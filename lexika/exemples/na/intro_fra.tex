\pagenumbering{roman}
{\LARGE \textbf{Introduction}}
	
«Philosophie»: qu'est-ce qui se joue dans la méfiance à l'égard d'une \textit{linguistique unique}?  comment argumenter 

\begin{quotation}
	Ce que nous tenons pour un monde ({\dots}) n'est-il pas tout d'abord, jusqu'au lointain des espèces, le champ qu'ont défriché et labouré des hommes occupés essentiellement de leur présence à un lieu, leurs objets reflétant donc ce désir, leurs savoirs ne cherchant en correspondances diverses que l'écho de ce projet de fonder? Dans ce cas les mots ne seraient que l'inscription, embroussaillée aujourd'hui sinon effacée, d'une pensée de la présence, élaborée comme telle, dans la «pierre» de ce dehors ({\dots}). Si les mots, disons cela autrement, n'étaient pas pour atteindre à l'en"=soi des choses mais pour approfondir le rapport d'une présence humaine et d'un lieu? S'ils étaient moins le «dictionnaire» de ce qu'on dira la nature que les marches de ce pays -- ce «séjour», disait Mallarmé -- où, par la grâce d'un horizon que des noms resserrent, nous pouvons être? Alors, bien sûr, chaque rencontre d'un être serait notre destin qui se joue, non la science qu'on vérifie. Et notre conscience en éveil se devrait de l'interroger comme son possible, même si déjà le temps nous en prive, au lieu de l'effacer de notre parole comme un exemple de plus de quelque schème éternel. (Yves Bonnefoy, «La fleur double, la sente étroite: la nuée», réimpr. in \textit{La Vérité de parole et autres essais}, Paris: Gallimard, 1995, pp. 566-567.)
\end{quotation}

Dans le cas du na: influence du tourisme.

\begin{quotation}
	[L]e tourisme s'est répandu comme une eau en crue et qui charrie des épaves. Il disloque le plus lointain des villages, il claironne dans la chambre la plus secrète des temples ses formules creuses et arrogantes, stéréotypes qui sont bien de notre Occident, hélas, dont la brochure publicitaire, cette parole de personne à propos de rien, est une des inventions spécifiques ({\dots}) [L]e texte touristique peut contraindre les habitants de ces lieux à se voir de par le dehors d'eux-mêmes, à se raconter comme le désire la paresse des arrivants, à substituer à leurs intuitions et leurs souvenirs les dessins bariolés qu'il faut à des visiteurs décontenancés ou hilares. (Yves Bonnefoy, \textit{L'Inachevable. Entretiens sur la poésie, 1990-2010}. Paris: Albin Michel, 2014, p. 232.)
\end{quotation}

La base de données lexicographique réalisée pour le na de Yongning est quadrilingue: les mots et exemples sont traduits en chinois, français et anglais. Pour le dictionnaire en ligne (HTML) et au format PDF, deux versions sont proposées: le présent document, na-chinois-français; et une version na-chinois-anglais. Ce choix paraissait préférable à la réalisation d'une unique version quadrilingue, car la présence du français et de l'anglais côte à côte, même distingués par la typographie, paraissait gêner la lecture.

La version française est pour l'heure essentiellement utilisée par son auteur; en conséquence, il ne paraisssait pas urgent de la doter d'une introduction en français, et on s'est contenté d'une introduction en anglais (pour la version na-chinois-anglais), à laquelle est renvoyé le lecteur souhaitant obtenir des précisions au sujet de ce dictionnaire. 

La présente version, première version publique, est numérotée 1.0. Il est prévu des mises à jour tous les ans ou tous les deux ans environ, à mesure de l'enrichissement du dictionnaire. Les lecteurs sont vivement encouragés à prendre contact (michaud.cnrs@gmail.com) pour signaler des erreurs et proposer des améliorations.

    La lexicologie est l'étude du lexique, et de la manière dont il est structuré dans chaque langue.  On peut vouloir l'analyser dans une seule langue en synchronie – par exemple, en rédigeant un dictionnaire, ou en décrivant un champ sémantique particulier dans cette langue. On peut aussi adopter une perspective typologique en comparant les lexiques de plusieurs langues dans un domaine donné : par exemple, comment les langues structurent-elles le domaine de la parenté ? de la magie ? des émotions ? de la parole et de la pensée ? Retrouve-t-on partout les mêmes concepts, les mêmes distinctions sémantiques ? La typologie lexicale peut permettre d’identifier des polysémies récurrentes, des zones de stabilité et de variation, des universaux du lexique.
    
    
    
    On peut également interroger ces structures lexicales dans leur dynamique, dans le temps et dans l'espace. Certaines polysémies, certaines associations sémantiques, sont apparues historiquement, ou au contraire ont disparu — soit par évolution interne, soit sous l'effet du contact avec d'autres langues. La tendance, chez les individus bilingues, à aligner les structures sémantiques des langues qu'ils parlent, a permis la diffusion de certaines catégorisations lexicales à l'échelle de vastes aires linguistiques et culturelles : c'est ainsi que certains découpages sémantiques, certaines polysémies ou phraséologies, deviennent les symptômes d’une aire donnée. Parfois, il est possible d'expliquer ces phénomènes aréaux par des liens entre pratiques langagières et pratiques sociales répandues dans la région: certains modes d’organisation familiale, par exemple, pourront être corrélés à des structures lexicales spécifiques dans le domaine de la parenté, ou dans le vocabulaire du mariage et des relations interpersonnelles.
    
    
    
    Au fil des prochaines années, l'idée de ce séminaire sera de mettre en valeur les données de première main que nous avons recueillies sur tous les continents, ressources considérables et souvent sous-exploitées. Les dictionnaires existants, livresques ou électroniques, pourront nous servir également; ainsi que nos corpus de textes, pour peu que nos méthodes impliquent d'y recourir. Par ailleurs, nous nourrirons nos réflexions théoriques et méthodologiques des publications diverses dans le domaine de la sémantique lexicale, de plus en plus nombreuses ces derniers temps – voyez la liste des références.
    
    
    
    Diverses approches sont possibles, et le LaCiTO pourra choisir d'en aborder plus d'une.  En fonction des souhaits des participants, nous pourrons choisir d'établir la typologie d'un domaine sémantique particulier, ou d'une sélection de différents domaines, à travers des études parallèles. Nous pourrons nous pencher sur la quête d'universaux, sur les cas de convergence aréale, sur les parcours étymologiques, sur la théorisation du changement sémantique. Nous pourrons confronter diverses approches méthodologiques: bases de données, création de questionnaires, élaboration et visualisation de statistiques, cartes sémantiques…
    
    
    
    
    Cette première séance commencera par un exposé d'Alexandre François visant à présenter le domaine – avec une attention particulière portée aux phénomènes de colexification (François 2008), et la possibilité de créer des cartes sémantiques en typologie lexicale.  
    
    L'exposé sera suivi d'une discussion collective pour réfléchir à la manière dont nous voudrions mener ce groupe de recherche au cours des prochaines années.
    
    
    
    Références
    
    François, Alexandre. 2008. Semantic maps and the typology of colexification: Intertwining polysemous networks across languages. In Martine Vanhove (ed.), From Polysemy to Semantic Change, Studies in Language Companion Series, vol. 106, 163–215. Amsterdam: John Benjamins. [accès en ligne]
    François, Alexandre. 2013. Shadows of bygone lives: The histories of spiritual words in northern Vanuatu. In Robert Mailhammer (ed.), 185-244.
    Juvonen, Päivi & Maria Koptjevskaja-Tamm (eds.). 2016. The Lexical Typlogy of Semantic Shifts. Cognitive Linguistics Research 58. Berlin: De Gruyter Mouton.
    Koptjevskaja-Tamm, Maria, Ekaterina Rakhilina & Martine Vanhove. 2015. The semantics of lexical typology. In Nick Riemer (ed.), The Routledge Handbook of Semantics. London: Routledge.
    Koptjevskaja-Tamm, Maria, Martine Vanhove, & Peter Koch. 2007. Typological approaches to lexical semantics. Linguistic Typology 11.1, 159-185.
    Mahieu, Marc-Antoine & Nicole Tersis (eds.) 2016. Questions de sémantique inuit / Topics in Inuit Semantics. Amerindia 38. 274 pp.
    Mailhammer, Robert (ed.). 2013. Lexical and structural etymology: Beyond word histories. Studies in Language Change, 11. Berlin: DeGruyter Mouton. 
    Moyse, Claire & Volker Gast & Ekkehard Koenig. 2014. Comparative lexicology and the typology of event descriptions : A programmatic study. In Doris Gerland, Christian Horn, Anja Latrouite, Albert Ortmann (eds), Meaning and Grammar of Nouns and Verbs. Studies in Language and Cognition, 1. Düsseldorf : Düsseldorf University Press, 145-183.
    Tersis, Nicole & Pascal Boyeldieu (eds.). 2017. Le langage de l'émotion : Variations linguistiques et culturelles. (Société d'Études Linguistiques et Anthropologiques de France, 469). Paris–Leuven: Peeters.
    Urban, Matthias. 2011. Asymmetries in overt marking and directionality in semantic change. Journal of Historical Linguistics 1(1). 3–47​.
    Vanhove, Martine (ed.) 2008. From Polysemy to Semantic Change. Studies in Language Companion Series, vol. 106. Amsterdam: John Benjamins.​

\cleardoublepage
\pagenumbering{arabic}
\setcounter{page}{1}

