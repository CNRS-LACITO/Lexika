
\documentclass[twoside,11pt]{article}
\title{Na dictionary}
\author{Alexis Michaud}
\usepackage[paperwidth=185mm,paperheight=260mm,top=16mm,bottom=16mm,left=15mm,right=20mm]{geometry}
\usepackage{multicol}
\setlength{\columnseprule}{1pt}
\setlength{\columnsep}{1.5cm}
\usepackage{changepage}
\usepackage[dvipsnames,table]{xcolor}
\usepackage{fancyhdr}
\pagestyle{fancy}
\fancyheadoffset{3.4em}
\fancyhead[LE,LO]{\rightmark}
\fancyhead[RE,RO]{\leftmark}
\usepackage{hyperref}
\hypersetup{pdftex,bookmarks=true,bookmarksnumbered,bookmarksopenlevel=5,bookmarksdepth=5,xetex,colorlinks=true,linkcolor=blue,citecolor=blue}
\usepackage[all]{hypcap}
\usepackage{fontspec}
\usepackage{natbib}
\usepackage{booktabs}
\usepackage{polyglossia}
\setdefaultlanguage{french}
\setotherlanguages{french,english}
\setmainfont{Charis SIL}
\usepackage{media9}
\usepackage{totcount}
\newcounter{compteur}
\setcounter{compteur}{0}
\regtotcounter{compteur}
\newfontfamily{\prin}[Mapping=tex-text,Ligatures=Common,Scale=MatchUppercase]{Gentium}
\newfontfamily{\nru}[Mapping=tex-text,Ligatures=Common,Scale=MatchUppercase]{Charis SIL}
\newfontfamily{\fra}[Mapping=tex-text,Ligatures=Common,Scale=MatchUppercase]{EB Garamond}
\newfontfamily{\cmn}[Mapping=tex-text,Ligatures=Common,Scale=MatchUppercase]{AR PL UMing CN}
\newfontfamily{\eng}[Mapping=tex-text,Ligatures=Common,Scale=MatchUppercase]{Liberation Serif}
\newfontfamily{\bod}[Mapping=tex-text,Ligatures=Common,Scale=MatchUppercase]{Gentium}
\newcommand{\pprin}[1]{\begin{cmn}{\prin #1}\end{cmn}}
\newcommand{\pnru}[1]{{\nru\textcolor{Blue}{#1}}}
\newcommand{\pfra}[1]{\begin{french}{\fra\textcolor{OliveGreen}{#1}}\end{french}}
\newcommand{\pcmn}[1]{{\cmn\textcolor{black}{#1}}}
\newcommand{\peng}[1]{\begin{english}{\eng\textcolor{Sepia}{#1}}\end{english}}
\newcommand{\pbod}[1]{{\bod\textcolor{black}{#1}}}
\newcommand{\cerclé}[1]{\raisebox{0pt}{\textcircled{\raisebox{-0.5pt} {\footnotesize{\pnru{#1}}}}}}
\newcommand{\caractère}[1]{\phantomsection\addcontentsline{toc}{section}{#1}{\begin{center}\textbf{\Large\pnru{#1}}\end{center}}}
\newenvironment{entrée}[3]{\hypertarget{#3}{}\phantomsection\addcontentsline{toc}{subsection}{#1\homonyme{#2}}\hspace*{-0.5cm}\textbf{\Large\pnru{#1 \homonyme{#2}}}\markright{#1 \homonyme{#2}}}{\stepcounter{compteur}\newline}
\newenvironment{sous-entrée}[3]{\par\hypertarget{#3}{}\phantomsection\addcontentsline{toc}{subsubsection}{#1 \homonyme{#2}}\begin{adjustwidth}{0.3cm}{}\pprin{■} \textbf{\Large\pnru{#1\homonyme{#2}}}}{\end{adjustwidth}}
\newcommand{\homonyme}[1]{#1}
\newcommand{\formedesurface}[1]{\hspace{0.5cm}/\pnru{#1}/\hspace{0.5cm}}
\newcommand{\formephonétique}[1]{\pnru{\textit{#1}}}
\newcommand{\ton}[1]{\cmn{声调类:}\prin{#1}\hspace{0.5cm}}
\newcommand{\classe}[1]{ \pcmn{\textcolor{PineGreen}{#1} }}
\newcommand{\paradigme}[1]{#1 }
\newcommand{\sens}[1]{ \cerclé{#1} }
\newenvironment{définition}{}{\hspace{5pt}}
\newenvironment{déclaration}{}{}
\newenvironment{exemple}{\pprin{¶} }{\hspace{5pt}}
\newenvironment{relationsémantique}{}{}
\newenvironment{forme-mot}{}{}
\newcommand{\synonyme}[1]{\pcmn{~【同义词】~\pnru{#1}}}
\newcommand{\antonyme}[1]{\pcmn{~【反义词】~\pnru{#1}}}
\newcommand{\confer}[1]{\pcmn{~【参考】~\pnru{#1}}}
\newcommand{\emprunt}[1]{\pcmn{~【借词】~#1}}
\newcommand{\étymologie}[1]{\pcmn{~【词源】~#1}}
\newcommand{\utilisation}[1]{\pcmn{~【用法】#1}}
\newcommand{\grammaire}[1]{\textsc{#1}}
\newcommand{\stylefv}[1]{\pnru{#1}}
\newcommand{\stylefn}[1]{\pcmn{#1}}
\newcommand{\stylefi}[1]{\textit{#1}}
\newcommand{\stylefg}[1]{\textsc{#1}}
\XeTeXlinebreaklocale "zh"
\XeTeXlinebreakskip = 0pt plus 1pt
% % Code spécial pour la gestion générique des césures applicable aux formes de surface
%\ExplSyntaxOn
%\RenewDocumentCommand{\formedesurface}{m}
%{
%% nouvelle variable « expression »
%\tl_set:Nn \expression { #1 }
%% remplace ˩˧˥ par ˩˧˥\-
%\regex_replace_all:nnN { (\B[˩˧˥]) } { \1\c{-} } \expression
%% renvoie la séquence totale
%{\tl_use: {\hspace{0.5cm}/\pnru{\expression}/\hspace{0.5cm}}}
%}
%\ExplSyntaxOff

\begin{document}
\pagenumbering{roman}
	{\LARGE \textbf{Introduction}}

This dictionary documents the lexicon of the Na language (\ipa{nɑ˩-ʐwɤ˥}) as spoken in and around the plain of Yongning, located in Southwestern China, at the border between Yunnan and Sichuan, at a latitude of 27°50’ N and a longitude of 100°41’ E. This language is known in China as ‘Mosuo'. 

To write a dictionary is to describe a language's lexical structures (Alex François). Progressing from a word list to a full-fledged dictionary requires delving into meaning, attempting to delineate connotations, polysemy, and relationships between words. Most dictionary entries could easily be expanded into essays. This dictionary is still at a relatively early stage xxxx


	\section{Broader research agenda} \label{sec:researchagenda}
	
	This dictionary is the outcome of linguistic fieldwork, an endeavour which consists in “going into a~community
	where a~language is spoken, collecting data from fluent native speakers, analysing the data, and
	providing a~comprehensive description, consisting of grammar, texts and dictionary”
	\citep[12]{Dixon2007}. The grammar, texts and dictionary are referred to as the “Boasian trilogy” \citep{foley1999} by reference to Franz Boas’s foundational work collecting North American languages \citep{boas1902, boasetal1911}; the trilogy can now be said to have become a \textit{tetralogy} as it integrates a multimedia component: audio and video recordings \citep{musgraveetal2014}. 
	
	My fieldwork on Yongning Na began in October 2006. A list of words was begun through elicitation, and gradually expanded and corrected as narratives were recorded and transcribed; addition of new words was therefore a slow process. An advantage of placing the emphasis on text collection is that a context is available to help clarify the meaning of newly encountered words, also offering a~basis for further discussion of their usage with language consultants. Systematic elicitation of large amounts of vocabulary was not carried out, hence the limited number of entries: currently on the order of 3,000.
	%currently \total{compteur}. 

\section{A review of Na language studies}
\label{sec:previousstudiesofthenalanguage}

This review of the literature about the Na language is summarized (and updated) from \citet[14-21]{michaud2017}. 

\subsection{Information about Na in the \textit{Brief description of the Naxi language}}

He Jiren \& Jiang Zhuyi’s \citeyear{heetal1985} \textit{Brief description of the Naxi language} (in Chinese) mainly focuses on the dialects spoken in the Lijiang plain, but the volume includes a~word list of Yongning Na (referred to as ‘Yongning plain \textit{patois} of the Eastern dialect of Naxi'), as well as some observations on phonology,
syntax, and dialectal diversity (pp. 107--116; see also \citealt{jiang1993}). The transcription is not phonemicized, and may not be fully consistent. Only four tones are
transcribed over monosyllables: LM (\ipa{˩˧}), M (\ipa{˧}), ML (\ipa{˧˩}), and H (\ipa{˥}), whereas
the analysis presented in \citet[53-92]{michaud2017} brings out six categories (LM, LH, M, L, H and MH). He \& Jiang based their linguistic research on
an~{analogy} with \ili{Naxi}, a~language which both of them could speak: He Jiren as a~native speaker, Jiang
Zhuyi as a~second"=language learner. \ili{Naxi} has a~four"=way tonal opposition over monosyllables:
High, Mid, Low (realized phonetically as low"=falling),
and Rising.  
There are also some issues with He \& Jiang’s transcription of vowels and consonants, as is to be
expected of initial field notes. Nasality is transcribed only in two syllables, /\ipa{xĩ}/ (as in
‘man’, transcribed /\ipa{xĩ˧}/; my data: //\ipa{hĩ˥}//) and /\ipa{ɣə̃r}/ (the only example is ‘bone’,
transcribed /\ipa{ʂa˧ɣə̃r˧}/; my transcription: //\ipa{ʂæ˩ɻ̍̃˩}//), whereas my investigation brings out eight nasal rhymes \citep[461-464]{michaud2017}. Another point of difference is that He \&
Jiang do not transcribe the uvular consonants reported here. Such
discrepancies may be due to the fact that the variety described by He \& Jiang had fewer phonemes
than that described here; but it is not implausible that they failed to distinguish some sounds that
were in fact contrastive. Conversely, some vowel differences transcribed by He \& Jiang may be spurious. The word list
contains examples of /\ipa{li}/ (as in /\ipa{li˧}/ ‘to look’) and /\ipa{lie}/ (as in /\ipa{lie˩˧}/
‘tea’). In my data ‘tea’ and ‘to look’ have the same initial and rhyme. The vowel /\ipa{i}/ is
slightly diphthongized towards [\ipa{e}], and thus close to [\ipa{lie}], which explains why it could be
sometimes heard as [\ipa{i}] and sometimes as [\ipa{ie}] before the investigator’s ear attunes to the
vowel system of Yongning Na. Once again, it is also theoretically possible that these two words did
not have the same phonemes in the dialect investigated by He \& Jiang.

\subsection{A~study of kinship terms, with phonetic observations: \citet{fu1980}}
\label{sec:fu1980astudyofkinshipterms}

The linguist Fu Maoji \zh{傅懋勣} visited Yongning in May and June 1979 with He Jiren and Jiang Zhuyi. He collected data in the
village of /\ipa{dʑɤ˩bv̩˧-ʁwɤ˩}/ (Jiabowa \zh{甲波瓦}) for a~study about kinship terms, presented at the 12th International Conference on {Sino"=Tibetan} Languages and Linguistics (Paris,
1979), then published in Chinese and in {French} translation (\citealt{fu1980,fu1983}). The article has an~appendix containing notes about phonetic transcription. 

\subsection{An outline of Yongning Na by Yang Zhenhong (\citeyear{yang2009})}
\label{sec:yang2009}

An outline of Yongning Na was published by Yang Zhenhong \zh{杨振洪}, a~speaker of this language
from /\ipa{ə˧bv̩˧-ʁwɤ˧}/ village (Abuwa \zh{阿布瓦村}), close to the current location of
the Yongning high school (original publication in Chinese: \citealt{yang2006d}; {English} translation
by Liberty Lidz, with improvements made after consulting with the author, published as \citealt{yang2009}). This outline by and large follows the structure of
He \& Jiang’s description of Naxi. 

\subsection{Lexical materials}
\label{sec:dictionary2013}

\textit{An anthology of everyday words and expressions in the Mosuo language} \citep{zhibaetal2013} presents vocabulary and expressions arranged by semantic field. The authors are a~native speaker from the Lake Lugu area and a~doctor in linguistics from Yunnan University. Their fieldwork is described as covering the Yongning plain and the Lake area, but with the Yongning plain as the main research area (p. 2). 

Approximations in phonetic notation are so numerous that they make the volume unreliable as a~work of reference. Voicing contrasts were challenging for the linguist in the team, whose training was mainly focused on the theory and practice of teaching Chinese as a~foreign language. Thus, the name of the mountain /\ipa{kɤ˧mv̩˧˥}/ is transcribed as /\ipa{gə⁵⁵mu⁵⁵}/, with a~voiced initial (p. 17 and elsewhere). The mountain's name in Chinese, \textit{Gemu} \zh{格姆山}, may have exerted an influence here. Conversely, the adjective /\ipa{dʑɤ˩\textsubscript{b}}/ ‘good’ is transcribed as /\ipa{tɕɑ¹³}/, with an unvoiced initial. Some phonemes, such as uvulars, are absent from the notations. 

\subsection{Liberty \citet{lidz2010}, \textit{A descriptive grammar of Yongning Na (Mosuo)}}
\label{sec:lidz2010}

Liberty Lidz’s
Ph.D.\ dissertation \citep{lidz2010}, \textit{A descriptive grammar of Yongning Na}
(\textit{Mosuo}), concerns the variety of Yongning Na spoken in the village of Luoshui \zh{落水},
on the shore of Lake Lugu. The dissertation, based on in"=depth fieldwork, provides a~description of the morphosyntax
of the language, and contains 150 pages of
transcribed and annotated narratives.


	\section{Chronology and method} \label{sec:method}

Unless otherwise stated, all the data are from one language consultant, Mrs. Latami Dashilame (\ipa{lɑ˧tʰɑ˧mi˥ ʈæ˧ʂɯ˧-lɑ˩mv˩}; Chinese: \zh{拉它米打史拉么}). She was born in 1950 in the hamlet called \ipa{ə˧lɑ˧-ʁwɤ\#˥} in Na, close to the monastery of Yongning. The administrative coordinates of this village are: Yúnnán province, Lìjiāng municipality, Nínglàng Yí autonomous county, Yǒngníng district, Ālāwǎ village (\zh{云南省丽江市宁蒗彝族自治县永宁乡阿拉瓦村}). The choice to work in one location only, and essentially with one consultant, is, again, based on the investigator's focus on the tone system. There is considerable dialectal diversity within the Na area (much more so than in the Naxi-speaking area); the tone systems of different villages are conspicuously different, and this geographical diversity combines with dramatic differences across social groups, and across generations. The obvious thing to do seemed to be an in-depth description and analysis of the language as spoken by one person (simultaneously making a few forays into other idiolects and dialects). Data from other speakers are indicated using their codes in the author's database of speakers of Naish languages. Table \ref{tab:consul} provides the speaker codes.

\begin{table}[H]
	\caption{Language consultant codes}
	\centering \label{tab:consul}
	\begin{tabular}{lllllll}
		\toprule
		speaker code &   name &  year of birth \\
		\midrule
			F4 (main consultant) & \ipa{lɑ˧tʰɑ˧mi˥ ʈæ˧ʂɯ˧-lɑ˩mv˩} & 1950 \\ 
			F5 &  \ipa{ki˧zo˧} & 1973  \\ 
			F6 &  \ipa{tɕʰi˧ɖv\#˥} & 1987 \\ 
			M18 &  \ipa{lɑ˧tʰɑ˧mi˥ ʈæ˧ʂɯ˧-ʈæ˩ʈv˩} & 1972 \\ 
			M21 & \ipa{ho˧dʑɤ˧tsʰe˥} & 1942 \\ 
			M23 & \ipa{ɖɯ˩ɖʐɯ˧} & 1974 \\
		\bottomrule
	\end{tabular}
\end{table}

The list of words as of 2011 was deposited in the STEDT database (http://stedt.berkeley.edu/). The same year, under the impetus of Guillaume Jacques and Aimée Lahaussois, plans were made to bring the word list closer to the standards of a full-fledged dictionary. A project was deposited with the Agence Nationale de la Recherche, accepted in 2012, and begun in 2013: the HimalCo project (ANR-12-CORP-0006). Céline Buret, a computing engineer, worked with the project team for two years (Nov. 2014-Oct. 2015). She converted the data to the format of the Field Linguist's Toolbox (MDF), then produced scripts for conversion to a XML format complying with the Lexical Markup Framework standard (LMF), allowing for automatic conversion to an online format as well as to LaTeX files (with PDF as the final output for circulation). The scripts constitute a Python 2 library called PyLMFlib, for: \textit{Python LMF library}. In 2015, version 1.0 of the online and PDF versions of the dictionary were produced and published online, along with the source document in MDF (Toolbox) format.

In 2016, Benjamin Galliot, working at CNRS-LACITO under a fixed-term (six-month) contract funded by CNRS (Délégation Paris-Villejuif), wrote a~new library using Python 3, which allows for native management of the Unicode standard. He also wrote a XSL script for generating the PDF versions of the dictionary. The new version of the dictionary released in the year 2017 (version 1.1) now has 

\begin{itemize}
	\item \textit{a phonetic transcription of tone as it surfaces on the item pronounced in isolation:} a surface-phonological transcription of tone, in addition to the indication of the underlying tone category
	\item \textit{a romanized representation:} a proposed spelling, devised by Roselle Dobbs and her collaborators (see more below)
	\item \textit{more cross-references} between entries, pointing to synonyms, etc.
\end{itemize}

\section{The Lexica series of dictionaries: format and vision} \label{sec:format}

The Na dictionary is one of the first in the Lexica series. \textit{Lexica} is the term adopted in 2017 for the dictionaries of the Pangloss Collection, an archive of (mostly) rare and endangered languages \citep{michailovskyetal2014}. The Lexica series aims to combine \textit{readability} (for users who browse through the dictionaries) with \textit{computer-readable encoding} (suitable for Natural Language Processing). 

	\subsection{Usefulness and limitations of the LMF format} \label{sec:lmf}

The Lexical Markup Framework \citep{francopoulo2013} is a pivotal format designed for machine-readable dictionaries. It is mostly compatible with SIL’s MDF format (Multi-Dictionary Formatter, used in particular by the Toolbox software). A limitation is that it places a constraint on what an entry can contain: subentries that belong to different grammatical categories need to have separate entries set up. For instance, \ipa{lɑ˧-kʰv̩˧˥} can mean both ‘year of the Tiger’ and ‘born in the year of the Tiger’ (in Chinese: \zh{虎年} and \zh{属虎}). From the linguist's point of view, it is desirable to set up two subentries within the same entry. But the part"=of"=speech categories are different (‘year of the Tiger’ is a noun phrase, and ‘born in the year of the Tiger’ is a predicate, categorized as an adjective), which, in the Lexical Markup Framework, necessitates setting up two different entries. 

\section{Guide to using the dictionary} \label{sec:howto}

	\subsection{Formats: trilingual Na-Chinese-English or Na-Chinese French} \label{sec:versions}

Entries and examples have translations into English, Chinese and French. Two language settings are offered for the PDF and online dictionary: either Na-Chinese-English, or Na-Chinese-French. The English and the French are not typeset alongside each other in the same document because distinguishing them visually is not obvious, even with the help of typographic devices such as using different fonts and colours. In the author's own experience, it was found that the presence of four languages alongside one another made consultation more difficult; specifically, English translations tended to be a distraction slowing down access to the French translations, and English users may similarly find that French clutters the layout. On the other hand, Chinese characters are visually well-distinct from Latin-based scripts, and so it did not appear necessary to separate the Chinese and produce a Na-Chinese version. Moreover, Chinese translations are often a useful complement to the translation in English (or French), as there are often closer equivalents: for instance \ipa{gɤ˧˥} translates straightforwardly as Chinese \zh{扛} whereas the English translation is more roundabout: ‘to carry on the shoulder'. Users who wish to have access to all the information can download the original file in Toolbox (MDF) format. 

	\subsection{Format of entries} \label{sec:entries}

Each entry contains
\begin{itemize}
	\item \textit{phonological transcription:} the form of the word in phonetic alphabet; tone is indicated in terms of phonological categories
	\item \textit{surface form:} the surface"=phonological form of the word in phonetic alphabet. Tone is indicated in terms of tonal realizations. This form can be read by anyone with a knowledge of the IPA, without requiring an understanding of the mapping of underlying phonological tone categories to surface tone in Yongning Na. 
	\item \textit{proposed orthographic representation:} notation in a~Romanized script devised by Roselle Dobbs with Mosuo collaborators. 
	\item \textit{part of speech:} an indication of the part of speech, using a simple set of labels
	\item \textit{tone:} the tone category of the word. This information is already present in the phonological transcription; having it repeated on its own facilitates searches
	\item \textit{definitions} in Chinese and English
	\item \textit{examples} with translations
	\item \textit{links} to related words, such as synonyms, or constituent parts of complex words 
	\item \textit{classifier:} for nouns, an indication on the more commonly associated classifiers
\end{itemize}

For surface"=phonological forms, the aim was to achieve greatest simplicity. Special symbols used in the word's underlying phonological form are removed from the surface form: dashes indicating junctures internal to the word, and tilde in reduplicated forms. For classifiers, the numeral ‘one' is added before the classifier, because classifiers are not free forms: they cannot be said on their own. For other bound morphemes, such as affixes and clitics, no surface form is indicated.

The proposed orthography in Latin alphabet was added by Roselle Dobbs (in the Summer of 2017). The transcription was developed by R. Dobbs with Na consultants, with a view to use within the Na community. Importantly, this is not a transliteration of the phonology: phonological forms are all based on the dialect of consultant F4, whereas proposed orthographic representations are intended by R. Dobbs and her collaborators as a cross"=dialect writing system. Given the high degree of dialectal diversity, proposing a transcription that is acceptable for speakers of several dialects implies some compromises. For instance, only the low tone (L), presumed to be more stable, is indicated in the orthography. Until a full description of this romanized writing system is published, requests for information about orthographic developments should be directed to Roselle Dobbs (rosellemay@hotmail.com). 

Among examples, those elicited to verify the output of certain combinations of tones are marked as ‘PHONO': examples elicited for the purpose of the phonological study. Proverbs and sayings are marked as ‘PROVERB'.

Some pieces of information are not shown in the PDF and online versions. These are:
\begin{itemize}
	\item An indication of \textit{semantic domain}: ‘society', ‘house', ‘body', ‘plant', ‘animal'... No attempt was made to use a fine-grained classification of the sort found in the WordNet database of English, where nouns, verbs, adjectives and adverbs are grouped into sets of cognitive synonym \citep{Fellbaum 2005}. This is simply a rough division into subsets for convenient sorting; the labelling relies partly on form, and partly on semantic contents. As for other aspects in the dictionary, choices made reflect the investigator's research priorities: for instance, the entries for ‘day’, ‘night’, ‘month’, ‘year’ were tagged as “classifiers", along with all other nouns that can appear immediately after a numeral. This allowed easy extraction of all classifiers for the purpose of a study of the tone patterns of classifiers \citep{Michaud2013}. These lexical items could just as well have been tagged as 'time', in view of their semantic field. The numbers ‘100’, ‘1,000’ and ‘10,000’ were likewise labelled as “classifiers" rather than numerals.
	\item \textit{Notes on past notations:} information tracing the history of notations, from the first fieldwork to the current version. For instance, the entry \ipa{ŋwɤ˧pʰæ˧˥} ‘tile' has a note that indicates that it was initially written with a M.H tone pattern, and with vowel \ipa{æ} in both syllables: *\ipa{ŋwæ˧pʰæ˥}. The note explains that the perception of \ipa{æ} in the first syllable is due to a phonetic tendency towards regressive vowel harmony. Verifications are also consigned in this field. About half the entries have information of this type.
	\item \textit{glosses:} glosses in English, Chinese and French, intended for the glossing of texts. The dictionary adopts the abbreviations recommended in the Leipzig Glossing Rules \citep{Comrie}; all other terms are provided in full. Glosses mostly follow the choices made by \citep{Lidz2010}.
\end{itemize}

	\subsection{Part-of-speech labelling} \label{sec:pos}
	
Dictionary entries carry a part-of-speech label. A rough-and-ready typology has been followed: see the table below. Needless to say, this system has limitations: a refined typology would require subcategories, e.g. defining classifiers as a subset among nouns; and categories such as ‘adverbs' raise greater difficulties, lacking a clear definition.
\begin{table}
	\caption{Parts of speech}
	\centering \label{tab:glosses}
	\begin{tabular}{lll}
		\toprule
		label & meaning & Leipzig Glossing Rules? y/n \\
		\midrule
		adj & adjective & y \\
		clf & classifier & y \\
		clitic & (same) & n \\
		cnj & conjunction & n \\
		ideophone & (same) & n \\
		disc.PTCL & discourse particle & n \\
		intj & interjection & n \\
		lnk & linker & n \\
		n & noun & y \\
		num & numeral & n \\
		pref & prefix & n \\
		postp & postposition & n \\
		pro & pronoun & n \\
		suff & suffix & n \\
		v & verb & y \\
		\bottomrule
	\end{tabular}
\end{table}

No attempt was made at including expressive noises in the dictionary, such as the sound \ipa{ɬː}. The meaning of this sound in Na can be characterized in the same way as that of words in the dictionary: the full definition would be that  \textit{it expresses enjoyment of food or drink (‘Yummy!'), and is also used to express admiration of a beautiful object, scene, or prospect}. But a reason for leaving it out is that, unlike interjections,  \ipa{ɬː} is not pronounced on expiratory airflow, but on inspired airflow. The air flows through the sides of the mouth, which is where saliva flows when one's mouth waters. Observations about such sounds (including clicks), like that of gestures, appeared to fall outside the scope of the dictionary.

	\subsection{Roots extracted from disyllables} \label{sec:roots}

xxxx Some roots can be extracted from disyllables. They are indicated by the symbol †. No surface form is provided, as these words are not in use in the language.

	\subsection{Words for which no Na equivalent was found} \label{sec:zeroanswer}
	
	The list that follows groups words for which no close equivalents could be found. These negative pieces of information contain hints about the consultants’ Na vocabulary and its ’soft shoulders’.
	
	English & Chinese & local Chinese name & French & comment
	Selaginella sp. & 卷柏 & & Selaginella sp. & F4 recognizes pictures of this plant, which is abundant in the mountains. But she does not have a name for it.
	hog, sand badger & 猪獾、狗獾 & 臭猪子 & blaireau &
	lynx & 猞猁& & lynx &
	pangolin & 穿山甲& & pangolin &
	earlobe & 耳垂 & & lobe de l'oreille & The word for 'ear' is used: there is no specific name for the earlobe.
	temples (on the head) & 太阳穴 & & tempes & In F4's Na, the temples are part of the forehead, /to˧kɤ#˥/. 
	wild man, savage & 野人 & & yéti & This is not part of local folklore. 
	Selaginella sp. & 卷柏 & & Selaginella sp. & F4 recognizes pictures of this plant, which is abundant in the mountains. But she does not have a name for it.
	Chinese angelica & 当归 & & Angelica sp. & 
	Anisodus tanguticus & 山茛菪 & & Angelica sp. & 
	Ligularia fischeria & 山紫菀 & & Ligularia fischeria & 
	dried cheese in thin, translucid slices & 乳扇 & & fromage séché en feuilles translucides & None was produced or consumed in Yongning in F4's lifetime.
	traditional toothbrush, made of pig hair & 用猪鬃毛作成的牙刷 & & brosse à dent traditionnelle en poil de cochon & Not in use in Yongning in F4's lifetime.
	catapult & 抛石机,弹弓 & & catapulte & Unknown to F4.
	slingshot & 绷弓子 & & lance-pierre, tire-boulettes (partie rigide en Y, et partie élastique permettant de catapulter un petit objet) & Unknown to F4.


	\subsection{Loanwords} \label{sec:loan}
	
Borrowings from Chinese and Tibetan are indicated as such in cases where identification seems straightforward. No efforts at systematic elicitation of borrowings from either language were made, but all loanwords occurring in texts were added to the dictionary. The information provided includes: donor language; form in the donor language; and explanations. When the number of syllables in the borrowed word is the same as in the donor language, the glosses in English (and French) start by the original word followed by two colons and a translation: e.g. ‘\zh{办法}::solution' for \ipa{pæ˧˥hwɤ˧}. 


	\section{Planned improvements and mid-term perspectives} \label{sec:improv}
	
This dictionary is conceived of as work-in-progress: successive versions will be released, probably every two years or so, (i) as an online dictionary in HTML format, (ii) as PDF documents, and (iii) in database format (native Toolbox/MDF format, then, in due course, the successors of this format). 

Planned improvements for future versions include the addition of
\begin{itemize}
	\item \textit{audio files for each head word:} this function has successfully been tested, but the editing of audio files still needs to be conducted
	\item \textit{links to the entire set of online recordings}: listing all textual occurrences in the lexicon entry, with links to the audio file and its aligned transcription. Textual occurrences ultimately constitute the best resource to document a word's usage. The examples currently presented in the dictionary are few in number, compared to the occurrences in texts; and their context of use may not be clear, despite efforts at clarifying their nature (singling out examples elicited for the purpose of morpho-phonological investigation by the mention ‘PHONO') and at providing contextual information for examples jotted down during fieldwork.
	\item \textit{additional cross-references} between entries, pointing to synonyms, etc.
	\item \textit{more Chinese loans:} indicating the Na pronunciation of Chinese words that are now commonly used by speakers of Na.
\end{itemize}

Collaborations are welcome for the following improvements :
\begin{itemize}
	\item \textit{the vocabulary of religion:} the field of religion remains mostly unexplored; the main consultant and I both lack the command of Tibetan that would be essential for this part of the investigation, and involvement of consultants from the Yongning monastery did not prove feasible in view of current restrictions on contacts with foreigners
	\item \textit{plants and animals:} as a dweller of the plain, the main consultant does not have extensive knowledge of wild plants and animals; the number of entries recorded so far remains small, and some definitions are currently limited to general indications such as ‘a type of pine'. To arrive at exact identification, and at more extensive lexicographic coverage, would require collaboration with other consultants, and with botanists.
\end{itemize}

	\section{Other resources about Yongning Na} \label{sec:resources}
	
	In the classical tradition of linguistic fieldwork \citep{Dixon2007}, a language description should include a dictionary, a grammar, and a collection of texts. 
	
	\begin{itemize}
		\item \textit{A set of Na recordings with time-aligned transcriptions} is available from the Pangloss Collection \citep{Michailovsky2014}; the current web address is lacito.vjf.cnrs.fr/pangloss/languages/Na\_en.htm 
		\item \textit{The grammar} is still in its early stages of preparation. A preliminary draft of a book-length study of Na morpho-tonology can be found online: https://halshs.archives-ouvertes.fr/halshs-01094049/document It also contains detailed information on the phonemic analysis.
	\end{itemize}
	
A review of the literature about Na and the other languages of the Naish  group is provided (in Chinese) by \citet{Li2015}. For an English-language introduction, see \citet{Michaud2015b}.

I would gratefully receive any comments or notifications of errors that the reader may wish to bring to my attention: please send e-mail to michaud.cnrs@gmail.com 


	\section{Acknowledgments} \label{sec:ackno}

Many thanks to Picus Ding for putting me in touch with the Mosuo scholar Latami Dashi. Special thanks to Latami Dashi for supporting and encouraging my work with his mother Latami Dashilame over the years. Many thanks to the main consultant, Latami Dashilame, and to all family members. 

Many thanks to Céline Buret and Séverine Guillaume for their much-appreciated computational expertise, and to Guillaume Jacques for suggestions all along the way. Many thanks to connoisseurs of the Na culture and language for useful exchanges: Lamu Gatusa \zh{拉木嘎吐萨} (Chinese pen-name: \zh{石高峰}), Liberty Lidz, Christine Mathieu, Pascale-Marie Milan and He Sana \zh{何梭娜}. Special thanks to Roselle Dobbs for extensive discussions and vigorous proof-reading over the years. Many thanks to A Hui \zh{阿慧} (to my knowledge the first speaker of Mosuo to read a M.A. degree in language and linguistics) for suggesting corrections. Remaining errors are my own responsibility. 

I am grateful for the opportunity allowed me by my home institution, Centre National de la Recherche Scientifique, of staying in China in 2011-2012 for extensive fieldwork, through a temporary affiliation with the CNRS’s research centre in China: CEFC (Centre d’Etudes Français sur la Chine contemporaine). From November 2012 to June 2016, I was based at the international research institute MICA, in Hanoi, in an exceptionally stimulating environment allowing for close collaboration with colleagues from Asia and elsewhere. Special thanks to the heads of the institute, Phạm Thị Ngọc Yến (succeeded in 2015 by Nguyễn Việt Sơn) and Eric Castelli, for their support and encouragement.

I am grateful to the Dongba Culture Research Institute (\zh{丽江市东巴文化研究院}) in Lijiang and the Horse-Tea Road Culture Research Centre (\zh{云南大学茶马古道文化研究所}) in Kunming for inviting me to become an Adjunct member (\zh{外聘研究员}), and for facilitating administrative and practical matters; special thanks to Li Dejing \zh{李德静} and to Mu Jihong\zh{木霁弘}. At Yunnan University, many thanks are due to Duan Bingchang \zh{段炳昌}, Wang Weidong \zh{王卫东}, Zhao Yanzhen \zh{赵燕珍} and Yang Liquan \zh{杨立权} for their careful and sensitive management of fieldwork-related administrative matters.
	
So many people have supported this project that I must apologize for those names that should be here but were inadvertently left off the list.

This work was supported financially by the ANR project HimalCo (ANR-12-CORP-0006), and constitutes a contribution to the LabEx “Empirical Foundations of Linguistics" project (ANR-10-LABX-0083).

\begin{thebibliography}{7}
	\providecommand{\natexlab}[1]{#1}
	\providecommand{\url}[1]{#1}
	\providecommand{\urlprefix}{}
	\expandafter\ifx\csname urlstyle\endcsname\relax
	\providecommand{\doi}[1]{doi:\discretionary{}{}{}#1}\else
	\providecommand{\doi}{doi:\discretionary{}{}{}\begingroup
		\urlstyle{rm}\Url}\fi
	
	\bibitem[{Li(2015)}]{Li2015}
	Li Zihe [\zh{李子鹤}]. 2015.
	\newblock
	\zh{纳西语言研究回顾------兼论语言在文化研究中的基础地位}
	[{A} review of {Naxi} language studies, with a discussion of the fundamental
	role of cultural studies for linguistic research].
	\newblock \zh{茶马古道研究期刊} 4. 125--131.
	
	\bibitem[{Comrie et~al.()Comrie, Haspelmath \& Bickel}]{Comrie}
	Comrie, Bernard, Martin Haspelmath \& Balthasar Bickel. 2008.
	\newblock Leipzig {Glossing Rules}.
	\newblock
	\urlprefix\url{http://www.eva.mpg.de/lingua/resources/glossing-rules.php}.
	
	\bibitem[{Dixon(2007)}]{Dixon2007}
	Dixon, Robert~M. 2007.
	\newblock Field linguistics: a minor manual.
	\newblock \emph{Sprachtypologie und Universalienforschung} 60(1). 12--31.
	
	\bibitem[{Lidz(2010)}]{Lidz2010}
	Lidz, Liberty. 2010.
	\newblock \emph{A descriptive grammar of {Yongning Na} ({Mosuo})}.
	\newblock Austin: University of Texas, Department of linguistics dissertation.
	\newblock
	\urlprefix\url{https://repositories.lib.utexas.edu/bitstream/handle/2152/ETD-UT-2010-12-2643/LIDZ-DISSERTATION.pdf}.
	\newblock Ph. D.
	
	\bibitem[{Michailovsky et~al.(2014)Michailovsky, Mazaudon, Michaud, Guillaume,
		Fran{\c{c}}ois \& Adamou}]{Michailovsky2014}
	Michailovsky, Boyd, Martine Mazaudon, Alexis Michaud, S{\'{e}}verine Guillaume,
	Alexandre Fran{\c{c}}ois \& Evangelia Adamou. 2014.
	\newblock Documenting and researching endangered languages: the {Pangloss
		Collection}.
	\newblock \emph{Language Documentation and Conservation} 8. 119--135.
	\newblock \urlprefix\url{http://hdl.handle.net/10125/4621}.
	
	\bibitem[{Michaud(2013)}]{Michaud2013}
	Michaud, Alexis. 2013.
	\newblock The tone patterns of numeral-plus-classifier phrases in {Yongning
		Na}: a synchronic description and analysis.
	\newblock In Nathan Hill \& Tom Owen-Smith (eds.), \emph{Transhimalayan
		{Linguistics}. {Historical} and {Descriptive} {Linguistics} of the
		{Himalayan} {Area}} (Trends in {Linguistics}. {Studies} and {Monographs}
	[{TiLSM}] 266), 275--311. Berlin: De Gruyter Mouton.
	
	\bibitem[{Michaud et~al.(2015)Michaud, Limin \& Yaoping}]{Michaud2015b}
	Michaud, Alexis, He~Limin \& Zhong Yaoping. 2015.
	\newblock Naxi / {Naish}.
	\newblock In Rint Sybesma, Wolfgang Behr, Zev Handel \& C.T.~James Huang
	(eds.), \emph{Encyclopedia of {Chinese} {Language} and {Linguistics}},
	Leiden: Brill.
	
\end{thebibliography}

\cleardoublepage
\pagenumbering{arabic}
\setcounter{page}{1}
\setlength{\parindent}{0pt}
\begin{multicols}{2}
\lhead{\firstmark}
\rhead{\botmark}
\newpage\caractère{ɑ}

\begin{entrée}
{ɑ˩˧}{}{ⓔɑ˩˧}\formedesurface{ɑ˩˥}\newline
\classe{名词}\ton{LM}
\paradigme{\pcmn{:} \p{}}
\begin{définition}\peng{Goose.}\end{définition}
\begin{définition}\pcmn{鹅}\end{définition}
\begin{définition}\pfra{Oie.}\end{définition}
\begin{exemple}\pnru{ɑ˩ dzɯ˥-ze˩}\hspace{5pt}\peng{…has eaten (a/the) goose}\hspace{5pt}\pcmn{吃了鹅}\hspace{5pt}\pfra{…a mangé (une) oie}\end{exemple}
\begin{exemple}\pnru{ɑ˩ hwæ˧-ze˧}\hspace{5pt}\peng{…has bought (a/the) goose}\hspace{5pt}\pcmn{买了鹅}\hspace{5pt}\pfra{…a acheté (une) oie}\end{exemple}
\end{entrée}

\begin{entrée}
{ɑ˩mi\#˥}{}{ⓔɑ˩mi\#˥}\formedesurface{ɑ˩mi˥}\newline
\classe{名词}\ton{LM+\#H}
\paradigme{\pcmn{:} \p{}}
\begin{définition}\peng{Female goose.}\end{définition}
\begin{définition}\pcmn{母鹅}\end{définition}
\begin{définition}\pfra{Oie (femelle).}\end{définition}
\begin{exemple}\pnru{ɑ˩mi˧-ɑ˥pʰv̩˩}\hspace{5pt}\peng{goose and gander}\hspace{5pt}\pcmn{公鹅与母鹅}\hspace{5pt}\pfra{oie et jars}\end{exemple}
\end{entrée}

\begin{entrée}
{ɑ˩pʰo˩}{}{ⓔɑ˩pʰo˩}\formedesurface{ɑ˩pʰo˩˥}\newline
\classe{助词}\ton{L}\begin{définition}\peng{Outside.}\end{définition}
\begin{définition}\pcmn{外面}\end{définition}
\begin{définition}\pfra{Dehors, à l'extérieur.}\end{définition}
\begin{exemple}\pnru{ɑ˩pʰo˩ bi˩˥}\hspace{5pt}\peng{to go outside; to attend to the call of nature}\hspace{5pt}\pcmn{出去,解手(委婉语)}\hspace{5pt}\pfra{aller dehors; aller faire ses besoins}\end{exemple}
\begin{exemple}\pnru{ɑ˩pʰo˩ bi˩-ze˥!}\hspace{5pt}\peng{Let's go out! / I must answer the call of nature!}\hspace{5pt}\pcmn{出去了! / 出去解手!}\hspace{5pt}\pfra{On sort! / [Je] vais faire mes besoins!}\end{exemple}
\begin{exemple}\pnru{ɑ˩pʰo˩ bi˩-ʂv̩˥ɖv̩˩!}\hspace{5pt}\peng{[She] wants to go out! (Context: on a sunny day, a family member senses that a toddler wants to go and play outside.)}\hspace{5pt}\pcmn{(她)想出去!(情景:婴儿看外边,觉得她好像想出去。)}\hspace{5pt}\pfra{[Elle] a envie de sortir! (Contexte: par une journée ensoleillée, un membre de la famille sent qu'un enfant aurait envie d'aller jouer dehors.)}\end{exemple}
\begin{exemple}\pnru{ə˧dɑ˥ | ə˩pʰo˩ hɯ˩-ze˥!}\hspace{5pt}\peng{Daddy went out! (Context: explanation provided to a little child who has just woken up and looks for her dad.)}\hspace{5pt}\pcmn{爸爸出去了!}\hspace{5pt}\pfra{Papa est sorti! (Adressé à une petite fille qui vient de se réveiller de sa sieste et cherche son père.)}\end{exemple}
\begin{exemple}\pnru{ə˩pʰo˩-bv̩˥ | lo˧ ʝi˧}\hspace{5pt}\peng{to work outside the household: to help other families (in particular for harvesting); to go and look for work in a city}\hspace{5pt}\pcmn{在外边工作:去帮别人家的忙(特别是收庄稼的时候),或者到城市里面打工}\hspace{5pt}\pfra{travailler à l'extérieur, aider d'autres familles (par exemple pour la récolte); aussi : aller chercher du travail à la ville}\end{exemple}
\end{entrée}

\begin{entrée}
{ɑ˩pʰo˩-hĩ˩}{}{ⓔɑ˩pʰo˩-hĩ˩}\formedesurface{ɑ˩pʰo˩-hĩ˩˥}\newline
\classe{名词}\ton{L}\begin{définition}\peng{Outsiders; strangers; other people.}\end{définition}
\begin{définition}\pcmn{外人,别人。当地人称从外地来的人为“外人”。}\end{définition}
\begin{définition}\pfra{Quelqu'un d'autre; une personne extérieure à la famille.}\end{définition}
\end{entrée}

\begin{entrée}
{ɑ˩pʰv̩˧˥}{}{ⓔɑ˩pʰv̩˧˥}\formedesurface{ɑ˩pʰv̩˧˥}\newline
\classe{动词}\ton{LM+MH\#}\begin{définition}\peng{To belch, to burp.}\end{définition}
\begin{définition}\pcmn{打饱嗝儿}\end{définition}
\begin{définition}\pfra{Roter.}\end{définition}
\end{entrée}

\begin{entrée}
{ɑ˩pʰv̩\#˥}{}{ⓔɑ˩pʰv̩\#˥}\formedesurface{ɑ˩pʰv̩˥}\newline
\classe{名词}\ton{LM+\#H}
\paradigme{\pcmn{:} \p{}}
\begin{définition}\peng{Gander; male goose.}\end{définition}
\begin{définition}\pcmn{公鹅}\end{définition}
\begin{définition}\pfra{Jars: mâle de l'oie.}\end{définition}
\end{entrée}

\begin{entrée}
{ɑ˩ʁo˧}{}{ⓔɑ˩ʁo˧}\formedesurface{ɑ˩ʁo˥}\newline
\classe{名词}\ton{LM}
\paradigme{\pcmn{:} \p{}}
\begin{définition}\peng{Home, central room in the house.}\end{définition}
\begin{définition}\pcmn{家、家里}\end{définition}
\begin{définition}\pfra{Le foyer, la pièce principale; la maison.}\end{définition}
\begin{exemple}\pnru{ɑ˩ʁo˧-hĩ\#˥}\hspace{5pt}\peng{members of the family, family members (who live under the same roof), lineage}\hspace{5pt}\pcmn{家人(住在一起的家人)}\hspace{5pt}\pfra{la maisonnée, les gens de la famille (proche: ceux qui habitent sous le même toit), la lignée}\end{exemple}
\begin{exemple}\pnru{ɑ˩ʁo˧=ɻæ˩}\hspace{5pt}\peng{the members of the family, the family group (living under the same roof)}\hspace{5pt}\pcmn{家人、家族(住在一起的家人)}\hspace{5pt}\pfra{la maisonnée, les gens de la famille}\end{exemple}
\begin{exemple}\pnru{njɤ˧ | ɑ˩ʁo˧}\hspace{5pt}\peng{my home, my house}\hspace{5pt}\pcmn{我家}\hspace{5pt}\pfra{mon foyer, ma maison}\end{exemple}
\begin{exemple}\pnru{njɤ˧ | ɑ˩ʁo˧=ɻ̍˩}\hspace{5pt}\peng{my family, my lineage, my kin}\hspace{5pt}\pcmn{我的家族}\hspace{5pt}\pfra{ma famille, ma lignée}\end{exemple}
\begin{exemple}\pnru{no˧ | ɑ˩ʁo˧}\hspace{5pt}\peng{your home, your house}\hspace{5pt}\pcmn{你家}\hspace{5pt}\pfra{ton foyer, ta maison}\end{exemple}
\begin{exemple}\pnru{ɑ˩ʁo˧ ʝi˧}\hspace{5pt}\peng{to take care of the household, to look after the affairs of the family (in particular: distributing work to the various members, and ensuring that the supplies are not running low)}\hspace{5pt}\pcmn{在家里做事情、管家(如:分配工作、做家务事等)}\hspace{5pt}\pfra{gérer la maisonnée, s'occuper de la famille (tâche de la personne qui répartit les travaux à accomplir, veille aux approvisionnements…)}\end{exemple}
\end{entrée}

\begin{entrée}
{ɑ˩zo\#˥}{}{ⓔɑ˩zo\#˥}\formedesurface{ɑ˩zo˥}\newline
\classe{名词}\ton{LM+\#H}\begin{définition}\peng{Gosling, baby goose.}\end{définition}
\begin{définition}\pcmn{小鹅}\end{définition}
\begin{définition}\pfra{Oison, petit de l'oie.}\end{définition}
\end{entrée}

\newpage\caractère{æ}

\begin{entrée}
{æ˧bæ˧}{}{ⓔæ˧bæ˧}\formedesurface{æ˧bæ˧}\newline
\classe{名词}\ton{M}
\paradigme{\pcmn{:} \p{}}
\begin{définition}\peng{Goitre.}\end{définition}
\begin{définition}\pcmn{甲状腺肿瘤}\end{définition}
\begin{définition}\pfra{Goître.}\end{définition}
\end{entrée}

\begin{entrée}
{æ˧bæ˧-ʈʂʰæ˧ɣɯ\#˥}{}{ⓔæ˧bæ˧-ʈʂʰæ˧ɣɯ\#˥}\formedesurface{æ˧bæ˧ʈʂʰæ˧ɣɯ˧}\newline
\classe{名词}\ton{\#H}\begin{définition}\peng{Kelp (literally “medicine against goiter", because kelp was introduced in Yongning as a means to provide iodine as a diet supplement, to prevent goiters).}\end{définition}
\begin{définition}\pcmn{海带(直译:“治甲状腺肿瘤的药”)。迂说法:一般用海带来防治甲状腺肿瘤。}\end{définition}
\begin{définition}\pfra{Algue; littéralement «médicament contre le goître» car tel était le motif de la diffusion à Yongning de l'algue, qui contient de l'iode.}\end{définition}
\end{entrée}

\begin{entrée}
{æ˩gv̩˩}{}{ⓔæ˩gv̩˩}\formedesurface{æ˩gv̩˩˥}\newline
\classe{名词}\ton{L}
\paradigme{\pcmn{:} \p{}}
\begin{définition}\peng{Ard. There are no distinct words for ‘ard' and ‘plough'; only ards were in use at the time of fieldwork.}\end{définition}
\begin{définition}\pcmn{犁头}\end{définition}
\begin{définition}\pfra{Araire. Il n'existe pas deux termes distincts, l'un pour l'araire et l'autre pour la charrue; à la date de l'enquête, seule l'araire était en usage.}\end{définition}
\begin{exemple}\pnru{æ˩gv̩˩ tʰv̩˩-nɑ˥}\hspace{5pt}\peng{|fg{n}+|fg{dem}+|fg{clf}}\hspace{5pt}\pcmn{这把犁头}\hspace{5pt}\pfra{|fg{n}+|fg{dem}+|fg{clf}}\end{exemple}
\begin{exemple}\pnru{æ˩mo˥}\hspace{5pt}\peng{used ard, plough which cannot be used anymore}\hspace{5pt}\pcmn{陈旧的犁架子(不能再用了)}\hspace{5pt}\pfra{araire usagée, vieille araire (hors d'usage du fait de l'usure)}\end{exemple}
\begin{exemple}\pnru{æ˩mo˥ tʰv̩˩-nɑ˩}\hspace{5pt}\peng{|fg{n}+|fg{dem}+|fg{clf}}\hspace{5pt}\pcmn{那个旧犁架子(犁杆)}\hspace{5pt}\pfra{|fg{n}+|fg{dem}+|fg{clf}}\end{exemple}
\begin{exemple}\pnru{æ˩-ʂɯ˩˥}\hspace{5pt}\peng{new ard, brand new ard}\hspace{5pt}\pcmn{新的犁架子}\hspace{5pt}\pfra{araire neuve}\end{exemple}
\end{entrée}

\begin{entrée}
{æ˩gv̩˩-mæ˩qo˥}{}{ⓔæ˩gv̩˩-mæ˩qo˥}\formedesurface{æ˩gv̩˩mæ˩qo˥}\newline
\classe{名词}\ton{L+H\#}
\paradigme{\pcmn{:} \p{}}
\begin{définition}\peng{Handle (stilt) of the ard, used to control the ard's direction and the furrow's depth.}\end{définition}
\begin{définition}\pcmn{犁把}\end{définition}
\begin{définition}\pfra{Mancheron de l'araire, manche de l'araire.}\end{définition}
\begin{exemple}\pnru{æ̃˩gv̩˩-mæ˩ ʑi˩-hĩ˥}\hspace{5pt}\peng{the person holding the handle of the ard}\hspace{5pt}\pcmn{抓着犁把的人}\hspace{5pt}\pfra{la personne qui tient le mancheron de la charrue}\end{exemple}
\begin{exemple}\pnru{æ̃˩gv̩˩-mæ˩qo˥ tʰv̩˩-nɑ˩}\hspace{5pt}\peng{|fg{n}+|fg{dem}+|fg{clf}}\hspace{5pt}\pcmn{这个犁把}\hspace{5pt}\pfra{|fg{n}+|fg{dem}+|fg{clf}}\end{exemple}
\end{entrée}

\begin{entrée}
{æ˧-hi˩hi˩}{}{ⓔæ˧-hi˩hi˩}\formedesurface{æ˧hi˩hi˩}\newline
\classe{感叹词}\ton{-L}\begin{définition}\peng{Cry of defiance or provocation. It can also express triumph and exultation.}\end{définition}
\begin{définition}\pcmn{啊嘿嘿!(蔑视、挑衅或欢呼愉悦的叫声)}\end{définition}
\begin{définition}\pfra{Cri qui exprime défiance ou provocation, mais également triomphe ou exultation.}\end{définition}
\begin{exemple}\pnru{æ˧-hi˩hi˩ kʰɯ˩}\hspace{5pt}\peng{to shout a cry of defiance or provocation}\hspace{5pt}\pcmn{喊出蔑视、挑衅或欢呼的叫声}\hspace{5pt}\pfra{lancer des cris de défiance}\end{exemple}
\end{entrée}

\begin{entrée}
{æ˧ʝi˧}{}{ⓔæ˧ʝi˧}\formedesurface{æ˧ʝi˧}\newline
\classe{感叹词}\ton{M}\begin{définition}\peng{Interjection: Ah nooon! / Ca va pas!}\end{définition}
\begin{définition}\pcmn{感叹词。表示劝阻语气(不好的方面),比如,两个人在谈话中,其中甲告诉乙自己准备买一件大衣,乙觉得甲非常不适合那款衣服,就会说“æ˧ʝi˧”,那个衣服不适合你。再比如,婴儿抓到脏东西(如泥巴)就往嘴里放,大人就会说“|fv{æ˧ʝi˧}”,脏}\end{définition}
\begin{définition}\pfra{Interjection.}\end{définition}
\end{entrée}

\begin{entrée}
{æ˧ʝi˩}{}{ⓔæ˧ʝi˩}\formedesurface{æ˧ʝi˩}\newline
\classe{名词}\ton{L\#}\begin{définition}\peng{Cry.}\end{définition}
\begin{définition}\pcmn{鸣声}\end{définition}
\begin{définition}\pfra{Cri.}\end{définition}
\begin{exemple}\pnru{æ˧ʝi˩ kʰɯ˩}\hspace{5pt}\peng{to shout}\hspace{5pt}\pcmn{叫}\hspace{5pt}\pfra{crier}\end{exemple}
\begin{exemple}\pnru{no˧ | æ˧ʝi˩ tʰɑ˩-kʰɯ˩! | no˧ se˧dʑæ˩ɻæ˩-gv̩˩! |}\hspace{5pt}\peng{Don't make noise! You are annoying everyone!}\hspace{5pt}\pcmn{请你不要大声/别那么大声!你就是叽叽喳喳!(你在麻烦大家、你很烦人)}\hspace{5pt}\pfra{Ne fais pas tant de bruit, tu embêtes le monde! (Contexte: réprimande qu'on adressait aux gens qui parlaient trop fort, qui élevaient la voix.)}\end{exemple}
\end{entrée}

\begin{entrée}
{æ˩mi˧-mv̩˧ʈv̩˥}{}{ⓔæ˩mi˧-mv̩˧ʈv̩˥}\formedesurface{æ˩mi˧mv̩˧ʈv̩˥}\newline
\classe{名词}\ton{LM+H\#}
\paradigme{\pcmn{:} \p{}}
\begin{définition}\peng{Anklebone, bone of the top of the foot.}\end{définition}
\begin{définition}\pcmn{踝骨}\end{définition}
\begin{définition}\pfra{Astragale (aussi appelé talus): os du tarse qui s'articule avec le tibia; os saillant sur le côté du pied.}\end{définition}
\end{entrée}

\begin{entrée}
{æ˩mi˧-ʁwɤ\#˥}{}{ⓔæ˩mi˧-ʁwɤ\#˥}\formedesurface{æ˩mi˧ʁwɤ˧}\newline
\classe{名词}\ton{LM+\#H}\begin{définition}\peng{Amiwa. This is the first village along the road from \stylefv{/qʰæ}˧tɕʰi˧/ to \stylefv{/ʈʂo}˧ʂɯ\#˥/. In traditional Na geography, which takes Lake Lugu as a point of origin, Amiwa is the third village of the plain.}\end{définition}
\begin{définition}\pcmn{阿咪瓦、阿米瓦(永宁坝子的一个村落)}\end{définition}
\begin{définition}\pfra{Amiwa, premier village que l'on rencontre sur la route entre \stylefv{/qʰæ}˧tɕʰi˧/ et \stylefv{/ʈʂo}˧ʂɯ\#˥/. Dans la géographie traditionnelle na, qui prend comme point d'origine le lac Lugu, Amiwa est le troisième village de la plaine de Yongning.}\end{définition}
\begin{exemple}\pnru{dʑɤ˩bv̩˧kɤ˧-sɑ˥ʁwɤ˩, | hi˩ʁwɤ˩-lo˥, | æ˩mi˧-ʁwɤ\#˥, | lɑ˧lo˧-ʁwɤ˥, | lɑ˧ŋwɤ˧, | bɤ˧tsʰo˧gv̩˥, | ə˧lɑ˧-ʁwɤ\#˥, | gæ˧ɻæ˩, | qʰæ˧tɕʰi˧, | tʰo˧ʈɯ\#˥}\hspace{5pt}\peng{The ten villages traditionally considered as part of Yongning.}\hspace{5pt}\pcmn{摩梭传统地理概念中,属于永宁的十个村落:佳部嘎萨瓦、习瓦洛、阿咪瓦、拉洛瓦、拉瓦、巴搓古、阿拉瓦、嘎拉、开基、拖支}\hspace{5pt}\pfra{Les dix villages comptant traditionnellement comme faisant partie de Yongning.}\end{exemple}
\end{entrée}

\begin{entrée}
{æ˩mo˧}{}{ⓔæ˩mo˧}\formedesurface{æ˩mo˥}\newline
\classe{名词}\ton{LM}
\paradigme{\pcmn{:} \p{}}
\begin{définition}\peng{Beam of the ard, pole of the ard: a long piece of wood linking the yoke to the sole.}\end{définition}
\begin{définition}\pcmn{犁杆}\end{définition}
\begin{définition}\pfra{Timon (âge, haie) de l'araire.}\end{définition}
\begin{exemple}\pnru{æ˩gv̩˩-mo˥}\hspace{5pt}\peng{same meaning}\hspace{5pt}\pcmn{同上}\hspace{5pt}\pfra{même sens}\end{exemple}
\end{entrée}

\begin{entrée}
{æ˧ɲi\#˥}{}{ⓔæ˧ɲi\#˥}\formedesurface{æ˧ɲi˧}\newline
\classe{名词}\ton{\#H}
\paradigme{\pcmn{:} \p{}}
\begin{définition}\peng{Suona, trumpet.}\end{définition}
\begin{définition}\pcmn{唢呐}\end{définition}
\begin{définition}\pfra{Clarinette.}\end{définition}
\end{entrée}

\begin{entrée}
{æ˩pʰæ˧˥}{}{ⓔæ˩pʰæ˧˥}\formedesurface{æ˩pʰæ˧˥}\newline
\classe{名词}\ton{LM+MH\#}
\paradigme{\pcmn{:} \p{}}
\begin{définition}\peng{Cliff, overhanging cliff. This term designates the top of the cliff: the relatively flat ground close to the precipice. To refer to the steep (vertical) side of the cliff, one adds \stylefv{/lɑ}˧bi˧/ ‘steep slope'.}\end{définition}
\begin{définition}\pcmn{岩石、悬崖、崖山、崖壁}\end{définition}
\begin{définition}\pfra{Falaise. Le terme désigne spécifiquement le dos d'une falaise: l'espace relativement plat en bord de précipice. Pour faire référence à la paroi (face verticale de la falaise), on ajoute \stylefv{/lɑ}˧bi˧/ ‘escarpement'.}\end{définition}
\begin{exemple}\pnru{æ˩pʰæ˧-lɑ˧bi˥}\hspace{5pt}\peng{The steep (vertical) side of the cliff.}\hspace{5pt}\pcmn{岩石陡峭面}\hspace{5pt}\pfra{La paroi d'une falaise.}\end{exemple}
\end{entrée}

\begin{entrée}
{æ˩qʰv̩˥}{}{ⓔæ˩qʰv̩˥}\formedesurface{æ˩qʰv̩˥}\newline
\classe{名词}\ton{LH}
\paradigme{\pcmn{:} \p{}}
\begin{définition}\peng{Cave, cavern, crevice (difficult to enter, or too small for a person to enter).}\end{définition}
\begin{définition}\pcmn{小山洞(很难钻进去,或者钻不进去的山洞)}\end{définition}
\begin{définition}\pfra{Crevasse, petite grotte (où il est difficile de pénétrer).}\end{définition}
\end{entrée}

\begin{entrée}
{æ˧ʁwæ˧}{}{ⓔæ˧ʁwæ˧}\formedesurface{æ˧ʁwæ˧}\newline
\classe{名词}\ton{M}
\paradigme{\pcmn{:} \p{}}
\begin{définition}\peng{Apricot.}\end{définition}
\begin{définition}\pcmn{杏}\end{définition}
\begin{définition}\pfra{Abricot.}\end{définition}
\begin{exemple}\pnru{æ˧ʁwæ˧ | ɖɯ˧-ɭɯ˧}\hspace{5pt}\peng{an apricot}\hspace{5pt}\pcmn{一颗杏}\hspace{5pt}\pfra{un abricot}\end{exemple}
\end{entrée}

\begin{entrée}
{æ˧ʂæ˧}{}{ⓔæ˧ʂæ˧}\formedesurface{æ˧ʂæ˧}\newline
\classe{名词}\ton{M}\begin{définition}\peng{Name of a mountain: one of the two main mountains in the vicinity of the Yongning plain. It is a masculine mountain ("the young man": \stylefv{/pʰæ}˧tɕi˥/), the counterpart to the feminine mountain \stylefv{/kɤ}˧mv̩˧˥/ ("the young woman": \stylefv{/mi}˩zɯ˩˥/).}\end{définition}
\begin{définition}\pcmn{一座山的名字:安山。位于永宁坝的西面,格姆女神山的对面。}\end{définition}
\begin{définition}\pfra{Nom d'une montagne: l'une des deux principales montagnes autour de la plaine de Yongning, la montagne masculine («le jeune homme»: \stylefv{/pʰæ}˧tɕi˥/); l'autre étant la montagne \stylefv{/kɤ}˧mv̩˧˥/, montagne féminine («la jeune femme»: \stylefv{/mi}˩zɯ˩˥/).}\end{définition}
\begin{exemple}\pnru{kɤ˧mv̩˧˥, | æ˧ʂæ˧, | ŋwɤ˧hɑ̃˩, | ʂwæ˧gv̩\#˥, | nɑ˩tsʰi˩˥ | -tɕʰɤ˧pɤ˧mi\#˥, | qv̩˧ɻ̍˧-ʈʂʰɑ˧nɑ˥ |}\hspace{5pt}\peng{The six mountains of Yongning that carry a name and have a definite symbolic value. The other mountains do not have comparable symbolic value, and fewer people use specific names for them.}\hspace{5pt}\pcmn{永宁地区有固定名字的六座山:格姆,安山,瓦哈,双古,纳慈巧吧咪,古尔川纳。}\hspace{5pt}\pfra{Les six montagnes de Yongning qui portent un nom. Les autres sommets du voisinage n'ont pas une valeur symbolique comparable, et ne portent pas de nom communément utilisé.}\end{exemple}
\end{entrée}

\begin{entrée}
{æ˧ʂæ\#˥}{}{ⓔæ˧ʂæ\#˥}\formedesurface{æ˧ʂæ˧}\newline
\classe{助词}\ton{\#H}\begin{définition}\peng{Yore, long ago.}\end{définition}
\begin{définition}\pcmn{古时候,从前}\end{définition}
\begin{définition}\pfra{Jadis, dans le passé.}\end{définition}
\begin{exemple}\pnru{æ˧ʂæ˧-kɤ˥ʈʂɯ˩}\hspace{5pt}\peng{Sayings of the old times, oral traditions of the old times}\hspace{5pt}\pcmn{古时候的说法、从前的说法}\hspace{5pt}\pfra{Dires du temps jadis, traditions orales}\end{exemple}
\end{entrée}

\begin{entrée}
{æ˧ʂæ˧-pi˧mv̩˧˥}{}{ⓔæ˧ʂæ˧-pi˧mv̩˧˥}\formedesurface{æ˧ʂæ˧pi˧mv̩˧˥}\newline
\classe{名词}\ton{-MH\#}
\étymologie{
æ˧ʂæ\#˥; pi˧mv̩˥\$
}
\paradigme{\pcmn{:} \p{}}
\begin{définition}\peng{Folk tale, tradition; this term is more colloquial than \stylefv{/æ}˧ʂæ˧-tɑ˩mv̩˩/.}\end{définition}
\begin{définition}\pcmn{古时候的说法、从前的说法,传统故事}\end{définition}
\begin{définition}\pfra{Conte, récit du temps jadis; terme plus familier que \stylefv{/æ}˧ʂæ˧-tɑ˩mv̩˩/.}\end{définition}
\end{entrée}

\begin{entrée}
{æ˧ʂæ˧-qʰwæ\#˥}{}{ⓔæ˧ʂæ˧-qʰwæ\#˥}\formedesurface{æ˧ʂæ˧qʰwæ˧}\newline
\classe{名词}\ton{\#H}
\étymologie{
æ˧ʂæ\#˥; qʰwæ˧
}\begin{définition}\peng{Oral tradition; literally: “messages from old times".}\end{définition}
\begin{définition}\pcmn{古时候的传说。直译:“(来自)古时候的寓意”}\end{définition}
\begin{définition}\pfra{Tradition orale; littéralement: «messages du temps jadis».}\end{définition}
\begin{exemple}\pnru{æ˧ʂæ˧-qʰwæ˧-ɳɯ˥ | dʑo˩-ɲi˥!}\hspace{5pt}\peng{“I'm not making this up:) this is part of what the old folks have passed down to us! / This is what our traditions say!" (Context: the speaker cites a proverb or saying, and emphasizes that it is to be taken seriously.)}\hspace{5pt}\pcmn{“(这些道理,不是我个人的意见:)传统中是这样讲的! / 咱们的口传文化中就是这么讲的!”(情景:一个人提到一个谚语,强调这是有根据、有源头的,是重要的一个道理。)}\hspace{5pt}\pfra{«(C'est pas moi qui invente ça:) c'est un dicton d'autrefois/ c'est quelque chose qui existe dans la tradition!» (Commentaire de quelqu'un qui cite un proverbe/dicton, et souligne qu'il ne s'agit pas de paroles en l'air, mais de vérités.)}\end{exemple}
\end{entrée}

\begin{entrée}
{æ˧ʂæ˧-qʰwɤ˧˥}{}{ⓔæ˧ʂæ˧-qʰwɤ˧˥}\formedesurface{æ˧ʂæ˧qʰwɤ˧˥}\newline
\classe{名词}\ton{MH\#}
\paradigme{\pcmn{:} \p{}}
\begin{définition}\peng{Story, folk tale.}\end{définition}
\begin{définition}\pcmn{故事}\end{définition}
\begin{définition}\pfra{Histoire, conte, récit traditionnel.}\end{définition}
\begin{exemple}\pnru{æ˧ʂæ˧qʰwɤ˧ ʐwɤ˧˥}\hspace{5pt}\peng{to tell a story}\hspace{5pt}\pcmn{讲故事}\hspace{5pt}\pfra{raconter une histoire}\end{exemple}
\begin{exemple}\pnru{ə˧ʝi˧-ʂɯ˥ʝi˩, | æ˧ʂæ˧qʰwɤ˧ ʐwɤ˧-kv̩˥!}\hspace{5pt}\peng{In the old times, (people) used to tell stories!}\hspace{5pt}\pcmn{在过去,大家经常讲故事!}\hspace{5pt}\pfra{Dans le temps, (on) racontait des histoires!}\end{exemple}
\end{entrée}

\begin{entrée}
{æ˧ʂæ˧-tɑ˩mv̩˩}{}{ⓔæ˧ʂæ˧-tɑ˩mv̩˩}\formedesurface{æ˧ʂæ˧tɑ˩mv̩˩}\newline
\classe{名词}\ton{-L}
\étymologie{
æ˧ʂæ\#˥; tɑ˩mv̩˩
}
\paradigme{\pcmn{:} \p{}}
\begin{définition}\peng{Proverb, saying, adage.}\end{définition}
\begin{définition}\pcmn{谚语、习语}\end{définition}
\begin{définition}\pfra{Proverbe, dicton.}\end{définition}
\begin{exemple}\pnru{æ˧ʂæ˧-tɑ˩mv̩˩ tʰv̩˩-kʰwɤ˩}\hspace{5pt}\peng{that proverb}\hspace{5pt}\pcmn{那个习语}\hspace{5pt}\pfra{ce dicton}\end{exemple}
\end{entrée}

\begin{entrée}
{æ˧tse˥-pʰæ˩}{}{ⓔæ˧tse˥-pʰæ˩}\formedesurface{æ˧tse˥pʰæ˩}\newline
\classe{名词}\ton{H\#-L}
\paradigme{\pcmn{:} \p{}}
\begin{définition}\peng{Kneebone.}\end{définition}
\begin{définition}\pcmn{膝盖骨}\end{définition}
\begin{définition}\pfra{Os du genou.}\end{définition}
\end{entrée}

\begin{entrée}
{æ˧tsɯ˥-pɤ˩lv̩˩}{}{ⓔæ˧tsɯ˥-pɤ˩lv̩˩}\formedesurface{æ˧tsɯ˥pɤ˩lv̩˩}\newline
\classe{名词}\ton{H\#-L}
\paradigme{\pcmn{:} \p{}}
\begin{définition}\peng{Nape of neck.}\end{définition}
\begin{définition}\pcmn{项、脖颈儿}\end{définition}
\begin{définition}\pfra{Nuque.}\end{définition}
\end{entrée}

\begin{entrée}
{æ˩ʈv̩˥}{}{ⓔæ˩ʈv̩˥}\formedesurface{æ˩ʈv̩˥}\newline
\classe{名词}\ton{LH}
\paradigme{\pcmn{:} \p{}}
\begin{définition}\peng{Large rock.}\end{définition}
\begin{définition}\pcmn{一大片大岩石}\end{définition}
\begin{définition}\pfra{Gros rocher, roc.}\end{définition}
\end{entrée}

\begin{entrée}
{æ˧ʈwɤ˩}{}{ⓔæ˧ʈwɤ˩}\formedesurface{æ˧ʈwɤ˩}\newline
\classe{名词}\ton{L\#}\begin{définition}\peng{The early morning; early in the morning.}\end{définition}
\begin{définition}\pcmn{凌晨、一大早(鸡叫的时候)}\end{définition}
\begin{définition}\pfra{Le petit matin.}\end{définition}
\begin{exemple}\pnru{æ˧ʈwɤ˩-mv̩˩kʰv̩˩}\hspace{5pt}\peng{From the early morning until nightfall}\hspace{5pt}\pcmn{从凌晨到傍晚}\hspace{5pt}\pfra{Du petit matin jusqu'à la tombée de la nuit.}\end{exemple}
\end{entrée}

\begin{entrée}
{æ̃˥}{}{ⓔæ̃˥}\formedesurface{æ̃˧}\newline
\classe{名词}\ton{H}\begin{définition}\peng{Brass, copper, bronze.}\end{définition}
\begin{définition}\pcmn{铜,包括黄铜、红铜、青铜}\end{définition}
\begin{définition}\pfra{Cuivre; bronze.}\end{définition}
\begin{exemple}\pnru{æ̃˧tso˧-æ̃˧mo˩}\hspace{5pt}\peng{instruments and objects made of brass}\hspace{5pt}\pcmn{铜做的工具、物品}\hspace{5pt}\pfra{instruments en cuivre, objets en cuivre}\end{exemple}
\begin{exemple}\pnru{æ̃˧ lɑ˩-zo˩-ɳɯ˩, | ʂe˧ mɤ˧-lɑ˧˥!}\hspace{5pt}\peng{“The man who works copper does not work iron!" These two specialties require different skills: physical strength for working iron; and delicate gestures for working copper. This saying is used to point out that each person has her/his own area of expertise.}\hspace{5pt}\pcmn{“打铜的人,不打铁!”这两种工作需要不同的能力:打铁的师傅需要体力,打铜的师傅需要技巧。这句谚语意指:每个人都有他的专攻(不能随便跨越到其它领域)。}\hspace{5pt}\pfra{«Celui qui travaille le cuivre, il ne doit pas travailler le fer/on ne doit pas lui confier de tâches de forgeron (=travail du fer)!» Ces deux spécialités demandent des qualités différentes: de la force physique pour le travail du fer, et du soin pour le travail du cuivre. Le dicton s'emploie pour souligner que chacun a son domaine de compétence.}\end{exemple}
\end{entrée}

\begin{entrée}
{æ̃˩˥}{}{ⓔæ̃˩˥}\formedesurface{æ̃˩˥}\newline
\classe{名词}\ton{LH}
\paradigme{\pcmn{:} \p{}}
\begin{définition}\peng{Soul (monosyllable).}\end{définition}
\begin{définition}\pcmn{灵魂}\end{définition}
\begin{définition}\pfra{Âme (monosyllabe).}\end{définition}
\end{entrée}

\begin{entrée}
{æ̃˩˧}{}{ⓔæ̃˩˧}\formedesurface{æ̃˩˥}\newline
\classe{名词}\ton{LM}
\paradigme{\pcmn{:} \p{}}
\begin{définition}\peng{Chicken.}\end{définition}
\begin{définition}\pcmn{鸡}\end{définition}
\begin{définition}\pfra{Poulet, poule.}\end{définition}
\begin{exemple}\pnru{æ̃˩ dzɯ˥-ze˩}\hspace{5pt}\peng{…has eaten (a/some) chicken}\hspace{5pt}\pcmn{吃了鸡}\hspace{5pt}\pfra{…a mangé (un/du) poulet}\end{exemple}
\begin{exemple}\pnru{æ̃˩ hwæ˧-ze˧}\hspace{5pt}\peng{…has bought (a) chicken}\hspace{5pt}\pcmn{买了鸡}\hspace{5pt}\pfra{…a acheté (un/du) poulet}\end{exemple}
\begin{exemple}\pnru{æ̃˩˥, | kʰv̩˧, | bo˩˥, | hwɤ˧˥, | ʝi˧, | lɑ˧, | tʰo˧li˧, | mv̩˧gv̩˧, | bv̩˧ʐv̩˧, | ʐwæ˧, | jo˧, | ʑi˩˥}\hspace{5pt}\peng{the twelve years of the duodenary cycle}\hspace{5pt}\pcmn{十二个生肖:鸡肉、狗、猪、鼠、牛、虎、兔、龙、蛇、马、羊、猴}\hspace{5pt}\pfra{les douze signes astrologiques}\end{exemple}
\begin{exemple}\pnru{æ̃˩-mɤ˥}\hspace{5pt}\peng{chicken grease, chicken fat}\hspace{5pt}\pcmn{鸡油}\hspace{5pt}\pfra{graisse de poulet}\end{exemple}
\begin{exemple}\pnru{æ̃˩-mɤ˥ dzɯ˩}\hspace{5pt}\peng{to eat chicken fat}\hspace{5pt}\pcmn{吃鸡油}\hspace{5pt}\pfra{manger de la graisse de poulet}\end{exemple}
\end{entrée}

\begin{entrée}
{æ̃˩α}{}{ⓔæ̃˩α}\formedesurface{ɖɯ˧ æ̃˩}\newline
\classe{量词}\ton{Lα}\begin{définition}\peng{Classifier for fires.}\end{définition}
\begin{définition}\pcmn{量词:火(一团)}\end{définition}
\begin{définition}\pfra{Classificateur des feux.}\end{définition}
\begin{exemple}\pnru{mv̩˧ | ʈʂʰɯ˧-æ̃˥}\hspace{5pt}\peng{this fire (tone: H\# / H\$)}\hspace{5pt}\pcmn{这团火}\hspace{5pt}\pfra{ce feu (ton: H\# / H\$)}\end{exemple}
\end{entrée}

\begin{entrée}
{æ̃˩α}{₁}{ⓔæ̃˩αⓗ1}\formedesurface{æ̃˩˥}\newline
\classe{动词}\ton{Lα}
1\begin{définition}\peng{To reflect (a mirror reflects light).}\end{définition}
\begin{définition}\pcmn{反射、辉映}\end{définition}
\begin{définition}\pfra{Réfléchir, renvoyer (un miroir renvoie la lumière; une cloison/toiture étanche renvoie la pluie =est étanche à la pluie).}\end{définition}
\end{entrée}

\begin{entrée}
{æ̃˩α}{₂}{ⓔæ̃˩αⓗ2}\formedesurface{æ̃˩˥}\newline
\classe{动词}\ton{Lα}
2\begin{définition}\peng{To get stuck.}\end{définition}
\begin{définition}\pcmn{堵塞、塞}\end{définition}
\begin{définition}\pfra{S'enliser; se coincer, se bloquer.}\end{définition}
\begin{exemple}\pnru{ʝi˩mi˩˥ | ɖʐæ˩qʰæ˧-qo˩ æ̃˩!}\hspace{5pt}\peng{The cow is stuck in the mud.}\hspace{5pt}\pcmn{牛陷在泥巴里。}\hspace{5pt}\pfra{La vache est enlisée dans la boue.}\end{exemple}
\end{entrée}

\begin{entrée}
{æ̃˩bi˩}{}{ⓔæ̃˩bi˩}\formedesurface{æ̃˩bi˩˥}\newline
\classe{名词}\ton{L}\begin{définition}\peng{A village just over the river on the Sichuan side of road to Qiansuo.}\end{définition}
\begin{définition}\pcmn{阿比村:从阿拉瓦村到前所乡路上经过的一个村落}\end{définition}
\begin{définition}\pfra{Abi: village sur le chemin de Qiansuo.}\end{définition}
\begin{exemple}\pnru{æ̃˩bi˩-ʁwɤ˥}\hspace{5pt}\peng{same meaning: the village of /æ̃˩bi˩/}\hspace{5pt}\pcmn{同上}\hspace{5pt}\pfra{même sens: le village de /æ̃˩bi˩/}\end{exemple}
\begin{exemple}\pnru{æ̃˩bi˩-hĩ˥ ɲi˩!}\hspace{5pt}\peng{[(S)he] is from the village of /æ̃˩bi˩/!}\hspace{5pt}\pcmn{是阿比村的人!}\hspace{5pt}\pfra{C'est quelqu'un du village de /æ̃˩bi˩/!}\end{exemple}
\end{entrée}

\begin{entrée}
{æ̃˩bv̩˥}{}{ⓔæ̃˩bv̩˥}\formedesurface{æ̃˩bv̩˥}\newline
\classe{名词}\ton{LH}
\paradigme{\pcmn{:} \p{}}
\begin{définition}\peng{Poultry yard.}\end{définition}
\begin{définition}\pcmn{鸡圈}\end{définition}
\begin{définition}\pfra{Poulailler.}\end{définition}
\end{entrée}

\begin{entrée}
{æ̃˩-kʰv̩˧˥}{₁}{ⓔæ̃˩-kʰv̩˧˥ⓗ1}\formedesurface{æ̃˩kʰv̩˧˥}\newline
\classe{名词}\ton{LM+MH\#}
1\begin{définition}\peng{Year of the Rooster.}\end{définition}
\begin{définition}\pcmn{鸡年,属鸡}\end{définition}
\begin{définition}\pfra{Année du Coq.}\end{définition}
\end{entrée}

\begin{entrée}
{æ̃˩-kʰv̩˧˥}{₂}{ⓔæ̃˩-kʰv̩˧˥ⓗ2}\formedesurface{æ̃˩kʰv̩˧˥}\newline
\classe{形容词}\ton{LM+MH\#}
2\begin{définition}\peng{Born in the year of the Rooster.}\end{définition}
\begin{définition}\pcmn{属鸡(属相)}\end{définition}
\begin{définition}\pfra{Né l'année du Coq.}\end{définition}
\end{entrée}

\begin{entrée}
{æ̃˩li˧pʰæ˥}{}{ⓔæ̃˩li˧pʰæ˥}\formedesurface{æ̃˩li˧pʰæ˥}\newline
\classe{名词}\ton{LM+H\#}
\paradigme{\pcmn{:} \p{}}
\begin{définition}\peng{Mirror.}\end{définition}
\begin{définition}\pcmn{镜子}\end{définition}
\begin{définition}\pfra{Miroir.}\end{définition}
\end{entrée}

\begin{entrée}
{æ̃˩ɬi\#˥}{}{ⓔæ̃˩ɬi\#˥}\formedesurface{æ̃˩ɬi˥}\newline
\classe{名词}\ton{LM+\#H}
\paradigme{\pcmn{:} \p{}}
\begin{définition}\peng{Soul.}\end{définition}
\begin{définition}\pcmn{灵魂、魂魄}\end{définition}
\begin{définition}\pfra{Âme.}\end{définition}
\end{entrée}

\begin{entrée}
{æ̃˩mi˧}{}{ⓔæ̃˩mi˧}\formedesurface{æ̃˩mi˥}\newline
\classe{名词}\ton{LM}
\paradigme{\pcmn{:} \p{}}
\begin{définition}\peng{Hen.}\end{définition}
\begin{définition}\pcmn{母鸡}\end{définition}
\begin{définition}\pfra{Poule.}\end{définition}
\begin{exemple}\pnru{æ̃˩mi˧-æ̃˧ʂwæ˥\#}\hspace{5pt}\peng{hen and rooster}\hspace{5pt}\pcmn{母鸡与公鸡}\hspace{5pt}\pfra{poule et coq}\end{exemple}
\begin{exemple}\pnru{æ̃˩mi˧-æ̃˧tsɯ˥\#}\hspace{5pt}\peng{hen and chick}\hspace{5pt}\pcmn{母鸡与稚鸡}\hspace{5pt}\pfra{poule et poussins}\end{exemple}
\end{entrée}

\begin{entrée}
{æ̃˧qæ˩}{₁}{ⓔæ̃˧qæ˩ⓗ1}\formedesurface{æ̃˧qæ˩}\newline
\classe{名词}\ton{L\#}
1
\paradigme{\pcmn{:} \p{}}
\begin{définition}\peng{Parrot.}\end{définition}
\begin{définition}\pcmn{鹦鹉}\end{définition}
\begin{définition}\pfra{Perroquet.}\end{définition}
\end{entrée}

\begin{entrée}
{æ̃˧qæ˩}{₂}{ⓔæ̃˧qæ˩ⓗ2}\formedesurface{æ̃˧qæ˩}\newline
\classe{形容词}\ton{L\#}
2\begin{définition}\peng{Blue-green; literally ‘parrot[-coloured]'.}\end{définition}
\begin{définition}\pcmn{像鹦鹉的颜色:介于青色、蓝色、绿色之间的颜色(摩梭话基本颜色为黄、红、青、黑、白、灰等)}\end{définition}
\begin{définition}\pfra{De couleur bleue/verte; couleur un peu plus légère que le vert de la plaine; équivalent du chinois \stylefn{青.} Littéralement: ‘[couleur] perroquet'.}\end{définition}
\begin{exemple}\pnru{æ̃˧qæ˩-ni˩gv̩˩}\hspace{5pt}\peng{vivid-coloured, blue-green; literally ‘like a parrot', i.e. ‘parrot-coloured'}\hspace{5pt}\pcmn{像鹦鹉的颜色:青、蓝色、绿色}\hspace{5pt}\pfra{couleur perroquet; littéralement ‘comme un perroquet'}\end{exemple}
\begin{exemple}\pnru{æ̃˧qæ˩-bɑ˩lɑ˩}\hspace{5pt}\peng{vivid-coloured, blue-green jacket: literally ‘parrot(-coloured) jacket'}\hspace{5pt}\pcmn{青、蓝色、绿色衣服}\hspace{5pt}\pfra{vêtement bleu; littéralement vêtement ‘couleur perroquet'}\end{exemple}
\begin{exemple}\pnru{æ̃˧qæ˩ni˩∼æ̃˧qæ˩ni˩gv̩˩}\hspace{5pt}\peng{|fg{red}; same meaning: blue-green}\hspace{5pt}\pcmn{重叠。同上:青色}\hspace{5pt}\pfra{|fg{red}; même sens: bleu-vert}\end{exemple}
\end{entrée}

\begin{entrée}
{æ̃˩ʁv̩˩}{}{ⓔæ̃˩ʁv̩˩}\formedesurface{æ̃˩ʁv̩˩˥}\newline
\classe{名词}\ton{L}
\paradigme{\pcmn{:} \p{}}
\begin{définition}\peng{Egg.}\end{définition}
\begin{définition}\pcmn{蛋}\end{définition}
\begin{définition}\pfra{Œuf.}\end{définition}
\begin{exemple}\pnru{bæ˧mi˧-æ̃˩ʁv̩˩}\hspace{5pt}\peng{cane egg}\hspace{5pt}\pcmn{鸭子蛋}\hspace{5pt}\pfra{œuf de cane}\end{exemple}
\begin{exemple}\pnru{æ̃˩ʁv̩˩ dzɯ˩˥}\hspace{5pt}\peng{to eat eggs}\hspace{5pt}\pcmn{吃蛋}\hspace{5pt}\pfra{manger des œufs}\end{exemple}
\end{entrée}

\begin{entrée}
{æ̃˩ʂe˩}{}{ⓔæ̃˩ʂe˩}\formedesurface{æ̃˩ʂe˩˥}\newline
\classe{名词}\ton{L}
\paradigme{\pcmn{:} \p{}}
\begin{définition}\peng{Muscle.}\end{définition}
\begin{définition}\pcmn{肌肉}\end{définition}
\begin{définition}\pfra{Muscle.}\end{définition}
\begin{exemple}\pnru{æ̃˩ʂe˩ tsʰi˩˥}\hspace{5pt}\peng{to run a temperature, to have a fever}\hspace{5pt}\pcmn{发烧}\hspace{5pt}\pfra{avoir la fièvre}\end{exemple}
\begin{exemple}\pnru{æ̃˩ʂe˩˥ | dʑɤ˩˥ | ʐwæ˩˥}\hspace{5pt}\peng{(He/she) is in great shape!}\hspace{5pt}\pcmn{身体状况很好!(包括气色好)}\hspace{5pt}\pfra{(Il) est en super forme!}\end{exemple}
\end{entrée}

\begin{entrée}
{æ̃˩ʂe˧li˥-mo˩}{}{ⓔæ̃˩ʂe˧li˥-mo˩}\formedesurface{æ̃˩ʂe˧li˥mo˩}\newline
\classe{名词}\ton{LM+H\#-}\begin{définition}\peng{“Chicken-meat mushroom": an edible mushroom, |\stylefi{Amanita spissa}.}\end{définition}
\begin{définition}\pcmn{麻母鸡菌:一种可以吃的菌子,块鳞灰毒鹅膏菌。以味鲜,类似于鸡肉味道而得名。}\end{définition}
\begin{définition}\pfra{«champignon viande-de-poulet»: un champignon comestible, |\stylefi{Amanita spissa}.}\end{définition}
\end{entrée}

\begin{entrée}
{æ̃˧ʂwæ˥}{}{ⓔæ̃˧ʂwæ˥}\formedesurface{æ̃˧ʂwæ˥}\newline
\classe{名词}\ton{H\#}
\paradigme{\pcmn{:} \p{}}
\begin{définition}\peng{Rooster.}\end{définition}
\begin{définition}\pcmn{公鸡}\end{définition}
\begin{définition}\pfra{Coq.}\end{définition}
\begin{exemple}\pnru{æ̃˧ʂwæ˥-æ̃˩mi˩}\hspace{5pt}\peng{cock and hen}\hspace{5pt}\pcmn{公鸡与母鸡}\hspace{5pt}\pfra{coq et poule}\end{exemple}
\end{entrée}

\begin{entrée}
{æ̃˧tsɯ˥}{}{ⓔæ̃˧tsɯ˥}\formedesurface{æ̃˧tsɯ˥}\newline
\classe{名词}\ton{H\#}
\paradigme{\pcmn{:} \p{}}
\begin{définition}\peng{Chick.}\end{définition}
\begin{définition}\pcmn{雏鸡、稚鸡}\end{définition}
\begin{définition}\pfra{Poussin.}\end{définition}
\end{entrée}

\begin{entrée}
{æ̃˧tsɯ˥-kʰɯ˩ʈʂɤ˩-mo˩}{}{ⓔæ̃˧tsɯ˥-kʰɯ˩ʈʂɤ˩-mo˩}\formedesurface{æ̃˧tsɯ˥kʰɯ˩ʈʂɤ˩mo˩}\newline
\classe{名词}\ton{H\#--}\begin{définition}\peng{“Chicken-claw mushroom": an edible mushroom.}\end{définition}
\begin{définition}\pcmn{扫把菌,扫帚菌(一种菌子),长得像鸡爪子的菌子。}\end{définition}
\begin{définition}\pfra{«champignon griffes-de-poulet»: champignon comestible.}\end{définition}
\end{entrée}

\begin{entrée}
{æ̃˧ʈwɤ˩-mv̩˩kʰv̩˩}{}{ⓔæ̃˧ʈwɤ˩-mv̩˩kʰv̩˩}\formedesurface{æ̃˧ʈwɤ˩mv̩˩kʰv̩˩}\newline
\classe{助词}\ton{L\#-L}\begin{définition}\peng{Constantly, all the time; literally: ‘[from] morning [till] evening'.}\end{définition}
\begin{définition}\pcmn{一直不停地,从早到晚。直译:‘(从)早上鸡叫的时候到(到)天黑’。}\end{définition}
\begin{définition}\pfra{Du matin au soir, constamment.}\end{définition}
\end{entrée}

\begin{entrée}
{æ̃˧-v̩\#˥}{}{ⓔæ̃˧-v̩\#˥}\formedesurface{æ̃˧v˧̩}\newline
\classe{名词}\ton{\#H}
\paradigme{\pcmn{:} \p{}}
\begin{définition}\peng{Copper pot.}\end{définition}
\begin{définition}\pcmn{铜锅}\end{définition}
\begin{définition}\pfra{Casserole en cuivre.}\end{définition}
\end{entrée}

\begin{entrée}
{æ̃˩zɯ˩}{}{ⓔæ̃˩zɯ˩}\formedesurface{æ̃˩zɯ˩˥}\newline
\classe{名词}\ton{L}
\paradigme{\pcmn{:} \p{}}
\begin{définition}\peng{Agate. Agate of various colours is used in ornamentation. Beads range from the size of a quail egg to that of a chicken's egg.}\end{définition}
\begin{définition}\pcmn{玛瑙}\end{définition}
\begin{définition}\pfra{Agate. Des perles d'agate de diverses couleurs sont utilisées en orfèvrerie. Elles sont de la taille d'un oeuf de caille, les plus gros approchent la taille d'un oeuf de poule. Les perles d'agate étaient intégrées aux bijoux et vêtements, dans une tradition d'inspiration tibétaine.}\end{définition}
\begin{exemple}\pnru{sɯ˧ɻ̍˧-æ̃˩zɯ˩}\hspace{5pt}\peng{pearl-shaped agate bead}\hspace{5pt}\pcmn{珠子形状的玛瑙}\hspace{5pt}\pfra{agate en forme de perle}\end{exemple}
\begin{exemple}\pnru{æ̃˩zɯ˩-ʂo˩∼ʂo˥}\hspace{5pt}\peng{with lots of agate on it (of a piece of clothing)}\hspace{5pt}\pcmn{(衣服上)都镶嵌着玛瑙}\hspace{5pt}\pfra{tout plein d'agate, plein de morceaux d'agate (d'un vêtement)}\end{exemple}
\end{entrée}

\newpage\caractère{b}

\begin{entrée}
{bɑ˩˥}{}{ⓔbɑ˩˥}\formedesurface{bɑ˩˥}\newline
\classe{语气助词}\ton{L?}\begin{définition}\peng{Affirmative final particle; comparable to question-tag in English.}\end{définition}
\begin{définition}\pcmn{句尾助词,表示肯定:“……是吧。”}\end{définition}
\begin{définition}\pfra{Particule finale affirmative: “…n'est-ce pas”.}\end{définition}
\end{entrée}

\begin{entrée}
{bɑ˩lɑ˩}{}{ⓔbɑ˩lɑ˩}\formedesurface{bɑ˩lɑ˩˥}\newline
\classe{名词}\ton{L}
\sens{1}\paradigme{\pcmn{:} \p{}}
\begin{définition}\peng{Jacket, upper outer garment; clothes.}\end{définition}
\begin{définition}\pcmn{上衣,衣服}\end{définition}
\begin{définition}\pfra{Chemise, veste; vêtement.}\end{définition}
\begin{exemple}\pnru{ɣɯ˩-bɑ˩lɑ˥ (+ɲi˩)}\hspace{5pt}\peng{leather jacket}\hspace{5pt}\pcmn{皮衣}\hspace{5pt}\pfra{veste de cuir}\end{exemple}\sens{2}\paradigme{\pcmn{:} \p{}}
\begin{définition}\peng{Placenta.}\end{définition}
\begin{définition}\pcmn{胎盘、衣胞}\end{définition}
\begin{définition}\pfra{Placenta.}\end{définition}
\end{entrée}

\begin{entrée}
{bɑ˧lɑ˧kʰɯ˧tsʰɤ˧}{}{ⓔbɑ˧lɑ˧kʰɯ˧tsʰɤ˧}\formedesurface{bɑ˧lɑ˧kʰɯ˧tsʰɤ˧}\newline
\classe{名词}\ton{M}
\paradigme{\pcmn{:} \p{}}
\begin{définition}\peng{Spider.}\end{définition}
\begin{définition}\pcmn{蜘蛛}\end{définition}
\begin{définition}\pfra{Araignée.}\end{définition}
\end{entrée}

\begin{entrée}
{bæ˧}{}{ⓔbæ˧}\formedesurface{bæ˧}\newline
\classe{形容词}\ton{M}\begin{définition}\peng{Stupid, idiot.}\end{définition}
\begin{définition}\pcmn{傻、笨、蠢}\end{définition}
\begin{définition}\pfra{Stupide, sot, idiot.}\end{définition}
\begin{exemple}\pnru{bæ˧-hĩ˧}\hspace{5pt}\peng{|fg{rel}}\hspace{5pt}\pcmn{傻的}\hspace{5pt}\pfra{|fg{rel}}\end{exemple}
\begin{exemple}\pnru{no˧ | bæ˧-ze˩!}\hspace{5pt}\peng{Well, you have had a hard time, haven't you! (Consolation to someone who complains about having had a hard time.)}\hspace{5pt}\pcmn{你这傻劲,你辛苦了!(人家诉苦的时候,安慰的话)}\hspace{5pt}\pfra{Eh bien, tu as eu bien du malheur! (Consolation dite à quelqu'un qui narre ses mésaventures.)}\end{exemple}
\begin{exemple}\pnru{bæ˧-ze˩ mæ˩!}\hspace{5pt}\peng{You have had a hard time! (Consolation to someone.)}\hspace{5pt}\pcmn{你这傻劲,你辛苦了!}\hspace{5pt}\pfra{Eh bien, voilà bien du malheur!}\end{exemple}
\end{entrée}

\begin{entrée}
{bæ˧˥}{}{ⓔbæ˧˥}\formedesurface{bæ˧˥}\newline
\classe{动词}\ton{MH}\begin{définition}\peng{To run.}\end{définition}
\begin{définition}\pcmn{跑}\end{définition}
\begin{définition}\pfra{Courir.}\end{définition}
\begin{exemple}\pnru{le˧-bæ˧-ze˥}\hspace{5pt}\peng{|fg{accomp} \_ |fg{pfv}}\hspace{5pt}\pcmn{跑了}\hspace{5pt}\pfra{|fg{accomp} \_ |fg{pfv}}\end{exemple}
\end{entrée}

\begin{entrée}
{bæ˧α}{}{ⓔbæ˧α}\formedesurface{ɖɯ˧ bæ˧}\newline
\classe{量词}\ton{Mα}\begin{définition}\peng{Classifier for sorts of things; used in statements of identity: “it is the same".}\end{définition}
\begin{définition}\pcmn{量词:东西(一样)}\end{définition}
\begin{définition}\pfra{Classificateur des espèces/sortes de choses. Proche de \stylefv{/ʁo}˩b/ ‘sorte, variété’. S'emploie dans la construction «c'est la même chose».}\end{définition}
\begin{exemple}\pnru{ɖɯ˧-bæ˧-lɑ˧ ɲi˥!}\hspace{5pt}\peng{It's the same!}\hspace{5pt}\pcmn{是一样的!}\hspace{5pt}\pfra{c'est pareil!/c'est la même chose!}\end{exemple}
\begin{exemple}\pnru{ʝi˧kʰv̩˥-dʑo˩, | ɲi˧-bæ˧ | ʐwɤ˩-tʰɑ˩˥! | ʝi˧kʰv̩˥-dʑo˩, | ɖɯ˧-bæ˧-lɑ˧ ʐwɤ˧-tʰɑ˥!}\hspace{5pt}\peng{Some (phrases/combinations between words) can be said two different ways; whereas others can only be said in one way / only have one possible realization! (Context: the investigation bears on tonal variants for phrases, such as numeral-plus-classifier phrases; the consultant confirms that some combinations admit two variants, whereas others only have one possible tone pattern.)}\hspace{5pt}\pcmn{有些(词组)有两种说法,有些只有一种说法!(情景:讨论的是一些有两种不同变调发音的词组,发音合作人确定:确实有些有两种不同的变调,而有些只有一种声调模型。)}\hspace{5pt}\pfra{Il y en a certaines (=des expressions/des combinaisons de mots), on peut les prononcer de deux façons/elles ont deux schémas tonals différents! Il y en a certaines, il n'y a qu'une façon de les dire/il n'y a qu'une sorte (de réalisation tonale possible)! (commentaire au sujet d'expressions qui ont deux variantes tonales)}\end{exemple}
\begin{exemple}\pnru{ɲi˧-bæ˧-ɳɯ˧ | ɖɯ˧-bæ˧ ʝi˧}\hspace{5pt}\peng{to confuse two things, e.g. to confuse two sounds (phonemes), and to write them in the same way, missing their opposition}\hspace{5pt}\pcmn{把两个音(其实是两个不同的音位)写成一样,等于把两者弄混淆了。}\hspace{5pt}\pfra{confondre deux choses (ex.: confondre deux sons, et les noter de la même façon, alors qu'ils s'opposent entre eux)}\end{exemple}
\end{entrée}

\begin{entrée}
{bæ˩}{₁}{ⓔbæ˩ⓗ1}\formedesurface{bæ˧}\newline
\classe{名词}\ton{L}
1
\paradigme{\pcmn{:} \p{}}
\begin{définition}\peng{Rope.}\end{définition}
\begin{définition}\pcmn{绳子}\end{définition}
\begin{définition}\pfra{Corde.}\end{définition}
\begin{exemple}\pnru{bæ˩ ʈʂʰɯ˩-kʰɯ˥}\hspace{5pt}\peng{|fg{n}+|fg{dem}+|fg{clf}}\hspace{5pt}\pcmn{这条绳子}\hspace{5pt}\pfra{|fg{n}+|fg{dem}+|fg{clf}}\end{exemple}
\end{entrée}

\begin{entrée}
{bæ˩}{₂}{ⓔbæ˩ⓗ2}\formedesurface{bæ˩˥}\newline
\classe{动词}\ton{L}
2\begin{définition}\peng{To fester (with pus), to suppurate, to be purulent.}\end{définition}
\begin{définition}\pcmn{化脓}\end{définition}
\begin{définition}\pfra{Suppurer, donner du pus.}\end{définition}
\begin{exemple}\pnru{bæ˩ bæ˧-ze˩}\hspace{5pt}\peng{the wound suppurates, the wound is pussy}\hspace{5pt}\pcmn{伤口在化脓}\hspace{5pt}\pfra{la blessure donne du pus, il y a du pus}\end{exemple}
\begin{exemple}\pnru{bæ˩˥ | le˧-bæ˩-ze˩}\hspace{5pt}\peng{the wound suppurates, the wound is pussy}\hspace{5pt}\pcmn{伤口在化脓}\hspace{5pt}\pfra{la blessure donne du pus, il y a du pus}\end{exemple}
\end{entrée}

\begin{entrée}
{bæ˩˥}{}{ⓔbæ˩˥}\formedesurface{bæ˩˥}\newline
\classe{名词}\ton{LH}
\paradigme{\pcmn{:} \p{}}
\begin{définition}\peng{Pus.}\end{définition}
\begin{définition}\pcmn{脓}\end{définition}
\begin{définition}\pfra{Pus.}\end{définition}
\begin{exemple}\pnru{bæ˩ bæ˧-ze˩}\hspace{5pt}\peng{the wound suppurates, the wound is pussy}\hspace{5pt}\pcmn{伤口在化脓}\hspace{5pt}\pfra{la blessure donne du pus, il y a du pus}\end{exemple}
\begin{exemple}\pnru{bæ˩˥ | le˧-bæ˩-ze˩}\hspace{5pt}\peng{the wound suppurates, the wound is pussy}\hspace{5pt}\pcmn{伤口化脓了。}\hspace{5pt}\pfra{la blessure donne du pus, il y a du pus}\end{exemple}
\end{entrée}

\begin{entrée}
{bæ˩˧}{}{ⓔbæ˩˧}\formedesurface{bæ˩˥}\newline
\classe{名词}\ton{LM}
\paradigme{\pcmn{:} \p{}}
\begin{définition}\peng{Crops.}\end{définition}
\begin{définition}\pcmn{庄稼}\end{définition}
\begin{définition}\pfra{Récolte; plantes que l'on a semées.}\end{définition}
\begin{exemple}\pnru{bæ˩ ɲi˧}\hspace{5pt}\peng{|fg{cop}}\hspace{5pt}\pcmn{是庄稼}\hspace{5pt}\pfra{|fg{cop}}\end{exemple}
\begin{exemple}\pnru{ɖɯ˧-kʰv̩˧ ʈv̩˧-bæ˥ mv̩˩, | ɕi˧-kʰv̩˧ | le˧-mɤ˧-dzɯ˧!}\hspace{5pt}\peng{“This year, even if we had had a thousand harvests, it would not have lasted a hundred years!" This proverb is a consolation for years of bad harvests: “If the harvest had been excellent, it would not have lasted forever anyway! Everything begins anew every year, so let us look forward!"}\hspace{5pt}\pcmn{“一年收千棵,不够吃百年!”(这个谚语,来慰藉收成不好的年份。)}\hspace{5pt}\pfra{«Quand bien même on aurait fait une récolte fabuleuse, ça ne nous durerait pas éternellement: ça se rejoue chaque année!» Littéralement: «si, une année, mille récoltes parvenaient à maturité, on n['en] mangerait pas [pour autant pendant] cent ans =on n'aurait pas à manger pour cent ans!» Le proverbe sert à se consoler d'une mauvaise récolte, qui va obliger à une année frugale: «Si belle soit la récolte, elle n'aurait de toute façon pas duré éternellement; tout est à recommencer l'année suivante, voyons donc de l'avant!»}\end{exemple}
\end{entrée}

\begin{entrée}
{bæ˩α}{₁}{ⓔbæ˩αⓗ1}\formedesurface{bæ˩˥}\newline
\classe{动词}\ton{Lα}
1\begin{définition}\peng{To sweep, to clean up.}\end{définition}
\begin{définition}\pcmn{扫}\end{définition}
\begin{définition}\pfra{Balayer.}\end{définition}
\begin{exemple}\pnru{ɖæ˩ bæ˧}\hspace{5pt}\peng{to sweep the dust, to sweep the floor}\hspace{5pt}\pcmn{扫地}\hspace{5pt}\pfra{balayer les saletés, balayer le sol}\end{exemple}
\begin{exemple}\pnru{le˧-bæ˧∼bæ˥}\hspace{5pt}\peng{|fg{accomp} |fg{red}}\hspace{5pt}\pcmn{扫一扫}\hspace{5pt}\pfra{|fg{accomp} |fg{red}}\end{exemple}
\begin{exemple}\pnru{ɖʐɤ˩ bæ˩˥}\hspace{5pt}\peng{to sweep the stairs}\hspace{5pt}\pcmn{扫楼梯}\hspace{5pt}\pfra{balayer l'escalier}\end{exemple}
\begin{exemple}\pnru{njɤ˧ | ɖʐɤ˩ bæ˩-zo˩-ho˥.}\hspace{5pt}\peng{I have to sweep the stairs!}\hspace{5pt}\pcmn{我要扫楼梯了!}\hspace{5pt}\pfra{Il va falloir que je balaie l'escalier!}\end{exemple}
\begin{exemple}\pnru{gi˩ bæ˩˥}\hspace{5pt}\peng{to sweep the granary}\hspace{5pt}\pcmn{扫仓廪}\hspace{5pt}\pfra{balayer le grenier à céréales}\end{exemple}
\begin{exemple}\pnru{njɤ˧ | gi˩ bæ˩-zo˩-ho˥.}\hspace{5pt}\peng{I have to sweep the granary!}\hspace{5pt}\pcmn{我要扫仓廪了!}\hspace{5pt}\pfra{Il va falloir que je balaie le grenier à céréales!}\end{exemple}
\end{entrée}

\begin{entrée}
{bæ˩α}{₂}{ⓔbæ˩αⓗ2}\formedesurface{bæ˩˥}\newline
\classe{动词}\ton{Lα}
2\begin{définition}\peng{To bloom.}\end{définition}
\begin{définition}\pcmn{开花}\end{définition}
\begin{définition}\pfra{S'ouvrir (fleur), fleurir.}\end{définition}
\begin{exemple}\pnru{bæ˩bæ˩ bæ˥-ze˩}\hspace{5pt}\peng{The flower has bloomed.}\hspace{5pt}\pcmn{花开了。}\hspace{5pt}\pfra{La fleur a fleuri.}\end{exemple}
\end{entrée}

\begin{entrée}
{bæ˩α}{₃}{ⓔbæ˩αⓗ3}\formedesurface{ɖɯ˧ bæ˩}\newline
\classe{量词}\ton{Lα}
3\begin{définition}\peng{Self-classifier for flowers.}\end{définition}
\begin{définition}\pcmn{量词:花(一朵)}\end{définition}
\begin{définition}\pfra{Auto-classificateur des fleurs.}\end{définition}
\begin{exemple}\pnru{tʰv̩˧-bæ˥}\hspace{5pt}\peng{|fg{dem} \_ (tone: H\# / H\$)}\hspace{5pt}\pcmn{指示代词 \_ :这朵(花)}\hspace{5pt}\pfra{|fg{dem} \_ (ton: H\# / H\$)}\end{exemple}
\end{entrée}

\begin{entrée}
{bæ˩bæ˩}{₁}{ⓔbæ˩bæ˩ⓗ1}\formedesurface{bæ˩bæ˩˥}\newline
\classe{名词}\ton{L}
1
\paradigme{\pcmn{:} \p{}}
\begin{définition}\peng{Flower.}\end{définition}
\begin{définition}\pcmn{花}\end{définition}
\begin{définition}\pfra{Fleur.}\end{définition}
\end{entrée}

\begin{entrée}
{bæ˩bæ˩}{₂}{ⓔbæ˩bæ˩ⓗ2}\formedesurface{bæ˩bæ˩˥}\newline
\classe{形容词}\ton{L}
2\begin{définition}\peng{Spotted.}\end{définition}
\begin{définition}\pcmn{花的(花蛋、花石头、花鸟)}\end{définition}
\begin{définition}\pfra{Bariolé, tacheté, moucheté (ex.: un oeuf moucheté, un oiseau au pelage moucheté, une pierre ayant plusieurs couleurs).}\end{définition}
\begin{exemple}\pnru{bæ˩bæ˩ tʰi˩-di˥}\hspace{5pt}\peng{same meaning: spotted (e.g. an egg, a bird, a stone)}\hspace{5pt}\pcmn{花的,有花纹}\hspace{5pt}\pfra{même sens: bariolé, tacheté (par ex.: oeuf, oiseau, pierre)}\end{exemple}
\end{entrée}

\begin{entrée}
{bæ˧bv̩˥}{}{ⓔbæ˧bv̩˥}\formedesurface{bæ˧bv̩˥}\newline
\classe{名词}\ton{H\#}
\paradigme{\pcmn{:} \p{}}
\begin{définition}\peng{Piglet.}\end{définition}
\begin{définition}\pcmn{猪崽}\end{définition}
\begin{définition}\pfra{Goret, porcelet, cochonnet, petit cochon.}\end{définition}
\begin{exemple}\pnru{bæ˧bv̩˥-zo˩}\hspace{5pt}\peng{same meaning: piglet}\hspace{5pt}\pcmn{猪崽}\hspace{5pt}\pfra{même sens: goret}\end{exemple}
\end{entrée}

\begin{entrée}
{bæ˩dʑɯ˥}{}{ⓔbæ˩dʑɯ˥}\formedesurface{bæ˩dʑɯ˥}\newline
\classe{名词}\ton{LH}
\paradigme{\pcmn{:} \p{}}
\begin{définition}\peng{Crops, harvest.}\end{définition}
\begin{définition}\pcmn{庄稼}\end{définition}
\begin{définition}\pfra{Récolte; plantes que l'on a semées.}\end{définition}
\begin{exemple}\pnru{bæ˩dʑɯ˥ | mɤ˧-dʑɤ˩!}\hspace{5pt}\peng{The harvest is not good!}\hspace{5pt}\pcmn{庄稼不好!/收成不好!}\hspace{5pt}\pfra{La récolte n'est pas bonne!}\end{exemple}
\begin{exemple}\pnru{bæ˩dʑɯ˧ | tv̩˧-bæ˩ le˩-mv̩˩-kʰɯ˩!}\hspace{5pt}\peng{May a thousand crops come to maturity! (A blessing to elders, used for instance during the rite of coming of age)}\hspace{5pt}\pcmn{祝:一千棵庄稼都成熟起来!(成年礼、过年等节庆时的祝福用语,晚辈对长辈的祝福)}\hspace{5pt}\pfra{Puissent mille récoltes venir à maturité! (Bénédiction qu'on dit aux aînés lors de cérémonies: par exemple lors du rite de passage à l'âge adulte)}\end{exemple}
\end{entrée}

\begin{entrée}
{bæ˧ɖæ˧}{}{ⓔbæ˧ɖæ˧}\formedesurface{bæ˧ɖæ˧}\newline
\classe{名词}\ton{M}
\paradigme{\pcmn{:} \p{}}
\begin{définition}\peng{Short rope.}\end{définition}
\begin{définition}\pcmn{短绳}\end{définition}
\begin{définition}\pfra{Corde courte}\end{définition}
\end{entrée}

\begin{entrée}
{bæ˩-lɑ˩∼lɑ˥}{}{ⓔbæ˩-lɑ˩∼lɑ˥}\formedesurface{bæ˩lɑ˩lɑ˥}\newline
\classe{形容词}\ton{L}\begin{définition}\peng{Soft, weak, pliant.}\end{définition}
\begin{définition}\pcmn{软,柔软、软塌塌、软绵绵}\end{définition}
\begin{définition}\pfra{Flasque, sans consistance.}\end{définition}
\end{entrée}

\begin{entrée}
{bæ˩-ljɤ˧∼ljɤ˧}{}{ⓔbæ˩-ljɤ˧∼ljɤ˧}\formedesurface{bæ˩ljɤ˧ljɤ˧}\newline
\classe{名词}\ton{L-}
\paradigme{\pcmn{:} \p{}}
\begin{définition}\peng{China fir cone.}\end{définition}
\begin{définition}\pcmn{松树果、杉树果}\end{définition}
\begin{définition}\pfra{Pomme de pin, fruit du sapin.}\end{définition}
\end{entrée}

\begin{entrée}
{bæ˧mi˧}{₁}{ⓔbæ˧mi˧ⓗ1}\formedesurface{bæ˧mi˧}\newline
\classe{名词}
1
\sens{1}\paradigme{\pcmn{:} \p{}}
\begin{définition}\peng{Duck (without a specification of gender).}\end{définition}
\begin{définition}\pcmn{鸭子}\end{définition}
\begin{définition}\pfra{Canard (sans préciser le sexe: canard ou cane).}\end{définition}\sens{2}
\begin{définition}\peng{Female duck.}\end{définition}
\begin{définition}\pcmn{母鸭子}\end{définition}
\begin{définition}\pfra{Cane.}\end{définition}
\begin{exemple}\pnru{bæ˧mi˧-bæ˧pʰv̩\#˥}\hspace{5pt}\peng{female duck and male duck}\hspace{5pt}\pcmn{母鸭子与公鸭子}\hspace{5pt}\pfra{cane et canard}\end{exemple}
\begin{exemple}\pnru{bæ˧mi˧-bæ˧zo\#˥}\hspace{5pt}\peng{female duck and duckling}\hspace{5pt}\pcmn{母鸭与小鸭子}\hspace{5pt}\pfra{cane et caneton}\end{exemple}
\end{entrée}

\begin{entrée}
{bæ˧mi˧}{₂}{ⓔbæ˧mi˧ⓗ2}\formedesurface{bæ˧mi˧}\newline
\classe{名词}\ton{M}
2
\sens{1}\paradigme{\pcmn{:} \p{}}
\begin{définition}\peng{Thick rope.}\end{définition}
\begin{définition}\pcmn{粗绳索}\end{définition}
\begin{définition}\pfra{Grosse corde.}\end{définition}\sens{2}\paradigme{\pcmn{:} \p{}}
\begin{définition}\peng{Long rope.}\end{définition}
\begin{définition}\pcmn{长绳索}\end{définition}
\begin{définition}\pfra{Longue corde.}\end{définition}
\end{entrée}

\begin{entrée}
{bæ˧mi˧-pʰv̩\#˥}{}{ⓔbæ˧mi˧-pʰv̩\#˥}\formedesurface{bæ˧mi˧pʰv̩˧}\newline
\classe{名词}\ton{\#H}
\paradigme{\pcmn{:} \p{}}
\begin{définition}\peng{Male duck.}\end{définition}
\begin{définition}\pcmn{公鸭子}\end{définition}
\begin{définition}\pfra{Canard (mâle).}\end{définition}
\begin{exemple}\pnru{bæ˧mi˧-pʰv̩˧ tʰv̩˧-mi˧˥}\hspace{5pt}\peng{|fg{n}+|fg{dem}+|fg{clf}}\hspace{5pt}\pcmn{这只公鸭子}\hspace{5pt}\pfra{|fg{n}+|fg{dem}+|fg{clf}}\end{exemple}
\end{entrée}

\begin{entrée}
{bæ˧pʰv̩\#˥}{}{ⓔbæ˧pʰv̩\#˥}\formedesurface{bæ˧pʰv̩˧}\newline
\classe{名词}\ton{\#H}
\paradigme{\pcmn{:} \p{}}
\begin{définition}\peng{Male duck.}\end{définition}
\begin{définition}\pcmn{公鸭子}\end{définition}
\begin{définition}\pfra{Canard (mâle).}\end{définition}
\begin{exemple}\pnru{bæ˧pʰv̩˧ tʰv̩˧-mi˧˥ / bæ˧pʰv̩˧ tʰv̩˧-mi˥\#}\hspace{5pt}\peng{|fg{n}+|fg{dem}+|fg{clf}}\hspace{5pt}\pcmn{这只公鸭子}\hspace{5pt}\pfra{|fg{n}+|fg{dem}+|fg{clf}}\end{exemple}
\begin{exemple}\pnru{bæ˧pʰv̩˧-bæ˧mi\#˥}\hspace{5pt}\peng{male duck and female duck}\hspace{5pt}\pcmn{公鸭子与母鸭子}\hspace{5pt}\pfra{canard et cane}\end{exemple}
\end{entrée}

\begin{entrée}
{bæ˩pʰv̩˥}{}{ⓔbæ˩pʰv̩˥}\formedesurface{bæ˩pʰv̩˥}\newline
\classe{名词}\ton{L+H\#}
\paradigme{\pcmn{:} \p{}}
\begin{définition}\peng{Crowndaisy chrysanthemum, |\stylefi{Glebionis coronaria}.}\end{définition}
\begin{définition}\pcmn{茼蒿}\end{définition}
\begin{définition}\pfra{Chrysanthème couronné, chrysanthème des jardins, chrysanthème comestible ou chrysanthème à couronnes, |\stylefi{Glebionis coronaria}.}\end{définition}
\begin{exemple}\pnru{bæ˩pʰv̩˥-bv̩˩ | bæ˩bæ˩˥}\hspace{5pt}\peng{the flower of crowndaisy chrysanthemum}\hspace{5pt}\pcmn{茼蒿的顶花}\hspace{5pt}\pfra{fleur de chrysanthème couronné}\end{exemple}
\begin{exemple}\pnru{bæ˩pʰv̩˥-bæ˩bæ˩}\hspace{5pt}\peng{crowndaisy chrysanthemum flower}\hspace{5pt}\pcmn{茼蒿顶花}\hspace{5pt}\pfra{fleur de chrysanthème couronné}\end{exemple}
\end{entrée}

\begin{entrée}
{bæ˩-ʁwæ˩∼ʁwæ˥}{}{ⓔbæ˩-ʁwæ˩∼ʁwæ˥}\formedesurface{bæ˩ʁwæ˩ʁwæ˥}\newline
\classe{形容词}\ton{L}\begin{définition}\peng{Loose, slack, lax.}\end{définition}
\begin{définition}\pcmn{松松的}\end{définition}
\begin{définition}\pfra{Relâché.}\end{définition}
\begin{exemple}\pnru{ʈʂʰɯ˧ | ɖwæ˧˥ | bæ˩ʁwæ˩∼ʁwæ˥-ʝi˩!}\hspace{5pt}\peng{It's loose! / It's not well-fastened! (About a load on a mule's back)}\hspace{5pt}\pcmn{松松的!松动了!(情景:看见驮在马上面的货物没系好)}\hspace{5pt}\pfra{C'est tout relâché, ce n'est pas bien serré! (Au sujet d'une charge sur le dos d'un mulet)}\end{exemple}
\end{entrée}

\begin{entrée}
{bæ˧ʁwɤ˧}{}{ⓔbæ˧ʁwɤ˧}\formedesurface{bæ˧ʁwɤ˧}\newline
\classe{名词}\ton{M}\begin{définition}\peng{A village close to the Hot Springs.}\end{définition}
\begin{définition}\pcmn{巴瓦:温泉乡的一个村落}\end{définition}
\begin{définition}\pfra{Un village proche des Source Chaudes.}\end{définition}
\begin{exemple}\pnru{bæ˧ʁwɤ˧-ʁwɤ˧}\hspace{5pt}\peng{same meaning: the village of /bæ˧ʁwɤ˧/}\hspace{5pt}\pcmn{同上:巴瓦村}\hspace{5pt}\pfra{même sens: le village de /bæ˧ʁwɤ˧/}\end{exemple}
\begin{exemple}\pnru{ə˧go˧-ʁwɤ˧, | ʁwɤ˧lɑ˩-bi˩, | bæ˧ʁwɤ˧, | tʰo˧tsʰe\#˥, | pi˧tsʰe˩-di˩, | pɤ˧dʑɤ˩-di˩, | ʁwɤ˧tv̩˧}\hspace{5pt}\peng{Seven villages that one encounters as one leaves the plain of Yongning (towards the Lake); the first two are perceived as villages with a high proportion of Na members, and the third as a mostly Na village, whereas the next two are Pumi (Prinmi); the last used to be predominantly Pumi, but as of the 2010s, it had an important Chinese (Han) population.}\hspace{5pt}\pcmn{永宁背向泸沽湖方向经过的七个村落:阿公瓦、瓦拉比、巴瓦、拖其、比其地、巴甲地、瓦都。前两个村落拥有相当大的摩梭人口比例,第三主要是摩梭村。拖其、比其地、巴甲地是普米村。瓦都,过去主要是普米族村,到了2010年代有了相当多的汉族人口。}\hspace{5pt}\pfra{Sept villages au sortir de la plaine de Yongning, dans la direction du Lac; les deux premiers comportent une population na; le troisième est un village na; les deux suivants sont essentiellement des villages pumi/prinmi; le dernier était un village pumi, et a désormais (dans les années 2010) une importante population chinoise (han).}\end{exemple}
\begin{exemple}\pnru{bæ˧ʁwɤ˧: | nɑ˩˥!}\hspace{5pt}\peng{/bæ˧ʁwɤ˧/ is a Na village!}\hspace{5pt}\pcmn{|fv{/bæ˧ʁwɤ˧/}是一个摩梭人村落!}\hspace{5pt}\pfra{/bæ˧ʁwɤ˧/, c'est un village na!}\end{exemple}
\begin{exemple}\pnru{bæ˧ʁwɤ˧-nɑ˩}\hspace{5pt}\peng{Mosuo people from /bæ˧ʁwɤ˧/.}\hspace{5pt}\pcmn{巴瓦摩梭}\hspace{5pt}\pfra{Les Mosuo de /bæ˧ʁwɤ˧/.}\end{exemple}
\end{entrée}

\begin{entrée}
{bæ˩ʈʂo˥}{}{ⓔbæ˩ʈʂo˥}\formedesurface{bæ˩ʈʂo˥}\newline
\classe{名词}\ton{LH}
\paradigme{\pcmn{:} \p{}}
\begin{définition}\peng{Broom.}\end{définition}
\begin{définition}\pcmn{扫帚}\end{définition}
\begin{définition}\pfra{Balai.}\end{définition}
\end{entrée}

\begin{entrée}
{bæ˩ʈʂwæ˩}{}{ⓔbæ˩ʈʂwæ˩}\formedesurface{bæ˩ʈʂwæ˩˥}\newline
\classe{名词}\ton{L}
\paradigme{\pcmn{:} \p{}}
\begin{définition}\peng{Reins.}\end{définition}
\begin{définition}\pcmn{缰绳}\end{définition}
\begin{définition}\pfra{Rênes.}\end{définition}
\begin{exemple}\pnru{ʐwæ˧-bæ˥ʈʂwæ˩}\hspace{5pt}\peng{horse's reins}\hspace{5pt}\pcmn{马缰绳}\hspace{5pt}\pfra{rênes du cheval}\end{exemple}
\end{entrée}

\begin{entrée}
{bæ˧zo\#˥}{}{ⓔbæ˧zo\#˥}\formedesurface{bæ˧zo˧}\newline
\classe{名词}\ton{\#H}
\paradigme{\pcmn{:} \p{}}
\begin{définition}\peng{Duckling.}\end{définition}
\begin{définition}\pcmn{小鸭子}\end{définition}
\begin{définition}\pfra{Caneton, petit canard.}\end{définition}
\begin{exemple}\pnru{bæ˧zo˧ tʰv̩˧-ɭɯ\#˥}\hspace{5pt}\peng{|fg{n}+|fg{dem}+|fg{clf}}\hspace{5pt}\pcmn{这只小鸭子}\hspace{5pt}\pfra{|fg{n}+|fg{dem}+|fg{clf}}\end{exemple}
\begin{exemple}\pnru{bæ˧zo˧-bæ˧mi\#˥}\hspace{5pt}\peng{duckling and female duck}\hspace{5pt}\pcmn{小鸭子与母鸭}\hspace{5pt}\pfra{caneton et cane}\end{exemple}
\end{entrée}

\begin{entrée}
{bɤ˥}{}{ⓔbɤ˥}\formedesurface{bɤ˧}\newline
\classe{名词}\ton{\#H}
\paradigme{\pcmn{:} \p{}}
\begin{définition}\peng{Pumi (Prinmi) (ethnic group).}\end{définition}
\begin{définition}\pcmn{普米族}\end{définition}
\begin{définition}\pfra{Pumi (Prinmi) (groupe ethnique).}\end{définition}
\begin{exemple}\pnru{bɤ˧-ʐwɤ˧ so˥}\hspace{5pt}\peng{to learn the Pumi language}\hspace{5pt}\pcmn{学普米语}\hspace{5pt}\pfra{apprendre la langue prinmi}\end{exemple}
\end{entrée}

\begin{entrée}
{bɤ˧˥α}{}{ⓔbɤ˧˥α}\formedesurface{ɖɯ˧ bɤ˧˥}\newline
\classe{量词}\ton{MHα}\begin{définition}\peng{Classifier for scarves.}\end{définition}
\begin{définition}\pcmn{量词:头帕(一条)}\end{définition}
\begin{définition}\pfra{Classificateur des fichus et foulards.}\end{définition}
\end{entrée}

\begin{entrée}
{bɤ˩α}{₁}{ⓔbɤ˩αⓗ1}\formedesurface{ɖɯ˧ bɤ˩}\newline
\classe{量词}\ton{Lα}
1\begin{définition}\peng{Classifier for corncobs.}\end{définition}
\begin{définition}\pcmn{量词:玉米棒子(一根)}\end{définition}
\begin{définition}\pfra{Classificateur des épis de maïs.}\end{définition}
\begin{exemple}\pnru{hɑ˧bɤ˥ | ɖɯ˧-bɤ˩}\hspace{5pt}\peng{a corncob}\hspace{5pt}\pcmn{一根玉米棒子}\hspace{5pt}\pfra{un épi de maïs}\end{exemple}
\begin{exemple}\pnru{tʰv̩˧-bɤ˥}\hspace{5pt}\peng{|fg{dem} \_ (tone: H\# / H\$)}\hspace{5pt}\pcmn{指示代词 \_:那根(玉米棒子)}\hspace{5pt}\pfra{|fg{dem} \_ (ton: H\# / H\$)}\end{exemple}
\end{entrée}

\begin{entrée}
{bɤ˩α}{₂}{ⓔbɤ˩αⓗ2}\formedesurface{ɖɯ˧ bɤ˩}\newline
\classe{量词}\ton{Lα}
2\begin{définition}\peng{Classifier for halves.}\end{définition}
\begin{définition}\pcmn{量词:半}\end{définition}
\begin{définition}\pfra{Classificateur des moitiés.}\end{définition}
\begin{exemple}\pnru{ɖɯ˧-bɤ˩-lɑ˩ tʰv̩˩-sɯ˩! | ɖɯ˧-hu˧-ɻ̍˥!}\hspace{5pt}\peng{I have only done one half as yet! Wait a bit! (Context: someone is sorting out clothes, and is midway through the task.)}\hspace{5pt}\pcmn{我才干了一半!稍等!(情景:一个在收拾衣服,告诉对方:才收拾了一半,还要等等。)}\hspace{5pt}\pfra{Je n'ai fait que la moitié! Attends un peu! (Contexte: quelqu'un est en train de trier les vêtements, et signale qu'il n'a pas fini)}\end{exemple}
\begin{exemple}\pnru{ɖɯ˧-bɤ˩-lɑ˩ tʰv̩˩-ze˩!}\hspace{5pt}\peng{You are only half-way through! (An observation about the investigator's progress in studying the Na language. It emphasizes the road ahead; still, this is a more encouraging formulation than the previous example: this one uses the perfective, acknowledging that part of the path that has already been covered, whereas |fv{ɖɯ˧-bɤ˩-lɑ˩ tʰv̩˩-sɯ˩} would essentially emphasize all that remains ahead.)}\hspace{5pt}\pcmn{(你)才到了一半!(合作人对调查者学摩梭话的评定)}\hspace{5pt}\pfra{Tu es à mi-chemin! / Tu n'as parcouru que la moitié du chemin! (Réflexion de la consultante principale au sujet de mon apprentissage de la langue na. Cette formulation souligne le chemin qui reste à parcourir, mais l'emploi du perfectif apporte une note d'encouragement, là où l'exemple précédant insisterait essentiellement sur le chemin restant à parcourir: |fv{ɖɯ˧-bɤ˩-lɑ˩ tʰv̩˩-sɯ˩}.)}\end{exemple}
\begin{exemple}\pnru{ʐæ˩ʂæ˥ | ʐwæ˩˥! | le˧-se˥, | ɖɯ˧-bɤ˩-qo˩-lɑ˩ tʰv̩˩-sɯ˩!}\hspace{5pt}\peng{It's really far! We have walked (for quite some time), and only got mid-way!}\hspace{5pt}\pcmn{真远啊!走啊走,才走了一半的路!}\hspace{5pt}\pfra{C'est bien loin! On a marché, et on n'est encore parvenu qu'à mi-chemin!}\end{exemple}
\begin{exemple}\pnru{ʐɤ˩mi˩˥ | ɖɯ˧-bɤ˩}\hspace{5pt}\peng{half the way}\hspace{5pt}\pcmn{路的一半}\hspace{5pt}\pfra{la moitié du chemin}\end{exemple}
\begin{exemple}\pnru{ə˧mi˧! | wɤ˩˥ | ɖɯ˧-bɤ˩ dʑo˩-sɯ˩-wɤ˩!}\hspace{5pt}\peng{Wow! (How far!) There is still half the way to go!}\hspace{5pt}\pcmn{啊呀嚒!还剩一半的路啊!}\hspace{5pt}\pfra{Houlàà! (Que c'est loin!) Il reste encore la moitié du chemin (à parcourir)!}\end{exemple}
\end{entrée}

\begin{entrée}
{bɤ˧dzi˩}{}{ⓔbɤ˧dzi˩}\formedesurface{bɤ˧dzi˩}\newline
\classe{名词}\ton{L\#}\begin{définition}\peng{A village in Yongning.}\end{définition}
\begin{définition}\pcmn{八珠(永宁坝子的一个村落)}\end{définition}
\begin{définition}\pfra{Un des villages de la plaine de Yongning.}\end{définition}
\begin{exemple}\pnru{ɖæ˩ʂɯ\#˥, | ʈʂo˧ʂɯ\#˥, | bɤ˩tɕʰɯ˩˥, | dɑ˧pʰo˥, | bɤ˧dzi˩, | dze˧bo˧}\hspace{5pt}\peng{Six villages of the plain of Yongning that lie relatively close to the Lake.}\hspace{5pt}\pcmn{永宁摩梭地理概念中,距离泸沽湖比较近的六个村落:扎实、忠实、八旗、达坡、八珠、者波。}\hspace{5pt}\pfra{Six villages de la plaine de Yongning qui sont relativement proches du Lac.}\end{exemple}
\end{entrée}

\begin{entrée}
{bɤ˧kɯ˧}{}{ⓔbɤ˧kɯ˧}\formedesurface{bɤ˧kɯ˧}\newline
\classe{名词}\ton{M}
\paradigme{\pcmn{:} \p{}}
\begin{définition}\peng{Sifter, sieve. RD Comment:Cf. bv̩˧ʈʂɯ˥}\end{définition}
\begin{définition}\pcmn{筛子}\end{définition}
\begin{définition}\pfra{Vannerie: tamis, crible, en forme de gourde; on y met des légumes, des choses à porter. Cette vannerie est commode à porter.}\end{définition}
\end{entrée}

\begin{entrée}
{bɤ˧mi\#˥}{}{ⓔbɤ˧mi\#˥}\formedesurface{bɤ˧mi˧}\newline
\classe{名词}\ton{\#H}
\paradigme{\pcmn{:} \p{}}
\begin{définition}\peng{Pumi woman.}\end{définition}
\begin{définition}\pcmn{普米族女人}\end{définition}
\begin{définition}\pfra{Femme pumi.}\end{définition}
\end{entrée}

\begin{entrée}
{bɤ˧mi˥-ʂe˩}{}{ⓔbɤ˧mi˥-ʂe˩}\formedesurface{bɤ˧mi˥ʂe˩}\newline
\classe{名词}\ton{H\#-}\begin{définition}\peng{Copper-nickel alloy.}\end{définition}
\begin{définition}\pcmn{白铜}\end{définition}
\begin{définition}\pfra{Cupronickel: alliage cuivre-nickel.}\end{définition}
\end{entrée}

\begin{entrée}
{bɤ˧ʂɯ˩}{}{ⓔbɤ˧ʂɯ˩}\formedesurface{bɤ˧ʂɯ˩}\newline
\classe{名词}\ton{L\#}\begin{définition}\peng{Baisha: name of a village of the Lijiang plain. The village had a tradition of trading and peddling to faraway places, hence its familiarity to people in Yongning. Its Naxi name is \stylefv{/bɤ}˧ʂɯ˩/.}\end{définition}
\begin{définition}\pcmn{白沙(丽江坝子里的一个村落)}\end{définition}
\begin{définition}\pfra{Nom d'un village de la plaine de Lijiang, d'où venaient de nombreux marchands, d'où le fait que son nom soit connu à Yongning. En naxi: \stylefv{/bɤ}˧ʂɯ˩/.}\end{définition}
\end{entrée}

\begin{entrée}
{bɤ˧tʰv̩˩}{}{ⓔbɤ˧tʰv̩˩}\formedesurface{bɤ˧tʰv̩˩}\newline
\classe{名词}\ton{L\#}
\paradigme{\pcmn{:} \p{}}
\begin{définition}\peng{Footprints.}\end{définition}
\begin{définition}\pcmn{脚印}\end{définition}
\begin{définition}\pfra{Empreintes, traces de pas, traces de pied.}\end{définition}
\begin{exemple}\pnru{hĩ˧-bɤ˧tʰv̩˥}\hspace{5pt}\peng{human footprints}\hspace{5pt}\pcmn{人的脚印}\hspace{5pt}\pfra{empreintes (de pied) d'homme}\end{exemple}
\begin{exemple}\pnru{kʰv̩˩mi˩-bɤ˩tʰv̩˥}\hspace{5pt}\peng{dog's footprints}\hspace{5pt}\pcmn{狗爪印}\hspace{5pt}\pfra{empreintes de (pattes de) chien}\end{exemple}
\end{entrée}

\begin{entrée}
{bɤ˩tɕʰɯ˩}{}{ⓔbɤ˩tɕʰɯ˩}\formedesurface{bɤ˩tɕʰɯ˩˥}\newline
\classe{名词}\ton{L}\begin{définition}\peng{A village in the plain of Lijiang.}\end{définition}
\begin{définition}\pcmn{八旗(永宁坝子的一个村落,也称作‘八七’)}\end{définition}
\begin{définition}\pfra{Un des villages de la plaine de Yongning.}\end{définition}
\begin{exemple}\pnru{ɖæ˩ʂɯ\#˥, | ʈʂo˧ʂɯ\#˥, | bɤ˩tɕʰɯ˩˥, | dɑ˧pʰo˥, | bɤ˧dzi˩, | dze˧bo˧}\hspace{5pt}\peng{Six villages of the plain of Yongning that lie relatively close to the Lake.}\hspace{5pt}\pcmn{永宁摩梭地理概念中,距离泸沽湖比较近的六个村落:扎实、忠实、八旗、达坡、八珠、者波。}\hspace{5pt}\pfra{Six villages de la plaine de Yongning qui sont relativement proches du Lac.}\end{exemple}
\end{entrée}

\begin{entrée}
{bɤ˧tsʰo˧gv̩˥}{}{ⓔbɤ˧tsʰo˧gv̩˥}\formedesurface{bɤ˧tsʰo˧gv̩˥}\newline
\classe{名词}\ton{H\#}\begin{définition}\peng{A village of the Lijiang plain: the central village of the plain, where the marketplace was still located in the early 21st century.}\end{définition}
\begin{définition}\pcmn{巴搓古(永宁坝子的一个村落)}\end{définition}
\begin{définition}\pfra{Un des villages de la plaine de Yongning; lieu de l'actuel marché; terme également employé pour désigner le lieu d'habitation des artisans Naxi).}\end{définition}
\begin{exemple}\pnru{bɤ˧tsʰo˧gv̩˥-hĩ˩}\hspace{5pt}\peng{someone from Bacuogu}\hspace{5pt}\pcmn{巴搓古的人}\hspace{5pt}\pfra{quelqu'un de Bacuogu}\end{exemple}
\begin{exemple}\pnru{dʑɤ˩bv̩˧kɤ˧-sɑ˥ʁwɤ˩, | hi˩ʁwɤ˩-lo˥, | æ˩mi˧-ʁwɤ\#˥, | lɑ˧lo˧-ʁwɤ˥, | lɑ˧ŋwɤ˧, | bɤ˧tsʰo˧gv̩˥, | ə˧lɑ˧-ʁwɤ\#˥, | gæ˧ɻæ˩, | qʰæ˧tɕʰi˧, | tʰo˧ʈɯ\#˥}\hspace{5pt}\peng{The ten Na villages considered in traditional geography as belonging to the vicinity of the Yongning temple.}\hspace{5pt}\pcmn{永宁摩梭地理概念中,距离扎美寺最近的十个村落:佳部嘎萨瓦、习瓦洛、阿咪瓦、拉洛瓦、拉瓦、巴搓古、阿拉瓦、嘎尔、开基、拖支。}\hspace{5pt}\pfra{Les dix villages na traditionnellement considérés comme appartenant au voisinage du temple de Yongning.}\end{exemple}
\end{entrée}

\begin{entrée}
{bɤ˧zo\#˥}{}{ⓔbɤ˧zo\#˥}\formedesurface{bɤ˧zo˧}\newline
\classe{名词}\ton{\#H}
\paradigme{\pcmn{:} \p{}}
\begin{définition}\peng{Pumi man.}\end{définition}
\begin{définition}\pcmn{普米族男人}\end{définition}
\begin{définition}\pfra{Homme pumi.}\end{définition}
\end{entrée}

\begin{entrée}
{‑bi}{}{ⓔ‑bi}\formedesurface{--}\newline
\classe{连接词}\ton{0?}\begin{définition}\peng{|fg{adversative}: no matter…}\end{définition}
\begin{définition}\pcmn{虽然……}\end{définition}
\begin{définition}\pfra{|fg{adversatif}: bien que, même si.}\end{définition}
\begin{exemple}\pnru{ʈʂʰɯ˧ | nɑ˩ ɲi˥-pi˩-bi˩-bi˩, | nɑ˩-ʐwɤ˧ | mɤ˧-kv̩˧˥!}\hspace{5pt}\peng{Although (s)he is Na, (s)he cannot speak the Na language!}\hspace{5pt}\pcmn{他虽然是摩梭人但不会讲摩梭话。}\hspace{5pt}\pfra{Bien qu'elle/il soit Na, elle/il ne sait pas parler la langue na!}\end{exemple}
\begin{exemple}\pnru{*ʈʂʰɯ˧ | nɑ˩ ɲi˥-bi˩, … / *ʈʂʰɯ˧ | nɑ˩ ɲi˥-bi˩-bi˩}\hspace{5pt}\peng{an ungrammatical sentence; the intended meaning was ‘although (s)he is Na…'}\hspace{5pt}\pcmn{病句:不能这样说“他虽然是摩梭人……”}\hspace{5pt}\pfra{phrase non acceptable; l'intention était de dire ‘bien qu'il soit Na…'}\end{exemple}
\end{entrée}

\begin{entrée}
{bi˥}{}{ⓔbi˥}\formedesurface{bi˧}\newline
\classe{形容词}\ton{H}\begin{définition}\peng{Thin; shallow.}\end{définition}
\begin{définition}\pcmn{薄,浅(水浅)}\end{définition}
\begin{définition}\pfra{Mince; peu profond.}\end{définition}
\begin{exemple}\pnru{bi˧ | ʐwæ˩˥!}\hspace{5pt}\peng{It's very shallow!}\hspace{5pt}\pcmn{很浅!}\hspace{5pt}\pfra{C'est très peu profond!}\end{exemple}
\begin{exemple}\pnru{dʑɯ˧ | ɖɯ˧-pi˧ bi˧˥}\hspace{5pt}\peng{The water is rather shallow.}\hspace{5pt}\pcmn{水有点浅。}\hspace{5pt}\pfra{L'eau est assez peu profonde.}\end{exemple}
\begin{exemple}\pnru{dʑɯ˧ bi˧-hĩ˧, | mɤ˧-ɖwæ˩!}\hspace{5pt}\peng{Shallow water is not frightening! / There is nothing frightening about shallow water! / Come on, don't be afraid: it's (just) shallow water!}\hspace{5pt}\pcmn{水很浅,不用怕!}\hspace{5pt}\pfra{L'eau pas profonde, ça fait pas peur! / L'eau n'est pas profonde; tu vas quand même pas avoir peur!}\end{exemple}
\end{entrée}

\begin{entrée}
{bi˥}{}{ⓔbi˥}\formedesurface{bi˧}\newline
\classe{名词}\ton{\#H}
\paradigme{\pcmn{:} \p{}}
\begin{définition}\peng{Snow.}\end{définition}
\begin{définition}\pcmn{雪}\end{définition}
\begin{définition}\pfra{Neige.}\end{définition}
\begin{exemple}\pnru{bi˧ gi˧-ze˩}\hspace{5pt}\peng{it is snowing; it has snowed}\hspace{5pt}\pcmn{下雪了}\hspace{5pt}\pfra{il neige}\end{exemple}
\end{entrée}

\begin{entrée}
{bi˧}{₂}{ⓔbi˧ⓗ2}\formedesurface{bi˧}\newline
\classe{名词}\ton{M}
2\begin{définition}\peng{Village; neighbours, people in the village.}\end{définition}
\begin{définition}\pcmn{村落,邻居、村里的人们}\end{définition}
\begin{définition}\pfra{Le village; les gens du village, les voisins.}\end{définition}
\end{entrée}

\begin{entrée}
{bi˧}{₃}{ⓔbi˧ⓗ3}\formedesurface{bi˧}\newline
\classe{动词}\ton{M}
3\begin{définition}\peng{To dare.}\end{définition}
\begin{définition}\pcmn{敢}\end{définition}
\begin{définition}\pfra{Oser.}\end{définition}
\begin{exemple}\pnru{ʝi˧-mɤ˧-bi˧}\hspace{5pt}\peng{Not to dare to.}\hspace{5pt}\pcmn{不敢做}\hspace{5pt}\pfra{Ne pas oser (faire quelque chose).}\end{exemple}
\end{entrée}

\begin{entrée}
{‑bi˧α}{}{ⓔ‑bi˧α}\formedesurface{bi˧}\newline
\classe{后缀}\ton{M}\begin{définition}\peng{Immediate future.}\end{définition}
\begin{définition}\pcmn{要(近将来)}\end{définition}
\begin{définition}\pfra{Futur immédiat.}\end{définition}
\end{entrée}

\begin{entrée}
{bi˧γ}{₁}{ⓔbi˧γⓗ1}\formedesurface{bi˧}\newline
\classe{动词}\ton{Mγ}
1\begin{définition}\peng{To go.}\end{définition}
\begin{définition}\pcmn{去}\end{définition}
\begin{définition}\pfra{Aller.}\end{définition}
\begin{exemple}\pnru{bi˧-tʰɑ˧!}\hspace{5pt}\peng{|fg{abilitive}}\hspace{5pt}\pcmn{可以去!}\hspace{5pt}\pfra{|fg{abilitive}: On peut y aller!}\end{exemple}
\begin{exemple}\pnru{bi˧-tʰɑ˧-ze˥!}\hspace{5pt}\peng{|fg{abilitive}+|fg{pfv}: We can go now!}\hspace{5pt}\pcmn{可以去了!}\hspace{5pt}\pfra{|fg{abilitive}+|fg{pfv}: Ca y est, on peut y aller!}\end{exemple}
\begin{exemple}\pnru{le˧-bi˩}\hspace{5pt}\peng{to go back}\hspace{5pt}\pcmn{回去,返回}\hspace{5pt}\pfra{retourner; s'en retourner}\end{exemple}
\end{entrée}

\begin{entrée}
{bi˩}{}{ⓔbi˩}\formedesurface{--}\newline
\classe{}\ton{L}\begin{définition}\peng{On; at.}\end{définition}
\begin{définition}\pcmn{向、至、往}\end{définition}
\begin{définition}\pfra{Sur; vers.}\end{définition}
\begin{exemple}\pnru{ʐæ˩sɯ˩-bi˥ | tʰi˧-ʈʂʰv̩˧˥}\hspace{5pt}\peng{to hold grip of the felt cape}\hspace{5pt}\pcmn{抓住毡子}\hspace{5pt}\pfra{[elle] a agrippé son vêtement}\end{exemple}
\begin{exemple}\pnru{pʰæ˧qʰwɤ˩ bi˩, | mɤ˩ tʰi˩-jɤ˩˥.}\hspace{5pt}\peng{To put oil onto the face (animal fat, to protect the skin); this phrase was also used to describe the investigator putting on suncream.}\hspace{5pt}\pcmn{脸上擦油(为了保护皮肤)。(看见调查者抹防晒霜,也这样描述了。)}\hspace{5pt}\pfra{S’étaler de l’huile sur le visage (traditionnellement: du saindoux, pour protéger la peau). Cette expression est également employée pour décrire le geste du visiteur qui se met de la crème solaire sur le visage.}\end{exemple}
\begin{exemple}\pnru{lo˩qʰwɤ˧ bi˩}\hspace{5pt}\peng{on the hand}\hspace{5pt}\pcmn{手上}\hspace{5pt}\pfra{sur la main}\end{exemple}
\begin{exemple}\pnru{gv̩˧mi˧-bi˩}\hspace{5pt}\peng{on the body}\hspace{5pt}\pcmn{身上}\hspace{5pt}\pfra{sur le corps}\end{exemple}
\begin{exemple}\pnru{kʰɯ˧tsʰɤ˧-bi˥}\hspace{5pt}\peng{on the feet}\hspace{5pt}\pcmn{脚上}\hspace{5pt}\pfra{sur les pieds}\end{exemple}
\end{entrée}

\begin{entrée}
{bi˩γ}{}{ⓔbi˩γ}\formedesurface{ɖɯ˧ bi˩}\newline
\classe{量词}\ton{Lγ}\begin{définition}\peng{Self-classifier for animal hooves; also used for footprints.}\end{définition}
\begin{définition}\pcmn{量词:脚或脚印(一只)}\end{définition}
\begin{définition}\pfra{Auto-classificateur des sabots d'animaux, et des traces qu'ils laissent sur le sol.}\end{définition}
\begin{exemple}\pnru{ʁwɤ˧ | ɖɯ˧-bi˩}\hspace{5pt}\peng{a mountain face EXPLAIN (By extension...)}\hspace{5pt}\pcmn{一片山}\hspace{5pt}\pfra{un pan de montagne}\end{exemple}
\end{entrée}

\begin{entrée}
{bi˩bi˧}{}{ⓔbi˩bi˧}\formedesurface{bi˩bi˥}\newline
\classe{名词}\ton{LM}
\paradigme{\pcmn{:} \p{}}
\begin{définition}\peng{Pod (of bean).}\end{définition}
\begin{définition}\pcmn{豆荚}\end{définition}
\begin{définition}\pfra{Cosse de haricot.}\end{définition}
\begin{exemple}\pnru{bi˩bi˧ ɲi˩}\hspace{5pt}\peng{|fg{cop}}\hspace{5pt}\pcmn{是豆荚}\hspace{5pt}\pfra{|fg{cop}}\end{exemple}
\begin{exemple}\pnru{nv̩˩ɭɯ˧-bi˩bi˩}\hspace{5pt}\peng{soybean pods}\hspace{5pt}\pcmn{黄豆荚}\hspace{5pt}\pfra{cosses de soja}\end{exemple}
\end{entrée}

\begin{entrée}
{bi˧bv̩˥}{}{ⓔbi˧bv̩˥}\formedesurface{bi˧bv̩˥}\newline
\classe{动词}\ton{H\#}\begin{définition}\peng{To burst out, to erupt.}\end{définition}
\begin{définition}\pcmn{暴发(如:洪水暴发)、爆炸、爆破、冲破、流淌,冲下去,下泻,很快地流}\end{définition}
\begin{définition}\pfra{Jaillir, entrer en éruption, couler en trombe; se déclencher (inondation).}\end{définition}
\begin{exemple}\pnru{tʰi˧-ʈwæ˧˥, | sɤ˧ | bi˧bv̩˥-ze˩!}\hspace{5pt}\peng{(She/he) fell down; blood is flowing profusely!}\hspace{5pt}\pcmn{(他)摔倒了,流了很多血/血破了!}\hspace{5pt}\pfra{[Il] est tombé [et s'est blessé]; le sang a jailli! (La blessure a aussitôt saigné à profusion.)}\end{exemple}
\begin{exemple}\pnru{dʑɯ˧ | bi˧bv̩˥-ze˩!}\hspace{5pt}\peng{The water is flowing profusely!}\hspace{5pt}\pcmn{水暴发了!/水流如注!}\hspace{5pt}\pfra{L'eau s'est déversée à flots!}\end{exemple}
\begin{exemple}\pnru{dʑɯ˩nɑ˩mi˩ bi˩bv̩˥-ze˩-pʰæ˩di˩!}\hspace{5pt}\peng{There seems to be a flood / landslide out on the mountain!}\hspace{5pt}\pcmn{山上的水爆发了的样子!(激流流得很冲,好像山肚子爆炸了一样)/山上好像有了泥石流!}\hspace{5pt}\pfra{On dirait qu'il y a une coulée de boue / un glissement de terrain sur la montagne!}\end{exemple}
\end{entrée}

\begin{entrée}
{bi˧ɕi˧kv̩˥}{}{ⓔbi˧ɕi˧kv̩˥}\formedesurface{bi˧ɕi˧kv̩˥}\newline
\classe{名词}\ton{H\#}
\paradigme{\pcmn{:} \p{}}
\begin{définition}\peng{Cheek.}\end{définition}
\begin{définition}\pcmn{腮,腮帮子}\end{définition}
\begin{définition}\pfra{Joue.}\end{définition}
\end{entrée}

\begin{entrée}
{bi˧hæ˧˥}{}{ⓔbi˧hæ˧˥}\formedesurface{bi˧hæ˧˥}\newline
\classe{名词}\ton{MH\#}
\paradigme{\pcmn{:} \p{}}
\begin{définition}\peng{Girth (for horse).}\end{définition}
\begin{définition}\pcmn{马肚带}\end{définition}
\begin{définition}\pfra{Sangle ventrale.}\end{définition}
\begin{exemple}\pnru{ʐwæ˧-bi˥hæ˩}\hspace{5pt}\peng{horse's girth}\hspace{5pt}\pcmn{马肚带}\hspace{5pt}\pfra{sangle de cheval}\end{exemple}
\end{entrée}

\begin{entrée}
{bi˧-lv̩˧∼lv̩˥}{}{ⓔbi˧-lv̩˧∼lv̩˥}\formedesurface{bi˧lv̩˧lv̩˥}\newline
\classe{名词}\ton{H\#}
\paradigme{\pcmn{:} \p{}}
\begin{définition}\peng{Snow flakes.}\end{définition}
\begin{définition}\pcmn{雪花}\end{définition}
\begin{définition}\pfra{Flocons de neige.}\end{définition}
\end{entrée}

\begin{entrée}
{bi˧mi˧}{}{ⓔbi˧mi˧}\formedesurface{bi˧mi˧}\newline
\classe{名词}\ton{M}
\paradigme{\pcmn{:} \p{}}
\begin{définition}\peng{Belly, abdomen.}\end{définition}
\begin{définition}\pcmn{肚子}\end{définition}
\begin{définition}\pfra{Ventre.}\end{définition}
\begin{exemple}\pnru{bi˧mi˧ ɖɯ˩}\hspace{5pt}\peng{Who has a big appetite. It also has the abstract meaning of broad-minded, to describe someone who does not haggle over small matters.}\hspace{5pt}\pcmn{能吃、饭量大。也指肚量大,不计较(随便说说不生气)}\hspace{5pt}\pfra{Qui a un gros appétit; ittéralement «qui a un gros ventre/gros estomac». L'expression a aussi le sens abstrait de: ouvert d'esprit, qui sait voir les choses de haut, qui ne se perd pas dans des finasseries.}\end{exemple}
\begin{exemple}\pnru{bi˧mi˧ tɕi˩}\hspace{5pt}\peng{Narrow-minded, parochial.}\hspace{5pt}\pcmn{肚量小,心胸狭隘}\hspace{5pt}\pfra{Etroit d'esprit.}\end{exemple}
\end{entrée}

\begin{entrée}
{bi˩mi˩}{}{ⓔbi˩mi˩}\formedesurface{bi˩mi˩˥}\newline
\classe{名词}\ton{L}
\paradigme{\pcmn{:} \p{}}
\begin{définition}\peng{Axe.}\end{définition}
\begin{définition}\pcmn{斧头}\end{définition}
\begin{définition}\pfra{Hache.}\end{définition}
\end{entrée}

\begin{entrée}
{bi˩pʰv̩˧˥}{}{ⓔbi˩pʰv̩˧˥}\formedesurface{bi˩pʰv̩˧˥}\newline
\classe{名词}\ton{LM+MH\#}\begin{définition}\peng{Bottle gourd; its fruit is the calabash, which becomes hard when dry.}\end{définition}
\begin{définition}\pcmn{葫芦}\end{définition}
\begin{définition}\pfra{Gourde (cucurbitacée); son fruit est la calebasse, qui devient dure en séchant.}\end{définition}
\end{entrée}

\begin{entrée}
{bi˩pʰv̩˧-dʑɯ˧˥}{}{ⓔbi˩pʰv̩˧-dʑɯ˧˥}\formedesurface{bi˩pʰv̩˧dʑɯ˧˥}\newline
\classe{名词}\ton{LM+MH\#}\begin{définition}\peng{Flood.}\end{définition}
\begin{définition}\pcmn{洪水}\end{définition}
\begin{définition}\pfra{Inondation.}\end{définition}
\end{entrée}

\begin{entrée}
{bi˩ʁo˥}{}{ⓔbi˩ʁo˥}\formedesurface{bi˩ʁo˥}\newline
\classe{名词}\ton{LH}
\paradigme{\pcmn{:} \p{}}
\begin{définition}\peng{Purse (used to be attached to the belt, on the inside).}\end{définition}
\begin{définition}\pcmn{钱包(过去:是系在腰上的)}\end{définition}
\begin{définition}\pfra{Bourse. La bourse était autrefois attachée à la ceinture, sur sa face interne.}\end{définition}
\begin{exemple}\pnru{bi˩ʁo˧ tʰi˧-ʂæ˧˥}\hspace{5pt}\peng{to attach (one's) purse (to the belt)}\hspace{5pt}\pcmn{系上钱包(在腰上)}\hspace{5pt}\pfra{attacher sa bourse (à sa ceinture)}\end{exemple}
\end{entrée}

\begin{entrée}
{bi˧tɑ˧}{}{ⓔbi˧tɑ˧}\formedesurface{bi˧tɑ˧}\newline
\classe{名词}\ton{M}
\paradigme{\pcmn{:} \p{}}
\begin{définition}\peng{Cape made of sheepskin and padded with felt, worn on the upper back. It kept the back warm and protected it against friction with loads carried on the shoulder.}\end{définition}
\begin{définition}\pcmn{垫背、披肩}\end{définition}
\begin{définition}\pfra{Cape en cuir de mouton, rembourrée de feutre, portée sur le haut du dos; elle maintenait le dos au chaud, et protégeait du frottement avec les charges portées sur le dos.}\end{définition}
\end{entrée}

\begin{entrée}
{bi˩tɑ˩}{}{ⓔbi˩tɑ˩}\formedesurface{bi˩tɑ˩˥}\newline
\classe{动词}\ton{L}\begin{définition}\peng{To pull, to drag.}\end{définition}
\begin{définition}\pcmn{退、退后}\end{définition}
\begin{définition}\pfra{Se reculer.}\end{définition}
\begin{exemple}\pnru{bi˩tɑ˩-ze˥}\hspace{5pt}\peng{|fg{pfv}}\hspace{5pt}\pcmn{退后了}\hspace{5pt}\pfra{|fg{pfv}}\end{exemple}
\end{entrée}

\begin{entrée}
{bi˧tɕɤ˩}{}{ⓔbi˧tɕɤ˩}\formedesurface{bi˧tɕɤ˩}\newline
\classe{名词}\ton{L\#}
\paradigme{\pcmn{:} \p{}}
\begin{définition}\peng{Navel.}\end{définition}
\begin{définition}\pcmn{肚脐}\end{définition}
\begin{définition}\pfra{Nombril.}\end{définition}
\end{entrée}

\begin{entrée}
{bi˧tɕo˧}{}{ⓔbi˧tɕo˧}\formedesurface{bi˧tɕo˧}\newline
\classe{名词}\ton{M}\begin{définition}\peng{Neighbourhood, vicinity.}\end{définition}
\begin{définition}\pcmn{外围。如:房屋的外围,是周围的村落。}\end{définition}
\begin{définition}\pfra{Les environs, le voisinage (d'une maison, d'un village). Par exemple, pour la consultante principale, le village, \stylefv{/ʁwɤ}˧qo˧/, c'est \stylefv{/ə}˧lɑ˧-ʁwɤ˧/; les villages environnants constituent le voisinage, \stylefv{/bi}˧tɕo˧/.}\end{définition}
\end{entrée}

\begin{entrée}
{bi˩-tsɯ˧tsɯ˧}{}{ⓔbi˩-tsɯ˧tsɯ˧}\formedesurface{bi˩tsɯ˧tsɯ˧}\newline
\classe{名词}\ton{L-}\begin{définition}\peng{Wild strawberry, |\stylefi{Fragaria vesca}.}\end{définition}
\begin{définition}\pcmn{野草莓}\end{définition}
\begin{définition}\pfra{Fraise sauvage, |\stylefi{Fragaria vesca}.}\end{définition}
\end{entrée}

\begin{entrée}
{bi˩ʈʂʰɤ\#˥}{}{ⓔbi˩ʈʂʰɤ\#˥}\formedesurface{bi˩ʈʂʰɤ˥}\newline
\classe{名词}\ton{LM+\#H}
\paradigme{\pcmn{:} \p{}}
\begin{définition}\peng{Whiskers.}\end{définition}
\begin{définition}\pcmn{胡须,络腮胡须}\end{définition}
\begin{définition}\pfra{Favoris, rouflaquettes.}\end{définition}
\end{entrée}

\begin{entrée}
{bi˩wɤ˧}{}{ⓔbi˩wɤ˧}\formedesurface{bi˩wɤ˥}\newline
\classe{名词}\ton{LM}
\paradigme{\pcmn{:} \p{}}
\begin{définition}\peng{Services (or money) offered as remuneration to a religious practitioner for ritual services rendered.}\end{définition}
\begin{définition}\pcmn{酬劳、报酬}\end{définition}
\begin{définition}\pfra{Services (ou argent) donnés en récompense à un moine pour avoir réalisé des rituels.}\end{définition}
\end{entrée}

\begin{entrée}
{bi˧zɯ˧}{}{ⓔbi˧zɯ˧}\formedesurface{bi˧zɯ˧}\newline
\classe{名词}\ton{M}
\paradigme{\pcmn{:} \p{}}
\begin{définition}\peng{Lower abdomen.}\end{définition}
\begin{définition}\pcmn{小肚子}\end{définition}
\begin{définition}\pfra{Bas-ventre.}\end{définition}
\end{entrée}

\begin{entrée}
{bo}{}{ⓔbo}\formedesurface{--}\newline
\classe{语气助词}\ton{0}\begin{définition}\peng{Final particle expressing vigorous affirmation.}\end{définition}
\begin{définition}\pcmn{句尾助词:啵(汉语借词)}\end{définition}
\begin{définition}\pfra{Particule finale empruntée au chinois, exprimant une affirmation vigoureuse: Ah que si!}\end{définition}
\end{entrée}

\begin{entrée}
{bo˧}{₁}{ⓔbo˧ⓗ1}\formedesurface{bo˧}\newline
\classe{名词}\ton{M}
1
\paradigme{\pcmn{:} \p{}}
\begin{définition}\peng{Small slope at the edge of a field; one can walk on it, sit on it, put objects down here... In the old times, when land was abundant, peasants would leave these small areas uncultivated. Today, peasants tend to plane them down to enlarge cultivated fields.}\end{définition}
\begin{définition}\pcmn{过去,自然形成的小坡,便于农民坐着休息、放东西。现在用拖拉机拉平,成田地。}\end{définition}
\begin{définition}\pfra{Petit talus en bord de champ, sur lequel on peut marcher, s'asseoir, poser des objets. Il s'agit de petites étendues de terres pas aisément cultivables, que les paysans d'autrefois, ayant des terres en abondance, ne se fatiguaient pas à aplanir pour les mettre en culture.}\end{définition}
\begin{exemple}\pnru{ʈʂʰɯ˧-qo˧ | bo˧ ɖɯ˧-ɭɯ˧ tʰi˧-di˩.}\hspace{5pt}\peng{Here, there is a small dike.}\hspace{5pt}\pcmn{这里有一个小坡。}\hspace{5pt}\pfra{ici, il y a une diguette.}\end{exemple}
\begin{exemple}\pnru{bo˧-kʰi˧}\hspace{5pt}\peng{the edge of the small dike}\hspace{5pt}\pcmn{小坡的边沿}\hspace{5pt}\pfra{le bord de la diguette}\end{exemple}
\begin{exemple}\pnru{bo˧ | ɖɯ˧-pʰæ˧˥}\hspace{5pt}\peng{a small dike}\hspace{5pt}\pcmn{一片坡}\hspace{5pt}\pfra{une diguette}\end{exemple}
\end{entrée}

\begin{entrée}
{bo˧}{₂}{ⓔbo˧ⓗ2}\formedesurface{bo˧}\newline
\classe{形容词}\ton{M}
2\begin{définition}\peng{Bright, shining.}\end{définition}
\begin{définition}\pcmn{光明,照耀,亮光}\end{définition}
\begin{définition}\pfra{Lumineux.}\end{définition}
\begin{exemple}\pnru{tʰi˧-bo˧-dʑo˧}\hspace{5pt}\peng{It casts light / it is bright. (Definition of a lamp.)}\hspace{5pt}\pcmn{(它)在发光。(描写灯)}\hspace{5pt}\pfra{Ca éclaire, c'est lumineux. (Définition d'une lampe.)}\end{exemple}
\begin{exemple}\pnru{bo˧-hĩ˧}\hspace{5pt}\peng{|fg{rel}}\hspace{5pt}\pcmn{发亮的、发光的}\hspace{5pt}\pfra{|fg{rel}}\end{exemple}
\end{entrée}

\begin{entrée}
{bo˧β}{}{ⓔbo˧β}\formedesurface{bo˧}\newline
\classe{动词}\ton{Mβ}\begin{définition}\peng{To spin; to winnow.}\end{définition}
\begin{définition}\pcmn{纺(麻线),使旋转,簸扬}\end{définition}
\begin{définition}\pfra{Faire tourner (la quenouille, pour filer le fil de lin); vanner. Ces deux procès peuvent paraître sans lien: pour vanner, on secoue; pour filer, on enroule le fil de façon circulaire. En fait, le fil ne tourne pas de façon lisse: il oscille sans cesse, d'où la proximité sémantique avec ‘vanner’.}\end{définition}
\begin{exemple}\pnru{tso˧∼tso˧ bo˧}\hspace{5pt}\peng{to winnow something; to spin something}\hspace{5pt}\pcmn{簸扬东西(用簸箕)}\hspace{5pt}\pfra{vanner quelque chose; faire tourner quelque chose (oscillation irrégulière)}\end{exemple}
\end{entrée}

\begin{entrée}
{bo˩˧}{}{ⓔbo˩˧}\formedesurface{bo˩˥}\newline
\classe{名词}\ton{LM}
\paradigme{\pcmn{:} \p{}}
\begin{définition}\peng{Pig.}\end{définition}
\begin{définition}\pcmn{猪}\end{définition}
\begin{définition}\pfra{Porc, cochon.}\end{définition}
\begin{exemple}\pnru{bo˩ hwæ˧-ze˧}\hspace{5pt}\peng{…bought (some/a) pig}\hspace{5pt}\pcmn{买了猪}\hspace{5pt}\pfra{…a acheté (du/un) porc}\end{exemple}
\begin{exemple}\pnru{bo˩ dzɯ˥-ze˩}\hspace{5pt}\peng{…ate (some/the) pig}\hspace{5pt}\pcmn{吃了猪}\hspace{5pt}\pfra{…a mangé (du/un) porc}\end{exemple}
\end{entrée}

\begin{entrée}
{bo˩α}{}{ⓔbo˩α}\formedesurface{bo˩˥}\newline
\classe{动词}\ton{Lα}\begin{définition}\peng{To kiss.}\end{définition}
\begin{définition}\pcmn{亲吻}\end{définition}
\begin{définition}\pfra{Embrasser.}\end{définition}
\begin{exemple}\pnru{ɖɯ˧-bo˩-ɻ̍˩}\hspace{5pt}\peng{|fg{delimitative} \_ |fg{inceptive}}\hspace{5pt}\pcmn{(主动的)亲吻一下(如:大人对小孩的一种表示)}\hspace{5pt}\pfra{|fg{délimitatif} \_ |fg{inchoatif}}\end{exemple}
\begin{exemple}\pnru{ɖɯ˧-bo˧∼bo˥-ɻ̍˩}\hspace{5pt}\peng{|fg{delimitative} \_ |fg{red} |fg{inceptive}}\hspace{5pt}\pcmn{相互亲吻一下}\hspace{5pt}\pfra{|fg{délimitatif} \_ |fg{red} |fg{inchoatif}}\end{exemple}
\end{entrée}

\begin{entrée}
{bo˩β}{}{ⓔbo˩β}\formedesurface{ɖɯ˧ bo˩}\newline
\classe{量词}\ton{Lβ}\begin{définition}\peng{Classifier for women's traditional hair dresses / headdresses.}\end{définition}
\begin{définition}\pcmn{量词:缎子发带(一条)}\end{définition}
\begin{définition}\pfra{Classificateur des coiffes (parures pour la chevelure des femmes).}\end{définition}
\end{entrée}

\begin{entrée}
{bo˩-bi˧mi˧}{}{ⓔbo˩-bi˧mi˧}\formedesurface{bo˩bi˧mi˧}\newline
\classe{名词}\ton{L-}
\paradigme{\pcmn{:} \p{}}
\begin{définition}\peng{Pig's belly.}\end{définition}
\begin{définition}\pcmn{猪肚子}\end{définition}
\begin{définition}\pfra{Ventre du cochon.}\end{définition}
\end{entrée}

\begin{entrée}
{bo˩-bv̩˥}{}{ⓔbo˩-bv̩˥}\formedesurface{bo˩bv̩˥}\newline
\classe{名词}\ton{LH}
\paradigme{\pcmn{:} \p{}}
\begin{définition}\peng{Pigsty, pigpen.}\end{définition}
\begin{définition}\pcmn{猪圈}\end{définition}
\begin{définition}\pfra{Enclos des porcs.}\end{définition}
\end{entrée}

\begin{entrée}
{bo˩dze˧}{}{ⓔbo˩dze˧}\formedesurface{bo˩dze˥}\newline
\classe{名词}\ton{LM}\begin{définition}\peng{Lark.}\end{définition}
\begin{définition}\pcmn{百灵鸟}\end{définition}
\begin{définition}\pfra{Alouette.}\end{définition}
\end{entrée}

\begin{entrée}
{bo˩dze˧-ko˩dze˩}{}{ⓔbo˩dze˧-ko˩dze˩}\formedesurface{bo˩dze˧ko˩dze˩}\newline
\classe{名词}\ton{LM-L}\begin{définition}\peng{Lark.}\end{définition}
\begin{définition}\pcmn{百灵鸟}\end{définition}
\begin{définition}\pfra{Alouette.}\end{définition}
\end{entrée}

\begin{entrée}
{bo˩-ɣɯ˥}{}{ⓔbo˩-ɣɯ˥}\formedesurface{bo˩ɣɯ˥}\newline
\classe{名词}\ton{LH}
\paradigme{\pcmn{:} \p{}}
\begin{définition}\peng{Pigskin, hogskin.}\end{définition}
\begin{définition}\pcmn{猪皮}\end{définition}
\begin{définition}\pfra{Couenne.}\end{définition}
\begin{exemple}\pnru{bo˩-ɣɯ˧kɯ˩}\hspace{5pt}\peng{same meaning: pigskin}\hspace{5pt}\pcmn{同上:猪皮}\hspace{5pt}\pfra{même sens: couenne}\end{exemple}
\end{entrée}

\begin{entrée}
{bo˩-hɑ\#˥}{}{ⓔbo˩-hɑ\#˥}\formedesurface{bo˩hɑ˥}\newline
\classe{名词}\ton{LM+\#H}
\paradigme{\pcmn{:} \p{}}
\begin{définition}\peng{Pig feed, swill.}\end{définition}
\begin{définition}\pcmn{猪食}\end{définition}
\begin{définition}\pfra{Nourriture du cochon/ pâtée des porcs.}\end{définition}
\end{entrée}

\begin{entrée}
{bo˩-kʰɯ˧}{}{ⓔbo˩-kʰɯ˧}\formedesurface{bo˩kʰɯ˥}\newline
\classe{名词}\ton{LM}
\paradigme{\pcmn{:} \p{}}
\begin{définition}\peng{Pig's feet: dried meat preserved in the skin of the pig's foot.}\end{définition}
\begin{définition}\pcmn{风干猪脚:把小猪腿的骨头取出来,在筒形的猪皮内塞满瘦肉和香料,缝起来,风干。}\end{définition}
\begin{définition}\pfra{Pieds de porc (pièce de boucherie): viande séchée conservée dans la peau du pied de cochon.}\end{définition}
\end{entrée}

\begin{entrée}
{bo˩-kʰv̩˧˥}{₁}{ⓔbo˩-kʰv̩˧˥ⓗ1}\formedesurface{bo˩kʰv̩˧˥}\newline
\classe{名词}\ton{LM+MH\#}
1\begin{définition}\peng{Year of the Pig.}\end{définition}
\begin{définition}\pcmn{猪年}\end{définition}
\begin{définition}\pfra{Année du Cochon.}\end{définition}
\end{entrée}

\begin{entrée}
{bo˩-kʰv̩˧˥}{₂}{ⓔbo˩-kʰv̩˧˥ⓗ2}\formedesurface{bo˩kʰv̩˧˥}\newline
\classe{形容词}\ton{LM+MH\#}
2\begin{définition}\peng{Born in the year of the Pig.}\end{définition}
\begin{définition}\pcmn{属猪(属相)}\end{définition}
\begin{définition}\pfra{Né l'année du Cochon.}\end{définition}
\end{entrée}

\begin{entrée}
{bo˩lo˧}{}{ⓔbo˩lo˧}\formedesurface{bo˩lo˥}\newline
\classe{名词}\ton{LM}
\paradigme{\pcmn{:} \p{}}
\begin{définition}\peng{Manger.}\end{définition}
\begin{définition}\pcmn{槽}\end{définition}
\begin{définition}\pfra{Mangeoire.}\end{définition}
\begin{exemple}\pnru{bo˩lo˥ | ɖɯ˧-ɭɯ˧}\hspace{5pt}\peng{a manger}\hspace{5pt}\pcmn{一个槽}\hspace{5pt}\pfra{une mangeoire}\end{exemple}
\end{entrée}

\begin{entrée}
{bo˩ɬɑ˥}{}{ⓔbo˩ɬɑ˥}\formedesurface{bo˩ɬɑ˥}\newline
\classe{名词}\ton{LH}
\paradigme{\pcmn{:} \p{}}
\begin{définition}\peng{Boar.}\end{définition}
\begin{définition}\pcmn{种猪、公猪}\end{définition}
\begin{définition}\pfra{Verrat.}\end{définition}
\end{entrée}

\begin{entrée}
{bo˩-ɬo˥}{}{ⓔbo˩-ɬo˥}\formedesurface{bo˩ɬo˥}\newline
\classe{名词}\ton{LH}
\paradigme{\pcmn{:} \p{}}
\begin{définition}\peng{Pork ribs.}\end{définition}
\begin{définition}\pcmn{猪肋骨}\end{définition}
\begin{définition}\pfra{Côtes de porc.}\end{définition}
\begin{exemple}\pnru{bo˩ɬo˥ | ɖɯ˧-do˥}\hspace{5pt}\peng{A large piece of pork ribs. This is the piece of meat that households within the extended family offered one another as a token of closeness and kinship, to emphasize the strong bonds between households inside the clan.}\hspace{5pt}\pcmn{一大块猪肋骨。用来祭祀。(情景:摩梭大家庭相互敬献,强调家屋与家屋之间的亲密关系。)}\hspace{5pt}\pfra{Un quartier de côtes de porc. C'est la pièce de boucherie qu'on offre aux autres maisonnées, à l'intérieur de la famille étendue, comme témoignage de solidarité au sein du clan, dont chaque maisonnée est solidaire des autres.}\end{exemple}
\end{entrée}

\begin{entrée}
{bo˩-mæ˧qv̩˩}{}{ⓔbo˩-mæ˧qv̩˩}\formedesurface{bo˩mæ˧qv̩˩}\newline
\classe{名词}\ton{L-L\#}
\paradigme{\pcmn{:} \p{}}
\begin{définition}\peng{Pig's tail.}\end{définition}
\begin{définition}\pcmn{猪尾巴}\end{définition}
\begin{définition}\pfra{Queue du cochon.}\end{définition}
\end{entrée}

\begin{entrée}
{bo˩-mɤ˥}{}{ⓔbo˩-mɤ˥}\formedesurface{bo˩mɤ˥}\newline
\classe{名词}\ton{LH}\begin{définition}\peng{Lard, pig fat.}\end{définition}
\begin{définition}\pcmn{猪油}\end{définition}
\begin{définition}\pfra{Saindoux (gras de porc).}\end{définition}
\end{entrée}

\begin{entrée}
{bo˩mi˧}{}{ⓔbo˩mi˧}\formedesurface{bo˩mi˥}\newline
\classe{名词}\ton{LM}
\paradigme{\pcmn{:} \p{}}
\begin{définition}\peng{Sow.}\end{définition}
\begin{définition}\pcmn{母猪}\end{définition}
\begin{définition}\pfra{Truie.}\end{définition}
\begin{exemple}\pnru{bo˩mi˧ ʑi˩}\hspace{5pt}\peng{to catch (a/the) sow}\hspace{5pt}\pcmn{抓母猪}\hspace{5pt}\pfra{attraper une truie}\end{exemple}
\begin{exemple}\pnru{bo˩mi˧ do˩ (+ze˩)}\hspace{5pt}\peng{…has seen (a/the) sow}\hspace{5pt}\pcmn{见了母猪}\hspace{5pt}\pfra{…a vu (une/la) truie}\end{exemple}
\begin{exemple}\pnru{bo˩mi˧-bæ˧bv̩˥}\hspace{5pt}\peng{sow and piglets}\hspace{5pt}\pcmn{母猪与猪仔}\hspace{5pt}\pfra{truie et porcelets}\end{exemple}
\end{entrée}

\begin{entrée}
{bo˩mi˧-dʑɯ˩pv̩˩}{}{ⓔbo˩mi˧-dʑɯ˩pv̩˩}\formedesurface{bo˩mi˧dʑɯ˩pv̩˩}\newline
\classe{名词}\ton{LM-L}\begin{définition}\peng{|\stylefi{(Dytiscus}, a predaceous diving beetle.}\end{définition}
\begin{définition}\pcmn{龙虱}\end{définition}
\begin{définition}\pfra{Dytique, |\stylefi{Dytiscus}.}\end{définition}
\end{entrée}

\begin{entrée}
{bo˩mi˧-dʑɯ˩pʰv̩˩}{}{ⓔbo˩mi˧-dʑɯ˩pʰv̩˩}\formedesurface{bo˩mi˧dʑɯ˩pʰv̩˩}\newline
\classe{名词}\ton{LM-L}\begin{définition}\peng{Weevil, snout beetle.}\end{définition}
\begin{définition}\pcmn{象鼻虫,米象}\end{définition}
\begin{définition}\pfra{Charançon.}\end{définition}
\end{entrée}

\begin{entrée}
{bo˩mi˧-ɳæ˧tɕʰɯ˩}{}{ⓔbo˩mi˧-ɳæ˧tɕʰɯ˩}\formedesurface{bo˩mi˧ɳæ˧tɕʰɯ˩}\newline
\classe{名词}\ton{LM-L\#}
\paradigme{\pcmn{:} \p{}}
\begin{définition}\peng{Dandelion.}\end{définition}
\begin{définition}\pcmn{蒲公英}\end{définition}
\begin{définition}\pfra{Pissenlit.}\end{définition}
\end{entrée}

\begin{entrée}
{bo˩mi˧-ʁo˩do˩}{}{ⓔbo˩mi˧-ʁo˩do˩}\formedesurface{bo˩mi˧ʁo˩do˩}\newline
\classe{名词}\ton{LM-L}
\paradigme{\pcmn{:} \p{}}
\begin{définition}\peng{|\stylefi{Datura stramonium Linn.}, a medicinal plant: juice extracted from the plant used to be applied to suppurating wounds.}\end{définition}
\begin{définition}\pcmn{曼陀罗。直译:‘母猪的核桃’。}\end{définition}
\begin{définition}\pfra{Pomme épineuse, Datura officinal, Stramoine ou Stramoine commune (|\stylefi{Datura stramonium Linn.}), littéralement «noix des truies»; était employé autrefois pour application sur les plaies qui suppuraient: on en extrayait le jus.}\end{définition}
\end{entrée}

\begin{entrée}
{bo˩pʰv̩˧}{}{ⓔbo˩pʰv̩˧}\formedesurface{bo˩pʰv̩˥}\newline
\classe{名词}\ton{LM}
\paradigme{\pcmn{:} \p{}}
\begin{définition}\peng{Boar.}\end{définition}
\begin{définition}\pcmn{种猪、公猪}\end{définition}
\begin{définition}\pfra{Verrat.}\end{définition}
\end{entrée}

\begin{entrée}
{bo˩qʰæ˧-pv̩˧ʈɤ˥-ɻ̍˩}{}{ⓔbo˩qʰæ˧-pv̩˧ʈɤ˥-ɻ̍˩}\formedesurface{bo˩qʰæ˧pv̩˧ʈɤ˥ɻ̍˩}\newline
\classe{名词}\ton{LM - H\# -}
\paradigme{\pcmn{:} \p{}}
\begin{définition}\peng{Dung beetle.}\end{définition}
\begin{définition}\pcmn{屎壳郎,蜣螂}\end{définition}
\begin{définition}\pfra{Bousier: sorte de scarabée, qui prolifère dans les étables lorsqu'il fait chaud.}\end{définition}
\end{entrée}

\begin{entrée}
{bo˩tv̩\#˥}{}{ⓔbo˩tv̩\#˥}\formedesurface{bo˩tv̩˥}\newline
\classe{名词}\ton{LM+\#H}
\paradigme{\pcmn{:} \p{}}
\begin{définition}\peng{Wild boar.}\end{définition}
\begin{définition}\pcmn{野猪}\end{définition}
\begin{définition}\pfra{Sanglier.}\end{définition}
\begin{exemple}\pnru{bo˩tv̩˧ hwæ˥}\hspace{5pt}\peng{to buy (a/the) wild boar}\hspace{5pt}\pcmn{买野猪}\hspace{5pt}\pfra{acheter un sanglier}\end{exemple}
\end{entrée}

\begin{entrée}
{bo˧tsi˩}{}{ⓔbo˧tsi˩}\formedesurface{bo˧tsi˩}\newline
\classe{名词}\ton{L\#}
\paradigme{\pcmn{:} \p{}}
\begin{définition}\peng{Mane.}\end{définition}
\begin{définition}\pcmn{(马)鬃}\end{définition}
\begin{définition}\pfra{Crinière.}\end{définition}
\begin{exemple}\pnru{ʐwæ˧-bo˧tsi˥\#}\hspace{5pt}\peng{horse mane}\hspace{5pt}\pcmn{马鬃}\hspace{5pt}\pfra{crinière du cheval}\end{exemple}
\begin{exemple}\pnru{bo˩-bo˧tsi˩}\hspace{5pt}\peng{Hog bristle.}\hspace{5pt}\pcmn{猪鬃}\hspace{5pt}\pfra{Soies de porc ou de sanglier. Elles servaient autrefois à confectionner de petites brosses pour les ustensiles de cuisine.}\end{exemple}
\end{entrée}

\begin{entrée}
{bo˩ʈʂʰæ˥}{}{ⓔbo˩ʈʂʰæ˥}\formedesurface{bo˩ʈʂʰæ˥}\newline
\classe{名词}\ton{LH}\begin{définition}\peng{Lard, fat meat of pig; also: boneless, fleshless preserved pork: cured pork made from a whole pig by removing all its internal organs from the opened stomach, seasoned with salt and spices and then the opening is stitched together. The whole sewn pig is then pressed with a slabstone.}\end{définition}
\begin{définition}\pcmn{猪膘,琵琶肉}\end{définition}
\begin{définition}\pfra{Lard; le même terme est employé pour désigner le cochon entier désossé et conservé dans sa peau (au moyen de salpêtre et de sel), qui se conserve une décennie, appelé «viande pipa» en chinois. La glose adoptée dans les textes est: cochon-conservé-entier.}\end{définition}
\end{entrée}

\begin{entrée}
{bo˩zɑ˧mi\#˥}{}{ⓔbo˩zɑ˧mi\#˥}\formedesurface{bo˩zɑ˧mi˧}\newline
\classe{名词}\ton{LM+\#H}\begin{définition}\peng{“Piggy-Sow": a term used as a temporary name for little girls, during the first months of their life, before they are given a real name. This ugly term is intended to disgust evil spirits, which will therefore turn their attention away from the infant. (In the early 21st century, the registry office requires a name to be given at birth; but this name only begins to be used by the family after the first months of life have elapsed.)}\end{définition}
\begin{définition}\pcmn{猪崽子(乳名、贱名:给刚出生的女孩起的名字,让鬼对她不感兴趣,不会来害小孩)}\end{définition}
\begin{définition}\pfra{«Petite Truie»: nom employé pour les petites filles pendant leurs premiers mois, avant qu'on leur donne un vrai nom. Le vilain nom dont on l'affuble vise à éviter que le nourrisson ne soit repéré par de mauvais esprits. (Actuellement, l'état-civil nécessite qu'un nom soit donné dès la naissance; mais celui-ci ne commence à être employé dans les conversations familiales qu'une fois passés les premiers mois.).}\end{définition}
\end{entrée}

\begin{entrée}
{bo˩zo˥}{}{ⓔbo˩zo˥}\formedesurface{bo˩zo˥}\newline
\classe{名词}\ton{LH}\begin{définition}\peng{“Piggy-Boy": a term used as a temporary name for little boys, during the first months of their life, before they are given a real name. This ugly term is intended to disgust evil spirits, which will therefore turn their attention away from the infant. (In the early 21st century, the registry office requires a name to be given at birth; but this name only begins to be used by the family after the first months of life have elapsed.)}\end{définition}
\begin{définition}\pcmn{猪崽子(乳名、贱名:给刚出生的男孩起的名字,让鬼对她不感兴趣,不会来害小孩)}\end{définition}
\begin{définition}\pfra{«Petit Porc»: nom employé pour les petits garçons pendant leurs premiers mois, avant qu'on leur donne un vrai nom. Le vilain nom dont on affuble l'enfant vise à éviter que le nourrisson ne soit repéré par de mauvais esprits. (Actuellement, l'état-civil nécessite qu'un nom soit donné dès la naissance; mais celui-ci ne commence à être employé dans les conversations familiales qu'une fois passés les premiers mois.)}\end{définition}
\end{entrée}

\begin{entrée}
{bo˧ʐæ˧}{}{ⓔbo˧ʐæ˧}\formedesurface{bo˧ʐæ˧}\newline
\classe{名词}\ton{M}\begin{définition}\peng{Glass (as a substance: window panes, drinking glasses… are made of glass).}\end{définition}
\begin{définition}\pcmn{玻璃}\end{définition}
\begin{définition}\pfra{Verre (matière).}\end{définition}
\begin{exemple}\pnru{bo˧ʐæ˧-tɕʰɤ˩ʈʂv̩˩}\hspace{5pt}\peng{goblet for drinking tea (made of glass)}\hspace{5pt}\pcmn{玻璃茶杯}\hspace{5pt}\pfra{gobelet à thé en verre}\end{exemple}
\end{entrée}

\begin{entrée}
{bo˧ʐæ˧ʈæ˧qʰv̩\#˥}{}{ⓔbo˧ʐæ˧ʈæ˧qʰv̩\#˥}\formedesurface{bo˧ʐæ˧ʈæ˧qʰv̩˧}\newline
\classe{名词}\ton{\#H}\begin{définition}\peng{Echo (in some places in the mountains, there is an echo).}\end{définition}
\begin{définition}\pcmn{有回音的地方,回音}\end{définition}
\begin{définition}\pfra{Écho (en certains endroits des montagnes, il y a un écho).}\end{définition}
\begin{exemple}\pnru{bo˧ʐæ˧ʈæ˧qʰv̩˧ | tʰi˧-ɖʐɯ˩∼ɖʐɯ˩!}\hspace{5pt}\peng{the echo resonates}\hspace{5pt}\pcmn{有回音,回音一阵阵}\hspace{5pt}\pfra{Il y a de l'écho!}\end{exemple}
\end{entrée}

\begin{entrée}
{bõ}{}{ⓔbõ}\formedesurface{bõ}\newline
\classe{}\ton{0}\begin{définition}\peng{Noise of a shock between two hard objects, for instance the sound of an axe splitting a log: Bang!}\end{définition}
\begin{définition}\pcmn{形声词:斧头把粗的木头开花成两半。砰! / 啪!}\end{définition}
\begin{définition}\pfra{Bruit d'un choc entre deux objets durs, par exemple un coup de hache qui fend une pièce de bois: Boum!}\end{définition}
\end{entrée}

\begin{entrée}
{bv̩˥}{}{ⓔbv̩˥}\formedesurface{bv̩˧}\newline
\classe{名词}\ton{\#H}
\paradigme{\pcmn{:} \p{}}
\begin{définition}\peng{Worm; insect.}\end{définition}
\begin{définition}\pcmn{虫}\end{définition}
\begin{définition}\pfra{Ver; insecte.}\end{définition}
\begin{exemple}\pnru{bv̩˧ tʰv̩˧-mi˧˥ / bv̩˧ tʰv̩˧-mi˥\#}\hspace{5pt}\peng{|fg{n}+|fg{dem}+|fg{clf}}\hspace{5pt}\pcmn{这只虫}\hspace{5pt}\pfra{|fg{n}+|fg{dem}+|fg{clf}}\end{exemple}
\end{entrée}

\begin{entrée}
{bv̩˥}{₁}{ⓔbv̩˥ⓗ1}\formedesurface{bv̩˧}\newline
\classe{形容词}\ton{H}
1\begin{définition}\peng{Thick (tree trunk); coarse (flour, powder).}\end{définition}
\begin{définition}\pcmn{粗(树粗大,粉末不精细……)}\end{définition}
\begin{définition}\pfra{Épais (tronc); grossier (farine, poudre).}\end{définition}
\begin{exemple}\pnru{qʰɑ˧-bv̩˧-gv̩˧}\hspace{5pt}\peng{very thick}\hspace{5pt}\pcmn{很粗、多粗的、好粗}\hspace{5pt}\pfra{très épais}\end{exemple}
\begin{exemple}\pnru{qʰɑ˧bv̩˧∼bv̩˧-gv̩˧}\hspace{5pt}\peng{very thick (as above)}\hspace{5pt}\pcmn{很粗、多粗的、好粗(同上)}\hspace{5pt}\pfra{très épais (idem ci-dessus)}\end{exemple}
\end{entrée}

\begin{entrée}
{bv̩˥}{₂}{ⓔbv̩˥ⓗ2}\formedesurface{bv̩˧}\newline
\classe{动词}\ton{H}
2\begin{définition}\peng{To distribute things, to allot things, to divide things between several persons.}\end{définition}
\begin{définition}\pcmn{分东西}\end{définition}
\begin{définition}\pfra{Partager, distribuer, répartir, diviser (ancien mot pour ‘donner’).}\end{définition}
\begin{exemple}\pnru{ɖɯ˧-v̩˧ ɖɯ˧-kʰwɤ˥ | le˧-bv̩˧∼bv̩˧}\hspace{5pt}\peng{to share, giving each person a piece}\hspace{5pt}\pcmn{分给一人一块}\hspace{5pt}\pfra{partager, un morceau par personne}\end{exemple}
\begin{exemple}\pnru{le˧-bv̩˧∼bv̩˧ tʰi˧-kwɤ˩}\hspace{5pt}\peng{to scatter all over the place, to lay out in no good order}\hspace{5pt}\pcmn{弄乱丢着,散开丢着}\hspace{5pt}\pfra{littéralement ‘séparer et poser’; sens: mettre en désordre, disposer en désordre}\end{exemple}
\end{entrée}

\begin{entrée}
{bv̩˥}{₃}{ⓔbv̩˥ⓗ3}\formedesurface{bv̩˧}\newline
\classe{动词}\ton{H}
3\begin{définition}\peng{To roast, to grill.}\end{définition}
\begin{définition}\pcmn{烤,炙}\end{définition}
\begin{définition}\pfra{Griller, faire griller.}\end{définition}
\begin{exemple}\pnru{hɑ˧ tʰi˧-bv̩˥}\hspace{5pt}\peng{to roast food, to roast cereals}\hspace{5pt}\pcmn{烤饭}\hspace{5pt}\pfra{faire griller du riz}\end{exemple}
\end{entrée}

\begin{entrée}
{bv̩˧}{}{ⓔbv̩˧}\formedesurface{bv̩˧}\newline
\classe{名词}\ton{M}
\paradigme{\pcmn{:} \p{}}
\begin{définition}\peng{Intestine.}\end{définition}
\begin{définition}\pcmn{肠子}\end{définition}
\begin{définition}\pfra{Intestin.}\end{définition}
\end{entrée}

\begin{entrée}
{=bv̩˧}{}{ⓔ=bv̩˧}\formedesurface{bv̩˧}\newline
\classe{附着词}\ton{M}\begin{définition}\peng{Possessive.}\end{définition}
\begin{définition}\pcmn{属式:的}\end{définition}
\begin{définition}\pfra{Possessif.}\end{définition}
\end{entrée}

\begin{entrée}
{bv̩˩}{}{ⓔbv̩˩}\formedesurface{bv̩˧}\newline
\classe{名词}\ton{L}
\paradigme{\pcmn{:} \p{}}
\begin{définition}\peng{Pen, corral for cattle.}\end{définition}
\begin{définition}\pcmn{牲畜圈(单音节)}\end{définition}
\begin{définition}\pfra{Enclos (monosyllabe).}\end{définition}
\begin{exemple}\pnru{bv̩˩ qo˩˥}\hspace{5pt}\peng{inside the corral}\hspace{5pt}\pcmn{牲畜圈里面}\hspace{5pt}\pfra{dans l'enclos}\end{exemple}
\begin{exemple}\pnru{bv̩˩qo˩ ʈæ˥}\hspace{5pt}\peng{to enclose (cattle) inside the pen}\hspace{5pt}\pcmn{关在圈里}\hspace{5pt}\pfra{enfermer dans l'étable}\end{exemple}
\end{entrée}

\begin{entrée}
{bv̩˩˧}{₁}{ⓔbv̩˩˧ⓗ1}\formedesurface{bv̩˩˥}\newline
\classe{名词}\ton{LM}
1
\paradigme{\pcmn{:} \p{}}
\begin{définition}\peng{Yak, Bos grunniens. The same term is used for wild yaks and domesticated yaks.}\end{définition}
\begin{définition}\pcmn{牦牛/野牦牛}\end{définition}
\begin{définition}\pfra{Yak, Bos grunniens (sauvage ou domestiqué).}\end{définition}
\begin{exemple}\pnru{bv̩˩-hṽ̩˩˥}\hspace{5pt}\peng{yak hair}\hspace{5pt}\pcmn{牦牛毛}\hspace{5pt}\pfra{poil de yak}\end{exemple}
\begin{exemple}\pnru{bv̩˩ dzɯ˧-ze˩}\hspace{5pt}\peng{…ate (some) yak}\hspace{5pt}\pcmn{吃了牦牛}\hspace{5pt}\pfra{…a mangé (du) yak}\end{exemple}
\begin{exemple}\pnru{bv̩˩ hwæ˧-ze˧}\hspace{5pt}\peng{…bought (some) yak}\hspace{5pt}\pcmn{买了牦牛}\hspace{5pt}\pfra{…a acheté (du) yak}\end{exemple}
\end{entrée}

\begin{entrée}
{bv̩˩˧}{₂}{ⓔbv̩˩˧ⓗ2}\formedesurface{bv̩˩˥}\newline
\classe{名词}\ton{LM}
2
\paradigme{\pcmn{:} \p{}}
\begin{définition}\peng{Food steamer.}\end{définition}
\begin{définition}\pcmn{蒸笼}\end{définition}
\begin{définition}\pfra{Étuve.}\end{définition}
\end{entrée}

\begin{entrée}
{bv̩˩α}{₁}{ⓔbv̩˩αⓗ1}\formedesurface{bv̩˩˥}\newline
\classe{动词}\ton{Lα}
1\begin{définition}\peng{To hatch, to incubate.}\end{définition}
\begin{définition}\pcmn{孵}\end{définition}
\begin{définition}\pfra{Couver.}\end{définition}
\begin{exemple}\pnru{æ˩mi˧ bv̩˩}\hspace{5pt}\peng{The hen is hatching eggs.}\hspace{5pt}\pcmn{母鸡孵蛋}\hspace{5pt}\pfra{La poule couve.}\end{exemple}
\end{entrée}

\begin{entrée}
{bv̩˩α}{₂}{ⓔbv̩˩αⓗ2}\formedesurface{bv̩˩˥}\newline
\classe{动词}\ton{Lα}
2\begin{définition}\peng{To steam, to cook by steaming.}\end{définition}
\begin{définition}\pcmn{蒸}\end{définition}
\begin{définition}\pfra{Cuire à la vapeur, étuver.}\end{définition}
\begin{exemple}\pnru{le˧-bv̩˩-ze˩}\hspace{5pt}\peng{|fg{accomp} \_ |fg{pfv}}\hspace{5pt}\pcmn{蒸了}\hspace{5pt}\pfra{|fg{accomp} \_ |fg{pfv}}\end{exemple}
\begin{exemple}\pnru{pɤ˩jɤ˧ bv̩˥}\hspace{5pt}\peng{to steam buns}\hspace{5pt}\pcmn{蒸馒头}\hspace{5pt}\pfra{cuire de la pâte à pain à la vapeur, faire des petits pains à la vapeur}\end{exemple}
\begin{exemple}\pnru{hɑ˧ bv̩˥∼bv̩˩}\hspace{5pt}\peng{to steam rice}\hspace{5pt}\pcmn{蒸米饭}\hspace{5pt}\pfra{cuire du riz à la vapeur}\end{exemple}
\end{entrée}

\begin{entrée}
{bv̩˩α}{₃}{ⓔbv̩˩αⓗ3}\formedesurface{bv̩˩˥}\newline
\classe{动词}\ton{Lα}
3\begin{définition}\peng{To live (one's life).}\end{définition}
\begin{définition}\pcmn{过(日子)}\end{définition}
\begin{définition}\pfra{Vivre, couler des jours, vivre (sa vie).}\end{définition}
\begin{exemple}\pnru{zɯ˧ bv̩˩}\hspace{5pt}\peng{to live one's life}\hspace{5pt}\pcmn{过日子}\hspace{5pt}\pfra{vivre sa vie}\end{exemple}
\begin{exemple}\pnru{hĩ˧-zɯ˧ bv̩˥, | lo˧hɑ˧!}\hspace{5pt}\peng{Living one's life is hard! / Life is tough!}\hspace{5pt}\pcmn{人一生,不容易!/ 生活,是艰难的!}\hspace{5pt}\pfra{Vivre sa vie, c'est difficile! / La vie humaine, c'est pas facile! / La vie est dure!}\end{exemple}
\begin{exemple}\pnru{hĩ˧-zɯ˧ | le˧-bv̩˩-ze˩.}\hspace{5pt}\peng{(Her/his) life has gone by! / (Her/his) life is over! (A reflection after someone's decease.)}\hspace{5pt}\pcmn{他的人生,就结束了!(情景:一个人去世了,葬礼的时候,有人这样说。)}\hspace{5pt}\pfra{(Sa) vie a passé! / (Sa) vie est terminée! (Réflexion après le décès de quelqu'un.)}\end{exemple}
\begin{exemple}\pnru{qʰwɤ˧-ɭɯ˥, | ʈʂʰæ˧-mɤ˧-dʑɯ˧! | ʈʂʰɯ˧ ɖɯ˧-zɯ˧ bv̩˩-ze˩!}\hspace{5pt}\peng{He never had to do the washing-up (in his entire life)! That's how he spent his lifetime (=without any practical concerns)! (About an Oxford professor whose every need in daily life was attended to by the college scouts.)}\hspace{5pt}\pcmn{他从来没有洗过碗!他这辈子,是这样过去的!(情景:关于一个英国知识分子,完全不用管家务、日常生活中的活儿:有人来管一切。)}\hspace{5pt}\pfra{Il n'a jamais eu à faire la vaisselle (de sa vie)! Voilà comment s'est passée toute sa vie! (Commentaire au sujet de la vie d'un universitaire d'Oxford qui s'était entièrement soustrait aux tâches manuelles.)}\end{exemple}
\begin{exemple}\pnru{ɖɯ˧-ɲi˥∼ɖɯ˩-ɲi˩ | bv̩˩ lo˩ fv̩˩˥!}\hspace{5pt}\peng{How easily days go by! / How time flies!}\hspace{5pt}\pcmn{日子过得真快!(直译:‘一天又一天,日子过得真容易’)}\hspace{5pt}\pfra{Les journées passent bien vite! / Comme le temps passe!}\end{exemple}
\end{entrée}

\begin{entrée}
{bv̩˩α}{₄}{ⓔbv̩˩αⓗ4}\formedesurface{bv̩˩˥}\newline
\classe{动词}\ton{Lα}
4
\sens{1}
\begin{définition}\peng{To sprinkle water.}\end{définition}
\begin{définition}\pcmn{泼水,浇(浇菜),撒(水)}\end{définition}
\begin{définition}\pfra{Asperger; arroser.}\end{définition}
\begin{exemple}\pnru{le˧-bv̩˩-ze˩}\hspace{5pt}\peng{|fg{accomp} \_ |fg{pfv}}\hspace{5pt}\pcmn{泼了}\hspace{5pt}\pfra{|fg{accomp} \_ |fg{pfv}}\end{exemple}
\begin{exemple}\pnru{dʑɯ˩ bv̩˩˥}\hspace{5pt}\peng{to sprinkle water}\hspace{5pt}\pcmn{泼水、撒水}\hspace{5pt}\pfra{asperger d'eau; arroser}\end{exemple}
\begin{exemple}\pnru{ɖɯ˧-bv̩˧∼bv̩˥-ɻ̍˩}\hspace{5pt}\peng{|fg{delimitative} |fg{red} |fg{inceptive}}\hspace{5pt}\pcmn{泼一泼}\hspace{5pt}\pfra{|fg{délimitatif} |fg{red} |fg{inchoatif}}\end{exemple}
\begin{exemple}\pnru{le˧-bv̩˧∼bv̩˥-ze˩}\hspace{5pt}\peng{|fg{accomp} |fg{red} |fg{pfv}}\hspace{5pt}\pcmn{泼了一点}\hspace{5pt}\pfra{|fg{accomp} |fg{red} |fg{pfv}}\end{exemple}\sens{2}
\begin{définition}\peng{To sow (seeds).}\end{définition}
\begin{définition}\pcmn{撒(种子)}\end{définition}
\begin{définition}\pfra{Disperser, semer (ex.: des graines).}\end{définition}
\begin{exemple}\pnru{ɻæ˩ bv̩˥}\hspace{5pt}\peng{to sow seeds}\hspace{5pt}\pcmn{撒种子}\hspace{5pt}\pfra{semer à la volée, répandre des graines (pour les semailles)}\end{exemple}
\begin{exemple}\pnru{tʰi˧-bv̩˩-ɻ̍˩}\hspace{5pt}\peng{Go ahead and sow!}\hspace{5pt}\pcmn{撒吧!}\hspace{5pt}\pfra{Sème donc!}\end{exemple}
\begin{exemple}\pnru{tʰi˧-bv̩˩-qɑ˩!}\hspace{5pt}\peng{Sow!}\hspace{5pt}\pcmn{撒吧!}\hspace{5pt}\pfra{Sème!}\end{exemple}
\end{entrée}

\begin{entrée}
{bv̩˩α}{₅}{ⓔbv̩˩αⓗ5}\formedesurface{bv̩˩˥}\newline
\classe{形容词}\ton{Lα}
5\begin{définition}\peng{Thin, scarce, sparse (e.g. hair).}\end{définition}
\begin{définition}\pcmn{少、薄}\end{définition}
\begin{définition}\pfra{Clairsemé, à nu.}\end{définition}
\begin{exemple}\pnru{ʁo˧ bv̩˧˥}\hspace{5pt}\peng{bald (literally “the head (has) scarce (hair)")}\hspace{5pt}\pcmn{头秃、头发很少}\hspace{5pt}\pfra{chauve (littéralement «la tête est à nu»)}\end{exemple}
\begin{exemple}\pnru{ʁo˧ bv̩˧ hĩ˥}\hspace{5pt}\peng{bald person}\hspace{5pt}\pcmn{头发少的人}\hspace{5pt}\pfra{un homme chauve}\end{exemple}
\begin{exemple}\pnru{ʁo˧ bv̩˧ zo˥}\hspace{5pt}\peng{Man who has scarce hair.}\hspace{5pt}\pcmn{头发少的男人}\hspace{5pt}\pfra{Homme à la tête dégarnie.}\end{exemple}
\begin{exemple}\pnru{ʈʂʰɯ˧ | ʁo˧ bv̩˧-ze˥}\hspace{5pt}\peng{He lost his hair, he went bald}\hspace{5pt}\pcmn{他秃头了,他头发掉了}\hspace{5pt}\pfra{il a perdu ses cheveux, il est devenu chauve}\end{exemple}
\begin{exemple}\pnru{ʁo˧qʰwɤ˩ | le˧-bv̩˩-ze˩}\hspace{5pt}\peng{(his) head has gone bald}\hspace{5pt}\pcmn{(他)秃头了。}\hspace{5pt}\pfra{(sa) tête s'est dégarnie, (sa) tête est devenue chauve}\end{exemple}
\begin{exemple}\pnru{bv̩˩-hĩ˩˥}\hspace{5pt}\peng{|fg{rel}}\hspace{5pt}\pcmn{秃的}\hspace{5pt}\pfra{|fg{rel}}\end{exemple}
\end{entrée}

\begin{entrée}
{bv̩˩di˩}{}{ⓔbv̩˩di˩}\formedesurface{bv̩˩di˩˥}\newline
\classe{名词}\ton{L}
\paradigme{\pcmn{:} \p{}}
\begin{définition}\peng{Food steamer.}\end{définition}
\begin{définition}\pcmn{蒸笼}\end{définition}
\begin{définition}\pfra{Étuve.}\end{définition}
\end{entrée}

\begin{entrée}
{bv̩˩dze˩}{₁}{ⓔbv̩˩dze˩ⓗ1}\formedesurface{bv̩˩dze˩˥}\newline
\classe{名词}\ton{L}
1
\paradigme{\pcmn{:} \p{}}
\begin{définition}\peng{Large spoon, used for rice and soup.}\end{définition}
\begin{définition}\pcmn{舀饭的勺,舀汤的勺}\end{définition}
\begin{définition}\pfra{Grosse cuillère (pour servir le riz, la soupe, etc.).}\end{définition}
\end{entrée}

\begin{entrée}
{bv̩˩dze˩}{₂}{ⓔbv̩˩dze˩ⓗ2}\formedesurface{ɖɯ˧ bv̩˩dze˩}\newline
\classe{量词}\ton{L}
2\begin{définition}\peng{Ladleful.}\end{définition}
\begin{définition}\pcmn{量词:勺}\end{définition}
\begin{définition}\pfra{Classificateur des cuillerées.}\end{définition}
\begin{exemple}\pnru{ɖɯ˧-bv̩˩dze˩}\hspace{5pt}\peng{one ladleful}\hspace{5pt}\pcmn{一勺}\hspace{5pt}\pfra{une louchée, une louche de}\end{exemple}
\end{entrée}

\begin{entrée}
{bv̩˧ɖæ˧}{}{ⓔbv̩˧ɖæ˧}\formedesurface{bv̩˧ɖæ˧}\newline
\classe{形容词}\ton{M}
\étymologie{
bv̩˧; ɖæ˥
}\begin{définition}\peng{With a bad temper. Literally ‘short-intestined': in folk representation, short intestines are associated with hasty emotional reactions, whereas long intestines allow their owner to digest vexations calmly.}\end{définition}
\begin{définition}\pcmn{脾气坏。直译:‘肠子短’。}\end{définition}
\begin{définition}\pfra{De mauvaise humeur, ayant mauvais caractère.}\end{définition}
\begin{exemple}\pnru{ʈʂʰɯ˧ | bv̩˧ɖæ˧-ze˩!}\hspace{5pt}\peng{He is is a bad mood now.}\hspace{5pt}\pcmn{他脾气坏了!/ 他生气了!}\hspace{5pt}\pfra{Il est de mauvais poil! / Il est de mauvaise humeur!}\end{exemple}
\end{entrée}

\begin{entrée}
{bv̩˧hu˧˥}{}{ⓔbv̩˧hu˧˥}\formedesurface{bv̩˩hu˧˥}\newline
\classe{名词}\ton{MH\#}
\paradigme{\pcmn{:} \p{}}
\begin{définition}\peng{Bowels: intestine+stomach.}\end{définition}
\begin{définition}\pcmn{胃与肠}\end{définition}
\begin{définition}\pfra{Tube digestif: estomac+intestin.}\end{définition}
\end{entrée}

\begin{entrée}
{bv̩˩hwɤ˩}{}{ⓔbv̩˩hwɤ˩}\formedesurface{bv̩˩hwɤ˩˥}\newline
\classe{名词}\ton{L}
\paradigme{\pcmn{:} \p{}}
\begin{définition}\peng{Wooden hut where shepherds stay while herding their flock on the mountain.}\end{définition}
\begin{définition}\pcmn{山上放牧的人暂时住的木头小房}\end{définition}
\begin{définition}\pfra{Cabane de berger, sur la montagne; n'est pas occupée à l'année, mais est assez solide pour être utilisée année après année, à la différence des cabanes provisoires construites lorsqu'on doit rester qq jours sur la montagne pour couper du bois.}\end{définition}
\end{entrée}

\begin{entrée}
{bv̩˧kʰɯ˧˥}{}{ⓔbv̩˧kʰɯ˧˥}\formedesurface{bv̩˧kʰɯ˧˥}\newline
\classe{名词}\ton{MH\#}
\paradigme{\pcmn{:} \p{}}
\begin{définition}\peng{Worm.}\end{définition}
\begin{définition}\pcmn{虫}\end{définition}
\begin{définition}\pfra{Ver.}\end{définition}
\end{entrée}

\begin{entrée}
{bv̩˩ɭɯ˩}{}{ⓔbv̩˩ɭɯ˩}\formedesurface{bv̩˩ɭɯ˩˥}\newline
\classe{名词}\ton{L}
\paradigme{\pcmn{:} \p{}}
\begin{définition}\peng{Kidneys.}\end{définition}
\begin{définition}\pcmn{肾}\end{définition}
\begin{définition}\pfra{Rein.}\end{définition}
\end{entrée}

\begin{entrée}
{bv̩˧mi˧}{₁}{ⓔbv̩˧mi˧ⓗ1}\formedesurface{bv̩˧mi˧}\newline
\classe{名词}\ton{M}
1
\paradigme{\pcmn{:} \p{}}
\begin{définition}\peng{Female yak, dri, drimo, nak.}\end{définition}
\begin{définition}\pcmn{母牦牛}\end{définition}
\begin{définition}\pfra{Yak femelle, dri, drimo, nak.}\end{définition}
\begin{exemple}\pnru{bv̩˧mi˧-bv̩˩ʂwæ˩}\hspace{5pt}\peng{female yak and castrated yak}\hspace{5pt}\pcmn{母牦牛与阉割牦牛}\hspace{5pt}\pfra{yak femelle et yak châtré}\end{exemple}
\begin{exemple}\pnru{bv̩˧mi˧-bv̩˧zo\#˥}\hspace{5pt}\peng{female yak and yak calf (baby yak)}\hspace{5pt}\pcmn{母牦牛与小牦牛}\hspace{5pt}\pfra{maman yack et petit yack}\end{exemple}
\end{entrée}

\begin{entrée}
{bv̩˧mi˧}{₂}{ⓔbv̩˧mi˧ⓗ2}\formedesurface{bv̩˧mi˧}\newline
\classe{名词}\ton{M}
2
\paradigme{\pcmn{:} \p{}}
\begin{définition}\peng{Large food steamer.}\end{définition}
\begin{définition}\pcmn{大蒸笼}\end{définition}
\begin{définition}\pfra{Grande étuve.}\end{définition}
\end{entrée}

\begin{entrée}
{bv̩˧-nɑ˥mi˩}{}{ⓔbv̩˧-nɑ˥mi˩}\formedesurface{bv̩˧nɑ˥mi˩}\newline
\classe{名词}\ton{bv̩˧nɑ˥mi˩}\begin{définition}\peng{|\stylefi{Mythimna separata (Walker)}.}\end{définition}
\begin{définition}\pcmn{玉米黏虫}\end{définition}
\begin{définition}\pfra{|\stylefi{Mythimna separata (Walker)}.}\end{définition}
\end{entrée}

\begin{entrée}
{bv̩˧nv̩˧}{₁}{ⓔbv̩˧nv̩˧ⓗ1}\formedesurface{bv̩˧nv̩˧}\newline
\classe{动词}\ton{M intrans}
1\begin{définition}\peng{To smell, to perceive by smelling.}\end{définition}
\begin{définition}\pcmn{嗅觉,闻到}\end{définition}
\begin{définition}\pfra{Sentir (par l'odorat).}\end{définition}
\begin{exemple}\pnru{no˧ | ɖɯ˧-bv̩˧nv̩˧-ɻ̍˩! | ɖwæ˩˥ | ɕjɤ˧!}\hspace{5pt}\peng{Have a smell! It smells great!}\hspace{5pt}\pcmn{你闻一闻吧!好香!}\hspace{5pt}\pfra{Sens donc! ça sent très bon/c'est très odorant/parfumé!}\end{exemple}
\end{entrée}

\begin{entrée}
{bv̩˧nv̩˧}{₂}{ⓔbv̩˧nv̩˧ⓗ2}\formedesurface{bv̩˧nv̩˧}\newline
\classe{形容词}\ton{M intrans}
2\begin{définition}\peng{Stinking, smelly.}\end{définition}
\begin{définition}\pcmn{臭}\end{définition}
\begin{définition}\pfra{Malodorant, puant, qui a une mauvaise odeur.}\end{définition}
\begin{exemple}\pnru{bv̩˧nv̩˧-ze˧}\hspace{5pt}\peng{|fg{pfv}}\hspace{5pt}\pcmn{臭了}\hspace{5pt}\pfra{|fg{pfv}}\end{exemple}
\end{entrée}

\begin{entrée}
{bv̩˧pʰv̩˧}{}{ⓔbv̩˧pʰv̩˧}\formedesurface{bv̩˧pʰv̩˧}\newline
\classe{名词}\ton{M}
\paradigme{\pcmn{:} \p{}}
\begin{définition}\peng{Male yak (elicited form; the commonly used form is \stylefv{/bv̩}˩ʂwæ˩/).}\end{définition}
\begin{définition}\pcmn{公牦牛}\end{définition}
\begin{définition}\pfra{Yak mâle. Ce mot est une forme élicitée; la forme usuelle est: \stylefv{/bv̩}˩ʂwæ˩/.}\end{définition}
\end{entrée}

\begin{entrée}
{bv̩˩qo˩-bv̩˧qʰæ˩}{}{ⓔbv̩˩qo˩-bv̩˧qʰæ˩}\formedesurface{bv̩˩qo˩bv̩˧qʰæ˩}\newline
\classe{名词}\ton{L-L\#}
\paradigme{\pcmn{:} \p{}}
\begin{définition}\peng{Manure, dung. Literally ‘manure in pen'.}\end{définition}
\begin{définition}\pcmn{农家粪(直译:‘牲口圈里的肥料’)、粪}\end{définition}
\begin{définition}\pfra{Fumier. Littéralement: ‘excréments d'étable'.}\end{définition}
\end{entrée}

\begin{entrée}
{bv̩˩qo˩-qʰæ˩}{}{ⓔbv̩˩qo˩-qʰæ˩}\formedesurface{bv̩˩qo˩qʰæ˩˥}\newline
\classe{名词}\ton{L}
\paradigme{\pcmn{:} \p{}}
\begin{définition}\peng{Manure, excrement.}\end{définition}
\begin{définition}\pcmn{圈粪、肥料}\end{définition}
\begin{définition}\pfra{Fumier.}\end{définition}
\begin{exemple}\pnru{bv̩˩qo˩-qʰæ˩ tʰv̩˩-ʁwɤ˥}\hspace{5pt}\peng{|fg{n}+|fg{dem}+|fg{clf}}\hspace{5pt}\pcmn{这堆肥料}\hspace{5pt}\pfra{|fg{n}+|fg{dem}+|fg{clf}}\end{exemple}
\end{entrée}

\begin{entrée}
{bv̩˩-qʰæ˩}{}{ⓔbv̩˩-qʰæ˩}\formedesurface{bv̩˩qʰæ˩˥}\newline
\classe{名词}\ton{L}
\paradigme{\pcmn{:} \p{}}
\begin{définition}\peng{Manure, dung.}\end{définition}
\begin{définition}\pcmn{肥料、粪}\end{définition}
\begin{définition}\pfra{Fumier.}\end{définition}
\begin{exemple}\pnru{bv̩˩qʰæ˩ tʰv̩˩-ʁwɤ˥}\hspace{5pt}\peng{|fg{n}+|fg{dem}+|fg{clf}}\hspace{5pt}\pcmn{这堆肥料}\hspace{5pt}\pfra{|fg{n}+|fg{dem}+|fg{clf}}\end{exemple}
\end{entrée}

\begin{entrée}
{bv̩˩qʰv̩˩}{}{ⓔbv̩˩qʰv̩˩}\formedesurface{bv̩˩qʰv̩˩˥}\newline
\classe{名词}\ton{L}
\sens{1}\paradigme{\pcmn{:} \p{}}
\begin{définition}\peng{Conch shell, |\stylefi{Turbinella pyrum L.}. It is used in ceremonies. Each family has a pair of conchs.}\end{définition}
\begin{définition}\pcmn{法螺、海螺、螺号}\end{définition}
\begin{définition}\pfra{Conque, |\stylefi{Turbinella pyrum L.}. On y souffle, comme dans une trompe, lors des cérémonies. Chaque famille en possède une paire.}\end{définition}\sens{2}
\begin{définition}\peng{Lines of the hand on the fingers' first phalanx, called 'conch lines' because of their shape.}\end{définition}
\begin{définition}\pcmn{手指螺纹(因长得像海螺形状而得名)。算命的时候,会看人的手指螺纹。}\end{définition}
\begin{définition}\pfra{Lignes de la main: spécifiquement, lignes de la première phalange des doigts, dont la forme en spirale évoque les conques, objet symbolique important dans la culture na.}\end{définition}
\begin{exemple}\pnru{lo˩qʰwɤ˧=bv̩˧ | bv̩˩qʰv̩˩˥}\hspace{5pt}\peng{the lines of the hand}\hspace{5pt}\pcmn{手指螺纹。直译:‘手指上的海螺’。}\hspace{5pt}\pfra{les lignes de la main}\end{exemple}
\end{entrée}

\begin{entrée}
{bv̩˧qʰv̩˧ʑi˩-hĩ˩}{}{ⓔbv̩˧qʰv̩˧ʑi˩-hĩ˩}\formedesurface{bv̩˧qʰv̩˧ʑi˩hĩ˩}\newline
\classe{名词}\ton{L\#}
\paradigme{\pcmn{:} \p{}}
\begin{définition}\peng{Snail.}\end{définition}
\begin{définition}\pcmn{蜗牛,螺蛳(直译:‘长角的’)}\end{définition}
\begin{définition}\pfra{Escargot.}\end{définition}
\end{entrée}

\begin{entrée}
{bv̩˧ɻ̍\#˥}{}{ⓔbv̩˧ɻ̍\#˥}\formedesurface{bv̩˧ɻ̍˧}\newline
\classe{名词}\ton{\#H}
\paradigme{\pcmn{:} \p{}}
\begin{définition}\peng{Fly.}\end{définition}
\begin{définition}\pcmn{苍蝇}\end{définition}
\begin{définition}\pfra{Mouche.}\end{définition}
\begin{exemple}\pnru{bv̩˧ɻ̍˧ ʈʂʰɯ˧-mi˧˥ / bv̩˧ɻ̍˧ ʈʂʰɯ˧-mi˥\#}\hspace{5pt}\peng{|fg{n}+|fg{dem}+|fg{clf}}\hspace{5pt}\pcmn{这只苍蝇}\hspace{5pt}\pfra{|fg{n}+|fg{dem}+|fg{clf}}\end{exemple}
\end{entrée}

\begin{entrée}
{bv̩˧ʂæ˧}{}{ⓔbv̩˧ʂæ˧}\formedesurface{bv̩˧ʂæ˧}\newline
\classe{形容词}\ton{M}
\étymologie{
bv̩˧; ʂæ˧
}\begin{définition}\peng{Good-tempered, with a good mood, good-humoured.}\end{définition}
\begin{définition}\pcmn{脾气好、随和。直译:‘肠子长’。}\end{définition}
\begin{définition}\pfra{De bonne humeur, ayant bon caractère.}\end{définition}
\begin{exemple}\pnru{ʈʂʰɯ˧ | bv̩˧ʂæ˧-ze˩}\hspace{5pt}\peng{He is in a good mood now.}\hspace{5pt}\pcmn{他脾气好了。}\hspace{5pt}\pfra{il est de bonne humeur}\end{exemple}
\begin{exemple}\pnru{bv̩˧ʂæ˧ | ʐwæ˩˥}\hspace{5pt}\peng{in a very good mood}\hspace{5pt}\pcmn{脾气很好}\hspace{5pt}\pfra{de très bonne humeur}\end{exemple}
\end{entrée}

\begin{entrée}
{bv̩˩ʂwæ˩}{}{ⓔbv̩˩ʂwæ˩}\formedesurface{bv̩˩ʂwæ˩˥}\newline
\classe{名词}\ton{L}
\paradigme{\pcmn{:} \p{}}
\begin{définition}\peng{Castrated yak.}\end{définition}
\begin{définition}\pcmn{阉割过的牦牛}\end{définition}
\begin{définition}\pfra{Yak châtré.}\end{définition}
\begin{exemple}\pnru{bv̩˩ʂwæ˩-bv̩˥mi˩}\hspace{5pt}\peng{castrated yak and female yak}\hspace{5pt}\pcmn{阉割过的公牦牛与母牦牛}\hspace{5pt}\pfra{yak châtré et yak femelle}\end{exemple}
\end{entrée}

\begin{entrée}
{bv̩˧tɕi˧}{}{ⓔbv̩˧tɕi˧}\formedesurface{bv̩˧tɕi˧}\newline
\classe{名词}\ton{M}
\paradigme{\pcmn{:} \p{}}
\begin{définition}\peng{Wild peach.}\end{définition}
\begin{définition}\pcmn{毛桃}\end{définition}
\begin{définition}\pfra{Pêche sauvage (de petite taille).}\end{définition}
\end{entrée}

\begin{entrée}
{bv̩˧ʈʂɯ˥}{}{ⓔbv̩˧ʈʂɯ˥}\formedesurface{bv̩˧ʈʂɯ˥}\newline
\classe{名词}\ton{H\#}
\paradigme{\pcmn{:} \p{}}
\begin{définition}\peng{Sifter, sieve.RD Comment:Cf. bɤ˧kɯ˧}\end{définition}
\begin{définition}\pcmn{筛子}\end{définition}
\begin{définition}\pfra{Vanneries: tamis où l'on fait sécher les graines de courge et autres produits de la ferme.}\end{définition}
\end{entrée}

\begin{entrée}
{bv̩˧ʈʂʰv̩˧}{}{ⓔbv̩˧ʈʂʰv̩˧}\formedesurface{bv̩˧ʈʂʰv̩˧}\newline
\classe{名词}\ton{M}
\paradigme{\pcmn{:} \p{}}
\begin{définition}\peng{Cymbals.}\end{définition}
\begin{définition}\pcmn{钹}\end{définition}
\begin{définition}\pfra{Cymbales.}\end{définition}
\end{entrée}

\begin{entrée}
{bv̩˧zo\#˥}{}{ⓔbv̩˧zo\#˥}\formedesurface{bv̩˧zo˧}\newline
\classe{名词}\ton{\#H}
\paradigme{\pcmn{:} \p{}}
\begin{définition}\peng{yak calf (baby yak).}\end{définition}
\begin{définition}\pcmn{小牦牛}\end{définition}
\begin{définition}\pfra{Petit du yak.}\end{définition}
\begin{exemple}\pnru{bv̩˧zo˧ tʰv̩˧-mi˧˥ / bv̩˧zo˧ tʰv̩˧-mi˥\#}\hspace{5pt}\peng{|fg{n}+|fg{dem}+|fg{clf}}\hspace{5pt}\pcmn{这头小牦牛}\hspace{5pt}\pfra{|fg{n}+|fg{dem}+|fg{clf}}\end{exemple}
\end{entrée}

\begin{entrée}
{bv̩˩zo˩}{}{ⓔbv̩˩zo˩}\formedesurface{bv̩˩zo˩˥}\newline
\classe{名词}\ton{L}
\paradigme{\pcmn{:} \p{}}
\begin{définition}\peng{Small food steamer.}\end{définition}
\begin{définition}\pcmn{小蒸笼}\end{définition}
\begin{définition}\pfra{Petite étuve.}\end{définition}
\end{entrée}

\begin{entrée}
{bv̩˧ʐv̩˧}{}{ⓔbv̩˧ʐv̩˧}\formedesurface{bv̩˧ʐv̩˧}\newline
\classe{名词}\ton{M}
\paradigme{\pcmn{:} \p{}}
\begin{définition}\peng{Dragon.}\end{définition}
\begin{définition}\pcmn{龙}\end{définition}
\begin{définition}\pfra{Dragon.}\end{définition}
\end{entrée}

\begin{entrée}
{bv̩˩ʐv̩˩-dzi˩}{}{ⓔbv̩˩ʐv̩˩-dzi˩}\formedesurface{bv̩˩ʐv̩˩dzi˩˥}\newline
\classe{名词}\ton{L}
\paradigme{\pcmn{:} \p{}}
\begin{définition}\peng{Ivy.}\end{définition}
\begin{définition}\pcmn{常春藤}\end{définition}
\begin{définition}\pfra{Lierre.}\end{définition}
\end{entrée}

\begin{entrée}
{bv̩˧ʐv̩˧-kʰv̩˧˥}{₁}{ⓔbv̩˧ʐv̩˧-kʰv̩˧˥ⓗ1}\formedesurface{bv̩˧ʐv̩˧kʰv̩˧˥}\newline
\classe{名词}\ton{MH\#}
1\begin{définition}\peng{Year of the Snake.}\end{définition}
\begin{définition}\pcmn{蛇年}\end{définition}
\begin{définition}\pfra{Année du Serpent.}\end{définition}
\end{entrée}

\begin{entrée}
{bv̩˧ʐv̩˧-kʰv̩˧˥}{₂}{ⓔbv̩˧ʐv̩˧-kʰv̩˧˥ⓗ2}\formedesurface{bv̩˧ʐv̩˧kʰv̩˧˥}\newline
\classe{形容词}\ton{MH\#}
2\begin{définition}\peng{Born in the year of the Snake.}\end{définition}
\begin{définition}\pcmn{属蛇}\end{définition}
\begin{définition}\pfra{Né l'année du Serpent.}\end{définition}
\end{entrée}

\newpage\caractère{ɕ}

\begin{entrée}
{ɕi˥α}{}{ⓔɕi˥α}\formedesurface{ɖɯ˧ ɕi˥}\newline
\classe{量词}\ton{Hα}\begin{définition}\peng{100.}\end{définition}
\begin{définition}\pcmn{百}\end{définition}
\begin{définition}\pfra{100.}\end{définition}
\begin{exemple}\pnru{ɖɯ˧-ɕi˥}\hspace{5pt}\peng{one hundred}\hspace{5pt}\pcmn{一百}\hspace{5pt}\pfra{cent}\end{exemple}
\begin{exemple}\pnru{ɖɯ˧-ɕi˧ kʰv̩˧˥}\hspace{5pt}\peng{one hundred years}\hspace{5pt}\pcmn{一百年,一百岁}\hspace{5pt}\pfra{cent ans, un siècle}\end{exemple}
\begin{exemple}\pnru{ɖɯ˧-ɕi˧ kʰv̩˧∼ɖɯ˥-ɕi˩ kʰv̩˩}\hspace{5pt}\peng{century after century}\hspace{5pt}\pcmn{一百年又一百年}\hspace{5pt}\pfra{siècle après siècle}\end{exemple}
\begin{exemple}\pnru{ɕi˧-kʰv̩˧˥}\hspace{5pt}\peng{a century, one hundred years (abridged formulation)}\hspace{5pt}\pcmn{百年(“一百年”的省略说法)}\hspace{5pt}\pfra{cent ans (formulation abrégée)}\end{exemple}
\end{entrée}

\begin{entrée}
{ɕi˥β}{}{ⓔɕi˥β}\formedesurface{ɖɯ˧ ɕi˥}\newline
\classe{量词}\ton{Hβ}\begin{définition}\peng{One hundredth of a yuan, one penny.}\end{définition}
\begin{définition}\pcmn{量词:分(一分钱)}\end{définition}
\begin{définition}\pfra{Centième d'unité monétaire.}\end{définition}
\end{entrée}

\begin{entrée}
{ɕi˧}{}{ⓔɕi˧}\formedesurface{ɕi˧}\newline
\classe{名词}\ton{M}\begin{définition}\peng{Rice (monosyllable).}\end{définition}
\begin{définition}\pcmn{米(单音节)}\end{définition}
\begin{définition}\pfra{Riz (monosyllabe).}\end{définition}
\end{entrée}

\begin{entrée}
{ɕi˩}{}{ⓔɕi˩}\formedesurface{ɕi˩˥}\newline
\classe{动词}\ton{La}\begin{définition}\peng{To be afraid of.}\end{définition}
\begin{définition}\pcmn{怕、害怕}\end{définition}
\begin{définition}\pfra{Craindre, avoir peur de. Verbe qui paraît suranné; il ne se trouve que dans quelques expressions.}\end{définition}
\begin{exemple}\pnru{njɤ˧ | no˩ ɕi˩ tʰɑ˥-mɤ˩-ʝi˩! | njɤ˧ | no˩ ɖwæ˩ tʰɑ˥-mɤ˩-ʝi˩!}\hspace{5pt}\peng{Don't you fancy I am afraid of you! / Don't you imagine you frighten me!}\hspace{5pt}\pcmn{我不会屈服于你!/ 我不会害怕你!(回敬的话)}\hspace{5pt}\pfra{Ne va pas croire que tu me fasses peur! / Si tu crois que j'ai peur de toi! Si tu crois que je te crains/que tu me fais peur!}\end{exemple}
\begin{exemple}\pnru{njɤ˧ | no˩ ɕi˩-mɤ˩-ʝi˥!}\hspace{5pt}\peng{You don't frighten me! / I'm not afraid of you!}\hspace{5pt}\pcmn{我不会害怕你!}\hspace{5pt}\pfra{Tu ne me fais pas peur!}\end{exemple}
\begin{exemple}\pnru{njɤ˧ | tʰv̩˧ ɕi˩-mɤ˩-ʝi˩!}\hspace{5pt}\peng{I'm not afraid of him!}\hspace{5pt}\pcmn{我不会害怕他!}\hspace{5pt}\pfra{Il ne me fait pas peur!}\end{exemple}
\begin{exemple}\pnru{njɤ˧ | ʈʂʰɯ˧-v̩˧ do˧˥, | ʁo˧ɕi˧˥ | ʐwæ˩˥!}\hspace{5pt}\peng{When I see him, I have a feeling of reverence! (Typically: seeing someone in the family who has authority, such as one's uncle.)}\hspace{5pt}\pcmn{我一看见他,有敬畏之感!}\hspace{5pt}\pfra{Quand je le vois, il m'inspire respect et révérence! (Un exemple typique: sentiment qu'inspire l'oncle, personne d'autorité.)}\end{exemple}
\end{entrée}

\begin{entrée}
{ɕi˩˥}{}{ⓔɕi˩˥}\formedesurface{ɕi˩˥}\newline
\classe{名词}\ton{LH}\begin{définition}\peng{Incense (second syllable).}\end{définition}
\begin{définition}\pcmn{香(单音节)}\end{définition}
\begin{définition}\pfra{Encens (monosyllabe).}\end{définition}
\begin{exemple}\pnru{ɕi˩ qæ˧˥}\hspace{5pt}\peng{to burn incense}\hspace{5pt}\pcmn{烧香}\hspace{5pt}\pfra{brûler de l'encens}\end{exemple}
\end{entrée}

\begin{entrée}
{ɕi˧ɕi˩-lo˩}{}{ⓔɕi˧ɕi˩-lo˩}\formedesurface{ɕi˧ɕi˩lo˩}\newline
\classe{名词}\ton{L\#-}\begin{définition}\peng{The smallest cutlets.}\end{définition}
\begin{définition}\pcmn{最细的肋骨}\end{définition}
\begin{définition}\pfra{Les plus petites des côtelettes.}\end{définition}
\end{entrée}

\begin{entrée}
{ɕi˩dv̩˥}{}{ⓔɕi˩dv̩˥}\formedesurface{ɕi˩dv̩˥}\newline
\classe{名词}\ton{LH}\begin{définition}\peng{Incense.}\end{définition}
\begin{définition}\pcmn{香,烧香火用的}\end{définition}
\begin{définition}\pfra{Encens; bâtonnet d'encens.}\end{définition}
\begin{exemple}\pnru{ɕi˩dv̩˥ qæ˩}\hspace{5pt}\peng{to burn incense}\hspace{5pt}\pcmn{烧香}\hspace{5pt}\pfra{brûler de l'encens}\end{exemple}
\end{entrée}

\begin{entrée}
{ɕi˩dzi˥}{}{ⓔɕi˩dzi˥}\formedesurface{ɕi˩dzi˥}\newline
\classe{名词}\ton{LH}
\paradigme{\pcmn{:} \p{}}
\begin{définition}\peng{Cypress.}\end{définition}
\begin{définition}\pcmn{柏树}\end{définition}
\begin{définition}\pfra{Genévrier; arbre dont des branchages sont employés lors des rituels (suivant la tradition tibétaine).}\end{définition}
\end{entrée}

\begin{entrée}
{ɕi˧-ho˩ʂɯ˩}{}{ⓔɕi˧-ho˩ʂɯ˩}\formedesurface{ɕi˧ho˩ʂɯ˩}\newline
\classe{名词}\ton{-L}\begin{définition}\peng{Tomato.}\end{définition}
\begin{définition}\pcmn{西红柿(汉语借词)}\end{définition}
\begin{définition}\pfra{Tomate.}\end{définition}
\end{entrée}

\begin{entrée}
{ɕi˧lv̩˧}{}{ⓔɕi˧lv̩˧}\formedesurface{ɕi˧lv̩˧}\newline
\classe{名词}\ton{M}
\paradigme{\pcmn{:} \p{}}
\begin{définition}\peng{Paddy field.}\end{définition}
\begin{définition}\pcmn{水稻田}\end{définition}
\begin{définition}\pfra{Champs de riz.}\end{définition}
\end{entrée}

\begin{entrée}
{ɕi˧lv̩˧-mv̩˧di˧˥}{}{ⓔɕi˧lv̩˧-mv̩˧di˧˥}\formedesurface{ɕi˧lv̩˧mv̩˧di˧˥}\newline
\classe{名词}\ton{-MH\#}
\paradigme{\pcmn{:} \p{}}
\begin{définition}\peng{Paddy field.RD Comment:Cf. lv̩˧pʰv̩˩, ɕi˧ɭɯ˧-lv̩˧pʰv̩˩}\end{définition}
\begin{définition}\pcmn{水田}\end{définition}
\begin{définition}\pfra{Champs de riz.}\end{définition}
\end{entrée}

\begin{entrée}
{ɕi˧ɭɯ˧}{}{ⓔɕi˧ɭɯ˧}\formedesurface{ɕi˧ɭɯ˧}\newline
\classe{名词}\ton{M}
\paradigme{\pcmn{:} \p{}}\paradigme{\pcmn{:} \p{}}
\begin{définition}\peng{Paddy rice; by extension: paddy field.}\end{définition}
\begin{définition}\pcmn{稻子}\end{définition}
\begin{définition}\pfra{Riz paddy; par extension: champ de riz.}\end{définition}
\end{entrée}

\begin{entrée}
{ɕi˧ɭɯ˧-lv̩˧pʰv̩˩}{}{ⓔɕi˧ɭɯ˧-lv̩˧pʰv̩˩}\formedesurface{ɕi˧ɭɯ˧lv̩˧pʰv̩˩}\newline
\classe{名词}\ton{-L\#}
\paradigme{\pcmn{:} \p{}}
\begin{définition}\peng{Paddy field.RD Comment:Cf. ɕi˧lv̩˧-mv̩˧di˧˥}\end{définition}
\begin{définition}\pcmn{水稻田}\end{définition}
\begin{définition}\pfra{Champs de riz.}\end{définition}
\end{entrée}

\begin{entrée}
{ɕi˧tv̩˧-di˩-lv̩˧}{}{ⓔɕi˧tv̩˧-di˩-lv̩˧}\formedesurface{ɕi˧tv̩˧di˩lv̩˧}\newline
\classe{名词}\ton{-L-}
\paradigme{\pcmn{:} \p{}}
\begin{définition}\peng{Paddy field.}\end{définition}
\begin{définition}\pcmn{种水稻的田}\end{définition}
\begin{définition}\pfra{Champs de riz.}\end{définition}
\end{entrée}

\begin{entrée}
{ɕi˧tɕʰi\#˥}{}{ⓔɕi˧tɕʰi\#˥}\formedesurface{ɕi˧tɕʰi˧}\newline
\classe{名词}\ton{\#H}
\paradigme{\pcmn{:} \p{}}
\begin{définition}\peng{Chaff; bran; husk (of rice).}\end{définition}
\begin{définition}\pcmn{米糠}\end{définition}
\begin{définition}\pfra{Son de riz.}\end{définition}
\end{entrée}

\begin{entrée}
{ɕi˩ʈʰæ˧˥}{₁}{ⓔɕi˩ʈʰæ˧˥ⓗ1}\formedesurface{ɕi˩ʈʰæ˧˥}\newline
\classe{形容词}\ton{LM+MH\#}
1\begin{définition}\peng{To be a stammerer; to have a stammer.}\end{définition}
\begin{définition}\pcmn{结巴}\end{définition}
\begin{définition}\pfra{Bègue, qui a un bégaiement.}\end{définition}
\begin{exemple}\pnru{ʈʂʰɯ˧ | ɖɯ˧-pi˧˥ | ɕi˩ʈʰæ˧˥}\hspace{5pt}\peng{(S)he has a stammer.}\hspace{5pt}\pcmn{他有一点结巴。}\hspace{5pt}\pfra{Il/elle est un peu bègue.}\end{exemple}
\end{entrée}

\begin{entrée}
{ɕi˩ʈʰæ˧˥}{₂}{ⓔɕi˩ʈʰæ˧˥ⓗ2}\formedesurface{ɕi˩ʈʰæ˧˥}\newline
\classe{名词}\ton{LM+MH\#}
2\begin{définition}\peng{Stammerer, stutterer.}\end{définition}
\begin{définition}\pcmn{结巴}\end{définition}
\begin{définition}\pfra{Bègue.}\end{définition}
\begin{exemple}\pnru{ʈʂʰɯ˧ | ɕi˩ʈʰæ˧ ɲi˥.}\hspace{5pt}\peng{(S)he is a stammerer.}\hspace{5pt}\pcmn{他是结巴的。}\hspace{5pt}\pfra{Il/elle est bègue.}\end{exemple}
\begin{exemple}\pnru{ʈʂʰɯ˧ | ɕi˩ʈʰæ˧-zo˥.}\hspace{5pt}\peng{(S)he is a stammerer.}\hspace{5pt}\pcmn{他是结巴。}\hspace{5pt}\pfra{Il/elle est bègue.}\end{exemple}
\end{entrée}

\begin{entrée}
{ɕi˧ʈʂʰwæ˧}{}{ⓔɕi˧ʈʂʰwæ˧}\formedesurface{ɕi˧ʈʂʰwæ˧}\newline
\classe{名词}\ton{M}\begin{définition}\peng{Husked rice.}\end{définition}
\begin{définition}\pcmn{大米}\end{définition}
\begin{définition}\pfra{Riz décortiqué.}\end{définition}
\begin{exemple}\pnru{ɕi˧ʈʂʰwæ˧-hɑ˧}\hspace{5pt}\peng{cooked rice; literally “cooked-rice food", specifying the term /hɑ˥/, which refers to food in general.}\hspace{5pt}\pcmn{米饭}\hspace{5pt}\pfra{riz cuit; littéralement: «nourriture-riz cuit»; formulation employée pour préciser le terme /hɑ˥/, qui désigne toutes les nourritures.}\end{exemple}
\end{entrée}

\begin{entrée}
{ɕjɤ˥}{}{ⓔɕjɤ˥}\formedesurface{ɕjɤ˧}\newline
\classe{动词}\ton{H}\begin{définition}\peng{To invent, to think out/up, to come up with (an idea, a solution).}\end{définition}
\begin{définition}\pcmn{想、发明、想出、找到(办法)(汉语借词)}\end{définition}
\begin{définition}\pfra{Inventer, trouver.}\end{définition}
\begin{exemple}\pnru{le˧-ɕjɤ˥}\hspace{5pt}\peng{|fg{accomp}}\hspace{5pt}\pcmn{想了}\hspace{5pt}\pfra{|fg{accomp}}\end{exemple}
\begin{exemple}\pnru{ʈʂʰɯ˧ | pæ˧˥hwɤ˧ | ɕjɤ˧ ɣɯ˧!}\hspace{5pt}\peng{(S)he knows to find solutions under all circumstances! / (S)he is good of finding solutions to all problems!}\hspace{5pt}\pcmn{他很会想办法的!}\hspace{5pt}\pfra{Il/elle a une solution à tout/ sait trouver une solution en toutes circonstances!}\end{exemple}
\end{entrée}

\begin{entrée}
{ɕjɤ˧˥}{}{ⓔɕjɤ˧˥}\formedesurface{ɕjɤ˧˥}\newline
\classe{动词}\ton{MH}\begin{définition}\peng{To try; to taste.}\end{définition}
\begin{définition}\pcmn{尝试、体会、经过}\end{définition}
\begin{définition}\pfra{Essayer, goûter, expérimenter.}\end{définition}
\begin{exemple}\pnru{le˧-ɕjɤ˧-ze˥}\hspace{5pt}\peng{|fg{accomp} \_ |fg{pfv}}\hspace{5pt}\pcmn{试了}\hspace{5pt}\pfra{|fg{accomp} \_ |fg{pfv}}\end{exemple}
\begin{exemple}\pnru{tso˧∼tso˧ ɕjɤ˩}\hspace{5pt}\peng{to try something}\hspace{5pt}\pcmn{试用一个东西}\hspace{5pt}\pfra{essayer quelque chose}\end{exemple}
\begin{exemple}\pnru{no˧ | ɖɯ˧-kʰwɤ˥ ɕjɤ˩!}\hspace{5pt}\peng{Have a taste! / Taste a bite!}\hspace{5pt}\pcmn{你尝一口吧!}\hspace{5pt}\pfra{Goûte un peu! goûte un morceau!}\end{exemple}
\begin{exemple}\pnru{ɖɯ˧-ɕjɤ˧-ɻ̍˥!}\hspace{5pt}\peng{Have a try!}\hspace{5pt}\pcmn{尝一尝吧! / 试一试吧!}\hspace{5pt}\pfra{Goûte voir! / Essaie voir!}\end{exemple}
\end{entrée}

\begin{entrée}
{ɕjɤ˧-bv̩˧nv̩˧}{}{ⓔɕjɤ˧-bv̩˧nv̩˧}\formedesurface{ɕjɤ˧bv̩˧nv̩˧}\newline
\classe{形容词}\ton{M}
\étymologie{
bv̩˧nv̩˧
}\begin{définition}\peng{Good (smell), fragrant.}\end{définition}
\begin{définition}\pcmn{香(香气、香味)}\end{définition}
\begin{définition}\pfra{Bonne (odeur).}\end{définition}
\begin{exemple}\pnru{ʈʂʰɯ˧ ɕjɤ˧-bv̩˧nv̩˧ ɲi˩.}\hspace{5pt}\peng{It smells good.}\hspace{5pt}\pcmn{这是香的(气味香)。}\hspace{5pt}\pfra{Ca sent bon!}\end{exemple}
\end{entrée}

\begin{entrée}
{ɕjɤ˩∼ɕjɤ˧˥}{}{ⓔɕjɤ˩∼ɕjɤ˧˥}\formedesurface{ɕjɤ˩ɕjɤ˧˥}\newline
\classe{动词}\ton{L}\begin{définition}\peng{To browbeat, to ill-treat.}\end{définition}
\begin{définition}\pcmn{欺负}\end{définition}
\begin{définition}\pfra{Maltraiter.}\end{définition}
\begin{exemple}\pnru{hĩ˧ ɕjɤ˥∼ɕjɤ˩}\hspace{5pt}\peng{to ill-treat someone, to ill-treat people}\hspace{5pt}\pcmn{欺负人}\hspace{5pt}\pfra{maltraiter quelqu'un}\end{exemple}
\begin{exemple}\pnru{no˧ | njɤ˩ ɕjɤ˩∼ɕjɤ˩-mv̩˩-zo˩˥! / no˧ | njɤ˩ ɕjɤ˩∼ɕjɤ˩˥!}\hspace{5pt}\peng{You are treating me badly! / You are bullying me!}\hspace{5pt}\pcmn{你不要欺负我!你欺负我!}\hspace{5pt}\pfra{Vous me maltraitez!}\end{exemple}
\begin{exemple}\pnru{no˧ | njɤ˩ ɕjɤ˩∼ɕjɤ˩-ze˥!}\hspace{5pt}\peng{You have treated me badly! / You have bullied me!}\hspace{5pt}\pcmn{你欺负了我!}\hspace{5pt}\pfra{Vous m'avez maltraité!}\end{exemple}
\end{entrée}

\begin{entrée}
{ɕjɤ˩jo˩}{}{ⓔɕjɤ˩jo˩}\formedesurface{ɕjɤ˩jo˩˥}\newline
\classe{名词}\ton{L}
\paradigme{\pcmn{:} \p{}}
\begin{définition}\peng{|\stylefi{Fritillaria cirrhosa}.}\end{définition}
\begin{définition}\pcmn{贝母}\end{définition}
\begin{définition}\pfra{|\stylefi{Fritillaria cirrhosa}.}\end{définition}
\end{entrée}

\begin{entrée}
{ɕjɤ˩tʰv̩˧˥}{}{ⓔɕjɤ˩tʰv̩˧˥}\formedesurface{ɕjɤ˩tʰv̩˧˥}\newline
\classe{动词}\ton{LM+MH\#}\begin{définition}\peng{To insult; to criticize.}\end{définition}
\begin{définition}\pcmn{骂,批评}\end{définition}
\begin{définition}\pfra{Insulter, maudire, se moquer; réprimander, gronder.}\end{définition}
\begin{exemple}\pnru{hĩ˧ ɕjɤ˥tʰv̩˩}\hspace{5pt}\peng{to insult people; to criticize people}\hspace{5pt}\pcmn{骂人、批评人}\hspace{5pt}\pfra{insulter quelqu'un/ critiquer quelqu'un}\end{exemple}
\begin{exemple}\pnru{ʈʂʰɯ˧ | njɤ˩ ɕjɤ˩tʰv̩˩.}\hspace{5pt}\peng{He insults me.}\hspace{5pt}\pcmn{他骂我。}\hspace{5pt}\pfra{Il m'insulte.}\end{exemple}
\end{entrée}

\begin{entrée}
{ɕjo˩li\#˥}{}{ⓔɕjo˩li\#˥}\formedesurface{ɕjo˩li˥}\newline
\classe{名词}\ton{LM+\#H}
\paradigme{\pcmn{:} \p{}}
\begin{définition}\peng{Flute.}\end{définition}
\begin{définition}\pcmn{笛子}\end{définition}
\begin{définition}\pfra{Flûte (type flûte traversière et non flûte à bec).}\end{définition}
\end{entrée}

\begin{entrée}
{ɕɯ˩α}{}{ⓔɕɯ˩α}\formedesurface{ɕɯ˩˥}\newline
\classe{动词}\ton{Lα}\begin{définition}\peng{To raise.}\end{définition}
\begin{définition}\pcmn{养}\end{définition}
\begin{définition}\pfra{Élever (terme plus relevé que \stylefn{ʐɤ}˧).}\end{définition}
\begin{exemple}\pnru{ɕɯ˩zo\#˥}\hspace{5pt}\peng{adopted child}\hspace{5pt}\pcmn{养儿}\hspace{5pt}\pfra{enfant adopté}\end{exemple}
\begin{exemple}\pnru{ho˧zo˧-ɕɯ˧zo˥, | æ̃˩ mɤ˧-tsɤ˧! | hĩ˧-zo˧mv˥, | ʐɤ˧ tʰɑ˧-mɤ˧-ʝi˧!}\hspace{5pt}\peng{The adopted baby pheasant does not become a chicken (=does not become domesticated)! One should not bring up other people's children! (Proverb which does not apply to the adoption of children who have lost ties with their biological family, but to the adoption of children who remain in touch with their relatives: no matter how much care one puts into bringing them up, they remain more attached to their lineage.)}\hspace{5pt}\pcmn{养的小雉,不会变成鸡!人家的孩子,不要养!(指的不是领养孤儿,而是无端端地养别人的孩子:无论多么关心孩子,最终他还是会无情,会更爱自己原来的家人。)}\hspace{5pt}\pfra{Un bébé faisan qu'on élève chez soi ne devient pas un poulet (n'est pas domestiqué pour autant)! Il ne faut pas élever les enfants d'autrui! (Proverbe qui ne s'applique pas à l'adoption d'enfants qui ont perdu leurs attaches à leur famille biologique, mais à l'adoption d'enfants qui restent en contact avec leurs proches: quelque soin que l'on consacre à leur éducation, ils restent plus attachés à leur famille d'origine.)}\end{exemple}
\end{entrée}

\newpage\caractère{d}

\begin{entrée}
{dɑ˧˥}{}{ⓔdɑ˧˥}\formedesurface{dɑ˧˥}\newline
\classe{名词}\ton{MH}\begin{définition}\peng{Misfortune, mishaps.}\end{définition}
\begin{définition}\pcmn{苦、苦难、悲戚}\end{définition}
\begin{définition}\pfra{Infortune, malheur.}\end{définition}
\begin{exemple}\pnru{dɑ˧-ʐwɤ˧˥}\hspace{5pt}\peng{to complain, to tell one's misfortunes}\hspace{5pt}\pcmn{诉苦}\hspace{5pt}\pfra{se plaindre de son infortune, gémir sur son sort}\end{exemple}
\begin{exemple}\pnru{ɻ̃˧-ʐwɤ˧ | dɑ˧-ʐwɤ˧-ɻ̍˥}\hspace{5pt}\peng{to bemoan one's misfortunes}\hspace{5pt}\pcmn{讲自己苦难的事情}\hspace{5pt}\pfra{raconter ses malheurs; se plaindre}\end{exemple}
\begin{exemple}\pnru{ʈʂʰɯ˧ | mɤ˧-dɑ˩-qʰwɤ˩, | ɻ̃˧-ʐwɤ˧ | dɑ˧-ʐwɤ˧-ɻ̍˥!}\hspace{5pt}\peng{(S)he is unhappy; (s)he is constantly complaining!}\hspace{5pt}\pcmn{他“玛达夸”,一直在诉说苦难的事情!}\hspace{5pt}\pfra{Il est malheureux; il passe son temps à se plaindre!}\end{exemple}
\end{entrée}

\begin{entrée}
{dɑ˧˥}{₁}{ⓔdɑ˧˥ⓗ1}\formedesurface{dɑ˧˥}\newline
\classe{动词}\ton{MH}
1\begin{définition}\peng{To build (a house…).}\end{définition}
\begin{définition}\pcmn{建(房子)}\end{définition}
\begin{définition}\pfra{Construire (une maison…).}\end{définition}
\begin{exemple}\pnru{ʑi˧mi˧ dɑ˧˥}\hspace{5pt}\peng{to build a house}\hspace{5pt}\pcmn{修建‘依咪’、建房}\hspace{5pt}\pfra{construire une maison}\end{exemple}
\end{entrée}

\begin{entrée}
{dɑ˧˥}{₂}{ⓔdɑ˧˥ⓗ2}\formedesurface{dɑ˧˥}\newline
\classe{动词}\ton{MH}
2\begin{définition}\peng{To fell (a tree); to cut into pieces (a large piece of meat); to create a breach (in a dike).}\end{définition}
\begin{définition}\pcmn{砍(树),砍(肉)}\end{définition}
\begin{définition}\pfra{Couper un arbre, abattre un arbre; ouvrir une brèche (dans une digue).}\end{définition}
\begin{exemple}\pnru{le˧-dɑ˧-ze˥}\hspace{5pt}\peng{|fg{accomp} \_ |fg{pfv}}\hspace{5pt}\pcmn{砍了(树),割了(肉)}\hspace{5pt}\pfra{|fg{accomp} \_ |fg{pfv}}\end{exemple}
\begin{exemple}\pnru{ɖɯ˧-dɑ˧ tʰi˥-dɑ˩}\hspace{5pt}\peng{to hit a blow}\hspace{5pt}\pcmn{砍一下}\hspace{5pt}\pfra{donner un coup}\end{exemple}
\begin{exemple}\pnru{dɑ˩∼dɑ˧˥}\hspace{5pt}\peng{|fg{red}}\hspace{5pt}\pcmn{重叠}\hspace{5pt}\pfra{|fg{red}}\end{exemple}
\begin{exemple}\pnru{le˧-dɑ˩∼dɑ˩(-ze˩)}\hspace{5pt}\peng{(I) have cut (e.g. a chicken) into pieces}\hspace{5pt}\pcmn{(我把一只鸡)砍成块了。}\hspace{5pt}\pfra{(j'ai) découpé (ex.: le poulet) en morceaux}\end{exemple}
\begin{exemple}\pnru{ʂe˧ dɑ˥∼dɑ˩}\hspace{5pt}\peng{to cut meat to small pieces}\hspace{5pt}\pcmn{把肉砍成小块}\hspace{5pt}\pfra{hacher de la viande, couper de la viande en morceaux}\end{exemple}
\begin{exemple}\pnru{tso˧∼tso˧ dɑ˩}\hspace{5pt}\peng{to cut things}\hspace{5pt}\pcmn{砍东西}\hspace{5pt}\pfra{couper des choses}\end{exemple}
\end{entrée}

\begin{entrée}
{dɑ˧˥β}{}{ⓔdɑ˧˥β}\formedesurface{ɖɯ˧ dɑ˧˥}\newline
\classe{量词}\ton{MHβ}\begin{définition}\peng{Self-classifier for blows.}\end{définition}
\begin{définition}\pcmn{量词:下(打一下)}\end{définition}
\begin{définition}\pfra{Auto-classificateur des coups.}\end{définition}
\begin{exemple}\pnru{ɖɯ˧-dɑ˧˥}\hspace{5pt}\peng{a blow}\hspace{5pt}\pcmn{(砍)一刀}\hspace{5pt}\pfra{un coup}\end{exemple}
\begin{exemple}\pnru{ɖɯ˧-dɑ˧ tʰi˥-dɑ˩}\hspace{5pt}\peng{to strike a blow, to give a blow}\hspace{5pt}\pcmn{砍一刀}\hspace{5pt}\pfra{donner un coup}\end{exemple}
\end{entrée}

\begin{entrée}
{dɑ˩}{}{ⓔdɑ˩}\formedesurface{dɑ˩˥}\newline
\classe{形容词}\ton{L}\begin{définition}\peng{Happy.}\end{définition}
\begin{définition}\pcmn{幸福、平安、安好}\end{définition}
\begin{définition}\pfra{Heureux.}\end{définition}
\begin{exemple}\pnru{mɤ˧-dɑ˩-qʰwɤ˩}\hspace{5pt}\peng{Melancholy expression, telling of one's unhappiness, lamenting one's hardships.}\hspace{5pt}\pcmn{“玛达夸”:叹息,有幸福、高兴、安好,有悲情、悲伤、哀痛,有珍惜、可惜、怜悯。}\hspace{5pt}\pfra{Expression mélancolique reflétant mélancolie, tristesse mêlées de joie, d'emportement et de regrets.}\end{exemple}
\begin{exemple}\pnru{mɤ˧-dɑ˩!}\hspace{5pt}\peng{As above. Introductory formula for melancholy songs, and sometimes at the beginning of stories. (The same formula is also used in the Laze language.)}\hspace{5pt}\pcmn{“玛达”。同上。}\hspace{5pt}\pfra{Idem ci-dessus. La formule est notamment employée en début de chanson mélancolique, et parfois au début d'un conte. (La même formule est en usage dans la langue lazé.)}\end{exemple}
\begin{exemple}\pnru{mɤ˧-dɑ˩-mi˩}\hspace{5pt}\peng{As above: same meaning as the form without a /-mi/ suffix.}\hspace{5pt}\pcmn{“玛达咪”。同上。}\hspace{5pt}\pfra{Comme ci-dessus: même sens que la forme sans suffixe /-mi/.}\end{exemple}
\begin{exemple}\pnru{ɖwæ˧˥ | hɤ˩-dɑ˥! | ɖwæ˧˥ | hɤ˩˥!}\hspace{5pt}\peng{Good job! / Well done! (Context: compliment to a toddler who has managed to do something impressive for her age.)}\hspace{5pt}\pcmn{很了不起啊!了不起!(情景:表扬一个小孩子成功地完成一件事情。)}\hspace{5pt}\pfra{Bravo, bravo! (Contexte: compliment saluant l'exploit d'une petite fille qui a réalisé une tâche pas évidente à son âge.)}\end{exemple}
\begin{exemple}\pnru{ɖʐɯ˩dɑ˥-kʰɤ˩dɑ˩-ɻ̍˩}\hspace{5pt}\peng{All is well. / All is for the best. (Used for instance to describe a period without food shortage, earthquake, epidemic, war or other catastrophe)}\hspace{5pt}\pcmn{一切都安好。(如:一段时间没有饥荒、地震、流行病、战争等灾难)}\hspace{5pt}\pfra{Tout va bien, tout est pour le mieux. (S'emploie par exemple pour décrire une période sans disette, ni tremblement de terre, ni épidémie, ni guerre ou autre catastrophe.)}\end{exemple}
\end{entrée}

\begin{entrée}
{dɑ˩β}{}{ⓔdɑ˩β}\formedesurface{dɑ˩˥}\newline
\classe{动词}\ton{Lβ}\begin{définition}\peng{To weave.}\end{définition}
\begin{définition}\pcmn{织}\end{définition}
\begin{définition}\pfra{Tisser.}\end{définition}
\begin{exemple}\pnru{ɣɯ˧ dɑ˩}\hspace{5pt}\peng{to weave fabric}\hspace{5pt}\pcmn{织布}\hspace{5pt}\pfra{tisser du tissu}\end{exemple}
\begin{exemple}\pnru{ɣɯ˧ | le˧-dɑ˩}\hspace{5pt}\peng{to weave fabric}\hspace{5pt}\pcmn{织布}\hspace{5pt}\pfra{tisser du tissu}\end{exemple}
\begin{exemple}\pnru{ɖɯ˧-dɑ˧∼dɑ˩-ɻ̍˩}\hspace{5pt}\peng{|fg{delimitative} \_ |fg{red} |fg{inceptive}}\hspace{5pt}\pcmn{织一下}\hspace{5pt}\pfra{|fg{délimitatif} \_ |fg{red} |fg{inchoatif}}\end{exemple}
\end{entrée}

\begin{entrée}
{dɑ˧ʝi˩}{}{ⓔdɑ˧ʝi˩}\formedesurface{dɑ˧ʝi˩}\newline
\classe{名词}\ton{L\#}
\paradigme{\pcmn{:} \p{}}
\begin{définition}\peng{Mule.}\end{définition}
\begin{définition}\pcmn{骡子}\end{définition}
\begin{définition}\pfra{Mule.}\end{définition}
\begin{exemple}\pnru{dɑ˧ʝi˩-dʑo˩, | ɖɯ˩mi˧ dʑo˧-kv̩˥-mæ˩! | ɖɯ˩zo˧ dʑo˧-kv̩˥-mæ˩!}\hspace{5pt}\peng{As for mules, there exist female mules, (and) male mules! / Among mules, there is a distinction between females and males! (Explanation provided to a city dweller on a visit, who knew precious little about animal breeding.)}\hspace{5pt}\pcmn{骡子呢,有母骡子!(也)有公骡子! / 骡子,分母的和公的!(这个说明是给一个不懂畜牧业的城里人听)}\hspace{5pt}\pfra{Les mules, ça se répartit en mules mâles et mules femelles! / Il existe une distinction de sexe parmi les mules! (Explication fournie à un visiteur citadin peu au fait de l'élevage des animaux.)}\end{exemple}
\end{entrée}

\begin{entrée}
{dɑ˩kʰɤ˩}{}{ⓔdɑ˩kʰɤ˩}\formedesurface{dɑ˩kʰɤ˩˥}\newline
\classe{名词}\ton{L}
\paradigme{\pcmn{:} \p{}}
\begin{définition}\peng{Drum.}\end{définition}
\begin{définition}\pcmn{鼓}\end{définition}
\begin{définition}\pfra{Tambour.}\end{définition}
\begin{exemple}\pnru{dɑ˩kʰɤ˩ lɑ˥(-ze˩)}\hspace{5pt}\peng{to play a drum}\hspace{5pt}\pcmn{打鼓(了)}\hspace{5pt}\pfra{jouer du tambour}\end{exemple}
\end{entrée}

\begin{entrée}
{dɑ˧pɤ˧}{}{ⓔdɑ˧pɤ˧}\formedesurface{dɑ˧pɤ˧}\newline
\classe{名词}\ton{M}
\paradigme{\pcmn{:} \p{}}
\begin{définition}\peng{Priest of the local religion.}\end{définition}
\begin{définition}\pcmn{宗教礼师。音译:达巴}\end{définition}
\begin{définition}\pfra{Prêtre de la religion locale.}\end{définition}
\begin{exemple}\pnru{dɑ˧pɤ˧ ʝi˧-hĩ˧ hĩ˧}\hspace{5pt}\peng{priest, person who performs the function of priest}\hspace{5pt}\pcmn{当达巴的人}\hspace{5pt}\pfra{prêtre, personne qui joue le rôle de prêtre/qui est prêtre}\end{exemple}
\end{entrée}

\begin{entrée}
{dɑ˧pv̩\#˥}{}{ⓔdɑ˧pv̩\#˥}\formedesurface{dɑ˧pv̩˧}\newline
\classe{名词}\ton{\#H}
\paradigme{\pcmn{:} \p{}}
\begin{définition}\peng{Host.}\end{définition}
\begin{définition}\pcmn{主人}\end{définition}
\begin{définition}\pfra{Maître de maison, hôte (personne qui accueille).}\end{définition}
\begin{exemple}\pnru{ʑi˧dv̩˧ dɑ˧pv̩˧}\hspace{5pt}\peng{the family host, the member of the family who has the role of host}\hspace{5pt}\pcmn{家的主人(依杜的主人。“依杜”指摩梭的人户。)}\hspace{5pt}\pfra{l'hôte de la maison}\end{exemple}
\begin{exemple}\pnru{ʑi˧dv̩˧-ʝi˧-hĩ˧ dɑ˧pv̩˧}\hspace{5pt}\peng{ditto}\hspace{5pt}\pcmn{同上}\hspace{5pt}\pfra{idem}\end{exemple}
\end{entrée}

\begin{entrée}
{dɑ˧pʰo˥}{}{ⓔdɑ˧pʰo˥}\formedesurface{dɑ˧pʰo˥}\newline
\classe{名词}\ton{LH}\begin{définition}\peng{Dapo.}\end{définition}
\begin{définition}\pcmn{达坡(永宁坝子的一个村落)}\end{définition}
\begin{définition}\pfra{Dapo (nom de village).}\end{définition}
\begin{exemple}\pnru{ɖæ˩ʂɯ\#˥, | ʈʂo˧ʂɯ\#˥, | bɤ˩tɕʰɯ˩˥, | dɑ˧pʰo˥, | bɤ˧dzi˩, | dze˧bo˧}\hspace{5pt}\peng{Six villages of the plain of Yongning that lie relatively close to the Lake.}\hspace{5pt}\pcmn{永宁摩梭地理概念中,距离泸沽湖比较近的六个村落:扎实、忠实、八旗、达坡、八珠、者波。}\hspace{5pt}\pfra{Six villages de la plaine de Yongning qui sont relativement proches du Lac.}\end{exemple}
\end{entrée}

\begin{entrée}
{dɑ˧ʁwɤ\#˥}{}{ⓔdɑ˧ʁwɤ\#˥}\formedesurface{dɑ˧ʁwɤ˧}\newline
\classe{名词}\ton{\#H}\begin{définition}\peng{A village downstream from Qiansuo; the language spoken there is reported to be close to that of the Yongning plain.}\end{définition}
\begin{définition}\pcmn{达瓦村:四川凉山州木里县、盐源县、丽江市宁蒗县交界的一个村落,在前所的下游。}\end{définition}
\begin{définition}\pfra{Un village en aval de Qiansuo; la langue parlée là-bas serait relativement proche de celle de la plaine de Yongning.}\end{définition}
\end{entrée}

\begin{entrée}
{dɑ˩to˩}{}{ⓔdɑ˩to˩}\formedesurface{dɑ˩to˩˥}\newline
\classe{助词}\ton{L}\begin{définition}\peng{At bottom, in reality, when all is said and done.}\end{définition}
\begin{définition}\pcmn{说到底,根本上,归根结底}\end{définition}
\begin{définition}\pfra{Au fond, en réalité, en définitive.}\end{définition}
\end{entrée}

\begin{entrée}
{dɑ˩to\#˥}{}{ⓔdɑ˩to\#˥}\formedesurface{dɑ˩to˥}\newline
\classe{助词}\ton{LM+\#H}\begin{définition}\peng{Politely. This term was only observed in association with the verb ‘to say', with the meaning ‘to say polite words, polite small-talk'.}\end{définition}
\begin{définition}\pcmn{客气地}\end{définition}
\begin{définition}\pfra{Poliment.}\end{définition}
\begin{exemple}\pnru{dɑ˩to˧ ʐwɤ˧˥}\hspace{5pt}\peng{to say some polite things}\hspace{5pt}\pcmn{说客气话}\hspace{5pt}\pfra{faire des politesses}\end{exemple}
\begin{exemple}\pnru{dɑ˩to˧ ʐwɤ˧-hĩ˥-lɑ˩ ɲi˩!}\hspace{5pt}\peng{It's just polite words! (Comment about an invitation by a neighbour, which was intended to be declined: it was not a true invitation.)}\hspace{5pt}\pcmn{只是说客气话而已!/这只是客气话而已!}\hspace{5pt}\pfra{C'est juste pour être poli! / C'est juste une façon de dire! (Commentaire de quelqu'un au sujet d'une invitation lancée par un voisin, qui est une simple politesse et pas une vraie invitation; il convient de la décliner.)}\end{exemple}
\end{entrée}

\begin{entrée}
{dɤ˩-qo˧}{}{ⓔdɤ˩-qo˧}\formedesurface{dɤ˩qo˧}\newline
\classe{助词}\ton{LM}\begin{définition}\peng{Way over there.}\end{définition}
\begin{définition}\pcmn{那里(远指)}\end{définition}
\begin{définition}\pfra{Par là-bas tout au loin, tout au loin là-bas.}\end{définition}
\end{entrée}

\begin{entrée}
{dɤ˩-tʰv̩˧-gi\#˥}{}{ⓔdɤ˩-tʰv̩˧-gi\#˥}\formedesurface{dɤ˩tʰv̩˧gi˧}\newline
\classe{助词}\ton{L-\#H}\begin{définition}\peng{Way over there.}\end{définition}
\begin{définition}\pcmn{那边(远指)}\end{définition}
\begin{définition}\pfra{Au loin, de ce côté-là.}\end{définition}
\end{entrée}

\begin{entrée}
{dɤ˩-tʰv̩˧qo˧}{}{ⓔdɤ˩-tʰv̩˧qo˧}\formedesurface{dɤ˩tʰv̩˧qo˧}\newline
\classe{助词}\ton{L-M}\begin{définition}\peng{Way over there.}\end{définition}
\begin{définition}\pcmn{那边(远指)}\end{définition}
\begin{définition}\pfra{Par là-bas tout au loin.}\end{définition}
\end{entrée}

\begin{entrée}
{dɤ˩-ʈʂʰɯ˧qo˧}{}{ⓔdɤ˩-ʈʂʰɯ˧qo˧}\formedesurface{dɤ˩ʈʂʰɯ˧qo˧}\newline
\classe{助词}\ton{L-M}\begin{définition}\peng{Way over there.}\end{définition}
\begin{définition}\pcmn{那边(远指)}\end{définition}
\begin{définition}\pfra{Par là-bas tout au loin.}\end{définition}
\end{entrée}

\begin{entrée}
{di˧˥}{₁}{ⓔdi˧˥ⓗ1}\formedesurface{di˧˥}\newline
\classe{动词}\ton{MH}
1\begin{définition}\peng{To hunt; to scatter, to drive out, to drive away.}\end{définition}
\begin{définition}\pcmn{打散,驱赶,撵,赶,打猎}\end{définition}
\begin{définition}\pfra{Poursuivre, chasser; disperser, repousser, faire déguerpir.}\end{définition}
\begin{exemple}\pnru{tɕʰɯ˩di˩˥}\hspace{5pt}\peng{to hunt the muntjac; to hunt}\hspace{5pt}\pcmn{赶麂子,狩猎}\hspace{5pt}\pfra{chasser le muntjac; chasser}\end{exemple}
\begin{exemple}\pnru{tɕʰɯ˩di˩-bi˩-ni˩gv̩˩˥}\hspace{5pt}\peng{to have a habit of hunting, to have a fondness for hunting}\hspace{5pt}\pcmn{有打猎的习惯、喜欢打猎}\hspace{5pt}\pfra{avoir l'habitude de chasser}\end{exemple}
\begin{exemple}\pnru{di˩∼di˧˥ / di˩∼di˧-ze˥}\hspace{5pt}\peng{|fg{red}: to hunt, to track}\hspace{5pt}\pcmn{重叠:跟着、追着}\hspace{5pt}\pfra{|fg{red}: suivre à la trace, pister}\end{exemple}
\begin{exemple}\pnru{tʰi˧-di˩∼di˩}\hspace{5pt}\peng{|fg{dur} |fg{red}}\hspace{5pt}\pcmn{|fg{dur} |fg{red}}\hspace{5pt}\pfra{|fg{dur} |fg{red}}\end{exemple}
\end{entrée}

\begin{entrée}
{di˧˥}{₂}{ⓔdi˧˥ⓗ2}\formedesurface{di˧˥}\newline
\classe{动词}\ton{MH}
2\begin{définition}\peng{To run; to have a runny belly = to have diarrhea.}\end{définition}
\begin{définition}\pcmn{拉(肚子)}\end{définition}
\begin{définition}\pfra{S'écouler, couler; avoir la courante = avoir la diarrhée.}\end{définition}
\begin{exemple}\pnru{bi˧mi˧ di˧˥}\hspace{5pt}\peng{to have diarrhea}\hspace{5pt}\pcmn{拉肚子}\hspace{5pt}\pfra{avoir la diarrhée}\end{exemple}
\end{entrée}

\begin{entrée}
{‑di˩}{}{ⓔ‑di˩}\formedesurface{di˩˥}\newline
\classe{后缀}\ton{L}\begin{définition}\peng{Nominalizer; locative or purposive.}\end{définition}
\begin{définition}\pcmn{名物化/处所格/目的格}\end{définition}
\begin{définition}\pfra{Nominalisateur; locatif; purposif.}\end{définition}
\begin{exemple}\pnru{tso˧∼tso˧-tɕɯ˧-di˧˥}\hspace{5pt}\peng{(piece of furniture/object) on which one can put things}\hspace{5pt}\pcmn{可以摆东西的(家具)}\hspace{5pt}\pfra{(meuble/objet) sur lequel on pose des choses}\end{exemple}
\end{entrée}

\begin{entrée}
{di˩˥}{}{ⓔdi˩˥}\formedesurface{di˩˥}\newline
\classe{名词}\ton{LH}
\paradigme{\pcmn{:} \p{}}
\begin{définition}\peng{Earth (as in: the sky and the earth).}\end{définition}
\begin{définition}\pcmn{地(天地的地)}\end{définition}
\begin{définition}\pfra{Terre (le ciel et la terre).}\end{définition}
\begin{exemple}\pnru{di˩ dv̩˩-ze˥}\hspace{5pt}\peng{to dig the earth}\hspace{5pt}\pcmn{挖土}\hspace{5pt}\pfra{creuser la terre}\end{exemple}
\begin{exemple}\pnru{di˩ hwæ˧-ze˩}\hspace{5pt}\peng{bought some earth}\hspace{5pt}\pcmn{买了土}\hspace{5pt}\pfra{a acheté de la terre}\end{exemple}
\begin{exemple}\pnru{ɖɯ˧-di˩ ɖɯ˩-bæ˩!}\hspace{5pt}\peng{Each place is different! (In particular, each place has its own pronunciation: its own dialect of the Na language)}\hspace{5pt}\pcmn{一个地方,一个样! = 每个地方有自己的特色(如:每个村落有自己的摩梭方言/土语)}\hspace{5pt}\pfra{C'est différent à chaque endroit/chaque endroit a ses choses propres (par exemple, en matière de langues, chaque village a sa prononciation, son dialecte)}\end{exemple}
\end{entrée}

\begin{entrée}
{di˩α}{}{ⓔdi˩α}\formedesurface{di˩˥}\newline
\classe{动词}\ton{Lα}\begin{définition}\peng{Existential verb: to have (a home); to have dirt on one's clothes; to have a different in length (two objects have a difference in length).}\end{définition}
\begin{définition}\pcmn{存在动词:有,拥有。例如:有家,有污垢在衣服上,有长短区别(两个物品有长短区别)}\end{définition}
\begin{définition}\pfra{Existentiel; posséder. Possession non amovible aussi bien que transitoire: avoir une maison, aussi bien que: avoir une tache de graisse sur la joue; avoir une différence de longueur (deux objets ont une différence de longueur).}\end{définition}
\begin{exemple}\pnru{ʈʰɯ˧ | ɑ˩ʁo˧ mɤ˧-di˩-hĩ˩.}\hspace{5pt}\peng{(S)he does not have a home. / (S)he is homeless.}\hspace{5pt}\pcmn{他没有家。}\hspace{5pt}\pfra{Elle/il n'a pas de maison, elle/il est sans domicile}\end{exemple}
\begin{exemple}\pnru{mɤ˧ tʰi˧-di˩}\hspace{5pt}\peng{there is grease (eg there is grease around the mouth of someone who has been biting away at large slabs of meat)}\hspace{5pt}\pcmn{有油(例如:一个人的嘴巴周围油乎乎,有油)}\hspace{5pt}\pfra{il y a de la graisse (ex.: autour de la bouche de quelqu'un qui vient de croquer de la viande à belles dents)}\end{exemple}
\begin{exemple}\pnru{ɖɯ˧-kʰwɤ˧ di˥}\hspace{5pt}\peng{there is something}\hspace{5pt}\pcmn{有一块东西}\hspace{5pt}\pfra{il y a quelque chose}\end{exemple}
\end{entrée}

\begin{entrée}
{di˩γ}{}{ⓔdi˩γ}\formedesurface{ɖɯ˧ di˩}\newline
\classe{量词}\ton{Lγ}\begin{définition}\peng{Self-classifier for plains, and places.}\end{définition}
\begin{définition}\pcmn{量词:坝子、地方(一个)}\end{définition}
\begin{définition}\pfra{Classificateur pour les plaines, les étendues de terre, les lieux.}\end{définition}
\begin{exemple}\pnru{ɖɯ˧-v̩˧ | ɖɯ˧-di˩ hɯ˩}\hspace{5pt}\peng{to separate, each going their separate ways}\hspace{5pt}\pcmn{分开,每个人去不同的地方}\hspace{5pt}\pfra{se séparer, partir chacun de son côté (par exemple: des frères se séparent et chacun va son chemin)}\end{exemple}
\end{entrée}

\begin{entrée}
{di˩-gɤ˩lɑ˥}{}{ⓔdi˩-gɤ˩lɑ˥}\formedesurface{di˩gɤ˩lɑ˥}\newline
\classe{名词}\ton{L+H\#}
\paradigme{\pcmn{:} \p{}}
\begin{définition}\peng{Earth spirit.}\end{définition}
\begin{définition}\pcmn{地菩萨}\end{définition}
\begin{définition}\pfra{Esprit de la terre, Bodhisattva terrestre.}\end{définition}
\end{entrée}

\begin{entrée}
{di˩li˩}{}{ⓔdi˩li˩}\formedesurface{di˩li˩˥}\newline
\classe{名词}\ton{L}
\paradigme{\pcmn{:} \p{}}
\begin{définition}\peng{Bracken.}\end{définition}
\begin{définition}\pcmn{蕨菜。俗名:龙爪菜}\end{définition}
\begin{définition}\pfra{Fougère. Elle était consommée comme nourriture en temps de disette.}\end{définition}
\begin{exemple}\pnru{di˩li˩-ʁo˩bv̩˥ (ton: L+H\#)}\hspace{5pt}\peng{bracken shoots}\hspace{5pt}\pcmn{蕨菜的萌芽}\hspace{5pt}\pfra{pousses de fougère}\end{exemple}
\begin{exemple}\pnru{di˩li˩-ʁo˩bv̩˥ hwæ˩}\hspace{5pt}\peng{to buy bracken shoots}\hspace{5pt}\pcmn{买蕨菜萌芽}\hspace{5pt}\pfra{acheter des pousses de fougère (verbe au ton M)}\end{exemple}
\begin{exemple}\pnru{di˩li˩-ʁo˩bv̩˥ tɕʰi˩}\hspace{5pt}\peng{to sell bracken shoots}\hspace{5pt}\pcmn{卖蕨菜萌芽}\hspace{5pt}\pfra{vendre des pousses de fougère}\end{exemple}
\begin{exemple}\pnru{di˩li˩-ʁo˩bv̩˥ dzɯ˩}\hspace{5pt}\peng{to eat bracken shoots}\hspace{5pt}\pcmn{吃蕨菜萌芽}\hspace{5pt}\pfra{manger des pousses de fougère}\end{exemple}
\begin{exemple}\pnru{di˩li˩-ʁo˩bv̩˥ dze˩}\hspace{5pt}\peng{to cut bracken shoots}\hspace{5pt}\pcmn{割蕨菜萌芽}\hspace{5pt}\pfra{couper des pousses de fougère}\end{exemple}
\begin{exemple}\pnru{di˩li˩-ʁo˩bv̩˥ tɕɤ˩}\hspace{5pt}\peng{to boil bracken shoots}\hspace{5pt}\pcmn{煮蕨菜嫩芽}\hspace{5pt}\pfra{faire bouillir des pousses de fougère}\end{exemple}
\end{entrée}

\begin{entrée}
{di˧mi˧}{}{ⓔdi˧mi˧}\formedesurface{di˧mi˧}\newline
\classe{名词}\ton{M}
\paradigme{\pcmn{:} \p{}}
\begin{définition}\peng{Large plain.}\end{définition}
\begin{définition}\pcmn{平坝、大平坝}\end{définition}
\begin{définition}\pfra{Grande plaine.}\end{définition}
\begin{exemple}\pnru{ɬi˧di˩-di˩mi˩}\hspace{5pt}\peng{the plain of Yongning}\hspace{5pt}\pcmn{永宁坝}\hspace{5pt}\pfra{la plaine de Yongning}\end{exemple}
\end{entrée}

\begin{entrée}
{di˧qo˧}{}{ⓔdi˧qo˧}\formedesurface{di˧qo˧}\newline
\classe{名词}\ton{M}
\paradigme{\pcmn{:} \p{}}
\begin{définition}\peng{Fields, cultivated lands.}\end{définition}
\begin{définition}\pcmn{田地}\end{définition}
\begin{définition}\pfra{Terres agricoles, champs.}\end{définition}
\end{entrée}

\begin{entrée}
{di˧ɻæ˧}{}{ⓔdi˧ɻæ˧}\formedesurface{di˧ɻæ˧}\newline
\classe{名词}\ton{M}
\paradigme{\pcmn{:} \p{}}
\begin{définition}\peng{Plain.}\end{définition}
\begin{définition}\pcmn{平地}\end{définition}
\begin{définition}\pfra{Plaine.}\end{définition}
\end{entrée}

\begin{entrée}
{do˥}{₁}{ⓔdo˥ⓗ1}\formedesurface{do˧}\newline
\classe{动词}\ton{H}
1\begin{définition}\peng{To climb.}\end{définition}
\begin{définition}\pcmn{爬,上去,上山}\end{définition}
\begin{définition}\pfra{Grimper, monter, escalader, gravir.}\end{définition}
\begin{exemple}\pnru{ʈʂo˩bo˩ do˩˥}\hspace{5pt}\peng{to climb a wall}\hspace{5pt}\pcmn{爬墙}\hspace{5pt}\pfra{grimper un mur, gravir un mur}\end{exemple}
\begin{exemple}\pnru{gɤ˩-do˧}\hspace{5pt}\peng{to escalate, to climb up}\hspace{5pt}\pcmn{爬上}\hspace{5pt}\pfra{grimper, gravir}\end{exemple}
\begin{exemple}\pnru{ʁwɤ˩ do˩˥}\hspace{5pt}\peng{to climb a mountain, to go hiking in the mountains}\hspace{5pt}\pcmn{爬山}\hspace{5pt}\pfra{gravir une montagne, grimper la montagne, faire de la montagne}\end{exemple}
\begin{exemple}\pnru{to˩ do˩˥}\hspace{5pt}\peng{to ascend a slope, to climb a slope}\hspace{5pt}\pcmn{爬坡}\hspace{5pt}\pfra{grimper la pente/ grimper une pente}\end{exemple}
\end{entrée}

\begin{entrée}
{do˥}{₂}{ⓔdo˥ⓗ2}\formedesurface{do˧}\newline
\classe{动词}\ton{H}
2\begin{définition}\peng{To mate with; to pair (of animal).}\end{définition}
\begin{définition}\pcmn{交配、交尾}\end{définition}
\begin{définition}\pfra{S'accoupler (animaux).}\end{définition}
\begin{exemple}\pnru{bo˩ɬɑ˥ | bo˩mi˧ do˧}\hspace{5pt}\peng{The pig mates with the sow.}\hspace{5pt}\pcmn{公猪与母猪交配。}\hspace{5pt}\pfra{Le verrat s'accouple avec la truie.}\end{exemple}
\end{entrée}

\begin{entrée}
{do˥α}{}{ⓔdo˥α}\formedesurface{ɖɯ˧ do˥}\newline
\classe{量词}\ton{Hα}\begin{définition}\peng{Classifier for partitions/walls.}\end{définition}
\begin{définition}\pcmn{量词.墙壁(一堵)}\end{définition}
\begin{définition}\pfra{Classificateur des cloisons et murs.}\end{définition}
\begin{exemple}\pnru{ʈʂʰɯ˧-do\#˥}\hspace{5pt}\peng{this partition/wall}\hspace{5pt}\pcmn{这堵(墙壁)}\hspace{5pt}\pfra{cette cloison}\end{exemple}
\end{entrée}

\begin{entrée}
{do˧}{₁}{ⓔdo˧ⓗ1}\formedesurface{do˧}\newline
\classe{形容词}\ton{M}
1\begin{définition}\peng{Stupid, silly, idiotic.}\end{définition}
\begin{définition}\pcmn{笨、愚蠢}\end{définition}
\begin{définition}\pfra{Bête, stupide.}\end{définition}
\begin{exemple}\pnru{zo˩ do˩˥}\hspace{5pt}\peng{idiot, village idiot}\hspace{5pt}\pcmn{傻瓜}\hspace{5pt}\pfra{un idiot, un fou du village; perçu comme: «un homme qui n'a pas grandi», «quelqu'un qui est resté enfant»}\end{exemple}
\end{entrée}

\begin{entrée}
{do˧}{₂}{ⓔdo˧ⓗ2}\formedesurface{do˧}\newline
\classe{形容词}\ton{M}
2\begin{définition}\peng{Sterile.}\end{définition}
\begin{définition}\pcmn{不能生育}\end{définition}
\begin{définition}\pfra{Stérile.}\end{définition}
\end{entrée}

\begin{entrée}
{do˩}{}{ⓔdo˩}\formedesurface{do˩˥}\newline
\classe{形容词}\ton{L}\begin{définition}\peng{Immature, lacking maturity.}\end{définition}
\begin{définition}\pcmn{不成熟、晚熟}\end{définition}
\begin{définition}\pfra{Immature.}\end{définition}
\begin{exemple}\pnru{ŋwɤ˩ɬi˩-mi˩˥, | ʂe˧ mɤ˧-mv̩˥, | ʂe˧ do˧˥! | tsʰe˩ŋwɤ˩ kʰv̩˥, | zo˧ mɤ˧-ti˩, | zo˧ do˧˥!}\hspace{5pt}\peng{In the fifth month, if cereals are still green (=if they do not yet yield grain), the crop is immature (and may not yield any harvest). At age 15, if a boy does not become an adult (=if a boy does not visit girls), he is immature (he is not developing normally)!}\hspace{5pt}\pcmn{五月份,谷物还是小草(还不出谷粒),算是晚熟!男人十五岁还不成熟(=还不见姑娘),算是晚熟!}\hspace{5pt}\pfra{Au cinquième mois, une céréale qui ne mûrit pas/qui ne donne pas de grain, c'est une récolte stérile/qui reste en herbe! A quinze ans, le garçon qui n'a pas encore acquis de maturité (=qui ne fréquente pas encore les filles), c'est qu'il a un problème/c'est un attardé!}\end{exemple}
\end{entrée}

\begin{entrée}
{do˩β}{}{ⓔdo˩β}\formedesurface{do˩˥}\newline
\classe{动词}\ton{Lβ}\begin{définition}\peng{To see; to come across someone.}\end{définition}
\begin{définition}\pcmn{看见,遇见,见}\end{définition}
\begin{définition}\pfra{Voir, apercevoir.}\end{définition}
\begin{exemple}\pnru{ɖɯ˧-do˥∼do˩-ɻ̍˩}\hspace{5pt}\peng{|fg{delimitative} \_ |fg{red} |fg{inceptive}}\hspace{5pt}\pcmn{见一见}\hspace{5pt}\pfra{|fg{délimitatif} \_ |fg{red} |fg{inchoatif}}\end{exemple}
\begin{exemple}\pnru{ɖɯ˧-kʰwɤ˧ do˧˥}\hspace{5pt}\peng{to see a piece}\hspace{5pt}\pcmn{看见一块(东西)}\hspace{5pt}\pfra{apercevoir un bout/un morceau}\end{exemple}
\begin{exemple}\pnru{tso˧∼tso˧ do˧˥}\hspace{5pt}\peng{to see things, to see something}\hspace{5pt}\pcmn{看见东西}\hspace{5pt}\pfra{apercevoir des choses/apercevoir quelque chose}\end{exemple}
\begin{exemple}\pnru{do˩-mɤ˩-ho˥}\hspace{5pt}\peng{\_ |fg{neg} |fg{desiderative}}\hspace{5pt}\pcmn{不许(看)见}\hspace{5pt}\pfra{\_ |fg{neg} |fg{désidératif}}\end{exemple}
\begin{exemple}\pnru{bo˩mi˧ do˩ (+ze˩)}\hspace{5pt}\peng{…has seen (a/the) sow}\hspace{5pt}\pcmn{看见了母猪}\hspace{5pt}\pfra{…a vu (une/la) truie}\end{exemple}
\end{entrée}

\begin{entrée}
{do˧bæ˧}{}{ⓔdo˧bæ˧}\formedesurface{do˧bæ˧}\newline
\classe{名词}\ton{M}
\paradigme{\pcmn{:} \p{}}
\begin{définition}\peng{Thigh.}\end{définition}
\begin{définition}\pcmn{大腿}\end{définition}
\begin{définition}\pfra{Cuisse.}\end{définition}
\begin{exemple}\pnru{do˧bæ˧ | ɖɯ˩-hĩ˩˥}\hspace{5pt}\peng{thigh}\hspace{5pt}\pcmn{大腿}\hspace{5pt}\pfra{cuisse}\end{exemple}
\begin{exemple}\pnru{do˧bæ˧ | tɕi˩-hĩ˩˥}\hspace{5pt}\peng{calf}\hspace{5pt}\pcmn{小腿}\hspace{5pt}\pfra{mollet}\end{exemple}
\end{entrée}

\begin{entrée}
{do˧bv̩˧}{}{ⓔdo˧bv̩˧}\formedesurface{do˧bv̩˧}\newline
\classe{名词}\ton{M}
\paradigme{\pcmn{:} \p{}}
\begin{définition}\peng{Buttocks.}\end{définition}
\begin{définition}\pcmn{屁股}\end{définition}
\begin{définition}\pfra{Fesse.}\end{définition}
\end{entrée}

\begin{entrée}
{do˩bv̩\#˥}{}{ⓔdo˩bv̩\#˥}\formedesurface{do˩bv̩˥}\newline
\classe{名词}\ton{LM+\#H}
\paradigme{\pcmn{:} \p{}}
\begin{définition}\peng{Mani wall, Mani pile: pile built of rubble and sand, with carved stone tablets, most with the inscription Om Mani Padme Hum. A Mani wall should be passed or circumvented from the left side, the clockwise direction in which the universe revolves, according to Buddhist doctrine.}\end{définition}
\begin{définition}\pcmn{嘛呢堆}\end{définition}
\begin{définition}\pfra{Mur de mani (le nom désigne l'ensemble du mur, pas seulement une des tablettes qui s'y trouvent). Le mur de mani est un mur de pierre sèche et de sable, comportant des tablettes de pierre sur lesquelles est gravé une inscription: le plus souvent Om Mani Padme Hum. Un mur de mani doit être contourné dans le sens des aiguilles d'une montre: le sens de rotation de l'univers, selon la doctrine bouddhiste.}\end{définition}
\end{entrée}

\begin{entrée}
{do˩kv̩\#˥}{}{ⓔdo˩kv̩\#˥}\formedesurface{do˩kv̩˥}\newline
\classe{名词}\ton{LM+\#H}
\paradigme{\pcmn{:} \p{}}
\begin{définition}\peng{Small beams upholding the ceiling of the ground floor.}\end{définition}
\begin{définition}\pcmn{小梁子,作为楼上(第二层)木地板的底}\end{définition}
\begin{définition}\pfra{Poutrelles soutenant le plancher du premier étage.}\end{définition}
\end{entrée}

\begin{entrée}
{dv̩˥}{}{ⓔdv̩˥}\formedesurface{dv̩˧}\newline
\classe{动词}\ton{H}\begin{définition}\peng{To dig.}\end{définition}
\begin{définition}\pcmn{挖}\end{définition}
\begin{définition}\pfra{Creuser.}\end{définition}
\begin{exemple}\pnru{ʈʂe˧ dv̩˧(-ze˩)}\hspace{5pt}\peng{to dig the earth}\hspace{5pt}\pcmn{挖土}\hspace{5pt}\pfra{piocher la terre, creuser la terre}\end{exemple}
\end{entrée}

\begin{entrée}
{dv̩˩}{₁}{ⓔdv̩˩ⓗ1}\formedesurface{ɖɯ˧ dv̩˩}\newline
\classe{量词}\ton{L *}
1\begin{définition}\peng{Classifier for flocks of cattle; only used in the singular.}\end{définition}
\begin{définition}\pcmn{量词:人、牲畜(一群、一队)}\end{définition}
\begin{définition}\pfra{Classificateur des troupeaux; ne s'utilise qu'au singulier.}\end{définition}
\end{entrée}

\begin{entrée}
{dv̩˩}{₂}{ⓔdv̩˩ⓗ2}\formedesurface{--}\newline
\classe{代词}\ton{L?}
2\begin{définition}\peng{Distal demonstrative, appearing in the phrase “this way, in this direction".}\end{définition}
\begin{définition}\pcmn{指示代词:那边(远指),从‘那个方向’这个短语提取出来的}\end{définition}
\begin{définition}\pfra{Démonstratif distal, qui apparaît dans l'indication de direction ‘par ici, dans cette direction’.}\end{définition}
\begin{exemple}\pnru{dv̩˩-tɕo˧}\hspace{5pt}\peng{that way}\hspace{5pt}\pcmn{那个方向}\hspace{5pt}\pfra{cette direction-là}\end{exemple}
\begin{exemple}\pnru{dv̩˩tɕo˧ fæ˧}\hspace{5pt}\peng{that way}\hspace{5pt}\pcmn{那个方向}\hspace{5pt}\pfra{cette direction-là}\end{exemple}
\end{entrée}

\begin{entrée}
{dv̩˩˧}{₁}{ⓔdv̩˩˧ⓗ1}\formedesurface{dv̩˩˥}\newline
\classe{名词}\ton{LM}
1
\paradigme{\pcmn{:} \p{}}
\begin{définition}\peng{Weasel.}\end{définition}
\begin{définition}\pcmn{黄鼠狼,黄喉貂}\end{définition}
\begin{définition}\pfra{Belette.}\end{définition}
\begin{exemple}\pnru{dv̩˩ hwæ˧-ze˧}\hspace{5pt}\peng{…bought (a) weasel}\hspace{5pt}\pcmn{买了黄鼠狼}\hspace{5pt}\pfra{…a acheté une belette}\end{exemple}
\begin{exemple}\pnru{dv̩˩ dzɯ˧-ze˩}\hspace{5pt}\peng{…ate a weasel}\hspace{5pt}\pcmn{吃了黄鼠狼}\hspace{5pt}\pfra{…a mangé une belette}\end{exemple}
\end{entrée}

\begin{entrée}
{dv̩˩˧}{₂}{ⓔdv̩˩˧ⓗ2}\formedesurface{dv̩˩˥}\newline
\classe{名词}\ton{LM}
2
\paradigme{\pcmn{:} \p{}}
\begin{définition}\peng{Poison.}\end{définition}
\begin{définition}\pcmn{毒}\end{définition}
\begin{définition}\pfra{Poison.}\end{définition}
\end{entrée}

\begin{entrée}
{dv̩˩α}{}{ⓔdv̩˩α}\formedesurface{dv̩˩˥}\newline
\classe{动词}\ton{Lα}
\sens{1}
\begin{définition}\peng{To poison.}\end{définition}
\begin{définition}\pcmn{毒害、毒化}\end{définition}
\begin{définition}\pfra{Empoisonner, rendre malade.}\end{définition}
\begin{exemple}\pnru{ʈʂʰɯ˧, | hĩ˧ dv̩˥-mɤ˩-kv̩˩! | ʈʂʰɯ˧, | hĩ˧ dv̩˥-kv̩˩!}\hspace{5pt}\peng{This one is not poisonous / is edible (literally “this one does not poison people")! That one [on the other hand] is poisonous / is not edible! (About different sorts of mushrooms.)}\hspace{5pt}\pcmn{这个,不会让人中毒!那个(反倒)会让人中毒!(情景:谈不同菌子种类。)}\hspace{5pt}\pfra{Celui-ci, il n'est pas dangereux / il est comestible! Celui-là [en revanche], il est vénéneux / il est dangereux / il peut vous rendre malade / il peut vous empoisonner / il n'est pas comestible! (Au sujet de diverses sortes de champignons.)}\end{exemple}
\begin{exemple}\pnru{ʈʂʰɯ˧, | dv̩˩-mɤ˩-kv̩˥!}\hspace{5pt}\peng{This one is not poisonous / is edible (literally “this one does not poison people")! (About a mushroom species.)}\hspace{5pt}\pcmn{这个,不会让人中毒!(情景:谈不同菌子种类。)}\hspace{5pt}\pfra{C'est inoffensif/comestible/pas vénéneux/pas dangereux! (Au sujet d'une sorte de champignon.)}\end{exemple}
\begin{relationsémantique}\{
renvoi
dv̩˩˧2
}\end{relationsémantique}\sens{2}
\begin{définition}\peng{To hate, to detest.}\end{définition}
\begin{définition}\pcmn{讨厌、恨}\end{définition}
\begin{définition}\pfra{Détester.}\end{définition}
\begin{exemple}\pnru{le˧-dv̩˩-ze˩}\hspace{5pt}\pfra{|fg{accomp} \_ |fg{pfv}}\end{exemple}
\begin{exemple}\pnru{njɤ˧ | ʈʂʰɯ˧ dv̩˥ | ʐwæ˩˥!}\hspace{5pt}\peng{I hate him/her!}\hspace{5pt}\pcmn{我很讨厌他!}\hspace{5pt}\pfra{je le déteste!}\end{exemple}
\begin{exemple}\pnru{dv̩˩-zo˧-mɤ˧-tʰɑ˧˥}\hspace{5pt}\peng{to hate deeply}\hspace{5pt}\pcmn{讨厌得不行}\hspace{5pt}\pfra{détester à mort}\end{exemple}
\end{entrée}

\begin{entrée}
{dv̩˩β}{}{ⓔdv̩˩β}\formedesurface{ɖɯ˧ dv̩˩}\newline
\classe{量词}\ton{Lβ}\begin{définition}\peng{Classifier for small groups of people: 3 or more.}\end{définition}
\begin{définition}\pcmn{量词:人(一些)}\end{définition}
\begin{définition}\pfra{Classificateur des petits groupes (de personnes): quelques-uns (plus de 3).}\end{définition}
\begin{exemple}\pnru{hĩ˧ ɖɯ˧-dv̩˩}\hspace{5pt}\peng{a few people, a group of people}\hspace{5pt}\pcmn{一些人}\hspace{5pt}\pfra{quelques personnes, un groupe de personnes}\end{exemple}
\begin{exemple}\pnru{hĩ˧ ʈʂʰɯ˧-dv̩˥}\hspace{5pt}\peng{these people, this group of people}\hspace{5pt}\pcmn{这些人}\hspace{5pt}\pfra{ce groupe de gens, ces qq personnes}\end{exemple}
\end{entrée}

\begin{entrée}
{dv̩˩bi˩}{}{ⓔdv̩˩bi˩}\formedesurface{dv̩˩bi˩˥}\newline
\classe{助词}\ton{L}\begin{définition}\peng{Opposite.}\end{définition}
\begin{définition}\pcmn{对面}\end{définition}
\begin{définition}\pfra{En face.}\end{définition}
\end{entrée}

\begin{entrée}
{dv̩˩mi\#˥}{}{ⓔdv̩˩mi\#˥}\formedesurface{dv̩˩mi˥}\newline
\classe{名词}\ton{LM+\#H}
\paradigme{\pcmn{:} \p{}}
\begin{définition}\peng{Female weasel.}\end{définition}
\begin{définition}\pcmn{母黄鼠狼}\end{définition}
\begin{définition}\pfra{Belette femelle.}\end{définition}
\begin{exemple}\pnru{dv̩˩mi˧-dv̩˥pʰv̩˩}\hspace{5pt}\peng{female weasel and male weasel}\hspace{5pt}\pcmn{母黄鼠狼与公黄鼠狼}\hspace{5pt}\pfra{belette femelle et belette mâle}\end{exemple}
\end{entrée}

\begin{entrée}
{dv̩˩pʰæ˧}{}{ⓔdv̩˩pʰæ˧}\formedesurface{dv̩˩pʰæ˥}\newline
\classe{名词}\ton{LM}
\paradigme{\pcmn{:} \p{}}
\begin{définition}\peng{The room in the main building of the farm where cereals were kept: the granary.}\end{définition}
\begin{définition}\pcmn{仓廪。摩梭话音译:‘独帕’}\end{définition}
\begin{définition}\pfra{Partie du bâtiment principal dans laquelle étaient conservées les céréales: le grenier à céréales.}\end{définition}
\end{entrée}

\begin{entrée}
{dv̩˩pʰv̩\#˥}{}{ⓔdv̩˩pʰv̩\#˥}\formedesurface{dv̩˩pʰv̩˥}\newline
\classe{名词}\ton{LM+\#H / LM}
\paradigme{\pcmn{:} \p{}}
\begin{définition}\peng{Male weasel.}\end{définition}
\begin{définition}\pcmn{公黄鼠狼}\end{définition}
\begin{définition}\pfra{Belette mâle.}\end{définition}
\end{entrée}

\begin{entrée}
{dv̩˩zo\#˥}{}{ⓔdv̩˩zo\#˥}\formedesurface{dv̩˩zo˥}\newline
\classe{名词}\ton{LM+\#H / LM}\begin{définition}\peng{Baby weasel.}\end{définition}
\begin{définition}\pcmn{黄鼠狼的崽子}\end{définition}
\begin{définition}\pfra{Bébé belette.}\end{définition}
\end{entrée}

\begin{entrée}
{dzɑ˥}{}{ⓔdzɑ˥}\formedesurface{dzɑ˥}\newline
\classe{形容词}\ton{H}
\sens{1}
\begin{définition}\peng{Bad, mean (action), inferior.}\end{définition}
\begin{définition}\pcmn{坏、差、下级(行为……)}\end{définition}
\begin{définition}\pfra{Mauvais (action…), inférieur, indigne.}\end{définition}
\begin{exemple}\pnru{ʈʂʰɯ˧-ɳɯ˧ | njɤ˧-ki˧ | dzɑ˧-ʝi˧ | ʐwæ˩˥!}\hspace{5pt}\peng{He really despises me!}\hspace{5pt}\pcmn{他很瞧不起我!}\hspace{5pt}\pfra{il me méprise vraiment!}\end{exemple}
\begin{exemple}\pnru{hĩ˧ ʈʂʰɯ˧-v̩˧ dʑo˩, | õ˧-ki˥ | dzɑ˧-ʝi˧-ze˩!}\hspace{5pt}\peng{This person has no self-respect! (literally: This person is doing herself harm)}\hspace{5pt}\pcmn{这个人,不尊重自己!}\hspace{5pt}\pfra{cette personne ne se respecte pas!}\end{exemple}
\begin{exemple}\pnru{mv̩˧ dzɑ˧.}\hspace{5pt}\peng{The weather is bad.}\hspace{5pt}\pcmn{天气很坏。}\hspace{5pt}\pfra{Il fait mauvais temps.}\end{exemple}
\begin{exemple}\pnru{mv̩˧ dzɑ˧-ze˩}\hspace{5pt}\peng{The weather is getting bad.}\hspace{5pt}\pcmn{天气变坏了。}\hspace{5pt}\pfra{Le temps se met au mauvais, il commence à faire mauvais temps.}\end{exemple}
\begin{exemple}\pnru{lo˧ dzɑ˧}\hspace{5pt}\peng{poor (work), bad (job: e.g. someone has done a bad job)}\hspace{5pt}\pcmn{(工作)差}\hspace{5pt}\pfra{bâclé, mal fait (travail)}\end{exemple}\sens{2}
\begin{définition}\peng{Poor (person).}\end{définition}
\begin{définition}\pcmn{穷(人)}\end{définition}
\begin{définition}\pfra{Indigent, pauvre (personne…).}\end{définition}
\begin{exemple}\pnru{dzɑ˧ | -ʐwæ˩-ze˥!}\hspace{5pt}\peng{(He/she) is really poor!}\hspace{5pt}\pcmn{他很穷!}\hspace{5pt}\pfra{(il/elle est) très pauvre!}\end{exemple}
\begin{exemple}\pnru{ɑ˩ʁo˧ | bo˩ʈʂʰæ˧ mɤ˧-dʑo˧, | dzɑ˧ ʈʂɤ˧-kv̩˩!}\hspace{5pt}\peng{If there is no fleshless preserved pork at home, it appears as if the family is really destitute!}\hspace{5pt}\pcmn{如果家里没有猪膘,会显得很穷!}\hspace{5pt}\pfra{Ne pas avoir de cochon-entier-conservé à la maison, ça fait vraiment mauvais effet/ça fait vraiment indigent/c'est la honte!}\end{exemple}
\begin{exemple}\pnru{dzɑ˧ ʈʂɤ˧ | ʐwæ˩˥!}\hspace{5pt}\peng{It's really a shame / it's really something to be ashamed of! (Talking about a socially stigmatized situation, such as not having the required food items or pieces of clothing for important ceremonies.)}\hspace{5pt}\pcmn{真羞耻啊!}\hspace{5pt}\pfra{C'est vraiment la honte/on paraît vraiment à la rue! (Au sujet de situations stigmatisées socialement, comme de ne pas posséder les nourritures ou vêtement nécessaires aux principaux rituels.)}\end{exemple}
\end{entrée}

\begin{entrée}
{dzɑ˩qʰwɤ˩}{}{ⓔdzɑ˩qʰwɤ˩}\formedesurface{dzɑ˧qʰwɤ˩˥}\newline
\classe{名词}\ton{L}
\paradigme{\pcmn{:} \p{}}
\begin{définition}\peng{Shoe.}\end{définition}
\begin{définition}\pcmn{鞋、鞋子}\end{définition}
\begin{définition}\pfra{Chaussure.}\end{définition}
\end{entrée}

\begin{entrée}
{dze˥}{}{ⓔdze˥}\formedesurface{dze˧}\newline
\classe{名词}\ton{\#H}\begin{définition}\peng{Sugar.}\end{définition}
\begin{définition}\pcmn{糖}\end{définition}
\begin{définition}\pfra{Sucre.}\end{définition}
\end{entrée}

\begin{entrée}
{dze˩}{}{ⓔdze˩}\formedesurface{dze˩˥}\newline
\classe{动词}\ton{L}\begin{définition}\peng{To be left over (food or drink).}\end{définition}
\begin{définition}\pcmn{剩下(饭或饮料)}\end{définition}
\begin{définition}\pfra{Rester, être en trop, devenir un reste (nourriture, boisson).}\end{définition}
\begin{exemple}\pnru{dzɯ˧-dze˥-ze˩!}\hspace{5pt}\peng{There are some leftovers! / The food has not been eaten up!}\hspace{5pt}\pcmn{剩了一些饭!/ 剩了一些吃的!}\hspace{5pt}\pfra{il y a des restes! / on n'a pas achevé de manger (un plat)!}\end{exemple}
\begin{exemple}\pnru{gɤ˩-dze˥ +ze˩!}\hspace{5pt}\peng{There are some leftovers!}\hspace{5pt}\pcmn{有剩下的!}\hspace{5pt}\pfra{Il en reste / il y a des restes!}\end{exemple}
\begin{exemple}\pnru{ʈʰɯ˩ dze˩-ze˥}\hspace{5pt}\peng{Some of the drink is left over! / (The drink) has not been drunk up!}\hspace{5pt}\pcmn{喝剩了、没喝完}\hspace{5pt}\pfra{Il en reste / on n'a pas achevé de boire (un verre…); ne pas être entièrement bu}\end{exemple}
\begin{exemple}\pnru{le˧-se˩-ze˩! | gɤ˩-mɤ˧-dze˩!}\hspace{5pt}\peng{It's completely finished (=eaten up / drunk up)! There are no leftovers!}\hspace{5pt}\pcmn{完了!(=全部吃/喝完了!)没有剩!}\hspace{5pt}\pfra{on a tout fini, il n'y a pas de restes! / tout a été mangé, bu…, il n'en reste plus !}\end{exemple}
\end{entrée}

\begin{entrée}
{dze˩˧}{}{ⓔdze˩˧}\formedesurface{dze˩˥}\newline
\classe{名词}\ton{LM}
\paradigme{\pcmn{:} \p{}}
\begin{définition}\peng{Wild pepper, Szechuan pepper.}\end{définition}
\begin{définition}\pcmn{花椒}\end{définition}
\begin{définition}\pfra{Xanthoxyle, poivre de Chine, poivre du Sichuan.}\end{définition}
\end{entrée}

\newpage\caractère{†}

\begin{entrée}
{†dze˩˧}{}{ⓔ†dze˩˧}\formedesurface{--}\newline
\classe{名词}\ton{LM}\begin{définition}\peng{Bee.}\end{définition}
\begin{définition}\pcmn{蜜蜂}\end{définition}
\begin{définition}\pfra{Abeille (racine déduite du disyllabe).}\end{définition}
\end{entrée}

\newpage\caractère{d}

\begin{entrée}
{dze˩α}{}{ⓔdze˩α}\formedesurface{ɖɯ˧ dze˩}\newline
\classe{量词}\ton{Lα}\begin{définition}\peng{Classifier for pairs of separable objects: a pair of pots, a pair of bottles…}\end{définition}
\begin{définition}\pcmn{量词:瓶子、锅(一对)}\end{définition}
\begin{définition}\pfra{Classificateur des lots de deux objets non indissociables; par ex.: lot de 2 casseroles; paire de haut-parleurs… Pour les paires non dissociables (ex.: paire de chaussures), on utilise: /dzi˧/.}\end{définition}
\begin{exemple}\pnru{zo˧mv̩˥ | ɖɯ˧-dze˩}\hspace{5pt}\peng{twins (literally: ‘a pair of children')}\hspace{5pt}\pcmn{双胞胎(直译:“一对孩子”)}\hspace{5pt}\pfra{des jumeaux (littéralement «une paire d'enfants»)}\end{exemple}
\begin{exemple}\pnru{ɖɯ˧-dze˩ dʑo˩-tsɯ˩ mv˩!}\hspace{5pt}\peng{She has twins! (literally: ‘(she) has a pair of children')}\hspace{5pt}\pcmn{双胞胎(直译:“一对孩子”)}\hspace{5pt}\pfra{Il paraît qu'elle a des jumeaux (littéralement «une paire d'enfants»)}\end{exemple}
\begin{exemple}\pnru{ʈʂʰɯ˧-dze˥}\hspace{5pt}\peng{|fg{dem} \_ (tone: H\# / H\$)}\hspace{5pt}\pcmn{指示代词 \_}\hspace{5pt}\pfra{|fg{dem} \_ (ton: H\# / H\$)}\end{exemple}
\end{entrée}

\begin{entrée}
{dze˩α}{₁}{ⓔdze˩αⓗ1}\formedesurface{dze˩˥}\newline
\classe{动词}\ton{Lα}
1\begin{définition}\peng{To fly.}\end{définition}
\begin{définition}\pcmn{飞}\end{définition}
\begin{définition}\pfra{Voler (dans les airs).}\end{définition}
\begin{exemple}\pnru{le˧-dze˩-hɯ˩-ze˩}\hspace{5pt}\peng{(The bird) has flown away.}\hspace{5pt}\pcmn{(鸟)飞走了。}\hspace{5pt}\pfra{(L'oiseau) est parti à tire-d'aile.}\end{exemple}
\begin{exemple}\pnru{mv̩˧ʁo˧ dze˧˥}\hspace{5pt}\peng{to fly in the sky}\hspace{5pt}\pcmn{在天空中飞}\hspace{5pt}\pfra{voler dans le ciel}\end{exemple}
\end{entrée}

\begin{entrée}
{dze˩α}{₂}{ⓔdze˩αⓗ2}\formedesurface{dze˩˥}\newline
\classe{动词}\ton{Lα}
2\begin{définition}\peng{To cut (with a knife).}\end{définition}
\begin{définition}\pcmn{切(用刀)}\end{définition}
\begin{définition}\pfra{Couper (avec un couteau).}\end{définition}
\begin{exemple}\pnru{le˧-dze˩}\hspace{5pt}\peng{|fg{accomp}}\hspace{5pt}\pcmn{|fg{accomp}}\hspace{5pt}\pfra{|fg{accomp}}\end{exemple}
\begin{exemple}\pnru{dze˧∼dze˥}\hspace{5pt}\peng{|fg{red}}\hspace{5pt}\pcmn{|fg{red}}\hspace{5pt}\pfra{|fg{red}}\end{exemple}
\begin{exemple}\pnru{le˧-dze˧∼dze˥}\hspace{5pt}\peng{|fg{accomp} \_ |fg{red}}\hspace{5pt}\pcmn{|fg{accomp} \_ |fg{red}}\hspace{5pt}\pfra{|fg{accomp} \_ |fg{red}}\end{exemple}
\begin{exemple}\pnru{v̩˩tsʰɤ˧ dze˧∼dze˥}\hspace{5pt}\peng{to cut vegetables}\hspace{5pt}\pcmn{切菜}\hspace{5pt}\pfra{découper des légumes}\end{exemple}
\begin{exemple}\pnru{nv̩˩dʑɯ˥ dze˩∼dze˩}\hspace{5pt}\peng{to cut tofu}\hspace{5pt}\pcmn{切豆腐}\hspace{5pt}\pfra{découper du tofu}\end{exemple}
\end{entrée}

\begin{entrée}
{dze˧bɤ˩}{}{ⓔdze˧bɤ˩}\formedesurface{dze˧bɤ˩}\newline
\classe{名词}\ton{L\#}
\paradigme{\pcmn{:} \p{}}
\begin{définition}\peng{Bat; used for all species, including the flying squirrel.}\end{définition}
\begin{définition}\pcmn{蝙蝠、飞鼠}\end{définition}
\begin{définition}\pfra{Chauve-souris; s'emploie pour toutes les espèces, y compris le renard volant.}\end{définition}
\begin{exemple}\pnru{dze˧bɤ˩-zo˩ | ɖɯ˧-ɭɯ˧}\hspace{5pt}\peng{a baby bat}\hspace{5pt}\pcmn{一只小蝙蝠}\hspace{5pt}\pfra{une petite chauve-souris, un bébé chauve-souris}\end{exemple}
\begin{exemple}\pnru{dze˧bɤ˩-pʰv̩˩ | ɖɯ˧-mi˩}\hspace{5pt}\peng{a male bat}\hspace{5pt}\pcmn{一只公蝙蝠}\hspace{5pt}\pfra{une chauve-souris mâle}\end{exemple}
\begin{exemple}\pnru{dze˧bɤ˩-mi˩ | ɖɯ˧-mi˩}\hspace{5pt}\peng{a female bat}\hspace{5pt}\pcmn{一只母蝙蝠}\hspace{5pt}\pfra{une chauve-souris femelle}\end{exemple}
\end{entrée}

\begin{entrée}
{dze˧bo˧}{}{ⓔdze˧bo˧}\formedesurface{dze˧bo˧}\newline
\classe{名词}
\sens{1}
\begin{définition}\peng{A family name from Yongning.}\end{définition}
\begin{définition}\pcmn{者波(姓)。这个家族有三个家庭。}\end{définition}
\begin{définition}\pfra{Nom de clan/famille étendue qui compte 3 familles. C'est également le nom d'un village de la plaine de Yongning.}\end{définition}
\begin{exemple}\pnru{dze˧bo˧=ɻ̍˩}\hspace{5pt}\peng{the /dze˧bo˧/ clan, the /dze˧bo˧/ family}\hspace{5pt}\pcmn{者波家族}\hspace{5pt}\pfra{le clan /dze˧bo˧/, la famille /dze˧bo˧/}\end{exemple}\sens{2}
\begin{définition}\peng{A village in the Yongning plain. It consists of two parts, “upper" and “lower: \stylefv{/gɤ}˩ʁwɤ˧/ and \stylefv{/mv̩}˩ʁwɤ˧/.}\end{définition}
\begin{définition}\pcmn{者波(永宁坝子的一个村落)。村落有两个部分,|fv{/gɤ˩ʁwɤ˧/}‘上村’与|fv{/mv̩˩ʁwɤ˧/}‘下村’.}\end{définition}
\begin{définition}\pfra{Zhebo, un village de la plaine de Yongning. Il est divisé en deux parties, «du haut» et «du bas»: \stylefv{/gɤ}˩ʁwɤ˧/ et \stylefv{/mv̩}˩ʁwɤ˧/.}\end{définition}
\begin{exemple}\pnru{ɖæ˩ʂɯ\#˥, | ʈʂo˧ʂɯ\#˥, | bɤ˩tɕʰɯ˩˥, | dɑ˧pʰo˥, | bɤ˧dzi˩, | dze˧bo˧}\hspace{5pt}\peng{Six villages of the plain of Yongning that lie relatively close to the Lake.}\hspace{5pt}\pcmn{永宁摩梭地理概念中,距离泸沽湖比较近的六个村落:扎实、忠实、八旗、达坡、八珠、者波。}\hspace{5pt}\pfra{Six villages de la plaine de Yongning qui sont relativement proches du Lac.}\end{exemple}
\end{entrée}

\begin{entrée}
{dze˧dv̩˩}{}{ⓔdze˧dv̩˩}\formedesurface{dze˧dv̩˩}\newline
\classe{名词}\ton{L\#}\begin{définition}\peng{Cake, bread.}\end{définition}
\begin{définition}\pcmn{饼}\end{définition}
\begin{définition}\pfra{Galette de céréale (blé, avoine…), pain.}\end{définition}
\begin{exemple}\pnru{dze˧dv̩˩-pɤ˩jɤ˩}\hspace{5pt}\peng{cake of cereals}\hspace{5pt}\pcmn{粮食饼}\hspace{5pt}\pfra{galette de céréale (même sens)}\end{exemple}
\end{entrée}

\begin{entrée}
{dze˩dʑɯ˧˥}{}{ⓔdze˩dʑɯ˧˥}\formedesurface{dze˩dʑɯ˧˥}\newline
\classe{形容词}\ton{LM+MH\#}\begin{définition}\peng{Arrogant, conceited.}\end{définition}
\begin{définition}\pcmn{骄傲,自以为好}\end{définition}
\begin{définition}\pfra{Orgueilleux, arrogant.}\end{définition}
\begin{exemple}\pnru{ʈʂʰɯ˧ | hĩ˧-bi˥ | mɤ˧-li˧! | dze˩dʑɯ˧˥ | ʐwæ˧˥!}\hspace{5pt}\peng{He despises others! He is very arrogant!}\hspace{5pt}\pcmn{他看不起别人!他很骄傲!}\hspace{5pt}\pfra{il méprise les autres! il est très orgueilleux!}\end{exemple}
\end{entrée}

\begin{entrée}
{dze˧hi˧}{}{ⓔdze˧hi˧}\formedesurface{dze˧hi˧}\newline
\classe{名词}\ton{M}\begin{définition}\peng{In-laws.}\end{définition}
\begin{définition}\pcmn{丈人}\end{définition}
\begin{définition}\pfra{Beaux-parents.}\end{définition}
\begin{exemple}\pnru{njɤ˧ | dze˧hi˧-ki˩ bi˩!}\hspace{5pt}\peng{I'm going to my in-laws' place! / I'm going to visit my in-laws!}\hspace{5pt}\pcmn{我去我丈人(那边)!}\hspace{5pt}\pfra{Je vais chez mes beaux-parents!}\end{exemple}
\begin{exemple}\pnru{no˧ | dze˧hi˧ | ə˩-to˩-ze˥? - le˧-to˩-ze˩!}\hspace{5pt}\peng{Do you have in-laws? / Do you stand in an ‘in-law' relationship? (=Are you married?) - Yes, I have entered into such a relationship! (=Yes, I am married!)}\hspace{5pt}\pcmn{你有丈人吗?(=你结婚了吗?)-有的!(=结婚了!)}\hspace{5pt}\pfra{Tu as une belle-famille? =Tu es marié(e)? -Oui!}\end{exemple}
\begin{exemple}\pnru{no˧ | dze˧hi˧ to˩ ə˩-bi˩?}\hspace{5pt}\peng{Do you have plans to get married? (Literally: Are you going to enter an ‘in-law' relationship?)}\hspace{5pt}\pcmn{你打算结婚吗?}\hspace{5pt}\pfra{Tu comptes te marier? (Littéralement: Tu vas te lier avec une belle-famille?)}\end{exemple}
\end{entrée}

\begin{entrée}
{dze˧kʰɤ˧˥}{}{ⓔdze˧kʰɤ˧˥}\formedesurface{dze˧kʰɤ˧˥}\newline
\classe{名词}\ton{MH\#}
\paradigme{\pcmn{:} \p{}}
\begin{définition}\peng{Commoner (second of the three ranks in feudal society).}\end{définition}
\begin{définition}\pcmn{百姓。音译:“责卡”}\end{définition}
\begin{définition}\pfra{Roturier, la 2e des 3 castes de la société ancienne, intermédiaire entre la noblesse et les serfs.}\end{définition}
\end{entrée}

\begin{entrée}
{dze˧ɭɯ˧}{}{ⓔdze˧ɭɯ˧}\formedesurface{dze˧ɭɯ˧}\newline
\classe{名词}\ton{M}\begin{définition}\peng{Wheat.}\end{définition}
\begin{définition}\pcmn{小麦}\end{définition}
\begin{définition}\pfra{Blé, froment.}\end{définition}
\end{entrée}

\begin{entrée}
{dze˧ɭɯ˧-ɻ̃\#˥}{}{ⓔdze˧ɭɯ˧-ɻ̃\#˥}\formedesurface{dze˧ɭɯ˧ɻ̃˧}\newline
\classe{名词}\ton{\#H}\begin{définition}\peng{Wheat straw.}\end{définition}
\begin{définition}\pcmn{麦杆}\end{définition}
\begin{définition}\pfra{Paille de blé.}\end{définition}
\end{entrée}

\begin{entrée}
{dze˩mi˧}{}{ⓔdze˩mi˧}\formedesurface{dze˧mi˧}\newline
\classe{名词}\ton{LM}
\paradigme{\pcmn{:} \p{}}
\begin{définition}\peng{Bee.}\end{définition}
\begin{définition}\pcmn{蜜蜂}\end{définition}
\begin{définition}\pfra{Abeille.}\end{définition}
\end{entrée}

\begin{entrée}
{dze˩mi˧-bæ˩bæ˩}{}{ⓔdze˩mi˧-bæ˩bæ˩}\formedesurface{dze˩mi˧bæ˩bæ˩}\newline
\classe{名词}\ton{LM-L}
\paradigme{\pcmn{:} \p{}}
\begin{définition}\peng{|\stylefi{Artemisia suboligata}.}\end{définition}
\begin{définition}\pcmn{茶绒蒿}\end{définition}
\begin{définition}\pfra{|\stylefi{Artemisia suboligata}; littéralement «la fleur aux abeilles».}\end{définition}
\end{entrée}

\begin{entrée}
{dze˩mi˧-dze\#˥}{}{ⓔdze˩mi˧-dze\#˥}\formedesurface{dze˩mi˧dze˧}\newline
\classe{名词}\ton{LM+\#H}
\étymologie{
dze˩mi˧; dze˥
}
\paradigme{\pcmn{:} \p{}}
\begin{définition}\peng{Honey.}\end{définition}
\begin{définition}\pcmn{蜂蜜}\end{définition}
\begin{définition}\pfra{Miel.}\end{définition}
\begin{exemple}\pnru{dze˩mi˧dze˧ dzɯ˧}\hspace{5pt}\peng{to eat honey}\hspace{5pt}\pcmn{吃蜂蜜}\hspace{5pt}\pfra{manger du miel}\end{exemple}
\end{entrée}

\begin{entrée}
{dze˩mi˧-kʰv̩˩}{}{ⓔdze˩mi˧-kʰv̩˩}\formedesurface{dze˩mi˧kʰv̩˩}\newline
\classe{名词}\ton{LM-L}
\paradigme{\pcmn{:} \p{}}
\begin{définition}\peng{Beehive, honeycomb.}\end{définition}
\begin{définition}\pcmn{蜂窝}\end{définition}
\begin{définition}\pfra{Ruche.}\end{définition}
\end{entrée}

\begin{entrée}
{dze˩mi˧-pv̩˥ɻ̍˩}{}{ⓔdze˩mi˧-pv̩˥ɻ̍˩}\formedesurface{dze˩mi˧pv̩˥ɻ̍˩}\newline
\classe{名词}\ton{LM+\#H-}\begin{définition}\peng{Straight ladybell, |\stylefi{Adenophora sp.} The Na word literally means ‘bees' resting place’, and is also applied to various other flowers of which bees are particularly fond.}\end{définition}
\begin{définition}\pcmn{沙参}\end{définition}
\begin{définition}\pfra{|\stylefi{Adenophora sp.} Le mot signifie littéralement ‘repos des abeilles’, et est également usité pour diverses fleurs dont les abeilles sont particulièrement friandes.}\end{définition}
\begin{exemple}\pnru{dze˩mi˧-pv̩˥ɻ̍˩-kʰɯ˩ʈɯ˩}\hspace{5pt}\peng{root of straight ladybell}\hspace{5pt}\pcmn{沙参根}\hspace{5pt}\pfra{racine d'|Adenophora sp.}\end{exemple}
\end{entrée}

\begin{entrée}
{dze˩mi˧-pv̩˥ɻ̍˩}{}{ⓔdze˩mi˧-pv̩˥ɻ̍˩}\formedesurface{dze˩mi˧pv̩˥ɻ̍˩}\newline
\classe{名词}\ton{LM+\#H-}
\paradigme{\pcmn{:} \p{}}
\begin{définition}\peng{Large-leaved gentian.}\end{définition}
\begin{définition}\pcmn{秦艽}\end{définition}
\begin{définition}\pfra{Gentiane.}\end{définition}
\begin{exemple}\pnru{dʑɯ˧qʰɑ˧-bæ˩bæ˩}\hspace{5pt}\peng{gentian flowers}\hspace{5pt}\pcmn{秦艽花}\hspace{5pt}\pfra{fleurs de gentiane}\end{exemple}
\end{entrée}

\begin{entrée}
{dze˧-ɻ̃\#˥}{}{ⓔdze˧-ɻ̃\#˥}\formedesurface{dze˧ɻ̃˧}\newline
\classe{名词}\ton{\#H}\begin{définition}\peng{Wheat straw.}\end{définition}
\begin{définition}\pcmn{小麦秆}\end{définition}
\begin{définition}\pfra{Paille de blé.}\end{définition}
\end{entrée}

\begin{entrée}
{dze˧-tɕʰi\#˥}{}{ⓔdze˧-tɕʰi\#˥}\formedesurface{dze˧tɕʰi˧}\newline
\classe{名词}\ton{\#H}\begin{définition}\peng{Wheat beard.}\end{définition}
\begin{définition}\pcmn{麦芒}\end{définition}
\begin{définition}\pfra{Barbe de blé.}\end{définition}
\end{entrée}

\begin{entrée}
{dze˧-ʈʂæ˥}{}{ⓔdze˧-ʈʂæ˥}\formedesurface{dze˧ʈʂæ˥}\newline
\classe{名词}\ton{H\#}
\paradigme{\pcmn{:} \p{}}
\begin{définition}\peng{Stinging organ of a bee.}\end{définition}
\begin{définition}\pcmn{蜜蜂的螫針}\end{définition}
\begin{définition}\pfra{Dard de l'abeille.}\end{définition}
\end{entrée}

\begin{entrée}
{dze˧ʈʂɯ˧}{}{ⓔdze˧ʈʂɯ˧}\formedesurface{dze˧ʈʂɯ˧}\newline
\classe{名词}\ton{M}
\paradigme{\pcmn{:} \p{}}
\begin{définition}\peng{Sifter, sieve.}\end{définition}
\begin{définition}\pcmn{筛子}\end{définition}
\begin{définition}\pfra{Vanneries: tamis, crible.}\end{définition}
\end{entrée}

\begin{entrée}
{dze˧ʈʂʰɤ\$˥}{}{ⓔdze˧ʈʂʰɤ\$˥}\formedesurface{dze˧ʈʂʰɤ˥}\newline
\classe{名词}\ton{H\$}\begin{définition}\peng{Cereals; the main cereal crop used to be barley, but the meaning of this word currently tends to become identified with the five main sorts of grains referred to in Chinese as ‘the five cereals', \stylefn{五谷,} namely rice, two kinds of millet, wheat, and beans.}\end{définition}
\begin{définition}\pcmn{粮食。现在,这个词的含义受到汉语‘五谷’这个词的影响,用来指代‘五谷杂粮’,相当于所有粮食类,如:小米类、稻谷、麦子、玉米以及豆类与薯类。}\end{définition}
\begin{définition}\pfra{Céréales; la céréale traditionnelle était l'orge, mais le sens de l'expression tend actuellement à se confondre avec celui de l'expression chinoise \stylefn{五谷} ‘les cinq céréales': le riz; deux sortes de millet; le blé; et les fèves.}\end{définition}
\end{entrée}

\begin{entrée}
{dzɤ˥β}{}{ⓔdzɤ˥β}\formedesurface{ɖɯ˧ dzɤ˥}\newline
\classe{量词}\ton{Hβ}\begin{définition}\peng{Classifier for sides.}\end{définition}
\begin{définition}\pcmn{量词:面}\end{définition}
\begin{définition}\pfra{Côté.}\end{définition}
\begin{exemple}\pnru{ʈʂʰɯ˧-dzɤ˧}\hspace{5pt}\peng{this side}\hspace{5pt}\pcmn{这面}\hspace{5pt}\pfra{ce côté-ci}\end{exemple}
\begin{exemple}\pnru{ɖɯ˧-dzɤ˥}\hspace{5pt}\peng{one side}\hspace{5pt}\pcmn{一面}\hspace{5pt}\pfra{un côté}\end{exemple}
\end{entrée}

\begin{entrée}
{dzɤ˩α}{}{ⓔdzɤ˩α}\formedesurface{dzɤ˩˥}\newline
\classe{动词}\ton{Lα}\begin{définition}\peng{To collapse, to topple over, to fall into ruin.}\end{définition}
\begin{définition}\pcmn{塌毁,倒塌 ,倒}\end{définition}
\begin{définition}\pfra{S’écrouler, s'effondrer (mur); tomber (arbre), se renverser, s'abattre.}\end{définition}
\begin{exemple}\pnru{mv̩˩tɕo˧ dzɤ˩}\hspace{5pt}\peng{same meaning: to collapse}\hspace{5pt}\pcmn{同上:塌毁}\hspace{5pt}\pfra{même sens: s'écrouler}\end{exemple}
\begin{exemple}\pnru{le˧-dzɤ˩-ze˩}\hspace{5pt}\peng{|fg{accomp} \_ |fg{pfv}}\hspace{5pt}\pcmn{塌毁了}\hspace{5pt}\pfra{|fg{accomp} \_ |fg{pfv}}\end{exemple}
\end{entrée}

\begin{entrée}
{dzi˥}{}{ⓔdzi˥}\formedesurface{dzi˧}\newline
\classe{名词}\ton{\#H}
\paradigme{\pcmn{:} \p{}}
\begin{définition}\peng{Chisel.}\end{définition}
\begin{définition}\pcmn{凿子}\end{définition}
\begin{définition}\pfra{Burin, ciseau.}\end{définition}
\end{entrée}

\begin{entrée}
{dzi˧˥}{}{ⓔdzi˧˥}\formedesurface{dzi˧˥}\newline
\classe{动词}\ton{MH}\begin{définition}\peng{To tremble, to shake.}\end{définition}
\begin{définition}\pcmn{颤抖、抖动}\end{définition}
\begin{définition}\pfra{Trembler.}\end{définition}
\begin{exemple}\pnru{njɤ˩ dzi˧˥}\hspace{5pt}\peng{the eyelid trembles (literally ‘the eye trembles')}\hspace{5pt}\pcmn{眼皮跳}\hspace{5pt}\pfra{la paupière tremble}\end{exemple}
\begin{exemple}\pnru{njɤ˩ dzi˧-ze˥}\hspace{5pt}\peng{the eyelid trembles}\hspace{5pt}\pcmn{眼皮跳}\hspace{5pt}\pfra{la paupière tremble}\end{exemple}
\begin{exemple}\pnru{njɤ˩ dzi˧˥ | ʐwæ˩˥}\hspace{5pt}\peng{the eyelid trembles terribly}\hspace{5pt}\pcmn{眼皮跳得很厉害}\hspace{5pt}\pfra{la paupière tremble très fort}\end{exemple}
\begin{exemple}\pnru{njɤ˩˥ | le˧-dzi˧-ze˥}\hspace{5pt}\peng{the eyelid trembles}\hspace{5pt}\pcmn{眼皮跳}\hspace{5pt}\pfra{la paupière tremble}\end{exemple}
\begin{exemple}\pnru{njɤ˩˥ | mɤ˧-dzi˧˥}\hspace{5pt}\peng{the eyelid does not tremble}\hspace{5pt}\pcmn{眼皮不跳}\hspace{5pt}\pfra{la paupière ne tremble pas}\end{exemple}
\end{entrée}

\begin{entrée}
{dzi˧β}{}{ⓔdzi˧β}\formedesurface{ɖɯ˧ dzi˧}\newline
\classe{量词}\ton{Mβ}\begin{définition}\peng{Classifier for pairs of objects, when the pair makes up a unit: e.g. a pair of shoes.}\end{définition}
\begin{définition}\pcmn{量词:鞋(一双)}\end{définition}
\begin{définition}\pfra{Paire d'objets qui constitue une unité: par exemple une paire de chaussures.}\end{définition}
\begin{exemple}\pnru{ɣɯ˩-dzɑ˩qʰwɤ˥ | ɖɯ˧-dzi˧}\hspace{5pt}\peng{a pair of leather shoes}\hspace{5pt}\pcmn{一双皮鞋}\hspace{5pt}\pfra{une paire de chaussures en cuir}\end{exemple}
\end{entrée}

\begin{entrée}
{dzi˩}{₁}{ⓔdzi˩ⓗ1}\formedesurface{dzi˩˥}\newline
\classe{动词}\ton{L}
1\begin{définition}\peng{To fall, to come (of night); to get (dark).}\end{définition}
\begin{définition}\pcmn{来(晚上来了)}\end{définition}
\begin{définition}\pfra{Tomber, venir (la nuit tombe, la nuit vient).}\end{définition}
\begin{exemple}\pnru{nɑ˩˥ | le˧-dzi˩-ze˩!}\hspace{5pt}\peng{The night has fallen! / It has got dark!}\hspace{5pt}\pcmn{天黑了!}\hspace{5pt}\pfra{la nuit est tombée! Il fait noir!}\end{exemple}
\begin{exemple}\pnru{nɑ˩˥ | le˧-dzi˩ | le˧-se˩-ze˩!}\hspace{5pt}\peng{It has got completely dark!}\hspace{5pt}\pcmn{天完全黑了!}\hspace{5pt}\pfra{il fait tout à fait nuit!}\end{exemple}
\end{entrée}

\begin{entrée}
{dzi˩}{₂}{ⓔdzi˩ⓗ2}\formedesurface{ɖɯ˧ dzi˩}\newline
\classe{量词}\ton{Lγ}
2\begin{définition}\peng{Classifier for entire dresses.}\end{définition}
\begin{définition}\pcmn{量词:衣服(一套)}\end{définition}
\begin{définition}\pfra{Un costume entier: tous les vêtements qu'on porte.}\end{définition}
\begin{exemple}\pnru{dʑi˧hṽ̩˥ | ɖɯ˧-dzi˩}\hspace{5pt}\peng{a full set of dress, a complete dress}\hspace{5pt}\pcmn{一套衣服}\hspace{5pt}\pfra{un costume entier, un vêtement}\end{exemple}
\begin{exemple}\pnru{dʑi˧hṽ̩˧ ɖɯ˧-dzi˩}\hspace{5pt}\peng{a full set of dress, a complete dress (same as preceding example, integrated into a single tone group)}\hspace{5pt}\pcmn{一套衣服(同上,但将短语合在一起,构成一个单一的声调短语)}\hspace{5pt}\pfra{un costume entier, un vêtement (même sens que l'exemple précédent; intégration en un seul groupe tonal)}\end{exemple}
\end{entrée}

\begin{entrée}
{dzi˩α}{₁}{ⓔdzi˩αⓗ1}\formedesurface{dzi˩˥}\newline
\classe{动词}
1
\sens{1}
\begin{définition}\peng{To sit.}\end{définition}
\begin{définition}\pcmn{坐}\end{définition}
\begin{définition}\pfra{S'asseoir, être assis.}\end{définition}
\begin{exemple}\pnru{tʰi˧-dzi˩!}\hspace{5pt}\peng{Sit down!}\hspace{5pt}\pcmn{坐下!}\hspace{5pt}\pfra{Asseyez-vous!}\end{exemple}
\begin{exemple}\pnru{hĩ˧bæ˧ ʈʂʰɯ˧-qo˧ dzi˩.}\hspace{5pt}\peng{The guest sits here.}\hspace{5pt}\pcmn{客人是坐在这边的。}\hspace{5pt}\pfra{L'invité s'asseoit ici.}\end{exemple}
\begin{exemple}\pnru{(ʈʂʰɯ˧ | ) tʰi˧-dzi˩-kʰɯ˩-se˩.}\hspace{5pt}\peng{(S)he has got seated.}\hspace{5pt}\pcmn{他坐下了。}\hspace{5pt}\pfra{Il est assis/installé, il a pris sa place.}\end{exemple}
\begin{exemple}\pnru{le˧-dzi˧∼dzi˥}\hspace{5pt}\peng{to remain seated; used as a euphemism to mean: to sit with others at a funeral wake, to keep a deathwatch}\hspace{5pt}\pcmn{坐一坐。来指:居丧、守灵(委婉语)}\hspace{5pt}\pfra{se tenir assis; s'emploie, par euphémisme, pour désigner la participation à une veillée funèbre}\end{exemple}\sens{2}
\begin{définition}\peng{To dwell, to live at a place.}\end{définition}
\begin{définition}\pcmn{住}\end{définition}
\begin{définition}\pfra{Demeurer, habiter.}\end{définition}
\begin{exemple}\pnru{dzi˩-bi˩-ni˩gv̩˩}\hspace{5pt}\peng{to be accustomed to; to get accustomed to, to feel at ease, to adapt (to an environment)}\hspace{5pt}\pcmn{习惯(一个新的环境、一个地方的饮食……)}\hspace{5pt}\pfra{s'habituer (à un environnement; à des habitudes de vie: nourriture, boisson…; à quelque chose; à quelqu'un)}\end{exemple}
\begin{exemple}\pnru{dzi˩-bi˩-ni˩-mɤ˩-gv̩˩˥}\hspace{5pt}\peng{not to get accustomed; to feel awkward}\hspace{5pt}\pcmn{不习惯}\hspace{5pt}\pfra{je n'aime pas/je ne m'y fais pas/ça ne me plaît pas (ex.: quelqu'un ne veut pas rester quelque part, il s'y trouve mal et a la nostalgie d'ailleurs)}\end{exemple}
\begin{exemple}\pnru{njɤ˧ | ʈʂʰɯ˧-qo˧ | dzi˩-bi˩-ni˩-mɤ˩-gv̩˩˥}\hspace{5pt}\peng{I can't get accustomed (to this place)! / (I) can't make myself at ease here! / (I) don't like it here!}\hspace{5pt}\pcmn{我不习惯这里! / 我不喜欢这里! / 我不想待了!}\hspace{5pt}\pfra{Je ne me fais pas à ici! / Je ne suis pas bien ici! (Paraphrase proposée par M23: «Je ne veux pas rester ici!»)}\end{exemple}
\end{entrée}

\begin{entrée}
{dzi˩α}{₂}{ⓔdzi˩αⓗ2}\formedesurface{dzi˩˥}\newline
\classe{动词}\ton{Lα}
2\begin{définition}\peng{To gather, to assemble (people gather together).}\end{définition}
\begin{définition}\pcmn{聚集}\end{définition}
\begin{définition}\pfra{Se rassembler (groupe de personnes).}\end{définition}
\begin{exemple}\pnru{ɖɯ˧-ʁwɤ˧ | le˧-dzi˧∼dzi˥}\hspace{5pt}\peng{the whole village gathered together}\hspace{5pt}\pcmn{全村都聚集在一起}\hspace{5pt}\pfra{tout le village se rassemble}\end{exemple}
\begin{exemple}\pnru{hĩ˧ ɖɯ˧-v̩˧ | le˧-ʂɯ˧-ze˧! le˧-dzi˧∼dzi˥ jo˩!}\hspace{5pt}\peng{Someone has passed away! Come and join the gathering!}\hspace{5pt}\pcmn{一个人去世了!来参加丧礼吧!}\hspace{5pt}\pfra{Quelqu'un est mort! Venez participer à la réunion/au rassemblement!}\end{exemple}
\end{entrée}

\begin{entrée}
{dzi˩α}{₃}{ⓔdzi˩αⓗ3}\formedesurface{dzi˩˥}\newline
\classe{动词}\ton{Lα}
3\begin{définition}\peng{To drop, to fall, to sink (e.g. boat slowly sinking down into a lake).}\end{définition}
\begin{définition}\pcmn{掉入、沉下去}\end{définition}
\begin{définition}\pfra{Tomber, sombrer (ex.: un bateau qui sombre peu à peu dans le lac).}\end{définition}
\begin{exemple}\pnru{mv̩˩tɕo˧ dzi˩}\hspace{5pt}\peng{to sink down}\hspace{5pt}\pcmn{往下掉入、沉下去}\hspace{5pt}\pfra{|fg{directionnel} \_ ; même sens: sombrer, couler}\end{exemple}
\end{entrée}

\begin{entrée}
{dzi˩β}{}{ⓔdzi˩β}\formedesurface{ɖɯ˧ dzi˩}\newline
\classe{量词}\ton{Lβ}\begin{définition}\peng{Classifier for trees, bamboos…}\end{définition}
\begin{définition}\pcmn{量词:树(一棵),竹子(一根)}\end{définition}
\begin{définition}\pfra{Classificateur des arbres.}\end{définition}
\begin{exemple}\pnru{si˧dzi˩ | ɖɯ˧-dzi˩}\hspace{5pt}\peng{a tree}\hspace{5pt}\pcmn{一棵树}\hspace{5pt}\pfra{un arbre}\end{exemple}
\begin{exemple}\pnru{tʰv̩˧-dzi˧˥}\hspace{5pt}\peng{that tree}\hspace{5pt}\pcmn{那棵树}\hspace{5pt}\pfra{cet arbre}\end{exemple}
\end{entrée}

\begin{entrée}
{dzi˧dzi˧}{}{ⓔdzi˧dzi˧}\formedesurface{dzi˧dzi˧}\newline
\classe{名词}\ton{M}\begin{définition}\peng{Oriental white oak.}\end{définition}
\begin{définition}\pcmn{青冈树、槲栎}\end{définition}
\begin{définition}\pfra{Chêne blanc oriental.}\end{définition}
\begin{exemple}\pnru{dzi˧dzi˧, | si˧dzi˩-mv̩˩!}\hspace{5pt}\peng{/dzi˧dzi˧/ is the name of a tree!}\hspace{5pt}\pcmn{|fv{dzi˧dzi˧}是一种树的名字!}\hspace{5pt}\pfra{/dzi˧dzi˧/, c'est un nom d'arbre! / C'est le nom d'un arbre!}\end{exemple}
\end{entrée}

\begin{entrée}
{dzi˧dzi˧-mo˧˥}{}{ⓔdzi˧dzi˧-mo˧˥}\formedesurface{dzi˧dzi˧mo˧˥}\newline
\classe{名词}\ton{MH\#}\begin{définition}\peng{A species of edible mushroom that grows on fallen trees; it is used as a medicine against stomach-ache.}\end{définition}
\begin{définition}\pcmn{一种可以吃的菌子,长在枯木上}\end{définition}
\begin{définition}\pfra{Champignon comestible, qui ne pousse pas sur la terre, seulement sur les arbres tombés; est utilisé comme médicament pour les maux d'estomac.}\end{définition}
\end{entrée}

\begin{entrée}
{dzi˧ɖæ˧}{}{ⓔdzi˧ɖæ˧}\formedesurface{dzi˧ɖæ˧}\newline
\classe{名词}\ton{M}
\paradigme{\pcmn{:} \p{}}
\begin{définition}\peng{Location.}\end{définition}
\begin{définition}\pcmn{位置、所在地}\end{définition}
\begin{définition}\pfra{Emplacement, localisation (emploi typique: emplacement d'une maison).}\end{définition}
\end{entrée}

\begin{entrée}
{dzi˩ʁo˩}{}{ⓔdzi˩ʁo˩}\formedesurface{dzi˩ʁo˩˥}\newline
\classe{名词}\ton{L}
\paradigme{\pcmn{:} \p{}}
\begin{définition}\peng{Seat, place.}\end{définition}
\begin{définition}\pcmn{座位}\end{définition}
\begin{définition}\pfra{Place, place assise.}\end{définition}
\end{entrée}

\begin{entrée}
{dzi˩tsʰɤ˩}{}{ⓔdzi˩tsʰɤ˩}\formedesurface{dzi˩tsʰɤ˩˥}\newline
\classe{名词}\ton{L}
\paradigme{\pcmn{:} \p{}}
\begin{définition}\peng{A shrub with sharp thorns.}\end{définition}
\begin{définition}\pcmn{永宁的一种灌木}\end{définition}
\begin{définition}\pfra{Arbuste: sorte de houx, de grande taille.}\end{définition}
\end{entrée}

\begin{entrée}
{dzo˥}{}{ⓔdzo˥}\formedesurface{dzo˧}\newline
\classe{名词}\ton{\#H}\begin{définition}\peng{Hail.}\end{définition}
\begin{définition}\pcmn{冰雹}\end{définition}
\begin{définition}\pfra{Grêle.}\end{définition}
\begin{exemple}\pnru{dzo˧ lɑ˩}\hspace{5pt}\peng{there is some hail}\hspace{5pt}\pcmn{下冰雹}\hspace{5pt}\pfra{il grêle}\end{exemple}
\begin{exemple}\pnru{dzo˧ gi˧-ze˩}\hspace{5pt}\peng{there is some hail}\hspace{5pt}\pcmn{下冰雹了}\hspace{5pt}\pfra{il tombe de la grêle}\end{exemple}
\end{entrée}

\begin{entrée}
{dzo˩}{}{ⓔdzo˩}\formedesurface{dzo˧}\newline
\classe{名词}\ton{L}
\paradigme{\pcmn{:} \p{}}
\begin{définition}\peng{Bridge.}\end{définition}
\begin{définition}\pcmn{桥}\end{définition}
\begin{définition}\pfra{Pont.}\end{définition}
\begin{exemple}\pnru{dzo˧ | ɖɯ˧ pɤ˩}\hspace{5pt}\peng{a bridge}\hspace{5pt}\pcmn{一辆桥}\hspace{5pt}\pfra{un pont}\end{exemple}
\begin{exemple}\pnru{dzo˩ bæ˩˥}\hspace{5pt}\peng{to sweep (a/the) bridge}\hspace{5pt}\pcmn{扫桥}\hspace{5pt}\pfra{balayer le pont}\end{exemple}
\begin{exemple}\pnru{njɤ˧ | dzo˩ bæ˩-zo˩-ho˥.}\hspace{5pt}\peng{I have to sweep the bridge.}\hspace{5pt}\pcmn{我要扫桥了。}\hspace{5pt}\pfra{Il va falloir que je balaie le pont.}\end{exemple}
\begin{exemple}\pnru{dzo˩ gv̩˩˥}\hspace{5pt}\peng{to build (/repair) a bridge}\hspace{5pt}\pcmn{建一辆桥}\hspace{5pt}\pfra{construire (ou réparer) un pont}\end{exemple}
\begin{exemple}\pnru{njɤ˧ | dzo˩ gv̩˩-zo˩-ho˥.}\hspace{5pt}\peng{I have to build (/repair) a bridge.}\hspace{5pt}\pfra{Il va falloir que je construise (/répare) un pont.}\end{exemple}
\end{entrée}

\begin{entrée}
{dzo˩˧}{}{ⓔdzo˩˧}\formedesurface{dzo˩˥}\newline
\classe{名词}\ton{LM}
\paradigme{\pcmn{:} \p{}}
\begin{définition}\peng{Lizard.}\end{définition}
\begin{définition}\pcmn{壁虎,蜥蜴,四脚蛇}\end{définition}
\begin{définition}\pfra{Lézard.}\end{définition}
\begin{exemple}\pnru{dzo˩ hwæ˧-ze˧}\hspace{5pt}\peng{…bought (a) lizard}\hspace{5pt}\pcmn{买了壁虎}\hspace{5pt}\pfra{…a acheté (un) lézard}\end{exemple}
\begin{exemple}\pnru{dzo˩ dzɯ˧-ze˩}\hspace{5pt}\peng{…ate (a) lizard}\hspace{5pt}\pcmn{吃了壁虎}\hspace{5pt}\pfra{…a mangé (un) lézard}\end{exemple}
\end{entrée}

\begin{entrée}
{dzo˩∼dzo˧˥}{}{ⓔdzo˩∼dzo˧˥}\formedesurface{dzo˩dzo˧˥}\newline
\classe{动词}\ton{MH}\begin{définition}\peng{To laugh at, to poke fun at, to mock, to ridicule.}\end{définition}
\begin{définition}\pcmn{嘲笑、取笑}\end{définition}
\begin{définition}\pfra{Se moquer de, rire de, ridiculiser.}\end{définition}
\begin{exemple}\pnru{hĩ˧ dzo˧-dzo˥-ho˩-ni˩zo˩!}\hspace{5pt}\peng{It looks like (he/she) is going to poke fun (at…)}\hspace{5pt}\pcmn{他好像要开始取笑人家了!}\hspace{5pt}\pfra{On dirait qu'(elle/il) va tourner quelqu'un en dérision}\end{exemple}
\begin{exemple}\pnru{tʰɑ˧ dzo˩∼dzo˩!}\hspace{5pt}\peng{Don't laugh at people!}\hspace{5pt}\pcmn{别嘲笑(人家)!}\hspace{5pt}\pfra{Ne vous moquez pas!/ Pas de sarcasmes!}\end{exemple}
\end{entrée}

\begin{entrée}
{dzo˧-lv̩˧∼lv̩˥}{}{ⓔdzo˧-lv̩˧∼lv̩˥}\formedesurface{dzo˧lv̩˧lv̩˥}\newline
\classe{名词}\ton{H\#}
\paradigme{\pcmn{:} \p{}}
\begin{définition}\peng{Hailstone.}\end{définition}
\begin{définition}\pcmn{冰雹}\end{définition}
\begin{définition}\pfra{Grêlon.}\end{définition}
\end{entrée}

\begin{entrée}
{dzo˧mi˧}{}{ⓔdzo˧mi˧}\formedesurface{dzo˧mi˧}\newline
\classe{名词}\ton{M}
\paradigme{\pcmn{:} \p{}}
\begin{définition}\peng{Large vat.}\end{définition}
\begin{définition}\pcmn{大桶}\end{définition}
\begin{définition}\pfra{Grande cuve, grand tonneau (sens très proche du précédent, peuvent s'employer de façon interchangeable pour certains des tonneaux).}\end{définition}
\end{entrée}

\begin{entrée}
{dzo˩mi\#˥}{}{ⓔdzo˩mi\#˥}\formedesurface{dzo˩mi˥}\newline
\classe{名词}\ton{LM+\#H}
\paradigme{\pcmn{:} \p{}}
\begin{définition}\peng{Female lizard.}\end{définition}
\begin{définition}\pcmn{母壁虎}\end{définition}
\begin{définition}\pfra{Lézard femelle.}\end{définition}
\begin{exemple}\pnru{dzo˧mi˧-dzo˩pʰv̩˩}\hspace{5pt}\peng{female lizard and male lizard}\hspace{5pt}\pcmn{母壁虎与公壁虎}\hspace{5pt}\pfra{lézard femelle et lézard mâle}\end{exemple}
\end{entrée}

\begin{entrée}
{dzo˩pʰv̩˩}{}{ⓔdzo˩pʰv̩˩}\formedesurface{dzo˩pʰv̩˩˥}\newline
\classe{名词}\ton{L}
\paradigme{\pcmn{:} \p{}}
\begin{définition}\peng{Male lizard.}\end{définition}
\begin{définition}\pcmn{公壁虎}\end{définition}
\begin{définition}\pfra{Lézard mâle.}\end{définition}
\begin{exemple}\pnru{dzo˩pʰv̩˩-dzo˩mi˥}\hspace{5pt}\peng{male lizard and female lizard}\hspace{5pt}\pcmn{公壁虎与母壁虎}\hspace{5pt}\pfra{lézard mâle et lézard femelle}\end{exemple}
\end{entrée}

\begin{entrée}
{dzo˧zo\#˥}{}{ⓔdzo˧zo\#˥}\formedesurface{dzo˧zo˧}\newline
\classe{名词}\ton{\#H}
\paradigme{\pcmn{:} \p{}}
\begin{définition}\peng{Small vat.}\end{définition}
\begin{définition}\pcmn{小桶}\end{définition}
\begin{définition}\pfra{Petite cuve.}\end{définition}
\end{entrée}

\begin{entrée}
{dzo˩zo\#˥}{}{ⓔdzo˩zo\#˥}\formedesurface{dzo˩zo˥}\newline
\classe{名词}\ton{LM+\#H}
\paradigme{\pcmn{:} \p{}}
\begin{définition}\peng{Baby lizard.}\end{définition}
\begin{définition}\pcmn{小壁虎}\end{définition}
\begin{définition}\pfra{Bébé lézard.}\end{définition}
\end{entrée}

\begin{entrée}
{dzɯ˥}{}{ⓔdzɯ˥}\formedesurface{dzɯ˧}\newline
\classe{动词}\ton{H}\begin{définition}\peng{To eat.}\end{définition}
\begin{définition}\pcmn{吃}\end{définition}
\begin{définition}\pfra{Manger.}\end{définition}
\begin{exemple}\pnru{le˧-dzɯ˥}\hspace{5pt}\peng{|fg{accomp}}\hspace{5pt}\pcmn{实施}\hspace{5pt}\pfra{|fg{accomp}}\end{exemple}
\begin{exemple}\pnru{hɑ˧ dzɯ˧}\hspace{5pt}\peng{to have a meal; to take some food}\hspace{5pt}\pcmn{吃饭}\hspace{5pt}\pfra{prendre un repas, manger de la nourriture}\end{exemple}
\begin{exemple}\pnru{njɤ˧ | hɑ˧ le˧-dzɯ˥-ze˩}\hspace{5pt}\peng{I have eaten. / I have had a meal.}\hspace{5pt}\pcmn{我吃过饭了。}\hspace{5pt}\pfra{J'ai mangé. J'ai pris mon repas.}\end{exemple}
\begin{exemple}\pnru{dzɯ˧-di˧˥}\hspace{5pt}\peng{food, thing to eat}\hspace{5pt}\pcmn{吃的(东西)}\hspace{5pt}\pfra{nourriture, chose à manger}\end{exemple}
\begin{exemple}\pnru{dzɯ˧-bi˩-ze˩!}\hspace{5pt}\peng{Let's eat! / It's time to eat!}\hspace{5pt}\pcmn{要吃饭了!}\hspace{5pt}\pfra{(On) va manger! / C'est l'heure de manger!}\end{exemple}
\end{entrée}

\begin{entrée}
{dzɯ˧tsɯ˧˥}{}{ⓔdzɯ˧tsɯ˧˥}\formedesurface{dzɯ˧tsɯ˧˥}\newline
\classe{名词}\ton{MH\#}
\paradigme{\pcmn{:} \p{}}
\begin{définition}\peng{A shrub that grows in Yongning.}\end{définition}
\begin{définition}\pcmn{永宁的一种灌木}\end{définition}
\begin{définition}\pfra{Sorte de houx, petites feuilles à piquants.}\end{définition}
\end{entrée}

\begin{entrée}
{dʑɤ˩β}{}{ⓔdʑɤ˩β}\formedesurface{dʑɤ˩˥}\newline
\classe{形容词}\ton{Lβ}\begin{définition}\peng{Good (good decision).}\end{définition}
\begin{définition}\pcmn{好}\end{définition}
\begin{définition}\pfra{Bon (bonne décision).}\end{définition}
\begin{exemple}\pnru{mɤ˧-dʑɤ˩}\hspace{5pt}\peng{bad}\hspace{5pt}\pcmn{坏}\hspace{5pt}\pfra{pas bien, mauvais}\end{exemple}
\begin{exemple}\pnru{dʑɤ˩-hĩ˥}\hspace{5pt}\peng{|fg{rel}}\hspace{5pt}\pcmn{|fg{rel}}\hspace{5pt}\pfra{|fg{rel}}\end{exemple}
\begin{exemple}\pnru{(no˧) ɖwæ˧˥ | dʑɤ˩˥!}\hspace{5pt}\peng{You're great!}\hspace{5pt}\pcmn{你很好!}\hspace{5pt}\pfra{Tu es quelqu'un de bien!}\end{exemple}
\begin{exemple}\pnru{dʑɤ˩-kʰɯ˥!}\hspace{5pt}\peng{A benediction used on the New Year: “Let there be good (things)!", i.e. “Prosperity!", “All the best for the New Year!"}\hspace{5pt}\pcmn{新年祝福:“祝一切好! / 万事如意!”}\hspace{5pt}\pfra{bénédiction dite au Nouvel An: «Bonnes (choses)!», «Prospérité!», «Bonne année!»}\end{exemple}
\begin{exemple}\pnru{no˧ | le˧-ʝi˥ | dʑɤ˩˥, | hĩ˧-ɳɯ˩ | do˩˥! | ʈʂʰɯ˧ | le˧-ʝi˥ | mɤ˧-dʑɤ˩, | hĩ˧-ɳɯ˩ | ʐwɤ˩˥!}\hspace{5pt}\peng{“If you do well, people see it / people realize so! If (s)he does badly, people say so!" = “A job well done earns recognition; a job badly done earns criticism!" (Context: talking about a bad book. In the Na world view as remembered by the consultant, there is no question that it is better to do good, that good deeds and good attitudes eventually get rewarded, and bad deeds and bad attitudes eventually get punished.}\hspace{5pt}\pcmn{你做得好,人家(会)发现!他做的不好,人家(会)说(他)!}\hspace{5pt}\pfra{«Si tu travailles bien, les gens s'en rendent compte! S'il travail mal, les gens le disent!» = «Le bon travail est reconnu; le mauvais travail reçoit des critiques!» (Contexte: commentaire au sujet d'un mauvais livre. C'est ce qu'on disait autrefois: la belle ouvrage est reconnue; le mauvais travail vous attire des critiques.)}\end{exemple}
\end{entrée}

\begin{entrée}
{dʑɤ˧bo˩}{}{ⓔdʑɤ˧bo˩}\formedesurface{dʑɤ˧bo˩}\newline
\classe{名词}\ton{L\#}
\paradigme{\pcmn{:} \p{}}
\begin{définition}\peng{Granary (a special building).}\end{définition}
\begin{définition}\pcmn{粮仓}\end{définition}
\begin{définition}\pfra{Grenier à céréale; selon M23, n'existe plus: était, dans le temps, un bâtiment exclusivement consacré à la conservation des céréales: un grenier.}\end{définition}
\end{entrée}

\begin{entrée}
{dʑɤ˩bv̩˥}{}{ⓔdʑɤ˩bv̩˥}\formedesurface{dʑɤ˩bv̩˥}\newline
\classe{动词}\ton{LH}\begin{définition}\peng{To play.}\end{définition}
\begin{définition}\pcmn{玩,玩耍}\end{définition}
\begin{définition}\pfra{Jouer.}\end{définition}
\begin{exemple}\pnru{dʑɤ˩bv̩˥ -bi˩/-ze˩}\hspace{5pt}\peng{\_ |fg{fut.imm}/|fg{pfv}}\hspace{5pt}\pcmn{要玩耍 / 玩耍了}\hspace{5pt}\pfra{\_ |fg{fut.imm}/|fg{pfv}}\end{exemple}
\begin{exemple}\pnru{le˧-dʑɤ˩bv̩˩ +ze˩}\hspace{5pt}\peng{|fg{accomp} \_ |fg{pfv}}\hspace{5pt}\pcmn{玩耍了}\hspace{5pt}\pfra{|fg{accomp} \_ |fg{pfv}}\end{exemple}
\begin{exemple}\pnru{tʰi˧-dʑɤ˩bv̩˩, | tʰi˧-dʑɤ˩bv̩˩, | le˧-fv̩˧!}\hspace{5pt}\peng{They play, they play… they're happy! / (By) playing on and on, they get really happy! (About children playing together)}\hspace{5pt}\pcmn{他们玩着玩着,很高兴!(情景:几个小孩子一起玩)}\hspace{5pt}\pfra{ils jouent, ils jouent… ils sont contents! (Au sujet d'enfants qui jouent ensemble)}\end{exemple}
\end{entrée}

\begin{entrée}
{dʑɤ˩bv̩˥-di˩}{}{ⓔdʑɤ˩bv̩˥-di˩}\formedesurface{dʑɤ˩bv̩˥di˩}\newline
\classe{名词}\ton{LH-}\begin{définition}\peng{Toy.}\end{définition}
\begin{définition}\pcmn{玩具}\end{définition}
\begin{définition}\pfra{Jouet.}\end{définition}
\end{entrée}

\begin{entrée}
{dʑɤ˩bv̩˧kɤ˧-sɑ˥ʁwɤ˩}{}{ⓔdʑɤ˩bv̩˧kɤ˧-sɑ˥ʁwɤ˩}\formedesurface{dʑɤ˩bv̩˧kɤ˧sɑ˥ʁwɤ˩}\newline
\classe{名词}\ton{LM+MH\#-}\begin{définition}\peng{Gaoming, a village north-east of Yongning).}\end{définition}
\begin{définition}\pcmn{高明 (摩梭话名称的汉译:佳部嘎萨瓦、也称作嘎撒瓦)(永宁坝子的一个村落)}\end{définition}
\begin{définition}\pfra{Gaoming, un village au nord-est de Yongning.}\end{définition}
\begin{exemple}\pnru{dʑɤ˩bv̩˧kɤ˧-sɑ˥ʁwɤ˩, | hi˩ʁwɤ˩-lo˥, | æ˩mi˧-ʁwɤ\#˥, | lɑ˧lo˧-ʁwɤ˥, | lɑ˧ŋwɤ˧, | bɤ˧tsʰo˧gv̩˥, | ə˧lɑ˧-ʁwɤ\#˥, | gæ˧ɻæ˩, | qʰæ˧tɕʰi˧, | tʰo˧ʈɯ\#˥}\hspace{5pt}\peng{The ten Na villages considered in traditional geography as belonging to the vicinity of the Yongning temple.}\hspace{5pt}\pcmn{永宁摩梭地理概念中,距离扎美寺最近的十个村落:佳部嘎萨瓦、习瓦洛、阿咪瓦、拉洛瓦、拉瓦、巴搓古、阿拉瓦、嘎尔、开基、拖支。}\hspace{5pt}\pfra{Les dix villages na traditionnellement considérés comme appartenant au voisinage du temple de Yongning.}\end{exemple}
\end{entrée}

\begin{entrée}
{dʑɤ˩bv̩˥-ʁwɤ˩}{}{ⓔdʑɤ˩bv̩˥-ʁwɤ˩}\formedesurface{dʑɤ˩bv̩˥ʁwɤ˩}\newline
\classe{名词}\ton{LH-}\begin{définition}\peng{Jiabowa (name of a village).}\end{définition}
\begin{définition}\pcmn{甲波瓦(永宁坝子的一个村落)}\end{définition}
\begin{définition}\pfra{Jiabowa (nom de village).}\end{définition}
\end{entrée}

\begin{entrée}
{dʑɤ˩ɕjɤ˩}{}{ⓔdʑɤ˩ɕjɤ˩}\formedesurface{dʑɤ˩ɕjɤ˩˥}\newline
\classe{名词}\ton{L}\begin{définition}\peng{Shoe-pad; insole.}\end{définition}
\begin{définition}\pcmn{鞋垫}\end{définition}
\begin{définition}\pfra{Semelle (en paille); chausson (en paille).}\end{définition}
\end{entrée}

\begin{entrée}
{dʑɤ˧do˩}{}{ⓔdʑɤ˧do˩}\formedesurface{dʑɤ˧do˩}\newline
\classe{名词}\ton{L\#}\begin{définition}\peng{Zhongdian (place name).}\end{définition}
\begin{définition}\pcmn{中甸}\end{définition}
\begin{définition}\pfra{Zhongdian (nom de lieu).}\end{définition}
\begin{exemple}\pnru{dʑɤ˧do˩-bɤ˩}\hspace{5pt}\peng{the Pumi people of Zhongdian}\hspace{5pt}\pcmn{中甸普米族}\hspace{5pt}\pfra{les Pumi de Zhongdian}\end{exemple}
\end{entrée}

\begin{entrée}
{dʑɤ˩kʰwɤ˧}{}{ⓔdʑɤ˩kʰwɤ˧}\formedesurface{dʑɤ˩kʰwɤ˥}\newline
\classe{名词}\ton{LM}\begin{définition}\peng{Distance.}\end{définition}
\begin{définition}\pcmn{距离}\end{définition}
\begin{définition}\pfra{Distance.}\end{définition}
\begin{exemple}\pnru{no˧ | ʈʂʰɯ˧ | ə˩-ʐæ˥ʂæ˩? | dʑɤ˩kʰwɤ˧ ə˩-di˩? | - dʑɤ˩˥ | dʑɤ˩kʰwɤ˧ mɤ˧-di˥! | mɤ˧-ʐæ˩ʂæ˩!}\hspace{5pt}\peng{Are you distant from him? Is there distance (between you)? - There is not much distance to speak of! We are not distant! (=we are close friends)}\hspace{5pt}\pcmn{你们很熟吗?/ 你们很亲吗? - 不很熟!/ 不很亲!}\hspace{5pt}\pfra{tu es loin de lui? Y a-t-il de la distance entre vous? (=vous êtes proches/intimes, ou pas?) - Non, il n'y a guère de distance! Nous ne somme pas éloignés!}\end{exemple}
\end{entrée}

\begin{entrée}
{dʑɤ˩pi\#˥}{}{ⓔdʑɤ˩pi\#˥}\formedesurface{dʑɤ˩pi˥}\newline
\classe{形容词}\ton{LM+\#H}\begin{définition}\peng{Plenty of.}\end{définition}
\begin{définition}\pcmn{多}\end{définition}
\begin{définition}\pfra{Beaucoup.}\end{définition}
\begin{exemple}\pnru{dʑɤ˩pi˧ ʝi˧}\hspace{5pt}\peng{It's very useful!}\hspace{5pt}\pcmn{很有用、有很多用处}\hspace{5pt}\pfra{C'est très utile}\end{exemple}
\begin{exemple}\pnru{dʑɤ˩pi˧ dʑo˧!}\hspace{5pt}\peng{(I) have plenty! / (I) have a lot! (possession)}\hspace{5pt}\pcmn{我有很多!}\hspace{5pt}\pfra{J'en ai beaucoup! (possession)}\end{exemple}
\begin{exemple}\pnru{dʑɤ˩pi˧ dʑo˧˥}\hspace{5pt}\peng{There is plenty / there is a lot. (Note: existence, and not possession)}\hspace{5pt}\pcmn{有很多。}\hspace{5pt}\pfra{Il en existe beaucoup, il y en a beaucoup. (Note: existentiel, et non possession)}\end{exemple}
\end{entrée}

\begin{entrée}
{dʑɤ˩qʰɑ˥}{}{ⓔdʑɤ˩qʰɑ˥}\formedesurface{dʑɤ˩qʰɑ˩}\newline
\classe{助词}\ton{LH}\begin{définition}\peng{Continuously, with full might.}\end{définition}
\begin{définition}\pcmn{一直、一个劲地}\end{définition}
\begin{définition}\pfra{Continuellement, par un effort soutenu.}\end{définition}
\begin{exemple}\pnru{dʑɤ˩qʰɑ˥ ʈɤ˩}\hspace{5pt}\peng{to pull with full might}\hspace{5pt}\pcmn{一个劲地拉}\hspace{5pt}\pfra{tirer en un effort continu}\end{exemple}
\begin{exemple}\pnru{dʑɤ˩qʰɑ˥ mi˩}\hspace{5pt}\peng{to push with full might}\hspace{5pt}\pcmn{一个劲地推}\hspace{5pt}\pfra{appuyer en un effort continu}\end{exemple}
\begin{exemple}\pnru{dʑɤ˩qʰɑ˥ lɑ˩}\hspace{5pt}\peng{to beat with full might}\hspace{5pt}\pcmn{一个劲地打}\hspace{5pt}\pfra{frapper continuellement, en un effort continu}\end{exemple}
\end{entrée}

\begin{entrée}
{dʑɤ˩so˥}{}{ⓔdʑɤ˩so˥}\formedesurface{dʑɤ˩so˥}\newline
\classe{形容词}\ton{LH}\begin{définition}\peng{Many, a great number of.}\end{définition}
\begin{définition}\pcmn{好几(个)}\end{définition}
\begin{définition}\pfra{En abondance, beaucoup.}\end{définition}
\begin{exemple}\pnru{dʑɤ˩-so˥ ɲi˩}\hspace{5pt}\peng{many days; a long time}\hspace{5pt}\pcmn{好几天}\hspace{5pt}\pfra{beaucoup de jours, longtemps}\end{exemple}
\begin{exemple}\pnru{dʑɤ˩-so˥ hɑ̃˩}\hspace{5pt}\peng{many days; a long time}\hspace{5pt}\pcmn{好几天}\hspace{5pt}\pfra{beaucoup de jours, longtemps}\end{exemple}
\begin{exemple}\pnru{dʑɤ˩-so˥ ɭɯ˩}\hspace{5pt}\peng{many...}\hspace{5pt}\pcmn{好几……}\hspace{5pt}\pfra{beaucoup de...}\end{exemple}
\end{entrée}

\begin{entrée}
{dʑɤ˩tsʰi\#˥}{}{ⓔdʑɤ˩tsʰi\#˥}\formedesurface{dʑɤ˩tsʰi˥}\newline
\classe{名词}\ton{LM+\#H}\begin{définition}\peng{Masculine given name.}\end{définition}
\begin{définition}\pcmn{男性名字}\end{définition}
\begin{définition}\pfra{Prénom masculin.}\end{définition}
\end{entrée}

\begin{entrée}
{dʑɤ˩tsʰi˧-ɖɯ˩mɑ˩}{}{ⓔdʑɤ˩tsʰi˧-ɖɯ˩mɑ˩}\formedesurface{dʑɤ˩tsʰi˧ɖɯ˩mɑ˩}\newline
\classe{名词}\ton{LM-L}\begin{définition}\peng{Feminine given name.}\end{définition}
\begin{définition}\pcmn{女性名字}\end{définition}
\begin{définition}\pfra{Prénom féminin.}\end{définition}
\end{entrée}

\begin{entrée}
{dʑɤ˩tsʰi˧-tsi˩mv̩˩}{}{ⓔdʑɤ˩tsʰi˧-tsi˩mv̩˩}\formedesurface{dʑɤ˩tsʰi˧tsi˩mv̩˩}\newline
\classe{名词}\ton{LM-L}
\paradigme{\pcmn{:} \p{}}
\begin{définition}\peng{Prayer flag.}\end{définition}
\begin{définition}\pcmn{经幡、风马旗(挂在家旁边的树上,或房顶上)}\end{définition}
\begin{définition}\pfra{Drapeau de prière; on en attache sur un arbre proche de la maison, ou sur le sommet de la maison.}\end{définition}
\end{entrée}

\begin{entrée}
{dʑɤ˩tsʰo˧}{}{ⓔdʑɤ˩tsʰo˧}\formedesurface{dʑɤ˩tsʰo˥}\newline
\classe{动词}\ton{LM}\begin{définition}\peng{To dance.}\end{définition}
\begin{définition}\pcmn{跳舞}\end{définition}
\begin{définition}\pfra{Danser.}\end{définition}
\begin{exemple}\pnru{ʈʂʰɯ˧ | dʑɤ˩tsʰo˧ mɤ˧-dʑɤ˩!}\hspace{5pt}\peng{(S)he does not dance well!}\hspace{5pt}\pcmn{他舞跳得不好!}\hspace{5pt}\pfra{il/elle ne danse pas bien!}\end{exemple}
\begin{exemple}\pnru{dʑɤ˩tsʰo˧ | ɖɯ˧-hɑ̃˧ tsʰo˧}\hspace{5pt}\peng{to dance all evening, to dance a whole night}\hspace{5pt}\pcmn{跳一整夜舞}\hspace{5pt}\pfra{danser (toute) une soirée}\end{exemple}
\begin{exemple}\pnru{ʑi˧qʰwɤ˧-ʂɯ˧-qo˧ | dʑɤ˩tsʰo˧ ʁɑ˧ ʂe˩}\hspace{5pt}\peng{to throw a housewarming party in a newly built house}\hspace{5pt}\pcmn{在新房子举办乔迁宴会}\hspace{5pt}\pfra{organiser une pendaison de crémaillère dans une nouvelle maison}\end{exemple}
\end{entrée}

\begin{entrée}
{dʑi˥}{₁}{ⓔdʑi˥ⓗ1}\formedesurface{dʑi˧}\newline
\classe{名词}\ton{\#H}
1\begin{définition}\peng{Urine.}\end{définition}
\begin{définition}\pcmn{尿}\end{définition}
\begin{définition}\pfra{Urine.}\end{définition}
\begin{exemple}\pnru{dʑi˧ bæ˥}\hspace{5pt}\peng{to sweep urine}\hspace{5pt}\pcmn{扫尿}\hspace{5pt}\pfra{balayer l'urine}\end{exemple}
\begin{exemple}\pnru{dʑi˧-lɑ˩ | qʰæ˧}\hspace{5pt}\peng{excrements: urine and faeces}\hspace{5pt}\pcmn{大小便的统称}\hspace{5pt}\pfra{excréments: urine et fèces}\end{exemple}
\end{entrée}

\begin{entrée}
{dʑi˥}{₂}{ⓔdʑi˥ⓗ2}\formedesurface{dʑi˧}\newline
\classe{名词}\ton{H}
2
\paradigme{\pcmn{:} \p{}}
\begin{définition}\peng{Clothes, clothing (monosyllabic form).}\end{définition}
\begin{définition}\pcmn{衣服}\end{définition}
\begin{définition}\pfra{Habit, vêtement (mot monosyllabique).}\end{définition}
\begin{exemple}\pnru{kʰv̩˧ʂɯ˥, | dʑi˧ qæ˧!}\hspace{5pt}\peng{On New Year's Eve, one changes one's clothing / one wears new clothes!}\hspace{5pt}\pcmn{过年,换衣服! / 过年,要穿新衣服!}\hspace{5pt}\pfra{Au Nouvel An, on change de vêtements/ on porte des vêtements neufs!}\end{exemple}
\begin{exemple}\pnru{dʑi˧ qæ˧-ze˩!}\hspace{5pt}\peng{(He/she) has changed clothes!}\hspace{5pt}\pcmn{换衣服了!}\hspace{5pt}\pfra{(il/elle) a changé de vêtements!}\end{exemple}
\begin{exemple}\pnru{nɑ˩-dʑi\#˥}\hspace{5pt}\peng{Na clothing}\hspace{5pt}\pcmn{摩梭服装}\hspace{5pt}\pfra{le vêtement des Na}\end{exemple}
\begin{exemple}\pnru{hæ˧-dʑi\#˥}\hspace{5pt}\peng{Chinese (Han) clothing}\hspace{5pt}\pcmn{汉族服装}\hspace{5pt}\pfra{le vêtement des Chinois (Han)}\end{exemple}
\end{entrée}

\begin{entrée}
{dʑi˧hṽ̩˥\$}{}{ⓔdʑi˧hṽ̩˥\$}\formedesurface{dʑi˧hṽ̩˥}\newline
\classe{名词}\ton{H\$}
\paradigme{\pcmn{:} \p{}}
\begin{définition}\peng{Clothes, clothing.}\end{définition}
\begin{définition}\pcmn{衣服}\end{définition}
\begin{définition}\pfra{Habit, vêtement (terme générique).}\end{définition}
\end{entrée}

\begin{entrée}
{dʑi˧mi\#˥}{}{ⓔdʑi˧mi\#˥}\formedesurface{dʑi˧mi˧}\newline
\classe{名词}\ton{\#H}
\paradigme{\pcmn{:} \p{}}
\begin{définition}\peng{Female water buffalo.}\end{définition}
\begin{définition}\pcmn{母水牛}\end{définition}
\begin{définition}\pfra{Buffle (femelle).}\end{définition}
\begin{exemple}\pnru{dʑi˧mi˧ tʰv̩˧-pʰo˩}\hspace{5pt}\peng{|fg{n}+|fg{dem}+|fg{clf}}\hspace{5pt}\pcmn{这头母水牛}\hspace{5pt}\pfra{|fg{n}+|fg{dem}+|fg{clf}}\end{exemple}
\begin{exemple}\pnru{dʑi˧mi˧-dʑi˧zo\#˥ / dʑi˧mi˧-dʑi˥zo˩}\hspace{5pt}\peng{female water buffalo and male water buffalo}\hspace{5pt}\pcmn{母水牛与公水牛}\hspace{5pt}\pfra{buffles femelle et mâle}\end{exemple}
\end{entrée}

\begin{entrée}
{dʑi˧tɕʰi˧}{}{ⓔdʑi˧tɕʰi˧}\formedesurface{dʑi˧tɕʰi˧}\newline
\classe{名词}\ton{M}
\paradigme{\pcmn{:} \p{}}
\begin{définition}\peng{Trip.}\end{définition}
\begin{définition}\pcmn{旅途}\end{définition}
\begin{définition}\pfra{Trajet, périple, voyage, déplacement.}\end{définition}
\begin{exemple}\pnru{dʑi˧tɕʰi˧ bi˧}\hspace{5pt}\peng{to travel}\hspace{5pt}\pcmn{旅游、出差}\hspace{5pt}\pfra{voyager}\end{exemple}
\begin{exemple}\pnru{dʑi˧tɕʰi˧ gwɤ˥}\hspace{5pt}\peng{to travel, to make a (long) trip}\hspace{5pt}\pcmn{旅游、出差}\hspace{5pt}\pfra{faire un grand voyage, faire de lointains périples}\end{exemple}
\begin{exemple}\pnru{dʑi˧tɕʰi˧ | ɖɯ˧-ʂɯ˩ gwɤ˩-bi˩}\hspace{5pt}\peng{to travel around}\hspace{5pt}\pcmn{旅游}\hspace{5pt}\pfra{faire un tour}\end{exemple}
\end{entrée}

\begin{entrée}
{dʑi˩wɤ˩}{}{ⓔdʑi˩wɤ˩}\formedesurface{dʑi˩wɤ˩˥}\newline
\classe{名词}\ton{L}
\paradigme{\pcmn{:} \p{}}
\begin{définition}\peng{Stirrup.}\end{définition}
\begin{définition}\pcmn{马镫}\end{définition}
\begin{définition}\pfra{Étriers.}\end{définition}
\end{entrée}

\begin{entrée}
{dʑi˧zo\#˥}{}{ⓔdʑi˧zo\#˥}\formedesurface{dʑi˧zo˧}\newline
\classe{名词}\ton{\#H}
\paradigme{\pcmn{:} \p{}}
\begin{définition}\peng{Male baby buffalo.}\end{définition}
\begin{définition}\pcmn{小水牛(水牛崽子),一般指公的小水牛}\end{définition}
\begin{définition}\pfra{Buffle (enfant mâle).}\end{définition}
\begin{exemple}\pnru{dʑi˧zo˧ tʰv̩˧-ɭɯ\#˥}\hspace{5pt}\peng{|fg{n}+|fg{dem}+|fg{clf}}\hspace{5pt}\pcmn{这只水牛崽子}\hspace{5pt}\pfra{|fg{n}+|fg{dem}+|fg{clf}}\end{exemple}
\end{entrée}

\begin{entrée}
{‑dʑo˥}{}{ⓔ‑dʑo˥}\formedesurface{dʑo˧}\newline
\classe{后缀}\ton{H}\begin{définition}\peng{Topic marker.}\end{définition}
\begin{définition}\pcmn{主题}\end{définition}
\begin{définition}\pfra{Marqueur de topic.}\end{définition}
\end{entrée}

\begin{entrée}
{‑dʑo˧}{}{ⓔ‑dʑo˧}\formedesurface{dʑo˧}\newline
\classe{后缀}\ton{M}\begin{définition}\peng{Progressive aspect.}\end{définition}
\begin{définition}\pcmn{进行式}\end{définition}
\begin{définition}\pfra{Aspect progressif.}\end{définition}
\end{entrée}

\begin{entrée}
{dʑo˧β}{}{ⓔdʑo˧β}\formedesurface{dʑo˧}\newline
\classe{动词}\ton{Mβ}\begin{définition}\peng{To possess.}\end{définition}
\begin{définition}\pcmn{有,拥有}\end{définition}
\begin{définition}\pfra{Posséder; y avoir; avoir un objet, avoir une pensée…; autrefois, il y avait une mère et sa fille…}\end{définition}
\begin{exemple}\pnru{mɤ˧-dʑo˧-ze˧!}\hspace{5pt}\peng{There isn't any left!}\hspace{5pt}\pcmn{没有了!}\hspace{5pt}\pfra{Il n'y en a plus!}\end{exemple}
\begin{exemple}\pnru{le˧-dʑo˧-ze˧!}\hspace{5pt}\peng{There is some, now!}\hspace{5pt}\pcmn{有了!}\hspace{5pt}\pfra{ça y est, il y en a!}\end{exemple}
\begin{exemple}\pnru{ʈʂʰɯ˧ | ɑ˩ʁo˧ | ɖɯ˧-sɑ˥ | mɤ˧-dʑo˧!}\hspace{5pt}\peng{At his home, they have nothing at all = he is needy}\hspace{5pt}\pcmn{他家什么也没有 = 他家贫穷}\hspace{5pt}\pfra{Il n'y a rien chez lui = sa maison est dans l'indigence}\end{exemple}
\begin{exemple}\pnru{njɤ˧ | mv̩˩zɯ˩-ni˥mi˩ | ŋi˧-kv̩˧ dʑo˧˥!}\hspace{5pt}\peng{I have two siblings!}\hspace{5pt}\pcmn{我有两个兄弟姐妹!}\hspace{5pt}\pfra{j’ai deux frères et sœurs!}\end{exemple}
\begin{exemple}\pnru{dʑo˧-tʰɑ˧˥!}\hspace{5pt}\peng{There could be some! / It's possible that there will be some!}\hspace{5pt}\pcmn{会有的!}\hspace{5pt}\pfra{Cela arrive / il peut y en avoir / c'est possible qu'il y en ait!}\end{exemple}
\begin{exemple}\pnru{tso˧∼tso˧ dʑo˧}\hspace{5pt}\peng{he has some things}\hspace{5pt}\pcmn{他有东西}\hspace{5pt}\pfra{il y a des choses}\end{exemple}
\begin{exemple}\pnru{njɤ˧-ɻ̍˩, | ɖɯ˧-ɭɯ˧-lɑ˧ dʑo˥!}\hspace{5pt}\peng{We only have one (child)!}\hspace{5pt}\pcmn{我们只有一个(孩子)!}\hspace{5pt}\pfra{Nous, on n'en a qu'un(, d'enfant)!}\end{exemple}
\begin{exemple}\pnru{ɖwæ˧˥ | dʑo˧-ɲi˥!}\hspace{5pt}\peng{There are lots! (For instance, when preparing to build a house, one needs large quantities of lumber; someone may comment: “There are lots!")}\hspace{5pt}\pcmn{有很多!(如:准备建房,积累的木材有很多)}\hspace{5pt}\pfra{Il y en a des quantités! (ex.: au sujet du bois de construction qu'on prépare en vue de la construction d'une maison)}\end{exemple}
\end{entrée}

\begin{entrée}
{dʑo˩β}{}{ⓔdʑo˩β}\formedesurface{dʑo˩˥}\newline
\classe{动词}\ton{Lβ}\begin{définition}\peng{Existential: there is someone in the toilets/at home; there are n people in the family.}\end{définition}
\begin{définition}\pcmn{存在动词:有,存在着。如:某人在家或家里有几口人}\end{définition}
\begin{définition}\pfra{Existentiel pour les êtres animés (dont les personnes).}\end{définition}
\begin{exemple}\pnru{mɤ˧-dʑo˩}\hspace{5pt}\peng{|fg{neg}}\hspace{5pt}\pcmn{没有、不在}\hspace{5pt}\pfra{|fg{neg}}\end{exemple}
\begin{exemple}\pnru{ʈʂʰɯ˧ | ɑ˩ʁo˧ mɤ˧-dʑo˩!}\hspace{5pt}\peng{(S)he is not at home!}\hspace{5pt}\pcmn{他不在家!}\hspace{5pt}\pfra{Il/elle n'est pas à la maison!}\end{exemple}
\end{entrée}

\begin{entrée}
{dʑɯ˧}{}{ⓔdʑɯ˧}\formedesurface{dʑɯ˧}\newline
\classe{名词}\ton{M}\begin{définition}\peng{Moment, time (of a certain event).}\end{définition}
\begin{définition}\pcmn{……的时间}\end{définition}
\begin{définition}\pfra{Le moment (de), l'heure (de).}\end{définition}
\begin{exemple}\pnru{ʈʂʰwɤ˩ dzɯ˩-bi˩-dʑɯ˩˥}\hspace{5pt}\peng{dinner-time}\hspace{5pt}\pcmn{吃晚餐的时间}\hspace{5pt}\pfra{l'heure du repas du soir}\end{exemple}
\begin{exemple}\pnru{ɑ˩pʰo˩ bi˩-dʑɯ˩˥}\hspace{5pt}\peng{time to go out; the right time to go outside}\hspace{5pt}\pcmn{出去的(合适)时间}\hspace{5pt}\pfra{l'heure d'aller dehors, le moment d'aller dehors}\end{exemple}
\begin{exemple}\pnru{le˧-ʑi˧-bi˧-dʑɯ˧ tʰv̩˧-ze˩!}\hspace{5pt}\peng{It is time to go to sleep!}\hspace{5pt}\pcmn{睡觉的时间到了!}\hspace{5pt}\pfra{Il est l'heure d'aller dormir!}\end{exemple}
\begin{exemple}\pnru{ʐo˩ dzɯ˩-bi˩-dʑɯ˩˥}\hspace{5pt}\peng{lunch-time}\hspace{5pt}\pcmn{午饭的时间}\hspace{5pt}\pfra{l'heure du déjeuner}\end{exemple}
\end{entrée}

\begin{entrée}
{dʑɯ˧˥}{}{ⓔdʑɯ˧˥}\formedesurface{dʑɯ˧˥}\newline
\classe{形容词}\ton{MH}\begin{définition}\peng{Many, much.}\end{définition}
\begin{définition}\pcmn{多}\end{définition}
\begin{définition}\pfra{Nombreux, beaucoup (dénombrable).}\end{définition}
\begin{exemple}\pnru{hĩ˧ dʑɯ˩}\hspace{5pt}\peng{There are many people.}\hspace{5pt}\pcmn{人多。}\hspace{5pt}\pfra{les gens sont nombreux, il y a beaucoup de monde}\end{exemple}
\end{entrée}

\begin{entrée}
{‑dʑɯ˧α}{}{ⓔ‑dʑɯ˧α}\formedesurface{dʑɯ˧}\newline
\classe{后缀}\ton{M}\begin{définition}\peng{|fg{experiential.}}\end{définition}
\begin{définition}\pcmn{……过}\end{définition}
\begin{définition}\pfra{|fg{expérientiel.}}\end{définition}
\end{entrée}

\begin{entrée}
{dʑɯ˩}{}{ⓔdʑɯ˩}\formedesurface{dʑɯ˧}\newline
\classe{名词}
\sens{1}\paradigme{\pcmn{:} \p{}}
\begin{définition}\peng{Water.}\end{définition}
\begin{définition}\pcmn{水}\end{définition}
\begin{définition}\pfra{Eau.}\end{définition}
\begin{exemple}\pnru{dʑɯ˧ ʈʰɯ˧}\hspace{5pt}\peng{to drink water}\hspace{5pt}\pcmn{喝水}\hspace{5pt}\pfra{boire de l'eau}\end{exemple}
\begin{exemple}\pnru{ʈʂʰɯ˧ dʑɯ˧ ʈʰɯ˧-dʑo˧!}\hspace{5pt}\peng{(S)he is drinking water}\hspace{5pt}\pcmn{他在喝水}\hspace{5pt}\pfra{il est en train de boire de l'eau!}\end{exemple}
\begin{exemple}\pnru{dʑɯ˧ | ɖɯ˧-ʈʰɤ˧ ʈʰɯ˧˥}\hspace{5pt}\peng{to drink a little water (literally ‘a drop of water')}\hspace{5pt}\pcmn{喝一点水(直译:‘一滴水’)}\hspace{5pt}\pfra{boire un peu d'eau (littéralement «une goutte d'eau»)}\end{exemple}
\begin{exemple}\pnru{dʑɯ˩ kʰɯ˩˥}\hspace{5pt}\peng{to put water}\hspace{5pt}\pcmn{放水}\hspace{5pt}\pfra{mettre de l'eau}\end{exemple}
\begin{exemple}\pnru{dʑɯ˩ mæ˩˥}\hspace{5pt}\peng{to irrigate, to water}\hspace{5pt}\pcmn{浇灌、灌溉}\hspace{5pt}\pfra{irriguer, arroser, mettre de l’eau}\end{exemple}
\begin{exemple}\pnru{dʑɯ˩ qæ˩, | hɑ˩ qæ˩˥ |}\hspace{5pt}\peng{a description of the traveller's changes in environment: ‘to change water, to change food'. This requires using strategies to avoid ailments: in particular, it was customary to boil in water a little earth of the place where one had arrived, and to drink this preparation.}\hspace{5pt}\pcmn{‘换水换土’:这个短语描述旅人到他人乡的情况,带来水土不服的危险。为了预防这类不良反应,摩梭旅人习惯水煮一点当地的土,喝下去,为了适应当地的水土。}\hspace{5pt}\pfra{description du dépaysement que connaît le voyageur qui arrive en pays étranger et doit ‘changer d'eau, changer de nourriture'. Ce dépaysement commande des stratégies de prévention de soucis de santé: en particulier, il était usuel de faire bouillir un peu de terre locale dans de l'eau, et de boire cette préparation de façon à s'accoutumer.}\end{exemple}
\begin{exemple}\pnru{dʑɯ˧ | mv̩˩tɕo˧ dɑ˧˥}\hspace{5pt}\peng{the water flows downwards}\hspace{5pt}\pcmn{水往下流}\hspace{5pt}\pfra{l'eau coule vers le bas}\end{exemple}\sens{2}
\begin{définition}\peng{River, waterway.}\end{définition}
\begin{définition}\pcmn{河流}\end{définition}
\begin{définition}\pfra{Rivière.}\end{définition}
\end{entrée}

\begin{entrée}
{dʑɯ˩α}{}{ⓔdʑɯ˩α}\formedesurface{dʑɯ˩˥}\newline
\classe{动词}\ton{Lα}\begin{définition}\peng{To twist (strings) together (to make a rope).}\end{définition}
\begin{définition}\pcmn{搓(搓绳子)}\end{définition}
\begin{définition}\pfra{Rouler, tordre (par exemple: rouler des brins, pour en faire une corde; ne s'emploie pas pour les fibres très fines, pour lesquelles on dit: \stylefv{/ʈʂwæ}˧˥/).}\end{définition}
\begin{exemple}\pnru{le˧-dʑɯ˩-ze˩}\hspace{5pt}\peng{|fg{accomp} \_ |fg{pfv}}\hspace{5pt}\pcmn{搓了}\hspace{5pt}\pfra{|fg{accomp} \_ |fg{pfv}}\end{exemple}
\begin{exemple}\pnru{bæ˩ dʑɯ˩˥}\hspace{5pt}\peng{to twist (strings into) a rope}\hspace{5pt}\pcmn{搓绳子}\hspace{5pt}\pfra{tordre une corde}\end{exemple}
\begin{exemple}\pnru{qʰv̩˩ɖʐæ˩ dʑɯ˥}\hspace{5pt}\peng{to twist a string, a small rope}\hspace{5pt}\pcmn{搓一根小绳子}\hspace{5pt}\pfra{faire une ficelle, une petite cordelette}\end{exemple}
\begin{exemple}\pnru{ɖɯ˧-kʰwɤ˧ dʑɯ˥}\hspace{5pt}\peng{to twist a little}\hspace{5pt}\pcmn{搓一下}\hspace{5pt}\pfra{tordre un peu / tordre quelque chose}\end{exemple}
\end{entrée}

\begin{entrée}
{dʑɯ˩-æ̃˩tsɯ˧}{}{ⓔdʑɯ˩-æ̃˩tsɯ˧}\formedesurface{dʑɯ˩æ̃˩tsɯ˥}\newline
\classe{名词}\ton{L-LM}
\paradigme{\pcmn{:} \p{}}
\begin{définition}\peng{Water fowl: used as a cover term for a variety of birds including sandpiper (|\stylefi{Calidris}), avocet, Baillon's crake, and blue-breasted banded rail.}\end{définition}
\begin{définition}\pcmn{水禽,包括几种不同的小鸟,如:鹬}\end{définition}
\begin{définition}\pfra{Gibier d'eau, sauvagine; employé pour divers oiseaux tels que: bécasseau, chevalier (|\stylefi{Calidris}), avocette, marouette, et râle.}\end{définition}
\end{entrée}

\begin{entrée}
{dʑɯ˧dv̩˧}{}{ⓔdʑɯ˧dv̩˧}\formedesurface{dʑɯ˧dv̩˧}\newline
\classe{名词}\ton{M}
\paradigme{\pcmn{:} \p{}}
\begin{définition}\peng{Earthworm.}\end{définition}
\begin{définition}\pcmn{蚯蚓}\end{définition}
\begin{définition}\pfra{Ver de terre.}\end{définition}
\begin{exemple}\pnru{dʑɯ˧dv̩˧-mi˩, | ə˩-dʑo˩˥?}\hspace{5pt}\peng{Do female earthworms exist? (An artificially designed question, so as to elicit a form of ‘earthworm' with the ‘female' suffix, with a view to understanding the synchronically productive tone assignment rules for the gender suffixes.)}\hspace{5pt}\pcmn{有母蚯蚓吗?}\hspace{5pt}\pfra{Les vers de terre femelle, ça existe? (Cette phrase permet d'éliciter une forme associant ‘ver de terre' au suffixe ‘femelle', dans l'idée d'étudier les règles tonales productives en synchronie pour les suffixes de genre.)}\end{exemple}
\begin{exemple}\pnru{dʑɯ˧dv̩˧-pʰv̩˩, | ə˩-dʑo˩˥?}\hspace{5pt}\peng{Do male earthworms exist? (An artificially designed question, so as to elicit a form of ‘earthworm' with the ‘male' suffix, with a view to understanding the synchronically productive tone assignment rules for the gender suffixes.)}\hspace{5pt}\pcmn{有公蚯蚓吗?}\hspace{5pt}\pfra{Les vers de terre mâle, ça existe? (Cette phrase permet d'éliciter une forme associant ‘ver de terre' au suffixe ‘mâle', dans l'idée d'étudier les règles tonales productives en synchronie pour les suffixes de genre.)}\end{exemple}
\end{entrée}

\begin{entrée}
{dʑɯ˩dze˩}{}{ⓔdʑɯ˩dze˩}\formedesurface{dʑɯ˩dze˩˥}\newline
\classe{名词}\ton{L}
\paradigme{\pcmn{:} \p{}}
\begin{définition}\peng{Ladle used for people's food.}\end{définition}
\begin{définition}\pcmn{舀汤的勺子}\end{définition}
\begin{définition}\pfra{Louche utilisée pour faire la cuisine, distribuer la soupe.}\end{définition}
\end{entrée}

\begin{entrée}
{dʑɯ˧dze˧mi\#˥}{}{ⓔdʑɯ˧dze˧mi\#˥}\formedesurface{dʑɯ˧dze˧mi˧}\newline
\classe{名词}\ton{\#H}
\paradigme{\pcmn{:} \p{}}
\begin{définition}\peng{Cicada.}\end{définition}
\begin{définition}\pcmn{蝉}\end{définition}
\begin{définition}\pfra{Cigale.}\end{définition}
\begin{exemple}\pnru{dʑɯ˧dze˧-mi˧ tʰv̩˧-mi˧˥ / dʑɯ˧dze˧-mi˧ tʰv̩˧-mi˥\#}\hspace{5pt}\peng{|fg{n}+|fg{dem}+|fg{clf}}\hspace{5pt}\pcmn{这只蝉}\hspace{5pt}\pfra{|fg{n}+|fg{dem}+|fg{clf}}\end{exemple}
\end{entrée}

\begin{entrée}
{dʑɯ˩gɤ˩di˩}{}{ⓔdʑɯ˩gɤ˩di˩}\formedesurface{dʑɯ˩gɤ˩di˩˥}\newline
\classe{名词}\ton{L}
\paradigme{\pcmn{:} \p{}}
\begin{définition}\peng{Carrying/shoulder pole.}\end{définition}
\begin{définition}\pcmn{扁担}\end{définition}
\begin{définition}\pfra{Palanche.}\end{définition}
\end{entrée}

\begin{entrée}
{dʑɯ˩gv̩˥}{}{ⓔdʑɯ˩gv̩˥}\formedesurface{dʑɯ˩gv̩˥}\newline
\classe{形容词}\ton{LH}\begin{définition}\peng{Round-shouldered, stooping.}\end{définition}
\begin{définition}\pcmn{驼背}\end{définition}
\begin{définition}\pfra{Voûté, qui a le dos rond, courbé.}\end{définition}
\end{entrée}

\begin{entrée}
{dʑɯ˩gv̩˩}{}{ⓔdʑɯ˩gv̩˩}\formedesurface{dʑɯ˩gv̩˩˥}\newline
\classe{名词}\ton{L}
\paradigme{\pcmn{:} \p{}}
\begin{définition}\peng{Large barrel where drinking water is kept; water trough.}\end{définition}
\begin{définition}\pcmn{大水桶,水槽}\end{définition}
\begin{définition}\pfra{Cuve où l'on conserve l'eau potable, tonneau d'eau. A la date de l'enquête, il s'agissait d'un baril en fer.}\end{définition}
\begin{exemple}\pnru{pv̩˩-dʑɯ˩gv̩˩˥}\hspace{5pt}\pfra{même sens}\end{exemple}
\end{entrée}

\begin{entrée}
{dʑɯ˩hṽ̩˧˥}{}{ⓔdʑɯ˩hṽ̩˧˥}\formedesurface{dʑɯ˩hṽ̩˧˥}\newline
\classe{名词}\ton{LM+MH\#}\begin{définition}\peng{Dough made of flour and water.}\end{définition}
\begin{définition}\pcmn{面和水和成的浆糊}\end{définition}
\begin{définition}\pfra{Mélange d'eau et de farine: par ex. de la pâte à pain, du tsamba avec de l'eau…}\end{définition}
\end{entrée}

\begin{entrée}
{dʑɯ˩-hwæ˩tsɯ˥}{}{ⓔdʑɯ˩-hwæ˩tsɯ˥}\formedesurface{dʑɯ˩hwæ˩tsɯ˥}\newline
\classe{名词}\ton{L+H\#}\begin{définition}\peng{Shrew: the consultant uses a periphrasis: “wild mouse".}\end{définition}
\begin{définition}\pcmn{尖鼠、鼩鼱}\end{définition}
\begin{définition}\pfra{Musaraigne; la locutrice emploie une périphrase: «souris sauvage».}\end{définition}
\end{entrée}

\begin{entrée}
{dʑɯ˧ki˥}{}{ⓔdʑɯ˧ki˥}\formedesurface{dʑɯ˩ki˥}\newline
\classe{名词}\ton{H\#}
\paradigme{\pcmn{:} \p{}}
\begin{définition}\peng{Girdle, waistband (a large piece of fabric worn at the waist; can be used to carry a child); belt.}\end{définition}
\begin{définition}\pcmn{布带子,用来背小孩的带子,腰带}\end{définition}
\begin{définition}\pfra{Gaine: large ceinture en tissu, qui peut servir à porter un enfant; aussi: ceinture (terme générique).}\end{définition}
\end{entrée}

\begin{entrée}
{dʑɯ˩kʰi˩}{}{ⓔdʑɯ˩kʰi˩}\formedesurface{dʑɯ˩kʰi˩˥}\newline
\classe{名词}\ton{L}\begin{définition}\peng{Water's edge.}\end{définition}
\begin{définition}\pcmn{水边}\end{définition}
\begin{définition}\pfra{Bord de l'eau.}\end{définition}
\end{entrée}

\begin{entrée}
{dʑɯ˩kʰv̩˩}{}{ⓔdʑɯ˩kʰv̩˩}\formedesurface{dʑɯ˩kʰv̩˩˥}\newline
\classe{名词}\ton{L}\begin{définition}\peng{Moss.}\end{définition}
\begin{définition}\pcmn{青苔}\end{définition}
\begin{définition}\pfra{Mousse.}\end{définition}
\end{entrée}

\begin{entrée}
{dʑɯ˧-li˧}{}{ⓔdʑɯ˧-li˧}\formedesurface{dʑɯ˧li˧}\newline
\classe{动词}\ton{M}\begin{définition}\peng{To irrigate.}\end{définition}
\begin{définition}\pcmn{灌溉}\end{définition}
\begin{définition}\pfra{Irriguer.}\end{définition}
\begin{exemple}\pnru{dʑɯ˧-li˧-ze˧}\hspace{5pt}\peng{|fg{pfv}}\hspace{5pt}\pcmn{灌溉了}\hspace{5pt}\pfra{|fg{pfv}}\end{exemple}
\begin{exemple}\pnru{dʑɯ˧-mɤ˧-li˧-hĩ˧ lv̩˧}\hspace{5pt}\peng{dry farmland, dry field: a field that is not irrigated}\hspace{5pt}\pcmn{旱田:不灌溉的田}\hspace{5pt}\pfra{champ sec/pluvial: «champ qu'on n'irrigue pas»}\end{exemple}
\end{entrée}

\begin{entrée}
{dʑɯ˧ɭɯ˧}{}{ⓔdʑɯ˧ɭɯ˧}\formedesurface{dʑɯ˧ɭɯ˧}\newline
\classe{名词}\ton{M}\begin{définition}\peng{Broomcorn millet, |\stylefi{Panicum miliaceum}.}\end{définition}
\begin{définition}\pcmn{黍,小米}\end{définition}
\begin{définition}\pfra{Millet, |\stylefi{Panicum miliaceum}.}\end{définition}
\begin{exemple}\pnru{dʑɯ˧ɭɯ˧-ho\#˥}\hspace{5pt}\peng{millet gruel}\hspace{5pt}\pcmn{小米粥}\hspace{5pt}\pfra{gruau de millet}\end{exemple}
\end{entrée}

\begin{entrée}
{dʑɯ˧mi˧}{}{ⓔdʑɯ˧mi˧}\formedesurface{dʑɯ˧mi˧}\newline
\classe{名词}\ton{M}
\paradigme{\pcmn{:} \p{}}
\begin{définition}\peng{Large river.}\end{définition}
\begin{définition}\pcmn{大河}\end{définition}
\begin{définition}\pfra{Grande rivière.}\end{définition}
\end{entrée}

\begin{entrée}
{dʑɯ˩nɑ˩hæ̃˩tʰɑ˩}{}{ⓔdʑɯ˩nɑ˩hæ̃˩tʰɑ˩}\formedesurface{dʑɯ˩nɑ˩hæ̃˩tʰɑ˩˥}\newline
\classe{名词}\ton{L}
\paradigme{\pcmn{:} \p{}}
\begin{définition}\peng{Water-mill.}\end{définition}
\begin{définition}\pcmn{水磨}\end{définition}
\begin{définition}\pfra{Moulin à eau.}\end{définition}
\end{entrée}

\begin{entrée}
{dʑɯ˩nɑ˩mi˩}{}{ⓔdʑɯ˩nɑ˩mi˩}\formedesurface{dʑɯ˩nɑ˩mi˩˥}\newline
\classe{名词}\ton{L}\begin{définition}\peng{Mountain areas (uncultivated), mountain forest, wilderness.}\end{définition}
\begin{définition}\pcmn{深山老林、高山上的地方}\end{définition}
\begin{définition}\pfra{Forêt d'altitude, régions sauvages en altitude.}\end{définition}
\end{entrée}

\begin{entrée}
{dʑɯ˩nɑ˩mi˩-ʁo˩}{}{ⓔdʑɯ˩nɑ˩mi˩-ʁo˩}\formedesurface{dʑɯ˩nɑ˩mi˩-ʁo˩˥}\newline
\classe{名词}\ton{L}\begin{définition}\peng{Mountain areas (uncultivated), mountain forest, wilderness.}\end{définition}
\begin{définition}\pcmn{深山老林、高山上的地方}\end{définition}
\begin{définition}\pfra{Forêt d'altitude, régions sauvages en altitude.}\end{définition}
\end{entrée}

\begin{entrée}
{dʑɯ˧njɤ˧}{}{ⓔdʑɯ˧njɤ˧}\formedesurface{dʑɯ˧njɤ˧}\newline
\classe{名词}\ton{M}\begin{définition}\peng{Broomcorn millet, |\stylefi{Panicum miliaceum}.}\end{définition}
\begin{définition}\pcmn{黍,小米}\end{définition}
\begin{définition}\pfra{Millet, |\stylefi{Panicum miliaceum}.}\end{définition}
\begin{exemple}\pnru{dʑɯ˧njɤ˧, | ʐɯ˧ tɕɤ˧˥!}\hspace{5pt}\peng{Millet is used to make wine!}\hspace{5pt}\pcmn{小米,用来酿酒!}\hspace{5pt}\pfra{Le millet, on s'en sert pour faire du vin!}\end{exemple}
\begin{exemple}\pnru{dʑɯ˧njɤ˧-hɑ\#˥}\hspace{5pt}\peng{cooked millet}\hspace{5pt}\pcmn{小米饭}\hspace{5pt}\pfra{millet cuit}\end{exemple}
\end{entrée}

\begin{entrée}
{dʑɯ˩pɤ˩-kv̩˧hĩ˩}{}{ⓔdʑɯ˩pɤ˩-kv̩˧hĩ˩}\formedesurface{dʑɯ˩pɤ˩kv̩˧hĩ˩}\newline
\classe{名词}\ton{L-L\#}\begin{définition}\peng{Spring (where water flows out).}\end{définition}
\begin{définition}\pcmn{水泉、山泉}\end{définition}
\begin{définition}\pfra{Source.}\end{définition}
\begin{exemple}\pnru{dʑɯ˩pɤ˩-kv̩˧hĩ˩ | tʰi˧-di˩}\hspace{5pt}\peng{there is a spring}\hspace{5pt}\pcmn{有水泉}\hspace{5pt}\pfra{il y a une source}\end{exemple}
\end{entrée}

\begin{entrée}
{dʑɯ˩pɤ˩qʰv̩˩}{}{ⓔdʑɯ˩pɤ˩qʰv̩˩}\formedesurface{dʑɯ˩pɤ˩qʰv̩˩˥}\newline
\classe{名词}\ton{L}\begin{définition}\peng{Spring (where water flows out).}\end{définition}
\begin{définition}\pcmn{水泉、山泉}\end{définition}
\begin{définition}\pfra{Source.}\end{définition}
\end{entrée}

\begin{entrée}
{dʑɯ˩pɤ˩tv̩˩qʰv̩˥}{}{ⓔdʑɯ˩pɤ˩tv̩˩qʰv̩˥}\formedesurface{dʑɯ˩pɤ˩tv̩˩qʰv̩˥}\newline
\classe{名词}\ton{L+H\#}\begin{définition}\peng{Spring (where water flows out).}\end{définition}
\begin{définition}\pcmn{水泉、山泉}\end{définition}
\begin{définition}\pfra{Source.}\end{définition}
\end{entrée}

\begin{entrée}
{dʑɯ˩pʰæ˩}{}{ⓔdʑɯ˩pʰæ˩}\formedesurface{dʑɯ˩pʰæ˩˥}\newline
\classe{名词}\ton{L}
\paradigme{\pcmn{:} \p{}}
\begin{définition}\peng{Ice.}\end{définition}
\begin{définition}\pcmn{冰}\end{définition}
\begin{définition}\pfra{Glace.}\end{définition}
\end{entrée}

\begin{entrée}
{dʑɯ˧qʰɑ˧}{}{ⓔdʑɯ˧qʰɑ˧}\formedesurface{dʑɯ˧qʰɑ˧}\newline
\classe{名词}\ton{M}
\paradigme{\pcmn{:} \p{}}
\begin{définition}\peng{Selfheal, |\stylefi{Prunella vulgaris}: a plant used in Chinese medicine. It yields a decoction that was used in Yongning to cure a sore throat. The decoction is bitter, hence the name.}\end{définition}
\begin{définition}\pcmn{夏枯草。煮出的夏枯草汤液,用来治嗓子痛。}\end{définition}
\begin{définition}\pfra{|\stylefi{Prunella vulgaris}, plante employée en médecine chinoise. A Yongning, on l'utilisait en décoction, contre le mal de gorge. La deuxième syllabe signifierait ‘amer’, du fait que la décoction est amère.}\end{définition}
\begin{exemple}\pnru{dʑɯ˧qʰɑ˧-bæ˩bæ˩}\hspace{5pt}\peng{Selfheal flowers.}\hspace{5pt}\pcmn{夏枯草花}\hspace{5pt}\pfra{Fleurs de |Prunella vulgaris.}\end{exemple}
\end{entrée}

\begin{entrée}
{dʑɯ˩qʰæ˩}{}{ⓔdʑɯ˩qʰæ˩}\formedesurface{dʑɯ˩qʰæ˩˥}\newline
\classe{名词}\ton{L}\begin{définition}\peng{Cold water.}\end{définition}
\begin{définition}\pcmn{凉水}\end{définition}
\begin{définition}\pfra{Eau froide.}\end{définition}
\end{entrée}

\begin{entrée}
{dʑɯ˧qʰv̩˧}{}{ⓔdʑɯ˧qʰv̩˧}\formedesurface{dʑɯ˧qʰv̩˧}\newline
\classe{名词}\ton{M}
\paradigme{\pcmn{:} \p{}}
\begin{définition}\peng{Well.}\end{définition}
\begin{définition}\pcmn{井、水井}\end{définition}
\begin{définition}\pfra{Puits.}\end{définition}
\begin{exemple}\pnru{ɑ˩ʁo˥ | dʑɯ˧qʰv̩˧ tʰi˧-di˩.}\hspace{5pt}\peng{There is a well at home / in the farm.}\hspace{5pt}\pcmn{家里有水井。}\hspace{5pt}\pfra{il y a un puits à la maison/dans la ferme.}\end{exemple}
\end{entrée}

\begin{entrée}
{dʑɯ˧qʰv̩˩}{}{ⓔdʑɯ˧qʰv̩˩}\formedesurface{dʑɯ˧qʰv̩˩}\newline
\classe{名词}\ton{L\#}
\paradigme{\pcmn{:} \p{}}
\begin{définition}\peng{A wild plant of Yongning.}\end{définition}
\begin{définition}\pcmn{永宁的一种植物}\end{définition}
\begin{définition}\pfra{Plante sauvage dont les graines forment de grosses boules de graines.}\end{définition}
\begin{exemple}\pnru{dʑɯ˧qʰv̩˩-lv̩˩lv̩˩}\hspace{5pt}\peng{the grains of this plant}\hspace{5pt}\pcmn{这种植物的种子}\hspace{5pt}\pfra{graines de la plante en question}\end{exemple}
\end{entrée}

\begin{entrée}
{dʑɯ˩qʰwɤ˩-lv̩˩}{}{ⓔdʑɯ˩qʰwɤ˩-lv̩˩}\formedesurface{dʑɯ˩qʰwɤ˩-lv̩˩˥}\newline
\classe{名词}\ton{L}
\paradigme{\pcmn{:} \p{}}
\begin{définition}\peng{Marsh, bog, swamp (unsuitable for agriculture).}\end{définition}
\begin{définition}\pcmn{沼泽、湿地}\end{définition}
\begin{définition}\pfra{Marais (terre impropre à la culture).}\end{définition}
\end{entrée}

\begin{entrée}
{dʑɯ˧ʁo˩}{}{ⓔdʑɯ˧ʁo˩}\formedesurface{dʑɯ˧ʁo˩}\newline
\classe{名词}\ton{L\#}\begin{définition}\peng{Peach.}\end{définition}
\begin{définition}\pcmn{桃子}\end{définition}
\begin{définition}\pfra{Pêche.}\end{définition}
\end{entrée}

\begin{entrée}
{dʑɯ˩ʁo˩}{}{ⓔdʑɯ˩ʁo˩}\formedesurface{dʑɯ˩ʁo˩˥}\newline
\classe{名词}\ton{L}\begin{définition}\peng{Mountain areas (uncultivated), mountain forest, wilderness.}\end{définition}
\begin{définition}\pcmn{深山老林、高山上的地方}\end{définition}
\begin{définition}\pfra{Forêt d'altitude, régions sauvages en altitude.}\end{définition}
\end{entrée}

\begin{entrée}
{dʑɯ˩ʁo˩-æ̃˧}{}{ⓔdʑɯ˩ʁo˩-æ̃˧}\formedesurface{dʑɯ˩ʁo˩æ̃˥}\newline
\classe{名词}\ton{L-M}
\paradigme{\pcmn{:} \p{}}
\begin{définition}\peng{Quail, rail, |\stylefi{Coturnix}; used when identifying pictures of various species of |\stylefi{Turnix}, |\stylefi{Coturnix}, and |\stylefi{Crex}.}\end{définition}
\begin{définition}\pcmn{鹌鹑}\end{définition}
\begin{définition}\pfra{Caille, |\stylefi{Coturnix}; terme employé pour divers oiseaux, dont des râles (|\stylefi{Crex}).}\end{définition}
\end{entrée}

\begin{entrée}
{dʑɯ˩ʁo˩-bo˧}{}{ⓔdʑɯ˩ʁo˩-bo˧}\formedesurface{dʑɯ˩ʁo˩bo˥}\newline
\classe{名词}\ton{L-M}
\paradigme{\pcmn{:} \p{}}
\begin{définition}\peng{Wild boar.}\end{définition}
\begin{définition}\pcmn{野猪}\end{définition}
\begin{définition}\pfra{Sanglier, porc sauvage.}\end{définition}
\end{entrée}

\begin{entrée}
{dʑɯ˩ʁo˩-dze˧}{}{ⓔdʑɯ˩ʁo˩-dze˧}\formedesurface{dʑɯ˩ʁo˩dze˥}\newline
\classe{名词}\ton{L-M}\begin{définition}\peng{Wild pepper.}\end{définition}
\begin{définition}\pcmn{野花椒}\end{définition}
\begin{définition}\pfra{Xanthoxyle sauvage.}\end{définition}
\end{entrée}

\begin{entrée}
{dʑɯ˩ʁo˩-hwɤ˩li˧}{}{ⓔdʑɯ˩ʁo˩-hwɤ˩li˧}\formedesurface{dʑɯ˩ʁo˩hwɤ˩li˥}\newline
\classe{名词}\ton{L-LM}
\paradigme{\pcmn{:} \p{}}
\begin{définition}\peng{Yunnan wild cat, |\stylefi{Felis temincki}.}\end{définition}
\begin{définition}\pcmn{野猫}\end{définition}
\begin{définition}\pfra{Chat sauvage, |\stylefi{Felis temincki}.}\end{définition}
\end{entrée}

\begin{entrée}
{dʑɯ˩ʁo˩-ɬi˩bi˧}{}{ⓔdʑɯ˩ʁo˩-ɬi˩bi˧}\formedesurface{dʑɯ˩ʁo˩ɬi˩bi˥}\newline
\classe{名词}\ton{L-LM}
\paradigme{\pcmn{:} \p{}}
\begin{définition}\peng{Chinese yam (shan-yao).}\end{définition}
\begin{définition}\pcmn{山药}\end{définition}
\begin{définition}\pfra{Igname de Chine (shan-yao).}\end{définition}
\end{entrée}

\begin{entrée}
{dʑɯ˩ʁo˩-zɯ˩}{}{ⓔdʑɯ˩ʁo˩-zɯ˩}\formedesurface{dʑɯ˩ʁo˩-zɯ˩˥}\newline
\classe{名词}\ton{L}
\paradigme{\pcmn{:} \p{}}
\begin{définition}\peng{Wild herbs.}\end{définition}
\begin{définition}\pcmn{野草}\end{définition}
\begin{définition}\pfra{Herbes de la montagne, herbes sauvages, foin poussant sur l'alpage.}\end{définition}
\begin{exemple}\pnru{ʈʂʰɯ˧ | dʑɯ˩ʁo˩-zɯ˩ ɲi˥.}\hspace{5pt}\peng{|fg{dem} \_ |fg{cop}}\hspace{5pt}\pcmn{指示代词 \_ 系词}\hspace{5pt}\pfra{|fg{dem} \_ |fg{cop}}\end{exemple}
\end{entrée}

\begin{entrée}
{dʑɯ˩si˩}{}{ⓔdʑɯ˩si˩}\formedesurface{dʑɯ˩si˩˥}\newline
\classe{名词}\ton{L}\begin{définition}\peng{Oriental white oak.}\end{définition}
\begin{définition}\pcmn{青冈树、槲栎}\end{définition}
\begin{définition}\pfra{Chêne blanc oriental.}\end{définition}
\end{entrée}

\begin{entrée}
{dʑɯ˩so˩}{}{ⓔdʑɯ˩so˩}\formedesurface{dʑɯ˩so˩˥}\newline
\classe{名词}\ton{L}
\paradigme{\pcmn{:} \p{}}
\begin{définition}\peng{Wave.}\end{définition}
\begin{définition}\pcmn{波浪}\end{définition}
\begin{définition}\pfra{Vague.}\end{définition}
\begin{exemple}\pnru{dʑɯ˩so˩ pʰv̩˩˥}\hspace{5pt}\peng{there is a wave, a wave breaks}\hspace{5pt}\pcmn{有波浪}\hspace{5pt}\pfra{il y a une vague, une vague déferle}\end{exemple}
\end{entrée}

\begin{entrée}
{dʑɯ˩ʂo˥}{}{ⓔdʑɯ˩ʂo˥}\formedesurface{dʑɯ˩ʂo˥}\newline
\classe{名词}\ton{L+H\#}\begin{définition}\peng{Name of a ritual.}\end{définition}
\begin{définition}\pcmn{一项仪式}\end{définition}
\begin{définition}\pfra{Nom d'un rituel.}\end{définition}
\begin{exemple}\pnru{dʑɯ˩ʂo˥-tsɑ˩bɤ˩}\hspace{5pt}\peng{flour used for ceremonies; it must not contain oatmeal. After the ceremony, it is thrown away (not eaten).}\hspace{5pt}\pcmn{做仪式时所使用的面粉。这种面粉里不要含有燕麦。仪式结束后,面粉被扔掉。}\hspace{5pt}\pfra{farine (tsamba) pouvant servir aux cérémonies; elle ne doit pas contenir d'avoine. A la fin de la cérémonie, on la jette}\end{exemple}
\end{entrée}

\begin{entrée}
{dʑɯ˩ʂwæ˩}{}{ⓔdʑɯ˩ʂwæ˩}\formedesurface{dʑɯ˩ʂwæ˩˥}\newline
\classe{名词}\ton{L}\begin{définition}\peng{|\stylefi{Lysimachia christinae Hance}, a medicinal plant}\end{définition}
\begin{définition}\pcmn{过路黄}\end{définition}
\begin{définition}\pfra{|\stylefi{Lysimachia christinae Hance}, plante médicinale.}\end{définition}
\begin{exemple}\pnru{dʑɯ˩ʂwæ˩-bæ˥bæ˩}\hspace{5pt}\peng{Flower of |Lysimachia christinae Hance.}\hspace{5pt}\pcmn{过路黄花}\hspace{5pt}\pfra{Fleur de |Lysimachia christinae Hance.}\end{exemple}
\end{entrée}

\begin{entrée}
{dʑɯ˩tɤ˩ɻ̍˥}{}{ⓔdʑɯ˩tɤ˩ɻ̍˥}\formedesurface{dʑɯ˩tɤ˩ɻ̍˥}\newline
\classe{形容词}\ton{L+H\#}\begin{définition}\peng{Humid, moist.}\end{définition}
\begin{définition}\pcmn{湿}\end{définition}
\begin{définition}\pfra{Mouillé, humide.}\end{définition}
\begin{exemple}\pnru{dʑɯ˩tɤ˩ɻ̍˥ gv̩˩-ze˩}\hspace{5pt}\peng{It got wet.}\hspace{5pt}\pcmn{湿了!}\hspace{5pt}\pfra{ça s'est mouillé}\end{exemple}
\end{entrée}

\begin{entrée}
{dʑɯ˩tɕʰɯ˩lɑ˩qʰɑ˥}{}{ⓔdʑɯ˩tɕʰɯ˩lɑ˩qʰɑ˥}\formedesurface{dʑɯ˩tɕʰɯ˩lɑ˩qʰɑ˥}\newline
\classe{名词}\ton{L+H\#}\begin{définition}\peng{Plum.}\end{définition}
\begin{définition}\pcmn{一种梅子}\end{définition}
\begin{définition}\pfra{Sorte de prunelle très acide, qui pousse à l'état sauvage; utilisée en décoction, en association avec du gingembre et de petites pommes séchées, contre les maux de gorge.}\end{définition}
\end{entrée}

\begin{entrée}
{dʑɯ˩tɕʰɯ˩lɑ˩qʰæ˥}{}{ⓔdʑɯ˩tɕʰɯ˩lɑ˩qʰæ˥}\formedesurface{dʑɯ˩tɕʰɯ˩lɑ˩qʰæ˥}\newline
\classe{名词}\ton{L+H\#}\begin{définition}\peng{Buckthorn, |\stylefi{Hippophae rhamnoides Linn.}.}\end{définition}
\begin{définition}\pcmn{沙棘}\end{définition}
\begin{définition}\pfra{Argousier, |\stylefi{Hippophae rhamnoides Linn.}.}\end{définition}
\end{entrée}

\begin{entrée}
{dʑɯ˩tsʰi˩}{}{ⓔdʑɯ˩tsʰi˩}\formedesurface{dʑɯ˩tsʰi˩˥}\newline
\classe{名词}\ton{L}\begin{définition}\peng{Boiled water, hot water.}\end{définition}
\begin{définition}\pcmn{开水,热水}\end{définition}
\begin{définition}\pfra{Eau bouillante, eau chaude.}\end{définition}
\end{entrée}

\begin{entrée}
{dʑɯ˩tsʰi˩ʈʰɯ˩di˩}{}{ⓔdʑɯ˩tsʰi˩ʈʰɯ˩di˩}\formedesurface{dʑɯ˩tsʰi˩ʈʰɯ˩di˩˥}\newline
\classe{名词}\ton{L}
\paradigme{\pcmn{:} \p{}}
\begin{définition}\peng{Small container for hot water (for 1 person).}\end{définition}
\begin{définition}\pcmn{口杯}\end{définition}
\begin{définition}\pfra{Petite gourde thermos individuelle.}\end{définition}
\end{entrée}

\begin{entrée}
{dʑɯ˩ʈv̩˧}{}{ⓔdʑɯ˩ʈv̩˧}\formedesurface{dʑɯ˩ʈv̩˥}\newline
\classe{形容词}\ton{LM}\begin{définition}\peng{To be a hunchback/humpback.}\end{définition}
\begin{définition}\pcmn{驼背(严重的病)}\end{définition}
\begin{définition}\pfra{Être gravement voûté, avoir une bosse.}\end{définition}
\begin{exemple}\pnru{dʑɯ˩ʈv̩˧-ze˩}\hspace{5pt}\peng{|fg{pfv}}\hspace{5pt}\pcmn{驼背了}\hspace{5pt}\pfra{|fg{pfv}}\end{exemple}
\end{entrée}

\begin{entrée}
{dʑɯ˧ʈʂʰwæ\#˥}{}{ⓔdʑɯ˧ʈʂʰwæ\#˥}\formedesurface{dʑɯ˧ʈʂʰwæ˧}\newline
\classe{名词}\ton{\#H}\begin{définition}\peng{Husked broomcorn millet, |\stylefi{Panicum miliaceum}.}\end{définition}
\begin{définition}\pcmn{已碾的小米}\end{définition}
\begin{définition}\pfra{Millet décortiqué, |\stylefi{Panicum miliaceum}.}\end{définition}
\end{entrée}

\begin{entrée}
{dʑɯ˩zo˩}{}{ⓔdʑɯ˩zo˩}\formedesurface{dʑɯ˩zo˩˥}\newline
\classe{名词}\ton{L}
\paradigme{\pcmn{:} \p{}}
\begin{définition}\peng{Brook, rivulet.}\end{définition}
\begin{définition}\pcmn{溪流}\end{définition}
\begin{définition}\pfra{Ruisseau.}\end{définition}
\end{entrée}

\begin{entrée}
{dʑɯ˩ʐv̩˩}{}{ⓔdʑɯ˩ʐv̩˩}\formedesurface{dʑɯ˩ʐv̩˩˥}\newline
\classe{动词}\ton{L}\begin{définition}\peng{To swim.}\end{définition}
\begin{définition}\pcmn{游泳}\end{définition}
\begin{définition}\pfra{Nager.}\end{définition}
\end{entrée}

\newpage\caractère{ɖ}

\begin{entrée}
{ɖæ˥}{}{ⓔɖæ˥}\formedesurface{ɖæ˧}\newline
\classe{形容词}\ton{H}\begin{définition}\peng{Short.}\end{définition}
\begin{définition}\pcmn{短}\end{définition}
\begin{définition}\pfra{Court.}\end{définition}
\end{entrée}

\begin{entrée}
{ɖæ˩˧}{}{ⓔɖæ˩˧}\formedesurface{ɖæ˩˥}\newline
\classe{名词}\ton{LM}
\sens{1}\paradigme{\pcmn{:} \p{}}
\begin{définition}\peng{Dust.}\end{définition}
\begin{définition}\pcmn{尘土}\end{définition}
\begin{définition}\pfra{Poussière.}\end{définition}
\begin{exemple}\pnru{ɖæ˩˥ | ɖɯ˧-ti˧ tʰi˧-di˥}\hspace{5pt}\peng{there is a layer of dust}\hspace{5pt}\pcmn{有一层灰}\hspace{5pt}\pfra{il y a une couche de poussière}\end{exemple}
\begin{exemple}\pnru{ɖæ˩ bæ˧}\hspace{5pt}\peng{to sweep the dust}\hspace{5pt}\pcmn{扫灰}\hspace{5pt}\pfra{balayer la poussière}\end{exemple}\sens{2}
\begin{définition}\peng{Dirt, filth.}\end{définition}
\begin{définition}\pcmn{污垢}\end{définition}
\begin{définition}\pfra{Saletés.}\end{définition}
\end{entrée}

\begin{entrée}
{ɖæ˩α}{}{ⓔɖæ˩α}\formedesurface{ɖæ˩˥}\newline
\classe{动词}\ton{Lα}\begin{définition}\peng{To pass over, to cross (a river on a boat, a mountain…).}\end{définition}
\begin{définition}\pcmn{渡(坐船渡河……)}\end{définition}
\begin{définition}\pfra{Passer, traverser (une rivière en bateau, une montagne…).}\end{définition}
\begin{exemple}\pnru{dʑɯ˩ ɖæ˩˥ / dʑɯ˩ ɖæ˩-ze˥}\hspace{5pt}\peng{to cross a river}\hspace{5pt}\pcmn{渡河}\hspace{5pt}\pfra{passer une rivière}\end{exemple}
\begin{exemple}\pnru{dʑɯ˧ | ɖɯ˧-kʰɯ˩ ɖæ˩}\hspace{5pt}\peng{as above}\hspace{5pt}\pcmn{同上}\hspace{5pt}\pfra{idem}\end{exemple}
\end{entrée}

\begin{entrée}
{ɖæ˩α}{}{ⓔɖæ˩α}\formedesurface{ɖɯ˧ ɖæ˩}\newline
\classe{量词}\ton{Lα}\begin{définition}\peng{A section of (road); a bolt of cloth.}\end{définition}
\begin{définition}\pcmn{量词:路(段)/布(匹)}\end{définition}
\begin{définition}\pfra{Classificateur des sections, pour objets pouvant être divisés dans le sens de la longueur. Pour une route, cette unité correspond à 1/2 journée de marche; pour du tissu, elle correspond à une pièce.}\end{définition}
\begin{exemple}\pnru{ʐɤ˩mi˩˥ | ɖɯ˧-ɖæ˩}\hspace{5pt}\peng{a section of road, a stretch of road}\hspace{5pt}\pcmn{一段路}\hspace{5pt}\pfra{un bout de chemin, un bout de route}\end{exemple}
\begin{exemple}\pnru{ɲi˧, ɲi˩-ɖæ˩! |}\hspace{5pt}\peng{Two stretches a day! (Set phrase: in one day, one can cover a distance of two ‘stretches'. If one can get somewhere before lunch, the distance counts as one stretch/section; if one can only arrive there in the afternoon, it counts as two stretches/sections.)}\hspace{5pt}\pcmn{一天两段路!(说明:早上出发,如果午饭前能到目的地,距离算是一段,如果下午晚上才到,算两段。)}\hspace{5pt}\pfra{formule figée traditionnel: un jour, ça fait deux étapes! (si on peut parvenir quelque part avant le déjeuner, la distance est considérée comme une seule étape; si on y parvient dans l'après-midi, on compte deux étapes)}\end{exemple}
\begin{exemple}\pnru{ʈʂʰɯ˧-ɖæ˥}\hspace{5pt}\peng{|fg{dem} \_ (tone: H\# / H\$)}\hspace{5pt}\pcmn{指示代词 \_}\hspace{5pt}\pfra{|fg{dem} \_ (ton: H\# / H\$)}\end{exemple}
\end{entrée}

\begin{entrée}
{ɖæ˩dʑɯ˥}{}{ⓔɖæ˩dʑɯ˥}\formedesurface{ɖæ˩dʑɯ˥}\newline
\classe{名词}\ton{LH}
\paradigme{\pcmn{:} \p{}}
\begin{définition}\peng{Dirt, filth.}\end{définition}
\begin{définition}\pcmn{污垢}\end{définition}
\begin{définition}\pfra{Détritus, saletés, crasse.}\end{définition}
\end{entrée}

\begin{entrée}
{ɖæ˧∼ɖæ˩}{}{ⓔɖæ˧∼ɖæ˩}\formedesurface{ɖæ˧ɖæ˩}\newline
\classe{形容词}\ton{L\#}\begin{définition}\peng{Horizontal.}\end{définition}
\begin{définition}\pcmn{横着(横躺在路上)}\end{définition}
\begin{définition}\pfra{Horizontal.}\end{définition}
\begin{exemple}\pnru{ɖæ˧∼ɖæ˩ | tʰi˧-tɕɯ˥}\hspace{5pt}\peng{to lay flat}\hspace{5pt}\pcmn{横着放}\hspace{5pt}\pfra{poser à plat}\end{exemple}
\end{entrée}

\begin{entrée}
{ɖæ˩-lɑ˧so\#˥}{}{ⓔɖæ˩-lɑ˧so\#˥}\formedesurface{ɖæ˩lɑ˧so˧}\newline
\classe{名词}\ton{L-\#H}\begin{définition}\peng{Name of a ceremony conducted at home once a year, during the first two weeks of the year, by one or two monks invited to the farm: offering grain (or fruit) to the gods. The aim is to ensure prosperity for the household.}\end{définition}
\begin{définition}\pcmn{一种祈福仪式,和尚在过年时主持行礼}\end{définition}
\begin{définition}\pfra{Nom de cérémonie que les moines (un ou deux) pratiquent une fois par an (pendant la première quinzaine de l'année) au domicile de la personne qui les y invite: offrande de céréales (ou de fruits) aux divinités. L'objet de la cérémonie est d'assurer la prospérité financière de la maisonnée.}\end{définition}
\begin{exemple}\pnru{ɖæ˩-lɑ˧so˧ qæ˩}\hspace{5pt}\peng{to carry out the Ddaelaso rituel}\hspace{5pt}\pcmn{进行这种仪式}\hspace{5pt}\pfra{réaliser le rituel Ddaelaso}\end{exemple}
\begin{exemple}\pnru{ɖæ˩-lɑ˧so˧ li˩}\hspace{5pt}\peng{to watch the Ddaelaso rituel}\hspace{5pt}\pcmn{看这种仪式}\hspace{5pt}\pfra{regarder le rituel Ddaelaso}\end{exemple}
\end{entrée}

\begin{entrée}
{ɖæ˩mi˧}{}{ⓔɖæ˩mi˧}\formedesurface{ɖæ˩mi˥}\newline
\classe{名词}\ton{LM}\begin{définition}\peng{The Yongning monastery.}\end{définition}
\begin{définition}\pcmn{永宁大寺}\end{définition}
\begin{définition}\pfra{Nom du temple de Yongning.}\end{définition}
\begin{exemple}\pnru{ɖæ˩mi˧-ʈæ˩bɤ˩}\hspace{5pt}\peng{a priest from the monastery}\hspace{5pt}\pcmn{永宁大寺的和尚}\hspace{5pt}\pfra{un prêtre du monastère}\end{exemple}
\end{entrée}

\begin{entrée}
{ɖæ˩mi˧-go˧bɤ˩}{}{ⓔɖæ˩mi˧-go˧bɤ˩}\formedesurface{ɖæ˩mi˧go˧bɤ˩}\newline
\classe{名词}\ton{LM-L\#}\begin{définition}\peng{Yongning temple.}\end{définition}
\begin{définition}\pcmn{永宁大寺}\end{définition}
\begin{définition}\pfra{Le temple de Yongning.}\end{définition}
\end{entrée}

\begin{entrée}
{ɖæ˩pʰv̩˥}{}{ⓔɖæ˩pʰv̩˥}\formedesurface{ɖæ˩pʰv̩˥}\newline
\classe{名词}\ton{LH}
\paradigme{\pcmn{:} \p{}}
\begin{définition}\peng{Dust, dirt.}\end{définition}
\begin{définition}\pcmn{灰尘}\end{définition}
\begin{définition}\pfra{Poussière.}\end{définition}
\end{entrée}

\begin{entrée}
{ɖæ˩ʂɯ\#˥}{}{ⓔɖæ˩ʂɯ\#˥}\formedesurface{ɖæ˩ʂɯ˥}\newline
\classe{名词}\ton{LM+\#H}\begin{définition}\peng{A village name.}\end{définition}
\begin{définition}\pcmn{扎实(永宁坝子的一个村落)}\end{définition}
\begin{définition}\pfra{Nom de village.}\end{définition}
\begin{exemple}\pnru{ɖæ˩ʂɯ˧-ʁwɤ\#˥}\hspace{5pt}\peng{same meaning}\hspace{5pt}\pcmn{同上:扎实村}\hspace{5pt}\pfra{même sens}\end{exemple}
\begin{exemple}\pnru{ɖæ˩ʂɯ\#˥, | ʈʂo˧ʂɯ\#˥, | bɤ˩tɕʰɯ˩˥, | dɑ˧pʰo˥, | bɤ˧dzi˩, | dze˧bo˧}\hspace{5pt}\peng{Six villages of the plain of Yongning that lie relatively close to the Lake.}\hspace{5pt}\pcmn{永宁摩梭地理概念中,距离泸沽湖比较近的六个村落:扎实、忠实、八旗、达坡、八珠、者波。}\hspace{5pt}\pfra{Six villages de la plaine de Yongning qui sont relativement proches du Lac.}\end{exemple}
\end{entrée}

\begin{entrée}
{ɖɤ˥}{}{ⓔɖɤ˥}\formedesurface{ɖɤ˧}\newline
\classe{动词}\ton{H}\begin{définition}\peng{To crawl, to creep.}\end{définition}
\begin{définition}\pcmn{爬行,匍匐}\end{définition}
\begin{définition}\pfra{Ramper.}\end{définition}
\begin{exemple}\pnru{ɖɤ˧∼ɖɤ˧ (-ze˩)}\hspace{5pt}\peng{|fg{red}}\hspace{5pt}\pcmn{重叠:爬一爬}\hspace{5pt}\pfra{|fg{red}}\end{exemple}
\begin{exemple}\pnru{ʈʂʰɯ˧ | ɖɤ˧∼ɖɤ˧-ʁo˧-ze˩!}\hspace{5pt}\peng{She can crawl! / She knows how to crawl! (About a baby that crawls around.)}\hspace{5pt}\pcmn{她会爬了!}\hspace{5pt}\pfra{Elle sait ramper! (Au sujet d'un bébé qui se déplace en rampant.)}\end{exemple}
\end{entrée}

\begin{entrée}
{ɖɤ˩α}{}{ⓔɖɤ˩α}\formedesurface{ɖɤ˩˥}\newline
\classe{形容词}\ton{Lα}\begin{définition}\peng{Hot (weather).}\end{définition}
\begin{définition}\pcmn{很热(天气),阳光强烈}\end{définition}
\begin{définition}\pfra{Brûlant, ardent (soleil), chaud (temps).}\end{définition}
\begin{exemple}\pnru{ɲi˧mi˧ | ɖɤ˩-ze˥!}\hspace{5pt}\peng{The sun is burning hot, scalding}\hspace{5pt}\pcmn{太阳很大、很强烈}\hspace{5pt}\pfra{le soleil est torride, le soleil est très vif}\end{exemple}
\begin{exemple}\pnru{ɖɤ˩-hĩ˩˥}\hspace{5pt}\peng{|fg{rel}}\hspace{5pt}\pcmn{热的}\hspace{5pt}\pfra{|fg{rel}}\end{exemple}
\end{entrée}

\begin{entrée}
{ɖɤ˧mi˧}{}{ⓔɖɤ˧mi˧}\formedesurface{ɖɤ˧mi˧}\newline
\classe{名词}\ton{M}
\paradigme{\pcmn{:} \p{}}
\begin{définition}\peng{Fox.}\end{définition}
\begin{définition}\pcmn{狐狸}\end{définition}
\begin{définition}\pfra{Renard.}\end{définition}
\begin{exemple}\pnru{ɖɤ˧mi˧-zo\#˥}\hspace{5pt}\peng{little fox, baby fox}\hspace{5pt}\pcmn{小狐狸}\hspace{5pt}\pfra{petit renard, renardeau}\end{exemple}
\begin{exemple}\pnru{ɖɤ˧mi˧-pʰv̩\#˥}\hspace{5pt}\peng{male fox}\hspace{5pt}\pcmn{公狐狸}\hspace{5pt}\pfra{renard mâle}\end{exemple}
\begin{exemple}\pnru{ɖɤ˧mi˧, | mi˩ ɲi˥!}\hspace{5pt}\peng{This fox is a female!}\hspace{5pt}\pcmn{这只狐狸是母的!}\hspace{5pt}\pfra{Ce renard, c'est une femelle!}\end{exemple}
\end{entrée}

\begin{entrée}
{ɖo˧α}{}{ⓔɖo˧α}\formedesurface{ɖo˧}\newline
\classe{动词}\ton{M intrans}\begin{définition}\peng{To allow, to permit; also: to order about; to run errands for.}\end{définition}
\begin{définition}\pcmn{让,指使、使唤}\end{définition}
\begin{définition}\pfra{Devoir, être obligé de; permettre, autoriser, accorder; ordonner, donner un ordre.}\end{définition}
\begin{exemple}\pnru{po˧ mɤ˧-ɖo˧!}\hspace{5pt}\peng{(You) are not allowed to take it! / You must not take it! (eg telling a child that (s)he is not allowed to take a knife)}\hspace{5pt}\pcmn{不许拿!}\hspace{5pt}\pfra{(Tu n'as) pas le droit de le prendre! (ex.: un enfant n'est pas autorisé à jouer avec un couteau)}\end{exemple}
\begin{exemple}\pnru{ʈʂʰɯ˧, | po˧ ɖo˧!}\hspace{5pt}\peng{That one, you can have it / you can take it / you can play with it! (Context: as above: telling a child what (s)he can and cannot toy with.)}\hspace{5pt}\pcmn{那个,是可以拿的! / 那个,是可以玩的!(情景同上:告诉一个小孩子什么东西可以拿,什么不可以拿。)}\hspace{5pt}\pfra{Ca, (tu) peux le prendre / tu peux jouer avec! (Même contexte que ci-dessus: on indique à un enfant ce qu'on a le droit de prendre, et ce qu'on n'a pas le droit de prendre.)}\end{exemple}
\begin{exemple}\pnru{gɤ˩ do˧ mɤ˧-ɖo˧!}\hspace{5pt}\peng{(You) are not allowed to climb (on the table,…)}\hspace{5pt}\pcmn{不许爬上(桌子……)}\hspace{5pt}\pfra{(tu) n'as pas le droit de grimper/monter sur (une table…)}\end{exemple}
\begin{exemple}\pnru{lɑ˧-kʰv̩˧˥, | ʑi˧qʰwɤ˧ tsʰi˧-mɤ˧-ɖo˧! | ʑi˩-kʰv̩˩˥, | ʑi˧qʰwɤ˧ tsʰi˧-mɤ˧-ɖo˧! |}\hspace{5pt}\peng{(During) the year of the Tiger, one should not build a house! (During) the year of the Monkey, one should not build a house! (These years are considered too “hard", /wu˧/, by astrology.)}\hspace{5pt}\pcmn{虎年,不要建房!猴年,不要建房!(这样的年,被认为是太‘硬’的。)}\hspace{5pt}\pfra{L'année du Tigre, l'année du Singe, il ne faut pas construire de maison/il ne faut pas se lancer dans la construction d'une maison! (Ce sont des années trop «dures», /wu˧/, selon l'astrologie traditionnelle)}\end{exemple}
\begin{exemple}\pnru{ʝi˧ mɤ˧-ɖo˧!}\hspace{5pt}\peng{(One) must not do (that)!}\hspace{5pt}\pcmn{不要做!}\hspace{5pt}\pfra{Il ne faut pas faire (ça)!}\end{exemple}
\end{entrée}

\begin{entrée}
{ɖɯ˧‑}{}{ⓔɖɯ˧‑}\formedesurface{ɖɯ˧}\newline
\classe{介词}\ton{M}\begin{définition}\peng{|fg{delimitative.}}\end{définition}
\begin{définition}\pcmn{进行时态}\end{définition}
\begin{définition}\pfra{|fg{délimitatif.}}\end{définition}
\end{entrée}

\begin{entrée}
{ɖɯ˧˥}{}{ⓔɖɯ˧˥}\formedesurface{ɖɯ˧˥}\newline
\classe{数词}\ton{MH}\begin{définition}\peng{One (numeral).}\end{définition}
\begin{définition}\pcmn{一}\end{définition}
\begin{définition}\pfra{Un (numéral).}\end{définition}
\end{entrée}

\begin{entrée}
{ɖɯ˧β}{}{ⓔɖɯ˧β}\formedesurface{ɖɯ˧}\newline
\classe{动词}\ton{Mβ}\begin{définition}\peng{To obtain, to get.}\end{définition}
\begin{définition}\pcmn{得到}\end{définition}
\begin{définition}\pfra{Obtenir, trouver.}\end{définition}
\begin{exemple}\pnru{le˧-ʂe˧ le˧-ɖɯ˧-ze˧!}\hspace{5pt}\peng{(I) have looked for something and found it! / I have found (something by looking around for it)!}\hspace{5pt}\pcmn{找到了!}\hspace{5pt}\pfra{j'ai cherché et je l'ai trouvé! j'ai trouvé (en cherchant)!}\end{exemple}
\begin{exemple}\pnru{ɖɯ˧-tʰɑ˧˥!}\hspace{5pt}\peng{It is possible to obtain it! / It can be obtained!}\hspace{5pt}\pcmn{可以得到的!}\hspace{5pt}\pfra{On peut obtenir!}\end{exemple}
\begin{exemple}\pnru{ɖɯ˧-tʰɑ˧-ze˥!}\hspace{5pt}\peng{We have managed to obtain it! / We found it possible to obtain it!}\hspace{5pt}\pcmn{(我们)成功地得到了!}\hspace{5pt}\pfra{On a réussi à obtenir!}\end{exemple}
\begin{exemple}\pnru{tso˧∼tso˧ ɖɯ˧ (+ze˧)}\hspace{5pt}\peng{to obtain something}\hspace{5pt}\pcmn{得到东西}\hspace{5pt}\pfra{obtenir quelque chose}\end{exemple}
\end{entrée}

\begin{entrée}
{ɖɯ˩α}{}{ⓔɖɯ˩α}\formedesurface{ɖɯ˩˥}\newline
\classe{形容词}\ton{Lα}\begin{définition}\peng{Big, large.}\end{définition}
\begin{définition}\pcmn{大}\end{définition}
\begin{définition}\pfra{Grand.}\end{définition}
\begin{exemple}\pnru{mɤ˧-ɖɯ˩}\hspace{5pt}\peng{|fg{neg}}\hspace{5pt}\pcmn{|fg{neg}}\hspace{5pt}\pfra{|fg{neg}}\end{exemple}
\begin{exemple}\pnru{ɖɯ˩-hĩ˩˥}\hspace{5pt}\peng{|fg{rel}}\hspace{5pt}\pcmn{|fg{rel}}\hspace{5pt}\pfra{|fg{rel}}\end{exemple}
\begin{exemple}\pnru{le˧-ɖɯ˩(-ze˩)}\hspace{5pt}\peng{|fg{accomp}+|fg{pfv}}\hspace{5pt}\pcmn{|fg{accomp}+|fg{pfv}}\hspace{5pt}\pfra{|fg{accomp}+|fg{pfv}: ça a grandi!/ il/elle a grandi!}\end{exemple}
\begin{exemple}\pnru{ə˧pɤ˥ɖɯ˩-gv̩˩}\hspace{5pt}\peng{very big}\hspace{5pt}\pcmn{好大、大得很}\hspace{5pt}\pfra{très grand}\end{exemple}
\end{entrée}

\begin{entrée}
{ɖɯ˩ɖʐɯ˧}{}{ⓔɖɯ˩ɖʐɯ˧}\formedesurface{ɖɯ˩ɖʐɯ˥}\newline
\classe{名词}\ton{LM}\begin{définition}\peng{Masculine given name.}\end{définition}
\begin{définition}\pcmn{男性名字:独知}\end{définition}
\begin{définition}\pfra{Prénom masculin.}\end{définition}
\end{entrée}

\begin{entrée}
{ɖɯ˩ɖʐɯ˧-tsʰɯ˩ɻ̍˩}{}{ⓔɖɯ˩ɖʐɯ˧-tsʰɯ˩ɻ̍˩}\formedesurface{ɖɯ˩ɖʐɯ˧tsʰɯ˩ɻ̍˩}\newline
\classe{名词}\ton{LM-L}\begin{définition}\peng{Masculine given name.}\end{définition}
\begin{définition}\pcmn{男性名字}\end{définition}
\begin{définition}\pfra{Prénom masculin.}\end{définition}
\end{entrée}

\begin{entrée}
{ɖɯ˩hĩ˩}{}{ⓔɖɯ˩hĩ˩}\formedesurface{ɖɯ˩hĩ˩˥}\newline
\classe{名词}\ton{L}
\paradigme{\pcmn{:} \p{}}
\begin{définition}\peng{Important people (including elders).}\end{définition}
\begin{définition}\pcmn{大人、重要的人(包括长辈)}\end{définition}
\begin{définition}\pfra{Gens importants.}\end{définition}
\end{entrée}

\begin{entrée}
{ɖɯ˩lo\#˥}{}{ⓔɖɯ˩lo\#˥}\formedesurface{ɖɯ˩lo˥}\newline
\classe{名词}\ton{LM+\#H}
\sens{1}\paradigme{\pcmn{:} \p{}}
\begin{définition}\peng{Tradition.}\end{définition}
\begin{définition}\pcmn{传统}\end{définition}
\begin{définition}\pfra{Coutume, tradition.}\end{définition}
\begin{exemple}\pnru{ɖɯ˩lo˧ ɖɯ˧-kʰwɤ˥ | tʰi˧-so˥-ɻ̍˩}\hspace{5pt}\peng{to teach a custom}\hspace{5pt}\pcmn{教授一个传统、一个习俗}\hspace{5pt}\pfra{enseigner une coutume}\end{exemple}\sens{2}
\begin{définition}\peng{Good manners.}\end{définition}
\begin{définition}\pcmn{礼仪、礼貌}\end{définition}
\begin{définition}\pfra{Savoir-vivre.}\end{définition}
\begin{exemple}\pnru{ʈʂʰɯ˧ | ɖɯ˩lo˧ dʑɤ˥!}\hspace{5pt}\peng{(S)he knows the customs / (s)he has good manners}\hspace{5pt}\pcmn{他懂礼貌、他会做人}\hspace{5pt}\pfra{Il/elle sait vivre/connaît les coutumes/a du savoir-vivre!}\end{exemple}\sens{3}
\begin{définition}\peng{The order of things.}\end{définition}
\begin{définition}\pcmn{道理}\end{définition}
\begin{définition}\pfra{Ordre des choses.}\end{définition}
\end{entrée}

\begin{entrée}
{ɖɯ˧-ɬi˧mi˧}{}{ⓔɖɯ˧-ɬi˧mi˧}\formedesurface{ɖɯ˧ɬi˧mi˧}\newline
\classe{名词}\ton{M}\begin{définition}\peng{1st month.}\end{définition}
\begin{définition}\pcmn{正月}\end{définition}
\begin{définition}\pfra{1er mois.}\end{définition}
\end{entrée}

\begin{entrée}
{ɖɯ˩mɑ\#˥}{}{ⓔɖɯ˩mɑ\#˥}\formedesurface{ɖɯ˩mɑ˥}\newline
\classe{名词}\ton{LM+\#H}\begin{définition}\peng{Feminine given name.}\end{définition}
\begin{définition}\pcmn{女性名字}\end{définition}
\begin{définition}\pfra{Prénom féminin.}\end{définition}
\end{entrée}

\begin{entrée}
{ɖɯ˩mɑ˧-ɬɑ˩tsʰo˩}{}{ⓔɖɯ˩mɑ˧-ɬɑ˩tsʰo˩}\formedesurface{ɖɯ˩mɑ˧ɬɑ˩tsʰo˩}\newline
\classe{名词}\ton{LM-L}\begin{définition}\peng{Feminine given name.}\end{définition}
\begin{définition}\pcmn{女性名字}\end{définition}
\begin{définition}\pfra{Prénom féminin.}\end{définition}
\end{entrée}

\begin{entrée}
{ɖɯ˩mɑ˧-pv̩˩ʈʰɯ˩}{}{ⓔɖɯ˩mɑ˧-pv̩˩ʈʰɯ˩}\formedesurface{ɖɯ˩mɑ˧pv̩˩ʈʰɯ˩}\newline
\classe{名词}\ton{LM-L}\begin{définition}\peng{Feminine given name.}\end{définition}
\begin{définition}\pcmn{女性名字}\end{définition}
\begin{définition}\pfra{Prénom féminin.}\end{définition}
\end{entrée}

\begin{entrée}
{ɖɯ˩mi\#˥}{}{ⓔɖɯ˩mi\#˥}\formedesurface{ɖɯ˩mi˥}\newline
\classe{名词}\ton{LM+\#H}
\paradigme{\pcmn{:} \p{}}
\begin{définition}\peng{Female mule. This is a sterile animal. It is docile, and suitable for tasks such as leading a caravan. It is therefore a highly valued animal.}\end{définition}
\begin{définition}\pcmn{母骡子、母马骡}\end{définition}
\begin{définition}\pfra{Mule femelle. C'est un animal stérile. Il est docile et fort, et peut accomplir des tâches importantes comme d'être l'animal de tête dans une caravane. C'est donc un animal de grand prix.}\end{définition}
\begin{exemple}\pnru{ɖɯ˩mi˧-ɖɯ˥zo˩}\hspace{5pt}\peng{Female mule and male mule}\hspace{5pt}\pcmn{母骡子与公骡子}\hspace{5pt}\pfra{mule femelle et mule mâle}\end{exemple}
\end{entrée}

\begin{entrée}
{ɖɯ˧-njɤ˧}{}{ⓔɖɯ˧-njɤ˧}\formedesurface{ɖɯ˧njɤ˧}\newline
\classe{助词}\ton{M}\begin{définition}\peng{Continuously, ceaselessly.}\end{définition}
\begin{définition}\pcmn{一直、一直不停}\end{définition}
\begin{définition}\pfra{Sans cesse; sans arrêt; toujours.}\end{définition}
\begin{exemple}\pnru{ɖɯ˧-njɤ˧ | so˩˥}\hspace{5pt}\peng{to study ceaselessly}\hspace{5pt}\pcmn{一直不停地学习}\hspace{5pt}\pfra{étudier sans arrêt}\end{exemple}
\begin{exemple}\pnru{ɖɯ˧-njɤ˧ | lo˧ ʝi˧}\hspace{5pt}\peng{to work ceaselessly}\hspace{5pt}\pcmn{一直不停地工作}\hspace{5pt}\pfra{travailler sans arrêt}\end{exemple}
\begin{exemple}\pnru{ɖɯ˧-njɤ˧-zo˥}\hspace{5pt}\peng{often}\hspace{5pt}\pcmn{经常、常}\hspace{5pt}\pfra{souvent}\end{exemple}
\begin{exemple}\pnru{ɖɯ˧-njɤ˧ hwæ˩; ɖɯ˧-njɤ˧ tɕʰi˧; ɖɯ˧-njɤ˧ dzɯ˧; ɖɯ˧-njɤ˧ dze˧˥; ɖɯ˧-njɤ˧ ʐwɤ˧˥; ɖɯ˧-njɤ˧ lɑ˧˥}\hspace{5pt}\peng{combinations with verbs in the six tones: to buy, to sell, to eat, to cut, to speak, to strike}\hspace{5pt}\pcmn{跟六个调类的动词结合:买,卖,吃,切,说,打}\hspace{5pt}\pfra{avec des verbes aux six tons, pour étudier les tons : acheter; vendre; manger; couper; parler; frapper}\end{exemple}
\begin{exemple}\pnru{ɖɯ˧-njɤ˧ | hwæ˧; ɖɯ˧-njɤ˧ | tɕʰi˧; ɖɯ˧-njɤ˧ | dzɯ˧; ɖɯ˧-njɤ˧ | dze˩˥; ɖɯ˧-njɤ˧ | ʐwɤ˩˥; ɖɯ˧-njɤ˧ | lɑ˧˥}\hspace{5pt}\peng{combinations with verbs in the six tones: to buy, to sell, to eat, to cut, to speak, to strike (separating into two tone groups)}\hspace{5pt}\pcmn{跟六个调类的动词结合:买,卖,吃,切,说,打}\hspace{5pt}\pfra{avec des verbes aux six tons, pour étudier les tons : acheter; vendre; manger; couper; parler; frapper; en séparant en groupes tonals}\end{exemple}
\end{entrée}

\begin{entrée}
{ɖɯ˧-ɲi˧-ɖɯ˥-hɑ̃˩}{}{ⓔɖɯ˧-ɲi˧-ɖɯ˥-hɑ̃˩}\formedesurface{ɖɯ˧ɲi˧ɖɯ˥hɑ̃˩}\newline
\classe{名词}\ton{\#H-}\begin{définition}\peng{One day and one night.}\end{définition}
\begin{définition}\pcmn{一天一夜}\end{définition}
\begin{définition}\pfra{Un jour et une nuit.}\end{définition}
\end{entrée}

\begin{entrée}
{ɖɯ˧-so˩}{}{ⓔɖɯ˧-so˩}\formedesurface{ɖɯ˧so˩}\newline
\classe{名词}\ton{Lα}\begin{définition}\peng{Some, a few. Made up of ‘one' and ‘three'.}\end{définition}
\begin{définition}\pcmn{一些、两三个(直译:‘一三(个)’}\end{définition}
\begin{définition}\pfra{Un petit nombre de, quelques-uns.}\end{définition}
\begin{exemple}\pnru{hĩ˧ | ɖɯ˧-so˩ kv̩˩}\hspace{5pt}\peng{a few people}\hspace{5pt}\pcmn{几个人}\hspace{5pt}\pfra{quelques personnes (deux, trois…)}\end{exemple}
\begin{exemple}\pnru{ɖɯ˧-so˩ ɲi˩}\hspace{5pt}\peng{a few days}\hspace{5pt}\pcmn{几天}\hspace{5pt}\pfra{quelques jours}\end{exemple}
\end{entrée}

\begin{entrée}
{ɖɯ˧ʈæ˩}{}{ⓔɖɯ˧ʈæ˩}\formedesurface{ɖɯ˧ʈæ˩}\newline
\classe{名词}\ton{L\#}\begin{définition}\peng{A ritual performed by monks after someone's decease.}\end{définition}
\begin{définition}\pcmn{一个葬礼仪式,由和尚主持}\end{définition}
\begin{définition}\pfra{Rite pratiqué par les moines du monastère après un décès.}\end{définition}
\end{entrée}

\begin{entrée}
{ɖɯ˩zo\#˥}{}{ⓔɖɯ˩zo\#˥}\formedesurface{ɖɯ˩zo˥}\newline
\classe{名词}\ton{LM+\#H}
\paradigme{\pcmn{:} \p{}}
\begin{définition}\peng{Male mule.}\end{définition}
\begin{définition}\pcmn{公骡子}\end{définition}
\begin{définition}\pfra{Mule mâle (animal moins prisé que la femelle).}\end{définition}
\begin{exemple}\pnru{ɖɯ˩zo˧-ɖɯ˥mi˩}\hspace{5pt}\peng{male mule and female mule}\hspace{5pt}\pcmn{公骡子与母骡子}\hspace{5pt}\pfra{mule mâle et mule femelle}\end{exemple}
\end{entrée}

\begin{entrée}
{ɖv̩˩}{}{ⓔɖv̩˩}\formedesurface{ɖv̩˧}\newline
\classe{名词}\ton{L}
\paradigme{\pcmn{:} \p{}}
\begin{définition}\peng{Wing (monosyllabic form; the disyllabic form is preferred).}\end{définition}
\begin{définition}\pcmn{翅膀}\end{définition}
\begin{définition}\pfra{Ailes (forme monosyllabique; la forme disyllabique est préférée).}\end{définition}
\end{entrée}

\begin{entrée}
{ɖv̩˧qæ˧}{}{ⓔɖv̩˧qæ˧}\formedesurface{ɖv̩˧qæ˧}\newline
\classe{名词}\ton{M}
\paradigme{\pcmn{:} \p{}}
\begin{définition}\peng{Wing.}\end{définition}
\begin{définition}\pcmn{翅膀}\end{définition}
\begin{définition}\pfra{Ailes.}\end{définition}
\begin{exemple}\pnru{kɤ˩nɑ˧mi˧-ɖv̩˧qæ˥}\hspace{5pt}\peng{eagle wings}\hspace{5pt}\pcmn{老鹰翅膀}\hspace{5pt}\pfra{aile d'aigle}\end{exemple}
\end{entrée}

\begin{entrée}
{ɖwæ˥}{}{ⓔɖwæ˥}\formedesurface{ɖwæ˧}\newline
\classe{形容词}\ton{H}\begin{définition}\peng{Muddy, turbid.}\end{définition}
\begin{définition}\pcmn{浑浊 (水)}\end{définition}
\begin{définition}\pfra{Trouble (le même terme est employé pour l'eau vive et pour l'eau stagnante).}\end{définition}
\begin{exemple}\pnru{dʑɯ˧ ɖwæ\#˥}\hspace{5pt}\peng{turbid water}\hspace{5pt}\pcmn{浑浊的水}\hspace{5pt}\pfra{eau trouble}\end{exemple}
\begin{exemple}\pnru{dʑɯ˧ | ɖwæ˧-ze˩!}\hspace{5pt}\peng{The water has become turbid.}\hspace{5pt}\pcmn{水浑浊了。}\hspace{5pt}\pfra{l'eau s'est troublée! l'eau est devenue trouble!}\end{exemple}
\end{entrée}

\begin{entrée}
{ɖwæ˥}{}{ⓔɖwæ˥}\formedesurface{ɖwæ˧}\newline
\classe{名词}
\sens{1}\paradigme{\pcmn{:} \p{}}
\begin{définition}\peng{Pond.}\end{définition}
\begin{définition}\pcmn{池塘}\end{définition}
\begin{définition}\pfra{Mare.}\end{définition}
\begin{exemple}\pnru{ɖwæ˩ɬo˩mi˧}\hspace{5pt}\peng{large pond}\hspace{5pt}\pcmn{大池塘}\hspace{5pt}\pfra{grand étang}\end{exemple}\sens{2}
\begin{définition}\peng{Pool (artificial).}\end{définition}
\begin{définition}\pcmn{水坑}\end{définition}
\begin{définition}\pfra{Réserve d'eau (artificielle).}\end{définition}
\end{entrée}

\begin{entrée}
{ɖwæ˥α}{}{ⓔɖwæ˥α}\formedesurface{ɖɯ˧ ɖwæ˥}\newline
\classe{量词}\ton{Hα}\begin{définition}\peng{Classifier for steps (of stairs).}\end{définition}
\begin{définition}\pcmn{量词:梯级、楼梯(一节)}\end{définition}
\begin{définition}\pfra{Classificateur des marches d'escalier.}\end{définition}
\begin{exemple}\pnru{ɖɯ˧-ɖwæ˧ ɲi˥}\hspace{5pt}\peng{It's a step (of stairs). (Elicited to investigate the word's tonal behaviour)}\hspace{5pt}\pcmn{是一节/一节阶梯。(引出这句是为了了解这个词在不同语境的声调变化。)}\hspace{5pt}\pfra{c'est une marche}\end{exemple}
\begin{exemple}\pnru{ʈʂʰɯ˧-ɖwæ\#˥}\hspace{5pt}\peng{this step}\hspace{5pt}\pcmn{这节阶梯}\hspace{5pt}\pfra{cette marche}\end{exemple}
\end{entrée}

\begin{entrée}
{ɖwæ˧˥}{₁}{ⓔɖwæ˧˥ⓗ1}\formedesurface{ɖwæ˧˥}\newline
\classe{动词}\ton{MH}
1\begin{définition}\peng{To whip.}\end{définition}
\begin{définition}\pcmn{鞭打、抽打、加鞭}\end{définition}
\begin{définition}\pfra{Fouetter, donner des coups (ex.: un tigre fouette le sol avec sa queue).}\end{définition}
\begin{exemple}\pnru{mæ˧qv̩˩-po˩-ɳɯ˩ | ɖwæ˧˥}\hspace{5pt}\peng{to whip with the tail (e.g. a tiger whips the ground with its tail)}\hspace{5pt}\pcmn{用尾巴来抽打(如:老虎用尾巴来抽打地面)}\hspace{5pt}\pfra{donner des coups de queue (ex.: le tigre fouette le sol de sa queue)}\end{exemple}
\end{entrée}

\begin{entrée}
{ɖwæ˧˥}{₂}{ⓔɖwæ˧˥ⓗ2}\formedesurface{ɖwæ˧˥}\newline
\classe{助词}\ton{MH}
2\begin{définition}\peng{Intensive: very, terribly.}\end{définition}
\begin{définition}\pcmn{很、极}\end{définition}
\begin{définition}\pfra{Intensif: très.}\end{définition}
\begin{exemple}\pnru{ʈʂʰɯ˧ | ɖwæ˧˥ | æ˧mv̩˩ fv̩˩!}\hspace{5pt}\peng{She likes her elder sister very much!}\hspace{5pt}\pcmn{她很喜欢她姐姐!}\hspace{5pt}\pfra{elle aime beaucoup sa grande sœur!}\end{exemple}
\end{entrée}

\begin{entrée}
{ɖwæ˩α}{}{ⓔɖwæ˩α}\formedesurface{ɖwæ˩˥}\newline
\classe{动词}\ton{Lα}\begin{définition}\peng{To be afraid.}\end{définition}
\begin{définition}\pcmn{害怕}\end{définition}
\begin{définition}\pfra{Avoir peur.}\end{définition}
\begin{exemple}\pnru{njɤ˧ | ɖwæ˩˥!}\hspace{5pt}\peng{I am afraid!}\hspace{5pt}\pcmn{我害怕!}\hspace{5pt}\pfra{J'ai peur!}\end{exemple}
\begin{exemple}\pnru{njɤ˧ | ʈʂʰɯ˧-v̩˧ | ɖwæ˩˥ | ʐwæ˩˥!}\hspace{5pt}\peng{I am really afraid of this person!}\hspace{5pt}\pcmn{我很害怕那个人!}\hspace{5pt}\pfra{J'ai très peur de lui!}\end{exemple}
\end{entrée}

\begin{entrée}
{ɖwæ˧-pɤ˧ɭɯ˥}{}{ⓔɖwæ˧-pɤ˧ɭɯ˥}\formedesurface{ɖwæ˧pɤ˧ɭɯ˥}\newline
\classe{名词}\ton{H\#}
\paradigme{\pcmn{:} \p{}}
\begin{définition}\peng{Puddle, pool (natural).}\end{définition}
\begin{définition}\pcmn{水潭}\end{définition}
\begin{définition}\pfra{Flaque (naturelle).}\end{définition}
\begin{exemple}\pnru{ɖwæ˧ tʰi˧-pɤ˥ɭɯ˩}\hspace{5pt}\peng{there is water in the pool; a puddle has formed}\hspace{5pt}\pcmn{有水潭}\hspace{5pt}\pfra{une flaque/petite mare s'est formée, il y a une flaque}\end{exemple}
\end{entrée}

\begin{entrée}
{ɖʐæ˧β}{}{ⓔɖʐæ˧β}\formedesurface{ɖʐæ˧}\newline
\classe{动词}\ton{Mβ}\begin{définition}\peng{To ride (a horse).}\end{définition}
\begin{définition}\pcmn{骑马}\end{définition}
\begin{définition}\pfra{Monter à cheval.}\end{définition}
\begin{exemple}\pnru{le˧-ɖʐæ˧-ze˧}\hspace{5pt}\peng{|fg{accomp} \_ |fg{pfv}}\hspace{5pt}\pcmn{|fg{accomp} \_ |fg{pfv}}\hspace{5pt}\pfra{|fg{accomp} \_ |fg{pfv}}\end{exemple}
\begin{exemple}\pnru{ʐwæ˧ ɖʐæ˧}\hspace{5pt}\peng{to ride a horse}\hspace{5pt}\pcmn{骑马}\hspace{5pt}\pfra{monter à cheval}\end{exemple}
\begin{exemple}\pnru{ɖʐæ˧-tʰɑ˧˥!}\hspace{5pt}\peng{It's possible to ride (it)! / It's OK to ride (it)!}\hspace{5pt}\pcmn{可以骑的!}\hspace{5pt}\pfra{On peut le monter!}\end{exemple}
\end{entrée}

\begin{entrée}
{ɖʐæ˩α}{}{ⓔɖʐæ˩α}\formedesurface{ɖʐæ˩˥}\newline
\classe{动词}\ton{Lα}\begin{définition}\peng{To melt; to thaw.}\end{définition}
\begin{définition}\pcmn{融化}\end{définition}
\begin{définition}\pfra{Fondre.}\end{définition}
\begin{exemple}\pnru{mɤ˧ | le˧-ɖʐæ˩-ze˩}\hspace{5pt}\peng{The grease has melted.}\hspace{5pt}\pcmn{油融化了。}\hspace{5pt}\pfra{la graisse a fondu (ex.: du saindoux qui fond dans un chaudron)}\end{exemple}
\begin{exemple}\pnru{dʑi˩pʰæ˩˥ | le˧-ɖʐæ˩-ze˩}\hspace{5pt}\peng{The ice has melted.}\hspace{5pt}\pcmn{冰融化了。}\hspace{5pt}\pfra{La glace a fondu.}\end{exemple}
\end{entrée}

\begin{entrée}
{ɖʐæ˩bv̩˩}{}{ⓔɖʐæ˩bv̩˩}\formedesurface{ɖʐæ˩bv̩˩˥}\newline
\classe{名词}\ton{L}
\paradigme{\pcmn{:} \p{}}
\begin{définition}\peng{Sorcerer.}\end{définition}
\begin{définition}\pcmn{法师}\end{définition}
\begin{définition}\pfra{Sorcier.}\end{définition}
\begin{exemple}\pnru{ə˧pʰv̩˧-ɖʐæ˩bv̩˩}\hspace{5pt}\peng{‘Grandfather sorcerer': a respectful term of address for a sorcerer who is advanced in years or considered to have great powers}\hspace{5pt}\pcmn{‘法师爷爷’:对年龄高(或被认为本事很大)的法师的尊重称呼}\hspace{5pt}\pfra{‘Grand-père sorcier': terme d'adresse respectueux pour un sorcier d'âge avancé, ou considéré comme ayant des pouvoirs considérables}\end{exemple}
\begin{exemple}\pnru{ə˧v̩˧-ɖʐæ˥bv̩˩}\hspace{5pt}\peng{‘Uncle sorcerer': a respectful term of address for a sorcerer}\hspace{5pt}\pcmn{‘法师舅舅’:对法师的尊重称呼}\hspace{5pt}\pfra{‘Oncle sorcier': terme d'adresse respectueux pour un sorcier}\end{exemple}
\end{entrée}

\begin{entrée}
{ɖʐæ˧qʰæ˥\$}{}{ⓔɖʐæ˧qʰæ˥\$}\formedesurface{ɖʐæ˧qʰæ˥}\newline
\classe{名词}\ton{H\$}\begin{définition}\peng{Mud.}\end{définition}
\begin{définition}\pcmn{泥巴}\end{définition}
\begin{définition}\pfra{Boue.}\end{définition}
\begin{exemple}\pnru{ɖʐæ˧qʰæ˧ ʐæ˥(-ze˩)}\hspace{5pt}\peng{There is mud; mud has formed.}\hspace{5pt}\pcmn{有泥巴了。}\hspace{5pt}\pfra{De la boue s'est formée; il y a de la boue, c'est tout boueux. (Littéralement: «de la boue s'est mélangée».)}\end{exemple}
\begin{exemple}\pnru{ɖʐæ˧qʰæ˧ ʐæ˧∼ʐæ˥}\hspace{5pt}\peng{There is mud; mud has formed.}\hspace{5pt}\pcmn{有泥巴了}\hspace{5pt}\pfra{De la boue s'est formée; il y a de la boue, c'est tout boueux. (Littéralement: «de la boue s'est mélangée».)}\end{exemple}
\end{entrée}

\begin{entrée}
{ɖʐe˧}{}{ⓔɖʐe˧}\formedesurface{ɖʐe˧}\newline
\classe{名词}\ton{M}\begin{définition}\peng{Money.}\end{définition}
\begin{définition}\pcmn{钱}\end{définition}
\begin{définition}\pfra{Argent (avoir de l'argent).}\end{définition}
\end{entrée}

\begin{entrée}
{ɖʐe˧gɯ˧}{}{ⓔɖʐe˧gɯ˧}\formedesurface{ɖʐe˧gɯ˧}\newline
\classe{名词}\ton{M}\begin{définition}\peng{Yongsheng (place name).}\end{définition}
\begin{définition}\pcmn{永胜(地名)}\end{définition}
\begin{définition}\pfra{Yongsheng (nom de comté).}\end{définition}
\begin{exemple}\pnru{ɖʐe˧gɯ˧-to˩mi˩}\hspace{5pt}\peng{a high mountain located in Yongsheng}\hspace{5pt}\pcmn{永胜的一座高山}\hspace{5pt}\pfra{une haute montagne située à Yongsheng}\end{exemple}
\begin{exemple}\pnru{ɖʐe˧gɯ˧-hæ˧}\hspace{5pt}\peng{Yongsheng Chinese (Han) (note: Yongsheng is mainly populated by Han Chinese)}\hspace{5pt}\pcmn{永胜汉族}\hspace{5pt}\pfra{Chinois de Yongsheng (note: le comté de Yongsheng était peuplé majoritairement de Chinois (Han).)}\end{exemple}
\begin{exemple}\pnru{ɖʐe˧gɯ˧-dʑo˧, | hæ˧-ʂo˧∼ʂo˩!}\hspace{5pt}\peng{In Yongsheng, there are lots of Chinese (Han) people!}\hspace{5pt}\pcmn{永胜,汉族群多!}\hspace{5pt}\pfra{A Yongsheng, il y a plein de Chinois (Han)!}\end{exemple}
\end{entrée}

\begin{entrée}
{ɖʐe˧ʁwɤ˧}{}{ⓔɖʐe˧ʁwɤ˧}\formedesurface{ɖʐe˧ʁwɤ˧}\newline
\classe{名词}\ton{M}\begin{définition}\peng{Money, wealth.}\end{définition}
\begin{définition}\pcmn{钱}\end{définition}
\begin{définition}\pfra{Argent (monnaie); richesse.}\end{définition}
\end{entrée}

\begin{entrée}
{ɖʐɤ˧˥}{₁}{ⓔɖʐɤ˧˥ⓗ1}\formedesurface{ɖʐɤ˧˥}\newline
\classe{动词}
1
\sens{1}
\begin{définition}\peng{To pluck (fruit, weeds), to pick (vegetables).}\end{définition}
\begin{définition}\pcmn{摘(果子、蔬菜)、扯(草)}\end{définition}
\begin{définition}\pfra{Cueillir (des fruits, des légumes); arracher (des mauvaises herbes).}\end{définition}
\begin{exemple}\pnru{le˧-ɖʐɤ˧-po˥-jo˩!}\hspace{5pt}\peng{Pluck some (fruit) and pass them over (to us)!}\hspace{5pt}\pcmn{(你)去给摘(一些)过来吧!}\hspace{5pt}\pfra{cueille-m'en qq-unes!/cueilles-en et passe-les(-nous) par ici!}\end{exemple}
\begin{exemple}\pnru{v̩˩tsʰɤ˧ ɖʐɤ˥}\hspace{5pt}\peng{to pick vegetables}\hspace{5pt}\pcmn{摘蔬菜}\hspace{5pt}\pfra{cueillir des légumes}\end{exemple}
\begin{exemple}\pnru{le˧-ɖʐɤ˧˥, | mv̩˩-tɕo˧ kwɤ˩}\hspace{5pt}\peng{to pluck and throw away (weeds)}\hspace{5pt}\pcmn{扯(荒草),扔掉}\hspace{5pt}\pfra{arracher et jeter (les mauvaises herbes)}\end{exemple}\sens{2}
\begin{définition}\peng{To snap, to cut (thread); to smash; to destroy (a building).}\end{définition}
\begin{définition}\pcmn{拆(线),拔,捣碎}\end{définition}
\begin{définition}\pfra{Déchirer, couper (fil); briser; broyer; détruire (une maison).}\end{définition}
\begin{exemple}\pnru{le˧-ɖʐɤ˩∼ɖʐɤ˩}\hspace{5pt}\peng{|fg{red}}\hspace{5pt}\pcmn{重叠:拆拆}\hspace{5pt}\pfra{|fg{red}}\end{exemple}
\begin{exemple}\pnru{ʑi˧qʰwɤ˧ ɖʐɤ˧˥}\hspace{5pt}\peng{to destroy a house}\hspace{5pt}\pcmn{拆房子}\hspace{5pt}\pfra{détruire une maison, démolir une maison}\end{exemple}
\begin{exemple}\pnru{le˧-ɖʐɤ˧˥ | ɲi˧-gi˧ gv̩˧}\hspace{5pt}\peng{to tear into two pieces}\hspace{5pt}\pcmn{拆成两半}\hspace{5pt}\pfra{déchirer en deux morceaux}\end{exemple}
\end{entrée}

\begin{entrée}
{ɖʐɤ˧˥}{₂}{ⓔɖʐɤ˧˥ⓗ2}\formedesurface{ɖʐɤ˧˥}\newline
\classe{动词}\ton{MH}
2\begin{définition}\peng{To prop open (a tent).}\end{définition}
\begin{définition}\pcmn{撑开(帐篷)}\end{définition}
\begin{définition}\pfra{Déployer, ouvrir en soutenant; ex.: déployer la tente.}\end{définition}
\begin{exemple}\pnru{le˧-ɖʐɤ˩∼ɖʐɤ˩}\hspace{5pt}\peng{|fg{red}}\hspace{5pt}\pcmn{重叠}\hspace{5pt}\pfra{|fg{red}}\end{exemple}
\end{entrée}

\begin{entrée}
{ɖʐɤ˩}{}{ⓔɖʐɤ˩}\formedesurface{ɖʐɤ˧}\newline
\classe{名词}
\sens{1}\paradigme{\pcmn{:} \p{}}
\begin{définition}\peng{Ladder.}\end{définition}
\begin{définition}\pcmn{梯子}\end{définition}
\begin{définition}\pfra{Échelle.}\end{définition}
\begin{exemple}\pnru{ɖʐɤ˩ do˧}\hspace{5pt}\peng{to climb a ladder}\hspace{5pt}\pcmn{爬上一个梯子}\hspace{5pt}\pfra{gravir une échelle}\end{exemple}
\begin{exemple}\pnru{ɖʐɤ˧ | gɤ˩-do˧}\hspace{5pt}\peng{to climb up a ladder}\hspace{5pt}\pcmn{爬上一个梯子}\hspace{5pt}\pfra{même sens, avec ajout d'un directionnel: gravir une échelle}\end{exemple}\sens{2}
\begin{définition}\peng{Stairs.}\end{définition}
\begin{définition}\pcmn{楼梯}\end{définition}
\begin{définition}\pfra{Escalier (en bois, sauf indication contraire).}\end{définition}
\begin{exemple}\pnru{lv̩˧mi˧-ɖʐɤ˩ (+ɲi˩)}\hspace{5pt}\peng{stone stairs}\hspace{5pt}\pcmn{石头楼梯}\hspace{5pt}\pfra{escalier en pierre}\end{exemple}
\end{entrée}

\begin{entrée}
{ɖʐɤ˩ɖwæ˩}{}{ⓔɖʐɤ˩ɖwæ˩}\formedesurface{ɖʐɤ˩ɖwæ˩˥}\newline
\classe{名词}\ton{L}
\paradigme{\pcmn{:} \p{}}
\begin{définition}\peng{Step of stairs.}\end{définition}
\begin{définition}\pcmn{台阶}\end{définition}
\begin{définition}\pfra{Marche d'escalier.}\end{définition}
\begin{exemple}\pnru{lv̩˧mi˧-ɖʐɤ˩ɖwæ˩}\hspace{5pt}\peng{stone step}\hspace{5pt}\pcmn{石头台阶}\hspace{5pt}\pfra{marche en pierre}\end{exemple}
\end{entrée}

\begin{entrée}
{ɖʐɤ˩kɤ˥\$}{}{ⓔɖʐɤ˩kɤ˥\$}\formedesurface{ɖʐɤ˩kɤ˥}\newline
\classe{名词}\ton{LM+H\$}\begin{définition}\peng{A family name from Yongning. There are two families in Yongning that carry this name.}\end{définition}
\begin{définition}\pcmn{一个姓。这个姓,永宁有两家}\end{définition}
\begin{définition}\pfra{Nom de clan/famille étendue. Deux familles portent ce nom à Yongning.}\end{définition}
\begin{exemple}\pnru{ɖʐɤ˩kɤ˧=ɻ̍˥\$}\hspace{5pt}\peng{the /ɖʐɤ˩kɤ˥\$/ clan}\hspace{5pt}\pcmn{|fv{/ɖʐɤ˩kɤ˥\$/}家族}\hspace{5pt}\pfra{le clan /ɖʐɤ˩kɤ˥\$/}\end{exemple}
\end{entrée}

\begin{entrée}
{ɖʐɤ˧qʰwɤ˧}{}{ⓔɖʐɤ˧qʰwɤ˧}\formedesurface{ɖʐɤ˧qʰwɤ˧}\newline
\classe{名词}\ton{M}
\paradigme{\pcmn{:} \p{}}
\begin{définition}\peng{Cold, flu.}\end{définition}
\begin{définition}\pcmn{感冒}\end{définition}
\begin{définition}\pfra{Rhume.}\end{définition}
\begin{exemple}\pnru{ɖʐɤ˧qʰwɤ˧ go˩}\hspace{5pt}\peng{to have a cold; to have a flu}\hspace{5pt}\pcmn{感冒}\hspace{5pt}\pfra{avoir un rhume, être enrhumé}\end{exemple}
\begin{exemple}\pnru{ɖʐɤ˧qʰwɤ˧ mɤ˧-go˩}\hspace{5pt}\peng{…has no cold / does not have a cold}\hspace{5pt}\pcmn{没感冒}\hspace{5pt}\pfra{…n'est pas enrhumé}\end{exemple}
\end{entrée}

\begin{entrée}
{ɖʐɤ˧qʰwɤ˧ʈʂe\#˥}{}{ⓔɖʐɤ˧qʰwɤ˧ʈʂe\#˥}\formedesurface{ɖʐɤ˧qʰwɤ˧ʈʂe˧}\newline
\classe{名词}\ton{\#H}
\paradigme{\pcmn{:} \p{}}
\begin{définition}\peng{Awl.}\end{définition}
\begin{définition}\pcmn{锥、锥子}\end{définition}
\begin{définition}\pfra{Poinçon, alène.}\end{définition}
\end{entrée}

\begin{entrée}
{ɖʐo˥}{}{ⓔɖʐo˥}\formedesurface{ɖʐo˧}\newline
\classe{形容词}\ton{H}\begin{définition}\peng{Cold (weather).}\end{définition}
\begin{définition}\pcmn{冷(天气……)}\end{définition}
\begin{définition}\pfra{Froid.}\end{définition}
\end{entrée}

\begin{entrée}
{ɖʐo˥}{}{ⓔɖʐo˥}\formedesurface{ɖʐo˧}\newline
\classe{名词}\ton{\#H}
\paradigme{\pcmn{:} \p{}}
\begin{définition}\peng{Major roof beam.}\end{définition}
\begin{définition}\pcmn{大梁}\end{définition}
\begin{définition}\pfra{Pièce de charpente carrée (côté: environ 18 cm), dans les parties du bâtiment qui n'ont pas de piliers: \stylefv{/gæ}˩pʰæ˧/, \stylefv{/mv̩}˩pʰæ˧/. Elles supportent la charpente. (M18 pense que ce terme désigne toute la structure du bâtiment.).}\end{définition}
\end{entrée}

\begin{entrée}
{ɖʐo˥α}{}{ⓔɖʐo˥α}\formedesurface{ɖɯ˧ ɖʐo˥}\newline
\classe{量词}\ton{Hα}\begin{définition}\peng{Classifier for beams (in carpentry).}\end{définition}
\begin{définition}\pcmn{量词:梁(一根)}\end{définition}
\begin{définition}\pfra{Classificateur des poutres.}\end{définition}
\end{entrée}

\begin{entrée}
{ɖʐo˩β}{}{ⓔɖʐo˩β}\formedesurface{ɖʐo˩˥}\newline
\classe{动词}\ton{Lβ}\begin{définition}\peng{To crush, to crumble (with the teeth or with a grindstone).}\end{définition}
\begin{définition}\pcmn{弄碎(用牙齿、手磨)}\end{définition}
\begin{définition}\pfra{Écraser (au moulin; ou avec les dents).}\end{définition}
\begin{exemple}\pnru{ʈʂo˧ɭɯ˧ ɖʐo˧˥}\hspace{5pt}\peng{to crush with a grindstone}\hspace{5pt}\pcmn{用手磨弄碎}\hspace{5pt}\pfra{écraser avec un moulin (/ʈʂu˧ɭɯ\#˥/: moulin)}\end{exemple}
\begin{exemple}\pnru{ɖɯ˧-kʰwɤ˧ ɖʐo˧˥}\hspace{5pt}\peng{to crush a piece (of something)}\hspace{5pt}\pcmn{弄碎一块}\hspace{5pt}\pfra{écraser un morceau (de quelque chose)}\end{exemple}
\begin{exemple}\pnru{ɖɯ˧-mɤ˩ ɖʐo˩}\hspace{5pt}\peng{to crush a little (of something)}\hspace{5pt}\pcmn{弄碎一点(东西)}\hspace{5pt}\pfra{écraser un peu (de quelque chose)}\end{exemple}
\end{entrée}

\begin{entrée}
{ɖʐɯ˥}{}{ⓔɖʐɯ˥}\formedesurface{ɖʐɯ˧}\newline
\classe{名词}\ton{\#H}
\paradigme{\pcmn{:} \p{}}
\begin{définition}\peng{Market.}\end{définition}
\begin{définition}\pcmn{集市(圩场,街子)}\end{définition}
\begin{définition}\pfra{Marché.}\end{définition}
\end{entrée}

\begin{entrée}
{ɖʐɯ˥α}{}{ⓔɖʐɯ˥α}\formedesurface{ɖɯ˧ ɖʐɯ˥}\newline
\classe{量词}\ton{Hα}\begin{définition}\peng{Self-classifier for marketplaces/towns.}\end{définition}
\begin{définition}\pcmn{量词:市场(一个),城市(一个)}\end{définition}
\begin{définition}\pfra{Auto-classificateur des marchés/villes.}\end{définition}
\begin{exemple}\pnru{ɖʐɯ˧ | ɖɯ˧-ɖʐɯ˥}\hspace{5pt}\peng{a marketplace, a town}\hspace{5pt}\pcmn{一个市场}\hspace{5pt}\pfra{un marché}\end{exemple}
\end{entrée}

\begin{entrée}
{ɖʐɯ˧∼ɖʐɯ˥}{}{ⓔɖʐɯ˧∼ɖʐɯ˥}\formedesurface{ɖʐɯ˧ɖʐɯ˥}\newline
\classe{名词}\ton{H\#}\begin{définition}\pcmn{(一)会儿}\end{définition}
\begin{définition}\pfra{Moment, instant.}\end{définition}
\begin{exemple}\pnru{ɖɯ˧-ɖʐɯ˧∼ɖʐɯ˥}\hspace{5pt}\peng{a moment}\hspace{5pt}\pcmn{一会儿}\hspace{5pt}\pfra{un moment}\end{exemple}
\begin{exemple}\pnru{ɖɯ˧-ɖʐɯ˧∼ɖʐɯ˥ ʝi˩}\hspace{5pt}\peng{to work for a moment}\hspace{5pt}\pcmn{工作一会儿}\hspace{5pt}\pfra{travailler un moment}\end{exemple}
\end{entrée}

\begin{entrée}
{ɖʐɯ˩∼ɖʐɯ˧˥}{}{ⓔɖʐɯ˩∼ɖʐɯ˧˥}\formedesurface{ɖʐɯ˩ɖʐɯ˧˥}\newline
\classe{动词}\ton{MH}\begin{définition}\peng{To shake (one's head).}\end{définition}
\begin{définition}\pcmn{摇(头)}\end{définition}
\begin{définition}\pfra{Secouer (la tête).}\end{définition}
\begin{exemple}\pnru{ʁo˧ ɖʐɯ˥∼ɖʐɯ˩}\hspace{5pt}\peng{to shake one's head}\hspace{5pt}\pcmn{摇头}\hspace{5pt}\pfra{secouer la tête}\end{exemple}
\begin{exemple}\pnru{ʁo˧ | le˧-ɖʐɯ˩∼ɖʐɯ˩-ze˩}\hspace{5pt}\peng{shook (her/his) head}\hspace{5pt}\pcmn{摇了头}\hspace{5pt}\pfra{a secoué la tête}\end{exemple}
\end{entrée}

\begin{entrée}
{ɖʐɯ˥kʰɤ˩}{}{ⓔɖʐɯ˥kʰɤ˩}\formedesurface{ɖʐɯ˧kʰɤ˩}\newline
\classe{名词}\begin{définition}\peng{Moment.}\end{définition}
\begin{définition}\pcmn{(一)会儿}\end{définition}
\begin{définition}\pfra{Un moment.}\end{définition}
\begin{exemple}\pnru{ɖɯ˧-ɖʐɯ˥kʰɤ˩}\hspace{5pt}\peng{a moment}\hspace{5pt}\pcmn{一会儿}\hspace{5pt}\pfra{un moment}\end{exemple}
\end{entrée}

\begin{entrée}
{ɖʐɯ˩kʰɤ˥}{}{ⓔɖʐɯ˩kʰɤ˥}\formedesurface{ɖʐɯ˩kʰɤ˥}\newline
\classe{名词}\ton{LH}
\étymologie{
ɖʐɯ˩a
}
\paradigme{\pcmn{:} \p{}}
\begin{définition}\peng{Period of time, era.}\end{définition}
\begin{définition}\pcmn{时代}\end{définition}
\begin{définition}\pfra{Époque, ère, état de la société.}\end{définition}
\end{entrée}

\begin{entrée}
{ɖʐɯ˧qo˩}{}{ⓔɖʐɯ˧qo˩}\formedesurface{ɖʐɯ˧qo˩}\newline
\classe{助词}\ton{L\#}\begin{définition}\peng{In town, in the street.}\end{définition}
\begin{définition}\pcmn{在城里、在市里}\end{définition}
\begin{définition}\pfra{En ville.}\end{définition}
\begin{exemple}\pnru{ɖʐɯ˧qo˩ kʰi˩}\hspace{5pt}\peng{to go into town}\hspace{5pt}\pcmn{上街}\hspace{5pt}\pfra{aller dans la rue, faire un tour en ville}\end{exemple}
\end{entrée}

\begin{entrée}
{ɖʐɯ˧ʂɯ˥}{}{ⓔɖʐɯ˧ʂɯ˥}\formedesurface{ɖʐɯ˧ʂɯ˥}\newline
\classe{名词}\ton{H\#}
\paradigme{\pcmn{:} \p{}}
\begin{définition}\peng{Chopsticks.}\end{définition}
\begin{définition}\pcmn{筷子}\end{définition}
\begin{définition}\pfra{Baguettes.}\end{définition}
\end{entrée}

\begin{entrée}
{ɖʐɯ˩tso\#˥}{}{ⓔɖʐɯ˩tso\#˥}\formedesurface{ɖʐɯ˩tso˥}\newline
\classe{名词}\ton{LM+\#H}
\paradigme{\pcmn{:} \p{}}
\begin{définition}\peng{Rules of society.}\end{définition}
\begin{définition}\pcmn{社会规矩}\end{définition}
\begin{définition}\pfra{Règles de conduite sociale, règles régissant la société (politique, société).}\end{définition}
\begin{exemple}\pnru{ɖʐɯ˩tso˥ | hĩ˧-qo˩-ɳɯ˩ | le˧-tsʰɯ˩-ɲi˩-tsɯ˩-mæ˩!}\hspace{5pt}\peng{The rules of society, the moral teachings (including proverbs, tales…) come from people / are human creations / are the fruit of human experience!}\hspace{5pt}\pcmn{社会规矩,是通过人类的经验形成的! / 社会规矩,是人(按一代代的经验)创造的!}\hspace{5pt}\pfra{ces morales (les contes, les proverbes…) ça provient des hommes! / la morale (des histoires, …) c'est le fruit de l'expérience des hommes!}\end{exemple}
\end{entrée}

\begin{entrée}
{ɖʐɯ˧ʈʂɯ˥}{}{ⓔɖʐɯ˧ʈʂɯ˥}\formedesurface{ɖʐɯ˧ʈʂɯ˥}\newline
\classe{名词}\ton{H\#}
\paradigme{\pcmn{:} \p{}}
\begin{définition}\peng{Sifter, sieve.}\end{définition}
\begin{définition}\pcmn{筛子}\end{définition}
\begin{définition}\pfra{Vannerie.}\end{définition}
\end{entrée}

\begin{entrée}
{ɖʐv̩˥}{}{ⓔɖʐv̩˥}\formedesurface{ɖʐv̩˧}\newline
\classe{名词}\ton{\#H}
\paradigme{\pcmn{:} \p{}}
\begin{définition}\peng{Large vein, artery.}\end{définition}
\begin{définition}\pcmn{动脉}\end{définition}
\begin{définition}\pfra{Artère.}\end{définition}
\end{entrée}

\begin{entrée}
{ɖʐv̩˥}{}{ⓔɖʐv̩˥}\formedesurface{ɖʐv̩˧}\newline
\classe{形容词}\ton{H}\begin{définition}\peng{Moist, wet, damp, humid.}\end{définition}
\begin{définition}\pcmn{湿}\end{définition}
\begin{définition}\pfra{Humide, mouillé.}\end{définition}
\begin{exemple}\pnru{le˧-ɖʐv̩˥-ze˩}\hspace{5pt}\peng{|fg{accomp} \_ |fg{pfv}}\hspace{5pt}\pcmn{|fg{accomp} \_ |fg{pfv}}\hspace{5pt}\pfra{|fg{accomp} \_ |fg{pfv}}\end{exemple}
\begin{exemple}\pnru{ɖʐv̩˧∼ɖʐv̩˧}\hspace{5pt}\peng{|fg{red}}\hspace{5pt}\pcmn{|fg{red}}\hspace{5pt}\pfra{|fg{red}}\end{exemple}
\begin{exemple}\pnru{ʈʂe˧ ɖʐv̩˧-ze˩!}\hspace{5pt}\peng{The earth is damp!}\hspace{5pt}\pcmn{土湿了。}\hspace{5pt}\pfra{La terre est mouillée!}\end{exemple}
\begin{exemple}\pnru{si˧ɖʐv̩\#˥}\hspace{5pt}\peng{green wood, freshly cut wood (antonym of: dry wood)}\hspace{5pt}\pcmn{新伐材、生材、湿材(反义词:干木)}\hspace{5pt}\pfra{bois vert, bois fraîchement coupé (antonyme de: bois sec)}\end{exemple}
\end{entrée}

\begin{entrée}
{ɖʐv̩˥}{}{ⓔɖʐv̩˥}\formedesurface{ɖʐv̩˧}\newline
\classe{动词}\ton{H}\begin{définition}\peng{To rise, to go up, to increase.}\end{définition}
\begin{définition}\pcmn{涨}\end{définition}
\begin{définition}\pfra{Augmenter.}\end{définition}
\begin{exemple}\pnru{hĩ˧ ɖʐv̩˧}\hspace{5pt}\peng{people become numerous}\hspace{5pt}\pcmn{人变多}\hspace{5pt}\pfra{Les gens deviennent nombreux, se multiplient}\end{exemple}
\begin{exemple}\pnru{mo˧ ɖʐv̩˥}\hspace{5pt}\peng{mushrooms multiply, become numerous}\hspace{5pt}\pcmn{菌子长得多}\hspace{5pt}\pfra{les champignons se multiplient}\end{exemple}
\end{entrée}

\begin{entrée}
{ɖʐv̩˧}{₁}{ⓔɖʐv̩˧ⓗ1}\formedesurface{ɖʐv̩˧}\newline
\classe{动词}\ton{M intrans}
1\begin{définition}\peng{To burn, to catch fire.}\end{définition}
\begin{définition}\pcmn{燃烧}\end{définition}
\begin{définition}\pfra{Brûler; prendre feu.}\end{définition}
\begin{exemple}\pnru{mv̩˧ ɖʐv̩˧-ze˩!}\hspace{5pt}\peng{It has caught fire!}\hspace{5pt}\pcmn{着火了!}\hspace{5pt}\pfra{Ca a pris feu! / Au feu!}\end{exemple}
\begin{exemple}\pnru{mv̩˧ le˧-ɖʐv̩˧-ze˧!}\hspace{5pt}\peng{The fire has caught!}\hspace{5pt}\pcmn{开始着火了!}\hspace{5pt}\pfra{Le feu a pris!}\end{exemple}
\begin{exemple}\pnru{tʰi˧-ɖʐv̩˧-dʑo˧!}\hspace{5pt}\peng{The fire is burning!}\hspace{5pt}\pcmn{火在燃烧!}\hspace{5pt}\pfra{C'est en train de brûler! / Le feu est en train de brûler!}\end{exemple}
\end{entrée}

\begin{entrée}
{ɖʐv̩˧}{₂}{ⓔɖʐv̩˧ⓗ2}\formedesurface{ɖʐv̩˧}\newline
\classe{名词}\ton{M}
2
\paradigme{\pcmn{:} \p{}}
\begin{définition}\peng{Friend, companion, partner.}\end{définition}
\begin{définition}\pcmn{朋友、伙伴、伴侣}\end{définition}
\begin{définition}\pfra{Ami/amie, compagnon/compagne.}\end{définition}
\begin{exemple}\pnru{njɤ˧ | ɖʐv̩˧ ɲi˩.}\hspace{5pt}\peng{[(S)he] is my friend.}\hspace{5pt}\pcmn{是我朋友。}\hspace{5pt}\pfra{C'est mon ami(e).}\end{exemple}
\begin{exemple}\pnru{õ˧ ɖʐv̩˥, õ˩ li˩! |}\hspace{5pt}\peng{‘One is easily influenced by one's friends!' (Literally: ‘One's friends, one observes'.) The proverb refers to influence from friends, good or bad: good friends exert good influences; bad friends exert bad influences.}\hspace{5pt}\pcmn{“大家都容易受朋友的影响!”(直译:“自己的朋友,自己看(=自己爱学他们的习惯)”)}\hspace{5pt}\pfra{‘On est influencé par ses amis!' (Littéralement: ‘On observe ses amis!') Le proverbe souligne l'influence des amis, en bien ou en mal selon qu'on a ou non choisi judicieusement.}\end{exemple}
\end{entrée}

\begin{entrée}
{ɖʐv̩˧}{₃}{ⓔɖʐv̩˧ⓗ3}\formedesurface{ɖʐv̩˧}\newline
\classe{名词}\ton{M}
3
\paradigme{\pcmn{:} \p{}}
\begin{définition}\peng{An important and unfortunate event, such as a serious accident.}\end{définition}
\begin{définition}\pcmn{事故,(不幸的)大事}\end{définition}
\begin{définition}\pfra{Accident (grave).}\end{définition}
\begin{exemple}\pnru{ɖʐv̩˧ kʰɯ˧˥}\hspace{5pt}\peng{to cause an accident; to commit a fault; something serious happens}\hspace{5pt}\pcmn{犯错误,出大事}\hspace{5pt}\pfra{causer un accident, commettre une faute; il se passe quelque chose de grave}\end{exemple}
\begin{exemple}\pnru{ɖʐv̩˧ kʰɯ˧-ze˥}\hspace{5pt}\peng{As above, with the |fg{pfv} morpheme}\hspace{5pt}\pcmn{同上,加上|fg{pfv}语素}\hspace{5pt}\pfra{Comme ci-dessus, avec ajout du |fg{pfv}}\end{exemple}
\begin{exemple}\pnru{ɖʐv̩˧ ɖɯ˧-ɖʐv̩˧ | kʰɯ˧-ze˥!}\hspace{5pt}\peng{An accident has happened! / There's been an accident!}\hspace{5pt}\pcmn{出大事了!}\hspace{5pt}\pfra{il est arrivé un accident!}\end{exemple}
\end{entrée}

\begin{entrée}
{ɖʐv̩˧}{₄}{ⓔɖʐv̩˧ⓗ4}\formedesurface{ɖʐv̩˧}\newline
\classe{名词}\ton{M}
4\begin{définition}\peng{Dew.}\end{définition}
\begin{définition}\pcmn{露水}\end{définition}
\begin{définition}\pfra{Rosée.}\end{définition}
\end{entrée}

\begin{entrée}
{ɖʐv̩˧β}{}{ⓔɖʐv̩˧β}\formedesurface{ɖɯ˧ ɖʐv̩˧}\newline
\classe{量词}\ton{Mβ}\begin{définition}\peng{Self-classifier for accidents.}\end{définition}
\begin{définition}\pcmn{量词:事故(一场)}\end{définition}
\begin{définition}\pfra{Auto-classificateur des accidents.}\end{définition}
\end{entrée}

\begin{entrée}
{ɖʐv̩˩α}{₁}{ⓔɖʐv̩˩αⓗ1}\formedesurface{ɖʐv̩˩˥}\newline
\classe{形容词}\ton{Lα}
1\begin{définition}\peng{Ugly.}\end{définition}
\begin{définition}\pcmn{丑陋}\end{définition}
\begin{définition}\pfra{Laid, vilain.}\end{définition}
\begin{exemple}\pnru{ɖʐv̩˩-hĩ˩˥}\hspace{5pt}\peng{|fg{rel}/|fg{nmlz}}\hspace{5pt}\pcmn{丑的}\hspace{5pt}\pfra{|fg{rel}/|fg{nmlz}}\end{exemple}
\begin{exemple}\pnru{ʈʂʰɯ˧-v̩˧ | ɖwæ˧˥ | ɖʐv̩˩˥!}\hspace{5pt}\peng{This one is really ugly!}\hspace{5pt}\pcmn{这个好丑!}\hspace{5pt}\pfra{celui-là/celle-là est vraiment méchant/mauvais}\end{exemple}
\end{entrée}

\begin{entrée}
{ɖʐv̩˩α}{₂}{ⓔɖʐv̩˩αⓗ2}\formedesurface{ɖʐv̩˩˥}\newline
\classe{动词}\ton{Lα}
2\begin{définition}\peng{To decide, to make a decision.}\end{définition}
\begin{définition}\pcmn{决定、选择、拿主意}\end{définition}
\begin{définition}\pfra{Décider, choisir.}\end{définition}
\begin{exemple}\pnru{njɤ˧-ɳɯ˧ | ɖʐv̩˧ ʝi˧-bi˧!}\hspace{5pt}\peng{I'm going to decide!}\hspace{5pt}\pcmn{我来决定吧!}\hspace{5pt}\pfra{C'est moi qui vais décider!}\end{exemple}
\end{entrée}

\begin{entrée}
{ɖʐv̩˧-nɑ˥mi˩}{}{ⓔɖʐv̩˧-nɑ˥mi˩}\formedesurface{ɖʐv̩˧nɑ˥mi˩}\newline
\classe{名词}\ton{\#H-}
\paradigme{\pcmn{:} \p{}}
\begin{définition}\peng{Heron.}\end{définition}
\begin{définition}\pcmn{鹳}\end{définition}
\begin{définition}\pfra{Héron: oiseau échassier, non migrateur.}\end{définition}
\end{entrée}

\begin{entrée}
{ɖʐv̩˧qʰɑ˧}{}{ⓔɖʐv̩˧qʰɑ˧}\formedesurface{ɖʐv̩˧qʰɑ˧}\newline
\classe{名词}\ton{M}\begin{définition}\peng{Dew.}\end{définition}
\begin{définition}\pcmn{露水}\end{définition}
\begin{définition}\pfra{Rosée.}\end{définition}
\end{entrée}

\begin{entrée}
{ɖʐv̩˩ti\#˥}{}{ⓔɖʐv̩˩ti\#˥}\formedesurface{ɖʐv̩˩ti˥}\newline
\classe{名词}\ton{LM+\#H}\begin{définition}\peng{Spear.}\end{définition}
\begin{définition}\pcmn{矛}\end{définition}
\begin{définition}\pfra{Lance.}\end{définition}
\end{entrée}

\begin{entrée}
{ɖʐv̩˧tsi˥}{}{ⓔɖʐv̩˧tsi˥}\newline
\classe{名词}
\sens{1}
\begin{relationsémantique}\{
renvoi
ɖʐv̩˥
}\end{relationsémantique}\paradigme{\pcmn{:} \p{}}
\begin{définition}\peng{Artery.}\end{définition}
\begin{définition}\pcmn{动脉}\end{définition}
\begin{définition}\pfra{Artère (du corps humain).}\end{définition}\sens{2}
\begin{définition}\peng{Stem, stalk.}\end{définition}
\begin{définition}\pcmn{茎}\end{définition}
\begin{définition}\pfra{Tige (d'une plante).}\end{définition}
\end{entrée}

\begin{entrée}
{ɖʐv̩˧ʐv̩˧-ɖʐv̩˧mi\#˥}{}{ⓔɖʐv̩˧ʐv̩˧-ɖʐv̩˧mi\#˥}\formedesurface{ɖʐv̩˧ʐv̩˧-ɖʐv̩˧mi˧}\newline
\classe{名词}\ton{\#H}\begin{définition}\peng{Friend, companion, partner.}\end{définition}
\begin{définition}\pcmn{朋友、伙伴、伴侣}\end{définition}
\begin{définition}\pfra{Ami(e).}\end{définition}
\end{entrée}

\begin{entrée}
{ɖʐwæ˥}{}{ⓔɖʐwæ˥}\formedesurface{ɖʐwæ˧}\newline
\classe{名词}\ton{\#H}
\paradigme{\pcmn{:} \p{}}
\begin{définition}\peng{Small hoe (smaller than \stylefv{/hwæ}˧pʰæ˩/).}\end{définition}
\begin{définition}\pcmn{锄头}\end{définition}
\begin{définition}\pfra{Petite houe (plus petite que \stylefv{/hwæ}˧pʰæ˩/).}\end{définition}
\end{entrée}

\begin{entrée}
{ɖʐwæ˧˥}{}{ⓔɖʐwæ˧˥}\formedesurface{ɖʐwæ˧˥}\newline
\classe{动词}\ton{MH}\begin{définition}\peng{To fall down; to release, to drop.}\end{définition}
\begin{définition}\pcmn{掉下}\end{définition}
\begin{définition}\pfra{Tomber; laisser tomber, lâcher (un objet qu'on tenait à la main).}\end{définition}
\begin{exemple}\pnru{mv̩˩tɕo˧ ɖʐwæ˧˥ / mv̩˩tɕo˧ ɖʐwæ˧-ze˥}\hspace{5pt}\peng{to fall down}\hspace{5pt}\pcmn{掉下去(+了)}\hspace{5pt}\pfra{tomber par terre; littéralement «tomber vers le bas»}\end{exemple}
\begin{exemple}\pnru{hæ̃˧ ɳɯ˧ | mv̩˧ʈʰæ˧ ɖʐwæ˩.}\hspace{5pt}\peng{The wind made it fall to the ground. (About a piece of clothing that was hung on a tree, on a hanger, to dry.)}\hspace{5pt}\pcmn{风让它掉下来了!(一件衣服,挂在树上晾干,风刮起来了,衣服掉下来了)}\hspace{5pt}\pfra{Le vent l'a fait tomber! (Au sujet d'un vêtement qu'on avait suspendu à une branche d'arbre pour le faire sécher.)}\end{exemple}
\end{entrée}

\begin{entrée}
{ɖʐwæ˩˧}{}{ⓔɖʐwæ˩˧}\formedesurface{ɖʐwæ˩˥}\newline
\classe{名词}\ton{LM}
\paradigme{\pcmn{:} \p{}}
\begin{définition}\peng{Sparrow (monosyllabic form; not in common use).}\end{définition}
\begin{définition}\pcmn{麻雀}\end{définition}
\begin{définition}\pfra{Moineau (forme monosyllabique; n'est pas d'usage courant).}\end{définition}
\end{entrée}

\begin{entrée}
{ɖʐwæ˩α}{}{ⓔɖʐwæ˩α}\formedesurface{ɖʐwæ˩˥}\newline
\classe{动词}\ton{Lα}\begin{définition}\peng{To quarrel.}\end{définition}
\begin{définition}\pcmn{吵架}\end{définition}
\begin{définition}\pfra{Se disputer (monosyllabe).}\end{définition}
\begin{exemple}\pnru{ɖʐwæ˧∼ɖʐwæ˥}\hspace{5pt}\peng{|fg{red}}\hspace{5pt}\pcmn{重叠}\hspace{5pt}\pfra{|fg{red}}\end{exemple}
\end{entrée}

\begin{entrée}
{ɖʐwæ˩hi˩}{}{ⓔɖʐwæ˩hi˩}\newline
\classe{名词}
\sens{1}\paradigme{\pcmn{:} \p{}}
\begin{définition}\peng{Canine tooth, fang.}\end{définition}
\begin{définition}\pcmn{獠牙}\end{définition}
\begin{définition}\pfra{Canine (dent), croc.}\end{définition}\sens{2}
\begin{définition}\peng{Fang.}\end{définition}
\begin{définition}\pcmn{动物的牙(犬牙)}\end{définition}
\begin{définition}\pfra{Crocs (de bête).}\end{définition}
\end{entrée}

\begin{entrée}
{ɖʐwæ˧lɑ˧-ʁo˧ɖɯ˧˥}{}{ⓔɖʐwæ˧lɑ˧-ʁo˧ɖɯ˧˥}\formedesurface{ɖʐwæ˧lɑ˧ʁo˧ɖɯ˧˥}\newline
\classe{名词}\ton{MH\#}
\paradigme{\pcmn{:} \p{}}
\begin{définition}\peng{A type of sparrow.}\end{définition}
\begin{définition}\pcmn{雀}\end{définition}
\begin{définition}\pfra{Oiseau ressemblant à un moineau, au corps blanc et noir; M23 croît le reconnaître dans: Pericrocotus divaricatus, mais cette espèce n'existe que dans le nord de la Chine.}\end{définition}
\end{entrée}

\begin{entrée}
{ɖʐwæ˧mi˧}{}{ⓔɖʐwæ˧mi˧}\formedesurface{ɖʐwæ˧mi˧}\newline
\classe{名词}\ton{M}
\paradigme{\pcmn{:} \p{}}
\begin{définition}\peng{Sparrow.}\end{définition}
\begin{définition}\pcmn{麻雀}\end{définition}
\begin{définition}\pfra{Moineau.}\end{définition}
\end{entrée}

\begin{entrée}
{ɖʐwæ˧pʰv̩\#˥}{}{ⓔɖʐwæ˧pʰv̩\#˥}\formedesurface{ɖʐwæ˧pʰv̩˧}\newline
\classe{名词}\ton{\#H}
\paradigme{\pcmn{:} \p{}}
\begin{définition}\peng{Male sparrow.}\end{définition}
\begin{définition}\pcmn{公麻雀}\end{définition}
\begin{définition}\pfra{Moineau mâle.}\end{définition}
\begin{exemple}\pnru{ɖʐwæ˧pʰv̩˧ tʰv̩˧-mi˧˥ / ɖʐwæ˧pʰv̩˧ tʰv̩˧-mi˥\#}\hspace{5pt}\peng{|fg{n}+|fg{dem}+|fg{clf}}\hspace{5pt}\pcmn{这只公麻雀}\hspace{5pt}\pfra{|fg{n}+|fg{dem}+|fg{clf}}\end{exemple}
\end{entrée}

\begin{entrée}
{ɖʐwæ˧zo\#˥}{}{ⓔɖʐwæ˧zo\#˥}\formedesurface{ɖʐwæ˧zo˧}\newline
\classe{名词}\ton{\#H}
\paradigme{\pcmn{:} \p{}}
\begin{définition}\peng{Baby sparrow, little sparrow.}\end{définition}
\begin{définition}\pcmn{小麻雀}\end{définition}
\begin{définition}\pfra{Moinillon, petit moineau, bébé moineau.}\end{définition}
\end{entrée}

\newpage\caractère{ə}

\begin{entrée}
{ə}{}{ⓔə}\formedesurface{ə}\newline
\classe{感叹词}\ton{--}\begin{définition}\peng{Interjection.}\end{définition}
\begin{définition}\pcmn{感叹词}\end{définition}
\begin{définition}\pfra{Interjection.}\end{définition}
\end{entrée}

\begin{entrée}
{ə˩‑}{}{ⓔə˩‑}\formedesurface{--}\newline
\classe{代词}\ton{L}\begin{définition}\peng{Total interrogation.}\end{définition}
\begin{définition}\pcmn{……吗?}\end{définition}
\begin{définition}\pfra{Interrogation totale.}\end{définition}
\begin{exemple}\pnru{dʑɯ˧ | ə˩-dʑo˧?}\hspace{5pt}\peng{Is there any water?}\hspace{5pt}\pcmn{有谁吗?}\hspace{5pt}\pfra{est-ce qu’il y a de l’eau ?}\end{exemple}
\begin{exemple}\pnru{ə˩-ŋi˩˥ ?}\hspace{5pt}\peng{Is that right? / Is that correct? / … isn't it?}\hspace{5pt}\pcmn{对吗? / 对吧?}\hspace{5pt}\pfra{Est-ce que c’est ça ?/ C'est bien ça? … n’est-ce pas ?}\end{exemple}
\end{entrée}

\begin{entrée}
{ə˧bɑ˩-lɑ˩bɑ˩}{}{ⓔə˧bɑ˩-lɑ˩bɑ˩}\formedesurface{ə˧bɑ˩lɑ˩bɑ˩}\newline
\classe{名词}\ton{L\#-}\begin{définition}\peng{Cactus.}\end{définition}
\begin{définition}\pcmn{仙人掌}\end{définition}
\begin{définition}\pfra{Cactus.}\end{définition}
\begin{exemple}\pnru{ə˧bɑ˩-lɑ˩bɑ˩ | ɖɯ˧-dzi˩}\hspace{5pt}\peng{a cactus plant}\hspace{5pt}\pcmn{一棵仙人掌}\hspace{5pt}\pfra{un cactus}\end{exemple}
\end{entrée}

\begin{entrée}
{ə˧bo˥\$}{}{ⓔə˧bo˥\$}\formedesurface{ə˧bo˥}\newline
\classe{名词}\ton{H\$}
\paradigme{\pcmn{:} \p{}}
\begin{définition}\peng{Paternal uncle.}\end{définition}
\begin{définition}\pcmn{父亲的兄弟}\end{définition}
\begin{définition}\pfra{Oncle paternel=frère du père (sens vérifié: renvoie à la famille du père).}\end{définition}
\begin{exemple}\pnru{ə˧bo˧-ɖɯ˧˥}\hspace{5pt}\peng{paternal uncle, father's elder brother}\hspace{5pt}\pcmn{伯父:父亲的哥哥}\hspace{5pt}\pfra{oncle paternel aîné du père}\end{exemple}
\begin{exemple}\pnru{ə˧bo˧-tɕi˥ (+ɲi˩)}\hspace{5pt}\peng{paternal uncle, father's younger brother}\hspace{5pt}\pcmn{叔叔:父亲的弟弟}\hspace{5pt}\pfra{oncle paternel cadet du père}\end{exemple}
\end{entrée}

\begin{entrée}
{ə˧bo˧tɕo˧li˧}{}{ⓔə˧bo˧tɕo˧li˧}\formedesurface{ə˧bo˧tɕo˧li˧}\newline
\classe{名词}\ton{M}
\paradigme{\pcmn{:} \p{}}
\begin{définition}\peng{Cricket.}\end{définition}
\begin{définition}\pcmn{蟋蟀}\end{définition}
\begin{définition}\pfra{Criquet.}\end{définition}
\end{entrée}

\begin{entrée}
{ə˧bv̩˩}{}{ⓔə˧bv̩˩}\formedesurface{ə˧bv̩˩}\newline
\classe{名词}\ton{L\#}
\paradigme{\pcmn{:} \p{}}
\begin{définition}\peng{Oven to make bricks, ceramics…}\end{définition}
\begin{définition}\pcmn{烤砖、陶器等用的烤炉}\end{définition}
\begin{définition}\pfra{Four où on cuit les briques, les objets en céramique…}\end{définition}
\end{entrée}

\begin{entrée}
{ə˧bv̩˧-ʁwɤ˧}{}{ⓔə˧bv̩˧-ʁwɤ˧}\formedesurface{ə˧bv̩˧ʁwɤ˧}\newline
\classe{名词}\ton{M}\begin{définition}\peng{Name of a village.}\end{définition}
\begin{définition}\pcmn{阿布瓦村}\end{définition}
\begin{définition}\pfra{Abuwa (nom de village).}\end{définition}
\end{entrée}

\begin{entrée}
{ə˧ɕjɤ˩}{}{ⓔə˧ɕjɤ˩}\formedesurface{ə˧ɕjɤ˩}\newline
\classe{名词}\ton{L\#}
\paradigme{\pcmn{:} \p{}}
\begin{définition}\peng{Lover, boy/girl-friend.}\end{définition}
\begin{définition}\pcmn{情人}\end{définition}
\begin{définition}\pfra{Petit(e) ami(e), amant(e).}\end{définition}
\end{entrée}

\begin{entrée}
{ə˧ɕjo˩}{}{ⓔə˧ɕjo˩}\formedesurface{ə˧ɕjo˩}\newline
\classe{名词}\ton{L\#}\begin{définition}\peng{A family name from Yongning. There are two families in Yongning that carry this name.}\end{définition}
\begin{définition}\pcmn{一个姓。这个姓,永宁有两家}\end{définition}
\begin{définition}\pfra{Nom de clan/famille étendue. Deux familles portent ce nom à Yongning.}\end{définition}
\begin{exemple}\pnru{ə˧ɕjo˩=ɻ̍˩}\hspace{5pt}\peng{the /ə˧ɕjo˩/ clan}\hspace{5pt}\pcmn{|fv{/ə˧ɕjo˩/}家族}\hspace{5pt}\pfra{le clan /ə˧ɕjo˩/}\end{exemple}
\end{entrée}

\begin{entrée}
{ə˧dɑ˥\$}{}{ⓔə˧dɑ˥\$}\formedesurface{ə˧dɑ˥}\newline
\classe{名词}\ton{H\$}
\paradigme{\pcmn{:} \p{}}
\begin{définition}\peng{Father.}\end{définition}
\begin{définition}\pcmn{父亲}\end{définition}
\begin{définition}\pfra{Père.}\end{définition}
\end{entrée}

\begin{entrée}
{ə˧dɑ˧-ə˧mi\#˥}{}{ⓔə˧dɑ˧-ə˧mi\#˥}\formedesurface{ə˧dɑ˧ə˧mi˧}\newline
\classe{名词}\ton{\#H}\begin{définition}\peng{Father and mother.}\end{définition}
\begin{définition}\pcmn{父母}\end{définition}
\begin{définition}\pfra{Père et mère.}\end{définition}
\begin{exemple}\pnru{ə˧dɑ˧-ə˧mi˧ ɲi˥-kv̩˩}\hspace{5pt}\peng{the father and mother, as a pair}\hspace{5pt}\pcmn{父母亲}\hspace{5pt}\pfra{le père et la mère, tous les deux; le couple formé du père et de la mère}\end{exemple}
\end{entrée}

\begin{entrée}
{ə˧dɑ˧-zo\#˥}{}{ⓔə˧dɑ˧-zo\#˥}\formedesurface{ə˧dɑ˧zo˧}\newline
\classe{名词}\ton{\#H}\begin{définition}\peng{Father and son.}\end{définition}
\begin{définition}\pcmn{父子}\end{définition}
\begin{définition}\pfra{Père et fils.}\end{définition}
\end{entrée}

\begin{entrée}
{ə˧dze˧}{}{ⓔə˧dze˧}\formedesurface{ə˧dze˧}\newline
\classe{名词}\ton{M}\begin{définition}\peng{Purple gromwell, red-root gromwell, |\stylefi{Lithospermum erythrorhizon Sieb. et Zucc.}.}\end{définition}
\begin{définition}\pcmn{紫草}\end{définition}
\begin{définition}\pfra{Grémil des teinturiers, |\stylefi{Lithospermum erythrorhizon Sieb. et Zucc.}.}\end{définition}
\begin{exemple}\pnru{ə˧dze˧-njɤ˩hṽ̩˩}\hspace{5pt}\peng{same meaning: purple gromwell}\hspace{5pt}\pcmn{紫草}\hspace{5pt}\pfra{même sens}\end{exemple}
\begin{exemple}\pnru{ə˧dze˧-bæ˩bæ˩}\hspace{5pt}\peng{gromwell flowers}\hspace{5pt}\pcmn{紫草花}\hspace{5pt}\pfra{fleurs de grémil}\end{exemple}
\end{entrée}

\begin{entrée}
{ə˧-dzɤ˥\$}{}{ⓔə˧-dzɤ˥\$}\formedesurface{ə˧dzɤ˥}\newline
\classe{助词}\ton{H\$}\begin{définition}\peng{Slowly.}\end{définition}
\begin{définition}\pcmn{慢}\end{définition}
\begin{définition}\pfra{Lentement, doucement.}\end{définition}
\begin{exemple}\pnru{ə˧-dzɤ˧ ʝi˧}\hspace{5pt}\peng{to work slowly, to do slowly}\hspace{5pt}\pcmn{慢慢做}\hspace{5pt}\pfra{travailler lentement, faire lentement}\end{exemple}
\begin{exemple}\pnru{ə˧dzɤ˧ le˧-hõ˩! |}\hspace{5pt}\peng{Goodbye! (Said by the host to their guest. Literally: “Walk slowly!" = “Take your time on the way!")}\hspace{5pt}\pcmn{慢走!}\hspace{5pt}\pfra{Au revoir! (Dit par l'hôte à la personne qui s'en va. Littéralement: «Allez doucement!» / «Prenez votre temps en chemin!»)}\end{exemple}
\begin{exemple}\pnru{ə˧dzɤ˥ | le˧-hõ˩! |}\hspace{5pt}\peng{Goodbye!}\hspace{5pt}\pcmn{慢走!}\hspace{5pt}\pfra{Au revoir!}\end{exemple}
\begin{exemple}\pnru{ə˧dzɤ˧ le˧-dzi˩! |}\hspace{5pt}\peng{Goodbye! (Said by the guest to their host. Literally: “Sit quietly!" = “Take it easy!")}\hspace{5pt}\pcmn{慢慢坐!}\hspace{5pt}\pfra{Au revoir! (Dit par l'invité à son hôte. Littéralement: «Restez assis doucement = tranquillement!»}\end{exemple}
\end{entrée}

\begin{entrée}
{ə˧-dzɤ˧∼dzɤ˥}{}{ⓔə˧-dzɤ˧∼dzɤ˥}\formedesurface{ə˧dzɤ˧dzɤ˥}\newline
\classe{助词}\ton{H\#}
\étymologie{
ə˧-dzɤ˥\$, ə˧ze˧
}\begin{définition}\peng{Slowly.}\end{définition}
\begin{définition}\pcmn{慢慢地}\end{définition}
\begin{définition}\pfra{Lentement, doucement.}\end{définition}
\begin{exemple}\pnru{ʈʂʰɯ˧ | ɖwæ˧˥ | ə˧-dzɤ˧∼dzɤ˥ ʝi˩-kv̩˩!}\hspace{5pt}\peng{(S)he works very carefully. (The literal meaning is ‘very slowly'; this is not a criticism, however: it means that they know to take their time in order to do a good job.)}\hspace{5pt}\pcmn{他工作很细致。(直译:‘他工作很慢’,但不是批评:意味着那个人懂得慢慢来做,做得更仔细。)}\hspace{5pt}\pfra{Il/elle travaille avec grand soin. (Le sens littéral est «Il/elle travaille très lentement», mais cela n'est pas une critique: cela signifie qu'il/elle sait prendre le temps pour réaliser du bon travail.)}\end{exemple}
\begin{exemple}\pnru{ə˧-dzɤ˧∼dzɤ˥ ʝi˩}\hspace{5pt}\peng{to do (something) slowly}\hspace{5pt}\pcmn{慢慢地做}\hspace{5pt}\pfra{travailler lentement, faire lentement}\end{exemple}
\begin{exemple}\pnru{ə˧-zɤ˧∼zɤ˥ ʝi˩}\hspace{5pt}\peng{to do (something) slowly}\hspace{5pt}\pcmn{慢慢地做}\hspace{5pt}\pfra{Travaille doucement! / Prends ton temps! / travailler lentement, faire lentement}\end{exemple}
\end{entrée}

\begin{entrée}
{ə˧ɖo˧}{}{ⓔə˧ɖo˧}\formedesurface{ə˧ɖo˧}\newline
\classe{名词}\ton{M}
\paradigme{\pcmn{:} \p{}}
\begin{définition}\peng{Lover, boy/girl-friend.}\end{définition}
\begin{définition}\pcmn{情人(音译:阿注)}\end{définition}
\begin{définition}\pfra{Petit ami, petite amie, amant(e).}\end{définition}
\end{entrée}

\begin{entrée}
{ə˧go˧}{}{ⓔə˧go˧}\formedesurface{ə˧go˧}\newline
\classe{名词}\ton{M}\begin{définition}\peng{A family name from Yongning. There are three families in Yongning that carry this name.}\end{définition}
\begin{définition}\pcmn{一个姓。这个姓,永宁有三家}\end{définition}
\begin{définition}\pfra{Nom de clan/famille étendue. Trois familles portent ce nom à Yongning.}\end{définition}
\begin{exemple}\pnru{ə˧go˧=ɻ̍˩}\hspace{5pt}\peng{the /ə˧go˧/ clan}\hspace{5pt}\pcmn{|fv{/ə˧go˧/}家族}\hspace{5pt}\pfra{le clan /ə˧go˧/}\end{exemple}
\begin{exemple}\pnru{ə˧go˧ | dʑɤ˩tsʰi˧}\hspace{5pt}\peng{The person called /dʑɤ˩tsʰi\#˥/, of the /ə˧go˧/ clan}\hspace{5pt}\pcmn{|fv{/ə˧go˧/} 家族名叫|fv{/dʑɤ˩tsʰi\#˥/}那个人}\hspace{5pt}\pfra{la personne prénommée /dʑɤ˩tsʰi\#˥/, du clan /ə˧go˧/}\end{exemple}
\end{entrée}

\begin{entrée}
{ə˧go˧-ʁwɤ˧}{}{ⓔə˧go˧-ʁwɤ˧}\formedesurface{ə˧go˧ʁwɤ˧}\newline
\classe{名词}\ton{M}\begin{définition}\peng{Name of a village of the Hot Springs area.}\end{définition}
\begin{définition}\pcmn{阿公瓦村:温泉乡的一个村落}\end{définition}
\begin{définition}\pfra{Un village proche de Wenquan.}\end{définition}
\begin{exemple}\pnru{ə˧go˧-ʁwɤ˧, | ʁwɤ˧lɑ˩-bi˩, | bæ˧ʁwɤ˧, | tʰo˧tsʰe\#˥, | pi˧tsʰe˩-di˩, | pɤ˧dʑɤ˩-di˩, | ʁwɤ˧tv̩˧}\hspace{5pt}\peng{Seven villages that one encounters as one leaves the plain of Yongning (towards the Lake); the first two are perceived as villages with a high proportion of Na members, and the third as a mostly Na village, whereas the next two are Pumi (Prinmi); the last used to be predominantly Pumi, but as of the 2010s, it had an important Chinese (Han) population.}\hspace{5pt}\pcmn{永宁背向泸沽湖方向经过的七个村落:阿公瓦、瓦拉比、巴瓦、拖其、比其地、巴甲地、瓦都。前两个村落拥有相当大的摩梭人口比例,第三主要是摩梭村。拖其、比其地、巴甲地是普米村。瓦都,过去主要是普米族村,到了2010年代有了相当多的汉族人口。}\hspace{5pt}\pfra{Sept villages au sortir de la plaine de Yongning, dans la direction du Lac; les deux premiers comportent une population na; le troisième est un village na; les deux suivants sont essentiellement des villages pumi/prinmi; le dernier était un village pumi, et a désormais (dans les années 2010) une importante population chinoise (han).}\end{exemple}
\end{entrée}

\begin{entrée}
{ə˧gɯ˩}{}{ⓔə˧gɯ˩}\formedesurface{ə˧gɯ˩}\newline
\classe{名词}\ton{L\#}
\paradigme{\pcmn{:} \p{}}
\begin{définition}\peng{Peppermint.}\end{définition}
\begin{définition}\pcmn{薄荷}\end{définition}
\begin{définition}\pfra{Menthe.}\end{définition}
\end{entrée}

\begin{entrée}
{ə˧hɑ˩-bɑ˩lɑ˩}{}{ⓔə˧hɑ˩-bɑ˩lɑ˩}\formedesurface{ə˧hɑ˩bɑ˩lɑ˩}\newline
\classe{名词}\ton{L\#-}
\paradigme{\pcmn{:} \p{}}
\begin{définition}\peng{Traditional song.}\end{définition}
\begin{définition}\pcmn{民歌}\end{définition}
\begin{définition}\pfra{Chanson traditionnelle.}\end{définition}
\begin{exemple}\pnru{ə˧hɑ˩bɑ˩lɑ˩ | ɖɯ˧-ɖʐo˩ gwɤ˩}\hspace{5pt}\peng{to sing a song}\hspace{5pt}\pcmn{唱一首摩梭歌}\hspace{5pt}\pfra{chanter une chanson}\end{exemple}
\end{entrée}

\begin{entrée}
{ə˧hĩ˥}{}{ⓔə˧hĩ˥}\formedesurface{ə˧hĩ˥}\newline
\classe{助词}\ton{H\#}\begin{définition}\peng{In a moment.}\end{définition}
\begin{définition}\pcmn{一会儿、待会儿、等一下}\end{définition}
\begin{définition}\pfra{Dans un moment.}\end{définition}
\begin{exemple}\pnru{ə˧hĩ˥-ɳɯ˩, | li˧-kʰɯ˧-bi˥!}\hspace{5pt}\peng{I will show you in a moment!}\hspace{5pt}\pcmn{待会儿,我给你看吧!}\hspace{5pt}\pfra{Tout à l’heure je vais te montrer! / dans un moment, je te montrerai!}\end{exemple}
\end{entrée}

\begin{entrée}
{ə˧hwɤ˧}{}{ⓔə˧hwɤ˧}\formedesurface{ə˧hwɤ˧}\newline
\classe{助词}\ton{M}\begin{définition}\peng{Yesterday evening.}\end{définition}
\begin{définition}\pcmn{昨晚}\end{définition}
\begin{définition}\pfra{Hier soir.}\end{définition}
\begin{exemple}\pnru{ə˧hwɤ˧ | mv̩˩kʰv̩˧˥}\hspace{5pt}\peng{yesterday evening, during the night}\hspace{5pt}\pcmn{昨晚,夜里}\hspace{5pt}\pfra{hier au soir, dans la nuit}\end{exemple}
\end{entrée}

\begin{entrée}
{ə˧hwɤ˧-zo˧hṽ̩˧˥}{}{ⓔə˧hwɤ˧-zo˧hṽ̩˧˥}\formedesurface{ə˧hwɤ˧zo˧hṽ̩˧˥}\newline
\classe{名词}\ton{-MH\#}\begin{définition}\peng{Newborn baby.}\end{définition}
\begin{définition}\pcmn{婴儿}\end{définition}
\begin{définition}\pfra{Bébé, nouveau-né, nourrisson.}\end{définition}
\end{entrée}

\begin{entrée}
{ə˧jɤ˩}{}{ⓔə˧jɤ˩}\formedesurface{ə˧jɤ˩}\newline
\classe{名词}\ton{L\#}
\paradigme{\pcmn{:} \p{}}
\begin{définition}\peng{Maternal aunt (mother's elder sister).}\end{définition}
\begin{définition}\pcmn{姨母 (比母亲大)}\end{définition}
\begin{définition}\pfra{Tante maternelle (sœur aînée de la mère).}\end{définition}
\end{entrée}

\begin{entrée}
{ə˧ʝi˥\$}{}{ⓔə˧ʝi˥\$}\formedesurface{ə˧ʝi˥}\newline
\classe{助词}\ton{H\$}\begin{définition}\peng{Last year.}\end{définition}
\begin{définition}\pcmn{去年}\end{définition}
\begin{définition}\pfra{L'année dernière, l'année passée, l'an passé, l'an dernier.}\end{définition}
\end{entrée}

\begin{entrée}
{ə˧ʝi˧-ʂɯ˥ʝi˩}{}{ⓔə˧ʝi˧-ʂɯ˥ʝi˩}\formedesurface{ə˧ʝi˧ʂɯ˥ʝi˩}\newline
\classe{助词}\ton{\#H-}\begin{définition}\peng{Long ago; in the past; once upon a time.}\end{définition}
\begin{définition}\pcmn{很久以前,古时候,传说古代}\end{définition}
\begin{définition}\pfra{Jadis, aux temps anciens, il était une fois.}\end{définition}
\end{entrée}

\begin{entrée}
{ə˧ʝi˧-tsʰi˧ʝi\#˥}{}{ⓔə˧ʝi˧-tsʰi˧ʝi\#˥}\formedesurface{ə˧ʝi˧tsʰi˧ʝi˧}\newline
\classe{助词}\ton{\#H}\begin{définition}\peng{These years, currently.}\end{définition}
\begin{définition}\pcmn{这几年、现在这个时代}\end{définition}
\begin{définition}\pfra{Ces années-ci, actuellement.}\end{définition}
\end{entrée}

\begin{entrée}
{ə˩kʰɯ˩}{}{ⓔə˩kʰɯ˩}\formedesurface{ə˩kʰɯ˩˥}\newline
\classe{名词}\ton{L}
\paradigme{\pcmn{:} \p{}}
\begin{définition}\peng{Turnip, wild cabbage, |\stylefi{Brassica rapa}.}\end{définition}
\begin{définition}\pcmn{芜菁 、扁萝卜、大头菜、蔓菁}\end{définition}
\begin{définition}\pfra{Navet, |\stylefi{Brassica rapa}.}\end{définition}
\begin{exemple}\pnru{ə˩kʰɯ˩-bv̩˧ | kʰɯ˩ʈɯ˩˥}\hspace{5pt}\peng{the root of wild cabbage}\hspace{5pt}\pcmn{芜菁的根}\hspace{5pt}\pfra{racine de navet}\end{exemple}
\end{entrée}

\begin{entrée}
{ə˧lɑ˧}{}{ⓔə˧lɑ˧}\formedesurface{ə˧lɑ˧}\newline
\classe{名词}\ton{M}\begin{définition}\peng{A family name from Yongning. There are three families in Yongning that carry this name. This is one of the three first Na clans who settled in the vicinity of the monastery, the other two being \stylefv{/kɤ}˧˥tʰɑ˩/ and \stylefv{/lɑ}˧tʰɑ˧mi˥\$/.}\end{définition}
\begin{définition}\pcmn{一个姓。这个姓,永宁有三家}\end{définition}
\begin{définition}\pfra{Nom de clan/famille étendue. Trois familles portent ce nom à Yongning. C'est l'un des trois clans qui se sont établis les premiers dans le voisinage du monastère de Yongning, les deux autres étant \stylefv{/kɤ}˧˥tʰɑ˩/ et \stylefv{/lɑ}˧tʰɑ˧mi˥\$/.}\end{définition}
\begin{exemple}\pnru{ə˧lɑ˧=ɻ̍˩}\hspace{5pt}\peng{the /ə˧lɑ˧/ clan}\hspace{5pt}\pcmn{|fv{/ə˧lɑ˧/}家族}\hspace{5pt}\pfra{le clan /ə˧lɑ˧/}\end{exemple}
\end{entrée}

\begin{entrée}
{ə˧lɑ˧-ʁwɤ\#˥}{}{ⓔə˧lɑ˧-ʁwɤ\#˥}\formedesurface{ə˧lɑ˧ʁwɤ˧}\newline
\classe{名词}\ton{\#H}\begin{définition}\peng{A hamlet of Yongning, close to the monastery.}\end{définition}
\begin{définition}\pcmn{阿拉瓦村:永宁寺旁边的村落(主合作人出生的地方)。(旧名:七家村,因为村落在1960年左右有七个家庭)}\end{définition}
\begin{définition}\pfra{Un hameau de Yongning, proche du monastère (lieu de naissance de la consultante principale). Nom chinois: Alawa.}\end{définition}
\begin{exemple}\pnru{dʑɤ˩bv̩˧kɤ˧-sɑ˥ʁwɤ˩, | hi˩ʁwɤ˩-lo˥, | æ˩mi˧-ʁwɤ\#˥, | lɑ˧lo˧-ʁwɤ˥, | lɑ˧ŋwɤ˧, | bɤ˧tsʰo˧gv̩˥, | ə˧lɑ˧-ʁwɤ\#˥, | gæ˧ɻæ˩, | qʰæ˧tɕʰi˧, | tʰo˧ʈɯ\#˥}\hspace{5pt}\peng{The ten Na villages considered in traditional geography as belonging to the vicinity of the Yongning temple.}\hspace{5pt}\pcmn{永宁摩梭地理概念中,距离扎美寺最近的十个村落:佳部嘎萨瓦、习瓦洛、阿咪瓦、拉洛瓦、拉瓦、巴搓古、阿拉瓦、嘎尔、开基、拖支。}\hspace{5pt}\pfra{Les dix villages na traditionnellement considérés comme appartenant au voisinage du temple de Yongning.}\end{exemple}
\end{entrée}

\begin{entrée}
{ə˩ljɤ˩hæ̃˩ʂɯ˥-mo˩}{}{ⓔə˩ljɤ˩hæ̃˩ʂɯ˥-mo˩}\formedesurface{ə˩ljɤ˩hæ̃˩ʂɯ˥mo˩}\newline
\classe{名词}\ton{L+H\#-}\begin{définition}\peng{A sort of mushroom: |\stylefi{Hygrophorus lucorum Kalc hbr.}.}\end{définition}
\begin{définition}\pcmn{柠檬黄蜡伞(一种菌子)}\end{définition}
\begin{définition}\pfra{Grand champignon jaune vif, comestible: |\stylefi{Hygrophorus lucorum Kalc hbr.}. Littéralement «champignon doré».}\end{définition}
\end{entrée}

\begin{entrée}
{ə˧mɑ˧}{}{ⓔə˧mɑ˧}\formedesurface{ə˧mɑ˧}\newline
\classe{名词}\ton{M}
\paradigme{\pcmn{:} \p{}}
\begin{définition}\peng{Mother (term of address used by children).}\end{définition}
\begin{définition}\pcmn{阿妈(孩子对母亲的称呼)}\end{définition}
\begin{définition}\pfra{Mère (terme d'adresse).}\end{définition}
\end{entrée}

\begin{entrée}
{ə˧mi˧}{}{ⓔə˧mi˧}\formedesurface{ə˧mi˧}\newline
\classe{名词}\ton{M}
\paradigme{\pcmn{:} \p{}}
\begin{définition}\peng{Mother; aunt.}\end{définition}
\begin{définition}\pcmn{母亲、姑母、姨母、伯母、叔母、大娘、婶、大妈、姨、伯母、舅母、大婶、大姨、阿姨、妗母、妗子、舅妈、婶母、婶娘、婶子、叔母、姨妈、姨母、姨娘}\end{définition}
\begin{définition}\pfra{Mère; le terme s'emploie aussi pour désigner les tantes.}\end{définition}
\begin{exemple}\pnru{ə˧mi˧=ɻæ˩}\hspace{5pt}\peng{\_ |fg{associative}}\hspace{5pt}\pcmn{母亲们 =长辈女性}\hspace{5pt}\pfra{\_ |fg{associatif}: les mères =les femmes de la génération supérieure}\end{exemple}
\end{entrée}

\begin{entrée}
{ə˧mi˧-ɖɯ˩}{}{ⓔə˧mi˧-ɖɯ˩}\formedesurface{ə˧mi˧ɖɯ˩}\newline
\classe{名词}\ton{L\#}
\paradigme{\pcmn{:} \p{}}
\begin{définition}\peng{Maternal aunt (mother's elder sister).}\end{définition}
\begin{définition}\pcmn{姨母 (比母亲大)}\end{définition}
\begin{définition}\pfra{Tante maternelle (sœur aînée de la mère).}\end{définition}
\end{entrée}

\begin{entrée}
{ə˧mi˧-mv̩˩}{}{ⓔə˧mi˧-mv̩˩}\formedesurface{ə˧mi˧mv̩˩}\newline
\classe{名词}\ton{L\#}\begin{définition}\peng{Mother and daughter.}\end{définition}
\begin{définition}\pcmn{母女,母亲与女孩}\end{définition}
\begin{définition}\pfra{Mère et fille.}\end{définition}
\end{entrée}

\begin{entrée}
{ə˧mi˧-tɕi˩}{}{ⓔə˧mi˧-tɕi˩}\formedesurface{ə˧mi˧tɕi˩}\newline
\classe{名词}\ton{L\#}
\paradigme{\pcmn{:} \p{}}
\begin{définition}\peng{Maternal aunt (mother's younger sister).}\end{définition}
\begin{définition}\pcmn{姨母 (比母亲小)}\end{définition}
\begin{définition}\pfra{Tante (soeur cadette de la mère).}\end{définition}
\end{entrée}

\begin{entrée}
{ə˧mi˧-ze˩mi˩}{}{ⓔə˧mi˧-ze˩mi˩}\formedesurface{ə˧mi˧ze˩mi˩}\newline
\classe{名词}\ton{-L}\begin{définition}\peng{Aunt and niece.}\end{définition}
\begin{définition}\pcmn{姑母、姨母、伯母、或叔母(=婶母)与侄女或外甥女}\end{définition}
\begin{définition}\pfra{Tante et nièce.}\end{définition}
\end{entrée}

\begin{entrée}
{ə˧mi˧-zo\#˥}{}{ⓔə˧mi˧-zo\#˥}\formedesurface{ə˧mi˧zo˧}\newline
\classe{名词}\ton{\#H}\begin{définition}\peng{Mother and son.}\end{définition}
\begin{définition}\pcmn{母子}\end{définition}
\begin{définition}\pfra{Mère et fils.}\end{définition}
\end{entrée}

\begin{entrée}
{ə˧mv̩˩}{}{ⓔə˧mv̩˩}\formedesurface{ə˧mv̩˩}\newline
\classe{名词}\ton{L\#}
\paradigme{\pcmn{:} \p{}}
\begin{définition}\peng{Elder sibling (brother or sister).}\end{définition}
\begin{définition}\pcmn{哥哥,姐姐(也指堂哥堂姐)}\end{définition}
\begin{définition}\pfra{Aîné: grand frère, grande sœur (employé aussi entre cousins).}\end{définition}
\begin{exemple}\pnru{æ˧mv̩˩=ɻæ˩}\hspace{5pt}\peng{|fg{associative}: elder siblings}\hspace{5pt}\pcmn{联想复数:哥哥们、姐姐们}\hspace{5pt}\pfra{|fg{associatif}: les aînés dans la fratrie: sœurs et frères aînés}\end{exemple}
\end{entrée}

\begin{entrée}
{ə˧mv̩˧-gi˥zɯ˩}{}{ⓔə˧mv̩˧-gi˥zɯ˩}\formedesurface{ə˧mv̩˧gi˥zɯ˩}\newline
\classe{名词}\ton{\#H-}\begin{définition}\peng{Brothers, irrespective of age (elder or younger).}\end{définition}
\begin{définition}\pcmn{兄弟(哥哥们与弟弟们)}\end{définition}
\begin{définition}\pfra{Frères, quel que soit leur âge (aînés ou cadets).}\end{définition}
\end{entrée}

\begin{entrée}
{ə˧mv̩˧-go˧mi˥}{}{ⓔə˧mv̩˧-go˧mi˥}\formedesurface{ə˧mv̩˧go˧mi˥}\newline
\classe{名词}\ton{-H\#}\begin{définition}\peng{Sisters (elder as well as younger).}\end{définition}
\begin{définition}\pcmn{姐妹}\end{définition}
\begin{définition}\pfra{Sœurs (aînées ou cadettes).}\end{définition}
\end{entrée}

\begin{entrée}
{ə˧mv̩˥-tɕi˩}{}{ⓔə˧mv̩˥-tɕi˩}\formedesurface{ə˧mv̩˥tɕi˩}\newline
\classe{形容词}\ton{H\#-}\begin{définition}\peng{Very small, diminutive.}\end{définition}
\begin{définition}\pcmn{小、细小}\end{définition}
\begin{définition}\pfra{Petit, tout petit, riquiqui.}\end{définition}
\begin{exemple}\pnru{ə˧mv̩˥-tɕi˩-gv̩˩}\hspace{5pt}\peng{very small, diminutive}\hspace{5pt}\pcmn{小、细小}\hspace{5pt}\pfra{tout petit}\end{exemple}
\begin{exemple}\pnru{ə˧mv̩˥-tɕi˩-hĩ˩}\hspace{5pt}\peng{very small}\hspace{5pt}\pcmn{细小的}\hspace{5pt}\pfra{tout petit}\end{exemple}
\begin{exemple}\pnru{ə˧mv̩˥tɕi˩ | ɖɯ˧-kʰwɤ˥}\hspace{5pt}\peng{a little piece, a little bit}\hspace{5pt}\pcmn{一小块}\hspace{5pt}\pfra{un petit morceau}\end{exemple}
\end{entrée}

\begin{entrée}
{ə˧ɲi˥\$}{}{ⓔə˧ɲi˥\$}\formedesurface{ə˧ɲi˥}\newline
\classe{助词}\ton{H\$}\begin{définition}\peng{Yesterday.}\end{définition}
\begin{définition}\pcmn{昨天}\end{définition}
\begin{définition}\pfra{Hier.}\end{définition}
\end{entrée}

\begin{entrée}
{ə˧ɲi˧mɤ˩}{}{ⓔə˧ɲi˧mɤ˩}\formedesurface{ə˧ɲi˧mɤ˩}\newline
\classe{名词}\ton{L\#}\begin{définition}\peng{Buddhist nun.}\end{définition}
\begin{définition}\pcmn{尼姑}\end{définition}
\begin{définition}\pfra{Nonne bouddhiste.}\end{définition}
\begin{exemple}\pnru{mi˩zɯ˩-ʈæ˩bɤ˥}\hspace{5pt}\peng{woman priest}\hspace{5pt}\pcmn{尼姑}\hspace{5pt}\pfra{femme-prêtre}\end{exemple}
\end{entrée}

\begin{entrée}
{ə˧ɲi˥-tsæ˩qæ˩}{}{ⓔə˧ɲi˥-tsæ˩qæ˩}\formedesurface{ə˧ɲi˥tsæ˩qæ˩}\newline
\classe{名词}\ton{H\#-L}
\paradigme{\pcmn{:} \p{}}
\begin{définition}\peng{Little finger.}\end{définition}
\begin{définition}\pcmn{小指}\end{définition}
\begin{définition}\pfra{Auriculaire.}\end{définition}
\end{entrée}

\begin{entrée}
{ə˧ɲi˧-tsʰi˧ɲi\#˥}{}{ⓔə˧ɲi˧-tsʰi˧ɲi\#˥}\formedesurface{ə˧ɲi˧tsʰi˧ɲi˧}\newline
\classe{助词}\ton{\#H}\begin{définition}\peng{These days.}\end{définition}
\begin{définition}\pcmn{近来}\end{définition}
\begin{définition}\pfra{Ces temps-ci, ces jours-ci.}\end{définition}
\end{entrée}

\begin{entrée}
{ə˧pʰv̩˧}{}{ⓔə˧pʰv̩˧}\formedesurface{ə˧pʰv̩˧}\newline
\classe{名词}\ton{M}
\paradigme{\pcmn{:} \p{}}
\begin{définition}\peng{Grandmother's brother=mother's uncle (on her mother's side); extended meaning: male elder two generations above oneself.}\end{définition}
\begin{définition}\pcmn{舅姥爷:姥姥的哥哥或弟弟(也就是母亲的舅舅)。泛指:“祖父”}\end{définition}
\begin{définition}\pfra{Frère de la grand-mère = oncle de l'oncle maternel = oncle de la mère; sens étendu: personnage masculin important 2 générations au-dessus de soi.}\end{définition}
\end{entrée}

\begin{entrée}
{ə˩qo˥}{}{ⓔə˩qo˥}\formedesurface{ə˩qo˥}\newline
\classe{助词}\ton{LH}\begin{définition}\peng{Inward.}\end{définition}
\begin{définition}\pcmn{往里}\end{définition}
\begin{définition}\pfra{À l'intérieur, vers l'intérieur.}\end{définition}
\end{entrée}

\begin{entrée}
{ə˧si˧}{}{ⓔə˧si˧}\formedesurface{ə˧si˧}\newline
\classe{名词}\ton{M}
\paradigme{\pcmn{:} \p{}}
\begin{définition}\peng{Great-grandmother (on the mother's side); extended meaning: great-grandmother and her brothers and sisters: third-generation elders.}\end{définition}
\begin{définition}\pcmn{祖母。泛指:祖母与其兄弟姐妹}\end{définition}
\begin{définition}\pfra{Arrière-grand-mère (bisaïeule); sens étendu: bisaïeule et ses frères et soeurs, c'est-à-dire les membres importants de la famille à la 3e génération.}\end{définition}
\end{entrée}

\begin{entrée}
{ə˧si˧-ə˧pʰv̩\#˥}{}{ⓔə˧si˧-ə˧pʰv̩\#˥}\formedesurface{ə˧si˧ə˧pʰv̩˧}\newline
\classe{名词}\ton{\#H}\begin{définition}\peng{Ancestors of the third and fourth generations.}\end{définition}
\begin{définition}\pcmn{祖宗(三、四代以前)}\end{définition}
\begin{définition}\pfra{Ancêtres aux 3e et 4e générations.}\end{définition}
\end{entrée}

\begin{entrée}
{ə˧so˧}{}{ⓔə˧so˧}\formedesurface{ə˧so˧}\newline
\classe{助词}\ton{M}\begin{définition}\peng{A short time ago, a moment ago.}\end{définition}
\begin{définition}\pcmn{刚才}\end{définition}
\begin{définition}\pfra{Tout à l'heure, il y a un moment.}\end{définition}
\end{entrée}

\begin{entrée}
{ə˧-sɯ˩kv̩˩}{}{ⓔə˧-sɯ˩kv̩˩}\formedesurface{ə˧sɯ˩kv̩˩, ə˩sɯ˧kv˥}\newline
\classe{代词}\ton{-L / LMH}\begin{définition}\peng{1pl inclusive.}\end{définition}
\begin{définition}\pcmn{咱们}\end{définition}
\begin{définition}\pfra{1e pers. pl., inclusive.}\end{définition}
\begin{exemple}\pnru{ə˧-sɯ˩kv̩˩ | ɖɯ˧-tɑ˧˥ | nɑ˩ ɲi˧!}\hspace{5pt}\peng{We are Na, all of us!}\hspace{5pt}\pcmn{咱们都是摩梭人!}\hspace{5pt}\pfra{On est tous Na!}\end{exemple}
\begin{exemple}\pnru{ə˧-sɯ˩kv̩˩ lɑ˩ ɲi˩!}\hspace{5pt}\peng{We are between ourselves! (Context: a grandmother encourages her one-year-old granddaughter to defecate; there are other people present, which may be intimidating to the child, so her grandmother reassures her: “These are all family members! You can relax, there's nothing to be afraid of!"}\hspace{5pt}\pcmn{只有我们!(=没有人在看!)(情景:一位奶奶鼓励一岁的小孙女拉屎。周围有人,这可能会让小姑娘害羞,所以奶奶安慰她:“都是家人,可以轻松!”)}\hspace{5pt}\pfra{On est entre nous! (Contexte: une grand-mère encourage sa petite-fille d'un an à déféquer dans son pot, alors qu'il y a du public, ce qui pourrait être intimidant; elle la rassure: «C'est la famille/on est entre nous!»)}\end{exemple}
\end{entrée}

\begin{entrée}
{ə˧ʂɯ˧ɲi˥}{}{ⓔə˧ʂɯ˧ɲi˥}\formedesurface{ə˧ʂɯ˧ɲi˥}\newline
\classe{助词}\ton{H\#}\begin{définition}\peng{The past few days.}\end{définition}
\begin{définition}\pcmn{前几天}\end{définition}
\begin{définition}\pfra{Ces derniers jours, les jours passés.}\end{définition}
\begin{exemple}\pnru{ə˧ʂɯ˧ɲi˥ | ɖɯ˧ɲi˥}\hspace{5pt}\peng{One day, some time ago.}\hspace{5pt}\pcmn{前段时间的一天}\hspace{5pt}\pfra{Un jour, il y a quelque temps}\end{exemple}
\end{entrée}

\begin{entrée}
{ə˧ti˥-dzi˩}{}{ⓔə˧ti˥-dzi˩}\formedesurface{ə˧ti˥dzi˩}\newline
\classe{名词}\ton{H\#-}\begin{définition}\peng{The city of Weixi, in Yunnan.}\end{définition}
\begin{définition}\pcmn{维西}\end{définition}
\begin{définition}\pfra{Weixi (localité du Yunnan).}\end{définition}
\end{entrée}

\begin{entrée}
{ə˧tɕi˩}{}{ⓔə˧tɕi˩}\formedesurface{ə˧tɕi˩}\newline
\classe{名词}\ton{L\#}
\paradigme{\pcmn{:} \p{}}
\begin{définition}\peng{Maternal aunt (mother's younger sister).}\end{définition}
\begin{définition}\pcmn{姨母 (比母亲小)}\end{définition}
\begin{définition}\pfra{Tante (soeur cadette de la mère).}\end{définition}
\begin{exemple}\pnru{ə˧tɕi˩=ɻæ˩}\hspace{5pt}\peng{|fg{associative}: aunts}\hspace{5pt}\pcmn{姨母们}\hspace{5pt}\pfra{|fg{associatif}: les tantes}\end{exemple}
\end{entrée}

\begin{entrée}
{ə˧tse˥\$}{}{ⓔə˧tse˥\$}\formedesurface{ə˧tse˥}\newline
\classe{助词}\ton{H\$}\begin{définition}\peng{Why.}\end{définition}
\begin{définition}\pcmn{为什么}\end{définition}
\begin{définition}\pfra{Pourquoi.}\end{définition}
\begin{exemple}\pnru{ə˧tse˧-ʝi˥ / ə˧tse˧-ʝi˧}\hspace{5pt}\peng{Why? (Two variants; same meaning)}\hspace{5pt}\pcmn{为什么?(有两个变体,意思一致)}\hspace{5pt}\pfra{Pourquoi? (Deux variantes, même sens)}\end{exemple}
\begin{exemple}\pnru{no˧ | ə˧tse˧-ʝi˥ | mɤ˧-tsʰɯ˩ ɲi˩? / no˧ | ə˧tse˧-ʝi˧-zo˥ | mɤ˧-tsʰɯ˩ ɲi˩?}\hspace{5pt}\peng{Why didn't you come?}\hspace{5pt}\pcmn{你为什么没有来?}\hspace{5pt}\pfra{pourquoi tu ne viens pas/n'es pas venu?}\end{exemple}
\begin{exemple}\pnru{no˧ | ə˧tse˧-ʝi˥ | mɤ˧-dzɯ˥? = no˧ | ə˧tse˧-ʝi˥ | mɤ˧-dzɯ˧-ɲi˥?}\hspace{5pt}\peng{Why don't you eat?}\hspace{5pt}\pcmn{你为什么不吃?}\hspace{5pt}\pfra{pourquoi tu ne manges pas?}\end{exemple}
\begin{exemple}\pnru{ʈʂʰɯ˧ | ə˧tse˧-ɲi˥-hɯ˩?}\hspace{5pt}\peng{What does that mean?}\hspace{5pt}\pcmn{这是怎么一回事?}\hspace{5pt}\pfra{qu'est-ce que ça veut dire?}\end{exemple}
\end{entrée}

\begin{entrée}
{ə˧tso˧}{}{ⓔə˧tso˧}\formedesurface{ə˧tso˧}\newline
\classe{代词}\ton{M}\begin{définition}\peng{What.}\end{définition}
\begin{définition}\pcmn{什么}\end{définition}
\begin{définition}\pfra{|fg{interrog.quoi} (quoi, pronom interrogatif).}\end{définition}
\begin{exemple}\pnru{ə˧tso˧ ɲi˩?}\hspace{5pt}\peng{What is it?}\hspace{5pt}\pcmn{是什么?}\hspace{5pt}\pfra{Qu'est-ce que c'est?}\end{exemple}
\begin{exemple}\pnru{no˧ | ə˧tso˧ ʝi˧ bi˧?}\hspace{5pt}\peng{What are you going to do?}\hspace{5pt}\pcmn{你要做什么?}\hspace{5pt}\pfra{Qu'est-ce que tu vas faire? (Cette phrase peut se substituer à «où tu vas?», |fv{/zo˩qo˧ bi˧?/}, comme salutation adressée à quelqu'un qui est en chemin.)}\end{exemple}
\begin{exemple}\pnru{no˧ | ə˧tse˧ bi˧?}\hspace{5pt}\peng{contracted form of the previous example: /tso˧/ and the following /ʝi˧/ fuse into a single syllable, [tse˧].}\hspace{5pt}\pcmn{上面例子的缩短格式:|fv{/tso˧/}与|fv{/ʝi˧/}合成一个音节,|fv{[tse˧]}。}\hspace{5pt}\pfra{forme contractée de 2: dans [tse˧], /tso˧/ et le /ʝi˧/ suivant fusionnent en une seule syllabe.}\end{exemple}
\end{entrée}

\begin{entrée}
{ə˧tso˧∼ə˧tso˥}{}{ⓔə˧tso˧∼ə˧tso˥}\formedesurface{ə˧tso˧ə˧tso˥}\newline
\classe{代词}\ton{H\#}\begin{définition}\peng{What (reduplicated).}\end{définition}
\begin{définition}\pcmn{什么(重叠)}\end{définition}
\begin{définition}\pfra{|fg{interrog.quoi}, rédupliqué.}\end{définition}
\end{entrée}

\begin{entrée}
{ə˧tso˧-mɤ˧-ɲi˩}{}{ⓔə˧tso˧-mɤ˧-ɲi˩}\formedesurface{ə˧tso˧mɤ˧ɲi˩}\newline
\classe{代词}\ton{L\#}\begin{définition}\peng{All, all sorts of.}\end{définition}
\begin{définition}\pcmn{各种}\end{définition}
\begin{définition}\pfra{Tout, toutes les sortes de.}\end{définition}
\end{entrée}

\begin{entrée}
{ə˧v̩˧˥}{₁}{ⓔə˧v̩˧˥ⓗ1}\formedesurface{ə˧v̩˧˥}\newline
\classe{形容词}\ton{MH\#}
1\begin{définition}\peng{Beautiful, pretty.}\end{définition}
\begin{définition}\pcmn{美,好看,美丽}\end{définition}
\begin{définition}\pfra{Beau, joli.}\end{définition}
\begin{exemple}\pnru{ɖwæ˧˥ | ə˧v̩˧˥}\hspace{5pt}\peng{|fg{intensive.very}}\hspace{5pt}\pcmn{很好看!}\hspace{5pt}\pfra{|fg{intensif.très}}\end{exemple}
\begin{exemple}\pnru{ə˧-mɤ˧-v̩˧˥}\hspace{5pt}\peng{|fg{neg}: ugly}\hspace{5pt}\pcmn{丑陋}\hspace{5pt}\pfra{|fg{neg}: vilain, laid}\end{exemple}
\end{entrée}

\begin{entrée}
{ə˧v̩˧˥}{₂}{ⓔə˧v̩˧˥ⓗ2}\formedesurface{ə˧v̩˧˥}\newline
\classe{名词}\ton{MH\#}
2
\paradigme{\pcmn{:} \p{}}
\begin{définition}\peng{Maternal uncle (mother's brother: same term for older brother and younger brother).}\end{définition}
\begin{définition}\pcmn{舅舅、舅父 (比母亲大或比母亲小不区分)}\end{définition}
\begin{définition}\pfra{Oncle maternel =frère de la mère (aîné ou cadet).}\end{définition}
\begin{exemple}\pnru{ə˧v̩˧-ɖɯ˧˥}\hspace{5pt}\peng{mother's elder brother}\hspace{5pt}\pcmn{比母亲大的舅舅}\hspace{5pt}\pfra{oncle, aîné de la mère}\end{exemple}
\begin{exemple}\pnru{ə˧v̩˧-tɕi˥}\hspace{5pt}\peng{mother's younger brother}\hspace{5pt}\pcmn{比母亲小的舅舅}\hspace{5pt}\pfra{oncle, cadet de la mère}\end{exemple}
\begin{exemple}\pnru{mv˧ʁo˥ | tʰi˧-dze˩, | kɤ˩-nɑ˧mi˧ ɖɯ˧˥ ! | di˧qo˧ ʈʰɯ˧-dʑo˩, | ə˧v˧ ɖɯ˧˥!}\hspace{5pt}\peng{“As the Eagle is greatest of all that fly in the sky, so the Uncle is greatest of all that walk the earth.”}\hspace{5pt}\pcmn{“天上飞的,是老鹰最大。天下走的,是舅舅最大。”}\hspace{5pt}\pfra{“Parmi tout ce qui vole dans le ciel, l'aigle est le plus grand; parmi tout ce qui marche sur la terre, l'oncle est le plus grand.”}\end{exemple}
\begin{exemple}\pnru{mv˧ʁo˥ dze˩hĩ˩-dʑo˥, | kɤ˩-nɑ˧mi˧; | di˧qo˧ se˧-dʑo˩, | ə˧v˧˥!}\hspace{5pt}\peng{“As the Eagle is greatest of all that fly in the sky, so the Uncle is greatest of all that walk the earth.”}\hspace{5pt}\pcmn{“天上飞的,是老鹰最大。天下走的,是舅舅最大。”}\hspace{5pt}\pfra{“Parmi tout ce qui vole dans le ciel, l'aigle est le plus grand; parmi tout ce qui marche sur la terre, l'oncle est le plus grand.”}\end{exemple}
\begin{exemple}\pnru{mv˧ʁo˥ dze˩hĩ˩˥ | -dʑo˥, | kɤ˩-nɑ˧mi˧; | di˧qo˧ se˧-dʑo˩, | ə˧v˧˥!}\hspace{5pt}\peng{“As the Eagle is greatest of all that fly in the sky, so the Uncle is greatest of all that walk the earth.”}\hspace{5pt}\pcmn{“天上飞的,是老鹰最大。天下走的,是舅舅最大。”}\hspace{5pt}\pfra{“Parmi tout ce qui vole dans le ciel, l'aigle est le plus grand; parmi tout ce qui marche sur la terre, l'oncle est le plus grand.”}\end{exemple}
\begin{exemple}\pnru{mv˧ʁo˥ dze˩hĩ˩-dʑo˥, | kɤ˩-nɑ˧mi˧; | di˧qo˧-dʑo˧, | ə˧v˧˥!}\hspace{5pt}\peng{“As the Eagle is greatest of all that fly in the sky, so the Uncle is greatest of all that walk the earth.”}\hspace{5pt}\pcmn{“天上飞的,是老鹰最大。天下走的,是舅舅最大。”}\hspace{5pt}\pfra{“Parmi tout ce qui vole dans le ciel, l'aigle est le plus grand; parmi tout ce qui marche sur la terre, l'oncle est le plus grand.”}\end{exemple}
\begin{exemple}\pnru{mv˧ʁo˥ | dze˩-hĩ˩-dʑo˥, | ɖɯ˩-hĩ˩-dʑo˥, | kɤ˩-nɑ˧mi˧! | mv˧di˧-qo˥ | ɖɯ˩-hĩ˩-dʑo˥, | ə˧v˧˥!}\hspace{5pt}\peng{“As the Eagle is greatest of all that fly in the sky, so the Uncle is greatest of all that walk the earth.”}\hspace{5pt}\pcmn{“天上飞的,是老鹰最大。天下走的,是舅舅最大。”}\hspace{5pt}\pfra{“Parmi tout ce qui vole dans le ciel, l'aigle est le plus grand; parmi tout ce qui marche sur la terre, l'oncle est le plus grand.”}\end{exemple}
\end{entrée}

\begin{entrée}
{ə˧v̩˧-ze˥v̩˩}{}{ⓔə˧v̩˧-ze˥v̩˩}\formedesurface{ə˧v̩˧ze˥v̩˩}\newline
\classe{名词}\ton{\#H-}\begin{définition}\peng{Uncle and nephew.}\end{définition}
\begin{définition}\pcmn{叔叔侄子}\end{définition}
\begin{définition}\pfra{Oncle et neveu.}\end{définition}
\end{entrée}

\begin{entrée}
{ə˧ze˧}{}{ⓔə˧ze˧}\formedesurface{ə˧ze˧}\newline
\classe{助词}\ton{H\#}\begin{définition}\peng{Slowly.}\end{définition}
\begin{définition}\pcmn{慢慢地}\end{définition}
\begin{définition}\pfra{Lentement, doucement.}\end{définition}
\begin{exemple}\pnru{ə˧ze˧ le˧-hõ˩!}\hspace{5pt}\peng{Walk slowly! / Take your time on the road! / Have a quiet and pleasant journey! (Polite salutation to someone who is leaving.)}\hspace{5pt}\pcmn{慢走!}\hspace{5pt}\pfra{Salutations à quelqu'un qui s'en va: «Au revoir!», littéralement «Marche doucement!»}\end{exemple}
\begin{exemple}\pnru{ə˧ze˧ le˧-dzi˩!}\hspace{5pt}\peng{Just stay seated! (Polite salutation when leaving someone.)}\hspace{5pt}\pcmn{慢慢坐!}\hspace{5pt}\pfra{Salutation lorsqu'on quitte quelqu'un: ‘Au revoir!', littéralement ‘Reste tranquillement assis!'}\end{exemple}
\end{entrée}

\begin{entrée}
{ə˧zo˩-ʁwɤ˩}{}{ⓔə˧zo˩-ʁwɤ˩}\formedesurface{ə˧zo˩ʁwɤ˧}\newline
\classe{名词}\ton{L\#-}\begin{définition}\peng{A village close to the Hot Springs.}\end{définition}
\begin{définition}\pcmn{温泉乡的一个村落}\end{définition}
\begin{définition}\pfra{Un village proche des Sources Chaudes.}\end{définition}
\end{entrée}

\begin{entrée}
{ə˧=zɯ˩}{}{ⓔə˧=zɯ˩}\formedesurface{ə˧zɯ˩}\newline
\classe{代词}\ton{L\# / L}\begin{définition}\peng{Dual inclusive first person pronoun: us two, the two of us (the speaker and the addressee).}\end{définition}
\begin{définition}\pcmn{咱们两个}\end{définition}
\begin{définition}\pfra{Pronom duel inclusif: nous deux (le locuteur et l'interlocuteur).}\end{définition}
\end{entrée}

\begin{entrée}
{ə˧ʐv̩˩}{}{ⓔə˧ʐv̩˩}\formedesurface{ə˧ʐv̩˩}\newline
\classe{形容词}\ton{L\#}\begin{définition}\peng{Old, used.}\end{définition}
\begin{définition}\pcmn{陈旧}\end{définition}
\begin{définition}\pfra{Ancien, usagé.}\end{définition}
\begin{exemple}\pnru{ʂe˧ ʐv̩˥}\hspace{5pt}\peng{old meat, meat that is not fresh}\hspace{5pt}\pcmn{陈肉、不新鲜的肉}\hspace{5pt}\pfra{de la vieille viande, de la viande pas fraîche}\end{exemple}
\end{entrée}

\begin{entrée}
{ə˧ʑi˧˥}{}{ⓔə˧ʑi˧˥}\formedesurface{ə˧ʑi˧˥}\newline
\classe{名词}\ton{MH\#}
\paradigme{\pcmn{:} \p{}}
\begin{définition}\peng{Grandmother (on mother's side); elderly woman.}\end{définition}
\begin{définition}\pcmn{祖母,姥姥,老妪}\end{définition}
\begin{définition}\pfra{Grand-mère, aïeule; vieille femme.}\end{définition}
\begin{exemple}\pnru{ə˧ʑi˧ ʝi˧ so˥-zo˩-ho˩-ze˩!}\hspace{5pt}\peng{I shall have to learn to be a grandmother! / I shall have to learn to behave as a grandmother! (Humorous remark by the main consultant, after a doctor has advised her to avoid low, soft seats such as sofas and to adopt a taller wooden chair. Paraphrase: “I guess I have entered the category of elderly persons!")}\hspace{5pt}\pcmn{我要学习当老太太了!(情景:一位医生建议合作人不要坐在小凳子或者软沙发上了,而要坐更高的木头椅子。她幽默地说:“看来我是老年人了!”)}\hspace{5pt}\pfra{Il va falloir que j'apprenne à me comporter (sagement) comme une grand-mère! (Contexte: remarque teintée d'humour de la consultante principale face à l'âge qui vient et ses soucis: un médecin lui déconseille les sofas/assises molles et lui recommande une chaise haute en bois; elle se fait la réflexion qu'elle a vieilli et doit maintenant apprendre à prendre des précautions.)}\end{exemple}
\end{entrée}

\begin{entrée}
{ə˧ʑi˧-ə˧pʰv̩˧˥}{}{ⓔə˧ʑi˧-ə˧pʰv̩˧˥}\formedesurface{ə˧ʑi˧ə˧pʰv̩˧˥}\newline
\classe{名词}\ton{MH\#}\begin{définition}\peng{Elders by two generations: the grandmother and her brothers.}\end{définition}
\begin{définition}\pcmn{奶奶与她的兄弟}\end{définition}
\begin{définition}\pfra{La grand-mère et ses frères: les aînés 2 génération au-dessus de soi.}\end{définition}
\end{entrée}

\begin{entrée}
{ə˧ʑi˧-ʐv̩˥mi˩}{}{ⓔə˧ʑi˧-ʐv̩˥mi˩}\formedesurface{ə˧ʑi˧ʐv̩˥mi˩}\newline
\classe{名词}\ton{\#H-}\begin{définition}\peng{Grandmother and granddaughter.}\end{définition}
\begin{définition}\pcmn{奶奶与孙女}\end{définition}
\begin{définition}\pfra{Grand-mère et petite-fille.}\end{définition}
\end{entrée}

\newpage\caractère{f}

\begin{entrée}
{fɑ˩α}{}{ⓔfɑ˩α}\formedesurface{fɑ˩˥}\newline
\classe{动词}\ton{Lα}\begin{définition}\peng{To ferment.}\end{définition}
\begin{définition}\pcmn{发酵(汉语借词:发)}\end{définition}
\begin{définition}\pfra{Fermenter.}\end{définition}
\begin{exemple}\pnru{tsɑ˧bɤ˧ ɖɯ˧-mɤ˩ | tʰi˧-fɑ˩}\hspace{5pt}\peng{to make a little flour ferment, to prepare a little bread dough}\hspace{5pt}\pcmn{发一点面}\hspace{5pt}\pfra{faire lever un peu de farine}\end{exemple}
\begin{exemple}\pnru{tsɑ˧bɤ˧ tʰi˧-fɑ˩! | pɤ˩jɤ˧ gv̩˥-bi˩!}\hspace{5pt}\peng{Make some flour to ferment! We're going to prepare buns!}\hspace{5pt}\pcmn{你发一点面吧!要做馒头!}\hspace{5pt}\pfra{Fais lever de la farine, on va faire des petits pains!}\end{exemple}
\end{entrée}

\begin{entrée}
{fɑ˧tɑ˧˥}{}{ⓔfɑ˧tɑ˧˥}\formedesurface{fɑ˧tɑ˧˥}\newline
\classe{形容词}\ton{MH}\begin{définition}\peng{Developed, flourishing.}\end{définition}
\begin{définition}\pcmn{发达}\end{définition}
\begin{définition}\pfra{Développé, florissant.}\end{définition}
\begin{exemple}\pnru{fɑ˧tɑ˧-ze˥}\hspace{5pt}\peng{|fg{pfv}}\hspace{5pt}\pcmn{很发达的了}\hspace{5pt}\pfra{|fg{pfv}}\end{exemple}
\end{entrée}

\begin{entrée}
{fæ˧}{}{ⓔfæ˧}\formedesurface{fæ˧}\newline
\classe{名词}\ton{M}\begin{définition}\peng{Direction.}\end{définition}
\begin{définition}\pcmn{方(方向的方)(汉语借词)}\end{définition}
\begin{définition}\pfra{Direction.}\end{définition}
\begin{exemple}\pnru{dv̩˩tɕo˧ fæ˧}\hspace{5pt}\peng{that way}\hspace{5pt}\pcmn{那个方向}\hspace{5pt}\pfra{cette direction-là}\end{exemple}
\end{entrée}

\begin{entrée}
{fv̩˧}{}{ⓔfv̩˧}\formedesurface{fv̩˧}\newline
\classe{形容词}\ton{M}\begin{définition}\peng{Glad, pleased, happy, delighted; to like.}\end{définition}
\begin{définition}\pcmn{高兴、起劲,喜欢、爱、愿意}\end{définition}
\begin{définition}\pfra{Content, joyeux; agréable; aimer, apprécier.}\end{définition}
\begin{exemple}\pnru{ɖwæ˧˥ | fv̩˧}\hspace{5pt}\peng{|fg{intensive.very}: really glad, very happy}\hspace{5pt}\pcmn{很高兴}\hspace{5pt}\pfra{|fg{intensif.très}: très content, tout content}\end{exemple}
\begin{exemple}\pnru{dʑɤ˩˥ | fv̩˧}\hspace{5pt}\peng{really glad, very happy}\hspace{5pt}\pcmn{很高兴}\hspace{5pt}\pfra{très content, tout content}\end{exemple}
\begin{exemple}\pnru{mɤ˧-fv̩˧ ʝi˧}\hspace{5pt}\peng{to get angry, to lose one's temper, to air one's anger}\hspace{5pt}\pcmn{生气}\hspace{5pt}\pfra{se mettre en colère, s'énerver}\end{exemple}
\begin{exemple}\pnru{ʈʂʰɯ˧ mɤ˧-fv̩˧ ʝi˧!}\hspace{5pt}\peng{He/she is angry.}\hspace{5pt}\pcmn{他在生气。}\hspace{5pt}\pfra{Il/elle est mécontent(e) / en colère.}\end{exemple}
\begin{exemple}\pnru{ɖwæ˧˥ | fv̩˧hĩ˧ ɖɯ˧-v̩˧ ɲi˩}\hspace{5pt}\peng{It's a very agreeable person.}\hspace{5pt}\pcmn{他是很善良的人。}\hspace{5pt}\pfra{c'est quelqu'un de très agréable}\end{exemple}
\end{entrée}

\begin{entrée}
{fv̩˩˧}{}{ⓔfv̩˩˧}\formedesurface{fv̩˩˥}\newline
\classe{名词}\ton{LM}\begin{définition}\peng{Neighbours.}\end{définition}
\begin{définition}\pcmn{邻居,村里的人们}\end{définition}
\begin{définition}\pfra{Le voisinage, les voisins.}\end{définition}
\end{entrée}

\begin{entrée}
{fv̩˩bi˩}{}{ⓔfv̩˩bi˩}\formedesurface{fv̩˩bi˩˥}\newline
\classe{名词}\ton{L}\begin{définition}\peng{Neighbourhood (in the extended sense: encompasses several small villages).}\end{définition}
\begin{définition}\pcmn{邻里、邻村:大家族居住的那片地方,包括几个小村落}\end{définition}
\begin{définition}\pfra{Contrée, voisinage, ensemble de villages où habitent des gens de la famille étendue.}\end{définition}
\end{entrée}

\begin{entrée}
{fv̩˧kʰo˥}{}{ⓔfv̩˧kʰo˥}\formedesurface{fv̩˧kʰo˥}\newline
\classe{名词}\ton{H\#}\begin{définition}\peng{Fengke: a village located close to the Yangtze river, on the right bank.}\end{définition}
\begin{définition}\pcmn{奉科(金沙江边的一个地区)}\end{définition}
\begin{définition}\pfra{Fengke: village situé au bord du Yang-tsé, sur la rive droite.}\end{définition}
\end{entrée}

\begin{entrée}
{fv̩˧ʂɯ˩}{}{ⓔfv̩˧ʂɯ˩}\formedesurface{fv̩˧ʂɯ˩}\newline
\classe{名词}\ton{L\#}\begin{définition}\peng{Rheumatism.}\end{définition}
\begin{définition}\pcmn{风湿(汉语借词)}\end{définition}
\begin{définition}\pfra{Rhumatismes.}\end{définition}
\begin{exemple}\pnru{fv̩˧ʂɯ˩ go˩}\hspace{5pt}\peng{to suffer from rheumatism, to have rheumatism}\hspace{5pt}\pcmn{有风湿、得风湿}\hspace{5pt}\pfra{souffrir de rhumatismes, avoir des rhumatismes}\end{exemple}
\end{entrée}

\newpage\caractère{g}

\begin{entrée}
{gæ˩ɖæ˧}{}{ⓔgæ˩ɖæ˧}\formedesurface{gæ˩ɖæ˥}\newline
\classe{名词}\ton{LM}\begin{définition}\peng{Top part of body.}\end{définition}
\begin{définition}\pcmn{上半身}\end{définition}
\begin{définition}\pfra{Le haut du corps.}\end{définition}
\end{entrée}

\begin{entrée}
{gæ˩pʰæ˧}{}{ⓔgæ˩pʰæ˧}\formedesurface{gæ˩pʰæ˥}\newline
\classe{名词}\ton{LM}
\paradigme{\pcmn{:} \p{}}
\begin{définition}\peng{Storeroom, larder: a room where food is kept.}\end{définition}
\begin{définition}\pcmn{储藏室、库房:存粮食、火腿的房间}\end{définition}
\begin{définition}\pfra{Resserre, pièce où on conserve certains produits: dans le même bâtiment que la cuisine-foyer-salle à manger, à sa gauche (vu depuis la cour).}\end{définition}
\end{entrée}

\begin{entrée}
{gæ˧ɻæ˩}{}{ⓔgæ˧ɻæ˩}\formedesurface{gæ˧ɻæ˩}\newline
\classe{名词}\ton{L\#}\begin{définition}\peng{The name of a village located about 1,500 meters West of \stylefv{/ə}˧lɑ˧-ʁwɤ\#˥/: to the left when leaving the plain of Yongning towards Eya; Chinese: Gaer.}\end{définition}
\begin{définition}\pcmn{嘎尔村,80年代起行政称作嘎拉村民小组。}\end{définition}
\begin{définition}\pfra{Village situé à environ 1,5 km à l'ouest de \stylefv{/ə}˧lɑ˧-ʁwɤ\#˥/: à main gauche en sortant de la vallée de Yongning, en direction de Eya. En chinois: Gaer.}\end{définition}
\begin{exemple}\pnru{dʑɤ˩bv̩˧kɤ˧-sɑ˥ʁwɤ˩, | hi˩ʁwɤ˩-lo˥, | æ˩mi˧-ʁwɤ\#˥, | lɑ˧lo˧-ʁwɤ˥, | lɑ˧ŋwɤ˧, | bɤ˧tsʰo˧gv̩˥, | ə˧lɑ˧-ʁwɤ\#˥, | gæ˧ɻæ˩, | qʰæ˧tɕʰi˧, | tʰo˧ʈɯ\#˥}\hspace{5pt}\peng{The ten Na villages considered in traditional geography as belonging to the vicinity of the Yongning temple.}\hspace{5pt}\pcmn{永宁摩梭地理概念中,距离扎美寺最近的十个村落:佳部嘎萨瓦、习瓦洛、阿咪瓦、拉洛瓦、拉瓦、巴搓古、阿拉瓦、嘎尔、开基、拖支。}\hspace{5pt}\pfra{Les dix villages na traditionnellement considérés comme appartenant au voisinage du temple de Yongning.}\end{exemple}
\end{entrée}

\begin{entrée}
{‑gɤ˧}{}{ⓔ‑gɤ˧}\formedesurface{gɤ˧}\newline
\classe{}
\sens{1}
\begin{définition}\peng{Place.}\end{définition}
\begin{définition}\pcmn{地方}\end{définition}
\begin{définition}\pfra{Lieu, endroit.}\end{définition}
\begin{exemple}\pnru{njɤ˧ | ɖɯ˧-ʝi˧ (-gɤ˧) bi˧-zo˧-ho˩!}\hspace{5pt}\peng{I have to go somewhere! / I have to make a trip! / I'm off!}\hspace{5pt}\pcmn{我要去一个别的地方! / 我要换一个地方了! / 我要走了!}\hspace{5pt}\pfra{Je dois me rendre quelque part! / Je dois faire un voyage! / Je m'en vais! (Contexte: lorsqu'on se prépare réellement à un voyage; ou lors d'une dispute, lorsqu'on menace de quitter la maison.)}\end{exemple}
\begin{exemple}\pnru{ze˩ gɤ˧}\hspace{5pt}\peng{which place}\hspace{5pt}\pcmn{什么地方}\hspace{5pt}\pfra{quel endroit}\end{exemple}
\begin{exemple}\pnru{ʈʂʰɯ˧-gɤ˧}\hspace{5pt}\peng{this place}\hspace{5pt}\pcmn{这个地方}\hspace{5pt}\pfra{cet endroit-ci}\end{exemple}
\begin{exemple}\pnru{tʰv̩˧-gɤ˧}\hspace{5pt}\peng{that place}\hspace{5pt}\pcmn{那个地方}\hspace{5pt}\pfra{cet endroit-là}\end{exemple}\sens{2}
\begin{définition}\peng{Moment.}\end{définition}
\begin{définition}\pcmn{时候}\end{définition}
\begin{définition}\pfra{Moment.}\end{définition}
\begin{exemple}\pnru{ʂɯ˧-ɬi˧mi˧-qo˧-gɤ˧ tʰv̩˧}\hspace{5pt}\peng{when the seventh month has come, when one is in the seventh month}\hspace{5pt}\pcmn{七月到了的时候}\hspace{5pt}\pfra{quand est venu le septième mois, quand on en est au septième mois}\end{exemple}
\begin{exemple}\pnru{ʂɯ˧-ɬi˧mi˧-qo˧-gɤ˧-dʑo˥}\hspace{5pt}\peng{in the seventh month, during the seventh month}\hspace{5pt}\pcmn{七月的时候}\hspace{5pt}\pfra{pendant le septième mois, au cours du septième mois}\end{exemple}
\end{entrée}

\begin{entrée}
{gɤ˧˥}{}{ⓔgɤ˧˥}\formedesurface{gɤ˧˥}\newline
\classe{动词}\ton{MH}\begin{définition}\peng{To carry on the shoulder; to carry on a shoulder pole.}\end{définition}
\begin{définition}\pcmn{扛,担}\end{définition}
\begin{définition}\pfra{Porter à l’épaule; porter sur une palanche.}\end{définition}
\begin{exemple}\pnru{tʰi˧-gɤ˧˥}\hspace{5pt}\peng{|fg{dur}}\hspace{5pt}\pcmn{|fg{dur}}\hspace{5pt}\pfra{|fg{dur}}\end{exemple}
\begin{exemple}\pnru{tʰi˧-gɤ˧-ze˥}\hspace{5pt}\peng{|fg{dur} \_ |fg{pfv}}\hspace{5pt}\pcmn{|fg{dur} \_ |fg{pfv}}\hspace{5pt}\pfra{|fg{dur} \_ |fg{pfv}}\end{exemple}
\begin{exemple}\pnru{le˧-gɤ˧-ze˥}\hspace{5pt}\peng{|fg{accomp} \_ |fg{pfv}}\hspace{5pt}\pcmn{扛了}\hspace{5pt}\pfra{|fg{accomp} \_ |fg{pfv}}\end{exemple}
\begin{exemple}\pnru{tso˧∼tso˧ gɤ˩}\hspace{5pt}\peng{to carry something on the shoulder}\hspace{5pt}\pcmn{扛东西}\hspace{5pt}\pfra{porter quelque chose à l'épaule}\end{exemple}
\begin{exemple}\pnru{njɤ˧(-ɳɯ˧) | gɤ˧-bi˥!}\hspace{5pt}\peng{Let me carry it!}\hspace{5pt}\pcmn{我来扛吧!}\hspace{5pt}\pfra{C'est moi qui porte!}\end{exemple}
\end{entrée}

\begin{entrée}
{gɤ˧β}{}{ⓔgɤ˧β}\formedesurface{gɤ˧}\newline
\classe{动词}\ton{Mβ}\begin{définition}\peng{To lack something (someone lacks a certain ability).}\end{définition}
\begin{définition}\pcmn{缺乏}\end{définition}
\begin{définition}\pfra{Manquer de.}\end{définition}
\begin{exemple}\pnru{mɤ˧-gɤ˧}\hspace{5pt}\peng{|fg{neg}: not to lack}\hspace{5pt}\pcmn{不缺乏}\hspace{5pt}\pfra{|fg{neg}: ne pas manquer de}\end{exemple}
\end{entrée}

\begin{entrée}
{gɤ˩}{}{ⓔgɤ˩}\formedesurface{gɤ˩˥}\newline
\classe{形容词}\ton{L}\begin{définition}\peng{Quarrelsome. This term is used to describe the personality associated with certain astrological signs: some, such as the Tiger and the Monkey, are considered as quarrelsome, making the people born during the corresponding years less suitable for participating in certain rites (e.g. the Coming of Age rite), and more suitable for certain other rites and occasions.}\end{définition}
\begin{définition}\pcmn{爱吵架}\end{définition}
\begin{définition}\pfra{Querelleur, belliqueux, batailleur. Ce terme s'emploie au sujet des signes astrologiques: certains sont considérés comme ‘bagarreurs', comme le Tigre et le Singe, ce qui rend les personnes nées cette année-là peu appropriées pour certains rites (ex.: lors du rite de passage à l'âge adulte), et au contraire très prisés pour d'autres.}\end{définition}
\begin{exemple}\pnru{kʰv̩˧ gɤ˧˥}\hspace{5pt}\peng{“quarrelsome year": a year whose astrological sign is a quarrelsome animal. Astrological signs such as the Tiger and the Monkey are considered as quarrelsome; people born during one of these years are said to be tough and quarrelsome.}\hspace{5pt}\pcmn{爱打架的年份/生肖:十二个生肖中,虎、猴……被认为是爱打架的。}\hspace{5pt}\pfra{signe belliqueux (concept astrologique: certains signes confèrent aux gens nés l'année correspondante un caractère dur/belliqueux)}\end{exemple}
\begin{exemple}\pnru{kʰv̩˧ gɤ˧-hĩ˥}\hspace{5pt}\peng{person whose astrological sign is a quarrelsome animal. Astrological signs such as the Tiger and the Monkey are considered as quarrelsome.}\hspace{5pt}\pcmn{属一个爱打架的年份/生肖的人。十二个生肖中,虎、猴……被认为是爱打架的。}\hspace{5pt}\pfra{personne d'une année batailleuse}\end{exemple}
\begin{exemple}\pnru{ʑi˩hṽ̩˥, | lɑ˧ : | kʰv̩˧ gɤ˧˥!}\hspace{5pt}\peng{The Monkey and the Ape are quarrelsome birth signs!}\hspace{5pt}\pcmn{属猴和属虎的人很爱吵架!}\hspace{5pt}\pfra{Les signes astrologiques du Singe et du Tigre sont des signes batailleurs!}\end{exemple}
\begin{exemple}\pnru{ʑi˩˥, | lɑ˧, | kʰv̩˧ gɤ˧˥!}\hspace{5pt}\peng{Same meaning as above; the investigator substituted the colloquial term for ‘ape, monkey'.}\hspace{5pt}\pcmn{同上}\hspace{5pt}\pfra{Même sens que ci-dessus; formulation modernisée par l'enquêteur, utilisant le terme usuel pour ‘singe'.}\end{exemple}
\end{entrée}

\begin{entrée}
{gɤ˩‑}{}{ⓔgɤ˩‑}\formedesurface{--}\newline
\classe{助词}\ton{L}\begin{définition}\peng{Directional prefix: upward.}\end{définition}
\begin{définition}\pcmn{向上、往上}\end{définition}
\begin{définition}\pfra{Préfixe directionnel: vers le haut.}\end{définition}
\end{entrée}

\begin{entrée}
{gɤ˩α}{₁}{ⓔgɤ˩αⓗ1}\formedesurface{gɤ˩˥}\newline
\classe{动词}\ton{Lα}
1\begin{définition}\peng{To go out (fire).}\end{définition}
\begin{définition}\pcmn{灭,熄}\end{définition}
\begin{définition}\pfra{S’éteindre.}\end{définition}
\begin{exemple}\pnru{mv̩˧ | le˧-gɤ˩(-ze˩)}\hspace{5pt}\peng{The fire has gone out. / The fire went out.}\hspace{5pt}\pcmn{火灭了。}\hspace{5pt}\pfra{Le feu s'est éteint.}\end{exemple}
\end{entrée}

\begin{entrée}
{gɤ˩α}{₂}{ⓔgɤ˩αⓗ2}\formedesurface{gɤ˩˥}\newline
\classe{动词}\ton{Lα}
2\begin{définition}\peng{To be satisfied/happy; to feel that things are fair.}\end{définition}
\begin{définition}\pcmn{满意,幸福,甘心,服气}\end{définition}
\begin{définition}\pfra{Être satisfait, content (de son sort), heureux.}\end{définition}
\begin{exemple}\pnru{hɤ˩-zo˥, | le˧-gɤ˩-ze˩!}\hspace{5pt}\peng{(S)he has made a good job of it; (s)he is satisfied/happy!}\hspace{5pt}\pcmn{很成功,真高兴! / 他成功了,很满意!}\hspace{5pt}\pfra{(il a réussi) habilement (ce qu'il voulait faire), il est content/satisfait!}\end{exemple}
\begin{exemple}\pnru{ʈʂʰɯ˧ | ɖwæ˧˥ | le˧-gɤ˩-ze˩!}\hspace{5pt}\peng{(S)he is very satisfied/happy!}\hspace{5pt}\pcmn{他很满意!}\hspace{5pt}\pfra{il est très content!}\end{exemple}
\begin{exemple}\pnru{no˩-se˥, | ɖwæ˧˥ | le˧-gɤ˩-ze˩: | zo˧mv̩˥ hɤ˩-zo˩!}\hspace{5pt}\peng{You have grounds for satisfaction: your children are really bright!}\hspace{5pt}\pcmn{你呢,(应该)很满意:(你的)孩子很成功!}\hspace{5pt}\pfra{Vous, vous avez bien de la chance/vous avez toutes raisons d'être satisfait(e)/vous avez des sujets de satisfaction: vos enfants sont brillants/habiles!}\end{exemple}
\begin{exemple}\pnru{mɤ˧-gɤ˩}\hspace{5pt}\peng{to be dissatisfied, not resigned, recalcitrant}\hspace{5pt}\pcmn{不满意、不甘心、不服气}\hspace{5pt}\pfra{être mécontent, ne pas se résigner, être récalcitrant}\end{exemple}
\end{entrée}

\begin{entrée}
{gɤ˩α}{₃}{ⓔgɤ˩αⓗ3}\formedesurface{gɤ˩˥}\newline
\classe{形容词}\ton{Lα}
3\begin{définition}\peng{Startled, amazed, shocked, awestruck; terrified.}\end{définition}
\begin{définition}\pcmn{震惊}\end{définition}
\begin{définition}\pfra{Surpris, étonné, abasourdi; terrifié.}\end{définition}
\begin{exemple}\pnru{le˧-gɤ˩-ze˩}\hspace{5pt}\peng{|fg{accomp} \_ |fg{pfv}}\hspace{5pt}\pcmn{震惊了}\hspace{5pt}\pfra{|fg{accomp} \_ |fg{pfv}}\end{exemple}
\begin{exemple}\pnru{no˧ | hĩ˧ gɤ˧-kʰɯ˥!}\hspace{5pt}\peng{You frighten people! / People are afraid of you!}\hspace{5pt}\pcmn{你让人害怕!}\hspace{5pt}\pfra{Tu fais peur aux gens!}\end{exemple}
\end{entrée}

\begin{entrée}
{gɤ˧bɤ˧}{}{ⓔgɤ˧bɤ˧}\formedesurface{gɤ˧bɤ˧}\newline
\classe{名词}\ton{M}
\paradigme{\pcmn{:} \p{}}
\begin{définition}\peng{Shadow.}\end{définition}
\begin{définition}\pcmn{影子}\end{définition}
\begin{définition}\pfra{Ombre.}\end{définition}
\begin{exemple}\pnru{gɤ˧bɤ˧ li˧}\hspace{5pt}\peng{to watch television (coinage to avoid the loanword ‘television')}\hspace{5pt}\pcmn{看电视}\hspace{5pt}\pfra{regarder la télé (néologisme)}\end{exemple}
\end{entrée}

\begin{entrée}
{‑gɤ˧bi\#˥}{}{ⓔ‑gɤ˧bi\#˥}\formedesurface{gɤ˧bi˧}\newline
\classe{}\ton{\#H}\begin{définition}\peng{On top.}\end{définition}
\begin{définition}\pcmn{上面}\end{définition}
\begin{définition}\pfra{Sur, dessus.}\end{définition}
\begin{exemple}\pnru{ʑi˧qʰwɤ˧-gɤ˧bi˧}\hspace{5pt}\peng{on the (roof of) the house = on the roof}\hspace{5pt}\pcmn{在房頂上}\hspace{5pt}\pfra{sur la maison, sur le toit}\end{exemple}
\end{entrée}

\begin{entrée}
{gɤ˩bv̩˧}{}{ⓔgɤ˩bv̩˧}\formedesurface{gɤ˩bv̩˥}\newline
\classe{动词}\ton{LM}\begin{définition}\peng{To overflow.}\end{définition}
\begin{définition}\pcmn{溢出来}\end{définition}
\begin{définition}\pfra{Déborder.}\end{définition}
\begin{exemple}\pnru{gɤ˩bv̩˧-ze˩}\hspace{5pt}\peng{|fg{pfv}}\hspace{5pt}\pcmn{溢出来了}\hspace{5pt}\pfra{|fg{pfv}}\end{exemple}
\end{entrée}

\begin{entrée}
{gɤ˩dzɤ˧}{}{ⓔgɤ˩dzɤ˧}\formedesurface{gɤ˩dzɤ˥}\newline
\classe{助词}\ton{LM}\begin{définition}\peng{At the top part: inside a room, at a table…, this is the place of honour.}\end{définition}
\begin{définition}\pcmn{在上部分,上座}\end{définition}
\begin{définition}\pfra{Haut, partie supérieure, partie noble (d'une salle, d'une tablée…) (symboliquement: «la tête»).}\end{définition}
\begin{exemple}\pnru{gɤ˩dzɤ˧ dzi˧˥}\hspace{5pt}\peng{to sit at a place of honour, to sit at the superior part (of a table, a room…)}\hspace{5pt}\pcmn{坐上座}\hspace{5pt}\pfra{être assis à une place d'honneur}\end{exemple}
\begin{exemple}\pnru{no˧ | gɤ˩dzɤ˧ dzi˧˥!}\hspace{5pt}\peng{Please be seated at the place of honour!}\hspace{5pt}\pcmn{请您坐在上座!}\hspace{5pt}\pfra{Veuillez vous installer à l'une des premières places! / Veuillez prendre l'une des places d'honneur!}\end{exemple}
\end{entrée}

\begin{entrée}
{gɤ˧lɑ˧}{}{ⓔgɤ˧lɑ˧}\formedesurface{gɤ˧lɑ˧}\newline
\classe{名词}\ton{M}
\paradigme{\pcmn{:} \p{}}
\begin{définition}\peng{God, Pusa, Buddha, Bodhisattva.}\end{définition}
\begin{définition}\pcmn{神,菩萨,佛}\end{définition}
\begin{définition}\pfra{Dieu, bouddha, bodhisattva.}\end{définition}
\end{entrée}

\begin{entrée}
{gɤ˧lɑ˧-pɤ\#˥}{}{ⓔgɤ˧lɑ˧-pɤ\#˥}\formedesurface{gɤ˧lɑ˧pɤ˧}\newline
\classe{名词}\ton{\#H}
\paradigme{\pcmn{:} \p{}}
\begin{définition}\peng{Image of Buddha.}\end{définition}
\begin{définition}\pcmn{佛像}\end{définition}
\begin{définition}\pfra{Image du bouddha.}\end{définition}
\end{entrée}

\begin{entrée}
{gɤ˧lɑ˧-ʑi˩}{}{ⓔgɤ˧lɑ˧-ʑi˩}\formedesurface{gɤ˧lɑ˧ʑi˩}\newline
\classe{名词}\ton{-L}
\paradigme{\pcmn{:} \p{}}
\begin{définition}\peng{Room where the ancestors are worshipped.}\end{définition}
\begin{définition}\pcmn{经堂(拜佛、拜祖先的房间)}\end{définition}
\begin{définition}\pfra{Pièce du culte: pièce des esprits, pièce des ancêtres, où se trouve un autel. Un rituel y est effectué chaque matin. Le nom désigne par extension l'intégralité d'un des quatre bâtiments de la ferme traditionnelle na.}\end{définition}
\begin{exemple}\pnru{gɤ˧lɑ˧-ʑi˩-di˩}\hspace{5pt}\peng{same meaning}\hspace{5pt}\pcmn{同上}\hspace{5pt}\pfra{même sens}\end{exemple}
\end{entrée}

\begin{entrée}
{gɤ˧qo˥}{}{ⓔgɤ˧qo˥}\formedesurface{gɤ˧qo˥}\newline
\classe{名词}\ton{MH}
\paradigme{\pcmn{:} \p{}}
\begin{définition}\peng{Higher part of the main room.}\end{définition}
\begin{définition}\pcmn{主屋的高处:人吃饭的地方}\end{définition}
\begin{définition}\pfra{Haut du foyer: partie de la pièce où l’on prend les repas, autour du foyer; c'est une structure en bois, surélevée d’une vingtaine de centimètres par rapport au sol cimenté.}\end{définition}
\end{entrée}

\begin{entrée}
{gɤ˩-qo˧}{}{ⓔgɤ˩-qo˧}\formedesurface{gɤ˩qo˥}\newline
\classe{助词}\ton{M}\begin{définition}\peng{Way up there.}\end{définition}
\begin{définition}\pcmn{那上面(指高处)}\end{définition}
\begin{définition}\pfra{Par là-bas tout en haut.}\end{définition}
\end{entrée}

\begin{entrée}
{gɤ˩qwɤ˧}{}{ⓔgɤ˩qwɤ˧}\formedesurface{gɤ˩qwɤ˥}\newline
\classe{名词}\ton{LM}
\paradigme{\pcmn{:} \p{}}
\begin{définition}\peng{Altar above the hearth (where gifts made to the family are displayed).}\end{définition}
\begin{définition}\pcmn{火炉旁边的祭坛(上面摆礼物等)}\end{définition}
\begin{définition}\pfra{Autel en contrehaut du foyer, où on dépose les présents qu'apportent les invités/les membres de la famille, les offrant ainsi aux ancêtres.}\end{définition}
\end{entrée}

\begin{entrée}
{gɤ˩ʁwɤ˧}{}{ⓔgɤ˩ʁwɤ˧}\formedesurface{gɤ˩ʁwɤ˥}\newline
\classe{名词}\ton{LM}\begin{définition}\peng{Upper reaches of a river; upstream.}\end{définition}
\begin{définition}\pcmn{上游}\end{définition}
\begin{définition}\pfra{Cours supérieur (d'une rivière), amont.}\end{définition}
\end{entrée}

\begin{entrée}
{gɤ˩ʁwɤ\#˥}{}{ⓔgɤ˩ʁwɤ\#˥}\formedesurface{gɤ˩ʁwɤ˥}\newline
\classe{名词}\ton{LM+\#H}\begin{définition}\peng{The village of Gewa.}\end{définition}
\begin{définition}\pcmn{格瓦村:永宁坝子的一个村落。直译:上村。音译:格瓦。}\end{définition}
\begin{définition}\pfra{Nom de village; en chinois: Gewa.}\end{définition}
\end{entrée}

\begin{entrée}
{gɤ˩-tʰv̩˧-gi\#˥}{}{ⓔgɤ˩-tʰv̩˧-gi\#˥}\formedesurface{gɤ˩tʰv̩˧gi˧}\newline
\classe{助词}\ton{L-\#H}\begin{définition}\peng{Way up there.}\end{définition}
\begin{définition}\pcmn{那里(指高处)}\end{définition}
\begin{définition}\pfra{Au loin par là-haut, de ce côté tout là-haut.}\end{définition}
\end{entrée}

\begin{entrée}
{gɤ˩-tʰv̩˧qo˧}{}{ⓔgɤ˩-tʰv̩˧qo˧}\formedesurface{gɤ˩tʰv̩˧qo˧}\newline
\classe{助词}\ton{L-\#H}\begin{définition}\peng{Way up there.}\end{définition}
\begin{définition}\pcmn{那里(指高处)}\end{définition}
\begin{définition}\pfra{Au loin par là-haut, de ce côté tout là-haut.}\end{définition}
\end{entrée}

\begin{entrée}
{gɤ˩ʈʂæ˧˥}{}{ⓔgɤ˩ʈʂæ˧˥}\formedesurface{gɤ˩ʈʂæ˧˥}\newline
\classe{名词}\ton{LM+MH\#}\begin{définition}\peng{Top part (of the body=above the waist).}\end{définition}
\begin{définition}\pcmn{上半(身)}\end{définition}
\begin{définition}\pfra{Haut du corps, partie supérieure du corps.}\end{définition}
\end{entrée}

\begin{entrée}
{gɤ˩ʈʂʰæ˧-hĩ˧˥}{}{ⓔgɤ˩ʈʂʰæ˧-hĩ˧˥}\formedesurface{gɤ˩ʈʂʰæ˧hĩ˧˥}\newline
\classe{名词}\ton{LM+MH\#}\begin{définition}\peng{Ancestors, past generations.}\end{définition}
\begin{définition}\pcmn{祖先}\end{définition}
\begin{définition}\pfra{Les générations passées, les ancêtres.}\end{définition}
\end{entrée}

\begin{entrée}
{gɤ˩-ʈʂʰɯ˧-gi\#˥}{}{ⓔgɤ˩-ʈʂʰɯ˧-gi\#˥}\formedesurface{gɤ˩ʈʂʰɯ˧gi˧}\newline
\classe{助词}\ton{L-\#H}\begin{définition}\peng{Way up there.}\end{définition}
\begin{définition}\pcmn{那里(指高处)}\end{définition}
\begin{définition}\pfra{Au loin par là-haut, de ce côté tout là-haut.}\end{définition}
\end{entrée}

\begin{entrée}
{gɤ˩-ʈʂʰɯ˧qo˧}{}{ⓔgɤ˩-ʈʂʰɯ˧qo˧}\formedesurface{gɤ˩ʈʂʰɯ˧qo˧}\newline
\classe{助词}\ton{L-\#H}\begin{définition}\peng{Way up there.}\end{définition}
\begin{définition}\pcmn{那里(指高处)}\end{définition}
\begin{définition}\pfra{Au loin par là-haut, de ce côté tout là-haut.}\end{définition}
\end{entrée}

\begin{entrée}
{gi˥}{₁}{ⓔgi˥ⓗ1}\formedesurface{gi˧}\newline
\classe{动词}\ton{H}
1\begin{définition}\peng{To fall (snow, rain), to snow/to rain.}\end{définition}
\begin{définition}\pcmn{下(雨,雪)}\end{définition}
\begin{définition}\pfra{Tomber (neige, pluie), neiger, pleuvoir.}\end{définition}
\begin{exemple}\pnru{bi˧ gi˧. / bi˧ gi˧-ze˩.}\hspace{5pt}\peng{It snows. / It has snowed.}\hspace{5pt}\pcmn{下雪。 / 下雪了。}\hspace{5pt}\pfra{Il neige. / Il a neigé.}\end{exemple}
\begin{exemple}\pnru{hi˩ gi˩˥. / hi˩ gi˩-ze˥.}\hspace{5pt}\peng{It rains. / It has rained.}\hspace{5pt}\pcmn{下雨。 / 下雨了。}\hspace{5pt}\pfra{Il pleut. / Il a plu.}\end{exemple}
\begin{exemple}\pnru{tsʰi˧-ɲi˧-dʑo˩, | hi˩ gi˩-ze˥, | le˧-gɤ˩-ze˩!}\hspace{5pt}\peng{Today, it is raining; that's good! / it's a good thing! (A comment made at the beginning of the rainy season, after a long drought.)}\hspace{5pt}\pcmn{今天,下雨了,真好!(情景:大旱灾过后,雨季终于来了,这对庄稼很好。)}\hspace{5pt}\pfra{Aujourd'hui, il s'est mis à pleuvoir / il a plu; c'est bien! (Commentaire au sujet de la pluie qui est venue, après une longue période de sécheresse.)}\end{exemple}
\end{entrée}

\begin{entrée}
{gi˥}{₂}{ⓔgi˥ⓗ2}\formedesurface{gi˧}\newline
\classe{动词}\ton{H}
2\begin{définition}\peng{To owe money.}\end{définition}
\begin{définition}\pcmn{欠(钱)}\end{définition}
\begin{définition}\pfra{Devoir de l'argent, avoir des dettes.}\end{définition}
\begin{exemple}\pnru{ɖʐe˧ | tʰi˧-gi˥}\hspace{5pt}\peng{to owe money}\hspace{5pt}\pcmn{欠钱}\hspace{5pt}\pfra{devoir de l'argent}\end{exemple}
\end{entrée}

\begin{entrée}
{gi˥α}{}{ⓔgi˥α}\newline
\classe{量词}
\sens{1}
\begin{définition}\peng{A half.}\end{définition}
\begin{définition}\pcmn{量词:一半}\end{définition}
\begin{définition}\pfra{Une moitié, un demi.}\end{définition}
\begin{exemple}\pnru{ɖɯ˧-gi˥}\hspace{5pt}\peng{one half}\hspace{5pt}\pcmn{一半}\hspace{5pt}\pfra{une moitié}\end{exemple}
\begin{exemple}\pnru{tsʰe˩ʐv̩˩-gi˥}\hspace{5pt}\peng{fourteen halves (combination elicited to determine the tonal category of the classifier)}\hspace{5pt}\pcmn{十四个半(注:这是为了确定调类而问的短语)}\hspace{5pt}\pfra{quatorze moitiés (combinaison permettant de déterminer la catégorie tonale de ce classificateur: elle établit que le ton est H1 et non H2)}\end{exemple}
\begin{exemple}\pnru{tv̩˧tsʰɯ˧ | ɖɯ˧-gi˥}\hspace{5pt}\peng{half the time, half the duration}\hspace{5pt}\pcmn{一半的时间}\hspace{5pt}\pfra{la moitié du temps, la moitié de la durée}\end{exemple}\sens{2}
\begin{définition}\peng{A side; a direction.}\end{définition}
\begin{définition}\pcmn{量词:一面(房屋的一面)}\end{définition}
\begin{définition}\pfra{Un côté (d'une pièce, d'une maison…); une direction.}\end{définition}
\begin{exemple}\pnru{ɖɯ˧-gi˧ hõ˧}\hspace{5pt}\peng{to go in a certain direction, to go one's way}\hspace{5pt}\pcmn{往一个方向走、走自己的方向}\hspace{5pt}\pfra{aller d'un certain côté, aller de son côté}\end{exemple}
\begin{exemple}\pnru{ɖɯ˧-v̩˧ | ɖɯ˧-gi˧ hɯ˧}\hspace{5pt}\peng{to go each one's separate way; to go each in a different direction}\hspace{5pt}\pcmn{分开:每个人去自己的方向}\hspace{5pt}\pfra{aller chacun de son côté; se séparer}\end{exemple}
\end{entrée}

\begin{entrée}
{‑gi˧˥}{}{ⓔ‑gi˧˥}\formedesurface{gi˧˥}\newline
\classe{}\ton{MH}\begin{définition}\peng{Behind.}\end{définition}
\begin{définition}\pcmn{后面,(最)后}\end{définition}
\begin{définition}\pfra{Derrière.}\end{définition}
\begin{exemple}\pnru{ə˧mɑ˧-gi˧˥}\hspace{5pt}\peng{behind mummy}\hspace{5pt}\pcmn{妈妈后面}\hspace{5pt}\pfra{derrière maman}\end{exemple}
\begin{exemple}\pnru{lɑ˧-gi˧˥}\hspace{5pt}\peng{behind the tiger}\hspace{5pt}\pcmn{老虎后面}\hspace{5pt}\pfra{derrière le tigre}\end{exemple}
\begin{exemple}\pnru{bo˩-gi˥}\hspace{5pt}\peng{behind the pig}\hspace{5pt}\pcmn{猪后面}\hspace{5pt}\pfra{derrière le cochon}\end{exemple}
\begin{exemple}\pnru{mv̩˩-gi˥}\hspace{5pt}\peng{behind the daughter}\hspace{5pt}\pcmn{女儿后面}\hspace{5pt}\pfra{derrière la fille}\end{exemple}
\begin{exemple}\pnru{ʐwæ˧-gi˥}\hspace{5pt}\peng{behind the horse}\hspace{5pt}\pcmn{马后面}\hspace{5pt}\pfra{derrière le cheval}\end{exemple}
\begin{exemple}\pnru{ʈʂʰɯ˧-gi˥ | tʰi˧-tɕʰo˩}\hspace{5pt}\peng{to hide in there (literally ‘behind there')}\hspace{5pt}\pcmn{藏那后面}\hspace{5pt}\pfra{se cacher là-derrière}\end{exemple}
\begin{exemple}\pnru{no˧-gi˧ njɤ˥ ʈʂwæ˩!}\hspace{5pt}\peng{I follow in your footsteps! / I follow you! / I imitate you!}\hspace{5pt}\pcmn{我跟你走! / 我都按你说的来做吧!}\hspace{5pt}\pfra{je te suis, je marche dans tes pas; je t'imite}\end{exemple}
\begin{exemple}\pnru{ɖɯ˧-v̩˧-gi˧˥, | ɖɯ˧-v̩˧ hwæ˧!}\hspace{5pt}\peng{to buy one after the other (context: someone buys one horse after the other, to put together a complete caravan of his own)}\hspace{5pt}\pcmn{一个接着一个地买(情景:一个人接二连三地买马,最后组成自己的马帮队)}\hspace{5pt}\pfra{en acheter un après l'autre (contexte: un caravanier achète des chevaux l'un après l'autre, afin de se constituer sa propre caravane)}\end{exemple}
\begin{exemple}\pnru{gi˧˥ | ɖɯ˧-qɑ˩ gv̩˩-bi˩!}\hspace{5pt}\peng{Let's do one last bundle! (Context: women are extracting flax fiber, processing bundle after bundle; towards the end of a long work session, someone says: “Let's do one last bundle! / One last bundle and we shall call it a day!")}\hspace{5pt}\pcmn{再做一捆吧!(情景:女人们在纺麻线,工作了很久,一个人就说:“再做最后一捆(就收工吧)!”)}\hspace{5pt}\pfra{On va en faire une dernière botte! (contexte: on travaille le lin, botte après botte; vers la fin d'une longue séance de travail, quelqu'un annonce: «On va en faire une dernière botte! / Une dernière botte, et on s'arrête!»)}\end{exemple}
\begin{exemple}\pnru{gi˧-se˧}\hspace{5pt}\peng{to walk after, to follow after}\hspace{5pt}\pcmn{在后面走,在后面跟着}\hspace{5pt}\pfra{marcher derrière, suivre derrière}\end{exemple}
\end{entrée}

\begin{entrée}
{gi˩}{}{ⓔgi˩}\formedesurface{gi˧}\newline
\classe{名词}\ton{L}
\paradigme{\pcmn{:} \p{}}
\begin{définition}\peng{Granary (room within the house where grain is stored).}\end{définition}
\begin{définition}\pcmn{粮仓}\end{définition}
\begin{définition}\pfra{Grenier à céréales; selon M23, est le lieu dans la maison où on stocke les céréales.}\end{définition}
\begin{exemple}\pnru{gi˧mi˧}\hspace{5pt}\peng{large granary}\hspace{5pt}\pcmn{大粮仓}\hspace{5pt}\pfra{grand grenier}\end{exemple}
\begin{exemple}\pnru{gi˩zo˩˥}\hspace{5pt}\peng{small granary}\hspace{5pt}\pcmn{小粮仓}\hspace{5pt}\pfra{petit grenier}\end{exemple}
\begin{exemple}\pnru{njɤ˧ | gi˩ gv̩˩-zo˩-ho˥}\hspace{5pt}\peng{I shall have to repair the granary!}\hspace{5pt}\pcmn{我应该修粮仓!}\hspace{5pt}\pfra{il va falloir que je répare le grenier à céréales!}\end{exemple}
\end{entrée}

\begin{entrée}
{gi˩}{}{ⓔgi˩}\formedesurface{gi˧}\newline
\classe{名词}\ton{L}\begin{définition}\peng{Bear.}\end{définition}
\begin{définition}\pcmn{大熊}\end{définition}
\begin{définition}\pfra{Ours.}\end{définition}
\end{entrée}

\begin{entrée}
{gi˩α}{}{ⓔgi˩α}\formedesurface{gi˩˥}\newline
\classe{形容词}\ton{Lα}
\étymologie{
/gɯ˩a 2/; /ʝi˥/
}\begin{définition}\peng{True, real; really, truly.}\end{définition}
\begin{définition}\pcmn{真,真的}\end{définition}
\begin{définition}\pfra{Vrai, vraiment.}\end{définition}
\begin{exemple}\pnru{mɤ˧-gi˩!}\hspace{5pt}\peng{(It) is not true!}\hspace{5pt}\pcmn{不是的! / 不是真的!}\hspace{5pt}\pfra{c'est pas vrai!}\end{exemple}
\begin{exemple}\pnru{ə˩-gi˩˥?}\hspace{5pt}\peng{Right? / Is that true? / It is true, isn't it?}\hspace{5pt}\pcmn{对吧? / 对吗?}\hspace{5pt}\pfra{c'est vrai? / n'est-ce pas?}\end{exemple}
\begin{exemple}\pnru{ə˩-gi˩˥ ? – gi˩˥!}\hspace{5pt}\peng{Is that right? - Yes, it is! (One speaker asks for confirmation; the other provides confirmation.)}\hspace{5pt}\pcmn{对吧? -对的!}\hspace{5pt}\pfra{N'est-ce pas? - Oui-da! (Le locuteur demande à son interlocuteur de confirmer qu’il adhère à son propos; l'autre donne son assentiment.)}\end{exemple}
\begin{exemple}\pnru{gi˩-hĩ˩ ʐwɤ˥}\hspace{5pt}\peng{to speak the truth, to tell the truth}\hspace{5pt}\pcmn{说实话,老实说}\hspace{5pt}\pfra{dire vrai}\end{exemple}
\begin{exemple}\pnru{gi˩˥ | -gɯ˩˥}\hspace{5pt}\peng{truly, veritably}\hspace{5pt}\pcmn{真的,真正的}\hspace{5pt}\pfra{vraiment, véritablement}\end{exemple}
\end{entrée}

\begin{entrée}
{gi˧dʑɯ˧}{}{ⓔgi˧dʑɯ˧}\formedesurface{gi˧dʑɯ˧}\newline
\classe{名词}\ton{M}\begin{définition}\peng{The Yangtze river (Yellow Sands river).}\end{définition}
\begin{définition}\pcmn{金沙江}\end{définition}
\begin{définition}\pfra{Le fleuve Yangtze.}\end{définition}
\begin{exemple}\pnru{gi˧dʑɯ˧-kʰi\#˥}\hspace{5pt}\peng{the banks of the Yangtze river: Fengke, Labai…}\hspace{5pt}\pcmn{金沙江边:奉科,拉伯……}\hspace{5pt}\pfra{le bord du fleuve Yangtze: Fengke, Labai…}\end{exemple}
\begin{exemple}\pnru{gi˧dʑɯ˧-kʰi˧-hĩ\#˥}\hspace{5pt}\peng{inhabitants of the banks of the Yangtze: people of Labai, Fengke…}\hspace{5pt}\pcmn{金沙江边的人:奉科人,拉伯人……}\hspace{5pt}\pfra{les riverains du Yangtze: gens de Labai, Fengke…}\end{exemple}
\end{entrée}

\begin{entrée}
{gi˩kɯ˩}{}{ⓔgi˩kɯ˩}\formedesurface{gi˩kɯ˩˥}\newline
\classe{名词}\ton{L}\begin{définition}\peng{Musk (literally: ‘bear's gall').}\end{définition}
\begin{définition}\pcmn{麝香(直译:大熊胆)}\end{définition}
\begin{définition}\pfra{Musc (littéralement: ‘bile d'ours').}\end{définition}
\end{entrée}

\begin{entrée}
{gi˧-nɑ˧mi\#˥}{}{ⓔgi˧-nɑ˧mi\#˥}\formedesurface{gi˧nɑ˧mi˧}\newline
\classe{名词}\ton{\#H}
\paradigme{\pcmn{:} \p{}}
\begin{définition}\peng{Bear; she-bear. There is no way to refer unambiguously to a female bear, as the same term is used for bears irrespective of sex.}\end{définition}
\begin{définition}\pcmn{熊,母熊}\end{définition}
\begin{définition}\pfra{Ours (mâle ou femelle). Il n'existe pas de terme désignant de façon non ambiguë une ourse.}\end{définition}
\begin{exemple}\pnru{gi˧-nɑ˧mi˧ tʰv̩˧-pʰo˩}\hspace{5pt}\peng{|fg{n}+|fg{dem}+|fg{clf}}\hspace{5pt}\pcmn{这只熊}\hspace{5pt}\pfra{|fg{n}+|fg{dem}+|fg{clf}}\end{exemple}
\end{entrée}

\begin{entrée}
{gi˧-nɑ˧mi˧-pʰv̩\#˥}{}{ⓔgi˧-nɑ˧mi˧-pʰv̩\#˥}\formedesurface{gi˧nɑ˧mi˧pʰv̩˧}\newline
\classe{名词}\ton{\#H}
\paradigme{\pcmn{:} \p{}}
\begin{définition}\peng{He-bear, male bear.}\end{définition}
\begin{définition}\pcmn{公熊}\end{définition}
\begin{définition}\pfra{Ours (mâle).}\end{définition}
\begin{exemple}\pnru{gi˧-nɑ˧mi˧-pʰv̩˧ tʰv̩˧-pʰo˩}\hspace{5pt}\peng{|fg{n}+|fg{dem}+|fg{clf}}\hspace{5pt}\pcmn{这只公熊}\hspace{5pt}\pfra{|fg{n}+|fg{dem}+|fg{clf}}\end{exemple}
\end{entrée}

\begin{entrée}
{gi˧-nɑ˧mi˧-zo\#˥}{}{ⓔgi˧-nɑ˧mi˧-zo\#˥}\formedesurface{gi˧nɑ˧mi˧zo˧}\newline
\classe{名词}\ton{\#H}
\paradigme{\pcmn{:} \p{}}
\begin{définition}\peng{Little bear, bear cub.}\end{définition}
\begin{définition}\pcmn{小熊}\end{définition}
\begin{définition}\pfra{Ourson (de sexe masculin).}\end{définition}
\begin{exemple}\pnru{gi˧-nɑ˧mi˧-zo˧ tʰv̩˧-ɭɯ\#˥}\hspace{5pt}\peng{|fg{n}+|fg{dem}+|fg{clf}}\hspace{5pt}\pcmn{这只小熊}\hspace{5pt}\pfra{|fg{n}+|fg{dem}+|fg{clf}}\end{exemple}
\end{entrée}

\begin{entrée}
{gi˧zɯ\#˥}{}{ⓔgi˧zɯ\#˥}\formedesurface{gi˧zɯ˧}\newline
\classe{名词}\ton{\#H}
\paradigme{\pcmn{:} \p{}}
\begin{définition}\peng{Little brother, younger brother; the term is also used to refer to younger cousins.}\end{définition}
\begin{définition}\pcmn{弟弟(也可指更年轻的表弟)}\end{définition}
\begin{définition}\pfra{Petit frère (employé aussi entre cousins).}\end{définition}
\begin{exemple}\pnru{gi˧zɯ˧=ɻæ˥}\hspace{5pt}\peng{|fg{associative}: younger brothers, cousins…}\hspace{5pt}\pcmn{联想复数:弟弟们,表弟们}\hspace{5pt}\pfra{|fg{associatif}: les petits frères}\end{exemple}
\end{entrée}

\begin{entrée}
{gi˧zɯ˧-go˧mi\#˥}{}{ⓔgi˧zɯ˧-go˧mi\#˥}\formedesurface{gi˧zɯ˧go˧mi˧}\newline
\classe{名词}\ton{\#H}\begin{définition}\peng{Younger siblings (brothers and sisters).}\end{définition}
\begin{définition}\pcmn{弟弟妹妹}\end{définition}
\begin{définition}\pfra{Cadets: petits frères+ petites sœurs.}\end{définition}
\end{entrée}

\begin{entrée}
{go˩α}{}{ⓔgo˩α}\formedesurface{go˩˥}\newline
\classe{动词}\ton{Lα}\begin{définition}\peng{To suffer; to be sick, to be ill.}\end{définition}
\begin{définition}\pcmn{痛,病 (生病)}\end{définition}
\begin{définition}\pfra{Souffrir, avoir mal; être malade.}\end{définition}
\begin{exemple}\pnru{njɤ˧ | go˩˥!}\hspace{5pt}\peng{I am suffering! / It hurts!}\hspace{5pt}\pcmn{我痛!}\hspace{5pt}\pfra{J'ai mal!}\end{exemple}
\begin{exemple}\pnru{njɤ˧ | go˩˥ | ʐwæ˩˥!}\hspace{5pt}\peng{I am suffering a lot! / It hurts a lot!}\hspace{5pt}\pcmn{我好疼!}\hspace{5pt}\pfra{J'ai très mal!}\end{exemple}
\begin{exemple}\pnru{go˩-hĩ˩˥}\hspace{5pt}\peng{|fg{nmlz}: patient, sick person}\hspace{5pt}\pcmn{病人,病的(人)}\hspace{5pt}\pfra{|fg{nmlz}: patient, malade}\end{exemple}
\begin{exemple}\pnru{hĩ˧ | go˩-hĩ˩˥}\hspace{5pt}\peng{sick person, person who is ill, patient}\hspace{5pt}\pcmn{病人}\hspace{5pt}\pfra{patient, personne malade, malade}\end{exemple}
\begin{exemple}\pnru{bi˧mi˧ go˩}\hspace{5pt}\peng{to have stomach-ache}\hspace{5pt}\pcmn{肚子疼}\hspace{5pt}\pfra{avoir mal au ventre}\end{exemple}
\end{entrée}

\begin{entrée}
{go˧bɤ˩}{}{ⓔgo˧bɤ˩}\formedesurface{go˧bɤ˩}\newline
\classe{名词}\ton{L\#}
\paradigme{\pcmn{:} \p{}}
\begin{définition}\peng{Temple, monastery.}\end{définition}
\begin{définition}\pcmn{庙,寺}\end{définition}
\begin{définition}\pfra{Temple, monastère.}\end{définition}
\end{entrée}

\begin{entrée}
{go˩bi˧}{}{ⓔgo˩bi˧}\formedesurface{go˩bi˥}\newline
\classe{名词}\ton{LM}\begin{définition}\peng{The city of Lijiang.}\end{définition}
\begin{définition}\pcmn{丽江城}\end{définition}
\begin{définition}\pfra{Lijiang (la ville).}\end{définition}
\begin{exemple}\pnru{go˩bi˧-ɖʐɯ˧qo˩}\hspace{5pt}\peng{the city of Lijiang}\hspace{5pt}\pcmn{丽江城}\hspace{5pt}\pfra{la ville de Lijiang}\end{exemple}
\end{entrée}

\begin{entrée}
{go˩bo˥}{}{ⓔgo˩bo˥}\formedesurface{go˩bo˥}\newline
\classe{名词}\ton{LH}
\paradigme{\pcmn{:} \p{}}
\begin{définition}\peng{Livestock.}\end{définition}
\begin{définition}\pcmn{牲畜}\end{définition}
\begin{définition}\pfra{Bétail, animaux domestiques.}\end{définition}
\end{entrée}

\begin{entrée}
{go˧mi˧}{}{ⓔgo˧mi˧}\formedesurface{go˧mi˧}\newline
\classe{名词}\ton{M}
\paradigme{\pcmn{:} \p{}}
\begin{définition}\peng{Younger sister.}\end{définition}
\begin{définition}\pcmn{妹妹}\end{définition}
\begin{définition}\pfra{Petite soeur (employé aussi pour les cousines plus jeunes).}\end{définition}
\begin{exemple}\pnru{go˧mi˧=ɻæ˩}\hspace{5pt}\peng{|fg{associative}: younger sisters, younger cousins}\hspace{5pt}\pcmn{联想复数:妹妹们,表妹们}\hspace{5pt}\pfra{|fg{associatif}: les petites sœurs, les jeunes cousines}\end{exemple}
\end{entrée}

\begin{entrée}
{gɯ˩α}{₁}{ⓔgɯ˩αⓗ1}\formedesurface{gɯ˩˥}\newline
\classe{动词}\ton{Lα}
1\begin{définition}\peng{To believe.}\end{définition}
\begin{définition}\pcmn{相信}\end{définition}
\begin{définition}\pfra{Croire.}\end{définition}
\begin{exemple}\pnru{ʈʂʰɯ˧-ɳɯ˧ ʐwɤ˩-hĩ˩, | njɤ˧ | mɤ˧-gɯ˩!}\hspace{5pt}\peng{I do not believe what (s)he says!}\hspace{5pt}\pcmn{他说的话,我不相信!}\hspace{5pt}\pfra{Je ne crois pas ce qu'il dit! / Je ne le crois pas! / Je ne crois pas un mot de ce qu'il raconte!}\end{exemple}
\end{entrée}

\begin{entrée}
{gɯ˩α}{₂}{ⓔgɯ˩αⓗ2}\formedesurface{gɯ˩˥}\newline
\classe{形容词}\ton{Lα}
2\begin{définition}\peng{True, authentic, veritable.}\end{définition}
\begin{définition}\pcmn{真,真的}\end{définition}
\begin{définition}\pfra{Vrai, authentique, véritable.}\end{définition}
\begin{exemple}\pnru{mɤ˧-gɯ˩}\hspace{5pt}\peng{not true}\hspace{5pt}\pcmn{不是真的}\hspace{5pt}\pfra{pas vrai}\end{exemple}
\begin{exemple}\pnru{gɯ˩-hĩ˩˥}\hspace{5pt}\peng{|fg{nmlz}}\hspace{5pt}\pcmn{真的}\hspace{5pt}\pfra{|fg{nmlz}}\end{exemple}
\begin{exemple}\pnru{ə˩-gɯ˩˥?}\hspace{5pt}\peng{Is that true?}\hspace{5pt}\pcmn{真的吗?}\hspace{5pt}\pfra{c'est vrai?}\end{exemple}
\begin{exemple}\pnru{gɯ˩ wɤ˩-ɻ̍˥!}\hspace{5pt}\peng{It's really like that! / Yes, it is indeed true!}\hspace{5pt}\pcmn{就是真的啊! / 的确是这样啊!}\hspace{5pt}\pfra{C'est bien ça! / C'est réellement ainsi!}\end{exemple}
\begin{exemple}\pnru{gɯ˩-ʝi˥?}\hspace{5pt}\peng{Really?}\hspace{5pt}\pcmn{原来是这样吗?}\hspace{5pt}\pfra{C'est vrai? Vraiment?}\end{exemple}
\begin{exemple}\pnru{gɯ˩ ʂv̩˩ɖv̩˩˥}\hspace{5pt}\peng{to believe in (something); literally: ‘to think (that something is) true'}\hspace{5pt}\pcmn{相信}\hspace{5pt}\pfra{croire (quelqu'un, quelque chose): littéralement «penser que c'est vrai»}\end{exemple}
\begin{exemple}\pnru{gɯ˧ ʐwɤ˧}\hspace{5pt}\peng{to say the truth}\hspace{5pt}\pcmn{说实话}\hspace{5pt}\pfra{dire la vérité}\end{exemple}
\end{entrée}

\begin{entrée}
{gɯ˩ɭɯ˧˥}{}{ⓔgɯ˩ɭɯ˧˥}\formedesurface{gɯ˩ɭɯ˧˥}\newline
\classe{动词}\ton{LM+MH\#}\begin{définition}\peng{To rub, to knead (e.g. rub one's hands).}\end{définition}
\begin{définition}\pcmn{揉}\end{définition}
\begin{définition}\pfra{Frotter (ex.: se frotter les yeux, frotter un vêtement).}\end{définition}
\begin{exemple}\pnru{gɯ˩ɭɯ˧-ze˥}\hspace{5pt}\peng{|fg{pfv}}\hspace{5pt}\pcmn{揉了}\hspace{5pt}\pfra{|fg{pfv}}\end{exemple}
\begin{exemple}\pnru{le˧-gɯ˩ɭɯ˩+ze˩}\hspace{5pt}\peng{|fg{accomp} \_ |fg{pfv}}\hspace{5pt}\pcmn{揉了}\hspace{5pt}\pfra{|fg{accomp} \_ |fg{pfv}}\end{exemple}
\begin{exemple}\pnru{le˧-gɯ˩ɭɯ˩∼le˧-gɯ˩ɭɯ˩}\hspace{5pt}\peng{|fg{accomp} |fg{red} |fg{pfv}}\hspace{5pt}\pcmn{揉一揉}\hspace{5pt}\pfra{|fg{accomp} |fg{red} |fg{pfv}}\end{exemple}
\end{entrée}

\begin{entrée}
{gv̩˥}{}{ⓔgv̩˥}\formedesurface{gv̩˧}\newline
\classe{动词}\ton{H}\begin{définition}\peng{To cross, to get over (a river, a lake…).}\end{définition}
\begin{définition}\pcmn{过(一条河、一个湖……)}\end{définition}
\begin{définition}\pfra{Passer, traverser (un cours d'eau, un lac…).}\end{définition}
\begin{exemple}\pnru{dʑɯ˩ gv̩˩˥}\hspace{5pt}\peng{to cross a river}\hspace{5pt}\pcmn{过河}\hspace{5pt}\pfra{traverser l'eau/traverser la rivière}\end{exemple}
\end{entrée}

\begin{entrée}
{gv̩˥}{}{ⓔgv̩˥}\formedesurface{gv̩˧}\newline
\classe{名词}\ton{\#H}
\paradigme{\pcmn{:} \p{}}
\begin{définition}\peng{Manger.}\end{définition}
\begin{définition}\pcmn{马槽}\end{définition}
\begin{définition}\pfra{Auge, mangeoire.}\end{définition}
\begin{exemple}\pnru{ʐwæ˧gv̩\#˥}\hspace{5pt}\peng{horse's manger}\hspace{5pt}\pcmn{马槽}\hspace{5pt}\pfra{auge du cheval}\end{exemple}
\end{entrée}

\begin{entrée}
{gv̩˧}{₁}{ⓔgv̩˧ⓗ1}\formedesurface{gv̩˧}\newline
\classe{动词}\ton{Mγ}
1\begin{définition}\peng{To flow, to go by, to elapse (time); to take place, to occur (event).}\end{définition}
\begin{définition}\pcmn{过去 (时间)、过,发生}\end{définition}
\begin{définition}\pfra{S'écouler, passer (le temps passe); se passer (un événement).}\end{définition}
\begin{exemple}\pnru{le˧-gv̩˩-ze˩}\hspace{5pt}\peng{|fg{accomp} \_ |fg{pfv}}\hspace{5pt}\pcmn{已经过去了}\hspace{5pt}\pfra{|fg{accomp} \_ |fg{pfv}}\end{exemple}
\begin{exemple}\pnru{ɖɯ˧-ɭɯ˧ gv̩˧}\hspace{5pt}\peng{an hour goes by}\hspace{5pt}\pcmn{一个小时过去了}\hspace{5pt}\pfra{une heure se passe}\end{exemple}
\begin{exemple}\pnru{tsʰe˩-ɲi˩ gv̩˩-ze˥!}\hspace{5pt}\peng{Ten days have gone by/ten days have elapsed}\hspace{5pt}\pcmn{十天过去了}\hspace{5pt}\pfra{Dix jours ont passé!}\end{exemple}
\begin{exemple}\pnru{mɤ˧-gv̩˧-ze˧!}\hspace{5pt}\peng{It won't do! / It won't work! / It's no good!}\hspace{5pt}\pcmn{不好了! / 不行了!}\hspace{5pt}\pfra{(ah là là,) ça ne va plus!}\end{exemple}
\begin{exemple}\pnru{ʈʂʰɯ˧ne˧-ʝi˥ | gv̩˧, -tsɯ˩-mv̩˩!}\hspace{5pt}\peng{They say that's how it happened!}\hspace{5pt}\pcmn{据说是这样发生的!}\hspace{5pt}\pfra{Ca c'est passé comme ça, à ce qu'on raconte!}\end{exemple}
\end{entrée}

\begin{entrée}
{gv̩˧}{₂}{ⓔgv̩˧ⓗ2}\formedesurface{gv̩˧}\newline
\classe{形容词}\ton{M}
2\begin{définition}\peng{Good (good heart).}\end{définition}
\begin{définition}\pcmn{好(心好)}\end{définition}
\begin{définition}\pfra{Bon (bon cœur).}\end{définition}
\begin{exemple}\pnru{ɖwæ˧˥ | gv̩˧!}\hspace{5pt}\peng{|fg{intensive.very}}\hspace{5pt}\pcmn{很好!}\hspace{5pt}\pfra{|fg{intensif.très}}\end{exemple}
\begin{exemple}\pnru{mɤ˧-gv̩˧!}\hspace{5pt}\peng{|fg{neg}}\hspace{5pt}\pcmn{不好}\hspace{5pt}\pfra{|fg{neg}}\end{exemple}
\end{entrée}

\begin{entrée}
{gv̩˧}{₃}{ⓔgv̩˧ⓗ3}\formedesurface{gv̩˧}\newline
\classe{数词}\ton{M? H\#? (pas L)}
3\begin{définition}\peng{Nine.}\end{définition}
\begin{définition}\pcmn{九}\end{définition}
\begin{définition}\pfra{Neuf.}\end{définition}
\end{entrée}

\begin{entrée}
{gv̩˧}{₄}{ⓔgv̩˧ⓗ4}\formedesurface{gv̩˧}\newline
\classe{动词}\ton{M}
4\begin{définition}\peng{To be; to become (stative verb).}\end{définition}
\begin{définition}\pcmn{系词}\end{définition}
\begin{définition}\pfra{Être, devenir (verbe statif).}\end{définition}
\begin{exemple}\pnru{ʈʂʰɯ˧ | no˧ | ɲi˧gɤ˧ | ʂwæ˧-mɤ˧-gv̩˧!}\hspace{5pt}\peng{Her nose is less straight than yours! (About a little girl whose nose does not resemble her father's straight nose)}\hspace{5pt}\pcmn{她的鼻子没有你的直!(关于一个鼻子比较扁的小女孩)}\hspace{5pt}\pfra{elle a le nez moins droit que toi! (Au sujet d'une petite fille dont le nez ne ressemble pas au nez droit de son père)}\end{exemple}
\begin{exemple}\pnru{ʐæ˧ni˩ | mɤ˧-gv̩˧}\hspace{5pt}\peng{not tall, not impressive, not great-looking}\hspace{5pt}\pcmn{个子不高}\hspace{5pt}\pfra{pas bien grand (en taille), pas bien impressionnant}\end{exemple}
\end{entrée}

\begin{entrée}
{gv̩˩α}{₁}{ⓔgv̩˩αⓗ1}\newline
\classe{动词}
1
\sens{1}
\begin{définition}\peng{To prepare (a meal), to cook.}\end{définition}
\begin{définition}\pcmn{做(饭)}\end{définition}
\begin{définition}\pfra{Cuisiner, préparer (un repas, de la nourriture).}\end{définition}
\begin{exemple}\pnru{hɑ˧ gv̩˥}\hspace{5pt}\peng{to cook, to prepare a meal}\hspace{5pt}\pcmn{做饭}\hspace{5pt}\pfra{faire la cuisine, cuisiner}\end{exemple}
\begin{exemple}\pnru{le˧-gv̩˩-ze˩}\hspace{5pt}\peng{|fg{accomp} \_ |fg{pfv}}\hspace{5pt}\pcmn{做(饭)了}\hspace{5pt}\pfra{|fg{accomp} \_ |fg{pfv}}\end{exemple}
\begin{exemple}\pnru{njɤ˧ | hɑ˧ gv̩˥-bi˩!}\hspace{5pt}\peng{Let me do the cooking! / I'm doing the cooking!}\hspace{5pt}\pcmn{我来做饭吧!}\hspace{5pt}\pfra{je vais faire la cuisine!}\end{exemple}\sens{2}
\begin{définition}\peng{To construct, to build (a house).}\end{définition}
\begin{définition}\pcmn{盖、建 (房子)}\end{définition}
\begin{définition}\pfra{Construire (une maison).}\end{définition}
\begin{exemple}\pnru{ʑi˧qʰwɤ˧ gv̩˩}\hspace{5pt}\peng{to build a house}\hspace{5pt}\pcmn{建房}\hspace{5pt}\pfra{construire un bâtiment}\end{exemple}\sens{3}
\begin{définition}\peng{To repair; to make (a tool, a machine…).}\end{définition}
\begin{définition}\pcmn{修理、做出来(工具)}\end{définition}
\begin{définition}\pfra{Fabriquer ou réparer (un outil).}\end{définition}
\begin{exemple}\pnru{le˧-gv̩˧∼gv̩˥}\hspace{5pt}\peng{|fg{red}: to repair}\hspace{5pt}\pcmn{重叠:修理}\hspace{5pt}\pfra{|fg{red}: réparer}\end{exemple}
\begin{exemple}\pnru{le˧-gv̩˩ | le˧-tʰv̩˧-ze˧!}\hspace{5pt}\peng{It's repaired! / It's done! / I have finished doing it!}\hspace{5pt}\pcmn{修理好了!/ 修理出来了!}\hspace{5pt}\pfra{Ca y est, c'est réparé/c'est fabriqué/c'est fini!}\end{exemple}
\end{entrée}

\begin{entrée}
{gv̩˩α}{₂}{ⓔgv̩˩αⓗ2}\formedesurface{gv̩˩˥}\newline
\classe{动词}\ton{Lα}
2\begin{définition}\peng{To tidy up, to sort out.}\end{définition}
\begin{définition}\pcmn{收拾}\end{définition}
\begin{définition}\pfra{Ranger.}\end{définition}
\begin{exemple}\pnru{tʰi˧-gv̩˧∼gv̩˥}\hspace{5pt}\peng{|fg{dur}}\hspace{5pt}\pcmn{|fg{dur}}\hspace{5pt}\pfra{|fg{dur}}\end{exemple}
\begin{exemple}\pnru{ɖɯ˧-gv̩˧∼gv̩˥-ɻ̍˩}\hspace{5pt}\peng{to clear up a little}\hspace{5pt}\pcmn{收拾一下}\hspace{5pt}\pfra{ranger un peu}\end{exemple}
\begin{exemple}\pnru{le˧-gv̩˧∼gv̩˥ | tʰi˧-tɕɯ˥}\hspace{5pt}\peng{to tidy up and put (everything) into place}\hspace{5pt}\pcmn{收拾,摆好}\hspace{5pt}\pfra{ranger et bien mettre à sa place}\end{exemple}
\end{entrée}

\begin{entrée}
{gv̩˩α}{₃}{ⓔgv̩˩αⓗ3}\formedesurface{gv̩˩˥}\newline
\classe{动词}\ton{Lα}
3\begin{définition}\peng{To set (the sun sets), to decline.}\end{définition}
\begin{définition}\pcmn{落下(太阳落山)}\end{définition}
\begin{définition}\pfra{Se coucher (le soleil se couche), décliner.}\end{définition}
\begin{exemple}\pnru{ɲi˧mi˧ gv̩˩-se˩}\hspace{5pt}\peng{after the sun has set, after sunset}\hspace{5pt}\pcmn{在太阳落山之后,在太阳落山了以后}\hspace{5pt}\pfra{à la nuit tombée, une fois la nuit tombée, après le coucher du soleil}\end{exemple}
\begin{exemple}\pnru{ɲi˧mi˧ | le˧-gv̩˩-ze˩.}\hspace{5pt}\peng{The sun has set.}\hspace{5pt}\pcmn{太阳落山了。}\hspace{5pt}\pfra{Le soleil s'est couché.}\end{exemple}
\begin{exemple}\pnru{ɲi˧mi˧ | mɤ˧-gv̩˩-sɯ˩.}\hspace{5pt}\peng{The sun has not set yet.}\hspace{5pt}\pcmn{太阳还没有落。}\hspace{5pt}\pfra{Le soleil ne s'est pas encore couché.}\end{exemple}
\end{entrée}

\begin{entrée}
{gv̩˧dv̩˧}{}{ⓔgv̩˧dv̩˧}\formedesurface{gv̩˧dv̩˧}\newline
\classe{名词}\ton{M}
\paradigme{\pcmn{:} \p{}}
\begin{définition}\peng{Back.}\end{définition}
\begin{définition}\pcmn{脊背}\end{définition}
\begin{définition}\pfra{Dos.}\end{définition}
\end{entrée}

\begin{entrée}
{gv̩˧dv̩˧-gv̩˧mi˧}{}{ⓔgv̩˧dv̩˧-gv̩˧mi˧}\formedesurface{gv̩˧dv̩˧gv̩˧mi˧}\newline
\classe{名词}\ton{M}
\étymologie{
gv̩˧dv̩˧; gv̩˧mi˧
}
\paradigme{\pcmn{:} \p{}}
\begin{définition}\peng{Body.}\end{définition}
\begin{définition}\pcmn{身体}\end{définition}
\begin{définition}\pfra{Corps.}\end{définition}
\end{entrée}

\begin{entrée}
{gv̩˩dʑɯ˩}{}{ⓔgv̩˩dʑɯ˩}\formedesurface{gv̩˩dʑɯ˩˥}\newline
\classe{形容词}\ton{L}\begin{définition}\peng{Angry; afflicted.}\end{définition}
\begin{définition}\pcmn{生气}\end{définition}
\begin{définition}\pfra{En colère, affligé.}\end{définition}
\end{entrée}

\begin{entrée}
{gv̩˧kv̩˩}{}{ⓔgv̩˧kv̩˩}\formedesurface{gv̩˧kv̩˩}\newline
\classe{名词}\ton{L\#}\begin{définition}\peng{Intonation, way of speaking; can be used, by extension, to refer to tones.}\end{définition}
\begin{définition}\pcmn{语调,声调}\end{définition}
\begin{définition}\pfra{Intonation, manière de s'exprimer; par extension, peut désigner les tons.}\end{définition}
\begin{exemple}\pnru{gv̩˧kv̩˩-gv̩˩li˩ | ʐwɤ˩˥}\hspace{5pt}\peng{to speak with a pleasant style, to deliver one's speech with elegance}\hspace{5pt}\pcmn{说话说得好听、有口才、口若悬河、能言善辩}\hspace{5pt}\pfra{parler avec une élocution soignée/agréable}\end{exemple}
\end{entrée}

\begin{entrée}
{gv̩˩ɬi˩mi˩}{}{ⓔgv̩˩ɬi˩mi˩}\formedesurface{gv̩˩ɬi˩mi˩˥}\newline
\classe{名词}\ton{L}\begin{définition}\peng{9th month.}\end{définition}
\begin{définition}\pcmn{九月}\end{définition}
\begin{définition}\pfra{9e mois.}\end{définition}
\end{entrée}

\begin{entrée}
{gv̩˧mɑ˧}{}{ⓔgv̩˧mɑ˧}\formedesurface{gv̩˧mɑ˧}\newline
\classe{名词}\ton{M}\begin{définition}\peng{Masculine given name.}\end{définition}
\begin{définition}\pcmn{男性名字}\end{définition}
\begin{définition}\pfra{Prénom masculin.}\end{définition}
\begin{exemple}\pnru{hĩ˧ | ʈʂʰɯ˧-v̩˧, | gv̩˧mɑ˧ mv̩˧ʈʂæ˧˥!}\hspace{5pt}\peng{This person is called /gv̩˧mɑ˧/!}\hspace{5pt}\pcmn{这个人,名叫|fv{/gv̩˧mɑ˧/}!}\hspace{5pt}\pfra{Cette personne s'appelle /gv̩˧mɑ˧/ !}\end{exemple}
\end{entrée}

\begin{entrée}
{gv̩˧mi˧}{}{ⓔgv̩˧mi˧}\formedesurface{gv̩˧mi˧}\newline
\classe{名词}\ton{M}
\paradigme{\pcmn{:} \p{}}
\begin{définition}\peng{Body.}\end{définition}
\begin{définition}\pcmn{身体}\end{définition}
\begin{définition}\pfra{Corps.}\end{définition}
\end{entrée}

\begin{entrée}
{gv̩˩pʰæ˩}{}{ⓔgv̩˩pʰæ˩}\formedesurface{gv̩˩pʰæ˩˥}\newline
\classe{名词}\ton{L}
\paradigme{\pcmn{:} \p{}}
\begin{définition}\peng{Thin plank.}\end{définition}
\begin{définition}\pcmn{相当薄的木板}\end{définition}
\begin{définition}\pfra{Planche de bois fine: trois ou quatre centimètres.}\end{définition}
\end{entrée}

\begin{entrée}
{gv̩˧sɯ˥-pv̩˩}{}{ⓔgv̩˧sɯ˥-pv̩˩}\formedesurface{gv̩˧sɯ˥pv̩˩}\newline
\classe{名词}\ton{H\#-L}
\paradigme{\pcmn{:} \p{}}
\begin{définition}\peng{Shoulderblade, scapula.}\end{définition}
\begin{définition}\pcmn{肩胛骨}\end{définition}
\begin{définition}\pfra{Omoplate.}\end{définition}
\end{entrée}

\begin{entrée}
{gv̩˧tɕʰɯ˧˥}{}{ⓔgv̩˧tɕʰɯ˧˥}\formedesurface{gv̩˧tɕʰɯ˧˥}\newline
\classe{动词}\ton{MH\#}\begin{définition}\peng{To catch a cold.}\end{définition}
\begin{définition}\pcmn{着凉}\end{définition}
\begin{définition}\pfra{Prendre froid, attraper un rhume, attraper froid.}\end{définition}
\end{entrée}

\begin{entrée}
{gv̩˧tsʰi˩}{}{ⓔgv̩˧tsʰi˩}\formedesurface{gv̩˧tsʰi˩}\newline
\classe{数词}\ton{L\#}\begin{définition}\peng{90.}\end{définition}
\begin{définition}\pcmn{90}\end{définition}
\begin{définition}\pfra{90.}\end{définition}
\end{entrée}

\begin{entrée}
{gwɤ˩α}{}{ⓔgwɤ˩α}\formedesurface{gwɤ˩˥}\newline
\classe{动词}\ton{Lα}\begin{définition}\peng{To sing.}\end{définition}
\begin{définition}\pcmn{唱、唱歌}\end{définition}
\begin{définition}\pfra{Chanter.}\end{définition}
\begin{exemple}\pnru{njɤ˧ | ɖɯ˧-ɖʐo˩ | gwɤ˩-ze˥!}\hspace{5pt}\peng{I have sung a song!}\hspace{5pt}\pcmn{我唱了一首歌!}\hspace{5pt}\pfra{j'ai chanté une chanson!}\end{exemple}
\begin{exemple}\pnru{no˧ | ɖɯ˧-ɖʐo˩ gwɤ˩!}\hspace{5pt}\peng{Please sing a song! / Go ahead and sing us a song!}\hspace{5pt}\pcmn{你唱一首吧!}\hspace{5pt}\pfra{chante-nous une chanson!}\end{exemple}
\begin{exemple}\pnru{ɖɯ˧-kʰwɤ˧ gwɤ˥}\hspace{5pt}\peng{to sing a song}\hspace{5pt}\pcmn{唱一下}\hspace{5pt}\pfra{chanter une chanson}\end{exemple}
\begin{exemple}\pnru{ɖɯ˧-kʰwɤ˧ gwɤ˥-ɻ̍˩}\hspace{5pt}\peng{to sing a song}\hspace{5pt}\pcmn{唱一下}\hspace{5pt}\pfra{chanter une chanson}\end{exemple}
\begin{exemple}\pnru{nɑ˩-gwɤ˥}\hspace{5pt}\peng{Na songs}\hspace{5pt}\pcmn{摩梭民歌}\hspace{5pt}\pfra{les chansons des Na}\end{exemple}
\begin{exemple}\pnru{ʈʂʰɯ˧ | nɑ˩-gwɤ˥ F | kv̩˧˥! | hæ˧-gwɤ˩ F | kv̩˧˥! | ʁo˧dzi˩-gwɤ˩ F | kv̩˧-ʝi˥! |}\hspace{5pt}\peng{He can sing (lots of different styles:) Na songs! and also Chinese (Han) songs! and also Tibetan songs!}\hspace{5pt}\pcmn{他会唱很多种风格的歌曲:摩梭的,会唱!汉族的,会唱!藏族的,会唱!}\hspace{5pt}\pfra{Il sait chanter (toutes sortes de styles:) les chansons na! les chansons chinoises! les chansons tibétaines!}\end{exemple}
\end{entrée}

\begin{entrée}
{gwɤ˩∼gwɤ˧˥}{}{ⓔgwɤ˩∼gwɤ˧˥}\formedesurface{gwɤ˩gwɤ˧˥}\newline
\classe{动词}\ton{L}\begin{définition}\peng{To stroll, to ramble, to roam.}\end{définition}
\begin{définition}\pcmn{逛,玩,游}\end{définition}
\begin{définition}\pfra{Se promener, se divertir, faire du tourisme.}\end{définition}
\begin{exemple}\pnru{le˧-gwɤ˩∼gwɤ˩ | le˧-tsʰɯ˩-ze˩!}\hspace{5pt}\peng{So you are back from a stroll! / You are back from your little walk, eh?}\hspace{5pt}\pcmn{你已经散步回来了!}\hspace{5pt}\pfra{(tu) reviens de promenade!/ Alors comme ça te voilà revenu de ta promenade!}\end{exemple}
\begin{exemple}\pnru{ʈʂʰɯ˧ | gwɤ˩∼gwɤ˩-hɯ˩-ze˥!}\hspace{5pt}\peng{(S)he has gone out for a walk!}\hspace{5pt}\pcmn{他散步去了!}\hspace{5pt}\pfra{Il/elle est parti(e) se promener!}\end{exemple}
\begin{exemple}\pnru{æ˧ʂæ˧ gwɤ˩; | qv̩˧ɻ̍˧ gwɤ˥; | nɑ˩tsʰi˩ gwɤ˥}\hspace{5pt}\peng{“to walk around Mount /æ˧ʂæ˧/; to walk around Mount /qv̩˧ɻ̍\#˥/; to walk around Mount /nɑ˩tsʰi˩/": i.e. to do rituals on these mountains, in particular to obtain fertility, or to obtain a cure for a child who did not learn to speak.}\hspace{5pt}\pcmn{绕|fv{æ˧ʂæ˧}山,绕|fv{qv̩˧ɻ̍\#˥}山,绕|fv{nɑ˩tsʰi˩}山(做“绕山”仪式,为了求生子等)}\hspace{5pt}\pfra{«faire le tour de la montagne /æ˧ʂæ˧/; faire le tour de la montagne /qv̩˧ɻ̍\#˥/; faire le tour de la montagne /nɑ˩tsʰi˩/»: façon de désigner les rites qu'on pratiquait sur ces montagnes: pour des enfants qui tardaient à apprendre à parler, etc.}\end{exemple}
\end{entrée}

\begin{entrée}
{gwɤ˩ʝi˧}{}{ⓔgwɤ˩ʝi˧}\formedesurface{gwɤ˩ʝi˥}\newline
\classe{助词}\ton{LM}\begin{définition}\peng{In good order.}\end{définition}
\begin{définition}\pcmn{整齐}\end{définition}
\begin{définition}\pfra{En ordre/rangé.}\end{définition}
\begin{exemple}\pnru{tso˧∼tso˧ | gwɤ˩ʝi˧ tʰi˧-tɕɯ˥ |}\hspace{5pt}\peng{to put things in good order}\hspace{5pt}\pcmn{把东西摆整齐}\hspace{5pt}\pfra{mettre des choses en ordre, ranger des choses}\end{exemple}
\end{entrée}

\newpage\caractère{ɣ}

\begin{entrée}
{ɣɯ˥}{}{ⓔɣɯ˥}\formedesurface{ɣɯ˧}\newline
\classe{形容词}\ton{H}\begin{définition}\peng{Competent, able.}\end{définition}
\begin{définition}\pcmn{能干、好(做事情做得好)}\end{définition}
\begin{définition}\pfra{Habile, compétent, bon.}\end{définition}
\begin{exemple}\pnru{mɤ˧-ɣɯ˥}\hspace{5pt}\peng{|fg{neg}}\hspace{5pt}\pcmn{不能干}\hspace{5pt}\pfra{|fg{neg}}\end{exemple}
\begin{exemple}\pnru{ʈʂʰɯ˧-ɳɯ˧, | bɑ˩lɑ˩ hwæ˧ | ɣɯ˧!}\hspace{5pt}\peng{He/she is very good at buying clothes! / He/she has talent for choosing clothes!}\hspace{5pt}\pcmn{他很会买衣服!}\hspace{5pt}\pfra{Il/elle s'entend à acheter des vêtements! / Il/elle a du talent pour acheter des vêtements!}\end{exemple}
\end{entrée}

\begin{entrée}
{ɣɯ˧}{}{ⓔɣɯ˧}\formedesurface{ɣɯ˧}\newline
\classe{名词}\ton{\#H}
\paradigme{\pcmn{:} \p{}}
\begin{définition}\peng{Cloth.}\end{définition}
\begin{définition}\pcmn{布料}\end{définition}
\begin{définition}\pfra{Tissu.}\end{définition}
\begin{exemple}\pnru{ɣɯ˧dzo˩, | ɣɯ˧ni˧˥, | ɣɯ˧, | ɖɯ˧-ʑi˩ ɲi˩-ze˩!}\hspace{5pt}\peng{The weaving-machine, the bamboo structure keeping the threads together, and fabric: these belong to the same family! / these are all part of the same sphere!}\hspace{5pt}\pcmn{织布机、竹子的框(让线不乱混)、布料,属于同一类!(直译:“都是一家的!”)}\hspace{5pt}\pfra{Le métier à tisser, la structure en bambou qui maintient les fils, le tissu, c'est de la même famille! / ça forme une famille! (Commentaire sémantique de la locutrice, au cours d'une séance où il était question de tissus et de tissage)}\end{exemple}
\end{entrée}

\begin{entrée}
{ɣɯ˩˥}{}{ⓔɣɯ˩˥}\formedesurface{ɣɯ˩˥}\newline
\classe{名词}\ton{LH}
\paradigme{\pcmn{:} \p{}}
\begin{définition}\peng{Skin.}\end{définition}
\begin{définition}\pcmn{皮肤}\end{définition}
\begin{définition}\pfra{Peau.}\end{définition}
\begin{exemple}\pnru{ɣɯ˩ dzɯ˩˥}\hspace{5pt}\peng{to eat skin}\hspace{5pt}\pcmn{吃皮}\hspace{5pt}\pfra{manger de la peau}\end{exemple}
\begin{exemple}\pnru{ɣɯ˩˥ | ɖɯ˧-ʂɯ˩ pʰv˩}\hspace{5pt}\peng{literally ‘to shed one's skin once'; meaning: to be worn out and physically hurt (by an exhausting task, such as felling trees high up on the mountains and carrying lumber back to the plain)}\hspace{5pt}\pcmn{直译:‘脱皮一次’。意思:疲劳而受伤(因为做了很辛苦的工作,如:在深山老林砍树、扛树干回到坝子)}\hspace{5pt}\pfra{littéralement ‘perdre sa peau'; sens: être épuisé et blessé par un travail ardu (par exemple: abattre des arbres sur la montagne et ramener les grumes jusque dans la plaine)}\end{exemple}
\end{entrée}

\begin{entrée}
{ɣɯ˧bo˩}{}{ⓔɣɯ˧bo˩}\formedesurface{ɣɯ˧bo˩}\newline
\classe{名词}\ton{L\#}
\paradigme{\pcmn{:} \p{}}
\begin{définition}\peng{Weft, weft thread, pick.}\end{définition}
\begin{définition}\pcmn{纬线、纬纱}\end{définition}
\begin{définition}\pfra{Trame (lorsqu'on tisse, il y a du fil de trame, et du fil de chaîne); la trame désigne l'ensemble des fils de trame.}\end{définition}
\end{entrée}

\begin{entrée}
{ɣɯ˧dzo˩}{}{ⓔɣɯ˧dzo˩}\formedesurface{ɣɯ˧dzo˩}\newline
\classe{名词}\ton{L\#}
\paradigme{\pcmn{:} \p{}}
\begin{définition}\peng{Loom.}\end{définition}
\begin{définition}\pcmn{织布机}\end{définition}
\begin{définition}\pfra{Métier à tisser, appareil à tisser.}\end{définition}
\end{entrée}

\begin{entrée}
{ɣɯ˩kɯ˧˥}{}{ⓔɣɯ˩kɯ˧˥}\newline
\classe{名词}
\sens{1}\paradigme{\pcmn{:} \p{}}
\begin{définition}\peng{Peel, rind.}\end{définition}
\begin{définition}\pcmn{皮、鸡蛋壳、麦麸}\end{définition}
\begin{définition}\pfra{Pelure de fruit ou de légume.}\end{définition}
\begin{exemple}\pnru{pʰi˩ko˧-ɣɯ˩kɯ˩}\hspace{5pt}\peng{peel of an apple}\hspace{5pt}\pcmn{苹果皮}\hspace{5pt}\pfra{pelure de pomme}\end{exemple}
\begin{exemple}\pnru{jɤ˩jo˧-ɣɯ˥kɯ˩}\hspace{5pt}\peng{potato peel}\hspace{5pt}\pcmn{洋芋皮}\hspace{5pt}\pfra{pelure de pomme de terre}\end{exemple}\sens{2}
\begin{définition}\peng{Fur, pelt, skin (of animal).}\end{définition}
\begin{définition}\pcmn{皮}\end{définition}
\begin{définition}\pfra{Fourrure, peau d'animal.}\end{définition}
\begin{exemple}\pnru{ʂe˧-ɣɯ˥kɯ˩}\hspace{5pt}\peng{skin of meat, i.e. skin on a piece of meat}\hspace{5pt}\pcmn{肉皮:鸡皮、猪肉的皮……}\hspace{5pt}\pfra{peau de la viande (peau de poulet, couenne de porc…)}\end{exemple}\sens{3}
\begin{définition}\peng{Eggshell.}\end{définition}
\begin{définition}\pcmn{蛋壳}\end{définition}
\begin{définition}\pfra{Coquille (d'oeuf).}\end{définition}\sens{4}
\begin{définition}\peng{Bran.}\end{définition}
\begin{définition}\pcmn{麸}\end{définition}
\begin{définition}\pfra{Son (de céréale).}\end{définition}
\begin{exemple}\pnru{dze˧ɭɯ˧-ɣɯ˩kɯ˩}\hspace{5pt}\peng{wheat bran}\hspace{5pt}\pcmn{小麦麸}\hspace{5pt}\pfra{son de blé}\end{exemple}
\end{entrée}

\begin{entrée}
{ɣɯ˩-nɑ˥mi˩}{}{ⓔɣɯ˩-nɑ˥mi˩}\formedesurface{ɣɯ˩nɑ˥mi˩}\newline
\classe{名词}\ton{\#H-}
\paradigme{\pcmn{:} \p{}}
\begin{définition}\peng{Yi (derogatory term).}\end{définition}
\begin{définition}\pcmn{彝族(带偏见的说法)}\end{définition}
\begin{définition}\pfra{Terme péjoratif pour désigner les Yi (groupe ethnique): «les peaux-noires».}\end{définition}
\begin{exemple}\pnru{ɣɯ˩-nɑ˥mi˩-zo˩}\hspace{5pt}\peng{Yi man}\hspace{5pt}\pcmn{彝族男人}\hspace{5pt}\pfra{homme yi}\end{exemple}
\begin{exemple}\pnru{ɣɯ˩-nɑ˥mi˩-mv̩˩}\hspace{5pt}\peng{Yi woman}\hspace{5pt}\pcmn{彝族女人}\hspace{5pt}\pfra{femme yi}\end{exemple}
\end{entrée}

\begin{entrée}
{ɣɯ˧ni˧˥}{}{ⓔɣɯ˧ni˧˥}\formedesurface{ɣɯ˧ni˧˥}\newline
\classe{名词}\ton{MH\#}
\paradigme{\pcmn{:} \p{}}
\begin{définition}\peng{A part of the loom: a small bamboo structure hanging from the top of the loom, keeping the threads together.}\end{définition}
\begin{définition}\pcmn{织布机的一部分:竹子的框,让线不乱混}\end{définition}
\begin{définition}\pfra{Petite structure en bambou maintenant les fils du métier à tisser; ses fils (blancs) sont verticaux, et passent à travers la trame.}\end{définition}
\end{entrée}

\newpage\caractère{h}

\begin{entrée}
{hɑ˥}{}{ⓔhɑ˥}\formedesurface{hɑ˧}\newline
\classe{名词}\ton{\#H}\begin{définition}\peng{Food.}\end{définition}
\begin{définition}\pcmn{饭,米饭}\end{définition}
\begin{définition}\pfra{Nourriture.}\end{définition}
\begin{exemple}\pnru{hɑ˧-ʈv̩˧∼ʈv̩˥}\hspace{5pt}\peng{ball of cereals}\hspace{5pt}\pcmn{饭坨坨、饭团}\hspace{5pt}\pfra{boule de céréale (riz ou autre)}\end{exemple}
\begin{exemple}\pnru{hɑ˧ dzɯ˧}\hspace{5pt}\peng{to eat}\hspace{5pt}\pcmn{吃饭}\hspace{5pt}\pfra{manger}\end{exemple}
\begin{exemple}\pnru{ʈʂʰɯ˧ | hɑ˧ dzɯ˧-dʑo˩!}\hspace{5pt}\peng{(S)he is eating!}\hspace{5pt}\pcmn{他在吃饭!}\hspace{5pt}\pfra{Il/elle est en train de manger!}\end{exemple}
\begin{exemple}\pnru{hɑ˧ʂɯ˩}\hspace{5pt}\peng{fresh cereals (freshly reaped; they yield especially good-tasting cakes)}\hspace{5pt}\pcmn{新鲜的粮食(可以用来烤很香的饼)}\hspace{5pt}\pfra{céréales fraîches (juste après la cueillette; on en prépare des galettes particulièrement savoureuses)}\end{exemple}
\end{entrée}

\begin{entrée}
{hɑ˩α}{}{ⓔhɑ˩α}\formedesurface{hɑ˩˥}\newline
\classe{动词}\ton{Lα}\begin{définition}\peng{To open (one's eyes).}\end{définition}
\begin{définition}\pcmn{睁开(眼睛)}\end{définition}
\begin{définition}\pfra{Ouvrir (les yeux); s'ouvrir (un sac).}\end{définition}
\begin{exemple}\pnru{tʰi˧-hɑ˩}\hspace{5pt}\peng{|fg{dur}}\hspace{5pt}\pcmn{|fg{dur}}\hspace{5pt}\pfra{|fg{dur}}\end{exemple}
\begin{exemple}\pnru{njɤ˩ɭɯ˥ | gɤ˩-hɑ˥ |}\hspace{5pt}\peng{to open one's eyes}\hspace{5pt}\pcmn{睁开眼睛}\hspace{5pt}\pfra{ouvrir les yeux}\end{exemple}
\begin{exemple}\pnru{njɤ˩ɭɯ˧ hɑ˩}\hspace{5pt}\peng{to open one's eyes}\hspace{5pt}\pcmn{睁开眼睛}\hspace{5pt}\pfra{ouvrir les yeux}\end{exemple}
\end{entrée}

\begin{entrée}
{hɑ˧bɤ˥}{}{ⓔhɑ˧bɤ˥}\formedesurface{hɑ˧bɤ˥}\newline
\classe{名词}\ton{H\#}
\paradigme{\pcmn{:} \p{}}
\begin{définition}\peng{Corncob.}\end{définition}
\begin{définition}\pcmn{玉米棒子}\end{définition}
\begin{définition}\pfra{Épi de maïs.}\end{définition}
\begin{exemple}\pnru{qʰɑ˧dze˧-hɑ˧bɤ˥}\hspace{5pt}\peng{sweetcorn ear}\hspace{5pt}\pcmn{玉米棒子}\hspace{5pt}\pfra{maïs en épi; épi de maïs}\end{exemple}
\end{entrée}

\begin{entrée}
{hɑ˧-bv̩˥∼bv̩˩-di˩}{}{ⓔhɑ˧-bv̩˥∼bv̩˩-di˩}\formedesurface{hɑ˧bv̩˥bv̩˩di˩}\newline
\classe{名词}\ton{\#H-}
\paradigme{\pcmn{:} \p{}}
\begin{définition}\peng{Rice steamer.}\end{définition}
\begin{définition}\pcmn{甑}\end{définition}
\begin{définition}\pfra{Étuve pour le riz.}\end{définition}
\end{entrée}

\begin{entrée}
{hɑ˧-gv̩˥-di˩}{}{ⓔhɑ˧-gv̩˥-di˩}\formedesurface{hɑ˧gv̩˥di˩}\newline
\classe{名词}\ton{H\#-}\begin{définition}\peng{Stove.}\end{définition}
\begin{définition}\pcmn{炉子、灶头}\end{définition}
\begin{définition}\pfra{Fourneau.}\end{définition}
\end{entrée}

\begin{entrée}
{hɑ˧ɭɯ\#˥}{}{ⓔhɑ˧ɭɯ\#˥}\formedesurface{hɑ˧ɭɯ˧}\newline
\classe{名词}\ton{\#H}\begin{définition}\peng{Cereals.}\end{définition}
\begin{définition}\pcmn{粮食}\end{définition}
\begin{définition}\pfra{Céréales.}\end{définition}
\end{entrée}

\begin{entrée}
{hɑ˧mi˥}{}{ⓔhɑ˧mi˥}\formedesurface{hɑ˧mi˥}\newline
\classe{动词}\ton{H\#}\begin{définition}\peng{To beg.}\end{définition}
\begin{définition}\pcmn{讨饭}\end{définition}
\begin{définition}\pfra{Mendier.}\end{définition}
\begin{exemple}\pnru{hɑ˧mi˥-hĩ˩}\hspace{5pt}\peng{\_ |fg{rel}: beggar, [person] who begs}\hspace{5pt}\pcmn{要饭的、乞丐}\hspace{5pt}\pfra{\_ |fg{rel}: mendiant, [personne] qui mendie}\end{exemple}
\end{entrée}

\begin{entrée}
{hɑ˧pv̩˩}{}{ⓔhɑ˧pv̩˩}\formedesurface{hɑ˧pv̩˩}\newline
\classe{名词}\ton{L\#}\begin{définition}\peng{‘dry' cooked rice: the type of rice usually served at meals, as distinct from watery rice gruel.}\end{définition}
\begin{définition}\pcmn{干的米饭(与稀饭不同)}\end{définition}
\begin{définition}\pfra{Riz cuit ‘sec': le riz tel qu'il est servi aux repas, par opposition avec le gruau de riz.}\end{définition}
\end{entrée}

\begin{entrée}
{hɑ˧ʂɯ˥}{}{ⓔhɑ˧ʂɯ˥}\formedesurface{hɑ˧ʂɯ˥}\newline
\classe{连词}\ton{H\#}\begin{définition}\peng{Gap-filler, borrowed from the Chinese: “still/also…".}\end{définition}
\begin{définition}\pcmn{还是(汉语借词)}\end{définition}
\begin{définition}\pfra{Explétif, emprunté au chinois: ‘quand même/aussi…’.}\end{définition}
\end{entrée}

\begin{entrée}
{hɑ˧-ʐwɤ˩}{}{ⓔhɑ˧-ʐwɤ˩}\formedesurface{hɑ˧ʐwɤ˩}\newline
\classe{形容词}\ton{L\#}\begin{définition}\peng{Hungry.}\end{définition}
\begin{définition}\pcmn{饿(饭)}\end{définition}
\begin{définition}\pfra{Avoir faim.}\end{définition}
\end{entrée}

\begin{entrée}
{hɑ̃˧˥}{₁}{ⓔhɑ̃˧˥ⓗ1}\formedesurface{hɑ̃˧˥}\newline
\classe{动词}\ton{MH}
1\begin{définition}\peng{To put out, to extinguish (e.g. the fire of the stove).}\end{définition}
\begin{définition}\pcmn{把火炉灭了}\end{définition}
\begin{définition}\pfra{Éteindre (par exemple le feu du foyer).}\end{définition}
\begin{exemple}\pnru{mv̩˧ le˧-hɑ̃˧˥}\hspace{5pt}\peng{to put out the fire}\hspace{5pt}\pcmn{灭火}\hspace{5pt}\pfra{éteindre le feu}\end{exemple}
\end{entrée}

\begin{entrée}
{hɑ̃˧˥}{₂}{ⓔhɑ̃˧˥ⓗ2}\formedesurface{hɑ̃˧˥}\newline
\classe{动词}\ton{MH}
2
\sens{1}
\begin{définition}\peng{To spend the night (at a certain place).}\end{définition}
\begin{définition}\pcmn{过夜}\end{définition}
\begin{définition}\pfra{Passer la nuit.}\end{définition}
\begin{exemple}\pnru{ɖɯ˧-hɑ̃˧ tʰi˥-hɑ̃˩ |}\hspace{5pt}\peng{to spend a night (somewhere), to stay for the night}\hspace{5pt}\pcmn{过夜}\hspace{5pt}\pfra{se reposer une soirée, passer une soirée/nuitée (qq part)}\end{exemple}
\begin{exemple}\pnru{ʑi˧qʰwɤ˧ ɖɯ˧-ɭɯ˧-qo˧ hɑ̃˧˥}\hspace{5pt}\peng{to spend the night in a house}\hspace{5pt}\pcmn{在一个人家过夜}\hspace{5pt}\pfra{passer la nuit dans une maison}\end{exemple}\sens{2}
\begin{définition}\peng{To perch, to rest: a bee lands on a flower, a bird lands on a branch}\end{définition}
\begin{définition}\pcmn{栖息、休息}\end{définition}
\begin{définition}\pfra{Se percher, se poser: une abeille se pose sur une fleur}\end{définition}
\begin{exemple}\pnru{v˩dze˩˥ | tʰi˧-hɑ̃˧-ze˥!}\hspace{5pt}\peng{(Look,) the bird has perched / has landed!}\hspace{5pt}\pcmn{(你看,)鸟栖息了!}\hspace{5pt}\pfra{(Regarde,) l'oiseau s'est posé / s'est perché!}\end{exemple}
\end{entrée}

\begin{entrée}
{hɑ̃˧˥α}{}{ⓔhɑ̃˧˥α}\formedesurface{ɖɯ˧ hɑ̃˧˥}\newline
\classe{量词}\ton{MHα}\begin{définition}\peng{Classifier for nights.}\end{définition}
\begin{définition}\pcmn{量词:夜}\end{définition}
\begin{définition}\pfra{Nuit; par extension: utilisé pour le décompte des jours.}\end{définition}
\begin{exemple}\pnru{ɖɯ˧-hɑ̃˧˥}\hspace{5pt}\peng{one night}\hspace{5pt}\pcmn{一夜}\hspace{5pt}\pfra{une nuit}\end{exemple}
\begin{exemple}\pnru{tsʰe˩-hɑ̃˩˥}\hspace{5pt}\peng{ten nights = ten days}\hspace{5pt}\pcmn{十夜(等于十天)}\hspace{5pt}\pfra{dix soirées=10 jours}\end{exemple}
\begin{exemple}\pnru{ɖɯ˧-hɑ̃˧ lɑ˥-dʑo˩!}\hspace{5pt}\peng{There is only one day left!}\hspace{5pt}\pcmn{只有一个晚上了!}\hspace{5pt}\pfra{Il ne reste qu'une soirée!}\end{exemple}
\begin{exemple}\pnru{ɖɯ˧-hɑ̃˧-ɳɯ˥ | le˧-li˧-le˧-se˩-ze˩!}\hspace{5pt}\peng{He has entirely read it in one night! / He has read the whole (book) in just one night! (Imagined context: someone is given a book; he finishes reading it within a day)}\hspace{5pt}\pcmn{一个晚上就读完了!/一天之内都读完了!(情景:送一个人一本书,他马上全部读完)}\hspace{5pt}\pfra{(Il) a tout lu en deux jours! (contexte imaginé: on offre un livre à quelqu'un; en deux jours il a tout lu)}\end{exemple}
\end{entrée}

\begin{entrée}
{hɑ̃˧mo˥}{}{ⓔhɑ̃˧mo˥}\formedesurface{hɑ̃˧mo˥}\newline
\classe{形容词}\ton{H\#}\begin{définition}\peng{Old (person).}\end{définition}
\begin{définition}\pcmn{年老}\end{définition}
\begin{définition}\pfra{Âgé, vieux (personne humaine).}\end{définition}
\begin{exemple}\pnru{hĩ˧ ʈʂʰɯ˧-v̩˧ | hɑ̃˧mo˥ | ʐwæ˩˥!}\hspace{5pt}\peng{This person is extremely old/extremely advanced in years!}\hspace{5pt}\pcmn{这个人,年纪非常大!}\hspace{5pt}\pfra{Cette personne est très âgée!}\end{exemple}
\end{entrée}

\begin{entrée}
{hæ˧}{}{ⓔhæ˧}\formedesurface{hæ˧}\newline
\classe{名词}\ton{M}
\paradigme{\pcmn{:} \p{}}
\begin{définition}\peng{Chinese (Han).}\end{définition}
\begin{définition}\pcmn{汉人}\end{définition}
\begin{définition}\pfra{Chinois (Han).}\end{définition}
\begin{exemple}\pnru{hæ˧-mi\#˥}\hspace{5pt}\peng{a Chinese woman, a Han Chinese woman}\hspace{5pt}\pcmn{汉族女人}\hspace{5pt}\pfra{une femme chinoise, une Chinoise (Han)}\end{exemple}
\begin{exemple}\pnru{hæ˧-mv̩˧ hæ˧-di˧˥}\hspace{5pt}\peng{(Han) Chinese territory: Chengdu, Kunming…}\hspace{5pt}\pcmn{汉族地区,包括成都、昆明等等}\hspace{5pt}\pfra{le territoire des Chinois (Han): Chengdu, Kunming…}\end{exemple}
\begin{exemple}\pnru{hæ˧-di˩}\hspace{5pt}\peng{(Han) Chinese territory: Chengdu, Kunming…; used to mean ‘the south'}\hspace{5pt}\pcmn{汉族地区,包括成都、昆明等等,来代指南方}\hspace{5pt}\pfra{le territoire des Chinois (Han): Chengdu, Kunming…; l'expression est employée pour désigner la direction du sud}\end{exemple}
\begin{exemple}\pnru{hæ˧-zo˧bæ˩}\hspace{5pt}\peng{Han Chinese man (derogatory: literally ‘Chinese idiot')}\hspace{5pt}\pcmn{汉男人(带偏见的称呼)}\hspace{5pt}\pfra{homme chinois han (terme péjoratif: littéralement ‘idiot de Chinois')}\end{exemple}
\end{entrée}

\begin{entrée}
{hæ˧˥}{₁}{ⓔhæ˧˥ⓗ1}\formedesurface{hæ˧˥}\newline
\classe{形容词}
1
\sens{1}
\begin{définition}\peng{Supple, lithe.}\end{définition}
\begin{définition}\pcmn{软、柔软(树枝……)}\end{définition}
\begin{définition}\pfra{Souple, mou (branche…).}\end{définition}
\begin{exemple}\pnru{hæ˧njæ˧˥ ◊ -gv̩˩}\hspace{5pt}\peng{soft, lithe, supple}\hspace{5pt}\pcmn{软、柔软(树枝……)}\hspace{5pt}\pfra{souple}\end{exemple}\sens{2}
\begin{définition}\peng{Thin, watery (soup, gruel).}\end{définition}
\begin{définition}\pcmn{稀(粥、汤)}\end{définition}
\begin{définition}\pfra{Léger, clair, délayé (gruau, soupe…).}\end{définition}
\end{entrée}

\begin{entrée}
{hæ˧˥}{₂}{ⓔhæ˧˥ⓗ2}\formedesurface{hæ˧˥}\newline
\classe{名词}\ton{MH}
2\begin{définition}\peng{Lime.}\end{définition}
\begin{définition}\pcmn{石灰}\end{définition}
\begin{définition}\pfra{Chaux.}\end{définition}
\begin{exemple}\pnru{hæ˧ hwæ˥}\hspace{5pt}\peng{to buy lime}\hspace{5pt}\pcmn{买石灰}\hspace{5pt}\pfra{acheter de la chaux}\end{exemple}
\begin{exemple}\pnru{hæ˧ tɕʰi˥}\hspace{5pt}\peng{to sell lime}\hspace{5pt}\pcmn{卖石灰}\hspace{5pt}\pfra{vendre de la chaux}\end{exemple}
\begin{exemple}\pnru{hæ˧ ki˥}\hspace{5pt}\peng{to give lime}\hspace{5pt}\pcmn{给石灰}\hspace{5pt}\pfra{donner de la chaux}\end{exemple}
\begin{exemple}\pnru{hæ˧ dv̩˥}\hspace{5pt}\peng{to dig lime}\hspace{5pt}\pcmn{挖石灰}\hspace{5pt}\pfra{piocher de la chaux}\end{exemple}
\begin{exemple}\pnru{hæ˧ bæ˥}\hspace{5pt}\peng{to sweep lime}\hspace{5pt}\pcmn{扫石灰}\hspace{5pt}\pfra{balayer de la chaux}\end{exemple}
\begin{exemple}\pnru{hæ˧ gɤ˥}\hspace{5pt}\peng{to carry lime}\hspace{5pt}\pcmn{扛石灰}\hspace{5pt}\pfra{porter de la chaux}\end{exemple}
\end{entrée}

\begin{entrée}
{hæ˩α}{}{ⓔhæ˩α}\formedesurface{hæ˩˥}\newline
\classe{动词}\ton{Lα}\begin{définition}\peng{To harm, to cause trouble.}\end{définition}
\begin{définition}\pcmn{祸害、害}\end{définition}
\begin{définition}\pfra{Causer du tort à (note: ce mot n'est pas un emprunt, la ressemble avec le mandarin est une coïncidence).}\end{définition}
\begin{exemple}\pnru{hĩ˧ hæ˥}\hspace{5pt}\peng{to harm people}\hspace{5pt}\pcmn{害人}\hspace{5pt}\pfra{causer du tort aux gens}\end{exemple}
\begin{exemple}\pnru{hĩ˧ hæ˥-kv̩˩}\hspace{5pt}\peng{who can harm people; cruel}\hspace{5pt}\pcmn{会害人的、残忍、凶狠}\hspace{5pt}\pfra{qui est susceptible de causer du tort/d'être cruel}\end{exemple}
\begin{exemple}\pnru{hĩ˧ hæ˥-zo˩}\hspace{5pt}\peng{terrifying person, frightening person, creepy person}\hspace{5pt}\pcmn{可怕的人}\hspace{5pt}\pfra{homme terrifiant}\end{exemple}
\end{entrée}

\begin{entrée}
{hæ˧di˩-ʈæ˩bɤ˩}{}{ⓔhæ˧di˩-ʈæ˩bɤ˩}\formedesurface{hæ˧di˩ʈæ˩bɤ˩}\newline
\classe{名词}\ton{-L}\begin{définition}\peng{Beggar-monk (of the Buddhist religion).}\end{définition}
\begin{définition}\pcmn{比丘、游僧}\end{définition}
\begin{définition}\pfra{Bhiksu, moine mendiant.}\end{définition}
\end{entrée}

\begin{entrée}
{hæ˧ɭɯ\#˥}{}{ⓔhæ˧ɭɯ\#˥}\formedesurface{hæ˧ɭɯ˧}\newline
\classe{名词}\ton{\#H}\begin{définition}\peng{Chinese sorghum.}\end{définition}
\begin{définition}\pcmn{高粱}\end{définition}
\begin{définition}\pfra{Sorgho, gaoliang; céréale dont on se sert pour faire du vin.}\end{définition}
\end{entrée}

\begin{entrée}
{hæ˧se˧}{}{ⓔhæ˧se˧}\formedesurface{hæ˧se˧}\newline
\classe{名词}\ton{M}\begin{définition}\peng{Agar.}\end{définition}
\begin{définition}\pcmn{石花菜、海参}\end{définition}
\begin{définition}\pfra{Agar (ressemble à une algue).}\end{définition}
\end{entrée}

\begin{entrée}
{hæ˧ʐwɤ˩}{}{ⓔhæ˧ʐwɤ˩}\formedesurface{hæ˧ʐwɤ˩}\newline
\classe{名词}\ton{L\#}\begin{définition}\peng{The Chinese language.}\end{définition}
\begin{définition}\pcmn{汉语}\end{définition}
\begin{définition}\pfra{La langue chinoise.}\end{définition}
\end{entrée}

\begin{entrée}
{hæ̃˧}{}{ⓔhæ̃˧}\formedesurface{hæ̃˧}\newline
\classe{名词}\ton{M}
\paradigme{\pcmn{:} \p{}}
\begin{définition}\peng{Wind.}\end{définition}
\begin{définition}\pcmn{风}\end{définition}
\begin{définition}\pfra{Vent.}\end{définition}
\begin{exemple}\pnru{hæ̃˧ tʰv̩˧ / hæ̃˧ tʰv̩˧-ze˧}\hspace{5pt}\peng{the wind has risen, the wind is blowing}\hspace{5pt}\pcmn{刮风了}\hspace{5pt}\pfra{il y a du vent, le vent souffle}\end{exemple}
\begin{exemple}\pnru{wɤ˩˥ | hæ̃˧ tʰv̩˧-ho˩-ze˩!}\hspace{5pt}\peng{There's going to be some wind again!}\hspace{5pt}\pcmn{风又要刮起来了!}\hspace{5pt}\pfra{Le vent va se lever à nouveau!/ On dirait que le vent va se remettre à souffler!}\end{exemple}
\end{entrée}

\begin{entrée}
{hæ̃˧˥}{}{ⓔhæ̃˧˥}\formedesurface{hæ̃˧˥}\newline
\classe{动词}
\sens{1}
\begin{définition}\peng{To cut (with a blade: sword…), e.g. to cut cloth (to make clothes).}\end{définition}
\begin{définition}\pcmn{切,裁}\end{définition}
\begin{définition}\pfra{Trancher, couper au moyen d'un instrument tranchant: couteau, épée…; ex.: tailler un vêtement.}\end{définition}
\begin{exemple}\pnru{le˧-hæ̃˧-ze˥}\hspace{5pt}\peng{|fg{accomp} \_ |fg{pfv}}\hspace{5pt}\pcmn{切了}\hspace{5pt}\pfra{|fg{accomp} \_ |fg{pfv}}\end{exemple}
\begin{exemple}\pnru{tʰɑ˧-hæ̃˧˥!}\hspace{5pt}\peng{|fg{prohib}}\hspace{5pt}\pcmn{别切!}\hspace{5pt}\pfra{|fg{prohib}}\end{exemple}
\begin{exemple}\pnru{bɑ˩lɑ˩˥ | le˧-hæ̃˧˥, | le˧-ʐv̩˧˥}\hspace{5pt}\peng{to cut cloth to make clothes, and to sew clothes}\hspace{5pt}\pcmn{裁(布料来做)衣服,又缝(衣服) / 先裁布料,再缝衣服}\hspace{5pt}\pfra{tailler des vêtements et les coudre}\end{exemple}\sens{2}
\begin{définition}\peng{To castrate.}\end{définition}
\begin{définition}\pcmn{阉割}\end{définition}
\begin{définition}\pfra{Castrer, châtrer.}\end{définition}
\end{entrée}

\begin{entrée}
{hæ̃˧α}{}{ⓔhæ̃˧α}\formedesurface{hæ̃˧}\newline
\classe{动词}\ton{Mα}\begin{définition}\peng{To use the wind to winnow.}\end{définition}
\begin{définition}\pcmn{扬(粮食)}\end{définition}
\begin{définition}\pfra{Vanner: verser doucement dans une vannerie; la balle s'envole à mesure, emportée par le vent.}\end{définition}
\begin{exemple}\pnru{hɑ˧ hæ̃˩}\hspace{5pt}\peng{to winnow cereals}\hspace{5pt}\pcmn{扬粮食}\hspace{5pt}\pfra{vanner du grain}\end{exemple}
\begin{exemple}\pnru{tso˧∼tso˧ hæ̃˩}\hspace{5pt}\peng{to winnow things}\hspace{5pt}\pcmn{扬东西}\hspace{5pt}\pfra{vanner des choses}\end{exemple}
\end{entrée}

\begin{entrée}
{hæ̃˩}{}{ⓔhæ̃˩}\formedesurface{hæ̃˧}\newline
\classe{名词}\ton{L}
\paradigme{\pcmn{:} \p{}}
\begin{définition}\peng{Gold.}\end{définition}
\begin{définition}\pcmn{金子}\end{définition}
\begin{définition}\pfra{Or (métal).}\end{définition}
\end{entrée}

\begin{entrée}
{hæ̃˩-bɑ˧lɑ˩}{}{ⓔhæ̃˩-bɑ˧lɑ˩}\formedesurface{hæ̃˩bɑ˧lɑ˩}\newline
\classe{名词}\ton{L-L\#}
\paradigme{\pcmn{:} \p{}}
\begin{définition}\peng{Silk.}\end{définition}
\begin{définition}\pcmn{丝绸}\end{définition}
\begin{définition}\pfra{Soie.}\end{définition}
\end{entrée}

\begin{entrée}
{hæ̃˩bæ˩}{}{ⓔhæ̃˩bæ˩}\formedesurface{hæ̃˩bæ˩˥}\newline
\classe{动词}\ton{L}\begin{définition}\peng{To dance a ritual dance.}\end{définition}
\begin{définition}\pcmn{跳大神}\end{définition}
\begin{définition}\pfra{Effectuer une danse rituelle.}\end{définition}
\end{entrée}

\begin{entrée}
{hæ̃˩di˩}{}{ⓔhæ̃˩di˩}\formedesurface{hæ̃˩di˩˥}\newline
\classe{名词}\ton{L}
\paradigme{\pcmn{:} \p{}}
\begin{définition}\peng{Ruler.}\end{définition}
\begin{définition}\pcmn{尺}\end{définition}
\begin{définition}\pfra{Règle.}\end{définition}
\end{entrée}

\begin{entrée}
{hæ̃˧do˧}{}{ⓔhæ̃˧do˧}\formedesurface{hæ̃˧do˧}\newline
\classe{名词}\ton{M}
\paradigme{\pcmn{:} \p{}}
\begin{définition}\peng{Threshing ground.}\end{définition}
\begin{définition}\pcmn{打场}\end{définition}
\begin{définition}\pfra{Aire à battre le grain.}\end{définition}
\begin{exemple}\pnru{hæ̃˧do˧ bæ˩}\hspace{5pt}\peng{to sweep the threshing ground}\hspace{5pt}\pcmn{清扫打场}\hspace{5pt}\pfra{balayer l'aire à battre le grain}\end{exemple}
\end{entrée}

\begin{entrée}
{hæ̃˧kʰɤ˧˥}{}{ⓔhæ̃˧kʰɤ˧˥}\formedesurface{hæ̃˧kʰɤ˧˥}\newline
\classe{名词}\ton{MH\#}
\paradigme{\pcmn{:} \p{}}
\begin{définition}\peng{Rafter; beam.}\end{définition}
\begin{définition}\pcmn{椽子}\end{définition}
\begin{définition}\pfra{Pièce de charpente: poutrelles de toiture: poutrelles courtes, installées en inclinaison, dans le sens de la largeur du bâtiment, sur les poutres horizontales, ʐv̩˩ɭɯ˥. Les tuiles (autrefois: les planches) reposent sur ces poutrelles.}\end{définition}
\end{entrée}

\begin{entrée}
{hæ̃˧kʰo˧}{}{ⓔhæ̃˧kʰo˧}\formedesurface{hæ̃˧kʰo˧}\newline
\classe{名词}\ton{M}
\paradigme{\pcmn{:} \p{}}
\begin{définition}\peng{Princess, young lady of the nobility.}\end{définition}
\begin{définition}\pcmn{小姐、公主}\end{définition}
\begin{définition}\pfra{Demoiselle de la noblesse.}\end{définition}
\begin{exemple}\pnru{hæ̃˧kʰo˧-mi˧}\hspace{5pt}\peng{same meaning: young lady}\hspace{5pt}\pcmn{同上:小姐、公主}\hspace{5pt}\pfra{même sens: demoiselle, princesse}\end{exemple}
\end{entrée}

\begin{entrée}
{hæ̃˧pɤ˧}{}{ⓔhæ̃˧pɤ˧}\formedesurface{hæ̃˧pɤ˧}\newline
\classe{名词}\ton{M}
\paradigme{\pcmn{:} \p{}}
\begin{définition}\peng{Plait; braid.}\end{définition}
\begin{définition}\pcmn{辫子}\end{définition}
\begin{définition}\pfra{Tresse.}\end{définition}
\end{entrée}

\begin{entrée}
{hæ̃˧qʰv̩˥\$}{}{ⓔhæ̃˧qʰv̩˥\$}\formedesurface{hæ̃˧qʰv̩˥}\newline
\classe{助词}\ton{H\$}\begin{définition}\peng{Late at night, in the middle of night.}\end{définition}
\begin{définition}\pcmn{半夜}\end{définition}
\begin{définition}\pfra{En pleine nuit, tard dans la nuit.}\end{définition}
\end{entrée}

\begin{entrée}
{hæ̃˩qʰv̩˩}{}{ⓔhæ̃˩qʰv̩˩}\formedesurface{hæ̃˩qʰv̩˩˥}\newline
\classe{名词}\ton{L}\begin{définition}\peng{Ephedra sinica.}\end{définition}
\begin{définition}\pcmn{草麻黄}\end{définition}
\begin{définition}\pfra{Ephedra sinica.}\end{définition}
\end{entrée}

\begin{entrée}
{hæ̃˩qʰwɤ˩}{}{ⓔhæ̃˩qʰwɤ˩}\formedesurface{hæ̃˩qʰwɤ˩˥}\newline
\classe{名词}\ton{L}
\paradigme{\pcmn{:} \p{}}
\begin{définition}\peng{Lavender.}\end{définition}
\begin{définition}\pcmn{薰衣草(永宁的一种植物)}\end{définition}
\begin{définition}\pfra{Lavande.}\end{définition}
\end{entrée}

\begin{entrée}
{hæ̃˩sɤ˩}{}{ⓔhæ̃˩sɤ˩}\formedesurface{hæ̃˩sɤ˩˥}\newline
\classe{名词}\ton{L}
\paradigme{\pcmn{:} \p{}}
\begin{définition}\peng{Magpie.}\end{définition}
\begin{définition}\pcmn{喜鹊}\end{définition}
\begin{définition}\pfra{Pie.}\end{définition}
\end{entrée}

\begin{entrée}
{hæ̃˧ʂɯ˩‑}{}{ⓔhæ̃˧ʂɯ˩‑}\formedesurface{hæ̃˧ʂɯ˩}\newline
\classe{名词}\ton{L\#}
\paradigme{\pcmn{:} \p{}}
\begin{définition}\peng{‘precious': a prefix added to certain nouns to coin a prestige term. This prefix is not currently productive: it cannot be added to terms such as ‘mother', ‘house'…}\end{définition}
\begin{définition}\pfra{‘précieux': préfixe ajouté à certains noms pour construire une appellation prestigieuse. ; n'est pas productif: on ne peut l'ajouter à: /ə˧mɑ˧/ ‘mère', /ɑ˩ʁo˧/ ‘maison', etc.}\end{définition}
\begin{exemple}\pnru{hæ̃˧ʂɯ˩-to˩mi˩}\hspace{5pt}\peng{the Precious Pillars, the Golden Pillars: a solemn designation for the two pillars of the main building}\hspace{5pt}\pcmn{‘黄金柱’、‘宝贵柱’:对主屋两个柱子的庄严称呼}\hspace{5pt}\pfra{les Piliers d'Or, les Précieux Piliers: appellation solennelle pour les deux piliers de la maison}\end{exemple}
\end{entrée}

\begin{entrée}
{hæ̃˧ʂɯ˩-pæ˩pʰæ˩}{}{ⓔhæ̃˧ʂɯ˩-pæ˩pʰæ˩}\formedesurface{hæ̃˧ʂɯ˩pæ˧pʰæ˧}\newline
\classe{名词}\ton{L\#-}
\paradigme{\pcmn{:} \p{}}
\begin{définition}\peng{Rack for drying grain.}\end{définition}
\begin{définition}\pcmn{粮架}\end{définition}
\begin{définition}\pfra{Espalier en bois, dans la cour des fermes, pour faire sécher les épis de maïs avant égrenage.}\end{définition}
\end{entrée}

\begin{entrée}
{hæ̃˧ʂv̩˧pɤ˥}{}{ⓔhæ̃˧ʂv̩˧pɤ˥}\formedesurface{hæ̃˧ʂv̩˧pɤ˥}\newline
\classe{名词}\ton{H\#}\begin{définition}\peng{Husband.}\end{définition}
\begin{définition}\pcmn{丈夫}\end{définition}
\begin{définition}\pfra{Mari.}\end{définition}
\end{entrée}

\begin{entrée}
{hæ̃˧ʐɤ˥}{}{ⓔhæ̃˧ʐɤ˥}\formedesurface{hæ̃˧ʐɤ˥}\newline
\classe{名词}\ton{H\#}
\paradigme{\pcmn{:} \p{}}
\begin{définition}\peng{Cut marks (such as those left by a saw or axe on wood).}\end{définition}
\begin{définition}\pcmn{切割的痕迹}\end{définition}
\begin{définition}\pfra{Trace de découpe, marque de coupure.}\end{définition}
\begin{exemple}\pnru{hæ̃˧ʐɤ˥ tʰv̩˩-kʰwɤ˩}\hspace{5pt}\peng{|fg{n}+|fg{dem}+|fg{clf}: this trace of cutting}\hspace{5pt}\pcmn{这道割痕}\hspace{5pt}\pfra{|fg{n}+|fg{dem}+|fg{clf}: cette trace de découpe}\end{exemple}
\end{entrée}

\begin{entrée}
{hɤ˧}{}{ⓔhɤ˧}\formedesurface{hɤ˧}\newline
\classe{名词}\ton{M}\begin{définition}\peng{All.}\end{définition}
\begin{définition}\pcmn{全部}\end{définition}
\begin{définition}\pfra{Tout.}\end{définition}
\begin{exemple}\pnru{ɖɯ˧-hɤ˧ | mɤ˧-go˩}\hspace{5pt}\peng{to have no ailment at all, to be free from any pain}\hspace{5pt}\pcmn{一点也没病、没有任何痛苦}\hspace{5pt}\pfra{n'avoir aucune maladie, ne souffrir de rien}\end{exemple}
\begin{exemple}\pnru{ɖɯ˧-hɤ˧ | mɤ˧-sɯ˥}\hspace{5pt}\peng{to be ignorant of everything (literally: not to know a thing)}\hspace{5pt}\pcmn{什么也不知道}\hspace{5pt}\pfra{ne pas savoir quoi que ce soit, être ignorant de tout, ne rien savoir du tout}\end{exemple}
\begin{exemple}\pnru{ʈʂʰɯ˧ | ɖɯ˧-hɤ˧ hwæ˧}\hspace{5pt}\peng{(S)he buys everything / buys the lot}\hspace{5pt}\pcmn{他全部都买。/他什么都买。}\hspace{5pt}\pfra{(il/elle) achète (le) tout}\end{exemple}
\end{entrée}

\begin{entrée}
{hɤ˩α}{₁}{ⓔhɤ˩αⓗ1}\formedesurface{hɤ˩˥}\newline
\classe{动词}\ton{Lα}
1\begin{définition}\peng{To dry beside or over a fire.}\end{définition}
\begin{définition}\pcmn{烘干}\end{définition}
\begin{définition}\pfra{Chauffer au feu, sécher au feu.}\end{définition}
\begin{exemple}\pnru{tʰi˧-hɤ˩}\hspace{5pt}\peng{|fg{dur}}\hspace{5pt}\pcmn{|fg{dur}}\hspace{5pt}\pfra{|fg{dur}}\end{exemple}
\begin{exemple}\pnru{le˧-hɤ˩}\hspace{5pt}\peng{|fg{accomp}}\hspace{5pt}\pcmn{|fg{accomp}}\hspace{5pt}\pfra{|fg{accomp}}\end{exemple}
\begin{exemple}\pnru{ɖɯ˧-hɤ˩-ɻ̍˩}\hspace{5pt}\peng{|fg{delimitative} \_ |fg{inceptive}}\hspace{5pt}\pcmn{烘干一下}\hspace{5pt}\pfra{|fg{délimitatif} \_ |fg{inchoatif}: chauffer un coup, chauffer un peu}\end{exemple}
\begin{exemple}\pnru{le˧-hɤ˩-ze˩, | le˧-pv̩˧-ze˧!}\hspace{5pt}\peng{It was dried beside the fire, and it got dry / and it is now dry!}\hspace{5pt}\pcmn{烘干了,(现在)干了!}\hspace{5pt}\pfra{on l'a chauffé au feu, ça a séché}\end{exemple}
\begin{exemple}\pnru{ɖɯ˧-kʰwɤ˧ hɤ˥}\hspace{5pt}\peng{to dry something beside the fire}\hspace{5pt}\pcmn{烘干一个东西}\hspace{5pt}\pfra{chauffer quelque chose}\end{exemple}
\end{entrée}

\begin{entrée}
{hɤ˩α}{₂}{ⓔhɤ˩αⓗ2}\formedesurface{hɤ˩˥}\newline
\classe{动词}\ton{Lα}
2\begin{définition}\peng{To go, past perfective form: has gone.}\end{définition}
\begin{définition}\pcmn{去,过去式,整体体}\end{définition}
\begin{définition}\pfra{Partir, forme passée perfective.}\end{définition}
\end{entrée}

\begin{entrée}
{hɤ˩α}{₃}{ⓔhɤ˩αⓗ3}\formedesurface{hɤ˩˥}\newline
\classe{形容词}\ton{Lα}
3\begin{définition}\peng{Appropriate, good; of person: able, good at a certain technique.}\end{définition}
\begin{définition}\pcmn{好(技巧好),好(表扬一个人的行为)}\end{définition}
\begin{définition}\pfra{Bon, approprié, bien; capable (adjectif), habile à une technique.}\end{définition}
\begin{exemple}\pnru{ɖwæ˧˥ | hɤ˩˥!}\hspace{5pt}\peng{|fg{intensive.very}}\hspace{5pt}\pcmn{很好!}\hspace{5pt}\pfra{|fg{intensif.très}: c'est très bien!}\end{exemple}
\begin{exemple}\pnru{mɤ˧-hɤ˩}\hspace{5pt}\peng{|fg{neg}}\hspace{5pt}\pcmn{不好}\hspace{5pt}\pfra{|fg{neg}}\end{exemple}
\begin{exemple}\pnru{hɤ˩-hĩ˩˥}\hspace{5pt}\peng{|fg{rel}/|fg{nmlz}}\hspace{5pt}\pcmn{好的}\hspace{5pt}\pfra{|fg{rel}/|fg{nmlz}}\end{exemple}
\end{entrée}

\begin{entrée}
{hɤ˧hwi˥}{}{ⓔhɤ˧hwi˥}\newline
\classe{动词}\begin{définition}\peng{To regret.}\end{définition}
\begin{définition}\pcmn{后悔(汉语借词)}\end{définition}
\begin{définition}\pfra{Regretter.}\end{définition}
\begin{exemple}\pnru{ʈʂʰɯ˧ne˧-ʝi˥ | ʐwɤ˩ mɤ˩-ɖo˩˥! | hɤ˧hwi˥-ze˩! | tʰɑ˧-ʐwɤ˩-tso˩ ɲi˩ mæ˩!}\hspace{5pt}\peng{I should not have spoken like that! I feel regret! It's something that I should have refrained from saying!}\hspace{5pt}\pcmn{不应该这么说!(我)后悔了!是不应该说的话!}\hspace{5pt}\pfra{Je n'aurais pas dû parler ainsi! J'en éprouve du regret/ je le regrette! C'est quelque chose que je n'aurais pas dû dire!}\end{exemple}
\end{entrée}

\begin{entrée}
{hi˥}{}{ⓔhi˥}\formedesurface{hi˧}\newline
\classe{名词}\ton{\#H}
\paradigme{\pcmn{:} \p{}}
\begin{définition}\peng{Tooth.}\end{définition}
\begin{définition}\pcmn{牙齿}\end{définition}
\begin{définition}\pfra{Dent.}\end{définition}
\begin{exemple}\pnru{hi˧ go˧˥}\hspace{5pt}\peng{(a) tooth aches; to have a tooth-ache}\hspace{5pt}\pcmn{牙疼}\hspace{5pt}\pfra{avoir mal aux dents}\end{exemple}
\end{entrée}

\newpage\caractère{†}

\begin{entrée}
{†hi˧}{}{ⓔ†hi˧}\formedesurface{--}\newline
\classe{形容词}\ton{H?}\begin{définition}\peng{Fast.}\end{définition}
\begin{définition}\pcmn{快}\end{définition}
\begin{définition}\pfra{Rapide, rapidement (racine extraite de la forme disyllabique).}\end{définition}
\begin{exemple}\pnru{hi˧le˩ ʝi˩}\hspace{5pt}\peng{to do quickly}\hspace{5pt}\pcmn{快速做}\hspace{5pt}\pfra{faire rapidement}\end{exemple}
\begin{exemple}\pnru{hi˧le˩ | le˧-jo˩!}\hspace{5pt}\peng{Come quickly!}\hspace{5pt}\pcmn{快来!}\hspace{5pt}\pfra{viens vite!}\end{exemple}
\begin{exemple}\pnru{ʈʂʰɯ˧ | ɖwæ˧˥ | hi˧le˩ | ʝi˧-kv̩˩!}\hspace{5pt}\peng{(S)he knows how to work really fast!}\hspace{5pt}\pcmn{他做事很麻利!}\hspace{5pt}\pfra{Lui, il sait travailler vite!}\end{exemple}
\end{entrée}

\newpage\caractère{h}

\begin{entrée}
{hi˩}{₁}{ⓔhi˩ⓗ1}\formedesurface{hi˧}\newline
\classe{名词}\ton{L}
1
\paradigme{\pcmn{:} \p{}}
\begin{définition}\peng{Lake (monosyllabic word).}\end{définition}
\begin{définition}\pcmn{湖、海(单音节)}\end{définition}
\begin{définition}\pfra{Lac (monosyllabe).}\end{définition}
\end{entrée}

\begin{entrée}
{hi˩}{₂}{ⓔhi˩ⓗ2}\formedesurface{hi˩˥}\newline
\classe{动词}\ton{L}
2\begin{définition}\peng{Existential verb, for unmovable objects: e.g.the Lake exists/is at a certain place.}\end{définition}
\begin{définition}\pcmn{存在动词:固定不动的物体,如:泸沽湖}\end{définition}
\begin{définition}\pfra{Exister, se trouver: verbe d'existence pour objets non mobiles, par exemple le Lac existe/se trouve à un endroit.}\end{définition}
\begin{exemple}\pnru{hi˩nɑ˧mi˧ | tʰi˧-hi˩}\hspace{5pt}\peng{the Lake exists/is there}\hspace{5pt}\pcmn{有(泸沽)湖(在那儿)}\hspace{5pt}\pfra{le Lac se trouve là/existe là/se trouve là, immuable}\end{exemple}
\end{entrée}

\begin{entrée}
{hi˩˥}{}{ⓔhi˩˥}\formedesurface{hi˩˥}\newline
\classe{名词}\ton{LH}
\paradigme{\pcmn{:} \p{}}
\begin{définition}\peng{Rain.}\end{définition}
\begin{définition}\pcmn{雨}\end{définition}
\begin{définition}\pfra{Pluie.}\end{définition}
\begin{exemple}\pnru{hi˩ gi˩˥ / hi˩ gi˩-ze˥}\hspace{5pt}\peng{it's raining}\hspace{5pt}\pcmn{下雨了}\hspace{5pt}\pfra{il pleut}\end{exemple}
\end{entrée}

\begin{entrée}
{hi˧dʑi˧}{}{ⓔhi˧dʑi˧}\formedesurface{hi˧dʑi˧}\newline
\classe{名词}\ton{M}
\paradigme{\pcmn{:} \p{}}
\begin{définition}\peng{Rain cape, rainware made from straw, rush…}\end{définition}
\begin{définition}\pcmn{蓑衣}\end{définition}
\begin{définition}\pfra{Cape de pluie, vêtement qui protège de la pluie (en paille, écorce…).}\end{définition}
\begin{exemple}\pnru{hi˩ gi˩-ze˥, | hi˧dʑi˧ tʰi˧-mv̩˧.}\hspace{5pt}\peng{It has begun to rain / it's raining; put on a rain cape.}\hspace{5pt}\pcmn{下雨了,披蓑衣(雨衣)吧。}\hspace{5pt}\pfra{Il pleut, mets une cape de pluie.}\end{exemple}
\end{entrée}

\begin{entrée}
{hi˩dʑɯ˩}{}{ⓔhi˩dʑɯ˩}\formedesurface{hi˩dʑɯ˩˥}\newline
\classe{名词}\ton{L}
\paradigme{\pcmn{:} \p{}}
\begin{définition}\peng{Charcoal.}\end{définition}
\begin{définition}\pcmn{炭}\end{définition}
\begin{définition}\pfra{Charbon de bois.}\end{définition}
\end{entrée}

\begin{entrée}
{hi˧kʰɯ\#˥}{}{ⓔhi˧kʰɯ\#˥}\formedesurface{hi˧kʰɯ˧}\newline
\classe{名词}\ton{\#H}\begin{définition}\peng{Gum; gingiva.}\end{définition}
\begin{définition}\pcmn{牙龈}\end{définition}
\begin{définition}\pfra{Gencive.}\end{définition}
\begin{exemple}\pnru{hi˧kʰɯ˧ ʈʂʰæ˧}\hspace{5pt}\peng{to brush one's teeth}\hspace{5pt}\pcmn{刷牙}\hspace{5pt}\pfra{se brosser les dents; /hi˧kʰɯ˧/ peut désigner tout ce qu'on lave quand on se brosse les dents: gencives et dents.}\end{exemple}
\begin{exemple}\pnru{hi˧kʰɯ˧-ʈv̩˥}\hspace{5pt}\peng{root of the teeth}\hspace{5pt}\pcmn{牙根}\hspace{5pt}\pfra{racine des dents}\end{exemple}
\end{entrée}

\begin{entrée}
{hi˩mi˩}{}{ⓔhi˩mi˩}\formedesurface{hi˩mi˩˥}\newline
\classe{名词}\ton{L}
\paradigme{\pcmn{:} \p{}}
\begin{définition}\peng{Tongue.}\end{définition}
\begin{définition}\pcmn{舌头}\end{définition}
\begin{définition}\pfra{Langue.}\end{définition}
\begin{exemple}\pnru{hi˩mi˩˥, | ɻ̃˧ mɤ˧-ʑi˧! | ə˧tso˧ ʐwɤ˩-bi˩, | õ˧-lɑ˥ ɖʐv̩˩!}\hspace{5pt}\peng{The tongue has no bone! Only oneself knows what one is going to say! (Proverb meaning that one is responsible for one's speech: only oneself knows whether one is telling the truth or not.)}\hspace{5pt}\pcmn{“舌头没有骨头。讲的是什么(=是否真的),只有自己才知道!”(谚语)}\hspace{5pt}\pfra{«La langue n'a pas d'os! Ce qu'on dit, soi seul sait (si c'est la vérité)!»}\end{exemple}
\end{entrée}

\begin{entrée}
{hi˩nɑ˧mi\#˥}{}{ⓔhi˩nɑ˧mi\#˥}\formedesurface{hi˩nɑ˧mi˧}\newline
\classe{名词}\ton{LM+\#H}
\paradigme{\pcmn{:} \p{}}
\begin{définition}\peng{Lake.}\end{définition}
\begin{définition}\pcmn{湖}\end{définition}
\begin{définition}\pfra{Lac.}\end{définition}
\end{entrée}

\begin{entrée}
{hi˩ɲi˩zo˩}{}{ⓔhi˩ɲi˩zo˩}\formedesurface{hi˩ɲi˩zo˩˥}\newline
\classe{名词}\ton{L}
\paradigme{\pcmn{:} \p{}}
\begin{définition}\peng{Salamander.}\end{définition}
\begin{définition}\pcmn{娃娃鱼}\end{définition}
\begin{définition}\pfra{Salamandre.}\end{définition}
\end{entrée}

\begin{entrée}
{hi˩qʰɑ˩}{}{ⓔhi˩qʰɑ˩}\formedesurface{hi˩qʰɑ˩˥}\newline
\classe{名词}\ton{L}
\paradigme{\pcmn{:} \p{}}
\begin{définition}\peng{Torrential rain, cloudburst.}\end{définition}
\begin{définition}\pcmn{暴雨}\end{définition}
\begin{définition}\pfra{Orage.}\end{définition}
\begin{exemple}\pnru{hi˩qʰɑ˩ lɑ˥(-ze˩)}\hspace{5pt}\peng{torrential rain is falling}\hspace{5pt}\pcmn{下暴雨了}\hspace{5pt}\pfra{l'orage éclate, il y a de l'orage}\end{exemple}
\end{entrée}

\begin{entrée}
{hi˧qʰwɤ˩}{}{ⓔhi˧qʰwɤ˩}\formedesurface{hi˧qʰwɤ˩}\newline
\classe{名词}\ton{L\#}
\paradigme{\pcmn{:} \p{}}
\begin{définition}\peng{Decayed teeth; dental caries.}\end{définition}
\begin{définition}\pcmn{蛀牙}\end{définition}
\begin{définition}\pfra{Dent gâtée, dent cariée, carie.}\end{définition}
\end{entrée}

\begin{entrée}
{hi˩ʁwɤ˩-lo˧}{}{ⓔhi˩ʁwɤ˩-lo˧}\formedesurface{hi˩ʁwɤ˩lo˥}\newline
\classe{名词}\ton{L-M}\begin{définition}\peng{The name of a village in the plain of Yongning.}\end{définition}
\begin{définition}\pcmn{习瓦洛村:永宁坝子的一个村落}\end{définition}
\begin{définition}\pfra{Un village de la plaine de Yongning.}\end{définition}
\begin{exemple}\pnru{dʑɤ˩bv̩˧kɤ˧-sɑ˥ʁwɤ˩, | hi˩ʁwɤ˩-lo˥, | æ˩mi˧-ʁwɤ\#˥, | lɑ˧lo˧-ʁwɤ˥, | lɑ˧ŋwɤ˧, | bɤ˧tsʰo˧gv̩˥, | ə˧lɑ˧-ʁwɤ\#˥, | gæ˧ɻæ˩, | qʰæ˧tɕʰi˧, | tʰo˧ʈɯ\#˥}\hspace{5pt}\peng{The ten Na villages considered in traditional geography as belonging to the vicinity of the Yongning temple.}\hspace{5pt}\pcmn{永宁摩梭地理概念中,距离扎美寺最近的十个村落:佳部嘎萨瓦、习瓦洛、阿咪瓦、拉洛瓦、拉瓦、巴搓古、阿拉瓦、嘎尔、开基、拖支。}\hspace{5pt}\pfra{Les dix villages na traditionnellement considérés comme appartenant au voisinage du temple de Yongning.}\end{exemple}
\end{entrée}

\begin{entrée}
{hi˧tʰɑ˩}{}{ⓔhi˧tʰɑ˩}\formedesurface{hi˧tʰɑ˩}\newline
\classe{形容词}\ton{L\#}\begin{définition}\peng{Sharp, keen (blade).}\end{définition}
\begin{définition}\pcmn{锋利}\end{définition}
\begin{définition}\pfra{Aiguisé, qui coupe bien, affûté.}\end{définition}
\end{entrée}

\begin{entrée}
{hi˧tʰo˧˥}{}{ⓔhi˧tʰo˧˥}\formedesurface{hi˧tʰo˧˥}\newline
\classe{名词}\ton{MH\#}
\paradigme{\pcmn{:} \p{}}
\begin{définition}\peng{Tooth.}\end{définition}
\begin{définition}\pcmn{牙齿}\end{définition}
\begin{définition}\pfra{Dent.}\end{définition}
\end{entrée}

\begin{entrée}
{hi˧tsɯ˩}{}{ⓔhi˧tsɯ˩}\formedesurface{hi˧tsɯ˩}\newline
\classe{名词}\ton{L\#}
\paradigme{\pcmn{:} \p{}}
\begin{définition}\peng{Incisors, front teeth.}\end{définition}
\begin{définition}\pcmn{门牙}\end{définition}
\begin{définition}\pfra{Incisives (dents).}\end{définition}
\end{entrée}

\begin{entrée}
{hi˩ʐæ˥}{}{ⓔhi˩ʐæ˥}\newline
\classe{名词}
\sens{1}\paradigme{\pcmn{:} \p{}}
\begin{définition}\peng{Uvula.}\end{définition}
\begin{définition}\pcmn{小舌}\end{définition}
\begin{définition}\pfra{Luette.}\end{définition}
\begin{exemple}\pnru{qv̩˧ʈʂæ˧-bv̩˥ | hi˩ʐæ˧}\hspace{5pt}\peng{the uvula; specifying ‘the throat's…' disambiguates between the uvula and the tendon of the tongue, which are referred to by the same term, /hi˩ʐæ˥/.}\hspace{5pt}\pcmn{小舌}\hspace{5pt}\pfra{la luette; la précision ‘…de la gorge' permet de lever l'ambiguïté lorsqu'il pourrait s'agir du tendon de la langue, lui aussi désigné comme /hi˩ʐæ˥/.}\end{exemple}\sens{2}
\begin{définition}\peng{Tendon of the tongue.}\end{définition}
\begin{définition}\pcmn{舌头的筋}\end{définition}
\begin{définition}\pfra{Tendon de la langue.}\end{définition}
\begin{exemple}\pnru{hi˩mi˩-bv̩˧ | hi˩ʐæ˧}\hspace{5pt}\peng{the tendon of the tongue; specifying ‘the tongue's…' disambiguates between the uvula and the tendon of the tongue, which are referred to by the same term, /hi˩ʐæ˥/.}\hspace{5pt}\pcmn{舌头的筋}\hspace{5pt}\pfra{le tendon de la langue; la précision ‘de la langue' permet de lever l'ambiguïté dans les cas où il pourrait aussi s'agir de la luette, elle aussi désignée comme /hi˩ʐæ˥/.}\end{exemple}
\end{entrée}

\begin{entrée}
{hĩ˥}{}{ⓔhĩ˥}\formedesurface{hĩ˧}\newline
\classe{名词}\ton{\#H}
\paradigme{\pcmn{:} \p{}}
\begin{définition}\peng{Person, human being, man (without any indication of gender).}\end{définition}
\begin{définition}\pcmn{人}\end{définition}
\begin{définition}\pfra{Personne; être humain; homme (sans indication de genre).}\end{définition}
\begin{exemple}\pnru{hĩ˧ | ɖɯ˧-v̩˧}\hspace{5pt}\peng{one person, an individual}\hspace{5pt}\pcmn{一个人}\hspace{5pt}\pfra{une personne}\end{exemple}
\begin{exemple}\pnru{hĩ˧-ɻ̃˧ | ɖɯ˧-lo˩}\hspace{5pt}\peng{a lineage, a family}\hspace{5pt}\pcmn{一个家族}\hspace{5pt}\pfra{une lignée, une famille}\end{exemple}
\begin{exemple}\pnru{hĩ˧-mv˥ hĩ˩-di˩}\hspace{5pt}\peng{other people's home, other people's place (as opposed to one's home place)}\hspace{5pt}\pcmn{人家的地方,人家的故乡(不是自己的地方)}\hspace{5pt}\pfra{les terres étrangères, les terres d'autres gens (par opposition avec sa propre terre natale)}\end{exemple}
\begin{exemple}\pnru{hĩ˧-mv˥ hĩ˩-di˩ | qʰɑ˧-dʑɤ˥∼dʑɤ˩, | õ˧-mv˥ õ˩-di˩ tsʰe˩ mɤ˩-gv˩!}\hspace{5pt}\peng{No matter how beautiful other people's places are, they can never be equal to one's own homeland!}\hspace{5pt}\pcmn{其他人的地方怎么好,也比不过自己的地方!}\hspace{5pt}\pfra{Si belles soient les terres d'autrui, elles n'auront jamais la beauté de ses propres terres / de la terre natale !}\end{exemple}
\end{entrée}

\begin{entrée}
{‑hĩ˥}{}{ⓔ‑hĩ˥}\formedesurface{hĩ˧}\newline
\classe{后缀}\ton{H}\begin{définition}\peng{Relativizer and nominalizer.}\end{définition}
\begin{définition}\pcmn{关系从句/名词化}\end{définition}
\begin{définition}\pfra{Relativisateur et nominalisateur.}\end{définition}
\end{entrée}

\begin{entrée}
{hĩ˧˥}{₁}{ⓔhĩ˧˥ⓗ1}\formedesurface{hĩ˧˥}\newline
\classe{动词}\ton{MH}
1\begin{définition}\peng{To stand, to stand upright.}\end{définition}
\begin{définition}\pcmn{站(站立)}\end{définition}
\begin{définition}\pfra{Être debout, se tenir debout.}\end{définition}
\end{entrée}

\begin{entrée}
{hĩ˧˥}{₂}{ⓔhĩ˧˥ⓗ2}\formedesurface{hĩ˧˥}\newline
\classe{动词}\ton{MH}
2\begin{définition}\peng{To have to, to be necessary.}\end{définition}
\begin{définition}\pcmn{应该}\end{définition}
\begin{définition}\pfra{Devoir, falloir.}\end{définition}
\begin{exemple}\pnru{mɤ˧-hĩ˧}\hspace{5pt}\peng{|fg{neg}}\hspace{5pt}\pcmn{否定}\hspace{5pt}\pfra{|fg{neg}}\end{exemple}
\begin{exemple}\pnru{no˧ | ʝi˧-hĩ˧˥!}\hspace{5pt}\peng{You have to do it!}\hspace{5pt}\pcmn{你应该做!}\hspace{5pt}\pfra{c'est à toi de le faire! / il faut que tu le fasses!}\end{exemple}
\begin{exemple}\pnru{njɤ˧ | ʝi˧-mɤ˧-hĩ˧-hĩ˥ | (ɖɯ˧-pi˧) ʝi˧-ze˩! |}\hspace{5pt}\peng{I have done something I shouldn't have!}\hspace{5pt}\pcmn{我做了一件不应该做的事!}\hspace{5pt}\pfra{j'ai fait quelque chose que j'aurais pas dû!}\end{exemple}
\begin{exemple}\pnru{no˧ | lo˧ ʝi˧-hĩ˧!}\hspace{5pt}\peng{You have to work! / You must work!}\hspace{5pt}\pcmn{你应该工作啊!}\hspace{5pt}\pfra{Il faut que tu travailles!}\end{exemple}
\end{entrée}

\begin{entrée}
{hĩ˧bæ\#˥}{}{ⓔhĩ˧bæ\#˥}\formedesurface{hĩ˧bæ˧}\newline
\classe{名词}\ton{\#H}
\paradigme{\pcmn{:} \p{}}
\begin{définition}\peng{Guest, visitor.}\end{définition}
\begin{définition}\pcmn{客人}\end{définition}
\begin{définition}\pfra{Invité, visiteur, hôte.}\end{définition}
\begin{exemple}\pnru{hĩ˧bæ˧ ʝi˧}\hspace{5pt}\peng{to be a guest, to be invited, to attend a party}\hspace{5pt}\pcmn{做客}\hspace{5pt}\pfra{participer à une fête en tant qu'invité, se rendre à une fête/à une invitation}\end{exemple}
\begin{exemple}\pnru{hĩ˧bæ˧ tsʰɯ˧-ze˥ ! |}\hspace{5pt}\peng{A guest has arrived!}\hspace{5pt}\pcmn{客人来了!}\hspace{5pt}\pfra{Un invité est arrivé!}\end{exemple}
\end{entrée}

\begin{entrée}
{hĩ˧hĩ\#˥}{}{ⓔhĩ˧hĩ\#˥}\formedesurface{hĩ˧hĩ˧}\newline
\classe{名词}\ton{\#H}
\paradigme{\pcmn{:} \p{}}
\begin{définition}\peng{Strangers, people outside the family.}\end{définition}
\begin{définition}\pcmn{外人}\end{définition}
\begin{définition}\pfra{Les gens extérieurs à la famille (s'oppose à: ‘les gens de la famille’).}\end{définition}
\end{entrée}

\begin{entrée}
{hĩ˧-lɑ˩-kv̩˩-hĩ˩}{}{ⓔhĩ˧-lɑ˩-kv̩˩-hĩ˩}\formedesurface{hĩ˧lɑ˩kv̩˩hĩ˩}\newline
\classe{名词}\ton{-L}\begin{définition}\peng{Dangerous person; enemy.}\end{définition}
\begin{définition}\pcmn{危险的人,仇人,敌人}\end{définition}
\begin{définition}\pfra{Personne dangereuse, ennemi, bandit; littéralement: «personne susceptible de frapper les gens».}\end{définition}
\end{entrée}

\begin{entrée}
{hĩ˧mɤ˧gɤ˥}{}{ⓔhĩ˧mɤ˧gɤ˥}\formedesurface{hĩ˧mɤ˧gɤ˥}\newline
\classe{动词}\ton{H\#}\begin{définition}\peng{To envy, to be jealous of.}\end{définition}
\begin{définition}\pcmn{妒忌}\end{définition}
\begin{définition}\pfra{Envier, être jaloux de.}\end{définition}
\begin{exemple}\pnru{hĩ˧mɤ˧gɤ˥ ʝi˩}\hspace{5pt}\peng{to be jealous (of someone), to have a fit of jealousy}\hspace{5pt}\pcmn{妒忌}\hspace{5pt}\pfra{envier (quelqu'un), faire une crise de jalousie}\end{exemple}
\end{entrée}

\begin{entrée}
{hĩ˧mo˥}{}{ⓔhĩ˧mo˥}\formedesurface{hĩ˧mo˥}\newline
\classe{名词}\ton{H\#}
\paradigme{\pcmn{:} \p{}}
\begin{définition}\peng{Elderly person.}\end{définition}
\begin{définition}\pcmn{老人}\end{définition}
\begin{définition}\pfra{Personne âgée, vieillard, vieillarde.}\end{définition}
\begin{exemple}\pnru{hĩ˧mo˥-hĩ˩}\hspace{5pt}\peng{\_ |fg{rel}; same meaning}\hspace{5pt}\pcmn{老人、老的人}\hspace{5pt}\pfra{\_ |fg{rel}; même sens}\end{exemple}
\end{entrée}

\begin{entrée}
{hĩ˧mo˩}{}{ⓔhĩ˧mo˩}\newline
\classe{名词}
\sens{1}\paradigme{\pcmn{:} \p{}}
\begin{définition}\peng{Corpse.}\end{définition}
\begin{définition}\pcmn{尸体}\end{définition}
\begin{définition}\pfra{Cadavre.}\end{définition}
\begin{exemple}\pnru{hĩ˧mo˩-kʰɯ˩-di˩}\hspace{5pt}\peng{coffin; literally ‘thing (in which) to put a corpse'}\hspace{5pt}\pcmn{棺材}\hspace{5pt}\pfra{cercueil (périphrase: «objet (dans lequel) on met le cadavre»)}\end{exemple}\sens{2}
\begin{définition}\peng{Tomb.}\end{définition}
\begin{définition}\pcmn{坟墓}\end{définition}
\begin{définition}\pfra{Tombe, tombeau.}\end{définition}
\end{entrée}

\begin{entrée}
{hĩ˧nv̩˥}{}{ⓔhĩ˧nv̩˥}\formedesurface{hĩ˧nv̩˥}\newline
\classe{动词}\ton{H\#}\begin{définition}\peng{To be disobedient, naughty.}\end{définition}
\begin{définition}\pcmn{不听话、调皮}\end{définition}
\begin{définition}\pfra{Être désobéissant, être espiègle, désobéir.}\end{définition}
\end{entrée}

\begin{entrée}
{hĩ˧-tɕʰɯ\#˥}{}{ⓔhĩ˧-tɕʰɯ\#˥}\formedesurface{hĩ˧tɕʰɯ˧}\newline
\classe{名词}\ton{\#H}\begin{définition}\peng{Family member belonging to the same generation: brother, sister, or cousin (on the mother's side).}\end{définition}
\begin{définition}\pcmn{同一辈的亲戚:兄弟姐妹、堂兄弟姐妹}\end{définition}
\begin{définition}\pfra{Membre de la famille de même génération: frère, sœur, cousin(e) (du côté maternel).}\end{définition}
\begin{exemple}\pnru{hĩ˧-tɕʰɯ˧ - hĩ˧-ʈʂɤ\#˥}\hspace{5pt}\peng{same meaning: the family members belonging to the same generation}\hspace{5pt}\pcmn{同一辈的亲戚:兄弟姐妹、堂兄弟姐妹}\hspace{5pt}\pfra{même sens: les gens de la même génération, dans la famille: frères, sœurs, mais aussi cousins du côté maternel}\end{exemple}
\begin{exemple}\pnru{ʈʂʰɯ˧ | njɤ˧ | hĩ˧ tɕʰɯ˧ ɲi˥!}\hspace{5pt}\peng{(S)he is someone of my generation! (=my cousin, my brother/sister…)}\hspace{5pt}\pcmn{他是跟我同一辈的亲戚!(=堂兄弟姐妹)}\hspace{5pt}\pfra{C'est mon cousin/ma cousine/quelqu'un de ma fratrie!}\end{exemple}
\begin{exemple}\pnru{hĩ˧-tɕʰɯ˧ mɤ˧-ɲi˥ F | hĩ˧-tɕʰɯ˧ ʝi˧ tʰɑ˩-kv̩˩!}\hspace{5pt}\peng{“Even if one is not (born) a family member, it is possible to become one!" A saying that refers to quasi-family links between friends, which amount to a form of adoption into the family circle.}\hspace{5pt}\pcmn{“不是亲戚,也可以变成亲戚!”这个俗语来形容朋友之间的深情,变成像家人之间的感情。}\hspace{5pt}\pfra{«Même si on n'est pas de la même famille (au départ), on peut le devenir!» Formule traditionnelle pour désigner les liens quasi-familiaux tissés entre amis, qui reviennent à des formes d'adoption au sein du cercle familial.}\end{exemple}
\end{entrée}

\begin{entrée}
{hĩ˧-ʈʂɤ\#˥}{}{ⓔhĩ˧-ʈʂɤ\#˥}\formedesurface{hĩ˧ʈʂɤ˧}\newline
\classe{名词}\ton{\#H}\begin{définition}\peng{Family member belonging to the same generation: brother, sister, or cousin (on the mother's side).}\end{définition}
\begin{définition}\pcmn{同一辈的亲戚:兄弟姐妹、堂兄弟姐妹}\end{définition}
\begin{définition}\pfra{Membre de la famille de même génération: frère, sœur, cousin(e) (du côté maternel).}\end{définition}
\begin{exemple}\pnru{hĩ˧-tɕʰɯ˧ - hĩ˧-ʈʂɤ\#˥}\hspace{5pt}\peng{same meaning: the family members belonging to the same generation}\hspace{5pt}\pcmn{同一辈的亲戚:兄弟姐妹、堂兄弟姐妹}\hspace{5pt}\pfra{même sens: les gens de la même génération, dans la famille: frères, sœurs, mais aussi cousins du côté maternel}\end{exemple}
\end{entrée}

\begin{entrée}
{ho˥}{₁}{ⓔho˥ⓗ1}\formedesurface{ho˧}\newline
\classe{名词}\ton{\#H}
1
\paradigme{\pcmn{:} \p{}}
\begin{définition}\peng{Partridge.RD QN: Partridge or pheasant?}\end{définition}
\begin{définition}\pcmn{雉}\end{définition}
\begin{définition}\pfra{Faisan (utilisé aussi pour: cailles, et certaines poules sauvages).}\end{définition}
\begin{exemple}\pnru{ho˧ tʰv̩˧-mi˧˥ / ho˧ tʰv̩˧-mi˥\#}\hspace{5pt}\peng{|fg{n}+|fg{dem}+|fg{clf}}\hspace{5pt}\pcmn{这只雉}\hspace{5pt}\pfra{|fg{n}+|fg{dem}+|fg{clf}}\end{exemple}
\end{entrée}

\begin{entrée}
{ho˥}{₂}{ⓔho˥ⓗ2}\formedesurface{ho˧}\newline
\classe{名词}\ton{\#H}
2\begin{définition}\peng{Porridge, gruel, congee.}\end{définition}
\begin{définition}\pcmn{粥}\end{définition}
\begin{définition}\pfra{Gruau.}\end{définition}
\begin{exemple}\pnru{ho˧ ʈʰɯ˧˥}\hspace{5pt}\peng{to drink gruel}\hspace{5pt}\pcmn{喝粥}\hspace{5pt}\pfra{boire du gruau}\end{exemple}
\end{entrée}

\begin{entrée}
{ho˧˥}{}{ⓔho˧˥}\formedesurface{ho˧˥}\newline
\classe{动词}\ton{MH}\begin{définition}\peng{To sip: to drink by small mouthfuls.}\end{définition}
\begin{définition}\pcmn{小口地喝}\end{définition}
\begin{définition}\pfra{Siroter, boire à petites gorgées.}\end{définition}
\begin{exemple}\pnru{ʐɯ˧ ho˧˥}\hspace{5pt}\peng{to sip wine}\hspace{5pt}\pcmn{小口地喝酒}\hspace{5pt}\pfra{siroter du vin}\end{exemple}
\begin{exemple}\pnru{ʐɯ˧ ho˧∼ho˥}\hspace{5pt}\peng{to sip wine}\hspace{5pt}\pcmn{小口地喝酒}\hspace{5pt}\pfra{siroter du vin}\end{exemple}
\begin{exemple}\pnru{ʐɯ˧ | ɖɯ˧-ho˧∼ho˥}\hspace{5pt}\peng{to sip wine}\hspace{5pt}\pcmn{喝一小口酒}\hspace{5pt}\pfra{siroter du vin}\end{exemple}
\end{entrée}

\begin{entrée}
{‑ho˩}{}{ⓔ‑ho˩}\formedesurface{ho˩˥}\newline
\classe{后缀}\ton{L}\begin{définition}\peng{Future / desiderative / conjecture.}\end{définition}
\begin{définition}\pcmn{未来\_愿望}\end{définition}
\begin{définition}\pfra{Future / desiderative / conjecture.}\end{définition}
\begin{exemple}\pnru{hi˩gi˩ ə˥-ho˩? - hi˩ gi˩ ho˥!}\hspace{5pt}\peng{Is it going to rain? - Yes, it is going to rain!}\hspace{5pt}\pcmn{要下雨了吗? - 是的,要下雨了!}\hspace{5pt}\pfra{Va-t-il pleuvoir ? - Oui!}\end{exemple}
\begin{exemple}\pnru{ʈʂʰɯ˧ | so˧ɲi˥ | le˧-jo˩ ho˩-hĩ˩ | ə˩-ɲi˩˥ ? - ʈʂʰɯ˧ | so˧ɲi˥ | le˧-jo˩-ho˩-ɲi˩-mæ˩.}\hspace{5pt}\peng{Is he going to come tomorrow? - (Yes,) he will come tomorrow.}\hspace{5pt}\pcmn{他明天要来买? - (是的,)他明天会来的。(回答表示:比较肯定。)}\hspace{5pt}\pfra{Viendra-t-il demain ? - (Oui) je pense qu’il viendra demain. (Qd on est presque sûr)}\end{exemple}
\begin{exemple}\pnru{so˧ɲi˥ | le˧-ɬi˥ | mɤ˧-ho˥!}\hspace{5pt}\peng{Tomorrow, they won't be on holiday anymore! (Context: on a Sunday, talking about a kindergarten that has been on holiday during the previous week.)}\hspace{5pt}\pcmn{明天就不休息了!}\hspace{5pt}\pfra{Demain, ils ne seront plus en vacances! (contexte: on parle d'une crèche qui a été en vacances la semaine précédente à l'occasion de la Fête des enseignants)}\end{exemple}
\begin{exemple}\pnru{tɕʰi˧ ə˧-ho˩?}\hspace{5pt}\peng{Is (she/he) going to sell?}\hspace{5pt}\pcmn{要卖吗?/ 会卖吗?}\hspace{5pt}\pfra{va(-t-il) vendre?}\end{exemple}
\begin{exemple}\pnru{hwæ˧ ə˧-ho˥?}\hspace{5pt}\peng{Is (she/he) going to buy?}\hspace{5pt}\pcmn{要买吗?/ 会买马?}\hspace{5pt}\pfra{va(-t-il) acheter?}\end{exemple}
\end{entrée}

\begin{entrée}
{ho˩α}{}{ⓔho˩α}\formedesurface{ho˩˥}\newline
\classe{形容词}\ton{Lα}\begin{définition}\peng{Correct; suitable, appropriate.}\end{définition}
\begin{définition}\pcmn{准确,合适}\end{définition}
\begin{définition}\pfra{Exact, correct; adapté, convenable.}\end{définition}
\begin{exemple}\pnru{mɤ˧-ho˩}\hspace{5pt}\peng{|fg{neg}}\hspace{5pt}\pcmn{不合适,不准,不对}\hspace{5pt}\pfra{|fg{neg}: faux, erroné, inapproprié}\end{exemple}
\begin{exemple}\pnru{ho˩-ze˧!}\hspace{5pt}\peng{|fg{pfv}}\hspace{5pt}\pcmn{对了! / 准确!}\hspace{5pt}\pfra{|fg{pfv}}\end{exemple}
\begin{exemple}\pnru{ho˩-hĩ˩˥}\hspace{5pt}\peng{|fg{rel}/|fg{nmlz}}\hspace{5pt}\pcmn{准确的}\hspace{5pt}\pfra{|fg{rel}/|fg{nmlz}}\end{exemple}
\end{entrée}

\begin{entrée}
{ho˧ɕjæ˩}{}{ⓔho˧ɕjæ˩}\formedesurface{ho˧ɕjæ˩}\newline
\classe{名词}\ton{L\#}\begin{définition}\peng{Cord to which fire is put in order to shoot.}\end{définition}
\begin{définition}\pcmn{火绳,导火索}\end{définition}
\begin{définition}\pfra{Mèche.}\end{définition}
\end{entrée}

\begin{entrée}
{ho˩ɕjæ˧}{}{ⓔho˩ɕjæ˧}\formedesurface{ho˩ɕjæ˥}\newline
\classe{名词}\ton{LM}\begin{définition}\peng{Wrinkled giant hyssop, |\stylefi{Elshotzia sp.}.}\end{définition}
\begin{définition}\pcmn{藿香}\end{définition}
\begin{définition}\pfra{Hysope, |\stylefi{Elshotzia sp.}.}\end{définition}
\end{entrée}

\begin{entrée}
{ho˧di˧}{}{ⓔho˧di˧}\formedesurface{ho˧di˧}\newline
\classe{名词}\ton{M}\begin{définition}\peng{Chinese (Han) areas of Sichuan: Yanyuan, Yanbian, Xichang…}\end{définition}
\begin{définition}\pcmn{四川(盐源、盐边、西昌……)}\end{définition}
\begin{définition}\pfra{Régions chinoises (Han) du Sichuan: Yanyuan, Yanbian, Xichang…}\end{définition}
\end{entrée}

\begin{entrée}
{ho˧dʑɯ˧˥}{}{ⓔho˧dʑɯ˧˥}\formedesurface{ho˧dʑɯ˧˥}\newline
\classe{名词}\ton{MH\#}\begin{définition}\peng{Paste; starch.}\end{définition}
\begin{définition}\pcmn{浆糊,浆子}\end{définition}
\begin{définition}\pfra{Pâte, colle à base de farine, liquide visqueux.}\end{définition}
\end{entrée}

\begin{entrée}
{ho˩dʑɯ˩}{}{ⓔho˩dʑɯ˩}\formedesurface{ho˩dʑɯ˩˥}\newline
\classe{形容词}\ton{L}\begin{définition}\peng{Destitute, impoverished, down and out.}\end{définition}
\begin{définition}\pcmn{穷苦、凋敝、寒苦、竭蹶、穷乏}\end{définition}
\begin{définition}\pfra{Indigent.}\end{définition}
\begin{exemple}\pnru{ho˩dʑɯ˩-ze˥}\hspace{5pt}\peng{|fg{pfv}}\hspace{5pt}\pcmn{变穷苦了}\hspace{5pt}\pfra{|fg{pfv}: qui se retrouve à la rue, qui devient démuni}\end{exemple}
\end{entrée}

\begin{entrée}
{ho˧dʑɯ˧-tɤ˥ɻ̍˩}{}{ⓔho˧dʑɯ˧-tɤ˥ɻ̍˩}\formedesurface{ho˧dʑɯ˧tɤ˥ɻ̍˩}\newline
\classe{名词}\ton{\#H-}\begin{définition}\peng{Paste; starch.}\end{définition}
\begin{définition}\pcmn{浆糊,浆子}\end{définition}
\begin{définition}\pfra{Pâte, colle à base de farine, liquide visqueux.}\end{définition}
\end{entrée}

\begin{entrée}
{ho˧ko˧}{}{ⓔho˧ko˧}\formedesurface{ho˧ko˧}\newline
\classe{名词}\ton{M}\begin{définition}\peng{Cooking pot for making hotpot; traditionally made of copper, with a hole in the centre.}\end{définition}
\begin{définition}\pcmn{火锅(汉语借词)}\end{définition}
\begin{définition}\pfra{Grand récipient pour faire la fondue mongole.}\end{définition}
\begin{exemple}\pnru{æ̃˧-ho˧ko˥}\hspace{5pt}\peng{copper pot for hotpot}\hspace{5pt}\pcmn{铜火锅}\hspace{5pt}\pfra{récipient pour fondue en cuivre}\end{exemple}
\end{entrée}

\begin{entrée}
{ho˩lo˧pv̩˥}{}{ⓔho˩lo˧pv̩˥}\formedesurface{ho˩lo˧pv̩˥}\newline
\classe{名词}\ton{LM+H\#}
\paradigme{\pcmn{:} \p{}}
\begin{définition}\peng{Carrot.}\end{définition}
\begin{définition}\pcmn{胡萝卜}\end{définition}
\begin{définition}\pfra{Carotte.}\end{définition}
\end{entrée}

\begin{entrée}
{ho˧mi\#˥}{}{ⓔho˧mi\#˥}\formedesurface{ho˧mi˧}\newline
\classe{名词}\ton{\#H}
\paradigme{\pcmn{:} \p{}}
\begin{définition}\peng{Hen pheasant, female pheasant.}\end{définition}
\begin{définition}\pcmn{母雉}\end{définition}
\begin{définition}\pfra{Faisan femelle.}\end{définition}
\begin{exemple}\pnru{ho˧mi˧ tʰv̩˧-mi˧˥ / ho˧mi˧ tʰv̩˧-mi˥\#}\hspace{5pt}\peng{|fg{n}+|fg{dem}+|fg{clf}}\hspace{5pt}\pcmn{这个母雉}\hspace{5pt}\pfra{|fg{n}+|fg{dem}+|fg{clf}}\end{exemple}
\begin{exemple}\pnru{ho˧mi˧-ho˧pʰv̩˥ / ho˧mi˧-ho˥pʰv̩˩}\hspace{5pt}\peng{female and male pheasant}\hspace{5pt}\pcmn{母雉与公雉}\hspace{5pt}\pfra{faisan femelle et faisan mâle}\end{exemple}
\end{entrée}

\begin{entrée}
{ho˧pʰv̩\#˥}{}{ⓔho˧pʰv̩\#˥}\formedesurface{ho˧pʰv̩˧}\newline
\classe{名词}\ton{\#H}
\paradigme{\pcmn{:} \p{}}
\begin{définition}\peng{Cock pheasant, male pheasant.}\end{définition}
\begin{définition}\pcmn{公雉}\end{définition}
\begin{définition}\pfra{Faisan mâle.}\end{définition}
\begin{exemple}\pnru{ho˧pʰv̩˧ tʰv̩˧-mi˧˥ / ho˧pʰv̩˧ tʰv̩˧-mi˥\#}\hspace{5pt}\peng{|fg{n}+|fg{dem}+|fg{clf}}\hspace{5pt}\pcmn{这只公雉}\hspace{5pt}\pfra{|fg{n}+|fg{dem}+|fg{clf}}\end{exemple}
\end{entrée}

\begin{entrée}
{ho˩to˩ʂæ˧}{}{ⓔho˩to˩ʂæ˧}\formedesurface{ho˩to˩ʂæ˥}\newline
\classe{名词}\ton{L.L.M}\begin{définition}\peng{Yew.}\end{définition}
\begin{définition}\pcmn{红豆杉}\end{définition}
\begin{définition}\pfra{If.}\end{définition}
\begin{exemple}\pnru{ho˩to˩ʂæ˧-bv˧ | bæ˩bæ˩ ɲi˥!}\hspace{5pt}\peng{It's a yew flower!}\hspace{5pt}\pcmn{是红豆杉花!}\hspace{5pt}\pfra{C'est une fleur d'if!}\end{exemple}
\end{entrée}

\begin{entrée}
{ho˧tʰv̩˧}{}{ⓔho˧tʰv̩˧}\formedesurface{ho˧tʰv̩˧}\newline
\classe{名词}\ton{M}
\paradigme{\pcmn{:} \p{}}
\begin{définition}\peng{Ham.}\end{définition}
\begin{définition}\pcmn{火腿(汉语借词)}\end{définition}
\begin{définition}\pfra{Jambon.}\end{définition}
\begin{exemple}\pnru{ho˧tʰv̩˧ gv̩˩, | hæ˧ ɳɯ˧ | so˩˥! |}\hspace{5pt}\peng{It's from the Chinese (Han) that we learnt to make ham! (The earlier recipe, for |fv{ʂe˧sɑ˩}, was slightly different.)}\hspace{5pt}\pcmn{(怎么)做火腿,(我们)是向汉人学的!}\end{exemple}
\end{entrée}

\begin{entrée}
{ho˧ʈʂɯ˧}{}{ⓔho˧ʈʂɯ˧}\formedesurface{ho˧ʈʂɯ˧}\newline
\classe{名词}\ton{M}\begin{définition}\peng{Mugwort, wormwood, |\stylefi{Artemisia vulgaris}.}\end{définition}
\begin{définition}\pcmn{蒿(汉语借词:蒿枝)}\end{définition}
\begin{définition}\pfra{Armoise, |\stylefi{Artemisia vulgaris}.}\end{définition}
\end{entrée}

\begin{entrée}
{ho˧zo\#˥}{}{ⓔho˧zo\#˥}\formedesurface{ho˧zo˧}\newline
\classe{名词}\ton{\#H}
\paradigme{\pcmn{:} \p{}}
\begin{définition}\peng{Baby pheasant, pheasant chick.}\end{définition}
\begin{définition}\pcmn{小雉}\end{définition}
\begin{définition}\pfra{Bébé faisan.}\end{définition}
\begin{exemple}\pnru{ho˧zo˧ tʰv̩˧-ɭɯ\#˥}\hspace{5pt}\peng{|fg{n}+|fg{dem}+|fg{clf}}\hspace{5pt}\pcmn{这只小雉}\hspace{5pt}\pfra{|fg{n}+|fg{dem}+|fg{clf}}\end{exemple}
\end{entrée}

\begin{entrée}
{hõ˧˥}{}{ⓔhõ˧˥}\formedesurface{hõ˧˥}\newline
\classe{数词}\ton{MH}\begin{définition}\peng{Eight.}\end{définition}
\begin{définition}\pcmn{八}\end{définition}
\begin{définition}\pfra{Huit.}\end{définition}
\end{entrée}

\begin{entrée}
{hõ˧α}{}{ⓔhõ˧α}\newline
\classe{动词}
\sens{1}
\begin{définition}\peng{To go away (imperative).}\end{définition}
\begin{définition}\pcmn{走(离开),命令式}\end{définition}
\begin{définition}\pfra{Partir (impératif).}\end{définition}
\begin{exemple}\pnru{no˧ hõ˧!}\hspace{5pt}\peng{Go!}\hspace{5pt}\pcmn{你走吧!}\hspace{5pt}\pfra{vas-y!}\end{exemple}
\begin{exemple}\pnru{no˧ | le˧-hõ˧!}\hspace{5pt}\peng{Go!}\hspace{5pt}\pcmn{你走吧!}\hspace{5pt}\pfra{Marche!/Vas-y!}\end{exemple}
\begin{exemple}\pnru{ə˧-ze˧∼ze˥ hõ˩! / ə˧-dzɤ˥ | le˧-hõ˧!}\hspace{5pt}\peng{Walk slowly! / Take your time on the road! / Have a quiet and pleasant journey! (Polite salutation to someone who is leaving.)}\hspace{5pt}\pcmn{慢走!}\hspace{5pt}\pfra{salutation respectueuse à quelqu'un qui se met en chemin: Prends ton temps en chemin!}\end{exemple}
\begin{exemple}\pnru{ɑ˩pʰo˩ hõ˩˥!}\hspace{5pt}\peng{Get out!}\hspace{5pt}\pcmn{出去!走开!滚出去!}\hspace{5pt}\pfra{Dehors! / Dégage!}\end{exemple}\sens{2}
\begin{définition}\peng{Imperative.}\end{définition}
\begin{définition}\pcmn{命令式}\end{définition}
\begin{définition}\pfra{Impératif.}\end{définition}
\begin{exemple}\pnru{no˧ | dzɯ˧-hõ˧!}\hspace{5pt}\peng{Eat!}\hspace{5pt}\pcmn{你吃吧!}\hspace{5pt}\pfra{Mange!}\end{exemple}
\end{entrée}

\begin{entrée}
{hõ˧-ɬi˧mi\#˥}{}{ⓔhõ˧-ɬi˧mi\#˥}\formedesurface{hõ˧ɬi˧mi˧}\newline
\classe{名词}\ton{\#H}\begin{définition}\peng{8th month.}\end{définition}
\begin{définition}\pcmn{八月}\end{définition}
\begin{définition}\pfra{8e mois.}\end{définition}
\end{entrée}

\begin{entrée}
{hõ˩tsʰi˧˥}{}{ⓔhõ˩tsʰi˧˥}\formedesurface{hõ˩tsʰi˧˥}\newline
\classe{数词}\ton{LM+MH\#}\begin{définition}\peng{80.}\end{définition}
\begin{définition}\pcmn{80}\end{définition}
\begin{définition}\pfra{80.}\end{définition}
\end{entrée}

\begin{entrée}
{hu˥}{}{ⓔhu˥}\formedesurface{hu˧}\newline
\classe{动词}\ton{H}\begin{définition}\peng{To wait.}\end{définition}
\begin{définition}\pcmn{等候}\end{définition}
\begin{définition}\pfra{Attendre.}\end{définition}
\begin{exemple}\pnru{le˧-hu˥-ze˩}\hspace{5pt}\peng{|fg{accomp} \_ |fg{pfv}}\hspace{5pt}\pcmn{等了}\hspace{5pt}\pfra{|fg{accomp} \_ |fg{pfv}}\end{exemple}
\begin{exemple}\pnru{ɖɯ˧-hu˥ / ɖɯ˧-hu˧-ɻ̍˥}\hspace{5pt}\peng{to wait a while / Wait a while!}\hspace{5pt}\pcmn{等一下 / 请等一下!}\hspace{5pt}\pfra{attendre un peu / Attends un peu!}\end{exemple}
\begin{exemple}\pnru{hĩ˧ hu˧}\hspace{5pt}\peng{to wait for someone}\hspace{5pt}\pcmn{等人}\hspace{5pt}\pfra{attendre quelqu'un}\end{exemple}
\end{entrée}

\begin{entrée}
{hu˧˥}{₁}{ⓔhu˧˥ⓗ1}\formedesurface{hu˧˥}\newline
\classe{动词}\ton{MH}
1\begin{définition}\peng{To miss, to long for, to be nostalgic about}\end{définition}
\begin{définition}\pcmn{想念}\end{définition}
\begin{définition}\pfra{Avoir la nostalgie de.}\end{définition}
\begin{exemple}\pnru{ə˧mi˧ hu˧˥}\hspace{5pt}\peng{to miss (one's) mother}\hspace{5pt}\pcmn{想念母亲}\hspace{5pt}\pfra{avoir la nostalgie de sa mère}\end{exemple}
\end{entrée}

\begin{entrée}
{hu˧˥}{₂}{ⓔhu˧˥ⓗ2}\formedesurface{hu˧˥}\newline
\classe{名词}\ton{MH}
2
\paradigme{\pcmn{:} \p{}}
\begin{définition}\peng{Entrails.}\end{définition}
\begin{définition}\pcmn{内脏:胃、肠子等}\end{définition}
\begin{définition}\pfra{Estomac au sens large: entrailles, tripes (tout le système digestif).}\end{définition}
\end{entrée}

\begin{entrée}
{hu˧mi˥\$}{}{ⓔhu˧mi˥\$}\formedesurface{hu˧mi˥}\newline
\classe{名词}\ton{H\$}
\paradigme{\pcmn{:} \p{}}
\begin{définition}\peng{Stomach.}\end{définition}
\begin{définition}\pcmn{胃}\end{définition}
\begin{définition}\pfra{Estomac.}\end{définition}
\end{entrée}

\begin{entrée}
{hɯ˧}{}{ⓔhɯ˧}\formedesurface{hɯ˧}\newline
\classe{动词}\ton{Mγ}\begin{définition}\peng{To go, past form.}\end{définition}
\begin{définition}\pcmn{走(过去式)}\end{définition}
\begin{définition}\pfra{Aller, forme passée.}\end{définition}
\begin{exemple}\pnru{(ki˧zo˧) | lo˧ ʝi˧-hɯ˧(-ze˩)!}\hspace{5pt}\peng{Kizo has gone to work!}\hspace{5pt}\pcmn{给若(人名)干活去了!}\hspace{5pt}\pfra{Kizo est partie travailler!}\end{exemple}
\begin{exemple}\pnru{le˧-hɯ˩-hĩ˩ hĩ˩}\hspace{5pt}\peng{euphemism for ‘deceased person': literally ‘person who has gone'}\hspace{5pt}\pcmn{委婉语:‘走了的人’=去世了的人}\hspace{5pt}\pfra{personne décédée; littéralement «personne qui est partie»}\end{exemple}
\end{entrée}

\begin{entrée}
{hɯ˧β}{}{ⓔhɯ˧β}\formedesurface{ɖɯ˧ hɯ˧}\newline
\classe{量词}\ton{Mβ}\begin{définition}\peng{A few, some, a little.}\end{définition}
\begin{définition}\pcmn{量词:一点}\end{définition}
\begin{définition}\pfra{Quelques-uns, un peu, une petite quantité de.}\end{définition}
\begin{exemple}\pnru{ʈʂʰæ˧ɣɯ˧ ɖɯ˧-hɯ˧}\hspace{5pt}\peng{some medicines, a few medicines}\hspace{5pt}\pcmn{一些药}\hspace{5pt}\pfra{quelques médicaments}\end{exemple}
\end{entrée}

\begin{entrée}
{hṽ̩˥}{}{ⓔhṽ̩˥}\formedesurface{hṽ̩˧}\newline
\classe{名词}\ton{\#H}
\paradigme{\pcmn{:} \p{}}
\begin{définition}\peng{Body hair; animal's hair; porcupine's quills.}\end{définition}
\begin{définition}\pcmn{毛、豪猪的翎}\end{définition}
\begin{définition}\pfra{Poils (pour les animaux, y compris les épines du hérisson; aussi pour les hommes).}\end{définition}
\end{entrée}

\begin{entrée}
{hṽ̩˥}{}{ⓔhṽ̩˥}\formedesurface{hṽ̩˧}\newline
\classe{动词}\ton{H}\begin{définition}\peng{To stir-fry.}\end{définition}
\begin{définition}\pcmn{炒(肉、菜)}\end{définition}
\begin{définition}\pfra{Frire (viande, légumes…), cuire au wok.}\end{définition}
\begin{exemple}\pnru{hṽ̩˧∼hṽ̩˧}\hspace{5pt}\peng{|fg{red}}\hspace{5pt}\pcmn{重叠}\hspace{5pt}\pfra{|fg{red}}\end{exemple}
\begin{exemple}\pnru{le˧-hṽ̩˧∼hṽ̩˧}\hspace{5pt}\peng{|fg{accomp} |fg{red}}\hspace{5pt}\pcmn{|fg{accomp} |fg{red}}\hspace{5pt}\pfra{|fg{accomp} |fg{red}}\end{exemple}
\begin{exemple}\pnru{hṽ̩˧∼hṽ̩˧-ze˩}\hspace{5pt}\peng{|fg{red} |fg{pfv}}\hspace{5pt}\pcmn{炒了}\hspace{5pt}\pfra{|fg{red} |fg{pfv}}\end{exemple}
\begin{exemple}\pnru{ʂe˧ hṽ̩˧∼hṽ̩˧}\hspace{5pt}\peng{to stir-fry some meat}\hspace{5pt}\pcmn{炒肉}\hspace{5pt}\pfra{frire de la viande}\end{exemple}
\begin{exemple}\pnru{v̩˩tsʰɤ˧ hṽ̩˧∼hṽ̩˧}\hspace{5pt}\peng{to stir-fry some vegetables}\hspace{5pt}\pcmn{炒蔬菜}\hspace{5pt}\pfra{frire des légumes}\end{exemple}
\begin{exemple}\pnru{læ˧tsɯ˥ hṽ̩˩∼hṽ̩˩}\hspace{5pt}\peng{to stir-fry chili peppers}\hspace{5pt}\pcmn{炒辣椒}\hspace{5pt}\pfra{frire des piments}\end{exemple}
\begin{exemple}\pnru{hɑ˧ hṽ̩˧∼hṽ̩˧}\hspace{5pt}\peng{to stir-fry some rice, to stir-fry some food}\hspace{5pt}\pcmn{炒饭}\hspace{5pt}\pfra{frire un plat/faire la cuisine/frire du riz/de la nourriture}\end{exemple}
\end{entrée}

\begin{entrée}
{hṽ̩˩α}{₁}{ⓔhṽ̩˩αⓗ1}\formedesurface{hṽ̩˩˥}\newline
\classe{形容词}\ton{Lα}
1\begin{définition}\peng{Red.}\end{définition}
\begin{définition}\pcmn{红色的}\end{définition}
\begin{définition}\pfra{Rouge (ex.: vêtement rouge, sang rouge).}\end{définition}
\end{entrée}

\begin{entrée}
{hṽ̩˩α}{₂}{ⓔhṽ̩˩αⓗ2}\formedesurface{hṽ̩˩˥}\newline
\classe{形容词}\ton{Lα}
2\begin{définition}\peng{Of small stature, small in stature, short, not tall; low.}\end{définition}
\begin{définition}\pcmn{矮、低}\end{définition}
\begin{définition}\pfra{De petite taille, de petite stature; bas.}\end{définition}
\end{entrée}

\begin{entrée}
{hṽ̩˧dɤ˧ɻ̍\#˥}{}{ⓔhṽ̩˧dɤ˧ɻ̍\#˥}\formedesurface{hṽ̩˧dɤ˧ɻ̍˧}\newline
\classe{形容词}\ton{\#H}\begin{définition}\peng{Clumsy, incapable.}\end{définition}
\begin{définition}\pcmn{笨拙,经常损坏东西}\end{définition}
\begin{définition}\pfra{Maladroit, incapable.}\end{définition}
\begin{exemple}\pnru{hṽ̩˩-hĩ˩˥}\hspace{5pt}\peng{|fg{rel}}\hspace{5pt}\pcmn{笨拙的}\hspace{5pt}\pfra{|fg{rel}}\end{exemple}
\begin{exemple}\pnru{hṽ̩˧dɤ˧ɻ̍˧∼hṽ̩˧dɤ˧ɻ̍˧-zo˥}\hspace{5pt}\peng{clumsy person / clumsy boy}\hspace{5pt}\pcmn{笨手笨脚的(男)人}\hspace{5pt}\pfra{un maladroit}\end{exemple}
\begin{exemple}\pnru{hṽ̩˩ɖɻ̍˩∼hṽ̩˧ɖɻ̍˧-gv̩˧}\hspace{5pt}\peng{|fg{red}}\hspace{5pt}\pcmn{重叠:笨笨的}\hspace{5pt}\pfra{|fg{red}}\end{exemple}
\end{entrée}

\begin{entrée}
{hṽ̩˩-ɖʐæ˩ɻæ˥}{}{ⓔhṽ̩˩-ɖʐæ˩ɻæ˥}\formedesurface{hṽ̩˩ɖʐæ˩ɻæ˥}\newline
\classe{形容词}\ton{L+H\#}
\étymologie{
hṽ̩˩a 1
}\begin{définition}\peng{Intensely red, red all over.}\end{définition}
\begin{définition}\pcmn{红红的}\end{définition}
\begin{définition}\pfra{Tout rouge.}\end{définition}
\begin{exemple}\pnru{hṽ̩˩ɖʐæ˩ɻæ˥-gv̩˩}\hspace{5pt}\peng{intensely red, red all over}\hspace{5pt}\pcmn{红红的}\hspace{5pt}\pfra{tout rouge}\end{exemple}
\begin{exemple}\pnru{hṽ̩˩ɖʐæ˩˥ | hṽ̩˩ɖʐæ˩˥ gv̩˩}\hspace{5pt}\peng{|fg{red}; the first two syllables are higher-pitched than the following two}\hspace{5pt}\pcmn{重叠}\hspace{5pt}\pfra{|fg{red}; les deux premières syllabes ont une fréquence fondamentale nettement plus haute que les deux suivantes}\end{exemple}
\end{entrée}

\begin{entrée}
{hṽ̩˩∼hṽ̩˩}{}{ⓔhṽ̩˩∼hṽ̩˩}\formedesurface{hṽ̩˧hṽ̩˩˥}\newline
\classe{动词}\ton{L}\begin{définition}\peng{To disturb, to interfere, to hinder, to obstruct, to impede.}\end{définition}
\begin{définition}\pcmn{干扰、防碍}\end{définition}
\begin{définition}\pfra{Ennuyer, empêcher, faire obstruction à.}\end{définition}
\begin{exemple}\pnru{hĩ˧ hṽ̩˥∼hṽ̩˩}\hspace{5pt}\peng{to annoy people}\hspace{5pt}\pcmn{干扰人家}\hspace{5pt}\pfra{ennuyer les gens}\end{exemple}
\end{entrée}

\begin{entrée}
{hṽ̩˧∼hṽ̩˩-ɖʐæ˩∼ɖʐæ˩}{}{ⓔhṽ̩˧∼hṽ̩˩-ɖʐæ˩∼ɖʐæ˩}\formedesurface{hṽ̩˧hṽ̩˩ɖʐæ˩ɖʐæ˩}\newline
\classe{形容词}\ton{L\#-}
\étymologie{
hṽ̩˩a 1
}\begin{définition}\peng{Intensely red, red all over.}\end{définition}
\begin{définition}\pcmn{红红的}\end{définition}
\begin{définition}\pfra{Tout rouge.}\end{définition}
\end{entrée}

\begin{entrée}
{hṽ̩˧nɑ˩}{}{ⓔhṽ̩˧nɑ˩}\formedesurface{hṽ̩˧nɑ˩}\newline
\classe{名词}\ton{L\#}
\paradigme{\pcmn{:} \p{}}
\begin{définition}\peng{Wild animal.}\end{définition}
\begin{définition}\pcmn{野兽}\end{définition}
\begin{définition}\pfra{Bête sauvage.}\end{définition}
\end{entrée}

\begin{entrée}
{hwɑ˩kwɤ˧}{}{ⓔhwɑ˩kwɤ˧}\formedesurface{hwɑ˩kwɤ˥}\newline
\classe{名词}\ton{LM}
\paradigme{\pcmn{:} \p{}}
\begin{définition}\peng{Cucumber.}\end{définition}
\begin{définition}\pcmn{黄瓜(汉语借词)}\end{définition}
\begin{définition}\pfra{Concombre.}\end{définition}
\end{entrée}

\begin{entrée}
{hwæ˧α}{}{ⓔhwæ˧α}\formedesurface{hwæ˧}\newline
\classe{动词}\ton{Mα}\begin{définition}\peng{To buy.}\end{définition}
\begin{définition}\pcmn{买}\end{définition}
\begin{définition}\pfra{Acheter.}\end{définition}
\begin{exemple}\pnru{le˧-hwæ˧}\hspace{5pt}\peng{|fg{accomp}}\hspace{5pt}\pcmn{|fg{accomp}}\hspace{5pt}\pfra{|fg{accomp}}\end{exemple}
\begin{exemple}\pnru{tso˧∼tso˧ hwæ˩}\hspace{5pt}\peng{to buy things}\hspace{5pt}\pcmn{买东西}\hspace{5pt}\pfra{acheter des choses}\end{exemple}
\begin{exemple}\pnru{ɖɯ˧-kʰwɤ˥ hwæ˩}\hspace{5pt}\peng{to buy a piece (of something)}\hspace{5pt}\pcmn{买一块}\hspace{5pt}\pfra{acheter un morceau}\end{exemple}
\begin{exemple}\pnru{hwæ˧∼hwæ˩}\hspace{5pt}\peng{|fg{red}}\hspace{5pt}\pcmn{重叠}\hspace{5pt}\pfra{|fg{red}}\end{exemple}
\end{entrée}

\begin{entrée}
{hwæ˩α}{₁}{ⓔhwæ˩αⓗ1}\formedesurface{hwæ˩˥}\newline
\classe{动词}\ton{Lα}
1\begin{définition}\peng{To close the door (from outside).}\end{définition}
\begin{définition}\pcmn{关(出门,就关门)}\end{définition}
\begin{définition}\pfra{Fermer (la porte).}\end{définition}
\begin{exemple}\pnru{kʰi˧ | tʰi˧-hwæ˩!}\hspace{5pt}\peng{Close the door!}\hspace{5pt}\pcmn{关门吧!}\hspace{5pt}\pfra{Ferme la porte!}\end{exemple}
\begin{exemple}\pnru{ʂe˧bæ˧ | le˧-wo˧-hwæ˥}\hspace{5pt}\pfra{mettre la chaîne à la porte (quand on sort de la maison, on ferme la porte avec une chaîne de fer, et un verrou)}\end{exemple}
\begin{exemple}\pnru{kʰi˧-bi˥ di˩-hĩ˩ ʂe˩bæ˩}\end{exemple}
\end{entrée}

\begin{entrée}
{hwæ˩α}{₂}{ⓔhwæ˩αⓗ2}\formedesurface{hwæ˩˥}\newline
\classe{动词}\ton{Lα}
2\begin{définition}\peng{To suspend, to hang (in a place).}\end{définition}
\begin{définition}\pcmn{悬挂、挂在墙上}\end{définition}
\begin{définition}\pfra{Accrocher, suspendre; être accroché, suspendu à; se tenir à.}\end{définition}
\begin{exemple}\pnru{tso˧∼tso˧ | gɤ˧bi˧ hwæ˥}\hspace{5pt}\peng{to suspend things up high (e.g. on a hook)}\hspace{5pt}\pcmn{挂东西在上面}\hspace{5pt}\pfra{accrocher des choses en hauteur}\end{exemple}
\begin{exemple}\pnru{tso˧∼tso˧ hwæ˥}\hspace{5pt}\peng{to suspend things}\hspace{5pt}\pcmn{挂东西}\hspace{5pt}\pfra{accrocher des choses}\end{exemple}
\begin{exemple}\pnru{ʂe˧ | tʰi˧-hwæ˩}\hspace{5pt}\peng{to hang meat (above the hearth, to smoke it)}\hspace{5pt}\pcmn{挂肉(在火塘上,为了熏肉)}\hspace{5pt}\pfra{accrocher de la viande (au-dessus du foyer, pour la fumer)}\end{exemple}
\begin{exemple}\pnru{tso˧∼tso˧ | tʰi˧-hwæ˩}\hspace{5pt}\peng{to suspend things}\hspace{5pt}\pcmn{挂东西}\hspace{5pt}\pfra{accrocher des choses}\end{exemple}
\begin{exemple}\pnru{ʂe˧-hwæ˥-di˩}\hspace{5pt}\peng{A small beam in the main room (right under the main beams) where meat is suspended to smoke it. The word literally means “thing to hang meat".}\hspace{5pt}\pcmn{主屋里面的小梁(大梁下面),用来挂肉,熏肉。直译:“挂肉的东西”。}\hspace{5pt}\pfra{Poutrelle de la pièce principale, juste en-dessous des poutres maîtresses, servant à suspendre de la viande qui sèche. Littéralement «objet pour accrocher de la viande»}\end{exemple}
\begin{exemple}\pnru{tso˧∼tso˧-hwæ˥-di˩}\hspace{5pt}\peng{object used to suspend things; this can refer to any object from a hook to a beam on which things are hung}\hspace{5pt}\pcmn{挂(东西)用的(东西),如:钩子、用来挂肉的小梁……}\hspace{5pt}\pfra{objet servant à suspendre des choses; cette périphrase peut par exemple désigner la poutrelle servant à accrocher de la viande, dans la pièce principale}\end{exemple}
\end{entrée}

\begin{entrée}
{hwæ˧ɖʐæ˥}{₁}{ⓔhwæ˧ɖʐæ˥ⓗ1}\formedesurface{hwæ˧ɖʐæ˥}\newline
\classe{名词}\ton{H\#}
1
\paradigme{\pcmn{:} \p{}}
\begin{définition}\peng{Squirrel.}\end{définition}
\begin{définition}\pcmn{松鼠,灰鼠}\end{définition}
\begin{définition}\pfra{Écureuil.}\end{définition}
\begin{exemple}\pnru{hwæ˧ɖʐæ˥-pʰv̩˩}\hspace{5pt}\peng{male squirrel}\hspace{5pt}\pcmn{公松鼠}\hspace{5pt}\pfra{écureuil mâle}\end{exemple}
\begin{exemple}\pnru{hwæ˧ɖʐæ˥-mi˩}\hspace{5pt}\peng{female squirrel}\hspace{5pt}\pcmn{母松鼠}\hspace{5pt}\pfra{écureuil femelle}\end{exemple}
\end{entrée}

\begin{entrée}
{hwæ˧ɖʐæ˥}{₂}{ⓔhwæ˧ɖʐæ˥ⓗ2}\formedesurface{hwæ˧ɖʐæ˥}\newline
\classe{名词}\ton{H\#}
2
\paradigme{\pcmn{:} \p{}}
\begin{définition}\peng{Wart.}\end{définition}
\begin{définition}\pcmn{瘊子、肉赘}\end{définition}
\begin{définition}\pfra{Verrue.}\end{définition}
\begin{exemple}\pnru{hwæ˧ʈʂæ˥ tʰv̩˩}\hspace{5pt}\peng{a wart forms}\hspace{5pt}\pcmn{长瘊子}\hspace{5pt}\pfra{une verrue se forme; attraper une verrue}\end{exemple}
\begin{exemple}\pnru{hwæ˧ʈʂæ˥ | le˧-tʰv̩˧-ze˧!}\hspace{5pt}\peng{A wart has formed!}\hspace{5pt}\pcmn{长瘊子了!}\hspace{5pt}\pfra{Une verrue s'est formée!}\end{exemple}
\end{entrée}

\begin{entrée}
{hwæ˧pʰæ˥}{}{ⓔhwæ˧pʰæ˥}\formedesurface{hwæ˧pʰæ˥}\newline
\classe{名词}\ton{H\#}
\paradigme{\pcmn{:} \p{}}
\begin{définition}\peng{A piece of cloth.}\end{définition}
\begin{définition}\pcmn{一块布}\end{définition}
\begin{définition}\pfra{Pièce de tissu.}\end{définition}
\end{entrée}

\begin{entrée}
{hwæ˧pʰæ˩}{}{ⓔhwæ˧pʰæ˩}\formedesurface{hwæ˧pʰæ˩}\newline
\classe{名词}\ton{L\#}
\paradigme{\pcmn{:} \p{}}
\begin{définition}\peng{Large hoe.}\end{définition}
\begin{définition}\pcmn{大锄}\end{définition}
\begin{définition}\pfra{Grosse houe.}\end{définition}
\begin{exemple}\pnru{hwæ˧pʰæ˩ tʰv̩˩-nɑ˩}\hspace{5pt}\peng{|fg{n}+|fg{dem}+|fg{clf}}\hspace{5pt}\pcmn{这把大锄}\hspace{5pt}\pfra{|fg{n}+|fg{dem}+|fg{clf}}\end{exemple}
\end{entrée}

\begin{entrée}
{hwæ˧pʰæ˩-gv̩˩-di˩}{}{ⓔhwæ˧pʰæ˩-gv̩˩-di˩}\formedesurface{hwæ˧pʰæ˩gv̩˩di˩}\newline
\classe{名词}\ton{L\#-}
\paradigme{\pcmn{:} \p{}}
\begin{définition}\peng{Loom.}\end{définition}
\begin{définition}\pcmn{织布机}\end{définition}
\begin{définition}\pfra{Métier à tisser.}\end{définition}
\begin{exemple}\pnru{hwæ˧pʰæ˩gv̩˩di˩-tɕi˩tɕʰi˧}\hspace{5pt}\peng{industrial sewing machine (formed of the Na word plus the Chinese word for ‘engine')}\hspace{5pt}\pcmn{工业织布机。直译:“织布机器”(在摩梭词后面加上汉语的“机器”)}\hspace{5pt}\pfra{métier à tisser industriel, machine à faire du tissu (formé de ‘métier à tisser' + le mot chinois pour ‘machine')}\end{exemple}
\end{entrée}

\begin{entrée}
{hwæ˧tsɯ˥}{}{ⓔhwæ˧tsɯ˥}\formedesurface{hwæ˧tsɯ˥}\newline
\classe{名词}\ton{H\#}
\paradigme{\pcmn{:} \p{}}
\begin{définition}\peng{Rat.}\end{définition}
\begin{définition}\pcmn{老鼠(汉语借词)}\end{définition}
\begin{définition}\pfra{Rat.}\end{définition}
\begin{exemple}\pnru{hwæ˧tsɯ˥-pʰv̩˩}\hspace{5pt}\peng{male rat}\hspace{5pt}\pcmn{公老鼠}\hspace{5pt}\pfra{rat mâle}\end{exemple}
\begin{exemple}\pnru{hwæ˧tsɯ˥-mi˩}\hspace{5pt}\peng{female rat}\hspace{5pt}\pcmn{母老鼠}\hspace{5pt}\pfra{rat femelle}\end{exemple}
\end{entrée}

\begin{entrée}
{hwæ˧tsɯ˥-njɤ˩di˩}{}{ⓔhwæ˧tsɯ˥-njɤ˩di˩}\formedesurface{hwæ˧tsɯ˥njɤ˩di˩}\newline
\classe{名词}\ton{H\#-}
\paradigme{\pcmn{:} \p{}}
\begin{définition}\peng{Setose thistle.}\end{définition}
\begin{définition}\pcmn{大蓟}\end{définition}
\begin{définition}\pfra{Chardon.}\end{définition}
\end{entrée}

\begin{entrée}
{hwæ˧tsɯ˥-njɤ˩di˩-si˩dzi˩}{}{ⓔhwæ˧tsɯ˥-njɤ˩di˩-si˩dzi˩}\formedesurface{hwæ˧tsɯ˥njɤ˩di˩si˩dzi˩}\newline
\classe{名词}\ton{H\#-}\begin{définition}\peng{Burdock.}\end{définition}
\begin{définition}\pcmn{牛蒡}\end{définition}
\begin{définition}\pfra{Bardane: |\stylefi{Arctium lappa}, plante dont les graines adhèrent à la laine et la queue des moutons. On en tire un médicament contre le rhume.}\end{définition}
\end{entrée}

\begin{entrée}
{hwɤ˥}{}{ⓔhwɤ˥}\formedesurface{hwɤ˧}\newline
\classe{动词}\ton{H}\begin{définition}\peng{To participate in a funeral ceremony (literally ‘to see [the deceased] out').}\end{définition}
\begin{définition}\pcmn{执绋送丧}\end{définition}
\begin{définition}\pfra{Participer à une cérémonie funèbre; littéralement «envoyer quelqu'un», c'est-à-dire accompagner quelqu'un vers l'au-delà.}\end{définition}
\end{entrée}

\begin{entrée}
{hwɤ˥}{}{ⓔhwɤ˥}\formedesurface{hwɤ˧}\newline
\classe{名词}\ton{\#H}
\paradigme{\pcmn{:} \p{}}
\begin{définition}\peng{Gift (usually money) made on important occasions (weddings, etc).}\end{définition}
\begin{définition}\pcmn{在大事发生的时候送的礼物(近期一般给钱):婚礼、葬礼}\end{définition}
\begin{définition}\pfra{Don d'argent à l'occasion des grands événements: décès, mariages.}\end{définition}
\begin{exemple}\pnru{hwɤ˧ | ɖɯ˧-kʰwɤ˥}\hspace{5pt}\peng{a gift, a financial contribution}\hspace{5pt}\pcmn{一个红包}\hspace{5pt}\pfra{un cadeau, un don d'argent}\end{exemple}
\end{entrée}

\begin{entrée}
{hwɤ˧}{}{ⓔhwɤ˧}\formedesurface{hwɤ˧}\newline
\classe{形容词}\ton{M}\begin{définition}\peng{Broad, vast, extensive; big (plain; piece of cloth, vegetable…).}\end{définition}
\begin{définition}\pcmn{宽,辽阔,宽敞}\end{définition}
\begin{définition}\pfra{Vaste (plaine), étendu, grand (pièce de tissu), de grande taille (objet, légume…).}\end{définition}
\begin{exemple}\pnru{qʰɑ˧-hwɤ˧-gv̩˧}\hspace{5pt}\peng{extremely vast}\hspace{5pt}\pcmn{非常宽敞}\hspace{5pt}\pfra{très vaste}\end{exemple}
\end{entrée}

\begin{entrée}
{hwɤ˧˥}{₁}{ⓔhwɤ˧˥ⓗ1}\formedesurface{hwɤ˧˥}\newline
\classe{动词}\ton{MH}
1\begin{définition}\peng{To mend, to patch.}\end{définition}
\begin{définition}\pcmn{补}\end{définition}
\begin{définition}\pfra{Repriser, raccommoder (vêtement).}\end{définition}
\begin{exemple}\pnru{le˧-hwɤ˧˥}\hspace{5pt}\peng{|fg{accomp}}\hspace{5pt}\pcmn{|fg{accomp}}\hspace{5pt}\pfra{|fg{accomp}}\end{exemple}
\begin{exemple}\pnru{bɑ˩lɑ˩ hwɤ˥}\hspace{5pt}\peng{to mend clothes}\hspace{5pt}\pcmn{补衣服}\hspace{5pt}\pfra{réparer un vêtement, recoudre un vêtement, rapetasser un vêtement}\end{exemple}
\end{entrée}

\begin{entrée}
{hwɤ˧˥}{₂}{ⓔhwɤ˧˥ⓗ2}\formedesurface{hwɤ˧˥}\newline
\classe{名词}\ton{MH}
2
\paradigme{\pcmn{:} \p{}}
\begin{définition}\peng{Cat (monosyllable).}\end{définition}
\begin{définition}\pcmn{猫(单音节)}\end{définition}
\begin{définition}\pfra{Chat (monosyllabique).}\end{définition}
\end{entrée}

\begin{entrée}
{hwɤ˧˥}{₃}{ⓔhwɤ˧˥ⓗ3}\formedesurface{hwɤ˧˥}\newline
\classe{名词}\ton{MH}
3\begin{définition}\peng{Rust (monosyllable).}\end{définition}
\begin{définition}\pcmn{锈(单音节)}\end{définition}
\begin{définition}\pfra{Rouille (monosyllabe).}\end{définition}
\end{entrée}

\begin{entrée}
{hwɤ˩}{}{ⓔhwɤ˩}\formedesurface{hwɤ˩˥}\newline
\classe{动词}
\sens{1}
\begin{définition}\peng{To pack, to tie together into a bundle.}\end{définition}
\begin{définition}\pcmn{捆(捆成捆儿)}\end{définition}
\begin{définition}\pfra{Attacher, emballer, préparer un fardeau/une charge, constituer un ballot (avec de l'herbe, des objets…), mettre en botte, mettre en ballot.}\end{définition}
\begin{exemple}\pnru{zɯ˧-wɤ˧ hwɤ˥}\hspace{5pt}\peng{to make a bundle of hay}\hspace{5pt}\pcmn{将草捆成一垛、捆一垛草}\hspace{5pt}\pfra{faire un ballot d'herbes}\end{exemple}
\begin{exemple}\pnru{si˧-wɤ˧ hwɤ˥}\hspace{5pt}\peng{to make a bundle of wood}\hspace{5pt}\pcmn{将木头捆成一堆、捆一堆木头}\hspace{5pt}\pfra{faire un ballot de bois}\end{exemple}
\begin{exemple}\pnru{hɑ˧-wɤ˧ hwɤ˥}\hspace{5pt}\peng{to make a bundle of cut cereals}\hspace{5pt}\pcmn{将粮食捆成一包、捆一包粮食}\hspace{5pt}\pfra{faire un ballot de céréales}\end{exemple}
\begin{exemple}\pnru{wɤ˩ hwɤ˩˥}\hspace{5pt}\peng{to tie together into a bundle, to make a bundle}\hspace{5pt}\pcmn{捆成一包}\hspace{5pt}\pfra{préparer un fardeau, mettre en ballot / mettre (des objets, des choses) en paquet, de façon à ce qu'une personne puisse le porter}\end{exemple}
\begin{exemple}\pnru{wɤ˩˥ | tʰi˧-hwɤ˩}\hspace{5pt}\peng{and then, (we) tie (it) into a bundle!}\hspace{5pt}\pcmn{然后,捆成一包!}\hspace{5pt}\pfra{et après, on en fait un ballot / on en fait un paquet / on attache ça ensemble!}\end{exemple}
\begin{exemple}\pnru{wɤ˩˥ | ɖɯ˧-wɤ˩ hwɤ˩}\hspace{5pt}\peng{to tie another bundle, to make (yet) another bundle}\hspace{5pt}\pcmn{又捆一包}\hspace{5pt}\pfra{faire un ballot de plus}\end{exemple}\sens{2}
\begin{définition}\peng{To be in trouble, to put oneself into trouble (figurative sense: as if one were all tied up with a rope, unable to move, to live one's life normally).}\end{définition}
\begin{définition}\pcmn{有困难、像把自己捆起来一样}\end{définition}
\begin{définition}\pfra{Sens figuré: avoir des embarras, se créer des embarras sur les bras, s'empêtrer.}\end{définition}
\begin{exemple}\pnru{hĩ˧, | wɤ˩ hwɤ˧ ʝi˧-ni˥gv̩˩!}\hspace{5pt}\peng{The people seem to be unhappy / under strain! (Figuratively: they look all tied up, as if they were tied with a rope, unable to move = to live a normal life.)}\hspace{5pt}\pcmn{人家难受,像被捆一样}\hspace{5pt}\pfra{Les gens, ils étaient accablés / ils étaient malheureux! (Image: les gens étaient comme ficelés, incapables de se mouvoir normalement, de vivre leur vie normalement.)}\end{exemple}
\begin{exemple}\pnru{wɤ˩hwɤ˧ ʝi˧-ni˥gv̩˩-ɲi˩-ze˩!}\hspace{5pt}\peng{I have put myself in trouble! / I have made a lot of trouble for myself! / I have put my knickers in a twist!}\hspace{5pt}\pcmn{我给自己找麻烦了!}\hspace{5pt}\pfra{(je me) suis mis des grosses complications sur les bras!}\end{exemple}
\end{entrée}

\begin{entrée}
{hwɤ˩α}{}{ⓔhwɤ˩α}\formedesurface{hwɤ˩˥}\newline
\classe{动词}\ton{Lα}\begin{définition}\peng{To hand over, to pass over, to send.}\end{définition}
\begin{définition}\pcmn{递过去}\end{définition}
\begin{définition}\pfra{Passer un objet, envoyer un objet (à quelqu'un).}\end{définition}
\begin{exemple}\pnru{hĩ˧-ki˧ | tso˧∼tso˧ hwɤ˥}\hspace{5pt}\peng{to send some stuff to someone}\hspace{5pt}\pcmn{给人家寄东西}\hspace{5pt}\pfra{envoyer des choses à quelqu'un}\end{exemple}
\end{entrée}

\begin{entrée}
{hwɤ˩dʑɯ˩}{}{ⓔhwɤ˩dʑɯ˩}\formedesurface{hwɤ˩dʑɯ˩˥}\newline
\classe{名词}\ton{L}
\paradigme{\pcmn{:} \p{}}
\begin{définition}\peng{Cabin, hut.}\end{définition}
\begin{définition}\pcmn{山上过夜的小木房}\end{définition}
\begin{définition}\pfra{Cabane, hutte (qu'on construit sur la montagne quand on doit y passer qq nuits, par exemple pour couper du bois).}\end{définition}
\end{entrée}

\begin{entrée}
{hwɤ˩kæ˧}{}{ⓔhwɤ˩kæ˧}\formedesurface{hwɤ˩kæ˥}\newline
\classe{名词}\ton{LM}\begin{définition}\peng{Red birch; its wood is good: it is used to make ards.}\end{définition}
\begin{définition}\pcmn{红桦树}\end{définition}
\begin{définition}\pfra{Bouleau rouge; son bois est bon, on s'en sert pour fabriquer des araires.}\end{définition}
\begin{exemple}\pnru{hwɤ˩kæ˧-si˧dzi˩}\hspace{5pt}\peng{same meaning: red birch}\hspace{5pt}\pcmn{同上:红桦树}\hspace{5pt}\pfra{même sens: bouleau rouge}\end{exemple}
\end{entrée}

\begin{entrée}
{hwɤ˧-kʰv̩˥}{₁}{ⓔhwɤ˧-kʰv̩˥ⓗ1}\formedesurface{hwɤ˧kʰv̩˥}\newline
\classe{名词}\ton{-H\#}1
\begin{définition}\peng{Year of the Cat (corresponding to the Chinese year of the Rat).}\end{définition}
\begin{définition}\pcmn{鼠年(摩梭话称作“猫年”)}\end{définition}
\begin{définition}\pfra{Année du Chat (correspondant à l'année chinoise du Rat).}\end{définition}
\end{entrée}

\begin{entrée}
{hwɤ˧-kʰv̩˥}{₂}{ⓔhwɤ˧-kʰv̩˥ⓗ2}\formedesurface{hwɤ˧kʰv̩˥}\newline
\classe{形容词}\ton{-H\#}2
\begin{définition}\peng{Born in the year of the Cat (corresponding to the Chinese year of the Rat).}\end{définition}
\begin{définition}\pcmn{属鼠(摩梭话称作“属猫”)}\end{définition}
\begin{définition}\pfra{Né l'année du Chat (correspondant à l'année chinoise du Rat).}\end{définition}
\end{entrée}

\begin{entrée}
{hwɤ˧li˧˥}{}{ⓔhwɤ˧li˧˥}\formedesurface{hwɤ˧li˧˥}\newline
\classe{名词}\ton{MH\#}
\paradigme{\pcmn{:} \p{}}
\begin{définition}\peng{Cat.}\end{définition}
\begin{définition}\pcmn{猫}\end{définition}
\begin{définition}\pfra{Chat.}\end{définition}
\end{entrée}

\begin{entrée}
{hwɤ˧li˧-bv̩˥}{}{ⓔhwɤ˧li˧-bv̩˥}\formedesurface{hwɤ˧li˧bv̩˥}\newline
\classe{名词}\ton{H\#}
\paradigme{\pcmn{:} \p{}}
\begin{définition}\peng{Lower balcony, mezzanine.}\end{définition}
\begin{définition}\pcmn{夹层:主屋的夹层。因为烟多,所以人不能将这个空间当卧室。只有一层薄的木地板。}\end{définition}
\begin{définition}\pfra{Mezzanine: espace de la pièce principale où un plancher est aménagé sous la charpente, formant comme une mezzanine, mais que les habitants humains n'utilisent pas: l'endroit étrant très enfumé, on n'y place qu'un plancher peu solide et sans rambarde; d'où le nom: «(la pièce) du chat». On y laisse parfois des objets (vanneries par exemples), qui y sont relativement préservés des insectes par la fumée.}\end{définition}
\end{entrée}

\begin{entrée}
{hwɤ˧li˧-hwæ˧qʰæ\#˥}{}{ⓔhwɤ˧li˧-hwæ˧qʰæ\#˥}\formedesurface{hwɤ˧li˧-hwæ˧qʰæ˧}\newline
\classe{名词}\ton{\#H}\begin{définition}\peng{Scabious.}\end{définition}
\begin{définition}\pcmn{山萝卜}\end{définition}
\begin{définition}\pfra{Cerfeuil.}\end{définition}
\end{entrée}

\begin{entrée}
{hwɤ˧li˧-se˧-di˧˥}{}{ⓔhwɤ˧li˧-se˧-di˧˥}\formedesurface{hwɤ˧li˧se˧di˧˥}\newline
\classe{名词}\ton{MH\#}
\paradigme{\pcmn{:} \p{}}
\begin{définition}\peng{Lower balcony, mezzanine.}\end{définition}
\begin{définition}\pcmn{夹层:主屋的夹层。因为烟多,所以人不能将这个空间当卧室。只有一层薄的木地板。}\end{définition}
\begin{définition}\pfra{Mezzanine: espace de la pièce principale où un plancher est aménagé sous la charpente, formant comme une mezzanine, mais que les habitants humains n'utilisent pas: l'endroit étrant très enfumé, on n'y place qu'un plancher peu solide et sans rambarde; d'où le nom: «(la pièce) du chat». On y laisse parfois des objets (vanneries par exemples), qui y sont relativement préservés des insectes par la fumée.}\end{définition}
\end{entrée}

\begin{entrée}
{hwɤ˧li˧-ʂɯ˧mo˥}{}{ⓔhwɤ˧li˧-ʂɯ˧mo˥}\formedesurface{hwɤ˧li˧ʂɯ˧mo˥}\newline
\classe{名词}\ton{H\#}
\paradigme{\pcmn{:} \p{}}
\begin{définition}\peng{Old cat (male or female).}\end{définition}
\begin{définition}\pcmn{老猫(不分公、母)}\end{définition}
\begin{définition}\pfra{Vieux chat, vieux matou (de l'un ou l'autre sexe).}\end{définition}
\end{entrée}

\begin{entrée}
{hwɤ˧li˧-zo˧˥}{}{ⓔhwɤ˧li˧-zo˧˥}\formedesurface{hwɤ˧li˧zo˧˥}\newline
\classe{名词}\ton{MH\#}
\paradigme{\pcmn{:} \p{}}
\begin{définition}\peng{Kitten, cub.}\end{définition}
\begin{définition}\pcmn{小猫}\end{définition}
\begin{définition}\pfra{Chaton.}\end{définition}
\end{entrée}

\begin{entrée}
{hwɤ˧mi˥\$}{}{ⓔhwɤ˧mi˥\$}\formedesurface{hwɤ˧mi˥}\newline
\classe{名词}\ton{H\$}
\paradigme{\pcmn{:} \p{}}
\begin{définition}\peng{She-cat, queen.}\end{définition}
\begin{définition}\pcmn{母猫}\end{définition}
\begin{définition}\pfra{Chatte.}\end{définition}
\begin{exemple}\pnru{hwɤ˧mi˧-hwɤ˥pʰv̩˩ / hwɤ˧mi˧-hwɤ˧pʰv̩˥\#}\hspace{5pt}\peng{she-cat and tom-cat}\hspace{5pt}\pcmn{母猫与公猫}\hspace{5pt}\pfra{chatte et matou}\end{exemple}
\end{entrée}

\begin{entrée}
{hwɤ˧pʰv̩\#˥}{}{ⓔhwɤ˧pʰv̩\#˥}\formedesurface{hwɤ˧pʰv̩˧}\newline
\classe{名词}\ton{\#H}
\paradigme{\pcmn{:} \p{}}
\begin{définition}\peng{Tom-cat, tom.}\end{définition}
\begin{définition}\pcmn{公猫}\end{définition}
\begin{définition}\pfra{Matou, chat mâle.}\end{définition}
\begin{exemple}\pnru{hwɤ˧pʰv̩˧ tʰv̩˧-mi˥\#}\hspace{5pt}\peng{|fg{n}+|fg{dem}+|fg{clf}}\hspace{5pt}\pcmn{那个公猫}\hspace{5pt}\pfra{|fg{n}+|fg{dem}+|fg{clf}}\end{exemple}
\begin{exemple}\pnru{hwɤ˧pʰv̩˧-hwɤ˧mi˥}\hspace{5pt}\peng{tom-cat and she-cat}\hspace{5pt}\pcmn{公猫与母猫}\hspace{5pt}\pfra{matou et chatte}\end{exemple}
\end{entrée}

\begin{entrée}
{hwɤ˧se˧}{}{ⓔhwɤ˧se˧}\formedesurface{hwɤ˧se˧}\newline
\classe{名词}\ton{M}\begin{définition}\peng{Peanuts.}\end{définition}
\begin{définition}\pcmn{花生}\end{définition}
\begin{définition}\pfra{Cacahuètes.}\end{définition}
\begin{exemple}\pnru{hwɤ˧se˧-qo˧tv̩˩}\hspace{5pt}\peng{peanuts}\hspace{5pt}\pcmn{花生米}\hspace{5pt}\pfra{même sens; littéralement «graines de cacahuètes»}\end{exemple}
\end{entrée}

\begin{entrée}
{hwɤ˧tɕi˥}{}{ⓔhwɤ˧tɕi˥}\formedesurface{hwɤ˧tɕi˥}\newline
\classe{名词}\ton{H\#}
\paradigme{\pcmn{:} \p{}}
\begin{définition}\peng{Curly dock, |\stylefi{Rumex crispus}. It is one of three sorts of plants used as pig fodder; it is also used as food for humans.}\end{définition}
\begin{définition}\pcmn{土大黄(学名:皱叶酸模)(喂猪的牧草)}\end{définition}
\begin{définition}\pfra{Parelle sauvage, oseille crépue, patience crépue, patience sauvage, |\stylefi{Rumex crispus}. Cette plante constitue l'une des trois sortes de fourrage utilisées pour les cochons; elle est aussi consommée par les humains.}\end{définition}
\begin{exemple}\pnru{hwɤ˧tɕi˥-bæ˩bæ˩}\hspace{5pt}\peng{same meaning}\hspace{5pt}\pcmn{同上}\hspace{5pt}\pfra{même sens}\end{exemple}
\begin{exemple}\pnru{hwɤ˧tɕʰi˥-ʁo˩bv̩˩}\hspace{5pt}\peng{sprouts of curly dock}\hspace{5pt}\pcmn{土大黄的嫩芽}\hspace{5pt}\pfra{pousses de parelle sauvage}\end{exemple}
\end{entrée}

\begin{entrée}
{hwɤ˩ʈi˥}{}{ⓔhwɤ˩ʈi˥}\formedesurface{hwɤ˩ʈi˥}\newline
\classe{动词}\ton{LH}\begin{définition}\peng{To become rusty, to get rusty, to rust.}\end{définition}
\begin{définition}\pcmn{生锈}\end{définition}
\begin{définition}\pfra{Rouiller.}\end{définition}
\begin{exemple}\pnru{hwɤ˩ʈi˥-ze˩}\hspace{5pt}\peng{|fg{pfv}: it has become rusty}\hspace{5pt}\pcmn{生锈了}\hspace{5pt}\pfra{|fg{pfv}: ça a rouillé}\end{exemple}
\end{entrée}

\begin{entrée}
{hwɤ˧zo\#˥}{}{ⓔhwɤ˧zo\#˥}\formedesurface{hwɤ˧zo˧}\newline
\classe{名词}\ton{\#H}
\paradigme{\pcmn{:} \p{}}
\begin{définition}\peng{Kitten.}\end{définition}
\begin{définition}\pcmn{小猫}\end{définition}
\begin{définition}\pfra{Chaton.}\end{définition}
\begin{exemple}\pnru{hwɤ˧zo˧ tʰv̩˧-ɭɯ\#˥}\hspace{5pt}\peng{|fg{n}+|fg{dem}+|fg{clf}}\hspace{5pt}\pcmn{那个小猫}\hspace{5pt}\pfra{|fg{n}+|fg{dem}+|fg{clf}}\end{exemple}
\begin{exemple}\pnru{hwɤ˧zo˧-hwɤ˧mi˥}\hspace{5pt}\peng{cats, the cat family: kitten and parents}\hspace{5pt}\pcmn{猫,包括小猫、母猫和公猫}\hspace{5pt}\pfra{chats (toute la famille: chatons et parents)}\end{exemple}
\end{entrée}

\begin{entrée}
{hwɤ̃˩α}{}{ⓔhwɤ̃˩α}\formedesurface{hwɤ̃˩˥}\newline
\classe{形容词}\ton{Lα}\begin{définition}\peng{Late.}\end{définition}
\begin{définition}\pcmn{迟,晚}\end{définition}
\begin{définition}\pfra{En retard.}\end{définition}
\begin{exemple}\pnru{hwɤ̃˩-hĩ˩˥}\hspace{5pt}\peng{|fg{rel}/|fg{nmlz}}\hspace{5pt}\pcmn{迟的}\hspace{5pt}\pfra{|fg{rel}/|fg{nmlz}}\end{exemple}
\begin{exemple}\pnru{ʈʂʰɯ˧ ʑi˧-ʈi˥ hwɤ̃˩!}\hspace{5pt}\peng{He/she gets up late!}\hspace{5pt}\pcmn{他起床起得晚!}\hspace{5pt}\pfra{il se lève tard}\end{exemple}
\end{entrée}

\newpage\caractère{i}

\begin{entrée}
{ĩ˧}{}{ⓔĩ˧}\formedesurface{ĩ˧}\newline
\classe{感叹词}\ton{M}\begin{définition}\peng{Yes, OK.}\end{définition}
\begin{définition}\pcmn{是的,好的}\end{définition}
\begin{définition}\pfra{Oui, d'accord.}\end{définition}
\begin{exemple}\pnru{ʈʂʰɯ˧-ɳɯ˧ | “ĩ˧! ĩ˧!" | pi˧. |}\hspace{5pt}\peng{(S)he said “Yes! yes!"}\hspace{5pt}\pcmn{他说:“是的,是的!”}\hspace{5pt}\pfra{il a dit: «Oui, oui!»}\end{exemple}
\end{entrée}

\newpage\caractère{j}

\begin{entrée}
{jɤ˧}{₁}{ⓔjɤ˧ⓗ1}\formedesurface{jɤ˧}\newline
\classe{形容词}\ton{M}
1\begin{définition}\peng{Good (only appears in negative construction).}\end{définition}
\begin{définition}\pcmn{好(只出现在否定词后面)}\end{définition}
\begin{définition}\pfra{Bien (ne s'utilise qu'en tournure négative).}\end{définition}
\begin{exemple}\pnru{mɤ˧-jɤ˧}\hspace{5pt}\peng{|fg{neg}: it's not good! It's not right! (About someone's behaviour)}\hspace{5pt}\pcmn{不好(形容一个人的行为)}\hspace{5pt}\pfra{|fg{neg}: ce n'est pas bien! / c'est pas beau, ça! (Au sujet du comportement de quelqu'un)}\end{exemple}
\end{entrée}

\begin{entrée}
{jɤ˧}{₂}{ⓔjɤ˧ⓗ2}\formedesurface{jɤ˧}\newline
\classe{形容词}\ton{M}
2\begin{définition}\peng{Flat.}\end{définition}
\begin{définition}\pcmn{平(土地)}\end{définition}
\begin{définition}\pfra{Plat.}\end{définition}
\begin{exemple}\pnru{mɤ˧-jɤ˧}\hspace{5pt}\peng{|fg{neg}: not flat; uneven}\hspace{5pt}\pcmn{不平}\hspace{5pt}\pfra{|fg{neg}: pas plat; inégal}\end{exemple}
\end{entrée}

\begin{entrée}
{jɤ˧}{₃}{ⓔjɤ˧ⓗ3}\formedesurface{jɤ˧}\newline
\classe{名词}\ton{M}
3
\paradigme{\pcmn{:} \p{}}
\begin{définition}\peng{Tobacco, cigarettes.}\end{définition}
\begin{définition}\pcmn{烟}\end{définition}
\begin{définition}\pfra{Tabac, cigarette.}\end{définition}
\begin{exemple}\pnru{jɤ˧ ʈʰɯ˩}\hspace{5pt}\peng{to smoke tobacco}\hspace{5pt}\pcmn{抽烟}\hspace{5pt}\pfra{fumer}\end{exemple}
\end{entrée}

\begin{entrée}
{jɤ˧˥}{₁}{ⓔjɤ˧˥ⓗ1}\formedesurface{jɤ˧˥}\newline
\classe{动词}\ton{MH}
1\begin{définition}\peng{To lick.}\end{définition}
\begin{définition}\pcmn{舔}\end{définition}
\begin{définition}\pfra{Lécher.}\end{définition}
\begin{exemple}\pnru{tso˧∼tso˧ jɤ˩}\hspace{5pt}\peng{to lick something}\hspace{5pt}\pcmn{舔东西}\hspace{5pt}\pfra{lécher quelque chose}\end{exemple}
\begin{exemple}\pnru{dzɯ˧-di˧ jɤ˥}\hspace{5pt}\peng{to lick food}\hspace{5pt}\pcmn{舔食品}\hspace{5pt}\pfra{lécher de la nourriture}\end{exemple}
\begin{exemple}\pnru{tso˧∼tso˧ ɖɯ˧-kʰwɤ˥ jɤ˩-ze˩}\hspace{5pt}\peng{(S)he has licked something.}\hspace{5pt}\pcmn{他舔了一个东西。}\hspace{5pt}\pfra{(elle/il) a léché quelque chose}\end{exemple}
\begin{exemple}\pnru{qʰwɤ˧ jɤ˥}\hspace{5pt}\peng{to lick a bowl (what one does when there is no food left; a beggar licks a bowl for the last bits of food left)}\hspace{5pt}\pcmn{舔一个碗(如:乞丐舔碗)}\hspace{5pt}\pfra{lécher le plat, lécher un bol (ce qu'on fait lorsqu'un plat est terminé: comme le petit doigt dans la comptine enfantine, ou comme le mendiant à qui on donne des restes)}\end{exemple}
\end{entrée}

\begin{entrée}
{jɤ˧˥}{₂}{ⓔjɤ˧˥ⓗ2}\formedesurface{jɤ˧˥}\newline
\classe{名词}\ton{MH}
2\begin{définition}\peng{A wild radish that grows on the mountains; it is edible; it is picked and eaten in the Spring, when vegetables are not ripe yet. Yi people harvest it and sell it in the plain.}\end{définition}
\begin{définition}\pcmn{红萝卜菜:一种山上的野菜。春天的时候,菜园的蔬菜还没有成熟的时候,永宁的人吃红萝卜菜。彝族在高山上采下来,在永宁卖。}\end{définition}
\begin{définition}\pfra{Radis sauvage qui pousse en montagne; on le consomme surtout au printemps, à une époque où il n'y a pas encore de légumes. Ce radis est récoltée par les Yi et vendu dans la plaine.}\end{définition}
\begin{exemple}\pnru{jɤ˧ dzɯ˧ | qʰɑ˧-sɯ˥∼sɯ˩, | jɤ˧ ʈʂɤ˥ ŋv̩˩-ɭɯ˩∼ɭɯ˩!}\hspace{5pt}\peng{“The wild radish is bitter; and its harvest costs tears! / The wild radish tastes bitter; and its harvest is bitter, too! / The wild radish is all bitterness inside, and all bitterness at the harvest!" This proverb evokes the difficulty of the harvest, which requires long wanderings up high on the mountain.}\hspace{5pt}\pcmn{“红萝卜菜,味道苦,去摘也要流眼泪! / 红萝卜菜,吃起来苦,摘起来也苦!”摘红萝卜菜,需要爬高山,寻找时间长,永宁坝子的农民觉得这比较苦。}\hspace{5pt}\pfra{«Le radis sauvage, ça a un goût amer quand on le mange, et ça vous fait pleurer pour le récolter!» (de: /qʰɑ˧/ ‘amer'+expressif) / «récolter le radis sauvage, ça fait pleurer!» Non pas à cause de la plante elle-même, pas comme un oignon qui piquerait les yeux: mais parce qu'on s'épuisait à aller le chercher en haute montagne.}\end{exemple}
\end{entrée}

\begin{entrée}
{jɤ˧˥}{₃}{ⓔjɤ˧˥ⓗ3}\formedesurface{jɤ˧˥}\newline
\classe{动词}\ton{MH}
3\begin{définition}\peng{To spread, to put on, to smear.}\end{définition}
\begin{définition}\pcmn{抹、涂抹}\end{définition}
\begin{définition}\pfra{Étendre, appliquer, mettre (ex.: appliquer un onguent).}\end{définition}
\begin{exemple}\pnru{pʰv˧ʂɯ˧ jɤ˧˥}\hspace{5pt}\peng{to put on beauty cream or sunscreen}\hspace{5pt}\pcmn{抹防晒霜}\hspace{5pt}\pfra{appliquer une crème de beauté ou de la crème solaire}\end{exemple}
\begin{exemple}\pnru{mɤ˩ jɤ˩˥}\hspace{5pt}\peng{to apply grease (e.g. to the skin)}\hspace{5pt}\pcmn{涂抹油}\hspace{5pt}\pfra{appliquer de la graisse (ex.: sur une peau sèche)}\end{exemple}
\begin{exemple}\pnru{tʰi˧-jɤ˧˥}\hspace{5pt}\pfra{|fg{dur} \_}\end{exemple}
\end{entrée}

\begin{entrée}
{jɤ˧˥α}{₁}{ⓔjɤ˧˥αⓗ1}\formedesurface{ɖɯ˧ jɤ˧˥}\newline
\classe{量词}\ton{MHα}
1\begin{définition}\peng{Classifier used for women, and for some female domestic animals; it does not carry any hint of deprecation, nor does it convey any hint of respect by itself.}\end{définition}
\begin{définition}\pcmn{量词:母性、雌性(人或动物)(一个/一只)}\end{définition}
\begin{définition}\pfra{Classificateur des créatures femelles; employé pour les personnes de sexe féminin (appellation qui ne marque pas de respect, mais n'est pas injurieuse), et pour certains animaux domestiques.}\end{définition}
\end{entrée}

\begin{entrée}
{jɤ˧˥α}{₂}{ⓔjɤ˧˥αⓗ2}\formedesurface{ɖɯ˧ jɤ˧˥}\newline
\classe{量词}\ton{MHα}
2\begin{définition}\peng{Classifier for dough balls and teacakes. One dough ball is the quantity of dough that can be prepared with one egg. Tea consumed in the Yongning area in the first half of the 20th century was green tea from a large leaf variety of Camellia sinensis (C. sinensis assamica) found in the mountains of southern Yunnan; it used to be pressed into ‘teacake' shape.}\end{définition}
\begin{définition}\pcmn{量词:面(一团),茶饼(一个)等。(一团面,是和了一个鸡蛋的面团的量。)}\end{définition}
\begin{définition}\pfra{Classificateur pour la pâte à pain: quantité de pâte aux œufs que l'on peut préparer avec un œuf. Ce classificateur est également utilisé pour le thé tassé en galettes.}\end{définition}
\begin{exemple}\pnru{æ˩ʁv̩˩-pɤ˥jɤ˩ | ɖɯ˧-jɤ˧˥}\hspace{5pt}\peng{a ball of egg dough}\hspace{5pt}\pcmn{一个鸡蛋面团}\hspace{5pt}\pfra{une boule de pâte à pain à l'oeuf}\end{exemple}
\begin{exemple}\pnru{ʝi˧-jɤ˧˥}\hspace{5pt}\peng{one ball/cake}\hspace{5pt}\pcmn{一个团/并}\hspace{5pt}\pfra{une boule/galette}\end{exemple}
\end{entrée}

\begin{entrée}
{jɤ˩α}{₁}{ⓔjɤ˩αⓗ1}\formedesurface{jɤ˩˥}\newline
\classe{动词}\ton{Lα}
1\begin{définition}\peng{To coil.}\end{définition}
\begin{définition}\pcmn{盘、盘绕(线)}\end{définition}
\begin{définition}\pfra{Enrouler (ex.: des fibres de lin).}\end{définition}
\begin{exemple}\pnru{sɑ˧ jɤ˥}\hspace{5pt}\peng{to coil linen thread}\hspace{5pt}\pcmn{盘麻线}\hspace{5pt}\pfra{enrouler du fil de lin}\end{exemple}
\begin{exemple}\pnru{sɑ˧ | le˧-jɤ˩}\hspace{5pt}\peng{to coil linen thread}\hspace{5pt}\pcmn{盘麻线}\hspace{5pt}\pfra{enrouler du fil de lin}\end{exemple}
\end{entrée}

\begin{entrée}
{jɤ˩α}{₂}{ⓔjɤ˩αⓗ2}\formedesurface{jɤ˩˥}\newline
\classe{形容词}\ton{Lα}
2\begin{définition}\peng{Overcooked, overdone, mushy, sodden, mushed.}\end{définition}
\begin{définition}\pcmn{(煮)烂}\end{définition}
\begin{définition}\pfra{Trop cuit, dissous, tout décomposé: ex.: pommes de terre qui éclatent, pois qui se décomposent en purée.}\end{définition}
\begin{exemple}\pnru{le˧-tɕɤ˧˥ | le˧-jɤ˩-ze˩!}\hspace{5pt}\peng{It got sodden after boiling! / After boiling, it got all mushy/overdone!}\hspace{5pt}\pcmn{煮烂了!}\hspace{5pt}\pfra{ça s'est décomposé à force de bouillir!}\end{exemple}
\begin{exemple}\pnru{jɤ˩-hĩ˩˥}\hspace{5pt}\peng{|fg{rel}/|fg{nmlz}}\hspace{5pt}\pcmn{烂的}\hspace{5pt}\pfra{|fg{rel}/|fg{nmlz}}\end{exemple}
\end{entrée}

\begin{entrée}
{jɤ˩β}{₁}{ⓔjɤ˩βⓗ1}\formedesurface{jɤ˩˥}\newline
\classe{动词}\ton{Lβ}
1\begin{définition}\peng{To be listless, to be dejected.}\end{définition}
\begin{définition}\pcmn{没精神}\end{définition}
\begin{définition}\pfra{Être fatigué, être sans entrain, être à la masse.}\end{définition}
\begin{exemple}\pnru{tʰi˧-jɤ˩-ho˩-ze˩!}\hspace{5pt}\peng{(S)he is getting listless/dispirited!}\hspace{5pt}\pcmn{他没精神了!}\hspace{5pt}\pfra{(Il/elle) est à la masse!}\end{exemple}
\begin{exemple}\pnru{ɑ˩ʁo˧ ʂv̩˧ɖv̩˧ | tʰi˧-jɤ˩-ho˩-tsɯ˩!}\hspace{5pt}\peng{When one misses home, one gets listless/dispirited!}\hspace{5pt}\pcmn{想家的时候,没精神!}\hspace{5pt}\pfra{Quand on a la nostalgie, on est sans entrain!}\end{exemple}
\begin{exemple}\pnru{ɑ˩ʁo˧ ʂv̩˧ɖv̩˧-zo˥, | tʰi˧-jɤ˩-ho˩!}\hspace{5pt}\peng{When one misses home, one gets listless/dispirited!}\hspace{5pt}\pcmn{想家的时候,没精神!}\hspace{5pt}\pfra{Quand on a la nostalgie, on est sans entrain!}\end{exemple}
\begin{exemple}\pnru{ɲi˧mi˧ tsʰi˧-zo˩, | tʰi˧-jɤ˩-ho˩!}\hspace{5pt}\peng{When the weather is hot, one gets listless/dispirited!}\hspace{5pt}\pcmn{天气很热,没精神!}\hspace{5pt}\pfra{Quand il fait très chaud, on est sans entrain!}\end{exemple}
\begin{exemple}\pnru{jɤ˩-mɤ˥-jɤ˩}\hspace{5pt}\peng{\_ |fg{neg} \_}\hspace{5pt}\pcmn{\_ |fg{neg} \_}\hspace{5pt}\pfra{\_ |fg{neg} \_}\end{exemple}
\end{entrée}

\begin{entrée}
{jɤ˩β}{₂}{ⓔjɤ˩βⓗ2}\formedesurface{ɖɯ˧ jɤ˩}\newline
\classe{量词}\ton{Lβ}
2\begin{définition}\peng{Row: classifier for rows of vegetables.}\end{définition}
\begin{définition}\pcmn{量词:排(一排菜)}\end{définition}
\begin{définition}\pfra{Classificateur des rangées de légumes (dans un potager, dans un champ).}\end{définition}
\begin{exemple}\pnru{v˩tsʰɤ˧˥ | ɖɯ˧-jɤ˩ tʰi˩-pʰo˩}\hspace{5pt}\peng{to plant a row of vegetables}\hspace{5pt}\pcmn{种一排菜}\hspace{5pt}\pfra{planter une rangée de légumes}\end{exemple}
\end{entrée}

\begin{entrée}
{jɤ˩ɕjo˧-dzɑ˧qʰwɤ˩}{}{ⓔjɤ˩ɕjo˧-dzɑ˧qʰwɤ˩}\formedesurface{jɤ˩ɕjo˧dzɑ˧qʰwɤ˩}\newline
\classe{名词}\ton{LM-L\#}
\paradigme{\pcmn{:} \p{}}
\begin{définition}\peng{Sandal.}\end{définition}
\begin{définition}\pcmn{凉鞋}\end{définition}
\begin{définition}\pfra{Sandale.}\end{définition}
\end{entrée}

\begin{entrée}
{jɤ˧gɯ˩}{}{ⓔjɤ˧gɯ˩}\formedesurface{jɤ˧gɯ˩}\newline
\classe{名词}\ton{L\#}
\paradigme{\pcmn{:} \p{}}
\begin{définition}\peng{Buckwheat, |\stylefi{Fagopyrum esculentum}.}\end{définition}
\begin{définition}\pcmn{甜荞/荞麦/花荞}\end{définition}
\begin{définition}\pfra{Sarrasin, blé noir, |\stylefi{Fagopyrum esculentum}.}\end{définition}
\end{entrée}

\begin{entrée}
{jɤ˩ho˧}{}{ⓔjɤ˩ho˧}\formedesurface{jɤ˩ho˥}\newline
\classe{名词}\ton{LM}
\paradigme{\pcmn{:} \p{}}
\begin{définition}\peng{Matches.}\end{définition}
\begin{définition}\pcmn{火柴(洋火)}\end{définition}
\begin{définition}\pfra{Allumette.}\end{définition}
\end{entrée}

\begin{entrée}
{jɤ˩jo\#˥}{}{ⓔjɤ˩jo\#˥}\formedesurface{jɤ˩jo˥}\newline
\classe{名词}\ton{LM+\#H}
\paradigme{\pcmn{:} \p{}}
\begin{définition}\peng{Potato.}\end{définition}
\begin{définition}\pcmn{洋芋、土豆 、马铃薯(汉语借词)}\end{définition}
\begin{définition}\pfra{Pomme de terre.}\end{définition}
\end{entrée}

\begin{entrée}
{jɤ˩jo˧-bv̩\#˥}{}{ⓔjɤ˩jo˧-bv̩\#˥}\formedesurface{jɤ˩jo˧-bv̩˧}\newline
\classe{名词}\ton{LM+\#H}
\étymologie{
jɤ˩jo\#˥; bv̩˥
}
\paradigme{\pcmn{:} \p{}}
\begin{définition}\peng{Potato grub, |\stylefi{Agriotes lineatus}.}\end{définition}
\begin{définition}\pcmn{蛴螬}\end{définition}
\begin{définition}\pfra{Larve de taupin, ver fil de fer, |\stylefi{Agriotes lineatus}: ver qui mange les tubercules.}\end{définition}
\end{entrée}

\begin{entrée}
{jɤ˧ŋɤ˧}{}{ⓔjɤ˧ŋɤ˧}\formedesurface{jɤ˧ŋɤ˧}\newline
\classe{名词}\ton{M}\begin{définition}\peng{The city of Chengdu, in Sichuan.}\end{définition}
\begin{définition}\pcmn{成都}\end{définition}
\begin{définition}\pfra{La ville de Chengdu, dans le Sichuan.}\end{définition}
\begin{exemple}\pnru{ho˧di˧-jɤ˧ŋɤ˧}\hspace{5pt}\peng{same meaning}\hspace{5pt}\pcmn{同上}\hspace{5pt}\pfra{même sens}\end{exemple}
\end{entrée}

\begin{entrée}
{jɤ˩pæ˧sɯ˥\$}{}{ⓔjɤ˩pæ˧sɯ˥\$}\formedesurface{jɤ˩pæ˧sɯ˥}\newline
\classe{名词}\ton{LM+H\$}\begin{définition}\peng{Yang Chieftain: a family name from Yongning, containing a name borrowed from Chinese (Yang \stylefn{杨)} plus a term referring to the lowest degree in the hierarchy of feudal leaders: the hamlet chieftain, \stylefn{把事.} Only one family in Yongning carries this name.}\end{définition}
\begin{définition}\pcmn{杨把事。这个姓,由两部分组成的:‘杨’姓(汉语借词)与封建社会最小领导层次:‘把事’。}\end{définition}
\begin{définition}\pfra{Petit Chef Yang: nom de famille constitué de l'expression chinoise \stylefn{杨把事,} formé du patronyme \stylefn{杨,} suivi du terme chinois renvoyant au plus bas degré de la hiérarchie féodale: le chef de hameau, \stylefn{把事.} Ce nom est propre à une seule famille de Yongning.}\end{définition}
\begin{exemple}\pnru{jɤ˩pæ˧sɯ˧=ɻ̍˥\$}\hspace{5pt}\peng{\_ |fg{associative}: the people of the Yang Chieftain family}\hspace{5pt}\pcmn{杨把事家族}\hspace{5pt}\pfra{\_ |fg{associatif}: les gens de la famille Petit Chef Yang}\end{exemple}
\begin{exemple}\pnru{jɤ˩pɑ˧sɯ˥ | ʈæ˧ʂɯ˧}\hspace{5pt}\peng{the proper name of a person of the Yang Chieftain family (given name: Dashi): ‘Dashi of the Yang Chieftain family'.}\hspace{5pt}\pcmn{杨把事家的一个人的名字:杨把事•达石}\hspace{5pt}\pfra{nom propre d'une personne de la famille Petit Chef Yang: ‘Dashi de la famille Petit Chef Yang'.}\end{exemple}
\end{entrée}

\begin{entrée}
{jɤ˩po˧}{}{ⓔjɤ˩po˧}\formedesurface{jɤ˩po˥}\newline
\classe{动词}\ton{LM}\begin{définition}\peng{To gamble, to bet, to wager.}\end{définition}
\begin{définition}\pcmn{赌博、打赌}\end{définition}
\begin{définition}\pfra{Parier.}\end{définition}
\begin{définition}\pfra{Parier.}\end{définition}
\end{entrée}

\begin{entrée}
{jɤ˧qʰɑ\#˥}{}{ⓔjɤ˧qʰɑ\#˥}\formedesurface{jɤ˧qʰɑ˧}\newline
\classe{名词}\ton{\#H}
\paradigme{\pcmn{:} \p{}}
\begin{définition}\peng{Bitter buckwheat, |\stylefi{Fagopyrum tataricum Gaertn}.}\end{définition}
\begin{définition}\pcmn{苦荞}\end{définition}
\begin{définition}\pfra{Sarrasin amer, |\stylefi{Fagopyrum tataricum Gaertn}.}\end{définition}
\end{entrée}

\begin{entrée}
{jɤ˧qʰɑ˧-pɤ˥jɤ˩-mo˩}{}{ⓔjɤ˧qʰɑ˧-pɤ˥jɤ˩-mo˩}\formedesurface{jɤ˧qʰɑ˧pɤ˥jɤ˩mo˩}\newline
\classe{名词}\ton{\#H-}\begin{définition}\peng{Cep, penny bun, porcino, |\stylefi{Boletus edulis} (a type of edible fungus); literally “buckwheat bun mushroom", due to its texture.}\end{définition}
\begin{définition}\pcmn{牛肝菌}\end{définition}
\begin{définition}\pfra{Bolet, cèpe, |\stylefi{Boletus edulis}; littéralement «champignon-galette de sarrasin», du fait de sa texture.}\end{définition}
\end{entrée}

\begin{entrée}
{jɤ˩tʰi˧-ʁæ˩bæ˩}{}{ⓔjɤ˩tʰi˧-ʁæ˩bæ˩}\formedesurface{jɤ˩tʰi˧ʁæ˩bæ˩}\newline
\classe{名词}\ton{LM-L}
\paradigme{\pcmn{:} \p{}}
\begin{définition}\peng{Porcelain plate.}\end{définition}
\begin{définition}\pcmn{瓷盘}\end{définition}
\begin{définition}\pfra{Assiette en faïence/porcelaine.}\end{définition}
\end{entrée}

\begin{entrée}
{jɤ˧wo˧˥}{}{ⓔjɤ˧wo˧˥}\formedesurface{jɤ˧wo˧˥}\newline
\classe{动词}\ton{MH\#}\begin{définition}\peng{To regress.}\end{définition}
\begin{définition}\pcmn{倒退、退步}\end{définition}
\begin{définition}\pfra{Régresser (le contraire de: faire des progrès).}\end{définition}
\begin{exemple}\pnru{no˧ | jɤ˧wo˧˥ | sɯ˧ɖʐæ˧! / no˧ | le˧-wo˥ | sɯ˧ɖʐæ˧!}\hspace{5pt}\peng{You are regressing! (Said to a child who had already developed a habit of going to the toilet in the previous weeks, but who, that day, soiled her trousers.)}\hspace{5pt}\pcmn{你这是在退步!(情景:一个小孩已经几个礼拜有了上厕所的习惯,那天又把屎拉在裤头里)}\hspace{5pt}\pfra{tu régresses! (adressé à un petit enfant qui a fait caca dans sa culotte, alors que depuis plusieurs semaines il avait pris l'habitude du pot)}\end{exemple}
\end{entrée}

\begin{entrée}
{je˧pʰi˧-jɤ\#˥}{}{ⓔje˧pʰi˧-jɤ\#˥}\formedesurface{je˧pʰi˧jɤ˧}\newline
\classe{名词}\ton{\#H}
\étymologie{
fn:鸦片; jɤ˧
}\begin{définition}\peng{Opium.}\end{définition}
\begin{définition}\pcmn{鸦片(汉语借词)}\end{définition}
\begin{définition}\pfra{Opium.}\end{définition}
\end{entrée}

\begin{entrée}
{je˩tʰi˧}{}{ⓔje˩tʰi˧}\formedesurface{je˩tʰi˥}\newline
\classe{名词}\ton{LM}\begin{définition}\peng{Enamel: found in enamel plates, for instance. Literally ‘Westerners' iron'.}\end{définition}
\begin{définition}\pcmn{搪瓷(汉语借词:‘洋铁’)}\end{définition}
\begin{définition}\pfra{Email: matière dont sont faites les assiettes émaillées, par exemple. Littéralement ‘fer des Occidentaux'.}\end{définition}
\end{entrée}

\begin{entrée}
{je˩ʐe˧}{}{ⓔje˩ʐe˧}\formedesurface{je˩ʐe˥}\newline
\classe{名词}\ton{LM}
\paradigme{\pcmn{:} \p{}}
\begin{définition}\peng{Westerner.}\end{définition}
\begin{définition}\pcmn{西方人(“洋人”)(汉语借词)}\end{définition}
\begin{définition}\pfra{Occidental.}\end{définition}
\end{entrée}

\begin{entrée}
{jo˥}{}{ⓔjo˥}\formedesurface{jo˧}\newline
\classe{名词}\ton{\#H}
\paradigme{\pcmn{:} \p{}}
\begin{définition}\peng{Jade.}\end{définition}
\begin{définition}\pcmn{玉石}\end{définition}
\begin{définition}\pfra{Jade (matière, pierre).}\end{définition}
\end{entrée}

\begin{entrée}
{jo˧}{}{ⓔjo˧}\formedesurface{jo˧}\newline
\classe{动词}\ton{M intrans}\begin{définition}\peng{To come; to come in.}\end{définition}
\begin{définition}\pcmn{来}\end{définition}
\begin{définition}\pfra{Venir, entrer.}\end{définition}
\begin{exemple}\pnru{le˧-jo˧-ze˧!}\hspace{5pt}\peng{|fg{accomp} \_ |fg{pfv}: (s)he has come}\hspace{5pt}\pcmn{来了!}\hspace{5pt}\pfra{|fg{accomp} \_ |fg{pfv}: (il/est) est arrivé(e) / est entré(e)!}\end{exemple}
\end{entrée}

\begin{entrée}
{jo˧˥}{}{ⓔjo˧˥}\formedesurface{jo˧˥}\newline
\classe{动词}\ton{MH}\begin{définition}\peng{To offer.}\end{définition}
\begin{définition}\pcmn{赠给}\end{définition}
\begin{définition}\pfra{Offrir.}\end{définition}
\end{entrée}

\begin{entrée}
{jo˩}{}{ⓔjo˩}\formedesurface{jo˧}\newline
\classe{名词}\ton{L}
\paradigme{\pcmn{:} \p{}}
\begin{définition}\peng{Sheep.}\end{définition}
\begin{définition}\pcmn{绵羊}\end{définition}
\begin{définition}\pfra{Mouton.}\end{définition}
\begin{exemple}\pnru{jo˩-ɣɯ˩˥}\hspace{5pt}\peng{sheep skin}\hspace{5pt}\pcmn{羊皮}\hspace{5pt}\pfra{peau de mouton}\end{exemple}
\begin{exemple}\pnru{jo˩hṽ̩˩˥}\hspace{5pt}\peng{wool (sheep wool)}\hspace{5pt}\pcmn{羊毛}\hspace{5pt}\pfra{laine (de mouton)}\end{exemple}
\end{entrée}

\begin{entrée}
{jo˩˧}{}{ⓔjo˩˧}\formedesurface{jo˩˥}\newline
\classe{名词}\ton{LM}\begin{définition}\peng{Right (opposite of: left).}\end{définition}
\begin{définition}\pcmn{右边}\end{définition}
\begin{définition}\pfra{Droite (opposé de: gauche).}\end{définition}
\end{entrée}

\begin{entrée}
{jo˩β}{}{ⓔjo˩β}\formedesurface{ɖɯ˧ jo˩}\newline
\classe{量词}\ton{Lβ}\begin{définition}\peng{An ounce.}\end{définition}
\begin{définition}\pcmn{量词:两(一两)}\end{définition}
\begin{définition}\pfra{Unité de poids: once.}\end{définition}
\end{entrée}

\begin{entrée}
{jo˩gi˩}{}{ⓔjo˩gi˩}\formedesurface{jo˩gi˩˥}\newline
\classe{名词}\ton{L}\begin{définition}\peng{Right (opposite of left).}\end{définition}
\begin{définition}\pcmn{右边}\end{définition}
\begin{définition}\pfra{Droite (contraire de: gauche).}\end{définition}
\begin{exemple}\pnru{jo˩gi˩dzɤ˩}\hspace{5pt}\peng{the side to the right, the right}\hspace{5pt}\pcmn{右边}\hspace{5pt}\pfra{du côté droit, à droite}\end{exemple}
\end{entrée}

\begin{entrée}
{jo˧gv̩˧}{}{ⓔjo˧gv̩˧}\formedesurface{jo˧gv̩˧}\newline
\classe{名词}\ton{M}\begin{définition}\peng{Lijiang.}\end{définition}
\begin{définition}\pcmn{丽江(包括丽江坝子)}\end{définition}
\begin{définition}\pfra{Lijiang (toute la région: la ville, et la plaine environnante).}\end{définition}
\end{entrée}

\begin{entrée}
{jo˧gv̩˧-ŋv̩˧lv̩˧}{}{ⓔjo˧gv̩˧-ŋv̩˧lv̩˧}\formedesurface{jo˧gv̩˧ŋv̩˧lv̩˧}\newline
\classe{名词}\ton{M}\begin{définition}\peng{Yulong snow mountain; literally ‘Lijiang's snow mountain'.}\end{définition}
\begin{définition}\pcmn{玉龙雪山}\end{définition}
\begin{définition}\pfra{La montagne Yulong: principale montagne de Lijiang.}\end{définition}
\end{entrée}

\begin{entrée}
{jo˩-kʰv̩˩}{₁}{ⓔjo˩-kʰv̩˩ⓗ1}\formedesurface{jo˩kʰv̩˩˥}\newline
\classe{名词}\ton{L}
1\begin{définition}\peng{Year of the Goat.}\end{définition}
\begin{définition}\pcmn{羊年}\end{définition}
\begin{définition}\pfra{Année du Mouton.}\end{définition}
\end{entrée}

\begin{entrée}
{jo˩-kʰv̩˩}{₂}{ⓔjo˩-kʰv̩˩ⓗ2}\formedesurface{jo˩kʰv̩˩˥}\newline
\classe{形容词}\ton{L}
2\begin{définition}\peng{Born in the year of the Goat.}\end{définition}
\begin{définition}\pcmn{属羊}\end{définition}
\begin{définition}\pfra{Né l'année du Mouton.}\end{définition}
\end{entrée}

\begin{entrée}
{jo˩lo˩}{}{ⓔjo˩lo˩}\formedesurface{jo˩lo˩˥}\newline
\classe{名词}\ton{L}\begin{définition}\peng{Right (opposite of left).}\end{définition}
\begin{définition}\pcmn{右边}\end{définition}
\begin{définition}\pfra{Droite (contraire de: gauche).}\end{définition}
\end{entrée}

\begin{entrée}
{jo˧lv̩\#˥}{}{ⓔjo˧lv̩\#˥}\formedesurface{jo˧lv̩˧}\newline
\classe{名词}\ton{\#H}\begin{définition}\peng{Prosperity.}\end{définition}
\begin{définition}\pcmn{繁荣、景气}\end{définition}
\begin{définition}\pfra{prospérité}\end{définition}
\begin{exemple}\pnru{jo˧lv̩˧ ɲi˥!}\hspace{5pt}\peng{Let there be prosperity! (A wish, a benediction.)}\hspace{5pt}\pcmn{财源广进!}\hspace{5pt}\pfra{Prospérité!! (Formule de bénédiction, de voeu.)}\end{exemple}
\end{entrée}

\begin{entrée}
{jo˧mi˧}{}{ⓔjo˧mi˧}\formedesurface{jo˧mi˧}\newline
\classe{名词}\ton{M}
\paradigme{\pcmn{:} \p{}}
\begin{définition}\peng{Ewe.}\end{définition}
\begin{définition}\pcmn{母绵羊}\end{définition}
\begin{définition}\pfra{Brebis.}\end{définition}
\begin{exemple}\pnru{jo˧mi˧-po˧lo˧}\hspace{5pt}\peng{ewe and ram}\hspace{5pt}\pcmn{母绵羊与公羊}\hspace{5pt}\pfra{brebis et bélier}\end{exemple}
\end{entrée}

\begin{entrée}
{jo˧mi˧-ʁwɤ˧}{}{ⓔjo˧mi˧-ʁwɤ˧}\formedesurface{jo˧mi˧ʁwɤ˧}\newline
\classe{名词}\ton{M}\begin{définition}\peng{The second village that one crosses when going from \stylefv{/qʰæ}˧tɕʰi˧/ to \stylefv{/ʈʂo}˧ʂɯ\#˥/.}\end{définition}
\begin{définition}\pcmn{有米瓦村}\end{définition}
\begin{définition}\pfra{Le second village que l'on rencontre sur le trajet entre \stylefv{/qʰæ}˧tɕʰi˧/ et \stylefv{/ʈʂo}˧ʂɯ\#˥/.}\end{définition}
\end{entrée}

\begin{entrée}
{jo˧mv̩\#˥}{}{ⓔjo˧mv̩\#˥}\formedesurface{jo˧mv̩˧}\newline
\classe{名词}\ton{\#H}\begin{définition}\peng{Youmi, a hamlet in Labai.}\end{définition}
\begin{définition}\pcmn{拉柏乡油米村}\end{définition}
\begin{définition}\pfra{Youmi, un hameau de Labai.}\end{définition}
\begin{exemple}\pnru{jo˧mv̩˧ hĩ\#˥}\hspace{5pt}\peng{People of Youmi.}\hspace{5pt}\pcmn{油米村人}\hspace{5pt}\pfra{Les gens du hameau de Youmi.}\end{exemple}
\end{entrée}

\begin{entrée}
{jo˩pv̩˧}{}{ⓔjo˩pv̩˧}\formedesurface{jo˩pv̩˥}\newline
\classe{名词}\ton{LM / LM+MH\#}
\paradigme{\pcmn{:} \p{}}
\begin{définition}\peng{Oilcloth; tarpaulin.}\end{définition}
\begin{définition}\pcmn{油布}\end{définition}
\begin{définition}\pfra{Toile cirée.}\end{définition}
\begin{exemple}\pnru{jo˩pv̩˧˥}\hspace{5pt}\peng{oilcloth (tonal variant)}\hspace{5pt}\pcmn{油布(声调变体)}\hspace{5pt}\pfra{toile cirée (variante tonale)}\end{exemple}
\end{entrée}

\begin{entrée}
{jo˩pʰv̩˩}{}{ⓔjo˩pʰv̩˩}\formedesurface{jo˩pʰv̩˩˥}\newline
\classe{名词}\ton{L}
\paradigme{\pcmn{:} \p{}}
\begin{définition}\peng{Male sheep.}\end{définition}
\begin{définition}\pcmn{公绵羊}\end{définition}
\begin{définition}\pfra{Bélier.}\end{définition}
\begin{exemple}\pnru{jo˧pʰv̩˧ tʰv̩˧-mi˥\#}\hspace{5pt}\peng{\_ |fg{dem} |fg{clf}: that ram}\hspace{5pt}\pcmn{这头公羊}\hspace{5pt}\pfra{\_ |fg{dem} |fg{clf}: ce bélier}\end{exemple}
\end{entrée}

\begin{entrée}
{jo˩ʂwæ˩}{}{ⓔjo˩ʂwæ˩}\formedesurface{jo˩ʂwæ˩˥}\newline
\classe{名词}\ton{L}
\paradigme{\pcmn{:} \p{}}
\begin{définition}\peng{Wether (castrated ram, neutered ram).}\end{définition}
\begin{définition}\pcmn{阉羊}\end{définition}
\begin{définition}\pfra{Bélier châtré.}\end{définition}
\end{entrée}

\begin{entrée}
{jo˩zo˩}{}{ⓔjo˩zo˩}\formedesurface{jo˩zo˩˥}\newline
\classe{名词}\ton{L}
\paradigme{\pcmn{:} \p{}}
\begin{définition}\peng{Lamb.}\end{définition}
\begin{définition}\pcmn{绵羊羔}\end{définition}
\begin{définition}\pfra{Agneau.}\end{définition}
\end{entrée}

\newpage\caractère{ʝ}

\begin{entrée}
{ʝi˥}{₁}{ⓔʝi˥ⓗ1}\formedesurface{ʝi˧}\newline
\classe{名词}\ton{\#H}
1
\paradigme{\pcmn{:} \p{}}
\begin{définition}\peng{Ox.}\end{définition}
\begin{définition}\pcmn{牛}\end{définition}
\begin{définition}\pfra{Vache, boeuf.}\end{définition}
\begin{exemple}\pnru{ʝi˧-ɣɯ˥}\hspace{5pt}\peng{ox skin}\hspace{5pt}\pcmn{牛皮}\hspace{5pt}\pfra{peau de vache}\end{exemple}
\begin{exemple}\pnru{ʝi˧ tʰv̩˧-pʰo˩}\hspace{5pt}\peng{|fg{n}+|fg{dem}+|fg{clf}}\hspace{5pt}\pcmn{那头牛}\hspace{5pt}\pfra{|fg{n}+|fg{dem}+|fg{clf}}\end{exemple}
\end{entrée}

\begin{entrée}
{ʝi˥}{₂}{ⓔʝi˥ⓗ2}\formedesurface{ʝi˧}\newline
\classe{动词}\ton{H}
2\begin{définition}\peng{To do, to work.}\end{définition}
\begin{définition}\pcmn{做,工作}\end{définition}
\begin{définition}\pfra{Travailler, faire.}\end{définition}
\begin{exemple}\pnru{ɖwæ˧˥ | lo˧ ʝi˧}\hspace{5pt}\peng{hard-working, who works hard}\hspace{5pt}\pcmn{勤劳、努力}\hspace{5pt}\pfra{travailleur, assidu, qui travaille beaucoup}\end{exemple}
\begin{exemple}\pnru{ɖɯ˧-sɑ˥ | mɤ˧-ʝi˥}\hspace{5pt}\peng{to do nothing at all}\hspace{5pt}\pcmn{什么也不干}\hspace{5pt}\pfra{ne rien faire du tout}\end{exemple}
\begin{exemple}\pnru{ə˧tso˧-mɤ˧-ɲi˩ | ʝi˧-bi˧-zo˧-ho˥!}\hspace{5pt}\peng{[I/we] will have to take charge of everything / [I/we] will have to do all the work!}\hspace{5pt}\pcmn{什么都要做! / 我什么都要干(/管)!}\hspace{5pt}\pfra{Il faut tout faire! / On va devoir m'occuper de tout!}\end{exemple}
\begin{exemple}\pnru{ʈʂʰɯ˧ne-ʝi˥ | ʝi˧-zo˧-ho˥-ɲi˩!}\hspace{5pt}\peng{This is how it must be done! / This is how it is done!}\hspace{5pt}\pcmn{应该这样做的!}\hspace{5pt}\pfra{Voilà comment il faut faire! / C'est comme ça qu'on fait!}\end{exemple}
\begin{exemple}\pnru{ɑ˩ʁo˧ ʝi˧}\hspace{5pt}\peng{to take care of the household, to look after the affairs of the family; in particular: distributing work to the various members, and ensuring that the supplies are not running low}\hspace{5pt}\pcmn{管理家里的大小事情(如:分配工作、家务等)}\hspace{5pt}\pfra{gérer la maisonnée, s'occuper de la famille (tâche de la personne qui répartit les travaux à accomplir, veille aux approvisionnements…)}\end{exemple}
\end{entrée}

\begin{entrée}
{ʝi˥}{₃}{ⓔʝi˥ⓗ3}\formedesurface{ʝi˧}\newline
\classe{动词}\ton{H}
3\begin{définition}\peng{To draw.}\end{définition}
\begin{définition}\pcmn{画}\end{définition}
\begin{définition}\pfra{Dessiner, tracer.}\end{définition}
\begin{exemple}\pnru{mɤ˧-ʝi˥}\hspace{5pt}\peng{|fg{neg}}\hspace{5pt}\pcmn{不画}\hspace{5pt}\pfra{|fg{neg}}\end{exemple}
\begin{exemple}\pnru{tʰɑ˧-ʝi˥!}\hspace{5pt}\peng{|fg{prohib}}\hspace{5pt}\pcmn{别画!}\hspace{5pt}\pfra{|fg{prohib}}\end{exemple}
\begin{exemple}\pnru{ʈʂɑ˧tɑ˥ ʝi˩}\hspace{5pt}\peng{to draw a sign}\hspace{5pt}\pcmn{画一个符号}\hspace{5pt}\pfra{tracer un signe, dessiner un signe}\end{exemple}
\end{entrée}

\begin{entrée}
{ʝi˥}{₄}{ⓔʝi˥ⓗ4}\formedesurface{ʝi˧}\newline
\classe{名词}\ton{\#H}
4
\paradigme{\pcmn{:} \p{}}
\begin{définition}\peng{Earthen jar.}\end{définition}
\begin{définition}\pcmn{坛子,罐子 (陶器)}\end{définition}
\begin{définition}\pfra{Jarre en terre cuite.}\end{définition}
\end{entrée}

\begin{entrée}
{ʝi˥}{₅}{ⓔʝi˥ⓗ5}\formedesurface{ʝi˧}\newline
\classe{动词}\ton{H}
5\begin{définition}\peng{To inform, to tell.}\end{définition}
\begin{définition}\pcmn{通知、告诉}\end{définition}
\begin{définition}\pfra{Informer.}\end{définition}
\begin{exemple}\pnru{le˧-ʝi˥-ze˩}\hspace{5pt}\peng{|fg{accomp} \_ |fg{pfv}}\hspace{5pt}\pcmn{通知了}\hspace{5pt}\pfra{|fg{accomp} \_ |fg{pfv}}\end{exemple}
\begin{exemple}\pnru{qʰwæ˧ mi˧ ʝi˧}\hspace{5pt}\peng{to provide a piece of news, to provide some information}\hspace{5pt}\pcmn{告诉(一个)消息}\hspace{5pt}\pfra{donner une nouvelle}\end{exemple}
\begin{exemple}\pnru{njɤ˧ | hĩ˧-ki˧ | qʰwæ˧mi˧ ʝi˧-ze˩}\hspace{5pt}\peng{I have told people the news.}\hspace{5pt}\pcmn{我告诉了人家(那个消息)。}\hspace{5pt}\pfra{j'ai annoncé la nouvelle aux gens / j'ai annoncé une nouvelle à quelqu'un}\end{exemple}
\end{entrée}

\begin{entrée}
{ʝi˥}{₆}{ⓔʝi˥ⓗ6}\formedesurface{ʝi˧}\newline
\classe{名词}\ton{\#H}
6\begin{définition}\peng{Man, male person.}\end{définition}
\begin{définition}\pcmn{男人}\end{définition}
\begin{définition}\pfra{Homme |\stylefi{(vir)}.}\end{définition}
\end{entrée}

\begin{entrée}
{ʝi˥}{₇}{ⓔʝi˥ⓗ7}\formedesurface{ʝi˧}\newline
\classe{动词}\ton{H}
7\begin{définition}\peng{Verb of existence, for movable things.}\end{définition}
\begin{définition}\pcmn{存在动词:有(可移动物品)}\end{définition}
\begin{définition}\pfra{Verbe d'existence: choses amovibles.}\end{définition}
\begin{exemple}\pnru{ə˧tso˧-mɤ˧-ɲi˩, | le˧-ʂe˧, | le˧-ʝi˥!}\hspace{5pt}\peng{We get all sorts of things (all the necessary paraphernalia for a ritual, a feast…) and we have it (at hand for when we need it) / We get all sorts of things ready (for the ritual / the feast)!}\hspace{5pt}\pcmn{所有(的东西都)找,(就)有了 = 所有的东西都备好了}\hspace{5pt}\pfra{(En vue d'un rituel, d'une fête…) on rassemble toutes sortes de choses; on en a (sous la main)/on a fait une provision! / On prépare tout par avance (pour le rituel/la fête)!}\end{exemple}
\end{entrée}

\begin{entrée}
{ʝi˧}{₁}{ⓔʝi˧ⓗ1}\formedesurface{ʝi˧}\newline
\classe{动词}\ton{Mγ}
1\begin{définition}\peng{To come.}\end{définition}
\begin{définition}\pcmn{来}\end{définition}
\begin{définition}\pfra{Venir.}\end{définition}
\begin{exemple}\pnru{lɑ˧ ʝi˧-ze˧!}\hspace{5pt}\peng{A tiger is coming! / A tiger has come round!}\hspace{5pt}\pcmn{老虎来了!}\hspace{5pt}\pfra{Voilà le tigre! / Un tigre arrive!}\end{exemple}
\begin{exemple}\pnru{lɑ˧ le˧-ʝi˩-ze˩!}\hspace{5pt}\peng{The tiger is coming back! / The tiger is coming again!}\hspace{5pt}\pcmn{老虎又来了!}\hspace{5pt}\pfra{Voilà le tigre qui revient! / Le tigre est revenu!/ Le tigre est de retour!}\end{exemple}
\begin{exemple}\pnru{mɤ˧-ʝi˧-ze˧!}\hspace{5pt}\peng{It's going wrong! / Something is going wrong! / We're in for trouble!}\hspace{5pt}\pcmn{不好了!不行了!}\hspace{5pt}\pfra{Ca ne va plus! / C'est la catastrophe!}\end{exemple}
\end{entrée}

\begin{entrée}
{ʝi˧}{₂}{ⓔʝi˧ⓗ2}\formedesurface{ʝi˧}\newline
\classe{名词}\ton{M}
2\begin{définition}\peng{One (restricted use: only in association with /ɭɯ˧/).}\end{définition}
\begin{définition}\pcmn{一}\end{définition}
\begin{définition}\pfra{Un (numéral, à emploi restreint; ne se combine qu'avec le classificateur /ɭɯ˧/).}\end{définition}
\begin{exemple}\pnru{zo˧mv̩˥ | ʝi˧-ɭɯ˧ ʂv̩˧}\hspace{5pt}\peng{to take care of a child}\hspace{5pt}\pcmn{管一个孩子}\hspace{5pt}\pfra{s'occuper d'un enfant}\end{exemple}
\end{entrée}

\begin{entrée}
{ʝi˧β}{}{ⓔʝi˧β}\formedesurface{ʝi˧}\newline
\classe{量词}\ton{Mβ}\begin{définition}\peng{Classifier for places.}\end{définition}
\begin{définition}\pcmn{量词:地方(一个)}\end{définition}
\begin{définition}\pfra{Classificateur des lieux.}\end{définition}
\begin{exemple}\pnru{ɖɯ˧-ʝi˧}\hspace{5pt}\peng{a place, somewhere}\hspace{5pt}\pcmn{一个地方}\hspace{5pt}\pfra{un endroit; qq part}\end{exemple}
\begin{exemple}\pnru{ɖɯ˧-ʝi˧ dzi˩}\hspace{5pt}\peng{to live somewhere; to move to somewhere}\hspace{5pt}\pcmn{住在一个地方,搬家到一个地方}\hspace{5pt}\pfra{habiter quelque part; emménager quelque part/déménager vers quelque part}\end{exemple}
\begin{exemple}\pnru{ɖɯ˧-v˧ | ɖɯ˧-ʝi˧ hɯ˧}\hspace{5pt}\peng{each goes her/his own way (context: explaining that, in many families, people go to live in different cities for professional reasons)}\hspace{5pt}\pcmn{个去个的地方!/ 每个人去不同的地方!(情景:由于工作原因,一家的成员经常需要去不同的城市工作。)}\hspace{5pt}\pfra{chacun s'en va de son côté (contexte: les membres d'une famille vont habiter en des lieux différents pour raisons professionnelles)}\end{exemple}
\end{entrée}

\begin{entrée}
{ʝi˩˥}{}{ⓔʝi˩˥}\formedesurface{ʝi˩˥}\newline
\classe{名词}\ton{LH}
\paradigme{\pcmn{:} \p{}}
\begin{définition}\peng{Spot, pimple.}\end{définition}
\begin{définition}\pcmn{痘痘}\end{définition}
\begin{définition}\pfra{Bouton.}\end{définition}
\begin{exemple}\pnru{ʝi˩ tʰv̩˩˥}\hspace{5pt}\peng{to have spots, to get pimples}\hspace{5pt}\pcmn{出痘痘}\hspace{5pt}\pfra{avoir des boutons}\end{exemple}
\end{entrée}

\begin{entrée}
{ʝi˧-bv̩˧˥}{}{ⓔʝi˧-bv̩˧˥}\formedesurface{ʝi˧bv̩˧˥}\newline
\classe{名词}\ton{MH\#}
\paradigme{\pcmn{:} \p{}}
\begin{définition}\peng{Cow pen.}\end{définition}
\begin{définition}\pcmn{牛圈}\end{définition}
\begin{définition}\pfra{Étable (des vaches).}\end{définition}
\end{entrée}

\begin{entrée}
{ʝi˩bv̩˩}{₁}{ⓔʝi˩bv̩˩ⓗ1}\formedesurface{ʝi˩bv̩˩˥}\newline
\classe{名词}\ton{L}
1
\paradigme{\pcmn{:} \p{}}
\begin{définition}\peng{Pockmarked person.}\end{définition}
\begin{définition}\pcmn{麻子}\end{définition}
\begin{définition}\pfra{Grêlé.}\end{définition}
\begin{exemple}\pnru{ʝi˩bv̩˩-ʝi˧ʈv̩˩ʈv̩˩}\hspace{5pt}\peng{same meaning}\hspace{5pt}\pcmn{同上}\hspace{5pt}\pfra{même sens}\end{exemple}
\end{entrée}

\begin{entrée}
{ʝi˩bv̩˩}{₂}{ⓔʝi˩bv̩˩ⓗ2}\formedesurface{ʝi˩bv̩˩˥}\newline
\classe{名词}\ton{L}
2
\paradigme{\pcmn{:} \p{}}
\begin{définition}\peng{Bull (male).}\end{définition}
\begin{définition}\pcmn{公牛}\end{définition}
\begin{définition}\pfra{Taureau.}\end{définition}
\end{entrée}

\begin{entrée}
{ʝi˩di˩-mi˥}{}{ⓔʝi˩di˩-mi˥}\formedesurface{ʝi˩di˩mi˥}\newline
\classe{名词}\ton{L+H\#}
\paradigme{\pcmn{:} \p{}}
\begin{définition}\peng{Heifer; also used for a female pianniu: hybrid of yak and cattle.}\end{définition}
\begin{définition}\pcmn{小牝牛(包括黄牛和小母犏牛)}\end{définition}
\begin{définition}\pfra{Génisse; s'emploie pour une petite vache, aussi bien que pour le pianniu (tibétain: dzomo).}\end{définition}
\end{entrée}

\begin{entrée}
{ʝi˧kʰv̩˥}{}{ⓔʝi˧kʰv̩˥}\formedesurface{ʝi˧kʰv̩˥}\newline
\classe{代词}\ton{H\#}\begin{définition}\peng{Some, a few.}\end{définition}
\begin{définition}\pcmn{一些}\end{définition}
\begin{définition}\pfra{Certains.}\end{définition}
\begin{exemple}\pnru{hĩ˧ ʝi˧kʰv̩˥}\hspace{5pt}\peng{some people, part of the people}\hspace{5pt}\pcmn{一些人}\hspace{5pt}\pfra{certaines personnes, une partie des gens}\end{exemple}
\end{entrée}

\begin{entrée}
{ʝi˧-kʰv̩˩}{₁}{ⓔʝi˧-kʰv̩˩ⓗ1}\formedesurface{ʝi˧kʰv̩˩}\newline
\classe{名词}\ton{L\#}
1\begin{définition}\peng{Year of the Ox.}\end{définition}
\begin{définition}\pcmn{牛年}\end{définition}
\begin{définition}\pfra{Année du Bœuf.}\end{définition}
\end{entrée}

\begin{entrée}
{ʝi˧-kʰv̩˩}{₂}{ⓔʝi˧-kʰv̩˩ⓗ2}\formedesurface{ʝi˧kʰv̩˩}\newline
\classe{形容词}\ton{L\#}
2\begin{définition}\peng{Born in the year of the Ox.}\end{définition}
\begin{définition}\pcmn{属牛}\end{définition}
\begin{définition}\pfra{Né l'année du Bœuf.}\end{définition}
\end{entrée}

\begin{entrée}
{ʝi˧kʰwɤ˥\$}{}{ⓔʝi˧kʰwɤ˥\$}\formedesurface{ʝi˧kʰwɤ˥}\newline
\classe{代词}\ton{H\$}\begin{définition}\peng{A little, some.}\end{définition}
\begin{définition}\pcmn{一点}\end{définition}
\begin{définition}\pfra{Un peu.}\end{définition}
\end{entrée}

\begin{entrée}
{ʝi˧lo\#˥}{}{ⓔʝi˧lo\#˥}\formedesurface{ʝi˧lo˧}\newline
\classe{名词}\ton{\#H}\begin{définition}\peng{Attitude towards others.}\end{définition}
\begin{définition}\pcmn{态度、对待的态度}\end{définition}
\begin{définition}\pfra{Traitement (d'autrui), attitude.}\end{définition}
\begin{exemple}\pnru{ʝi˧lo˧ dʑɤ˥!}\hspace{5pt}\peng{(He/she) has a good attitude!}\hspace{5pt}\pcmn{态度积极}\hspace{5pt}\pfra{(Il / elle) a une attitude positive}\end{exemple}
\begin{exemple}\pnru{ʈʂʰɯ˧ | ʝi˧lo˧ | dʑɤ˩˥! | hĩ˧-ki˧ | dʑɤ˩-ʝi˥!}\hspace{5pt}\peng{He/she has a good attitude towards people! He/she is kind to people / does some good around him/her!}\hspace{5pt}\pcmn{他(对人)态度好!对人好/做好事!}\hspace{5pt}\pfra{Il/elle traite bien les gens! Il/elle fait de bonnes actions!}\end{exemple}
\begin{exemple}\pnru{ʝi˧lo˧ dzɑ˧}\hspace{5pt}\peng{(to have) a bad attitude: to be lazy, dissipated…}\hspace{5pt}\pcmn{态度不好}\hspace{5pt}\pfra{(avoir une) mauvaise attitude: paresseuse, dissipée…}\end{exemple}
\begin{exemple}\pnru{njɤ˧-ɳɯ˧ hɑ˧ gv̩˥, | ʝi˧lo˧ dzɑ˧!}\hspace{5pt}\peng{When I cook, I don't make a good job of it / I don't (manage to) put any heart into it / I make a mess of it!}\hspace{5pt}\pcmn{我做饭,集中不了精神 / 做的乱七八糟!}\hspace{5pt}\pfra{Quand je fais la cuisine, je ne suis pas bien concentrée/je travaille n'importe comment!}\end{exemple}
\begin{exemple}\pnru{ʈʂʰɯ˧ | ə˧tso˧ ʝi˧lo˧ ɲi˥?}\hspace{5pt}\peng{What sort of an attitude is this? (Criticism of someone who does not have a proper attitude)}\hspace{5pt}\pcmn{这是什么态度啊?(批评一个人的态度)}\hspace{5pt}\pfra{Qu'est-ce que c'est que cette attitude? (critique adressée à quelqu'un qui fait n'importe quoi)}\end{exemple}
\end{entrée}

\begin{entrée}
{ʝi˧mi˧}{}{ⓔʝi˧mi˧}\formedesurface{ʝi˧mi˧}\newline
\classe{名词}\ton{M}
\paradigme{\pcmn{:} \p{}}
\begin{définition}\peng{Jar.}\end{définition}
\begin{définition}\pcmn{坛子,罐子 (陶器)}\end{définition}
\begin{définition}\pfra{Jarre.}\end{définition}
\end{entrée}

\begin{entrée}
{ʝi˩mi˩}{}{ⓔʝi˩mi˩}\formedesurface{ʝi˩mi˩˥}\newline
\classe{名词}\ton{L}
\paradigme{\pcmn{:} \p{}}
\begin{définition}\peng{Cow (female).}\end{définition}
\begin{définition}\pcmn{母牛}\end{définition}
\begin{définition}\pfra{Vache.}\end{définition}
\begin{exemple}\pnru{ʝi˩mi˩-ʐɤ˥qo˩}\hspace{5pt}\peng{cow and calf}\hspace{5pt}\pcmn{母牛与小牛}\hspace{5pt}\pfra{vache et veau}\end{exemple}
\end{entrée}

\begin{entrée}
{ʝi˩næ˩-se˧}{}{ⓔʝi˩næ˩-se˧}\formedesurface{ʝi˩næ˩se˧}\newline
\classe{名词}\ton{L-M}\begin{définition}\peng{Kunming, and the Eastern part of the province of Yunnan.}\end{définition}
\begin{définition}\pcmn{云南,昆明……}\end{définition}
\begin{définition}\pfra{Kunming et la partie orientale du Yunnan, une fois passés Lijiang et Dali.}\end{définition}
\begin{exemple}\pnru{sɯ˧pʰi˧ | ʝi˩næ˩-se˧-qo˧ hɯ˧-ɲi˥!}\hspace{5pt}\peng{The (feudal) lord has gone to Kunming!}\hspace{5pt}\pcmn{土司到昆明去了!}\hspace{5pt}\pfra{Le seigneur est parti à Kunming!}\end{exemple}
\end{entrée}

\begin{entrée}
{ʝi˩ŋɤ˧˥}{}{ⓔʝi˩ŋɤ˧˥}\formedesurface{ʝi˩ŋɤ˧˥}\newline
\classe{动词}\ton{LM+MH\#}\begin{définition}\peng{To bend (one's body) backwards.}\end{définition}
\begin{définition}\pcmn{往后仰}\end{définition}
\begin{définition}\pfra{Se courber vers l’arrière.}\end{définition}
\begin{exemple}\pnru{ʝi˩ŋɤ˧-ze˥}\hspace{5pt}\peng{|fg{pfv}}\hspace{5pt}\pcmn{往后仰了}\hspace{5pt}\pfra{|fg{pfv}}\end{exemple}
\begin{exemple}\pnru{ʝi˩ŋɤ˧˥ | tʰi˧-dzi˩}\hspace{5pt}\peng{to be seated leaning backwards, to lean against the back of one's seat}\hspace{5pt}\pcmn{坐着往后仰}\hspace{5pt}\pfra{être assis en s'inclinant vers l'arrière}\end{exemple}
\end{entrée}

\begin{entrée}
{ʝi˧pʰv̩\#˥}{}{ⓔʝi˧pʰv̩\#˥}\formedesurface{ʝi˧pʰv̩˧}\newline
\classe{名词}\ton{\#H}
\paradigme{\pcmn{:} \p{}}
\begin{définition}\peng{Male ox, bull.}\end{définition}
\begin{définition}\pcmn{公牛}\end{définition}
\begin{définition}\pfra{Taureau.}\end{définition}
\begin{exemple}\pnru{ʝi˧pʰv̩˧ tʰv̩˧-mi˥\#}\hspace{5pt}\peng{|fg{n}+|fg{dem}+|fg{clf}}\hspace{5pt}\pcmn{那头公牛}\hspace{5pt}\pfra{|fg{n}+|fg{dem}+|fg{clf}}\end{exemple}
\begin{exemple}\pnru{ʝi˧pʰv̩˧ tʰv̩˧-ɭɯ\#˥}\hspace{5pt}\peng{|fg{n}+|fg{dem}+|fg{clf.animaux}}\hspace{5pt}\pcmn{那头公牛}\hspace{5pt}\pfra{|fg{n}+|fg{dem}+|fg{clf.animaux}}\end{exemple}
\end{entrée}

\begin{entrée}
{ʝi˧qv̩˥}{}{ⓔʝi˧qv̩˥}\formedesurface{ʝi˧qv̩˥}\newline
\classe{名词}\ton{H\#}
\paradigme{\pcmn{:} \p{}}
\begin{définition}\peng{Neck strap: a part of the buffalo's harness for ploughing: a strap that fastens the yoke.}\end{définition}
\begin{définition}\pcmn{轭的一个部分,将牛轭安在牛的脖子上}\end{définition}
\begin{définition}\pfra{Collier: une partie du harnais utilisé pour les labours, qui maintient le joug en place; cette corde passe sous le cou de l'animal, et est fixée au joug.}\end{définition}
\end{entrée}

\begin{entrée}
{ʝi˩qʰv̩˩}{}{ⓔʝi˩qʰv̩˩}\formedesurface{ʝi˩qʰv̩˩˥}\newline
\classe{名词}\ton{L}
\paradigme{\pcmn{:} \p{}}
\begin{définition}\peng{Sleeve.}\end{définition}
\begin{définition}\pcmn{袖子}\end{définition}
\begin{définition}\pfra{Manche.}\end{définition}
\end{entrée}

\begin{entrée}
{ʝi˧ʁæ˥}{}{ⓔʝi˧ʁæ˥}\formedesurface{ʝi˧ʁæ˥}\newline
\classe{名词}\ton{H\#}
\paradigme{\pcmn{:} \p{}}
\begin{définition}\peng{Cow, beef.}\end{définition}
\begin{définition}\pcmn{黄牛}\end{définition}
\begin{définition}\pfra{Vache, boeuf.}\end{définition}
\end{entrée}

\begin{entrée}
{ʝi˧ʁo\#˥}{}{ⓔʝi˧ʁo\#˥}\formedesurface{ʝi˧ʁo˧}\newline
\classe{形容词}\ton{\#H}
\étymologie{
ʝi˥ 1; ʁo˧ 2
}\begin{définition}\peng{Able, capable, able-bodied.}\end{définition}
\begin{définition}\pcmn{能干、不缺劳力}\end{définition}
\begin{définition}\pfra{Capable; littéralement ‘qui sait faire’.}\end{définition}
\begin{exemple}\pnru{ʈʂʰɯ˧ | ʝi˧ʁo˧-hĩ˧ | ɖɯ˧-v̩˧ ɲi˩}\hspace{5pt}\peng{It's an able/competent person.}\hspace{5pt}\pcmn{他是一个能干/称职的人。}\hspace{5pt}\pfra{c'est quelqu'un d'habile/de capable}\end{exemple}
\begin{exemple}\pnru{ʝi˧ʁo˧-zo˥}\hspace{5pt}\peng{a competent lad, a capable fellow}\hspace{5pt}\pcmn{一个能干的男人}\hspace{5pt}\pfra{un homme capable/habile, un gaillard compétent}\end{exemple}
\begin{exemple}\pnru{ʝi˧ʁo˧ ɲi˥}\hspace{5pt}\peng{|fg{cop}}\hspace{5pt}\pcmn{|fg{cop}}\hspace{5pt}\pfra{|fg{cop}}\end{exemple}
\end{entrée}

\begin{entrée}
{ʝi˧sɑ˧}{}{ⓔʝi˧sɑ˧}\formedesurface{ʝi˧sɑ˧}\newline
\classe{名词}\ton{M}
\paradigme{\pcmn{:} \p{}}
\begin{définition}\peng{Umbrella.}\end{définition}
\begin{définition}\pcmn{雨伞}\end{définition}
\begin{définition}\pfra{Parapluie (emprunt).}\end{définition}
\end{entrée}

\begin{entrée}
{ʝi˧se˧}{₁}{ⓔʝi˧se˧ⓗ1}\formedesurface{ʝi˧se˧}\newline
\classe{名词}\ton{M}
1\begin{définition}\peng{Doctor.}\end{définition}
\begin{définition}\pcmn{医生(汉语借词)}\end{définition}
\begin{définition}\pfra{Médecin.}\end{définition}
\end{entrée}

\begin{entrée}
{ʝi˧se˧}{₂}{ⓔʝi˧se˧ⓗ2}\formedesurface{ʝi˧se˧}\newline
\classe{形容词}\ton{M}
2\begin{définition}\peng{Wild (as opposed to: cultivated; e.g. wild plants, wild animals).}\end{définition}
\begin{définition}\pcmn{野生(汉语借词)}\end{définition}
\begin{définition}\pfra{Sauvage, spontané: plantes qui poussent spontanément (par opposition aux plantes cultivées), animaux sauvages.}\end{définition}
\begin{exemple}\pnru{ʝi˧se˧-hĩ˧}\hspace{5pt}\peng{\_ |fg{rel}/|fg{nmlz}}\hspace{5pt}\pcmn{野生的}\hspace{5pt}\pfra{\_ |fg{rel}/|fg{nmlz}}\end{exemple}
\end{entrée}

\begin{entrée}
{ʝi˧sɯ˥}{}{ⓔʝi˧sɯ˥}\formedesurface{ʝi˧sɯ˥}\newline
\classe{名词}\ton{H\#}\begin{définition}\peng{Meaning, sense.}\end{définition}
\begin{définition}\pcmn{意思(汉语借词)}\end{définition}
\begin{définition}\pfra{Signification, sens.}\end{définition}
\end{entrée}

\begin{entrée}
{ʝi˧ʂæ˧tsʰɤ˩}{}{ⓔʝi˧ʂæ˧tsʰɤ˩}\formedesurface{ʝi˧ʂæ˧tsʰɤ˩}\newline
\classe{名词}\ton{L\#}\begin{définition}\peng{A wild radish that grows on the mountains; it is edible; it is picked and eaten in the Spring, when vegetables are not ripe yet. Yi people harvest it and sell it in the plain.}\end{définition}
\begin{définition}\pcmn{红萝卜菜(汉语借词:野山菜):一种山上的野菜。春天的时候,菜园的蔬菜还没有成熟的时候,永宁的人吃红萝卜菜。彝族人从高山上采下来,在永宁卖。}\end{définition}
\begin{définition}\pfra{Radis sauvage qui pousse en montagne; on le consomme surtout au printemps, à une époque où il n'y a pas encore de légumes. Ce radis est récoltée par les Yi et vendu dans la plaine.}\end{définition}
\end{entrée}

\begin{entrée}
{ʝi˧ʂɯ˥}{}{ⓔʝi˧ʂɯ˥}\formedesurface{ʝi˧ʂɯ˥}\newline
\classe{名词}\ton{H\#}\begin{définition}\peng{Masculine given name.}\end{définition}
\begin{définition}\pcmn{男性名字}\end{définition}
\begin{définition}\pfra{Prénom masculin.}\end{définition}
\end{entrée}

\begin{entrée}
{ʝi˧tɕi˧}{}{ⓔʝi˧tɕi˧}\formedesurface{ʝi˧tɕi˧}\newline
\classe{名词}\ton{M}\begin{définition}\peng{Feminine given name.}\end{définition}
\begin{définition}\pcmn{女性名字}\end{définition}
\begin{définition}\pfra{Prénom féminin.}\end{définition}
\end{entrée}

\begin{entrée}
{ʝi˧tɕi˧-ɖɯ˩mɑ˩}{}{ⓔʝi˧tɕi˧-ɖɯ˩mɑ˩}\formedesurface{ʝi˧tɕi˧ɖɯ˩mɑ˩}\newline
\classe{名词}\ton{-L}\begin{définition}\peng{Feminine given name.}\end{définition}
\begin{définition}\pcmn{女性名字}\end{définition}
\begin{définition}\pfra{Prénom féminin.}\end{définition}
\end{entrée}

\begin{entrée}
{ʝi˧tɕi˧-ɬɑ˩mv̩˩}{}{ⓔʝi˧tɕi˧-ɬɑ˩mv̩˩}\formedesurface{ʝi˧tɕi˧ɬɑ˩mv̩˩}\newline
\classe{名词}\ton{-L}\begin{définition}\peng{Feminine given name.}\end{définition}
\begin{définition}\pcmn{女性名字}\end{définition}
\begin{définition}\pfra{Prénom féminin.}\end{définition}
\end{entrée}

\begin{entrée}
{ʝi˧tsɯ˧}{}{ⓔʝi˧tsɯ˧}\formedesurface{ʝi˧tsɯ˧}\newline
\classe{名词}\ton{M}
\paradigme{\pcmn{:} \p{}}
\begin{définition}\peng{Chair (borrowing).}\end{définition}
\begin{définition}\pcmn{椅子}\end{définition}
\begin{définition}\pfra{Chaise (emprunt).}\end{définition}
\end{entrée}

\begin{entrée}
{ʝi˩ʈʂæ˧˥}{}{ⓔʝi˩ʈʂæ˧˥}\formedesurface{ʝi˩ʈʂæ˧˥}\newline
\classe{名词}\ton{LM+MH\#}
\paradigme{\pcmn{:} \p{}}
\begin{définition}\peng{Waist.}\end{définition}
\begin{définition}\pcmn{腰}\end{définition}
\begin{définition}\pfra{Taille.}\end{définition}
\end{entrée}

\begin{entrée}
{ʝi˧ʈʂʰe˥-mi˩}{}{ⓔʝi˧ʈʂʰe˥-mi˩}\formedesurface{ʝi˧ʈʂʰe˥mi˩}\newline
\classe{名词}\ton{H\#-}\begin{définition}\peng{South.}\end{définition}
\begin{définition}\pcmn{南方}\end{définition}
\begin{définition}\pfra{Sud.}\end{définition}
\begin{exemple}\pnru{ʝi˧ʈʂʰe˥mi˩-gi˩dzɤ˩ se˩}\hspace{5pt}\peng{to walk towards the south}\hspace{5pt}\pcmn{往南方走}\hspace{5pt}\pfra{marcher en direction du sud}\end{exemple}
\end{entrée}

\begin{entrée}
{ʝi˧zo\#˥}{}{ⓔʝi˧zo\#˥}\formedesurface{ʝi˧zo˧}\newline
\classe{名词}\ton{\#H}
\paradigme{\pcmn{:} \p{}}
\begin{définition}\peng{Calf.}\end{définition}
\begin{définition}\pcmn{小牛}\end{définition}
\begin{définition}\pfra{Veau.}\end{définition}
\begin{exemple}\pnru{ʝi˧zo˧ tʰv̩˧-ɭɯ\#˥}\hspace{5pt}\peng{|fg{n}+|fg{dem}+|fg{clf}}\hspace{5pt}\pcmn{那头小牛}\hspace{5pt}\pfra{|fg{n}+|fg{dem}+|fg{clf}}\end{exemple}
\end{entrée}

\newpage\caractère{k}

\begin{entrée}
{kæ˧ʈʂe˧}{}{ⓔkæ˧ʈʂe˧}\formedesurface{kæ˧ʈʂe˧}\newline
\classe{名词}\ton{M}\begin{définition}\peng{Acupuncture needles; acupuncture.}\end{définition}
\begin{définition}\pcmn{针灸(汉语借词:干针)}\end{définition}
\begin{définition}\pfra{Acupuncture.}\end{définition}
\begin{exemple}\pnru{kæ˧ʈʂe˧ lɑ˧˥}\hspace{5pt}\peng{to do an acupuncture session, to use acupuncture needles}\hspace{5pt}\pcmn{扎针灸}\hspace{5pt}\pfra{faire une séance d'acupuncture, placer des aiguilles d'acupuncture}\end{exemple}
\end{entrée}

\begin{entrée}
{kæ˧ʈʂɯ˧}{}{ⓔkæ˧ʈʂɯ˧}\formedesurface{kæ˧ʈʂɯ˧}\newline
\classe{名词}\ton{M}\begin{définition}\peng{Sugar cane.}\end{définition}
\begin{définition}\pcmn{甘蔗}\end{définition}
\begin{définition}\pfra{Canne à sucre.}\end{définition}
\end{entrée}

\begin{entrée}
{kɤ˧˥}{}{ⓔkɤ˧˥}\formedesurface{kɤ˧˥}\newline
\classe{动词}\ton{MH}\begin{définition}\peng{To knock on the door.}\end{définition}
\begin{définition}\pcmn{敲门}\end{définition}
\begin{définition}\pfra{Frapper à la porte, heurter à la porte.}\end{définition}
\begin{exemple}\pnru{tʰi˧-kɤ˧˥}\hspace{5pt}\peng{|fg{dur}}\hspace{5pt}\pcmn{|fg{dur}}\hspace{5pt}\pfra{|fg{dur}}\end{exemple}
\end{entrée}

\begin{entrée}
{kɤ˧˥α}{₁}{ⓔkɤ˧˥αⓗ1}\formedesurface{ɖɯ˧ kɤ˧˥}\newline
\classe{量词}\ton{MHα}
1\begin{définition}\peng{Classifier for sticks/rods.}\end{définition}
\begin{définition}\pcmn{量词:棍子、树枝(一根)}\end{définition}
\begin{définition}\pfra{Classificateur des bâtons.}\end{définition}
\begin{exemple}\pnru{si˧-kɤ˧˥ | ɖɯ˧-kɤ˧˥}\hspace{5pt}\peng{a branch (of a tree)}\hspace{5pt}\pcmn{一根树枝}\hspace{5pt}\pfra{une branche (d'arbre)}\end{exemple}
\end{entrée}

\begin{entrée}
{kɤ˧˥α}{₂}{ⓔkɤ˧˥αⓗ2}\formedesurface{ɖɯ˧ kɤ˧˥}\newline
\classe{量词}\ton{MHα}
2\begin{définition}\peng{A tract of land.}\end{définition}
\begin{définition}\pcmn{量词:地(一片)}\end{définition}
\begin{définition}\pfra{Une étendue de terre.}\end{définition}
\end{entrée}

\begin{entrée}
{kɤ˩}{}{ⓔkɤ˩}\formedesurface{kɤ˧}\newline
\classe{名词}\ton{L}
\paradigme{\pcmn{:} \p{}}
\begin{définition}\peng{Bottle.}\end{définition}
\begin{définition}\pcmn{瓶子}\end{définition}
\begin{définition}\pfra{Bouteille.}\end{définition}
\begin{exemple}\pnru{ʐɯ˧-kɤ˩}\hspace{5pt}\peng{wine bottle}\hspace{5pt}\pcmn{酒瓶}\hspace{5pt}\pfra{bouteille d'alcool}\end{exemple}
\begin{exemple}\pnru{ʐɯ˧ ɖɯ˧-kɤ˩}\hspace{5pt}\peng{one bottle of wine}\hspace{5pt}\pcmn{一瓶酒}\hspace{5pt}\pfra{une bouteille d'alcool}\end{exemple}
\end{entrée}

\begin{entrée}
{kɤ˩˧}{}{ⓔkɤ˩˧}\formedesurface{kɤ˩˥}\newline
\classe{名词}\ton{LM}
\paradigme{\pcmn{:} \p{}}
\begin{définition}\peng{Falcon.}\end{définition}
\begin{définition}\pcmn{隼、“小鹰”}\end{définition}
\begin{définition}\pfra{Buse, faucon.}\end{définition}
\begin{exemple}\pnru{kɤ˩ hwæ˧-ze˧}\hspace{5pt}\peng{…bought (a) falcon}\hspace{5pt}\pcmn{买了隼}\hspace{5pt}\pfra{…a acheté (un) faucon}\end{exemple}
\begin{exemple}\pnru{kɤ˩ dzɯ˧-ze˩}\hspace{5pt}\peng{…ate (a) falcon}\hspace{5pt}\pcmn{吃了隼}\hspace{5pt}\pfra{…a mangé (un) faucon}\end{exemple}
\end{entrée}

\begin{entrée}
{kɤ˩α}{}{ⓔkɤ˩α}\formedesurface{ɖɯ˧ kɤ˩}\newline
\classe{量词}\ton{Lα}\begin{définition}\peng{A bottle of.}\end{définition}
\begin{définition}\pcmn{量词:瓶}\end{définition}
\begin{définition}\pfra{Classificateur des bouteilles.}\end{définition}
\begin{exemple}\pnru{kɤ˩zo˩˥}\hspace{5pt}\peng{small bottle}\hspace{5pt}\pcmn{一小瓶}\hspace{5pt}\pfra{petite bouteille}\end{exemple}
\begin{exemple}\pnru{ʈʂʰɯ˧-kɤ˥}\hspace{5pt}\peng{|fg{dem} \_ (tone: H\# / H\$)}\hspace{5pt}\pcmn{指示代词 \_}\hspace{5pt}\pfra{|fg{dem} \_ (tone: H\# / H\$)}\end{exemple}
\end{entrée}

\begin{entrée}
{kɤ˩β}{}{ⓔkɤ˩β}\formedesurface{kɤ˩˥}\newline
\classe{动词}\ton{Lβ}\begin{définition}\peng{To saw.}\end{définition}
\begin{définition}\pcmn{锯(木头)}\end{définition}
\begin{définition}\pfra{Scier (du bois).}\end{définition}
\begin{exemple}\pnru{si˧ | ɖɯ˧-pʰæ˧ kɤ˥}\hspace{5pt}\peng{to saw a piece of wood}\hspace{5pt}\pcmn{锯一块木头}\hspace{5pt}\pfra{scier un morceau de bois}\end{exemple}
\end{entrée}

\begin{entrée}
{kɤ˧dzi˧}{}{ⓔkɤ˧dzi˧}\formedesurface{kɤ˧dzi˧}\newline
\classe{动词}\ton{M}\begin{définition}\peng{To take a seat, to get seated.}\end{définition}
\begin{définition}\pcmn{坐下(在饭桌)}\end{définition}
\begin{définition}\pfra{Prendre place (lors d'un repas, d'une cérémonie…).}\end{définition}
\begin{exemple}\pnru{ɑ˩ʁo˧-hĩ˧ | kɤ˧dzi˧-ze˧.}\hspace{5pt}\peng{The members of the family took their seats / got seated.}\hspace{5pt}\pcmn{家人入座了。}\hspace{5pt}\pfra{Les gens de la famille se sont assis/ont pris place.}\end{exemple}
\end{entrée}

\begin{entrée}
{kɤ˧kɤ˩}{}{ⓔkɤ˧kɤ˩}\formedesurface{kɤ˧kɤ˩}\newline
\classe{助词}\ton{L\#}\begin{définition}\peng{Next to, close to.}\end{définition}
\begin{définition}\pcmn{挨着(坐……)}\end{définition}
\begin{définition}\pfra{Proche de, à côté de.}\end{définition}
\begin{exemple}\pnru{(tso˧∼tso˧) kɤ˧kɤ˩ | tʰi˧-tɕɯ˥}\hspace{5pt}\peng{to arrange, to put in good order}\hspace{5pt}\pcmn{摆整齐、使均匀,如:一排排挨着}\hspace{5pt}\pfra{ranger des choses en bon ordre}\end{exemple}
\begin{exemple}\pnru{kɤ˧kɤ˩ | tʰi˧-se˥}\hspace{5pt}\peng{to walk in a line, one behind the other}\hspace{5pt}\pcmn{并排走}\hspace{5pt}\pfra{marcher en file indienne}\end{exemple}
\begin{exemple}\pnru{kɤ˧kɤ˩ | tʰi˧-dzi˩}\hspace{5pt}\peng{to sit close to one another}\hspace{5pt}\pcmn{挨着坐}\hspace{5pt}\pfra{être assis les uns à côté des autres, proches les uns des autres}\end{exemple}
\end{entrée}

\begin{entrée}
{kɤ˩∼kɤ˧˥}{}{ⓔkɤ˩∼kɤ˧˥}\formedesurface{kɤ˩kɤ˧˥}\newline
\classe{动词}\ton{L+MH\#}\begin{définition}\peng{To knock, to tap, to poke.}\end{définition}
\begin{définition}\pcmn{敲、拍}\end{définition}
\begin{définition}\pfra{Tapoter.}\end{définition}
\begin{exemple}\pnru{kʰi˧ kɤ˥∼kɤ˩}\hspace{5pt}\peng{to knock at the door}\hspace{5pt}\pcmn{敲门}\hspace{5pt}\pfra{frapper à la porte}\end{exemple}
\begin{exemple}\pnru{njɤ˧-ɳɯ˧ | no˧ | kɤ˩∼kɤ˧-bi˥!}\hspace{5pt}\peng{I am going to slap your buttocks! (An adult threatens a child.)}\hspace{5pt}\pcmn{我要打你屁股了!(大人对孩子说)}\hspace{5pt}\pfra{Je vais te donner une tape/une fessée! (Menace d'un adulte à un enfant)}\end{exemple}
\begin{exemple}\pnru{ʈʂo˧tsɯ˥ kɤ˩∼kɤ˩ (-ze˩/-bi˩)}\hspace{5pt}\peng{to tap the table, to rap on the table}\hspace{5pt}\pcmn{拍拍桌子}\hspace{5pt}\pfra{heurter la table, taper sur la table}\end{exemple}
\begin{exemple}\pnru{gv̩˧dv̩˧ kɤ˧∼kɤ˩}\hspace{5pt}\peng{to tap someone's back (to relieve back pain)}\hspace{5pt}\pcmn{敲敲背}\hspace{5pt}\pfra{tapoter sur le dos de quelqu'un (pour soulager un mal de dos)}\end{exemple}
\end{entrée}

\begin{entrée}
{kɤ˧ljɤ˩}{}{ⓔkɤ˧ljɤ˩}\formedesurface{kɤ˧ljɤ˩}\newline
\classe{名词}\ton{L\#}\begin{définition}\peng{Chinese sorghum.}\end{définition}
\begin{définition}\pcmn{高粱(汉语借词)}\end{définition}
\begin{définition}\pfra{Sorgho, gaoliang; céréale dont on se sert pour faire du vin.}\end{définition}
\end{entrée}

\begin{entrée}
{kɤ˩lo˧˥}{}{ⓔkɤ˩lo˧˥}\formedesurface{kɤ˩lo˧˥}\newline
\classe{名词}\ton{LM+MH\#}
\paradigme{\pcmn{:} \p{}}
\begin{définition}\peng{Branch.}\end{définition}
\begin{définition}\pcmn{树枝}\end{définition}
\begin{définition}\pfra{Branche.}\end{définition}
\begin{exemple}\pnru{si˧dzi˩-kɤ˩lo˩}\hspace{5pt}\peng{branch of tree}\hspace{5pt}\pcmn{树枝}\hspace{5pt}\pfra{branche d'arbre}\end{exemple}
\begin{exemple}\pnru{si˧-kɤ˥lo˩}\hspace{5pt}\peng{as above}\hspace{5pt}\pcmn{同上}\hspace{5pt}\pfra{idem}\end{exemple}
\end{entrée}

\begin{entrée}
{kɤ˧mi˧}{₁}{ⓔkɤ˧mi˧ⓗ1}\formedesurface{kɤ˧mi˧}\newline
\classe{名词}\ton{M}
1
\paradigme{\pcmn{:} \p{}}
\begin{définition}\peng{Female falcon.}\end{définition}
\begin{définition}\pcmn{母隼}\end{définition}
\begin{définition}\pfra{Faucon femelle.}\end{définition}
\begin{exemple}\pnru{kɤ˩mi˩-kɤ˩pʰv̩˥}\hspace{5pt}\peng{female falcon and male falcon}\hspace{5pt}\pcmn{母隼与公隼}\hspace{5pt}\pfra{faucon femelle et faucon mâle}\end{exemple}
\end{entrée}

\begin{entrée}
{kɤ˧mi˧}{₂}{ⓔkɤ˧mi˧ⓗ2}\formedesurface{kɤ˧mi˧}\newline
\classe{名词}\ton{M}
2
\paradigme{\pcmn{:} \p{}}
\begin{définition}\peng{A large jar; a large bottle.}\end{définition}
\begin{définition}\pcmn{大坛子,大瓶}\end{définition}
\begin{définition}\pfra{Grande jarre; grande bouteille.}\end{définition}
\end{entrée}

\begin{entrée}
{kɤ˧mv̩˧˥}{}{ⓔkɤ˧mv̩˧˥}\formedesurface{kɤ˧mv̩˧˥}\newline
\classe{名词}\ton{MH\#}\begin{définition}\peng{The Gemu mountain (Yongning).}\end{définition}
\begin{définition}\pcmn{格母山}\end{définition}
\begin{définition}\pfra{La montagne Gemu (Yongning).}\end{définition}
\begin{exemple}\pnru{ɬi˧di˩-kɤ˩mv̩˩}\hspace{5pt}\peng{Mount Gemu, in Yongning}\hspace{5pt}\pcmn{永宁格姆山}\hspace{5pt}\pfra{la montagne Gemu de Yongning}\end{exemple}
\begin{exemple}\pnru{kɤ˧mv̩˧-hæ̃˧kʰo˥}\hspace{5pt}\peng{the Gemu princess: another name for Mount Gemu, considered as a female deity}\hspace{5pt}\pcmn{格姆公主:格姆山别名(格姆山被看作女神)}\hspace{5pt}\pfra{«la princesse Gemu»; autre nom de la montagne Gemu, considérée comme une divinité féminine}\end{exemple}
\begin{exemple}\pnru{kɤ˧mv̩˧˥, | æ˧ʂæ˧, | ŋwɤ˧hɑ̃˩, | ʂwæ˧gv̩\#˥, | nɑ˩tsʰi˩˥ | -tɕʰɤ˧pɤ˧mi\#˥, | qv̩˧ɻ̍˧-ʈʂʰɑ˧nɑ˥ |}\hspace{5pt}\peng{The six mountains of Yongning that carry a name and have a definite symbolic value. The other mountains do not have comparable symbolic value, and fewer people use specific names for them.}\hspace{5pt}\pcmn{永宁地区有固定名字的六座山:格姆,安山,瓦哈,双古,纳慈巧吧咪,古尔川纳。}\hspace{5pt}\pfra{Les six montagnes de Yongning qui portent un nom. Les autres sommets du voisinage n'ont pas une valeur symbolique comparable, et ne portent pas de nom communément utilisé.}\end{exemple}
\end{entrée}

\begin{entrée}
{kɤ˩-nɑ˧mi˧}{}{ⓔkɤ˩-nɑ˧mi˧}\formedesurface{kɤ˩nɑ˧mi˧}\newline
\classe{名词}\ton{L-}
\paradigme{\pcmn{:} \p{}}
\begin{définition}\peng{Eagle.}\end{définition}
\begin{définition}\pcmn{老鹰}\end{définition}
\begin{définition}\pfra{Aigle.}\end{définition}
\end{entrée}

\begin{entrée}
{kɤ˩pʰv̩˩}{}{ⓔkɤ˩pʰv̩˩}\formedesurface{kɤ˩pʰv̩˩˥}\newline
\classe{名词}\ton{L}
\paradigme{\pcmn{:} \p{}}
\begin{définition}\peng{Male falcon.}\end{définition}
\begin{définition}\pcmn{公隼}\end{définition}
\begin{définition}\pfra{Faucon mâle.}\end{définition}
\begin{exemple}\pnru{kɤ˩pʰv̩˩-kɤ˩mi˥}\hspace{5pt}\peng{male falcon and female falcon}\hspace{5pt}\pcmn{公隼与母隼}\hspace{5pt}\pfra{faucon mâle et faucon femelle}\end{exemple}
\end{entrée}

\begin{entrée}
{kɤ˩-tjɤ˧ljɤ\#˥}{}{ⓔkɤ˩-tjɤ˧ljɤ\#˥}\formedesurface{kɤ˩tjɤ˧ljɤ˧}\newline
\classe{名词}\ton{L-\#H}
\paradigme{\pcmn{:} \p{}}
\begin{définition}\peng{Small bell hung to an animal's neck (e.g. horse's bell).}\end{définition}
\begin{définition}\pcmn{铃铛:如,挂在妈的脖子上的铃铛}\end{définition}
\begin{définition}\pfra{Clochette s'accrochant autour du cou (ex.: clochette d'un cheval).}\end{définition}
\end{entrée}

\begin{entrée}
{kɤ˧˥tʰɑ˩}{}{ⓔkɤ˧˥tʰɑ˩}\formedesurface{kɤ˧˥tʰɑ˩}\newline
\classe{名词}\ton{MH+L}\begin{définition}\peng{A family name from Yongning. There are two families in Yongning that carry this name. This is one of the first three clans who settled in the vicinity of the Yongning monastery, the other two being \stylefv{/ə}˧lɑ˧/ and \stylefv{/lɑ}˧tʰɑ˧mi˥\$/.}\end{définition}
\begin{définition}\pcmn{一个姓。这个姓,永宁有两家}\end{définition}
\begin{définition}\pfra{Nom de clan/famille étendue. Deux familles portent ce nom à Yongning. C'est l'un des trois premiers clans à s'être établis à proximité du monastère de Yongning, les deux autres étant \stylefv{/ə}˧lɑ˧/ et \stylefv{/lɑ}˧tʰɑ˧mi˥\$/.}\end{définition}
\begin{exemple}\pnru{kɤ˧˥tʰɑ˩=ɻ̍˩}\hspace{5pt}\peng{the /kɤ˧˥tʰɑ˩/ clan}\hspace{5pt}\pcmn{|fv{/kɤ˧˥tʰɑ˩/}家族}\hspace{5pt}\pfra{le clan /kɤ˧˥tʰɑ˩/}\end{exemple}
\end{entrée}

\begin{entrée}
{kɤ˧ʈʂɯ˩}{₁}{ⓔkɤ˧ʈʂɯ˩ⓗ1}\formedesurface{kɤ˧ʈʂɯ˩}\newline
\classe{动词}\ton{L\#}
1\begin{définition}\peng{To tell.}\end{définition}
\begin{définition}\pcmn{讲}\end{définition}
\begin{définition}\pfra{Parler, raconter.}\end{définition}
\begin{exemple}\pnru{hĩ˧-ki˧ | tʰɑ˧-kɤ˧ʈʂɯ˩!}\hspace{5pt}\peng{Don't tell it! / Don't tell anyone!}\hspace{5pt}\pcmn{不要告诉人家!}\hspace{5pt}\pfra{il ne faut pas le dire aux gens! / c'est secret!}\end{exemple}
\begin{exemple}\pnru{kɤ˧-tʰɑ˥-ʈʂɯ˩!}\hspace{5pt}\peng{Don't tell it! / Don't tell anyone!}\hspace{5pt}\pcmn{不要告诉人家!}\hspace{5pt}\pfra{il ne faut pas le dire! / c'est secret!}\end{exemple}
\begin{exemple}\pnru{njɤ˧-ɳɯ˧ | kɤ˧ʈʂɯ˩-bi˩!}\hspace{5pt}\peng{I'm going to (jump in and) say something!}\hspace{5pt}\pcmn{我要说一点事情!}\hspace{5pt}\pfra{je vais intervenir/je vais dire quelque chose!}\end{exemple}
\begin{exemple}\pnru{no˧ | kɤ˧ʈʂɯ˩ dʑo˩-ɲi˩!}\hspace{5pt}\peng{You have to say something!}\hspace{5pt}\pcmn{你得说话啊!}\hspace{5pt}\pfra{il faut que tu dises quelque chose!}\end{exemple}
\begin{exemple}\pnru{ʈʂʰɯ˧ | kɤ˧ʈʂɯ˩ | dʑɤ˩˥ | mɤ˧-mv̩˧-sɯ˥! / ʈʂʰɯ˧ | kɤ˧ʈʂɯ˩ dʑɤ˩˥ | mɤ˧-mv̩˧∼mv̩˧-sɯ˥!}\hspace{5pt}\peng{She does not really understand yet! (About a toddler aged 2 who does not yet speak distinctly or follow conversations)}\hspace{5pt}\pcmn{她还不怎么听得懂话!(关于一个不会说话的两岁小孩)}\hspace{5pt}\pfra{elle ne comprend pas encore grand'chose à ce qu'on dit! / elle ne sait pas encore comprendre ce qu'on dit! (au sujet d'une fillette de moins de 2 ans qui ne parle pas encore)}\end{exemple}
\begin{exemple}\pnru{tʰɑ˧-kɤ˧ʈʂɯ˩! | hĩ˧ ɳv̩˧ tʰɑ˧-kʰɯ˩!}\hspace{5pt}\peng{Don't talk about it! Don't let people know!}\hspace{5pt}\pcmn{不要告诉(人家)!别让人家知道!}\hspace{5pt}\pfra{N'en parle pas! il ne faut pas que les gens le sachent!}\end{exemple}
\begin{exemple}\pnru{hĩ˧-ki˧ | kɤ˧-mɤ˧-ʈʂɯ˩}\hspace{5pt}\peng{not to tell people; (to do something secretly) without telling anyone}\hspace{5pt}\pcmn{不跟人家说(自己做什么事)}\hspace{5pt}\pfra{(faire quelque chose) en cachette, sans le dire à personne}\end{exemple}
\begin{exemple}\pnru{kɤ˧ʈʂɯ˩ ɲi˩}\hspace{5pt}\peng{well-behaved, obedient (child) (literally: who listens to what (s)he is told)}\hspace{5pt}\pcmn{听话,乖(来形容一个孩子)}\hspace{5pt}\pfra{sage (au sujet d'un enfant) (littéralement: qui écoute ce qu'on lui dit)}\end{exemple}
\end{entrée}

\begin{entrée}
{kɤ˧ʈʂɯ˩}{₂}{ⓔkɤ˧ʈʂɯ˩ⓗ2}\formedesurface{kɤ˧ʈʂɯ˩}\newline
\classe{名词}\ton{L\#}
2
\paradigme{\pcmn{:} \p{}}
\begin{définition}\peng{Speech.}\end{définition}
\begin{définition}\pcmn{话}\end{définition}
\begin{définition}\pfra{Parole.}\end{définition}
\begin{exemple}\pnru{kɤ˧ʈʂɯ˩ ʝi˩}\hspace{5pt}\peng{to promise; to make an oath; also: to swear before the gods: when people had a disagreement that they were unable to settle, they would go to the monastery and present their point of view before the gods, swearing that they were telling the truth; the gods would then punish the guilty one (through plagues and misfortunes).}\hspace{5pt}\pcmn{答应,誓、发誓。两个人发生矛盾的时候,如果无法协调,他们会去大寺,在神像前讲述他们各自的观点,发誓他们自己讲的是真的。神会惩罚说谎的人(他家会有祸害)。}\hspace{5pt}\pfra{promettre; aussi: jurer ses grands dieux, prêter serment devant les Dieux: lorsque deux personnes avaient un différend qu'elles ne parvenaient pas à trancher, elles allaient raconter chacune sa version des faits devant les Dieux (au monastère); ceux-ci punissaient ensuite le coupable (par des calamités qui frappaient la famille du coupable).}\end{exemple}
\begin{exemple}\pnru{ʈʂʰɯ˧ | kɤ˧ʈʂɯ˩-ʝi˩}\hspace{5pt}\peng{(s)he promises}\hspace{5pt}\pcmn{他答应}\hspace{5pt}\pfra{il/elle promet}\end{exemple}
\begin{exemple}\pnru{ʈʂʰɯ˧ | kɤ˧ʈʂɯ˩ | mɤ˧-ʝi˥!}\hspace{5pt}\peng{(s)he has not promised!}\hspace{5pt}\pcmn{他没有答应!}\hspace{5pt}\pfra{il n'a pas promis!}\end{exemple}
\begin{exemple}\pnru{hĩ˧-kɤ˧ʈʂɯ˥ ɲi˩}\hspace{5pt}\peng{to listen to people's advice, to pay attention to what other people say (a good attitude in the consultant's view)}\hspace{5pt}\pcmn{听别人的建议、把别人的话当回事}\hspace{5pt}\pfra{écouter les conseils d'autrui, prêter attention à la parole d'autrui, écouter les bons conseils (attitude jugée positive et souhaitable par la consultante)}\end{exemple}
\begin{exemple}\pnru{hĩ˧-kɤ˧ʈʂɯ˥ | le˧-ɲi˥}\hspace{5pt}\peng{as above}\hspace{5pt}\pcmn{同上}\hspace{5pt}\pfra{même sens}\end{exemple}
\begin{exemple}\pnru{hĩ˧-kɤ˧ʈʂɯ˥ | mɤ˧-ɲi˥}\hspace{5pt}\peng{to fail to listen to people's advice}\hspace{5pt}\pcmn{听不进去别人的意见与建议}\hspace{5pt}\pfra{ne pas écouter les bons conseils, ne pas prêter attention à ce qu'on vous dit}\end{exemple}
\end{entrée}

\begin{entrée}
{kɤ˧v̩\#˥}{}{ⓔkɤ˧v̩\#˥}\formedesurface{kɤ˧v̩˧}\newline
\classe{名词}\ton{\#H}
\paradigme{\pcmn{:} \p{}}
\begin{définition}\peng{Amulet.}\end{définition}
\begin{définition}\pcmn{护符,护身符}\end{définition}
\begin{définition}\pfra{Amulette.}\end{définition}
\end{entrée}

\begin{entrée}
{kɤ˧wɤ\#˥}{}{ⓔkɤ˧wɤ\#˥}\formedesurface{kɤ˧wɤ˧}\newline
\classe{名词}\ton{\#H}\begin{définition}\peng{Predestination, predestined affinity.}\end{définition}
\begin{définition}\pcmn{缘分、共同命运}\end{définition}
\begin{définition}\pfra{Destinée, affinité prédestinée.}\end{définition}
\begin{exemple}\pnru{kɤ˧wɤ˧-ljɤ˧˥}\hspace{5pt}\peng{to have a predestined affinity; to have a common destiny}\hspace{5pt}\pcmn{有缘分、有共同命运}\hspace{5pt}\pfra{avoir une affinité prédestinée, avoir un destin commun}\end{exemple}
\end{entrée}

\begin{entrée}
{kɤ˧zo\#˥}{}{ⓔkɤ˧zo\#˥}\formedesurface{kɤ˧zo˧}\newline
\classe{名词}\ton{\#H}\begin{définition}\peng{Masculine given name.}\end{définition}
\begin{définition}\pcmn{男性名字}\end{définition}
\begin{définition}\pfra{Prénom masculin.}\end{définition}
\end{entrée}

\begin{entrée}
{kɤ˩zo˩}{₁}{ⓔkɤ˩zo˩ⓗ1}\formedesurface{kɤ˩zo˩˥}\newline
\classe{名词}\ton{L}
1\begin{définition}\peng{Baby falcon.}\end{définition}
\begin{définition}\pcmn{小隼}\end{définition}
\begin{définition}\pfra{Bébé faucon.}\end{définition}
\end{entrée}

\begin{entrée}
{kɤ˩zo˩}{₂}{ⓔkɤ˩zo˩ⓗ2}\formedesurface{kɤ˩zo˩˥}\newline
\classe{名词}\ton{L}
2\begin{définition}\peng{Small bottle.}\end{définition}
\begin{définition}\pcmn{小瓶子}\end{définition}
\begin{définition}\pfra{Petite bouteille.}\end{définition}
\end{entrée}

\begin{entrée}
{kɤ˧zo˧-tsʰɯ˩ɻ̍˩}{}{ⓔkɤ˧zo˧-tsʰɯ˩ɻ̍˩}\formedesurface{kɤ˧zo˧tsʰɯ˩ɻ̍˩}\newline
\classe{名词}\ton{-L}\begin{définition}\peng{Masculine given name.}\end{définition}
\begin{définition}\pcmn{男性名字}\end{définition}
\begin{définition}\pfra{Prénom masculin.}\end{définition}
\end{entrée}

\begin{entrée}
{ki˥α}{}{ⓔki˥α}\formedesurface{ɖɯ˧ ki˥}\newline
\classe{量词}\ton{Hα}\begin{définition}\peng{In association with the numeral ‘one', this classifier means ‘together'.}\end{définition}
\begin{définition}\pcmn{量词:加上数词‘一’,这个量词表示‘一起’。}\end{définition}
\begin{définition}\pfra{En association avec le numéral ‘un', ce classificateur signifie ‘ensemble'.}\end{définition}
\begin{exemple}\pnru{ɖɯ˧-ki˥}\hspace{5pt}\peng{together}\hspace{5pt}\pcmn{一起(共事)}\hspace{5pt}\pfra{ensemble}\end{exemple}
\begin{exemple}\pnru{ɖɯ˧-ki˧ tʰv̩˧}\hspace{5pt}\peng{to arrive together/at the same time}\hspace{5pt}\pcmn{同时到达}\hspace{5pt}\pfra{arriver ensemble/en même temps}\end{exemple}
\begin{exemple}\pnru{ɖɯ˧-ʝi˧-ɳɯ˧ tsʰɯ˧˥, | ɖɯ˧-ki˧ tʰv̩˧!}\hspace{5pt}\peng{Coming from the same place, we arrive together!}\hspace{5pt}\pcmn{从一个地方,一起到!}\hspace{5pt}\pfra{Venant du même endroit, (on) arrive ensemble!}\end{exemple}
\begin{exemple}\pnru{ɖɯ˧-ki˧ dzi˧˥}\hspace{5pt}\peng{to live together}\hspace{5pt}\pcmn{住在一起}\hspace{5pt}\pfra{habiter ensemble}\end{exemple}
\end{entrée}

\begin{entrée}
{‑ki˧}{}{ⓔ‑ki˧}\formedesurface{ki˧}\newline
\classe{后缀}\ton{M}\begin{définition}\peng{Dative (particle indicating the recipient) / allative (indicating a direction).}\end{définition}
\begin{définition}\pcmn{与格/向格:对/向……}\end{définition}
\begin{définition}\pfra{Datif/allatif.}\end{définition}
\begin{exemple}\pnru{ʈʂʰɯ˧-ki˧ ʐwɤ˧˥}\hspace{5pt}\peng{to speak to her/him}\hspace{5pt}\pcmn{给他讲}\hspace{5pt}\pfra{lui dire, lui parler}\end{exemple}
\begin{exemple}\pnru{ʈʂʰɯ˧-ki˧ ʐwɤ˧-bi˥}\hspace{5pt}\peng{as above, with immediate future}\hspace{5pt}\pcmn{要给他讲}\hspace{5pt}\pfra{idem+futur immédiat}\end{exemple}
\begin{exemple}\pnru{ə˧mɑ˧-ɳɯ˧ | njɤ˧-ki˧ | nɑ˩-ʐwɤ˧ so˩!}\hspace{5pt}\peng{Ama teaches me the Na language!}\hspace{5pt}\pcmn{阿妈教我摩梭话!}\hspace{5pt}\pfra{Ama m'enseigne la langue na!}\end{exemple}
\end{entrée}

\begin{entrée}
{ki˧α}{}{ⓔki˧α}\formedesurface{ki˧}\newline
\classe{动词}\ton{Mα}\begin{définition}\peng{To give, to pass on, to transmit, to offer.}\end{définition}
\begin{définition}\pcmn{给、传、献给、发(工资),嫁给}\end{définition}
\begin{définition}\pfra{Donner, passer, transmettre.}\end{définition}
\begin{exemple}\pnru{ki˧∼ki˩}\hspace{5pt}\peng{|fg{red}}\hspace{5pt}\pcmn{重叠}\hspace{5pt}\pfra{|fg{red}}\end{exemple}
\begin{exemple}\pnru{tso˧∼tso˧-ki˩}\hspace{5pt}\peng{to give things}\hspace{5pt}\pcmn{给东西}\hspace{5pt}\pfra{donner des choses}\end{exemple}
\begin{exemple}\pnru{tso˧∼tso˧ ki˧∼ki˥}\hspace{5pt}\peng{to give things}\hspace{5pt}\pcmn{给东西}\hspace{5pt}\pfra{donner des choses}\end{exemple}
\begin{exemple}\pnru{hĩ˧-ki˧ ki˩}\hspace{5pt}\peng{1. to give to someone. 2. to give oneself (in marriage) to someone, to marry someone (for a woman)}\hspace{5pt}\pcmn{1.许配给人家。2.嫁给人}\hspace{5pt}\pfra{1. donner à quelqu'un. 2. se donner en mariage à quelqu'un (pour une femme)}\end{exemple}
\begin{exemple}\pnru{hĩ˧-ki˧ | ɖwæ˧˥ | tʰi˧-ki˧}\hspace{5pt}\peng{to be generous, to be open-handed}\hspace{5pt}\pcmn{大方}\hspace{5pt}\pfra{être généreux}\end{exemple}
\begin{exemple}\pnru{hĩ˧-ki˧ ki˩ fv̩˩}\hspace{5pt}\peng{to like to make gifts, to like to give things to people}\hspace{5pt}\pcmn{爱送礼,爱给别人送东西}\hspace{5pt}\pfra{qui aime faire des cadeaux, qui aime donner des choses aux gens}\end{exemple}
\begin{exemple}\pnru{pʰɤ˧bɤ˧ ki˧ (-bi˧)}\hspace{5pt}\peng{to offer gifts}\hspace{5pt}\pcmn{送礼物}\hspace{5pt}\pfra{donner des cadeaux}\end{exemple}
\begin{exemple}\pnru{hɑ˧ ki˩}\hspace{5pt}\peng{to feed, to give food}\hspace{5pt}\pcmn{喂饭}\hspace{5pt}\pfra{nourrir, donner à manger}\end{exemple}
\end{entrée}

\begin{entrée}
{ki˩α}{}{ⓔki˩α}\formedesurface{ki˩˥}\newline
\classe{动词}\ton{Lα}\begin{définition}\peng{To put on (a skirt, trousers…).}\end{définition}
\begin{définition}\pcmn{穿上(裤子、袜子、鞋子)}\end{définition}
\begin{définition}\pfra{Enfiler, porter, mettre (une robe, un pantalon…).}\end{définition}
\begin{exemple}\pnru{ɬi˧qʰwɤ˩ | ɖɯ˧-ɭɯ˧ ki˩}\hspace{5pt}\peng{to put on trousers}\hspace{5pt}\pcmn{穿上裤子}\hspace{5pt}\pfra{enfiler un pantalon}\end{exemple}
\begin{exemple}\pnru{dzɑ˩qʰwɤ˩˥ | ɖɯ˧-dzi˧ ki˩}\hspace{5pt}\peng{to put on a pair of shoes}\hspace{5pt}\pcmn{穿上一双鞋}\hspace{5pt}\pfra{enfiler une paire de chaussures}\end{exemple}
\begin{exemple}\pnru{ʈʰæ˩ ki˩˥}\hspace{5pt}\peng{‘to put on a skirt'; this is the name of the ritual of entry into adulthood, after a girl has reached age 13}\hspace{5pt}\pcmn{“穿裙”:这是成年礼的名称(穿裙礼:女孩满13岁,即为成人)}\hspace{5pt}\pfra{porter la jupe; nom du rituel de passage à l'âge adulte (à treize ans révolus) pour les jeunes filles}\end{exemple}
\begin{exemple}\pnru{ɬi˧ ki˥}\hspace{5pt}\peng{‘to put on trousers'; this is the name of the ritual of entry into adulthood, after a boy has reached age 13}\hspace{5pt}\pcmn{“穿裤”:这是成年礼的名称(穿裤礼:男孩满了13岁,即为成人)}\hspace{5pt}\pfra{porter le pantalon; nom du rituel de passage à l'âge adulte (à treize ans révolus) pour les jeunes gens}\end{exemple}
\end{entrée}

\begin{entrée}
{ki˧li˥}{}{ⓔki˧li˥}\formedesurface{ki˧li˥}\newline
\classe{助词}\ton{H\#}\begin{définition}\peng{In a mess.}\end{définition}
\begin{définition}\pcmn{乱七八糟}\end{définition}
\begin{définition}\pfra{En désordre (formulation expressive, quasi-onomatopéique).}\end{définition}
\end{entrée}

\begin{entrée}
{ki˩mi˧}{}{ⓔki˩mi˧}\formedesurface{ki˩mi˥}\newline
\classe{名词}\ton{LM}
\paradigme{\pcmn{:} \p{}}
\begin{définition}\peng{A large fly with a green head; its larvae are particularly harmful.}\end{définition}
\begin{définition}\pcmn{绿头苍蝇}\end{définition}
\begin{définition}\pfra{Grosse mouche, dont les larves sont redoutées.}\end{définition}
\end{entrée}

\begin{entrée}
{ki˩tɑ\#˥}{}{ⓔki˩tɑ\#˥}\formedesurface{ki˩tɑ˥}\newline
\classe{名词}\ton{LM+\#H}\begin{définition}\peng{Bag made of leather and linen, in which silver coins used to be kept, buried somewhere in the house to hide it from robbers.}\end{définition}
\begin{définition}\pcmn{皮袋,来装家里的财物:金币、银币……这个皮袋,埋在房子里的一个保密地方,防贼。为了让它很结实,袋子有四、五层麻布内衬。可以保存很长时间。}\end{définition}
\begin{définition}\pfra{Sac de cuir et de lin, dans lequel on plaçait ce que possédait la maisonnée: or, argent… qu'on enterrait quelque part dans la maison, pour se prémunir contre les voleurs; les matières choisies, cuir et lin, se conservaient très longtemps; le sac avait quatre ou cinq épaisseurs de tissu, pour le rendre plus solide.}\end{définition}
\begin{exemple}\pnru{æ˧-tse˥pʰæ˩ | ɖɯ˧-ki˩tɑ˩}\hspace{5pt}\peng{a bag of bronze coins}\hspace{5pt}\pcmn{一袋铜币}\hspace{5pt}\pfra{un sac de pièces de cuivre}\end{exemple}
\end{entrée}

\begin{entrée}
{ki˩ti\#˥}{}{ⓔki˩ti\#˥}\formedesurface{ki˩ti˥}\newline
\classe{名词}\ton{LM+\#H}
\paradigme{\pcmn{:} \p{}}
\begin{définition}\peng{Leather belt.}\end{définition}
\begin{définition}\pcmn{皮腰带}\end{définition}
\begin{définition}\pfra{Ceinture en cuir.}\end{définition}
\end{entrée}

\begin{entrée}
{ki˧zo\#˥}{}{ⓔki˧zo\#˥}\formedesurface{ki˧zo˧}\newline
\classe{名词}\ton{\#H}\begin{définition}\peng{A unixex given name: a given name used for both men and women.}\end{définition}
\begin{définition}\pcmn{男女通用名}\end{définition}
\begin{définition}\pfra{Prénom unisexe: prénom utilisé pour les deux sexes.}\end{définition}
\end{entrée}

\begin{entrée}
{ki˧zo˧-ɖɯ˩mɑ˩}{}{ⓔki˧zo˧-ɖɯ˩mɑ˩}\formedesurface{ki˧zo˧ɖɯ˩mɑ˩}\newline
\classe{名词}\ton{-L}\begin{définition}\peng{Feminine given name.}\end{définition}
\begin{définition}\pcmn{女性名字}\end{définition}
\begin{définition}\pfra{Prénom féminin.}\end{définition}
\end{entrée}

\begin{entrée}
{ki˧zo˧-ɬɑ˩mv̩˩}{}{ⓔki˧zo˧-ɬɑ˩mv̩˩}\formedesurface{ki˧zo˧ɬɑ˩mv̩˩}\newline
\classe{名词}\ton{-L}\begin{définition}\peng{Feminine given name.}\end{définition}
\begin{définition}\pcmn{女性名字}\end{définition}
\begin{définition}\pfra{Prénom féminin.}\end{définition}
\end{entrée}

\begin{entrée}
{ko˥}{}{ⓔko˥}\formedesurface{ko˧}\newline
\classe{名词}\ton{\#H}
\paradigme{\pcmn{:} \p{}}
\begin{définition}\peng{Hill, small mountain.}\end{définition}
\begin{définition}\pcmn{小山}\end{définition}
\begin{définition}\pfra{Colline, petite montagne.}\end{définition}
\end{entrée}

\begin{entrée}
{ko˧˥}{₁}{ⓔko˧˥ⓗ1}\formedesurface{ko˧˥}\newline
\classe{助词}\ton{MH}
1\begin{définition}\peng{Too much, excessively.}\end{définition}
\begin{définition}\pcmn{过于,太(汉语借词)}\end{définition}
\begin{définition}\pfra{Trop, excessivement.}\end{définition}
\end{entrée}

\begin{entrée}
{ko˧˥}{₂}{ⓔko˧˥ⓗ2}\formedesurface{ko˧˥}\newline
\classe{动词}\ton{MH}
2\begin{définition}\peng{To happen, to take place, to pass, to go by (days, existence).}\end{définition}
\begin{définition}\pcmn{过(汉语借词)}\end{définition}
\begin{définition}\pfra{Se passer, avoir lieu: les jours passent, la vie se passe; couler (des jours/des années).}\end{définition}
\begin{exemple}\pnru{se˧ʐɯ˩ ko˩}\hspace{5pt}\peng{to celebrate a birthday}\hspace{5pt}\pcmn{过生日}\hspace{5pt}\pfra{fêter un anniversaire}\end{exemple}
\end{entrée}

\begin{entrée}
{ko˧α}{}{ⓔko˧α}\formedesurface{ɖɯ˧ ko˧}\newline
\classe{量词}\ton{Mα}\begin{définition}\peng{Classifier for small objects, e.g. cigarettes.}\end{définition}
\begin{définition}\pcmn{量词:小东西,例如烟(一只)}\end{définition}
\begin{définition}\pfra{Classificateur des petits objets, tels que des cigarettes.}\end{définition}
\end{entrée}

\begin{entrée}
{ko˩α}{}{ⓔko˩α}\formedesurface{ko˩˥}\newline
\classe{动词}\ton{Lα}\begin{définition}\peng{To warm oneself at a fire; to bask in the sun.}\end{définition}
\begin{définition}\pcmn{烤火取暖,晒太阳}\end{définition}
\begin{définition}\pfra{Se chauffer au feu ou au soleil; prendre le soleil.}\end{définition}
\begin{exemple}\pnru{mv̩˧ ko˥}\hspace{5pt}\peng{to warm oneself at a fire}\hspace{5pt}\pcmn{烤火取暖}\hspace{5pt}\pfra{se chauffer au feu}\end{exemple}
\begin{exemple}\pnru{le˧-ko˩-ze˩}\hspace{5pt}\peng{|fg{accomp} \_ |fg{pfv}}\hspace{5pt}\pcmn{烤火了}\hspace{5pt}\pfra{|fg{accomp} \_ |fg{pfv}}\end{exemple}
\begin{exemple}\pnru{ɲi˧mi˧ ko˩}\hspace{5pt}\peng{to bask in the sun}\hspace{5pt}\pcmn{晒太阳}\hspace{5pt}\pfra{se réchauffer au soleil}\end{exemple}
\begin{exemple}\pnru{ɲi˧mi˧ ɖɯ˧-ko˩-ɻ̍˩}\hspace{5pt}\peng{to bask in the sun for a while}\hspace{5pt}\pcmn{晒晒太阳}\hspace{5pt}\pfra{se réchauffer un moment au soleil}\end{exemple}
\begin{exemple}\pnru{ɲi˧mi˧ ɖɯ˧-ko˧∼ko˥-ɻ̍˩}\hspace{5pt}\peng{to bask in the sun for a while}\hspace{5pt}\pcmn{晒晒太阳}\hspace{5pt}\pfra{se réchauffer un moment au soleil}\end{exemple}
\end{entrée}

\begin{entrée}
{ko˩dze˧}{}{ⓔko˩dze˧}\formedesurface{ko˩dze˥}\newline
\classe{名词}\ton{LM}
\paradigme{\pcmn{:} \p{}}
\begin{définition}\peng{A sort of dove.}\end{définition}
\begin{définition}\pcmn{一种鸽子}\end{définition}
\begin{définition}\pfra{Sorte de colombe.}\end{définition}
\end{entrée}

\begin{entrée}
{ko˩dʑo˩}{}{ⓔko˩dʑo˩}\formedesurface{ko˩dʑo˩˥}\newline
\classe{名词}\ton{L}\begin{définition}\peng{Peacock.}\end{définition}
\begin{définition}\pcmn{孔雀}\end{définition}
\begin{définition}\pfra{Paon.}\end{définition}
\end{entrée}

\begin{entrée}
{ko˧ɖæ\#˥}{}{ⓔko˧ɖæ\#˥}\formedesurface{ko˧ɖæ˧}\newline
\classe{名词}\ton{\#H}
\paradigme{\pcmn{:} \p{}}
\begin{définition}\peng{Sculpture of Buddha (Tibetan borrowing).}\end{définition}
\begin{définition}\pcmn{佛像}\end{définition}
\begin{définition}\pfra{Statue de bouddha.}\end{définition}
\begin{exemple}\pnru{ko˧ɖæ˧-zo˧}\hspace{5pt}\peng{small statue of Buddha}\hspace{5pt}\pcmn{小佛像}\hspace{5pt}\pfra{statuette du Bouddha}\end{exemple}
\end{entrée}

\begin{entrée}
{ko˩ɖʐo˩}{}{ⓔko˩ɖʐo˩}\formedesurface{ko˩ɖʐo˩˥}\newline
\classe{名词}\ton{L}
\paradigme{\pcmn{:} \p{}}
\begin{définition}\peng{Flail.}\end{définition}
\begin{définition}\pcmn{连枷}\end{définition}
\begin{définition}\pfra{Fléau pour battre le grain.}\end{définition}
\end{entrée}

\begin{entrée}
{ko˧li\#˥}{}{ⓔko˧li\#˥}\formedesurface{ko˧li˧}\newline
\classe{名词}\ton{\#H}
\paradigme{\pcmn{:} \p{}}
\begin{définition}\peng{Blow tube: tube to blow on a fire.}\end{définition}
\begin{définition}\pcmn{吹火筒,用来吹火的小管子}\end{définition}
\begin{définition}\pfra{Soufflet à bouche: un tube dans lequel on souffle pour attiser le feu.}\end{définition}
\end{entrée}

\begin{entrée}
{ko˧no˧-ʁo\#˥}{}{ⓔko˧no˧-ʁo\#˥}\formedesurface{ko˧no˧-ʁo˧}\newline
\classe{名词}\ton{\#H}
\paradigme{\pcmn{:} \p{}}
\begin{définition}\peng{Mountain ridge; bridge in the mountains.}\end{définition}
\begin{définition}\pcmn{山梁}\end{définition}
\begin{définition}\pfra{Crête, ligne de faîte (en montagne).}\end{définition}
\end{entrée}

\begin{entrée}
{ko˩qʰɑ˧-dʑɯ\#˥}{}{ⓔko˩qʰɑ˧-dʑɯ\#˥}\formedesurface{ko˩qʰɑ˧-dʑɯ˧}\newline
\classe{名词}\ton{LM+\#H}\begin{définition}\peng{Golden plum blossom (a local term; identification not established yet)}\end{définition}
\begin{définition}\pcmn{一种植物,当地名称:金梅花。金梅花}\end{définition}
\begin{définition}\pfra{Prune dorée. Terme local; l'identification reste à faire.}\end{définition}
\begin{exemple}\pnru{ko˩qʰɑ˧-dʑɯ˧-bæ˥bæ˩}\hspace{5pt}\peng{flower of golden plum}\hspace{5pt}\pcmn{金梅花的花}\hspace{5pt}\pfra{fleur de prune dorée}\end{exemple}
\end{entrée}

\begin{entrée}
{ko˧sɯ\#˥}{}{ⓔko˧sɯ\#˥}\formedesurface{ko˧sɯ˧}\newline
\classe{名词}\ton{\#H}\begin{définition}\peng{Shop.}\end{définition}
\begin{définition}\pcmn{商店、小卖部(汉语借词:公司)}\end{définition}
\begin{définition}\pfra{Boutique.}\end{définition}
\end{entrée}

\begin{entrée}
{kɯ˥}{}{ⓔkɯ˥}\formedesurface{kɯ˧}\newline
\classe{形容词}\ton{H}\begin{définition}\peng{Tight, tense.}\end{définition}
\begin{définition}\pcmn{紧}\end{définition}
\begin{définition}\pfra{Serré, tendu.}\end{définition}
\begin{exemple}\pnru{le˧-tsɯ˥ | le˧-kɯ˥-kʰɯ˩}\hspace{5pt}\peng{to attach tightly, to attach so that it will be quite tight}\hspace{5pt}\pcmn{绑紧}\hspace{5pt}\pfra{attacher très serré}\end{exemple}
\begin{exemple}\pnru{le˧-kɯ˥-se˩}\hspace{5pt}\peng{|fg{accomp} \_ |fg{pfv}}\hspace{5pt}\pcmn{紧了}\hspace{5pt}\pfra{|fg{accomp} \_ |fg{pfv}}\end{exemple}
\end{entrée}

\begin{entrée}
{kɯ˥}{}{ⓔkɯ˥}\newline
\classe{名词}
\sens{1}\paradigme{\pcmn{:} \p{}}
\begin{définition}\peng{Gallbladder.}\end{définition}
\begin{définition}\pcmn{胆}\end{définition}
\begin{définition}\pfra{Vésicule biliaire.}\end{définition}\sens{2}
\begin{définition}\peng{Gall.}\end{définition}
\begin{définition}\pcmn{胆汁}\end{définition}
\begin{définition}\pfra{Bile.}\end{définition}
\end{entrée}

\begin{entrée}
{kɯ˧}{}{ⓔkɯ˧}\formedesurface{kɯ˧}\newline
\classe{名词}\ton{M}
\paradigme{\pcmn{:} \p{}}
\begin{définition}\peng{Star.}\end{définition}
\begin{définition}\pcmn{星星}\end{définition}
\begin{définition}\pfra{Étoile.}\end{définition}
\begin{exemple}\pnru{mv̩˧ʁo˥ | kɯ˧}\hspace{5pt}\peng{there are stars in the sky, one can see stars}\hspace{5pt}\pcmn{天上有星星、天上看得见星星}\hspace{5pt}\pfra{le ciel est étoilé, on voit les étoiles du ciel}\end{exemple}
\begin{exemple}\pnru{nɑ˩-ʈʂʰɯ˥, | kɯ˧ mɤ˧-li˧! | di˧mi˧-lɑ˧ li˥!}\hspace{5pt}\peng{The Na do not look at the stars! They only look at the plain (=at the plain of Yongning)! (A comment by the consultant about her lack of knowledge of the names of stars and constellations.)}\hspace{5pt}\pcmn{摩梭呢,不看星星,只看平坝(=永宁坝子)!(合作人说明为什么她不知道星星、星座的名字:摩梭人本来对天文不太感兴趣。)}\hspace{5pt}\pfra{Les Na, ils ne regardent pas les étoiles! Ils ne regardent que la plaine (=leur plaine: la plaine de Yongning)! (commentaire de F4 au sujet de son ignorance des noms d'étoiles et de constellations)}\end{exemple}
\end{entrée}

\begin{entrée}
{kɯ˧β}{}{ⓔkɯ˧β}\formedesurface{ɖɯ˧ kɯ˧}\newline
\classe{量词}\ton{Mβ}\begin{définition}\peng{Self-classifier for stars.}\end{définition}
\begin{définition}\pcmn{量词:星星(一个)}\end{définition}
\begin{définition}\pfra{Auto-classificateur des étoiles; classificateur des jours propices.}\end{définition}
\end{entrée}

\begin{entrée}
{kɯ˩α}{}{ⓔkɯ˩α}\formedesurface{kɯ˩˥}\newline
\classe{动词}\ton{Lα}\begin{définition}\peng{To ignore someone who would need help, to leave someone alone with difficulties one could help with. This is a term for which no straightforward Chinese equivalent has been found; it refers to a situation where lack of real attachment to someone shows up in the lack of impulse to go out of one's way and help them.}\end{définition}
\begin{définition}\pcmn{不理需要帮忙的人:知道一个人需要帮助,自己也有能力帮忙,但假装没看见、什么事没有}\end{définition}
\begin{définition}\pfra{Laisser quelqu'un en rade, laisser quelqu'un seul au moment où il aurait besoin d'aide, faire mine d'ignorer quelqu'un qui aurait besoin d'aide, négliger d'assister quelqu'un. Il n'a pas été trouvé d'équivalent chinois simple pour ce terme, qui renvoie à la situation où un manque de réel sympathie pour quelqu'un se traduit par le fait qu'on n'est pas tenté de faire l'effort de l'aider lorsqu'il en aurait besoin: on fait alors comme si de rien n'était, comme si on n'était pas au courant de la situation de cette personne.}\end{définition}
\begin{exemple}\pnru{hĩ˧ kɯ˥}\hspace{5pt}\peng{same meaning}\hspace{5pt}\pcmn{同上}\hspace{5pt}\pfra{même sens}\end{exemple}
\begin{exemple}\pnru{hĩ˧-ɳɯ˩ | kɯ˩-kv̩˥!}\hspace{5pt}\peng{People will sometimes ignore you when you are in need! / (You will realize that, in cases where you need help) people will sometimes ignore you and leave you alone with your difficulties!}\hspace{5pt}\pcmn{人家在你需要帮忙的时候就会不理你的!(如果处不好关系,人家对你没有什么好感,到时候你需要帮忙他们就不理你了。)}\hspace{5pt}\pfra{Il arrive que les gens te laissent tomber! / Il arrive que les gens ne t'apportent pas leur aide quand tu en aurais besoin!}\end{exemple}
\begin{exemple}\pnru{kɯ˩-mɤ˩-kv̩˥!}\hspace{5pt}\peng{(They) are not going to help you! / You're not going to get any help (from them)!}\hspace{5pt}\pcmn{人家在你需要帮忙的时候就会不理你的!}\hspace{5pt}\pfra{(ils/elles) n(e t)'aideront pas!}\end{exemple}
\begin{exemple}\pnru{hĩ˧-ɳɯ˩ | kɯ˩-tʰɑ˩-kʰɯ˥!}\hspace{5pt}\peng{Don't (behave in such a way as to) let people ignore you when you are in need! (Explanation: one should build trust for oneself, making others feel real trust and gratitude, so that they will help as a matter of course when the need for it arises; otherwise they will ignore us when we are in need of help.)}\hspace{5pt}\pcmn{别让人家(在你需要帮忙的时候)不理你!}\hspace{5pt}\pfra{Fais en sorte que les gens ne te laissent pas en rade (quand tu auras besoin d'aide)! (Explication: il faut faire en sorte de gagner une estime et une sympathie réelles de la part des gens que l'on connaît; de la sorte, ils vous aideront spontanément lorsque vous en aurez besoin. Sinon, leur manque de réelle sympathie se traduira par le fait qu'ils ne feront pas l'effort de donner un coup de main lorsqu'on en aura besoin: ils feront alors comme si de rien n'était, comme s'ils n'étaient pas conscients de notre besoin.)}\end{exemple}
\begin{exemple}\pnru{njɤ˧ | no˩ kɯ˩-hĩ˥ mɤ˩-ɲi˩! | njɤ˧ | no˧-ki˧ | dʑɤ˩-so˥-ɲi˩!}\hspace{5pt}\peng{I am not neglecting you at all! (On the contrary) I am teaching you good things / I am doing my best to teach you! (Context: a student considers himself neglected by a teacher; the teacher realizes that the student is dissatisfied, and provides a clarification.)}\hspace{5pt}\pcmn{我不是不重视你!(刚好相反:)我是用心教你的 / 我努力教你最好的!(情景:一名学生认为老师忽视他,老师发现学生不高兴,就说明。)}\hspace{5pt}\pfra{Je ne te néglige pas du tout! (Bien au contraire:) je t'enseigne bien / je fais de mon mieux pour t'apprendre des choses! (Contexte imaginé: un étudiant s'estime négligé par un enseignant; celui-ci se rend compte du mécontentement de l'étudiant, et lui dit qu'il interprète mal.)}\end{exemple}
\end{entrée}

\begin{entrée}
{kɯ˧ɭɯ˧}{}{ⓔkɯ˧ɭɯ˧}\formedesurface{kɯ˧ɭɯ˧}\newline
\classe{名词}\ton{M}
\paradigme{\pcmn{:} \p{}}
\begin{définition}\peng{Spirit.}\end{définition}
\begin{définition}\pcmn{神}\end{définition}
\begin{définition}\pfra{Esprit bienfaisant.}\end{définition}
\begin{exemple}\pnru{kɯ˧ɭɯ˧ | ɖɯ˧-dze˩}\hspace{5pt}\peng{a pair of (good) spirits, two (good) spirits}\hspace{5pt}\pcmn{两个(好)神}\hspace{5pt}\pfra{deux esprits bienfaisants}\end{exemple}
\end{entrée}

\begin{entrée}
{kɯ˧qʰæ˧ʂe˧˥}{}{ⓔkɯ˧qʰæ˧ʂe˧˥}\formedesurface{kɯ˧qʰæ˧ʂe˧˥}\newline
\classe{名词}\ton{MH\#}
\paradigme{\pcmn{:} \p{}}
\begin{définition}\peng{Comet.}\end{définition}
\begin{définition}\pcmn{流星}\end{définition}
\begin{définition}\pfra{Comète (littéralement «les étoiles défèquent»).}\end{définition}
\end{entrée}

\begin{entrée}
{kɯ˩ɻ̍˧}{}{ⓔkɯ˩ɻ̍˧}\formedesurface{kɯ˩ɻ̍˥}\newline
\classe{名词}\ton{LM}
\paradigme{\pcmn{:} \p{}}
\begin{définition}\peng{Two-string violin, erhu.}\end{définition}
\begin{définition}\pcmn{胡琴,二胡}\end{définition}
\begin{définition}\pfra{Violon à deux cordes, erhu.}\end{définition}
\begin{exemple}\pnru{kɯ˩ɻ̍˧ ʈɤ˧}\hspace{5pt}\peng{to play erhu}\hspace{5pt}\pcmn{拉二胡}\hspace{5pt}\pfra{jouer du erhu}\end{exemple}
\end{entrée}

\begin{entrée}
{kɯ˧ʈʂʰwɤ˩}{}{ⓔkɯ˧ʈʂʰwɤ˩}\formedesurface{kɯ˧ʈʂʰwɤ˩}\newline
\classe{名词}\ton{L\#}\begin{définition}\peng{A light meal taken at night, at about midnight, after one had worked or played late into the night. Literally ‘stars' dinner', because at that time of night (about midnight) the farm was asleep, and only the stars were present to keep company.}\end{définition}
\begin{définition}\pcmn{半夜小餐:如果工作或玩到半夜,在晚餐以外,还会在半夜十二点左右准备一顿小吃。直译:‘星餐’,因为在那时间全家都已入睡,只有天上的星星陪伴熬夜的人们。}\end{définition}
\begin{définition}\pfra{Repas de minuit: une collation prise vers minuit, au terme d'une longue soirée de travail ou de divertissement. L'expression signifie littéralement «dîner des étoiles»: à cette heure-là, toute la ferme est endormie, et les étoiles sont la seule présence qui accompagne les veilleurs.}\end{définition}
\begin{exemple}\pnru{kɯ˧ʈʂʰwɤ˩ gv˩}\hspace{5pt}\peng{To prepare a midnight meal.}\hspace{5pt}\pcmn{准备半夜小餐。}\hspace{5pt}\pfra{Préparer un repas des étoiles, confectionner la collation de minuit.}\end{exemple}
\end{entrée}

\begin{entrée}
{kv̩˥}{}{ⓔkv̩˥}\formedesurface{kv̩˧}\newline
\classe{名词}\ton{\#H}
\paradigme{\pcmn{:} \p{}}
\begin{définition}\peng{Garlic, |\stylefi{Allium sativum}.}\end{définition}
\begin{définition}\pcmn{大蒜}\end{définition}
\begin{définition}\pfra{Ail, |\stylefi{Allium sativum}.}\end{définition}
\end{entrée}

\begin{entrée}
{kv̩˧˥}{}{ⓔkv̩˧˥}\formedesurface{kv̩˧˥}\newline
\classe{动词}\ton{MH}\begin{définition}\peng{To be able to.}\end{définition}
\begin{définition}\pcmn{会、有能力做}\end{définition}
\begin{définition}\pfra{Pouvoir, être capable de, avoir la compétence pour (verbe de modalité épistémique).}\end{définition}
\end{entrée}

\begin{entrée}
{‑kv̩˧˥}{}{ⓔ‑kv̩˧˥}\formedesurface{kv̩˧˥}\newline
\classe{后缀}\ton{MH}\begin{définition}\peng{|fg{abilitive}; also indicates future.}\end{définition}
\begin{définition}\pcmn{能}\end{définition}
\begin{définition}\pfra{|fg{abilitive}; a aussi des emplois de futur.}\end{définition}
\end{entrée}

\begin{entrée}
{kv̩˩α}{₁}{ⓔkv̩˩αⓗ1}\formedesurface{kv̩˩˥}\newline
\classe{动词}
1
\sens{1}
\begin{définition}\peng{To pick up (from the ground), to gather.}\end{définition}
\begin{définition}\pcmn{捡起来,拾}\end{définition}
\begin{définition}\pfra{Ramasser; cueillir (des baies, des choses qu'on se baisse pour cueillir).}\end{définition}
\begin{exemple}\pnru{kv̩˧∼kv̩˥}\hspace{5pt}\peng{|fg{red}}\hspace{5pt}\pcmn{重叠}\hspace{5pt}\pfra{|fg{red}}\end{exemple}
\begin{exemple}\pnru{gɤ˩-kv̩˧∼kv̩˥}\hspace{5pt}\peng{to pick up (something that was on the ground)}\hspace{5pt}\pcmn{捡起来(地上的东西)}\hspace{5pt}\pfra{ramasser (quelque chose qui se trouvait à terre)}\end{exemple}
\begin{exemple}\pnru{le˧-kv̩˧∼kv̩˥}\hspace{5pt}\peng{to pick up (something that was on the ground)}\hspace{5pt}\pcmn{捡起来(地上的东西)}\hspace{5pt}\pfra{ramasser (quelque chose qui se trouvait à terre)}\end{exemple}
\begin{exemple}\pnru{le˧-ko˧∼ko˥ | po˧tsʰɯ˧ (+ tʰv̩˧-v̩˧ / zo˧mv̩˥)}\hspace{5pt}\peng{[This child] has been adopted (literally “has been picked up")}\hspace{5pt}\pcmn{(这个孩子)是被领养的。}\hspace{5pt}\pfra{il a été adopté; littéralement: «il a été ramassé, celui-là/cet enfant»}\end{exemple}\sens{2}
\begin{définition}\peng{To fish.}\end{définition}
\begin{définition}\pcmn{钓鱼}\end{définition}
\begin{définition}\pfra{Pêcher.}\end{définition}
\begin{exemple}\pnru{ɲi˧zo˧ kv̩˥}\hspace{5pt}\peng{to fish}\hspace{5pt}\pcmn{钓鱼}\hspace{5pt}\pfra{pêcher du poisson}\end{exemple}
\end{entrée}

\begin{entrée}
{kv̩˩α}{₂}{ⓔkv̩˩αⓗ2}\formedesurface{kv̩˩˥}\newline
\classe{动词}\ton{Lα}
2\begin{définition}\peng{To cross.}\end{définition}
\begin{définition}\pcmn{过}\end{définition}
\begin{définition}\pfra{Traverser.}\end{définition}
\begin{exemple}\pnru{ʈʂʰwæ˩ kv̩˥}\hspace{5pt}\peng{to cross (a river) in a boat}\hspace{5pt}\pcmn{坐船过(河)}\hspace{5pt}\pfra{traverser en bateau}\end{exemple}
\end{entrée}

\begin{entrée}
{kv̩˧dʑɯ˧˥}{}{ⓔkv̩˧dʑɯ˧˥}\formedesurface{kv̩˧dʑɯ˧˥}\newline
\classe{名词}\ton{MH}
\paradigme{\pcmn{:} \p{}}
\begin{définition}\peng{Tent.}\end{définition}
\begin{définition}\pcmn{帐篷}\end{définition}
\begin{définition}\pfra{Tente.}\end{définition}
\begin{exemple}\pnru{kv̩˧dʑɯ˧ lɑ˥}\hspace{5pt}\peng{to put up a tent, to set up a tent}\hspace{5pt}\pcmn{安装帐篷、搭建帐篷}\hspace{5pt}\pfra{déplier une tente, installer une tente}\end{exemple}
\end{entrée}

\begin{entrée}
{kv̩˧ʝi˥}{}{ⓔkv̩˧ʝi˥}\formedesurface{kv̩˧ʝi˥}\newline
\classe{助词}\ton{H\#}\begin{définition}\peng{Truly, really, for good.}\end{définition}
\begin{définition}\pcmn{真的、的确、确实}\end{définition}
\begin{définition}\pfra{Pour de vrai, réellement.}\end{définition}
\end{entrée}

\begin{entrée}
{kv̩˧ʝi˥\$}{}{ⓔkv̩˧ʝi˥\$}\formedesurface{kv̩˧ʝi˥}\newline
\classe{名词}\ton{H\$}\begin{définition}\peng{Life, existence, lifetime.}\end{définition}
\begin{définition}\pcmn{生命}\end{définition}
\begin{définition}\pfra{Vie, existence.}\end{définition}
\end{entrée}

\begin{entrée}
{kv̩˩kv̩˩}{}{ⓔkv̩˩kv̩˩}\formedesurface{kv̩˩kv̩˩˥}\newline
\classe{名词}\ton{L}
\paradigme{\pcmn{:} \p{}}
\begin{définition}\peng{Cheekbone.}\end{définition}
\begin{définition}\pcmn{颧骨}\end{définition}
\begin{définition}\pfra{Pommettes.}\end{définition}
\end{entrée}

\begin{entrée}
{kv̩˧lv̩˧lv̩˥}{}{ⓔkv̩˧lv̩˧lv̩˥}\formedesurface{kv̩˧lv̩˧lv̩˥}\newline
\classe{名词}\ton{H\#}
\paradigme{\pcmn{:} \p{}}
\begin{définition}\peng{Garlic braid: garlic bulbs with long leaves, braided into a large clump.}\end{définition}
\begin{définition}\pcmn{蒜瓣}\end{définition}
\begin{définition}\pfra{Tresse d'ail, ail tressé.}\end{définition}
\end{entrée}

\begin{entrée}
{kv̩˩nɑ˧˥}{}{ⓔkv̩˩nɑ˧˥}\formedesurface{kv̩˩nɑ˧˥}\newline
\classe{名词}\ton{LM+MH\#}
\paradigme{\pcmn{:} \p{}}
\begin{définition}\peng{Silk.}\end{définition}
\begin{définition}\pcmn{丝绸}\end{définition}
\begin{définition}\pfra{Soie.}\end{définition}
\begin{exemple}\pnru{kv̩˩nɑ˧-bɑ˧lɑ˥}\hspace{5pt}\peng{silk garment}\hspace{5pt}\pcmn{丝绸衣服}\hspace{5pt}\pfra{vêtement en soie}\end{exemple}
\end{entrée}

\begin{entrée}
{kv̩˧ɲi˥}{}{ⓔkv̩˧ɲi˥}\formedesurface{kv̩˧ɲi˥}\newline
\classe{形容词}\ton{H\#}\begin{définition}\peng{Empty.}\end{définition}
\begin{définition}\pcmn{空手,空}\end{définition}
\begin{définition}\pfra{Vide, sans rien.}\end{définition}
\begin{exemple}\pnru{bi˩ʁo˧ | kv̩˧ɲi˥-kʰɯ˩}\hspace{5pt}\peng{to empty (someone's) purse, i.e. to take someone's money}\hspace{5pt}\pcmn{(把一个人的)钱包弄空}\hspace{5pt}\pfra{vider la bourse (de quelqu'un), c'est-à-dire lui prendre son argent)}\end{exemple}
\begin{exemple}\pnru{tɕʰɯ˩ di˩-hɯ˩˥, | mɤ˧-ɖɯ˧, | kv̩˧ɲi˥ | le˧-tsʰɯ˩!}\hspace{5pt}\peng{He went to hunt the muntjac, but did not kill any, and came back empty-handed!}\hspace{5pt}\pcmn{他去狩猎,没得(任何猎物),空手回来!}\hspace{5pt}\pfra{Il est parti chasser le muntjac, il n'en a pas tué, et est revenu bredouille!}\end{exemple}
\end{entrée}

\begin{entrée}
{kv̩˧ʁo˧bv̩˥}{}{ⓔkv̩˧ʁo˧bv̩˥}\formedesurface{kv̩˧ʁo˧bv̩˥}\newline
\classe{名词}\ton{H\#}
\paradigme{\pcmn{:} \p{}}
\begin{définition}\peng{Garlic sprouts (consumed as a vegetable).}\end{définition}
\begin{définition}\pcmn{蒜苗}\end{définition}
\begin{définition}\pfra{Pousses d'ail (aliment).}\end{définition}
\end{entrée}

\begin{entrée}
{kv̩˧ʂe˥\$}{}{ⓔkv̩˧ʂe˥\$}\formedesurface{kv̩˧ʂe˥}\newline
\classe{名词}\ton{H\$}
\paradigme{\pcmn{:} \p{}}
\begin{définition}\peng{Flea.}\end{définition}
\begin{définition}\pcmn{跳蚤}\end{définition}
\begin{définition}\pfra{Puce.}\end{définition}
\end{entrée}

\begin{entrée}
{kv̩˩tɑ˩}{}{ⓔkv̩˩tɑ˩}\formedesurface{kv̩˩tɑ˩˥}\newline
\classe{动词}\ton{L}\begin{définition}\peng{To assemble, to group, to bring together (e.g. after felling trees, putting pieces of timber together).}\end{définition}
\begin{définition}\pcmn{集中在一起(如:砍木材后,把木材堆在一起)}\end{définition}
\begin{définition}\pfra{Regrouper, rassembler (ex.: des troncs, après leur abattage).}\end{définition}
\end{entrée}

\begin{entrée}
{kv̩˧tsʰɑ˥\$}{}{ⓔkv̩˧tsʰɑ˥\$}\formedesurface{kv̩˧tsʰɑ˥}\newline
\classe{名词}\ton{H\$}\begin{définition}\peng{Family name (that of the Muli feudal lord, belonging to the Pumi/Prinmi ethnic group).}\end{définition}
\begin{définition}\pcmn{一个姓(木里土司,普米族,的姓)}\end{définition}
\begin{définition}\pfra{Nom de clan/famille étendue. Ce nom était celui de la famille des seigneurs (pumi/prinmi) de Muli.}\end{définition}
\begin{exemple}\pnru{kv̩˧tsʰɑ˧=ɻ̍˥\$}\hspace{5pt}\peng{the /kv̩˧tsʰɑ˥\$/ clan, the /kv̩˧tsʰɑ˥\$/ family}\hspace{5pt}\pcmn{|fv{/kv̩˧tsʰɑ˥\$/}家族}\hspace{5pt}\pfra{le clan /kv̩˧tsʰɑ˥\$/, la famille /kv̩˧tsʰɑ˥\$/}\end{exemple}
\begin{exemple}\pnru{kv̩˧tsʰɑ˧=ɻ̍˧ pi˥-zo˩!}\hspace{5pt}\peng{They were called “the /kv̩˧tsʰɑ˥\$/ family"!}\hspace{5pt}\pcmn{人家把他们称作“|fv{/kv̩˧tsʰɑ˥\$/}家族”!}\hspace{5pt}\pfra{On les appelait «les /kv̩˧tsʰɑ˥\$/»!}\end{exemple}
\begin{exemple}\pnru{ɬi˧di˩ ʈʂv˩fv˩, | mv˧ɭɯ˩ | kv̩˧tsʰɑ˥!}\hspace{5pt}\peng{In Yongning, the Prefect [is the authority]; in Muli, the |fv{/kv̩˧tsʰɑ˥\$/} family [is the authority]! (A saying that describes the situation towards the middle of the 20th century.)}\hspace{5pt}\pcmn{在永宁,是知府(说的算)。在木里,是|fv{/kv̩˧tsʰɑ˥\$/}家族(说的算)!}\hspace{5pt}\pfra{A Yongning, c’est le Préfet qui décide; à Muli, c’est la famille |fv{/kv̩˧tsʰɑ˥\$/}! (Dicton qui résume la situation vers le milieu du XXe siècle.)}\end{exemple}
\end{entrée}

\begin{entrée}
{kv̩˧tsʰɤ˩}{}{ⓔkv̩˧tsʰɤ˩}\formedesurface{kv̩˧tsʰɤ˩}\newline
\classe{名词}\ton{L\#}
\paradigme{\pcmn{:} \p{}}
\begin{définition}\peng{Head of garlic.}\end{définition}
\begin{définition}\pcmn{蒜头}\end{définition}
\begin{définition}\pfra{Tête d'ail.}\end{définition}
\end{entrée}

\begin{entrée}
{kv̩˧ʈʂɯ˧˥}{}{ⓔkv̩˧ʈʂɯ˧˥}\formedesurface{kv̩˧ʈʂɯ˧˥}\newline
\classe{名词}\ton{MH\#}
\paradigme{\pcmn{:} \p{}}
\begin{définition}\peng{(finger)nail, (toe)nail.}\end{définition}
\begin{définition}\pcmn{指甲}\end{définition}
\begin{définition}\pfra{Ongle.}\end{définition}
\end{entrée}

\begin{entrée}
{kwɑ˧fæ˩}{}{ⓔkwɑ˧fæ˩}\formedesurface{kwɑ˧fæ˩}\newline
\classe{名词}\ton{L\#}\begin{définition}\peng{Name of a hotel.}\end{définition}
\begin{définition}\pcmn{官房(汉语借词),酒店名称}\end{définition}
\begin{définition}\pfra{Nom d'un hôtel.}\end{définition}
\begin{exemple}\pnru{kwɑ˧fæ˩}\hspace{5pt}\peng{the abridged name of a five-star hotel where one of the main consultant's daughters works}\hspace{5pt}\pcmn{丽江官房大酒店的简称。注:发音合作人的女儿在丽江官房大酒店工作。}\hspace{5pt}\pfra{le nom abrégé d'un hôtel cinq étoiles où travaille l'une des filles de la consultante principale.}\end{exemple}
\end{entrée}

\begin{entrée}
{kwɑ˧tsʰɑ˧}{}{ⓔkwɑ˧tsʰɑ˧}\formedesurface{kwɑ˧tsʰɑ˧}\newline
\classe{名词}\ton{M}
\paradigme{\pcmn{:} \p{}}
\begin{définition}\peng{Coffin.}\end{définition}
\begin{définition}\pcmn{棺材(汉语借词)}\end{définition}
\begin{définition}\pfra{Cercueil.}\end{définition}
\begin{exemple}\pnru{kwɑ˧tsʰɑ˧, | hĩ˧-mo˩-kʰɯ˩-di˩ ɲi˩!}\hspace{5pt}\peng{The coffin is the thing in which the corpse is put! / The coffin is the thing to put the corpse!}\hspace{5pt}\pcmn{棺材,是装尸体的! / 棺材,是用来装尸体的!}\hspace{5pt}\pfra{Le cercueil, c'est là où on met le cadavre! / Le cercueil, c'est l'objet qui accueille le cadavre!}\end{exemple}
\end{entrée}

\begin{entrée}
{kwæ˧}{}{ⓔkwæ˧}\formedesurface{kwæ˧}\newline
\classe{动词}\ton{M}\begin{définition}\peng{To take care of, to take charge of.}\end{définition}
\begin{définition}\pcmn{管(汉语借词)}\end{définition}
\begin{définition}\pfra{S'occuper de, se charger de.}\end{définition}
\end{entrée}

\begin{entrée}
{kwæ˧fæ˥}{}{ⓔkwæ˧fæ˥}\formedesurface{kwæ˧fæ˥}\newline
\classe{名词}\ton{H\#}
\paradigme{\pcmn{:} \p{}}
\begin{définition}\peng{Medium-sized beam.}\end{définition}
\begin{définition}\pcmn{中等大小的梁}\end{définition}
\begin{définition}\pfra{Poutre intermédiaire: pièce de charpente horizontale, posée sur une poutre maîtresse, et soutenant deux des poutres du toit, /ʐv̩˩ɭɯ˧/.}\end{définition}
\end{entrée}

\begin{entrée}
{kwæ˧pæ˥}{}{ⓔkwæ˧pæ˥}\formedesurface{kwæ˧pæ˥}\newline
\classe{名词}\ton{H\#}
\paradigme{\pcmn{:} \p{}}
\begin{définition}\peng{Harrow (made of wood).}\end{définition}
\begin{définition}\pcmn{耙(可能是汉语借词。原来借来的词:刮板?? 刮耙??)}\end{définition}
\begin{définition}\pfra{Herse en bois (vraisemblablement un emprunt au chinois; terme emprunté: pas identifié avec certitude).}\end{définition}
\end{entrée}

\begin{entrée}
{kwæ˧tsɯ˧}{}{ⓔkwæ˧tsɯ˧}\formedesurface{kwæ˧tsɯ˧}\newline
\classe{名词}\ton{M}\begin{définition}\peng{Sunflower seed.}\end{définition}
\begin{définition}\pcmn{葵花瓜籽(汉语借词)}\end{définition}
\begin{définition}\pfra{Graine de tournesol.}\end{définition}
\end{entrée}

\begin{entrée}
{‑kwɤ}{}{ⓔ‑kwɤ}\formedesurface{--}\newline
\classe{连接词}\ton{0}\begin{définition}\peng{When.}\end{définition}
\begin{définition}\pcmn{……的时候}\end{définition}
\begin{définition}\pfra{Lorsque; ne peut s'employer seul, mais apparaît dans la formule \stylefv{/kwɤ-tɕɯ-lɑ/.}}\end{définition}
\end{entrée}

\begin{entrée}
{kwɤ˩α}{}{ⓔkwɤ˩α}\formedesurface{ɖɯ˧ kwɤ˩}\newline
\classe{量词}\ton{Lα}\begin{définition}\peng{A string, a cluster of.}\end{définition}
\begin{définition}\pcmn{量词:串}\end{définition}
\begin{définition}\pfra{Classificateur des objets tressés, enfilés ou liés ensemble.}\end{définition}
\begin{exemple}\pnru{kv̩˧ | ɖɯ˧-kwɤ˩}\hspace{5pt}\peng{a braid of garlic}\hspace{5pt}\pcmn{一辫大蒜}\hspace{5pt}\pfra{une tresse d'aïl}\end{exemple}
\begin{exemple}\pnru{lɑ˧tsɯ˥ | ɖɯ˧-kwɤ˩}\hspace{5pt}\peng{a braid of hot peppers}\hspace{5pt}\pcmn{一辫辣椒}\hspace{5pt}\pfra{une ligature de piments}\end{exemple}
\begin{exemple}\pnru{ʈʂʰɯ˧-kwɤ˥}\hspace{5pt}\peng{|fg{dem} \_ (tone: H\# / H\$)}\hspace{5pt}\pcmn{指示代词 \_}\hspace{5pt}\pfra{|fg{dem} \_ (ton: H\# / H\$)}\end{exemple}
\end{entrée}

\begin{entrée}
{kwɤ˩α}{₁}{ⓔkwɤ˩αⓗ1}\formedesurface{kwɤ˩˥}\newline
\classe{动词}\ton{Lα}
1\begin{définition}\peng{To throw away (rubbish).}\end{définition}
\begin{définition}\pcmn{扔掉}\end{définition}
\begin{définition}\pfra{Jeter.}\end{définition}
\begin{exemple}\pnru{mv̩˩tɕo˧ kwɤ˩}\hspace{5pt}\peng{to throw away (rubbish)}\hspace{5pt}\pcmn{扔掉(垃圾)}\hspace{5pt}\pfra{jeter (détritus); littéralement: «jeter en bas»}\end{exemple}
\begin{exemple}\pnru{tso˧∼tso˧ kwɤ˥}\hspace{5pt}\peng{to throw stuff away}\hspace{5pt}\pcmn{扔东西}\hspace{5pt}\pfra{jeter des choses}\end{exemple}
\end{entrée}

\begin{entrée}
{kwɤ˩α}{₂}{ⓔkwɤ˩αⓗ2}\formedesurface{kwɤ˩˥}\newline
\classe{动词}\ton{Lα}
2\begin{définition}\peng{To manage, to be in charge of, to take care of.}\end{définition}
\begin{définition}\pcmn{管(汉语借词)}\end{définition}
\begin{définition}\pfra{S'occuper de, gérer, superviser.}\end{définition}
\begin{exemple}\pnru{ɖɯ˧-kʰwɤ˧ kwɤ˥}\hspace{5pt}\peng{to supervise a bit}\hspace{5pt}\pcmn{管一些}\hspace{5pt}\pfra{s'occuper un peu/s'occuper d'une partie (combinaison élicitée pour vérifier que le ton est L et non LM)}\end{exemple}
\end{entrée}

\begin{entrée}
{kwɤ˧ɭɯ˩}{}{ⓔkwɤ˧ɭɯ˩}\formedesurface{kwɤ˧ɭɯ˩}\newline
\classe{名词}\ton{L\#}
\paradigme{\pcmn{:} \p{}}
\begin{définition}\peng{Jug; jar; pitcher; also: treasure, valuable possession.}\end{définition}
\begin{définition}\pcmn{坛子、罐子 (陶器),宝贝}\end{définition}
\begin{définition}\pfra{Jarre; trésor, objet de grande valeur.}\end{définition}
\begin{exemple}\pnru{ʈʂʰɯ˧ | njɤ˧ kwɤ˧ɭɯ˩ ɲi˩!}\hspace{5pt}\peng{(S)he is my treasure! (About a child)}\hspace{5pt}\pcmn{他是我宝贝!(母亲说孩子是她的宝贝)}\hspace{5pt}\pfra{C'est mon petit trésor! (dit au sujet d'un enfant)}\end{exemple}
\end{entrée}

\begin{entrée}
{kwɤ˧pɤ˧}{}{ⓔkwɤ˧pɤ˧}\formedesurface{kwɤ˧pɤ˧}\newline
\classe{名词}\ton{M}
\paradigme{\pcmn{:} \p{}}
\begin{définition}\peng{Teaching, explanation.}\end{définition}
\begin{définition}\pcmn{解释,教导、教诲}\end{définition}
\begin{définition}\pfra{Explication, enseignement.}\end{définition}
\begin{exemple}\pnru{kwɤ˧pɤ˧ ɖɯ˧-kʰwɤ˥ lɑ˩}\hspace{5pt}\peng{to provide an explanation, to teach something}\hspace{5pt}\pcmn{解释一个道理、教一件事}\hspace{5pt}\pfra{enseigner quelque chose à quelqu'un, expliquer quelque chose à quelqu'un}\end{exemple}
\begin{exemple}\pnru{kwɤ˧pɤ˧ ɖɯ˧-kʰwɤ˥ | tʰi˧-lɑ˩-ɻ̍˩}\hspace{5pt}\peng{As above: to provide an explanation, to teach something}\hspace{5pt}\pcmn{同上:解释一个道理、教一件事}\hspace{5pt}\pfra{comme ci-dessus: enseigner quelque chose à quelqu'un, expliquer quelque chose à quelqu'un}\end{exemple}
\begin{exemple}\pnru{kwɤ˧pɤ˧ lɑ˧˥}\hspace{5pt}\peng{to teach}\hspace{5pt}\pcmn{教、解释}\hspace{5pt}\pfra{enseigner}\end{exemple}
\end{entrée}

\begin{entrée}
{kwɤ˩to˥}{}{ⓔkwɤ˩to˥}\formedesurface{kwɤ˩to˥}\newline
\classe{名词}\ton{LH}
\paradigme{\pcmn{:} \p{}}
\begin{définition}\peng{Jawbone, mandible, lower jaw.}\end{définition}
\begin{définition}\pcmn{颌骨}\end{définition}
\begin{définition}\pfra{Mandibule, mâchoire inférieure.}\end{définition}
\end{entrée}

\begin{entrée}
{‑kwɤ˧tɕɯ˥}{}{ⓔ‑kwɤ˧tɕɯ˥}\formedesurface{kwɤ˧tɕɯ˥}\newline
\classe{连接词}\ton{H\#}\begin{définition}\peng{After; because, since, as.}\end{définition}
\begin{définition}\pcmn{因为,由于,既然}\end{définition}
\begin{définition}\pfra{Comme; après; puisque.}\end{définition}
\begin{exemple}\pnru{-kwɤ˧tɕɯ˥-lɑ˩}\hspace{5pt}\peng{same meaning}\hspace{5pt}\pcmn{同上}\hspace{5pt}\pfra{même sens}\end{exemple}
\begin{exemple}\pnru{ʈʂʰɯ˧ | go˩-kwɤ˩tɕɯ˥-lɑ˩, | hɑ˧ mɤ˧-dzɯ˥.}\hspace{5pt}\peng{Because (s)he is ill, (s)he does not eat.}\hspace{5pt}\pcmn{他病了,吃不下饭。}\hspace{5pt}\pfra{Comme il est malade, il ne mange pas.}\end{exemple}
\begin{exemple}\pnru{ʈʂʰɯ˧ne˧-ʝi˥ | pi˧-kwɤ˩tɕɯ˩-lɑ˩, | wɤ˩˥ | lɑ˧hɑ˥ | ɖɯ˧-kʰwɤ˧ ʐwɤ˧˥.}\hspace{5pt}\peng{After he said that, he went on to say something different / he changed his mind and said something quite different.}\hspace{5pt}\pcmn{他这样说完以后,又讲了些其它的。}\hspace{5pt}\pfra{ayant dit cela, il dit à nouveau autre chose (après avoir dit ça, il a dit ajouté autre chose!)}\end{exemple}
\begin{exemple}\pnru{ʈʂʰɯ˧ | tʰi˧-dzi˩-kwɤ˩-tɕɯ˩, | ɖɯ˧-kʰwɤ˧ ʐwɤ˧-ɻ̍˥: | “sɤ˧sɤ˧˥ | ʐwæ˧˥!"}\hspace{5pt}\peng{After he got seated, he said the following: “How comfortable!"}\hspace{5pt}\pcmn{他坐下后,说了这么一句:“真舒服!”}\hspace{5pt}\pfra{après s’être assis/lorsqu’il fut assis, il dit une phrase: «quel confort!»}\end{exemple}
\end{entrée}

\begin{entrée}
{kʰɤ˧˥}{₁}{ⓔkʰɤ˧˥ⓗ1}\formedesurface{kʰɤ˧˥}\newline
\classe{动词}\ton{MH}
1\begin{définition}\peng{To put out (a fire).}\end{définition}
\begin{définition}\pcmn{灭(火)}\end{définition}
\begin{définition}\pfra{Éteindre le foyer.}\end{définition}
\end{entrée}

\begin{entrée}
{kʰɤ˧˥}{₂}{ⓔkʰɤ˧˥ⓗ2}\formedesurface{kʰɤ˧˥}\newline
\classe{名词}\ton{MH}
2
\paradigme{\pcmn{:} \p{}}
\begin{définition}\peng{Basket carried on the back.}\end{définition}
\begin{définition}\pcmn{背篓}\end{définition}
\begin{définition}\pfra{Hotte.}\end{définition}
\end{entrée}

\begin{entrée}
{kʰɤ˧˥α}{}{ⓔkʰɤ˧˥α}\formedesurface{ɖɯ˧ kʰɤ˧˥}\newline
\classe{量词}\ton{MHα}\begin{définition}\peng{A basket of.}\end{définition}
\begin{définition}\pcmn{量词:筐}\end{définition}
\begin{définition}\pfra{Classificateur des cageots.}\end{définition}
\end{entrée}

\begin{entrée}
{kʰɤ˧ɕjɤ˧}{}{ⓔkʰɤ˧ɕjɤ˧}\formedesurface{kʰɤ˧ɕjɤ˧}\newline
\classe{动词}\ton{M}\begin{définition}\peng{To spend (money).}\end{définition}
\begin{définition}\pcmn{花掉}\end{définition}
\begin{définition}\pfra{Dépenser.}\end{définition}
\begin{exemple}\pnru{kʰɤ˧ɕjɤ˧-ze˩!}\hspace{5pt}\peng{|fg{pfv}}\hspace{5pt}\pcmn{花掉了!}\hspace{5pt}\pfra{|fg{pfv}}\end{exemple}
\begin{exemple}\pnru{le˧-kʰɤ˧ɕjɤ˧ | le˧-se˩-kʰɯ˩}\hspace{5pt}\peng{to spend all the money}\hspace{5pt}\pcmn{全部都花完}\hspace{5pt}\pfra{tout dépenser (argent)}\end{exemple}
\begin{exemple}\pnru{mɤ˧-dʑo˧, | ɖwæ˩-mɤ˧-zo˧! | tʰi˧-ʂe˧ tʰi˧-kʰɤ˧ɕjɤ˧-tsæ˥-ɲi˩!}\hspace{5pt}\peng{If we have nothing/if we are poor, we should not worry! Go ahead: earn some money and spend it! (This is one of the teachings of the elders: money is something that circulates, not something to be accumulated for its own sake.)}\hspace{5pt}\pcmn{如果没有(钱),不用担心!要赚也要花!(这是长辈的一个教诲:钱赚了又花、花了又赚,不是来积累的。)}\hspace{5pt}\pfra{Si on n'a rien/qu'on est pauvre, il ne faut pas se faire de souci! On n'a qu'à aller gagner (littéralement ‘chercher’) puis dépenser! (Philosophie qu'enseignent les aînés: l'argent c'est quelque chose qui circule, pas quelque chose qui se thésaurise.)}\end{exemple}
\end{entrée}

\begin{entrée}
{kʰɤ˧mi˥\$}{}{ⓔkʰɤ˧mi˥\$}\formedesurface{kʰɤ˧mi˥}\newline
\classe{名词}\ton{H\$}
\paradigme{\pcmn{:} \p{}}
\begin{définition}\peng{Large basket carried on the back.}\end{définition}
\begin{définition}\pcmn{大背篓}\end{définition}
\begin{définition}\pfra{Grande hotte.}\end{définition}
\end{entrée}

\begin{entrée}
{kʰɤ˩njɤ˩∼kʰɤ˧njɤ˧}{}{ⓔkʰɤ˩njɤ˩∼kʰɤ˧njɤ˧}\formedesurface{kʰɤ˩njɤ˩kʰɤ˧njɤ˧}\newline
\classe{形容词}\ton{L-M}\begin{définition}\peng{Supple (movement).}\end{définition}
\begin{définition}\pcmn{柔软(动作)}\end{définition}
\begin{définition}\pfra{Souple (mouvement).}\end{définition}
\end{entrée}

\begin{entrée}
{kʰɤ˧ʂɯ˧}{}{ⓔkʰɤ˧ʂɯ˧}\formedesurface{kʰɤ˧ʂɯ˧}\newline
\classe{动词}\ton{M}\begin{définition}\peng{To begin.}\end{définition}
\begin{définition}\pcmn{开始(汉语借词)}\end{définition}
\begin{définition}\pfra{Commencer.}\end{définition}
\end{entrée}

\begin{entrée}
{kʰɤ˧zo˥\$}{}{ⓔkʰɤ˧zo˥\$}\formedesurface{kʰɤ˧zo˥}\newline
\classe{名词}\ton{H\$}
\paradigme{\pcmn{:} \p{}}
\begin{définition}\peng{Small basket carried on the back.}\end{définition}
\begin{définition}\pcmn{小背篓}\end{définition}
\begin{définition}\pfra{Petite hotte.}\end{définition}
\end{entrée}

\begin{entrée}
{kʰi˥}{}{ⓔkʰi˥}\formedesurface{kʰi˧}\newline
\classe{动词}\ton{H}\begin{définition}\peng{To separate, to take apart (e.g. to separate fibers of linen to make thread; to split lumber into planks).}\end{définition}
\begin{définition}\pcmn{拆开、分离(几根线),划开(木根成木板)}\end{définition}
\begin{définition}\pfra{Séparer, défaire (des fibres de lin: on sépare les fibres pour faire du fil), débiter (débiter une grume en planches).}\end{définition}
\begin{exemple}\pnru{sɑ˧ | le˧-kʰi˥}\hspace{5pt}\peng{to separate linen fibres (to make thread)}\hspace{5pt}\pcmn{拆开粗麻(为了纺出细麻线)}\hspace{5pt}\pfra{défaire des fibres de lin (pour fabriquer du fil)}\end{exemple}
\end{entrée}

\begin{entrée}
{kʰi˥}{₁}{ⓔkʰi˥ⓗ1}\formedesurface{kʰi˧}\newline
\classe{名词}\ton{\#H}
1
\paradigme{\pcmn{:} \p{}}
\begin{définition}\peng{Door.}\end{définition}
\begin{définition}\pcmn{门}\end{définition}
\begin{définition}\pfra{Porte.}\end{définition}
\begin{exemple}\pnru{kʰi˧-zo\#˥}\hspace{5pt}\peng{small door}\hspace{5pt}\pcmn{小门}\hspace{5pt}\pfra{petite porte}\end{exemple}
\end{entrée}

\begin{entrée}
{kʰi˥}{₂}{ⓔkʰi˥ⓗ2}\formedesurface{kʰi˧}\newline
\classe{名词}\ton{\#H}
2\begin{définition}\peng{Edge (monosyllable).}\end{définition}
\begin{définition}\pcmn{边(单音节)}\end{définition}
\begin{définition}\pfra{Bord (monosyllabe).}\end{définition}
\end{entrée}

\begin{entrée}
{kʰi˧˥}{}{ⓔkʰi˧˥}\formedesurface{kʰi˧˥}\newline
\classe{动词}\ton{MH}\begin{définition}\peng{Past form of verb ‘to leave'.}\end{définition}
\begin{définition}\pcmn{走(过去式)}\end{définition}
\begin{définition}\pfra{Forme passée du verbe ‘partir'.}\end{définition}
\begin{exemple}\pnru{ʈʂʰɯ˧ | zo˩qo˧ kʰi˧?}\hspace{5pt}\peng{Where has (s)he gone to? / Where has (s)he left for?}\hspace{5pt}\pcmn{他到哪里去了?}\hspace{5pt}\pfra{Elle/il est parti où?}\end{exemple}
\begin{exemple}\pnru{no˧ | tsʰi˧ɲi˧ | ɑ˩pʰo˩˥ | ə˩-kʰi˩˥?}\hspace{5pt}\peng{Did you go outside today? / Did you take a stroll today?}\hspace{5pt}\pcmn{你今天出去了吗?}\hspace{5pt}\pfra{tu es allé faire un tour dehors, aujourd'hui?/tu es sorti, aujourd'hui? (Contexte: question posée par un consultant alors que je le raccompagne après une séance de travail vespérale)}\end{exemple}
\end{entrée}

\begin{entrée}
{kʰi˧bɤ\#˥}{}{ⓔkʰi˧bɤ\#˥}\formedesurface{kʰi˧bɤ˧}\newline
\classe{名词}\ton{\#H}
\paradigme{\pcmn{:} \p{}}
\begin{définition}\peng{Threshold.}\end{définition}
\begin{définition}\pcmn{门槛}\end{définition}
\begin{définition}\pfra{Seuil.}\end{définition}
\end{entrée}

\begin{entrée}
{kʰi˧-bv̩˧lv̩˩}{}{ⓔkʰi˧-bv̩˧lv̩˩}\formedesurface{kʰi˧bv̩˧lv̩˩}\newline
\classe{名词}\ton{-L\#}
\paradigme{\pcmn{:} \p{}}
\begin{définition}\peng{Hinge.}\end{définition}
\begin{définition}\pcmn{(门的)合页}\end{définition}
\begin{définition}\pfra{Gonds (d'une porte).}\end{définition}
\end{entrée}

\begin{entrée}
{‑kʰi˧∼kʰi˧}{}{ⓔ‑kʰi˧∼kʰi˧}\formedesurface{kʰi˧kʰi˧}\newline
\classe{}\ton{\#H}\begin{définition}\peng{Around, close to, near, nearby.}\end{définition}
\begin{définition}\pcmn{周围、左右、旁边}\end{définition}
\begin{définition}\pfra{Aux alentours de, au bord de, auprès de.}\end{définition}
\begin{exemple}\pnru{ʑi˧qʰwɤ˧-kʰi˧∼kʰi˧}\hspace{5pt}\peng{near the house, in the vicinity of the house}\hspace{5pt}\pcmn{房子周围}\hspace{5pt}\pfra{aux alentours de la maison}\end{exemple}
\begin{exemple}\pnru{njɤ˧-bv̩˧ | kʰi˧∼kʰi˧}\hspace{5pt}\peng{near me, around me}\hspace{5pt}\pcmn{我的周围}\hspace{5pt}\pfra{à côté de moi, autour de moi}\end{exemple}
\begin{exemple}\pnru{ʐɤ˩mi˩-kʰi˩∼kʰi˩ se˩˥}\hspace{5pt}\peng{to walk on the roadside, to walk by the side of the road}\hspace{5pt}\pcmn{走在马路边}\hspace{5pt}\pfra{marcher au bord de la route}\end{exemple}
\end{entrée}

\begin{entrée}
{kʰi˧mi˧}{}{ⓔkʰi˧mi˧}\formedesurface{kʰi˧mi˧}\newline
\classe{名词}\ton{M}
\paradigme{\pcmn{:} \p{}}
\begin{définition}\peng{Main entrance, main door.}\end{définition}
\begin{définition}\pcmn{大门}\end{définition}
\begin{définition}\pfra{Grande porte (ex.: porte d'entrée d'une maison).}\end{définition}
\end{entrée}

\begin{entrée}
{kʰi˧qʰv̩\#˥}{}{ⓔkʰi˧qʰv̩\#˥}\formedesurface{kʰi˧qʰv̩˧}\newline
\classe{名词}\ton{\#H}\begin{définition}\peng{Door.}\end{définition}
\begin{définition}\pcmn{门口}\end{définition}
\begin{définition}\pfra{Porte.}\end{définition}
\begin{exemple}\pnru{kʰi˧qʰv˧ tʰv˧-ɲi˥}\hspace{5pt}\peng{to reach the door, to get to the door (context: reaching the door of one's home, getting back home from a long trip)}\hspace{5pt}\pcmn{到达(家)门口(情景:从远方回家,到达家门)}\hspace{5pt}\pfra{parvenir à la porte, atteindre la porte (contexte: retour d'un lointain périple)}\end{exemple}
\begin{exemple}\pnru{ɑ˩ʁo˧ kʰi˧qʰv˧ tʰv˧}\hspace{5pt}\peng{to reach the door of the house}\hspace{5pt}\pcmn{到达家门口}\hspace{5pt}\pfra{parvenir à la porte de la maison}\end{exemple}
\begin{exemple}\pnru{no˧ | kʰi˧qʰv˧ le˧ tʰv˧, | ə˧tso˧ ʝi˧ | mɤ˧-tsʰɯ˩? | ɑ˩ʁo˧ jo˩!}\hspace{5pt}\peng{You have reached the door of the house, so why won't you enter? Come in!}\hspace{5pt}\pcmn{既然到达家门口了,你怎么不进去?来吧!}\hspace{5pt}\pfra{Eh bien puisque te voilà parvenu au seuil, pourquoi que tu entres pas? Viens donc dans la maison / Entre donc! (Contexte: quelqu'un hésite au seuil d'une maison; l'hôte l'encourage.)}\end{exemple}
\end{entrée}

\begin{entrée}
{kʰi˧-qʰwɤ˩}{}{ⓔkʰi˧-qʰwɤ˩}\formedesurface{kʰi˧qʰwɤ˩}\newline
\classe{名词}\ton{L\#}
\paradigme{\pcmn{:} \p{}}
\begin{définition}\peng{Hinge.}\end{définition}
\begin{définition}\pcmn{门的合页}\end{définition}
\begin{définition}\pfra{Gonds (d'une porte).}\end{définition}
\end{entrée}

\begin{entrée}
{kʰi˧tɕʰɯ˩-mo˩}{}{ⓔkʰi˧tɕʰɯ˩-mo˩}\formedesurface{kʰi˧tɕʰɯ˩mo˩}\newline
\classe{名词}\ton{L\#-}\begin{définition}\peng{A poisonous mushroom.}\end{définition}
\begin{définition}\pcmn{一种有毒的菌子}\end{définition}
\begin{définition}\pfra{Un champignon vénéneux.}\end{définition}
\begin{exemple}\pnru{ʈʂæ˧mo˧-kʰi˧tɕʰɯ˩-mo˩}\hspace{5pt}\peng{same meaning}\hspace{5pt}\pcmn{同上}\hspace{5pt}\pfra{même sens}\end{exemple}
\end{entrée}

\begin{entrée}
{kʰo˥}{}{ⓔkʰo˥}\formedesurface{kʰo˧}\newline
\classe{动词}\ton{H}\begin{définition}\peng{To spread (e.g. to do a bed; to spread/scatter objects all over the floor).}\end{définition}
\begin{définition}\pcmn{铺(床……)、铺得满地(果子、工具……)}\end{définition}
\begin{définition}\pfra{Étendre (un matelas), étaler (des fruits, des outils… partout par terre).}\end{définition}
\begin{exemple}\pnru{kʰwæ˧ɻæ˧ kʰo˧}\hspace{5pt}\peng{to spread a mat}\hspace{5pt}\pcmn{铺垫子}\hspace{5pt}\pfra{étendre une natte}\end{exemple}
\end{entrée}

\begin{entrée}
{kʰo˧bɤ˧}{}{ⓔkʰo˧bɤ˧}\formedesurface{kʰo˧bɤ˧}\newline
\classe{名词}\ton{M}
\paradigme{\pcmn{:} \p{}}
\begin{définition}\peng{Home (solemn, formal word).}\end{définition}
\begin{définition}\pcmn{家(文言):母亲生活的空间:有家人,有火塘,有母亲在那里生活的那个空间}\end{définition}
\begin{définition}\pfra{Foyer. Mot ancien, qui n'est utilisé que dans un registre soutenu; il désigne un espace de vie.}\end{définition}
\begin{exemple}\pnru{dʑi˧tɕʰi˧ le˧-gwɤ˩ | qo˩ tɑ˧-ze˥, | njɤ˧-ɕi˩ ə˩mɑ˩ kʰo˩bɤ˩ dʑɤ˩. |}\hspace{5pt}\peng{After long journeys come to an end, how beautiful is the home of the mother who raised me! (A song about homesickness.)}\hspace{5pt}\pcmn{很远的路程结束了,养我的母亲家里多么美!}\hspace{5pt}\pfra{Au terme de lointains périples, comme il est beau, le foyer de la mère qui m'a élevée! (Chanson au sujet de la nostalgie du foyer.)}\end{exemple}
\begin{exemple}\pnru{dʑi˧tɕʰi˧ le˧-gwɤ˩ | qo˩ tɑ˧-ze˥, | njɤ˧-ɕi˩ ə˩mɑ˩ kʰo˩bɤ˩-qo˩. |}\hspace{5pt}\peng{as above}\hspace{5pt}\pcmn{同上}\hspace{5pt}\pfra{idem ci-dessus}\end{exemple}
\begin{exemple}\pnru{dʑi˧tɕʰi˧ le˧-gwɤ˩ qo˩ tɑ˩-ze˩, | njɤ˧-ɕi˩ ə˩mɑ˩ kʰo˩bɤ˩ dʑɤ˩. |}\hspace{5pt}\peng{as above}\hspace{5pt}\pcmn{同上}\hspace{5pt}\pfra{idem}\end{exemple}
\end{entrée}

\begin{entrée}
{kʰo˧lo˧}{}{ⓔkʰo˧lo˧}\formedesurface{kʰo˧lo˧}\newline
\classe{名词}\ton{M}
\paradigme{\pcmn{:} \p{}}
\begin{définition}\peng{Prayer wheel.}\end{définition}
\begin{définition}\pcmn{转经筒}\end{définition}
\begin{définition}\pfra{Moulin à prière (aussi bien les très grands, fixés à des axes verticaux dans les monastères, que les plus petits, tenus à la main).}\end{définition}
\end{entrée}

\begin{entrée}
{kʰɯ˧˥}{₁}{ⓔkʰɯ˧˥ⓗ1}\newline
\classe{动词}
1
\sens{1}
\begin{définition}\peng{To put into (e.g. to put into a bag); to dibble in seeds.}\end{définition}
\begin{définition}\pcmn{放,装(如:装进袋里),点种,收下}\end{définition}
\begin{définition}\pfra{Mettre, mettre dans (ex.: mettre de la farine dans une casserole); libérer, lâcher (ex.: un poulet qu'on tenait par les pattes); semer en enfonçant les graines; ranger, remettre à sa place.}\end{définition}
\begin{exemple}\pnru{kʰɯ˩∼kʰɯ˧˥}\hspace{5pt}\peng{|fg{red}}\hspace{5pt}\pcmn{|fg{red}}\hspace{5pt}\pfra{|fg{red}}\end{exemple}
\begin{exemple}\pnru{qwɤ˧-qo˧ | si˧ tʰi˧-kʰɯ˧˥}\hspace{5pt}\peng{to add wood into the fire}\hspace{5pt}\pcmn{放木头在火中}\hspace{5pt}\pfra{mettre/ajouter du bois dans le feu}\end{exemple}\sens{2}
\begin{définition}\peng{To allow; to let; to cause (causative value).}\end{définition}
\begin{définition}\pcmn{让,使动}\end{définition}
\begin{définition}\pfra{Autoriser; a aussi valeur causative, mais ce causatif issu du verbe ‘mettre’ paraît avoir un sens plus proche de ‘laisser’: par exemple ‘laisser sécher au soleil’ plutôt que ‘faire sécher au soleil’.}\end{définition}
\begin{exemple}\pnru{kʰv̩˩mi˩ zɯ˩∼zɯ˩˥, | le˧-ɖæ˥-kʰɯ˩! | hĩ˧-zɯ˧∼zɯ˥, | le˧-ʂæ˧-kʰɯ˥!}\hspace{5pt}\peng{Dog's lifespan was made shorter, and man's lifespan was made longer! (A summary of the legend “How dog and man exchanged their lifespans".)}\hspace{5pt}\pcmn{够的寿命,变短了/使得变短!(而)人的寿命,变长了/使得变长!(《狗和人交换寿命》故事的一个提要)}\hspace{5pt}\pfra{La vie des chiens s'en est trouvée écourtée, et celle des hommes allongée! (Résumé en quelques mots du récit «Le chien échange sa longévité avec l'homme»)}\end{exemple}
\begin{exemple}\pnru{hwæ˧ kʰɯ˧ ə˥-bi˩? | - hwæ˧ kʰɯ˧-bi˥!}\hspace{5pt}\peng{Do you agree to buy? - Yes!}\hspace{5pt}\pcmn{(你)让买吗? - 让买!}\hspace{5pt}\pfra{Tu es d'accord pour acheter? - Oui!}\end{exemple}
\begin{exemple}\pnru{tɕʰi˧ kʰɯ˧ ə˥-bi˩?}\hspace{5pt}\peng{Do you agree to sell?}\hspace{5pt}\pcmn{(你)让卖吗?}\hspace{5pt}\pfra{Tu es d'accord pour vendre?}\end{exemple}
\begin{exemple}\pnru{dzɯ˧ kʰɯ˩ ə˩-bi˩?}\hspace{5pt}\peng{Do you agree to eat?}\hspace{5pt}\pcmn{(你)让吃吗?}\hspace{5pt}\pfra{Tu es d'accord pour manger?}\end{exemple}
\begin{exemple}\pnru{tɕi˩ kʰɯ˥ ə˩-bi˩?}\hspace{5pt}\peng{Do you agree to write?}\hspace{5pt}\pcmn{(你)让写吗?}\hspace{5pt}\pfra{Tu es d'accord pour écrire?}\end{exemple}
\begin{exemple}\pnru{ʈʰɯ˩ kʰɯ˩ ə˥-bi˩?}\hspace{5pt}\peng{Do you agree to drink?}\hspace{5pt}\pcmn{(你)让喝吗?}\hspace{5pt}\pfra{Tu es d'accord pour boire?}\end{exemple}
\begin{exemple}\pnru{ʐv̩˧ kʰɯ˥ ə˩-bi˩?}\hspace{5pt}\peng{Do you agree to sew?}\hspace{5pt}\pcmn{(你)让缝吗?}\hspace{5pt}\pfra{Tu es d'accord pour coudre?}\end{exemple}
\end{entrée}

\begin{entrée}
{kʰɯ˧˥}{₂}{ⓔkʰɯ˧˥ⓗ2}\formedesurface{kʰɯ˧˥}\newline
\classe{动词}\ton{MH}
2\begin{définition}\peng{To throw.}\end{définition}
\begin{définition}\pcmn{甩、扔(石头)}\end{définition}
\begin{définition}\pfra{Lancer, jeter.}\end{définition}
\begin{exemple}\pnru{le˧-kʰɯ˧-ze˥}\hspace{5pt}\peng{|fg{accomp} \_ |fg{pfv}}\hspace{5pt}\pcmn{甩了}\hspace{5pt}\pfra{|fg{accomp} \_ |fg{pfv}}\end{exemple}
\begin{exemple}\pnru{lv̩˧mi˧ kʰɯ˧˥}\hspace{5pt}\peng{to throw a stone}\hspace{5pt}\pcmn{扔石头}\hspace{5pt}\pfra{jeter une pierre}\end{exemple}
\end{entrée}

\begin{entrée}
{kʰɯ˧˥}{₃}{ⓔkʰɯ˧˥ⓗ3}\formedesurface{kʰɯ˧˥}\newline
\classe{动词}\ton{MH}
3\begin{définition}\peng{To wear (a bracelet).}\end{définition}
\begin{définition}\pcmn{戴(手镯)}\end{définition}
\begin{définition}\pfra{Porter (un bracelet).}\end{définition}
\begin{exemple}\pnru{le˧-kʰɯ˧-ze˥}\hspace{5pt}\peng{|fg{accomp} \_ |fg{pfv}}\hspace{5pt}\pcmn{戴了}\hspace{5pt}\pfra{|fg{accomp} \_ |fg{pfv}}\end{exemple}
\begin{exemple}\pnru{lo˩dʑo˧ kʰɯ˩}\hspace{5pt}\peng{to wear a bracelet}\hspace{5pt}\pcmn{戴手镯}\hspace{5pt}\pfra{porter un bracelet}\end{exemple}
\end{entrée}

\begin{entrée}
{kʰɯ˩}{}{ⓔkʰɯ˩}\formedesurface{kʰɯ˧}\newline
\classe{名词}\ton{L}
\paradigme{\pcmn{:} \p{}}
\begin{définition}\peng{Thread.}\end{définition}
\begin{définition}\pcmn{线}\end{définition}
\begin{définition}\pfra{Fil.}\end{définition}
\end{entrée}

\begin{entrée}
{kʰɯ˩β}{}{ⓔkʰɯ˩β}\formedesurface{ɖɯ˧ kʰɯ˩}\newline
\classe{量词}\ton{Lβ}\begin{définition}\peng{Classifier for threads.}\end{définition}
\begin{définition}\pcmn{量词:线(一根、一条)}\end{définition}
\begin{définition}\pfra{Brin (d'herbe, de fil, de ficelle…).}\end{définition}
\begin{exemple}\pnru{kʰɯ˧ | ɖɯ˧-kʰɯ˩}\hspace{5pt}\peng{a thread of string}\hspace{5pt}\pcmn{一根线}\hspace{5pt}\pfra{un brin de fil}\end{exemple}
\begin{exemple}\pnru{zɯ˧ | ɖɯ˧-kʰɯ˩}\hspace{5pt}\peng{a blade of grass}\hspace{5pt}\pcmn{一根草}\hspace{5pt}\pfra{un brin d'herbe}\end{exemple}
\begin{exemple}\pnru{bæ˩ ɖɯ˥-kʰɯ˩}\hspace{5pt}\peng{a thread of rope}\hspace{5pt}\pcmn{一条绳子}\hspace{5pt}\pfra{un brin de corde, un bout de corde}\end{exemple}
\begin{exemple}\pnru{kʰɯ˧ | ʈʂʰɯ˧-kʰɯ˧˥}\hspace{5pt}\peng{this thread (note: irregular tone pattern)}\hspace{5pt}\pcmn{这根线}\hspace{5pt}\pfra{ce brin (note: schéma tonal irrégulier)}\end{exemple}
\end{entrée}

\begin{entrée}
{kʰɯ˧di˧˥}{}{ⓔkʰɯ˧di˧˥}\formedesurface{kʰɯ˧di˧˥}\newline
\classe{名词}\ton{MH\#}
\paradigme{\pcmn{:} \p{}}
\begin{définition}\peng{Container (general term).}\end{définition}
\begin{définition}\pcmn{容器}\end{définition}
\begin{définition}\pfra{Récipient (terme générique).}\end{définition}
\end{entrée}

\begin{entrée}
{kʰɯ˧dv̩\#˥}{}{ⓔkʰɯ˧dv̩\#˥}\formedesurface{kʰɯ˧dv̩˧}\newline
\classe{名词}\ton{\#H}
\paradigme{\pcmn{:} \p{}}
\begin{définition}\peng{Cripple, lame person.}\end{définition}
\begin{définition}\pcmn{跛}\end{définition}
\begin{définition}\pfra{Boiteux.}\end{définition}
\begin{exemple}\pnru{kʰɯ˧dv̩˧-hĩ˧}\hspace{5pt}\peng{cripple}\hspace{5pt}\pcmn{跛}\hspace{5pt}\pfra{boiteux}\end{exemple}
\begin{exemple}\pnru{kʰɯ˧dv̩˧-tsʰo˧qʰwɤ˧}\hspace{5pt}\peng{lame demon}\hspace{5pt}\pcmn{跛鬼}\hspace{5pt}\pfra{démon boiteux}\end{exemple}
\end{entrée}

\begin{entrée}
{kʰɯ˧dʑɯ˧˥}{}{ⓔkʰɯ˧dʑɯ˧˥}\formedesurface{kʰɯ˧dʑɯ˧˥}\newline
\classe{名词}\ton{MH\#}
\paradigme{\pcmn{:} \p{}}
\begin{définition}\peng{Leggings, puttee.}\end{définition}
\begin{définition}\pcmn{裹腿}\end{définition}
\begin{définition}\pfra{Bande molletière.}\end{définition}
\end{entrée}

\begin{entrée}
{kʰɯ˧pi˧}{}{ⓔkʰɯ˧pi˧}\formedesurface{kʰɯ˧pi˧}\newline
\classe{动词}\begin{définition}\peng{To stumble, to trip.}\end{définition}
\begin{définition}\pcmn{绊}\end{définition}
\begin{définition}\pfra{Trébucher.}\end{définition}
\begin{exemple}\pnru{njɤ˧ kʰɯ˧pi˧-ze˧!}\hspace{5pt}\peng{I have stumbled!}\hspace{5pt}\pcmn{我绊了一跤!}\hspace{5pt}\pfra{j'ai trébuché!}\end{exemple}
\end{entrée}

\begin{entrée}
{kʰɯ˩pv̩˩}{}{ⓔkʰɯ˩pv̩˩}\formedesurface{kʰɯ˩pv̩˩˥}\newline
\classe{名词}\ton{L}
\paradigme{\pcmn{:} \p{}}
\begin{définition}\peng{Shuttle on a weaving loom.}\end{définition}
\begin{définition}\pcmn{梭,梭子}\end{définition}
\begin{définition}\pfra{Navette du métier à tisser; elle est actuellement confectionnée au plus simple, en prenant une tige de tournesol ou un bambou fin.}\end{définition}
\end{entrée}

\begin{entrée}
{kʰɯ˧pʰv̩˩}{}{ⓔkʰɯ˧pʰv̩˩}\formedesurface{kʰɯ˧pʰv̩˩}\newline
\classe{名词}\ton{L\#}
\paradigme{\pcmn{:} \p{}}
\begin{définition}\peng{Chinese, Han.}\end{définition}
\begin{définition}\pcmn{汉族}\end{définition}
\begin{définition}\pfra{Chinois.}\end{définition}
\end{entrée}

\begin{entrée}
{kʰɯ˧tʰo˧˥}{}{ⓔkʰɯ˧tʰo˧˥}\formedesurface{kʰɯ˧tʰo˧˥}\newline
\classe{名词}\ton{H\#}
\paradigme{\pcmn{:} \p{}}
\begin{définition}\peng{Chains (to tie a criminal's feet), made of iron.}\end{définition}
\begin{définition}\pcmn{脚链}\end{définition}
\begin{définition}\pfra{Chaîne de fer, pour attacher les chevilles d'un criminel.}\end{définition}
\begin{exemple}\pnru{kʰɯ˧tʰo˧ kʰɯ˥}\hspace{5pt}\peng{to put chains (on someone's feet)}\hspace{5pt}\pcmn{戴上脚链(在一个人的脚上)}\hspace{5pt}\pfra{mettre les chaînes (aux pieds de quelqu'un)}\end{exemple}
\end{entrée}

\begin{entrée}
{kʰɯ˧tʰv̩\#˥}{}{ⓔkʰɯ˧tʰv̩\#˥}\formedesurface{kʰɯ˧tʰv̩˧}\newline
\classe{名词}\ton{\#H}
\paradigme{\pcmn{:} \p{}}
\begin{définition}\peng{Pedal of the loom (to invert the vertical position of the threads).}\end{définition}
\begin{définition}\pcmn{织布机的脚蹬子=踏板}\end{définition}
\begin{définition}\pfra{Pédale du métier à tisser (pour inverser la position verticale des fils de trame entre 2 passages du volant).}\end{définition}
\end{entrée}

\begin{entrée}
{kʰɯ˧tsɯ˧bæ˥}{}{ⓔkʰɯ˧tsɯ˧bæ˥}\formedesurface{kʰɯ˧tsɯ˧bæ˥}\newline
\classe{名词}\ton{H\#}\begin{définition}\peng{Strip of decorative fabric used to tie around the ankles of wide-legged trousers; originally from Tibetan regions.}\end{définition}
\begin{définition}\pcmn{绑腿布:用来绑裤腿的一块缠布,也有装饰功能(从藏族地区传过来的)}\end{définition}
\begin{définition}\pfra{Bande de tissu large d'une dizaine de centimères, utilisée autrefois pour serrer le pantalon, qui était très ample; c'était un élément fonctionnel mais également décoratif, préparé en belle étoffe; il provenait des régions tibétaines.}\end{définition}
\end{entrée}

\begin{entrée}
{kʰɯ˧tsʰɤ˧˥}{}{ⓔkʰɯ˧tsʰɤ˧˥}\formedesurface{kʰɯ˧tsʰɤ˧˥}\newline
\classe{名词}\ton{MH\#}
\paradigme{\pcmn{:} \p{}}
\begin{définition}\peng{Leg.}\end{définition}
\begin{définition}\pcmn{腿,脚}\end{définition}
\begin{définition}\pfra{Jambe.}\end{définition}
\end{entrée}

\begin{entrée}
{kʰɯ˩ʈɯ˩}{}{ⓔkʰɯ˩ʈɯ˩}\formedesurface{kʰɯ˩ʈɯ˩˥}\newline
\classe{名词}\ton{L}
\paradigme{\pcmn{:} \p{}}
\begin{définition}\peng{Root.}\end{définition}
\begin{définition}\pcmn{根}\end{définition}
\begin{définition}\pfra{Racine.}\end{définition}
\begin{exemple}\pnru{si˧dzi˩-kʰɯ˩ʈɯ˩}\hspace{5pt}\peng{tree root}\hspace{5pt}\pcmn{树根}\hspace{5pt}\pfra{racines d'arbre}\end{exemple}
\end{entrée}

\begin{entrée}
{kʰɯ˧ʈʂæ˧˥}{}{ⓔkʰɯ˧ʈʂæ˧˥}\formedesurface{kʰɯ˧ʈʂæ˧˥}\newline
\classe{名词}\ton{MH\#}
\paradigme{\pcmn{:} \p{}}
\begin{définition}\peng{Ankle.}\end{définition}
\begin{définition}\pcmn{踝关节}\end{définition}
\begin{définition}\pfra{Cheville.}\end{définition}
\end{entrée}

\begin{entrée}
{kʰɯ˧ʈʂɤ\#˥}{}{ⓔkʰɯ˧ʈʂɤ\#˥}\formedesurface{kʰɯ˧ʈʂɤ˧}\newline
\classe{名词}\ton{\#H}
\paradigme{\pcmn{:} \p{}}
\begin{définition}\peng{Chicken feet.}\end{définition}
\begin{définition}\pcmn{鸡爪}\end{définition}
\begin{définition}\pfra{Griffes de poulet.}\end{définition}
\begin{exemple}\pnru{kʰɯ˧ʈʂɤ˧ tʰv̩˧-ɭɯ\#˥}\hspace{5pt}\peng{|fg{n}+|fg{dem}+|fg{clf}}\hspace{5pt}\pcmn{这只鸡爪}\hspace{5pt}\pfra{|fg{n}+|fg{dem}+|fg{clf}}\end{exemple}
\begin{exemple}\pnru{kʰɯ˧ʈʂɤ˧ tʰv̩˧-ʈv̩˥\#}\hspace{5pt}\peng{|fg{n}+|fg{dem}+|fg{clf}}\hspace{5pt}\pcmn{这只鸡爪}\hspace{5pt}\pfra{|fg{n}+|fg{dem}+|fg{clf}}\end{exemple}
\end{entrée}

\begin{entrée}
{kʰɯ˧ʐɯ˥\$}{}{ⓔkʰɯ˧ʐɯ˥\$}\formedesurface{kʰɯ˧ʐɯ˥}\newline
\classe{名词}\ton{H\$}
\paradigme{\pcmn{:} \p{}}
\begin{définition}\peng{Rice wine (low alcohol).}\end{définition}
\begin{définition}\pcmn{黄酒、苏里玛酒}\end{définition}
\begin{définition}\pfra{Vin de riz (faiblement alcoolisé).}\end{définition}
\end{entrée}

\begin{entrée}
{kʰv̩˥}{₁}{ⓔkʰv̩˥ⓗ1}\formedesurface{kʰv̩˧}\newline
\classe{名词}\ton{\#H}
1
\paradigme{\pcmn{:} \p{}}
\begin{définition}\peng{Nest (monosyllable).}\end{définition}
\begin{définition}\pcmn{(鸟)巢}\end{définition}
\begin{définition}\pfra{Nid (monosyllabe).}\end{définition}
\begin{exemple}\pnru{kʰv̩˧ ʈʂʰɯ˧-ɭɯ\#˥}\hspace{5pt}\peng{|fg{n}+|fg{dem}+|fg{clf}}\hspace{5pt}\pcmn{这只鸟巢}\hspace{5pt}\pfra{|fg{n}+|fg{dem}+|fg{clf}}\end{exemple}
\end{entrée}

\begin{entrée}
{kʰv̩˥}{₂}{ⓔkʰv̩˥ⓗ2}\formedesurface{kʰv̩˧}\newline
\classe{动词}\ton{H}
2\begin{définition}\peng{To harvest grass, to cut grass.}\end{définition}
\begin{définition}\pcmn{割(草)}\end{définition}
\begin{définition}\pfra{Couper (ex.: de l'herbe) pour récolter.}\end{définition}
\begin{exemple}\pnru{le˧-kʰv̩˥-ze˩}\hspace{5pt}\peng{|fg{accomp} \_ |fg{pfv}}\hspace{5pt}\pcmn{割了}\hspace{5pt}\pfra{|fg{accomp} \_ |fg{pfv}}\end{exemple}
\begin{exemple}\pnru{zɯ˧-kʰv̩˧}\hspace{5pt}\peng{to cut grass}\hspace{5pt}\pcmn{割草}\hspace{5pt}\pfra{couper de l'herbe}\end{exemple}
\end{entrée}

\begin{entrée}
{kʰv̩˥}{₃}{ⓔkʰv̩˥ⓗ3}\formedesurface{kʰv̩˧}\newline
\classe{名词}\ton{\#H}
3
\paradigme{\pcmn{:} \p{}}
\begin{définition}\peng{Dog (monosyllable).}\end{définition}
\begin{définition}\pcmn{狗}\end{définition}
\begin{définition}\pfra{Chien (monosyllabe).}\end{définition}
\begin{exemple}\pnru{kʰv̩˧-ʂe˧ dzɯ˧}\hspace{5pt}\peng{to eat dog meat (a practice which is strongly antagonistic to Na culture, which considers dog as man's benefactor)}\hspace{5pt}\pcmn{吃狗肉}\hspace{5pt}\pfra{manger de la viande de chien (pratique qui va droit à l'encontre de la culture na, dans laquelle le chien est considéré comme bienfaiteur de l'homme)}\end{exemple}
\begin{exemple}\pnru{kʰv̩˧-zɯ˧∼zɯ˥}\hspace{5pt}\peng{dog's existence, dog's life (which dog exchanged with man, according to the legend)}\hspace{5pt}\pcmn{狗的生命(传说狗与人交换了生命)}\hspace{5pt}\pfra{l'existence du chien, la vie du chien (qu'il a échangée avec l'homme, selon la légende)}\end{exemple}
\begin{exemple}\pnru{kʰv̩˧ tʰv̩˧-mi˥\#}\hspace{5pt}\peng{|fg{n}+|fg{dem}+|fg{clf}}\hspace{5pt}\pcmn{那条狗}\hspace{5pt}\pfra{|fg{n}+|fg{dem}+|fg{clf}}\end{exemple}
\begin{exemple}\pnru{kʰv̩˧-gɤ˥ljɤ˩}\hspace{5pt}\peng{roving dog}\hspace{5pt}\pcmn{流浪狗}\hspace{5pt}\pfra{chien errant}\end{exemple}
\end{entrée}

\begin{entrée}
{kʰv̩˥}{₄}{ⓔkʰv̩˥ⓗ4}\formedesurface{kʰv̩˧}\newline
\classe{动词}\ton{H}
4\begin{définition}\peng{To steal.}\end{définition}
\begin{définition}\pcmn{偷}\end{définition}
\begin{définition}\pfra{Voler.}\end{définition}
\begin{exemple}\pnru{hĩ˧-bv̩˧ tso˧∼tso˧ kʰv̩˧}\hspace{5pt}\peng{to steal someone's stuff, to steal someone else's property}\hspace{5pt}\pcmn{偷别人的东西}\hspace{5pt}\pfra{voler les affaires de quelqu'un}\end{exemple}
\end{entrée}

\begin{entrée}
{kʰv̩˧˥}{}{ⓔkʰv̩˧˥}\newline
\classe{名词}
\sens{1}
\begin{définition}\peng{Year; year of age.}\end{définition}
\begin{définition}\pcmn{年、岁}\end{définition}
\begin{définition}\pfra{Année, an.}\end{définition}
\begin{exemple}\pnru{kʰv̩˧-mæ˥}\hspace{5pt}\peng{end of the year}\hspace{5pt}\pcmn{年尾}\hspace{5pt}\pfra{fin de l'année}\end{exemple}
\begin{exemple}\pnru{kʰv̩˧-mæ˥ ʂæ˩}\hspace{5pt}\peng{intercalary year: a year with 13 months; this happens every 4 years or so}\hspace{5pt}\pcmn{闰年(有13个月)}\hspace{5pt}\pfra{année longue, de 13 mois; cela a lieu tous les 4 ans environ}\end{exemple}
\begin{exemple}\pnru{kʰv̩˧-mæ˥ ɖæ˩}\hspace{5pt}\peng{normal year, usual year: a year that has 12 months}\hspace{5pt}\pcmn{正常的年份,普通年:一年十二个月}\hspace{5pt}\pfra{année normale, à douze mois}\end{exemple}\sens{2}
\begin{définition}\peng{Astrological sign.}\end{définition}
\begin{définition}\pcmn{生肖}\end{définition}
\begin{définition}\pfra{Signe astrologique.}\end{définition}
\begin{exemple}\pnru{no˧ | ə˧tso˧ kʰv̩˧ ɲi˥?}\hspace{5pt}\peng{What is your astrological sign?}\hspace{5pt}\pcmn{你是属什么的?}\hspace{5pt}\pfra{De quel signe es-tu?}\end{exemple}
\end{entrée}

\begin{entrée}
{kʰv̩˧˥α}{}{ⓔkʰv̩˧˥α}\formedesurface{ɖɯ˧ kʰv̩˧˥}\newline
\classe{量词}\ton{MHα}\begin{définition}\peng{Year; year of age.}\end{définition}
\begin{définition}\pcmn{量词:年、岁}\end{définition}
\begin{définition}\pfra{Année.}\end{définition}
\begin{exemple}\pnru{ɖɯ˧-kʰv̩˧˥}\hspace{5pt}\peng{one year}\hspace{5pt}\pcmn{一年}\hspace{5pt}\pfra{une année}\end{exemple}
\end{entrée}

\begin{entrée}
{kʰv̩˧bv̩˧˥}{}{ⓔkʰv̩˧bv̩˧˥}\formedesurface{kʰv̩˧bv̩˧˥}\newline
\classe{名词}\ton{MH\#}
\paradigme{\pcmn{:} \p{}}
\begin{définition}\peng{Kennel, doghouse.}\end{définition}
\begin{définition}\pcmn{狗窝}\end{définition}
\begin{définition}\pfra{Chenil.}\end{définition}
\end{entrée}

\begin{entrée}
{kʰv̩˧kwæ˧}{}{ⓔkʰv̩˧kwæ˧}\formedesurface{kʰv̩˧kwæ˧}\newline
\classe{名词}\ton{M}
\paradigme{\pcmn{:} \p{}}
\begin{définition}\peng{Bitter melon.}\end{définition}
\begin{définition}\pcmn{苦瓜}\end{définition}
\begin{définition}\pfra{Concombre amer.}\end{définition}
\end{entrée}

\begin{entrée}
{kʰv̩˩-kʰɤ˩}{}{ⓔkʰv̩˩-kʰɤ˩}\formedesurface{kʰv̩˩kʰɤ˩˥}\newline
\classe{名词}\ton{L}
\paradigme{\pcmn{:} \p{}}
\begin{définition}\peng{Chicken nest.}\end{définition}
\begin{définition}\pcmn{鸡窝}\end{définition}
\begin{définition}\pfra{Nid de poule, pondoir, endroit où la poule pond.}\end{définition}
\end{entrée}

\begin{entrée}
{kʰv̩˧-kʰv̩˩}{₁}{ⓔkʰv̩˧-kʰv̩˩ⓗ1}\formedesurface{kʰv̩˧kʰv̩˩}\newline
\classe{名词}\ton{L\#}
1\begin{définition}\peng{Year of the Dog.}\end{définition}
\begin{définition}\pcmn{狗年}\end{définition}
\begin{définition}\pfra{Année du Chien.}\end{définition}
\end{entrée}

\begin{entrée}
{kʰv̩˧-kʰv̩˩}{₂}{ⓔkʰv̩˧-kʰv̩˩ⓗ2}\formedesurface{kʰv̩˧kʰv̩˩}\newline
\classe{形容词}\ton{L\#}
2\begin{définition}\peng{Born in the year of the Dog.}\end{définition}
\begin{définition}\pcmn{属狗}\end{définition}
\begin{définition}\pfra{Né l'année du Chien.}\end{définition}
\end{entrée}

\begin{entrée}
{kʰv̩˧mæ˧}{}{ⓔkʰv̩˧mæ˧}\formedesurface{kʰv̩˧mæ˧}\newline
\classe{名词}\ton{M}
\paradigme{\pcmn{:} \p{}}
\begin{définition}\peng{Robber, bandit.}\end{définition}
\begin{définition}\pcmn{强盗}\end{définition}
\begin{définition}\pfra{Voleur, bandit.}\end{définition}
\begin{exemple}\pnru{kʰv̩˧mæ˧ ʝi˧-hĩ˧-hĩ˧}\hspace{5pt}\peng{person who robs, robber}\hspace{5pt}\pcmn{当强盗的人=强盗}\hspace{5pt}\pfra{personne qui est un bandit, bandit}\end{exemple}
\begin{exemple}\pnru{kʰv̩˧mæ˧-ni˩-zo˩! | hĩ˧ lɑ˩-ho˩!}\hspace{5pt}\peng{He's like a bandit! He may hit people!}\hspace{5pt}\pcmn{他像强盗似的!会打人的!}\hspace{5pt}\pfra{Ca doit être un voleur! Il se pourrait qu'il frappe les gens!}\end{exemple}
\begin{exemple}\pnru{kʰv̩˧mæ˧-ʑi˩}\hspace{5pt}\peng{prison: literally “house for thieves"}\hspace{5pt}\pcmn{监狱。直译:“贼家”}\hspace{5pt}\pfra{prison: littéralement «maison des voleurs»}\end{exemple}
\begin{exemple}\pnru{kʰv̩˧mæ˧-ʝi˧-hĩ˧, | lo˧ʑi˥bv̩˩-qo˩ ʈæ˩!}\hspace{5pt}\peng{Thieves are tied up in prisons / are sent to prison!}\hspace{5pt}\pcmn{贼,被关在监狱!}\hspace{5pt}\pfra{les voleurs, on les met en prison!}\end{exemple}
\begin{exemple}\pnru{no˧ | kʰv̩˧mæ˧-pʰæ˧qʰwɤ˩-ne˩-ʝi˩-zo˩!}\hspace{5pt}\peng{You have the face of a thief! / You really look like a thief! (An accusation about someone one thinks is a thief)}\hspace{5pt}\pcmn{你有一张贼脸!(控告一个人)}\hspace{5pt}\pfra{Toi, tu m'as une tête de voleur! (accusation lancée à quelqu'un qu'on pense être un voleur)}\end{exemple}
\end{entrée}

\begin{entrée}
{kʰv̩˩mi˩}{}{ⓔkʰv̩˩mi˩}\formedesurface{kʰv̩˩mi˩˥}\newline
\classe{名词}\ton{L}
\paradigme{\pcmn{:} \p{}}\paradigme{\pcmn{:} \p{}}
\begin{définition}\peng{Dog (either he-dog or she-dog).}\end{définition}
\begin{définition}\pcmn{狗}\end{définition}
\begin{définition}\pfra{Chien (sans spécifier le sexe).}\end{définition}
\begin{exemple}\pnru{kʰv̩˩mi˩ ʈʂʰɯ˩-jɤ˧}\hspace{5pt}\peng{|fg{n}+|fg{dem}+|fg{clf}}\hspace{5pt}\pcmn{这条狗}\hspace{5pt}\pfra{|fg{n}+|fg{dem}+|fg{clf}}\end{exemple}
\begin{exemple}\pnru{di˧qo˧-kʰv̩˩mi˩}\hspace{5pt}\peng{the dogs of the plain (which, unlike dogs in mountain hamlets, get to see lots of passers-by, and are less likely to bite strangers)}\hspace{5pt}\pcmn{平坝的狗}\hspace{5pt}\pfra{les chiens de la plaine (qui à la différence des chiens des petits hameaux de montagne voient beaucoup de passage et sont moins susceptibles de mordre les inconnus de passage)}\end{exemple}
\begin{exemple}\pnru{kʰv̩˩mi˩-gɤ˥ljɤ˩}\hspace{5pt}\peng{roving dog}\hspace{5pt}\pcmn{流浪的狗}\hspace{5pt}\pfra{chien errant}\end{exemple}
\end{entrée}

\begin{entrée}
{kʰv̩˧mv̩˥}{}{ⓔkʰv̩˧mv̩˥}\formedesurface{kʰv̩˧mv̩˥}\newline
\classe{名词}\ton{H\#}
\paradigme{\pcmn{:} \p{}}
\begin{définition}\peng{Female puppy. The term is also used as a temporary name for little girls, during the first months of their life, before they are given a real name. This ugly term is intended to disgust evil spirits, which will therefore turn their attention away from the infant. (In the early 21st century, the registry office requires a name to be given at birth; but this name only begins to be used by the family after the first months of life have elapsed.).}\end{définition}
\begin{définition}\pcmn{小母狗(给刚出生的女孩起的名字,让鬼对她不感兴趣,不会来害小孩)}\end{définition}
\begin{définition}\pfra{Chienne, petit chiot femelle. Le terme est également employé comme nom provisoire pour les fillettes pendant leurs premiers mois, avant qu'on ne leur donne un vrai nom. Le vilain nom dont on l'affuble vise à éviter que le nourrisson ne soit repéré par de mauvais esprits. (Actuellement, l'état-civil nécessite qu'un nom soit donné dès la naissance; mais celui-ci ne commence à être employé dans les conversations familiales qu'une fois passés les premiers mois.).}\end{définition}
\end{entrée}

\begin{entrée}
{kʰv̩˧nɑ˥}{}{ⓔkʰv̩˧nɑ˥}\formedesurface{kʰv̩˧nɑ˥}\newline
\classe{名词}\ton{H\#}
\paradigme{\pcmn{:} \p{}}
\begin{définition}\peng{Dog (formal word, used in elevated speech).}\end{définition}
\begin{définition}\pcmn{狗}\end{définition}
\begin{définition}\pfra{Chien (registre de langage relevé).}\end{définition}
\end{entrée}

\begin{entrée}
{kʰv̩˧pʰæ˧}{}{ⓔkʰv̩˧pʰæ˧}\formedesurface{kʰv̩˧pʰæ˧}\newline
\classe{名词}\ton{M}\begin{définition}\peng{Age.}\end{définition}
\begin{définition}\pcmn{年龄}\end{définition}
\begin{définition}\pfra{Âge.}\end{définition}
\begin{exemple}\pnru{kʰv̩˧pʰæ˧ tɕi˩}\hspace{5pt}\peng{young}\hspace{5pt}\pcmn{年轻}\hspace{5pt}\pfra{jeune}\end{exemple}
\begin{exemple}\pnru{kʰv̩˧pʰæ˧ | tɕi˩-hĩ˩˥}\hspace{5pt}\peng{young}\hspace{5pt}\pcmn{年轻的}\hspace{5pt}\pfra{jeune}\end{exemple}
\end{entrée}

\begin{entrée}
{kʰv̩˧-pʰo˥}{}{ⓔkʰv̩˧-pʰo˥}\formedesurface{kʰv̩˧pʰo˥}\newline
\classe{名词}\ton{H\#}\begin{définition}\peng{Half a year.}\end{définition}
\begin{définition}\pcmn{半年}\end{définition}
\begin{définition}\pfra{Une demi-année.}\end{définition}
\begin{exemple}\pnru{ɖɯ˧-kʰv̩˧-kʰv̩˥-pʰo˩}\hspace{5pt}\peng{one year and a half}\hspace{5pt}\pcmn{一年半}\hspace{5pt}\pfra{un an et demi}\end{exemple}
\end{entrée}

\begin{entrée}
{kʰv̩˧pʰv̩\#˥}{}{ⓔkʰv̩˧pʰv̩\#˥}\formedesurface{kʰv̩˧pʰv̩˧}\newline
\classe{名词}\ton{\#H}
\paradigme{\pcmn{:} \p{}}
\begin{définition}\peng{He-dog.}\end{définition}
\begin{définition}\pcmn{公狗}\end{définition}
\begin{définition}\pfra{Chien mâle (forme élicitée).}\end{définition}
\begin{exemple}\pnru{kʰv̩˧pʰv̩˧ ʈʂʰɯ˧-ɭɯ\#˥}\hspace{5pt}\peng{|fg{n}+|fg{dem}+|fg{clf}}\hspace{5pt}\pcmn{这只公狗}\hspace{5pt}\pfra{|fg{n}+|fg{dem}+|fg{clf}}\end{exemple}
\begin{exemple}\pnru{kʰv̩˧pʰv̩˧ tʰv̩˧-mi˧˥}\hspace{5pt}\peng{|fg{n}+|fg{dem}+|fg{clf}}\hspace{5pt}\pcmn{这只公狗}\hspace{5pt}\pfra{|fg{n}+|fg{dem}+|fg{clf}}\end{exemple}
\begin{exemple}\pnru{kʰv̩˧pʰv̩˧ tʰv̩˧-v̩\#˥}\hspace{5pt}\peng{|fg{n}+|fg{dem}+|fg{clf}}\hspace{5pt}\pcmn{这个公狗}\hspace{5pt}\pfra{|fg{n}+|fg{dem}+|fg{clf}}\end{exemple}
\end{entrée}

\begin{entrée}
{kʰv̩˧qʰwɤ˧˥}{}{ⓔkʰv̩˧qʰwɤ˧˥}\formedesurface{kʰv̩˧qʰwɤ˧˥}\newline
\classe{名词}\ton{MH\#}\begin{définition}\peng{Bad year, year when the crops are bad.}\end{définition}
\begin{définition}\pcmn{庄稼收成不好的(一)年}\end{définition}
\begin{définition}\pfra{Mauvaise année, année de disette.}\end{définition}
\begin{exemple}\pnru{kʰv̩˧qʰwɤ˧ tʰv̩˧˥}\hspace{5pt}\peng{the year is bad; crops are bad this year; a bad year has come}\hspace{5pt}\pcmn{今年,收成不好。}\hspace{5pt}\pfra{une mauvaise année a lieu, une année de mauvaise récolte/de disette}\end{exemple}
\end{entrée}

\begin{entrée}
{kʰv̩˧sɯ˧sɯ˩}{}{ⓔkʰv̩˧sɯ˧sɯ˩}\formedesurface{kʰv̩˧sɯ˧sɯ˩}\newline
\classe{名词}\ton{L\#}
\paradigme{\pcmn{:} \p{}}
\begin{définition}\peng{A flowering plant in the legume family: |\stylefi{Flemingia strobilifera}, also known as |\stylefi{Moghania fruticulosa}.}\end{définition}
\begin{définition}\pcmn{球穗千斤拔、半灌木千斤拔、大苞千斤拔}\end{définition}
\begin{définition}\pfra{Plante à fleurs, |\stylefi{Flemingia strobilifera}, aussi appelée |\stylefi{Moghania fruticulosa} (nom en chinois local: «oreille de souris», du fait de la forme de la feuille).}\end{définition}
\end{entrée}

\begin{entrée}
{kʰv̩˧ʂæ˧˥}{}{ⓔkʰv̩˧ʂæ˧˥}\formedesurface{kʰv̩˧ʂæ˧˥}\newline
\classe{动词}\ton{MH}\begin{définition}\peng{To hunt (leading a dog).}\end{définition}
\begin{définition}\pcmn{打猎、赶走、驱逐}\end{définition}
\begin{définition}\pfra{Chasser; mener un chien de chasse.}\end{définition}
\begin{exemple}\pnru{kʰv̩˧ʂæ˧ hɯ˧˥}\hspace{5pt}\peng{(He/she) has gone hunting}\hspace{5pt}\pcmn{狩猎去了}\hspace{5pt}\pfra{(Elle/il) est parti(e) chasser}\end{exemple}
\end{entrée}

\begin{entrée}
{kʰv̩˧ʂɯ˥}{}{ⓔkʰv̩˧ʂɯ˥}\formedesurface{kʰv̩˧ʂɯ˥}\newline
\classe{动词}\begin{définition}\peng{To celebrate the New Year.}\end{définition}
\begin{définition}\pcmn{过年}\end{définition}
\begin{définition}\pfra{Fêter le Nouvel An.}\end{définition}
\end{entrée}

\begin{entrée}
{kʰv̩˧tɕʰi˥\$}{}{ⓔkʰv̩˧tɕʰi˥\$}\formedesurface{kʰv̩˧tɕʰi˥}\newline
\classe{名词}\ton{H\$}\begin{définition}\peng{Solution, way out.}\end{définition}
\begin{définition}\pcmn{办法}\end{définition}
\begin{définition}\pfra{Solution, méthode.}\end{définition}
\begin{exemple}\pnru{kʰv̩˧tɕʰi˥ | mɤ˧-dʑo˧-ze˧! | ɻ̃˧-ɻ̍˧ tʰo˩!}\hspace{5pt}\peng{There is nothing we can do anymore! It's a catastrophe!}\hspace{5pt}\pcmn{没有办法了!糟糕了!}\hspace{5pt}\pfra{Il n'y a plus rien à faire! C'est la catastrophe!}\end{exemple}
\end{entrée}

\begin{entrée}
{kʰv̩˩tsɤ˩mi˥}{}{ⓔkʰv̩˩tsɤ˩mi˥}\formedesurface{kʰv̩˩tsɤ˩mi˥}\newline
\classe{名词}\ton{L+H\#}
\paradigme{\pcmn{:} \p{}}
\begin{définition}\peng{She-dog.}\end{définition}
\begin{définition}\pcmn{母狗}\end{définition}
\begin{définition}\pfra{Chienne.}\end{définition}
\end{entrée}

\begin{entrée}
{kʰv̩˧tsʰi˧-bo˥tsʰi˩}{}{ⓔkʰv̩˧tsʰi˧-bo˥tsʰi˩}\formedesurface{kʰv̩˧tsʰi˧bo˥tsʰi˩}\newline
\classe{名词}\ton{\#H-}
\paradigme{\pcmn{:} \p{}}
\begin{définition}\peng{Mole shrew.}\end{définition}
\begin{définition}\pcmn{鼹鼠}\end{définition}
\begin{définition}\pfra{Taupe.}\end{définition}
\end{entrée}

\begin{entrée}
{kʰv̩˧zo˥\$}{}{ⓔkʰv̩˧zo˥\$}\formedesurface{kʰv̩˧zo˥}\newline
\classe{名词}\ton{H\$}\begin{définition}\peng{A family name from Yongning. There are two families in Yongning that carry this name.}\end{définition}
\begin{définition}\pcmn{一个姓。这个姓,永宁有两家}\end{définition}
\begin{définition}\pfra{Nom de clan/famille étendue. Deux familles portent ce nom à Yongning.}\end{définition}
\begin{exemple}\pnru{kʰv̩˧zo˧=ɻ̍˥\$}\hspace{5pt}\peng{the /kʰv̩˧zo˥\$/ clan, the /kʰv̩˧zo˥\$/ family}\hspace{5pt}\pcmn{|fv{/kʰv̩˧zo˥\$/}家族}\hspace{5pt}\pfra{La famille /kʰv̩˧zo˥\$/, les /kʰv̩˧zo˥\$/}\end{exemple}
\begin{exemple}\pnru{kʰv̩˧zo˥-tsʰɯ˩ɻ̍˩}\hspace{5pt}\peng{the name of a person, containing both a family name: /kʰv̩˧zo˥\$/, and a given name: /tsʰɯ˧ɻ̍\#˥/}\hspace{5pt}\pcmn{一个人的名字:姓为|fv{/kʰv̩˧zo˥\$/},名为|fv{/tsʰɯ˧ɻ̍\#˥/}}\hspace{5pt}\pfra{nom d'une personne, comportant un nom de famille (/kʰv̩˧zo˥\$/) et un prénom (/tsʰɯ˧ɻ̍\#˥/)}\end{exemple}
\end{entrée}

\begin{entrée}
{kʰv̩˧zo\#˥}{}{ⓔkʰv̩˧zo\#˥}\formedesurface{kʰv̩˧zo˧}\newline
\classe{名词}\ton{\#H}
\paradigme{\pcmn{:} \p{}}
\begin{définition}\peng{Male dog. The term is also used as a temporary name for little boys, during the first months of their life, before they are given a real name. This ugly term is intended to disgust evil spirits, which will therefore turn their attention away from the infant. (In the early 21st century, the registry office requires a name to be given at birth; but this name only begins to be used by the family after the first months of life have elapsed.).}\end{définition}
\begin{définition}\pcmn{公狗(给刚出生的男孩子的名字,让鬼对他不感兴趣,不过来害小孩)}\end{définition}
\begin{définition}\pfra{Chien (mâle), chiot. Le terme est également employé comme nom provisoire pour les garçonnets pendant leurs premiers mois, avant qu'on ne leur donne un vrai nom. Le vilain nom dont on l'affuble vise à éviter que le nourrisson ne soit repéré par de mauvais esprits. (Actuellement, l'état-civil nécessite qu'un nom soit donné dès la naissance; mais celui-ci ne commence à être employé dans les conversations familiales qu'une fois passés les premiers mois.).}\end{définition}
\begin{exemple}\pnru{kʰv̩˧zo˧ ʈʂʰɯ˧-ɭɯ\#˥}\hspace{5pt}\peng{|fg{n}+|fg{dem}+|fg{clf}}\hspace{5pt}\pcmn{这只公狗}\hspace{5pt}\pfra{|fg{n}+|fg{dem}+|fg{clf}}\end{exemple}
\begin{exemple}\pnru{kʰv̩˧zo˧ tʰv̩˧-mi˧˥}\hspace{5pt}\peng{|fg{n}+|fg{dem}+|fg{clf}}\hspace{5pt}\pcmn{这只公狗}\hspace{5pt}\pfra{|fg{n}+|fg{dem}+|fg{clf}}\end{exemple}
\begin{exemple}\pnru{kʰv̩˧zo˧ tʰv̩˧-v̩\#˥}\hspace{5pt}\peng{|fg{n}+|fg{dem}+|fg{clf}}\hspace{5pt}\pcmn{这只公狗}\hspace{5pt}\pfra{|fg{n}+|fg{dem}+|fg{clf}}\end{exemple}
\begin{exemple}\pnru{kʰv̩˧zo˥-kʰv̩˩mv̩˩}\hspace{5pt}\peng{puppy and she-dog}\hspace{5pt}\pcmn{小狗与母狗}\hspace{5pt}\pfra{chien et chienne}\end{exemple}
\end{entrée}

\begin{entrée}
{kʰv̩˧zo˧ | -bo˩zo\#˥}{}{ⓔkʰv̩˧zo˧ | -bo˩zo\#˥}\formedesurface{kʰv̩˧zo˧bo˩zo˥}\newline
\classe{名词}\ton{-LM+\#H}
\paradigme{\pcmn{:} \p{}}
\begin{définition}\peng{Dog-pig, doggy-piggy. The term is used as a temporary name for little boys, during the first months of their life, before they are given a real name. This ugly term is intended to disgust evil spirits, which will therefore turn their attention away from the infant. (In the early 21st century, the registry office requires a name to be given at birth; but this name only begins to be used by the family after the first months of life have elapsed.).}\end{définition}
\begin{définition}\pcmn{小畜生,直译:猪崽子、狗崽子}\end{définition}
\begin{définition}\pfra{Chien-cochon, cochon-clébard. Terme employé comme nom provisoire pour les garçonnets pendant leurs premiers mois, avant qu'on ne leur donne un vrai nom. Le vilain nom dont on l'affuble vise à éviter que le nourrisson ne soit repéré par de mauvais esprits. (Actuellement, l'état-civil nécessite qu'un nom soit donné dès la naissance; mais celui-ci ne commence à être employé dans les conversations familiales qu'une fois passés les premiers mois.).}\end{définition}
\end{entrée}

\begin{entrée}
{kʰwæ˧hwæ˩}{}{ⓔkʰwæ˧hwæ˩}\newline
\classe{动词}\begin{définition}\peng{To drape oneself in (a cape, a piece of fabric).}\end{définition}
\begin{définition}\pcmn{披上}\end{définition}
\begin{définition}\pfra{Se draper de, endosser, mettre sur son dos.}\end{définition}
\end{entrée}

\begin{entrée}
{kʰwæ˧ɻæ\#˥}{}{ⓔkʰwæ˧ɻæ\#˥}\formedesurface{kʰwæ˧ɻæ˧}\newline
\classe{名词}\ton{\#H}
\paradigme{\pcmn{:} \p{}}
\begin{définition}\peng{Felt; extended use: mat (even if not made of felt), cushion…}\end{définition}
\begin{définition}\pcmn{毡子。也用来指席子,垫子等。}\end{définition}
\begin{définition}\pfra{Feutre; par extension: natte, tapis (même en vannerie), coussin…}\end{définition}
\begin{exemple}\pnru{kʰwæ˧ɻæ˧ tʰi˧-kʰo˥}\hspace{5pt}\peng{to spread a mat}\hspace{5pt}\pcmn{铺席子}\hspace{5pt}\pfra{étendre la natte}\end{exemple}
\end{entrée}

\begin{entrée}
{kʰwɤ˥α}{}{ⓔkʰwɤ˥α}\formedesurface{ɖɯ˧ kʰwɤ˥}\newline
\classe{量词}\ton{Hα}\begin{définition}\peng{A piece of, a chunk of; a mouthful of.}\end{définition}
\begin{définition}\pcmn{量词:块。一块肉、一口饭}\end{définition}
\begin{définition}\pfra{Classificateur des morceaux/bouchées.}\end{définition}
\begin{exemple}\pnru{ɖɯ˧-kʰwɤ˥∼ɖɯ˩-kʰwɤ˩}\hspace{5pt}\peng{chunk by chunk, one chunk after the other}\hspace{5pt}\pcmn{一块一块地}\hspace{5pt}\pfra{par petites bouchées, par petits morceaux}\end{exemple}
\begin{exemple}\pnru{kʰwɤ˧ | ɖɯ˧-ʂe˧-ɻ̍˩!}\hspace{5pt}\peng{Go ahead and decide! / Please make a decision!}\hspace{5pt}\pcmn{你们得要做出决定!}\hspace{5pt}\pfra{Décidez! / Il faut vous décider!}\end{exemple}
\begin{exemple}\pnru{ɖɯ˧-kʰwɤ˧ so˧˥, | ɖɯ˧-kʰwɤ˥ fv̩˩!}\hspace{5pt}\peng{Each new word is a new joy! (A comment by the consultant about the investigator's enjoyment of fieldwork. She takes a look at a draft dictionary, and comments that it represents a great deal of work, and that what matters is that the investigator should feel an interest in it, considering each new ‘piece' – each addition to the dictionary – as a source of joy.)}\hspace{5pt}\pcmn{学一点,就高兴一点!(评说语言调查工作:合作人看着本词典的初稿,说:这是一项很大的工程,关键的是调查者要有兴趣,欣赏每个新学的语言信息。)}\hspace{5pt}\pfra{Chaque mot appris représente une joie de plus! (Commentaire de la locutrice au sujet du travail de l'enquêteur. Tenant en main le manuscrit de ce dictionnaire, elle commente: cela représente un travail immense; l'important est que l'enquêteur y trouve de l'intérêt: chaque information nouvelle – chaque «morceau» de langue – ajoutée au dictionnaire est une joie pour l'enquêteur.)}\end{exemple}
\end{entrée}

\begin{entrée}
{kʰwɤ˧pʰv̩˧}{}{ⓔkʰwɤ˧pʰv̩˧}\formedesurface{kʰwɤ˧pʰv̩˧}\newline
\classe{名词}\ton{M}\begin{définition}\peng{Meadow.}\end{définition}
\begin{définition}\pcmn{草坪、草地}\end{définition}
\begin{définition}\pfra{Pré: soit prairie de plaine, soit prairie d'altitude (alpage).}\end{définition}
\end{entrée}

\begin{entrée}
{kʰwɤ˧pʰv̩˧-mo˧˥}{}{ⓔkʰwɤ˧pʰv̩˧-mo˧˥}\formedesurface{kʰwɤ˧pʰv̩˧mo˧˥}\newline
\classe{名词}\ton{MH\#}\begin{définition}\peng{Meadow mushroom: a sort of edible mushroom that grows on meadows (not yet identified; perhaps |\stylefi{Agaricus campestris}).}\end{définition}
\begin{définition}\pcmn{可以吃的一种菌子:可能是四孢蘑菇。直译:“草坪菌”}\end{définition}
\begin{définition}\pfra{Champignon des prés: une sorte de champignon comestible (pas encore identifiée): agaric champêtre ou rosé des prés, |\stylefi{Agaricus campestris}?.}\end{définition}
\end{entrée}

\newpage\caractère{l}

\begin{entrée}
{lɑ˧}{₁}{ⓔlɑ˧ⓗ1}\formedesurface{lɑ˧}\newline
\classe{名词}\ton{M}
1
\paradigme{\pcmn{:} \p{}}
\begin{définition}\peng{Tiger.}\end{définition}
\begin{définition}\pcmn{老虎}\end{définition}
\begin{définition}\pfra{Tigre.}\end{définition}
\end{entrée}

\begin{entrée}
{lɑ˧}{₂}{ⓔlɑ˧ⓗ2}\formedesurface{lɑ˧}\newline
\classe{助词}\ton{M}
2\begin{définition}\peng{Too, also, and.}\end{définition}
\begin{définition}\pcmn{和、与、跟}\end{définition}
\begin{définition}\pfra{Et, aussi.}\end{définition}
\begin{exemple}\pnru{ɖɯ˧-kʰv̩˧ lɑ˥ | so˩-ɬi˩˥}\hspace{5pt}\peng{one year and three months (context: indicating the age of an infant)}\hspace{5pt}\pcmn{一岁三个月}\hspace{5pt}\pfra{un an et trois mois (contexte: on indique l'âge d'un petit enfant)}\end{exemple}
\begin{exemple}\pnru{ʈʂʰɯ˧ lɑ˧ | mɤ˧-bi˧, | njɤ˧ lɑ˧ mɤ˧-bi˧!}\hspace{5pt}\peng{(If) (s)he does not go, I'm not going either!}\hspace{5pt}\pcmn{他不去(的话),我也不去!}\hspace{5pt}\pfra{s'il n'y va pas, moi non plus!}\end{exemple}
\begin{exemple}\pnru{hĩ˧ lɑ˩ | dʑɤ˧˥, | mv̩˧di˧ lɑ˥ | dʑɤ˧˥! / hĩ˧ lɑ˩ | dʑɤ˧˥, | lv̩˧ lɑ˧ | dʑɤ˧˥!}\hspace{5pt}\peng{The people are good; and the land is good! / The people are good; and the fields are good! (A set phrase to recommend a family which a young woman is considering joining through marriage: the people are good, and their land is good.)}\hspace{5pt}\pcmn{人也好,田也好!(习语:将女孩嫁出去前,一家人打听对方家如何,推荐的人保证:“他们家,人也好,田也好!”)}\hspace{5pt}\pfra{les gens (y) sont bons, (et) la terre (y) est bonne (formule de recommandation pour la famille que va rejoindre une jeune femme lors de son mariage)}\end{exemple}
\begin{exemple}\pnru{hĩ˧ F | dʑɤ˧˥, | mv̩˧di˧˥ F | dʑɤ˧˥!}\hspace{5pt}\peng{as above}\hspace{5pt}\pcmn{同上}\hspace{5pt}\pfra{même sens}\end{exemple}
\begin{exemple}\pnru{mɤ˧ lɑ˧ dʑɤ˧˥!}\hspace{5pt}\peng{The grease too is good! (Elicited variant on the preceding examples)}\hspace{5pt}\pcmn{猪油也好!(按照上面例子的变体)}\hspace{5pt}\pfra{la graisse aussi (y) est bonne! (variation élicitée à partir des exemples qui précèdent)}\end{exemple}
\begin{exemple}\pnru{qæ˩ lɑ˥ | dʑɤ˧˥!}\hspace{5pt}\peng{The oil too is good! (Elicited variant on the preceding examples)}\hspace{5pt}\pcmn{油也好!}\hspace{5pt}\pfra{l’huile aussi (y) est bonne! (variation à partir de l'exemple qui précède)}\end{exemple}
\begin{exemple}\pnru{ʈʂʰɯ˧ lɑ˧ | mɤ˧-bi˧, | njɤ˧ | mɤ˧-bi˧-ze˧! / ʈʰɯ˧ mɤ˧-bi˧-ze˧-dʑo˧, | njɤ˧ lɑ˧ | mɤ˧-bi˧-ze˧!}\hspace{5pt}\peng{If he doesn't go, I'm not going either!}\hspace{5pt}\pcmn{他如果不去,我也不去!}\hspace{5pt}\pfra{s'il n'y va pas, moi non plus!}\end{exemple}
\end{entrée}

\begin{entrée}
{lɑ˧}{₃}{ⓔlɑ˧ⓗ3}\formedesurface{lɑ˧}\newline
\classe{助词}\ton{M}
3\begin{définition}\peng{Only.}\end{définition}
\begin{définition}\pcmn{只,才}\end{définition}
\begin{définition}\pfra{Seulement.}\end{définition}
\begin{exemple}\pnru{ʈʂʰɯ˧-lɑ˩ ɲi˩-ze˩-mæ˩!}\hspace{5pt}\peng{That's all!}\hspace{5pt}\pcmn{就这些了! / 就这些而已! / 就这样!}\hspace{5pt}\pfra{C'est tout ! / Voilà tout !}\end{exemple}
\end{entrée}

\begin{entrée}
{lɑ˧˥}{₁}{ⓔlɑ˧˥ⓗ1}\formedesurface{lɑ˧˥}\newline
\classe{动词}\ton{MH}
1\begin{définition}\peng{To strike someone, to beat someone.}\end{définition}
\begin{définition}\pcmn{打(打人,钉钉子……)}\end{définition}
\begin{définition}\pfra{Battre quelque chose, frapper quelque chose, enfoncer un clou, casser des cailloux; donner (une injonction…).}\end{définition}
\begin{exemple}\pnru{hĩ˧ lɑ˩}\hspace{5pt}\peng{to strike someone}\hspace{5pt}\pcmn{打人}\hspace{5pt}\pfra{frapper quelqu'un}\end{exemple}
\begin{exemple}\pnru{hɑ˧ lɑ˩}\hspace{5pt}\peng{to beat the grain}\hspace{5pt}\pcmn{打粮食}\hspace{5pt}\pfra{battre le grain}\end{exemple}
\begin{exemple}\pnru{nv̩˩ɭɯ˧ lɑ˧}\hspace{5pt}\peng{to beat soy}\hspace{5pt}\pcmn{打大豆}\hspace{5pt}\pfra{battre les cosses de soja}\end{exemple}
\begin{exemple}\pnru{sɯ˩tʰi˩-po˥-ɳɯ˩ | lɑ˧˥}\hspace{5pt}\peng{to break with a knife (brick tea: compressed tea leaves)}\hspace{5pt}\pcmn{用刀子来砍(沱茶、茶饼)}\hspace{5pt}\pfra{casser au moyen d'un couteau (du thé compressé en galettes ou en briques, à l'ancienne)}\end{exemple}
\begin{exemple}\pnru{ə˧ʝi˧-ʂɯ˥ʝi˩, | ɬi˧di˩-dʑo˩, | æ˧ lɑ˩-hĩ˩ F | dʑo˩˥! | ʂe˧ lɑ˧-hĩ˥ F | dʑo˩˥! | hæ̃˩ lɑ˩-hĩ˥ F | dʑo˩˥! | ŋv̩˩ lɑ˩-hĩ˥ F | dʑo˩˥!}\hspace{5pt}\peng{In the past, in Yongning, there were craftsmen who forged copper! craftsmen who forged iron! craftsmen who forged gold! and craftsmen who forged silver!}\hspace{5pt}\pcmn{过去,在永宁,有铜匠、铁匠、金匠、银匠。}\hspace{5pt}\pfra{Autrefois, à Yongning, il y avait des artisans qui travaillaient le cuivre! Il y avait des artisans qui travaillaient le fer! Il y avait des artisans qui travaillaient l'or! Il y avait des artisans qui travaillaient l'argent!}\end{exemple}
\begin{exemple}\pnru{ə˧ʝi˧-ʂɯ˥ʝi˩, | ɬi˧di˩-dʑo˩, | æ˧ lɑ˩-hĩ˩ dʑo˩, | ʂe˧ lɑ˧-hĩ˥ dʑo˩, | hæ̃˩ lɑ˩-hĩ˥ dʑo˩, | ŋv̩˩ lɑ˩-hĩ˥ dʑo˩.}\hspace{5pt}\peng{In the past, in Yongning, there were craftsmen who forged copper; craftsmen who forged iron; craftsmen who forged gold; and craftsmen who forged silver.}\hspace{5pt}\pcmn{过去,在永宁,有铜匠、铁匠、金匠、银匠。}\hspace{5pt}\pfra{Autrefois, à Yongning, il y avait des artisans qui travaillaient le cuivre. Il y avait des artisans qui travaillaient le fer. Il y avait des artisans qui travaillaient l'or. Il y avait des artisans qui travaillaient l'argent.}\end{exemple}
\end{entrée}

\begin{entrée}
{lɑ˧˥}{₂}{ⓔlɑ˧˥ⓗ2}\formedesurface{lɑ˧˥}\newline
\classe{动词}\ton{MH}
2\begin{définition}\peng{To form, to be there, to have appeared (dew).}\end{définition}
\begin{définition}\pcmn{有,结(露水)}\end{définition}
\begin{définition}\pfra{Apparaître, y avoir (de la rosée).}\end{définition}
\begin{exemple}\pnru{ɖʐv̩˧ lɑ˧˥}\hspace{5pt}\peng{Some dew has appeared; there is some dew}\hspace{5pt}\pcmn{结露水了。}\hspace{5pt}\pfra{Il y a de la rosée; de la rosée s'est formée}\end{exemple}
\begin{exemple}\pnru{ɖʐv̩˧qʰɑ˧ lɑ˧˥}\hspace{5pt}\peng{Some dew has appeared; there is some dew}\hspace{5pt}\pcmn{结露水了。}\hspace{5pt}\pfra{Il y a de la rosée; de la rosée s'est formée}\end{exemple}
\end{entrée}

\begin{entrée}
{lɑ˧bi\#˥}{}{ⓔlɑ˧bi\#˥}\formedesurface{lɑ˧bi˧}\newline
\classe{名词}\ton{\#H}\begin{définition}\peng{Steep slope.}\end{définition}
\begin{définition}\pcmn{陡坡、土坡、斜坡}\end{définition}
\begin{définition}\pfra{Escarpement, pente raide, terrain escarpé.}\end{définition}
\begin{exemple}\pnru{lɑ˧bi˧-tsɤ˧}\hspace{5pt}\peng{steep (literally ‘like a steep slope')}\hspace{5pt}\pcmn{‘像陡坡’,等于:很陡}\hspace{5pt}\pfra{raide, escarpé (littéralement ‘comme un escarpement')}\end{exemple}
\begin{exemple}\pnru{lɑ˧bi˧-tsɤ˧ | ʐwæ˩˥!}\hspace{5pt}\peng{It is really steep!}\hspace{5pt}\pcmn{陡得很!}\hspace{5pt}\pfra{C'est très pentu! (Littéralement: ‘Ca ressemble vraiment à une pente raide!')}\end{exemple}
\end{entrée}

\begin{entrée}
{lɑ˧do\#˥}{}{ⓔlɑ˧do\#˥}\formedesurface{lɑ˧do˧}\newline
\classe{名词}\ton{\#H}\begin{définition}\peng{Horse groom.}\end{définition}
\begin{définition}\pcmn{马夫(参加马帮)}\end{définition}
\begin{définition}\pfra{Palefrenier, caravanier (employé, pas chef de caravane).}\end{définition}
\end{entrée}

\begin{entrée}
{lɑ˩gv̩˧}{}{ⓔlɑ˩gv̩˧}\formedesurface{lɑ˩gv̩˥}\newline
\classe{形容词}\ton{LM}\begin{définition}\peng{Curved, crooked, bent (e.g. tree).}\end{définition}
\begin{définition}\pcmn{弯(树…)}\end{définition}
\begin{définition}\pfra{Recourbé, tordu, courbe.}\end{définition}
\begin{exemple}\pnru{si˧dzi˩ | lɑ˩-gv̩˧-ze˩}\hspace{5pt}\peng{The tree got crooked.}\hspace{5pt}\pcmn{树弯了。}\hspace{5pt}\pfra{L'arbre est devenu courbé.}\end{exemple}
\end{entrée}

\begin{entrée}
{lɑ˩gv̩˧-lɑ˩ɲi˩}{}{ⓔlɑ˩gv̩˧-lɑ˩ɲi˩}\formedesurface{lɑ˩gv̩˧lɑ˩ɲi˩}\newline
\classe{形容词}\ton{LM-L}
\étymologie{
lɑ˩gv̩˧
}\begin{définition}\peng{Crooked, curved, bent (e.g. road, person's limbs).}\end{définition}
\begin{définition}\pcmn{弯(路,植物,人的四肢)}\end{définition}
\begin{définition}\pfra{Tout tordu, tout recourbé.}\end{définition}
\end{entrée}

\begin{entrée}
{lɑ˧hwɤ˩}{}{ⓔlɑ˧hwɤ˩}\formedesurface{lɑ˧hwɤ˩}\newline
\classe{名词}\ton{L\#}\begin{définition}\peng{A Na village outside the Yongning plain, close to the Lake, not far from \stylefv{/lɑ}˧tʰɑ˧-di˧˥/.}\end{définition}
\begin{définition}\pcmn{村落名}\end{définition}
\begin{définition}\pfra{Village na hors de la plaine de Yongning, vers le Lac, non loin de \stylefv{/lɑ}˧tʰɑ˧-di˧˥/.}\end{définition}
\end{entrée}

\begin{entrée}
{lɑ˩jɤ˧-ɬi˧}{}{ⓔlɑ˩jɤ˧-ɬi˧}\formedesurface{lɑ˩jɤ˧ɬi˧}\newline
\classe{名词}\ton{LM-}\begin{définition}\peng{12th month.}\end{définition}
\begin{définition}\pcmn{十二月}\end{définition}
\begin{définition}\pfra{Le douzième mois.}\end{définition}
\end{entrée}

\begin{entrée}
{lɑ˧kɤ˩}{}{ⓔlɑ˧kɤ˩}\formedesurface{lɑ˧kɤ˩}\newline
\classe{名词}\ton{L\#}
\paradigme{\pcmn{:} \p{}}
\begin{définition}\peng{Small jar used to preserve wine.}\end{définition}
\begin{définition}\pcmn{小坛子,用来存酒}\end{définition}
\begin{définition}\pfra{Petite cruche, petit pot pour l'alcool; sert pour le conserver longtemps, pas seulement pour le verser.}\end{définition}
\end{entrée}

\begin{entrée}
{lɑ˧-kʰv̩˧˥}{₁}{ⓔlɑ˧-kʰv̩˧˥ⓗ1}\formedesurface{lɑ˧kʰv̩˧˥}\newline
\classe{名词}\ton{MH\#}
1\begin{définition}\peng{Year of the Tiger.}\end{définition}
\begin{définition}\pcmn{虎年}\end{définition}
\begin{définition}\pfra{Année du Tigre.}\end{définition}
\end{entrée}

\begin{entrée}
{lɑ˧-kʰv̩˧˥}{₂}{ⓔlɑ˧-kʰv̩˧˥ⓗ2}\formedesurface{lɑ˧kʰv̩˧˥}\newline
\classe{形容词}\ton{MH\#}
2\begin{définition}\peng{Born in the year of the Tiger.}\end{définition}
\begin{définition}\pcmn{属虎}\end{définition}
\begin{définition}\pfra{Né l'année du Tigre.}\end{définition}
\end{entrée}

\begin{entrée}
{lɑ˧∼lɑ˧}{}{ⓔlɑ˧∼lɑ˧}\formedesurface{lɑ˧lɑ˧}\newline
\classe{形容词}\ton{M}\begin{définition}\peng{Flaccid, flabby.}\end{définition}
\begin{définition}\pcmn{松弛}\end{définition}
\begin{définition}\pfra{Ballant, flasque.}\end{définition}
\end{entrée}

\begin{entrée}
{lɑ˧∼lɑ˧β}{}{ⓔlɑ˧∼lɑ˧β}\formedesurface{lɑ˧lɑ˧}\newline
\classe{动词}\ton{Mβ}\begin{définition}\peng{To dilute in water.}\end{définition}
\begin{définition}\pcmn{掺水}\end{définition}
\begin{définition}\pfra{Diluer (dans l’eau).}\end{définition}
\begin{exemple}\pnru{(dʑɯ˧-qo˧) le˧-lɑ˧∼lɑ˧}\hspace{5pt}\peng{to dilute in water}\hspace{5pt}\pcmn{掺水}\hspace{5pt}\pfra{diluer dans de l’eau}\end{exemple}
\end{entrée}

\begin{entrée}
{lɑ˩∼lɑ˧˥}{}{ⓔlɑ˩∼lɑ˧˥}\formedesurface{lɑ˩lɑ˧˥}\newline
\classe{动词}\ton{MH}\begin{définition}\peng{To fight, to scuffle, to come to blows.}\end{définition}
\begin{définition}\pcmn{打架、吵架}\end{définition}
\begin{définition}\pfra{Se disputer, se battre.}\end{définition}
\begin{exemple}\pnru{lɑ˩lɑ˧-hĩ˥ | ʈʂʰɯ˧-tɕi˩}\hspace{5pt}\peng{those people who are fighting}\hspace{5pt}\pcmn{打架的这些(人)}\hspace{5pt}\pfra{ces (gens) qui se disputent}\end{exemple}
\end{entrée}

\begin{entrée}
{lɑ˧lo˧-ʁwɤ˥}{}{ⓔlɑ˧lo˧-ʁwɤ˥}\formedesurface{lɑ˧lo˧ʁwɤ˥}\newline
\classe{名词}\ton{H\#}\begin{définition}\peng{A village of Yongning; Chinese name: Laluowa.}\end{définition}
\begin{définition}\pcmn{拉洛瓦村(永宁坝子的一个村落)}\end{définition}
\begin{définition}\pfra{Un village de Yongning; prononciation chinoise: Laluowa.}\end{définition}
\begin{exemple}\pnru{dʑɤ˩bv̩˧kɤ˧-sɑ˥ʁwɤ˩, | hi˩ʁwɤ˩-lo˥, | æ˩mi˧-ʁwɤ\#˥, | lɑ˧lo˧-ʁwɤ˥, | lɑ˧ŋwɤ˧, | bɤ˧tsʰo˧gv̩˥, | ə˧lɑ˧-ʁwɤ\#˥, | gæ˧ɻæ˩, | qʰæ˧tɕʰi˧, | tʰo˧ʈɯ\#˥}\hspace{5pt}\peng{The ten Na villages considered in traditional geography as belonging to the vicinity of the Yongning temple.}\hspace{5pt}\pcmn{永宁摩梭地理概念中,距离扎美寺最近的十个村落:佳部嘎萨瓦、习瓦洛、阿咪瓦、拉洛瓦、拉瓦、巴搓古、阿拉瓦、嘎尔、开基、拖支。}\hspace{5pt}\pfra{Les dix villages na traditionnellement considérés comme appartenant au voisinage du temple de Yongning.}\end{exemple}
\end{entrée}

\begin{entrée}
{lɑ˧ɬɑ˧˥}{₁}{ⓔlɑ˧ɬɑ˧˥ⓗ1}\formedesurface{lɑ˧ɬɑ˧˥}\newline
\classe{代词}\ton{MH\#}
1\begin{définition}\peng{Other.}\end{définition}
\begin{définition}\pcmn{别的}\end{définition}
\begin{définition}\pfra{Autre, autres.}\end{définition}
\begin{exemple}\pnru{lɑ˧ɬɑ˧˥ | ɖɯ˧-tɕi˥}\hspace{5pt}\peng{some others, a few others}\hspace{5pt}\pcmn{其它一些}\hspace{5pt}\pfra{quelques autres}\end{exemple}
\begin{exemple}\pnru{lɑ˧ɬɑ˧˥ | ʈʂʰɯ˧-tɕi˩}\hspace{5pt}\peng{those other, those few others, the few that remained}\hspace{5pt}\pcmn{其它的那些}\hspace{5pt}\pfra{ces quelques autres, ceux qui restent}\end{exemple}
\begin{exemple}\pnru{lɑ˧ɬɑ˧˥ | ɖɯ˧-ʁo˩ ɲi˩!}\hspace{5pt}\peng{It's something different! / That's a different matter!}\hspace{5pt}\pcmn{是另一回事! / 是另一码事!}\hspace{5pt}\pfra{C'est autre chose! / Ca, c'est différent!}\end{exemple}
\end{entrée}

\begin{entrée}
{lɑ˧ɬɑ˧˥}{₂}{ⓔlɑ˧ɬɑ˧˥ⓗ2}\formedesurface{lɑ˧ɬɑ˧˥}\newline
\classe{形容词}\ton{MH\#}
2\begin{définition}\peng{Other.}\end{définition}
\begin{définition}\pcmn{别的}\end{définition}
\begin{définition}\pfra{Autre.}\end{définition}
\begin{exemple}\pnru{lɑ˧ɬɑ˧ hĩ˥}\hspace{5pt}\peng{other people}\hspace{5pt}\pcmn{其它人}\hspace{5pt}\pfra{les autres gens}\end{exemple}
\begin{exemple}\pnru{ɖɯ˧-bæ˧ | le˧-se˩, | ɖɯ˧-bæ˧ ʝi˧! / ɖɯ˧-bæ˧ | le˧-se˩, | wɤ˩˥ | lɑ˧ɬɑ˧˥ | ɖɯ˧-bæ˧ ʝi˧! |}\hspace{5pt}\peng{When one has finished one task, one moves on to another!}\hspace{5pt}\pcmn{做完一件事情,就轮到另一个!}\hspace{5pt}\pfra{Quand on a fini une chose/une tâche, on en fait une autre / on passe à une autre!}\end{exemple}
\end{entrée}

\begin{entrée}
{lɑ˧ɬɑ˧˥}{₃}{ⓔlɑ˧ɬɑ˧˥ⓗ3}\formedesurface{lɑ˧ɬɑ˧˥}\newline
\classe{连接词}\ton{MH\#}
3\begin{définition}\peng{Apart from, aside of, other than.}\end{définition}
\begin{définition}\pcmn{这以外}\end{définition}
\begin{définition}\pfra{À part, en dehors de.}\end{définition}
\begin{exemple}\pnru{tsɑ˧bɤ˧ mɤ˧-pʰv̩˧ɖɯ˧! | lɑ˧ɬɑ˧˥, | ə˧tso˧-mɤ˧-ɲi˩ | pʰv̩˩ɖɯ˩˥!}\hspace{5pt}\peng{Flour is not expensive; apart from it, everything is expensive! / Flour is cheap; but everything else is expensive! (An observation about the cost of living in early 21st-century Yongning)}\hspace{5pt}\pcmn{面粉不贵。其它的呢,什么都贵!(题目:讲今日永宁食品物价)}\hspace{5pt}\pfra{La farine n'est pas chère; à part ça, tout est cher! (Réflexion au sujet du coût de la vie dans la région aujourd'hui)}\end{exemple}
\end{entrée}

\begin{entrée}
{lɑ˧mɑ˧}{}{ⓔlɑ˧mɑ˧}\formedesurface{lɑ˧mɑ˧}\newline
\classe{名词}\ton{M}
\paradigme{\pcmn{:} \p{}}
\begin{définition}\peng{Lama.}\end{définition}
\begin{définition}\pcmn{喇嘛}\end{définition}
\begin{définition}\pfra{Lama.}\end{définition}
\begin{exemple}\pnru{hæ˧-lɑ˩mɑ˩}\hspace{5pt}\peng{Chinese lama}\hspace{5pt}\pcmn{汉族喇嘛}\hspace{5pt}\pfra{lama chinois}\end{exemple}
\end{entrée}

\begin{entrée}
{lɑ˩mɑ˩}{}{ⓔlɑ˩mɑ˩}\formedesurface{lɑ˩mɑ˩˥}\newline
\classe{名词}\ton{L}\begin{définition}\peng{A family name from Yongning. There are four families in Yongning that carry this name.}\end{définition}
\begin{définition}\pcmn{一个姓。这个姓,永宁有四个家}\end{définition}
\begin{définition}\pfra{Nom de clan/famille étendue. Quatre familles portent ce nom à Yongning.}\end{définition}
\begin{exemple}\pnru{lɑ˩mɑ˩=ɻ̍˥\$}\hspace{5pt}\peng{the /lɑ˩mɑ˩/ clan, the /lɑ˩mɑ˩/ family}\hspace{5pt}\pcmn{|fv{/lɑ˩mɑ˩/}家族}\hspace{5pt}\pfra{le clan /lɑ˩mɑ˩/, la famille /lɑ˩mɑ˩/}\end{exemple}
\begin{exemple}\pnru{lɑ˩mɑ˩-gv̩˥mɑ˩}\hspace{5pt}\peng{the name of a person, containing both a family name: /lɑ˩mɑ˩/, and a given name: /gv̩˧mɑ˧/}\hspace{5pt}\pcmn{一个人的名字:姓为|fv{/lɑ˩mɑ˩/},名为|fv{/gv̩˧mɑ˧/}}\hspace{5pt}\pfra{nom d'une personne, comportant un nom de famille (/lɑ˩mɑ˩/) et un prénom (/gv̩˧mɑ˧/)}\end{exemple}
\end{entrée}

\begin{entrée}
{lɑ˧mi\#˥}{}{ⓔlɑ˧mi\#˥}\formedesurface{lɑ˧mi˧}\newline
\classe{名词}\ton{\#H}
\paradigme{\pcmn{:} \p{}}
\begin{définition}\peng{Female tiger.}\end{définition}
\begin{définition}\pcmn{母老虎}\end{définition}
\begin{définition}\pfra{Tigresse.}\end{définition}
\begin{exemple}\pnru{lɑ˧mi˧ tʰv̩˧-mi˧˥ / lɑ˧mi˧ tʰv̩˧-mi˥\#}\hspace{5pt}\peng{|fg{n}+|fg{dem}+|fg{clf}}\hspace{5pt}\pcmn{那只老虎}\hspace{5pt}\pfra{|fg{n}+|fg{dem}+|fg{clf}}\end{exemple}
\end{entrée}

\begin{entrée}
{lɑ˧ŋwɤ˧}{}{ⓔlɑ˧ŋwɤ˧}\formedesurface{lɑ˧ŋwɤ˧}\newline
\classe{名词}\ton{M}
\sens{1}
\begin{définition}\peng{The name of a mountain on the way from Yongning to Wujiao.}\end{définition}
\begin{définition}\pcmn{拉瓦山:一座山的名字。}\end{définition}
\begin{définition}\pfra{Nom de montagne, sur le chemin de Yongning à Wujiao.}\end{définition}\sens{2}
\begin{définition}\peng{The name of a hamlet on the slope of the Langua mountain.}\end{définition}
\begin{définition}\pcmn{拉瓦村:拉瓦山上的一个村落。}\end{définition}
\begin{définition}\pfra{Nom d'un hameau qui se trouvent sur la montagne Langua.}\end{définition}
\begin{exemple}\pnru{dʑɤ˩bv̩˧kɤ˧-sɑ˥ʁwɤ˩, | hi˩ʁwɤ˩-lo˥, | æ˩mi˧-ʁwɤ\#˥, | lɑ˧lo˧-ʁwɤ˥, | lɑ˧ŋwɤ˧, | bɤ˧tsʰo˧gv̩˥, | ə˧lɑ˧-ʁwɤ\#˥, | gæ˧ɻæ˩, | qʰæ˧tɕʰi˧, | tʰo˧ʈɯ\#˥}\hspace{5pt}\peng{The ten Na villages considered in traditional geography as belonging to the vicinity of the Yongning temple.}\hspace{5pt}\pcmn{永宁摩梭地理概念中,距离扎美寺最近的十个村落:佳部嘎萨瓦、习瓦洛、阿咪瓦、拉洛瓦、拉瓦、巴搓古、阿拉瓦、嘎尔、开基、拖支。}\hspace{5pt}\pfra{Les dix villages na traditionnellement considérés comme appartenant au voisinage du temple de Yongning.}\end{exemple}
\end{entrée}

\begin{entrée}
{lɑ˧pʰv̩\#˥}{}{ⓔlɑ˧pʰv̩\#˥}\formedesurface{lɑ˧pʰv̩˧}\newline
\classe{名词}\ton{\#H}
\paradigme{\pcmn{:} \p{}}
\begin{définition}\peng{Male tiger.}\end{définition}
\begin{définition}\pcmn{公老虎}\end{définition}
\begin{définition}\pfra{Tigre (mâle).}\end{définition}
\begin{exemple}\pnru{lɑ˧pʰv̩˧ tʰv̩˧-ɭɯ\#˥}\hspace{5pt}\peng{|fg{n}+|fg{dem}+|fg{clf}}\hspace{5pt}\pcmn{那只老虎}\hspace{5pt}\pfra{|fg{n}+|fg{dem}+|fg{clf}}\end{exemple}
\end{entrée}

\begin{entrée}
{lɑ˩tɑ˧}{}{ⓔlɑ˩tɑ˧}\formedesurface{lɑ˩tɑ˥}\newline
\classe{形容词}\ton{LM}\begin{définition}\peng{Askew, slanting (e.g. hat).}\end{définition}
\begin{définition}\pcmn{歪,偏 (帽子戴得歪)}\end{définition}
\begin{définition}\pfra{De biais, de travers (ex.: porter son chapeau de travers).}\end{définition}
\end{entrée}

\begin{entrée}
{‑lɑ˩tɑ˩}{}{ⓔ‑lɑ˩tɑ˩}\formedesurface{lɑ˩tɑ˩˥}\newline
\classe{}\ton{L}\begin{définition}\peng{Close to.}\end{définition}
\begin{définition}\pfra{À proximité de.}\end{définition}
\begin{exemple}\pnru{ɑ˩ʁo˧ | -lɑ˩tɑ˩˥}\hspace{5pt}\peng{the perimeter of the house: the surface on which the house (farm) extends}\hspace{5pt}\pcmn{家的面积}\hspace{5pt}\pfra{le périmètre de la maison, là où s'étend le domaine de la maison}\end{exemple}
\end{entrée}

\begin{entrée}
{lɑ˧tʰɑ˧-di˧˥}{}{ⓔlɑ˧tʰɑ˧-di˧˥}\formedesurface{lɑ˧tʰɑ˧di˧˥}\newline
\classe{名词}\ton{MH\#}\begin{définition}\peng{The entire Na area around Lake Lugu, including Zuosuo (currently Luguhu Zhen) and the village of Luoshui.}\end{définition}
\begin{définition}\pcmn{泸沽湖周边的摩梭地区,包括左所(今为泸沽湖镇)、洛水村等}\end{définition}
\begin{définition}\pfra{La région na qui entoure le lac Lugu: Zuosuo (actuel Luguhu Zhen), le village de Luoshui, et les autres localités du bord du Lac.}\end{définition}
\begin{exemple}\pnru{ɬi˧ki˧, | ɲi˧se˩, | tɑ˧dzi˩, | mv̩˧qʰwæ˩, | lɑ˧tʰɑ˧-di˧˥}\hspace{5pt}\peng{Villages that one passes when moving away from the Yongning plain, towards Lake Lugu. These villages do not count as part of Yongning proper. The last, /lɑ˧tʰɑ˧-di˧˥/, is not a village name like the preceding four: it refers to the entire Na area beyond the fourth village.}\hspace{5pt}\pcmn{从永宁往泸沽湖所经过的村落,依次是:里格、尼赛、大祖、木垮,然后到拉塔地(拉塔地指的是泸沽湖周边的摩梭地区,包括左所、洛水村等)}\hspace{5pt}\pfra{Villages dans l'ordre, après la plaine de Yongning, ne comptant pas comme faisant partie de Yongning. Le dernier, /lɑ˧tʰɑ˧-di˧˥/, désigne toute la région na au-delà du quatrième village.}\end{exemple}
\end{entrée}

\begin{entrée}
{lɑ˧tʰɑ˧mi˥\$}{}{ⓔlɑ˧tʰɑ˧mi˥\$}\formedesurface{lɑ˧tʰɑ˧mi˥}\newline
\classe{名词}\ton{H\$}\begin{définition}\peng{A family name from Yongning. There are five families in Yongning that carry this name. This is one of the first three clans who settled in the vicinity of the Yongning monastery, the other two being \stylefv{/kɤ}˧˥tʰɑ˩/ and \stylefv{/ə}˧lɑ˧/.}\end{définition}
\begin{définition}\pcmn{一个姓。这个姓,永宁有五个家。音译:拉他咪}\end{définition}
\begin{définition}\pfra{Nom de clan/famille étendue. Cinq familles portent ce nom à Yongning. C'est l'un des trois premiers clans à s'être établis à proximité du monastère de Yongning, les deux autres étant \stylefv{/kɤ}˧˥tʰɑ˩/ et \stylefv{/ə}˧lɑ˧/.}\end{définition}
\begin{exemple}\pnru{lɑ˧tʰɑ˧mi˧=ɻ̍˥\$}\hspace{5pt}\peng{the /lɑ˧tʰɑ˧mi˥\$/ clan, the /lɑ˧tʰɑ˧mi˥\$/ family}\hspace{5pt}\pcmn{|fv{/lɑ˧tʰɑ˧mi˥\$/}家族}\hspace{5pt}\pfra{le clan /lɑ˧tʰɑ˧mi˥\$/, la famille /lɑ˧tʰɑ˧mi˥\$/}\end{exemple}
\end{entrée}

\begin{entrée}
{lɑ˧tʰɑ˧mi˥-ʈæ˧ʂɯ˧-lɑ˩mv̩˩}{}{ⓔlɑ˧tʰɑ˧mi˥-ʈæ˧ʂɯ˧-lɑ˩mv̩˩}\formedesurface{lɑ˧tʰɑ˧mi˥ʈæ˧ʂɯ˧lɑ˩mv̩˩}\newline
\classe{名词}\ton{H\#-M-L}\begin{définition}\peng{Proper name of the main consultant (reference speaker) for this volume (speaker code: F4).}\end{définition}
\begin{définition}\pcmn{拉他咪•达石拉么:本著作的标准发音合作人}\end{définition}
\begin{définition}\pfra{Nom propre (nom de famille et prénom) de la consultante de référence du présent travail (code locutrice: F4).}\end{définition}
\end{entrée}

\begin{entrée}
{lɑ˩ʈʂv̩˩}{}{ⓔlɑ˩ʈʂv̩˩}\formedesurface{lɑ˩ʈʂv̩˩˥}\newline
\classe{名词}\ton{L}
\paradigme{\pcmn{:} \p{}}
\begin{définition}\peng{Candle.}\end{définition}
\begin{définition}\pcmn{蜡烛}\end{définition}
\begin{définition}\pfra{Bougie.}\end{définition}
\end{entrée}

\begin{entrée}
{lɑ˧zi˥}{}{ⓔlɑ˧zi˥}\formedesurface{lɑ˧zi˥}\newline
\classe{名词}\ton{H\#}\begin{définition}\peng{Painter.}\end{définition}
\begin{définition}\pcmn{画家}\end{définition}
\begin{définition}\pfra{Peintre (activité qui n'est pas réservée aux moines).}\end{définition}
\begin{exemple}\pnru{ʈʂʰɯ˧-v̩˧, | lɑ˧zi˥ ɲi˩!}\hspace{5pt}\peng{(S)he is a painter! / (S)he can paint!}\hspace{5pt}\pcmn{他是画家!}\hspace{5pt}\pfra{elle/il est peintre! / elle/il sait peindre!}\end{exemple}
\end{entrée}

\begin{entrée}
{lɑ˧zo\#˥}{}{ⓔlɑ˧zo\#˥}\formedesurface{lɑ˧zo˧}\newline
\classe{名词}\ton{\#H}
\paradigme{\pcmn{:} \p{}}
\begin{définition}\peng{Baby tiger, child of tiger.}\end{définition}
\begin{définition}\pcmn{小老虎}\end{définition}
\begin{définition}\pfra{Petit tigre.}\end{définition}
\begin{exemple}\pnru{lɑ˧zo˧ tʰv̩˧-ɭɯ\#˥}\hspace{5pt}\peng{|fg{n}+|fg{dem}+|fg{clf}}\hspace{5pt}\pcmn{那只小老虎}\hspace{5pt}\pfra{|fg{n}+|fg{dem}+|fg{clf}}\end{exemple}
\end{entrée}

\begin{entrée}
{‑læ˧}{}{ⓔ‑læ˧}\formedesurface{læ˧}\newline
\classe{后缀}\ton{M}\begin{définition}\peng{This |fg{top} marker introduces a new element, without necessarily contrasting it with others. Possible gloss: concerning… .}\end{définition}
\begin{définition}\pcmn{主题:……的话、关于……}\end{définition}
\begin{définition}\pfra{Topique, introduisant un élément nouveau, pas nécessairement en contraste avec ce qui précède. Gloses possibles: pour ce qui est de, en ce qui concerne, quant à.}\end{définition}
\begin{exemple}\pnru{ɖɯ˩mɑ˧ | -læ˧…}\hspace{5pt}\peng{Concerning (my granddaughter) ɖɯ˩mɑ˧, …}\hspace{5pt}\pcmn{关于独妈呢,……}\hspace{5pt}\pfra{pour ce qui est de ma petite-fille ɖɯ˩mɑ˧, eh bien…}\end{exemple}
\begin{exemple}\pnru{lɑ˩mv̩˩˥ | -læ˧…}\hspace{5pt}\peng{Concerning lɑ˩mv̩˩˥ [a given name], …}\hspace{5pt}\pcmn{关于拉姆呢,……}\hspace{5pt}\pfra{pour ce qui est de lɑ˩mv̩˩˥ [nom propre], …}\end{exemple}
\begin{exemple}\pnru{ti˧ɖo˥ | -læ˧…}\hspace{5pt}\peng{Concerning ti˧ɖo˥ [a given name], …}\hspace{5pt}\pcmn{关于|fv{ti˧ɖo˥}[人的名字]呢,……}\hspace{5pt}\pfra{pour ce qui est de ti˧ɖo˥ [nom propre], …}\end{exemple}
\end{entrée}

\begin{entrée}
{læ˧dæ˧qæ˥}{}{ⓔlæ˧dæ˧qæ˥}\formedesurface{læ˧dæ˧qæ˥}\newline
\classe{名词}\ton{H\#}
\paradigme{\pcmn{:} \p{}}
\begin{définition}\peng{Armpit.}\end{définition}
\begin{définition}\pcmn{腋下}\end{définition}
\begin{définition}\pfra{Aisselle.}\end{définition}
\end{entrée}

\begin{entrée}
{læ˧ʁæ˥\$}{}{ⓔlæ˧ʁæ˥\$}\formedesurface{læ˧ʁæ˥}\newline
\classe{名词}\ton{H\$}
\paradigme{\pcmn{:} \p{}}
\begin{définition}\peng{Raven.}\end{définition}
\begin{définition}\pcmn{乌鸦}\end{définition}
\begin{définition}\pfra{Corbeau.}\end{définition}
\end{entrée}

\begin{entrée}
{læ˧ʁæ˧mi˥\$}{}{ⓔlæ˧ʁæ˧mi˥\$}\formedesurface{læ˧ʁæ˧mi˥}\newline
\classe{名词}\ton{H\$}
\paradigme{\pcmn{:} \p{}}
\begin{définition}\peng{Female raven.}\end{définition}
\begin{définition}\pcmn{母乌鸦}\end{définition}
\begin{définition}\pfra{Corbeau femelle.}\end{définition}
\end{entrée}

\begin{entrée}
{læ˧ʁæ˧-pʰv̩\#˥}{}{ⓔlæ˧ʁæ˧-pʰv̩\#˥}\formedesurface{læ˧ʁæ˧pʰv̩˧}\newline
\classe{名词}\ton{\#H}
\paradigme{\pcmn{:} \p{}}
\begin{définition}\peng{Male raven.}\end{définition}
\begin{définition}\pcmn{公乌鸦}\end{définition}
\begin{définition}\pfra{Corbeau mâle.}\end{définition}
\begin{exemple}\pnru{læ˧ʁæ˧-pʰv̩˧ tʰv̩˧-mi˥\$}\hspace{5pt}\peng{|fg{n}+|fg{dem}+|fg{clf}}\hspace{5pt}\pcmn{那只公乌鸦}\hspace{5pt}\pfra{|fg{n}+|fg{dem}+|fg{clf}}\end{exemple}
\end{entrée}

\begin{entrée}
{læ˧tsɯ˥}{}{ⓔlæ˧tsɯ˥}\formedesurface{læ˧tsɯ˥}\newline
\classe{名词}\ton{H\#}
\paradigme{\pcmn{:} \p{}}
\begin{définition}\peng{Chilly peppers.}\end{définition}
\begin{définition}\pcmn{辣椒(汉语借词:辣子)}\end{définition}
\begin{définition}\pfra{Piment.}\end{définition}
\begin{exemple}\pnru{læ˧tsɯ˥ hṽ̩˩∼hṽ̩˩}\hspace{5pt}\peng{to fry chilly peppers}\hspace{5pt}\pcmn{炒辣椒}\hspace{5pt}\pfra{frire des piments}\end{exemple}
\end{entrée}

\begin{entrée}
{le˧‑}{}{ⓔle˧‑}\formedesurface{le˧}\newline
\classe{前缀}\ton{M/0}\begin{définition}\peng{|fg{accomplished} aspect.}\end{définition}
\begin{définition}\pcmn{实施}\end{définition}
\begin{définition}\pfra{|fg{accomp.}}\end{définition}
\end{entrée}

\begin{entrée}
{le˩}{}{ⓔle˩}\formedesurface{--}\newline
\classe{语气助词}\ton{L?}\begin{définition}\peng{Exclamative final particle.}\end{définition}
\begin{définition}\pcmn{句尾助词:感叹}\end{définition}
\begin{définition}\pfra{Particule finale exclamative.}\end{définition}
\begin{exemple}\pnru{dʑɤ˩ le˥!}\hspace{5pt}\peng{Well done! / Great!}\hspace{5pt}\pcmn{好了!/ 太好了!}\hspace{5pt}\pfra{Bravo!}\end{exemple}
\end{entrée}

\begin{entrée}
{le˧-tɑ˧˥}{}{ⓔle˧-tɑ˧˥}\formedesurface{le˧tɑ˧˥}\newline
\classe{连接词}\ton{MH}\begin{définition}\peng{Up to, all the way to; even.}\end{définition}
\begin{définition}\pcmn{到……为止、一直到……、连……}\end{définition}
\begin{définition}\pfra{Jusqu'à; même.}\end{définition}
\end{entrée}

\begin{entrée}
{le˧-wo˥}{}{ⓔle˧-wo˥}\formedesurface{le˧wo˥}\newline
\classe{助词}\ton{H\#}\begin{définition}\peng{Over again, once over again; back.}\end{définition}
\begin{définition}\pcmn{再、又、重新}\end{définition}
\begin{définition}\pfra{À nouveau, de nouveau.}\end{définition}
\end{entrée}

\begin{entrée}
{le˧-wo˧}{}{ⓔle˧-wo˧}\formedesurface{le˧wo˧}\newline
\classe{助词}\ton{M}\begin{définition}\peng{Again; back.}\end{définition}
\begin{définition}\pcmn{又,……回去}\end{définition}
\begin{définition}\pfra{À nouveau.}\end{définition}
\begin{exemple}\pnru{le˧-wo˧ jo˧}\hspace{5pt}\peng{to come back}\hspace{5pt}\pcmn{回}\hspace{5pt}\pfra{revenir}\end{exemple}
\begin{exemple}\pnru{le˧-wo˧ le˧-gv̩˩}\hspace{5pt}\peng{to do over again}\hspace{5pt}\pcmn{从头开始}\hspace{5pt}\pfra{recommencer}\end{exemple}
\begin{exemple}\pnru{le˧-wo˧ le˧-gv̩˧∼gv̩˥}\hspace{5pt}\peng{to build anew, to make anew, to rebuild}\hspace{5pt}\pcmn{重新做、重新建}\hspace{5pt}\pfra{refaire, réinstaller, reconstruire}\end{exemple}
\begin{exemple}\pnru{le˧-wo˧ le˥-tɕo˩ ʐwɤ˩}\hspace{5pt}\peng{to speak over and over again, to rant, to repeat ceaselessly}\hspace{5pt}\pcmn{重复讲说过的话}\hspace{5pt}\pfra{répéter sans arrêt}\end{exemple}
\end{entrée}

\begin{entrée}
{le˧wo˧-tʰo˥tɕo˩}{}{ⓔle˧wo˧-tʰo˥tɕo˩}\formedesurface{le˧wo˧tʰo˥tɕo˩}\newline
\classe{助词}\ton{\#H-}\begin{définition}\peng{Then, afterwards.}\end{définition}
\begin{définition}\pcmn{后来(原意:翻篇)}\end{définition}
\begin{définition}\pfra{Ensuite; vient d'un sens littéral ‘se retourner’.}\end{définition}
\end{entrée}

\begin{entrée}
{li˧α}{}{ⓔli˧α}\newline
\classe{动词}
\sens{1}
\begin{définition}\peng{To look at.}\end{définition}
\begin{définition}\pcmn{看}\end{définition}
\begin{définition}\pfra{Regarder.}\end{définition}
\begin{exemple}\pnru{tʰi˧-li˧-dʑo˧}\hspace{5pt}\peng{|fg{dur} \_ |fg{prog}}\hspace{5pt}\pcmn{正在看}\hspace{5pt}\pfra{|fg{dur} \_ |fg{prog}}\end{exemple}
\begin{exemple}\pnru{tso˧∼tso˧ li˩}\hspace{5pt}\peng{to look at things}\hspace{5pt}\pcmn{看东西}\hspace{5pt}\pfra{regarder des choses}\end{exemple}\sens{2}
\begin{définition}\peng{To manage, to be in charge of.}\end{définition}
\begin{définition}\pcmn{管理}\end{définition}
\begin{définition}\pfra{S'occuper de.}\end{définition}
\begin{exemple}\pnru{ɑ˩ʁo˧ li˧}\hspace{5pt}\peng{to manage the household, to look after the house; to keep watch over the house}\hspace{5pt}\pcmn{管家、管家里的事情,看守家}\hspace{5pt}\pfra{s'occuper de la maison, veiller aux affaires de la maison; surveiller la maison}\end{exemple}\sens{3}
\begin{définition}\peng{To visit, to go and see (someone).}\end{définition}
\begin{définition}\pcmn{访问}\end{définition}
\begin{définition}\pfra{Rendre visite à, aller voir (quelqu'un).}\end{définition}
\begin{exemple}\pnru{pʰæ˧tɕi˥-zo˩-ɳɯ˩ | mv̩˩zo˩ li˥}\hspace{5pt}\peng{The young man sees the young woman. (Euphemistic phrasing, meaning “the young man has sexual intercourse with the young woman".)}\hspace{5pt}\pcmn{小伙子去拜访年轻女人(委婉语,指性交)}\hspace{5pt}\pfra{Le jeune homme voit (=va voir) la jeune femme. (Euphémisme pour désigner les relations sexuelles.)}\end{exemple}
\end{entrée}

\begin{entrée}
{li˩˥}{}{ⓔli˩˥}\formedesurface{li˩˥}\newline
\classe{名词}\ton{LH}
\paradigme{\pcmn{:} \p{}}
\begin{définition}\peng{Tea.}\end{définition}
\begin{définition}\pcmn{茶}\end{définition}
\begin{définition}\pfra{Thé.}\end{définition}
\begin{exemple}\pnru{li˩qʰɑ˩}\hspace{5pt}\peng{‘bitter tea': herbal tea prepared with leaves of Chinese peony, when there was no tea available; it had medicinal properties}\hspace{5pt}\pcmn{‘苦茶’:用白芍药来泡的饮料,没有茶的时候就喝这种‘苦茶’。它有医疗作用,帮助消化。}\hspace{5pt}\pfra{‘thé amer': décoction de feuilles de pivoine blanche de Chine, que l'on buvait lorsqu'il n'y avait pas de thé à la maison; cela possédait des vertus médicinales.}\end{exemple}
\begin{exemple}\pnru{li˩ tɕɤ˧-bi˧! |}\hspace{5pt}\peng{Tea time! / Let's prepare some tea! (Literally ‘boil tea': tea used to be prepared as a decoction, not as an infusion.)}\hspace{5pt}\pcmn{煮茶吧!(在过去,茶不是泡在开水中,而是煮在锅里。)}\hspace{5pt}\pfra{On va faire du thé! (Le verbe employé est ‘faire bouillir' et non ‘faire infuser': autrefois, le thé était préparé en décoction, non en infusion.)}\end{exemple}
\begin{exemple}\pnru{li˩ ʈʰɯ˩-bi˩˥!}\hspace{5pt}\peng{Let's drink tea! / We're going to drink tea!}\hspace{5pt}\pcmn{喝茶吧!}\hspace{5pt}\pfra{On va boire du thé! / Prenons du thé!}\end{exemple}
\end{entrée}

\begin{entrée}
{li˩pi˥}{}{ⓔli˩pi˥}\formedesurface{li˩pi˥}\newline
\classe{名词}\ton{LH}
\paradigme{\pcmn{:} \p{}}
\begin{définition}\peng{Tea that has infused for too long, tea dregs.}\end{définition}
\begin{définition}\pcmn{已经泡了太久的茶叶}\end{définition}
\begin{définition}\pfra{Feuille de thé qui a trop infusé, qui est bonne à jeter.}\end{définition}
\end{entrée}

\begin{entrée}
{li˩pʰv̩˩}{}{ⓔli˩pʰv̩˩}\formedesurface{li˩pʰv̩˩˥}\newline
\classe{名词}\ton{L}\begin{définition}\peng{Whiteworm Lichen, |\stylefi{Thamnolia vermicularis}; it used to be gathered on the seventh lunar month. It was used as a herbal tea.}\end{définition}
\begin{définition}\pcmn{雪茶}\end{définition}
\begin{définition}\pfra{Un lichen de montagne, |\stylefi{Thamnolia vermicularis}, employé en décoction. Au souvenir de F4, ce lichen se cueillait au septième mois, seule période où il était abondant; on allait le cueillir en haute montagne.}\end{définition}
\begin{exemple}\pnru{ŋwɤ˧hɑ̃˩-li˩pʰv˩}\hspace{5pt}\peng{lichen tea from the mountain ŋwɤ˧hɑ̃˩ (this type of lichen grows abundantly on that mountain, and was generally harvested there)}\hspace{5pt}\pcmn{ŋwɤ˧hɑ̃˩ 山的雪茶(说明:这种苔藓在那座山上多,七月份人家去采)}\hspace{5pt}\pfra{thé de lichen de la montagne ŋwɤ˧hɑ̃˩ (ce lichen y est abondant; c'est généralement là-bas qu'on le cueillait)}\end{exemple}
\end{entrée}

\begin{entrée}
{li˧ʐv̩˩}{}{ⓔli˧ʐv̩˩}\formedesurface{li˧ʐv̩˩}\newline
\classe{名词}\ton{L\#}\begin{définition}\peng{Tenderloins.}\end{définition}
\begin{définition}\pcmn{里脊肉}\end{définition}
\begin{définition}\pfra{Filet-mignon.}\end{définition}
\end{entrée}

\begin{entrée}
{ljɤ˩˥}{₁}{ⓔljɤ˩˥ⓗ1}\formedesurface{ljɤ˩˥}\newline
\classe{名词}\ton{LH}
1
\paradigme{\pcmn{:} \p{}}
\begin{définition}\peng{Beam.}\end{définition}
\begin{définition}\pcmn{梁}\end{définition}
\begin{définition}\pfra{Poutre.}\end{définition}
\end{entrée}

\begin{entrée}
{ljɤ˩˥}{₂}{ⓔljɤ˩˥ⓗ2}\formedesurface{ljɤ˩˥}\newline
\classe{名词}\ton{LM?LH?}
2
\paradigme{\pcmn{:} \p{}}
\begin{définition}\peng{Life, existence, destiny, fate.}\end{définition}
\begin{définition}\pcmn{命、生命、命运}\end{définition}
\begin{définition}\pfra{Sort, lot, existence, vie, destinée.}\end{définition}
\begin{exemple}\pnru{no˧ | ljɤ˩ ʈʂʰɯ˧-ljɤ˩-dʑo˩, | qʰæ˩˥ | ʐwæ˩˥!}\hspace{5pt}\peng{You really have a happy lot!}\hspace{5pt}\pcmn{你命好!}\hspace{5pt}\pfra{Tu as une belle vie! Tu as la vie belle!}\end{exemple}
\begin{exemple}\pnru{hĩ˧-ljɤ˥}\hspace{5pt}\peng{human existence, the human lot}\hspace{5pt}\pcmn{人命、人类的命运}\hspace{5pt}\pfra{l'existence humaine}\end{exemple}
\end{entrée}

\begin{entrée}
{ljɤ˩α}{}{ⓔljɤ˩α}\formedesurface{ɖɯ˧ ljɤ˩}\newline
\classe{量词}\ton{Lα}\begin{définition}\peng{Self-classifier for lives/destinies.}\end{définition}
\begin{définition}\pcmn{量词:命、命运}\end{définition}
\begin{définition}\pfra{Auto-classificateur des vies/destins.}\end{définition}
\begin{exemple}\pnru{ʈʂʰɯ˧-ljɤ˥}\hspace{5pt}\peng{|fg{dem} \_ (tone: H\# / H\$)}\hspace{5pt}\pcmn{指示代词 \_}\hspace{5pt}\pfra{|fg{dem} \_ (tone: H\# / H\$)}\end{exemple}
\end{entrée}

\begin{entrée}
{ljɤ˩mi˥}{}{ⓔljɤ˩mi˥}\formedesurface{ljɤ˩mi˥}\newline
\classe{名词}\ton{LH}
\paradigme{\pcmn{:} \p{}}
\begin{définition}\peng{Major (supporting) beam.}\end{définition}
\begin{définition}\pcmn{大梁}\end{définition}
\begin{définition}\pfra{Poutre importante.}\end{définition}
\end{entrée}

\begin{entrée}
{ljɤ˩mi˥-ʈæ˩qo˩}{}{ⓔljɤ˩mi˥-ʈæ˩qo˩}\formedesurface{ljɤ˩mi˥ʈæ˩qo˩}\newline
\classe{名词}\ton{LH-}
\paradigme{\pcmn{:} \p{}}
\begin{définition}\peng{Decoration of major (supporting) beam: symbolically, this is the beam's ‘ear'.}\end{définition}
\begin{définition}\pcmn{大梁的装饰:大梁的‘耳朵’}\end{définition}
\begin{définition}\pfra{Enjolivement sous une poutre; est perçu symboliquement comme ‘l'oreille' de la poutre.}\end{définition}
\end{entrée}

\begin{entrée}
{ljɤ˩ʂɯ˩}{}{ⓔljɤ˩ʂɯ˩}\formedesurface{ljɤ˩ʂɯ˩˥}\newline
\classe{名词}\ton{L}\begin{définition}\peng{Cereals.}\end{définition}
\begin{définition}\pcmn{粮食(汉语借词)}\end{définition}
\begin{définition}\pfra{Céréales.}\end{définition}
\end{entrée}

\begin{entrée}
{lje˩fe˧}{}{ⓔlje˩fe˧}\formedesurface{lje˩fe˥}\newline
\classe{名词}\ton{LM}\begin{définition}\peng{Mungo bean jelly.}\end{définition}
\begin{définition}\pcmn{凉粉}\end{définition}
\begin{définition}\pfra{Liangfen: spécialité de Dali et des environs.}\end{définition}
\end{entrée}

\begin{entrée}
{lo˧}{}{ⓔlo˧}\formedesurface{lo˧}\newline
\classe{名词}
\sens{1}\paradigme{\pcmn{:} \p{}}
\begin{définition}\peng{Work, occupation, task.}\end{définition}
\begin{définition}\pcmn{事情}\end{définition}
\begin{définition}\pfra{Occupation, travail, tâche.}\end{définition}
\begin{exemple}\pnru{lo˧ dʑo˧}\hspace{5pt}\peng{to be busy, to have work to do}\hspace{5pt}\pcmn{忙,有活要干}\hspace{5pt}\pfra{avoir du travail, être occupé}\end{exemple}
\begin{exemple}\pnru{njɤ˧ | lo˧ mɤ˧-dʑo˧.}\hspace{5pt}\peng{I am not busy. / I have some free time.}\hspace{5pt}\pcmn{我不忙。}\hspace{5pt}\pfra{Je ne suis pas occupé. / J'ai du temps libre. / Je suis disponible.}\end{exemple}\sens{2}
\begin{définition}\peng{Usefulness.}\end{définition}
\begin{définition}\pcmn{用处}\end{définition}
\begin{définition}\pfra{Utilité.}\end{définition}
\begin{exemple}\pnru{lo˧ mɤ˧-dʑo˧}\hspace{5pt}\peng{It's no use / it does not have any usefulness. (Context: talking about ivy, which cannot be fed to cattle and is not used for medical purposes, or for firewood, or for making ropes, tools…)}\hspace{5pt}\pcmn{没有用!(情景:谈到常春藤,说它是没有用处的植物)}\hspace{5pt}\pfra{C'est inutile / ça n'a aucune utilité. (Contexte: discussion au sujet du lierre, plante qui n'est utilisée ni comme fourrage, ni comme médicament, ni comme combustible, ni pour la confection de cordes ou autres outils ou objets)}\end{exemple}
\end{entrée}

\begin{entrée}
{lo˧˥}{₁}{ⓔlo˧˥ⓗ1}\formedesurface{lo˧˥}\newline
\classe{形容词}\ton{MH}
1\begin{définition}\peng{Thick.}\end{définition}
\begin{définition}\pcmn{厚}\end{définition}
\begin{définition}\pfra{Épais.}\end{définition}
\begin{exemple}\pnru{ʈʂʰɯ˧ | lo˧-pæ˧-ɻæ˥-gv̩˩!}\hspace{5pt}\peng{It's really thick!}\hspace{5pt}\pcmn{很厚啊!}\hspace{5pt}\pfra{c'est très épais!}\end{exemple}
\end{entrée}

\begin{entrée}
{lo˧˥}{₂}{ⓔlo˧˥ⓗ2}\formedesurface{lo˧˥}\newline
\classe{动词}\ton{MH}
2\begin{définition}\peng{To join hands in an indication of submission.}\end{définition}
\begin{définition}\pcmn{拱手作揖}\end{définition}
\begin{définition}\pfra{Joindre les mains en signe de soumission.}\end{définition}
\begin{exemple}\pnru{tsʰɤ˧tsʰɤ˧ lo˧˥}\hspace{5pt}\peng{to join hands in an indication of submission}\hspace{5pt}\pcmn{拱手作揖}\hspace{5pt}\pfra{rendre hommage à, joindre les mains en signe de soumission/respect}\end{exemple}
\begin{exemple}\pnru{tsʰɤ˧tsʰɤ˧ | le˧-lo˧-ze˥}\hspace{5pt}\peng{|fg{accomp} \_ |fg{pfv}}\hspace{5pt}\pcmn{|fg{accomp} \_ |fg{pfv}}\hspace{5pt}\pfra{|fg{accomp} \_ |fg{pfv}}\end{exemple}
\end{entrée}

\begin{entrée}
{lo˧β}{}{ⓔlo˧β}\formedesurface{ɖɯ˧ lo˧}\newline
\classe{量词}\ton{Mβ}\begin{définition}\peng{Self-classifier for tasks/occupations.}\end{définition}
\begin{définition}\pcmn{量词:事情(一件)、活(一个)}\end{définition}
\begin{définition}\pfra{Auto-classificateur des travaux/occupations.}\end{définition}
\end{entrée}

\begin{entrée}
{lo˩}{₁}{ⓔlo˩ⓗ1}\formedesurface{lo˩˥}\newline
\classe{动词}\ton{L}
1\begin{définition}\peng{To cross (a mountain pass).}\end{définition}
\begin{définition}\pcmn{过(垭口)}\end{définition}
\begin{définition}\pfra{Passer, franchir (un col).}\end{définition}
\begin{exemple}\pnru{mv̩˩tɕo˧-lo˩}\hspace{5pt}\peng{to go down (after crossing a mountain pass)}\hspace{5pt}\pcmn{往下过去(过了垭口以后)}\hspace{5pt}\pfra{descendre (après avoir passé un col)}\end{exemple}
\end{entrée}

\begin{entrée}
{lo˩}{₂}{ⓔlo˩ⓗ2}\formedesurface{lo˧}\newline
\classe{名词}\ton{L}
2
\paradigme{\pcmn{:} \p{}}
\begin{définition}\peng{Mountain valley.}\end{définition}
\begin{définition}\pcmn{山谷}\end{définition}
\begin{définition}\pfra{Vallée de montagne.}\end{définition}
\begin{exemple}\pnru{lo˧-qo˧}\hspace{5pt}\peng{in the valley}\hspace{5pt}\pcmn{山谷里}\hspace{5pt}\pfra{dans la vallée}\end{exemple}
\end{entrée}

\begin{entrée}
{lo˩˧}{}{ⓔlo˩˧}\formedesurface{lo˩˥}\newline
\classe{名词}\ton{LM}\begin{définition}\peng{Thumb.}\end{définition}
\begin{définition}\pcmn{大拇指(单音节,按照双音节词构拟出来的)}\end{définition}
\begin{définition}\pfra{Pouce (forme reconstruite d'après le disyllabe).}\end{définition}
\end{entrée}

\begin{entrée}
{lo˩β}{}{ⓔlo˩β}\formedesurface{ɖɯ˧ lo˩}\newline
\classe{量词}\ton{Lβ}\begin{définition}\peng{Self-classifier for valleys.}\end{définition}
\begin{définition}\pcmn{量词:谷}\end{définition}
\begin{définition}\pfra{Classificateur des vallées.}\end{définition}
\begin{exemple}\pnru{hĩ˧-ɻ̃˧ | ɖɯ˧-lo˩}\hspace{5pt}\peng{literally ‘a valley of people', to mean: all the population of that valley}\hspace{5pt}\pcmn{住在一座山谷里的所有人(直译:‘一山谷的人’)}\hspace{5pt}\pfra{tous les membres d'une grande famille: ‘[toute la population d']une vallée'}\end{exemple}
\begin{exemple}\pnru{si˧dzi˩ | ɖɯ˧-lo˩}\hspace{5pt}\peng{‘a valley [of/covered with] trees', i.e. a large tract of woodland}\hspace{5pt}\pcmn{一山谷的树,一片森林(直译:‘一山谷的树’)}\hspace{5pt}\pfra{une grande quantité d'arbres: ‘une vallée [couverte] d'arbres'}\end{exemple}
\end{entrée}

\begin{entrée}
{lo˧bæ˧˥}{}{ⓔlo˧bæ˧˥}\formedesurface{lo˧bæ˧˥}\newline
\classe{名词}\ton{MH\#}\begin{définition}\peng{Suspended bridge; zip line, flying fox.}\end{définition}
\begin{définition}\pcmn{索桥,溜索}\end{définition}
\begin{définition}\pfra{Pont suspendu; pont de corde. La corde du pont suspendu aurait été faite d'écorces d'arbres, non de chanvre, car les cordes en chanvre se détériorent rapidement quand elles sont exposées à la pluie.}\end{définition}
\end{entrée}

\begin{entrée}
{lo˩bɤ˩}{}{ⓔlo˩bɤ˩}\formedesurface{lo˩bɤ˩˥}\newline
\classe{名词}\ton{L}
\paradigme{\pcmn{:} \p{}}
\begin{définition}\peng{Palm of the hand.}\end{définition}
\begin{définition}\pcmn{手掌}\end{définition}
\begin{définition}\pfra{Paume.}\end{définition}
\end{entrée}

\begin{entrée}
{lo˩bv̩˧-ɭɯ˩}{}{ⓔlo˩bv̩˧-ɭɯ˩}\formedesurface{lo˩bv̩˧ɭɯ˩}\newline
\classe{名词}\ton{LM-L}
\paradigme{\pcmn{:} \p{}}
\begin{définition}\peng{Elbow.}\end{définition}
\begin{définition}\pcmn{肘}\end{définition}
\begin{définition}\pfra{Partie saillante du coude, qd le bras est replié.}\end{définition}
\end{entrée}

\begin{entrée}
{lo˧bv̩˩-ʈʂʰɯ˩}{}{ⓔlo˧bv̩˩-ʈʂʰɯ˩}\formedesurface{lo˧bv̩˩ʈʂʰɯ˩}\newline
\classe{名词}\ton{L\#-}
\paradigme{\pcmn{:} \p{}}
\begin{définition}\peng{Elephant.}\end{définition}
\begin{définition}\pcmn{象、大象}\end{définition}
\begin{définition}\pfra{Éléphant.}\end{définition}
\end{entrée}

\begin{entrée}
{lo˩dv̩\#˥}{}{ⓔlo˩dv̩\#˥}\formedesurface{lo˩dv̩˥}\newline
\classe{名词}\ton{LM+\#H}
\paradigme{\pcmn{:} \p{}}
\begin{définition}\peng{Person with a single arm or hand, one-armed (or one-handed) person.}\end{définition}
\begin{définition}\pcmn{独臂人:缺一只胳膊(手)的人}\end{définition}
\begin{définition}\pfra{Manchot.}\end{définition}
\end{entrée}

\begin{entrée}
{lo˩dzi˩}{}{ⓔlo˩dzi˩}\formedesurface{ɖɯ˧ lo˩dzi˩}\newline
\classe{量词}\ton{L}\begin{définition}\peng{A handful (using both hands).}\end{définition}
\begin{définition}\pcmn{量词:捧(用两只手)}\end{définition}
\begin{définition}\pfra{Classificateur des poignées (à deux mains).}\end{définition}
\end{entrée}

\begin{entrée}
{lo˩dʑo˥}{}{ⓔlo˩dʑo˥}\formedesurface{lo˩dʑo˥}\newline
\classe{名词}\ton{LH}
\paradigme{\pcmn{:} \p{}}
\begin{définition}\peng{Bracelet.}\end{définition}
\begin{définition}\pcmn{手镯}\end{définition}
\begin{définition}\pfra{Bracelet.}\end{définition}
\begin{exemple}\pnru{ŋv̩˩-lo˩dʑo˧ (+ɲi˩)}\hspace{5pt}\peng{silver bracelet}\hspace{5pt}\pcmn{银手镯}\hspace{5pt}\pfra{bracelet en argent}\end{exemple}
\begin{exemple}\pnru{hæ̃˩-lo˩dʑo˥ (+ɲi˩)}\hspace{5pt}\peng{gold bracelet}\hspace{5pt}\pcmn{金手镯}\hspace{5pt}\pfra{bracelet en or}\end{exemple}
\begin{exemple}\pnru{jo˧-lo˥dʑo˩}\hspace{5pt}\peng{jade bracelet}\hspace{5pt}\pcmn{玉手镯}\hspace{5pt}\pfra{bracelet en jade}\end{exemple}
\begin{exemple}\pnru{lo˩dʑo˥ kʰɯ˩}\hspace{5pt}\peng{to put on a bracelet}\hspace{5pt}\pcmn{戴上手镯}\hspace{5pt}\pfra{mettre un bracelet}\end{exemple}
\end{entrée}

\begin{entrée}
{lo˩ɖɯ˧}{}{ⓔlo˩ɖɯ˧}\formedesurface{lo˩ɖɯ˥}\newline
\classe{形容词}\ton{LM}\begin{définition}\peng{Generous.}\end{définition}
\begin{définition}\pcmn{大方}\end{définition}
\begin{définition}\pfra{Généreux.}\end{définition}
\end{entrée}

\begin{entrée}
{lo˧ɖʐɤ˩}{}{ⓔlo˧ɖʐɤ˩}\formedesurface{lo˧ɖʐɤ˩}\newline
\classe{名词}\ton{L\#}
\paradigme{\pcmn{:} \p{}}
\begin{définition}\peng{Weeding hoe: hand instrument with three spikes perpendicular to the handle, to loosen the soil. At the time of fieldwork, this tool had a metal head.}\end{définition}
\begin{définition}\pcmn{三齿耙}\end{définition}
\begin{définition}\pfra{Serfouette, croc à trois dents: instrument à trois dents perpendiculaires au manche, pour ameublir la terre. Les modèles actuellement utilisés ont une tête en métal.}\end{définition}
\begin{exemple}\pnru{lo˧ɖʐɤ˩ ʈʂʰɯ˩-nɑ˩}\hspace{5pt}\peng{|fg{n}+|fg{dem}+|fg{clf}}\hspace{5pt}\pcmn{这把三齿耙}\hspace{5pt}\pfra{|fg{n}+|fg{dem}+|fg{clf}}\end{exemple}
\end{entrée}

\begin{entrée}
{lo˧fv̩˧}{}{ⓔlo˧fv̩˧}\formedesurface{lo˧fv̩˧}\newline
\classe{形容词}\begin{définition}\peng{Easy.}\end{définition}
\begin{définition}\pcmn{容易,容易做}\end{définition}
\begin{définition}\pfra{Facile à faire.}\end{définition}
\begin{exemple}\pnru{lo˧fv̩˧ | ʐwæ˩˥}\hspace{5pt}\peng{very easy}\hspace{5pt}\pcmn{很容易}\hspace{5pt}\pfra{très facile}\end{exemple}
\end{entrée}

\begin{entrée}
{lo˧gv̩˩}{}{ⓔlo˧gv̩˩}\formedesurface{lo˧gv̩˩}\newline
\classe{名词}\ton{L\#}\begin{définition}\peng{Ninglang.}\end{définition}
\begin{définition}\pcmn{宁蒗}\end{définition}
\begin{définition}\pfra{Ninglang; actuellement utilisé pour désigner un village na du comté de Ninglang, relativement proche du centre administratif.}\end{définition}
\begin{exemple}\pnru{lo˧gv̩˩-di˩mi˩}\hspace{5pt}\peng{the Ninglang plain}\hspace{5pt}\pcmn{宁蒗坝子}\hspace{5pt}\pfra{la plaine de Ninglang}\end{exemple}
\end{entrée}

\begin{entrée}
{lo˩-gv̩˧dv̩˧}{}{ⓔlo˩-gv̩˧dv̩˧}\formedesurface{lo˩gv̩˧dv̩˧}\newline
\classe{名词}\ton{L-}
\paradigme{\pcmn{:} \p{}}
\begin{définition}\peng{Back of the hand.}\end{définition}
\begin{définition}\pcmn{手背}\end{définition}
\begin{définition}\pfra{Dos de la main.}\end{définition}
\end{entrée}

\begin{entrée}
{lo˧hɑ˧}{}{ⓔlo˧hɑ˧}\formedesurface{lo˧hɑ˧}\newline
\classe{形容词}\ton{M}\begin{définition}\peng{Difficult, hard to do.}\end{définition}
\begin{définition}\pcmn{难做}\end{définition}
\begin{définition}\pfra{Difficile, dur à faire.}\end{définition}
\begin{exemple}\pnru{ʝi˧ lo˧hɑ˧}\hspace{5pt}\peng{difficult to do}\hspace{5pt}\pcmn{难做}\hspace{5pt}\pfra{difficile à faire}\end{exemple}
\begin{exemple}\pnru{ʝi˧ lo˧hɑ˧ | ʐwæ˩˥}\hspace{5pt}\peng{extremely difficult to do}\hspace{5pt}\pcmn{非常难做}\hspace{5pt}\pfra{très difficile à faire}\end{exemple}
\end{entrée}

\begin{entrée}
{lo˩jɤ˧}{}{ⓔlo˩jɤ˧}\formedesurface{lo˩jɤ˥}\newline
\classe{名词}\ton{LM}\begin{définition}\peng{Silver coin, silver yuan.}\end{définition}
\begin{définition}\pcmn{银元}\end{définition}
\begin{définition}\pfra{Pièce d'argent.}\end{définition}
\begin{exemple}\pnru{lo˩jɤ˧ | ɖɯ˧-pʰæ˧˥}\hspace{5pt}\peng{one silver coin}\hspace{5pt}\pcmn{一块银元}\hspace{5pt}\pfra{une pièce d'argent}\end{exemple}
\end{entrée}

\begin{entrée}
{lo˧ʝi˧}{}{ⓔlo˧ʝi˧}\formedesurface{lo˧ʝi˧}\newline
\classe{动词}\ton{M}\begin{définition}\peng{To work.}\end{définition}
\begin{définition}\pcmn{劳动}\end{définition}
\begin{définition}\pfra{Travailler.}\end{définition}
\end{entrée}

\begin{entrée}
{lo˧ʝi˧-hĩ˧-hĩ\#˥}{}{ⓔlo˧ʝi˧-hĩ˧-hĩ\#˥}\formedesurface{lo˧ʝi˧hĩ˧hĩ˧}\newline
\classe{名词}\ton{\#H}
\paradigme{\pcmn{:} \p{}}
\begin{définition}\peng{Worker (in the fields or elsewhere).}\end{définition}
\begin{définition}\pcmn{劳动人民,农民}\end{définition}
\begin{définition}\pfra{Travailleur (paysan, ouvrier…).}\end{définition}
\end{entrée}

\begin{entrée}
{lo˩ko˧}{}{ⓔlo˩ko˧}\formedesurface{lo˩ko˥}\newline
\classe{名词}\ton{LM}
\paradigme{\pcmn{:} \p{}}
\begin{définition}\peng{Pot for cooking rice, soup…; used to be made of copper.}\end{définition}
\begin{définition}\pcmn{煮饭或煮汤的锣锅。在过去,锣锅一般是铜做的。}\end{définition}
\begin{définition}\pfra{Casserole, pour cuire les céréales, les légumes, les soupes… Elle était autrefois en cuivre.}\end{définition}
\begin{exemple}\pnru{lo˩ko˧: | hɑ˧ tɕɤ˩-di˩! | æ˧-v̩˧, | dʑɯ˩-kʰɯ˩-di˩˥! | ʈʂʰɤ˧ho˥, | dʑɯ˩ tɕɯ˩-di˩˥! |}\hspace{5pt}\peng{The cooking pot is for cooking cereals! The copper pot is for putting water! The boiler is for boiling water! (A summary of the respective uses of the three types of pots in use in Yongning about the middle of the 20th century.)}\hspace{5pt}\pcmn{锣锅,是用来煮饭的!铜锅,是放水用的!水壶,是来煮水的!(描写永宁二十世纪中使用的三种锅)}\hspace{5pt}\pfra{La casserole (/lo˩ko˧/), ça sert à cuire la nourriture! La casserole de cuivre (/æ˧-v̩˧/), ça sert à mettre de l'eau! La bouilloire (/ʈʂʰɤ˩ho˥/), ça sert à faire bouillir l'eau! (Résumé des emplois des trois sortes de casseroles qui étaient en usage à Yongning vers le milieu du XXe siècle.)}\end{exemple}
\end{entrée}

\begin{entrée}
{lo˧lo˧}{}{ⓔlo˧lo˧}\formedesurface{lo˧lo˧}\newline
\classe{名词}\ton{M}
\paradigme{\pcmn{:} \p{}}
\begin{définition}\peng{Yi (ethnic group).}\end{définition}
\begin{définition}\pcmn{彝族}\end{définition}
\begin{définition}\pfra{Yi (groupe ethnique).}\end{définition}
\end{entrée}

\begin{entrée}
{lo˩mæ˩}{}{ⓔlo˩mæ˩}\formedesurface{lo˩mæ˩˥}\newline
\classe{名词}\ton{L}\begin{définition}\peng{The village of Qiansuo.}\end{définition}
\begin{définition}\pcmn{前所}\end{définition}
\begin{définition}\pfra{Qiansuo (localité perçue par F4 comme comportant beaucoup de Yi, et des Chinois/Han, en plus des Na, d'où des contacts linguistiques/emprunts/mélanges).}\end{définition}
\end{entrée}

\begin{entrée}
{lo˩mi˧}{}{ⓔlo˩mi˧}\formedesurface{lo˩mi˥}\newline
\classe{名词}\ton{LM}
\paradigme{\pcmn{:} \p{}}
\begin{définition}\peng{Thumb.}\end{définition}
\begin{définition}\pcmn{大拇指}\end{définition}
\begin{définition}\pfra{Pouce.}\end{définition}
\end{entrée}

\begin{entrée}
{lo˩mi˧-qɑ˩}{}{ⓔlo˩mi˧-qɑ˩}\formedesurface{lo˩mi˧qɑ˩}\newline
\classe{名词}\ton{LM-L}
\paradigme{\pcmn{:} \p{}}
\begin{définition}\peng{Space between thumb and index finger.}\end{définition}
\begin{définition}\pcmn{虎口}\end{définition}
\begin{définition}\pfra{Espace entre le pouce et l'index.}\end{définition}
\end{entrée}

\begin{entrée}
{lo˧nɑ˥-bv̩˩}{}{ⓔlo˧nɑ˥-bv̩˩}\formedesurface{lo˧nɑ˥bv̩˩}\newline
\classe{名词}\ton{H\#-}\begin{définition}\peng{Prison.}\end{définition}
\begin{définition}\pcmn{监狱}\end{définition}
\begin{définition}\pfra{Prison.}\end{définition}
\begin{exemple}\pnru{lo˧nɑ˥-bv̩˩-qo˩ ʈæ˩}\hspace{5pt}\peng{to jail, to put into prison}\hspace{5pt}\pcmn{放在监狱、关在监狱里}\hspace{5pt}\pfra{enfermer en prison, mettre en prison}\end{exemple}
\end{entrée}

\begin{entrée}
{lo˧ɲi˥}{}{ⓔlo˧ɲi˥}\formedesurface{lo˧ɲi˥}\newline
\classe{名词}\ton{H\#}
\paradigme{\pcmn{:} \p{}}
\begin{définition}\peng{Finger.}\end{définition}
\begin{définition}\pcmn{手指}\end{définition}
\begin{définition}\pfra{Doigt.}\end{définition}
\end{entrée}

\begin{entrée}
{lo˧ɲi˥ | ɖɯ˧ɭɯ˧}{}{ⓔlo˧ɲi˥ | ɖɯ˧ɭɯ˧}\formedesurface{lo˧ɲi˥ɖɯ˧ɭɯ˧}\newline
\classe{名词}\ton{H\# | M}\begin{définition}\peng{Index finger.}\end{définition}
\begin{définition}\pcmn{食指}\end{définition}
\begin{définition}\pfra{Index.}\end{définition}
\end{entrée}

\begin{entrée}
{lo˧ɲi˥ | ɲi˧ɭɯ˧}{}{ⓔlo˧ɲi˥ | ɲi˧ɭɯ˧}\formedesurface{lo˧ɲi˥ɲi˧ɭɯ˧}\newline
\classe{名词}\ton{H\# | M}\begin{définition}\peng{Middle finger.}\end{définition}
\begin{définition}\pcmn{中指}\end{définition}
\begin{définition}\pfra{Majeur.}\end{définition}
\end{entrée}

\begin{entrée}
{lo˧ɲi˥ | so˧-ɭɯ˧}{}{ⓔlo˧ɲi˥ | so˧-ɭɯ˧}\formedesurface{lo˧ɲi˥so˧-ɭɯ˧}\newline
\classe{名词}\ton{H\# | M}\begin{définition}\peng{Ring finger, fourth finger.}\end{définition}
\begin{définition}\pcmn{第四根手指}\end{définition}
\begin{définition}\pfra{Annulaire.}\end{définition}
\end{entrée}

\begin{entrée}
{lo˩pv̩˧˥}{}{ⓔlo˩pv̩˧˥}\formedesurface{lo˩pv̩˧˥}\newline
\classe{名词}\ton{LM+MH\#}
\paradigme{\pcmn{:} \p{}}
\begin{définition}\peng{Ring.}\end{définition}
\begin{définition}\pcmn{戒指}\end{définition}
\begin{définition}\pfra{Anneau.}\end{définition}
\begin{exemple}\pnru{ŋv̩˩-lo˩pv̩˩}\hspace{5pt}\peng{silver ring}\hspace{5pt}\pcmn{银戒指}\hspace{5pt}\pfra{anneau en argent}\end{exemple}
\begin{exemple}\pnru{hæ̃˩-lo˩pv̩˩}\hspace{5pt}\peng{gold ring}\hspace{5pt}\pcmn{金戒指}\hspace{5pt}\pfra{anneau en or}\end{exemple}
\end{entrée}

\begin{entrée}
{lo˩qʰv̩˩}{}{ⓔlo˩qʰv̩˩}\formedesurface{lo˩qʰv̩˩˥}\newline
\classe{名词}\ton{L}
\paradigme{\pcmn{:} \p{}}
\begin{définition}\peng{Gully; ravine; valley.}\end{définition}
\begin{définition}\pcmn{山沟}\end{définition}
\begin{définition}\pfra{Vallée, gorge, ravin.}\end{définition}
\end{entrée}

\begin{entrée}
{lo˩qʰwɤ˧}{}{ⓔlo˩qʰwɤ˧}\newline
\classe{名词}
\sens{1}\paradigme{\pcmn{:} \p{}}
\begin{définition}\peng{Arm.}\end{définition}
\begin{définition}\pcmn{胳膊}\end{définition}
\begin{définition}\pfra{Bras.}\end{définition}
\begin{exemple}\pnru{lo˩qʰwɤ˧ li˧}\hspace{5pt}\peng{to look at (the) arm}\hspace{5pt}\pcmn{看胳膊}\hspace{5pt}\pfra{regarder le bras}\end{exemple}\sens{2}
\begin{définition}\peng{Hand.}\end{définition}
\begin{définition}\pcmn{手}\end{définition}
\begin{définition}\pfra{Main.}\end{définition}
\begin{exemple}\pnru{lo˩qʰwɤ˧ ʈʂʰæ˧}\hspace{5pt}\peng{to wash one's hands}\hspace{5pt}\pcmn{洗手}\hspace{5pt}\pfra{se laver les mains}\end{exemple}
\end{entrée}

\begin{entrée}
{lo˩qʰwɤ˧-kʰɯ˧ʑi˧˥}{}{ⓔlo˩qʰwɤ˧-kʰɯ˧ʑi˧˥}\formedesurface{lo˩qʰwɤ˧kʰɯ˧ʑi˧˥}\newline
\classe{名词}\ton{LM+MH\#}\begin{définition}\peng{Glove.}\end{définition}
\begin{définition}\pcmn{手套}\end{définition}
\begin{définition}\pfra{Gant.}\end{définition}
\end{entrée}

\begin{entrée}
{lo˩ʁwæ\#˥}{}{ⓔlo˩ʁwæ\#˥}\formedesurface{lo˩ʁwæ˥}\newline
\classe{名词}\ton{LM+\#H}\begin{définition}\peng{Left-handed person.}\end{définition}
\begin{définition}\pcmn{左撇子}\end{définition}
\begin{définition}\pfra{Gaucher.}\end{définition}
\end{entrée}

\begin{entrée}
{lo˧ʂv̩˩}{}{ⓔlo˧ʂv̩˩}\formedesurface{lo˧ʂv̩˩}\newline
\classe{名词}\ton{L\#}\begin{définition}\peng{The village of Luoshui.}\end{définition}
\begin{définition}\pcmn{洛水村}\end{définition}
\begin{définition}\pfra{Luoshui: village du bord du Lac.}\end{définition}
\end{entrée}

\begin{entrée}
{lo˧ʂv̩˩ | hi˩-nɑ˧mi˧}{}{ⓔlo˧ʂv̩˩ | hi˩-nɑ˧mi˧}\formedesurface{lo˧ʂv̩˩hi˩nɑ˧mi˧}\newline
\classe{名词}\ton{L\# | L-}\begin{définition}\peng{Lake Lugu.}\end{définition}
\begin{définition}\pcmn{泸沽湖}\end{définition}
\begin{définition}\pfra{Lac Lugu.}\end{définition}
\end{entrée}

\begin{entrée}
{lo˧tɑ˧-lo˧tɕi\#˥}{}{ⓔlo˧tɑ˧-lo˧tɕi\#˥}\formedesurface{lo˧tɑ˧lo˧tɕi˧}\newline
\classe{名词}\ton{\#H}
\paradigme{\pcmn{:} \p{}}
\begin{définition}\peng{Streamer of scriptures.}\end{définition}
\begin{définition}\pcmn{经幡、风马旗(挂在山上)}\end{définition}
\begin{définition}\pfra{Drapeau de prières.}\end{définition}
\begin{exemple}\pnru{lo˧tɑ˧-lo˧tɕi˧ | le˧-lɑ˧˥}\hspace{5pt}\peng{to print a streamer of scriptures; more generally: to string together a streamer of scriptures}\hspace{5pt}\pcmn{直译:印出一个经幡。也来指准备经幡的工作(到山上去挂之前)}\hspace{5pt}\pfra{imprimer un drapeau de prières; sens plus général: confectionner un drapeau de prières (chez soi, avant de se rendre sur le lieu où on l'installe)}\end{exemple}
\end{entrée}

\begin{entrée}
{lo˩tʰo˧}{}{ⓔlo˩tʰo˧}\formedesurface{lo˩tʰo˥}\newline
\classe{名词}\ton{LM}
\paradigme{\pcmn{:} \p{}}
\begin{définition}\peng{Handcuffs, chains to tie a criminal's hands.}\end{définition}
\begin{définition}\pcmn{手铐}\end{définition}
\begin{définition}\pfra{Menottes: chaîne de fer pour attacher les mains d'un criminel.}\end{définition}
\begin{exemple}\pnru{lo˩tʰo˧ kʰɯ˧˥}\hspace{5pt}\peng{to put handcuffs, to put on chains (on someone's hands)}\hspace{5pt}\pcmn{戴上手铐}\hspace{5pt}\pfra{passer les menottes à quelqu'un}\end{exemple}
\end{entrée}

\begin{entrée}
{lo˩tsʰɯ˥-sɑ˩}{}{ⓔlo˩tsʰɯ˥-sɑ˩}\formedesurface{lo˩tsʰɯ˥sɑ˩}\newline
\classe{名词}\ton{LH-}\begin{définition}\peng{Meat of the front legs of cattle.}\end{définition}
\begin{définition}\pcmn{牲畜前腿的肉}\end{définition}
\begin{définition}\pfra{Viande des membres antérieurs.}\end{définition}
\end{entrée}

\begin{entrée}
{lo˩ʈv̩˧}{}{ⓔlo˩ʈv̩˧}\formedesurface{lo˩ʈv̩˥}\newline
\classe{名词}\ton{LM}
\paradigme{\pcmn{:} \p{}}
\begin{définition}\peng{Fist.}\end{définition}
\begin{définition}\pcmn{拳}\end{définition}
\begin{définition}\pfra{Poing.}\end{définition}
\end{entrée}

\begin{entrée}
{lo˩ʈʰɯ˧}{}{ⓔlo˩ʈʰɯ˧}\formedesurface{lo˩ʈʰɯ˥}\newline
\classe{名词}\ton{LM}
\paradigme{\pcmn{:} \p{}}
\begin{définition}\peng{Elbow.}\end{définition}
\begin{définition}\pcmn{肘}\end{définition}
\begin{définition}\pfra{Coude.}\end{définition}
\end{entrée}

\begin{entrée}
{lo˩ʈʂæ˧˥}{}{ⓔlo˩ʈʂæ˧˥}\formedesurface{lo˩ʈʂæ˧˥}\newline
\classe{名词}\ton{LM+MH\#}
\paradigme{\pcmn{:} \p{}}
\begin{définition}\peng{Joints of the arm: wrist, elbow.}\end{définition}
\begin{définition}\pcmn{手臂的关节(手腕、肘弯)}\end{définition}
\begin{définition}\pfra{Articulations du bras: le poignet, mais aussi le coude.}\end{définition}
\end{entrée}

\begin{entrée}
{lv̩˥}{}{ⓔlv̩˥}\formedesurface{lv̩˧}\newline
\classe{动词}\ton{H}\begin{définition}\peng{To wind, to coil, to wrap.}\end{définition}
\begin{définition}\pcmn{缠(线……)、裹(毡子……)}\end{définition}
\begin{définition}\pfra{Enrouler (du fil); emballer.}\end{définition}
\begin{exemple}\pnru{le˧-qo˥-lv̩˩}\hspace{5pt}\peng{to wrap, to coil}\hspace{5pt}\pcmn{裹起来}\hspace{5pt}\pfra{enrouler}\end{exemple}
\begin{exemple}\pnru{kʰɯ˧ qo˧-lv̩˥}\hspace{5pt}\peng{to wind a thread}\hspace{5pt}\pcmn{缠线}\hspace{5pt}\pfra{enrouler du fil}\end{exemple}
\begin{exemple}\pnru{qo˧-lv̩˩}\hspace{5pt}\peng{to wrap, to coil}\hspace{5pt}\pcmn{裹}\hspace{5pt}\pfra{même sens: enrouler}\end{exemple}
\end{entrée}

\begin{entrée}
{lv̩˧}{₁}{ⓔlv̩˧ⓗ1}\formedesurface{lv̩˧}\newline
\classe{名词}\ton{M}
1
\paradigme{\pcmn{:} \p{}}
\begin{définition}\peng{Field.}\end{définition}
\begin{définition}\pcmn{田地}\end{définition}
\begin{définition}\pfra{Champs.}\end{définition}
\end{entrée}

\begin{entrée}
{lv̩˧}{₂}{ⓔlv̩˧ⓗ2}\formedesurface{lv̩˧}\newline
\classe{名词}\ton{M}
2\begin{définition}\peng{Cereals for horses or cattle.}\end{définition}
\begin{définition}\pcmn{喂给马或牛的粮食}\end{définition}
\begin{définition}\pfra{Grain (pour chevaux ou vaches), picotin.}\end{définition}
\begin{exemple}\pnru{ʐwæ˧-lv̩˧}\hspace{5pt}\peng{cereals fed to horses; same meaning as /ʐwæ˧-ɭɯ\#˥/}\hspace{5pt}\pcmn{喂给马的粮食}\hspace{5pt}\pfra{grain pour pour cheval, picotin; même sens que: /ʐwæ˧-ɭɯ\#˥/}\end{exemple}
\end{entrée}

\begin{entrée}
{lv̩˧}{₃}{ⓔlv̩˧ⓗ3}\formedesurface{lv̩˧}\newline
\classe{名词}\ton{M}
3\begin{définition}\peng{Stone (monosyllable).}\end{définition}
\begin{définition}\pcmn{石}\end{définition}
\begin{définition}\pfra{Pierre (monosyllabe).}\end{définition}
\begin{exemple}\pnru{lv̩˧-ʁæ˧bæ˥}\hspace{5pt}\peng{Porcelain plate. Literally ‘stone plate': the Na, being accustomed to wooden plates, first gave the noun ‘stone plate' to clear-coloured porcelain dishes.}\hspace{5pt}\pcmn{瓷盘子。直译:‘石盘’。摩梭人传统用木头盘子,将瓷盘子称作‘石头盘子’。}\hspace{5pt}\pfra{Assiette en faïence. Littéralement ‘assiette de pierre': les Na, accoutumés aux assiettes en bois, avaient d'abord donné à la vaisselle de couleur claire le nom d'assiettes de pierre.}\end{exemple}
\end{entrée}

\begin{entrée}
{lv̩˧˥}{}{ⓔlv̩˧˥}\formedesurface{lv̩˧˥}\newline
\classe{名词}\ton{MH}\begin{définition}\peng{Maggot.}\end{définition}
\begin{définition}\pcmn{蛆}\end{définition}
\begin{définition}\pfra{Larve.}\end{définition}
\end{entrée}

\begin{entrée}
{lv̩˧˥}{₁}{ⓔlv̩˧˥ⓗ1}\formedesurface{lv̩˧˥}\newline
\classe{动词}\ton{MH}
1\begin{définition}\peng{To herd.}\end{définition}
\begin{définition}\pcmn{放牧}\end{définition}
\begin{définition}\pfra{Garder les animaux, mener paître les animaux.}\end{définition}
\begin{exemple}\pnru{go˩bo˥ lv̩˩}\hspace{5pt}\peng{to graze cattle, to herd cattle}\hspace{5pt}\pcmn{放牧牲畜}\hspace{5pt}\pfra{mener paître le bétail, garder le bétail}\end{exemple}
\begin{exemple}\pnru{ʐwæ˧ lv̩˩}\hspace{5pt}\peng{to graze horses, to herd horses}\hspace{5pt}\pcmn{放马}\hspace{5pt}\pfra{mener paître les chevaux}\end{exemple}
\begin{exemple}\pnru{ʝi˧ lv̩˩}\hspace{5pt}\peng{to graze cows, to herd cows}\hspace{5pt}\pcmn{放牛}\hspace{5pt}\pfra{mener paître les vaches}\end{exemple}
\begin{exemple}\pnru{bo˩ lv̩˩˥}\hspace{5pt}\peng{to herd pigs}\hspace{5pt}\pcmn{放猪}\hspace{5pt}\pfra{garder les cochons}\end{exemple}
\begin{exemple}\pnru{tsʰɯ˧ lv̩˥}\hspace{5pt}\peng{to graze goats, to herd goats}\hspace{5pt}\pcmn{放山羊}\hspace{5pt}\pfra{mener paître les chèvres}\end{exemple}
\begin{exemple}\pnru{ɖɯ˧-hɤ˧ mɤ˧-lv̩˩∼lv̩˩}\hspace{5pt}\peng{lazy, who does not take care of anything}\hspace{5pt}\pcmn{懒,什么也不管}\hspace{5pt}\pfra{paresseux, qui ne s'occupe de rien}\end{exemple}
\end{entrée}

\begin{entrée}
{lv̩˧˥}{₂}{ⓔlv̩˧˥ⓗ2}\formedesurface{lv̩˧˥}\newline
\classe{动词}\ton{MH}
2\begin{définition}\peng{To escape, to flee.}\end{définition}
\begin{définition}\pcmn{逃跑,逃掉}\end{définition}
\begin{définition}\pfra{S'enfuir.}\end{définition}
\end{entrée}

\begin{entrée}
{lv̩˩α}{₁}{ⓔlv̩˩αⓗ1}\formedesurface{lv̩˩˥}\newline
\classe{动词}\ton{Lα}
1\begin{définition}\peng{To bark (a dog barks).}\end{définition}
\begin{définition}\pcmn{狗吠}\end{définition}
\begin{définition}\pfra{Aboyer.}\end{définition}
\begin{exemple}\pnru{kʰv̩˩mi˩ lv̩˥ |}\hspace{5pt}\peng{the dog barks}\hspace{5pt}\pcmn{狗吠}\hspace{5pt}\pfra{le chien aboie}\end{exemple}
\begin{exemple}\pnru{kʰv̩˩ lv̩˥-dʑo˩ |}\hspace{5pt}\peng{the dog is barking}\hspace{5pt}\pcmn{狗在叫}\hspace{5pt}\pfra{le chien est en train d'aboyer}\end{exemple}
\begin{exemple}\pnru{ɖɯ˧-lv̩˧∼lv̩˥-ɻ̍˩}\hspace{5pt}\peng{|fg{delimitative} \_ |fg{red} |fg{inceptive}}\hspace{5pt}\pcmn{叫一叫}\hspace{5pt}\pfra{|fg{délimitatif} \_ |fg{red} |fg{inchoatif}}\end{exemple}
\end{entrée}

\begin{entrée}
{lv̩˩α}{₂}{ⓔlv̩˩αⓗ2}\formedesurface{lv̩˩˥}\newline
\classe{动词}\ton{Lα}
2\begin{définition}\peng{To roll, to coil (fabric).}\end{définition}
\begin{définition}\pcmn{把布卷起来}\end{définition}
\begin{définition}\pfra{Enrouler (un tissu).}\end{définition}
\begin{exemple}\pnru{le˧-qæ˥-lv̩˩}\hspace{5pt}\peng{to coil}\hspace{5pt}\pcmn{卷起来}\hspace{5pt}\pfra{enrouler}\end{exemple}
\begin{exemple}\pnru{le˧-lv̩˧∼lv̩˧}\hspace{5pt}\pfra{|fg{accomp} |fg{red}}\end{exemple}
\begin{exemple}\pnru{tso˧∼tso˧ lv̩˧∼lv̩˧}\hspace{5pt}\peng{to coil things}\hspace{5pt}\pcmn{卷东西}\hspace{5pt}\pfra{enrouler des choses}\end{exemple}
\begin{exemple}\pnru{ɖɯ˧-kʰwɤ˧ lv̩˥}\hspace{5pt}\peng{to coil something}\hspace{5pt}\pcmn{卷一块(东西)}\hspace{5pt}\pfra{enrouler quelque chose}\end{exemple}
\end{entrée}

\begin{entrée}
{lv̩˩α}{₃}{ⓔlv̩˩αⓗ3}\formedesurface{lv̩˩˥}\newline
\classe{动词}\ton{Lα}
3\begin{définition}\peng{To plough, to till.}\end{définition}
\begin{définition}\pcmn{耕种}\end{définition}
\begin{définition}\pfra{Labourer.}\end{définition}
\begin{exemple}\pnru{le˧-lv̩˩-ze˩}\hspace{5pt}\peng{|fg{accomp} \_ |fg{pfv}}\hspace{5pt}\pcmn{耕种了}\hspace{5pt}\pfra{|fg{accomp} \_ |fg{pfv}}\end{exemple}
\begin{exemple}\pnru{ʝi˧-lv̩˧˥}\hspace{5pt}\peng{to plough}\hspace{5pt}\pcmn{耕种}\hspace{5pt}\pfra{labourer}\end{exemple}
\begin{exemple}\pnru{dʑi˧mi˧ lv̩˧˥ / dʑi˧mi˧ lv̩˧-ze˥}\hspace{5pt}\peng{to plough with a water buffalo}\hspace{5pt}\pcmn{用水牛耕田}\hspace{5pt}\pfra{labourer avec un buffle}\end{exemple}
\begin{exemple}\pnru{ʝi˧ ɖɯ˧-lv̩˧∼lv̩˥-ɻ̍˩}\hspace{5pt}\peng{to plough a little}\hspace{5pt}\pcmn{耕一耕}\hspace{5pt}\pfra{labourer un peu}\end{exemple}
\end{entrée}

\begin{entrée}
{lv̩˩α}{₄}{ⓔlv̩˩αⓗ4}\formedesurface{lv̩˩˥}\newline
\classe{动词}\ton{Lα}
4\begin{définition}\peng{To suffice, to be enough.}\end{définition}
\begin{définition}\pcmn{足够}\end{définition}
\begin{définition}\pfra{Suffire.}\end{définition}
\begin{exemple}\pnru{ə˩-lv̩˩˥? / ə˩-lv̩˩-ze˥?}\hspace{5pt}\peng{Is it enough? Is it sufficient?}\hspace{5pt}\pcmn{够了吗?}\hspace{5pt}\pfra{est-ce que ça (te) suffit ?}\end{exemple}
\end{entrée}

\begin{entrée}
{lv̩˧bv̩˧}{}{ⓔlv̩˧bv̩˧}\formedesurface{lv̩˧bv̩˧}\newline
\classe{名词}\ton{M}
\paradigme{\pcmn{:} \p{}}
\begin{définition}\peng{Vegetable bed.}\end{définition}
\begin{définition}\pcmn{菜畦}\end{définition}
\begin{définition}\pfra{Lit à légumes (dans le potager).}\end{définition}
\begin{exemple}\pnru{v˩tsʰɤ˧-lv˧bv̩\#˥}\hspace{5pt}\peng{same meaning: vegetable bed (in a vegetable garden)}\hspace{5pt}\pcmn{同上:菜畦}\hspace{5pt}\pfra{même sens: lit à légumes (dans le potager)}\end{exemple}
\begin{exemple}\pnru{qʰwæ˧ɭɯ˧-qo˧ | v˩tsʰɤ˧-lv˧bv˧ | le˧-gv˩, v˩tsʰɤ˧˥ | ɖɯ˧-jɤ˩ tʰi˩-pʰo˩}\hspace{5pt}\peng{to make a vegetable bed in the vegetable garden, and to sow a row of vegetables}\hspace{5pt}\pcmn{菜园里建菜畦,种一排菜}\hspace{5pt}\pfra{bâtir un lit à légumes dans le potager, et semer une rangée de légumes}\end{exemple}
\end{entrée}

\begin{entrée}
{lv̩˧dʑɯ˥}{}{ⓔlv̩˧dʑɯ˥}\formedesurface{lv̩˧dʑɯ˥}\newline
\classe{名词}\ton{H\#}
\paradigme{\pcmn{:} \p{}}
\begin{définition}\peng{Stone chips, little slabs of stone.}\end{définition}
\begin{définition}\pcmn{零碎的石头块}\end{définition}
\begin{définition}\pfra{Éclats de pierre, débris de pierre, petits bouts de pierre (ne veut pas dire ‘sable’).}\end{définition}
\end{entrée}

\begin{entrée}
{lv̩˩ʝi˧}{}{ⓔlv̩˩ʝi˧}\formedesurface{lv̩˩ʝi˥}\newline
\classe{动词}\ton{LM}\begin{définition}\peng{To record sound.}\end{définition}
\begin{définition}\pcmn{录音(汉语借词)}\end{définition}
\begin{définition}\pfra{Enregistrer.}\end{définition}
\begin{exemple}\pnru{hɑ˧ le˧-dzɯ˧-se˥, | lv̩˩ ʝi˧-bi˧ !}\hspace{5pt}\peng{After the meal, we'll do a recording!}\hspace{5pt}\pcmn{吃完饭,就录音吧! / 吃完饭就可以录音了!}\hspace{5pt}\pfra{Quand (on) aura fini de manger, (on) fera un enregistrement !}\end{exemple}
\end{entrée}

\begin{entrée}
{lv̩˩∼lv̩˧˥}{}{ⓔlv̩˩∼lv̩˧˥}\formedesurface{lv̩˩lv̩˧˥}\newline
\classe{动词}\ton{L+MH}\begin{définition}\peng{To move.}\end{définition}
\begin{définition}\pcmn{动(虫、桌子、小孩子动)}\end{définition}
\begin{définition}\pfra{Bouger, faire des mouvements.}\end{définition}
\begin{exemple}\pnru{lv̩˩∼lv̩˧-ze˥}\hspace{5pt}\peng{|fg{pfv}}\hspace{5pt}\pcmn{动了}\hspace{5pt}\pfra{|fg{pfv}}\end{exemple}
\begin{exemple}\pnru{tʰi˧-lv̩˩∼lv̩˩(-ze˩)}\hspace{5pt}\peng{|fg{dur} |fg{red}}\hspace{5pt}\pcmn{|fg{dur} |fg{red}}\hspace{5pt}\pfra{|fg{dur} |fg{red}}\end{exemple}
\begin{exemple}\pnru{tʰi˧-lv̩˩∼lv̩˩ | se˧}\hspace{5pt}\peng{to walk askance, to walk askew: e.g. a lame person walks with difficulty}\hspace{5pt}\pcmn{歪着走、扭着走、例如:残疾人走路有困难}\hspace{5pt}\pfra{marcher en se trémoussant, marcher de travers, marcher en se contorsionnant}\end{exemple}
\begin{exemple}\pnru{kʰɯ˧tsʰɤ˧ lv̩˥∼lv̩˩}\hspace{5pt}\peng{to move one's leg around}\hspace{5pt}\pcmn{活动一下(自己的)腿}\hspace{5pt}\pfra{bouger la jambe, remuer la jambe}\end{exemple}
\end{entrée}

\begin{entrée}
{lv̩˧mi˧}{}{ⓔlv̩˧mi˧}\formedesurface{lv̩˧mi˧}\newline
\classe{名词}\ton{M}
\paradigme{\pcmn{:} \p{}}
\begin{définition}\peng{Stone.}\end{définition}
\begin{définition}\pcmn{石头}\end{définition}
\begin{définition}\pfra{Pierre.}\end{définition}
\begin{exemple}\pnru{kʰv̩˧pʰæ˧tɕi˩, | lv̩˧mi˧ dzɯ˧-bi˧-ʁo˧-ho˩!}\hspace{5pt}\peng{‘When one is young, one could eat stones!' (Meaning: when one is young, one can eat anything, one has an excellent digestion; as one gets old, one is less tolerant of food that is not easy to digest.)}\hspace{5pt}\pcmn{‘年轻人,石头都能吃!’(意思:年轻人消化好,吃什么都行,而人变老就不那么容易消化了,要注意吃什么。)}\hspace{5pt}\pfra{‘Quand on est jeune, on mangerait des pierres!' (Signification: quand on est jeune, on mange de tout, on a une digestion solide; tandis que quand on est vieux, on a facilement mal au ventre, dès qu'on mange quelque chose d'un peu indigeste, une nourriture «trop dure».)}\end{exemple}
\end{entrée}

\begin{entrée}
{lv̩˧mi˧-bo\#˥}{}{ⓔlv̩˧mi˧-bo\#˥}\formedesurface{lv̩˧mi˧bo˧}\newline
\classe{名词}\ton{\#H}
\paradigme{\pcmn{:} \p{}}
\begin{définition}\peng{Stone wall.}\end{définition}
\begin{définition}\pcmn{石墙}\end{définition}
\begin{définition}\pfra{Mur en pierre.}\end{définition}
\end{entrée}

\begin{entrée}
{lv̩˧mi˧-dʑɯ˧dʑɯ˩}{}{ⓔlv̩˧mi˧-dʑɯ˧dʑɯ˩}\formedesurface{lv̩˧mi˧dʑɯ˧dʑɯ˩}\newline
\classe{名词}\ton{-L\#}
\paradigme{\pcmn{:} \p{}}
\begin{définition}\peng{Little slabs of stone, stone chips.}\end{définition}
\begin{définition}\pcmn{零碎的石块}\end{définition}
\begin{définition}\pfra{Éclats de pierre, débris de pierre, petits bouts de pierre (ne veut pas dire ‘sable’).}\end{définition}
\end{entrée}

\begin{entrée}
{lv̩˧pʰv̩˩}{₁}{ⓔlv̩˧pʰv̩˩ⓗ1}\formedesurface{lv̩˧pʰv̩˩}\newline
\classe{动词}\ton{L\#}
1\begin{définition}\peng{To open up new land for cultivation, to cultivate virgin land.}\end{définition}
\begin{définition}\pcmn{开荒}\end{définition}
\begin{définition}\pfra{Défricher.}\end{définition}
\begin{exemple}\pnru{lv̩˧pʰv̩˩-hɯ˩}\hspace{5pt}\peng{[(S)he] has gone to open up new land.}\hspace{5pt}\pcmn{他开荒去了。}\hspace{5pt}\pfra{(il/elle) est parti(e) défricher}\end{exemple}
\begin{exemple}\pnru{lv̩˧pʰv̩˩-bi˩-tso˩-ɲi˩}\hspace{5pt}\peng{It's necessary to go and open up new land for cultivation. / We're going to have to open up new land for cultivation.}\hspace{5pt}\pcmn{应该去开荒了。}\hspace{5pt}\pfra{Il va falloir aller défricher. / Il va falloir qu'on défriche de nouvelles terres.}\end{exemple}
\end{entrée}

\begin{entrée}
{lv̩˧pʰv̩˩}{₂}{ⓔlv̩˧pʰv̩˩ⓗ2}\formedesurface{lv̩˧pʰv̩˩}\newline
\classe{名词}\ton{L\#}
2
\paradigme{\pcmn{:} \p{}}
\begin{définition}\peng{Paddy field.RD Comment:Cf. ɕi˧lv̩˧-mv̩˧di˧˥}\end{définition}
\begin{définition}\pcmn{水田}\end{définition}
\begin{définition}\pfra{Champs inondés.}\end{définition}
\end{entrée}

\begin{entrée}
{lv̩˧qæ\#˥}{}{ⓔlv̩˧qæ\#˥}\formedesurface{lv̩˧qæ˧}\newline
\classe{名词}\ton{\#H}\begin{définition}\peng{Limit, boundary between fields belonging to different families. It is typically materialized by a small dike.}\end{définition}
\begin{définition}\pcmn{地界:不同家庭田地之间的界限}\end{définition}
\begin{définition}\pfra{Limite de propriété: limite entre les champs appartenant à des familles différentes. Elle est souvent matérialisée par une diguette.}\end{définition}
\end{entrée}

\begin{entrée}
{lv̩˧sɯ˥}{}{ⓔlv̩˧sɯ˥}\formedesurface{lv̩˧sɯ˥}\newline
\classe{名词}\ton{H\#}
\paradigme{\pcmn{:} \p{}}
\begin{définition}\peng{Lisu (ethnic group).}\end{définition}
\begin{définition}\pcmn{傈僳族}\end{définition}
\begin{définition}\pfra{Lisu (groupe ethnique).}\end{définition}
\end{entrée}

\begin{entrée}
{lv̩˩tɕʰɯ˧}{}{ⓔlv̩˩tɕʰɯ˧}\formedesurface{lv̩˩tɕʰɯ˥}\newline
\classe{名词}\ton{LM}\begin{définition}\peng{The village of Fengke (close to the Yangtze river): this is the former name of the area in Chinese.}\end{définition}
\begin{définition}\pcmn{六区,今奉科乡(汉语借词)}\end{définition}
\begin{définition}\pfra{Fengke: nom chinois ancien du village de Fengke, au bord du Yangtze.}\end{définition}
\end{entrée}

\begin{entrée}
{lv̩˩tɕʰɯ˧-hĩ\#˥}{}{ⓔlv̩˩tɕʰɯ˧-hĩ\#˥}\formedesurface{lv̩˩tɕʰɯ˧hĩ˧}\newline
\classe{名词}\ton{LM+\#H}\begin{définition}\peng{The inhabitants of the village of Fengke (Fv-kho).}\end{définition}
\begin{définition}\pcmn{奉科的人}\end{définition}
\begin{définition}\pfra{Gens de Fengke (Fv-kho).}\end{définition}
\end{entrée}

\begin{entrée}
{lv̩˧tsɯ˥}{}{ⓔlv̩˧tsɯ˥}\formedesurface{lv̩˧tsɯ˥}\newline
\classe{名词}\ton{H\#}
\paradigme{\pcmn{:} \p{}}
\begin{définition}\peng{Oven.}\end{définition}
\begin{définition}\pcmn{炉子(汉语借词)}\end{définition}
\begin{définition}\pfra{Four.}\end{définition}
\end{entrée}

\begin{entrée}
{lwæ˩pʰv̩˩}{}{ⓔlwæ˩pʰv̩˩}\formedesurface{lwæ˩pʰv̩˩˥}\newline
\classe{名词}\ton{L}\begin{définition}\peng{Ashes.}\end{définition}
\begin{définition}\pcmn{灰}\end{définition}
\begin{définition}\pfra{Cendres.}\end{définition}
\begin{exemple}\pnru{lwæ˩pʰv̩˩-ni˥gv̩˩}\hspace{5pt}\peng{grey; literally: “like ashes"}\hspace{5pt}\pcmn{灰色的(直译:“像白灰”)}\hspace{5pt}\pfra{de couleur grise; littéralement «comme de la cendre»}\end{exemple}
\end{entrée}

\begin{entrée}
{lwɤ˩˥}{}{ⓔlwɤ˩˥}\formedesurface{lwɤ˩˥}\newline
\classe{名词}\ton{LH}
\paradigme{\pcmn{:} \p{}}
\begin{définition}\peng{Ashes (of plants, charcoal…), cinders.}\end{définition}
\begin{définition}\pcmn{灰,灰烬(包括草木灰等等)}\end{définition}
\begin{définition}\pfra{Cendre (cendre végétale ou cendre de charbon); scories.}\end{définition}
\begin{exemple}\pnru{lwɤ˩-pʰæ˧di˩}\hspace{5pt}\peng{like ashes; gray-coloured}\hspace{5pt}\pcmn{像灰烬,灰色}\hspace{5pt}\pfra{comme de la cendre}\end{exemple}
\end{entrée}

\newpage\caractère{ɭ}

\begin{entrée}
{ɭɯ˧˥α}{}{ⓔɭɯ˧˥α}\formedesurface{ɖɯ˧ ɭɯ˧˥}\newline
\classe{量词}\ton{MHα}\begin{définition}\pcmn{量词:衣服(一件)}\end{définition}
\begin{définition}\pfra{Classificateur des vêtements.}\end{définition}
\begin{exemple}\pnru{ʈʰæ˧qʰwɤ˧ ɖɯ˧-ɭɯ˧˥}\hspace{5pt}\peng{a skirt}\hspace{5pt}\pcmn{一件裙子}\hspace{5pt}\pfra{une jupe}\end{exemple}
\begin{exemple}\pnru{bɑ˩lɑ˩˥ | ɖɯ˧-ɭɯ˧˥ |}\hspace{5pt}\peng{a piece of clothing; a shirt}\hspace{5pt}\pcmn{一件衣服}\hspace{5pt}\pfra{un vêtement}\end{exemple}
\begin{exemple}\pnru{*dʑi˧hṽ̩˥\$+ɖɯ˧-ɭɯ˧˥}\hspace{5pt}\peng{This classifier cannot combine with /dʑi˧hṽ̩˥\$/, which takes /ɖɯ˧-dzi˩/ as its classifier.}\hspace{5pt}\pcmn{(这个量词不能与/dʑi˧hṽ̩˥\$/结合。)}\hspace{5pt}\pfra{Ce classificateur ne se combine pas avec /dʑi˧hṽ̩˥\$/, qui prend pour classificateur: /ɖɯ˧-dzi˩/.}\end{exemple}
\end{entrée}

\begin{entrée}
{ɭɯ˧β}{}{ⓔɭɯ˧β}\formedesurface{ɖɯ˧ ɭɯ˧}\newline
\classe{量词}\ton{Mβ}\begin{définition}\peng{Originally a classifier for round objects: grains, bowls… Now a generic classifier, used e.g. for pieces of clothing.}\end{définition}
\begin{définition}\pcmn{最常用的量词,相当于汉语中‘个’的用法。本意是圆形颗粒。一粒(米……),一个(碗……),件(衣服……)}\end{définition}
\begin{définition}\pfra{Classificateur générique; à l'origine, classificateur pour les objets ronds, à l'emploi maintenant élargi.}\end{définition}
\begin{exemple}\pnru{ɕi˧ ɖɯ˧-ɭɯ˧ |}\hspace{5pt}\peng{a grain of rice}\hspace{5pt}\pcmn{一粒米}\hspace{5pt}\pfra{un grain de riz}\end{exemple}
\begin{exemple}\pnru{hõ˧-ɭɯ˥}\hspace{5pt}\peng{eight grains}\hspace{5pt}\pcmn{八粒}\hspace{5pt}\pfra{huit grains}\end{exemple}
\begin{exemple}\pnru{ɖɯ˧-ɭɯ˧ hwæ˧-mɤ˧-ɖo˧! | le˧-qʰwæ˧-kv̩˥!}\hspace{5pt}\peng{Don't buy just one / Don't buy a single one: it would break! (Explanation: objects must be bought by pairs: 2, 4, 6, 8, 10…, not by sets of odd numbers (1, 3, 5, 7, 9…), otherwise it bears ill luck and the objects get broken or lost)}\hspace{5pt}\pcmn{不要(只)买一个!会碎的!(东西要一对一对买:2、4、6、8、10……,单数不吉利,东西会碎的。)}\hspace{5pt}\pfra{N'en achète pas un (unique)! Ca va se casser! (Explication: il faut acheter les objets par paires: 2, 4, 6, 8, 10…, pas en nombre impair, sinon cela porte malheur et les objets cassent, se perdent…)}\end{exemple}
\begin{exemple}\pnru{ʈʂʰɯ˧ | zo˧hṽ̩˥ | dʑɤ˩-ɭɯ˥ dʑo˩!}\hspace{5pt}\peng{She has a really pretty child! (Context: the main consultant had a polite conversation with a neighbour who had a lovely grandson; later on, she told me: “She has a really pretty child!")}\hspace{5pt}\pcmn{她有个很漂亮的孩子!}\hspace{5pt}\pfra{Elle a un bien bel enfant! (Contexte: lors d'une sortie, la consultante principale dit quelques politesses à une voisine qui se promenait avec un petit-fils attendrissant; elle me dit ensuite: «Elle a un bien bel enfant!» Littéralement: «elle, (d')enfant(s), (elle) en a un (de) bien!»)}\end{exemple}
\end{entrée}

\newpage\caractère{ɬ}

\begin{entrée}
{ɬɑ˧˥}{}{ⓔɬɑ˧˥}\formedesurface{ɬɑ˧˥}\newline
\classe{形容词}\ton{MH}\begin{définition}\peng{Numerous, abundant, plentiful.}\end{définition}
\begin{définition}\pcmn{多、丰富、充分}\end{définition}
\begin{définition}\pfra{Abondant, nombreux.}\end{définition}
\begin{exemple}\pnru{dʑɤ˩-hĩ˩˥, | le˧-ɳɯ˥! | mɤ˧-dʑɤ˩-hĩ˩, | le˧-ɬɑ˧˥!}\hspace{5pt}\peng{Good ones are few! Not-so-good ones are numerous! (Context: discussing universities, among which high-school graduates choose.)}\hspace{5pt}\pcmn{好的,不多!不好的,就很多了!(情景:谈高中学生想入大学)}\hspace{5pt}\pfra{Les bons, il n'y en a guère; les médiocres, il y en a en quantité! (Contexte: au sujet des établissements universitaires entre lesquels les titulaires du baccalauréat chinois ont à choisir)}\end{exemple}
\end{entrée}

\begin{entrée}
{ɬɑ˧mv̩˥\$}{}{ⓔɬɑ˧mv̩˥\$}\formedesurface{ɬɑ˧mv̩˥}\newline
\classe{名词}\ton{H\$}\begin{définition}\peng{Feminine given name.}\end{définition}
\begin{définition}\pcmn{女性名字}\end{définition}
\begin{définition}\pfra{Prénom féminin.}\end{définition}
\end{entrée}

\begin{entrée}
{ɬɑ˧pɤ˩}{}{ⓔɬɑ˧pɤ˩}\formedesurface{ɬɑ˧pɤ˩}\newline
\classe{助词}\ton{L\#}\begin{définition}\peng{A lot, hard.}\end{définition}
\begin{définition}\pcmn{多、使劲}\end{définition}
\begin{définition}\pfra{Beaucoup.}\end{définition}
\begin{exemple}\pnru{ɬɑ˧pɤ˩ ʝi˩}\hspace{5pt}\peng{to work hard, to work in a concentrated manner}\hspace{5pt}\pcmn{使劲工作、使劲干}\hspace{5pt}\pfra{en faire beaucoup}\end{exemple}
\begin{exemple}\pnru{ɬɑ˧pɤ˩ | ɖɯ˧-kʰwɤ˧ ʝi˧}\hspace{5pt}\peng{to work hard for a while, to get some solid work done}\hspace{5pt}\pcmn{使劲工作一下}\hspace{5pt}\pfra{en mettre un coup, beaucoup travailler, bien avancer dans son travail}\end{exemple}
\begin{exemple}\pnru{ɬɑ˧pɤ˩ | ɖɯ˧-kʰwɤ˧ so˥}\hspace{5pt}\peng{to study hard, to make headway in one's studies}\hspace{5pt}\pcmn{努力学习一下}\hspace{5pt}\pfra{beaucoup étudier, faire un bon progrès dans l'étude}\end{exemple}
\end{entrée}

\begin{entrée}
{ɬɑ˧sɑ˧}{}{ⓔɬɑ˧sɑ˧}\formedesurface{ɬɑ˧sɑ˧}\newline
\classe{名词}\ton{M}\begin{définition}\peng{Lhasa.}\end{définition}
\begin{définition}\pcmn{拉萨}\end{définition}
\begin{définition}\pfra{Lhasa (capitale du Tibet).}\end{définition}
\end{entrée}

\begin{entrée}
{ɬɑ˧tɑ˥}{}{ⓔɬɑ˧tɑ˥}\formedesurface{ɬɑ˧tɑ˥}\newline
\classe{名词}\ton{H\#}
\paradigme{\pcmn{:} \p{}}
\begin{définition}\peng{Jerkin, leather vest.}\end{définition}
\begin{définition}\pcmn{皮革背心}\end{définition}
\begin{définition}\pfra{Gilet de cuir (mot sorti d'usage, n'apparaît que dans un proverbe).}\end{définition}
\end{entrée}

\begin{entrée}
{ɬɑ˧tsʰo\#˥}{}{ⓔɬɑ˧tsʰo\#˥}\formedesurface{ɬɑ˧tsʰo˧}\newline
\classe{名词}\ton{\#H}\begin{définition}\peng{Feminine given name.}\end{définition}
\begin{définition}\pcmn{女性名字}\end{définition}
\begin{définition}\pfra{Prénom féminin.}\end{définition}
\end{entrée}

\begin{entrée}
{ɬi˥}{}{ⓔɬi˥}\formedesurface{ɬi˧}\newline
\classe{动词}\ton{H}\begin{définition}\peng{To rest, to relax.}\end{définition}
\begin{définition}\pcmn{休息,松懈}\end{définition}
\begin{définition}\pfra{Se reposer, se détendre.}\end{définition}
\begin{exemple}\pnru{le˧-ɬi˥}\hspace{5pt}\peng{|fg{accomp} \_}\hspace{5pt}\pcmn{|fg{accomp} \_}\hspace{5pt}\pfra{|fg{accomp} \_}\end{exemple}
\end{entrée}

\begin{entrée}
{ɬi˧˥}{}{ⓔɬi˧˥}\formedesurface{ɬi˧˥}\newline
\classe{动词}\ton{MH}\begin{définition}\peng{To dry in the sun.}\end{définition}
\begin{définition}\pcmn{晒干}\end{définition}
\begin{définition}\pfra{Faire sécher au soleil.}\end{définition}
\begin{exemple}\pnru{le˧-pv̩˧ tʰi˧-ɬi˧˥}\hspace{5pt}\peng{to put in the sun to dry}\hspace{5pt}\pcmn{晒干}\hspace{5pt}\pfra{exposer au soleil afin de faire sécher}\end{exemple}
\end{entrée}

\begin{entrée}
{ɬi˧β}{}{ⓔɬi˧β}\formedesurface{ɖɯ˧ ɬi˧}\newline
\classe{量词}\ton{Mβ}\begin{définition}\peng{Month.}\end{définition}
\begin{définition}\pcmn{量词:月}\end{définition}
\begin{définition}\pfra{Mois.}\end{définition}
\end{entrée}

\begin{entrée}
{ɬi˩}{₁}{ⓔɬi˩ⓗ1}\formedesurface{ɬi˩˥}\newline
\classe{动词}\ton{Lα}
1\begin{définition}\peng{To measure (e.g. a piece of fabric) to find its length, in armspans.}\end{définition}
\begin{définition}\pcmn{量(一块布料……)有多长:有多少庹}\end{définition}
\begin{définition}\pfra{Toiser: mesurer (une pièce de tissu…) à l'aune de la toise: distance entre les deux bras écartés.}\end{définition}
\begin{exemple}\pnru{ɬi˩-se˥ (-ze˩)}\hspace{5pt}\peng{\_ |fg{achev} (|fg{pfv})}\hspace{5pt}\pcmn{量完(了)}\hspace{5pt}\pfra{\_ |fg{achev} (|fg{pfv})}\end{exemple}
\end{entrée}

\begin{entrée}
{ɬi˩}{₂}{ⓔɬi˩ⓗ2}\formedesurface{ɬi˧}\newline
\classe{名词}\ton{L}
2
\paradigme{\pcmn{:} \p{}}
\begin{définition}\peng{Roebuck, hornless river deer.}\end{définition}
\begin{définition}\pcmn{獐子}\end{définition}
\begin{définition}\pfra{Chevrotain.}\end{définition}
\end{entrée}

\begin{entrée}
{ɬi˩β}{}{ⓔɬi˩β}\formedesurface{ɖɯ˧ ɬi˩}\newline
\classe{量词}\ton{Lβ}\begin{définition}\peng{A span, an armspread.}\end{définition}
\begin{définition}\pcmn{量词:庹}\end{définition}
\begin{définition}\pfra{Toise: envergure des bras =longueur des deux bras écartés. Cette unité correspond à environ 5 pieds chinois (1 mètre 78).}\end{définition}
\begin{exemple}\pnru{tsʰe˧-ɬi˧}\hspace{5pt}\peng{10 spans, 10 armspreads}\hspace{5pt}\pcmn{十庹}\hspace{5pt}\pfra{10 toises}\end{exemple}
\end{entrée}

\begin{entrée}
{ɬi˩bi˩}{}{ⓔɬi˩bi˩}\formedesurface{ɬi˩bi˩˥}\newline
\classe{名词}\ton{L}
\paradigme{\pcmn{:} \p{}}
\begin{définition}\peng{Turnip; radish.}\end{définition}
\begin{définition}\pcmn{萝卜}\end{définition}
\begin{définition}\pfra{Navet, gros radis.}\end{définition}
\end{entrée}

\begin{entrée}
{ɬi˧bo\#˥}{}{ⓔɬi˧bo\#˥}\formedesurface{ɬi˧bo˧}\newline
\classe{名词}\ton{\#H}
\paradigme{\pcmn{:} \p{}}
\begin{définition}\peng{Deaf person.}\end{définition}
\begin{définition}\pcmn{聋子}\end{définition}
\begin{définition}\pfra{Sourd, personne sourde.}\end{définition}
\begin{exemple}\pnru{ɬi˧bo˧-hĩ˧}\hspace{5pt}\peng{deaf person}\hspace{5pt}\pcmn{耳朵聋的人}\hspace{5pt}\pfra{personne sourde}\end{exemple}
\end{entrée}

\begin{entrée}
{ɬi˧bv̩˧}{}{ⓔɬi˧bv̩˧}\formedesurface{ɬi˧bv̩˧}\newline
\classe{名词}\ton{M}
\paradigme{\pcmn{:} \p{}}
\begin{définition}\peng{Bai (ethnic group).}\end{définition}
\begin{définition}\pcmn{白族}\end{définition}
\begin{définition}\pfra{Bai (groupe ethnique).}\end{définition}
\end{entrée}

\begin{entrée}
{ɬi˧bv̩˩ | dʑɤ˩tsʰi˧si\#˥}{}{ⓔɬi˧bv̩˩ | dʑɤ˩tsʰi˧si\#˥}\formedesurface{ɬi˧bv̩˩dʑɤ˩tsʰi˧si˧}\newline
\classe{名词}\ton{L\# | LM+\#H}\begin{définition}\peng{Chinese toon, fragrant cedar, |\stylefi{Ailanthus chinensis}.}\end{définition}
\begin{définition}\pcmn{香椿、香椿树}\end{définition}
\begin{définition}\pfra{Ailante, |\stylefi{Ailanthus chinensis}; arbre très odorant.}\end{définition}
\end{entrée}

\begin{entrée}
{ɬi˧bv̩˧-mi\#˥}{}{ⓔɬi˧bv̩˧-mi\#˥}\formedesurface{ɬi˧bv̩˧mi˧}\newline
\classe{名词}\ton{\#H}
\paradigme{\pcmn{:} \p{}}
\begin{définition}\peng{Bai woman.}\end{définition}
\begin{définition}\pcmn{白族女人}\end{définition}
\begin{définition}\pfra{Femme bai.}\end{définition}
\end{entrée}

\begin{entrée}
{ɬi˧bv̩˧-zo\#˥}{}{ⓔɬi˧bv̩˧-zo\#˥}\formedesurface{ɬi˧bv̩˧zo˧}\newline
\classe{名词}\ton{\#H}
\paradigme{\pcmn{:} \p{}}
\begin{définition}\peng{Bai man.}\end{définition}
\begin{définition}\pfra{Homme bai.}\end{définition}
\end{entrée}

\begin{entrée}
{ɬi˧di˩}{}{ⓔɬi˧di˩}\formedesurface{ɬi˧di˩}\newline
\classe{名词}\ton{L\#}\begin{définition}\peng{Yongning (place name).}\end{définition}
\begin{définition}\pcmn{永宁(地名)}\end{définition}
\begin{définition}\pfra{Yongning (nom de lieu).}\end{définition}
\end{entrée}

\begin{entrée}
{ɬi˧di˩-hĩ˩}{}{ⓔɬi˧di˩-hĩ˩}\formedesurface{ɬi˧di˩hĩ˩}\newline
\classe{名词}\ton{L\#-}
\paradigme{\pcmn{:} \p{}}
\begin{définition}\peng{People of Yongning. Unless otherwise specified, this is mainly understood as referring to the Na (Mosuo).}\end{définition}
\begin{définition}\pcmn{永宁人(纳人)}\end{définition}
\begin{définition}\pfra{Les gens de Yongning.}\end{définition}
\end{entrée}

\begin{entrée}
{ɬi˧dʑɯ˩}{}{ⓔɬi˧dʑɯ˩}\formedesurface{ɬi˧dʑɯ˩}\newline
\classe{名词}\ton{L\#}
\paradigme{\pcmn{:} \p{}}
\begin{définition}\peng{The river that flows through the plain of Yongning.}\end{définition}
\begin{définition}\pcmn{永宁坝子的河流}\end{définition}
\begin{définition}\pfra{La rivière qui traverse la plaine de Yongning.}\end{définition}
\end{entrée}

\begin{entrée}
{ɬi˧gv̩\#˥}{}{ⓔɬi˧gv̩\#˥}\formedesurface{ɬi˧gv̩˧}\newline
\classe{名词}\ton{\#H}\begin{définition}\peng{Middle part; (in) the centre.}\end{définition}
\begin{définition}\pcmn{中部,中间}\end{définition}
\begin{définition}\pfra{Partie intermédiaire, milieu; au milieu.}\end{définition}
\begin{exemple}\pnru{ɬi˧gv̩˧ dzi˥}\hspace{5pt}\peng{to sit in the centre}\hspace{5pt}\pcmn{坐在中间}\hspace{5pt}\pfra{être assis au milieu}\end{exemple}
\end{entrée}

\begin{entrée}
{ɬi˧hĩ\#˥}{₁}{ⓔɬi˧hĩ\#˥ⓗ1}\formedesurface{ɬi˧hĩ˧}\newline
\classe{名词}\ton{\#H}
1\begin{définition}\peng{Man in middle position among siblings: neither eldest brother nor youngest brother; literal translation: “person in the middle".}\end{définition}
\begin{définition}\pcmn{兄弟里面夹中的男孩(上有哥哥下有弟弟的孩子)}\end{définition}
\begin{définition}\pfra{Homme en position intermédiaire dans la fratrie: ni aîné ni cadet; traduction littérale: «personne du milieu».}\end{définition}
\end{entrée}

\begin{entrée}
{ɬi˧hĩ\#˥}{₂}{ⓔɬi˧hĩ\#˥ⓗ2}\formedesurface{ɬi˧hĩ˧}\newline
\classe{名词}\ton{\#H}
2\begin{définition}\peng{Inhabitant of Yongning; as used by the main consultant, the term includes Pumi (Prinmi) people along with Na people.}\end{définition}
\begin{définition}\pcmn{永宁的人}\end{définition}
\begin{définition}\pfra{Habitant de Yongning. Peut désigner les Prinmi qui habitent dans la plaine, aussi bien que les Na.}\end{définition}
\end{entrée}

\begin{entrée}
{ɬi˧ki˥}{}{ⓔɬi˧ki˥}\formedesurface{ɬi˧ki˥}\newline
\classe{名词}\ton{H\#}\begin{définition}\peng{Ritual for boys coming of age, i.e. reaching the age of 13 years: “wearing trousers"; at that age adolescents begin to wear trousers instead of children's robes.}\end{définition}
\begin{définition}\pcmn{男性成年礼:直译“穿裤”}\end{définition}
\begin{définition}\pfra{Cérémonie pour les garçons atteignant 13 ans: littéralement ‘porter/enfiler/mettre le pantalon’. Après cette cérémonie, l'adolescent porte un pantalon, au lieu du vêtement unisexe des enfants. (Rituel parallèle avec fv{/ʈʰæ˩ ki˩˥/}, ‘porter/enfiler/mettre la jupe’, pour les jeunes filles.).}\end{définition}
\end{entrée}

\begin{entrée}
{ɬi˧ki\#˥}{}{ⓔɬi˧ki\#˥}\formedesurface{ɬi˧ki˧}\newline
\classe{名词}\ton{\#H}\begin{définition}\peng{The name of a Na village, outside the plain of Yongning, close to the Lake.}\end{définition}
\begin{définition}\pcmn{泸沽湖附近的一个村落}\end{définition}
\begin{définition}\pfra{Village na, hors de la plaine, proche du Lac.}\end{définition}
\begin{exemple}\pnru{ɬi˧ki˧, | ɲi˧se˩, | tɑ˧dzi˩, | mv̩˧qʰwæ˩, | lɑ˧tʰɑ˧-di˧˥}\hspace{5pt}\peng{Villages that one passes when moving away from the Yongning plain, towards Lake Lugu. These villages do not count as part of Yongning proper. The last, /lɑ˧tʰɑ˧-di˧˥/, is not a village name like the preceding four: it refers to the entire Na area beyond the fourth village.}\hspace{5pt}\pcmn{永宁到泸沽湖所经过的村落,依次是:里格、尼赛、大祖、木垮,然后到拉塔地(拉塔地指的是泸沽湖周边的摩梭地区,包括左所、洛水村等)}\hspace{5pt}\pfra{Villages dans l'ordre, après la plaine de Yongning, ne comptant pas comme faisant partie de Yongning. Le dernier, /lɑ˧tʰɑ˧-di˧˥/, désigne toute la région na au-delà du quatrième village.}\end{exemple}
\end{entrée}

\begin{entrée}
{ɬi˧mi˧}{₁}{ⓔɬi˧mi˧ⓗ1}\formedesurface{ɬi˧mi˧}\newline
\classe{名词}
1
\sens{1}\paradigme{\pcmn{:} \p{}}
\begin{définition}\peng{Moon (disyllable).}\end{définition}
\begin{définition}\pcmn{月亮(双音节)}\end{définition}
\begin{définition}\pfra{Lune (disyllabe).}\end{définition}\sens{2}
\begin{définition}\peng{Month (disyllable).}\end{définition}
\begin{définition}\pcmn{月(双音节)}\end{définition}
\begin{définition}\pfra{Mois.}\end{définition}
\begin{exemple}\pnru{ɬi˧mi˧ ɖɯ˧-gi˥}\hspace{5pt}\peng{half a month}\hspace{5pt}\pcmn{半个月}\hspace{5pt}\pfra{une quinzaine, la moitié d'un mois}\end{exemple}
\begin{exemple}\pnru{ɬi˧mi˧ le˧-gv̩˩}\hspace{5pt}\peng{the latter half of the month; literally ‘the declining half of the month'}\hspace{5pt}\pcmn{下半月份}\hspace{5pt}\pfra{le mois décroît; expression qui peut désigner la seconde période du mois}\end{exemple}
\end{entrée}

\begin{entrée}
{ɬi˧mi˧}{₂}{ⓔɬi˧mi˧ⓗ2}\formedesurface{ɬi˧mi˧}\newline
\classe{名词}\ton{M}
2
\paradigme{\pcmn{:} \p{}}
\begin{définition}\peng{Female roebuck.}\end{définition}
\begin{définition}\pcmn{母獐子}\end{définition}
\begin{définition}\pfra{Chevrotain femelle.}\end{définition}
\end{entrée}

\begin{entrée}
{ɬi˧mi˧dɑ˧dzɯ\#˥}{}{ⓔɬi˧mi˧dɑ˧dzɯ\#˥}\formedesurface{ɬi˧mi˧dɑ˧dzɯ˧}\newline
\classe{名词}\ton{\#H}
\paradigme{\pcmn{:} \p{}}
\begin{définition}\peng{Lunar eclipse.}\end{définition}
\begin{définition}\pcmn{月蚀}\end{définition}
\begin{définition}\pfra{Éclipse de lune.}\end{définition}
\begin{exemple}\pnru{ɬi˧mi˧dɑ˧dzɯ˧ tʰv̩˧}\hspace{5pt}\peng{there is a lunar eclipse}\hspace{5pt}\pcmn{有月蚀}\hspace{5pt}\pfra{il y a une éclipse de lune}\end{exemple}
\begin{exemple}\pnru{ʈʂʰɯ˧ | ɬi˧mi˧dɑ˧dzɯ˧ ɲi˥!}\hspace{5pt}\peng{This is a lunar eclipse! (Answer to the question ‘What is happening? / What is that supposed to mean?')}\hspace{5pt}\pcmn{这是月蚀!(一个人问:‘这是怎么回事?’,另一个回答:‘这是月蚀!’)}\hspace{5pt}\pfra{c'est une éclipse de lune! (réponse à la question ‘Qu'est-ce qui se passe?')}\end{exemple}
\end{entrée}

\begin{entrée}
{ɬi˧ɳæ˩}{}{ⓔɬi˧ɳæ˩}\formedesurface{ɬi˧ɳæ˩}\newline
\classe{名词}\ton{L\#}
\paradigme{\pcmn{:} \p{}}
\begin{définition}\peng{Menses; period.}\end{définition}
\begin{définition}\pcmn{月经}\end{définition}
\begin{définition}\pfra{Menstrues.}\end{définition}
\begin{exemple}\pnru{ʈʂʰɯ˧ | ɬi˧ɳæ˩-ze˩}\hspace{5pt}\peng{She is having her menses.}\hspace{5pt}\pcmn{她来了月经。}\hspace{5pt}\pfra{Elle est en train d'avoir ses règles.}\end{exemple}
\begin{exemple}\pnru{ɬi˧ɳæ˩ go˩}\hspace{5pt}\peng{to have painful menses}\hspace{5pt}\pcmn{来了月经,疼}\hspace{5pt}\pfra{avoir des menstrues douloureuses}\end{exemple}
\end{entrée}

\begin{entrée}
{ɬi˧pæ˥}{}{ⓔɬi˧pæ˥}\formedesurface{ɬi˧pæ˥}\newline
\classe{名词}\ton{H\#}
\paradigme{\pcmn{:} \p{}}
\begin{définition}\peng{Earring.}\end{définition}
\begin{définition}\pcmn{耳环}\end{définition}
\begin{définition}\pfra{Boucle d'oreille.}\end{définition}
\begin{exemple}\pnru{ŋv̩˩-ɬi˩pæ˥ (+ɲi˩)}\hspace{5pt}\peng{silver earring}\hspace{5pt}\pcmn{银耳环}\hspace{5pt}\pfra{boucle d'oreille en argent}\end{exemple}
\begin{exemple}\pnru{hæ̃˩-ɬi˩pæ˥ (+ɲi˩)}\hspace{5pt}\peng{gold earring}\hspace{5pt}\pcmn{金耳环}\hspace{5pt}\pfra{boucle d'oreille en or}\end{exemple}
\end{entrée}

\begin{entrée}
{ɬi˧pi˩}{}{ⓔɬi˧pi˩}\formedesurface{ɬi˧pi˩}\newline
\classe{名词}\ton{L\#}
\paradigme{\pcmn{:} \p{}}
\begin{définition}\peng{Ear.}\end{définition}
\begin{définition}\pcmn{耳朵}\end{définition}
\begin{définition}\pfra{Oreille.}\end{définition}
\begin{exemple}\pnru{ɬi˧pi˩-qʰv̩˩dʑɯ˩}\hspace{5pt}\peng{hole in the ear (to put earrings)}\hspace{5pt}\pcmn{耳垂的洞(来访耳环)}\hspace{5pt}\pfra{trou dans l'oreille (pour y mettre une boucle d'oreille)}\end{exemple}
\end{entrée}

\begin{entrée}
{ɬi˧pv̩˧lv̩˥}{}{ⓔɬi˧pv̩˧lv̩˥}\formedesurface{ɬi˧pv̩˧lv̩˥}\newline
\classe{名词}\ton{H\#}
\paradigme{\pcmn{:} \p{}}
\begin{définition}\peng{Ear tumour, pathological excrescence of the ear.}\end{définition}
\begin{définition}\pcmn{耳朵瘤}\end{définition}
\begin{définition}\pfra{Tumeur de l'oreille, excroissance pathologique de l'oreille.}\end{définition}
\end{entrée}

\begin{entrée}
{ɬi˧pʰv̩\#˥}{}{ⓔɬi˧pʰv̩\#˥}\formedesurface{ɬi˧pʰv̩˧}\newline
\classe{名词}\ton{\#H}
\paradigme{\pcmn{:} \p{}}
\begin{définition}\peng{Male roebuck, male hornless river deer.}\end{définition}
\begin{définition}\pcmn{公獐子}\end{définition}
\begin{définition}\pfra{Chevrotain mâle.}\end{définition}
\begin{exemple}\pnru{ɬi˧pʰv̩˧ tʰv̩˧-mi˥\# / ɬi˧pʰv̩˧ tʰv̩˧-mi˧˥}\hspace{5pt}\peng{|fg{n}+|fg{dem}+|fg{clf}}\hspace{5pt}\pcmn{那只公獐子}\hspace{5pt}\pfra{|fg{n}+|fg{dem}+|fg{clf}}\end{exemple}
\end{entrée}

\begin{entrée}
{ɬi˧qʰæ\#˥}{}{ⓔɬi˧qʰæ\#˥}\formedesurface{ɬi˧qʰæ˧}\newline
\classe{名词}\ton{\#H}
\paradigme{\pcmn{:} \p{}}
\begin{définition}\peng{Earwax.}\end{définition}
\begin{définition}\pcmn{耳垢}\end{définition}
\begin{définition}\pfra{Cérumen.}\end{définition}
\end{entrée}

\begin{entrée}
{ɬi˧qʰv̩\#˥}{}{ⓔɬi˧qʰv̩\#˥}\formedesurface{ɬi˧qʰv̩˧}\newline
\classe{名词}\ton{\#H}
\paradigme{\pcmn{:} \p{}}
\begin{définition}\peng{Auditory canal.}\end{définition}
\begin{définition}\pcmn{耳孔}\end{définition}
\begin{définition}\pfra{Conduit auditif.}\end{définition}
\begin{exemple}\pnru{ʈʂʰɯ˧ | ɬi˧qʰv̩˧ | ɖɯ˧-pi˧˥ | tʰɑ˩˥!}\hspace{5pt}\peng{She has a sensitive ear! (Context: about a 2-year old girl who wakes up from her siesta as soon as guests come in.)}\hspace{5pt}\pcmn{她耳朵很好使! / 她耳朵很尖!(情景:一有客人到家的声音,睡午觉的两岁女孩子立即醒来。)}\hspace{5pt}\pfra{Elle a l'oreille fine! (Contexte: au sujet d'une petite fille de 2 ans qui se réveille aussitôt de sa sieste lorsqu'elle entend l'arrivée de visiteurs.)}\end{exemple}
\end{entrée}

\begin{entrée}
{ɬi˩qʰwɤ˩}{}{ⓔɬi˩qʰwɤ˩}\formedesurface{ɬi˩qʰwɤ˩˥}\newline
\classe{名词}\ton{L}
\paradigme{\pcmn{:} \p{}}
\begin{définition}\peng{Trousers.}\end{définition}
\begin{définition}\pcmn{裤子}\end{définition}
\begin{définition}\pfra{Pantalon.}\end{définition}
\end{entrée}

\begin{entrée}
{ɬi˩ʁɑ˩}{}{ⓔɬi˩ʁɑ˩}\formedesurface{ɬi˩ʁɑ˩˥}\newline
\classe{形容词}\ton{L}\begin{définition}\peng{Infuriated, in a rage (connotation: attitude of a violent and overbearing person).}\end{définition}
\begin{définition}\pcmn{大发雷霆}\end{définition}
\begin{définition}\pfra{Furieux, en rage (attitude d'une personne violente et présomptueuse).}\end{définition}
\begin{exemple}\pnru{ɬi˩ʁɑ˩ ʝi˧}\hspace{5pt}\peng{to abandon oneself to one's rage}\hspace{5pt}\pcmn{大发雷霆}\hspace{5pt}\pfra{se livrer au courroux}\end{exemple}
\end{entrée}

\begin{entrée}
{ɬi˩ʈɯ˩mæ˥}{}{ⓔɬi˩ʈɯ˩mæ˥}\formedesurface{ɬi˩ʈɯ˩mæ˥}\newline
\classe{名词}\ton{L+H\#}
\paradigme{\pcmn{:} \p{}}
\begin{définition}\peng{Lower part of the main room.}\end{définition}
\begin{définition}\pcmn{主屋里面没有火铺的地方:没有木地板、小狗可以偶尔进来的地方(家人就给它扔骨头)}\end{définition}
\begin{définition}\pfra{Contrebas du foyer: place dans la salle principale entre le foyer et la porte (où les chiens sont tolérés en fin de repas; on leur y jette des os et autres débris de nourriture; dans la maison de F4, à la date de l’enquête, c’est un endroit où rien ne recouvre le sol cimenté.}\end{définition}
\begin{exemple}\pnru{u˧=ɻ̍˩, | kʰv̩˩mi˩ ʈʂʰɯ˩-jɤ˥ | ɖɯ˧-njɤ˧-zo˥ | ɬi˩ʈɯ˩mæ˥ hĩ˩ dʑo˩.}\hspace{5pt}\peng{Us (=in our family), this dog is often seated in the lower part of the room.}\hspace{5pt}\pcmn{咱们家这只狗经常呆在主屋火塘下面的地方。}\hspace{5pt}\pfra{Nous (=dans notre maison), ce chien, il se tient souvent assis en contrebas du foyer.}\end{exemple}
\end{entrée}

\begin{entrée}
{ɬi˧ʈv̩˥}{}{ⓔɬi˧ʈv̩˥}\formedesurface{ɬi˧ʈv̩˥}\newline
\classe{名词}\ton{H\#}
\paradigme{\pcmn{:} \p{}}
\begin{définition}\peng{Asiatic plantain.}\end{définition}
\begin{définition}\pcmn{车前草}\end{définition}
\begin{définition}\pfra{Plantain (utilisé par les Na pour ses vertus médicinales; est abondant à Yongning).}\end{définition}
\end{entrée}

\begin{entrée}
{ɬi˩zo˩}{}{ⓔɬi˩zo˩}\formedesurface{ɬi˩zo˩˥}\newline
\classe{名词}\ton{L}\begin{définition}\peng{Baby roebuck.}\end{définition}
\begin{définition}\pcmn{小獐子}\end{définition}
\begin{définition}\pfra{Bébé chevrotain.}\end{définition}
\end{entrée}

\begin{entrée}
{ɬo˥}{}{ⓔɬo˥}\formedesurface{ɬo˧}\newline
\classe{名词}\ton{\#H}
\paradigme{\pcmn{:} \p{}}
\begin{définition}\peng{Rib.}\end{définition}
\begin{définition}\pcmn{肋骨}\end{définition}
\begin{définition}\pfra{Côte.}\end{définition}
\begin{exemple}\pnru{bo˩ɬo˧}\hspace{5pt}\peng{pork rib}\hspace{5pt}\pcmn{猪肋骨}\hspace{5pt}\pfra{côtes de porc}\end{exemple}
\end{entrée}

\begin{entrée}
{ɬo˧˥}{}{ⓔɬo˧˥}\formedesurface{ɬo˧˥}\newline
\classe{形容词}\ton{MH}\begin{définition}\peng{Deep (water).}\end{définition}
\begin{définition}\pcmn{深(水深)}\end{définition}
\begin{définition}\pfra{Profond (eau).}\end{définition}
\end{entrée}

\begin{entrée}
{ɬo˩kɤ˩}{}{ⓔɬo˩kɤ˩}\formedesurface{ɬo˩kɤ˩˥}\newline
\classe{名词}\ton{L}
\paradigme{\pcmn{:} \p{}}
\begin{définition}\peng{Rib.}\end{définition}
\begin{définition}\pcmn{肋骨}\end{définition}
\begin{définition}\pfra{Côte (partie du corps).}\end{définition}
\end{entrée}

\begin{entrée}
{ɬo˧kʰv̩˧}{}{ⓔɬo˧kʰv̩˧}\formedesurface{ɬo˧kʰv̩˧}\newline
\classe{名词}\ton{M}
\paradigme{\pcmn{:} \p{}}
\begin{définition}\peng{Hip.}\end{définition}
\begin{définition}\pcmn{胯}\end{définition}
\begin{définition}\pfra{Hanche.}\end{définition}
\end{entrée}

\begin{entrée}
{ɬo˧pɤ˥}{}{ⓔɬo˧pɤ˥}\formedesurface{ɬo˧pɤ˥}\newline
\classe{名词}\ton{H\#}
\paradigme{\pcmn{:} \p{}}
\begin{définition}\peng{Blister (on the hands or feet).}\end{définition}
\begin{définition}\pcmn{水泡}\end{définition}
\begin{définition}\pfra{Ampoule.}\end{définition}
\begin{exemple}\pnru{ɬo˧pɤ˥ qʰwæ˩-ze˩!}\hspace{5pt}\peng{(I/you/(s)he) got a blister!}\hspace{5pt}\pcmn{起了水泡!}\hspace{5pt}\pfra{(Il s'est/Tu t'es/Je me suis) fait une ampoule!}\end{exemple}
\begin{exemple}\pnru{ɬo˧pɤ˥ | ɖɯ˧-ɭɯ˧ | qʰwæ˧-ze˥!}\hspace{5pt}\peng{(I/you/(s)he) got a blister!}\hspace{5pt}\pcmn{起了一个水泡!}\hspace{5pt}\pfra{(Il s'est/Tu t'es/Je me suis) fait une ampoule!}\end{exemple}
\begin{exemple}\pnru{ɬo˧pɤ˥ | ʁo˩-po˥-ɳɯ˩ | ʈʂe˩˥}\hspace{5pt}\peng{to pierce a blister with a needle}\hspace{5pt}\pcmn{用针来扎水泡}\hspace{5pt}\pfra{percer une ampoule à l'aide d'une aiguille}\end{exemple}
\end{entrée}

\begin{entrée}
{ɬo˧pv̩˥}{}{ⓔɬo˧pv̩˥}\formedesurface{ɬo˧pv̩˥}\newline
\classe{名词}\ton{H\#}\begin{définition}\peng{Kowtow.}\end{définition}
\begin{définition}\pcmn{跪下磕头 (叩头)}\end{définition}
\begin{définition}\pfra{Prosternation, kow-tow (très probablement emprunt tibétain).}\end{définition}
\begin{exemple}\pnru{ɬo˧pv̩˥ ti˩}\hspace{5pt}\peng{to kowtow}\hspace{5pt}\pcmn{跪下磕头}\hspace{5pt}\pfra{se prosterner}\end{exemple}
\begin{exemple}\pnru{ɬo˧pv̩˥ | le˧-ti˩}\hspace{5pt}\peng{to kowtow}\hspace{5pt}\pcmn{跪下磕头}\hspace{5pt}\pfra{se prosterner}\end{exemple}
\end{entrée}

\begin{entrée}
{ɬo˧tɑ˧}{}{ⓔɬo˧tɑ˧}\formedesurface{ɬo˧tɑ˧}\newline
\classe{介词}\ton{M}\begin{définition}\peng{On the side of, beside.}\end{définition}
\begin{définition}\pcmn{旁边}\end{définition}
\begin{définition}\pfra{À côté de, sur le côté de.}\end{définition}
\begin{exemple}\pnru{ɬo˧tɑ˧ ɻ̍˩}\hspace{5pt}\peng{to turn to the side}\hspace{5pt}\pcmn{向侧面转}\hspace{5pt}\pfra{se tourner vers le côté, se tourner de côté}\end{exemple}
\begin{exemple}\pnru{ʁo˧qʰwɤ˩ | ɬo˧tɑ˧ | go˩˥}\hspace{5pt}\peng{to have a headache; one's temples are throbbing (literally: ‘to hurt on the sides of the head')}\hspace{5pt}\pcmn{头疼,太阳穴阵痛}\hspace{5pt}\pfra{avoir mal sur le côté de la tête, avoir les tempes qui bourdonnent (littéralement: ‘avoir mal sur les côtés de la tête')}\end{exemple}
\end{entrée}

\begin{entrée}
{ɬo˧tɑ˧-ɬo˩ɲi˩}{}{ⓔɬo˧tɑ˧-ɬo˩ɲi˩}\formedesurface{ɬo˧tɑ˧ɬo˩ɲi˩}\newline
\classe{助词}\ton{-L}\begin{définition}\peng{In the vicinity, in the surroundings.}\end{définition}
\begin{définition}\pcmn{周围,附近}\end{définition}
\begin{définition}\pfra{À droite et à gauche, aux alentours; autour de.}\end{définition}
\end{entrée}

\begin{entrée}
{ɬo˩tsʰe˩mæ˥}{}{ⓔɬo˩tsʰe˩mæ˥}\formedesurface{ɬo˩tsʰe˩mæ˥}\newline
\classe{名词}\ton{L+H\#}
\paradigme{\pcmn{:} \p{}}
\begin{définition}\peng{Hip.}\end{définition}
\begin{définition}\pcmn{胯}\end{définition}
\begin{définition}\pfra{Hanche.}\end{définition}
\end{entrée}

\begin{entrée}
{ɬv̩˧˥}{}{ⓔɬv̩˧˥}\newline
\classe{名词}
\sens{1}\paradigme{\pcmn{:} \p{}}
\begin{définition}\peng{Brains.}\end{définition}
\begin{définition}\pcmn{脑子、脑髓}\end{définition}
\begin{définition}\pfra{Cerveau, cervelle.}\end{définition}\sens{2}
\begin{définition}\peng{Marrow.}\end{définition}
\begin{définition}\pcmn{骨髓}\end{définition}
\begin{définition}\pfra{Moëlle (des os).}\end{définition}
\end{entrée}

\begin{entrée}
{ɬv̩˩α}{₁}{ⓔɬv̩˩αⓗ1}\formedesurface{ɬv̩˩˥}\newline
\classe{动词}\ton{Lα}
1\begin{définition}\peng{To hold in the mouth; to let melt in the mouth.}\end{définition}
\begin{définition}\pcmn{含在嘴里、在嘴巴里溶化}\end{définition}
\begin{définition}\pfra{Garder dans la bouche, laisser fondre dans la bouche.}\end{définition}
\begin{exemple}\pnru{tso˧∼tso˧ ɬv̩˥}\hspace{5pt}\peng{to hold something in the mouth, to have something in the mouth (context: a small child who does not yet know to distinguish between food and non-edible stuff puts things in its mouth)}\hspace{5pt}\pcmn{含在嘴里(情景:一个小孩把不能吃的东西含在嘴巴里)}\hspace{5pt}\pfra{mettre des choses dans sa bouche (contexte: un petit enfant qui ne fait pas encore la différence entre nourriture et choses non comestibles met des choses dans sa bouche)}\end{exemple}
\end{entrée}

\begin{entrée}
{ɬv̩˩α}{₂}{ⓔɬv̩˩αⓗ2}\formedesurface{ɬv̩˩˥}\newline
\classe{形容词}\ton{Lα}
2\begin{définition}\peng{Warm.}\end{définition}
\begin{définition}\pcmn{温暖,暖和}\end{définition}
\begin{définition}\pfra{Chaud, tiède (agréablement tiède, pas froid).}\end{définition}
\begin{exemple}\pnru{dʑɤ˩˥ | ɬv̩˩˥}\hspace{5pt}\peng{nice and warm}\hspace{5pt}\pcmn{温暖}\hspace{5pt}\pfra{très tiède, bien tiède}\end{exemple}
\begin{exemple}\pnru{ɖwæ˧˥ | ɬv̩˩˥}\hspace{5pt}\peng{|fg{intensive.very}}\hspace{5pt}\pcmn{很暖和}\hspace{5pt}\pfra{|fg{intensif.très}: très tiède}\end{exemple}
\begin{exemple}\pnru{ɬv̩˩-hĩ˩˥}\hspace{5pt}\peng{|fg{rel}/|fg{nmlz}}\hspace{5pt}\pcmn{温暖的}\hspace{5pt}\pfra{|fg{rel}/|fg{nmlz}}\end{exemple}
\end{entrée}

\begin{entrée}
{ɬv̩˩α}{₃}{ⓔɬv̩˩αⓗ3}\formedesurface{ɬv̩˩˥}\newline
\classe{动词}\ton{Lα}
3
\étymologie{
ɬv̩˩a 2
}\begin{définition}\peng{To warm up (food).}\end{définition}
\begin{définition}\pcmn{热饭}\end{définition}
\begin{définition}\pfra{Réchauffer de la nourriture.}\end{définition}
\begin{exemple}\pnru{hɑ˧ ɬv̩˧˥}\hspace{5pt}\peng{to warm up rice / food}\hspace{5pt}\pcmn{热饭}\hspace{5pt}\pfra{réchauffer du riz / de la nourriture}\end{exemple}
\begin{exemple}\pnru{hɑ˧ | le˧-ɬv̩˩}\hspace{5pt}\peng{to warm up rice / food}\hspace{5pt}\pcmn{热饭}\hspace{5pt}\pfra{réchauffer du riz / de la nourriture}\end{exemple}
\begin{exemple}\pnru{hɑ˧ | ɖɯ˧-ɬv̩˧∼ɬv̩˥-ɻ̍˩}\hspace{5pt}\peng{to warm up food a little}\hspace{5pt}\pcmn{饭热一热}\hspace{5pt}\pfra{réchauffer un peu la nourriture}\end{exemple}
\end{entrée}

\begin{entrée}
{ɬv̩˧gv̩\#˥}{}{ⓔɬv̩˧gv̩\#˥}\formedesurface{ɬv̩˧gv̩˧}\newline
\classe{名词}\ton{\#H}\begin{définition}\peng{Ritual offering of food to the deceased, seven days after cremation.}\end{définition}
\begin{définition}\pcmn{火葬后第七天的送食物仪式}\end{définition}
\begin{définition}\pfra{Nourriture qu'on offre rituellement au défunt, sept jours après sa crémation.}\end{définition}
\end{entrée}

\begin{entrée}
{ɬv̩˧mi˧mæ˧dv̩˧mi\#˥}{}{ⓔɬv̩˧mi˧mæ˧dv̩˧mi\#˥}\formedesurface{ɬv̩˧mi˧mæ˧dv̩˧mi˧}\newline
\classe{名词}\ton{\#H}
\paradigme{\pcmn{:} \p{}}
\begin{définition}\peng{Praying mantis.}\end{définition}
\begin{définition}\pcmn{螳螂}\end{définition}
\begin{définition}\pfra{Mante religieuse.}\end{définition}
\begin{exemple}\pnru{ɬv̩˧mi˧mæ˧dv̩˧mi˧ tʰv̩˧-mi˧˥ / ɬv̩˧mi˧mæ˧dv̩˧mi˧ tʰv̩˧-mi˥\#}\hspace{5pt}\peng{|fg{n}+|fg{dem}+|fg{clf}}\hspace{5pt}\pcmn{那只螳螂}\hspace{5pt}\pfra{|fg{n}+|fg{dem}+|fg{clf}}\end{exemple}
\end{entrée}

\begin{entrée}
{ɬv̩˩pʰæ˩}{}{ⓔɬv̩˩pʰæ˩}\formedesurface{ɬv̩˩pʰæ˩˥}\newline
\classe{形容词}\ton{L}\begin{définition}\peng{Nice and warm.}\end{définition}
\begin{définition}\pcmn{温暖,暖和}\end{définition}
\begin{définition}\pfra{Bien tiède.}\end{définition}
\end{entrée}

\begin{entrée}
{ɬv̩˧ʁwɤ\#˥}{}{ⓔɬv̩˧ʁwɤ\#˥}\formedesurface{ɬv̩˧ʁwɤ˧}\newline
\classe{名词}\ton{\#H}\begin{définition}\peng{Village name.}\end{définition}
\begin{définition}\pcmn{村落名}\end{définition}
\begin{définition}\pfra{Village en aval de Qiansuo; leur langue serait relativement proche de celle de la vallée de Yongning.}\end{définition}
\end{entrée}

\newpage\caractère{m}

\begin{entrée}
{mɑ˩dzɑ˩}{}{ⓔmɑ˩dzɑ˩}\formedesurface{mɑ˩dzɑ˩˥}\newline
\classe{名词}\ton{L}
\paradigme{\pcmn{:} \p{}}
\begin{définition}\peng{Ink (solid).}\end{définition}
\begin{définition}\pcmn{墨}\end{définition}
\begin{définition}\pfra{Encre (solide).}\end{définition}
\end{entrée}

\begin{entrée}
{mɑ˩ɳɯ\#˥}{}{ⓔmɑ˩ɳɯ\#˥}\formedesurface{mɑ˩ɳɯ˥}\newline
\classe{名词}\ton{LM+\#H}
\paradigme{\pcmn{:} \p{}}
\begin{définition}\peng{Mani wall, Mani pile: pile built of rubble and sand, with carved stone tablets, most with the inscription Om Mani Padme Hum. A Mani wall should be passed or circumvented from the left side, the clockwise direction in which the universe revolves, according to Buddhist doctrine.}\end{définition}
\begin{définition}\pcmn{嘛呢堆}\end{définition}
\begin{définition}\pfra{Mur de mani: mur de pierre sèche et de sable, comportant des tablettes de pierre sur lesquelles est gravé une inscription: le plus souvent Om Mani Padme Hum. Un mur de mani doit être contourné dans le sens des aiguilles d'une montre: le sens de rotation de l'univers, selon la doctrine bouddhiste.}\end{définition}
\end{entrée}

\begin{entrée}
{mɑ˩ɳɯ˧-do˥bv̩˩}{}{ⓔmɑ˩ɳɯ˧-do˥bv̩˩}\formedesurface{mɑ˩ɳɯ˧do˥bv̩˩}\newline
\classe{名词}\ton{LM+\#H-}
\paradigme{\pcmn{:} \p{}}
\begin{définition}\peng{Mani wall, Mani pile: pile built of rubble and sand, with carved stone tablets, most with the inscription Om Mani Padme Hum. A Mani wall should be passed or circumvented from the left side, the clockwise direction in which the universe revolves, according to Buddhist doctrine.}\end{définition}
\begin{définition}\pcmn{嘛呢堆}\end{définition}
\begin{définition}\pfra{Mur de mani: mur de pierre sèche et de sable, comportant des tablettes de pierre sur lesquelles est gravé une inscription: le plus souvent Om Mani Padme Hum. Un mur de mani doit être contourné dans le sens des aiguilles d'une montre: le sens de rotation de l'univers, selon la doctrine bouddhiste.}\end{définition}
\end{entrée}

\begin{entrée}
{mɑ˧pʰv̩˧}{}{ⓔmɑ˧pʰv̩˧}\formedesurface{mɑ˧pʰv̩˧}\newline
\classe{名词}\ton{M}\begin{définition}\peng{Butter.}\end{définition}
\begin{définition}\pcmn{酥油}\end{définition}
\begin{définition}\pfra{Beurre (pour la préparation du thé au beurre).}\end{définition}
\end{entrée}

\begin{entrée}
{mɑ˧tsɑ˥}{}{ⓔmɑ˧tsɑ˥}\formedesurface{mɑ˧tsɑ˥}\newline
\classe{名词}\ton{H\#}
\paradigme{\pcmn{:} \p{}}
\begin{définition}\peng{Origin, distant cause, remote cause.}\end{définition}
\begin{définition}\pcmn{来历、发源地、深层原因/来源、来龙去脉、脉络}\end{définition}
\begin{définition}\pfra{Origine, cause (lointaine).}\end{définition}
\begin{exemple}\pnru{mɑ˧tsɑ˥ | ʈʂʰɯ˧-qo˧ le˧-tsʰɯ˩-ɲi˩! |}\hspace{5pt}\peng{(Of an event:) It comes from afar! / It does not take place simply by chance: there is a long story behind it!}\hspace{5pt}\pcmn{这(件事情)出处很远! / 有它的来龙去脉(=不是突然一下子出现的)!}\hspace{5pt}\pfra{(Au sujet d'un événement) Ca vient de loin! / ça a une origine/ça n'arrive pas par hasard!}\end{exemple}
\begin{exemple}\pnru{mɑ˧tsɑ˥ ʈʂʰɯ˩-kʰwɤ˩ |}\hspace{5pt}\peng{|fg{n}+|fg{dem}+|fg{clf}: this cause, this origin}\hspace{5pt}\pcmn{这个来历}\hspace{5pt}\pfra{|fg{n}+|fg{dem}+|fg{clf}: cette cause, cette origine}\end{exemple}
\end{entrée}

\begin{entrée}
{mæ˧}{}{ⓔmæ˧}\formedesurface{mæ˧}\newline
\classe{语气助词}\ton{M}\begin{définition}\peng{Final particle conveying obviousness.}\end{définition}
\begin{définition}\pcmn{句尾助词,表示显然、理所当然:“……呗!”}\end{définition}
\begin{définition}\pfra{Particule finale exprimant l'évidence.}\end{définition}
\begin{exemple}\pnru{hu˧mi˧-ʈʂʰæ˧ɣɯ˧ | le˧-ʈʰɯ˩, | le˧-qʰwɤ˧-ze˧ mæ˧! |}\hspace{5pt}\peng{[Nowadays] one simply takes medicines for the stomach, and one is healed! [unlike in the old times, when there were no hospitals]}\hspace{5pt}\pcmn{吃了胃药,就好了呗!}\hspace{5pt}\pfra{On prend des médicaments pour l'estomac, et ça guérit!}\end{exemple}
\end{entrée}

\begin{entrée}
{mæ˧}{₁}{ⓔmæ˧ⓗ1}\formedesurface{mæ˧}\newline
\classe{动词}\ton{M}
1\begin{définition}\peng{To be free to, to have time to.}\end{définition}
\begin{définition}\pcmn{(有)空}\end{définition}
\begin{définition}\pfra{Avoir le temps de, être libre.}\end{définition}
\begin{exemple}\pnru{njɤ˧ | mɤ˧-mæ˧.}\hspace{5pt}\peng{I do not have the time; I am busy}\hspace{5pt}\pcmn{我忙、我没有空}\hspace{5pt}\pfra{je suis occupé, je n'ai pas le temps}\end{exemple}
\begin{exemple}\pnru{njɤ˧ | mæ˧-mɤ˧-ho˩.}\hspace{5pt}\peng{I won't have the time.}\hspace{5pt}\pcmn{我不会有时间。}\hspace{5pt}\pfra{Je ne vais pas avoir le temps.}\end{exemple}
\end{entrée}

\begin{entrée}
{mæ˧}{₂}{ⓔmæ˧ⓗ2}\formedesurface{mæ˧}\newline
\classe{动词}\ton{M}
2\begin{définition}\peng{To manage (to do something).}\end{définition}
\begin{définition}\pcmn{能够(做)}\end{définition}
\begin{définition}\pfra{Parvenir à.}\end{définition}
\begin{exemple}\pnru{ɖɯ˩-hĩ˩ qʰɑ˥ mæ˩∼mæ˩! | tɕi˩-hĩ˩ lə˥-mɤ˩-mæ˩! / ɖɯ˩-hĩ˩˥, | qʰɑ˧ mæ˥∼mæ˩! | tɕi˩-hĩ˩˥, | le˧-mɤ˧-mæ˧!}\hspace{5pt}\peng{“Adults can manage all sorts of things, (whereas) children can't manage (that much) yet!" This saying is used when someone puts high demands on children or adolescents: Let the children play! To each age its occupations: children should play, not work. Adults' tasks are not their business!}\hspace{5pt}\pcmn{“大人管干活,小孩管玩耍!”这个谚语的意思是:不要让孩子干活,每个年龄有自己的事,孩子的事就是玩。成年人的活儿,不是他们的事!}\hspace{5pt}\pfra{«Les adultes peuvent tout faire; les enfants, eux, n'y arrivent pas/n'en sont pas capables!» Sens: s'adresse à quelqu'un qui assigne des tâches aux enfants et adolescents: Laissez les enfants jouer! A chacun ses occupations: les adultes travaillent; les enfants, leur tâche, c'est de s'amuser entre eux, pas de travailler. Le travail des grands, c'est pas leur affaire!}\end{exemple}
\end{entrée}

\begin{entrée}
{mæ˧˥}{}{ⓔmæ˧˥}\newline
\classe{动词}
\sens{1}
\begin{définition}\peng{To close (the mouth).}\end{définition}
\begin{définition}\pcmn{闭(嘴)}\end{définition}
\begin{définition}\pfra{Fermer (la bouche).}\end{définition}
\begin{exemple}\pnru{ɲi˧to˧ | tʰi˧-mæ˧˥}\hspace{5pt}\peng{to close the mouth}\hspace{5pt}\pcmn{闭嘴}\hspace{5pt}\pfra{fermer la bouche}\end{exemple}
\begin{exemple}\pnru{mæ˩∼mæ˧˥}\hspace{5pt}\peng{|fg{red}}\hspace{5pt}\pcmn{重叠}\hspace{5pt}\pfra{|fg{red}}\end{exemple}
\begin{exemple}\pnru{ɲi˧to˧ | tʰi˧-mæ˩∼mæ˩}\hspace{5pt}\peng{to close the mouth}\hspace{5pt}\pcmn{闭嘴}\hspace{5pt}\pfra{fermer la bouche}\end{exemple}\sens{2}
\begin{définition}\peng{To purse (one's lips).}\end{définition}
\begin{définition}\pcmn{抿(嘴巴)}\end{définition}
\begin{définition}\pfra{Pincer (les lèvres).}\end{définition}
\end{entrée}

\begin{entrée}
{mæ˧α}{}{ⓔmæ˧α}\newline
\classe{动词}
\sens{1}
\begin{définition}\peng{To clutch, to catch hold of.}\end{définition}
\begin{définition}\pcmn{钩住(东西)}\end{définition}
\begin{définition}\pfra{Attraper (un objet en hauteur).}\end{définition}
\begin{exemple}\pnru{tʰi˧-mæ˧-ze˧}\hspace{5pt}\peng{|fg{dur} \_ |fg{pfv}}\hspace{5pt}\pcmn{钩住了}\hspace{5pt}\pfra{|fg{dur} \_ |fg{pfv}}\end{exemple}\sens{2}
\begin{définition}\peng{To catch up with (someone).}\end{définition}
\begin{définition}\pcmn{跟上}\end{définition}
\begin{définition}\pfra{Rattraper, rejoindre (quelqu'un qui est plus avant sur un chemin/une route).}\end{définition}
\end{entrée}

\begin{entrée}
{mæ˧β}{}{ⓔmæ˧β}\formedesurface{mæ˧}\newline
\classe{动词}\ton{M}\begin{définition}\peng{To achieve, to succeed in, to complete (a task).}\end{définition}
\begin{définition}\pcmn{……成、……成功}\end{définition}
\begin{définition}\pfra{Parvenir à, réussir à.}\end{définition}
\begin{exemple}\pnru{njɤ˧ ɖʐɤ˧˥ | tʰi˧-mɤ˧-mæ˧!}\hspace{5pt}\peng{I can't fetch it!}\hspace{5pt}\pcmn{我够不着!(例如:够不着树枝上的果子)}\hspace{5pt}\pfra{je ne parviens pas à attraper (ex.: un fruit sur une branche trop élevée)}\end{exemple}
\end{entrée}

\begin{entrée}
{mæ˩}{}{ⓔmæ˩}\formedesurface{mæ˩˥}\newline
\classe{动词}\ton{L}\begin{définition}\peng{To water, to irrigate (making small trenches and pouring water into them).}\end{définition}
\begin{définition}\pcmn{灌溉}\end{définition}
\begin{définition}\pfra{Irriguer (en faisant couler de l'eau dans de petites tranchées).}\end{définition}
\begin{exemple}\pnru{dʑɯ˩ mæ˩˥}\hspace{5pt}\peng{to irrigate, to water}\hspace{5pt}\pcmn{浇灌}\hspace{5pt}\pfra{irriguer, arroser, mettre de l’eau}\end{exemple}
\begin{exemple}\pnru{dʑɯ˧ | le˧-mæ˩}\hspace{5pt}\peng{|fg{accomp}: to water, to irrigate}\hspace{5pt}\pcmn{浇灌}\hspace{5pt}\pfra{|fg{accomp}: irriguer, arroser}\end{exemple}
\end{entrée}

\begin{entrée}
{mæ˩α}{}{ⓔmæ˩α}\formedesurface{mæ˩˥}\newline
\classe{动词}\ton{Lα}\begin{définition}\peng{To aim at; to point at.}\end{définition}
\begin{définition}\pcmn{瞄准,指}\end{définition}
\begin{définition}\pfra{Viser; pointer, montrer du doigt.}\end{définition}
\begin{exemple}\pnru{tʰi˧-mæ˩-ze˩, | qʰæ˧-bi˥-ze˩.}\hspace{5pt}\peng{[(S)he] has aimed; [(s)he] will now shoot.}\hspace{5pt}\pcmn{瞄准了,要开枪了。}\hspace{5pt}\pfra{(Il) a visé, (il) va tirer.}\end{exemple}
\begin{exemple}\pnru{lo˧ɲi˥ mæ˩}\hspace{5pt}\peng{to point at with the finger}\hspace{5pt}\pcmn{用手指出}\hspace{5pt}\pfra{montrer du doigt}\end{exemple}
\begin{exemple}\pnru{tso˧∼tso˧ mæ˥}\hspace{5pt}\peng{to point at things}\hspace{5pt}\pcmn{指东西}\hspace{5pt}\pfra{pointer des choses du doigt}\end{exemple}
\end{entrée}

\begin{entrée}
{mæ˩α}{₁}{ⓔmæ˩αⓗ1}\formedesurface{ɖɯ˧ mæ˩}\newline
\classe{量词}\ton{Lα}
1\begin{définition}\peng{Monetary unit: yuan.}\end{définition}
\begin{définition}\pcmn{量词:钱(一元)}\end{définition}
\begin{définition}\pfra{Unité monétaire: un yuan.}\end{définition}
\begin{exemple}\pnru{ʈʂʰɯ˧-mæ˥}\hspace{5pt}\peng{|fg{dem} \_ (tone: H\# / H\$)}\hspace{5pt}\pcmn{指示代词 \_}\hspace{5pt}\pfra{|fg{dem} \_ (tone: H\# / H\$)}\end{exemple}
\end{entrée}

\begin{entrée}
{mæ˩α}{₂}{ⓔmæ˩αⓗ2}\formedesurface{ɖɯ˧ mæ˩}\newline
\classe{量词}\ton{Lα}
2\begin{définition}\peng{10,000.}\end{définition}
\begin{définition}\pcmn{万(数词充当量词)}\end{définition}
\begin{définition}\pfra{10.000.}\end{définition}
\begin{exemple}\pnru{ɖɯ˧-mæ˩}\hspace{5pt}\peng{10,000}\hspace{5pt}\pcmn{一万}\hspace{5pt}\pfra{10.000}\end{exemple}
\begin{exemple}\pnru{tsʰe˩-tv̩˩ mæ˥}\hspace{5pt}\peng{ten thousand times 10,000, i.e. one hundred million}\hspace{5pt}\pcmn{十千万,等于一亿}\hspace{5pt}\pfra{dix mille fois 10.000, soit cent millions}\end{exemple}
\begin{exemple}\pnru{ɖɯ˧-ɕi˧ mæ˩}\hspace{5pt}\peng{one hundred times 10,000, i.e. one million}\hspace{5pt}\pcmn{一百万}\hspace{5pt}\pfra{cent fois 10.000, soit un million}\end{exemple}
\end{entrée}

\begin{entrée}
{mæ˩ɖʐo˥}{}{ⓔmæ˩ɖʐo˥}\formedesurface{mæ˩ɖʐo˥}\newline
\classe{名词}\ton{LH}
\paradigme{\pcmn{:} \p{}}
\begin{définition}\peng{Whip.}\end{définition}
\begin{définition}\pcmn{鞭子}\end{définition}
\begin{définition}\pfra{Fouet.}\end{définition}
\begin{exemple}\pnru{ʐwæ˧-mæ˥ɖʐo˩}\hspace{5pt}\peng{horse whip}\hspace{5pt}\pcmn{马鞭}\hspace{5pt}\pfra{fouet de cheval}\end{exemple}
\end{entrée}

\begin{entrée}
{mæ˩ko˥}{}{ⓔmæ˩ko˥}\formedesurface{mæ˩ko˥}\newline
\classe{名词}\ton{LH}
\paradigme{\pcmn{:} \p{}}
\begin{définition}\peng{Harness.}\end{définition}
\begin{définition}\pcmn{挽具}\end{définition}
\begin{définition}\pfra{Harnais.}\end{définition}
\begin{exemple}\pnru{ʐwæ˧-mæ˥ko˩}\hspace{5pt}\peng{horse harness}\hspace{5pt}\pcmn{马挽具}\hspace{5pt}\pfra{harnais de cheval}\end{exemple}
\end{entrée}

\begin{entrée}
{‑mæ˧mæ˥}{}{ⓔ‑mæ˧mæ˥}\formedesurface{mæ˧mæ˥}\newline
\classe{}\ton{H\#}\begin{définition}\peng{At the end of, towards the end of, in the latter part of.}\end{définition}
\begin{définition}\pcmn{……末/底}\end{définition}
\begin{définition}\pfra{À la fin de, vers la fin de.}\end{définition}
\begin{exemple}\pnru{ɲi˧ɬi˧mi˧-mæ˧mæ˥, | qʰɑ˧dze˧ tv̩˧!}\hspace{5pt}\peng{Towards the end of the second month, one plants sweet corn!}\hspace{5pt}\pcmn{二月底,就种玉米! / 玉米是在二月底种的!}\hspace{5pt}\pfra{vers la fin du deuxième mois, on plante le maïs!}\end{exemple}
\begin{exemple}\pnru{gv̩˩ɬi˩mi˩-mæ˩-mæ˥}\hspace{5pt}\peng{towards the end of the ninth month}\hspace{5pt}\pcmn{九月底}\hspace{5pt}\pfra{vers la fin du neuvième mois}\end{exemple}
\end{entrée}

\begin{entrée}
{mæ˧pæ˧}{}{ⓔmæ˧pæ˧}\formedesurface{mæ˧pæ˧}\newline
\classe{名词}\ton{M}
\paradigme{\pcmn{:} \p{}}
\begin{définition}\peng{Large sifter.}\end{définition}
\begin{définition}\pcmn{大筛子}\end{définition}
\begin{définition}\pfra{Vannerie.}\end{définition}
\end{entrée}

\begin{entrée}
{mæ˧qo˩}{}{ⓔmæ˧qo˩}\formedesurface{mæ˧qo˩}\newline
\classe{助词}\ton{L\#}\begin{définition}\peng{At the extremity, at the end; at the bottom, in the lower part.}\end{définition}
\begin{définition}\pcmn{在尽头、在极点,在下面、在后面}\end{définition}
\begin{définition}\pfra{En bas, au fond; à l'arrière, derrière.}\end{définition}
\end{entrée}

\begin{entrée}
{‑mæ˧qo˩}{}{ⓔ‑mæ˧qo˩}\formedesurface{mæ˧qo˩}\newline
\classe{}\ton{L\#}\begin{définition}\peng{Below, behind.}\end{définition}
\begin{définition}\pcmn{下面,后面}\end{définition}
\begin{définition}\pfra{Derrière, sous.}\end{définition}
\end{entrée}

\begin{entrée}
{mæ˧qv̩˩}{}{ⓔmæ˧qv̩˩}\formedesurface{mæ˧qv̩˩}\newline
\classe{名词}\ton{L\#}
\paradigme{\pcmn{:} \p{}}
\begin{définition}\peng{Tail.}\end{définition}
\begin{définition}\pcmn{尾巴}\end{définition}
\begin{définition}\pfra{Queue.}\end{définition}
\begin{exemple}\pnru{ʝi˧-mæ˧qv̩˥}\hspace{5pt}\peng{cow's tail}\hspace{5pt}\pcmn{牛尾巴}\hspace{5pt}\pfra{queue de la vache}\end{exemple}
\end{entrée}

\begin{entrée}
{mæ˧ɻ̃\#˥}{}{ⓔmæ˧ɻ̃\#˥}\formedesurface{mæ˧ɻ̃˧}\newline
\classe{名词}\ton{\#H}
\paradigme{\pcmn{:} \p{}}
\begin{définition}\peng{Coccyx.}\end{définition}
\begin{définition}\pcmn{尾椎骨}\end{définition}
\begin{définition}\pfra{Coccyx.}\end{définition}
\end{entrée}

\begin{entrée}
{mæ˧ɻæ˩}{}{ⓔmæ˧ɻæ˩}\formedesurface{mæ˧ɻæ˩}\newline
\classe{名词}\ton{L\#}\begin{définition}\peng{Vegetable oil.}\end{définition}
\begin{définition}\pcmn{植物油}\end{définition}
\begin{définition}\pfra{Huile végétale.}\end{définition}
\end{entrée}

\begin{entrée}
{mɤ˧‑}{}{ⓔmɤ˧‑}\formedesurface{mɤ˧}\newline
\classe{前缀}\ton{M}\begin{définition}\peng{Negation.}\end{définition}
\begin{définition}\pcmn{否定:不,没}\end{définition}
\begin{définition}\pfra{Negation.}\end{définition}
\end{entrée}

\begin{entrée}
{mɤ˩}{}{ⓔmɤ˩}\formedesurface{mɤ˧}\newline
\classe{名词}\ton{L}\begin{définition}\peng{Animal fat.}\end{définition}
\begin{définition}\pcmn{动物油}\end{définition}
\begin{définition}\pfra{Huile animale, graisse.}\end{définition}
\begin{exemple}\pnru{njɤ˧ | mɤ˩ mɤ˩ dzɯ˩˥!}\hspace{5pt}\peng{I don't eat animal fat! (One of the investigator's peculiarities)}\hspace{5pt}\pcmn{我不吃猪油!(这是调查者的特点之一)}\hspace{5pt}\pfra{Je ne mange pas de graisse/de saindoux! (C'est là l'une des particularités de l'enquêteur)}\end{exemple}
\end{entrée}

\begin{entrée}
{mɤ˩α}{}{ⓔmɤ˩α}\formedesurface{ɖɯ˧ mɤ˩}\newline
\classe{量词}\ton{Lα}\begin{définition}\peng{A few, a little.}\end{définition}
\begin{définition}\pcmn{量词:一些、一点}\end{définition}
\begin{définition}\pfra{Classificateur des petites quantités: quelques-uns, quelque peu de, un peu de.}\end{définition}
\begin{exemple}\pnru{ɕi˧ɭɯ˧-ɻæ˩ | ɖɯ˧-mɤ˩}\hspace{5pt}\peng{a few seeds of rice}\hspace{5pt}\pcmn{一些稻谷种子}\hspace{5pt}\pfra{quelques graines de riz}\end{exemple}
\begin{exemple}\pnru{ɻæ˩˥ | ɖɯ˧-mɤ˩}\hspace{5pt}\peng{a few seeds}\hspace{5pt}\pcmn{一些种子}\hspace{5pt}\pfra{quelques graines}\end{exemple}
\begin{exemple}\pnru{tsɑ˧bɤ˧ | ɖɯ˧-mɤ˩, | tsɑ˧bɤ˧ | ɲi˧-mɤ˩}\hspace{5pt}\peng{a small quantity of flour; two small quantities of flour; etc}\hspace{5pt}\pcmn{一小捧面粉、两小捧面粉……}\hspace{5pt}\pfra{un peu de farine; deux poignées/petites quantités de farine; etc.}\end{exemple}
\begin{exemple}\pnru{ʈʂʰɯ˧-mɤ˥}\hspace{5pt}\peng{|fg{dem} \_ (tone: H\# / H\$)}\hspace{5pt}\pcmn{指示代词 \_}\hspace{5pt}\pfra{|fg{dem} \_ (tone: H\# / H\$)}\end{exemple}
\end{entrée}

\begin{entrée}
{mɤ˩β}{}{ⓔmɤ˩β}\formedesurface{mɤ˩˥}\newline
\classe{动词}\ton{Lβ}\begin{définition}\peng{To eat food in powder form, typically tsamba.}\end{définition}
\begin{définition}\pcmn{将粉状的食品放在嘴里(如:干糌粑)}\end{définition}
\begin{définition}\pfra{Prendre dans la bouche un aliment en poudre.}\end{définition}
\begin{exemple}\pnru{tsɑ˧bɤ˧ mɤ˩}\hspace{5pt}\peng{to eat dry tsamba: one takes a spoonful, pours it into the mouth, and lets it get impregnated with saliva}\hspace{5pt}\pcmn{吃干糌粑}\hspace{5pt}\pfra{manger du tsamba sec: on en prend une cuillère qu'on renverse dans sa bouche, et on laisse la farine s'imprégner de salive}\end{exemple}
\begin{exemple}\pnru{tsɑ˧bɤ˧ | ɖɯ˧-mɤ˧∼mɤ˩-ɻ̍˩}\hspace{5pt}\peng{to eat a little dry tsamba, to take the time to appreciate some dry tsamba}\hspace{5pt}\pcmn{品干糌粑、慢慢享受一点干糌粑}\hspace{5pt}\pfra{savourer un peu de tsamba}\end{exemple}
\end{entrée}

\begin{entrée}
{mɤ˧-dɑ˩}{}{ⓔmɤ˧-dɑ˩}\formedesurface{mɤ˧dɑ˩}\newline
\classe{感叹词}\ton{-L}\begin{définition}\peng{Alas!}\end{définition}
\begin{définition}\pcmn{感叹词:唉呀!(自怨自艾的语气)}\end{définition}
\begin{définition}\pfra{Hélas!}\end{définition}
\end{entrée}

\begin{entrée}
{mɤ˧-dɑ˩mi˩}{}{ⓔmɤ˧-dɑ˩mi˩}\formedesurface{mɤ˧dɑ˩mi˩}\newline
\classe{感叹词}\ton{-L}\begin{définition}\peng{Alas!}\end{définition}
\begin{définition}\pcmn{感叹词:唉呀啊!(自怨自艾的语气)}\end{définition}
\begin{définition}\pfra{Hélas!}\end{définition}
\end{entrée}

\begin{entrée}
{mɤ˧-dɑ˩qʰwɤ˩}{}{ⓔmɤ˧-dɑ˩qʰwɤ˩}\formedesurface{mɤ˧dɑ˩qʰwɤ˩}\newline
\classe{感叹词}\ton{-L}\begin{définition}\peng{Alas!}\end{définition}
\begin{définition}\pcmn{感叹词:唉呀啊!(自怨自艾的语气)}\end{définition}
\begin{définition}\pfra{Hélas!}\end{définition}
\end{entrée}

\begin{entrée}
{mɤ˩ɬi˩}{}{ⓔmɤ˩ɬi˩}\formedesurface{mɤ˩ɬi˩˥}\newline
\classe{名词}\ton{L}
\paradigme{\pcmn{:} \p{}}
\begin{définition}\peng{Butter tea.}\end{définition}
\begin{définition}\pcmn{酥油茶}\end{définition}
\begin{définition}\pfra{Thé au beurre.}\end{définition}
\end{entrée}

\begin{entrée}
{mɤ˩mv̩˩}{}{ⓔmɤ˩mv̩˩}\formedesurface{mɤ˩mv̩˩˥}\newline
\classe{名词}\ton{L}
\paradigme{\pcmn{:} \p{}}
\begin{définition}\peng{Candle holder.}\end{définition}
\begin{définition}\pcmn{烛台}\end{définition}
\begin{définition}\pfra{Porte-bougie: objet en cuivre dans lequel on verse de la paraffine fondue, ou de la graisse, et dans lequel on place une mèche; est utilisé dans les rituels.}\end{définition}
\end{entrée}

\begin{entrée}
{mɤ˧-ni˩∼ni˩}{}{ⓔmɤ˧-ni˩∼ni˩}\formedesurface{mɤ˧ni˩ni˩}\newline
\classe{形容词}\ton{-L}
\sens{1}
\begin{définition}\peng{Different from, not identical with.}\end{définition}
\begin{définition}\pcmn{不一样、有区别}\end{définition}
\begin{définition}\pfra{Différent, pas identique.}\end{définition}
\begin{exemple}\pnru{mɤ˧-ni˩∼ni˩-hĩ˩-lɑ˩ ɲi˩-ze˩!}\hspace{5pt}\peng{It's not the same!}\hspace{5pt}\pcmn{不是一样的!不是一回事!}\hspace{5pt}\pfra{Ce n'est pas pareil!}\end{exemple}\sens{2}
\begin{définition}\peng{Incredible, extraordinary, astounding.}\end{définition}
\begin{définition}\pcmn{难以相信、了不起、很特别}\end{définition}
\begin{définition}\pfra{Incroyable, extraordinaire, merveilleux.}\end{définition}
\begin{exemple}\pnru{mɤ˧-ni˩∼ni˩-hĩ˩-lɑ˩ ɲi˩-ze˩!}\hspace{5pt}\peng{It's not just the same old thing! / It's really extraordinary!}\hspace{5pt}\pcmn{这非常特别!}\hspace{5pt}\pfra{C'est absolument extraordinaire!}\end{exemple}
\begin{exemple}\pnru{ʈʂʰɯ˧ | mɤ˧-ni˩∼ni˩-hĩ˩ | ɖɯ˧-v̩˧ ɲi˩!}\hspace{5pt}\peng{(S)he is someone really exceptional!}\hspace{5pt}\pcmn{这是很利害的一个人!}\hspace{5pt}\pfra{C'est quelqu'un d'exceptionnel/d'extraordinaire!}\end{exemple}
\end{entrée}

\begin{entrée}
{mɤ˩tʰɑ˧}{}{ⓔmɤ˩tʰɑ˧}\formedesurface{mɤ˩tʰɑ˥}\newline
\classe{名词}\ton{LM}\begin{définition}\peng{Sesame candy.}\end{définition}
\begin{définition}\pcmn{麻糖(汉语借词)}\end{définition}
\begin{définition}\pfra{Confiserie au sésame.}\end{définition}
\end{entrée}

\begin{entrée}
{mɤ˧ʈʂʰɤ˧}{}{ⓔmɤ˧ʈʂʰɤ˧}\formedesurface{mɤ˧ʈʂʰɤ˧}\newline
\classe{名词}\ton{M}\begin{définition}\peng{Cart.}\end{définition}
\begin{définition}\pcmn{马车(汉语借词)}\end{définition}
\begin{définition}\pfra{Charrette.}\end{définition}
\end{entrée}

\begin{entrée}
{mi˧}{}{ⓔmi˧}\formedesurface{mi˧}\newline
\classe{名词}\ton{M}
\paradigme{\pcmn{:} \p{}}
\begin{définition}\peng{Wound.}\end{définition}
\begin{définition}\pcmn{伤口}\end{définition}
\begin{définition}\pfra{Blessure, plaie.}\end{définition}
\end{entrée}

\begin{entrée}
{mi˧˥}{}{ⓔmi˧˥}\formedesurface{mi˧˥}\newline
\classe{动词}\ton{MH}\begin{définition}\peng{To push.}\end{définition}
\begin{définition}\pcmn{推、拥挤}\end{définition}
\begin{définition}\pfra{Pousser.}\end{définition}
\begin{exemple}\pnru{le˧-mi˧-ze˥}\hspace{5pt}\peng{|fg{accomp} \_ |fg{pfv}}\hspace{5pt}\pcmn{推开了}\hspace{5pt}\pfra{|fg{accomp} \_ |fg{pfv}}\end{exemple}
\begin{exemple}\pnru{le˧-mi˧˥}\hspace{5pt}\peng{|fg{accomp}}\hspace{5pt}\pcmn{推}\hspace{5pt}\pfra{|fg{accomp}}\end{exemple}
\begin{exemple}\pnru{tʰi˧-mi˧˥}\hspace{5pt}\peng{|fg{dur}}\hspace{5pt}\pcmn{推}\hspace{5pt}\pfra{|fg{dur}}\end{exemple}
\begin{exemple}\pnru{tso˧∼tso˧ mi˩}\hspace{5pt}\peng{to push something}\hspace{5pt}\pcmn{推开一个东西}\hspace{5pt}\pfra{pousser quelque chose}\end{exemple}
\begin{exemple}\pnru{mi˩∼mi˧˥}\hspace{5pt}\peng{|fg{red}: to push and squeeze}\hspace{5pt}\pcmn{重叠:推、拥挤}\hspace{5pt}\pfra{|fg{red}}\end{exemple}
\begin{exemple}\pnru{mi˩∼mi˧-ɻ̍˥}\hspace{5pt}\peng{|fg{red} |fg{inceptive}}\hspace{5pt}\pcmn{重叠:推、拥挤}\hspace{5pt}\pfra{|fg{red} |fg{inchoatif}}\end{exemple}
\end{entrée}

\begin{entrée}
{mi˩˧}{}{ⓔmi˩˧}\formedesurface{mi˩˥}\newline
\classe{名词}\ton{LM}
\paradigme{\pcmn{:} \p{}}
\begin{définition}\peng{Female (animal).}\end{définition}
\begin{définition}\pcmn{母的(动物)}\end{définition}
\begin{définition}\pfra{Femelle (animal femelle).}\end{définition}
\begin{exemple}\pnru{ʈʂʰɯ˧, | mi˩˥! / ʈʂʰɯ˧, | mi˩ ɲi˥!}\hspace{5pt}\peng{It's a female!}\hspace{5pt}\pcmn{是母的!}\hspace{5pt}\pfra{C'est une femelle!}\end{exemple}
\end{entrée}

\begin{entrée}
{‑mi˩˧}{}{ⓔ‑mi˩˧}\newline
\classe{后缀}
\sens{1}\paradigme{\pcmn{:} \p{}}
\begin{définition}\peng{Feminine suffix.}\end{définition}
\begin{définition}\pcmn{阴性后缀}\end{définition}
\begin{définition}\pfra{Suffixe féminin.}\end{définition}\sens{2}\paradigme{\pcmn{:} \p{}}
\begin{définition}\peng{Augmentative suffix.}\end{définition}
\begin{définition}\pcmn{指大词}\end{définition}
\begin{définition}\pfra{Suffixe augmentatif.}\end{définition}
\end{entrée}

\begin{entrée}
{mi˩α}{}{ⓔmi˩α}\formedesurface{mi˩˥}\newline
\classe{动词}\ton{Lα}\begin{définition}\peng{To ask for.}\end{définition}
\begin{définition}\pcmn{请求、要,讨饭}\end{définition}
\begin{définition}\pfra{Demander, quémander.}\end{définition}
\begin{exemple}\pnru{hɑ˧ mi˥}\hspace{5pt}\peng{to beg (literally: ‘to ask for food')}\hspace{5pt}\pcmn{讨饭}\hspace{5pt}\pfra{mendier (littéralement: ‘demander à manger')}\end{exemple}
\begin{exemple}\pnru{hɑ˧ | ɖɯ˧-mi˧∼mi˥-ɻ̍˩}\hspace{5pt}\peng{to beg a little, to ask around for some food}\hspace{5pt}\pcmn{讨点饭}\hspace{5pt}\pfra{mendier un peu}\end{exemple}
\end{entrée}

\begin{entrée}
{mi˩β}{}{ⓔmi˩β}\formedesurface{ɖɯ˧ mi˩}\newline
\classe{量词}\ton{Lβ}\begin{définition}\peng{Classifier for small animals.}\end{définition}
\begin{définition}\pcmn{量词:小动物(一只)}\end{définition}
\begin{définition}\pfra{Classificateur des petits animaux (poules…).}\end{définition}
\begin{exemple}\pnru{ʈʂʰɯ˧-mi˧˥}\hspace{5pt}\peng{this animal}\hspace{5pt}\pcmn{这只}\hspace{5pt}\pfra{cet animal}\end{exemple}
\end{entrée}

\begin{entrée}
{mi˩hwɑ˧}{}{ⓔmi˩hwɑ˧}\formedesurface{mi˩hwɑ˥}\newline
\classe{名词}\ton{LM}\begin{définition}\peng{Cotton.}\end{définition}
\begin{définition}\pcmn{棉花(汉语借词)}\end{définition}
\begin{définition}\pfra{Coton.}\end{définition}
\begin{exemple}\pnru{mi˩hwɑ˧-bɑ˩lɑ˩}\hspace{5pt}\peng{cotton clothes}\hspace{5pt}\pcmn{棉布衣服}\hspace{5pt}\pfra{vêtement de coton}\end{exemple}
\end{entrée}

\begin{entrée}
{mi˧kʰwɤ\#˥}{}{ⓔmi˧kʰwɤ\#˥}\formedesurface{mi˧kʰwɤ˧}\newline
\classe{名词}
\sens{1}\paradigme{\pcmn{:} \p{}}
\begin{définition}\peng{Wound.}\end{définition}
\begin{définition}\pcmn{伤口}\end{définition}
\begin{définition}\pfra{Blessure, plaie.}\end{définition}\sens{2}
\begin{définition}\peng{Ulcer.}\end{définition}
\begin{définition}\pcmn{疮}\end{définition}
\begin{définition}\pfra{Ulcère.}\end{définition}
\end{entrée}

\begin{entrée}
{mi˩ɬi˩}{}{ⓔmi˩ɬi˩}\formedesurface{mi˩ɬi˩˥}\newline
\classe{名词}\ton{L}
\paradigme{\pcmn{:} \p{}}
\begin{définition}\peng{Large bamboo.}\end{définition}
\begin{définition}\pcmn{大竹子}\end{définition}
\begin{définition}\pfra{Grand bambou.}\end{définition}
\begin{exemple}\pnru{mi˩ɬi˩-bæ˩ʈʂo˥}\hspace{5pt}\peng{bamboo broom}\hspace{5pt}\pcmn{竹扫帚}\hspace{5pt}\pfra{balai en petites tiges de bambou}\end{exemple}
\begin{exemple}\pnru{mi˩ɬi˩-ʈʂæ˥do˩}\hspace{5pt}\peng{bamboo bucket to carry water (on one's back)}\hspace{5pt}\pcmn{竹桶,用来背水}\hspace{5pt}\pfra{seau en bambou pour porter de l'eau (sur le dos)}\end{exemple}
\end{entrée}

\begin{entrée}
{mi˩ɬi˩-ʁo˩bv̩˥}{}{ⓔmi˩ɬi˩-ʁo˩bv̩˥}\formedesurface{mi˩ɬi˩ʁo˩bv̩˥}\newline
\classe{名词}\ton{L+H\#}
\paradigme{\pcmn{:} \p{}}
\begin{définition}\peng{Bamboo shoot.}\end{définition}
\begin{définition}\pcmn{竹笋}\end{définition}
\begin{définition}\pfra{Pousse de bambou.}\end{définition}
\end{entrée}

\begin{entrée}
{mi˧ɬo\#˥}{}{ⓔmi˧ɬo\#˥}\formedesurface{mi˩ɬo˥}\newline
\classe{名词}\ton{\#H}
\paradigme{\pcmn{:} \p{}}
\begin{définition}\peng{Prayer.}\end{définition}
\begin{définition}\pcmn{祈祷}\end{définition}
\begin{définition}\pfra{Prière.}\end{définition}
\begin{exemple}\pnru{mi˧ɬo˧ lɑ˩}\hspace{5pt}\peng{to pray}\hspace{5pt}\pcmn{祈祷}\hspace{5pt}\pfra{prier}\end{exemple}
\end{entrée}

\begin{entrée}
{mi˧mi˧}{}{ⓔmi˧mi˧}\formedesurface{mi˧mi˧}\newline
\classe{名词}\ton{M}\begin{définition}\peng{Kernel (of a seed).}\end{définition}
\begin{définition}\pcmn{核,仁}\end{définition}
\begin{définition}\pfra{Amande (d'un noyau).}\end{définition}
\end{entrée}

\begin{entrée}
{mi˩mo˩}{}{ⓔmi˩mo˩}\formedesurface{mi˩mo˩˥}\newline
\classe{名词}\ton{L}
\paradigme{\pcmn{:} \p{}}
\begin{définition}\peng{Small sifter.}\end{définition}
\begin{définition}\pcmn{小筛子}\end{définition}
\begin{définition}\pfra{Petit crible.}\end{définition}
\end{entrée}

\begin{entrée}
{mi˧pɤ\#˥}{}{ⓔmi˧pɤ\#˥}\formedesurface{mi˧pɤ˧}\newline
\classe{名词}\ton{\#H}
\paradigme{\pcmn{:} \p{}}
\begin{définition}\peng{Scar.}\end{définition}
\begin{définition}\pcmn{疤}\end{définition}
\begin{définition}\pfra{Cicatrice.}\end{définition}
\end{entrée}

\begin{entrée}
{mi˩pʰv̩˩}{}{ⓔmi˩pʰv̩˩}\formedesurface{mi˩pʰv̩˩˥}\newline
\classe{名词}\ton{L}
\paradigme{\pcmn{:} \p{}}
\begin{définition}\peng{Nettle.}\end{définition}
\begin{définition}\pcmn{荨麻}\end{définition}
\begin{définition}\pfra{Ortie.}\end{définition}
\end{entrée}

\begin{entrée}
{mi˧tʰv̩\#˥}{}{ⓔmi˧tʰv̩\#˥}\formedesurface{mi˧tʰv̩˧}\newline
\classe{名词}\ton{\#H}
\paradigme{\pcmn{:} \p{}}
\begin{définition}\peng{Walking-stick.}\end{définition}
\begin{définition}\pcmn{拐棍}\end{définition}
\begin{définition}\pfra{Bâton, canne pour marcher.}\end{définition}
\end{entrée}

\begin{entrée}
{mi˩zɯ˩}{}{ⓔmi˩zɯ˩}\formedesurface{mi˩zɯ˩˥}\newline
\classe{名词}\ton{L}
\paradigme{\pcmn{:} \p{}}
\begin{définition}\peng{Woman; also the name of the second pillar in the main room (the feminine pillar).}\end{définition}
\begin{définition}\pcmn{女人。主屋的第二个柱子(代表女性的那个柱子)也是用这个名字。}\end{définition}
\begin{définition}\pfra{Femme; aussi: nom du deuxième pilier de la maison (le pilier féminin).}\end{définition}
\end{entrée}

\begin{entrée}
{mje˧˥}{}{ⓔmje˧˥}\formedesurface{mje˧˥}\newline
\classe{名词}\ton{MH}\begin{définition}\peng{Noodles.}\end{définition}
\begin{définition}\pcmn{面条}\end{définition}
\begin{définition}\pfra{Nouilles, pâtes alimentaires.}\end{définition}
\begin{exemple}\pnru{mjæ˧˥ | dzɯ˧-bi˧! ∼ mjæ˧ dzɯ˧-bi˧! ∼ mjæ˧ dzɯ˥-bi˩!}\hspace{5pt}\peng{Let's eat noodles!}\hspace{5pt}\pcmn{吃面吧!}\hspace{5pt}\pfra{On va manger des nouilles!}\end{exemple}
\begin{exemple}\pnru{mjæ˧˥ | ɖɯ˧-qʰwɤ˧ tɕɤ˥}\hspace{5pt}\peng{to boil a bowl of noodles, to cook a bowl of noodles}\hspace{5pt}\pcmn{煮一碗面}\hspace{5pt}\pfra{faire cuire un bol de nouilles}\end{exemple}
\begin{exemple}\pnru{mjæ˧ hwæ˥-bi˩}\hspace{5pt}\peng{(we) will buy noodles}\hspace{5pt}\pcmn{要买面}\hspace{5pt}\pfra{(on) va acheter des nouilles}\end{exemple}
\end{entrée}

\begin{entrée}
{mo˥α}{}{ⓔmo˥α}\formedesurface{ɖɯ˧ mo˥}\newline
\classe{量词}\ton{Hα}\begin{définition}\peng{One Chinese acre, amounting to one-sixth of an acre.}\end{définition}
\begin{définition}\pcmn{量词:地(一亩地)(汉语借词)}\end{définition}
\begin{définition}\pfra{Acre chinois: un sixième d'acre; 0,0667 hectare.}\end{définition}
\end{entrée}

\begin{entrée}
{mo˧}{}{ⓔmo˧}\formedesurface{mo˧}\newline
\classe{形容词}\ton{M}\begin{définition}\peng{Greedy, covetous.}\end{définition}
\begin{définition}\pcmn{贪心不足,贪吃、贪喝酒、贪抽烟}\end{définition}
\begin{définition}\pfra{Avide, glouton. S'applique à la nourriture, mais aussi à l'alcool, au tabac…}\end{définition}
\begin{exemple}\pnru{hɑ˧ mo˧}\hspace{5pt}\peng{as above: greedy; refers specifically to food}\hspace{5pt}\pcmn{贪吃}\hspace{5pt}\pfra{avide de nourriture}\end{exemple}
\begin{exemple}\pnru{hɑ˧ mo˧ | ʐwæ˩˥}\hspace{5pt}\peng{extremely greedy for food}\hspace{5pt}\pcmn{很贪吃}\hspace{5pt}\pfra{très avide de nourriture}\end{exemple}
\begin{exemple}\pnru{mɤ˧-hɑ˧mo˧!}\hspace{5pt}\peng{|fg{neg}: not greedy}\hspace{5pt}\pcmn{不贪吃}\hspace{5pt}\pfra{|fg{neg}: pas avide}\end{exemple}
\begin{exemple}\pnru{hɑ˧ | mɤ˧-mo˧!}\hspace{5pt}\peng{|fg{neg}: not greedy}\hspace{5pt}\pcmn{不贪吃}\hspace{5pt}\pfra{|fg{neg}: pas avide}\end{exemple}
\end{entrée}

\begin{entrée}
{mo˧˥}{}{ⓔmo˧˥}\formedesurface{mo˧˥}\newline
\classe{名词}\ton{MH}
\paradigme{\pcmn{:} \p{}}
\begin{définition}\peng{Mushroom.}\end{définition}
\begin{définition}\pcmn{菌子、蘑菇}\end{définition}
\begin{définition}\pfra{Champignon.}\end{définition}
\end{entrée}

\begin{entrée}
{mo˧˥α}{}{ⓔmo˧˥α}\formedesurface{ɖɯ˧ mo˧˥}\newline
\classe{量词}\ton{MHα}\begin{définition}\peng{Self-classifier for mushrooms.}\end{définition}
\begin{définition}\pcmn{量词:蘑菇(一只)}\end{définition}
\begin{définition}\pfra{Auto-classificateur des champignons.}\end{définition}
\end{entrée}

\begin{entrée}
{mo˧α}{}{ⓔmo˧α}\formedesurface{ɖɯ˧ mo˧}\newline
\classe{量词}\ton{Mα}\begin{définition}\peng{Classifier for corpses.}\end{définition}
\begin{définition}\pcmn{量词:尸体}\end{définition}
\begin{définition}\pfra{Classificateur des cadavres et tombeaux.}\end{définition}
\end{entrée}

\begin{entrée}
{mo˩}{}{ⓔmo˩}\formedesurface{--}\newline
\classe{语气助词}\ton{L}\begin{définition}\peng{Final particle indicating invitation/suggestion to do something.}\end{définition}
\begin{définition}\pcmn{句尾助词:请……}\end{définition}
\begin{définition}\pfra{Particule indiquant l'invitation à faire quelque chose.}\end{définition}
\begin{exemple}\pnru{no˧ | ɖɯ˧-ʈʰɯ˩-ɻ̍˩ mo˩!}\hspace{5pt}\peng{Please drink a little! / Do have a sip!}\hspace{5pt}\pcmn{请你喝一点!}\hspace{5pt}\pfra{Bois donc un peu!}\end{exemple}
\end{entrée}

\begin{entrée}
{mo˩α}{₁}{ⓔmo˩αⓗ1}\formedesurface{mo˩˥}\newline
\classe{形容词}\ton{Lα}
1\begin{définition}\peng{Old.}\end{définition}
\begin{définition}\pcmn{年老}\end{définition}
\begin{définition}\pfra{Vieux, âgé.}\end{définition}
\begin{exemple}\pnru{mo˩ hĩ˩˥}\hspace{5pt}\peng{old person}\hspace{5pt}\pcmn{老人}\hspace{5pt}\pfra{vieille personne}\end{exemple}
\begin{exemple}\pnru{si˧ mo˥}\hspace{5pt}\peng{old wood; old tree}\hspace{5pt}\pcmn{老树、老木头}\hspace{5pt}\pfra{vieux bois, vieil arbre}\end{exemple}
\begin{exemple}\pnru{le˧-mo˩-ze˩}\hspace{5pt}\peng{|fg{accomp} \_ |fg{pfv}: (he/she) has become old / has aged.}\hspace{5pt}\pcmn{|fg{accomp} \_ |fg{pfv}}\hspace{5pt}\pfra{|fg{accomp} \_ |fg{pfv}: (il/elle) a vieilli}\end{exemple}
\begin{exemple}\pnru{le˧-mo˩-hĩ˩}\hspace{5pt}\peng{Old person, person who has become old}\hspace{5pt}\pcmn{老了的人}\hspace{5pt}\pfra{vieille personne, personne qui a vieilli}\end{exemple}
\begin{exemple}\pnru{hĩ˧ mo˥, | õ˧-di˧ fv̩˥! | ʐwæ˧ mo˥, | to˩ do˩ ɖwæ˥!}\hspace{5pt}\peng{Old folk like their own place (=their own home); old horses are afraid to climb slopes! (Proverb.)}\hspace{5pt}\pcmn{老人爱自家,老马怕山坡!(谚语,描写不爱到处跑的老年人)}\hspace{5pt}\pfra{«Les vieilles personnes aiment leur chez-eux; les vieux chevaux ont peur de grimper les pentes!» (Sens: avec l'âge, on devient moins entreprenant.)}\end{exemple}
\begin{exemple}\pnru{lv̩˧ mo˥ F | dʑɯ˧ | le˧-qv̩˩; | si˧ mo˥ F | le˧-dze˩-kv̩˩! | no˧ F | ə˧tse˧ | le˧-ʂɯ˧-mɤ˧-tʰɑ˧˥ | di˩!}\hspace{5pt}\peng{Old stones are carried away by the stream; and old wood gets chopped down! And you, why can't you die? (Mocking an elderly person. Na tradition assigns man a lifespan of sixty years; people getting past seventy are considered to have reached the end of their lifespan.)}\hspace{5pt}\pcmn{老石头要被河流冲走,老木头要被砍掉。你呢,怎么还不死? (嘲笑一个年龄很高的人。摩梭传统中,人的寿命是六十岁:过了七十岁的人,被认为是已经到达了命的尽头。)}\hspace{5pt}\pfra{Les vieilles pierres, le courant les emporte; le vieux bois, on le coupe! Alors pourquoi toi te voilà qui ne veux pas mourir! (Moquerie à l'égard d'une personne très âgée.)}\end{exemple}
\end{entrée}

\begin{entrée}
{mo˩α}{₂}{ⓔmo˩αⓗ2}\formedesurface{mo˩˥}\newline
\classe{动词}\ton{Lα}
2\begin{définition}\peng{To die.}\end{définition}
\begin{définition}\pcmn{死、去世}\end{définition}
\begin{définition}\pfra{Mourir.}\end{définition}
\begin{exemple}\pnru{mɤ˧-mo˩-sɯ˩!}\hspace{5pt}\peng{(She/he/it) is not dead yet!}\hspace{5pt}\pcmn{还没死!}\hspace{5pt}\pfra{(Il n'est) pas encore mort!}\end{exemple}
\begin{exemple}\pnru{si˧ mo˩}\hspace{5pt}\peng{dead wood}\hspace{5pt}\pcmn{老干柴(直译:死了的木头)}\hspace{5pt}\pfra{bois mort}\end{exemple}
\end{entrée}

\begin{entrée}
{mo˧ɖʐv̩˥}{}{ⓔmo˧ɖʐv̩˥}\formedesurface{mo˧ɖʐv̩˥}\newline
\classe{名词}\ton{H\#}\begin{définition}\peng{Morel, hickory chick: an edible mushroom.}\end{définition}
\begin{définition}\pcmn{羊肚菌}\end{définition}
\begin{définition}\pfra{Morille: champignon comestible, particulièrement apprécié pour sa texture.}\end{définition}
\begin{exemple}\pnru{ʂɯ˧-ɬi˧mi˧, | mo˧ɖʐv̩˥!}\hspace{5pt}\peng{The seventh month is the season of morels!}\hspace{5pt}\pcmn{七月份,是羊肚菌的季节!}\hspace{5pt}\pfra{Le septième mois, c'est la saison des morilles! (cette sorte de champignon) (Il pousse des paquets de champignons si compacts qu'on n'arrive même pas à les séparer.)}\end{exemple}
\begin{exemple}\pnru{ʂɯ˧-ɬi˧mi˧ | mo˧ɖʐv̩˥-ne˩-ʝi˩-zo˩!}\hspace{5pt}\peng{‘[They have kids] like (=as numerous as) morels in the seventh month!', i.e. they have children in great abundance. This is a humorous comment made about people who had one child after the other. The abundance of morels in the seventh month is spectacular and proverbial.}\hspace{5pt}\pcmn{(你们家孩子)生得像七月份的羊肚菌一样!(来形容一家有很多孩子出生,一个又一个。在永宁地区,七月份羊肚菌很多。)}\hspace{5pt}\pfra{«Il vous en vient comme des morilles au septième mois!» Commentaire humoristique: ce qu'on disait au sujet des gens qui avaient beaucoup d'enfants, qui avaient enfant après enfant: «Ca prolifère comme les morilles au septième mois!»}\end{exemple}
\end{entrée}

\begin{entrée}
{mo˧jo˩-mi˩}{}{ⓔmo˧jo˩-mi˩}\formedesurface{mo˧jo˩mi˩}\newline
\classe{名词}\ton{L\#-}
\paradigme{\pcmn{:} \p{}}
\begin{définition}\peng{Owl.}\end{définition}
\begin{définition}\pcmn{猫头鹰}\end{définition}
\begin{définition}\pfra{Chouette, hibou (toutes les espèces de |\stylefi{bubo} et |\stylefi{strix}).}\end{définition}
\end{entrée}

\begin{entrée}
{mo˧jo˩mi˩-pʰv̩˩}{}{ⓔmo˧jo˩mi˩-pʰv̩˩}\formedesurface{mo˧jo˩mi˩pʰv̩˩}\newline
\classe{名词}\ton{L\#-}
\paradigme{\pcmn{:} \p{}}
\begin{définition}\peng{Male owl.}\end{définition}
\begin{définition}\pcmn{公猫头鹰}\end{définition}
\begin{définition}\pfra{Hibou mâle.}\end{définition}
\end{entrée}

\begin{entrée}
{mo˧jo˩mi˩-zo˩}{}{ⓔmo˧jo˩mi˩-zo˩}\formedesurface{mo˧jo˩mi˩zo˩}\newline
\classe{名词}\ton{L\#-}
\paradigme{\pcmn{:} \p{}}
\begin{définition}\peng{Baby owl, owlet.}\end{définition}
\begin{définition}\pcmn{小的猫头鹰}\end{définition}
\begin{définition}\pfra{Bébé hibou.}\end{définition}
\end{entrée}

\begin{entrée}
{mo˧kɤ˩}{}{ⓔmo˧kɤ˩}\formedesurface{mo˧kɤ˩}\newline
\classe{名词}\ton{L\#}\begin{définition}\peng{Azalea.}\end{définition}
\begin{définition}\pcmn{杜鹃花、踯躅、山石榴、照山红、唐杜鹃}\end{définition}
\begin{définition}\pfra{Azalée. Cette plante est perçue comme vénéneuse; on ne consomme pas les champignons qui poussent dans son voisinage.}\end{définition}
\begin{exemple}\pnru{mo˧kɤ˩-bæ˩bæ˩}\hspace{5pt}\peng{azalea flowers}\hspace{5pt}\pcmn{杜鹃花}\hspace{5pt}\pfra{fleurs d'azalée}\end{exemple}
\end{entrée}

\begin{entrée}
{mo˩kv̩\#˥}{}{ⓔmo˩kv̩\#˥}\formedesurface{mo˩kv̩˥}\newline
\classe{名词}\ton{LM+\#H}
\paradigme{\pcmn{:} \p{}}
\begin{définition}\peng{Mushrooms that grows on fallen trunks, e.g. oaks.}\end{définition}
\begin{définition}\pcmn{蘑菇:长在倒在地上的树(如青冈等树木)上的菌子(汉语借词)}\end{définition}
\begin{définition}\pfra{Sorte de champignon qui pousse sur les chênes (sur les troncs tombés, sur les arbres morts).}\end{définition}
\begin{exemple}\pnru{mo˩kv̩˥, | si˧dzi˩-mo˩!}\hspace{5pt}\peng{/mo˩kv̩\#˥/ refers to mushrooms that grow on trees!}\hspace{5pt}\pcmn{|fv{/mo˩kv̩\#˥/},指的是长在(倒在地上的)树上的菌子!}\hspace{5pt}\pfra{/mo˩kv̩\#˥/, ça désigne les champignons qui pousse sur les arbres! (littéralement: «les champignons d'arbres», par opposition aux «champignons de terre»)}\end{exemple}
\end{entrée}

\begin{entrée}
{mo˧ɬɑ˥}{}{ⓔmo˧ɬɑ˥}\formedesurface{mo˧ɬɑ˥}\newline
\classe{名词}\ton{H\#}\begin{définition}\peng{Slander.}\end{définition}
\begin{définition}\pcmn{诬蔑、坏话}\end{définition}
\begin{définition}\pfra{Médisance, calomnie.}\end{définition}
\begin{exemple}\pnru{mo˧ɬɑ˥ ʐwɤ˩}\hspace{5pt}\peng{to slander, to speak ill of others}\hspace{5pt}\pcmn{说人的坏话}\hspace{5pt}\pfra{médire de quelqu'un, calomnier quelqu'un}\end{exemple}
\end{entrée}

\begin{entrée}
{mo˧mo˥}{}{ⓔmo˧mo˥}\formedesurface{mo˧mo˥}\newline
\classe{名词}\ton{H\#}
\paradigme{\pcmn{:} \p{}}
\begin{définition}\peng{Steamed bun.}\end{définition}
\begin{définition}\pcmn{馒头、包子}\end{définition}
\begin{définition}\pfra{Petits pains (pouvant contenir de la farine de maïs; mais surtout farine de blé) cuits à la vapeur.}\end{définition}
\end{entrée}

\begin{entrée}
{mo˧nɑ˥}{₁}{ⓔmo˧nɑ˥ⓗ1}\formedesurface{mo˧nɑ˥}\newline
\classe{名词}\ton{H\#}
1\begin{définition}\peng{Gossip.}\end{définition}
\begin{définition}\pcmn{闲话}\end{définition}
\begin{définition}\pfra{Médisance, ragot.}\end{définition}
\begin{exemple}\pnru{mo˧nɑ˥ ʐwɤ˩}\hspace{5pt}\peng{to indulge in gossip, to speak badly of others}\hspace{5pt}\pcmn{八卦、讲别人的坏话}\hspace{5pt}\pfra{ragoter, médire}\end{exemple}
\begin{exemple}\pnru{ʈʂʰɯ˧ | ɖɯ˧-ɲi˥ | mo˧nɑ˥ ʐwɤ˩-dʑo˩!}\hspace{5pt}\peng{(S)he gossips all day!}\hspace{5pt}\pcmn{他一天到晚都在八卦!}\hspace{5pt}\pfra{Il/elle ragote toute la journée!}\end{exemple}
\begin{exemple}\pnru{mo˧nɑ˥-ɕi˩mi˩}\hspace{5pt}\peng{same meaning: gossip}\hspace{5pt}\pcmn{同上:八卦、坏话}\hspace{5pt}\pfra{même sens: ragot, médisance}\end{exemple}
\begin{exemple}\pnru{mo˧nɑ˥-ɕi˩mi˩ ʐwɤ˩}\hspace{5pt}\peng{to indulge in gossip, to speak badly of others}\hspace{5pt}\pcmn{八卦、讲别人的坏话}\hspace{5pt}\pfra{ragoter, médire}\end{exemple}
\begin{exemple}\pnru{hĩ˧ | ʈʂʰɯ˧-v̩˧, | mo˧nɑ˥-ɕi˩mi˩ | ɖɯ˧-v̩˧ ɲi˩!}\hspace{5pt}\peng{He's a gossiper, he talks badly of other people}\hspace{5pt}\pcmn{他爱八卦、爱说别人坏话}\hspace{5pt}\pfra{C'est un ragoteur, il est médisant}\end{exemple}
\end{entrée}

\begin{entrée}
{mo˧nɑ˥}{₂}{ⓔmo˧nɑ˥ⓗ2}\formedesurface{mo˧nɑ˥}\newline
\classe{名词}\ton{H\#}
2\begin{définition}\peng{Chopped straw, used when preparing pickled vegetables.}\end{définition}
\begin{définition}\pcmn{剁成粉的秸杆}\end{définition}
\begin{définition}\pfra{Paille hachée, utilisée dans la préparation des légumes en saumure.}\end{définition}
\begin{exemple}\pnru{mv˩zɯ˩-mo˩nɑ˥}\hspace{5pt}\peng{chopped oat straw}\hspace{5pt}\pcmn{剁成粉的燕麦秸杆}\hspace{5pt}\pfra{paille d'avoine hachée}\end{exemple}
\end{entrée}

\begin{entrée}
{mo˧qʰwɤ˥}{}{ⓔmo˧qʰwɤ˥}\formedesurface{mo˧qʰwɤ˥}\newline
\classe{名词}\ton{H\#}
\paradigme{\pcmn{:} \p{}}
\begin{définition}\peng{Wooden shuttle of zip line (flying fox): it glides along the rope; the passenger, horse, or load of goods is tied to the shuttle.}\end{définition}
\begin{définition}\pcmn{溜索上往返移动的木头梭}\end{définition}
\begin{définition}\pfra{Navette en bois d'un pont de corde: la navette coulisse sur la corde; passager, cheval ou chargement y sont attachés.}\end{définition}
\end{entrée}

\begin{entrée}
{mo˧qʰwɤ˧˥}{}{ⓔmo˧qʰwɤ˧˥}\formedesurface{mo˧qʰwɤ˧˥}\newline
\classe{形容词}\ton{MH}\begin{définition}\peng{Fond of food; voracious (can range from neutral uses to clearly negative uses).}\end{définition}
\begin{définition}\pcmn{胃口好,或:贪吃}\end{définition}
\begin{définition}\pfra{Qui a bon appétit, qui a un solide appétit; gourmand, vorace (peut être neutre, ou franchement négatif).}\end{définition}
\end{entrée}

\begin{entrée}
{mo˩ɻ̍\#˥}{}{ⓔmo˩ɻ̍\#˥}\formedesurface{mo˩ɻ̍˥}\newline
\classe{名词}\ton{LM+\#H}
\paradigme{\pcmn{:} \p{}}
\begin{définition}\peng{Black mushroom, ‘wood ear' (an edible fungus).}\end{définition}
\begin{définition}\pcmn{木耳(汉语借词)}\end{définition}
\begin{définition}\pfra{Champignon noir.}\end{définition}
\end{entrée}

\begin{entrée}
{mo˩zo\#˥}{}{ⓔmo˩zo\#˥}\formedesurface{mo˩zo˥}\newline
\classe{名词}\ton{LM+\#H}
\paradigme{\pcmn{:} \p{}}
\begin{définition}\peng{Soldier.}\end{définition}
\begin{définition}\pcmn{士兵}\end{définition}
\begin{définition}\pfra{Militaire, soldat.}\end{définition}
\begin{exemple}\pnru{mo˩zo˧ ʝi˧-hɯ˧ ɲi˥!}\hspace{5pt}\peng{He went to the army! / He joined the army! / He became a soldier!}\hspace{5pt}\pcmn{当兵去了!}\hspace{5pt}\pfra{Il est parti à l'armée! / Il s'est fait soldat!}\end{exemple}
\end{entrée}

\begin{entrée}
{mv̩˥}{}{ⓔmv̩˥}\newline
\classe{动词}
\sens{1}
\begin{définition}\peng{To hear.}\end{définition}
\begin{définition}\pcmn{懂,听见}\end{définition}
\begin{définition}\pfra{Entendre.}\end{définition}
\begin{exemple}\pnru{njɤ˧ | le˧-mv̩˥-ze˩}\hspace{5pt}\peng{I have heard}\hspace{5pt}\pcmn{我听见了}\hspace{5pt}\pfra{j'ai entendu}\end{exemple}\sens{2}
\begin{définition}\peng{To understand.}\end{définition}
\begin{définition}\pcmn{懂}\end{définition}
\begin{définition}\pfra{Comprendre.}\end{définition}
\begin{exemple}\pnru{njɤ˧ | le˧-mv̩˥-ze˩}\hspace{5pt}\peng{I have understood}\hspace{5pt}\pcmn{我懂了}\hspace{5pt}\pfra{j'ai compris}\end{exemple}
\end{entrée}

\begin{entrée}
{mv̩˥}{₁}{ⓔmv̩˥ⓗ1}\formedesurface{mv̩˧}\newline
\classe{名词}\ton{\#H}
1
\paradigme{\pcmn{:} \p{}}
\begin{définition}\peng{Sky.}\end{définition}
\begin{définition}\pcmn{天}\end{définition}
\begin{définition}\pfra{Ciel.}\end{définition}
\begin{exemple}\pnru{mv̩˧tʰv̩˧(-ze˩)}\hspace{5pt}\peng{the day is bright, the sky is clear}\hspace{5pt}\pcmn{天晴,天色亮}\hspace{5pt}\pfra{il fait clair, il fait grand jour, le ciel est clair}\end{exemple}
\begin{exemple}\pnru{hĩ˧-ɳɯ˩ mɤ˩-do˩, | mv̩˧-ɳɯ˩ | do˩˥!}\hspace{5pt}\peng{“What humans do not see, the Heavens see it!" (Meaning: a good deed earns one happiness in future; and a count of bad deeds, even those that go unseen by humans, is also kept in the Heavens.)}\hspace{5pt}\pcmn{“人看不见的,老天能看见!”}\hspace{5pt}\pfra{«Ce que les hommes ne voient pas, le ciel le voit!» (Sens: une bonne action n'est jamais perdue, et une mauvaise reçoit sa punition dans le monde d'en haut.)}\end{exemple}
\begin{exemple}\pnru{tsʰi˧ɲi˧, | mv̩˧ dʑɤ˥!}\hspace{5pt}\peng{Today, the weather is fair! / The weather is fine today!}\hspace{5pt}\pcmn{今天天气好!}\hspace{5pt}\pfra{Aujourd'hui, il fait beau!}\end{exemple}
\end{entrée}

\begin{entrée}
{mv̩˥}{₂}{ⓔmv̩˥ⓗ2}\formedesurface{mv̩˧}\newline
\classe{名词}\ton{\#H}
2
\paradigme{\pcmn{:} \p{}}
\begin{définition}\peng{Fire.}\end{définition}
\begin{définition}\pcmn{火}\end{définition}
\begin{définition}\pfra{Feu.}\end{définition}
\begin{exemple}\pnru{mv̩˧ kʰɯ˩}\hspace{5pt}\peng{to light a fire, to do a fire}\hspace{5pt}\pcmn{点火}\hspace{5pt}\pfra{allumer un feu, faire un feu}\end{exemple}
\end{entrée}

\begin{entrée}
{mv̩˧}{}{ⓔmv̩˧}\formedesurface{mv̩˧}\newline
\classe{名词}\ton{M}\begin{définition}\peng{Name (given name or family name).}\end{définition}
\begin{définition}\pcmn{姓名}\end{définition}
\begin{définition}\pfra{Nom (nom de famille ou prénom: nom donné à un individu).}\end{définition}
\begin{exemple}\pnru{ɑ˩ʁo˧-bv̩˧ | mv̩˧ (+ɲi˩)}\hspace{5pt}\peng{This is the family name! / This is my family name!}\hspace{5pt}\pcmn{这是家里的姓! / 这是我家的姓!}\hspace{5pt}\pfra{c'est le nom de la famille / c'est mon nom de famille!}\end{exemple}
\begin{exemple}\pnru{njɤ˧ | mv̩˧ ɖɯ˧-kʰwɤ˥ | ʂe˧-zo˧-ho˩!}\hspace{5pt}\peng{I have to go and get a name (from the monks at the monastery) (for a newborn child)}\hspace{5pt}\pcmn{我得去(向大寺里的和尚)求一个名字(给刚出生的孩子起名)}\hspace{5pt}\pfra{Il va falloir que j'aille chercher un nom (auprès des moines du monastère) (pour un enfant qui vient de naître)!}\end{exemple}
\end{entrée}

\begin{entrée}
{mv̩˧}{}{ⓔmv̩˧}\formedesurface{mv̩˧}\newline
\classe{语气助词}\ton{M}\begin{définition}\peng{Affirmative final particle.}\end{définition}
\begin{définition}\pcmn{句尾助词,表示肯定:“嘛”}\end{définition}
\begin{définition}\pfra{Particule finale affirmative.}\end{définition}
\end{entrée}

\begin{entrée}
{mv̩˧‑}{}{ⓔmv̩˧‑}\formedesurface{mv̩˧}\newline
\classe{前缀}\ton{M}\begin{définition}\peng{Aspect/mood: the event is about to take place: the event is imminent.}\end{définition}
\begin{définition}\pcmn{即将、快要、马上会、立即}\end{définition}
\begin{définition}\pfra{Aspect/mode: l'événement est imminent: sur le point de se produire.}\end{définition}
\begin{exemple}\pnru{ʈʂʰɯ˧ | mv̩˧-dzɯ˧-kwɤ˩-tɕɯ˩!}\hspace{5pt}\peng{Come on, eat it up! / Come on, finish your bowl!}\hspace{5pt}\pcmn{你吃完吧!}\hspace{5pt}\pfra{Mange-le donc! / Finis donc ça! (Contexte: à table, quelqu'un ne finit pas son bol; sa mère ou grand-mère lui enjoint de finir, pour ne pas gaspiller de nourriture.)}\end{exemple}
\begin{exemple}\pnru{tʰi˧-mv̩˧-dzɯ˧-kwɤ˩-tɕɯ˩!}\hspace{5pt}\peng{Same as previous example, with the |fg{durative}}\hspace{5pt}\pcmn{同上}\hspace{5pt}\pfra{Comme l'exemple précédent, avec le |fg{duratif}}\end{exemple}
\begin{exemple}\pnru{ʈʂʰɯ˧ mv̩˧-ʂɯ˧ bi˩-ni˩gv̩˩! njɤ˧ | gv̩˩dʑɯ˩˥ | ʐwæ˩˥! |}\hspace{5pt}\peng{(S)he is going to die! I am devastated!}\hspace{5pt}\pcmn{他要死了!我很伤心!}\hspace{5pt}\pfra{Il/elle va mourir! Je suis au désespoir!}\end{exemple}
\begin{exemple}\pnru{hĩ˧ ʈʂʰɯ˧-v̩˧ tʰv̩˧ mv̩˧-ʂɯ˧-kwɤ˧tɕɯ˥-lɑ˩…}\hspace{5pt}\peng{as this person is going to die soon…}\hspace{5pt}\pcmn{因为这个人快要去世……}\hspace{5pt}\pfra{du fait que cette personne va mourir très bientôt…}\end{exemple}
\begin{exemple}\pnru{mv̩˧-dzɯ˧-bi˩-ze˩!}\hspace{5pt}\peng{[We are] about to eat! / We are going to eat right now!}\hspace{5pt}\pcmn{马上要吃了!}\hspace{5pt}\pfra{[On] va manger tout de suite!}\end{exemple}
\begin{exemple}\pnru{mv̩˧-hwæ˧}\hspace{5pt}\peng{about to buy}\hspace{5pt}\pcmn{即将买}\hspace{5pt}\pfra{sur le point d'acheter}\end{exemple}
\begin{exemple}\pnru{mv̩˧-tɕʰi˧}\hspace{5pt}\peng{about to sell}\hspace{5pt}\pcmn{即将卖}\hspace{5pt}\pfra{sur le point de vendre}\end{exemple}
\begin{exemple}\pnru{mv̩˧-dzɯ˧-kwɤ˧tɕɯ˥-lɑ˩…}\hspace{5pt}\peng{since (she/he) is about to eat…}\hspace{5pt}\pcmn{因为马上要吃……}\hspace{5pt}\pfra{puisqu'elle/il est sur le point de manger…}\end{exemple}
\begin{exemple}\pnru{mv̩˧-lɑ˩-kwɤ˩tɕɯ˩-lɑ˩…}\hspace{5pt}\peng{since (she/he) is about to strike…}\hspace{5pt}\pcmn{因为要打……}\hspace{5pt}\pfra{puisqu'elle/il est sur le point de frapper…}\end{exemple}
\end{entrée}

\begin{entrée}
{mv̩˧α}{}{ⓔmv̩˧α}\formedesurface{mv̩˧}\newline
\classe{动词}\ton{Mα}\begin{définition}\peng{To put on (a shirt, a jacket).}\end{définition}
\begin{définition}\pcmn{穿(衣服、上衣)}\end{définition}
\begin{définition}\pfra{Mettre, porter, enfiler, endosser (une chemise, une veste); se vêtir d'un habit.}\end{définition}
\begin{exemple}\pnru{bɑ˩lɑ˩ mv̩˥}\hspace{5pt}\peng{to put on a shirt/jacket}\hspace{5pt}\pcmn{穿衣服}\hspace{5pt}\pfra{mettre une chemise/veste}\end{exemple}
\begin{exemple}\pnru{bɑ˩lɑ˩˥ | tʰi˧-mv̩˧}\hspace{5pt}\peng{to put on a shirt/jacket}\hspace{5pt}\pcmn{穿衣服}\hspace{5pt}\pfra{mettre une chemise/veste}\end{exemple}
\begin{exemple}\pnru{dʑi˧hṽ̩˧ mv̩˩}\hspace{5pt}\peng{to put on clothes}\hspace{5pt}\pcmn{穿衣服}\hspace{5pt}\pfra{enfiler un habit}\end{exemple}
\end{entrée}

\begin{entrée}
{mv̩˩˥}{}{ⓔmv̩˩˥}\formedesurface{mv̩˩˥}\newline
\classe{名词}\ton{LH}
\paradigme{\pcmn{:} \p{}}
\begin{définition}\peng{Daughter.}\end{définition}
\begin{définition}\pcmn{女儿}\end{définition}
\begin{définition}\pfra{Fille.}\end{définition}
\end{entrée}

\begin{entrée}
{mv̩˩α}{₁}{ⓔmv̩˩αⓗ1}\formedesurface{mv̩˩˥}\newline
\classe{动词}\ton{Lα}
1\begin{définition}\peng{To blow (e.g. to blow the fire, to blow a horn).}\end{définition}
\begin{définition}\pcmn{吹(灰,乐器)}\end{définition}
\begin{définition}\pfra{Souffler (ex.: souffler sur le feu, attiser le feu; souffler dans un instrument à vent).}\end{définition}
\begin{exemple}\pnru{mv̩˧∼mv̩˥(-ze˩)}\hspace{5pt}\peng{|fg{red}}\hspace{5pt}\pcmn{重叠:吹吹}\hspace{5pt}\pfra{|fg{red}}\end{exemple}
\begin{exemple}\pnru{ʝi˧qʰv̩˧ mv̩˥}\hspace{5pt}\peng{to blow a horn}\hspace{5pt}\pcmn{吹号角}\hspace{5pt}\pfra{souffler dans une corne}\end{exemple}
\end{entrée}

\begin{entrée}
{mv̩˩α}{₂}{ⓔmv̩˩αⓗ2}\formedesurface{mv̩˩˥}\newline
\classe{动词}\ton{Lα}
2\begin{définition}\peng{To wash away, to sweep away (of water), to carry away (of water): a strong current swept a swimmer away.}\end{définition}
\begin{définition}\pcmn{冲(走)}\end{définition}
\begin{définition}\pfra{Emporter (le courant emporte un nageur), balayer (une vague balaie une épave de bateau).}\end{définition}
\end{entrée}

\begin{entrée}
{mv̩˩α}{₃}{ⓔmv̩˩αⓗ3}\formedesurface{mv̩˩˥}\newline
\classe{形容词}
3
\sens{1}
\begin{définition}\peng{Ripe.}\end{définition}
\begin{définition}\pcmn{熟、成熟(植物、水果)}\end{définition}
\begin{définition}\pfra{Mûr (produit agricole).}\end{définition}
\begin{exemple}\pnru{mv̩˩-hĩ˩˥}\hspace{5pt}\peng{|fg{rel}}\hspace{5pt}\pcmn{熟的}\hspace{5pt}\pfra{|fg{rel}}\end{exemple}\sens{2}
\begin{définition}\peng{Cooked, well-cooked, done.}\end{définition}
\begin{définition}\pcmn{熟(食物)}\end{définition}
\begin{définition}\pfra{Cuit (aliment).}\end{définition}
\end{entrée}

\begin{entrée}
{mv̩˩α}{₄}{ⓔmv̩˩αⓗ4}\formedesurface{mv̩˩˥}\newline
\classe{动词}\ton{Lα}
4\begin{définition}\peng{To burn, to become consumed (e.g. a body on the funeral pyre becomes consumed).}\end{définition}
\begin{définition}\pcmn{燃烧}\end{définition}
\begin{définition}\pfra{Brûler, se consumer (ex.: un corps sur le bûcher).}\end{définition}
\end{entrée}

\begin{entrée}
{mv̩˩-bæ˧mi˩}{}{ⓔmv̩˩-bæ˧mi˩}\formedesurface{mv̩˩bæ˧mi˩}\newline
\classe{名词}\ton{L-L\#}
\paradigme{\pcmn{:} \p{}}
\begin{définition}\peng{Fool, idiot (female).}\end{définition}
\begin{définition}\pcmn{傻女人、笨女人}\end{définition}
\begin{définition}\pfra{Imbécile, idiote.}\end{définition}
\end{entrée}

\begin{entrée}
{mv̩˧bɤ\#˥}{}{ⓔmv̩˧bɤ\#˥}\formedesurface{mv̩˧bɤ˧}\newline
\classe{名词}\ton{\#H}
\paradigme{\pcmn{:} \p{}}
\begin{définition}\peng{Sole of the foot.}\end{définition}
\begin{définition}\pcmn{脚底}\end{définition}
\begin{définition}\pfra{Plante du pied.}\end{définition}
\end{entrée}

\begin{entrée}
{mv̩˧bv̩˧ʐv̩˥}{}{ⓔmv̩˧bv̩˧ʐv̩˥}\formedesurface{mv̩˧bv̩˧ʐv̩˥}\newline
\classe{名词}\ton{H\#}
\paradigme{\pcmn{:} \p{}}
\begin{définition}\peng{Dragon.}\end{définition}
\begin{définition}\pcmn{龙}\end{définition}
\begin{définition}\pfra{Dragon.}\end{définition}
\end{entrée}

\begin{entrée}
{mv̩˧ɕi˥}{}{ⓔmv̩˧ɕi˥}\formedesurface{mv̩˧ɕi˥}\newline
\classe{名词}\ton{H\#}
\paradigme{\pcmn{:} \p{}}
\begin{définition}\peng{Spark.}\end{définition}
\begin{définition}\pcmn{火花}\end{définition}
\begin{définition}\pfra{Étincelle.}\end{définition}
\end{entrée}

\begin{entrée}
{mv̩˧ɕi˥dʑɯ˩ʈʰɯ˩}{}{ⓔmv̩˧ɕi˥dʑɯ˩ʈʰɯ˩}\formedesurface{mv̩˧ɕi˥dʑɯ˩ʈʰɯ˩}\newline
\classe{名词}\ton{H\#-L}
\paradigme{\pcmn{:} \p{}}
\begin{définition}\peng{Rainbow.}\end{définition}
\begin{définition}\pcmn{彩虹}\end{définition}
\begin{définition}\pfra{Arc-en-ciel.}\end{définition}
\begin{exemple}\pnru{mv̩˧ɕi˥}\hspace{5pt}\peng{simplified form; same meaning: rainbow}\hspace{5pt}\pcmn{同上:彩虹(简化)}\hspace{5pt}\pfra{forme simplifiée; même sens: arc-en-ciel}\end{exemple}
\end{entrée}

\begin{entrée}
{mv̩˧di˧˥}{}{ⓔmv̩˧di˧˥}\formedesurface{mv̩˧di˧˥}\newline
\classe{名词}
\sens{1}\paradigme{\pcmn{:} \p{}}
\begin{définition}\peng{Field.}\end{définition}
\begin{définition}\pcmn{田地}\end{définition}
\begin{définition}\pfra{Champs (quel que soit ce qu'on y cultive).}\end{définition}\sens{2}
\begin{définition}\peng{The Earth, the place where mankind lives (as opposed to the Heavens).}\end{définition}
\begin{définition}\pcmn{天下}\end{définition}
\begin{définition}\pfra{La Terre, là où habitent les hommes (par opposition au ciel).}\end{définition}
\end{entrée}

\begin{entrée}
{mv̩˩do˩}{}{ⓔmv̩˩do˩}\formedesurface{mv̩˩do˩˥}\newline
\classe{动词}\ton{L}\begin{définition}\peng{To ask.}\end{définition}
\begin{définition}\pcmn{问}\end{définition}
\begin{définition}\pfra{Demander.}\end{définition}
\begin{exemple}\pnru{le˧-mv̩˩do˩}\hspace{5pt}\peng{|fg{accomp}}\hspace{5pt}\pcmn{|fg{accomp}}\hspace{5pt}\pfra{|fg{accomp}}\end{exemple}
\begin{exemple}\pnru{mv̩˩do˩-ze˥}\hspace{5pt}\peng{|fg{pfv}}\hspace{5pt}\pcmn{问了}\hspace{5pt}\pfra{|fg{pfv}}\end{exemple}
\begin{exemple}\pnru{ə˧tso˧ mv̩˩do˩-bi˩? |}\hspace{5pt}\peng{What would [you] like to ask? / What is your question?}\hspace{5pt}\pcmn{要问什么呢?}\hspace{5pt}\pfra{qu'est-ce que (tu) vas demander?}\end{exemple}
\end{entrée}

\begin{entrée}
{mv̩˧dze˧}{}{ⓔmv̩˧dze˧}\formedesurface{mv̩˧dze˧}\newline
\classe{名词}\ton{M}
\paradigme{\pcmn{:} \p{}}
\begin{définition}\peng{Barley, |\stylefi{Hordeum vulgare L}.}\end{définition}
\begin{définition}\pcmn{大麦}\end{définition}
\begin{définition}\pfra{Orge, |\stylefi{Hordeum vulgare L}.}\end{définition}
\end{entrée}

\begin{entrée}
{mv̩˧dze˧-tɕʰi\#˥}{}{ⓔmv̩˧dze˧-tɕʰi\#˥}\formedesurface{mv̩˧dze˧tɕʰi˧}\newline
\classe{名词}\ton{\#H}\begin{définition}\peng{Highland barley beard.}\end{définition}
\begin{définition}\pcmn{青稞芒}\end{définition}
\begin{définition}\pfra{Barbe d'orge.}\end{définition}
\end{entrée}

\begin{entrée}
{mv̩˩dzɤ˧}{}{ⓔmv̩˩dzɤ˧}\formedesurface{mv̩˩dzɤ˥}\newline
\classe{名词}\ton{LM}\begin{définition}\peng{Bottom part (symbolically: “the tail").}\end{définition}
\begin{définition}\pcmn{下面部分}\end{définition}
\begin{définition}\pfra{Bas, partie inférieure (symboliquement: «la queue»).}\end{définition}
\begin{exemple}\pnru{mv̩˩dzɤ˧ dzi˧˥}\hspace{5pt}\peng{to be seated in the bottom part (of the room)}\hspace{5pt}\pcmn{坐在(房间的)下面部分}\hspace{5pt}\pfra{être assis au fond de la salle}\end{exemple}
\begin{exemple}\pnru{no˧ | mv̩˩dzɤ˧ dzi˧˥!}\hspace{5pt}\peng{Go and get seated in the bottom part (of the room)!}\hspace{5pt}\pcmn{你到下面去坐!}\hspace{5pt}\pfra{Assieds-toi en bas!}\end{exemple}
\end{entrée}

\begin{entrée}
{mv̩˩ɖæ˧}{}{ⓔmv̩˩ɖæ˧}\formedesurface{mv̩˩ɖæ˥}\newline
\classe{名词}\ton{LM}\begin{définition}\peng{Bottom part of body.}\end{définition}
\begin{définition}\pcmn{下半身}\end{définition}
\begin{définition}\pfra{Le bas du corps.}\end{définition}
\end{entrée}

\begin{entrée}
{mv̩˧ɖɯ˩}{}{ⓔmv̩˧ɖɯ˩}\formedesurface{mv̩˧ɖɯ˩}\newline
\classe{名词}\ton{L\#}\begin{définition}\peng{Muddee mountain (Yongning).}\end{définition}
\begin{définition}\pcmn{木地箐(一座山与山上的村落)}\end{définition}
\begin{définition}\pfra{La montagne Muddee (Mudiqing, Yongning). Elle était habitée par des Pumi.}\end{définition}
\begin{exemple}\pnru{mv̩˧ɖɯ˩-qo˩qɑ˩}\hspace{5pt}\peng{The pass of Mount Muddeeq (Mudiqing), in Yongning}\hspace{5pt}\pcmn{木地箐垭口}\hspace{5pt}\pfra{Le col de la montagne Muddeeq de Yongning}\end{exemple}
\end{entrée}

\begin{entrée}
{mv̩˩ɖɯ˩}{}{ⓔmv̩˩ɖɯ˩}\formedesurface{mv̩˩ɖɯ˩˥}\newline
\classe{名词}\ton{L}\begin{définition}\peng{Eldest daughter.}\end{définition}
\begin{définition}\pcmn{大女儿}\end{définition}
\begin{définition}\pfra{Fille aînée.}\end{définition}
\begin{exemple}\pnru{zo˧ɖɯ˧-mv̩˥ɖɯ˩}\hspace{5pt}\peng{eldest son and eldest daughter (i.e. eldest male and female siblings)}\hspace{5pt}\pcmn{大儿子与大女儿}\hspace{5pt}\pfra{fils aîné et fille aînée: les aînés}\end{exemple}
\end{entrée}

\begin{entrée}
{mv̩˧-gɤ˥gɤ˩}{}{ⓔmv̩˧-gɤ˥gɤ˩}\formedesurface{mv̩˧gɤ˥gɤ˩}\newline
\classe{名词}\ton{\#H-}\begin{définition}\peng{Descendants.}\end{définition}
\begin{définition}\pcmn{下一代、后裔、后人}\end{définition}
\begin{définition}\pfra{Les descendants, la descendance.}\end{définition}
\end{entrée}

\begin{entrée}
{mv̩˧-gɤ˧lɑ˥}{}{ⓔmv̩˧-gɤ˧lɑ˥}\formedesurface{mv̩˧gɤ˧lɑ˥}\newline
\classe{名词}\ton{H\#}
\paradigme{\pcmn{:} \p{}}
\begin{définition}\peng{Sky spirit.}\end{définition}
\begin{définition}\pcmn{天宫菩萨}\end{définition}
\begin{définition}\pfra{Esprit du ciel, Bodhisattva céleste.}\end{définition}
\end{entrée}

\begin{entrée}
{mv̩˧gv̩\#˥}{}{ⓔmv̩˧gv̩\#˥}\formedesurface{mv̩˧gv̩˧}\newline
\classe{名词}\ton{\#H}
\paradigme{\pcmn{:} \p{}}
\begin{définition}\peng{Clap of thunder.}\end{définition}
\begin{définition}\pcmn{雷、雷声}\end{définition}
\begin{définition}\pfra{Tonnerre.}\end{définition}
\begin{exemple}\pnru{mv̩˧gv̩˧ | gv̩˧-ze˩}\hspace{5pt}\peng{there is a clap of thunder}\hspace{5pt}\pcmn{打雷了}\hspace{5pt}\pfra{le tonnerre gronde}\end{exemple}
\begin{exemple}\pnru{mv̩˧gv̩˧ lɑ˩}\hspace{5pt}\peng{there is a clap of thunder}\hspace{5pt}\pcmn{打雷了}\hspace{5pt}\pfra{il y a un coup de tonnerre}\end{exemple}
\end{entrée}

\begin{entrée}
{mv̩˧-gv̩˧dv̩˧}{}{ⓔmv̩˧-gv̩˧dv̩˧}\formedesurface{mv̩˧gv̩˧dv̩˧}\newline
\classe{名词}\ton{M}
\paradigme{\pcmn{:} \p{}}
\begin{définition}\peng{Instep, top part of the foot.}\end{définition}
\begin{définition}\pcmn{脚背}\end{définition}
\begin{définition}\pfra{Partie supérieure du pied.}\end{définition}
\end{entrée}

\begin{entrée}
{mv̩˧gv̩˧-kʰv̩˩}{₁}{ⓔmv̩˧gv̩˧-kʰv̩˩ⓗ1}\formedesurface{mv̩˧gv̩˧kʰv̩˩}\newline
\classe{名词}\ton{L\#}
1\begin{définition}\peng{Year of the Dragon.}\end{définition}
\begin{définition}\pcmn{龙年}\end{définition}
\begin{définition}\pfra{Année du Dragon.}\end{définition}
\end{entrée}

\begin{entrée}
{mv̩˧gv̩˧-kʰv̩˩}{₂}{ⓔmv̩˧gv̩˧-kʰv̩˩ⓗ2}\formedesurface{mv̩˧gv̩˧kʰv̩˩}\newline
\classe{形容词}\ton{L\#}
2\begin{définition}\peng{Born in the year of the Dragon.}\end{définition}
\begin{définition}\pcmn{属龙}\end{définition}
\begin{définition}\pfra{Né l'année du Dragon.}\end{définition}
\end{entrée}

\begin{entrée}
{mv̩˧kʰv̩˧˥}{}{ⓔmv̩˧kʰv̩˧˥}\formedesurface{mv̩˧kʰv̩˧˥}\newline
\classe{名词}\ton{MH\#}
\paradigme{\pcmn{:} \p{}}
\begin{définition}\peng{Smoke.}\end{définition}
\begin{définition}\pcmn{烟}\end{définition}
\begin{définition}\pfra{Fumée.}\end{définition}
\begin{exemple}\pnru{mv̩˧kʰv̩˧ lv̩˥}\hspace{5pt}\peng{there is a lot of smoke}\hspace{5pt}\pcmn{烟很多}\hspace{5pt}\pfra{ça enfume tout le monde}\end{exemple}
\end{entrée}

\begin{entrée}
{mv̩˩kʰv̩˧˥}{}{ⓔmv̩˩kʰv̩˧˥}\formedesurface{mv̩˩kʰv̩˧˥}\newline
\classe{名词}\ton{LM+MH\#}\begin{définition}\peng{Evening (starting when it begins to get dark).}\end{définition}
\begin{définition}\pcmn{晚上}\end{définition}
\begin{définition}\pfra{Soir, soirée (dès 17h, 18h, quand approche la tombée de la nuit).}\end{définition}
\end{entrée}

\begin{entrée}
{mv̩˧ɭɯ˩}{}{ⓔmv̩˧ɭɯ˩}\formedesurface{mv̩˧ɭɯ˩}\newline
\classe{名词}\ton{L\#}\begin{définition}\peng{Muli county.}\end{définition}
\begin{définition}\pcmn{木里}\end{définition}
\begin{définition}\pfra{Muli (localité dans le Sichuan, proche de Yongning).}\end{définition}
\end{entrée}

\begin{entrée}
{mv̩˩ɬi˥}{}{ⓔmv̩˩ɬi˥}\formedesurface{mv̩˩ɬi˥}\newline
\classe{名词}\ton{LH}\begin{définition}\peng{Second daughter; literally “middle daughter".}\end{définition}
\begin{définition}\pcmn{二女儿}\end{définition}
\begin{définition}\pfra{Cadette, puinée (fille deuxième née); littéralement: «fille du milieu».}\end{définition}
\end{entrée}

\begin{entrée}
{mv̩˧mi˧}{}{ⓔmv̩˧mi˧}\formedesurface{mv̩˧mi˧}\newline
\classe{名词}\ton{M}
\paradigme{\pcmn{:} \p{}}
\begin{définition}\peng{Woman.}\end{définition}
\begin{définition}\pcmn{女人}\end{définition}
\begin{définition}\pfra{Femme.}\end{définition}
\begin{exemple}\pnru{mv̩˧mi˧ so˩tsʰi˩-kʰv̩˩, | qʰo˧mo˥ gi˩ le˩-ʈɤ˩! | ʝi˧=ɻæ˧ qʰv̩˧tsʰi˧-kʰv̩˩, | bɤ˧di˩ lɑ˩ hṽ̩˩ ɖʐæ˩!}\hspace{5pt}\peng{“A woman of thirty must be pulled along like an old cow; a man of sixty stills rides tigers bareback in the land of the Pumi!" This proverb is about ageing in both sexes, with special emphasis on the appeal that they exert on the opposite sex: at thirty, a woman is old; at sixty, a man is still ready for the greatest exploits. The proverb is reported to be used by women, as an ironic (covertly mocking) comment about an ageing man.}\hspace{5pt}\pcmn{“女人,到三十岁就算是得拉着的老牛。男人,到六十岁还能在普米山上骑老虎!”这个谚语讲男人与女人老化过程,特别描写相互吸引的程度:三十岁女人算是老了,六十岁男人还认为自己有伟大的威力。女人可以用这个谚语隐蔽地嘲弄一个老男人。}\hspace{5pt}\pfra{«A trente ans, la femme est déjà comme une vieille vache qu'il faut tirer pour qu'elle avance (=à trente ans, une femme, c'est déjà une vieille); à soixante, l'homme chevauche sur une peau [littéralement: des poils] de tigre au pays des Pumi!» (=pour l'homme, soixante ans c'est un âge qui permet encore les exploits) (Dicton au sujet de la façon dont vieillissent les deux sexes, au plan de l'attirance qu'ils exercent sur l'autre sexe; employé par une femme, peut véhiculer une nuance de moquerie à l'égard d'un homme âgé)}\end{exemple}
\end{entrée}

\begin{entrée}
{mv̩˧∼mv̩\#˥}{}{ⓔmv̩˧∼mv̩\#˥}\formedesurface{mv̩˧mv̩˧}\newline
\classe{形容词}\ton{\#H}
\étymologie{
mv̩˥
}\begin{définition}\peng{Clear (speech).}\end{définition}
\begin{définition}\pcmn{清楚(话、事情)}\end{définition}
\begin{définition}\pfra{Clair (parole, événement…).}\end{définition}
\begin{exemple}\pnru{ʐwɤ˧ mv̩˧∼mv̩˧}\hspace{5pt}\peng{to speak clearly; clear speech}\hspace{5pt}\pcmn{讲清楚}\hspace{5pt}\pfra{parler clairement}\end{exemple}
\begin{exemple}\pnru{le˧-mv̩˧∼mv̩˧-kʰɯ˩}\hspace{5pt}\peng{to clarify, to explain}\hspace{5pt}\pcmn{弄明白、讲清楚}\hspace{5pt}\pfra{éclaircir, tirer au clair, expliquer}\end{exemple}
\end{entrée}

\begin{entrée}
{mv̩˧-mv̩˥-di˩}{}{ⓔmv̩˧-mv̩˥-di˩}\formedesurface{mv̩˧mv̩˥di˩}\newline
\classe{名词}\ton{H\#-}
\paradigme{\pcmn{:} \p{}}
\begin{définition}\peng{Bellows.}\end{définition}
\begin{définition}\pcmn{风箱}\end{définition}
\begin{définition}\pfra{Soufflet.}\end{définition}
\end{entrée}

\begin{entrée}
{mv̩˧ɲi˧}{}{ⓔmv̩˧ɲi˧}\formedesurface{mv̩˧ɲi˧}\newline
\classe{名词}\ton{M}
\paradigme{\pcmn{:} \p{}}
\begin{définition}\peng{Toe.}\end{définition}
\begin{définition}\pcmn{脚趾}\end{définition}
\begin{définition}\pfra{Orteil.}\end{définition}
\end{entrée}

\begin{entrée}
{mv̩˩pʰæ˧}{}{ⓔmv̩˩pʰæ˧}\formedesurface{mv̩˩pʰæ˥}\newline
\classe{名词}\ton{LM}
\paradigme{\pcmn{:} \p{}}
\begin{définition}\peng{Kitchen: the room where pig swill is cooked, wine is distilled, and some of the food for people is prepared.}\end{définition}
\begin{définition}\pcmn{备料房:煮猪食、煮酒的地方,有时候也在那边准备人的饭}\end{définition}
\begin{définition}\pfra{Office, cuisine: pièce où on cuisine la pâtée des cochons, où on distille le vin, et où on prépare certains des plats pour les humains. Elle est située dans le même bâtiment que le foyer-salle à manger, à sa droite (vu depuis la cour).}\end{définition}
\end{entrée}

\begin{entrée}
{mv̩˩pʰæ˧˥}{}{ⓔmv̩˩pʰæ˧˥}\formedesurface{mv̩˩pʰæ˧˥}\newline
\classe{动词}\ton{LM+MH\#}\begin{définition}\peng{To forget.}\end{définition}
\begin{définition}\pcmn{忘记、落下}\end{définition}
\begin{définition}\pfra{Oublier.}\end{définition}
\begin{exemple}\pnru{mv̩˩pʰæ˧-ze˥}\hspace{5pt}\peng{(I) forgot!}\hspace{5pt}\pcmn{忘记了}\hspace{5pt}\pfra{(j'ai) oublié!}\end{exemple}
\begin{exemple}\pnru{le˧-mv̩˩pʰæ˩(-ze˩)}\hspace{5pt}\peng{|fg{accomp} \_ |fg{pfv}}\hspace{5pt}\pcmn{忘记了}\hspace{5pt}\pfra{|fg{accomp} \_ |fg{pfv}}\end{exemple}
\begin{exemple}\pnru{ə˧tso˧-lɑ˧ ʂv̩˧ɖv̩˧, | mɤ˧-do˩! | tso˧∼tso˧ mv̩˥pʰæ˩!}\hspace{5pt}\peng{I have no idea what (s)he has in mind; (s)he keeps forgetting things!}\hspace{5pt}\pcmn{不知道(他)在想什么呢!他(一直在)落东西!/ 他一直在丢三落四}\hspace{5pt}\pfra{Je ne sais pas à quoi il pense/je ne sais pas ce qu'il a dans la tête: il oublie les choses!/il oublie tout!}\end{exemple}
\end{entrée}

\begin{entrée}
{mv̩˧qo˩}{}{ⓔmv̩˧qo˩}\formedesurface{mv̩˧qo˩}\newline
\classe{名词}\ton{L\#}
\paradigme{\pcmn{:} \p{}}
\begin{définition}\peng{Papaya.}\end{définition}
\begin{définition}\pcmn{木瓜}\end{définition}
\begin{définition}\pfra{Papaye.}\end{définition}
\begin{exemple}\pnru{mv̩˧qo˩-dʑɯ˩}\hspace{5pt}\peng{a liquid prepared from the papaya, which served as an equivalent of vinegar (vinegar was introduced late: it was bought in Chinese areas)}\hspace{5pt}\pcmn{用木瓜做的一种汁,用法类似于醋。过去,永宁没有醋,醋是从内地(汉族地区)买来的。}\hspace{5pt}\pfra{un liquide préparé à base de papaye, servant d'équivalent de vinaigre (le vinaigre a été introduit tardivement; il était acheté en pays chinois)}\end{exemple}
\end{entrée}

\begin{entrée}
{mv̩˧qʰwæ˩}{}{ⓔmv̩˧qʰwæ˩}\formedesurface{mv̩˧qʰwæ˩}\newline
\classe{名词}\ton{L\#}\begin{définition}\peng{The name of a village outside the plain of Yongning, close to the Lake.}\end{définition}
\begin{définition}\pcmn{木垮:村落名}\end{définition}
\begin{définition}\pfra{Village na hors de la plaine de Yongning, vers le Lac.}\end{définition}
\begin{exemple}\pnru{ɬi˧ki˧, | ɲi˧se˩, | tɑ˧dzi˩, | mv̩˧qʰwæ˩, | lɑ˧tʰɑ˧-di˧˥}\hspace{5pt}\peng{Villages that one passes when moving away from the Yongning plain, towards Lake Lugu. These villages do not count as part of Yongning proper. The last, /lɑ˧tʰɑ˧-di˧˥/, is not a village name like the preceding four: it refers to the entire Na area beyond the fourth village.}\hspace{5pt}\pcmn{永宁到泸沽湖所经过的村落,依次是:里格、尼赛、大祖、木垮,然后到拉塔地(拉塔地指的是泸沽湖周边的摩梭地区,包括左所、洛水村等)}\hspace{5pt}\pfra{Villages dans l'ordre, après la plaine de Yongning, ne comptant pas comme faisant partie de Yongning. Le dernier, /lɑ˧tʰɑ˧-di˧˥/, désigne toute la région na au-delà du quatrième village.}\end{exemple}
\end{entrée}

\begin{entrée}
{mv̩˧ʁo˥\$}{}{ⓔmv̩˧ʁo˥\$}\formedesurface{mv̩˧ʁo˥}\newline
\classe{名词}\ton{H\$}
\paradigme{\pcmn{:} \p{}}
\begin{définition}\peng{Heavens, sky.}\end{définition}
\begin{définition}\pcmn{天空}\end{définition}
\begin{définition}\pfra{Le ciel, les cieux.}\end{définition}
\end{entrée}

\begin{entrée}
{mv̩˩ʁwɤ˧}{₁}{ⓔmv̩˩ʁwɤ˧ⓗ1}\formedesurface{mv̩˩ʁwɤ˥}\newline
\classe{名词}\ton{LM}
1\begin{définition}\peng{Lower reaches of a river; downstream.}\end{définition}
\begin{définition}\pcmn{下游}\end{définition}
\begin{définition}\pfra{Cours inférieur, aval.}\end{définition}
\end{entrée}

\begin{entrée}
{mv̩˩ʁwɤ˧}{₂}{ⓔmv̩˩ʁwɤ˧ⓗ2}\formedesurface{mv̩˩ʁwɤ˥}\newline
\classe{名词}\ton{LM}
2\begin{définition}\peng{The name of a village.}\end{définition}
\begin{définition}\pcmn{下村,比如者波下村(永宁的一个村落)}\end{définition}
\begin{définition}\pfra{«le village du bas»: nom courant pour désigner un hameau d'un village, ou un village entier, par exemple le hameau du bas du village de Zhubo.}\end{définition}
\end{entrée}

\begin{entrée}
{mv̩˩si˧˥}{}{ⓔmv̩˩si˧˥}\formedesurface{mv̩˩si˧˥}\newline
\classe{名词}\ton{LM+MH\#}\begin{définition}\peng{Morning.}\end{définition}
\begin{définition}\pcmn{早晨}\end{définition}
\begin{définition}\pfra{Matin.}\end{définition}
\end{entrée}

\begin{entrée}
{mv̩˩si˧-njɤ˧˥}{}{ⓔmv̩˩si˧-njɤ˧˥}\formedesurface{mv̩˩si˧njɤ˧˥}\newline
\classe{助词}\ton{LM+MH\#}\begin{définition}\peng{Early in the morning.}\end{définition}
\begin{définition}\pcmn{一大早}\end{définition}
\begin{définition}\pfra{Tôt le matin.}\end{définition}
\end{entrée}

\begin{entrée}
{mv̩˩tɑ\#˥}{}{ⓔmv̩˩tɑ\#˥}\formedesurface{mv̩˩tɑ˥}\newline
\classe{动词}\ton{LM+\#H}\begin{définition}\peng{To praise, to commend.}\end{définition}
\begin{définition}\pcmn{表扬}\end{définition}
\begin{définition}\pfra{Louer, faire l'éloge de.}\end{définition}
\begin{exemple}\pnru{mv̩˩tɑ˧ ʝi˧}\hspace{5pt}\peng{to praise}\hspace{5pt}\pcmn{表扬}\hspace{5pt}\pfra{louer, faire l'éloge de}\end{exemple}
\begin{exemple}\pnru{hĩ˧-ɳɯ˩ | mv̩˩tɑ˥ F | ʝi˧ le˧-hɯ˩-ze˩.}\hspace{5pt}\peng{(She/he did some good things, and) people praised him.}\hspace{5pt}\pcmn{(他做了好事情,于是)人家大大地表扬他了。}\hspace{5pt}\pfra{(Il a fait de bonnes choses, et) les gens l'ont loué/ ont chanté ses louanges}\end{exemple}
\end{entrée}

\begin{entrée}
{mv̩˩tv̩˩}{}{ⓔmv̩˩tv̩˩}\formedesurface{mv̩˩tv̩˩˥}\newline
\classe{名词}\ton{L}\begin{définition}\peng{Only daughter.}\end{définition}
\begin{définition}\pcmn{独生女}\end{définition}
\begin{définition}\pfra{Fille unique.}\end{définition}
\begin{exemple}\pnru{mv̩˩tv̩˩˥ | ɖɯ˧-v̩˧-lɑ˧ dʑo˧˥!}\hspace{5pt}\peng{(She) just has an only daughter!}\hspace{5pt}\pcmn{她只有一个独生女!}\hspace{5pt}\pfra{(elle) n'a qu'une fille unique!}\end{exemple}
\end{entrée}

\begin{entrée}
{mv̩˩tʰi˩}{}{ⓔmv̩˩tʰi˩}\formedesurface{mv̩˩tʰi˩˥}\newline
\classe{形容词}\ton{L}
\étymologie{
mv̩˩˥; tʰi˧
}\begin{définition}\peng{Intelligent.}\end{définition}
\begin{définition}\pcmn{聪明}\end{définition}
\begin{définition}\pfra{Intelligente (d'une femme).}\end{définition}
\begin{exemple}\pnru{ʈʂʰɯ˧ | mv̩˩tʰi˩˥ | ʐwæ˩˥!}\hspace{5pt}\peng{She is very intelligent!}\hspace{5pt}\pcmn{她很聪明!}\hspace{5pt}\pfra{elle est très intelligente!}\end{exemple}
\end{entrée}

\begin{entrée}
{mv̩˧tʰv̩˧˥}{}{ⓔmv̩˧tʰv̩˧˥}\formedesurface{mv̩˧tʰv̩˧˥}\newline
\classe{名词}\ton{MH\#}
\paradigme{\pcmn{:} \p{}}
\begin{définition}\peng{Torch.}\end{définition}
\begin{définition}\pcmn{火把}\end{définition}
\begin{définition}\pfra{Torche.}\end{définition}
\end{entrée}

\begin{entrée}
{mv̩˩tɕi˥}{}{ⓔmv̩˩tɕi˥}\formedesurface{mv̩˩tɕi˥}\newline
\classe{名词}\ton{LH}\begin{définition}\peng{Youngest daughter.}\end{définition}
\begin{définition}\pcmn{最小的女儿}\end{définition}
\begin{définition}\pfra{Benjamine, plus jeune fille.}\end{définition}
\end{entrée}

\begin{entrée}
{mv̩˩tɕo˧}{}{ⓔmv̩˩tɕo˧}\formedesurface{mv̩˩tɕo˥}\newline
\classe{助词}\ton{LM}\begin{définition}\peng{Downward.}\end{définition}
\begin{définition}\pcmn{往下}\end{définition}
\begin{définition}\pfra{Vers le bas.}\end{définition}
\begin{exemple}\pnru{mv̩˩tɕo˧ kwɤ˩}\hspace{5pt}\peng{to throw down}\hspace{5pt}\pcmn{往下扔}\hspace{5pt}\pfra{jeter vers le bas}\end{exemple}
\begin{exemple}\pnru{mv̩˩tɕo˧ se˧!}\hspace{5pt}\peng{Get down! Go down! (Command to the dog if it climbs onto the floorboard of the house, contrary to the rule)}\hspace{5pt}\pcmn{下去!(命令狗从主屋的地板下去:狗不准来上面)}\hspace{5pt}\pfra{Descends! (Ce qu'on dit au chien qui monte sur la partie haute de la cuisine, contrevenant à la règle)}\end{exemple}
\end{entrée}

\begin{entrée}
{mv̩˧tsɯ˧˥}{}{ⓔmv̩˧tsɯ˧˥}\formedesurface{mv̩˧tsɯ˧˥}\newline
\classe{名词}\ton{MH\#}
\paradigme{\pcmn{:} \p{}}
\begin{définition}\peng{Beard.}\end{définition}
\begin{définition}\pcmn{胡子}\end{définition}
\begin{définition}\pfra{Barbe.}\end{définition}
\begin{exemple}\pnru{mv̩˧tsɯ˧ ʑi˥}\hspace{5pt}\peng{to have a beard}\hspace{5pt}\pcmn{有胡子}\hspace{5pt}\pfra{avoir de la barbe}\end{exemple}
\end{entrée}

\begin{entrée}
{mv̩˧tsʰi\#˥}{}{ⓔmv̩˧tsʰi\#˥}\formedesurface{mv̩˧tsʰi˧}\newline
\classe{名词}\ton{\#H}\begin{définition}\peng{Dry season (winter and spring: from the 9th lunar month to the 2nd lunar month).}\end{définition}
\begin{définition}\pcmn{旱季(冬天与春天:农历九月到二月)}\end{définition}
\begin{définition}\pfra{Saison sèche (hiver et printemps; du 9e mois au 2e mois du calendrier lunaire compris).}\end{définition}
\begin{exemple}\pnru{mv̩˧tsʰi˧-qo˩}\hspace{5pt}\peng{during the dry season}\hspace{5pt}\pcmn{旱季的时候}\hspace{5pt}\pfra{durant la saison sèche}\end{exemple}
\end{entrée}

\begin{entrée}
{mv̩˩tsʰo˩}{}{ⓔmv̩˩tsʰo˩}\formedesurface{mv̩˩tsʰo˩˥}\newline
\classe{名词}\ton{L}
\paradigme{\pcmn{:} \p{}}
\begin{définition}\peng{Firewood full of resin, used to start a fire.}\end{définition}
\begin{définition}\pcmn{含很多树脂的木头,用来引火}\end{définition}
\begin{définition}\pfra{Bois gorgé de résine, pour faire partir le feu (pièces de la taille d'une bûche, qu'on débite en petits morceaux pour faire partir le feu).}\end{définition}
\end{entrée}

\begin{entrée}
{mv̩˧ʈʰæ\#˥}{}{ⓔmv̩˧ʈʰæ\#˥}\formedesurface{mv̩˧ʈʰæ˧}\newline
\classe{助词}\ton{\#H}\begin{définition}\peng{Under.}\end{définition}
\begin{définition}\pcmn{下面}\end{définition}
\begin{définition}\pfra{Dessous, en bas.}\end{définition}
\begin{exemple}\pnru{ʈʂʰɯ˧ | mv̩˧ʈʰæ˧-lɑ˩ li˩! | gɤ˧bi˧ mɤ˧-li˩!}\hspace{5pt}\peng{He only looks down, he never glances up! (About someone who constantly sits at his desk, and complains about headaches and a sore neck: the speaker points out that it may be due to a bad posture while at work.)}\hspace{5pt}\pcmn{他老低头是往下看,不往上看!(情景:有人经常脖子疼、头疼,阿妈提出,这应该跟工作姿势不对有关:那个人一直坐在办公桌前,低着头)}\hspace{5pt}\pfra{Il regarde tout le temps vers le bas! il ne regarde pas vers le haut! (au sujet d'une personne constamment assise à son bureau, et qui se plaint de mots de tête)}\end{exemple}
\end{entrée}

\begin{entrée}
{mv̩˧ʈʰɯ˧}{}{ⓔmv̩˧ʈʰɯ˧}\formedesurface{mv̩˧ʈʰɯ˧}\newline
\classe{名词}\ton{M}
\paradigme{\pcmn{:} \p{}}
\begin{définition}\peng{Heel.}\end{définition}
\begin{définition}\pcmn{脚跟}\end{définition}
\begin{définition}\pfra{Talon.}\end{définition}
\end{entrée}

\begin{entrée}
{mv̩˧ʈʂæ˧˥}{}{ⓔmv̩˧ʈʂæ˧˥}\formedesurface{mv̩˧ʈʂæ˧˥}\newline
\classe{动词}\ton{MH\#}\begin{définition}\peng{To call, to give the name… to, to refer to… as…}\end{définition}
\begin{définition}\pcmn{叫做、称作、名叫}\end{définition}
\begin{définition}\pfra{S'appeler, avoir… pour nom.}\end{définition}
\begin{exemple}\pnru{(ʈʂʰɯ˧ | ) ə˧tso˧ mv̩˧ʈʂæ˧˥?}\hspace{5pt}\peng{What's her/his name? / What is (it/he/she) called?}\hspace{5pt}\pcmn{他叫什么名字?}\hspace{5pt}\pfra{comment il s'appelle? Quel est son nom?}\end{exemple}
\begin{exemple}\pnru{njɤ˧ | … mv̩˧ʈʂæ˧˥}\hspace{5pt}\peng{My name is…}\hspace{5pt}\pcmn{我名字叫……}\hspace{5pt}\pfra{Je m'appelle…}\end{exemple}
\end{entrée}

\begin{entrée}
{mv̩˩ʈʂæ˧˥}{}{ⓔmv̩˩ʈʂæ˧˥}\formedesurface{mv̩˩ʈʂæ˧˥}\newline
\classe{名词}\ton{LM+MH\#}\begin{définition}\peng{Lower part (of the body=below the waist).}\end{définition}
\begin{définition}\pcmn{下半(身)}\end{définition}
\begin{définition}\pfra{Bas du corps, partie inférieure du corps.}\end{définition}
\end{entrée}

\begin{entrée}
{mv̩˧ʈʂo˩-ti˩-bv̩˩}{}{ⓔmv̩˧ʈʂo˩-ti˩-bv̩˩}\formedesurface{mv̩˧ʈʂo˩ti˩bv̩˩}\newline
\classe{名词}\ton{L\#-}\begin{définition}\peng{Water striders, water bugs, magic bugs, pond skaters, |\stylefi{Gerridae}.}\end{définition}
\begin{définition}\pcmn{水黽}\end{définition}
\begin{définition}\pfra{Araignée d'eau, |\stylefi{Gerridae}.}\end{définition}
\end{entrée}

\begin{entrée}
{mv̩˧ʈʂv̩˥}{₁}{ⓔmv̩˧ʈʂv̩˥ⓗ1}\newline
\classe{形容词}
1
\sens{1}
\begin{définition}\peng{Creased.}\end{définition}
\begin{définition}\pcmn{皱(衣服)}\end{définition}
\begin{définition}\pfra{Plissé, froissé.}\end{définition}\sens{2}
\begin{définition}\peng{Wrinkled.}\end{définition}
\begin{définition}\pcmn{(脸)有皱纹}\end{définition}
\begin{définition}\pfra{Ridé.}\end{définition}
\begin{exemple}\pnru{to˧kɤ˧ | mv̩˧ʈʂv̩˥ ze˩.}\hspace{5pt}\peng{(His/her) forehead became wrinkled.}\hspace{5pt}\pcmn{他的前额有了皱纹。}\hspace{5pt}\pfra{(Son) front s'est ridé / son front a pris des rides.}\end{exemple}
\begin{exemple}\pnru{to˧kɤ˧ | le˧-mv̩˧ʈʂv̩˥}\hspace{5pt}\peng{(His/her) forehead is wrinkled.}\hspace{5pt}\pcmn{他的前额有皱纹。}\hspace{5pt}\pfra{(Son) front est ridé.}\end{exemple}
\begin{exemple}\pnru{æ˩ʂe˩˥ | le˧-mv̩˧ʈʂv̩˥}\hspace{5pt}\peng{The skin is wrinkled (literally “the flesh is wrinkled")}\hspace{5pt}\pcmn{皮肤有皱纹(直译:“肉有皱纹”)}\hspace{5pt}\pfra{La peau est ridée (littéralement: «la chair est ridée»)}\end{exemple}
\begin{relationsémantique}\{
renvoi
mv̩˧ʈʂv̩˥2
}\end{relationsémantique}\sens{3}
\begin{définition}\peng{Withered.}\end{définition}
\begin{définition}\pcmn{谢(花谢了)}\end{définition}
\begin{définition}\pfra{Fané.}\end{définition}
\begin{exemple}\pnru{bæ˩bæ˩˥ | le˧-mv̩˧ʈʂv̩˥-ze˩}\hspace{5pt}\peng{The flower has withered.}\hspace{5pt}\pcmn{花谢了。}\hspace{5pt}\pfra{La fleur s'est fanée.}\end{exemple}
\end{entrée}

\begin{entrée}
{mv̩˧ʈʂv̩˥}{₂}{ⓔmv̩˧ʈʂv̩˥ⓗ2}\formedesurface{mv̩˧ʈʂv̩˥}\newline
\classe{名词}\ton{H\#}
2
\paradigme{\pcmn{:} \p{}}
\begin{définition}\peng{Wrinkle.}\end{définition}
\begin{définition}\pcmn{皱纹}\end{définition}
\begin{définition}\pfra{Rides.}\end{définition}
\end{entrée}

\begin{entrée}
{mv̩˧ʈʂv̩˩}{}{ⓔmv̩˧ʈʂv̩˩}\formedesurface{mv̩˧ʈʂv̩˩}\newline
\classe{名词}\ton{L\#}
\paradigme{\pcmn{:} \p{}}
\begin{définition}\peng{Mortar.}\end{définition}
\begin{définition}\pcmn{臼}\end{définition}
\begin{définition}\pfra{Mortier.}\end{définition}
\end{entrée}

\begin{entrée}
{mv̩˧ʈʂv̩˩-nv̩˩mi˩}{}{ⓔmv̩˧ʈʂv̩˩-nv̩˩mi˩}\formedesurface{mv̩˧ʈʂv̩˩nv̩˩mi˩}\newline
\classe{名词}\ton{L\#-}
\étymologie{
mv̩˧ʈʂv̩˩; nv̩˩mi˩
}
\paradigme{\pcmn{:} \p{}}
\begin{définition}\peng{Small pestle.}\end{définition}
\begin{définition}\pcmn{杵}\end{définition}
\begin{définition}\pfra{Petit pilon.}\end{définition}
\end{entrée}

\begin{entrée}
{mv̩˧ʈʂʰɤ˩}{}{ⓔmv̩˧ʈʂʰɤ˩}\formedesurface{mv̩˧ʈʂʰɤ˩}\newline
\classe{名词}\ton{L\#}
\paradigme{\pcmn{:} \p{}}
\begin{définition}\peng{Chin.}\end{définition}
\begin{définition}\pcmn{下巴}\end{définition}
\begin{définition}\pfra{Menton.}\end{définition}
\end{entrée}

\begin{entrée}
{mv̩˩zo˩}{}{ⓔmv̩˩zo˩}\formedesurface{mv̩˩zo˩˥}\newline
\classe{名词}\ton{L}
\paradigme{\pcmn{:} \p{}}
\begin{définition}\peng{Young lady.}\end{définition}
\begin{définition}\pcmn{姑娘}\end{définition}
\begin{définition}\pfra{Jeune fille.}\end{définition}
\begin{exemple}\pnru{mv̩˩zo˩=ɻæ˧}\hspace{5pt}\peng{young ladies}\hspace{5pt}\pcmn{姑娘们}\hspace{5pt}\pfra{les jeunes filles}\end{exemple}
\end{entrée}

\begin{entrée}
{mv̩˩zo˩-ə˩mi˥}{}{ⓔmv̩˩zo˩-ə˩mi˥}\formedesurface{mv̩˩zo˩ə˩mi˥}\newline
\classe{名词}\ton{L+H\#}\begin{définition}\peng{A young lady and her mother.}\end{définition}
\begin{définition}\pcmn{姑娘与母亲}\end{définition}
\begin{définition}\pfra{(une) jeune fille et (sa) mère.}\end{définition}
\end{entrée}

\begin{entrée}
{mv̩˩zɯ˩}{₁}{ⓔmv̩˩zɯ˩ⓗ1}\formedesurface{mv̩˩zɯ˩˥}\newline
\classe{名词}\ton{L}
1
\paradigme{\pcmn{:} \p{}}
\begin{définition}\peng{Brothers.}\end{définition}
\begin{définition}\pcmn{兄弟(哥哥们与弟弟们)}\end{définition}
\begin{définition}\pfra{Frères (aînés ou cadets).}\end{définition}
\begin{exemple}\pnru{ʈʂʰɯ˧ | nɑ˧dʑi˧-bv̩˧ | mv̩˩zɯ˩-ʝi˥-hĩ˩ ɲi˩!}\hspace{5pt}\peng{He is nɑ˧dʑi˧/'s brother!}\hspace{5pt}\pcmn{他是|fv{nɑ˧dʑi˧/}的兄弟!}\hspace{5pt}\pfra{il est frère de nɑ˧dʑi˧/!}\end{exemple}
\end{entrée}

\begin{entrée}
{mv̩˩zɯ˩}{₂}{ⓔmv̩˩zɯ˩ⓗ2}\formedesurface{mv̩˩zɯ˩˥}\newline
\classe{名词}\ton{L}
2
\paradigme{\pcmn{:} \p{}}
\begin{définition}\peng{Oats.}\end{définition}
\begin{définition}\pcmn{燕麦}\end{définition}
\begin{définition}\pfra{Avoine.}\end{définition}
\end{entrée}

\begin{entrée}
{mv̩˩zɯ˩-ni˥mi˩}{}{ⓔmv̩˩zɯ˩-ni˥mi˩}\formedesurface{mv̩˩zɯ˩ni˥mi˩}\newline
\classe{名词}\ton{L+\#H-}\begin{définition}\peng{Brothers and sisters, siblings.}\end{définition}
\begin{définition}\pcmn{兄弟姐妹,堂兄弟姐妹}\end{définition}
\begin{définition}\pfra{Frères et sœurs (tous les frères et sœurs; s'applique aussi aux cousins).}\end{définition}
\end{entrée}

\begin{entrée}
{mv̩˧ʐe˧˥}{₁}{ⓔmv̩˧ʐe˧˥ⓗ1}\formedesurface{mv̩˧ʐe˧˥}\newline
\classe{名词}\ton{MH\#}
1\begin{définition}\peng{Rainy season (summer and autumn: from the 3rd to the 8th month of the lunar calendar).}\end{définition}
\begin{définition}\pcmn{雨季(夏天与秋天:三月份至八月份)}\end{définition}
\begin{définition}\pfra{Saison des pluies (été et automne: du 3e au 8e mois du calendrier lunaire).}\end{définition}
\begin{exemple}\pnru{mv̩˧ʐe˧-qo˥}\hspace{5pt}\peng{during the rainy season}\hspace{5pt}\pcmn{雨季的时候}\hspace{5pt}\pfra{pendant la saison des pluies}\end{exemple}
\end{entrée}

\begin{entrée}
{mv̩˧ʐe\#˥}{₂}{ⓔmv̩˧ʐe\#˥ⓗ2}\formedesurface{mv̩˧ʐe˧}\newline
\classe{名词}\ton{\#H}
2
\paradigme{\pcmn{:} \p{}}
\begin{définition}\peng{Gun; firelock.}\end{définition}
\begin{définition}\pcmn{枪,明火枪}\end{définition}
\begin{définition}\pfra{Arme à feu, fusil; arquebuse.}\end{définition}
\end{entrée}

\begin{entrée}
{mv̩˧ʐe˧-ʈʂʰæ˧ɣɯ\#˥}{}{ⓔmv̩˧ʐe˧-ʈʂʰæ˧ɣɯ\#˥}\formedesurface{mv̩˧ʐe˧ʈʂʰæ˧ɣɯ˧}\newline
\classe{名词}\ton{\#H}
\paradigme{\pcmn{:} \p{}}
\begin{définition}\peng{Gunpowder.}\end{définition}
\begin{définition}\pcmn{火药}\end{définition}
\begin{définition}\pfra{Poudre à canon.}\end{définition}
\end{entrée}

\begin{entrée}
{mv̩˩ʐɤ˩}{}{ⓔmv̩˩ʐɤ˩}\formedesurface{mv̩˩ʐɤ˩˥}\newline
\classe{名词}\ton{L}\begin{définition}\peng{Adopted daughter.}\end{définition}
\begin{définition}\pcmn{义女}\end{définition}
\begin{définition}\pfra{Fille adoptive.}\end{définition}
\end{entrée}

\begin{entrée}
{mv̩˧ʑi˩}{}{ⓔmv̩˧ʑi˩}\formedesurface{mv̩˧ʑi˩}\newline
\classe{名词}\ton{L\#}
\paradigme{\pcmn{:} \p{}}
\begin{définition}\peng{News, gossip.}\end{définition}
\begin{définition}\pcmn{消息、闲话、八卦}\end{définition}
\begin{définition}\pfra{Nouvelle, ragot, information, histoire.}\end{définition}
\begin{exemple}\pnru{mv̩˧ʑi˩ | ɖɯ˧-kʰwɤ˥}\hspace{5pt}\peng{a piece of gossip}\hspace{5pt}\pcmn{一个八卦}\hspace{5pt}\pfra{une nouvelle, un ragot, une information}\end{exemple}
\end{entrée}

\newpage\caractère{n}

\begin{entrée}
{nɑ˥}{}{ⓔnɑ˥}\formedesurface{nɑ˧}\newline
\classe{形容词}\ton{H}\begin{définition}\peng{Important, serious (e.g. a wound).}\end{définition}
\begin{définition}\pcmn{严重,重要}\end{définition}
\begin{définition}\pfra{Grave, sérieux (ex.: une blessure).}\end{définition}
\begin{exemple}\pnru{mɤ˧-nɑ˥}\hspace{5pt}\peng{not serious}\hspace{5pt}\pcmn{不严重}\hspace{5pt}\pfra{bénin, pas grave, sans conséquence (ex.: une écorchure)}\end{exemple}
\end{entrée}

\begin{entrée}
{nɑ˧˥}{}{ⓔnɑ˧˥}\formedesurface{nɑ˧˥}\newline
\classe{动词}\ton{MH}\begin{définition}\peng{To tremble.}\end{définition}
\begin{définition}\pcmn{发抖,颤抖}\end{définition}
\begin{définition}\pfra{Trembler.}\end{définition}
\begin{exemple}\pnru{nɑ˩∼nɑ˧-ze˥}\hspace{5pt}\peng{|fg{red} |fg{pfv}}\hspace{5pt}\pcmn{发抖了}\hspace{5pt}\pfra{|fg{red} |fg{pfv}}\end{exemple}
\begin{exemple}\pnru{le˧-nɑ˩∼nɑ˩}\hspace{5pt}\peng{|fg{accomp} |fg{red}}\hspace{5pt}\pcmn{|fg{accomp} |fg{red}}\hspace{5pt}\pfra{|fg{accomp} |fg{red}}\end{exemple}
\begin{exemple}\pnru{lo˩qʰwɤ˥ | nɑ˩∼nɑ˧˥}\hspace{5pt}\peng{the hand trembles}\hspace{5pt}\pcmn{手抖}\hspace{5pt}\pfra{la main tremble}\end{exemple}
\end{entrée}

\begin{entrée}
{nɑ˧α}{}{ⓔnɑ˧α}\formedesurface{ɖɯ˧ nɑ˧}\newline
\classe{量词}\ton{Mα}\begin{définition}\peng{Classifier for tools.}\end{définition}
\begin{définition}\pcmn{量词:工具(一把)}\end{définition}
\begin{définition}\pfra{Classificateur des outils.}\end{définition}
\begin{exemple}\pnru{ɖɯ˧-nɑ˧ dʑo˧}\hspace{5pt}\peng{there is one (tool)}\hspace{5pt}\pcmn{有一把(工具)}\hspace{5pt}\pfra{il y en a un; il y a un outil}\end{exemple}
\end{entrée}

\begin{entrée}
{nɑ˩˧}{}{ⓔnɑ˩˧}\formedesurface{nɑ˩˥}\newline
\classe{名词}\ton{LM}
\paradigme{\pcmn{:} \p{}}
\begin{définition}\peng{Endonym: Na.}\end{définition}
\begin{définition}\pcmn{自称:摩梭族}\end{définition}
\begin{définition}\pfra{Endonyme: les Na.}\end{définition}
\begin{exemple}\pnru{nɑ˩-mv̩˧ nɑ˥-di˩ |}\hspace{5pt}\peng{Na territory}\hspace{5pt}\pcmn{摩梭人地区}\hspace{5pt}\pfra{le territoire des Na}\end{exemple}
\begin{exemple}\pnru{ə˧ʝi˧-ʂɯ˥ʝi˩, | nɑ˩zo˧-tɑ˥mv̩˩-ɳɯ˩ | dʑo˧-ɲi˥-tsɯ˩!}\hspace{5pt}\peng{Na traditions used to mention this! / There used to be Na traditions about this! (Context: when reference is made to local customs, to explain what is allowed and what is not.)}\hspace{5pt}\pcmn{过去,摩梭人的传统(里)有(关于这些问题的说法)嘛!}\hspace{5pt}\pfra{Autrefois, notre tradition, elle en parlait ! / Notre tradition, elle en parle! (Contexte: quand on fait référence à la coutume locale: ce qu'il est interdit de faire, ce qu'on est autorisé à faire…)}\end{exemple}
\end{entrée}

\begin{entrée}
{nɑ˩β}{}{ⓔnɑ˩β}\formedesurface{nɑ˩˥}\newline
\classe{形容词}\ton{Lβ}\begin{définition}\peng{Black.}\end{définition}
\begin{définition}\pcmn{黑,暗(颜色,天色)}\end{définition}
\begin{définition}\pfra{Noir, sombre.}\end{définition}
\begin{exemple}\pnru{nɑ˩-hĩ˥}\hspace{5pt}\peng{|fg{rel}}\hspace{5pt}\pcmn{黑的}\hspace{5pt}\pfra{|fg{rel}}\end{exemple}
\begin{exemple}\pnru{mɤ˧-nɑ˩}\hspace{5pt}\peng{|fg{neg}}\hspace{5pt}\pcmn{不黑}\hspace{5pt}\pfra{|fg{neg}}\end{exemple}
\end{entrée}

\begin{entrée}
{nɑ˩bɑ˧-ʁɑ˧ɭɯ\#˥}{}{ⓔnɑ˩bɑ˧-ʁɑ˧ɭɯ\#˥}\formedesurface{nɑ˩bɑ˧ʁɑ˧ɭɯ˧}\newline
\classe{名词}\ton{LM+\#H}\begin{définition}\peng{Name of a mountain.}\end{définition}
\begin{définition}\pcmn{一座山的名字}\end{définition}
\begin{définition}\pfra{Nom d'une montagne de Yongning.}\end{définition}
\end{entrée}

\begin{entrée}
{nɑ˩dzi˧}{}{ⓔnɑ˩dzi˧}\formedesurface{nɑ˩dzi˥}\newline
\classe{助词}\ton{LM}\begin{définition}\peng{Dark (at twilight, dusk).}\end{définition}
\begin{définition}\pcmn{暗(黄昏/暮的时候,天变暗)}\end{définition}
\begin{définition}\pfra{Sombre (au crépuscule, il se met à faire sombre).}\end{définition}
\begin{exemple}\pnru{nɑ˩dzi˧-ze˩!}\hspace{5pt}\peng{It has got dark! Twilight has come!}\hspace{5pt}\pcmn{天变暗了! / 黄昏到了!}\hspace{5pt}\pfra{le crépuscule est venu! / c'est le crépuscule!}\end{exemple}
\begin{exemple}\pnru{nɑ˩dzi˧-ho˩-ze˩!}\hspace{5pt}\peng{It's going to get dark!}\hspace{5pt}\pcmn{(天)要变暗了!}\hspace{5pt}\pfra{Il va faire sombre! La nuit va commencer à tomber!}\end{exemple}
\end{entrée}

\begin{entrée}
{nɑ˧dʑi\#˥}{}{ⓔnɑ˧dʑi\#˥}\formedesurface{nɑ˧dʑi˧}\newline
\classe{名词}\ton{\#H}\begin{définition}\peng{Feminine given name.}\end{définition}
\begin{définition}\pcmn{女性名字}\end{définition}
\begin{définition}\pfra{Prénom féminin.}\end{définition}
\end{entrée}

\begin{entrée}
{nɑ˩hĩ\#˥}{}{ⓔnɑ˩hĩ\#˥}\formedesurface{nɑ˩hĩ˥}\newline
\classe{名词}\ton{LM+\#H}
\paradigme{\pcmn{:} \p{}}
\begin{définition}\peng{Naxi (ethnic group).}\end{définition}
\begin{définition}\pcmn{纳西族}\end{définition}
\begin{définition}\pfra{Naxi (groupe ethnique).}\end{définition}
\begin{exemple}\pnru{nɑ˩hĩ˧-mi˧ ɲi˥!}\hspace{5pt}\peng{She is a Naxi women! / It's a Naxi woman!}\hspace{5pt}\pcmn{她是纳西族女人!}\hspace{5pt}\pfra{c'est une femme naxi!}\end{exemple}
\begin{exemple}\pnru{nɑ˩hĩ˧-bɑ˧lɑ˥}\hspace{5pt}\peng{the Naxi costume, Naxi garments}\hspace{5pt}\pcmn{纳西族服装}\hspace{5pt}\pfra{vêtements naxi, costume naxi}\end{exemple}
\begin{exemple}\pnru{nɑ˩hĩ˧-ʐwɤ˧ so˥}\hspace{5pt}\peng{to study the Naxi language}\hspace{5pt}\pcmn{学纳西语}\hspace{5pt}\pfra{apprendre la langue naxi}\end{exemple}
\begin{exemple}\pnru{nɑ˩hĩ˧-tʰæ˧ɻæ˥}\hspace{5pt}\peng{Naxi books}\hspace{5pt}\pcmn{纳西族的书}\hspace{5pt}\pfra{livres naxi}\end{exemple}
\end{entrée}

\begin{entrée}
{nɑ˩kwɤ˧}{}{ⓔnɑ˩kwɤ˧}\formedesurface{nɑ˩kwɤ˥}\newline
\classe{名词}\ton{LM}
\paradigme{\pcmn{:} \p{}}
\begin{définition}\peng{Pumpkin; cushaw.}\end{définition}
\begin{définition}\pcmn{南瓜}\end{définition}
\begin{définition}\pfra{Potiron.}\end{définition}
\end{entrée}

\begin{entrée}
{nɑ˧mi\#˥}{}{ⓔnɑ˧mi\#˥}\formedesurface{nɑ˧mi˧}\newline
\classe{名词}\ton{\#H}
\paradigme{\pcmn{:} \p{}}
\begin{définition}\peng{Difficulties, complications, hardship, overwork, great fatigue.}\end{définition}
\begin{définition}\pcmn{受累、劳累、辛苦、困难、艰难、艰苦}\end{définition}
\begin{définition}\pfra{Épuisement, misère, difficultés.}\end{définition}
\begin{exemple}\pnru{nɑ˧mi˧ tʰv̩˧!}\hspace{5pt}\peng{Hardship has come!}\hspace{5pt}\pcmn{现在是艰苦的时候! / 现在很贫困!}\hspace{5pt}\pfra{Misère! / Des difficultés surviennent, on rencontre des difficultés; on est dans une période difficile}\end{exemple}
\end{entrée}

\begin{entrée}
{nɑ˩mi\#˥}{}{ⓔnɑ˩mi\#˥}\formedesurface{nɑ˩mi˥}\newline
\classe{名词}\ton{LM+\#H}\begin{définition}\peng{Na woman.}\end{définition}
\begin{définition}\pcmn{摩梭女人}\end{définition}
\begin{définition}\pfra{Une femme Na.}\end{définition}
\end{entrée}

\begin{entrée}
{nɑ˩mv̩˥-nɑ˩dzi˩dzi˩}{}{ⓔnɑ˩mv̩˥-nɑ˩dzi˩dzi˩}\formedesurface{nɑ˩mv̩˥nɑ˩dzi˩dzi˩}\newline
\classe{形容词}\ton{H\#-}\begin{définition}\peng{All dark, quite dark (at twilight, dusk).}\end{définition}
\begin{définition}\pcmn{很暗(天变得很暗)}\end{définition}
\begin{définition}\pfra{Tout sombre, tout noir (il fait nuit noire).}\end{définition}
\end{entrée}

\begin{entrée}
{nɑ˧∼nɑ˥}{}{ⓔnɑ˧∼nɑ˥}\formedesurface{nɑ˧nɑ˥}\newline
\classe{助词}\ton{H\#}
\étymologie{
nɑ˩
}\begin{définition}\peng{Secretly.}\end{définition}
\begin{définition}\pcmn{偷偷、暗暗、私自、暗地、暗里}\end{définition}
\begin{définition}\pfra{En cachette, furtivement.}\end{définition}
\begin{exemple}\pnru{nɑ˧nɑ˥ | le˧-li˧ le˧-do˧ |}\hspace{5pt}\peng{to see (someone) for whom one was secretly on the lookout}\hspace{5pt}\pcmn{偷偷地看见}\hspace{5pt}\pfra{apercevoir (quelqu'un) qu'on guettait en cachette}\end{exemple}
\begin{exemple}\pnru{nɑ˧∼nɑ˥ se˩ |}\hspace{5pt}\peng{to walk furtively}\hspace{5pt}\pcmn{贼头贼脑地走}\hspace{5pt}\pfra{marcher furtivement}\end{exemple}
\end{entrée}

\begin{entrée}
{nɑ˩pv̩˧-qʰwɤ˧}{}{ⓔnɑ˩pv̩˧-qʰwɤ˧}\formedesurface{nɑ˩pv̩˧qʰwɤ˧}\newline
\classe{名词}\ton{LM-}\begin{définition}\peng{Emperor (borrowed from the Mongolian?).}\end{définition}
\begin{définition}\pcmn{皇帝}\end{définition}
\begin{définition}\pfra{Empereur (emprunt au mongole?).}\end{définition}
\begin{exemple}\pnru{ʈʂʰɯ˧ | nɑ˩pʰv̩˧-qʰwɤ˧-ni˩gv̩˩!}\hspace{5pt}\peng{He's got an empereror's looks! / He thinks he's the emperor! (Mocking someone who thinks he or she can impose his/her decisions to everyone, who thinks (s)he is a great leader.)}\hspace{5pt}\pcmn{他摆出做皇帝的样子! / 他以为他是皇帝吧!(嘲笑一个自以为是的人)}\hspace{5pt}\pfra{Il vous prend des airs d'empereur! / Il se prend pour l'empereur! (Façon de se moquer d'un personnage qui veut en imposer à tous, qui se prend pour un grand chef.)}\end{exemple}
\end{entrée}

\begin{entrée}
{nɑ˩tsʰi˩}{}{ⓔnɑ˩tsʰi˩}\formedesurface{nɑ˩tsʰi˩˥}\newline
\classe{名词}\ton{L}\begin{définition}\peng{Name of a mountain.}\end{définition}
\begin{définition}\pcmn{一座山的名字}\end{définition}
\begin{définition}\pfra{Nom d'une montagne de Yongning.}\end{définition}
\begin{exemple}\pnru{kɤ˧mv̩˧˥, | æ˧ʂæ˧, | ŋwɤ˧hɑ̃˩, | ʂwæ˧gv̩\#˥, | nɑ˩tsʰi˩˥ | -tɕʰɤ˧pɤ˧mi\#˥, | qv̩˧ɻ̍˧-ʈʂʰɑ˧nɑ˥ |}\hspace{5pt}\peng{The six mountains of Yongning that carry a name and have a definite symbolic value. The other mountains do not have comparable symbolic value, and fewer people use specific names for them.}\hspace{5pt}\pcmn{永宁地区有固定名字的六座山:格姆,安山,瓦哈,双古,纳慈巧吧咪,古尔川纳。}\hspace{5pt}\pfra{Les six montagnes de Yongning qui portent un nom. Les autres sommets du voisinage n'ont pas une valeur symbolique comparable, et ne portent pas de nom communément utilisé.}\end{exemple}
\end{entrée}

\begin{entrée}
{nɑ˧ʈʂʰõ˧-õ˩di˩-pɤ˩}{}{ⓔnɑ˧ʈʂʰõ˧-õ˩di˩-pɤ˩}\formedesurface{nɑ˧ʈʂʰõ˧õ˩di˩pɤ˩}\newline
\classe{名词}\ton{-L}
\paradigme{\pcmn{:} \p{}}
\begin{définition}\peng{Streamer of scriptures, attached to pillars within the house.}\end{définition}
\begin{définition}\pcmn{经幡、风马旗(挂在家里的柱子上)}\end{définition}
\begin{définition}\pfra{Drapeau de prières attaché à des piliers de la maison.}\end{définition}
\end{entrée}

\begin{entrée}
{nɑ˩zo\#˥}{}{ⓔnɑ˩zo\#˥}\formedesurface{nɑ˩zo˥}\newline
\classe{名词}\ton{LM+\#H}\begin{définition}\peng{Na man.}\end{définition}
\begin{définition}\pcmn{摩梭男人}\end{définition}
\begin{définition}\pfra{Un homme Na.}\end{définition}
\end{entrée}

\begin{entrée}
{nɑ˩-ʐwɤ˥}{}{ⓔnɑ˩-ʐwɤ˥}\formedesurface{nɑ˩ʐwɤ˥}\newline
\classe{名词}\ton{LH}\begin{définition}\peng{Autonym of the language: the Na language.}\end{définition}
\begin{définition}\pcmn{本语言:摩梭话(纳语)}\end{définition}
\begin{définition}\pfra{Langue na: endonyme de la langue na.}\end{définition}
\end{entrée}

\begin{entrée}
{næ˩-qʰæ˥ʈʂʰe˩}{}{ⓔnæ˩-qʰæ˥ʈʂʰe˩}\newline
\classe{形容词}\begin{définition}\peng{Dark all over, completely dark.}\end{définition}
\begin{définition}\pcmn{黑乎乎、很黑}\end{définition}
\begin{définition}\pfra{Noir, sombre.}\end{définition}
\end{entrée}

\begin{entrée}
{‑ne}{}{ⓔ‑ne}\formedesurface{--}\newline
\classe{后缀}\ton{0?}\begin{définition}\peng{Like, as if.}\end{définition}
\begin{définition}\pcmn{像}\end{définition}
\begin{définition}\pfra{Comme.}\end{définition}
\begin{exemple}\pnru{no˧-ɳɯ˧ hwæ˧-ne˧-ʝi˥ | zo˧!}\hspace{5pt}\peng{as if you were buying}\hspace{5pt}\pcmn{就像你来买一样!}\hspace{5pt}\pfra{c'est toi qui achètes!}\end{exemple}
\begin{exemple}\pnru{no˧-ɳɯ˧ lɑ˧˥ | -ne˧-ʝi˥ | zo˧!}\hspace{5pt}\peng{as if you were striking}\hspace{5pt}\pcmn{就像你来打一样!}\hspace{5pt}\pfra{c'est toi qui frappes!}\end{exemple}
\begin{exemple}\pnru{no˧-ɳɯ˧ pi˧-ne˧-ʝi˥ | zo˧!}\hspace{5pt}\peng{as if you decided}\hspace{5pt}\pcmn{就像你来说一样!}\hspace{5pt}\pfra{c'est comme tu dis/c'est toi qui décides!}\end{exemple}
\begin{exemple}\pnru{ʈʂʰɯ˧ne˧-ʝi˥}\hspace{5pt}\peng{thus, in this way}\hspace{5pt}\pcmn{这样}\hspace{5pt}\pfra{ainsi, de la sorte}\end{exemple}
\begin{exemple}\pnru{tʰv̩˧-kʰv̩˩-ne˩-ʝi˩-zo˩}\hspace{5pt}\peng{like that year}\hspace{5pt}\pcmn{像那年一样、就像那年}\hspace{5pt}\pfra{comme cette année-là, de la même façon que cette année-là}\end{exemple}
\begin{exemple}\pnru{tʰv̩˧-ɬi˧-ne˧-ʝi˥-zo˩}\hspace{5pt}\peng{like that month}\hspace{5pt}\pcmn{像那个月一样、就像那个月}\hspace{5pt}\pfra{comme ce mois-là, de la même façon que ce mois-là}\end{exemple}
\begin{exemple}\pnru{tʰv̩˧-hɑ̃˩-ne˩-ʝi˩-zo˩}\hspace{5pt}\peng{like that evening}\hspace{5pt}\pcmn{像那天晚上一样、就像那个晚上}\hspace{5pt}\pfra{comme ce soir-là, de la même façon que ce soir-là}\end{exemple}
\begin{exemple}\pnru{tʰv̩˧-ʂɯ˥-ne˩-ʝi˩-zo˩}\hspace{5pt}\peng{like that time}\hspace{5pt}\pcmn{像那次一样、就像那次}\hspace{5pt}\pfra{comme cette fois-là, de la même façon que cette fois-là}\end{exemple}
\begin{exemple}\pnru{tʰv̩˧ɲi˧-ne˧-ʝi˥-zo˩}\hspace{5pt}\peng{like that day}\hspace{5pt}\pcmn{像那天一样、就像那天}\hspace{5pt}\pfra{comme ce jour-là, de la même façon que ce jour-là}\end{exemple}
\end{entrée}

\begin{entrée}
{ni˥}{}{ⓔni˥}\formedesurface{ni˧}\newline
\classe{名词}\ton{\#H}
\paradigme{\pcmn{:} \p{}}
\begin{définition}\peng{Amaranth.}\end{définition}
\begin{définition}\pcmn{苋米}\end{définition}
\begin{définition}\pfra{Amaranthe, |\stylefi{Amaranthus}: minuscule graine qui n'est pas une céréale mais a une valeur nutritionnelle comparable aux céréales.}\end{définition}
\end{entrée}

\begin{entrée}
{ni˧fv̩˥}{}{ⓔni˧fv̩˥}\formedesurface{ni˧fv̩˥}\newline
\classe{名词}\ton{H\#}
\paradigme{\pcmn{:} \p{}}
\begin{définition}\peng{A very large bag, either made of leather (to carry products over long distances by caravan) or of linen (to wrap up a corpse for temporary inhumation).}\end{définition}
\begin{définition}\pcmn{大包:用来包装物品的皮包(马帮用的),或者来装尸体的麻布包(为了在火葬前暂时存放尸体)}\end{définition}
\begin{définition}\pfra{Très grand sac/très grande poche; en cuir, pour emballer les produits que l'on transportait sur de longues distances; s'emploie aussi pour désigner le sac de toile dans lequel on place le corps d'un défunt pendant son inhumation provisoire.}\end{définition}
\begin{exemple}\pnru{jɤ˧ŋɤ˧-ni˧fv̩˥}\hspace{5pt}\peng{Chengdu bag (note: this kind of large, solid bag was often purchased in the area of Chengdu, hence their association with this place name.)}\hspace{5pt}\pcmn{成都大包。(据说这类的包一般是成都地区生产的。)}\hspace{5pt}\pfra{grand sac de Chengdu. Expression employée car les sacs de ce type provenaient généralement de la région de Chengdu.}\end{exemple}
\end{entrée}

\begin{entrée}
{‑ni˧gv̩˧˥}{}{ⓔ‑ni˧gv̩˧˥}\formedesurface{ni˧gv̩˧˥}\newline
\classe{助词}\ton{MH\#}\begin{définition}\peng{To be like, to seem like.}\end{définition}
\begin{définition}\pcmn{如、像}\end{définition}
\begin{définition}\pfra{Comme (être comme, être semblable à).}\end{définition}
\begin{exemple}\pnru{zɯ˧hṽ̩˩-ni˩gv̩˩}\hspace{5pt}\peng{like grass, i.e. vivid green}\hspace{5pt}\pcmn{像草,等于绿色}\hspace{5pt}\pfra{comme de l’herbe, c'est-à-dire vert}\end{exemple}
\begin{exemple}\pnru{æ̃˧qæ˩-ni˩gv̩˩}\hspace{5pt}\peng{like a parrot (i.e. blue/green-coloured)}\hspace{5pt}\pcmn{像鹦鹉,等于青色}\hspace{5pt}\pfra{bleu-vert; littéralement ‘comme un perroquet’, c'est-à-dire ‘couleur perroquet’}\end{exemple}
\begin{exemple}\pnru{lwæ˩pʰv̩˩-ni˥gv̩˩}\hspace{5pt}\peng{like ashes, i.e. grey-coloured}\hspace{5pt}\pcmn{灰色}\hspace{5pt}\pfra{comme de la cendre = de couleur grise}\end{exemple}
\begin{exemple}\pnru{sɯ˧pv̩˩-ni˩gv̩˩}\hspace{5pt}\peng{like a urinary bladder, in the shape of a bladder}\hspace{5pt}\pcmn{像膀胱}\hspace{5pt}\pfra{comme une vessie, en forme de vessie}\end{exemple}
\begin{exemple}\pnru{(nv̩˩mi˩˥ | ) ɖɯ˧-v̩˧-ni˩gv̩˩}\hspace{5pt}\peng{(their heart is) like one, as one}\hspace{5pt}\pcmn{一条心,想得一致}\hspace{5pt}\pfra{comme un seul cœur, (leur) cœur est à l'unisson}\end{exemple}
\begin{exemple}\pnru{dzi˩bi˩-ni˩gv̩˩˥}\hspace{5pt}\peng{to be accustomed to, to get accustomed to}\hspace{5pt}\pcmn{习惯(一个环境)}\hspace{5pt}\pfra{s’habituer, s'accoutumer, prendre ses habitudes (dans un environnement)}\end{exemple}
\begin{exemple}\pnru{li˩ ʈʰɯ˩-bi˩-ni˩-gv̩˩˥}\hspace{5pt}\peng{to have a habit of drinking tea, to be a tea-drinker}\hspace{5pt}\pcmn{习惯喝茶、有喝茶的习惯}\hspace{5pt}\pfra{avoir l'habitude de boire du thé, être un buveur de thé}\end{exemple}
\begin{exemple}\pnru{ʈʂʰɯ˧ | ʂɯ˧-bi˧-ni˩gv̩˩!}\hspace{5pt}\peng{It looks like it's going to die! / It looks as if it were going to die!}\hspace{5pt}\pcmn{他好像要死了!}\hspace{5pt}\pfra{On dirait qu'il/elle va mourir!}\end{exemple}
\end{entrée}

\begin{entrée}
{ni˧mi\#˥}{}{ⓔni˧mi\#˥}\formedesurface{ni˧mi˧}\newline
\classe{名词}\ton{\#H}
\paradigme{\pcmn{:} \p{}}
\begin{définition}\peng{Sisters.}\end{définition}
\begin{définition}\pcmn{姐妹}\end{définition}
\begin{définition}\pfra{Soeurs (aînées ou cadettes).}\end{définition}
\begin{exemple}\pnru{ʈʂʰɯ˧ | ʈæ˧ʂɯ˧-bv̩˧ | ni˧mi˧ ɲi˥.}\hspace{5pt}\peng{She is /ʈæ˧ʂɯ˧/’s sister.}\hspace{5pt}\pcmn{她是达石的姐姐(或妹妹)}\hspace{5pt}\pfra{Elle est soeur de /ʈæ˧ʂɯ˧/.}\end{exemple}
\end{entrée}

\begin{entrée}
{njæ˧bæ˥}{}{ⓔnjæ˧bæ˥}\formedesurface{njæ˧bæ˥}\newline
\classe{名词}\ton{H\#}
\paradigme{\pcmn{:} \p{}}
\begin{définition}\peng{Tear.}\end{définition}
\begin{définition}\pcmn{眼泪}\end{définition}
\begin{définition}\pfra{Larme.}\end{définition}
\end{entrée}

\begin{entrée}
{njæ˩pʰv̩˧}{}{ⓔnjæ˩pʰv̩˧}\formedesurface{njæ˩pʰv̩˥}\newline
\classe{名词}\ton{LM}
\paradigme{\pcmn{:} \p{}}
\begin{définition}\peng{White of the eye.}\end{définition}
\begin{définition}\pcmn{白眼球}\end{définition}
\begin{définition}\pfra{Blanc des yeux.}\end{définition}
\end{entrée}

\begin{entrée}
{njæ˥-qv̩˩}{}{ⓔnjæ˥-qv̩˩}\formedesurface{njæ˧qv̩˩}\newline
\classe{动词}\ton{H\#-}\begin{définition}\peng{To look away from.}\end{définition}
\begin{définition}\pcmn{看别的方向(蔑视态度)}\end{définition}
\begin{définition}\pfra{Détourner le regard, détourner la tête, se détourner.}\end{définition}
\begin{exemple}\pnru{hĩ˧ njæ˧qv̩˥}\hspace{5pt}\peng{to turn away the head from, to look away from (someone that one despises, hates…)}\hspace{5pt}\pcmn{看别的方向,不直接看(蔑视态度)}\hspace{5pt}\pfra{se détourner de quelqu'un, détourner la tête face à quelqu'un (que l'on méprise, déteste…)}\end{exemple}
\begin{exemple}\pnru{mɤ˧-njæ˥qv̩˩}\hspace{5pt}\peng{|fg{neg}}\hspace{5pt}\pcmn{|fg{neg}}\hspace{5pt}\pfra{|fg{neg}: ne pas détourner le regard (face à quelqu'un)}\end{exemple}
\end{entrée}

\begin{entrée}
{njæ˩qwæ˧˥}{}{ⓔnjæ˩qwæ˧˥}\formedesurface{njæ˩qwæ˧˥}\newline
\classe{形容词}\ton{LM+MH\#}
\paradigme{\pcmn{:} \p{}}
\begin{définition}\peng{Blind.}\end{définition}
\begin{définition}\pcmn{眼睛瞎了}\end{définition}
\begin{définition}\pfra{Aveugle.}\end{définition}
\begin{exemple}\pnru{ʈʂʰɯ˧ | njæ˩qwæ˧-ze˥}\hspace{5pt}\peng{(S)he went blind.}\hspace{5pt}\pcmn{他眼睛瞎了。}\hspace{5pt}\pfra{Elle/il est devenu(e) aveugle.}\end{exemple}
\begin{exemple}\pnru{ʈʂʰɯ˧ | njæ˩qwæ˧ ɲi˥.}\hspace{5pt}\peng{(S)he is blind.}\hspace{5pt}\pcmn{他是瞎子。}\hspace{5pt}\pfra{Elle/il est aveugle.}\end{exemple}
\begin{exemple}\pnru{njæ˩qwæ˧-mi\#˥}\hspace{5pt}\peng{blind woman}\hspace{5pt}\pcmn{眼睛瞎了的女人}\hspace{5pt}\pfra{femme aveugle}\end{exemple}
\begin{exemple}\pnru{njæ˩qwæ˧-zo\#˥}\hspace{5pt}\peng{blind man}\hspace{5pt}\pcmn{眼睛瞎了的男人}\hspace{5pt}\pfra{homme aveugle}\end{exemple}
\begin{exemple}\pnru{njæ˩qwæ˧-hĩ\#˥}\hspace{5pt}\peng{blind person}\hspace{5pt}\pcmn{瞎子}\hspace{5pt}\pfra{personne aveugle}\end{exemple}
\end{entrée}

\begin{entrée}
{njæ˩qʰæ\#˥}{}{ⓔnjæ˩qʰæ\#˥}\formedesurface{njæ˩qʰæ˥}\newline
\classe{名词}\ton{LM+\#H}
\paradigme{\pcmn{:} \p{}}
\begin{définition}\peng{Eye sand, gum in the eyes, rheum.}\end{définition}
\begin{définition}\pcmn{眼屎}\end{définition}
\begin{définition}\pfra{Chassie.}\end{définition}
\end{entrée}

\begin{entrée}
{njæ˧-sɯ˩kv̩˩}{}{ⓔnjæ˧-sɯ˩kv̩˩}\formedesurface{njæ˧sɯ˩kv̩˩}\newline
\classe{代词}\ton{-L}\begin{définition}\peng{First person plural pronoun, exclusive.}\end{définition}
\begin{définition}\pcmn{我们}\end{définition}
\begin{définition}\pfra{1e personne du pluriel, exclusive.}\end{définition}
\begin{exemple}\pnru{njæ˧-sɯ˩-kv̩˩ | mɤ˧-sɯ˥!}\hspace{5pt}\peng{We (exclusive) don't know!}\hspace{5pt}\pcmn{我们不知道!}\hspace{5pt}\pfra{Nous ne savons pas! / Nous ne sommes pas au courant!}\end{exemple}
\begin{exemple}\pnru{ʈʂʰɯ˧-sɯ˩-kv̩˩, | njæ˧-sɯ˩-kv̩˩-bv̩˩ | mɤ˧-sɯ˥!}\hspace{5pt}\peng{They do not understand us! (Context: about people coming from other areas, who do not understand local custom and habits)}\hspace{5pt}\pcmn{他们不了解我们!}\hspace{5pt}\pfra{Ils ne nous comprennent pas! (Contexte: au sujet de gens venant d'autres régions, qui ne comprennent pas les us et coutumes locales)}\end{exemple}
\end{entrée}

\begin{entrée}
{njæ˧tsɯ˩}{}{ⓔnjæ˧tsɯ˩}\newline
\classe{名词}
\sens{1}\paradigme{\pcmn{:} \p{}}
\begin{définition}\peng{Eyebrow.}\end{définition}
\begin{définition}\pcmn{眉毛}\end{définition}
\begin{définition}\pfra{Sourcil.}\end{définition}
\begin{exemple}\pnru{njæ˧tsɯ˩-ɖæ˩}\hspace{5pt}\peng{eyebrow (this formulation avoids ambiguity between ‘eyebrow' and ‘eyelashes')}\hspace{5pt}\pcmn{眉毛}\hspace{5pt}\pfra{sourcil (formulation permettant de lever l'ambiguïté du terme, qui peut signifier ‘sourcil' aussi bien que ‘cil')}\end{exemple}
\begin{exemple}\pnru{njæ˧tsɯ˩ | mv̩˩tɕo˧ kʰɯ˧˥}\hspace{5pt}\peng{to knit the brows (literally “to lower the brows")}\hspace{5pt}\pcmn{皱眉毛}\hspace{5pt}\pfra{froncer les sourcils (littéralement «abaisser les sourcils»)}\end{exemple}\sens{2}
\begin{définition}\peng{Eyelashes.}\end{définition}
\begin{définition}\pcmn{睫毛、眼睫毛、眼毛}\end{définition}
\begin{définition}\pfra{Cil.}\end{définition}
\begin{exemple}\pnru{njæ˧tsɯ˩-ʂæ˩}\hspace{5pt}\pfra{cil (formulation permettant de lever l'ambiguïté du terme, qui peut signifier ‘sourcil’ aussi bien que ‘cil’)}\end{exemple}
\end{entrée}

\begin{entrée}
{njæ˧=zɯ˩}{}{ⓔnjæ˧=zɯ˩}\formedesurface{njæ˧zɯ˩}\newline
\classe{代词}\ton{L\#}\begin{définition}\peng{Dual exclusive first person pronoun: us two, the two of us (the speaker plus another person who is not the addressee).}\end{définition}
\begin{définition}\pcmn{我们两个}\end{définition}
\begin{définition}\pfra{Pronom de première personne duelle exclusive: nous deux (le locuteur et une autre personne qui n'est pas l'interlocuteur).}\end{définition}
\begin{exemple}\pnru{ɑ˩ʁo˧(-hĩ˧) | njæ˧zɯ˩ ho˩-dʑo˩!}\hspace{5pt}\peng{(We cannot stay any longer because) our family is waiting for us!}\hspace{5pt}\pcmn{(我们两个不能再呆在这里了,)家里在等我们!}\hspace{5pt}\pfra{(On ne peut pas rester car) les gens de la famille nous attendent!}\end{exemple}
\end{entrée}

\begin{entrée}
{njɤ˧˥}{₁}{ⓔnjɤ˧˥ⓗ1}\formedesurface{njɤ˧˥}\newline
\classe{动词}\ton{MH}
1\begin{définition}\peng{To glue (two objects together).}\end{définition}
\begin{définition}\pcmn{贴}\end{définition}
\begin{définition}\pfra{Coller (2 objets ensemble).}\end{définition}
\begin{exemple}\pnru{le˧-njɤ˧-ze˥!}\hspace{5pt}\peng{|fg{accomp} \_ |fg{pfv}: It's glued!}\hspace{5pt}\pcmn{粘在一起了!}\hspace{5pt}\pfra{C'est recollé! / Ca y est, c'est collé!}\end{exemple}
\begin{exemple}\pnru{tso˧∼tso˧ le˧-ɖʐɤ˧, | le˧-njɤ˧˥!}\hspace{5pt}\peng{When something (e.g. a book) is torn, (we) glue it together!}\hspace{5pt}\pcmn{东西撕破了,粘在一起(就好了)}\hspace{5pt}\pfra{(pour un livre, par ex.) Quand un truc est déchiré, on le recolle!}\end{exemple}
\end{entrée}

\begin{entrée}
{njɤ˧˥}{₂}{ⓔnjɤ˧˥ⓗ2}\formedesurface{njɤ˧˥}\newline
\classe{形容词}\ton{MH}
2\begin{définition}\peng{Sticky.}\end{définition}
\begin{définition}\pcmn{黏(胶,树脂)}\end{définition}
\begin{définition}\pfra{Poisseux, collant, visqueux (colle, résine…).}\end{définition}
\end{entrée}

\begin{entrée}
{njɤ˧˥}{₃}{ⓔnjɤ˧˥ⓗ3}\formedesurface{njɤ˧˥}\newline
\classe{形容词}\ton{MH}
3\begin{définition}\peng{Early (to rise early).}\end{définition}
\begin{définition}\pcmn{早}\end{définition}
\begin{définition}\pfra{Tôt.}\end{définition}
\end{entrée}

\begin{entrée}
{njɤ˩}{}{ⓔnjɤ˩}\formedesurface{njɤ˧}\newline
\classe{代词}\ton{L}\begin{définition}\peng{1st singular pronoun, |fg{1sg.}}\end{définition}
\begin{définition}\pcmn{我}\end{définition}
\begin{définition}\pfra{Pronom de 1e personne du singulier.}\end{définition}
\begin{exemple}\pnru{njɤ˩ ɲi˩˥!}\hspace{5pt}\peng{It's me! (Typical answer at the door)}\hspace{5pt}\pcmn{是我!(情景:一个人敲门,里面的人问是谁,人家回答:“是我!”)}\hspace{5pt}\pfra{C'est moi! (Contexte: quelqu'un frappe à la porte, on demande qui c'est, et on reçoit pour réponse: «C'est moi!»)}\end{exemple}
\begin{exemple}\pnru{njɤ˧ no˧ lɑ˧˥}\hspace{5pt}\peng{I strike you}\hspace{5pt}\pcmn{我打你}\hspace{5pt}\pfra{je te frappe}\end{exemple}
\end{entrée}

\begin{entrée}
{njɤ˩˥}{}{ⓔnjɤ˩˥}\formedesurface{njɤ˩˥}\newline
\classe{名词}\ton{LH}
\paradigme{\pcmn{:} \p{}}
\begin{définition}\peng{Eye (monosyllable).}\end{définition}
\begin{définition}\pcmn{眼睛(单音节)}\end{définition}
\begin{définition}\pfra{Œil (monosyllabe).}\end{définition}
\end{entrée}

\begin{entrée}
{njɤ˩β}{}{ⓔnjɤ˩β}\formedesurface{njɤ˩˥}\newline
\classe{动词}\ton{Lβ}\begin{définition}\peng{To husk.}\end{définition}
\begin{définition}\pcmn{舂米}\end{définition}
\begin{définition}\pfra{Décortiquer le riz.}\end{définition}
\begin{exemple}\pnru{le˧-njɤ˩-ze˩}\hspace{5pt}\peng{|fg{accomp} \_ |fg{pfv}}\hspace{5pt}\pcmn{舂了}\hspace{5pt}\pfra{|fg{accomp} \_ |fg{pfv}}\end{exemple}
\begin{exemple}\pnru{hɑ˧ njɤ˧˥}\hspace{5pt}\peng{to husk rice}\hspace{5pt}\pcmn{舂米}\hspace{5pt}\pfra{décortiquer du riz}\end{exemple}
\begin{exemple}\pnru{hɑ˧ | le˧-njɤ˩}\hspace{5pt}\peng{to husk rice}\hspace{5pt}\pcmn{舂米}\hspace{5pt}\pfra{décortiquer du riz}\end{exemple}
\begin{exemple}\pnru{hɑ˧ | ɖɯ˧-njɤ˧∼njɤ˩-ɻ̍˩}\hspace{5pt}\peng{rice - |fg{delimitative} |fg{red} |fg{inceptive}}\hspace{5pt}\pcmn{把米舂一舂}\hspace{5pt}\pfra{riz - |fg{délimitatif} \_ |fg{red} |fg{inchoatif} : décortiquer un peu le riz}\end{exemple}
\end{entrée}

\begin{entrée}
{njɤ˩bi˥}{}{ⓔnjɤ˩bi˥}\formedesurface{njɤ˩bi˥}\newline
\classe{名词}\ton{LH}
\paradigme{\pcmn{:} \p{}}
\begin{définition}\peng{Eyelid (top eyelid).}\end{définition}
\begin{définition}\pcmn{上眼皮}\end{définition}
\begin{définition}\pfra{Paupière supérieure.}\end{définition}
\end{entrée}

\begin{entrée}
{njɤ˧di˧˥}{}{ⓔnjɤ˧di˧˥}\formedesurface{njɤ˧di˧˥}\newline
\classe{名词}\ton{MH\#}
\paradigme{\pcmn{:} \p{}}
\begin{définition}\peng{Glue.}\end{définition}
\begin{définition}\pcmn{胶}\end{définition}
\begin{définition}\pfra{Colle.}\end{définition}
\end{entrée}

\begin{entrée}
{njɤ˩-gɤ˧lɑ˩}{}{ⓔnjɤ˩-gɤ˧lɑ˩}\formedesurface{njɤ˩gɤ˧lɑ˩}\newline
\classe{名词}\ton{L-L\#}
\paradigme{\pcmn{:} \p{}}
\begin{définition}\peng{Eyeball.}\end{définition}
\begin{définition}\pcmn{眼珠}\end{définition}
\begin{définition}\pfra{Prunelle.}\end{définition}
\end{entrée}

\begin{entrée}
{njɤ˧kv̩˩}{}{ⓔnjɤ˧kv̩˩}\formedesurface{njɤ˧kv̩˩}\newline
\classe{名词}\ton{L\#}
\paradigme{\pcmn{:} \p{}}
\begin{définition}\peng{Cheekbone.}\end{définition}
\begin{définition}\pcmn{颧骨}\end{définition}
\begin{définition}\pfra{Pommettes.}\end{définition}
\end{entrée}

\begin{entrée}
{njɤ˧kv̩˩-njɤ˩tsʰɤ˩}{}{ⓔnjɤ˧kv̩˩-njɤ˩tsʰɤ˩}\formedesurface{njɤ˧kv̩˩njɤ˩tsʰɤ˩}\newline
\classe{形容词}\ton{L\#-}\begin{définition}\peng{Beautiful; with a pretty face (of a woman).}\end{définition}
\begin{définition}\pcmn{美丽、面貌美}\end{définition}
\begin{définition}\pfra{Belle; qui a un beau visage, qui a des traits gracieux.}\end{définition}
\begin{exemple}\pnru{ə˧mi˧! | mv̩˩zo˩ ʈʂʰɯ˩-ɭɯ˥ | njɤ˧kv̩˩-njɤ˩tsʰɤ˩! | ɖwæ˧˥ | ə˧v̩˧˥!}\hspace{5pt}\peng{Wow! This young lady is really beautiful! Very pretty!}\hspace{5pt}\pcmn{啊呀,这个少女真美丽!很漂亮!}\hspace{5pt}\pfra{Eh bien, cette jeune fille est vraiment belle! Très jolie!}\end{exemple}
\begin{exemple}\pnru{njɤ˧kv̩˩njɤ˩tsʰɤ˩ | ʐwæ˩˥}\hspace{5pt}\peng{extremely beautiful}\hspace{5pt}\pcmn{非常美}\hspace{5pt}\pfra{particulièrement belle}\end{exemple}
\end{entrée}

\begin{entrée}
{njɤ˩kʰi\#˥}{}{ⓔnjɤ˩kʰi\#˥}\formedesurface{njɤ˩kʰi˥}\newline
\classe{名词}\ton{LM+\#H}
\paradigme{\pcmn{:} \p{}}
\begin{définition}\peng{Bottom eyelid.}\end{définition}
\begin{définition}\pcmn{下眼皮}\end{définition}
\begin{définition}\pfra{Paupière inférieure.}\end{définition}
\end{entrée}

\begin{entrée}
{njɤ˧lɑ˥}{}{ⓔnjɤ˧lɑ˥}\formedesurface{njɤ˧lɑ˥}\newline
\classe{助词}\ton{H\#}\begin{définition}\peng{Early.}\end{définition}
\begin{définition}\pcmn{早、黎明的时候}\end{définition}
\begin{définition}\pfra{Tôt.}\end{définition}
\begin{exemple}\pnru{njɤ˧lɑ˥ | gɤ˩-ʈi˧}\hspace{5pt}\peng{to rise early}\hspace{5pt}\pcmn{起得早}\hspace{5pt}\pfra{se lever tôt}\end{exemple}
\end{entrée}

\begin{entrée}
{njɤ˧le˧gv̩\#˥}{}{ⓔnjɤ˧le˧gv̩\#˥}\formedesurface{njɤ˧le˧gv̩˧}\newline
\classe{名词}\ton{\#H}\begin{définition}\peng{Daytime.}\end{définition}
\begin{définition}\pcmn{白天、大白天}\end{définition}
\begin{définition}\pfra{Journée, plein jour.}\end{définition}
\begin{exemple}\pnru{ɲi˧mi˧-njɤ˩le˩gv̩˩}\hspace{5pt}\peng{daytime}\hspace{5pt}\pcmn{白天}\hspace{5pt}\pfra{même sens}\end{exemple}
\end{entrée}

\begin{entrée}
{njɤ˩ɭɯ˧}{}{ⓔnjɤ˩ɭɯ˧}\formedesurface{njɤ˩ɭɯ˥}\newline
\classe{名词}\ton{LM}
\paradigme{\pcmn{:} \p{}}
\begin{définition}\peng{Eye.}\end{définition}
\begin{définition}\pcmn{眼睛}\end{définition}
\begin{définition}\pfra{Œil.}\end{définition}
\end{entrée}

\begin{entrée}
{njɤ˧mv̩˥}{}{ⓔnjɤ˧mv̩˥}\formedesurface{njɤ˧mv̩˥}\newline
\classe{名词}\ton{H\#}
\paradigme{\pcmn{:} \p{}}
\begin{définition}\peng{A plant used as fodder for the pigs, |\stylefi{Chenopodium album}.}\end{définition}
\begin{définition}\pcmn{灰条菜、灰灰菜:喂猪的牧草}\end{définition}
\begin{définition}\pfra{Sorte de fourrage pour les cochons, |\stylefi{Chenopodium album}. (Il y a en tout trois sortes de fourrage pour les cochons.).}\end{définition}
\end{entrée}

\begin{entrée}
{njɤ˧mv̩˥-mi˩}{}{ⓔnjɤ˧mv̩˥-mi˩}\formedesurface{njɤ˧mv̩˥mi˩}\newline
\classe{名词}\ton{H\#-L}
\paradigme{\pcmn{:} \p{}}
\begin{définition}\peng{Camel.}\end{définition}
\begin{définition}\pcmn{骆驼}\end{définition}
\begin{définition}\pfra{Chameau.}\end{définition}
\begin{exemple}\pnru{njɤ˧mv̩˥mi˩-zo˩}\hspace{5pt}\peng{baby camel}\hspace{5pt}\pcmn{小骆驼}\hspace{5pt}\pfra{enfant du chameau, petit chameau}\end{exemple}
\begin{exemple}\pnru{njɤ˧mv̩˥mi˩-pʰv̩˩}\hspace{5pt}\peng{male camel}\hspace{5pt}\pcmn{公骆驼}\hspace{5pt}\pfra{chameau mâle}\end{exemple}
\end{entrée}

\begin{entrée}
{njɤ˧nɑ˩}{}{ⓔnjɤ˧nɑ˩}\formedesurface{njɤ˧nɑ˩}\newline
\classe{名词}\ton{L\#}
\paradigme{\pcmn{:} \p{}}
\begin{définition}\peng{Eyeball.}\end{définition}
\begin{définition}\pcmn{眼珠}\end{définition}
\begin{définition}\pfra{Prunelle.}\end{définition}
\end{entrée}

\begin{entrée}
{njɤ˩qʰwɤ˧˥}{}{ⓔnjɤ˩qʰwɤ˧˥}\formedesurface{njɤ˩qʰwɤ˧˥}\newline
\classe{名词}\ton{LM+MH\#}
\paradigme{\pcmn{:} \p{}}
\begin{définition}\peng{Orbit; eye socket.}\end{définition}
\begin{définition}\pcmn{眼眶}\end{définition}
\begin{définition}\pfra{Orbite (de l'œil).}\end{définition}
\end{entrée}

\begin{entrée}
{njɤ˩-tse˧∼tse˩}{}{ⓔnjɤ˩-tse˧∼tse˩}\formedesurface{njɤ˩tse˧tse˩}\newline
\classe{名词}\ton{L-L\#}
\paradigme{\pcmn{:} \p{}}
\begin{définition}\peng{Branched horsetail, |\stylefi{Equisetum ramosissimum Desf.} This is a wild herb used in traditional medicine; its stem consists of small segments; when pulled/plucked, the stem breaks at one of these articulations.}\end{définition}
\begin{définition}\pcmn{节节草}\end{définition}
\begin{définition}\pfra{Prêle ramifiée, prêle rameuse, |\stylefi{Equisetum ramosissimum Desf.} Herbe sauvage, utilisée en pharmacopée traditionnelle; sa tige est divisée en petits segments, et elle se brise à l'une de ces articulations si on l'arrache.}\end{définition}
\end{entrée}

\begin{entrée}
{njɤ˧ʈʂɤ˥}{}{ⓔnjɤ˧ʈʂɤ˥}\formedesurface{njɤ˧ʈʂɤ˥}\newline
\classe{名词}\ton{H\#}\begin{définition}\peng{Black-jack, beggar-ticks, cobbler's pegs, Spanish needle, |\stylefi{Bidens pilosa L.}, a species of flowering plant in the aster family. The barbed awns of the seeds catch onto fur or clothing, and can injure flesh.}\end{définition}
\begin{définition}\pcmn{鬼针草}\end{définition}
\begin{définition}\pfra{Sornet, herbe à aiguilles, |\stylefi{Bidens pilosa L.}: plante de la famille des Asteraceae, dont les graines noires, fines et allongées, de 5 à 10 mm, s'accrochent aux vêtements et aux poils d'animaux par deux piquants fins, situés à l'une de leurs extrémités.}\end{définition}
\end{entrée}

\begin{entrée}
{njɤ˩ʈʂv̩˧˥}{}{ⓔnjɤ˩ʈʂv̩˧˥}\formedesurface{njɤ˩ʈʂv̩˧˥}\newline
\classe{名词}\ton{LM+MH\#}
\paradigme{\pcmn{:} \p{}}
\begin{définition}\peng{Loach (a kind of fish).}\end{définition}
\begin{définition}\pcmn{泥鳅}\end{définition}
\begin{définition}\pfra{Loche (poisson).}\end{définition}
\end{entrée}

\begin{entrée}
{njo˥}{}{ⓔnjo˥}\formedesurface{njo˧}\newline
\classe{名词}\ton{\#H}\begin{définition}\peng{Cucurbit flute, hulusi: a free reed wind instrument.}\end{définition}
\begin{définition}\pcmn{葫芦丝、葫芦箫}\end{définition}
\begin{définition}\pfra{Flûte à calebasse, hulusi: instrument à vent à anche libre.}\end{définition}
\begin{exemple}\pnru{njo˧ mv̩˥}\hspace{5pt}\peng{to play the cucurbit flute}\hspace{5pt}\pcmn{吹响葫芦丝}\hspace{5pt}\pfra{jouer de la flûte à calebasse}\end{exemple}
\end{entrée}

\begin{entrée}
{njo˧}{}{ⓔnjo˧}\formedesurface{njo˧}\newline
\classe{名词}\ton{M}\begin{définition}\peng{Ear (of wheat, barley).}\end{définition}
\begin{définition}\pcmn{谷穗}\end{définition}
\begin{définition}\pfra{Épi (de blé, d'orge, de riz…).}\end{définition}
\begin{exemple}\pnru{hɑ˧-njo˩}\hspace{5pt}\peng{ear of cereals}\hspace{5pt}\pcmn{谷穗}\hspace{5pt}\pfra{épi de céréales}\end{exemple}
\begin{exemple}\pnru{mv̩˧dze˧-njo˧ (+ɲi˩)}\hspace{5pt}\peng{ear of barley}\hspace{5pt}\pcmn{大麦穗}\hspace{5pt}\pfra{épi d'orge}\end{exemple}
\begin{exemple}\pnru{tsʰi˧zi˧-hɑ˧njo˥ (+ɲi˩)}\hspace{5pt}\peng{ear of highland barley}\hspace{5pt}\pcmn{青稞穗}\hspace{5pt}\pfra{épi d'orge d'altitude}\end{exemple}
\end{entrée}

\begin{entrée}
{njo˩˥}{}{ⓔnjo˩˥}\formedesurface{njo˩˥}\newline
\classe{名词}\ton{LH}\begin{définition}\peng{Milk.}\end{définition}
\begin{définition}\pcmn{奶汁}\end{définition}
\begin{définition}\pfra{Lait.}\end{définition}
\begin{exemple}\pnru{njo˩ ki˧}\hspace{5pt}\peng{to breast-feed (literally ‘to give milk')}\hspace{5pt}\pcmn{给(喂)奶}\hspace{5pt}\pfra{donner le sein, donner la tétée, nourrir (un nourrisson); littéralement «donner du lait»}\end{exemple}
\begin{exemple}\pnru{njo˩ ʈʰɯ˩˥}\hspace{5pt}\peng{to drink milk}\hspace{5pt}\pcmn{喝奶}\hspace{5pt}\pfra{boire du lait}\end{exemple}
\end{entrée}

\begin{entrée}
{njo˩bi˥}{}{ⓔnjo˩bi˥}\formedesurface{njo˩bi˥}\newline
\classe{名词}\ton{LH}
\paradigme{\pcmn{:} \p{}}
\begin{définition}\peng{Breast.}\end{définition}
\begin{définition}\pcmn{乳房}\end{définition}
\begin{définition}\pfra{Sein, mamelle.}\end{définition}
\begin{exemple}\pnru{ʝi˧-njo˥bi˩}\hspace{5pt}\peng{cow's breast}\hspace{5pt}\pcmn{牛的奶头}\hspace{5pt}\pfra{mamelle de la vache}\end{exemple}
\begin{exemple}\pnru{njo˩bi˧-ʁo˧qʰwɤ˩}\hspace{5pt}\peng{nipple, teat}\hspace{5pt}\pcmn{乳头}\hspace{5pt}\pfra{téton}\end{exemple}
\end{entrée}

\begin{entrée}
{njo˧bi˧li˥}{}{ⓔnjo˧bi˧li˥}\formedesurface{njo˧bi˧li˥}\newline
\classe{名词}\ton{H\#}
\paradigme{\pcmn{:} \p{}}
\begin{définition}\peng{Lips.}\end{définition}
\begin{définition}\pcmn{嘴唇}\end{définition}
\begin{définition}\pfra{Lèvres.}\end{définition}
\end{entrée}

\begin{entrée}
{njo˩kæ˧-tɕi˩˥}{}{ⓔnjo˩kæ˧-tɕi˩˥}\formedesurface{njo˩kæ˧-tɕi˩˥}\newline
\classe{名词}\ton{LM-LH}\begin{définition}\peng{Cep, |\stylefi{Boletus edulis}.}\end{définition}
\begin{définition}\pcmn{牛肝菌(汉语借词)}\end{définition}
\begin{définition}\pfra{Bolet, cèpe, |\stylefi{Boletus edulis}.}\end{définition}
\end{entrée}

\begin{entrée}
{njo˩pɤ˩lv̩˥}{}{ⓔnjo˩pɤ˩lv̩˥}\formedesurface{njo˩pɤ˩lv̩˥}\newline
\classe{名词}\ton{L+H\#}
\paradigme{\pcmn{:} \p{}}
\begin{définition}\peng{Udder.}\end{définition}
\begin{définition}\pcmn{牛的奶头}\end{définition}
\begin{définition}\pfra{Pis de la vache.}\end{définition}
\end{entrée}

\begin{entrée}
{‑no˧˥}{}{ⓔ‑no˧˥}\formedesurface{no˧˥}\newline
\classe{语气助词}\ton{MH}\begin{définition}\peng{Contrastive topic.}\end{définition}
\begin{définition}\pcmn{主题:对于、关于}\end{définition}
\begin{définition}\pfra{Topique contrastif. Gloses possibles: …, en revanche, …; …, pour sa part, …; quant à …}\end{définition}
\begin{exemple}\pnru{qæ˧do˧ | -no˧˥}\hspace{5pt}\peng{As for lumber, …}\hspace{5pt}\pcmn{关于木材,……}\hspace{5pt}\pfra{en ce qui concerne le bois de construction, …}\end{exemple}
\begin{exemple}\pnru{ʑi˧qʰwɤ˧ | -no˧˥}\hspace{5pt}\peng{As for the main room, …}\hspace{5pt}\pcmn{关于主屋……}\hspace{5pt}\pfra{en ce qui concerne la pièce principale, …}\end{exemple}
\end{entrée}

\begin{entrée}
{no˩}{}{ⓔno˩}\formedesurface{no˧}\newline
\classe{代词}\ton{L}\begin{définition}\peng{Second person singular pronoun.}\end{définition}
\begin{définition}\pcmn{你}\end{définition}
\begin{définition}\pfra{Pronom de deuxième personne du singulier.}\end{définition}
\begin{exemple}\pnru{no˩ ɲi˩˥}\hspace{5pt}\peng{It's you!}\hspace{5pt}\pcmn{是你!}\hspace{5pt}\pfra{c'est toi!}\end{exemple}
\end{entrée}

\begin{entrée}
{no˩α}{}{ⓔno˩α}\formedesurface{no˩˥}\newline
\classe{动词}\ton{Lα}\begin{définition}\peng{To add, to blend in, to mix.}\end{définition}
\begin{définition}\pcmn{搀}\end{définition}
\begin{définition}\pfra{Mélanger, ajouter.}\end{définition}
\begin{exemple}\pnru{ɳæ˧ | tsɑ˧bɤ˧-qo˧ tʰi˧-no˩}\hspace{5pt}\peng{to mix grilled flour with milk, to add milk to grilled flour}\hspace{5pt}\pcmn{在糌粑里搀奶、糌粑里搀奶}\hspace{5pt}\pfra{mettre du lait dans la farine grillée, mélanger la farine grillée avec du lait}\end{exemple}
\begin{exemple}\pnru{le˧-no˩}\hspace{5pt}\peng{|fg{accomp} \_}\hspace{5pt}\pcmn{|fg{accomp} \_}\hspace{5pt}\pfra{|fg{accomp} \_}\end{exemple}
\begin{exemple}\pnru{hɑ˧-qo˩ | tɕæ˧ɻæ˩ tʰi˩-no˩: hɑ˧-qo˩ | tɕæ˧ɻæ˩ tʰi˩-kʰɯ˩}\hspace{5pt}\peng{a paraphrase to explain the verb's meaning: ‘to blend pickled vegetables into the food/rice, that means: to add picked vegetables to the food.'}\hspace{5pt}\pcmn{关于这个动词的说明:‘饭里面搀泡菜,就是说:在饭里面放泡菜。’}\hspace{5pt}\pfra{paraphrase pour expliquer le sens du verbe: ‘ajouter des légumes en saumure dans la nourriture, ça veut dire: mettre des légumes en saumure dans la nourriture.'}\end{exemple}
\begin{exemple}\pnru{hɑ˧-qo˩ | tɕæ˧ɻæ˩ tʰi˩-no˩: tɕæ˧ɻæ˩-lɑ˩ | hɑ˧ | ɖɯ˧-tɕʰo˩ dzɯ˩}\hspace{5pt}\peng{a paraphrase to explain the verb's meaning: ‘to blend pickled vegetables into the food/rice, that means: to eat picked vegetables and food/rice together.'}\hspace{5pt}\pcmn{关于这个动词的说明:‘饭里面搀泡菜,就是说:把泡菜和饭一起吃。’}\hspace{5pt}\pfra{paraphrase pour expliquer le sens du verbe: ‘ajouter des légumes en saumure dans la nourriture, ça veut dire: manger des légumes en saumure et de la nourriture (du riz) ensemble.'}\end{exemple}
\end{entrée}

\begin{entrée}
{no˩bv̩˧}{}{ⓔno˩bv̩˧}\formedesurface{no˩bv̩˥}\newline
\classe{名词}\ton{LM}\begin{définition}\peng{Masculine given name.}\end{définition}
\begin{définition}\pcmn{男性名字}\end{définition}
\begin{définition}\pfra{Prénom masculin.}\end{définition}
\end{entrée}

\begin{entrée}
{no˧no˧}{}{ⓔno˧no˧}\formedesurface{no˧no˧}\newline
\classe{名词}\ton{M}\begin{définition}\peng{Feminine given name.}\end{définition}
\begin{définition}\pcmn{女性名字}\end{définition}
\begin{définition}\pfra{Prénom féminin.}\end{définition}
\end{entrée}

\begin{entrée}
{no˧no˧-ɖɯ˩mɑ˩}{}{ⓔno˧no˧-ɖɯ˩mɑ˩}\newline
\classe{名词}\ton{-L}\begin{définition}\peng{Feminine given name.}\end{définition}
\begin{définition}\pcmn{女性名字}\end{définition}
\begin{définition}\pfra{Prénom féminin.}\end{définition}
\end{entrée}

\begin{entrée}
{no˩qo˥}{}{ⓔno˩qo˥}\formedesurface{no˩qo˥}\newline
\classe{助词}\ton{LH}\begin{définition}\peng{Close to, next to.}\end{définition}
\begin{définition}\pcmn{……附近}\end{définition}
\begin{définition}\pfra{À proximité de, à côté de.}\end{définition}
\end{entrée}

\begin{entrée}
{no˧=ɻ̍˩}{}{ⓔno˧=ɻ̍˩}\formedesurface{no˧ɻ̍˩}\newline
\classe{代词}\ton{L\#}\begin{définition}\peng{Second person plural.}\end{définition}
\begin{définition}\pcmn{你们}\end{définition}
\begin{définition}\pfra{Deuxième personne du pluriel.}\end{définition}
\end{entrée}

\begin{entrée}
{no˧-sɯ˩kv̩˩}{}{ⓔno˧-sɯ˩kv̩˩}\formedesurface{no˧sɯ˩kv̩˩}\newline
\classe{代词}\ton{-L}\begin{définition}\peng{Second-person plural pronoun.}\end{définition}
\begin{définition}\pcmn{你们}\end{définition}
\begin{définition}\pfra{Pronom de deuxième personne du pluriel.}\end{définition}
\begin{exemple}\pnru{no˧-sɯ˩-kv̩˩ | mɤ˧-sɯ˥!}\hspace{5pt}\peng{You (plural) don't know (about this)!}\hspace{5pt}\pcmn{你们不知道!}\hspace{5pt}\pfra{vous ne savez pas!}\end{exemple}
\begin{exemple}\pnru{njɤ˧ | no˧-sɯ˩-kv̩˩ | mɤ˧-sɯ˥!}\hspace{5pt}\peng{I don't know you! / I am not familiar with your situation/business!}\hspace{5pt}\pcmn{我不认识你们! / 我不熟悉你们!}\hspace{5pt}\pfra{je ne vous connais pas!}\end{exemple}
\begin{exemple}\pnru{njɤ˧ | no˧-sɯ˩-kv̩˩ lɑ˩ / njɤ˧ | no˧-sɯ˩-kv̩˩ lɑ˩-bi˩}\hspace{5pt}\peng{I strike you (|fg{pl}) / I am going to strike you}\hspace{5pt}\pcmn{我打你们 / 我要打你们}\hspace{5pt}\pfra{je vous frappe / je vais vous frapper}\end{exemple}
\begin{exemple}\pnru{no˧-sɯ˩-kv̩˩ lɑ˩-bi˩}\hspace{5pt}\peng{you are going to strike}\hspace{5pt}\pcmn{你们要打了}\hspace{5pt}\pfra{vous allez frapper}\end{exemple}
\begin{exemple}\pnru{no˧-sɯ˩-kv̩˩ | sɯ˧!}\hspace{5pt}\peng{you know, you are aware (of this)}\hspace{5pt}\pcmn{你们知道}\hspace{5pt}\pfra{vous savez, vous êtes au courant}\end{exemple}
\begin{exemple}\pnru{no˧-sɯ˩-kv̩˩ | le˧-li˧}\hspace{5pt}\peng{you are looking}\hspace{5pt}\pcmn{你们在看}\hspace{5pt}\pfra{vous regardez}\end{exemple}
\begin{exemple}\pnru{no˧-sɯ˩-kv̩˩ | li˧-bi˧!}\hspace{5pt}\peng{you are going to look}\hspace{5pt}\pcmn{你们去看吧!}\hspace{5pt}\pfra{vous allez regarder}\end{exemple}
\end{entrée}

\begin{entrée}
{no˧=zɯ˩}{}{ⓔno˧=zɯ˩}\formedesurface{no˧zɯ˩}\newline
\classe{代词}\ton{L\#}\begin{définition}\peng{Dual second person pronoun: you two.}\end{définition}
\begin{définition}\pcmn{你们俩}\end{définition}
\begin{définition}\pfra{Pronom personnel de deuxième personne duel: vous deux.}\end{définition}
\end{entrée}

\begin{entrée}
{no˩zɯ˧˥}{}{ⓔno˩zɯ˧˥}\formedesurface{no˩zɯ˧˥}\newline
\classe{代词}\ton{LM+MH\#}\begin{définition}\peng{Dual second person pronoun: you two.}\end{définition}
\begin{définition}\pcmn{你们俩}\end{définition}
\begin{définition}\pfra{Pronom personnel de deuxième personne duel: vous deux.}\end{définition}
\end{entrée}

\begin{entrée}
{nv̩˥}{₁}{ⓔnv̩˥ⓗ1}\formedesurface{nv̩˧}\newline
\classe{动词}\ton{H}
1\begin{définition}\peng{To chase after, to pursue.}\end{définition}
\begin{définition}\pcmn{追赶}\end{définition}
\begin{définition}\pfra{Suivre à la trace, pister.}\end{définition}
\begin{exemple}\pnru{le˧-nv̩˥}\hspace{5pt}\peng{|fg{accomp}}\hspace{5pt}\pcmn{|fg{accomp}}\hspace{5pt}\pfra{|fg{accomp}}\end{exemple}
\begin{exemple}\pnru{ʈʂɤ˩nv̩˩}\hspace{5pt}\peng{to chase after, to pursue}\hspace{5pt}\pcmn{追赶}\hspace{5pt}\pfra{suivre à la trace, pister}\end{exemple}
\begin{exemple}\pnru{le˧-ʈʂɤ˩nv̩˩}\hspace{5pt}\peng{to chase after, to pursue}\hspace{5pt}\pcmn{追赶}\hspace{5pt}\pfra{suivre à la trace, pister}\end{exemple}
\begin{exemple}\pnru{le˧-ʈʂɤ˩nv̩˩ | le˧-hɯ˩}\hspace{5pt}\peng{He went to chase after}\hspace{5pt}\pcmn{追赶去了}\hspace{5pt}\pfra{Il est parti à la poursuite de / à la chasse de}\end{exemple}
\end{entrée}

\begin{entrée}
{nv̩˥}{₂}{ⓔnv̩˥ⓗ2}\formedesurface{nv̩˧}\newline
\classe{动词}\ton{H}
2\begin{définition}\peng{To bury.}\end{définition}
\begin{définition}\pcmn{埋}\end{définition}
\begin{définition}\pfra{Enterrer.}\end{définition}
\end{entrée}

\begin{entrée}
{nv̩˩dʑɯ˥}{}{ⓔnv̩˩dʑɯ˥}\formedesurface{nv̩˩dʑɯ˥}\newline
\classe{名词}\ton{LH}
\paradigme{\pcmn{:} \p{}}
\begin{définition}\peng{Tofu, bean curd.}\end{définition}
\begin{définition}\pcmn{豆腐}\end{définition}
\begin{définition}\pfra{Tofu.}\end{définition}
\end{entrée}

\begin{entrée}
{nv̩˩ho\#˥}{}{ⓔnv̩˩ho\#˥}\formedesurface{nv̩˩ho˥}\newline
\classe{名词}\ton{LM+\#H}
\paradigme{\pcmn{:} \p{}}
\begin{définition}\peng{Long-boiled soft beancurd.}\end{définition}
\begin{définition}\pcmn{豆花}\end{définition}
\begin{définition}\pfra{Tofu léger, longuement bouilli.}\end{définition}
\end{entrée}

\begin{entrée}
{nv̩˧hṽ̩˩}{}{ⓔnv̩˧hṽ̩˩}\formedesurface{nv̩˧hṽ̩˩}\newline
\classe{名词}\ton{L\#}
\paradigme{\pcmn{:} \p{}}
\begin{définition}\peng{Bean; string bean, kidney bean.}\end{définition}
\begin{définition}\pcmn{豆子,四季豆,花腰豆}\end{définition}
\begin{définition}\pfra{Haricot: terme générique.}\end{définition}
\end{entrée}

\begin{entrée}
{nv̩˧hṽ̩˩-bi˩bi˩}{}{ⓔnv̩˧hṽ̩˩-bi˩bi˩}\formedesurface{nv̩˧hṽ̩˩bi˩bi˩}\newline
\classe{名词}\ton{L\#-}
\paradigme{\pcmn{:} \p{}}
\begin{définition}\peng{Green bean, snap bean, string bean; one consumes the pod with the seed inside.}\end{définition}
\begin{définition}\pcmn{四季豆、玉豆、帶莢豌豆、菜豆、刀豆、豆角、敏豆仔、敏豆}\end{définition}
\begin{définition}\pfra{Haricot vert; on consomme la cosse fraîche et la graine qu'il contient.}\end{définition}
\end{entrée}

\begin{entrée}
{nv̩˩ɭɯ˧}{}{ⓔnv̩˩ɭɯ˧}\formedesurface{nv̩˩ɭɯ˥}\newline
\classe{名词}\ton{LM}
\paradigme{\pcmn{:} \p{}}
\begin{définition}\peng{Soy beans, soya beans.}\end{définition}
\begin{définition}\pcmn{黄豆}\end{définition}
\begin{définition}\pfra{Soja.}\end{définition}
\end{entrée}

\begin{entrée}
{nv̩˩mi˩}{}{ⓔnv̩˩mi˩}\formedesurface{nv̩˩mi˩˥}\newline
\classe{名词}
\sens{1}\paradigme{\pcmn{:} \p{}}
\begin{définition}\peng{Heart.}\end{définition}
\begin{définition}\pcmn{心脏}\end{définition}
\begin{définition}\pfra{Cœur.}\end{définition}
\begin{exemple}\pnru{hĩ˧ ʈʂʰɯ˧-v̩˧, | nv̩˩mi˩ tɕi˥! |}\hspace{5pt}\peng{This person lacks courage! (literally “…(his/her) heart is small")}\hspace{5pt}\pcmn{这个人,胆小!(直译:“心小”)}\hspace{5pt}\pfra{Celui-là, il manque de courage! (littéralement «(a) un petit coeur», «son coeur est petit»)}\end{exemple}
\begin{exemple}\pnru{nv̩˩mi˩˥ | ɖɯ˧-ɭɯ˧ tsɤ˧ |}\hspace{5pt}\peng{in sympathy, in unison}\hspace{5pt}\pcmn{情投意合}\hspace{5pt}\pfra{être en sympathie, à l'unisson}\end{exemple}
\begin{exemple}\pnru{nv̩˩mi˩˥ | tʰi˧-tɕɯ˥ | so˩˥}\hspace{5pt}\peng{to study patiently / to teach patiently}\hspace{5pt}\pcmn{耐心地学习 / 耐心地教}\hspace{5pt}\pfra{enseigner patiemment/apprendre patiemment}\end{exemple}
\begin{exemple}\pnru{nv̩˩mi˩-qo˥ | tʰi˧-ʑi˥}\hspace{5pt}\peng{to remember, to keep in mind}\hspace{5pt}\pcmn{记住、记得(直译:‘心里存着’、‘心里有’)}\hspace{5pt}\pfra{se souvenir de, garder à l'esprit, avoir à l'esprit}\end{exemple}
\begin{exemple}\pnru{nv̩˩mi˩-qo˥ | tʰi˧-kʰɯ˧˥}\hspace{5pt}\peng{to make an effort to remember, to carry in mind}\hspace{5pt}\pcmn{记住(直译:‘放在心里’)}\hspace{5pt}\pfra{faire l'effort de se souvenir, garder à l'esprit, garder en mémoire}\end{exemple}\sens{2}
\begin{définition}\peng{State of mind.}\end{définition}
\begin{définition}\pcmn{心情}\end{définition}
\begin{définition}\pfra{État d'esprit.}\end{définition}
\begin{exemple}\pnru{nv̩˩mi˩ dzɯ˩∼dzɯ˩-ɻ̍˥}\hspace{5pt}\peng{not to get along well; to quarrel all the time; to poison each other's life}\hspace{5pt}\pcmn{经常吵架、过不到一起去}\hspace{5pt}\pfra{ne pas bien s'entendre; se chamailler sans répit; s'empoisonner mutuellement la vie, se faire la vie impossible, se bouffer le nez}\end{exemple}
\end{entrée}

\begin{entrée}
{nv̩˩mi˩-ɖɯ˩}{}{ⓔnv̩˩mi˩-ɖɯ˩}\formedesurface{nv̩˩mi˩ɖɯ˩˥}\newline
\classe{形容词}\ton{L}
\étymologie{
nv̩˩mi˩; ɖɯ˩a
}\begin{définition}\peng{Courageous, brave.}\end{définition}
\begin{définition}\pcmn{勇敢、有勇气的}\end{définition}
\begin{définition}\pfra{Courageux, audacieux.}\end{définition}
\begin{exemple}\pnru{ʈʂʰɯ˧ | nv̩˩mi˩˥ | ɖwæ˧˥ | ɖɯ˩˥! | hĩ˧ | mɤ˧-ɖwæ˥!}\hspace{5pt}\peng{(S)he is very brave! (S)he is not afraid of others / (s)he fears no one!}\hspace{5pt}\pcmn{他很勇敢!谁也不怕!}\hspace{5pt}\pfra{il est très courageux! il n'a peur de personne!}\end{exemple}
\begin{exemple}\pnru{pɤ˩mi˩˥ | nv̩˩mi˩ ɖɯ˩˥, | ʝi˧-ɳɯ˧ tʰv̩˧˥; | mi˩zɯ˩ nv̩˥mi˩ ɖɯ˩ (-dʑo˩), | hĩ˧-ɳɯ˧ lɑ˧˥!}\hspace{5pt}\peng{“If the frog is brave, it gets stamped on by the ox; if the woman is brave, she gets beaten!" (Explanation: weaker creatures must not be too brave: if a frog fears nothing and ventures onto the roads, it can easily get crushed to death; if a woman behaves with masculine self-assurance and courage, she gets into situations where people come to hands, and she eventually has the lower hand.)}\hspace{5pt}\pcmn{“勇敢的青蛙,被牛压死。勇敢的女人,被人家打!”(说明:青蛙太勇敢,上马路,就容易被压死,而女人太勇敢,容易跟别人发生矛盾,最后就打不过男人。)}\hspace{5pt}\pfra{«La grenouille courageuse, elle se fait écraser par le bœuf; la femme courageuse, elle se fait frapper!» (Explication: les créatures qui ne sont pas les plus fortes doivent se garder d'être trop courageuses: la grenouille qui n'a peur de rien et s'aventure sur le grand chemin, elle se fait écraser; la femme qui se comporte avec une mâle assurance, elle finit par entrer dans des conflits où on en vient aux mains et où elle a finalement le dessous.)}\end{exemple}
\end{entrée}

\begin{entrée}
{nv̩˩mi˩-ki˧ki˩}{}{ⓔnv̩˩mi˩-ki˧ki˩}\formedesurface{nv̩˩mi˩ki˧ki˩}\newline
\classe{形容词}\ton{L-L\#}
\étymologie{
nv̩˩mi˩; ki˧a
}\begin{définition}\peng{With similar mood/frame of mind.}\end{définition}
\begin{définition}\pcmn{心意相通}\end{définition}
\begin{définition}\pfra{En harmonie, à l'unisson.}\end{définition}
\end{entrée}

\begin{entrée}
{nv̩˩mi˩-ʈʰi˩}{}{ⓔnv̩˩mi˩-ʈʰi˩}\formedesurface{nv̩˩mi˩ʈʰi˩˥}\newline
\classe{形容词}\ton{L}
\étymologie{
nv̩˩mi˩; ʈʰi˩a
}\begin{définition}\peng{Weak, worn out.}\end{définition}
\begin{définition}\pcmn{累得没精神了}\end{définition}
\begin{définition}\pfra{Découragé, nostalgique, mélancolique.}\end{définition}
\end{entrée}

\begin{entrée}
{nv̩˧pɤ˩}{}{ⓔnv̩˧pɤ˩}\formedesurface{nv̩˧pɤ˩}\newline
\classe{名词}\ton{L\#}\begin{définition}\peng{Broad beans; lima beans.}\end{définition}
\begin{définition}\pcmn{蚕豆}\end{définition}
\begin{définition}\pfra{Fèves.}\end{définition}
\end{entrée}

\begin{entrée}
{nv̩˩pi˧}{}{ⓔnv̩˩pi˧}\formedesurface{nv̩˩pi˥}\newline
\classe{名词}\ton{LM}\begin{définition}\peng{Soybean dregs.}\end{définition}
\begin{définition}\pcmn{豆粕}\end{définition}
\begin{définition}\pfra{Tourteau de soja: le reste du soja, après qu'on en a tiré le lait de soja; sert de nourriture pour les porcs.}\end{définition}
\end{entrée}

\begin{entrée}
{nv̩˧tv̩˥}{}{ⓔnv̩˧tv̩˥}\formedesurface{nv̩˧tv̩˥}\newline
\classe{名词}\ton{H\#}
\paradigme{\pcmn{:} \p{}}
\begin{définition}\peng{Nosebag.}\end{définition}
\begin{définition}\pcmn{(挂在马脖子下面的)饲料袋子、马粮袋子}\end{définition}
\begin{définition}\pfra{Musette à grain, sac à grain: sac dans lequel on donnait à manger au cheval; le sac est pendu au cou du cheval.}\end{définition}
\end{entrée}

\begin{entrée}
{nv̩˩tɕʰi\#˥}{}{ⓔnv̩˩tɕʰi\#˥}\formedesurface{nv̩˩tɕʰi˥}\newline
\classe{名词}\ton{LM+\#H}
\paradigme{\pcmn{:} \p{}}
\begin{définition}\peng{Fine chaff of beans (used to feed cows).}\end{définition}
\begin{définition}\pcmn{豆类的细糠秕,来喂牛}\end{définition}
\begin{définition}\pfra{Balle de légumineuse (fine, pour nourrir les bovidés).}\end{définition}
\end{entrée}

\begin{entrée}
{nv̩˩tsɑ˧˥}{}{ⓔnv̩˩tsɑ˧˥}\formedesurface{nv̩˩tsɑ˧˥}\newline
\classe{名词}\ton{LM+MH\#}
\paradigme{\pcmn{:} \p{}}
\begin{définition}\peng{Coarse chaff of beans.}\end{définition}
\begin{définition}\pcmn{粗的豆糠}\end{définition}
\begin{définition}\pfra{Balle grossière de légumineuses.}\end{définition}
\end{entrée}

\begin{entrée}
{nv̩˩ze˧}{}{ⓔnv̩˩ze˧}\formedesurface{nv̩˩ze˥}\newline
\classe{名词}\ton{LM}\begin{définition}\peng{Chickpea, |\stylefi{Cicer arietinum}, black-coloured; the dish \stylefn{黑色凉粉} is made out of this pea.}\end{définition}
\begin{définition}\pcmn{鹰嘴豆、桃尔豆、鸡豆、鸡心豆}\end{définition}
\begin{définition}\pfra{Pois chiche, |\stylefi{Cicer arietinum}, de couleur noire, dont on prépare la spécialité de Dali: \stylefn{黑色凉粉.}}\end{définition}
\end{entrée}

\newpage\caractère{ɳ}

\begin{entrée}
{ɳæ˥}{}{ⓔɳæ˥}\formedesurface{ɳæ˧}\newline
\classe{动词}\ton{H}\begin{définition}\peng{To hide, to conceal oneself.}\end{définition}
\begin{définition}\pcmn{躲藏}\end{définition}
\begin{définition}\pfra{Se cacher.}\end{définition}
\begin{exemple}\pnru{tʰi˧-ɳæ˥}\hspace{5pt}\peng{|fg{dur} \_}\hspace{5pt}\pcmn{|fg{dur} \_}\hspace{5pt}\pfra{|fg{dur}}\end{exemple}
\end{entrée}

\begin{entrée}
{ɳæ˧˥}{}{ⓔɳæ˧˥}\newline
\classe{动词}
\sens{1}
\begin{définition}\peng{To press, to push down (with the hand); to press flat, to flatten; to squeeze.}\end{définition}
\begin{définition}\pcmn{按(用手)、压扁、挤压}\end{définition}
\begin{définition}\pfra{Aplatir; appuyer, peser sur; presser.}\end{définition}
\begin{exemple}\pnru{mv̩˩tɕo˧ ɳæ˧˥}\hspace{5pt}\peng{to push down, to press down}\hspace{5pt}\pcmn{往下按}\hspace{5pt}\pfra{appuyer vers le bas, peser sur}\end{exemple}
\begin{exemple}\pnru{le˧-ɳæ˩∼ɳæ˩}\hspace{5pt}\pfra{|fg{accomp} |fg{red}}\end{exemple}\sens{2}
\begin{définition}\peng{To oppress.}\end{définition}
\begin{définition}\pcmn{压迫}\end{définition}
\begin{définition}\pfra{Opprimer, accabler, écraser de son autorité, en imposer par la violence.}\end{définition}
\begin{exemple}\pnru{hĩ˧ kʰv̩˧, | hĩ˧ ɳæ˩}\hspace{5pt}\peng{to steal and oppress (description of a despot's behaviour)}\hspace{5pt}\pcmn{偷和迫(描述专制统治者的行为)}\hspace{5pt}\pfra{voler et oppresser (description du comportement d'un despote)}\end{exemple}
\end{entrée}

\begin{entrée}
{ɳæ˩=ɻæ˧}{}{ⓔɳæ˩=ɻæ˧}\formedesurface{ɳæ˩ɻæ˥}\newline
\classe{代词}\ton{LM}\begin{définition}\peng{Second person associative plural.}\end{définition}
\begin{définition}\pcmn{你们家、你们家族}\end{définition}
\begin{définition}\pfra{Deuxième personne, pluriel associatif: vous autres.}\end{définition}
\end{entrée}

\begin{entrée}
{ɳæ˧=ɻ̍˩}{}{ⓔɳæ˧=ɻ̍˩}\formedesurface{ɳæ˧ɻ̍˩}\newline
\classe{代词}\ton{L\#}\begin{définition}\peng{Second person plural. This is a variant of |fv{/no˧=ɻ̍˩/}; the form |fv{/no˧=ɻ̍˩/} is considered more correct.}\end{définition}
\begin{définition}\pcmn{你们。这是|fv{/no˧=ɻ̍˩/}的一个变体。|fv{/no˧=ɻ̍˩/}被认为是更标准的。}\end{définition}
\begin{définition}\pfra{Deuxième personne du pluriel: vous autres. Variante de |fv{/no˧=ɻ̍˩/}; la forme |fv{/no˧=ɻ̍˩/} est jugée plus correcte.}\end{définition}
\end{entrée}

\begin{entrée}
{‑ɳɯ}{₂}{ⓔ‑ɳɯⓗ2}\formedesurface{--}\newline
\classe{连接词}\ton{0?}
2\begin{définition}\peng{Or.}\end{définition}
\begin{définition}\pcmn{或者、还是}\end{définition}
\begin{définition}\pfra{Ou, ou alors.}\end{définition}
\end{entrée}

\begin{entrée}
{ɳɯ˥}{}{ⓔɳɯ˥}\formedesurface{ɳɯ˧}\newline
\classe{形容词}\ton{H}\begin{définition}\peng{Few.}\end{définition}
\begin{définition}\pcmn{少}\end{définition}
\begin{définition}\pfra{Peu, peu nombreux (dénombrable).}\end{définition}
\begin{exemple}\pnru{hĩ˧ ɳɯ˧}\hspace{5pt}\peng{people are few / there are few people}\hspace{5pt}\pcmn{好的,不多!不好的,就很多了!}\hspace{5pt}\pfra{les gens sont peu nombreux}\end{exemple}
\begin{exemple}\pnru{tso˧∼tso˧ | ɳɯ˧-ze˩}\hspace{5pt}\peng{there are fewer things, the amount of things has decreased}\hspace{5pt}\pcmn{东西(变)少了}\hspace{5pt}\pfra{il y a moins de choses, la quantité a diminué}\end{exemple}
\begin{exemple}\pnru{dʑɤ˩-hĩ˩˥, | le˧-ɳɯ˥! | mɤ˧-dʑɤ˩-hĩ˩, | le˧-dʑɯ˧˥!}\hspace{5pt}\peng{Good one are few; bad ones are many! / There are few good ones, but many bad ones! (A comment about higher education institutions, among which laureates of the national University entrance examination are given a choice.)}\hspace{5pt}\pcmn{好的少,不好的多!(关于大学:高考后,学生要报志愿)}\hspace{5pt}\pfra{Les bons, il n'y en a guère; les médiocres, il y en a en quantité! (contexte: au sujet des établissements universitaires entre lesquels les lauréats du concours national d'entrée à l'université ont à choisir)}\end{exemple}
\end{entrée}

\begin{entrée}
{‑ɳɯ˧}{₁}{ⓔ‑ɳɯ˧ⓗ1}\formedesurface{ɳɯ˧}\newline
\classe{}\ton{M}
1\begin{définition}\peng{Ablative, agent, and topic marker.}\end{définition}
\begin{définition}\pcmn{离格(从格),施动者,主题。接近汉语的‘由’}\end{définition}
\begin{définition}\pfra{Ablatif, agent, et marqueur de topique.}\end{définition}
\end{entrée}

\begin{entrée}
{ɳɯ˧˥}{}{ⓔɳɯ˧˥}\formedesurface{ɳɯ˧˥}\newline
\classe{动词}\ton{MH}\begin{définition}\peng{To wring, to tighten, to clamp.}\end{définition}
\begin{définition}\pcmn{拧}\end{définition}
\begin{définition}\pfra{Serrer.}\end{définition}
\begin{exemple}\pnru{le˧-ɳɯ˧-ze˥}\hspace{5pt}\peng{|fg{accomp} \_ |fg{pfv}}\hspace{5pt}\pcmn{拧了}\hspace{5pt}\pfra{|fg{accomp} \_ |fg{pfv}}\end{exemple}
\begin{exemple}\pnru{ʁo˧qɑ˥ | ʈʰɯ˧-ɭɯ˧ | le˧-ɳɯ˧-qɑ˥-jo˩!}\hspace{5pt}\peng{Tighten the lid! (of a glass jar, used as drinking glass)}\hspace{5pt}\pcmn{(你)拧一下盖子吧!}\hspace{5pt}\pfra{Serre donc ce couvercle! (celui d'un bocal en verre, utilisé comme verre)}\end{exemple}
\end{entrée}

\begin{entrée}
{ɳɯ˧ɕi˩}{}{ⓔɳɯ˧ɕi˩}\formedesurface{ɳɯ˧ɕi˩}\newline
\classe{形容词}\ton{L\#}\begin{définition}\peng{Lovely.}\end{définition}
\begin{définition}\pcmn{可爱}\end{définition}
\begin{définition}\pfra{Mignon, joli.}\end{définition}
\end{entrée}

\begin{entrée}
{ɳɯ˧go˧˥}{}{ⓔɳɯ˧go˧˥}\formedesurface{ɳɯ˧go˧˥}\newline
\classe{形容词}\ton{MH\#}\begin{définition}\peng{Pitiable, wretched, pitiful.}\end{définition}
\begin{définition}\pcmn{可怜}\end{définition}
\begin{définition}\pfra{Pitoyable, qui suscite la pitié.}\end{définition}
\begin{exemple}\pnru{no˧ | ʈʰɯ˧-ki˧ | ɖwæ˧˥ | ɳɯ˧go˧˥! / …ɖwæ˧˥ | ɳɯ˧go˧ ʝi˥!}\hspace{5pt}\peng{You pity her/him a lot! / You feel a lot of pity for her/him!}\hspace{5pt}\pcmn{你真的很可怜他。}\hspace{5pt}\pfra{Tu la/le plains beaucoup! / Tu as vraiment pitié d'elle/de lui!}\end{exemple}
\end{entrée}

\begin{entrée}
{ɳv̩˥}{}{ⓔɳv̩˥}\newline
\classe{动词}
\sens{1}
\begin{définition}\peng{To sniff.}\end{définition}
\begin{définition}\pcmn{闻嗅}\end{définition}
\begin{définition}\pfra{Sentir, renifler.}\end{définition}\sens{2}
\begin{définition}\peng{To hear, to get to know (good news…).}\end{définition}
\begin{définition}\pcmn{听到(消息)、风闻}\end{définition}
\begin{définition}\pfra{Apprendre une nouvelle; être au courant de.}\end{définition}
\begin{exemple}\pnru{mɤ˧-ɳv̩˥}\hspace{5pt}\peng{I am not aware of this piece of news! / I didn't know about that!}\hspace{5pt}\pcmn{(我)不知道这个消息!}\hspace{5pt}\pfra{|fg{neg}: je ne suis pas au courant!}\end{exemple}
\begin{exemple}\pnru{no˧ ə˧tso˧ ɳv̩˥?}\hspace{5pt}\peng{Which piece of news did you get? / What did you get to know?}\hspace{5pt}\pcmn{你听到了什么消息呢?}\hspace{5pt}\pfra{Quelle nouvelle as-tu apprise?}\end{exemple}
\end{entrée}

\begin{entrée}
{ɳv̩˩˧}{}{ⓔɳv̩˩˧}\formedesurface{ɳv̩˩˥}\newline
\classe{名词}\ton{LM}
\paradigme{\pcmn{:} \p{}}
\begin{définition}\peng{Moth; insect that eats into wood, books, clothes etc.}\end{définition}
\begin{définition}\pcmn{蛀虫}\end{définition}
\begin{définition}\pfra{Mite (insecte qui mange les vêments).}\end{définition}
\end{entrée}

\newpage\caractère{ɲ}

\begin{entrée}
{ɲi˥}{₁}{ⓔɲi˥ⓗ1}\formedesurface{ɲi˧}\newline
\classe{动词}\ton{H}
1\begin{définition}\peng{To listen.}\end{définition}
\begin{définition}\pcmn{听}\end{définition}
\begin{définition}\pfra{Écouter.}\end{définition}
\begin{exemple}\pnru{tʰi˧-ɲi˥}\hspace{5pt}\peng{|fg{dur}}\hspace{5pt}\pcmn{|fg{dur}}\hspace{5pt}\pfra{|fg{dur}}\end{exemple}
\begin{exemple}\pnru{tso˧∼tso˧ ɲi˧}\hspace{5pt}\peng{to listen to things}\hspace{5pt}\pcmn{听东西}\hspace{5pt}\pfra{écouter des choses}\end{exemple}
\begin{exemple}\pnru{le˧-ɲi˥-ze˩}\hspace{5pt}\peng{|fg{accomp} \_ |fg{pfv}}\hspace{5pt}\pcmn{听了}\hspace{5pt}\pfra{|fg{accomp} \_ |fg{pfv}}\end{exemple}
\end{entrée}

\begin{entrée}
{ɲi˥}{₂}{ⓔɲi˥ⓗ2}\formedesurface{ɲi˧}\newline
\classe{动词}\ton{H}
2\begin{définition}\peng{To borrow from someone.}\end{définition}
\begin{définition}\pcmn{向别人借}\end{définition}
\begin{définition}\pfra{Emprunter (un objet).}\end{définition}
\begin{exemple}\pnru{hĩ˧-ki˧ | tso˧∼tso˧ ɲi˧ |}\hspace{5pt}\peng{to borrow things from someone}\hspace{5pt}\pcmn{向别人借东西}\hspace{5pt}\pfra{emprunter des choses à quelqu'un}\end{exemple}
\end{entrée}

\begin{entrée}
{ɲi˥}{₃}{ⓔɲi˥ⓗ3}\formedesurface{ɲi˧}\newline
\classe{动词}\ton{H}
3\begin{définition}\peng{To lose, to be defeated.}\end{définition}
\begin{définition}\pcmn{败、输}\end{définition}
\begin{définition}\pfra{Échouer, perdre.}\end{définition}
\end{entrée}

\begin{entrée}
{ɲi˥β}{}{ⓔɲi˥β}\formedesurface{ɖɯ˧ ɲi˥}\newline
\classe{量词}\ton{Hβ}\begin{définition}\peng{Day.}\end{définition}
\begin{définition}\pcmn{日、天}\end{définition}
\begin{définition}\pfra{Un jour.}\end{définition}
\begin{exemple}\pnru{ɖɯ˧-ɲi˥}\hspace{5pt}\peng{one day}\hspace{5pt}\pcmn{一天}\hspace{5pt}\pfra{un jour}\end{exemple}
\end{entrée}

\begin{entrée}
{ɲi˧}{}{ⓔɲi˧}\formedesurface{ɲi˧}\newline
\classe{形容词}\ton{M}\begin{définition}\peng{Full, satiated.}\end{définition}
\begin{définition}\pcmn{饱}\end{définition}
\begin{définition}\pfra{Rassasié, repu.}\end{définition}
\begin{exemple}\pnru{le˧-ɲi˧-ze˧}\hspace{5pt}\peng{|fg{accomp} \_ |fg{pfv}}\hspace{5pt}\pcmn{饱了}\hspace{5pt}\pfra{|fg{accomp} \_ |fg{pfv}}\end{exemple}
\begin{exemple}\pnru{hɑ˧-ɲi˧(-ze˩)}\hspace{5pt}\peng{I am full. / I am satiated.}\hspace{5pt}\pcmn{吃饱了。 / 吃饱饭了。}\hspace{5pt}\pfra{(je) suis rassasié}\end{exemple}
\begin{exemple}\pnru{njɤ˧ | le˧-ɲi˧-ze˧!}\hspace{5pt}\peng{I am full. / I am satiated.}\hspace{5pt}\pcmn{我饱了。}\hspace{5pt}\pfra{je suis rassasié}\end{exemple}
\end{entrée}

\begin{entrée}
{ɲi˧˥}{}{ⓔɲi˧˥}\formedesurface{ɲi˧˥}\newline
\classe{数词}\ton{MH}\begin{définition}\peng{2.}\end{définition}
\begin{définition}\pcmn{2}\end{définition}
\begin{définition}\pfra{2.}\end{définition}
\end{entrée}

\begin{entrée}
{ɲi˧α}{}{ⓔɲi˧α}\formedesurface{ɲi˧}\newline
\classe{动词}\ton{Mα}\begin{définition}\peng{To need.}\end{définition}
\begin{définition}\pcmn{需要}\end{définition}
\begin{définition}\pfra{Avoir besoin de, vouloir.}\end{définition}
\begin{exemple}\pnru{no˧ | ə˩-ɲi˧? | mɤ˧-ɲi˧!}\hspace{5pt}\peng{Do you want (some)? - No!}\hspace{5pt}\pcmn{你要吗?- 不要!}\hspace{5pt}\pfra{Tu en veux? - (Non,) je n'en veux pas/je n'en ai pas besoin!}\end{exemple}
\end{entrée}

\begin{entrée}
{‑ɲi˩}{}{ⓔ‑ɲi˩}\formedesurface{ɲi˩˥}\newline
\classe{语气助词}\ton{L}\begin{définition}\peng{A particle derived from the copula, described by L. Lidz (2010:497) as conveying “an epistemic strategy that marks a high degree of certitude".}\end{définition}
\begin{définition}\pcmn{肯定(系词)}\end{définition}
\begin{définition}\pfra{Particule indiquant la certitude; dérivée du verbe copule.}\end{définition}
\end{entrée}

\begin{entrée}
{ɲi˩}{₁}{ⓔɲi˩ⓗ1}\formedesurface{ɲi˩˥}\newline
\classe{动词}\ton{Lα}
1\begin{définition}\peng{To press, to hold (clamped under the arm, between the legs…).}\end{définition}
\begin{définition}\pcmn{夹、夹持}\end{définition}
\begin{définition}\pfra{Serrer, tenir (ex.: tenir quelque chose serré sous le bras, serrer quelque chose entre les jambes).}\end{définition}
\begin{exemple}\pnru{ɖɯ˧-ɲi˧∼ɲi˥-ɻ̍˩}\hspace{5pt}\peng{to pressa little}\hspace{5pt}\pcmn{夹一点}\hspace{5pt}\pfra{serrer un peu}\end{exemple}
\end{entrée}

\begin{entrée}
{ɲi˩}{₂}{ⓔɲi˩ⓗ2}\formedesurface{ɲi˩˥}\newline
\classe{代词}\ton{L}
2\begin{définition}\peng{Who.}\end{définition}
\begin{définition}\pcmn{谁}\end{définition}
\begin{définition}\pfra{Pronom interrogatif: qui?.}\end{définition}
\begin{exemple}\pnru{ɲi˩ ɲi˧?}\hspace{5pt}\peng{Who is that?}\hspace{5pt}\pcmn{是谁?}\hspace{5pt}\pfra{C'est qui?}\end{exemple}
\begin{exemple}\pnru{no˧ | ɲi˩ ɲi˧?}\hspace{5pt}\peng{Who are you?}\hspace{5pt}\pcmn{你是谁?}\hspace{5pt}\pfra{Qui êtes-vous?}\end{exemple}
\begin{exemple}\pnru{ʈʂʰɯ˧ | ɲi˩ ɲi˧?}\hspace{5pt}\peng{Who is this person?}\hspace{5pt}\pcmn{他是谁?}\hspace{5pt}\pfra{Qui est-ce?}\end{exemple}
\begin{exemple}\pnru{no˧ | ɲi˩˥ ◊ -ki˩ bi˩-pi˩, | ɖɯ˧-bæ˧ lɑ˧ ɲi˥!}\hspace{5pt}\peng{No matter where you go, it's the same everywhere!}\hspace{5pt}\pcmn{无论你去谁(家),都一样!}\hspace{5pt}\pfra{Peu importe chez qui tu vas, c'est pareil partout!}\end{exemple}
\begin{exemple}\pnru{no˧ | ɲi˩-ki˥ bi˩-pi˩, | ɖɯ˧-bæ˧ lɑ˧ ɲi˥!}\hspace{5pt}\peng{As previous example, with a different division into tone groups}\hspace{5pt}\pcmn{同上,声调段界不同}\hspace{5pt}\pfra{Comme l'exemple précédent, avec une division en groupes tonals différente}\end{exemple}
\end{entrée}

\begin{entrée}
{ɲi˩α}{₁}{ⓔɲi˩αⓗ1}\formedesurface{ɲi˩˥}\newline
\classe{动词}\ton{Lα}
1\begin{définition}\peng{To twine, to wind; twist with the fingers (e.g. linen, to make thread).}\end{définition}
\begin{définition}\pcmn{捻,缠线}\end{définition}
\begin{définition}\pfra{Tordre avec les doigts, enrouler, filer (pour fabriquer du fil de lin, pour tisser des vêtements).}\end{définition}
\begin{exemple}\pnru{le˧-ɲi˩}\hspace{5pt}\peng{|fg{accomp}}\hspace{5pt}\pcmn{|fg{accomp}}\hspace{5pt}\pfra{|fg{accomp}}\end{exemple}
\begin{exemple}\pnru{sɑ˧ ɲi˥}\hspace{5pt}\peng{to twine hemp (to make thread)}\hspace{5pt}\pcmn{捻麻}\hspace{5pt}\pfra{tordre le chanvre/le lin (pour faire du fil)}\end{exemple}
\begin{exemple}\pnru{ɖɯ˧-ɲi˧∼ɲi˥-ɻ̍˩}\hspace{5pt}\peng{|fg{delimitative} |fg{red} |fg{inceptive}}\hspace{5pt}\pcmn{捻一捻}\hspace{5pt}\pfra{|fg{délimitatif} |fg{red} |fg{inchoatif}}\end{exemple}
\end{entrée}

\begin{entrée}
{ɲi˩α}{₂}{ⓔɲi˩αⓗ2}\formedesurface{ɲi˩˥}\newline
\classe{动词}\ton{Lα}
2\begin{définition}\peng{To break (tool), to be broken.}\end{définition}
\begin{définition}\pcmn{设备坏了}\end{définition}
\begin{définition}\pfra{S'abîmer, se casser; tomber en panne (ex.: appareil photo).}\end{définition}
\begin{exemple}\pnru{le˧-ɲi˩-ze˩}\hspace{5pt}\peng{|fg{accomp} \_ |fg{pfv}: it's broken!}\hspace{5pt}\pcmn{坏了!/破了!}\hspace{5pt}\pfra{|fg{accomp} \_ |fg{pfv}: c'est cassé!}\end{exemple}
\begin{exemple}\pnru{tso˧∼tso˧ ɲi˥}\hspace{5pt}\peng{to break things}\hspace{5pt}\pcmn{东西坏了}\hspace{5pt}\pfra{casser des choses}\end{exemple}
\end{entrée}

\begin{entrée}
{ɲi˩α}{₃}{ⓔɲi˩αⓗ3}\formedesurface{ɲi˩˥}\newline
\classe{动词}\ton{Lα}
3\begin{définition}\peng{Copula.}\end{définition}
\begin{définition}\pcmn{系词:是}\end{définition}
\begin{définition}\pfra{Verbe copule.}\end{définition}
\end{entrée}

\begin{entrée}
{ɲi˩bv̩˩}{}{ⓔɲi˩bv̩˩}\formedesurface{ɲi˩bv̩˩˥}\newline
\classe{名词}\ton{L}
\paradigme{\pcmn{:} \p{}}
\begin{définition}\peng{Grasshopper, cricket.}\end{définition}
\begin{définition}\pcmn{蟋蟀}\end{définition}
\begin{définition}\pfra{Criquet.}\end{définition}
\end{entrée}

\begin{entrée}
{ɲi˩bv̩˩-ʂe˩sɑ˧}{}{ⓔɲi˩bv̩˩-ʂe˩sɑ˧}\formedesurface{ɲi˩bv̩˩ʂe˩sɑ˥}\newline
\classe{名词}\ton{-LM}
\paradigme{\pcmn{:} \p{}}
\begin{définition}\peng{Dragonfly.}\end{définition}
\begin{définition}\pcmn{蜻蜓}\end{définition}
\begin{définition}\pfra{Libellule.}\end{définition}
\end{entrée}

\begin{entrée}
{ɲi˧dʑɯ\#˥}{}{ⓔɲi˧dʑɯ\#˥}\formedesurface{ɲi˧dʑɯ˧}\newline
\classe{名词}\ton{H\#}
\paradigme{\pcmn{:} \p{}}
\begin{définition}\peng{Penis.}\end{définition}
\begin{définition}\pcmn{男生殖器}\end{définition}
\begin{définition}\pfra{Pénis, organe sexuel masculin.}\end{définition}
\end{entrée}

\begin{entrée}
{ɲi˧gɤ\#˥}{}{ⓔɲi˧gɤ\#˥}\formedesurface{ɲi˧gɤ˧}\newline
\classe{名词}\ton{\#H}
\paradigme{\pcmn{:} \p{}}
\begin{définition}\peng{Nose.}\end{définition}
\begin{définition}\pcmn{鼻子}\end{définition}
\begin{définition}\pfra{Nez.}\end{définition}
\end{entrée}

\begin{entrée}
{ɲi˧gɤ˧-bæ˧˥}{}{ⓔɲi˧gɤ˧-bæ˧˥}\formedesurface{ɲi˧gɤ˧bæ˧˥}\newline
\classe{名词}\ton{MH\#}
\paradigme{\pcmn{:} \p{}}
\begin{définition}\peng{Rope attached to a cow's nasal ring.}\end{définition}
\begin{définition}\pcmn{牛鼻绳。也可以来指牛鼻圈。}\end{définition}
\begin{définition}\pfra{Corde accrochée à l'anneau nasal, longe de vache; aussi utilisé par extension pour l'anneau nasal, pour lequel aucun terme propre n'existe.}\end{définition}
\end{entrée}

\begin{entrée}
{ɲi˧gɤ˧-dʑɯ˧˥}{}{ⓔɲi˧gɤ˧-dʑɯ˧˥}\formedesurface{ɲi˧gɤ˧dʑɯ˧˥}\newline
\classe{名词}\ton{-MH}\begin{définition}\peng{Nasal mucus, snivel.}\end{définition}
\begin{définition}\pcmn{鼻涕}\end{définition}
\begin{définition}\pfra{Mucus, morve.}\end{définition}
\end{entrée}

\begin{entrée}
{ɲi˧ɬi˧mi˧}{}{ⓔɲi˧ɬi˧mi˧}\formedesurface{ɲi˧ɬi˧mi˧}\newline
\classe{名词}\ton{M}\begin{définition}\peng{Second month.}\end{définition}
\begin{définition}\pcmn{二月}\end{définition}
\begin{définition}\pfra{Le deuxième mois.}\end{définition}
\end{entrée}

\begin{entrée}
{ɲi˩mɑ\#˥}{}{ⓔɲi˩mɑ\#˥}\formedesurface{ɲi˩mɑ˥}\newline
\classe{名词}\ton{LM+\#H}\begin{définition}\peng{Masculine given name used for the elder of two twins (the child who is born first).}\end{définition}
\begin{définition}\pcmn{男性名字,起给双胞胎中的老大}\end{définition}
\begin{définition}\pfra{Prénom masculin pour l'aîné des jumeaux (l'enfant né en premier).}\end{définition}
\end{entrée}

\begin{entrée}
{ɲi˧mi˧}{}{ⓔɲi˧mi˧}\newline
\classe{名词}
\sens{1}\paradigme{\pcmn{:} \p{}}
\begin{définition}\peng{Sun.}\end{définition}
\begin{définition}\pcmn{太阳}\end{définition}
\begin{définition}\pfra{Soleil.}\end{définition}
\begin{exemple}\pnru{ɲi˧mi˧ tʰv̩˧}\hspace{5pt}\peng{the sun rises}\hspace{5pt}\pcmn{太阳出来、日出}\hspace{5pt}\pfra{le soleil se lève}\end{exemple}\sens{2}
\begin{définition}\peng{Day; daytime; time.}\end{définition}
\begin{définition}\pcmn{日、时间}\end{définition}
\begin{définition}\pfra{La journée; le temps.}\end{définition}
\end{entrée}

\begin{entrée}
{ɲi˧mi˧dɑ˧dzɯ\#˥}{}{ⓔɲi˧mi˧dɑ˧dzɯ\#˥}\formedesurface{ɲi˧mi˧dɑ˧dzɯ˧}\newline
\classe{名词}\ton{\#H}
\paradigme{\pcmn{:} \p{}}
\begin{définition}\peng{Solar eclipse.}\end{définition}
\begin{définition}\pcmn{日蚀}\end{définition}
\begin{définition}\pfra{Éclipse solaire.}\end{définition}
\begin{exemple}\pnru{ɲi˧mi˧dɑ˧dzɯ˧ tʰv̩˧}\hspace{5pt}\peng{there is a solar eclipse}\hspace{5pt}\pcmn{有日蚀}\hspace{5pt}\pfra{il y a une éclipse de soleil}\end{exemple}
\begin{exemple}\pnru{ɲi˧mi˧dɑ˧dzɯ˧ ɲi˥!}\hspace{5pt}\peng{Yes, it's a solar eclipse!}\hspace{5pt}\pcmn{是的,是日蚀!}\hspace{5pt}\pfra{Oui, c'est bien une éclipse de soleil!}\end{exemple}
\end{entrée}

\begin{entrée}
{ɲi˧mi˧-gv̩˩}{}{ⓔɲi˧mi˧-gv̩˩}\formedesurface{ɲi˧mi˧gv̩˩}\newline
\classe{名词}\ton{-L}\begin{définition}\peng{West: “[the direction where] the sun sets".}\end{définition}
\begin{définition}\pcmn{西方}\end{définition}
\begin{définition}\pfra{Ouest; «[la direction dans laquelle] le soleil se couche».}\end{définition}
\begin{exemple}\pnru{ɲi˧mi˧-gv̩˩-gi˩-dzɤ˩ se˩}\hspace{5pt}\peng{to walk towards the west}\hspace{5pt}\pcmn{往西边走}\hspace{5pt}\pfra{marcher vers l'ouest}\end{exemple}
\end{entrée}

\begin{entrée}
{ɲi˧mi˧-kʰɯ˧ʂɯ˧}{}{ⓔɲi˧mi˧-kʰɯ˧ʂɯ˧}\formedesurface{ɲi˧mi˧kʰɯ˧ʂɯ˧}\newline
\classe{名词}\ton{M}
\paradigme{\pcmn{:} \p{}}
\begin{définition}\peng{Rays (of sunshine).}\end{définition}
\begin{définition}\pcmn{太阳的光线}\end{définition}
\begin{définition}\pfra{Rayons du soleil.}\end{définition}
\end{entrée}

\begin{entrée}
{ɲi˧mi˧tʰv̩\#˥}{}{ⓔɲi˧mi˧tʰv̩\#˥}\formedesurface{ɲi˧mi˧tʰv̩˧}\newline
\classe{名词}\ton{\#H}\begin{définition}\peng{East, orient.}\end{définition}
\begin{définition}\pcmn{东方}\end{définition}
\begin{définition}\pfra{Est, orient.}\end{définition}
\begin{exemple}\pnru{ɲi˧mi˧tʰv̩˧-gi˧}\hspace{5pt}\peng{the direction of the east}\hspace{5pt}\pcmn{东方方向}\hspace{5pt}\pfra{la direction de l'est}\end{exemple}
\begin{exemple}\pnru{ɲi˧mi˧tʰv̩˧-gi˧ | se˧}\hspace{5pt}\peng{to walk towards the east}\hspace{5pt}\pcmn{向东面走}\hspace{5pt}\pfra{marcher vers l'est}\end{exemple}
\begin{exemple}\pnru{ɲi˧mi˧tʰv̩˧-gi˧ | dʑo˩˥}\hspace{5pt}\peng{to live in the East, to live in the Orient. (Context: the consultant imagines that I am in Europe, thinking of her, saying: ‘She lives in the Orient'.)}\hspace{5pt}\pcmn{住在东方(合作人想象我在欧洲,想着她说:‘她住在东方’。)}\hspace{5pt}\pfra{se trouver à l'est, habiter en Orient (contexte: la locutrice m'imagine, depuis l'Europe, pensant à elle, et disant: «elle habite en Orient».}\end{exemple}
\end{entrée}

\begin{entrée}
{ɲi˧mi˧-ʈæ˧ʂɯ˧}{}{ⓔɲi˧mi˧-ʈæ˧ʂɯ˧}\formedesurface{ɲi˧mi˧ʈæ˧ʂɯ˧}\newline
\classe{名词}\ton{M}
\paradigme{\pcmn{:} \p{}}
\begin{définition}\peng{Sunflower.}\end{définition}
\begin{définition}\pcmn{葵花}\end{définition}
\begin{définition}\pfra{Tournesol.}\end{définition}
\end{entrée}

\begin{entrée}
{ɲi˧nɑ˩}{}{ⓔɲi˧nɑ˩}\formedesurface{ɲi˧nɑ˩}\newline
\classe{名词}\ton{L\#}\begin{définition}\peng{Cane; rattan.}\end{définition}
\begin{définition}\pcmn{藤子}\end{définition}
\begin{définition}\pfra{Liane, rattan, vigne vierge, lierre.}\end{définition}
\end{entrée}

\begin{entrée}
{ɲi˧ŋwɤ˩}{}{ⓔɲi˧ŋwɤ˩}\formedesurface{ɲi˧ŋwɤ˩}\newline
\classe{名词}\ton{L\#}\begin{définition}\peng{Auspitious day.}\end{définition}
\begin{définition}\pcmn{吉利日}\end{définition}
\begin{définition}\pfra{Un jour propice.}\end{définition}
\begin{exemple}\pnru{ɲi˧ŋwɤ˩hɑ̃˩tʰɑ˩}\hspace{5pt}\peng{same meaning: auspitious day}\hspace{5pt}\pcmn{同上:吉利日}\hspace{5pt}\pfra{même sens: jour propice}\end{exemple}
\begin{exemple}\pnru{ɲi˧ŋwɤ˩hɑ̃˩tʰɑ˩ | ɖɯ˧-ɭɯ˧}\hspace{5pt}\peng{an auspitious day}\hspace{5pt}\pcmn{吉利的一天}\hspace{5pt}\pfra{un jour propice}\end{exemple}
\end{entrée}

\begin{entrée}
{ɲi˧pʰv̩˩}{}{ⓔɲi˧pʰv̩˩}\formedesurface{ɲi˧pʰv̩˩}\newline
\classe{名词}\ton{L\#}\begin{définition}\peng{Frost.}\end{définition}
\begin{définition}\pcmn{霜}\end{définition}
\begin{définition}\pfra{Givre.}\end{définition}
\begin{exemple}\pnru{ɲi˧pʰv̩˩ lɑ˩-ze˩}\hspace{5pt}\peng{there is some frost}\hspace{5pt}\pcmn{有霜}\hspace{5pt}\pfra{il y a du givre}\end{exemple}
\end{entrée}

\begin{entrée}
{ɲi˩pʰv̩˩}{}{ⓔɲi˩pʰv̩˩}\formedesurface{ɲi˩pʰv̩˩˥}\newline
\classe{名词}\ton{L}\begin{définition}\peng{A mountain plant; the consultant proposes this term for water mint, |\stylefi{Mentha aquatica}, |\stylefi{Mentha hirsuta Huds.} but this is unlikely to be the correct identification.}\end{définition}
\begin{définition}\pcmn{一种植物。合作人看水薄荷的图片就觉得像这种植物,但很可能不是。李达珠等(2015:98)翻译为“野牡丹”但这好像也不准确。}\end{définition}
\begin{définition}\pfra{Une plante de montagne; la locutrice pense la reconnaître sur une photo de menthe aquatique, |\stylefi{Mentha aquatica}, |\stylefi{Mentha hirsuta Huds.} mais ce n'est vraisemblablement pas la bonne identification.}\end{définition}
\begin{exemple}\pnru{ɲi˩pʰv̩˩-bæ˥bæ˩}\hspace{5pt}\peng{the flower of this plant}\hspace{5pt}\pcmn{这种植物的花}\hspace{5pt}\pfra{la fleur de cette plante}\end{exemple}
\end{entrée}

\begin{entrée}
{ɲi˧qʰv̩˧}{}{ⓔɲi˧qʰv̩˧}\newline
\classe{名词}
\sens{1}\paradigme{\pcmn{:} \p{}}
\begin{définition}\peng{Nostril.}\end{définition}
\begin{définition}\pcmn{鼻孔}\end{définition}
\begin{définition}\pfra{Narine.}\end{définition}\sens{2}
\begin{définition}\peng{Snivel, nasal mucus.}\end{définition}
\begin{définition}\pcmn{鼻涕}\end{définition}
\begin{définition}\pfra{Mucus, morve.}\end{définition}
\end{entrée}

\begin{entrée}
{ɲi˩=ɻ̍˥}{}{ⓔɲi˩=ɻ̍˥}\formedesurface{ɲi˩ɻ̍˥}\newline
\classe{代词}\ton{LM+H\#}\begin{définition}\peng{Second person associative pronoun: you and your clan/family/friends.}\end{définition}
\begin{définition}\pcmn{第二人称,联想复数:你与周边的人(家人、家族、亲戚、朋友们……)}\end{définition}
\begin{définition}\pfra{Pronom de 2e personne associatif: toi et les tiens.}\end{définition}
\end{entrée}

\begin{entrée}
{ɲi˧se˩}{}{ⓔɲi˧se˩}\formedesurface{ɲi˧se˩}\newline
\classe{名词}\ton{L\#}\begin{définition}\peng{The name of a village.}\end{définition}
\begin{définition}\pcmn{小落水(村落名)}\end{définition}
\begin{définition}\pfra{Un village du bord du Lac.}\end{définition}
\begin{exemple}\pnru{ɲi˧se˩, | nɑ˩-lɑ˧ ɲi˥!}\hspace{5pt}\peng{Nhissei is a thoroughly Na village! / Na is populated entirely by Na people!}\hspace{5pt}\pcmn{小落水,是纯摩梭的一个村落!}\hspace{5pt}\pfra{Nhissei, c'est un village entièrement Na!}\end{exemple}
\begin{exemple}\pnru{ɬi˧ki˧, | ɲi˧se˩, | tɑ˧dzi˩, | mv̩˧qʰwæ˩, | lɑ˧tʰɑ˧-di˧˥}\hspace{5pt}\peng{Villages that one passes when moving away from the Yongning plain, towards Lake Lugu. These villages do not count as part of Yongning proper. The last, /lɑ˧tʰɑ˧-di˧˥/, is not a village name like the preceding four: it refers to the entire Na area beyond the fourth village.}\hspace{5pt}\pcmn{永宁到泸沽湖所经过的村落,依次是:里格、尼赛、大祖、木垮,然后到拉塔地(拉塔地指的是泸沽湖周边的摩梭地区,包括左所、洛水村等)}\hspace{5pt}\pfra{Villages dans l'ordre, après la plaine de Yongning, ne comptant pas comme faisant partie de Yongning. Le dernier, /lɑ˧tʰɑ˧-di˧˥/, désigne toute la région na au-delà du quatrième village.}\end{exemple}
\end{entrée}

\begin{entrée}
{ɲi˧to˧}{}{ⓔɲi˧to˧}\formedesurface{ɲi˧to˧}\newline
\classe{名词}\ton{M}
\paradigme{\pcmn{:} \p{}}
\begin{définition}\peng{The mouth, seen as including the part of the face surrounding the mouth, in particular the jaw.}\end{définition}
\begin{définition}\pcmn{嘴巴,包括嘴周围的部位:颚等}\end{définition}
\begin{définition}\pfra{Bouche/pourtour de la bouche (autour des lèvre).}\end{définition}
\begin{exemple}\pnru{ɲi˧to˧ ʈʂʰwæ˩}\hspace{5pt}\peng{talkative}\hspace{5pt}\pcmn{多嘴、拉不断扯不断(直译:“嘴快”)}\hspace{5pt}\pfra{bavard (littéralement «bouche rapide»)}\end{exemple}
\end{entrée}

\begin{entrée}
{ɲi˧tʰo˧}{}{ⓔɲi˧tʰo˧}\formedesurface{ɲi˧tʰo˧}\newline
\classe{名词}\ton{H\#? M? à voir yyyyama}\begin{définition}\peng{Rumours.}\end{définition}
\begin{définition}\pcmn{谣言}\end{définition}
\begin{définition}\pfra{Rumeurs, ragots.}\end{définition}
\begin{exemple}\pnru{hĩ˧-ɲi˩tʰo˩ | le˧-ɖɯ˧}\hspace{5pt}\peng{to be an object of rumour}\hspace{5pt}\pcmn{被人家说三到四}\hspace{5pt}\pfra{faire l'objet de rumeurs}\end{exemple}
\begin{exemple}\pnru{hĩ˧-ɲi˩tʰo˩ ɖɯ˩}\hspace{5pt}\peng{to be an object of rumour}\hspace{5pt}\pcmn{被人家说三到四}\hspace{5pt}\pfra{faire l'objet de rumeurs}\end{exemple}
\end{entrée}

\begin{entrée}
{ɲi˧tsi˧}{}{ⓔɲi˧tsi˧}\formedesurface{ɲi˧tsi˧}\newline
\classe{数词}\ton{M}\begin{définition}\peng{20.}\end{définition}
\begin{définition}\pcmn{20}\end{définition}
\begin{définition}\pfra{20.}\end{définition}
\end{entrée}

\begin{entrée}
{ɲi˧tsi˧-ɖɯ˧˥}{}{ⓔɲi˧tsi˧-ɖɯ˧˥}\formedesurface{ɲi˧tsi˧ɖɯ˧˥}\newline
\classe{数词}\ton{-MH\#}\begin{définition}\peng{21.}\end{définition}
\begin{définition}\pcmn{21}\end{définition}
\begin{définition}\pfra{21.}\end{définition}
\end{entrée}

\begin{entrée}
{ɲi˧tsi˧-gv̩˧}{}{ⓔɲi˧tsi˧-gv̩˧}\formedesurface{ɲi˧tsi˧gv̩˧}\newline
\classe{数词}\ton{M}\begin{définition}\peng{29.}\end{définition}
\begin{définition}\pcmn{29}\end{définition}
\begin{définition}\pfra{29.}\end{définition}
\end{entrée}

\begin{entrée}
{ɲi˧tsi˧-hõ˧˥}{}{ⓔɲi˧tsi˧-hõ˧˥}\formedesurface{ɲi˧tsi˧hõ˧˥}\newline
\classe{数词}\ton{-MH\#}\begin{définition}\peng{28.}\end{définition}
\begin{définition}\pcmn{28}\end{définition}
\begin{définition}\pfra{28.}\end{définition}
\end{entrée}

\begin{entrée}
{ɲi˧tsi˧-ɲi˧˥}{}{ⓔɲi˧tsi˧-ɲi˧˥}\formedesurface{ɲi˧tsi˧ɲi˧˥}\newline
\classe{数词}\ton{-MH\#}\begin{définition}\peng{22.}\end{définition}
\begin{définition}\pcmn{22}\end{définition}
\begin{définition}\pfra{22.}\end{définition}
\end{entrée}

\begin{entrée}
{ɲi˧tsi˧-ŋwɤ˧}{}{ⓔɲi˧tsi˧-ŋwɤ˧}\formedesurface{ɲi˧tsi˧ŋwɤ˧}\newline
\classe{数词}\ton{M}\begin{définition}\peng{25.}\end{définition}
\begin{définition}\pcmn{25}\end{définition}
\begin{définition}\pfra{25.}\end{définition}
\end{entrée}

\begin{entrée}
{ɲi˧tsi˧-qʰv̩˧˥}{}{ⓔɲi˧tsi˧-qʰv̩˧˥}\formedesurface{ɲi˧tsi˧qʰv̩˧˥}\newline
\classe{数词}\ton{-MH\#}\begin{définition}\peng{26.}\end{définition}
\begin{définition}\pcmn{26}\end{définition}
\begin{définition}\pfra{26.}\end{définition}
\end{entrée}

\begin{entrée}
{ɲi˧tsi˧-so˩}{}{ⓔɲi˧tsi˧-so˩}\formedesurface{ɲi˧tsi˧so˩}\newline
\classe{数词}\ton{-L}\begin{définition}\peng{23.}\end{définition}
\begin{définition}\pcmn{23}\end{définition}
\begin{définition}\pfra{23.}\end{définition}
\end{entrée}

\begin{entrée}
{ɲi˧tsi˧-ʂɯ˧}{}{ⓔɲi˧tsi˧-ʂɯ˧}\formedesurface{ɲi˧tsi˧ʂɯ˧}\newline
\classe{数词}\ton{M}\begin{définition}\peng{27.}\end{définition}
\begin{définition}\pcmn{27}\end{définition}
\begin{définition}\pfra{27.}\end{définition}
\end{entrée}

\begin{entrée}
{ɲi˧tsi˧-ʐv̩˧}{}{ⓔɲi˧tsi˧-ʐv̩˧}\formedesurface{ɲi˧tsi˧ʐv̩˧}\newline
\classe{数词}\ton{M}\begin{définition}\peng{24.}\end{définition}
\begin{définition}\pcmn{24}\end{définition}
\begin{définition}\pfra{24.}\end{définition}
\end{entrée}

\begin{entrée}
{ɲi˩tsɯ\#˥}{}{ⓔɲi˩tsɯ\#˥}\formedesurface{ɲi˩tsɯ˥}\newline
\classe{名词}\ton{LM+\#H}
\paradigme{\pcmn{:} \p{}}
\begin{définition}\peng{Hmong (ethnic group).}\end{définition}
\begin{définition}\pcmn{苗族}\end{définition}
\begin{définition}\pfra{Hmông (groupe ethnique).}\end{définition}
\end{entrée}

\begin{entrée}
{ɲi˧-ʈʂæ˧-ʑi˧˥}{}{ⓔɲi˧-ʈʂæ˧-ʑi˧˥}\formedesurface{ɲi˧ʈʂæ˧ʑi˧˥}\newline
\classe{名词}\ton{MH\#}
\paradigme{\pcmn{:} \p{}}
\begin{définition}\peng{The building inside the farm where the bedrooms are located, and a living-room (downstairs in the centre). Literally ‘the two-floor building', as this is the only building that has rooms on two floors.}\end{définition}
\begin{définition}\pcmn{二层房:农场里面的一栋楼,正对着农场大门}\end{définition}
\begin{définition}\pfra{Bâtiment d'habitation; littéralement ‘le bâtiment à deux étages', car c'est le seul qui ait des salles sur deux étages. Ce bâtiment se trouve face à l'entrée de la ferme.}\end{définition}
\begin{exemple}\pnru{ɲi˧-ʈʂæ˧-ʑi˧-di˥}\hspace{5pt}\peng{same meaning}\hspace{5pt}\pcmn{同上}\hspace{5pt}\pfra{même sens}\end{exemple}
\end{entrée}

\begin{entrée}
{ɲi˩ʈʂe˩}{}{ⓔɲi˩ʈʂe˩}\formedesurface{ɲi˩ʈʂe˩˥}\newline
\classe{名词}\ton{L}\begin{définition}\peng{Door bar.}\end{définition}
\begin{définition}\pcmn{门闩}\end{définition}
\begin{définition}\pfra{Barre de porte: barre pour fermer la porte principale de la ferme.}\end{définition}
\begin{exemple}\pnru{ɲi˩ʈʂe˩ tʰi˥-kʰɯ˩, | tʰi˧-ʈæ˩!}\hspace{5pt}\peng{Put on the door bar, to lock (the main door)!}\hspace{5pt}\pcmn{放门闩,(好好)锁(门)!}\hspace{5pt}\pfra{On met la barre à la porte; on verrouille! / On met la barre à la porte, et c'est fermé!}\end{exemple}
\end{entrée}

\begin{entrée}
{ɲi˧ze˧-hæ̃˩ze˩}{}{ⓔɲi˧ze˧-hæ̃˩ze˩}\formedesurface{ɲi˧ze˧hæ̃˩ze˩}\newline
\classe{名词}\ton{-L}
\paradigme{\pcmn{:} \p{}}
\begin{définition}\peng{Swift.}\end{définition}
\begin{définition}\pcmn{雨燕}\end{définition}
\begin{définition}\pfra{Hirondelle.}\end{définition}
\end{entrée}

\begin{entrée}
{ɲi˧zo\#˥}{}{ⓔɲi˧zo\#˥}\formedesurface{ɲi˧zo˧}\newline
\classe{名词}\ton{\#H}
\paradigme{\pcmn{:} \p{}}
\begin{définition}\peng{Fish.}\end{définition}
\begin{définition}\pcmn{鱼}\end{définition}
\begin{définition}\pfra{Poisson.}\end{définition}
\begin{exemple}\pnru{ɲi˧zo˧ tʰv̩˧-mi˥\# / ɲi˧zo˧ tʰv̩˧-mi˧˥}\hspace{5pt}\peng{|fg{n}+|fg{dem}+|fg{clf}}\hspace{5pt}\pcmn{那条鱼}\hspace{5pt}\pfra{|fg{n}+|fg{dem}+|fg{clf}}\end{exemple}
\begin{exemple}\pnru{ɲi˧zo˧-tɑ˧pv̩˥}\hspace{5pt}\peng{dried fish}\hspace{5pt}\pcmn{干鱼}\hspace{5pt}\pfra{poisson séché}\end{exemple}
\end{entrée}

\newpage\caractère{ŋ}

\begin{entrée}
{ŋæ˧ʝi˩}{}{ⓔŋæ˧ʝi˩}\formedesurface{ŋæ˧ʝi˩}\newline
\classe{形容词}\ton{L\#}\begin{définition}\peng{Easy and comfortable, at ease. Borrowing from Southwestern Mandarin.}\end{définition}
\begin{définition}\pcmn{安逸(汉语借词)}\end{définition}
\begin{définition}\pfra{À l'aise, dans le confort, dans l'abondance. Emprunt au dialecte mandarin du sud-ouest.}\end{définition}
\end{entrée}

\begin{entrée}
{ŋɤ˩ŋɤ˩}{}{ⓔŋɤ˩ŋɤ˩}\formedesurface{ŋɤ˩ŋɤ˩˥}\newline
\classe{名词}\ton{L}
\paradigme{\pcmn{:} \p{}}
\begin{définition}\peng{Palate.}\end{définition}
\begin{définition}\pcmn{上腭}\end{définition}
\begin{définition}\pfra{Palais.}\end{définition}
\end{entrée}

\begin{entrée}
{ŋv̩˩}{}{ⓔŋv̩˩}\formedesurface{ŋv̩˧}\newline
\classe{名词}
\sens{1}
\begin{définition}\peng{Silver; money.}\end{définition}
\begin{définition}\pcmn{银子}\end{définition}
\begin{définition}\pfra{Argent (métal).}\end{définition}
\begin{exemple}\pnru{ŋv˧hæ̃˩/ or et ærgent càd ærgent, pætrimoine}\hspace{5pt}\peng{money, wealth; literally ‘silver and gold'}\hspace{5pt}\pcmn{金钱、钱财、财富。直译:‘银子与金子’}\hspace{5pt}\pfra{argent, patrimoine, fortune; littéralement ‘or et argent'}\end{exemple}\sens{2}
\begin{définition}\peng{Money.}\end{définition}
\begin{définition}\pcmn{钱}\end{définition}
\begin{définition}\pfra{Argent (argent-papier et pièces de monnaie).}\end{définition}
\end{entrée}

\begin{entrée}
{ŋv̩˩α}{}{ⓔŋv̩˩α}\formedesurface{ŋv̩˩˥}\newline
\classe{动词}\ton{Lα}\begin{définition}\peng{To cry, to weep.}\end{définition}
\begin{définition}\pcmn{哭}\end{définition}
\begin{définition}\pfra{Pleurer.}\end{définition}
\begin{exemple}\pnru{(tʰi˧-)ŋv̩˧∼ŋv̩˥}\hspace{5pt}\peng{|fg{dur} |fg{red}}\hspace{5pt}\pcmn{哭一场}\hspace{5pt}\pfra{|fg{dur} |fg{red}}\end{exemple}
\end{entrée}

\begin{entrée}
{ŋwæ˧qʰv̩˧}{}{ⓔŋwæ˧qʰv̩˧}\formedesurface{ŋwæ˧qʰv̩˧}\newline
\classe{名词}\ton{M}
\paradigme{\pcmn{:} \p{}}
\begin{définition}\peng{Oven to make tiles.}\end{définition}
\begin{définition}\pcmn{烧瓦的烤炉}\end{définition}
\begin{définition}\pfra{Four où on cuit les tuiles.}\end{définition}
\begin{exemple}\pnru{ŋwæ˧qʰv̩˧ ʂɯ˧-ʑi˩}\hspace{5pt}\peng{‘the seven families of the Tile Oven': an expression formerly used to designate the people from Alawa village, at a time when there were only seven families living there.}\hspace{5pt}\pcmn{‘瓦炉七家’:过去来指阿拉瓦村的人,当时那里只有七家住}\hspace{5pt}\pfra{‘les sept familles du Four à tuiles': expression dont on désignait autrefois les gens du village de Alawa, du temps où il n'y avait là que sept familles}\end{exemple}
\begin{exemple}\pnru{ə˧lɑ˧-ʁwɤ˧ | ŋwæ˧qʰv̩˧ | tsʰe˧ɲi˧ ʑi˩}\hspace{5pt}\peng{‘the twelve families of Alawa and the Tile Oven': an expression formerly used to designate the people from Alawa village, at a time when the number of families had increased from seven to twelve through migration.}\hspace{5pt}\pcmn{‘阿拉瓦瓦炉十二家’:过去来指阿拉瓦村的人,当时那里住的人家,从七家已经增加到十二家}\hspace{5pt}\pfra{«les douze familles de Alawa»: expression dont on désignait autrefois les gens du village de Alawa, du temps où le nombre de familles était passé de sept à douze par l'arrivée de nouveaux venus.}\end{exemple}
\end{entrée}

\begin{entrée}
{ŋwɤ˧}{}{ⓔŋwɤ˧}\formedesurface{ŋwɤ˧}\newline
\classe{数词}\ton{M? H\#? (pas L)}\begin{définition}\peng{Five.}\end{définition}
\begin{définition}\pcmn{五}\end{définition}
\begin{définition}\pfra{Cinq.}\end{définition}
\end{entrée}

\begin{entrée}
{ŋwɤ˧˥}{}{ⓔŋwɤ˧˥}\formedesurface{ŋwɤ˧˥}\newline
\classe{动词}\ton{MH}\begin{définition}\peng{To sting, to pierce.}\end{définition}
\begin{définition}\pcmn{刺痛}\end{définition}
\begin{définition}\pfra{Percer, piquer.}\end{définition}
\begin{exemple}\pnru{tɕʰi˧ ŋwɤ˩-ze˩}\hspace{5pt}\peng{(He/she) was stung by a thorn}\hspace{5pt}\pcmn{(他)被刺扎疼了。}\hspace{5pt}\pfra{(Elle/il) a pris une écharde}\end{exemple}
\end{entrée}

\begin{entrée}
{ŋwɤ˧hɑ̃˩}{}{ⓔŋwɤ˧hɑ̃˩}\formedesurface{ŋwɤ˧hɑ̃˩}\newline
\classe{名词}\ton{L\#}\begin{définition}\peng{A mountain to the South-West of Yongning.}\end{définition}
\begin{définition}\pcmn{位于永宁西南的一座山}\end{définition}
\begin{définition}\pfra{Nom d'une montagne au sud-ouest de Yongning.}\end{définition}
\begin{exemple}\pnru{kɤ˧mv̩˧˥, | æ˧ʂæ˧, | ŋwɤ˧hɑ̃˩, | ʂwæ˧gv̩\#˥, | nɑ˩tsʰi˩˥ | -tɕʰɤ˧pɤ˧mi\#˥, | qv̩˧ɻ̍˧-ʈʂʰɑ˧nɑ˥ |}\hspace{5pt}\peng{The six mountains of Yongning that carry a name and have a definite symbolic value. The other mountains do not have comparable symbolic value, and fewer people use specific names for them.}\hspace{5pt}\pcmn{永宁地区有固定名字的六座山:格姆,安山,瓦哈,双古,纳慈巧吧咪,古尔川纳。}\hspace{5pt}\pfra{Les six montagnes de Yongning qui portent un nom. Les autres sommets du voisinage n'ont pas une valeur symbolique comparable, et ne portent pas de nom communément utilisé.}\end{exemple}
\end{entrée}

\begin{entrée}
{ŋwɤ˩ɭɯ˧-tse˥pʰæ˩}{}{ⓔŋwɤ˩ɭɯ˧-tse˥pʰæ˩}\formedesurface{ŋwɤ˩ɭɯ˧tse˥pʰæ˩}\newline
\classe{名词}\ton{LM+\#H-}
\paradigme{\pcmn{:} \p{}}
\begin{définition}\peng{Kneebone.}\end{définition}
\begin{définition}\pcmn{膝盖骨}\end{définition}
\begin{définition}\pfra{Os du genou.}\end{définition}
\end{entrée}

\begin{entrée}
{ŋwɤ˩ɬi˩mi˩}{}{ⓔŋwɤ˩ɬi˩mi˩}\formedesurface{ŋwɤ˩ɬi˩mi˩˥}\newline
\classe{名词}\ton{L}\begin{définition}\peng{5th month.}\end{définition}
\begin{définition}\pcmn{五月}\end{définition}
\begin{définition}\pfra{5e mois.}\end{définition}
\end{entrée}

\begin{entrée}
{ŋwɤ˩ɬv̩˧˥}{}{ⓔŋwɤ˩ɬv̩˧˥}\formedesurface{ŋwɤ˩ɬv̩˧˥}\newline
\classe{名词}\ton{LM+MH\#}
\paradigme{\pcmn{:} \p{}}
\begin{définition}\peng{Cartilages of the knee; literally “marrow of the knee". This expression emphasizes the fragility of this articulation.}\end{définition}
\begin{définition}\pcmn{膝盖(直译:“膝盖髓”)。这个说法强调膝盖的脆弱。}\end{définition}
\begin{définition}\pfra{Genou, cartilages du genou, articulation du genou: littéralement «moëlle du genou». L'expression insiste sur le caractère fragile de cette articulation.}\end{définition}
\begin{exemple}\pnru{ŋwɤ˩ɬv̩˧-ko˧lo˥ go˩.}\hspace{5pt}\peng{to feel pain inside the knee}\hspace{5pt}\pcmn{膝盖疼。}\hspace{5pt}\pfra{avoir mal dans le genou}\end{exemple}
\end{entrée}

\begin{entrée}
{ŋwɤ˧pʰæ˧˥}{}{ⓔŋwɤ˧pʰæ˧˥}\formedesurface{ŋwɤ˧pʰæ˧˥}\newline
\classe{名词}\ton{MH\#}
\paradigme{\pcmn{:} \p{}}
\begin{définition}\peng{Tile (roof tile).}\end{définition}
\begin{définition}\pcmn{瓦(汉语借词)}\end{définition}
\begin{définition}\pfra{Tuile (pour la toiture).}\end{définition}
\end{entrée}

\begin{entrée}
{ŋwɤ˧qo˥}{}{ⓔŋwɤ˧qo˥}\formedesurface{ŋwɤ˧qo˥}\newline
\classe{名词}\ton{H\#}
\paradigme{\pcmn{:} \p{}}
\begin{définition}\peng{Knee.}\end{définition}
\begin{définition}\pcmn{膝盖}\end{définition}
\begin{définition}\pfra{Genou.}\end{définition}
\end{entrée}

\begin{entrée}
{ŋwɤ˧tsʰi˩}{}{ⓔŋwɤ˧tsʰi˩}\formedesurface{ŋwɤ˧tsʰi˩}\newline
\classe{数词}\ton{L\#}\begin{définition}\peng{50.}\end{définition}
\begin{définition}\pcmn{50}\end{définition}
\begin{définition}\pfra{50.}\end{définition}
\end{entrée}

\begin{entrée}
{ŋwɤ˧ʈʂwæ˧}{}{ⓔŋwɤ˧ʈʂwæ˧}\formedesurface{ŋwɤ˧ʈʂwæ˧}\newline
\classe{名词}\ton{M}\begin{définition}\peng{Tile and brick: characterization of the main materials used for Chinese-style house construction.}\end{définition}
\begin{définition}\pcmn{瓦与砖(汉语借词)}\end{définition}
\begin{définition}\pfra{Tuile et brique: caractérisation des principaux matériaux de construction des maisons à la façon chinoise.}\end{définition}
\end{entrée}

\newpage\caractère{o}

\begin{entrée}
{õ˧˥}{}{ⓔõ˧˥}\formedesurface{õ˧˥}\newline
\classe{代词}\ton{MH}\begin{définition}\peng{(one)self.}\end{définition}
\begin{définition}\pcmn{自己}\end{définition}
\begin{définition}\pfra{Soi-même, propre.}\end{définition}
\begin{exemple}\pnru{õ˧-ɑ˥ʁo˩}\hspace{5pt}\peng{one's house}\hspace{5pt}\pcmn{自己家}\hspace{5pt}\pfra{sa propre maison}\end{exemple}
\begin{exemple}\pnru{õ˧-dʑɯ˥, õ˩ ʈʰɯ˩! |}\hspace{5pt}\peng{Each drinks from her own bottle! (Context: a toddler has grabbed another's milk bottle; parents prevent her from drinking from it.)}\hspace{5pt}\pcmn{自己喝自己的!(情景:一个婴儿抓另一个婴儿的奶瓶。)}\hspace{5pt}\pfra{Chacun boit sa propre boisson! (Contexte: un petit enfant s'empare du biberon d'un autre et s'apprête à boire; on l'en empêche.)}\end{exemple}
\begin{exemple}\pnru{õ˧-ʂe˥, õ˩ ʈʰæ˩! |}\hspace{5pt}\peng{Each person eats their own slab of meat! (Describing table manners: each person used to receive one slice of meat and eat it up, unlike Chinese custom, in which each guest picks food mouthful by mouthful, with chopsticks, from the dishes placed on the table.)}\hspace{5pt}\pcmn{自己吃自己的(那块)肉!(关于饮食习惯:吃饭的时候,每人分得一块肉,自己吃完。当地人认为,汉族没有这种分吃的习惯。)}\hspace{5pt}\pfra{chacun mange son propre morceau de viande! (Description des manières de table: dans le temps, on donnait un bout de viande à chacun et chacun mangeait son morceau, pas comme la coutume chinoise qui veut qu'on prélève bouchée par bouchée, avec ses baguettes, dans les bols/assiettes posés sur la table.)}\end{exemple}
\begin{exemple}\pnru{õ˧-bv̩˥-õ˩ ʝi˩-ɳɯ˩ | sɯ˧-kv̩˩!}\hspace{5pt}\peng{One learns by practising oneself! / It's by practising oneself that one really masters a skill!}\hspace{5pt}\pcmn{自己做,就能学会!/ 要学会,就得自己熟练!}\hspace{5pt}\pfra{c'est en faisant soi-même qu'on apprend!}\end{exemple}
\begin{exemple}\pnru{õ˧-bv̩˥-õ˩ +N |}\hspace{5pt}\peng{one's own N}\hspace{5pt}\pcmn{自己的(+名词)}\hspace{5pt}\pfra{son propre N (soi-même+|fg{poss}+soi-même)}\end{exemple}
\begin{exemple}\pnru{õ˧-bv̩˥-õ˩ ʐwæ˩}\hspace{5pt}\peng{one's own horse}\hspace{5pt}\pcmn{自己的马}\hspace{5pt}\pfra{son propre cheval}\end{exemple}
\begin{exemple}\pnru{õ˧-bv̩˥-õ˩ ʝi˩}\hspace{5pt}\peng{one's own cow}\hspace{5pt}\pcmn{自己的牛}\hspace{5pt}\pfra{sa propre vache}\end{exemple}
\begin{exemple}\pnru{õ˧-bv̩˥-õ˩ lv̩˩}\hspace{5pt}\peng{one's own field}\hspace{5pt}\pcmn{自己的田地}\hspace{5pt}\pfra{son propre champ}\end{exemple}
\begin{exemple}\pnru{õ˧-bv̩˥-õ˩ ɖʐe˩}\hspace{5pt}\peng{one's own money}\hspace{5pt}\pcmn{自己的钱}\hspace{5pt}\pfra{son propre argent}\end{exemple}
\begin{exemple}\pnru{õ˧mv̩˥-õ˩di˩}\hspace{5pt}\peng{birth place}\hspace{5pt}\pcmn{出生的地方、老家、故乡}\hspace{5pt}\pfra{lieu de naissance, lieu d'origine}\end{exemple}
\begin{exemple}\pnru{hĩ˧-mv˥ hĩ˩-di˩ | qʰɑ˧-dʑɤ˥∼dʑɤ˩, | õ˧-mv˥ õ˩-di˩ tsʰe˩ mɤ˩-gv˩!}\hspace{5pt}\peng{No matter how beautiful other people's places are, they can never be equal to one's own homeland!}\hspace{5pt}\pcmn{其他人的地方怎么好,也比不过自己的地方!}\hspace{5pt}\pfra{Si belles soient les terres d'autrui, elles n'auront jamais la beauté de ses propres terres / de la terre natale !}\end{exemple}
\begin{exemple}\pnru{õ˧-ə˧mv̩˥ / õ˧-ə˥mv̩˩ / õ˧-ə˧mv̩˧˥}\hspace{5pt}\peng{one's own elder (brother or sister)}\hspace{5pt}\pcmn{自家姐姐(或哥哥)}\hspace{5pt}\pfra{son propre aîné (frère ou soeur)}\end{exemple}
\begin{exemple}\pnru{õ˧-ə˧v̩˥ / õ˧-ə˥v̩˩}\hspace{5pt}\peng{one's own maternal uncle}\hspace{5pt}\pcmn{自家舅舅(母亲的兄弟)}\hspace{5pt}\pfra{son propre oncle}\end{exemple}
\begin{exemple}\pnru{õ˧-ʐɤ˥mi˩, õ˩ ɲi˩! |}\hspace{5pt}\peng{One's path, that is one's identity / one's destiny! / The path you choose is your destiny!}\hspace{5pt}\pcmn{自己的道路,就是自己!/ 每个人有自己的命运!}\hspace{5pt}\pfra{Chacun a son chemin! / Chacun vit sa vie! / A chacun sa destinée!}\end{exemple}
\end{entrée}

\begin{entrée}
{õ˧dɤ˧ɻ̍˧}{}{ⓔõ˧dɤ˧ɻ̍˧}\formedesurface{õ˧dɤ˧ɻ̍˧}\newline
\classe{名词}\ton{M}\begin{définition}\peng{Foundation, fundamentals.}\end{définition}
\begin{définition}\pcmn{根本}\end{définition}
\begin{définition}\pfra{Fondement/fondamentalement.}\end{définition}
\begin{exemple}\pnru{õ˧dɤ˧ɻ̍˧-ɳɯ˧, | hĩ˧ ʈʂʰɯ˧-v̩˧ | ʈʂʰɯ˧ne˧ gv̩˧˥ ◊ -ɲi˩!}\hspace{5pt}\peng{So that is how he really behaves / does! (Comment on someone whose behaviour is not respectful of good manners)}\hspace{5pt}\pcmn{他原来是这样做事情的! / 他原来这么不懂事!}\hspace{5pt}\pfra{Voilà comment il se comporte en réalité/au fond! (Se dit de quelqu'un dont le comportement est irrespectueux des règles de savoir-vivre)}\end{exemple}
\end{entrée}

\begin{entrée}
{õ˩dv̩˧˥}{}{ⓔõ˩dv̩˧˥}\formedesurface{õ˩dv̩˧˥}\newline
\classe{名词}\ton{LM+MH\#}
\paradigme{\pcmn{:} \p{}}
\begin{définition}\peng{Wolf.}\end{définition}
\begin{définition}\pcmn{狼}\end{définition}
\begin{définition}\pfra{Loup.}\end{définition}
\end{entrée}

\begin{entrée}
{õ˩dv̩˧-kʰv̩˥mi˩}{}{ⓔõ˩dv̩˧-kʰv̩˥mi˩}\formedesurface{õ˩dv̩˧kʰv̩˥mi˩}\newline
\classe{名词}\ton{LM+\#H-}
\paradigme{\pcmn{:} \p{}}
\begin{définition}\peng{Wolfhound.}\end{définition}
\begin{définition}\pcmn{狼狗}\end{définition}
\begin{définition}\pfra{Chien-loup.}\end{définition}
\end{entrée}

\begin{entrée}
{õ˩dv̩˧-mi˥}{}{ⓔõ˩dv̩˧-mi˥}\formedesurface{õ˩dv̩˧mi˥}\newline
\classe{名词}\ton{LM+H\#}
\paradigme{\pcmn{:} \p{}}
\begin{définition}\peng{Female wolf.}\end{définition}
\begin{définition}\pcmn{母狼}\end{définition}
\begin{définition}\pfra{Louve.}\end{définition}
\end{entrée}

\begin{entrée}
{õ˩dv̩˧-pʰv̩\#˥}{}{ⓔõ˩dv̩˧-pʰv̩\#˥}\formedesurface{õ˩dv̩˧pʰv̩˧}\newline
\classe{名词}\ton{LM+\#H}
\paradigme{\pcmn{:} \p{}}
\begin{définition}\peng{Male wolf.}\end{définition}
\begin{définition}\pcmn{公狼}\end{définition}
\begin{définition}\pfra{Loup mâle.}\end{définition}
\end{entrée}

\begin{entrée}
{õ˩dv̩˧-zo\#˥}{}{ⓔõ˩dv̩˧-zo\#˥}\formedesurface{õ˩dv̩˧zo˧}\newline
\classe{名词}\ton{LM+\#H}\begin{définition}\peng{Little wolf.}\end{définition}
\begin{définition}\pcmn{小狼}\end{définition}
\begin{définition}\pfra{Louveteau.}\end{définition}
\end{entrée}

\begin{entrée}
{õ˧tʰv̩˧ɲi˧}{}{ⓔõ˧tʰv̩˧ɲi˧}\formedesurface{õ˧tʰv̩˧ɲi˧}\newline
\classe{名词}\ton{M}\begin{définition}\peng{That day (long ago).}\end{définition}
\begin{définition}\pcmn{(早以前的)那天}\end{définition}
\begin{définition}\pfra{Ce jour-là (il y a longtemps).}\end{définition}
\end{entrée}

\begin{entrée}
{õ˧ʈʂwɤ˧}{}{ⓔõ˧ʈʂwɤ˧}\formedesurface{õ˧ʈʂwɤ˧}\newline
\classe{名词}\ton{M}
\paradigme{\pcmn{:} \p{}}
\begin{définition}\peng{Mosquito.}\end{définition}
\begin{définition}\pcmn{蚊子}\end{définition}
\begin{définition}\pfra{Moustique.}\end{définition}
\begin{exemple}\pnru{õ˧ʈʂwɤ˧ le˧-tʰv̩˧-ze˧!}\hspace{5pt}\peng{Here comes a mosquito! / A mosquito has come in! (=into the room, into the mosquito net…)}\hspace{5pt}\pcmn{有一只蚊子!}\hspace{5pt}\pfra{voilà un moustique! / un moustique est entré (dans la pièce, sous la moustiquaire…)}\end{exemple}
\begin{exemple}\pnru{ʂɯ˧-ɬi˧mi˧, | õ˧ʈʂwɤ˧! |}\hspace{5pt}\peng{In the seventh month, there are lots of mosquitoes!}\hspace{5pt}\pcmn{七月份,蚊子多! / 七月份,是蚊子多的一个月!}\hspace{5pt}\pfra{Le septième mois, c'est un mois à moustiques!}\end{exemple}
\end{entrée}

\begin{entrée}
{õ˧ʈʂwɤ˧-kv̩˧dʑɯ˧˥}{}{ⓔõ˧ʈʂwɤ˧-kv̩˧dʑɯ˧˥}\formedesurface{õ˧ʈʂwɤ˧kv̩˧dʑɯ˧˥}\newline
\classe{名词}\ton{-MH\#}
\paradigme{\pcmn{:} \p{}}
\begin{définition}\peng{Mosquito net.}\end{définition}
\begin{définition}\pcmn{蚊帐}\end{définition}
\begin{définition}\pfra{Moustiquaire.}\end{définition}
\end{entrée}

\begin{entrée}
{õ˧ʈʂʰɯ˧ne˧-ʝi˥}{}{ⓔõ˧ʈʂʰɯ˧ne˧-ʝi˥}\formedesurface{õ˧ʈʂʰɯ˧ne˧-ʝi˥}\newline
\classe{助词}\ton{H\#}\begin{définition}\peng{In that way.}\end{définition}
\begin{définition}\pcmn{那样}\end{définition}
\begin{définition}\pfra{De cette façon-là.}\end{définition}
\end{entrée}

\newpage\caractère{p}

\begin{entrée}
{pɑ˧tɕɤ˧}{}{ⓔpɑ˧tɕɤ˧}\formedesurface{pɑ˧tɕɤ˧}\newline
\classe{名词}\ton{M}\begin{définition}\peng{Plantain.}\end{définition}
\begin{définition}\pcmn{芭蕉(汉语借词)}\end{définition}
\begin{définition}\pfra{Bananier plantain.}\end{définition}
\end{entrée}

\begin{entrée}
{pæ˥}{}{ⓔpæ˥}\formedesurface{pæ˧}\newline
\classe{动词}\ton{H}\begin{définition}\peng{To move house.}\end{définition}
\begin{définition}\pcmn{搬(家)}\end{définition}
\begin{définition}\pfra{Déménager.}\end{définition}
\end{entrée}

\begin{entrée}
{pæ˥α}{}{ⓔpæ˥α}\formedesurface{ɖɯ˧ pæ˥}\newline
\classe{量词}\ton{Hα}\begin{définition}\peng{Classifier for packs/herds (of horses…), troops (of soldiers)…}\end{définition}
\begin{définition}\pcmn{量词:马、军人……(一队)}\end{définition}
\begin{définition}\pfra{Troupe (de chevaux, de soldats…).}\end{définition}
\end{entrée}

\begin{entrée}
{pæ˧˥}{₁}{ⓔpæ˧˥ⓗ1}\formedesurface{pæ˧˥}\newline
\classe{动词}\ton{MH}
1\begin{définition}\peng{To cultivate land.}\end{définition}
\begin{définition}\pcmn{种(地)}\end{définition}
\begin{définition}\pfra{Cultiver (une terre).}\end{définition}
\end{entrée}

\begin{entrée}
{pæ˧˥}{₂}{ⓔpæ˧˥ⓗ2}\formedesurface{pæ˧˥}\newline
\classe{动词}\ton{MH}
2\begin{définition}\peng{To exceed; to let slip.}\end{définition}
\begin{définition}\pcmn{超过,错过}\end{définition}
\begin{définition}\pfra{Dépasser, outrepasser; laisser passer (une occasion).}\end{définition}
\begin{exemple}\pnru{pæ˧˥ ◊ -kʰɯ˩-pi˩, | mɤ˧-tsɤ˧! |}\hspace{5pt}\peng{It's not good to let (an auspicious day) slip by! / It's not good to miss the opportunity (of an auspicious days; for the building of a house, for instance)}\hspace{5pt}\pcmn{错过(一个吉日),不好!}\hspace{5pt}\pfra{Ce n'est pas bien de laisser passer (un jour propice: pour la construction d'une maison, par exemple)!}\end{exemple}
\begin{exemple}\pnru{pæ˧˥ | -tʰɑ˧-kʰɯ˩}\hspace{5pt}\peng{Don't let (this opportunity) slip by! / (You/we) mustn't miss this opportunity!}\hspace{5pt}\pcmn{不要错过(机会)!}\hspace{5pt}\pfra{Il ne faut pas laisser passer/filer (une occasion/un moment propice)!}\end{exemple}
\begin{exemple}\pnru{le˧-pæ˧-ze˥!}\hspace{5pt}\peng{It's too late! / We have let the opportunity slip by!}\hspace{5pt}\pcmn{错过了!}\hspace{5pt}\pfra{(On) a laissé filer (une occasion)/ c'est passé, c'est trop tard!}\end{exemple}
\end{entrée}

\begin{entrée}
{pæ˩α}{}{ⓔpæ˩α}\formedesurface{pæ˩˥}\newline
\classe{动词}\ton{Mα}\begin{définition}\peng{To lay (the table).}\end{définition}
\begin{définition}\pcmn{摆桌子、供应饭菜}\end{définition}
\begin{définition}\pfra{Mettre (la table), servir.}\end{définition}
\begin{exemple}\pnru{hɑ˧ tʰi˧-pæ˩ tsæ˩-ɲi˩-ze˩! | hɑ˧ dzɯ˧-bi˧-ze˩!}\hspace{5pt}\peng{The table is set / everything is ready! Let's eat!}\hspace{5pt}\pcmn{饭摆好了!吃饭了!}\hspace{5pt}\pfra{C'est servi! A table!}\end{exemple}
\end{entrée}

\begin{entrée}
{pæ˧˥hwɤ˧}{}{ⓔpæ˧˥hwɤ˧}\formedesurface{pæ˧˥hwɤ˧}\newline
\classe{名词}\ton{MH.M}
\paradigme{\pcmn{:} \p{}}
\begin{définition}\peng{Solution, method (early borrowing from Chinese).}\end{définition}
\begin{définition}\pcmn{办法(早期汉语借词)}\end{définition}
\begin{définition}\pfra{Solution, méthode (emprunt chinois ancien).}\end{définition}
\begin{exemple}\pnru{ʈʂʰɯ˧ | pæ˧˥hwɤ˧ | ɕjɤ˩ ɣɯ˧ (+ | ʐwæ˩˥)!}\hspace{5pt}\peng{He/she is great at finding solutions / at handling all sorts of difficult situations!}\hspace{5pt}\pcmn{他很会想办法的!}\hspace{5pt}\pfra{Il/elle excelle à trouver des solutions/ il a une solution à tout!}\end{exemple}
\end{entrée}

\begin{entrée}
{pæ˧kʰwɤ\#˥}{}{ⓔpæ˧kʰwɤ\#˥}\formedesurface{pæ˧kʰwɤ˧}\newline
\classe{名词}\ton{\#H}
\paradigme{\pcmn{:} \p{}}
\begin{définition}\peng{Silver coin of the imperial times.}\end{définition}
\begin{définition}\pcmn{民国之前的银币}\end{définition}
\begin{définition}\pfra{Pièce d'argent de l'époque impériale.}\end{définition}
\begin{exemple}\pnru{ə˧mi˧! | pæ˧kʰwɤ˧ so˧-ɭɯ˥ ki˩-mæ˩!}\hspace{5pt}\peng{Wow! [(S)he] is giving you three silver coins!! (According to the main consultant's memories, this is the type of comment that uncles and aunts would make when a child who turned 13 received significant amounts of money on the occasion of their coming of age. The equivalent today would be about half a month's salary. To give only one coin would not be right, because gifts have to come in pairs. To give two coins is fully sufficient: a beautiful gift. To give three coins is an impressive gift, beyond expectations.)}\hspace{5pt}\pcmn{哇!(他)给三块银币!(在一个孩子成年时,亲戚会给银币。给一块,不合适,因为礼物不能只给一个,要给两个。给两块银币,是合适的,也是够的。给三块银币,超出期望,是大礼物了。按现在的标准/说法,三个银币等于半个月的工资左右。)}\hspace{5pt}\pfra{Waouuu! [Il/elle] te donne trois pièces d'argent! (D'après le souvenir qu'en a la consultante principale, c'est le type de commentaire que faisaient autrefois les tantes ou oncles d'un enfant à qui on offrait une forte somme d'argent à l'occasion de son passage à l'âge adulte, à treize ans. Cela correspondrait aujourd'hui à la moitié d'un mois de salaire. Donner une seule pièce, c'est symboliquement inapproprié: on offre par paires. Donner deux pièces, c'est un beau cadeau, approprié et suffisant. Donner trois pièces, c'est un cadeau considérable, qui dépasse les attentes.)}\end{exemple}
\begin{exemple}\pnru{pæ˧kʰwɤ˧ ɖɯ˧-ɭɯ˥\# ; pæ˧kʰwɤ˧ ɲi˧-ɭɯ˥\# ; pæ˧kʰwɤ˧ so˧-ɭɯ˥\#}\hspace{5pt}\peng{one silver coin, two silver coins, three silver coins}\hspace{5pt}\pcmn{一块银币,两块银币,三块银币}\hspace{5pt}\pfra{une pièce d'argent; deux pièces d'argent; trois pièces d'argent}\end{exemple}
\begin{exemple}\pnru{pæ˧kʰwɤ˧ ɖɯ˧-ki˩tɑ˩}\hspace{5pt}\peng{a bag of silver coins (to be interred in a secret place)}\hspace{5pt}\pcmn{一包银币(埋在地里,为了藏)}\hspace{5pt}\pfra{un sac de pièces d'argent, destiné à être caché/enterré}\end{exemple}
\end{entrée}

\begin{entrée}
{pæ˧li˩}{}{ⓔpæ˧li˩}\formedesurface{pæ˧li˩}\newline
\classe{名词}\ton{L\#}\begin{définition}\peng{Chinese chestnut.}\end{définition}
\begin{définition}\pcmn{板栗}\end{définition}
\begin{définition}\pfra{Châtaigne.}\end{définition}
\begin{exemple}\pnru{pæ˧li˩-si˩dzi˩}\hspace{5pt}\peng{chestnut tree}\hspace{5pt}\pcmn{板栗树}\hspace{5pt}\pfra{châtaignier}\end{exemple}
\begin{exemple}\pnru{pæ˧li˩-dzi˩}\hspace{5pt}\peng{chestnut tree}\hspace{5pt}\pcmn{板栗树}\hspace{5pt}\pfra{châtaignier}\end{exemple}
\end{entrée}

\begin{entrée}
{pæ˩pʰæ˧˥}{₁}{ⓔpæ˩pʰæ˧˥ⓗ1}\newline
\classe{名词}
1
\sens{1}\paradigme{\pcmn{:} \p{}}
\begin{définition}\peng{Thick wood plank. A well-prepared plank, used in construction, could last a hundred years.}\end{définition}
\begin{définition}\pcmn{厚的木板、 木板子}\end{définition}
\begin{définition}\pfra{Grosse planche de bois, épaisse d'une dizaine de centimètres, utilisée pour la charpente des maisons.}\end{définition}\sens{2}
\begin{définition}\peng{Harrow; the term is the same as that for ‘plank', as the harrow essentially consisted in a large, squared piece of lumber, without teeth.}\end{définition}
\begin{définition}\pcmn{耙}\end{définition}
\begin{définition}\pfra{Herse en bois, qui consiste essentiellement en une grosse pièce de bois, sans dents, d'où l'emploi (par extension) du terme qui signifie ‘planche’.}\end{définition}
\end{entrée}

\begin{entrée}
{pæ˩pʰæ˧˥}{₂}{ⓔpæ˩pʰæ˧˥ⓗ2}\formedesurface{pæ˩pʰæ˧˥}\newline
\classe{名词}\ton{LM+MH\#}
2\begin{définition}\peng{Masculine given name.}\end{définition}
\begin{définition}\pcmn{男性名字}\end{définition}
\begin{définition}\pfra{Prénom masculin.}\end{définition}
\end{entrée}

\begin{entrée}
{pæ˧ɻæ˩-ʈʂʰo˩}{}{ⓔpæ˧ɻæ˩-ʈʂʰo˩}\formedesurface{pæ˧ɻæ˩ʈʂʰo˩}\newline
\classe{名词}\ton{L\#-}\begin{définition}\peng{Hongqiao, a (mostly Han Chinese) village on the road from Ninglang to Yongning.}\end{définition}
\begin{définition}\pcmn{红桥}\end{définition}
\begin{définition}\pfra{Hongqiao, village sur la route entre Ninglang et Yongning (principalement peuplé de Chinois Han).}\end{définition}
\begin{exemple}\pnru{no˧ | pæ˧ɻæ˩ʈʂʰo˩-hĩ˩-ni˩-zo˩!}\hspace{5pt}\peng{“You look like someone from Hongqiao!" This is an insult, meaning “You are ugly". Popular Na geography had it that the people of Hongqiao (a village which the caravans crossed) had coarse, unlovely physical features, such as big snub noses.}\hspace{5pt}\pcmn{解放前用的侮辱语句:“你像红桥人!”=“你很丑!”摩梭民间文化中,红桥(马帮路过的一个乡)的人被认为难看,面貌不“眉清目秀”,比如有扁鼻子。}\hspace{5pt}\pfra{«Tu ressembles à quelqu'un de Hongqiao!» Insulte, pour dire de quelqu'un qu'il a un physique disgracieux. La géographie populaire na attribuait des traits grossiers aux gens de Hongqiao (localité que traversaient les caravanes): gros nez camus, en particulier.}\end{exemple}
\end{entrée}

\begin{entrée}
{pæ˧sɯ˧}{}{ⓔpæ˧sɯ˧}\formedesurface{pæ˧sɯ˧}\newline
\classe{名词}\ton{M}\begin{définition}\peng{The lowest rank in the hierarchy of feudal officials.}\end{définition}
\begin{définition}\pcmn{把事(封建官员系统中的最低等级)(汉语借词)}\end{définition}
\begin{définition}\pfra{Rang (le plus bas) dans la hiérarchie des fonctionnaires féodaux.}\end{définition}
\end{entrée}

\begin{entrée}
{pæ˧te˩}{}{ⓔpæ˧te˩}\formedesurface{pæ˧te˩}\newline
\classe{名词}\ton{L\#}
\paradigme{\pcmn{:} \p{}}
\begin{définition}\peng{Bench, stool.}\end{définition}
\begin{définition}\pcmn{板凳}\end{définition}
\begin{définition}\pfra{Banc, tabouret.}\end{définition}
\end{entrée}

\begin{entrée}
{pe˧ʂe˧}{}{ⓔpe˧ʂe˧}\formedesurface{pe˧ʂe˧}\newline
\classe{助词}\ton{M}\begin{définition}\peng{Itself, per se.}\end{définition}
\begin{définition}\pcmn{本身(汉语借词)}\end{définition}
\begin{définition}\pfra{En soi.}\end{définition}
\end{entrée}

\begin{entrée}
{pɤ˥}{}{ⓔpɤ˥}\formedesurface{pɤ˧}\newline
\classe{动词}\ton{H}\begin{définition}\peng{To curl up; to hunch, to huddle up.}\end{définition}
\begin{définition}\pcmn{蜷曲、蜷缩}\end{définition}
\begin{définition}\pfra{S'accroupir, se mettre en boule, se recroqueviller sur soi-même.}\end{définition}
\begin{exemple}\pnru{æ˩ ʈʂʰɯ˧-mi˥ | si˧dzi˩-ʈʰæ˩qo˩ | tʰi˧-pɤ˥-dʑo˩!}\hspace{5pt}\peng{The hen has huddled up under a tree!}\hspace{5pt}\pcmn{那只鸡,在树下蜷缩着!}\hspace{5pt}\pfra{La poule est recroquevillée sous l'arbre/est accroupie sous l'arbre!}\end{exemple}
\begin{exemple}\pnru{ʈʂʰɯ˧-qo˧ ɖɯ˧-pɤ˥ ɕjɤ˩-ɻ̍˩!}\hspace{5pt}\peng{Come and lay here (for a rest)!}\hspace{5pt}\pcmn{过来这边躺一下!}\hspace{5pt}\pfra{Viens t'allonger par ici (pour te reposer)!}\end{exemple}
\end{entrée}

\begin{entrée}
{pɤ˥}{}{ⓔpɤ˥}\formedesurface{pɤ˧}\newline
\classe{名词}\ton{\#H}
\paradigme{\pcmn{:} \p{}}
\begin{définition}\peng{Drawing, painting.}\end{définition}
\begin{définition}\pcmn{画}\end{définition}
\begin{définition}\pfra{Dessin, peinture.}\end{définition}
\end{entrée}

\begin{entrée}
{pɤ˥α}{}{ⓔpɤ˥α}\formedesurface{ɖɯ˧ pɤ˥}\newline
\classe{量词}\ton{Hα}\begin{définition}\peng{Classifier for spoonfuls (of food)}\end{définition}
\begin{définition}\pcmn{量词:一瓢(饭)}\end{définition}
\begin{définition}\pfra{Classificateur des cuillerées de riz}\end{définition}
\begin{exemple}\pnru{hɑ˧ | ɖɯ˧-pɤ˥}\hspace{5pt}\peng{a spoonful of rice}\hspace{5pt}\pcmn{一瓢饭}\hspace{5pt}\pfra{une cuillerée de riz}\end{exemple}
\begin{exemple}\pnru{v˩dʑɯ˩˥ | ɖɯ˧-qʰwɤ˧pɤ˧}\hspace{5pt}\peng{a bowl of soup}\hspace{5pt}\pcmn{一碗汤}\hspace{5pt}\pfra{un bol de soupe}\end{exemple}
\end{entrée}

\begin{entrée}
{pɤ˥β}{}{ⓔpɤ˥β}\formedesurface{ɖɯ˧ pɤ˥}\newline
\classe{量词}\ton{Hβ}\begin{définition}\peng{Classifier for statues, paintings…}\end{définition}
\begin{définition}\pcmn{量词:雕像,如:佛像(一尊)}\end{définition}
\begin{définition}\pfra{Classificateur des images, peintures…}\end{définition}
\begin{exemple}\pnru{gɤ˧lɑ˧ | ɖɯ˧-pɤ˥}\hspace{5pt}\peng{a god's statue}\hspace{5pt}\pcmn{一尊佛像}\hspace{5pt}\pfra{une statue de divinité}\end{exemple}
\end{entrée}

\begin{entrée}
{pɤ˧˥}{}{ⓔpɤ˧˥}\formedesurface{pɤ˧˥}\newline
\classe{动词}\ton{MH}\begin{définition}\peng{To harrow.}\end{définition}
\begin{définition}\pcmn{耙地}\end{définition}
\begin{définition}\pfra{Passer la herse, aplanir (à l'aide d'une herse/instrument permettant de lisser le champ après labourage, afin qu'il soit prêt pour qu'on y repique le riz).}\end{définition}
\begin{exemple}\pnru{ʝi˧ pɤ˥}\hspace{5pt}\peng{to harrow}\hspace{5pt}\pcmn{耙地}\hspace{5pt}\pfra{passer la herse}\end{exemple}
\begin{exemple}\pnru{ɕi˧ tv̩˧-dʑo˧, | ʝi˧ le˧-pɤ˩!}\hspace{5pt}\peng{When one plants rice, one must harrow the field (first)!}\hspace{5pt}\pcmn{种稻谷,要(先)耙地!}\hspace{5pt}\pfra{Quand on plante du riz (=avant de planter le riz), il faut passer la herse!}\end{exemple}
\end{entrée}

\begin{entrée}
{pɤ˧α}{}{ⓔpɤ˧α}\formedesurface{pɤ˧}\newline
\classe{动词}\ton{Mα}\begin{définition}\peng{To carry on one's back.}\end{définition}
\begin{définition}\pcmn{背(水、柴、孩子……)}\end{définition}
\begin{définition}\pfra{Porter sur son dos (le bois, …).}\end{définition}
\begin{exemple}\pnru{pɤ˧∼pɤ˧}\hspace{5pt}\peng{|fg{red}}\hspace{5pt}\pcmn{重叠:背一背}\hspace{5pt}\pfra{|fg{red}}\end{exemple}
\begin{exemple}\pnru{tʰi˧-pɤ˥∼pɤ˩}\hspace{5pt}\peng{|fg{dur} |fg{red}}\hspace{5pt}\pcmn{背一背}\hspace{5pt}\pfra{|fg{dur} |fg{red}}\end{exemple}
\begin{exemple}\pnru{qʰæ˧ pɤ˧∼pɤ˥}\hspace{5pt}\peng{to carry manure}\hspace{5pt}\pcmn{背肥料}\hspace{5pt}\pfra{porter des engrais/ du fumier}\end{exemple}
\begin{exemple}\pnru{kʰɤ˧ pɤ˧∼pɤ˥}\hspace{5pt}\peng{to carry a dorsal basket}\hspace{5pt}\pcmn{背背篓}\hspace{5pt}\pfra{porter un panier dorsal}\end{exemple}
\begin{exemple}\pnru{zɯ˧ pɤ˧∼pɤ˥}\hspace{5pt}\peng{to carry grass}\hspace{5pt}\pcmn{背草}\hspace{5pt}\pfra{porter de l'herbe}\end{exemple}
\begin{exemple}\pnru{tso˧∼tso˧ pɤ˧∼pɤ˥}\hspace{5pt}\peng{to carry things}\hspace{5pt}\pcmn{背东西}\hspace{5pt}\pfra{porter des choses}\end{exemple}
\begin{exemple}\pnru{*tso˧∼tso˧ pɤ˩}\hspace{5pt}\peng{to carry things (this expression is well-formed syntactically, but apparently not in use)}\hspace{5pt}\pcmn{背东西(语法上,这个短语没有问题,但发音合作人不那么说。)}\hspace{5pt}\pfra{porter des choses (l'expression est bien formée, mais pas usitée)}\end{exemple}
\begin{exemple}\pnru{njɤ˧-ɳɯ˧ pɤ˧∼pɤ˩ (+bi˩)!}\hspace{5pt}\peng{I'll do the carrying! / Let me carry (it)!}\hspace{5pt}\pcmn{我来背!}\hspace{5pt}\pfra{c'est moi qui porte!}\end{exemple}
\begin{exemple}\pnru{dʑɯ˩ pɤ˩∼pɤ˥}\hspace{5pt}\peng{to carry water}\hspace{5pt}\pcmn{背水}\hspace{5pt}\pfra{porter de l'eau}\end{exemple}
\begin{exemple}\pnru{zo˧mv̩˥ pɤ˩∼pɤ˩}\hspace{5pt}\peng{to carry a child on the back}\hspace{5pt}\pcmn{背孩子}\hspace{5pt}\pfra{porter un enfant sur le dos}\end{exemple}
\end{entrée}

\begin{entrée}
{pɤ˩α}{}{ⓔpɤ˩α}\formedesurface{pɤ˩˥}\newline
\classe{动词}\ton{Lα}\begin{définition}\peng{To come out, to emerge, to appear.}\end{définition}
\begin{définition}\pcmn{出现、出来、浮现}\end{définition}
\begin{définition}\pfra{Sortir, émerger, apparaître.}\end{définition}
\begin{exemple}\pnru{dʑɯ˩ pɤ˩˥}\hspace{5pt}\peng{some water comes out}\hspace{5pt}\pcmn{涌出水来}\hspace{5pt}\pfra{de l'eau sort}\end{exemple}
\begin{exemple}\pnru{dʑɯ˧qʰv̩˧-qo˧ | dʑɯ˩ pɤ˩-ze˥}\hspace{5pt}\peng{Water emerges at the source.}\hspace{5pt}\pcmn{水泉里面,涌出水来。}\hspace{5pt}\pfra{De l'eau apparaît à la source / de l'eau coule à la source}\end{exemple}
\begin{exemple}\pnru{tʰi˧-pɤ˩-dʑo˩}\hspace{5pt}\peng{|fg{dur} \_ |fg{prog}: it is emerging}\hspace{5pt}\pcmn{正在涌出水来}\hspace{5pt}\pfra{|fg{dur} \_ |fg{prog}: ça sort, ça coule, ça émerge (ex.: de l'eau de source)}\end{exemple}
\begin{exemple}\pnru{gɤ˩-pɤ˥}\hspace{5pt}\peng{to emerge, to come up, to appear (e.g. the sun comes out)}\hspace{5pt}\pcmn{出现、上来:太阳出来}\hspace{5pt}\pfra{émerger, se lever: le soleil se lève}\end{exemple}
\end{entrée}

\begin{entrée}
{pɤ˩β}{}{ⓔpɤ˩β}\formedesurface{ɖɯ˧ pɤ˩}\newline
\classe{量词}\ton{Lβ}\begin{définition}\peng{Classifier for ladders, doors…}\end{définition}
\begin{définition}\pcmn{量词:木工件,如梯子、门等等(一扇门,一把梯子)}\end{définition}
\begin{définition}\pfra{Classificateur des éléments de menuiserie/charpente: échelles, portes…}\end{définition}
\end{entrée}

\begin{entrée}
{pɤ˧dʑɤ˩-di˩}{}{ⓔpɤ˧dʑɤ˩-di˩}\formedesurface{pɤ˧dʑɤ˩di˩}\newline
\classe{名词}\ton{L\#-}\begin{définition}\peng{A village close to the Hot Springs.}\end{définition}
\begin{définition}\pcmn{巴甲地村:温泉乡的一个村落}\end{définition}
\begin{définition}\pfra{Un village proche des Sources Chaudes.}\end{définition}
\begin{exemple}\pnru{ə˧go˧-ʁwɤ˧, | ʁwɤ˧lɑ˩-bi˩, | bæ˧ʁwɤ˧, | tʰo˧tsʰe\#˥, | pi˧tsʰe˩-di˩, | pɤ˧dʑɤ˩-di˩, | ʁwɤ˧tv̩˧}\hspace{5pt}\peng{Seven villages that one encounters as one leaves the plain of Yongning (towards the Lake); the first two are perceived as villages with a high proportion of Na members, and the third as a mostly Na village, whereas the next two are Pumi (Prinmi); the last used to be predominantly Pumi, but as of the 2010s, it had an important Chinese (Han) population.}\hspace{5pt}\pcmn{永宁背向泸沽湖方向经过的七个村落:阿公瓦、瓦拉比、巴瓦、拖其、比其地、巴甲地、瓦都。前两个村落拥有相当大的摩梭人口比例,第三主要是摩梭村。拖其、比其地、巴甲地是普米村。瓦都,过去主要是普米族村,到了2010年代有了相当多的汉族人口。}\hspace{5pt}\pfra{Sept villages au sortir de la plaine de Yongning, dans la direction du Lac; les deux premiers comportent une population na; le troisième est un village na; les deux suivants sont essentiellement des villages pumi/prinmi; le dernier était un village pumi, et a désormais (dans les années 2010) une importante population chinoise (han).}\end{exemple}
\end{entrée}

\begin{entrée}
{pɤ˩dʑɯ˩}{}{ⓔpɤ˩dʑɯ˩}\formedesurface{pɤ˩dʑɯ˩˥}\newline
\classe{名词}\ton{L}\begin{définition}\peng{Spring water.}\end{définition}
\begin{définition}\pcmn{泉水}\end{définition}
\begin{définition}\pfra{Eau de source.}\end{définition}
\end{entrée}

\begin{entrée}
{pɤ˩-ho˩∼ho˥}{}{ⓔpɤ˩-ho˩∼ho˥}\formedesurface{pɤ˩ho˩ho˥}\newline
\classe{形容词}\ton{L+H\#}\begin{définition}\peng{Soft.}\end{définition}
\begin{définition}\pcmn{柔软}\end{définition}
\begin{définition}\pfra{Mou.}\end{définition}
\begin{exemple}\pnru{pɤ˩-ho˩∼ho˥-gv̩˩}\hspace{5pt}\peng{soft}\hspace{5pt}\pcmn{柔软}\hspace{5pt}\pfra{mou}\end{exemple}
\begin{exemple}\pnru{ʁo˧qʰwɤ˩ | pɤ˩-ho˩∼ho˥-gv̩˩-hĩ˩ | tʰv̩˧-kʰwɤ˥}\hspace{5pt}\peng{the place where the head is soft =the fontanel}\hspace{5pt}\pcmn{头上软软的那块 =囟门}\hspace{5pt}\pfra{l'endroit où la tête est toute molle =la fontanelle, chez les bébés}\end{exemple}
\end{entrée}

\begin{entrée}
{pɤ˩jɤ˧˥}{}{ⓔpɤ˩jɤ˧˥}\newline
\classe{名词}
\sens{1}\paradigme{\pcmn{:} \p{}}
\begin{définition}\peng{Dough for making bread (steamed bread, as well as bread that is fried or cooked on a griddle.}\end{définition}
\begin{définition}\pcmn{做面包的面团(可以蒸成馒头)}\end{définition}
\begin{définition}\pfra{Pâte à pain (à cuire à la vapeur, dans l'huile ou sur une plaque).}\end{définition}\sens{2}\paradigme{\pcmn{:} \p{}}
\begin{définition}\peng{Bread, cake (typically round and flat).}\end{définition}
\begin{définition}\pcmn{饼}\end{définition}
\begin{définition}\pfra{Galette, pain.}\end{définition}
\begin{exemple}\pnru{li˩-pɤ˥jɤ˩ | ɖɯ˧-ɭɯ˧}\hspace{5pt}\peng{a piece of brick tea, a brick of tea (tea leaves pressed into the shape of a round flat cake)}\hspace{5pt}\pcmn{一块茶饼}\hspace{5pt}\pfra{une galette de thé (feuilles de thé pressées en forme de galette)}\end{exemple}
\begin{exemple}\pnru{ɕi˧ʈʂʰwæ˧-pɤ˩jɤ˩}\hspace{5pt}\peng{rice cake}\hspace{5pt}\pcmn{米饼}\hspace{5pt}\pfra{galette de riz}\end{exemple}
\begin{exemple}\pnru{dze˧ɭɯ˧-pɤ˩jɤ˩}\hspace{5pt}\peng{wheat cake, wheat bread}\hspace{5pt}\pcmn{小麦饼}\hspace{5pt}\pfra{galette de froment, galette à la farine de blé, pain de froment}\end{exemple}
\begin{exemple}\pnru{qʰɑ˧dze˧-pɤ˩jɤ˩}\hspace{5pt}\peng{sweetcorn cake}\hspace{5pt}\pcmn{玉米饼}\hspace{5pt}\pfra{galette de maïs, galette à la farine de maïs}\end{exemple}
\begin{exemple}\pnru{tsʰi˧zi˧-pɤ˥jɤ˩}\hspace{5pt}\peng{highland barley cake}\hspace{5pt}\pcmn{青稞饼}\hspace{5pt}\pfra{galette à l'orge d'altitude}\end{exemple}
\begin{exemple}\pnru{jɤ˧gɯ˩-pɤ˩jɤ˩}\hspace{5pt}\peng{buckwheat cake}\hspace{5pt}\pcmn{甜荞饼}\hspace{5pt}\pfra{galette de sarrasin}\end{exemple}
\begin{exemple}\pnru{jɤ˧qʰɑ˧-pɤ˥jɤ˩}\hspace{5pt}\peng{bitter buckwheat cake}\hspace{5pt}\pcmn{苦荞饼}\hspace{5pt}\pfra{galette de sarrasin amer}\end{exemple}
\end{entrée}

\begin{entrée}
{pɤ˩jɤ˧bv̩˥-di˩}{}{ⓔpɤ˩jɤ˧bv̩˥-di˩}\formedesurface{pɤ˩jɤ˧bv̩˥di˩}\newline
\classe{名词}\ton{LM+H\#-}
\paradigme{\pcmn{:} \p{}}
\begin{définition}\peng{Steamer used for bread (buns).}\end{définition}
\begin{définition}\pcmn{用来蒸面团(馒头等等)的蒸笼}\end{définition}
\begin{définition}\pfra{Étuve pour cuire la pâte/le pain.}\end{définition}
\end{entrée}

\begin{entrée}
{pɤ˧lɑ˩}{}{ⓔpɤ˧lɑ˩}\formedesurface{pɤ˧lɑ˩}\newline
\classe{名词}\ton{L\#}
\paradigme{\pcmn{:} \p{}}
\begin{définition}\peng{Photo, photography (newly coined word).}\end{définition}
\begin{définition}\pcmn{相片,照片}\end{définition}
\begin{définition}\pfra{Photo, photographie (néologisme).}\end{définition}
\end{entrée}

\begin{entrée}
{pɤ˩lv̩˧}{}{ⓔpɤ˩lv̩˧}\formedesurface{pɤ˩lv̩˥}\newline
\classe{名词}\ton{LM}\begin{définition}\peng{Warehouse, storehouse: a one-floor building, opposite the main building (\stylefv{/ʑi}˧mi˧/); it is used for storing objects, such as the ard, and preserved meat.}\end{définition}
\begin{définition}\pcmn{仓库:主屋对面的房子,只有一层。用来收藏大工具,例如犁,或者腊肉}\end{définition}
\begin{définition}\pfra{Réserve, magasin: bâtiment à un seul étage, face au bâtiment principal (\stylefv{/ʑi}˧mi˧/), dans lequel on range les gros outils, tels que l'araire, et la viande séchée.}\end{définition}
\end{entrée}

\begin{entrée}
{pɤ˩lv̩˩}{}{ⓔpɤ˩lv̩˩}\formedesurface{pɤ˩lv̩˩˥}\newline
\classe{名词}\ton{L}
\paradigme{\pcmn{:} \p{}}
\begin{définition}\peng{Nape of neck.}\end{définition}
\begin{définition}\pcmn{项背 、项、脖颈儿}\end{définition}
\begin{définition}\pfra{Nuque.}\end{définition}
\end{entrée}

\begin{entrée}
{pɤ˩mi˩}{}{ⓔpɤ˩mi˩}\formedesurface{pɤ˩mi˩˥}\newline
\classe{名词}\ton{L}
\paradigme{\pcmn{:} \p{}}
\begin{définition}\peng{Frog.}\end{définition}
\begin{définition}\pcmn{青蛙}\end{définition}
\begin{définition}\pfra{Grenouille.}\end{définition}
\begin{exemple}\pnru{pɤ˩mi˩-pɤ˥pʰv̩˩}\hspace{5pt}\peng{female frog and male frog}\hspace{5pt}\pcmn{母青蛙与公青蛙}\hspace{5pt}\pfra{grenouille femelle et grenouille mâle}\end{exemple}
\begin{exemple}\pnru{pɤ˩mi˩-ʝi˥pʰv̩˩}\hspace{5pt}\peng{A species of large frog or toad, which is abundant in the Yongning plain. This is one of three species distinguished by the consultant. It is not eaten by the Naxi (nor by the Na, who do not eat any sort of frog). This is the term used for |Kaloula verrucosa and |Rana chaochiaoensis.}\hspace{5pt}\pcmn{一种大青蛙,在永宁坝子很常见。这是发音合作人认识的三种蛙之一。纳西族人不吃这种动物(摩梭人不吃任何蛙类动物)。}\hspace{5pt}\pfra{grosse grenouille (ou crapaud); animal très courant dans la plaine. C'est l'une des trois sortes de grenouilles que connaît la locutrice. Cet animal n'est pas consommé par les Naxi (ni par les Na, qui ne mangent aucune grenouille). La locutrice emploie ce terme pour |Kaloula verrucosa et |Rana chaochiaoensis.}\end{exemple}
\begin{exemple}\pnru{pɤ˩mi˩-ʝi˥pʰv̩˩-mi˩}\hspace{5pt}\peng{same meaning}\hspace{5pt}\pcmn{同上}\hspace{5pt}\pfra{même sens}\end{exemple}
\begin{exemple}\pnru{hæ̃˧ʂɯ˩-pɤ˩mi˩}\hspace{5pt}\peng{A beautiful species of frog, with a long body. It is only found in the forest, on the mountain. This is the second of three species distinguished by the consultant.}\hspace{5pt}\pcmn{一种很美的青蛙,身体很长。只出现在山上森林里。这是发音合作人认识的第二种青蛙。}\hspace{5pt}\pfra{Belle grenouille, de longue taille. Elle ne s'observe qu'en forêt, dans la montagne. C'est la deuxième des trois sortes de grenouilles que connaît la locutrice.}\end{exemple}
\begin{exemple}\pnru{dʑɯ˩-pɤ˩mi˩˥}\hspace{5pt}\peng{A species of frog with a small head and large eyes, considered by the consultant as spending most of the time in the water. This is the third of three species of frogs distinguished by the consultant. The Naxi hunt it, especially in the fifth month.}\hspace{5pt}\pcmn{一种青蛙,头小、眼睛大。这是发音合作人认识的第三种青蛙。纳西族吃这种青蛙。}\hspace{5pt}\pfra{Grenouille ayant une petite tête et de grands yeux, qui passerait le plus clair de son temps dans l'eau. C'est la troisième des trois sortes de grenouilles que connaît la locutrice. Les Naxi la chassent, la dénichant sous les cailloux des ruisseaux, surtout au cinquième mois.}\end{exemple}
\begin{exemple}\pnru{nɑ˩hĩ˥ | pɤ˧-ʂe˧ dzɯ˧; | pɤ˧-ɣɯ˧ | ɬɑ˧tɑ˥ mv̩˩! | pɤ˧-mæ˧, | bæ˧ʈʂo˥ ʝi˩!}\hspace{5pt}\peng{Proverb: “The Naxi eat frog meat; they wear vests made of frog skin; and they make brooms with frog tails!"}\hspace{5pt}\pcmn{谚语:“纳西人吃青蛙,披青蛙皮衣,蛙尾巴当扫帚!”}\hspace{5pt}\pfra{«Les Naxi mangent de la viande de grenouille; ils se vêtent de gilets en peau de grenouille; ils se font des balais avec la queue des grenouilles!»}\end{exemple}
\end{entrée}

\begin{entrée}
{pɤ˩pʰv̩˩}{}{ⓔpɤ˩pʰv̩˩}\formedesurface{pɤ˩pʰv̩˩˥}\newline
\classe{名词}\ton{L}
\paradigme{\pcmn{:} \p{}}
\begin{définition}\peng{Male frog.}\end{définition}
\begin{définition}\pcmn{公青蛙}\end{définition}
\begin{définition}\pfra{Grenouille mâle.}\end{définition}
\end{entrée}

\begin{entrée}
{pɤ˧ʁɑ˧}{}{ⓔpɤ˧ʁɑ˧}\formedesurface{ɖɯ˧ pɤ˧ʁɑ˧}\newline
\classe{量词}\ton{M}\begin{définition}\peng{A big step.}\end{définition}
\begin{définition}\pcmn{量词:一大步}\end{définition}
\begin{définition}\pfra{Un grand pas.}\end{définition}
\begin{exemple}\pnru{ɖɯ˧-pɤ˧ʁɑ˧∼ɖɯ˧-pɤ˧ʁɑ˧}\hspace{5pt}\peng{with great strides}\hspace{5pt}\pcmn{大步流星地}\hspace{5pt}\pfra{à grands pas}\end{exemple}
\begin{exemple}\pnru{ɲi˧-pɤ˧ʁɑ˧}\hspace{5pt}\peng{two great strides}\hspace{5pt}\pcmn{两大步}\hspace{5pt}\pfra{deux grandes enjambées}\end{exemple}
\end{entrée}

\begin{entrée}
{pɤ˩ti\#˥}{}{ⓔpɤ˩ti\#˥}\formedesurface{pɤ˩ti˥}\newline
\classe{名词}\ton{LM+\#H}
\paradigme{\pcmn{:} \p{}}
\begin{définition}\peng{Stool, small bench.}\end{définition}
\begin{définition}\pcmn{凳子}\end{définition}
\begin{définition}\pfra{Tabouret, petit banc.}\end{définition}
\end{entrée}

\begin{entrée}
{‑pɤ˧to˩}{}{ⓔ‑pɤ˧to˩}\formedesurface{pɤ˧to˩}\newline
\classe{助词}\ton{L\#}\begin{définition}\peng{Even.}\end{définition}
\begin{définition}\pcmn{连}\end{définition}
\begin{définition}\pfra{Même.}\end{définition}
\begin{exemple}\pnru{ʈʂʰɯ˧ | li˩-pɤ˥to˩ | ʈʰɯ˩-ɲi˥!}\hspace{5pt}\peng{She even drinks tea! (About the eating and drinking habits of a one-year-old child)}\hspace{5pt}\pcmn{她连茶都喝!(关于一个一岁孩子的饮食习惯)}\hspace{5pt}\pfra{Elle boit même du thé! (au sujet de l'alimentation d'un enfant d'un an)}\end{exemple}
\begin{exemple}\pnru{ʈʂʰɯ˧ | pɤ˩jɤ˧-pɤ˥to˩ | dzɯ˩-ɲi˥!}\hspace{5pt}\peng{She even eats bread! (About the eating and drinking habits of a one-year-old child)}\hspace{5pt}\pcmn{她连面包都吃!}\hspace{5pt}\pfra{Elle mange même du pain! (au sujet de l'alimentation d'un enfant d'un an)}\end{exemple}
\begin{exemple}\pnru{hæ˧, | kʰv̩˩mi˩-ʂe˩-pɤ˥to˩ dzɯ˩-kv̩˩!}\hspace{5pt}\peng{The (Han) Chinese even eat dog meat! (Note: consumption of dog meat is forbidden in Na culture)}\hspace{5pt}\pcmn{汉族连狗肉都吃!(注:摩梭人不吃狗肉)}\hspace{5pt}\pfra{les Chinois, ils mangent même du chien! (Note: l'un des interdits alimentaires na concerne la viande de chien, le chien étant un animal sacré dans la culture na.)}\end{exemple}
\begin{exemple}\pnru{hæ˧, | kʰv̩˩mi˩-ʂe˩˥ F dzɯ˩-kv̩˩!}\hspace{5pt}\peng{as above}\hspace{5pt}\pcmn{同上}\hspace{5pt}\pfra{même sens}\end{exemple}
\begin{exemple}\pnru{bo˩-pɤ˥to˩; lɑ˧-pɤ˧to˩; mv̩˩-pɤ˥to˩; ʐwæ˧-pɤ˧to˩; ɬi˧mi˧-pɤ˧to˩; ɲi˧mi˧-pɤ˧to˩; hwɤ˧li˧-pɤ˥-to˩; hwɤ˧mi˧-pɤ˥to˩; kʰv̩˩mi˩-pɤ˥-to˩; ʁo˧dzi˩-pɤ˩to˩; ʝi˩ʈʂæ˧-pɤ˥to˩; nɑ˩hĩ˧-pɤ˧to˩; bo˩mi˧-pɤ˧to˩; bo˩ɬɑ˧-pɤ˩to˩; ʁæ˧ʈv̩˥-pɤ˩to˩}\hspace{5pt}\peng{combinations with nouns of the various tone categories}\hspace{5pt}\pcmn{与不同声调类的名词结合}\hspace{5pt}\pfra{en association avec des noms des diverses catégories tonales}\end{exemple}
\end{entrée}

\begin{entrée}
{pɤ˧tv̩˥}{}{ⓔpɤ˧tv̩˥}\formedesurface{pɤ˧tv̩˥}\newline
\classe{名词}\ton{H\#}\begin{définition}\peng{Wickerwork basket.}\end{définition}
\begin{définition}\pcmn{篮子、竹篮}\end{définition}
\begin{définition}\pfra{Panier de vannerie.}\end{définition}
\begin{exemple}\pnru{ɖʐɯ˧ʂɯ˥-pɤ˩tv̩˩}\hspace{5pt}\peng{small basket where chopsticks are kept}\hspace{5pt}\pcmn{筷子篮}\hspace{5pt}\pfra{panier (traditionnellement: en vannerie) dans lequel on range les baguettes}\end{exemple}
\end{entrée}

\begin{entrée}
{pɤ˧tʰi˩}{}{ⓔpɤ˧tʰi˩}\formedesurface{pɤ˧tʰi˩}\newline
\classe{名词}\ton{L\#}\begin{définition}\peng{A family name from Yongning. There are two families in Yongning that carry this name.}\end{définition}
\begin{définition}\pcmn{一个姓。这个姓,永宁有两家}\end{définition}
\begin{définition}\pfra{Nom de clan/famille étendue. Deux familles portent ce nom à Yongning.}\end{définition}
\begin{exemple}\pnru{pɤ˧tʰi˩=ɻ̍˩}\hspace{5pt}\peng{the /pɤ˧tʰi˩/ clan, the /pɤ˧tʰi˩/ family}\hspace{5pt}\pcmn{|fv{/pɤ˧tʰi˩/}家族}\hspace{5pt}\pfra{le clan /pɤ˧tʰi˩/, la famille /pɤ˧tʰi˩/}\end{exemple}
\end{entrée}

\begin{entrée}
{pɤ˩tɕɯ˧˥}{}{ⓔpɤ˩tɕɯ˧˥}\formedesurface{pɤ˩tɕɯ˧˥}\newline
\classe{名词}\ton{LM+MH\#}
\paradigme{\pcmn{:} \p{}}
\begin{définition}\peng{Tadpole.}\end{définition}
\begin{définition}\pcmn{蝌蚪}\end{définition}
\begin{définition}\pfra{Têtard.}\end{définition}
\end{entrée}

\begin{entrée}
{pɤ˩tɕɯ˧-pɤ˥mi˩}{}{ⓔpɤ˩tɕɯ˧-pɤ˥mi˩}\formedesurface{pɤ˩tɕɯ˧pɤ˥mi˩}\newline
\classe{名词}\ton{LM+\#H-}
\paradigme{\pcmn{:} \p{}}
\begin{définition}\peng{Tadpole.}\end{définition}
\begin{définition}\pcmn{蝌蚪}\end{définition}
\begin{définition}\pfra{Têtard.}\end{définition}
\end{entrée}

\begin{entrée}
{pɤ˩tɕɯ˧-ʁo˧ɖɯ˧˥}{}{ⓔpɤ˩tɕɯ˧-ʁo˧ɖɯ˧˥}\formedesurface{pɤ˩tɕɯ˧ʁo˧ɖɯ˧˥}\newline
\classe{名词}\ton{LM-MH\#}
\paradigme{\pcmn{:} \p{}}
\begin{définition}\peng{Tadpole.}\end{définition}
\begin{définition}\pcmn{蝌蚪}\end{définition}
\begin{définition}\pfra{Têtard.}\end{définition}
\end{entrée}

\begin{entrée}
{pɤ˧ʈʂʰwæ˩}{}{ⓔpɤ˧ʈʂʰwæ˩}\formedesurface{pɤ˧ʈʂʰwæ˩}\newline
\classe{形容词}\ton{L\#}\begin{définition}\peng{Bechuae.}\end{définition}
\begin{définition}\pcmn{(鼻子)扁、被压扁、凹下去}\end{définition}
\begin{définition}\pfra{Écrasé, plat, aplati, raplapla, écrabouillé.}\end{définition}
\begin{exemple}\pnru{ɲi˧gɤ˧ pɤ˧ʈʂʰwæ˥}\hspace{5pt}\peng{flat nose}\hspace{5pt}\pcmn{扁鼻子}\hspace{5pt}\pfra{nez aplati/camus; littéralement «nez écrasé»}\end{exemple}
\begin{exemple}\pnru{le˧-pɤ˥ʈʂʰwæ˩-ze˩}\hspace{5pt}\peng{|fg{accomp} \_ |fg{pfv}}\hspace{5pt}\pcmn{被压扁}\hspace{5pt}\pfra{|fg{accomp} \_ |fg{pfv}}\end{exemple}
\begin{exemple}\pnru{ʈʂʰɯ˧-v̩˧ le˧-ʈʰi˩-zo˩, | dɤ˧-qo˧ | tʰi˧-pɤ˥ʈʂʰwæ˩-ze˩!}\hspace{5pt}\peng{That one (=that man/woman) was exhausted, (s)he collapsed over there!}\hspace{5pt}\pcmn{那个(人)筋疲力尽,躺倒在那边了!}\hspace{5pt}\pfra{il s'est effondré là-bas, recru de fatigue!}\end{exemple}
\begin{exemple}\pnru{hæ˧mi˧-pɤ˧ʈʂʰwæ˧}\hspace{5pt}\peng{flat(-faced) Chinese woman (derogatory comment on noses that are not high enough by local standards)}\hspace{5pt}\pcmn{(脸、鼻子)扁的汉族女人(带偏见的称呼)}\hspace{5pt}\pfra{Chinoise Han (au visage) raplapla (commentaire désobligeant à l'égard de dames au nez trop peu saillant selon les critères locaux)}\end{exemple}
\end{entrée}

\begin{entrée}
{pɤ˩˧ʐv̩˩}{}{ⓔpɤ˩˧ʐv̩˩}\formedesurface{pɤ˩˧ʐv̩˩}\newline
\classe{名词}\ton{LM-L}
\paradigme{\pcmn{:} \p{}}
\begin{définition}\peng{Mattress.}\end{définition}
\begin{définition}\pcmn{褥子(汉语借词:被褥)}\end{définition}
\begin{définition}\pfra{Matelas.}\end{définition}
\end{entrée}

\begin{entrée}
{pi˥}{}{ⓔpi˥}\formedesurface{pi˧}\newline
\classe{动词}\ton{H}\begin{définition}\peng{To say.}\end{définition}
\begin{définition}\pcmn{说}\end{définition}
\begin{définition}\pfra{Dire.}\end{définition}
\begin{exemple}\pnru{tʰɑ˧-pi˥!}\hspace{5pt}\peng{Don't say it! / Don't speak about it!}\hspace{5pt}\pcmn{别说!}\hspace{5pt}\pfra{Il ne faut pas (le) dire!}\end{exemple}
\begin{exemple}\pnru{ə˧tso˧ pi˧?}\hspace{5pt}\peng{What did you say? (Call for repetition)}\hspace{5pt}\pcmn{(你刚才)说什么?(请人家重新说一遍)}\hspace{5pt}\pfra{Que dis-tu? (employé pour demander à quelqu'un de répéter)}\end{exemple}
\begin{exemple}\pnru{ə˧tso˧ pi˧-ɲi˥?}\hspace{5pt}\peng{What did you say? (Call for repetition)}\hspace{5pt}\pcmn{(你刚才)说什么?(请人家重新说一遍)}\hspace{5pt}\pfra{Que dis-tu? (employé pour demander à quelqu'un de répéter)}\end{exemple}
\end{entrée}

\begin{entrée}
{pi˧˥α}{}{ⓔpi˧˥α}\formedesurface{ɖɯ˧ pi˧˥}\newline
\classe{量词}\ton{MHα}\begin{définition}\peng{A little (noncount); mostly appears in combination with the numeral ‘one'.}\end{définition}
\begin{définition}\pcmn{量词:少}\end{définition}
\begin{définition}\pfra{Peu (indénombrable), un peu; souvent employé comme hypocoristique.}\end{définition}
\begin{exemple}\pnru{ɖɯ˧-pi˧˥}\hspace{5pt}\peng{a little}\hspace{5pt}\pcmn{一点}\hspace{5pt}\pfra{un peu}\end{exemple}
\begin{exemple}\pnru{qʰæ˧-pi˩}\hspace{5pt}\peng{a little manure}\hspace{5pt}\pcmn{一点粪肥}\hspace{5pt}\pfra{un peu de crottin (ramassé comme engrais)}\end{exemple}
\begin{exemple}\pnru{ŋv̩˧-pi˧}\hspace{5pt}\peng{a little money}\hspace{5pt}\pcmn{一点钱}\hspace{5pt}\pfra{un peu d’argent}\end{exemple}
\begin{exemple}\pnru{ŋv̩˧ | ɖɯ˧-pi˧˥}\hspace{5pt}\peng{a little money}\hspace{5pt}\pcmn{一点钱}\hspace{5pt}\pfra{un peu d’argent}\end{exemple}
\begin{exemple}\pnru{lwɤ˧˥ | ɖɯ˧ pi˧˥}\hspace{5pt}\peng{a little ashes}\hspace{5pt}\pcmn{一点灰}\hspace{5pt}\pfra{un peu de cendre; on ne peut dire: *|fv{lwɤ˧-pi˥}, non plus que: *tsʰe˧-pi˩ (pour ‘un peu de sel’)}\end{exemple}
\begin{exemple}\pnru{ʈʂʰɯ˧ | ɖʐe˧ ɖɯ˧-pi˧ dʑo˧!}\hspace{5pt}\peng{He has a little money! / He is rather affluent!}\hspace{5pt}\pcmn{他有一些钱!}\hspace{5pt}\pfra{il a un peu d'argent!}\end{exemple}
\end{entrée}

\begin{entrée}
{pi˩˥}{}{ⓔpi˩˥}\formedesurface{pi˩˥}\newline
\classe{名词}\ton{LH}\begin{définition}\peng{Brush for writing or painting (Chinese borrowing).}\end{définition}
\begin{définition}\pcmn{笔}\end{définition}
\begin{définition}\pfra{Pinceau pour écrire ou peindre (emprunt ancien).}\end{définition}
\begin{exemple}\pnru{tʰæ˧ɻæ˩ tɕɯ˩-di˩, | pi˩˥!}\hspace{5pt}\peng{The thing used to write is called “pen"!}\hspace{5pt}\pcmn{用来写字的那个东西,(叫做)“笔”!}\hspace{5pt}\pfra{Le truc pour écrire, ça s'appelle ‘pinceau’!}\end{exemple}
\end{entrée}

\begin{entrée}
{pi˧lv̩\#˥}{}{ⓔpi˧lv̩\#˥}\formedesurface{pi˧lv̩˧}\newline
\classe{名词}\ton{\#H}\begin{définition}\peng{Residue left by the production of alcohol, distiller's grains: grains that are fed to the pigs.}\end{définition}
\begin{définition}\pcmn{酒糟:煮酒剩下的渣滓(一般给猪吃)}\end{définition}
\begin{définition}\pfra{Déchet de la distillation: ce qui reste après la production de l'alcool; grain qu'on donne aux animaux.}\end{définition}
\begin{exemple}\pnru{pi˧lv̩˧, | hĩ˧ | dzɯ˧-mɤ˧-kv̩˩!}\hspace{5pt}\peng{Distiller's grains are not suitable for human consumption! / People don't eat distiller's grains!}\hspace{5pt}\pcmn{酒糟,人不能吃!}\hspace{5pt}\pfra{Les grains après distillation, ça ne se mange pas! / ce n'est pas propre à la consommation humaine!}\end{exemple}
\end{entrée}

\begin{entrée}
{pi˧mɑ˧}{}{ⓔpi˧mɑ˧}\formedesurface{pi˧mɑ˧}\newline
\classe{名词}\ton{M}\begin{définition}\peng{A unixex given name: a given name used for both men and women.}\end{définition}
\begin{définition}\pcmn{男女通用名}\end{définition}
\begin{définition}\pfra{Prénom unisexe: prénom utilisé pour les deux sexes.}\end{définition}
\end{entrée}

\begin{entrée}
{pi˧mɑ˧-ɬɑ˩mv̩˩}{}{ⓔpi˧mɑ˧-ɬɑ˩mv̩˩}\formedesurface{pi˧mɑ˧ɬɑ˩mv̩˩}\newline
\classe{名词}\ton{-L}\begin{définition}\peng{Feminine given name.}\end{définition}
\begin{définition}\pcmn{女性名字}\end{définition}
\begin{définition}\pfra{Prénom féminin.}\end{définition}
\end{entrée}

\begin{entrée}
{pi˧mɑ˧-ɬɑ˩tsʰo˩}{}{ⓔpi˧mɑ˧-ɬɑ˩tsʰo˩}\formedesurface{pi˧mɑ˧ɬɑ˩tsʰo˩}\newline
\classe{名词}\ton{-L}\begin{définition}\peng{Feminine given name.}\end{définition}
\begin{définition}\pcmn{女性名字}\end{définition}
\begin{définition}\pfra{Prénom féminin.}\end{définition}
\end{entrée}

\begin{entrée}
{pi˧mv̩˥\$}{}{ⓔpi˧mv̩˥\$}\formedesurface{pi˧mv̩˥}\newline
\classe{名词}\ton{H\$}\begin{définition}\peng{Set phrase, idiom, adage.}\end{définition}
\begin{définition}\pcmn{成语、俗语}\end{définition}
\begin{définition}\pfra{Dicton, parole du temps jadis, adage.}\end{définition}
\end{entrée}

\begin{entrée}
{pi˩mv̩˥}{}{ⓔpi˩mv̩˥}\formedesurface{pi˩mv̩˥}\newline
\classe{名词}\ton{L}
\paradigme{\pcmn{:} \p{}}
\begin{définition}\peng{|\stylefi{Fritillaria cirrhosa}.}\end{définition}
\begin{définition}\pcmn{贝母(汉语借词)}\end{définition}
\begin{définition}\pfra{|\stylefi{Fritillaria cirrhosa}.}\end{définition}
\end{entrée}

\begin{entrée}
{pi˩ɻ̍˥}{}{ⓔpi˩ɻ̍˥}\formedesurface{pi˩ɻ̍˥}\newline
\classe{名词}\ton{LH}
\paradigme{\pcmn{:} \p{}}
\begin{définition}\peng{Double chin; flesh under the chin.}\end{définition}
\begin{définition}\pcmn{双下巴}\end{définition}
\begin{définition}\pfra{Double menton, bourrelet de chair sous le menton.}\end{définition}
\end{entrée}

\begin{entrée}
{pi˩ti\#˥}{}{ⓔpi˩ti\#˥}\formedesurface{pi˩ti˥}\newline
\classe{名词}\ton{LM+\#H}
\paradigme{\pcmn{:} \p{}}
\begin{définition}\peng{Silver nugget, piece of raw silver.}\end{définition}
\begin{définition}\pcmn{银块}\end{définition}
\begin{définition}\pfra{Pépite d'argent.}\end{définition}
\end{entrée}

\begin{entrée}
{pi˧tsʰe˩-di˩}{}{ⓔpi˧tsʰe˩-di˩}\formedesurface{pi˧tsʰe˩di˩}\newline
\classe{名词}\ton{L\#-}\begin{définition}\peng{A village close to the Hot Springs.}\end{définition}
\begin{définition}\pcmn{比其地村:温泉乡的一个村落}\end{définition}
\begin{définition}\pfra{Un village proche des Sources Chaudes.}\end{définition}
\begin{exemple}\pnru{ə˧go˧-ʁwɤ˧, | ʁwɤ˧lɑ˩-bi˩, | bæ˧ʁwɤ˧, | tʰo˧tsʰe\#˥, | pi˧tsʰe˩-di˩, | pɤ˧dʑɤ˩-di˩, | ʁwɤ˧tv̩˧}\hspace{5pt}\peng{Seven villages that one encounters as one leaves the plain of Yongning (towards the Lake); the first two are perceived as villages with a high proportion of Na members, and the third as a mostly Na village, whereas the next two are Pumi (Prinmi); the last used to be predominantly Pumi, but as of the 2010s, it had an important Chinese (Han) population.}\hspace{5pt}\pcmn{永宁背向泸沽湖方向经过的七个村落:阿公瓦、瓦拉比、巴瓦、拖其、比其地、巴甲地、瓦都。前两个村落拥有相当大的摩梭人口比例,第三主要是摩梭村。拖其、比其地、巴甲地是普米村。瓦都,过去主要是普米族村,到了2010年代有了相当多的汉族人口。}\hspace{5pt}\pfra{Sept villages au sortir de la plaine de Yongning, dans la direction du Lac; les deux premiers comportent une population na; le troisième est un village na; les deux suivants sont essentiellement des villages pumi/prinmi; le dernier était un village pumi, et a désormais (dans les années 2010) une importante population chinoise (han).}\end{exemple}
\begin{exemple}\pnru{pi˧tsʰe˩: bɤ˩! |}\hspace{5pt}\peng{/pi˧tsʰe˧/ is a Pumi village!}\hspace{5pt}\pcmn{/pi˧tsʰe˩/是一个普米族村落!}\hspace{5pt}\pfra{/pi˧tsʰe˩/, c'est un village pumi!}\end{exemple}
\end{entrée}

\begin{entrée}
{pjɤ˥}{}{ⓔpjɤ˥}\formedesurface{pjɤ˧}\newline
\classe{形容词}\ton{H}\begin{définition}\peng{Square.}\end{définition}
\begin{définition}\pcmn{方形的}\end{définition}
\begin{définition}\pfra{Carré/anguleux (visage, pilier…).}\end{définition}
\begin{exemple}\pnru{tɑ˧-pjɤ˧∼pjɤ˥ (-zo˩)}\hspace{5pt}\peng{(of a face or object) unpleasantly squarish, lacking smoothness}\hspace{5pt}\pcmn{(脸、物品)太方,不圆滑}\hspace{5pt}\pfra{un peu anguleux/carré (terme péjoratif: objet / physique trop peu lisse pour être plaisant au regard ou au toucher)}\end{exemple}
\end{entrée}

\begin{entrée}
{po˥}{}{ⓔpo˥}\formedesurface{po˧}\newline
\classe{动词}\begin{définition}\peng{To pack.}\end{définition}
\begin{définition}\pcmn{包(量词)(汉语借词)}\end{définition}
\begin{définition}\pfra{Emballer.}\end{définition}
\begin{exemple}\pnru{le˧-po˥}\hspace{5pt}\peng{|fg{accomp}}\hspace{5pt}\pcmn{|fg{accomp}}\hspace{5pt}\pfra{|fg{accomp}}\end{exemple}
\end{entrée}

\begin{entrée}
{po˧˥}{}{ⓔpo˧˥}\newline
\classe{动词}
\sens{1}
\begin{définition}\peng{To bring; to send (a letter), to deliver (a message).}\end{définition}
\begin{définition}\pcmn{寄信、服送、带过来、拿、送}\end{définition}
\begin{définition}\pfra{Amener, apporter; ramener, rapporter; faire cadeau de; envoyer (un message), transmettre; utiliser.}\end{définition}
\begin{exemple}\pnru{qʰwæ˧ po˧˥}\hspace{5pt}\peng{to bring a letter/a message}\hspace{5pt}\pcmn{带来一封信/一个消息}\hspace{5pt}\pfra{amener une lettre/un message}\end{exemple}
\begin{exemple}\pnru{tso˧∼tso˧ tʰi˧-po˧˥}\hspace{5pt}\peng{to bring something}\hspace{5pt}\pcmn{带来一个东西}\hspace{5pt}\pfra{amener quelque chose}\end{exemple}
\begin{exemple}\pnru{hĩ˧ ɖɯ˧-v̩˧ | tso˧∼tso˧ ɖɯ˧-kʰwɤ˥ | tʰi˧-po˧˥}\hspace{5pt}\peng{someone brings something}\hspace{5pt}\pcmn{有人带东西过来}\hspace{5pt}\pfra{quelqu'un prend/amène quelque chose}\end{exemple}
\begin{exemple}\pnru{ʈʂʰwæ˧˥ | po˧-jo˥!}\hspace{5pt}\peng{Bring it over, quick!}\hspace{5pt}\pcmn{快拿过来吧!/ 快带过来吧!}\hspace{5pt}\pfra{amène vite!}\end{exemple}\sens{2}
\begin{définition}\peng{To carry (a child), i.e. to be pregnant.}\end{définition}
\begin{définition}\pcmn{怀孕}\end{définition}
\begin{définition}\pfra{Porter un enfant, c'est-à-dire être enceinte.}\end{définition}
\begin{exemple}\pnru{ʈʂʰɯ˧ | zo˧mv̩˥ po˩.}\hspace{5pt}\peng{She is pregnant.}\hspace{5pt}\pcmn{她怀孕了。}\hspace{5pt}\pfra{Elle est enceinte.}\end{exemple}
\begin{exemple}\pnru{zo˧ po˩ (+ze˩)}\hspace{5pt}\peng{to carry a child, i.e. to be pregnant}\hspace{5pt}\pcmn{怀孕}\hspace{5pt}\pfra{porter un enfant, être enceinte.}\end{exemple}
\end{entrée}

\begin{entrée}
{po˧α}{}{ⓔpo˧α}\formedesurface{ɖɯ˧ po˧}\newline
\classe{量词}\ton{Mα}\begin{définition}\peng{Classifier for plants with a stalk; also used for pieces of clothing.}\end{définition}
\begin{définition}\pcmn{量词:有根的植物,衣服(一棵,一件)}\end{définition}
\begin{définition}\pfra{Classificateur des plantes à tiges (fleurs, poireaux…); aussi utilisé pour compter les types/catégories de vêtements.}\end{définition}
\end{entrée}

\begin{entrée}
{po˩β}{}{ⓔpo˩β}\formedesurface{ɖɯ˧ po˩}\newline
\classe{量词}\ton{Lβ}\begin{définition}\peng{Classifier for packs (e.g. a pack of cigarettes).}\end{définition}
\begin{définition}\pcmn{量词:包(汉语借词)}\end{définition}
\begin{définition}\pfra{Classificateur des paquets.}\end{définition}
\end{entrée}

\begin{entrée}
{po˧ɖʐɯ\#˥}{}{ⓔpo˧ɖʐɯ\#˥}\formedesurface{po˧ɖʐɯ˧}\newline
\classe{名词}\ton{\#H}
\paradigme{\pcmn{:} \p{}}
\begin{définition}\peng{Craftsman.}\end{définition}
\begin{définition}\pcmn{工匠}\end{définition}
\begin{définition}\pfra{Artisan.}\end{définition}
\begin{exemple}\pnru{po˧ɖʐɯ˧ ʝi˧-hĩ˧-hĩ˧}\hspace{5pt}\peng{person who works as a craftsman}\hspace{5pt}\pcmn{当工匠的人}\hspace{5pt}\pfra{personne qui travaille comme artisan}\end{exemple}
\end{entrée}

\begin{entrée}
{po˧lo˧}{}{ⓔpo˧lo˧}\formedesurface{po˩lo˧}\newline
\classe{名词}\ton{M}
\paradigme{\pcmn{:} \p{}}
\begin{définition}\peng{Ram.}\end{définition}
\begin{définition}\pcmn{公绵羊}\end{définition}
\begin{définition}\pfra{Bélier; bouc.}\end{définition}
\begin{exemple}\pnru{po˧lo˧ lɑ˧˥}\hspace{5pt}\peng{to strike a ram}\hspace{5pt}\pcmn{打公绵羊}\hspace{5pt}\pfra{frapper un bélier}\end{exemple}
\end{entrée}

\begin{entrée}
{po˧po˧}{}{ⓔpo˧po˧}\formedesurface{po˧po˧}\newline
\classe{名词}\ton{M}
\paradigme{\pcmn{:} \p{}}
\begin{définition}\peng{Ball.}\end{définition}
\begin{définition}\pcmn{球}\end{définition}
\begin{définition}\pfra{Ballon.}\end{définition}
\begin{exemple}\pnru{po˧po˧ lɑ˧˥}\hspace{5pt}\peng{to play (foot)ball}\hspace{5pt}\pcmn{打球}\hspace{5pt}\pfra{jouer au ballon}\end{exemple}
\end{entrée}

\begin{entrée}
{po˧po˧tsʰɤ˧˥}{}{ⓔpo˧po˧tsʰɤ˧˥}\formedesurface{po˧po˧tsʰɤ˧˥}\newline
\classe{名词}\ton{MH\#}
\paradigme{\pcmn{:} \p{}}
\begin{définition}\peng{Cabbage.}\end{définition}
\begin{définition}\pcmn{圆白菜}\end{définition}
\begin{définition}\pfra{Chou.}\end{définition}
\end{entrée}

\begin{entrée}
{pv̩˧}{₁}{ⓔpv̩˧ⓗ1}\formedesurface{pv̩˧}\newline
\classe{动词}\ton{Mγ}
1\begin{définition}\peng{To perform (a sacrifice, a ritual), to celebrate (a festival), to chant (a ritual).}\end{définition}
\begin{définition}\pcmn{祭}\end{définition}
\begin{définition}\pfra{Faire un sacrifice, faire un rituel, psalmodier.}\end{définition}
\begin{exemple}\pnru{kʰv̩˧ pv̩˥}\hspace{5pt}\peng{to do the New Year ceremony, to celebrate the New Year}\hspace{5pt}\pcmn{做过年的祭礼}\hspace{5pt}\pfra{faire la veillée du Nouvel An (la veille de la nouvelle année), célébrer la veillée du Nouvel An}\end{exemple}
\begin{exemple}\pnru{tsʰi˧ɲi˧, | kʰv̩˧ pv̩˥-tso˩-ɲi˩!}\hspace{5pt}\peng{Tonight, we are going to celebrate the New Year!}\hspace{5pt}\pcmn{今天就要过年了!}\hspace{5pt}\pfra{ce soir, on va fêter le Nouvel An!}\end{exemple}
\end{entrée}

\begin{entrée}
{pv̩˧}{₂}{ⓔpv̩˧ⓗ2}\formedesurface{pv̩˧}\newline
\classe{形容词}\ton{M}
2\begin{définition}\peng{Dry.}\end{définition}
\begin{définition}\pcmn{干燥}\end{définition}
\begin{définition}\pfra{Sec.}\end{définition}
\begin{exemple}\pnru{le˧-pv̩˧-ze˧}\hspace{5pt}\peng{|fg{accomp} \_ |fg{pfv}}\hspace{5pt}\pcmn{干了}\hspace{5pt}\pfra{|fg{accomp} \_ |fg{pfv}}\end{exemple}
\begin{exemple}\pnru{le˧-pv̩˧ le˧-ʐwæ˩-ze˩}\hspace{5pt}\peng{It has dried up / it has completely dried / it is now completely dry}\hspace{5pt}\pcmn{干透了}\hspace{5pt}\pfra{ça a complètement séché/c'est entièrement sec/c'est parfaitement sec}\end{exemple}
\begin{exemple}\pnru{pv̩˧-kæ˧-ɻæ˩-gv̩˩}\hspace{5pt}\peng{all dry, completely dry}\hspace{5pt}\pcmn{全干、完全干}\hspace{5pt}\pfra{tout sec}\end{exemple}
\begin{exemple}\pnru{si˧ pv̩˩}\hspace{5pt}\peng{dry wood}\hspace{5pt}\pcmn{干的木头}\hspace{5pt}\pfra{bois sec}\end{exemple}
\end{entrée}

\begin{entrée}
{pv̩˧˥}{₁}{ⓔpv̩˧˥ⓗ1}\formedesurface{pv̩˧˥}\newline
\classe{动词}\ton{MH}
1\begin{définition}\peng{To pull out (weeds), to weed.}\end{définition}
\begin{définition}\pcmn{拔、扯(草)}\end{définition}
\begin{définition}\pfra{Enlever, arracher (les mauvaises herbes); couper du fourrage pour les animaux domestiques.}\end{définition}
\begin{exemple}\pnru{zɯ˧ pv̩˩}\hspace{5pt}\peng{to pull out (weeds), to weed; to cut grass for cattle}\hspace{5pt}\pcmn{拔草}\hspace{5pt}\pfra{arracher les mauvaises herbes; couper du fourrage pour les animaux domestiques}\end{exemple}
\begin{exemple}\pnru{zɯ˧ | le˧-pv̩˧˥}\hspace{5pt}\peng{to pull out (weeds), to weed; to cut grass for cattle}\hspace{5pt}\pcmn{拔草}\hspace{5pt}\pfra{arracher les mauvaises herbes; couper du fourrage pour les animaux domestiques}\end{exemple}
\end{entrée}

\begin{entrée}
{pv̩˧˥}{₂}{ⓔpv̩˧˥ⓗ2}\formedesurface{pv̩˧˥}\newline
\classe{动词}\ton{MH}
2\begin{définition}\peng{To draw (a weapon), to take out of its sheath.}\end{définition}
\begin{définition}\pcmn{拉出(剑……)}\end{définition}
\begin{définition}\pfra{Dégainer (une arme blanche), sortir du fourreau.}\end{définition}
\begin{exemple}\pnru{ʁæ˧mi˧ | tʰi˧-pv̩˧˥}\hspace{5pt}\peng{to draw a sword}\hspace{5pt}\pcmn{拉出剑}\hspace{5pt}\pfra{dégainer une épée}\end{exemple}
\begin{exemple}\pnru{gæ˩-pv̩˧˥}\hspace{5pt}\peng{to draw (a weapon), to take out of its sheath}\hspace{5pt}\pcmn{拉出(剑……)}\hspace{5pt}\pfra{dégainer, sortir une arme de son fourreau}\end{exemple}
\begin{exemple}\pnru{ʁæ˧mi˧ | gæ˩-pv̩˧˥}\hspace{5pt}\peng{to draw a sword}\hspace{5pt}\pcmn{拉出剑}\hspace{5pt}\pfra{dégainer une épée}\end{exemple}
\end{entrée}

\begin{entrée}
{pv̩˧˥}{₃}{ⓔpv̩˧˥ⓗ3}\formedesurface{ɖɯ˧ pv̩˧˥}\newline
\classe{量词}\ton{MHα}
3\begin{définition}\peng{Classifier for steps / strides.}\end{définition}
\begin{définition}\pcmn{量词:步}\end{définition}
\begin{définition}\pfra{Classificateur des pas/enjambées; emprunt au chinois.}\end{définition}
\end{entrée}

\begin{entrée}
{pv̩˩}{}{ⓔpv̩˩}\formedesurface{pv̩˩˥}\newline
\classe{动词}\ton{L}\begin{définition}\peng{To go by, to flow (of time).}\end{définition}
\begin{définition}\pcmn{过、过去(时间过去、日子过去)}\end{définition}
\begin{définition}\pfra{Passer, s'écouler: le temps passe, les jours passent.}\end{définition}
\begin{exemple}\pnru{ɲi˧mi˧ pv̩˩}\hspace{5pt}\peng{time goes by; literally: the day goes by}\hspace{5pt}\pcmn{时间过去。直译:(一)天(慢慢)过(去)}\hspace{5pt}\pfra{le temps passe, la journée passe}\end{exemple}
\begin{exemple}\pnru{ɲi˧mi˧ | le˧-pv̩˩-ze˩}\hspace{5pt}\peng{time has gone by, the day has gone by}\hspace{5pt}\pcmn{时间过去了,(一)天过去了}\hspace{5pt}\pfra{le temps a passé, la journée a passé}\end{exemple}
\begin{exemple}\pnru{dʑɤ˩-dzɯ˧ qʰwɤ˧-dzɯ˥, | bi˧mi˧ ʂv̩˧-qʰwɤ˧-ɻ̍˥; | dʑɤ˩-ʐwɤ˥ qʰwɤ˩-ʐwɤ˩, | ɲi˧mi˧ ʂæ˧ pv̩˩-di˩!}\hspace{5pt}\peng{Whether one eats good stuff or bad stuff, that fills the stomach / that does the trick of filling your belly! Whether one tells good stories or bad ones, that helps make the long day go by / that does the trick of chipping a long (and tedious) day away/of filling a day pleasantly! (A laid-back proverb in praise of small talk and gossip.)}\hspace{5pt}\pcmn{“吃好吃坏,(都)能填满肚子/(都)能吃饱!说好说坏,(都)能让一天(轻松)过去!”(这个谚语,说闲聊的好。)}\hspace{5pt}\pfra{Qu'on mange bien ou mal, on arrive à se remplir le ventre / Que la nourriture soit bonne ou mauvaise, peu importe au fond, tant qu'on a le ventre plein; qu'on dise des bonnes choses (=des éloges d'autrui) ou des mauvaises (=des ragots), on arrive à passer la journée / le jour se passe agréablement! (Proverbe qui fait l'éloge des vertus du bavardage et du commérage.)}\end{exemple}
\begin{exemple}\pnru{dʑɤ˩-dzɯ˧ qʰwɤ˧-dzɯ˥, | bi˧mi˧ ʂv̩˧˥; | dʑɤ˩-ʐwɤ˥ qʰwɤ˩-ʐwɤ˩, | ɲi˧mi˧ ʂæ˧-pv̩˩-di˩!}\hspace{5pt}\peng{Variant of the above proverb.}\hspace{5pt}\pcmn{上述谚语的变体}\hspace{5pt}\pfra{Variante du proverbe ci-dessus.}\end{exemple}
\end{entrée}

\begin{entrée}
{pv̩˩α}{₁}{ⓔpv̩˩αⓗ1}\formedesurface{pv̩˩˥}\newline
\classe{动词}\ton{Lα}
1\begin{définition}\peng{To see off.}\end{définition}
\begin{définition}\pcmn{送行}\end{définition}
\begin{définition}\pfra{Raccompagner; escorter; mener, conduire (du bétail).}\end{définition}
\begin{exemple}\pnru{hĩ˧bæ˧ pv̩˥}\hspace{5pt}\peng{to see a guest off}\hspace{5pt}\pcmn{送客}\hspace{5pt}\pfra{raccompagner un invité}\end{exemple}
\end{entrée}

\begin{entrée}
{pv̩˩α}{₂}{ⓔpv̩˩αⓗ2}\formedesurface{pv̩˩˥}\newline
\classe{动词}\ton{Lα}
2\begin{définition}\peng{To allow, to authorize (someone to do something, e.g. to marry); to finance (i.e. to invest money in a caravan); to require (someone to do something).}\end{définition}
\begin{définition}\pcmn{让,安排,投资,要求}\end{définition}
\begin{définition}\pfra{Autoriser (ex.: un mariage); demander (à quelqu'un de faire quelque chose), faire faire; commanditer, être commanditaire/investisseur (ex.: pour une caravane).}\end{définition}
\begin{exemple}\pnru{sɯ˧pʰi˧-ɳɯ˧ | pv̩˩-kʰɯ˥-ɲi˩!}\hspace{5pt}\peng{It was the feudal lord who financed (the caravan)!}\hspace{5pt}\pcmn{(马帮)是土司来投资的!}\hspace{5pt}\pfra{c'est le seigneur qui était le commanditaire!}\end{exemple}
\begin{exemple}\pnru{ʈʂʰɯ˧ | ɖʐe˧ ʂe˧ pv̩˩-kʰɯ˩-tso˩-ɲi˩!}\hspace{5pt}\peng{(S)he is bringing the capital! / (S)he is financing the whole thing! (e.g. a caravan)}\hspace{5pt}\pcmn{是他来投资的!(如:马帮)}\hspace{5pt}\pfra{c'est elle/lui qui apporte le capital/qui commandite! (ex.: pour une caravane)}\end{exemple}
\begin{exemple}\pnru{hĩ˧-ɳɯ˩ | pv̩˩-mɤ˩-kʰɯ˥!}\hspace{5pt}\peng{People do not allow access! / Access is not allowed! (Context: a discussion about difficulties for the investigator to be allowed to stay in an area of Sichuan where Naish languages are spoken. The consultant summarizes as: “Access is not allowed!")}\hspace{5pt}\pcmn{人家不让去!}\hspace{5pt}\pfra{On n'est pas autorisé à y aller! (Contexte: discussion au sujet des difficultés pour l'enquêteur d'accès à une localité où sont parlées des langues naish, dans le Sichuan. La consultante résume: «On n'est pas autorisé à y aller! / L'accès n'est pas autorisé!»)}\end{exemple}
\end{entrée}

\begin{entrée}
{pv̩˩α}{₃}{ⓔpv̩˩αⓗ3}\formedesurface{pv̩˩˥}\newline
\classe{动词}\ton{Lα}
3\begin{définition}\peng{To comb.}\end{définition}
\begin{définition}\pcmn{梳}\end{définition}
\begin{définition}\pfra{Peigner.}\end{définition}
\begin{exemple}\pnru{ʁo˧qʰwɤ˩ pv̩˩}\hspace{5pt}\peng{to comb one's head}\hspace{5pt}\pcmn{梳头}\hspace{5pt}\pfra{se peigner}\end{exemple}
\begin{exemple}\pnru{ʁo˧ pv̩˥}\hspace{5pt}\peng{to comb one's head}\hspace{5pt}\pcmn{梳头}\hspace{5pt}\pfra{se peigner}\end{exemple}
\end{entrée}

\begin{entrée}
{pv̩˧lv̩˧}{}{ⓔpv̩˧lv̩˧}\formedesurface{pv̩˧lv̩˧}\newline
\classe{名词}\ton{M}
\paradigme{\pcmn{:} \p{}}
\begin{définition}\peng{Nonirrigated farmland; dry land.}\end{définition}
\begin{définition}\pcmn{旱地}\end{définition}
\begin{définition}\pfra{Champ sec/pluvial.}\end{définition}
\end{entrée}

\begin{entrée}
{pv̩˧ɭɯ˧}{}{ⓔpv̩˧ɭɯ˧}\formedesurface{pv̩˧ɭɯ˧}\newline
\classe{动词}\ton{M}\begin{définition}\peng{To roll (a stone rolls down a slope).}\end{définition}
\begin{définition}\pcmn{滚动(石头滚动)}\end{définition}
\begin{définition}\pfra{Rouler (une pierre roule).}\end{définition}
\begin{exemple}\pnru{pv̩˧ɭɯ˧-ze˩}\hspace{5pt}\peng{|fg{pfv}}\hspace{5pt}\pcmn{滚动了}\hspace{5pt}\pfra{|fg{pfv}}\end{exemple}
\begin{exemple}\pnru{le˧-pv̩˧-ɭɯ˧}\hspace{5pt}\peng{|fg{accomp}}\hspace{5pt}\pcmn{|fg{accomp}}\hspace{5pt}\pfra{|fg{accomp}}\end{exemple}
\end{entrée}

\begin{entrée}
{pv̩˩ɭɯ˥}{}{ⓔpv̩˩ɭɯ˥}\formedesurface{pv̩˩ɭɯ˥}\newline
\classe{名词}\ton{LH}
\paradigme{\pcmn{:} \p{}}
\begin{définition}\peng{Button.}\end{définition}
\begin{définition}\pcmn{扣子}\end{définition}
\begin{définition}\pfra{Bouton (sur un vêtement).}\end{définition}
\end{entrée}

\begin{entrée}
{pv̩˩mi˩}{}{ⓔpv̩˩mi˩}\formedesurface{pv̩˩mi˩˥}\newline
\classe{名词}\ton{L}
\paradigme{\pcmn{:} \p{}}
\begin{définition}\peng{Comb (coarse).}\end{définition}
\begin{définition}\pcmn{粗齿梳子}\end{définition}
\begin{définition}\pfra{Peigne grossier, à dents relativement écartées.}\end{définition}
\end{entrée}

\begin{entrée}
{pv̩˩pv̩˧}{}{ⓔpv̩˩pv̩˧}\formedesurface{pv̩˩pv̩˥}\newline
\classe{名词}\ton{LM}\begin{définition}\peng{Pocket.}\end{définition}
\begin{définition}\pcmn{衣兜}\end{définition}
\begin{définition}\pfra{Poche.}\end{définition}
\begin{exemple}\pnru{bɑ˩lɑ˩-pv̩˥pv̩˩}\hspace{5pt}\peng{pocket of the shirt; it can contain small objects such as tobacco and coins.}\hspace{5pt}\pcmn{上衣兜子}\hspace{5pt}\pfra{poche intérieure de chemise; on y serrait de petits objets: pièces de monnaie, tabac…}\end{exemple}
\end{entrée}

\begin{entrée}
{pv̩˧qʰwɤ˥}{}{ⓔpv̩˧qʰwɤ˥}\formedesurface{pv̩˧qʰwɤ˥}\newline
\classe{名词}\ton{H\#}
\paradigme{\pcmn{:} \p{}}
\begin{définition}\peng{Shuttle on a weaving loom (traditional shuttle made of wood).}\end{définition}
\begin{définition}\pcmn{梭,梭子(传统的,木头做的)}\end{définition}
\begin{définition}\pfra{Navette du métier à tisser: navette traditionnelle, en bois (n'est plus en usage actuellement, remplacée par une navette plus simple).}\end{définition}
\begin{exemple}\pnru{ɣɯ˧dzo˩-bv̩˩ | pv̩˧qʰwɤ˥}\hspace{5pt}\peng{the shuttle of the loom}\hspace{5pt}\pcmn{织布机的梭子}\hspace{5pt}\pfra{la navette du métier à tisser}\end{exemple}
\end{entrée}

\begin{entrée}
{pv̩˧ɻ̍\#˥}{}{ⓔpv̩˧ɻ̍\#˥}\formedesurface{pv̩˧ɻ̍˧}\newline
\classe{名词}\ton{\#H}
\paradigme{\pcmn{:} \p{}}
\begin{définition}\peng{Tibetan wool fabric.}\end{définition}
\begin{définition}\pcmn{氆氇}\end{définition}
\begin{définition}\pfra{Habit tibétain en laine (vêtement de grand prix).}\end{définition}
\end{entrée}

\begin{entrée}
{pv̩˧ʂɯ˩}{}{ⓔpv̩˧ʂɯ˩}\formedesurface{pv̩˧ʂɯ˩}\newline
\classe{名词}\ton{L\#}
\paradigme{\pcmn{:} \p{}}
\begin{définition}\peng{Amber.}\end{définition}
\begin{définition}\pcmn{琥珀}\end{définition}
\begin{définition}\pfra{Ambre.}\end{définition}
\end{entrée}

\begin{entrée}
{pv̩˩tɑ˩}{}{ⓔpv̩˩tɑ˩}\formedesurface{pv̩˩tɑ˩˥}\newline
\classe{名词}\ton{L}
\paradigme{\pcmn{:} \p{}}
\begin{définition}\peng{Bucket, pail.}\end{définition}
\begin{définition}\pcmn{桶}\end{définition}
\begin{définition}\pfra{Seau.}\end{définition}
\end{entrée}

\begin{entrée}
{pv̩˩tsɯ˧˥}{}{ⓔpv̩˩tsɯ˧˥}\newline
\classe{名词}
\sens{1}\paradigme{\pcmn{:} \p{}}
\begin{définition}\peng{Fine comb (used to comb out lice).}\end{définition}
\begin{définition}\pcmn{篦子(用来梳虱子)}\end{définition}
\begin{définition}\pfra{Peigne fin (utilisé pour épouiller).}\end{définition}\sens{2}
\begin{définition}\peng{Iron threads in a wooden frame (like a comb in which the weft is caught), used to tamp down the threads while weaving.}\end{définition}
\begin{définition}\pcmn{用来夯实布料的木头架子,里面有铁丝}\end{définition}
\begin{définition}\pfra{Fils de fer dans un cadre de bois: sorte de peigne dans lequel la trame est emprisonnée, et qui sert à tasser les fils à mesure que l'on tisse.}\end{définition}
\end{entrée}

\begin{entrée}
{pv̩˩tsɯ˧-pv̩˥mi˩}{}{ⓔpv̩˩tsɯ˧-pv̩˥mi˩}\formedesurface{pv̩˩tsɯ˧pv̩˥mi˩}\newline
\classe{名词}\ton{LM+\#H-}\begin{définition}\peng{Combs.}\end{définition}
\begin{définition}\pcmn{梳子(总称)}\end{définition}
\begin{définition}\pfra{Peignes.}\end{définition}
\end{entrée}

\begin{entrée}
{pv̩˩ʈʰɯ˧}{}{ⓔpv̩˩ʈʰɯ˧}\formedesurface{pv̩˩ʈʰɯ˥}\newline
\classe{名词}\ton{LM}\begin{définition}\peng{Feminine given name.}\end{définition}
\begin{définition}\pcmn{女性名字}\end{définition}
\begin{définition}\pfra{Prénom féminin.}\end{définition}
\end{entrée}

\begin{entrée}
{pv̩˧ʈʂɯ˩}{}{ⓔpv̩˧ʈʂɯ˩}\formedesurface{pv̩˧ʈʂɯ˩}\newline
\classe{动词}\ton{L\#}\begin{définition}\peng{To press, to squeeze.}\end{définition}
\begin{définition}\pcmn{挤、挤压}\end{définition}
\begin{définition}\pfra{Presser, serrer.}\end{définition}
\begin{exemple}\pnru{njɤ˧-ɳɯ˧ | pv̩˧ʈʂɯ˩-bi˩!}\hspace{5pt}\peng{I'm going to press (it)! / Let me press it!}\hspace{5pt}\pcmn{我来压吧!}\hspace{5pt}\pfra{Je vais presser ça! / je m'occupe de serrer ça/presser ça!}\end{exemple}
\begin{exemple}\pnru{le˧-pv̩˥ʈʂɯ˩}\hspace{5pt}\peng{|fg{accomp}}\hspace{5pt}\pcmn{|fg{accomp}}\hspace{5pt}\pfra{|fg{accomp}}\end{exemple}
\end{entrée}

\begin{entrée}
{pʰæ˧˥}{}{ⓔpʰæ˧˥}\formedesurface{pʰæ˧˥}\newline
\classe{动词}\ton{MH}\begin{définition}\peng{To shove, to push away.}\end{définition}
\begin{définition}\pcmn{推搡}\end{définition}
\begin{définition}\pfra{Écarter, pousser, jouer des coudes.}\end{définition}
\begin{exemple}\pnru{ɖɯ˩-tɕo˧ pʰæ˧˥, | ʈʂʰɯ˧-tɕo˧ pʰæ˧˥}\hspace{5pt}\peng{to shove on this side, to shove on that side (e.g. at a station, when lots of people are shoving their way towards the ticket counter)}\hspace{5pt}\pcmn{东推西挤}\hspace{5pt}\pfra{pousser par ici, pousser par là / jouer des coudes par ci, jouer des coudes par là (ex.: à la gare, quand il y a presse pour acheter un billet de train)}\end{exemple}
\begin{exemple}\pnru{ʈʂe˧ | le˧-pʰæ˩∼pʰæ˩}\hspace{5pt}\peng{to throw earth here and there: a chicken is scratching the soil to find food, and sends spurts of earth here and there}\hspace{5pt}\pcmn{将土扔这里扔那里:一只鸡在抓地找吃的,让土飞这里飞那里}\hspace{5pt}\pfra{rejeter la terre de droite et de gauche : une poule gratte la terre à la recherche de nourriture, et fait voler de la terre de droite et de gauche}\end{exemple}
\end{entrée}

\begin{entrée}
{pʰæ˧˥α}{}{ⓔpʰæ˧˥α}\formedesurface{ɖɯ˧ pʰæ˧˥}\newline
\classe{量词}\ton{MHα}\begin{définition}\peng{Classifier for flat objects, e.g. a sheet (of paper).}\end{définition}
\begin{définition}\pcmn{量词:平面的东西,如:纸(一张、一片)}\end{définition}
\begin{définition}\pfra{Classificateur des objets plats: feuilles de papier…}\end{définition}
\end{entrée}

\begin{entrée}
{pʰæ˧β}{}{ⓔpʰæ˧β}\newline
\classe{动词}
\sens{1}
\begin{définition}\peng{To tie, to fasten (an animal).}\end{définition}
\begin{définition}\pcmn{拴(牛……)}\end{définition}
\begin{définition}\pfra{Attacher (un animal).}\end{définition}
\begin{exemple}\pnru{tʰi˧-pʰæ˧+ze˧}\hspace{5pt}\peng{|fg{dur} \_ |fg{pfv}}\hspace{5pt}\pcmn{|fg{dur} \_ |fg{pfv}}\hspace{5pt}\pfra{|fg{dur} \_ |fg{pfv}}\end{exemple}
\begin{exemple}\pnru{pʰæ˧∼pʰæ˧}\hspace{5pt}\peng{|fg{red}}\hspace{5pt}\pcmn{重叠}\hspace{5pt}\pfra{|fg{red}}\end{exemple}\sens{2}
\begin{définition}\peng{To be linked, to have ties: e.g. the members of a family have ties.}\end{définition}
\begin{définition}\pcmn{有联系,有缘分,有深的关系}\end{définition}
\begin{définition}\pfra{Être lié, avoir des liens étroits: par exemple, les membres d'une famille ont des liens profonds.}\end{définition}
\begin{exemple}\pnru{pʰæ˧∼pʰæ˧=ɻæ˩ ɲi˩!}\hspace{5pt}\peng{They are united / they make up a couple / they are united into a couple (about two young people)}\hspace{5pt}\pcmn{他们有联系了/他们成了一俩了!(关于两个年轻人)}\hspace{5pt}\pfra{[Ils] sont unis, ils sont en couple! (au sujet de deux jeunes personnes)}\end{exemple}
\end{entrée}

\begin{entrée}
{‑pʰæ˥di˩}{}{ⓔ‑pʰæ˥di˩}\formedesurface{pʰæ˧di˩}\newline
\classe{助词}\ton{H.L}\begin{définition}\peng{Like; as if.}\end{définition}
\begin{définition}\pcmn{像、好像}\end{définition}
\begin{définition}\pfra{Semblable à; comme; comme si; on dirait que.}\end{définition}
\begin{exemple}\pnru{mɤ˧-pʰæ˥di˩}\hspace{5pt}\peng{unlike; for example: seeing a child after several years have elapsed, one finds that (s)he does not look the same as before / is greatly changed}\hspace{5pt}\pcmn{不像(比如:几年没见一个孩子,见的时候,觉得不像以前的样子了)}\hspace{5pt}\pfra{différent, peu ressemblant; par exemple, rencontrant un enfant qui a beaucoup grandi en l'espace de quelques années, on peut faire la réflexion selon laquelle il/elle ne ressemble plus à ce qu'il était auparavant}\end{exemple}
\end{entrée}

\begin{entrée}
{pʰæ˧qʰwɤ˩}{}{ⓔpʰæ˧qʰwɤ˩}\formedesurface{pʰæ˧qʰwɤ˩}\newline
\classe{名词}\ton{L\#}
\paradigme{\pcmn{:} \p{}}
\begin{définition}\peng{Face.}\end{définition}
\begin{définition}\pcmn{脸}\end{définition}
\begin{définition}\pfra{Visage.}\end{définition}
\end{entrée}

\begin{entrée}
{pʰæ˧ʂv̩˧-di˧˥}{}{ⓔpʰæ˧ʂv̩˧-di˧˥}\formedesurface{pʰæ˧ʂv̩˧di˧˥}\newline
\classe{名词}\ton{MH\#}\begin{définition}\peng{Scarf.}\end{définition}
\begin{définition}\pcmn{围巾}\end{définition}
\begin{définition}\pfra{Foulard (périphrase); autrefois, on utilisait un fichu, \stylefv{/qʰwæ}˧ʈɯ˥/.}\end{définition}
\end{entrée}

\begin{entrée}
{pʰæ˧tɕi˥}{}{ⓔpʰæ˧tɕi˥}\newline
\classe{名词}
\sens{1}\paradigme{\pcmn{:} \p{}}
\begin{définition}\peng{Young man, young chap, young lad.}\end{définition}
\begin{définition}\pcmn{小伙子、 青年男子}\end{définition}
\begin{définition}\pfra{Jeune homme, petit gars.}\end{définition}
\begin{exemple}\pnru{pʰæ˧tɕi˥-zo˩}\hspace{5pt}\peng{young man}\hspace{5pt}\pcmn{小伙子}\hspace{5pt}\pfra{jeune homme}\end{exemple}
\begin{exemple}\pnru{pʰæ˧tɕi˥=ɻæ˩}\hspace{5pt}\peng{young men}\hspace{5pt}\pcmn{小伙子们}\hspace{5pt}\pfra{jeunes hommes; les jeunes hommes}\end{exemple}\sens{2}\paradigme{\pcmn{:} \p{}}
\begin{définition}\peng{Name of the first pillar in the main room, the one closest to the door (masculine pillar, the other one being feminine).}\end{définition}
\begin{définition}\pcmn{第一根柱子的名称(代表男人、男性的那根柱子)}\end{définition}
\begin{définition}\pfra{Nom du premier pilier (il y a deux grands piliers dans la maison traditionnelle), celui qui est le plus près de la porte: c'est le pilier masculin, «le jeune homme», le second étant féminin, «la jeune femme».}\end{définition}
\end{entrée}

\begin{entrée}
{pʰæ˧ʈʂʰæ˧lo\#˥}{}{ⓔpʰæ˧ʈʂʰæ˧lo\#˥}\formedesurface{pʰæ˧ʈʂʰæ˧lo˧}\newline
\classe{名词}\ton{\#H}
\paradigme{\pcmn{:} \p{}}
\begin{définition}\peng{Washbasin, basin to wash one's face.}\end{définition}
\begin{définition}\pcmn{脸盆,木盆}\end{définition}
\begin{définition}\pfra{Bassine pour se laver le visage; le même ustensile est utilisé pour servir les nourritures si un bol serait trop petit: pour servir le riz, les soupes…}\end{définition}
\end{entrée}

\begin{entrée}
{pʰe˧}{}{ⓔpʰe˧}\formedesurface{pʰe˧}\newline
\classe{感叹词}\ton{M}\begin{définition}\peng{Interjection: No way! The speaker signals that the interlocutor is making wrong statements, and that (s)he (the speaker) will now put forward different views.}\end{définition}
\begin{définition}\pcmn{呸!(表示唾弃的感叹词)}\end{définition}
\begin{définition}\pfra{Mais pas du tout! Mais non, enfin! Interjection par laquelle le locuteur signale qu'il reprend la main: que ses interlocuteurs lui paraissent être dans l'erreur, et qu'il va rectifier.}\end{définition}
\end{entrée}

\begin{entrée}
{pʰɤ˩˧}{}{ⓔpʰɤ˩˧}\formedesurface{pʰɤ˩˥}\newline
\classe{名词}\ton{LM}
\paradigme{\pcmn{:} \p{}}
\begin{définition}\peng{Jackal, hyena.}\end{définition}
\begin{définition}\pcmn{豺}\end{définition}
\begin{définition}\pfra{Hyène, chacal.}\end{définition}
\begin{exemple}\pnru{pʰɤ˩ hwæ˧-ze˧}\hspace{5pt}\peng{…bought (a/the) jackal}\hspace{5pt}\pcmn{买了豺}\hspace{5pt}\pfra{…a acheté (une) hyène}\end{exemple}
\begin{exemple}\pnru{pʰɤ˩ dzɯ˧-ze˩}\hspace{5pt}\peng{…ate jackal}\hspace{5pt}\pcmn{吃了豺}\hspace{5pt}\pfra{…a mangé (une) hyène}\end{exemple}
\end{entrée}

\begin{entrée}
{pʰɤ˧bɤ˧}{}{ⓔpʰɤ˧bɤ˧}\formedesurface{pʰɤ˧bɤ˧}\newline
\classe{名词}\ton{M}
\paradigme{\pcmn{:} \p{}}
\begin{définition}\peng{Gift, present (typical gifts are tobacco, tea leaf, candies, and wine; one does not usually offer clothes, apart from specific ritual occasions).}\end{définition}
\begin{définition}\pcmn{礼物}\end{définition}
\begin{définition}\pfra{Cadeau (choses à manger ou boire; essentiellement: tabac, thé, bonbons, vin; on n'offre généralement pas de vêtements).}\end{définition}
\begin{exemple}\pnru{pʰɤ˧bɤ˧ po˧-tsʰɯ˧˥}\hspace{5pt}\peng{to bring gifts}\hspace{5pt}\pcmn{带礼物}\hspace{5pt}\pfra{amener des cadeaux}\end{exemple}
\begin{exemple}\pnru{ʈʂʰɯ˧ | ʈæ˧ʂɯ˧ ki˧-hĩ˧ pʰɤ˧bɤ˧ ŋi˩.}\hspace{5pt}\peng{This is a gift from Dashi!}\hspace{5pt}\pcmn{这是达石给的礼物!}\hspace{5pt}\pfra{C'est un cadeau que m'a donné Dashi!}\end{exemple}
\begin{exemple}\pnru{ʈʂʰɯ˧ | ʈæ˧ʂɯ˧ tʰi˧-ki˧-hĩ˧ pʰɤ˧bɤ˧ ŋi˩.}\hspace{5pt}\peng{This is a gift from Dashi! / Here is a gift for you from Dashi!}\hspace{5pt}\pcmn{这是达石送你的礼物!}\hspace{5pt}\pfra{C'est un cadeau que Dashi te fait! Voici un cadeau de la part de Dashi!}\end{exemple}
\end{entrée}

\begin{entrée}
{pʰɤ˧fv̩˩}{}{ⓔpʰɤ˧fv̩˩}\formedesurface{pʰɤ˧fv̩˩}\newline
\classe{名词}\ton{L\#}\begin{définition}\peng{Teapot.}\end{définition}
\begin{définition}\pcmn{茶壶}\end{définition}
\begin{définition}\pfra{Théière.}\end{définition}
\end{entrée}

\begin{entrée}
{pʰɤ˩mi˩}{}{ⓔpʰɤ˩mi˩}\formedesurface{pʰɤ˩mi˩˥}\newline
\classe{名词}\ton{L}
\paradigme{\pcmn{:} \p{}}
\begin{définition}\peng{Female jackal.}\end{définition}
\begin{définition}\pcmn{母豺}\end{définition}
\begin{définition}\pfra{Femelle chacal.}\end{définition}
\end{entrée}

\begin{entrée}
{pʰɤ˧pʰv̩\#˥}{}{ⓔpʰɤ˧pʰv̩\#˥}\formedesurface{pʰɤ˧pʰv̩˧}\newline
\classe{名词}\ton{\#H}
\paradigme{\pcmn{:} \p{}}
\begin{définition}\peng{Male jackal.}\end{définition}
\begin{définition}\pcmn{公豺}\end{définition}
\begin{définition}\pfra{Chacal mâle.}\end{définition}
\end{entrée}

\begin{entrée}
{pʰɤ˩-so˩∼so˥}{}{ⓔpʰɤ˩-so˩∼so˥}\formedesurface{pʰɤ˩so˩so˥}\newline
\classe{形容词}\begin{définition}\peng{Loose (the soil is loose after being forked over).}\end{définition}
\begin{définition}\pcmn{松(土)}\end{définition}
\begin{définition}\pfra{Meuble: la terre est meuble.}\end{définition}
\begin{exemple}\pnru{ʈʂe˧ | pʰɤ˩-so˩∼so˥-gv̩˩}\hspace{5pt}\peng{the soil is loose, the soil has been loosened}\hspace{5pt}\pcmn{土是松的}\hspace{5pt}\pfra{la terre est meuble, la terre a été ameublie}\end{exemple}
\end{entrée}

\begin{entrée}
{pʰɤ˩zo˩}{}{ⓔpʰɤ˩zo˩}\formedesurface{pʰɤ˩zo˩˥}\newline
\classe{名词}\ton{L}
\paradigme{\pcmn{:} \p{}}
\begin{définition}\peng{Baby jackal.}\end{définition}
\begin{définition}\pcmn{豺崽子}\end{définition}
\begin{définition}\pfra{Petit chacal.}\end{définition}
\end{entrée}

\begin{entrée}
{pʰi˧}{}{ⓔpʰi˧}\formedesurface{pʰi˧}\newline
\classe{名词}\ton{M}
\paradigme{\pcmn{:} \p{}}
\begin{définition}\peng{Linen cloth.}\end{définition}
\begin{définition}\pcmn{麻布,亚麻布}\end{définition}
\begin{définition}\pfra{Tissu de lin; anciennement le tissu dont étaient faits tous les vêtements.}\end{définition}
\begin{exemple}\pnru{pʰi˩ dɑ˩˥}\hspace{5pt}\peng{to weave linen}\hspace{5pt}\pcmn{织麻布}\hspace{5pt}\pfra{tisser le lin, faire du tissu de lin}\end{exemple}
\end{entrée}

\begin{entrée}
{pʰi˧˥}{}{ⓔpʰi˧˥}\formedesurface{pʰi˧˥}\newline
\classe{动词}\ton{MH}\begin{définition}\peng{To vomit.}\end{définition}
\begin{définition}\pcmn{呕吐}\end{définition}
\begin{définition}\pfra{Vomir.}\end{définition}
\begin{exemple}\pnru{le˧-pʰi˧-ze˥}\hspace{5pt}\peng{|fg{accomp} \_ |fg{pfv}}\hspace{5pt}\pcmn{呕吐了}\hspace{5pt}\pfra{|fg{accomp} \_ |fg{pfv}}\end{exemple}
\end{entrée}

\begin{entrée}
{pʰi˧β}{}{ⓔpʰi˧β}\formedesurface{pʰi˧}\newline
\classe{动词}\ton{Mβ}\begin{définition}\peng{To winnow with a fan.}\end{définition}
\begin{définition}\pcmn{簸}\end{définition}
\begin{définition}\pfra{Vanner à l'aide d'un crible (vannerie ronde): on fait «sauter» le grain dans un crible, et le vent emporte la balle.}\end{définition}
\begin{exemple}\pnru{hɑ˧ pʰi˧}\hspace{5pt}\peng{to winnow cereals}\hspace{5pt}\pcmn{簸粮食}\hspace{5pt}\pfra{vanner du grain}\end{exemple}
\begin{exemple}\pnru{pʰi˧∼pʰi˧}\hspace{5pt}\peng{|fg{red}}\hspace{5pt}\pcmn{重叠:簸一簸}\hspace{5pt}\pfra{|fg{red}}\end{exemple}
\begin{exemple}\pnru{le˧-pʰi˧(-ze˧)}\hspace{5pt}\peng{|fg{accomp} \_ (|fg{pfv})}\hspace{5pt}\pcmn{簸了}\hspace{5pt}\pfra{|fg{accomp} \_ (|fg{pfv})}\end{exemple}
\end{entrée}

\begin{entrée}
{pʰi˩}{}{ⓔpʰi˩}\formedesurface{pʰi˩˥}\newline
\classe{形容词}\ton{L}\begin{définition}\peng{Flat.}\end{définition}
\begin{définition}\pcmn{平(汉语借词)}\end{définition}
\begin{définition}\pfra{Plat.}\end{définition}
\end{entrée}

\begin{entrée}
{pʰi˩hæ˩}{}{ⓔpʰi˩hæ˩}\formedesurface{pʰi˩hæ˩˥}\newline
\classe{名词}\ton{L}
\paradigme{\pcmn{:} \p{}}
\begin{définition}\peng{Sandal.}\end{définition}
\begin{définition}\pcmn{凉鞋。汉语借词:第一个音节:皮,第二个音节:不明确,同|fv{/tɕæ˧hæ˩/}。}\end{définition}
\begin{définition}\pfra{Sandale.}\end{définition}
\end{entrée}

\begin{entrée}
{pʰi˩ko˧}{}{ⓔpʰi˩ko˧}\formedesurface{pʰi˩ko˥}\newline
\classe{名词}\ton{LM}
\paradigme{\pcmn{:} \p{}}
\begin{définition}\peng{Apple.}\end{définition}
\begin{définition}\pcmn{苹果}\end{définition}
\begin{définition}\pfra{Pomme.}\end{définition}
\end{entrée}

\begin{entrée}
{pʰi˧kʰv̩˧}{}{ⓔpʰi˧kʰv̩˧}\formedesurface{pʰi˧kʰv̩˧}\newline
\classe{名词}\ton{M}
\paradigme{\pcmn{:} \p{}}
\begin{définition}\peng{Dustpan, wicker scoop, dirt-scooping implement.}\end{définition}
\begin{définition}\pcmn{畚箕}\end{définition}
\begin{définition}\pfra{Pelle à poussière.}\end{définition}
\end{entrée}

\begin{entrée}
{pʰi˧kʰv̩˧}{}{ⓔpʰi˧kʰv̩˧}\formedesurface{pʰi˧kʰv̩˧}\newline
\classe{名词}\ton{M}\begin{définition}\peng{Clamshell.}\end{définition}
\begin{définition}\pcmn{贝壳}\end{définition}
\begin{définition}\pfra{Coquillage.}\end{définition}
\end{entrée}

\begin{entrée}
{pʰi˧li˩}{}{ⓔpʰi˧li˩}\formedesurface{pʰi˧li˩}\newline
\classe{名词}\ton{L\#}
\paradigme{\pcmn{:} \p{}}
\begin{définition}\peng{Butterfly.}\end{définition}
\begin{définition}\pcmn{蝴蝶}\end{définition}
\begin{définition}\pfra{Papillon.}\end{définition}
\end{entrée}

\begin{entrée}
{pʰi˧mo˩}{₁}{ⓔpʰi˧mo˩ⓗ1}\formedesurface{pʰi˧mo˩}\newline
\classe{名词}\ton{L\#}
1
\paradigme{\pcmn{:} \p{}}
\begin{définition}\peng{Winnowing fan.}\end{définition}
\begin{définition}\pcmn{簸箕(用来簸粮食等)}\end{définition}
\begin{définition}\pfra{Vanneuse.}\end{définition}
\end{entrée}

\begin{entrée}
{pʰi˧mo˩}{₂}{ⓔpʰi˧mo˩ⓗ2}\formedesurface{pʰi˧mo˩}\newline
\classe{名词}\ton{L\#}
2\begin{définition}\peng{Snare to catch birds.}\end{définition}
\begin{définition}\pcmn{抓鸟的圈套}\end{définition}
\begin{définition}\pfra{Piège pour attraper des oiseaux.}\end{définition}
\begin{exemple}\pnru{v̩˩dze˩ qo˥-di˩, | pʰi˧mo˩!}\hspace{5pt}\peng{The thing to catch birds is called “snare"!}\hspace{5pt}\pcmn{抓鸟的东西,(叫做)圈套!}\hspace{5pt}\pfra{Ce dont on se sert pour attraper les oiseaux, on appelle ça ‘piège pour oiseaux’!}\end{exemple}
\end{entrée}

\begin{entrée}
{pʰi˩tʰɑ˩}{}{ⓔpʰi˩tʰɑ˩}\formedesurface{pʰi˩tʰɑ˩˥}\newline
\classe{名词}\ton{L}
\paradigme{\pcmn{:} \p{}}
\begin{définition}\peng{Tinder, touchwood.}\end{définition}
\begin{définition}\pcmn{火草}\end{définition}
\begin{définition}\pfra{Amadou.}\end{définition}
\end{entrée}

\begin{entrée}
{pʰi˧tsʰo\#˥}{}{ⓔpʰi˧tsʰo\#˥}\formedesurface{pʰi˧tsʰo˧}\newline
\classe{名词}\ton{\#H}\begin{définition}\peng{Masculine given name.}\end{définition}
\begin{définition}\pcmn{男性名字}\end{définition}
\begin{définition}\pfra{Prénom masculin.}\end{définition}
\end{entrée}

\begin{entrée}
{pʰi˧tsʰo˧-ɖɯ˩ɖʐɯ˩}{}{ⓔpʰi˧tsʰo˧-ɖɯ˩ɖʐɯ˩}\formedesurface{pʰi˧tsʰo˧ɖɯ˩ɖʐɯ˩}\newline
\classe{名词}\ton{-L}\begin{définition}\peng{Masculine given name.}\end{définition}
\begin{définition}\pcmn{男性名字}\end{définition}
\begin{définition}\pfra{Prénom masculin.}\end{définition}
\end{entrée}

\begin{entrée}
{pʰi˧ʈʂæ˧}{}{ⓔpʰi˧ʈʂæ˧}\formedesurface{pʰi˧ʈʂæ˧}\newline
\classe{动词}\ton{M}\begin{définition}\peng{To drape oneself in a felt cape, to drape a piece of felt over one's shoulders.}\end{définition}
\begin{définition}\pcmn{披毡(汉语借词)}\end{définition}
\begin{définition}\pfra{Se draper d'un feutre.}\end{définition}
\end{entrée}

\begin{entrée}
{pʰo˥}{}{ⓔpʰo˥}\formedesurface{pʰo˧}\newline
\classe{动词}\ton{H}\begin{définition}\peng{To drape oneself in (a cape, a piece of fabric), without buttoning up.}\end{définition}
\begin{définition}\pcmn{披上(不系扣子)}\end{définition}
\begin{définition}\pfra{Se draper de, endosser, mettre sur son dos. Le fait de porter un vêtement sur les épaules sans le boutonner était considéré comme mal élevé: seuls les voleurs gardent la veste ouverte, pour y fourrer subrepticement leur butin.}\end{définition}
\begin{exemple}\pnru{mɤ˧-pʰo˥}\hspace{5pt}\peng{|fg{neg}}\hspace{5pt}\pcmn{不披}\hspace{5pt}\pfra{|fg{neg}}\end{exemple}
\begin{exemple}\pnru{bɑ˩lɑ˩ qɑ˩-pʰo˩˥}\hspace{5pt}\peng{to put on a shirt without buttoning up}\hspace{5pt}\pcmn{披上衣服(不系扣子)}\hspace{5pt}\pfra{endosser un habit, se mettre un habit sur les épaules (sans boutonner)}\end{exemple}
\end{entrée}

\begin{entrée}
{pʰo˧˥}{}{ⓔpʰo˧˥}\formedesurface{pʰo˧˥}\newline
\classe{动词}\ton{MH}\begin{définition}\peng{To sow.}\end{définition}
\begin{définition}\pcmn{撒 (撒种子)、播(种子)}\end{définition}
\begin{définition}\pfra{Semer à la volée.}\end{définition}
\begin{exemple}\pnru{ɻæ˩ pʰo˧˥}\hspace{5pt}\peng{to sow seeds}\hspace{5pt}\pcmn{撒种子}\hspace{5pt}\pfra{semer des graines à la volée}\end{exemple}
\end{entrée}

\begin{entrée}
{pʰo˧˥α}{}{ⓔpʰo˧˥α}\formedesurface{ɖɯ˧ pʰo˧˥}\newline
\classe{量词}\ton{MHα}\begin{définition}\peng{A member of a pair; also used for some large domestic animals, e.g. oxen.}\end{définition}
\begin{définition}\pcmn{量词:一对中的一只(例如一只鞋),一头大牲畜(牛……)}\end{définition}
\begin{définition}\pfra{Classificateur des membres d'une paire. Par exemple: une chaussure, pas une paire. Ce classificateur est également employé pour le gros bétail: vaches, buffles…}\end{définition}
\end{entrée}

\begin{entrée}
{pʰo˧β}{}{ⓔpʰo˧β}\formedesurface{pʰo˧}\newline
\classe{动词}\ton{Mβ}\begin{définition}\peng{To open (e.g. a door).}\end{définition}
\begin{définition}\pcmn{打开(例如:开门)}\end{définition}
\begin{définition}\pfra{Ouvrir (ex.: porte).}\end{définition}
\begin{exemple}\pnru{gɤ˩-pʰo˧ (-ze˧)}\hspace{5pt}\peng{to open up}\hspace{5pt}\pcmn{打开}\hspace{5pt}\pfra{ouvrir}\end{exemple}
\begin{exemple}\pnru{kʰi˧ pʰo˧}\hspace{5pt}\peng{to open the door}\hspace{5pt}\pcmn{开门}\hspace{5pt}\pfra{ouvrir la porte}\end{exemple}
\begin{exemple}\pnru{kʰi˧mi˧ le˧-pʰo˧}\hspace{5pt}\peng{to open the door}\hspace{5pt}\pcmn{开门}\hspace{5pt}\pfra{ouvrir la porte}\end{exemple}
\begin{exemple}\pnru{tso˧∼tso˧ pʰo˧}\hspace{5pt}\peng{to open something}\hspace{5pt}\pcmn{打开东西}\hspace{5pt}\pfra{ouvrir quelque chose}\end{exemple}
\end{entrée}

\begin{entrée}
{pʰo˩α}{}{ⓔpʰo˩α}\formedesurface{pʰo˩˥}\newline
\classe{动词}\ton{Lα}\begin{définition}\peng{To flee, to rush away, to escape.}\end{définition}
\begin{définition}\pcmn{逃,逃跑,逃掉}\end{définition}
\begin{définition}\pfra{S'échapper, s'enfuir; détaler.}\end{définition}
\begin{exemple}\pnru{le˧-pʰo˩-ze˩}\hspace{5pt}\peng{|fg{accomp} \_ |fg{pfv}}\hspace{5pt}\pcmn{逃跑了}\hspace{5pt}\pfra{|fg{accomp} \_ |fg{pfv}}\end{exemple}
\begin{exemple}\pnru{le˧-pʰo˩-hɯ˩-ze˩!}\hspace{5pt}\peng{(She/he) has escaped!}\hspace{5pt}\pcmn{(他)逃跑了!}\hspace{5pt}\pfra{(Elle/il) s'est enfui(e)!}\end{exemple}
\end{entrée}

\begin{entrée}
{pʰo˩lɑ˧˥}{}{ⓔpʰo˩lɑ˧˥}\formedesurface{pʰo˩lɑ˧˥}\newline
\classe{动词}\ton{LM+MH\#}\begin{définition}\peng{To wage war.}\end{définition}
\begin{définition}\pcmn{战争、打仗}\end{définition}
\begin{définition}\pfra{Faire la guerre.}\end{définition}
\begin{exemple}\pnru{mɤ˧-pʰo˩lɑ˩}\hspace{5pt}\peng{|fg{neg}}\hspace{5pt}\pcmn{不打仗}\hspace{5pt}\pfra{|fg{neg}}\end{exemple}
\begin{exemple}\pnru{pʰo˩lɑ˧˥ | ɖɯ˧-kʰv̩˧˥}\hspace{5pt}\peng{a year of war, a year during which there was war}\hspace{5pt}\pcmn{打仗的一年}\hspace{5pt}\pfra{une année de guerre}\end{exemple}
\end{entrée}

\begin{entrée}
{pʰv̩˧}{₁}{ⓔpʰv̩˧ⓗ1}\formedesurface{pʰv̩˧}\newline
\classe{名词}\ton{M}
1
\paradigme{\pcmn{:} \p{}}
\begin{définition}\peng{Male.}\end{définition}
\begin{définition}\pcmn{公的}\end{définition}
\begin{définition}\pfra{Mâle.}\end{définition}
\begin{exemple}\pnru{ʈʂʰɯ˧, | pʰv̩˧ ɲi˩!}\hspace{5pt}\peng{It's a male!}\hspace{5pt}\pcmn{这(只动物)是公的!}\hspace{5pt}\pfra{C'est un mâle!}\end{exemple}
\begin{exemple}\pnru{ʈʂʰɯ˧, | pʰv̩˧!}\hspace{5pt}\peng{It's a male!}\hspace{5pt}\pcmn{这(只动物)是公的!}\hspace{5pt}\pfra{C'est un mâle!}\end{exemple}
\end{entrée}

\begin{entrée}
{pʰv̩˧}{₂}{ⓔpʰv̩˧ⓗ2}\formedesurface{pʰv̩˧}\newline
\classe{名词}\ton{M}
2\begin{définition}\peng{Price.}\end{définition}
\begin{définition}\pcmn{价格}\end{définition}
\begin{définition}\pfra{Prix.}\end{définition}
\end{entrée}

\begin{entrée}
{pʰv̩˧˥}{}{ⓔpʰv̩˧˥}\formedesurface{pʰv̩˧˥}\newline
\classe{动词}\ton{MH}\begin{définition}\peng{To take off (clothes).}\end{définition}
\begin{définition}\pcmn{脱(衣服)}\end{définition}
\begin{définition}\pfra{Ôter, retirer (un vêtement).}\end{définition}
\end{entrée}

\begin{entrée}
{pʰv̩˧˥}{₁}{ⓔpʰv̩˧˥ⓗ1}\formedesurface{pʰv̩˧˥}\newline
\classe{动词}\ton{MH}
1\begin{définition}\peng{To boil, to cook in water.}\end{définition}
\begin{définition}\pcmn{煮(鸡蛋、洋芋……)}\end{définition}
\begin{définition}\pfra{Faire bouillir, faire cuire à l'eau (œuf, patates…).}\end{définition}
\begin{exemple}\pnru{jɤ˩jo˥ F | pʰv̩˧˥! | æ˩ʁv̩˩˥ F | pʰv̩˧˥!}\hspace{5pt}\peng{Potatoes can be boiled! Eggs can be boiled!}\hspace{5pt}\pcmn{洋芋,是(可以)煮的!鸡蛋,是(可以)煮的!}\hspace{5pt}\pfra{Les pommes de terre, ça se cuit à l'eau! Les œufs, ça se cuit à l'eau!}\end{exemple}
\begin{exemple}\pnru{æ˩ʁv̩˩ pʰv̩˥}\hspace{5pt}\peng{to cook eggs in water}\hspace{5pt}\pcmn{煮鸡蛋}\hspace{5pt}\pfra{cuire des œufs à l'eau, faire des oeufs durs}\end{exemple}
\begin{exemple}\pnru{jɤ˩jo˥ pʰv̩˩}\hspace{5pt}\peng{to boil potatoes}\hspace{5pt}\pcmn{煮洋芋}\hspace{5pt}\pfra{cuire des pommes de terre à l'eau}\end{exemple}
\begin{exemple}\pnru{le˧-pʰv̩˧ | le˧-mv̩˩-ze˩!}\hspace{5pt}\peng{It is cooked (from boiling)! / It has been boiled to the point when it is now well-done/cooked}\hspace{5pt}\pcmn{煮熟了!}\hspace{5pt}\pfra{C'est cuit (à l'eau)! Résultatif: ça a été suffisamment bouilli pour que ce soit maintenant cuit}\end{exemple}
\end{entrée}

\begin{entrée}
{pʰv̩˧˥}{₂}{ⓔpʰv̩˧˥ⓗ2}\formedesurface{pʰv̩˧˥}\newline
\classe{动词}\ton{MH}
2\begin{définition}\peng{To pour, to spill.}\end{définition}
\begin{définition}\pcmn{倒(酒……),倒出来}\end{définition}
\begin{définition}\pfra{Verser; renverser; répandre; jeter.}\end{définition}
\begin{exemple}\pnru{ʐɯ˧ pʰv̩˧˥}\hspace{5pt}\peng{to pour wine, to serve wine}\hspace{5pt}\pcmn{倒酒}\hspace{5pt}\pfra{verser du vin, servir du vin}\end{exemple}
\begin{exemple}\pnru{dʑɯ˩ pʰv̩˩˥}\hspace{5pt}\peng{to pour water, to serve water (as a beverage)}\hspace{5pt}\pcmn{倒水}\hspace{5pt}\pfra{verser de l'eau}\end{exemple}
\begin{exemple}\pnru{mv̩˩tɕo˧ pʰv̩˧˥}\hspace{5pt}\peng{to pour out, to spill on the floor}\hspace{5pt}\pcmn{往外倒}\hspace{5pt}\pfra{renverser, verser à terre, jeter à terre}\end{exemple}
\begin{exemple}\pnru{ɖæ˩˥ | mv̩˩tɕo˧ pʰv̩˥}\hspace{5pt}\peng{to throw out garbage, to pour garbage (out of a bucket onto a dirt heap)}\hspace{5pt}\pcmn{倒垃圾}\hspace{5pt}\pfra{jeter des ordures}\end{exemple}
\end{entrée}

\begin{entrée}
{pʰv̩˧˥}{₃}{ⓔpʰv̩˧˥ⓗ3}\formedesurface{pʰv̩˧˥}\newline
\classe{动词}\ton{MH}
3\begin{définition}\peng{To turn over (when lying down).}\end{définition}
\begin{définition}\pcmn{翻身、翻来翻去}\end{définition}
\begin{définition}\pfra{Retourner; se retourner (quelqu'un est allongé et se retourne).}\end{définition}
\begin{exemple}\pnru{le˧-wo˧ tsɤ˥-pʰv̩˩ |}\hspace{5pt}\peng{to turn over (when lying down)}\hspace{5pt}\pcmn{翻身}\hspace{5pt}\pfra{se retourner}\end{exemple}
\begin{exemple}\pnru{ɖɯ˧-tɕo˥ tsɤ˩-pʰv̩˩, | ʈʂʰɯ˧-tɕo˥ tsɤ˩-pʰv̩˩}\hspace{5pt}\peng{to turn over this way and that (when lying down: turning over restlessly)}\hspace{5pt}\pcmn{翻来翻去}\hspace{5pt}\pfra{se retourner par-ci, se retourner par-là}\end{exemple}
\end{entrée}

\begin{entrée}
{pʰv̩˩α}{}{ⓔpʰv̩˩α}\formedesurface{pʰv̩˩˥}\newline
\classe{形容词}\ton{Lα}\begin{définition}\peng{White.}\end{définition}
\begin{définition}\pcmn{白色(脸、衣服)}\end{définition}
\begin{définition}\pfra{Blanc (visage, habits, cheveux…).}\end{définition}
\begin{exemple}\pnru{pʰv̩˩-hĩ˩˥}\hspace{5pt}\peng{|fg{rel}}\hspace{5pt}\pcmn{白的}\hspace{5pt}\pfra{|fg{rel}}\end{exemple}
\end{entrée}

\begin{entrée}
{pʰv̩˩α}{}{ⓔpʰv̩˩α}\formedesurface{ɖɯ˧ pʰv̩˩}\newline
\classe{量词}\ton{Lβ}\begin{définition}\peng{Classifier for fields.}\end{définition}
\begin{définition}\pcmn{量词:田地(一块)}\end{définition}
\begin{définition}\pfra{Classificateur des parcelles de terre, des champs.}\end{définition}
\begin{exemple}\pnru{lv̩˧ | ɖɯ˧-pʰv̩˩}\hspace{5pt}\peng{a field}\hspace{5pt}\pcmn{一块田}\hspace{5pt}\pfra{un champ; une parcelle}\end{exemple}
\end{entrée}

\begin{entrée}
{pʰv̩˩β}{₁}{ⓔpʰv̩˩βⓗ1}\formedesurface{pʰv̩˩˥}\newline
\classe{动词}\ton{Lβ}
1\begin{définition}\peng{To move around.}\end{définition}
\begin{définition}\pcmn{摇动、翻滚}\end{définition}
\begin{définition}\pfra{S'agiter.}\end{définition}
\begin{exemple}\pnru{bo˩˥ | tʰi˧-pʰv̩˩-dʑo˩}\hspace{5pt}\peng{The pig is moving around (restlessly).}\hspace{5pt}\pcmn{猪在翻滚}\hspace{5pt}\pfra{le cochon s'agite dans son box}\end{exemple}
\begin{exemple}\pnru{bo˩-ɳɯ˧ | pʰv̩˧∼pʰv̩˩}\hspace{5pt}\peng{same meaning as above}\hspace{5pt}\pcmn{猪在翻滚}\hspace{5pt}\pfra{même sens}\end{exemple}
\end{entrée}

\begin{entrée}
{pʰv̩˩β}{₂}{ⓔpʰv̩˩βⓗ2}\formedesurface{pʰv̩˩˥}\newline
\classe{动词}\ton{Lβ}
2\begin{définition}\peng{To expand, to spread, to extend.}\end{définition}
\begin{définition}\pcmn{扩散、发展}\end{définition}
\begin{définition}\pfra{Connaître une expansion, se répandre, s'étendre.}\end{définition}
\begin{exemple}\pnru{zo˧mv̩˥ | tʰi˧-pʰv̩˩}\hspace{5pt}\peng{the children spread into new territory; the family spreads, expands into new areas}\hspace{5pt}\pcmn{孩子们扩散(到新的地方)}\hspace{5pt}\pfra{les enfants se répandent, la famille essaime}\end{exemple}
\end{entrée}

\begin{entrée}
{pʰv̩˧dʑo\#˥}{}{ⓔpʰv̩˧dʑo\#˥}\formedesurface{pʰv̩˧dʑo˧}\newline
\classe{名词}\ton{\#H}\begin{définition}\peng{The village of Labai.}\end{définition}
\begin{définition}\pcmn{拉伯}\end{définition}
\begin{définition}\pfra{Village de Labai.}\end{définition}
\begin{exemple}\pnru{pʰv̩˧dʑo˧ dzi˧˥}\hspace{5pt}\peng{to live in Labai, to dwell in Labai}\hspace{5pt}\pcmn{在拉柏住}\hspace{5pt}\pfra{habiter à Labai}\end{exemple}
\end{entrée}

\begin{entrée}
{pʰv̩˧dʑo˧-hĩ\#˥}{}{ⓔpʰv̩˧dʑo˧-hĩ\#˥}\formedesurface{pʰv̩˧dʑo˧hĩ˧}\newline
\classe{名词}\ton{\#H}\begin{définition}\peng{Inhabitant of Labai, person from Labai.}\end{définition}
\begin{définition}\pcmn{拉伯的人}\end{définition}
\begin{définition}\pfra{Personne de Labai, gens de Labai.}\end{définition}
\end{entrée}

\begin{entrée}
{pʰv̩˧ɖɯ˧˥}{}{ⓔpʰv̩˧ɖɯ˧˥}\formedesurface{pʰv̩˧ɖɯ˧˥}\newline
\classe{形容词}\ton{MH\#}
\étymologie{
pʰv̩˧ 2; ɖɯ˩a
}\begin{définition}\peng{Expensive.}\end{définition}
\begin{définition}\pcmn{贵}\end{définition}
\begin{définition}\pfra{Coûteux, cher.}\end{définition}
\begin{exemple}\pnru{pʰv̩˧ɖɯ˧ ʝi˥}\hspace{5pt}\peng{to care for, to give great attention to}\hspace{5pt}\pcmn{关心}\hspace{5pt}\pfra{prêter attention à}\end{exemple}
\end{entrée}

\begin{entrée}
{pʰv̩˧kɤ˧}{}{ⓔpʰv̩˧kɤ˧}\formedesurface{pʰv̩˧kɤ˧}\newline
\classe{名词}\ton{M}
\paradigme{\pcmn{:} \p{}}
\begin{définition}\peng{Blanket.}\end{définition}
\begin{définition}\pcmn{被子}\end{définition}
\begin{définition}\pfra{Couverture.}\end{définition}
\end{entrée}

\begin{entrée}
{pʰv̩˧ɭɯ\#˥}{}{ⓔpʰv̩˧ɭɯ\#˥}\formedesurface{pʰv̩˧ɭɯ˧}\newline
\classe{名词}\ton{\#H}
\paradigme{\pcmn{:} \p{}}
\begin{définition}\peng{Tibetan wool fabric.}\end{définition}
\begin{définition}\pcmn{氆氇}\end{définition}
\begin{définition}\pfra{Tissu de laine tibétain.}\end{définition}
\end{entrée}

\begin{entrée}
{pʰv̩˧ɭɯ˧-ʈʰæ˧qʰwɤ˥}{}{ⓔpʰv̩˧ɭɯ˧-ʈʰæ˧qʰwɤ˥}\formedesurface{pʰv̩˧ɭɯ˧ʈʰæ˧qʰwɤ˥}\newline
\classe{名词}\ton{H\#}
\paradigme{\pcmn{:} \p{}}
\begin{définition}\peng{Woolen skirt. (Not in common use in Yongning.).}\end{définition}
\begin{définition}\pcmn{羊毛裙子}\end{définition}
\begin{définition}\pfra{Jupe de laine. (Ce n'est pas un vêtement courant à Yongning.).}\end{définition}
\end{entrée}

\begin{entrée}
{pʰv̩˧ʂɯ˧}{}{ⓔpʰv̩˧ʂɯ˧}\formedesurface{pʰv̩˧ʂɯ˧}\newline
\classe{名词}\ton{M}\begin{définition}\peng{Beauty cream; now used to refer to sunscreen (sunblock).}\end{définition}
\begin{définition}\pcmn{美容膏。也来指防晒霜}\end{définition}
\begin{définition}\pfra{Crème de beauté; s'emploie aussi pour la crème solaire.}\end{définition}
\begin{exemple}\pnru{pʰv˧ʂɯ˧ jɤ˧˥}\hspace{5pt}\peng{to put on beauty cream; to apply sunscreen}\hspace{5pt}\pcmn{抹美容膏,抹防晒霜}\hspace{5pt}\pfra{étaler de la crème solaire, mettre de la crème solaire}\end{exemple}
\begin{exemple}\pnru{pʰv˧ʂɯ˧ lɑ˧˥}\hspace{5pt}\peng{to put on beauty cream; to apply sunscreen}\hspace{5pt}\pcmn{抹美容膏,抹防晒霜}\hspace{5pt}\pfra{étaler de la crème solaire, mettre de la crème solaire}\end{exemple}
\end{entrée}

\begin{entrée}
{pʰv̩˧tv̩˥}{}{ⓔpʰv̩˧tv̩˥}\formedesurface{pʰv̩˧tv̩˥}\newline
\classe{名词}\ton{H\#}
\paradigme{\pcmn{:} \p{}}
\begin{définition}\peng{Male water buffalo.}\end{définition}
\begin{définition}\pcmn{公水牛}\end{définition}
\begin{définition}\pfra{Buffle mâle.}\end{définition}
\begin{exemple}\pnru{dʑi˧mi˧-pʰv̩˩tv̩˩}\hspace{5pt}\peng{same meaning: male water buffalo}\hspace{5pt}\pcmn{同上:公水牛}\hspace{5pt}\pfra{même sens: buffle mâle}\end{exemple}
\begin{exemple}\pnru{dʑi˧mi˧ ʈʂʰɯ˧-pʰo˩ dʑo˩, | pʰv̩˧tv̩˥ ɲi˩!}\hspace{5pt}\peng{This buffalo is a male!}\hspace{5pt}\pcmn{这头水牛是公的/是公水牛!}\hspace{5pt}\pfra{ce buffle, c'est un mâle!}\end{exemple}
\end{entrée}

\begin{entrée}
{pʰv̩˩-tɕæ˩ɻæ˥}{}{ⓔpʰv̩˩-tɕæ˩ɻæ˥}\formedesurface{pʰv̩˩tɕæ˩ɻæ˥}\newline
\classe{形容词}\ton{L+H\#}
\étymologie{
pʰv̩˩a
}\begin{définition}\peng{Very white.}\end{définition}
\begin{définition}\pcmn{很白(脸、衣服、头发)}\end{définition}
\begin{définition}\pfra{Blanc (visage, habits, cheveux…).}\end{définition}
\begin{exemple}\pnru{pʰv̩˩tɕæ˩ɻæ˥-gv̩˩}\hspace{5pt}\peng{very white}\hspace{5pt}\pcmn{很白}\hspace{5pt}\pfra{tout blanc}\end{exemple}
\begin{exemple}\pnru{pʰv̩˩↑tɕæ˥ɻæ˩-gv̩˩}\hspace{5pt}\peng{very white}\hspace{5pt}\pcmn{很白}\hspace{5pt}\pfra{tout blanc}\end{exemple}
\begin{exemple}\pnru{pʰæ˧qʰwɤ˩ | pʰv̩˩tɕæ˩ɻæ˥-gv̩˩}\hspace{5pt}\peng{the face is very white}\hspace{5pt}\pcmn{脸很白}\hspace{5pt}\pfra{le visage est très blanc}\end{exemple}
\end{entrée}

\begin{entrée}
{pʰv̩˥ʈʂʰe˩}{}{ⓔpʰv̩˥ʈʂʰe˩}\formedesurface{pʰv̩˧ʈʂʰe˩}\newline
\classe{动词}\ton{HL}\begin{définition}\peng{To distinguish.}\end{définition}
\begin{définition}\pcmn{分开、区分、区别开来}\end{définition}
\begin{définition}\pfra{Distinguer, voir la différence (par exemple entre diverses espèces de champignons).}\end{définition}
\begin{exemple}\pnru{le˧-pʰv̩˥ʈʂʰe˩}\hspace{5pt}\peng{to distinguish, to tell apart (e.g. species of mushrooms)}\hspace{5pt}\pcmn{分开、区分、区别开来}\hspace{5pt}\pfra{distinguer, voir la différence (par exemple entre diverses espèces de champignons)}\end{exemple}
\begin{exemple}\pnru{ɖɯ˧-pʰv̩˥ʈʂʰe˩=ɻ̍˩}\hspace{5pt}\peng{|fg{delimitative} \_ |fg{inceptive}}\hspace{5pt}\pcmn{试着区分}\hspace{5pt}\pfra{|fg{délimitatif} \_ |fg{inchoatif}}\end{exemple}
\begin{exemple}\pnru{mɤ˧-pʰv̩˥ʈʂʰe˩}\hspace{5pt}\peng{not to distinguish, not to make any difference (e.g. between different species of mushrooms)}\hspace{5pt}\pcmn{不分开,不分,不区分}\hspace{5pt}\pfra{ne pas distinguer, ne pas faire de différence, ne pas voir la différence (par exemple entre diverses espèces de champignons)}\end{exemple}
\end{entrée}

\begin{entrée}
{pʰv̩˧ʐo˧˥}{}{ⓔpʰv̩˧ʐo˧˥}\formedesurface{pʰv̩˧ʐo˧˥}\newline
\classe{形容词}\ton{MH\#}
\étymologie{
pʰv̩˧ 2; ʐo˩a
}\begin{définition}\peng{Cheap.}\end{définition}
\begin{définition}\pcmn{便宜}\end{définition}
\begin{définition}\pfra{Bon marché.}\end{définition}
\end{entrée}

\newpage\caractère{q}

\begin{entrée}
{qɑ˧˥}{}{ⓔqɑ˧˥}\formedesurface{qɑ˧˥}\newline
\classe{动词}\ton{MH}\begin{définition}\peng{To help.}\end{définition}
\begin{définition}\pcmn{帮助}\end{définition}
\begin{définition}\pfra{Aider.}\end{définition}
\begin{exemple}\pnru{tʰi˧-qɑ˧˥}\hspace{5pt}\peng{|fg{dur}}\hspace{5pt}\pcmn{|fg{dur}}\hspace{5pt}\pfra{|fg{dur}}\end{exemple}
\begin{exemple}\pnru{qɑ˩∼qɑ˧˥}\hspace{5pt}\peng{|fg{red}}\hspace{5pt}\pcmn{重叠:帮帮忙}\hspace{5pt}\pfra{|fg{red}}\end{exemple}
\begin{exemple}\pnru{hĩ˧ qɑ˩∼qɑ˩}\hspace{5pt}\peng{to help people; to go and work at someone else's place (e.g. during the harvest)}\hspace{5pt}\pcmn{帮人,到别人家去工作(例如收庄稼的时候)}\hspace{5pt}\pfra{aider des gens; aller travailler chez autrui (par ex. pendant les récoltes)}\end{exemple}
\begin{exemple}\pnru{njɤ˧ no˧ qɑ˧∼qɑ˥}\hspace{5pt}\peng{I help you}\hspace{5pt}\pcmn{我帮你}\hspace{5pt}\pfra{je t'aide}\end{exemple}
\end{entrée}

\begin{entrée}
{‑qɑ˧˥}{₁}{ⓔ‑qɑ˧˥ⓗ1}\formedesurface{qɑ˧˥}\newline
\classe{}\ton{MH}
1\begin{définition}\peng{Dative (to); comitative (with).}\end{définition}
\begin{définition}\pcmn{给、对}\end{définition}
\begin{définition}\pfra{À (datif); avec (comitatif).}\end{définition}
\begin{exemple}\pnru{no˧-qɑ˧ | tʰi˧-qɑ˧˥}\hspace{5pt}\peng{to help you}\hspace{5pt}\pcmn{帮助你,给你帮助}\hspace{5pt}\pfra{t'aider, t'apporter de l'aide}\end{exemple}
\begin{exemple}\pnru{no˧-qɑ˧ bi˧}\hspace{5pt}\peng{to go with you}\hspace{5pt}\pcmn{跟你去}\hspace{5pt}\pfra{aller avec toi}\end{exemple}
\begin{exemple}\pnru{no-qɑ˧ tɕʰo˧˥}\hspace{5pt}\peng{to accompany you}\hspace{5pt}\pcmn{跟你,跟你一起}\hspace{5pt}\pfra{t'accompagner}\end{exemple}
\end{entrée}

\begin{entrée}
{‑qɑ˧˥}{₂}{ⓔ‑qɑ˧˥ⓗ2}\formedesurface{qɑ˧˥}\newline
\classe{}\ton{?}
2\begin{définition}\peng{topic marker: as for... This morpheme is systematically accompanied by intonational focalization of the preceding word.}\end{définition}
\begin{définition}\pcmn{关于……}\end{définition}
\begin{définition}\pfra{marqueur de topicalisation: en ce qui concerne... Ce marqueur est systématiquement accompagné d'une focalisation intonative du mot qui précède.}\end{définition}
\begin{exemple}\pnru{le˧-nv̩˥ F -qɑ˩}\hspace{5pt}\peng{In fact, he knew (all about it)!}\hspace{5pt}\pcmn{因为他好像知道(那件事)的!}\hspace{5pt}\pfra{En fait, il était au courant de tout!}\end{exemple}
\begin{exemple}\pnru{le˧-ʈʂæ˧˥ F -qɑ˩}\hspace{5pt}\peng{As for the installation / concerning the process of installing...}\hspace{5pt}\pcmn{关于安装,那件事……}\hspace{5pt}\pfra{pour ce qui est de l'installation...}\end{exemple}
\begin{exemple}\pnru{le˧-bi˧ F -qɑ˧˥}\hspace{5pt}\peng{as for going...}\hspace{5pt}\pcmn{关于去……}\hspace{5pt}\pfra{pour ce qui est d'aller...}\end{exemple}
\begin{exemple}\pnru{le˧-dzi˩ F -qɑ˩...}\hspace{5pt}\peng{as for sitting...}\hspace{5pt}\pcmn{关于坐……}\hspace{5pt}\pfra{pour ce qui est de s'asseoir...}\end{exemple}
\end{entrée}

\begin{entrée}
{qɑ˩α}{}{ⓔqɑ˩α}\newline
\classe{动词}
\sens{1}
\begin{définition}\peng{To cover (e.g. cover a pot with a lid).}\end{définition}
\begin{définition}\pcmn{盖、覆盖}\end{définition}
\begin{définition}\pfra{Couvrir: par exemple mettre un couvercle, ou couvrir un plat d'une coupelle pour éviter que les mouches ne s'y posent.}\end{définition}
\begin{exemple}\pnru{le˧-qɑ˩-ze˩}\hspace{5pt}\peng{|fg{accomp} \_ |fg{pfv}}\hspace{5pt}\pcmn{盖了}\hspace{5pt}\pfra{|fg{accomp} \_ |fg{pfv}}\end{exemple}
\begin{exemple}\pnru{tʰi˧-qɑ˩-ze˩}\hspace{5pt}\peng{|fg{dur} \_ |fg{pfv}}\hspace{5pt}\pcmn{|fg{dur} \_ |fg{pfv}}\hspace{5pt}\pfra{|fg{dur} \_ |fg{pfv}}\end{exemple}
\begin{exemple}\pnru{ɖɯ˧-kʰwɤ˥ | tʰi˧-qɑ˥}\hspace{5pt}\peng{to cover (a television set) with a piece of fabric (to protect it from dust)}\hspace{5pt}\pcmn{用一块(布料)来盖(电视机,为了防灰)}\hspace{5pt}\pfra{couvrir (un téléviseur) d'un morceau (de tissu) (pour le préserver de la poussière)}\end{exemple}
\begin{exemple}\pnru{hæ̃˧qʰv̩˥ | tʰi˧-qɑ˩!}\hspace{5pt}\peng{At night, we cover (the television set with a piece of fabric)!}\hspace{5pt}\pcmn{晚上,(要)盖上! / 我们晚上盖电视机(为了防灰)!}\hspace{5pt}\pfra{le soir, on recouvre (le téléviseur d'un tissu)! / (on) (le) met le soir (sur le téléviseur)/ (on) (en) recouvre (le téléviseur) le soir!}\end{exemple}
\begin{exemple}\pnru{tso˧∼tso˧ qɑ˥}\hspace{5pt}\peng{to cover things}\hspace{5pt}\pcmn{覆盖东西}\hspace{5pt}\pfra{recouvrir quelque chose}\end{exemple}\sens{2}
\begin{définition}\peng{To hide from view.}\end{définition}
\begin{définition}\pcmn{遮(云遮月)、遮挡}\end{définition}
\begin{définition}\pfra{Voiler, bloquer (la lumière), cacher au regard.}\end{définition}
\end{entrée}

\begin{entrée}
{qɑ˩γ}{}{ⓔqɑ˩γ}\formedesurface{ɖɯ˧ qɑ˩}\newline
\classe{量词}
\sens{1}
\begin{définition}\peng{Classifier for armfuls: of firewood, objects…}\end{définition}
\begin{définition}\pcmn{量词:抱}\end{définition}
\begin{définition}\pfra{Classificateur : une brassée (de bois coupé pour le feu, d'objets…).}\end{définition}
\begin{exemple}\pnru{ʈʂʰɯ˧-qɑ˥}\hspace{5pt}\peng{this armful}\hspace{5pt}\pcmn{这一抱}\hspace{5pt}\pfra{cette brassée}\end{exemple}\sens{2}
\begin{définition}\peng{A large bundle of cut cereals, made of about 10 sheaves. Each sheaf is tied using one stalk, then sheaves are tied together using string. A mule can carry 4 large bundles. Also for: an armful.}\end{définition}
\begin{définition}\pcmn{量词:粮食垛、干草垛}\end{définition}
\begin{définition}\pfra{Classificateur des bottes de céréales coupées, faites d'une dizaine de gerbes. Chaque gerbe est nouée à l'aide d'une tige, puis les gerbes sont liées ensemble avec de la ficelle. Une mule peut porter 4 bottes.}\end{définition}
\begin{exemple}\pnru{dze˧ɭɯ˧ ɖɯ˧-qɑ˩}\hspace{5pt}\peng{a bundle of corn (cut cereals)}\hspace{5pt}\pcmn{一垛小麦(收割时,将十束绑在一起成一垛)}\hspace{5pt}\pfra{une botte de blé (lors de la récolte)}\end{exemple}
\end{entrée}

\begin{entrée}
{qæ˥}{₁}{ⓔqæ˥ⓗ1}\formedesurface{qæ˧}\newline
\classe{动词}\ton{H}
1\begin{définition}\peng{To displace, to move (e.g. earth from one spot to another).}\end{définition}
\begin{définition}\pcmn{搬}\end{définition}
\begin{définition}\pfra{Déplacer, transporter (ex.: transporter de la terre, après avoir creusé).}\end{définition}
\begin{exemple}\pnru{le˧-qæ˥}\hspace{5pt}\peng{|fg{accomp}}\hspace{5pt}\pcmn{|fg{accomp}}\hspace{5pt}\pfra{|fg{accomp}}\end{exemple}
\end{entrée}

\begin{entrée}
{qæ˥}{₂}{ⓔqæ˥ⓗ2}\formedesurface{qæ˧}\newline
\classe{动词}\ton{H}
2\begin{définition}\peng{To change.}\end{définition}
\begin{définition}\pcmn{换}\end{définition}
\begin{définition}\pfra{Changer.}\end{définition}
\begin{exemple}\pnru{le˧-qæ˥-ze˩}\hspace{5pt}\peng{|fg{accomp} \_ |fg{pfv}}\hspace{5pt}\pcmn{换了}\hspace{5pt}\pfra{|fg{accomp} \_ |fg{pfv}}\end{exemple}
\begin{exemple}\pnru{dʑi˧hṽ̩˧ qæ˧}\hspace{5pt}\peng{to change clothes}\hspace{5pt}\pcmn{换衣服}\hspace{5pt}\pfra{changer de vêtements}\end{exemple}
\begin{exemple}\pnru{bɑ˩lɑ˩˥ | tʰi˧-qæ˥}\hspace{5pt}\peng{to change clothes}\hspace{5pt}\pcmn{换衣服}\hspace{5pt}\pfra{changer de vêtements}\end{exemple}
\begin{exemple}\pnru{qæ˧∼qæ˧}\hspace{5pt}\peng{|fg{red}: to exchange (an object for another)}\hspace{5pt}\pcmn{重叠:交换}\hspace{5pt}\pfra{|fg{red}: échanger (un objet contre un autre)}\end{exemple}
\begin{exemple}\pnru{qæ˧∼qæ˧-ɻ̍˥}\hspace{5pt}\peng{|fg{red} |fg{inceptive}}\hspace{5pt}\pcmn{|fg{red} |fg{inceptive}}\hspace{5pt}\pfra{|fg{red} |fg{inchoatif}}\end{exemple}
\begin{exemple}\pnru{qæ˧∼qæ˧-ɻ̍˧-ze˥}\hspace{5pt}\peng{|fg{red} |fg{inceptive} |fg{pfv}}\hspace{5pt}\pcmn{|fg{red} |fg{inceptive} |fg{pfv}}\hspace{5pt}\pfra{|fg{red} |fg{inchoatif} |fg{pfv}}\end{exemple}
\begin{exemple}\pnru{tso˧∼tso˧ qæ˧∼qæ˧}\hspace{5pt}\peng{to exchange things}\hspace{5pt}\pcmn{交换东西}\hspace{5pt}\pfra{échanger des choses}\end{exemple}
\begin{exemple}\pnru{le˧-qæ˧∼qæ˧(-ze˩)}\hspace{5pt}\peng{|fg{accomp} |fg{red} (|fg{pfv})}\hspace{5pt}\pcmn{|fg{accomp} |fg{red} (|fg{pfv})}\hspace{5pt}\pfra{|fg{accomp} |fg{red} (|fg{pfv})}\end{exemple}
\end{entrée}

\begin{entrée}
{qæ˥}{₃}{ⓔqæ˥ⓗ3}\formedesurface{qæ˧}\newline
\classe{动词}\ton{H}
3\begin{définition}\peng{To sculpt.}\end{définition}
\begin{définition}\pcmn{雕}\end{définition}
\begin{définition}\pfra{Sculpter.}\end{définition}
\begin{exemple}\pnru{le˧-qæ˥-ze˩}\hspace{5pt}\peng{|fg{accomp} \_ |fg{pfv}}\hspace{5pt}\pcmn{雕了}\hspace{5pt}\pfra{|fg{accomp} \_ |fg{pfv}}\end{exemple}
\begin{exemple}\pnru{bæ˩bæ˩ qæ˥}\hspace{5pt}\peng{to sculpt a flower}\hspace{5pt}\pcmn{雕花}\hspace{5pt}\pfra{sculpter une fleur}\end{exemple}
\end{entrée}

\begin{entrée}
{qæ˧˥}{₁}{ⓔqæ˧˥ⓗ1}\formedesurface{qæ˧˥}\newline
\classe{动词}\ton{MH}
1\begin{définition}\peng{To burn something, e.g. to cremate a corpse.}\end{définition}
\begin{définition}\pcmn{燃烧,如:烧尸体(进行火葬时)}\end{définition}
\begin{définition}\pfra{Brûler quelque chose; par exemple: incinérer un corps.}\end{définition}
\begin{exemple}\pnru{mv̩˧ qæ˩-ze˩}\hspace{5pt}\peng{the fire has started, the fire is blazing; a fire has caught}\hspace{5pt}\pcmn{火烧着了 / 着火了}\hspace{5pt}\pfra{le feu est parti, ça brûle, ça flambe; un incendie est parti}\end{exemple}
\begin{exemple}\pnru{mv̩˧ le˧-qæ˧˥ / mv̩˧ le˧-qæ˧-ze˥}\hspace{5pt}\peng{the fire is burning; a fire has caught}\hspace{5pt}\pcmn{火在烧 / 着火了}\hspace{5pt}\pfra{ça brûle; il y a un incendie}\end{exemple}
\begin{exemple}\pnru{mv̩˧ qæ˥-ɻ̍˩}\hspace{5pt}\peng{the fire is burning}\hspace{5pt}\pcmn{火在烧 / 火烧着了}\hspace{5pt}\pfra{ça brûle; il y a un incendie}\end{exemple}
\begin{exemple}\pnru{mv̩˧ qæ˥-ɻ̍˩ kʰɯ˩}\hspace{5pt}\peng{to start a fire (as an act of destruction/war), to commit arson}\hspace{5pt}\pcmn{(有人)放火}\hspace{5pt}\pfra{lancer un incendie, déclencher un incendie, mettre le feu (acte criminel)}\end{exemple}
\begin{exemple}\pnru{mv̩˧qæ˥-ɻ̍˩-hɯ˩}\hspace{5pt}\peng{a fire has started}\hspace{5pt}\pcmn{(有人)放火了!}\hspace{5pt}\pfra{un incendie est parti}\end{exemple}
\end{entrée}

\begin{entrée}
{qæ˧˥}{₂}{ⓔqæ˧˥ⓗ2}\formedesurface{qæ˧˥}\newline
\classe{动词}\ton{MH}
2\begin{définition}\peng{To suffer, to have pain.}\end{définition}
\begin{définition}\pcmn{疼}\end{définition}
\begin{définition}\pfra{Souffrir, avoir mal.}\end{définition}
\begin{exemple}\pnru{bi˧mi˧ qæ˧˥}\hspace{5pt}\peng{to have a stomach-ache}\hspace{5pt}\pcmn{肚子疼}\hspace{5pt}\pfra{avoir mal au ventre}\end{exemple}
\begin{exemple}\pnru{ɬo˧kʰv̩˧ qæ˧˥}\hspace{5pt}\peng{the waist hurts, the lower back hurts}\hspace{5pt}\pcmn{腰疼}\hspace{5pt}\pfra{avoir mal à la hanche}\end{exemple}
\begin{exemple}\pnru{ʁo˧qʰwɤ˩ qæ˩}\hspace{5pt}\peng{to have a headache}\hspace{5pt}\pcmn{头疼}\hspace{5pt}\pfra{avoir mal à la tête}\end{exemple}
\end{entrée}

\begin{entrée}
{qæ˩˥}{₁}{ⓔqæ˩˥ⓗ1}\formedesurface{qæ˩˥}\newline
\classe{名词}\ton{LH}
1\begin{définition}\peng{Oil; cooking oil.}\end{définition}
\begin{définition}\pcmn{油,食用油}\end{définition}
\begin{définition}\pfra{Huile (terme générique; huile de friture).}\end{définition}
\end{entrée}

\begin{entrée}
{qæ˩˥}{₂}{ⓔqæ˩˥ⓗ2}\formedesurface{qæ˩˥}\newline
\classe{名词}\ton{LH}
2
\paradigme{\pcmn{:} \p{}}
\begin{définition}\peng{Glue.}\end{définition}
\begin{définition}\pcmn{胶}\end{définition}
\begin{définition}\pfra{Colle.}\end{définition}
\end{entrée}

\begin{entrée}
{qæ˩α}{}{ⓔqæ˩α}\formedesurface{qæ˩˥}\newline
\classe{动词}\ton{Lα}\begin{définition}\peng{To coax (a child).}\end{définition}
\begin{définition}\pcmn{哄(孩子)}\end{définition}
\begin{définition}\pfra{Cajoler un enfant.}\end{définition}
\begin{exemple}\pnru{zo˧ qæ˥}\hspace{5pt}\peng{to coax a child}\hspace{5pt}\pcmn{哄孩子}\hspace{5pt}\pfra{cajoler un enfant}\end{exemple}
\begin{exemple}\pnru{le˧-qæ˧∼qæ˥ | le˧-ʑi˧-kʰɯ˥}\hspace{5pt}\peng{to put asleep by coaxing, to coax asleep}\hspace{5pt}\pcmn{哄睡着}\hspace{5pt}\pfra{endormir (un enfant) en le cajolant}\end{exemple}
\end{entrée}

\begin{entrée}
{qæ˩β}{}{ⓔqæ˩β}\formedesurface{qæ˩˥}\newline
\classe{动词}\ton{Lβ}\begin{définition}\peng{To cheat, to deceive.}\end{définition}
\begin{définition}\pcmn{欺骗}\end{définition}
\begin{définition}\pfra{Tromper.}\end{définition}
\begin{exemple}\pnru{hĩ˧ qæ˥-kv̩˩}\hspace{5pt}\peng{sly, who is good at deceiving people}\hspace{5pt}\pcmn{狡猾、很能骗人的}\hspace{5pt}\pfra{rusé, qui sait tromper son monde}\end{exemple}
\begin{exemple}\pnru{hĩ˧ qæ˥ | ʐwæ˩˥}\hspace{5pt}\peng{sly, who is good at deceiving people}\hspace{5pt}\pcmn{狡猾、很能骗人的}\hspace{5pt}\pfra{qui trompe magistralement son monde}\end{exemple}
\begin{exemple}\pnru{hĩ˧ qæ˥ mɤ˩-ɖo˩!}\hspace{5pt}\peng{One must not cheat others! / One must not deceive people! (A precept taught by the main consultant's grandmother)}\hspace{5pt}\pcmn{不要骗人!(这个信条,是发音合作人的祖母教的)}\hspace{5pt}\pfra{il ne faut pas tromper (autrui)! (précepte inculqué à la locutrice par sa grand-mère)}\end{exemple}
\begin{exemple}\pnru{qæ˩-mɤ˩-ɖo˩˥!}\hspace{5pt}\peng{One must not cheat (others)! / One must not deceive people! (A precept taught by the main consultant's grandmother)}\hspace{5pt}\pcmn{不要骗人!(这个信条,是发音合作人的祖母教的)}\hspace{5pt}\pfra{il ne faut pas tromper (autrui)! (précepte inculqué à la locutrice par sa grand-mère)}\end{exemple}
\begin{exemple}\pnru{mɤ˧-qæ˩}\hspace{5pt}\peng{|fg{neg}}\hspace{5pt}\pcmn{不骗}\hspace{5pt}\pfra{|fg{neg}}\end{exemple}
\begin{exemple}\pnru{hĩ˧ qæ˥-tso˩∼tso˩!}\hspace{5pt}\peng{Shoddy stuff! (Literally: ‘deceitful stuff!') (Context: a comment about thread of poor quality, bought at the market)}\hspace{5pt}\pcmn{骗人的东西!(关于买来的一团线,质量不好)}\hspace{5pt}\pfra{C'est de la camelote! / C'est un truc d'arnaqueurs! (au sujet d'une bobine de fil de mauvaise qualité, achetée dans le commerce)}\end{exemple}
\end{entrée}

\begin{entrée}
{qæ˩di˩}{}{ⓔqæ˩di˩}\formedesurface{qæ˩di˩˥}\newline
\classe{动词}\ton{L}\begin{définition}\peng{To flick, to flip.}\end{définition}
\begin{définition}\pcmn{弹(弹脸)}\end{définition}
\begin{définition}\pfra{Donner une chiquenaude.}\end{définition}
\end{entrée}

\begin{entrée}
{qæ˧do˧}{}{ⓔqæ˧do˧}\formedesurface{qæ˧do˧}\newline
\classe{名词}\ton{M}
\paradigme{\pcmn{:} \p{}}
\begin{définition}\peng{Timber, lumber.}\end{définition}
\begin{définition}\pcmn{木材、木料}\end{définition}
\begin{définition}\pfra{Bois de charpente, tronc coupé.}\end{définition}
\begin{exemple}\pnru{ʑi˧mi˧-qæ˩do˩}\hspace{5pt}\peng{lumber for the construction of the main building of a Na farm}\hspace{5pt}\pcmn{建主房的木材}\hspace{5pt}\pfra{bois de charpente utilisé pour le bâtiment principal}\end{exemple}
\begin{exemple}\pnru{ʑi˧qʰwɤ˧-qæ˧do\#˥}\hspace{5pt}\peng{lumber for the construction of a building}\hspace{5pt}\pcmn{建房子的木材}\hspace{5pt}\pfra{bois de charpente, bois pour la construction d'un bâtiment}\end{exemple}
\end{entrée}

\begin{entrée}
{qæ˧dzɯ˩}{}{ⓔqæ˧dzɯ˩}\formedesurface{qæ˧dzɯ˩}\newline
\classe{名词}\ton{L\#}\begin{définition}\peng{A family name from Yongning. There are two families in Yongning that carry this name.}\end{définition}
\begin{définition}\pcmn{一个姓。这个姓,永宁有两家}\end{définition}
\begin{définition}\pfra{Nom de clan/famille étendue. Deux familles portent ce nom à Yongning.}\end{définition}
\begin{exemple}\pnru{qæ˧dzɯ˩-ɻ̍˩}\hspace{5pt}\peng{the /qæ˧dzɯ˩/ clan, the /qæ˧dzɯ˩/ family}\hspace{5pt}\pcmn{|fv{/qæ˧dzɯ˩/}家族}\hspace{5pt}\pfra{le clan /qæ˧dzɯ˩/, la famille /qæ˧dzɯ˩/}\end{exemple}
\begin{exemple}\pnru{qæ˧dzɯ˩ | -tsʰɯ˧ɻ̍˧}\hspace{5pt}\peng{the name of a person, containing both a family name: /lqæ˧dzɯ˩/, and a given name: /tsʰɯ˧ɻ̍\#˥/}\hspace{5pt}\pcmn{一个人的名字:姓为|fv{/qæ˧dzɯ˩/},名为|fv{/tsʰɯ˧ɻ̍\#˥/}}\hspace{5pt}\pfra{nom d'une personne, comportant un nom de famille (/qæ˧dzɯ˩/) et un prénom (/tsʰɯ˧ɻ̍\#˥/)}\end{exemple}
\end{entrée}

\begin{entrée}
{qæ˧ɻ̍˧}{}{ⓔqæ˧ɻ̍˧}\formedesurface{qæ˧ɻ̍˧}\newline
\classe{名词}\ton{M}
\paradigme{\pcmn{:} \p{}}
\begin{définition}\peng{Timber, lumber.}\end{définition}
\begin{définition}\pcmn{木材、木料}\end{définition}
\begin{définition}\pfra{Bois de charpente, tronc coupé.}\end{définition}
\end{entrée}

\begin{entrée}
{qi˧qi˧}{}{ⓔqi˧qi˧}\formedesurface{qi˧qi˧}\newline
\classe{助词}\ton{M}\begin{définition}\peng{Originally, to begin with.}\end{définition}
\begin{définition}\pcmn{原来、一开始}\end{définition}
\begin{définition}\pfra{À l'origine.}\end{définition}
\end{entrée}

\begin{entrée}
{qo˥}{₁}{ⓔqo˥ⓗ1}\formedesurface{qo˧}\newline
\classe{动词}\ton{H}
1\begin{définition}\peng{To kneel down.}\end{définition}
\begin{définition}\pcmn{跪下}\end{définition}
\begin{définition}\pfra{S'agenouiller (les mains au sol).}\end{définition}
\end{entrée}

\begin{entrée}
{qo˥}{₂}{ⓔqo˥ⓗ2}\formedesurface{qo˧}\newline
\classe{动词}\ton{H}
2\begin{définition}\peng{To love.}\end{définition}
\begin{définition}\pcmn{爱,关心}\end{définition}
\begin{définition}\pfra{Aimer d'amour.}\end{définition}
\begin{exemple}\pnru{mɤ˧-qo˧}\hspace{5pt}\peng{|fg{neg}}\hspace{5pt}\pcmn{不爱}\hspace{5pt}\pfra{|fg{neg}}\end{exemple}
\begin{exemple}\pnru{zo˧mv̩˥zo˩ qo˩}\hspace{5pt}\peng{to love (one's) children}\hspace{5pt}\pcmn{爱孩子}\hspace{5pt}\pfra{aimer (ses) enfants}\end{exemple}
\begin{exemple}\pnru{õ˧-hĩ˥ qo˩}\hspace{5pt}\peng{to love one's family}\hspace{5pt}\pcmn{爱自己家人}\hspace{5pt}\pfra{aimer sa famille}\end{exemple}
\end{entrée}

\begin{entrée}
{‑qo˧}{}{ⓔ‑qo˧}\formedesurface{qo˧}\newline
\classe{}\ton{M}\begin{définition}\peng{In, inside.}\end{définition}
\begin{définition}\pcmn{里}\end{définition}
\begin{définition}\pfra{Dans.}\end{définition}
\end{entrée}

\begin{entrée}
{qo˩α}{}{ⓔqo˩α}\formedesurface{qo˩˥}\newline
\classe{动词}\ton{Lα}\begin{définition}\peng{To put away, to preserve (e.g. to put leftovers in a box so flies won't land on it).}\end{définition}
\begin{définition}\pcmn{放、储存}\end{définition}
\begin{définition}\pfra{Garder, serrer, ranger (de la nourriture dans un récipient à l'abri des mouches).}\end{définition}
\end{entrée}

\begin{entrée}
{qo˩ho˧˥}{}{ⓔqo˩ho˧˥}\formedesurface{qo˩ho˧˥}\newline
\classe{名词}\ton{LM+MH\#}
\paradigme{\pcmn{:} \p{}}
\begin{définition}\peng{Round wicker/bamboo box used to carry gifts.}\end{définition}
\begin{définition}\pcmn{礼盒}\end{définition}
\begin{définition}\pfra{Boîte en vannerie ronde, dans laquelle on place les cadeaux qu’on vient offrir; est formée de deux parties qui s’emboîtent; on la porte lorsqu'on se rend chez quelqu'un dans le cadre d'un événement social important. Cf récit F4. Une photo de cet objet est présente dans le rapport d'enquête de terrain publié en 1986 en 3 volumes (永宁纳西族……调查).}\end{définition}
\end{entrée}

\begin{entrée}
{qo˧lo˩}{}{ⓔqo˧lo˩}\formedesurface{qo˧lo˩}\newline
\classe{助词}\ton{L\#}\begin{définition}\peng{Inside, within.}\end{définition}
\begin{définition}\pcmn{里面}\end{définition}
\begin{définition}\pfra{Dedans, à l'intérieur de, dans.}\end{définition}
\end{entrée}

\begin{entrée}
{‑qo˧lo˩}{}{ⓔ‑qo˧lo˩}\formedesurface{qo˧lo˩}\newline
\classe{}\ton{L\#}\begin{définition}\peng{In.}\end{définition}
\begin{définition}\pcmn{里面}\end{définition}
\begin{définition}\pfra{Dans.}\end{définition}
\begin{exemple}\pnru{ʈʂʰɯ˧ | ɑ˩ʁo˧-qo˧lo˩ dʑo˩}\hspace{5pt}\peng{(S)he is in the house. / (S)he is inside.}\hspace{5pt}\pcmn{他在家里。}\hspace{5pt}\pfra{Il/elle est à la maison/dans la maison.}\end{exemple}
\end{entrée}

\begin{entrée}
{qo˧lo˧ʂv̩˥}{}{ⓔqo˧lo˧ʂv̩˥}\formedesurface{qo˧lo˧ʂv̩˥}\newline
\classe{形容词}\ton{H\#}\begin{définition}\peng{Well-behaved.}\end{définition}
\begin{définition}\pcmn{乖、听话}\end{définition}
\begin{définition}\pfra{Obéissant, sage (enfant).}\end{définition}
\end{entrée}

\begin{entrée}
{qo˧pv̩˧}{}{ⓔqo˧pv̩˧}\formedesurface{qo˧pv̩˧}\newline
\classe{形容词}\ton{M}\begin{définition}\peng{Thirsty.}\end{définition}
\begin{définition}\pcmn{渴}\end{définition}
\begin{définition}\pfra{Assoiffé, ayant soif.}\end{définition}
\begin{exemple}\pnru{qo˧pv̩˧-ze˧}\hspace{5pt}\peng{|fg{pfv}}\hspace{5pt}\pcmn{渴了}\hspace{5pt}\pfra{|fg{pfv}}\end{exemple}
\begin{exemple}\pnru{qo˧pv̩˧ mɤ˧-tʰɑ˧-ze˥}\hspace{5pt}\peng{terribly thirsty}\hspace{5pt}\pcmn{渴得不行}\hspace{5pt}\pfra{avoir soif comme c'est pas possible, avoir terriblement soif}\end{exemple}
\end{entrée}

\begin{entrée}
{qo˧pv̩˩}{}{ⓔqo˧pv̩˩}\formedesurface{qo˧pv̩˩}\newline
\classe{名词}\ton{L\#}
\paradigme{\pcmn{:} \p{}}
\begin{définition}\peng{Cuckoo.}\end{définition}
\begin{définition}\pcmn{布谷鸟}\end{définition}
\begin{définition}\pfra{Coucou.}\end{définition}
\begin{exemple}\pnru{qo˧pv̩˩-ɻwæ˩ | ɖɯ˧-ɲi˥}\hspace{5pt}\peng{Ancestors' Day, Tomb-Sweeping Day, on the first day of the fifth month; literally: ‘the day when the cuckoo sings'}\hspace{5pt}\pcmn{清明节。直译:“布谷鸟叫的那天”}\hspace{5pt}\pfra{Le Jour des Ancêtres, au 1er jour du 5e mois. Littéralement: «le jour où chante le coucou».}\end{exemple}
\end{entrée}

\begin{entrée}
{qo˧pv̩˩-ʐwæ˩ɖʐæ˩}{}{ⓔqo˧pv̩˩-ʐwæ˩ɖʐæ˩}\formedesurface{qo˧pv̩˩ʐwæ˩ɖʐæ˩}\newline
\classe{名词}\ton{L\#-}\begin{définition}\peng{Jay, |\stylefi{Garrulus glandarius sinensis}.}\end{définition}
\begin{définition}\pcmn{松鸦}\end{définition}
\begin{définition}\pfra{Geai, |\stylefi{Garrulus glandarius sinensis}.}\end{définition}
\end{entrée}

\begin{entrée}
{qo˩qɑ˩}{}{ⓔqo˩qɑ˩}\formedesurface{qo˩qɑ˩˥}\newline
\classe{名词}\ton{L}
\paradigme{\pcmn{:} \p{}}
\begin{définition}\peng{Mountain pass.}\end{définition}
\begin{définition}\pcmn{垭口}\end{définition}
\begin{définition}\pfra{Col (de montagne).}\end{définition}
\end{entrée}

\begin{entrée}
{qo˩tɑ˧˥}{}{ⓔqo˩tɑ˧˥}\formedesurface{qo˩tɑ˧˥}\newline
\classe{动词}\ton{L+MH\#}\begin{définition}\peng{To stop, to come to a halt.}\end{définition}
\begin{définition}\pcmn{停下来、停止}\end{définition}
\begin{définition}\pfra{S'interrompre, cesser, s'arrêter}\end{définition}
\begin{exemple}\pnru{le˧-se˥, | ɖɯ˧-di˩ tʰv˩, | qo˩tɑ˧-ze˥!}\hspace{5pt}\peng{We walked; (when) we arrived at a (certain) place, we stopped!}\hspace{5pt}\pcmn{走路到一个地方就停下来了!}\hspace{5pt}\pfra{On a marché, et, arrivé quelque part, on s'est arrêté.}\end{exemple}
\begin{exemple}\pnru{le˧-gwɤ˩, | qo˩ tɑ˧-ze˥.}\hspace{5pt}\peng{(We) had a stroll, (then) we stopped.}\hspace{5pt}\pcmn{散步(一会),后来停下来了。}\hspace{5pt}\pfra{On s'est promené, (et puis) on s'est arrêté.}\end{exemple}
\begin{exemple}\pnru{le˧-so˩, | qo˩ tɑ˧-ze˥.}\hspace{5pt}\peng{(S)he has got seated.}\hspace{5pt}\pcmn{学习了(一会),后来停止了。}\hspace{5pt}\pfra{(On) a étudié/fait nos devoirs un moment, puis on a arrêté / on s'en est tenu là.}\end{exemple}
\end{entrée}

\begin{entrée}
{qo˧tv̩˩}{}{ⓔqo˧tv̩˩}\formedesurface{qo˧tv̩˩}\newline
\classe{名词}\ton{L\#/LM}
\paradigme{\pcmn{:} \p{}}
\begin{définition}\peng{Kernel, fruit stone, pit.}\end{définition}
\begin{définition}\pcmn{果核}\end{définition}
\begin{définition}\pfra{Noyau (aussi pour: graines de tournesol; et pour: bobines de fil).}\end{définition}
\begin{exemple}\pnru{dʑi˧ʁo˩-qo˩tv̩˩}\hspace{5pt}\peng{peach kernel}\hspace{5pt}\pcmn{桃子果核}\hspace{5pt}\pfra{noyau de pêche}\end{exemple}
\end{entrée}

\begin{entrée}
{qo˩tv̩˩-lv̩˥}{}{ⓔqo˩tv̩˩-lv̩˥}\formedesurface{qo˩tv̩˩lv̩˥}\newline
\classe{名词}\ton{L+H\#}
\paradigme{\pcmn{:} \p{}}
\begin{définition}\peng{Ball, lump.}\end{définition}
\begin{définition}\pcmn{团}\end{définition}
\begin{définition}\pfra{Boule.}\end{définition}
\begin{exemple}\pnru{li˩-qo˩tv̩˥-lv̩˩}\hspace{5pt}\peng{tea leaves compressed in bowl shape}\hspace{5pt}\pcmn{沱茶}\hspace{5pt}\pfra{thé comprimé en boule}\end{exemple}
\begin{exemple}\pnru{li˩-qo˩tv̩˥-lv̩˩ | ɖɯ˧-qʰwɤ˧ tɕɤ˥}\hspace{5pt}\peng{to make a bowl of tea, using tea leaves compressed in bowl shape}\hspace{5pt}\pcmn{煮一碗沱茶}\hspace{5pt}\pfra{préparer un bol de thé avec du thé comprimé en boule}\end{exemple}
\end{entrée}

\begin{entrée}
{qv̩˧˥}{}{ⓔqv̩˧˥}\formedesurface{qv̩˧˥}\newline
\classe{动词}\ton{MH}\begin{définition}\peng{To frighten.}\end{définition}
\begin{définition}\pcmn{吓(吓唬)}\end{définition}
\begin{définition}\pfra{Faire peur, effrayer.}\end{définition}
\begin{exemple}\pnru{hĩ˧ qv̩˩}\hspace{5pt}\peng{to frighten people}\hspace{5pt}\pcmn{吓人}\hspace{5pt}\pfra{faire peur aux gens}\end{exemple}
\begin{exemple}\pnru{no˧ | hĩ˧ qv̩˩-zo˩! / ʈʂʰɯ˧-ɳɯ˧ | hĩ˧ qv̩˩-zo˩!}\hspace{5pt}\peng{You frighten people! / He frightens people!}\hspace{5pt}\pcmn{你吓人! / 他吓人!}\hspace{5pt}\pfra{tu fais peur aux gens! Il fait peur aux gens!}\end{exemple}
\begin{exemple}\pnru{ʈʂʰɯ˧ | njæ˩ qv̩˩-tsʰɯ˩˥!}\hspace{5pt}\peng{He frightens people!}\hspace{5pt}\pcmn{他吓人!}\hspace{5pt}\pfra{il fait peur!}\end{exemple}
\begin{exemple}\pnru{njɤ˧ɳɯ˧ | ʈʂʰɯ˧ qv̩˩-bi˩!}\hspace{5pt}\peng{I am going to frighten her/him!}\hspace{5pt}\pcmn{我要吓唬他一下!}\hspace{5pt}\pfra{Je vais lui faire peur!}\end{exemple}
\begin{exemple}\pnru{tʰɑ˧-qv̩˧˥!}\hspace{5pt}\peng{|fg{prohib}}\hspace{5pt}\pcmn{别吓唬(人家)!}\hspace{5pt}\pfra{|fg{prohib}}\end{exemple}
\end{entrée}

\begin{entrée}
{qv̩˩˥}{}{ⓔqv̩˩˥}\formedesurface{qv̩˩˥}\newline
\classe{名词}\ton{LH}
\paradigme{\pcmn{:} \p{}}
\begin{définition}\peng{Handle.}\end{définition}
\begin{définition}\pcmn{把手}\end{définition}
\begin{définition}\pfra{Poignée, manche (d'une valise, d'une bouteille thermos, d'une louche…).}\end{définition}
\end{entrée}

\begin{entrée}
{qv̩˩α}{}{ⓔqv̩˩α}\formedesurface{qv̩˩˥}\newline
\classe{动词}\ton{Lα}\begin{définition}\peng{To wash (something) along (of water); to be carried (by water) (heavy objects, e.g. rocks are carried by a stream; the verb cannot be used for light objects, such as leaves).}\end{définition}
\begin{définition}\pcmn{冲走}\end{définition}
\begin{définition}\pfra{Faire rouler, emporter, charrier (un objet lourd: par exemple, le courant emporte des cailloux, les charriant au loin; le verbe ne peut s'employer pour des objets légers, par exemple des feuilles).}\end{définition}
\begin{exemple}\pnru{le˧-qv̩˩ | le˧-po˧-tsʰɯ˧˥}\hspace{5pt}\peng{to carry to a certain place, to wash along all the way to a certain place}\hspace{5pt}\pcmn{冲到某个地方}\hspace{5pt}\pfra{charrier, amener en faisant rouler: un torrent en crue charrie des cailloux jusque dans la plaine}\end{exemple}
\begin{exemple}\pnru{lv̩˧mi˧ | ɬi˧dʑɯ˩-ɳɯ˩ | qv̩˩˥.}\hspace{5pt}\peng{The stones are carried (down into the plain) by (the strong current of) the river of Yongning.}\hspace{5pt}\pcmn{石头被永宁河水冲(到坝子)}\hspace{5pt}\pfra{les pierres sont amenées par (le courant de) la rivière de Yongning}\end{exemple}
\begin{exemple}\pnru{dʑɯ˧-ɳɯ˧ | le˧-qv̩˩ | le˧-po˧-tsʰɯ˧-hĩ˥ | lv̩˧mi˧}\hspace{5pt}\peng{stones carried over (to this place) by the stream}\hspace{5pt}\pcmn{水流冲下来的石头}\hspace{5pt}\pfra{pierres amenées par la rivière, pierres charriées (jusqu'ici) par la rivière}\end{exemple}
\end{entrée}

\begin{entrée}
{qv̩˧dzi˩}{}{ⓔqv̩˧dzi˩}\formedesurface{qv̩˧dzi˩}\newline
\classe{名词}\ton{L\#}
\paradigme{\pcmn{:} \p{}}
\begin{définition}\peng{|\stylefi{Pinus massoniana D.Don in Lamb.}, Masson's pine, Chinese red pine, horsetail pine. Its seeds are not edible (the fish eat them, but they are poisonous for humans).}\end{définition}
\begin{définition}\pcmn{马尾松}\end{définition}
\begin{définition}\pfra{|\stylefi{Pinus massoniana D.Don in Lamb.}, conifère de la famille des |\stylefi{Pinaceae}. Ses pignes ne sont pas comestibles: les poissons les mangent, mais pour les hommes elles sont vénéneuses.}\end{définition}
\begin{exemple}\pnru{qv̩˧dzi˩-lv̩˩∼lv̩˩, | dzɯ˧ mɤ˧-ɖo˧!}\hspace{5pt}\peng{One must not eat the seeds of Masson's pine! (It is poisonous)}\hspace{5pt}\pcmn{马松树的果子,不要吃!(有毒)}\hspace{5pt}\pfra{Il ne faut pas manger les pignes du pin de Masson! (Elles sont vénéneuses.)}\end{exemple}
\end{entrée}

\begin{entrée}
{qv̩˧ɻ̍\#˥}{}{ⓔqv̩˧ɻ̍\#˥}\formedesurface{qv̩˧ɻ̍˧}\newline
\classe{名词}\ton{\#H}\begin{définition}\peng{Name of a mountain in Yongning.}\end{définition}
\begin{définition}\pcmn{永宁的一座山}\end{définition}
\begin{définition}\pfra{Une montagne de Yongning.}\end{définition}
\begin{exemple}\pnru{qv̩˧ɻ̍˧-ʁo˧-qʰwɤ˥}\hspace{5pt}\peng{the top of the /qv̩˧ɻ̍˧/ mountain}\hspace{5pt}\pcmn{|fv{/qv̩˧ɻ̍˧/}山的山顶}\hspace{5pt}\pfra{le sommet de la montagne /qv̩˧ɻ̍˧/}\end{exemple}
\end{entrée}

\begin{entrée}
{qv̩˧tɕi˥}{}{ⓔqv̩˧tɕi˥}\formedesurface{qv̩˧tɕi˥}\newline
\classe{名词}\ton{H\#}\begin{définition}\peng{Spittle, phlegm, sputum.}\end{définition}
\begin{définition}\pcmn{痰}\end{définition}
\begin{définition}\pfra{Crachat, mucus.}\end{définition}
\end{entrée}

\begin{entrée}
{qv̩˧ʈʂæ˧˥}{}{ⓔqv̩˧ʈʂæ˧˥}\newline
\classe{名词}
\sens{1}\paradigme{\pcmn{:} \p{}}
\begin{définition}\peng{Throat.}\end{définition}
\begin{définition}\pcmn{喉咙}\end{définition}
\begin{définition}\pfra{Gorge.}\end{définition}\sens{2}
\begin{définition}\peng{Voice.}\end{définition}
\begin{définition}\pcmn{声音}\end{définition}
\begin{définition}\pfra{Voix.}\end{définition}
\begin{exemple}\pnru{ʈʂʰɯ˧ | qv̩˧ʈʂæ˧ dʑɤ˥!}\hspace{5pt}\peng{(S)he has a beautiful voice.}\hspace{5pt}\pcmn{他嗓子好。}\hspace{5pt}\pfra{Elle/il a une belle voix.}\end{exemple}
\begin{exemple}\pnru{ʈʂʰɯ˧ | qv̩˧ʈʂæ˧˥ | ɖwæ˧˥ | dʑɤ˩˥!}\hspace{5pt}\peng{(S)he has a really beautiful voice.}\hspace{5pt}\pcmn{他嗓子很好。}\hspace{5pt}\pfra{Elle/il a une très belle voix.}\end{exemple}
\end{entrée}

\begin{entrée}
{qwæ˧}{}{ⓔqwæ˧}\formedesurface{qwæ˧}\newline
\classe{名词}\ton{M}
\paradigme{\pcmn{:} \p{}}
\begin{définition}\peng{Mat, bed mat.}\end{définition}
\begin{définition}\pcmn{床垫子}\end{définition}
\begin{définition}\pfra{Sommier (de lit); banc large.}\end{définition}
\begin{exemple}\pnru{qwæ˧mi\#˥}\hspace{5pt}\peng{large mat}\hspace{5pt}\pcmn{大床垫子}\hspace{5pt}\pfra{grand sommier}\end{exemple}
\end{entrée}

\begin{entrée}
{qwæ˧˥}{₁}{ⓔqwæ˧˥ⓗ1}\newline
\classe{动词}
1
\sens{1}
\begin{définition}\peng{To dig.}\end{définition}
\begin{définition}\pcmn{挖(土)}\end{définition}
\begin{définition}\pfra{Creuser, piocher (dans la terre meuble).}\end{définition}
\begin{exemple}\pnru{tv̩˧qʰv̩˧ qwæ˧˥}\hspace{5pt}\peng{to dig a hole}\hspace{5pt}\pcmn{挖洞}\hspace{5pt}\pfra{creuser un trou}\end{exemple}
\begin{exemple}\pnru{ʈʂe˧ qwæ˩}\hspace{5pt}\peng{to dig out the soil}\hspace{5pt}\pcmn{挖土}\hspace{5pt}\pfra{creuser la terre}\end{exemple}
\begin{exemple}\pnru{qʰæ˧lo˧ qwæ˥}\hspace{5pt}\peng{to dig a ditch}\hspace{5pt}\pcmn{挖水沟}\hspace{5pt}\pfra{dégager une rigole}\end{exemple}
\begin{exemple}\pnru{jɤ˩jo˥ qwæ˩}\hspace{5pt}\peng{to dig out potatoes, to harvest potatoes}\hspace{5pt}\pcmn{挖洋芋}\hspace{5pt}\pfra{déterrer des pommes de terre, récolter des pommes de terre}\end{exemple}\sens{2}
\begin{définition}\peng{To scoop (water).}\end{définition}
\begin{définition}\pcmn{舀(水)}\end{définition}
\begin{définition}\pfra{Puiser (de l'eau).}\end{définition}
\begin{exemple}\pnru{dʑɯ˩ qwæ˩˥}\hspace{5pt}\peng{to scoop water}\hspace{5pt}\pcmn{舀水}\hspace{5pt}\pfra{puiser de l'eau}\end{exemple}
\end{entrée}

\begin{entrée}
{qwæ˧˥}{₂}{ⓔqwæ˧˥ⓗ2}\formedesurface{qwæ˧˥}\newline
\classe{动词}\ton{MH}
2\begin{définition}\peng{To engrave.}\end{définition}
\begin{définition}\pcmn{雕刻}\end{définition}
\begin{définition}\pfra{Graver.}\end{définition}
\begin{exemple}\pnru{bæ˩bæ˩ qwæ˥}\hspace{5pt}\peng{to engrave a flower}\hspace{5pt}\pcmn{刻花}\hspace{5pt}\pfra{graver une fleur}\end{exemple}
\begin{exemple}\pnru{qwæ˩∼qwæ˧˥}\hspace{5pt}\peng{|fg{red}}\hspace{5pt}\pcmn{重叠}\hspace{5pt}\pfra{|fg{red}}\end{exemple}
\begin{exemple}\pnru{bæ˩bæ˩ qwæ˥∼qwæ˩}\hspace{5pt}\peng{to engrave flowers}\hspace{5pt}\pcmn{刻一朵花}\hspace{5pt}\pfra{graver des fleurs}\end{exemple}
\end{entrée}

\begin{entrée}
{qwæ˧˥}{₃}{ⓔqwæ˧˥ⓗ3}\formedesurface{qwæ˧˥}\newline
\classe{名词}\ton{\#H}
3
\paradigme{\pcmn{:} \p{}}
\begin{définition}\peng{The bench of the main room, close to the hearth, where guests are seated.}\end{définition}
\begin{définition}\pcmn{主屋里面的长凳:客人和老人坐的地方}\end{définition}
\begin{définition}\pfra{Banc de la pièce principale, proche du foyer, où s'asseyent les hôtes.}\end{définition}
\end{entrée}

\begin{entrée}
{qwæ˩˥}{}{ⓔqwæ˩˥}\formedesurface{qwæ˩˥}\newline
\classe{名词}\ton{LH}
\paradigme{\pcmn{:} \p{}}
\begin{définition}\peng{Jaw (monosyllable).}\end{définition}
\begin{définition}\pcmn{嘴巴(单音节)}\end{définition}
\begin{définition}\pfra{Mâchoire (monosyllabe).}\end{définition}
\end{entrée}

\begin{entrée}
{qwæ˩ɖʐæ˩}{}{ⓔqwæ˩ɖʐæ˩}\formedesurface{qwæ˩ɖʐæ˩˥}\newline
\classe{名词}\ton{L}
\paradigme{\pcmn{:} \p{}}
\begin{définition}\peng{Jaw; mouth.}\end{définition}
\begin{définition}\pcmn{颚、嘴、嘴巴、口}\end{définition}
\begin{définition}\pfra{Mâchoire; bouche.}\end{définition}
\begin{exemple}\pnru{qwæ˩ɖʐæ˩-qo˥-ɳɯ˩ | ʈʰæ˧˥}\hspace{5pt}\peng{to masticate, to gnaw}\hspace{5pt}\pcmn{咬在嘴里}\hspace{5pt}\pfra{mastiquer, ronger}\end{exemple}
\end{entrée}

\begin{entrée}
{qwæ˧lo˧˥}{}{ⓔqwæ˧lo˧˥}\formedesurface{qwæ˧lo˧˥}\newline
\classe{名词}\ton{MH\#}
\paradigme{\pcmn{:} \p{}}
\begin{définition}\peng{Passageway, small lane, small path.}\end{définition}
\begin{définition}\pcmn{过道、小道}\end{définition}
\begin{définition}\pfra{Petit passage, petit sentier.}\end{définition}
\begin{exemple}\pnru{qwæ˧lo˧-qo˥ | gɤ˩tɕo˧ le˧-jo˩}\hspace{5pt}\peng{to come by the small lane}\hspace{5pt}\pcmn{抄小道}\hspace{5pt}\pfra{venir par le petit chemin/la sente (contexte: on demande à l'enquêteur si, pour se rendre de la maison de la consultante à son hameau natal, tout proche, il est passé par la rue principale de Yongning, ou a emprunté le petit chemin de derrière, parmi les champs)}\end{exemple}
\end{entrée}

\begin{entrée}
{qwæ˧mæ\#˥}{}{ⓔqwæ˧mæ\#˥}\formedesurface{qwæ˧mæ˧}\newline
\classe{名词}\ton{\#H}
\paradigme{\pcmn{:} \p{}}
\begin{définition}\peng{Middle part of the main room.}\end{définition}
\begin{définition}\pcmn{主屋的中庭:在主屋上半部分与门之间的空间}\end{définition}
\begin{définition}\pfra{Partie médiane du foyer: sur la partie surélevée, mais «côté cuisine», pas la partie la plus noble de l'espace où on prend les repas.}\end{définition}
\begin{exemple}\pnru{qwæ˧mæ˧-qo˩}\hspace{5pt}\peng{In the middle part of the main room.}\hspace{5pt}\pcmn{在主屋的中庭}\hspace{5pt}\pfra{Dans la partie médiane du foyer.}\end{exemple}
\begin{exemple}\pnru{qwæ˧mæ˧, | mv̩˧kʰɯ˩-di˩ mæ˩qo˩!}\hspace{5pt}\peng{The middle part of the main room is behind (=around) the place where we make fire!}\hspace{5pt}\pcmn{主屋的中庭,在点火的地方后面(=周围)!}\hspace{5pt}\pfra{La partie médiane du foyer, c'est ce qu'il y a derrière l'endroit où on fait le feu!}\end{exemple}
\end{entrée}

\begin{entrée}
{qwæ˩∼qwæ˧˥}{}{ⓔqwæ˩∼qwæ˧˥}\formedesurface{qwæ˩qwæ˧˥}\newline
\classe{动词}\begin{définition}\peng{To scratch.}\end{définition}
\begin{définition}\pcmn{抠痒}\end{définition}
\begin{définition}\pfra{Se gratter; gratter, gratouiller.}\end{définition}
\begin{exemple}\pnru{le˧-qwæ˧∼qwæ˩-ze˩}\hspace{5pt}\peng{|fg{accomp} \_ |fg{red} |fg{pfv}}\hspace{5pt}\pcmn{|fg{accomp} \_ |fg{red} |fg{pfv}}\hspace{5pt}\pfra{|fg{accomp} \_ |fg{red} |fg{pfv}}\end{exemple}
\end{entrée}

\begin{entrée}
{qwæ˧ʁo\#˥}{}{ⓔqwæ˧ʁo\#˥}\formedesurface{qwæ˧ʁo˧}\newline
\classe{名词}\ton{\#H}
\paradigme{\pcmn{:} \p{}}
\begin{définition}\peng{The bench of the main room, close to the hearth, where guests are seated.}\end{définition}
\begin{définition}\pcmn{主屋里面的长凳:客人和老人坐的地方}\end{définition}
\begin{définition}\pfra{Banc de la pièce principale, proche du foyer, où s'asseyent les hôtes.}\end{définition}
\end{entrée}

\begin{entrée}
{qwæ˧ʂe\#˥}{}{ⓔqwæ˧ʂe\#˥}\formedesurface{qwæ˧ʂe˥}\newline
\classe{名词}\ton{\#H}
\paradigme{\pcmn{:} \p{}}
\begin{définition}\peng{Bedbug.}\end{définition}
\begin{définition}\pcmn{臭虫}\end{définition}
\begin{définition}\pfra{Punaise.}\end{définition}
\end{entrée}

\begin{entrée}
{qwæ˧ʂe˧lɑ˧bv̩˥}{}{ⓔqwæ˧ʂe˧lɑ˧bv̩˥}\formedesurface{qwæ˧ʂe˧lɑ˧bv̩˥}\newline
\classe{名词}\ton{H\#}
\paradigme{\pcmn{:} \p{}}
\begin{définition}\peng{A species of worm.}\end{définition}
\begin{définition}\pcmn{一种蠕虫}\end{définition}
\begin{définition}\pfra{Sorte de ver.}\end{définition}
\end{entrée}

\begin{entrée}
{qwæ˩ʂv̩˧˥}{}{ⓔqwæ˩ʂv̩˧˥}\formedesurface{qwæ˩ʂv̩˧˥}\newline
\classe{名词}\ton{LM+MH\#}
\paradigme{\pcmn{:} \p{}}
\begin{définition}\peng{Bit (of a bridle).}\end{définition}
\begin{définition}\pcmn{马嚼子}\end{définition}
\begin{définition}\pfra{Mors.}\end{définition}
\begin{exemple}\pnru{ʐwæ˧-qwæ˥ʂv̩˩}\hspace{5pt}\peng{bit of a horse's bridle}\hspace{5pt}\pcmn{马嚼子}\hspace{5pt}\pfra{mors de cheval}\end{exemple}
\end{entrée}

\begin{entrée}
{qwæ˧zo˧zo˩}{}{ⓔqwæ˧zo˧zo˩}\formedesurface{qwæ˧zo˧zo˩}\newline
\classe{名词}\ton{L\#}
\paradigme{\pcmn{:} \p{}}
\begin{définition}\peng{The bench of the main room, close to the hearth, where guests are seated.}\end{définition}
\begin{définition}\pcmn{主屋的长凳,离火塘近。这是客人的尊座。}\end{définition}
\begin{définition}\pfra{Banc de la pièce principale, proche du foyer, où s'asseyent les hôtes.}\end{définition}
\end{entrée}

\begin{entrée}
{qwɤ˧}{}{ⓔqwɤ˧}\formedesurface{qwɤ˧}\newline
\classe{名词}\ton{M}
\paradigme{\pcmn{:} \p{}}
\begin{définition}\peng{Fire pit.}\end{définition}
\begin{définition}\pcmn{火塘}\end{définition}
\begin{définition}\pfra{Foyer, âtre, lieu où on fait du feu dans la maison.}\end{définition}
\begin{exemple}\pnru{qwɤ˧, | mv̩˧ kʰɯ˩-di˩!}\hspace{5pt}\peng{The fire pit is the place where one puts fire / where one does a fire!}\hspace{5pt}\pcmn{火塘,就是升火的地方!}\hspace{5pt}\pfra{Le foyer, c'est là où on allume le feu!}\end{exemple}
\end{entrée}

\begin{entrée}
{qwɤ˧α}{}{ⓔqwɤ˧α}\formedesurface{qwɤ˧}\newline
\classe{动词}\ton{Mα}\begin{définition}\peng{To accuse, to denounce.}\end{définition}
\begin{définition}\pcmn{告状}\end{définition}
\begin{définition}\pfra{Accuser.}\end{définition}
\begin{exemple}\pnru{mɤ˧-qwɤ˧}\hspace{5pt}\peng{|fg{neg}}\hspace{5pt}\pcmn{不告状}\hspace{5pt}\pfra{|fg{neg}}\end{exemple}
\begin{exemple}\pnru{hĩ˧ qwɤ˩}\hspace{5pt}\peng{to accuse someone, to denounce someone}\hspace{5pt}\pcmn{告一个人}\hspace{5pt}\pfra{dénoncer quelqu'un}\end{exemple}
\begin{exemple}\pnru{njɤ˧-ɳɯ˧ | qwɤ˧-bi˧!}\hspace{5pt}\peng{I am going to denounce/accuse}\hspace{5pt}\pcmn{我要告状!}\hspace{5pt}\pfra{je vais (te) dénoncer!}\end{exemple}
\begin{exemple}\pnru{no˧ | le˧-qwɤ˧-hõ˧!}\hspace{5pt}\peng{Go and denounce (him/her)!}\hspace{5pt}\pcmn{你去告状吧!}\hspace{5pt}\pfra{va (le) dénoncer!}\end{exemple}
\begin{exemple}\pnru{qwɤ˧∼qwɤ˩}\hspace{5pt}\peng{|fg{red}}\hspace{5pt}\pcmn{重叠}\hspace{5pt}\pfra{|fg{red}}\end{exemple}
\end{entrée}

\begin{entrée}
{qwɤ˩α}{}{ⓔqwɤ˩α}\formedesurface{qwɤ˩˥}\newline
\classe{动词}\ton{Lα}\begin{définition}\peng{To grow.}\end{définition}
\begin{définition}\pcmn{生长、长}\end{définition}
\begin{définition}\pfra{Pousser, grandir.}\end{définition}
\begin{exemple}\pnru{gɤ˩-qwɤ˥}\hspace{5pt}\peng{to grow}\hspace{5pt}\pcmn{长大,生长}\hspace{5pt}\pfra{grandir, pousser}\end{exemple}
\begin{exemple}\pnru{ʈʂʰɯ˧ | gɤ˩-qwɤ˥-ze˩!}\hspace{5pt}\peng{(S)he has grown up! / (S)he has grown a lot! (About a child)}\hspace{5pt}\pcmn{他长大了!(关于一个小孩)}\hspace{5pt}\pfra{Il/elle a grandi! (Au sujet d'un enfant qu'on revoit après un certain temps)}\end{exemple}
\end{entrée}

\begin{entrée}
{qwɤ˧ɭɯ\#˥}{}{ⓔqwɤ˧ɭɯ\#˥}\formedesurface{qwɤ˧ɭɯ˧}\newline
\classe{名词}\ton{\#H}\begin{définition}\peng{Fireplace. It can be the fireplace in the main room of the house, or a campfire.}\end{définition}
\begin{définition}\pcmn{火塘。包括家里的火塘和野外的临时营火、篝火。}\end{définition}
\begin{définition}\pfra{Foyer. Il peut aussi bien s'agir du foyer dans la maison que d'un feu de camp: foyer bâti à l'extérieur, provisoirement, lorsqu'on campe sur la montagne.}\end{définition}
\begin{exemple}\pnru{qwɤ˧ɭɯ˧-pʰɤ˧bɤ˥}\hspace{5pt}\peng{the gifts offered to the ancestors, at the fireplace: even when building a campfire for one day only on the mountain, one offers a little food to the ancestors before beginning the meal (in the same way as is done at home)}\hspace{5pt}\pcmn{敬给祖先的礼物:即使在山上升起篝火野餐,还是要像在家里一样,用餐前先敬给祖先一些饭。}\hspace{5pt}\pfra{les cadeaux offerts aux ancêtres: même lorsqu'il ne s'agit que d'un foyer provisoire, bâti pour une seule journée dans un campement en montagne, on pratique l'offrande d'un peu de nourriture}\end{exemple}
\end{entrée}

\begin{entrée}
{qwɤ˩pi˩}{}{ⓔqwɤ˩pi˩}\formedesurface{qwɤ˩pi˩˥}\newline
\classe{名词}\ton{L}
\paradigme{\pcmn{:} \p{}}
\begin{définition}\peng{Mouth.}\end{définition}
\begin{définition}\pcmn{嘴巴}\end{définition}
\begin{définition}\pfra{Bouche.}\end{définition}
\begin{exemple}\pnru{qwɤ˩pi˩-qo˩lo˥}\hspace{5pt}\peng{inside the mouth}\hspace{5pt}\pcmn{嘴巴里}\hspace{5pt}\pfra{à l'intérieur de la bouche}\end{exemple}
\begin{exemple}\pnru{ko˩pi˩-ko˩lo˧}\hspace{5pt}\peng{inside the mouth}\hspace{5pt}\pcmn{嘴巴里}\hspace{5pt}\pfra{dans la bouche, à l'intérieur de la bouche}\end{exemple}
\end{entrée}

\begin{entrée}
{qʰɑ˥}{}{ⓔqʰɑ˥}\formedesurface{qʰɑ˧}\newline
\classe{形容词}\ton{H}\begin{définition}\peng{Bitter.}\end{définition}
\begin{définition}\pcmn{苦}\end{définition}
\begin{définition}\pfra{Amer.}\end{définition}
\end{entrée}

\begin{entrée}
{qʰɑ˧‑}{}{ⓔqʰɑ˧‑}\formedesurface{qʰɑ˧}\newline
\classe{助词}\begin{définition}\peng{Very, extremely.}\end{définition}
\begin{définition}\pcmn{多么、非常}\end{définition}
\begin{définition}\pfra{Particulièrement, très.}\end{définition}
\begin{exemple}\pnru{qʰɑ˧-ɖɯ˧-hĩ˧}\hspace{5pt}\peng{extremely big}\hspace{5pt}\pcmn{非常大}\hspace{5pt}\pfra{extrêmement gros}\end{exemple}
\begin{exemple}\pnru{qʰɑ˧-ɖɯ˧-gv̩˧}\hspace{5pt}\peng{extremely large; how large!}\hspace{5pt}\pcmn{非常大}\hspace{5pt}\pfra{particulièrement grand}\end{exemple}
\begin{exemple}\pnru{qʰɑ˧-ʂwæ˧-gv̩˧}\hspace{5pt}\peng{extremely tall; how tall!}\hspace{5pt}\pcmn{很高、非常高}\hspace{5pt}\pfra{particulièrement grand, de très grande taille}\end{exemple}
\begin{exemple}\pnru{qʰɑ˧-ʂwæ˧-mi˧zo˥}\hspace{5pt}\peng{extremely tall}\hspace{5pt}\pcmn{很高}\hspace{5pt}\pfra{très grand, de très haute taille}\end{exemple}
\begin{exemple}\pnru{qʰɑ˧-ɖɯ˧-mi˧zo˥}\hspace{5pt}\peng{extremely big}\hspace{5pt}\pcmn{很大}\hspace{5pt}\pfra{très gros, de très grande envergure}\end{exemple}
\end{entrée}

\begin{entrée}
{qʰɑ˧}{₁}{ⓔqʰɑ˧ⓗ1}\formedesurface{qʰɑ˧}\newline
\classe{代词}\ton{M}
1\begin{définition}\peng{How many (small number).}\end{définition}
\begin{définition}\pcmn{几、多少}\end{définition}
\begin{définition}\pfra{Combien.}\end{définition}
\begin{exemple}\pnru{hĩ˧ | qʰɑ˧-kv̩˧˥?}\hspace{5pt}\peng{how many people?}\hspace{5pt}\pcmn{几个人?}\hspace{5pt}\pfra{combien de gens?}\end{exemple}
\begin{exemple}\pnru{bæ˩bæ˩˥ | qʰɑ˧-bæ˩?}\hspace{5pt}\peng{how many flowers?}\hspace{5pt}\pcmn{几朵花?}\hspace{5pt}\pfra{combien de fleurs?}\end{exemple}
\begin{exemple}\pnru{qʰɑ˧-ʑi˩?}\hspace{5pt}\peng{how many families?}\hspace{5pt}\pcmn{几家?}\hspace{5pt}\pfra{combien de familles?}\end{exemple}
\begin{exemple}\pnru{hɑ˧ | qʰɑ˧-tɕʰi˩?}\hspace{5pt}\peng{how many meals?}\hspace{5pt}\pcmn{几顿饭?}\hspace{5pt}\pfra{combien de repas?}\end{exemple}
\begin{exemple}\pnru{qʰɑ˧-ɲi˧?}\hspace{5pt}\peng{how many days?}\hspace{5pt}\pcmn{几天?}\hspace{5pt}\pfra{combien de jours?}\end{exemple}
\begin{exemple}\pnru{qʰɑ˧-kʰv̩˧ gv̩˧-ze˩?}\hspace{5pt}\peng{How old are you / is (s)he?}\hspace{5pt}\pcmn{几岁了?}\hspace{5pt}\pfra{quel âge avez-(vous)?}\end{exemple}
\begin{exemple}\pnru{qʰɑ˧-kʰv̩˧˥?}\hspace{5pt}\peng{how many years?}\hspace{5pt}\pcmn{几年?}\hspace{5pt}\pfra{combien d'années?}\end{exemple}
\begin{exemple}\pnru{qʰɑ˧-kʰwɤ˧˥?}\hspace{5pt}\peng{how many pieces?}\hspace{5pt}\pcmn{几块?}\hspace{5pt}\pfra{combien de morceaux?}\end{exemple}
\begin{exemple}\pnru{qʰɑ˧-nɑ˧?}\hspace{5pt}\peng{how many (tools…)?}\hspace{5pt}\pcmn{几把?}\hspace{5pt}\pfra{combien (d'outils…)?}\end{exemple}
\begin{exemple}\pnru{sɯ˩tʰi˩˥ | qʰɑ˧-nɑ˧ dʑo˧?}\hspace{5pt}\peng{How many knives are there?}\hspace{5pt}\pcmn{有几把刀?}\hspace{5pt}\pfra{Combien y a-t-il de couteaux?}\end{exemple}
\begin{exemple}\pnru{qʰɑ˧-kʰɯ˩}\hspace{5pt}\peng{how many (long objects)}\hspace{5pt}\pcmn{几条}\hspace{5pt}\pfra{combien (d'objets longs)}\end{exemple}
\begin{exemple}\pnru{qʰɑ˧-kʰɯ˩ dʑo˩?}\hspace{5pt}\peng{How many (long objects) are there?}\hspace{5pt}\pcmn{有几条?}\hspace{5pt}\pfra{Combien y a-t-il (d'objets longs)?}\end{exemple}
\begin{exemple}\pnru{qʰɑ˧-mæ˩ dʑo˩?}\hspace{5pt}\peng{How much money do (you) have?}\hspace{5pt}\pcmn{有几块(钱)?}\hspace{5pt}\pfra{Combien (tu) as d'argent?}\end{exemple}
\begin{exemple}\pnru{si˧dzi˩ | qʰɑ˧-dzi˩?}\hspace{5pt}\peng{how many trees?}\hspace{5pt}\pcmn{几棵树?}\hspace{5pt}\pfra{combien d'arbres?}\end{exemple}
\begin{exemple}\pnru{si˧kɤ˧˥ | qʰɑ˧-kɤ˧˥?}\hspace{5pt}\peng{how many branches?}\hspace{5pt}\pcmn{几枝树枝?}\hspace{5pt}\pfra{combien de branches?}\end{exemple}
\begin{exemple}\pnru{qʰɑ˧-kʰɤ˧˥?}\hspace{5pt}\peng{how many baskets?}\hspace{5pt}\pcmn{几筐?}\hspace{5pt}\pfra{combien de cageots?}\end{exemple}
\end{entrée}

\begin{entrée}
{qʰɑ˧}{₂}{ⓔqʰɑ˧ⓗ2}\formedesurface{qʰɑ˧}\newline
\classe{助词}\ton{M}
2\begin{définition}\peng{A few; several; some.}\end{définition}
\begin{définition}\pcmn{几(如:十几个)}\end{définition}
\begin{définition}\pfra{Quelques, plusieurs.}\end{définition}
\begin{exemple}\pnru{tsʰe˩-qʰɑ˩˥}\hspace{5pt}\peng{ten and a few more (i.e. between ten and twenty)}\hspace{5pt}\pcmn{十几个、十来个}\hspace{5pt}\pfra{dix et plus (entre dix et vingt)}\end{exemple}
\begin{exemple}\pnru{tsʰe˩-qʰɑ˩-kv̩˩˥}\hspace{5pt}\peng{ten and a few more (i.e. between ten and twenty)}\hspace{5pt}\pcmn{十几个、十来个}\hspace{5pt}\pfra{dix et plus (entre dix et vingt)}\end{exemple}
\end{entrée}

\begin{entrée}
{qʰɑ˧dze˧}{}{ⓔqʰɑ˧dze˧}\formedesurface{qʰɑ˧dze˧}\newline
\classe{名词}\ton{M}
\paradigme{\pcmn{:} \p{}}
\begin{définition}\peng{Sweet corn; maize; Indian corn.}\end{définition}
\begin{définition}\pcmn{玉米、包谷}\end{définition}
\begin{définition}\pfra{Maïs.}\end{définition}
\begin{exemple}\pnru{qʰɑ˧dze˧-kʰɯ˩ʈɯ˩}\hspace{5pt}\peng{the roots of the sweetcorn plant}\hspace{5pt}\pcmn{玉米的根}\hspace{5pt}\pfra{racines des plants de maïs}\end{exemple}
\begin{exemple}\pnru{qʰɑ˧dze˧ qʰæ˩}\hspace{5pt}\peng{to cut ears of sweetcorn, to snap off ears of sweetcorn}\hspace{5pt}\pcmn{采玉米:折断玉米棒子}\hspace{5pt}\pfra{cueillir le maïs: arracher les épis de maïs}\end{exemple}
\begin{exemple}\pnru{qʰɑ˧dze˧ ɖʐɤ˧˥}\hspace{5pt}\peng{to harvest sweetcorn, to pick sweetcorn}\hspace{5pt}\pcmn{采玉米}\hspace{5pt}\pfra{cueillir le maïs}\end{exemple}
\begin{exemple}\pnru{qʰɑ˧dze˧-tsɑ˩bɤ˩ | ɖɯ˧-mɤ˩}\hspace{5pt}\peng{a little sweetcorn flour}\hspace{5pt}\pcmn{一点玉米粉}\hspace{5pt}\pfra{un peu de farine de maïs}\end{exemple}
\begin{exemple}\pnru{qʰɑ˧dze˧-hɑ˧bɤ˥, | qʰɑ˧dze˧-hɑ˧ɭɯ\#˥, | qʰɑ˧dze˧-tsɑ˩bɤ˩}\hspace{5pt}\peng{three forms of sweetcorn: sweetcorn ear; sweetcorn grains; sweetcorn flour}\hspace{5pt}\pcmn{玉米的三种形态:玉米棒子,玉米粒,玉米粉}\hspace{5pt}\pfra{le maïs sous trois formes: épis de maïs; maïs en grains; farine de maïs}\end{exemple}
\end{entrée}

\begin{entrée}
{qʰɑ˧dze˧-hwæ˩-di˩}{}{ⓔqʰɑ˧dze˧-hwæ˩-di˩}\formedesurface{qʰɑ˧dze˧hwæ˩di˩}\newline
\classe{名词}\ton{-L}
\paradigme{\pcmn{:} \p{}}
\begin{définition}\peng{Horizontal beams of the rack for drying grain: ‘the place to hang sweetcorn'.}\end{définition}
\begin{définition}\pcmn{粮架的横梁}\end{définition}
\begin{définition}\pfra{Poutrelle d'espalier à sécher le maïs: partie horizontale de la structure en bois. Périphrase: ‘endroit où on accroche le maïs'.}\end{définition}
\end{entrée}

\begin{entrée}
{qʰɑ˧dze˧-lv̩˧}{}{ⓔqʰɑ˧dze˧-lv̩˧}\formedesurface{qʰɑ˧dze˧lv̩˧}\newline
\classe{名词}\ton{M}
\paradigme{\pcmn{:} \p{}}
\begin{définition}\peng{Maize field.}\end{définition}
\begin{définition}\pcmn{包谷田、玉米田}\end{définition}
\begin{définition}\pfra{Champ de maïs.}\end{définition}
\end{entrée}

\begin{entrée}
{qʰɑ˩jɤ˩}{}{ⓔqʰɑ˩jɤ˩}\formedesurface{qʰɑ˩jɤ˩˥}\newline
\classe{代词}\ton{L}\begin{définition}\peng{How many.}\end{définition}
\begin{définition}\pcmn{多少}\end{définition}
\begin{définition}\pfra{Combien.}\end{définition}
\begin{exemple}\pnru{qʰɑ˩jɤ˩ tʰi˥-ki˩?}\hspace{5pt}\peng{How much does it cost?}\hspace{5pt}\pcmn{要给多少? = 多少钱?}\hspace{5pt}\pfra{combien donner? / c'est combien?}\end{exemple}
\begin{exemple}\pnru{qʰɑ˩jɤ˩ ɲi˧?}\hspace{5pt}\peng{How much do (you) need?}\hspace{5pt}\pcmn{要多少?}\hspace{5pt}\pfra{combien (t'en) faut-il?}\end{exemple}
\begin{exemple}\pnru{ɖʐe˧ | qʰɑ˩jɤ˩ ɲi˧?}\hspace{5pt}\peng{How much money does it cost?}\hspace{5pt}\pcmn{要多少钱?}\hspace{5pt}\pfra{Combien d'argent ça coûte?}\end{exemple}
\end{entrée}

\begin{entrée}
{qʰɑ˩ne˩}{}{ⓔqʰɑ˩ne˩}\formedesurface{qʰɑ˩ne˩˥}\newline
\classe{代词}\ton{L}\begin{définition}\peng{How.}\end{définition}
\begin{définition}\pcmn{怎么样}\end{définition}
\begin{définition}\pfra{Pronom interrogatif: comment?.}\end{définition}
\begin{exemple}\pnru{qʰɑ˩ne˩ ʝi˥?}\hspace{5pt}\peng{how to do / how is it done?}\hspace{5pt}\pcmn{怎么做?}\hspace{5pt}\pfra{comment faire?}\end{exemple}
\begin{exemple}\pnru{qʰɑ˩ne˩ ʝi˥-tso˩-ɲi˩?}\hspace{5pt}\peng{how must one do / how is it done?}\hspace{5pt}\pcmn{要怎么做?}\hspace{5pt}\pfra{comment faut-il faire?}\end{exemple}
\begin{exemple}\pnru{qʰɑ˩ne˩ gv̩˩˥?}\hspace{5pt}\peng{how to do / how is it done?}\hspace{5pt}\pcmn{怎么做?}\hspace{5pt}\pfra{comment faire?}\end{exemple}
\begin{exemple}\pnru{qʰɑ˩ne˩ gv̩˩-ho˥-ze˩?}\hspace{5pt}\peng{What happened?}\hspace{5pt}\pcmn{怎么样了?发展到什么程度?}\hspace{5pt}\pfra{que s'est-il passé?}\end{exemple}
\end{entrée}

\begin{entrée}
{qʰɑ˧tɑ˧}{}{ⓔqʰɑ˧tɑ˧}\formedesurface{qʰɑ˧tɑ˧}\newline
\classe{代词}\ton{M}\begin{définition}\peng{When.}\end{définition}
\begin{définition}\pcmn{什么时候}\end{définition}
\begin{définition}\pfra{Quand.}\end{définition}
\begin{exemple}\pnru{qʰɑ˧tɑ˧ bi˧?}\hspace{5pt}\peng{When will you go?}\hspace{5pt}\pcmn{你什么时候去?}\hspace{5pt}\pfra{quand (y) vas(-tu)?}\end{exemple}
\end{entrée}

\begin{entrée}
{qʰɑ˧-ʈʂʰɤ˧∼ʈʂʰɤ˥}{}{ⓔqʰɑ˧-ʈʂʰɤ˧∼ʈʂʰɤ˥}\formedesurface{qʰɑ˧ʈʂʰɤ˧ʈʂʰɤ˥}\newline
\classe{形容词}\ton{H\#}\begin{définition}\peng{Completely dry.}\end{définition}
\begin{définition}\pcmn{充分干燥}\end{définition}
\begin{définition}\pfra{Tout sec.}\end{définition}
\begin{exemple}\pnru{qʰɑ˧-ʈʂʰɤ˧∼ʈʂʰɤ˥-gv̩˩}\hspace{5pt}\peng{completely dry}\hspace{5pt}\pcmn{充分干燥}\hspace{5pt}\pfra{complètement sec}\end{exemple}
\begin{exemple}\pnru{le˧-pv̩˧-zo˩, | qʰɑ˧ʈʂʰɤ˧ʈʂʰɤ˥-gv̩˩-ze˩!}\hspace{5pt}\peng{Now it is dry: completely dry!}\hspace{5pt}\pcmn{现在,干了:全干了!}\hspace{5pt}\pfra{c'est maintenant sec, et complètement sec!}\end{exemple}
\end{entrée}

\begin{entrée}
{qʰæ˥}{₁}{ⓔqʰæ˥ⓗ1}\formedesurface{qʰæ˧}\newline
\classe{动词}\ton{H}
1\begin{définition}\peng{To gnaw, to nibble.}\end{définition}
\begin{définition}\pcmn{啃(啃骨头)}\end{définition}
\begin{définition}\pfra{Ronger.}\end{définition}
\end{entrée}

\begin{entrée}
{qʰæ˥}{₂}{ⓔqʰæ˥ⓗ2}\formedesurface{qʰæ˧}\newline
\classe{形容词}\ton{H}
2\begin{définition}\peng{Cold (water).}\end{définition}
\begin{définition}\pcmn{冷(水)}\end{définition}
\begin{définition}\pfra{Froide (eau).}\end{définition}
\begin{exemple}\pnru{dʑɯ˩qʰæ˩}\hspace{5pt}\peng{cold water}\hspace{5pt}\pcmn{凉水}\hspace{5pt}\pfra{eau froide}\end{exemple}
\begin{exemple}\pnru{qʰæ˧-ɕjæ˧-gv̩˧}\hspace{5pt}\peng{very cold}\hspace{5pt}\pcmn{冷得很}\hspace{5pt}\pfra{très froid}\end{exemple}
\end{entrée}

\begin{entrée}
{qʰæ˥}{₃}{ⓔqʰæ˥ⓗ3}\formedesurface{qʰæ˧}\newline
\classe{名词}\ton{\#H}
3
\paradigme{\pcmn{:} \p{}}
\begin{définition}\peng{Trench (monosyllable).}\end{définition}
\begin{définition}\pcmn{水沟(单音节)}\end{définition}
\begin{définition}\pfra{Canal, rigole.}\end{définition}
\end{entrée}

\begin{entrée}
{qʰæ˥}{₄}{ⓔqʰæ˥ⓗ4}\newline
\classe{名词}
4
\sens{1}\paradigme{\pcmn{:} \p{}}
\begin{définition}\peng{Excrement, dung, droppings.}\end{définition}
\begin{définition}\pcmn{屎、垃圾、 肥料}\end{définition}
\begin{définition}\pfra{Excréments, fèces.}\end{définition}
\begin{exemple}\pnru{qʰæ˧ ɖɯ˧-pɤ˧ ʂe˧˥}\hspace{5pt}\peng{to defecate}\hspace{5pt}\pcmn{拉一泡屎}\hspace{5pt}\pfra{faire une crotte}\end{exemple}
\begin{exemple}\pnru{qʰæ˧ kv̩˥}\hspace{5pt}\peng{to pick up dung}\hspace{5pt}\pcmn{捡(马……)屎}\hspace{5pt}\pfra{ramasser du crottin}\end{exemple}
\begin{exemple}\pnru{qʰæ˧-pi˩ kv̩˩}\hspace{5pt}\peng{to pick up a little dung (to fertilize the fields)}\hspace{5pt}\pcmn{捡一点(马……)屎}\hspace{5pt}\pfra{ramasser un peu de crottin}\end{exemple}
\begin{exemple}\pnru{qʰæ˧ ɖɯ˧-pi˧ kv̩˥}\hspace{5pt}\peng{as above: to pick up a little dung (to fertilize the fields)}\hspace{5pt}\pcmn{同上:捡一点(马……)屎}\hspace{5pt}\pfra{même sens: ramasser un peu de crottin}\end{exemple}\sens{2}\paradigme{\pcmn{:} \p{}}
\begin{définition}\peng{Flatulence, fart.}\end{définition}
\begin{définition}\pcmn{屁}\end{définition}
\begin{définition}\pfra{Pet.}\end{définition}
\begin{exemple}\pnru{qʰæ˧ kʰɯ˩}\hspace{5pt}\peng{to fart}\hspace{5pt}\pcmn{放屁}\hspace{5pt}\pfra{péter}\end{exemple}
\begin{exemple}\pnru{qʰæ˧ | ɖɯ˧-pɤ˥ kʰɯ˩}\hspace{5pt}\peng{to fart, to make a fart}\hspace{5pt}\pcmn{放一个屁}\hspace{5pt}\pfra{faire un pet}\end{exemple}\sens{3}
\begin{définition}\peng{Refuse, garbage.}\end{définition}
\begin{définition}\pcmn{垃圾}\end{définition}
\begin{définition}\pfra{Ordure, détritus.}\end{définition}
\end{entrée}

\begin{entrée}
{qʰæ˧˥}{₁}{ⓔqʰæ˧˥ⓗ1}\formedesurface{qʰæ˧˥}\newline
\classe{动词}\ton{MH}
1\begin{définition}\peng{To come out (moon, sun).}\end{définition}
\begin{définition}\pcmn{出来(月亮,太阳)}\end{définition}
\begin{définition}\pfra{Paraître, se lever (lune, soleil).}\end{définition}
\begin{exemple}\pnru{tʰi˧-qʰæ˧-ze˥}\hspace{5pt}\peng{|fg{dur} \_ |fg{pfv}}\hspace{5pt}\pcmn{|fg{dur} \_ |fg{pfv}}\hspace{5pt}\pfra{|fg{dur} \_ |fg{pfv}}\end{exemple}
\end{entrée}

\begin{entrée}
{qʰæ˧˥}{₂}{ⓔqʰæ˧˥ⓗ2}\formedesurface{qʰæ˧˥}\newline
\classe{动词}\ton{MH}
2\begin{définition}\peng{To pull down, to dismantle.}\end{définition}
\begin{définition}\pcmn{拆}\end{définition}
\begin{définition}\pfra{Démolir.}\end{définition}
\begin{exemple}\pnru{ʑi˧qʰwɤ˧ qʰæ˧˥}\hspace{5pt}\peng{to demolish a house}\hspace{5pt}\pcmn{拆房子}\hspace{5pt}\pfra{démolir une maison}\end{exemple}
\end{entrée}

\begin{entrée}
{qʰæ˧˥}{₃}{ⓔqʰæ˧˥ⓗ3}\formedesurface{qʰæ˧˥}\newline
\classe{动词}\ton{MH}
3\begin{définition}\peng{To share: several people share something among themselves; someone shares out something.}\end{définition}
\begin{définition}\pcmn{分东西、(大家)平分东西}\end{définition}
\begin{définition}\pfra{Partager, répartir.}\end{définition}
\end{entrée}

\begin{entrée}
{qʰæ˧˥}{₄}{ⓔqʰæ˧˥ⓗ4}\formedesurface{qʰæ˧˥}\newline
\classe{动词}\ton{MH}
4\begin{définition}\peng{To shoot (with a gun).}\end{définition}
\begin{définition}\pcmn{开枪}\end{définition}
\begin{définition}\pfra{Tirer (avec une arme à feu, une arbalète; aussi avec un arc: tirer une flèche).}\end{définition}
\begin{exemple}\pnru{le˧-qʰæ˧-ze˥}\hspace{5pt}\peng{|fg{accomp} \_ |fg{pfv}}\hspace{5pt}\pcmn{开枪了}\hspace{5pt}\pfra{|fg{accomp} \_ |fg{pfv}}\end{exemple}
\begin{exemple}\pnru{mv̩˧ʐe˧ qʰæ˩(-ze˩)}\hspace{5pt}\peng{to shoot with a gun}\hspace{5pt}\pcmn{开枪}\hspace{5pt}\pfra{tirer avec une arme à feu}\end{exemple}
\end{entrée}

\begin{entrée}
{qʰæ˧˥}{₅}{ⓔqʰæ˧˥ⓗ5}\formedesurface{qʰæ˧˥}\newline
\classe{形容词}\ton{MH}
5\begin{définition}\peng{Happy, content, peaceful, at peace.}\end{définition}
\begin{définition}\pcmn{幸福,安逸,平安}\end{définition}
\begin{définition}\pfra{Heureux.}\end{définition}
\begin{exemple}\pnru{le˧-qʰæ˧-ze˥}\hspace{5pt}\peng{|fg{accomp} \_ |fg{pfv}}\hspace{5pt}\pcmn{|fg{accomp} \_ |fg{pfv}}\hspace{5pt}\pfra{|fg{accomp} \_ |fg{pfv}}\end{exemple}
\begin{exemple}\pnru{lo˧ qʰæ˩}\hspace{5pt}\peng{to work in a quiet, relaxed manner}\hspace{5pt}\pcmn{轻松工作}\hspace{5pt}\pfra{travailler de façon tranquille, détendue, paisible}\end{exemple}
\end{entrée}

\begin{entrée}
{qʰæ˧˥}{₆}{ⓔqʰæ˧˥ⓗ6}\formedesurface{qʰæ˧˥}\newline
\classe{动词}\ton{MH}
6\begin{définition}\peng{To burn, to go brown: food or oil gets close to burning point (but remains edible).}\end{définition}
\begin{définition}\pcmn{糊、变黑(高温让油、食物变黑,变糊了)}\end{définition}
\begin{définition}\pfra{Brûler, griller, noircir: la graisse chauffée à feu vif noircit, fume et donne de l'acroléine; le riz devient sec, très/trop cuit. La nourriture reste comestible.}\end{définition}
\begin{exemple}\pnru{le˧-qʰæ˧-ze˥}\hspace{5pt}\peng{|fg{accomp} \_ |fg{pfv}}\hspace{5pt}\pcmn{|fg{accomp} \_ |fg{pfv}}\hspace{5pt}\pfra{|fg{accomp} \_ |fg{pfv}}\end{exemple}
\begin{exemple}\pnru{mɤ˧ | le˧-qʰæ˧-ze˥}\hspace{5pt}\peng{The oil has burned / has reached boiling point / has gone black!}\hspace{5pt}\pcmn{油焦了!}\hspace{5pt}\pfra{L'huile a noirci / l'huile est parvenue à une très haute température.}\end{exemple}
\begin{exemple}\pnru{hɑ˧ | le˧-qʰæ˧-ze˥}\hspace{5pt}\peng{The rice has burned / is overcooked.}\hspace{5pt}\pcmn{饭糊了。}\hspace{5pt}\pfra{Le riz a brûlé. / Le riz est trop cuit.}\end{exemple}
\begin{exemple}\pnru{v̩˩tsʰɤ˩˥ | hṽ̩˧∼hṽ̩˧ F | le˧-qʰæ˧-ze˥!}\hspace{5pt}\peng{The vegetables are going brown / are overcooked / are getting burnt from frying!}\hspace{5pt}\pcmn{菜都炒糊了!}\hspace{5pt}\pfra{Les légumes, à force de frire, les voilà brûlés! / les voilà trop cuits!}\end{exemple}
\begin{exemple}\pnru{ʂe˧ | hṽ̩˧∼hṽ̩˧ F | le˧-qʰæ˧-ze˥!}\hspace{5pt}\peng{The meat is going brown / is overcooked / is getting burnt from frying!}\hspace{5pt}\pcmn{肉都炒焦了!}\hspace{5pt}\pfra{La viande, à force de frire, la voilà brûlée! / la voilà trop cuite!}\end{exemple}
\end{entrée}

\begin{entrée}
{qʰæ˧β}{}{ⓔqʰæ˧β}\formedesurface{qʰæ˧}\newline
\classe{动词}\ton{Mβ}\begin{définition}\peng{To break (a stick breaks).}\end{définition}
\begin{définition}\pcmn{断,破(棍子,竹竿)}\end{définition}
\begin{définition}\pfra{(se) briser, (se) casser (ex.: un bâton).}\end{définition}
\begin{exemple}\pnru{le˧-qʰæ˧-ze˧}\hspace{5pt}\peng{|fg{accomp} \_ |fg{pfv}}\hspace{5pt}\pcmn{断了}\hspace{5pt}\pfra{|fg{accomp} \_ |fg{pfv}}\end{exemple}
\begin{exemple}\pnru{si˧ qʰæ˩}\hspace{5pt}\peng{to break wood}\hspace{5pt}\pcmn{砸木头}\hspace{5pt}\pfra{briser du bois}\end{exemple}
\end{entrée}

\begin{entrée}
{qʰæ˩}{}{ⓔqʰæ˩}\formedesurface{qʰæ˩˥}\newline
\classe{动词}\ton{Lα}\begin{définition}\peng{To crack, to snap off.}\end{définition}
\begin{définition}\pcmn{折断}\end{définition}
\begin{définition}\pfra{Casser; ex.: casser une branche, récolter un épi de maïs en le cassant du plant de maïs.}\end{définition}
\begin{exemple}\pnru{qʰɑ˧dze˧ qʰæ˩}\hspace{5pt}\peng{to harvest sweet corn (literally: to snap off ears of sweet corn)}\hspace{5pt}\pcmn{采玉米}\hspace{5pt}\pfra{cueillir du maïs}\end{exemple}
\begin{exemple}\pnru{qʰɑ˧dze˧ | le˧-qʰæ˩-ze˩}\hspace{5pt}\peng{The sweetcorn has been harvested.}\hspace{5pt}\pcmn{玉米收好了。}\hspace{5pt}\pfra{le maïs est cueilli}\end{exemple}
\begin{exemple}\pnru{qʰɑ˧dze˧ | ɖɯ˧-qʰæ˧∼qʰæ˥-ɻ̍˩}\hspace{5pt}\peng{to harvest some sweet corn}\hspace{5pt}\pcmn{去采些玉米}\hspace{5pt}\pfra{cueillir un peu de maïs}\end{exemple}
\end{entrée}

\begin{entrée}
{qʰæ˩α}{₁}{ⓔqʰæ˩αⓗ1}\formedesurface{qʰæ˩˥}\newline
\classe{形容词}\ton{Lα}
1\begin{définition}\peng{False, fake.}\end{définition}
\begin{définition}\pcmn{假}\end{définition}
\begin{définition}\pfra{Faux, mensonger.}\end{définition}
\begin{exemple}\pnru{qʰæ˩-hĩ˩˥, | tʰɑ˧-ʐwɤ˩!}\hspace{5pt}\peng{Do not tell lies! / Do not tell things that are false!}\hspace{5pt}\pcmn{假话,不要说! =不要撒谎!}\hspace{5pt}\pfra{ne dis pas de mensonges!}\end{exemple}
\begin{exemple}\pnru{qʰæ˧ ʐwɤ˧}\hspace{5pt}\peng{to tell lies}\hspace{5pt}\pcmn{撒谎、说谎}\hspace{5pt}\pfra{dire des mensonges}\end{exemple}
\end{entrée}

\begin{entrée}
{qʰæ˩α}{₂}{ⓔqʰæ˩αⓗ2}\formedesurface{qʰæ˩˥}\newline
\classe{形容词}
2
\sens{1}
\begin{définition}\peng{Well (to feel well); quiet.}\end{définition}
\begin{définition}\pcmn{平静、安静,安乐、(身体)健康}\end{définition}
\begin{définition}\pfra{En paix, tranquille, paisible (époque); en bonne santé (corps).}\end{définition}
\begin{exemple}\pnru{hĩ˧ | ə˩-qʰæ˩˥?}\hspace{5pt}\peng{How are you?}\hspace{5pt}\pcmn{你好吗? / 一切好吗?}\hspace{5pt}\pfra{est-ce que ça va bien?/tu vas bien? (Formule équivalente du /ə˥-lɑ˧∼lɑ˩/ de Lijiang ‘Est-ce que vous êtes en bonne santé?', qui à Yongning évoque malencontreusement le rédupliqué /ə˩-lɑ˩∼lɑ˧˥/ ‘est-ce que [vous] vous disputez?')}\end{exemple}
\begin{exemple}\pnru{njɤ˧ | mɤ˧-qʰæ˩.}\hspace{5pt}\peng{I don't feel well.}\hspace{5pt}\pcmn{我不舒服。}\hspace{5pt}\pfra{je ne me sens pas bien.}\end{exemple}\sens{2}
\begin{définition}\peng{Light, easy (work).}\end{définition}
\begin{définition}\pcmn{轻松}\end{définition}
\begin{définition}\pfra{Léger, peu fatigant (travail).}\end{définition}
\begin{exemple}\pnru{qʰæ˩-hĩ˩˥}\hspace{5pt}\peng{|fg{rel}}\hspace{5pt}\pcmn{轻松的}\hspace{5pt}\pfra{|fg{rel}}\end{exemple}
\end{entrée}

\begin{entrée}
{qʰæ˩bæ˩}{}{ⓔqʰæ˩bæ˩}\formedesurface{qʰæ˩bæ˩˥}\newline
\classe{名词}\ton{L}
\paradigme{\pcmn{:} \p{}}
\begin{définition}\peng{Spoon, used for salt, tsamba… It corresponds to European teaspoons and tablespoons.}\end{définition}
\begin{définition}\pcmn{调羹}\end{définition}
\begin{définition}\pfra{Cuillère de petite taille: pour le sel, le tsamba… Elle correspond aux cuillères à café et cuillères à soupe du paradigme européen.}\end{définition}
\end{entrée}

\begin{entrée}
{qʰæ˧kʰwɤ\#˥}{}{ⓔqʰæ˧kʰwɤ\#˥}\formedesurface{qʰæ˧kʰwɤ˧}\newline
\classe{名词}\ton{\#H}
\paradigme{\pcmn{:} \p{}}
\begin{définition}\peng{Small dam (in canal; made of stones and earth).}\end{définition}
\begin{définition}\pcmn{小水坝,来堵塞田地里的小水渠}\end{définition}
\begin{définition}\pfra{Petit barrage pour bloquer un canal d'irrigation, fait de pierres et de terre. Pour irriguer, on l'ouvre à coups de houe.}\end{définition}
\begin{exemple}\pnru{qʰæ˧kʰwɤ˧ ɖɯ˧-ɭɯ˧}\hspace{5pt}\peng{a small dam}\hspace{5pt}\pcmn{一个小水坝}\hspace{5pt}\pfra{un petit canal}\end{exemple}
\end{entrée}

\begin{entrée}
{qʰæ˧lo˧˥}{}{ⓔqʰæ˧lo˧˥}\formedesurface{qʰæ˧lo˧˥}\newline
\classe{名词}\ton{MH\#}
\paradigme{\pcmn{:} \p{}}
\begin{définition}\peng{Small gulley, small trench.}\end{définition}
\begin{définition}\pcmn{小水渠}\end{définition}
\begin{définition}\pfra{Petite rigole, petit canal.}\end{définition}
\end{entrée}

\begin{entrée}
{qʰæ˧mi˧}{}{ⓔqʰæ˧mi˧}\formedesurface{qʰæ˧mi˧}\newline
\classe{名词}\ton{M}
\paradigme{\pcmn{:} \p{}}
\begin{définition}\peng{Large trench, canal.}\end{définition}
\begin{définition}\pcmn{大水渠}\end{définition}
\begin{définition}\pfra{Grand canal.}\end{définition}
\end{entrée}

\begin{entrée}
{qʰæ˧mo˩}{}{ⓔqʰæ˧mo˩}\formedesurface{qʰæ˧mo˩}\newline
\classe{名词}\ton{L\#}\begin{définition}\peng{A poisonous mushroom.}\end{définition}
\begin{définition}\pcmn{有毒的一种菌子}\end{définition}
\begin{définition}\pfra{Un champignon vénéneux.}\end{définition}
\end{entrée}

\begin{entrée}
{qʰæ˧pɤ\#˥}{}{ⓔqʰæ˧pɤ\#˥}\formedesurface{qʰæ˧pɤ˧}\newline
\classe{名词}\ton{\#H}
\paradigme{\pcmn{:} \p{}}
\begin{définition}\peng{Excrements, dung, dropping.}\end{définition}
\begin{définition}\pcmn{屎}\end{définition}
\begin{définition}\pfra{Crotte.}\end{définition}
\begin{exemple}\pnru{qʰæ˧pɤ˧ | ɖɯ˧-pɤ˧ ʂe˧˥}\hspace{5pt}\peng{to defecate}\hspace{5pt}\pcmn{拉一泡屎}\hspace{5pt}\pfra{faire une crotte}\end{exemple}
\begin{exemple}\pnru{qʰæ˧pɤ˧ tʰi˧-ʂe˧˥}\hspace{5pt}\peng{to defecate}\hspace{5pt}\pcmn{拉一泡屎}\hspace{5pt}\pfra{faire une crotte}\end{exemple}
\end{entrée}

\begin{entrée}
{qʰæ˧tv̩˧}{}{ⓔqʰæ˧tv̩˧}\formedesurface{qʰæ˧tv̩˧}\newline
\classe{名词}\ton{M}
\paradigme{\pcmn{:} \p{}}
\begin{définition}\peng{Anus.}\end{définition}
\begin{définition}\pcmn{肛门}\end{définition}
\begin{définition}\pfra{Anus.}\end{définition}
\end{entrée}

\begin{entrée}
{qʰæ˧tɕʰi˧}{}{ⓔqʰæ˧tɕʰi˧}\formedesurface{qʰæ˧tɕʰi˧}\newline
\classe{名词}\ton{M}\begin{définition}\peng{A village of Yongning; Chinese name: Kaiji.}\end{définition}
\begin{définition}\pcmn{开基(永宁的一个村落)}\end{définition}
\begin{définition}\pfra{Un village de Yongning; nom chinois: Kaiji.}\end{définition}
\begin{exemple}\pnru{ʈʂʰɯ˧ | qʰæ˧tɕʰi˧-hĩ˧ ɲi˥!}\hspace{5pt}\peng{(S)he is from the village of Kaiji!}\hspace{5pt}\pcmn{他是开基村人!}\hspace{5pt}\pfra{c'est quelqu'un de Kaiji!}\end{exemple}
\begin{exemple}\pnru{dʑɤ˩bv̩˧kɤ˧-sɑ˥ʁwɤ˩, | hi˩ʁwɤ˩-lo˥, | æ˩mi˧-ʁwɤ\#˥, | lɑ˧lo˧-ʁwɤ˥, | lɑ˧ŋwɤ˧, | bɤ˧tsʰo˧gv̩˥, | ə˧lɑ˧-ʁwɤ\#˥, | gæ˧ɻæ˩, | qʰæ˧tɕʰi˧, | tʰo˧ʈɯ\#˥}\hspace{5pt}\peng{The ten Na villages considered in traditional geography as belonging to the vicinity of the Yongning temple.}\hspace{5pt}\pcmn{永宁摩梭地理概念中,距离扎美寺最近的十个村落:佳部嘎萨瓦、习瓦洛、阿咪瓦、拉洛瓦、拉瓦、巴搓古、阿拉瓦、嘎尔、开基、拖支。}\hspace{5pt}\pfra{Les dix villages na traditionnellement considérés comme appartenant au voisinage du temple de Yongning.}\end{exemple}
\end{entrée}

\begin{entrée}
{qʰæ˧tɕʰi˧-ɬi˧dʑɯ˩}{}{ⓔqʰæ˧tɕʰi˧-ɬi˧dʑɯ˩}\formedesurface{qʰæ˧tɕʰi˧ɬi˧dʑɯ˩}\newline
\classe{名词}\ton{-L\#}\begin{définition}\peng{The river which runs through the plain of Yongning.}\end{définition}
\begin{définition}\pcmn{永宁坝的河流}\end{définition}
\begin{définition}\pfra{Nom de la rivière qui traverse la plaine de Yongning.}\end{définition}
\end{entrée}

\begin{entrée}
{qʰæ˧ʈæ˧˥}{}{ⓔqʰæ˧ʈæ˧˥}\formedesurface{qʰæ˧ʈæ˧˥}\newline
\classe{形容词}\ton{MH\#}
\étymologie{
qʰæ˩a 1
}\begin{définition}\peng{Quiet, at peace.}\end{définition}
\begin{définition}\pcmn{安静}\end{définition}
\begin{définition}\pfra{Paisible, tranquille (personnalité, trait de caractère).}\end{définition}
\begin{exemple}\pnru{qʰæ˧ʈæ˧˥ | tʰi˧-dzi˩}\hspace{5pt}\peng{to sit quietly, free from toil and care}\hspace{5pt}\pcmn{安静地坐着}\hspace{5pt}\pfra{être assis tranquillement, être tranquille, avoir l'esprit libre}\end{exemple}
\begin{exemple}\pnru{qʰæ˧ʈæ˧˥ | tʰi˧-ʝi˧}\hspace{5pt}\peng{to work quietly}\hspace{5pt}\pcmn{安静地工作}\hspace{5pt}\pfra{travailler paisiblement}\end{exemple}
\end{entrée}

\begin{entrée}
{qʰæ˩ʈv̩˩ɻæ˥}{}{ⓔqʰæ˩ʈv̩˩ɻæ˥}\formedesurface{qʰæ˩ʈv̩˩ɻæ˥}\newline
\classe{形容词}\ton{L+H\#}\begin{définition}\peng{Quiet, peaceful.}\end{définition}
\begin{définition}\pcmn{安宁}\end{définition}
\begin{définition}\pfra{Serein.}\end{définition}
\begin{exemple}\pnru{qʰæ˩ʈv̩˩ɻæ˥ | ɖɯ˧-dzi˩-ɻ̍˩}\hspace{5pt}\peng{to sit quietly for a while}\hspace{5pt}\pcmn{安静地坐一会}\hspace{5pt}\pfra{être assis tranquille, dans le calme}\end{exemple}
\begin{exemple}\pnru{qʰæ˩ʈv̩˩ɻæ˥-gv̩˩}\hspace{5pt}\peng{peacefully}\hspace{5pt}\pcmn{安宁地}\hspace{5pt}\pfra{tranquillement}\end{exemple}
\begin{exemple}\pnru{ʈʂʰwɤ˧ | ʈʂʰɯ˧-tɕʰi˧˥, | qʰæ˩ʈv̩˩ɻæ˥ dzɯ˩!}\hspace{5pt}\peng{The evening meal is a meal that one enjoys peacefully! (Explanation: unlike lunch, which is eaten quickly before going back to work, the evening meal is taken in a relaxed setting, enjoying more abundant food than at lunch, and quietly transitioning into rest time.)}\hspace{5pt}\pcmn{晚餐这顿,吃得很安静!(说明:中午饭,吃得快然后继续干活。晚餐,吃得很安静,慢慢吃多一点,慢慢进入晚上休息时间。)}\hspace{5pt}\pfra{Le dîner, c'est un repas qu'on mange sereinement, bien à son aise! (Explication: à la différence du déjeuner, pris rapidement avant de retourner au travail, le repas du soir est pris en famille, tranquillement; on mange plus qu'à midi; on passe doucement au temps du repos et de la nuit.)}\end{exemple}
\end{entrée}

\begin{entrée}
{qʰæ˧zo\#˥}{}{ⓔqʰæ˧zo\#˥}\formedesurface{qʰæ˧zo˧}\newline
\classe{名词}\ton{\#H}
\paradigme{\pcmn{:} \p{}}
\begin{définition}\peng{Small trench/canal.}\end{définition}
\begin{définition}\pcmn{小水渠}\end{définition}
\begin{définition}\pfra{Petit canal.}\end{définition}
\end{entrée}

\begin{entrée}
{qʰo˧˥}{₁}{ⓔqʰo˧˥ⓗ1}\formedesurface{qʰo˧˥}\newline
\classe{动词}\ton{MH}
1\begin{définition}\peng{To peck.}\end{définition}
\begin{définition}\pcmn{啄}\end{définition}
\begin{définition}\pfra{Picorer.}\end{définition}
\begin{exemple}\pnru{hɑ˧ qʰo˩(-ze˩)}\hspace{5pt}\peng{to peck cereals}\hspace{5pt}\pcmn{啄粮食}\hspace{5pt}\pfra{picorer des céréales}\end{exemple}
\begin{exemple}\pnru{hɑ˧ qʰo˥∼qʰo˩ (-dʑo˩)}\hspace{5pt}\peng{to peck cereals}\hspace{5pt}\pcmn{啄粮食}\hspace{5pt}\pfra{picorer des céréales}\end{exemple}
\begin{exemple}\pnru{æ˩-ɳɯ˥ | hɑ˧ qʰo˩}\hspace{5pt}\peng{the chicken is pecking cereals}\hspace{5pt}\pcmn{鸡在啄粮食}\hspace{5pt}\pfra{la poule picore}\end{exemple}
\end{entrée}

\begin{entrée}
{qʰo˧˥}{₂}{ⓔqʰo˧˥ⓗ2}\formedesurface{qʰo˧˥}\newline
\classe{动词}\ton{MH}
2\begin{définition}\peng{To kill; to slaughter (an animal).}\end{définition}
\begin{définition}\pcmn{杀,宰牲畜}\end{définition}
\begin{définition}\pfra{Tuer; abattre un animal.}\end{définition}
\begin{exemple}\pnru{bo˩ qʰo˧˥ / bo˩ qʰo˧-ze˥}\hspace{5pt}\peng{to slaughter a pig}\hspace{5pt}\pcmn{杀猪}\hspace{5pt}\pfra{tuer le cochon}\end{exemple}
\begin{exemple}\pnru{bo˩˥ | le˧-qʰo˧-ze˥}\hspace{5pt}\peng{the pig has been slaughtered}\hspace{5pt}\pcmn{杀了猪}\hspace{5pt}\pfra{le cochon a été abattu}\end{exemple}
\begin{exemple}\pnru{æ˩ qʰo˧˥}\hspace{5pt}\peng{to kill a chicken}\hspace{5pt}\pcmn{杀鸡}\hspace{5pt}\pfra{tuer un poulet}\end{exemple}
\begin{exemple}\pnru{ʝi˧ qʰo˩}\hspace{5pt}\peng{to kill a cow}\hspace{5pt}\pcmn{杀牛}\hspace{5pt}\pfra{tuer une vache}\end{exemple}
\end{entrée}

\begin{entrée}
{qʰo˧α}{}{ⓔqʰo˧α}\formedesurface{qʰo˧}\newline
\classe{动词}\ton{Mα}\begin{définition}\peng{To pile up (e.g. stones).}\end{définition}
\begin{définition}\pcmn{堆起来}\end{définition}
\begin{définition}\pfra{Empiler (par exemple des pierres).}\end{définition}
\begin{exemple}\pnru{lv̩˧mi˧ tʰi˧-qʰo˧}\hspace{5pt}\peng{to pile up stones}\hspace{5pt}\pcmn{石头堆起来}\hspace{5pt}\pfra{empiler des pierres}\end{exemple}
\end{entrée}

\begin{entrée}
{qʰo˩β}{}{ⓔqʰo˩β}\formedesurface{qʰo˩˥}\newline
\classe{动词}\ton{Lβ}\begin{définition}\peng{To invite, to treat.}\end{définition}
\begin{définition}\pcmn{邀请、请}\end{définition}
\begin{définition}\pfra{Inviter.}\end{définition}
\begin{exemple}\pnru{hĩ˧ qʰo˧˥}\hspace{5pt}\peng{to invite someone}\hspace{5pt}\pcmn{邀请人}\hspace{5pt}\pfra{inviter quelqu'un}\end{exemple}
\begin{exemple}\pnru{hĩ˧bæ˧ qʰo˧˥}\hspace{5pt}\peng{to invite a guest}\hspace{5pt}\pcmn{邀请客人}\hspace{5pt}\pfra{inviter un hôte, convier un invité}\end{exemple}
\begin{exemple}\pnru{hĩ˧bæ˧ | qʰo˧-zo˧-ho˥}\hspace{5pt}\peng{We should invite guests!}\hspace{5pt}\pcmn{需要请一下客人!}\hspace{5pt}\pfra{Il va falloir inviter des hôtes!}\end{exemple}
\begin{exemple}\pnru{hĩ˧bæ˧ qʰo˧-di˧˥}\hspace{5pt}\peng{Euphemism for ‘rat poison'. This phrase is intended not to attract the mice's attention to these preparations.}\hspace{5pt}\pcmn{‘待客的东西’(老鼠药的委婉语。如果说出来要买老鼠药,老鼠会知道,就不会吃的。)}\hspace{5pt}\pfra{Euphémisme pour désigner la mort-aux-rats. La croyance veut que si on expose clairement le projet, les rats vont se méfier et ne prendront pas cette nourriture empoisonnée.}\end{exemple}
\begin{exemple}\pnru{ɖɯ˧-qʰo˥∼qʰo˩-ɻ̍˩}\hspace{5pt}\peng{|fg{delimitative} |fg{red} |fg{inceptive}}\hspace{5pt}\pcmn{|fg{delimitative} |fg{red} |fg{inceptive}:请一下}\hspace{5pt}\pfra{|fg{délimitative} |fg{red} |fg{inchoatif}}\end{exemple}
\begin{exemple}\pnru{qʰo˩-mɤ˥-qʰo˩}\hspace{5pt}\peng{to invite or not}\hspace{5pt}\pcmn{请不请}\hspace{5pt}\pfra{inviter ou ne pas inviter}\end{exemple}
\begin{exemple}\pnru{qʰo˩-mɤ˩-ho˥}\hspace{5pt}\peng{…will not invite}\hspace{5pt}\pcmn{不请了 / 不要请了}\hspace{5pt}\pfra{… ne vais pas inviter}\end{exemple}
\end{entrée}

\begin{entrée}
{qʰo˩dv̩˧˥}{}{ⓔqʰo˩dv̩˧˥}\formedesurface{qʰo˩dv̩˧˥}\newline
\classe{名词}\ton{LM+MH\#}
\paradigme{\pcmn{:} \p{}}
\begin{définition}\peng{Hammer; typically a large wood hammer.}\end{définition}
\begin{définition}\pcmn{大锤子}\end{définition}
\begin{définition}\pfra{Masse, marteau de grande taille; typiquement: masse en bois utilisée pour défoncer les grosses mottes après les labours.}\end{définition}
\begin{exemple}\pnru{ʂe˩-qʰo˩dv̩˧˥}\hspace{5pt}\peng{iron hammer}\hspace{5pt}\pcmn{铁锤子}\hspace{5pt}\pfra{masse en fer}\end{exemple}
\end{entrée}

\begin{entrée}
{qʰo˧lo˧}{}{ⓔqʰo˧lo˧}\formedesurface{qʰo˧lo˧}\newline
\classe{名词}\ton{M}
\paradigme{\pcmn{:} \p{}}
\begin{définition}\peng{Wheel.}\end{définition}
\begin{définition}\pcmn{轮子}\end{définition}
\begin{définition}\pfra{Roue.}\end{définition}
\end{entrée}

\begin{entrée}
{qʰo˧mo˥}{}{ⓔqʰo˧mo˥}\formedesurface{qʰo˧mo˥}\newline
\classe{名词}\ton{H\#}
\paradigme{\pcmn{:} \p{}}
\begin{définition}\peng{Old cow (which does not give milk anymore).}\end{définition}
\begin{définition}\pcmn{老牛(不产奶了)}\end{définition}
\begin{définition}\pfra{Vieille vache (qui n'a plus de lait).}\end{définition}
\end{entrée}

\begin{entrée}
{qʰo˩mv̩˩}{}{ⓔqʰo˩mv̩˩}\formedesurface{qʰo˩mv̩˩˥}\newline
\classe{名词}\ton{L}
\paradigme{\pcmn{:} \p{}}
\begin{définition}\peng{Straw hat.}\end{définition}
\begin{définition}\pcmn{斗笠}\end{définition}
\begin{définition}\pfra{Chapeau de paille.}\end{définition}
\end{entrée}

\begin{entrée}
{qʰo˩tv̩˧˥}{}{ⓔqʰo˩tv̩˧˥}\formedesurface{qʰo˩tv̩˧˥}\newline
\classe{名词}\ton{LM+MH\#}
\paradigme{\pcmn{:} \p{}}
\begin{définition}\peng{Tree stump.}\end{définition}
\begin{définition}\pcmn{树墩、树桩}\end{définition}
\begin{définition}\pfra{Souche.}\end{définition}
\end{entrée}

\begin{entrée}
{qʰv̩˥}{}{ⓔqʰv̩˥}\formedesurface{qʰv̩˧}\newline
\classe{名词}\ton{\#H}
\paradigme{\pcmn{:} \p{}}
\begin{définition}\peng{Noise, sound.}\end{définition}
\begin{définition}\pcmn{声音}\end{définition}
\begin{définition}\pfra{Son, bruit.}\end{définition}
\begin{exemple}\pnru{ʈʂʰɯ˧ | ə˧tso˧ qʰv̩˧ ɲi˥?}\hspace{5pt}\peng{What is this sound?}\hspace{5pt}\pcmn{这是什么声音?}\hspace{5pt}\pfra{c'est quoi ce bruit?}\end{exemple}
\end{entrée}

\begin{entrée}
{qʰv̩˥α}{}{ⓔqʰv̩˥α}\formedesurface{ɖɯ˧ qʰv̩˥}\newline
\classe{量词}\ton{Hα}\begin{définition}\peng{Classifier for hamlets / small villages.}\end{définition}
\begin{définition}\pcmn{量词:村落}\end{définition}
\begin{définition}\pfra{Classificateur des hameaux.}\end{définition}
\begin{exemple}\pnru{ŋwɤ˧-qʰv̩˧, | tsʰe˧ɲi˧-ʑi˩}\hspace{5pt}\peng{Five hamlets, twelve families! (This formula summarizes the statistics of the village of /ə˧lɑ˧-ʁwɤ\#˥/)}\hspace{5pt}\pcmn{五个村落,十二个家庭!(描写阿拉瓦村的情况)}\hspace{5pt}\pfra{Cinq hameaux, douze familles! (Formule résumant la statistique du village de /ə˧lɑ˧-ʁwɤ\#˥/)}\end{exemple}
\end{entrée}

\begin{entrée}
{qʰv̩˧}{₁}{ⓔqʰv̩˧ⓗ1}\newline
\classe{名词}
1
\sens{1}\paradigme{\pcmn{:} \p{}}
\begin{définition}\peng{Hole.}\end{définition}
\begin{définition}\pcmn{洞}\end{définition}
\begin{définition}\pfra{Trou.}\end{définition}\sens{2}
\begin{définition}\peng{Burrow.}\end{définition}
\begin{définition}\pcmn{野兽的洞穴、野兽的窝}\end{définition}
\begin{définition}\pfra{Terrier.}\end{définition}
\begin{exemple}\pnru{ɖɤ˧-qʰv̩˧}\hspace{5pt}\peng{fox burrow}\hspace{5pt}\pcmn{狐狸的窝}\hspace{5pt}\pfra{terrier de renard}\hspace{5pt}\pcmn{tone: M}\end{exemple}
\begin{exemple}\pnru{ʂwæ˧ qʰv̩˧}\hspace{5pt}\peng{otter's burrow}\hspace{5pt}\pcmn{水獭的窝}\hspace{5pt}\pfra{terrier de loutre}\hspace{5pt}\pcmn{tone: M}\end{exemple}
\end{entrée}

\begin{entrée}
{qʰv̩˧}{₂}{ⓔqʰv̩˧ⓗ2}\formedesurface{qʰv̩˧}\newline
\classe{名词}\ton{M}
2
\paradigme{\pcmn{:} \p{}}
\begin{définition}\peng{Horn.}\end{définition}
\begin{définition}\pcmn{犄角(锯下来的)}\end{définition}
\begin{définition}\pfra{Corne.}\end{définition}
\begin{exemple}\pnru{ʝi˧-qʰv̩\#˥}\hspace{5pt}\peng{Ox horn. Ox horns are used as containers for drinking.}\hspace{5pt}\pcmn{牛角(过去,用牛角来当饮料容器)}\hspace{5pt}\pfra{Corne de boeuf. La corne de boeuf était autrefois utilisée comme récipient pour boissons.}\end{exemple}
\begin{exemple}\pnru{ʈʂʰæ˧-qʰv̩˥}\hspace{5pt}\peng{stag horn}\hspace{5pt}\pcmn{鹿角}\hspace{5pt}\pfra{corne de cerf}\end{exemple}
\end{entrée}

\begin{entrée}
{qʰv̩˧˥}{₁}{ⓔqʰv̩˧˥ⓗ1}\formedesurface{qʰv̩˧˥}\newline
\classe{动词}\ton{MH}
1\begin{définition}\peng{To huddle up, to curl up.}\end{définition}
\begin{définition}\pcmn{蜷缩}\end{définition}
\begin{définition}\pfra{Se recroqueviller.}\end{définition}
\begin{exemple}\pnru{ɲi˧-qʰv̩˧˥ | tʰi˧-dzi˩}\hspace{5pt}\peng{to be seated, leaning forward, torso bent towards the thighs}\hspace{5pt}\pcmn{坐着身体缩成一团}\hspace{5pt}\pfra{être assis penché en avant, le torse penché vers les cuisses}\end{exemple}
\begin{exemple}\pnru{ɲi˧-qʰv̩˧-ʝi˥ | tʰi˧-dzi˩}\hspace{5pt}\peng{to be seated, leaning forward, torso bent towards the thighs}\hspace{5pt}\pcmn{坐着身体缩成一团}\hspace{5pt}\pfra{être assis penché en avant, le torse penché vers les cuisses}\end{exemple}
\end{entrée}

\begin{entrée}
{qʰv̩˧˥}{₂}{ⓔqʰv̩˧˥ⓗ2}\formedesurface{qʰv̩˧˥}\newline
\classe{数词}\ton{MH}
2\begin{définition}\peng{Six.}\end{définition}
\begin{définition}\pcmn{六}\end{définition}
\begin{définition}\pfra{Six.}\end{définition}
\end{entrée}

\begin{entrée}
{qʰv̩˧dʑɯ˥\$}{}{ⓔqʰv̩˧dʑɯ˥\$}\formedesurface{qʰv̩˧dʑɯ˥}\newline
\classe{名词}\ton{H\$}
\paradigme{\pcmn{:} \p{}}
\begin{définition}\peng{Hole, cavity (e.g. mouse hole, or trap to catch large animals).}\end{définition}
\begin{définition}\pcmn{窟窿}\end{définition}
\begin{définition}\pfra{Cavité, trou (ex.: trou de souris; ou piège où on fait tomber les animaux sauvages).}\end{définition}
\begin{exemple}\pnru{hwæ˧tsɯ˥-qʰv̩˩dʑi˩}\hspace{5pt}\peng{mousehole}\hspace{5pt}\pcmn{耗子洞}\hspace{5pt}\pfra{trou de souris}\end{exemple}
\begin{exemple}\pnru{qʰv̩˧dʑɯ˧ tsʰi˧ (-ze˩)}\hspace{5pt}\peng{to bore a hole}\hspace{5pt}\pcmn{挖一个洞}\hspace{5pt}\pfra{percer un trou}\end{exemple}
\end{entrée}

\begin{entrée}
{qʰv̩˩ɖɯ˩}{}{ⓔqʰv̩˩ɖɯ˩}\formedesurface{qʰv̩˩ɖɯ˩˥}\newline
\classe{名词}\ton{L}\begin{définition}\peng{Attachment (to someone): found in the phrase ‘to be attached to someone, to care for someone'.}\end{définition}
\begin{définition}\pcmn{关心}\end{définition}
\begin{définition}\pfra{Attachement envers quelqu'un; observé seulement dans l'expression ‘être attaché à, faire cas de, attacher du prix à’.}\end{définition}
\begin{exemple}\pnru{qʰv̩˩ɖɯ˩ pʰv̩˥}\hspace{5pt}\peng{to care for someone, to respect, to feel attachment to someone}\hspace{5pt}\pcmn{关心(一个人),重视(如:孩子重视父母)}\hspace{5pt}\pfra{attacher de l'importance à, respecter, être attaché à (ex.: relation des enfants à leurs parents)}\end{exemple}
\end{entrée}

\begin{entrée}
{qʰv̩˩ɖʐæ˩}{}{ⓔqʰv̩˩ɖʐæ˩}\formedesurface{qʰv̩˩ɖʐæ˩˥}\newline
\classe{名词}\ton{L}
\paradigme{\pcmn{:} \p{}}
\begin{définition}\peng{String; small rope.}\end{définition}
\begin{définition}\pcmn{小绳子,细的绳子}\end{définition}
\begin{définition}\pfra{Cordelette, ficelle.}\end{définition}
\begin{exemple}\pnru{qʰv̩˩ɖʐæ˩ ʈʂʰɯ˩-kʰɯ˥}\hspace{5pt}\peng{|fg{n}+|fg{dem}+|fg{clf}}\hspace{5pt}\pcmn{一条细的绳子}\hspace{5pt}\pfra{|fg{n}+|fg{dem}+|fg{clf}}\end{exemple}
\end{entrée}

\begin{entrée}
{qʰv̩˧ɬi˧mi\#˥}{}{ⓔqʰv̩˧ɬi˧mi\#˥}\formedesurface{qʰv̩˧ɬi˧mi˧}\newline
\classe{名词}\ton{\#H}\begin{définition}\peng{6th month.}\end{définition}
\begin{définition}\pcmn{六月}\end{définition}
\begin{définition}\pfra{6e mois.}\end{définition}
\end{entrée}

\begin{entrée}
{qʰv̩˩∼qʰv̩˧˥}{}{ⓔqʰv̩˩∼qʰv̩˧˥}\formedesurface{qʰv̩˩qʰv̩˧˥}\newline
\classe{动词}\ton{MH}\begin{définition}\peng{To fold (clothes).}\end{définition}
\begin{définition}\pcmn{折叠、裹起来}\end{définition}
\begin{définition}\pfra{Plier (vêtements).}\end{définition}
\begin{exemple}\pnru{qʰv̩˩∼qʰv̩˧-ze˥}\hspace{5pt}\peng{|fg{pfv}}\hspace{5pt}\pcmn{折起来了}\hspace{5pt}\pfra{|fg{pfv}}\end{exemple}
\begin{exemple}\pnru{le˧-qʰv̩˩∼qʰv̩˩}\hspace{5pt}\peng{|fg{accomp}}\hspace{5pt}\pcmn{|fg{accomp}}\hspace{5pt}\pfra{|fg{accomp}}\end{exemple}
\end{entrée}

\begin{entrée}
{qʰv̩˧tʰv̩\#˥}{}{ⓔqʰv̩˧tʰv̩\#˥}\formedesurface{ɖɯ˧ qʰv̩˧tʰv̩˧}\newline
\classe{量词}\ton{\#H}\begin{définition}\peng{Classifier: a hornful. The quantity of liquid (or powder) that can be contained in an ox's horn. Ox horns used to serve as containers for water.}\end{définition}
\begin{définition}\pcmn{量词:一个牛角的容量}\end{définition}
\begin{définition}\pfra{Classificateur: quantité de liquide (ou de poudre) que tient une corne de vache.}\end{définition}
\begin{exemple}\pnru{ɖɯ˧-qʰv̩˧tʰv̩\#˥, | ɲi˧-qʰv̩˧tʰv̩\#˥, | so˩-qʰv̩˩tʰv̩˩˥, | ʐv̩˧-qʰv̩˧tʰv̩\#˥, | ŋwɤ˧-qʰv̩˧tʰv̩\#˥, | qʰv̩˧-qʰv̩˧tʰv̩\#˥, | ʂɯ˧-qʰv̩˧tʰv̩\#˥, | hõ˧-qʰv̩˧tʰv̩\#˥, | gv̩˧-qʰv̩˧tʰv̩\#˥, | tsʰe˩-qʰv̩˩tʰv̩˩˥}\hspace{5pt}\peng{association with numerals from 1 to 10}\hspace{5pt}\pcmn{与数词结合,一至十}\hspace{5pt}\pfra{association avec des numéraux, de 1 à 10}\end{exemple}
\end{entrée}

\begin{entrée}
{qʰv̩˧tʰv̩\#˥}{}{ⓔqʰv̩˧tʰv̩\#˥}\formedesurface{qʰv̩˧tʰv̩˧}\newline
\classe{名词}\ton{\#H}
\paradigme{\pcmn{:} \p{}}
\begin{définition}\peng{Horn.}\end{définition}
\begin{définition}\pcmn{(牛)角}\end{définition}
\begin{définition}\pfra{Corne (de vache).}\end{définition}
\begin{exemple}\pnru{qʰv˧tʰv˥ | ɖɯ˧-ɭɯ˧}\hspace{5pt}\peng{a horn}\hspace{5pt}\pcmn{一个角}\hspace{5pt}\pfra{une corne}\end{exemple}
\begin{exemple}\pnru{qʰv˧tʰv˧ ɲi˥}\hspace{5pt}\peng{It's a horn.}\hspace{5pt}\pcmn{是(牛)角。}\hspace{5pt}\pfra{C'est une corne.}\end{exemple}
\end{entrée}

\begin{entrée}
{qʰv̩˩tsʰi˧˥}{}{ⓔqʰv̩˩tsʰi˧˥}\formedesurface{qʰv̩˩tsʰi˧˥}\newline
\classe{数词}\ton{LM+MH\#}\begin{définition}\peng{60.}\end{définition}
\begin{définition}\pcmn{60}\end{définition}
\begin{définition}\pfra{60.}\end{définition}
\end{entrée}

\begin{entrée}
{qʰwæ˧}{}{ⓔqʰwæ˧}\formedesurface{qʰwæ˧}\newline
\classe{名词}\ton{M}
\paradigme{\pcmn{:} \p{}}
\begin{définition}\peng{Message (monosyllable).}\end{définition}
\begin{définition}\pcmn{信息,信}\end{définition}
\begin{définition}\pfra{Lettre, message, parole/récit.}\end{définition}
\begin{exemple}\pnru{qʰwæ˧ po˧˥}\hspace{5pt}\peng{to carry a letter; to convey a message}\hspace{5pt}\pcmn{带信息、传信息,传一封信}\hspace{5pt}\pfra{envoyer une lettre}\end{exemple}
\begin{exemple}\pnru{qʰwæ˧ kʰwɤ˧˥}\hspace{5pt}\peng{to be in touch (with someone)}\hspace{5pt}\pcmn{互相通信息、有联系(两个人互相通信息)}\hspace{5pt}\pfra{être en contact, être en correspondance; être en relation}\end{exemple}
\begin{exemple}\pnru{dɑ˧pɤ˧-qʰwæ\#˥}\hspace{5pt}\peng{the tales of the /dɑ˧pɤ˧/ priests}\hspace{5pt}\pcmn{达巴的故事}\hspace{5pt}\pfra{les récits des prêtres /dɑ˧pɤ˧/}\end{exemple}
\end{entrée}

\begin{entrée}
{qʰwæ˧˥}{₁}{ⓔqʰwæ˧˥ⓗ1}\formedesurface{qʰwæ˧˥}\newline
\classe{动词}\ton{MH}
1\begin{définition}\peng{To break (bowl, jar), to crack (nuts).}\end{définition}
\begin{définition}\pcmn{弄碎}\end{définition}
\begin{définition}\pfra{Briser (verre, vaisselle…), faire éclater; casser (des noix).}\end{définition}
\begin{exemple}\pnru{ʁo˧do˧ qʰwæ˧˥}\hspace{5pt}\peng{to crack walnuts}\hspace{5pt}\pcmn{敲开坚果(在永宁,不用夹子:用锤子敲开)}\hspace{5pt}\pfra{casser des noix}\end{exemple}
\end{entrée}

\begin{entrée}
{qʰwæ˧˥}{₂}{ⓔqʰwæ˧˥ⓗ2}\formedesurface{qʰwæ˧˥}\newline
\classe{动词}\ton{MH}
2\begin{définition}\peng{To slap.}\end{définition}
\begin{définition}\pcmn{掴、打}\end{définition}
\begin{définition}\pfra{Gifler.}\end{définition}
\begin{exemple}\pnru{le˧-qʰwæ˧-ze˥}\hspace{5pt}\peng{|fg{accomp} \_ |fg{pfv}}\hspace{5pt}\pcmn{掴了}\hspace{5pt}\pfra{|fg{accomp} \_ |fg{pfv}}\end{exemple}
\begin{exemple}\pnru{zɯ˧ɻ̍˧ qʰwæ˩}\hspace{5pt}\peng{to slap/smack someone's cheek}\hspace{5pt}\pcmn{打嘴巴}\hspace{5pt}\pfra{gifler}\end{exemple}
\begin{exemple}\pnru{zɯ˧ɻ̍˧ | ɖɯ˧-ɭɯ˧ | tʰi˧-qʰwæ˧-bi˥!}\hspace{5pt}\peng{I'm going to slap your cheek! (Said by an adult to a child)}\hspace{5pt}\pcmn{我要打嘴巴了!(对孩子说)}\hspace{5pt}\pfra{Je vais te gifler! / Je vais te flanquer une gifle! (A un enfant)}\end{exemple}
\end{entrée}

\begin{entrée}
{qʰwæ˧˥α}{}{ⓔqʰwæ˧˥α}\formedesurface{ɖɯ˧ qʰwæ˧˥}\newline
\classe{量词}\ton{MHα}\begin{définition}\peng{Classifier for filaments of hemp before spinning.}\end{définition}
\begin{définition}\pcmn{量词:丝,如纺之前的麻丝(一根)}\end{définition}
\begin{définition}\pfra{Classificateur des filaments de chanvre avant filage.}\end{définition}
\end{entrée}

\begin{entrée}
{qʰwæ˩}{}{ⓔqʰwæ˩}\formedesurface{qʰwæ˩}\newline
\classe{名词}\ton{L}
\paradigme{\pcmn{:} \p{}}
\begin{définition}\peng{Fence, made of bamboo or of thorny shrub branches.}\end{définition}
\begin{définition}\pcmn{篱笆}\end{définition}
\begin{définition}\pfra{Haie, faite de bambou ou de broussailles épineuses.}\end{définition}
\end{entrée}

\begin{entrée}
{qʰwæ˩α}{}{ⓔqʰwæ˩α}\formedesurface{qʰwæ˩˥}\newline
\classe{动词}\ton{Lα}\begin{définition}\peng{To block.}\end{définition}
\begin{définition}\pcmn{挡住}\end{définition}
\begin{définition}\pfra{Bloquer.}\end{définition}
\end{entrée}

\begin{entrée}
{qʰwæ˩kɤ˩}{}{ⓔqʰwæ˩kɤ˩}\formedesurface{qʰwæ˩kɤ˩˥}\newline
\classe{名词}\ton{L}
\paradigme{\pcmn{:} \p{}}
\begin{définition}\peng{A sort of shrub, reaching 1.5 to 2 meters in height.}\end{définition}
\begin{définition}\pcmn{一种灌木,1.5至2米高,可以当篱笆用}\end{définition}
\begin{définition}\pfra{Une sorte d'arbuste d'environ 1 mètre 50 à 2 mètres de haut.}\end{définition}
\begin{exemple}\pnru{qʰwæ˩kɤ˩-dzi˩˥}\hspace{5pt}\peng{same meaning}\hspace{5pt}\pcmn{同上}\hspace{5pt}\pfra{même sens}\end{exemple}
\end{entrée}

\begin{entrée}
{qʰwæ˧kʰwɤ\#˥}{}{ⓔqʰwæ˧kʰwɤ\#˥}\formedesurface{qʰwæ˧kʰwɤ˧}\newline
\classe{名词}\ton{\#H}
\paradigme{\pcmn{:} \p{}}
\begin{définition}\peng{Gossip, idle chatter.}\end{définition}
\begin{définition}\pcmn{闲话、流言、蜚语、闲言碎语、八卦}\end{définition}
\begin{définition}\pfra{Récit, racontar, ragot, histoire.}\end{définition}
\begin{exemple}\pnru{ɖɯ˧-zɯ˧ qʰwæ˧kʰwɤ˧}\hspace{5pt}\peng{to tell a piece of gossip}\hspace{5pt}\pcmn{讲一点八卦}\hspace{5pt}\pfra{raconter un petit racontar, rapporter un petit ragot}\end{exemple}
\end{entrée}

\begin{entrée}
{qʰwæ˧ɭɯ˧}{}{ⓔqʰwæ˧ɭɯ˧}\formedesurface{qʰwæ˧ɭɯ˧}\newline
\classe{名词}\ton{M}
\paradigme{\pcmn{:} \p{}}
\begin{définition}\peng{Vegetable garden.}\end{définition}
\begin{définition}\pcmn{菜园}\end{définition}
\begin{définition}\pfra{Potager.}\end{définition}
\end{entrée}

\begin{entrée}
{qʰwæ˧mi\#˥}{}{ⓔqʰwæ˧mi\#˥}\formedesurface{qʰwæ˧mi˧}\newline
\classe{名词}\ton{\#H}
\paradigme{\pcmn{:} \p{}}
\begin{définition}\peng{Message, information (extended meaning: letter).}\end{définition}
\begin{définition}\pcmn{口信, 信息}\end{définition}
\begin{définition}\pfra{Message, information (d'où: lettre).}\end{définition}
\begin{exemple}\pnru{qʰwæ˧mi˧ ʝi˧}\hspace{5pt}\peng{to carry a message}\hspace{5pt}\pcmn{带一个口信}\hspace{5pt}\pfra{porter un message}\end{exemple}
\end{entrée}

\begin{entrée}
{qʰwæ˧ʈɯ˥}{}{ⓔqʰwæ˧ʈɯ˥}\formedesurface{qʰwæ˩ʈɯ˥}\newline
\classe{名词}\ton{H\#}
\paradigme{\pcmn{:} \p{}}
\begin{définition}\peng{Scarf, kerchief.}\end{définition}
\begin{définition}\pcmn{头帕}\end{définition}
\begin{définition}\pfra{Fichu (tissu qu'on porte sur la tête).}\end{définition}
\end{entrée}

\begin{entrée}
{qʰwɤ˧}{}{ⓔqʰwɤ˧}\formedesurface{qʰwɤ˧}\newline
\classe{名词}\ton{M}
\paradigme{\pcmn{:} \p{}}
\begin{définition}\peng{Tracks (left by an animal); spoor.}\end{définition}
\begin{définition}\pcmn{痕迹}\end{définition}
\begin{définition}\pfra{Traces, piste (d'un animal; lorsqu'on chasse, on suit la piste d'un animal, on le suit à la trace).}\end{définition}
\end{entrée}

\begin{entrée}
{qʰwɤ˧˥}{₁}{ⓔqʰwɤ˧˥ⓗ1}\formedesurface{qʰwɤ˧˥}\newline
\classe{名词}\ton{MH}
1
\paradigme{\pcmn{:} \p{}}
\begin{définition}\peng{Bowl.}\end{définition}
\begin{définition}\pcmn{碗}\end{définition}
\begin{définition}\pfra{Bol.}\end{définition}
\end{entrée}

\begin{entrée}
{qʰwɤ˧˥}{₂}{ⓔqʰwɤ˧˥ⓗ2}\formedesurface{qʰwɤ˧˥}\newline
\classe{名词}\ton{MH}
2
\paradigme{\pcmn{:} \p{}}
\begin{définition}\peng{Tale, story, yarn.}\end{définition}
\begin{définition}\pcmn{故事}\end{définition}
\begin{définition}\pfra{Histoire, récit.}\end{définition}
\begin{exemple}\pnru{æ˧ʂæ˧-qʰwɤ˧˥}\hspace{5pt}\peng{tale, folk tale}\hspace{5pt}\pcmn{老故事}\hspace{5pt}\pfra{récit d'autrefois, conte}\end{exemple}
\end{entrée}

\begin{entrée}
{qʰwɤ˧˥α}{}{ⓔqʰwɤ˧˥α}\formedesurface{ɖɯ˧ qʰwɤ˧˥}\newline
\classe{量词}\ton{MHα}\begin{définition}\peng{A bowl(ful) of.}\end{définition}
\begin{définition}\pcmn{量词:碗}\end{définition}
\begin{définition}\pfra{Classificateur des bols (utilisés comme quantité de mesure du non dénombrable).}\end{définition}
\end{entrée}

\begin{entrée}
{qʰwɤ˧α}{}{ⓔqʰwɤ˧α}\formedesurface{qʰwɤ˧}\newline
\classe{动词}\ton{Mα}\begin{définition}\peng{To heal (wound, disease, broken bone…).}\end{définition}
\begin{définition}\pcmn{治好(骨折、病)}\end{définition}
\begin{définition}\pfra{Se guérir (blessure, maladie); se rétablir (une fracture).}\end{définition}
\begin{exemple}\pnru{le˧-qʰwɤ˧-ɲi˥!}\hspace{5pt}\peng{It is healed! / It has healed!}\hspace{5pt}\pcmn{治好了!}\hspace{5pt}\pfra{C'est guéri! / La fracture est rétablie!}\end{exemple}
\end{entrée}

\begin{entrée}
{qʰwɤ˩α}{₁}{ⓔqʰwɤ˩αⓗ1}\formedesurface{qʰwɤ˩˥}\newline
\classe{形容词}\ton{Lα}
1\begin{définition}\peng{Intelligent.}\end{définition}
\begin{définition}\pcmn{聪明}\end{définition}
\begin{définition}\pfra{Intelligent.}\end{définition}
\begin{exemple}\pnru{qʰwɤ˩-hĩ˩˥}\hspace{5pt}\peng{|fg{rel}/|fg{nmlz}}\hspace{5pt}\pcmn{聪明的}\hspace{5pt}\pfra{|fg{rel}/|fg{nmlz}}\end{exemple}
\begin{exemple}\pnru{qʰwɤ˩-le˥!}\hspace{5pt}\peng{(You are/(s)he is) clever! (A comment when someone says/does something clever)}\hspace{5pt}\pcmn{很聪明! / 太聪明了!}\hspace{5pt}\pfra{(il est/tu es) intelligent! (Commentaire lorsque quelqu'un dit ou fait quelque chose d'astucieux)}\end{exemple}
\begin{exemple}\pnru{ɖwæ˧˥ | qʰwɤ˩˥!}\hspace{5pt}\peng{|fg{intensive.very} \_}\hspace{5pt}\pcmn{很聪明!}\hspace{5pt}\pfra{|fg{intensif.très} \_: très intelligent}\end{exemple}
\end{entrée}

\begin{entrée}
{qʰwɤ˩α}{₂}{ⓔqʰwɤ˩αⓗ2}\formedesurface{qʰwɤ˩˥}\newline
\classe{形容词}\ton{Lα}
2\begin{définition}\peng{Bad.}\end{définition}
\begin{définition}\pcmn{坏}\end{définition}
\begin{définition}\pfra{Mauvais.}\end{définition}
\begin{exemple}\pnru{kʰv̩˧ | qʰwɤ˩-hĩ˩˥}\hspace{5pt}\peng{a bad year (a year when crops are not good)}\hspace{5pt}\pcmn{(收成)不好的一年}\hspace{5pt}\pfra{une mauvaise année}\end{exemple}
\begin{exemple}\pnru{kʰv̩˧ qʰwɤ˧˥}\hspace{5pt}\peng{a bad year (a year when crops are not good)}\hspace{5pt}\pcmn{(收成)不好的一年}\hspace{5pt}\pfra{une mauvaise année}\end{exemple}
\begin{exemple}\pnru{tsʰi˧-ʝi˧, | kʰv̩˧qʰwɤ˧ tʰv̩˧˥!}\hspace{5pt}\peng{This year is a bad year! (=a year when crops are not good)}\hspace{5pt}\pcmn{今年,年景不好!(收成不好)}\hspace{5pt}\pfra{cette année, c'est une mauvaise année (les récoltes sont mauvaises)!}\end{exemple}
\begin{exemple}\pnru{ʈʂʰɯ˧ | nv̩˩mi˩˥ | ɖwæ˧˥ | qʰwɤ˩˥!}\hspace{5pt}\peng{He has a really bad heart! / He is a really bad man!}\hspace{5pt}\pcmn{他心很坏!}\hspace{5pt}\pfra{Il a l'âme bien noire!}\end{exemple}
\begin{exemple}\pnru{qʰwɤ˩-ʝi˩˥}\hspace{5pt}\peng{to do bad things: to damage stuff; to annoy people…}\hspace{5pt}\pcmn{干坏事:损坏东西,干扰人家……}\hspace{5pt}\pfra{faire des bêtises, faire du mal: abîmer des choses, faire des misères aux gens…}\end{exemple}
\begin{exemple}\pnru{ʈʂʰɯ˧-ɳɯ˧ | njɤ˧-bv̩˧ tso˧∼tso˧ | le˧-qʰwɤ˩-ʝi˩-ze˩!}\hspace{5pt}\peng{(S)he has damaged my stuff!}\hspace{5pt}\pcmn{他弄坏了我的东西!}\hspace{5pt}\pfra{Il a abîmé mes affaires!}\end{exemple}
\begin{exemple}\pnru{hĩ˧ qʰwɤ˧-ʝi˥}\hspace{5pt}\peng{to annoy people}\hspace{5pt}\pcmn{干扰人家、麻烦人}\hspace{5pt}\pfra{embêter les gens, faire des misères aux gens}\end{exemple}
\begin{exemple}\pnru{ʈʂʰɯ˧ | to˩to˧mi˥ hĩ˩ qʰwɤ˩-ʝi˩!}\hspace{5pt}\peng{(S)he purposedly annoys people! / (S)he annoys people on purpose!}\hspace{5pt}\pcmn{他故意麻烦人!}\hspace{5pt}\pfra{Il/elle fait exprès d'embêter les gens!}\end{exemple}
\end{entrée}

\begin{entrée}
{qʰwɤ˧bi˩}{}{ⓔqʰwɤ˧bi˩}\newline
\classe{名词}
\sens{1}\paradigme{\pcmn{:} \p{}}
\begin{définition}\peng{Hoof (of horse); foot (of dog).}\end{définition}
\begin{définition}\pcmn{马蹄}\end{définition}
\begin{définition}\pfra{Sabot, patte.}\end{définition}
\begin{exemple}\pnru{ʐwæ˧-qʰwɤ˧bi˥\#}\hspace{5pt}\peng{horse hoof}\hspace{5pt}\pcmn{马蹄、(马、狗……的)脚}\hspace{5pt}\pfra{sabot de cheval}\end{exemple}
\begin{exemple}\pnru{kʰv̩˩-qʰwɤ˩bi˥\#}\hspace{5pt}\peng{dog's foot}\hspace{5pt}\pcmn{狗脚}\hspace{5pt}\pfra{patte de chien}\end{exemple}\sens{2}
\begin{définition}\peng{Track, trail, spoor, footprints.}\end{définition}
\begin{définition}\pcmn{脚的痕迹、行径}\end{définition}
\begin{définition}\pfra{Traces, piste.}\end{définition}
\end{entrée}

\begin{entrée}
{qʰwɤ˩ɖɯ˩}{}{ⓔqʰwɤ˩ɖɯ˩}\formedesurface{qʰwɤ˩ɖɯ˩˥}\newline
\classe{名词}\ton{L}
\paradigme{\pcmn{:} \p{}}
\begin{définition}\peng{Relatives, members of the family.}\end{définition}
\begin{définition}\pcmn{亲戚}\end{définition}
\begin{définition}\pfra{Membres de la famille (étendue).}\end{définition}
\begin{exemple}\pnru{ə˧zɯ˩ | qʰwɤ˩ɖɯ˩ ɲi˥}\hspace{5pt}\peng{We two belong to the same family.}\hspace{5pt}\pcmn{咱们两个是一家人。}\hspace{5pt}\pfra{Tous deux, on est de la même famille}\end{exemple}
\begin{exemple}\pnru{qʰwɤ˩ɖɯ˩˥, | v̩˩dze˩˥}\hspace{5pt}\peng{family and friends, extended family circle}\hspace{5pt}\pcmn{亲人:泛指亲戚与亲密朋友们}\hspace{5pt}\pfra{famille et amis, cercle familial élargi}\end{exemple}
\begin{exemple}\pnru{qʰwɤ˩ɖɯ˩ to˥}\hspace{5pt}\peng{to establish family ties between two families (through marriage)}\hspace{5pt}\pcmn{建立起两个家庭之间的联系(通过婚姻)}\hspace{5pt}\pfra{établir des liens familiaux, unir deux familles (par un mariage)}\end{exemple}
\end{entrée}

\begin{entrée}
{qʰwɤ˧mi˥\$}{}{ⓔqʰwɤ˧mi˥\$}\formedesurface{qʰwɤ˧mi˥}\newline
\classe{名词}\ton{H\$}\begin{définition}\peng{Large bowl; it used to be made of wood.}\end{définition}
\begin{définition}\pcmn{大碗(以前碗是用木头做的)}\end{définition}
\begin{définition}\pfra{Grand bol (autrefois, les bols étaient en bois).}\end{définition}
\end{entrée}

\begin{entrée}
{qʰwɤ˧pɤ˥\$}{}{ⓔqʰwɤ˧pɤ˥\$}\formedesurface{qʰwɤ˧pɤ˥}\newline
\classe{名词}\ton{H\$}\begin{définition}\peng{Large bowl.}\end{définition}
\begin{définition}\pcmn{大碗}\end{définition}
\begin{définition}\pfra{Grand bol.}\end{définition}
\end{entrée}

\begin{entrée}
{qʰwɤ˧ʂe˩}{}{ⓔqʰwɤ˧ʂe˩}\formedesurface{qʰwɤ˧ʂe˩}\newline
\classe{名词}\ton{L\#}
\paradigme{\pcmn{:} \p{}}
\begin{définition}\peng{Horseshoe.}\end{définition}
\begin{définition}\pcmn{马蹄铁}\end{définition}
\begin{définition}\pfra{Fer à cheval.}\end{définition}
\begin{exemple}\pnru{ʐwæ˧-qʰwɤ˧ʂe˥ (+ɲi˩)}\hspace{5pt}\peng{horseshoe}\hspace{5pt}\pcmn{马蹄铁}\hspace{5pt}\pfra{fer à cheval}\end{exemple}
\end{entrée}

\begin{entrée}
{qʰwɤ˧to˩}{}{ⓔqʰwɤ˧to˩}\formedesurface{qʰwɤ˧to˩}\newline
\classe{名词}\ton{L\#}
\paradigme{\pcmn{:} \p{}}
\begin{définition}\peng{Tip of the shoulder.}\end{définition}
\begin{définition}\pcmn{肩膀的末端}\end{définition}
\begin{définition}\pfra{Épaule (extrémité de l'épaule, bout de l'épaule).}\end{définition}
\begin{exemple}\pnru{hĩ˧ ʈʂʰɯ˧-v̩˧, | qʰwɤ˧to˩ | ɖɯ˧-pi˧˥ | ʂwæ˧-hṽ̩˩-di˩!}\hspace{5pt}\peng{This person's shoulders are not quite straight! / His/her shoulders don't align!}\hspace{5pt}\pcmn{这个人的肩膀不正,一高一低!}\hspace{5pt}\pfra{Ce type, il a les épaules un peu de travers! / Il a une épaule plus haute que l'autre!}\end{exemple}
\end{entrée}

\begin{entrée}
{qʰwɤ˧tʰv̩\#˥}{}{ⓔqʰwɤ˧tʰv̩\#˥}\formedesurface{qʰwɤ˧tʰv̩˧}\newline
\classe{名词}\ton{\#H}
\paradigme{\pcmn{:} \p{}}
\begin{définition}\peng{Bamboo basket to carry water (on back).}\end{définition}
\begin{définition}\pcmn{竹篓}\end{définition}
\begin{définition}\pfra{Hotte en bambou pour porter de l'eau.}\end{définition}
\end{entrée}

\begin{entrée}
{qʰwɤ˧tsʰi˩}{}{ⓔqʰwɤ˧tsʰi˩}\formedesurface{qʰwɤ˧tsʰi˩}\newline
\classe{名词}\ton{L\#}
\paradigme{\pcmn{:} \p{}}
\begin{définition}\peng{Shoulder.}\end{définition}
\begin{définition}\pcmn{肩膀}\end{définition}
\begin{définition}\pfra{Épaule.}\end{définition}
\begin{exemple}\pnru{qʰwɤ˧tsʰi˩-ʁo˩ | hwæ˧pʰæ˩ | ɖɯ˧-nɑ˧-ʈʂʰɯ˧ gɤ˧˥}\hspace{5pt}\peng{to carry a hoe on the shoulder}\hspace{5pt}\pcmn{肩上扛一把锄头}\hspace{5pt}\pfra{porter une houe à l'épaule}\end{exemple}
\end{entrée}

\begin{entrée}
{qʰwɤ˧zo˥\$}{}{ⓔqʰwɤ˧zo˥\$}\formedesurface{qʰwɤ˧zo˥}\newline
\classe{名词}\ton{H\$}\begin{définition}\peng{Small bowl.}\end{définition}
\begin{définition}\pcmn{小碗}\end{définition}
\begin{définition}\pfra{Petit bol.}\end{définition}
\end{entrée}

\newpage\caractère{ɻ}

\begin{entrée}
{ɻ̃˥}{}{ⓔɻ̃˥}\formedesurface{ɻ̃˧}\newline
\classe{形容词}\ton{H}\begin{définition}\peng{Destitute, impoverished, poor; troubled, helpless.}\end{définition}
\begin{définition}\pcmn{困难、贫穷}\end{définition}
\begin{définition}\pfra{Démuni, en mauvaise passe.}\end{définition}
\begin{exemple}\pnru{le˧-ɻ̃˥-ze˩!}\hspace{5pt}\peng{[(S)he] is really poor/helpless!}\hspace{5pt}\pcmn{(他)真的很穷苦!}\hspace{5pt}\pfra{(il) est en mauvaise passe!/il est à la rue!}\end{exemple}
\begin{exemple}\pnru{le˧-ɻ̃˧-bi˧}\hspace{5pt}\peng{|fg{accomp} \_ |fg{fut\_imm}}\hspace{5pt}\pcmn{|fg{accomp} \_ |fg{fut\_imm}}\hspace{5pt}\pfra{|fg{accomp} \_ |fg{fut\_imm}}\end{exemple}
\begin{exemple}\pnru{mɤ˧-ɻ̃˥}\hspace{5pt}\peng{|fg{neg}}\hspace{5pt}\pcmn{|fg{neg}}\hspace{5pt}\pfra{|fg{neg}}\end{exemple}
\begin{exemple}\pnru{le˧-ɻ̃˧-zo˥, | ɻ̃˧-lɑ˩ bi˩-mɤ˩-dʑɯ˩!}\hspace{5pt}\peng{“Sure, we're in poverty/we're hungry, but not to the point where bones are bare!" Play on words on ‘poor, destitute' and ‘bone', which are homophonous. The proverb is used to relativize people's perceived degree of misfortune.}\hspace{5pt}\pcmn{很困难,也还没有到饿死的程度啊! / 再困难,也还没饿死!(直译:“再困难,也没有露出骨头!”这个成语,来安慰认为自己太可怜的人。)}\hspace{5pt}\pfra{«Pour démuni/mal nourri/affamé qu'on soit, on n'en est pas encore maigre au point d'avoir les os à découvert!» Jeu de mots sur ‘démuni' et ‘ossement', qui sont homophones. Le proverbe sert à relativiser le malheur ressenti par quelqu'un.}\end{exemple}
\begin{exemple}\pnru{ɻ̃˧-ʐwɤ˧˥}\hspace{5pt}\peng{to complain}\hspace{5pt}\pcmn{诉苦、抱怨}\hspace{5pt}\pfra{se plaindre}\end{exemple}
\begin{exemple}\pnru{ɻ̃˧-ʐwɤ˧ | dɑ˧-ʐwɤ˧-ɻ̍˥}\hspace{5pt}\peng{to tell one's miseries, to complain about one's fate}\hspace{5pt}\pcmn{诉苦、讲自己的不幸}\hspace{5pt}\pfra{raconter ses malheurs; se plaindre}\end{exemple}
\begin{exemple}\pnru{ʈʂʰɯ˧ | mɑ˧dɑ˩-qʰwɤ˩, | ɻ̃˧-ʐwɤ˧ | dɑ˧-ʐwɤ˧-ɻ̍˥!}\hspace{5pt}\peng{He is unhappy; he spends his time complaining / he is always complaining!}\hspace{5pt}\pcmn{他不幸福,一直在讲自己怎么可怜!}\hspace{5pt}\pfra{Il est malheureux; il passe son temps à se plaindre!}\end{exemple}
\end{entrée}

\begin{entrée}
{ɻ̃˥}{}{ⓔɻ̃˥}\formedesurface{ɻ̃˧}\newline
\classe{名词}\ton{\#H}
\paradigme{\pcmn{:} \p{}}
\begin{définition}\peng{Bone.}\end{définition}
\begin{définition}\pcmn{骨头}\end{définition}
\begin{définition}\pfra{Os, ossement.}\end{définition}
\end{entrée}

\begin{entrée}
{ɻ̃˧}{}{ⓔɻ̃˧}\formedesurface{ɻ̃˧}\newline
\classe{名词}\ton{M}
\paradigme{\pcmn{:} \p{}}
\begin{définition}\peng{Clan.}\end{définition}
\begin{définition}\pcmn{家族}\end{définition}
\begin{définition}\pfra{Clan: ensemble de familles.}\end{définition}
\begin{exemple}\pnru{ɻ̃˧ ɖɯ˧-ɻ̃˧}\hspace{5pt}\peng{one clan}\hspace{5pt}\pcmn{一个家族}\hspace{5pt}\pfra{un clan}\end{exemple}
\end{entrée}

\begin{entrée}
{ɻ̃˧β}{}{ⓔɻ̃˧β}\formedesurface{ɖɯ˧ ɻ̃˧}\newline
\classe{量词}\ton{Mβ}\begin{définition}\peng{Classifier for clans / extended families; literally ‘one bone'. This unit is located one level higher up than the ‘family community' in Fu Maoji's (1983) terminology.}\end{définition}
\begin{définition}\pcmn{量词:家族}\end{définition}
\begin{définition}\pfra{Classificateur des clans / groupes de familles: littéralement ‘un os'. Echelon supérieur à celui de la ‘communauté familiale' dans la terminologie de Fu Maoji (1983).}\end{définition}
\end{entrée}

\begin{entrée}
{=ɻæ˩}{}{ⓔ=ɻæ˩}\formedesurface{ɻæ˩˥}\newline
\classe{附着词}\ton{L}\begin{définition}\peng{Plural.}\end{définition}
\begin{définition}\pcmn{多数}\end{définition}
\begin{définition}\pfra{Pluriel.}\end{définition}
\begin{exemple}\pnru{ʈʂʰɯ˧-ɻæ˥\$}\hspace{5pt}\peng{these things, this sort of things}\hspace{5pt}\pcmn{这类的东西,……之类}\hspace{5pt}\pfra{ces choses-ci, cette sorte de choses}\end{exemple}
\end{entrée}

\begin{entrée}
{ɻæ˩˥}{₁}{ⓔɻæ˩˥ⓗ1}\formedesurface{ɻæ˩˥}\newline
\classe{名词}\ton{LH}
1
\paradigme{\pcmn{:} \p{}}
\begin{définition}\peng{Seed.}\end{définition}
\begin{définition}\pcmn{种子}\end{définition}
\begin{définition}\pfra{Graine.}\end{définition}
\end{entrée}

\begin{entrée}
{ɻæ˩˥}{₂}{ⓔɻæ˩˥ⓗ2}\formedesurface{ɻæ˩˥}\newline
\classe{名词}\ton{LH}
2
\paradigme{\pcmn{:} \p{}}
\begin{définition}\peng{Yoke (for one or two animals).}\end{définition}
\begin{définition}\pcmn{牛轭(单行或双行)}\end{définition}
\begin{définition}\pfra{Joug (le terme est le même pour un ou deux animaux).}\end{définition}
\begin{exemple}\pnru{ʝi˧-ɻæ˥}\hspace{5pt}\peng{same meaning as the monosyllabic form: yoke (literally ‘ox yoke')}\hspace{5pt}\pcmn{牛轭}\hspace{5pt}\pfra{même sens que la forme monosyllabique: joug; littéralement ‘joug de bœuf/buffle'}\end{exemple}
\begin{exemple}\pnru{ɻæ˩ ʈʂʰɯ˩-ɭɯ˥ / ɻæ˩ ʈʂʰɯ˩-ɭɯ˩˥}\hspace{5pt}\peng{|fg{n}+|fg{dem}+|fg{clf}; allows two variants}\hspace{5pt}\pcmn{这个牛轭}\hspace{5pt}\pfra{|fg{n}+|fg{dem}+|fg{clf}; cette expression possède deux variantes tonales}\end{exemple}
\end{entrée}

\begin{entrée}
{ɻæ˩α}{}{ⓔɻæ˩α}\formedesurface{ɻæ˩˥}\newline
\classe{形容词}\ton{Lα}\begin{définition}\peng{Shrivelled, flat, shrunken.}\end{définition}
\begin{définition}\pcmn{瘪}\end{définition}
\begin{définition}\pfra{Plat, de forme plate, aplati.}\end{définition}
\begin{exemple}\pnru{ɻæ˩-hĩ˩˥}\hspace{5pt}\peng{|fg{rel}/|fg{nmlz}}\hspace{5pt}\pcmn{瘪的}\hspace{5pt}\pfra{|fg{rel}/|fg{nmlz}}\end{exemple}
\begin{exemple}\pnru{ɻæ˩ti˩ɻæ˥ (-gv̩˩)}\hspace{5pt}\peng{shrivelled, flat, shrunken}\hspace{5pt}\pcmn{瘪瘪的}\hspace{5pt}\pfra{raplapla, ratatiné}\end{exemple}
\end{entrée}

\begin{entrée}
{ɻ̃˧hæ˩}{}{ⓔɻ̃˧hæ˩}\formedesurface{ɻ̃˧hæ˩}\newline
\classe{名词}\ton{L\#}
\paradigme{\pcmn{:} \p{}}
\begin{définition}\peng{Cartilage.}\end{définition}
\begin{définition}\pcmn{软骨}\end{définition}
\begin{définition}\pfra{Cartilage.}\end{définition}
\end{entrée}

\begin{entrée}
{ɻ̃˧kɤ˩}{}{ⓔɻ̃˧kɤ˩}\formedesurface{ɻ̃˧kɤ˩}\newline
\classe{名词}\ton{L\#}
\paradigme{\pcmn{:} \p{}}
\begin{définition}\peng{Backbone.}\end{définition}
\begin{définition}\pcmn{脊椎骨}\end{définition}
\begin{définition}\pfra{Colonne vertébrale.}\end{définition}
\end{entrée}

\begin{entrée}
{ɻ̃˧ko˩}{}{ⓔɻ̃˧ko˩}\formedesurface{ɻ̃˧ko˩}\newline
\classe{名词}\ton{L\#}
\paradigme{\pcmn{:} \p{}}
\begin{définition}\peng{Shinbone, tibia.}\end{définition}
\begin{définition}\pcmn{胫骨}\end{définition}
\begin{définition}\pfra{Tibia.}\end{définition}
\begin{exemple}\pnru{hĩ˧-dzɑ˧ | ɖʐe˧ tʰɑ˧-ʝi˥, | ɻ̃˧ko˩ mi˩ tʰɑ˩-tʰv̩˩. |}\hspace{5pt}\peng{The poor must not borrow money; the shinbone must not receive wounds! (Proverb, to explain that one must avoid hitting weak/sensitive spots.)}\hspace{5pt}\pcmn{“穷人莫借钱,胫骨莫受伤!”}\hspace{5pt}\pfra{«Le pauvre ne doit pas emprunter d'argent; le tibia ne doit pas recevoir de blessure!» (Ce proverbe enseigne qu'il ne faut pas toucher les points les plus sensibles, les plus fragiles.)}\end{exemple}
\end{entrée}

\begin{entrée}
{ɻ̃˧mi˧}{}{ⓔɻ̃˧mi˧}\formedesurface{ɻ̃˧mi˧}\newline
\classe{名词}\ton{M}
\paradigme{\pcmn{:} \p{}}
\begin{définition}\peng{Tree trunk.}\end{définition}
\begin{définition}\pcmn{树干}\end{définition}
\begin{définition}\pfra{Tronc.}\end{définition}
\begin{exemple}\pnru{si˧dzi˩-ɻ̃˩mi˩}\hspace{5pt}\peng{tree trunk}\hspace{5pt}\pcmn{树干}\hspace{5pt}\pfra{tronc d'arbre}\end{exemple}
\end{entrée}

\begin{entrée}
{ɻ̃˧ʈʂæ˩}{}{ⓔɻ̃˧ʈʂæ˩}\formedesurface{ɻ̃˧ʈʂæ˩}\newline
\classe{名词}\ton{L\#}
\paradigme{\pcmn{:} \p{}}
\begin{définition}\peng{Ankle, joint (between the foot and the leg, the arm and the hand…).}\end{définition}
\begin{définition}\pcmn{关节部位,关节}\end{définition}
\begin{définition}\pfra{Articulations (de la jambe: la cheville, le genou…; du bras: le poignet, le coude…).}\end{définition}
\end{entrée}

\begin{entrée}
{ɻ̃˧ʈʂwæ˩}{}{ⓔɻ̃˧ʈʂwæ˩}\formedesurface{ɻ̃˧ʈʂwæ˩}\newline
\classe{名词}\ton{L\#}\begin{définition}\peng{Sambucus, |\stylefi{Toricellia angulata Oliv.}}\end{définition}
\begin{définition}\pcmn{接骨丹}\end{définition}
\begin{définition}\pfra{Sambucus, |\stylefi{Toricellia angulata Oliv.}}\end{définition}
\begin{exemple}\pnru{ɻ̃˧ʈʂwæ˩-si˩}\hspace{5pt}\peng{same meaning}\hspace{5pt}\pcmn{同上}\hspace{5pt}\pfra{même sens}\end{exemple}
\end{entrée}

\begin{entrée}
{ɻwæ˥}{}{ⓔɻwæ˥}\newline
\classe{动词}
\sens{1}
\begin{définition}\peng{To cry (man, and animals: cat, cow, horse, donkey, chicken, lion, wolf…); to call out.}\end{définition}
\begin{définition}\pcmn{喊、吼、叫(人、猫、牛、猪、羊、狼、驴、狮子、老虎、豺狼……)}\end{définition}
\begin{définition}\pfra{Crier, hurler; miauler; braire; hennir; rugir (chat, bœuf, cochon, mouton, loup, lion).}\end{définition}
\begin{exemple}\pnru{mɤ˧-ɻwæ˥}\hspace{5pt}\peng{|fg{neg}}\hspace{5pt}\pcmn{不叫}\hspace{5pt}\pfra{|fg{neg}}\end{exemple}
\begin{exemple}\pnru{ɻwæ˧∼ɻwæ˧}\hspace{5pt}\peng{|fg{red}}\hspace{5pt}\pcmn{重叠}\hspace{5pt}\pfra{|fg{red}}\end{exemple}
\begin{exemple}\pnru{hĩ˧ ɻwæ˧-dʑo˩}\hspace{5pt}\peng{Someone is shouting}\hspace{5pt}\pcmn{有人在叫。}\hspace{5pt}\pfra{Il y a quelqu'un qui est est en train d'appeler/de crier}\end{exemple}
\begin{exemple}\pnru{ɖɯ˧-ɻwæ˧-ɻ̍˥}\hspace{5pt}\peng{to call out}\hspace{5pt}\pcmn{叫一声}\hspace{5pt}\pfra{appeler, lancer un appel}\end{exemple}
\begin{exemple}\pnru{hwɤ˧li˧ ɻwæ˥-dʑo˩}\hspace{5pt}\peng{the cat is calling/crying}\hspace{5pt}\pcmn{猫在叫}\hspace{5pt}\pfra{le chat miaule}\end{exemple}
\begin{exemple}\pnru{æ̃˩ ɻwæ˥}\hspace{5pt}\pfra{la poule caquette}\hspace{5pt}\peng{the chicken is cackling}\hspace{5pt}\pcmn{鸡在叫}\end{exemple}
\begin{exemple}\pnru{ʐwæ˧pʰæ˧di˧˥ | tʰi˧-ɻwæ˥-dʑo˩}\hspace{5pt}\peng{the donkey is braying}\hspace{5pt}\pcmn{驴在叫}\hspace{5pt}\pfra{l'âne brait}\end{exemple}
\begin{exemple}\pnru{ʐwæ˧ | tʰi˧-ɻwæ˥-dʑo˩}\hspace{5pt}\peng{the horse is whinnying}\hspace{5pt}\pcmn{马在嘶}\hspace{5pt}\pfra{le cheval est en train de hennir}\end{exemple}
\begin{exemple}\pnru{ʐwæ˧ ɻwæ˧-dʑo˩!}\hspace{5pt}\peng{the horse is whinnying}\hspace{5pt}\pcmn{马在嘶}\hspace{5pt}\pfra{le cheval est en train de hennir}\end{exemple}\sens{2}
\begin{définition}\peng{To invite, to call over.}\end{définition}
\begin{définition}\pcmn{请、叫(来)}\end{définition}
\begin{définition}\pfra{Inviter, faire venir.}\end{définition}
\begin{exemple}\pnru{ɖɯ˧-ɻwæ˧-ɻ̍˥}\hspace{5pt}\peng{|fg{delimitative} \_ |fg{inceptive}}\hspace{5pt}\pcmn{请来一下}\hspace{5pt}\pfra{|fg{délimitatif} \_ |fg{inchoatif}}\end{exemple}
\begin{exemple}\pnru{tʰɑ˧-ɻwæ˥!}\hspace{5pt}\peng{|fg{prohib}}\hspace{5pt}\pcmn{不要请!}\hspace{5pt}\pfra{|fg{prohib}}\end{exemple}
\begin{exemple}\pnru{ɻwæ˧-mɤ˧-bi˧!}\hspace{5pt}\peng{(We)'re not inviting (him/her)!}\hspace{5pt}\pcmn{不请他!}\hspace{5pt}\pfra{(on) ne l'invite pas!}\end{exemple}
\end{entrée}

\begin{entrée}
{ɻwæ˥β}{}{ⓔɻwæ˥β}\formedesurface{ɖɯ˧ ɻwæ˥}\newline
\classe{量词}\ton{Hβ}\begin{définition}\peng{Classifier for places.}\end{définition}
\begin{définition}\pcmn{量词:地方(一处)}\end{définition}
\begin{définition}\pfra{Classificateur des lieux, des endroits.}\end{définition}
\begin{exemple}\pnru{tʰv̩˧-ɻwæ˧-qo˥ | mɤ˧-tʰv̩˧-sɯ˩!}\hspace{5pt}\peng{…has never been to those places}\hspace{5pt}\pcmn{还没到这些地方}\hspace{5pt}\pfra{…n'est jamais allé dans ces lieux-là}\end{exemple}
\end{entrée}

\begin{entrée}
{ɻ̍˩}{}{ⓔɻ̍˩}\formedesurface{ɻ̍˧}\newline
\classe{名词}\ton{L}\begin{définition}\peng{Side, direction.}\end{définition}
\begin{définition}\pcmn{面(一个四方形物品的四面)}\end{définition}
\begin{définition}\pfra{Direction, côté.}\end{définition}
\begin{exemple}\pnru{ʐv̩˧-ɻ̍˥}\hspace{5pt}\peng{the four directions, the four sides (e.g. of a house)}\hspace{5pt}\pcmn{四面}\hspace{5pt}\pfra{les quatre directions, les quatre côtés (d'une maison)}\end{exemple}
\end{entrée}

\begin{entrée}
{‑ɻ̍˩}{}{ⓔ‑ɻ̍˩}\formedesurface{ɻ̍˩˥}\newline
\classe{后缀}\ton{L}\begin{définition}\peng{|fg{inceptive} (|fg{inchoative}).}\end{définition}
\begin{définition}\pcmn{发端}\end{définition}
\begin{définition}\pfra{|fg{inchoatif.}}\end{définition}
\end{entrée}

\begin{entrée}
{=ɻ̍˩}{}{ⓔ=ɻ̍˩}\formedesurface{ɻ̍˩˥}\newline
\classe{附着词}\ton{L}\begin{définition}\peng{Associative plural.}\end{définition}
\begin{définition}\pcmn{联想复数:一家人、一族人、一辈人……}\end{définition}
\begin{définition}\pfra{Pluriel associatif, ou pluriel d'accompagnement, couramment utilisé avec les termes de parenté, les noms de clans…}\end{définition}
\begin{exemple}\pnru{ʈʂʰɯ˧-ʑi˧=ɻ̍˥}\hspace{5pt}\peng{this household; the people of this family}\hspace{5pt}\pcmn{这家的人}\hspace{5pt}\pfra{les gens de cette famille; cette maisonnée-ci}\end{exemple}
\end{entrée}

\begin{entrée}
{ɻ̍˩β}{}{ⓔɻ̍˩β}\formedesurface{ɻ̍˩˥}\newline
\classe{动词}\ton{Lβ}\begin{définition}\peng{To face, to turn toward.}\end{définition}
\begin{définition}\pcmn{对着}\end{définition}
\begin{définition}\pfra{Regarder, se tourner vers, faire face à.}\end{définition}
\begin{exemple}\pnru{mɤ˧-ɻ̍˩}\hspace{5pt}\peng{|fg{neg}}\hspace{5pt}\pcmn{|fg{neg}}\hspace{5pt}\pfra{|fg{neg}}\end{exemple}
\begin{exemple}\pnru{ɖɯ˧-ɻ̍˧∼ɻ̍˩}\hspace{5pt}\peng{|fg{delimitative} \_ |fg{red}}\hspace{5pt}\pcmn{|fg{delimitative} \_ |fg{red}}\hspace{5pt}\pfra{|fg{délimitatif} \_ |fg{red}}\end{exemple}
\begin{exemple}\pnru{ze˩gi˧ ɻ̍˥?}\hspace{5pt}\peng{In which direction should (I) look? / Which direction should I turn to?}\hspace{5pt}\pcmn{(我要)往哪边转?}\hspace{5pt}\pfra{dans quelle direction regarder?}\end{exemple}
\begin{exemple}\pnru{no˧ | ʈʂʰɯ˧tɕo˧ ɻ̍˩!}\hspace{5pt}\peng{Turn this way! / Turn towards this direction!}\hspace{5pt}\pcmn{你往这里转/往这里看!}\hspace{5pt}\pfra{Tourne-toi par ici! / Regarde par ici!}\end{exemple}
\begin{exemple}\pnru{gɤ˩-ɻ̍˥ mv̩˩-ɻ̍˩, | ə˧tso˧ li˧?}\hspace{5pt}\peng{You turn in all directions; what are you looking for/at? / What are you looking for in all directions?}\hspace{5pt}\pcmn{你左转右转,(到底)在看什么?}\hspace{5pt}\pfra{Qu’as-tu à regarder de toutes parts ? / Tu regardes de toutes parts, que cherches-tu ?}\end{exemple}
\end{entrée}

\begin{entrée}
{ɻ̍˧bɤ˧}{}{ⓔɻ̍˧bɤ˧}\formedesurface{ɻ̍˧bɤ˧}\newline
\classe{名词}\ton{M}
\paradigme{\pcmn{:} \p{}}
\begin{définition}\peng{The truth; the facts.}\end{définition}
\begin{définition}\pcmn{实情,真理}\end{définition}
\begin{définition}\pfra{La vérité; le vrai et le faux; les faits authentiques.}\end{définition}
\begin{exemple}\pnru{njɤ˧-ɳɯ˧ | ɻ̍˧bɤ˧ | ʐwɤ˩-bi˩˥!}\hspace{5pt}\peng{I am going to tell the truth!}\hspace{5pt}\pcmn{我要把实情说出来!}\hspace{5pt}\pfra{je vais dire toute la vérité/je vais faire la lumière!}\end{exemple}
\end{entrée}

\begin{entrée}
{ɻ̍˧qʰv̩˧}{}{ⓔɻ̍˧qʰv̩˧}\formedesurface{ɻ̍˧qʰv̩˧}\newline
\classe{名词}\ton{M}
\paradigme{\pcmn{:} \p{}}
\begin{définition}\peng{Warm springs.}\end{définition}
\begin{définition}\pcmn{温泉}\end{définition}
\begin{définition}\pfra{Source chaude.}\end{définition}
\begin{exemple}\pnru{ɻ̍˧qʰv̩˧-dʑɯ˩}\hspace{5pt}\peng{warm spring water (not drinkable)}\hspace{5pt}\pcmn{温泉水(不可饮用)}\hspace{5pt}\pfra{eau de source chaude (non potable)}\end{exemple}
\end{entrée}

\begin{entrée}
{ɻ̍˩ɻ̍˧-lo˩}{}{ⓔɻ̍˩ɻ̍˧-lo˩}\formedesurface{ɻ̍˩ɻ̍˧lo˩}\newline
\classe{名词}\ton{LM-L}\begin{définition}\peng{The horse walking in second position in a caravan.}\end{définition}
\begin{définition}\pcmn{马帮中的第二匹马}\end{définition}
\begin{définition}\pfra{Le cheval qui marche en second (derrière le cheval de tête), dans une caravan.}\end{définition}
\end{entrée}

\begin{entrée}
{ɻ̍˧tɑ˧}{}{ⓔɻ̍˧tɑ˧}\formedesurface{ɻ̍˧tɑ˧}\newline
\classe{名词}\ton{M}
\paradigme{\pcmn{:} \p{}}
\begin{définition}\peng{Lymph nodes, glands.}\end{définition}
\begin{définition}\pcmn{淋巴结}\end{définition}
\begin{définition}\pfra{Ganglions.}\end{définition}
\end{entrée}

\begin{entrée}
{ɻ̍˩ʈʂʰe\#˥}{}{ⓔɻ̍˩ʈʂʰe\#˥}\formedesurface{ɻ̍˩ʈʂʰe˥}\newline
\classe{名词}\ton{LM+\#H}\begin{définition}\peng{Masculine given name.}\end{définition}
\begin{définition}\pcmn{男性名字}\end{définition}
\begin{définition}\pfra{Prénom masculin.}\end{définition}
\end{entrée}

\begin{entrée}
{ɻ̍˩ʈʂʰe˧-ɖɯ˩mɑ˩}{}{ⓔɻ̍˩ʈʂʰe˧-ɖɯ˩mɑ˩}\formedesurface{ɻ̍˩ʈʂʰe˧ɖɯ˩mɑ˩}\newline
\classe{名词}\ton{LM-L}\begin{définition}\peng{Feminine given name.}\end{définition}
\begin{définition}\pcmn{女性名字}\end{définition}
\begin{définition}\pfra{Prénom féminin.}\end{définition}
\end{entrée}

\begin{entrée}
{ɻ̍˩ʈʂʰe˧-tsʰɯ˩ɻ̍˩}{}{ⓔɻ̍˩ʈʂʰe˧-tsʰɯ˩ɻ̍˩}\formedesurface{ɻ̍˩ʈʂʰe˧tsʰɯ˩ɻ̍˩}\newline
\classe{名词}\ton{LM-L}\begin{définition}\peng{Masculine given name.}\end{définition}
\begin{définition}\pcmn{男性名字}\end{définition}
\begin{définition}\pfra{Prénom masculin.}\end{définition}
\end{entrée}

\newpage\caractère{ʁ}

\begin{entrée}
{ʁɑ˥}{}{ⓔʁɑ˥}\formedesurface{ʁɑ˧}\newline
\classe{名词}\ton{\#H}\begin{définition}\peng{Strength.}\end{définition}
\begin{définition}\pcmn{力气}\end{définition}
\begin{définition}\pfra{Force.}\end{définition}
\begin{exemple}\pnru{ʁɑ˧ ʑi˧}\hspace{5pt}\peng{to have strength}\hspace{5pt}\pcmn{有力量}\hspace{5pt}\pfra{avoir de la force}\end{exemple}
\begin{exemple}\pnru{no˧ɻ̍˩ | hĩ˧tɕʰi˧ ʁɑ˧ ʑi˧!}\hspace{5pt}\peng{Your family/clan is powerful!}\hspace{5pt}\pcmn{你们家族很强大!}\hspace{5pt}\pfra{votre famille/lignée/tribu est puissante!}\end{exemple}
\begin{exemple}\pnru{ʁɑ˧ tʰv̩˧ (+ze˩)}\hspace{5pt}\peng{to exert oneself, to make efforts}\hspace{5pt}\pcmn{尽力}\hspace{5pt}\pfra{faire des efforts, donner toutes ses forces, s'impliquer (dans une tâche)}\end{exemple}
\end{entrée}

\begin{entrée}
{ʁɑ˥}{₁}{ⓔʁɑ˥ⓗ1}\formedesurface{ʁɑ˧}\newline
\classe{动词}\ton{H}
1\begin{définition}\peng{To invite, to call over.}\end{définition}
\begin{définition}\pcmn{请}\end{définition}
\begin{définition}\pfra{Convier, faire venir, inviter.}\end{définition}
\end{entrée}

\begin{entrée}
{ʁɑ˥}{₂}{ⓔʁɑ˥ⓗ2}\formedesurface{ʁɑ˧}\newline
\classe{动词}\ton{H}
2\begin{définition}\peng{To win, to succeed.}\end{définition}
\begin{définition}\pcmn{赢}\end{définition}
\begin{définition}\pfra{Gagner.}\end{définition}
\begin{exemple}\pnru{le˧-ʁɑ˥-ze˩}\hspace{5pt}\peng{|fg{accomp} \_ |fg{pfv}}\hspace{5pt}\pcmn{赢了}\hspace{5pt}\pfra{|fg{accomp} \_ |fg{pfv}}\end{exemple}
\end{entrée}

\begin{entrée}
{ʁɑ˧}{₁}{ⓔʁɑ˧ⓗ1}\formedesurface{ʁɑ˧}\newline
\classe{形容词}\ton{M}
1\begin{définition}\peng{Good (of good quality).}\end{définition}
\begin{définition}\pcmn{好(质量好,品质好,脾气好)}\end{définition}
\begin{définition}\pfra{Bon, fiable: objet de bonne qualité; travail de bonne tenue; personne ayant bon caractère.}\end{définition}
\begin{exemple}\pnru{mɤ˧-ʁɑ˧-hĩ˧ ʂe˧}\hspace{5pt}\peng{bad meat, meat of poor quality}\hspace{5pt}\pcmn{不好的肉(质量不好)}\hspace{5pt}\pfra{de la mauvaise viande}\end{exemple}
\begin{exemple}\pnru{mɤ˧-ʁɑ˧-hĩ˧ ʂe˧-kʰwɤ˧ ki˩}\hspace{5pt}\peng{to give a piece of bad meat}\hspace{5pt}\pcmn{给一块不好的肉}\hspace{5pt}\pfra{donner un morceau de mauvaise viande}\end{exemple}
\begin{exemple}\pnru{pʰi˩ko˧ | mɤ˧-ʁɑ˧-ze˧!}\hspace{5pt}\peng{The apples are not good anymore! (Context: in March, apples from the previous harvest are not good anymore: they are either rotten or sour.)}\hspace{5pt}\pcmn{苹果不好了! / 苹果不新鲜了!(三月份,上一季收获的苹果已经不好吃的了,或者烂了,或者变酸)}\hspace{5pt}\pfra{Les pommes ne sont plus bonnes! (Contexte: au mois de mars, les pommes de la récolte précédente ne sont plus bonnes, elles sont fripées ou pourries.)}\end{exemple}
\begin{exemple}\pnru{hĩ˧ ɖɯ˧-v̩˧ | ʁɑ˧-mɤ˧-ʑi˧-hĩ˧ ʐwɤ˧˥!}\hspace{5pt}\peng{Someone is talking nonsense!}\hspace{5pt}\pcmn{有人在乱说话!}\hspace{5pt}\pfra{quelqu'un dit n'importe quoi}\end{exemple}
\end{entrée}

\begin{entrée}
{ʁɑ˧}{₂}{ⓔʁɑ˧ⓗ2}\formedesurface{ʁɑ˧}\newline
\classe{动词}\ton{M}
2\begin{définition}\peng{To ask for forgiveness, to apologize.}\end{définition}
\begin{définition}\pcmn{道歉、请人家原谅}\end{définition}
\begin{définition}\pfra{Présenter ses excuses, demander pardon.}\end{définition}
\begin{exemple}\pnru{ʁɑ˧-ze˧!}\hspace{5pt}\peng{Please accept my apologies! (To a person of high rank)}\hspace{5pt}\pcmn{抱歉! / 请原谅!(对地位比自己高的人说)}\hspace{5pt}\pfra{Pardon! (adressé à une personne de haut rang)}\end{exemple}
\end{entrée}

\begin{entrée}
{ʁɑ˧β}{}{ⓔʁɑ˧β}\formedesurface{ʁɑ˧}\newline
\classe{动词}\ton{Mβ}\begin{définition}\peng{To stride over (an obstacle), to straddle, to go beyond, to cross.}\end{définition}
\begin{définition}\pcmn{跨(跨过小沟)}\end{définition}
\begin{définition}\pfra{Enjamber, franchir (un ruisseau, le seuil d'une maison…).}\end{définition}
\begin{exemple}\pnru{le˧-ʁɑ˧-ze˧}\hspace{5pt}\peng{|fg{accomp} \_ |fg{pfv}}\hspace{5pt}\pcmn{跨过了}\hspace{5pt}\pfra{|fg{accomp} \_ |fg{pfv}}\end{exemple}
\begin{exemple}\pnru{kʰi˧mi˧ | le˧-ʁɑ˧}\hspace{5pt}\peng{to cross the main door (into the farm) (striding over the threshold)}\hspace{5pt}\pcmn{过大门(过门槛)}\hspace{5pt}\pfra{franchir la porte d'entrée de la ferme (on passe le seuil, en marchant; il ne s'agit pas de grimper par-dessus)}\end{exemple}
\begin{exemple}\pnru{kʰi˧mi˧ ʁɑ˧-ze˧}\hspace{5pt}\peng{crossed the main door}\hspace{5pt}\pcmn{过了大门}\hspace{5pt}\pfra{a franchi la porte d'entrée}\end{exemple}
\end{entrée}

\begin{entrée}
{ʁɑ˧dzi˧}{}{ⓔʁɑ˧dzi˧}\formedesurface{ʁɑ˧dzi˧}\newline
\classe{名词}\ton{M}
\paradigme{\pcmn{:} \p{}}
\begin{définition}\peng{Poplar.}\end{définition}
\begin{définition}\pcmn{杨树}\end{définition}
\begin{définition}\pfra{Peuplier.}\end{définition}
\end{entrée}

\begin{entrée}
{ʁɑ˧ɭɯ\#˥}{}{ⓔʁɑ˧ɭɯ\#˥}\formedesurface{ʁɑ˧ɭɯ˧}\newline
\classe{名词}\ton{\#H}
\paradigme{\pcmn{:} \p{}}
\begin{définition}\peng{Cairn: a human-made pile of stones, used as trail marker. The stones of the cairn are not engraved.}\end{définition}
\begin{définition}\pcmn{石堆}\end{définition}
\begin{définition}\pfra{Cairn: tas de pierres qui aide à repérer un sentier.}\end{définition}
\begin{exemple}\pnru{qo˩qɑ˩-ʁɑ˥ɭɯ˩}\hspace{5pt}\peng{mountain pass cairn: a cairn at a mountain pass}\hspace{5pt}\pcmn{垭口石堆:垭口上的石堆}\hspace{5pt}\pfra{cairn situé à un col}\end{exemple}
\end{entrée}

\begin{entrée}
{ʁɑ˩mi˥}{}{ⓔʁɑ˩mi˥}\formedesurface{ʁɑ˩mi˥}\newline
\classe{动词}\ton{LH}\begin{définition}\peng{To apologize.}\end{définition}
\begin{définition}\pcmn{道歉}\end{définition}
\begin{définition}\pfra{Demander pardon; formule de requête, et de remerciement. Le spectre des significations rappelle l'étymologie de «merci»: de «crier merci» (implorer la vie sauve) à un emploi comme formule de politesse courante.}\end{définition}
\begin{exemple}\pnru{ʁɑ˩mi˥-ze˩!}\hspace{5pt}\peng{Thank you!}\hspace{5pt}\pcmn{谢谢!}\hspace{5pt}\pfra{Merci!}\end{exemple}
\end{entrée}

\begin{entrée}
{ʁɑ˧pv̩˧}{}{ⓔʁɑ˧pv̩˧}\formedesurface{ʁɑ˧pv̩˧}\newline
\classe{名词}\ton{M}
\paradigme{\pcmn{:} \p{}}
\begin{définition}\peng{Chest.}\end{définition}
\begin{définition}\pcmn{胸脯、胸膛}\end{définition}
\begin{définition}\pfra{Poitrine.}\end{définition}
\end{entrée}

\begin{entrée}
{ʁɑ˧pv̩˧-ɻ̃\#˥}{}{ⓔʁɑ˧pv̩˧-ɻ̃\#˥}\formedesurface{ʁɑ˧pv̩˧ɻ̃˧}\newline
\classe{名词}\ton{\#H}
\paradigme{\pcmn{:} \p{}}
\begin{définition}\peng{Clavicle; collarbone.}\end{définition}
\begin{définition}\pcmn{锁骨}\end{définition}
\begin{définition}\pfra{Clavicule.}\end{définition}
\end{entrée}

\begin{entrée}
{ʁɑ˧pʰv̩\#˥}{}{ⓔʁɑ˧pʰv̩\#˥}\formedesurface{ʁɑ˧pʰv̩˧}\newline
\classe{名词}\ton{\#H}
\paradigme{\pcmn{:} \p{}}
\begin{définition}\peng{Salary, price paid for the work done by a worker.}\end{définition}
\begin{définition}\pcmn{工资, 工钱}\end{définition}
\begin{définition}\pfra{Salaire, littéralement «prix du travail».}\end{définition}
\end{entrée}

\begin{entrée}
{ʁɑ˥ʂe˩}{}{ⓔʁɑ˥ʂe˩}\formedesurface{ʁɑ˧ʂe˩}\newline
\classe{动词}\ton{H.L}\begin{définition}\peng{To invite to come over, to call upon the services of.}\end{définition}
\begin{définition}\pcmn{请来(和尚、医生……)}\end{définition}
\begin{définition}\pfra{Faire venir, faire appel à (un moine, un médecin…).}\end{définition}
\begin{exemple}\pnru{ʈæ˧bɤ˧ ʁɑ˧-ʂe˩}\hspace{5pt}\peng{to ask a monk to come over}\hspace{5pt}\pcmn{请和尚到家来}\hspace{5pt}\pfra{faire venir un moine, faire appel à un moine}\end{exemple}
\begin{exemple}\pnru{ʈʂʰæ˧ɣɯ˧ ki˩-hĩ˩ | le˧-ʁɑ˥-ʂe˩ | le˧-po˧-jo˥ / ʈʂʰæ˧ɣɯ˧ ki˩-hĩ˩ | ɖɯ˧-ʁɑ˥-ʂe˩-ɻ̍˩ hõ˩}\hspace{5pt}\peng{to call the doctor}\hspace{5pt}\pcmn{请医生来}\hspace{5pt}\pfra{faire venir le médecin}\end{exemple}
\begin{exemple}\pnru{njɤ˧-ɳɯ˧ | no˩ ʁɑ˩-ʂe˥}\hspace{5pt}\peng{I invite you to come over}\hspace{5pt}\pcmn{我请你来}\hspace{5pt}\pfra{je t'invite/je te convie/je te fais venir}\end{exemple}
\begin{exemple}\pnru{mɤ˧-ʁɑ˥-ʂe˩}\hspace{5pt}\peng{|fg{neg}}\hspace{5pt}\pcmn{|fg{neg}}\hspace{5pt}\pfra{|fg{neg}: ne pas faire appel à}\end{exemple}
\end{entrée}

\begin{entrée}
{ʁɑ˩ʂɯ˧}{}{ⓔʁɑ˩ʂɯ˧}\formedesurface{ʁɑ˩ʂɯ˥}\newline
\classe{助词}\ton{LM}\begin{définition}\peng{In fact.}\end{définition}
\begin{définition}\pcmn{其实、事实上}\end{définition}
\begin{définition}\pfra{En fait, en réalité.}\end{définition}
\end{entrée}

\begin{entrée}
{ʁɑ˧-ʐwɤ˧˥}{}{ⓔʁɑ˧-ʐwɤ˧˥}\formedesurface{ʁɑ˧ʐwɤ˧˥}\newline
\classe{动词}\ton{MH\#}\begin{définition}\peng{To browbeat; to take advantage of; to pick on.}\end{définition}
\begin{définition}\pcmn{欺负}\end{définition}
\begin{définition}\pfra{Provoquer/humilier.}\end{définition}
\begin{exemple}\pnru{ʈʂʰɯ˧-v̩˧ | hĩ˧-ki˧ ʁɑ˧-ʐwɤ˧-ʝi˥!}\hspace{5pt}\peng{(s)he is picking on someone}\hspace{5pt}\pcmn{他欺负人、他对人发脾气}\hspace{5pt}\pfra{elle/il provoque/humilie quelqu'un d'autre}\end{exemple}
\begin{exemple}\pnru{ʁɑ˧ ʐwɤ˧-ɻ̍˥}\hspace{5pt}\peng{as above}\hspace{5pt}\pcmn{同上}\hspace{5pt}\pfra{même sens}\end{exemple}
\begin{exemple}\pnru{no˧ | ʁɑ˧ ʐwɤ˧-tʰɑ˧-ɻ̍˥!}\hspace{5pt}\peng{Don't browbeat people!}\hspace{5pt}\pcmn{你不要欺负人!}\hspace{5pt}\pfra{Ne provoque pas les gens!}\end{exemple}
\end{entrée}

\begin{entrée}
{ʁæ˥}{}{ⓔʁæ˥}\formedesurface{ʁæ˧}\newline
\classe{名词}\ton{\#H}
\paradigme{\pcmn{:} \p{}}
\begin{définition}\peng{Neck (monosyllable).}\end{définition}
\begin{définition}\pcmn{脖子(单音节)}\end{définition}
\begin{définition}\pfra{Cou (monosyllabe; moins usité que le disyllabe).}\end{définition}
\end{entrée}

\begin{entrée}
{ʁæ˧}{}{ⓔʁæ˧}\formedesurface{ʁæ˧}\newline
\classe{形容词}\ton{M}\begin{définition}\peng{Rich.}\end{définition}
\begin{définition}\pcmn{富}\end{définition}
\begin{définition}\pfra{Riche.}\end{définition}
\end{entrée}

\begin{entrée}
{ʁæ˧˥}{}{ⓔʁæ˧˥}\formedesurface{ʁæ˧˥}\newline
\classe{形容词}\ton{MH}\begin{définition}\peng{Nauseous, disgusting.}\end{définition}
\begin{définition}\pcmn{不好吃,恶心}\end{définition}
\begin{définition}\pfra{Écoeurant, dégoûtant, pas bon au goût (pas forcément à cause d'un excès de graisse: par exemple, selon les critères gastronomiques locaux, mes flocons d'avoine entrent dans cette catégorie).}\end{définition}
\end{entrée}

\begin{entrée}
{ʁæ˩˥}{}{ⓔʁæ˩˥}\formedesurface{ʁæ˩˥}\newline
\classe{名词}\ton{LH}\begin{définition}\peng{Sap.}\end{définition}
\begin{définition}\pcmn{树液}\end{définition}
\begin{définition}\pfra{Sève, résine.}\end{définition}
\begin{exemple}\pnru{tʰo˩ʁæ˩˥}\hspace{5pt}\peng{same meaning}\hspace{5pt}\pcmn{同上}\hspace{5pt}\pfra{même sens}\end{exemple}
\end{entrée}

\begin{entrée}
{ʁæ˩α}{₁}{ⓔʁæ˩αⓗ1}\formedesurface{ʁæ˩˥}\newline
\classe{动词}\ton{Lα}
1\begin{définition}\peng{To fall apart, to scatter, to melt (e.g. clods of dry earth melting in water when a field is irrigated after ploughing).}\end{définition}
\begin{définition}\pcmn{散、散开,化,溶化(一块土在水里面散开)}\end{définition}
\begin{définition}\pfra{Se défaire, fondre, se dissoudre: une motte de terre plongée dans l'eau se défait.}\end{définition}
\begin{exemple}\pnru{le˧-ʁæ˩-ze˩}\hspace{5pt}\peng{|fg{accomp} \_ |fg{pfv}}\hspace{5pt}\pcmn{|fg{accomp} \_ |fg{pfv}}\hspace{5pt}\pfra{|fg{accomp} \_ |fg{pfv}}\end{exemple}
\begin{exemple}\pnru{le˧-ʁæ˧∼ʁæ˥ (-ze˩ / -bi˩)}\hspace{5pt}\peng{|fg{red}}\hspace{5pt}\pcmn{重叠}\hspace{5pt}\pfra{|fg{red}}\end{exemple}
\begin{exemple}\pnru{ɖɯ˧-kʰwɤ˧ ʁæ˥}\hspace{5pt}\peng{a lump (of earth) melts}\hspace{5pt}\pcmn{一块(土)散开}\hspace{5pt}\pfra{un morceau (de terre) se défait}\end{exemple}
\begin{exemple}\pnru{ʈʂe˧ʈv̩˥ | le˧-ʁæ˩-ze˩}\hspace{5pt}\peng{Clods of earth fall apart (after ploughing, the fields are irrigated; clods of earth melt into the water)}\hspace{5pt}\pcmn{土块散开在了(耕田后灌溉,土块散在水里)}\hspace{5pt}\pfra{les mottes de terre se défond, se dissolvent (dans l'eau dont on inonde les champs après les labours)}\end{exemple}
\end{entrée}

\begin{entrée}
{ʁæ˩α}{₂}{ⓔʁæ˩αⓗ2}\formedesurface{ʁæ˩˥}\newline
\classe{形容词}\ton{Lα}
2\begin{définition}\peng{Drunk.}\end{définition}
\begin{définition}\pcmn{醉}\end{définition}
\begin{définition}\pfra{Ivre, saoul.}\end{définition}
\begin{exemple}\pnru{ʐɯ˧ ʁæ˩(-ze˩)}\hspace{5pt}\peng{drunk}\hspace{5pt}\pcmn{醉酒}\hspace{5pt}\pfra{ivre d'alcool}\end{exemple}
\end{entrée}

\begin{entrée}
{ʁæ˩α}{₃}{ⓔʁæ˩αⓗ3}\formedesurface{ʁæ˩˥}\newline
\classe{形容词}\ton{Lα}
3\begin{définition}\peng{Appropriate; auspicious.}\end{définition}
\begin{définition}\pcmn{合适,吉利}\end{définition}
\begin{définition}\pfra{Approprié; propice, favorable.}\end{définition}
\begin{exemple}\pnru{ʁæ˧ mɤ˧-ʑi˧}\hspace{5pt}\peng{not propicious / not favourable}\hspace{5pt}\pcmn{不吉利、不合适}\hspace{5pt}\pfra{ce n'est pas propice/favorable}\end{exemple}
\begin{exemple}\pnru{ʁæ˧ mɤ˧-ʑi˧, | ʝi˧ mɤ˧-tʰɑ˩! / ʝi˧-mɤ˧-ɖo˧!}\hspace{5pt}\peng{It is not appropriate / the situation is not propitious; it must not / should not be done! (A phrase to caution others against doing something)}\hspace{5pt}\pcmn{不吉利 / 不合适(的事情),不能做!/ 不要做!(警告)}\hspace{5pt}\pfra{les circonstances ne sont pas propices / ce n'est pas une bonne idée, il ne faut pas le faire! (Mise en garde)}\end{exemple}
\begin{exemple}\pnru{ʁæ˧ mɤ˧-ʑi˧, | ʐwɤ˩ mɤ˩-tʰɑ˥! / ʁæ˧ mɤ˧-ʑi˧, | ʐwɤ˩ mɤ˩-ɖo˩˥!}\hspace{5pt}\peng{It's not appropriate; one must not talk about it! / One should not talk nonsense! (A phrase to caution others against being carelessly talkative)}\hspace{5pt}\pcmn{不合适(的话),不能说! / 不合适(的话),不要说!(警告)}\hspace{5pt}\pfra{Ca ne convient pas; on ne peut pas le dire / on ne doit pas le dire! / Il ne faut pas tenir de propos inappropriés! / Il faut faire attention à ce qu'on dit! (Mise en garde)}\end{exemple}
\begin{exemple}\pnru{ʁæ˧-mɤ˧-ʑi˧, | tɕi˩-mɤ˩-ɖo˩˥!}\hspace{5pt}\peng{(You) must not transcribe the bad ones! / You must not transcribe the messy ones! (Context: the investigator was explaining his wish to choose, among the wealth of recorded narratives, those that are the most interesting and successful, to do a transcription and complete translation and annotation. By her answer, the consultant indicates her approval, at the same time as she shows her understanding of the idea: any materials that may be inappropriate in any way should be left out, and not put to writing.)}\hspace{5pt}\pcmn{乱七八糟的,不要记录! / 不好的,不要记录!(情景:选择一个故事来做记音翻译等。合作人提出,要考虑好记录哪些、选择好的资料,不能什么都记录。)}\hspace{5pt}\pfra{Il ne faut pas transcrire ceux qui sont pas bons! Il ne faut pas écrire n'importe quoi! (Contexte: cette phrase récapitule le principe qui préside au choix des récits à transcrire et traduire. J'expliquais de mon mieux mon souhait de choisir, parmi les récits enregistrés --relativement nombreux--, ceux qui sont les plus intéressants, et les plus réussis. Par cet énoncé, la locutrice apporte son assentiment, en même temps qu'elle indique qu'elle comprend l'idée: il faut transcrire les récits qui sont bons; il faut bien choisir, et écarter ceux qui ne seraient pas appropriés en quoi que ce soit. Par le même énoncé, la locutrice témoigne en outre de sa modestie: elle ne défend pas l'idée selon laquelle tous ses récits sont d'égale qualité, et accepte de bonne grâce l'idée que certains sont plus réussis que d'autres, ou plus adéquats pour le propos du linguiste.)}\end{exemple}
\begin{exemple}\pnru{ʈʂʰɯ˧ | lo˧ | ʁæ˧-mɤ˧-ʑi˧ ʝi˧!}\hspace{5pt}\peng{He does a bad job of it! / He makes a mess of his work!}\hspace{5pt}\pcmn{他工作做得乱七八糟!}\hspace{5pt}\pfra{Il ne fait pas attention dans son travail! il ne travaille pas avec soin! il fait n'importe quoi!}\end{exemple}
\end{entrée}

\begin{entrée}
{ʁæ˩α}{₄}{ⓔʁæ˩αⓗ4}\formedesurface{ʁæ˩˥}\newline
\classe{形容词}\ton{Lα}
4\begin{définition}\peng{Ugly.}\end{définition}
\begin{définition}\pcmn{丑陋}\end{définition}
\begin{définition}\pfra{Laid, vilain.}\end{définition}
\end{entrée}

\begin{entrée}
{ʁæ˧bæ˧}{}{ⓔʁæ˧bæ˧}\formedesurface{ʁæ˧bæ˧}\newline
\classe{名词}\ton{M}
\paradigme{\pcmn{:} \p{}}
\begin{définition}\peng{Dish, plate.}\end{définition}
\begin{définition}\pcmn{盘子}\end{définition}
\begin{définition}\pfra{Assiette.}\end{définition}
\end{entrée}

\begin{entrée}
{ʁæ˩bæ˩}{}{ⓔʁæ˩bæ˩}\formedesurface{ɖɯ˧ ʁæ˩bæ˩}\newline
\classe{量词}\ton{L}\begin{définition}\peng{Classifier: a plateful, the contents of a plate.}\end{définition}
\begin{définition}\pcmn{量词:盘}\end{définition}
\begin{définition}\pfra{Assiettée, contenue d'une assiette.}\end{définition}
\begin{exemple}\pnru{ɖɯ˧-ʁæ˩bæ˩, | ɲi˧-ʁæ˩bæ˩, | so˩-ʁæ˩bæ˩˥, | ʐv̩˧-ʁæ˥bæ˩, | ŋwɤ˧-ʁæ˥bæ˩, | qʰv̩˧-ʁæ˥bæ˩, | ʂɯ˧-ʁæ˩bæ˩, | hõ˧-ʁæ˥bæ˩, | gv̩˧-ʁæ˥bæ˩, | tsʰe˩-ʁæ˩bæ˩˥}\hspace{5pt}\peng{association with numerals from 1 to 10}\hspace{5pt}\pcmn{与数词结合,一至十}\hspace{5pt}\pfra{association avec des numéraux, de 1 à 10}\end{exemple}
\end{entrée}

\begin{entrée}
{ʁæ˩ɖʐv̩˧}{}{ⓔʁæ˩ɖʐv̩˧}\formedesurface{ʁæ˩ɖʐv̩˥}\newline
\classe{形容词}\ton{LM}\begin{définition}\peng{Ugly.}\end{définition}
\begin{définition}\pcmn{丑陋}\end{définition}
\begin{définition}\pfra{Laid, vilain.}\end{définition}
\end{entrée}

\begin{entrée}
{ʁæ˧ɭɯ˥}{}{ⓔʁæ˧ɭɯ˥}\formedesurface{ʁæ˧ɭɯ˥}\newline
\classe{名词}\ton{H\#}
\paradigme{\pcmn{:} \p{}}
\begin{définition}\peng{Fetters (wooden fetters, around the neck); yoke.}\end{définition}
\begin{définition}\pcmn{枷锁(木头做的)}\end{définition}
\begin{définition}\pfra{Carcan (était en bois); littéralement «[objet dans lequel] on met le cou».}\end{définition}
\begin{exemple}\pnru{ʁæ˧ɭɯ˥ | ɖɯ˧-ɭɯ˧ kʰɯ˧˥}\hspace{5pt}\peng{to put fetters (on someone's neck)}\hspace{5pt}\pcmn{套上一个枷锁(在一个人的脖子上)}\hspace{5pt}\pfra{mettre un joug (à quelqu'un)}\end{exemple}
\begin{exemple}\pnru{ʁæ˧ɭɯ˥ kʰɯ˩}\hspace{5pt}\peng{to put fetters (on someone's neck)}\hspace{5pt}\pcmn{套上枷锁(在一个人的脖子上)}\hspace{5pt}\pfra{mettre le joug (à quelqu'un)}\end{exemple}
\end{entrée}

\begin{entrée}
{ʁæ˧mi˧}{}{ⓔʁæ˧mi˧}\formedesurface{ʁæ˧mi˧}\newline
\classe{名词}\ton{M}
\paradigme{\pcmn{:} \p{}}
\begin{définition}\peng{Sword.}\end{définition}
\begin{définition}\pcmn{剑}\end{définition}
\begin{définition}\pfra{Épée.}\end{définition}
\end{entrée}

\begin{entrée}
{ʁæ˧ŋv̩˥}{}{ⓔʁæ˧ŋv̩˥}\formedesurface{ʁæ˧ŋv̩˥}\newline
\classe{名词}\ton{H\#}
\paradigme{\pcmn{:} \p{}}
\begin{définition}\peng{Silver-embellished collar (a precious part of the dress, with silver thread).}\end{définition}
\begin{définition}\pcmn{银衣领}\end{définition}
\begin{définition}\pfra{Col (précieux, avec des fils d'argent).}\end{définition}
\end{entrée}

\begin{entrée}
{ʁæ˧ɻ̍˥}{}{ⓔʁæ˧ɻ̍˥}\formedesurface{ʁæ˧ɻ̍˥}\newline
\classe{名词}\ton{H\#}
\paradigme{\pcmn{:} \p{}}
\begin{définition}\peng{Neck.}\end{définition}
\begin{définition}\pcmn{脖子}\end{définition}
\begin{définition}\pfra{Cou.}\end{définition}
\end{entrée}

\begin{entrée}
{ʁæ˧tɑ˩}{}{ⓔʁæ˧tɑ˩}\formedesurface{ʁæ˧tɑ˩}\newline
\classe{名词}\ton{L\#}
\paradigme{\pcmn{:} \p{}}
\begin{définition}\peng{Withers: part of the ox's body on which the yoke rests.}\end{définition}
\begin{définition}\pcmn{肩隆}\end{définition}
\begin{définition}\pfra{Garrot: partie du corps de l'animal sur lequel repose le joug.}\end{définition}
\begin{exemple}\pnru{ʝi˧-ʁæ˧tɑ˥}\hspace{5pt}\peng{ox's withers}\hspace{5pt}\pcmn{牛肩隆}\hspace{5pt}\pfra{garrot de vache}\end{exemple}
\begin{exemple}\pnru{ʁæ˧tɑ˩ tʰv̩˩-ɭɯ˩}\hspace{5pt}\peng{|fg{n}+|fg{dem}+|fg{clf}}\hspace{5pt}\pcmn{这只肩隆}\hspace{5pt}\pfra{|fg{n}+|fg{dem}+|fg{clf}}\end{exemple}
\end{entrée}

\begin{entrée}
{ʁæ˧ʈv̩˥}{}{ⓔʁæ˧ʈv̩˥}\formedesurface{ʁæ˧ʈv̩˥}\newline
\classe{名词}\ton{H\#}
\paradigme{\pcmn{:} \p{}}
\begin{définition}\peng{Neck.}\end{définition}
\begin{définition}\pcmn{脖子}\end{définition}
\begin{définition}\pfra{Cou.}\end{définition}
\end{entrée}

\begin{entrée}
{ʁæ˧zo\#˥}{}{ⓔʁæ˧zo\#˥}\formedesurface{ʁæ˧zo˧}\newline
\classe{名词}\ton{\#H}\begin{définition}\peng{Short sword.}\end{définition}
\begin{définition}\pcmn{短剑}\end{définition}
\begin{définition}\pfra{Petite épée.}\end{définition}
\end{entrée}

\begin{entrée}
{ʁæ˧ʑi˧}{}{ⓔʁæ˧ʑi˧}\formedesurface{ʁæ˧ʑi˧}\newline
\classe{动词}\ton{M}\begin{définition}\peng{To mind something; to take into account; to take into consideration; to care about.}\end{définition}
\begin{définition}\pcmn{考虑}\end{définition}
\begin{définition}\pfra{S'occuper de; se mêler de; prendre en considération.}\end{définition}
\begin{exemple}\pnru{hĩ˧ | qʰɑ˧-kv̩˧ dʑo˧˥ | mɤ˧-ʁæ˧ʑi˧, | njɤ˧-ɳɯ˧ qʰæ˧˥! |}\hspace{5pt}\peng{No matter how many people (guests) there are, I (go to participate and) help! (Context: the consultant explains how, following Na traditions, she volunteers her time to help on important occasions, such as funerals, to help other families.)}\hspace{5pt}\pcmn{无论有多少个人,我都会去帮助!(情景:合作人描写她在永宁有大事时怎么去帮其它家庭的忙,不考虑活多么累,只考虑怎么能给予帮助)}\hspace{5pt}\pfra{«Moi, j'aide, sans m'inquiéter de savoir combien il y a d'invités (littéralement ‘de gens’)!» (contexte: F4 explique comment on se dévouait autrefois pour aider les amis, non membres de la famille, lors des grandes occasions, telles que les funérailles)}\end{exemple}
\begin{exemple}\pnru{no˧ | mɤ˧-ʁæ˧ʑi˧!}\hspace{5pt}\peng{Leave me alone! / Leave me in peace! / Mind your own business!}\hspace{5pt}\pcmn{别管我了! / 请让我安静! / 请不要打扰我了!}\hspace{5pt}\pfra{Fiche-moi la paix! / Laisse-moi tranquille! / Mêle-toi de tes affaires!}\end{exemple}
\end{entrée}

\begin{entrée}
{‑ʁo}{}{ⓔ‑ʁo}\formedesurface{--}\newline
\classe{}\ton{0?}\begin{définition}\peng{On, on top of.}\end{définition}
\begin{définition}\pcmn{上面}\end{définition}
\begin{définition}\pfra{Sur.}\end{définition}
\end{entrée}

\begin{entrée}
{ʁo˥}{}{ⓔʁo˥}\newline
\classe{名词}
\sens{1}\paradigme{\pcmn{:} \p{}}
\begin{définition}\peng{Head (monosyllable).}\end{définition}
\begin{définition}\pcmn{头(单音节)}\end{définition}
\begin{définition}\pfra{Tête (monosyllabique).}\end{définition}\sens{2}
\begin{définition}\peng{Beginning.}\end{définition}
\begin{définition}\pcmn{开头}\end{définition}
\begin{définition}\pfra{Début.}\end{définition}
\begin{exemple}\pnru{ɬi˧-ʁo\#˥}\hspace{5pt}\peng{the beginning of the month, the first days of the month}\hspace{5pt}\pcmn{月初}\hspace{5pt}\pfra{le début du mois}\end{exemple}
\begin{exemple}\pnru{kʰv̩˧-ʁo˥\$}\hspace{5pt}\peng{the beginning of the year}\hspace{5pt}\pcmn{年初}\hspace{5pt}\pfra{le début de l'année}\end{exemple}
\begin{exemple}\pnru{*ɲi˧-ʁo˩}\hspace{5pt}\peng{*the beginning of the day}\hspace{5pt}\pcmn{*天初}\hspace{5pt}\pfra{*le début de la journée}\end{exemple}
\end{entrée}

\begin{entrée}
{ʁo˧}{₁}{ⓔʁo˧ⓗ1}\formedesurface{ʁo˧}\newline
\classe{动词}\ton{M intrans}
1\begin{définition}\peng{To lay eggs.}\end{définition}
\begin{définition}\pcmn{下蛋}\end{définition}
\begin{définition}\pfra{Pondre.}\end{définition}
\begin{exemple}\pnru{æ˩ ʁo˥}\hspace{5pt}\peng{to lay eggs}\hspace{5pt}\pcmn{下蛋}\hspace{5pt}\pfra{pondre des œufs}\end{exemple}
\begin{exemple}\pnru{æ˩mi˧ tʰi˧-ʁo˧-dʑo˧!}\hspace{5pt}\peng{The hen is laying eggs!}\hspace{5pt}\pcmn{母鸡在下蛋!}\hspace{5pt}\pfra{la poule est en train de pondre!}\end{exemple}
\begin{exemple}\pnru{æ˩mi˧ | æ˩ ʁo˧-ze˩!}\hspace{5pt}\peng{The hen has laid eggs!}\hspace{5pt}\pcmn{母鸡下蛋了!}\hspace{5pt}\pfra{la poule a pondu!}\end{exemple}
\end{entrée}

\begin{entrée}
{ʁo˧}{₂}{ⓔʁo˧ⓗ2}\formedesurface{ʁo˧}\newline
\classe{动词}\ton{M intrans}
2\begin{définition}\peng{To be able to, to manage to.}\end{définition}
\begin{définition}\pcmn{能……、有能力做}\end{définition}
\begin{définition}\pfra{Arriver à, parvenir à.}\end{définition}
\begin{exemple}\pnru{njɤ˧ | tɕi˩-mɤ˩-ʁo˩˥!}\hspace{5pt}\peng{I can't write! / I am not able to write! (Said by someone who has not learnt to write)}\hspace{5pt}\pcmn{我写不出来! / 我不会写!}\hspace{5pt}\pfra{je ne parviens pas à écrire/je ne sais pas écrire!}\end{exemple}
\begin{exemple}\pnru{njɤ˧ | tɕi˩-ʁo˩˥!}\hspace{5pt}\peng{I can write! / I am able to write! / I know how to write!}\hspace{5pt}\pcmn{我会写! / 我写得出来!}\hspace{5pt}\pfra{je parviens à écrire/je sais écrire!}\end{exemple}
\end{entrée}

\begin{entrée}
{ʁo˧˥}{}{ⓔʁo˧˥}\formedesurface{ʁo˧˥}\newline
\classe{名词}\ton{MH}
\paradigme{\pcmn{:} \p{}}
\begin{définition}\peng{Needle.}\end{définition}
\begin{définition}\pcmn{针}\end{définition}
\begin{définition}\pfra{Aiguille.}\end{définition}
\end{entrée}

\begin{entrée}
{ʁo˩}{}{ⓔʁo˩}\formedesurface{ʁo˩˥}\newline
\classe{动词}\ton{L}\begin{définition}\peng{To sink (e.g. a boat slowly sinking down into a lake).}\end{définition}
\begin{définition}\pcmn{掉入、沉下去}\end{définition}
\begin{définition}\pfra{Tomber, sombrer (ex.: quelqu'un coule dans l'eau; un bateau sombre peu à peu dans le lac).}\end{définition}
\begin{exemple}\pnru{mv̩˩tɕo˥ ʁo˩}\hspace{5pt}\peng{to sink}\hspace{5pt}\pcmn{掉入}\hspace{5pt}\pfra{s'enfoncer, être englouti (par l'eau…)}\end{exemple}
\end{entrée}

\begin{entrée}
{ʁo˩β}{}{ⓔʁo˩β}\formedesurface{ʁo˩˥}\newline
\classe{动词}\ton{Lβ}\begin{définition}\peng{To form, to appear: e.g. a callus has formed.}\end{définition}
\begin{définition}\pcmn{出现、形成(如:出了茧子)}\end{définition}
\begin{définition}\pfra{Se former, apparaître (un cor se forme, un durillon se forme). Ce verbe n'a été observé qu'en association avec le mot ‘durillon, cor'.}\end{définition}
\begin{exemple}\pnru{sɯ˧ʈv̩˥ ʁo˩-ze˩! |}\hspace{5pt}\peng{A callus has formed!}\hspace{5pt}\pcmn{磨出了茧子!}\hspace{5pt}\pfra{Un durillon s'est formé!}\end{exemple}
\begin{exemple}\pnru{ʁo˩-mɤ˩-ho˥}\hspace{5pt}\peng{\_ |fg{neg} |fg{desiderative}}\hspace{5pt}\pcmn{不会出(茧子)}\hspace{5pt}\pfra{\_ |fg{neg} |fg{désidératif}}\end{exemple}
\end{entrée}

\begin{entrée}
{ʁo˩β}{}{ⓔʁo˩β}\formedesurface{ɖɯ˧ ʁo˩}\newline
\classe{量词}\ton{Lβ}\begin{définition}\peng{A sort of.}\end{définition}
\begin{définition}\pcmn{量词:种}\end{définition}
\begin{définition}\pfra{Classificateur des variétés/sortes de choses.}\end{définition}
\begin{exemple}\pnru{ɖɯ˧-ʁo˩}\hspace{5pt}\peng{one type (of clothing, food…)}\hspace{5pt}\pcmn{一种(衣服、食物……)}\hspace{5pt}\pfra{une sorte (de vêtement, de nourriture…)}\end{exemple}
\begin{exemple}\pnru{ʈʂʰɯ˧-ʁo˥}\hspace{5pt}\peng{this type (of clothing, food…)}\hspace{5pt}\pcmn{这种(衣服、食物……)}\hspace{5pt}\pfra{cette sorte (de vêtement, de nourriture…)}\end{exemple}
\end{entrée}

\begin{entrée}
{ʁo˧bv̩˧}{}{ⓔʁo˧bv̩˧}\formedesurface{ʁo˧bv̩˧}\newline
\classe{名词}\ton{M}
\paradigme{\pcmn{:} \p{}}
\begin{définition}\peng{Sprout, bud.}\end{définition}
\begin{définition}\pcmn{树的萌芽、新发出来的叶子}\end{définition}
\begin{définition}\pfra{Pousse d'arbre.}\end{définition}
\begin{exemple}\pnru{tʰo˧-ʁo˧bv˥}\hspace{5pt}\peng{bud of pine tree}\hspace{5pt}\pcmn{小松树尖}\hspace{5pt}\pfra{pousse de sapin}\end{exemple}
\begin{exemple}\pnru{tʰo˩ʂv˩-ʁo˥bv˩}\hspace{5pt}\peng{bud of pine tree}\hspace{5pt}\pcmn{小松树尖}\hspace{5pt}\pfra{pousse de sapin; littéralement «pousses d'aiguilles de sapin»}\end{exemple}
\end{entrée}

\begin{entrée}
{ʁo˧dɑ˧}{}{ⓔʁo˧dɑ˧}\formedesurface{ʁo˧dɑ˧}\newline
\classe{助词}\ton{M}\begin{définition}\peng{In front of.}\end{définition}
\begin{définition}\pcmn{前面,之前}\end{définition}
\begin{définition}\pfra{Devant, avant, auparavant.}\end{définition}
\begin{exemple}\pnru{ʂɯ˧-kʰv̩˧-ʁo˧dɑ˧}\hspace{5pt}\peng{seven years ago}\hspace{5pt}\pcmn{七年前}\hspace{5pt}\pfra{il y a sept ans}\end{exemple}
\begin{exemple}\pnru{ʁo˧dɑ˧ ɖɯ˧-so˩ ɲi˩}\hspace{5pt}\peng{the past few days}\hspace{5pt}\pcmn{前几天}\hspace{5pt}\pfra{ces derniers jours, les quelques jours passés, il y a quelques jours}\end{exemple}
\end{entrée}

\begin{entrée}
{ʁo˩di˥}{}{ⓔʁo˩di˥}\formedesurface{ʁo˩di˥}\newline
\classe{名词}\ton{LH}
\paradigme{\pcmn{:} \p{}}
\begin{définition}\peng{Mad person.}\end{définition}
\begin{définition}\pcmn{疯子}\end{définition}
\begin{définition}\pfra{Fou, aliéné.}\end{définition}
\end{entrée}

\begin{entrée}
{ʁo˧do˧}{₁}{ⓔʁo˧do˧ⓗ1}\formedesurface{ʁo˧do˧}\newline
\classe{名词}\ton{M}
1
\paradigme{\pcmn{:} \p{}}
\begin{définition}\peng{Walnut.}\end{définition}
\begin{définition}\pcmn{核桃}\end{définition}
\begin{définition}\pfra{Noix.}\end{définition}
\begin{exemple}\pnru{ʁo˧do˧ qʰwæ˧˥}\hspace{5pt}\peng{to crack walnuts}\hspace{5pt}\pcmn{开核桃}\hspace{5pt}\pfra{casser des noix}\end{exemple}
\begin{exemple}\pnru{ʁo˧do˧ ʐwæ˧}\hspace{5pt}\peng{to weigh walnuts}\hspace{5pt}\pcmn{称核桃}\hspace{5pt}\pfra{peser des noix}\end{exemple}
\end{entrée}

\begin{entrée}
{ʁo˧do˧}{₂}{ⓔʁo˧do˧ⓗ2}\formedesurface{ʁo˧do˧}\newline
\classe{名词}\ton{M}
2
\paradigme{\pcmn{:} \p{}}
\begin{définition}\peng{Interests.}\end{définition}
\begin{définition}\pcmn{利息}\end{définition}
\begin{définition}\pfra{Intérêts.}\end{définition}
\end{entrée}

\begin{entrée}
{ʁo˧dzi˥}{}{ⓔʁo˧dzi˥}\formedesurface{ʁo˧dzi˥}\newline
\classe{动词}\ton{H\#}\begin{définition}\peng{To collide, to run into.}\end{définition}
\begin{définition}\pcmn{碰撞}\end{définition}
\begin{définition}\pfra{Heurter.}\end{définition}
\begin{exemple}\pnru{le˧-ʁo˧dzi˥}\hspace{5pt}\peng{|fg{accomp}}\hspace{5pt}\pcmn{|fg{accomp}}\hspace{5pt}\pfra{|fg{accomp}}\end{exemple}
\begin{exemple}\pnru{hĩ˧ | tʰi˧-ʁo˧dzi˥ tsʰɯ˩(-ze˩)}\hspace{5pt}\peng{People have ran into one another}\hspace{5pt}\pcmn{人们(互相)碰撞}\hspace{5pt}\pfra{deux personnes se sont heurtées/sont entrées en collision}\end{exemple}
\end{entrée}

\begin{entrée}
{ʁo˧dzi˩}{}{ⓔʁo˧dzi˩}\formedesurface{ʁo˧dzi˩}\newline
\classe{名词}\ton{L\#}
\paradigme{\pcmn{:} \p{}}
\begin{définition}\peng{Tibetan.}\end{définition}
\begin{définition}\pcmn{藏族}\end{définition}
\begin{définition}\pfra{Tibétain.}\end{définition}
\end{entrée}

\begin{entrée}
{ʁo˧dzi˩-di˩}{}{ⓔʁo˧dzi˩-di˩}\newline
\classe{名词}
\sens{1}
\begin{définition}\peng{Tibet (literally: ‘the Tibetan land').}\end{définition}
\begin{définition}\pcmn{西藏}\end{définition}
\begin{définition}\pfra{Le Tibet (littéralement: «les contrées tibétaines»).}\end{définition}\sens{2}
\begin{définition}\peng{North (literally: ‘the Tibetan land').}\end{définition}
\begin{définition}\pcmn{北方(直译:‘藏族地区’)}\end{définition}
\begin{définition}\pfra{Nord (littéralement: «les contrées tibétaines»).}\end{définition}
\end{entrée}

\begin{entrée}
{ʁo˧dzi˩-tʰæ˩ɻæ˩}{}{ⓔʁo˧dzi˩-tʰæ˩ɻæ˩}\formedesurface{ʁo˧dzi˩tʰæ˩ɻæ˩}\newline
\classe{名词}\ton{L\#-}
\paradigme{\pcmn{:} \p{}}
\begin{définition}\peng{Flag, banner, pennant (literally: Tibetan writings).}\end{définition}
\begin{définition}\pcmn{旗子}\end{définition}
\begin{définition}\pfra{Drapeau, fanion (littéralement «écritures tibétaines»).}\end{définition}
\end{entrée}

\begin{entrée}
{ʁo˧dʑɯ˧}{}{ⓔʁo˧dʑɯ˧}\formedesurface{ʁo˧dʑɯ˧}\newline
\classe{名词}\ton{M}
\paradigme{\pcmn{:} \p{}}
\begin{définition}\peng{Bridle; halter.}\end{définition}
\begin{définition}\pcmn{马笼头}\end{définition}
\begin{définition}\pfra{Œillères.}\end{définition}
\begin{exemple}\pnru{ʐwæ˧-ʁo˧dʑɯ˥ (ʈʂʰɯ˧ | ʐwæ˧-ʁo˧dʑɯ˥ ɲi˩)}\hspace{5pt}\peng{horse's halter}\hspace{5pt}\pcmn{马笼头}\hspace{5pt}\pfra{œillères de cheval}\end{exemple}
\end{entrée}

\begin{entrée}
{ʁo˧ɖɯ˧˥}{}{ⓔʁo˧ɖɯ˧˥}\formedesurface{ʁo˧ɖɯ˧˥}\newline
\classe{名词}\ton{MH\#}\begin{définition}\peng{Tadpole.}\end{définition}
\begin{définition}\pcmn{蝌蚪}\end{définition}
\begin{définition}\pfra{Têtard.}\end{définition}
\end{entrée}

\begin{entrée}
{ʁo˩ɖɯ˩so˥}{}{ⓔʁo˩ɖɯ˩so˥}\formedesurface{ʁo˩ɖɯ˩so˥}\newline
\classe{助词}\ton{L+H\#}\begin{définition}\peng{In three days.}\end{définition}
\begin{définition}\pcmn{大后天}\end{définition}
\begin{définition}\pfra{Dans trois jours.}\end{définition}
\end{entrée}

\begin{entrée}
{ʁo˧gv̩\#˥}{}{ⓔʁo˧gv̩\#˥}\formedesurface{ʁo˧gv̩˧}\newline
\classe{名词}\ton{\#H}
\paradigme{\pcmn{:} \p{}}
\begin{définition}\peng{Pillow.}\end{définition}
\begin{définition}\pcmn{枕头}\end{définition}
\begin{définition}\pfra{Oreiller.}\end{définition}
\end{entrée}

\begin{entrée}
{ʁo˩hi˩}{}{ⓔʁo˩hi˩}\formedesurface{ʁo˩hi˩˥}\newline
\classe{名词}\ton{L}
\paradigme{\pcmn{:} \p{}}
\begin{définition}\peng{Molars and premolars.}\end{définition}
\begin{définition}\pcmn{臼齿+后臼齿}\end{définition}
\begin{définition}\pfra{Molaires et prémolaires.}\end{définition}
\end{entrée}

\begin{entrée}
{ʁo˧hṽ̩˧˥}{}{ⓔʁo˧hṽ̩˧˥}\formedesurface{ʁo˧hṽ̩˧˥}\newline
\classe{名词}\ton{MH\#}
\paradigme{\pcmn{:} \p{}}
\begin{définition}\peng{Hair (of the head).}\end{définition}
\begin{définition}\pcmn{头发}\end{définition}
\begin{définition}\pfra{Cheveux.}\end{définition}
\end{entrée}

\begin{entrée}
{ʁo˧ʝi˧}{}{ⓔʁo˧ʝi˧}\formedesurface{ʁo˧ʝi˧}\newline
\classe{助词}\ton{M}\begin{définition}\peng{The year after next.}\end{définition}
\begin{définition}\pcmn{后年}\end{définition}
\begin{définition}\pfra{Dans deux ans.}\end{définition}
\begin{exemple}\pnru{ʁo˧ʝi˧ ɖɯ˧-kʰv̩˧˥}\hspace{5pt}\peng{the year after next}\hspace{5pt}\pcmn{后年}\hspace{5pt}\pfra{l'année dans deux ans}\end{exemple}
\end{entrée}

\begin{entrée}
{ʁo˧kɤ˩}{}{ⓔʁo˧kɤ˩}\formedesurface{ʁo˧kɤ˩}\newline
\classe{名词}\ton{L\#}
\paradigme{\pcmn{:} \p{}}
\begin{définition}\peng{Woven headdress for women who already have children; for young women who do not yet have children, this same item is called \stylefv{/ʁo}˧ni˥/.}\end{définition}
\begin{définition}\pcmn{用来将长辫缠成盘头的黑色丝头饰(已经有孩子的女人戴的)。还没有孩子的青年女人,也戴这种头饰,但称作|fv{/ʁo˧ni˥/})}\end{définition}
\begin{définition}\pfra{Coiffe en fils tressés; chez une jeune femme (qui a atteint ses 13 ans, mais n'a pas encore d'enfants), ce même objet est désigné comme \stylefv{/ʁo}˧ni˥/.}\end{définition}
\end{entrée}

\begin{entrée}
{ʁo˩kʰv̩˩}{}{ⓔʁo˩kʰv̩˩}\formedesurface{ʁo˩kʰv̩˩˥}\newline
\classe{名词}\ton{L}\begin{définition}\peng{Sandalwood, sandlewood.}\end{définition}
\begin{définition}\pcmn{香木}\end{définition}
\begin{définition}\pfra{Arbre à épice, arbre à encens de petite taille, qui pousse en montagne, dans les espaces ombragés.}\end{définition}
\begin{exemple}\pnru{ʁo˩kʰv̩˩-si˩}\hspace{5pt}\peng{same meaning}\hspace{5pt}\pcmn{同上}\hspace{5pt}\pfra{même sens}\end{exemple}
\end{entrée}

\begin{entrée}
{ʁo˧lv̩˧}{}{ⓔʁo˧lv̩˧}\formedesurface{ʁo˧lv̩˧}\newline
\classe{动词}\ton{M}\begin{définition}\peng{To lose one's way, to become lost.}\end{définition}
\begin{définition}\pcmn{迷路}\end{définition}
\begin{définition}\pfra{Se perdre, perdre son chemin.}\end{définition}
\begin{exemple}\pnru{le˧-ʁo˧lv̩˧}\hspace{5pt}\peng{|fg{accomp}}\hspace{5pt}\pcmn{|fg{accomp}}\hspace{5pt}\pfra{|fg{accomp}}\end{exemple}
\end{entrée}

\begin{entrée}
{ʁo˧ɬi˥}{}{ⓔʁo˧ɬi˥}\formedesurface{ʁo˧ɬi˥}\newline
\classe{名词}\ton{H\#}
\paradigme{\pcmn{:} \p{}}
\begin{définition}\peng{Large needle with which animal hide can be sewn.}\end{définition}
\begin{définition}\pcmn{大粗针,用来缝琵琶肉}\end{définition}
\begin{définition}\pfra{Grosse aiguille avec laquelle on coud le paquet de viande de cochon au salpêtre qui se conserve une décennie; contrairement à ce que dit M21, n'est pas utilisé pour coudre les peaux d'animaux (mouton, bœuf, yak…): pour cela, il faut utiliser un poinçon.}\end{définition}
\begin{exemple}\pnru{ʁo˧ɬi˥, | bo˩ʈʂʰæ˧ ʐv̩˩-di˩ ɲi˩.}\hspace{5pt}\peng{The large needle is used to sew pipa meat.}\hspace{5pt}\pcmn{大针,是用来缝琵琶肉的。}\hspace{5pt}\pfra{La grosse aiguille, ça sert à coudre le cochon-conservé-entier (viande séchée «pipa»).}\end{exemple}
\end{entrée}

\begin{entrée}
{ʁo˧mæ˧}{}{ⓔʁo˧mæ˧}\formedesurface{ʁo˧mæ˧}\newline
\classe{动词}\ton{M}\begin{définition}\peng{To care for, to take care of (the aged, children, people in need…).}\end{définition}
\begin{définition}\pcmn{照顾好、管好、关心(老人、孩子、需要帮助的人)}\end{définition}
\begin{définition}\pfra{S'occuper de (personnes âgées, enfants, personnes ayant besoin d'aide).}\end{définition}
\begin{exemple}\pnru{ʁo˧mæ˧-ze˩}\hspace{5pt}\peng{|fg{pfv}}\hspace{5pt}\pcmn{管好了}\hspace{5pt}\pfra{|fg{pfv}}\end{exemple}
\begin{exemple}\pnru{hĩ˧ ʁo˧mæ˧}\hspace{5pt}\peng{to take care of people}\hspace{5pt}\pcmn{照顾人}\hspace{5pt}\pfra{s'occuper des gens}\end{exemple}
\begin{exemple}\pnru{no˧ | njɤ˧ ʁo˧mæ˧˥!}\hspace{5pt}\peng{You take good care of me! (A satisfied comment by an elderly person to someone who takes care of them)}\hspace{5pt}\pcmn{你很关心我啊!(老人满意地说)}\hspace{5pt}\pfra{Tu t'occupes de moi! / tu es aux petits soins pour moi! (Commentaire satisfait/élogieux)}\end{exemple}
\begin{exemple}\pnru{ʈʂʰɯ˧ | ɖwæ˧˥ | hĩ˧ ʁo˧mæ˧-kv̩˩!}\hspace{5pt}\peng{She is great at taking care of people! (A comment about the lady of the house)}\hspace{5pt}\pcmn{她很会管家!(对女主人的表扬)}\hspace{5pt}\pfra{Elle sait à merveille s'occuper des gens/prendre soin des gens! (Commentaire au sujet d'une maîtresse de maison)}\end{exemple}
\end{entrée}

\begin{entrée}
{ʁo˧mi˥\$}{}{ⓔʁo˧mi˥\$}\formedesurface{ʁo˧mi˥}\newline
\classe{名词}\ton{H\$}
\paradigme{\pcmn{:} \p{}}
\begin{définition}\peng{Large needle.}\end{définition}
\begin{définition}\pcmn{大针}\end{définition}
\begin{définition}\pfra{Grosse aiguille.}\end{définition}
\end{entrée}

\begin{entrée}
{ʁo˧mi˧}{}{ⓔʁo˧mi˧}\formedesurface{ʁo˧mi˧}\newline
\classe{名词}\ton{M}
\paradigme{\pcmn{:} \p{}}
\begin{définition}\peng{King; high official; chief.}\end{définition}
\begin{définition}\pcmn{国王、大臣、头领}\end{définition}
\begin{définition}\pfra{Roi; haut dignitaire, grand mandarin; chef.}\end{définition}
\begin{exemple}\pnru{ʁo˧mi˧ ʝi˧-hĩ˧ hĩ˧}\hspace{5pt}\peng{person who has a role as king/high official/chief}\hspace{5pt}\pcmn{当国王、土司、大臣、头领……的人}\hspace{5pt}\pfra{personne qui a fonction de dignitaire/chef}\end{exemple}
\begin{exemple}\pnru{kʰv̩˧mæ˧-ʁo˧mi˧}\hspace{5pt}\peng{head of (a band of) robbers}\hspace{5pt}\pcmn{土匪的头领}\hspace{5pt}\pfra{chef des brigands, capitaine d'une troupe de brigands}\end{exemple}
\end{entrée}

\begin{entrée}
{ʁo˧ni˥}{}{ⓔʁo˧ni˥}\formedesurface{ʁo˧ni˥}\newline
\classe{名词}\ton{H\#}
\paradigme{\pcmn{:} \p{}}
\begin{définition}\peng{Woven headdress for young women who do not yet have children; for women who already have children, this same item is called \stylefv{/ʁo}˧kɤ˩/.}\end{définition}
\begin{définition}\pcmn{用来将长辫缠成盘头的黑色丝头饰(还没有孩子的青年女人戴的)。已经有孩子的女人,也戴这种头饰,但称作|fv{/ʁo˧kɤ˩/})}\end{définition}
\begin{définition}\pfra{Coiffe en fils tressés des jeunes femmes qui n'ont pas encore d'enfant. Les femmes qui ont des enfants portent également cette pièce de costume, mais elle est alors désignée comme \stylefv{/ʁo}˧kɤ˩/.}\end{définition}
\end{entrée}

\begin{entrée}
{ʁo˥pv̩˩}{}{ⓔʁo˥pv̩˩}\formedesurface{ʁo˧pv̩˩}\newline
\classe{动词}\ton{H.L}
\sens{1}
\begin{définition}\peng{To come across (someone), to meet someone (typically: on a path, meeting someone walking in the other direction).}\end{définition}
\begin{définition}\pcmn{遇见(如:在路上遇见一个人)}\end{définition}
\begin{définition}\pfra{Croiser, faire la rencontre de quelqu'un (typiquement: sur un sentier, rencontrer quelqu'un qui marche en sens inverse).}\end{définition}
\begin{exemple}\pnru{tʰi˧-ʁo˥pv̩˩}\hspace{5pt}\peng{|fg{dur}}\hspace{5pt}\pcmn{|fg{dur}}\hspace{5pt}\pfra{|fg{dur}}\end{exemple}\sens{2}
\begin{définition}\peng{To intercept (bandits intercept a caravan), to attack on the road.}\end{définition}
\begin{définition}\pcmn{阻截、在路上攻击}\end{définition}
\begin{définition}\pfra{Intercepter, attaquer en route (des brigands attaquent une caravane).}\end{définition}
\end{entrée}

\begin{entrée}
{ʁo˧pʰɤ˩-ʁo˩dv̩˩lv̩˩}{}{ⓔʁo˧pʰɤ˩-ʁo˩dv̩˩lv̩˩}\formedesurface{ʁo˧pʰɤ˩ʁo˩dv̩˩lv̩˩}\newline
\classe{名词}\ton{L\#-}\begin{définition}\peng{|\stylefi{Eugeron breviscapus} (a type of daisy).}\end{définition}
\begin{définition}\pcmn{短葶飞蓬}\end{définition}
\begin{définition}\pfra{|\stylefi{Eugeron breviscapus} (une sorte de pâquerette).}\end{définition}
\end{entrée}

\begin{entrée}
{ʁo˧qɑ˥}{}{ⓔʁo˧qɑ˥}\formedesurface{ʁo˧qɑ˥}\newline
\classe{名词}\ton{H\#}
\paradigme{\pcmn{:} \p{}}
\begin{définition}\peng{Lid.}\end{définition}
\begin{définition}\pcmn{锅盖、盖子}\end{définition}
\begin{définition}\pfra{Couvercle.}\end{définition}
\end{entrée}

\begin{entrée}
{ʁo˧qʰwɤ˩}{}{ⓔʁo˧qʰwɤ˩}\newline
\classe{名词}
\sens{1}\paradigme{\pcmn{:} \p{}}
\begin{définition}\peng{Head.}\end{définition}
\begin{définition}\pcmn{头,上面部分}\end{définition}
\begin{définition}\pfra{Tête.}\end{définition}
\begin{exemple}\pnru{ʁo˧qʰwɤ˩ dzi˩}\hspace{5pt}\peng{to sit in a place of honour}\hspace{5pt}\pcmn{坐在贵宾的位置上}\hspace{5pt}\pfra{être assis à une place d'honneur}\end{exemple}
\begin{exemple}\pnru{õ˧-ʁo˥qʰwɤ˩}\hspace{5pt}\peng{one's own head}\hspace{5pt}\pcmn{自己的头}\hspace{5pt}\pfra{sa propre tête}\end{exemple}
\begin{exemple}\pnru{õ˧-ʁo˥qʰwɤ˩ lɑ˩}\hspace{5pt}\peng{to hit one's own head (context: a child hits its own head rhythmically with a stick)}\hspace{5pt}\pcmn{打自己的头(情景:一个小孩用小棍子敲打自己的头)}\hspace{5pt}\pfra{se taper sur la tête (contexte: un enfant tape en rythme sur sa propre tête avec une baguette)}\end{exemple}\sens{2}
\begin{définition}\peng{Top part, upper part.}\end{définition}
\begin{définition}\pcmn{上面部分}\end{définition}
\begin{définition}\pfra{Partie supérieure de.}\end{définition}
\end{entrée}

\begin{entrée}
{ʁo˧so˩}{}{ⓔʁo˧so˩}\formedesurface{ʁo˧so˩}\newline
\classe{助词}\ton{L\#}\begin{définition}\peng{The day after tomorrow.}\end{définition}
\begin{définition}\pcmn{后天}\end{définition}
\begin{définition}\pfra{Après-demain.}\end{définition}
\begin{exemple}\pnru{ʁo˧so˩ | -ɖɯ˧ɲi˥}\hspace{5pt}\peng{the day after tomorrow}\hspace{5pt}\pcmn{后天}\hspace{5pt}\pfra{la journée d'après-demain}\end{exemple}
\end{entrée}

\begin{entrée}
{ʁo˧ʂv̩˧}{}{ⓔʁo˧ʂv̩˧}\formedesurface{ʁo˧ʂv̩˧}\newline
\classe{动词}\ton{M}\begin{définition}\peng{To guide, to show the way.}\end{définition}
\begin{définition}\pcmn{带头、带路}\end{définition}
\begin{définition}\pfra{Conduire, guider.}\end{définition}
\begin{exemple}\pnru{ʐɤ˩mi˩ ʁo˩ʂv̩˩˥}\hspace{5pt}\peng{to show the way}\hspace{5pt}\pcmn{带路}\hspace{5pt}\pfra{montrer le chemin}\end{exemple}
\begin{exemple}\pnru{ɖɯ˧-ʑi˩-ɳɯ˩ | ʁo˧ʂv̩˧}\hspace{5pt}\peng{a family shows the way/ sets an example (which other families follow): for instance, one family begins to harvest rice, and others follow their example}\hspace{5pt}\pcmn{有一家带头:例如收庄稼时,一个家先开始收割,于是其它家庭也跟着开始收割。}\hspace{5pt}\pfra{une famille montre l'exemple: par exemple, une famille commence à récolter le riz, et les autres suivent son exemple}\end{exemple}
\begin{exemple}\pnru{ʁo˧ʂv̩˧-ze˧}\hspace{5pt}\peng{|fg{pfv}}\hspace{5pt}\pcmn{带了路}\hspace{5pt}\pfra{|fg{pfv}}\end{exemple}
\begin{exemple}\pnru{njɤ˧=ɻ̍˩-ɳɯ˩ | ʁo˧ʂv̩˧!}\hspace{5pt}\peng{We are showing the way! / We are setting an example for others! (Context: for agricultural activities, one household started first, and the others followed suit.)}\hspace{5pt}\pcmn{是我们带头的!(其他家庭是跟着我们来的!)(情景:农业活动,如:收庄稼,是一个家庭先开始的,然后其他家庭也跟着来。)}\hspace{5pt}\pfra{C'est nous qui lançons le mouvement!/ C'est nous qui donnons l'exemple aux autres! (explication: pour les travaux des champs, une maisonnée s'y attelait en premier, et les autres suivaient)}\end{exemple}
\end{entrée}

\begin{entrée}
{‑ʁo˧to˩}{}{ⓔ‑ʁo˧to˩}\newline
\classe{}
\sens{1}
\begin{définition}\peng{On top of.}\end{définition}
\begin{définition}\pcmn{……之上}\end{définition}
\begin{définition}\pfra{Sur.}\end{définition}
\begin{exemple}\pnru{qo˩qɑ˩-ʁo˩to˥}\hspace{5pt}\peng{at the top of the mountain pass, at the mountain pass}\hspace{5pt}\pcmn{垭口上}\hspace{5pt}\pfra{en haut du col}\end{exemple}
\begin{exemple}\pnru{ʁo˧qʰwɤ˩-ʁo˩to˩}\hspace{5pt}\peng{on the head, on top of the head}\hspace{5pt}\pcmn{头上}\hspace{5pt}\pfra{sur la tête, sur le sommet du crâne}\end{exemple}
\begin{exemple}\pnru{ʑi˧qʰwɤ˧-ʁo˧to˩}\hspace{5pt}\peng{on the house; e.g. there is a bird's nest on the top of the house}\hspace{5pt}\pcmn{房子上面:例如:有鸟窝在房顶上}\hspace{5pt}\pfra{sur la maison; ex.: il y a un nid d’oiseaux sur la maison}\end{exemple}\sens{2}
\begin{définition}\peng{While, at the time that, during the time that.}\end{définition}
\begin{définition}\pcmn{……的时候}\end{définition}
\begin{définition}\pfra{Pendant, au moment de.}\end{définition}
\begin{exemple}\pnru{hɑ˧dzɯ˧-ʁo˧to˩, | ʈʂʰɯ˧-ɳɯ˧ | mɤ˧-fv̩˧-ʝi˧.}\hspace{5pt}\peng{During the meal, he felt displeased/he got angry.}\hspace{5pt}\pcmn{吃饭的时候,他不高兴了/生气了。}\hspace{5pt}\pfra{Au cours du repas, il se mit en colère/ devint triste.}\end{exemple}\sens{3}
\begin{définition}\peng{To, at, towards.}\end{définition}
\begin{définition}\pcmn{向、往}\end{définition}
\begin{définition}\pfra{À l'endroit de, à l'égard de, en direction de.}\end{définition}\sens{4}
\begin{définition}\peng{Compared to.}\end{définition}
\begin{définition}\pcmn{跟……相比}\end{définition}
\begin{définition}\pfra{En comparaison de.}\end{définition}
\end{entrée}

\begin{entrée}
{‑ʁo˧tʰo˩}{}{ⓔ‑ʁo˧tʰo˩}\formedesurface{ʁo˧tʰo˩}\newline
\classe{}\ton{L\#}\begin{définition}\peng{Behind; since.}\end{définition}
\begin{définition}\pcmn{后面,自从}\end{définition}
\begin{définition}\pfra{Derrière; depuis.}\end{définition}
\begin{exemple}\pnru{ʑi˧-tʰo˩}\hspace{5pt}\peng{behind the house (=the place where there is a vegetable garden)}\hspace{5pt}\pcmn{家后院(=菜园的地方)}\hspace{5pt}\pfra{l'arrière de la maison (où il y a le potager)}\end{exemple}
\begin{exemple}\pnru{ʑi˧-ʁo˥tʰo˩}\hspace{5pt}\peng{as above: behind the house}\hspace{5pt}\pcmn{同上:家后院}\hspace{5pt}\pfra{idem: derrière la maison, l'arrière de la maison}\end{exemple}
\begin{exemple}\pnru{ʑi˧qʰwɤ˧-ʁo˧tʰo˩}\hspace{5pt}\peng{as above: behind the house}\hspace{5pt}\pcmn{同上:家后院}\hspace{5pt}\pfra{idem: derrière la maison, l'arrière de la maison}\end{exemple}
\end{entrée}

\begin{entrée}
{ʁo˧tɕʰɤ\#˥}{}{ⓔʁo˧tɕʰɤ\#˥}\formedesurface{ʁo˧tɕʰɤ˧}\newline
\classe{形容词}\ton{\#H}\begin{définition}\peng{Sharp, pointed.}\end{définition}
\begin{définition}\pcmn{尖}\end{définition}
\begin{définition}\pfra{Pointu.}\end{définition}
\begin{exemple}\pnru{ʁo˧tɕʰɤ˧∼tɕʰɤ˧-gv̩˧}\hspace{5pt}\peng{sharp}\hspace{5pt}\pcmn{尖}\hspace{5pt}\pfra{pointu}\end{exemple}
\end{entrée}

\begin{entrée}
{ʁo˧tsʰe˧ʁo\#˥}{}{ⓔʁo˧tsʰe˧ʁo\#˥}\formedesurface{ʁo˧tsʰe˧ʁo˧}\newline
\classe{名词}\ton{\#H}\begin{définition}\peng{Top (e.g. mountain top).}\end{définition}
\begin{définition}\pcmn{顶上,如:山顶}\end{définition}
\begin{définition}\pfra{Sommet de, en haut de.}\end{définition}
\begin{exemple}\pnru{ʁwɤ˧-bv̩˧ | ʁo˧tsʰe˧ʁo˧}\hspace{5pt}\peng{the top of the mountain, the mountain top}\hspace{5pt}\pcmn{山的顶,山顶}\hspace{5pt}\pfra{le sommet de la montagne}\end{exemple}
\begin{exemple}\pnru{ʁo˧qʰwɤ˩-ʁo˩tsʰe˩}\hspace{5pt}\peng{the top of the head}\hspace{5pt}\pcmn{头顶}\hspace{5pt}\pfra{le sommet de la tête}\end{exemple}
\end{entrée}

\begin{entrée}
{ʁo˧ʈv̩˧ʈv̩˥}{}{ⓔʁo˧ʈv̩˧ʈv̩˥}\formedesurface{ʁo˧ʈv̩˧ʈv̩˥}\newline
\classe{名词}\ton{H\#}
\paradigme{\pcmn{:} \p{}}
\begin{définition}\peng{Yi (derogatory term: “ungroomed heads", “messy heads").}\end{définition}
\begin{définition}\pcmn{彝族(带偏见的说法:“乱糟糟的头发”)}\end{définition}
\begin{définition}\pfra{Yi (groupe ethnique): terme péjoratif: «les hirsutes», «les ébouriffés».}\end{définition}
\end{entrée}

\begin{entrée}
{ʁo˧ʈʂe˩}{}{ⓔʁo˧ʈʂe˩}\formedesurface{ʁo˧ʈʂe˩}\newline
\classe{名词}\ton{L\#}
\paradigme{\pcmn{:} \p{}}
\begin{définition}\peng{Tinea, ringworm.}\end{définition}
\begin{définition}\pcmn{癣}\end{définition}
\begin{définition}\pfra{Teigne.}\end{définition}
\begin{exemple}\pnru{ʁu˧ʈʂɯ˩ ɖwæ˧˥ tʰi˧ di˩!}\hspace{5pt}\peng{(She/he) has a big patch of tinea!}\hspace{5pt}\pcmn{他长了很多癣!}\hspace{5pt}\pfra{(il) a vraiment la teigne à la tête!}\end{exemple}
\end{entrée}

\begin{entrée}
{ʁo˧zo\#˥}{}{ⓔʁo˧zo\#˥}\formedesurface{ʁo˧zo˧}\newline
\classe{名词}\ton{\#H}
\paradigme{\pcmn{:} \p{}}
\begin{définition}\peng{Small needle.}\end{définition}
\begin{définition}\pcmn{小针}\end{définition}
\begin{définition}\pfra{Petite aiguille.}\end{définition}
\end{entrée}

\begin{entrée}
{ʁo˥-ʐv̩˩}{}{ⓔʁo˥-ʐv̩˩}\formedesurface{ʁo˧ʐv̩˩}\newline
\classe{动词}\ton{H.L}\begin{définition}\peng{Bless and protect.}\end{définition}
\begin{définition}\pcmn{保佑}\end{définition}
\begin{définition}\pfra{Bénir et protéger.}\end{définition}
\begin{exemple}\pnru{mɤ˧-ʁo˥ʐv̩˩}\hspace{5pt}\peng{|fg{neg}}\hspace{5pt}\pcmn{|fg{neg}}\hspace{5pt}\pfra{|fg{neg}}\end{exemple}
\begin{exemple}\pnru{gɤ˧lɑ˧ | ɖɯ˧-ʁo˥ʐv̩˩-ɻ̍˩!}\hspace{5pt}\peng{May the gods bless (you/us)!}\hspace{5pt}\pcmn{菩萨保佑!}\hspace{5pt}\pfra{Que les esprits [te/nous] bénissent!}\end{exemple}
\end{entrée}

\begin{entrée}
{ʁo˧ʑi˧˥}{}{ⓔʁo˧ʑi˧˥}\formedesurface{ʁo˧ʑi˧˥}\newline
\classe{助词}\ton{MH\#}\begin{définition}\peng{As from…, starting…}\end{définition}
\begin{définition}\pcmn{从……开始}\end{définition}
\begin{définition}\pfra{À partir de.}\end{définition}
\begin{exemple}\pnru{ɖɯ˧ɬi˧mi˧-ʁo˧ʑi˧˥}\hspace{5pt}\peng{from the first month}\hspace{5pt}\pcmn{一月份开始}\hspace{5pt}\pfra{à partir du premier mois}\end{exemple}
\begin{exemple}\pnru{tsʰi˧ɲi˧-ʁo˧ʑi˧˥}\hspace{5pt}\peng{as from today}\hspace{5pt}\pcmn{今天开始}\hspace{5pt}\pfra{à partir d'aujourd'hui}\end{exemple}
\begin{exemple}\pnru{tsʰi˧ʝi˧ ɖɯ˧-kʰv̩˧˥-ʁo˧ʑi˧˥}\hspace{5pt}\peng{from this year on, as from this year}\hspace{5pt}\pcmn{今年开始}\hspace{5pt}\pfra{à partir de cette année}\end{exemple}
\begin{exemple}\pnru{gv̩˩ɬi˩mi˩-ʁo˩ʑi˩˥}\hspace{5pt}\peng{as from the 9th month, starting from the 9th month}\hspace{5pt}\pcmn{九月份开始}\hspace{5pt}\pfra{à partir du 9e mois}\end{exemple}
\begin{exemple}\pnru{ʐe˧ʈæ˥ɬi˩-ʁo˩ʑi˩}\hspace{5pt}\peng{as from the 11th month, starting from the 11th month}\hspace{5pt}\pcmn{十一月份开始}\hspace{5pt}\pfra{à partir du 11e mois}\end{exemple}
\end{entrée}

\begin{entrée}
{ʁv̩˥}{}{ⓔʁv̩˥}\formedesurface{ʁv̩˧}\newline
\classe{动词}\ton{H}\begin{définition}\peng{To swallow.}\end{définition}
\begin{définition}\pcmn{吞,咽}\end{définition}
\begin{définition}\pfra{Avaler, déglutir.}\end{définition}
\begin{exemple}\pnru{le˧-ʁv̩˥}\hspace{5pt}\peng{|fg{accomp}}\hspace{5pt}\pcmn{|fg{accomp}}\hspace{5pt}\pfra{|fg{accomp}}\end{exemple}
\end{entrée}

\begin{entrée}
{ʁv̩˧˥}{}{ⓔʁv̩˧˥}\formedesurface{ʁv̩˧˥}\newline
\classe{名词}\ton{MH}
\paradigme{\pcmn{:} \p{}}
\begin{définition}\peng{Crane (a migratory bird).}\end{définition}
\begin{définition}\pcmn{黑颈鹤(候鸟)}\end{définition}
\begin{définition}\pfra{Grue, Grus nigricollis Przew et autres espèces similaires. Il s'agit d'un oiseau migrateur.}\end{définition}
\begin{exemple}\pnru{ʁv̩˧nɑ˥mi˩}\hspace{5pt}\peng{same meaning: crane}\hspace{5pt}\pcmn{同上:黑颈鹤}\hspace{5pt}\pfra{même sens: grue}\end{exemple}
\begin{exemple}\pnru{ʁv̩˧ dzɯ˥-ze˩}\hspace{5pt}\peng{…ate the crane}\hspace{5pt}\pcmn{吃了黑颈鹤}\hspace{5pt}\pfra{…a mangé une grue}\end{exemple}
\begin{exemple}\pnru{ʁv̩˧ hwæ˥-ze˩}\hspace{5pt}\peng{…bought (a/the) crane}\hspace{5pt}\pcmn{买了黑颈鹤}\hspace{5pt}\pfra{…a acheté une grue}\end{exemple}
\end{entrée}

\begin{entrée}
{ʁv̩˧-bv̩˥}{}{ⓔʁv̩˧-bv̩˥}\formedesurface{ʁv̩˧bv̩˥}\newline
\classe{名词}\ton{H\#}
\paradigme{\pcmn{:} \p{}}
\begin{définition}\peng{Field penny-cress, |\stylefi{Thlaspi arvense}: a foetid plant with round flat pods.}\end{définition}
\begin{définition}\pcmn{菥蓂}\end{définition}
\begin{définition}\pfra{Monnoyère, tablier des champs, |\stylefi{Thlaspi arvense}; littéralement «le légume de la grue».}\end{définition}
\end{entrée}

\begin{entrée}
{ʁv̩˧mi˥\$}{}{ⓔʁv̩˧mi˥\$}\formedesurface{ʁv̩˧mi˥}\newline
\classe{名词}\ton{H\$}
\paradigme{\pcmn{:} \p{}}
\begin{définition}\peng{Female crane.}\end{définition}
\begin{définition}\pcmn{母鹤}\end{définition}
\begin{définition}\pfra{Grue femelle.}\end{définition}
\end{entrée}

\begin{entrée}
{ʁv̩˧pʰv̩\#˥}{}{ⓔʁv̩˧pʰv̩\#˥}\formedesurface{ʁv̩˧pʰv̩˧}\newline
\classe{名词}\ton{\#H}
\paradigme{\pcmn{:} \p{}}
\begin{définition}\peng{Male crane.}\end{définition}
\begin{définition}\pcmn{公鹤}\end{définition}
\begin{définition}\pfra{Grue mâle.}\end{définition}
\begin{exemple}\pnru{ʁv̩˧pʰv̩˧-ʁv̩˧mi\#˥}\hspace{5pt}\peng{male crane and female crane}\hspace{5pt}\pcmn{公鹤与母鹤}\hspace{5pt}\pfra{grue mâle et grue femelle}\end{exemple}
\end{entrée}

\begin{entrée}
{ʁv̩˧zo\#˥}{}{ⓔʁv̩˧zo\#˥}\formedesurface{ʁv̩˧zo˧}\newline
\classe{名词}\ton{\#H}
\paradigme{\pcmn{:} \p{}}
\begin{définition}\peng{Baby crane.}\end{définition}
\begin{définition}\pcmn{小鹤}\end{définition}
\begin{définition}\pfra{Enfant grue.}\end{définition}
\end{entrée}

\begin{entrée}
{ʁwæ˥}{}{ⓔʁwæ˥}\formedesurface{ʁwæ˧}\newline
\classe{名词}\ton{\#H}\begin{définition}\peng{Left (monosyllable).}\end{définition}
\begin{définition}\pcmn{左边(单音节)}\end{définition}
\begin{définition}\pfra{Gauche (monosyllabe).}\end{définition}
\end{entrée}

\begin{entrée}
{ʁwæ˧gi\#˥}{}{ⓔʁwæ˧gi\#˥}\formedesurface{ʁwæ˧gi˧}\newline
\classe{名词}\ton{\#H}\begin{définition}\peng{Left side, left.}\end{définition}
\begin{définition}\pcmn{左边}\end{définition}
\begin{définition}\pfra{Gauche, côté gauche.}\end{définition}
\end{entrée}

\begin{entrée}
{ʁwæ˧gi˧dzɤ\#˥}{}{ⓔʁwæ˧gi˧dzɤ\#˥}\formedesurface{ʁwæ˧gi˧dzɤ˧}\newline
\classe{名词}\ton{\#H}\begin{définition}\peng{Left, left side, left direction.}\end{définition}
\begin{définition}\pcmn{左、左边}\end{définition}
\begin{définition}\pfra{Gauche, côté gauche, direction de gauche.}\end{définition}
\end{entrée}

\begin{entrée}
{ʁwæ˧lo˥}{}{ⓔʁwæ˧lo˥}\formedesurface{ʁwæ˧lo˥}\newline
\classe{名词}\ton{H\#}\begin{définition}\peng{Left side, left direction.}\end{définition}
\begin{définition}\pcmn{左边,左手}\end{définition}
\begin{définition}\pfra{Gauche; direction de gauche.}\end{définition}
\end{entrée}

\begin{entrée}
{ʁwæ˧tsɯ˥}{}{ⓔʁwæ˧tsɯ˥}\formedesurface{ʁwæ˧tsɯ˥}\newline
\classe{名词}\ton{H\#}\begin{définition}\peng{Socks.}\end{définition}
\begin{définition}\pcmn{袜子}\end{définition}
\begin{définition}\pfra{Chaussettes.}\end{définition}
\end{entrée}

\begin{entrée}
{ʁwæ˧ʈʂʰe˩}{}{ⓔʁwæ˧ʈʂʰe˩}\formedesurface{ʁwæ˧ʈʂʰe˩}\newline
\classe{动词}\ton{L\#}\begin{définition}\peng{To accomplish, to complete.}\end{définition}
\begin{définition}\pcmn{完成(汉语借词)}\end{définition}
\begin{définition}\pfra{Achever, mener à terme.}\end{définition}
\begin{exemple}\pnru{le˧-ʁwæ˧ʈʂʰe˩-ze˩!}\hspace{5pt}\peng{It's complete! / It's finished!}\hspace{5pt}\pcmn{完成了!}\hspace{5pt}\pfra{C'est achevé!}\end{exemple}
\end{entrée}

\begin{entrée}
{ʁwɤ˧}{₁}{ⓔʁwɤ˧ⓗ1}\formedesurface{ʁwɤ˧}\newline
\classe{名词}\ton{M}
1
\paradigme{\pcmn{:} \p{}}
\begin{définition}\peng{Mountain.}\end{définition}
\begin{définition}\pcmn{山}\end{définition}
\begin{définition}\pfra{Montagne.}\end{définition}
\begin{exemple}\pnru{ʁo˧-ʂwæ˧}\hspace{5pt}\peng{high mountain}\hspace{5pt}\pcmn{高山}\hspace{5pt}\pfra{haute montagne}\end{exemple}
\end{entrée}

\begin{entrée}
{ʁwɤ˧}{₂}{ⓔʁwɤ˧ⓗ2}\formedesurface{ʁwɤ˧}\newline
\classe{名词}\ton{M}
2
\paradigme{\pcmn{:} \p{}}
\begin{définition}\peng{Village, hamlet.}\end{définition}
\begin{définition}\pcmn{村寨,村落}\end{définition}
\begin{définition}\pfra{Village, hameau.}\end{définition}
\begin{exemple}\pnru{ʁwɤ˧-qo˧}\hspace{5pt}\peng{in the village}\hspace{5pt}\pcmn{村子里}\hspace{5pt}\pfra{dans le village}\end{exemple}
\begin{exemple}\pnru{ɖɯ˧-ʁwɤ˧ mɤ˧-ɲi˩: | ʈʂʰɯ˧-ʁwɤ˧… | ʈʂʰɯ˧-ʁwɤ˧…}\hspace{5pt}\peng{These do not belong to the same village: here, it is the village named…; over there, it is the village named…}\hspace{5pt}\pcmn{它们不属于一个村落:这边,是……村,而那边,是……村。}\hspace{5pt}\pfra{Ce n'est pas le même endroit (littéralement: ce n'est pas le même village): ici, c'est (le village de)…; là, c'est (le village de)…}\end{exemple}
\end{entrée}

\begin{entrée}
{ʁwɤ˧}{₃}{ⓔʁwɤ˧ⓗ3}\formedesurface{ʁwɤ˧}\newline
\classe{名词}\ton{M}
3\begin{définition}\peng{Money.}\end{définition}
\begin{définition}\pcmn{钱}\end{définition}
\begin{définition}\pfra{Argent (non pas le métal, mais la monnaie).}\end{définition}
\begin{exemple}\pnru{ɖʐe˧-ʁwɤ˧}\hspace{5pt}\peng{money}\hspace{5pt}\pcmn{钱}\hspace{5pt}\pfra{argent}\end{exemple}
\end{entrée}

\begin{entrée}
{ʁwɤ˧α}{}{ⓔʁwɤ˧α}\formedesurface{ʁwɤ˧}\newline
\classe{动词}\ton{Mα}\begin{définition}\peng{To make a heap of (e.g. cereals), to pile up.}\end{définition}
\begin{définition}\pcmn{堆 (例如:堆积泥土)}\end{définition}
\begin{définition}\pfra{Amasser, entasser.}\end{définition}
\begin{exemple}\pnru{ɖɯ˧-ʁwɤ˧ tʰi˧-ʁwɤ˧}\hspace{5pt}\peng{to make a heap, to heap together}\hspace{5pt}\pcmn{堆在一起}\hspace{5pt}\pfra{faire un tas, amasser en tas}\end{exemple}
\begin{exemple}\pnru{ɖɯ˧-ʁwɤ˧ tʰi˧-tɕɯ˥}\hspace{5pt}\peng{to arrange into a heap}\hspace{5pt}\pcmn{收拾成一堆}\hspace{5pt}\pfra{ranger en tas}\end{exemple}
\begin{exemple}\pnru{tso˧∼tso˧ | gɤ˩-ʁwɤ˥ lv̩˩}\hspace{5pt}\peng{to pile up objects}\hspace{5pt}\pcmn{东西堆起来}\hspace{5pt}\pfra{entasser des objets}\end{exemple}
\begin{exemple}\pnru{ɖɯ˧-ʁwɤ˥-lv̩˩}\hspace{5pt}\peng{to make into a heap (e.g. nuts, fruit… scattered around)}\hspace{5pt}\pcmn{收拾成一堆(如:有果子散在地上,把它们堆在一起)}\hspace{5pt}\pfra{rassembler en un tas (ex.: des piments épars, des noix, des fruits…)}\end{exemple}
\begin{exemple}\pnru{tso˧∼tso˧ ʁwɤ˩}\hspace{5pt}\peng{to pile up things}\hspace{5pt}\pcmn{东西堆在一起}\hspace{5pt}\pfra{entasser des choses}\end{exemple}
\end{entrée}

\begin{entrée}
{ʁwɤ˧α}{}{ⓔʁwɤ˧α}\formedesurface{ɖɯ˧ ʁwɤ˧}\newline
\classe{量词}\ton{Mα}\begin{définition}\peng{A heap (e.g. of grains, of cut wood); literally: ‘a mountain of'.}\end{définition}
\begin{définition}\pcmn{量词:堆(一堆粮食、一堆柴……)}\end{définition}
\begin{définition}\pfra{Classificateur des tas / amoncellements (de céréales, de bois coupé…); littéralement: ‘une montagne de'.}\end{définition}
\end{entrée}

\newpage\caractère{†}

\begin{entrée}
{†ʁwɤ˩α}{}{ⓔ†ʁwɤ˩α}\formedesurface{--}\newline
\classe{动词}\ton{Lα}\begin{définition}\peng{To negotiate (monosyllabic root extracted from the reduplicated form).}\end{définition}
\begin{définition}\pcmn{商量(单音节)}\end{définition}
\begin{définition}\pfra{Discuter, négocier (racine extraite de la forme rédupliquée).}\end{définition}
\end{entrée}

\newpage\caractère{ʁ}

\begin{entrée}
{ʁwɤ˧lɑ˩-bi˩}{}{ⓔʁwɤ˧lɑ˩-bi˩}\formedesurface{ʁwɤ˧lɑ˩bi˩}\newline
\classe{名词}\ton{L\#-}\begin{définition}\peng{Walabie, a village of the Yongning plain. It is inhabited by both Na and Pumi.}\end{définition}
\begin{définition}\pcmn{瓦拉比,也称作瓦拉别(永宁的一个村落)}\end{définition}
\begin{définition}\pfra{Walabie, un village de la plaine de Yongning. Il est peuplé de Na et de Pumi.}\end{définition}
\begin{exemple}\pnru{ə˧go˧-ʁwɤ˧, | ʁwɤ˧lɑ˩-bi˩, | bæ˧ʁwɤ˧, | tʰo˧tsʰe\#˥, | pi˧tsʰe˩-di˩, | pɤ˧dʑɤ˩-di˩, | ʁwɤ˧tv̩˧}\hspace{5pt}\peng{Seven villages that one encounters as one leaves the plain of Yongning (towards the Lake); the first two are perceived as villages with a high proportion of Na members, and the third as a mostly Na village, whereas the next two are Pumi (Prinmi); the last used to be predominantly Pumi, but as of the 2010s, it had an important Chinese (Han) population.}\hspace{5pt}\pcmn{永宁背向泸沽湖方向经过的七个村落:阿公瓦、瓦拉比、巴瓦、拖其、比其地、巴甲地、瓦都。前两个村落拥有相当大的摩梭人口比例,第三主要是摩梭村。拖其、比其地、巴甲地是普米村。瓦都,过去主要是普米族村,到了2010年代有了相当多的汉族人口。}\hspace{5pt}\pfra{Sept villages au sortir de la plaine de Yongning, dans la direction du Lac; les deux premiers comportent une population na; le troisième est un village na; les deux suivants sont essentiellement des villages pumi/prinmi; le dernier était un village pumi, et a désormais (dans les années 2010) une importante population chinoise (han).}\end{exemple}
\end{entrée}

\begin{entrée}
{ʁwɤ˧qʰv̩˧}{}{ⓔʁwɤ˧qʰv̩˧}\formedesurface{ʁwɤ˧qʰv̩˧}\newline
\classe{名词}\ton{M}
\paradigme{\pcmn{:} \p{}}
\begin{définition}\peng{Cave, cavern.}\end{définition}
\begin{définition}\pcmn{山洞}\end{définition}
\begin{définition}\pfra{Grotte, caverne.}\end{définition}
\end{entrée}

\begin{entrée}
{ʁwɤ˧qʰv̩˧dʑɯ\#˥}{}{ⓔʁwɤ˧qʰv̩˧dʑɯ\#˥}\formedesurface{ʁwɤ˧qʰv̩˧dʑɯ˧}\newline
\classe{名词}\ton{\#H}
\paradigme{\pcmn{:} \p{}}
\begin{définition}\peng{Cave, cavern.}\end{définition}
\begin{définition}\pcmn{山洞}\end{définition}
\begin{définition}\pfra{Grotte, caverne (où il est facile d'entrer).}\end{définition}
\end{entrée}

\begin{entrée}
{ʁwɤ˩ʁo˩}{}{ⓔʁwɤ˩ʁo˩}\formedesurface{ʁwɤ˩ʁo˩˥}\newline
\classe{名词}\ton{L}
\paradigme{\pcmn{:} \p{}}
\begin{définition}\peng{Hillside.}\end{définition}
\begin{définition}\pcmn{山坡}\end{définition}
\begin{définition}\pfra{Collines, versant des montagnes (pas forcément très escarpé, plutôt collines que très fortes pentes).}\end{définition}
\begin{exemple}\pnru{ʁwɤ˩ʁo˩ dʑɤ˥bv̩˩ ə˩bi˩?}\hspace{5pt}\peng{You want to come and have fun on the mountain? Would you like to go and take a stroll on the mountain?}\hspace{5pt}\pcmn{去山上玩,好吗?}\hspace{5pt}\pfra{tu viens te détendre sur la montagne?}\end{exemple}
\end{entrée}

\begin{entrée}
{ʁwɤ˧∼ʁwɤ˥α}{}{ⓔʁwɤ˧∼ʁwɤ˥α}\formedesurface{ʁwɤ˧ʁwɤ˥}\newline
\classe{动词}\ton{Lα}\begin{définition}\peng{To discuss, to negotiate.}\end{définition}
\begin{définition}\pcmn{商量}\end{définition}
\begin{définition}\pfra{Discuter, négocier.}\end{définition}
\begin{exemple}\pnru{ɖɯ˧-ʁwɤ˧∼ʁwɤ˥-ɻ̍˩}\hspace{5pt}\peng{|fg{delimitative} \_ |fg{red} |fg{inceptive}}\hspace{5pt}\pcmn{商量商量}\hspace{5pt}\pfra{|fg{délimitatif} \_ |fg{red} |fg{inchoatif}}\end{exemple}
\end{entrée}

\begin{entrée}
{ʁwɤ˧tv̩˧}{}{ⓔʁwɤ˧tv̩˧}\formedesurface{ʁwɤ˧tv̩˧}\newline
\classe{名词}\ton{M}\begin{définition}\peng{A village near the Hot Springs.}\end{définition}
\begin{définition}\pcmn{瓦都村:温泉乡的一个村落}\end{définition}
\begin{définition}\pfra{Un village proche de Wenquan.}\end{définition}
\begin{exemple}\pnru{ʁwɤ˧tv̩˧-ʁwɤ˧}\hspace{5pt}\peng{same meaning}\hspace{5pt}\pcmn{同上}\hspace{5pt}\pfra{même sens}\end{exemple}
\begin{exemple}\pnru{ə˧go˧-ʁwɤ˧, | ʁwɤ˧lɑ˩-bi˩, | bæ˧ʁwɤ˧, | tʰo˧tsʰe\#˥, | pi˧tsʰe˩-di˩, | pɤ˧dʑɤ˩-di˩, | ʁwɤ˧tv̩˧}\hspace{5pt}\peng{Seven villages that one encounters as one leaves the plain of Yongning (towards the Lake); the first two are perceived as villages with a high proportion of Na members, and the third as a mostly Na village, whereas the next two are Pumi (Prinmi); the last used to be predominantly Pumi, but as of the 2010s, it had an important Chinese (Han) population.}\hspace{5pt}\pcmn{永宁背向泸沽湖方向经过的七个村落:阿公瓦、瓦拉比、巴瓦、拖其、比其地、巴甲地、瓦都。前两个村落拥有相当大的摩梭人口比例,第三主要是摩梭村。拖其、比其地、巴甲地是普米村。瓦都,过去主要是普米族村,到了2010年代有了相当多的汉族人口。}\hspace{5pt}\pfra{Sept villages au sortir de la plaine de Yongning, dans la direction du Lac; les deux premiers comportent une population na; le troisième est un village na; les deux suivants sont essentiellement des villages pumi/prinmi; le dernier était un village pumi, et a désormais (dans les années 2010) une importante population chinoise (han).}\end{exemple}
\begin{exemple}\pnru{ʁwɤ˧tv̩˧: | bɤ˧!}\hspace{5pt}\peng{/ʁwɤ˧tv̩˧/ is a Pumi village!}\hspace{5pt}\pcmn{/ʁwɤ˧tv̩˧/是一个普米族村落!}\hspace{5pt}\pfra{/ʁwɤ˧tv̩˧/, c'est un village pumi!}\end{exemple}
\end{entrée}

\begin{entrée}
{ʁwɤ˧ʐv̩\#˥}{}{ⓔʁwɤ˧ʐv̩\#˥}\formedesurface{ʁwɤ˧ʐv̩˧}\newline
\classe{名词}\ton{\#H}\begin{définition}\peng{The village of Qiansuo.}\end{définition}
\begin{définition}\pcmn{四川省凉山州盐源县前所乡}\end{définition}
\begin{définition}\pfra{Qiansuo (localité perçue par F4 comme comportant beaucoup de Yi, et des Chinois/Han, en plus des Na, d'où des contacts linguistiques/emprunts/mélanges).}\end{définition}
\begin{exemple}\pnru{ʁwɤ˧ʐv̩˧-lo˩mæ˩}\hspace{5pt}\peng{same meaning}\hspace{5pt}\pcmn{同上}\hspace{5pt}\pfra{même sens}\end{exemple}
\begin{exemple}\pnru{ʁwɤ˧ʐv̩˧, | jɤ˧qʰɑ˧ dʑɤ˥; | hwɤ˧li˧-hɑ˧ mɤ˧-dʑo˧˥!}\hspace{5pt}\peng{Adage: “In Qiansuo, bitter buckwheat grows beautifully; there's nothing for cats to eat!" Explanation: cats do not eat bitter buckwheat.}\hspace{5pt}\pcmn{俗语:“前所,苦荞(庄稼)很好。猫,没得吃!”(说明:猫不吃苦荞。)}\hspace{5pt}\pfra{dicton ancien: «A Qiansuo, le sarrasin amer pousse à merveille; les chats n'y ont rien à manger!» (Explication: les chats ne mangent pas de sarrasin. Voyant la maisonnée se régaler, le chat passe en miaulant, mais il n'y a rien pour lui: rien à son goût.)}\end{exemple}
\begin{exemple}\pnru{ʁwɤ˧ʐv˧, | jɤ˧qʰɑ˧ dʑɤ˥, | hwɤ˧li˧˥ | hɑ˧ mɤ˧-dʑo˧!}\hspace{5pt}\peng{as above}\hspace{5pt}\pcmn{同上}\hspace{5pt}\pfra{comme ci-dessus}\end{exemple}
\begin{exemple}\pnru{di˩ʁo˩˥ ◊ -dʑo˩, | gæ˧ɻæ˩! | di˩mæ˩˥ ◊ -dʑo˩, | ʁwɤ˧ʐv̩˧! |}\hspace{5pt}\peng{yyyy reporter: di˩ʁo˩, di˩mæ˩}\hspace{5pt}\pcmn{di˩ʁo˩上面, di˩mæ˩下面}\hspace{5pt}\pfra{Vers le haut, c'est le village de |fv{gæ˧ɻæ˩}! Vers le bas, c'est le village de |fv{ʁwɤ˧ʐv̩˧}! (Au sujet de l'orientation de la plaine de Yongning, et de la situation du hameau de Alawa.)}\end{exemple}
\end{entrée}

\newpage\caractère{s}

\begin{entrée}
{sɑ˥}{₁}{ⓔsɑ˥ⓗ1}\newline
\classe{名词}
1
\sens{1}\paradigme{\pcmn{:} \p{}}
\begin{définition}\peng{Flax, |\stylefi{Linum usitatissimum}.}\end{définition}
\begin{définition}\pcmn{亚麻}\end{définition}
\begin{définition}\pfra{Lin, |\stylefi{Linum usitatissimum}, plante textile et oléagineuse.}\end{définition}\sens{2}\paradigme{\pcmn{:} \p{}}
\begin{définition}\peng{Hemp, |\stylefi{Cannabis sativa}.}\end{définition}
\begin{définition}\pcmn{火麻、胡麻}\end{définition}
\begin{définition}\pfra{Chanvre, |\stylefi{Cannabis sativa}, plante textile.}\end{définition}
\end{entrée}

\begin{entrée}
{sɑ˥}{₂}{ⓔsɑ˥ⓗ2}\formedesurface{ɖɯ˧ sɑ˥}\newline
\classe{量词}\ton{H*}
2\begin{définition}\peng{A thing (no plural; only used in the negative construction “there is not a thing").}\end{définition}
\begin{définition}\pcmn{量词:样,如:‘一样都没有’}\end{définition}
\begin{définition}\pfra{Classificateur des choses/objets, utilisé seulement en tournure négative: ‘quoi que ce soit'.}\end{définition}
\begin{exemple}\pnru{ɖɯ˧-sɑ˥ | mɤ˧-dʑo˧!}\hspace{5pt}\peng{There is simply nothing at all! (A polite statement made by the host when welcoming a guest for a meal, apologizing, in self-deprecation, for not offering a meal commensurate to one's wishes.)}\hspace{5pt}\pcmn{一样也没有! / 没什么东西!(请客时的礼貌、自我贬低说法:请客人原谅菜不够丰盛)}\hspace{5pt}\pfra{Il n’y a rien du tout [à manger]! (phrase polie qd on invite quelqu'un à manger: on prie le convive d’excuser la pauvreté des mets proposés)}\end{exemple}
\end{entrée}

\begin{entrée}
{sɑ˧˥}{}{ⓔsɑ˧˥}\formedesurface{sɑ˧˥}\newline
\classe{动词}\ton{MH}\begin{définition}\peng{To deliver (something to a place).}\end{définition}
\begin{définition}\pcmn{运送(货到目的地)}\end{définition}
\begin{définition}\pfra{Livrer (à destination).}\end{définition}
\begin{exemple}\pnru{le˧-sɑ˧-tʰi˥-ki˩}\hspace{5pt}\peng{to deliver (to someone's place)}\hspace{5pt}\pcmn{送(东西到人家里)}\hspace{5pt}\pfra{même sens: livrer (un objet, une marchandise)}\end{exemple}
\end{entrée}

\begin{entrée}
{sɑ˧˥α}{}{ⓔsɑ˧˥α}\formedesurface{ɖɯ˧ sɑ˧˥}\newline
\classe{量词}\ton{MHα}\begin{définition}\peng{Classifier for salted, smoked hog legs.}\end{définition}
\begin{définition}\pcmn{量词:腊猪脚(烟熏腊猪蹄子)(一只)}\end{définition}
\begin{définition}\pfra{Classificateur des pattes de cochon conservées.}\end{définition}
\begin{exemple}\pnru{ʂe˧sɑ˩ | ɖɯ˧-sɑ˧˥}\hspace{5pt}\peng{a salted, smoked hog leg}\hspace{5pt}\pcmn{一只腊猪脚}\hspace{5pt}\pfra{une patte de cochon conservée (viande des membres du cochon, conservée --séchée-- avec l'os)}\end{exemple}
\end{entrée}

\begin{entrée}
{sɑ˧bo\#˥}{}{ⓔsɑ˧bo\#˥}\formedesurface{sɑ˧bo˧}\newline
\classe{名词}\ton{\#H}
\paradigme{\pcmn{:} \p{}}
\begin{définition}\peng{Distaff: a stick or spindle onto which wool or flax is wound for spinning.}\end{définition}
\begin{définition}\pcmn{卷线杆、拉线棒}\end{définition}
\begin{définition}\pfra{Quenouille: instrument en bois pour enrouler le fil, pour filer le chanvre.}\end{définition}
\begin{exemple}\pnru{sɑ˧bo˧-di˧˥}\hspace{5pt}\peng{same meaning}\hspace{5pt}\pcmn{同上}\hspace{5pt}\pfra{même sens}\end{exemple}
\end{entrée}

\begin{entrée}
{sɑ˩mi˩}{}{ⓔsɑ˩mi˩}\formedesurface{sɑ˩mi˩˥}\newline
\classe{名词}\ton{L}
\paradigme{\pcmn{:} \p{}}
\begin{définition}\peng{Marijuana, cannabis, |\stylefi{Cannabis indica}.}\end{définition}
\begin{définition}\pcmn{大麻}\end{définition}
\begin{définition}\pfra{Chanvre indien, kanja, marijuana, |\stylefi{Cannabis indica} (plante euphorisante/psychotrope, qui est également comestible: les Na en tiraient de l'huile).}\end{définition}
\begin{exemple}\pnru{sɑ˩mi˩-mæ˩ɻæ˥, | dzɯ˧-kv̩˩!}\hspace{5pt}\peng{Cannabis oil is edible!}\hspace{5pt}\pcmn{大麻油,是可以吃的!}\hspace{5pt}\pfra{L'huile de lin, c'est comestible/ça se mange!}\end{exemple}
\end{entrée}

\begin{entrée}
{sɑ˧pʰv̩˧˥}{}{ⓔsɑ˧pʰv̩˧˥}\formedesurface{sɑ˧pʰv̩˧˥}\newline
\classe{名词}\ton{MH\#}
\paradigme{\pcmn{:} \p{}}
\begin{définition}\peng{Thread of linen, |\stylefi{Cannabis sativa}.}\end{définition}
\begin{définition}\pcmn{麻线}\end{définition}
\begin{définition}\pfra{Fil de lin, |\stylefi{Cannabis sativa}.}\end{définition}
\begin{exemple}\pnru{sɑ˧pʰv̩˧-sɑ˧jɤ˥}\hspace{5pt}\peng{linen thread}\hspace{5pt}\pcmn{麻线}\hspace{5pt}\pfra{fil de lin}\end{exemple}
\end{entrée}

\begin{entrée}
{sɑ˧tɕɯ˧}{}{ⓔsɑ˧tɕɯ˧}\formedesurface{sɑ˧tɕɯ˧}\newline
\classe{名词}\ton{M}
\paradigme{\pcmn{:} \p{}}
\begin{définition}\peng{Vagina.}\end{définition}
\begin{définition}\pcmn{女生殖器}\end{définition}
\begin{définition}\pfra{Organe sexuel féminin, vagin (mot tabou).}\end{définition}
\end{entrée}

\begin{entrée}
{sɑ˩tsʰi˩}{₁}{ⓔsɑ˩tsʰi˩ⓗ1}\formedesurface{sɑ˩tsʰi˩˥}\newline
\classe{名词}\ton{L}
1
\paradigme{\pcmn{:} \p{}}
\begin{définition}\peng{Oar.}\end{définition}
\begin{définition}\pcmn{桨}\end{définition}
\begin{définition}\pfra{Rame.}\end{définition}
\end{entrée}

\begin{entrée}
{sɑ˩tsʰi˩}{₂}{ⓔsɑ˩tsʰi˩ⓗ2}\formedesurface{sɑ˩tsʰi˩˥}\newline
\classe{名词}\ton{L}
2\begin{définition}\peng{Wooden instrument resembling an oar, used to stir pigswill.}\end{définition}
\begin{définition}\pcmn{像桨的木头工具,来搅拌猪食}\end{définition}
\begin{définition}\pfra{Pale en bois utilisée pour touiller la pâtée des cochons; ressemble à une rame: l'ustensile de cuisine est de taille nettement plus petite que la rame des bateaux, mais de forme similaire.}\end{définition}
\end{entrée}

\begin{entrée}
{sɑ˧tsʰv̩˩}{}{ⓔsɑ˧tsʰv̩˩}\formedesurface{sɑ˧tsʰv̩˩}\newline
\classe{名词}\ton{L\#}\begin{définition}\peng{Vinegar; sour vinegar.}\end{définition}
\begin{définition}\pcmn{酸醋(汉语借词)}\end{définition}
\begin{définition}\pfra{Vinaigre.}\end{définition}
\end{entrée}

\begin{entrée}
{sæ˧tsʰɤ˩}{}{ⓔsæ˧tsʰɤ˩}\formedesurface{sæ˧tsʰɤ˩}\newline
\classe{名词}\ton{L\#}\begin{définition}\peng{Pickled vegetables.}\end{définition}
\begin{définition}\pcmn{酸菜(汉语借词)、泡菜}\end{définition}
\begin{définition}\pfra{Légumes en saumure.}\end{définition}
\end{entrée}

\begin{entrée}
{‑se}{}{ⓔ‑se}\formedesurface{--}\newline
\classe{语气助词}\ton{0?}\begin{définition}\peng{Topic marker, which singles out a precise element or set of elements and delineates it as a unit, against the background of elements that neighbour on it in the conceptual space.}\end{définition}
\begin{définition}\pcmn{主题}\end{définition}
\begin{définition}\pfra{Marqueur de topique qui délimite l'objet ou l'ensemble concerné, le faisant ressortir contre l'arrière-plan formé par les éléments proches de lui dans l'espace conceptuel. Plutôt que d'un contraste, il s'agit d'une appréhension de l'objet concerné dans son unité.}\end{définition}
\end{entrée}

\begin{entrée}
{se˥}{}{ⓔse˥}\formedesurface{se˧}\newline
\classe{动词}\ton{H}\begin{définition}\peng{To walk.}\end{définition}
\begin{définition}\pcmn{走、走路}\end{définition}
\begin{définition}\pfra{Marcher.}\end{définition}
\begin{exemple}\pnru{le˧-se˥-ze˩}\hspace{5pt}\peng{|fg{accomp} \_ |fg{pfv}}\hspace{5pt}\pcmn{走了}\hspace{5pt}\pfra{|fg{accomp} \_ |fg{pfv}}\end{exemple}
\begin{exemple}\pnru{se˧-ho˥-ze˩!}\hspace{5pt}\peng{[The baby] will soon walk / will soon be able to walk!}\hspace{5pt}\pcmn{(婴儿)很快就学会走路了!}\hspace{5pt}\pfra{[Le bébé] va bientôt marcher / va bientôt savoir marcher!}\end{exemple}
\begin{exemple}\pnru{ʐɤ˩mi˩-qo˥ | so˩-hɑ̃˩ se˩˥}\hspace{5pt}\peng{to spend three days on the road, to make a trip that lasts three days}\hspace{5pt}\pcmn{走在路上三天时间、走三天}\hspace{5pt}\pfra{passer trois nuits en route / faire un voyage qui va durer trois jours (au sujet d'un trajet de trois nuits de Lijiang à Hanoi: un train de nuit; un car de nuit le lendemain; et un second train de nuit le troisième jour)}\end{exemple}
\end{entrée}

\begin{entrée}
{se˧}{}{ⓔse˧}\formedesurface{se˧}\newline
\classe{名词}\ton{M}
\paradigme{\pcmn{:} \p{}}
\begin{définition}\peng{Himalayan goral (|\stylefi{Naemorhedus goral}), blue sheep.}\end{définition}
\begin{définition}\pcmn{岩羊}\end{définition}
\begin{définition}\pfra{|\stylefi{Naemorhedus goral}. Le même terme est employé par la locutrice pour décrire des photos de |\stylefi{Pseudois schaeferi}, sorte de bouquetin.}\end{définition}
\end{entrée}

\begin{entrée}
{‑se˩}{}{ⓔ‑se˩}\formedesurface{se˩˥}\newline
\classe{后缀}\ton{L}\begin{définition}\peng{Suffix indicating the completion of an action: the action has reached its end.}\end{définition}
\begin{définition}\pcmn{完成}\end{définition}
\begin{définition}\pfra{Suffixe indiquant l'achèvement d'une action: l'action est conduite à son terme.}\end{définition}
\begin{exemple}\pnru{se˩-ze˥!}\hspace{5pt}\peng{It's finished! / It's completed!}\hspace{5pt}\pcmn{完了!}\hspace{5pt}\pfra{C'est fini!}\end{exemple}
\begin{exemple}\pnru{no˧ | tʰi˧-dzi˩-kʰɯ˩-se˩-dʑo˩, | dʑɯ˩-tsʰi˧ ɖɯ˧-qʰwɤ˧ pʰv̩˥ | tʰi˧-ki˧!}\hspace{5pt}\peng{After you have been seated, (I) pour out a bowl of hot water (for you).}\hspace{5pt}\pcmn{让(你)坐下以后,(我)给你倒一杯开水。}\hspace{5pt}\pfra{Après que tu te sois assis, je te verse un verre d’eau chaude.}\end{exemple}
\end{entrée}

\begin{entrée}
{se˩α}{}{ⓔse˩α}\formedesurface{se˩˥}\newline
\classe{动词}\ton{Lα}\begin{définition}\peng{To finish, to complete.}\end{définition}
\begin{définition}\pcmn{完成}\end{définition}
\begin{définition}\pfra{Achever.}\end{définition}
\begin{exemple}\pnru{le˧-ʝi˥ | le˧-se˩-ze˩!}\hspace{5pt}\peng{It's done and finished! / (I) have finished (the job)!}\hspace{5pt}\pcmn{做完了! / 完成了!}\hspace{5pt}\pfra{(je) l'ai fait, j'ai fini!}\end{exemple}
\begin{exemple}\pnru{le˧-se˧∼se˥-ze˩!}\hspace{5pt}\peng{It's finished, it's completed! / It's now over and done with!}\hspace{5pt}\pcmn{完成了!}\hspace{5pt}\pfra{C'est fini, c'est achevé!}\end{exemple}
\begin{exemple}\pnru{mɤ˧-se˩}\hspace{5pt}\peng{It's not finished!}\hspace{5pt}\pcmn{没有完!}\hspace{5pt}\pfra{Ce n'est pas fini!}\end{exemple}
\begin{exemple}\pnru{se˩˥ ◊ -dʑo˩, | se˩-mɤ˩-tʰɑ˩˥! | dʑɤ˩˥ ◊ -dʑo˩, | dʑɤ˩-kʰɯ˧ tʰɑ˥!}\hspace{5pt}\peng{A comment about linguistic documentation, summarizing an explanation provided by a student: “One cannot complete the task (=one cannot collect everything: every single word, every single turn of phrase, every single story…); but one can do nice things (=collect stories which constitute complete, interesting documents)".}\hspace{5pt}\pcmn{(想做)完,但是没办法做完!不过最后还是可以做得很好!(情景:谈及收集语言的工作)}\hspace{5pt}\pfra{Parenthèse au sujet de la documentation linguistique: on ne peut pas en voir le bout (tout collecter de façon exhaustive); mais on peut réaliser de belles choses!}\end{exemple}
\end{entrée}

\begin{entrée}
{se˩di˩}{}{ⓔse˩di˩}\formedesurface{se˩di˩˥}\newline
\classe{名词}\ton{L}
\paradigme{\pcmn{:} \p{}}
\begin{définition}\peng{Saw.}\end{définition}
\begin{définition}\pcmn{锯}\end{définition}
\begin{définition}\pfra{Scie.}\end{définition}
\begin{exemple}\pnru{se˩di˩˥ | ɖɯ˩-hĩ˩˥ | ɖɯ˧-nɑ˧}\hspace{5pt}\peng{a large saw}\hspace{5pt}\pcmn{一把大锯}\hspace{5pt}\pfra{une grande scie}\end{exemple}
\begin{exemple}\pnru{se˩di˩˥ | tɕi˩-hĩ˩˥ | ɖɯ˧-nɑ˧}\hspace{5pt}\peng{a small saw}\hspace{5pt}\pcmn{一把小锯}\hspace{5pt}\pfra{une petite scie}\end{exemple}
\begin{exemple}\pnru{se˩di˩˥ | ɬi˧-hĩ˧ | ɖɯ˧-nɑ˩}\hspace{5pt}\peng{a medium-sized saw}\hspace{5pt}\pcmn{中间大小的锯子}\hspace{5pt}\pfra{une scie de taille moyenne}\end{exemple}
\end{entrée}

\begin{entrée}
{se˧-dʑæ˩ɻæ˩}{}{ⓔse˧-dʑæ˩ɻæ˩}\formedesurface{se˧dʑæ˩ɻæ˩}\newline
\classe{动词}\ton{-L}\begin{définition}\peng{To annoy, to make trouble.}\end{définition}
\begin{définition}\pcmn{烦人}\end{définition}
\begin{définition}\pfra{Déranger, ennuyer les gens, agacer.}\end{définition}
\begin{exemple}\pnru{mɤ˧-se˥-dʑæ˩ɻæ˩ gv̩˩!}\hspace{5pt}\peng{(No,) you're not annoying (us)! (Context: someone apologizes for coming round to a friend's place at a bad time; the host reassures the visitor.)}\hspace{5pt}\pcmn{不麻烦,不麻烦!}\hspace{5pt}\pfra{Non, tu ne déranges pas! (Contexte: quelqu'un s'excuse de passer à un mauvais moment; l'hôte le rassure.)}\end{exemple}
\begin{exemple}\pnru{no˧ | se˧dʑæ˩ɻæ˩-gv˩! |}\hspace{5pt}\peng{You're annoying people! You're annoying everyone!}\hspace{5pt}\pcmn{你在麻烦大家/你很烦人!}\hspace{5pt}\pfra{Tu embêtes le monde! Tu déranges tout le monde!}\end{exemple}
\end{entrée}

\begin{entrée}
{se˧gi\#˥}{}{ⓔse˧gi\#˥}\formedesurface{se˧gi˧}\newline
\classe{名词}\ton{\#H}\begin{définition}\peng{The Tibetan name of the mountain \stylefv{/kɤ}˧mv̩˧˥/ (Chinese name: Gemu).}\end{définition}
\begin{définition}\pcmn{格姆山的藏语名字}\end{définition}
\begin{définition}\pfra{Nom anciennement donné par les Tibétains à la montagne \stylefv{/kɤ}˧mv̩˧˥/ (nom chinois: Gemu).}\end{définition}
\begin{exemple}\pnru{se˧gi˧-kɤ˩mv̩˩}\hspace{5pt}\peng{same meaning}\hspace{5pt}\pcmn{同上}\hspace{5pt}\pfra{même sens}\end{exemple}
\end{entrée}

\begin{entrée}
{se˩gwɤ˩mi˥}{}{ⓔse˩gwɤ˩mi˥}\formedesurface{se˩gwɤ˩mi˥}\newline
\classe{名词}\ton{L+H\#}
\paradigme{\pcmn{:} \p{}}
\begin{définition}\peng{Vulture. This term is not restricted to female vultures, and hence does not provide an indication on sex.}\end{définition}
\begin{définition}\pcmn{雕(不仅来指母雕)}\end{définition}
\begin{définition}\pfra{Vautour. Le terme n'est pas restreint aux vautours femelles; dans l'état actuel de la langue, il ne fournit pas d'indication de sexe.}\end{définition}
\begin{exemple}\pnru{se˩gwɤ˩mi˥-pʰv̩˩}\hspace{5pt}\peng{male vulture}\hspace{5pt}\pcmn{公雕}\hspace{5pt}\pfra{vautour mâle}\end{exemple}
\begin{exemple}\pnru{se˩gwɤ˩mi˥-zo˩}\hspace{5pt}\peng{baby vulture}\hspace{5pt}\pcmn{小雕}\hspace{5pt}\pfra{petit vautour, bébé vautour}\end{exemple}
\begin{exemple}\pnru{se˩gwɤ˩mi˥-ʈʂʰɯ˩, | mi˩ ɲi˥!}\hspace{5pt}\peng{This vulture is a female!}\hspace{5pt}\pcmn{这只雕是母的!}\hspace{5pt}\pfra{Ce vautour, c'est une femelle!}\end{exemple}
\end{entrée}

\begin{entrée}
{se˧kʰɯ˩}{}{ⓔse˧kʰɯ˩}\formedesurface{se˧kʰɯ˩}\newline
\classe{名词}\ton{L\#}
\paradigme{\pcmn{:} \p{}}
\begin{définition}\peng{Satin.}\end{définition}
\begin{définition}\pcmn{缎子}\end{définition}
\begin{définition}\pfra{Satin.}\end{définition}
\begin{exemple}\pnru{se˧kʰɯ˩-ʁo˩ni˩}\hspace{5pt}\peng{satin headdress}\hspace{5pt}\pcmn{缎子发带}\hspace{5pt}\pfra{coiffe en satin}\end{exemple}
\end{entrée}

\begin{entrée}
{se˧mi\#˥}{}{ⓔse˧mi\#˥}\formedesurface{se˧mi˧}\newline
\classe{名词}\ton{\#H}
\paradigme{\pcmn{:} \p{}}
\begin{définition}\peng{Female goral (|\stylefi{Naemorhedus goral}), female blue sheep.}\end{définition}
\begin{définition}\pcmn{母岩羊}\end{définition}
\begin{définition}\pfra{|\stylefi{Naemorhedus goral} femelle.}\end{définition}
\end{entrée}

\begin{entrée}
{se˧nɑ\#˥}{}{ⓔse˧nɑ\#˥}\formedesurface{se˧nɑ˧}\newline
\classe{形容词}\ton{\#H}\begin{définition}\peng{Stingy, miserly.}\end{définition}
\begin{définition}\pcmn{吝啬}\end{définition}
\begin{définition}\pfra{Avare.}\end{définition}
\begin{exemple}\pnru{ʈʂʰɯ˧ | se˧nɑ˧-hĩ˧ ɖɯ˧-v̩˧ ɲi˩!}\hspace{5pt}\peng{It's a stingy person!}\hspace{5pt}\pcmn{他是一个吝啬的人!}\hspace{5pt}\pfra{C'est quelqu'un d'avare!}\end{exemple}
\begin{exemple}\pnru{ʈʂʰɯ˧ | ə˧-se˧nɑ˧? - se˧nɑ˧ | ʐwæ˩˥!}\hspace{5pt}\peng{Is he stingy? - Oh yes, very much so!}\hspace{5pt}\pcmn{他吝啬吗? - 非常吝啬!}\hspace{5pt}\pfra{Est-il avare? - Oui, très avare!}\end{exemple}
\end{entrée}

\begin{entrée}
{se˧pʰɤ˧}{}{ⓔse˧pʰɤ˧}\formedesurface{se˧pʰɤ˧}\newline
\classe{名词}\ton{M}
\paradigme{\pcmn{:} \p{}}
\begin{définition}\peng{Fuss.}\end{définition}
\begin{définition}\pcmn{大惊小怪,麻烦}\end{définition}
\begin{définition}\pfra{Complications.}\end{définition}
\begin{exemple}\pnru{se˧pʰɤ˧ ʝi˧}\hspace{5pt}\peng{to make a big fuss about something}\hspace{5pt}\pcmn{小事大作}\hspace{5pt}\pfra{se faire toute une affaire de quelque chose, s’en faire au point de porter comme une pierre dans le cœur}\end{exemple}
\end{entrée}

\begin{entrée}
{se˧pʰv̩\#˥}{}{ⓔse˧pʰv̩\#˥}\formedesurface{se˧pʰv̩˧}\newline
\classe{名词}\ton{\#H}
\paradigme{\pcmn{:} \p{}}
\begin{définition}\peng{Male goral (|\stylefi{Naemorhedus goral}), male blue sheep.}\end{définition}
\begin{définition}\pcmn{公岩羊}\end{définition}
\begin{définition}\pfra{|\stylefi{Naemorhedus goral} mâle.}\end{définition}
\end{entrée}

\begin{entrée}
{se˧ʂɯ˩}{}{ⓔse˧ʂɯ˩}\formedesurface{se˧ʂɯ˩}\newline
\classe{动词}\ton{L\#}\begin{définition}\peng{To waste.}\end{définition}
\begin{définition}\pcmn{浪费}\end{définition}
\begin{définition}\pfra{Gaspiller.}\end{définition}
\begin{exemple}\pnru{ɖwæ˧˥ | se˧ʂɯ˩!}\hspace{5pt}\peng{It's a waste!}\hspace{5pt}\pcmn{很浪费! / 太浪费了!}\hspace{5pt}\pfra{C'est du gaspillage!}\end{exemple}
\end{entrée}

\begin{entrée}
{se˧tʰo˩}{}{ⓔse˧tʰo˩}\formedesurface{se˧tʰo˩}\newline
\classe{名词}\ton{L}
\paradigme{\pcmn{:} \p{}}
\begin{définition}\peng{Tenon.}\end{définition}
\begin{définition}\pcmn{榫头(汉语借词)}\end{définition}
\begin{définition}\pfra{Tenon.}\end{définition}
\begin{exemple}\pnru{se˧tʰo˩ | ɖɯ˧-ɭɯ˧}\hspace{5pt}\peng{a tenon}\hspace{5pt}\pcmn{一个榫头}\hspace{5pt}\pfra{un tenon}\end{exemple}
\end{entrée}

\begin{entrée}
{se˧zo\#˥}{}{ⓔse˧zo\#˥}\formedesurface{se˧zo˧}\newline
\classe{名词}\ton{\#H}
\paradigme{\pcmn{:} \p{}}
\begin{définition}\peng{Baby goral, baby blue sheep.}\end{définition}
\begin{définition}\pcmn{小岩羊}\end{définition}
\begin{définition}\pfra{Petit de |\stylefi{Naemorhedus goral}.}\end{définition}
\end{entrée}

\begin{entrée}
{se˧ʐɯ˩}{}{ⓔse˧ʐɯ˩}\formedesurface{se˧ʐɯ˩}\newline
\classe{名词}\ton{L\#}\begin{définition}\peng{Birthday.}\end{définition}
\begin{définition}\pcmn{生日(汉语借词)}\end{définition}
\begin{définition}\pfra{Anniversaire.}\end{définition}
\begin{exemple}\pnru{se˧ʐɯ˩ ko˩}\hspace{5pt}\peng{to celebrate a birthday}\hspace{5pt}\pcmn{过生日}\hspace{5pt}\pfra{fêter un anniversaire}\end{exemple}
\end{entrée}

\begin{entrée}
{sɤ˥}{}{ⓔsɤ˥}\formedesurface{sɤ˧}\newline
\classe{名词}\ton{\#H}
\paradigme{\pcmn{:} \p{}}
\begin{définition}\peng{Blood.}\end{définition}
\begin{définition}\pcmn{血}\end{définition}
\begin{définition}\pfra{Sang.}\end{définition}
\end{entrée}

\begin{entrée}
{sɤ˩˥}{}{ⓔsɤ˩˥}\formedesurface{sɤ˩˥}\newline
\classe{名词}\ton{LH}
\paradigme{\pcmn{:} \p{}}
\begin{définition}\peng{Mole; pigmented naevus.}\end{définition}
\begin{définition}\pcmn{黑痣}\end{définition}
\begin{définition}\pfra{Grain de beauté.}\end{définition}
\end{entrée}

\begin{entrée}
{sɤ˧ɭɯ˩}{}{ⓔsɤ˧ɭɯ˩}\formedesurface{sɤ˧ɭɯ˩}\newline
\classe{名词}\ton{L\#}
\paradigme{\pcmn{:} \p{}}
\begin{définition}\peng{Pear.}\end{définition}
\begin{définition}\pcmn{梨子}\end{définition}
\begin{définition}\pfra{Poire.}\end{définition}
\end{entrée}

\begin{entrée}
{sɤ˧sɤ˧˥}{}{ⓔsɤ˧sɤ˧˥}\formedesurface{sɤ˧sɤ˧˥}\newline
\classe{形容词}\ton{MH\#}\begin{définition}\peng{Pleasant (circumstances).}\end{définition}
\begin{définition}\pcmn{舒畅}\end{définition}
\begin{définition}\pfra{Agréable, plaisant (circonstances).}\end{définition}
\begin{exemple}\pnru{si˧dzi˩-ʈʰæ˩qo˩dzi˩, | sɤ˧sɤ˧˥ | ʐwæ˩˥!}\hspace{5pt}\peng{Being seated under this tree is especially pleasant!}\hspace{5pt}\pcmn{在树下坐着,感到很舒畅!}\hspace{5pt}\pfra{assis sous cet arbre, c'est le bonheur!}\end{exemple}
\begin{exemple}\pnru{ʈʂʰɯ˧-ɳɯ˧ | ɖɯ˧-ɖʐɯ˩ gwɤ˩-dʑo˩, | sɤ˧sɤ˧˥ | ʐwæ˩˥!}\hspace{5pt}\peng{He has sung for a while; it was really pleasant!}\hspace{5pt}\pcmn{他唱了一会,真舒畅!}\hspace{5pt}\pfra{Il a chanté un moment; c'était vraiment plaisant!}\end{exemple}
\end{entrée}

\begin{entrée}
{sɤ˧tʰo˧˥}{}{ⓔsɤ˧tʰo˧˥}\formedesurface{sɤ˧tʰo˧˥}\newline
\classe{名词}\ton{MH\#}
\paradigme{\pcmn{:} \p{}}
\begin{définition}\peng{A type of pine.}\end{définition}
\begin{définition}\pcmn{一种松树}\end{définition}
\begin{définition}\pfra{Sorte de pin.}\end{définition}
\begin{exemple}\pnru{sɤ˧tʰo˧-dzi˧˥}\hspace{5pt}\peng{same meaning}\hspace{5pt}\pcmn{同上}\hspace{5pt}\pfra{même sens (désigne une espèce de pin)}\end{exemple}
\end{entrée}

\begin{entrée}
{sɤ˧tsi˥}{}{ⓔsɤ˧tsi˥}\formedesurface{sɤ˧tsi˥}\newline
\classe{名词}\ton{H\#}
\paradigme{\pcmn{:} \p{}}
\begin{définition}\peng{Vein.}\end{définition}
\begin{définition}\pcmn{血管}\end{définition}
\begin{définition}\pfra{Veines.}\end{définition}
\end{entrée}

\begin{entrée}
{si˥}{}{ⓔsi˥}\formedesurface{si˧}\newline
\classe{名词}\ton{\#H}
\paradigme{\pcmn{:} \p{}}
\begin{définition}\peng{Wood.}\end{définition}
\begin{définition}\pcmn{木头}\end{définition}
\begin{définition}\pfra{Bois.}\end{définition}
\begin{exemple}\pnru{si˧-mo˩}\hspace{5pt}\peng{dead wood}\hspace{5pt}\pcmn{枯木}\hspace{5pt}\pfra{bois mort}\end{exemple}
\end{entrée}

\begin{entrée}
{si˧˥}{₁}{ⓔsi˧˥ⓗ1}\formedesurface{si˧˥}\newline
\classe{动词}\ton{MH}
1\begin{définition}\peng{To shave (the beard or the head); to scrub (e.g. to scrub earth off vegetables).}\end{définition}
\begin{définition}\pcmn{剔,刮}\end{définition}
\begin{définition}\pfra{Raser (la barbe); gratter (la terre collée à un champignon).}\end{définition}
\begin{exemple}\pnru{mo˧ si˥}\hspace{5pt}\peng{to scrub mushrooms (to take off the earth, moss…)}\hspace{5pt}\pcmn{刮菌子(刮掉污垢)}\hspace{5pt}\pfra{gratter des champignons, pour en retirer la terre, les aiguilles de pin… Cela se fait souvent à sec.}\end{exemple}
\begin{exemple}\pnru{mv̩˧tsɯ˧ si˥}\hspace{5pt}\peng{to shave (one's) beard}\hspace{5pt}\pcmn{刮胡子}\hspace{5pt}\pfra{raser la barbe}\end{exemple}
\begin{exemple}\pnru{ʁo˧qʰwɤ˩ si˩}\hspace{5pt}\peng{to shave one's head}\hspace{5pt}\pcmn{剃头}\hspace{5pt}\pfra{raser le crâne, raser la tête}\end{exemple}
\begin{exemple}\pnru{ʁo˧qʰwɤ˩ si˩-di˩}\hspace{5pt}\peng{Razor: object used to shave the head or the beard. (In the main consultant's youth, not every family had a razor. One would call someone to the house to shave the head or the beard. It was mostly monks and elderly people who had their heads and beards shaved.)}\hspace{5pt}\pcmn{理发刮刀}\hspace{5pt}\pfra{Rasoir, objet utilisé pour raser le crâne, mais aussi pour raser la barbe. Dans la jeunesse de F4, il existait quelques rasoirs; chaque famille n'en possédait pas. On faisait venir une personne sachant manier l'instrument. Ce sont les moines et les vieilles personnes qui faisaient le plus fréquemment appel à ces services.}\end{exemple}
\end{entrée}

\begin{entrée}
{si˧˥}{₂}{ⓔsi˧˥ⓗ2}\formedesurface{si˧˥}\newline
\classe{动词}\ton{MH}
2\begin{définition}\peng{To murder, to kill (a human being).}\end{définition}
\begin{définition}\pcmn{杀(人)}\end{définition}
\begin{définition}\pfra{Assassiner, tuer (un homme).}\end{définition}
\begin{exemple}\pnru{hĩ˧ si˩}\hspace{5pt}\peng{to kill someone}\hspace{5pt}\pcmn{杀人}\hspace{5pt}\pfra{tuer quelqu'un, assassiner quelqu'un}\end{exemple}
\end{entrée}

\begin{entrée}
{si˧α}{}{ⓔsi˧α}\formedesurface{si˧}\newline
\classe{动词}\ton{Mα}\begin{définition}\peng{To choose, to select.}\end{définition}
\begin{définition}\pcmn{挑选}\end{définition}
\begin{définition}\pfra{Choisir.}\end{définition}
\begin{exemple}\pnru{le˧-si˧-ze˧}\hspace{5pt}\peng{|fg{accomp} \_ |fg{pfv}}\hspace{5pt}\pcmn{选了}\hspace{5pt}\pfra{|fg{accomp} \_ |fg{pfv}}\end{exemple}
\begin{exemple}\pnru{no˧ si˧-bi˧!}\hspace{5pt}\peng{You choose! / Go ahead and choose!}\hspace{5pt}\pcmn{你要选!}\hspace{5pt}\pfra{Tu choisis! / A toi le choix!}\end{exemple}
\begin{exemple}\pnru{njɤ˧-ɳɯ˧ si˧-bi˧!}\hspace{5pt}\peng{I choose! / Let me choose!}\hspace{5pt}\pcmn{是我来选!}\hspace{5pt}\pfra{C'est moi qui choisis!}\end{exemple}
\begin{exemple}\pnru{le˧-si˥∼si˩}\hspace{5pt}\peng{|fg{accomp} \_ |fg{red}}\hspace{5pt}\pcmn{|fg{accomp} \_ |fg{red}}\hspace{5pt}\pfra{|fg{accomp} \_ |fg{red}}\end{exemple}
\begin{exemple}\pnru{tso˧∼tso˧ si˩}\hspace{5pt}\peng{to choose things}\hspace{5pt}\pcmn{选东西}\hspace{5pt}\pfra{choisir des choses}\end{exemple}
\begin{exemple}\pnru{tso˧∼tso˧ si˧∼si˥}\hspace{5pt}\peng{to choose things}\hspace{5pt}\pcmn{选选东西}\hspace{5pt}\pfra{choisir des choses}\end{exemple}
\begin{exemple}\pnru{dʑɤ˩-hĩ˥ | si˧}\hspace{5pt}\peng{to choose good ones}\hspace{5pt}\pcmn{挑好的}\hspace{5pt}\pfra{choisir les plus beaux; en choisir de beaux (par exemple: sur la montagne, lorsqu'on choisit des arbres à abattre pour donner du bois du charpente)}\end{exemple}
\end{entrée}

\begin{entrée}
{si˩˥}{}{ⓔsi˩˥}\formedesurface{si˩˥}\newline
\classe{名词}\ton{LH}
\paradigme{\pcmn{:} \p{}}
\begin{définition}\peng{Liver.}\end{définition}
\begin{définition}\pcmn{肝}\end{définition}
\begin{définition}\pfra{Foie.}\end{définition}
\end{entrée}

\begin{entrée}
{si˧bv̩˧}{}{ⓔsi˧bv̩˧}\formedesurface{si˧bv̩˧}\newline
\classe{名词}\ton{M}
\paradigme{\pcmn{:} \p{}}
\begin{définition}\peng{Evil spirit.}\end{définition}
\begin{définition}\pcmn{鬼}\end{définition}
\begin{définition}\pfra{Démon (forme obtenue par élicitation; nettement moins courante que la forme féminine).}\end{définition}
\end{entrée}

\begin{entrée}
{si˧bv̩˧-mi\#˥}{}{ⓔsi˧bv̩˧-mi\#˥}\formedesurface{si˧bv̩˧mi˧}\newline
\classe{名词}\ton{\#H}
\paradigme{\pcmn{:} \p{}}
\begin{définition}\peng{Evil spirit (female).}\end{définition}
\begin{définition}\pcmn{妖精}\end{définition}
\begin{définition}\pfra{Démone.}\end{définition}
\end{entrée}

\begin{entrée}
{si˧bv̩˧-zo\#˥}{}{ⓔsi˧bv̩˧-zo\#˥}\formedesurface{si˧bv̩˧zo˧}\newline
\classe{名词}\ton{\#H}
\paradigme{\pcmn{:} \p{}}
\begin{définition}\peng{Evil spirit (masculine).}\end{définition}
\begin{définition}\pcmn{鬼}\end{définition}
\begin{définition}\pfra{Démon masculin (forme élicitée, sur la base de la forme féminine; est un mot qui existe, mais peu courant).}\end{définition}
\end{entrée}

\begin{entrée}
{si˧ɕi˧˥}{}{ⓔsi˧ɕi˧˥}\formedesurface{si˧ɕi˧˥}\newline
\classe{名词}\ton{MH\#}
\paradigme{\pcmn{:} \p{}}
\begin{définition}\peng{Forest.}\end{définition}
\begin{définition}\pcmn{森林}\end{définition}
\begin{définition}\pfra{Forêt (clairsemée).}\end{définition}
\begin{exemple}\pnru{tʰo˧ɕi˧˥}\hspace{5pt}\peng{pine forest}\hspace{5pt}\pcmn{松树森林}\hspace{5pt}\pfra{forêt de pins}\end{exemple}
\end{entrée}

\begin{entrée}
{si˧dzi˩}{}{ⓔsi˧dzi˩}\formedesurface{si˧dzi˩}\newline
\classe{名词}\ton{L\#}
\paradigme{\pcmn{:} \p{}}
\begin{définition}\peng{Tree.}\end{définition}
\begin{définition}\pcmn{树}\end{définition}
\begin{définition}\pfra{Arbre.}\end{définition}
\end{entrée}

\begin{entrée}
{si˧dzi˩-mv̩˩tsɯ˩}{}{ⓔsi˧dzi˩-mv̩˩tsɯ˩}\formedesurface{si˧dzi˩mv̩˩tsɯ˩}\newline
\classe{名词}\ton{\#L-L}\begin{définition}\peng{Radicel, rootlet, small root.}\end{définition}
\begin{définition}\pcmn{胚根}\end{définition}
\begin{définition}\pfra{Radicelle, petites racine.}\end{définition}
\end{entrée}

\begin{entrée}
{si˧dʑɯ˥}{}{ⓔsi˧dʑɯ˥}\formedesurface{si˧dʑɯ˥}\newline
\classe{名词}\ton{H\#}
\paradigme{\pcmn{:} \p{}}
\begin{définition}\peng{Kindling.}\end{définition}
\begin{définition}\pcmn{火煤、火捻、火种、劈柴、引柴}\end{définition}
\begin{définition}\pfra{Petit bois, pour faire démarrer le feu; à Yongning, ce qu'on utilise: des morceaux de pin gorgés de résine, utilisés spécialement à cet effet.}\end{définition}
\end{entrée}

\begin{entrée}
{si˧gɯ˧}{}{ⓔsi˧gɯ˧}\formedesurface{si˧gɯ˧}\newline
\classe{名词}\ton{M}
\paradigme{\pcmn{:} \p{}}
\begin{définition}\peng{Lion.}\end{définition}
\begin{définition}\pcmn{狮子}\end{définition}
\begin{définition}\pfra{Lion.}\end{définition}
\end{entrée}

\begin{entrée}
{si˧gɯ˧-mi˩}{}{ⓔsi˧gɯ˧-mi˩}\formedesurface{si˧gɯ˧mi˩}\newline
\classe{名词}\ton{-L}
\paradigme{\pcmn{:} \p{}}
\begin{définition}\peng{Lioness.}\end{définition}
\begin{définition}\pcmn{母狮}\end{définition}
\begin{définition}\pfra{Lionne.}\end{définition}
\end{entrée}

\begin{entrée}
{si˧gɯ˧-pʰv̩\#˥}{}{ⓔsi˧gɯ˧-pʰv̩\#˥}\formedesurface{si˧gɯ˧pʰv̩˧}\newline
\classe{名词}\ton{\#H}
\paradigme{\pcmn{:} \p{}}
\begin{définition}\peng{Male lion.}\end{définition}
\begin{définition}\pcmn{公狮子}\end{définition}
\begin{définition}\pfra{Lion (mâle).}\end{définition}
\end{entrée}

\begin{entrée}
{si˧gɯ˧-tsʰo\#˥}{}{ⓔsi˧gɯ˧-tsʰo\#˥}\formedesurface{si˧gɯ˧tsʰo˧}\newline
\classe{名词}\ton{\#H}\begin{définition}\peng{Lion Dance: a show organized for the feudal lord.}\end{définition}
\begin{définition}\pcmn{狮子舞:土司准备的礼仪性表演。土司也亲自参与舞蹈。}\end{définition}
\begin{définition}\pfra{Danse du Lion: spectacle masqué, commandité par le seigneur féodal, qui participait lui-même à certaines des danses.}\end{définition}
\end{entrée}

\begin{entrée}
{si˧gɯ˧-zo\#˥}{}{ⓔsi˧gɯ˧-zo\#˥}\formedesurface{si˧gɯ˧zo˧}\newline
\classe{名词}\ton{\#H}
\paradigme{\pcmn{:} \p{}}
\begin{définition}\peng{Lion cub.}\end{définition}
\begin{définition}\pcmn{小狮子}\end{définition}
\begin{définition}\pfra{Lionceau, petit lion.}\end{définition}
\end{entrée}

\begin{entrée}
{si˧kɤ˧˥}{}{ⓔsi˧kɤ˧˥}\formedesurface{si˧kɤ˧˥}\newline
\classe{名词}\ton{MH\#}
\paradigme{\pcmn{:} \p{}}
\begin{définition}\peng{Branch; rod, stick.}\end{définition}
\begin{définition}\pcmn{树枝、小树枝,棍子}\end{définition}
\begin{définition}\pfra{Branche; petite branche; bâton, gourdin, canne pour marcher.}\end{définition}
\end{entrée}

\begin{entrée}
{si˩kwæ˧}{}{ⓔsi˩kwæ˧}\formedesurface{si˩kwæ˥}\newline
\classe{动词}\ton{LM}\begin{définition}\peng{Straw (for drinking liquids).}\end{définition}
\begin{définition}\pcmn{吸管(汉语借词)}\end{définition}
\begin{définition}\pfra{Paille (pour boire un liquide à la paille).}\end{définition}
\begin{exemple}\pnru{si˩kwæ˧-qo˧-ɳɯ˧ | ʈʰɯ˩˥}\hspace{5pt}\peng{to drink with a straw}\hspace{5pt}\pcmn{用吸管喝}\hspace{5pt}\pfra{boire à la paille}\end{exemple}
\end{entrée}

\begin{entrée}
{si˧kwɤ˩}{}{ⓔsi˧kwɤ˩}\formedesurface{si˧kwɤ˩}\newline
\classe{名词}\ton{L\#}
\paradigme{\pcmn{:} \p{}}
\begin{définition}\peng{Wooden structure (of a house), carpentry.}\end{définition}
\begin{définition}\pcmn{木头框架,如:房子的木头框架}\end{définition}
\begin{définition}\pfra{Structure, charpente, gros œuvre en bois (d'une maison).}\end{définition}
\begin{exemple}\pnru{ʑi˧mi˧-si˧kwɤ˩}\hspace{5pt}\peng{a house's carpentry, a house's wooden structure}\hspace{5pt}\pcmn{房子的木头框架}\hspace{5pt}\pfra{la charpente d'une maison}\end{exemple}
\end{entrée}

\begin{entrée}
{si˧kʰɯ\#˥}{}{ⓔsi˧kʰɯ\#˥}\formedesurface{si˧kʰɯ˧}\newline
\classe{名词}\ton{H\#}\begin{définition}\peng{An unidentified plant; local name: \stylefn{根三香.}}\end{définition}
\begin{définition}\pcmn{色疙瘩}\end{définition}
\begin{définition}\pfra{Plante identifiée par Li Dazhu (dans son ouvrage au sujet de la médecine mosuo) par le nom local \stylefn{根三香,} dont un équivalent serait \stylefn{色疙瘩.} Aucun de ces deux termes n'a pu être retrouvé dans les nomenclatures.}\end{définition}
\begin{exemple}\pnru{si˧kʰɯ˧-bæ˥bæ˩}\hspace{5pt}\peng{the flower of this plant}\hspace{5pt}\pcmn{色疙瘩花}\hspace{5pt}\pfra{fleur de cette plante}\end{exemple}
\end{entrée}

\begin{entrée}
{si˧kʰɯ˧-ɭɯ˧bv̩˥}{}{ⓔsi˧kʰɯ˧-ɭɯ˧bv̩˥}\formedesurface{si˧kʰɯ˧ɭɯ˧bv̩˥}\newline
\classe{名词}\ton{H\#}
\paradigme{\pcmn{:} \p{}}
\begin{définition}\peng{White Chinese herbaceous peony, |\stylefi{Paeonia lactiflora}.}\end{définition}
\begin{définition}\pcmn{白芍药}\end{définition}
\begin{définition}\pfra{Pivoine blanche de Chine, |\stylefi{Paeonia lactiflora}.}\end{définition}
\end{entrée}

\begin{entrée}
{si˧ɭɯ\#˥}{}{ⓔsi˧ɭɯ\#˥}\formedesurface{si˧ɭɯ˧}\newline
\classe{名词}\ton{\#H}
\paradigme{\pcmn{:} \p{}}
\begin{définition}\peng{Timber, lumber.}\end{définition}
\begin{définition}\pcmn{木材、木料}\end{définition}
\begin{définition}\pfra{Bois de charpente, tronc coupé.}\end{définition}
\begin{exemple}\pnru{ʑi˧mi˧-si˩ɭɯ˩}\hspace{5pt}\peng{lumber for the construction of the main building of a Na farm}\hspace{5pt}\pcmn{建主房的木材}\hspace{5pt}\pfra{bois de charpente utilisé pour le bâtiment principal}\end{exemple}
\begin{exemple}\pnru{ʑi˧qʰwɤ˧-si˧ɭɯ\#˥}\hspace{5pt}\peng{lumber for the construction of a building}\hspace{5pt}\pcmn{建房子的木材}\hspace{5pt}\pfra{bois de charpente, bois pour la construction d'un bâtiment}\end{exemple}
\begin{exemple}\pnru{si˧ɭɯ˧-ʑi˧qʰwɤ˥}\hspace{5pt}\peng{Lumber house, log house: house made of wood. (Na houses used to be made of lumber.)}\hspace{5pt}\pcmn{木房}\hspace{5pt}\pfra{Maison en bois.}\end{exemple}
\end{entrée}

\begin{entrée}
{si˧nɑ˥}{}{ⓔsi˧nɑ˥}\formedesurface{si˧nɑ˥}\newline
\classe{名词}\ton{H\#}
\paradigme{\pcmn{:} \p{}}
\begin{définition}\peng{Deep forest.}\end{définition}
\begin{définition}\pcmn{森林深处(难走路)}\end{définition}
\begin{définition}\pfra{Forêt épaisse.}\end{définition}
\end{entrée}

\begin{entrée}
{si˩qʰɑ˩}{}{ⓔsi˩qʰɑ˩}\formedesurface{si˩qʰɑ˩˥}\newline
\classe{名词}\ton{L}\begin{définition}\peng{Plum tree, prune tree.}\end{définition}
\begin{définition}\pcmn{梅子}\end{définition}
\begin{définition}\pfra{Abricotier du Japon (essence bien représentée à Yongning).}\end{définition}
\begin{exemple}\pnru{si˩qʰɑ˩-dʑɯ˩}\hspace{5pt}\peng{a liquid prepared from plums, which served as an equivalent of vinegar (vinegar was introduced late: it was bought in Chinese areas)}\hspace{5pt}\pcmn{用梅子做的一种汁,用法类似于醋。过去,永宁没有醋,醋是从内地买来的。}\hspace{5pt}\pfra{un liquide préparé à base de prunelles d'abricotier du Japon, servant d'équivalent de vinaigre (le vinaigre a été introduit tardivement; il était acheté en pays chinois)}\end{exemple}
\end{entrée}

\begin{entrée}
{si˧qʰwæ˩}{}{ⓔsi˧qʰwæ˩}\formedesurface{si˧qʰwæ˩}\newline
\classe{动词}\ton{L\#}\begin{définition}\peng{To chop wood, to split wood.}\end{définition}
\begin{définition}\pcmn{劈、剖}\end{définition}
\begin{définition}\pfra{Couper du bois, fendre du bois.}\end{définition}
\end{entrée}

\begin{entrée}
{si˧-ʁæ˧bæ˥}{}{ⓔsi˧-ʁæ˧bæ˥}\formedesurface{si˧ʁæ˧bæ˥}\newline
\classe{名词}\ton{-H\#}
\paradigme{\pcmn{:} \p{}}
\begin{définition}\peng{Wooden plate.}\end{définition}
\begin{définition}\pcmn{木盘子}\end{définition}
\begin{définition}\pfra{Assiette en bois.}\end{définition}
\end{entrée}

\begin{entrée}
{si˧ʁo\#˥}{}{ⓔsi˧ʁo\#˥}\formedesurface{si˧ʁo˧}\newline
\classe{名词}\ton{\#H}
\paradigme{\pcmn{:} \p{}}
\begin{définition}\peng{Fruit tree.}\end{définition}
\begin{définition}\pcmn{果树}\end{définition}
\begin{définition}\pfra{Arbre fruitier.}\end{définition}
\end{entrée}

\begin{entrée}
{si˧ʁo˧si˧ɭɯ\#˥}{}{ⓔsi˧ʁo˧si˧ɭɯ\#˥}\formedesurface{si˧ʁo˧si˧ɭɯ˧}\newline
\classe{名词}\ton{\#H}
\paradigme{\pcmn{:} \p{}}
\begin{définition}\peng{Fruit.}\end{définition}
\begin{définition}\pcmn{水果}\end{définition}
\begin{définition}\pfra{Fruit.}\end{définition}
\begin{exemple}\pnru{si˧ʁo˧si˧ɭɯ˧ ɲi˥}\hspace{5pt}\peng{|fg{cop}}\hspace{5pt}\pcmn{是水果。}\hspace{5pt}\pfra{|fg{cop}}\end{exemple}
\end{entrée}

\begin{entrée}
{si˧-sæ˥qʰv̩˩}{}{ⓔsi˧-sæ˥qʰv̩˩}\formedesurface{si˧sæ˥qʰv̩˩}\newline
\classe{名词}\ton{\#H-}\begin{définition}\peng{Birch, |\stylefi{Betula szechuanica (Betula Pendula var. szechuanica)}.}\end{définition}
\begin{définition}\pcmn{四川桦树,白桦树}\end{définition}
\begin{définition}\pfra{Bouleau, |\stylefi{Betula szechuanica (Betula Pendula var. szechuanica)}.}\end{définition}
\end{entrée}

\begin{entrée}
{si˧tʰv̩\#˥}{}{ⓔsi˧tʰv̩\#˥}\formedesurface{si˧tʰv̩˧}\newline
\classe{名词}\ton{\#H}\begin{définition}\peng{A piece of furniture of the main room, which constitutes the symbolic dwelling of ancestors, and serves as an altar; on the New Year, some candles are lighted on it.}\end{définition}
\begin{définition}\pcmn{供桌:主屋里面的一个家具,是祖先的象征性住所。}\end{définition}
\begin{définition}\pfra{Meuble-autel des ancêtres, dans la pièce principale, qui constitue le lieu symbolique où résident les ancêtres; on y met des bougies au Nouvel An.}\end{définition}
\begin{exemple}\pnru{ʑi˧dv̩˧-nv̩˩mi˩, | si˧tʰv̩˧!}\hspace{5pt}\peng{The heart of the house is the altar to the ancestors!}\hspace{5pt}\pcmn{屋子的中心,就是祖先的供桌!}\hspace{5pt}\pfra{le cœur de la maison, c'est le meuble-autel des ancêtres!}\end{exemple}
\end{entrée}

\begin{entrée}
{si˩tsʰɤ˩}{}{ⓔsi˩tsʰɤ˩}\newline
\classe{名词}
\sens{1}\paradigme{\pcmn{:} \p{}}
\begin{définition}\peng{Leaf.}\end{définition}
\begin{définition}\pcmn{叶子}\end{définition}
\begin{définition}\pfra{Feuille.}\end{définition}
\begin{exemple}\pnru{si˧dzi˩-si˩tsʰɤ˩}\hspace{5pt}\peng{tree leaf}\hspace{5pt}\pcmn{树叶}\hspace{5pt}\pfra{feuilles d'arbre}\end{exemple}\sens{2}
\begin{définition}\peng{Cock's comb.}\end{définition}
\begin{définition}\pcmn{鸡冠}\end{définition}
\begin{définition}\pfra{Crête (du coq, d'un oiseau).}\end{définition}
\begin{exemple}\pnru{æ̃˧ʂwæ˥-si˩tsʰɤ˩}\hspace{5pt}\peng{comb of (a) cock}\hspace{5pt}\pcmn{公鸡冠}\hspace{5pt}\pfra{crête de coq}\end{exemple}
\end{entrée}

\begin{entrée}
{so˥}{₁}{ⓔso˥ⓗ1}\formedesurface{so˧}\newline
\classe{名词}\ton{\#H}
1\begin{définition}\peng{Offering to the gods, given to them in the morning; it comprises tea, butter, flour, and honey; it is burnt over a fire of pine needles.}\end{définition}
\begin{définition}\pcmn{早上献给神的食物(含茶、酥油、面、蜂蜜),扔进松针火里烧}\end{définition}
\begin{définition}\pfra{Offrande aux esprits: repas qu'on leur offre le matin; on y met du thé, du beurre, de la farine, et du miel (et éventuellement des fleurs: \stylefv{/so}˧dze˧-bæ˩bæ˩/); on le fait brûler sur un feu d'épines de pin.}\end{définition}
\begin{exemple}\pnru{so˧ dze˧ tʰi˧-qæ˩}\hspace{5pt}\peng{to burn honey as an offering}\hspace{5pt}\pcmn{烧蜂蜜献给神}\hspace{5pt}\pfra{faire brûler du miel en offrande}\end{exemple}
\begin{exemple}\pnru{so˧ qæ˩}\hspace{5pt}\peng{to burn an offering}\hspace{5pt}\pcmn{烧献给神(食物,……)}\hspace{5pt}\pfra{brûler une offrande; traditionnellement, du pin gorgé de résine.}\end{exemple}
\end{entrée}

\begin{entrée}
{so˥}{₂}{ⓔso˥ⓗ2}\formedesurface{ɖɯ˧ so˥}\newline
\classe{量词}\ton{H*}
2\begin{définition}\peng{A thing (no plural; only used in the negative construction “there is not a thing").}\end{définition}
\begin{définition}\pcmn{量词:样东西,如:‘一样东西都没有’}\end{définition}
\begin{définition}\pfra{Classificateur des choses/objets, utilisé seulement en tournure négative: ‘quoi que ce soit'.}\end{définition}
\begin{exemple}\pnru{ɖɯ˧-so˥ | mɤ˧-dʑo˧!}\hspace{5pt}\peng{There is simply nothing at all! (A polite statement made by the host when welcoming a guest for a meal, apologizing, in self-deprecation, for not offering a meal commensurate to one's wishes.)}\hspace{5pt}\pcmn{一样也没有! / 没什么东西!(请客时的礼貌、自我贬低说法:请客人原谅菜不够丰盛)}\hspace{5pt}\pfra{Il n’y a rien du tout [à manger]! (phrase polie qd on invite quelqu'un à manger: on prie le convive d’excuser la pauvreté des mets proposés)}\end{exemple}
\end{entrée}

\begin{entrée}
{so˧˥}{}{ⓔso˧˥}\newline
\classe{名词}
\sens{1}\paradigme{\pcmn{:} \p{}}
\begin{définition}\peng{Breath.}\end{définition}
\begin{définition}\pcmn{(一口)气}\end{définition}
\begin{définition}\pfra{Souffle.}\end{définition}\sens{2}
\begin{définition}\peng{Vapour.}\end{définition}
\begin{définition}\pcmn{蒸汽}\end{définition}
\begin{définition}\pfra{Vapeur.}\end{définition}
\begin{exemple}\pnru{so˧ tʰv̩˥-ze˩}\hspace{5pt}\peng{Vapour is coming out.}\hspace{5pt}\pcmn{热气冒出来了。}\hspace{5pt}\pfra{il y a de la vapeur qui sort, ça fait de la vapeur}\end{exemple}
\end{entrée}

\begin{entrée}
{so˧α}{}{ⓔso˧α}\formedesurface{ɖɯ˧ so˧}\newline
\classe{量词}\ton{Mα}\begin{définition}\peng{Classifier for mornings.}\end{définition}
\begin{définition}\pcmn{量词:早晨(一个)}\end{définition}
\begin{définition}\pfra{Classificateur des matinées. Il existe trois expressions pour compter les journées: on peut dire: un jour; une matinée; ou une nuit.}\end{définition}
\begin{exemple}\pnru{mv̩˩si˧-njɤ˧˥ | ɖɯ˧-so˧, | njɤ˧le˧gv̩˧ | ɖɯ˧-ɲi˥, | mv̩˧kʰv̩˥ | ɖɯ˧-hɑ̃˧˥!}\hspace{5pt}\peng{One morning; one day; [or] one night! (A sentence that summarizes the three ways to count days: a day can be referred to as “one morning", “one day", or “one night".)}\hspace{5pt}\pcmn{一个早晨,一个白天,(或者说)一个晚上!(这句话,总结数日子的三个方式:‘一天’,可以说成‘一个早晨’、‘一个白天’、或‘一个晚上’。)}\hspace{5pt}\pfra{Une matinée; une journée; [ou] une nuit! (Expression didactique résumant les trois façons de compter les jours: on peut compter les matinées, les journées, ou les soirées.)}\end{exemple}
\begin{exemple}\pnru{tʰv̩˧-so˩}\hspace{5pt}\peng{that morning}\hspace{5pt}\pcmn{那天早上}\hspace{5pt}\pfra{ce matin-là}\end{exemple}
\end{entrée}

\begin{entrée}
{so˩}{}{ⓔso˩}\formedesurface{so˩˥}\newline
\classe{数词}\ton{L}\begin{définition}\peng{Three.}\end{définition}
\begin{définition}\pcmn{三}\end{définition}
\begin{définition}\pfra{Trois.}\end{définition}
\end{entrée}

\begin{entrée}
{so˩α}{₁}{ⓔso˩αⓗ1}\formedesurface{so˩˥}\newline
\classe{形容词}\ton{Lα}
1\begin{définition}\peng{Good, pleasant to the taste or smell.}\end{définition}
\begin{définition}\pcmn{香(吃得香,气味香)}\end{définition}
\begin{définition}\pfra{Agréable, bon (goût, odeur).}\end{définition}
\end{entrée}

\begin{entrée}
{so˩α}{₂}{ⓔso˩αⓗ2}\newline
\classe{动词}
2
\sens{1}
\begin{définition}\peng{To study.}\end{définition}
\begin{définition}\pcmn{学习}\end{définition}
\begin{définition}\pfra{Étudier.}\end{définition}
\begin{exemple}\pnru{tʰæ˧ɻæ˩ so˩}\hspace{5pt}\peng{to study (books)}\hspace{5pt}\pcmn{读书、学习}\hspace{5pt}\pfra{étudier (des livres)}\end{exemple}
\begin{exemple}\pnru{so˩ mɤ˩-se˥!}\hspace{5pt}\peng{There's no end of it! / One is never done with studying! (A comment about the linguist's endeavour to study a language: unlike manual work, it is never really finished.)}\hspace{5pt}\pcmn{学不完!(关于语言学家的工作:做不完,不像做手工可以有一个明确的终点。)}\hspace{5pt}\pfra{c'est sans fin! / tu n'as jamais fini d'étudier! (au sujet du travail du linguiste, étudier une langue: à la différence des travaux manuels, ce n'est jamais fini, on n'en voit jamais le bout)}\end{exemple}
\begin{exemple}\pnru{ɖɯ˧-so˧∼so˥-ɻ̍˩}\hspace{5pt}\peng{to study a little}\hspace{5pt}\pcmn{学一学}\hspace{5pt}\pfra{étudier un peu}\end{exemple}
\begin{exemple}\pnru{le˧-so˩-ɳɯ˩, | kv̩˧ ɲi˥-mæ˩!}\hspace{5pt}\peng{By learning, one becomes able (to do things)! (About the necessity of schooling and learning: those who do not learn computers, foreign languages... are unable to find a good job.)}\hspace{5pt}\pcmn{通过学习,就会了!}\hspace{5pt}\pfra{En apprenant, on devient capable (de réaliser toutes sortes de choses!) (Contexte: discussion au sujet de la nécessité de l'éducation et des apprentissages, sans lesquels on ne peut trouver une bonne situation dans la société.)}\end{exemple}\sens{2}
\begin{définition}\peng{To follow the example of someone, to imitate someone.}\end{définition}
\begin{définition}\pcmn{学一个人、模仿一个人}\end{définition}
\begin{définition}\pfra{Imiter.}\end{définition}
\begin{exemple}\pnru{tʰv̩˧ tʰɑ˧-so˧˥!}\hspace{5pt}\peng{Don't follow his example! / Don't do like him!}\hspace{5pt}\pcmn{别学他! / 别做得像他一样!}\hspace{5pt}\pfra{Ne t'avise pas de suivre son exemple!/Ne va pas faire comme lui!/Ne va pas prendre exemple sur lui!}\end{exemple}\sens{3}
\begin{définition}\peng{To teach.}\end{définition}
\begin{définition}\pcmn{教}\end{définition}
\begin{définition}\pfra{Enseigner.}\end{définition}
\begin{exemple}\pnru{tʰæ˧ɻæ˩ so˩}\hspace{5pt}\peng{to teach}\hspace{5pt}\pcmn{教书}\hspace{5pt}\pfra{enseigner}\end{exemple}
\begin{exemple}\pnru{njɤ˧-ɳɯ˧ | no˧ so˧-bi˧!}\hspace{5pt}\peng{I'm going to teach you! / Let me teach you!}\hspace{5pt}\pfra{Je vais t'enseigner/t'apprendre!}\end{exemple}
\end{entrée}

\begin{entrée}
{so˧dʑɯ\#˥}{}{ⓔso˧dʑɯ\#˥}\formedesurface{so˧dʑɯ˧}\newline
\classe{名词}\ton{\#H}
\paradigme{\pcmn{:} \p{}}
\begin{définition}\peng{Pitfall, pit, trap.}\end{définition}
\begin{définition}\pcmn{陷阱}\end{définition}
\begin{définition}\pfra{Piège.}\end{définition}
\begin{exemple}\pnru{so˧dʑɯ˧ | ɖɯ˧-ɭɯ˧ | qwæ˧˥}\hspace{5pt}\peng{to dig a pitfall, to dig a trap}\hspace{5pt}\pcmn{挖一个陷阱}\hspace{5pt}\pfra{creuser une fosse, un piège}\end{exemple}
\end{entrée}

\begin{entrée}
{so˧hɑ̃˩}{}{ⓔso˧hɑ̃˩}\formedesurface{so˧hɑ̃˩}\newline
\classe{助词}\ton{L\#}\begin{définition}\peng{Tomorrow evening.}\end{définition}
\begin{définition}\pcmn{明晚}\end{définition}
\begin{définition}\pfra{Demain soir.}\end{définition}
\begin{exemple}\pnru{so˧hɑ̃˩ | -ɖɯ˩hɑ̃˩˥}\hspace{5pt}\peng{tomorrow evening}\hspace{5pt}\pcmn{明天晚上}\hspace{5pt}\pfra{demain soir}\end{exemple}
\end{entrée}

\begin{entrée}
{so˧hwɤ˩}{}{ⓔso˧hwɤ˩}\formedesurface{so˧hwɤ˩}\newline
\classe{助词}\ton{L\#}\begin{définition}\peng{Afterwards; later; from now on.}\end{définition}
\begin{définition}\pcmn{后来、以后,从此以后}\end{définition}
\begin{définition}\pfra{Ensuite; par la suite; à partir de maintenant, désormais, dorénavant.}\end{définition}
\end{entrée}

\begin{entrée}
{so˧ʝi˥\$}{}{ⓔso˧ʝi˥\$}\formedesurface{so˧ʝi˥}\newline
\classe{助词}\ton{H\$}\begin{définition}\peng{Next year.}\end{définition}
\begin{définition}\pcmn{明年}\end{définition}
\begin{définition}\pfra{L'année prochaine, l'an prochain.}\end{définition}
\end{entrée}

\begin{entrée}
{so˧lo˧}{}{ⓔso˧lo˧}\formedesurface{so˧lo˧}\newline
\classe{名词}\ton{M}
\paradigme{\pcmn{:} \p{}}
\begin{définition}\peng{Influence, example (in education).}\end{définition}
\begin{définition}\pcmn{影响,榜样}\end{définition}
\begin{définition}\pfra{Influence; exemple (dans l'éducation de quelqu'un).}\end{définition}
\begin{exemple}\pnru{so˧lo˧ dzɑ˧! | mɤ˧-dʑɤ˩-hĩ˩ | ɖɯ˧-ʑi˩ ɲi˩!}\hspace{5pt}\peng{He/she has a bad influence / he/she gives a bad example! (His/her family) is a bad family!}\hspace{5pt}\pcmn{他(对周围的人)有一个不好的影响!(他的家庭)是个不好的家庭!}\hspace{5pt}\pfra{Il/elle exerce une mauvaise influence / il/elle donne un mauvais exemple! (Sa famille,) c'est une mauvaise famille!}\end{exemple}
\begin{exemple}\pnru{so˧lo˧ mɤ˧-dʑɤ˩!}\hspace{5pt}\peng{Same meaning as previous example: His/her example/influence is not good.}\hspace{5pt}\pcmn{同上:(他对别人的)影响不好。}\hspace{5pt}\pfra{Même sens que l'exemple précédent: Son exemple n'est pas bon / son influence n'est pas bonne.}\end{exemple}
\begin{exemple}\pnru{so˧lo˧ dʑɤ˩}\hspace{5pt}\peng{good influence; good example; good education}\hspace{5pt}\pcmn{好榜样、好例子、好教育}\hspace{5pt}\pfra{bonne influence; bon exemple; bonne éducation}\end{exemple}
\end{entrée}

\begin{entrée}
{so˧ɬi˧mi˧}{}{ⓔso˧ɬi˧mi˧}\formedesurface{so˧ɬi˧mi˧}\newline
\classe{名词}\ton{M}\begin{définition}\peng{Third month.}\end{définition}
\begin{définition}\pcmn{三月}\end{définition}
\begin{définition}\pfra{3e mois.}\end{définition}
\end{entrée}

\begin{entrée}
{so˧ɲi˥}{}{ⓔso˧ɲi˥}\formedesurface{so˧ɲi˥}\newline
\classe{助词}\begin{définition}\peng{Tomorrow.}\end{définition}
\begin{définition}\pcmn{明天、第二天}\end{définition}
\begin{définition}\pfra{Demain, le lendemain.}\end{définition}
\end{entrée}

\begin{entrée}
{so˩∼so˧˥}{}{ⓔso˩∼so˧˥}\formedesurface{so˩so˧˥}\newline
\classe{动词}\ton{MH}\begin{définition}\peng{To rub in one's hands.}\end{définition}
\begin{définition}\pcmn{揉在手里}\end{définition}
\begin{définition}\pfra{Frotter dans ses mains.}\end{définition}
\begin{exemple}\pnru{le˧-so˩∼so˩}\hspace{5pt}\peng{|fg{accomp} \_ |fg{red}}\hspace{5pt}\pcmn{揉来揉去}\hspace{5pt}\pfra{|fg{accomp} \_ |fg{red}}\end{exemple}
\end{entrée}

\begin{entrée}
{so˧tsʰi˥}{}{ⓔso˧tsʰi˥}\formedesurface{so˧tsʰi˥}\newline
\classe{动词}\ton{H\#}\begin{définition}\peng{To breathe.}\end{définition}
\begin{définition}\pcmn{呼吸}\end{définition}
\begin{définition}\pfra{Respirer.}\end{définition}
\begin{exemple}\pnru{so˧tsʰi˥ | ʐwæ˩˥}\hspace{5pt}\peng{to breathe heavily, to pant}\hspace{5pt}\pcmn{喘气}\hspace{5pt}\pfra{respirer très vite, haleter}\end{exemple}
\begin{exemple}\pnru{mɤ˧-so˧tsʰi˥}\hspace{5pt}\peng{|fg{neg}}\hspace{5pt}\pcmn{不喘气}\hspace{5pt}\pfra{|fg{neg}}\end{exemple}
\end{entrée}

\begin{entrée}
{so˧tsʰi˧}{}{ⓔso˧tsʰi˧}\formedesurface{so˧tsʰi˧}\newline
\classe{数词}\ton{M}\begin{définition}\peng{30.}\end{définition}
\begin{définition}\pcmn{30}\end{définition}
\begin{définition}\pfra{30.}\end{définition}
\end{entrée}

\begin{entrée}
{so˧tsʰɯ˧ɲi˧}{}{ⓔso˧tsʰɯ˧ɲi˧}\formedesurface{so˧tsʰɯ˧ɲi˧}\newline
\classe{名词}\ton{M}\begin{définition}\peng{The 30th day of the month.}\end{définition}
\begin{définition}\pcmn{三十号}\end{définition}
\begin{définition}\pfra{Le 30e jour.}\end{définition}
\end{entrée}

\begin{entrée}
{sɯ˥}{₁}{ⓔsɯ˥ⓗ1}\formedesurface{sɯ˧}\newline
\classe{动词}\ton{H}
1\begin{définition}\peng{To whet.}\end{définition}
\begin{définition}\pcmn{磨(刀)}\end{définition}
\begin{définition}\pfra{Aiguiser.}\end{définition}
\begin{exemple}\pnru{ɖɯ˧-sɯ˧∼sɯ˧-ɻ̍˥}\hspace{5pt}\peng{to whet a little}\hspace{5pt}\pcmn{磨一磨}\hspace{5pt}\pfra{aiguiser un peu}\end{exemple}
\begin{exemple}\pnru{sɯ˩tʰi˩ sɯ˩˥}\hspace{5pt}\peng{to whet a knife}\hspace{5pt}\pcmn{磨刀}\hspace{5pt}\pfra{aiguiser un couteau}\end{exemple}
\end{entrée}

\begin{entrée}
{sɯ˥}{₂}{ⓔsɯ˥ⓗ2}\formedesurface{sɯ˧}\newline
\classe{动词}\ton{H}
2\begin{définition}\peng{To know.}\end{définition}
\begin{définition}\pcmn{知道}\end{définition}
\begin{définition}\pfra{Savoir.}\end{définition}
\begin{exemple}\pnru{mɤ˧-sɯ˥}\hspace{5pt}\peng{|fg{neg}}\hspace{5pt}\pcmn{不知道}\hspace{5pt}\pfra{|fg{neg}}\end{exemple}
\end{entrée}

\begin{entrée}
{‑sɯ˧}{}{ⓔ‑sɯ˧}\formedesurface{sɯ˧}\newline
\classe{后缀}\ton{M}\begin{définition}\peng{First, at first, in the first place; anymore (in “not anymore").}\end{définition}
\begin{définition}\pcmn{首先、先}\end{définition}
\begin{définition}\pfra{D'abord; encore (dans la tournure: pas encore).}\end{définition}
\begin{exemple}\pnru{njɤ˧ ʈʂʰɯ˧-sɯ˩ | dzɯ˧-bi˧!}\hspace{5pt}\peng{Let me eat this one first! / I'll eat this one first!}\hspace{5pt}\pcmn{我要先吃这个!}\hspace{5pt}\pfra{je vais d'abord manger celui-ci!}\end{exemple}
\begin{exemple}\pnru{njɤ˧ | ʈʂʰɯ˧-sɯ˩ | li˧-bi˧!}\hspace{5pt}\peng{I'll read this one first! (Context: examining two books, and deciding which one to read first)}\hspace{5pt}\pcmn{我要先读这本!}\hspace{5pt}\pfra{je vais d'abord lire celui-ci! (au sujet de deux livres)}\end{exemple}
\begin{exemple}\pnru{ʈʂʰɯ˧ sɯ˩ | hwæ˧-bi˧!}\hspace{5pt}\peng{Let's buy this one first!}\hspace{5pt}\pcmn{先买这个吧!}\hspace{5pt}\pfra{(je) vais d'abord acheter celui-ci!}\end{exemple}
\begin{exemple}\pnru{ʈʂʰɯ˧ sɯ˩ | tɕʰi˧-bi˧!}\hspace{5pt}\peng{Let's sell this one first!}\hspace{5pt}\pcmn{先卖这个吧!}\hspace{5pt}\pfra{(je) vais d'abord vendre celui-ci!}\end{exemple}
\begin{exemple}\pnru{ʈʂʰɯ˧ sɯ˩ | dzɯ˧-bi˧!}\hspace{5pt}\peng{Let's eat this one first!}\hspace{5pt}\pcmn{先吃这个吧!}\hspace{5pt}\pfra{(je) vais d'abord manger celui-ci!}\end{exemple}
\begin{exemple}\pnru{ʈʂʰɯ˧ sɯ˩ | ʑi˩-bi˩˥}\hspace{5pt}\peng{Let's pick up this one first!}\hspace{5pt}\pcmn{先拿这个吧!}\hspace{5pt}\pfra{(je) vais d'abord prendre celui-ci!}\end{exemple}
\begin{exemple}\pnru{ʈʂʰɯ˧ sɯ˩ | ʈʰɯ˩-bi˩˥}\hspace{5pt}\peng{Let's drink this one first!}\hspace{5pt}\pcmn{先喝这个吧!}\hspace{5pt}\pfra{(je) vais d'abord boire celui-ci!}\end{exemple}
\begin{exemple}\pnru{ʈʂʰɯ˧ sɯ˩ | lɑ˧-bi˥}\hspace{5pt}\peng{Let's beat this one first!}\hspace{5pt}\pcmn{先打这个吧!}\hspace{5pt}\pfra{(je) vais d'abord battre celui-ci!}\end{exemple}
\begin{exemple}\pnru{tv̩˧tv̩˥ sɯ˩ | tʰi˧-tsʰi˥!}\hspace{5pt}\peng{Put on your hat first! (Injunction to a little child before an outing)}\hspace{5pt}\pcmn{你先戴上帽子!(情景:出门前,让孩子戴上帽子)}\hspace{5pt}\pfra{Mets d'abord ton chapeau! (injonction à un petit enfant, avant une sortie)}\end{exemple}
\begin{exemple}\pnru{no˧ | le˧-sɯ˧ gv̩˧∼gv̩˥!}\hspace{5pt}\peng{Do your own work first! / Please work on your own for a start! (Context: when I arrive for a morning class, the consultant is busy; she knows that I have various tasks to do, some of which I can do on my own, such as verifying texts that have already been transcribed; she tells me: “Please work on your own for a start!")}\hspace{5pt}\pcmn{你先自己工作(一会)吧!(情景:调查者早上到合作人的家,但她忙着,而她知道调查者有不同类型的工作要做,其中有一些可以自己做,比如重新核对记录过的长篇语料。她说:“你先忙自己的一会吧!”)}\hspace{5pt}\pfra{Commence par travailler tout seul! (Consigne de la locutrice quand j'arrive pour ma leçon du matin. Elle est occupée; et elle sait que j'ai de quoi m'occuper seul en l'attendant: toiletter des textes déjà transcrits, etc. Elle me dit «Commence par travailler tout seul! / Commence par les tâches que tu peux faire tout seul!»}\end{exemple}
\end{entrée}

\begin{entrée}
{sɯ˧α}{}{ⓔsɯ˧α}\newline
\classe{动词}
\sens{1}
\begin{définition}\peng{To string (beads).}\end{définition}
\begin{définition}\pcmn{串(珠)}\end{définition}
\begin{définition}\pfra{Enfiler (des perles).}\end{définition}
\begin{exemple}\pnru{sɯ˧ɻ̍˧ sɯ˧}\hspace{5pt}\peng{to string beads}\hspace{5pt}\pcmn{串珠}\hspace{5pt}\pfra{enfiler des perles}\end{exemple}
\begin{exemple}\pnru{le˧-sɯ˧-se˩-ze˩}\hspace{5pt}\peng{(I) have finished to string (beads)}\hspace{5pt}\pcmn{串完了!}\hspace{5pt}\pfra{(j'ai) fini d'enfiler}\end{exemple}
\begin{exemple}\pnru{tso˧∼tso˧ sɯ˩}\hspace{5pt}\peng{to string things}\hspace{5pt}\pcmn{串东西}\hspace{5pt}\pfra{enfiler des choses}\end{exemple}\sens{2}
\begin{définition}\peng{To put on (a skirt).}\end{définition}
\begin{définition}\pcmn{穿(裙子)}\end{définition}
\begin{définition}\pfra{Enfiler (une jupe).}\end{définition}
\begin{exemple}\pnru{ʈʰæ˧qʰwɤ˧ sɯ˧}\hspace{5pt}\peng{to put on a skirt}\hspace{5pt}\pcmn{穿裙子}\hspace{5pt}\pfra{enfiler une robe}\end{exemple}
\end{entrée}

\begin{entrée}
{sɯ˩α}{}{ⓔsɯ˩α}\formedesurface{sɯ˩˥}\newline
\classe{动词}\ton{Lα}\begin{définition}\peng{To live, to be alive.}\end{définition}
\begin{définition}\pcmn{活}\end{définition}
\begin{définition}\pfra{Vivre, être vivant.}\end{définition}
\begin{exemple}\pnru{ʈʂʰɯ˧ tʰi˧-sɯ˩-dʑo˩!}\hspace{5pt}\peng{(S)he is alive!}\hspace{5pt}\pcmn{他活着!}\hspace{5pt}\pfra{Elle/il est vivant(e)!}\end{exemple}
\begin{exemple}\pnru{ʈʂʰɯ˧ | mɤ˧-ʂɯ˧! | tʰi˧-sɯ˩-dʑo˩!}\hspace{5pt}\peng{It's not dead! It's alive! (About a plant or animal that looked dead)}\hspace{5pt}\pcmn{它没死,还活着!(一个植物、动物)}\hspace{5pt}\pfra{ce n'est pas mort! c'est encore vivant! (au sujet d'une plante/d'un animal qui paraissait mort(e))}\end{exemple}
\end{entrée}

\begin{entrée}
{sɯ˧gv̩\#˥}{}{ⓔsɯ˧gv̩\#˥}\formedesurface{sɯ˧gv̩˧}\newline
\classe{名词}\ton{\#H}
\paradigme{\pcmn{:} \p{}}
\begin{définition}\peng{Box, case.}\end{définition}
\begin{définition}\pcmn{箱子,柜子}\end{définition}
\begin{définition}\pfra{Caisse, coffre; par extension: armoire.}\end{définition}
\end{entrée}

\begin{entrée}
{sɯ˧kʰɯ˩}{}{ⓔsɯ˧kʰɯ˩}\formedesurface{sɯ˧kʰɯ˩}\newline
\classe{名词}\ton{L\#}\begin{définition}\peng{Ritual performed for the death of a female relative who left her maternal home to marry.}\end{définition}
\begin{définition}\pcmn{斯克:嫁到外边的女人去世时进行的仪式}\end{définition}
\begin{définition}\pfra{Rituel lors du décès d'une femme qui a quitté la maison de sa mère pour se marier.}\end{définition}
\end{entrée}

\begin{entrée}
{sɯ˧ljɤ˧˥}{}{ⓔsɯ˧ljɤ˧˥}\formedesurface{sɯ˧ljɤ˧˥}\newline
\classe{名词}\ton{MH\#}\begin{définition}\peng{Plastic.}\end{définition}
\begin{définition}\pcmn{塑料(汉语借词)}\end{définition}
\begin{définition}\pfra{Plastique.}\end{définition}
\end{entrée}

\begin{entrée}
{sɯ˧ljɤ˧tʰo˧˥}{}{ⓔsɯ˧ljɤ˧tʰo˧˥}\formedesurface{sɯ˧ljɤ˧tʰo˧˥}\newline
\classe{名词}\ton{MH\#}
\paradigme{\pcmn{:} \p{}}
\begin{définition}\peng{Plastic jerrican; used to store and transport drinking water.}\end{définition}
\begin{définition}\pcmn{塑料桶(汉语借词)}\end{définition}
\begin{définition}\pfra{Container pour liquides, en matière plastique.}\end{définition}
\end{entrée}

\begin{entrée}
{sɯ˧mɤ˩}{}{ⓔsɯ˧mɤ˩}\formedesurface{sɯ˧mɤ˩}\newline
\classe{名词}\ton{L\#}\begin{définition}\peng{Purple perilla, |\stylefi{Perilla frutescens}, akajiso.}\end{définition}
\begin{définition}\pcmn{紫苏}\end{définition}
\begin{définition}\pfra{Shiso, |\stylefi{Perilla frutescens}, akajiso: plante alimentaire, aromatique, médicinale et ornementale de la famille des Lamiacées. Ses petites graines noires ressemblent au sésame. Elle a été introduite à Yongning récemment (années 1980?) On se sert des graines pour confectionner des confiseries; et on en extrait de l'huile, qui se mange.}\end{définition}
\begin{exemple}\pnru{sɯ˧mɤ˩, | ɬi˧di˩ | tv̩˧-kv̩˧˥!}\hspace{5pt}\peng{Purple perilla is cultivated in Yongning! / Purple perilla is among the crops that are grown in Yongning!}\hspace{5pt}\pcmn{在永宁,有紫苏!/ 有人种紫苏!}\hspace{5pt}\pfra{Le shiso, on en cultive à Yongning!}\end{exemple}
\begin{exemple}\pnru{sɯ˧mɤ˩-dze˩}\hspace{5pt}\pfra{friandise contenant des graines de shiso}\hspace{5pt}\peng{candy containing purple perilla seeds}\hspace{5pt}\pcmn{含紫苏的糖果}\end{exemple}
\begin{exemple}\pnru{sɯ˧mɤ˩-mæ˩ɻæ˩}\hspace{5pt}\pfra{huile de shiso}\hspace{5pt}\peng{purple perilla oil}\hspace{5pt}\pcmn{紫苏油}\end{exemple}
\end{entrée}

\begin{entrée}
{sɯ˧pv̩˩}{}{ⓔsɯ˧pv̩˩}\formedesurface{sɯ˧pv̩˩}\newline
\classe{名词}\ton{L\#}
\paradigme{\pcmn{:} \p{}}
\begin{définition}\peng{Urinary bladder.}\end{définition}
\begin{définition}\pcmn{膀胱}\end{définition}
\begin{définition}\pfra{Vessie.}\end{définition}
\end{entrée}

\begin{entrée}
{sɯ˩pv̩˩}{}{ⓔsɯ˩pv̩˩}\formedesurface{sɯ˩pv̩˩˥}\newline
\classe{名词}
\paradigme{\pcmn{:} \p{}}
\begin{définition}\peng{Raised spot, blister.}\end{définition}
\begin{définition}\pcmn{水泡(例如:开水烫了手,会形成水泡)}\end{définition}
\begin{définition}\pfra{Cloque (par exemple, après qu'on se soit ébouillanté).}\end{définition}
\begin{exemple}\pnru{sɯ˩pv̩˩ qʰwæ˥-ze˩!}\hspace{5pt}\peng{a raised spot has formed!}\hspace{5pt}\pcmn{起了水泡!}\hspace{5pt}\pfra{une cloque s'est formée!}\end{exemple}
\end{entrée}

\begin{entrée}
{sɯ˧pv̩˩-ni˩gv̩˩}{}{ⓔsɯ˧pv̩˩-ni˩gv̩˩}\formedesurface{sɯ˧pv̩˩ni˩gv̩˩}\newline
\classe{形容词}\ton{L\#-}\begin{définition}\peng{Swollen: literally: ‘like a bladder'.}\end{définition}
\begin{définition}\pcmn{膀肿}\end{définition}
\begin{définition}\pfra{Gonflé, galbé, rond: littéralement «comme une vessie». Une vessie de porc que l'on gonfle prend une forme toute ronde.}\end{définition}
\end{entrée}

\begin{entrée}
{sɯ˧pv̩˧-sɯ˥nɑ˩}{}{ⓔsɯ˧pv̩˧-sɯ˥nɑ˩}\formedesurface{sɯ˧pv̩˧sɯ˥nɑ˩}\newline
\classe{名词}\ton{\#H-}
\paradigme{\pcmn{:} \p{}}
\begin{définition}\peng{Caterpillar.}\end{définition}
\begin{définition}\pcmn{毛虫}\end{définition}
\begin{définition}\pfra{Chenille.}\end{définition}
\end{entrée}

\begin{entrée}
{sɯ˧pʰi˧}{}{ⓔsɯ˧pʰi˧}\formedesurface{sɯ˧pʰi˧}\newline
\classe{名词}\ton{M}
\paradigme{\pcmn{:} \p{}}
\begin{définition}\peng{Chieftain, nobleman, lord: the highest of the three castes (ranks) in feudal society.}\end{définition}
\begin{définition}\pcmn{贵族,土司,奴隶主,官。音译:“司沛”}\end{définition}
\begin{définition}\pfra{Noble, seigneur, chef: la plus haute des 3 castes de la société ancienne.}\end{définition}
\begin{exemple}\pnru{ɖæ˩mi˧-sɯ˩pʰi˩}\hspace{5pt}\peng{the noblemen at the monastery}\hspace{5pt}\pcmn{大寺贵族}\hspace{5pt}\pfra{les nobles du monastère}\end{exemple}
\begin{exemple}\pnru{sɯ˧pʰi˧ hĩ˩}\hspace{5pt}\peng{the nobleman's subjects, the nobleman's people}\hspace{5pt}\pcmn{贵族的臣子、贵族手下的人}\hspace{5pt}\pfra{les sujets du seigneur, les gens du seigneur}\end{exemple}
\end{entrée}

\begin{entrée}
{sɯ˧pʰi˧-zo˧}{}{ⓔsɯ˧pʰi˧-zo˧}\formedesurface{sɯ˧pʰi˧zo˧}\newline
\classe{名词}\ton{H\#}
\paradigme{\pcmn{:} \p{}}
\begin{définition}\peng{Young man of the nobility.}\end{définition}
\begin{définition}\pcmn{少爷}\end{définition}
\begin{définition}\pfra{Jeune homme de la noblesse, fils de mandarin.}\end{définition}
\end{entrée}

\begin{entrée}
{sɯ˧ɻ̃\#˥}{}{ⓔsɯ˧ɻ̃\#˥}\formedesurface{sɯ˧ɻ̃˧}\newline
\classe{名词}\ton{\#H}
\paradigme{\pcmn{:} \p{}}
\begin{définition}\peng{Tree trunk.}\end{définition}
\begin{définition}\pcmn{树干}\end{définition}
\begin{définition}\pfra{Tronc.}\end{définition}
\begin{exemple}\pnru{si˧dzi˩ tʰv̩˩-dzi˩, | sɯ˧ɻ̃˧ dʑɤ˥!}\hspace{5pt}\peng{This tree has a good trunk! (i.e. it is suitable for use in carpentry, making furniture…)}\hspace{5pt}\pcmn{这是棵好树!(可以用做木料)}\hspace{5pt}\pfra{Cet arbre, il a un beau tronc! (c'est-à-dire qu'il est utilisable en menuiserie)}\end{exemple}
\end{entrée}

\begin{entrée}
{sɯ˧ɻæ˧}{}{ⓔsɯ˧ɻæ˧}\formedesurface{sɯ˧ɻæ˧}\newline
\classe{名词}\ton{M}
\paradigme{\pcmn{:} \p{}}
\begin{définition}\peng{Table.}\end{définition}
\begin{définition}\pcmn{桌子}\end{définition}
\begin{définition}\pfra{Table (à Yongning, vers le milieu du XXe siècle, elles étaient basses et carrées).}\end{définition}
\end{entrée}

\begin{entrée}
{sɯ˧ɻ̃˧mi\#˥}{}{ⓔsɯ˧ɻ̃˧mi\#˥}\formedesurface{sɯ˧ɻ̃˧mi˧}\newline
\classe{名词}\ton{\#H}
\paradigme{\pcmn{:} \p{}}
\begin{définition}\peng{Backbone.}\end{définition}
\begin{définition}\pcmn{脊椎骨}\end{définition}
\begin{définition}\pfra{Colonne vertébrale.}\end{définition}
\end{entrée}

\begin{entrée}
{sɯ˧ɻ̍˧}{}{ⓔsɯ˧ɻ̍˧}\formedesurface{sɯ˧ɻ̍˧}\newline
\classe{名词}\ton{M}
\paradigme{\pcmn{:} \p{}}
\begin{définition}\peng{Bead, pearl.}\end{définition}
\begin{définition}\pcmn{珠,珠子,珍珠}\end{définition}
\begin{définition}\pfra{Perle.}\end{définition}
\end{entrée}

\begin{entrée}
{sɯ˩ɻ̍˩}{}{ⓔsɯ˩ɻ̍˩}\formedesurface{sɯ˩ɻ̍˩˥}\newline
\classe{名词}\ton{L}
\paradigme{\pcmn{:} \p{}}
\begin{définition}\peng{Whetting-stone.}\end{définition}
\begin{définition}\pcmn{磨刀石}\end{définition}
\begin{définition}\pfra{Pierre à aiguiser, «fusil».}\end{définition}
\end{entrée}

\begin{entrée}
{sɯ˧sɯ˩}{}{ⓔsɯ˧sɯ˩}\formedesurface{sɯ˧sɯ˩}\newline
\classe{形容词}\ton{L\#}\begin{définition}\peng{Raw.}\end{définition}
\begin{définition}\pcmn{生(不熟)}\end{définition}
\begin{définition}\pfra{Cru.}\end{définition}
\begin{exemple}\pnru{ʂe˧ sɯ˧∼sɯ˥}\hspace{5pt}\peng{raw meat}\hspace{5pt}\pcmn{生肉}\hspace{5pt}\pfra{viande crue}\end{exemple}
\begin{exemple}\pnru{ʈʂe˧ sɯ˧∼sɯ˥}\hspace{5pt}\peng{‘raw earth': immature soil, earth that has not been prepared for agriculture by adding manure, etc}\hspace{5pt}\pcmn{‘生土’:没有经过加工(加肥料等等)的土,还不适合种农作物}\hspace{5pt}\pfra{‘terre crue': terre qui n'a pas été préparée pour l'agriculture par l'ajout de fumier, etc}\end{exemple}
\end{entrée}

\begin{entrée}
{sɯ˩tʰi˩}{}{ⓔsɯ˩tʰi˩}\formedesurface{sɯ˩tʰi˩˥}\newline
\classe{名词}\ton{L}
\paradigme{\pcmn{:} \p{}}
\begin{définition}\peng{Knife.}\end{définition}
\begin{définition}\pcmn{刀}\end{définition}
\begin{définition}\pfra{Couteau.}\end{définition}
\end{entrée}

\begin{entrée}
{sɯ˩tʰi˩-kʰɯ˥ʑi˩}{}{ⓔsɯ˩tʰi˩-kʰɯ˥ʑi˩}\formedesurface{sɯ˩tʰi˩kʰɯ˥ʑi˩}\newline
\classe{名词}\ton{L+\#H-}
\paradigme{\pcmn{:} \p{}}
\begin{définition}\peng{Knife sheath.}\end{définition}
\begin{définition}\pcmn{刀鞘}\end{définition}
\begin{définition}\pfra{Fourreau du couteau, gaine du couteau.}\end{définition}
\end{entrée}

\begin{entrée}
{sɯ˧tsɯ˧}{}{ⓔsɯ˧tsɯ˧}\formedesurface{sɯ˧tsɯ˧}\newline
\classe{名词}\ton{M}
\paradigme{\pcmn{:} \p{}}
\begin{définition}\peng{Lion.}\end{définition}
\begin{définition}\pcmn{狮子(汉语借词)}\end{définition}
\begin{définition}\pfra{Lion.}\end{définition}
\end{entrée}

\begin{entrée}
{sɯ˧tsɯ˩}{}{ⓔsɯ˧tsɯ˩}\formedesurface{sɯ˧tsɯ˩}\newline
\classe{名词}\ton{L\#}\begin{définition}\peng{Camphor.}\end{définition}
\begin{définition}\pcmn{樟}\end{définition}
\begin{définition}\pfra{Camphre (arbre).}\end{définition}
\begin{exemple}\pnru{sɯ˧tsɯ˩-dzi˩}\hspace{5pt}\peng{camphor tree}\hspace{5pt}\pcmn{樟树}\hspace{5pt}\pfra{arbre à camphre}\end{exemple}
\end{entrée}

\begin{entrée}
{sɯ˧ʈv̩˥}{}{ⓔsɯ˧ʈv̩˥}\formedesurface{sɯ˧ʈv̩˥}\newline
\classe{名词}\ton{H\#}
\paradigme{\pcmn{:} \p{}}
\begin{définition}\peng{Callus.}\end{définition}
\begin{définition}\pcmn{茧子}\end{définition}
\begin{définition}\pfra{Cor, durillon.}\end{définition}
\begin{exemple}\pnru{hĩ˧ ʈʂʰɯ˧-v̩˧-bv̩˧ | mv̩˧ɲi˧, | sɯ˧ʈv̩˥ ʁo˩!}\hspace{5pt}\peng{This person's thumb has a callus / developed a callus!}\hspace{5pt}\pcmn{这个人的拇指有茧子!}\hspace{5pt}\pfra{le pouce de cette personne a un cor/un durillon!}\end{exemple}
\begin{exemple}\pnru{sɯ˧ʈv̩˥ | mɤ˧-ʁo˩-ze˩!}\hspace{5pt}\peng{The callus is gone! / There is no callus anymore!}\hspace{5pt}\pcmn{没有茧子了!}\hspace{5pt}\pfra{(il n'y a/je n'ai) plus de durillon!}\end{exemple}
\begin{exemple}\pnru{sɯ˧ʈv̩˥ ʁo˩-ze˩!}\hspace{5pt}\peng{A callus has formed!}\hspace{5pt}\pcmn{磨出了茧子!}\hspace{5pt}\pfra{Un durillon s'est formé!}\end{exemple}
\end{entrée}

\begin{entrée}
{sɯ˧zɯ\#˥}{}{ⓔsɯ˧zɯ\#˥}\formedesurface{sɯ˧zɯ˧}\newline
\classe{名词}\ton{\#H}
\paradigme{\pcmn{:} \p{}}
\begin{définition}\peng{Family community.}\end{définition}
\begin{définition}\pcmn{家族、支系}\end{définition}
\begin{définition}\pfra{Communauté familiale; échelon inférieur à «os».}\end{définition}
\begin{exemple}\pnru{sɯ˧zɯ˧ ɖɯ˧-lo˩}\hspace{5pt}\peng{one family community}\hspace{5pt}\pcmn{一个支系,一条线}\hspace{5pt}\pfra{une communauté familiale}\end{exemple}
\begin{exemple}\pnru{sɯ˧zɯ˧ ɖɯ˧-ʁwɤ˧}\hspace{5pt}\peng{one family community}\hspace{5pt}\pcmn{一个支系,一条线}\hspace{5pt}\pfra{une communauté familiale}\end{exemple}
\begin{exemple}\pnru{sɯ˧zɯ˧ ə˩-dʑo˩?}\hspace{5pt}\peng{Is there a (complete) family community? / Is the family large? (Question asked as part of discussions preliminary to marriage: Will the bride have a large family around her, be surrounded by a large family? A small family is considered much less attractive than a large one.)}\hspace{5pt}\pcmn{家族齐全吗?/ 家族,人多吗?(谈婚姻前的题目之一:男方家族人多不多。以人多为好。)}\hspace{5pt}\pfra{est-ce qu'il a une grande famille/est-ce que sa famille est nombreuse? =est-ce que la mariée sera bien entourée, intégrée dans une grande famille? (Question que l'on pose lors des discussions préliminaires aux mariages: on s'inquiète des qualités de la maisonnée que la jeune femme va rejoindre)}\end{exemple}
\end{entrée}

\newpage\caractère{ʂ}

\begin{entrée}
{ʂæ˧}{}{ⓔʂæ˧}\formedesurface{ʂæ˧}\newline
\classe{形容词}
\sens{1}
\begin{définition}\peng{Long.}\end{définition}
\begin{définition}\pcmn{长}\end{définition}
\begin{définition}\pfra{Long.}\end{définition}
\begin{exemple}\pnru{qʰɑ˧-ʂæ˧-gv̩˧}\hspace{5pt}\peng{extremely long}\hspace{5pt}\pcmn{非常长}\hspace{5pt}\pfra{très long}\end{exemple}
\begin{exemple}\pnru{le˧-ʈɤ˧-le˧-ʂæ˧ (+kʰɯ˧˥)}\hspace{5pt}\peng{to lengthen}\hspace{5pt}\pcmn{拉长}\hspace{5pt}\pfra{allonger, étirer}\end{exemple}\sens{2}
\begin{définition}\peng{Distant, far.}\end{définition}
\begin{définition}\pcmn{远}\end{définition}
\begin{définition}\pfra{Lointain, distant, éloigné.}\end{définition}
\end{entrée}

\begin{entrée}
{ʂæ˧˥}{₁}{ⓔʂæ˧˥ⓗ1}\formedesurface{ʂæ˧˥}\newline
\classe{动词}\ton{MH}
1\begin{définition}\peng{To lead along (by hand, halter…).}\end{définition}
\begin{définition}\pcmn{牵(牵着牛)}\end{définition}
\begin{définition}\pfra{Tenir un chien en laisse, mener un chien; mener, guider, amener (les vaches aux pâturages, etc).}\end{définition}
\begin{exemple}\pnru{kʰv̩˧ ʂæ˧˥}\hspace{5pt}\peng{to lead a dog; to hunt}\hspace{5pt}\pcmn{遛狗,狩猎}\hspace{5pt}\pfra{mener un chien; chasser}\end{exemple}
\begin{exemple}\pnru{kʰv̩˧ʂæ˧ hɯ˧˥}\hspace{5pt}\peng{gone hunting, out hunting}\hspace{5pt}\pcmn{狩猎去了}\hspace{5pt}\pfra{parti chasser, parti à la chasse}\end{exemple}
\end{entrée}

\begin{entrée}
{ʂæ˧˥}{₂}{ⓔʂæ˧˥ⓗ2}\formedesurface{ʂæ˧˥}\newline
\classe{动词}
2
\sens{1}
\begin{définition}\peng{To tie into bundles.}\end{définition}
\begin{définition}\pcmn{捆成一包}\end{définition}
\begin{définition}\pfra{Attacher, nouer en bottes.}\end{définition}
\begin{exemple}\pnru{le˧-ʂæ˧˥}\hspace{5pt}\peng{|fg{accomp}}\hspace{5pt}\pcmn{|fg{accomp}}\hspace{5pt}\pfra{|fg{accomp}}\end{exemple}
\begin{exemple}\pnru{hɑ˧ ʂæ˩}\hspace{5pt}\peng{to tie freshly cut rice into bundles}\hspace{5pt}\pcmn{刚收割的稻子,捆成捆}\hspace{5pt}\pfra{nouer le riz coupé en bottes}\end{exemple}\sens{2}
\begin{définition}\peng{To wrap, to pack.}\end{définition}
\begin{définition}\pcmn{包}\end{définition}
\begin{définition}\pfra{Envelopper, emballer (monosyllabique).}\end{définition}
\begin{exemple}\pnru{ʂæ˩∼ʂæ˧˥}\hspace{5pt}\peng{|fg{red}: to wrap, to pack}\hspace{5pt}\pcmn{重叠:包一包}\hspace{5pt}\pfra{|fg{red}: emballer, envelopper}\end{exemple}
\begin{exemple}\pnru{ʂæ˩∼ʂæ˧-ze˥}\hspace{5pt}\peng{|fg{red} |fg{accomp}}\hspace{5pt}\pcmn{|fg{red} |fg{accomp}}\hspace{5pt}\pfra{|fg{red} |fg{accomp}}\end{exemple}
\begin{exemple}\pnru{tso˧∼tso˧ ʂæ˥∼ʂæ˩}\hspace{5pt}\peng{to wrap things}\hspace{5pt}\pcmn{包一包东西}\hspace{5pt}\pfra{emballer des choses}\end{exemple}
\end{entrée}

\begin{entrée}
{ʂæ˧˥α}{}{ⓔʂæ˧˥α}\formedesurface{ɖɯ˧ ʂæ˧˥}\newline
\classe{量词}\ton{MHα}\begin{définition}\peng{A sheaf of cut rice or other crop (the amount cut at one go with a sickle and immediately tied together with one sprig).}\end{définition}
\begin{définition}\pcmn{量词:捆}\end{définition}
\begin{définition}\pfra{Classificateur des gerbes: ce qu'on coupe en un coup de faucille et attache d'un brin.}\end{définition}
\begin{exemple}\pnru{zɯ˧ | ɖɯ˧-ʂæ˧˥}\hspace{5pt}\peng{a sheaf of grass}\hspace{5pt}\pcmn{一捆草}\hspace{5pt}\pfra{une gerbe d'herbe (nouée ensemble par un brin)}\end{exemple}
\begin{exemple}\pnru{ɕi˧ɭɯ˧ | ɖɯ˧-ʂæ˧˥}\hspace{5pt}\peng{a sheaf of rice}\hspace{5pt}\pcmn{一捆稻谷}\hspace{5pt}\pfra{une gerbe de riz (nouée par un brin)}\end{exemple}
\end{entrée}

\begin{entrée}
{ʂæ˧ɖæ\#˥}{}{ⓔʂæ˧ɖæ\#˥}\formedesurface{ʂæ˧ɖæ˧}\newline
\classe{名词}\ton{\#H}
\paradigme{\pcmn{:} \p{}}
\begin{définition}\peng{Difference in length.}\end{définition}
\begin{définition}\pcmn{长度区别}\end{définition}
\begin{définition}\pfra{Différence de longueur.}\end{définition}
\begin{exemple}\pnru{ʂæ˧ɖæ˧ di˥, | mɤ˧-dʑɤ˩!}\hspace{5pt}\peng{If there are differences in length, it's not good / it won't do! (Context: explaining which trees to fell when in need of timber for housebuilding; the trees need to be about the same size.)}\hspace{5pt}\pcmn{如果长短不一,不好!/不行!(情景:解释砍树时如何选择合适的树)}\hspace{5pt}\pfra{S'il y a des différences de longueur, c'est vilain/ça ne convient pas! (Contexte: explication au sujet du choix d'arbres à abattre pour obtenir du bois de charpente.)}\end{exemple}
\begin{exemple}\pnru{ʂæ˧ɖæ˧ | mɤ˧-di˩!}\hspace{5pt}\peng{There are no differences in length! (i.e. the timber is suitable for use in construction; same context as previous example)}\hspace{5pt}\pcmn{没有长度区别,都一样齐!(等于是好的建房木料)(情景:同上)}\hspace{5pt}\pfra{il n'y a pas de différences de longueur (=c'est très bien)! (Même contexte que ci-dessus: choix d'arbres à abattre pour obtenir du bois de charpente)}\end{exemple}
\end{entrée}

\begin{entrée}
{ʂæ˥-ljɤ˩}{}{ⓔʂæ˥-ljɤ˩}\formedesurface{ʂæ˧ljɤ˩}\newline
\classe{动词}\ton{H.L}\begin{définition}\peng{To discuss.}\end{définition}
\begin{définition}\pcmn{商量(汉语借词}\end{définition}
\begin{définition}\pfra{Discuter.}\end{définition}
\end{entrée}

\begin{entrée}
{ʂæ˧-lo˩pv˩}{}{ⓔʂæ˧-lo˩pv˩}\formedesurface{ʂæ˧lo˩pv˩}\newline
\classe{名词}\ton{-L}\begin{définition}\peng{Scabious.}\end{définition}
\begin{définition}\pcmn{山萝卜}\end{définition}
\begin{définition}\pfra{Cerfeuil.}\end{définition}
\end{entrée}

\begin{entrée}
{ʂæ˧pʰi˧}{}{ⓔʂæ˧pʰi˧}\formedesurface{ʂæ˧pʰi˧}\newline
\classe{名词}\ton{M}\begin{définition}\peng{Commodity, goods, merchandise.}\end{définition}
\begin{définition}\pcmn{商品}\end{définition}
\begin{définition}\pfra{Marchandise, objet qui peut se vendre au marché.}\end{définition}
\end{entrée}

\begin{entrée}
{ʂæ˩ɻ̃˩}{}{ⓔʂæ˩ɻ̃˩}\formedesurface{ʂæ˩ɻ̃˩˥}\newline
\classe{名词}\ton{L}
\paradigme{\pcmn{:} \p{}}
\begin{définition}\peng{Bone.}\end{définition}
\begin{définition}\pcmn{骨头}\end{définition}
\begin{définition}\pfra{Os, ossement.}\end{définition}
\end{entrée}

\begin{entrée}
{ʂæ˧ʁwɤ˩}{}{ⓔʂæ˧ʁwɤ˩}\formedesurface{ʂæ˧ʁwɤ˩}\newline
\classe{名词}\ton{L\#}\begin{définition}\peng{Shuhe: the name of a village in the Lijiang plain.}\end{définition}
\begin{définition}\pcmn{束河(旧称:龙泉):丽江坝子里的一个村落。由于束河商人多,经常有束河人到永宁等地,使得相当多的永宁人熟悉那个村落名。}\end{définition}
\begin{définition}\pfra{Shuhe: nom d'un village de la plaine de Lijiang (anciennement Longquan). Les terres de ce village étaient médiocres, et beaucoup de ses habitants se tournaient vers le commerce et voyageaient dans toute la région, d'où le fait que le nom de ce village soit connu à Yongning.}\end{définition}
\end{entrée}

\begin{entrée}
{ʂæ˧tsɯ˧}{}{ⓔʂæ˧tsɯ˧}\formedesurface{ʂæ˧tsɯ˧}\newline
\classe{名词}\ton{M}
\paradigme{\pcmn{:} \p{}}
\begin{définition}\peng{Kaftan: clothing that children used to wear before they came of age: a loose robe (the same for girls and boys); also worn by adult men in earlier times.}\end{définition}
\begin{définition}\pcmn{裋、卡夫坦长衣:成年前男女小孩均穿的裋,成年男人也穿}\end{définition}
\begin{définition}\pfra{Caftan: vêtement que portaient les enfants avant leurs treize ans: robe ample (la même pour les filles et les garçons); anciennement, les hommes aussi portaient ce type de vêtement.}\end{définition}
\end{entrée}

\begin{entrée}
{ʂe˥}{₁}{ⓔʂe˥ⓗ1}\formedesurface{ʂe˧}\newline
\classe{名词}\ton{\#H}
1\begin{définition}\peng{Meat, flesh.}\end{définition}
\begin{définition}\pcmn{肉,肌肉}\end{définition}
\begin{définition}\pfra{Viande, chair.}\end{définition}
\end{entrée}

\begin{entrée}
{ʂe˥}{₂}{ⓔʂe˥ⓗ2}\formedesurface{ʂe˧}\newline
\classe{名词}\ton{\#H}
2\begin{définition}\peng{Unripe cereals.}\end{définition}
\begin{définition}\pcmn{未熟粮食}\end{définition}
\begin{définition}\pfra{Céréales pas encore mûres: céréales en herbe, dont on voit déjà l'épi mais dont l'épi ne s'est pas encore incliné sous le poids du grain.}\end{définition}
\begin{exemple}\pnru{ʂe˧do˧˥}\hspace{5pt}\peng{unripe cereals}\hspace{5pt}\pcmn{未熟粮食}\hspace{5pt}\pfra{même sens}\end{exemple}
\end{entrée}

\begin{entrée}
{ʂe˧α}{}{ⓔʂe˧α}\formedesurface{ʂe˧}\newline
\classe{动词}\ton{Mα}\begin{définition}\peng{To look for, to search for; to procure, to get.}\end{définition}
\begin{définition}\pcmn{寻找}\end{définition}
\begin{définition}\pfra{Chercher; se procurer.}\end{définition}
\begin{exemple}\pnru{le˧-ʂe˧ le˧-ɖɯ˧-ze˧!}\hspace{5pt}\peng{(I) looked for something, and I found it!}\hspace{5pt}\pcmn{(我)找了……就找到了! / 找到了!}\hspace{5pt}\pfra{(j'ai) cherché et (j'ai) trouvé!}\end{exemple}
\begin{exemple}\pnru{hĩ˧ ɖɯ˧-v̩˧ ʂe˧}\hspace{5pt}\peng{literally ‘to look for someone'; meaning: to visit someone of the opposite sex, to frequent someone of the opposite sex (this is typically a masculine activity)}\hspace{5pt}\pcmn{直译:‘找一个人’。实际含义:去访问异性的人(一般是男人去访问女人)}\hspace{5pt}\pfra{littéralement ‘chercher quelqu'un'; sens: fréquenter quelqu'un du sexe opposé, rendre visite à quelqu'un du sexe opposé (généralement: se dit d'un homme)}\end{exemple}
\begin{exemple}\pnru{hĩ˧ ʂe˩}\hspace{5pt}\peng{to take a wife, to marry a wife}\hspace{5pt}\pcmn{娶媳妇}\hspace{5pt}\pfra{prendre femme, épouser une femme}\end{exemple}
\begin{exemple}\pnru{tso˧∼tso˧ ʂe˩}\hspace{5pt}\peng{to look for things}\hspace{5pt}\pcmn{找东西}\hspace{5pt}\pfra{chercher quelque chose}\end{exemple}
\begin{exemple}\pnru{lo˧ mɤ˧-dʑo˧, | lo˧ ʂe˧!}\hspace{5pt}\peng{[(S)he] looks for complications / creates unnecessary complications! (Literally: ‘to look for work when there isn't any'.)}\hspace{5pt}\pcmn{没事找事!}\hspace{5pt}\pfra{[Il/elle] se crée des complications / se donner du travail!}\end{exemple}
\begin{exemple}\pnru{le˧-ʂe˧ tʰi˧-tɕɯ˥}\hspace{5pt}\peng{to prepare (e.g. ingredients for a recipe, luggage for travel), to get (something) ready}\hspace{5pt}\pcmn{准备(做饭的材料、旅途用品……)}\hspace{5pt}\pfra{préparer (des ingrédients pour une recette, ses bagages…)}\end{exemple}
\end{entrée}

\begin{entrée}
{ʂe˩}{}{ⓔʂe˩}\formedesurface{ʂe˧}\newline
\classe{名词}\ton{L}\begin{définition}\peng{Iron (monosyllable).}\end{définition}
\begin{définition}\pcmn{铁(单音节)}\end{définition}
\begin{définition}\pfra{Fer (monosyllabe).}\end{définition}
\end{entrée}

\begin{entrée}
{ʂe˩β}{}{ⓔʂe˩β}\formedesurface{ʂe˩˥}\newline
\classe{动词}\ton{Lβ}\begin{définition}\peng{To urinate.}\end{définition}
\begin{définition}\pcmn{小便,尿; 屙尿; 解溲; 拉(屎)}\end{définition}
\begin{définition}\pfra{Uriner, pisser, faire pipi; déféquer.}\end{définition}
\begin{exemple}\pnru{dʑi˧ ʂe˧˥}\hspace{5pt}\peng{to urinate}\hspace{5pt}\pcmn{屙尿}\hspace{5pt}\pfra{pisser}\end{exemple}
\begin{exemple}\pnru{qʰæ˧ ʂe˧˥}\hspace{5pt}\peng{to defecate}\hspace{5pt}\pcmn{拉屎}\hspace{5pt}\pfra{déféquer}\end{exemple}
\begin{exemple}\pnru{le˧-ʂe˩-ze˩}\hspace{5pt}\peng{|fg{accomp} \_ |fg{pfv}}\hspace{5pt}\pcmn{尿了}\hspace{5pt}\pfra{|fg{accomp} \_ |fg{pfv}}\end{exemple}
\begin{exemple}\pnru{ɖɯ˧-ʈʰɤ˧ ʂe˧˥}\hspace{5pt}\peng{to urinate a drop}\hspace{5pt}\pcmn{尿一滴尿}\hspace{5pt}\pfra{pisser une goutte}\end{exemple}
\end{entrée}

\begin{entrée}
{ʂe˧bæ˧}{}{ⓔʂe˧bæ˧}\formedesurface{ʂe˧bæ˧}\newline
\classe{名词}\ton{M}
\paradigme{\pcmn{:} \p{}}
\begin{définition}\peng{Necklace; chain.}\end{définition}
\begin{définition}\pcmn{项圈、项链,锁链}\end{définition}
\begin{définition}\pfra{Collier; chaîne.}\end{définition}
\begin{exemple}\pnru{ŋv̩˩-ʂe˩bæ˥}\hspace{5pt}\peng{silver necklace}\hspace{5pt}\pcmn{银项链}\hspace{5pt}\pfra{collier en argent}\end{exemple}
\begin{exemple}\pnru{hæ̃˩-ʂe˩bæ˥}\hspace{5pt}\peng{gold necklace}\hspace{5pt}\pcmn{金项链}\hspace{5pt}\pfra{collier en or}\end{exemple}
\begin{exemple}\pnru{ʂe˧mo˧ʂe˧bæ˧, | kʰv̩˩mi˩ pʰæ˩˥!}\hspace{5pt}\peng{The iron necklace is used to tie the dog!}\hspace{5pt}\pcmn{铁链,是来用拴狗的!}\hspace{5pt}\pfra{Le collier de fer, c'est pour attacher le chien!}\end{exemple}
\begin{exemple}\pnru{kʰi˧-ʂe˧bæ˥, | ʂe˧mo˧ po˧-ɳɯ˧ | gv̩˩˥!}\hspace{5pt}\peng{The door's chain (the chain used to lock the door) is made of iron!}\hspace{5pt}\pcmn{铁链,是来用拴狗的!}\end{exemple}
\end{entrée}

\begin{entrée}
{ʂe˧bv̩\#˥}{}{ⓔʂe˧bv̩\#˥}\formedesurface{ʂe˧bv̩˧}\newline
\classe{名词}\ton{\#H}\begin{définition}\peng{Sausage, dried meat preserved in intestines.}\end{définition}
\begin{définition}\pcmn{香肠,把瘦肉装在肠子里}\end{définition}
\begin{définition}\pfra{Saucisse; viande séchée conservée dans les intestins.}\end{définition}
\end{entrée}

\begin{entrée}
{ʂe˧di˩}{}{ⓔʂe˧di˩}\formedesurface{ʂe˧di˩}\newline
\classe{形容词}\ton{L\#}\begin{définition}\peng{Fat (person).}\end{définition}
\begin{définition}\pcmn{胖}\end{définition}
\begin{définition}\pfra{Gros.}\end{définition}
\begin{exemple}\pnru{ʂe˧ di˩-ze˩!}\hspace{5pt}\peng{(He/she) has put on weight!}\hspace{5pt}\pcmn{胖了!}\hspace{5pt}\pfra{(il/elle) a grossi!}\end{exemple}
\end{entrée}

\begin{entrée}
{ʂe˧dzo\#˥}{}{ⓔʂe˧dzo\#˥}\formedesurface{ʂe˧dzo˧}\newline
\classe{名词}\ton{\#H}
\paradigme{\pcmn{:} \p{}}
\begin{définition}\peng{Cooking table: a wooden piece of furniture on which one places the chopping board.}\end{définition}
\begin{définition}\pcmn{放案板的家具}\end{définition}
\begin{définition}\pfra{Meuble de cuisine: structure en bois sur laquelle on fait la cuisine: on y pose la planche à découper, les ustensiles…}\end{définition}
\end{entrée}

\begin{entrée}
{ʂe˧kʰɯ˧}{}{ⓔʂe˧kʰɯ˧}\formedesurface{ʂe˧kʰɯ˧}\newline
\classe{名词}\ton{M}
\paradigme{\pcmn{:} \p{}}
\begin{définition}\peng{Tripod.}\end{définition}
\begin{définition}\pcmn{三脚架}\end{définition}
\begin{définition}\pfra{Trépied de fer (dans le foyer, sur lequel on pose une casserole, une poële, une bouilloire…).}\end{définition}
\end{entrée}

\begin{entrée}
{ʂe˩lɑ˩}{}{ⓔʂe˩lɑ˩}\formedesurface{ʂe˩lɑ˩˥}\newline
\classe{名词}\ton{L}\begin{définition}\peng{To forge.}\end{définition}
\begin{définition}\pcmn{打铁}\end{définition}
\begin{définition}\pfra{Forger, battre le fer.}\end{définition}
\end{entrée}

\begin{entrée}
{ʂe˩-lɑ˩-hĩ˥}{}{ⓔʂe˩-lɑ˩-hĩ˥}\formedesurface{ʂe˩lɑ˩hĩ˥}\newline
\classe{名词}\ton{L+H\#}
\paradigme{\pcmn{:} \p{}}
\begin{définition}\peng{Blacksmith.}\end{définition}
\begin{définition}\pcmn{铁匠}\end{définition}
\begin{définition}\pfra{Forgeron.}\end{définition}
\begin{exemple}\pnru{ʂe˩lɑ˩-hĩ˥ hĩ˩}\hspace{5pt}\peng{blacksmith}\hspace{5pt}\pcmn{铁匠}\hspace{5pt}\pfra{forgeron}\end{exemple}
\end{entrée}

\begin{entrée}
{ʂe˩mɤ˩}{}{ⓔʂe˩mɤ˩}\formedesurface{ʂe˩mɤ˩˥}\newline
\classe{名词}\ton{L}\begin{définition}\peng{Fat meat.}\end{définition}
\begin{définition}\pcmn{肥肉}\end{définition}
\begin{définition}\pfra{Viande grasse.}\end{définition}
\end{entrée}

\begin{entrée}
{ʂe˧mi˧}{}{ⓔʂe˧mi˧}\formedesurface{ʂe˧mi˧}\newline
\classe{名词}\ton{M}
\paradigme{\pcmn{:} \p{}}
\begin{définition}\peng{Louse.}\end{définition}
\begin{définition}\pcmn{虱子}\end{définition}
\begin{définition}\pfra{Pou.}\end{définition}
\end{entrée}

\begin{entrée}
{ʂe˧mo˧}{}{ⓔʂe˧mo˧}\formedesurface{ʂe˧mo˧}\newline
\classe{名词}\ton{M}\begin{définition}\peng{Iron (disyllable).}\end{définition}
\begin{définition}\pcmn{铁(双音节)}\end{définition}
\begin{définition}\pfra{Fer (disyllabe).}\end{définition}
\end{entrée}

\begin{entrée}
{ʂe˩-mo˧˥}{}{ⓔʂe˩-mo˧˥}\formedesurface{ʂe˩mo˧˥}\newline
\classe{名词}\ton{LM+MH\#}\begin{définition}\peng{Pine mushroom, matsutake, |\stylefi{Tricholoma matsutake}.}\end{définition}
\begin{définition}\pcmn{松茸}\end{définition}
\begin{définition}\pfra{Champignon des pins, matsutake, |\stylefi{Tricholoma matsutake}: un champignon comestible, rare et très apprécié.}\end{définition}
\end{entrée}

\begin{entrée}
{ʂe˧nɑ˩}{}{ⓔʂe˧nɑ˩}\formedesurface{ʂe˧nɑ˩}\newline
\classe{名词}\ton{L\#}\begin{définition}\peng{Lean meat.}\end{définition}
\begin{définition}\pcmn{瘦肉}\end{définition}
\begin{définition}\pfra{Viande maigre.}\end{définition}
\end{entrée}

\begin{entrée}
{ʂe˧ɲi˩}{}{ⓔʂe˧ɲi˩}\formedesurface{ʂe˧ɲi˩}\newline
\classe{名词}\ton{L\#}
\paradigme{\pcmn{:} \p{}}
\begin{définition}\peng{Fire tongs.}\end{définition}
\begin{définition}\pcmn{火钳}\end{définition}
\begin{définition}\pfra{Pince à braises.}\end{définition}
\end{entrée}

\begin{entrée}
{ʂe˧pv̩˩}{}{ⓔʂe˧pv̩˩}\formedesurface{ʂe˧pv̩˩}\newline
\classe{名词}\ton{L\#}\begin{définition}\peng{Cured meat; bacon.}\end{définition}
\begin{définition}\pcmn{腊肉}\end{définition}
\begin{définition}\pfra{Viande séchée.}\end{définition}
\end{entrée}

\begin{entrée}
{ʂe˧qʰv̩˧}{}{ⓔʂe˧qʰv̩˧}\formedesurface{ʂe˧qʰv̩˧}\newline
\classe{名词}\ton{M}
\paradigme{\pcmn{:} \p{}}
\begin{définition}\peng{Iron nail; nail.}\end{définition}
\begin{définition}\pcmn{铁钉,钉子}\end{définition}
\begin{définition}\pfra{Clou en fer.}\end{définition}
\begin{exemple}\pnru{ʂe˧qʰv̩˧ lɑ˧˥}\hspace{5pt}\peng{to hammer in a nail, to hit a nail}\hspace{5pt}\pcmn{钉钉子}\hspace{5pt}\pfra{enfoncer un clou, planter un clou}\end{exemple}
\end{entrée}

\begin{entrée}
{ʂe˧sɑ˩}{}{ⓔʂe˧sɑ˩}\formedesurface{ʂe˧sɑ˩}\newline
\classe{名词}\ton{L\#}
\paradigme{\pcmn{:} \p{}}
\begin{définition}\peng{Meat of the limbs of pig. This includes the four limbs; it usually refers to preserved meat, but can also be used to refer to the limbs of the living animal.}\end{définition}
\begin{définition}\pcmn{风干猪腿。把猪大腿的皮刮下来,留一层薄薄的瘦肉筋,使其绷紧,撑开,形成扇面,风干。}\end{définition}
\begin{définition}\pfra{Viande des membres du cochon: les membres postérieurs aussi bien que les membres antérieurs. Le terme s'emploie pour la pièce de boucherie: de la viande conservée (séchée) avec l'os; mais le même terme peut également s'employer pour désigner les membres de la bête vivante.}\end{définition}
\end{entrée}

\begin{entrée}
{ʂe˧-sɯ˧sɯ˥}{}{ⓔʂe˧-sɯ˧sɯ˥}\formedesurface{ʂe˧sɯ˧sɯ˥}\newline
\classe{名词}\ton{H\#}\begin{définition}\peng{Raw meat.}\end{définition}
\begin{définition}\pcmn{生肉}\end{définition}
\begin{définition}\pfra{Viande crue.}\end{définition}
\end{entrée}

\begin{entrée}
{ʂe˧ʂe˧}{}{ⓔʂe˧ʂe˧}\formedesurface{ʂe˧ʂe˧}\newline
\classe{动词}\ton{M}\begin{définition}\peng{To catch a cold.}\end{définition}
\begin{définition}\pcmn{着凉}\end{définition}
\begin{définition}\pfra{Prendre froid, attraper un rhume, attraper froid.}\end{définition}
\begin{exemple}\pnru{ʂe˧ʂe˧-ze˩}\hspace{5pt}\peng{|fg{pfv}}\hspace{5pt}\pcmn{着凉了}\hspace{5pt}\pfra{|fg{pfv}}\end{exemple}
\end{entrée}

\begin{entrée}
{ʂe˩ʂv̩˩}{}{ⓔʂe˩ʂv̩˩}\formedesurface{ʂe˩ʂv̩˩˥}\newline
\classe{名词}\ton{L}
\paradigme{\pcmn{:} \p{}}
\begin{définition}\peng{Nit, egg of louse.}\end{définition}
\begin{définition}\pcmn{虮子}\end{définition}
\begin{définition}\pfra{Lente, oeuf de pou.}\end{définition}
\end{entrée}

\begin{entrée}
{ʂe˧ʈʂe˩}{}{ⓔʂe˧ʈʂe˩}\formedesurface{ʂe˧ʈʂe˩}\newline
\classe{名词}\ton{L\#}
\paradigme{\pcmn{:} \p{}}
\begin{définition}\peng{Cotton fabric, cloth.}\end{définition}
\begin{définition}\pcmn{棉布,布料}\end{définition}
\begin{définition}\pfra{Tissu de coton.}\end{définition}
\end{entrée}

\begin{entrée}
{ʂe˧ʐe\#˥}{}{ⓔʂe˧ʐe\#˥}\formedesurface{ʂe˧ʐe˧}\newline
\classe{名词}\ton{\#H}
\paradigme{\pcmn{:} \p{}}
\begin{définition}\peng{Preserved pork meat.}\end{définition}
\begin{définition}\pcmn{腊肉,包括不同几类的腊肉,如火腿等。}\end{définition}
\begin{définition}\pfra{Viande de cochon préservée. Le terme recouvre diverses pièces de boucherie, dont le jambon.}\end{définition}
\end{entrée}

\begin{entrée}
{ʂɤ˩α}{}{ⓔʂɤ˩α}\formedesurface{ʂɤ˩˥}\newline
\classe{动词}\ton{Lα}\begin{définition}\peng{To tear, to rip.}\end{définition}
\begin{définition}\pcmn{撕(纸……)}\end{définition}
\begin{définition}\pfra{Déchirer (ex.: du papier).}\end{définition}
\begin{exemple}\pnru{tso˧∼tso˧ ʂɤ˥}\hspace{5pt}\peng{to tear things}\hspace{5pt}\pcmn{撕东西}\hspace{5pt}\pfra{déchirer des choses}\end{exemple}
\begin{exemple}\pnru{tso˧∼tso˧ ʂɤ˧∼ʂɤ˥ (+ze˩)}\hspace{5pt}\peng{to tear things}\hspace{5pt}\pcmn{撕东西}\hspace{5pt}\pfra{déchirer des choses}\end{exemple}
\begin{exemple}\pnru{le˧-ʂɤ˧∼ʂɤ˥+ze˩}\hspace{5pt}\peng{|fg{accomp} \_ |fg{red} |fg{pfv}}\hspace{5pt}\pcmn{撕了}\hspace{5pt}\pfra{|fg{accomp} \_ |fg{red} |fg{pfv}}\end{exemple}
\end{entrée}

\begin{entrée}
{ʂɤ˧do˧˥}{}{ⓔʂɤ˧do˧˥}\formedesurface{ʂɤ˧do˧˥}\newline
\classe{形容词}
\sens{1}
\begin{définition}\peng{Ashamed, embarrassed.}\end{définition}
\begin{définition}\pcmn{害羞}\end{définition}
\begin{définition}\pfra{Honteux.}\end{définition}
\begin{exemple}\pnru{ʂɤ˧do˧ mɤ˧-sɯ˥!}\hspace{5pt}\peng{[(S)he] is sullen / impudent / has no sense of shame}\hspace{5pt}\pcmn{不知羞耻!}\hspace{5pt}\pfra{(il/elle) est effronté(e), ne connaît pas la politesse, est malpoli}\end{exemple}\sens{2}
\begin{définition}\peng{Modest, demure, discreet, polite.}\end{définition}
\begin{définition}\pcmn{娴静、礼貌}\end{définition}
\begin{définition}\pfra{Pudique; poli.}\end{définition}
\begin{exemple}\pnru{ʈʂʰɯ˧ ʂɤ˧do˧-zo˥! / ʂɤ˧do˧ ʝi˥!}\hspace{5pt}\peng{(S)he is very modest/discreet/polite!}\hspace{5pt}\pcmn{他/她很娴静 / 很持重!}\hspace{5pt}\pfra{Cette personne a de la pudeur!}\end{exemple}
\end{entrée}

\begin{entrée}
{ʂɤ˧ɲi\#˥}{}{ⓔʂɤ˧ɲi\#˥}\formedesurface{ʂɤ˧ɲi˧}\newline
\classe{名词}\ton{\#H}
\paradigme{\pcmn{:} \p{}}
\begin{définition}\peng{Advice, suggestion, recommendation.}\end{définition}
\begin{définition}\pcmn{建议、意见}\end{définition}
\begin{définition}\pfra{Conseil, avis.}\end{définition}
\begin{exemple}\pnru{ʂɤ˧ɲi˧ ʑi˧˥}\hspace{5pt}\peng{to ask for advice / to ask for counsel}\hspace{5pt}\pcmn{请求意见,求教}\hspace{5pt}\pfra{demander un conseil / prendre le conseil (de quelqu'un)}\end{exemple}
\begin{exemple}\pnru{no˧ | hĩ˧-ki˧ | ʂɤ˧ɲi˧ mɤ˧-ʑi˧-zo˥!}\hspace{5pt}\peng{You shouldn't ask around for advice! / There is no need for you to ask for anyone's advice! (=You can make a decision by yourself.)}\hspace{5pt}\pcmn{你不要问人家的意见!}\hspace{5pt}\pfra{Tu n'as pas à prendre son conseil! / Tu n'as pas à prendre le conseil d'autrui [à ce sujet: à toi de décider]!}\end{exemple}
\begin{exemple}\pnru{ə˧tse˧ʝi˧-zo˥ | ʂɤ˧ɲi˧ ʑi˧-tso˧-ɲi˥?}\hspace{5pt}\peng{Why would you want to ask for (his/her) advice?}\hspace{5pt}\pcmn{你为什么要问(他的)意见!}\hspace{5pt}\pfra{Pourquoi donc lui demandes-tu conseil?}\end{exemple}
\end{entrée}

\begin{entrée}
{ʂɤ˩ŋɤ\#˥}{}{ⓔʂɤ˩ŋɤ\#˥}\formedesurface{ʂɤ˩ŋɤ˥}\newline
\classe{名词}\ton{LM+\#H}
\paradigme{\pcmn{:} \p{}}
\begin{définition}\peng{Gong.}\end{définition}
\begin{définition}\pcmn{锣}\end{définition}
\begin{définition}\pfra{Gong.}\end{définition}
\begin{exemple}\pnru{ʂɤ˩ŋɤ˧ lɑ˩}\hspace{5pt}\peng{to play the gong}\hspace{5pt}\pcmn{打锣}\hspace{5pt}\pfra{jouer du gong}\end{exemple}
\end{entrée}

\begin{entrée}
{ʂo˥}{}{ⓔʂo˥}\formedesurface{ʂo˧}\newline
\classe{动词}\ton{H}\begin{définition}\peng{To reap, to gather in.}\end{définition}
\begin{définition}\pcmn{收割}\end{définition}
\begin{définition}\pfra{Récolter.}\end{définition}
\begin{exemple}\pnru{le˧-ʂo˥-ze˩}\hspace{5pt}\peng{|fg{accomp} \_ |fg{pfv}}\hspace{5pt}\pcmn{收割了}\hspace{5pt}\pfra{|fg{accomp} \_ |fg{pfv}}\end{exemple}
\begin{exemple}\pnru{ɖɯ˧-kʰv̩˥ ɖɯ˧-ʂɯ˩ | gɤ˩-ʂo˥-ze˩!}\hspace{5pt}\peng{We have one harvest (of rice) every year!}\hspace{5pt}\pcmn{每年收一次稻谷!}\hspace{5pt}\pfra{on récolte (le riz) une fois par an!}\end{exemple}
\end{entrée}

\begin{entrée}
{ʂo˧}{}{ⓔʂo˧}\formedesurface{ʂo˧}\newline
\classe{感叹词}\ton{M}\begin{définition}\peng{Interjection to get pigs to move forward.}\end{définition}
\begin{définition}\pcmn{赶猪用的叹词:走!走!}\end{définition}
\begin{définition}\pfra{Interjection employée pour faire avancer les cochons, lorsqu'on les guide sur le chemin du pâturage: «Zou! / Allez!».}\end{définition}
\begin{exemple}\pnru{ʂo˧! / ʂo˧bɤ˩!}\hspace{5pt}\peng{interjection to get pigs to move forward}\hspace{5pt}\pcmn{赶猪用的叹词}\hspace{5pt}\pfra{Interjection employée pour faire avancer les cochons, lorsqu'on les guide sur le chemin du pâturage: «Zou! / Allez!»}\end{exemple}
\end{entrée}

\begin{entrée}
{ʂo˧}{}{ⓔʂo˧}\formedesurface{ʂo˧}\newline
\classe{动词}\ton{M}\begin{définition}\peng{To gather.}\end{définition}
\begin{définition}\pcmn{收集}\end{définition}
\begin{définition}\pfra{Rassembler, assembler, accumuler.}\end{définition}
\begin{exemple}\pnru{le˧-ʂo˧∼ʂo˧}\hspace{5pt}\peng{|fg{accomp} \_ |fg{red}}\hspace{5pt}\pcmn{|fg{accomp} \_ |fg{red}}\hspace{5pt}\pfra{|fg{accomp} \_ |fg{red}}\end{exemple}
\begin{exemple}\pnru{ʂo˧∼ʂo˧-zo˧-ho˩-ze˩}\hspace{5pt}\peng{We are going to have to gather (things)}\hspace{5pt}\pcmn{该收集一些了。}\hspace{5pt}\pfra{Il va falloir rassembler/assembler.}\end{exemple}
\end{entrée}

\begin{entrée}
{ʂo˧˥}{₁}{ⓔʂo˧˥ⓗ1}\formedesurface{ʂo˧˥}\newline
\classe{动词}\ton{MH}
1\begin{définition}\peng{To slip, to slide.}\end{définition}
\begin{définition}\pcmn{滑,光滑(路……)}\end{définition}
\begin{définition}\pfra{Glisser.}\end{définition}
\begin{exemple}\pnru{mv̩˩tɕo˧ ʂo˧˥}\hspace{5pt}\peng{to slide down, to slip to the floor}\hspace{5pt}\pcmn{滑下、滑倒}\hspace{5pt}\pfra{glisser vers le bas, glisser par terre}\end{exemple}
\begin{exemple}\pnru{ʈʂʰɯ˧ | le˧-ʂo˧˥, | tʰi˧-ʈwæ˧-ze˥}\hspace{5pt}\peng{(S)he slipped and fell down}\hspace{5pt}\pcmn{他滑了一跤}\hspace{5pt}\pfra{il a glissé et il est tombé}\end{exemple}
\begin{exemple}\pnru{ʂo˩∼ʂo˧˥}\hspace{5pt}\peng{|fg{red}}\hspace{5pt}\pcmn{重叠}\hspace{5pt}\pfra{|fg{red}}\end{exemple}
\begin{exemple}\pnru{ɖæ˩ʂo˩˥ / ɖæ˩ʂo˩-ze˥}\hspace{5pt}\peng{to slide down}\hspace{5pt}\pcmn{往下滑}\hspace{5pt}\pfra{glisser; dévaler une pente en glissant}\end{exemple}
\begin{exemple}\pnru{no˧ | ɖæ˩ʂo˩∼ɖæ˥ʂo˩! |}\hspace{5pt}\peng{You are really cunning! (A criticism of someone who is not direct, not honest, who does not have a proper attitude: giving a slimy impression.)}\hspace{5pt}\pcmn{你真滑头!}\hspace{5pt}\pfra{Tu es bien malhonnête! (Critique de quelqu'un qui n'est pas franc et direct, qui est faux jeton, qui n'a pas une bonne attitude, donnant une impression huileuse: qui s'esquive et se dérobe, comme un objet glissant qui se dérobe à la prise.)}\end{exemple}
\end{entrée}

\begin{entrée}
{ʂo˧˥}{₂}{ⓔʂo˧˥ⓗ2}\formedesurface{ʂo˧˥}\newline
\classe{形容词}\ton{MH}
2\begin{définition}\peng{Slippery.}\end{définition}
\begin{définition}\pcmn{光滑(路……)}\end{définition}
\begin{définition}\pfra{Lisse, glissant.}\end{définition}
\begin{exemple}\pnru{mɤ˩ ʂo˩-ʂo˥ |}\hspace{5pt}\peng{slippery with grease}\hspace{5pt}\pcmn{油腻腻、滑腻}\hspace{5pt}\pfra{huileux, tout poisseux de graisse}\end{exemple}
\begin{exemple}\pnru{ɲi˧to˧ ɖɯ˧-ɭɯ˧ | dze˧-ʂo˧∼ʂo˥}\hspace{5pt}\peng{(her/his) whole mouth was slippery with sugar}\hspace{5pt}\pcmn{他嘴巴被糖粘得黏黏的}\hspace{5pt}\pfra{toute (sa) bouche est/était pleine de sucre / toute poisseuse à force de sucre}\end{exemple}
\end{entrée}

\begin{entrée}
{ʂo˩α}{}{ⓔʂo˩α}\formedesurface{ʂo˩˥}\newline
\classe{形容词}\ton{Lα}\begin{définition}\peng{Clean; clear (water).}\end{définition}
\begin{définition}\pcmn{干净、整洁,本质干净,清(水)}\end{définition}
\begin{définition}\pfra{Propre (sens propre ou figuré); claire (eau).}\end{définition}
\begin{exemple}\pnru{ʂo˩-hĩ˩˥}\hspace{5pt}\peng{|fg{nmlz}}\hspace{5pt}\pcmn{干净的}\hspace{5pt}\pfra{|fg{nmlz}}\end{exemple}
\begin{exemple}\pnru{mɤ˧-ʂo˩}\hspace{5pt}\peng{not clean}\hspace{5pt}\pcmn{不干净、脏}\hspace{5pt}\pfra{sale, malpropre}\end{exemple}
\begin{exemple}\pnru{ʈʂʰɯ˧ | ʂo˩-hĩ˩ ɲi˥. |}\hspace{5pt}\peng{This is clean.}\hspace{5pt}\pcmn{这是干净的。}\hspace{5pt}\pfra{C'est propre.}\end{exemple}
\begin{exemple}\pnru{dʑɯ˧ ʂo˧}\hspace{5pt}\peng{clean water, clear water}\hspace{5pt}\pcmn{清水、干净的水}\hspace{5pt}\pfra{de l'eau claire, de l'eau propre}\end{exemple}
\end{entrée}

\begin{entrée}
{ʂo˩qæ˩}{}{ⓔʂo˩qæ˩}\formedesurface{ʂo˩qæ˩˥}\newline
\classe{形容词}\ton{L}\begin{définition}\peng{Very clean.}\end{définition}
\begin{définition}\pcmn{很干净}\end{définition}
\begin{définition}\pfra{Tout propre.}\end{définition}
\begin{exemple}\pnru{ʂo˩qæ˩˥ ◊ -gv̩˩}\hspace{5pt}\peng{very clean}\hspace{5pt}\pcmn{很干净}\hspace{5pt}\pfra{tout propre}\end{exemple}
\begin{exemple}\pnru{ɑ˩ʁo˧ | le˧-gv̩˧∼gv̩˥ | ʂo˩qæ˩˥ ◊ -gv̩˩}\hspace{5pt}\peng{to put the house in order, that it be very clean}\hspace{5pt}\pcmn{家收拾得干干净净}\hspace{5pt}\pfra{ranger la maison, qu'elle soit bien propre}\end{exemple}
\end{entrée}

\begin{entrée}
{ʂo˧ʂo˧}{}{ⓔʂo˧ʂo˧}\formedesurface{ʂo˧ʂo˧}\newline
\classe{动词}\ton{M}\begin{définition}\peng{To prepare.}\end{définition}
\begin{définition}\pcmn{准备}\end{définition}
\begin{définition}\pfra{Préparer.}\end{définition}
\begin{exemple}\pnru{le˧-ʂo˧∼ʂo˧}\hspace{5pt}\peng{|fg{accomp}}\hspace{5pt}\pcmn{|fg{accomp}}\hspace{5pt}\pfra{|fg{accomp}}\end{exemple}
\end{entrée}

\begin{entrée}
{‑ʂo˧∼ʂo˩}{}{ⓔ‑ʂo˧∼ʂo˩}\formedesurface{--}\newline
\classe{后缀}\ton{L\#}\begin{définition}\peng{In association with a |fg{n}, indicates “plenty of, abundance of".}\end{définition}
\begin{définition}\pcmn{形容词化:……乎乎}\end{définition}
\begin{définition}\pfra{Morphème d'adjectivation: signifie ‘tout N-eux’. Par exemple, ajouté à ‘graisse’, cela donne ‘tout graisseux, plein de graisse’.}\end{définition}
\begin{exemple}\pnru{mɤ˩-ʂo˩∼ʂo˥}\hspace{5pt}\peng{full of grease, greasy}\hspace{5pt}\pcmn{油乎乎}\hspace{5pt}\pfra{plein de graisse}\end{exemple}
\begin{exemple}\pnru{dze˧-ʂo˧∼ʂo˥}\hspace{5pt}\peng{full of sugar, sugary all over}\hspace{5pt}\pcmn{甜乎乎}\hspace{5pt}\pfra{plein de sucre}\end{exemple}
\begin{exemple}\pnru{si˧-ʂo˧∼ʂo˥}\hspace{5pt}\peng{full of wood, replete with wood, entirely made of wood}\hspace{5pt}\pcmn{有很多木头,如:来形容一个新建的木头房子:面面都是新鲜木头,给人的感觉是“木头呼呼”}\hspace{5pt}\pfra{plein de bois, tout plein de (beau) bois; ex.: d'une maison de bois nouvellement construite: jusque dans chaque recoin, c'est de bon bois, c'est «pur bois».}\end{exemple}
\begin{exemple}\pnru{ʂe˧-ʂo˧∼ʂo˥}\hspace{5pt}\peng{full of meat, replete with meat, with lots of meat (e.g. a dish contains lots of meat, not just a few small chippings drowned in vegetables)}\hspace{5pt}\pcmn{肉乎乎:比如一个菜含有很多肉,而不是一点点肉淹在一大堆蔬菜里面。}\hspace{5pt}\pfra{plein de viande: par exemple, un plat contient de la viande en abondance, pas juste quelques petits bouts noyés dans une masse de légumes}\end{exemple}
\begin{exemple}\pnru{tɕi˧-ʂo˧∼ʂo˥; ʁwɤ˧-ʂo˧∼ʂo˥; qʰv̩˧-ʂo˧∼ʂo˥}\hspace{5pt}\peng{full of clouds}\hspace{5pt}\pcmn{云乎乎}\hspace{5pt}\pfra{plein de nuages}\end{exemple}
\begin{exemple}\pnru{ʁwɤ˧-ʂo˧∼ʂo˥}\hspace{5pt}\peng{full of mountains, covered with mountains, with mountains on all sides}\hspace{5pt}\pcmn{“山乎乎”:到处都是山}\hspace{5pt}\pfra{plein de montagnes, entouré de montagnes de toutes parts}\end{exemple}
\begin{exemple}\pnru{qʰv̩˧-ʂo˧∼ʂo˥}\hspace{5pt}\peng{full of holes}\hspace{5pt}\pcmn{“洞乎乎”,有很多洞的}\hspace{5pt}\pfra{plein de trous}\end{exemple}
\begin{exemple}\pnru{ɬv̩˧-ʂo˧∼ʂo˥}\hspace{5pt}\peng{full of brains, with lots of brains}\hspace{5pt}\pcmn{脑髓乎乎、有很多脑髓(如:一道菜,里面都是脑髓)}\hspace{5pt}\pfra{plein de cervelle}\end{exemple}
\end{entrée}

\begin{entrée}
{ʂɻ̍˧˥}{}{ⓔʂɻ̍˧˥}\formedesurface{ʂɻ̍˧˥}\newline
\classe{形容词}\ton{MH}\begin{définition}\peng{Full.}\end{définition}
\begin{définition}\pcmn{满}\end{définition}
\begin{définition}\pfra{Rempli, plein.}\end{définition}
\begin{exemple}\pnru{le˧-ʂɻ̍˧-ze˥}\hspace{5pt}\peng{|fg{accomp} \_ |fg{pfv}}\hspace{5pt}\pcmn{满了}\hspace{5pt}\pfra{|fg{accomp} \_ |fg{pfv}}\end{exemple}
\end{entrée}

\begin{entrée}
{ʂɯ˧}{}{ⓔʂɯ˧}\formedesurface{ʂɯ˧}\newline
\classe{数词}\ton{M? H\#? (pas L)}\begin{définition}\peng{Seven.}\end{définition}
\begin{définition}\pcmn{七}\end{définition}
\begin{définition}\pfra{Sept.}\end{définition}
\end{entrée}

\begin{entrée}
{ʂɯ˧˥}{₁}{ⓔʂɯ˧˥ⓗ1}\formedesurface{ʂɯ˧˥}\newline
\classe{动词}\ton{MH}
1\begin{définition}\peng{To peel (with a knife).}\end{définition}
\begin{définition}\pcmn{削(用刀)}\end{définition}
\begin{définition}\pfra{Éplucher, peler, décortiquer (avec un instrument).}\end{définition}
\begin{exemple}\pnru{ɣɯ˩ ʂɯ˧˥ / ɣɯ˩ʂɯ˧ ze˥}\hspace{5pt}\peng{to peel, to peel off the skin}\hspace{5pt}\pcmn{削皮}\hspace{5pt}\pfra{éplucher la peau}\end{exemple}
\begin{exemple}\pnru{ɣɯ˩kɯ˧ ʂɯ˥}\hspace{5pt}\peng{to peel, to peel off the skin}\hspace{5pt}\pcmn{削皮}\hspace{5pt}\pfra{éplucher la peau}\end{exemple}
\begin{exemple}\pnru{jɤ˩jo˧ ɣɯ˥ʂɯ˩}\hspace{5pt}\peng{to peel potatoes}\hspace{5pt}\pcmn{削洋芋皮}\hspace{5pt}\pfra{peler des patates}\end{exemple}
\begin{exemple}\pnru{tso˧∼tso˧ ɣɯ˥ʂɯ˩}\hspace{5pt}\peng{to peel things}\hspace{5pt}\pcmn{削东西}\hspace{5pt}\pfra{éplucher des choses}\end{exemple}
\end{entrée}

\begin{entrée}
{ʂɯ˧˥}{₂}{ⓔʂɯ˧˥ⓗ2}\formedesurface{ʂɯ˧˥}\newline
\classe{形容词}\ton{MH}
2\begin{définition}\peng{New, fresh.}\end{définition}
\begin{définition}\pcmn{新}\end{définition}
\begin{définition}\pfra{Nouveau, neuf, frais.}\end{définition}
\begin{exemple}\pnru{ʂɯ˧-hĩ˧ ɲi˥!}\hspace{5pt}\peng{It's new!}\hspace{5pt}\pcmn{是新的!}\hspace{5pt}\pfra{c'est neuf!}\end{exemple}
\begin{exemple}\pnru{ʂe˧ ʂɯ˩}\hspace{5pt}\peng{fresh meat}\hspace{5pt}\pcmn{新鲜的肉}\hspace{5pt}\pfra{de la viande fraîche}\end{exemple}
\end{entrée}

\begin{entrée}
{ʂɯ˧α}{₁}{ⓔʂɯ˧αⓗ1}\formedesurface{ʂɯ˧}\newline
\classe{动词}\ton{Mα}
1\begin{définition}\peng{To leak.}\end{définition}
\begin{définition}\pcmn{漏}\end{définition}
\begin{définition}\pfra{Fuir, s'écouler, se répandre, se vider.}\end{définition}
\begin{exemple}\pnru{tʰi˧-ʂɯ˥∼ʂɯ˩(-ze˩)}\hspace{5pt}\peng{it is leaking}\hspace{5pt}\pcmn{漏了!}\hspace{5pt}\pfra{Ca fuit / ça se vide!}\end{exemple}
\begin{exemple}\pnru{mɤ˧-ʂɯ˥∼ʂɯ˩! | mɤ˧-ʑi˧!}\hspace{5pt}\peng{It does not leak; it does not flow out!}\hspace{5pt}\pcmn{没漏,没流出去!}\hspace{5pt}\pfra{Ca ne s'écoule pas, ça ne fuit pas! (/ʑi˧/: ‘s'écouler’)}\end{exemple}
\end{entrée}

\begin{entrée}
{ʂɯ˧α}{₂}{ⓔʂɯ˧αⓗ2}\formedesurface{ʂɯ˧}\newline
\classe{动词}\ton{Mα}
2\begin{définition}\peng{To die.}\end{définition}
\begin{définition}\pcmn{死}\end{définition}
\begin{définition}\pfra{Mourir, décéder.}\end{définition}
\begin{exemple}\pnru{le˧-ʂɯ˧-ho˩-ze˩}\hspace{5pt}\peng{It's going to die! (About a sick plant or animal)}\hspace{5pt}\pcmn{快要死了!(病了的植物、动物)}\hspace{5pt}\pfra{ça va mourir! (au sujet d'une plante ou d'un animal malade)}\end{exemple}
\begin{exemple}\pnru{mɤ˧-ʂɯ˧-sɯ˩!}\hspace{5pt}\peng{(He/she/it) is not dead yet!}\hspace{5pt}\pcmn{还没死!}\hspace{5pt}\pfra{(Il/elle/ce) n'est pas encore mort!}\end{exemple}
\begin{exemple}\pnru{no˧ | le˧-ʂɯ˧-bi˧-tsæ˧-ɲi˧-ze˩!}\hspace{5pt}\peng{Go and die! / May you die! (Imprecation)}\hspace{5pt}\pcmn{你去死吧!}\hspace{5pt}\pfra{Crève donc! / Crève, charogne! (imprécation/malédiction, qu'on lance sous le coup de la colère)}\end{exemple}
\begin{exemple}\pnru{ʂɯ˧-ze˧! | ʂɯ˧-ze˧!}\hspace{5pt}\peng{We're dead! (Expression of terror on the part of someone who thinks that (s)he is going to die. Context: someone taking a plane for the first time is terrified by the plane's changes of direction, and thinks that the plane is falling to the ground and is going to crash.)}\hspace{5pt}\pcmn{死了!死了!(情景:一个人第一次坐飞机,感到飞机飞得不稳定,会坠毁,大家会马上死。)}\hspace{5pt}\pfra{Expression de terreur de quelqu'un qui croit sa dernière heure arrivée; ex.: Ama prenant l'avion pour Kunming, et ayant l'impression que l'avion se renverse.}\end{exemple}
\end{entrée}

\begin{entrée}
{ʂɯ˩β}{}{ⓔʂɯ˩β}\formedesurface{ɖɯ˧ ʂɯ˩}\newline
\classe{量词}\ton{Lβ}\begin{définition}\peng{Times (repeating an action: doing something n times).}\end{définition}
\begin{définition}\pcmn{量词:次数}\end{définition}
\begin{définition}\pfra{Classificateur des fois (répétitions d'une action).}\end{définition}
\end{entrée}

\begin{entrée}
{ʂɯ˧dʑi˧}{}{ⓔʂɯ˧dʑi˧}\formedesurface{ʂɯ˧dʑi˧}\newline
\classe{名词}\ton{M}\begin{définition}\peng{Shroud, burial suit.}\end{définition}
\begin{définition}\pcmn{寿衣}\end{définition}
\begin{définition}\pfra{Linceul, suaire, vêtement mortuaire (de: ‘mourir’ et ‘habit’).}\end{définition}
\begin{exemple}\pnru{ʂɯ˧dʑi˧ ʐv̩˥}\hspace{5pt}\peng{to sew the burial suit, to sew the shroud}\hspace{5pt}\pcmn{缝寿衣}\hspace{5pt}\pfra{coudre les vêtements mortuaires, coudre le linceul}\end{exemple}
\end{entrée}

\begin{entrée}
{ʂɯ˩ʝi\#˥}{}{ⓔʂɯ˩ʝi\#˥}\formedesurface{ʂɯ˩ʝi˥}\newline
\classe{助词}\ton{LM+\#H}\begin{définition}\peng{Two years ago.}\end{définition}
\begin{définition}\pcmn{前年}\end{définition}
\begin{définition}\pfra{Il y a deux ans.}\end{définition}
\begin{exemple}\pnru{ʂɯ˩ʝi˥ | ɖɯ˧-kʰv̩˧˥}\hspace{5pt}\peng{two years ago}\hspace{5pt}\pcmn{前年}\hspace{5pt}\pfra{il y a deux ans, l'année il y a deux ans}\end{exemple}
\end{entrée}

\begin{entrée}
{ʂɯ˩kwæ˩ɻæ˥}{}{ⓔʂɯ˩kwæ˩ɻæ˥}\formedesurface{ʂɯ˩kwæ˩ɻæ˥}\newline
\classe{形容词}\ton{L+H\#}\begin{définition}\peng{Yellow.}\end{définition}
\begin{définition}\pcmn{黄}\end{définition}
\begin{définition}\pfra{Jaune.}\end{définition}
\begin{exemple}\pnru{ʂɯ˩kwæ˩ɻæ˥-hĩ˩ gv̩˩-ze˩}\hspace{5pt}\peng{[the book] has turned yellow!}\hspace{5pt}\pcmn{[书]变黄了!}\hspace{5pt}\pfra{[le livre] a jauni!}\end{exemple}
\begin{exemple}\pnru{ʂɯ˩kwæ˩˥ | ʂɯ˩kwæ˩˥ | gv̩˩}\hspace{5pt}\peng{very yellow}\hspace{5pt}\pcmn{深黄}\hspace{5pt}\pfra{tout jaune}\end{exemple}
\end{entrée}

\begin{entrée}
{ʂɯ˧-ɬi˧mi˧}{}{ⓔʂɯ˧-ɬi˧mi˧}\formedesurface{ʂɯ˧ɬi˧mi˧}\newline
\classe{名词}\ton{M}\begin{définition}\peng{7th month.}\end{définition}
\begin{définition}\pcmn{七月}\end{définition}
\begin{définition}\pfra{7e mois.}\end{définition}
\end{entrée}

\begin{entrée}
{ʂɯ˧ɲi˥}{}{ⓔʂɯ˧ɲi˥}\formedesurface{ʂɯ˧ɲi˥}\newline
\classe{助词}\ton{H\#}\begin{définition}\peng{The day before yesterday.}\end{définition}
\begin{définition}\pcmn{前天}\end{définition}
\begin{définition}\pfra{Avant-hier.}\end{définition}
\begin{exemple}\pnru{ʂɯ˧ɲi˥ | -ɖɯ˧ɲi˥}\hspace{5pt}\peng{the day before yesterday}\hspace{5pt}\pcmn{前天}\hspace{5pt}\pfra{avant-hier}\end{exemple}
\end{entrée}

\begin{entrée}
{ʂɯ˧ʂɯ˧-dzi˩}{}{ⓔʂɯ˧ʂɯ˧-dzi˩}\formedesurface{ʂɯ˧ʂɯ˧dzi˩}\newline
\classe{名词}\ton{-L}
\paradigme{\pcmn{:} \p{}}
\begin{définition}\peng{Root of Anhwei Barberry (|\stylefi{Berberis anhweiensis Ahrendt}), a plant used in Chinese medicine.}\end{définition}
\begin{définition}\pcmn{安徽刺黄柏、黄柏、刺黄柏、三颗针(一种中药)}\end{définition}
\begin{définition}\pfra{Racine de berberis, plante utilisée en médecine chinoise (|\stylefi{Berberis anhweiensis Ahrendt}).}\end{définition}
\end{entrée}

\begin{entrée}
{ʂɯ˧tɤ˧ɻ̍\#˥}{}{ⓔʂɯ˧tɤ˧ɻ̍\#˥}\formedesurface{ʂɯ˧tɤ˧ɻ̍˧}\newline
\classe{形容词}\ton{\#H}\begin{définition}\peng{Smooth (e.g. carefully planed wood).}\end{définition}
\begin{définition}\pcmn{平滑}\end{définition}
\begin{définition}\pfra{Lisse; par exemple: un pilier en bois, qui devient bien lisse par le travail du menuisier.}\end{définition}
\begin{exemple}\pnru{ʂɯ˧tɤ˧ɻ̍˧-zo˥}\hspace{5pt}\peng{very smooth}\hspace{5pt}\pcmn{很平滑}\hspace{5pt}\pfra{bien lisse}\end{exemple}
\begin{exemple}\pnru{ʂɯ˧tɤ˧ɻ̍˧ gv̩˧-ze˩}\hspace{5pt}\peng{(it) was made nice and smooth}\hspace{5pt}\pcmn{弄得平滑了}\hspace{5pt}\pfra{on l'a bien lissé, on l'a rendu bien lisse}\end{exemple}
\end{entrée}

\begin{entrée}
{ʂɯ˩tsɯ˧}{}{ⓔʂɯ˩tsɯ˧}\formedesurface{ʂɯ˩tsɯ˥}\newline
\classe{名词}\ton{LM}
\paradigme{\pcmn{:} \p{}}
\begin{définition}\peng{Pistol.}\end{définition}
\begin{définition}\pcmn{手枪}\end{définition}
\begin{définition}\pfra{Pistolet.}\end{définition}
\begin{exemple}\pnru{ʂɯ˩tsɯ˧ | ɖɯ˧-nɑ˧ | tʰi˧-pɤ˥∼pɤ˩}\hspace{5pt}\peng{to carry a pistol}\hspace{5pt}\pcmn{带手枪}\hspace{5pt}\pfra{porter un pistolet}\end{exemple}
\end{entrée}

\begin{entrée}
{ʂɯ˩tsɯ˧}{}{ⓔʂɯ˩tsɯ˧}\formedesurface{ʂɯ˩tsɯ˥}\newline
\classe{名词}\ton{LM}\begin{définition}\peng{Persimmon.}\end{définition}
\begin{définition}\pcmn{柿子(汉语借词)}\end{définition}
\begin{définition}\pfra{Kaki.}\end{définition}
\begin{exemple}\pnru{ʂɯ˩tsɯ˧ | ɖɯ˧-so˩-ɭɯ˩ hwæ˩-bi˩!}\hspace{5pt}\peng{Let's buy a few persimmons!}\hspace{5pt}\pcmn{买一些柿子吧!}\hspace{5pt}\pfra{(Je) vais acheter quelques kakis!}\end{exemple}
\end{entrée}

\begin{entrée}
{ʂɯ˧tsʰi˩}{}{ⓔʂɯ˧tsʰi˩}\formedesurface{ʂɯ˧tsʰi˩}\newline
\classe{数词}\ton{L\#}\begin{définition}\peng{70.}\end{définition}
\begin{définition}\pcmn{70}\end{définition}
\begin{définition}\pfra{70.}\end{définition}
\end{entrée}

\begin{entrée}
{ʂv̩˥}{}{ⓔʂv̩˥}\formedesurface{ʂv̩˧}\newline
\classe{名词}\ton{H}
\paradigme{\pcmn{:} \p{}}
\begin{définition}\peng{Dice.}\end{définition}
\begin{définition}\pcmn{骰子}\end{définition}
\begin{définition}\pfra{Dé.}\end{définition}
\begin{exemple}\pnru{ʂv̩˧ | ʐv̩˩-ɭɯ˩˥}\hspace{5pt}\peng{four dice (dice came in pairs)}\hspace{5pt}\pcmn{四个骰子}\hspace{5pt}\pfra{quatre dés (les dés allaient par quatre)}\end{exemple}
\end{entrée}

\begin{entrée}
{ʂv̩˧˥}{}{ⓔʂv̩˧˥}\formedesurface{ʂv̩˧˥}\newline
\classe{动词}\ton{MH}\begin{définition}\peng{To twist, to wring.}\end{définition}
\begin{définition}\pcmn{拧(拧毛巾)}\end{définition}
\begin{définition}\pfra{Tordre, essorer (vêtement).}\end{définition}
\begin{exemple}\pnru{le˧-ʂv̩˧-ze˥}\hspace{5pt}\peng{|fg{accomp} \_ |fg{pfv}}\hspace{5pt}\pcmn{拧了}\hspace{5pt}\pfra{|fg{accomp} \_ |fg{pfv}}\end{exemple}
\begin{exemple}\pnru{dʑi˧hṽ̩˧ ʂv̩˩}\hspace{5pt}\peng{to wring out clothes}\hspace{5pt}\pcmn{拧衣服}\hspace{5pt}\pfra{essorer des vêtements}\end{exemple}
\end{entrée}

\begin{entrée}
{ʂv̩˧β}{}{ⓔʂv̩˧β}\newline
\classe{动词}
\sens{1}
\begin{définition}\peng{To look after, to take care of (children).}\end{définition}
\begin{définition}\pcmn{带(孩子……)}\end{définition}
\begin{définition}\pfra{S'occuper de, surveiller.}\end{définition}
\begin{exemple}\pnru{zo˧mv̩˥ | ɖɯ˧-ɭɯ˧ ʂv̩˧}\hspace{5pt}\peng{to take care of a child, to look after a child}\hspace{5pt}\pcmn{带个孩子}\hspace{5pt}\pfra{garder un enfant, s'occuper d'un enfant}\end{exemple}
\begin{exemple}\pnru{zo˧mv̩˥ ʂv̩˩}\hspace{5pt}\peng{to take care of a child, to look after a child}\hspace{5pt}\pcmn{带孩子}\hspace{5pt}\pfra{garder un enfant, s'occuper d'un enfant}\end{exemple}
\begin{exemple}\pnru{le˧-ʂv̩˧ tʰi˧-kʰɯ˧˥ | tʰæ˧ɻ̍˩ so˩}\hspace{5pt}\peng{to oblige to study (a mother obliges a child to study)}\hspace{5pt}\pcmn{让他学习、要求他学习(家长管孩子,让他学习)}\hspace{5pt}\pfra{obliger à étudier (une mère oblige son enfant à faire ses devoirs)}\end{exemple}\sens{2}
\begin{définition}\peng{To lead (the way).}\end{définition}
\begin{définition}\pcmn{带(路)}\end{définition}
\begin{définition}\pfra{Mener, guider.}\end{définition}
\begin{exemple}\pnru{ʈæ˧ʂɯ˧ | ɖʐv̩˧ ʂv̩˧-po˧-bi˧-ho˥!}\hspace{5pt}\peng{Dashi is going to take care of his friends [taking them on a tourist trip to Yongning]}\hspace{5pt}\pcmn{达石要管朋友(带他们去永宁旅游)}\hspace{5pt}\pfra{Dashi va accompagner ses amis [=les emmener en excursion à Yongning]!}\end{exemple}
\begin{exemple}\pnru{ʈæ˧ʂɯ˧ | ɖʐv̩˧ ʂv̩˧-bi˧-ho˩!}\hspace{5pt}\peng{Dashi is going to take care of his friends [taking them on a tourist trip to Yongning]}\hspace{5pt}\pcmn{达石要管朋友(带他们去永宁旅游)}\hspace{5pt}\pfra{Dashi va accompagner ses amis [=les emmener en excursion à Yongning]!}\end{exemple}
\end{entrée}

\begin{entrée}
{ʂv̩˧ɖv̩˧}{}{ⓔʂv̩˧ɖv̩˧}\formedesurface{ʂv̩˧ɖv̩˧}\newline
\classe{动词}
\sens{1}
\begin{définition}\peng{To think.}\end{définition}
\begin{définition}\pcmn{想}\end{définition}
\begin{définition}\pfra{Penser, réfléchir.}\end{définition}
\begin{exemple}\pnru{ə˧tso˧ ʂv̩˧ɖv̩˧?}\hspace{5pt}\peng{What are you thinking about? / Where's your mind?}\hspace{5pt}\pcmn{在想什么?}\hspace{5pt}\pfra{A quoi [tu] penses?}\end{exemple}
\begin{exemple}\pnru{njɤ˧ | ɖɯ˧ bæ˧ ʂv̩˧dv̩˧}\hspace{5pt}\peng{I'm thinking about something.}\hspace{5pt}\pcmn{我在想一件事情。}\hspace{5pt}\pfra{je pense à quelque chose}\end{exemple}
\begin{exemple}\pnru{ʂv̩˧ɖv̩˧ tʰv̩˧}\hspace{5pt}\peng{to understand}\hspace{5pt}\pcmn{明白,想起}\hspace{5pt}\pfra{comprendre; se souvenir}\end{exemple}
\begin{exemple}\pnru{njɤ˧ | ʂv̩˧ɖv̩˧ tʰv̩˧}\hspace{5pt}\peng{I understand.}\hspace{5pt}\pcmn{我明白。}\hspace{5pt}\pfra{Je comprends.}\end{exemple}
\begin{exemple}\pnru{ʈʂʰɯ˧ | le˧-ʂv̩˧ɖv̩˧-le˧-tʰv̩˧-ze˧}\hspace{5pt}\peng{He has understood.}\hspace{5pt}\pcmn{他明白了。}\hspace{5pt}\pfra{Il a compris.}\end{exemple}\sens{2}
\begin{définition}\peng{To remember, to recollect, to recall.}\end{définition}
\begin{définition}\pcmn{想起、回忆}\end{définition}
\begin{définition}\pfra{Se souvenir.}\end{définition}
\begin{exemple}\pnru{ʂv̩˧ɖv̩˧ tʰv̩˧}\hspace{5pt}\peng{to remember}\hspace{5pt}\pcmn{想起}\hspace{5pt}\pfra{se souvenir}\end{exemple}
\begin{exemple}\pnru{njɤ˧ | ʂv̩˧ɖv̩˧ tʰv̩˧}\hspace{5pt}\peng{I remember}\hspace{5pt}\pcmn{我想起}\hspace{5pt}\pfra{je me souviens}\end{exemple}
\begin{exemple}\pnru{ʈʂʰɯ˧ | le˧-ʂv̩˧ɖv̩˧-le˧-tʰv̩˧-ze˧}\hspace{5pt}\peng{He remembers, he recollects}\hspace{5pt}\pcmn{他想起来了}\hspace{5pt}\pfra{il se rappelle, ça lui revient}\end{exemple}\sens{3}
\begin{définition}\peng{To miss, to long for; to feel sorrowful, sad, grieved.}\end{définition}
\begin{définition}\pcmn{想念、感到悲哀}\end{définition}
\begin{définition}\pfra{Avoir la nostalgie de.}\end{définition}
\begin{exemple}\pnru{ʂv̩˧ɖv̩˧ tʰv̩˧ | ʐwæ˩˥}\hspace{5pt}\peng{to be full of nostalgia}\hspace{5pt}\pcmn{特别想念}\hspace{5pt}\pfra{être plongé dans le chagrin}\end{exemple}
\begin{exemple}\pnru{njɤ˧ | no˩ ʂv̩˩ɖv̩˩˥}\hspace{5pt}\peng{I miss you!}\hspace{5pt}\pcmn{我想你!}\hspace{5pt}\pfra{tu me manques!}\end{exemple}
\begin{exemple}\pnru{ʂv̩˧ɖv̩˧ qʰwɤ˧ tʰv̩˧˥}\hspace{5pt}\peng{to have a fit of nostalgia}\hspace{5pt}\pcmn{想念}\hspace{5pt}\pfra{être nostalgique, avoir une crise de nostalgie}\end{exemple}
\begin{exemple}\pnru{ʂv̩˧ɖv̩˧ mɤ˧-zo˧}\hspace{5pt}\peng{There's no need to worry / feel unhappy}\hspace{5pt}\pcmn{不用发愁}\hspace{5pt}\pfra{on n'a pas à se faire de souci, il n'y a pas lieu de se morfondre}\end{exemple}
\end{entrée}

\begin{entrée}
{‑ʂv̩˧ɖv̩˧}{}{ⓔ‑ʂv̩˧ɖv̩˧}\formedesurface{ʂv̩˧ɖv̩˧}\newline
\classe{后缀}\ton{M}\begin{définition}\peng{Volitive: to want, to wish.}\end{définition}
\begin{définition}\pcmn{想、意志}\end{définition}
\begin{définition}\pfra{Volitif: vouloir, souhaiter.}\end{définition}
\end{entrée}

\begin{entrée}
{ʂv̩˩gv̩˩}{}{ⓔʂv̩˩gv̩˩}\formedesurface{ʂv̩˩gv̩˩˥}\newline
\classe{名词}\ton{L}
\paradigme{\pcmn{:} \p{}}
\begin{définition}\peng{Sickle.}\end{définition}
\begin{définition}\pcmn{镰刀}\end{définition}
\begin{définition}\pfra{Faucille.}\end{définition}
\end{entrée}

\begin{entrée}
{ʂv̩˧kʰɯ˩}{}{ⓔʂv̩˧kʰɯ˩}\formedesurface{ʂv̩˧kʰɯ˩}\newline
\classe{动词}\ton{L\#}\begin{définition}\peng{To bet.}\end{définition}
\begin{définition}\pcmn{赌博}\end{définition}
\begin{définition}\pfra{Parier.}\end{définition}
\begin{exemple}\pnru{ʂv̩˧kʰɯ˩ | -jɤ˩po˧}\hspace{5pt}\peng{same meaning}\hspace{5pt}\pcmn{同上}\hspace{5pt}\pfra{même sens}\end{exemple}
\end{entrée}

\begin{entrée}
{ʂv̩˩njɤ˥}{}{ⓔʂv̩˩njɤ˥}\formedesurface{ʂv̩˩njɤ˥}\newline
\classe{名词}\ton{LH}
\paradigme{\pcmn{:} \p{}}
\begin{définition}\peng{Tree bur; burl.}\end{définition}
\begin{définition}\pcmn{树瘤}\end{définition}
\begin{définition}\pfra{Nœud (sur un arbre).}\end{définition}
\begin{exemple}\pnru{ʂv̩˩njɤ˥ ɲi˩}\hspace{5pt}\peng{|fg{cop}}\hspace{5pt}\pcmn{是树瘤。}\hspace{5pt}\pfra{|fg{cop}}\end{exemple}
\end{entrée}

\begin{entrée}
{ʂv̩˧ʂv̩˧˥}{}{ⓔʂv̩˧ʂv̩˧˥}\formedesurface{ʂv̩˧ʂv̩˧˥}\newline
\classe{名词}\ton{MH\#}
\paradigme{\pcmn{:} \p{}}
\begin{définition}\peng{Paper.}\end{définition}
\begin{définition}\pcmn{纸}\end{définition}
\begin{définition}\pfra{Papier.}\end{définition}
\begin{exemple}\pnru{ʂv̩˧ʂv̩˧˥ | ɖɯ˧-pʰæ˧˥}\hspace{5pt}\peng{a sheet of paper}\hspace{5pt}\pcmn{一张纸}\hspace{5pt}\pfra{une feuille de papier}\end{exemple}
\begin{exemple}\pnru{ʂv̩˧ʂv̩˧ ɖɯ˧ pʰæ˧˥}\hspace{5pt}\peng{a sheet of paper}\hspace{5pt}\pcmn{一张纸}\hspace{5pt}\pfra{une feuille de papier}\end{exemple}
\end{entrée}

\begin{entrée}
{ʂwæ˧}{}{ⓔʂwæ˧}\formedesurface{ʂwæ˧}\newline
\classe{形容词}\ton{M}\begin{définition}\peng{Tall.}\end{définition}
\begin{définition}\pcmn{高}\end{définition}
\begin{définition}\pfra{Haut, de haute taille, grand.}\end{définition}
\begin{exemple}\pnru{qʰɑ˧-ʂwæ˧-gv̩˧}\hspace{5pt}\peng{very tall}\hspace{5pt}\pcmn{非常高}\hspace{5pt}\pfra{très grand}\end{exemple}
\begin{exemple}\pnru{ʈʂʰɯ˧ | ə˧pɤ˧ | -ʂwæ˩-gv̩˩˥!}\hspace{5pt}\peng{(S)he is extremely tall!}\hspace{5pt}\pcmn{他非常高!}\hspace{5pt}\pfra{Elle/il est très grand(e)!}\end{exemple}
\begin{exemple}\pnru{gv̩˧mi˧ ʂwæ˧}\hspace{5pt}\peng{tall; literally ‘with a tall body'}\hspace{5pt}\pcmn{高、身材高}\hspace{5pt}\pfra{de grande taille; littéralement ‘(qui a un) grand corps'}\end{exemple}
\end{entrée}

\begin{entrée}
{ʂwæ˧}{}{ⓔʂwæ˧}\formedesurface{ʂwæ˧}\newline
\classe{名词}\ton{M}
\paradigme{\pcmn{:} \p{}}
\begin{définition}\peng{Otter.}\end{définition}
\begin{définition}\pcmn{水獭}\end{définition}
\begin{définition}\pfra{Loutre.}\end{définition}
\begin{exemple}\pnru{ʂwæ˧-ɣɯ˩}\hspace{5pt}\peng{otter skin}\hspace{5pt}\pcmn{水獭皮}\hspace{5pt}\pfra{peau de loutre}\end{exemple}
\end{entrée}

\begin{entrée}
{ʂwæ˧˥}{}{ⓔʂwæ˧˥}\formedesurface{ʂwæ˧˥}\newline
\classe{动词}\ton{MH}\begin{définition}\peng{To defecate.}\end{définition}
\begin{définition}\pcmn{拉(屎)}\end{définition}
\begin{définition}\pfra{Déféquer, faire caca.}\end{définition}
\begin{exemple}\pnru{qʰæ˧ ʂwæ˩}\hspace{5pt}\peng{to defecate}\hspace{5pt}\pcmn{拉屎}\hspace{5pt}\pfra{déféquer}\end{exemple}
\end{entrée}

\begin{entrée}
{ʂwæ˧α}{}{ⓔʂwæ˧α}\formedesurface{ʂwæ˧}\newline
\classe{动词}\ton{Mα}\begin{définition}\peng{To stir.}\end{définition}
\begin{définition}\pcmn{搅拌}\end{définition}
\begin{définition}\pfra{Remuer (monosyllabe).}\end{définition}
\begin{exemple}\pnru{le˧-ʂwæ˧}\hspace{5pt}\peng{|fg{accomp}}\hspace{5pt}\pcmn{|fg{accomp}}\hspace{5pt}\pfra{|fg{accomp}}\end{exemple}
\begin{exemple}\pnru{mɤ˧-ʂwæ˧}\hspace{5pt}\peng{|fg{neg}}\hspace{5pt}\pcmn{不搅拌}\hspace{5pt}\pfra{|fg{neg}}\end{exemple}
\begin{exemple}\pnru{tso˧∼tso˧ ʂwæ˩}\hspace{5pt}\peng{to stir things}\hspace{5pt}\pcmn{搅拌东西}\hspace{5pt}\pfra{remuer des choses, touiller des choses}\end{exemple}
\end{entrée}

\begin{entrée}
{ʂwæ˩˥}{}{ⓔʂwæ˩˥}\formedesurface{ʂwæ˩˥}\newline
\classe{名词}\ton{LH}
\paradigme{\pcmn{:} \p{}}
\begin{définition}\peng{Wedge.}\end{définition}
\begin{définition}\pcmn{楔子}\end{définition}
\begin{définition}\pfra{Coin.}\end{définition}
\begin{exemple}\pnru{ʂwæ˩ lɑ˧˥ / ʂwæ˩ lɑ˧-ze˥}\hspace{5pt}\peng{to strike a wedge}\hspace{5pt}\pcmn{打一个楔子}\hspace{5pt}\pfra{frapper un coin}\end{exemple}
\begin{exemple}\pnru{ʂwæ˩ hwæ˥-ze˩}\hspace{5pt}\peng{…bought a wedge}\hspace{5pt}\pcmn{买了楔子}\hspace{5pt}\pfra{…a acheté un coin}\end{exemple}
\begin{exemple}\pnru{ʂwæ˩ tʰv̩˩-ɭɯ˩˥ / ʂwæ˩ tʰv̩˩-ɭɯ˥}\hspace{5pt}\peng{\_ |fg{dem}+|fg{clf}}\hspace{5pt}\pcmn{这个楔子}\hspace{5pt}\pfra{\_ |fg{dem}+|fg{clf.générique}}\end{exemple}
\begin{exemple}\pnru{ʂwæ˩ tʰv̩˩-kʰwɤ˩˥}\hspace{5pt}\peng{\_ |fg{dem}+|fg{clf}}\hspace{5pt}\pcmn{这个楔子}\hspace{5pt}\pfra{|fg{n}+|fg{dem}+|fg{clf.morceaux}}\end{exemple}
\begin{exemple}\pnru{ʂwæ˩ kʰɯ˥}\hspace{5pt}\peng{to place a wedge, to put a wedge}\hspace{5pt}\pcmn{放一个楔子}\hspace{5pt}\pfra{mettre un coin}\end{exemple}
\end{entrée}

\begin{entrée}
{ʂwæ˩α}{}{ⓔʂwæ˩α}\formedesurface{ʂwæ˩˥}\newline
\classe{动词}\ton{Lα}\begin{définition}\peng{To cure (meat etc) with smoke.}\end{définition}
\begin{définition}\pcmn{熏}\end{définition}
\begin{définition}\pfra{Fumer (aliment).}\end{définition}
\begin{exemple}\pnru{ʂe˧ ʂwæ˥}\hspace{5pt}\peng{to cure meat with smoke}\hspace{5pt}\pcmn{熏肉}\hspace{5pt}\pfra{fumer de la viande}\end{exemple}
\end{entrée}

\begin{entrée}
{ʂwæ˧bæ˩}{}{ⓔʂwæ˧bæ˩}\formedesurface{ʂwæ˧bæ˩}\newline
\classe{名词}\ton{L\#}\begin{définition}\peng{Camellia flower.}\end{définition}
\begin{définition}\pcmn{映山红}\end{définition}
\begin{définition}\pfra{Camélia.}\end{définition}
\begin{exemple}\pnru{ʂwæ˧bæ˩ bæ˩ |}\hspace{5pt}\peng{The camellia flowers are in bloom.}\hspace{5pt}\pcmn{山茶花开了。}\hspace{5pt}\pfra{Les camélias sont en fleurs.}\end{exemple}
\begin{exemple}\pnru{so˧-ɬi˧mi˧, | ʂwæ˧bæ˩ bæ˩! |}\hspace{5pt}\peng{Camellia flowers bloom in the third month!}\hspace{5pt}\pcmn{山茶花是在三月份开花的!}\hspace{5pt}\pfra{Les camélias fleurissent au troisième mois!}\end{exemple}
\begin{exemple}\pnru{ʂwæ˧bæ˩-si˩}\hspace{5pt}\peng{camellia tree}\hspace{5pt}\pcmn{山茶树}\hspace{5pt}\pfra{camélia (arbre); littéralement «arbre des fleurs de camélia»}\end{exemple}
\end{entrée}

\begin{entrée}
{ʂwæ˧gv̩\#˥}{}{ⓔʂwæ˧gv̩\#˥}\formedesurface{ʂwæ˧gv̩˧}\newline
\classe{名词}\ton{\#H}\begin{définition}\peng{A mountain to the North-West of Yongning, called “Jiaze Mountain" in Chinese.}\end{définition}
\begin{définition}\pcmn{加泽大山(位于永宁西北的一座山)}\end{définition}
\begin{définition}\pfra{Une montagne au nord-ouest de Yongning.}\end{définition}
\begin{exemple}\pnru{kɤ˧mv̩˧˥, | æ˧ʂæ˧, | ŋwɤ˧hɑ̃˩, | ʂwæ˧gv̩\#˥, | nɑ˩tsʰi˩˥ | -tɕʰɤ˧pɤ˧mi\#˥, | qv̩˧ɻ̍˧-ʈʂʰɑ˧nɑ˥ |}\hspace{5pt}\peng{The six mountains of Yongning that carry a name and have a definite symbolic value. The other mountains do not have comparable symbolic value, and fewer people use specific names for them.}\hspace{5pt}\pcmn{永宁地区有固定名字的六座山:格姆,安山,瓦哈,双古,纳慈巧吧咪,古尔川纳。}\hspace{5pt}\pfra{Les six montagnes de Yongning qui portent un nom. Les autres sommets du voisinage n'ont pas une valeur symbolique comparable, et ne portent pas de nom communément utilisé.}\end{exemple}
\end{entrée}

\begin{entrée}
{ʂwæ˩gv̩˩}{}{ⓔʂwæ˩gv̩˩}\formedesurface{ʂwæ˩gv̩˩˥}\newline
\classe{名词}\ton{L}
\paradigme{\pcmn{:} \p{}}
\begin{définition}\peng{Kitchen cabinet, kitchen dresser.}\end{définition}
\begin{définition}\pcmn{柜子}\end{définition}
\begin{définition}\pfra{Buffet (où est rangée la vaisselle).}\end{définition}
\end{entrée}

\begin{entrée}
{ʂwæ˧pɤ˧tʰe˧}{}{ⓔʂwæ˧pɤ˧tʰe˧}\formedesurface{ʂwæ˧pɤ˧tʰe˧}\newline
\classe{名词}\ton{-L}\begin{définition}\peng{Twins.}\end{définition}
\begin{définition}\pcmn{双胞胎 (汉语借词)}\end{définition}
\begin{définition}\pfra{Jumeaux.}\end{définition}
\end{entrée}

\begin{entrée}
{ʂwæ˧ɻæ˧}{}{ⓔʂwæ˧ɻæ˧}\formedesurface{ʂwæ˧ɻæ˧}\newline
\classe{形容词}\ton{M}\begin{définition}\peng{Lame in the legs.}\end{définition}
\begin{définition}\pcmn{瘸腿}\end{définition}
\begin{définition}\pfra{Être paralysé des jambes.}\end{définition}
\begin{exemple}\pnru{ʈʂʰɯ˧ | ʂwæ˧ɻæ˧-hĩ˧ ɖɯ˧-v̩˧ ɲi˩!}\hspace{5pt}\peng{He is lame in the legs!}\hspace{5pt}\pcmn{他腿瘸了!}\hspace{5pt}\pfra{Il est paralysé des jambes! (=il ne peut que se traîner sur le sol)}\end{exemple}
\end{entrée}

\begin{entrée}
{ʂwæ˧si\#˥}{}{ⓔʂwæ˧si\#˥}\formedesurface{ʂwæ˧si˧}\newline
\classe{名词}\ton{\#H}\begin{définition}\peng{Camellia tree.}\end{définition}
\begin{définition}\pcmn{山茶树}\end{définition}
\begin{définition}\pfra{Camélia (arbre).}\end{définition}
\end{entrée}

\begin{entrée}
{ʂwæ˧tsɯ˥}{}{ⓔʂwæ˧tsɯ˥}\formedesurface{ʂwæ˧tsɯ˥}\newline
\classe{名词}\ton{H\#}
\paradigme{\pcmn{:} \p{}}
\begin{définition}\peng{Brush.}\end{définition}
\begin{définition}\pcmn{刷子}\end{définition}
\begin{définition}\pfra{Brosse.}\end{définition}
\end{entrée}

\begin{entrée}
{ʂwɤ˧ljɤ˧-kwɤ˩}{}{ⓔʂwɤ˧ljɤ˧-kwɤ˩}\formedesurface{ʂwɤ˧ljɤ˧kwɤ˩}\newline
\classe{名词}\ton{-L}
\paradigme{\pcmn{:} \p{}}
\begin{définition}\peng{Small bell that was attached to horses' breastplate.}\end{définition}
\begin{définition}\pcmn{挂在马胸前的铃铛}\end{définition}
\begin{définition}\pfra{Clochette qui se fixait sur le poitrail des chevaux.}\end{définition}
\end{entrée}

\newpage\caractère{t}

\begin{entrée}
{tɑ˥}{}{ⓔtɑ˥}\formedesurface{tɑ˧}\newline
\classe{形容词}\ton{H}\begin{définition}\peng{Reliable, trustworthy.}\end{définition}
\begin{définition}\pcmn{可靠}\end{définition}
\begin{définition}\pfra{Fiable.}\end{définition}
\begin{exemple}\pnru{le˧-tɑ˥ (| ʐwæ˩˥)}\hspace{5pt}\peng{very reliable}\hspace{5pt}\pcmn{很靠谱}\hspace{5pt}\pfra{très fiable}\end{exemple}
\begin{exemple}\pnru{le˧ mɤ˧-tɑ˥ (| ʐwæ˩˥)}\hspace{5pt}\peng{not reliable at all}\hspace{5pt}\pcmn{不靠谱}\hspace{5pt}\pfra{pas fiable}\end{exemple}
\begin{exemple}\pnru{no˧ | le˧-mɤ˧-tɑ˥-hĩ˩ ɖɯ˧-v̩˧ ɲi˩!}\hspace{5pt}\peng{You are an irresponsible person! / You are not a reliable person!}\hspace{5pt}\pcmn{你是不靠谱的人!}\hspace{5pt}\pfra{Tu es quelqu'un de pas fiable/pas responsable!}\end{exemple}
\end{entrée}

\begin{entrée}
{tɑ˧˥}{}{ⓔtɑ˧˥}\formedesurface{tɑ˧˥}\newline
\classe{动词}\ton{MH}\begin{définition}\peng{To give way, to fall behind.}\end{définition}
\begin{définition}\pcmn{退后}\end{définition}
\begin{définition}\pfra{Reculer, se retirer.}\end{définition}
\begin{exemple}\pnru{ʁo˧tʰo˩ tɑ˩}\hspace{5pt}\peng{to give way, to fall behind}\hspace{5pt}\pcmn{往后退}\hspace{5pt}\pfra{reculer, se retirer vers l'arrière}\end{exemple}
\end{entrée}

\begin{entrée}
{tɑ˧˥α}{}{ⓔtɑ˧˥α}\formedesurface{ɖɯ˧ tɑ˧˥}\newline
\classe{量词}\ton{MHα}\begin{définition}\peng{Entirely, all; everyone.}\end{définition}
\begin{définition}\pcmn{量词:全部、一切,大家}\end{définition}
\begin{définition}\pfra{Entièrement, tout, tout le monde.}\end{définition}
\begin{exemple}\pnru{ɖɯ˧-tɑ˧˥}\hspace{5pt}\peng{entirely, all; everyone}\hspace{5pt}\pcmn{全部、一切,大家}\hspace{5pt}\pfra{entièrement, tout, tout le monde}\end{exemple}
\begin{exemple}\pnru{ɖɯ˧-tɑ˧=ɻæ˥}\hspace{5pt}\peng{entirely, all; everyone (same meaning as above, with a plural morpheme)}\hspace{5pt}\pcmn{全部、一切,大家(同上,加上多数词素)}\hspace{5pt}\pfra{entièrement, tout, tout le monde (même sens que ci-dessus, avec le morphème de pluriel)}\end{exemple}
\end{entrée}

\begin{entrée}
{tɑ˧α}{}{ⓔtɑ˧α}\formedesurface{tɑ˧}\newline
\classe{动词}\ton{Mα}\begin{définition}\peng{To dry beside or over a fire.}\end{définition}
\begin{définition}\pcmn{烘干}\end{définition}
\begin{définition}\pfra{Chauffer au feu.}\end{définition}
\begin{exemple}\pnru{kwɤ˧-kʰɯ˧ tʰi˧-tɑ˧}\hspace{5pt}\peng{to warm up (food…) beside a fire}\hspace{5pt}\pcmn{放在火炉旁边热一下(饭)}\hspace{5pt}\pfra{réchauffer (la nourriture…) auprès du feu}\end{exemple}
\end{entrée}

\begin{entrée}
{tɑ˩α}{}{ⓔtɑ˩α}\formedesurface{ɖɯ˧ tɑ˩}\newline
\classe{量词}\ton{Lα}\begin{définition}\peng{Classifier for sums of money.}\end{définition}
\begin{définition}\pcmn{量词:钱(一笔)}\end{définition}
\begin{définition}\pfra{Classificateur des fortes sommes d'argent.}\end{définition}
\begin{exemple}\pnru{ɖʐe˧ | ɖɯ˧-tɑ˩}\hspace{5pt}\peng{a (big) sum of money}\hspace{5pt}\pcmn{一笔钱}\hspace{5pt}\pfra{un paquet d'argent, une liasse de billets…}\end{exemple}
\end{entrée}

\begin{entrée}
{tɑ˩dv̩˧˥}{}{ⓔtɑ˩dv̩˧˥}\formedesurface{tɑ˩dv̩˧˥}\newline
\classe{名词}\ton{LM+MH\#}
\paradigme{\pcmn{:} \p{}}
\begin{définition}\peng{Pocket.}\end{définition}
\begin{définition}\pcmn{口袋、衣袋、兜子}\end{définition}
\begin{définition}\pfra{Poche.}\end{définition}
\begin{exemple}\pnru{njɤ˧ | tɑ˩dv̩˧-qo˥ | tsʰe˩mæ˩-tɑ˥kɤ˩-lɑ˩ dʑo˩!}\hspace{5pt}\peng{I only have ten yuan in my pocket!}\hspace{5pt}\pcmn{我兜子里只有十元钱!}\hspace{5pt}\pfra{Je n’ai que dix yuan en poche!}\end{exemple}
\end{entrée}

\begin{entrée}
{tɑ˧dzi˩}{}{ⓔtɑ˧dzi˩}\formedesurface{tɑ˧dzi˩}\newline
\classe{名词}\ton{L\#}\begin{définition}\peng{A Na village down below Nisei, upward from Lataddi.}\end{définition}
\begin{définition}\pcmn{村落名}\end{définition}
\begin{définition}\pfra{Village na en contrebas de Nisei, en contrehaut de Lataddi (\stylefv{/lɑ}˧tʰɑ˧-di˧˥/).}\end{définition}
\begin{exemple}\pnru{ɬi˧ki˧, | ɲi˧se˩, | tɑ˧dzi˩, | mv̩˧qʰwæ˩, | lɑ˧tʰɑ˧-di˧˥}\hspace{5pt}\peng{Villages that one passes when moving away from the Yongning plain, towards Lake Lugu. These villages do not count as part of Yongning proper. The last, /lɑ˧tʰɑ˧-di˧˥/, is not a village name like the preceding four: it refers to the entire Na area beyond the fourth village.}\hspace{5pt}\pcmn{永宁到泸沽湖所经过的村落,依次是:里格、尼赛、大祖、木垮,然后到拉塔地(拉塔地指的是泸沽湖周边的摩梭地区,包括左所、洛水村等)}\hspace{5pt}\pfra{Villages dans l'ordre, après la plaine de Yongning, ne comptant pas comme faisant partie de Yongning. Le dernier, /lɑ˧tʰɑ˧-di˧˥/, désigne toute la région na au-delà du quatrième village.}\end{exemple}
\end{entrée}

\begin{entrée}
{tɑ˩dʑɤ\#˥}{}{ⓔtɑ˩dʑɤ\#˥}\formedesurface{tɑ˩dʑɤ˥}\newline
\classe{名词}\ton{LM+\#H}\begin{définition}\peng{Masculine given name given to the second among twins.}\end{définition}
\begin{définition}\pcmn{男性名字,双胞胎中老二的名字}\end{définition}
\begin{définition}\pfra{Prénom masculin employé pour le second des jumeaux.}\end{définition}
\end{entrée}

\begin{entrée}
{tɑ˩ɖʐo˧dzi˧˥}{}{ⓔtɑ˩ɖʐo˧dzi˧˥}\formedesurface{tɑ˩ɖʐo˧dzi˧˥}\newline
\classe{名词}\ton{LM+MH\#}
\paradigme{\pcmn{:} \p{}}
\begin{définition}\peng{Small prayer flag.}\end{définition}
\begin{définition}\pcmn{小经幡}\end{définition}
\begin{définition}\pfra{Petit drapeau de prière.}\end{définition}
\end{entrée}

\begin{entrée}
{tɑ˩˥fv˩˥}{}{ⓔtɑ˩˥fv˩˥}\formedesurface{tɑ˩˥fv˩˥}\newline
\classe{名词}\ton{LH.LH}\begin{définition}\peng{Excrement.}\end{définition}
\begin{définition}\pcmn{大粪(汉语借词)}\end{définition}
\begin{définition}\pfra{Excrément.}\end{définition}
\end{entrée}

\begin{entrée}
{tɑ˧gɤ˩}{}{ⓔtɑ˧gɤ˩}\formedesurface{tɑ˧gɤ˩}\newline
\classe{形容词}\ton{L\#}\begin{définition}\peng{Gaunt, emaciated.}\end{définition}
\begin{définition}\pcmn{瘦弱、枯瘦}\end{définition}
\begin{définition}\pfra{Maigre (personne maigre).}\end{définition}
\end{entrée}

\begin{entrée}
{tɑ˧ho˧}{}{ⓔtɑ˧ho˧}\formedesurface{tɑ˧ho˧}\newline
\classe{助词}\ton{M}\begin{définition}\peng{Together.}\end{définition}
\begin{définition}\pcmn{一起}\end{définition}
\begin{définition}\pfra{Ensemble.}\end{définition}
\begin{exemple}\pnru{ɖɯ˧-ʁwɤ˧ tɑ˧ho˧ kʰi˧˥}\hspace{5pt}\peng{The whole village went together.}\hspace{5pt}\pcmn{全村一起去了。}\hspace{5pt}\pfra{Tout le village y est allé ensemble.}\end{exemple}
\begin{exemple}\pnru{tɑ˧ho˧ ʝi˧}\hspace{5pt}\peng{to work together}\hspace{5pt}\pcmn{一起工作}\hspace{5pt}\pfra{travailler ensemble}\end{exemple}
\begin{exemple}\pnru{tɑ˧ho˧ tsʰo˧}\hspace{5pt}\peng{to dance together}\hspace{5pt}\pcmn{一起跳舞}\hspace{5pt}\pfra{danser ensemble}\end{exemple}
\end{entrée}

\begin{entrée}
{tɑ˩hwɤ˩}{}{ⓔtɑ˩hwɤ˩}\formedesurface{tɑ˩hwɤ˩˥}\newline
\classe{动词}
\sens{1}
\begin{définition}\peng{To offer gifts outside the family circle.}\end{définition}
\begin{définition}\pcmn{送礼(给家里以外的人)}\end{définition}
\begin{définition}\pfra{Offrir des présents à des gens extérieurs à la famille.}\end{définition}
\begin{exemple}\pnru{hĩ˧-ki˧ | ɖɯ˧-kʰwɤ˧ tɑ˥hwɤ˩-zo˩-ʝi˩!}\hspace{5pt}\peng{We shall have to make a present to people! / It's going to be an occasion to make a present to people! (For instance, when a child goes through the “Coming of age" rite.)}\hspace{5pt}\pcmn{应该给人家送礼了!(例如,人家为孩子进行成年礼时,要送礼。)}\hspace{5pt}\pfra{Il va falloir faire un présent aux gens! / Ca va être l'occasion de faire un présent aux gens! (par exemple à l'occasion d'un rituel de passage à l'âge adulte)}\end{exemple}
\begin{exemple}\pnru{zo˧mv̩˥-ki˩, | tɑ˩hwɤ˩ mɤ˥-kv̩˩!}\hspace{5pt}\peng{Presents are not for the kids! / We don't give big presents to children!}\hspace{5pt}\pcmn{不会专门给孩子送(大)礼的!(说明:送礼,是送给家里的主人)}\hspace{5pt}\pfra{On ne fait pas de présents aux enfants! (Explication: le présent donné par une famille à une autre de façon ritualisée est offert aux aînés, pas aux enfants; c'est d'une nature différente des petits cadeaux qu'on peut leur faire au quotidien.)}\end{exemple}
\begin{exemple}\pnru{ʐɯ˧ tɑ˩hwɤ˩}\hspace{5pt}\peng{to offer wine as a present}\hspace{5pt}\pcmn{送酒(作为礼物)}\hspace{5pt}\pfra{offrir du vin comme présent}\end{exemple}
\begin{exemple}\pnru{li˩ tɑ˥hwɤ˩}\hspace{5pt}\peng{to offer tea as a present}\hspace{5pt}\pcmn{送茶(作为礼物)}\hspace{5pt}\pfra{offrir du thé comme présent}\end{exemple}
\begin{exemple}\pnru{dze˧ tɑ˥hwɤ˩}\hspace{5pt}\peng{to offer sweets as a present}\hspace{5pt}\pcmn{送糖(作为礼物)}\hspace{5pt}\pfra{offrir des sucreries/des bonbons comme présent}\end{exemple}\sens{2}
\begin{définition}\peng{To give a dowry: to give goods to a young woman when she goes to her new home after her wedding. The dowry used to be brought on horseback, in two wood boxes: gifts must come in pairs, and the dowry is no exception.}\end{définition}
\begin{définition}\pcmn{送陪嫁(嫁妆、陪奁)}\end{définition}
\begin{définition}\pfra{Fournir la dot: donner des biens à une jeune femme lorsqu'elle rejoint sa nouvelle famille lors du mariage. (Note: la dot est apportée à dos de cheval; elle est rangée dans deux caisses en bois: comme en d'autres circonstances, les présents doivent aller par deux.).}\end{définition}
\begin{exemple}\pnru{ə˧tso˧ tɑ˩hwɤ˩-ʝi˩? | ə˧-sɯ˩kv̩˩ (-dʑo˩), | ɖɯ˧-li˧-ɻ̍˩-bi˩!}\hspace{5pt}\peng{What did they give as a dowry? Let's go and have a look! (At a wedding, the gifts given as a dowry are put on public display, for everyone to appreciate the parents' generosity.)}\hspace{5pt}\pcmn{给的是什么嫁妆?咱们去看一看吧!(结婚的时候,陪嫁展示在大家眼前,显示女方家的大方程度)}\hspace{5pt}\pfra{Qu'est-ce qu'ils ont donné comme dot? Regardons un peu! / Allons voir! (Ce que disent les villageois invités à un mariage; les biens offerts en dot sont alors exposés, de façon à ce que chacun puisse apprécier la générosité des parents.)}\end{exemple}
\begin{exemple}\pnru{ti˧tsɯ˥ | qʰɑ˧-ɭɯ˧ tɑ˩hwɤ˩? - ti˧tsɯ˥ | ɲi˧-ɭɯ˧ tɑ˩hwɤ˩!}\hspace{5pt}\peng{How many boxes are there in the dowry? - The dowry consists of two boxes!}\hspace{5pt}\pcmn{陪嫁有几个木箱? - 陪嫁有两个(木箱)!}\hspace{5pt}\pfra{Combien de caisses sont offertes en dot/ de combien de caisses la dot se compose-t-elle? - De deux caisses!}\end{exemple}
\end{entrée}

\begin{entrée}
{‑tɑ.kɤ}{}{ⓔ‑tɑ.kɤ}\formedesurface{--}\newline
\classe{后缀}\ton{?}\begin{définition}\peng{Alone, only.}\end{définition}
\begin{définition}\pcmn{只、才}\end{définition}
\begin{définition}\pfra{Seulement (dénombrable), seul.}\end{définition}
\begin{exemple}\pnru{õ˧-tɑ˧kɤ˥, | ʐwɤ˩ mɤ˩-ʁo˩˥!}\hspace{5pt}\peng{On one's own, it's really hard to talk! (Reflection about a recording: the speaker was struggling to put together a coherent narrative with the linguist as sole listener.)}\hspace{5pt}\pcmn{自己一个人,说不出话来!/ 一个人说话,很难讲下去!(描述一个发音合作人在录音时的困难:如果只有半懂不懂的调查者在听,很难流利地讲话。)}\hspace{5pt}\pfra{«Quand on est tout seul, on n'arrive pas à parler!» (Réflexion à l'écoute d'un enregistrement: le locuteur avait bien de la peine à déployer un récit avec pour seul public le linguiste-enquêteur.)}\end{exemple}
\begin{exemple}\pnru{njɤ˩-tɑ˩kɤ˥}\hspace{5pt}\peng{me alone / me, on my own}\hspace{5pt}\pcmn{我一个人}\hspace{5pt}\pfra{moi seul}\end{exemple}
\begin{exemple}\pnru{no˩-tɑ˩kɤ˥}\hspace{5pt}\peng{you alone / you, on your own}\hspace{5pt}\pcmn{你一个人}\hspace{5pt}\pfra{toi seul}\end{exemple}
\begin{exemple}\pnru{ʈʂʰɯ˧-tɑ˧kɤ˥}\hspace{5pt}\peng{he alone / he, on his own}\hspace{5pt}\pcmn{他一个人}\hspace{5pt}\pfra{lui seul}\end{exemple}
\begin{exemple}\pnru{ʈʂʰɯ˧ ə˧mi˧ | ə˧ɲi˧-tɑ˧kɤ˥ | le˧-ɬi˥!}\hspace{5pt}\peng{Her mother only got a day off yesterday (=today, she is back to work)! (Context: talking about a child whose mother is absent, as she is at work.)}\hspace{5pt}\pcmn{她母亲昨天那天才有假期!(情景:谈一个孩子的母亲,说她前一天才能休息,而当天在上班。)}\hspace{5pt}\pfra{sa mère n'a eu de congés que pour la journée d'hier! (contexte: on parle d'un petit enfant dont la mère est absente, au travail)}\end{exemple}
\begin{exemple}\pnru{ɑ˩ʁo˧, | hĩ˧ ɖɯ˧-v̩˧-tɑ˧kɤ˩ dʑo˩!}\hspace{5pt}\peng{There is only one person at home!}\hspace{5pt}\pcmn{只有一个人在家!}\hspace{5pt}\pfra{Il n'y a qu'une seule personne à la maison!}\end{exemple}
\end{entrée}

\begin{entrée}
{tɑ˩kɤ˧}{}{ⓔtɑ˩kɤ˧}\formedesurface{tɑ˩kɤ˥}\newline
\classe{动词}\begin{définition}\peng{To tease (by gestures).}\end{définition}
\begin{définition}\pcmn{逗弄(动作)}\end{définition}
\begin{définition}\pfra{Taquiner.}\end{définition}
\end{entrée}

\begin{entrée}
{tɑ˧ko˧}{}{ⓔtɑ˧ko˧}\formedesurface{tɑ˧ko˧}\newline
\classe{动词}\ton{M}\begin{définition}\peng{To do manual work, to get a job, to do odd jobs to make some money.}\end{définition}
\begin{définition}\pcmn{打工(汉语借词)}\end{définition}
\begin{définition}\pfra{Trouver du travail non qualifié (sur un chantier…), gagner de l'argent en faisant des petits boulots.}\end{définition}
\begin{exemple}\pnru{tɑ˧ko˧ hɯ˧-ze˩!}\hspace{5pt}\peng{(S)he has gone (to the city, to another place…) to do odd jobs to make some money!}\hspace{5pt}\pcmn{(他)打工去了!}\hspace{5pt}\pfra{(Elle/il) est parti(e) gagner de l'argent en faisant des petits boulots!}\end{exemple}
\end{entrée}

\begin{entrée}
{tɑ˧ko˩}{}{ⓔtɑ˧ko˩}\formedesurface{tɑ˧ko˩}\newline
\classe{动词}\ton{L\#}\begin{définition}\peng{To delay, to hold up.}\end{définition}
\begin{définition}\pcmn{耽误}\end{définition}
\begin{définition}\pfra{Retarder.}\end{définition}
\begin{exemple}\pnru{hĩ˧ tɑ˧ko˥}\hspace{5pt}\peng{to delay people}\hspace{5pt}\pcmn{耽误人家}\hspace{5pt}\pfra{retarder les gens}\end{exemple}
\begin{exemple}\pnru{ʈʂʰɯ˧ hĩ˧ tɑ˧ko˥ | ʐwæ˩˥!}\hspace{5pt}\peng{(S)he delays people a lot!}\hspace{5pt}\pcmn{他耽误大家很多!}\hspace{5pt}\pfra{Il/elle retarde tout le monde!}\end{exemple}
\end{entrée}

\begin{entrée}
{tɑ˩li˥}{}{ⓔtɑ˩li˥}\formedesurface{tɑ˩li˥}\newline
\classe{名词}\ton{LH}\begin{définition}\peng{Dali (city name).}\end{définition}
\begin{définition}\pcmn{大理(汉语借词)}\end{définition}
\begin{définition}\pfra{Dali (nom de ville).}\end{définition}
\end{entrée}

\begin{entrée}
{tɑ˥mo˩}{}{ⓔtɑ˥mo˩}\formedesurface{tɑ˧mo˩}\newline
\classe{动词}\ton{H.L}\begin{définition}\peng{To wilt, to wither (flower…).}\end{définition}
\begin{définition}\pcmn{萎、萎蔫}\end{définition}
\begin{définition}\pfra{Faner.}\end{définition}
\begin{exemple}\pnru{le˧-tɑ˥mo˩-ze˩!}\hspace{5pt}\peng{It has wilted!}\hspace{5pt}\pcmn{萎蔫了!}\hspace{5pt}\pfra{Ca a fané! (Exemple: une fleur coupée, une fleur abîmée par le vent ou par un soleil trop ardent.)}\end{exemple}
\end{entrée}

\begin{entrée}
{tɑ˩mv̩˩}{}{ⓔtɑ˩mv̩˩}\formedesurface{tɑ˩mv̩˩˥}\newline
\classe{动词}\ton{L}\begin{définition}\peng{Proverb.}\end{définition}
\begin{définition}\pcmn{谚语}\end{définition}
\begin{définition}\pfra{Proverbe.}\end{définition}
\begin{exemple}\pnru{æ˧ʂæ˧-tɑ˩mv̩˩}\hspace{5pt}\peng{same meaning: proverb (literally ‘proverb of yore')}\hspace{5pt}\pcmn{同上:谚语(直译:‘从前的老话’)}\hspace{5pt}\pfra{même sens: proverbe (littéralement: ‘proverbe ancien')}\end{exemple}
\begin{exemple}\pnru{æ˧ʂæ˧-tɑ˥mv̩˩}\hspace{5pt}\peng{proverb; traditional story}\hspace{5pt}\pcmn{谚语、传统故事}\hspace{5pt}\pfra{proverbe; histoire ancienne}\end{exemple}
\end{entrée}

\begin{entrée}
{tɑ˧nɑ˩}{}{ⓔtɑ˧nɑ˩}\formedesurface{tɑ˧nɑ˩}\newline
\classe{名词}\ton{L\#}
\paradigme{\pcmn{:} \p{}}
\begin{définition}\peng{Crossbow.}\end{définition}
\begin{définition}\pcmn{弩弓}\end{définition}
\begin{définition}\pfra{Arbalète.}\end{définition}
\end{entrée}

\begin{entrée}
{tɑ˧pi˧}{₁}{ⓔtɑ˧pi˧ⓗ1}\formedesurface{tɑ˧pi˧}\newline
\classe{形容词}\ton{M}
1\begin{définition}\peng{Identical to, like, to the likeness of.}\end{définition}
\begin{définition}\pcmn{如、像、像……那样}\end{définition}
\begin{définition}\pfra{Identique à, pareil à, semblable à, à l'exemple de.}\end{définition}
\begin{exemple}\pnru{no˧-bi˧ tɑ˩pi˩, …}\hspace{5pt}\peng{like you; following your example}\hspace{5pt}\pcmn{像你}\hspace{5pt}\pfra{selon ton exemple; comme toi; identique à toi}\end{exemple}
\begin{exemple}\pnru{njɤ˧-bi˧ tɑ˩pi˩…}\hspace{5pt}\peng{like me; following my example}\hspace{5pt}\pcmn{像我}\hspace{5pt}\pfra{comme moi; à mon exemple}\end{exemple}
\begin{exemple}\pnru{no˧=ɻ̍˩-bv̩˩, | njɤ˧=ɻ̍˩-bv̩˩, | tɑ˧pi˧!}\hspace{5pt}\peng{Yours and ours are built on the same pattern / are identical! (Context: discussing the farms of the village: they are all built on the same model, in the same way, and thus identical.)}\hspace{5pt}\pcmn{你家的房子,我家的房子,都是一样的!(如:一个村子里的房子,都是按同一个模式建设的。)}\hspace{5pt}\pfra{Le mien et le tien, ils sont faits sur le même exemple =ils sont pareils! (au sujet de bâtiments, par exemple: les maisons d'un même village sont bâties sur le même modèle)}\end{exemple}
\begin{exemple}\pnru{no˧-ɳɯ˧ gv̩˩, | njɤ˧-ɳɯ˧-gv̩˩, | tɑ˧pi˧!}\hspace{5pt}\peng{Whether it's you or me who's building [the house], it's the same / the result is the same!}\hspace{5pt}\pcmn{无论是谁来盖房,盖出来的都一样!}\hspace{5pt}\pfra{que ce soit toi ou moi qui construise [une maison], c'est pareil! / c'est sur le même modèle!}\end{exemple}
\begin{exemple}\pnru{ʈʂʰɯ˧-bi˩ | tɑ˧pi˧, | njɤ˧-ɳɯ˧ dɑ˧-bi˥-ze˩!}\hspace{5pt}\peng{I am going to build [a house] like that one! / I am going to build [a house] that will be identical to his!}\hspace{5pt}\pcmn{我要盖跟这一样的房子!}\hspace{5pt}\pfra{je vais construire [une maison] comme celle-là/ je vais imiter cette maison-là! / Je vais construire une maison qui sera pareille à la sienne!}\end{exemple}
\begin{exemple}\pnru{no˧=ɻ˩ njɤ˩=ɻ˩, | tɑ˧pi˧ mɤ˧-tʰɑ˩!}\hspace{5pt}\peng{There's no point comparing oneself to others!}\hspace{5pt}\pcmn{你的,我的,不要相比!/不要跟他人相比!}\hspace{5pt}\pfra{Il ne faut pas se comparer aux autres!}\end{exemple}
\end{entrée}

\begin{entrée}
{tɑ˧pi˧}{₂}{ⓔtɑ˧pi˧ⓗ2}\formedesurface{tɑ˧pi˧}\newline
\classe{动词}\ton{M}
2\begin{définition}\peng{To take as an example, to draw an analog.}\end{définition}
\begin{définition}\pcmn{打比方(汉语借词:当地汉语方言‘打比’)}\end{définition}
\begin{définition}\pfra{Prendre pour exemple.}\end{définition}
\begin{exemple}\pnru{tɑ˧pi˧-ze˩}\hspace{5pt}\peng{|fg{pfv}}\hspace{5pt}\pcmn{打比方}\hspace{5pt}\pfra{|fg{pfv}}\end{exemple}
\end{entrée}

\begin{entrée}
{tɑ˧pi˧}{₃}{ⓔtɑ˧pi˧ⓗ3}\formedesurface{tɑ˧pi˧}\newline
\classe{名词}\ton{M}
3\begin{définition}\peng{Example, analog.}\end{définition}
\begin{définition}\pcmn{比喻(汉语借词:‘打比方’,当地汉语方言‘打比’)}\end{définition}
\begin{définition}\pfra{Exemple, analogie.}\end{définition}
\begin{exemple}\pnru{tɑ˧pi˧ ɲi˩-ze˩ mæ˩! |}\hspace{5pt}\peng{I only say this as an example!}\hspace{5pt}\pcmn{只是比方而已!}\hspace{5pt}\pfra{Je dis juste ça à titre d'exemple! (Contexte: F4 explique que tous les enfants ne profitent pas également des bons conseils qu'on leur donne: 'Toi, tes parents t'ont encouragé à étudier, et tu as réussi! Ton frère, il n'a pas réussi!' Me voyant prêt à corriger (pour dire que mon frère n'a pas moins bien réussi), elle souligne: 'Je dis juste ça à titre d'exemple!')}\end{exemple}
\end{entrée}

\begin{entrée}
{tɑ˧pʰi˩}{}{ⓔtɑ˧pʰi˩}\formedesurface{tɑ˧pʰi˩}\newline
\classe{名词}\ton{L\#}
\paradigme{\pcmn{:} \p{}}
\begin{définition}\peng{Chinese mugwort, |\stylefi{Artemisia argyi}.}\end{définition}
\begin{définition}\pcmn{艾、艾蒿}\end{définition}
\begin{définition}\pfra{Armoise de Chine, |\stylefi{Artemisia argyi}.}\end{définition}
\end{entrée}

\begin{entrée}
{tɑ˩so˩kʰo˥}{}{ⓔtɑ˩so˩kʰo˥}\formedesurface{tɑ˩so˩kʰo˥}\newline
\classe{助词}\ton{L+H\#}\begin{définition}\peng{Crosslegged (bodily posture).}\end{définition}
\begin{définition}\pcmn{盘腿(而坐)}\end{définition}
\begin{définition}\pfra{En tailleur (posture assise).}\end{définition}
\begin{exemple}\pnru{tɑ˩so˩kʰo˥ | tʰi˧-dzi˩}\hspace{5pt}\peng{To sit crosslegged. (This is the usual posture for monks, and also a usual posture for commoners.)}\hspace{5pt}\pcmn{打坐、盘腿而坐(和尚的坐姿)}\hspace{5pt}\pfra{être assis en tailleur (posture assise des moines, et posture également courante chez les gens du commun)}\end{exemple}
\end{entrée}

\begin{entrée}
{tɑ˧∼tɑ˧}{₁}{ⓔtɑ˧∼tɑ˧ⓗ1}\formedesurface{tɑ˧tɑ˧}\newline
\classe{形容词}\ton{M}
1\begin{définition}\peng{Serious, reliable, careful; clear (to see clearly).}\end{définition}
\begin{définition}\pcmn{严肃认真、细心、细致,(看得)清楚、清晰}\end{définition}
\begin{définition}\pfra{Sérieux, attentif, soigneux; exact, précis (une chaussure convient précisément à un pied; quelqu'un observe avec précision/exactitude).}\end{définition}
\begin{exemple}\pnru{mɤ˧-tɑ˧∼tɑ˧}\hspace{5pt}\peng{sloppy, not careful}\hspace{5pt}\pcmn{邋遢、草率、潦草}\hspace{5pt}\pfra{pas sérieux, négligé (au sujet d'un travail)}\end{exemple}
\begin{exemple}\pnru{lo˧ ʝi˧ mɤ˧-tɑ˧∼tɑ˧}\hspace{5pt}\peng{to do sloppy work}\hspace{5pt}\pcmn{工作草率}\hspace{5pt}\pfra{travailler sans soin, de façon négligée}\end{exemple}
\begin{exemple}\pnru{hĩ˧ ʈʂʰɯ˧-v̩˧, | tɑ˧∼tɑ˧!}\hspace{5pt}\peng{(S)he works carefully!}\hspace{5pt}\pcmn{他很认真!}\hspace{5pt}\pfra{lui, il est soigneux!}\end{exemple}
\end{entrée}

\begin{entrée}
{tɑ˧∼tɑ˧}{₂}{ⓔtɑ˧∼tɑ˧ⓗ2}\formedesurface{tɑ˧tɑ˧}\newline
\classe{助词}\ton{M}
2\begin{définition}\peng{Exactly (right), just (right).}\end{définition}
\begin{définition}\pcmn{刚(好)、正(好)}\end{définition}
\begin{définition}\pfra{Précisément, justement, juste à point nommé.}\end{définition}
\begin{exemple}\pnru{tɑ˧∼tɑ˧ | ho˩˥! |}\hspace{5pt}\peng{Just right, exactly right. (Example: a pair of shoes fits perfectly.)}\hspace{5pt}\pcmn{刚刚好!(如:一双鞋刚好合适)}\hspace{5pt}\pfra{Ca convient exactement/ça convient précisément (ex.: au sujet d'une paire de chaussures qu'on vient de vous offrir)}\end{exemple}
\begin{exemple}\pnru{le˧-li˧ tɑ˧∼tɑ˧}\hspace{5pt}\peng{to see clearly}\hspace{5pt}\pcmn{看清楚}\hspace{5pt}\pfra{voir clairement}\end{exemple}
\end{entrée}

\begin{entrée}
{tæ˧pv̩˩}{}{ⓔtæ˧pv̩˩}\formedesurface{tæ˧pv̩˩}\newline
\classe{形容词}\ton{L\#}\begin{définition}\peng{Dry (fruit, vegetables), dried in the sun.}\end{définition}
\begin{définition}\pcmn{晒干的(水果、蔬菜……)}\end{définition}
\begin{définition}\pfra{Séché (au soleil) (ex.: légumes).}\end{définition}
\begin{exemple}\pnru{v̩˩tsʰɤ˧-tæ˧pv̩˥}\hspace{5pt}\peng{dry vegetables, vegetables dried in the sun}\hspace{5pt}\pcmn{晒干的蔬菜}\hspace{5pt}\pfra{légumes séchés au soleil}\end{exemple}
\begin{exemple}\pnru{ʂe˧-tæ˧pv̩˥}\hspace{5pt}\peng{dry meat, meat dried in the sun}\hspace{5pt}\pcmn{晒干的肉}\hspace{5pt}\pfra{viande séchée au soleil}\end{exemple}
\begin{exemple}\pnru{v˩tsʰɤ˧˥ | le˧-ʈæ˥pv˩ kʰɯ˩}\hspace{5pt}\peng{to dry vegetables}\hspace{5pt}\pcmn{将蔬菜弄干(晒干)}\hspace{5pt}\pfra{faire sécher des légumes}\end{exemple}
\begin{exemple}\pnru{v˩tsʰɤ˧˥ | ʈæ˧pv˩ gv˩}\hspace{5pt}\peng{to dry vegetables}\hspace{5pt}\pcmn{将蔬菜弄干(晒干)}\hspace{5pt}\pfra{faire sécher des légumes}\end{exemple}
\end{entrée}

\begin{entrée}
{tæ˧ɻæ˩}{}{ⓔtæ˧ɻæ˩}\formedesurface{tæ˧ɻæ˩}\newline
\classe{名词}\ton{L\#}
\paradigme{\pcmn{:} \p{}}
\begin{définition}\peng{Oesophagus; Adam's apple.}\end{définition}
\begin{définition}\pcmn{喉管、喉结}\end{définition}
\begin{définition}\pfra{Pomme d'Adam; larynx, gorge, oesophage.}\end{définition}
\end{entrée}

\begin{entrée}
{ti˧}{}{ⓔti˧}\formedesurface{ti˧}\newline
\classe{动词}\ton{M}\begin{définition}\peng{To become mature, to become an adult.}\end{définition}
\begin{définition}\pcmn{成熟(人成熟)}\end{définition}
\begin{définition}\pfra{Mûrir, devenir adulte (d'une personne).}\end{définition}
\begin{exemple}\pnru{hĩ˧ tʰv̩˧-v̩˧, | gɤ˩-ti˧-ze˧!}\hspace{5pt}\peng{This person has become an adult! / This person has grown up/has become mature!}\hspace{5pt}\pcmn{这个人,成熟了! / 是大人了!}\hspace{5pt}\pfra{Cette personne a grandi / est devenue adulte / a mûri!}\end{exemple}
\begin{exemple}\pnru{hĩ˧ tʰv̩˧-v̩˧, | mɤ˧-ti˧-sɯ˩!}\hspace{5pt}\peng{This person is not mature yet! / This person is not an adult yet!}\hspace{5pt}\pcmn{这个人,还不成熟!}\hspace{5pt}\pfra{Cette personne n'est pas encore adulte!}\end{exemple}
\begin{exemple}\pnru{zo˩mv̩˧ | gɤ˩-ti˧, | lo˧ hɑ˧!}\hspace{5pt}\peng{For a child to grow/to become an adult is no easy business! (Refers to difficulty both for the child and for the family)}\hspace{5pt}\pcmn{孩子长成熟(的过程),还是挺难的!}\hspace{5pt}\pfra{La croissance d'un enfant (jusqu'à l'âge adulte), c'est pas facile! (La difficulté est pour les parents, et aussi pour l'enfant)}\end{exemple}
\end{entrée}

\begin{entrée}
{ti˧˥}{}{ⓔti˧˥}\formedesurface{ti˧˥}\newline
\classe{动词}\ton{MH}\begin{définition}\peng{To settle, to decide (Chinese borrowing).}\end{définition}
\begin{définition}\pcmn{决定(汉语借词:定)}\end{définition}
\begin{définition}\pfra{Décider, fixer, arrêter.}\end{définition}
\end{entrée}

\begin{entrée}
{ti˧˥α}{}{ⓔti˧˥α}\formedesurface{ɖɯ˧ ti˧˥}\newline
\classe{量词}\ton{MHα}\begin{définition}\peng{Classifier for layers (of dust, of boards…).}\end{définition}
\begin{définition}\pcmn{量词:层(一层灰、一层木板……)}\end{définition}
\begin{définition}\pfra{Couche (de poussière; de planches constituant un plancher; de tissu…).}\end{définition}
\begin{exemple}\pnru{dʑɯ˩-nɑ˩mi˩˥ | gv̩˧-ti˩-qo˩ tʰv̩˩}\hspace{5pt}\peng{To arrive at the heart of the heart of the alpine forest. Literally ‘in the ninth layer of alpine forest'; ‘ninth' here serves as the highest numeral, to refer to an extreme; there is no such thing as a ‘first layer', a ‘second layer' and so on.}\hspace{5pt}\pcmn{到深山老林的最深处。直译:‘到深山老林的第九层’。这里的‘九’作为最高的数字,表示‘极深’的意思:不能说‘深山老林的第一层’、‘第二层’等。}\hspace{5pt}\pfra{se retrouver au plus profond de la forêt: littéralement «dans la neuvième couche de forêt/d'alpage» (=au plus profond; on ne compte pas au-delà de la 9e «couche»; ce décompte est métaphorique, il ne correspond pas à un décompte en étapes à pied, par exemple.}\end{exemple}
\end{entrée}

\begin{entrée}
{ti˩α}{}{ⓔti˩α}\newline
\classe{动词}
\sens{1}
\begin{définition}\peng{To pound, e.g. pounding Szechuan pepper with a small metal pestle, or pounding earth to build a wall of earth.}\end{définition}
\begin{définition}\pcmn{捣(花椒、大蒜……)}\end{définition}
\begin{définition}\pfra{Piler: réduire quelque chose en poudre dans un mortier, par des coups répétés; comprimer de la terre pour former un mur de terre.}\end{définition}
\begin{exemple}\pnru{læ˧tsɯ˥ ti˩}\hspace{5pt}\peng{to pound hot peppers}\hspace{5pt}\pcmn{捣辣椒}\hspace{5pt}\pfra{piler le piment, réduire le piment en poudre}\end{exemple}
\begin{exemple}\pnru{tsʰo˧ko˧ ti˩}\hspace{5pt}\peng{to pound cardamom}\hspace{5pt}\pcmn{捣草果}\hspace{5pt}\pfra{piler la cardamome, réduire de cardamome en poudre}\end{exemple}
\begin{exemple}\pnru{dze˩ ti˥}\hspace{5pt}\peng{to pound Szechuan pepper}\hspace{5pt}\pcmn{捣花椒}\hspace{5pt}\pfra{piler le xanthoxyle, réduire le xanthoxyle en poudre}\end{exemple}
\begin{exemple}\pnru{ʈʂo˩bo˩ ti˥}\hspace{5pt}\peng{to build a wall of earth, by pounding the earth}\hspace{5pt}\pcmn{垒土墙}\hspace{5pt}\pfra{construire un mur en terre en comprimant la terre à coups de masse}\end{exemple}\sens{2}
\begin{définition}\peng{To hit, to strike lightly.}\end{définition}
\begin{définition}\pcmn{拍打}\end{définition}
\begin{définition}\pfra{Donner une tape, tapoter, frapper quelqu'un légèrement.}\end{définition}
\begin{exemple}\pnru{hĩ˧ ti˥}\hspace{5pt}\peng{to slap someone, to hit someone mildly}\hspace{5pt}\pcmn{拍打人}\hspace{5pt}\pfra{donner une tape à quelqu'un}\end{exemple}
\begin{exemple}\pnru{hĩ˧ | ɖɯ˧-v̩˧ ti˩-ze˩}\hspace{5pt}\peng{(She/he) has slapped someone.}\hspace{5pt}\pcmn{(他)拍打了某人。}\hspace{5pt}\pfra{(Elle/il) a donné une tape à quelqu'un.}\end{exemple}
\end{entrée}

\begin{entrée}
{ti˧ɖo˥}{}{ⓔti˧ɖo˥}\formedesurface{ti˧ɖo˥}\newline
\classe{名词}\ton{H\#}\begin{définition}\peng{Masculine given name.}\end{définition}
\begin{définition}\pcmn{男性名字}\end{définition}
\begin{définition}\pfra{Prénom masculin.}\end{définition}
\end{entrée}

\begin{entrée}
{ti˩pʰo˩}{}{ⓔti˩pʰo˩}\formedesurface{ti˩pʰo˩˥}\newline
\classe{名词}\ton{L}
\paradigme{\pcmn{:} \p{}}
\begin{définition}\peng{Ceiling.}\end{définition}
\begin{définition}\pcmn{天花板}\end{définition}
\begin{définition}\pfra{Plafond.}\end{définition}
\end{entrée}

\begin{entrée}
{ti˧pʰv̩\#˥}{}{ⓔti˧pʰv̩\#˥}\formedesurface{ti˧pʰv̩˧}\newline
\classe{名词}\ton{\#H}
\paradigme{\pcmn{:} \p{}}
\begin{définition}\peng{Copper cup for offerings in religious rituals.}\end{définition}
\begin{définition}\pcmn{铜杯盏,做仪式用的}\end{définition}
\begin{définition}\pfra{Coupe de cuivre pour les offrandes; elle est évasée, et de la taille d'un petit gobelet à alcool.}\end{définition}
\end{entrée}

\begin{entrée}
{ti˩tje˧}{}{ⓔti˩tje˧}\formedesurface{ti˩tje˧}\newline
\classe{动词}\ton{LM}\begin{définition}\peng{To treat, to handle (someone).}\end{définition}
\begin{définition}\pcmn{对待(汉语借词)}\end{définition}
\begin{définition}\pfra{Traiter (quelqu'un).}\end{définition}
\end{entrée}

\begin{entrée}
{ti˧tsɯ˥}{}{ⓔti˧tsɯ˥}\formedesurface{ti˧tsɯ˥}\newline
\classe{名词}\ton{H\#}
\paradigme{\pcmn{:} \p{}}
\begin{définition}\peng{Box (woven out of bamboo or wicker).}\end{définition}
\begin{définition}\pcmn{竹箱}\end{définition}
\begin{définition}\pfra{Boîte en vannerie (objet qui n'est plus en usage aujourd'hui).}\end{définition}
\end{entrée}

\begin{entrée}
{ti˧ʈʂʰɯ˩}{}{ⓔti˧ʈʂʰɯ˩}\formedesurface{ti˧ʈʂʰɯ˩}\newline
\classe{名词}\ton{L\#}
\paradigme{\pcmn{:} \p{}}
\begin{définition}\peng{Hammer.}\end{définition}
\begin{définition}\pcmn{铁锤}\end{définition}
\begin{définition}\pfra{Marteau.}\end{définition}
\end{entrée}

\begin{entrée}
{tjɤ˧hwɑ˧˥}{}{ⓔtjɤ˧hwɑ˧˥}\formedesurface{tjɤ˧hwɑ˧˥}\newline
\classe{名词}\ton{MH\#}
\paradigme{\pcmn{:} \p{}}
\begin{définition}\peng{Telephone.}\end{définition}
\begin{définition}\pcmn{电话(汉语借词)}\end{définition}
\begin{définition}\pfra{Téléphone.}\end{définition}
\begin{exemple}\pnru{tjɤ˧hwɑ˧ li˥}\hspace{5pt}\peng{to watch one's telephone (an important activity for smartphone owners in the late 2010s)}\hspace{5pt}\pcmn{看电话(2010年代末拥有手机的人的一个重要名堂)}\hspace{5pt}\pfra{regarder son téléphone (une des principales préoccupations des possesseurs d'ordiphones vers la fin des années 2010)}\end{exemple}
\end{entrée}

\begin{entrée}
{tjɤ˧po˧}{}{ⓔtjɤ˧po˧}\formedesurface{tjɤ˧po˧}\newline
\classe{名词}\ton{M}
\paradigme{\pcmn{:} \p{}}
\begin{définition}\peng{Fort; pillbox; blockhouse for military use.}\end{définition}
\begin{définition}\pcmn{碉堡(汉语借词)}\end{définition}
\begin{définition}\pfra{Fortin, fort, forteresse.}\end{définition}
\end{entrée}

\begin{entrée}
{tjɤ˩˥ʂɯ˧}{}{ⓔtjɤ˩˥ʂɯ˧}\formedesurface{tjɤ˩˥ʂɯ˧}\newline
\classe{名词}\ton{LH.M}\begin{définition}\peng{Television.}\end{définition}
\begin{définition}\pcmn{电视(汉语借词)}\end{définition}
\begin{définition}\pfra{Télévision.}\end{définition}
\begin{exemple}\pnru{tjɤ˩ʂɯ˧ li˥}\hspace{5pt}\peng{to watch television}\hspace{5pt}\pcmn{看电视}\hspace{5pt}\pfra{regarder la télévision}\end{exemple}
\begin{exemple}\pnru{tjɤ˩ʂɯ˧-qo˥}\hspace{5pt}\peng{on television}\hspace{5pt}\pcmn{电视上}\hspace{5pt}\pfra{à la télévision}\end{exemple}
\end{entrée}

\begin{entrée}
{to˥α}{}{ⓔto˥α}\formedesurface{ɖɯ˧ to˥}\newline
\classe{量词}\ton{Hα}\begin{définition}\peng{An armful of.}\end{définition}
\begin{définition}\pcmn{量词:抱}\end{définition}
\begin{définition}\pfra{Une grande brassée, ce qu'on peut prendre dans les bras: par exemple lors de la récolte: une brassée de riz coupé.}\end{définition}
\begin{exemple}\pnru{qʰv̩˩ɖʐæ˩˥ | ɖɯ˧-to˥}\hspace{5pt}\peng{an armful of string (i.e. a huge quantity of string; see the narrative TraderAndHisSon)}\hspace{5pt}\pcmn{一抱绳子}\hspace{5pt}\pfra{toute une brassée de ficelle/cordelette (voir le récit TraderAndHisSon)}\end{exemple}
\end{entrée}

\begin{entrée}
{to˩˥}{}{ⓔto˩˥}\formedesurface{to˩˥}\newline
\classe{名词}\ton{LH}
\paradigme{\pcmn{:} \p{}}
\begin{définition}\peng{Mountain slope, hillside.}\end{définition}
\begin{définition}\pcmn{山坡,岗}\end{définition}
\begin{définition}\pfra{Pente, versant escarpé de montagne/colline.}\end{définition}
\begin{exemple}\pnru{to˩ do˩˥}\hspace{5pt}\peng{to climb a hillside}\hspace{5pt}\pcmn{爬山坡}\hspace{5pt}\pfra{grimper une pente}\end{exemple}
\begin{exemple}\pnru{ʁwɤ˧-to˩}\hspace{5pt}\peng{mountain slope}\hspace{5pt}\pcmn{山坡}\hspace{5pt}\pfra{pente de montagne}\end{exemple}
\end{entrée}

\begin{entrée}
{to˩α}{₁}{ⓔto˩αⓗ1}\formedesurface{to˩˥}\newline
\classe{动词}\ton{Lα}
1\begin{définition}\peng{To wrestle.}\end{définition}
\begin{définition}\pcmn{摔交}\end{définition}
\begin{définition}\pfra{Lutter, faire de la lutte.}\end{définition}
\begin{exemple}\pnru{le˧-to˩-ze˩}\hspace{5pt}\peng{|fg{accomp} \_ |fg{pfv}}\hspace{5pt}\pcmn{|fg{accomp} \_ |fg{pfv}}\hspace{5pt}\pfra{|fg{accomp} \_ |fg{pfv}}\end{exemple}
\begin{exemple}\pnru{ɖʐæ˧∼ɖʐæ˧ to˩}\hspace{5pt}\peng{to wrestle}\hspace{5pt}\pcmn{摔交}\hspace{5pt}\pfra{lutter, faire de la lutte}\end{exemple}
\end{entrée}

\begin{entrée}
{to˩α}{₂}{ⓔto˩αⓗ2}\formedesurface{to˩˥}\newline
\classe{动词}\ton{Lα}
2\begin{définition}\peng{To lie down.}\end{définition}
\begin{définition}\pcmn{躺下}\end{définition}
\begin{définition}\pfra{S'allonger.}\end{définition}
\begin{exemple}\pnru{tʰi˧-to˩-ɕjɤ˩}\hspace{5pt}\peng{to lie down and rest}\hspace{5pt}\pcmn{躺着休息}\hspace{5pt}\pfra{se reposer en position allongée, s'allonger pour prendre un peu de repos}\end{exemple}
\end{entrée}

\begin{entrée}
{to˩α}{₃}{ⓔto˩αⓗ3}\formedesurface{to˩˥}\newline
\classe{动词}\ton{Lα}
3\begin{définition}\peng{To stand in a family relationship, to have family ties.}\end{définition}
\begin{définition}\pcmn{有亲属关系}\end{définition}
\begin{définition}\pfra{Être en relation, entretenir un lien de parenté.}\end{définition}
\begin{exemple}\pnru{le˧-to˧∼to˥}\hspace{5pt}\peng{|fg{accomp} \_ |fg{red}}\hspace{5pt}\pcmn{|fg{accomp} \_ |fg{red}}\hspace{5pt}\pfra{|fg{accomp} \_ |fg{red}}\end{exemple}
\begin{exemple}\pnru{qʰwɤ˩ɖɯ˩˥ | le˧-to˩-ze˩}\hspace{5pt}\peng{We have acquired a family tie! (through adoption, marriage…)}\hspace{5pt}\pcmn{我们成了亲戚!(通过领养、婚姻……)}\hspace{5pt}\pfra{(nous) avons acquis un lien de parenté! (par adoption, mariage…)}\end{exemple}
\end{entrée}

\begin{entrée}
{to˧bɤ\#˥}{}{ⓔto˧bɤ\#˥}\formedesurface{to˧bɤ˧}\newline
\classe{形容词}\ton{\#H}\begin{définition}\peng{Empty.}\end{définition}
\begin{définition}\pcmn{空}\end{définition}
\begin{définition}\pfra{Vide.}\end{définition}
\begin{exemple}\pnru{to˧bɤ˧-ze˩}\hspace{5pt}\peng{|fg{pfv}: it's empty, there is nothing left (e.g. a bowl is entirely emptied)}\hspace{5pt}\pcmn{空了}\hspace{5pt}\pfra{il n'y a plus rien (ex.: un bol est complètement vidé)}\end{exemple}
\begin{exemple}\pnru{to˧bɤ˧ ɲi˥}\hspace{5pt}\peng{\_ |fg{cop}: it's empty}\hspace{5pt}\pcmn{是空的}\hspace{5pt}\pfra{\_ |fg{cop}: c'est vide}\end{exemple}
\end{entrée}

\begin{entrée}
{to˩bi˩}{}{ⓔto˩bi˩}\formedesurface{ɖɯ˧ to˩bi˩}\newline
\classe{量词}\ton{L}\begin{définition}\peng{Self-classifier for bottles.}\end{définition}
\begin{définition}\pcmn{量词:瓶}\end{définition}
\begin{définition}\pfra{Auto-classificateur des bouteilles.}\end{définition}
\begin{exemple}\pnru{ɖɯ˧-to˩bi˩, so˩-to˩bi˩˥, ʐv̩˧-to˥bi˩, qʰv̩˧-to˥bi˩, ʂɯ˧-to˩bi˩, gv̩˧-to˥bi˩, tsʰe˩-to˩bi˩˥}\hspace{5pt}\peng{association with numerals from 1 to 10}\hspace{5pt}\pcmn{与数词结合,一至十}\hspace{5pt}\pfra{association avec des numéraux, de 1 à 10. Comportement tonal identique pour 1 et 2, 4 et 5, 6 et 8.}\end{exemple}
\end{entrée}

\begin{entrée}
{to˩bi\#˥}{}{ⓔto˩bi\#˥}\formedesurface{to˩bi˥}\newline
\classe{名词}\ton{LM+\#H}
\paradigme{\pcmn{:} \p{}}
\begin{définition}\peng{Bottle.}\end{définition}
\begin{définition}\pcmn{瓶子}\end{définition}
\begin{définition}\pfra{Bouteille.}\end{définition}
\end{entrée}

\begin{entrée}
{to˧kɤ\#˥}{}{ⓔto˧kɤ\#˥}\newline
\classe{名词}
\sens{1}\paradigme{\pcmn{:} \p{}}
\begin{définition}\peng{Forehead.}\end{définition}
\begin{définition}\pcmn{额头}\end{définition}
\begin{définition}\pfra{Front.}\end{définition}\sens{2}
\begin{définition}\peng{Luck, good fortune.}\end{définition}
\begin{définition}\pcmn{运气}\end{définition}
\begin{définition}\pfra{Chance, bonne fortune.}\end{définition}
\begin{exemple}\pnru{to˧kɤ˧ dʑɤ˥}\hspace{5pt}\peng{to be lucky; to have a good karma}\hspace{5pt}\pcmn{好运气,运气好}\hspace{5pt}\pfra{avoir de la chance; avoir un bon karma}\end{exemple}
\begin{exemple}\pnru{njɤ˧ | tsʰi˧ʝi˧ | to˧kɤ˧ dʑjɤ˥ (+ | ʐwæ˩˥)}\hspace{5pt}\peng{This year, I am lucky! / This is an auspicious year for me!}\hspace{5pt}\pcmn{我今年运气好!}\hspace{5pt}\pfra{Cette année, j'ai de la chance!}\end{exemple}
\end{entrée}

\begin{entrée}
{to˧kɤ˧-qʰæ˩di˩ | bæ˩bæ˩˥}{}{ⓔto˧kɤ˧-qʰæ˩di˩ | bæ˩bæ˩˥}\formedesurface{to˧kɤ˧qʰæ˩di˩bæ˩bæ˩˥}\newline
\classe{名词}\ton{-L|L}
\paradigme{\pcmn{:} \p{}}
\begin{définition}\peng{Plant with long filaments.}\end{définition}
\begin{définition}\pcmn{永宁的一种植物}\end{définition}
\begin{définition}\pfra{Plante à longs filaments.}\end{définition}
\end{entrée}

\begin{entrée}
{to˩kʰv̩˩mi˥}{}{ⓔto˩kʰv̩˩mi˥}\formedesurface{to˩kʰv̩˩mi˥}\newline
\classe{名词}\ton{L+H\#}
\paradigme{\pcmn{:} \p{}}
\begin{définition}\peng{Male dog.}\end{définition}
\begin{définition}\pcmn{公狗}\end{définition}
\begin{définition}\pfra{Chien (animal mâle).}\end{définition}
\end{entrée}

\begin{entrée}
{to˩mi˩}{₁}{ⓔto˩mi˩ⓗ1}\formedesurface{to˩mi˩˥}\newline
\classe{名词}\ton{L}
1
\paradigme{\pcmn{:} \p{}}
\begin{définition}\peng{Pillar.}\end{définition}
\begin{définition}\pcmn{柱子}\end{définition}
\begin{définition}\pfra{Pilier.}\end{définition}
\begin{exemple}\pnru{hæ̃˧ʂɯ˩-to˩mi˩}\hspace{5pt}\peng{the Precious Pillars, the Golden Pillars: a solemn designation for the two pillars of the main building}\hspace{5pt}\pcmn{‘黄金柱’、‘宝贵柱’:对主屋两个柱子的庄严称呼}\hspace{5pt}\pfra{les Piliers d'Or, les Précieux Piliers: appellation solennelle pour les deux piliers de la maison}\end{exemple}
\end{entrée}

\begin{entrée}
{to˩mi˩}{₂}{ⓔto˩mi˩ⓗ2}\formedesurface{to˩mi˩˥}\newline
\classe{名词}\ton{L}
2\begin{définition}\peng{Large slope.}\end{définition}
\begin{définition}\pcmn{大山坡}\end{définition}
\begin{définition}\pfra{Grande pente (mot attesté, pas combinaison artificielle).}\end{définition}
\end{entrée}

\begin{entrée}
{to˩pi˩}{}{ⓔto˩pi˩}\formedesurface{ɖɯ˧ to˩pi˩}\newline
\classe{量词}\ton{L}\begin{définition}\peng{Classifier for times: n times as many/much as…}\end{définition}
\begin{définition}\pcmn{量词:倍(多几倍、少几倍等等)}\end{définition}
\begin{définition}\pfra{Fois, multiple de.}\end{définition}
\begin{exemple}\pnru{ɖɯ˧-to˩pi˩, ɲi˧-to˩pi˩, so˩-to˩pi˩˥, ʐv̩˧-to˥pi˩, qʰv̩˧-to˥pi˩, ʂɯ˧-to˩pi˩, gv̩˧-to˥pi˩, tsʰe˩-to˩pi˩˥}\hspace{5pt}\peng{association with numerals from 1 to 10}\hspace{5pt}\pcmn{与数词结合,一至十}\hspace{5pt}\pfra{association avec des numéraux, de 1 à 10. Comportement tonal identique pour 1 et 2, 4 et 5, 6 et 8.}\end{exemple}
\end{entrée}

\begin{entrée}
{to˩pv̩˧}{}{ⓔto˩pv̩˧}\formedesurface{to˩pv̩˥}\newline
\classe{助词}\ton{LM}\begin{définition}\peng{To begin with, at first, in the first place.}\end{définition}
\begin{définition}\pcmn{最初}\end{définition}
\begin{définition}\pfra{Au début, pour commencer.}\end{définition}
\begin{exemple}\pnru{ʁo˧dɑ˧ | to˩pv˥ | so˩˥, | ɖɯ˧-pi˧˥ | hw̃ɤ˩˥! | gɯ˩-ʝi˥ gɯ˩-dʑo˩ | ɖɯ˧-pi˧˥ | le˧-ʈʂʰwæ˩-ze˩! | ɖɯ˧-pi˧˥ | le˧-kv˧˥ ◊ -dʑo˩, | le˧-ʈʂʰwæ˩-ze˩!}\hspace{5pt}\peng{Before, at first, (your) learning (Mosuo) was a little slow. (Now,) it is really getting a bit faster! You know some (Mosuo, and so) it gets faster!}\hspace{5pt}\pcmn{以前,最初的时候,(你)学习(摩梭话)有一点慢!(到了现在,)真的有一点快了!因为你会一些(摩梭话),所以快起来了!}\hspace{5pt}\pfra{Autrefois, au début, ton apprentissage, c'était un peu lent! Vrai de vrai, ça devient un peu plus rapide! Comme tu connais maintenant un peu (la langue), (le travail de transcription) va plus vite!' (Commentaire de F4 en 2017, au sujet des progrès dans le travail de transcription de textes.)}\end{exemple}
\end{entrée}

\begin{entrée}
{to˧qɑ˧}{}{ⓔto˧qɑ˧}\formedesurface{to˧qɑ˧}\newline
\classe{名词}\ton{M}
\paradigme{\pcmn{:} \p{}}
\begin{définition}\peng{Kid (baby goat, young goat).}\end{définition}
\begin{définition}\pcmn{羔羊}\end{définition}
\begin{définition}\pfra{Chevreau.}\end{définition}
\end{entrée}

\begin{entrée}
{to˩qo˧˥}{}{ⓔto˩qo˧˥}\formedesurface{to˩qo˧˥}\newline
\classe{动词}\ton{L+MH\#}\begin{définition}\peng{To turn upside down.}\end{définition}
\begin{définition}\pcmn{倒过来、倒放倒置}\end{définition}
\begin{définition}\pfra{Renverser, verser; à l'envers (ex: renverser le contenu d'une boîte sur la table).}\end{définition}
\begin{exemple}\pnru{to˩qo˧˥ | tɕɯ˧}\hspace{5pt}\peng{to put upside down}\hspace{5pt}\pcmn{倒过来放}\hspace{5pt}\pfra{mettre à l'envers, renverser}\end{exemple}
\begin{exemple}\pnru{njɤ˧-ɳɯ˧ | to˩qo˧-bi˧!}\hspace{5pt}\peng{I am going to turn (this object) upside down!}\hspace{5pt}\pcmn{我要(将这个东西)倒过来放!}\hspace{5pt}\pfra{je vais renverser (ce pot, cette assiette…)}\end{exemple}
\begin{exemple}\pnru{to˩qo˧-ze˥}\hspace{5pt}\peng{|fg{pfv}}\hspace{5pt}\pcmn{倒过来了}\hspace{5pt}\pfra{|fg{pfv}}\end{exemple}
\end{entrée}

\begin{entrée}
{to˩qo˩lv̩˥}{}{ⓔto˩qo˩lv̩˥}\formedesurface{to˩qo˩lv̩˥}\newline
\classe{形容词}\ton{L+H\#}\begin{définition}\peng{Round in shape.}\end{définition}
\begin{définition}\pcmn{圆形(球很圆)}\end{définition}
\begin{définition}\pfra{Rond.}\end{définition}
\begin{exemple}\pnru{to˩qo˩lv̩˥-gv̩˩}\hspace{5pt}\peng{round in shape}\hspace{5pt}\pcmn{圆形}\hspace{5pt}\pfra{rond}\end{exemple}
\end{entrée}

\begin{entrée}
{to˧∼to˧β}{}{ⓔto˧∼to˧β}\formedesurface{to˧to˧}\newline
\classe{动词}\ton{Mβ}\begin{définition}\peng{To hold a child in one's arms; to hug.}\end{définition}
\begin{définition}\pcmn{抱小孩子、搂,互相拥抱}\end{définition}
\begin{définition}\pfra{Prendre un enfant dans ses bras.}\end{définition}
\begin{exemple}\pnru{zo˧mv̩˥ to˩∼to˩}\hspace{5pt}\peng{to hold a child in one's arms, to hug a child}\hspace{5pt}\pcmn{抱小孩子}\hspace{5pt}\pfra{porter un enfant dans ses bras}\end{exemple}
\end{entrée}

\begin{entrée}
{to˩to˧mi˥}{}{ⓔto˩to˧mi˥}\formedesurface{to˩to˧mi˥}\newline
\classe{助词}\ton{LM+H\#}
\sens{1}
\begin{définition}\peng{Carefully.}\end{définition}
\begin{définition}\pcmn{认真地}\end{définition}
\begin{définition}\pfra{Soigneusement, attentivement.}\end{définition}
\begin{exemple}\pnru{njɤ˧-ɳɯ˧ | to˩to˧ mi˥ | ʐwɤ˩-bi˩˥! |}\hspace{5pt}\peng{I will explain carefully! / I will explain very clearly, step by step!}\hspace{5pt}\pcmn{我要认真地讲!}\hspace{5pt}\pfra{je vais parler soigneusement/je vais bien expliquer!}\end{exemple}
\begin{exemple}\pnru{to˩to˧-mi˥ | so˩˥}\hspace{5pt}\peng{to study with great care}\hspace{5pt}\pcmn{认真地学习}\hspace{5pt}\pfra{étudier attentivement}\end{exemple}\sens{2}
\begin{définition}\peng{Intentionally, purposedly, on purpose.}\end{définition}
\begin{définition}\pcmn{故意地}\end{définition}
\begin{définition}\pfra{Volontairement, délibérément, de propos délibéré: quelqu'un fait exprès de faire quelque chose.}\end{définition}
\end{entrée}

\begin{entrée}
{to˩ʈɯ˩}{}{ⓔto˩ʈɯ˩}\formedesurface{to˩ʈɯ˩˥}\newline
\classe{形容词}\ton{L}\begin{définition}\peng{Short (of person).}\end{définition}
\begin{définition}\pcmn{矮}\end{définition}
\begin{définition}\pfra{Petit (d'un homme).}\end{définition}
\begin{exemple}\pnru{to˩ʈɯ˩∼ʈɯ˥}\hspace{5pt}\peng{short}\hspace{5pt}\pcmn{矮}\hspace{5pt}\pfra{petit, de petite taille}\end{exemple}
\end{entrée}

\begin{entrée}
{to˩zo˩}{}{ⓔto˩zo˩}\formedesurface{to˩zo˩˥}\newline
\classe{名词}\ton{L}\begin{définition}\peng{Small slope.}\end{définition}
\begin{définition}\pcmn{小山坡}\end{définition}
\begin{définition}\pfra{Petite pente (mot attesté, pas combinaison artificielle).}\end{définition}
\end{entrée}

\begin{entrée}
{tõ˧kwɤ˧}{}{ⓔtõ˧kwɤ˧}\formedesurface{tõ˧kwɤ˧}\newline
\classe{名词}\ton{M}\begin{définition}\peng{Wax gourd, white gourd, winter melon.}\end{définition}
\begin{définition}\pcmn{冬瓜}\end{définition}
\begin{définition}\pfra{Benincasa hispida.}\end{définition}
\end{entrée}

\begin{entrée}
{tv̩˧˥}{₁}{ⓔtv̩˧˥ⓗ1}\formedesurface{tv̩˧˥}\newline
\classe{动词}\ton{MH}
1\begin{définition}\peng{To support, to stabilize, to consolidate.}\end{définition}
\begin{définition}\pcmn{搀扶、撑住、稳住}\end{définition}
\begin{définition}\pfra{Soutenir.}\end{définition}
\end{entrée}

\begin{entrée}
{tv̩˧˥}{₂}{ⓔtv̩˧˥ⓗ2}\formedesurface{tv̩˧˥}\newline
\classe{动词}\ton{MH}
2\begin{définition}\peng{To pour (a liquid) into someone's mouth, to pour down someone's throat (e.g. pouring medicines into the throat of a sick person).}\end{définition}
\begin{définition}\pcmn{喂,喂到嘴里}\end{définition}
\begin{définition}\pfra{Verser (un liquide) dans la bouche de quelqu'un, faire boire à quelqu'un.}\end{définition}
\begin{exemple}\pnru{ʈʂʰæ˧ɣɯ˧ | tʰi˧-tv̩˧˥}\hspace{5pt}\peng{to give medicines, to pour medicines into the throat of a sick person}\hspace{5pt}\pcmn{喂药}\hspace{5pt}\pfra{verser un médicament dans la bouche de quelqu'un, faire boire un médicament à quelqu'un}\end{exemple}
\end{entrée}

\begin{entrée}
{tv̩˧α}{}{ⓔtv̩˧α}\formedesurface{tv̩˧}\newline
\classe{动词}\ton{Mα}\begin{définition}\peng{To plant, to bed out (rice).}\end{définition}
\begin{définition}\pcmn{耕种、插秧}\end{définition}
\begin{définition}\pfra{Planter; aussi: repiquer (le riz).}\end{définition}
\begin{exemple}\pnru{ɕi˧ tv̩˧}\hspace{5pt}\peng{to bed out rice}\hspace{5pt}\pcmn{插秧}\hspace{5pt}\pfra{repiquer le riz}\end{exemple}
\begin{exemple}\pnru{le˧-tv̩˧-ze˧}\hspace{5pt}\pfra{|fg{accomp} \_ |fg{pfv}}\end{exemple}
\begin{exemple}\pnru{le˧-tv̩˥-tv̩˩-ze˩}\hspace{5pt}\pfra{|fg{red}}\end{exemple}
\end{entrée}

\begin{entrée}
{tv̩˧α}{₁}{ⓔtv̩˧αⓗ1}\formedesurface{ɖɯ˧ tv̩˧}\newline
\classe{量词}\ton{Mα}
1\begin{définition}\peng{1,000.}\end{définition}
\begin{définition}\pcmn{千(数词充当量词)}\end{définition}
\begin{définition}\pfra{1.000.}\end{définition}
\begin{exemple}\pnru{ɖɯ˧-tv̩˧}\hspace{5pt}\peng{one thousand}\hspace{5pt}\pcmn{一千}\hspace{5pt}\pfra{mille}\end{exemple}
\begin{exemple}\pnru{ɖɯ˧-tv̩˧ tv̩˧}\hspace{5pt}\peng{one thousand thousands = one million}\hspace{5pt}\pcmn{一千千,等于一百万}\hspace{5pt}\pfra{mille milliers = un million}\end{exemple}
\begin{exemple}\pnru{tsʰe˩-tv̩˩ mæ˥}\hspace{5pt}\peng{ten thousand times 10,000, i.e. one hundred million}\hspace{5pt}\pcmn{十千万,等于一亿}\hspace{5pt}\pfra{dix mille fois 10.000, soit cent millions}\end{exemple}
\end{entrée}

\begin{entrée}
{tv̩˧α}{₂}{ⓔtv̩˧αⓗ2}\formedesurface{ɖɯ˧ tv̩˧}\newline
\classe{量词}\ton{Mα}
2\begin{définition}\peng{Classifier: a dime, i.e. one tenth of the monetary unit.}\end{définition}
\begin{définition}\pcmn{量词:角(钱),一元的十分之一}\end{définition}
\begin{définition}\pfra{Dixième d'unité monétaire.}\end{définition}
\end{entrée}

\begin{entrée}
{tv̩˧ɕi˩}{}{ⓔtv̩˧ɕi˩}\formedesurface{tv̩˧ɕi˩}\newline
\classe{名词}\ton{L\#}
\paradigme{\pcmn{:} \p{}}
\begin{définition}\peng{Centipede.}\end{définition}
\begin{définition}\pcmn{蜈蚣}\end{définition}
\begin{définition}\pfra{Millepattes.}\end{définition}
\end{entrée}

\begin{entrée}
{tv̩˩ɭɯ˧˥}{}{ⓔtv̩˩ɭɯ˧˥}\formedesurface{tv̩˩ɭɯ˧˥}\newline
\classe{名词}\ton{LM+MH\#}
\paradigme{\pcmn{:} \p{}}
\begin{définition}\peng{Fine, high-quality basket carried on the back; it had a trough shape. Not in use anymore at the time of fieldwork.}\end{définition}
\begin{définition}\pcmn{高级的背篓,过去用它放礼物}\end{définition}
\begin{définition}\pfra{Hotte de grande qualité, dans laquelle on offrait des cadeaux; n'existe plus actuellement; était resserrée au milieu: de forme concave, pas convexe.}\end{définition}
\end{entrée}

\begin{entrée}
{tv̩˧po˩}{}{ⓔtv̩˧po˩}\formedesurface{tv̩˧po˩}\newline
\classe{动词}\ton{L\#}\begin{définition}\peng{To gamble, to bet, to wager.}\end{définition}
\begin{définition}\pcmn{赌博(汉语借词)}\end{définition}
\begin{définition}\pfra{Parier, jouer à des jeux d'argent.}\end{définition}
\end{entrée}

\begin{entrée}
{tv̩˧qʰv̩˧}{}{ⓔtv̩˧qʰv̩˧}\formedesurface{tv̩˧qʰv̩˧}\newline
\classe{名词}\ton{M}\begin{définition}\peng{Temporary tomb, where the body is placed prior to cremation.}\end{définition}
\begin{définition}\pcmn{临时坟墓}\end{définition}
\begin{définition}\pfra{Tombe provisoire, où on place le corps du défunt avant la crémation.}\end{définition}
\end{entrée}

\begin{entrée}
{tv̩˧tv̩˥}{}{ⓔtv̩˧tv̩˥}\formedesurface{tv̩˧tv̩˥}\newline
\classe{名词}\ton{H\#}
\paradigme{\pcmn{:} \p{}}
\begin{définition}\peng{Hat.}\end{définition}
\begin{définition}\pcmn{帽子}\end{définition}
\begin{définition}\pfra{Chapeau.}\end{définition}
\end{entrée}

\begin{entrée}
{tv̩˩tv̩˩}{}{ⓔtv̩˩tv̩˩}\newline
\classe{形容词}
\sens{1}
\begin{définition}\peng{Upright.}\end{définition}
\begin{définition}\pcmn{直,笔直的(如:站直)}\end{définition}
\begin{définition}\pfra{Droit, bien d'aplomb.}\end{définition}\sens{2}
\begin{définition}\peng{Upright, righteous, honest.}\end{définition}
\begin{définition}\pcmn{耿直}\end{définition}
\begin{définition}\pfra{Droit, intègre, honnête.}\end{définition}
\end{entrée}

\begin{entrée}
{tv̩˧tsʰɯ˧}{}{ⓔtv̩˧tsʰɯ˧}\newline
\classe{名词}
\sens{1}\paradigme{\pcmn{:} \p{}}
\begin{définition}\peng{Time.}\end{définition}
\begin{définition}\pcmn{时间}\end{définition}
\begin{définition}\pfra{Temps.}\end{définition}
\begin{exemple}\pnru{njɤ˧ | tv̩˧tsʰɯ˧ mɤ˧-dʑo˧.}\hspace{5pt}\peng{I don't have the time.}\hspace{5pt}\pcmn{我没时间。}\hspace{5pt}\pfra{Je n'ai pas le temps.}\end{exemple}
\begin{exemple}\pnru{njɤ˧ | tv̩˧tsʰɯ˧ dʑo˧.}\hspace{5pt}\peng{I have time. / I have some free time. / I have the time.}\hspace{5pt}\pcmn{我有时间。}\hspace{5pt}\pfra{J'ai du temps libre. / J'ai le temps.}\end{exemple}\sens{2}
\begin{définition}\peng{Spell of time; hour.}\end{définition}
\begin{définition}\pcmn{时间段、小时}\end{définition}
\begin{définition}\pfra{Période de temps, heure.}\end{définition}
\begin{exemple}\pnru{tv̩˧tsʰɯ˧ | ɖɯ˧-ɭɯ˧}\hspace{5pt}\peng{one hour}\hspace{5pt}\pcmn{一个小时}\hspace{5pt}\pfra{une heure}\end{exemple}
\begin{exemple}\pnru{tv̩˧tsʰɯ˧ ɖɯ˧-ɭɯ˧ gv̩˧-ze˧!}\hspace{5pt}\peng{One hour has gone by.}\hspace{5pt}\pcmn{一个小时过去了。}\hspace{5pt}\pfra{Une heure a passé.}\end{exemple}
\begin{exemple}\pnru{tv̩˧tsʰɯ˧ ɖɯ˧-ɭɯ˧ le˧-hɯ˩-ze˩.}\hspace{5pt}\peng{One hour has gone by.}\hspace{5pt}\pcmn{一个小时过去了。}\hspace{5pt}\pfra{Une heure s'est écoulée.}\end{exemple}
\begin{exemple}\pnru{tv̩˧tsʰɯ˧ qʰɑ˧-ɭɯ˧?}\hspace{5pt}\peng{What time is it? (Literally: ‘How many hours?’)}\hspace{5pt}\pcmn{几点了?}\hspace{5pt}\pfra{Quelle heure est-il?}\end{exemple}
\end{entrée}

\begin{entrée}
{tv̩˧tsʰɯ˧li˧di˩}{}{ⓔtv̩˧tsʰɯ˧li˧di˩}\formedesurface{tv̩˧tsʰɯ˧li˧di˩}\newline
\classe{名词}\ton{L\#}\begin{définition}\peng{Clock; literally: ‘thing for seeing the time’.}\end{définition}
\begin{définition}\pcmn{钟、手表}\end{définition}
\begin{définition}\pfra{Horloge; montre. Littéralement: ‘chose pour voir l'heure’.}\end{définition}
\end{entrée}

\begin{entrée}
{tʰɑ˧‑}{}{ⓔtʰɑ˧‑}\formedesurface{tʰɑ˧}\newline
\classe{前缀}\ton{M/0}\begin{définition}\peng{Prohibitive.}\end{définition}
\begin{définition}\pcmn{禁止式:不要、别}\end{définition}
\begin{définition}\pfra{Prohibitif.}\end{définition}
\begin{exemple}\pnru{tʰɑ˧-lɑ˩∼lɑ˩-ze˩!}\hspace{5pt}\peng{Don't quarrel!}\hspace{5pt}\pcmn{别吵架了!}\hspace{5pt}\pfra{Arrêtez de vous disputer! / Ne vous disputez pas!}\end{exemple}
\begin{exemple}\pnru{tʰɑ˧-dzo˧∼dzo˥!}\hspace{5pt}\peng{Don't touch!}\hspace{5pt}\pcmn{不要动来动去! / 不要碰来碰去!}\hspace{5pt}\pfra{Ne pas toucher! / Ne touchez pas!}\end{exemple}
\end{entrée}

\begin{entrée}
{tʰɑ˧˥}{₁}{ⓔtʰɑ˧˥ⓗ1}\formedesurface{tʰɑ˧˥}\newline
\classe{形容词}\ton{MH}
1\begin{définition}\peng{Sharp, keen.}\end{définition}
\begin{définition}\pcmn{锋利}\end{définition}
\begin{définition}\pfra{Aiguisé, qui coupe bien, affûté.}\end{définition}
\end{entrée}

\begin{entrée}
{tʰɑ˧˥}{₂}{ⓔtʰɑ˧˥ⓗ2}\formedesurface{tʰɑ˧˥}\newline
\classe{动词}\ton{MH}
2\begin{définition}\peng{To be possible, to be allowed: |fg{permissive.}}\end{définition}
\begin{définition}\pcmn{可以,允许}\end{définition}
\begin{définition}\pfra{Être possible, être autorisé: |fg{permissif.}}\end{définition}
\begin{exemple}\pnru{mɤ˧-tʰɑ˧˥ | mɤ˧-ʐv̩˩! | njɤ˧ | dzɯ˧-bi˧ni˧-mɤ˧-gv̩˧˥!}\hspace{5pt}\peng{It's not that good! I don't like to eat that!}\hspace{5pt}\pcmn{不怎么好吃!我不喜欢吃!}\hspace{5pt}\pfra{Ce n'est pas vraiment bon! Je n'aime pas en manger!}\end{exemple}
\begin{exemple}\pnru{ə˩ljɤ˩hæ̃˩ʂɯ˥-mo˩-ʈʂʰɯ˩-dʑo˩, | hĩ˧ | mɤ˧-tʰɑ˧˥ | dv̩˩-mɤ˧-kv̩˧˥!}\hspace{5pt}\peng{The Golden Mushroom is not all that poisonous! / The Golden Mushroom is not really dangerous!}\hspace{5pt}\pcmn{黄蜡伞不怎么会让人中毒!/ 毒性不太大!}\hspace{5pt}\pfra{Le Champignon Doré, il n'est pas si vénéneux que ça!}\end{exemple}
\end{entrée}

\begin{entrée}
{tʰɑ˩˥}{}{ⓔtʰɑ˩˥}\formedesurface{tʰɑ˩˥}\newline
\classe{名词}\ton{LH}
\paradigme{\pcmn{:} \p{}}
\begin{définition}\peng{Water buffalo (monosyllabic form, extracted from disyllables such as \stylefn{/tʰɑ}˩mi\#˥/ ‘female buffalo').}\end{définition}
\begin{définition}\pcmn{水牛}\end{définition}
\begin{définition}\pfra{Buffle; forme monosyllabique, qui n'est pas utilisée telle quelle, seulement dans des formes telles que \stylefn{/tʰɑ}˩mi\#˥/ ‘buffle femelle'.}\end{définition}
\end{entrée}

\begin{entrée}
{tʰɑ˩lo˧}{}{ⓔtʰɑ˩lo˧}\formedesurface{tʰɑ˩lo˥}\newline
\classe{名词}\ton{LM}\begin{définition}\peng{The name given to the Yongning plain by the Tibetans.}\end{définition}
\begin{définition}\pcmn{永宁的藏语名称}\end{définition}
\begin{définition}\pfra{Prononciation par les Na de thar lam, nom anciennement donné par les Tibétains à la plaine de Yongning.}\end{définition}
\begin{exemple}\pnru{tʰɑ˩lo˧-go˧bɤ˩}\hspace{5pt}\peng{the temple of Thar Lam}\hspace{5pt}\pcmn{永宁大寺}\hspace{5pt}\pfra{le temple de Thar lam =le temple de Yongning, tel que l'appellent les Tibétains}\end{exemple}
\begin{exemple}\pnru{tʰɑ˩lo˧ se˧gi˧ kɤ˩mv̩˩}\hspace{5pt}\peng{mount Gemu, in Yongning}\hspace{5pt}\pcmn{永宁格姆山}\hspace{5pt}\pfra{la montagne Gemu de Yongning}\end{exemple}
\end{entrée}

\begin{entrée}
{tʰɑ˩mi\#˥}{}{ⓔtʰɑ˩mi\#˥}\formedesurface{tʰɑ˩mi˥}\newline
\classe{名词}\ton{LM+\#H}
\paradigme{\pcmn{:} \p{}}
\begin{définition}\peng{Female water buffalo.}\end{définition}
\begin{définition}\pcmn{母水牛}\end{définition}
\begin{définition}\pfra{Buffle femelle.}\end{définition}
\begin{exemple}\pnru{dʑi˧mi˧-tʰɑ˩mi˩}\hspace{5pt}\peng{same meaning: female water buffalo}\hspace{5pt}\pcmn{母水牛}\hspace{5pt}\pfra{même sens: buffle femelle}\end{exemple}
\begin{exemple}\pnru{dʑi˧mi˧ ʈʂʰɯ˧-pʰo˩ dʑo˩, | tʰɑ˩mi˧ ɲi˥!}\hspace{5pt}\peng{This buffalo is a female!}\hspace{5pt}\pcmn{这头水牛是母的!}\hspace{5pt}\pfra{ce buffle, c'est une femelle!}\end{exemple}
\end{entrée}

\begin{entrée}
{tʰɑ˧-ni˥-ni˩-gv̩˩}{}{ⓔtʰɑ˧-ni˥-ni˩-gv̩˩}\formedesurface{tʰɑ˧ni˥ni˩gv̩˩}\newline
\classe{助词}\ton{-H.L-}\begin{définition}\peng{Similar to.}\end{définition}
\begin{définition}\pcmn{类似,相近}\end{définition}
\begin{définition}\pfra{Comparable à, semblable à, similaire à.}\end{définition}
\end{entrée}

\begin{entrée}
{tʰɑ˩pʰv̩\#˥}{}{ⓔtʰɑ˩pʰv̩\#˥}\formedesurface{tʰɑ˩pʰv̩˥}\newline
\classe{名词}\ton{LM+\#H}
\paradigme{\pcmn{:} \p{}}
\begin{définition}\peng{Male water buffalo.}\end{définition}
\begin{définition}\pcmn{公水牛}\end{définition}
\begin{définition}\pfra{Buffle mâle.}\end{définition}
\end{entrée}

\begin{entrée}
{tʰɑ˩tʰɑ˩}{}{ⓔtʰɑ˩tʰɑ˩}\formedesurface{tʰɑ˩tʰɑ˩˥}\newline
\classe{名词}\ton{L}
\paradigme{\pcmn{:} \p{}}
\begin{définition}\peng{A good spot, a good place to find a certain species of plant: for instance, mushrooms that will grow there every year.}\end{définition}
\begin{définition}\pcmn{采野生植物如菌子等的好地方}\end{définition}
\begin{définition}\pfra{Bon coin pour la cueillette de champignons, de plantes sauvages…}\end{définition}
\begin{exemple}\pnru{tʰɑ˩tʰɑ˩˥ | ɖɯ˧-kʰwɤ˥}\hspace{5pt}\peng{a good spot (for hunting a certain kind of wild plant)}\hspace{5pt}\pcmn{一个好地方}\hspace{5pt}\pfra{un bon coin (pour la cueillette)}\end{exemple}
\end{entrée}

\begin{entrée}
{tʰɑ˧v̩˥}{}{ⓔtʰɑ˧v̩˥}\formedesurface{tʰɑ˧v̩˥}\newline
\classe{名词}\ton{H\#}
\paradigme{\pcmn{:} \p{}}
\begin{définition}\peng{Room for guests (local Chinese word; meaning in standard Chinese: central room, main hall). There is no direct equivalent in Na because the traditional house did not have a room for guests.}\end{définition}
\begin{définition}\pcmn{堂屋(汉语借词),来指客房}\end{définition}
\begin{définition}\pfra{Chambre des invités (emprunt au chinois local; sens en chinois standard: pièce centrale, salle de séjour). Il n'y a pas d'équivalent direct dans la maison na traditionnelle: c'est dans la resserre qu'on pouvait improviser une chambre supplémentaire.}\end{définition}
\end{entrée}

\begin{entrée}
{tʰɑ˩zo\#˥}{}{ⓔtʰɑ˩zo\#˥}\formedesurface{tʰɑ˩zo˥}\newline
\classe{名词}\ton{LM+\#H}
\paradigme{\pcmn{:} \p{}}
\begin{définition}\peng{Baby water buffalo.}\end{définition}
\begin{définition}\pcmn{小水牛}\end{définition}
\begin{définition}\pfra{Petit buffle, enfant du buffle.}\end{définition}
\end{entrée}

\begin{entrée}
{tʰɑ˩-ʐwæ˧mi˧}{}{ⓔtʰɑ˩-ʐwæ˧mi˧}\formedesurface{tʰɑ˩ʐwæ˧mi˧}\newline
\classe{名词}\ton{L-M}
\paradigme{\pcmn{:} \p{}}
\begin{définition}\peng{Donkey, ass (either jack or jenny or foal).}\end{définition}
\begin{définition}\pcmn{驴子}\end{définition}
\begin{définition}\pfra{Âne (mâle ou femelle).}\end{définition}
\end{entrée}

\begin{entrée}
{tʰæ˧ɻæ˩}{}{ⓔtʰæ˧ɻæ˩}\formedesurface{tʰæ˧ɻæ˩}\newline
\classe{名词}\ton{L\#}
\paradigme{\pcmn{:} \p{}}
\begin{définition}\peng{Book.}\end{définition}
\begin{définition}\pcmn{书}\end{définition}
\begin{définition}\pfra{Livre.}\end{définition}
\end{entrée}

\begin{entrée}
{tʰæ˩tsɯ˧}{}{ⓔtʰæ˩tsɯ˧}\formedesurface{tʰæ˩tsɯ˥}\newline
\classe{名词}\ton{LM}\begin{définition}\peng{Jar.}\end{définition}
\begin{définition}\pcmn{坛子(汉语借词)}\end{définition}
\begin{définition}\pfra{Jarre.}\end{définition}
\end{entrée}

\begin{entrée}
{tʰi˧}{}{ⓔtʰi˧}\formedesurface{tʰi˧}\newline
\classe{形容词}\ton{M}\begin{définition}\peng{Able; capable; competent; clever.}\end{définition}
\begin{définition}\pcmn{能干}\end{définition}
\begin{définition}\pfra{Compétent, habile.}\end{définition}
\begin{exemple}\pnru{ɖwæ˧˥ | tʰi˧}\hspace{5pt}\peng{|fg{intensive.very}}\hspace{5pt}\pcmn{很能干}\hspace{5pt}\pfra{|fg{intensif.très}: très habile}\end{exemple}
\begin{exemple}\pnru{mv̩˩tʰi˩ tʰv̩˩-v̩˩˥}\hspace{5pt}\peng{that clever woman}\hspace{5pt}\pcmn{那个聪明女人}\hspace{5pt}\pfra{cette femme intelligente}\end{exemple}
\begin{exemple}\pnru{zo˧tʰi˧}\hspace{5pt}\peng{clever man}\hspace{5pt}\pcmn{聪明男人}\hspace{5pt}\pfra{homme intelligent}\end{exemple}
\end{entrée}

\begin{entrée}
{tʰi˧‑}{}{ⓔtʰi˧‑}\formedesurface{tʰi˧}\newline
\classe{前缀}\ton{M/0}\begin{définition}\peng{Durative (|fg{dur}).}\end{définition}
\begin{définition}\pcmn{持续体}\end{définition}
\begin{définition}\pfra{Duratif (|fg{dur}).}\end{définition}
\begin{exemple}\pnru{tʰi˧-dzɯ˥-dʑo˩!}\hspace{5pt}\peng{(She) is eating!}\hspace{5pt}\pcmn{她在吃东西!}\hspace{5pt}\pfra{(Elle) est en train de manger! / Elle mange! (Contexte: on constate avec joie qu'un enfant qui ne mangeait plus depuis deux jours est en train de ronger à belles dents un épi de maïs.)}\end{exemple}
\begin{exemple}\pnru{tʰi˧-mɤ˧-ɲi˥}\hspace{5pt}\peng{otherwise, or else}\hspace{5pt}\pcmn{否则、要不然}\hspace{5pt}\pfra{faute de quoi}\end{exemple}
\end{entrée}

\begin{entrée}
{tʰi˩˥}{₁}{ⓔtʰi˩˥ⓗ1}\formedesurface{tʰi˩˥}\newline
\classe{名词}\ton{LH}
1
\paradigme{\pcmn{:} \p{}}
\begin{définition}\peng{Plane.}\end{définition}
\begin{définition}\pcmn{刨}\end{définition}
\begin{définition}\pfra{Rabot.}\end{définition}
\end{entrée}

\begin{entrée}
{tʰi˩˥}{₂}{ⓔtʰi˩˥ⓗ2}\formedesurface{tʰi˩˥}\newline
\classe{语气助词}\ton{LM? LH?}
2\begin{définition}\peng{Discourse particle: so, then, and then.}\end{définition}
\begin{définition}\pcmn{然后}\end{définition}
\begin{définition}\pfra{Particule de discours: alors, donc, après.}\end{définition}
\end{entrée}

\begin{entrée}
{tʰi˩α}{}{ⓔtʰi˩α}\formedesurface{tʰi˩˥}\newline
\classe{动词}\ton{Lα}\begin{définition}\peng{To plane (wood flat).}\end{définition}
\begin{définition}\pcmn{刨}\end{définition}
\begin{définition}\pfra{Raboter.}\end{définition}
\begin{exemple}\pnru{tso˧∼tso˧ tʰi˥(-ze˩)}\hspace{5pt}\peng{to plane things}\hspace{5pt}\pcmn{刨东西}\hspace{5pt}\pfra{raboter quelque chose}\end{exemple}
\begin{exemple}\pnru{le˧-tʰi˩-ze˩}\hspace{5pt}\peng{|fg{accomp} \_ |fg{pfv}}\hspace{5pt}\pcmn{刨了}\hspace{5pt}\pfra{|fg{accomp} \_ |fg{pfv}}\end{exemple}
\begin{exemple}\pnru{tso˧∼tso˧ | le˧-tʰi˩(-ze˩)}\hspace{5pt}\peng{to plane things}\hspace{5pt}\pcmn{刨东西}\hspace{5pt}\pfra{raboter quelque chose}\end{exemple}
\begin{exemple}\pnru{pæ˩pʰæ˧ tʰi˥}\hspace{5pt}\peng{to plane a plank}\hspace{5pt}\pcmn{刨木板}\hspace{5pt}\pfra{raboter une planche}\end{exemple}
\end{entrée}

\begin{entrée}
{tʰi˩mi\#˥}{}{ⓔtʰi˩mi\#˥}\formedesurface{tʰi˩mi˥}\newline
\classe{名词}\ton{LM+\#H}
\paradigme{\pcmn{:} \p{}}
\begin{définition}\peng{Large plane.}\end{définition}
\begin{définition}\pcmn{大刨}\end{définition}
\begin{définition}\pfra{Grand rabot.}\end{définition}
\end{entrée}

\begin{entrée}
{tʰi˩zo\#˥}{}{ⓔtʰi˩zo\#˥}\formedesurface{tʰi˧zo˥}\newline
\classe{名词}\ton{LM+\#H}
\paradigme{\pcmn{:} \p{}}
\begin{définition}\peng{Small plane.}\end{définition}
\begin{définition}\pcmn{小刨}\end{définition}
\begin{définition}\pfra{Petit rabot.}\end{définition}
\end{entrée}

\begin{entrée}
{tʰo˥α}{}{ⓔtʰo˥α}\formedesurface{ɖɯ˧ tʰo˥}\newline
\classe{量词}\ton{Hα}\begin{définition}\peng{Classifier for sets.}\end{définition}
\begin{définition}\pcmn{量词:套(汉语借词)}\end{définition}
\begin{définition}\pfra{Classificateur des ensembles, des lots.}\end{définition}
\end{entrée}

\begin{entrée}
{tʰo˥α}{}{ⓔtʰo˥α}\formedesurface{ɖɯ˧ tʰo˥}\newline
\classe{量词}\ton{Hα}\begin{définition}\peng{Classifier for solutions.}\end{définition}
\begin{définition}\pcmn{量词:办法,解决的方法(一个)}\end{définition}
\begin{définition}\pfra{Classificateur des solutions / issues heureuses.}\end{définition}
\begin{exemple}\pnru{ə˧tso˧ tʰo˧ dʑo˧-kv̩˩?}\hspace{5pt}\peng{What can we do about it? / What can be done about it?}\hspace{5pt}\pcmn{有什么办法?}\hspace{5pt}\pfra{Qu'est-ce qu'on y peut? / Qu'est-ce qu'on peut y faire?}\end{exemple}
\end{entrée}

\begin{entrée}
{tʰo˩α}{}{ⓔtʰo˩α}\formedesurface{tʰo˩˥}\newline
\classe{动词}\ton{Lα}\begin{définition}\peng{To lean on.}\end{définition}
\begin{définition}\pcmn{靠}\end{définition}
\begin{définition}\pfra{S’adosser à, s'appuyer.}\end{définition}
\begin{exemple}\pnru{tʰi˧-tʰo˩}\hspace{5pt}\peng{|fg{dur}}\hspace{5pt}\pcmn{|fg{dur}}\hspace{5pt}\pfra{|fg{dur}}\end{exemple}
\begin{exemple}\pnru{tʰi˧-tʰo˩-ɻ̍˩}\hspace{5pt}\peng{|fg{dur} \_ |fg{inceptive}}\hspace{5pt}\pcmn{|fg{dur} \_ |fg{inceptive}}\hspace{5pt}\pfra{|fg{dur} \_ |fg{inchoatif}}\end{exemple}
\begin{exemple}\pnru{ɖɯ˧-tʰo˩-ɻ̍˩}\hspace{5pt}\peng{|fg{delimitative} \_ |fg{inceptive}}\hspace{5pt}\pcmn{|fg{delimitative} \_ |fg{inceptive}}\hspace{5pt}\pfra{|fg{délimitatif} \_ |fg{inchoatif}}\end{exemple}
\end{entrée}

\begin{entrée}
{tʰo˧ɕi˧˥}{}{ⓔtʰo˧ɕi˧˥}\formedesurface{tʰo˧ɕi˧˥}\newline
\classe{名词}\ton{MH\#}
\paradigme{\pcmn{:} \p{}}
\begin{définition}\peng{Forest of conifers.}\end{définition}
\begin{définition}\pcmn{松树林}\end{définition}
\begin{définition}\pfra{Forêt de conifères.}\end{définition}
\end{entrée}

\begin{entrée}
{tʰo˧ɕi˩}{}{ⓔtʰo˧ɕi˩}\formedesurface{tʰo˧ɕi˩}\newline
\classe{名词}\ton{L\#}
\paradigme{\pcmn{:} \p{}}
\begin{définition}\peng{Messenger.}\end{définition}
\begin{définition}\pcmn{通信员(汉语借词)}\end{définition}
\begin{définition}\pfra{Messager.}\end{définition}
\end{entrée}

\begin{entrée}
{tʰo˧dzi˩}{}{ⓔtʰo˧dzi˩}\formedesurface{tʰo˧dzi˩}\newline
\classe{名词}\ton{L\#}
\paradigme{\pcmn{:} \p{}}
\begin{définition}\peng{Pine tree.}\end{définition}
\begin{définition}\pcmn{松树}\end{définition}
\begin{définition}\pfra{Pin.}\end{définition}
\end{entrée}

\begin{entrée}
{tʰo˧dzi˩-hwæ˩tsɯ˩}{}{ⓔtʰo˧dzi˩-hwæ˩tsɯ˩}\formedesurface{tʰo˧dzi˩hwæ˩tsɯ˩}\newline
\classe{名词}\ton{L\#-L}\begin{définition}\peng{Hedgehog. Literally: pine tree mouse.}\end{définition}
\begin{définition}\pcmn{刺猬}\end{définition}
\begin{définition}\pfra{Hérisson; littéralement «souris des pins».}\end{définition}
\end{entrée}

\begin{entrée}
{tʰo˧fv̩˧}{}{ⓔtʰo˧fv̩˧}\formedesurface{tʰo˧fv̩˧}\newline
\classe{名词}\ton{M}\begin{définition}\peng{Bandit, brigand.}\end{définition}
\begin{définition}\pcmn{土匪(汉语借词)}\end{définition}
\begin{définition}\pfra{Bandit, maraudeur.}\end{définition}
\end{entrée}

\begin{entrée}
{tʰo˧lɑ˧tɕi˧}{}{ⓔtʰo˧lɑ˧tɕi˧}\formedesurface{tʰo˧lɑ˧tɕi˧}\newline
\classe{名词}\ton{M}
\paradigme{\pcmn{:} \p{}}
\begin{définition}\peng{Tractor.}\end{définition}
\begin{définition}\pcmn{拖拉机(汉语借词)}\end{définition}
\begin{définition}\pfra{Tracteur.}\end{définition}
\begin{exemple}\pnru{bo˩mi˧-tʰo˧lɑ˧tɕi˧}\hspace{5pt}\peng{‘sow-tractor': a small tractor (the first type that was introduced into Yongning)}\hspace{5pt}\pcmn{‘母猪拖拉机’:小型拖拉机}\hspace{5pt}\pfra{‘tracteur-truie': petit tracteur (le premier modèle introduit à Yongning)}\end{exemple}
\end{entrée}

\begin{entrée}
{tʰo˧li˧}{}{ⓔtʰo˧li˧}\formedesurface{tʰo˧li˧}\newline
\classe{名词}\ton{M}
\paradigme{\pcmn{:} \p{}}
\begin{définition}\peng{Rabbit.}\end{définition}
\begin{définition}\pcmn{兔子}\end{définition}
\begin{définition}\pfra{Lapin.}\end{définition}
\end{entrée}

\begin{entrée}
{tʰo˧li˧-kʰv̩˧˥}{₁}{ⓔtʰo˧li˧-kʰv̩˧˥ⓗ1}\formedesurface{tʰo˧li˧kʰv̩˧˥}\newline
\classe{名词}\ton{MH\#}
1\begin{définition}\peng{Year of the Rabbit.}\end{définition}
\begin{définition}\pcmn{兔年}\end{définition}
\begin{définition}\pfra{Année du Lapin.}\end{définition}
\end{entrée}

\begin{entrée}
{tʰo˧li˧-kʰv̩˧˥}{₂}{ⓔtʰo˧li˧-kʰv̩˧˥ⓗ2}\formedesurface{tʰo˧li˧kʰv̩˧˥}\newline
\classe{名词}\ton{MH\#}
2\begin{définition}\peng{Born in the year of the Rabbit.}\end{définition}
\begin{définition}\pcmn{属兔}\end{définition}
\begin{définition}\pfra{Né l'année du Lapin.}\end{définition}
\end{entrée}

\begin{entrée}
{tʰo˧li˧-mi˩}{}{ⓔtʰo˧li˧-mi˩}\formedesurface{tʰo˧li˧mi˩}\newline
\classe{名词}\ton{-L}
\paradigme{\pcmn{:} \p{}}
\begin{définition}\peng{Doe hare, jill.}\end{définition}
\begin{définition}\pcmn{母兔}\end{définition}
\begin{définition}\pfra{Lapin femelle.}\end{définition}
\end{entrée}

\begin{entrée}
{tʰo˧li˧-pʰv̩\#˥}{}{ⓔtʰo˧li˧-pʰv̩\#˥}\formedesurface{tʰo˧li˧pʰv̩˧}\newline
\classe{名词}\ton{\#H}
\paradigme{\pcmn{:} \p{}}
\begin{définition}\peng{Male rabbit.}\end{définition}
\begin{définition}\pcmn{公兔}\end{définition}
\begin{définition}\pfra{Lapin mâle.}\end{définition}
\end{entrée}

\begin{entrée}
{tʰo˧li˧-zo\#˥}{}{ⓔtʰo˧li˧-zo\#˥}\formedesurface{tʰo˧li˧zo˧}\newline
\classe{名词}\ton{\#H}
\paradigme{\pcmn{:} \p{}}
\begin{définition}\peng{Baby rabbit.}\end{définition}
\begin{définition}\pcmn{小兔}\end{définition}
\begin{définition}\pfra{Petit lapin, bébé lapin.}\end{définition}
\end{entrée}

\begin{entrée}
{tʰo˩lo˧}{}{ⓔtʰo˩lo˧}\formedesurface{tʰo˩lo˥}\newline
\classe{名词}\ton{LM}\begin{définition}\peng{The horse walking in front in a caravan.}\end{définition}
\begin{définition}\pcmn{头马:马帮里走在最前面的那匹马}\end{définition}
\begin{définition}\pfra{Cheval de tête, dans une caravane.}\end{définition}
\end{entrée}

\begin{entrée}
{tʰo˧-mo˩}{}{ⓔtʰo˧-mo˩}\formedesurface{tʰo˧mo˩}\newline
\classe{名词}\ton{L\#}\begin{définition}\peng{“Pine-tree mushroom": an edible mushroom often found close to pine trees.}\end{définition}
\begin{définition}\pcmn{“松树菌”:一种菌子}\end{définition}
\begin{définition}\pfra{«champignon des sapins»: champignon comestible, ainsi nommé parce qu'il pousse au pied des sapins.}\end{définition}
\end{entrée}

\begin{entrée}
{tʰo˩pʰv̩˧tɕʰɤ˧}{}{ⓔtʰo˩pʰv̩˧tɕʰɤ˧}\formedesurface{tʰo˩pʰv̩˧tɕʰɤ˧}\newline
\classe{名词}\ton{LM}
\paradigme{\pcmn{:} \p{}}
\begin{définition}\peng{Gun; firelock; rifle.}\end{définition}
\begin{définition}\pcmn{枪,明火枪}\end{définition}
\begin{définition}\pfra{Arme à feu, fusil; arquebuse.}\end{définition}
\end{entrée}

\begin{entrée}
{tʰo˧ɻæ˥}{}{ⓔtʰo˧ɻæ˥}\formedesurface{tʰo˧ɻæ˥}\newline
\classe{名词}\ton{H\#}
\paradigme{\pcmn{:} \p{}}
\begin{définition}\peng{Pine-nut kernel.}\end{définition}
\begin{définition}\pcmn{松子}\end{définition}
\begin{définition}\pfra{Pignon de pin (graine comestible).}\end{définition}
\end{entrée}

\begin{entrée}
{tʰo˩ʁæ˩}{}{ⓔtʰo˩ʁæ˩}\formedesurface{tʰo˩ʁæ˩˥}\newline
\classe{名词}\ton{L}
\paradigme{\pcmn{:} \p{}}
\begin{définition}\peng{Pine resin; colophony.}\end{définition}
\begin{définition}\pcmn{松香}\end{définition}
\begin{définition}\pfra{Résine de pin.}\end{définition}
\end{entrée}

\begin{entrée}
{tʰo˩ʂv̩˩}{}{ⓔtʰo˩ʂv̩˩}\formedesurface{tʰo˩ʂv̩˩˥}\newline
\classe{名词}\ton{L}
\paradigme{\pcmn{:} \p{}}
\begin{définition}\peng{Pine needles.}\end{définition}
\begin{définition}\pcmn{树针}\end{définition}
\begin{définition}\pfra{Aiguilles de pin.}\end{définition}
\end{entrée}

\begin{entrée}
{tʰo˩tɕi˧˥}{}{ⓔtʰo˩tɕi˧˥}\formedesurface{tʰo˩tɕi˧˥}\newline
\classe{名词}\ton{LM+MH\#}
\paradigme{\pcmn{:} \p{}}
\begin{définition}\peng{Brick.}\end{définition}
\begin{définition}\pcmn{砖}\end{définition}
\begin{définition}\pfra{Brique à l'ancienne: brique crue.}\end{définition}
\end{entrée}

\begin{entrée}
{tʰo˧tɕo˧}{}{ⓔtʰo˧tɕo˧}\formedesurface{tʰo˧tɕo˧}\newline
\classe{助词}\ton{M}\begin{définition}\peng{Backward, to the back.}\end{définition}
\begin{définition}\pcmn{往后}\end{définition}
\begin{définition}\pfra{Vers l'arrière.}\end{définition}
\begin{exemple}\pnru{tʰo˧tɕo˧ li˧}\hspace{5pt}\peng{to look back}\hspace{5pt}\pcmn{往后看}\hspace{5pt}\pfra{regarder derrière (soi)}\end{exemple}
\begin{exemple}\pnru{tʰo˧tɕo˧ ɖɯ˧-li˧-ɻ̍˧}\hspace{5pt}\peng{to glance backward}\hspace{5pt}\pcmn{往后看一眼}\hspace{5pt}\pfra{jeter un coup d'œil en arrière}\end{exemple}
\end{entrée}

\begin{entrée}
{tʰo˧tsʰe˧-ʁwɤ\#˥}{}{ⓔtʰo˧tsʰe˧-ʁwɤ\#˥}\formedesurface{tʰo˧tsʰe˧ʁwɤ˧}\newline
\classe{名词}\ton{\#H}\begin{définition}\peng{A village close to the Hot Springs.}\end{définition}
\begin{définition}\pcmn{拖其村:温泉乡的一个村落}\end{définition}
\begin{définition}\pfra{Un village proche des Sources Chaudes.}\end{définition}
\begin{exemple}\pnru{tʰo˧tsʰe\#˥}\hspace{5pt}\peng{same meaning}\hspace{5pt}\pcmn{同上}\hspace{5pt}\pfra{même sens}\end{exemple}
\begin{exemple}\pnru{ə˧go˧-ʁwɤ˧, | ʁwɤ˧lɑ˩-bi˩, | bæ˧ʁwɤ˧, | tʰo˧tsʰe\#˥, | pi˧tsʰe˩-di˩, | pɤ˧dʑɤ˩-di˩, | ʁwɤ˧tv̩˧}\hspace{5pt}\peng{Seven villages that one encounters as one leaves the plain of Yongning (towards the Lake); the first two are perceived as villages with a high proportion of Na members, and the third as a mostly Na village, whereas the next two are Pumi (Prinmi); the last used to be predominantly Pumi, but as of the 2010s, it had an important Chinese (Han) population.}\hspace{5pt}\pcmn{永宁背向泸沽湖方向经过的七个村落:阿公瓦、瓦拉比、巴瓦、拖其、比其地、巴甲地、瓦都。前两个村落拥有相当大的摩梭人口比例,第三主要是摩梭村。拖其、比其地、巴甲地是普米村。瓦都,过去主要是普米族村,到了2010年代有了相当多的汉族人口。}\hspace{5pt}\pfra{Sept villages au sortir de la plaine de Yongning, dans la direction du Lac; les deux premiers comportent une population na; le troisième est un village na; les deux suivants sont essentiellement des villages pumi/prinmi; le dernier était un village pumi, et a désormais (dans les années 2010) une importante population chinoise (han).}\end{exemple}
\begin{exemple}\pnru{tʰo˧tsʰe˧: | bɤ˧!}\hspace{5pt}\peng{/tʰo˧tsʰe˧/ is a Pumi village!}\hspace{5pt}\pcmn{拖其村是一个普米族村落!}\hspace{5pt}\pfra{/tʰo˧tsʰe˧/, c'est un village pumi!}\end{exemple}
\end{entrée}

\begin{entrée}
{tʰo˧ʈɯ\#˥}{}{ⓔtʰo˧ʈɯ\#˥}\formedesurface{tʰo˧ʈɯ˧}\newline
\classe{名词}\ton{\#H}\begin{définition}\peng{A village in Yongning; Chinese: Tuozhikaiji.}\end{définition}
\begin{définition}\pcmn{拖支开基村(永宁的一个村落)}\end{définition}
\begin{définition}\pfra{Un village de Yongning: Tuozhikaiji.}\end{définition}
\begin{exemple}\pnru{dʑɤ˩bv̩˧kɤ˧-sɑ˥ʁwɤ˩, | hi˩ʁwɤ˩-lo˥, | æ˩mi˧-ʁwɤ\#˥, | lɑ˧lo˧-ʁwɤ˥, | lɑ˧ŋwɤ˧, | bɤ˧tsʰo˧gv̩˥, | ə˧lɑ˧-ʁwɤ\#˥, | gæ˧ɻæ˩, | qʰæ˧tɕʰi˧, | tʰo˧ʈɯ\#˥}\hspace{5pt}\peng{The ten Na villages considered in traditional geography as belonging to the vicinity of the Yongning temple.}\hspace{5pt}\pcmn{永宁摩梭地理概念中,距离扎美寺最近的十个村落:佳部嘎萨瓦、习瓦洛、阿咪瓦、拉洛瓦、拉瓦、巴搓古、阿拉瓦、嘎尔、开基、拖支。}\hspace{5pt}\pfra{Les dix villages na traditionnellement considérés comme appartenant au voisinage du temple de Yongning.}\end{exemple}
\end{entrée}

\begin{entrée}
{tʰo˧ʐv̩˥}{}{ⓔtʰo˧ʐv̩˥}\formedesurface{tʰo˧ʐv̩˥}\newline
\classe{名词}\ton{H\#}
\paradigme{\pcmn{:} \p{}}
\begin{définition}\peng{Pigeon.}\end{définition}
\begin{définition}\pcmn{鸽子}\end{définition}
\begin{définition}\pfra{Pigeon.}\end{définition}
\begin{exemple}\pnru{tʰo˧ʐv̩˥-mi˩}\hspace{5pt}\peng{female pigeon}\hspace{5pt}\pcmn{母鸽子}\hspace{5pt}\pfra{pigeon femelle}\end{exemple}
\begin{exemple}\pnru{tʰo˧ʐv̩˥-pʰv̩˩}\hspace{5pt}\peng{male pigeon}\hspace{5pt}\pcmn{公鸽子}\hspace{5pt}\pfra{pigeon mâle}\end{exemple}
\begin{exemple}\pnru{tʰo˧ʐv̩˥-zo˩}\hspace{5pt}\peng{baby pigeon}\hspace{5pt}\pcmn{小鸽子}\hspace{5pt}\pfra{petit pigeon}\end{exemple}
\end{entrée}

\begin{entrée}
{tʰv̩˥}{₁}{ⓔtʰv̩˥ⓗ1}\formedesurface{tʰv̩˧}\newline
\classe{代词}\ton{\#H}
1\begin{définition}\peng{That; distal demonstrative.}\end{définition}
\begin{définition}\pcmn{那指示.远指}\end{définition}
\begin{définition}\pfra{Démonstratif distal, qui forme un couple avec le démonstratif proximal.}\end{définition}
\begin{exemple}\pnru{tʰv̩˧ ɲi˥!}\hspace{5pt}\peng{It's that one! / That's the one!}\hspace{5pt}\pcmn{是那个!}\hspace{5pt}\pfra{c'est celui-là!}\end{exemple}
\begin{exemple}\pnru{tʰv̩˧-v̩\#˥}\hspace{5pt}\peng{that one}\hspace{5pt}\pcmn{那个}\hspace{5pt}\pfra{celui-là (|fg{dem.dist}-|fg{clf.individu})}\end{exemple}
\end{entrée}

\begin{entrée}
{tʰv̩˥}{₂}{ⓔtʰv̩˥ⓗ2}\formedesurface{tʰv̩˧}\newline
\classe{代词}\ton{\#H}
2\begin{définition}\peng{3rd person singular.}\end{définition}
\begin{définition}\pcmn{他}\end{définition}
\begin{définition}\pfra{Pronom de troisième personne du singulier; provient du démonstratif distal.}\end{définition}
\begin{exemple}\pnru{tʰv̩˧=ɻ̍˩}\hspace{5pt}\peng{his family, his household, his clan, his kin}\hspace{5pt}\pcmn{他家、他家族、他的人}\hspace{5pt}\pfra{sa famille, sa maisonnée, les siens}\end{exemple}
\end{entrée}

\begin{entrée}
{-tʰv̩˥}{₃}{ⓔ-tʰv̩˥ⓗ3}\formedesurface{tʰv̩˧}\newline
\classe{后缀}\ton{\#H}
3\begin{définition}\peng{Topic marker; grammaticalized from the distal demonstrative.}\end{définition}
\begin{définition}\pcmn{主题(指示.远指)}\end{définition}
\begin{définition}\pfra{Focalisateur; grammaticalisé à partir du démonstratif distal.}\end{définition}
\end{entrée}

\begin{entrée}
{‑tʰv̩˧}{₁}{ⓔ‑tʰv̩˧ⓗ1}\formedesurface{tʰv̩˧}\newline
\classe{}\ton{M}
1\begin{définition}\peng{Temporal postposition: up to, up until.}\end{définition}
\begin{définition}\pcmn{到……为止}\end{définition}
\begin{définition}\pfra{Postposition temporelle: jusqu'à.}\end{définition}
\end{entrée}

\begin{entrée}
{‑tʰv̩˧}{₂}{ⓔ‑tʰv̩˧ⓗ2}\formedesurface{tʰv̩˧}\newline
\classe{后缀}\ton{M}
2\begin{définition}\peng{To achieve, to attain (a goal), to complete successfully (an action); grammaticalized from the verb ‘to come out'.}\end{définition}
\begin{définition}\pcmn{……成}\end{définition}
\begin{définition}\pfra{Parvenir à, réussir à, réaliser avec succès; grammaticalisé à partir du verbe ‘sortir'.}\end{définition}
\begin{exemple}\pnru{lo˧ ʝi˧-mɤ˧-tʰv̩˧}\hspace{5pt}\peng{not to be able to complete one's task, to be unable to do one's work fully (example: someone is constantly being disturbed, and consequently can't achieve what they wanted to/can't work in a focused way)}\hspace{5pt}\pcmn{活做不出来、活做不成(比如:一个人经常被打扰,所以不能集中工作,没有效率,要做的事做不成)}\hspace{5pt}\pfra{ne pas parvenir à venir à bien d'une tâche; ex.: une personne est constamment dérangée et ne parvient pas à travailler de façon concentrée}\end{exemple}
\end{entrée}

\begin{entrée}
{tʰv̩˧˥}{₁}{ⓔtʰv̩˧˥ⓗ1}\formedesurface{tʰv̩˧˥}\newline
\classe{动词}\ton{MH}
1\begin{définition}\peng{To step on, to tread on, to trample.}\end{définition}
\begin{définition}\pcmn{踩}\end{définition}
\begin{définition}\pfra{Fouler du pied, marcher sur, écraser.}\end{définition}
\begin{exemple}\pnru{tʰv̩˩∼tʰv̩˧˥}\hspace{5pt}\peng{|fg{red}}\hspace{5pt}\pcmn{重叠}\hspace{5pt}\pfra{|fg{red}}\end{exemple}
\begin{exemple}\pnru{ɖɯ˧-tʰv̩˧ tʰi˥-tʰv̩˩}\hspace{5pt}\peng{to give a kick, to stamp the ground}\hspace{5pt}\pcmn{踢一脚}\hspace{5pt}\pfra{donner un coup de pied par terre, fouler le sol du pied}\end{exemple}
\begin{exemple}\pnru{kʰɯ˧tsʰɤ˧ tʰv̩˥-tsʰɯ˩}\hspace{5pt}\peng{to give a kick, to stamp the ground}\hspace{5pt}\pcmn{踢一脚}\hspace{5pt}\pfra{donner un coup de pied par terre, fouler le sol du pied}\end{exemple}
\begin{exemple}\pnru{kʰɯ˧tsʰɤ˧ tʰv̩˥∼tʰv̩˩}\hspace{5pt}\peng{to give a kick, to stamp the ground}\hspace{5pt}\pcmn{踢一脚}\hspace{5pt}\pfra{donner un coup de pied par terre, fouler le sol du pied}\end{exemple}
\begin{exemple}\pnru{kʰɯ˧tsʰɤ˧ tʰɑ˧-tʰv̩˧˥!}\hspace{5pt}\peng{Do not stamp the ground! / Do not kick/tread on something!}\hspace{5pt}\pcmn{别踢!}\hspace{5pt}\pfra{Ne donne pas de coup de pied!}\end{exemple}
\end{entrée}

\begin{entrée}
{tʰv̩˧˥}{₂}{ⓔtʰv̩˧˥ⓗ2}\formedesurface{tʰv̩˧˥}\newline
\classe{动词}\ton{MH}
2\begin{définition}\peng{To take charge of, to foot the bill (e.g. someone invites the whole village to a feast; that person provides the food, but does not necessarily do the cooking).}\end{définition}
\begin{définition}\pcmn{负担(某个活动的费用,如:请全村人吃饭)}\end{définition}
\begin{définition}\pfra{Se charger de, préparer, offrir (quelqu'un se charge d'offrir un repas aux gens du village; c'est lui qui paie, pas forcément qui fait la cuisine).}\end{définition}
\end{entrée}

\begin{entrée}
{tʰv̩˧˥α}{}{ⓔtʰv̩˧˥α}\formedesurface{ɖɯ˧ tʰv̩˧˥}\newline
\classe{量词}\ton{MHα}\begin{définition}\peng{Classifier for steps (in walking).}\end{définition}
\begin{définition}\pcmn{量词:步}\end{définition}
\begin{définition}\pfra{Pas, enjambée.}\end{définition}
\begin{exemple}\pnru{ɖɯ˧-tʰv̩˧∼ɖɯ˥-tʰv̩˩}\hspace{5pt}\peng{step by step, one step after the other}\hspace{5pt}\pcmn{一步一步}\hspace{5pt}\pfra{pas à pas}\end{exemple}
\begin{exemple}\pnru{ɖɯ˧-tʰv̩˧˥, | ɖɯ˧-tʰv̩˧˥}\hspace{5pt}\peng{step by step, one step after the other; same as above, but detaching the two parts of the phrase; this is closer to repetition than to reduplication}\hspace{5pt}\pcmn{一步又一步}\hspace{5pt}\pfra{idem, détachant les deux parties; cette forme est plus proche d'une répétition que d'une réduplication}\end{exemple}
\end{entrée}

\begin{entrée}
{tʰv̩˧α}{₁}{ⓔtʰv̩˧αⓗ1}\formedesurface{tʰv̩˧}\newline
\classe{动词}
1
\sens{1}
\begin{définition}\peng{To come out, to emerge.}\end{définition}
\begin{définition}\pcmn{出来}\end{définition}
\begin{définition}\pfra{Sortir.}\end{définition}
\begin{exemple}\pnru{ɑ˩pʰo˩ tʰv̩˩˥}\hspace{5pt}\peng{to come out: e.g. an animal comes out of its burrow}\hspace{5pt}\pcmn{出来,如:动物从地洞里爬出来}\hspace{5pt}\pfra{sortir, ex.: un animal sort de son terrier}\end{exemple}
\begin{exemple}\pnru{ɲi˧mi˧ tʰv̩˧}\hspace{5pt}\peng{the sun comes out}\hspace{5pt}\pcmn{太阳出来}\hspace{5pt}\pfra{le soleil paraît}\end{exemple}\sens{2}
\begin{définition}\peng{To rise (wind).}\end{définition}
\begin{définition}\pcmn{刮(风)}\end{définition}
\begin{définition}\pfra{Souffler (vent).}\end{définition}\sens{3}
\begin{définition}\peng{To bud, to sprout (a tree sprouts).}\end{définition}
\begin{définition}\pcmn{发芽、抽芽}\end{définition}
\begin{définition}\pfra{Germer, bourgeonner, donner des bourgeons.}\end{définition}
\begin{exemple}\pnru{si˧dzi˩ | ʁo˧bv̩˧ tʰv̩˧}\hspace{5pt}\peng{the tree buds}\hspace{5pt}\pcmn{树抽芽}\hspace{5pt}\pfra{l'arbre fait des bourgeons}\end{exemple}\sens{4}
\begin{définition}\peng{To appear, to happen, to get (a wound).}\end{définition}
\begin{définition}\pcmn{出现}\end{définition}
\begin{définition}\pfra{Apparaître, se faire: une blessure apparaît, on reçoit une blessure.}\end{définition}
\begin{exemple}\pnru{mi˧ tʰv̩˧}\hspace{5pt}\peng{to get wounded}\hspace{5pt}\pcmn{受伤}\hspace{5pt}\pfra{se faire une blessure/avoir une blessure/se blesser}\end{exemple}
\begin{exemple}\pnru{ɖɯ˧-v̩˧ mi˧ tʰv̩˧-ze˧!}\hspace{5pt}\peng{someone has got wounded!}\hspace{5pt}\pcmn{有人受伤了!}\hspace{5pt}\pfra{quelqu'un s'est blessé!}\end{exemple}
\end{entrée}

\begin{entrée}
{tʰv̩˧α}{₂}{ⓔtʰv̩˧αⓗ2}\formedesurface{tʰv̩˧}\newline
\classe{动词}\ton{Mα}
2\begin{définition}\peng{To create; to found.}\end{définition}
\begin{définition}\pcmn{建立、创造、制造出来}\end{définition}
\begin{définition}\pfra{Créer, fonder; se trouver, se fabriquer.}\end{définition}
\begin{exemple}\pnru{ʑi˩ tʰv̩˩}\hspace{5pt}\peng{to found a new home}\hspace{5pt}\pcmn{分家、建立新家}\hspace{5pt}\pfra{créer une nouvelle maison, fonder une nouvelle maison; traduit en chinois par 分家, concept en fait assez différent dans la mesure où |fv{ʑi˩ tʰv̩˩} évoque un essaimage, plutôt qu'une séparation.}\end{exemple}
\begin{exemple}\pnru{ʈʂʰɯ˧ | ʑi˩ tʰv̩˩-ze˥!}\hspace{5pt}\peng{(S)he founded a new home!}\hspace{5pt}\pcmn{他建了新家!}\hspace{5pt}\pfra{Il/elle a fondé sa propre maisonnée!}\end{exemple}
\begin{exemple}\pnru{ʈʂʰɯ˧ | ʑi˩ tʰv̩˩-bi˩˥!}\hspace{5pt}\peng{(S)he is going to found a new home!}\hspace{5pt}\pcmn{他要建个新家!}\hspace{5pt}\pfra{Il/elle va fonder sa propre maisonnée!}\end{exemple}
\begin{exemple}\pnru{lo˧ mɤ˧-dʑo˧, | lo˧ tʰv̩˧˥! / no˧ | lo˧ mɤ˧-dʑo˧, | lo˧ tʰv̩˧-ɲi˥!}\hspace{5pt}\peng{[(S)he] has no obligations, and yet (s)he works a lot / (s)he finds tasks to do! (A compliment to a civil servant who could be content to pocket a salary every month but who sets goals for her/himself and looks for useful tasks to accomplish. The sentence can also be used negatively, to criticize someone who takes up unnecessary tasks instead of keeping quiet.)}\hspace{5pt}\pcmn{自找麻烦!(这句,除贬义用法,还能用来表扬,如表扬一位当官的人努力去做好事,给自己找有意义的事情干。)}\hspace{5pt}\pfra{Il n'a pas d'obligations, et pourtant il travaille! (Compliment à l'endroit d'un fonctionnaire qui pourrait se contenter de percevoir son salaire, mais qui se donne à lui-même des objectifs et des tâches à accomplir. La phrase peut également être employée de façon négative, pour critiquer quelqu'un qui déploie une activité inutile au lieu de se tenir tranquille.)}\end{exemple}
\end{entrée}

\begin{entrée}
{tʰv̩˧β}{}{ⓔtʰv̩˧β}\formedesurface{tʰv̩˧}\newline
\classe{动词}\ton{Mβ}\begin{définition}\peng{To lend.}\end{définition}
\begin{définition}\pcmn{借给人}\end{définition}
\begin{définition}\pfra{Prêter (un objet).}\end{définition}
\begin{exemple}\pnru{tso˧∼tso˧ tʰv̩˧}\hspace{5pt}\peng{to lend something}\hspace{5pt}\pcmn{借东西(给人)}\hspace{5pt}\pfra{prêter quelque chose}\end{exemple}
\end{entrée}

\begin{entrée}
{tʰv̩˩β}{}{ⓔtʰv̩˩β}\formedesurface{ɖɯ˧ tʰv̩˩}\newline
\classe{量词}\ton{Lβ}\begin{définition}\peng{Classifier for sets of tens. The term used to refer to sets of eight. The word remains in use, but its meaning has shifted towards the meaning of ‘sets of ten', following the generalized use of decimal numeration.}\end{définition}
\begin{définition}\pcmn{量词:一套(有十个)。更早的意思是八个。}\end{définition}
\begin{définition}\pfra{Classificateur des dizaines. Autrefois, le terme servait à compter par ensembles de 8. Le terme a demeuré, mais son sens s'est déplacé vers le sens de ‘dizaine', suivant la généralisation du système de numération à base dix.}\end{définition}
\begin{exemple}\pnru{qʰwɤ˩˥ | ɖɯ˧-tʰv̩˩}\hspace{5pt}\peng{a set of ten bowls}\hspace{5pt}\pcmn{一套十个碗}\hspace{5pt}\pfra{un lot de dix bols}\end{exemple}
\begin{exemple}\pnru{ɖʐɯ˧ʂɯ˥ | ɖɯ˧-tʰv̩˩}\hspace{5pt}\peng{a set of ten (pairs of) chopsticks}\hspace{5pt}\pcmn{一套十(双)筷子}\hspace{5pt}\pfra{un paquet de dix (paires de) baguettes}\end{exemple}
\end{entrée}

\begin{entrée}
{tʰv̩˧gi˧}{}{ⓔtʰv̩˧gi˧}\formedesurface{tʰv̩˧gi˧}\newline
\classe{助词}\ton{M}\begin{définition}\peng{In that direction.}\end{définition}
\begin{définition}\pcmn{那边}\end{définition}
\begin{définition}\pfra{Là-bas, de ce côté-là.}\end{définition}
\end{entrée}

\begin{entrée}
{tʰv̩˧ne˧-ʝi˥}{}{ⓔtʰv̩˧ne˧-ʝi˥}\formedesurface{tʰv̩˧ne˧ʝi˥}\newline
\classe{助词}\ton{H\#}\begin{définition}\peng{In that way.}\end{définition}
\begin{définition}\pcmn{那样}\end{définition}
\begin{définition}\pfra{Ainsi, de cette façon (adverbe de manière), contenant le démonstratif distal.}\end{définition}
\end{entrée}

\begin{entrée}
{tʰv̩˧ɲi\#˥}{}{ⓔtʰv̩˧ɲi\#˥}\formedesurface{tʰv̩˧ɲi˧}\newline
\classe{助词}\ton{\#H}\begin{définition}\peng{That day.}\end{définition}
\begin{définition}\pcmn{那天}\end{définition}
\begin{définition}\pfra{Ce jour-là (déictique lointain).}\end{définition}
\end{entrée}

\begin{entrée}
{tʰv̩˧qo˧}{}{ⓔtʰv̩˧qo˧}\formedesurface{tʰv̩˧qo˧}\newline
\classe{代词}\ton{M}\begin{définition}\peng{There; that place.}\end{définition}
\begin{définition}\pcmn{那里、那个地方}\end{définition}
\begin{définition}\pfra{Là-bas; cet endroit-là.}\end{définition}
\end{entrée}

\begin{entrée}
{tʰv̩˧-se˩-gɤ˩}{}{ⓔtʰv̩˧-se˩-gɤ˩}\formedesurface{tʰv̩˧se˩gɤ˩}\newline
\classe{助词}\ton{-L}\begin{définition}\peng{Henceforth.}\end{définition}
\begin{définition}\pcmn{今后、从此、此后}\end{définition}
\begin{définition}\pfra{Désormais, dorénavant.}\end{définition}
\end{entrée}

\begin{entrée}
{tʰv̩˧-si˥}{}{ⓔtʰv̩˧-si˥}\formedesurface{tʰv̩˧si˥}\newline
\classe{助词}\ton{H\#}\begin{définition}\peng{Numerous.}\end{définition}
\begin{définition}\pcmn{多}\end{définition}
\begin{définition}\pfra{Nombreux.}\end{définition}
\begin{exemple}\pnru{mv̩˧ʁo˧=ɻ̍˥-dʑo˩, | ɻæ˩˥ | tʰv̩˧-si˥ | tʰv̩˩-jɤ˩ dʑo˩˥!}\hspace{5pt}\peng{The People of the Sky had seeds in profusion!}\hspace{5pt}\pcmn{天上的人,有许多许多种子!}\hspace{5pt}\pfra{Les gens du Ciel, ils avaient des semences en abondance!}\end{exemple}
\end{entrée}

\begin{entrée}
{tʰv̩˧-sɯ˩kv̩˩}{}{ⓔtʰv̩˧-sɯ˩kv̩˩}\formedesurface{tʰv̩˧sɯ˩kv̩˩}\newline
\classe{代词}\ton{-L}\begin{définition}\peng{Third-person plural pronoun.}\end{définition}
\begin{définition}\pcmn{他们}\end{définition}
\begin{définition}\pfra{Pronom de troisième personne du pluriel.}\end{définition}
\end{entrée}

\begin{entrée}
{tɕæ˧hæ˩}{}{ⓔtɕæ˧hæ˩}\formedesurface{tɕæ˧hæ˩}\newline
\classe{名词}\ton{L\#}
\paradigme{\pcmn{:} \p{}}
\begin{définition}\peng{Rubber.}\end{définition}
\begin{définition}\pcmn{橡胶(汉语借词。第二个音节:未确定。)}\end{définition}
\begin{définition}\pfra{Caoutchouc.}\end{définition}
\begin{exemple}\pnru{tɕæ˧hæ˩-dzɑ˩qʰwɤ˩}\hspace{5pt}\peng{rubber shoe, shoe with a rubber sole, sports shoe}\hspace{5pt}\pcmn{橡胶鞋、橡胶底鞋}\hspace{5pt}\pfra{chaussures à semelle en gomme/en caoutchouc; baskets}\end{exemple}
\end{entrée}

\begin{entrée}
{tɕæ˧pʰv̩˩}{}{ⓔtɕæ˧pʰv̩˩}\formedesurface{tɕæ˧pʰv̩˩}\newline
\classe{形容词}\ton{L\#}\begin{définition}\peng{White.}\end{définition}
\begin{définition}\pcmn{白(脸、衣服)}\end{définition}
\begin{définition}\pfra{Blanc (visage, habits, cheveux…).}\end{définition}
\begin{exemple}\pnru{tɕæ˧pʰv̩˩-bɑ˩lɑ˩}\hspace{5pt}\peng{white clothes}\hspace{5pt}\pcmn{白的衣服}\hspace{5pt}\pfra{vêtement blanc}\end{exemple}
\begin{exemple}\pnru{tɕæ˧pʰv̩˩-ʈæ˩qʰwɤ˩}\hspace{5pt}\peng{white skirt}\hspace{5pt}\pcmn{白色裙子}\hspace{5pt}\pfra{robe blanche}\end{exemple}
\end{entrée}

\begin{entrée}
{tɕæ˧ɻæ˩}{}{ⓔtɕæ˧ɻæ˩}\formedesurface{tɕæ˧ɻæ˩}\newline
\classe{名词}\ton{L\#}\begin{définition}\peng{Pickled vegetables.}\end{définition}
\begin{définition}\pcmn{酸菜、泡菜}\end{définition}
\begin{définition}\pfra{Légumes en saumure. On en mangeait une sorte chaque jour pendant la saison d'hiver: un jour navet en saumure, etc.}\end{définition}
\begin{exemple}\pnru{wo˩-tɕæ˩ɻæ˥}\hspace{5pt}\peng{pickled turnip leaves}\hspace{5pt}\pcmn{圆根叶子酸菜}\hspace{5pt}\pfra{feuilles de navet conservées dans la saumure}\end{exemple}
\begin{exemple}\pnru{tsʰɑ˧-tɕæ˧ɻæ˥}\end{exemple}
\begin{exemple}\pnru{ɬi˩bi˩-tɕæ˩ɻæ˥}\hspace{5pt}\peng{pickled turnip}\hspace{5pt}\pcmn{圆根酸菜}\hspace{5pt}\pfra{navet conservé dans la saumure}\end{exemple}
\begin{exemple}\pnru{pɤ˧pɤ˧tsʰɯ˧-tɕæ˧ɻæ˥}\hspace{5pt}\peng{picked Chinese cabbage}\hspace{5pt}\pcmn{圆白菜酸菜}\hspace{5pt}\pfra{chou chinois en saumure}\end{exemple}
\end{entrée}

\begin{entrée}
{tɕæ˧tsʰe˩}{}{ⓔtɕæ˧tsʰe˩}\formedesurface{tɕæ˧tsʰe˩}\newline
\classe{名词}\ton{L\#}\begin{définition}\peng{Jiaze, a hamlet to the north of Labai}\end{définition}
\begin{définition}\pcmn{拉伯乡加泽村(汉语借词)}\end{définition}
\begin{définition}\pfra{Jiaze, un hameau au nord de Labai}\end{définition}
\end{entrée}

\begin{entrée}
{tɕɤ}{}{ⓔtɕɤ}\formedesurface{tɕɤ!}\newline
\classe{感叹词}\ton{0}\begin{définition}\peng{Interjection: hey!}\end{définition}
\begin{définition}\pcmn{感叹词:嘿!}\end{définition}
\begin{définition}\pfra{Interjection: tiens! eh!}\end{définition}
\end{entrée}

\begin{entrée}
{tɕɤ˥}{}{ⓔtɕɤ˥}\formedesurface{tɕɤ˧}\newline
\classe{动词}\ton{H}\begin{définition}\peng{To fade (of colours).}\end{définition}
\begin{définition}\pcmn{褪色}\end{définition}
\begin{définition}\pfra{S'effacer (couleur).}\end{définition}
\begin{exemple}\pnru{le˧-tɕɤ˥-ze˩}\hspace{5pt}\peng{|fg{accomp} \_ |fg{pfv}}\hspace{5pt}\pcmn{褪色了}\hspace{5pt}\pfra{|fg{accomp} \_ |fg{pfv}}\end{exemple}
\end{entrée}

\begin{entrée}
{tɕɤ˧˥}{}{ⓔtɕɤ˧˥}\formedesurface{tɕɤ˧˥}\newline
\classe{动词}\ton{MH}\begin{définition}\peng{To boil, to cook thoroughly; to cook in a pot.}\end{définition}
\begin{définition}\pcmn{煮}\end{définition}
\begin{définition}\pfra{Bouillir; cuire en faisant bouillir; cuire dans une casserole.}\end{définition}
\begin{exemple}\pnru{ʂe˧ tɕɤ˩}\hspace{5pt}\peng{to boil meat}\hspace{5pt}\pcmn{煮肉}\hspace{5pt}\pfra{faire bouillir de la viande, faire cuire de la viande à l'eau}\end{exemple}
\begin{exemple}\pnru{bo˩-hɑ˧ tɕɤ˩}\hspace{5pt}\peng{to boil pigswill, to cook pigswill}\hspace{5pt}\pcmn{煮猪食}\hspace{5pt}\pfra{faire bouillir la pâtée des cochons}\end{exemple}
\begin{exemple}\pnru{ho˧ tɕɤ˩}\hspace{5pt}\peng{to cook stew}\hspace{5pt}\pcmn{煮粥}\hspace{5pt}\pfra{faire du ragoût}\end{exemple}
\begin{exemple}\pnru{dʑɯ˩ʁo˩˥, | mo˧-no˥, | mo˧ tɕɤ˥-hĩ˩ lɑ˩-ɲi˩-mæ˩! |}\hspace{5pt}\peng{Up on the mountain, to cook mushrooms, (we) simply cook them in a pot! (This does not refer to boiling in the sense of ‘cooking in hot water': the mushrooms are put in a pot; one adds grease and salt, and the mushrooms cook in their own juice.)}\hspace{5pt}\pcmn{在山上,菌子,就是简单煮一下而已!(放在锅里,加油、加盐。用菌子自身的水分)}\hspace{5pt}\pfra{(Quand on se trouve sur) la montagne, les champignons, on les fait simplement cuire dans une casserole! (Littéralement: «on se contente de les faire bouillir».) (On mettait simplement les champignons dans une casserole, sans eau, avec du sel et de la graisse; les champignons cuisaient alors dans leur propre eau.)}\end{exemple}
\end{entrée}

\begin{entrée}
{tɕɤ˩}{}{ⓔtɕɤ˩}\formedesurface{tɕɤ˩˥}\newline
\classe{动词}\ton{L}\begin{définition}\peng{To bind together.}\end{définition}
\begin{définition}\pcmn{打结、系上}\end{définition}
\begin{définition}\pfra{Attacher (ex.: un joug sur une vache; des troncs…).}\end{définition}
\begin{exemple}\pnru{ʁæ˧ɻ̍˥ | tʰi˧-tɕɤ˩}\hspace{5pt}\peng{to attach the yoke (to a buffalo)}\hspace{5pt}\pcmn{系上牛轭}\hspace{5pt}\pfra{fixer (un joug sur un buffle)}\end{exemple}
\end{entrée}

\begin{entrée}
{tɕɤ˧fv̩˩}{}{ⓔtɕɤ˧fv̩˩}\formedesurface{tɕɤ˧fv̩˩}\newline
\classe{名词}\ton{L\#}
\paradigme{\pcmn{:} \p{}}
\begin{définition}\peng{Container for liquids, such as plastic jerricans; used to store and transport drinking water.}\end{définition}
\begin{définition}\pcmn{塑料桶等存水用的容器}\end{définition}
\begin{définition}\pfra{Container pour liquides; s'emploie pour désigner les containers en matière plastique.}\end{définition}
\end{entrée}

\begin{entrée}
{tɕɤ˧ho˩pæ˧}{}{ⓔtɕɤ˧ho˩pæ˧}\formedesurface{tɕɤ˧ho˩pæ˧}\newline
\classe{名词}\ton{MLM}\begin{définition}\peng{Plywood, veneer board.}\end{définition}
\begin{définition}\pcmn{胶合板(汉语借词)}\end{définition}
\begin{définition}\pfra{Contreplaqué, panneau en contreplaqué.}\end{définition}
\end{entrée}

\begin{entrée}
{tɕɤ˩ho˩tsɯ˥}{}{ⓔtɕɤ˩ho˩tsɯ˥}\formedesurface{tɕɤ˩ho˩tsɯ˥}\newline
\classe{名词}\ton{L+H\#}
\paradigme{\pcmn{:} \p{}}
\begin{définition}\peng{Swindler, cheat.}\end{définition}
\begin{définition}\pcmn{骗子}\end{définition}
\begin{définition}\pfra{Escroc.}\end{définition}
\begin{exemple}\pnru{ʈʂʰɯ˧ | hĩ˧ ʈʂʰɯ˧-v̩˧ | tɕɤ˩ho˩tsɯ˥ ɲi˩.}\hspace{5pt}\peng{This man is a swindler!}\hspace{5pt}\pcmn{这个人是骗子!}\hspace{5pt}\pfra{Cet homme, c'est un escroc!}\end{exemple}
\end{entrée}

\begin{entrée}
{tɕɤ˧jo˩}{}{ⓔtɕɤ˧jo˩}\formedesurface{tɕɤ˧jo˩}\newline
\classe{名词}\ton{L\#}\begin{définition}\peng{Prison.}\end{définition}
\begin{définition}\pcmn{监狱(汉语借词)}\end{définition}
\begin{définition}\pfra{Prison.}\end{définition}
\begin{exemple}\pnru{tɕɤ˧jo˩-qo˩ | tʰi˧-ʈæ˩}\hspace{5pt}\peng{to put into prison, to imprison}\hspace{5pt}\pcmn{关在监狱}\hspace{5pt}\pfra{enfermer en prison, mettre en prison}\end{exemple}
\begin{exemple}\pnru{tɕɤ˧jo˩-qo˩ ʈæ˩-hɯ˩-ze˩!}\hspace{5pt}\peng{(He/she) has been jailed/sent to prison!}\hspace{5pt}\pcmn{被关在监狱!}\hspace{5pt}\pfra{(On l')a jeté en prison! / (On l')a mis en prison!}\end{exemple}
\end{entrée}

\begin{entrée}
{tɕɤ˧qʰɑ\#˥}{}{ⓔtɕɤ˧qʰɑ\#˥}\formedesurface{tɕɤ˧qʰɑ˧}\newline
\classe{名词}\ton{\#H}\begin{définition}\peng{Mugwort, wormwood, |\stylefi{Artemisia vulgaris}.}\end{définition}
\begin{définition}\pcmn{蒿、青蒿}\end{définition}
\begin{définition}\pfra{Armoise, |\stylefi{Artemisia vulgaris}.}\end{définition}
\begin{exemple}\pnru{tɕɤ˧qʰɑ˧-mo˩}\hspace{5pt}\peng{a type of edible mushroom, called ‘mugwort mushroom' because it grows close to mugwort}\hspace{5pt}\pcmn{一种可以吃的菌子,长在蒿附近}\hspace{5pt}\pfra{un champignon comestible, nommé ‘champignon de l'armoise' parce qu'il croît à proximité de l'armoise}\end{exemple}
\end{entrée}

\begin{entrée}
{tɕɤ˧tɑ˧}{}{ⓔtɕɤ˧tɑ˧}\formedesurface{tɕɤ˧tɑ˧}\newline
\classe{名词}\ton{M}
\paradigme{\pcmn{:} \p{}}
\begin{définition}\peng{Yoke.}\end{définition}
\begin{définition}\pcmn{牛轭(单行)(汉语借词)}\end{définition}
\begin{définition}\pfra{Joug.}\end{définition}
\begin{exemple}\pnru{tɕɤ˧tɑ˧ tʰv̩˧-ɭɯ˧}\hspace{5pt}\peng{|fg{n}+|fg{dem}+|fg{clf}}\hspace{5pt}\pcmn{这个牛轭}\hspace{5pt}\pfra{|fg{n}+|fg{dem}+|fg{clf}}\end{exemple}
\end{entrée}

\begin{entrée}
{tɕɤ˧tɑ˧-bæ˩}{}{ⓔtɕɤ˧tɑ˧-bæ˩}\formedesurface{tɕɤ˧tɑ˧bæ˩}\newline
\classe{名词}\ton{-L}
\étymologie{
tɕɤ˧tɑ˧; bæ˩
}
\paradigme{\pcmn{:} \p{}}
\begin{définition}\peng{Tracking rope, towrope, towline.}\end{définition}
\begin{définition}\pcmn{牛皮绳 ,犁具连轭之绳}\end{définition}
\begin{définition}\pfra{Courroies entre le joug et l'araire.}\end{définition}
\begin{exemple}\pnru{tɕɤ˧tɑ˧-bæ˩ tʰv̩˩-kʰɯ˩}\hspace{5pt}\peng{|fg{n}+|fg{dem}+|fg{clf}}\hspace{5pt}\pcmn{这条牛皮绳}\hspace{5pt}\pfra{|fg{n}+|fg{dem}+|fg{clf}}\end{exemple}
\end{entrée}

\begin{entrée}
{tɕɤ˧∼tɕɤ˧}{}{ⓔtɕɤ˧∼tɕɤ˧}\formedesurface{tɕɤ˧tɕɤ˧}\newline
\classe{助词}\ton{M}\begin{définition}\peng{Just; exactly.}\end{définition}
\begin{définition}\pcmn{将将(汉语借词)、刚刚}\end{définition}
\begin{définition}\pfra{Précisément, exactement (ex.: au moment précis où, juste au moment où).}\end{définition}
\end{entrée}

\begin{entrée}
{tɕi˥}{}{ⓔtɕi˥}\formedesurface{tɕi˧}\newline
\classe{动词}\ton{H}\begin{définition}\peng{To shake (e.g. clothes after washing; to shake one's head).}\end{définition}
\begin{définition}\pcmn{抖、抖动,摇动}\end{définition}
\begin{définition}\pfra{Secouer (ex.: pour défroisser des vêtements après lavage; aussi: secouer la tête).}\end{définition}
\begin{exemple}\pnru{le˧-tɕi˧∼tɕi˧-ze˩}\hspace{5pt}\peng{|fg{accomp} \_ |fg{pfv}}\hspace{5pt}\pcmn{|fg{accomp} \_ |fg{pfv}}\hspace{5pt}\pfra{|fg{accomp} \_ |fg{pfv}}\end{exemple}
\begin{exemple}\pnru{tʰi˧-tɕi˧∼tɕi˧+ze˩}\hspace{5pt}\peng{|fg{dur} \_ |fg{pfv}}\hspace{5pt}\pcmn{|fg{dur} \_ |fg{pfv}}\hspace{5pt}\pfra{|fg{dur} \_ |fg{pfv}}\end{exemple}
\begin{exemple}\pnru{ʁo˧qʰwɤ˩ tɕi˩∼tɕi˩}\hspace{5pt}\peng{to shake one's head}\hspace{5pt}\pcmn{摇头}\hspace{5pt}\pfra{agiter la tête, secouer la tête}\end{exemple}
\begin{exemple}\pnru{ɖɯ˧-tɕi˧∼tɕi˧-ɻ̍˥}\hspace{5pt}\peng{|fg{demilitative} |fg{red} |fg{inceptive}}\hspace{5pt}\pcmn{摇一摇}\hspace{5pt}\pfra{|fg{délimitatif} \_ |fg{red} |fg{inchoatif}}\end{exemple}
\end{entrée}

\begin{entrée}
{tɕi˧}{₁}{ⓔtɕi˧ⓗ1}\formedesurface{tɕi˧}\newline
\classe{形容词}\ton{M}
1
\sens{1}
\begin{définition}\peng{Sour, acidic.}\end{définition}
\begin{définition}\pcmn{酸}\end{définition}
\begin{définition}\pfra{Acide.}\end{définition}
\begin{exemple}\pnru{tɕʰɯ˩-hĩ˩˥}\end{exemple}
\begin{exemple}\pnru{tɕi˧-hĩ˧ pʰi˩}\hspace{5pt}\peng{to have acid reflux}\hspace{5pt}\pcmn{吐酸水}\hspace{5pt}\pfra{avoir des remontées acides}\end{exemple}\sens{2}
\begin{définition}\peng{Sour, fermented.}\end{définition}
\begin{définition}\pcmn{(通过发酵的)酸}\end{définition}
\begin{définition}\pfra{Fermenté.}\end{définition}
\end{entrée}

\begin{entrée}
{tɕi˧}{₂}{ⓔtɕi˧ⓗ2}\formedesurface{tɕi˧}\newline
\classe{名词}\ton{M}
2
\paradigme{\pcmn{:} \p{}}
\begin{définition}\peng{Snare, trap, trick.}\end{définition}
\begin{définition}\pcmn{圈套}\end{définition}
\begin{définition}\pfra{Piège.}\end{définition}
\begin{exemple}\pnru{tɕi˧ kʰɯ˧˥}\hspace{5pt}\peng{to set a trap}\hspace{5pt}\pcmn{设下圈套}\hspace{5pt}\pfra{poser un piège}\end{exemple}
\end{entrée}

\begin{entrée}
{tɕi˧˥}{}{ⓔtɕi˧˥}\formedesurface{tɕi˧˥}\newline
\classe{动词}\ton{MH}\begin{définition}\peng{To invite (guests) to drink wine.}\end{définition}
\begin{définition}\pcmn{敬(酒)}\end{définition}
\begin{définition}\pfra{Inviter (des hôtes) à boire du vin.}\end{définition}
\begin{exemple}\pnru{ʐɯ˧ tɕi˧˥}\hspace{5pt}\peng{to invite (guests) to drink wine}\hspace{5pt}\pcmn{敬酒}\hspace{5pt}\pfra{inviter (des hôtes) à boire du vin}\end{exemple}
\end{entrée}

\begin{entrée}
{tɕi˧α}{}{ⓔtɕi˧α}\formedesurface{ɖɯ˧ tɕi˧}\newline
\classe{量词}\ton{Mα}\begin{définition}\peng{Some, a few.}\end{définition}
\begin{définition}\pcmn{量词:一些}\end{définition}
\begin{définition}\pfra{Quelques-uns, certains, une partie.}\end{définition}
\begin{exemple}\pnru{ɖɯ˧-tɕi˧}\hspace{5pt}\peng{some, a few}\hspace{5pt}\pcmn{一些}\hspace{5pt}\pfra{quelques-uns, certains}\end{exemple}
\begin{exemple}\pnru{ʈʂʰɯ˧-tɕi˩}\hspace{5pt}\peng{these few}\hspace{5pt}\pcmn{这些}\hspace{5pt}\pfra{ceux-ci}\end{exemple}
\end{entrée}

\begin{entrée}
{tɕi˩˥}{}{ⓔtɕi˩˥}\formedesurface{tɕi˩˥}\newline
\classe{名词}\ton{LH}
\paradigme{\pcmn{:} \p{}}
\begin{définition}\peng{Saddle.}\end{définition}
\begin{définition}\pcmn{马鞍}\end{définition}
\begin{définition}\pfra{Selle.}\end{définition}
\begin{exemple}\pnru{ʐwæ˧-tɕi˥}\hspace{5pt}\peng{horse saddle}\hspace{5pt}\pcmn{马鞍}\hspace{5pt}\pfra{selle de cheval}\end{exemple}
\end{entrée}

\begin{entrée}
{tɕi˩α}{}{ⓔtɕi˩α}\formedesurface{tɕi˩˥}\newline
\classe{形容词}\ton{Lα}\begin{définition}\peng{Small; short (not tall).}\end{définition}
\begin{définition}\pcmn{矮,低,小}\end{définition}
\begin{définition}\pfra{Petit.}\end{définition}
\begin{exemple}\pnru{tɕi˩-hĩ˩˥}\hspace{5pt}\peng{|fg{nmlz}}\hspace{5pt}\pcmn{矮的}\hspace{5pt}\pfra{(qui est) petit}\end{exemple}
\begin{exemple}\pnru{gv̩˧mi˧ tɕi˩}\hspace{5pt}\peng{short (not tall)}\hspace{5pt}\pcmn{矮}\hspace{5pt}\pfra{de petite taille}\end{exemple}
\end{entrée}

\begin{entrée}
{tɕi˧do˩}{}{ⓔtɕi˧do˩}\formedesurface{tɕi˧do˩}\newline
\classe{名词}\ton{L\#}
\paradigme{\pcmn{:} \p{}}
\begin{définition}\peng{Tangerine.}\end{définition}
\begin{définition}\pcmn{橘子}\end{définition}
\begin{définition}\pfra{Mandarine.}\end{définition}
\end{entrée}

\begin{entrée}
{tɕi˧-dʑɯ˩}{}{ⓔtɕi˧-dʑɯ˩}\newline
\classe{名词}
\sens{1}
\begin{définition}\peng{Acid potion: a preparation from sour plums or wild berries, used to make people vomit when they had food poisoning (e.g. from eating poisonous mushrooms).}\end{définition}
\begin{définition}\pcmn{用梅子等野生果子做出来的一种药品(酸水),食物中毒的情况下给病人和这种酸水让他呕吐}\end{définition}
\begin{définition}\pfra{Potion acide: une préparation à base de prunelles acides ou baies sauvages, utilisée pour faire vomir les personnes victimes d'un empoisonnement alimentaire (par exemple par des champignons vénéneux).}\end{définition}\sens{2}
\begin{définition}\peng{Vinegar.}\end{définition}
\begin{définition}\pcmn{醋}\end{définition}
\begin{définition}\pfra{Vinaigre.}\end{définition}
\end{entrée}

\begin{entrée}
{tɕi˧kwɤ˧}{}{ⓔtɕi˧kwɤ˧}\formedesurface{tɕi˧kwɤ˧}\newline
\classe{名词}\ton{M}
\paradigme{\pcmn{:} \p{}}
\begin{définition}\peng{Melon, gourd.}\end{définition}
\begin{définition}\pcmn{瓜}\end{définition}
\begin{définition}\pfra{Courge (inclut les courgettes).}\end{définition}
\begin{exemple}\pnru{tɕi˧kwɤ˧ bv̩˧-ɻ̍˧ (+ɲi˩)}\hspace{5pt}\peng{small melon}\hspace{5pt}\pcmn{小瓜}\hspace{5pt}\pfra{petite courge}\end{exemple}
\begin{exemple}\pnru{tɕi˧kwɤ˧ kwɤ˧mo˩}\hspace{5pt}\peng{large melon}\hspace{5pt}\pcmn{大瓜}\hspace{5pt}\pfra{grosse courge}\end{exemple}
\end{entrée}

\begin{entrée}
{tɕi˩nv̩˧˥}{}{ⓔtɕi˩nv̩˧˥}\formedesurface{tɕi˩nv̩˧˥}\newline
\classe{名词}\ton{LM+MH\#}
\paradigme{\pcmn{:} \p{}}
\begin{définition}\peng{Saddle mat.}\end{définition}
\begin{définition}\pcmn{马鞍下面的毯子}\end{définition}
\begin{définition}\pfra{Tapis de selle.}\end{définition}
\begin{exemple}\pnru{ʐwæ˧-tɕi˥nv̩˩}\hspace{5pt}\peng{horse saddle mat}\hspace{5pt}\pcmn{马鞍毯子}\hspace{5pt}\pfra{tapis de selle de cheval}\end{exemple}
\end{entrée}

\begin{entrée}
{tɕi˩qɑ˥}{}{ⓔtɕi˩qɑ˥}\formedesurface{tɕi˩qɑ˥}\newline
\classe{名词}\ton{LH}
\paradigme{\pcmn{:} \p{}}
\begin{définition}\peng{Carpet.}\end{définition}
\begin{définition}\pcmn{毯子}\end{définition}
\begin{définition}\pfra{Tapis.}\end{définition}
\end{entrée}

\begin{entrée}
{tɕi˧sɯ˧pɤ˧}{}{ⓔtɕi˧sɯ˧pɤ˧}\formedesurface{tɕi˧sɯ˧pɤ˧}\newline
\classe{名词}\ton{M}\begin{définition}\peng{Cheese made of yak milk. First, the milk is creamed, then boiled again, with an additive to make it curdle; finally, the preparation is left to dry and harden. It is used in cooking (some of it can be added to gruel), and also as a treatment for diaorrhea. It can keep for a long time.}\end{définition}
\begin{définition}\pcmn{牦牛奶酪}\end{définition}
\begin{définition}\pfra{Fromage au lait de yak. On commençait par écrémer le lait, puis on faisait bouillir, avec un additif pour le faire cailler; enfin la préparation se solidifiait. Cette préparation était utilisée dans l'alimentation: on en mettait dans les bouillies de céréales. Elle pouvait se conserver. Ce fromage, acide et dur, était recommandé aux personnes ayant des soucis digestifs (comme remède à la diarrhée), et aux personnes âgées.}\end{définition}
\begin{exemple}\pnru{mv̩˧ɭɯ˩-pʰɤ˩bɤ˩, | tɕi˧sɯ˧pɤ˧!}\hspace{5pt}\peng{The gift from Muli is yak cheese! / The gift that people usually bring back from their trips to Muli is yak cheese! / Yak cheese is a specialty of Muli! (Yak cheese used to be one of the delicacies that young men offered to young ladies when coming back from caravan journeys.)}\hspace{5pt}\pcmn{木里的礼物:牦牛的奶酪! / 牦牛奶酪,是木里的特产!}\hspace{5pt}\pfra{Le cadeau (qu'on ramène de Muli), c'est le fromage de yak! / La spécialité de Muli, c'est le fromage de yak! (Autrefois, c'était un des cadeaux que les jeunes gens offraient aux jeunes filles au retour de leurs voyages.)}\end{exemple}
\begin{exemple}\pnru{mv̩˧ɭɯ˩ pʰɤ˩bɤ˩, | tɕi˧sɯ˧pɤ˧! | ə˧ɖo˧ ʁo˧ ɖʐɯ˥∼ɖʐɯ˩ ʝi˩-ze˩!}\hspace{5pt}\peng{The gift from Muli is yak cheese! (My) beloved will shake her head (when tasting the delightfully acid cheese)! (Words from a song that used to be sung when travelling, imagining the return to Yongning.)}\hspace{5pt}\pcmn{(从)木里(带回来)的礼物,就是牦牛奶酪!亲爱的(=收礼物的那个人),会摇头的!(吃、喝的时候会摇头,是因为牦牛奶奶酪比较酸)}\hspace{5pt}\pfra{Le cadeau (qu'on ramène de Muli), c'est le fromage de yak! Ma bien-aimée va secouer la tête (lorsqu'elle goûtera à ce fromage, très acide)! (Paroles d'une chanson qu'on chantait en chemin, en imaginant le retour.)}\end{exemple}
\begin{exemple}\pnru{tɕi˧sɯ˧pɤ˧, | ɖɯ˧-tɑ˧˥ | gv̩˧-mɤ˧-kv̩˥! | ʝi˧-kʰv̩˥-lɑ˩ gv̩˩-kv̩˩!}\hspace{5pt}\peng{Not everyone knew how to make yak cheese! Only a few had this know-how!}\hspace{5pt}\pcmn{不是每个人都会做牦牛奶酪!只有少数(人)才会做!}\hspace{5pt}\pfra{Ce n'est pas tout le monde qui savait faire du fromage de yak! Il n'y a que certaines (personnes/familles) qui savaient le faire!}\end{exemple}
\begin{exemple}\pnru{tɕi˧sɯ˧pɤ˧-dʑɯ˩}\hspace{5pt}\peng{water in which one has diluted some yak cheese; it has medicinal properties}\hspace{5pt}\pcmn{一种饮料:将牦牛奶酪溶化在水里}\hspace{5pt}\pfra{eau dans laquelle on a dilué du fromage de yak; elle a des propriétés médicinales}\end{exemple}
\begin{exemple}\pnru{tɕi˧sɯ˧pɤ˧ ʈʰɯ˩}\hspace{5pt}\peng{to drink water in which one has diluted some yak cheese; literally: ‘to drink yak cheese'}\hspace{5pt}\pcmn{喝溶化在水里的牦牛奶酪(直译:喝牦牛奶酪)}\hspace{5pt}\pfra{boire de l'eau dans laquelle on a dilué du fromage de yak; littéralement: ‘boire du fromage de yak'}\end{exemple}
\end{entrée}

\begin{entrée}
{tɕi˧tɕi˧ | læ˩sæ˧-dzi˩}{}{ⓔtɕi˧tɕi˧ | læ˩sæ˧-dzi˩}\formedesurface{tɕi˧tɕi˧læ˩sæ˧dzi˩}\newline
\classe{名词}\ton{M | LH-L}\begin{définition}\peng{A type of hardwood (not identified yet).}\end{définition}
\begin{définition}\pcmn{一种树,木质很硬}\end{définition}
\begin{définition}\pfra{Un arbre au bois très dur.}\end{définition}
\end{entrée}

\begin{entrée}
{tɕo˥}{}{ⓔtɕo˥}\formedesurface{tɕo˧}\newline
\classe{名词}\ton{H}\begin{définition}\peng{Direction.}\end{définition}
\begin{définition}\pcmn{方向}\end{définition}
\begin{définition}\pfra{Sens, direction.}\end{définition}
\begin{exemple}\pnru{ʈʂʰɯ˧-tɕo˧}\hspace{5pt}\peng{this way}\hspace{5pt}\pcmn{这个方向,向这里}\hspace{5pt}\pfra{dans cette direction-ci}\end{exemple}
\begin{exemple}\pnru{ɖɯ˧-tɕo˥}\hspace{5pt}\peng{one side, in one direction}\hspace{5pt}\pcmn{一边}\hspace{5pt}\pfra{d'un côté, dans une direction}\end{exemple}
\begin{exemple}\pnru{gɤ˩-tɕo˧}\hspace{5pt}\peng{upward, towards the top}\hspace{5pt}\pcmn{向上,往上}\hspace{5pt}\pfra{vers le haut}\end{exemple}
\begin{exemple}\pnru{dv̩˩tɕo˧}\hspace{5pt}\peng{that way}\hspace{5pt}\pcmn{那边}\hspace{5pt}\pfra{dans cette direction-là}\end{exemple}
\end{entrée}

\begin{entrée}
{tɕo˩ɕjo˧}{}{ⓔtɕo˩ɕjo˧}\formedesurface{tɕo˩ɕjo˥}\newline
\classe{名词}\ton{LM}\begin{définition}\peng{Whistle, whistling noise.}\end{définition}
\begin{définition}\pcmn{口哨}\end{définition}
\begin{définition}\pfra{Sifflement.}\end{définition}
\begin{exemple}\pnru{tɕo˩ɕjo˧ | ɖɯ˧-ɖʐo˩ kʰɯ˩}\hspace{5pt}\peng{to whistle a little, to whistle a few notes}\hspace{5pt}\pcmn{吹口哨、吹一声口哨}\hspace{5pt}\pfra{siffler un air, siffler un coup}\end{exemple}
\end{entrée}

\begin{entrée}
{tɕo˩mv̩˧}{}{ⓔtɕo˩mv̩˧}\formedesurface{tɕo˩mv̩˥}\newline
\classe{名词}\ton{LM}
\paradigme{\pcmn{:} \p{}}
\begin{définition}\peng{Wife of maternal uncle. The word consists of a Chinese borrowing, \stylefn{舅} ‘maternal uncle', to which is added the Na word for ‘woman'.}\end{définition}
\begin{définition}\pcmn{舅妈(舅:汉语借词,妈:摩梭话“女人”)}\end{définition}
\begin{définition}\pfra{Femme de l'oncle maternel; constitué d'un emprunt chinois, \stylefn{舅} ‘oncle maternel', et d'un mot na: ‘femme'.}\end{définition}
\end{entrée}

\begin{entrée}
{‑tɕɯ}{}{ⓔ‑tɕɯ}\formedesurface{--}\newline
\classe{后缀}\ton{?}\begin{définition}\peng{Grammaticalized form of the verb \stylefv{/tɕɯ}˥/ ‘to put, to place'; expresses that the action is over and done with: that its goal has been reached, and one now moves on to something different; in the same way as, after an object has been put in the right place, one may turn one's attention to other objects.}\end{définition}
\begin{définition}\pcmn{表示:已完成,可以轮到其它的了}\end{définition}
\begin{définition}\pfra{Forme grammaticalisée de \stylefv{/tɕɯ}˥/ ‘mettre, placer'; elle exprime que l'action est finie, que son terme est maintenant dépassé, et qu'on peut passer à autre chose: de même que, une fois un objet posé à sa place, on peut tourner son attention vers un autre.}\end{définition}
\begin{exemple}\pnru{gɤ˩-ʈʂʰwæ˧-tɕɯ˧˥ / gɤ˩-ʈʂʰwæ˧-tɕɯ˧-zo˥}\hspace{5pt}\peng{to wake up, to awaken}\hspace{5pt}\pcmn{醒过来、醒来}\hspace{5pt}\pfra{être éveillé, se réveiller}\end{exemple}
\begin{exemple}\pnru{ʈʂʰɯ˧-qo˧ dzi˩-tɕɯ˩-zo˩}\hspace{5pt}\peng{to sit here}\hspace{5pt}\pcmn{在这里坐下来}\hspace{5pt}\pfra{s'asseoir ici}\end{exemple}
\begin{exemple}\pnru{ʐɯ˧ ɖɯ˧-qʰwɤ˧ tʰi˧-mv̩˧-ʈʰɯ˧˥ | -tɕɯ˩-zo˩!}\hspace{5pt}\peng{(Go ahead and) drink this bowl of wine!}\hspace{5pt}\pcmn{把这碗酒喝了下去!}\hspace{5pt}\pfra{Bois donc ce bol de vin!}\end{exemple}
\end{entrée}

\begin{entrée}
{tɕɯ˥}{}{ⓔtɕɯ˥}\newline
\classe{动词}
\sens{1}
\begin{définition}\peng{To put, to lay up.}\end{définition}
\begin{définition}\pcmn{放置}\end{définition}
\begin{définition}\pfra{Poser, ranger, mettre, placer.}\end{définition}
\begin{exemple}\pnru{tʰi˧-tɕɯ˥}\hspace{5pt}\peng{|fg{dur}}\hspace{5pt}\pcmn{|fg{dur}}\hspace{5pt}\pfra{|fg{dur}}\end{exemple}
\begin{exemple}\pnru{ɖɯ˩hĩ˧ | ɖɯ˩˧ | tʰi˧-tɕɯ˥, | tɕi˩hĩ˧ | tɕi˩˧ | tʰi˧-tɕɯ˥}\hspace{5pt}\peng{to put big ones with big ones, small ones with small ones}\hspace{5pt}\pcmn{大小归类}\hspace{5pt}\pfra{mettre les grands avec les grands, les petits avec les petits}\end{exemple}\sens{2}
\begin{définition}\peng{To settle, to decide.}\end{définition}
\begin{définition}\pcmn{决定、定下来}\end{définition}
\begin{définition}\pfra{Fixer, décider (ex.: les puissances suprêmes fixent la durée de la vie humaine).}\end{définition}
\begin{exemple}\pnru{le˧-ʐwɤ˩ | tʰi˧-tɕɯ˥}\hspace{5pt}\peng{to settle}\hspace{5pt}\pcmn{说好、决定}\hspace{5pt}\pfra{fixer; arrêter; décider que}\end{exemple}
\begin{exemple}\pnru{le˧-ʐwɤ˩ | tʰi˧-tɕɯ˧-ɲi˥-tsɯ˩!}\hspace{5pt}\peng{It's settled!}\hspace{5pt}\pcmn{说好了! / 决定好了!}\hspace{5pt}\pfra{C'est fixé/c'est décidé/c'est arrêté!}\end{exemple}
\end{entrée}

\begin{entrée}
{tɕɯ˧}{}{ⓔtɕɯ˧}\formedesurface{tɕɯ˧}\newline
\classe{名词}\ton{M}
\paradigme{\pcmn{:} \p{}}
\begin{définition}\peng{Cloud.}\end{définition}
\begin{définition}\pcmn{云}\end{définition}
\begin{définition}\pfra{Nuage.}\end{définition}
\begin{exemple}\pnru{mv̩˧tɕɯ˥}\hspace{5pt}\pfra{il y a des nuages, le temps est nuageux}\hspace{5pt}\peng{the weather is cloudy}\hspace{5pt}\pcmn{天上多云}\end{exemple}
\begin{exemple}\pnru{mv̩˧ʁo˥, | tɕɯ˧!}\hspace{5pt}\peng{The sky is cloudy!}\hspace{5pt}\pcmn{天上有云!}\hspace{5pt}\pfra{le ciel est nuageux!}\end{exemple}
\begin{exemple}\pnru{mv̩˧ʁo˥ tɕɯ˩ pʰv̩˩ |}\hspace{5pt}\peng{The sky is cloudy!}\hspace{5pt}\pcmn{天上有云!}\hspace{5pt}\pfra{le ciel est nuageux!}\end{exemple}
\begin{exemple}\pnru{tɕɯ˧pʰv̩˩; tɕɯ˧ | pʰv̩˩tɕæ˩˥ ◊ -gv̩˩}\hspace{5pt}\peng{white cloud}\hspace{5pt}\pcmn{白云、白色的云}\hspace{5pt}\pfra{nuage blanc}\end{exemple}
\begin{exemple}\pnru{mv̩˧nɑ˥-tɕɯ˩nɑ˩-ɻ̍˩!}\hspace{5pt}\peng{the sky is dark / the sky is very cloudy}\hspace{5pt}\pcmn{天很黑,有很多乌云}\hspace{5pt}\pfra{il fait sombre/ le ciel est très nuageux!}\end{exemple}
\end{entrée}

\begin{entrée}
{tɕɯ˧˥}{₁}{ⓔtɕɯ˧˥ⓗ1}\formedesurface{tɕɯ˧˥}\newline
\classe{动词}\ton{MH}
1\begin{définition}\peng{To pack-transport.}\end{définition}
\begin{définition}\pcmn{驮运}\end{définition}
\begin{définition}\pfra{Transporter à dos d'animaux.}\end{définition}
\begin{exemple}\pnru{ʐwæ˧ tɕɯ˩}\hspace{5pt}\peng{to pack-transport, to transport on horseback}\hspace{5pt}\pcmn{用马驮运、做马帮}\hspace{5pt}\pfra{faire du commerce par caravanes, transporter en caravane, organiser une caravane}\end{exemple}
\begin{exemple}\pnru{ʐwæ˧ʁo˧ tʰi˧-tɕɯ˧˥}\hspace{5pt}\peng{to transport on horseback}\hspace{5pt}\pcmn{用马驮运}\hspace{5pt}\pfra{transporter à dos de cheval}\end{exemple}
\begin{exemple}\pnru{ʐwæ˧-tɕɯ˩-zo˩}\hspace{5pt}\peng{person who takes part in a caravan, who works in a caravan}\hspace{5pt}\pcmn{加入马帮的男人}\hspace{5pt}\pfra{caravanier, personne qui va avec les caravanes}\end{exemple}
\end{entrée}

\begin{entrée}
{tɕɯ˧˥}{₂}{ⓔtɕɯ˧˥ⓗ2}\formedesurface{tɕɯ˧˥}\newline
\classe{名词}\ton{MH}
2
\paradigme{\pcmn{:} \p{}}
\begin{définition}\peng{Leech.}\end{définition}
\begin{définition}\pcmn{水蛭、蚂蟥}\end{définition}
\begin{définition}\pfra{Sangsue.}\end{définition}
\end{entrée}

\begin{entrée}
{tɕɯ˧˥}{₃}{ⓔtɕɯ˧˥ⓗ3}\formedesurface{tɕɯ˧˥}\newline
\classe{名词}\ton{MH}
3
\paradigme{\pcmn{:} \p{}}
\begin{définition}\peng{Wasp.}\end{définition}
\begin{définition}\pcmn{马蜂 (黄蜂)}\end{définition}
\begin{définition}\pfra{Guêpe.}\end{définition}
\begin{exemple}\pnru{tɕɯ˧mi˥\$}\hspace{5pt}\peng{female wasp (elicited combination)}\hspace{5pt}\pcmn{母蚂蜂(人工的词)}\hspace{5pt}\pfra{guêpe femelle (élicité pour le propos de l'étude tonale)}\end{exemple}
\begin{exemple}\pnru{tɕɯ˧pʰv̩\#˥}\hspace{5pt}\peng{male wasp (elicited combination)}\hspace{5pt}\pcmn{公马蜂}\hspace{5pt}\pfra{guêpe mâle (élicité pour le propos de l'étude tonale)}\end{exemple}
\begin{exemple}\pnru{tɕɯ˧zo\#˥}\hspace{5pt}\peng{baby wasp (elicited combination)}\hspace{5pt}\pcmn{小马蜂}\hspace{5pt}\pfra{bébé guêpe (élicité pour le propos de l'étude tonale)}\end{exemple}
\end{entrée}

\begin{entrée}
{tɕɯ˧˥}{₄}{ⓔtɕɯ˧˥ⓗ4}\formedesurface{tɕɯ˧˥}\newline
\classe{名词}\ton{MH}
4
\paradigme{\pcmn{:} \p{}}
\begin{définition}\peng{Scales (for weighing something).}\end{définition}
\begin{définition}\pcmn{称}\end{définition}
\begin{définition}\pfra{Balance.}\end{définition}
\end{entrée}

\begin{entrée}
{tɕɯ˧˥α}{₁}{ⓔtɕɯ˧˥αⓗ1}\formedesurface{ɖɯ˧ tɕɯ˧˥}\newline
\classe{量词}\ton{MHα}
1\begin{définition}\peng{Classifier for loads carried by a pack-animal.}\end{définition}
\begin{définition}\pcmn{量词:驮子(一匹)}\end{définition}
\begin{définition}\pfra{Classificateur des charges sur une bête de somme.}\end{définition}
\end{entrée}

\begin{entrée}
{tɕɯ˧˥α}{₂}{ⓔtɕɯ˧˥αⓗ2}\formedesurface{ɖɯ˧ tɕɯ˧˥}\newline
\classe{量词}\ton{MHα}
2\begin{définition}\peng{Classifier: a pound of.}\end{définition}
\begin{définition}\pcmn{量词:斤(用于固体,也用于液体)(汉语借词)}\end{définition}
\begin{définition}\pfra{Livre (aussi pour les liquides: pinte).}\end{définition}
\begin{exemple}\pnru{ʐɯ˧ | ɖɯ˧-tɕɯ˧˥}\hspace{5pt}\peng{a pint of wine}\hspace{5pt}\pcmn{一斤酒}\hspace{5pt}\pfra{une pinte de vin}\end{exemple}
\end{entrée}

\begin{entrée}
{tɕɯ˧β}{}{ⓔtɕɯ˧β}\formedesurface{tɕɯ˧}\newline
\classe{动词}\ton{Mβ}\begin{définition}\peng{To shake.}\end{définition}
\begin{définition}\pcmn{摇晃}\end{définition}
\begin{définition}\pfra{Secouer (monosyllabe).}\end{définition}
\begin{exemple}\pnru{tso˧∼tso˧ tɕɯ˧}\hspace{5pt}\peng{to shake things}\hspace{5pt}\pcmn{摇东西}\hspace{5pt}\pfra{secouer des choses}\end{exemple}
\end{entrée}

\begin{entrée}
{tɕɯ˩˥}{}{ⓔtɕɯ˩˥}\formedesurface{tɕɯ˩˥}\newline
\classe{名词}\ton{LH}\begin{définition}\peng{Saliva.}\end{définition}
\begin{définition}\pcmn{口水、唾、唾沫、唾液}\end{définition}
\begin{définition}\pfra{Salive.}\end{définition}
\end{entrée}

\begin{entrée}
{tɕɯ˩α}{}{ⓔtɕɯ˩α}\formedesurface{tɕɯ˩˥}\newline
\classe{动词}\ton{Lα}\begin{définition}\peng{To write.}\end{définition}
\begin{définition}\pcmn{写}\end{définition}
\begin{définition}\pfra{Écrire.}\end{définition}
\begin{exemple}\pnru{le˧-tɕɯ˩-ze˩}\hspace{5pt}\peng{|fg{accomp}+|fg{pfv}}\hspace{5pt}\pcmn{写了}\hspace{5pt}\pfra{|fg{accomp}+|fg{pfv}}\end{exemple}
\begin{exemple}\pnru{tʰæ˧ɻæ˩ tɕɯ˩}\hspace{5pt}\peng{to write, to write a text, to write a book}\hspace{5pt}\pcmn{写、写书}\hspace{5pt}\pfra{écrire quelque chose/écrire du texte/écrire un livre}\end{exemple}
\begin{exemple}\pnru{ɖɯ˧-kʰv̩˥ | tsʰe˧-ɲi˧ ɬi˧, | njɤ˧ | tsʰe˧-ɲi˧ bæ˧ tɕɯ˩-bi˩-ʂv̩˩ɖv̩˩!}\hspace{5pt}\peng{There are twelve months in one year; I would like to transcribe twelve stories (in the coming year)! (Context: the consultant notices that I completed the transcription of two texts in two months; by providing this example sentence, she suggests to me the project of keeping up the same rhythm, transcribing twelve stories in the coming year.)}\hspace{5pt}\pcmn{一年有十二个月,我就想(一年之内)记十二个故事!(情景:我两个月内完成了两个故事的记录工作。发音合作人举这个例句,鼓励我坚持这种速度,一年内再记十二个故事。)}\hspace{5pt}\pfra{Dans une année il y a douze mois; je voudrais transcrire douze histoires (au cours de l'année qui vient)! (contexte: en septembre 2011, Ama remarque que j'ai transcrit deux contes en deux mois; en m'offrant cet exemple, elle me souffle le projet de garder le rythme et de transcrire une histoire par mois soit douze pendant l'année qui vient)}\end{exemple}
\begin{exemple}\pnru{ɖɯ˧-tɕɯ˧∼tɕɯ˥-ɻ̍˩}\hspace{5pt}\peng{|fg{delimitative} \_ |fg{red} |fg{inceptive}}\hspace{5pt}\pcmn{|fg{delimitative} \_ |fg{red} |fg{inceptive}}\hspace{5pt}\pfra{|fg{délimitatif} \_ |fg{red} |fg{inchoatif}}\end{exemple}
\begin{exemple}\pnru{tɕɯ˩-di˩˥}\hspace{5pt}\peng{brush, pen; literally ‘thing to write'}\hspace{5pt}\pcmn{笔。直译:‘(用来)书写的(东西)’}\hspace{5pt}\pfra{pinceau; littéralement ‘chose pour écrire'}\end{exemple}
\begin{exemple}\pnru{tʰæ˧ɻæ˩-tɕɯ˩-di˩}\hspace{5pt}\peng{brush, pen; literally ‘thing to write books'}\hspace{5pt}\pcmn{笔。直译:‘(用来)写书的(东西)’}\hspace{5pt}\pfra{pinceau; littéralement ‘chose pour écrire des livres'}\end{exemple}
\begin{exemple}\pnru{ʈʂʰɯ˧ | tʰi˧-tɕɯ˧∼tɕɯ˥ dʑo˩}\hspace{5pt}\peng{(S)he is writing}\hspace{5pt}\pcmn{他正在写写东西。}\hspace{5pt}\pfra{Elle/il est en train d'écrire}\end{exemple}
\end{entrée}

\begin{entrée}
{tɕɯ˩lv̩˩ho˥}{}{ⓔtɕɯ˩lv̩˩ho˥}\formedesurface{tɕɯ˩lv̩˩ho˥}\newline
\classe{名词}\ton{L+H\#}
\paradigme{\pcmn{:} \p{}}
\begin{définition}\peng{Sling (for throwing an object).}\end{définition}
\begin{définition}\pcmn{弹弓}\end{définition}
\begin{définition}\pfra{Fronde (bande de tissu permettant de catapulter un objet).}\end{définition}
\end{entrée}

\begin{entrée}
{tɕɯ˧ɭɯ˧}{}{ⓔtɕɯ˧ɭɯ˧}\formedesurface{tɕɯ˧ɭɯ˧}\newline
\classe{动词}\ton{M}\begin{définition}\peng{To roll, to spool, to reel.}\end{définition}
\begin{définition}\pcmn{缠绕}\end{définition}
\begin{définition}\pfra{Enrouler, embobiner.}\end{définition}
\begin{exemple}\pnru{njɤ˧-ɳɯ˧ | tɕɯ˧ɭɯ˧-bi˧!}\hspace{5pt}\peng{Let me reel! / Let me do the reeling!}\hspace{5pt}\pcmn{让我来缠吧!}\hspace{5pt}\pfra{Je me charge d'enrouler! / C'est moi qui vais enrouler!}\end{exemple}
\end{entrée}

\begin{entrée}
{tɕɯ˩ɭɯ˩}{}{ⓔtɕɯ˩ɭɯ˩}\formedesurface{tɕɯ˩ɭɯ˩˥}\newline
\classe{名词}\ton{L}
\paradigme{\pcmn{:} \p{}}
\begin{définition}\peng{Shrike, |\stylefi{Lanius tephronotus} (a species of bird).}\end{définition}
\begin{définition}\pcmn{伯劳鸟}\end{définition}
\begin{définition}\pfra{Pie grièche du Tibet, |\stylefi{Lanius tephronotus}.}\end{définition}
\end{entrée}

\begin{entrée}
{tɕɯ˩ɭɯ˩-qʰæ˥bæ˩}{}{ⓔtɕɯ˩ɭɯ˩-qʰæ˥bæ˩}\formedesurface{tɕɯ˩ɭɯ˩qʰæ˥bæ˩}\newline
\classe{名词}\ton{L+\#H-}
\paradigme{\pcmn{:} \p{}}
\begin{définition}\peng{Chinese star jessamine, an evergreen woody liana (|\stylefi{Trachelospermum jasminoides}). Its name in Na literally means ‘spoon of the shrike’.}\end{définition}
\begin{définition}\pcmn{络石藤,别名石鲮、明石、悬石、云珠、云丹、红对叶肾、白花藤}\end{définition}
\begin{définition}\pfra{Jasmin étoilé: plante grimpante de la famille des Apocynaceae, aussi appelé faux jasmin ou jasmin des Indes (|\stylefi{Trachelospermum jasminoides}). Son nom en na signifie littéralement ‘cuillère de la pie grièche’.}\end{définition}
\end{entrée}

\begin{entrée}
{tɕɯ˧mi˥\$}{}{ⓔtɕɯ˧mi˥\$}\formedesurface{tɕɯ˧mi˥}\newline
\classe{名词}\ton{H\$}\begin{définition}\peng{Large scales (for weighing things).}\end{définition}
\begin{définition}\pcmn{大称}\end{définition}
\begin{définition}\pfra{Grande balance.}\end{définition}
\end{entrée}

\begin{entrée}
{tɕɯ˩mi˥}{}{ⓔtɕɯ˩mi˥}\formedesurface{tɕɯ˩mi˥}\newline
\classe{名词}\ton{LH}
\paradigme{\pcmn{:} \p{}}
\begin{définition}\peng{Chinese Hwamei or Melodious Laughingthrush (|\stylefi{Leucodioptron canorum}).}\end{définition}
\begin{définition}\pcmn{画眉鸟}\end{définition}
\begin{définition}\pfra{Passereau de la famille des Leiothrichidae: (|\stylefi{Leucodioptron canorum}). Le nom ‘hwamei’ signifie «sourcils peints» en référence à la marque distinctive autour des yeux de l'oiseau.}\end{définition}
\begin{exemple}\pnru{tɕɯ˩mi˥ | ə˧mi˧ ɲi˩!}\hspace{5pt}\peng{It's a mummy hwamei! (=a female)}\hspace{5pt}\pcmn{是一个画眉鸟妈妈!(=是母的画眉鸟)}\hspace{5pt}\pfra{c'est une maman hwamei! (=une femelle)}\end{exemple}
\begin{exemple}\pnru{tɕɯ˩mi˥ | zo˧ ɲi˥!}\hspace{5pt}\peng{It's a baby hwamei!}\hspace{5pt}\pcmn{是一个小画眉鸟!}\hspace{5pt}\pfra{c'est un petit hwamei! (=un enfant/bébé)}\end{exemple}
\begin{exemple}\pnru{tɕɯ˩mi˥ | pʰv̩˧ ɲi˩!}\hspace{5pt}\peng{It's a male hwamei!}\hspace{5pt}\pcmn{是公的画眉鸟!}\hspace{5pt}\pfra{C'est un hwamei mâle!}\end{exemple}
\end{entrée}

\begin{entrée}
{tɕɯ˧pv̩˧}{}{ⓔtɕɯ˧pv̩˧}\formedesurface{tɕɯ˧pv̩˧}\newline
\classe{形容词}\ton{M}\begin{définition}\peng{At ease.}\end{définition}
\begin{définition}\pcmn{轻松快乐、舒畅}\end{définition}
\begin{définition}\pfra{À l'aise, peinard.}\end{définition}
\begin{exemple}\pnru{ʈʂʰɯ˧qo˧ | tɕɯ˧pv̩˧-ʂe˧∼ʂe˧ | ɖɯ˧-dzi˩-zo˩-ho˩!}\hspace{5pt}\peng{Have a seat here, happy and relaxed!}\hspace{5pt}\pcmn{在这边舒畅地坐一会吧!}\hspace{5pt}\pfra{assieds-toi ici, bien peinard!}\end{exemple}
\begin{exemple}\pnru{ʈʂʰɯ˧qo˧ | tɕɯ˧pv̩˧-ʂe˧∼ʂe˧-zo˥ | ɖɯ˧-dzi˩-bi˩-ɻ̍˩!}\hspace{5pt}\peng{Let's have a seat here, happy and relaxed!}\hspace{5pt}\pcmn{在这边舒畅地坐一会吧!}\hspace{5pt}\pfra{asseyons-nous donc ici, bien peinards!}\end{exemple}
\end{entrée}

\begin{entrée}
{tɕɯ˧sɯ˧˥}{}{ⓔtɕɯ˧sɯ˧˥}\formedesurface{tɕɯ˧sɯ˧˥}\newline
\classe{名词}\ton{MH\#}
\paradigme{\pcmn{:} \p{}}
\begin{définition}\peng{Mist, fog.}\end{définition}
\begin{définition}\pcmn{雾}\end{définition}
\begin{définition}\pfra{Brume.}\end{définition}
\begin{exemple}\pnru{tɕɯ˧sɯ˧mv̩˥}\hspace{5pt}\peng{there is some fog, there is some mist}\hspace{5pt}\pcmn{有雾}\hspace{5pt}\pfra{il y a de la brume}\end{exemple}
\end{entrée}

\begin{entrée}
{tɕɯ˧wɤ˧}{}{ⓔtɕɯ˧wɤ˧}\formedesurface{tɕɯ˧wɤ˧}\newline
\classe{动词}\ton{M}\begin{définition}\peng{To reincarnate.}\end{définition}
\begin{définition}\pcmn{转生、转世}\end{définition}
\begin{définition}\pfra{Se réincarner.}\end{définition}
\begin{exemple}\pnru{le˧-tɕɯ˧wɤ˧-ho˥!}\hspace{5pt}\peng{(She/he) is going to get reincarnated! (About a deceased person)}\hspace{5pt}\pcmn{他要转生了!}\hspace{5pt}\pfra{(La défunte / le défunt) va se réincarner!}\end{exemple}
\end{entrée}

\begin{entrée}
{tɕɯ˧zo˥\$}{}{ⓔtɕɯ˧zo˥\$}\formedesurface{tɕɯ˧zo˥}\newline
\classe{名词}\ton{H\$}\begin{définition}\peng{Small scales (for weighing things).}\end{définition}
\begin{définition}\pcmn{小称}\end{définition}
\begin{définition}\pfra{Petite balance.}\end{définition}
\end{entrée}

\begin{entrée}
{tɕʰɤ˧˥}{}{ⓔtɕʰɤ˧˥}\formedesurface{tɕʰɤ˧˥}\newline
\classe{动词}\ton{MH}\begin{définition}\peng{To cheat on someone, to deceive.}\end{définition}
\begin{définition}\pcmn{欺骗}\end{définition}
\begin{définition}\pfra{Tromper.}\end{définition}
\begin{exemple}\pnru{le˧-tɕʰɤ˧-ze˥}\hspace{5pt}\peng{|fg{accomp} \_ |fg{pfv}}\hspace{5pt}\pcmn{欺骗了}\hspace{5pt}\pfra{|fg{accomp} \_ |fg{pfv}}\end{exemple}
\begin{exemple}\pnru{hĩ˧ tɕʰɤ˧(-ze˩)}\hspace{5pt}\peng{to cheat on people, to deceive people}\hspace{5pt}\pcmn{骗人}\hspace{5pt}\pfra{tromper les gens}\end{exemple}
\begin{exemple}\pnru{no˧ | hĩ˧ tɕʰɤ˧!}\hspace{5pt}\peng{You cheat people! / You deceive people!}\hspace{5pt}\pcmn{你骗人!}\hspace{5pt}\pfra{vous trompez les gens!}\end{exemple}
\begin{exemple}\pnru{(hĩ˧ |) no˩ tɕʰɤ˩˥!}\hspace{5pt}\peng{People cheat you!}\hspace{5pt}\pcmn{人家骗你!}\hspace{5pt}\pfra{les gens vous trompent!}\end{exemple}
\begin{exemple}\pnru{(no˧ |) njɤ˩ tɕʰɤ˩˥!}\hspace{5pt}\peng{You cheat on me!}\hspace{5pt}\pcmn{你骗我!}\hspace{5pt}\pfra{Vous me trompez!}\end{exemple}
\end{entrée}

\begin{entrée}
{tɕʰɤ˩lv̩˩}{}{ⓔtɕʰɤ˩lv̩˩}\formedesurface{tɕʰɤ˩lv̩˩˥}\newline
\classe{名词}\ton{L}\begin{définition}\peng{Lily, lily buds.}\end{définition}
\begin{définition}\pcmn{百合}\end{définition}
\begin{définition}\pfra{Lis.}\end{définition}
\begin{exemple}\pnru{tɕʰɤ˩lv̩˩-hṽ̩˩hṽ̩˩˥}\hspace{5pt}\peng{stir-fried lily buds}\hspace{5pt}\pcmn{炒百合}\hspace{5pt}\pfra{lis cuits au wok}\end{exemple}
\begin{exemple}\pnru{tɕʰɤ˩lv̩˩˥, | kv̩˧-pʰæ˧di˥!}\hspace{5pt}\peng{Lily buds look like garlic!}\hspace{5pt}\pcmn{百合,像大蒜!}\hspace{5pt}\pfra{Le lis, ça ressemble à de l'ail!}\end{exemple}
\begin{exemple}\pnru{tɕʰɤ˩lv̩˩˥, | dʑɯ˩-nɑ˩mi˩-ʁo˥ dʑɯ˩-nɑ˩mi˩-ʁo˥ di˩-kv̩˩!}\hspace{5pt}\peng{Lilies grow high up on the mountain!}\hspace{5pt}\pcmn{百合长在高山上!}\hspace{5pt}\pfra{Le lis, ça pousse dans la montagne/en haute montagne!}\end{exemple}
\begin{exemple}\pnru{tɕʰɤ˩lv̩˩˥, | dʑɯ˩-nɑ˩mi˩-ʁo˥ | di˩-kv̩˩˥! |}\hspace{5pt}\peng{Lilies grow high up on the mountain!}\hspace{5pt}\pcmn{百合长在高山上!}\hspace{5pt}\pfra{Le lis, ça pousse dans la montagne/en haute montagne!}\end{exemple}
\end{entrée}

\begin{entrée}
{tɕʰɤ˧ɲi˧-ne˧-ʝi˥}{}{ⓔtɕʰɤ˧ɲi˧-ne˧-ʝi˥}\formedesurface{tɕʰɤ˧ɲi˧ne˧ʝi˥}\newline
\classe{助词}\ton{H\#}\begin{définition}\peng{Every day.}\end{définition}
\begin{définition}\pcmn{每天}\end{définition}
\begin{définition}\pfra{Tous les jours.}\end{définition}
\end{entrée}

\begin{entrée}
{tɕʰɤ˧pɤ˧-mi\#˥}{}{ⓔtɕʰɤ˧pɤ˧-mi\#˥}\formedesurface{tɕʰɤ˧pɤ˧mi˧}\newline
\classe{名词}\ton{\#H}\begin{définition}\peng{The name of a sacred spring located in a cave on mount \stylefv{/nɑ}˩tsʰi˩/.}\end{définition}
\begin{définition}\pcmn{一处神泉}\end{définition}
\begin{définition}\pfra{Nom d'une source sacrée, située dans une grotte, sur la montagne \stylefv{/nɑ}˩tsʰi˩/.}\end{définition}
\begin{exemple}\pnru{nɑ˩tsʰi˩˥ | tɕʰɤ˧pɤ˧-mi\#˥}\hspace{5pt}\peng{full name of the mountain where the spring is located}\hspace{5pt}\pcmn{神泉所在山的全称}\hspace{5pt}\pfra{nom complet de la montagne où se trouve la source sacrée}\end{exemple}
\end{entrée}

\begin{entrée}
{tɕʰɤ˧ʂo\#˥}{}{ⓔtɕʰɤ˧ʂo\#˥}\formedesurface{tɕʰɤ˧ʂo˧}\newline
\classe{名词}\ton{\#H}
\paradigme{\pcmn{:} \p{}}
\begin{définition}\peng{Altar.}\end{définition}
\begin{définition}\pcmn{祭坛}\end{définition}
\begin{définition}\pfra{Autel, lieu où on brûle de l'encens (dans la maison: l'autel principal est à l'étage, dans le bâtiment face à la porte de la ferme).}\end{définition}
\end{entrée}

\begin{entrée}
{tɕʰɤ˧ti\#˥}{}{ⓔtɕʰɤ˧ti\#˥}\formedesurface{tɕʰɤ˧ti˧}\newline
\classe{名词}\ton{\#H}
\sens{1}\paradigme{\pcmn{:} \p{}}
\begin{définition}\peng{Stupa, tower.}\end{définition}
\begin{définition}\pcmn{塔}\end{définition}
\begin{définition}\pfra{Stupa, tour.}\end{définition}\sens{2}
\begin{définition}\peng{Chorten (reliquary).}\end{définition}
\begin{définition}\pcmn{佛塔}\end{définition}
\begin{définition}\pfra{Chörten (reliquaire).}\end{définition}
\end{entrée}

\begin{entrée}
{tɕʰɤ˧tɕo˩}{}{ⓔtɕʰɤ˧tɕo˩}\formedesurface{tɕʰɤ˧tɕo˩}\newline
\classe{名词}\ton{L\#}\begin{définition}\peng{Two-man saw: a saw designed for use by two sawyers.}\end{définition}
\begin{définition}\pcmn{双人锯:以前用于把圆木截成板材的大的双人锯(汉语借词)}\end{définition}
\begin{définition}\pfra{Scie passe-partout: grande scie avec une poignée à chaque extrémité, maniée par deux bûcherons.}\end{définition}
\end{entrée}

\begin{entrée}
{tɕʰɤ˧tɕʰɤ˧˥}{}{ⓔtɕʰɤ˧tɕʰɤ˧˥}\formedesurface{tɕʰɤ˧tɕʰɤ˧˥}\newline
\classe{助词}\ton{MH\#}\begin{définition}\peng{Entirely, completely, totally.}\end{définition}
\begin{définition}\pcmn{彻底}\end{définition}
\begin{définition}\pfra{Entièrement, tout à fait, complètement.}\end{définition}
\end{entrée}

\begin{entrée}
{tɕʰɤ˩ʈʂv̩˧}{}{ⓔtɕʰɤ˩ʈʂv̩˧}\formedesurface{tɕʰɤ˩ʈʂv̩˥}\newline
\classe{名词}\ton{LM}
\paradigme{\pcmn{:} \p{}}
\begin{définition}\peng{Drinking glass, goblet; used for wine.}\end{définition}
\begin{définition}\pcmn{酒杯、杯子}\end{définition}
\begin{définition}\pfra{Gobelet pour le vin ou autres liquides (sans anse); en verre ou autre matériau.}\end{définition}
\begin{exemple}\pnru{bo˧ʐæ˧-tɕʰɤ˩ʈʂv˩}\hspace{5pt}\peng{goblet for drinking tea (made of glass)}\hspace{5pt}\pcmn{玻璃茶杯}\hspace{5pt}\pfra{gobelet à thé en verre}\end{exemple}
\end{entrée}

\begin{entrée}
{tɕʰi˥}{}{ⓔtɕʰi˥}\formedesurface{tɕʰi˧}\newline
\classe{名词}\ton{\#H}
\paradigme{\pcmn{:} \p{}}
\begin{définition}\peng{Thorn.}\end{définition}
\begin{définition}\pcmn{刺}\end{définition}
\begin{définition}\pfra{Épine.}\end{définition}
\end{entrée}

\begin{entrée}
{tɕʰi˧β}{}{ⓔtɕʰi˧β}\formedesurface{tɕʰi˧}\newline
\classe{动词}\ton{Mβ}\begin{définition}\peng{To sell.}\end{définition}
\begin{définition}\pcmn{卖}\end{définition}
\begin{définition}\pfra{Vendre.}\end{définition}
\end{entrée}

\begin{entrée}
{tɕʰi˩β}{}{ⓔtɕʰi˩β}\formedesurface{ɖɯ˧ tɕʰi˩}\newline
\classe{量词}\ton{Lβ}\begin{définition}\peng{Classifier for meals.}\end{définition}
\begin{définition}\pcmn{量词:饭(一顿)}\end{définition}
\begin{définition}\pfra{Classificateur des repas.}\end{définition}
\begin{exemple}\pnru{ɖɯ˧-tɕʰi˩ dzɯ˩}\hspace{5pt}\peng{to have a meal, to eat a meal}\hspace{5pt}\pcmn{吃一顿}\hspace{5pt}\pfra{prendre un repas}\end{exemple}
\begin{exemple}\pnru{gv̩˧-tɕʰi˥}\hspace{5pt}\peng{nine meals}\hspace{5pt}\pcmn{就顿(饭)}\hspace{5pt}\pfra{neuf repas}\end{exemple}
\begin{exemple}\pnru{tɕʰi˩ tʰv̩˩˥}\hspace{5pt}\peng{to contribute food for the meals during a funeral ceremony: when one is invited to a funeral, one brings food as a contribution to the funeral}\hspace{5pt}\pcmn{带饭,“出(一)顿(饭)”:被请参加守孝时,要给那家主人带上饭)}\hspace{5pt}\pfra{apporter de la nourriture, apporter un repas: lorsqu'on est invité à participer à des cérémonies funéraires, on apporte à manger, pour contribuer aux repas collectifs}\end{exemple}
\begin{exemple}\pnru{tɕʰi˩tʰv̩˩-hĩ˥}\hspace{5pt}\peng{the person who provides the meal at a wake (following a funeral); it is generally someone who is not from the household.}\hspace{5pt}\pcmn{给大家供饭的那个人(不一定是主人)}\hspace{5pt}\pfra{la personne qui se charge du repas / qui nourrit tous les participants (lors d'un repas de veillée funéraire)}\end{exemple}
\end{entrée}

\begin{entrée}
{tɕʰi˧ɖv̩\#˥}{}{ⓔtɕʰi˧ɖv̩\#˥}\formedesurface{tɕʰi˧ɖv̩˧}\newline
\classe{名词}\ton{\#H}\begin{définition}\peng{Feminine given name.}\end{définition}
\begin{définition}\pcmn{女性名字}\end{définition}
\begin{définition}\pfra{Prénom féminin.}\end{définition}
\end{entrée}

\begin{entrée}
{tɕʰi˧nɑ˥}{}{ⓔtɕʰi˧nɑ˥}\formedesurface{tɕʰi˧nɑ˧}\newline
\classe{名词}\ton{H\#}\begin{définition}\peng{Prinsepia, |\stylefi{Prinsepia utilis Royle}; its seeds yield a highly valued oil, for both cooking and massaging on people's bodies.}\end{définition}
\begin{définition}\pcmn{青刺果、青刺尖、阿娜斯果}\end{définition}
\begin{définition}\pfra{Prinsepia, |\stylefi{Prinsepia utilis Royle}; végétal qui sert pour les haies, à grosses épines, petites fleurs jaunes, et tige verte vernissée. On tire de ses graines une huile de grand prix, utilisée dans des préparations alimentaires et comme cosmétique/huile de massage.}\end{définition}
\begin{exemple}\pnru{tɕʰi˧nɑ˥-dzi˩}\hspace{5pt}\peng{prinsepia plant}\hspace{5pt}\pcmn{青刺尖}\hspace{5pt}\pfra{prinsepia (la plante)}\end{exemple}
\begin{exemple}\pnru{tɕʰi˧nɑ˥-bæ˩bæ˩}\hspace{5pt}\peng{prinsepia flower}\hspace{5pt}\pcmn{青刺果花}\hspace{5pt}\pfra{fleur de prinsepia}\end{exemple}
\end{entrée}

\begin{entrée}
{tɕʰi˩tɕʰɤ˩}{}{ⓔtɕʰi˩tɕʰɤ˩}\formedesurface{tɕʰi˩tɕʰɤ˩˥}\newline
\classe{形容词}\ton{L}\begin{définition}\peng{Complete, all in readiness.}\end{définition}
\begin{définition}\pcmn{齐全(汉语借词)}\end{définition}
\begin{définition}\pfra{Complet, au grand complet.}\end{définition}
\end{entrée}

\begin{entrée}
{tɕʰi˩tsɯ˧}{}{ⓔtɕʰi˩tsɯ˧}\formedesurface{tɕʰi˧tsɯ˥}\newline
\classe{名词}\ton{LM}\begin{définition}\peng{Eggplant.}\end{définition}
\begin{définition}\pcmn{茄子}\end{définition}
\begin{définition}\pfra{Aubergine.}\end{définition}
\end{entrée}

\begin{entrée}
{tɕʰi˧ʈʂʰɤ˥}{}{ⓔtɕʰi˧ʈʂʰɤ˥}\formedesurface{tɕʰi˧ʈʂʰɤ˥}\newline
\classe{名词}\ton{H\#}
\paradigme{\pcmn{:} \p{}}
\begin{définition}\peng{Car.}\end{définition}
\begin{définition}\pcmn{汽车(汉语借词)}\end{définition}
\begin{définition}\pfra{Voiture, automobile.}\end{définition}
\end{entrée}

\begin{entrée}
{tɕʰĩ˩tsʰɤ˩}{}{ⓔtɕʰĩ˩tsʰɤ˩}\formedesurface{tɕʰĩ˩tsʰɤ˩˥}\newline
\classe{名词}\ton{L}\begin{définition}\peng{Celery.}\end{définition}
\begin{définition}\pcmn{芹菜}\end{définition}
\begin{définition}\pfra{Céleri.}\end{définition}
\end{entrée}

\begin{entrée}
{tɕʰo˧˥}{}{ⓔtɕʰo˧˥}\formedesurface{tɕʰo˧˥}\newline
\classe{动词}\ton{MH}\begin{définition}\peng{To square (off).}\end{définition}
\begin{définition}\pcmn{(将木料)砍成方形}\end{définition}
\begin{définition}\pfra{Équarrir (une grosse pièce de bois de charpente).}\end{définition}
\begin{exemple}\pnru{bi˩mi˩-ɳɯ˥ | tɕʰo˧˥}\hspace{5pt}\peng{to square off with an axe}\hspace{5pt}\pcmn{用斧头砍成方形}\hspace{5pt}\pfra{équarrir à la hache}\end{exemple}
\end{entrée}

\begin{entrée}
{tɕʰo˩}{}{ⓔtɕʰo˩}\formedesurface{ɖɯ˧ tɕʰo˩}\newline
\classe{量词}\ton{L *}\begin{définition}\peng{Classifier: in combination with ‘one', means ‘together'; no plural form.}\end{définition}
\begin{définition}\pcmn{量词:一起}\end{définition}
\begin{définition}\pfra{Ensemble.}\end{définition}
\begin{exemple}\pnru{ɖɯ˧-tɕʰo˩}\hspace{5pt}\peng{together}\hspace{5pt}\pcmn{一起}\hspace{5pt}\pfra{ensemble}\end{exemple}
\begin{exemple}\pnru{le˧-tɕʰo˥∼tɕʰo˩}\hspace{5pt}\peng{same meaning as above: together}\hspace{5pt}\pcmn{同上:一起}\hspace{5pt}\pfra{même sens que ci-dessus: ensemble}\end{exemple}
\end{entrée}

\begin{entrée}
{tɕʰo˩˧}{}{ⓔtɕʰo˩˧}\formedesurface{tɕʰo˩˥}\newline
\classe{名词}\ton{LM}
\paradigme{\pcmn{:} \p{}}
\begin{définition}\peng{Ladle, scoop used for water.}\end{définition}
\begin{définition}\pcmn{勺子、瓢}\end{définition}
\begin{définition}\pfra{Louche (de grande taille: grosse louche pour puiser l'eau; tient plus d'un litre).}\end{définition}
\end{entrée}

\begin{entrée}
{tɕʰo˩α}{}{ⓔtɕʰo˩α}\formedesurface{tɕʰo˩˥}\newline
\classe{动词}\ton{Lα}\begin{définition}\peng{To accompany someone, to go along with someone.}\end{définition}
\begin{définition}\pcmn{陪伴、一起去、跟着}\end{définition}
\begin{définition}\pfra{Accompagner, suivre (quelqu'un lors d'un voyage, par exemple); aller avec.}\end{définition}
\begin{exemple}\pnru{hĩ˧ tɕʰo˥}\hspace{5pt}\peng{to accompany someone}\hspace{5pt}\pcmn{陪伴某人}\hspace{5pt}\pfra{suivre quelqu'un}\end{exemple}
\begin{exemple}\pnru{ɖɯ˧-tɕʰo˩ tʰi˩-tɕʰo˩ |}\hspace{5pt}\peng{to make up a set, to go with each other/one another: for instance, in the main room, the thangka above the hearth and the paintings on the cupboard that hosts the altar to the ancestors make up a set, they go with each other}\hspace{5pt}\pcmn{陪伴某人}\hspace{5pt}\pfra{aller ensemble, former un ensemble: par exemple, dans la pièce principale de la maison, le thangka au-dessus du foyer et les peintures sur le buffet-autel des ancêtres forment un tout, elles vont ensemble}\end{exemple}
\end{entrée}

\begin{entrée}
{tɕʰo˩mi\#˥}{}{ⓔtɕʰo˩mi\#˥}\formedesurface{tɕʰo˩mi˥}\newline
\classe{名词}\ton{LM+\#H}
\paradigme{\pcmn{:} \p{}}
\begin{définition}\peng{Large ladle.}\end{définition}
\begin{définition}\pcmn{大瓢}\end{définition}
\begin{définition}\pfra{Grande louche.}\end{définition}
\end{entrée}

\begin{entrée}
{tɕʰo˩qʰwɤ˧}{}{ⓔtɕʰo˩qʰwɤ˧}\formedesurface{tɕʰo˩qʰwɤ˥}\newline
\classe{名词}\ton{LM}
\paradigme{\pcmn{:} \p{}}
\begin{définition}\peng{Ladle used for pigswill.}\end{définition}
\begin{définition}\pcmn{用来煮猪食的勺子}\end{définition}
\begin{définition}\pfra{Louche utilisée pour les aliments des animaux; à la date de l'enquête, c'était un objet en aluminium, tandis que celui utilisé pour puiser l'eau, que l'on peut porter à la bouche, est en étain.}\end{définition}
\end{entrée}

\begin{entrée}
{tɕʰo˩zo\#˥}{}{ⓔtɕʰo˩zo\#˥}\formedesurface{tɕʰo˩zo˥}\newline
\classe{名词}\ton{LM+\#H}
\paradigme{\pcmn{:} \p{}}
\begin{définition}\peng{Small ladle.}\end{définition}
\begin{définition}\pcmn{小瓢}\end{définition}
\begin{définition}\pfra{Petite louche.}\end{définition}
\end{entrée}

\begin{entrée}
{tɕʰɯ˥}{}{ⓔtɕʰɯ˥}\formedesurface{tɕʰɯ˧}\newline
\classe{动词}\ton{H}\begin{définition}\peng{To pierce (e.g. a cow's nose).}\end{définition}
\begin{définition}\pcmn{穿刺、 刺破}\end{définition}
\begin{définition}\pfra{Percer, transpercer.}\end{définition}
\begin{exemple}\pnru{ʝi˧ ʈʂʰɯ˧-pʰo˩, | ɲi˧ tɕʰi˧-ze˩!}\hspace{5pt}\peng{This ox's nose was pierced (to put a ring)}\hspace{5pt}\pcmn{这头牛的鼻子被穿刺(为了安一个牛鼻圈)}\hspace{5pt}\pfra{Ce bœuf, on lui a percé le museau (pour y placer un anneau)!}\end{exemple}
\end{entrée}

\begin{entrée}
{tɕʰɯ˧˥}{}{ⓔtɕʰɯ˧˥}\formedesurface{tɕʰɯ˧˥}\newline
\classe{名词}\ton{MH}\begin{définition}\peng{Lacquer, paint.}\end{définition}
\begin{définition}\pcmn{漆}\end{définition}
\begin{définition}\pfra{Peinture, laque.}\end{définition}
\begin{exemple}\pnru{tɕʰɯ˧ jɤ˥-zo˩-ho˩!}\hspace{5pt}\peng{It's time to paint (the room, the house…)! / We're going to have to paint (the room, the house…)!}\hspace{5pt}\pcmn{该刷漆了!}\hspace{5pt}\pfra{Il va falloir (re)peindre}\end{exemple}
\end{entrée}

\begin{entrée}
{tɕʰɯ˧˥}{₁}{ⓔtɕʰɯ˧˥ⓗ1}\formedesurface{tɕʰɯ˧˥}\newline
\classe{动词}\ton{MH}
1\begin{définition}\peng{To throw away.}\end{définition}
\begin{définition}\pcmn{扔(垃圾)}\end{définition}
\begin{définition}\pfra{Jeter, se débarrasser de (poubelles, détritus…); abandonner.}\end{définition}
\begin{exemple}\pnru{ɖæ˩˥ | tʰi˧-tɕʰɯ˧˥}\hspace{5pt}\peng{to throw garbage}\hspace{5pt}\pcmn{扔垃圾}\hspace{5pt}\pfra{jeter des détritus}\end{exemple}
\end{entrée}

\begin{entrée}
{tɕʰɯ˧˥}{₂}{ⓔtɕʰɯ˧˥ⓗ2}\formedesurface{tɕʰɯ˧˥}\newline
\classe{动词}\ton{MH}
2\begin{définition}\peng{To spit.}\end{définition}
\begin{définition}\pcmn{吐(吐口水)}\end{définition}
\begin{définition}\pfra{Cracher.}\end{définition}
\end{entrée}

\begin{entrée}
{tɕʰɯ˧˥}{₃}{ⓔtɕʰɯ˧˥ⓗ3}\formedesurface{tɕʰɯ˧˥}\newline
\classe{动词}\ton{MH}
3\begin{définition}\peng{To lose, to misplace.}\end{définition}
\begin{définition}\pcmn{丢失、弄丢}\end{définition}
\begin{définition}\pfra{Perdre.}\end{définition}
\begin{exemple}\pnru{le˧-tɕʰɯ˧-ze˥}\hspace{5pt}\peng{|fg{accomp} \_ |fg{pfv}}\hspace{5pt}\pcmn{丢了}\hspace{5pt}\pfra{|fg{accomp} \_ |fg{pfv}}\end{exemple}
\begin{exemple}\pnru{le˧-tɕʰɯ˧-hɯ˥-ze˩!}\hspace{5pt}\peng{It's lost!}\hspace{5pt}\pcmn{丢掉了!}\hspace{5pt}\pfra{c'est perdu!}\end{exemple}
\end{entrée}

\begin{entrée}
{tɕʰɯ˧˥}{₄}{ⓔtɕʰɯ˧˥ⓗ4}\formedesurface{tɕʰɯ˧˥}\newline
\classe{形容词}\ton{MH}
4\begin{définition}\peng{Anxious, worried.}\end{définition}
\begin{définition}\pcmn{担心}\end{définition}
\begin{définition}\pfra{Inquiet, angoissé, tourmenté, oppressé.}\end{définition}
\begin{exemple}\pnru{nv̩˩mi˩ tɕʰɯ˥}\hspace{5pt}\peng{anxious, worried}\hspace{5pt}\pcmn{担心}\hspace{5pt}\pfra{inquiet}\end{exemple}
\end{entrée}

\begin{entrée}
{tɕʰɯ˧˥}{₅}{ⓔtɕʰɯ˧˥ⓗ5}\formedesurface{tɕʰɯ˧˥}\newline
\classe{形容词}\ton{MH}
5\begin{définition}\peng{At ease, comfortable.}\end{définition}
\begin{définition}\pcmn{舒服}\end{définition}
\begin{définition}\pfra{Observé seulement en tournure négative: ne pas avoir (de quoi vivre); être démuni. On peut imaginer comme sens ancien ‘bien doté, à l'aise’.}\end{définition}
\begin{exemple}\pnru{mɤ˧-tɕʰɯ˧-bi˥ / mɤ˧-tɕʰɯ˧˥ |-bi˩}\hspace{5pt}\peng{even if one is in need, …}\hspace{5pt}\pcmn{虽然很贫穷,……}\hspace{5pt}\pfra{même si on est dans le besoin/quoi qu'on soit dans le besoin, …}\end{exemple}
\begin{exemple}\pnru{mɤ˧-dʑo˧ mɤ˧-tɕʰɯ˧-ɻ̍˧-bi˥, | ɖwæ˩ mɤ˧-zo˧!}\hspace{5pt}\peng{Even if one is in need, one should not worry! (because the Gods will do something to save us)}\hspace{5pt}\pcmn{虽然穷,莫担心!(因为菩萨会救好人)}\hspace{5pt}\pfra{«Même si on est sans rien, dans le besoin, il ne faut pas s'inquiéter!» (car le Ciel vient en aide aux gens qui font de leur mieux)}\end{exemple}
\begin{exemple}\pnru{hĩ˧ ʈʂʰɯ˧-v̩˧-dʑo˩, | ɖwæ˧˥ | mɤ˧-tɕʰɯ˧˥!}\hspace{5pt}\peng{This person is really in need!}\hspace{5pt}\pcmn{这个人,真的很穷!}\hspace{5pt}\pfra{Il est vraiment dans le besoin/nécessiteux!}\end{exemple}
\end{entrée}

\begin{entrée}
{tɕʰɯ˧α}{₁}{ⓔtɕʰɯ˧αⓗ1}\formedesurface{tɕʰɯ˧}\newline
\classe{动词}\ton{Mα}
1\begin{définition}\peng{To raise (one's arm).}\end{définition}
\begin{définition}\pcmn{举、抬(胳膊)}\end{définition}
\begin{définition}\pfra{Lever (le bras…).}\end{définition}
\begin{exemple}\pnru{lo˩qʰwɤ˥ | gɤ˩-tɕʰɯ˧}\hspace{5pt}\peng{to raise one's arm}\hspace{5pt}\pcmn{举手、抬胳膊}\hspace{5pt}\pfra{lever le bras}\end{exemple}
\begin{exemple}\pnru{kʰɯ˧tsʰɤ˧˥ | gɤ˩-tɕʰɯ˧}\hspace{5pt}\peng{to raise one's leg}\hspace{5pt}\pcmn{抬脚}\hspace{5pt}\pfra{lever la jambe}\end{exemple}
\begin{exemple}\pnru{gɤ˩-mɤ˧-tɕʰɯ˧}\hspace{5pt}\peng{not to raise}\hspace{5pt}\pcmn{不抬起来}\hspace{5pt}\pfra{ne pas lever}\end{exemple}
\end{entrée}

\begin{entrée}
{tɕʰɯ˧α}{₂}{ⓔtɕʰɯ˧αⓗ2}\formedesurface{tɕʰɯ˧}\newline
\classe{动词}
2
\sens{1}
\begin{définition}\peng{To guard, to keep guard, to keep watch.}\end{définition}
\begin{définition}\pcmn{守卫}\end{définition}
\begin{définition}\pfra{Garder, faire la garde, surveiller.}\end{définition}
\begin{exemple}\pnru{ɑ˩ʁo˧ tɕʰɯ˧}\hspace{5pt}\peng{to watch over the house, to guard the house}\hspace{5pt}\pcmn{守护家}\hspace{5pt}\pfra{garder la maison}\end{exemple}
\begin{exemple}\pnru{ɑ˩ʁo˧ tʰi˧-tɕʰɯ˧-dʑo˧}\hspace{5pt}\peng{watching over the house}\hspace{5pt}\pcmn{守着家}\hspace{5pt}\pfra{en train de surveiller la maison}\end{exemple}
\begin{exemple}\pnru{tso˧∼tso˧ tɕʰɯ˧}\hspace{5pt}\peng{to watch over things}\hspace{5pt}\pcmn{守着东西}\hspace{5pt}\pfra{surveiller des objets}\end{exemple}\sens{2}
\begin{définition}\peng{To keep a deathwatch, to sit with others at a funeral wake.}\end{définition}
\begin{définition}\pcmn{居丧、守灵}\end{définition}
\begin{définition}\pfra{Veiller un défunt, lors d'une veillée funèbre.}\end{définition}
\begin{exemple}\pnru{hĩ˧ tɕʰɯ˧}\hspace{5pt}\peng{same meaning: to keep a deathwatch for a deceased person}\hspace{5pt}\pcmn{同上:守灵}\hspace{5pt}\pfra{même sens: veiller un défunt}\end{exemple}
\end{entrée}

\begin{entrée}
{tɕʰɯ˩˥}{}{ⓔtɕʰɯ˩˥}\formedesurface{tɕʰɯ˧}\newline
\classe{名词}\ton{LH}
\paradigme{\pcmn{:} \p{}}
\begin{définition}\peng{Muntjac, barking deer.}\end{définition}
\begin{définition}\pcmn{麂子}\end{définition}
\begin{définition}\pfra{Muntjac.}\end{définition}
\begin{exemple}\pnru{tɕʰɯ˩ hwæ˧-ze˩}\hspace{5pt}\peng{…has bought (a/the) muntjac}\hspace{5pt}\pcmn{买麂子}\hspace{5pt}\pfra{…a acheté un muntjac}\end{exemple}
\begin{exemple}\pnru{tɕʰɯ˩ dzɯ˩-ze˥}\hspace{5pt}\peng{…has eaten muntjac}\hspace{5pt}\pcmn{吃了麂子}\hspace{5pt}\pfra{…a mangé un muntjac}\end{exemple}
\end{entrée}

\begin{entrée}
{tɕʰɯ˩α}{}{ⓔtɕʰɯ˩α}\formedesurface{tɕʰɯ˩˥}\newline
\classe{形容词}\ton{Lα}\begin{définition}\peng{Sweet.}\end{définition}
\begin{définition}\pcmn{甜}\end{définition}
\begin{définition}\pfra{Sucré.}\end{définition}
\end{entrée}

\begin{entrée}
{tɕʰɯ˧bo˧˥}{}{ⓔtɕʰɯ˧bo˧˥}\formedesurface{tɕʰɯ˧bo˧˥}\newline
\classe{形容词}\ton{MH\#}\begin{définition}\peng{Fresh, cool.}\end{définition}
\begin{définition}\pcmn{凉快}\end{définition}
\begin{définition}\pfra{Frais.}\end{définition}
\end{entrée}

\begin{entrée}
{tɕʰɯ˩di˩}{}{ⓔtɕʰɯ˩di˩}\formedesurface{tɕʰɯ˩di˩˥}\newline
\classe{动词}\ton{L}\begin{définition}\peng{To hunt.}\end{définition}
\begin{définition}\pcmn{狩猎}\end{définition}
\begin{définition}\pfra{Chasser.}\end{définition}
\end{entrée}

\begin{entrée}
{tɕʰɯ˩di˩kʰv̩˩}{}{ⓔtɕʰɯ˩di˩kʰv̩˩}\formedesurface{tɕʰɯ˩di˩kʰv̩˩˥}\newline
\classe{名词}\ton{L}
\paradigme{\pcmn{:} \p{}}
\begin{définition}\peng{Hunting dog, hound.}\end{définition}
\begin{définition}\pcmn{猎狗}\end{définition}
\begin{définition}\pfra{Chien de chasse.}\end{définition}
\begin{exemple}\pnru{tɕʰɯ˩di˩-kʰv̩˥mi˩}\hspace{5pt}\peng{same meaning}\hspace{5pt}\pcmn{猎狗}\hspace{5pt}\pfra{même sens}\end{exemple}
\end{entrée}

\begin{entrée}
{tɕʰɯ˧lo\#˥}{}{ⓔtɕʰɯ˧lo\#˥}\formedesurface{tɕʰɯ˧lo˧}\newline
\classe{名词}\ton{\#H}
\paradigme{\pcmn{:} \p{}}
\begin{définition}\peng{Large plate.}\end{définition}
\begin{définition}\pcmn{大盘子}\end{définition}
\begin{définition}\pfra{Grande assiette.}\end{définition}
\end{entrée}

\begin{entrée}
{tɕʰɯ˩mi\#˥}{}{ⓔtɕʰɯ˩mi\#˥}\formedesurface{tɕʰɯ˩mi˥}\newline
\classe{名词}\ton{LM+\#H / L}
\paradigme{\pcmn{:} \p{}}
\begin{définition}\peng{Female muntjac, muntjac doe, female barking deer, barking deer doe.}\end{définition}
\begin{définition}\pcmn{母麂子}\end{définition}
\begin{définition}\pfra{Muntjac femelle.}\end{définition}
\end{entrée}

\begin{entrée}
{tɕʰɯ˩pʰv̩\#˥}{}{ⓔtɕʰɯ˩pʰv̩\#˥}\formedesurface{tɕʰɯ˩pʰv̩˥}\newline
\classe{名词}\ton{LM+\#H / L}
\paradigme{\pcmn{:} \p{}}
\begin{définition}\peng{Male muntjac, muntjac stag, male barking deer, barking deer stag.}\end{définition}
\begin{définition}\pcmn{公麂子}\end{définition}
\begin{définition}\pfra{Muntjac mâle.}\end{définition}
\end{entrée}

\begin{entrée}
{tɕʰɯ˩-ʁo˩-tɕʰɯ˥}{}{ⓔtɕʰɯ˩-ʁo˩-tɕʰɯ˥}\formedesurface{tɕʰɯ˩ʁo˩tɕʰɯ˥!}\newline
\classe{助词}\ton{L+H\#}\begin{définition}\peng{Bless you! (what one says when someone sneezes).}\end{définition}
\begin{définition}\pcmn{旁边的人打嚏喷时说的祝愿话}\end{définition}
\begin{définition}\pfra{A vos souhaits! (formule que l'on dit lorsque quelqu'un éternue).}\end{définition}
\end{entrée}

\begin{entrée}
{tɕʰɯ˧si˩-dʑɤ˩pv̩˩}{}{ⓔtɕʰɯ˧si˩-dʑɤ˩pv̩˩}\formedesurface{tɕʰɯ˧si˩dʑɤ˩pv̩˩}\newline
\classe{名词}\ton{L\#-}\begin{définition}\peng{Monster, demon.}\end{définition}
\begin{définition}\pcmn{妖怪}\end{définition}
\begin{définition}\pfra{Monstre, revenant.}\end{définition}
\begin{exemple}\pnru{no˧ | tɕʰɯ˧si˩-dʑɤ˩pv̩˩-ki˩ | le˧-hɯ˩-ɲi˩-ze˩!}\hspace{5pt}\peng{You have gone away to the world of monstres (and should not come back to trouble the living)! (Speech addressed to a ghost that one beseeches should not come back)}\hspace{5pt}\pcmn{你已经到妖怪的世界那边(就恳求你不要回来了)!(对鬼说的话)}\hspace{5pt}\pfra{Tu es parti rejoindre les monstres! (propos tenus à un revenant qu'on enjoint de ne plus revenir hanter les vivants)}\end{exemple}
\end{entrée}

\begin{entrée}
{tɕʰɯ˧sɯ˥}{}{ⓔtɕʰɯ˧sɯ˥}\formedesurface{tɕʰɯ˧sɯ˥}\newline
\classe{形容词}\ton{H\#}\begin{définition}\peng{Sad, grieved.}\end{définition}
\begin{définition}\pcmn{悲哀、伤心}\end{définition}
\begin{définition}\pfra{Triste, dans l'affliction, plongé dans le chagrin.}\end{définition}
\end{entrée}

\begin{entrée}
{tɕʰɯ˩∼tɕʰɯ˧˥}{}{ⓔtɕʰɯ˩∼tɕʰɯ˧˥}\formedesurface{tɕʰɯ˩tɕʰɯ˧˥}\newline
\classe{动词}\ton{MH}\begin{définition}\peng{To suck.}\end{définition}
\begin{définition}\pcmn{吸吮}\end{définition}
\begin{définition}\pfra{Sucer.}\end{définition}
\begin{exemple}\pnru{lo˩mi˧ tɕʰi˩∼tɕʰi˩}\hspace{5pt}\peng{to suck one's thumb}\hspace{5pt}\pcmn{吮拇指}\hspace{5pt}\pfra{sucer son pouce}\end{exemple}
\end{entrée}

\begin{entrée}
{tɕʰɯ˩zo\#˥}{}{ⓔtɕʰɯ˩zo\#˥}\formedesurface{tɕʰɯ˩zo˥}\newline
\classe{名词}\ton{LM+\#H / L}
\paradigme{\pcmn{:} \p{}}
\begin{définition}\peng{Baby muntjac, barking deer fawn}\end{définition}
\begin{définition}\pcmn{麂子崽子}\end{définition}
\begin{définition}\pfra{Petit muntjac.}\end{définition}
\end{entrée}

\begin{entrée}
{tsɑ˧}{}{ⓔtsɑ˧}\formedesurface{tsɑ˧}\newline
\classe{形容词}\ton{M}\begin{définition}\peng{Busy.}\end{définition}
\begin{définition}\pcmn{忙}\end{définition}
\begin{définition}\pfra{Occupé, affairé, pressé.}\end{définition}
\begin{exemple}\pnru{ɖwæ˧˥ | tsɑ˧}\hspace{5pt}\peng{|fg{intensive.very}: very busy}\hspace{5pt}\pcmn{很忙}\hspace{5pt}\pfra{|fg{intensif.très}: très occupé}\end{exemple}
\begin{exemple}\pnru{tsɑ˧ | ʐwæ˩˥}\hspace{5pt}\peng{extremely busy}\hspace{5pt}\pcmn{非常忙}\hspace{5pt}\pfra{extrêmement occupé}\end{exemple}
\end{entrée}

\begin{entrée}
{tsɑ˧˥}{₁}{ⓔtsɑ˧˥ⓗ1}\formedesurface{tsɑ˧˥}\newline
\classe{动词}
1
\sens{1}
\begin{définition}\peng{To kick, to smash (clods of earth).}\end{définition}
\begin{définition}\pcmn{打碎(坷拉),踢(一脚)}\end{définition}
\begin{définition}\pfra{Donner un coup de pied; briser (les mottes de terre, après le labour, avec une bêche, ou une masse en bois).}\end{définition}
\begin{exemple}\pnru{le˧-tsɑ˧-ze˥}\hspace{5pt}\peng{|fg{accomp}+|fg{pfv}}\hspace{5pt}\pcmn{|fg{accomp}+|fg{pfv}}\hspace{5pt}\pfra{|fg{accomp}+|fg{pfv}}\end{exemple}
\begin{exemple}\pnru{ʈʂe˧ tsɑ˩}\hspace{5pt}\peng{to smash clods of earth, after ploughing (with a hand instrument, such as a hoe)}\hspace{5pt}\pcmn{打碎土坷垃}\hspace{5pt}\pfra{briser les mottes de terre après le labour (avec un instrument manuel: houe, bêche)}\end{exemple}
\begin{exemple}\pnru{ɖɯ˧-tsɑ˧ tʰi˥-tsɑ˩}\hspace{5pt}\peng{to kick repeatedly, to give one kick after the other}\hspace{5pt}\pcmn{踢了又踢}\hspace{5pt}\pfra{donner une succession de coups de pied}\end{exemple}\sens{2}
\begin{définition}\peng{To row (a boat).}\end{définition}
\begin{définition}\pcmn{划(船)}\end{définition}
\begin{définition}\pfra{Ramer (=geste comparable à celui de briser les mottes: geste répétitif, exerçant la force coup après coup, sur l'eau, comme on le ferait sur des mottes de terre).}\end{définition}
\begin{exemple}\pnru{tsɑ˧-hɯ˥-tsɑ˩-ɻ̍˩}\hspace{5pt}\peng{to row in a sustained way, to row with great vigour}\hspace{5pt}\pcmn{用力地划船、一直划船}\hspace{5pt}\pfra{ramer de façon soutenue}\end{exemple}
\end{entrée}

\begin{entrée}
{tsɑ˧˥}{₂}{ⓔtsɑ˧˥ⓗ2}\formedesurface{tsɑ˧˥}\newline
\classe{动词}\ton{MH}
2\begin{définition}\peng{To lay (up/down), to place.}\end{définition}
\begin{définition}\pcmn{放置、放下}\end{définition}
\begin{définition}\pfra{Déposer, poser.}\end{définition}
\begin{exemple}\pnru{mv̩˩tɕo˧ tsɑ˧˥}\hspace{5pt}\peng{to lay down, to put down on the floor}\hspace{5pt}\pcmn{放下、放在地上}\hspace{5pt}\pfra{poser à terre}\end{exemple}
\end{entrée}

\begin{entrée}
{tsɑ˩}{}{ⓔtsɑ˩}\formedesurface{tsɑ˩˥}\newline
\classe{动词}\ton{L}\begin{définition}\peng{To wink (as a discreet sign of mutual understanding).}\end{définition}
\begin{définition}\pcmn{眨眼}\end{définition}
\begin{définition}\pfra{Faire un clin d'oeil (discret signe d'intelligence).}\end{définition}
\begin{exemple}\pnru{ʈʂʰɯ˧ | njɤ˩ɭɯ˧ tsɑ˩∼tsɑ˩-dʑo˩!}\hspace{5pt}\peng{|fg{red}: (S)he is winking!}\hspace{5pt}\pcmn{重叠:他在眨眨眼!}\hspace{5pt}\pfra{|fg{red}: Elle/il est en train de faire un clin d'oeil!}\end{exemple}
\begin{exemple}\pnru{ʈʂʰɯ˧ | njɤ˩ɭɯ˧ tsɑ˩-dʑo˩!}\hspace{5pt}\peng{(S)he is winking!}\hspace{5pt}\pcmn{他在眨眼!}\hspace{5pt}\pfra{Elle/il est en train de faire un clin d'oeil!}\end{exemple}
\begin{exemple}\pnru{tsɑ˩∼tsɑ˧˥}\hspace{5pt}\peng{|fg{red}}\hspace{5pt}\pcmn{重叠}\hspace{5pt}\pfra{|fg{red}}\end{exemple}
\begin{exemple}\pnru{mɤ˧-tsɑ˩∼tsɑ˩}\hspace{5pt}\peng{|fg{neg} |fg{red}}\hspace{5pt}\pcmn{不眨眼}\hspace{5pt}\pfra{|fg{neg} |fg{red}}\end{exemple}
\end{entrée}

\begin{entrée}
{tsɑ˧bɤ˧}{}{ⓔtsɑ˧bɤ˧}\formedesurface{tsɑ˧bɤ˧}\newline
\classe{名词}\ton{M}\begin{définition}\peng{Powder; flour.}\end{définition}
\begin{définition}\pcmn{糌粑、面粉、粉、粉末}\end{définition}
\begin{définition}\pfra{Poudre; farine.}\end{définition}
\begin{exemple}\pnru{qʰɑ˧dze˧-tsɑ˩bɤ˩}\hspace{5pt}\peng{sweetcorn flour}\hspace{5pt}\pcmn{玉米粉}\hspace{5pt}\pfra{farine de maïs}\end{exemple}
\begin{exemple}\pnru{dze˧ɭɯ˧-tsɑ˩bɤ˩}\hspace{5pt}\peng{wheat flour}\hspace{5pt}\pcmn{小麦面}\hspace{5pt}\pfra{farine de blé}\end{exemple}
\begin{exemple}\pnru{lv̩˧mi˧-tsɑ˩bɤ˩}\hspace{5pt}\peng{stone powder, powdered stone}\hspace{5pt}\pcmn{石头粉、被磨成粉的石头}\hspace{5pt}\pfra{poudre de pierre, pierre pulvérisée}\end{exemple}
\begin{exemple}\pnru{tsʰi˧zi˧-tsɑ˧bɤ˥}\hspace{5pt}\peng{highland barley flour}\hspace{5pt}\pcmn{青稞面粉}\hspace{5pt}\pfra{farine d'orge}\end{exemple}
\begin{exemple}\pnru{mv̩˩zɯ˩-tsɑ˩bɤ˥}\hspace{5pt}\peng{oatmeal flour}\hspace{5pt}\pcmn{燕麦面粉}\hspace{5pt}\pfra{farine d'avoine}\end{exemple}
\begin{exemple}\pnru{jɤ˩jo˧-tsɑ˧bɤ˥}\hspace{5pt}\peng{potato flour (elicited combination)}\hspace{5pt}\pcmn{洋芋面粉}\hspace{5pt}\pfra{farine de pommes de terre}\end{exemple}
\begin{exemple}\pnru{nv̩˩ɭɯ˧-tsɑ˩bɤ˩}\hspace{5pt}\peng{soy flour}\hspace{5pt}\pcmn{黄豆面粉}\hspace{5pt}\pfra{farine de soja}\end{exemple}
\begin{exemple}\pnru{læ˧tsɯ˥-tsɑ˩bɤ˩}\hspace{5pt}\peng{chili powder}\hspace{5pt}\pcmn{辣椒粉}\hspace{5pt}\pfra{piment en poudre}\end{exemple}
\begin{exemple}\pnru{ʈʂʰæ˧ɣɯ˧-tsɑ˧bɤ˥}\hspace{5pt}\peng{Medicine powder, medicine in powder form. For instance: the disinfectant commonly used in Yongning at the time of fieldwork, of the brand 云南白药}\hspace{5pt}\pcmn{药粉,粉状药品。如:“云南白药”消毒粉。}\hspace{5pt}\pfra{Médicament en poudre. (Exemple: le désinfectant en poudre actuellement utilisé, de la marque 云南白药.)}\end{exemple}
\begin{exemple}\pnru{ʂæ˩ɻ̃˩-tsɑ˩bɤ˥}\hspace{5pt}\peng{bone powder}\hspace{5pt}\pcmn{骨头粉}\hspace{5pt}\pfra{poudre d'os}\end{exemple}
\begin{exemple}\pnru{jɤ˧-tsɑ˧bɤ˧}\hspace{5pt}\peng{tobacco powder}\hspace{5pt}\pcmn{烟草粉}\hspace{5pt}\pfra{tabac en poudre}\end{exemple}
\begin{exemple}\pnru{jɤ˧ɻ̃˧-tsɑ˧bɤ˥}\hspace{5pt}\peng{tobacco powder}\hspace{5pt}\pcmn{烟草粉}\hspace{5pt}\pfra{poudre de tabac, tabac en poudre}\end{exemple}
\begin{exemple}\pnru{dze˩-tsɑ˩bɤ˥}\hspace{5pt}\peng{Szechuan pepper powder}\hspace{5pt}\pcmn{花椒粉}\hspace{5pt}\pfra{xanthoxyle en poudre}\end{exemple}
\begin{exemple}\pnru{dze˧-tsɑ˧bɤ˥}\hspace{5pt}\peng{powdered sugar, granulated sugar}\hspace{5pt}\pcmn{砂糖}\hspace{5pt}\pfra{sucre en poudre}\end{exemple}
\begin{exemple}\pnru{tsɑ˧bɤ˧ mɤ˩}\hspace{5pt}\peng{to eat dry flour (made of grilled barley)}\hspace{5pt}\pfra{manger du tsamba sec}\end{exemple}
\begin{exemple}\pnru{tsɑ˧bɤ˧ gv̩˩}\hspace{5pt}\peng{to cook tsamba (grilled flour)}\hspace{5pt}\pcmn{炒糌粑,制作糌粑}\hspace{5pt}\pfra{préparer du tsamba/de la farine grillée}\end{exemple}
\end{entrée}

\begin{entrée}
{tsɑ˩tɕi˩}{}{ⓔtsɑ˩tɕi˩}\formedesurface{tsɑ˩tɕi˩˥}\newline
\classe{名词}\ton{L}
\paradigme{\pcmn{:} \p{}}
\begin{définition}\peng{Various mushrooms, mixed mushrooms.}\end{définition}
\begin{définition}\pcmn{杂菌(汉语借词)}\end{définition}
\begin{définition}\pfra{Champignons divers, champignons variés.}\end{définition}
\end{entrée}

\begin{entrée}
{tsɑ˧ʐo˩}{}{ⓔtsɑ˧ʐo˩}\formedesurface{tsɑ˧ʐo˩}\newline
\classe{形容词}\ton{L\#}\begin{définition}\peng{Diligent, conscientious.}\end{définition}
\begin{définition}\pcmn{勤快}\end{définition}
\begin{définition}\pfra{Zélé, assidu.}\end{définition}
\end{entrée}

\begin{entrée}
{‑tsæ˧α}{}{ⓔ‑tsæ˧α}\formedesurface{tsæ˧}\newline
\classe{后缀}\ton{M}\begin{définition}\peng{Causative.}\end{définition}
\begin{définition}\pcmn{使动:让}\end{définition}
\begin{définition}\pfra{Causatif.}\end{définition}
\begin{exemple}\pnru{tʰi˧-dzɯ˥-kʰɯ˩-tsæ˩-ɲi˩!}\hspace{5pt}\peng{(We) have to get her to eat! (Context: comment made by a family member when a young child refused to have a meal)}\hspace{5pt}\pcmn{必须让她吃!(情景:一个小女孩拒绝吃饭,家人就说这句。)}\hspace{5pt}\pfra{il faut l'obliger à manger/il faut la faire manger! (Commentaire d'un membre de la famille au sujet d'une petite fille qui refuse un repas)}\end{exemple}
\begin{exemple}\pnru{tʰi˧-ʐwɤ˩-kʰɯ˩-tsæ˩-ɲi˩!}\hspace{5pt}\peng{(We) have to get (him/her) to talk! (Variant based on the preceding example)}\hspace{5pt}\pcmn{必须让他说!(在以上例子的基础上编的句子)}\hspace{5pt}\pfra{il faut le faire parler/il faut l'obliger à parler! (Variante créée par analogie avec l'exemple précédent)}\end{exemple}
\end{entrée}

\begin{entrée}
{tsæ˧qæ˥}{}{ⓔtsæ˧qæ˥}\newline
\classe{名词}
\sens{1}\paradigme{\pcmn{:} \p{}}
\begin{définition}\peng{Hook.}\end{définition}
\begin{définition}\pcmn{钩子}\end{définition}
\begin{définition}\pfra{Crochet.}\end{définition}\sens{2}
\begin{définition}\peng{Firing pin (of a gun).}\end{définition}
\begin{définition}\pcmn{撞针}\end{définition}
\begin{définition}\pfra{Percuteur (de fusil).}\end{définition}
\end{entrée}

\begin{entrée}
{tse˩˥}{}{ⓔtse˩˥}\formedesurface{tse˩˥}\newline
\classe{名词}\ton{LH}
\paradigme{\pcmn{:} \p{}}
\begin{définition}\peng{Lock.}\end{définition}
\begin{définition}\pcmn{锁}\end{définition}
\begin{définition}\pfra{Serrure, verrou.}\end{définition}
\begin{exemple}\pnru{æ˧tse˥}\hspace{5pt}\peng{bronze lock}\hspace{5pt}\pcmn{铜锁}\hspace{5pt}\pfra{verrou en bronze}\end{exemple}
\end{entrée}

\begin{entrée}
{tse˩α}{₁}{ⓔtse˩αⓗ1}\formedesurface{tse˩˥}\newline
\classe{动词}\ton{Lα}
1\begin{définition}\peng{To chase after; to pursue.}\end{définition}
\begin{définition}\pcmn{追赶}\end{définition}
\begin{définition}\pfra{Suivre à la trace, poursuivre, pister.}\end{définition}
\begin{exemple}\pnru{hĩ˧ tse˥}\hspace{5pt}\peng{to chase after someone}\hspace{5pt}\pcmn{追赶某人}\hspace{5pt}\pfra{suivre quelqu'un à la trace}\end{exemple}
\end{entrée}

\begin{entrée}
{tse˩α}{₂}{ⓔtse˩αⓗ2}\formedesurface{tse˩˥}\newline
\classe{动词}\ton{Lα}
2\begin{définition}\peng{To float.}\end{définition}
\begin{définition}\pcmn{漂浮 (浮在水上)}\end{définition}
\begin{définition}\pfra{Flotter.}\end{définition}
\begin{exemple}\pnru{gɤ˩tse˧}\hspace{5pt}\peng{as above: to float}\hspace{5pt}\pcmn{同上:漂浮 (浮在水上)}\hspace{5pt}\pfra{même sens: flotter}\end{exemple}
\begin{exemple}\pnru{ɖɯ˧-tse˧∼tse˥-ɻ̍˩}\hspace{5pt}\peng{|fg{delimitative} \_ |fg{red} |fg{inceptive}}\hspace{5pt}\pcmn{|fg{delimitative} \_ |fg{red} |fg{inceptive}}\hspace{5pt}\pfra{|fg{délimitatif} \_ |fg{red} |fg{inchoatif}}\end{exemple}
\begin{exemple}\pnru{dʑɯ˩ʁo˩˥ | tʰi˧-tse˩ (-dʑo˩)}\hspace{5pt}\peng{to float (in a torrent) on the mountain (e.g. timber is floated downstream)}\hspace{5pt}\pcmn{让木头漂到下游}\hspace{5pt}\pfra{faire flotter (dans un torrent), en montagne (ex.: des troncs qu'on ramène du lieu d'abattage jusqu'à la plaine)}\end{exemple}
\begin{exemple}\pnru{dʑɯ˩ʁo˩ tse˧}\hspace{5pt}\peng{as above: to float down from the mountain}\hspace{5pt}\pcmn{同上:让木头漂到下游}\hspace{5pt}\pfra{même sens: ramener de la montagne en faisant descendre la rivière}\end{exemple}
\end{entrée}

\begin{entrée}
{tse˩α}{₃}{ⓔtse˩αⓗ3}\formedesurface{tse˩˥}\newline
\classe{动词}\ton{Lα}
3\begin{définition}\peng{To lock.}\end{définition}
\begin{définition}\pcmn{锁门}\end{définition}
\begin{définition}\pfra{Fermer à clef.}\end{définition}
\begin{exemple}\pnru{kʰi˧ tse˥(-ze˩) / kʰi˧ tʰi˧-tse˩}\hspace{5pt}\peng{to lock the door}\hspace{5pt}\pcmn{锁门}\hspace{5pt}\pfra{verrouiller la porte}\end{exemple}
\end{entrée}

\begin{entrée}
{tse˧bæ˥}{}{ⓔtse˧bæ˥}\formedesurface{tse˧bæ˥}\newline
\classe{名词}\ton{H\#}
\paradigme{\pcmn{:} \p{}}
\begin{définition}\peng{Tinder.}\end{définition}
\begin{définition}\pcmn{火绒}\end{définition}
\begin{définition}\pfra{Amadou.}\end{définition}
\end{entrée}

\begin{entrée}
{tse˧bo\#˥}{}{ⓔtse˧bo\#˥}\formedesurface{tse˧bo˧}\newline
\classe{名词}\ton{\#H}
\paradigme{\pcmn{:} \p{}}
\begin{définition}\peng{Small bell.}\end{définition}
\begin{définition}\pcmn{铃铛}\end{définition}
\begin{définition}\pfra{Clochette (portée par le bétail: chevaux, parfois chiens).}\end{définition}
\end{entrée}

\begin{entrée}
{tse˧di˧}{}{ⓔtse˧di˧}\formedesurface{tse˧di˧}\newline
\classe{名词}\ton{M}\begin{définition}\peng{Sandalwood, sandlewood.}\end{définition}
\begin{définition}\pcmn{檀香木、檀香、檀木}\end{définition}
\begin{définition}\pfra{Bois de santal arbre à épice, arbre à encens.}\end{définition}
\begin{exemple}\pnru{tse˧di˧-si\#˥}\hspace{5pt}\peng{same meaning}\hspace{5pt}\pcmn{同上}\hspace{5pt}\pfra{même sens}\end{exemple}
\end{entrée}

\begin{entrée}
{tse˧kʰo˩}{}{ⓔtse˧kʰo˩}\formedesurface{tse˧kʰo˩}\newline
\classe{名词}\ton{L\#}
\paradigme{\pcmn{:} \p{}}
\begin{définition}\peng{Sanctuary (small sanctuary on the mountain).}\end{définition}
\begin{définition}\pcmn{佛龛}\end{définition}
\begin{définition}\pfra{Sanctuaire (petit sanctuaire sur la montagne; n'est pas habitable).}\end{définition}
\end{entrée}

\begin{entrée}
{tse˧lv̩˥}{}{ⓔtse˧lv̩˥}\formedesurface{tse˧lv̩˥}\newline
\classe{名词}\ton{H\#}
\paradigme{\pcmn{:} \p{}}
\begin{définition}\peng{Flint.}\end{définition}
\begin{définition}\pcmn{燧石}\end{définition}
\begin{définition}\pfra{Silex.}\end{définition}
\end{entrée}

\begin{entrée}
{tse˧mi˥}{}{ⓔtse˧mi˥}\formedesurface{tse˧mi˥}\newline
\classe{名词}\ton{H\#}
\paradigme{\pcmn{:} \p{}}
\begin{définition}\peng{Lighter.}\end{définition}
\begin{définition}\pcmn{火镰}\end{définition}
\begin{définition}\pfra{Briquet.}\end{définition}
\end{entrée}

\begin{entrée}
{tse˧mi˥-dʑɯ˩ʁo˩}{}{ⓔtse˧mi˥-dʑɯ˩ʁo˩}\formedesurface{tse˧mi˥dʑɯ˩ʁo˩}\newline
\classe{名词}\ton{H\#-L}\begin{définition}\peng{The village of Wenquan, in the plain of Yongning, where hot springs are located, hence the Chinese name Wenquan, ‘hot springs'.}\end{définition}
\begin{définition}\pcmn{温泉乡的主要村落}\end{définition}
\begin{définition}\pfra{Le village de Wenquan (possède des sources chaudes).}\end{définition}
\end{entrée}

\begin{entrée}
{tse˩pʰæ˧˥}{}{ⓔtse˩pʰæ˧˥}\formedesurface{tse˩pʰæ˧˥}\newline
\classe{名词}\ton{LM+MH\#}
\paradigme{\pcmn{:} \p{}}
\begin{définition}\peng{Coins of the imperial times.}\end{définition}
\begin{définition}\pcmn{民国之前的货币}\end{définition}
\begin{définition}\pfra{Pièces de l'époque impériale.}\end{définition}
\begin{exemple}\pnru{æ˧-tse˥pʰæ˩}\hspace{5pt}\peng{bronze coins of the imperial times}\hspace{5pt}\pcmn{民国之前的铜币}\hspace{5pt}\pfra{pièce en bronze de l'époque impériale}\end{exemple}
\end{entrée}

\begin{entrée}
{tse˩qwæ˧˥}{}{ⓔtse˩qwæ˧˥}\formedesurface{tse˩qwæ˧˥}\newline
\classe{名词}\ton{LM+MH\#}
\paradigme{\pcmn{:} \p{}}
\begin{définition}\peng{Key.}\end{définition}
\begin{définition}\pcmn{钥匙}\end{définition}
\begin{définition}\pfra{Clef.}\end{définition}
\end{entrée}

\begin{entrée}
{tse˩tɑ˧˥}{}{ⓔtse˩tɑ˧˥}\formedesurface{tse˩tɑ˧˥}\newline
\classe{名词}\ton{LM+MH\#}
\paradigme{\pcmn{:} \p{}}
\begin{définition}\peng{Scissors.}\end{définition}
\begin{définition}\pcmn{剪刀}\end{définition}
\begin{définition}\pfra{Ciseaux.}\end{définition}
\end{entrée}

\begin{entrée}
{tse˩ʈʂʰv̩˩}{}{ⓔtse˩ʈʂʰv̩˩}\formedesurface{tse˩ʈʂʰv̩˩˥}\newline
\classe{名词}\ton{L}\begin{définition}\peng{Derogatory term of address for a dog.}\end{définition}
\begin{définition}\pcmn{骂狗的话}\end{définition}
\begin{définition}\pfra{Sac à puces: terme d'insulte pour un chien.}\end{définition}
\end{entrée}

\begin{entrée}
{tse˩ʈʂʰv̩˩-kʰv̩˥}{}{ⓔtse˩ʈʂʰv̩˩-kʰv̩˥}\formedesurface{tse˩ʈʂʰv̩˩kʰv̩˥}\newline
\classe{名词}\ton{L+H\#}\begin{définition}\peng{Derogatory term of address for a dog.}\end{définition}
\begin{définition}\pcmn{骂狗的话}\end{définition}
\begin{définition}\pfra{Sac à puces: terme d'insulte pour un chien.}\end{définition}
\begin{exemple}\pnru{tse˩ʈʂʰv̩˩-kʰv̩˧ ! | mv̩˩tɕo˧ se˥ !}\hspace{5pt}\peng{Come down, you damn dog!}\hspace{5pt}\pcmn{你这坏狗,下去!}\hspace{5pt}\pfra{Descends, sac à puces! (Injonction adressée à un chien qui s'aventurait dans la partie haute de la salle à manger)}\end{exemple}
\end{entrée}

\begin{entrée}
{tsɤ˧}{₁}{ⓔtsɤ˧ⓗ1}\formedesurface{tsɤ˧}\newline
\classe{动词}\ton{M intrans}
1\begin{définition}\peng{To become, tu turn into; to be.}\end{définition}
\begin{définition}\pcmn{形成,变成}\end{définition}
\begin{définition}\pfra{Se transformer, créer, devenir; être.}\end{définition}
\begin{exemple}\pnru{sɯ˧pv̩˧-sɯ˥nɑ˩-ʈʂʰɯ˩ | ə˧dzɤ˧∼dzɤ˥-zo˩ | pʰi˧li˩ tsɤ˩-ɲi˩-kv̩˩-tsɯ˩ ◊ -mv̩˩!}\hspace{5pt}\peng{The caterpillar gradually becomes a butterfly, doesn't it!}\hspace{5pt}\pcmn{毛虫能慢慢变成蝴蝶,不是吗?}\hspace{5pt}\pfra{la chenille devient peu à peu papillon!}\end{exemple}
\begin{exemple}\pnru{ɖɯ˧-bæ˧ mɤ˧-tsɤ˧}\hspace{5pt}\peng{It's not the same}\hspace{5pt}\pcmn{有区别、不一样}\hspace{5pt}\pfra{ce n'est pas la même chose, ce n'est pas pareil}\end{exemple}
\end{entrée}

\begin{entrée}
{tsɤ˧}{₂}{ⓔtsɤ˧ⓗ2}\formedesurface{tsɤ˧}\newline
\classe{形容词}\ton{M}
2\begin{définition}\peng{Suitable, correct.}\end{définition}
\begin{définition}\pcmn{对,合适,成}\end{définition}
\begin{définition}\pfra{Correct, qui va bien.}\end{définition}
\begin{exemple}\pnru{(le˧-)tsɤ˧-ze˧!}\hspace{5pt}\peng{It's okay! / It's arranged! / Things have been made good!}\hspace{5pt}\pcmn{好了! / 弄好了! / 成!}\hspace{5pt}\pfra{c'est bon, c'est arrangé!}\end{exemple}
\begin{exemple}\pnru{tsɤ˧-ʝi˧!}\hspace{5pt}\peng{Okay, fine! (Indication of acquiescence to an instruction)}\hspace{5pt}\pcmn{行! / 好的!(表示同意或接受命令)}\hspace{5pt}\pfra{OK! C'est bon! (formule très courante, pour indiquer son acquiescement à une instruction reçue)}\end{exemple}
\begin{exemple}\pnru{tsɤ˧ ɲi˥!}\hspace{5pt}\peng{That's fine!}\hspace{5pt}\pcmn{好的!}\hspace{5pt}\pfra{c'est bon!}\end{exemple}
\begin{exemple}\pnru{no˧ | mɤ˧-bi˧ mɤ˧-tsɤ˧!}\hspace{5pt}\peng{It wouldn't be right for you not to go!}\hspace{5pt}\pcmn{你如果不去,就不对! =你不能不去!}\hspace{5pt}\pfra{Tu ne peux pas ne pas y aller!, littéralement «que tu n'y ailles pas, ça ne va pas!»}\end{exemple}
\begin{exemple}\pnru{ʈʂʰɯ˧ | ɖɯ˧-pi˧˥ | mɤ˧-tsɤ˧!}\hspace{5pt}\peng{He is not quite OK! / There's something wrong with him!}\hspace{5pt}\pcmn{他有一点不对劲吧!}\hspace{5pt}\pfra{Lui, il est pas très net! / Y'a quelque chose qui va pas chez lui!}\end{exemple}
\begin{exemple}\pnru{tsɤ˧ mɤ˧-ʝi˧-ze˧!}\hspace{5pt}\peng{It won't do! / It won't work! / It's no good!}\hspace{5pt}\pcmn{不好了!/不行了!}\hspace{5pt}\pfra{Ca ne va plus!}\end{exemple}
\begin{exemple}\pnru{hĩ˧-ɳɯ˩ | le˧-so˩, | tsɤ˧!}\hspace{5pt}\peng{When people teach you something, it's fortunate / it's good / it's an opportunity to seize! (Context: discussing the behaviour of someone who would not listen to good advice.)}\hspace{5pt}\pcmn{人家教,是好事! / 人家教,是要珍惜的! / 有人愿意教你,是件好事!}\hspace{5pt}\pfra{Quand on t'apprend quelque chose, c'est une chance à saisir ! / Quand il se trouve quelqu'un qui est disposé à t'apprendre quelque chose, c'est une chance à saisir! / Si tu écoutes les bons conseils, tout ira bien! (Contexte: on évoque quelqu'un qui n'est pas enclin à écouter les bons conseils: qui se braque quand on lui fournit d'utiles conseils.)}\end{exemple}
\begin{exemple}\pnru{hĩ˧-ɳɯ˩ | le˧-so˩, | tsɤ˧-kv˧˥!}\hspace{5pt}\peng{as above}\hspace{5pt}\pcmn{同上}\hspace{5pt}\pfra{même sens}\end{exemple}
\begin{exemple}\pnru{mɤ˧-tsɤ˧-ze˧! | tʰɑ˧-ʐwɤ˩-tso˩ ɲi˩ mæ˩!}\hspace{5pt}\peng{Aïe, aïe, aïe, je regrette / j'ai fait erreur / j'ai fait une bêtise! C'est quelque chose que je n'aurais pas dû dire! (Contexte: quelqu'un s'est emporté pendant une conversation; plus tard, la colère fait place à des regrets. Cette formule, |fv{mɤ˧-tsɤ˧-ze˧}, est le plus proche équivalent proposé par F4 en langue na pour dire le regret sans employer le mot chinois.)}\end{exemple}
\end{entrée}

\begin{entrée}
{tsɤ˧}{₃}{ⓔtsɤ˧ⓗ3}\formedesurface{tsɤ˧}\newline
\classe{形容词}\ton{M}
3\begin{définition}\peng{Fine (powder).}\end{définition}
\begin{définition}\pcmn{细(粉状)}\end{définition}
\begin{définition}\pfra{Fine (poudre).}\end{définition}
\begin{exemple}\pnru{tsɑ˧bɤ˧ tsɤ\#˥}\hspace{5pt}\peng{fine flour}\hspace{5pt}\pcmn{细粮}\hspace{5pt}\pfra{farine fine}\end{exemple}
\end{entrée}

\begin{entrée}
{tsɤ˧}{₄}{ⓔtsɤ˧ⓗ4}\formedesurface{tsɤ˧}\newline
\classe{形容词}\ton{M}
4\begin{définition}\peng{Greedy.}\end{définition}
\begin{définition}\pcmn{嘴馋}\end{définition}
\begin{définition}\pfra{Gourmand.}\end{définition}
\end{entrée}

\begin{entrée}
{tsɤ˧di˧}{}{ⓔtsɤ˧di˧}\formedesurface{tsɤ˧di˧}\newline
\classe{名词}\ton{M}\begin{définition}\peng{Sandalwood, sandlewood; a tall tree, not a shrub.}\end{définition}
\begin{définition}\pcmn{香木}\end{définition}
\begin{définition}\pfra{Arbre à épice, arbre à encens de grande taille.}\end{définition}
\begin{exemple}\pnru{tsɤ˧di˧-dzi˩}\hspace{5pt}\peng{same meaning}\hspace{5pt}\pcmn{同上}\hspace{5pt}\pfra{même sens}\end{exemple}
\end{entrée}

\begin{entrée}
{tsɤ˧ɖɯ˧}{}{ⓔtsɤ˧ɖɯ˧}\formedesurface{tsɤ˧ɖɯ˧}\newline
\classe{动词}\begin{définition}\peng{To give birth (cattle).}\end{définition}
\begin{définition}\pcmn{生崽子(牛类)}\end{définition}
\begin{définition}\pfra{Mettre bas (bovidé).}\end{définition}
\begin{exemple}\pnru{tsɤ˧ɖɯ˧-ze˩}\hspace{5pt}\peng{|fg{pfv}}\hspace{5pt}\pcmn{生崽子了}\hspace{5pt}\pfra{|fg{pfv}}\end{exemple}
\begin{exemple}\pnru{(dʑi˧mi˧) tsɤ˧ɖɯ˧-ze˩}\hspace{5pt}\peng{(the water buffalo) has given birth.}\hspace{5pt}\pcmn{水牛生崽子了。}\hspace{5pt}\pfra{(le buffle) a enfanté.}\end{exemple}
\end{entrée}

\begin{entrée}
{tsɤ˩pʰv̩˧-tsɤ˥li˩}{}{ⓔtsɤ˩pʰv̩˧-tsɤ˥li˩}\formedesurface{tsɤ˩pʰv̩˧tsɤ˥li˩}\newline
\classe{动词}\ton{LM+\#H-}
\sens{1}
\begin{définition}\peng{To turn in all directions, to turn and twist.}\end{définition}
\begin{définition}\pcmn{东翻西滚}\end{définition}
\begin{définition}\pfra{(se) tourner en tous sens, (se) retourner dans tous les sens.}\end{définition}
\begin{exemple}\pnru{ʈʂʰɯ˧ | tsɤ˩pʰv̩˧-tsɤ˥li˩-ɻ̍˩!}\hspace{5pt}\peng{(S)he is turning and twisting in all directions!}\hspace{5pt}\pcmn{他在东翻西滚!}\hspace{5pt}\pfra{il/elle se retourne en tous sens! (ex.: un enfant qui se roule par terre)}\end{exemple}\sens{2}
\begin{définition}\peng{To be two-faced, to be double-dealing, deceitful, hypocritical.}\end{définition}
\begin{définition}\pcmn{伪善}\end{définition}
\begin{définition}\pfra{Être hypocrite, faux-jeton.}\end{définition}
\end{entrée}

\begin{entrée}
{tsɤ˧ʁo˧-tsʰi˧ʁo˥}{}{ⓔtsɤ˧ʁo˧-tsʰi˧ʁo˥}\formedesurface{tsɤ˧ʁo˧tsʰi˧ʁo˥}\newline
\classe{形容词}\ton{H\#}\begin{définition}\peng{Greedy.}\end{définition}
\begin{définition}\pcmn{馋}\end{définition}
\begin{définition}\pfra{Gourmand.}\end{définition}
\begin{exemple}\pnru{tsɤ˧ʁo˧-tsʰi˧ʁo˥ tsʰi˩}\hspace{5pt}\peng{to be greedy}\hspace{5pt}\pcmn{馋}\hspace{5pt}\pfra{être gourmand}\end{exemple}
\end{entrée}

\begin{entrée}
{tsɤ˧ʑi˧}{}{ⓔtsɤ˧ʑi˧}\newline
\classe{动词}\begin{définition}\peng{To be pregnant.}\end{définition}
\begin{définition}\pcmn{怀孕}\end{définition}
\begin{définition}\pfra{Être grosse, être enceinte.}\end{définition}
\end{entrée}

\begin{entrée}
{tsi˥}{}{ⓔtsi˥}\newline
\classe{名词}
\sens{1}
\begin{relationsémantique}\{
renvoi
tsi˧gi˥\$
}\end{relationsémantique}\paradigme{\pcmn{:} \p{}}
\begin{définition}\peng{Crack.}\end{définition}
\begin{définition}\pcmn{裂缝、缝隙}\end{définition}
\begin{définition}\pfra{Fissure, interstice.}\end{définition}
\begin{exemple}\pnru{tsi˧ qʰwæ˧-ze˥!}\hspace{5pt}\peng{A crack has appeared!}\hspace{5pt}\pcmn{有了裂缝!}\hspace{5pt}\pfra{il y a une fissure qui s'est faite! / ça s'est fendu! to crack; to develop a chink/crack/fissure}\end{exemple}
\begin{exemple}\pnru{tsi˧ hɯ˧-ze˧!}\hspace{5pt}\peng{A crack has appeared!}\hspace{5pt}\pcmn{有了裂缝!}\hspace{5pt}\pfra{ça s'est fissuré!}\end{exemple}\sens{2}
\begin{définition}\peng{Stitch.}\end{définition}
\begin{définition}\pcmn{针脚}\end{définition}
\begin{définition}\pfra{Couture.}\end{définition}
\end{entrée}

\begin{entrée}
{tsi˧α}{}{ⓔtsi˧α}\formedesurface{tsi˧}\newline
\classe{形容词}\ton{Mα}\begin{définition}\peng{Spicy.}\end{définition}
\begin{définition}\pcmn{辣}\end{définition}
\begin{définition}\pfra{Piquant, pimenté.}\end{définition}
\begin{exemple}\pnru{ʈʂʰɯ˧ tsi˧-hĩ˧ ɲi˥!}\hspace{5pt}\peng{It's spicy!}\hspace{5pt}\pcmn{这是辣的!}\hspace{5pt}\pfra{c'est piquant/c'est pimenté!}\end{exemple}
\end{entrée}

\begin{entrée}
{tsi˧β}{}{ⓔtsi˧β}\formedesurface{tsi˧}\newline
\classe{动词}\ton{Mβ}\begin{définition}\peng{To set up, to install.}\end{définition}
\begin{définition}\pcmn{安装}\end{définition}
\begin{définition}\pfra{Fixer, installer, mettre en place (ex.: un pilier, dans une maison en construction).}\end{définition}
\begin{exemple}\pnru{tso˧∼tso˧ tsi˧}\hspace{5pt}\peng{to set up something}\hspace{5pt}\pcmn{安装东西}\hspace{5pt}\pfra{installer quelque chose}\end{exemple}
\begin{exemple}\pnru{tso˧∼tso˧ | gɤ˩-tsi˧-ɻ̍˥}\hspace{5pt}\peng{to set up something}\hspace{5pt}\pcmn{安装东西}\hspace{5pt}\pfra{installer quelque chose, mettre quelque chose en place}\end{exemple}
\begin{exemple}\pnru{gɤ˩-tsi˧ tʰi˧-tɕɯ˥}\hspace{5pt}\peng{to set up (vertically)}\hspace{5pt}\pcmn{立起来}\hspace{5pt}\pfra{(re)lever, (re)mettre d'aplomb}\end{exemple}
\end{entrée}

\begin{entrée}
{tsi˩α}{}{ⓔtsi˩α}\formedesurface{tsi˩˥}\newline
\classe{动词}\ton{Lα}\begin{définition}\peng{To boil, to bring to a boil.}\end{définition}
\begin{définition}\pcmn{烧开}\end{définition}
\begin{définition}\pfra{Faire bouillir.}\end{définition}
\begin{exemple}\pnru{dʑɯ˧ | le˧-tsi˩-tʰv̩˩-ze˩!}\hspace{5pt}\peng{The water is boiling!}\hspace{5pt}\pcmn{水开了!}\hspace{5pt}\pfra{L'eau bout!}\end{exemple}
\begin{exemple}\pnru{dʑɯ˩ tsi˩-tʰv̩˩-ze˥!}\hspace{5pt}\peng{The water is boiling!}\hspace{5pt}\pcmn{水开了!}\hspace{5pt}\pfra{L'eau bout!}\end{exemple}
\begin{exemple}\pnru{mɤ˧-tsi˩-tʰv̩˩-sɯ˩!}\hspace{5pt}\peng{It is not boiling yet!}\hspace{5pt}\pcmn{还没有烧开!}\hspace{5pt}\pfra{Ca ne bout pas encore!}\end{exemple}
\begin{exemple}\pnru{ɖɯ˧-tsi˩-tʰv̩˩-ɻ̍˩-kʰɯ˩}\hspace{5pt}\peng{to leave to boil for a while}\hspace{5pt}\pcmn{煮一会儿}\hspace{5pt}\pfra{laisser bouillir un moment}\end{exemple}
\begin{exemple}\pnru{ɖɯ˧-tsi˧∼tsi˥-ɻ̍˩ kʰɯ˩}\hspace{5pt}\peng{to boil a while}\hspace{5pt}\pcmn{煮一会儿}\hspace{5pt}\pfra{faire bouillir un moment}\end{exemple}
\end{entrée}

\begin{entrée}
{tsi˧gi˥\$}{}{ⓔtsi˧gi˥\$}\formedesurface{tsi˧gi˥}\newline
\classe{名词}\ton{H\$}
\paradigme{\pcmn{:} \p{}}
\begin{définition}\peng{Crack, fissure.}\end{définition}
\begin{définition}\pcmn{缝隙,例如:墙上的}\end{définition}
\begin{définition}\pfra{Fissure.}\end{définition}
\begin{exemple}\pnru{tsi˧gi˥ | ɖɯ˧-kʰwɤ˥}\hspace{5pt}\peng{a fissure}\hspace{5pt}\pcmn{一个缝隙}\hspace{5pt}\pfra{une fissure}\end{exemple}
\begin{exemple}\pnru{tsi˧gi˥ | ɖɯ˧-kʰwɤ˧ tʰi˧-di˥}\hspace{5pt}\peng{there is a fissure}\hspace{5pt}\pcmn{有一个缝隙}\hspace{5pt}\pfra{il y a une fissure}\end{exemple}
\end{entrée}

\begin{entrée}
{tsi˩ɭɯ˩}{}{ⓔtsi˩ɭɯ˩}\formedesurface{tsi˩ɭɯ˩˥}\newline
\classe{名词}\ton{L}
\paradigme{\pcmn{:} \p{}}
\begin{définition}\peng{A species of small bird.}\end{définition}
\begin{définition}\pcmn{一种小鸟}\end{définition}
\begin{définition}\pfra{Petit oiseau de couleur bleue/verte.}\end{définition}
\end{entrée}

\begin{entrée}
{tsi˩ɭɯ˩-bv̩˥ | -qʰæ˩bæ˩˥}{}{ⓔtsi˩ɭɯ˩-bv̩˥ | -qʰæ˩bæ˩˥}\formedesurface{tsi˩ɭɯ˩bv̩˥qʰæ˩bæ˩˥}\newline
\classe{名词}\ton{L+H\#|L}\begin{définition}\peng{Trailing plant.}\end{définition}
\begin{définition}\pcmn{蔓草}\end{définition}
\begin{définition}\pfra{Herbe rampante.}\end{définition}
\end{entrée}

\begin{entrée}
{‑tso}{}{ⓔ‑tso}\formedesurface{--}\newline
\classe{后缀}\ton{?}\begin{définition}\peng{Nominalizer.}\end{définition}
\begin{définition}\pcmn{名物化}\end{définition}
\begin{définition}\pfra{Nominalizer.}\end{définition}
\begin{exemple}\pnru{bæ˩-ʂo˧-tso˧ tʰi˧-tʰv̩˧-ho˩-ze˩!}\hspace{5pt}\peng{The crop is going to reach maturity!}\hspace{5pt}\pcmn{庄稼快要熟了!}\hspace{5pt}\pfra{la récolte va parvenir à terme/ les produits de la récolte seront bientôt mûrs!}\end{exemple}
\begin{exemple}\pnru{ʈʰɯ˩-tso˩˥}\hspace{5pt}\peng{Liquid that can be drunk; the meaning is not entirely identical with /ʈʰɯ˩-di˩˥/ ‘drink'.}\hspace{5pt}\pcmn{喝的东西}\hspace{5pt}\pfra{Chose liquide qui se boit; pas exactement identique avec /ʈʰɯ˩-di˩˥/ ‘chose à boire, boisson'.}\end{exemple}
\begin{exemple}\pnru{v̩˩dʑɯ˩˥, | ʈʰɯ˩-tso˩ ɲi˥!}\hspace{5pt}\peng{Soup is liquid / is something that one drinks! / Soup counts among liquids!}\hspace{5pt}\pcmn{汤,是喝的东西! / 汤是来喝的! / 汤,算是属于饮料类的!}\hspace{5pt}\pfra{la soupe, c'est liquide/ça se boit!}\end{exemple}
\begin{exemple}\pnru{dʑɤ˩bv̩˧-tso˩}\hspace{5pt}\peng{distraction, game, amusement, piece of entertainment; the meaning is distinct from /dʑɤ˩bv̩˧-di˩/ ‘toy', which refers specifically to objects}\hspace{5pt}\pcmn{娱乐、游戏}\hspace{5pt}\pfra{distraction, jeu, divertissement; le sens est distinct d'une autre forme nominalisée, /dʑɤ˩bv̩˧-di˩/ ‘jouet' (objet)}\end{exemple}
\begin{exemple}\pnru{ʝi˧-tso˧ mɤ˧-ʂv̩˧ɖv̩˧, | dʑɤ˩bv̩˧-tso˩-lɑ˩ ʂv̩˩ɖv̩˩!}\hspace{5pt}\peng{(He) does not think about his duties, only about his amusements/his pleasures!}\hspace{5pt}\pcmn{不考虑任务,只想到娱乐! / 干正经事不想,只想玩!}\hspace{5pt}\pfra{(il) ne pense pas à ses tâches/à ses devoirs/à ce qu'il doit faire, il ne pense qu'à ses distractions!}\end{exemple}
\begin{exemple}\pnru{tv̩˧tso˧}\hspace{5pt}\peng{thing to be planted; crop. (Another nominalized form exists: /tv̩˧-di˩/, but there is homonymy with the compound made of ‘to plant' and ‘earth': /tv̩˧-di˩/, meaning ‘cultivable land, cultivable soil'. The conflict is resolved by using /tv̩˧-tso˧/ rather than /tv̩˧-di˩/ to mean ‘thing to plant, thing that can be planted'.}\hspace{5pt}\pcmn{种的东西=农作物。存在有另外一种名物化:|fv{/tv̩˧-di˩/},可是那个与‘可以种的地、农业土地’|fv{/tv̩˧-di˩/}同音,于是用|fv{/tv̩˧-tso˧/}来表示‘农作物’而不用|fv{/tv̩˧-di˩/}。}\hspace{5pt}\pfra{chose que l'on plante. (La forme nominalisée /tv̩˧-di˩/ existe aussi mais il y a un conflit homophonique entre ‘planter'+'terre', /tv̩˧-di˩/, et le suffixe /-di/ ‘nominalisateur', de sorte que pour dire ‘chose qu'on plante', on ne dit pas /tv̩˧-di˩/, mais /tv̩˧-tso˧/.}\end{exemple}
\end{entrée}

\begin{entrée}
{‑tso˧α}{}{ⓔ‑tso˧α}\formedesurface{tso˧}\newline
\classe{后缀}\ton{M}\begin{définition}\peng{|fg{volitive.}}\end{définition}
\begin{définition}\pcmn{意志}\end{définition}
\begin{définition}\pfra{|fg{volitif.}}\end{définition}
\begin{exemple}\pnru{dʑɤ˩bv̩˥-tso˩-ɲi˩-mæ˩!}\hspace{5pt}\peng{Let's go and play!}\hspace{5pt}\pcmn{玩一玩吧!}\hspace{5pt}\pfra{On joue, d'accord? / Allez, on va jouer!}\end{exemple}
\begin{exemple}\pnru{ə˧tso˧ tv̩˧-tso˧-ɲi˥ ?}\hspace{5pt}\peng{What do you plan to plant? / Which crop are you going to plant?}\hspace{5pt}\pcmn{要种什么东西?}\hspace{5pt}\pfra{Qu'est-ce que vous comptez planter?}\end{exemple}
\begin{exemple}\pnru{ə˧tso˧ ʝi˧-bi˧-tso˧-ɲi˥?}\hspace{5pt}\peng{What do you plan to do now?}\hspace{5pt}\pcmn{要做什么了?}\hspace{5pt}\pfra{Qu'est-ce que vous comptez faire maintenant?}\end{exemple}
\end{entrée}

\begin{entrée}
{tso˩α}{}{ⓔtso˩α}\formedesurface{tso˩˥}\newline
\classe{动词}\ton{Lα}\begin{définition}\peng{To build a wall, a bridge…}\end{définition}
\begin{définition}\pcmn{砌(墙)、建(桥梁)}\end{définition}
\begin{définition}\pfra{Construire un mur, un pont…}\end{définition}
\begin{exemple}\pnru{ɖʐɤ˩ tso˩˥}\hspace{5pt}\peng{to build stairs}\hspace{5pt}\pcmn{修一座楼梯}\hspace{5pt}\pfra{construire un escalier}\end{exemple}
\begin{exemple}\pnru{dzo˩ tso˩˥}\hspace{5pt}\peng{to build a bridge}\hspace{5pt}\pcmn{修一座桥}\hspace{5pt}\pfra{construire un pont}\end{exemple}
\begin{exemple}\pnru{ʐɤ˩mi˩ tso˥}\hspace{5pt}\peng{to build a road}\hspace{5pt}\pcmn{修一条路}\hspace{5pt}\pfra{construire une route}\end{exemple}
\begin{exemple}\pnru{qʰæ˧lo˧ tso˥}\hspace{5pt}\peng{to dig a ditch, to make a ditch}\hspace{5pt}\pcmn{挖水沟、修一条水沟}\hspace{5pt}\pfra{creuser un canal, construire un canal}\end{exemple}
\begin{exemple}\pnru{ɖɯ˧-tso˧∼tso˥-ɻ̍˩}\hspace{5pt}\peng{to build something}\hspace{5pt}\pcmn{修东西}\hspace{5pt}\pfra{construire quelque chose}\end{exemple}
\end{entrée}

\begin{entrée}
{tso˩γ}{}{ⓔtso˩γ}\formedesurface{ɖɯ˧ tso˩}\newline
\classe{量词}\ton{Lγ}\begin{définition}\peng{Classifier for rooms.}\end{définition}
\begin{définition}\pcmn{量词:间(房间,分隔间,包间)}\end{définition}
\begin{définition}\pfra{Classificateur des pièces (dans la maison), des compartiments (dans un grenier).}\end{définition}
\begin{exemple}\pnru{ʈʂʰɯ˧-tso˥}\hspace{5pt}\peng{this room}\hspace{5pt}\pcmn{这间}\hspace{5pt}\pfra{cette pièce}\end{exemple}
\end{entrée}

\begin{entrée}
{tso˧kʰwɤ\#˥}{}{ⓔtso˧kʰwɤ\#˥}\formedesurface{tso˧kʰwɤ˧}\newline
\classe{名词}\ton{\#H}
\paradigme{\pcmn{:} \p{}}
\begin{définition}\peng{Bag (of fabric or leather).}\end{définition}
\begin{définition}\pcmn{袋子}\end{définition}
\begin{définition}\pfra{Sac (était fait en toile ou en cuir).}\end{définition}
\end{entrée}

\begin{entrée}
{tso˧lo˧-mv̩˥tso˩}{}{ⓔtso˧lo˧-mv̩˥tso˩}\formedesurface{tso˧lo˧mv̩˥tso˩}\newline
\classe{名词}\ton{\#H-}
\paradigme{\pcmn{:} \p{}}
\begin{définition}\peng{Tool; thing, object.}\end{définition}
\begin{définition}\pcmn{东西,工具}\end{définition}
\begin{définition}\pfra{Outil; chose, objet, truc.}\end{définition}
\end{entrée}

\begin{entrée}
{tso˧qwɤ˧}{}{ⓔtso˧qwɤ˧}\formedesurface{tso˧qwɤ˧}\newline
\classe{名词}\ton{M}
\paradigme{\pcmn{:} \p{}}
\begin{définition}\peng{Sleeping corner: a part of the main room where there is bedding; some people can sit there during meals or family reunions. Newborn babies sleep there. After a decease, corpses are placed on that bed.}\end{définition}
\begin{définition}\pcmn{小床角:主屋里面的一个角落,有床垫子。用餐、招待客人的时候,会有人在上面坐。刚出生的婴儿也在此处睡觉。人去世后,尸体先放在那个地方。}\end{définition}
\begin{définition}\pfra{Chambrette: partie de la pièce principale dans laquelle se trouve un couchage; on y place provisoirement les nouveaux-nés, et les défunts.}\end{définition}
\end{entrée}

\begin{entrée}
{tso˩qʰv̩˩ɻ̍˥}{}{ⓔtso˩qʰv̩˩ɻ̍˥}\formedesurface{tso˩qʰv̩˩ɻ̍˥}\newline
\classe{名词}\ton{H\#}
\paradigme{\pcmn{:} \p{}}
\begin{définition}\peng{Porch, enclosed porch, vestibule: a narrow area between the door and the courtyard, covered by the roof (and hence sheltered from rain), and, in some houses, shut off from the coutyard by a wall with one door approximately in the middle. This porch is the area that one reaches when crossing the threshold, coming out from the main room. In the main consultant's house, where the porch is not enclosed, it is exposed to sunshine until the afternoon, and tasks such as chopping vegetables are carried out sitting in this area.}\end{définition}
\begin{définition}\pcmn{玄关、门厅}\end{définition}
\begin{définition}\pfra{Porche, vestibule: espace situé entre la cour et la pièce principale, c'est-à-dire l'espace, protégé de la pluie par la toiture, où l'on parvient lorsqu'on passe le seuil en sortant de la pièce principale. Dans certaines maisons, ce porche est séparé de la cour par une cloison de bois, percée d'une porte à peu près au milieu de sa longueur.}\end{définition}
\end{entrée}

\begin{entrée}
{tso˧tɕɤ˩}{}{ⓔtso˧tɕɤ˩}\formedesurface{tso˧tɕɤ˩}\newline
\classe{连词}\ton{L\#}\begin{définition}\peng{religion}\end{définition}
\begin{définition}\pcmn{宗教(汉语借词)}\end{définition}
\begin{définition}\pfra{religion}\end{définition}
\end{entrée}

\begin{entrée}
{tso˧∼tso\#˥}{}{ⓔtso˧∼tso\#˥}\formedesurface{tso˧tso˧}\newline
\classe{名词}\ton{\#H}
\paradigme{\pcmn{:} \p{}}
\begin{définition}\peng{Thing, thingummy, stuff.}\end{définition}
\begin{définition}\pcmn{东西}\end{définition}
\begin{définition}\pfra{Chose, truc, bidule, objet, machin.}\end{définition}
\begin{exemple}\pnru{tso˧∼tso˧-zo˧∼zo˧-mv̩˧∼mv̩˥}\hspace{5pt}\peng{thingummy, stuff}\hspace{5pt}\pcmn{各种东西、各种各样乱七八糟东西}\hspace{5pt}\pfra{bidule}\end{exemple}
\begin{exemple}\pnru{tso˧∼tso˧ hwæ˩}\hspace{5pt}\peng{to buy things}\hspace{5pt}\pcmn{买东西}\hspace{5pt}\pfra{acheter des choses}\end{exemple}
\begin{exemple}\pnru{tso˧∼tso˧ tɕʰi˧(-ze˩)}\hspace{5pt}\peng{to sell things}\hspace{5pt}\pcmn{卖东西}\hspace{5pt}\pfra{vendre des choses}\end{exemple}
\begin{exemple}\pnru{tso˧∼tso˧ dzɯ˧(-ze˩)}\hspace{5pt}\peng{to eat things}\hspace{5pt}\pcmn{吃东西}\hspace{5pt}\pfra{manger des choses}\end{exemple}
\begin{exemple}\pnru{tso˧∼tso˧ dze˥}\hspace{5pt}\peng{to cut things}\hspace{5pt}\pcmn{切东西}\hspace{5pt}\pfra{couper des choses}\end{exemple}
\begin{exemple}\pnru{tso˧∼tso˧ ʈʰɯ˩}\hspace{5pt}\peng{to drink things}\hspace{5pt}\pcmn{喝东西}\hspace{5pt}\pfra{boire des choses}\end{exemple}
\begin{exemple}\pnru{tso˧∼tso˧ lɑ˩}\hspace{5pt}\peng{to beat things}\hspace{5pt}\pcmn{打东西}\hspace{5pt}\pfra{frapper des choses}\end{exemple}
\end{entrée}

\begin{entrée}
{tso˩∼tso˧˥}{}{ⓔtso˩∼tso˧˥}\formedesurface{tso˩tso˧˥}\newline
\classe{动词}\ton{MH}\begin{définition}\peng{To mix, to prepare (dog food).}\end{définition}
\begin{définition}\pcmn{拌好狗食}\end{définition}
\begin{définition}\pfra{Touiller, faire la pâtée (du chien…).}\end{définition}
\end{entrée}

\begin{entrée}
{tsɯ˥}{₁}{ⓔtsɯ˥ⓗ1}\newline
\classe{动词}
1
\sens{1}
\begin{définition}\peng{To tie (with a rope).}\end{définition}
\begin{définition}\pcmn{绑、捆、栓}\end{définition}
\begin{définition}\pfra{Attacher.}\end{définition}
\begin{exemple}\pnru{dʑi˧mi˧ tʰi˧-tsɯ˥}\hspace{5pt}\peng{to tie a water buffalo}\hspace{5pt}\pcmn{栓水牛}\hspace{5pt}\pfra{attacher le buffle}\end{exemple}
\begin{exemple}\pnru{tsɯ˧∼tsɯ˧}\hspace{5pt}\peng{|fg{red}}\hspace{5pt}\pcmn{重叠}\hspace{5pt}\pfra{|fg{red}}\end{exemple}
\begin{exemple}\pnru{le˧-tsɯ˧∼tsɯ˧}\hspace{5pt}\peng{|fg{accomp} |fg{red}}\hspace{5pt}\pcmn{|fg{accomp} |fg{red}}\hspace{5pt}\pfra{|fg{accomp} |fg{red}}\end{exemple}
\begin{exemple}\pnru{tʰi˧-tsɯ˧∼tsɯ˧}\hspace{5pt}\peng{|fg{dur} |fg{red}}\hspace{5pt}\pcmn{|fg{dur} |fg{red}}\hspace{5pt}\pfra{|fg{dur} |fg{red}}\end{exemple}\sens{2}
\begin{définition}\peng{To hang oneself.}\end{définition}
\begin{définition}\pcmn{上吊自杀、缢}\end{définition}
\begin{définition}\pfra{Se pendre.}\end{définition}
\begin{exemple}\pnru{ʁæ˧tsɯ˧ le˧-ʂɯ˧ +ze˧}\hspace{5pt}\peng{to hang oneself}\hspace{5pt}\pcmn{上吊自杀、缢}\hspace{5pt}\pfra{se pendre}\end{exemple}
\end{entrée}

\begin{entrée}
{tsɯ˥}{₂}{ⓔtsɯ˥ⓗ2}\formedesurface{tsɯ˧}\newline
\classe{动词}\ton{H}
2\begin{définition}\peng{Dredge for, fish for, scoop up out of water.}\end{définition}
\begin{définition}\pcmn{打捞}\end{définition}
\begin{définition}\pfra{Prendre avec une écumoire, récupérer dans l'eau.}\end{définition}
\end{entrée}

\begin{entrée}
{tsɯ˧}{₁}{ⓔtsɯ˧ⓗ1}\formedesurface{tsɯ˧}\newline
\classe{名词}\ton{M}
1\begin{définition}\peng{Letter, Chinese character.}\end{définition}
\begin{définition}\pcmn{字}\end{définition}
\begin{définition}\pfra{Lettre, caractère chinois.}\end{définition}
\end{entrée}

\begin{entrée}
{tsɯ˧˥}{}{ⓔtsɯ˧˥}\formedesurface{tsɯ˧˥}\newline
\classe{动词}\ton{MH}\begin{définition}\peng{To call, to give the name of…}\end{définition}
\begin{définition}\pcmn{叫、叫做}\end{définition}
\begin{définition}\pfra{Appeler, nommer, désigner.}\end{définition}
\begin{exemple}\pnru{ʈæ˧ʂɯ˧-ɳɯ˧ | no˧-ki˥ | jɤ˩-ʐe˧ ɲi˩-tsɯ˩}\hspace{5pt}\peng{/ʈæ˧ʂɯ˧/ calls you “foreigner"!}\hspace{5pt}\pcmn{达石把你叫作“老外”!}\hspace{5pt}\pfra{/ʈæ˧ʂɯ˧/ t'appelle «l'étranger» / il te traite d'étranger!}\end{exemple}
\begin{exemple}\pnru{ʈæ˧ʂɯ˧ ɲi˩-tsɯ˩.}\hspace{5pt}\peng{His name is /ʈæ˧ʂɯ˧/. / He is called /ʈæ˧ʂɯ˧/.}\hspace{5pt}\pcmn{他名字叫达石。}\hspace{5pt}\pfra{Son nom est /ʈæ˧ʂɯ˧/. / Il s'appelle /ʈæ˧ʂɯ˧/.}\end{exemple}
\begin{exemple}\pnru{ʈæ˧ʂɯ˧ tsɯ˧˥.}\hspace{5pt}\peng{His name is /ʈæ˧ʂɯ˧/. / He is called /ʈæ˧ʂɯ˧/.}\hspace{5pt}\pcmn{他名字叫达石。}\hspace{5pt}\pfra{Son nom est /ʈæ˧ʂɯ˧/. / Il s'appelle /ʈæ˧ʂɯ˧/.}\end{exemple}
\end{entrée}

\begin{entrée}
{tsɯ˧˥}{}{ⓔtsɯ˧˥}\formedesurface{tsɯ˧˥}\newline
\classe{语气助词}\ton{MH}\begin{définition}\peng{Reported/hearsay evidential: the speaker indicates that they are not in a position to vouch personally for what they are saying, that it is based on indirect knowledge, from hearsay.}\end{définition}
\begin{définition}\pcmn{传闻据素:据说……}\end{définition}
\begin{définition}\pfra{Particule d'évidentialité rapportée: elle indique une connaissance indirecte, par ouï-dire, et non par connaissance directe.}\end{définition}
\end{entrée}

\begin{entrée}
{tsɯ˧α}{₁}{ⓔtsɯ˧αⓗ1}\formedesurface{tsɯ˧}\newline
\classe{名词}\ton{Mα}
1\begin{définition}\peng{To filter.}\end{définition}
\begin{définition}\pcmn{过滤}\end{définition}
\begin{définition}\pfra{Filtrer.}\end{définition}
\begin{exemple}\pnru{ɖɯ˧-ʈʰɤ˥ tsɯ˩}\hspace{5pt}\peng{To filter a drop.}\hspace{5pt}\pcmn{过滤一滴}\hspace{5pt}\pfra{Filtrer une goutte.}\end{exemple}
\end{entrée}

\begin{entrée}
{tsɯ˩α}{}{ⓔtsɯ˩α}\formedesurface{tsɯ˩˥}\newline
\classe{动词}\ton{Lα}\begin{définition}\peng{To block up.}\end{définition}
\begin{définition}\pcmn{堵塞、塞住洞口}\end{définition}
\begin{définition}\pfra{Boucher/être bouché; obstruer (ex.: obstruer l'entrée d'un trou).}\end{définition}
\end{entrée}

\begin{entrée}
{tsɯ˩pʰɤ˩}{}{ⓔtsɯ˩pʰɤ˩}\formedesurface{tsɯ˩pʰɤ˩˥}\newline
\classe{动词}
\sens{1}
\begin{définition}\peng{To blink.}\end{définition}
\begin{définition}\pcmn{眨眼}\end{définition}
\begin{définition}\pfra{Cligner des yeux.}\end{définition}
\begin{exemple}\pnru{mɤ˧-tsɯ˩pʰɤ˩}\hspace{5pt}\peng{|fg{neg}}\hspace{5pt}\pcmn{不眨眼}\hspace{5pt}\pfra{|fg{neg}}\end{exemple}
\begin{exemple}\pnru{ɖɯ˧-tsɯ˧∼tsɯ˥-ɻ̍˩}\hspace{5pt}\peng{|fg{delimitative} |fg{red} |fg{inceptive}}\hspace{5pt}\pcmn{|fg{delimitative} |fg{red} |fg{inceptive}}\hspace{5pt}\pfra{|fg{délimitatif} |fg{red} |fg{inchoatif}}\end{exemple}
\begin{exemple}\pnru{njɤ˩ɭɯ˧ tsɯ˩pʰɤ˩}\hspace{5pt}\peng{to blink}\hspace{5pt}\pcmn{眨眼}\hspace{5pt}\pfra{cligner des yeux}\end{exemple}\sens{2}
\begin{définition}\peng{To wink (as a discreet sign of mutual understanding).}\end{définition}
\begin{définition}\pcmn{眨眼}\end{définition}
\begin{définition}\pfra{Faire un clin d'oeil (discret signe d'intelligence).}\end{définition}
\begin{exemple}\pnru{ʈʂʰɯ˧ | njɤ˩ɭɯ˧ tsɯ˩pʰɤ˩-dʑo˩!}\hspace{5pt}\peng{(S)he is winking!}\hspace{5pt}\pcmn{他在眨眼!}\hspace{5pt}\pfra{Elle/il est en train de faire un clin d'oeil!}\end{exemple}
\end{entrée}

\begin{entrée}
{tsʰɑ˧bo\#˥}{}{ⓔtsʰɑ˧bo\#˥}\formedesurface{tsʰɑ˧bo˧}\newline
\classe{名词}\ton{\#H}\begin{définition}\peng{Cook.}\end{définition}
\begin{définition}\pcmn{厨师}\end{définition}
\begin{définition}\pfra{Cuisinier.}\end{définition}
\begin{exemple}\pnru{tsʰɑ˧bo˧ lɑ˩}\hspace{5pt}\peng{to be a cook, to work as a cook, to get employed as a cook}\hspace{5pt}\pcmn{当厨师}\hspace{5pt}\pfra{être cuisinier, s'engager comme cuisinier, faire le travail de cuisinier}\end{exemple}
\begin{exemple}\pnru{tsʰɑ˧bo˧ ʝi˧}\hspace{5pt}\peng{to be a cook, to work as a cook, to get employed as a cook}\hspace{5pt}\pcmn{当厨师}\hspace{5pt}\pfra{être cuisinier, s'engager comme cuisinier, faire le travail de cuisinier}\end{exemple}
\end{entrée}

\begin{entrée}
{tsʰɑ˧kv̩˩}{}{ⓔtsʰɑ˧kv̩˩}\formedesurface{tsʰɑ˧kv̩˩}\newline
\classe{名词}\ton{L\#}\begin{définition}\peng{Warehouse, storehouse.}\end{définition}
\begin{définition}\pcmn{仓库(汉语借词)}\end{définition}
\begin{définition}\pfra{Réserve, magasin.}\end{définition}
\end{entrée}

\begin{entrée}
{tsʰɑ˩pʰɑ˩lɑ˥}{}{ⓔtsʰɑ˩pʰɑ˩lɑ˥}\formedesurface{tsʰɑ˩pʰɑ˩lɑ˥}\newline
\classe{名词}\ton{L+H\#}\begin{définition}\peng{Husk of sweet corn (maize) cobs.}\end{définition}
\begin{définition}\pcmn{苞谷叶(玉米穰子的叶子)}\end{définition}
\begin{définition}\pfra{Feuilles d'épi de maïs: les feuilles qui entourent l'épi de maïs.}\end{définition}
\begin{exemple}\pnru{qʰɑ˧dze˧-tsʰɑ˩pʰɑ˩lɑ˩}\hspace{5pt}\peng{same meaning}\hspace{5pt}\pcmn{同上}\hspace{5pt}\pfra{même sens}\end{exemple}
\end{entrée}

\begin{entrée}
{tsʰɑ˧tɕɤ˧˥}{}{ⓔtsʰɑ˧tɕɤ˧˥}\formedesurface{tsʰɑ˧tɕɤ˧˥}\newline
\classe{名词}\ton{MH}\begin{définition}\peng{Seedlings.}\end{définition}
\begin{définition}\pcmn{青菜幼苗}\end{définition}
\begin{définition}\pfra{Jeunes pousses, petites pousses qu'on récolte pour les manger.}\end{définition}
\end{entrée}

\begin{entrée}
{tsʰæ˧pʰv˧˥}{}{ⓔtsʰæ˧pʰv˧˥}\formedesurface{tsʰæ˧pʰv˧˥}\newline
\classe{名词}\ton{MH\#}
\paradigme{\pcmn{:} \p{}}
\begin{définition}\peng{Chinese cabbage. This is a calque from the Chinese ‘white vegetable', using the Chinese word for ‘vegetable' in association with the Na word for ‘white'.}\end{définition}
\begin{définition}\pcmn{白菜(借汉语‘白菜’的第二个音节来充当这个名词的第一个音节:按摩梭话句法,形容词在名词后面,跟汉语相反)}\end{définition}
\begin{définition}\pfra{Chou chinois. Il s'agit d'un calque du chinois ‘légume blanc', employant le nom chinois pour ‘légume' associé à l'adjectif na pour ‘blanc'.}\end{définition}
\end{entrée}

\begin{entrée}
{tsʰe˧}{}{ⓔtsʰe˧}\formedesurface{tsʰe˧}\newline
\classe{数词}\ton{M}\begin{définition}\peng{Ten.}\end{définition}
\begin{définition}\pcmn{十}\end{définition}
\begin{définition}\pfra{Dix.}\end{définition}
\end{entrée}

\begin{entrée}
{tsʰe˩β}{}{ⓔtsʰe˩β}\formedesurface{ɖɯ˧ tsʰe˩}\newline
\classe{量词}\ton{Lβ}\begin{définition}\peng{Classifier for knots, e.g. in a braid.}\end{définition}
\begin{définition}\pcmn{量词:数辫子的节(一节)}\end{définition}
\begin{définition}\pfra{Classificateur des nœuds dans une tresse.}\end{définition}
\end{entrée}

\begin{entrée}
{tsʰe˩β}{}{ⓔtsʰe˩β}\formedesurface{ɖɯ˧ tsʰe˩}\newline
\classe{量词}\ton{Lβ}\begin{définition}\peng{Classifier: an inch (1/3 decimeter).}\end{définition}
\begin{définition}\pcmn{量词:寸(汉语借词)}\end{définition}
\begin{définition}\pfra{Pouce (cette unité de mesure n'était pas en usage chez les Na avant son introduction par emprunt au chinois).}\end{définition}
\end{entrée}

\begin{entrée}
{tsʰe\#˥}{}{ⓔtsʰe\#˥}\formedesurface{tsʰe˧}\newline
\classe{名词}\ton{\#H}\begin{définition}\peng{Salt.}\end{définition}
\begin{définition}\pcmn{盐}\end{définition}
\begin{définition}\pfra{Sel.}\end{définition}
\end{entrée}

\begin{entrée}
{tsʰe˧di˧}{}{ⓔtsʰe˧di˧}\formedesurface{tsʰe˧di˧}\newline
\classe{名词}\ton{M}\begin{définition}\peng{Jiaze, a hamlet to the north of Labai}\end{définition}
\begin{définition}\pcmn{拉伯乡加泽村}\end{définition}
\begin{définition}\pfra{Jiaze, un hameau au nord de Labai}\end{définition}
\begin{exemple}\pnru{tsʰe˧di˧-qo˩qɑ˩}\hspace{5pt}\peng{Jiaze Pass.}\hspace{5pt}\pcmn{加泽垭口}\hspace{5pt}\pfra{Le col de Jiaze.}\end{exemple}
\end{entrée}

\begin{entrée}
{tsʰe˧do˧˥}{}{ⓔtsʰe˧do˧˥}\formedesurface{tsʰe˧do˧˥}\newline
\classe{助词}\ton{MH\# | L}\begin{définition}\peng{The beginning of the month.}\end{définition}
\begin{définition}\pcmn{月初}\end{définition}
\begin{définition}\pfra{Début du mois.}\end{définition}
\begin{exemple}\pnru{tsʰe˧do˧-ɖɯ˧ɲi\#˥ / tsʰe˧do˧-ɖɯ˧ɲi˥}\hspace{5pt}\peng{the 1st day of the month}\hspace{5pt}\pcmn{初一}\hspace{5pt}\pfra{le 1er du mois}\end{exemple}
\begin{exemple}\pnru{tsʰe˧do˧-ɲi˧ɲi\#˥ / tsʰe˧do˧-ɲi˧ɲi˥}\hspace{5pt}\peng{the second day of the month}\hspace{5pt}\pcmn{初二}\hspace{5pt}\pfra{le deuxième jour du mois}\end{exemple}
\begin{exemple}\pnru{tsʰe˧do˧˥ | -so˩ɲi˩˥}\hspace{5pt}\peng{the third day of the month}\hspace{5pt}\pcmn{初三}\hspace{5pt}\pfra{le 3e du mois}\end{exemple}
\begin{exemple}\pnru{tsʰe˧do˧-ŋwɤ˥ɲi˩}\hspace{5pt}\peng{the 5th day of the month}\hspace{5pt}\pcmn{初五}\hspace{5pt}\pfra{le 5e jour du mois}\end{exemple}
\begin{exemple}\pnru{tsʰe˧do˧-hõ˥ɲi˩}\hspace{5pt}\peng{the 8th day of the month}\hspace{5pt}\pcmn{初八}\hspace{5pt}\pfra{le 8e jour du mois}\end{exemple}
\begin{exemple}\pnru{tsʰe˧do˧˥ | -tsʰe˩ɲi˩˥}\hspace{5pt}\peng{the 10th day of the month}\hspace{5pt}\pcmn{初十}\hspace{5pt}\pfra{le 10e jour du mois}\end{exemple}
\begin{exemple}\pnru{tsʰe˧do˧˥ | -tsʰe˩ɖɯ˩ɲi˩˥}\hspace{5pt}\peng{the 11th day of the month}\hspace{5pt}\pcmn{十一日}\hspace{5pt}\pfra{le 11e jour du mois}\end{exemple}
\end{entrée}

\begin{entrée}
{tsʰe˧ɖɯ˧}{}{ⓔtsʰe˧ɖɯ˧}\formedesurface{tsʰe˧ɖɯ˧}\newline
\classe{数词}\ton{M}\begin{définition}\peng{11.}\end{définition}
\begin{définition}\pcmn{11}\end{définition}
\begin{définition}\pfra{11.}\end{définition}
\end{entrée}

\begin{entrée}
{tsʰe˩gv̩˩}{}{ⓔtsʰe˩gv̩˩}\formedesurface{tsʰe˩gv̩˩˥}\newline
\classe{数词}\ton{L}\begin{définition}\peng{19.}\end{définition}
\begin{définition}\pcmn{19}\end{définition}
\begin{définition}\pfra{19.}\end{définition}
\end{entrée}

\begin{entrée}
{tsʰe˩hõ˩}{}{ⓔtsʰe˩hõ˩}\formedesurface{tsʰe˩hõ˩˥}\newline
\classe{数词}\ton{L}\begin{définition}\peng{18.}\end{définition}
\begin{définition}\pcmn{18}\end{définition}
\begin{définition}\pfra{18.}\end{définition}
\end{entrée}

\begin{entrée}
{tsʰe˧hṽ̩˧˥}{}{ⓔtsʰe˧hṽ̩˧˥}\formedesurface{tsʰe˧hṽ̩˧˥}\newline
\classe{名词}\ton{MH\#}
\paradigme{\pcmn{:} \p{}}
\begin{définition}\peng{Chinese evergreen.}\end{définition}
\begin{définition}\pcmn{万年青}\end{définition}
\begin{définition}\pfra{Aspidistra.}\end{définition}
\begin{exemple}\pnru{tsʰe˧hṽ̩˧-dzi˧˥}\hspace{5pt}\peng{Chinese evergreen tree}\hspace{5pt}\pcmn{万年青树}\hspace{5pt}\pfra{arbre/plant d'aspidistra}\end{exemple}
\begin{exemple}\pnru{tsʰe˧hṽ̩˧-bæ˥bæ˩}\hspace{5pt}\peng{flowers of Chinese evergreen}\hspace{5pt}\pcmn{万年青花}\hspace{5pt}\pfra{fleurs d'aspidistra}\end{exemple}
\end{entrée}

\begin{entrée}
{tsʰe˧jɤ˧mi˥}{}{ⓔtsʰe˧jɤ˧mi˥}\formedesurface{tsʰe˧jɤ˧mi˥}\newline
\classe{名词}\ton{H\#}
\paradigme{\pcmn{:} \p{}}
\begin{définition}\peng{Marsh, bog, swamp.}\end{définition}
\begin{définition}\pcmn{沼泽}\end{définition}
\begin{définition}\pfra{Marécage.}\end{définition}
\begin{exemple}\pnru{tsʰe˧jɤ˧mi˥-qo˩, | ʈʰæ˧tɕi˧ɭɯ˥ | tʰi˧-di˩!}\hspace{5pt}\peng{In marshes, there are clumps of wild herbs!}\hspace{5pt}\pcmn{沼泽里,只长野草!}\hspace{5pt}\pfra{sur les terres marécageuses, il ne pousse que des petites touffes d'herbe!}\end{exemple}
\end{entrée}

\begin{entrée}
{tsʰe˧kʰv̩˧}{}{ⓔtsʰe˧kʰv̩˧}\formedesurface{tsʰe˧kʰv̩˧}\newline
\classe{动词}\ton{M}\begin{définition}\peng{To hiccup.}\end{définition}
\begin{définition}\pcmn{打嗝}\end{définition}
\begin{définition}\pfra{Avoir le hoquet.}\end{définition}
\end{entrée}

\begin{entrée}
{tsʰe˧ɬi˧mi˧}{}{ⓔtsʰe˧ɬi˧mi˧}\formedesurface{tsʰe˧ɬi˧mi˧}\newline
\classe{名词}\ton{M}\begin{définition}\peng{10th month.}\end{définition}
\begin{définition}\pcmn{十月}\end{définition}
\begin{définition}\pfra{10e mois.}\end{définition}
\end{entrée}

\begin{entrée}
{tsʰe˧ɲi˧}{}{ⓔtsʰe˧ɲi˧}\formedesurface{tsʰe˧ɲi˧}\newline
\classe{数词}\ton{M}\begin{définition}\peng{12.}\end{définition}
\begin{définition}\pcmn{12}\end{définition}
\begin{définition}\pfra{12.}\end{définition}
\end{entrée}

\begin{entrée}
{tsʰe˩ŋwɤ˩}{}{ⓔtsʰe˩ŋwɤ˩}\formedesurface{tsʰe˩ŋwɤ˩˥}\newline
\classe{数词}\ton{L}\begin{définition}\peng{15.}\end{définition}
\begin{définition}\pcmn{15}\end{définition}
\begin{définition}\pfra{15.}\end{définition}
\end{entrée}

\begin{entrée}
{tsʰe˩ŋwɤ˩ɲi˩}{}{ⓔtsʰe˩ŋwɤ˩ɲi˩}\formedesurface{tsʰe˩ŋwɤ˩ɲi˩˥}\newline
\classe{名词}\ton{L}\begin{définition}\peng{The 15th day of the month.}\end{définition}
\begin{définition}\pcmn{十五号}\end{définition}
\begin{définition}\pfra{Le 15e jour du mois.}\end{définition}
\end{entrée}

\begin{entrée}
{tsʰe˧qʰɑ˩}{}{ⓔtsʰe˧qʰɑ˩}\formedesurface{tsʰe˧qʰɑ˩}\newline
\classe{形容词}\ton{L\#}\begin{définition}\peng{Too salty.}\end{définition}
\begin{définition}\pcmn{太咸}\end{définition}
\begin{définition}\pfra{Trop salé.}\end{définition}
\end{entrée}

\begin{entrée}
{tsʰe˩qʰv̩˩}{}{ⓔtsʰe˩qʰv̩˩}\formedesurface{tsʰe˩qʰv̩˩˥}\newline
\classe{数词}\ton{L}\begin{définition}\peng{16.}\end{définition}
\begin{définition}\pcmn{16}\end{définition}
\begin{définition}\pfra{16.}\end{définition}
\end{entrée}

\begin{entrée}
{tsʰe˩qʰv̩˩ɲi˩}{}{ⓔtsʰe˩qʰv̩˩ɲi˩}\formedesurface{tsʰe˩qʰv̩˩ɲi˩˥}\newline
\classe{名词}\ton{L}\begin{définition}\peng{The 16th day of the month.}\end{définition}
\begin{définition}\pcmn{十六号}\end{définition}
\begin{définition}\pfra{Le 16e jour du mois.}\end{définition}
\begin{exemple}\pnru{tsʰe˧do˧˥tsʰe˩kʰʶv̩˩ɲi˩˧}\hspace{5pt}\peng{same meaning}\hspace{5pt}\pcmn{同上}\hspace{5pt}\pfra{même sens}\end{exemple}
\end{entrée}

\begin{entrée}
{tsʰe˧so˧}{}{ⓔtsʰe˧so˧}\formedesurface{tsʰe˧so˧}\newline
\classe{数词}\ton{M}\begin{définition}\peng{13.}\end{définition}
\begin{définition}\pcmn{13}\end{définition}
\begin{définition}\pfra{13.}\end{définition}
\end{entrée}

\begin{entrée}
{tsʰe˧so˧˥}{}{ⓔtsʰe˧so˧˥}\formedesurface{tsʰe˧so˧˥}\newline
\classe{形容词}\ton{MH\#}\begin{définition}\peng{Salty (pleasantly salty).}\end{définition}
\begin{définition}\pcmn{咸}\end{définition}
\begin{définition}\pfra{Salé (agréablement: salé à point).}\end{définition}
\end{entrée}

\begin{entrée}
{tsʰe˧ʂɯ˧}{}{ⓔtsʰe˧ʂɯ˧}\formedesurface{tsʰe˧ʂɯ˧}\newline
\classe{数词}\ton{M}\begin{définition}\peng{17.}\end{définition}
\begin{définition}\pcmn{17}\end{définition}
\begin{définition}\pfra{17.}\end{définition}
\end{entrée}

\begin{entrée}
{tsʰe˧tʰv̩\#˥}{}{ⓔtsʰe˧tʰv̩\#˥}\newline
\classe{名词}
\sens{1}\paradigme{\pcmn{:} \p{}}
\begin{définition}\peng{Smallpox.}\end{définition}
\begin{définition}\pcmn{天花}\end{définition}
\begin{définition}\pfra{Variole, petite vérole.}\end{définition}
\begin{exemple}\pnru{tsʰe˧tʰv̩˧ | bæ˩bæ˩ bæ˥-ze˩}\hspace{5pt}\peng{Smallpox has broken out.}\hspace{5pt}\pcmn{天花/麻疹犯了。}\hspace{5pt}\pfra{La variole s'est déclarée.}\end{exemple}\sens{2}
\begin{définition}\peng{Measles.}\end{définition}
\begin{définition}\pcmn{麻疹,疹子}\end{définition}
\begin{définition}\pfra{Rougeole.}\end{définition}
\begin{exemple}\pnru{gv̩˧-kʰv̩˩ mɤ˩-gv̩˩, | tsʰe˧ mɤ˧-tʰv̩˧, | hĩ˧ ʈʂɤ˧-mɤ˧-kv̩˩!}\hspace{5pt}\peng{If one does not catch the measles before age nine, one cannot become an adult (literally ‘a person')!}\hspace{5pt}\pcmn{九岁前不得麻疹,不能成人! / 得麻疹,就是小孩生长过程中必须要的一件事情!}\hspace{5pt}\pfra{Si, avant l'âge de neuf ans, on ne contracte pas la rougeole, on ne peut pas devenir adulte! / Attraper la rougeole, ça fait partie du processus de croissance vers l'âge adulte!}\end{exemple}
\end{entrée}

\begin{entrée}
{tsʰe˧ʈʂæ˧}{}{ⓔtsʰe˧ʈʂæ˧}\formedesurface{tsʰe˧ʈʂæ˧}\newline
\classe{名词}\ton{M}\begin{définition}\peng{Village head, small-ranking official.}\end{définition}
\begin{définition}\pcmn{村长}\end{définition}
\begin{définition}\pfra{Chef de village, petit officiel.}\end{définition}
\end{entrée}

\begin{entrée}
{tsʰe˩ʐv̩˩}{}{ⓔtsʰe˩ʐv̩˩}\formedesurface{tsʰe˩ʐv̩˩˥}\newline
\classe{数词}\ton{L}\begin{définition}\peng{14.}\end{définition}
\begin{définition}\pcmn{14}\end{définition}
\begin{définition}\pfra{14.}\end{définition}
\end{entrée}

\begin{entrée}
{tsʰɤ˧˥}{₁}{ⓔtsʰɤ˧˥ⓗ1}\formedesurface{tsʰɤ˧˥}\newline
\classe{动词}\ton{MH}
1\begin{définition}\peng{To milk.}\end{définition}
\begin{définition}\pcmn{挤奶}\end{définition}
\begin{définition}\pfra{Traire (vache, brebis).}\end{définition}
\begin{exemple}\pnru{tso˧∼tso˧ tsʰɤ˩ + ze˩}\hspace{5pt}\peng{to milk things}\hspace{5pt}\pcmn{挤出东西}\hspace{5pt}\pfra{traire des choses}\end{exemple}
\begin{exemple}\pnru{ʝi˧-bv̩˧ | ɳæ˧ tsʰɤ˩}\hspace{5pt}\peng{to milk a cow}\hspace{5pt}\pcmn{挤牛奶}\hspace{5pt}\pfra{traire (le lait de) la vache}\end{exemple}
\end{entrée}

\begin{entrée}
{tsʰɤ˧˥}{₂}{ⓔtsʰɤ˧˥ⓗ2}\formedesurface{tsʰɤ˧˥}\newline
\classe{动词}\ton{MH}
2\begin{définition}\peng{To rub (e.g. rough fabric rubs against the skin).}\end{définition}
\begin{définition}\pcmn{摩擦}\end{définition}
\begin{définition}\pfra{Frotter, faire une friction (ex.: un tissu grossier frotte sur la peau, et l'irrite).}\end{définition}
\end{entrée}

\begin{entrée}
{tsʰɤ˧˥}{₃}{ⓔtsʰɤ˧˥ⓗ3}\formedesurface{tsʰɤ˧˥}\newline
\classe{动词}\ton{MH}
3\begin{définition}\peng{To attack, to pillage (e.g. bandits attack a caravan).}\end{définition}
\begin{définition}\pcmn{抢}\end{définition}
\begin{définition}\pfra{Attaquer, piller, s'en prendre à (des brigands attaquent un convoi).}\end{définition}
\end{entrée}

\begin{entrée}
{tsʰɤ˧˥}{₄}{ⓔtsʰɤ˧˥ⓗ4}\formedesurface{tsʰɤ˧˥}\newline
\classe{动词}\ton{MH}
4\begin{définition}\peng{To give back, to return.}\end{définition}
\begin{définition}\pcmn{还(东西)}\end{définition}
\begin{définition}\pfra{Rendre.}\end{définition}
\end{entrée}

\begin{entrée}
{tsʰɤ˧˥α}{}{ⓔtsʰɤ˧˥α}\formedesurface{ɖɯ˧ tsʰɤ˧˥}\newline
\classe{量词}\ton{MHα}\begin{définition}\peng{Classifier for dented or bumpy objects: cockscombs, leaves, and bulbs of garlic.}\end{définition}
\begin{définition}\pcmn{量词:凸凹的物品,如:鸡冠(一顶)、叶子(一片)、蒜(一头)}\end{définition}
\begin{définition}\pfra{Classificateur des objets bosselés: feuilles, crêtes de coq, têtes d'ail (une tête d'ail se divise en gousses un peu comme on effeuille une branche ou une fleur).}\end{définition}
\begin{exemple}\pnru{bæ˩bæ˩˥ | ɖɯ˧-tsʰɤ˧˥}\hspace{5pt}\peng{a flower}\hspace{5pt}\pcmn{一朵花}\hspace{5pt}\pfra{une fleur}\end{exemple}
\end{entrée}

\begin{entrée}
{tsʰɤ˩α}{}{ⓔtsʰɤ˩α}\formedesurface{tsʰɤ˩˥}\newline
\classe{动词}\ton{Lα}\begin{définition}\peng{To plait, to weave (hair, thread).}\end{définition}
\begin{définition}\pcmn{编(头发,线)}\end{définition}
\begin{définition}\pfra{Tresser (les cheveux, fils).}\end{définition}
\begin{exemple}\pnru{ʁo˧qʰwɤ˩ tsʰɤ˩}\hspace{5pt}\peng{to weave the hair}\hspace{5pt}\pcmn{编辫子}\hspace{5pt}\pfra{tresser les cheveux, littéralement «tresser la tête»}\end{exemple}
\begin{exemple}\pnru{hæ̃˧pɤ˧ le˧-tsʰɤ˩}\hspace{5pt}\peng{to plait hair}\hspace{5pt}\pcmn{梳一条辫子}\hspace{5pt}\pfra{faire une tresse}\end{exemple}
\begin{exemple}\pnru{ɖɯ˧-tsʰɤ˧∼tsʰɤ˥-ɻ̍˩}\hspace{5pt}\peng{|fg{delimitative} \_ |fg{red} |fg{inceptive}}\hspace{5pt}\pcmn{|fg{delimitative} \_ |fg{red} |fg{inceptive}}\hspace{5pt}\pfra{|fg{délimitatif} \_ |fg{red} |fg{inchoatif}}\end{exemple}
\end{entrée}

\begin{entrée}
{tsʰi˥α}{}{ⓔtsʰi˥α}\formedesurface{ɖɯ˧ tsʰi˥}\newline
\classe{量词}\ton{Hα}\begin{définition}\peng{Classifier for pelts / hides (animal skins), and for pieces of fabric.}\end{définition}
\begin{définition}\pcmn{量词:动物皮(一张),布料(一块)}\end{définition}
\begin{définition}\pfra{Classificateur des peaux d'animaux, et des pièces de tissu.}\end{définition}
\begin{exemple}\pnru{ɖɯ˧-tsʰi˥}\hspace{5pt}\peng{one pelt}\hspace{5pt}\pcmn{一张动物皮}\hspace{5pt}\pfra{une peau}\end{exemple}
\begin{exemple}\pnru{ɖɯ˧-tsʰi˧ ɲi˥}\hspace{5pt}\peng{It's a pelt.}\hspace{5pt}\pcmn{这是一张(动物皮)}\hspace{5pt}\pfra{c'est une peau}\end{exemple}
\end{entrée}

\begin{entrée}
{tsʰi˧}{₁}{ⓔtsʰi˧ⓗ1}\formedesurface{tsʰi˧}\newline
\classe{形容词}\ton{M}
1\begin{définition}\peng{Hot; scalding.}\end{définition}
\begin{définition}\pcmn{热,烫}\end{définition}
\begin{définition}\pfra{Chaud.}\end{définition}
\begin{exemple}\pnru{tsʰi˧-zo˧ mɤ˧-tʰɑ˧˥!}\hspace{5pt}\peng{The heat is unbearable! / The weather is unbearably hot!}\hspace{5pt}\pcmn{热得受不了!}\hspace{5pt}\pfra{il fait une chaleur insupportable!}\end{exemple}
\end{entrée}

\begin{entrée}
{tsʰi˧}{₂}{ⓔtsʰi˧ⓗ2}\formedesurface{tsʰi˧}\newline
\classe{形容词}\ton{M}
2\begin{définition}\peng{Bright.}\end{définition}
\begin{définition}\pcmn{明亮}\end{définition}
\begin{définition}\pfra{Brillant, lumineux, ardent.}\end{définition}
\begin{exemple}\pnru{ɲi˧mi˧ tsʰi˧}\hspace{5pt}\peng{the sun shines, the sunlight is strong}\hspace{5pt}\pcmn{太阳很晒}\hspace{5pt}\pfra{le soleil est très fort/le soleil tape dur}\end{exemple}
\begin{exemple}\pnru{ɬi˧mi˧ tsʰi˧}\hspace{5pt}\peng{the moon shines, the moonlight is strong}\hspace{5pt}\pcmn{月亮很亮、月光很明亮}\hspace{5pt}\pfra{la lune brille, la lune luit, on y voit clair à la lumière de la lune}\end{exemple}
\end{entrée}

\begin{entrée}
{tsʰi˧˥}{₁}{ⓔtsʰi˧˥ⓗ1}\formedesurface{tsʰi˧˥}\newline
\classe{动词}\ton{MH}
1\begin{définition}\peng{To construct a house, to build a house.}\end{définition}
\begin{définition}\pcmn{盖,建 (房子)}\end{définition}
\begin{définition}\pfra{Construire.}\end{définition}
\begin{exemple}\pnru{ʑi˧qʰwɤ˧ tsʰi˧˥}\hspace{5pt}\peng{to build a house}\hspace{5pt}\pcmn{建 房子}\hspace{5pt}\pfra{construire un bâtiment}\end{exemple}
\end{entrée}

\begin{entrée}
{tsʰi˧˥}{₂}{ⓔtsʰi˧˥ⓗ2}\formedesurface{tsʰi˧˥}\newline
\classe{动词}\ton{MH}
2\begin{définition}\peng{To bore a hole, to punch a hole.}\end{définition}
\begin{définition}\pcmn{穿一个洞}\end{définition}
\begin{définition}\pfra{Percer un trou, faire un trou (ex.: dans un tissu, un mur…).}\end{définition}
\end{entrée}

\begin{entrée}
{tsʰi˧˥}{₃}{ⓔtsʰi˧˥ⓗ3}\formedesurface{tsʰi˧˥}\newline
\classe{动词}\ton{MH}
3\begin{définition}\peng{To start (a fire).}\end{définition}
\begin{définition}\pcmn{点(火)}\end{définition}
\begin{définition}\pfra{Allumer (un feu).}\end{définition}
\begin{exemple}\pnru{mv̩˧ tsʰi˧˥}\hspace{5pt}\peng{to start a fire}\hspace{5pt}\pcmn{点火}\hspace{5pt}\pfra{allumer un feu}\end{exemple}
\begin{exemple}\pnru{njɤ˧-ɳɯ˧ | mv̩˧tsʰi˧-bi˥}\hspace{5pt}\peng{I am going to start a fire}\hspace{5pt}\pcmn{我要点个火}\hspace{5pt}\pfra{je vais allumer le feu}\end{exemple}
\begin{exemple}\pnru{mv̩˩tsʰo˩ tsʰi˧}\hspace{5pt}\peng{to put fire to a piece of wood full of sap (to start a fire)}\hspace{5pt}\pcmn{用含很多树脂的木头来引火}\hspace{5pt}\pfra{mettre le feu à un bout de bois plein de sève (pour faire partir le feu)}\end{exemple}
\end{entrée}

\begin{entrée}
{tsʰi˧˥}{₄}{ⓔtsʰi˧˥ⓗ4}\formedesurface{tsʰi˧˥}\newline
\classe{动词}\ton{MH}
4\begin{définition}\peng{To squat.}\end{définition}
\begin{définition}\pcmn{蹲}\end{définition}
\begin{définition}\pfra{S'accroupir.}\end{définition}
\begin{exemple}\pnru{le˧-tsʰi˩∼tsʰi˩ | tʰi˧-dzi˩}\hspace{5pt}\peng{to sit cross-legged}\hspace{5pt}\pcmn{盘腿坐}\hspace{5pt}\pfra{être accroupi (être assis avec les genoux regroupés sur la poitrine)}\end{exemple}
\begin{exemple}\pnru{gɤ˩-tsʰi˧∼tsʰi˩ tʰi˧-dzi˩}\hspace{5pt}\peng{as above}\hspace{5pt}\pcmn{盘腿坐}\hspace{5pt}\pfra{même sens}\end{exemple}
\end{entrée}

\begin{entrée}
{tsʰi˧˥}{₅}{ⓔtsʰi˧˥ⓗ5}\formedesurface{tsʰi˧˥}\newline
\classe{形容词}\ton{MH}
5\begin{définition}\peng{Sick, ill.}\end{définition}
\begin{définition}\pcmn{病}\end{définition}
\begin{définition}\pfra{Malade, souffrant.}\end{définition}
\begin{exemple}\pnru{mɤ˧-go˩ mɤ˩-tsʰi˩-ɻ̍˩ |}\hspace{5pt}\peng{to be in good health: not sick, not suffering}\hspace{5pt}\pcmn{健康:不病、不痛}\hspace{5pt}\pfra{être bien portant, ne pas être malade}\end{exemple}
\end{entrée}

\begin{entrée}
{tsʰi˧β}{}{ⓔtsʰi˧β}\formedesurface{tsʰi˧}\newline
\classe{动词}\ton{Mβ}\begin{définition}\peng{To wear (a hat).}\end{définition}
\begin{définition}\pcmn{戴帽子}\end{définition}
\begin{définition}\pfra{Porter (un chapeau).}\end{définition}
\begin{exemple}\pnru{tv̩˧tv̩˥ tsʰi˩}\hspace{5pt}\peng{to put on a hat}\hspace{5pt}\pcmn{戴上帽子}\hspace{5pt}\pfra{mettre un chapeau}\end{exemple}
\end{entrée}

\begin{entrée}
{tsʰi˩α}{}{ⓔtsʰi˩α}\formedesurface{tsʰi˩˥}\newline
\classe{形容词}\ton{Lα}\begin{définition}\peng{Fine, thin.}\end{définition}
\begin{définition}\pcmn{细(树、体型细小)}\end{définition}
\begin{définition}\pfra{Fin (objet).}\end{définition}
\begin{exemple}\pnru{tsʰi˩-hĩ˩˥}\hspace{5pt}\peng{|fg{nmlz}}\hspace{5pt}\pcmn{细的}\hspace{5pt}\pfra{|fg{nmlz}}\end{exemple}
\begin{exemple}\pnru{qʰɑ˧-tsʰi˧-gv̩˧}\hspace{5pt}\peng{very thin}\hspace{5pt}\pcmn{非常细}\hspace{5pt}\pfra{très fin}\end{exemple}
\begin{exemple}\pnru{dʑɤ˧˥ | tsʰi˩-njæ˩˥ | -gv̩˩!}\hspace{5pt}\peng{It is really thin!}\hspace{5pt}\pcmn{真细!}\hspace{5pt}\pfra{C'est vraiment fin!}\end{exemple}
\end{entrée}

\begin{entrée}
{tsʰi\#˥}{}{ⓔtsʰi\#˥}\formedesurface{tsʰi˧}\newline
\classe{名词}\ton{\#H}\begin{définition}\peng{Dry season (winter and spring: from the 9th lunar month to the 2nd lunar month).}\end{définition}
\begin{définition}\pcmn{旱季(冬天至春天:农历九月到来年二月)}\end{définition}
\begin{définition}\pfra{Saison sèche (hiver et printemps; du 9e mois au 2e mois du calendrier lunaire compris).}\end{définition}
\end{entrée}

\begin{entrée}
{tsʰi˧bv̩˩}{}{ⓔtsʰi˧bv̩˩}\formedesurface{tsʰi˧bv̩˩}\newline
\classe{形容词}\ton{L\#}\begin{définition}\peng{Muggy, sultry, oppressively hot.}\end{définition}
\begin{définition}\pcmn{闷热}\end{définition}
\begin{définition}\pfra{Étouffant.}\end{définition}
\end{entrée}

\begin{entrée}
{tsʰi˧ʝi\#˥}{}{ⓔtsʰi˧ʝi\#˥}\formedesurface{tsʰi˧ʝi˧}\newline
\classe{助词}\ton{\#H}\begin{définition}\peng{This year.}\end{définition}
\begin{définition}\pcmn{今年}\end{définition}
\begin{définition}\pfra{Cette année.}\end{définition}
\begin{exemple}\pnru{tsʰi˧ʝi˧-se˥, | …}\hspace{5pt}\peng{Until this year, …}\hspace{5pt}\pcmn{到了今年,……}\hspace{5pt}\pfra{Jusqu'à cette année, …}\end{exemple}
\end{entrée}

\begin{entrée}
{tsʰi˩mv̩˩tʰv̩˩}{}{ⓔtsʰi˩mv̩˩tʰv̩˩}\formedesurface{tsʰi˩mv̩˩tʰv̩˩˥}\newline
\classe{名词}\ton{L}\begin{définition}\peng{Dancing demon.}\end{définition}
\begin{définition}\pcmn{跳着的鬼}\end{définition}
\begin{définition}\pfra{Démon qui danse.}\end{définition}
\end{entrée}

\begin{entrée}
{tsʰi˧ɲi\#˥}{}{ⓔtsʰi˧ɲi\#˥}\formedesurface{tsʰi˧ɲi˧}\newline
\classe{助词}\ton{\#H}\begin{définition}\peng{Today.}\end{définition}
\begin{définition}\pcmn{今天}\end{définition}
\begin{définition}\pfra{Aujourd'hui.}\end{définition}
\begin{exemple}\pnru{tsʰi˧ɲi˧-ʁo˧dɑ˧}\hspace{5pt}\peng{before today; previously}\hspace{5pt}\pcmn{今天之前}\hspace{5pt}\pfra{avant ce jour, avant aujourd'hui; précédemment}\end{exemple}
\end{entrée}

\begin{entrée}
{tsʰi˧qʰæ˧˥}{}{ⓔtsʰi˧qʰæ˧˥}\formedesurface{tsʰi˧qʰæ˧˥}\newline
\classe{助词}\ton{MH}\begin{définition}\peng{Now, currently, these days.}\end{définition}
\begin{définition}\pcmn{现在}\end{définition}
\begin{définition}\pfra{En ce moment, actuellement, maintenant.}\end{définition}
\end{entrée}

\begin{entrée}
{tsʰi˧si˩-dʑɤ˩pv̩˩}{}{ⓔtsʰi˧si˩-dʑɤ˩pv̩˩}\formedesurface{tsʰi˧si˩dʑɤ˩pv̩˩}\newline
\classe{名词}\ton{L\#-}\begin{définition}\peng{The world of spirits, the world of the dead.}\end{définition}
\begin{définition}\pcmn{神灵的世界、死人的世界}\end{définition}
\begin{définition}\pfra{Le monde des esprits, le monde des morts.}\end{définition}
\end{entrée}

\begin{entrée}
{tsʰi˧ti\#˥}{}{ⓔtsʰi˧ti\#˥}\formedesurface{tsʰi˧ti˧}\newline
\classe{名词}\ton{\#H}\begin{définition}\peng{Masculine given name.}\end{définition}
\begin{définition}\pcmn{男性名字}\end{définition}
\begin{définition}\pfra{Prénom masculin.}\end{définition}
\end{entrée}

\begin{entrée}
{tsʰi˩tv̩˩}{}{ⓔtsʰi˩tv̩˩}\formedesurface{tsʰi˩tv̩˩˥}\newline
\classe{名词}\ton{L}
\paradigme{\pcmn{:} \p{}}
\begin{définition}\peng{Bone marrow.}\end{définition}
\begin{définition}\pcmn{骨髓}\end{définition}
\begin{définition}\pfra{Moëlle.}\end{définition}
\end{entrée}

\begin{entrée}
{tsʰi˩tsʰi˩}{}{ⓔtsʰi˩tsʰi˩}\formedesurface{tsʰi˩tsʰi˩˥}\newline
\classe{名词}\ton{L}
\paradigme{\pcmn{:} \p{}}
\begin{définition}\peng{Peas, garden peas.}\end{définition}
\begin{définition}\pcmn{豌豆}\end{définition}
\begin{définition}\pfra{Pois, petits pois.}\end{définition}
\end{entrée}

\begin{entrée}
{tsʰi˧zi\#˥}{}{ⓔtsʰi˧zi\#˥}\formedesurface{tsʰi˧zi˧}\newline
\classe{名词}\ton{\#H}
\paradigme{\pcmn{:} \p{}}
\begin{définition}\peng{Highland barley, |\stylefi{Hordeum vulgare var. nudum Hook. f.}.}\end{définition}
\begin{définition}\pcmn{青稞}\end{définition}
\begin{définition}\pfra{Orge d'altitude, |\stylefi{Hordeum vulgare var. nudum Hook. f.}.}\end{définition}
\begin{exemple}\pnru{tsʰi˧zi˧ | nɑ˩-hĩ˩˥}\hspace{5pt}\peng{black barley}\hspace{5pt}\pcmn{黑青稞}\hspace{5pt}\pfra{orge noir}\end{exemple}
\begin{exemple}\pnru{tsʰi˧zi˧ | pʰv̩˩-hĩ˩˥}\hspace{5pt}\peng{white barley}\hspace{5pt}\pcmn{白青稞}\hspace{5pt}\pfra{orge blanc}\end{exemple}
\end{entrée}

\begin{entrée}
{tsʰi˧zi˧-ɻ̃\#˥}{}{ⓔtsʰi˧zi˧-ɻ̃\#˥}\formedesurface{tsʰi˧zi˧ɻ̃˧}\newline
\classe{名词}\ton{\#H}
\paradigme{\pcmn{:} \p{}}
\begin{définition}\peng{Highland barley straw.}\end{définition}
\begin{définition}\pcmn{青稞杆}\end{définition}
\begin{définition}\pfra{Paille d'orge.}\end{définition}
\end{entrée}

\begin{entrée}
{tsʰo˥}{}{ⓔtsʰo˥}\formedesurface{tsʰo˧}\newline
\classe{形容词}\ton{H}\begin{définition}\peng{Complete, all in readiness.}\end{définition}
\begin{définition}\pcmn{齐全}\end{définition}
\begin{définition}\pfra{Complet, au grand complet.}\end{définition}
\begin{exemple}\pnru{ə˧tso˧-mɤ˧-ɲi˩, | tʰi˧-tsʰo˥-ze˩!}\hspace{5pt}\peng{All is in readiness! Everything is now ready! (Context: preparation for a feast, a meal…)}\hspace{5pt}\pcmn{什么都准备得很齐全!}\hspace{5pt}\pfra{Tout y est! Tout est prêt! (Au sujet de préparatifs pour une fête, un repas…)}\end{exemple}
\begin{exemple}\pnru{mɤ˧-tsʰo˧-sɯ˥! | wɤ˩˥ | ɲi˧-bæ˧ hwæ˧-zo˧-ho˩!}\hspace{5pt}\peng{[Decoration] is not complete yet! [I] still need to purchase a few items! (Context: visitors admire a newly furbished apartment in town; the landlord answers their compliments by saying ‘The work is not finished yet!')}\hspace{5pt}\pcmn{还不算齐全! / 还没有装饰齐全!(情景:朋友们表扬新装修的丽江房子,主人谦虚回答:‘还不算齐全!’)}\hspace{5pt}\pfra{On n'y est pas encore tout à fait / ce n'est pas encore tout à fait prêt! Il reste deux ou trois trucs à acheter! (Contexte: on achève la décoration d'un appartement à la ville; aux compliments des visiteurs, l'heureux propriétaire répond: ‘Ce n'est pas encore terminé!')}\end{exemple}
\begin{exemple}\pnru{tʰi˧-tsʰo˥-kʰɯ˩}\hspace{5pt}\peng{|fg{dur} \_ |fg{caus}: to complete, to bring to complete readiness}\hspace{5pt}\pcmn{|fg{dur} \_ |fg{caus}:完成、弄齐全}\hspace{5pt}\pfra{|fg{dur} \_ |fg{caus}: porter à son point d'achèvement, porter au grand complet}\end{exemple}
\end{entrée}

\begin{entrée}
{tsʰo˧˥}{}{ⓔtsʰo˧˥}\formedesurface{tsʰo˧˥}\newline
\classe{名词}\ton{MH}\begin{définition}\peng{Respect, attention, esteem.}\end{définition}
\begin{définition}\pcmn{重视、关心、恭敬}\end{définition}
\begin{définition}\pfra{Respect, attention, estime.}\end{définition}
\begin{exemple}\pnru{ʈʂʰɯ˧-ɳɯ˧ | njɤ˧-ki˧ | ɖwæ˧˥ | tsʰo˧ ʝi˥!}\hspace{5pt}\peng{He/she treats me with great respect/attention.}\hspace{5pt}\pcmn{他很重视我 / 他对我很尊敬、很关心。}\hspace{5pt}\pfra{Il/elle me traite avec les plus grands égards / est aux petits soins pour moi!}\end{exemple}
\end{entrée}

\begin{entrée}
{tsʰo˧β}{}{ⓔtsʰo˧β}\formedesurface{tsʰo˧}\newline
\classe{动词}\ton{Mβ}\begin{définition}\peng{To jump.}\end{définition}
\begin{définition}\pcmn{跳}\end{définition}
\begin{définition}\pfra{Sauter.}\end{définition}
\begin{exemple}\pnru{bæ˧ tsʰo˧}\hspace{5pt}\peng{to skip rope}\hspace{5pt}\pcmn{跳绳}\hspace{5pt}\pfra{sauter à la corde}\end{exemple}
\begin{exemple}\pnru{tsʰo˧∼tsʰo˧}\hspace{5pt}\peng{|fg{red}}\hspace{5pt}\pcmn{重叠}\hspace{5pt}\pfra{forme rédupliquée: trépigner, sautiller ici et là}\end{exemple}
\end{entrée}

\begin{entrée}
{tsʰo˩}{}{ⓔtsʰo˩}\formedesurface{tsʰo˧}\newline
\classe{名词}\ton{L}\begin{définition}\peng{Human beings; the human species.}\end{définition}
\begin{définition}\pcmn{人类}\end{définition}
\begin{définition}\pfra{Espèce humaine, êtres humains; terme ancien apparaissant dans certains proverbes.}\end{définition}
\end{entrée}

\begin{entrée}
{tsʰo˧ɖɯ˩}{}{ⓔtsʰo˧ɖɯ˩}\formedesurface{tsʰo˧ɖɯ˩}\newline
\classe{名词}\ton{L\#}\begin{définition}\peng{Group dance.}\end{définition}
\begin{définition}\pcmn{集体舞}\end{définition}
\begin{définition}\pfra{Danse en groupe: parfois dix personnes, parfois jusqu'à cent (un village entier).}\end{définition}
\begin{exemple}\pnru{tsʰo˧ɖɯ˩ tsʰo˩}\hspace{5pt}\peng{to perform a group dance}\hspace{5pt}\pcmn{跳一个集体舞}\hspace{5pt}\pfra{faire une grande danse collective}\end{exemple}
\end{entrée}

\begin{entrée}
{tsʰo˧ɖwæ\#˥}{}{ⓔtsʰo˧ɖwæ\#˥}\formedesurface{tsʰo˧ɖwæ˧}\newline
\classe{名词}\ton{\#H}
\paradigme{\pcmn{:} \p{}}
\begin{définition}\peng{Stone step.}\end{définition}
\begin{définition}\pcmn{石头台阶}\end{définition}
\begin{définition}\pfra{Marche en pierre.}\end{définition}
\end{entrée}

\begin{entrée}
{tsʰo˧ko˧}{}{ⓔtsʰo˧ko˧}\formedesurface{tsʰo˧ko˧}\newline
\classe{名词}\ton{M}
\paradigme{\pcmn{:} \p{}}
\begin{définition}\peng{Cardamom, |\stylefi{Elletaria cardamomum}.}\end{définition}
\begin{définition}\pcmn{小豆蔻}\end{définition}
\begin{définition}\pfra{Cardamome, |\stylefi{Elletaria cardamomum}.}\end{définition}
\end{entrée}

\begin{entrée}
{tsʰo˩mo˩}{}{ⓔtsʰo˩mo˩}\formedesurface{tsʰo˩mo˩˥}\newline
\classe{名词}\ton{L}\begin{définition}\peng{Old man.}\end{définition}
\begin{définition}\pcmn{老头}\end{définition}
\begin{définition}\pfra{Vieil homme, vieillard.}\end{définition}
\end{entrée}

\begin{entrée}
{tsʰo˧pæ\#˥}{}{ⓔtsʰo˧pæ\#˥}\formedesurface{tsʰo˧pæ˧}\newline
\classe{名词}\ton{\#H}\begin{définition}\peng{Caravan chief.}\end{définition}
\begin{définition}\pcmn{马帮头领}\end{définition}
\begin{définition}\pfra{Chef de caravane.}\end{définition}
\end{entrée}

\begin{entrée}
{tsʰo˧pjɤ˧}{}{ⓔtsʰo˧pjɤ˧}\formedesurface{tsʰo˧pjɤ˧}\newline
\classe{名词}\ton{M}
\paradigme{\pcmn{:} \p{}}
\begin{définition}\peng{Soap. Presumably borrowed from a language of Burma: cp. Nung /tshɑ³¹ pi⁵⁵ iɔ⁵⁵/ [Dai 1992], Luxi Achang and Lianghe Achang /tshɑu⁵⁵ pjɑu⁵⁵/ [Dai 1985], Longchuan Achang /tshau³¹ piau³¹/ [Dai 1992]. Culturally, it is not unlikely that soap was first introduced through contact/commerce with ethnic groups of Burma.}\end{définition}
\begin{définition}\pcmn{肥皂}\end{définition}
\begin{définition}\pfra{Savon. Sans doute mot emprunté à une langue de birmanie: cp. nung: tshɑ³¹ pi⁵⁵ iɔ⁵⁵ [Dai 1992]; achang de Luxi et Lianghe: tshɑu⁵⁵ pjɑu⁵⁵ [Dai 1985]; achang de Longchuan: tshau³¹ piau³¹ [Dai 1992]. Culturellement, il est plausible que le savon ait été introduit par le contact/commerce avec des groupes ethniques de Birmanie.}\end{définition}
\end{entrée}

\begin{entrée}
{tsʰo˧qʰwɤ˩}{}{ⓔtsʰo˧qʰwɤ˩}\formedesurface{tsʰo˧qʰwɤ˩}\newline
\classe{助词}\ton{L\#}\begin{définition}\peng{Tonight.}\end{définition}
\begin{définition}\pcmn{今晚}\end{définition}
\begin{définition}\pfra{Ce soir.}\end{définition}
\begin{exemple}\pnru{tsʰo˧qʰwɤ˩ | mv̩˩kʰv̩˧˥}\hspace{5pt}\peng{same meaning: tonight}\hspace{5pt}\pcmn{同上:今晚}\hspace{5pt}\pfra{même sens: ce soir}\end{exemple}
\end{entrée}

\begin{entrée}
{tsʰo˧qʰwɤ˧mi\#˥}{}{ⓔtsʰo˧qʰwɤ˧mi\#˥}\formedesurface{tsʰo˧qʰwɤ˧mi˧}\newline
\classe{名词}\ton{\#H}
\paradigme{\pcmn{:} \p{}}
\begin{définition}\peng{Demon, ghost.}\end{définition}
\begin{définition}\pcmn{鬼}\end{définition}
\begin{définition}\pfra{Démon, fantôme.}\end{définition}
\end{entrée}

\begin{entrée}
{tsʰo˧qʰwɤ˧mi˧-bæ˥bæ˩}{}{ⓔtsʰo˧qʰwɤ˧mi˧-bæ˥bæ˩}\formedesurface{tsʰo˧qʰwɤ˧mi˧bæ˥bæ˩}\newline
\classe{名词}\ton{\#H-}\begin{définition}\peng{A blue flower, |\stylefi{Delphinium grandiflorum}.}\end{définition}
\begin{définition}\pcmn{翠雀花}\end{définition}
\begin{définition}\pfra{Une fleur bleue, |\stylefi{Delphinium grandiflorum}.}\end{définition}
\end{entrée}

\begin{entrée}
{tsʰo˧qʰwɤ˧zo\#˥}{}{ⓔtsʰo˧qʰwɤ˧zo\#˥}\formedesurface{tsʰo˧qʰwɤ˧zo˧}\newline
\classe{名词}\ton{\#H}\begin{définition}\peng{Demon, ghost (this word is less common than that with a feminine suffix).}\end{définition}
\begin{définition}\pcmn{鬼}\end{définition}
\begin{définition}\pfra{Démon, fantôme (forme moins courante que celle comportant un suffixe féminin).}\end{définition}
\end{entrée}

\begin{entrée}
{tsʰo˧ʁo\#˥}{}{ⓔtsʰo˧ʁo\#˥}\formedesurface{tsʰo˧ʁo˧}\newline
\classe{名词}\ton{\#H}
\paradigme{\pcmn{:} \p{}}
\begin{définition}\peng{Barn: the building at the entrance of the farm, through which one comes when entering the farm. It is made of wood. On the ground floor, there are stables and pigsties; hay and straw are stored on the first floor.}\end{définition}
\begin{définition}\pcmn{牲畜圈:家门口的那栋楼,下为畜厩,上存饲料或另辟为房间}\end{définition}
\begin{définition}\pfra{Étable: bâtiment à l'entrée de la ferme, que l'on traverse en entrant dans la ferme. Construit en bois. Au rez-de-chaussée se trouvent les étables des porcs; à l'étage un grenier à foin.}\end{définition}
\end{entrée}

\begin{entrée}
{tsʰo˧tsɯ˧}{}{ⓔtsʰo˧tsɯ˧}\formedesurface{tsʰo˧tsɯ˧}\newline
\classe{名词}\ton{M}
\paradigme{\pcmn{:} \p{}}
\begin{définition}\peng{Onion; leek.}\end{définition}
\begin{définition}\pcmn{葱,韭葱}\end{définition}
\begin{définition}\pfra{Poireau, oignon.}\end{définition}
\end{entrée}

\begin{entrée}
{tsʰo˩tsɯ˧}{}{ⓔtsʰo˩tsɯ˧}\formedesurface{tsʰo˩tsɯ˥}\newline
\classe{名词}\ton{LM}
\paradigme{\pcmn{:} \p{}}
\begin{définition}\peng{File (tool).}\end{définition}
\begin{définition}\pcmn{锉刀}\end{définition}
\begin{définition}\pfra{Lime.}\end{définition}
\end{entrée}

\begin{entrée}
{tsʰɯ˧˥}{₁}{ⓔtsʰɯ˧˥ⓗ1}\formedesurface{tsʰɯ˧˥}\newline
\classe{动词}\ton{MH}
1\begin{définition}\peng{To cut to pieces (e.g. to cut away at a piece of clothing with scissors).}\end{définition}
\begin{définition}\pcmn{剪成片}\end{définition}
\begin{définition}\pfra{Taillader (ex.: taillader un vêtement, le découper avec des ciseaux; n'est pas: tailler du tissu pour faire des vêtements).}\end{définition}
\begin{exemple}\pnru{tʰɑ˧-tsʰɯ˧˥!}\hspace{5pt}\peng{|fg{prohib}}\hspace{5pt}\pcmn{|fg{prohib}}\hspace{5pt}\pfra{|fg{prohib}}\end{exemple}
\begin{exemple}\pnru{dʑi˧hṽ̩˧ tsʰɯ˩}\hspace{5pt}\peng{to cut clothes to pieces}\hspace{5pt}\pcmn{把衣服剪成片}\hspace{5pt}\pfra{couper des vêtements en morceaux}\end{exemple}
\end{entrée}

\begin{entrée}
{tsʰɯ˧˥}{₂}{ⓔtsʰɯ˧˥ⓗ2}\formedesurface{tsʰɯ˧˥}\newline
\classe{名词}\ton{MH}
2
\paradigme{\pcmn{:} \p{}}
\begin{définition}\peng{Goat (male or female).}\end{définition}
\begin{définition}\pcmn{山羊}\end{définition}
\begin{définition}\pfra{Chèvre/bouc.}\end{définition}
\end{entrée}

\begin{entrée}
{tsʰɯ˩α}{}{ⓔtsʰɯ˩α}\formedesurface{tsʰɯ˩˥}\newline
\classe{动词}\ton{Lα}\begin{définition}\peng{To come (|fg{pst}).}\end{définition}
\begin{définition}\pcmn{来(过去式)}\end{définition}
\begin{définition}\pfra{Venir (|fg{pst}).}\end{définition}
\begin{exemple}\pnru{le˧-gwɤ˩∼gwɤ˩ | le˧-tsʰɯ˩-ze˩}\hspace{5pt}\peng{to be back from a stroll}\hspace{5pt}\pcmn{散步回来}\hspace{5pt}\pfra{revenir de promenade}\end{exemple}
\begin{exemple}\pnru{le˧-tsʰɯ˩-ze˩}\hspace{5pt}\peng{to be back}\hspace{5pt}\pcmn{回来了}\hspace{5pt}\pfra{être de retour}\end{exemple}
\begin{exemple}\pnru{ɖɯ˧-ʝi˧-ɳɯ˧ tsʰɯ˧˥, | ɖɯ˧-ki˧ tʰv̩˧!}\hspace{5pt}\peng{“We have come from different places, and now we arrive in the same place / we come together!" This turn of phrase is not intelligible without prior learning, as it literally means “Coming from one place; arriving in one place".}\hspace{5pt}\pcmn{“我们都来自不同的地方,但现在在一起了!”}\hspace{5pt}\pfra{«Venus de différents endroits, nous voici réunis en ce lieu!» L'expression est obscure pour qui ne l'a pas apprise (par exemple pour des locuteurs du bord du Lac): son sens littéral est simplement «On vient d'un endroit; on arrive à un endroit!»}\end{exemple}
\end{entrée}

\begin{entrée}
{tsʰɯ˧hṽ̩˥\$}{}{ⓔtsʰɯ˧hṽ̩˥\$}\formedesurface{tsʰɯ˧hṽ̩˥}\newline
\classe{名词}\ton{H\$}
\paradigme{\pcmn{:} \p{}}
\begin{définition}\peng{Cashmere, goat wool.}\end{définition}
\begin{définition}\pcmn{山羊毛}\end{définition}
\begin{définition}\pfra{Laine de chèvre, cachemire.}\end{définition}
\end{entrée}

\begin{entrée}
{tsʰɯ˧mi˥\$}{}{ⓔtsʰɯ˧mi˥\$}\formedesurface{tsʰɯ˧mi˥}\newline
\classe{名词}\ton{H\$}
\paradigme{\pcmn{:} \p{}}
\begin{définition}\peng{Nanny goat.}\end{définition}
\begin{définition}\pcmn{母山羊}\end{définition}
\begin{définition}\pfra{Chèvre.}\end{définition}
\begin{exemple}\pnru{tsʰɯ˧mi˧-po˧lo˥}\hspace{5pt}\peng{nanny goat and billy goat}\hspace{5pt}\pcmn{母山羊与公山羊}\hspace{5pt}\pfra{chèvre et bouc}\end{exemple}
\end{entrée}

\begin{entrée}
{tsʰɯ˧mi˧-to˧qɑ˥\$}{}{ⓔtsʰɯ˧mi˧-to˧qɑ˥\$}\formedesurface{tsʰɯ˧mi˧to˧qɑ˥}\newline
\classe{名词}\ton{H\$}
\paradigme{\pcmn{:} \p{}}
\begin{définition}\peng{Male goat; also used to refer to a young male goat, or even to goats in general, male and female.}\end{définition}
\begin{définition}\pcmn{公山羊(包括公山羊羔)(可以来指所有的山羊,包括母的和公的)}\end{définition}
\begin{définition}\pfra{Bouc; s'emploie aussi pour un chevreau (cabri), ou même pour toute l'espèce, y compris les chèvres.}\end{définition}
\end{entrée}

\begin{entrée}
{tsʰɯ˧pʰv̩\#˥}{}{ⓔtsʰɯ˧pʰv̩\#˥}\formedesurface{tsʰɯ˧pʰv̩˧}\newline
\classe{名词}\ton{\#H}
\paradigme{\pcmn{:} \p{}}
\begin{définition}\peng{Billy goat.}\end{définition}
\begin{définition}\pcmn{公山羊}\end{définition}
\begin{définition}\pfra{Bouc (terme élicité; plus courant: \stylefv{/po}˧lo˧/).}\end{définition}
\end{entrée}

\begin{entrée}
{tsʰɯ˧ɻ̍\#˥}{}{ⓔtsʰɯ˧ɻ̍\#˥}\formedesurface{tsʰɯ˧ɻ̍˧}\newline
\classe{名词}\ton{\#H}\begin{définition}\peng{A unixex given name: a given name used for both men and women.}\end{définition}
\begin{définition}\pcmn{男女通用名}\end{définition}
\begin{définition}\pfra{Prénom unisexe: prénom utilisé pour les deux sexes.}\end{définition}
\end{entrée}

\begin{entrée}
{tsʰɯ˧ɻ̍˧-ɖɯ˩mɑ˩}{}{ⓔtsʰɯ˧ɻ̍˧-ɖɯ˩mɑ˩}\newline
\classe{名词}\ton{tsʰɯ˧ɻ̍˧ɖɯ˩mɑ˩}\begin{définition}\peng{Feminine given name.}\end{définition}
\begin{définition}\pcmn{女性名字}\end{définition}
\begin{définition}\pfra{Prénom féminin.}\end{définition}
\end{entrée}

\begin{entrée}
{tsʰɯ˧ɻ̍˧-ɬɑ˩mv̩˩}{}{ⓔtsʰɯ˧ɻ̍˧-ɬɑ˩mv̩˩}\formedesurface{tsʰɯ˧ɻ̍˧ɬɑ˩mv̩˩}\newline
\classe{名词}\ton{-L}\begin{définition}\peng{Feminine given name.}\end{définition}
\begin{définition}\pcmn{女性名字}\end{définition}
\begin{définition}\pfra{Prénom féminin.}\end{définition}
\end{entrée}

\begin{entrée}
{tsʰɯ˧ɻ̍˧-pʰi˩tsʰo˩}{}{ⓔtsʰɯ˧ɻ̍˧-pʰi˩tsʰo˩}\formedesurface{tsʰɯ˧ɻ̍˧pʰi˩tsʰo˩}\newline
\classe{名词}\ton{-L}\begin{définition}\peng{Masculine given name.}\end{définition}
\begin{définition}\pcmn{男性名字}\end{définition}
\begin{définition}\pfra{Prénom masculin.}\end{définition}
\end{entrée}

\begin{entrée}
{tsʰɯ˧ʂwæ˥}{}{ⓔtsʰɯ˧ʂwæ˥}\formedesurface{tsʰɯ˧ʂwæ˥}\newline
\classe{名词}\ton{H\#}
\paradigme{\pcmn{:} \p{}}
\begin{définition}\peng{Wether (castrated goat, neutered goat).}\end{définition}
\begin{définition}\pcmn{阉山羊}\end{définition}
\begin{définition}\pfra{Bouc castré.}\end{définition}
\end{entrée}

\begin{entrée}
{tsʰɯ˧-to˧qɑ˥}{}{ⓔtsʰɯ˧-to˧qɑ˥}\formedesurface{tsʰɯ˧to˧qɑ˥}\newline
\classe{名词}\ton{H\#}
\paradigme{\pcmn{:} \p{}}
\begin{définition}\peng{Kid (baby goat).}\end{définition}
\begin{définition}\pcmn{羔羊、羔子}\end{définition}
\begin{définition}\pfra{Chevreau, cabri.}\end{définition}
\end{entrée}

\begin{entrée}
{tsʰɯ˩tsʰɯ˩ɻ̃˩}{}{ⓔtsʰɯ˩tsʰɯ˩ɻ̃˩}\formedesurface{tsʰɯ˩tsʰɯ˩ɻ̃˩˥}\newline
\classe{名词}\ton{L}
\paradigme{\pcmn{:} \p{}}
\begin{définition}\peng{Dry plant of garden peas, garden peas hay.}\end{définition}
\begin{définition}\pcmn{豌豆干草}\end{définition}
\begin{définition}\pfra{Paille de petits pois.}\end{définition}
\begin{exemple}\pnru{ʈʂʰɯ˧ | tsʰɯ˩tsʰɯ˩ɻ̃˩ ɲi˥.}\hspace{5pt}\peng{This is garden pea hay.}\hspace{5pt}\pcmn{这是豌豆干草。}\hspace{5pt}\pfra{C'est de la paille de haricots.}\end{exemple}
\end{entrée}

\begin{entrée}
{tsʰɯ˧zo\#˥}{}{ⓔtsʰɯ˧zo\#˥}\formedesurface{tsʰɯ˧zo˧}\newline
\classe{名词}\ton{\#H}
\paradigme{\pcmn{:} \p{}}
\begin{définition}\peng{Baby goat, male or female: kid or young nanny goat.}\end{définition}
\begin{définition}\pcmn{山羊羔(公的或母的)}\end{définition}
\begin{définition}\pfra{Chevreau (cabri) ou chevrette.}\end{définition}
\begin{exemple}\pnru{tsʰɯ˧zo˧-to˧qɑ˥}\hspace{5pt}\peng{young nanny goat(s) and young kid(s)}\hspace{5pt}\pcmn{母山羊羔与公山羊羔}\hspace{5pt}\pfra{chevrettes et chevreaux}\end{exemple}
\end{entrée}

\begin{entrée}
{tsʰv̩˩˥}{}{ⓔtsʰv̩˩˥}\formedesurface{tsʰv̩˩˥}\newline
\classe{名词}\ton{LH}\begin{définition}\peng{Vinegar.}\end{définition}
\begin{définition}\pcmn{醋(汉语借词)}\end{définition}
\begin{définition}\pfra{Vinaigre.}\end{définition}
\end{entrée}

\newpage\caractère{ʈ}

\begin{entrée}
{ʈæ˩α}{}{ⓔʈæ˩α}\formedesurface{ʈæ˩˥}\newline
\classe{动词}
\sens{1}
\begin{définition}\peng{To lock up (animals), to close (a door).}\end{définition}
\begin{définition}\pcmn{关(门、羊)}\end{définition}
\begin{définition}\pfra{Fermer, refermer; enfermer (ex.: des moutons); aussi: fermer une route.}\end{définition}
\begin{exemple}\pnru{bv̩˩qo˩ ʈæ˥}\hspace{5pt}\peng{to close the stable}\hspace{5pt}\pcmn{关牛圈}\hspace{5pt}\pfra{fermer l'étable}\end{exemple}
\begin{exemple}\pnru{tʰi˧-ʈæ˩}\hspace{5pt}\peng{|fg{dur}: to close}\hspace{5pt}\pcmn{关门}\hspace{5pt}\pfra{|fg{dur}: fermer}\end{exemple}
\begin{exemple}\pnru{kʰi˧ ʈæ˥}\hspace{5pt}\peng{to close the door}\hspace{5pt}\pcmn{关门}\hspace{5pt}\pfra{fermer la porte}\end{exemple}\sens{2}
\begin{définition}\peng{To tie (a knot).}\end{définition}
\begin{définition}\pcmn{扣(扣子)、系、结}\end{définition}
\begin{définition}\pfra{Nouer (un noeud).}\end{définition}
\end{entrée}

\begin{entrée}
{ʈæ˧bɤ˧}{}{ⓔʈæ˧bɤ˧}\formedesurface{ʈæ˧bɤ˧}\newline
\classe{名词}\ton{M}
\paradigme{\pcmn{:} \p{}}
\begin{définition}\peng{Buddhist monk, lama, Buddhist nun.}\end{définition}
\begin{définition}\pcmn{和尚,尼姑}\end{définition}
\begin{définition}\pfra{Moine bouddhiste, nonne bouddhiste.}\end{définition}
\begin{exemple}\pnru{ʈæ˧bɤ˧ʈʂʰo˧}\hspace{5pt}\peng{same meaning}\hspace{5pt}\pcmn{同上}\hspace{5pt}\pfra{même sens}\end{exemple}
\begin{exemple}\pnru{ʈæ˧bɤ˧ ʝi˧-hĩ˧-hĩ˧}\hspace{5pt}\peng{person who is a monk}\hspace{5pt}\pcmn{当和尚的人}\hspace{5pt}\pfra{même sens}\end{exemple}
\begin{exemple}\pnru{hæ˧ʈæ˩bɤ˩}\hspace{5pt}\peng{Chinese monk}\hspace{5pt}\pcmn{汉人和尚}\hspace{5pt}\pfra{moine chinois}\end{exemple}
\end{entrée}

\begin{entrée}
{ʈæ˩ɖɯ˧}{}{ⓔʈæ˩ɖɯ˧}\formedesurface{ʈæ˩ɖɯ˥}\newline
\classe{形容词}\ton{LM}\begin{définition}\peng{Satisfied, quiet.}\end{définition}
\begin{définition}\pcmn{安乐}\end{définition}
\begin{définition}\pfra{Satisfait, tranquille.}\end{définition}
\begin{exemple}\pnru{mɤ˧-ʈæ˩ɖɯ˩}\hspace{5pt}\peng{dissatisfied, restless}\hspace{5pt}\pcmn{不高兴、不安}\hspace{5pt}\pfra{mécontent, furieux}\end{exemple}
\begin{exemple}\pnru{ə˧mɑ˧ | tsʰi˧-ɲi˧ | ʈæ˩ɖɯ˧ tʰi˧-dzi˩-dʑo˩!}\hspace{5pt}\peng{Today, Ama is sitting quietly!}\hspace{5pt}\pcmn{今天,阿妈安乐地坐着。}\hspace{5pt}\pfra{Aujourd'hui, Ama est assise bien tranquille!}\end{exemple}
\begin{exemple}\pnru{ʈʂʰɯ˧-ɳɯ˧ | njɤ˧-ki˧ | mɤ˧-ʈæ˩ɖɯ˩-hĩ˩ ʐwɤ˩!}\hspace{5pt}\peng{He told me unpleasant things! / He told me vexing things! / He told me some things that make me frustrated/dissatisfied!}\hspace{5pt}\pcmn{他跟我说了一些让我不安的(事情)! / 他跟我说的,让我生气!}\hspace{5pt}\pfra{il m'a dit des choses qui fâchent! (=il m'a vexé, il m'a dit des choses désobligeantes)}\end{exemple}
\end{entrée}

\begin{entrée}
{ʈæ˧kwæ˧˥}{}{ⓔʈæ˧kwæ˧˥}\formedesurface{ʈæ˧kwæ˧˥}\newline
\classe{形容词}\ton{MH\#}\begin{définition}\peng{Prodigal, wasteful.}\end{définition}
\begin{définition}\pcmn{爱浪费}\end{définition}
\begin{définition}\pfra{Prodigue, qui dépense tout.}\end{définition}
\begin{exemple}\pnru{ʈʂʰɯ˧ | ʈæ˧kwæ˧-hĩ˥ | ɖɯ˧-v̩˧ ɲi˩.}\hspace{5pt}\peng{(S)he is a prodigal person.}\hspace{5pt}\pcmn{他是爱浪费的人。}\hspace{5pt}\pfra{C'est un prodigue/quelqu'un qui dépense tout/qui mène la maison à la ruine.}\end{exemple}
\end{entrée}

\begin{entrée}
{ʈæ˧pv̩˩}{}{ⓔʈæ˧pv̩˩}\formedesurface{ʈæ˧pv̩˩}\newline
\classe{形容词}\ton{L\#}\begin{définition}\peng{Skinny, thin (person).}\end{définition}
\begin{définition}\pcmn{瘦(人很瘦)}\end{définition}
\begin{définition}\pfra{Maigre, sec (personne maigre, au corps sec).}\end{définition}
\end{entrée}

\begin{entrée}
{ʈæ˧qo˧}{}{ⓔʈæ˧qo˧}\formedesurface{ʈæ˧qo˧}\newline
\classe{助词}\ton{M}\begin{définition}\peng{At bottom, at the bottom of.}\end{définition}
\begin{définition}\pcmn{底下}\end{définition}
\begin{définition}\pfra{Au fond de.}\end{définition}
\begin{exemple}\pnru{hi˩nɑ˧mi˧-ʈæ˧qo˥}\hspace{5pt}\peng{at the bottom of the Lake}\hspace{5pt}\pcmn{在湖底下}\hspace{5pt}\pfra{au fond du Lac}\end{exemple}
\begin{exemple}\pnru{ʈæ˧qo˧ tɕɯ˧}\hspace{5pt}\peng{to place at the bottom of…}\hspace{5pt}\pcmn{放在底下}\hspace{5pt}\pfra{mettre au fond de…}\end{exemple}
\begin{exemple}\pnru{hi˩nɑ˧mi˧, | ʈæ˧ mɤ˧-do˩; | hĩ˧-nv̩˥mi˩, | ɳv̩˧ mɤ˧-tʰɑ˩.}\hspace{5pt}\peng{“One can't see to the bottom of the Lake; one can't know the heart of men." (Proverb that comes up in courting songs.)}\hspace{5pt}\pcmn{“人的心,湖底藏:看不清,摸不透!” 直译:“湖,(我们)看不到(它的)底下。人的心,是知道不了的!”(情歌里的一个谚语)}\hspace{5pt}\pfra{«On ne voit pas le fond du lac; on ne connaît pas le cœur des hommes!» (Proverbe qui apparaît dans les chansons que se chantaient les jeunes gens se faisant la cour.)}\end{exemple}
\end{entrée}

\begin{entrée}
{ʈæ˧ʂɯ˧}{}{ⓔʈæ˧ʂɯ˧}\formedesurface{ʈæ˧ʂɯ˧}\newline
\classe{名词}\ton{M}\begin{définition}\peng{Masculine given name.}\end{définition}
\begin{définition}\pcmn{男性名字}\end{définition}
\begin{définition}\pfra{Prénom masculin.}\end{définition}
\end{entrée}

\begin{entrée}
{ʈæ˧ʂɯ˧-ɖɯ˩mɑ˩}{}{ⓔʈæ˧ʂɯ˧-ɖɯ˩mɑ˩}\formedesurface{ʈæ˧ʂɯ˧ɖɯ˩mɑ˩}\newline
\classe{名词}\ton{-L}\begin{définition}\peng{Feminine given name.}\end{définition}
\begin{définition}\pcmn{女性名字}\end{définition}
\begin{définition}\pfra{Prénom féminin.}\end{définition}
\end{entrée}

\begin{entrée}
{ʈæ˧ʂɯ˧-ɬɑ˩mv̩˩}{}{ⓔʈæ˧ʂɯ˧-ɬɑ˩mv̩˩}\formedesurface{ʈæ˧ʂɯ˧ɬɑ˩mv̩˩}\newline
\classe{名词}\ton{-L}\begin{définition}\peng{Feminine given name.}\end{définition}
\begin{définition}\pcmn{女性名字}\end{définition}
\begin{définition}\pfra{Prénom féminin.}\end{définition}
\end{entrée}

\begin{entrée}
{ʈæ˧ʂɯ˧-pæ˩pʰæ˩}{}{ⓔʈæ˧ʂɯ˧-pæ˩pʰæ˩}\formedesurface{ʈæ˧ʂɯ˧pæ˩pʰæ˩}\newline
\classe{名词}\ton{-L}\begin{définition}\peng{Masculine given name.}\end{définition}
\begin{définition}\pcmn{男性名字}\end{définition}
\begin{définition}\pfra{Prénom masculin.}\end{définition}
\end{entrée}

\begin{entrée}
{ʈæ˧ʂɯ˧-tsʰi˩ti˩}{}{ⓔʈæ˧ʂɯ˧-tsʰi˩ti˩}\formedesurface{ʈæ˧ʂɯ˧tsʰi˩ti˩}\newline
\classe{名词}\ton{-L}\begin{définition}\peng{Masculine given name.}\end{définition}
\begin{définition}\pcmn{男性名字}\end{définition}
\begin{définition}\pfra{Prénom masculin.}\end{définition}
\end{entrée}

\begin{entrée}
{ʈæ˧ʂɯ˧-ʈæ˩ʈv̩˩}{}{ⓔʈæ˧ʂɯ˧-ʈæ˩ʈv̩˩}\formedesurface{ʈæ˧ʂɯ˧ʈæ˩ʈv̩˩}\newline
\classe{名词}\ton{-L}\begin{définition}\peng{Masculine given name.}\end{définition}
\begin{définition}\pcmn{男性名字}\end{définition}
\begin{définition}\pfra{Prénom masculin.}\end{définition}
\end{entrée}

\begin{entrée}
{ʈæ˩tsʰo\#˥}{₁}{ⓔʈæ˩tsʰo\#˥ⓗ1}\formedesurface{ʈæ˩tsʰo˥}\newline
\classe{名词}\ton{LM+\#H}
1
\paradigme{\pcmn{:} \p{}}
\begin{définition}\peng{Class, group, set (of monks).}\end{définition}
\begin{définition}\pcmn{班、小组}\end{définition}
\begin{définition}\pfra{Classe, groupe, ensemble (de prêtres).}\end{définition}
\begin{exemple}\pnru{ʈæ˩tsʰo˧ | ɖɯ˧-ɭɯ˧}\hspace{5pt}\peng{a group (of priests)}\hspace{5pt}\pcmn{一个小组、一帮(和尚)}\hspace{5pt}\pfra{une classe, un groupe (de prêtres)}\end{exemple}
\end{entrée}

\begin{entrée}
{ʈæ˩tsʰo\#˥}{₂}{ⓔʈæ˩tsʰo\#˥ⓗ2}\formedesurface{ɖɯ˧ ʈæ˩tsʰo˩}\newline
\classe{量词}\ton{LM+\#H}
2\begin{définition}\peng{Daeco.}\end{définition}
\begin{définition}\pcmn{量词:和尚(一帮、一班)}\end{définition}
\begin{définition}\pfra{Auto-classificateur des classes/groupes (de prêtres).}\end{définition}
\begin{exemple}\pnru{ɖɯ˧-ʈæ˩tsʰo˩}\hspace{5pt}\peng{a group (of monks)}\hspace{5pt}\pcmn{一班(和尚)}\hspace{5pt}\pfra{un groupe (de prêtres)}\end{exemple}
\end{entrée}

\begin{entrée}
{ʈæ˩ʈv̩\#˥}{}{ⓔʈæ˩ʈv̩\#˥}\formedesurface{ʈæ˩ʈv̩˥}\newline
\classe{名词}\ton{LM+\#H}\begin{définition}\peng{Masculine given name.}\end{définition}
\begin{définition}\pcmn{男性名字}\end{définition}
\begin{définition}\pfra{Prénom masculin.}\end{définition}
\end{entrée}

\begin{entrée}
{ʈɤ˧α}{}{ⓔʈɤ˧α}\formedesurface{ʈɤ˧}\newline
\classe{动词}\ton{Mα}\begin{définition}\peng{To pull.}\end{définition}
\begin{définition}\pcmn{拉、拽}\end{définition}
\begin{définition}\pfra{Tirer.}\end{définition}
\begin{exemple}\pnru{tso˧∼tso˧ ʈɤ˩(-ze˩)}\hspace{5pt}\peng{to pull something}\hspace{5pt}\pcmn{拉拽东西}\hspace{5pt}\pfra{tirer quelque chose}\end{exemple}
\begin{exemple}\pnru{mv̩˧ʐe˧ qʰæ˩ | le˧-wo˧-ʈɤ˥-di˩}\hspace{5pt}\peng{periphrase to refer to the trigger of a gun: the part that one pulls to shoot}\hspace{5pt}\pcmn{扳机}\hspace{5pt}\pfra{périphrase pour désigner la gâchette d'un pistolet: ce qu'on tire vers soi pour faire feu}\end{exemple}
\end{entrée}

\begin{entrée}
{ʈi˥α}{}{ⓔʈi˥α}\formedesurface{ɖɯ˧ ʈi˥}\newline
\classe{量词}\ton{Hα}\begin{définition}\peng{A handspan (between the thumb and index). The distance between the thumb and the middle finger is not commonly used.}\end{définition}
\begin{définition}\pcmn{量词:拃(大拇指和食指之间的距离。一般不用大拇指和中指之间的距离。)}\end{définition}
\begin{définition}\pfra{Empan: distance entre le pouce et l'index écartés. D'ordinaire, on n'emploie pas la distance entre pouce et majeur.}\end{définition}
\end{entrée}

\begin{entrée}
{ʈi˩α}{}{ⓔʈi˩α}\formedesurface{ʈi˩˥}\newline
\classe{动词}\ton{Lα}\begin{définition}\peng{To get up.}\end{définition}
\begin{définition}\pcmn{起(如:起来,起床)}\end{définition}
\begin{définition}\pfra{Se lever.}\end{définition}
\begin{exemple}\pnru{gɤ˩-ʈi˧}\hspace{5pt}\peng{to get up}\hspace{5pt}\pcmn{起来}\hspace{5pt}\pfra{se lever}\end{exemple}
\begin{exemple}\pnru{ʑi˧ ʈi˥}\hspace{5pt}\peng{to wake up}\hspace{5pt}\pcmn{醒来}\hspace{5pt}\pfra{se réveiller}\end{exemple}
\begin{exemple}\pnru{ʑi˧ gɤ˧-ʈi˩}\hspace{5pt}\peng{to wake up}\hspace{5pt}\pcmn{醒来}\hspace{5pt}\pfra{se réveiller}\end{exemple}
\begin{exemple}\pnru{gɤ˩ mɤ˥-ʈi˩}\hspace{5pt}\peng{not to get up}\hspace{5pt}\pcmn{不起床}\hspace{5pt}\pfra{ne pas se lever}\end{exemple}
\begin{exemple}\pnru{mɤ˧-ʈi˩-sɯ˩!}\hspace{5pt}\peng{(She/he) has not got up yet / is not up yet!}\hspace{5pt}\pcmn{还没起床!}\hspace{5pt}\pfra{(Il/elle) n'est pas encore levé(e)!}\end{exemple}
\begin{exemple}\pnru{le˧-ʈi˩-ze˩!}\hspace{5pt}\peng{(She/he) has got up!}\hspace{5pt}\pcmn{起床了!}\hspace{5pt}\pfra{(il/elle) s'est levé(e)!}\end{exemple}
\begin{exemple}\pnru{ɖɯ˧-ʈi˧∼ʈi˥-ɻ̍˩}\hspace{5pt}\peng{|fg{delimitative} |fg{red} |fg{inceptive}}\hspace{5pt}\pcmn{起来一下}\hspace{5pt}\pfra{|fg{délimitatif} |fg{red} |fg{inchoatif}}\end{exemple}
\end{entrée}

\begin{entrée}
{ʈɯ˧˥}{}{ⓔʈɯ˧˥}\formedesurface{ʈɯ˧˥}\newline
\classe{动词}\ton{MH}\begin{définition}\peng{To blanch: to scald with boiling water, as a preliminary stage in cooking (e.g. for dried vegetables) or in making thread (from linen).}\end{définition}
\begin{définition}\pcmn{以滚水将蔬菜或亚麻灼过}\end{définition}
\begin{définition}\pfra{Blanchir à l'eau bouillante: du lin pour préparer du fil pour le tissage, des légumes séchés avant de les utiliser pour la cuisine…}\end{définition}
\begin{exemple}\pnru{tʰi˧-ʈɯ˧˥}\hspace{5pt}\peng{|fg{dur}}\hspace{5pt}\pcmn{|fg{dur}}\hspace{5pt}\pfra{|fg{dur}}\end{exemple}
\begin{exemple}\pnru{dʑɯ˩tsʰi˩-qo˥ | tʰi˧-ʈɯ˧˥ / dʑɯ˩tsʰi˩-qo˥ | ʈɯ˧˥}\hspace{5pt}\peng{to blanch with boiling water}\hspace{5pt}\pcmn{以滚水灼过}\hspace{5pt}\pfra{blanchir à l'eau bouillante}\end{exemple}
\begin{exemple}\pnru{dʑɯ˧-qo˧ | ʈɯ˧˥}\hspace{5pt}\peng{to blanch with water}\hspace{5pt}\pcmn{以水灼过}\hspace{5pt}\pfra{blanchir à l'eau}\end{exemple}
\begin{exemple}\pnru{v˩tsʰɤ˧ ʈɯ˥}\hspace{5pt}\peng{to blanch vegetables}\hspace{5pt}\pcmn{灼蔬菜}\hspace{5pt}\pfra{blanchir des légumes}\end{exemple}
\begin{exemple}\pnru{sɑ˧, | ʈɯ˧-kv˥!}\hspace{5pt}\peng{Linen needs to be blanched!}\hspace{5pt}\pcmn{亚麻,要灼过!}\hspace{5pt}\pfra{Le chanvre, ça se blanchit! (Au cours de la préparation du chanvre pour en faire du fil, il faut le blanchir.)}\end{exemple}
\end{entrée}

\begin{entrée}
{ʈɯ˧α}{}{ⓔʈɯ˧α}\formedesurface{ʈɯ˧}\newline
\classe{动词}\ton{Mα}\begin{définition}\peng{To set in place.}\end{définition}
\begin{définition}\pcmn{安装、摆好}\end{définition}
\begin{définition}\pfra{Mettre en place, installer à sa juste place.}\end{définition}
\begin{exemple}\pnru{ʂe˧kʰɯ˧ tʰi˧-ʈɯ˧˥, | v̩˧ | tʰi˧-ʈɯ˧}\hspace{5pt}\peng{to set the tripod in place (in the hearth); to set the large pot in place (as part of the final steps in setting up the new home, after a new house has been built)}\hspace{5pt}\pcmn{(建完新房后)安装三脚架、把锅摆好(在三脚架上)}\hspace{5pt}\pfra{mettre en place le trépied de fer dans le foyer, mettre en place la grande casserole (sur le trépied) (Contexte: description de la «pendaison de crémaillère», dans une nouvelle maison)}\end{exemple}
\begin{exemple}\pnru{tsʰo˩-ɻ̃˩˥ | dʑɯ˩ mɤ˩-ʈɯ˩˥, | lɑ˧-ʂe˧ | kʰv̩˧ tʰɑ˩-ki˩!}\hspace{5pt}\peng{“Human bones must not be put in water; tiger's flesh must not be given to the dog!" (Explanation: corpses were not buried in water, unlike in certain Tibetan customs. Neither the body, nor the ashes of cremation, must be put in water.)}\hspace{5pt}\pcmn{“人骨头,莫碰水!老虎肉,莫给狗!”(这个谚语,来强调摩梭与藏族的一些不同习惯:摩梭禁止让尸体或骨灰沾水。)}\hspace{5pt}\pfra{«Les ossements humains, on ne les met pas à l'eau! La chair du tigre, on ne la donne pas au chien!» Sens: on n'enterre pas les gens dans l'eau (à la différence de certaines coutumes tibétaines); on prenait soin de n'immerger dans l'eau, ni le corps, ni les cendres après la crémation.}\end{exemple}
\end{entrée}

\begin{entrée}
{ʈɯ˧ʈʰæ\#˥}{}{ⓔʈɯ˧ʈʰæ\#˥}\formedesurface{ʈɯ˧ʈʰæ˧}\newline
\classe{名词}\ton{\#H}
\paradigme{\pcmn{:} \p{}}
\begin{définition}\peng{Patrimony, family wealth, property.}\end{définition}
\begin{définition}\pcmn{家底、财产(贵重物品)}\end{définition}
\begin{définition}\pfra{Patrimoine.}\end{définition}
\begin{exemple}\pnru{ɑ˩ʁo˧ ʈɯ˧ʈʰæ˧!}\hspace{5pt}\peng{Prosperity to the family!}\hspace{5pt}\pcmn{祝你们家发财!}\hspace{5pt}\pfra{Prospérité à la famille!}\end{exemple}
\begin{exemple}\pnru{ʈʂʰɯ˧ | ʈɯ˧ʈʰæ˧ | ɖwæ˧˥ | dʑo˧-ʝi˧!}\hspace{5pt}\peng{(S)he has a large patrimony / His/her family is rich!}\hspace{5pt}\pcmn{他家底很好! / 他家有钱!}\hspace{5pt}\pfra{Il/elle est riche! / Sa famille est riche!}\end{exemple}
\end{entrée}

\begin{entrée}
{ʈv̩˩}{}{ⓔʈv̩˩}\formedesurface{ʈv̩˧}\newline
\classe{名词}\ton{L}
\paradigme{\pcmn{:} \p{}}
\begin{définition}\peng{Knot.}\end{définition}
\begin{définition}\pcmn{死扣、死结}\end{définition}
\begin{définition}\pfra{Noeud.}\end{définition}
\begin{exemple}\pnru{ɖɯ˧-ʈv̩˩}\hspace{5pt}\peng{a knot}\hspace{5pt}\pcmn{一个死结}\hspace{5pt}\pfra{un noeud}\end{exemple}
\begin{exemple}\pnru{ɖɯ˧-ʈv̩˩ | tʰi˧-ʈv̩˩}\hspace{5pt}\peng{to tie a knot}\hspace{5pt}\pcmn{打一个死结}\hspace{5pt}\pfra{faire un noeud}\end{exemple}
\end{entrée}

\begin{entrée}
{ʈv̩˩α}{₁}{ⓔʈv̩˩αⓗ1}\formedesurface{ʈv̩˩˥}\newline
\classe{动词}\ton{Lα}
1\begin{définition}\peng{To weave (bamboo).}\end{définition}
\begin{définition}\pcmn{编(竹子)}\end{définition}
\begin{définition}\pfra{Tresser (vannerie).}\end{définition}
\begin{exemple}\pnru{qʰwɤ˧tʰv̩˧ ʈv̩˥}\hspace{5pt}\peng{to weave a basket for carrying water}\hspace{5pt}\pcmn{编背水的背篓}\hspace{5pt}\pfra{tresser une hotte dorsale (en bambou) pour porter l'eau}\end{exemple}
\begin{exemple}\pnru{mi˩ɬi˩ ʈv̩˥}\hspace{5pt}\peng{to weave bamboo}\hspace{5pt}\pcmn{编竹子}\hspace{5pt}\pfra{tresser du bambou}\end{exemple}
\begin{exemple}\pnru{tso˧∼tso˧ ʈv̩˥}\hspace{5pt}\peng{to weave things}\hspace{5pt}\pcmn{编东西}\hspace{5pt}\pfra{tresser des choses}\end{exemple}
\end{entrée}

\begin{entrée}
{ʈv̩˩α}{₂}{ⓔʈv̩˩αⓗ2}\formedesurface{ʈv̩˩˥}\newline
\classe{动词}\ton{Lα}
2\begin{définition}\peng{To throw (a stone at someone).}\end{définition}
\begin{définition}\pcmn{掷(掷石头)}\end{définition}
\begin{définition}\pfra{Lancer (une pierre sur quelqu'un).}\end{définition}
\begin{exemple}\pnru{mɤ˧-ʈv̩˩}\hspace{5pt}\peng{|fg{neg}}\hspace{5pt}\pcmn{|fg{neg}}\hspace{5pt}\pfra{|fg{neg}}\end{exemple}
\begin{exemple}\pnru{lv̩˧mi˧ ʈv̩˩}\hspace{5pt}\peng{to throw a stone}\hspace{5pt}\pcmn{掷石头}\hspace{5pt}\pfra{lancer une pierre}\end{exemple}
\begin{exemple}\pnru{tso˧∼tso˧ ʈv̩˥}\hspace{5pt}\peng{to throw things}\hspace{5pt}\pcmn{掷东西}\hspace{5pt}\pfra{jeter des choses}\end{exemple}
\end{entrée}

\begin{entrée}
{ʈv̩˩β}{}{ⓔʈv̩˩β}\formedesurface{ɖɯ˧ ʈv̩˩}\newline
\classe{量词}\ton{Lβ}\begin{définition}\peng{Classifier: a large chunk of: a piece larger than a mouthful. The size can range from a chunk of meat corresponding to one serving for one guest, to a quarter of meat weighing several kilos.}\end{définition}
\begin{définition}\pcmn{量词:大块,如:一块肉,从一个人的份到几公斤的重量}\end{définition}
\begin{définition}\pfra{Classificateur des quartiers/pièces: morceaux de taille supérieure à une bouchée. Il peut s'agir d'un morceau de viande qu'on donne à un convive, et qui se mange en plusieurs bouchées, mais aussi d'un gros quartier de viande (plusieurs kilos).}\end{définition}
\end{entrée}

\begin{entrée}
{ʈv̩˩qʰv̩˩}{}{ⓔʈv̩˩qʰv̩˩}\formedesurface{ʈv̩˩qʰv̩˩˥}\newline
\classe{名词}\ton{L}
\paradigme{\pcmn{:} \p{}}
\begin{définition}\peng{Slipknot.}\end{définition}
\begin{définition}\pcmn{活扣}\end{définition}
\begin{définition}\pfra{Noeud coulant.}\end{définition}
\begin{exemple}\pnru{ʈv̩˩qʰv̩˩˥ | tʰi˧-ʈv̩˩}\hspace{5pt}\peng{to tie a slipknot}\hspace{5pt}\pcmn{打活扣}\hspace{5pt}\pfra{faire un noeud coulant}\end{exemple}
\begin{exemple}\pnru{ʈv̩˩qʰv̩˩˥ | ɖɯ˧-ɭɯ˧ | tʰi˧-ʈv̩˩}\hspace{5pt}\peng{to tie a slipknot}\hspace{5pt}\pcmn{打一个活扣}\hspace{5pt}\pfra{faire un noeud coulant}\end{exemple}
\end{entrée}

\begin{entrée}
{ʈwæ˧˥}{}{ⓔʈwæ˧˥}\formedesurface{ʈwæ˧˥}\newline
\classe{动词}\ton{MH}\begin{définition}\peng{To fall down (on a slippery road).}\end{définition}
\begin{définition}\pcmn{跌倒(路很滑)}\end{définition}
\begin{définition}\pfra{Tomber (en glissant).}\end{définition}
\end{entrée}

\begin{entrée}
{ʈwæ˩α}{}{ⓔʈwæ˩α}\formedesurface{ʈwæ˩˥}\newline
\classe{动词}\ton{Lα}\begin{définition}\peng{To freeze, to solidify.}\end{définition}
\begin{définition}\pcmn{冻}\end{définition}
\begin{définition}\pfra{Geler, se figer.}\end{définition}
\begin{exemple}\pnru{dʑɯ˩ ʈwæ˩˥}\hspace{5pt}\peng{water freezes}\hspace{5pt}\pcmn{水冻成冰}\hspace{5pt}\pfra{l'eau gèle, il gèle}\end{exemple}
\begin{exemple}\pnru{dʑɯ˩pʰæ˩ ʈwæ˧-ze˩}\hspace{5pt}\peng{ice has formed}\hspace{5pt}\pcmn{水冻成冰了。}\hspace{5pt}\pfra{l'eau a gelé, de la glace s'est formée}\end{exemple}
\begin{exemple}\pnru{ɖɯ˧-ʈwæ˧∼ʈwæ˥-ɻ̍˩ kʰɯ˩}\hspace{5pt}\peng{to put to freeze, to put in the deep freeze}\hspace{5pt}\pcmn{冷冻,放在冷箱}\hspace{5pt}\pfra{faire geler, mettre à congeler}\end{exemple}
\end{entrée}

\begin{entrée}
{ʈwɤ˧α}{}{ⓔʈwɤ˧α}\formedesurface{ʈwɤ˧}\newline
\classe{动词}\ton{Mα}\begin{définition}\peng{To sing (of bird), to cock-a-doodle-doo (cock).}\end{définition}
\begin{définition}\pcmn{啼,鸡叫}\end{définition}
\begin{définition}\pfra{Le coq chante/fait cocorico; un oiseau chante.}\end{définition}
\begin{exemple}\pnru{æ̃˩ ʈwɤ˧ (+ze˧)}\hspace{5pt}\peng{The cock sings.}\hspace{5pt}\pcmn{鸡叫。}\hspace{5pt}\pfra{Le coq chante/fait cocorico.}\end{exemple}
\end{entrée}

\begin{entrée}
{ʈʰæ˧˥}{}{ⓔʈʰæ˧˥}\newline
\classe{动词}
\sens{1}
\begin{définition}\peng{To bite; to sting.}\end{définition}
\begin{définition}\pcmn{咬、叮}\end{définition}
\begin{définition}\pfra{Mordre (mordre à belles dents dans quelque chose); piquer (une abeille pique quelqu'un).}\end{définition}
\begin{exemple}\pnru{tso˧∼tso˧ ʈʰæ˩(-ze˩)}\hspace{5pt}\peng{to bite things}\hspace{5pt}\pcmn{咬东西}\hspace{5pt}\pfra{mordre quelque chose}\end{exemple}
\begin{exemple}\pnru{hĩ˧ ʈʰæ˩}\hspace{5pt}\peng{to bite someone (e.g. a dog bites a stranger)}\hspace{5pt}\pcmn{咬人}\hspace{5pt}\pfra{mordre quelqu'un (ex.: un chien mord un inconnu de passage)}\end{exemple}\sens{2}
\begin{définition}\peng{To fit, to adjust, to match: e.g. when building a house, to pieces of carpentry fit each other exactly, and ‘bite' into each other to perfection.}\end{définition}
\begin{définition}\pcmn{对号、合适、相配:建房时,两块木材调剂地刚好合适,好像互相“咬紧”的样子}\end{définition}
\begin{définition}\pfra{S'emboîter, s'ajuster (au sujet de pièces de charpenterie); emploi figuré de ‘mordre’: les pièces s'ajustent comme si elles mordaient les unes dans les autres.}\end{définition}
\end{entrée}

\newpage\caractère{†}

\begin{entrée}
{†ʈʰæ˩}{}{ⓔ†ʈʰæ˩}\formedesurface{--}\newline
\classe{名词}\ton{L}\begin{définition}\peng{Skirt (monosyllabic form extracted from the set phrase ‘to wear the skirt').}\end{définition}
\begin{définition}\pcmn{裙子(单音节)}\end{définition}
\begin{définition}\pfra{Jupe; monosyllabe extrait d'après le comportement dans l'expression figée /ʈʰæ˩ ki˩˥/ ‘enfiler la jupe', avec verbe au ton La (nom du rituel de passage à l'âge adulte). Le monosyllabe n'est pas usité hors de cette expression. Par exemple, */ʈʰæ˩ ɲi˩˥/ ‘c'est une jupe' est catégoriquement refusé par F4.}\end{définition}
\begin{exemple}\pnru{ʈʰæ˧ | le˧-ki˩}\hspace{5pt}\peng{to put on the skirt (|fg{accomp})}\hspace{5pt}\pcmn{穿上裙子}\hspace{5pt}\pfra{enfiler la jupe (|fg{accomp})}\end{exemple}
\begin{exemple}\pnru{ʈʰæ˩ ki˩˥}\hspace{5pt}\peng{the ritual of coming of age for women: “putting on the skirt"}\hspace{5pt}\pcmn{女人成年的礼仪:“穿裙子”}\hspace{5pt}\pfra{rituel de passage à l'âge adulte, pour les jeunes femmes: «enfiler la jupe»}\end{exemple}
\end{entrée}

\newpage\caractère{ʈ}

\begin{entrée}
{ʈʰæ˩ki˩}{}{ⓔʈʰæ˩ki˩}\formedesurface{ʈʰæ˩ki˩˥}\newline
\classe{动词}\ton{L}\begin{définition}\peng{To perform the ceremony for a person's coming of age.}\end{définition}
\begin{définition}\pcmn{举行女孩的成年礼}\end{définition}
\begin{définition}\pfra{Réaliser la cérémonie de passage à l'âge adulte des femmes.}\end{définition}
\begin{exemple}\pnru{ʈʰæ˩ki˩-ze˥!}\hspace{5pt}\peng{She has come of age! / The ceremony for her coming of age has been performed!}\hspace{5pt}\pcmn{穿裙了! / 行过穿裙礼了! / 她成年了!}\hspace{5pt}\pfra{Elle est adulte maintenant! / La cérémonie de passage à l'âge adulte a été réalisée!}\end{exemple}
\end{entrée}

\begin{entrée}
{ʈʰæ˧-mɤ˧-ʝi\#˥}{}{ⓔʈʰæ˧-mɤ˧-ʝi\#˥}\formedesurface{ʈʰæ˧mɤ˧ʝi˧}\newline
\classe{形容词}\ton{\#H}\begin{définition}\peng{Disorderly.}\end{définition}
\begin{définition}\pcmn{乱}\end{définition}
\begin{définition}\pfra{Désordonné.}\end{définition}
\begin{exemple}\pnru{ɑ˩ʁo˧ | ʈʰæ˧-mɤ˧-ʝi˧! |}\hspace{5pt}\peng{The house is in a mess!}\hspace{5pt}\pcmn{家很乱!}\hspace{5pt}\pfra{la maison est en grand désordre!}\end{exemple}
\begin{exemple}\pnru{ʈʰæ˧-mɤ˧-ʝi˧ ɲi˥! |}\hspace{5pt}\peng{It's a real mess!}\hspace{5pt}\pcmn{真乱!}\hspace{5pt}\pfra{C'est vraiment le désordre!}\end{exemple}
\end{entrée}

\begin{entrée}
{ʈʰæ˧mi˧-ɳɯ˩}{}{ⓔʈʰæ˧mi˧-ɳɯ˩}\formedesurface{ʈʰæ˧mi˧ɳɯ˩}\newline
\classe{助词}\ton{L\#}\begin{définition}\peng{Really, actually.}\end{définition}
\begin{définition}\pcmn{真的}\end{définition}
\begin{définition}\pfra{Vraiment, réellement.}\end{définition}
\begin{exemple}\pnru{ʈʂʰɯ˧ | ʈʰæ˧mi˧-ɳɯ˩ | go˩˥!}\hspace{5pt}\peng{(S)he is really ill!}\hspace{5pt}\pcmn{他真的病了!}\hspace{5pt}\pfra{Il/elle est vraiment malade!}\end{exemple}
\end{entrée}

\begin{entrée}
{‑ʈʰæ˧qo˩}{}{ⓔ‑ʈʰæ˧qo˩}\formedesurface{ʈʰæ˧qo˩}\newline
\classe{}\ton{L\#}\begin{définition}\peng{Under.}\end{définition}
\begin{définition}\pcmn{……之下、下面}\end{définition}
\begin{définition}\pfra{Sous (sous le ciel; sous la tente); au pied (d'une montagne).}\end{définition}
\end{entrée}

\begin{entrée}
{ʈʰæ˧qʰwɤ˧}{}{ⓔʈʰæ˧qʰwɤ˧}\formedesurface{ʈʰæ˧qʰwɤ˧}\newline
\classe{名词}\ton{M}
\paradigme{\pcmn{:} \p{}}
\begin{définition}\peng{Skirt.}\end{définition}
\begin{définition}\pcmn{裙子}\end{définition}
\begin{définition}\pfra{Jupe.}\end{définition}
\end{entrée}

\begin{entrée}
{ʈʰæ˩∼ʈʰæ˧˥}{}{ⓔʈʰæ˩∼ʈʰæ˧˥}\formedesurface{ʈʰæ˩ʈʰæ˧˥}\newline
\classe{动词}\ton{MH}\begin{définition}\peng{To itch.}\end{définition}
\begin{définition}\pcmn{痒}\end{définition}
\begin{définition}\pfra{Démanger.}\end{définition}
\begin{exemple}\pnru{le˧-ʈʰæ˩∼ʈʰæ˩-ze˩}\hspace{5pt}\peng{|fg{accomp} |fg{red} |fg{pfv}}\hspace{5pt}\pcmn{|fg{accomp} |fg{red} |fg{pfv}}\hspace{5pt}\pfra{|fg{accomp} |fg{red} |fg{pfv}}\end{exemple}
\end{entrée}

\begin{entrée}
{ʈʰɤ˥α}{}{ⓔʈʰɤ˥α}\formedesurface{ɖɯ˧ ʈʰɤ˥}\newline
\classe{量词}\ton{Hα}\begin{définition}\peng{A drop (of liquid).}\end{définition}
\begin{définition}\pcmn{量词:滴}\end{définition}
\begin{définition}\pfra{Goutte (une goutte de liquide).}\end{définition}
\end{entrée}

\begin{entrée}
{ʈʰɤ˧˥}{}{ⓔʈʰɤ˧˥}\formedesurface{ʈʰɤ˧˥}\newline
\classe{动词}\ton{MH}\begin{définition}\peng{To drip, to dribble.}\end{définition}
\begin{définition}\pcmn{滴(水往下滴)}\end{définition}
\begin{définition}\pfra{Goutter, dégouliner, couler goutte à goutte.}\end{définition}
\begin{exemple}\pnru{tʰi˧-ʈʰɤ˩∼ʈʰɤ˩}\hspace{5pt}\peng{|fg{dur} |fg{red}}\hspace{5pt}\pcmn{滴着滴着}\hspace{5pt}\pfra{|fg{dur} |fg{red}}\end{exemple}
\end{entrée}

\begin{entrée}
{ʈʰi˩α}{}{ⓔʈʰi˩α}\formedesurface{ʈʰi˩˥}\newline
\classe{形容词}\ton{Lα}\begin{définition}\peng{Tired, weary.}\end{définition}
\begin{définition}\pcmn{累、疲倦、精疲力竭}\end{définition}
\begin{définition}\pfra{Fatigué.}\end{définition}
\begin{exemple}\pnru{le˧-ʈʰi˩-ze˩}\hspace{5pt}\peng{|fg{accomp} \_ |fg{pfv}}\hspace{5pt}\pcmn{累了}\hspace{5pt}\pfra{|fg{accomp} \_ |fg{pfv}}\end{exemple}
\begin{exemple}\pnru{njɤ˧ | ʈʰi˩˥!}\hspace{5pt}\peng{I am tired!}\hspace{5pt}\pcmn{我累了!}\hspace{5pt}\pfra{je suis fatigué!}\end{exemple}
\begin{exemple}\pnru{njɤ˧ | ʈʰi˩-ze˥!}\hspace{5pt}\peng{I am tired!}\hspace{5pt}\pcmn{我累了!}\hspace{5pt}\pfra{je suis fatigué!}\end{exemple}
\end{entrée}

\begin{entrée}
{ʈʰɯ˩α}{}{ⓔʈʰɯ˩α}\formedesurface{ʈʰɯ˩˥}\newline
\classe{动词}\ton{Lα}\begin{définition}\peng{To sneeze.}\end{définition}
\begin{définition}\pcmn{打喷嚏}\end{définition}
\begin{définition}\pfra{Éternuer.}\end{définition}
\begin{exemple}\pnru{ɖɯ˧-ʈʰɯ˧∼ʈʰɯ˥}\hspace{5pt}\peng{|fg{inceptive} |fg{red}}\hspace{5pt}\pfra{|fg{inchoatif} |fg{red}}\end{exemple}
\end{entrée}

\begin{entrée}
{ʈʰɯ˩β}{}{ⓔʈʰɯ˩β}\formedesurface{ʈʰɯ˩˥}\newline
\classe{动词}\ton{Lβ}\begin{définition}\peng{To drink.}\end{définition}
\begin{définition}\pcmn{喝}\end{définition}
\begin{définition}\pfra{Boire.}\end{définition}
\begin{exemple}\pnru{njɤ˧ | mɤ˧-ʈʰɯ˩}\hspace{5pt}\peng{I don't drink}\hspace{5pt}\pcmn{我不喝}\hspace{5pt}\pfra{je ne bois pas}\end{exemple}
\begin{exemple}\pnru{ʈʰɯ˩-ze˥}\hspace{5pt}\peng{|fg{pfv}}\hspace{5pt}\pcmn{喝了}\hspace{5pt}\pfra{|fg{pfv}}\end{exemple}
\begin{exemple}\pnru{le˧-ʈʰɯ˩-ze˩}\hspace{5pt}\peng{|fg{accomp} \_ |fg{pfv}}\hspace{5pt}\pcmn{|fg{accomp} \_ |fg{pfv}}\hspace{5pt}\pfra{|fg{accomp} \_ |fg{pfv}}\end{exemple}
\begin{exemple}\pnru{ʐɯ˧ ʈʰɯ˩}\hspace{5pt}\peng{to drink wine}\hspace{5pt}\pcmn{喝酒}\hspace{5pt}\pfra{boire du vin}\end{exemple}
\begin{exemple}\pnru{jɤ˧ ʈʰɯ˩}\hspace{5pt}\peng{to smoke (tobacco)}\hspace{5pt}\pcmn{抽烟}\hspace{5pt}\pfra{fumer (du tabac)}\end{exemple}
\begin{exemple}\pnru{dʑɯ˩qʰæ˩ ʈʰɯ˩˥}\hspace{5pt}\peng{to drink cold water}\hspace{5pt}\pcmn{喝凉水}\hspace{5pt}\pfra{boire de l'eau froide}\end{exemple}
\begin{exemple}\pnru{dʑɯ˩tsʰi˩ ʈʰɯ˩˥}\hspace{5pt}\peng{to drink hot water}\hspace{5pt}\pcmn{喝热水}\hspace{5pt}\pfra{boire de l'eau chaude}\end{exemple}
\begin{exemple}\pnru{li˩ ʈʰɯ˩}\hspace{5pt}\peng{to drink tea}\hspace{5pt}\pcmn{喝茶}\hspace{5pt}\pfra{boire du thé}\end{exemple}
\begin{exemple}\pnru{v̩˩dʑɯ˩ ʈʰɯ˩˥}\hspace{5pt}\peng{to drink soup}\hspace{5pt}\pcmn{喝汤}\hspace{5pt}\pfra{boire de la soupe}\end{exemple}
\begin{exemple}\pnru{dʑɯ˧ ʈʰɯ˧}\hspace{5pt}\peng{to drink water}\hspace{5pt}\pcmn{喝水}\hspace{5pt}\pfra{boire de l'eau}\end{exemple}
\begin{exemple}\pnru{njɤ˧ | dʑɯ˧ ʈʰɯ˧-ze˧}\hspace{5pt}\peng{I have drunk some water}\hspace{5pt}\pcmn{我喝了水}\hspace{5pt}\pfra{j'ai bu de l'eau}\end{exemple}
\begin{exemple}\pnru{njɤ˧ | dʑɯ˧ ʈʰɯ˧-zo˧-ho˩}\hspace{5pt}\peng{I'm going to have to drink water.}\hspace{5pt}\pcmn{我应该喝水了。}\hspace{5pt}\pfra{Il va falloir que je boive de l'eau.}\end{exemple}
\end{entrée}

\begin{entrée}
{ʈʂɑ˧tɑ˥}{}{ⓔʈʂɑ˧tɑ˥}\formedesurface{ʈʂɑ˧tɑ˥}\newline
\classe{名词}\ton{H\#}
\paradigme{\pcmn{:} \p{}}
\begin{définition}\peng{Sign.}\end{définition}
\begin{définition}\pcmn{记号}\end{définition}
\begin{définition}\pfra{Signe.}\end{définition}
\begin{exemple}\pnru{ʈʂɑ˧tɑ˥ ʝi˩}\hspace{5pt}\peng{to make a mark, to write a sign}\hspace{5pt}\pcmn{写一个符号、画一个符号}\hspace{5pt}\pfra{faire une marque, inscrire un signe}\end{exemple}
\begin{exemple}\pnru{ʈʂɑ˧tɑ˥ tɕi˩}\hspace{5pt}\peng{to write signs, to make marks}\hspace{5pt}\pcmn{写符号、画符号}\hspace{5pt}\pfra{écrire des signes, faire des marques (pas pour un unique signe: se rapproche de l'écriture d'un message/texte)}\end{exemple}
\end{entrée}

\begin{entrée}
{ʈʂæ˧˥}{₁}{ⓔʈʂæ˧˥ⓗ1}\formedesurface{ʈʂæ˧˥}\newline
\classe{动词}\ton{MH}
1\begin{définition}\peng{To rob, to steal.}\end{définition}
\begin{définition}\pcmn{抢劫、抢}\end{définition}
\begin{définition}\pfra{Voler, s'emparer de, extorquer, arracher.}\end{définition}
\begin{exemple}\pnru{le˧-ʈʂæ˧-ze˥}\hspace{5pt}\peng{|fg{accomp} \_ |fg{pfv}}\hspace{5pt}\pcmn{抢了}\hspace{5pt}\pfra{|fg{accomp} \_ |fg{pfv}}\end{exemple}
\begin{exemple}\pnru{tso˧∼tso˧ ʈʂæ˩}\hspace{5pt}\peng{to steal things}\hspace{5pt}\pcmn{抢东西}\hspace{5pt}\pfra{voler des choses}\end{exemple}
\begin{exemple}\pnru{le˧-ʈʂæ˧-po˥-hɯ˩(-ze˩)}\hspace{5pt}\peng{[(S)he] stole away (something)}\hspace{5pt}\pcmn{把东西抢走了}\hspace{5pt}\pfra{(il) a extorqué quelque chose et est parti avec}\end{exemple}
\begin{exemple}\pnru{hĩ˧ ʈʂæ˩}\hspace{5pt}\peng{to rob people, to steal from people}\hspace{5pt}\pcmn{抢劫}\hspace{5pt}\pfra{voler les gens, extorquer des choses aux gens}\end{exemple}
\end{entrée}

\begin{entrée}
{ʈʂæ˧˥}{₂}{ⓔʈʂæ˧˥ⓗ2}\formedesurface{ʈʂæ˧˥}\newline
\classe{动词}\ton{MH}
2\begin{définition}\peng{To send someone.}\end{définition}
\begin{définition}\pcmn{派人}\end{définition}
\begin{définition}\pfra{Envoyer qqun.}\end{définition}
\begin{exemple}\pnru{ɖɯ˧-v̩˧ ʈʂæ˧˥}\hspace{5pt}\peng{to send someone}\hspace{5pt}\pcmn{派一个人}\hspace{5pt}\pfra{envoyer quelqu'un}\end{exemple}
\begin{exemple}\pnru{hĩ˧ ʈʂæ˩}\hspace{5pt}\peng{as above}\hspace{5pt}\pcmn{同上}\hspace{5pt}\pfra{idem}\end{exemple}
\end{entrée}

\begin{entrée}
{ʈʂæ˧˥}{₃}{ⓔʈʂæ˧˥ⓗ3}\formedesurface{ʈʂæ˧˥}\newline
\classe{动词}\ton{MH}
3\begin{définition}\peng{To set, to attach (e.g. to sew a button, to put a saddle on a horse).}\end{définition}
\begin{définition}\pcmn{安上(如:缝扣子、安上马鞍)}\end{définition}
\begin{définition}\pfra{Fixer, accrocher (ex.: coudre un bouton sur un vêtement; attacher la selle sur un cheval).}\end{définition}
\begin{exemple}\pnru{pv̩˩ɭɯ˥ ʈʂæ˩}\hspace{5pt}\peng{to sew a button}\hspace{5pt}\pcmn{缝扣子}\hspace{5pt}\pfra{coudre un bouton (sur un vêtement)}\end{exemple}
\begin{exemple}\pnru{ʐwæ˧tɕi˥ ʈʂæ˩}\hspace{5pt}\peng{to put a saddle on a horse, to saddle a horse}\hspace{5pt}\pcmn{备鞍}\hspace{5pt}\pfra{attacher la selle d'un cheval; seller un cheval}\end{exemple}
\begin{exemple}\pnru{ɖɯ˧-ɲi˥, | so˧-ʂɯ˧ ʈʂæ˧˥!}\hspace{5pt}\peng{In one day [of caravan journey], one saddles (horses) three times!}\hspace{5pt}\pcmn{(走马帮,)一天备鞍三次!}\hspace{5pt}\pfra{Au cours d'une journée, on selle trois fois (les chevaux, lorsqu'on est parti en caravane)!}\end{exemple}
\end{entrée}

\begin{entrée}
{ʈʂæ˧˥}{₄}{ⓔʈʂæ˧˥ⓗ4}\formedesurface{ʈʂæ˧˥}\newline
\classe{名词}
4
\sens{1}\paradigme{\pcmn{:} \p{}}
\begin{définition}\peng{Articulation.}\end{définition}
\begin{définition}\pcmn{关节}\end{définition}
\begin{définition}\pfra{Articulation.}\end{définition}\sens{2}
\begin{définition}\peng{Period, epoch, age, era, span of time.}\end{définition}
\begin{définition}\pcmn{段(时间)、时代}\end{définition}
\begin{définition}\pfra{Période, époque; segment de temps.}\end{définition}
\end{entrée}

\begin{entrée}
{ʈʂæ˧˥α}{}{ⓔʈʂæ˧˥α}\formedesurface{ɖɯ˧ ʈʂæ˧˥}\newline
\classe{量词}\ton{MHα}\begin{définition}\peng{Classifier for ears (of sweet corn).}\end{définition}
\begin{définition}\pcmn{量词.玉米(一棒)}\end{définition}
\begin{définition}\pfra{Classificateur des épis de maïs (mûrs).}\end{définition}
\begin{exemple}\pnru{qʰɑ˧dze˧ | ɖɯ˧-ʈʂæ˧˥}\hspace{5pt}\peng{an ear of sweet corn}\hspace{5pt}\pcmn{一棒玉米}\hspace{5pt}\pfra{un épi de maïs}\end{exemple}
\begin{exemple}\pnru{qʰɑ˧dze˧ | ɖɯ˧-ʈʂæ˧ ɖʐɤ˥}\hspace{5pt}\peng{to pick an ear of sweet corn}\hspace{5pt}\pcmn{掰一棒玉米}\hspace{5pt}\pfra{arracher un épi de maïs, récolter un épi de maïs}\end{exemple}
\end{entrée}

\begin{entrée}
{ʈʂæ˩do\#˥}{}{ⓔʈʂæ˩do\#˥}\formedesurface{ʈʂæ˩do˥}\newline
\classe{名词}\ton{LM+\#H}
\paradigme{\pcmn{:} \p{}}
\begin{définition}\peng{Container in which butter-tea is mixed; also: butter churn.}\end{définition}
\begin{définition}\pcmn{打酥油茶的罐、酥油茶搅拌器,黄油搅乳器}\end{définition}
\begin{définition}\pfra{Récipient dans lequel on bat le thé au beurre (tube-baratte en bois); aussi: grande baratte pour baratter le beurre.}\end{définition}
\end{entrée}

\begin{entrée}
{ʈʂæ˧mo\#˥}{}{ⓔʈʂæ˧mo\#˥}\formedesurface{ʈʂæ˧mo˧}\newline
\classe{名词}\ton{\#H}\begin{définition}\peng{A poisonous mushroom.}\end{définition}
\begin{définition}\pcmn{一种有毒的菌子}\end{définition}
\begin{définition}\pfra{Un champignon vénéneux.}\end{définition}
\begin{exemple}\pnru{ʈʂæ˧mo˧-kʰi˧tɕʰɯ˩-mo˩ / kʰi˧tɕʰɯ˩-mo˩}\hspace{5pt}\peng{same meaning}\hspace{5pt}\pcmn{同上}\hspace{5pt}\pfra{même sens}\end{exemple}
\end{entrée}

\begin{entrée}
{ʈʂæ˧ʈʂɯ˧}{}{ⓔʈʂæ˧ʈʂɯ˧}\formedesurface{ʈʂæ˧ʈʂɯ˧}\newline
\classe{助词}\ton{M}\begin{définition}\peng{Truthfully, accurately, really.}\end{définition}
\begin{définition}\pcmn{确切、真的}\end{définition}
\begin{définition}\pfra{Véritablement, vraiment, pour de vrai.}\end{définition}
\end{entrée}

\begin{entrée}
{ʈʂæ˧wɤ˩}{}{ⓔʈʂæ˧wɤ˩}\formedesurface{ʈʂæ˧wɤ˩}\newline
\classe{名词}\ton{L\#}
\paradigme{\pcmn{:} \p{}}
\begin{définition}\peng{Servant.}\end{définition}
\begin{définition}\pcmn{仆人,佣人}\end{définition}
\begin{définition}\pfra{Serviteur.}\end{définition}
\end{entrée}

\begin{entrée}
{ʈʂe˥}{₁}{ⓔʈʂe˥ⓗ1}\formedesurface{ʈʂe˧}\newline
\classe{名词}\ton{\#H}
1\begin{définition}\peng{Earth.}\end{définition}
\begin{définition}\pcmn{土壤}\end{définition}
\begin{définition}\pfra{Terre.}\end{définition}
\begin{exemple}\pnru{ʈʂe˧pv̩˩}\hspace{5pt}\peng{dry earth}\hspace{5pt}\pcmn{干土}\hspace{5pt}\pfra{terre sèche}\end{exemple}
\begin{exemple}\pnru{ʈʂe˧ sɯ˧∼sɯ˥}\hspace{5pt}\peng{‘raw earth': immature soil, earth that has not been prepared for agriculture by adding manure, etc}\hspace{5pt}\pcmn{‘生土’:没有经过加工(加肥料等等)的土,还不适合种农作物}\hspace{5pt}\pfra{‘terre crue': terre qui n'a pas été préparée pour l'agriculture par l'ajout de fumier, etc}\end{exemple}
\end{entrée}

\begin{entrée}
{ʈʂe˥}{₂}{ⓔʈʂe˥ⓗ2}\formedesurface{ʈʂe˧}\newline
\classe{名词}\ton{\#H}
2
\paradigme{\pcmn{:} \p{}}
\begin{définition}\peng{Needle.}\end{définition}
\begin{définition}\pcmn{针(汉语借词)}\end{définition}
\begin{définition}\pfra{Aiguille.}\end{définition}
\end{entrée}

\begin{entrée}
{ʈʂe˩α}{}{ⓔʈʂe˩α}\formedesurface{ʈʂe˩˥}\newline
\classe{动词}\ton{Lα}\begin{définition}\peng{To sting, to pierce (e.g. a thorn).}\end{définition}
\begin{définition}\pcmn{刺痛}\end{définition}
\begin{définition}\pfra{Percer, transpercer (une écharde, un piquant de plante…).}\end{définition}
\begin{exemple}\pnru{le˧-ʈʂe˩-ze˩}\hspace{5pt}\peng{|fg{accomp} \_ |fg{pfv}}\hspace{5pt}\pcmn{|fg{accomp} \_ |fg{pfv}}\hspace{5pt}\pfra{|fg{accomp} \_ |fg{pfv}}\end{exemple}
\begin{exemple}\pnru{tɕʰi˧-ɳɯ˧ ʈʂe˩-ze˩}\hspace{5pt}\peng{to be pierced by a thorn, to catch a thorn}\hspace{5pt}\pcmn{被刺所刺痛}\hspace{5pt}\pfra{être piqué par une épine, se prendre une épine}\end{exemple}
\begin{exemple}\pnru{tso˧∼tso˧ ʈʂe˥}\hspace{5pt}\peng{to pierce something}\hspace{5pt}\pcmn{刺到一个东西}\hspace{5pt}\pfra{percer quelque chose}\end{exemple}
\end{entrée}

\begin{entrée}
{ʈʂe˧dɑ˥}{}{ⓔʈʂe˧dɑ˥}\formedesurface{ʈʂe˧dɑ˥}\newline
\classe{名词}\ton{H\#}
\paradigme{\pcmn{:} \p{}}
\begin{définition}\peng{Partition.}\end{définition}
\begin{définition}\pcmn{隔板}\end{définition}
\begin{définition}\pfra{Cloison.}\end{définition}
\end{entrée}

\begin{entrée}
{ʈʂe˧gi˥\$}{}{ⓔʈʂe˧gi˥\$}\formedesurface{ʈʂe˧gi˥}\newline
\classe{助词}\ton{H\$}\begin{définition}\peng{In-between, in the middle of.}\end{définition}
\begin{définition}\pcmn{中间、之间、间}\end{définition}
\begin{définition}\pfra{Entre, au milieu de.}\end{définition}
\begin{exemple}\pnru{ə˧-sɯ˩kv̩˩-ʈʂe˩gi˩}\hspace{5pt}\peng{(in the space) between us, in the space that separates us}\hspace{5pt}\pcmn{在咱们之间(的空间)}\hspace{5pt}\pfra{entre nous, dans l'espace qui nous sépare}\end{exemple}
\end{entrée}

\begin{entrée}
{ʈʂe˩kʰɯ˩}{}{ⓔʈʂe˩kʰɯ˩}\formedesurface{ʈʂe˩kʰɯ˩˥}\newline
\classe{名词}\ton{L}
\paradigme{\pcmn{:} \p{}}
\begin{définition}\peng{Seam.}\end{définition}
\begin{définition}\pcmn{缝}\end{définition}
\begin{définition}\pfra{Couture (d'un vêtement).}\end{définition}
\end{entrée}

\begin{entrée}
{ʈʂe˩ʂwæ˧˥}{}{ⓔʈʂe˩ʂwæ˧˥}\formedesurface{ʈʂe˩ʂwæ˧˥}\newline
\classe{名词}\ton{LM+MH\#}\begin{définition}\peng{Grit, gravel.}\end{définition}
\begin{définition}\pcmn{砾石}\end{définition}
\begin{définition}\pfra{Gravier, sable grossier.}\end{définition}
\end{entrée}

\begin{entrée}
{ʈʂɤ˧α}{}{ⓔʈʂɤ˧α}\newline
\classe{动词}
\sens{1}
\begin{définition}\peng{To count; to calculate.}\end{définition}
\begin{définition}\pcmn{数、算}\end{définition}
\begin{définition}\pfra{Compter; calculer.}\end{définition}
\begin{exemple}\pnru{ʈʂɤ˧∼ʈʂɤ˩}\hspace{5pt}\peng{|fg{red}}\hspace{5pt}\pcmn{重叠:算一算}\hspace{5pt}\pfra{|fg{red}}\end{exemple}
\begin{exemple}\pnru{ɖɯ˧-ʈʂɤ˥∼ʈʂɤ˩-ɻ̍˩}\hspace{5pt}\peng{to do some counting, to take a count}\hspace{5pt}\pcmn{算一下}\hspace{5pt}\pfra{faire quelques calculs}\end{exemple}
\begin{exemple}\pnru{tso˧∼tso˧ ʈʂɤ˩}\hspace{5pt}\peng{to count things}\hspace{5pt}\pcmn{数东西}\hspace{5pt}\pfra{compter des choses}\end{exemple}
\begin{exemple}\pnru{hĩ˧ ʈʂɤ˩}\hspace{5pt}\peng{to count people}\hspace{5pt}\pcmn{数人}\hspace{5pt}\pfra{compter les gens}\end{exemple}
\begin{exemple}\pnru{bo˩ ʈʂɤ˧}\hspace{5pt}\peng{to count pigs}\hspace{5pt}\pcmn{数猪}\hspace{5pt}\pfra{compter les porcs}\end{exemple}
\begin{exemple}\pnru{le˧-ʈʂɤ˧-ze˧}\hspace{5pt}\peng{|fg{accomp} \_ |fg{pfv}}\hspace{5pt}\pcmn{数了}\hspace{5pt}\pfra{|fg{accomp} \_ |fg{pfv}}\end{exemple}\sens{2}
\begin{définition}\peng{To do divination, to do fortune-telling.}\end{définition}
\begin{définition}\pcmn{算命}\end{définition}
\begin{définition}\pfra{Dire la bonne aventure, pratiquer la divination.}\end{définition}
\begin{exemple}\pnru{le˧-ʈʂɤ˥∼ʈʂɤ˩}\hspace{5pt}\peng{to do divination, to do fortune-telling}\hspace{5pt}\pcmn{算命}\hspace{5pt}\pfra{dire la bonne aventure, pratiquer la divination}\end{exemple}
\begin{exemple}\pnru{ɖɯ˧-ʈʂɤ˥∼ʈʂɤ˩-ɻ̍˩}\hspace{5pt}\peng{to do some fortune-telling}\hspace{5pt}\pcmn{算一下命}\hspace{5pt}\pfra{dire la bonne aventure}\end{exemple}
\begin{exemple}\pnru{ɲi˧ŋwɤ˩hɑ̃˩tʰɑ˩ | ɖɯ˧-ɭɯ˧ | ʈʂɤ˧-bi˧!}\hspace{5pt}\peng{(We are) going to look for an auspicious day (for an event such as a wedding or the building of a house)}\hspace{5pt}\pcmn{要掐算一下日子}\hspace{5pt}\pfra{On va chercher un jour propice! (pour un événement tel qu'un mariage ou la construction d'une maison)}\end{exemple}
\begin{exemple}\pnru{kɯ˧ ʈʂɤ˧, | hɑ̃˧ ʈʂɤ˧}\hspace{5pt}\peng{to look for an auspicious day for an important event; literally: “to count stars, to count days"}\hspace{5pt}\pcmn{掐算一下。直译:“算星星,算日子”。}\hspace{5pt}\pfra{chercher un jour propice pour un événement, tel que le début de la construction d'une maison; littéralement: «compter les étoiles, compter les jours»}\end{exemple}\sens{3}
\begin{définition}\peng{To count as.}\end{définition}
\begin{définition}\pcmn{算是,当作}\end{définition}
\begin{définition}\pfra{Compter comme, être, avoir fonction de, avoir rôle de.}\end{définition}
\begin{exemple}\pnru{hĩ˧ ɖɯ˧-v̩˧ ʈʂɤ˧-ze˧!}\hspace{5pt}\peng{(She/he) now counts as a (grown-up) person! / (She/he) can now be considered an adult! (A comment typically made when a child reaches adulthood, at age 13.)}\hspace{5pt}\pcmn{变成大人了!(十三岁成年礼时常用的一句话)}\hspace{5pt}\pfra{(Elle/il) compte maintenant comme une grande personne! / C'est un(e) adulte, maintenant! (Ce qu'on dit d'un enfant qui atteint l'âge adulte: 13 ans.)}\end{exemple}
\begin{exemple}\pnru{dʑɤ˩ ʈʂɤ˧}\hspace{5pt}\peng{to be good, to count as good, to be considered as good}\hspace{5pt}\pcmn{算是很好的}\hspace{5pt}\pfra{être très bien}\end{exemple}
\begin{exemple}\pnru{ʈʂʰɯ˧ | õ˧-bv̩˥-õ˩ | dʑɤ˩ʈʂɤ˧ (+ | ʐwæ˩˥)}\hspace{5pt}\peng{He thinks highly of himself! / He is proud of himself / conceited!}\hspace{5pt}\pcmn{他觉得自己很了不起!}\hspace{5pt}\pfra{Il a une haute idée de lui-même! / Il est orgueilleux!}\end{exemple}
\begin{exemple}\pnru{hɤ˩ ʈʂɤ˩˥}\hspace{5pt}\peng{to count as remarkable, to be considered as remarkable}\hspace{5pt}\pcmn{算很了不起的,算很能干的}\hspace{5pt}\pfra{être habile / admirable / remarquable, être considéré comme habile, compter comme (quelqu'un d')habile}\end{exemple}
\begin{exemple}\pnru{ɖwæ˧˥ | hɤ˩ ʈʂɤ˩˥}\hspace{5pt}\peng{same meaning}\hspace{5pt}\pcmn{同上}\hspace{5pt}\pfra{même sens}\end{exemple}
\begin{exemple}\pnru{ʈʂʰɯ˧ | gi˧zɯ˧ ʈʂɤ˧-tso˧-ɲi˥.}\hspace{5pt}\peng{He has the status of little brother! (Comment made to emphasize someone's role in the family.)}\hspace{5pt}\pcmn{他是做弟弟的!(强调该人的社会角色)}\hspace{5pt}\pfra{C'est le petit frère / il a le statut de petit frère! (Commentaire qui rappelle le statut familial de la personne concernée.)}\end{exemple}
\begin{exemple}\pnru{ʈʂʰɯ˧ | gi˧zɯ˧ ʈʂɤ˧-ɲi˥!}\hspace{5pt}\peng{He has the status of little brother! (Comment made to emphasize someone's role in the family.)}\hspace{5pt}\pcmn{他是做弟弟的!(强调该人的社会角色)}\hspace{5pt}\pfra{C'est le petit frère / il a le statut de petit frère! (Commentaire qui rappelle le statut familial de la personne concernée.)}\end{exemple}
\begin{exemple}\pnru{ʈʂʰɯ˧ | bɤ˧ ʈʂɤ˧-tso˧-ɲi˥!}\hspace{5pt}\peng{He/she is Pumi! (Comment made to emphasize this aspect of someone's identity.)}\hspace{5pt}\pcmn{他是普米族!(强调该人的民族)}\hspace{5pt}\pfra{Il/elle est pumi! (Commentaire qui rappelle un élément de l'identité de la personne concernée.)}\end{exemple}
\begin{exemple}\pnru{ʈʂʰɯ˧ | nɑ˩ ʈʂɤ˧-tso˧-ɲi˥!}\hspace{5pt}\peng{He/she is Na! (Comment made to emphasize this aspect of someone's identity.)}\hspace{5pt}\pcmn{他是摩梭人!(强调该人的民族身份)}\hspace{5pt}\pfra{Il/elle est na! (Commentaire qui rappelle un élément de l'identité de la personne concernée)}\end{exemple}
\begin{exemple}\pnru{ʈʂʰɯ˧ | æ˧mv̩˩ ʈʂɤ˩-ɲi˩!}\hspace{5pt}\peng{She has the status of elder sister! / He has the status of elder brother! (Comment made to emphasize someone's role in the family.)}\hspace{5pt}\pcmn{她是做姐姐的! / 他是做哥哥的!(强调该人的社会角色)}\hspace{5pt}\pfra{C'est elle la grande soeur! / C'est lui le grand frère! (Commentaire qui rappelle le statut familial de la personne concernée.)}\end{exemple}
\begin{exemple}\pnru{ʈʂʰɯ˧ | gi˧zɯ˧-go˩mi˩ ʈʂɤ˩-ɲi˩!}\hspace{5pt}\peng{They are brothers and sisters!}\hspace{5pt}\pcmn{他们是(兄)弟(姐)妹!}\hspace{5pt}\pfra{Ils sont frère et sœur!}\end{exemple}
\begin{exemple}\pnru{ʈʂʰɯ˧ | æ˧mv̩˧-go˧mi˥ | ʈʂɤ˧-tso˧ mɤ˧-ɲi˥! | mɤ˧-ʈʂɤ˧!}\hspace{5pt}\peng{They can't be considered as brothers and sisters! / They are not actually brothers and sisters!}\hspace{5pt}\pcmn{他们不算是兄弟姐妹!}\hspace{5pt}\pfra{Ils ne sont pas frères et sœurs! / Ils n'ont pas cette relation-là!}\end{exemple}
\begin{exemple}\pnru{ʐwæ˩ ʈʂɤ˩}\hspace{5pt}\peng{remarkable, great, exceptional, outstanding}\hspace{5pt}\pcmn{了不起}\hspace{5pt}\pfra{remarquable, extraordinaire, exceptionnel}\end{exemple}
\begin{exemple}\pnru{ʈʂʰɯ˧ | ə˧tso˧ ʐwæ˩ ʈʂɤ˩-tso˩ dʑo˩?}\hspace{5pt}\peng{What's so remarkable about her/him?}\hspace{5pt}\pcmn{他有什么了不起的?}\hspace{5pt}\pfra{Qu'est-ce qu'elle/il a de si remarquable? / Quelles qualités exceptionnelles a-t-il/elle (que je doive prendre son conseil/suivre son avis)?}\end{exemple}
\begin{exemple}\pnru{ʈʂʰɯ˧ | ʐwæ˩ ʈʂɤ˩˥ | ʐwæ˩˥!}\hspace{5pt}\peng{It's really an outstanding person!}\hspace{5pt}\pcmn{他非常了不起!}\hspace{5pt}\pfra{C'est quelqu'un de vraiment remarquable/extraordinaire!}\end{exemple}
\end{entrée}

\begin{entrée}
{ʈʂo˩}{}{ⓔʈʂo˩}\formedesurface{ɖɯ˧ ʈʂo˩}\newline
\classe{量词}\ton{L}\begin{définition}\peng{Classifier for meals.}\end{définition}
\begin{définition}\pcmn{量词:饭(一顿)}\end{définition}
\begin{définition}\pfra{Classificateur des repas.}\end{définition}
\begin{exemple}\pnru{ɖɯ˧-ʈʂo˩ tʰi˩-pæ˩ |}\hspace{5pt}\peng{to serve a meal, to set a meal}\hspace{5pt}\pcmn{摆饭,摆饭桌}\hspace{5pt}\pfra{servir un repas}\end{exemple}
\begin{exemple}\pnru{ʐo˩˥, | njɤ˧ ɖɯ˧-ʈʂo˩ pæ˩-bi˩!}\hspace{5pt}\peng{For lunch, I will serve a (whole) meal! / I'm taking charge of lunch!}\hspace{5pt}\pcmn{我来管午饭这一顿!}\hspace{5pt}\pfra{au déjeuner, je vais (vous) servir (tout le) repas!}\end{exemple}
\begin{exemple}\pnru{hɑ˧ ɖɯ˧-ʈʂo˩}\hspace{5pt}\peng{a meal}\hspace{5pt}\pcmn{一顿饭}\hspace{5pt}\pfra{un repas}\end{exemple}
\end{entrée}

\begin{entrée}
{ʈʂo˩bo˩}{}{ⓔʈʂo˩bo˩}\formedesurface{ʈʂo˩bo˩˥}\newline
\classe{名词}\ton{L}
\paradigme{\pcmn{:} \p{}}
\begin{définition}\peng{Earthen wall.}\end{définition}
\begin{définition}\pcmn{土墙}\end{définition}
\begin{définition}\pfra{Mur en terre.}\end{définition}
\begin{exemple}\pnru{ʈʂo˩bo˩ ti˥}\hspace{5pt}\peng{to build a wall of earth, by pounding the earth}\hspace{5pt}\pcmn{垒墙}\hspace{5pt}\pfra{construire un mur en terre (en comprimant la terre)}\end{exemple}
\end{entrée}

\begin{entrée}
{ʈʂo˧kʰɯ˩}{}{ⓔʈʂo˧kʰɯ˩}\formedesurface{ʈʂo˧kʰɯ˩}\newline
\classe{名词}\ton{L\#}\begin{définition}\peng{Ritual performed on the occasion of the death of a family member.}\end{définition}
\begin{définition}\pcmn{忠克:亲人去世时举行的仪式}\end{définition}
\begin{définition}\pfra{Rituel pour la mort d'une personne de sa famille.}\end{définition}
\end{entrée}

\begin{entrée}
{ʈʂo˧ɭɯ\#˥}{}{ⓔʈʂo˧ɭɯ\#˥}\formedesurface{ʈʂo˧ɭɯ˧}\newline
\classe{名词}\ton{\#H}
\paradigme{\pcmn{:} \p{}}
\begin{définition}\peng{Hand-operated grindstone.}\end{définition}
\begin{définition}\pcmn{手推磨}\end{définition}
\begin{définition}\pfra{Moulin à main.}\end{définition}
\begin{exemple}\pnru{ʈʂo˧ɭɯ˧-nv̩˥mi˩}\hspace{5pt}\peng{the axis of the grindstone (literally: its hear)}\hspace{5pt}\pcmn{手推磨的轴心}\hspace{5pt}\pfra{l'axe du moulin (littéralement: son cœur)}\end{exemple}
\end{entrée}

\begin{entrée}
{ʈʂo˧ɭɯ˧ʈʂo˧˥}{}{ⓔʈʂo˧ɭɯ˧ʈʂo˧˥}\formedesurface{ʈʂo˧ɭɯ˧ʈʂo˧˥}\newline
\classe{名词}\ton{MH\#}
\paradigme{\pcmn{:} \p{}}
\begin{définition}\peng{A water insect.}\end{définition}
\begin{définition}\pcmn{一种水虫}\end{définition}
\begin{définition}\pfra{Insecte aquatique.}\end{définition}
\end{entrée}

\begin{entrée}
{ʈʂo˩mv̩˩}{}{ⓔʈʂo˩mv̩˩}\formedesurface{ʈʂo˩mv̩˩˥}\newline
\classe{名词}\ton{L}\begin{définition}\peng{Fine sand.}\end{définition}
\begin{définition}\pcmn{沙子}\end{définition}
\begin{définition}\pfra{Sable fin.}\end{définition}
\end{entrée}

\begin{entrée}
{ʈʂo˧ʂɯ\#˥}{}{ⓔʈʂo˧ʂɯ\#˥}\formedesurface{ʈʂo˧ʂɯ˧}\newline
\classe{名词}\ton{\#H}\begin{définition}\peng{Name of a village: Zhongshi.}\end{définition}
\begin{définition}\pcmn{忠实(永宁的一个村落)}\end{définition}
\begin{définition}\pfra{Village de Zhongshi.}\end{définition}
\begin{exemple}\pnru{ɖæ˩ʂɯ\#˥, | ʈʂo˧ʂɯ\#˥, | bɤ˩tɕʰɯ˩˥, | dɑ˧pʰo˥, | bɤ˧dzi˩, | dze˧bo˧}\hspace{5pt}\peng{Six villages of the plain of Yongning that lie relatively close to the Lake.}\hspace{5pt}\pcmn{永宁摩梭地理概念中,距离泸沽湖比较近的六个村落:扎实、忠实、八旗、达坡、八珠、者波。}\hspace{5pt}\pfra{Six villages de la plaine de Yongning qui sont relativement proches du Lac.}\end{exemple}
\end{entrée}

\begin{entrée}
{ʈʂo˧tsɯ˥}{}{ⓔʈʂo˧tsɯ˥}\formedesurface{ʈʂo˧tsɯ˥}\newline
\classe{名词}\ton{H\#}
\paradigme{\pcmn{:} \p{}}
\begin{définition}\peng{Table.}\end{définition}
\begin{définition}\pcmn{桌子(汉语借词)}\end{définition}
\begin{définition}\pfra{Table.}\end{définition}
\end{entrée}

\begin{entrée}
{ʈʂɻ̍˥}{}{ⓔʈʂɻ̍˥}\formedesurface{ʈʂɻ̍˧}\newline
\classe{动词}\ton{H}\begin{définition}\peng{To cough.}\end{définition}
\begin{définition}\pcmn{咳嗽}\end{définition}
\begin{définition}\pfra{Tousser.}\end{définition}
\begin{exemple}\pnru{ʈʂʰɯ˧ | tʰi˧-ʈʂɻ̍˥-dʑo˩}\hspace{5pt}\peng{(S)he is coughing.}\hspace{5pt}\pcmn{他在咳嗽}\hspace{5pt}\pfra{il tousse}\end{exemple}
\end{entrée}

\begin{entrée}
{ʈʂɯ˧}{}{ⓔʈʂɯ˧}\formedesurface{ʈʂɯ˧}\newline
\classe{名词}\ton{M}
\paradigme{\pcmn{:} \p{}}
\begin{définition}\peng{Claws.}\end{définition}
\begin{définition}\pcmn{爪子}\end{définition}
\begin{définition}\pfra{Griffes (d'un animal); serres (d'un oiseau).}\end{définition}
\end{entrée}

\begin{entrée}
{ʈʂɯ˧˥}{}{ⓔʈʂɯ˧˥}\formedesurface{ʈʂɯ˧˥}\newline
\classe{动词}\ton{MH}\begin{définition}\peng{To sift.}\end{définition}
\begin{définition}\pcmn{筛}\end{définition}
\begin{définition}\pfra{Tamiser.}\end{définition}
\begin{exemple}\pnru{le˧-ʈʂɯ˧-ze˥}\hspace{5pt}\peng{|fg{accomp} \_ |fg{pfv}}\hspace{5pt}\pcmn{|fg{accomp} \_ |fg{pfv}}\hspace{5pt}\pfra{|fg{accomp} \_ |fg{pfv}}\end{exemple}
\begin{exemple}\pnru{ɖɯ˧-ʈʂɯ˧-ɻ̍˥}\hspace{5pt}\peng{|fg{delimitative} \_ |fg{inceptive}}\hspace{5pt}\pcmn{|fg{delimitative} \_ |fg{inceptive}}\hspace{5pt}\pfra{|fg{délimitatif} \_ |fg{inchoatif}}\end{exemple}
\end{entrée}

\begin{entrée}
{ʈʂɯ˧dzi˩}{}{ⓔʈʂɯ˧dzi˩}\formedesurface{ʈʂɯ˧dzi˩}\newline
\classe{名词}\ton{L\#}
\paradigme{\pcmn{:} \p{}}
\begin{définition}\peng{Lacquer tree, varnish tree.}\end{définition}
\begin{définition}\pcmn{漆树}\end{définition}
\begin{définition}\pfra{Arbre à vernis.}\end{définition}
\end{entrée}

\begin{entrée}
{ʈʂɯ˧fv̩\#˥}{}{ⓔʈʂɯ˧fv̩\#˥}\formedesurface{ʈʂɯ˧fv̩˧}\newline
\classe{名词}\ton{\#H}\begin{définition}\peng{Local government.}\end{définition}
\begin{définition}\pcmn{(土)知府,如:永宁知府(汉语借词)}\end{définition}
\begin{définition}\pfra{Gouvernement local, pouvoir local.}\end{définition}
\begin{exemple}\pnru{no˧ | ɬi˧di˩-ʈʂɯ˩fv̩˩-ni˩-zo˩!}\hspace{5pt}\peng{You think you're the government, hey?! (A criticism to people who keep telling others how they should behave, as if they lorded it over everyone else.)}\hspace{5pt}\pcmn{你像永宁土知府! / 你是永宁土知府吧!(批评独断的人、一手包办的人)}\hspace{5pt}\pfra{Tu te prends pour le seigneur! (critique qu'on s'adressait aux gens qui se mêlaient de dicter leur conduite aux autres, comme s'ils étaient les maîtres des lieux)}\end{exemple}
\begin{exemple}\pnru{no˧ | ʈʂɯ˧fv̩˧-mi˧-ni˧˥ | -zo˩!}\hspace{5pt}\peng{You are the Princess of Yongning, hey?! (Criticism addressed to an overbearing woman)}\hspace{5pt}\pcmn{你好像是永宁大公主! / 你好像是永宁知府女主人!(批评一个独断的女人)}\hspace{5pt}\pfra{Tu joues les princesses! (littéralement ‘les femmes du pouvoir’) Critique adressée à une femme qui prend de grands airs.}\end{exemple}
\end{entrée}

\begin{entrée}
{ʈʂɯ˧mɤ˩}{}{ⓔʈʂɯ˧mɤ˩}\formedesurface{ʈʂɯ˧mɤ˩}\newline
\classe{名词}\ton{L\#}\begin{définition}\peng{Sesame.}\end{définition}
\begin{définition}\pcmn{芝麻(汉语借词)}\end{définition}
\begin{définition}\pfra{Sésame.}\end{définition}
\begin{exemple}\pnru{ʈʂɯ˧mɤ˩, | ɬi˧di˩ | mɤ˧-tʰv̩˧-ɲi˥!}\hspace{5pt}\peng{Sesame does not grow in Yongning! / Sesame isn't grown in Yongning!}\hspace{5pt}\pcmn{永宁不产芝麻!}\hspace{5pt}\pfra{Le sésame ne pousse pas à Yongning!}\end{exemple}
\end{entrée}

\begin{entrée}
{ʈʂv̩˩}{}{ⓔʈʂv̩˩}\formedesurface{ʈʂv̩˩˥}\newline
\classe{形容词}\ton{L}\begin{définition}\peng{Peaceful, soft (astrological sign).}\end{définition}
\begin{définition}\pcmn{平和的(生肖)}\end{définition}
\begin{définition}\pfra{Paisible, aimable, pacifique, doux (signe astrologique); s'emploie au sujet des signes astrologiques: certains sont considérés comme ‘paisibles', comme le Boeuf, le Lapin et la Chèvre, ce qui rend les personnes nées cette année-là appropriées pour certains rites/certaines tâches (ex.: lors du rite de passage à l'âge adulte), et au contraire non appropriées pour d'autres.}\end{définition}
\begin{exemple}\pnru{kʰv̩˧ ʈʂv̩˧˥}\hspace{5pt}\peng{a peaceful, soft astrological sign, such as the Ox, the Rabbit and the Goat}\hspace{5pt}\pcmn{平和的生肖,如牛、兔、羊}\hspace{5pt}\pfra{signe pacifique, calme, non belliqueux}\end{exemple}
\end{entrée}

\begin{entrée}
{ʈʂv̩˩˥}{}{ⓔʈʂv̩˩˥}\formedesurface{ʈʂv̩˩˥}\newline
\classe{名词}\ton{LH}\begin{définition}\peng{Sweat (monosyllable).}\end{définition}
\begin{définition}\pcmn{汗(单音节)}\end{définition}
\begin{définition}\pfra{Sueur (monosyllabe).}\end{définition}
\begin{exemple}\pnru{ʈʂv̩˧ bv̩˧nv̩˩}\hspace{5pt}\peng{smelling of sweat, reeking of sweat}\hspace{5pt}\pcmn{有汗(臭)的味道}\hspace{5pt}\pfra{qui sent la sueur, malodorant}\end{exemple}
\end{entrée}

\begin{entrée}
{ʈʂv̩˩α}{₁}{ⓔʈʂv̩˩αⓗ1}\formedesurface{ʈʂv̩˩˥}\newline
\classe{动词}\ton{Lα}
1\begin{définition}\peng{To contaminate, to infect.}\end{définition}
\begin{définition}\pcmn{传染}\end{définition}
\begin{définition}\pfra{Contaminer, infecter.}\end{définition}
\begin{exemple}\pnru{hĩ˧ ʈʂv̩˥-ho˩}\hspace{5pt}\peng{(the disease) is going to contaminate someone / is going to contaminate people}\hspace{5pt}\pcmn{(病毒)会传染人的}\hspace{5pt}\pfra{(la maladie) va contaminer quelqu'un}\end{exemple}
\begin{exemple}\pnru{ʈʂv̩˧∼ʈʂv̩˥}\hspace{5pt}\peng{|fg{red}}\hspace{5pt}\pcmn{重叠}\hspace{5pt}\pfra{|fg{red}}\end{exemple}
\begin{exemple}\pnru{ʈʂv̩˧∼ʈʂv̩˥-ɻ̍˩ ho˩}\hspace{5pt}\peng{(the disease) is going to contaminate (people)}\hspace{5pt}\pcmn{(病毒)会传染的。}\hspace{5pt}\pfra{(la maladie) va contaminer (des gens)}\end{exemple}
\begin{exemple}\pnru{njɤ˧-ɳɯ˧ | no˧ ʈʂv̩˧-ʝi˥!}\hspace{5pt}\peng{(Be careful,) I may contaminate you / pass on my cold to you!}\hspace{5pt}\pcmn{(要小心:)我会传染你的!}\hspace{5pt}\pfra{(Attention,) je vais te contaminer/te passer (mon rhume)!}\end{exemple}
\end{entrée}

\begin{entrée}
{ʈʂv̩˩α}{₂}{ⓔʈʂv̩˩αⓗ2}\formedesurface{ʈʂv̩˩˥}\newline
\classe{动词}\ton{Lα}
2\begin{définition}\peng{To light (a candle).}\end{définition}
\begin{définition}\pcmn{点(蜡烛……)}\end{définition}
\begin{définition}\pfra{Allumer (une bougie).}\end{définition}
\end{entrée}

\begin{entrée}
{ʈʂv̩˧di˧˥}{}{ⓔʈʂv̩˧di˧˥}\formedesurface{ʈʂv̩˧di˧˥}\newline
\classe{名词}\ton{MH\#}\begin{définition}\peng{Name of a village outside the plain of Lijiang, in the vicinity of the Lake, close to \stylefv{/lɑ}˧tʰɑ˧-di˧˥/.}\end{définition}
\begin{définition}\pcmn{村落名}\end{définition}
\begin{définition}\pfra{Village na hors de la plaine de Yongning, vers le Lac, non loin de \stylefv{/lɑ}˧tʰɑ˧-di˧˥/.}\end{définition}
\end{entrée}

\begin{entrée}
{ʈʂv̩˩dʑɯ˥}{}{ⓔʈʂv̩˩dʑɯ˥}\formedesurface{ʈʂv̩˩dʑɯ˥}\newline
\classe{名词}\ton{LH}\begin{définition}\peng{Sweat.}\end{définition}
\begin{définition}\pcmn{汗}\end{définition}
\begin{définition}\pfra{Sueur.}\end{définition}
\end{entrée}

\begin{entrée}
{ʈʂv̩˧pɤ˩}{}{ⓔʈʂv̩˧pɤ˩}\formedesurface{ʈʂv̩˧pɤ˩}\newline
\classe{名词}\ton{L\#}
\paradigme{\pcmn{:} \p{}}
\begin{définition}\peng{Cutting-board; vessel or cutting board for meat.}\end{définition}
\begin{définition}\pcmn{菜板、俎}\end{définition}
\begin{définition}\pfra{Planche à découper.}\end{définition}
\end{entrée}

\begin{entrée}
{ʈʂv̩˧tɕɯ˥}{}{ⓔʈʂv̩˧tɕɯ˥}\formedesurface{ʈʂv̩˧tɕɯ˥}\newline
\classe{名词}\ton{H\#}\begin{définition}\peng{Spittle, phlegm, sputum.}\end{définition}
\begin{définition}\pcmn{痰}\end{définition}
\begin{définition}\pfra{Crachat.}\end{définition}
\end{entrée}

\begin{entrée}
{ʈʂwɑ˧∼ʈʂwɑ˧-nɑ˧∼nɑ\#˥}{}{ⓔʈʂwɑ˧∼ʈʂwɑ˧-nɑ˧∼nɑ\#˥}\formedesurface{ʈʂwɑ˧ʈʂwɑ˧nɑ˧nɑ˧}\newline
\classe{形容词}\ton{\#H}\begin{définition}\peng{Mixed; diverse, heterogeneous; messy.}\end{définition}
\begin{définition}\pcmn{杂、混杂}\end{définition}
\begin{définition}\pfra{Divers, varié, désordonné.}\end{définition}
\begin{exemple}\pnru{ʈʂwɑ˧∼ʈʂwɑ˧-nɑ˧∼nɑ˧-hĩ˥}\hspace{5pt}\peng{|fg{rel}/nmlz}\hspace{5pt}\pcmn{混杂的}\hspace{5pt}\pfra{|fg{rel}/nmlz}\end{exemple}
\begin{exemple}\pnru{ʈʂwɑ˧∼ʈʂwɑ˧-nɑ˧∼nɑ˧-ɻ̍˥}\hspace{5pt}\peng{messy}\hspace{5pt}\pcmn{混杂}\hspace{5pt}\pfra{désordonné}\end{exemple}
\end{entrée}

\begin{entrée}
{ʈʂwæ˥α}{}{ⓔʈʂwæ˥α}\formedesurface{ɖɯ˧ ʈʂwæ˥}\newline
\classe{量词}\ton{Hα}\begin{définition}\peng{Classifier for journeys.}\end{définition}
\begin{définition}\pcmn{量词:征途、路程、路途、征程,趟}\end{définition}
\begin{définition}\pfra{Classificateur des trajets.}\end{définition}
\begin{exemple}\pnru{ɖɯ˧-ɲi˥ | ɖɯ˧-ʈʂwæ˧ bi˧}\hspace{5pt}\peng{to go once a day, to go one time each day}\hspace{5pt}\pcmn{一天去一趟}\hspace{5pt}\pfra{(y) aller une fois par jour}\end{exemple}
\end{entrée}

\begin{entrée}
{ʈʂwæ˧˥}{₁}{ⓔʈʂwæ˧˥ⓗ1}\newline
\classe{动词}
1
\sens{1}
\begin{définition}\peng{To set up, to install.}\end{définition}
\begin{définition}\pcmn{安装}\end{définition}
\begin{définition}\pfra{Installer.}\end{définition}
\begin{exemple}\pnru{tjɤ˧hwɑ˧ ʈʂwæ˥}\hspace{5pt}\peng{to set up the telephone, to put up a telephone line (in a house that did not have it before)}\hspace{5pt}\pcmn{安装电话(座机)}\hspace{5pt}\pfra{installer le téléphone (dans une demeure qui n'y était pas reliée précédemment)}\end{exemple}
\begin{exemple}\pnru{le˧-ʈʂwæ˧˥ le˧-tse˧-ze˧!}\hspace{5pt}\peng{It's installed! / It is now well installed!}\hspace{5pt}\pcmn{装好了!}\hspace{5pt}\pfra{C'est bien installé!}\end{exemple}\sens{2}
\begin{définition}\peng{To repair, to cure (a tooth).}\end{définition}
\begin{définition}\pcmn{补(牙)、修好(坏牙)}\end{définition}
\begin{définition}\pfra{Soigner, réparer (une dent).}\end{définition}
\begin{exemple}\pnru{hi˧ ʈʂwæ˩}\hspace{5pt}\peng{to cure a tooth}\hspace{5pt}\pcmn{补牙、修好坏牙}\hspace{5pt}\pfra{soigner une dent; littéralement ‘remettre une dent’}\end{exemple}
\begin{exemple}\pnru{hi˧ | le˧-ʈʂwæ˧-ze˥!}\hspace{5pt}\peng{The tooth is cured!}\hspace{5pt}\pcmn{牙补好了!}\hspace{5pt}\pfra{La dent est soignée! / La dent est réparée!}\end{exemple}\sens{3}
\begin{définition}\peng{To tie (a string), to make a knot to tie two pieces of thread together.}\end{définition}
\begin{définition}\pcmn{结线}\end{définition}
\begin{définition}\pfra{Nouer (des fils).}\end{définition}
\begin{exemple}\pnru{kʰɯ˩ ʈʂwæ˩˥}\hspace{5pt}\peng{to tie pieces of thread together}\hspace{5pt}\pcmn{结线}\hspace{5pt}\pfra{attacher des brins de fil ensemble, nouer des fils (par ex. lorsqu'on prépare le métier à tisser)}\end{exemple}
\end{entrée}

\begin{entrée}
{ʈʂwæ˧˥}{₂}{ⓔʈʂwæ˧˥ⓗ2}\formedesurface{ʈʂwæ˧˥}\newline
\classe{动词}\ton{MH}
2\begin{définition}\peng{To savour, to enjoy, to relish.}\end{définition}
\begin{définition}\pcmn{欣赏、品尝(饮食、音乐……)}\end{définition}
\begin{définition}\pfra{Savourer, déguster, siroter (nourriture ou boisson).}\end{définition}
\begin{exemple}\pnru{no˧ | li˩ ʈʂwæ˧-ɻ̍˥! |}\hspace{5pt}\peng{Please enjoy a cup of tea! (A polite invitation)}\hspace{5pt}\pcmn{请您品一点茶!(礼貌说法)}\hspace{5pt}\pfra{Veuillez prendre un peu de thé! (Invitation polie)}\end{exemple}
\begin{exemple}\pnru{ʐɯ˧ F | ʈʂwæ˧˥! | li˩˥ F | ʈʂwæ˧˥! hɑ˧ F | ʈʂwæ˧˥!}\hspace{5pt}\peng{Wine is something to be savoured! Tea is something to be savoured! Food is something to be savoured! (An explanation about the use of the verb.)}\hspace{5pt}\pcmn{酒,是可以品尝的!茶,是可以品尝的!饭,是可以品尝的!(关于‘品尝’这个动词的说明)}\hspace{5pt}\pfra{L'alcool, ça se savoure; le thé, ça se savoure! (Explication au sujet des emplois du verbe)}\end{exemple}
\begin{exemple}\pnru{hɑ˧ ʈʂwæ˩}\hspace{5pt}\peng{to savour food}\hspace{5pt}\pcmn{品尝食物}\hspace{5pt}\pfra{savourer de la nourriture}\end{exemple}
\begin{exemple}\pnru{li˩ ʈʂwæ˧˥}\hspace{5pt}\peng{to savour tea}\hspace{5pt}\pcmn{品茶}\hspace{5pt}\pfra{savourer du thé}\end{exemple}
\begin{exemple}\pnru{ʐɯ˧ ʈʂwæ˧˥}\hspace{5pt}\peng{to savour wine}\hspace{5pt}\pcmn{品酒}\hspace{5pt}\pfra{savourer de l'alcool}\end{exemple}
\begin{exemple}\pnru{ə˩kʰɯ˩ ʈʂwæ˥}\hspace{5pt}\peng{to savour turnip (an ironic but fully acceptable statement)}\hspace{5pt}\pcmn{尝尝圆根(玩笑话,因为圆根没有什么滋味)}\hspace{5pt}\pfra{savourer des feuilles de navet (formulation ironique mais tout à fait acceptable)}\end{exemple}
\end{entrée}

\begin{entrée}
{ʈʂwæ˩ho˧ɻ̍˧}{}{ⓔʈʂwæ˩ho˧ɻ̍˧}\formedesurface{ʈʂwæ˧ho˧ɻ̍˧}\newline
\classe{名词}\ton{LM}\begin{définition}\peng{Drill.}\end{définition}
\begin{définition}\pcmn{钻子}\end{définition}
\begin{définition}\pfra{Perceuse.}\end{définition}
\end{entrée}

\begin{entrée}
{ʈʂwæ˧tʰo˩}{}{ⓔʈʂwæ˧tʰo˩}\formedesurface{ʈʂwæ˧tʰo˩}\newline
\classe{名词}\ton{L\#}\begin{définition}\peng{Brick.}\end{définition}
\begin{définition}\pcmn{砖头(汉语借词)}\end{définition}
\begin{définition}\pfra{Brique.}\end{définition}
\end{entrée}

\begin{entrée}
{ʈʂwæ˧∼ʈʂwæ˧}{}{ⓔʈʂwæ˧∼ʈʂwæ˧}\formedesurface{ʈʂwæ˧ʈʂwæ˧}\newline
\classe{动词}\ton{M}\begin{définition}\peng{To mix.}\end{définition}
\begin{définition}\pcmn{搅拌、使混合}\end{définition}
\begin{définition}\pfra{Mélanger.}\end{définition}
\end{entrée}

\begin{entrée}
{ʈʂwæ˩∼ʈʂwæ˧˥}{}{ⓔʈʂwæ˩∼ʈʂwæ˧˥}\formedesurface{ʈʂwæ˩ʈʂwæ˧˥}\newline
\classe{动词}\ton{MH}\begin{définition}\peng{To hold by the hand, to hold each other's hands.}\end{définition}
\begin{définition}\pcmn{手拉手}\end{définition}
\begin{définition}\pfra{Se tenir par la main, se tenir la main.}\end{définition}
\begin{exemple}\pnru{le˧-ʈʂwæ˧∼ʈʂwæ˧-ze˧!}\hspace{5pt}\peng{|fg{accomp} |fg{red} |fg{pfv}}\hspace{5pt}\pcmn{|fg{accomp} |fg{red} |fg{pfv}}\hspace{5pt}\pfra{|fg{accomp} |fg{red} |fg{pfv}}\end{exemple}
\begin{exemple}\pnru{ʈʂwæ˩∼ʈʂwæ˧-ɻ̍˥}\hspace{5pt}\peng{|fg{red} |fg{inceptive}}\hspace{5pt}\pcmn{|fg{red} |fg{inceptive}}\hspace{5pt}\pfra{|fg{red} |fg{inchoatif}; même sens: se tenir par la main}\end{exemple}
\end{entrée}

\begin{entrée}
{ʈʂwɤ˧α}{}{ⓔʈʂwɤ˧α}\formedesurface{ʈʂwɤ˧}\newline
\classe{动词}\ton{Mα}\begin{définition}\peng{To scratch (with claws, e.g. of tiger).}\end{définition}
\begin{définition}\pcmn{抓(用爪子抓)}\end{définition}
\begin{définition}\pfra{Griffer (ex.: un tigre griffe).}\end{définition}
\begin{exemple}\pnru{tso˧∼tso˧ ʈʂwɤ˩}\hspace{5pt}\peng{to scratch objects}\hspace{5pt}\pcmn{抓东西}\hspace{5pt}\pfra{griffer des objets}\end{exemple}
\end{entrée}

\begin{entrée}
{ʈʂwɤ˧α}{}{ⓔʈʂwɤ˧α}\formedesurface{ɖɯ˧ ʈʂwɤ˧}\newline
\classe{量词}\ton{Mα}\begin{définition}\peng{A handful (with one single hand).}\end{définition}
\begin{définition}\pcmn{量词:捧}\end{définition}
\begin{définition}\pfra{Classificateur des poignées: ce que l'on peut prendre dans une seule main.}\end{définition}
\end{entrée}

\begin{entrée}
{ʈʂwɤ˥∼ʈʂwɤ˩}{}{ⓔʈʂwɤ˥∼ʈʂwɤ˩}\formedesurface{ʈʂwɤ˧ʈʂwɤ˩}\newline
\classe{动词}\ton{H.L}\begin{définition}\peng{To touch.}\end{définition}
\begin{définition}\pcmn{触碰}\end{définition}
\begin{définition}\pfra{Toucher.}\end{définition}
\begin{exemple}\pnru{ə˧tso˧ mɤ˧-ɲi˩ | ʈʂwɤ˧∼ʈʂwɤ˩!}\hspace{5pt}\peng{You really touch all and everything, don't you! (Mildly scolding a baby that crawls around on a table and grabs every object in turn)}\hspace{5pt}\pcmn{你什么都碰,是吗!(小孩爬在桌子上,试着拿每个东西)}\hspace{5pt}\pfra{(tu) touches vraiment à tout! (doux reproche adressé à un bébé qui se promène sur une table et s'empare de tout ce qui s'y trouve)}\end{exemple}
\end{entrée}

\begin{entrée}
{ʈʂʰɑ˧lɑ˧}{}{ⓔʈʂʰɑ˧lɑ˧}\formedesurface{ʈʂʰɑ˧lɑ˧}\newline
\classe{动词}\ton{M}\begin{définition}\peng{To discuss, to have a talk, to chat.}\end{définition}
\begin{définition}\pcmn{商量、交谈、谈天、聊天}\end{définition}
\begin{définition}\pfra{Discuter, bavarder.}\end{définition}
\begin{exemple}\pnru{hĩ˧-qɑ˩ ʈʂʰɑ˩lɑ˩}\hspace{5pt}\peng{to have a chat with someone}\hspace{5pt}\pcmn{跟人聊天}\hspace{5pt}\pfra{bavarder avec quelqu'un}\end{exemple}
\begin{exemple}\pnru{ɖɯ˧-kʰwɤ˧ ʈʂʰɑ˧lɑ˥}\hspace{5pt}\peng{to have a small chat}\hspace{5pt}\pcmn{聊聊天}\hspace{5pt}\pfra{bavarder un peu, avoir une causerie avec}\end{exemple}
\begin{exemple}\pnru{njɤ˧ | no˧-qɑ˧ ʈʂʰɑ˧lɑ˥}\hspace{5pt}\peng{I tell you, I narrate to you}\hspace{5pt}\pcmn{我给你讲、我跟你聊聊天}\hspace{5pt}\pfra{je te raconte, je te dis}\end{exemple}
\end{entrée}

\begin{entrée}
{ʈʂʰɑ˧lɑ˧-mv̩˧lɑ˩}{}{ⓔʈʂʰɑ˧lɑ˧-mv̩˧lɑ˩}\formedesurface{ʈʂʰɑ˧lɑ˧mv̩˧lɑ˩}\newline
\classe{动词}\ton{-L\#}\begin{définition}\peng{To discuss, to have a talk, to chat.}\end{définition}
\begin{définition}\pcmn{商量、交谈、谈天、聊天}\end{définition}
\begin{définition}\pfra{Discuter, bavarder.}\end{définition}
\begin{exemple}\pnru{ʈʂʰɑ˧lɑ˧-mv̩˧lɑ˩-ɻ̍˩}\hspace{5pt}\peng{to have a chat}\hspace{5pt}\pcmn{聊聊天}\hspace{5pt}\pfra{bavarder un peu, avoir une causerie}\end{exemple}
\end{entrée}

\begin{entrée}
{ʈʂʰɑ˧nɑ˥}{}{ⓔʈʂʰɑ˧nɑ˥}\formedesurface{ʈʂʰɑ˧nɑ˥}\newline
\classe{名词}\ton{H\#}\begin{définition}\peng{The name of a sacred spring, at the foot of a cliff, on the mountain \stylefv{/qv̩}˧ɻ̍\#˥/.}\end{définition}
\begin{définition}\pcmn{一眼山泉的名字}\end{définition}
\begin{définition}\pfra{Nom d'une source sacrée, située au pied d'une falaise, sur la montagne \stylefv{/qv̩}˧ɻ̍\#˥/; on disait que son eau sortait du ventre de la montagne. Le récit DumbChildren raconte comment son eau était utilisée comme remède de fertilité.}\end{définition}
\begin{exemple}\pnru{qv̩˧ɻ̍˧-ʈʂʰɑ˧nɑ˥\#}\hspace{5pt}\peng{the full name of the mountain}\hspace{5pt}\pcmn{山的全名,包括水泉名}\hspace{5pt}\pfra{nom complet de la montagne}\end{exemple}
\begin{exemple}\pnru{kɤ˧mv̩˧˥, | æ˧ʂæ˧, | ŋwɤ˧hɑ̃˩, | ʂwæ˧gv̩\#˥, | nɑ˩tsʰi˩˥ | -tɕʰɤ˧pɤ˧mi\#˥, | qv̩˧ɻ̍˧-ʈʂʰɑ˧nɑ˥ |}\hspace{5pt}\peng{The six mountains of Yongning that carry a name and have a definite symbolic value. The other mountains do not have comparable symbolic value, and fewer people use specific names for them.}\hspace{5pt}\pcmn{永宁地区有固定名字的六座山:格姆,安山,瓦哈,双古,纳慈巧吧咪,古尔川纳。}\hspace{5pt}\pfra{Les six montagnes de Yongning qui portent un nom. Les autres sommets du voisinage n'ont pas une valeur symbolique comparable, et ne portent pas de nom communément utilisé.}\end{exemple}
\end{entrée}

\begin{entrée}
{ʈʂʰæ˥}{}{ⓔʈʂʰæ˥}\formedesurface{ʈʂʰæ˧}\newline
\classe{动词}\ton{H}\begin{définition}\peng{To wash (clothes, oneself…).}\end{définition}
\begin{définition}\pcmn{洗(洗衣服,洗澡……)}\end{définition}
\begin{définition}\pfra{Laver (les habits, la vaisselle…), rincer (le riz…).}\end{définition}
\begin{exemple}\pnru{dʑi˧hṽ̩˧ ʈʂʰæ˧}\hspace{5pt}\peng{to wash clothes}\hspace{5pt}\pcmn{洗衣服}\hspace{5pt}\pfra{laver des vêtements}\end{exemple}
\begin{exemple}\pnru{bɑ˩lɑ˩ ʈʂʰæ˩˥}\hspace{5pt}\peng{to wash shirts}\hspace{5pt}\pcmn{洗上衣}\hspace{5pt}\pfra{laver des chemises}\end{exemple}
\begin{exemple}\pnru{ɬi˧qʰwɤ˩ ʈʂʰæ˩}\hspace{5pt}\peng{to wash trousers}\hspace{5pt}\pcmn{洗裤子}\hspace{5pt}\pfra{laver des pantalons}\end{exemple}
\begin{exemple}\pnru{gv̩˧mi˧ ʈʂʰæ˧}\hspace{5pt}\peng{to wash oneself, to take a bath/shower}\hspace{5pt}\pcmn{洗澡}\hspace{5pt}\pfra{se laver, prendre un bain/une douche}\end{exemple}
\begin{exemple}\pnru{gv̩˧mi˧ ʈʂʰæ˧∼ʈʂʰæ˧}\hspace{5pt}\peng{to wash oneself a bit, to do a quick clean-up}\hspace{5pt}\pcmn{洗一下身体}\hspace{5pt}\pfra{se laver un coup}\end{exemple}
\begin{exemple}\pnru{hɑ˧ ʈʂʰæ˧}\hspace{5pt}\peng{to rinse cereals (before cooking)}\hspace{5pt}\pcmn{淘洗粮食}\hspace{5pt}\pfra{rincer une céréale (avant de la cuire)}\end{exemple}
\begin{exemple}\pnru{ɕi˧ʈʂʰwæ˧ ʈʂʰæ˧(-ze˩)}\hspace{5pt}\peng{to rinse rice (before cooking)}\hspace{5pt}\pcmn{淘米}\hspace{5pt}\pfra{rincer le riz (avant de le cuire)}\end{exemple}
\end{entrée}

\begin{entrée}
{ʈʂʰæ˧˥}{₁}{ⓔʈʂʰæ˧˥ⓗ1}\formedesurface{ʈʂʰæ˧˥}\newline
\classe{名词}\ton{MH}
1
\paradigme{\pcmn{:} \p{}}
\begin{définition}\peng{Deer, red deer, |\stylefi{Cervus elaphus kansuensis}.}\end{définition}
\begin{définition}\pcmn{马鹿}\end{définition}
\begin{définition}\pfra{Cerf, |\stylefi{Cervus elaphus kansuensis}.}\end{définition}
\end{entrée}

\begin{entrée}
{ʈʂʰæ˧˥}{₂}{ⓔʈʂʰæ˧˥ⓗ2}\formedesurface{ɖɯ˧ ʈʂʰæ˧˥}\newline
\classe{量词}\ton{MH}
2\begin{définition}\peng{Classifier for generations.}\end{définition}
\begin{définition}\pcmn{量词:代、世、辈、世代}\end{définition}
\begin{définition}\pfra{Classificateur des générations.}\end{définition}
\end{entrée}

\begin{entrée}
{ʈʂʰæ˧ɣɯ\#˥}{}{ⓔʈʂʰæ˧ɣɯ\#˥}\formedesurface{ʈʂʰæ˧ɣɯ˧}\newline
\classe{名词}\ton{\#H}\begin{définition}\peng{Medicine.}\end{définition}
\begin{définition}\pcmn{药}\end{définition}
\begin{définition}\pfra{Médicament.}\end{définition}
\begin{exemple}\pnru{ʈʂʰæ˧ɣɯ˧ ʈʰɯ˧˥}\hspace{5pt}\peng{to take a medicine; literally “to drink a medicine"}\hspace{5pt}\pcmn{吃药(直译:“喝药”)}\hspace{5pt}\pfra{Prendre un médicament. Littéralement: ‘boire un médicament’ (collocation différente du chinois 吃药 ‘manger un médicament’)}\end{exemple}
\begin{exemple}\pnru{ʈʂʰæ˧ɣɯ˧ lɑ˩}\hspace{5pt}\peng{to spread pesticides (in an orchard, a vegetable garden or a field)}\hspace{5pt}\pcmn{打农药}\hspace{5pt}\pfra{répandre des pesticides, traiter (un verger, un potager, un champ…)}\end{exemple}
\end{entrée}

\begin{entrée}
{ʈʂʰæ˧ɣɯ˧-ki˩-hĩ˩-hĩ˩}{}{ⓔʈʂʰæ˧ɣɯ˧-ki˩-hĩ˩-hĩ˩}\formedesurface{ʈʂʰæ˧ɣɯ˧ki˩hĩ˩hĩ˩}\newline
\classe{名词}\ton{-L}
\paradigme{\pcmn{:} \p{}}
\begin{définition}\peng{Doctor; literally: “person who gives medicines".}\end{définition}
\begin{définition}\pcmn{医生}\end{définition}
\begin{définition}\pfra{Médecin, docteur; littéralement: ‘personne qui donne des médicaments’.}\end{définition}
\end{entrée}

\begin{entrée}
{ʈʂʰæ˧mi˥\$}{}{ⓔʈʂʰæ˧mi˥\$}\formedesurface{ʈʂʰæ˧mi˥}\newline
\classe{名词}\ton{H\$}
\paradigme{\pcmn{:} \p{}}
\begin{définition}\peng{Doe, hind.}\end{définition}
\begin{définition}\pcmn{母马鹿}\end{définition}
\begin{définition}\pfra{Biche.}\end{définition}
\end{entrée}

\begin{entrée}
{ʈʂʰæ˧nɑ˥}{}{ⓔʈʂʰæ˧nɑ˥}\formedesurface{ʈʂʰæ˧nɑ˥}\newline
\classe{名词}\ton{H\#}
\paradigme{\pcmn{:} \p{}}
\begin{définition}\peng{Black stag: a legendary species, which only spirits are able to hunt down.}\end{définition}
\begin{définition}\pcmn{黑鹿}\end{définition}
\begin{définition}\pfra{Cerf noir: espèce légendaire, que seuls les esprits sont à même de chasser et abattre.}\end{définition}
\end{entrée}

\begin{entrée}
{ʈʂʰæ˧pʰv̩\#˥}{}{ⓔʈʂʰæ˧pʰv̩\#˥}\formedesurface{ʈʂʰæ˧pʰv̩˧}\newline
\classe{名词}\ton{\#H}
\paradigme{\pcmn{:} \p{}}
\begin{définition}\peng{Male deer.}\end{définition}
\begin{définition}\pcmn{公马鹿}\end{définition}
\begin{définition}\pfra{Cerf (mâle).}\end{définition}
\end{entrée}

\begin{entrée}
{ʈʂʰæ˧qʰv̩˥\$}{}{ⓔʈʂʰæ˧qʰv̩˥\$}\formedesurface{ʈʂʰæ˧qʰv̩˥}\newline
\classe{名词}\ton{H\$}
\paradigme{\pcmn{:} \p{}}
\begin{définition}\peng{Antlers; pilose antler (of young stags).}\end{définition}
\begin{définition}\pcmn{鹿角,鹿茸}\end{définition}
\begin{définition}\pfra{Bois d'un cerf (même mot pour les bois d'un jeune cerf, utilisés comme aphrodisiaque en médecine traditionnelle).}\end{définition}
\end{entrée}

\begin{entrée}
{ʈʂʰæ˧sɯ˩}{}{ⓔʈʂʰæ˧sɯ˩}\formedesurface{ʈʂʰæ˧sɯ˩}\newline
\classe{助词}\ton{L\#}\begin{définition}\peng{Unfortunate, a pity.}\end{définition}
\begin{définition}\pcmn{太可惜了!}\end{définition}
\begin{définition}\pfra{Dommage, regrettable.}\end{définition}
\begin{exemple}\pnru{ʈʂʰæ˧sɯ˩!}\hspace{5pt}\peng{It's a pity!}\hspace{5pt}\pcmn{可惜!}\hspace{5pt}\pfra{Dommage!}\end{exemple}
\end{entrée}

\begin{entrée}
{ʈʂʰæ˧∼ʈʂʰæ˧}{}{ⓔʈʂʰæ˧∼ʈʂʰæ˧}\formedesurface{ʈʂʰæ˧ʈʂʰæ˧}\newline
\classe{形容词}\ton{M}\begin{définition}\peng{Solid, of good quality.}\end{définition}
\begin{définition}\pcmn{结实、质量好,(东西)耐用,(人)可靠}\end{définition}
\begin{définition}\pfra{Solide, de bonne qualité, résistant (vêtement, outil, objet…).}\end{définition}
\end{entrée}

\begin{entrée}
{ʈʂʰæ˧zo\#˥}{}{ⓔʈʂʰæ˧zo\#˥}\formedesurface{ʈʂʰæ˧zo˧}\newline
\classe{名词}\ton{\#H}
\paradigme{\pcmn{:} \p{}}
\begin{définition}\peng{Baby deer.}\end{définition}
\begin{définition}\pcmn{小鹿}\end{définition}
\begin{définition}\pfra{Faon.}\end{définition}
\end{entrée}

\begin{entrée}
{ʈʂʰe˧β}{}{ⓔʈʂʰe˧β}\formedesurface{ʈʂʰe˧}\newline
\classe{动词}\ton{Mβ}\begin{définition}\peng{To stretch (one's hand…).}\end{définition}
\begin{définition}\pcmn{伸(伸手)}\end{définition}
\begin{définition}\pfra{Tendre, étendre (la main).}\end{définition}
\begin{exemple}\pnru{le˧-ʈʂʰe˧-ze˧}\hspace{5pt}\peng{|fg{accomp} \_ |fg{pfv}}\hspace{5pt}\pcmn{|fg{accomp} \_ |fg{pfv}}\hspace{5pt}\pfra{|fg{accomp} \_ |fg{pfv}}\end{exemple}
\begin{exemple}\pnru{mv̩˩tɕo˧ ʈʂʰe˧}\hspace{5pt}\peng{to stretch down}\hspace{5pt}\pcmn{向下伸展}\hspace{5pt}\pfra{étendre vers le bas}\end{exemple}
\begin{exemple}\pnru{lo˩qʰwɤ˧ | ə˩pʰo˩ ʈʂʰe˩˥}\hspace{5pt}\peng{to strech one's hand outside (e.g. out the window)}\hspace{5pt}\pcmn{手伸到外边}\hspace{5pt}\pfra{étendre la main à l'extérieur (par une fenêtre)}\end{exemple}
\begin{exemple}\pnru{tso˧∼tso˧ ʈʂʰe˧}\hspace{5pt}\peng{to extend something, to stick out something (e.g. to extend a cane out the window of a car)}\hspace{5pt}\pcmn{伸出一个东西,如:从车窗里伸出一个棍子}\hspace{5pt}\pfra{étendre quelque chose: par exemple, faire sortir un bâton par une fenêtre}\end{exemple}
\end{entrée}

\begin{entrée}
{ʈʂʰe˩ko˧}{}{ⓔʈʂʰe˩ko˧}\formedesurface{ʈʂʰe˩ko˥}\newline
\classe{动词}\ton{LM}\begin{définition}\peng{To succeed.}\end{définition}
\begin{définition}\pcmn{成功(汉语借词)}\end{définition}
\begin{définition}\pfra{Réussir.}\end{définition}
\end{entrée}

\begin{entrée}
{ʈʂʰe˧∼ʈʂʰe˧}{}{ⓔʈʂʰe˧∼ʈʂʰe˧}\formedesurface{ɖɯ˧ ʈʂʰe˧ʈʂʰe˧}\newline
\classe{量词}\ton{M}
\étymologie{
ʈʂʰe˧b
}\begin{définition}\peng{Classifiers for walls, i.e. the width of a room: for instance, a cupboard can be described as extending over an entire wall, i.e. occupying the entire width of a room.}\end{définition}
\begin{définition}\pcmn{量词:一面(墙)}\end{définition}
\begin{définition}\pfra{Classificateur pour les murs, et donc pour la largeur de toute une pièce: un buffet/placard occupe toute la largeur d'une pièce, par exemple.}\end{définition}
\end{entrée}

\begin{entrée}
{ʈʂʰɤ˩α}{}{ⓔʈʂʰɤ˩α}\formedesurface{ʈʂʰɤ˩˥}\newline
\classe{动词}\ton{Lα}\begin{définition}\peng{To share (several people share something among themselves; someone shares out something).}\end{définition}
\begin{définition}\pcmn{分}\end{définition}
\begin{définition}\pfra{Répartir, diviser.}\end{définition}
\begin{exemple}\pnru{ɖɯ˧-v̩˧ ɖɯ˧-kʰwɤ˥ | le˧-ʈʂʰɤ˧∼ʈʂʰɤ˥}\hspace{5pt}\peng{to share: one piece for each person}\hspace{5pt}\pcmn{平分}\hspace{5pt}\pfra{répartir un morceau par personne}\end{exemple}
\end{entrée}

\begin{entrée}
{ʈʂʰɤ˩ho˥}{}{ⓔʈʂʰɤ˩ho˥}\formedesurface{ʈʂʰɤ˩ho˥}\newline
\classe{名词}\ton{LH}
\paradigme{\pcmn{:} \p{}}
\begin{définition}\peng{Kettle.}\end{définition}
\begin{définition}\pcmn{水壶(汉语借词:茶壶)}\end{définition}
\begin{définition}\pfra{Bouilloire.}\end{définition}
\end{entrée}

\begin{entrée}
{ʈʂʰɤ˩kɤ˧}{}{ⓔʈʂʰɤ˩kɤ˧}\formedesurface{ʈʂʰɤ˩kɤ˥}\newline
\classe{名词}\ton{LM}
\paradigme{\pcmn{:} \p{}}
\begin{définition}\peng{Goblet.}\end{définition}
\begin{définition}\pcmn{缸子,杯子}\end{définition}
\begin{définition}\pfra{Gobelet (avec anse); en métal ou poterie.}\end{définition}
\end{entrée}

\begin{entrée}
{ʈʂʰɤ˩qo˧}{}{ⓔʈʂʰɤ˩qo˧}\formedesurface{ʈʂʰɤ˩qo˥}\newline
\classe{名词}\ton{LM}
\paradigme{\pcmn{:} \p{}}
\begin{définition}\peng{Attention, interest, care.}\end{définition}
\begin{définition}\pcmn{关注、关心}\end{définition}
\begin{définition}\pfra{Attention, intérêt.}\end{définition}
\begin{exemple}\pnru{ʈʂʰɤ˩qo˧ kʰɯ˧˥}\hspace{5pt}\peng{to pay attention to, to care for}\hspace{5pt}\pcmn{关心、关注}\hspace{5pt}\pfra{se soucier de, prêter attention à}\end{exemple}
\begin{exemple}\pnru{ʈʂʰɤ˩qo˧ | ɖwæ˧˥ | tʰi˧-kʰɯ˧˥}\hspace{5pt}\peng{to pay great attention to, to care greatly for (e.g. a grandmother paying great attention to a little child's feeding)}\hspace{5pt}\pcmn{很关心、很关注}\hspace{5pt}\pfra{prêter une grande attention à, être très attentif à (ex.: une grand-mère très attentive à l'alimentation d'un petit enfant)}\end{exemple}
\begin{exemple}\pnru{ʈʂʰɤ˩qo˧ | mɤ˧-kʰɯ˧˥}\hspace{5pt}\peng{to pay little attention to, not to care for}\hspace{5pt}\pcmn{不关心、不关注}\hspace{5pt}\pfra{être insensible à, ne pas prêter attention à}\end{exemple}
\end{entrée}

\begin{entrée}
{ʈʂʰɤ˩tɕʰɯ˩}{}{ⓔʈʂʰɤ˩tɕʰɯ˩}\formedesurface{ʈʂʰɤ˩tɕʰɯ˩˥}\newline
\classe{形容词}\ton{L}\begin{définition}\peng{Admirable, with high qualities.}\end{définition}
\begin{définition}\pcmn{利害,值得崇拜}\end{définition}
\begin{définition}\pfra{Très doué, très calé, possédant des qualités admirables (par ex.: personne très savante).}\end{définition}
\end{entrée}

\begin{entrée}
{ʈʂʰɤ˧tsɯ˧}{}{ⓔʈʂʰɤ˧tsɯ˧}\formedesurface{ʈʂʰɤ˧tsɯ˧}\newline
\classe{名词}\ton{M}\begin{définition}\peng{Car, bus.}\end{définition}
\begin{définition}\pcmn{车子(汉语借词)}\end{définition}
\begin{définition}\pfra{Voiture, automobile, car.}\end{définition}
\end{entrée}

\begin{entrée}
{ʈʂʰɤ˩∼ʈʂʰɤ˧˥}{}{ⓔʈʂʰɤ˩∼ʈʂʰɤ˧˥}\formedesurface{ʈʂʰɤ˩ʈʂʰɤ˧˥}\newline
\classe{动词}\ton{MH}\begin{définition}\peng{To feel, to touch, to stroke.}\end{définition}
\begin{définition}\pcmn{抚摸}\end{définition}
\begin{définition}\pfra{Toucher.}\end{définition}
\begin{exemple}\pnru{ʈʂʰɤ˩ʈʂʰɤ˧ mɤ˥-tʰɑ˩!}\hspace{5pt}\peng{One must not touch!}\hspace{5pt}\pcmn{禁止触碰!}\hspace{5pt}\pfra{il ne faut pas toucher!}\end{exemple}
\begin{exemple}\pnru{tʰɑ˧-ʈʂʰɤ˩ʈʂʰɤ˩!}\hspace{5pt}\peng{Do not touch!}\hspace{5pt}\pcmn{别碰!}\hspace{5pt}\pfra{ne touchez pas!}\end{exemple}
\begin{exemple}\pnru{tso˧∼tso˧ ʈʂʰɤ˥ʈʂʰɤ˩}\hspace{5pt}\peng{to touch something}\hspace{5pt}\pcmn{抚摸东西}\hspace{5pt}\pfra{toucher quelque chose}\end{exemple}
\end{entrée}

\begin{entrée}
{ʈʂʰɤ˧zo˥-ʈʂʰɤ˩mv̩˩}{}{ⓔʈʂʰɤ˧zo˥-ʈʂʰɤ˩mv̩˩}\formedesurface{ʈʂʰɤ˧zo˥ʈʂʰɤ˩mv̩˩}\newline
\classe{名词}\ton{H\#-L}\begin{définition}\peng{Love child.}\end{définition}
\begin{définition}\pcmn{私生子:没有名分的孩子、不明来路}\end{définition}
\begin{définition}\pfra{Enfant naturel.}\end{définition}
\begin{exemple}\pnru{ə˧dɑ˥ | ɲi˩-ɲi˥ | mɤ˧-sɯ˥ | ʈʂʰɯ˧-v̩˧, | ʈʂʰɤ˧zo˥-ʈʂʰɤ˩mv̩˩ mv̩˩ʈʂæ˩.}\hspace{5pt}\peng{Someone who does not know who his father is, is called a “love child".}\hspace{5pt}\pcmn{一个人不知道他父亲是谁,就称作“私生子”。}\hspace{5pt}\pfra{Celui qui ne sait pas qui est son père, on l'appelle «enfant naturel».}\end{exemple}
\begin{exemple}\pnru{ə˧ʝi˧-ʂɯ˥ʝi˩, | ʈʂʰɤ˧zo˥-ʈʂʰɤ˩mv̩˩ ʐɤ˩-hĩ˩-lɑ˩ ɲi˩!}\hspace{5pt}\peng{In the past, one used to bring up love children, and that was that! / In the past, one used to bring up love children without making any fuss!}\hspace{5pt}\pcmn{过去,大家会公开把“私生子”养大,不会大惊小怪的!}\hspace{5pt}\pfra{Autrefois, les enfants naturels, on les élevait et voilà tout!/on les élevait tout simplement, sans faire de difficultés!}\end{exemple}
\end{entrée}

\begin{entrée}
{ʈʂʰo˥}{}{ⓔʈʂʰo˥}\formedesurface{ʈʂʰo˧}\newline
\classe{动词}\ton{H}\begin{définition}\peng{To pray (to a god): to recite prayers, to chant prayers.}\end{définition}
\begin{définition}\pcmn{拜(神)}\end{définition}
\begin{définition}\pfra{Prier (une divinité): réciter des prières, psalmodier des prières.}\end{définition}
\begin{exemple}\pnru{ʈʂʰo˧do˩ ʈʂʰo˩}\hspace{5pt}\peng{to pray to the spirit of the home}\hspace{5pt}\pcmn{祭祀祖先}\hspace{5pt}\pfra{prier l'esprit du foyer}\end{exemple}
\begin{exemple}\pnru{hĩ˧-mo˥, | zo˩qo˧ ʂɯ˧, | zo˩qo˧-ɳɯ˧ ʈʂʰo˧-zo˧!}\hspace{5pt}\peng{Deceased members of the family are honoured at the place where they passed away!}\hspace{5pt}\pcmn{要在家人去世地点进行祭拜!}\hspace{5pt}\pfra{Les membres décédés de la famille, c'est à l'endroit où ils sont morts qu'on leur rend hommage!}\end{exemple}
\end{entrée}

\begin{entrée}
{ʈʂʰo˧β}{}{ⓔʈʂʰo˧β}\formedesurface{ʈʂʰo˧}\newline
\classe{动词}\ton{Mβ}\begin{définition}\peng{To read aloud.}\end{définition}
\begin{définition}\pcmn{朗读}\end{définition}
\begin{définition}\pfra{Lire à haute voix.}\end{définition}
\begin{exemple}\pnru{le˧-ʈʂʰo˧-ze˧}\hspace{5pt}\peng{|fg{accomp} \_ |fg{pfv}}\hspace{5pt}\pcmn{朗读了}\hspace{5pt}\pfra{|fg{accomp} \_ |fg{pfv}}\end{exemple}
\begin{exemple}\pnru{le˧-ʈʂʰo˧-le˧-se˩}\hspace{5pt}\peng{(I) have finished reading aloud}\hspace{5pt}\pcmn{朗读完了。}\hspace{5pt}\pfra{(j'ai) fini de lire}\end{exemple}
\begin{exemple}\pnru{tʰæ˧ɻæ˩ ʈʂʰo˩}\hspace{5pt}\peng{to read a book aloud}\hspace{5pt}\pcmn{朗读一本书}\hspace{5pt}\pfra{lire un livre}\end{exemple}
\begin{exemple}\pnru{ʈʂʰo˧∼ʈʂʰo˧}\hspace{5pt}\peng{|fg{red}}\hspace{5pt}\pcmn{重叠}\hspace{5pt}\pfra{|fg{red}}\end{exemple}
\end{entrée}

\begin{entrée}
{ʈʂʰo˧bɤ\#˥}{}{ⓔʈʂʰo˧bɤ\#˥}\formedesurface{ʈʂʰo˧bɤ˧}\newline
\classe{名词}\ton{\#H}
\paradigme{\pcmn{:} \p{}}
\begin{définition}\peng{Masculine clothing worn on special occasions.}\end{définition}
\begin{définition}\pcmn{男上衣}\end{définition}
\begin{définition}\pfra{Vêtement masculin, que les hommes portaient à partir de 13 ans: sorte de veste serrée à la ceinture, qu'on portait sur la chemise lors des grandes occasions: mariage, invitations…}\end{définition}
\end{entrée}

\begin{entrée}
{ʈʂʰo˧bv̩˩}{}{ⓔʈʂʰo˧bv̩˩}\formedesurface{ʈʂʰo˧bv̩˩}\newline
\classe{名词}\ton{L\#}
\paradigme{\pcmn{:} \p{}}
\begin{définition}\peng{Calamus, sweet flag, bitterroot, |\stylefi{Acorus calamus} (a tall wetland plant).}\end{définition}
\begin{définition}\pcmn{菖蒲}\end{définition}
\begin{définition}\pfra{Acore odorant, jonc odorant, |\stylefi{Acorus calamus}: plante herbacée aquatique, pérenne, rhizomateuse.}\end{définition}
\end{entrée}

\begin{entrée}
{ʈʂʰo˧do˩}{}{ⓔʈʂʰo˧do˩}\formedesurface{ʈʂʰo˧do˩}\newline
\classe{名词}\ton{L\#}
\paradigme{\pcmn{:} \p{}}
\begin{définition}\peng{Small eminence next to the hearth, symbolising the ancestors, on top of which some food is offered at the beginning of each meal.}\end{définition}
\begin{définition}\pcmn{火塘上面祖先灵位}\end{définition}
\begin{définition}\pfra{Petite éminence juste à côté du foyer, en contrebas de l'autel où on offre des cadeaux aux ancêtres; c'est sur cette petite éminence qu'on dépose un peu de nourriture au début de chaque repas, en offrande aux ancêtres.}\end{définition}
\end{entrée}

\begin{entrée}
{ʈʂʰo˧lo\#˥}{}{ⓔʈʂʰo˧lo\#˥}\formedesurface{ʈʂʰo˧lo˧}\newline
\classe{名词}\ton{\#H}
\paradigme{\pcmn{:} \p{}}
\begin{définition}\peng{Frying pan (large, with flat bottom).}\end{définition}
\begin{définition}\pcmn{平底大锅(直径大概半米),用来煎洋芋饼等等}\end{définition}
\begin{définition}\pfra{Grande poêle à fond plat, diamètre un peu supérieur à 50 cm, pour frire des aliments (galettes de pomme de terre, fèves).}\end{définition}
\end{entrée}

\begin{entrée}
{ʈʂʰɻ̍}{}{ⓔʈʂʰɻ̍}\formedesurface{ʈʂʰɻ!}\newline
\classe{}\ton{--}\begin{définition}\peng{Hissing noise of water that comes in contact with red-hot metal or incandescent wood: Pssshhh!}\end{définition}
\begin{définition}\pcmn{形声词:噗嗤!(水浇在很热的金属上的声音)}\end{définition}
\begin{définition}\pfra{Bruit de l'eau qui crépite au contact d'un métal rougi ou de bois incandescent: psshhhh!}\end{définition}
\end{entrée}

\begin{entrée}
{ʈʂʰɻ̍˧}{}{ⓔʈʂʰɻ̍˧}\formedesurface{ʈʂʰɻ̍˧}\newline
\classe{名词}\ton{M}
\paradigme{\pcmn{:} \p{}}
\begin{définition}\peng{Ploughshare.}\end{définition}
\begin{définition}\pcmn{铧头,犁铧}\end{définition}
\begin{définition}\pfra{Soc de l'araire.}\end{définition}
\begin{exemple}\pnru{ʈʂʰɻ̍˧ ʈʂʰɯ˧-ɭɯ˧}\hspace{5pt}\peng{|fg{n}+|fg{dem}+|fg{clf}}\hspace{5pt}\pcmn{这把铧头}\hspace{5pt}\pfra{|fg{n}+|fg{dem}+|fg{clf}}\end{exemple}
\end{entrée}

\begin{entrée}
{ʈʂʰɻ̍˧˥}{₁}{ⓔʈʂʰɻ̍˧˥ⓗ1}\formedesurface{ʈʂʰɻ̍˧˥}\newline
\classe{动词}\ton{MH}
1\begin{définition}\peng{To grasp (e.g. a sword hilt).}\end{définition}
\begin{définition}\pcmn{握 (握刀把)}\end{définition}
\begin{définition}\pfra{Empoigner, prendre en main, saisir, tenir fermement (ex.: couteau); serrer, crisper (le poing).}\end{définition}
\begin{exemple}\pnru{sɯ˩tʰi˩˥ | (ɖɯ˧)-nɑ˧ | tʰi˧-ʈʂʰɻ̍˧˥ (+dʑo˩)}\hspace{5pt}\peng{to grasp a knife}\hspace{5pt}\pcmn{手里握刀}\hspace{5pt}\pfra{empoigner un couteau}\end{exemple}
\begin{exemple}\pnru{ʈʂʰɻ̍˧ mɤ˧-bi˧!}\hspace{5pt}\peng{I won't grasp (this knife, …)}\hspace{5pt}\pcmn{我不要拿(刀)!}\hspace{5pt}\pfra{je ne veux pas empoigner/ pas question que j'empoigne (ce couteau,…)}\end{exemple}
\begin{exemple}\pnru{lo˩qʰwɤ˧ ʈʂʰɻ̍˩∼ʈʂʰɻ̍˩ |}\hspace{5pt}\peng{to tighten the fist}\hspace{5pt}\pcmn{攥紧拳头}\hspace{5pt}\pfra{serrer le poing}\end{exemple}
\end{entrée}

\begin{entrée}
{ʈʂʰɻ̍˧˥}{₂}{ⓔʈʂʰɻ̍˧˥ⓗ2}\formedesurface{ʈʂʰɻ̍˧˥}\newline
\classe{名词}\ton{MH}
2
\paradigme{\pcmn{:} \p{}}
\begin{définition}\peng{Lung.}\end{définition}
\begin{définition}\pcmn{肺}\end{définition}
\begin{définition}\pfra{Poumon.}\end{définition}
\end{entrée}

\begin{entrée}
{ʈʂʰɻ̍˧˥α}{}{ⓔʈʂʰɻ̍˧˥α}\formedesurface{ɖɯ˧ ʈʂʰɻ̍˧˥}\newline
\classe{量词}\ton{MHα}\begin{définition}\peng{Classifier for handfuls / balls: loose substance shaped into ball form by compressing it in the hand, for instance a handful of cooked cereals.}\end{définition}
\begin{définition}\pcmn{量词:团,掐。指的是一只手里能拿的量,压成团,如:手里拿煮熟的粮食,压成饭团。}\end{définition}
\begin{définition}\pfra{Classificateur des boules/poignées: la quantité que l'on compacte en la serrant dans une main, par exemple une poignée de céréale cuite qu'on compresse en boule.}\end{définition}
\end{entrée}

\begin{entrée}
{ʈʂʰɯ˥}{₁}{ⓔʈʂʰɯ˥ⓗ1}\formedesurface{ʈʂʰɯ˧}\newline
\classe{代词}\ton{\#H}
1\begin{définition}\peng{This; proximal demonstrative.}\end{définition}
\begin{définition}\pcmn{指示.近指:这}\end{définition}
\begin{définition}\pfra{Démonstratif proximal, qui forme un couple avec le démonstratif distal.}\end{définition}
\begin{exemple}\pnru{ʈʂʰɯ˧ ɲi˥!}\hspace{5pt}\peng{This is it! / That's it! / That's right!}\hspace{5pt}\pcmn{是这个! / 对的!}\hspace{5pt}\pfra{C'est ça!}\end{exemple}
\begin{exemple}\pnru{ʈʂʰɯ˧-v̩\#˥}\hspace{5pt}\peng{this one}\hspace{5pt}\pcmn{这个}\hspace{5pt}\pfra{celui-ci (|fg{dem\_prox}-|fg{clf.individu})}\end{exemple}
\begin{exemple}\pnru{ʈʂʰɯ˧=ɻæ˥}\hspace{5pt}\peng{these things}\hspace{5pt}\pcmn{这些}\hspace{5pt}\pfra{|fg{pl}: ces choses-là}\end{exemple}
\end{entrée}

\begin{entrée}
{ʈʂʰɯ˥}{₂}{ⓔʈʂʰɯ˥ⓗ2}\formedesurface{ʈʂʰɯ˧}\newline
\classe{代词}\ton{\#H}
2\begin{définition}\peng{3rd person singular pronoun.}\end{définition}
\begin{définition}\pcmn{他}\end{définition}
\begin{définition}\pfra{Pronom de troisième personne du singulier.}\end{définition}
\begin{exemple}\pnru{ʈʂʰɯ˧ ɲi˥!}\hspace{5pt}\peng{That's her/him!}\hspace{5pt}\pcmn{是他!}\hspace{5pt}\pfra{C'est elle/lui!}\end{exemple}
\end{entrée}

\begin{entrée}
{ʈʂʰɯ˧}{}{ⓔʈʂʰɯ˧}\formedesurface{ʈʂʰɯ˧}\newline
\classe{后缀}\ton{M}\begin{définition}\peng{Topic marker; grammaticalized from the proximal demonstrative.}\end{définition}
\begin{définition}\pcmn{主题(来自近指指示代词的语法化)}\end{définition}
\begin{définition}\pfra{Topicalisateur; grammaticalisé à partir du démonstratif proximal.}\end{définition}
\end{entrée}

\begin{entrée}
{ʈʂʰɯ˧-gɤ˧}{}{ⓔʈʂʰɯ˧-gɤ˧}\formedesurface{ʈʂʰɯ˧gɤ˧}\newline
\classe{助词}\ton{M}\begin{définition}\peng{Here.}\end{définition}
\begin{définition}\pcmn{这里}\end{définition}
\begin{définition}\pfra{Ici, à cet endroit-ci.}\end{définition}
\end{entrée}

\begin{entrée}
{ʈʂʰɯ˧gi\#˥}{}{ⓔʈʂʰɯ˧gi\#˥}\formedesurface{ʈʂʰɯ˧gi˧}\newline
\classe{助词}\ton{\#H}\begin{définition}\peng{Here.}\end{définition}
\begin{définition}\pcmn{这边}\end{définition}
\begin{définition}\pfra{Ici, à cet endroit-ci.}\end{définition}
\end{entrée}

\begin{entrée}
{ʈʂʰɯ˧ne˧-ʝi˥}{}{ⓔʈʂʰɯ˧ne˧-ʝi˥}\formedesurface{ʈʂʰɯ˧ne˧ʝi˥}\newline
\classe{助词}\ton{H\#}\begin{définition}\peng{Thus, in this way.}\end{définition}
\begin{définition}\pcmn{这样,这么}\end{définition}
\begin{définition}\pfra{Ainsi, de cette façon (adverbe de manière).}\end{définition}
\begin{exemple}\pnru{ʈʂʰɯ˧ne˧-ʝi˥ | le˧-ʐwɤ˩!}\hspace{5pt}\peng{This is how it's said!}\hspace{5pt}\pcmn{是这样讲的!}\hspace{5pt}\pfra{c'est comme ça qu'on dit!}\end{exemple}
\begin{exemple}\pnru{ʈʂʰɯ˧ne˧-ʝi˥ | le˧-pi˥!}\hspace{5pt}\peng{This is how it's said!}\hspace{5pt}\pcmn{是这样说的!}\hspace{5pt}\pfra{c'est comme ça qu'on parle!}\end{exemple}
\begin{exemple}\pnru{ʈʂʰɯ˧ne˧-ʝi˥ | le˧-ʝi˥!}\hspace{5pt}\peng{This is how it's done!}\hspace{5pt}\pcmn{是这样做的!}\hspace{5pt}\pfra{c'est comme ça qu'on fait!}\end{exemple}
\end{entrée}

\begin{entrée}
{ʈʂʰɯ˧qɑ˧}{}{ⓔʈʂʰɯ˧qɑ˧}\formedesurface{ʈʂʰɯ˧qɑ˧}\newline
\classe{助词}\begin{définition}\peng{Together.}\end{définition}
\begin{définition}\pcmn{一起}\end{définition}
\begin{définition}\pfra{Ensemble.}\end{définition}
\end{entrée}

\begin{entrée}
{ʈʂʰɯ˧-qo˧}{}{ⓔʈʂʰɯ˧-qo˧}\formedesurface{ʈʂʰɯ˧qo˧}\newline
\classe{助词}\ton{M}\begin{définition}\peng{Here.}\end{définition}
\begin{définition}\pcmn{这里}\end{définition}
\begin{définition}\pfra{Ici, à cet endroit-ci.}\end{définition}
\end{entrée}

\begin{entrée}
{ʈʂʰɯ˧-sɯ˩-kv̩˩}{}{ⓔʈʂʰɯ˧-sɯ˩-kv̩˩}\formedesurface{ʈʂʰɯ˧sɯ˩kv̩˩}\newline
\classe{代词}\ton{-L}\begin{définition}\peng{3rd person plural.}\end{définition}
\begin{définition}\pcmn{他们}\end{définition}
\begin{définition}\pfra{3e personne du pluriel.}\end{définition}
\begin{exemple}\pnru{ʈʂʰɯ˧-sɯ˩-kv̩˩-bv̩˩ | mɤ˧-sɯ˥!}\hspace{5pt}\peng{(We) don't know about their business! / (We) are not privy to their family affairs!}\hspace{5pt}\pcmn{我们不知道他们的事!}\hspace{5pt}\pfra{On ne connaît pas leurs histoires!}\end{exemple}
\end{entrée}

\begin{entrée}
{ʈʂʰɯ˧ʂo˧}{}{ⓔʈʂʰɯ˧ʂo˧}\formedesurface{ʈʂʰɯ˧ʂo˧}\newline
\classe{助词}\ton{M}\begin{définition}\peng{This morning.}\end{définition}
\begin{définition}\pcmn{今天早上}\end{définition}
\begin{définition}\pfra{Ce matin (à partir du petit matin).}\end{définition}
\end{entrée}

\begin{entrée}
{ʈʂʰɯ˧tɕi˩}{}{ⓔʈʂʰɯ˧tɕi˩}\formedesurface{ʈʂʰɯ˧tɕi˩}\newline
\classe{代词}\ton{L\#}\begin{définition}\peng{Third-person plural pronoun.}\end{définition}
\begin{définition}\pcmn{他们}\end{définition}
\begin{définition}\pfra{Pronom de troisième personne pluriel.}\end{définition}
\end{entrée}

\begin{entrée}
{ʈʂʰɯ˧=zɯ˩}{}{ⓔʈʂʰɯ˧=zɯ˩}\formedesurface{ʈʂʰɯ˧zɯ˩}\newline
\classe{代词}\ton{L\#}\begin{définition}\peng{Dual third-person pronoun: the two of them.}\end{définition}
\begin{définition}\pcmn{他们两个}\end{définition}
\begin{définition}\pfra{Pronom de troisième personne duel: eux deux.}\end{définition}
\end{entrée}

\begin{entrée}
{ʈʂʰv̩˧}{}{ⓔʈʂʰv̩˧}\formedesurface{ʈʂʰv̩˧}\newline
\classe{名词}\ton{M}\begin{définition}\peng{Breakfast.}\end{définition}
\begin{définition}\pcmn{早饭}\end{définition}
\begin{définition}\pfra{Repas du matin/ petit déjeuner.}\end{définition}
\begin{exemple}\pnru{ʈʂʰv̩˧ dzɯ˧(-ze˩)}\hspace{5pt}\peng{to have breakfast, to eat breakfast}\hspace{5pt}\pcmn{吃早饭}\hspace{5pt}\pfra{prendre le petit déjeuner}\end{exemple}
\begin{exemple}\pnru{bæ˧qʰæ˧ ʈʂʰv̩\#˥}\hspace{5pt}\peng{The breakfast shared when coming back from the cremation ceremony. Guests stop at the house of the deceased, where they are offered breakfast before they set home.}\hspace{5pt}\pcmn{丧礼早餐:参加火葬仪式的人留在去世的人家,一起吃一点早饭再回家。}\hspace{5pt}\pfra{le petit déjeuner qu'on prend au retour de la cérémonie de crémation: revenant du lieu où a eu lieu la crémation, les invités, en nombre relativement restreint, font une pause dans la maison du défunt, où on leur offre une collation avant qu'ils ne s'en retournent.}\end{exemple}
\end{entrée}

\begin{entrée}
{ʈʂʰv̩˧˥}{}{ⓔʈʂʰv̩˧˥}\formedesurface{ʈʂʰv̩˧˥}\newline
\classe{动词}\ton{MH}\begin{définition}\peng{To add water, to pour extra water.}\end{définition}
\begin{définition}\pcmn{掺和}\end{définition}
\begin{définition}\pfra{Ajouter de l’eau, verser de l'eau.}\end{définition}
\begin{exemple}\pnru{le˧-ʈʂʰv̩˧-ze˥}\hspace{5pt}\peng{|fg{accomp} \_ |fg{pfv}}\hspace{5pt}\pcmn{|fg{accomp} \_ |fg{pfv}}\hspace{5pt}\pfra{|fg{accomp} \_ |fg{pfv}}\end{exemple}
\begin{exemple}\pnru{dʑɯ˩ ʈʂʰv̩˩˥}\hspace{5pt}\peng{to add water (e.g. in a pot)}\hspace{5pt}\pcmn{加水(如:往锅里添加水)}\hspace{5pt}\pfra{ajouter de l'eau (dans une marmite, …)}\end{exemple}
\end{entrée}

\begin{entrée}
{ʈʂʰv̩˩}{₁}{ⓔʈʂʰv̩˩ⓗ1}\formedesurface{ʈʂʰv̩˩˥}\newline
\classe{动词}\ton{L}
1\begin{définition}\peng{To complete, to finish.}\end{définition}
\begin{définition}\pcmn{完成}\end{définition}
\begin{définition}\pfra{Achever, terminer.}\end{définition}
\begin{exemple}\pnru{le˧-ʈʂʰv̩˩-se˩}\hspace{5pt}\peng{|fg{accomp} \_ |fg{cmpl}}\hspace{5pt}\pcmn{完成了}\hspace{5pt}\pfra{|fg{accomp} \_ |fg{cmpl}}\end{exemple}
\begin{exemple}\pnru{tsʰi˧-ɲi˧-bv̩˧ | lo˧ | le˧-ʈʂʰv̩˩! | le˧-se˩-ze˩!}\hspace{5pt}\peng{Today's work is completed! It's finished!}\hspace{5pt}\pcmn{今天的工作完成了!就算完工了吧!}\hspace{5pt}\pfra{Le travail d'aujourd'hui… on tourne la page! Il est fini!}\end{exemple}
\end{entrée}

\begin{entrée}
{ʈʂʰv̩˩}{₂}{ⓔʈʂʰv̩˩ⓗ2}\formedesurface{ʈʂʰv̩˩˥}\newline
\classe{动词}\ton{L}
2\begin{définition}\peng{To set aside, to set apart, to distinguish.}\end{définition}
\begin{définition}\pcmn{除开}\end{définition}
\begin{définition}\pfra{Mettre à part.}\end{définition}
\begin{exemple}\pnru{gɤ˩-ʈʂʰv̩˧, | mv̩˩-ʈʂʰv̩˧-tsæ˩-ɲi˩}\hspace{5pt}\peng{to set aside, to distinguish, not to mix}\hspace{5pt}\pcmn{不算在里面、不算在一起}\hspace{5pt}\pfra{laisser à part, distinguer, ne pas mettre ensemble, ne pas fourrer dans le même sac}\end{exemple}
\begin{exemple}\pnru{no˧-bv̩˧ | gɤ˩-ʈʂʰv̩˧! | njɤ˧-bv̩˧, | mv̩˩-ʈʂʰv̩˧!}\hspace{5pt}\peng{Your stuff belongs to you; and mine belongs to me!}\hspace{5pt}\pcmn{你的算你的,我的算我的!}\hspace{5pt}\pfra{Ce qui est à toi est à toi; ce qui est à moi est à moi!}\end{exemple}
\end{entrée}

\begin{entrée}
{ʈʂʰv̩˩α}{}{ⓔʈʂʰv̩˩α}\formedesurface{ʈʂʰv̩˩˥}\newline
\classe{动词}\ton{Lα}\begin{définition}\peng{To dye.}\end{définition}
\begin{définition}\pcmn{染}\end{définition}
\begin{définition}\pfra{Teindre.}\end{définition}
\begin{exemple}\pnru{mɤ˧-ʈʂʰv̩˩}\hspace{5pt}\peng{|fg{neg}}\hspace{5pt}\pcmn{|fg{neg}}\hspace{5pt}\pfra{|fg{neg}}\end{exemple}
\begin{exemple}\pnru{ʈʂʰv̩˩ mɤ˩-bi˩˥!}\hspace{5pt}\peng{\_ |fg{neg} |fg{fut\_imm}}\hspace{5pt}\pcmn{\_ |fg{neg} |fg{fut\_imm}}\hspace{5pt}\pfra{\_ |fg{neg} |fg{fut\_imm}}\end{exemple}
\begin{exemple}\pnru{tso˧∼tso˧ ʈʂʰv̩˥}\hspace{5pt}\peng{to dye things}\hspace{5pt}\pcmn{染东西}\hspace{5pt}\pfra{teindre des choses}\end{exemple}
\end{entrée}

\begin{entrée}
{ʈʂʰv̩˧dʑɯ˧}{}{ⓔʈʂʰv̩˧dʑɯ˧}\formedesurface{ʈʂʰv̩˧dʑɯ˧}\newline
\classe{名词}\ton{M}
\paradigme{\pcmn{:} \p{}}
\begin{définition}\peng{Dye, dyestuff.}\end{définition}
\begin{définition}\pcmn{染料}\end{définition}
\begin{définition}\pfra{Teinture.}\end{définition}
\begin{exemple}\pnru{dʑi˧hṽ̩˧-ʈʂʰv̩˧dʑɯ˧}\hspace{5pt}\peng{dye for clothes}\hspace{5pt}\pcmn{衣服染料}\hspace{5pt}\pfra{teinture pour vêtements}\end{exemple}
\begin{exemple}\pnru{ʈʂʰv̩˧dʑɯ˧ | hṽ̩˩-hĩ˩˥}\hspace{5pt}\peng{red dye}\hspace{5pt}\pcmn{红色的染料}\hspace{5pt}\pfra{teinture rouge}\end{exemple}
\end{entrée}

\begin{entrée}
{ʈʂʰv̩˧mi˧}{}{ⓔʈʂʰv̩˧mi˧}\formedesurface{ʈʂʰv̩˧mi˧}\newline
\classe{名词}\ton{M}
\paradigme{\pcmn{:} \p{}}
\begin{définition}\peng{Wife.}\end{définition}
\begin{définition}\pcmn{太太、老婆、媳妇}\end{définition}
\begin{définition}\pfra{Épouse, femme.}\end{définition}
\end{entrée}

\begin{entrée}
{ʈʂʰv̩˧ɻ̍˥\$}{}{ⓔʈʂʰv̩˧ɻ̍˥\$}\formedesurface{ʈʂʰv̩˧ɻ̍˥}\newline
\classe{名词}\ton{H\$}
\paradigme{\pcmn{:} \p{}}
\begin{définition}\peng{Ant.}\end{définition}
\begin{définition}\pcmn{蚂蚁}\end{définition}
\begin{définition}\pfra{Fourmi.}\end{définition}
\begin{exemple}\pnru{ʈʂʰv̩˧ɻ̍˧ | tɕi˩-hĩ˩˥}\hspace{5pt}\peng{small ant}\hspace{5pt}\pcmn{小蚂蚁}\hspace{5pt}\pfra{petite fourmi}\end{exemple}
\end{entrée}

\begin{entrée}
{ʈʂʰv̩˧ɻ̍˧qʰv̩\#˥}{}{ⓔʈʂʰv̩˧ɻ̍˧qʰv̩\#˥}\formedesurface{ʈʂʰv̩˧ɻ̍˧qʰv̩˧}\newline
\classe{名词}\ton{\#H}
\paradigme{\pcmn{:} \p{}}
\begin{définition}\peng{Ant nest.}\end{définition}
\begin{définition}\pcmn{蚂蚁巢}\end{définition}
\begin{définition}\pfra{Fourmilière.}\end{définition}
\begin{exemple}\pnru{ʈʂʰv̩˧ɻ̍˧qʰv̩˧ ɲi˥!}\hspace{5pt}\peng{It's an ant nest!}\hspace{5pt}\pcmn{是蚂蚁巢!}\hspace{5pt}\pfra{c'est une fourmilière!}\end{exemple}
\end{entrée}

\begin{entrée}
{ʈʂʰv̩˩∼ʈʂʰv̩˩˧}{}{ⓔʈʂʰv̩˩∼ʈʂʰv̩˩˧}\formedesurface{ʈʂʰv̩˩ʈʂʰv̩˩˧}\newline
\classe{动词}\ton{MH}\begin{définition}\peng{To rub, to wipe.}\end{définition}
\begin{définition}\pcmn{擦}\end{définition}
\begin{définition}\pfra{Frotter, essuyer.}\end{définition}
\begin{exemple}\pnru{ʈʂʰv̩˩∼ʈʂʰv̩˩-ze˧}\hspace{5pt}\peng{|fg{red} |fg{pfv}}\hspace{5pt}\pcmn{|fg{red} |fg{pfv}}\hspace{5pt}\pfra{|fg{red} |fg{pfv}}\end{exemple}
\begin{exemple}\pnru{le˧-ʈʂʰv̩˩∼ʈʂʰv̩˩-ze˩}\hspace{5pt}\peng{|fg{accomp} \_ |fg{pfv}}\hspace{5pt}\pcmn{|fg{accomp} \_ |fg{pfv}}\hspace{5pt}\pfra{|fg{accomp} \_ |fg{pfv}}\end{exemple}
\begin{exemple}\pnru{tso˧∼tso˧ ʈʂʰv̩˥∼ʈʂʰv̩˩}\hspace{5pt}\peng{to wipe things}\hspace{5pt}\pcmn{擦东西}\hspace{5pt}\pfra{frotter des choses}\end{exemple}
\end{entrée}

\begin{entrée}
{ʈʂʰwæ˧˥}{₁}{ⓔʈʂʰwæ˧˥ⓗ1}\formedesurface{ʈʂʰwæ˧˥}\newline
\classe{动词}\ton{MH}
1\begin{définition}\peng{To hide (an object).}\end{définition}
\begin{définition}\pcmn{藏(东西)}\end{définition}
\begin{définition}\pfra{Cacher: cacher un objet.}\end{définition}
\end{entrée}

\begin{entrée}
{ʈʂʰwæ˧˥}{₂}{ⓔʈʂʰwæ˧˥ⓗ2}\formedesurface{ʈʂʰwæ˧˥}\newline
\classe{动词}\ton{MH}
2\begin{définition}\peng{To stab.}\end{définition}
\begin{définition}\pcmn{插、戳}\end{définition}
\begin{définition}\pfra{Enfoncer (un couteau) d'un geste brutal: poignarder, de haut en bas, avec force; insérer, planter, ficher (ex.: un couteau, une aiguille…).}\end{définition}
\end{entrée}

\begin{entrée}
{ʈʂʰwæ˧α}{₁}{ⓔʈʂʰwæ˧αⓗ1}\formedesurface{ʈʂʰwæ˧}\newline
\classe{动词}\ton{Mα}
1\begin{définition}\peng{To rot.}\end{définition}
\begin{définition}\pcmn{腐烂}\end{définition}
\begin{définition}\pfra{Pourrir.}\end{définition}
\begin{exemple}\pnru{ʈʂʰwæ˧-ze˧}\hspace{5pt}\peng{|fg{pfv}}\hspace{5pt}\pcmn{烂了}\hspace{5pt}\pfra{|fg{pfv}}\end{exemple}
\begin{exemple}\pnru{le˧-ʈʂʰwæ˧-ze˧}\hspace{5pt}\peng{|fg{accomp} \_ |fg{pfv}}\hspace{5pt}\pcmn{|fg{accomp} \_ |fg{pfv}}\hspace{5pt}\pfra{|fg{accomp} \_ |fg{pfv}}\end{exemple}
\begin{exemple}\pnru{hĩ˧-ɳɯ˩ | mɤ˧-dzɯ˥, | le˧-ʈʂʰwæ˧-ze˧! |}\hspace{5pt}\peng{No one ate it, and now it's rotten! (About a water melon that was forgotten in the cupboard.)}\hspace{5pt}\pcmn{没人吃,就烂了!(一个西瓜被忘记在橱柜里,就腐烂了)}\hspace{5pt}\pfra{On a oublié de la manger, et maintenant c'est pourri! (au sujet d'une pastèque qui a traîné dans le garde-manger et est maintenant incomestible)}\end{exemple}
\end{entrée}

\begin{entrée}
{ʈʂʰwæ˧α}{₂}{ⓔʈʂʰwæ˧αⓗ2}\formedesurface{ʈʂʰwæ˧}\newline
\classe{动词}\ton{Mα}
2\begin{définition}\peng{To wake up.}\end{définition}
\begin{définition}\pcmn{醒来}\end{définition}
\begin{définition}\pfra{Se réveiller.}\end{définition}
\begin{exemple}\pnru{le˧-ʈʂʰwæ˧-ze˧}\hspace{5pt}\peng{|fg{accomp} \_ |fg{pfv}}\hspace{5pt}\pcmn{|fg{accomp} \_ |fg{pfv}}\hspace{5pt}\pfra{|fg{accomp} \_ |fg{pfv}}\end{exemple}
\begin{exemple}\pnru{gɤ˩ʈʂʰwæ˧}\hspace{5pt}\peng{to wake up}\hspace{5pt}\pcmn{醒来}\hspace{5pt}\pfra{se réveiller}\end{exemple}
\begin{exemple}\pnru{gɤ˩ʈʂʰwæ˧-ze˧!}\hspace{5pt}\peng{[(S)he] has woken up!}\hspace{5pt}\pcmn{醒来了!}\hspace{5pt}\pfra{(il) s'est réveillé!}\end{exemple}
\end{entrée}

\begin{entrée}
{ʈʂʰwæ˩˧}{}{ⓔʈʂʰwæ˩˧}\formedesurface{ʈʂʰwæ˩˥}\newline
\classe{名词}\ton{LM}
\paradigme{\pcmn{:} \p{}}
\begin{définition}\peng{Boat.}\end{définition}
\begin{définition}\pcmn{船(汉语借词)}\end{définition}
\begin{définition}\pfra{Bateau (emprunt chinois ancien); désormais, c'est le terme employé pour les grands bateaux, par ex. les barges pour passer le Yangtze; ce sont des Chinois et des Hmong qui auraient installé le bateau permettant de passer le Yangtze, d'où l'utilisation d'un mot chinois.}\end{définition}
\end{entrée}

\begin{entrée}
{ʈʂʰwæ˩α}{}{ⓔʈʂʰwæ˩α}\formedesurface{ʈʂʰwæ˩˥}\newline
\classe{形容词}\ton{Lα}\begin{définition}\peng{Rapid, fast.}\end{définition}
\begin{définition}\pcmn{快(动作快,跑得快)}\end{définition}
\begin{définition}\pfra{Rapide.}\end{définition}
\begin{exemple}\pnru{ʈʂʰwæ˩-hĩ˩˥}\hspace{5pt}\peng{|fg{rel}/|fg{nmlz}}\hspace{5pt}\pcmn{快的}\hspace{5pt}\pfra{|fg{rel}/|fg{nmlz}}\end{exemple}
\begin{exemple}\pnru{ɲi˧to˧ ʈʂʰwæ˩}\hspace{5pt}\peng{who has a loose tongue, to talks too much}\hspace{5pt}\pcmn{嘴快}\hspace{5pt}\pfra{bavard, qui parle sans réfléchir suffisamment}\end{exemple}
\end{entrée}

\begin{entrée}
{ʈʂʰwæ˧-bv̩˧nv̩\#˥}{}{ⓔʈʂʰwæ˧-bv̩˧nv̩\#˥}\formedesurface{ʈʂʰwæ˧bv̩˧nv̩˧}\newline
\classe{形容词}\ton{\#H}\begin{définition}\peng{Bad, spoilt, rotten.}\end{définition}
\begin{définition}\pcmn{食物变味,有臭味道了}\end{définition}
\begin{définition}\pfra{Puant, à l'odeur de pourriture.}\end{définition}
\begin{exemple}\pnru{ʈʂʰwæ˧-bv̩˧nv̩˧ ɲi˥!}\hspace{5pt}\peng{It stinks! / It smells rotten!}\hspace{5pt}\pcmn{臭了、有臭味道了}\hspace{5pt}\pfra{Ca sent le pourri! / Ca pue la pourriture! / C'est vraiment malodorant!}\end{exemple}
\end{entrée}

\begin{entrée}
{ʈʂʰwæ˧kɯ˧}{}{ⓔʈʂʰwæ˧kɯ˧}\formedesurface{ʈʂʰwæ˧kɯ˧}\newline
\classe{名词}\ton{M}
\paradigme{\pcmn{:} \p{}}
\begin{définition}\peng{Net.}\end{définition}
\begin{définition}\pcmn{网}\end{définition}
\begin{définition}\pfra{Filet.}\end{définition}
\begin{exemple}\pnru{ʈʂʰwæ˧kɯ˧ tʰv̩˧-nɑ˩}\hspace{5pt}\peng{|fg{n}+|fg{dem}+|fg{clf}}\hspace{5pt}\pcmn{这个网}\hspace{5pt}\pfra{|fg{n}+|fg{dem}+|fg{clf}}\end{exemple}
\end{entrée}

\begin{entrée}
{ʈʂʰwæ˧pi\#˥}{}{ⓔʈʂʰwæ˧pi\#˥}\formedesurface{ʈʂʰwæ˧pi˧}\newline
\classe{名词}\ton{M}\begin{définition}\peng{Fermented rice wine. This type of alcohol is sweet, not very strong.}\end{définition}
\begin{définition}\pcmn{米酒(甜酒,酒精度低)}\end{définition}
\begin{définition}\pfra{Alcool de riz fermenté. Ce type d'alcool est sucré, et son degré d'alcool est moins élevé que celui des alcools distillés.}\end{définition}
\end{entrée}

\begin{entrée}
{ʈʂʰwæ˧tsɯ˧}{}{ⓔʈʂʰwæ˧tsɯ˧}\formedesurface{ʈʂʰwæ˧tsɯ˧}\newline
\classe{名词}\ton{M}
\paradigme{\pcmn{:} \p{}}
\begin{définition}\peng{Window.}\end{définition}
\begin{définition}\pcmn{窗户}\end{définition}
\begin{définition}\pfra{Fenêtre.}\end{définition}
\end{entrée}

\begin{entrée}
{ʈʂʰwæ˩tsʰɯ˩}{}{ⓔʈʂʰwæ˩tsʰɯ˩}\formedesurface{ʈʂʰwæ˩tsʰɯ˩˥}\newline
\classe{动词}\ton{L}\begin{définition}\peng{To create.}\end{définition}
\begin{définition}\pcmn{创造(汉语借词)}\end{définition}
\begin{définition}\pfra{Créer.}\end{définition}
\end{entrée}

\begin{entrée}
{ʈʂʰwæ˧ʈʂʰwæ˧}{}{ⓔʈʂʰwæ˧ʈʂʰwæ˧}\formedesurface{ʈʂʰwæ˧ʈʂʰwæ˧}\newline
\classe{名词}\ton{M}
\paradigme{\pcmn{:} \p{}}
\begin{définition}\peng{Cymbals (Chinese borrowing).}\end{définition}
\begin{définition}\pcmn{钹}\end{définition}
\begin{définition}\pfra{Cymbales (mot emprunté au chinois).}\end{définition}
\end{entrée}

\begin{entrée}
{ʈʂʰwɤ˩}{}{ⓔʈʂʰwɤ˩}\formedesurface{ʈʂʰwɤ˧}\newline
\classe{名词}\ton{L}\begin{définition}\peng{Dinner, evening meal.}\end{définition}
\begin{définition}\pcmn{晚饭}\end{définition}
\begin{définition}\pfra{Repas du soir, dîner.}\end{définition}
\begin{exemple}\pnru{ʈʂʰwɤ˩ gv̩˩˥}\hspace{5pt}\peng{to cook dinner}\hspace{5pt}\pcmn{做晚饭}\hspace{5pt}\pfra{cuisiner le dîner}\end{exemple}
\begin{exemple}\pnru{ʈʂʰwɤ˩ tʰv̩˩˥}\hspace{5pt}\peng{to take charge of dinner, to prepare dinner, to provide dinner (this can refer to providing the ingredients for the meal, not necessarily preparing it oneself)}\hspace{5pt}\pcmn{请吃晚饭,提供晚餐(不一定自己做:意思是提供原料)}\hspace{5pt}\pfra{offrir à dîner, se charger du dîner (personne qui invite, pas nécessairement qui fait la cuisine elle-même)}\end{exemple}
\begin{exemple}\pnru{ʈʂʰwɤ˩ dzɯ˩˥}\hspace{5pt}\peng{to eat dinner}\hspace{5pt}\pcmn{吃晚饭}\hspace{5pt}\pfra{prendre le repas du soir}\end{exemple}
\end{entrée}

\begin{entrée}
{ʈʂʰwɤ˧tsʰi˧˥}{}{ⓔʈʂʰwɤ˧tsʰi˧˥}\formedesurface{ʈʂʰwɤ˧tsʰi˧˥}\newline
\classe{形容词}\ton{MH\#}\begin{définition}\peng{Narrow.}\end{définition}
\begin{définition}\pcmn{窄}\end{définition}
\begin{définition}\pfra{Étroit.}\end{définition}
\end{entrée}

\newpage\caractère{u}

\begin{entrée}
{u˧}{}{ⓔu˧}\formedesurface{u˧}\newline
\classe{代词}\ton{M}\begin{définition}\peng{First person, associative: my family/my people. This root is only attested together with a plural or associative clitic.}\end{définition}
\begin{définition}\pcmn{我家人、我家族}\end{définition}
\begin{définition}\pfra{Pronom de 1e personne, associatif: les miens. Cette racine n'apparaît qu'en combinaison avec un clitique pluriel ou associatif.}\end{définition}
\begin{exemple}\pnru{u˧=ɻ˩, ʈʂʰɯ˧=ɻ˩}\hspace{5pt}\peng{My clan, his clan: two terms that stand in a relation of opposition}\hspace{5pt}\pcmn{我的人,他的人(有对立的两个代词)}\hspace{5pt}\pfra{mon clan, son clan : deux termes qui forment une opposition}\end{exemple}
\begin{exemple}\pnru{u˧ɻ̍˩ | ə˧si˧}\hspace{5pt}\peng{my great-grandmother}\hspace{5pt}\pcmn{我家祖母}\hspace{5pt}\pfra{mon arrière-grand-mère}\end{exemple}
\begin{exemple}\pnru{u˧=ɻæ˩, ʈʂʰɯ˧=ɻæ˩}\hspace{5pt}\peng{Us, them: two terms that stand in a relation of opposition}\hspace{5pt}\pfra{Nous autres, eux : deux termes qui forment une opposition.}\end{exemple}
\end{entrée}

\newpage\caractère{v}

\begin{entrée}
{v̩˥}{}{ⓔv̩˥}\formedesurface{v̩˧}\newline
\classe{名词}\ton{\#H}
\paradigme{\pcmn{:} \p{}}
\begin{définition}\peng{Pot.}\end{définition}
\begin{définition}\pcmn{锅}\end{définition}
\begin{définition}\pfra{Casserole (terme générique).}\end{définition}
\end{entrée}

\begin{entrée}
{v̩˥β}{}{ⓔv̩˥β}\formedesurface{ɖɯ˧ v̩˥}\newline
\classe{量词}\ton{Hβ}\begin{définition}\peng{Self-classifier for pots; classifier for potfuls (of food, liquid…).}\end{définition}
\begin{définition}\pcmn{量词:锅(一口),或锅的容量}\end{définition}
\begin{définition}\pfra{Auto-classificateur des casseroles; et classificateur des casserolées (utilisant la casserole comme mesure de quantité de nourriture, liquide ou solide).}\end{définition}
\end{entrée}

\begin{entrée}
{v̩˧}{}{ⓔv̩˧}\formedesurface{ɖɯ˧ v̩˧}\newline
\classe{量词}\ton{M *}\begin{définition}\peng{Classifier for one individual (human); can only be used after the numeral ‘one', i.e. either in the singular or in association with numbers whose last figure is ‘one'.}\end{définition}
\begin{définition}\pcmn{量词:人(一个人)。只能用于单数。}\end{définition}
\begin{définition}\pfra{Classificateur pour un individu humain; ne peut s'employer que pour le chiffre ‘un', autrement dit soit au singulier, soit avec des nombres dont le dernier chiffre est ‘un'.}\end{définition}
\begin{exemple}\pnru{ɖɯ˧-v̩\#˥; ɖɯ˧-v̩˧ ɲi˥}\hspace{5pt}\peng{one person; it is one person (elicited to verify tone)}\hspace{5pt}\pcmn{一个人,是一个人(为了确认声调而问的短语)}\hspace{5pt}\pfra{1 personne; c'est 1 personne (élicité pour vérifier le ton)}\end{exemple}
\begin{exemple}\pnru{tsʰe˧ɖɯ˧-v̩˧}\hspace{5pt}\peng{11 persons}\hspace{5pt}\pcmn{十一个人}\hspace{5pt}\pfra{11 personnes}\end{exemple}
\begin{exemple}\pnru{ɲi˧tsi˧ɖɯ˧-v̩˧}\hspace{5pt}\peng{21 persons}\hspace{5pt}\pcmn{二十一个人}\hspace{5pt}\pfra{21 personnes}\end{exemple}
\begin{exemple}\pnru{so˧tsʰi˧ɖɯ˧-v̩˧}\hspace{5pt}\peng{31 persons}\hspace{5pt}\pcmn{三十一个人}\hspace{5pt}\pfra{31 personnes}\end{exemple}
\begin{exemple}\pnru{ʐv̩˧tsʰi˩ɖɯ˩-v̩˩}\hspace{5pt}\peng{41 persons}\hspace{5pt}\pcmn{四十一个人}\hspace{5pt}\pfra{41 personnes}\end{exemple}
\begin{exemple}\pnru{ŋwɤ˧tsʰi˩ɖɯ˩-v̩˩}\hspace{5pt}\peng{51 persons}\hspace{5pt}\pcmn{五十一个人}\hspace{5pt}\pfra{51 personnes}\end{exemple}
\begin{exemple}\pnru{qʰv̩˧tsʰi˧ɖɯ˧-v̩˥}\hspace{5pt}\peng{61 persons}\hspace{5pt}\pcmn{六十一个人}\hspace{5pt}\pfra{61 personnes}\end{exemple}
\begin{exemple}\pnru{ʂɯ˧tsʰi˩ɖɯ˩-v̩˩}\hspace{5pt}\peng{71 persons}\hspace{5pt}\pcmn{七十一个人}\hspace{5pt}\pfra{71 personnes}\end{exemple}
\begin{exemple}\pnru{hõ˧tsʰi˩ɖɯ˩-v̩˩}\hspace{5pt}\peng{81 persons}\hspace{5pt}\pcmn{八十一个人}\hspace{5pt}\pfra{81 personnes}\end{exemple}
\begin{exemple}\pnru{gv̩˧tsʰi˩ɖɯ˩-v̩˩}\hspace{5pt}\peng{91 persons}\hspace{5pt}\pcmn{九十一个人}\hspace{5pt}\pfra{91 personnes}\end{exemple}
\end{entrée}

\begin{entrée}
{v̩˩}{}{ⓔv̩˩}\formedesurface{v̩˩˥}\newline
\classe{动词}\ton{Lα}\begin{définition}\peng{To hug, to embrace.}\end{définition}
\begin{définition}\pcmn{搂(人的脖子)}\end{définition}
\begin{définition}\pfra{Embrasser, prendre dans ses bras (monosyllabe extrait de la forme \stylefv{/le}˧-v̩˧∼v̩˥/).}\end{définition}
\begin{exemple}\pnru{ʁæ˧ | le˧-v̩˧∼v̩˥}\hspace{5pt}\peng{to embrace someone's neck}\hspace{5pt}\pcmn{搂人的脖子}\hspace{5pt}\pfra{prendre le cou (de quelqu'un) dans son bras}\end{exemple}
\begin{exemple}\pnru{ʁæ˧ | le˧-v̩˩}\hspace{5pt}\peng{as above}\hspace{5pt}\pcmn{同上}\hspace{5pt}\pfra{même sens}\end{exemple}
\begin{exemple}\pnru{ʁæ˧ v̩˥ se˩}\hspace{5pt}\peng{to walk together with someone, arm curled around the neck}\hspace{5pt}\pcmn{互相搂着走}\hspace{5pt}\pfra{marcher/se promener en tenant le cou de quelqu'un/enlacé avec quelqu'un}\end{exemple}
\end{entrée}

\begin{entrée}
{v̩˩dze˩}{}{ⓔv̩˩dze˩}\formedesurface{v̩˩dze˩˥}\newline
\classe{名词}\ton{L}
\paradigme{\pcmn{:} \p{}}
\begin{définition}\peng{Bird.}\end{définition}
\begin{définition}\pcmn{鸟}\end{définition}
\begin{définition}\pfra{Oiseau.}\end{définition}
\begin{exemple}\pnru{v̩˩dze˩-bi˥ | hṽ̩˧ ʑi˥}\hspace{5pt}\peng{There are feathers on the bird.}\hspace{5pt}\pcmn{鸟(身)上有(羽)毛。}\hspace{5pt}\pfra{Sur l'oiseau, il y a des plumes.}\end{exemple}
\begin{exemple}\pnru{v̩˩dze˩-mi˩}\hspace{5pt}\peng{female bird}\hspace{5pt}\pcmn{母鸟}\hspace{5pt}\pfra{oiseau femelle}\end{exemple}
\begin{exemple}\pnru{v̩˩dze˩-pʰv̩˩}\hspace{5pt}\peng{male bird}\hspace{5pt}\pcmn{公鸟}\hspace{5pt}\pfra{oiseau mâle}\end{exemple}
\begin{exemple}\pnru{v̩˩dze˩-zo˩}\hspace{5pt}\peng{baby bird}\hspace{5pt}\pcmn{小鸟}\hspace{5pt}\pfra{petit oiseau}\end{exemple}
\end{entrée}

\begin{entrée}
{v̩˩dze˩-kʰv̩˩}{}{ⓔv̩˩dze˩-kʰv̩˩}\formedesurface{v̩˩dze˩-kʰv̩˩˥}\newline
\classe{名词}\ton{L}
\paradigme{\pcmn{:} \p{}}
\begin{définition}\peng{Nest.}\end{définition}
\begin{définition}\pcmn{鸟窝,鸟巢}\end{définition}
\begin{définition}\pfra{Nid.}\end{définition}
\begin{exemple}\pnru{v̩˩dze˩kʰv̩˩ ɲi˥.}\hspace{5pt}\peng{\_ |fg{cop}}\hspace{5pt}\pcmn{是鸟窝}\hspace{5pt}\pfra{\_ |fg{cop}}\end{exemple}
\end{entrée}

\begin{entrée}
{v̩˧dʑo\#˥}{}{ⓔv̩˧dʑo\#˥}\formedesurface{v̩˧dʑo˧}\newline
\classe{名词}\ton{\#H}\begin{définition}\peng{Wujiao township.}\end{définition}
\begin{définition}\pcmn{屋脚(村落名)}\end{définition}
\begin{définition}\pfra{Wujiao (nom de village).}\end{définition}
\end{entrée}

\begin{entrée}
{v̩˩dʑɯ˩}{}{ⓔv̩˩dʑɯ˩}\formedesurface{v̩˩dʑɯ˩˥}\newline
\classe{名词}\ton{L}\begin{définition}\peng{Soup.}\end{définition}
\begin{définition}\pcmn{汤}\end{définition}
\begin{définition}\pfra{Soupe.}\end{définition}
\begin{exemple}\pnru{æ˩ʂe˧-v̩˥dʑɯ˩}\hspace{5pt}\peng{chicken soup}\hspace{5pt}\pcmn{鸡汤}\hspace{5pt}\pfra{soupe de poulet}\end{exemple}
\end{entrée}

\begin{entrée}
{v̩˧ko˧}{}{ⓔv̩˧ko˧}\formedesurface{v̩˧ko˧}\newline
\classe{名词}\ton{M}\begin{définition}\peng{Tortoise.}\end{définition}
\begin{définition}\pcmn{乌龟(汉语借词)}\end{définition}
\begin{définition}\pfra{Tortue.}\end{définition}
\end{entrée}

\begin{entrée}
{v̩˧lɑ˩-ʝi˩}{}{ⓔv̩˧lɑ˩-ʝi˩}\formedesurface{v̩˧lɑ˩ʝi˩}\newline
\classe{动词}\ton{L\#-}\begin{définition}\peng{To trade, to do business.}\end{définition}
\begin{définition}\pcmn{做生意}\end{définition}
\begin{définition}\pfra{Faire du commerce.}\end{définition}
\end{entrée}

\begin{entrée}
{v̩˧lɑ˩-ʝi˩-hĩ˩-hĩ˩}{}{ⓔv̩˧lɑ˩-ʝi˩-hĩ˩-hĩ˩}\formedesurface{v̩˧lɑ˩ʝi˩hĩ˩hĩ˩}\newline
\classe{名词}\ton{L\#-}
\paradigme{\pcmn{:} \p{}}
\begin{définition}\peng{Merchant.}\end{définition}
\begin{définition}\pcmn{商人}\end{définition}
\begin{définition}\pfra{Marchand.}\end{définition}
\begin{exemple}\pnru{v̩˧lɑ˩-ʝi˩-hĩ˩}\hspace{5pt}\peng{merchant}\hspace{5pt}\pcmn{商人}\hspace{5pt}\pfra{marchand}\end{exemple}
\end{entrée}

\begin{entrée}
{v̩˧mi\#˥}{}{ⓔv̩˧mi\#˥}\formedesurface{v̩˧mi˧}\newline
\classe{名词}\ton{\#H}\begin{définition}\peng{Large cooking pot.}\end{définition}
\begin{définition}\pcmn{大锅}\end{définition}
\begin{définition}\pfra{Grande casserole.}\end{définition}
\end{entrée}

\begin{entrée}
{v̩˩tsʰɤ˧˥}{}{ⓔv̩˩tsʰɤ˧˥}\formedesurface{v̩˩tsʰɤ˧˥}\newline
\classe{名词}\ton{LM+MH\#}
\paradigme{\pcmn{:} \p{}}
\begin{définition}\peng{Vegetables (in a broad sense including fresh vegetables and picked vegetables).}\end{définition}
\begin{définition}\pcmn{蔬菜}\end{définition}
\begin{définition}\pfra{Légumes.}\end{définition}
\begin{exemple}\pnru{v˩tsʰɤ˧-tsʰɑ˧nɑ˥}\hspace{5pt}\peng{fresh vegetables. Literally ‘dark vegetables'. This does not refer to one species in particular, but to all sorts of fresh vegetables, as opposed to pickled vegetables. In the process of preserving vegetables, their original darker colours tend to fade away.}\hspace{5pt}\pcmn{新鲜蔬菜。直译:‘绿油油的青菜’。指的不是某种具体的青菜,而是任何新鲜蔬菜,分别于酸菜。制造酸菜的过程中,蔬菜(萝卜等等)褪色:失去原来的深色。}\hspace{5pt}\pfra{légumes frais. Littéralement ‘légume de couleur sombre. L'expression ne renvoie pas à une espèce en particulier, mais désigne globalement les légumes verts, par opposition aux légumes conservés, qui perdaient de leur couleur au cours du processus de fermentation.}\end{exemple}
\end{entrée}

\begin{entrée}
{v̩˩tsʰɤ˧-bv̩\#˥}{}{ⓔv̩˩tsʰɤ˧-bv̩\#˥}\formedesurface{v̩˩tsʰɤ˧bv̩˧}\newline
\classe{名词}\ton{LM+\#H}
\paradigme{\pcmn{:} \p{}}
\begin{définition}\peng{Ladybug, ladybird.}\end{définition}
\begin{définition}\pcmn{瓢虫}\end{définition}
\begin{définition}\pfra{Coccinelle.}\end{définition}
\end{entrée}

\begin{entrée}
{v̩˩tsʰɤ˧-pʰv̩˥}{}{ⓔv̩˩tsʰɤ˧-pʰv̩˥}\formedesurface{v̩˩tsʰɤ˧pʰv̩˥}\newline
\classe{名词}\ton{LM+H\#}
\paradigme{\pcmn{:} \p{}}
\begin{définition}\peng{Chinese cabbage.}\end{définition}
\begin{définition}\pcmn{白菜}\end{définition}
\begin{définition}\pfra{Chou chinois.}\end{définition}
\end{entrée}

\begin{entrée}
{v̩˩tsʰɤ˧-v̩˥ɲi˩}{}{ⓔv̩˩tsʰɤ˧-v̩˥ɲi˩}\formedesurface{v̩˩tsʰɤ˧v̩˥ɲi˩}\newline
\classe{名词}\ton{LM+\#H-}
\paradigme{\pcmn{:} \p{}}
\begin{définition}\peng{Vegetables.}\end{définition}
\begin{définition}\pcmn{蔬菜}\end{définition}
\begin{définition}\pfra{Légumes.}\end{définition}
\end{entrée}

\begin{entrée}
{v̩˧∼v̩˧α}{}{ⓔv̩˧∼v̩˧α}\formedesurface{v̩˧v̩˧}\newline
\classe{动词}\ton{Mα}\begin{définition}\peng{To chew; to chew the cud.}\end{définition}
\begin{définition}\pcmn{嚼}\end{définition}
\begin{définition}\pfra{Mâcher.}\end{définition}
\begin{exemple}\pnru{le˧-v̩˧∼v̩˧ +ze˩}\hspace{5pt}\peng{|fg{accomp}}\hspace{5pt}\pcmn{|fg{accomp}}\hspace{5pt}\pfra{|fg{accomp}}\end{exemple}
\begin{exemple}\pnru{le˧-wo˧ v̩˧∼v̩˧}\hspace{5pt}\peng{to chew the cud}\hspace{5pt}\pcmn{反刍}\hspace{5pt}\pfra{ruminer (la vache rumine)}\end{exemple}
\end{entrée}

\begin{entrée}
{v̩˧zo\#˥}{}{ⓔv̩˧zo\#˥}\formedesurface{v̩˧zo˧}\newline
\classe{名词}\ton{\#H}\begin{définition}\peng{Small cooking pot.}\end{définition}
\begin{définition}\pcmn{小锅}\end{définition}
\begin{définition}\pfra{Petite casserole.}\end{définition}
\end{entrée}

\newpage\caractère{w}

\begin{entrée}
{wɤ˧}{₁}{ⓔwɤ˧ⓗ1}\formedesurface{wɤ˧}\newline
\classe{名词}\ton{M}
1
\paradigme{\pcmn{:} \p{}}
\begin{définition}\peng{Serf, slave (lowest of the 3 ranks in feudal society).}\end{définition}
\begin{définition}\pcmn{奴隶,农奴。音译:“俄”}\end{définition}
\begin{définition}\pfra{Serf, esclave (la plus basse caste de la société ancienne).}\end{définition}
\end{entrée}

\begin{entrée}
{wɤ˧}{₂}{ⓔwɤ˧ⓗ2}\formedesurface{wɤ˧}\newline
\classe{语气助词}\ton{M}
2\begin{définition}\peng{Final particle conveying exclamation, with a nuance of obviousness.}\end{définition}
\begin{définition}\pcmn{句尾助词:吧、呗}\end{définition}
\begin{définition}\pfra{Particule finale exclamative, avec une nuance d'évidence.}\end{définition}
\end{entrée}

\begin{entrée}
{wɤ˩˥}{}{ⓔwɤ˩˥}\formedesurface{wɤ˩˥}\newline
\classe{助词}\ton{LM? LH?}\begin{définition}\peng{Again; also.}\end{définition}
\begin{définition}\pcmn{又,再}\end{définition}
\begin{définition}\pfra{À nouveau, encore; aussi.}\end{définition}
\begin{exemple}\pnru{wɤ˩˥ | ɖɯ˧-ʂɯ˩}\hspace{5pt}\peng{once again, once more, one more time}\hspace{5pt}\pcmn{再一次、又一次}\hspace{5pt}\pfra{une nouvelle fois, une fois de plus}\end{exemple}
\end{entrée}

\begin{entrée}
{wɤ˩α}{}{ⓔwɤ˩α}\formedesurface{wɤ˩˥}\newline
\classe{动词}\ton{Lα}\begin{définition}\peng{To depend on.}\end{définition}
\begin{définition}\pcmn{依赖}\end{définition}
\begin{définition}\pfra{Dépendre de, se reposer sur.}\end{définition}
\begin{exemple}\pnru{hĩ˧-bi˥ | wɤ˩-mɤ˩-bi˩˥!}\hspace{5pt}\peng{One should not depend on others!}\hspace{5pt}\pcmn{不要依赖别人!}\hspace{5pt}\pfra{Il ne faut pas dépendre des autres!}\end{exemple}
\begin{exemple}\pnru{hĩ˧-bi˥ | wɤ˩-v̩˩-tʰv̩˩˥!}\hspace{5pt}\peng{(Whether one wants or not) one depends on others (in some respect or other)!}\hspace{5pt}\pcmn{(无论如何)人都会依靠别人的!(意思是:人不能完全独立,人活在人间就会或多或少需要依靠别人。)}\hspace{5pt}\pfra{On se trouve dépendre des autres!}\end{exemple}
\end{entrée}

\begin{entrée}
{wɤ˩β}{}{ⓔwɤ˩β}\formedesurface{ɖɯ˧ wɤ˩}\newline
\classe{量词}\ton{Lβ}\begin{définition}\peng{Load, charge, weight.}\end{définition}
\begin{définition}\pcmn{量词:担,负荷}\end{définition}
\begin{définition}\pfra{Classificateur des charges/fardeaux qu'une personne peut porter.}\end{définition}
\begin{exemple}\pnru{ɖɯ˧-wɤ˩ pɤ˩∼pɤ˩ |}\hspace{5pt}\peng{to carry a load}\hspace{5pt}\pcmn{背一担}\hspace{5pt}\pfra{porter une charge}\end{exemple}
\begin{exemple}\pnru{ɖɯ˧-wɤ˩, | ɖɯ˧-wɤ˩ | le˧-kʰɯ˩∼kʰɯ˩ | tʰi˧-tɕɯ˥ |}\hspace{5pt}\peng{to pile up loads, one after the other}\hspace{5pt}\pcmn{将驮的大包堆起来}\hspace{5pt}\pfra{entasser les charges, l'une après l'autre}\end{exemple}
\end{entrée}

\begin{entrée}
{wɤ˩∼wɤ˩}{}{ⓔwɤ˩∼wɤ˩}\formedesurface{wɤ˩wɤ˩˥}\newline
\classe{动词}\ton{L}\begin{définition}\peng{To detour past, to bypass.}\end{définition}
\begin{définition}\pcmn{绕过}\end{définition}
\begin{définition}\pfra{Contourner.}\end{définition}
\begin{exemple}\pnru{le˧-wɤ˩-ze˩}\hspace{5pt}\peng{|fg{accomp} \_ |fg{pfv}}\hspace{5pt}\pcmn{绕了}\hspace{5pt}\pfra{|fg{accomp} \_ pfv}\end{exemple}
\begin{exemple}\pnru{ɖɯ˧-wɤ˩∼wɤ˩-ɻ̍˩}\hspace{5pt}\peng{|fg{delimitative} \_ |fg{red} |fg{inceptive}}\hspace{5pt}\pcmn{绕一绕}\hspace{5pt}\pfra{|fg{délimitatif} \_ |fg{red} |fg{inchoatif}}\end{exemple}
\begin{exemple}\pnru{wɤ˩∼wɤ˩ bi˩˥}\hspace{5pt}\peng{|fg{imm\_fut}}\hspace{5pt}\pcmn{|fg{imm\_fut}}\hspace{5pt}\pfra{|fg{fut\_imm}}\end{exemple}
\begin{exemple}\pnru{wɤ˩∼wɤ˩-ze˥}\hspace{5pt}\peng{|fg{pfv}}\hspace{5pt}\pcmn{绕了}\hspace{5pt}\pfra{|fg{pfv}}\end{exemple}
\begin{exemple}\pnru{le˧-wɤ˩∼wɤ˩ | le˧-se˥}\hspace{5pt}\peng{to bypass on foot; to walk past, bypassing (a certain place)}\hspace{5pt}\pcmn{走路绕过}\hspace{5pt}\pfra{contourner à pied}\end{exemple}
\end{entrée}

\begin{entrée}
{wo˥}{}{ⓔwo˥}\formedesurface{wo˧}\newline
\classe{形容词}\ton{H}\begin{définition}\peng{Hard, solid, resilient.}\end{définition}
\begin{définition}\pcmn{硬,坚硬,结实}\end{définition}
\begin{définition}\pfra{Dur, solide, résistant.}\end{définition}
\begin{exemple}\pnru{le˧-wo˥-ze˩}\hspace{5pt}\peng{|fg{accomp} \_ |fg{pfv}: it hardened}\hspace{5pt}\pcmn{硬了}\hspace{5pt}\pfra{|fg{accomp} \_ |fg{pfv}: ça a durci, c'est devenu dur}\end{exemple}
\end{entrée}

\begin{entrée}
{wo˧˥}{}{ⓔwo˧˥}\formedesurface{wo˧˥}\newline
\classe{动词}\ton{MH}\begin{définition}\peng{To do (something) over again.}\end{définition}
\begin{définition}\pcmn{重新做、再来做}\end{définition}
\begin{définition}\pfra{Se retourner (quelqu'un est assis et se retourne: mouvement du torse).}\end{définition}
\begin{exemple}\pnru{le˧-wo˧ ʐwɤ˧˥}\hspace{5pt}\peng{to answer}\hspace{5pt}\pcmn{回答}\hspace{5pt}\pfra{répondre, donner une réponse}\end{exemple}
\begin{exemple}\pnru{le˧-wo˧-ɻ̍˥}\hspace{5pt}\peng{to turn around (e.g. in order to look back)}\hspace{5pt}\pcmn{转身}\hspace{5pt}\pfra{se retourner}\end{exemple}
\begin{exemple}\pnru{le˧-wo˧ li˥}\hspace{5pt}\peng{to look back}\hspace{5pt}\pcmn{往后看}\hspace{5pt}\pfra{regarder derrière soi}\end{exemple}
\begin{exemple}\pnru{lə-˧wo˧ tʰo˥-tɕo˩}\hspace{5pt}\peng{to turn around (e.g. in order to look back)}\hspace{5pt}\pcmn{转身}\hspace{5pt}\pfra{se retourner}\end{exemple}
\begin{exemple}\pnru{le˧-wo˧-tɕo˥!}\hspace{5pt}\peng{Turn around! (Said to a baby who is about to get down a bed head first)}\hspace{5pt}\pcmn{转身!(婴儿爬下床,头朝下。奶奶告诉她:要先转身)}\hspace{5pt}\pfra{retourne-toi! (adressé à un bébé qui s'apprête à descendre d'un lit la tête la première)}\end{exemple}
\begin{exemple}\pnru{le˧-wo˧˥ | le˧-hɯ˩}\hspace{5pt}\peng{has gone back, went back}\hspace{5pt}\pcmn{回去了}\hspace{5pt}\pfra{…est reparti}\end{exemple}
\end{entrée}

\begin{entrée}
{wo˩˥}{}{ⓔwo˩˥}\formedesurface{wo˩˥}\newline
\classe{名词}\ton{LH}\begin{définition}\peng{Turnip leaves; they used to be eaten as a vegetable.}\end{définition}
\begin{définition}\pcmn{圆根的叶子}\end{définition}
\begin{définition}\pfra{Feuilles du navet.}\end{définition}
\begin{exemple}\pnru{wo˩bɤ˧˥}\hspace{5pt}\peng{same meaning: turnip leaves}\hspace{5pt}\pcmn{同上:圆根叶子}\hspace{5pt}\pfra{même sens: feuilles du navet}\end{exemple}
\begin{exemple}\pnru{wo˩-v˥tsʰɤ˩}\hspace{5pt}\peng{same meaning: turnip leaves; literally ‘turnip leaves vegetable', emphasizing the fact that they are used as a vegetable: as an ingredient in a recipe}\hspace{5pt}\pcmn{同上:圆根叶子}\hspace{5pt}\pfra{même sens: feuilles du navet; littéralement ‘légume-feuilles du navet'; l'expression souligne qu'il s'agit d'une variété de légume: d'un ingrédient pour la cuisine}\end{exemple}
\begin{exemple}\pnru{wo˩-tɕæ˩ɻæ˥}\hspace{5pt}\peng{pickled turnip leaves}\hspace{5pt}\pcmn{圆根叶子酸菜}\hspace{5pt}\pfra{feuilles du navet conservées dans la saumure}\end{exemple}
\end{entrée}

\begin{entrée}
{wo˩β}{}{ⓔwo˩β}\formedesurface{ɖɯ˧ wo˩}\newline
\classe{量词}\ton{Lβ}\begin{définition}\peng{Classifier for teams of oxen. In Yongning, the ard is drawn by two oxen, or two small water buffaloes, or one strong water buffalo.}\end{définition}
\begin{définition}\pcmn{量词:牛(一架)}\end{définition}
\begin{définition}\pfra{Classificateur des paires de bœufs; attelage de bœufs pour tirer l'araire. A Yongning, l'attelage comporte deux bœufs, ou deux petits buffles, ou un seul buffle vigoureux.}\end{définition}
\begin{exemple}\pnru{dʑi˧mi˧ | ɲi˧-pʰo˧˥, | ɖɯ˧-wo˩!}\hspace{5pt}\peng{Two water buffaloes make up one team!}\hspace{5pt}\pcmn{两头水牛,等于一架!}\hspace{5pt}\pfra{Deux buffles, cela forme un attelage!}\end{exemple}
\end{entrée}

\begin{entrée}
{wo˩kɤ\#˥}{}{ⓔwo˩kɤ\#˥}\formedesurface{wo˩kɤ˥}\newline
\classe{名词}\ton{LM+\#H}
\paradigme{\pcmn{:} \p{}}
\begin{définition}\peng{Swing.}\end{définition}
\begin{définition}\pcmn{秋千(鞦韆)}\end{définition}
\begin{définition}\pfra{Balançoire.}\end{définition}
\begin{exemple}\pnru{wo˩kɤ˧-tsɑ˧-di˧˥}\hspace{5pt}\peng{same meaning: swing}\hspace{5pt}\pcmn{同上:秋千}\hspace{5pt}\pfra{même sens: balançoire}\end{exemple}
\begin{exemple}\pnru{wo˩kɤ˧ tsɑ˧˥}\hspace{5pt}\peng{same meaning: swing}\hspace{5pt}\pcmn{同上:秋千}\hspace{5pt}\pfra{même sens: balançoire}\end{exemple}
\end{entrée}

\begin{entrée}
{w̃æ˧}{}{ⓔw̃æ˧}\formedesurface{w̃æ˧}\newline
\classe{动词}\ton{M intrans}\begin{définition}\peng{To swell, to inflate (e.g. the belly is swollen).}\end{définition}
\begin{définition}\pcmn{肿,膨胀,(肚子)胀}\end{définition}
\begin{définition}\pfra{Se gonfler, enfler (ventre).}\end{définition}
\begin{exemple}\pnru{ɻ̍˧tɑ˧ w̃æ˧ (-ze˧)}\hspace{5pt}\peng{glands are swollen}\hspace{5pt}\pcmn{淋巴结肿了}\hspace{5pt}\pfra{les ganglions sont enflés}\end{exemple}
\begin{exemple}\pnru{tso˧∼tso˧ w̃æ˩}\hspace{5pt}\peng{something has swollen}\hspace{5pt}\pcmn{东西膨胀了}\hspace{5pt}\pfra{quelque chose a enflé}\end{exemple}
\end{entrée}

\newpage\caractère{z}

\begin{entrée}
{zɑ˥}{}{ⓔzɑ˥}\formedesurface{zɑ˧}\newline
\classe{形容词}\ton{H}\begin{définition}\peng{Restricted to, limited to.}\end{définition}
\begin{définition}\pcmn{仅仅}\end{définition}
\begin{définition}\pfra{Limité à, restreint à (en tournure négative).}\end{définition}
\begin{exemple}\pnru{ʁwɤ˧-qo˧-ɳɯ˧-lɑ˧ mɤ˧-zɑ˥ (…)}\hspace{5pt}\peng{not only the people from the village}\hspace{5pt}\pcmn{不仅有村子里的人}\hspace{5pt}\pfra{Il n'y avait pas que les gens du village (…)}\end{exemple}
\end{entrée}

\begin{entrée}
{zɑ˩α}{}{ⓔzɑ˩α}\formedesurface{zɑ˩˥}\newline
\classe{动词}\ton{Lα}\begin{définition}\peng{To go downward (a mountain), to descend.}\end{définition}
\begin{définition}\pcmn{下(山……)}\end{définition}
\begin{définition}\pfra{Descendre (redescendre de la montagne).}\end{définition}
\begin{exemple}\pnru{ʁwɤ˩ zɑ˩˥}\hspace{5pt}\peng{to go down the mountain}\hspace{5pt}\pcmn{下山}\hspace{5pt}\pfra{descendre de la montagne}\end{exemple}
\begin{exemple}\pnru{mɤ˧-zɑ˩-sɯ˩}\hspace{5pt}\peng{not to go down yet}\hspace{5pt}\pcmn{还没下来}\hspace{5pt}\pfra{ne pas descendre encore}\end{exemple}
\begin{exemple}\pnru{ɖɯ˧-zɑ˧∼zɑ˥-ɻ̍˩}\hspace{5pt}\peng{|fg{delimitative} \_ |fg{red} |fg{inceptive}}\hspace{5pt}\pcmn{下来一下}\hspace{5pt}\pfra{|fg{délimitatif} \_ |fg{red} |fg{inchoatif}}\end{exemple}
\end{entrée}

\begin{entrée}
{zɑ˩-bɑ˧lɑ˩}{}{ⓔzɑ˩-bɑ˧lɑ˩}\formedesurface{zɑ˩bɑ˧lɑ˩}\newline
\classe{名词}
\sens{1}
\begin{définition}\peng{Religious painting (thangka) on wood, on the wall next to the hearth.}\end{définition}
\begin{définition}\pcmn{火塘旁边墙上的壁画(唐卡:内容来自藏传佛教)}\end{définition}
\begin{définition}\pfra{Peinture religieuse (thangka) sur bois, dans une niche sur le mur, au-dessus du foyer.}\end{définition}\sens{2}
\begin{définition}\peng{Divinity of fire, of the hearth, and of the house.}\end{définition}
\begin{définition}\pcmn{火,火塘与家的神}\end{définition}
\begin{définition}\pfra{Divinité du feu, du foyer et de la maison.}\end{définition}
\end{entrée}

\begin{entrée}
{zɑ˧ɭɯ˧}{}{ⓔzɑ˧ɭɯ˧}\formedesurface{zɑ˧ɭɯ˧}\newline
\classe{名词}\ton{M}
\paradigme{\pcmn{:} \p{}}
\begin{définition}\peng{Barrow, castrated male pig, neutered pig.}\end{définition}
\begin{définition}\pcmn{阉猪}\end{définition}
\begin{définition}\pfra{Porc castré.}\end{définition}
\end{entrée}

\begin{entrée}
{zɑ˩ɲi˥-ʂɤ˩}{}{ⓔzɑ˩ɲi˥-ʂɤ˩}\formedesurface{zɑ˩ɲi˥ʂɤ˩}\newline
\classe{名词}\ton{LH-}\begin{définition}\peng{Vampire: a demon of human shape (the size of a large person), who feeds on blood.}\end{définition}
\begin{définition}\pcmn{吸血鬼}\end{définition}
\begin{définition}\pfra{Vampire; démon malfaisant de forme humaine (de la taille d'un humain de grande taille), qui ne mange pas de viande, et se nourrit de sang.}\end{définition}
\end{entrée}

\begin{entrée}
{zɑ˧zɑ˧}{}{ⓔzɑ˧zɑ˧}\formedesurface{zɑ˧zɑ˧}\newline
\classe{形容词}\ton{M}\begin{définition}\peng{Careful.}\end{définition}
\begin{définition}\pcmn{细心、细致}\end{définition}
\begin{définition}\pfra{Attentif, soigneux.}\end{définition}
\end{entrée}

\begin{entrée}
{‑ze˧β}{}{ⓔ‑ze˧β}\formedesurface{ze˧}\newline
\classe{后缀}\ton{M}\begin{définition}\peng{Perfective, |fg{pfv}}\end{définition}
\begin{définition}\pcmn{整体体}\end{définition}
\begin{définition}\pfra{Perfectif, |fg{pfv}}\end{définition}
\end{entrée}

\begin{entrée}
{ze˩}{}{ⓔze˩}\formedesurface{ze˩˥}\newline
\classe{形容词}\ton{L}\begin{définition}\peng{Pure.}\end{définition}
\begin{définition}\pcmn{纯洁}\end{définition}
\begin{définition}\pfra{Pur.}\end{définition}
\begin{exemple}\pnru{mɤ˧-ʂo˩ F | le˧-ʂo˩-kʰɯ˩! | mɤ˧-ze˩ F | le˧-ze˩-kʰɯ˩!}\hspace{5pt}\peng{Let what is not pure be purified! Let what is not clean be cleaned up!}\hspace{5pt}\pcmn{让不干净的变得干净!让不纯洁的变得纯洁!(仪式用语)}\hspace{5pt}\pfra{Que ce qui n'est pas propre, soit nettoyé / devienne propre! Que ce qui est impur soit purifié! (formule dite lors de rituels de purification: voir le récit Mountains)}\end{exemple}
\end{entrée}

\begin{entrée}
{ze˩}{}{ⓔze˩}\formedesurface{ze˩˥}\newline
\classe{代词}\ton{L}\begin{définition}\peng{Which.}\end{définition}
\begin{définition}\pcmn{哪}\end{définition}
\begin{définition}\pfra{Quel, lequel.}\end{définition}
\end{entrée}

\begin{entrée}
{ze˩bæ˧}{}{ⓔze˩bæ˧}\formedesurface{ze˩bæ˥}\newline
\classe{代词}\ton{LM}\begin{définition}\peng{Which; which kind.}\end{définition}
\begin{définition}\pcmn{哪,哪个 (哪个碗),哪一种}\end{définition}
\begin{définition}\pfra{Quelle sorte de, lequel.}\end{définition}
\begin{exemple}\pnru{ze˩bæ˧ ɲi˥?}\hspace{5pt}\peng{Which one is it? / Which kind is it?}\hspace{5pt}\pcmn{是哪个?是哪一样?}\hspace{5pt}\pfra{c'est lequel?/c'est de quelle sorte?}\end{exemple}
\end{entrée}

\begin{entrée}
{ze˩bæ˩}{}{ⓔze˩bæ˩}\formedesurface{ze˩bæ˩˥}\newline
\classe{名词}\ton{L}
\paradigme{\pcmn{:} \p{}}
\begin{définition}\peng{Flash of lightning, thunderbolt.}\end{définition}
\begin{définition}\pcmn{闪电、打闪电、霹雷}\end{définition}
\begin{définition}\pfra{Éclair.}\end{définition}
\begin{exemple}\pnru{ze˩bæ˩-ze˥!}\hspace{5pt}\peng{There has been a flash of lightning!}\hspace{5pt}\pcmn{打闪电了!}\hspace{5pt}\pfra{il y a eu un éclair!}\end{exemple}
\begin{exemple}\pnru{ze˩bæ˩˥ ◊ -dʑo˩!}\hspace{5pt}\peng{There are flashes of lightning!}\hspace{5pt}\pcmn{打着闪电!}\hspace{5pt}\pfra{il y a des éclairs!}\end{exemple}
\end{entrée}

\begin{entrée}
{ze˩gɤ˧}{}{ⓔze˩gɤ˧}\formedesurface{ze˩gɤ˥}\newline
\classe{代词}\ton{LM}\begin{définition}\peng{At which place, where.}\end{définition}
\begin{définition}\pcmn{哪里,什么地方}\end{définition}
\begin{définition}\pfra{Où, à quel endroit.}\end{définition}
\end{entrée}

\begin{entrée}
{ze˩mi˩}{}{ⓔze˩mi˩}\formedesurface{ze˩mi˩˥}\newline
\classe{名词}\ton{L}
\paradigme{\pcmn{:} \p{}}
\begin{définition}\peng{Niece.}\end{définition}
\begin{définition}\pcmn{甥女(姐妹的女儿)}\end{définition}
\begin{définition}\pfra{Nièce (enfant d'une soeur).}\end{définition}
\end{entrée}

\begin{entrée}
{ze˩v̩˩}{}{ⓔze˩v̩˩}\formedesurface{ze˩v̩˩˥}\newline
\classe{名词}\ton{L}
\paradigme{\pcmn{:} \p{}}
\begin{définition}\peng{Nephew (son of one's sister).}\end{définition}
\begin{définition}\pcmn{外甥(姐妹的儿子)}\end{définition}
\begin{définition}\pfra{Neveu (fils d'une sœur).}\end{définition}
\end{entrée}

\begin{entrée}
{ze˩v̩˩-ze˧mi˩}{}{ⓔze˩v̩˩-ze˧mi˩}\formedesurface{ze˩v̩˩ze˧mi˩}\newline
\classe{名词}\ton{L-L\#}\begin{définition}\peng{Nephews and nieces.}\end{définition}
\begin{définition}\pcmn{外甥甥女(姐妹的儿女)}\end{définition}
\begin{définition}\pfra{Neveux et nièces (du côté des sœurs: enfants des sœurs).}\end{définition}
\end{entrée}

\begin{entrée}
{‑zo}{}{ⓔ‑zo}\formedesurface{--}\newline
\classe{后缀}\ton{?}\begin{définition}\peng{Adverbializer, |fg{advb.}}\end{définition}
\begin{définition}\pcmn{副词化:……地}\end{définition}
\begin{définition}\pfra{Adverbialisateur, |fg{advb.}}\end{définition}
\begin{exemple}\pnru{gv̩˩dʑɯ˩-zo˥}\hspace{5pt}\peng{sadly, with great sadness}\hspace{5pt}\pcmn{难过乎乎地}\hspace{5pt}\pfra{tristement}\end{exemple}
\begin{exemple}\pnru{ʂv̩˧ɖv̩˧-zo˩}\hspace{5pt}\peng{pensively}\hspace{5pt}\pcmn{沉思地}\hspace{5pt}\pfra{pensivement}\end{exemple}
\begin{exemple}\pnru{ʐæ˧-zo˩}\hspace{5pt}\peng{smilingly}\hspace{5pt}\pcmn{笑着地}\hspace{5pt}\pfra{en souriant}\end{exemple}
\begin{exemple}\pnru{ɳɯ˧ɕi˩-zo˩}\hspace{5pt}\peng{in a lovely/amiable way}\hspace{5pt}\pcmn{可爱地}\hspace{5pt}\pfra{de façon mignonne}\end{exemple}
\begin{exemple}\pnru{ʈʂʰɯ˧ne˧-ʝi˥ | pi˧-zo˩, | njɤ˧ | mɤ˧-gɤ˩ | ʐwæ˩˥! (M18)}\hspace{5pt}\peng{His speaking that way makes me terribly cross/discontents me greatly!}\hspace{5pt}\pcmn{(他)这么说/这样的说法,让我非常不满意!}\hspace{5pt}\pfra{Qu’il parle ainsi, cela me contrarie terriblement!}\end{exemple}
\begin{exemple}\pnru{ə˧v̩˧-zo˥!}\hspace{5pt}\peng{Very pretty! / That's really sweet! (Said to a little girl who is showing around a pair of brand new shoes)}\hspace{5pt}\pcmn{真漂亮!(情景:一个小姑娘让大家看她的一双新鞋子,奶奶就给予所需称赞)}\hspace{5pt}\pfra{C'est joli! C'est mignon tout plein! (Commentaire adressé à une fillette qui montre à la ronde des souliers tout neufs.)}\end{exemple}
\end{entrée}

\begin{entrée}
{zo˥}{}{ⓔzo˥}\newline
\classe{名词}
\sens{1}\paradigme{\pcmn{:} \p{}}
\begin{définition}\peng{Son.}\end{définition}
\begin{définition}\pcmn{儿子}\end{définition}
\begin{définition}\pfra{Fils.}\end{définition}
\begin{exemple}\pnru{zo˧ ɲi˥-kv̩˩}\hspace{5pt}\peng{two sons}\hspace{5pt}\pcmn{两个儿子}\hspace{5pt}\pfra{deux fils}\end{exemple}\sens{2}
\begin{définition}\peng{Man, |\stylefi{Vir}.}\end{définition}
\begin{définition}\pcmn{男人}\end{définition}
\begin{définition}\pfra{Homme, |\stylefi{Vir}.}\end{définition}
\end{entrée}

\begin{entrée}
{zo˧α}{}{ⓔzo˧α}\formedesurface{zo˧}\newline
\classe{动词}\ton{Mα}\begin{définition}\peng{To have to, to be necessary.}\end{définition}
\begin{définition}\pcmn{要,应该}\end{définition}
\begin{définition}\pfra{Devoir.}\end{définition}
\begin{exemple}\pnru{mɤ˧-zo˧ (-ze˧)! | tʰi˧-kwɤ˩-kʰɯ˩!}\hspace{5pt}\peng{It's not necessary! Forget it!}\hspace{5pt}\pcmn{不用了!算了吧!}\hspace{5pt}\pfra{ce n'est pas la peine! laisse tomber!}\end{exemple}
\begin{exemple}\pnru{ʈʂʰɯ˧ne˧-ʝi˥ | ʝi˧-zo˧-ho˥-ɲi˩!}\hspace{5pt}\peng{That's how one must do! / That's how it's done!}\hspace{5pt}\pcmn{是应该这样做的!}\hspace{5pt}\pfra{Il faut faire comme ça!}\end{exemple}
\end{entrée}

\begin{entrée}
{‑zo˧α}{}{ⓔ‑zo˧α}\formedesurface{zo˧}\newline
\classe{后缀}\ton{M}\begin{définition}\peng{Obligative.}\end{définition}
\begin{définition}\pcmn{应该、必须}\end{définition}
\begin{définition}\pfra{Obligatif.}\end{définition}
\end{entrée}

\begin{entrée}
{zo˧bæ˩}{}{ⓔzo˧bæ˩}\formedesurface{zo˧bæ˩}\newline
\classe{名词}\ton{L\#}
\paradigme{\pcmn{:} \p{}}
\begin{définition}\peng{Fool, idiot.}\end{définition}
\begin{définition}\pcmn{笨人、傻瓜}\end{définition}
\begin{définition}\pfra{Imbécile, idiot.}\end{définition}
\begin{exemple}\pnru{mɤ˧-zo˧bæ˩!}\hspace{5pt}\peng{No, (you) are not an idiot! (A reassuring answer to someone who deprecates himself as an idiot.)}\hspace{5pt}\pcmn{(你)不是笨蛋!(情景:一个人批评自己是笨蛋,人家安慰他。)}\hspace{5pt}\pfra{(Non, tu n'es) pas idiot(e)! (Propos rassurant adressé à un interlocuteur accablé par ses propres maladresses.)}\end{exemple}
\begin{exemple}\pnru{zo˧bæ˩-mv̩˩bæ˩}\hspace{5pt}\peng{silly people, idiots (of both sexes)}\hspace{5pt}\pcmn{傻瓜们(不分男女)}\hspace{5pt}\pfra{idiots, imbéciles (sans distinction de sexe)}\end{exemple}
\end{entrée}

\begin{entrée}
{zo˩bv̩˥li˩}{}{ⓔzo˩bv̩˥li˩}\formedesurface{zo˩bv̩˥li˩}\newline
\classe{名词}\ton{L.H.L}\begin{définition}\peng{Universe.}\end{définition}
\begin{définition}\pcmn{宇宙}\end{définition}
\begin{définition}\pfra{Univers.}\end{définition}
\begin{exemple}\pnru{sɑ˧ | -zo˩bv̩˥-li˩}\hspace{5pt}\peng{the universe}\hspace{5pt}\pcmn{宇宙}\hspace{5pt}\pfra{l'univers}\end{exemple}
\end{entrée}

\begin{entrée}
{zo˧ɖɯ\#˥}{}{ⓔzo˧ɖɯ\#˥}\formedesurface{zo˧ɖɯ˧}\newline
\classe{名词}\ton{\#H}\begin{définition}\peng{Eldest son.}\end{définition}
\begin{définition}\pcmn{大儿子}\end{définition}
\begin{définition}\pfra{Fils aîné.}\end{définition}
\begin{exemple}\pnru{zo˧ɖɯ˧-mv̩˥ɖɯ˩}\hspace{5pt}\peng{eldest son and eldest daughter}\hspace{5pt}\pcmn{大儿子与大女儿}\hspace{5pt}\pfra{fils et fille aînés}\end{exemple}
\end{entrée}

\begin{entrée}
{zo˧hṽ̩˧˥}{}{ⓔzo˧hṽ̩˧˥}\newline
\classe{名词}
\sens{1}\paradigme{\pcmn{:} \p{}}
\begin{définition}\peng{Son.}\end{définition}
\begin{définition}\pcmn{儿子}\end{définition}
\begin{définition}\pfra{Fils.}\end{définition}
\begin{exemple}\pnru{zo˧hṽ̩˧=ɻæ˥}\hspace{5pt}\peng{the sons}\hspace{5pt}\pcmn{儿子们}\hspace{5pt}\pfra{les fils}\end{exemple}\sens{2}
\begin{définition}\peng{Young chap, young lad, young man.}\end{définition}
\begin{définition}\pcmn{小伙子、 青年男子}\end{définition}
\begin{définition}\pfra{Jeune homme, petit gars.}\end{définition}
\end{entrée}

\begin{entrée}
{zo˧hṽ̩˧-mv̩˥zo˩}{}{ⓔzo˧hṽ̩˧-mv̩˥zo˩}\formedesurface{zo˧hṽ̩˧mv̩˥zo˩}\newline
\classe{名词}\ton{MH\#-}\begin{définition}\peng{Descendants; sons and daughters.}\end{définition}
\begin{définition}\pcmn{后代}\end{définition}
\begin{définition}\pfra{Les descendants.}\end{définition}
\begin{exemple}\pnru{zo˧hṽ̩˧mv̩˥zo˩=ɻæ˩}\hspace{5pt}\peng{\_ |fg{associative}}\hspace{5pt}\pcmn{\_ 联想复数}\hspace{5pt}\pfra{\_ |fg{associatif}}\end{exemple}
\end{entrée}

\begin{entrée}
{zo˧mv̩˥}{}{ⓔzo˧mv̩˥}\formedesurface{zo˧mv̩˥}\newline
\classe{名词}\ton{H\#}
\paradigme{\pcmn{:} \p{}}
\begin{définition}\peng{Child.}\end{définition}
\begin{définition}\pcmn{孩子}\end{définition}
\begin{définition}\pfra{Enfant.}\end{définition}
\begin{exemple}\pnru{zo˧mv̩˥ | æ˧mv̩˥tɕi˩-hĩ˩}\hspace{5pt}\peng{newborn baby, infant}\hspace{5pt}\pcmn{新生婴儿}\hspace{5pt}\pfra{nouveau-né, nourrisson}\end{exemple}
\end{entrée}

\begin{entrée}
{zo˩no˧}{}{ⓔzo˩no˧}\formedesurface{zo˩no˥}\newline
\classe{助词}\ton{LM}\begin{définition}\peng{Now.}\end{définition}
\begin{définition}\pcmn{现在}\end{définition}
\begin{définition}\pfra{Maintenant, actuellement: désigne le moment présent (heure de la journée), comme la période présente (époque contemporaine, par opposition à d'autres époques); également employé comme élément phatique: ‘alors…’; ‘eh bien…’.}\end{définition}
\begin{exemple}\pnru{zo˩no˥ | gɤ˩-ʈi˧!}\hspace{5pt}\peng{She only just woke up! (Context: someone walks into the house in the afternoon, sees a little child playing, and notes: “She has got up!" The child's grandmother answers: “She only just woke up!")}\hspace{5pt}\pcmn{刚起床! / 刚才才起床!}\hspace{5pt}\pfra{Elle vient de se réveiller/de se lever! / Elle s'est réveillée à l'instant! (contexte: quelqu'un entre dans la maison, voit un petit enfant en train de jouer et constate: «Elle est réveillée!» Sa grand-mère répond: «Elle vient de se réveiller!»)}\end{exemple}
\end{entrée}

\begin{entrée}
{zo˩qo˧}{}{ⓔzo˩qo˧}\formedesurface{zo˩qo˥}\newline
\classe{代词}\ton{LM}\begin{définition}\peng{Where.}\end{définition}
\begin{définition}\pcmn{哪里}\end{définition}
\begin{définition}\pfra{Où.}\end{définition}
\begin{exemple}\pnru{no˧ | zo˩qo˧ bi˧?}\hspace{5pt}\peng{Where are you going?}\hspace{5pt}\pcmn{你去哪里?}\hspace{5pt}\pfra{Où tu vas?}\end{exemple}
\begin{exemple}\pnru{zo˩qo˧-ɳɯ˧ | tsʰɯ˩˥?}\hspace{5pt}\peng{Where (are you) coming from?}\hspace{5pt}\pcmn{从哪里来?}\hspace{5pt}\pfra{D'où (tu) viens?}\end{exemple}
\begin{exemple}\pnru{no˧ | hɑ˧ | zo˩qo˧ dzɯ˧-bi˧-pi˧, | ɖɯ˧-bæ˧ lɑ˧ ɲi˥!}\hspace{5pt}\peng{No matter which one you choose: they're all the same! (Context: discussing the restaurant scene in Yongning; in the speaker's view, the newly opened restaurants all share the same qualities and shortcomings, for instance concerning hygiene.)}\hspace{5pt}\pcmn{无论你到哪里去吃,都一样!(情景:新开的饭馆)}\hspace{5pt}\pfra{Peu importe où tu vas manger, c'est partout pareil! (contexte: au sujet des restaurants récemment ouverts à Yongning, qui partagent les mêmes qualités et défauts dont des problèmes d'hygiène)}\end{exemple}
\begin{exemple}\pnru{zo˩qo˧ tʰv̩˧?}\hspace{5pt}\peng{Where are you? (Typical question when calling someone on their mobile phone)}\hspace{5pt}\pcmn{你到哪里了?(打手机)}\hspace{5pt}\pfra{Où tu es? (question typique quand on appelle quelqu'un sur son téléphone portable)}\end{exemple}
\end{entrée}

\begin{entrée}
{zo˧ʂv̩˧kʰv̩˥-ɲi˩tʰv̩˩}{}{ⓔzo˧ʂv̩˧kʰv̩˥-ɲi˩tʰv̩˩}\formedesurface{zo˧ʂv̩˧kʰv̩˥ɲi˩tʰv̩˩}\newline
\classe{名词}\ton{H\#-}\begin{définition}\peng{Orphan.}\end{définition}
\begin{définition}\pcmn{孤儿}\end{définition}
\begin{définition}\pfra{Orphelin.}\end{définition}
\end{entrée}

\begin{entrée}
{zo˧tv̩\#˥}{}{ⓔzo˧tv̩\#˥}\formedesurface{zo˧tv̩˧}\newline
\classe{名词}\ton{\#H}\begin{définition}\peng{Only son.}\end{définition}
\begin{définition}\pcmn{独生子,独生男孩}\end{définition}
\begin{définition}\pfra{Fils unique.}\end{définition}
\begin{exemple}\pnru{zo˧tv̩˧ ɖɯ˧-v̩˧-lɑ˧ dʑo˧˥!}\hspace{5pt}\peng{(She) just has an only son!}\hspace{5pt}\pcmn{(她)只有一个独生男孩子!}\hspace{5pt}\pfra{(elle) n'a qu'un fils unique!}\end{exemple}
\begin{exemple}\pnru{ʂɯ˧-ɬi˧mi˧, | zo˧tv̩˧ ʐɤ˥-tʰɑ˩-se˩!}\hspace{5pt}\peng{“In the seventh month, let not an only son take the road!" (The seventh month is the peak of the rainy season; it was considered as a wrong time for long travels.)}\hspace{5pt}\pcmn{“七月份,独生子不要上路!”(七月份是大雨季,摩梭人认为七月份的路最不安全:有生命危险)}\hspace{5pt}\pfra{«Au septième mois, un fils unique ne doit pas aller par les chemins!» (Le septième mois, au plus fort des grandes pluies, était considéré comme un mois défavorable pour voyager; les voyageurs risquaient d'y rester.)}\end{exemple}
\end{entrée}

\begin{entrée}
{zo˧tv̩˧-mv̩˥tv̩˩}{}{ⓔzo˧tv̩˧-mv̩˥tv̩˩}\formedesurface{zo˧tv̩˧mv̩˥tv̩˩}\newline
\classe{名词}\ton{\#H-}\begin{définition}\peng{Only child (boy or girl).}\end{définition}
\begin{définition}\pcmn{独生子(男女通用)}\end{définition}
\begin{définition}\pfra{Enfant unique (fils unique ou fille unique).}\end{définition}
\end{entrée}

\begin{entrée}
{zo˧tʰi˧}{}{ⓔzo˧tʰi˧}\formedesurface{zo˧tʰi˧}\newline
\classe{形容词}\ton{M}\begin{définition}\peng{Intelligent.}\end{définition}
\begin{définition}\pcmn{聪明}\end{définition}
\begin{définition}\pfra{Intelligent.}\end{définition}
\begin{exemple}\pnru{ʈʂʰɯ˧ | zo˧tʰi˧ | ʐwæ˩˥!}\hspace{5pt}\peng{He is very clever!}\hspace{5pt}\pcmn{他很聪明!}\hspace{5pt}\pfra{il est très intelligent!}\end{exemple}
\begin{exemple}\pnru{ʈʂʰɯ˧ | mɤ˧-tʰi˧!}\hspace{5pt}\peng{He is not clever!}\hspace{5pt}\pcmn{他不聪明!}\hspace{5pt}\pfra{il n'est pas intelligent/il n'est pas bien malin! (on ne peut dire: /*mɤ˧-zo˧tʰi˧/)}\end{exemple}
\end{entrée}

\begin{entrée}
{zo˧tʰi˧}{}{ⓔzo˧tʰi˧}\formedesurface{zo˧tʰi˧}\newline
\classe{名词}\ton{M}\begin{définition}\peng{Intelligent person.}\end{définition}
\begin{définition}\pcmn{聪明的人}\end{définition}
\begin{définition}\pfra{Personne intelligente.}\end{définition}
\begin{exemple}\pnru{zo˧tʰi˧ ɖɯ˧-v̩˧}\hspace{5pt}\peng{an intelligent person}\hspace{5pt}\pcmn{一个聪明的人}\hspace{5pt}\pfra{une personne intelligente}\end{exemple}
\end{entrée}

\begin{entrée}
{zo˧tɕi˥}{}{ⓔzo˧tɕi˥}\formedesurface{zo˧tɕi˥}\newline
\classe{名词}\ton{H\#}\begin{définition}\peng{Youngest son.}\end{définition}
\begin{définition}\pcmn{最小的儿子}\end{définition}
\begin{définition}\pfra{Fils dernier-né, benjamin.}\end{définition}
\begin{exemple}\pnru{zo˧tɕi˥-mv̩˩tɕi˩}\hspace{5pt}\peng{youngest son and youngest daughter}\hspace{5pt}\pcmn{最小的儿子与女儿}\hspace{5pt}\pfra{le benjamin et la benjamine: les plus jeunes enfants}\end{exemple}
\end{entrée}

\begin{entrée}
{zo˧∼zo˧-mv̩˧∼mv̩˥}{}{ⓔzo˧∼zo˧-mv̩˧∼mv̩˥}\formedesurface{zo˧zo˧mv̩˧mv̩˥}\newline
\classe{名词}\ton{H\#}
\paradigme{\pcmn{:} \p{}}
\begin{définition}\peng{Thing, thingummy.}\end{définition}
\begin{définition}\pcmn{东西}\end{définition}
\begin{définition}\pfra{Truc, bidule.}\end{définition}
\end{entrée}

\begin{entrée}
{zo˧ʐɤ\#˥}{}{ⓔzo˧ʐɤ\#˥}\formedesurface{zo˧ʐɤ˧}\newline
\classe{名词}\ton{\#H}\begin{définition}\peng{Adoptive son, foster son.}\end{définition}
\begin{définition}\pcmn{义子}\end{définition}
\begin{définition}\pfra{Fils adoptif.}\end{définition}
\end{entrée}

\begin{entrée}
{zɯ˥}{}{ⓔzɯ˥}\formedesurface{zɯ˧}\newline
\classe{名词}\ton{\#H}
\paradigme{\pcmn{:} \p{}}
\begin{définition}\peng{Grass.}\end{définition}
\begin{définition}\pcmn{草}\end{définition}
\begin{définition}\pfra{Herbe.}\end{définition}
\end{entrée}

\begin{entrée}
{zɯ˧}{}{ⓔzɯ˧}\formedesurface{zɯ˧}\newline
\classe{名词}\ton{M}\begin{définition}\peng{Life, existence.}\end{définition}
\begin{définition}\pcmn{生命}\end{définition}
\begin{définition}\pfra{Vie, existence.}\end{définition}
\begin{exemple}\pnru{zɯ˧ʂæ\#˥}\hspace{5pt}\peng{long life}\hspace{5pt}\pcmn{长命、长的人生}\hspace{5pt}\pfra{longue vie}\end{exemple}
\begin{exemple}\pnru{zɯ˧ ʂæ˧ | hɑ̃˧-ʝi˧-kʰɯ˩!}\hspace{5pt}\peng{May you have a long life!}\hspace{5pt}\pcmn{祝你长寿!}\hspace{5pt}\pfra{Puisses-tu avoir longue vie! (bénédiction)}\end{exemple}
\begin{exemple}\pnru{zɯ˧ɖæ\#˥}\hspace{5pt}\peng{short life}\hspace{5pt}\pcmn{短命}\hspace{5pt}\pfra{courte vie}\end{exemple}
\end{entrée}

\begin{entrée}
{zɯ˧β}{}{ⓔzɯ˧β}\formedesurface{ɖɯ˧ zɯ˧}\newline
\classe{量词}\ton{Mβ}\begin{définition}\peng{Self-classifier for life, existence.}\end{définition}
\begin{définition}\pcmn{量词:辈子}\end{définition}
\begin{définition}\pfra{Auto-classificateur de la vie, de l'existence entière.}\end{définition}
\begin{exemple}\pnru{ɖɯ˧-zɯ˧}\hspace{5pt}\peng{all of (one's) life, a lifetime}\hspace{5pt}\pcmn{一辈子(的时间)}\hspace{5pt}\pfra{toute la vie}\end{exemple}
\end{entrée}

\begin{entrée}
{zɯ˧hṽ̩˩}{}{ⓔzɯ˧hṽ̩˩}\formedesurface{zɯ˧hṽ̩˩}\newline
\classe{形容词}\ton{L\#}\begin{définition}\peng{Green.}\end{définition}
\begin{définition}\pcmn{绿(布料、线)}\end{définition}
\begin{définition}\pfra{Vert.}\end{définition}
\begin{exemple}\pnru{zɯ˧hṽ̩˩-ni˩gv̩˩}\hspace{5pt}\peng{green}\hspace{5pt}\pcmn{绿}\hspace{5pt}\pfra{de couleur verte}\end{exemple}
\begin{exemple}\pnru{zɯ˧hṽ̩˩ | ∼zɯ˧hṽ̩˩-ni˩gv̩˩}\hspace{5pt}\peng{all green, green all over}\hspace{5pt}\pcmn{全绿}\hspace{5pt}\pfra{tout vert}\end{exemple}
\end{entrée}

\begin{entrée}
{zɯ˧pv̩˩}{}{ⓔzɯ˧pv̩˩}\formedesurface{zɯ˧pv̩˩}\newline
\classe{名词}\ton{L\#}
\paradigme{\pcmn{:} \p{}}
\begin{définition}\peng{Hay, dry grass.}\end{définition}
\begin{définition}\pcmn{干草}\end{définition}
\begin{définition}\pfra{Foin; s'emploie aussi parfois pour désigner la paille: dans la maison, on ne stocke que de la paille de riz, pas de foin; l'herbe cueillie verte puis séchée (foin) n'est pas entreposée, mais aussitôt donnée aux animaux.}\end{définition}
\end{entrée}

\begin{entrée}
{zɯ˧-qʰɑ˧mi\#˥}{}{ⓔzɯ˧-qʰɑ˧mi\#˥}\formedesurface{zɯ˧-qʰɑ˧mi˧}\newline
\classe{名词}\ton{\#H}
\paradigme{\pcmn{:} \p{}}
\begin{définition}\peng{Sabai grass, |\stylefi{Eulaliopsis binata (Retz.) C. E. Hubb.}.}\end{définition}
\begin{définition}\pcmn{蓑草、山草、山草根、龙须草、山茅草、羊草、拟金茅}\end{définition}
\begin{définition}\pfra{|\stylefi{Eulaliopsis binata (Retz.) C. E. Hubb.}, herbe sauvage. L'herbe n'est pas prisée du bétail, non plus que ses racines, et n'est jamais consommée par les humains. Les racines de cette herbe sont utilisées dans les rituels: elles ont une odeur forte à la combustion.}\end{définition}
\end{entrée}

\begin{entrée}
{zɯ˧ɻ̍\#˥}{}{ⓔzɯ˧ɻ̍\#˥}\formedesurface{zɯ˧ɻ̍˧}\newline
\classe{名词}\ton{\#H}
\paradigme{\pcmn{:} \p{}}
\begin{définition}\peng{Cheek.}\end{définition}
\begin{définition}\pcmn{腮、腮帮子}\end{définition}
\begin{définition}\pfra{Joue (partie basse, en-dessous des pommettes; vers l'articulation des deux mâchoires).}\end{définition}
\begin{exemple}\pnru{zɯ˧ɻ̍˧ qʰwæ˩}\hspace{5pt}\peng{to slap/smack someone's cheek}\hspace{5pt}\pcmn{掌掴、打嘴巴}\hspace{5pt}\pfra{gifler}\end{exemple}
\end{entrée}

\begin{entrée}
{zɯ˧∼zɯ˧}{}{ⓔzɯ˧∼zɯ˧}\formedesurface{zɯ˧zɯ˧}\newline
\classe{名词}\ton{M}
\paradigme{\pcmn{:} \p{}}
\begin{définition}\peng{Life, existence.}\end{définition}
\begin{définition}\pcmn{生命}\end{définition}
\begin{définition}\pfra{Vie, existence.}\end{définition}
\begin{exemple}\pnru{hĩ˧-zɯ˧∼zɯ˥\$}\hspace{5pt}\peng{human life}\hspace{5pt}\pcmn{人生}\hspace{5pt}\pfra{la vie humaine}\end{exemple}
\begin{exemple}\pnru{hĩ˧ zɯ˧ | ʂæ˧ | ʐwæ˩˥}\hspace{5pt}\peng{a very long life / life is very long}\hspace{5pt}\pcmn{很长的人生 / 人生很长}\hspace{5pt}\pfra{une très longue vie / la vie est longue}\end{exemple}
\end{entrée}

\begin{entrée}
{zɯ˩∼zɯ˩}{}{ⓔzɯ˩∼zɯ˩}\formedesurface{zɯ˩zɯ˩˥}\newline
\classe{动词}\ton{L}\begin{définition}\peng{To be numb.}\end{définition}
\begin{définition}\pcmn{麻木}\end{définition}
\begin{définition}\pfra{Être engourdi.}\end{définition}
\begin{exemple}\pnru{gv̩˧dv̩˧gv̩˧mi˧ | zɯ˩∼zɯ˩˥}\hspace{5pt}\peng{to be numb (whole body)}\hspace{5pt}\pcmn{身体麻木、全身麻木}\hspace{5pt}\pfra{avoir le corps engourdi}\end{exemple}
\begin{exemple}\pnru{gv̩˧mi˧ | zɯ˩∼zɯ˩˥}\hspace{5pt}\peng{to be numb (whole body)}\hspace{5pt}\pcmn{身体麻木、全身麻木}\hspace{5pt}\pfra{avoir le corps engourdi}\end{exemple}
\begin{exemple}\pnru{tʰi˧-zɯ˩∼zɯ˩}\hspace{5pt}\peng{|fg{dur} |fg{red}}\hspace{5pt}\pcmn{|fg{dur} |fg{red}}\hspace{5pt}\pfra{|fg{dur} |fg{red}}\end{exemple}
\end{entrée}

\newpage\caractère{ʐ}

\begin{entrée}
{ʐ}{}{ⓔʐ}\formedesurface{ʐ!}\newline
\classe{}\ton{--}\begin{définition}\peng{Rumbling sound of heavy loads carried over a wooden floor, of lorries… Brrroom!}\end{définition}
\begin{définition}\pcmn{形声词:轰隆隆!(拉很重的物品在地板上的隆隆声,卡车的隆隆声)}\end{définition}
\begin{définition}\pfra{Bruit de grondement des grosses charges qu'on traîne sur le sol, des moteurs de camions: Brrroum!}\end{définition}
\end{entrée}

\begin{entrée}
{ʐæ˧}{}{ⓔʐæ˧}\formedesurface{ʐæ˧}\newline
\classe{形容词}\ton{M}\begin{définition}\peng{Tall and big; great; impressive.}\end{définition}
\begin{définition}\pcmn{高大}\end{définition}
\begin{définition}\pfra{Grand et fort, massif, baraqué.}\end{définition}
\begin{exemple}\pnru{ʐæ˧-ni˩gv̩˩}\hspace{5pt}\peng{tall and big; great; impressive}\hspace{5pt}\pcmn{高大}\hspace{5pt}\pfra{grand et fort}\end{exemple}
\begin{exemple}\pnru{hĩ˧ | ʈʂʰɯ˧-v̩˧, | ʐæ˧-ni˩gv̩˩!}\hspace{5pt}\peng{This person looks impressive / tall and big!}\hspace{5pt}\pcmn{这人很高大!}\hspace{5pt}\pfra{Elle/il est grand(e) et fort(e) / impressionnant(e)!}\end{exemple}
\begin{exemple}\pnru{ʐæ˧ni˩ | mɤ˧-gv̩˧}\hspace{5pt}\peng{not tall, not impressive, not great-looking}\hspace{5pt}\pcmn{个子不高}\hspace{5pt}\pfra{pas bien grand (en taille), pas bien impressionnant}\end{exemple}
\begin{exemple}\pnru{ʐæ˧ | ʐwæ˩˥}\hspace{5pt}\peng{very tall and big}\hspace{5pt}\pcmn{很高大}\hspace{5pt}\pfra{très grand et fort}\end{exemple}
\end{entrée}

\begin{entrée}
{ʐæ˧α}{}{ⓔʐæ˧α}\formedesurface{ʐæ˧}\newline
\classe{动词}
\sens{1}
\begin{définition}\peng{To laugh.}\end{définition}
\begin{définition}\pcmn{笑}\end{définition}
\begin{définition}\pfra{Rire.}\end{définition}
\begin{exemple}\pnru{zo˧hṽ̩˥ | hĩ˧ ʐæ˧∼ʐæ˥-kʰɯ˩}\hspace{5pt}\peng{The kids make people laugh}\hspace{5pt}\pcmn{孩子们把大家逗笑了。}\hspace{5pt}\pfra{les enfants taquinent les gens, les font rire}\end{exemple}
\begin{exemple}\pnru{hĩ˧ | ʐæ˧∼ʐæ˥ kʰɯ˩}\hspace{5pt}\peng{to make people laugh, to amuse people}\hspace{5pt}\pcmn{让大家笑一笑}\hspace{5pt}\pfra{faire rire les gens, amuser les gens, faire rire le public}\end{exemple}
\begin{exemple}\pnru{ʐæ˧∼ʐæ˩-di˩}\hspace{5pt}\peng{jokes, funny talk}\hspace{5pt}\pcmn{笑话,好笑的话}\hspace{5pt}\pfra{plaisanteries, blagues}\end{exemple}\sens{2}
\begin{définition}\peng{To laugh at; to be impertinent; to deride, to make fun of (people).}\end{définition}
\begin{définition}\pcmn{嘲笑别人、出言不逊}\end{définition}
\begin{définition}\pfra{Être impertinent, déranger, se moquer du monde.}\end{définition}
\begin{exemple}\pnru{le˧-ʐæ˧-ze˧}\hspace{5pt}\peng{|fg{accomp} \_ |fg{pfv}}\hspace{5pt}\pcmn{出言不逊了}\hspace{5pt}\pfra{|fg{accomp} \_ |fg{pfv}}\end{exemple}
\begin{exemple}\pnru{le˧-ʐæ˥∼ʐæ˩}\hspace{5pt}\peng{|fg{red}}\hspace{5pt}\pcmn{笑一笑(别人)}\hspace{5pt}\pfra{|fg{red}}\end{exemple}
\begin{exemple}\pnru{hĩ˧ ʐæ˩}\hspace{5pt}\peng{to make fun of other people}\hspace{5pt}\pcmn{嘲笑人家}\hspace{5pt}\pfra{être impertinent avec les gens, déranger les gens}\end{exemple}
\begin{exemple}\pnru{le˧-ʐæ˥∼ʐæ˩-ze˩}\hspace{5pt}\peng{|fg{red} |fg{pfv}}\hspace{5pt}\pcmn{嘲笑了}\hspace{5pt}\pfra{|fg{red} |fg{pfv}}\end{exemple}
\end{entrée}

\begin{entrée}
{ʐæ˩˥}{}{ⓔʐæ˩˥}\formedesurface{ʐæ˩˥}\newline
\classe{名词}\ton{LH}
\paradigme{\pcmn{:} \p{}}
\begin{définition}\peng{Leopard, panther (note: these two terms are homonymous in English).}\end{définition}
\begin{définition}\pcmn{豹子}\end{définition}
\begin{définition}\pfra{Léopard, panthère (note: ces deux termes sont homonymes en français).}\end{définition}
\begin{exemple}\pnru{ʐæ˩ dzɯ˧-ze˩}\hspace{5pt}\peng{…ate (a/the) leopard}\hspace{5pt}\pcmn{吃了豹子}\hspace{5pt}\pfra{…a mangé (un/le) léopard}\end{exemple}
\begin{exemple}\pnru{ʐæ˩ hwæ˧-ze˩}\hspace{5pt}\peng{…bought (a/the) leopard}\hspace{5pt}\pcmn{买了豹子}\hspace{5pt}\pfra{…a acheté (un/le) léopard}\end{exemple}
\end{entrée}

\begin{entrée}
{ʐæ˩β}{}{ⓔʐæ˩β}\formedesurface{ʐæ˩˥}\newline
\classe{动词}\ton{Lβ}\begin{définition}\peng{To mix.}\end{définition}
\begin{définition}\pcmn{搅拌合混}\end{définition}
\begin{définition}\pfra{Mélanger, tourner (un mélange, une préparation).}\end{définition}
\begin{exemple}\pnru{le˧-ʐæ˧∼ʐæ˥}\hspace{5pt}\peng{|fg{accomp} \_ |fg{red}}\hspace{5pt}\pcmn{搅拌}\hspace{5pt}\pfra{|fg{accomp} \_ |fg{red}}\end{exemple}
\end{entrée}

\begin{entrée}
{ʐæ˩mi\#˥}{}{ⓔʐæ˩mi\#˥}\formedesurface{ʐæ˩mi˥}\newline
\classe{名词}\ton{LM+\#H}
\paradigme{\pcmn{:} \p{}}
\begin{définition}\peng{Female leopard.}\end{définition}
\begin{définition}\pcmn{母豹子}\end{définition}
\begin{définition}\pfra{Léopard femelle.}\end{définition}
\begin{exemple}\pnru{ʐæ˩mi˧-ʐæ˥zo˩}\hspace{5pt}\peng{female leopard and male leopard}\hspace{5pt}\pcmn{母豹子与公豹子}\hspace{5pt}\pfra{léopard femelle et léopard mâle}\end{exemple}
\end{entrée}

\begin{entrée}
{ʐæ˩pʰv̩˧}{}{ⓔʐæ˩pʰv̩˧}\formedesurface{ʐæ˩pʰv̩˥}\newline
\classe{名词}\ton{LM}
\paradigme{\pcmn{:} \p{}}
\begin{définition}\peng{Male leopard.}\end{définition}
\begin{définition}\pcmn{公豹子}\end{définition}
\begin{définition}\pfra{Léopard mâle.}\end{définition}
\begin{exemple}\pnru{ʐæ˩pʰv̩˧-ʐæ˩mi˩}\hspace{5pt}\peng{male leopard and female leopard}\hspace{5pt}\pcmn{公豹子与母豹子}\hspace{5pt}\pfra{léopard mâle et léopard femelle}\end{exemple}
\end{entrée}

\begin{entrée}
{ʐæ˩sɯ˩}{}{ⓔʐæ˩sɯ˩}\formedesurface{ʐæ˩sɯ˩˥}\newline
\classe{名词}\ton{L}
\paradigme{\pcmn{:} \p{}}
\begin{définition}\peng{Rough felt made only of sheep wool. One drapes it over one's shoulders as an outdoor protection from the cold.}\end{définition}
\begin{définition}\pcmn{披毡}\end{définition}
\begin{définition}\pfra{Feutre grossier, fait uniquement de laine de mouton, dont on se drape en extérieur pour se protéger du froid.}\end{définition}
\end{entrée}

\begin{entrée}
{ʐæ˩sɯ˩-kʰwæ˩ɻæ˧}{}{ⓔʐæ˩sɯ˩-kʰwæ˩ɻæ˧}\formedesurface{ʐæ˩sɯ˩kʰwæ˩ɻæ˥}\newline
\classe{名词}
\paradigme{\pcmn{:} \p{}}
\begin{définition}\peng{Felt mat.}\end{définition}
\begin{définition}\pcmn{毡子(真正的毡子)做的垫子}\end{définition}
\begin{définition}\pfra{Natte en feutre. Le terme désigne spécifiquement les nattes/matelas/tissus en feutre véritable, par opposition avec le sens étendu que peut avoir \stylefv{/kʰwæ}˧ɻæ\#˥/.}\end{définition}
\end{entrée}

\begin{entrée}
{ʐæ˩ʂæ˧}{}{ⓔʐæ˩ʂæ˧}\formedesurface{ʐæ˩ʂæ˥}\newline
\classe{形容词}\ton{LM}\begin{définition}\peng{Far, distant.}\end{définition}
\begin{définition}\pcmn{远}\end{définition}
\begin{définition}\pfra{Loin, lointain.}\end{définition}
\begin{exemple}\pnru{no˧ | ʈʂʰɯ˧ | ə˩-ʐæ˥ʂæ˩? | dʑɤ˩kʰwɤ˧ ə˩-di˩? | - dʑɤ˩˥ | dʑɤ˩kʰwɤ˧ mɤ˧-di˥! | mɤ˧-ʐæ˩ʂæ˩!}\hspace{5pt}\peng{Are you distant from him? Is there distance (between you)? - There is not much distance to speak of! We are not distant! (=we are close friends)}\hspace{5pt}\pcmn{你们很熟吗? - 不很熟!}\hspace{5pt}\pfra{tu es loin de lui? Y a-t-il de la distance entre vous? (=vous êtes proches/intimes, ou pas?) - Non, il n'y a guère de distance! Nous ne somme pas éloignés!}\end{exemple}
\end{entrée}

\begin{entrée}
{ʐæ˩tsɯ˧˥}{}{ⓔʐæ˩tsɯ˧˥}\formedesurface{ʐæ˩tsɯ˧˥}\newline
\classe{名词}\ton{LM+MH\#}
\paradigme{\pcmn{:} \p{}}
\begin{définition}\peng{Path, trail.}\end{définition}
\begin{définition}\pcmn{小路、径道}\end{définition}
\begin{définition}\pfra{Sentier, petit chemin.}\end{définition}
\begin{exemple}\pnru{ʐæ˩tsɯ˧-ʐɤ˥mi˩}\hspace{5pt}\peng{small trail}\hspace{5pt}\pcmn{径道}\hspace{5pt}\pfra{chemin de traverse, raccourci}\end{exemple}
\end{entrée}

\begin{entrée}
{ʐæ˧v̩˩-tʰv̩˩}{}{ⓔʐæ˧v̩˩-tʰv̩˩}\formedesurface{ʐæ˧v̩˩tʰv̩˩}\newline
\classe{动词}\ton{L\#-}\begin{définition}\peng{To joke, to crack a joke.}\end{définition}
\begin{définition}\pcmn{开玩笑}\end{définition}
\begin{définition}\pfra{Blaguer, faire une blague, faire une plaisanterie.}\end{définition}
\begin{exemple}\pnru{ʐæ˧v̩˩-tʰv̩˩ | ʐwæ˩˥}\hspace{5pt}\peng{to crack jokes all the time, to make lots of jokes}\hspace{5pt}\pcmn{开很多玩笑、一直开玩笑}\hspace{5pt}\pfra{plaisanter follement, rire beaucoup}\end{exemple}
\begin{exemple}\pnru{ʐæ˧v̩˩-tʰv̩˩-hĩ˩ ʐwɤ˩}\hspace{5pt}\peng{to crack a joke}\hspace{5pt}\pcmn{开个玩笑}\hspace{5pt}\pfra{lancer une blague, dire une plaisanterie}\end{exemple}
\end{entrée}

\begin{entrée}
{ʐæ˩zo\#˥}{}{ⓔʐæ˩zo\#˥}\formedesurface{ʐæ˩zo˥}\newline
\classe{名词}\ton{LM+\#H}
\paradigme{\pcmn{:} \p{}}
\begin{définition}\peng{Leopard cub, baby leopard.}\end{définition}
\begin{définition}\pcmn{小豹子}\end{définition}
\begin{définition}\pfra{Bébé léopard, petit léopard.}\end{définition}
\begin{exemple}\pnru{ʐæ˩zo˧-ʐæ˥mi˩}\hspace{5pt}\peng{baby leopard and female leopard}\hspace{5pt}\pcmn{小豹子与母豹子}\hspace{5pt}\pfra{petit léopard et léopard femelle}\end{exemple}
\end{entrée}

\begin{entrée}
{ʐe˥}{}{ⓔʐe˥}\formedesurface{ɖɯ˧ ʐe˥}\newline
\classe{量词}\ton{Hα}\begin{définition}\peng{Classifier for quarters of preserved meat.}\end{définition}
\begin{définition}\pcmn{量词:熏肉(一块)}\end{définition}
\begin{définition}\pfra{Classificateur des morceaux de viande conservée.}\end{définition}
\end{entrée}

\begin{entrée}
{ʐe˥}{₁}{ⓔʐe˥ⓗ1}\formedesurface{ʐe˧}\newline
\classe{名词}\ton{\#H}
1
\paradigme{\pcmn{:} \p{}}
\begin{définition}\peng{Arrow.}\end{définition}
\begin{définition}\pcmn{箭}\end{définition}
\begin{définition}\pfra{Flèche.}\end{définition}
\begin{exemple}\pnru{ʐe˧ɻ̃˧ | ɖɯ˧-kʰɯ˩}\hspace{5pt}\peng{an arrow; also, metaphorically: a family, a lineage}\hspace{5pt}\pcmn{一枝箭。也来指一个家庭}\hspace{5pt}\pfra{une flèche; désigne aussi, de façon métaphorique, une lignée/une famille}\end{exemple}
\end{entrée}

\begin{entrée}
{ʐe˥}{₂}{ⓔʐe˥ⓗ2}\formedesurface{ʐe˧}\newline
\classe{名词}\ton{\#H}
2\begin{définition}\peng{Rainy season (summer and autumn: from the 3rd to the 8th month of the lunar calendar).}\end{définition}
\begin{définition}\pcmn{雨季(夏天至秋天:三月份至八月份)}\end{définition}
\begin{définition}\pfra{Saison des pluies (été et automne: du 3e au 8e mois du calendrier lunaire).}\end{définition}
\end{entrée}

\begin{entrée}
{ʐe˧ʈæ˥-ɬi˩}{}{ⓔʐe˧ʈæ˥-ɬi˩}\formedesurface{ʐe˧ʈæ˥ɬi˩}\newline
\classe{名词}\ton{H\#-L}\begin{définition}\peng{11th month.}\end{définition}
\begin{définition}\pcmn{十一月}\end{définition}
\begin{définition}\pfra{11e mois.}\end{définition}
\end{entrée}

\begin{entrée}
{ʐe˧v̩\#˥}{}{ⓔʐe˧v̩\#˥}\formedesurface{ʐe˧v̩˧}\newline
\classe{名词}\ton{\#H}
\paradigme{\pcmn{:} \p{}}
\begin{définition}\peng{Castrated ox, neutered ox, steer.}\end{définition}
\begin{définition}\pcmn{阉牛}\end{définition}
\begin{définition}\pfra{Taureau castré.}\end{définition}
\end{entrée}

\begin{entrée}
{ʐe˧zo\#˥}{}{ⓔʐe˧zo\#˥}\formedesurface{ʐe˧zo˧}\newline
\classe{名词}\ton{\#H}\begin{définition}\peng{Arrow.}\end{définition}
\begin{définition}\pcmn{箭}\end{définition}
\begin{définition}\pfra{Flèche.}\end{définition}
\begin{exemple}\pnru{ʐe˧zo˧ | ɖɯ˧-kʰɯ˩}\hspace{5pt}\peng{one arrow}\hspace{5pt}\pcmn{一枝箭}\hspace{5pt}\pfra{une flèche}\end{exemple}
\end{entrée}

\begin{entrée}
{ʐe˩ʐe˧-bæ˩bæ˩}{}{ⓔʐe˩ʐe˧-bæ˩bæ˩}\formedesurface{ʐe˩ʐe˧bæ˩bæ˩}\newline
\classe{名词}\ton{LM-L}\begin{définition}\peng{Wild cotton (literally: “Westerners' flower").}\end{définition}
\begin{définition}\pcmn{野棉花(直译:‘洋人花’)}\end{définition}
\begin{définition}\pfra{Coton sauvage; littéralement «la fleur des Occidentaux».}\end{définition}
\end{entrée}

\begin{entrée}
{ʐe˩ʐe˧-læ˧tsɯ˥}{}{ⓔʐe˩ʐe˧-læ˧tsɯ˥}\formedesurface{ʐe˩ʐe˧læ˧tsɯ˥}\newline
\classe{名词}\ton{LM-H\#}\begin{définition}\peng{One of the three main types of plants used for pig fodder.}\end{définition}
\begin{définition}\pcmn{喂猪的牧草}\end{définition}
\begin{définition}\pfra{Sorte de fourrage pour les cochons (il y en a trois en tout).}\end{définition}
\end{entrée}

\begin{entrée}
{ʐɤ˧β}{}{ⓔʐɤ˧β}\formedesurface{ʐɤ˧}\newline
\classe{动词}\ton{Mβ}\begin{définition}\peng{To raise (animals, or children); to care for (the elderly).}\end{définition}
\begin{définition}\pcmn{饲养(动物)、养(孩子)、管(老人)}\end{définition}
\begin{définition}\pfra{Élever (des enfants ou des animaux); s'occuper de (personnes âgées).}\end{définition}
\begin{exemple}\pnru{bo˩ ʐɤ˧}\hspace{5pt}\peng{to raise pigs}\hspace{5pt}\pcmn{养猪}\hspace{5pt}\pfra{élever des cochons}\end{exemple}
\begin{exemple}\pnru{ʐwæ˧zo˧ ʐɤ˧}\hspace{5pt}\peng{to raise colts}\hspace{5pt}\pcmn{养小马}\hspace{5pt}\pfra{élever des poulains}\end{exemple}
\end{entrée}

\begin{entrée}
{ʐɤ˩˧}{}{ⓔʐɤ˩˧}\formedesurface{ʐɤ˩˥}\newline
\classe{名词}\ton{LM}
\paradigme{\pcmn{:} \p{}}
\begin{définition}\peng{Road (monosyllable).}\end{définition}
\begin{définition}\pcmn{路(单音节)}\end{définition}
\begin{définition}\pfra{Route (monosyllabe).}\end{définition}
\begin{exemple}\pnru{ʐɤ˩mi˩-qo˥}\hspace{5pt}\peng{on the road, on the way}\hspace{5pt}\pcmn{路上}\hspace{5pt}\pfra{sur le chemin}\end{exemple}
\begin{exemple}\pnru{ʐɤ˩mi˩-qo˥, | hĩ˧ se˧! |}\hspace{5pt}\peng{People are walking on the road/path!}\hspace{5pt}\pcmn{路上有人走!}\hspace{5pt}\pfra{il y a des gens qui passent sur le chemin!}\end{exemple}
\begin{exemple}\pnru{ʐɤ˩ se˩-zo˩˥}\hspace{5pt}\peng{traveller, person who travels; specifically: person who does commerce by caravans}\hspace{5pt}\pcmn{旅人,特别指走马帮的商人}\hspace{5pt}\pfra{voyageur, homme qui voyage; spécifiquement: personne partant faire du commerce en caravane}\end{exemple}
\end{entrée}

\begin{entrée}
{ʐɤ˩α}{}{ⓔʐɤ˩α}\formedesurface{ʐɤ˩˥}\newline
\classe{形容词}\ton{Lα}\begin{définition}\peng{Clean.}\end{définition}
\begin{définition}\pcmn{干净}\end{définition}
\begin{définition}\pfra{Propre.}\end{définition}
\begin{exemple}\pnru{ʈʂʰɯ˧ | ʐɤ˩-hĩ˩ ɲi˥}\hspace{5pt}\peng{It is clean}\hspace{5pt}\pcmn{这是干净的}\hspace{5pt}\pfra{c'est propre}\end{exemple}
\begin{exemple}\pnru{mɤ˧-ʐɤ˩}\hspace{5pt}\peng{not clean, dirty}\hspace{5pt}\pcmn{不干净}\hspace{5pt}\pfra{crasseux, dégoûtant (vêtements, nourriture…)}\end{exemple}
\end{entrée}

\begin{entrée}
{ʐɤ˩γ}{}{ⓔʐɤ˩γ}\formedesurface{ɖɯ˧ ʐɤ˩}\newline
\classe{量词}\ton{Lγ}\begin{définition}\peng{Classifier for lines/patterns (in weaving, drawing, painting…).}\end{définition}
\begin{définition}\pcmn{量词:图案(画画或织布)(一个)}\end{définition}
\begin{définition}\pfra{Classificateur des motifs, tracés, lignes, dans les dessins, peintures et tissages.}\end{définition}
\end{entrée}

\begin{entrée}
{ʐɤ˩mi˩}{}{ⓔʐɤ˩mi˩}\formedesurface{ʐɤ˩mi˩˥}\newline
\classe{名词}\ton{L}
\paradigme{\pcmn{:} \p{}}
\begin{définition}\peng{Road.}\end{définition}
\begin{définition}\pcmn{路}\end{définition}
\begin{définition}\pfra{Route.}\end{définition}
\begin{exemple}\pnru{hĩ˧ | ɖɯ˧-v̩˧∼ɖɯ˧-v̩˧ | le˧-se˥, | ʐɤ˩mi˩ tsɤ˩˥!}\hspace{5pt}\peng{People walk, one after the other, and they create a path! (Context: when people go to fell trees on the mountain, where there was no path before, their passage open a new path, whose traces remain visible and may become a customary path.)}\hspace{5pt}\pcmn{路是人走出来的!}\hspace{5pt}\pfra{Contexte: on va couper du bois en montagne, à un endroit où il n'y a pas de chemin. Les gens se succèdent, et cela finit par ouvrir un chemin/former une sorte de chemin}\end{exemple}
\end{entrée}

\begin{entrée}
{ʐɤ˩ni˧˥}{}{ⓔʐɤ˩ni˧˥}\formedesurface{ʐɤ˩ni˧˥}\newline
\classe{形容词}\ton{L+MH\#}\begin{définition}\peng{Near.}\end{définition}
\begin{définition}\pcmn{近}\end{définition}
\begin{définition}\pfra{Proche.}\end{définition}
\end{entrée}

\begin{entrée}
{ʐɤ˩qo˩}{}{ⓔʐɤ˩qo˩}\newline
\classe{名词}
\sens{1}\paradigme{\pcmn{:} \p{}}
\begin{définition}\peng{Calf.}\end{définition}
\begin{définition}\pcmn{小牛}\end{définition}
\begin{définition}\pfra{Veau.}\end{définition}\sens{2}
\begin{définition}\peng{Male pianniu (hybrid of yak and cattle).}\end{définition}
\begin{définition}\pcmn{公犏牛}\end{définition}
\begin{définition}\pfra{Pianniu, pienniu, dzo, zopiok (mâle).}\end{définition}
\end{entrée}

\begin{entrée}
{ʐɤ˩ʐɤ˧˥}{}{ⓔʐɤ˩ʐɤ˧˥}\formedesurface{ʐɤ˩ʐɤ˧˥}\newline
\classe{名词}\ton{LM+MH\#}
\paradigme{\pcmn{:} \p{}}
\begin{définition}\peng{Lines, pattern.}\end{définition}
\begin{définition}\pcmn{花纹、图案}\end{définition}
\begin{définition}\pfra{Motif.}\end{définition}
\begin{exemple}\pnru{ʐɤ˩ʐɤ˧ tʰi˧-di˥}\hspace{5pt}\peng{with lines/patterns / there are lines/patterns}\hspace{5pt}\pcmn{有花纹}\hspace{5pt}\pfra{qui a des motifs, des dessins (ex.: un tissu)}\end{exemple}
\begin{exemple}\pnru{bɑ˩lɑ˩˥ | ʈʰɯ˧-ɭɯ˥-bi˩ | ʐɤ˩ʐɤ˧ tʰi˧-di˥}\hspace{5pt}\peng{There are lines/patterns on this piece of clothing.}\hspace{5pt}\pcmn{这衣服上面有花纹。}\hspace{5pt}\pfra{sur ce vêtement il y a un motif}\end{exemple}
\end{entrée}

\begin{entrée}
{ʐo˩}{}{ⓔʐo˩}\formedesurface{ʐo˧}\newline
\classe{名词}\ton{L}\begin{définition}\peng{Noon; lunch.}\end{définition}
\begin{définition}\pcmn{中午}\end{définition}
\begin{définition}\pfra{Midi; repas de midi/déjeuner.}\end{définition}
\begin{exemple}\pnru{ʐo˩ dzɯ˩˥}\hspace{5pt}\peng{to have lunch}\hspace{5pt}\pcmn{吃午饭}\hspace{5pt}\pfra{prendre son déjeuner}\end{exemple}
\end{entrée}

\begin{entrée}
{ʐo˩α}{₁}{ⓔʐo˩αⓗ1}\formedesurface{ʐo˩˥}\newline
\classe{动词}\ton{Lα}
1\begin{définition}\peng{To swing back and forth.}\end{définition}
\begin{définition}\pcmn{甩来甩去}\end{définition}
\begin{définition}\pfra{Se balancer.}\end{définition}
\begin{exemple}\pnru{ɖɯ˧-ʐo˩-ɻ̍˩}\hspace{5pt}\peng{to swing back and forth}\hspace{5pt}\pcmn{甩来甩去}\hspace{5pt}\pfra{se balancer un peu}\end{exemple}
\begin{exemple}\pnru{ʐo˩∼ʐo˧-ze˥}\hspace{5pt}\peng{|fg{red} |fg{pfv}}\hspace{5pt}\pcmn{|fg{red} |fg{pfv}}\hspace{5pt}\pfra{|fg{red} |fg{pfv}}\end{exemple}
\begin{exemple}\pnru{le˧-ʐo˩∼ʐo˩}\hspace{5pt}\peng{|fg{accomp} |fg{red}}\hspace{5pt}\pcmn{|fg{accomp} |fg{red}}\hspace{5pt}\pfra{|fg{accomp} |fg{red}}\end{exemple}
\end{entrée}

\begin{entrée}
{ʐo˩α}{₂}{ⓔʐo˩αⓗ2}\formedesurface{ʐo˩˥}\newline
\classe{形容词}\ton{Lα}
2\begin{définition}\peng{Light.}\end{définition}
\begin{définition}\pcmn{轻}\end{définition}
\begin{définition}\pfra{Léger.}\end{définition}
\begin{exemple}\pnru{ʐo˩-tɕʰæ˩ɻæ˥ (ʐo˩-tɕʰæ˩ɻæ˥-gv̩˩)}\hspace{5pt}\peng{very light}\hspace{5pt}\pcmn{轻飘飘}\hspace{5pt}\pfra{tout léger}\end{exemple}
\end{entrée}

\begin{entrée}
{ʐo˩dzɯ˩}{}{ⓔʐo˩dzɯ˩}\formedesurface{ʐo˩dzɯ˩˥}\newline
\classe{动词}\ton{L}\begin{définition}\peng{To eat lunch.}\end{définition}
\begin{définition}\pcmn{吃午饭}\end{définition}
\begin{définition}\pfra{Déjeuner, prendre le repas de midi.}\end{définition}
\begin{exemple}\pnru{ʐo˩ dzɯ˩˥}\hspace{5pt}\peng{to have lunch}\hspace{5pt}\pcmn{吃午饭}\hspace{5pt}\pfra{déjeuner (verbe), prendre le déjeuner}\end{exemple}
\begin{exemple}\pnru{ʐo˩ dzɯ˩-se˥}\hspace{5pt}\peng{afternoon}\hspace{5pt}\pcmn{下午}\hspace{5pt}\pfra{l'après-midi}\end{exemple}
\end{entrée}

\begin{entrée}
{ʐo˩∼ʐo˧˥}{}{ⓔʐo˩∼ʐo˧˥}\formedesurface{ʐo˩ʐo˧˥}\newline
\classe{动词}\ton{MH}\begin{définition}\peng{To swing.}\end{définition}
\begin{définition}\pcmn{摔、摇摆}\end{définition}
\begin{définition}\pfra{Se balancer.}\end{définition}
\end{entrée}

\begin{entrée}
{ʐɯ˥}{}{ⓔʐɯ˥}\formedesurface{ʐɯ˧}\newline
\classe{形容词}\ton{H}\begin{définition}\peng{Heavy.}\end{définition}
\begin{définition}\pcmn{重}\end{définition}
\begin{définition}\pfra{Lourd.}\end{définition}
\begin{exemple}\pnru{ʐɯ˧-ʈʂʰæ˩ɻæ˩ (-gv̩˩)}\hspace{5pt}\peng{very heavy}\hspace{5pt}\pcmn{很重的}\hspace{5pt}\pfra{très lourd}\end{exemple}
\end{entrée}

\begin{entrée}
{ʐɯ˧}{}{ⓔʐɯ˧}\formedesurface{ʐɯ˧}\newline
\classe{名词}\ton{M}\begin{définition}\peng{Fermented alcohol, liquor, spirits, wine.}\end{définition}
\begin{définition}\pcmn{酒}\end{définition}
\begin{définition}\pfra{Alcool fermenté, chang, vin.}\end{définition}
\begin{exemple}\pnru{ʐɯ˧ pʰv̩˧˥}\hspace{5pt}\peng{to pour wine}\hspace{5pt}\pcmn{斟酒}\hspace{5pt}\pfra{verser à boire}\end{exemple}
\end{entrée}

\begin{entrée}
{ʐɯ˩dzi˥}{}{ⓔʐɯ˩dzi˥}\formedesurface{ʐɯ˩dzi˥}\newline
\classe{名词}\ton{LH}
\paradigme{\pcmn{:} \p{}}
\begin{définition}\peng{Cedar.}\end{définition}
\begin{définition}\pcmn{杉树}\end{définition}
\begin{définition}\pfra{Cèdre.}\end{définition}
\end{entrée}

\begin{entrée}
{ʐɯ˩gv̩˩}{}{ⓔʐɯ˩gv̩˩}\formedesurface{ʐɯ˩gv̩˩˥}\newline
\classe{名词}\ton{L}
\paradigme{\pcmn{:} \p{}}
\begin{définition}\peng{Boat.}\end{définition}
\begin{définition}\pcmn{船}\end{définition}
\begin{définition}\pfra{Canot, bateau (utilisé uniquement pour les barques circulant sur le Lac, pas pour les autres bateaux).}\end{définition}
\begin{exemple}\pnru{ʐɯ˩gv̩˩ dzi˩˥}\hspace{5pt}\peng{to sit in a boat}\hspace{5pt}\pcmn{坐船}\hspace{5pt}\pfra{être assis dans un bateau, être à bord d'un bateau}\end{exemple}
\end{entrée}

\begin{entrée}
{ʐɯ˧ɭɯ˧}{}{ⓔʐɯ˧ɭɯ˧}\formedesurface{ʐɯ˧ɭɯ˧}\newline
\classe{动词}\ton{M}\begin{définition}\peng{To shake (of earth), earthquake.}\end{définition}
\begin{définition}\pcmn{地震}\end{définition}
\begin{définition}\pfra{Tremblement de terre/la terre tremble.}\end{définition}
\begin{exemple}\pnru{ʐɯ˧ɭɯ˧-ze˧!}\hspace{5pt}\peng{There is an earthquake!}\hspace{5pt}\pcmn{地震了!}\hspace{5pt}\pfra{Il y a un tremblement de terre! / La terre tremble!}\end{exemple}
\end{entrée}

\begin{entrée}
{ʐɯ˩-mo˧˥}{}{ⓔʐɯ˩-mo˧˥}\formedesurface{ʐɯ˩mo˧˥}\newline
\classe{名词}\ton{LM+MH\#}\begin{définition}\peng{“mushroom of the cedar tree": a sort of mushroom often found close to cedar trees.}\end{définition}
\begin{définition}\pcmn{“杉树菌”:一种菌子}\end{définition}
\begin{définition}\pfra{«champignon des cèdres»; champignon comestible, de la même famille que le «champignon des sapins», \stylefv{/tʰo}˧-mo˩/.}\end{définition}
\begin{exemple}\pnru{tʰo˧mo˩-ʐɯ˩mo˩}\hspace{5pt}\peng{Pine-tree mushroom and cedar-tree mushroom}\hspace{5pt}\pcmn{松树菌与杉树菌}\hspace{5pt}\pfra{champignon des sapins et champignon des cèdres}\end{exemple}
\end{entrée}

\begin{entrée}
{ʐɯ˧nɑ˩}{}{ⓔʐɯ˧nɑ˩}\formedesurface{ʐɯ˧nɑ˩}\newline
\classe{名词}\ton{L\#}\begin{définition}\peng{Strong liquor, high-quality spirits.}\end{définition}
\begin{définition}\pcmn{醇酒,好酒}\end{définition}
\begin{définition}\pfra{Alcool fort; alcool de qualité supérieure.}\end{définition}
\end{entrée}

\begin{entrée}
{ʐɯ˩tse˧}{}{ⓔʐɯ˩tse˧}\formedesurface{ʐɯ˩tse˥}\newline
\classe{名词}\ton{LM}
\paradigme{\pcmn{:} \p{}}
\begin{définition}\peng{Mountain spirit.}\end{définition}
\begin{définition}\pcmn{山神}\end{définition}
\begin{définition}\pfra{Esprit de la montagne.}\end{définition}
\end{entrée}

\begin{entrée}
{ʐɯ˩tse˧-mæ˧ʂæ˩}{}{ⓔʐɯ˩tse˧-mæ˧ʂæ˩}\formedesurface{ʐɯ˩tse˧mæ˧ʂæ˩}\newline
\classe{名词}\ton{LM-L\#}\begin{définition}\peng{Golden pheasant.}\end{définition}
\begin{définition}\pcmn{锦鸡}\end{définition}
\begin{définition}\pfra{Faisan doré.}\end{définition}
\end{entrée}

\begin{entrée}
{ʐɯ˩tsɯ˧}{}{ⓔʐɯ˩tsɯ˧}\formedesurface{ʐɯ˩tsɯ˥}\newline
\classe{名词}\ton{LM}\begin{définition}\peng{Days; life; time.}\end{définition}
\begin{définition}\pcmn{日子(汉语借词)}\end{définition}
\begin{définition}\pfra{Jours, temps.}\end{définition}
\begin{exemple}\pnru{ʐɯ˩tsɯ˧ ʈʂɤ˧}\hspace{5pt}\peng{to look for an auspicious date (for building a house or other important project)}\hspace{5pt}\pcmn{算日子(为了选择吉利的一天)}\hspace{5pt}\pfra{rechercher une date propice (pour la construction d'une maison ou autre projet important)}\end{exemple}
\end{entrée}

\begin{entrée}
{ʐɯ˩tsɯ˧-mɤ˩ʈʂʰɤ˩}{}{ⓔʐɯ˩tsɯ˧-mɤ˩ʈʂʰɤ˩}\formedesurface{ʐɯ˩tsɯ˧mɤ˩ʈʂʰɤ˩}\newline
\classe{名词}\ton{LM-L}
\paradigme{\pcmn{:} \p{}}
\begin{définition}\peng{Mattress.}\end{définition}
\begin{définition}\pcmn{褥子}\end{définition}
\begin{définition}\pfra{Matelas.}\end{définition}
\end{entrée}

\begin{entrée}
{ʐv̩˧}{}{ⓔʐv̩˧}\formedesurface{ʐv̩˧}\newline
\classe{数词}\ton{M? H\#?}\begin{définition}\peng{Four.}\end{définition}
\begin{définition}\pcmn{四}\end{définition}
\begin{définition}\pfra{Quatre.}\end{définition}
\end{entrée}

\begin{entrée}
{ʐv̩˧˥}{}{ⓔʐv̩˧˥}\formedesurface{ʐv̩˧˥}\newline
\classe{动词}\ton{MH}\begin{définition}\peng{To sew.}\end{définition}
\begin{définition}\pcmn{缝}\end{définition}
\begin{définition}\pfra{Coudre.}\end{définition}
\end{entrée}

\begin{entrée}
{ʐv̩˩α}{₁}{ⓔʐv̩˩αⓗ1}\newline
\classe{动词}
1
\sens{1}
\begin{définition}\peng{To knead (dough).}\end{définition}
\begin{définition}\pcmn{揉(面)}\end{définition}
\begin{définition}\pfra{Pétrir (la pâte), malaxer.}\end{définition}
\begin{exemple}\pnru{pɤ˩jɤ˧ ʐv̩˥}\hspace{5pt}\peng{to knead dough}\hspace{5pt}\pcmn{揉面}\hspace{5pt}\pfra{pétrir la pâte}\end{exemple}
\begin{exemple}\pnru{ʐv̩˧∼ʐv̩˥}\hspace{5pt}\peng{|fg{red}}\hspace{5pt}\pcmn{重叠}\hspace{5pt}\pfra{|fg{red}}\end{exemple}
\begin{exemple}\pnru{ɖɯ˧-kʰwɤ˧ ʐv̩˥}\hspace{5pt}\peng{to knead a piece (of dough)}\hspace{5pt}\pcmn{揉一块(面团)}\hspace{5pt}\pfra{pétrir un morceau}\end{exemple}\sens{2}
\begin{définition}\peng{To crease, to crumple, to wrinkle.}\end{définition}
\begin{définition}\pcmn{皱(衣服)}\end{définition}
\begin{définition}\pfra{Froisser, plisser.}\end{définition}
\begin{exemple}\pnru{bɑ˩lɑ˩ ʐv̩˥(-ze˩)}\hspace{5pt}\peng{to crease clothes; the clothes have been creased; the clothes are creased}\hspace{5pt}\pcmn{衣服皱了}\hspace{5pt}\pfra{les vêtements sont froissés, les vêtements ont été froissés}\end{exemple}
\begin{exemple}\pnru{ʐv̩˧∼ʐv̩˥}\hspace{5pt}\peng{|fg{red}}\hspace{5pt}\pcmn{重叠}\hspace{5pt}\pfra{|fg{red}}\end{exemple}
\begin{exemple}\pnru{le˧-ʐv̩˧∼ʐv̩˥-ze˩}\hspace{5pt}\peng{|fg{accomp} \_ |fg{red} |fg{pfv}}\hspace{5pt}\pcmn{|fg{accomp} \_ |fg{red} |fg{pfv}}\hspace{5pt}\pfra{|fg{accomp} \_ |fg{red} |fg{pfv}}\end{exemple}
\end{entrée}

\begin{entrée}
{ʐv̩˩α}{₂}{ⓔʐv̩˩αⓗ2}\formedesurface{ʐv̩˩˥}\newline
\classe{形容词}\ton{Lα}
2\begin{définition}\peng{Delicious, good (to the taste).}\end{définition}
\begin{définition}\pcmn{好吃}\end{définition}
\begin{définition}\pfra{Bon (au goût).}\end{définition}
\end{entrée}

\begin{entrée}
{ʐv̩˧bæ˧}{}{ⓔʐv̩˧bæ˧}\formedesurface{ʐv̩˧bæ˧}\newline
\classe{名词}\ton{M}
\paradigme{\pcmn{:} \p{}}
\begin{définition}\peng{Snake, serpent.}\end{définition}
\begin{définition}\pcmn{蛇}\end{définition}
\begin{définition}\pfra{Serpent.}\end{définition}
\begin{exemple}\pnru{ʐv̩˧bæ˧ ɣɯ˩ pʰv̩˩}\hspace{5pt}\peng{The snake sheds skin / exuviates}\hspace{5pt}\pcmn{蛇蜕皮}\hspace{5pt}\pfra{Le serpent mue}\end{exemple}
\end{entrée}

\begin{entrée}
{ʐv̩˧bæ˧-bv̩˧-hɑ\#˥}{}{ⓔʐv̩˧bæ˧-bv̩˧-hɑ\#˥}\formedesurface{ʐv̩˧bæ˧bv̩˧hɑ˧}\newline
\classe{名词}\ton{\#H}
\paradigme{\pcmn{:} \p{}}
\begin{définition}\peng{One of the three types of pig fodder.}\end{définition}
\begin{définition}\pcmn{能喂给猪的三种草之一}\end{définition}
\begin{définition}\pfra{L'une des trois sortes de fourrage que l'on donne aux cochons; M18 propose comme étymologie populaire «serpent en colère», du fait que cette plante a des feuilles grasses qui ressemblent à un serpent, et qui sont entortillées.}\end{définition}
\end{entrée}

\begin{entrée}
{ʐv̩˧bæ˧-mi˩}{}{ⓔʐv̩˧bæ˧-mi˩}\formedesurface{ʐv̩˧bæ˧mi˩}\newline
\classe{名词}\ton{-L}
\paradigme{\pcmn{:} \p{}}
\begin{définition}\peng{Female snake.}\end{définition}
\begin{définition}\pcmn{母蛇}\end{définition}
\begin{définition}\pfra{Serpent femelle.}\end{définition}
\end{entrée}

\begin{entrée}
{ʐv̩˧bæ˧-pʰv̩\#˥}{}{ⓔʐv̩˧bæ˧-pʰv̩\#˥}\formedesurface{ʐv̩˧bæ˧pʰv̩˧}\newline
\classe{名词}\ton{\#H}
\paradigme{\pcmn{:} \p{}}
\begin{définition}\peng{Male snake.}\end{définition}
\begin{définition}\pcmn{公蛇}\end{définition}
\begin{définition}\pfra{Serpent mâle.}\end{définition}
\end{entrée}

\begin{entrée}
{ʐv̩˧bæ˧-zo\#˥}{}{ⓔʐv̩˧bæ˧-zo\#˥}\formedesurface{ʐv̩˧bæ˧zo˧}\newline
\classe{名词}\ton{\#H}
\paradigme{\pcmn{:} \p{}}
\begin{définition}\peng{Baby snake.}\end{définition}
\begin{définition}\pcmn{小蛇}\end{définition}
\begin{définition}\pfra{Petit serpent.}\end{définition}
\end{entrée}

\begin{entrée}
{ʐv̩˧bæ˧-ʐv̩˧qʰɑ\#˥}{}{ⓔʐv̩˧bæ˧-ʐv̩˧qʰɑ\#˥}\formedesurface{ʐv̩˧bæ˧ʐv̩˧qʰɑ˧}\newline
\classe{名词}\ton{\#H}\begin{définition}\peng{Jack-in-the-pulpit, |\stylefi{Arisaema consanguineum} (a type of flowering plant).}\end{définition}
\begin{définition}\pcmn{天南星}\end{définition}
\begin{définition}\pfra{|\stylefi{Arisaema consanguineum}.}\end{définition}
\end{entrée}

\begin{entrée}
{ʐv̩˧bɤ\#˥}{}{ⓔʐv̩˧bɤ\#˥}\formedesurface{ʐv̩˧bɤ˧}\newline
\classe{名词}\ton{\#H}
\paradigme{\pcmn{:} \p{}}
\begin{définition}\peng{The Pumi (Prinmi) people of the mountains.}\end{définition}
\begin{définition}\pcmn{高山普米族(永宁以北地区:木里等)}\end{définition}
\begin{définition}\pfra{Les Pumi des montagnes, du côté de Muli et Jiaze.}\end{définition}
\end{entrée}

\begin{entrée}
{ʐv̩˧di˧˥}{}{ⓔʐv̩˧di˧˥}\formedesurface{ʐv̩˧di˧˥}\newline
\classe{名词}\ton{MH\#}\begin{définition}\peng{The warm area on the banks of the Yangtze river: Fengke, Labai…}\end{définition}
\begin{définition}\pcmn{金沙江边的地方(气候热)}\end{définition}
\begin{définition}\pfra{Les rives du Yangtze; le climat y est chaud et humide. Ces régions sont perçus par les Na de Yongning comme peuplées de Pumi; ils imaginent que les habitants de Fengke et Labai seraient des descendants des Pumi. (Source: consultants F4, F5, M21.).}\end{définition}
\end{entrée}

\begin{entrée}
{ʐv̩˧dzi˩}{}{ⓔʐv̩˧dzi˩}\formedesurface{ʐv̩˧dzi˩}\newline
\classe{名词}\ton{L\#}
\paradigme{\pcmn{:} \p{}}
\begin{définition}\peng{Willow tree.}\end{définition}
\begin{définition}\pcmn{柳树,杨柳}\end{définition}
\begin{définition}\pfra{Saule.}\end{définition}
\end{entrée}

\begin{entrée}
{ʐv̩˧hĩ\#˥}{}{ⓔʐv̩˧hĩ\#˥}\formedesurface{ʐv̩˧hĩ˧}\newline
\classe{名词}\ton{\#H}
\paradigme{\pcmn{:} \p{}}
\begin{définition}\peng{One of the designations of the Pumi (ethnic group).}\end{définition}
\begin{définition}\pcmn{普米族}\end{définition}
\begin{définition}\pfra{Désignation des Pumi.}\end{définition}
\end{entrée}

\begin{entrée}
{ʐv̩˩ɭɯ˥}{}{ⓔʐv̩˩ɭɯ˥}\formedesurface{ʐv̩˩ɭɯ˥}\newline
\classe{名词}\ton{LH}
\paradigme{\pcmn{:} \p{}}
\begin{définition}\peng{Beam.}\end{définition}
\begin{définition}\pcmn{支撑顶板的梁}\end{définition}
\begin{définition}\pfra{Poutre soutenant la toiture, posée horizontalement, dans le sens de la longueur du bâtiment. Sur elle reposent les poutrelles courtes posées inclinées dans le sens de la largeur du bâtiment, /hæ̃˧kʰɤ˧˥/.}\end{définition}
\end{entrée}

\begin{entrée}
{ʐv̩˩-ɬi˩mi˩}{}{ⓔʐv̩˩-ɬi˩mi˩}\formedesurface{ʐv̩˩ɬi˩mi˩˥}\newline
\classe{名词}\ton{L}\begin{définition}\peng{4th month.}\end{définition}
\begin{définition}\pcmn{四月}\end{définition}
\begin{définition}\pfra{4e mois.}\end{définition}
\end{entrée}

\begin{entrée}
{ʐv̩˧mi\#˥}{}{ⓔʐv̩˧mi\#˥}\formedesurface{ʐv̩˧mi˧}\newline
\classe{名词}\ton{\#H}
\paradigme{\pcmn{:} \p{}}
\begin{définition}\peng{Granddaughter.}\end{définition}
\begin{définition}\pcmn{孙女}\end{définition}
\begin{définition}\pfra{Petite-fille.}\end{définition}
\begin{exemple}\pnru{njɤ˧ | ʐv̩˧mi˧ | ɖɯ˧-ɭɯ˧ dʑo˧}\hspace{5pt}\peng{I have a granddaughter.}\hspace{5pt}\pcmn{我有一个孙女。}\hspace{5pt}\pfra{J'ai une petite-fille.}\end{exemple}
\end{entrée}

\begin{entrée}
{ʐv̩˩mi˩}{}{ⓔʐv̩˩mi˩}\formedesurface{ʐv̩˩mi˩˥}\newline
\classe{名词}\ton{L}
\paradigme{\pcmn{:} \p{}}
\begin{définition}\peng{Bow (archery bow).}\end{définition}
\begin{définition}\pcmn{弓}\end{définition}
\begin{définition}\pfra{Arc.}\end{définition}
\end{entrée}

\begin{entrée}
{ʐv̩˧mv̩˧lɑ˧di˧˥}{}{ⓔʐv̩˧mv̩˧lɑ˧di˧˥}\formedesurface{ʐv̩˧mv̩˧lɑ˧di˧˥}\newline
\classe{名词}\ton{MH\#}
\paradigme{\pcmn{:} \p{}}
\begin{définition}\peng{The territory of the Pumi people on the banks of the Yangtze river. This area is perceived as less central and pleasant than Yongning.}\end{définition}
\begin{définition}\pcmn{江边普米族地区(带偏见的说法)}\end{définition}
\begin{définition}\pfra{Les territoires des Pumi, au bord du fleuve Yangtze. Le terme est connoté péjorativement: cette région est perçue comme périphérique et moins plaisante que la plaine de Yongning.}\end{définition}
\begin{exemple}\pnru{ʐv̩˧mv̩˧lɑ˧di˧-hĩ˥}\hspace{5pt}\peng{people from the Pumi territories}\hspace{5pt}\pcmn{普米族地区的人们}\hspace{5pt}\pfra{habitants des territoires pumi des bords du fleuve; personnes pumi}\end{exemple}
\end{entrée}

\begin{entrée}
{ʐv̩˧-ɲi˧-ʁo˧tʰo˥}{}{ⓔʐv̩˧-ɲi˧-ʁo˧tʰo˥}\formedesurface{ʐv̩˧ɲi˧ʁo˧tʰo˥}\newline
\classe{助词}\ton{H\#}\begin{définition}\peng{In four days.}\end{définition}
\begin{définition}\pcmn{四天以后}\end{définition}
\begin{définition}\pfra{Dans quatre jours.}\end{définition}
\end{entrée}

\begin{entrée}
{ʐv̩˧ɻ̍˥}{}{ⓔʐv̩˧ɻ̍˥}\formedesurface{ʐv̩˧ɻ̍˥}\newline
\classe{形容词}\ton{H\#}\begin{définition}\peng{Square.}\end{définition}
\begin{définition}\pcmn{正方形}\end{définition}
\begin{définition}\pfra{Carré.}\end{définition}
\begin{exemple}\pnru{ʐv̩˩-hĩ˩˥}\hspace{5pt}\peng{|fg{nmlz}}\hspace{5pt}\pcmn{方形的}\hspace{5pt}\pfra{|fg{nmlz}}\end{exemple}
\begin{exemple}\pnru{ʐv̩˧ɻ̍˥-gv̩˩}\hspace{5pt}\peng{square}\hspace{5pt}\pcmn{方形的}\hspace{5pt}\pfra{carré}\end{exemple}
\end{entrée}

\begin{entrée}
{ʐv̩˧-tsʰi˩}{}{ⓔʐv̩˧-tsʰi˩}\formedesurface{ʐv̩˧tsʰi˩}\newline
\classe{数词}\ton{L\#}\begin{définition}\peng{40.}\end{définition}
\begin{définition}\pcmn{40}\end{définition}
\begin{définition}\pfra{40.}\end{définition}
\end{entrée}

\begin{entrée}
{ʐv̩˧v̩\#˥}{}{ⓔʐv̩˧v̩\#˥}\formedesurface{ʐv̩˧v̩˧}\newline
\classe{名词}\ton{\#H}
\paradigme{\pcmn{:} \p{}}
\begin{définition}\peng{Grandson.}\end{définition}
\begin{définition}\pcmn{孙子}\end{définition}
\begin{définition}\pfra{Petit-fils.}\end{définition}
\begin{exemple}\pnru{njɤ˧ | ʐv̩˧v̩˧ ɖɯ˧-ɭɯ˧ dʑo˧.}\hspace{5pt}\peng{I have a grandson.}\hspace{5pt}\pcmn{我有一个孙子。}\hspace{5pt}\pfra{j'ai un petit-fils}\end{exemple}
\end{entrée}

\begin{entrée}
{ʐv̩˧v̩˥-ʐv̩˩mi˩}{}{ⓔʐv̩˧v̩˥-ʐv̩˩mi˩}\formedesurface{ʐv̩˧v̩˥ʐv̩˩mi˩}\newline
\classe{名词}\ton{H\#-}\begin{définition}\peng{Grandchildren.}\end{définition}
\begin{définition}\pcmn{孙子孙女}\end{définition}
\begin{définition}\pfra{Petits-enfants.}\end{définition}
\end{entrée}

\begin{entrée}
{ʐv̩˧-zo\#˥}{}{ⓔʐv̩˧-zo\#˥}\formedesurface{ʐv̩˧zo˧}\newline
\classe{名词}\ton{\#H}
\paradigme{\pcmn{:} \p{}}
\begin{définition}\peng{Small bow (archery bow).}\end{définition}
\begin{définition}\pcmn{小弓}\end{définition}
\begin{définition}\pfra{Petit arc.}\end{définition}
\end{entrée}

\begin{entrée}
{ʐwæ˥}{}{ⓔʐwæ˥}\formedesurface{ʐwæ˧}\newline
\classe{名词}\ton{\#H}
\paradigme{\pcmn{:} \p{}}
\begin{définition}\peng{Horse.}\end{définition}
\begin{définition}\pcmn{马}\end{définition}
\begin{définition}\pfra{Cheval.}\end{définition}
\begin{exemple}\pnru{dʑɯ˩ʁo˩-ʐwæ˩}\hspace{5pt}\peng{wild horse}\hspace{5pt}\pcmn{野马}\hspace{5pt}\pfra{cheval sauvage, non domestiqué}\end{exemple}
\begin{exemple}\pnru{o-ho-ho! ʐwæ˧-ɳɯ˩ | dzɯ˧-po˧-hɯ˥-ze˩!}\hspace{5pt}\peng{Oops! The horse scoffed the lot!}\hspace{5pt}\pcmn{啊呀嚒!马把饲料都吃光了!}\hspace{5pt}\pfra{Houlàlà! Le cheval nous l'a mangé! (Contexte: on laisse dans la cour des céréales, ou du fourrage, et pendant qu'on a le dos tourné, le cheval chaparde cette nourriture.)}\end{exemple}
\end{entrée}

\begin{entrée}
{ʐwæ˧˥}{}{ⓔʐwæ˧˥}\formedesurface{ʐwæ˧˥}\newline
\classe{动词}\ton{MH}\begin{définition}\peng{To hoe weeds.}\end{définition}
\begin{définition}\pcmn{薅锄、锄草}\end{définition}
\begin{définition}\pfra{Sarcler, biner.}\end{définition}
\begin{exemple}\pnru{ʐwæ˩∼ʐwæ˧˥}\hspace{5pt}\peng{|fg{red}}\hspace{5pt}\pcmn{重叠}\hspace{5pt}\pfra{|fg{red}}\end{exemple}
\begin{exemple}\pnru{jɤ˩jo˥ ʐwæ˩}\hspace{5pt}\peng{to hoe potatoes, to weed a potato field}\hspace{5pt}\pcmn{洋芋地里锄草}\hspace{5pt}\pfra{sarcler des pommes de terre}\end{exemple}
\begin{exemple}\pnru{jɤ˩jo˧ ʐwæ˧∼ʐwæ˥}\hspace{5pt}\peng{to hoe potatoes, to weed a potato field}\hspace{5pt}\pcmn{洋芋地里锄草}\hspace{5pt}\pfra{sarcler des pommes de terre}\end{exemple}
\begin{exemple}\pnru{qʰɑ˧dze˧ ʐwæ˧˥}\hspace{5pt}\peng{to hoe sweetcorn, to weed a sweetcorn field}\hspace{5pt}\pcmn{苞谷地里锄草}\hspace{5pt}\pfra{sarcler du maïs}\end{exemple}
\begin{exemple}\pnru{qʰɑ˧dze˧ ʐwæ˧∼ʐwæ˥}\hspace{5pt}\peng{to hoe sweetcorn, to weed a sweetcorn field}\hspace{5pt}\pcmn{苞谷地里锄草}\hspace{5pt}\pfra{sarcler du maïs}\end{exemple}
\end{entrée}

\begin{entrée}
{ʐwæ˧α}{}{ⓔʐwæ˧α}\formedesurface{ʐwæ˧}\newline
\classe{动词}\ton{Mα}\begin{définition}\peng{To weigh (with scales).}\end{définition}
\begin{définition}\pcmn{称}\end{définition}
\begin{définition}\pfra{Peser (à l'aide d'une balance).}\end{définition}
\begin{exemple}\pnru{mɤ˧-ʐwæ˧}\hspace{5pt}\peng{|fg{neg}}\hspace{5pt}\pcmn{不称}\hspace{5pt}\pfra{|fg{neg}}\end{exemple}
\begin{exemple}\pnru{le˧-ʐwæ˧-ze˧}\hspace{5pt}\peng{|fg{accomp} \_ |fg{pfv}}\hspace{5pt}\pcmn{称了}\hspace{5pt}\pfra{|fg{accomp} \_ |fg{pfv}}\end{exemple}
\begin{exemple}\pnru{tso˧∼tso˧ ʐwæ˩}\hspace{5pt}\peng{to weigh things}\hspace{5pt}\pcmn{称东西}\hspace{5pt}\pfra{peser des choses}\end{exemple}
\begin{exemple}\pnru{ʁo˧do˧ ʐwæ˧}\hspace{5pt}\peng{to weigh walnuts}\hspace{5pt}\pcmn{称核桃}\hspace{5pt}\pfra{peser des noix}\end{exemple}
\end{entrée}

\begin{entrée}
{ʐwæ˩}{}{ⓔʐwæ˩}\formedesurface{ʐwæ˩˥}\newline
\classe{助词}\ton{L}\begin{définition}\peng{Extremely.}\end{définition}
\begin{définition}\pcmn{很、极}\end{définition}
\begin{définition}\pfra{Extrêmement.}\end{définition}
\begin{exemple}\pnru{ʐwæ˩-ze˥!}\hspace{5pt}\peng{That's too much! / There's too much!}\hspace{5pt}\pcmn{太多了!}\hspace{5pt}\pfra{Il y en a trop! / ça fait trop!}\end{exemple}
\end{entrée}

\begin{entrée}
{ʐwæ˩α}{₁}{ⓔʐwæ˩αⓗ1}\formedesurface{ʐwæ˩˥}\newline
\classe{动词}\ton{Lα}
1\begin{définition}\peng{To swoon.}\end{définition}
\begin{définition}\pcmn{昏,昏厥}\end{définition}
\begin{définition}\pfra{S'évanouir.}\end{définition}
\begin{exemple}\pnru{le˧-ʈʰi˩ | le˧-ʐwæ˩-ze˩}\hspace{5pt}\peng{to be so tired as to fall into a swoon, to swoon from exhaustion}\hspace{5pt}\pcmn{累得都昏倒了}\hspace{5pt}\pfra{s'évanouir à force de fatigue, s'évanouir d'épuisement}\end{exemple}
\end{entrée}

\begin{entrée}
{ʐwæ˩α}{₂}{ⓔʐwæ˩αⓗ2}\formedesurface{ʐwæ˩˥}\newline
\classe{形容词}\ton{Lα}
2\begin{définition}\peng{Good, well (working well, strongly).}\end{définition}
\begin{définition}\pcmn{好,能干}\end{définition}
\begin{définition}\pfra{Habile, bon, capable.}\end{définition}
\begin{exemple}\pnru{ʐwæ˩-hĩ˩˥}\hspace{5pt}\peng{|fg{nmlz}}\hspace{5pt}\pcmn{能干的}\hspace{5pt}\pfra{|fg{nmlz}}\end{exemple}
\begin{exemple}\pnru{ʈʂʰɯ˧ ɖwæ˧˥ | ʐwæ˩˥!}\hspace{5pt}\peng{He is very capable!}\hspace{5pt}\pcmn{他很能干!}\hspace{5pt}\pfra{Il est très habile! / Il est formidable!}\end{exemple}
\end{entrée}

\begin{entrée}
{ʐwæ˧bv̩˧˥}{}{ⓔʐwæ˧bv̩˧˥}\formedesurface{ʐwæ˧bv̩˧˥}\newline
\classe{名词}\ton{MH\#}
\paradigme{\pcmn{:} \p{}}
\begin{définition}\peng{Horse's stable.}\end{définition}
\begin{définition}\pcmn{马圈}\end{définition}
\begin{définition}\pfra{Enclos des chevaux.}\end{définition}
\end{entrée}

\begin{entrée}
{ʐwæ˧-hɑ\#˥}{}{ⓔʐwæ˧-hɑ\#˥}\formedesurface{ʐwæ˧hɑ˧}\newline
\classe{名词}\ton{\#H}\begin{définition}\peng{Horse feed.}\end{définition}
\begin{définition}\pcmn{马料、马饲料}\end{définition}
\begin{définition}\pfra{Nourriture pour cheval.}\end{définition}
\end{entrée}

\begin{entrée}
{ʐwæ˧-kʰv̩˩}{₁}{ⓔʐwæ˧-kʰv̩˩ⓗ1}\formedesurface{ʐwæ˧kʰv̩˩}\newline
\classe{名词}\ton{L\#}
1\begin{définition}\peng{Year of the Horse.}\end{définition}
\begin{définition}\pcmn{马年}\end{définition}
\begin{définition}\pfra{Année du Cheval.}\end{définition}
\end{entrée}

\begin{entrée}
{ʐwæ˧-kʰv̩˩}{₂}{ⓔʐwæ˧-kʰv̩˩ⓗ2}\formedesurface{ʐwæ˧kʰv̩˩}\newline
\classe{形容词}\ton{L\#}
2\begin{définition}\peng{Born in the year of the Horse.}\end{définition}
\begin{définition}\pcmn{属马}\end{définition}
\begin{définition}\pfra{Né l'année du Cheval.}\end{définition}
\end{entrée}

\begin{entrée}
{ʐwæ˧-ɭɯ\#˥}{}{ⓔʐwæ˧-ɭɯ\#˥}\formedesurface{ʐwæ˧ɭɯ˧}\newline
\classe{名词}\ton{\#H}\begin{définition}\peng{Cereals for the horse, horse fodder.}\end{définition}
\begin{définition}\pcmn{马料(粮食)}\end{définition}
\begin{définition}\pfra{Nourriture pour cheval (céréales).}\end{définition}
\end{entrée}

\begin{entrée}
{ʐwæ˩mi˩}{}{ⓔʐwæ˩mi˩}\formedesurface{ʐwæ˩mi˩˥}\newline
\classe{名词}\ton{L}
\paradigme{\pcmn{:} \p{}}
\begin{définition}\peng{Mare.}\end{définition}
\begin{définition}\pcmn{母马}\end{définition}
\begin{définition}\pfra{Jument; également employé pour une jeune jument: pouliche, «bébé cheval» de sexe féminin.}\end{définition}
\begin{exemple}\pnru{ʂe˩-ʐwæ˩mi˥}\hspace{5pt}\peng{bicycle}\hspace{5pt}\pcmn{自行车(“铁马”)}\hspace{5pt}\pfra{vélo; néologisme introduit par F4 d'après le taïwanais tiěmǎ 铁马 que j'essayais de traduire en na.}\end{exemple}
\begin{exemple}\pnru{ʐwæ˩mi˩-ʐwæ˩zo˩}\hspace{5pt}\peng{mare and colt}\hspace{5pt}\pcmn{母马与马驹子}\hspace{5pt}\pfra{jument et poulain}\end{exemple}
\end{entrée}

\begin{entrée}
{ʐwæ˧pʰæ˧di˧˥}{}{ⓔʐwæ˧pʰæ˧di˧˥}\formedesurface{ʐwæ˧pʰæ˧di˧˥}\newline
\classe{名词}\ton{MH\#}
\paradigme{\pcmn{:} \p{}}
\begin{définition}\peng{Lunge, tether (for a horse).}\end{définition}
\begin{définition}\pcmn{拉马链子}\end{définition}
\begin{définition}\pfra{Longe, objet pour accrocher le cheval.}\end{définition}
\end{entrée}

\begin{entrée}
{ʐwæ˧-qʰæ\#˥}{}{ⓔʐwæ˧-qʰæ\#˥}\formedesurface{ʐwæ˧qʰæ˧}\newline
\classe{名词}\ton{\#H}
\paradigme{\pcmn{:} \p{}}
\begin{définition}\peng{Horse manure.}\end{définition}
\begin{définition}\pcmn{马粪}\end{définition}
\begin{définition}\pfra{Crottin de cheval.}\end{définition}
\end{entrée}

\begin{entrée}
{ʐwæ˧ʁo˩}{}{ⓔʐwæ˧ʁo˩}\formedesurface{ʐwæ˧ʁo˩}\newline
\classe{名词}\ton{L\#}
\paradigme{\pcmn{:} \p{}}
\begin{définition}\peng{Castrated horse, gelding, neutered horse.}\end{définition}
\begin{définition}\pcmn{骟马}\end{définition}
\begin{définition}\pfra{Cheval castré.}\end{définition}
\end{entrée}

\begin{entrée}
{ʐwæ˧sɯ˩}{}{ⓔʐwæ˧sɯ˩}\formedesurface{ʐwæ˧sɯ˩}\newline
\classe{名词}\ton{L\#}
\paradigme{\pcmn{:} \p{}}
\begin{définition}\peng{Stallion.}\end{définition}
\begin{définition}\pcmn{公马}\end{définition}
\begin{définition}\pfra{Étalon.}\end{définition}
\begin{exemple}\pnru{ʐwæ˧sɯ˩-ʐwæ˩mi˩}\hspace{5pt}\peng{stallion and mare}\hspace{5pt}\pcmn{公马与母马}\hspace{5pt}\pfra{étalon et jument}\end{exemple}
\begin{exemple}\pnru{ʐwæ˧sɯ˩-ʐwæ˩zo˩}\hspace{5pt}\peng{stallion and colt}\hspace{5pt}\pcmn{公马与小马}\hspace{5pt}\pfra{étalon et poulain}\end{exemple}
\end{entrée}

\begin{entrée}
{ʐwæ˧zo\#˥}{}{ⓔʐwæ˧zo\#˥}\formedesurface{ʐwæ˧zo˧}\newline
\classe{名词}\ton{\#H}
\paradigme{\pcmn{:} \p{}}
\begin{définition}\peng{Colt, pony, filly, foal.}\end{définition}
\begin{définition}\pcmn{马驹子}\end{définition}
\begin{définition}\pfra{Poulain.}\end{définition}
\begin{exemple}\pnru{ʐwæ˧zo˧-ʐwæ˥mi˩}\hspace{5pt}\peng{colt and mare}\hspace{5pt}\pcmn{马驹子与母马}\hspace{5pt}\pfra{poulain et jument}\end{exemple}
\end{entrée}

\begin{entrée}
{ʐwæ˧-zɯ\#˥}{}{ⓔʐwæ˧-zɯ\#˥}\formedesurface{ʐwæ˧zɯ˧}\newline
\classe{名词}\ton{\#H}\begin{définition}\peng{Hay for horses, horse hay.}\end{définition}
\begin{définition}\pcmn{喂马的草}\end{définition}
\begin{définition}\pfra{Fourrage pour le cheval, foin pour cheval.}\end{définition}
\end{entrée}

\begin{entrée}
{ʐwæ˧∼ʐwæ˧}{}{ⓔʐwæ˧∼ʐwæ˧}\newline
\classe{动词}
\sens{1}
\begin{définition}\peng{To put (things) in order.}\end{définition}
\begin{définition}\pcmn{收拾}\end{définition}
\begin{définition}\pfra{Ranger (des objets).}\end{définition}
\begin{exemple}\pnru{tso˧∼tso˧ ʐwæ˧∼ʐwæ˧(-ze˩)}\hspace{5pt}\peng{to put things in order}\hspace{5pt}\pcmn{收拾东西}\hspace{5pt}\pfra{ranger des choses}\end{exemple}
\begin{exemple}\pnru{le˧-ʐwæ˧∼ʐwæ˧ ɖɯ˧-ʝi˧-tɕɯ˥}\hspace{5pt}\peng{to put things in order in one place, to arrange things together in one place}\hspace{5pt}\pcmn{把东西收拾在一起}\hspace{5pt}\pfra{ranger des choses et les mettre à leur place, ranger des choses ensemble}\end{exemple}\sens{2}
\begin{définition}\peng{To gather (people).}\end{définition}
\begin{définition}\pcmn{聚集}\end{définition}
\begin{définition}\pfra{Rassembler (des gens).}\end{définition}
\end{entrée}

\begin{entrée}
{ʐwɤ˧}{}{ⓔʐwɤ˧}\formedesurface{ʐwɤ˧}\newline
\classe{形容词}\ton{M}\begin{définition}\peng{Hungry (monosyllable).}\end{définition}
\begin{définition}\pcmn{饿}\end{définition}
\begin{définition}\pfra{Qui a faim (forme monosyllabique). Se combine en disyllabe avec le mot ‘nourriture’.}\end{définition}
\begin{exemple}\pnru{hɑ˧-ʐwɤ˩}\hspace{5pt}\peng{to be hungry}\hspace{5pt}\pcmn{饿}\hspace{5pt}\pfra{avoir faim}\end{exemple}
\end{entrée}

\begin{entrée}
{ʐwɤ˩β}{}{ⓔʐwɤ˩β}\formedesurface{ʐwɤ˩˥}\newline
\classe{动词}\ton{Lβ}\begin{définition}\peng{To speak.}\end{définition}
\begin{définition}\pcmn{讲话}\end{définition}
\begin{définition}\pfra{Parler.}\end{définition}
\begin{exemple}\pnru{ʐwɤ˧∼ʐwɤ˩ mɤ˩-hĩ˩}\hspace{5pt}\peng{dumb person, person who is not able to speak}\hspace{5pt}\pcmn{哑巴、不会讲话的人}\hspace{5pt}\pfra{muet, personne muette, personne qui ne parle pas}\end{exemple}
\begin{exemple}\pnru{ʐwɤ˧∼ʐwɤ˩ mɤ˩-hĩ˩, | ʈʂʰɯ˧-v̩˧!}\hspace{5pt}\peng{(S)he is not able to speak! / (S)he won't speak!}\hspace{5pt}\pcmn{不会讲话,这个人! / 这个人,不会讲话!}\hspace{5pt}\pfra{Elle/il ne sait pas parler, elle/lui!}\end{exemple}
\begin{exemple}\pnru{le˧-ʐwɤ˩-ze˩}\hspace{5pt}\peng{|fg{accomp} \_ |fg{pfv}}\hspace{5pt}\pcmn{讲了}\hspace{5pt}\pfra{|fg{accomp} \_ |fg{pfv}}\end{exemple}
\begin{exemple}\pnru{no˧ | ə˧tso˧ ʐwɤ˩-ɲi˩?}\hspace{5pt}\peng{What are you saying? / What do you mean?}\hspace{5pt}\pcmn{你说什么?}\hspace{5pt}\pfra{Que dis-tu? / Qu'est-ce que tu veux dire?}\end{exemple}
\begin{exemple}\pnru{ʐwɤ˧∼ʐwɤ˩}\hspace{5pt}\peng{|fg{red}}\hspace{5pt}\pcmn{重叠}\hspace{5pt}\pfra{|fg{red}}\end{exemple}
\begin{exemple}\pnru{le˧-ʐwɤ˩}\hspace{5pt}\peng{to answer}\hspace{5pt}\pcmn{回答}\hspace{5pt}\pfra{répondre, donner une réponse}\end{exemple}
\begin{exemple}\pnru{le˧-wo˧ ʐwɤ˧˥}\hspace{5pt}\peng{to answer}\hspace{5pt}\pcmn{回答}\hspace{5pt}\pfra{répondre, donner une réponse}\end{exemple}
\begin{exemple}\pnru{ʈʂʰɯ˧ | le˧-ʐwɤ˩-bi˩-dʑo˩…}\hspace{5pt}\peng{According to her/him… / From her/his point of view…}\hspace{5pt}\pcmn{依照他的说法……}\hspace{5pt}\pfra{A ce qu'elle/il dit…}\end{exemple}
\begin{exemple}\pnru{hĩ˧-qɑ˧ ʐwɤ˧∼ʐwɤ˥}\hspace{5pt}\peng{to speak to people}\hspace{5pt}\pcmn{对人家讲}\hspace{5pt}\pfra{parler aux gens}\end{exemple}
\begin{exemple}\pnru{le˧-ʐwɤ˧∼ʐwɤ˥-ze˩}\hspace{5pt}\peng{|fg{accomp} |fg{red} |fg{pfv}}\hspace{5pt}\pcmn{|fg{accomp} |fg{red} |fg{pfv}}\hspace{5pt}\pfra{|fg{accomp} |fg{red} |fg{pfv}}\end{exemple}
\begin{exemple}\pnru{ʐwɤ˧qʰwæ˧-ʈʂɯ˧qʰwæ˧ ɲi˩!}\hspace{5pt}\peng{That's how the saying goes! (Context: explaining that an idea is not just the speaker's opinion, but conforms to established wisdom.)}\hspace{5pt}\pcmn{是老说法!(指出:说的是老说法,不是说话人的个人意见。)}\hspace{5pt}\pfra{C'est ce qu'on dit! / C'est comme ça qu'on dit! (Autrement dit, ce qu'on vient d'expliquer n'est pas seulement l'opinion du locuteur, mais reflète le sens commun.)}\end{exemple}
\end{entrée}

\begin{entrée}
{ʐwɤ˧mv̩˧}{}{ⓔʐwɤ˧mv̩˧}\formedesurface{ʐwɤ˧mv̩˧}\newline
\classe{名词}\ton{M}
\étymologie{
ʐwɤ˩b
}
\paradigme{\pcmn{:} \p{}}
\begin{définition}\peng{Idiom, set phrase, fixed expression.}\end{définition}
\begin{définition}\pcmn{惯用语、习惯语、习语}\end{définition}
\begin{définition}\pfra{Formule toute faite, expression toute faite, expression idiomatique.}\end{définition}
\begin{exemple}\pnru{ʐwɤ˧mv̩˧ dʑo˧-kv̩˧˥ !}\hspace{5pt}\peng{This is how they say! / There's such a set phrase!}\hspace{5pt}\pcmn{有这么一句老话! / 有这么一个说法!}\hspace{5pt}\pfra{C'est comme ça qu'on dit! / Il y a une expression comme ça!}\end{exemple}
\begin{exemple}\pnru{æ˧ʂæ˧-ʐwɤ˧mv̩˧ | ɖɯ˧-kʰwɤ˥}\hspace{5pt}\peng{a saying, a set phrase of the old times}\hspace{5pt}\pcmn{一句老话、一个传统的说法}\hspace{5pt}\pfra{une expression toute faite du temps jadis, un dicton}\end{exemple}
\begin{exemple}\pnru{ʐwɤ˧mv̩˧-pi˧-mv̩˥ (≈ʐwɤ˧mv̩˧-pi˧mv̩\#˥)}\hspace{5pt}\peng{a saying, a set phrase of the old times}\hspace{5pt}\pcmn{一句老话、一个传统的说法}\hspace{5pt}\pfra{une expression toute faite du temps jadis, un dicton}\end{exemple}
\begin{exemple}\pnru{ʐwɤ˧mv̩˧-pi˧mv̩˧ ɲi˥!}\hspace{5pt}\peng{It's a saying of the old times!}\hspace{5pt}\pcmn{是老说法!}\hspace{5pt}\pfra{C'est ce qu'on dit! / C'est ce que dit le proverbe!}\end{exemple}
\end{entrée}

\newpage\caractère{ʑ}

\begin{entrée}
{ʑi˥}{}{ⓔʑi˥}\formedesurface{ʑi˧}\newline
\classe{动词}\ton{H}\begin{définition}\peng{To be present: abstract entity (courage, strength) or concrete entity (beard).}\end{définition}
\begin{définition}\pcmn{有,拥有(抽象:有力量,有勇气)}\end{définition}
\begin{définition}\pfra{Être présent, y avoir; propriété du corps, de l'âme, d'un objet… Ex.: avoir de la force; avoir de la barbe; il y a une resserre dans la maison.}\end{définition}
\begin{exemple}\pnru{mɤ˧-ʑi˥}\hspace{5pt}\peng{|fg{neg}}\hspace{5pt}\pcmn{没有}\hspace{5pt}\pfra{|fg{neg}}\end{exemple}
\end{entrée}

\begin{entrée}
{ʑi˧˥}{₁}{ⓔʑi˧˥ⓗ1}\formedesurface{ʑi˧˥}\newline
\classe{动词}\ton{MH}
1\begin{définition}\peng{To sleep.}\end{définition}
\begin{définition}\pcmn{睡觉}\end{définition}
\begin{définition}\pfra{Dormir.}\end{définition}
\begin{exemple}\pnru{le˧-ʑi˧˥}\hspace{5pt}\peng{|fg{accomp}}\hspace{5pt}\pcmn{|fg{accomp}}\hspace{5pt}\pfra{|fg{accomp}}\end{exemple}
\begin{exemple}\pnru{le˧-ʑi˧-ze˥}\hspace{5pt}\peng{|fg{accomp}|fg{pfv}}\hspace{5pt}\pcmn{|fg{accomp}|fg{pfv}}\hspace{5pt}\pfra{|fg{accomp}|fg{pfv}}\end{exemple}
\begin{exemple}\pnru{æ˩ ʑi˧-ze˥}\hspace{5pt}\peng{the chicken has gone asleep}\hspace{5pt}\pcmn{鸡睡觉了}\hspace{5pt}\pfra{la poule s'est endormie}\end{exemple}
\begin{exemple}\pnru{le˧-ʑi˧-bi˧-ze˩!}\hspace{5pt}\peng{I'm going to sleep!}\hspace{5pt}\pcmn{要睡觉了!}\hspace{5pt}\pfra{( je) vais dormir!/(je) vais me coucher!}\end{exemple}
\begin{exemple}\pnru{le˧-ʑi˧˥, | ʑi˧-mɤ˥-tʰɑ˩!}\hspace{5pt}\peng{Were I to try to sleep, I would not be able to! / I would like to sleep, but I can't!}\hspace{5pt}\pcmn{想睡,但睡不了!}\hspace{5pt}\pfra{j'essaierais de dormir, que je n'y arriverais pas! / Dormir, il ne faut pas y penser/je n'y arriverais pas! (contexte: une personne âgée se plaint de maux de tête; on lui suggère d'aller se reposer/faire une sieste)}\end{exemple}
\begin{exemple}\pnru{pʰæ˧tɕi˥-zo˩-ɳɯ˩ | mv̩˩zo˩-qɑ˥ ʑi˩}\hspace{5pt}\peng{The young man sleeps with the young woman. (This phrasing refers rather bluntly to sexual intercourse, and is considered extremely rude.)}\hspace{5pt}\pcmn{小伙子跟年轻女人睡!(庸俗说法)}\hspace{5pt}\pfra{Le jeune homme couche avec la jeune femme. (Formulation considérée comme vulgaire; ce thème est tabou.)}\end{exemple}
\end{entrée}

\begin{entrée}
{ʑi˧α}{}{ⓔʑi˧α}\newline
\classe{动词}
\sens{1}
\begin{définition}\peng{To leak.}\end{définition}
\begin{définition}\pcmn{漏(水)}\end{définition}
\begin{définition}\pfra{Couler, avoir une fuite; s'écouler (fleuve).}\end{définition}
\begin{exemple}\pnru{mv̩˩tɕo˧ ʑi˧}\hspace{5pt}\peng{to leak, to drip down}\hspace{5pt}\pcmn{(水)往下漏}\hspace{5pt}\pfra{fuir, avoir une fuite}\end{exemple}\sens{2}
\begin{définition}\peng{To flow (a river flows).}\end{définition}
\begin{définition}\pcmn{流(河水流着)}\end{définition}
\begin{définition}\pfra{S'écouler, couler (rivière).}\end{définition}
\begin{exemple}\pnru{mv̩˩tɕo˧ ʑi˧}\hspace{5pt}\peng{to flow down (the water of a brook flows down)}\hspace{5pt}\pcmn{(河)往下游流}\hspace{5pt}\pfra{couler (rivière)}\end{exemple}
\end{entrée}

\begin{entrée}
{ʑi˩}{₂}{ⓔʑi˩ⓗ2}\formedesurface{ɖɯ˧ ʑi˩}\newline
\classe{量词}\ton{Lβ}
2\begin{définition}\peng{Family.}\end{définition}
\begin{définition}\pcmn{家庭(一户人)}\end{définition}
\begin{définition}\pfra{Famille.}\end{définition}
\begin{exemple}\pnru{hĩ˧ | ɖɯ˧-ʑi˩}\hspace{5pt}\peng{a family}\hspace{5pt}\pcmn{一家人}\hspace{5pt}\pfra{une famille}\end{exemple}
\begin{exemple}\pnru{ʈʂʰɯ˧-ʑi˥}\hspace{5pt}\peng{this family}\hspace{5pt}\pcmn{这家}\hspace{5pt}\pfra{cette famille-ci}\end{exemple}
\begin{exemple}\pnru{ŋwæ˧-qʰv̩˧, | tsʰe˧ɲi˧-ʑi˩}\hspace{5pt}\peng{Five hamlets, twelve families! (A summary of the statistics of the village of /ə˧lɑ˧-ʁwɤ\#˥/.)}\hspace{5pt}\pcmn{五个村落,十二个家庭!(阿拉瓦村的人口简介)}\hspace{5pt}\pfra{Cinq hameaux, douze familles! (Formule résumant la statistique du village de /ə˧lɑ˧-ʁwɤ\#˥/.)}\end{exemple}
\end{entrée}

\begin{entrée}
{ʑi˩˥}{}{ⓔʑi˩˥}\formedesurface{ʑi˩˥}\newline
\classe{名词}\ton{LH}
\paradigme{\pcmn{:} \p{}}
\begin{définition}\peng{Monkey, ape.}\end{définition}
\begin{définition}\pcmn{猴子}\end{définition}
\begin{définition}\pfra{Singe.}\end{définition}
\begin{exemple}\pnru{ʑi˩ dzɯ˧-ze˩}\hspace{5pt}\peng{…has eaten (a/the) monkey}\hspace{5pt}\pcmn{吃了猴子}\hspace{5pt}\pfra{(sujet non spécifié: un tigre…) a mangé (le) singe}\end{exemple}
\begin{exemple}\pnru{ʑi˩ hwæ˧-ze˩}\hspace{5pt}\peng{…has bought (a/the) monkey}\hspace{5pt}\pcmn{买了猴子}\hspace{5pt}\pfra{(sujet non spécifié: quelqu'un) a acheté (le) singe}\end{exemple}
\end{entrée}

\newpage\caractère{†}

\begin{entrée}
{†ʑi˩˧}{}{ⓔ†ʑi˩˧}\formedesurface{--}\newline
\classe{名词}\ton{LM? LH?}\begin{définition}\peng{Building; houses.}\end{définition}
\begin{définition}\pcmn{房屋}\end{définition}
\begin{définition}\pfra{Maison, bâtiment d'habitation; monosyllabe extrait de l'expression /ʑi˩ tsʰi˩˥, | æ̃˩ tsʰi˧/ ‘bâtir une demeure': le schéma tonal, avec un verbe au ton MH, peut provenir d'un nom au ton LM ou LH.}\end{définition}
\begin{exemple}\pnru{ʑi˩ tsʰi˧˥, | æ̃˩ tsʰi˥}\hspace{5pt}\peng{to build a house, to build a home (set phrase)}\hspace{5pt}\pcmn{建房立家(固定词语)}\hspace{5pt}\pfra{Bâtir une demeure, bâtir un foyer (expression proverbiale)}\end{exemple}
\begin{exemple}\pnru{ʑi˩ tʰv̩˩˥}\hspace{5pt}\peng{to found a new home, to build a new house}\hspace{5pt}\pcmn{分家,建立自己的新房屋比如:孩子多,一个孩子建自己的房子)}\hspace{5pt}\pfra{Fonder une nouvelle demeure: lorsque la famille est nombreuse, un de ses membres peut bâtir sa propre demeure}\end{exemple}
\begin{exemple}\pnru{ʑi˩ qʰæ˧˥}\hspace{5pt}\peng{to demolish a house}\hspace{5pt}\pcmn{拆房子}\hspace{5pt}\pfra{démolir une demeure}\end{exemple}
\begin{exemple}\pnru{*ʑi˩ hwæ˧}\hspace{5pt}\peng{*to buy a house (example coined by the investigator, to investigate the monosyllable's potential to combine with other verbs, and its tone category; this example was refused by speaker F4)}\hspace{5pt}\pcmn{*买房(这个例子是调查者构造的,F4确定是不可以说的。造这个例子的目的有两个:看单音节词根“家”能不能跟其它动词结合,也试着确定它的调类。)}\hspace{5pt}\pfra{*acheter une demeure; forme non correcte, proposée pour voir dans quelle mesure le monosyllabe peut se combiner avec divers verbes.}\end{exemple}
\end{entrée}

\newpage\caractère{ʑ}

\begin{entrée}
{ʑi˩β}{}{ⓔʑi˩β}\formedesurface{ʑi˩˥}\newline
\classe{动词}
\sens{1}
\begin{définition}\peng{To clutch, to grasp, to grab.}\end{définition}
\begin{définition}\pcmn{拿,捉 (捉鸡)}\end{définition}
\begin{définition}\pfra{Attraper, saisir, prendre (ex.: un animal récalcitrant).}\end{définition}
\begin{exemple}\pnru{æ˩ ʑi˧}\hspace{5pt}\peng{to catch a chicken}\hspace{5pt}\pcmn{捉鸡}\hspace{5pt}\pfra{attraper (un/le) poulet}\end{exemple}
\begin{exemple}\pnru{æ˩˥ | le˧-ʑi˩}\hspace{5pt}\peng{to catch a chicken}\hspace{5pt}\pcmn{捉鸡}\hspace{5pt}\pfra{attraper (un/le) poulet}\end{exemple}
\begin{exemple}\pnru{hĩ˧ ʑi˧˥}\hspace{5pt}\peng{to grab someone}\hspace{5pt}\pcmn{抓人}\hspace{5pt}\pfra{attraper quelqu'un}\end{exemple}
\begin{exemple}\pnru{ʁæ˧ ʑi˧}\hspace{5pt}\peng{to hold sb in one's arms (arm over the neck)}\hspace{5pt}\pcmn{搂(用胳膊搂脖子)}\hspace{5pt}\pfra{passer le bras autour du cou de quelqu'un}\end{exemple}\sens{2}
\begin{définition}\peng{To take along, to bring along.}\end{définition}
\begin{définition}\pcmn{带、拿过来}\end{définition}
\begin{définition}\pfra{Amener, prendre avec soi.}\end{définition}
\begin{exemple}\pnru{tso˧∼tso˧ ʑi˧˥}\hspace{5pt}\peng{to bring something}\hspace{5pt}\pcmn{带东西过来}\hspace{5pt}\pfra{attraper quelque chose}\end{exemple}
\end{entrée}

\begin{entrée}
{ʑi˧dv̩˧}{}{ⓔʑi˧dv̩˧}\formedesurface{ʑi˧dv̩˧}\newline
\classe{名词}
\sens{1}\paradigme{\pcmn{:} \p{}}
\begin{définition}\peng{The household.}\end{définition}
\begin{définition}\pcmn{家}\end{définition}
\begin{définition}\pfra{Maisonnée.}\end{définition}
\begin{exemple}\pnru{ɖɯ˧-ʑi˩dv̩˩}\hspace{5pt}\peng{a household}\hspace{5pt}\pcmn{一家人,包括所有成员}\hspace{5pt}\pfra{une maisonnée/ toute la maison}\end{exemple}
\begin{exemple}\pnru{ɑ˩ʁo˧-ʑi˧dv̩˧ ʝi˧}\hspace{5pt}\peng{to take care of the house, to look after the household, to be in charge of the household}\hspace{5pt}\pcmn{管家}\hspace{5pt}\pfra{s'occuper de la maison, veiller au bon fonctionnement de la maison}\end{exemple}\sens{2}\paradigme{\pcmn{:} \p{}}
\begin{définition}\peng{The entire farmhouse, comprising the main house and the other buildings.}\end{définition}
\begin{définition}\pcmn{农舍,包括院子、人住的楼、动物住的楼等}\end{définition}
\begin{définition}\pfra{L'ensemble de la ferme, comprenant plusieurs bâtiments, le bétail et les gens.}\end{définition}
\end{entrée}

\begin{entrée}
{ʑi˧dv̩˧ʝi˧-hĩ\#˥}{}{ⓔʑi˧dv̩˧ʝi˧-hĩ\#˥}\formedesurface{ʑi˧dv̩˧ʝi˧hĩ˧}\newline
\classe{名词}\ton{\#H}\begin{définition}\peng{The person in charge of the house, the master/mistress of the house.}\end{définition}
\begin{définition}\pcmn{一家之主、家长}\end{définition}
\begin{définition}\pfra{La personne qui s'occupe de la maison, le maître/la maîtresse de céans.}\end{définition}
\end{entrée}

\begin{entrée}
{ʑi˩hṽ̩\#˥}{}{ⓔʑi˩hṽ̩\#˥}\formedesurface{ʑi˩hṽ̩˥}\newline
\classe{名词}\ton{LM+\#H}
\paradigme{\pcmn{:} \p{}}
\begin{définition}\peng{Body hair (of humans).}\end{définition}
\begin{définition}\pcmn{人身上的毛}\end{définition}
\begin{définition}\pfra{Poils corporels.}\end{définition}
\end{entrée}

\begin{entrée}
{ʑi˧kv̩˧wo˧}{}{ⓔʑi˧kv̩˧wo˧}\formedesurface{ʑi˧kv̩˧wo˧}\newline
\classe{名词}\ton{M}
\paradigme{\pcmn{:} \p{}}
\begin{définition}\peng{Roof.}\end{définition}
\begin{définition}\pcmn{房顶}\end{définition}
\begin{définition}\pfra{Toit.}\end{définition}
\end{entrée}

\begin{entrée}
{ʑi˩-kʰv̩˧˥}{₁}{ⓔʑi˩-kʰv̩˧˥ⓗ1}\formedesurface{ʑi˩kʰv̩˧˥}\newline
\classe{名词}\ton{LM+MH\#}
1\begin{définition}\peng{Year of the Monkey.}\end{définition}
\begin{définition}\pcmn{猴年}\end{définition}
\begin{définition}\pfra{Année du Singe.}\end{définition}
\end{entrée}

\begin{entrée}
{ʑi˩-kʰv̩˧˥}{₂}{ⓔʑi˩-kʰv̩˧˥ⓗ2}\formedesurface{ʑi˩kʰv̩˧˥}\newline
\classe{形容词}\ton{LM+MH\#}
2\begin{définition}\peng{Born in the year of the Monkey.}\end{définition}
\begin{définition}\pcmn{属猴}\end{définition}
\begin{définition}\pfra{Né l'année du Singe.}\end{définition}
\end{entrée}

\begin{entrée}
{ʑi˧mi˧}{}{ⓔʑi˧mi˧}\formedesurface{ʑi˧mi˧}\newline
\classe{名词}
\sens{1}\paradigme{\pcmn{:} \p{}}
\begin{définition}\peng{The main building of the house/farm: the building where the hearth is located.}\end{définition}
\begin{définition}\pcmn{家里有火塘的那个房子(“祖母房”)}\end{définition}
\begin{définition}\pfra{Le bâtiment de la maison où se trouve le foyer.}\end{définition}\sens{2}
\begin{définition}\peng{The entire house, the entire farm.}\end{définition}
\begin{définition}\pcmn{整个家园}\end{définition}
\begin{définition}\pfra{L'ensemble de la maison; l'ensemble de la ferme.}\end{définition}
\end{entrée}

\begin{entrée}
{ʑi˩mi\#˥}{}{ⓔʑi˩mi\#˥}\formedesurface{ʑi˩mi˥}\newline
\classe{名词}\ton{LM+\#H}
\paradigme{\pcmn{:} \p{}}
\begin{définition}\peng{Female monkey.}\end{définition}
\begin{définition}\pcmn{母猴}\end{définition}
\begin{définition}\pfra{Singe femelle.}\end{définition}
\end{entrée}

\begin{entrée}
{ʑi˧mv̩˧˥}{}{ⓔʑi˧mv̩˧˥}\formedesurface{ʑi˧mv̩˧˥}\newline
\classe{名词}\ton{MH\#}
\paradigme{\pcmn{:} \p{}}
\begin{définition}\peng{Dream.}\end{définition}
\begin{définition}\pcmn{梦}\end{définition}
\begin{définition}\pfra{Rêve.}\end{définition}
\begin{exemple}\pnru{ʑi˧mv̩˧ qʰwɤ˧˥}\hspace{5pt}\peng{to have a dream}\hspace{5pt}\pcmn{做梦}\hspace{5pt}\pfra{faire un rêve}\end{exemple}
\begin{exemple}\pnru{ʑi˧mv̩˧ sɯ˧}\hspace{5pt}\peng{to sleep-walk (somnambulism); also: to speak in one's dreams}\hspace{5pt}\pcmn{梦游,梦呓}\hspace{5pt}\pfra{être somnambule; parler dans son sommeil}\end{exemple}
\begin{exemple}\pnru{njɤ˧ | ə˧hwɤ˧ | ʑi˧mv̩˥ | mɤ˧-dʑɤ˩!}\hspace{5pt}\peng{I have not had good dreams yesterday night! / I had a nightmare yesterday night!}\hspace{5pt}\pcmn{我昨天做了恶梦!}\hspace{5pt}\pfra{J'ai fait un cauchemar hier!}\end{exemple}
\end{entrée}

\begin{entrée}
{ʑi˧ŋɤ˥}{}{ⓔʑi˧ŋɤ˥}\formedesurface{ʑi˧ŋɤ˥}\newline
\classe{动词}\ton{H\#}\begin{définition}\peng{To doze off, to nod.}\end{définition}
\begin{définition}\pcmn{打瞌睡}\end{définition}
\begin{définition}\pfra{Somnoler.}\end{définition}
\begin{exemple}\pnru{ʑi˧ŋɤ˥-ze˩}\hspace{5pt}\peng{|fg{pfv}}\hspace{5pt}\pcmn{|fg{pfv}}\hspace{5pt}\pfra{|fg{pfv}}\end{exemple}
\end{entrée}

\begin{entrée}
{ʑi˧ŋv̩˥}{}{ⓔʑi˧ŋv̩˥}\formedesurface{ʑi˧ŋv̩˥}\newline
\classe{动词}\ton{H\#}\begin{définition}\peng{To sleep.}\end{définition}
\begin{définition}\pcmn{睡觉}\end{définition}
\begin{définition}\pfra{Dormir.}\end{définition}
\begin{exemple}\pnru{le˧-ʑi˧ŋv̩˥}\hspace{5pt}\peng{|fg{accomp}}\hspace{5pt}\pcmn{|fg{accomp}}\hspace{5pt}\pfra{|fg{accomp}}\end{exemple}
\begin{exemple}\pnru{ʑi˧ŋv̩˥-ho˩}\hspace{5pt}\peng{is going to sleep}\hspace{5pt}\pcmn{要睡了}\hspace{5pt}\pfra{qui va s'endormir, qui est ensommeillé}\end{exemple}
\end{entrée}

\begin{entrée}
{ʑi˩pʰv̩\#˥}{}{ⓔʑi˩pʰv̩\#˥}\formedesurface{ʑi˩pʰv̩˥}\newline
\classe{名词}\ton{LM+\#H}
\paradigme{\pcmn{:} \p{}}
\begin{définition}\peng{Male monkey.}\end{définition}
\begin{définition}\pcmn{公猴}\end{définition}
\begin{définition}\pfra{Singe mâle.}\end{définition}
\end{entrée}

\begin{entrée}
{ʑi˧qʰwɤ˧}{}{ⓔʑi˧qʰwɤ˧}\formedesurface{ʑi˧qʰwɤ˧}\newline
\classe{名词}\ton{M}
\paradigme{\pcmn{:} \p{}}
\begin{définition}\peng{Building; houses.}\end{définition}
\begin{définition}\pcmn{房屋}\end{définition}
\begin{définition}\pfra{Bâtiment, bâtiment d'habitation, pièce d'habitation; en naxi, a aussi le sens de ‘maisonnée’.}\end{définition}
\begin{exemple}\pnru{ʑi˧qʰwɤ˧ gv̩˩}\hspace{5pt}\peng{to build a building}\hspace{5pt}\pcmn{建房}\hspace{5pt}\pfra{bâtir une maison}\end{exemple}
\begin{exemple}\pnru{ʑi˧qʰwɤ˧-lɑ˧ do˥!}\hspace{5pt}\peng{One can only see buildings! (A comment by the consultant about the city of Lijiang: the plain is now thoroughly covered by buildings, and one cannot see fields anymore, unlike in Yongning, where until recently there were only a few hamlets scattered among a landscape of groves, pastures, and cultivated fields.)}\hspace{5pt}\pcmn{只看到房子! / 能看见的只有房子!(合作者说,丽江市区都是房子,看不到田。这一点,不像永宁坝:二十世纪的永宁,只有一些小村落分散在一大片田地中。)}\hspace{5pt}\pfra{on ne voit que des maisons/des bâtiments! (commentaires au sujet de la ville de Lijiang, où on ne voit pas les champs, à la différence de la plaine de Yongning: campagne où il y avait peu de maisons et de grands espaces cultivés.}\end{exemple}
\begin{exemple}\pnru{ɕjo˩ɕjɤ˩-ʑi˩qʰwɤ˥}\hspace{5pt}\peng{the building(s) of the school, the school buildings}\hspace{5pt}\pcmn{学校的楼(‘学校’:汉语借词)}\hspace{5pt}\pfra{les bâtiments de l'école (du chinois 学校 ‘école’)}\end{exemple}
\begin{exemple}\pnru{ʑi˧qʰwɤ˧ ʈʂʰv̩˩}\hspace{5pt}\peng{to paint a house}\hspace{5pt}\pcmn{给房子刷颜色(直译:‘染房’)}\hspace{5pt}\pfra{peindre une maison; littéralement: ‘teindre une maison’}\end{exemple}
\begin{exemple}\pnru{ʑi˧qʰwɤ˧ tɕʰi˧-hĩ˧ kʰv̩˥mi˩}\hspace{5pt}\peng{watchdog}\hspace{5pt}\pcmn{看门狗}\hspace{5pt}\pfra{chien de garde, chien qui garde la maison}\end{exemple}
\begin{exemple}\pnru{ʑi˧qʰwɤ˧ tɕʰi˧-hĩ˧ kʰv̩˥}\hspace{5pt}\peng{watchdog}\hspace{5pt}\pcmn{看门狗}\hspace{5pt}\pfra{chien de garde, chien qui garde la maison}\end{exemple}
\end{entrée}

\begin{entrée}
{ʑi˧ʁæ˥\$}{}{ⓔʑi˧ʁæ˥\$}\formedesurface{ʑi˧ʁæ˥}\newline
\classe{名词}\ton{H\$}
\paradigme{\pcmn{:} \p{}}
\begin{définition}\peng{The back of the house, the space behind the house.}\end{définition}
\begin{définition}\pcmn{房屋的上后方}\end{définition}
\begin{définition}\pfra{L'espace situé derrière la maison: entre le bâtiment et les murs de la ferme.}\end{définition}
\begin{exemple}\pnru{ɲi˧ʈʂæ˧ʑi˧-ʁo˧tʰo˥, | ʑi˧ʁæ˧ ɲi˥ mæ˩!}\hspace{5pt}\peng{The place behind the two-storey building is called ‘ʑi˧ʁæ˥\$'!}\hspace{5pt}\pcmn{两层楼房后面(这块地方)叫做“房屋的上后方”! / 房屋背后(这块地方)叫做“房屋的上后方”!}\hspace{5pt}\pfra{Derrière le bâtiment à deux étages, c'est ‘ʑi˧ʁæ˥\$'! / Derrière le bâtiment à deux étages, il y a ce qu'on appelle ‘l'espace derrière la maison'!}\end{exemple}
\end{entrée}

\begin{entrée}
{ʑi˧ʁo˥\$}{}{ⓔʑi˧ʁo˥\$}\formedesurface{ʑi˧ʁo˥}\newline
\classe{名词}\ton{H\$}
\paradigme{\pcmn{:} \p{}}
\begin{définition}\peng{Bed.}\end{définition}
\begin{définition}\pcmn{床}\end{définition}
\begin{définition}\pfra{Lit (le couchage entier).}\end{définition}
\end{entrée}

\begin{entrée}
{ʑi˩zo\#˥}{}{ⓔʑi˩zo\#˥}\formedesurface{ʑi˩zo˥}\newline
\classe{名词}\ton{LM+\#H}
\paradigme{\pcmn{:} \p{}}
\begin{définition}\peng{Infant monkey/ape.}\end{définition}
\begin{définition}\pcmn{小猴子}\end{définition}
\begin{définition}\pfra{Petit singe.}\end{définition}
\end{entrée}

\end{multicols}
\end{document}