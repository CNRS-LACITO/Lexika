\pagenumbering{roman}

{\LARGE \textbf{\pcmn{前言}}}
	\section{ \pcmn{缘起}} \label{sec:language}

 \pcmn{ 在北纬27.50度,东经100.41度的交界点上生活着一个知名的族群——摩梭人。它奇特的民族风俗使它名扬天下,但是,它的语言研究还远远跟不上它的名气。作为一名语音学家找到了摩梭语言就如同进入了阿拉丁的宝库:摩梭语音的丰富多彩,尤其是声调在整个语法系统里所起的核心作用,独特而又迷人。
 	本词典的全部信息来自于一位充满独特个人魅力的摩梭阿妈——拉它米打史拉么。1950年出生的这位阿妈,差不多与新中国同年龄,她身上的故事就是摩梭人今天与昨天的缩影。在云南省丽江市宁蒗县永宁乡平静村独自养育了四个儿女,其中一位就是摩梭著名学者拉他咪达石(拉他咪王勇)。有幸结识拉他咪达石先生与他亲爱的母亲是这本词典得以与世人见面的关键,在此向他们表示最深切的感谢。}
 
因为方言(土语)之间差异大,所以严格以一个地方,一个人的发音为标准音。如:粗绳索,永宁阿拉瓦村是bæ˧mi˧。这个词语‘鸭子’同音。而左所舍夸村的摩梭话里,bæ˧mi˧表示鸭子,bæ˧mi˩才表示粗绳索,大绳索。

	\section{ \pcmn{词典}} \label{sec:method}

 \pcmn{2006年在纳西语语音研究告一段落之后,我开始了对摩梭语音的分析。一开始的初衷只是进行音系分析,但却惊奇地发现摩梭声调不仅是单纯的音系问题,同时与语法有着千丝万缕的联系。这一发现,促使我开始对摩梭话进行全方位的研究。研究方法以搜集来的长篇语料为蓝本,在搜集语料的同时进行音系实验(系统的机械问答),并以长篇语料中的词汇为基准,汇集成为本部词典。词典目前的条目数量不到三千,但我的初衷并不致力于单纯的词汇收集而是希望能在有限的条目中最大限度地呈现摩梭语音的独特性。

本词典以国际音标为注音基础。声调标注系统由于比较复杂,因此启用了一套专门符号,该符号的用法及说明见}Michaud 2015\pcmn{。

本词典有三种呈现格式,在线词典、PDF文本、与Toolbox资料库。内容将陆续进行更正及删补。同时欢迎读者的指正与批评。来信请寄}alexis.michaud@cnrs.fr \pcmn{。}

\section{ \pcmn{其它}} \label{sec:other}

 \pcmn{著名语言学家孙天心提出,语言学家有三件宝:长篇语料、词典与参考语法。以这个标准来看,长篇语料就是我囊中最大的宝贝,历时十年搜集而来的近二十个小时的资料及逐句标注翻译,已经全部在线公开。在线地址:“泛语资料库”(}Pangloss Collection \pcmn{),}lacito.vjf.cnrs.fr/pangloss/languages/Na\_en.htm \pcmn{。下一步的相关工作,是完成一本《摩梭话声调研究》,初稿也已上线,地址:}https://halshs.archives-ouvertes.fr/halshs-01094049/document \pcmn{。}

	\section{ \pcmn{致谢}} \label{sec:thks}
	
\pcmn{拉它米打史拉么阿妈及全家人、拉他咪达石(摩梭知名学者,出版有多部摩梭文化研究专著)、李德静(丽江市东巴文化研究院院长)、黄行(中央社会科学院民族学研究所所长)、段炳昌(云南大学人文学院院长)、木霁弘(云南大学茶马古道文化研究所所长)、王卫东(云南大学人文学院中文系系主任)、和学光(云南党校图书馆馆长)、赵燕珍(云南大学人文学院中文系教授)、杨立权(云南大学人文学院中文系教授)、丁思之教授(普米语研究专家)、拉木嘎吐萨(摩梭知名学者)、阿慧(摩梭人研究生)、}Céline Buret\pcmn{(工程师)、}Séverine Guillaume\pcmn{(工程师)、向柏霖(}Guillaume Jacques\pcmn{,语言学专家)、
李力(}Liberty Lidz\pcmn{,摩梭话研究专家)、杜玫瑰(}Roselle Dobbs\pcmn{,摩梭研究者)、}Pascale-Marie Milan\pcmn{(人类学家,摩梭文化研究专家)、}Christine Mathieu\pcmn{(人类学家,摩梭文化研究专家):感谢你们的大力支持。

2016年,云南大学硕士阿慧(摩梭人)核对了词典1至23页(条目的百分之八)。

在此也向成书过程中曾给予帮助的许多朋友和专家们一并致谢!}




	\section{ \pcmn{参考书目}} \label{sec:refs}
	\begin{itemize}
		\item \pcmn{李子鹤. 纳西语言研究回顾——兼论语言在文化研究中的基础地位[J]. 茶马古道研究期刊, 2015, 4: 125–131.}
		\item \pcmn{孙天心. 藏缅语的调查[J]. 语言学论丛, 2007, 36: 98–107.}
		\item LIDZ L. A descriptive grammar of Yongning Na (Mosuo) [D]. Austin: University of Texas, Department of linguistics, 2010.
		\pcmn{(下载地址:}https://repositories.lib.utexas.edu/bitstream/handle/2152/ETD-UT-2010-12-2643/LIDZ-DISSERTATION.pdf\pcmn{)}
		\item MICHAUD A. Phrasing, prominence, and morphotonology: How utterances are divided into tone groups in Yongning Na [J]. Bulletin of Chinese Linguistics, 2015. \pcmn{(下载出版前版本的地址:}https://halshs.archives-ouvertes.fr/halshs-01162331\pcmn{)}
	\end{itemize}

\cleardoublepage
%\pagenumbering{arabic}
%\setcounter{page}{1}
 

	

