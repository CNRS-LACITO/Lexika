\pagenumbering{roman}
	{\LARGE \textbf{Introduction}}

This dictionary documents the lexicon of the Na language (\ipa{nɑ˩-ʐwɤ˥}) as spoken in and around the plain of Yongning, located in Southwestern China, at the border between Yunnan and Sichuan, at a latitude of 27°50’ N and a longitude of 100°41’ E. This language is known in China as ‘Mosuo'. 

To write a dictionary is to describe a language's lexical structures (Alex François). Progressing from a word list to a full-fledged dictionary requires delving into meaning, attempting to delineate connotations, polysemy, and relationships between words. Most dictionary entries could easily be expanded into essays. This dictionary is still at a relatively early stage xxxx


	\section{Broader research agenda} \label{sec:researchagenda}
	
	This dictionary is the outcome of linguistic fieldwork, an endeavour which consists in “going into a~community
	where a~language is spoken, collecting data from fluent native speakers, analysing the data, and
	providing a~comprehensive description, consisting of grammar, texts and dictionary”
	\citep[12]{Dixon2007}. The grammar, texts and dictionary are referred to as the “Boasian trilogy” \citep{foley1999} by reference to Franz Boas’s foundational work collecting North American languages \citep{boas1902, boasetal1911}; the trilogy can now be said to have become a \textit{tetralogy} as it integrates a multimedia component: audio and video recordings \citep{musgraveetal2014}. 
	
	My fieldwork on Yongning Na began in October 2006. A list of words was begun through elicitation, and gradually expanded and corrected as narratives were recorded and transcribed; addition of new words was therefore a slow process. An advantage of placing the emphasis on text collection is that a context is available to help clarify the meaning of newly encountered words, also offering a~basis for further discussion of their usage with language consultants. Systematic elicitation of large amounts of vocabulary was not carried out, hence the limited number of entries: currently on the order of 3,000.
	%currently \total{compteur}. 

\section{A review of Na language studies}
\label{sec:previousstudiesofthenalanguage}

This review of the literature about the Na language is summarized (and updated) from \citet[14-21]{michaud2017}. 

\subsection{Information about Na in the \textit{Brief description of the Naxi language}}

He Jiren \& Jiang Zhuyi’s \citeyear{heetal1985} \textit{Brief description of the Naxi language} (in Chinese) mainly focuses on the dialects spoken in the Lijiang plain, but the volume includes a~word list of Yongning Na (referred to as ‘Yongning plain \textit{patois} of the Eastern dialect of Naxi'), as well as some observations on phonology,
syntax, and dialectal diversity (pp. 107--116; see also \citealt{jiang1993}). The transcription is not phonemicized, and may not be fully consistent. Only four tones are
transcribed over monosyllables: LM (\ipa{˩˧}), M (\ipa{˧}), ML (\ipa{˧˩}), and H (\ipa{˥}), whereas
the analysis presented in \citet[53-92]{michaud2017} brings out six categories (LM, LH, M, L, H and MH). He \& Jiang based their linguistic research on
an~{analogy} with \ili{Naxi}, a~language which both of them could speak: He Jiren as a~native speaker, Jiang
Zhuyi as a~second"=language learner. \ili{Naxi} has a~four"=way tonal opposition over monosyllables:
High, Mid, Low (realized phonetically as low"=falling),
and Rising.  
There are also some issues with He \& Jiang’s transcription of vowels and consonants, as is to be
expected of initial field notes. Nasality is transcribed only in two syllables, /\ipa{xĩ}/ (as in
‘man’, transcribed /\ipa{xĩ˧}/; my data: //\ipa{hĩ˥}//) and /\ipa{ɣə̃r}/ (the only example is ‘bone’,
transcribed /\ipa{ʂa˧ɣə̃r˧}/; my transcription: //\ipa{ʂæ˩ɻ̍̃˩}//), whereas my investigation brings out eight nasal rhymes \citep[461-464]{michaud2017}. Another point of difference is that He \&
Jiang do not transcribe the uvular consonants reported here. Such
discrepancies may be due to the fact that the variety described by He \& Jiang had fewer phonemes
than that described here; but it is not implausible that they failed to distinguish some sounds that
were in fact contrastive. Conversely, some vowel differences transcribed by He \& Jiang may be spurious. The word list
contains examples of /\ipa{li}/ (as in /\ipa{li˧}/ ‘to look’) and /\ipa{lie}/ (as in /\ipa{lie˩˧}/
‘tea’). In my data ‘tea’ and ‘to look’ have the same initial and rhyme. The vowel /\ipa{i}/ is
slightly diphthongized towards [\ipa{e}], and thus close to [\ipa{lie}], which explains why it could be
sometimes heard as [\ipa{i}] and sometimes as [\ipa{ie}] before the investigator’s ear attunes to the
vowel system of Yongning Na. Once again, it is also theoretically possible that these two words did
not have the same phonemes in the dialect investigated by He \& Jiang.

\subsection{A~study of kinship terms, with phonetic observations: \citet{fu1980}}
\label{sec:fu1980astudyofkinshipterms}

The linguist Fu Maoji \zh{傅懋勣} visited Yongning in May and June 1979 with He Jiren and Jiang Zhuyi. He collected data in the
village of /\ipa{dʑɤ˩bv̩˧-ʁwɤ˩}/ (Jiabowa \zh{甲波瓦}) for a~study about kinship terms, presented at the 12th International Conference on {Sino"=Tibetan} Languages and Linguistics (Paris,
1979), then published in Chinese and in {French} translation (\citealt{fu1980,fu1983}). The article has an~appendix containing notes about phonetic transcription. 

\subsection{An outline of Yongning Na by Yang Zhenhong (\citeyear{yang2009})}
\label{sec:yang2009}

An outline of Yongning Na was published by Yang Zhenhong \zh{杨振洪}, a~speaker of this language
from /\ipa{ə˧bv̩˧-ʁwɤ˧}/ village (Abuwa \zh{阿布瓦村}), close to the current location of
the Yongning high school (original publication in Chinese: \citealt{yang2006d}; {English} translation
by Liberty Lidz, with improvements made after consulting with the author, published as \citealt{yang2009}). This outline by and large follows the structure of
He \& Jiang’s description of Naxi. 

\subsection{Lexical materials}
\label{sec:dictionary2013}

\textit{An anthology of everyday words and expressions in the Mosuo language} \citep{zhibaetal2013} presents vocabulary and expressions arranged by semantic field. The authors are a~native speaker from the Lake Lugu area and a~doctor in linguistics from Yunnan University. Their fieldwork is described as covering the Yongning plain and the Lake area, but with the Yongning plain as the main research area (p. 2). 

Approximations in phonetic notation are so numerous that they make the volume unreliable as a~work of reference. Voicing contrasts were challenging for the linguist in the team, whose training was mainly focused on the theory and practice of teaching Chinese as a~foreign language. Thus, the name of the mountain /\ipa{kɤ˧mv̩˧˥}/ is transcribed as /\ipa{gə⁵⁵mu⁵⁵}/, with a~voiced initial (p. 17 and elsewhere). The mountain's name in Chinese, \textit{Gemu} \zh{格姆山}, may have exerted an influence here. Conversely, the adjective /\ipa{dʑɤ˩\textsubscript{b}}/ ‘good’ is transcribed as /\ipa{tɕɑ¹³}/, with an unvoiced initial. Some phonemes, such as uvulars, are absent from the notations. 

\subsection{Liberty \citet{lidz2010}, \textit{A descriptive grammar of Yongning Na (Mosuo)}}
\label{sec:lidz2010}

Liberty Lidz’s
Ph.D.\ dissertation \citep{lidz2010}, \textit{A descriptive grammar of Yongning Na}
(\textit{Mosuo}), concerns the variety of Yongning Na spoken in the village of Luoshui \zh{落水},
on the shore of Lake Lugu. The dissertation, based on in"=depth fieldwork, provides a~description of the morphosyntax
of the language, and contains 150 pages of
transcribed and annotated narratives.


	\section{Chronology and method} \label{sec:method}

Unless otherwise stated, all the data are from one language consultant, Mrs. Latami Dashilame (\ipa{lɑ˧tʰɑ˧mi˥ ʈæ˧ʂɯ˧-lɑ˩mv˩}; Chinese: \zh{拉它米打史拉么}). She was born in 1950 in the hamlet called \ipa{ə˧lɑ˧-ʁwɤ\#˥} in Na, close to the monastery of Yongning. The administrative coordinates of this village are: Yúnnán province, Lìjiāng municipality, Nínglàng Yí autonomous county, Yǒngníng district, Ālāwǎ village (\zh{云南省丽江市宁蒗彝族自治县永宁乡阿拉瓦村}). The choice to work in one location only, and essentially with one consultant, is, again, based on the investigator's focus on the tone system. There is considerable dialectal diversity within the Na area (much more so than in the Naxi-speaking area); the tone systems of different villages are conspicuously different, and this geographical diversity combines with dramatic differences across social groups, and across generations. The obvious thing to do seemed to be an in-depth description and analysis of the language as spoken by one person (simultaneously making a few forays into other idiolects and dialects). Data from other speakers are indicated using their codes in the author's database of speakers of Naish languages. Table \ref{tab:consul} provides the speaker codes.

\begin{table}[H]
	\caption{Language consultant codes}
	\centering \label{tab:consul}
	\begin{tabular}{lllllll}
		\toprule
		speaker code &   name &  year of birth \\
		\midrule
			F4 (main consultant) & \ipa{lɑ˧tʰɑ˧mi˥ ʈæ˧ʂɯ˧-lɑ˩mv˩} & 1950 \\ 
			F5 &  \ipa{ki˧zo˧} & 1973  \\ 
			F6 &  \ipa{tɕʰi˧ɖv\#˥} & 1987 \\ 
			M18 &  \ipa{lɑ˧tʰɑ˧mi˥ ʈæ˧ʂɯ˧-ʈæ˩ʈv˩} & 1972 \\ 
			M21 & \ipa{ho˧dʑɤ˧tsʰe˥} & 1942 \\ 
			M23 & \ipa{ɖɯ˩ɖʐɯ˧} & 1974 \\
		\bottomrule
	\end{tabular}
\end{table}

The list of words as of 2011 was deposited in the STEDT database (http://stedt.berkeley.edu/). The same year, under the impetus of Guillaume Jacques and Aimée Lahaussois, plans were made to bring the word list closer to the standards of a full-fledged dictionary. A project was deposited with the Agence Nationale de la Recherche, accepted in 2012, and begun in 2013: the HimalCo project (ANR-12-CORP-0006). Céline Buret, a computing engineer, worked with the project team for two years (Nov. 2014-Oct. 2015). She converted the data to the format of the Field Linguist's Toolbox (MDF), then produced scripts for conversion to a XML format complying with the Lexical Markup Framework standard (LMF), allowing for automatic conversion to an online format as well as to LaTeX files (with PDF as the final output for circulation). The scripts constitute a Python 2 library called PyLMFlib, for: \textit{Python LMF library}. In 2015, version 1.0 of the online and PDF versions of the dictionary were produced and published online, along with the source document in MDF (Toolbox) format.

In 2016, Benjamin Galliot, working at CNRS-LACITO under a fixed-term (six-month) contract funded by CNRS (Délégation Paris-Villejuif), wrote a~new library using Python 3, which allows for native management of the Unicode standard. He also wrote a XSL script for generating the PDF versions of the dictionary. The new version of the dictionary released in the year 2017 (version 1.1) now has 

\begin{itemize}
	\item \textit{a phonetic transcription of tone as it surfaces on the item pronounced in isolation:} a surface-phonological transcription of tone, in addition to the indication of the underlying tone category
	\item \textit{a romanized representation:} a proposed spelling, devised by Roselle Dobbs and her collaborators (see more below)
	\item \textit{more cross-references} between entries, pointing to synonyms, etc.
\end{itemize}

\section{The Lexica series of dictionaries: format and vision} \label{sec:format}

The Na dictionary is one of the first in the Lexica series. \textit{Lexica} is the term adopted in 2017 for the dictionaries of the Pangloss Collection, an archive of (mostly) rare and endangered languages \citep{michailovskyetal2014}. The Lexica series aims to combine \textit{readability} (for users who browse through the dictionaries) with \textit{computer-readable encoding} (suitable for Natural Language Processing). 

	\subsection{Usefulness and limitations of the LMF format} \label{sec:lmf}

The Lexical Markup Framework \citep{francopoulo2013} is a pivotal format designed for machine-readable dictionaries. It is mostly compatible with SIL’s MDF format (Multi-Dictionary Formatter, used in particular by the Toolbox software). A limitation is that it places a constraint on what an entry can contain: subentries that belong to different grammatical categories need to have separate entries set up. For instance, \ipa{lɑ˧-kʰv̩˧˥} can mean both ‘year of the Tiger’ and ‘born in the year of the Tiger’ (in Chinese: \zh{虎年} and \zh{属虎}). From the linguist's point of view, it is desirable to set up two subentries within the same entry. But the part"=of"=speech categories are different (‘year of the Tiger’ is a noun phrase, and ‘born in the year of the Tiger’ is a predicate, categorized as an adjective), which, in the Lexical Markup Framework, necessitates setting up two different entries. 

\section{Guide to using the dictionary} \label{sec:howto}

	\subsection{Formats: trilingual Na-Chinese-English or Na-Chinese French} \label{sec:versions}

Entries and examples have translations into English, Chinese and French. Two language settings are offered for the PDF and online dictionary: either Na-Chinese-English, or Na-Chinese-French. The English and the French are not typeset alongside each other in the same document because distinguishing them visually is not obvious, even with the help of typographic devices such as using different fonts and colours. In the author's own experience, it was found that the presence of four languages alongside one another made consultation more difficult; specifically, English translations tended to be a distraction slowing down access to the French translations, and English users may similarly find that French clutters the layout. On the other hand, Chinese characters are visually well-distinct from Latin-based scripts, and so it did not appear necessary to separate the Chinese and produce a Na-Chinese version. Moreover, Chinese translations are often a useful complement to the translation in English (or French), as there are often closer equivalents: for instance \ipa{gɤ˧˥} translates straightforwardly as Chinese \zh{扛} whereas the English translation is more roundabout: ‘to carry on the shoulder'. Users who wish to have access to all the information can download the original file in Toolbox (MDF) format. 

	\subsection{Format of entries} \label{sec:entries}

Each entry contains
\begin{itemize}
	\item \textit{phonological transcription:} the form of the word in phonetic alphabet; tone is indicated in terms of phonological categories
	\item \textit{surface form:} the surface"=phonological form of the word in phonetic alphabet. Tone is indicated in terms of tonal realizations. This form can be read by anyone with a knowledge of the IPA, without requiring an understanding of the mapping of underlying phonological tone categories to surface tone in Yongning Na. 
	\item \textit{proposed orthographic representation:} notation in a~Romanized script devised by Roselle Dobbs with Mosuo collaborators. 
	\item \textit{part of speech:} an indication of the part of speech, using a simple set of labels
	\item \textit{tone:} the tone category of the word. This information is already present in the phonological transcription; having it repeated on its own facilitates searches
	\item \textit{definitions} in Chinese and English
	\item \textit{examples} with translations
	\item \textit{links} to related words, such as synonyms, or constituent parts of complex words 
	\item \textit{classifier:} for nouns, an indication on the more commonly associated classifiers
\end{itemize}

For surface"=phonological forms, the aim was to achieve greatest simplicity. Special symbols used in the word's underlying phonological form are removed from the surface form: dashes indicating junctures internal to the word, and tilde in reduplicated forms. For classifiers, the numeral ‘one' is added before the classifier, because classifiers are not free forms: they cannot be said on their own. For other bound morphemes, such as affixes and clitics, no surface form is indicated.

The proposed orthography in Latin alphabet was added by Roselle Dobbs (in the Summer of 2017). The transcription was developed by R. Dobbs with Na consultants, with a view to use within the Na community. Importantly, this is not a transliteration of the phonology: phonological forms are all based on the dialect of consultant F4, whereas proposed orthographic representations are intended by R. Dobbs and her collaborators as a cross"=dialect writing system. Given the high degree of dialectal diversity, proposing a transcription that is acceptable for speakers of several dialects implies some compromises. For instance, only the low tone (L), presumed to be more stable, is indicated in the orthography. Until a full description of this romanized writing system is published, requests for information about orthographic developments should be directed to Roselle Dobbs (rosellemay@hotmail.com). 

Among examples, those elicited to verify the output of certain combinations of tones are marked as ‘PHONO': examples elicited for the purpose of the phonological study. Proverbs and sayings are marked as ‘PROVERB'.

Some pieces of information are not shown in the PDF and online versions. These are:
\begin{itemize}
	\item An indication of \textit{semantic domain}: ‘society', ‘house', ‘body', ‘plant', ‘animal'... No attempt was made to use a fine-grained classification of the sort found in the WordNet database of English, where nouns, verbs, adjectives and adverbs are grouped into sets of cognitive synonym \citep{Fellbaum 2005}. This is simply a rough division into subsets for convenient sorting; the labelling relies partly on form, and partly on semantic contents. As for other aspects in the dictionary, choices made reflect the investigator's research priorities: for instance, the entries for ‘day’, ‘night’, ‘month’, ‘year’ were tagged as “classifiers", along with all other nouns that can appear immediately after a numeral. This allowed easy extraction of all classifiers for the purpose of a study of the tone patterns of classifiers \citep{Michaud2013}. These lexical items could just as well have been tagged as 'time', in view of their semantic field. The numbers ‘100’, ‘1,000’ and ‘10,000’ were likewise labelled as “classifiers" rather than numerals.
	\item \textit{Notes on past notations:} information tracing the history of notations, from the first fieldwork to the current version. For instance, the entry \ipa{ŋwɤ˧pʰæ˧˥} ‘tile' has a note that indicates that it was initially written with a M.H tone pattern, and with vowel \ipa{æ} in both syllables: *\ipa{ŋwæ˧pʰæ˥}. The note explains that the perception of \ipa{æ} in the first syllable is due to a phonetic tendency towards regressive vowel harmony. Verifications are also consigned in this field. About half the entries have information of this type.
	\item \textit{glosses:} glosses in English, Chinese and French, intended for the glossing of texts. The dictionary adopts the abbreviations recommended in the Leipzig Glossing Rules \citep{Comrie}; all other terms are provided in full. Glosses mostly follow the choices made by \citep{Lidz2010}.
\end{itemize}

	\subsection{Part-of-speech labelling} \label{sec:pos}
	
Dictionary entries carry a part-of-speech label. A rough-and-ready typology has been followed: see the table below. Needless to say, this system has limitations: a refined typology would require subcategories, e.g. defining classifiers as a subset among nouns; and categories such as ‘adverbs' raise greater difficulties, lacking a clear definition.
\begin{table}
	\caption{Parts of speech}
	\centering \label{tab:glosses}
	\begin{tabular}{lll}
		\toprule
		label & meaning & Leipzig Glossing Rules? y/n \\
		\midrule
		adj & adjective & y \\
		clf & classifier & y \\
		clitic & (same) & n \\
		cnj & conjunction & n \\
		ideophone & (same) & n \\
		disc.PTCL & discourse particle & n \\
		intj & interjection & n \\
		lnk & linker & n \\
		n & noun & y \\
		num & numeral & n \\
		pref & prefix & n \\
		postp & postposition & n \\
		pro & pronoun & n \\
		suff & suffix & n \\
		v & verb & y \\
		\bottomrule
	\end{tabular}
\end{table}

No attempt was made at including expressive noises in the dictionary, such as the sound \ipa{ɬː}. The meaning of this sound in Na can be characterized in the same way as that of words in the dictionary: the full definition would be that  \textit{it expresses enjoyment of food or drink (‘Yummy!'), and is also used to express admiration of a beautiful object, scene, or prospect}. But a reason for leaving it out is that, unlike interjections,  \ipa{ɬː} is not pronounced on expiratory airflow, but on inspired airflow. The air flows through the sides of the mouth, which is where saliva flows when one's mouth waters. Observations about such sounds (including clicks), like that of gestures, appeared to fall outside the scope of the dictionary.

	\subsection{Roots extracted from disyllables} \label{sec:roots}

xxxx Some roots can be extracted from disyllables. They are indicated by the symbol †. No surface form is provided, as these words are not in use in the language.

	\subsection{Words for which no Na equivalent was found} \label{sec:zeroanswer}
	
	The list that follows groups words for which no close equivalents could be found. These negative pieces of information contain hints about the consultants’ Na vocabulary and its ’soft shoulders’.
	
	English & Chinese & local Chinese name & French & comment
	Selaginella sp. & 卷柏 & & Selaginella sp. & F4 recognizes pictures of this plant, which is abundant in the mountains. But she does not have a name for it.
	hog, sand badger & 猪獾、狗獾 & 臭猪子 & blaireau &
	lynx & 猞猁& & lynx &
	pangolin & 穿山甲& & pangolin &
	earlobe & 耳垂 & & lobe de l'oreille & The word for 'ear' is used: there is no specific name for the earlobe.
	temples (on the head) & 太阳穴 & & tempes & In F4's Na, the temples are part of the forehead, /to˧kɤ#˥/. 
	wild man, savage & 野人 & & yéti & This is not part of local folklore. 
	Selaginella sp. & 卷柏 & & Selaginella sp. & F4 recognizes pictures of this plant, which is abundant in the mountains. But she does not have a name for it.
	Chinese angelica & 当归 & & Angelica sp. & 
	Anisodus tanguticus & 山茛菪 & & Angelica sp. & 
	Ligularia fischeria & 山紫菀 & & Ligularia fischeria & 
	dried cheese in thin, translucid slices & 乳扇 & & fromage séché en feuilles translucides & None was produced or consumed in Yongning in F4's lifetime.
	traditional toothbrush, made of pig hair & 用猪鬃毛作成的牙刷 & & brosse à dent traditionnelle en poil de cochon & Not in use in Yongning in F4's lifetime.
	catapult & 抛石机,弹弓 & & catapulte & Unknown to F4.
	slingshot & 绷弓子 & & lance-pierre, tire-boulettes (partie rigide en Y, et partie élastique permettant de catapulter un petit objet) & Unknown to F4.


	\subsection{Loanwords} \label{sec:loan}
	
Borrowings from Chinese and Tibetan are indicated as such in cases where identification seems straightforward. No efforts at systematic elicitation of borrowings from either language were made, but all loanwords occurring in texts were added to the dictionary. The information provided includes: donor language; form in the donor language; and explanations. When the number of syllables in the borrowed word is the same as in the donor language, the glosses in English (and French) start by the original word followed by two colons and a translation: e.g. ‘\zh{办法}::solution' for \ipa{pæ˧˥hwɤ˧}. 


	\section{Planned improvements and mid-term perspectives} \label{sec:improv}
	
This dictionary is conceived of as work-in-progress: successive versions will be released, probably every two years or so, (i) as an online dictionary in HTML format, (ii) as PDF documents, and (iii) in database format (native Toolbox/MDF format, then, in due course, the successors of this format). 

Planned improvements for future versions include the addition of
\begin{itemize}
	\item \textit{audio files for each head word:} this function has successfully been tested, but the editing of audio files still needs to be conducted
	\item \textit{links to the entire set of online recordings}: listing all textual occurrences in the lexicon entry, with links to the audio file and its aligned transcription. Textual occurrences ultimately constitute the best resource to document a word's usage. The examples currently presented in the dictionary are few in number, compared to the occurrences in texts; and their context of use may not be clear, despite efforts at clarifying their nature (singling out examples elicited for the purpose of morpho-phonological investigation by the mention ‘PHONO') and at providing contextual information for examples jotted down during fieldwork.
	\item \textit{additional cross-references} between entries, pointing to synonyms, etc.
	\item \textit{more Chinese loans:} indicating the Na pronunciation of Chinese words that are now commonly used by speakers of Na.
\end{itemize}

Collaborations are welcome for the following improvements :
\begin{itemize}
	\item \textit{the vocabulary of religion:} the field of religion remains mostly unexplored; the main consultant and I both lack the command of Tibetan that would be essential for this part of the investigation, and involvement of consultants from the Yongning monastery did not prove feasible in view of current restrictions on contacts with foreigners
	\item \textit{plants and animals:} as a dweller of the plain, the main consultant does not have extensive knowledge of wild plants and animals; the number of entries recorded so far remains small, and some definitions are currently limited to general indications such as ‘a type of pine'. To arrive at exact identification, and at more extensive lexicographic coverage, would require collaboration with other consultants, and with botanists.
\end{itemize}

	\section{Other resources about Yongning Na} \label{sec:resources}
	
	In the classical tradition of linguistic fieldwork \citep{Dixon2007}, a language description should include a dictionary, a grammar, and a collection of texts. 
	
	\begin{itemize}
		\item \textit{A set of Na recordings with time-aligned transcriptions} is available from the Pangloss Collection \citep{Michailovsky2014}; the current web address is lacito.vjf.cnrs.fr/pangloss/languages/Na\_en.htm 
		\item \textit{The grammar} is still in its early stages of preparation. A preliminary draft of a book-length study of Na morpho-tonology can be found online: https://halshs.archives-ouvertes.fr/halshs-01094049/document It also contains detailed information on the phonemic analysis.
	\end{itemize}
	
A review of the literature about Na and the other languages of the Naish  group is provided (in Chinese) by \citet{Li2015}. For an English-language introduction, see \citet{Michaud2015b}.

I would gratefully receive any comments or notifications of errors that the reader may wish to bring to my attention: please send e-mail to michaud.cnrs@gmail.com 


	\section{Acknowledgments} \label{sec:ackno}

Many thanks to Picus Ding for putting me in touch with the Mosuo scholar Latami Dashi. Special thanks to Latami Dashi for supporting and encouraging my work with his mother Latami Dashilame over the years. Many thanks to the main consultant, Latami Dashilame, and to all family members. 

Many thanks to Céline Buret and Séverine Guillaume for their much-appreciated computational expertise, and to Guillaume Jacques for suggestions all along the way. Many thanks to connoisseurs of the Na culture and language for useful exchanges: Lamu Gatusa \zh{拉木嘎吐萨} (Chinese pen-name: \zh{石高峰}), Liberty Lidz, Christine Mathieu, Pascale-Marie Milan and He Sana \zh{何梭娜}. Special thanks to Roselle Dobbs for extensive discussions and vigorous proof-reading over the years. Many thanks to A Hui \zh{阿慧} (to my knowledge the first speaker of Mosuo to read a M.A. degree in language and linguistics) for suggesting corrections. Remaining errors are my own responsibility. 

I am grateful for the opportunity allowed me by my home institution, Centre National de la Recherche Scientifique, of staying in China in 2011-2012 for extensive fieldwork, through a temporary affiliation with the CNRS’s research centre in China: CEFC (Centre d’Etudes Français sur la Chine contemporaine). From November 2012 to June 2016, I was based at the international research institute MICA, in Hanoi, in an exceptionally stimulating environment allowing for close collaboration with colleagues from Asia and elsewhere. Special thanks to the heads of the institute, Phạm Thị Ngọc Yến (succeeded in 2015 by Nguyễn Việt Sơn) and Eric Castelli, for their support and encouragement.

I am grateful to the Dongba Culture Research Institute (\zh{丽江市东巴文化研究院}) in Lijiang and the Horse-Tea Road Culture Research Centre (\zh{云南大学茶马古道文化研究所}) in Kunming for inviting me to become an Adjunct member (\zh{外聘研究员}), and for facilitating administrative and practical matters; special thanks to Li Dejing \zh{李德静} and to Mu Jihong\zh{木霁弘}. At Yunnan University, many thanks are due to Duan Bingchang \zh{段炳昌}, Wang Weidong \zh{王卫东}, Zhao Yanzhen \zh{赵燕珍} and Yang Liquan \zh{杨立权} for their careful and sensitive management of fieldwork-related administrative matters.
	
So many people have supported this project that I must apologize for those names that should be here but were inadvertently left off the list.

This work was supported financially by the ANR project HimalCo (ANR-12-CORP-0006), and constitutes a contribution to the LabEx “Empirical Foundations of Linguistics" project (ANR-10-LABX-0083).

\begin{thebibliography}{7}
	\providecommand{\natexlab}[1]{#1}
	\providecommand{\url}[1]{#1}
	\providecommand{\urlprefix}{}
	\expandafter\ifx\csname urlstyle\endcsname\relax
	\providecommand{\doi}[1]{doi:\discretionary{}{}{}#1}\else
	\providecommand{\doi}{doi:\discretionary{}{}{}\begingroup
		\urlstyle{rm}\Url}\fi
	
	\bibitem[{Li(2015)}]{Li2015}
	Li Zihe [\zh{李子鹤}]. 2015.
	\newblock
	\zh{纳西语言研究回顾------兼论语言在文化研究中的基础地位}
	[{A} review of {Naxi} language studies, with a discussion of the fundamental
	role of cultural studies for linguistic research].
	\newblock \zh{茶马古道研究期刊} 4. 125--131.
	
	\bibitem[{Comrie et~al.()Comrie, Haspelmath \& Bickel}]{Comrie}
	Comrie, Bernard, Martin Haspelmath \& Balthasar Bickel. 2008.
	\newblock Leipzig {Glossing Rules}.
	\newblock
	\urlprefix\url{http://www.eva.mpg.de/lingua/resources/glossing-rules.php}.
	
	\bibitem[{Dixon(2007)}]{Dixon2007}
	Dixon, Robert~M. 2007.
	\newblock Field linguistics: a minor manual.
	\newblock \emph{Sprachtypologie und Universalienforschung} 60(1). 12--31.
	
	\bibitem[{Lidz(2010)}]{Lidz2010}
	Lidz, Liberty. 2010.
	\newblock \emph{A descriptive grammar of {Yongning Na} ({Mosuo})}.
	\newblock Austin: University of Texas, Department of linguistics dissertation.
	\newblock
	\urlprefix\url{https://repositories.lib.utexas.edu/bitstream/handle/2152/ETD-UT-2010-12-2643/LIDZ-DISSERTATION.pdf}.
	\newblock Ph. D.
	
	\bibitem[{Michailovsky et~al.(2014)Michailovsky, Mazaudon, Michaud, Guillaume,
		Fran{\c{c}}ois \& Adamou}]{Michailovsky2014}
	Michailovsky, Boyd, Martine Mazaudon, Alexis Michaud, S{\'{e}}verine Guillaume,
	Alexandre Fran{\c{c}}ois \& Evangelia Adamou. 2014.
	\newblock Documenting and researching endangered languages: the {Pangloss
		Collection}.
	\newblock \emph{Language Documentation and Conservation} 8. 119--135.
	\newblock \urlprefix\url{http://hdl.handle.net/10125/4621}.
	
	\bibitem[{Michaud(2013)}]{Michaud2013}
	Michaud, Alexis. 2013.
	\newblock The tone patterns of numeral-plus-classifier phrases in {Yongning
		Na}: a synchronic description and analysis.
	\newblock In Nathan Hill \& Tom Owen-Smith (eds.), \emph{Transhimalayan
		{Linguistics}. {Historical} and {Descriptive} {Linguistics} of the
		{Himalayan} {Area}} (Trends in {Linguistics}. {Studies} and {Monographs}
	[{TiLSM}] 266), 275--311. Berlin: De Gruyter Mouton.
	
	\bibitem[{Michaud et~al.(2015)Michaud, Limin \& Yaoping}]{Michaud2015b}
	Michaud, Alexis, He~Limin \& Zhong Yaoping. 2015.
	\newblock Naxi / {Naish}.
	\newblock In Rint Sybesma, Wolfgang Behr, Zev Handel \& C.T.~James Huang
	(eds.), \emph{Encyclopedia of {Chinese} {Language} and {Linguistics}},
	Leiden: Brill.
	
\end{thebibliography}

\cleardoublepage
\pagenumbering{arabic}
\setcounter{page}{1}