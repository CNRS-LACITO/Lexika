
\documentclass[twoside,11pt]{article}
\title{Essai d’un dictionnaire alphabétique}
\author{Isabelle Bril}
\usepackage[paperwidth=185mm,paperheight=260mm,top=16mm,bottom=16mm,left=15mm,right=20mm]{geometry}
\usepackage{multicol}
\setlength{\columnseprule}{1pt}
\setlength{\columnsep}{1.5cm}
\usepackage{titlesec}
\usepackage{changepage}
\usepackage[dvipsnames,table]{xcolor}
\usepackage{fancyhdr}
\pagestyle{fancy}
\fancyheadoffset{3.4em}
\fancyhead[LE,LO]{\rightmark}
\fancyhead[RE,RO]{\leftmark}
\usepackage{hyperref}
\hypersetup{pdftex,bookmarks=true,bookmarksnumbered,bookmarksopenlevel=5,bookmarksdepth=5,xetex,colorlinks=true,linkcolor=blue,citecolor=blue}
\usepackage[all]{hypcap}
\usepackage{fontspec}
\usepackage{natbib}
\usepackage{booktabs}
\usepackage{polyglossia}
\setdefaultlanguage{french}
\setmainfont{Liberation Serif}
\newfontfamily{\déf}[Mapping=tex-text,Ligatures=Common,Scale=MatchUppercase]{Liberation Serif}
\newfontfamily{\nua}[Mapping=tex-text,Ligatures=Common,Scale=MatchUppercase]{Charis SIL}
\newfontfamily{\fra}[Mapping=tex-text,Ligatures=Common,Scale=MatchUppercase]{EB Garamond}
\newcommand{\pdéf}[1]{\déf #1}
\newcommand{\pfra}[1]{\fra #1}
\newcommand{\pnua}[1]{\nua #1}
\newcommand{\cerclé}[1]{\raisebox{0pt}{\textcircled{\raisebox{-0.5pt} {\footnotesize{\pdéf{#1}}}}}}
\newcommand{\lettrine}[1]{\phantomsection\addcontentsline{toc}{section}{#1}{\begin{center}\textbf{\Large\pnua{#1}}\end{center}}}
\newenvironment{entrée}[3]{\hypertarget{#3}{}\phantomsection\addcontentsline{toc}{subsection}{#1\homonyme{#2}}\hspace*{-1cm}\textbf{\Large\pnua{\textcolor{Sepia}{#1}~\homonyme{#2}}}\markright{#1\homonyme{#2}}}{}
\newenvironment{sous-entrée}[2]{\hypertarget{#2}{}\phantomsection\addcontentsline{toc}{subsubsection}{#1}\pdéf{■}~\textbf{\large\pnua{\textcolor{Sepia}{#1}}}}{}
\newenvironment{glose}{\pdéf{◊}}{}
\newenvironment{exemple}{\pdéf{¶}}{}
\newcommand{\nomscientifique}[1]{\emph{#1}}
\newcommand{\homonyme}[1]{\textcolor{Red}{#1}}
\newcommand{\formephonétique}[1]{/\pnua{#1}/}
\newcommand{\région}[1]{\textcolor{Green}{[#1]}}
\newcommand{\variante}[1]{\textcolor{Sepia}{(#1)}}
\newcommand{\groupe}[1]{\cerclé{#1}}
\newcommand{\classe}[1]{\pfra{\textcolor{Blue}{\emph{#1}}}}
\newcommand{\sens}[1]{\cerclé{#1}}
\newcommand{\domainesémantique}[1]{\pfra{\textit{#1}}}
\newcommand{\relationsémantique}[2]{\pfra{\emph{#1}~:~}\pnua{\textcolor{Sepia}{#2}}}
\newcommand{\emprunt}[1]{\pfra{Empr.~:~}#1}
\newcommand{\étymologie}[1]{\pfra{Étym.~:~}#1}
\newcommand{\morphologie}[1]{\pfra{Morph.~:~}#1}
\newcommand{\langue}[1]{\pfra{#1}}
\newcommand{\étymon}[1]{\textbf{#1}}
\newcommand{\glosecourte}[1]{'#1'}
\newcommand{\auteur}[1]{d'après \textsc{#1}}
\newcommand{\note}[3]{\textit{Note (#2)~:~#1 '#3'}}
\newcommand{\lien}[2]{\hyperlink{#1}{\pnua{#2}}}
\setcounter{secnumdepth}{4}
\titleformat{\paragraph}
{\normalfont\normalsize\bfseries}{\theparagraph}{1em}{}
\titlespacing*{\paragraph}
{0pt}{3.25ex plus 1ex minus .2ex}{1.5ex plus .2ex}\begin{document}
�introduction
\begin{multicols}{2}
\lhead{\firstmark}
\rhead{\botmark}\newpage

\lettrine{
a
ã
}\begin{entrée}{a}{1}{ⓔaⓗ1}
\région{GOs}
\variante{%
ò
\région{BO PA}}
(\domainesémantique{Verbes de déplacement et moyens de déplacement})
\classe{v}
\begin{glose}
\pfra{partir ; aller}
\end{glose}
\newline
\begin{sous-entrée}{a-da}{ⓔaⓗ1ⓝa-da}
\begin{glose}
\pfra{monter}
\end{glose}
\end{sous-entrée}
\newline
\begin{sous-entrée}{a-du}{ⓔaⓗ1ⓝa-du}
\begin{glose}
\pfra{descendre}
\end{glose}
\end{sous-entrée}
\newline
\begin{sous-entrée}{a-mi}{ⓔaⓗ1ⓝa-mi}
\begin{glose}
\pfra{venir (vers le locuteur)}
\end{glose}
\end{sous-entrée}
\newline
\begin{sous-entrée}{a-e}{ⓔaⓗ1ⓝa-e}
\begin{glose}
\pfra{s'éloigner (du locuteur)}
\end{glose}
\end{sous-entrée}
\newline
\begin{sous-entrée}{a-mwa-mi}{ⓔaⓗ1ⓝa-mwa-mi}
\begin{glose}
\pfra{revenir}
\end{glose}
\end{sous-entrée}
\newline
\begin{sous-entrée}{a-mwa-e}{ⓔaⓗ1ⓝa-mwa-e}
\begin{glose}
\pfra{repartir}
\end{glose}
\end{sous-entrée}
\newline
\begin{sous-entrée}{a hayu}{ⓔaⓗ1ⓝa hayu}
\begin{glose}
\pfra{aller n'importe où}
\end{glose}
\newline
\begin{exemple}
\région{BO PA}
\textbf{\pnua{ò hayu}}
\pfra{aller n'importe où}
\end{exemple}
\newline
\begin{exemple}
\région{BO PA}
\textbf{\pnua{ò maya}}
\pfra{aller doucement, ralentir}
\end{exemple}
\newline
\note{ò [BO, PA] : forme de "a" en composition}{grammaire}{}
\end{sous-entrée}
\end{entrée}

\begin{entrée}{a}{2}{ⓔaⓗ2}
\région{GOs}
\variante{%
hai, ha, ai
\région{PA BO}}
(\domainesémantique{Conjonction})
\classe{CNJ}
\begin{glose}
\pfra{ou bien}
\end{glose}
\newline
\begin{exemple}
\textbf{\pnua{inu a ijö}}
\pfra{moi ou toi}
\end{exemple}
\end{entrée}

\begin{entrée}{a}{3}{ⓔaⓗ3}
(\domainesémantique{Pronoms})
\classe{PRO 2° pers. PL (sujet)}
\begin{glose}
\pfra{vous}
\end{glose}
\end{entrée}

\begin{entrée}{a}{4}{ⓔaⓗ4}
\région{GOs}
\variante{%
al
\région{PA BO}}
(\domainesémantique{Astres})
\classe{nom}
\begin{glose}
\pfra{soleil ; beau temps}
\end{glose}
\newline
\begin{exemple}
\région{GO}
\textbf{\pnua{ulu a}}
\pfra{le soleil se couche}
\end{exemple}
\newline
\begin{sous-entrée}{gòòn-a}{ⓔaⓗ4ⓝgòòn-a}
\région{GO}
\begin{glose}
\pfra{midi, zénith}
\end{glose}
\newline
\begin{exemple}
\région{PA}
\textbf{\pnua{gòòn-al}}
\end{exemple}
\newline
\begin{exemple}
\textbf{\pnua{midi, zénith}}
\end{exemple}
\newline
\begin{exemple}
\région{PA}
\textbf{\pnua{waya èno al ? (ou) waya yino al ?}}
\end{exemple}
\newline
\begin{exemple}
\textbf{\pnua{quelle heure est-il?}}
\end{exemple}
\end{sous-entrée}
\newline
\étymologie{
\langue{POc}
\étymon{*qaco, *qaso}
\glosecourte{soleil}}
\end{entrée}

\begin{entrée}{a}{5}{ⓔaⓗ5}
\région{BO}
(\domainesémantique{Conjonction})
\classe{REL ou DEM}
\begin{glose}
\pfra{que}
\end{glose}
\newline
\begin{exemple}
\textbf{\pnua{eje we a nu kido-le kõnõbwõn}}
\pfra{voici l'eau que j'ai bue hier}
\end{exemple}
\newline
\begin{exemple}
\textbf{\pnua{i ra tòne a dili a i nyama}}
\pfra{il entend la terre qui bouge}
\end{exemple}
\end{entrée}

\begin{entrée}{a}{6}{ⓔaⓗ6}
\région{PA BO}
(\domainesémantique{Relateurs et relateurs possessifs})
\classe{voyelle euphonique}
\begin{glose}
\pfra{voyelle euphonique (parfois réalisée schwa)}
\end{glose}
\newline
\begin{exemple}
\région{BO}
\textbf{\pnua{ho-ny a lavian}}
\pfra{ma viande}
\end{exemple}
\newline
\begin{exemple}
\région{BO}
\textbf{\pnua{ho-ny a nò lana}}
\pfra{ce sont mes poissons à manger}
\end{exemple}
\newline
\note{cette voyelle euphonique apparaît dans la détermination}{grammaire}{}
\end{entrée}

\begin{entrée}{a-}{1}{ⓔa-ⓗ1}
\région{GO PA}
\variante{%
aa-
}
(\domainesémantique{Démonstratifs})
\classe{PREF.NMLZ (n. d'agent)}
\begin{glose}
\pfra{agent (préfixe des noms d')}
\end{glose}
\newline
\begin{sous-entrée}{a-vhaa}{ⓔa-ⓗ1ⓝa-vhaa}
\begin{glose}
\pfra{bavard}
\end{glose}
\end{sous-entrée}
\newline
\begin{sous-entrée}{a-palu}{ⓔa-ⓗ1ⓝa-palu}
\begin{glose}
\pfra{avare}
\end{glose}
\end{sous-entrée}
\newline
\begin{sous-entrée}{a-punò}{ⓔa-ⓗ1ⓝa-punò}
\région{GO}
\begin{glose}
\pfra{orateur}
\end{glose}
\end{sous-entrée}
\newline
\begin{sous-entrée}{aa-punòl}{ⓔa-ⓗ1ⓝaa-punòl}
\région{PA}
\begin{glose}
\pfra{orateur}
\end{glose}
\newline
\note{(selon Dubois) a- marque une action présente: nu a-puyòl 'je fais la cuisine' (lit. moi cuisinier)}{grammaire}{}
\end{sous-entrée}
\end{entrée}

\begin{entrée}{a-}{2}{ⓔa-ⓗ2}
\région{GOs PA}
\variante{%
aa-
}
(\domainesémantique{Préfixes classificateurs numériques})
\classe{CLF.NUM (animés)}
\begin{glose}
\pfra{classificateur des animés}
\end{glose}
\newline
\begin{exemple}
\textbf{\pnua{a-xe, a-tru, a-ko, a-pa, a-ni, a-ni-ma-xe kuau}}
\pfra{un, deux, trois, quatre, cinq, six chiens}
\end{exemple}
\newline
\relationsémantique{Cf.}{\lien{}{po-, go-, we-, pepo-}}
\glosecourte{classificateurs}
\end{entrée}

\begin{entrée}{-a}{}{ⓔ-a}
\région{GOs BO PA}
(\domainesémantique{Relateurs et relateurs possessifs})
\classe{REL.POSS}
\begin{glose}
\pfra{relateur possessif: de}
\end{glose}
\newline
\begin{exemple}
\région{GOs BO PA}
\région{GOs}
\textbf{\pnua{wòòdro-w-a êgu}}
\pfra{les discussions des gens}
\end{exemple}
\newline
\begin{exemple}
\région{GOs}
\textbf{\pnua{mõlò-w-a êgu}}
\pfra{les coutumesdes gens}
\end{exemple}
\newline
\begin{exemple}
\région{GOs}
\textbf{\pnua{nobwò-w-a êgu}}
\pfra{les tâches des gens}
\end{exemple}
\newline
\begin{exemple}
\région{GOs}
\textbf{\pnua{lòtò-w-a (la) êgu}}
\pfra{la voiture des gens}
\end{exemple}
\newline
\begin{exemple}
\région{GOs}
\textbf{\pnua{lòtò i nu}}
\pfra{ma voiture}
\end{exemple}
\newline
\begin{exemple}
\région{GOs}
\textbf{\pnua{lòtò i la êgu}}
\pfra{la voiture de ces gens}
\end{exemple}
\newline
\begin{exemple}
\région{PA}
\textbf{\pnua{doo-a-ny}}
\pfra{ma marmite}
\end{exemple}
\end{entrée}

\begin{entrée}{ã-}{}{ⓔã-}
\région{GOs WEM WE}
(\domainesémantique{Démonstratifs})
\classe{DEM}
\begin{glose}
\pfra{celui-ci (humain)}
\end{glose}
\newline
\begin{sous-entrée}{ã-èni}{ⓔã-ⓝã-èni}
\begin{glose}
\pfra{cet homme-là (DX2)}
\end{glose}
\end{sous-entrée}
\newline
\begin{sous-entrée}{ã-èba}{ⓔã-ⓝã-èba}
\begin{glose}
\pfra{cet homme-là (DX2 sur le côté)}
\end{glose}
\end{sous-entrée}
\newline
\begin{sous-entrée}{ã-õli}{ⓔã-ⓝã-õli}
\begin{glose}
\pfra{cet homme-là (DX3)}
\end{glose}
\end{sous-entrée}
\newline
\begin{sous-entrée}{ã-èdu mu}{ⓔã-ⓝã-èdu mu}
\begin{glose}
\pfra{cet homme-là derrière}
\end{glose}
\end{sous-entrée}
\newline
\begin{sous-entrée}{ã-èda}{ⓔã-ⓝã-èda}
\begin{glose}
\pfra{cet homme là-haut}
\end{glose}
\end{sous-entrée}
\newline
\begin{sous-entrée}{ã-èdu}{ⓔã-ⓝã-èdu}
\begin{glose}
\pfra{cet homme-là en bas}
\end{glose}
\end{sous-entrée}
\newline
\begin{sous-entrée}{ã-èbòli}{ⓔã-ⓝã-èbòli}
\begin{glose}
\pfra{cet homme-là loin en bas}
\end{glose}
\end{sous-entrée}
\end{entrée}

\begin{entrée}{-ã}{}{ⓔ-ã}
\région{GOs PA}
(\domainesémantique{Démonstratifs})
\classe{DEIC.1; ANAPH}
\begin{glose}
\pfra{ceci}
\end{glose}
\newline
\begin{exemple}
\région{PA}
\textbf{\pnua{nye ègu-ã}}
\pfra{cet homme-ci}
\end{exemple}
\newline
\begin{exemple}
\textbf{\pnua{ã ègu-ã}}
\pfra{cet homme-ci}
\end{exemple}
\newline
\begin{exemple}
\textbf{\pnua{ègumãli-ã}}
\pfra{ces deux personnes-ci}
\end{exemple}
\newline
\relationsémantique{Cf.}{\lien{}{-õli}}
\glosecourte{là (déict. distal)}
\newline
\relationsémantique{Cf.}{\lien{ⓔ-ba}{-ba}}
\glosecourte{là (déict, latéralement,visible)}
\newline
\relationsémantique{Cf.}{\lien{ⓔ-òⓗ1}{-ò}}
\glosecourte{là-bas (anaph)}
\end{entrée}

\begin{entrée}{-ãã}{}{ⓔ-ãã}
\formephonétique{ɛ̃ː}
\région{GOs PA BO}
(\domainesémantique{Pronoms})
\classe{PRO 1° pers. incl. PL (OBJ ou POSS)}
\begin{glose}
\pfra{nous ; nos}
\end{glose}
\end{entrée}

\begin{entrée}{aa-baatro}{}{ⓔaa-baatro}
\formephonétique{'aːbaːɽo}
\région{GOs}
(\domainesémantique{Caractéristiques et propriétés des personnes})
\classe{nom}
\begin{glose}
\pfra{paresseux ; fainéant}
\end{glose}
\newline
\relationsémantique{Syn.}{\lien{ⓔkônôô}{kônôô}}
\glosecourte{paresseux}
\end{entrée}

\begin{entrée}{aaleni}{}{ⓔaaleni}
\région{PA}
\variante{%
halèèni
\région{BO [Corne]}}
(\domainesémantique{Verbes d'action (en général)})
\classe{v.t.}
\begin{glose}
\pfra{dévier qqch.}
\end{glose}
\newline
\begin{exemple}
\région{BO}
\textbf{\pnua{nu halèèni we}}
\pfra{je dévie l'eau}
\end{exemple}
\end{entrée}

\begin{entrée}{aamu}{}{ⓔaamu}
\région{GOs}
\variante{%
amu
\région{PA}, 
ããmu
\région{BO}}
\newline
\sens{1}
(\domainesémantique{Insectes})
\classe{nom}
\begin{glose}
\pfra{mouche (grosse) [GOs]}
\end{glose}
\newline
\relationsémantique{Cf.}{\lien{}{ne phû}}
\glosecourte{mouche bleue}
\newline
\sens{2}
(\domainesémantique{Insectes})
\classe{nom}
\begin{glose}
\pfra{abeille [PA, BO]}
\end{glose}
\newline
\begin{sous-entrée}{pi-ããmu, we-aamu}{ⓔaamuⓢ2ⓝpi-ããmu, we-aamu}
\région{PA BO}
\begin{glose}
\pfra{miel}
\end{glose}
\end{sous-entrée}
\end{entrée}

\begin{entrée}{aa-palu}{}{ⓔaa-palu}
\formephonétique{aː'palu}
\région{GOs}
(\domainesémantique{Dons, échanges, achat et vente, vol})
\classe{v}
\begin{glose}
\pfra{avare, égoïste, qui ne partage pas}
\end{glose}
\end{entrée}

\begin{entrée}{aa-pubwe wè-ce}{}{ⓔaa-pubwe wè-ce}
\région{GOs}
(\domainesémantique{Poissons})
\classe{nom}
\begin{glose}
\pfra{poisson "million"}
\end{glose}
\nomscientifique{Poecilia reticulata (Poeciliidés)}
\end{entrée}

\begin{entrée}{aari}{}{ⓔaari}
\région{PA}
\variante{%
hari
\région{PA}}
(\domainesémantique{Aliments, alimentation})
\classe{nom}
\begin{glose}
\pfra{riz}
\end{glose}
\newline
\emprunt{riz (FR)}
\end{entrée}

\begin{entrée}{aava}{}{ⓔaava}
\formephonétique{'aːva}
\région{GOs}
\classe{v.stat.}
\newline
\sens{1}
(\domainesémantique{Processus liés aux plantes})
\begin{glose}
\pfra{vert ; pas mûr}
\end{glose}
\begin{glose}
\pfra{jeune (fruit)}
\end{glose}
\newline
\sens{2}
(\domainesémantique{Caractéristiques et propriétés des personnes})
\begin{glose}
\pfra{fragile (nourrisson)}
\end{glose}
\newline
\begin{exemple}
\textbf{\pnua{e aava phagoo-je}}
\pfra{son corps est fragile (se dit d'un nourrisson)}
\end{exemple}
\newline
\relationsémantique{Ant.}{\lien{ⓔzeenô}{zeenô}}
\glosecourte{mûr}
\end{entrée}

\begin{entrée}{aavhe}{}{ⓔaavhe}
\formephonétique{aːβe}
\région{PA BO}
\variante{%
avwe
\région{GO(s)}}
\classe{nom}
\newline
\sens{1}
(\domainesémantique{Société})
\begin{glose}
\pfra{étranger ; inconnu (personnes, objets)}
\end{glose}
\newline
\begin{exemple}
\région{GO}
\textbf{\pnua{aavwe-ã}}
\pfra{elle est étrangère pour nous (lit. notre étrangère)}
\end{exemple}
\newline
\begin{exemple}
\région{GO}
\textbf{\pnua{mèni ka aavwe}}
\pfra{un oiseau inconnu, jamais vu}
\end{exemple}
\newline
\sens{2}
(\domainesémantique{Société})
\begin{glose}
\pfra{clan venus de l'extérieur (pour des cérémonies)}
\end{glose}
\newline
\relationsémantique{Cf.}{\lien{}{apoxapenu, awoxavhenu}}
\glosecourte{personnes ou clans qui accueillent les 'aavhe'}
\end{entrée}

\begin{entrée}{aazi}{}{ⓔaazi}
\formephonétique{aːði}
\région{GOs}
(\domainesémantique{Poissons})
\classe{nom}
\begin{glose}
\pfra{"bossu doré"}
\end{glose}
\nomscientifique{Lethrinus atkinsoni (Lethrinidés)}
\end{entrée}

\begin{entrée}{aazo}{}{ⓔaazo}
\formephonétique{aːðo}
\région{GOs PA}
(\domainesémantique{Organisation sociale})
\classe{nom}
\begin{glose}
\pfra{chef ; grand-chef}
\end{glose}
\end{entrée}

\begin{entrée}{ãbaa}{1}{ⓔãbaaⓗ1}
\formephonétique{ɛ̃'mbaː}
\région{GOs}
\variante{%
ãbaa-n
\région{BO PA}}
(\domainesémantique{Quantificateurs})
\classe{QNT}
\begin{glose}
\pfra{autre (un, d') ; un autre}
\end{glose}
\begin{glose}
\pfra{un bout de}
\end{glose}
\begin{glose}
\pfra{certains ; quelques}
\end{glose}
\newline
\begin{exemple}
\textbf{\pnua{ãbaa-la}}
\pfra{certains d'entre eux}
\end{exemple}
\newline
\begin{exemple}
\région{PA}
\textbf{\pnua{ãbaa-la êgu}}
\pfra{certaines de ces personnes}
\end{exemple}
\newline
\begin{exemple}
\région{PA}
\textbf{\pnua{koen-xa ãbaa wony}}
\pfra{certains bateaux ont disparu}
\end{exemple}
\newline
\begin{exemple}
\région{PA}
\textbf{\pnua{koen-xa ãbaa êgu}}
\pfra{certaines personnes sont absentes}
\end{exemple}
\newline
\begin{exemple}
\région{GOs}
\textbf{\pnua{ge le xa ãbaa-we ne zoma a iò ne thrõbo}}
\pfra{certains d'entre vous partiront ce soir (tout à l'heure au soir)}
\end{exemple}
\newline
\begin{exemple}
\région{PA}
\textbf{\pnua{ge le xa ãbaa wony a kòen}}
\pfra{certains bateaux ont disparu}
\end{exemple}
\newline
\begin{exemple}
\région{GOs}
\textbf{\pnua{ge le ãbaa wõ xa la kòi-ò}}
\pfra{certains bateaux ont disparu}
\end{exemple}
\newline
\begin{exemple}
\textbf{\pnua{la a, novwö la abaa la tree yu}}
\pfra{ils sont partis, mais les autres sont restés}
\end{exemple}
\newline
\begin{exemple}
\région{BO}
\textbf{\pnua{i phe abaa la ko}}
\pfra{il a pris un des poulets}
\end{exemple}
\newline
\begin{exemple}
\textbf{\pnua{thu ãbaa mwani}}
\pfra{ajoute de l'argent}
\end{exemple}
\newline
\begin{exemple}
\textbf{\pnua{na ãbaa mwani}}
\pfra{donne plus d'argent}
\end{exemple}
\end{entrée}

\begin{entrée}{ãbaa-}{2}{ⓔãbaa-ⓗ2}
\formephonétique{ɛ̃'mbaː}
\région{GOs PA BO}
(\domainesémantique{Parenté})
\classe{nom}
\begin{glose}
\pfra{frère}
\end{glose}
\begin{glose}
\pfra{soeur}
\end{glose}
\begin{glose}
\pfra{cousins parallèles (enfants de soeur de mère, enfants de frère de père)}
\end{glose}
\newline
\begin{exemple}
\région{PA}
\textbf{\pnua{pe-ãbaa-la}}
\pfra{ils sont frères et soeurs}
\end{exemple}
\newline
\begin{exemple}
\région{PA}
\textbf{\pnua{ãbaa-kee-ny}}
\pfra{le frère de mon père}
\end{exemple}
\newline
\étymologie{
\étymon{*apə}
\auteur{Lynch}}
\end{entrée}

\begin{entrée}{ãbaa êmwê}{}{ⓔãbaa êmwê}
\région{GOs PA}
\variante{%
ãbaa êmwên whamã
\région{PA}}
(\domainesémantique{Parenté})
\classe{nom}
\begin{glose}
\pfra{frère}
\end{glose}
\newline
\begin{exemple}
\région{GO}
\textbf{\pnua{ãbaa-nu êmwê}}
\pfra{mon frère aîné}
\end{exemple}
\newline
\begin{exemple}
\région{PA}
\textbf{\pnua{ãbaa-ny êmwê}}
\pfra{mon frère aîné}
\end{exemple}
\end{entrée}

\begin{entrée}{ãbaa êmwê whamã}{}{ⓔãbaa êmwê whamã}
\région{GOs}
\variante{%
ãbaa êmwên whamã
\région{PA BO}}
(\domainesémantique{Parenté})
\classe{nom}
\begin{glose}
\pfra{frère aîné}
\end{glose}
\newline
\begin{exemple}
\textbf{\pnua{ãbaa-ny êmwên whamã [PA]}}
\pfra{mon frère aîné}
\end{exemple}
\end{entrée}

\begin{entrée}{ãbaa thoomwã}{}{ⓔãbaa thoomwã}
\région{GOsPA}
\variante{%
ãbaa thòòmwa, ãbaa-dòòmwa
\région{PA}}
(\domainesémantique{Parenté})
\classe{nom}
\begin{glose}
\pfra{soeur}
\end{glose}
\newline
\begin{exemple}
\région{GO}
\textbf{\pnua{ãbaa-nu thoomwã}}
\pfra{ma soeur}
\end{exemple}
\newline
\begin{exemple}
\région{PA}
\textbf{\pnua{ãbaa-ny thoomwã}}
\pfra{ma soeur}
\end{exemple}
\end{entrée}

\begin{entrée}{ãbaa thoomwã whamã}{}{ⓔãbaa thoomwã whamã}
\région{GOsPA}
\variante{%
ãbaa thòòmwa, ãbaa-dòòmwa whamã
\région{PA}}
(\domainesémantique{Parenté})
\classe{nom}
\begin{glose}
\pfra{soeur aînée}
\end{glose}
\end{entrée}

\begin{entrée}{ãbaa-xa}{}{ⓔãbaa-xa}
\formephonétique{̃ɛ̃mbaːɣa}
\région{GOs}
(\domainesémantique{Quantificateurs})
\classe{QNT}
\begin{glose}
\pfra{en complément ; en plus}
\end{glose}
\newline
\begin{exemple}
\textbf{\pnua{nu phe-mi hovwo vwo ãbaa-xa lhãã co pavwange}}
\pfra{j'ai apporté de la nourriture en complément de ce que tu as préparé}
\end{exemple}
\end{entrée}

\begin{entrée}{a-baxoo}{}{ⓔa-baxoo}
\région{GOs}
(\domainesémantique{Verbes de déplacement et moyens de déplacement})
\classe{v}
\begin{glose}
\pfra{aller tout droit}
\end{glose}
\end{entrée}

\begin{entrée}{ãbe}{}{ⓔãbe}
\région{GOs}
(\domainesémantique{Fonctions intellectuelles})
\classe{v}
\begin{glose}
\pfra{croire tout savoir}
\end{glose}
\newline
\begin{exemple}
\textbf{\pnua{e za ãbe òri !}}
\pfra{il croit tout savoir !}
\end{exemple}
\newline
\begin{exemple}
\textbf{\pnua{kebwa ãbe !}}
\pfra{arrête de faire celui qui sait tout !}
\end{exemple}
\newline
\begin{exemple}
\textbf{\pnua{jo za hine ãbe !}}
\pfra{tu crois tout savoir ! tu sais toujours tout !}
\end{exemple}
\end{entrée}

\begin{entrée}{a-bwaayu}{}{ⓔa-bwaayu}
\région{WEM}
(\domainesémantique{Caractéristiques et propriétés des personnes})
\classe{v}
\begin{glose}
\pfra{travailleur ; courageux}
\end{glose}
\end{entrée}

\begin{entrée}{a-çaabò}{}{ⓔa-çaabò}
\formephonétique{a'ʒaːbɔ}
\région{GOs}
\variante{%
ayabòl
\région{PA}, 
ayabwòl
\région{BO (Corne)}}
(\domainesémantique{Parenté})
\classe{nom}
\begin{glose}
\pfra{maternels ; parenté ou famille côté maternel}
\end{glose}
\newline
\begin{exemple}
\textbf{\pnua{a-çaabò i hâ}}
\pfra{notre parenté maternelle}
\end{exemple}
\newline
\relationsémantique{Cf.}{\lien{ⓔcabo}{cabo}}
\glosecourte{sortir, naître}
\end{entrée}

\begin{entrée}{a-choomu}{}{ⓔa-choomu}
\région{GOs}
(\domainesémantique{Fonctions intellectuelles})
\classe{nom}
\begin{glose}
\pfra{enseignant}
\end{glose}
\end{entrée}

\begin{entrée}{ã-da}{}{ⓔã-da}
\région{GOs}
(\domainesémantique{Verbes de déplacement et moyens de déplacement})
\classe{v.DIR}
\begin{glose}
\pfra{monter}
\end{glose}
\begin{glose}
\pfra{entrer dans une maison}
\end{glose}
\begin{glose}
\pfra{aller vers l'intérieur du pays}
\end{glose}
\begin{glose}
\pfra{aller en amont d'un cours d'eau ; sortir de l'eau, etc.}
\end{glose}
\newline
\begin{sous-entrée}{ã-daa-mi}{ⓔã-daⓝã-daa-mi}
\région{GOs}
\begin{glose}
\pfra{monter vers ici}
\end{glose}
\end{sous-entrée}
\newline
\begin{sous-entrée}{ã-da-ò}{ⓔã-daⓝã-da-ò}
\région{GOs}
\begin{glose}
\pfra{monter en s'éloignant}
\end{glose}
\newline
\begin{exemple}
\région{GOs}
\textbf{\pnua{nu ã-da Numia}}
\pfra{je vais à Nouméa}
\end{exemple}
\newline
\begin{exemple}
\région{GOs}
\textbf{\pnua{nu ã-da na Frans}}
\pfra{je reviens de France}
\end{exemple}
\newline
\begin{exemple}
\région{GOs}
\textbf{\pnua{nu ã-da na bwaabu}}
\pfra{je reviens d'en bas (= de France)}
\end{exemple}
\newline
\begin{exemple}
\région{GOs}
\textbf{\pnua{nu uja-da na Frans}}
\pfra{j'arrive de France}
\end{exemple}
\end{sous-entrée}
\end{entrée}

\begin{entrée}{ã-da-ò}{}{ⓔã-da-ò}
\région{GOs}
(\domainesémantique{Verbes de déplacement et moyens de déplacement})
\classe{v.DIR}
\begin{glose}
\pfra{monter (en s'éloignantdu locuteur)}
\end{glose}
\end{entrée}

\begin{entrée}{ã-du}{}{ⓔã-du}
\région{GOs PA}
(\domainesémantique{Verbes de déplacement et moyens de déplacement})
\classe{v.DIR}
\begin{glose}
\pfra{descendre}
\end{glose}
\begin{glose}
\pfra{sortir (de la maison)}
\end{glose}
\newline
\begin{sous-entrée}{ã-du-mi}{ⓔã-duⓝã-du-mi}
\begin{glose}
\pfra{descends vers moi}
\end{glose}
\end{sous-entrée}
\newline
\begin{sous-entrée}{ã-du-ò}{ⓔã-duⓝã-du-ò}
\begin{glose}
\pfra{descends en t'éloignant}
\end{glose}
\newline
\begin{exemple}
\textbf{\pnua{nu ã-du Frans}}
\pfra{je vais en France}
\end{exemple}
\newline
\begin{exemple}
\région{GOs}
\textbf{\pnua{nu ã-du Pum}}
\pfra{je vais à Poum}
\end{exemple}
\newline
\begin{exemple}
\région{GOs}
\textbf{\pnua{nu ã-du Aramwa}}
\pfra{je vais à Arama}
\end{exemple}
\newline
\begin{exemple}
\textbf{\pnua{e ã-du pwa}}
\pfra{il est sorti (de la maison)}
\end{exemple}
\newline
\relationsémantique{Ant.}{\lien{ⓔã-da}{ã-da}}
\glosecourte{entrer (dans la maison)}
\end{sous-entrée}
\end{entrée}

\begin{entrée}{a-è}{}{ⓔa-è}
\région{GOs}
(\domainesémantique{Verbes de déplacement et moyens de déplacement})
\classe{v.DIR}
\begin{glose}
\pfra{aller dans une direction transverse ; passer}
\end{glose}
\newline
\begin{exemple}
\textbf{\pnua{e a-è}}
\pfra{il va dans une direction transverse}
\end{exemple}
\newline
\begin{exemple}
\textbf{\pnua{nu a-è Pwebo}}
\pfra{je vais à Pouébo}
\end{exemple}
\newline
\begin{exemple}
\textbf{\pnua{la a-è Hiengen}}
\pfra{ils sont allés à Hienghiène}
\end{exemple}
\end{entrée}

\begin{entrée}{ã-e}{}{ⓔã-e}
\région{GOs}
(\domainesémantique{Démonstratifs})
\classe{PRO.DEIC.2 (3° pers. masc. SG)}
\begin{glose}
\pfra{lui-là}
\end{glose}
\end{entrée}

\begin{entrée}{aeke}{}{ⓔaeke}
\région{WE}
(\domainesémantique{Interjection})
\classe{INTJ}
\begin{glose}
\pfra{salut ! (se dit quand on est loin)}
\end{glose}
\end{entrée}

\begin{entrée}{aguko}{}{ⓔaguko}
\région{BO PA}
(\domainesémantique{Types de maison, architecture de la maison})
\classe{nom}
\begin{glose}
\pfra{solive verticale}
\end{glose}
\newline
\note{Pièce de bois réunissant la poutre maîtresse à la poutre de faîtage des maisons carrées (Dubois)}{glose}{}
\end{entrée}

\begin{entrée}{a-hãbu}{}{ⓔa-hãbu}
\région{GOs}
(\domainesémantique{Verbes de déplacement et moyens de déplacement})
\classe{v}
\begin{glose}
\pfra{marcher en tête ; passer devant}
\end{glose}
\end{entrée}

\begin{entrée}{a-hò}{}{ⓔa-hò}
\région{GOs}
\variante{%
a-ò
\région{GO(s)}}
(\domainesémantique{Verbes de déplacement et moyens de déplacement})
\classe{v.DIR}
\begin{glose}
\pfra{éloigner (s')}
\end{glose}
\begin{glose}
\pfra{partir}
\end{glose}
\begin{glose}
\pfra{aller (s'en)}
\end{glose}
\newline
\begin{exemple}
\textbf{\pnua{a-hò na inu !}}
\pfra{éloigne-toi de moi !}
\end{exemple}
\newline
\begin{exemple}
\textbf{\pnua{a-ò !}}
\pfra{va t'en ! (en s'éloignant du locuteur)}
\end{exemple}
\end{entrée}

\begin{entrée}{a-hõ}{}{ⓔa-hõ}
\région{GOs}
\variante{%
a-õ
\région{GO(s) BO}}
(\domainesémantique{Vêtements, parure})
\classe{v.stat.}
\begin{glose}
\pfra{nu}
\end{glose}
\newline
\begin{exemple}
\textbf{\pnua{e pe-a-õ}}
\pfra{il marche tout nu, sans rien}
\end{exemple}
\end{entrée}

\begin{entrée}{a-hovwo}{}{ⓔa-hovwo}
\région{GOs PA}
(\domainesémantique{Aliments, alimentation})
\classe{v}
\begin{glose}
\pfra{gourmand ; vorace}
\end{glose}
\end{entrée}

\begin{entrée}{a-hû-mi}{}{ⓔa-hû-mi}
\région{GOs}
(\domainesémantique{Verbes de déplacement et moyens de déplacement})
\classe{v}
\begin{glose}
\pfra{approcher (s')}
\end{glose}
\newline
\begin{exemple}
\textbf{\pnua{a-(h)û-mi !}}
\pfra{approche-toi !}
\end{exemple}
\end{entrée}

\begin{entrée}{ai-}{}{ⓔai-}
\région{GOs PA BO}
\variante{%
awi-
}
(\domainesémantique{Sentiments})
\classe{nom}
\begin{glose}
\pfra{coeur ; amour ; volonté ; envie de}
\end{glose}
\newline
\begin{exemple}
\région{GO}
\textbf{\pnua{ai-nu nye wô-ã}}
\pfra{j'ai envie de ce bateau, j'aime bien ce bateau}
\end{exemple}
\newline
\begin{exemple}
\région{GO}
\textbf{\pnua{ai-nu wö po pe-me-õ}}
\pfra{je voudrais y aller avec vous}
\end{exemple}
\newline
\begin{exemple}
\région{GO}
\textbf{\pnua{la êgu ai-la nye wô-ã}}
\pfra{j'ai envie de ce bateau, j'aime bien ce bateau}
\end{exemple}
\newline
\begin{exemple}
\région{GO}
\textbf{\pnua{a-xa wôjô-nu na wame nye}}
\pfra{j'ai envie d'un bateau qui soit comme ça}
\end{exemple}
\newline
\begin{exemple}
\région{GO}
\textbf{\pnua{ai-je}}
\pfra{son coeur}
\end{exemple}
\newline
\begin{exemple}
\région{GO}
\textbf{\pnua{ai-nu}}
\pfra{mon coeur}
\end{exemple}
\newline
\begin{exemple}
\région{PA}
\textbf{\pnua{phwe-ai-n}}
\pfra{son coeur}
\end{exemple}
\newline
\begin{exemple}
\région{PA}
\textbf{\pnua{ai-n}}
\pfra{il veut}
\end{exemple}
\newline
\begin{exemple}
\textbf{\pnua{ai amalaò mèèni (p)u la a-du kale}}
\pfra{les oiseaux veulent descendre à la pêche}
\end{exemple}
\newline
\begin{exemple}
\région{PA}
\textbf{\pnua{ai-m da ? avu-yu phe nya hele ?}}
\pfra{que veux-tu ? désires-tu prendre ce couteau ?}
\end{exemple}
\newline
\begin{exemple}
\région{PA}
\textbf{\pnua{ai-ny u nu a}}
\pfra{je veux partir}
\end{exemple}
\newline
\begin{exemple}
\région{PA}
\textbf{\pnua{ai-m da?}}
\pfra{que veux-tu ?}
\end{exemple}
\newline
\begin{sous-entrée}{kixa ai}{ⓔai-ⓝkixa ai}
\begin{glose}
\pfra{pas dressé (animal) (lit. qui n'a pas de coeur)}
\end{glose}
\end{sous-entrée}
\newline
\begin{sous-entrée}{kixa ai}{ⓔai-ⓝkixa ai}
\begin{glose}
\pfra{qui n'a pas l'âge de raison (enfant) (lit. qui n'a pas de coeur)}
\end{glose}
\end{sous-entrée}
\end{entrée}

\begin{entrée}{ai-xa}{}{ⓔai-xa}
\région{GOs}
\variante{%
ai xa
\région{PA}}
(\domainesémantique{Sentiments})
\classe{v}
\begin{glose}
\pfra{vouloir ; envie de (avoir)}
\end{glose}
\newline
\begin{exemple}
\région{GO}
\textbf{\pnua{axa wôjô-nu}}
\pfra{j'ai envie d'un bateau}
\end{exemple}
\newline
\begin{exemple}
\région{GO}
\textbf{\pnua{axa mõõ-nu ma avwo-nu marie}}
\pfra{j'ai envie d'une épouse et je veux me marier}
\end{exemple}
\newline
\begin{exemple}
\région{PA}
\textbf{\pnua{axa i yo}}
\pfra{ta volonté, ton désir}
\end{exemple}
\newline
\note{ai-xa est raccourci sous la forme: axa et aa [PA]}{grammaire}{}
\end{entrée}

\begin{entrée}{a-kããle}{}{ⓔa-kããle}
\région{GOs PA}
(\domainesémantique{Remèdes, médecine})
\classe{nom}
\begin{glose}
\pfra{médecin (celui qui soigne)}
\end{glose}
\newline
\begin{exemple}
\textbf{\pnua{a-kããle-hã}}
\pfra{celui qui nous soigne}
\end{exemple}
\end{entrée}

\begin{entrée}{a-kai}{}{ⓔa-kai}
\région{GOs BO}
(\domainesémantique{Verbes de déplacement et moyens de déplacement})
\classe{v}
\begin{glose}
\pfra{suivre}
\end{glose}
\begin{glose}
\pfra{accompagner (lit. aller suivre)}
\end{glose}
\newline
\begin{exemple}
\région{GO}
\textbf{\pnua{a-wö-nu a-kai-we}}
\pfra{je voudrais/ j'ai envie de vous suivre/accompagner}
\end{exemple}
\newline
\begin{exemple}
\région{GO}
\textbf{\pnua{nu a-kai-jö}}
\pfra{je suis venu avec toi en te suivant}
\end{exemple}
\newline
\begin{exemple}
\région{BO}
\textbf{\pnua{nu ru a-kai-m}}
\pfra{j'irai avec toi, je t'accompagnerai}
\end{exemple}
\newline
\begin{sous-entrée}{a-kai-ne}{ⓔa-kaiⓝa-kai-ne}
\begin{glose}
\pfra{ensemble}
\end{glose}
\newline
\relationsémantique{Cf.}{\lien{}{höze dè}}
\glosecourte{suivre un chemin}
\end{sous-entrée}
\end{entrée}

\begin{entrée}{a-kalu}{}{ⓔa-kalu}
\région{GOs}
(\domainesémantique{Parenté})
\classe{nom}
\begin{glose}
\pfra{maternels (ceux qui reçoivent, qui s'inclinent pour recevoir?)}
\end{glose}
\end{entrée}

\begin{entrée}{a-kaze}{}{ⓔa-kaze}
\région{GOs}
\variante{%
a-kale
\région{PA}}
(\domainesémantique{Verbes de déplacement et moyens de déplacement
, Pêche})
\classe{v}
\begin{glose}
\pfra{aller à la pêche (à la mer) ;}
\end{glose}
\begin{glose}
\pfra{pêcher à la mer}
\end{glose}
\end{entrée}

\begin{entrée}{a-kênô}{}{ⓔa-kênô}
\région{GOs}
(\domainesémantique{Verbes de déplacement et moyens de déplacement})
\classe{v}
\begin{glose}
\pfra{contourner ; faire le tour}
\end{glose}
\newline
\begin{exemple}
\textbf{\pnua{e ka-kênô}}
\pfra{il se retourne}
\end{exemple}
\end{entrée}

\begin{entrée}{a-kênõge}{}{ⓔa-kênõge}
\région{GOs}
(\domainesémantique{Verbes de déplacement et moyens de déplacement})
\classe{v.t.}
\begin{glose}
\pfra{encercler}
\end{glose}
\begin{glose}
\pfra{entourer}
\end{glose}
\newline
\begin{exemple}
\textbf{\pnua{la thrê a-kênõge nò}}
\pfra{ils courent en encerclant les poissons}
\end{exemple}
\end{entrée}

\begin{entrée}{a-kò}{}{ⓔa-kò}
\région{GOs}
\variante{%
a-kòn
\région{BO PA}}
(\domainesémantique{Numéraux cardinaux})
\classe{NUM}
\begin{glose}
\pfra{trois (animés)}
\end{glose}
\end{entrée}

\begin{entrée}{a-kò êgu}{}{ⓔa-kò êgu}
\région{GOs}
(\domainesémantique{Numéraux cardinaux})
\classe{NUM}
\begin{glose}
\pfra{soixante (lit. trois hommes)}
\end{glose}
\end{entrée}

\begin{entrée}{a-kò êgu bwa truuçi}{}{ⓔa-kò êgu bwa truuçi}
\région{GOs}
(\domainesémantique{Numéraux cardinaux})
\classe{NUM}
\begin{glose}
\pfra{soixante-dix (lit. trois hommes et 10)}
\end{glose}
\end{entrée}

\begin{entrée}{akònòbòn}{}{ⓔakònòbòn}
\région{BO [BM]}
(\domainesémantique{Temps})
\classe{ADV}
\begin{glose}
\pfra{passé ; dernier}
\end{glose}
\newline
\begin{exemple}
\région{BO}
\textbf{\pnua{je ka akònòbòn}}
\pfra{l'année dernière [BM]}
\end{exemple}
\end{entrée}

\begin{entrée}{a-kha-kujaxo}{}{ⓔa-kha-kujaxo}
\région{GOs}
(\domainesémantique{Verbes de déplacement et moyens de déplacement})
\classe{v}
\begin{glose}
\pfra{rendre visite à un malade}
\end{glose}
\end{entrée}

\begin{entrée}{a-kha-pwiò}{}{ⓔa-kha-pwiò}
\région{GOs}
(\domainesémantique{Pêche
, Verbes de déplacement et moyens de déplacement})
\classe{v}
\begin{glose}
\pfra{aller pêcher à la senne (lit. aller lancer le filet)}
\end{glose}
\end{entrée}

\begin{entrée}{a-khazia}{}{ⓔa-khazia}
\région{GOs}
(\domainesémantique{Verbes de déplacement et moyens de déplacement})
\classe{v}
\begin{glose}
\pfra{passer près de}
\end{glose}
\end{entrée}

\begin{entrée}{ala}{}{ⓔala}
\région{GOs}
(\domainesémantique{Caractéristiques et propriétés des personnes})
\classe{v}
\begin{glose}
\pfra{maladroit ; gauche}
\end{glose}
\newline
\begin{exemple}
\région{GO}
\textbf{\pnua{ala la mwêje-je}}
\pfra{il a des manières gauches, il est maladroit}
\end{exemple}
\newline
\begin{exemple}
\région{GO}
\textbf{\pnua{e ala òri ègu ba !}}
\pfra{ce qu'il est maladroit cet homme !}
\end{exemple}
\end{entrée}

\begin{entrée}{ala-}{}{ⓔala-}
\région{GOs PA BO}
\classe{n ; PREF. sémantique (référant à une surface extérieure)}
\newline
\sens{1}
(\domainesémantique{Corps humain})
\begin{glose}
\pfra{face ; visage ; avant}
\end{glose}
\newline
\begin{exemple}
\région{PA BO}
\textbf{\pnua{ala-mè-n}}
\pfra{son visage}
\end{exemple}
\newline
\begin{sous-entrée}{bwa ala-mè-ny}{ⓔala-ⓢ1ⓝbwa ala-mè-ny}
\région{BO}
\begin{glose}
\pfra{devant moi}
\end{glose}
\end{sous-entrée}
\newline
\begin{sous-entrée}{bwa ala-mwa ; bwa ala-mè mwa}{ⓔala-ⓢ1ⓝbwa ala-mwa ; bwa ala-mè mwa}
\begin{glose}
\pfra{devant la maison}
\end{glose}
\end{sous-entrée}
\newline
\sens{2}
(\domainesémantique{Configuration des objets})
\begin{glose}
\pfra{façade ; surface}
\end{glose}
\newline
\begin{sous-entrée}{ala-hi-je}{ⓔala-ⓢ2ⓝala-hi-je}
\région{GO}
\begin{glose}
\pfra{paume de la main}
\end{glose}
\end{sous-entrée}
\newline
\begin{sous-entrée}{ala-hi-n}{ⓔala-ⓢ2ⓝala-hi-n}
\région{BO}
\begin{glose}
\pfra{paume de la main}
\end{glose}
\end{sous-entrée}
\newline
\begin{sous-entrée}{ala-kòò-n}{ⓔala-ⓢ2ⓝala-kòò-n}
\région{BO}
\begin{glose}
\pfra{plante du pied ; chaussure}
\end{glose}
\end{sous-entrée}
\newline
\sens{3}
(\domainesémantique{Préfixes sémantiques divers})
\begin{glose}
\pfra{contenant vide}
\end{glose}
\newline
\begin{sous-entrée}{ala-gu}{ⓔala-ⓢ3ⓝala-gu}
\région{BO}
\begin{glose}
\pfra{valve de coquillage}
\end{glose}
\end{sous-entrée}
\newline
\begin{sous-entrée}{ala-nu}{ⓔala-ⓢ3ⓝala-nu}
\région{BO}
\begin{glose}
\pfra{coquille de coco vide}
\end{glose}
\end{sous-entrée}
\newline
\begin{sous-entrée}{ala-we}{ⓔala-ⓢ3ⓝala-we}
\région{BO}
\begin{glose}
\pfra{calebasse}
\end{glose}
\end{sous-entrée}
\newline
\begin{sous-entrée}{ala-dau}{ⓔala-ⓢ3ⓝala-dau}
\région{BO}
\begin{glose}
\pfra{îlot inhabité}
\end{glose}
\end{sous-entrée}
\newline
\sens{4}
(\domainesémantique{Caractéristiques et propriétés des personnes})
\begin{glose}
\pfra{apparence ; aspect}
\end{glose}
\newline
\étymologie{
\langue{POc}
\étymon{*qadop}
\glosecourte{face, devant}}
\end{entrée}

\begin{entrée}{alaaba}{}{ⓔalaaba}
\région{GOs BO PA}
(\domainesémantique{Feu : objets et actions liés au feu})
\classe{nom}
\begin{glose}
\pfra{braise}
\end{glose}
\end{entrée}

\begin{entrée}{alaal}{}{ⓔalaal}
\région{BO}
(\domainesémantique{Poissons})
\classe{nom}
\begin{glose}
\pfra{picot rayé [Corne]}
\end{glose}
\nomscientifique{Siganus sp. (Siganidae)}
\newline
\note{non vérifié}{général}{}
\end{entrée}

\begin{entrée}{alabo}{}{ⓔalabo}
\région{GOs WEM}
(\domainesémantique{Corps humain})
\classe{nom}
\begin{glose}
\pfra{côté (du corps)}
\end{glose}
\newline
\begin{sous-entrée}{alabo bwa mhõ}{ⓔalaboⓝalabo bwa mhõ}
\begin{glose}
\pfra{côté gauche}
\end{glose}
\end{sous-entrée}
\end{entrée}

\begin{entrée}{alabo-}{}{ⓔalabo-}
\région{PA}
(\domainesémantique{Préfixes classificateurs numériques})
\classe{CLF.NUM}
\begin{glose}
\pfra{quart ou moitié de tortue ou de bœœuf}
\end{glose}
\newline
\relationsémantique{Cf.}{\lien{}{mai-xe [PA WEM]}}
\glosecourte{morceau de viande,de tortue, etc.}
\end{entrée}

\begin{entrée}{ala-hi}{}{ⓔala-hi}
\région{GOs BO PA}
\newline
\groupe{A}
(\domainesémantique{Corps humain})
\classe{nom}
\begin{glose}
\pfra{paume}
\end{glose}
\newline
\begin{exemple}
\région{PA}
\textbf{\pnua{ala-hi-n}}
\pfra{sa paume}
\end{exemple}
\newline
\groupe{B}
(\domainesémantique{Caractéristiques et propriétés des personnes})
\classe{v}
\begin{glose}
\pfra{habile}
\end{glose}
\newline
\begin{exemple}
\région{GO}
\textbf{\pnua{ala-hii-je}}
\pfra{il est adroit, habile}
\end{exemple}
\newline
\begin{exemple}
\région{GO}
\textbf{\pnua{za kavwö jö ala-hi}}
\pfra{tu ne sais pas t'y prendre, tu es maladroit}
\end{exemple}
\newline
\begin{exemple}
\région{GO}
\textbf{\pnua{jö za ala-hi}}
\pfra{tu es habile, tu es adroit}
\end{exemple}
\end{entrée}

\begin{entrée}{ala-kò}{}{ⓔala-kò}
\région{GOs BO}
\variante{%
ala-xò
\région{GO(s)}, 
ala-kò
\région{PA}}
\classe{nom}
\newline
\sens{1}
(\domainesémantique{Corps humain})
\begin{glose}
\pfra{plante de pied}
\end{glose}
\newline
\begin{exemple}
\région{GO}
\textbf{\pnua{ala-xòò-je}}
\pfra{sa plante de pied, ses chaussures}
\end{exemple}
\newline
\sens{2}
(\domainesémantique{Vêtements, parure})
\begin{glose}
\pfra{chaussure}
\end{glose}
\newline
\begin{exemple}
\région{PA}
\textbf{\pnua{ala-kòò-ny}}
\pfra{mes chaussures}
\end{exemple}
\newline
\begin{exemple}
\région{GO}
\textbf{\pnua{ala-xòò-nu}}
\pfra{mes chaussures}
\end{exemple}
\newline
\begin{sous-entrée}{ala-xò thi}{ⓔala-kòⓢ2ⓝala-xò thi}
\begin{glose}
\pfra{claquettes}
\end{glose}
\end{sous-entrée}
\newline
\begin{sous-entrée}{ala-xò-thixò}{ⓔala-kòⓢ2ⓝala-xò-thixò}
\begin{glose}
\pfra{chaussure à talons}
\end{glose}
\end{sous-entrée}
\end{entrée}

\begin{entrée}{ala-me}{}{ⓔala-me}
\région{GOs PA BO}
\classe{nom}
\newline
\sens{1}
(\domainesémantique{Corps humain})
\begin{glose}
\pfra{visage ; face}
\end{glose}
\newline
\begin{exemple}
\région{GO}
\textbf{\pnua{tûûni ala-me-ju !}}
\pfra{essuie ton visage !}
\end{exemple}
\newline
\begin{exemple}
\région{PA}
\textbf{\pnua{ala-mee-n}}
\pfra{son visage}
\end{exemple}
\newline
\begin{exemple}
\région{BO}
\textbf{\pnua{bwa ala-mee-ny}}
\pfra{son visage}
\end{exemple}
\newline
\sens{2}
(\domainesémantique{Noms locatifs})
\begin{glose}
\pfra{devant}
\end{glose}
\newline
\begin{sous-entrée}{ala-me-ko}{ⓔala-meⓢ2ⓝala-me-ko}
\begin{glose}
\pfra{le devant de la jambe}
\end{glose}
\end{sous-entrée}
\newline
\begin{sous-entrée}{ala-me mwa}{ⓔala-meⓢ2ⓝala-me mwa}
\begin{glose}
\pfra{le devant de la maison}
\end{glose}
\end{sous-entrée}
\end{entrée}

\begin{entrée}{alamwi}{}{ⓔalamwi}
\région{BO PA}
\classe{nom}
\newline
\sens{1}
(\domainesémantique{Objets coutumiers})
\begin{glose}
\pfra{corde de monnaie de coquillage (Dubois)}
\end{glose}
\newline
\note{fibre lavée et séchée, sert à faire des cordes pour les frondes, les doigtiers de sagaie ou les ceintures de guerre 'wa-bwanu' (Charles)}{glose}{}
\newline
\sens{2}
(\domainesémantique{Cordes, cordages})
\begin{glose}
\pfra{fibre de jeune rejet de 'phuleng'}
\end{glose}
\newline
\note{fibre lavée et séchée, sert à faire des cordes pour les frondes, les doigtiers de sagaie ou les ceintures de guerre 'wa-bwanu' (Charles)}{glose}{}
\end{entrée}

\begin{entrée}{alavwu}{}{ⓔalavwu}
\région{GOs WEM PA BO}
\variante{%
alapu
\région{vx}}
(\domainesémantique{Fonctions naturelles humaines})
\classe{v.stat.}
\begin{glose}
\pfra{faim (avoir)}
\end{glose}
\end{entrée}

\begin{entrée}{alawe}{}{ⓔalawe}
\région{GOs PA}
\variante{%
olawe, olaè, olaa
\région{BO}}
(\domainesémantique{Interjection})
\classe{INTJ}
\begin{glose}
\pfra{au revoir !}
\end{glose}
\newline
\begin{exemple}
\région{GO}
\textbf{\pnua{alawe i we}}
\pfra{au revoir ! (à plusieurs personnes)}
\end{exemple}
\newline
\begin{exemple}
\région{GO}
\textbf{\pnua{alawe i jò}}
\pfra{au revoir à vous 2 !}
\end{exemple}
\newline
\begin{exemple}
\région{GO}
\textbf{\pnua{ba-alawe}}
\pfra{coutume d'au revoir !}
\end{exemple}
\newline
\begin{exemple}
\région{PA}
\textbf{\pnua{alawe-m}}
\pfra{au revoir à toi !}
\end{exemple}
\newline
\begin{exemple}
\région{BO}
\textbf{\pnua{olaè-m}}
\pfra{au revoir à toi !}
\end{exemple}
\newline
\begin{exemple}
\région{BO}
\textbf{\pnua{olaa-m}}
\pfra{au revoir à toi}
\end{exemple}
\end{entrée}

\begin{entrée}{alaxe}{}{ⓔalaxe}
\région{GOs PA BO}
\classe{v}
\newline
\sens{1}
(\domainesémantique{Noms locatifs})
\begin{glose}
\pfra{côté (sur le)}
\end{glose}
\begin{glose}
\pfra{travers (de) ; pas droit}
\end{glose}
\newline
\begin{exemple}
\région{GO}
\textbf{\pnua{tre-alaxe dröö-a jö}}
\pfra{ta marmite est posée de travers (i.e. plus sur un côté, pas au milieu)}
\end{exemple}
\newline
\begin{exemple}
\région{GO PA}
\textbf{\pnua{kô-alaxe}}
\pfra{couché de travers}
\end{exemple}
\newline
\sens{2}
(\domainesémantique{Modalité, verbes modaux})
\begin{glose}
\pfra{travers (de) ; mal fait}
\end{glose}
\newline
\begin{exemple}
\région{GO}
\textbf{\pnua{la nee-alaxee-ni la môgu i ã}}
\pfra{ils ont mal fait notre travail}
\end{exemple}
\newline
\begin{exemple}
\région{PA}
\textbf{\pnua{la ne-alaxee-ni la nyama i ã}}
\pfra{ils ont mal fait notre travail}
\end{exemple}
\newline
\relationsémantique{Ant.}{\lien{}{baxòòl}}
\glosecourte{droit}
\end{entrée}

\begin{entrée}{aleleang}{}{ⓔaleleang}
\région{PA}
(\domainesémantique{Insectes})
\classe{nom}
\begin{glose}
\pfra{cigale (petite et rouge)}
\end{glose}
\end{entrée}

\begin{entrée}{a-lixee}{}{ⓔa-lixee}
\région{GOs}
(\domainesémantique{Verbes de déplacement et moyens de déplacement})
\classe{v}
\begin{glose}
\pfra{aller à flanc de montagne}
\end{glose}
\end{entrée}

\begin{entrée}{alö}{}{ⓔalö}
\région{GOs BO}
\variante{%
alu
\région{PA}}
(\domainesémantique{Fonctions naturelles humaines})
\classe{v.t.}
\begin{glose}
\pfra{regarder ; observer ; guetter}
\end{glose}
\newline
\begin{exemple}
\textbf{\pnua{e alö-le loto}}
\pfra{il regarde la voiture}
\end{exemple}
\newline
\begin{exemple}
\textbf{\pnua{e alö-le thoomwa ã}}
\pfra{il regarde cette femme}
\end{exemple}
\newline
\begin{exemple}
\textbf{\pnua{e alö-le hênua èmwê ã}}
\pfra{il regarde la photo de cet homme}
\end{exemple}
\newline
\begin{exemple}
\textbf{\pnua{e alöe-nu}}
\pfra{il me regarde}
\end{exemple}
\newline
\begin{exemple}
\textbf{\pnua{e alöe-je}}
\pfra{il le/la regarde}
\end{exemple}
\newline
\begin{exemple}
\textbf{\pnua{ciia ! alöle zine !}}
\pfra{poulpe ! regarde le rat !}
\end{exemple}
\newline
\begin{exemple}
\textbf{\pnua{e alö-le ciia xo zine}}
\pfra{le rat regarde le poulpe}
\end{exemple}
\newline
\begin{exemple}
\textbf{\pnua{novwo zine ca, e alö-le ciia}}
\pfra{le rat, il regarde le poulpe}
\end{exemple}
\newline
\begin{exemple}
\textbf{\pnua{novwo ciia ca, e alöle-je xo zine}}
\pfra{quant au poulpe, le rat le regarde}
\end{exemple}
\newline
\note{alö-le (v.t.)}{grammaire}{regarder qqch.}
\end{entrée}

\begin{entrée}{alobo}{}{ⓔalobo}
\région{GOs PA}
(\domainesémantique{Fonctions naturelles humaines})
\classe{v.i.}
\begin{glose}
\pfra{fixer du regard ; dévisager}
\end{glose}
\newline
\begin{exemple}
\textbf{\pnua{kebwa alobo !}}
\pfra{ne fixe pas du regard !}
\end{exemple}
\end{entrée}

\begin{entrée}{ãmã}{}{ⓔãmã}
\formephonétique{ɛ̃mɛ̃}
\région{GOs}
\classe{nom}
\newline
\sens{1}
(\domainesémantique{Corps humain})
\begin{glose}
\pfra{palais (bouche)}
\end{glose}
\newline
\sens{2}
(\domainesémantique{Sons, bruits})
\begin{glose}
\pfra{brouhaha des voix}
\end{glose}
\newline
\begin{exemple}
\textbf{\pnua{ãmã-la}}
\pfra{leur brouhaha}
\end{exemple}
\end{entrée}

\begin{entrée}{ã-mã-}{}{ⓔã-mã-}
\région{GOs PA BO}
(\domainesémantique{Démonstratifs})
\classe{DEM duel ou PL}
\begin{glose}
\pfra{les ; ces}
\end{glose}
\newline
\begin{exemple}
\textbf{\pnua{ã-mã-lã-ôli}}
\pfra{ceux-là}
\end{exemple}
\newline
\begin{exemple}
\textbf{\pnua{ã-mã-li-nã}}
\pfra{les deux hommes}
\end{exemple}
\newline
\begin{exemple}
\textbf{\pnua{ã-mã-la-nã}}
\pfra{les hommes}
\end{exemple}
\end{entrée}

\begin{entrée}{amaèk}{}{ⓔamaèk}
\région{BO}
(\domainesémantique{Noms des plantes})
\classe{nom}
\begin{glose}
\pfra{mimosa (faux) [Corne]}
\end{glose}
\nomscientifique{Leucaena glauca}
\newline
\note{non vérifié}{général}{}
\end{entrée}

\begin{entrée}{ã-mala-e}{}{ⓔã-mala-e}
\région{GOs}
(\domainesémantique{Démonstratifs})
\classe{PRO.DEIC.2 (3° pers. masc. PL)}
\begin{glose}
\pfra{eux là}
\end{glose}
\end{entrée}

\begin{entrée}{ã-mala-na}{}{ⓔã-mala-na}
\région{GOs PA BO}
(\domainesémantique{Démonstratifs})
\classe{PRO 3° pers. masc. PL}
\begin{glose}
\pfra{eux là}
\end{glose}
\end{entrée}

\begin{entrée}{ã-mala-ò}{}{ⓔã-mala-ò}
\région{GOs PA BO}
(\domainesémantique{Démonstratifs})
\classe{PRO.DEIC.3 (3° pers. masc. PL)}
\begin{glose}
\pfra{eux là-bas}
\end{glose}
\end{entrée}

\begin{entrée}{ã-mala-òòli}{}{ⓔã-mala-òòli}
\région{GOs}
(\domainesémantique{Démonstratifs})
\classe{PRO.DEIC.3 (3° pers. masc. PL)}
\begin{glose}
\pfra{eux là-bas}
\end{glose}
\end{entrée}

\begin{entrée}{ã-mali-na}{}{ⓔã-mali-na}
\région{GOs BO}
(\domainesémantique{Démonstratifs})
\classe{PRO 3° pers. masc. duel (DX ou ANAPH)}
\begin{glose}
\pfra{eux-deux là; hé ! les 2 hommes !}
\end{glose}
\end{entrée}

\begin{entrée}{ã-mani}{}{ⓔã-mani}
\région{BO}
(\domainesémantique{Organisation sociale})
\classe{nom}
\begin{glose}
\pfra{porte-parole [BM]}
\end{glose}
\end{entrée}

\begin{entrée}{ãmatri}{}{ⓔãmatri}
\formephonétique{̃̃̃ɛ̃maɽi}
\région{GOs}
\variante{%
ãmari
\région{GO(s)}}
(\domainesémantique{Bananiers et bananes})
\classe{nom}
\begin{glose}
\pfra{banane "amérique"}
\end{glose}
\end{entrée}

\begin{entrée}{a-maxo}{}{ⓔa-maxo}
\région{GOs}
(\domainesémantique{Organisation sociale})
\classe{nom}
\begin{glose}
\pfra{porte-parole (du chef) ; émissaire}
\end{glose}
\end{entrée}

\begin{entrée}{amee}{}{ⓔamee}
\région{GOs}
(\domainesémantique{Société})
\classe{nom}
\begin{glose}
\pfra{compagnon}
\end{glose}
\newline
\begin{exemple}
\région{GO}
\textbf{\pnua{amee-nu}}
\pfra{(c'est) mon compagnon, ma compagne}
\end{exemple}
\newline
\begin{exemple}
\région{GO}
\textbf{\pnua{amee-nu Brigit}}
\pfra{Brigit est ma compagne}
\end{exemple}
\newline
\begin{exemple}
\région{GO}
\textbf{\pnua{ge ea amee-jö ?}}
\pfra{où est ton compagnon, ta compagne ?}
\end{exemple}
\newline
\relationsémantique{Syn.}{\lien{ⓔbalaⓗ2}{bala}}
\glosecourte{partenaire}
\newline
\relationsémantique{Syn.}{\lien{ⓔthilò}{thilò}}
\glosecourte{paire, l'autre d'une paire}
\end{entrée}

\begin{entrée}{ã-mi}{}{ⓔã-mi}
\région{GOs}
\variante{%
ô-mi
\région{BO}}
(\domainesémantique{Verbes de déplacement et moyens de déplacement})
\classe{v.DIR}
\begin{glose}
\pfra{venir vers ego}
\end{glose}
\newline
\begin{exemple}
\textbf{\pnua{nu ã-mi na bwa wamwa}}
\pfra{je reviens du champ}
\end{exemple}
\newline
\begin{exemple}
\textbf{\pnua{e ã-mi ena}}
\pfra{il est venu ici}
\end{exemple}
\newline
\begin{exemple}
\textbf{\pnua{li pe-hivwine-li ã-mi na (ni) nye ba-egu}}
\pfra{ils s'ignorent à cause de cette femme (venir de cette femme)}
\end{exemple}
\newline
\begin{exemple}
\textbf{\pnua{e gi ã-mi na nu}}
\pfra{il pleure à cause de moi}
\end{exemple}
\newline
\begin{sous-entrée}{ã-mi na (ni)}{ⓔã-miⓝã-mi na (ni)}
\begin{glose}
\pfra{à cause de (provenir de)}
\end{glose}
\end{sous-entrée}
\end{entrée}

\begin{entrée}{a-mõnu}{}{ⓔa-mõnu}
\région{GOs BO}
(\domainesémantique{Verbes de déplacement et moyens de déplacement})
\classe{v.DIR}
\begin{glose}
\pfra{aller près de}
\end{glose}
\begin{glose}
\pfra{approcher}
\end{glose}
\newline
\begin{exemple}
\textbf{\pnua{ao-mi mõnu}}
\pfra{approche-toi tout près}
\end{exemple}
\end{entrée}

\begin{entrée}{amwidra}{}{ⓔamwidra}
\région{GOs}
\variante{%
aamwida
\région{BO}}
(\domainesémantique{Mollusques})
\classe{nom}
\begin{glose}
\pfra{bernacle ; anatife (chapeau chinois, clovisse)}
\end{glose}
\end{entrée}

\begin{entrée}{ã-na !}{}{ⓔã-na !}
\formephonétique{ɛ̃ɳa}
\région{GOs}
(\domainesémantique{Interpellation
, Démonstratifs})
\classe{INTJ}
\begin{glose}
\pfra{eh ! l'homme là !}
\end{glose}
\newline
\relationsémantique{Cf.}{\lien{ⓔijèè !}{ijèè !}}
\glosecourte{eh ! la femme !}
\end{entrée}

\begin{entrée}{ãnã}{}{ⓔãnã}
\formephonétique{ɛ̃ɳɛ̃}
\région{GOs BO}
\newline
\sens{1}
(\domainesémantique{Démonstratifs})
\classe{DEM.DEIC}
\begin{glose}
\pfra{celui-là}
\end{glose}
\newline
\begin{exemple}
\région{BO}
\textbf{\pnua{mwa ãnã}}
\pfra{la maison de cet homme}
\end{exemple}
\newline
\sens{2}
(\domainesémantique{Interpellation})
\begin{glose}
\pfra{eh l'homme !}
\end{glose}
\end{entrée}

\begin{entrée}{ani êgu}{}{ⓔani êgu}
\région{GOs}
\variante{%
anim êgu
\région{PA}}
(\domainesémantique{Numéraux cardinaux})
\classe{NUM}
\begin{glose}
\pfra{cent (lit. cinq hommes)}
\end{glose}
\end{entrée}

\begin{entrée}{a-ni-ma-xe}{}{ⓔa-ni-ma-xe}
\région{GOs}
\variante{%
a-nim-a-xe
\région{BO PA}}
(\domainesémantique{Numéraux cardinaux})
\classe{NUM (animés)}
\begin{glose}
\pfra{six}
\end{glose}
\end{entrée}

\begin{entrée}{a-niza ?}{}{ⓔa-niza ?}
\région{GOs}
\région{PA BO}
\variante{%
a-nira ?
}
(\domainesémantique{Interrogatifs})
\classe{INT}
\begin{glose}
\pfra{combien? (animés)}
\end{glose}
\end{entrée}

\begin{entrée}{a-nûû}{}{ⓔa-nûû}
\région{GOs}
(\domainesémantique{Verbes de déplacement et moyens de déplacement})
\classe{v}
\begin{glose}
\pfra{aller à la pêche ou la chasse à la torche}
\end{glose}
\newline
\begin{exemple}
\textbf{\pnua{e a-nûû kula}}
\pfra{aller à la pêche aux crevettes à la torche}
\end{exemple}
\newline
\begin{exemple}
\textbf{\pnua{e a-nûû peenã}}
\pfra{aller à la pêche aux anguilles à la torche}
\end{exemple}
\newline
\begin{exemple}
\textbf{\pnua{e a-nûû dube}}
\pfra{aller à la chasse au cerf à la torche}
\end{exemple}
\end{entrée}

\begin{entrée}{aò}{}{ⓔaò}
\région{BO PA}
(\domainesémantique{Organisation sociale})
\classe{v}
\begin{glose}
\pfra{sorcier ; emboucaneur}
\end{glose}
\end{entrée}

\begin{entrée}{a-ò !}{}{ⓔa-ò !}
\région{GOs PA}
(\domainesémantique{Injonction})
\classe{INTJ}
\begin{glose}
\pfra{va-t-en ! ; pars ! ; vas-y !}
\end{glose}
\begin{glose}
\pfra{vas-y ! (utilisé lors de la couverture du toit et la ligature de la paille, lorsque la personne sur le toit pique l'alène vers le bas)}
\end{glose}
\newline
\note{appels prononcés par la personne qui est à l'intérieur de la maison à l'adresse de celui qui est sur le toit pour que ce dernier pique l'alène vers lui, vers le bas)}{glose}{}
\newline
\relationsémantique{Cf.}{\lien{ⓔcabòl !}{cabòl !}}
\glosecourte{sors ! (appel de celui qui est sur le toit à l'adresse de celui qui est à l'intérieur de la maison indiquant à ce dernier qu'il doit piquer l'alène à travers la paille vers le haut )}
\end{entrée}

\begin{entrée}{ã-ò}{}{ⓔã-ò}
\région{GOs BO}
(\domainesémantique{Démonstratifs})
\classe{DEM.ANAPH}
\begin{glose}
\pfra{celui-là (en question)}
\end{glose}
\newline
\begin{exemple}
\textbf{\pnua{i nee xo ã-ò}}
\pfra{celui-là (en question) l'a fait}
\end{exemple}
\end{entrée}

\begin{entrée}{ã-òòli}{}{ⓔã-òòli}
\région{GOs}
(\domainesémantique{Démonstratifs})
\classe{PRO.DEIC.3 (3° pers.)}
\begin{glose}
\pfra{lui là-bas}
\end{glose}
\end{entrée}

\begin{entrée}{apa êgu}{}{ⓔapa êgu}
\région{GOs}
\variante{%
aapa
\région{BO}}
(\domainesémantique{Numéraux cardinaux})
\classe{NUM}
\begin{glose}
\pfra{quatre-vingt (lit. quatre hommes)}
\end{glose}
\end{entrée}

\begin{entrée}{apa êgu bwa truuçi}{}{ⓔapa êgu bwa truuçi}
\région{GOs}
(\domainesémantique{Numéraux cardinaux})
\classe{NUM}
\begin{glose}
\pfra{quatre-vingt dix (lit. quatre hommes et 10)}
\end{glose}
\end{entrée}

\begin{entrée}{a-pe-haze}{}{ⓔa-pe-haze}
\région{GOs}
\région{PA BO}
\variante{%
a-ve-hale
}
(\domainesémantique{Verbes de déplacement et moyens de déplacement})
\classe{v}
\begin{glose}
\pfra{aller chacun de son côté ; divorcer}
\end{glose}
\newline
\begin{exemple}
\région{PA}
\textbf{\pnua{li u a-da mwa a-ve-hale na lina pwamwa}}
\pfra{ils s'en retournent dans leur terroir respectif}
\end{exemple}
\newline
\begin{exemple}
\région{GO}
\textbf{\pnua{la a-pe-haze, la a-ve-haze}}
\pfra{ils se dispersent}
\end{exemple}
\newline
\begin{exemple}
\région{BO}
\textbf{\pnua{la pe-a pe-ale}}
\pfra{ils se dispersent}
\end{exemple}
\end{entrée}

\begin{entrée}{a-pe-hê-kòlò}{}{ⓔa-pe-hê-kòlò}
\région{BO}
(\domainesémantique{Couples de parenté
, Parenté})
\classe{couple PAR}
\begin{glose}
\pfra{enfant de frère et de cousins mutuels (femme parlant)}
\end{glose}
\begin{glose}
\pfra{famille; allié}
\end{glose}
\newline
\begin{exemple}
\région{BO}
\textbf{\pnua{li a-pe-hê-kòlò-li}}
\pfra{ils sont tante et neveux/nièces (enfants de frère)}
\end{exemple}
\end{entrée}

\begin{entrée}{a-pe-paçaxai}{}{ⓔa-pe-paçaxai}
\région{GOs}
\variante{%
a-pe-pha-caaxai
\région{PA}}
(\domainesémantique{Caractéristiques et propriétés des personnes})
\classe{nom}
\begin{glose}
\pfra{farceur ; qui joue des tours ; turbulent}
\end{glose}
\end{entrée}

\begin{entrée}{a-pii}{}{ⓔa-pii}
\région{GOs}
(\domainesémantique{Crustacés, crabes})
\classe{nom}
\begin{glose}
\pfra{crabe en train de muer (et de jeter sa carapace)}
\end{glose}
\end{entrée}

\begin{entrée}{apoxapenu}{}{ⓔapoxapenu}
\région{GOs}
\variante{%
avhoxavhenu
\région{PA}}
(\domainesémantique{Organisation sociale})
\classe{nom}
\begin{glose}
\pfra{clan allié du clan maternel}
\end{glose}
\begin{glose}
\pfra{personnes connues [PA]}
\end{glose}
\newline
\note{(qui vient soutenir les maternels pour offrir les dons coutumiers à ceux qui viennent d'ailleurs ('aavhe'), lors des dons dans les cérémonies (de deuil par ex.)}{glose}{}
\newline
\relationsémantique{Cf.}{\lien{ⓔaavhe}{aavhe}}
\glosecourte{clan extérieur (venus pour des cérémonies)}
\newline
\relationsémantique{Cf.}{\lien{ⓔa-xalu}{a-xalu}}
\glosecourte{clan maternel (dans les cérémonies de deuil)}
\newline
\relationsémantique{Cf.}{\lien{}{a-kalu whaniri}}
\glosecourte{clan maternel (dans les cérémonies de deuil)}
\end{entrée}

\begin{entrée}{a-poxe}{}{ⓔa-poxe}
\région{GOs PA}
(\domainesémantique{Verbes de déplacement et moyens de déplacement})
\classe{v}
\begin{glose}
\pfra{aller ensemble (lit. aller un)}
\end{glose}
\newline
\begin{exemple}
\région{GO}
\textbf{\pnua{la a-poxe}}
\pfra{aller ensemble (lit. aller un)}
\end{exemple}
\end{entrée}

\begin{entrée}{a-puunõ}{}{ⓔa-puunõ}
\région{GOs}
\variante{%
a-puunol
\région{PA}}
(\domainesémantique{Discours, échanges verbaux})
\classe{nom}
\begin{glose}
\pfra{orateur}
\end{glose}
\end{entrée}

\begin{entrée}{a-phènô}{}{ⓔa-phènô}
\région{GOs}
(\domainesémantique{Dons, échanges, achat et vente, vol})
\classe{nom}
\begin{glose}
\pfra{voleur}
\end{glose}
\end{entrée}

\begin{entrée}{a-phe-vhaa}{}{ⓔa-phe-vhaa}
\région{GOs PA}
(\domainesémantique{Organisation sociale})
\classe{nom}
\begin{glose}
\pfra{messager (lit. celui qui apporte le message)}
\end{glose}
\newline
\begin{exemple}
\textbf{\pnua{a-phe-vha i aazo}}
\pfra{le messager du chef}
\end{exemple}
\end{entrée}

\begin{entrée}{a-phònò}{}{ⓔa-phònò}
\région{GOs}
\variante{%
a-phònòng
\région{PA}}
(\domainesémantique{Organisation sociale})
\classe{nom}
\begin{glose}
\pfra{sorcier ('emboucaneur')}
\end{glose}
\newline
\relationsémantique{Cf.}{\lien{}{phònò}}
\glosecourte{sorcellerie ('boucan')}
\end{entrée}

\begin{entrée}{ara-hogo}{}{ⓔara-hogo}
\région{PA}
(\domainesémantique{Topographie})
\classe{nom}
\begin{glose}
\pfra{flanc de la montagne}
\end{glose}
\newline
\begin{exemple}
\région{PA}
\textbf{\pnua{ni ara-hogo}}
\pfra{sur le flanc de la montagne}
\end{exemple}
\end{entrée}

\begin{entrée}{aru}{}{ⓔaru}
\région{PA BO}
(\domainesémantique{Cultures, techniques, boutures})
\classe{nom}
\begin{glose}
\pfra{butte de terre de la tarodière}
\end{glose}
\newline
\note{orientée vers le versant de la montagne, à terre moins remuée que le "penu", donne de moins bons taros (Dubois + Charles)}{glose}{}
\newline
\relationsémantique{Cf.}{\lien{ⓔpeenu}{peenu}}
\glosecourte{tarodière}
\end{entrée}

\begin{entrée}{atibuda}{}{ⓔatibuda}
\région{GOs}
(\domainesémantique{Oiseaux})
\classe{nom}
\begin{glose}
\pfra{oiseau (petit, noir)}
\end{glose}
\newline
\begin{sous-entrée}{atibunu}{ⓔatibudaⓝatibunu}
\begin{glose}
\pfra{oiseau (à ventre rouge)}
\end{glose}
\end{sous-entrée}
\end{entrée}

\begin{entrée}{a-thu drogò}{}{ⓔa-thu drogò}
\région{GOs}
(\domainesémantique{Société
, Travail bois})
\classe{nom}
\begin{glose}
\pfra{sculpteur}
\end{glose}
\end{entrée}

\begin{entrée}{a-thu kônya}{}{ⓔa-thu kônya}
\région{GOs}
(\domainesémantique{Relations et interaction sociales})
\classe{nom}
\begin{glose}
\pfra{pitre (qqn qui fait le)}
\end{glose}
\end{entrée}

\begin{entrée}{atre thrõbõ}{}{ⓔatre thrõbõ}
\formephonétique{aʈe ʈʰõmbõ}
\région{GOs}
\variante{%
te-thõbwõn, te-a thõbwõn
\région{BO}}
(\domainesémantique{Astres})
\classe{nom}
\begin{glose}
\pfra{étoile du soir ; Vénus}
\end{glose}
\end{entrée}

\begin{entrée}{atrilò}{}{ⓔatrilò}
\formephonétique{aʈilɔ}
\région{GOs BO PA}
(\domainesémantique{Corps humain})
\classe{nom}
\begin{glose}
\pfra{glotte}
\end{glose}
\end{entrée}

\begin{entrée}{a-tru êgu}{}{ⓔa-tru êgu}
\formephonétique{aʈu}
\région{GOs}
\variante{%
aaru êgu
\région{BO}}
(\domainesémantique{Numéraux cardinaux})
\classe{NUM}
\begin{glose}
\pfra{quarante (lit. deux hommes)}
\end{glose}
\end{entrée}

\begin{entrée}{a-tru êgu bwa truuçi}{}{ⓔa-tru êgu bwa truuçi}
\région{GOs}
(\domainesémantique{Numéraux cardinaux})
\classe{NUM}
\begin{glose}
\pfra{cinquante (lit. deux hommes et 10)}
\end{glose}
\end{entrée}

\begin{entrée}{au mwaji-n}{}{ⓔau mwaji-n}
\région{PA}
(\domainesémantique{Temps})
\classe{nom}
\begin{glose}
\pfra{retard (son) ; il est en retard}
\end{glose}
\newline
\begin{exemple}
\textbf{\pnua{au mwaji-n}}
\pfra{il est en retard}
\end{exemple}
\end{entrée}

\begin{entrée}{auva}{}{ⓔauva}
\région{GOs BO}
(\domainesémantique{Mammifères marins})
\classe{nom}
\begin{glose}
\pfra{dugong ; vache marine}
\end{glose}
\end{entrée}

\begin{entrée}{a-uxi-ce}{}{ⓔa-uxi-ce}
\région{GOs}
(\domainesémantique{Insectes})
\classe{nom}
\begin{glose}
\pfra{termite (lit. qui perce le bois)}
\end{glose}
\end{entrée}

\begin{entrée}{a-vwe bulu}{}{ⓔa-vwe bulu}
\formephonétique{aβe}
\région{GOs}
(\domainesémantique{Relations et interaction sociales})
\classe{v.COLL}
\begin{glose}
\pfra{ensemble (être)}
\end{glose}
\begin{glose}
\pfra{relation réciproque}
\end{glose}
\newline
\begin{exemple}
\région{GO}
\textbf{\pnua{li pe-a-vwe bulu}}
\pfra{ils se déplacent ensemble}
\end{exemple}
\newline
\begin{exemple}
\région{GO}
\textbf{\pnua{la a-vwe bulu}}
\pfra{ils restent ensemble}
\end{exemple}
\newline
\begin{exemple}
\région{GO}
\textbf{\pnua{la avwe bulu vwo la a hovwo}}
\pfra{ils sont ensemble pour aller manger}
\end{exemple}
\newline
\relationsémantique{Cf.}{\lien{ⓔa-poxe}{a-poxe}}
\glosecourte{aller ensemble (lit. aller un)}
\newline
\morphologie{a-vwe bulu vient de : a-pe-bulu}
\end{entrée}

\begin{entrée}{a-vwe-da}{}{ⓔa-vwe-da}
\formephonétique{aβe}
\région{GOs}
(\domainesémantique{Verbes de déplacement et moyens de déplacement})
\classe{v.DIR}
\begin{glose}
\pfra{aller en montant (sans destination précise)}
\end{glose}
\newline
\begin{sous-entrée}{a-wâ-vwe-da}{ⓔa-vwe-daⓝa-wâ-vwe-da}
\begin{glose}
\pfra{monter comme ça (sans destination précise)}
\end{glose}
\end{sous-entrée}
\end{entrée}

\begin{entrée}{a-vwe-du}{}{ⓔa-vwe-du}
\formephonétique{aβe}
\région{GOs}
(\domainesémantique{Verbes de déplacement et moyens de déplacement})
\classe{v.DIR}
\begin{glose}
\pfra{aller en descendant (sans destination précise)}
\end{glose}
\newline
\begin{sous-entrée}{a-wâ-vwe-du}{ⓔa-vwe-duⓝa-wâ-vwe-du}
\begin{glose}
\pfra{descendre comme ça (sans destination précise)}
\end{glose}
\end{sous-entrée}
\end{entrée}

\begin{entrée}{a-vwe-e}{}{ⓔa-vwe-e}
\formephonétique{aβe}
\région{GOs}
(\domainesémantique{Verbes de déplacement et moyens de déplacement})
\classe{v.DIR}
\begin{glose}
\pfra{aller sur le côté ; aller dans une direction transverse}
\end{glose}
\newline
\morphologie{a-vwe-e vient de : a-pe-e}
\end{entrée}

\begin{entrée}{avwi-}{}{ⓔavwi-}
\formephonétique{aβi}
\région{GOs}
\variante{%
avhi-
\région{PA}, 
api
\région{GO(s)}}
(\domainesémantique{Parenté})
\classe{nom}
\begin{glose}
\pfra{affins du côté maternel (maternels parlant)}
\end{glose}
\newline
\note{Terme d'évitement pour référer au frère ou à la soeur quand on est de sexe opposé.}{glose}{}
\newline
\begin{exemple}
\région{GO}
\textbf{\pnua{avwi-la i nu}}
\pfra{mes maternels}
\end{exemple}
\newline
\begin{exemple}
\région{GO}
\textbf{\pnua{api-nu}}
\pfra{ma soeur}
\end{exemple}
\newline
\begin{exemple}
\région{PA}
\textbf{\pnua{i avhi-ny}}
\pfra{il est de mon clan maternel}
\end{exemple}
\newline
\relationsémantique{Cf.}{\lien{}{ayabòl [PA]}}
\glosecourte{affins du côté maternel}
\end{entrée}

\begin{entrée}{a-vwö}{}{ⓔa-vwö}
\formephonétique{aβω}
\région{GOs}
\variante{%
apo
\région{GO(s)}, 
a- (forme courte)
\région{GO(s)}, 
a-wu-; avo-
\région{PA}}
(\domainesémantique{Sentiments})
\classe{v}
\begin{glose}
\pfra{envie de (avoir)}
\end{glose}
\begin{glose}
\pfra{vouloir}
\end{glose}
\newline
\begin{exemple}
\textbf{\pnua{ai-nu vwö/po/wu nu ...}}
\pfra{je veux ... que}
\end{exemple}
\newline
\begin{exemple}
\textbf{\pnua{ai-nu vwo nu kido}}
\pfra{j'ai envie de boire (lit. désir-mon boire)}
\end{exemple}
\newline
\begin{exemple}
\textbf{\pnua{apo-nu kudo}}
\pfra{j'ai envie de boire}
\end{exemple}
\newline
\begin{exemple}
\textbf{\pnua{ai-nu vwö-nu imê}}
\pfra{j'ai envie d'uriner}
\end{exemple}
\newline
\begin{exemple}
\région{PA}
\textbf{\pnua{ai-nu phe ai-m meni koi-m}}
\pfra{je veux prendre ton coeur et ton foie}
\end{exemple}
\newline
\note{a-vwö est la forme contractée et incorrecte de : ai-... vwö.}{grammaire}{}
\end{entrée}

\begin{entrée}{a-vwö kudo}{}{ⓔa-vwö kudo}
\formephonétique{aβω}
\région{GOs}
\variante{%
aapo kudo
\région{GO(s) vx}}
(\domainesémantique{Fonctions naturelles humaines})
\classe{v}
\begin{glose}
\pfra{soif (avoir) (lit. vouloir boire)}
\end{glose}
\newline
\begin{exemple}
\région{GO}
\textbf{\pnua{a-vwö-nu kudo}}
\pfra{j'ai envie de boire}
\end{exemple}
\end{entrée}

\begin{entrée}{avwònò}{}{ⓔavwònò}
\formephonétique{aβɔ̃ɳɔ̃}
\région{GOs}
\variante{%
avwònò
\région{PA BO}, 
apono
\région{vx}}
\classe{nom}
\newline
\sens{1}
(\domainesémantique{Types de maison, architecture de la maison})
\begin{glose}
\pfra{maison ; demeure}
\end{glose}
\newline
\sens{2}
(\domainesémantique{Habitat})
\begin{glose}
\pfra{village ; ensemble de maisons}
\end{glose}
\newline
\begin{exemple}
\région{PA}
\textbf{\pnua{avwònòò-n}}
\pfra{sa demeure, son village}
\end{exemple}
\newline
\sens{3}
(\domainesémantique{Noms locatifs})
\classe{n.LOC}
\begin{glose}
\pfra{chez}
\end{glose}
\end{entrée}

\begin{entrée}{a-vwö-nu kûûni}{}{ⓔa-vwö-nu kûûni}
\formephonétique{aβωɳu}
\région{GOs}
(\domainesémantique{Aliments, alimentation})
\classe{v}
\begin{glose}
\pfra{j'ai envie de le manger}
\end{glose}
\begin{glose}
\pfra{appétissant ; beau à voir (légumes)}
\end{glose}
\end{entrée}

\begin{entrée}{a-wãã-da}{}{ⓔa-wãã-da}
\région{GOs}
(\domainesémantique{Verbes de déplacement et moyens de déplacement})
\classe{v}
\begin{glose}
\pfra{monter (sans destination précise)}
\end{glose}
\end{entrée}

\begin{entrée}{a-wãã-du}{}{ⓔa-wãã-du}
\région{GOs}
(\domainesémantique{Verbes de déplacement et moyens de déplacement})
\classe{v}
\begin{glose}
\pfra{descendre (sans destination précise)}
\end{glose}
\newline
\begin{exemple}
\région{GO}
\textbf{\pnua{e a-wã-du Kaavwo kòlò we-za}}
\pfra{Kaavwo part du côté de la mer}
\end{exemple}
\newline
\begin{exemple}
\région{GO}
\textbf{\pnua{e a-wã-du xo Kaavwo kòlò we-za}}
\pfra{Kaavwo part du côté de la mer (contrastif sur les personnes)}
\end{exemple}
\newline
\begin{exemple}
\région{GO}
\textbf{\pnua{e a-wã-du kòlò we-za xo Kaavwo}}
\pfra{Kaavwo part du côté de la mer}
\end{exemple}
\newline
\note{-wã- indique une direction (sans destination précise)}{grammaire}{}
\end{entrée}

\begin{entrée}{a-wãã-e}{}{ⓔa-wãã-e}
\région{GOs}
(\domainesémantique{Verbes de déplacement et moyens de déplacement})
\classe{v.DIR}
\begin{glose}
\pfra{aller sur le côté (sans destination précise)}
\end{glose}
\end{entrée}

\begin{entrée}{a-wãã-ò}{}{ⓔa-wãã-ò}
\région{GOs}
(\domainesémantique{Verbes de déplacement et moyens de déplacement})
\classe{v.DIR}
\begin{glose}
\pfra{aller en s'éloignant (sans destination précise)}
\end{glose}
\end{entrée}

\begin{entrée}{a waya ?}{}{ⓔa waya ?}
\région{GOs}
(\domainesémantique{Verbes de déplacement et moyens de déplacement})
\classe{INT}
\begin{glose}
\pfra{aller vers où ?}
\end{glose}
\end{entrée}

\begin{entrée}{awaze}{}{ⓔawaze}
\région{GOs}
(\domainesémantique{Interjection})
\classe{v}
\begin{glose}
\pfra{dégage ! ; sors de là !}
\end{glose}
\end{entrée}

\begin{entrée}{a-wi?}{}{ⓔa-wi?}
\région{GOs PA}
(\domainesémantique{Verbes de déplacement et moyens de déplacement})
\classe{INT.LOC (dynamique)}
\begin{glose}
\pfra{où (aller) ? ; aller où ?}
\end{glose}
\newline
\begin{exemple}
\textbf{\pnua{jö a-wi ?}}
\pfra{où vas-tu ?}
\end{exemple}
\newline
\begin{exemple}
\textbf{\pnua{la a-wi ?}}
\pfra{où vont-ils ?}
\end{exemple}
\newline
\begin{exemple}
\textbf{\pnua{e a-wi kuau ?}}
\pfra{où va ce chien ?}
\end{exemple}
\newline
\begin{exemple}
\textbf{\pnua{e a-wi lòtò-ã ?}}
\pfra{où va cette voiture ?}
\end{exemple}
\end{entrée}

\begin{entrée}{a-wône-vwo}{}{ⓔa-wône-vwo}
\région{GOs}
(\domainesémantique{Relations et interaction sociales})
\classe{nom}
\begin{glose}
\pfra{remplaçant}
\end{glose}
\end{entrée}

\begin{entrée}{a-whili}{}{ⓔa-whili}
\région{GOs}
\variante{%
a-huli
\région{PA}}
(\domainesémantique{Relations et interaction sociales})
\classe{nom}
\begin{glose}
\pfra{guide}
\end{glose}
\end{entrée}

\begin{entrée}{a whili paa}{}{ⓔa whili paa}
\région{GOs PA}
(\domainesémantique{Organisation sociale
, Guerre})
\classe{nom}
\begin{glose}
\pfra{chef de guerre}
\end{glose}
\end{entrée}

\begin{entrée}{a-whili pò}{}{ⓔa-whili pò}
\région{GOs}
(\domainesémantique{Poissons})
\classe{nom}
\begin{glose}
\pfra{poisson "sabre"}
\end{glose}
\nomscientifique{Chirocentrus dorab (Chirocentridés)}
\end{entrée}

\begin{entrée}{a-xalu}{}{ⓔa-xalu}
\région{GOs}
\variante{%
a-kalu
\région{GO(s)}}
(\domainesémantique{Alliance})
\classe{nom}
\begin{glose}
\pfra{clan maternel (dans les cérémonies de deuil)}
\end{glose}
\newline
\relationsémantique{Cf.}{\lien{ⓔapoxapenu}{apoxapenu}}
\glosecourte{clan paternel (dans les cérémonies de deuil)}
\end{entrée}

\begin{entrée}{axe}{}{ⓔaxe}
\région{GO PA}
\variante{%
haxe
\région{GO PA}}
(\domainesémantique{Conjonction})
\classe{CNJ}
\begin{glose}
\pfra{et ; mais}
\end{glose}
\newline
\begin{exemple}
\région{GO}
\textbf{\pnua{Haxe kavwö jaxa vwö jö zòò}}
\pfra{Mais il ne sera pas possible que tu nages}
\end{exemple}
\end{entrée}

\begin{entrée}{a-xè}{}{ⓔa-xè}
\région{GOs PA BO}
\région{GOs}
\variante{%
a-kè
}
(\domainesémantique{Numéraux cardinaux})
\classe{NUM (animés)}
\begin{glose}
\pfra{un (animé : homme ou animal)}
\end{glose}
\newline
\begin{exemple}
\région{GO}
\textbf{\pnua{a-ru ; a-tru}}
\pfra{deux}
\end{exemple}
\newline
\begin{exemple}
\région{GO}
\textbf{\pnua{a-ko}}
\pfra{trois}
\end{exemple}
\newline
\begin{exemple}
\région{PA}
\textbf{\pnua{axe (h)ada pòi-n}}
\pfra{il n'a qu'un enfant}
\end{exemple}
\newline
\begin{exemple}
\région{GO}
\textbf{\pnua{axe pwaje}}
\pfra{un crabe (vivant ou mort)}
\end{exemple}
\end{entrée}

\begin{entrée}{axè êgu}{}{ⓔaxè êgu}
\région{GOs PA}
\variante{%
aaxe ãgu
\région{BO}, 
aaxe êgu
\région{PA}}
(\domainesémantique{Numéraux cardinaux})
\classe{NUM}
\begin{glose}
\pfra{vingt (lit. un homme)}
\end{glose}
\newline
\begin{exemple}
\textbf{\pnua{e u uvwi axè êgu jivwa po-mã}}
\pfra{il a acheté 20 mangues}
\end{exemple}
\newline
\begin{exemple}
\textbf{\pnua{e u uvwi axè êgu jivwa po-mã}}
\pfra{il a acheté 20 mangues}
\end{exemple}
\newline
\begin{exemple}
\textbf{\pnua{axè êgu jivwa ci-kãbwa}}
\pfra{20 tissus}
\end{exemple}
\end{entrée}

\begin{entrée}{axè êgu bwa truuçi}{}{ⓔaxè êgu bwa truuçi}
\région{GO}
\variante{%
aaxè ãgu bwa truji
\région{BO}}
(\domainesémantique{Numéraux cardinaux})
\classe{NUM}
\begin{glose}
\pfra{trente (lit. un homme et 10)}
\end{glose}
\end{entrée}

\begin{entrée}{axè êguu}{}{ⓔaxè êguu}
\région{GOs PA}
(\domainesémantique{Numéraux cardinaux})
\classe{NUM}
\begin{glose}
\pfra{vingt personnes}
\end{glose}
\end{entrée}

\begin{entrée}{axe poxee}{}{ⓔaxe poxee}
\région{GOs}
(\domainesémantique{Conjonction})
\classe{CNJ}
\begin{glose}
\pfra{mais en revanche}
\end{glose}
\end{entrée}

\begin{entrée}{axe poxe hõ ma}{}{ⓔaxe poxe hõ ma}
\région{PA}
(\domainesémantique{Conjonction})
\classe{CNJ}
\begin{glose}
\pfra{mais seulement}
\end{glose}
\end{entrée}

\begin{entrée}{a-yu}{}{ⓔa-yu}
\région{GOs}
(\domainesémantique{Société})
\classe{v}
\begin{glose}
\pfra{habitant (être) de}
\end{glose}
\newline
\begin{exemple}
\textbf{\pnua{e a-yu Gomen}}
\pfra{il vit à Gomen}
\end{exemple}
\newline
\relationsémantique{Cf.}{\lien{}{êgu Gomen}}
\glosecourte{les habitants de Gomen}
\end{entrée}

\begin{entrée}{a-zaala}{}{ⓔa-zaala}
\formephonétique{aðaːla}
\région{GOs}
\classe{v}
\newline
\sens{1}
(\domainesémantique{Pêche})
\begin{glose}
\pfra{aller à la pêche (sur le plâtier)}
\end{glose}
\newline
\begin{sous-entrée}{a-zaala}{ⓔa-zaalaⓢ1ⓝa-zaala}
\begin{glose}
\pfra{aller à la chasse / pêche}
\end{glose}
\end{sous-entrée}
\newline
\sens{2}
(\domainesémantique{Chasse})
\begin{glose}
\pfra{aller à la chasse (cerf)}
\end{glose}
\end{entrée}

\begin{entrée}{azi}{}{ⓔazi}
\formephonétique{aði}
\région{GOs}
\région{PA}
\variante{%
kââli
, 
kâli
\région{BO}, 
we-khâli
\région{PA BO}}
(\domainesémantique{Corps humain})
\classe{nom}
\begin{glose}
\pfra{bile ; fiel}
\end{glose}
\begin{glose}
\pfra{vésicule biliaire}
\end{glose}
\newline
\étymologie{
\langue{POc}
\étymon{*qasu}}
\end{entrée}

\begin{entrée}{azoo}{}{ⓔazoo}
\formephonétique{aðoː}
\région{GOs}
\variante{%
alòò- ; alu
\région{PA BO}}
(\domainesémantique{Parenté})
\classe{nom}
\begin{glose}
\pfra{époux ; mari}
\end{glose}
\newline
\begin{exemple}
\région{GO}
\textbf{\pnua{azoo-nu}}
\pfra{mon mari}
\end{exemple}
\newline
\begin{exemple}
\région{PA}
\textbf{\pnua{i alòò-ny, i alu-ny}}
\pfra{c'est mon mari}
\end{exemple}
\newline
\begin{sous-entrée}{phe alòò-n}{ⓔazooⓝphe alòò-n}
\région{PA}
\begin{glose}
\pfra{se marier (prendre époux )}
\end{glose}
\newline
\relationsémantique{Cf.}{\lien{}{hazo [GOs], halòòn [PA]}}
\glosecourte{se marier}
\end{sous-entrée}
\newline
\étymologie{
\langue{POc}
\étymon{*qasawa}}
\end{entrée}

\newpage

\lettrine{b}\begin{entrée}{ba}{1}{ⓔbaⓗ1}
\formephonétique{mba}
\région{GOs BO PA}
(\domainesémantique{Poissons})
\classe{nom}
\begin{glose}
\pfra{gobie}
\end{glose}
\begin{glose}
\pfra{loche ; sardine}
\end{glose}
\nomscientifique{Awaous guamensis (Gobidés)}
\end{entrée}

\begin{entrée}{ba}{2}{ⓔbaⓗ2}
\région{BO (Corne)}
(\domainesémantique{Types de maison, architecture de la maison})
\classe{nom}
\begin{glose}
\pfra{plateforme ; place aménagée devant la case}
\end{glose}
\newline
\note{non vérifié}{général}{}
\end{entrée}

\begin{entrée}{ba}{3}{ⓔbaⓗ3}
\région{WEM PA}
(\domainesémantique{Cultures, techniques, boutures})
\classe{nom}
\begin{glose}
\pfra{mur de soutènement de la tarodière [PA]}
\end{glose}
\begin{glose}
\pfra{barrage pour dévier l'eau vers la tarodière [WEM]}
\end{glose}
\newline
\begin{sous-entrée}{ba-khia}{ⓔbaⓗ3ⓝba-khia}
\begin{glose}
\pfra{mur de soutènement du champ d'igname}
\end{glose}
\end{sous-entrée}
\newline
\begin{sous-entrée}{ba-peenu}{ⓔbaⓗ3ⓝba-peenu}
\begin{glose}
\pfra{mur de soutènement de la tarodière irriguée}
\end{glose}
\end{sous-entrée}
\newline
\begin{sous-entrée}{ba-mwa}{ⓔbaⓗ3ⓝba-mwa}
\begin{glose}
\pfra{mur de soutènement de la maison}
\end{glose}
\end{sous-entrée}
\newline
\étymologie{
\langue{POc}
\étymon{*mpaa}}
\end{entrée}

\begin{entrée}{ba-}{}{ⓔba-}
\région{GOs PA}
(\domainesémantique{Démonstratifs})
\classe{PREF.NMLZ (instrumental)}
\begin{glose}
\pfra{instrument à ; sert à}
\end{glose}
\newline
\begin{sous-entrée}{ba-iazo}{ⓔba-ⓝba-iazo}
\begin{glose}
\pfra{pierre d'herminette (adze blade)}
\end{glose}
\end{sous-entrée}
\newline
\begin{sous-entrée}{ba-hal}{ⓔba-ⓝba-hal}
\région{BO}
\begin{glose}
\pfra{rame}
\end{glose}
\end{sous-entrée}
\newline
\begin{sous-entrée}{ba-he}{ⓔba-ⓝba-he}
\région{BO}
\begin{glose}
\pfra{allume-feu}
\end{glose}
\end{sous-entrée}
\newline
\begin{sous-entrée}{ba-kaa-da}{ⓔba-ⓝba-kaa-da}
\begin{glose}
\pfra{échelle}
\end{glose}
\end{sous-entrée}
\newline
\begin{sous-entrée}{ba-kham}{ⓔba-ⓝba-kham}
\begin{glose}
\pfra{louche}
\end{glose}
\end{sous-entrée}
\newline
\begin{sous-entrée}{ka-kido}{ⓔba-ⓝka-kido}
\région{BO PA}
\begin{glose}
\pfra{bol, verre}
\end{glose}
\end{sous-entrée}
\newline
\begin{sous-entrée}{ba-tabi}{ⓔba-ⓝba-tabi}
\begin{glose}
\pfra{mortier (food-pounder)}
\end{glose}
\end{sous-entrée}
\newline
\begin{sous-entrée}{ba-tenge}{ⓔba-ⓝba-tenge}
\région{BO}
\begin{glose}
\pfra{fil}
\end{glose}
\end{sous-entrée}
\newline
\begin{sous-entrée}{ba-tipu}{ⓔba-ⓝba-tipu}
\région{BO}
\begin{glose}
\pfra{perche (pour bateau)}
\end{glose}
\end{sous-entrée}
\newline
\begin{sous-entrée}{ba-tu da ?}{ⓔba-ⓝba-tu da ?}
\begin{glose}
\pfra{pourquoi faire ? (Dubois, ms)}
\end{glose}
\end{sous-entrée}
\newline
\begin{sous-entrée}{ba-ul}{ⓔba-ⓝba-ul}
\région{PA}
\begin{glose}
\pfra{éventail}
\end{glose}
\end{sous-entrée}
\end{entrée}

\begin{entrée}{-ba}{}{ⓔ-ba}
\région{GOs}
(\domainesémantique{Démonstratifs})
\classe{DEM.DEIC.2}
\begin{glose}
\pfra{celui-là (latéralement,visible)}
\end{glose}
\newline
\begin{exemple}
\textbf{\pnua{nye mwa e-ba}}
\pfra{la maison à côté}
\end{exemple}
\newline
\begin{exemple}
\textbf{\pnua{ijè-ba}}
\pfra{celle-là (femme)}
\end{exemple}
\newline
\begin{exemple}
\textbf{\pnua{ãã-ba}}
\pfra{celui-là (homme)}
\end{exemple}
\end{entrée}

\begin{entrée}{baa}{1}{ⓔbaaⓗ1}
\région{GOs PA BO}
(\domainesémantique{Quantificateurs})
\classe{COLL}
\begin{glose}
\pfra{ensemble (dans les interpellations)}
\end{glose}
\newline
\begin{sous-entrée}{baa-roomwa !}{ⓔbaaⓗ1ⓝbaa-roomwa !}
\begin{glose}
\pfra{les femmes !}
\end{glose}
\end{sous-entrée}
\newline
\begin{sous-entrée}{baa-êmwên !}{ⓔbaaⓗ1ⓝbaa-êmwên !}
\begin{glose}
\pfra{les hommes !}
\end{glose}
\end{sous-entrée}
\end{entrée}

\begin{entrée}{baa}{2}{ⓔbaaⓗ2}
\région{GOs}
\variante{%
baang
\région{PA}, 
baan
\région{BO}}
(\domainesémantique{Couleurs})
\classe{v.stat.}
\begin{glose}
\pfra{noir ; noirci}
\end{glose}
\newline
\begin{exemple}
\région{BO}
\textbf{\pnua{baan u pubu yaai}}
\pfra{noirci par la fumée du feu}
\end{exemple}
\newline
\begin{sous-entrée}{dili baa}{ⓔbaaⓗ2ⓝdili baa}
\région{GO}
\begin{glose}
\pfra{terre noire}
\end{glose}
\end{sous-entrée}
\end{entrée}

\begin{entrée}{baa}{3}{ⓔbaaⓗ3}
\région{GOs}
(\domainesémantique{Mouvements ou actions faits avec le corps, les bras, les mains, les pieds})
\classe{v}
\begin{glose}
\pfra{frapper}
\end{glose}
\newline
\begin{exemple}
\textbf{\pnua{la baa-nu}}
\pfra{ils m'ont frappé}
\end{exemple}
\newline
\begin{exemple}
\textbf{\pnua{i baa Kawèngwa}}
\pfra{il a frappé Kawèngwa (ogre)}
\end{exemple}
\newline
\note{baani (v.t.)}{grammaire}{frapper, tuer qqn ou qqch}
\end{entrée}

\begin{entrée}{bãã}{}{ⓔbãã}
\région{BO}
(\domainesémantique{Poissons})
\classe{nom}
\begin{glose}
\pfra{silure (de rivière) [Corne]}
\end{glose}
\newline
\note{non vérifié}{général}{}
\end{entrée}

\begin{entrée}{baaba}{}{ⓔbaaba}
\région{GOs BO PA}
(\domainesémantique{Parties du corps humain : doigts, orteil})
\classe{nom}
\begin{glose}
\pfra{annulaire}
\end{glose}
\end{entrée}

\begin{entrée}{baa-bo}{}{ⓔbaa-bo}
\région{GOs}
(\domainesémantique{Chasse})
\classe{v}
\begin{glose}
\pfra{chasser la roussette}
\end{glose}
\newline
\note{baani (v.t.)}{grammaire}{chasser}
\end{entrée}

\begin{entrée}{baado}{}{ⓔbaado}
\région{GOs PA}
(\domainesémantique{Corps humain})
\classe{n (inaliénable)}
\begin{glose}
\pfra{menton}
\end{glose}
\newline
\begin{exemple}
\région{PA}
\textbf{\pnua{baadò-n}}
\pfra{son menton}
\end{exemple}
\newline
\begin{exemple}
\région{GO}
\textbf{\pnua{baadò-nu}}
\pfra{mon menton}
\end{exemple}
\end{entrée}

\begin{entrée}{baa-ja}{}{ⓔbaa-ja}
\région{GOs}
\classe{nom}
\newline
\sens{1}
(\domainesémantique{Organisation sociale})
\begin{glose}
\pfra{loi ; règle}
\end{glose}
\newline
\sens{2}
(\domainesémantique{Objets et meubles de la maison})
\begin{glose}
\pfra{balance}
\end{glose}
\end{entrée}

\begin{entrée}{baalaba}{}{ⓔbaalaba}
\région{PA BO}
(\domainesémantique{Objets et meubles de la maison})
\classe{nom}
\begin{glose}
\pfra{lit}
\end{glose}
\end{entrée}

\begin{entrée}{ba-alawe}{}{ⓔba-alawe}
\région{GOs}
(\domainesémantique{Coutumes, dons coutumiers})
\classe{nom}
\begin{glose}
\pfra{coutume de départ}
\end{glose}
\end{entrée}

\begin{entrée}{baani}{}{ⓔbaani}
\formephonétique{mbaːɳi}
\région{GOs PA BO}
(\domainesémantique{Mouvements ou actions faits avec le corps, les bras, les mains, les pieds})
\classe{v.t.}
\begin{glose}
\pfra{tuer}
\end{glose}
\begin{glose}
\pfra{frapper (de haut en bas)}
\end{glose}
\begin{glose}
\pfra{abattre (animal)}
\end{glose}
\begin{glose}
\pfra{tuer (les moustiques en tapant)}
\end{glose}
\begin{glose}
\pfra{faire fuir (animal)}
\end{glose}
\newline
\note{e baa vwo (vwo = haivwo)}{grammaire}{il en tue beaucoup}
\newline
\note{baa (v.i.)}{grammaire}{}
\end{entrée}

\begin{entrée}{baaro}{}{ⓔbaaro}
\formephonétique{baːɽω}
\région{GOs WEM BO PA}
\variante{%
baatro
\région{GO(s)}}
(\domainesémantique{Caractéristiques et propriétés des personnes})
\classe{v.stat.}
\begin{glose}
\pfra{paresseux (humain)}
\end{glose}
\newline
\begin{sous-entrée}{a-baatrö, a-baaro}{ⓔbaaroⓝa-baatrö, a-baaro}
\begin{glose}
\pfra{un paresseux, fainéant}
\end{glose}
\newline
\begin{exemple}
\textbf{\pnua{e a-baatro}}
\pfra{il est paresseux}
\end{exemple}
\newline
\relationsémantique{Cf.}{\lien{ⓔkônôô}{kônôô}}
\glosecourte{animal domestique; paresseux (dort toute la journée comme un animal domestique)}
\newline
\relationsémantique{Ant.}{\lien{ⓔa-bwaayu}{a-bwaayu}}
\glosecourte{travailleur}
\end{sous-entrée}
\end{entrée}

\begin{entrée}{baaròl}{}{ⓔbaaròl}
\région{PA BO}
(\domainesémantique{Poissons})
\classe{nom}
\begin{glose}
\pfra{loche blanche de rivière}
\end{glose}
\end{entrée}

\begin{entrée}{baatro}{}{ⓔbaatro}
\formephonétique{baːɽo}
\région{GOs}
\variante{%
baaro
\région{GO(s)}}
(\domainesémantique{Poissons})
\classe{nom}
\begin{glose}
\pfra{lochon (petit poisson d'eau douce)}
\end{glose}
\nomscientifique{Eleotris fusca (Eleotridés)}
\newline
\relationsémantique{Cf.}{\lien{ⓔthraanõ}{thraanõ}}
\glosecourte{lochon (grande taille)}
\end{entrée}

\begin{entrée}{ba-atru}{}{ⓔba-atru}
\formephonétique{ba.aʈu}
\région{GOs}
(\domainesémantique{Numéraux ordinaux})
\classe{ORD}
\begin{glose}
\pfra{deuxième}
\end{glose}
\newline
\begin{exemple}
\textbf{\pnua{ba-atru ẽnõ}}
\pfra{deuxième enfant}
\end{exemple}
\end{entrée}

\begin{entrée}{baaxò}{}{ⓔbaaxò}
\formephonétique{baːɣɔ}
\région{GOs WEM WE}
\variante{%
baaxòl
\région{BO}, 
baxòòl
\formephonétique{baɣɔːl}
\région{PA}}
\newline
\groupe{A}
\newline
\sens{1}
(\domainesémantique{Fonctions intellectuelles})
\classe{nom}
\begin{glose}
\pfra{droit ; droiture}
\end{glose}
\newline
\begin{exemple}
\région{BO}
\textbf{\pnua{ra thu(-xa) baxòòl i nu ?}}
\pfra{ai-je le droit ? (BM)}
\end{exemple}
\newline
\groupe{B}
\newline
\sens{2}
(\domainesémantique{Préfixes et verbes de position})
\classe{v.stat.}
\begin{glose}
\pfra{droit (être) ; vertical ; d'aplomb}
\end{glose}
\newline
\begin{sous-entrée}{a-baaxo}{ⓔbaaxòⓢ2ⓝa-baaxo}
\région{GO}
\begin{glose}
\pfra{marcher tout droit}
\end{glose}
\end{sous-entrée}
\newline
\begin{sous-entrée}{kô-baaxòl}{ⓔbaaxòⓢ2ⓝkô-baaxòl}
\région{PA}
\begin{glose}
\pfra{être couché en long}
\end{glose}
\newline
\note{baxòòle (v.t.)}{grammaire}{}
\end{sous-entrée}
\newline
\relationsémantique{Ant.}{\lien{}{phòng [BO]}}
\glosecourte{tordu}
\end{entrée}

\begin{entrée}{baazò}{}{ⓔbaazò}
\région{GOs}
(\domainesémantique{Préfixes et verbes de position})
\classe{LOC}
\begin{glose}
\pfra{en travers}
\end{glose}
\newline
\begin{exemple}
\textbf{\pnua{e kô-baazò cee bwa de}}
\pfra{l'arbre est couché en travers de la route}
\end{exemple}
\newline
\relationsémantique{Ant.}{\lien{ⓔku-gòò}{ku-gòò}}
\glosecourte{être droit}
\newline
\relationsémantique{Ant.}{\lien{ⓔbaaxò}{baaxò}}
\glosecourte{être droit, vertical}
\end{entrée}

\begin{entrée}{ba-cabi}{}{ⓔba-cabi}
\région{GOs PA}
\variante{%
marto
\région{GOs}}
(\domainesémantique{Outils})
\classe{nom}
\begin{glose}
\pfra{marteau ; instrument pour taper}
\end{glose}
\end{entrée}

\begin{entrée}{ba-còòxe}{}{ⓔba-còòxe}
\région{GOs}
(\domainesémantique{Instruments})
\classe{nom}
\begin{glose}
\pfra{ciseaux}
\end{glose}
\end{entrée}

\begin{entrée}{ba-êgu}{}{ⓔba-êgu}
\région{GOs}
(\domainesémantique{Société})
\classe{nom}
\begin{glose}
\pfra{femme (terme respectueux)}
\end{glose}
\end{entrée}

\begin{entrée}{ba-hã}{}{ⓔba-hã}
\région{GOs}
(\domainesémantique{Instruments})
\classe{nom}
\begin{glose}
\pfra{volant (voiture)}
\end{glose}
\end{entrée}

\begin{entrée}{ba-kudo}{}{ⓔba-kudo}
\région{GOs}
\variante{%
ba-xudo
\région{GO(s)}, 
ba-kido
\région{PA BO}}
(\domainesémantique{Ustensiles})
\classe{nom}
\begin{glose}
\pfra{verre ; bol}
\end{glose}
\end{entrée}

\begin{entrée}{ba-kûûni}{}{ⓔba-kûûni}
\formephonétique{bakûːɳi}
\région{GOs}
\variante{%
ba-kuuni
\région{BO}}
(\domainesémantique{Aspect})
\classe{nom}
\begin{glose}
\pfra{finalement ; en guise de fin ; fin}
\end{glose}
\end{entrée}

\begin{entrée}{ba-kha-da}{}{ⓔba-kha-da}
\région{PA}
(\domainesémantique{Instruments})
\classe{nom}
\begin{glose}
\pfra{échelle}
\end{glose}
\end{entrée}

\begin{entrée}{ba-kheevwo phwa}{}{ⓔba-kheevwo phwa}
\région{GOs}
\variante{%
ba-kham phwa
\région{PA BO}}
(\domainesémantique{Ustensiles})
\classe{nom}
\begin{glose}
\pfra{écumoire (lit. écope à trou)}
\end{glose}
\end{entrée}

\begin{entrée}{ba-kheevwo thô}{}{ⓔba-kheevwo thô}
\région{GOs}
\variante{%
ba-kham thô
\région{PA}, 
ba-kham thõn
\région{BO}}
(\domainesémantique{Ustensiles})
\classe{nom}
\begin{glose}
\pfra{louche (lit. écope fermée)}
\end{glose}
\newline
\relationsémantique{Cf.}{\lien{ⓔthô}{thô}}
\glosecourte{fermer}
\end{entrée}

\begin{entrée}{bala}{2}{ⓔbalaⓗ2}
\région{GOs BO}
(\domainesémantique{Relations et interaction sociales})
\classe{nom}
\begin{glose}
\pfra{partenaire ; co-équipier ; complice (groupe de plus de 2 personnes)}
\end{glose}
\newline
\begin{exemple}
\textbf{\pnua{bala-nu}}
\pfra{mon équipier}
\end{exemple}
\newline
\begin{exemple}
\textbf{\pnua{pe-bala-î}}
\pfra{nous sommes co-équipiers}
\end{exemple}
\newline
\begin{exemple}
\textbf{\pnua{pe-bala-lò}}
\pfra{ils 3 font équipe}
\end{exemple}
\newline
\begin{sous-entrée}{bala-yu}{ⓔbalaⓗ2ⓝbala-yu}
\begin{glose}
\pfra{compagnon de résidence, serviteur}
\end{glose}
\end{sous-entrée}
\end{entrée}

\begin{entrée}{bala}{3}{ⓔbalaⓗ3}
\région{GOs}
\variante{%
bala-n
\région{PA BO}}
(\domainesémantique{Configuration des objets})
\classe{nom}
\begin{glose}
\pfra{limite ; bout ; fin}
\end{glose}
\newline
\begin{exemple}
\textbf{\pnua{bala mwa}}
\pfra{les extrémités de la maison}
\end{exemple}
\newline
\begin{exemple}
\textbf{\pnua{la phe bala}}
\pfra{ils prennent la suite de / poursuivent (une oeuvre)}
\end{exemple}
\newline
\begin{exemple}
\région{PA}
\textbf{\pnua{i ra u a u bala-n}}
\pfra{il est parti définitivement}
\end{exemple}
\end{entrée}

\begin{entrée}{bala}{4}{ⓔbalaⓗ4}
\région{GOs}
\newline
\sens{1}
(\domainesémantique{Modalité, verbes modaux})
\classe{PTCL.MODAL (adversatif, hypothétique)}
\begin{glose}
\pfra{adversatif ; incertain}
\end{glose}
\newline
\begin{exemple}
\région{GOs}
\textbf{\pnua{nòme jö bala a, jö thomã-nu}}
\pfra{si jamais tu t'en vas, tu m'appelles (au cas où tu t'en irais)}
\end{exemple}
\newline
\begin{exemple}
\région{GOs}
\textbf{\pnua{awö-nu a-da bwa kavegu, xa nye bala mudra hõbwoli-nu pune digöö}}
\pfra{je voulais aller à 'eika', mais malheureusement ma robe a été déchirée par les 'cassis'}
\end{exemple}
\newline
\begin{exemple}
\région{GOs}
\textbf{\pnua{e bala uja Kumwa, xa nye e ci thraa loto i je}}
\pfra{il est arrivé à Kumac, alors même que sa voiture marchait très mal}
\end{exemple}
\newline
\begin{exemple}
\région{GOs}
\textbf{\pnua{e zaxoe kibao mèni, axe e bala tha}}
\pfra{il tentait de tuer l'oiseau, mais il l'a raté}
\end{exemple}
\newline
\begin{exemple}
\région{GOs}
\textbf{\pnua{bala thrûã !}}
\pfra{pas de chance (expression: thûã = "faire semblant"; s'emploie quand on fait quelque chose qui échoue, et qu'on fait semblant de ne pas avoir voulu le faire. La personne elle-même ou quelqu'un d'autre peut le dire)}
\end{exemple}
\newline
\sens{2}
(\domainesémantique{Modalité, verbes modaux})
\classe{PTCL.MODAL (contrastif)}
\begin{glose}
\pfra{contrastif}
\end{glose}
\newline
\begin{exemple}
\région{GOs}
\textbf{\pnua{jö bala yuu ?}}
\pfra{mais alors tu es resté ? (alors que tu devais partir)}
\end{exemple}
\newline
\begin{exemple}
\région{GOs}
\textbf{\pnua{e bala a}}
\pfra{elle poursuit son chemin (malgré les appels)}
\end{exemple}
\newline
\sens{3}
(\domainesémantique{Modalité, verbes modaux})
\classe{PTCL.MODAL}
\begin{glose}
\pfra{complètement}
\end{glose}
\newline
\begin{exemple}
\région{GOs}
\textbf{\pnua{e bala tua mwa kô-chòva ò}}
\pfra{la corde du cheval en question s'est complètement détachée}
\end{exemple}
\newline
\begin{exemple}
\région{GOs}
\textbf{\pnua{u bala thraa}}
\pfra{c'est complètement fichu !}
\end{exemple}
\newline
\begin{exemple}
\région{GOs}
\textbf{\pnua{cii bala kô-raa !}}
\pfra{c'est vraiment impossible !}
\end{exemple}
\newline
\sens{4}
(\domainesémantique{Modalité, verbes modaux})
\classe{PTCL.MODAL}
\begin{glose}
\pfra{pour toujours ; à jamais ; révolu}
\end{glose}
\newline
\begin{exemple}
\région{GOs}
\textbf{\pnua{ezoma bala mõgu}}
\pfra{elle va travailler là tout le temps (pour toujours)}
\end{exemple}
\newline
\begin{exemple}
\textbf{\pnua{e za u a vwo bala mwa jena}}
\pfra{elle est partie à jamais là-bas (pas souhaitable)}
\end{exemple}
\end{entrée}

\begin{entrée}{bala-}{1}{ⓔbala-ⓗ1}
\région{GOs BO PA}
\classe{nom}
\newline
\sens{1}
(\domainesémantique{Préfixes classificateurs sémantiques})
\begin{glose}
\pfra{morceau (allongé) ; partie ; moitié}
\end{glose}
\newline
\begin{sous-entrée}{bala-ce}{ⓔbala-ⓗ1ⓢ1ⓝbala-ce}
\région{BO}
\begin{glose}
\pfra{bout de bois}
\end{glose}
\end{sous-entrée}
\newline
\begin{sous-entrée}{bala-mada}{ⓔbala-ⓗ1ⓢ1ⓝbala-mada}
\région{PA}
\begin{glose}
\pfra{morceau d'étoffe}
\end{glose}
\end{sous-entrée}
\newline
\begin{sous-entrée}{bala-paang}{ⓔbala-ⓗ1ⓢ1ⓝbala-paang}
\région{PA}
\begin{glose}
\pfra{partie désherbée}
\end{glose}
\end{sous-entrée}
\newline
\begin{sous-entrée}{balaa-gò}{ⓔbala-ⓗ1ⓢ1ⓝbalaa-gò}
\begin{glose}
\pfra{longue-vue, jumelles (ressemblait à un bout de bambou)}
\end{glose}
\end{sous-entrée}
\newline
\sens{2}
(\domainesémantique{Préfixes classificateurs numériques})
\classe{CLF.NUM}
\begin{glose}
\pfra{canne à sucre, bois}
\end{glose}
\newline
\begin{sous-entrée}{bala-xe, bala-tru, etc.}{ⓔbala-ⓗ1ⓢ2ⓝbala-xe, bala-tru, etc.}
\begin{glose}
\pfra{un morceau de bois, deux, trois}
\end{glose}
\end{sous-entrée}
\end{entrée}

\begin{entrée}{balaa-mãã}{}{ⓔbalaa-mãã}
\région{GOs}
(\domainesémantique{Marées})
\classe{nom}
\begin{glose}
\pfra{marée haute à l'aurore}
\end{glose}
\end{entrée}

\begin{entrée}{bala-khazia}{}{ⓔbala-khazia}
\région{GOs}
(\domainesémantique{Verbes d'action (en général)})
\classe{v}
\begin{glose}
\pfra{trouver qqch par hasard}
\end{glose}
\newline
\begin{exemple}
\textbf{\pnua{nu bala-khazia nye ce}}
\pfra{j'ai trouvé cet arbre par hasard}
\end{exemple}
\newline
\relationsémantique{Cf.}{\lien{ⓔkhazia}{khazia}}
\glosecourte{(être) à côté, à proximité de}
\end{entrée}

\begin{entrée}{bala-khò}{}{ⓔbala-khò}
\région{PA}
(\domainesémantique{Configuration des objets})
\classe{nom}
\begin{glose}
\pfra{limite du labour (là où on s'est arrêté)}
\end{glose}
\end{entrée}

\begin{entrée}{bala-n}{}{ⓔbala-n}
\région{PA}
(\domainesémantique{Modalité, verbes modaux})
\classe{MODAL}
\begin{glose}
\pfra{à la suite, dans la foulée}
\end{glose}
\newline
\begin{exemple}
\textbf{\pnua{i ra u a wu bala-n}}
\pfra{elle est partie à tout jamais}
\end{exemple}
\newline
\begin{exemple}
\textbf{\pnua{li hovwo jo li bala mani ?}}
\pfra{ils ontmangé et ils sont restés dormir (à la suite)}
\end{exemple}
\newline
\begin{exemple}
\région{PA}
\textbf{\pnua{i bala kool mwa}}
\pfra{il est resté (alors qu'il était sur le point de partir)}
\end{exemple}
\newline
\begin{exemple}
\région{PA}
\textbf{\pnua{i havha kêê-n, jo nu bala khobwe}}
\pfra{son père est arrivé et je lui en ai parlé (en profitant de l'occasion)}
\end{exemple}
\newline
\begin{exemple}
\région{PA}
\textbf{\pnua{nu bala kha-phe-je}}
\pfra{je l'ai pris en route}
\end{exemple}
\end{entrée}

\begin{entrée}{bala-nyama}{}{ⓔbala-nyama}
\région{PA}
(\domainesémantique{Verbes d'action (en général)})
\classe{nom}
\begin{glose}
\pfra{travail interrompu, non fini}
\end{glose}
\end{entrée}

\begin{entrée}{bala-pho}{}{ⓔbala-pho}
\région{GOs}
(\domainesémantique{Tressage})
\classe{nom}
\begin{glose}
\pfra{limite du tressage (là où on s'est arrêté)}
\end{glose}
\end{entrée}

\begin{entrée}{bala-xè}{}{ⓔbala-xè}
\région{GOs PA}
(\domainesémantique{Préfixes classificateurs numériques})
\classe{CLF.NUM (morceaux de bois)}
\begin{glose}
\pfra{un (morceau de bois)}
\end{glose}
\newline
\begin{exemple}
\textbf{\pnua{bala-xè, bala-tru, etc.}}
\pfra{un, deux morceau(x), etc.}
\end{exemple}
\end{entrée}

\begin{entrée}{ba- ... (le)}{}{ⓔba- ... (le)}
\région{GOs PA BO}
\variante{%
na-
\région{BO vx}}
(\domainesémantique{Numéraux ordinaux})
\classe{ORD}
\begin{glose}
\pfra{n-ième}
\end{glose}
\newline
\begin{exemple}
\région{GO}
\textbf{\pnua{e phe nye baa-we-tru-le}}
\pfra{elle prend le deuxième (objet long: bout de bois)}
\end{exemple}
\newline
\begin{exemple}
\région{GO}
\textbf{\pnua{ba-pò-xe, ba-pò-tru}}
\pfra{premier, deuxième (inanimé)}
\end{exemple}
\newline
\begin{exemple}
\région{GO}
\textbf{\pnua{ba-õko xa nu nõõ-je ni tree}}
\pfra{c'est la troisième fois que je le vois dans la journée}
\end{exemple}
\newline
\begin{exemple}
\région{PA}
\textbf{\pnua{ba-pwòru, ba-poru}}
\pfra{deuxième}
\end{exemple}
\newline
\begin{exemple}
\textbf{\pnua{ba-axè êgu}}
\pfra{vingtième}
\end{exemple}
\newline
\begin{exemple}
\région{GO}
\textbf{\pnua{ba-õxe xa nu nõõ-je}}
\pfra{c'est la première fois que je le vois}
\end{exemple}
\newline
\begin{exemple}
\région{BO}
\textbf{\pnua{na-pòxe}}
\pfra{premier (inanimé) (Dubois)}
\end{exemple}
\end{entrée}

\begin{entrée}{bale}{}{ⓔbale}
\région{GOs}
(\domainesémantique{Instruments})
\classe{v ; n}
\begin{glose}
\pfra{balayer ; balai}
\end{glose}
\newline
\begin{exemple}
\textbf{\pnua{bale goo}}
\pfra{un balai (fait avec la nervure centrale des folioles de palmes de cocotier)}
\end{exemple}
\newline
\emprunt{balai (FR)}
\end{entrée}

\begin{entrée}{balevhi}{}{ⓔbalevhi}
\formephonétique{baleβi}
\région{GA}
(\domainesémantique{Topographie})
\classe{nom}
\begin{glose}
\pfra{talus}
\end{glose}
\end{entrée}

\begin{entrée}{balexa}{}{ⓔbalexa}
\région{GOs BO}
(\domainesémantique{Bananiers et bananes})
\classe{nom}
\begin{glose}
\pfra{goût d'une banane (minyô)}
\end{glose}
\newline
\note{cette banane colle à la langue quand elle n'est pas mûre (par extension, réfère à tous les fruits qui ont le même goût)}{glose}{}
\end{entrée}

\begin{entrée}{bali-cee}{}{ⓔbali-cee}
\région{GOs}
(\domainesémantique{Ponts})
\classe{nom}
\begin{glose}
\pfra{pont en bois (sur une rivière)}
\end{glose}
\end{entrée}

\begin{entrée}{ba-nhõî}{}{ⓔba-nhõî}
\formephonétique{baɳʰɔ̃î}
\région{GOs PA}
(\domainesémantique{Cordes, cordages})
\classe{nom}
\begin{glose}
\pfra{lien (lit. qui sert à attacher)}
\end{glose}
\end{entrée}

\begin{entrée}{ba-ogine}{}{ⓔba-ogine}
\région{GOs BO}
\variante{%
ba-ogin-en
\région{PA}}
(\domainesémantique{Numéraux ordinaux})
\classe{ORD}
\begin{glose}
\pfra{dernier (le ) ; fin}
\end{glose}
\newline
\begin{exemple}
\région{BO}
\textbf{\pnua{ni ba-ogine mwhããnu}}
\pfra{à la fin du mois}
\end{exemple}
\newline
\begin{exemple}
\région{PA}
\textbf{\pnua{ba-ogin-en}}
\pfra{pour terminer, pour finir}
\end{exemple}
\end{entrée}

\begin{entrée}{ba-õxe}{}{ⓔba-õxe}
\région{GOs}
(\domainesémantique{Numéraux ordinaux})
\classe{ORD}
\begin{glose}
\pfra{première fois}
\end{glose}
\newline
\begin{exemple}
\région{GOs}
\textbf{\pnua{ba-õxe xa nu nõõ-je}}
\pfra{c'est la première fois que je le vois}
\end{exemple}
\newline
\relationsémantique{Cf.}{\lien{}{ba-õtru, ba-õko etc.}}
\glosecourte{2ème, 3ème fois, etc.}
\end{entrée}

\begin{entrée}{ba-paaba}{}{ⓔba-paaba}
\région{GOs}
(\domainesémantique{Navigation})
\classe{nom}
\begin{glose}
\pfra{rame}
\end{glose}
\end{entrée}

\begin{entrée}{ba-pase}{}{ⓔba-pase}
\région{GOs}
(\domainesémantique{Instruments})
\classe{nom}
\begin{glose}
\pfra{tamis}
\end{glose}
\newline
\begin{sous-entrée}{ba-pase ô}{ⓔba-paseⓝba-pase ô}
\begin{glose}
\pfra{tamis}
\end{glose}
\end{sous-entrée}
\newline
\begin{sous-entrée}{ba-pase kafe}{ⓔba-paseⓝba-pase kafe}
\begin{glose}
\pfra{filtre à café}
\end{glose}
\end{sous-entrée}
\newline
\begin{sous-entrée}{ba-pase nu}{ⓔba-paseⓝba-pase nu}
\begin{glose}
\pfra{passoire (lit. qui sert à passer le coco)}
\end{glose}
\end{sous-entrée}
\newline
\emprunt{passer (FR)}
\end{entrée}

\begin{entrée}{ba-pe-ravhi}{}{ⓔba-pe-ravhi}
\région{PA}
(\domainesémantique{Soins du corps
, Instruments})
\classe{nom}
\begin{glose}
\pfra{rasoir}
\end{glose}
\end{entrée}

\begin{entrée}{ba-pe-thra}{}{ⓔba-pe-thra}
\formephonétique{bapeʈʰa}
\région{GOs}
\classe{nom}
(\domainesémantique{Soins du corps
, Instruments})
\begin{glose}
\pfra{rasoir}
\end{glose}
\newline
\relationsémantique{Cf.}{\lien{ⓔpe-thra}{pe-thra}}
\glosecourte{raser (se)}
\end{entrée}

\begin{entrée}{ba-phe-ẽnõ}{}{ⓔba-phe-ẽnõ}
\région{PA}
(\domainesémantique{Coutumes, dons coutumiers})
\classe{nom}
\begin{glose}
\pfra{don lors d'une adoption (pour prendre l'enfant)}
\end{glose}
\end{entrée}

\begin{entrée}{ba-phe-vwo}{}{ⓔba-phe-vwo}
\région{GOs}
(\domainesémantique{Portage})
\classe{nom}
\begin{glose}
\pfra{bât (cheval)}
\end{glose}
\begin{glose}
\pfra{porte-bagages}
\end{glose}
\end{entrée}

\begin{entrée}{ba-thaavwu}{}{ⓔba-thaavwu}
\région{GOs}
\variante{%
ba-thaavwun
\région{PA}}
(\domainesémantique{Aspect})
\classe{ASP}
\begin{glose}
\pfra{pour commencer}
\end{glose}
\end{entrée}

\begin{entrée}{ba-thiçe}{}{ⓔba-thiçe}
\formephonétique{batʰiʒe}
\région{GOs}
(\domainesémantique{Feu : objets et actions liés au feu
, Instruments})
\classe{nom}
\begin{glose}
\pfra{tisonnier}
\end{glose}
\end{entrée}

\begin{entrée}{ba-thi-halelewa}{}{ⓔba-thi-halelewa}
\région{GO PA BO}
(\domainesémantique{Instruments})
\classe{nom}
\begin{glose}
\pfra{bâton à glu (sur lequel on colle les cigales)}
\end{glose}
\end{entrée}

\begin{entrée}{ba-thu-khia}{}{ⓔba-thu-khia}
\région{PA BO}
(\domainesémantique{Organisation sociale
, Cultures, techniques, boutures})
\classe{v ; n}
\begin{glose}
\pfra{labourer le champ d'igname du chef}
\end{glose}
\begin{glose}
\pfra{ouverture du champ d'igname du chef}
\end{glose}
\end{entrée}

\begin{entrée}{ba-thu-pwaalu}{}{ⓔba-thu-pwaalu}
\région{GOs}
(\domainesémantique{Coutumes, dons coutumiers})
\classe{nom}
\begin{glose}
\pfra{geste de respect}
\end{glose}
\end{entrée}

\begin{entrée}{ba-trabwa}{}{ⓔba-trabwa}
\formephonétique{baɽabwa}
\région{GOs BO PA}
\variante{%
ba-rabwa
\région{GO(s)}}
(\domainesémantique{Objets et meubles de la maison})
\classe{nom}
\begin{glose}
\pfra{chaise ; banc ; chaise ; siège}
\end{glose}
\newline
\begin{sous-entrée}{ba-tabwa cova}{ⓔba-trabwaⓝba-tabwa cova}
\région{PA}
\begin{glose}
\pfra{selle}
\end{glose}
\end{sous-entrée}
\newline
\begin{sous-entrée}{mhenõ-(t)rabwa}{ⓔba-trabwaⓝmhenõ-(t)rabwa}
\begin{glose}
\pfra{siège (tout ce qui sert à s'asseoir)}
\end{glose}
\end{sous-entrée}
\end{entrée}

\begin{entrée}{ba-tröi}{}{ⓔba-tröi}
\formephonétique{baɽωi baɽui}
\région{GOs}
\variante{%
ba-rui
\région{PA}}
(\domainesémantique{Ustensiles})
\classe{nom}
\begin{glose}
\pfra{cuillère}
\end{glose}
\newline
\begin{sous-entrée}{ba-tröi pònò}{ⓔba-tröiⓝba-tröi pònò}
\begin{glose}
\pfra{petite cuillère}
\end{glose}
\end{sous-entrée}
\newline
\begin{sous-entrée}{ba-tröi waa}{ⓔba-tröiⓝba-tröi waa}
\begin{glose}
\pfra{grande cuillère}
\end{glose}
\newline
\relationsémantique{Cf.}{\lien{ⓔtröi}{tröi}}
\glosecourte{puiser}
\newline
\relationsémantique{Cf.}{\lien{ⓔtruu}{truu}}
\glosecourte{plonger}
\end{sous-entrée}
\end{entrée}

\begin{entrée}{ba-thrôbo}{}{ⓔba-thrôbo}
\formephonétique{baʈʰôbo}
\région{GOs}
(\domainesémantique{Description des objets, formes, consistance, taille})
\classe{nom}
\begin{glose}
\pfra{talus}
\end{glose}
\end{entrée}

\begin{entrée}{ba-u}{}{ⓔba-u}
\région{GOs}
\variante{%
ba-ul
\région{WEM WE BO PA}}
(\domainesémantique{Feu : objets et actions liés au feu
, Instruments})
\classe{nom}
\begin{glose}
\pfra{éventail (en feuille de cocotier pour le feu)}
\end{glose}
\newline
\relationsémantique{Cf.}{\lien{ⓔulaⓗ1}{ula}}
\glosecourte{éventer}
\end{entrée}

\begin{entrée}{ba-uxi-cee}{}{ⓔba-uxi-cee}
\formephonétique{bauɣicɨ}
\région{GOs}
(\domainesémantique{Outils})
\classe{nom}
\begin{glose}
\pfra{perceuse}
\end{glose}
\end{entrée}

\begin{entrée}{bavala}{}{ⓔbavala}
\région{PA BO}
(\domainesémantique{Préfixes et verbes de position})
\classe{v}
\begin{glose}
\pfra{aligné ;}
\end{glose}
\begin{glose}
\pfra{côte à côte}
\end{glose}
\newline
\begin{sous-entrée}{tee-bavala}{ⓔbavalaⓝtee-bavala}
\région{PA}
\begin{glose}
\pfra{assis côte à côte}
\end{glose}
\end{sous-entrée}
\newline
\begin{sous-entrée}{a-bavala}{ⓔbavalaⓝa-bavala}
\région{PA}
\begin{glose}
\pfra{marcher côte à côte}
\end{glose}
\end{sous-entrée}
\newline
\begin{sous-entrée}{ku-bavala}{ⓔbavalaⓝku-bavala}
\région{PA}
\begin{glose}
\pfra{debout en ligne, côte à côte}
\end{glose}
\end{sous-entrée}
\newline
\begin{sous-entrée}{ta-bavala}{ⓔbavalaⓝta-bavala}
\région{BO}
\begin{glose}
\pfra{assis en ligne [BM]}
\end{glose}
\end{sous-entrée}
\end{entrée}

\begin{entrée}{ba-zo}{}{ⓔba-zo}
\région{GOs}
\variante{%
ba-zòl
\région{PA}, 
ba-yòl
\région{BO}}
(\domainesémantique{Ustensiles})
\classe{nom}
\begin{glose}
\pfra{grattoir}
\end{glose}
\end{entrée}

\begin{entrée}{ba-zo cee}{}{ⓔba-zo cee}
\formephonétique{bazocɨ}
\région{GOs PA}
\variante{%
ba-yo
\région{BO}}
(\domainesémantique{Outils})
\classe{nom}
\begin{glose}
\pfra{scie}
\end{glose}
\end{entrée}

\begin{entrée}{be}{}{ⓔbe}
\région{GOs PA BO}
(\domainesémantique{Insectes})
\classe{nom}
\begin{glose}
\pfra{ver de terre}
\end{glose}
\newline
\étymologie{
\langue{POc}
\étymon{*mpasa, *mpaya}
\glosecourte{ver de terre}}
\end{entrée}

\begin{entrée}{bè}{1}{ⓔbèⓗ1}
\région{GOs}
(\domainesémantique{Arbre})
\classe{nom}
\begin{glose}
\pfra{banian (à racines aériennes)}
\end{glose}
\nomscientifique{Ficus sp.}
\newline
\relationsémantique{Cf.}{\lien{ⓔbumi}{bumi}}
\glosecourte{banian}
\newline
\étymologie{
\langue{PSO (Proto-South Oceanic)}
\étymon{*baqa}
\auteur{Geraghty}}
\end{entrée}

\begin{entrée}{bè}{2}{ⓔbèⓗ2}
\région{GOs}
\variante{%
bèn
\région{BO PA}}
(\domainesémantique{Oiseaux})
\classe{nom}
\begin{glose}
\pfra{ralle (oiseau) ; bécassine}
\end{glose}
\nomscientifique{Rallus philippensis swindellsi}
\end{entrée}

\begin{entrée}{bea}{}{ⓔbea}
\région{BO}
(\domainesémantique{Ignames})
\classe{nom}
\begin{glose}
\pfra{igname blanche (Dubois)}
\end{glose}
\end{entrée}

\begin{entrée}{bee}{1}{ⓔbeeⓗ1}
\région{GOs}
\variante{%
been
\région{PA BO}}
(\domainesémantique{Description des objets, formes, consistance, taille})
\classe{v.stat.}
\begin{glose}
\pfra{mouillé ; humide}
\end{glose}
\newline
\begin{exemple}
\région{GO}
\textbf{\pnua{dròò-chaamwa bee}}
\pfra{des feuilles fraîches de bananier}
\end{exemple}
\newline
\begin{exemple}
\région{GO}
\textbf{\pnua{nuu-phò bee}}
\pfra{des fibres fraîches de pandanus}
\end{exemple}
\newline
\note{pa-beene}{grammaire}{mouiller}
\end{entrée}

\begin{entrée}{bee}{2}{ⓔbeeⓗ2}
\région{PA}
(\domainesémantique{Processus liés aux plantes})
\classe{v}
\begin{glose}
\pfra{vert (tubercules, fruits)}
\end{glose}
\newline
\relationsémantique{Ant.}{\lien{ⓔte}{te}}
\glosecourte{commencer à mûrir}
\end{entrée}

\begin{entrée}{bee-}{}{ⓔbee-}
\région{GOs PA BO}
(\domainesémantique{Alliance})
\classe{nom}
\begin{glose}
\pfra{parenté par alliance}
\end{glose}
\begin{glose}
\pfra{soeur du mari ; frère d'épouse ; mari de soeur}
\end{glose}
\begin{glose}
\pfra{soeur ou frère du beau-frère ; soeur ou frère de la belle-soeur (désigne aussi 'homme parlant' les cousins parallèles de l'épouse: fils de frère de père, fils de soeur de mère et les cousins croisés de l'épouse: fils de frère de mère, fils de soeur de père)}
\end{glose}
\newline
\note{parents par alliance; anciennement "bee-" ne faisait référence qu'aux hommes, (aux 'beaux-frères) et phalawu- référait aux femmes}{glose}{}
\newline
\begin{exemple}
\région{PA}
\textbf{\pnua{bee-ny dòòmwa, thòòmwa}}
\pfra{ma belle-soeur}
\end{exemple}
\newline
\begin{exemple}
\région{GO}
\textbf{\pnua{bee-nu thòòmwa}}
\pfra{ma belle-soeur}
\end{exemple}
\newline
\begin{exemple}
\région{GO}
\textbf{\pnua{bee-nu èmwê}}
\pfra{mon beau-frère}
\end{exemple}
\newline
\relationsémantique{Cf.}{\lien{ⓔmõõ-}{mõõ-}}
\glosecourte{soeur de l'épouse}
\end{entrée}

\begin{entrée}{beela}{}{ⓔbeela}
\région{GOs PA BO}
\classe{v}
\newline
\sens{1}
(\domainesémantique{Verbes de mouvement})
\begin{glose}
\pfra{ramper (enfant) ; marcher à 4 pattes (enfant)}
\end{glose}
\newline
\sens{2}
(\domainesémantique{Verbes de mouvement})
\begin{glose}
\pfra{monter aux arbres (en serrant le tronc)}
\end{glose}
\newline
\begin{exemple}
\textbf{\pnua{e beela bwa ce}}
\pfra{il grimpe à l'arbre}
\end{exemple}
\end{entrée}

\begin{entrée}{bèèxu}{}{ⓔbèèxu}
\région{BO}
(\domainesémantique{Fonctions naturelles des animaux})
\classe{v}
\begin{glose}
\pfra{en rut [Corne]}
\end{glose}
\newline
\note{non vérifié}{général}{}
\end{entrée}

\begin{entrée}{beloo}{}{ⓔbeloo}
\région{GOs PA}
\classe{v.stat.}
\newline
\sens{1}
(\domainesémantique{Santé, maladie})
\begin{glose}
\pfra{faible ; fragile}
\end{glose}
\newline
\sens{2}
(\domainesémantique{Caractéristiques et propriétés des personnes})
\begin{glose}
\pfra{mou ; sans énergie}
\end{glose}
\begin{glose}
\pfra{lent ; indolent}
\end{glose}
\newline
\begin{exemple}
\textbf{\pnua{e môgu beloo}}
\pfra{il a travaillé mollement}
\end{exemple}
\newline
\relationsémantique{Ant.}{\lien{ⓔthani}{thani}}
\glosecourte{dynamique}
\end{entrée}

\begin{entrée}{bemãbe}{}{ⓔbemãbe}
\région{GOs}
(\domainesémantique{Caractéristiques et propriétés des personnes})
\classe{v.stat.}
\begin{glose}
\pfra{courageux ; travailleur ; débrouillard}
\end{glose}
\newline
\relationsémantique{Ant.}{\lien{ⓔkônôô}{kônôô}}
\glosecourte{paresseux}
\newline
\relationsémantique{Ant.}{\lien{}{baatro, baaro}}
\glosecourte{paresseux}
\end{entrée}

\begin{entrée}{bèn}{}{ⓔbèn}
\région{PA BO}
(\domainesémantique{Oiseaux})
\classe{nom}
\begin{glose}
\pfra{bécasse}
\end{glose}
\end{entrée}

\begin{entrée}{benõõ}{}{ⓔbenõõ}
\formephonétique{benɔ̃ː}
\région{PA BO WEM}
(\domainesémantique{Nattes})
\classe{nom}
\begin{glose}
\pfra{natte (de palmes de cocotier tressées)}
\end{glose}
\end{entrée}

\begin{entrée}{bèsè}{}{ⓔbèsè}
\région{GOs}
(\domainesémantique{Ustensiles})
\classe{nom}
\begin{glose}
\pfra{bassine ; cuvette}
\end{glose}
\newline
\emprunt{bassine (FR)}
\end{entrée}

\begin{entrée}{bèxè}{}{ⓔbèxè}
\région{GOs}
(\domainesémantique{Santé, maladie})
\classe{nom}
\begin{glose}
\pfra{bourbouille}
\end{glose}
\end{entrée}

\begin{entrée}{be-yaza}{}{ⓔbe-yaza}
\région{GOs}
\variante{%
be-ala-
\région{PA}, 
be-(y)ala
\région{WEM WE}}
(\domainesémantique{Société})
\classe{v}
\begin{glose}
\pfra{homonyme (être) (qui porte le même nom)}
\end{glose}
\newline
\begin{exemple}
\région{GO}
\textbf{\pnua{nu be-yaza gee}}
\pfra{j'ai le même nom que grand-mère}
\end{exemple}
\newline
\begin{exemple}
\région{GO}
\textbf{\pnua{pe-be-yaza-bî}}
\pfra{nous avons le même nom; nous sommes homonymes}
\end{exemple}
\newline
\begin{exemple}
\région{GO}
\textbf{\pnua{pe-be-yaza-li}}
\pfra{ils (2) sont homonymes}
\end{exemple}
\newline
\begin{exemple}
\région{PA}
\textbf{\pnua{i be-ala-ny}}
\pfra{c'est mon homonyme}
\end{exemple}
\newline
\begin{exemple}
\région{PA}
\textbf{\pnua{pe-be-(y)ala-bî}}
\pfra{nous avons le même nom}
\end{exemple}
\newline
\begin{exemple}
\région{PA}
\textbf{\pnua{li evhi /evhe be-alaa-n}}
\pfra{ils portent le même nom}
\end{exemple}
\newline
\begin{exemple}
\région{WEM}
\textbf{\pnua{ã-be-(y)ala-ny nyòli}}
\pfra{celui-là a le même nom que moi}
\end{exemple}
\end{entrée}

\begin{entrée}{bi}{1}{ⓔbiⓗ1}
\région{GOs}
\variante{%
biny
\région{PA BO WE}}
(\domainesémantique{Santé, maladie})
\classe{v.stat.}
\begin{glose}
\pfra{maigre}
\end{glose}
\begin{glose}
\pfra{dégonflé}
\end{glose}
\begin{glose}
\pfra{coupé (souffle)}
\end{glose}
\newline
\begin{exemple}
\textbf{\pnua{e bi bo}}
\pfra{le ballon est dégonflé}
\end{exemple}
\newline
\begin{exemple}
\textbf{\pnua{e bi phalawe}}
\pfra{il a le souffle coupé}
\end{exemple}
\end{entrée}

\begin{entrée}{bi}{2}{ⓔbiⓗ2}
\région{GOs}
(\domainesémantique{Jeux divers})
\classe{nom}
\begin{glose}
\pfra{bille (de petite taille ; jeu)}
\end{glose}
\newline
\emprunt{bille (FR)}
\end{entrée}

\begin{entrée}{bi}{3}{ⓔbiⓗ3}
\région{GOsPA}
(\domainesémantique{Pronoms})
\classe{PRO 1° pers. duel excl. (sujet, OBJ ou POSS)}
\begin{glose}
\pfra{nous deux (excl.) ; notre}
\end{glose}
\end{entrée}

\begin{entrée}{-bi}{}{ⓔ-bi}
\région{GOs}
\variante{%
-bin
\région{PA BO}}
(\domainesémantique{Pronoms})
\classe{PRO 1° pers. duel excl. (OBJ ou POSS)}
\begin{glose}
\pfra{nous deux (excl.) ; notre}
\end{glose}
\end{entrée}

\begin{entrée}{bibi}{}{ⓔbibi}
\région{GOs BO}
(\domainesémantique{Parenté})
\classe{nom}
\begin{glose}
\pfra{cousin croisé de même sexe (terme d'appellation)}
\end{glose}
\newline
\relationsémantique{Cf.}{\lien{ⓔebiigi}{ebiigi}}
\glosecourte{cousin croisé de même sexe}
\end{entrée}

\begin{entrée}{biça}{}{ⓔbiça}
\formephonétique{biʒa}
\région{WEM}
\classe{v ; QNT}
\newline
\sens{1}
(\domainesémantique{Verbes de déplacement et moyens de déplacement})
\begin{glose}
\pfra{dépasser}
\end{glose}
\newline
\begin{exemple}
\textbf{\pnua{nu biça i je}}
\pfra{je la dépasse}
\end{exemple}
\newline
\begin{exemple}
\textbf{\pnua{va pe-thu-mhenõ, nu kha-biça i je}}
\pfra{nous nous promenons, je la dépasse à pied}
\end{exemple}
\newline
\sens{2}
(\domainesémantique{Marques de degré})
\begin{glose}
\pfra{trop}
\end{glose}
\begin{glose}
\pfra{plus encore}
\end{glose}
\newline
\begin{exemple}
\textbf{\pnua{la ra u mara thu wöloo, mara u biça mwa !}}
\pfra{ils se sont mis à troubler l'eau et elle l'était encore plus}
\end{exemple}
\newline
\begin{exemple}
\textbf{\pnua{ra u biça halelewa!}}
\pfra{il y a énormément de cigales}
\end{exemple}
\newline
\begin{exemple}
\textbf{\pnua{biça mwani i je}}
\pfra{il a beaucoup d'argent}
\end{exemple}
\end{entrée}

\begin{entrée}{biçô}{}{ⓔbiçô}
\formephonétique{biʒô}
\région{GOs}
\classe{v}
\newline
\sens{1}
(\domainesémantique{Mouvements ou actions faits avec le corps, les bras, les mains, les pieds})
\begin{glose}
\pfra{accrocher (s') ; suspendre (se)}
\end{glose}
\newline
\begin{exemple}
\région{GO}
\textbf{\pnua{e za u biçôô-je}}
\pfra{elle s'accroche à lui}
\end{exemple}
\newline
\sens{2}
(\domainesémantique{Préfixes et verbes de position})
\begin{glose}
\pfra{accroché ; suspendu}
\end{glose}
\newline
\relationsémantique{Cf.}{\lien{ⓔcôô}{côô}}
\glosecourte{suspendu}
\end{entrée}

\begin{entrée}{bii}{1}{ⓔbiiⓗ1}
\région{GOs}
(\domainesémantique{Description des objets, formes, consistance, taille})
\classe{v ; n}
\begin{glose}
\pfra{enfoncé ; cabossé}
\end{glose}
\begin{glose}
\pfra{choc ; impact}
\end{glose}
\newline
\begin{exemple}
\textbf{\pnua{e bi wãge}}
\pfra{il s'est enfoncé la cage thoracique}
\end{exemple}
\newline
\begin{sous-entrée}{tha-bii loto}{ⓔbiiⓗ1ⓝtha-bii loto}
\begin{glose}
\pfra{cabosser la voiture}
\end{glose}
\end{sous-entrée}
\end{entrée}

\begin{entrée}{bii}{2}{ⓔbiiⓗ2}
\région{GOs PA}
(\domainesémantique{Cultures, techniques, boutures})
\classe{nom}
\begin{glose}
\pfra{conduite d'eau en bambou (amène l'eau de la rivière à la tarodière) ; canalisation}
\end{glose}
\newline
\begin{exemple}
\textbf{\pnua{lha thu bii}}
\pfra{elles font une conduite d'eau}
\end{exemple}
\end{entrée}

\begin{entrée}{biibu}{}{ⓔbiibu}
\région{GOs}
\variante{%
bibwo
\région{GO(s)}}
(\domainesémantique{Mouvements ou actions avec la tête, les yeux, la bouche})
\classe{v}
\begin{glose}
\pfra{embrasser ; faire un baiser}
\end{glose}
\end{entrée}

\begin{entrée}{biije}{}{ⓔbiije}
\formephonétique{biːɲɟe}
\région{GOs PA BO}
(\domainesémantique{Aliments, alimentation})
\classe{v}
\begin{glose}
\pfra{mâcher (pour extraire le jus) ; mastiquer (des fibres, l'écorce du bourao, du magnania)}
\end{glose}
\newline
\relationsémantique{Cf.}{\lien{ⓔhovwo}{hovwo}}
\glosecourte{manger (général)}
\newline
\relationsémantique{Cf.}{\lien{}{cèni, cani}}
\glosecourte{manger (féculents)}
\newline
\relationsémantique{Cf.}{\lien{ⓔhuu}{huu}}
\glosecourte{manger (nourriture carnée)}
\newline
\relationsémantique{Cf.}{\lien{}{huli}}
\glosecourte{manger (de la viande, du coco)}
\newline
\relationsémantique{Cf.}{\lien{ⓔkûûniⓗ2}{kûûni}}
\glosecourte{manger (fruits)}
\end{entrée}

\begin{entrée}{bile}{}{ⓔbile}
\région{GOs BO}
\variante{%
bire
\région{GOs}}
(\domainesémantique{Mouvements ou actions faits avec le corps, les bras, les mains, les pieds
, Cordes, cordages})
\classe{v ; n}
\begin{glose}
\pfra{rouler (un toron de corde sur la cuisse)}
\end{glose}
\begin{glose}
\pfra{enrouler}
\end{glose}
\begin{glose}
\pfra{filer}
\end{glose}
\begin{glose}
\pfra{corde à torons tordus}
\end{glose}
\newline
\begin{sous-entrée}{bile wa}{ⓔbileⓝbile wa}
\begin{glose}
\pfra{rouler un toron}
\end{glose}
\end{sous-entrée}
\newline
\begin{sous-entrée}{wa-bile}{ⓔbileⓝwa-bile}
\begin{glose}
\pfra{fil de filet}
\end{glose}
\end{sous-entrée}
\newline
\étymologie{
\langue{POc}
\étymon{*piri}
\glosecourte{plait a cord, twist}
\auteur{Blust}}
\end{entrée}

\begin{entrée}{bilòò}{}{ⓔbilòò}
\région{GOs BO PA}
(\domainesémantique{Mouvements ou actions faits avec le corps, les bras, les mains, les pieds})
\classe{v}
\begin{glose}
\pfra{tourner}
\end{glose}
\begin{glose}
\pfra{tordre; enrouler (une corde)}
\end{glose}
\begin{glose}
\pfra{moudre}
\end{glose}
\begin{glose}
\pfra{essorer}
\end{glose}
\newline
\begin{exemple}
\textbf{\pnua{e khaa-bilòò kòò-je}}
\pfra{elle s'est tordu la cheville (lit. appuyer tordre)}
\end{exemple}
\newline
\begin{exemple}
\textbf{\pnua{e khaa-bilòò hîî-je}}
\pfra{elle s'est tordu le poignet (en tombant et en appuyant dessus)}
\end{exemple}
\newline
\begin{sous-entrée}{bilòò kafe}{ⓔbilòòⓝbilòò kafe}
\begin{glose}
\pfra{moudre le café (en tournant le moulin)}
\end{glose}
\end{sous-entrée}
\newline
\begin{sous-entrée}{bilòò-du}{ⓔbilòòⓝbilòò-du}
\begin{glose}
\pfra{visser}
\end{glose}
\end{sous-entrée}
\newline
\begin{sous-entrée}{bilòò-da}{ⓔbilòòⓝbilòò-da}
\begin{glose}
\pfra{dévisser}
\end{glose}
\end{sous-entrée}
\end{entrée}

\begin{entrée}{biluu}{}{ⓔbiluu}
\région{GOs PA BO}
(\domainesémantique{Mollusques})
\classe{nom}
\begin{glose}
\pfra{escargot de terre ; bulime}
\end{glose}
\nomscientifique{Auricula auris}
\end{entrée}

\begin{entrée}{bîni}{}{ⓔbîni}
\formephonétique{bîɳi}
\région{GOs}
\variante{%
biini
\région{PA}, 
bîni
\région{BO}}
(\domainesémantique{Mouvements ou actions faits avec le corps, les bras, les mains, les pieds})
\classe{v.t.}
\begin{glose}
\pfra{plier (linge)}
\end{glose}
\begin{glose}
\pfra{enrouler (tissu, natte)}
\end{glose}
\newline
\begin{exemple}
\textbf{\pnua{nu bî hõbwò}}
\pfra{j'ai plié le linge}
\end{exemple}
\newline
\begin{exemple}
\textbf{\pnua{i bîni wal}}
\pfra{j'ai plié le linge}
\end{exemple}
\end{entrée}

\begin{entrée}{bizi}{}{ⓔbizi}
\région{GOs}
\variante{%
biri
\région{BO}}
(\domainesémantique{Mouvements ou actions faits avec le corps, les bras, les mains, les pieds})
\classe{v}
\begin{glose}
\pfra{étrangler (avec les mains)}
\end{glose}
\newline
\begin{exemple}
\région{BO}
\textbf{\pnua{i biri nõõ-n}}
\pfra{il l'étrangle (lit. lui serre le cou)}
\end{exemple}
\end{entrée}

\begin{entrée}{bizigi}{}{ⓔbizigi}
\région{GOs}
\variante{%
birigi
\région{GO(s) WE}, 
bitigi
\région{PA BO}}
\classe{v.stat.}
\newline
\sens{1}
(\domainesémantique{Description des objets, formes, consistance, taille})
\begin{glose}
\pfra{collé}
\end{glose}
\begin{glose}
\pfra{collant (sous la dent) [PA]}
\end{glose}
\newline
\begin{exemple}
\région{PA}
\textbf{\pnua{i bitigi nye kui}}
\pfra{cette igname est pâteuse, collante}
\end{exemple}
\newline
\sens{2}
(\domainesémantique{Matière, matériaux})
\begin{glose}
\pfra{résine [PA] (de sapin, kaori) ; collant comme de la résine}
\end{glose}
\newline
\begin{sous-entrée}{pha-bizigi-ni}{ⓔbizigiⓢ2ⓝpha-bizigi-ni}
\région{GO}
\begin{glose}
\pfra{coller qqch}
\end{glose}
\newline
\relationsémantique{Cf.}{\lien{ⓔtigiⓗ1}{tigi}}
\glosecourte{être pris (dans un filet)}
\end{sous-entrée}
\end{entrée}

\begin{entrée}{bo}{}{ⓔbo}
\formephonétique{mbo}
\région{GOs PA BO}
\variante{%
bwo
\région{GO(s) BO}, 
bon, bwon
\région{BO}}
(\domainesémantique{Fonctions naturelles humaines})
\classe{v ; n}
\begin{glose}
\pfra{sentir (odeur) ; odeur ; avoir une odeur}
\end{glose}
\newline
\begin{exemple}
\textbf{\pnua{bô-jo}}
\pfra{ton odeur}
\end{exemple}
\newline
\begin{sous-entrée}{bo-raa; bo traa}{ⓔboⓝbo-raa; bo traa}
\région{GO}
\begin{glose}
\pfra{sentir mauvais}
\end{glose}
\end{sous-entrée}
\newline
\begin{sous-entrée}{bo-zo}{ⓔboⓝbo-zo}
\begin{glose}
\pfra{sentir bon}
\end{glose}
\end{sous-entrée}
\newline
\begin{sous-entrée}{me-bo zo}{ⓔboⓝme-bo zo}
\région{BO}
\begin{glose}
\pfra{bonne odeur}
\end{glose}
\newline
\note{v.t. bole}{grammaire}{}
\end{sous-entrée}
\newline
\étymologie{
\langue{POc}
\étymon{*boni}}
\end{entrée}

\begin{entrée}{bò}{1}{ⓔbòⓗ1}
\région{GOs}
\variante{%
bòl
\région{BO}}
(\domainesémantique{Mouvements ou actions faits avec le corps, les bras, les mains, les pieds})
\classe{v}
\begin{glose}
\pfra{taper avec un bâton}
\end{glose}
\newline
\begin{exemple}
\textbf{\pnua{e aa-bò}}
\pfra{il frappe souvent}
\end{exemple}
\newline
\begin{exemple}
\région{GO}
\textbf{\pnua{e bòzi-nu}}
\pfra{il m'a tapé}
\end{exemple}
\newline
\begin{sous-entrée}{ba-bò}{ⓔbòⓗ1ⓝba-bò}
\begin{glose}
\pfra{bâton}
\end{glose}
\newline
\note{bòli [PA], bòzi [GOs] (v.t.)}{grammaire}{frapper qqch}
\end{sous-entrée}
\end{entrée}

\begin{entrée}{bò}{2}{ⓔbòⓗ2}
\formephonétique{bɔŋ}
\région{GOs}
\variante{%
bòng
\région{BO PA}}
(\domainesémantique{Topographie})
\classe{nom}
\begin{glose}
\pfra{pente (de montagne) ; ravin}
\end{glose}
\newline
\begin{exemple}
\région{GO}
\textbf{\pnua{kòlò bò}}
\pfra{le bord du ravin}
\end{exemple}
\newline
\begin{exemple}
\région{BO}
\textbf{\pnua{kòlò bòng}}
\pfra{le bord du ravin, le talus}
\end{exemple}
\end{entrée}

\begin{entrée}{bò}{3}{ⓔbòⓗ3}
\région{GOs}
\variante{%
bòng
\région{BO [BM]}}
(\domainesémantique{Processus liés aux plantes})
\classe{v}
\begin{glose}
\pfra{faner (feuilles d'arbre après un feu)}
\end{glose}
\begin{glose}
\pfra{roussi (par le feu)}
\end{glose}
\newline
\relationsémantique{Cf.}{\lien{ⓔmènõ}{mènõ}}
\glosecourte{faner naturellement}
\end{entrée}

\begin{entrée}{bö}{}{ⓔbö}
\région{WE BO}
\variante{%
bwö
\région{BO}}
(\domainesémantique{Feu : objets et actions liés au feu})
\classe{v}
\begin{glose}
\pfra{éteindre un feu}
\end{glose}
\newline
\begin{exemple}
\région{BO}
\textbf{\pnua{u bö yaai}}
\pfra{le feu s'est éteint [BM]}
\end{exemple}
\newline
\begin{exemple}
\région{WEM}
\textbf{\pnua{bööni yaai}}
\pfra{éteindre le feu}
\end{exemple}
\newline
\relationsémantique{Cf.}{\lien{}{thi-bööni [GOs]}}
\glosecourte{éteindre un feu}
\end{entrée}

\begin{entrée}{bòdra}{}{ⓔbòdra}
\formephonétique{bɔɖa}
\région{GOs}
\variante{%
bòda
\région{BO}}
\classe{nom}
\newline
\sens{1}
(\domainesémantique{Couleurs})
\begin{glose}
\pfra{teinture noire}
\end{glose}
\newline
\sens{2}
(\domainesémantique{Noms des plantes})
\begin{glose}
\pfra{Alphitonia}
\end{glose}
\nomscientifique{Alphitonia neocaledonica}
\end{entrée}

\begin{entrée}{böji-hi}{}{ⓔböji-hi}
\région{GOs}
(\domainesémantique{Corps humain})
\classe{nom}
\begin{glose}
\pfra{coude ; l'articulation de (tout) le bras}
\end{glose}
\begin{glose}
\pfra{poignet}
\end{glose}
\end{entrée}

\begin{entrée}{böji-kò}{}{ⓔböji-kò}
\région{GOs}
(\domainesémantique{Corps humain})
\classe{nom}
\begin{glose}
\pfra{articulation de (toute) la jambe}
\end{glose}
\end{entrée}

\begin{entrée}{böleõne bö}{}{ⓔböleõne bö}
\formephonétique{bωle}
\région{GOs BO}
\variante{%
bööle
\région{PA}}
(\domainesémantique{Fonctions naturelles humaines})
\classe{v}
\begin{glose}
\pfra{sentir (volontairement)}
\end{glose}
\begin{glose}
\pfra{humer}
\end{glose}
\newline
\begin{exemple}
\région{GO}
\textbf{\pnua{e böle mu-cee}}
\pfra{il sent (le parfum de) la fleur (en prenant la fleur)}
\end{exemple}
\newline
\begin{exemple}
\textbf{\pnua{e bö yai hôboli-nu}}
\pfra{ma robe sent le feu}
\end{exemple}
\newline
\relationsémantique{Cf.}{\lien{}{trõne bö [GOs]}}
\glosecourte{sentir (sans intention)}
\end{entrée}

\begin{entrée}{bòli}{}{ⓔbòli}
\région{GOs}
(\domainesémantique{Directions})
\classe{DIR}
\begin{glose}
\pfra{là-bas en bas}
\end{glose}
\newline
\begin{exemple}
\textbf{\pnua{nòò-je bòli !}}
\pfra{regarde-le/là là-bas en bas !}
\end{exemple}
\end{entrée}

\begin{entrée}{bòlòmaxaò}{}{ⓔbòlòmaxaò}
\région{GOs}
\variante{%
boloxao
\région{GO(s)}, 
vaaci
\région{WE}, 
vaci
\région{PA}}
(\domainesémantique{Mammifères})
\classe{nom}
\begin{glose}
\pfra{bétail}
\end{glose}
\newline
\emprunt{bull and cow (GB)}
\end{entrée}

\begin{entrée}{bò-na}{}{ⓔbò-na}
\formephonétique{bɔɳa}
\région{GOs}
\variante{%
bwòn-na
\région{PA BO}, 
bòna
\région{BO}}
(\domainesémantique{Adverbes déictiques de temps})
\classe{ADV}
\begin{glose}
\pfra{après-demain ; le lendemain}
\end{glose}
\begin{glose}
\pfra{à l'avenir}
\end{glose}
\newline
\begin{sous-entrée}{bò-na ô-xe}{ⓔbò-naⓝbò-na ô-xe}
\begin{glose}
\pfra{après-après-demain; dans 3 jours}
\end{glose}
\end{sous-entrée}
\end{entrée}

\begin{entrée}{böö}{}{ⓔböö}
\formephonétique{mbω}
\région{GOs}
(\domainesémantique{Santé, maladie})
\classe{v}
\begin{glose}
\pfra{réaction cutanée (avoir une) à qqch ; s'infecter avec la sève de l'acajou}
\end{glose}
\newline
\begin{exemple}
\région{GO}
\textbf{\pnua{e böö phwa-je xo pò-mha}}
\pfra{sa bouche a eu une réaction cutanée à cause de la mangue}
\end{exemple}
\end{entrée}

\begin{entrée}{booli}{}{ⓔbooli}
\région{PA WEM BO}
\variante{%
bwooli
\région{WEM BO}}
(\domainesémantique{Arbre})
\classe{nom}
\begin{glose}
\pfra{arbre (à bois dur, bois qui fait des étincelles, n'est donc pas utilisé pour le feu de nuit)}
\end{glose}
\nomscientifique{Acronychia laevis, Corne)}
\end{entrée}

\begin{entrée}{bööni}{}{ⓔbööni}
\formephonétique{mbωːɳi}
\région{GOs PA BO}
(\domainesémantique{Types de maison, architecture de la maison})
\classe{v}
\begin{glose}
\pfra{construire (un mur) ; faire un mur}
\end{glose}
\newline
\note{bö (v.i.)}{grammaire}{construire}
\end{entrée}

\begin{entrée}{bosu}{}{ⓔbosu}
\région{GOs}
(\domainesémantique{Relations et interaction sociales})
\classe{v}
\begin{glose}
\pfra{saluer (se)}
\end{glose}
\newline
\begin{exemple}
\textbf{\pnua{mo pe-bosu}}
\pfra{nous nous saluons}
\end{exemple}
\newline
\emprunt{bonjour (FR)}
\end{entrée}

\begin{entrée}{boxola}{}{ⓔboxola}
\région{PA}
(\domainesémantique{Noms des plantes})
\classe{nom}
\begin{glose}
\pfra{herbe coupante et malodorante}
\end{glose}
\end{entrée}

\begin{entrée}{bòzi}{}{ⓔbòzi}
\région{GOs}
\variante{%
bòli
\région{BO PA}}
(\domainesémantique{Mouvements ou actions faits avec le corps, les bras, les mains, les pieds})
\classe{v}
\begin{glose}
\pfra{fouetter}
\end{glose}
\begin{glose}
\pfra{corriger ; châtier}
\end{glose}
\end{entrée}

\begin{entrée}{bozo}{}{ⓔbozo}
\région{GOs}
\variante{%
bolo
\région{PA WE}}
(\domainesémantique{Corps humain})
\classe{nom}
\begin{glose}
\pfra{cordon ombilical ; nom de l'arbre planté à la naissance (et sous lequel on enterrait le cordon ombilical)}
\end{glose}
\newline
\begin{exemple}
\textbf{\pnua{bozo-nu}}
\pfra{mon cordon ombilical}
\end{exemple}
\newline
\begin{sous-entrée}{phwe-bozo}{ⓔbozoⓝphwe-bozo}
\begin{glose}
\pfra{nombril}
\end{glose}
\end{sous-entrée}
\newline
\étymologie{
\langue{POc}
\étymon{*mpusos}}
\end{entrée}

\begin{entrée}{bu}{1}{ⓔbuⓗ1}
\région{GOs PA BO}
(\domainesémantique{Aliments, alimentation
, Processus liés aux plantes})
\classe{v.stat.}
\begin{glose}
\pfra{vert (fruit) ; pas mûr}
\end{glose}
\newline
\relationsémantique{Ant.}{\lien{ⓔmii}{mii}}
\glosecourte{mûr}
\end{entrée}

\begin{entrée}{bu}{2}{ⓔbuⓗ2}
\région{GOs PA}
(\domainesémantique{Cultures, techniques, boutures})
\classe{nom}
\begin{glose}
\pfra{talus}
\end{glose}
\begin{glose}
\pfra{tertre}
\end{glose}
\begin{glose}
\pfra{tas}
\end{glose}
\begin{glose}
\pfra{billon}
\end{glose}
\newline
\begin{exemple}
\région{PA}
\textbf{\pnua{bu dili}}
\pfra{billon de terre}
\end{exemple}
\newline
\étymologie{
\langue{PEOc (Proto-Eastern Oceanic)}
\étymon{*apu}
\glosecourte{mound for house site}
\auteur{Blust}}
\end{entrée}

\begin{entrée}{bu}{3}{ⓔbuⓗ3}
\région{GOs}
\variante{%
bul
\région{BO PA}}
(\domainesémantique{Verbes de mouvement
, Navigation})
\classe{v}
\begin{glose}
\pfra{sombrer ; couler ; noyer (se)}
\end{glose}
\begin{glose}
\pfra{chavirer ; retourner (se)}
\end{glose}
\newline
\begin{exemple}
\région{PA}
\textbf{\pnua{i bule wòny}}
\pfra{le bateau a coulé}
\end{exemple}
\newline
\begin{exemple}
\région{GO}
\textbf{\pnua{nu pa-bu-ni wô}}
\pfra{j'ai fait chavirer le bateau}
\end{exemple}
\end{entrée}

\begin{entrée}{bu}{4}{ⓔbuⓗ4}
\région{GOs PA}
(\domainesémantique{Mouvements ou actions faits avec le corps, les bras, les mains, les pieds})
\classe{v}
\begin{glose}
\pfra{frapper ; taper (une cible)}
\end{glose}
\newline
\begin{exemple}
\région{GO}
\textbf{\pnua{e bu-i je}}
\pfra{il l'a frappé, il est entré en collision avec lui}
\end{exemple}
\newline
\begin{exemple}
\région{GO}
\textbf{\pnua{e bu ni ce xo loto}}
\pfra{la voiture est entrée dans l'arbre}
\end{exemple}
\newline
\begin{exemple}
\région{GO}
\textbf{\pnua{li pe-bu-i li}}
\pfra{ils sont entrés en collision}
\end{exemple}
\newline
\begin{exemple}
\textbf{\pnua{e bu hòò}}
\pfra{ça a tapé loin}
\end{exemple}
\newline
\begin{sous-entrée}{jige bu hòò}{ⓔbuⓗ4ⓝjige bu hòò}
\région{GO}
\begin{glose}
\pfra{fusil à longue portée (lit. qui tape loin)}
\end{glose}
\newline
\note{bule (v.t)}{grammaire}{frapper}
\end{sous-entrée}
\end{entrée}

\begin{entrée}{bu}{5}{ⓔbuⓗ5}
\région{GOs}
(\domainesémantique{Reptiles marins})
\classe{nom}
\begin{glose}
\pfra{tortue "bonne écaille"}
\end{glose}
\end{entrée}

\begin{entrée}{bu}{6}{ⓔbuⓗ6}
\région{PA}
(\domainesémantique{Sentiments})
\classe{v}
\begin{glose}
\pfra{refuser ; ne pas vouloir}
\end{glose}
\newline
\begin{exemple}
\région{PA}
\textbf{\pnua{bu je ne i zange je vo nyama}}
\pfra{il refuse de l'aider à travailler}
\end{exemple}
\newline
\begin{exemple}
\région{PA}
\textbf{\pnua{bu je yaai na nûû}}
\pfra{il ne veut pas que le feu éclaire}
\end{exemple}
\end{entrée}

\begin{entrée}{bu-}{}{ⓔbu-}
\région{GOs PA}
(\domainesémantique{Préfixes classificateurs numériques})
\classe{CLF.NUM}
\begin{glose}
\pfra{billon (d'igname), monticule}
\end{glose}
\newline
\begin{exemple}
\textbf{\pnua{bu-xe, bu-ru,}}
\pfra{un, deux}
\end{exemple}
\newline
\begin{exemple}
\région{PA}
\textbf{\pnua{bu-xe bu kui, bu-ru bu kui}}
\pfra{un, deux buttes d'igname}
\end{exemple}
\newline
\begin{exemple}
\région{PA}
\textbf{\pnua{bu-ru bu õn}}
\pfra{deux tas de sable}
\end{exemple}
\end{entrée}

\begin{entrée}{bua}{}{ⓔbua}
\formephonétique{bu.a}
\région{GOs PA BO}
(\domainesémantique{Relations et interaction sociales})
\classe{v}
\begin{glose}
\pfra{héler ; faire signe}
\end{glose}
\begin{glose}
\pfra{crier (pour annoncer sa présence)}
\end{glose}
\end{entrée}

\begin{entrée}{buaõ}{}{ⓔbuaõ}
\formephonétique{buaɔ̃}
\région{GOs}
\variante{%
buaôn
\région{BO}, 
bwaô
\région{PA}}
(\domainesémantique{Reptiles marins})
\classe{nom}
\begin{glose}
\pfra{serpent de mer (tricot rayé) ; plature}
\end{glose}
\newline
\relationsémantique{Cf.}{\lien{ⓔbwaaⓗ1}{bwaa}}
\glosecourte{serpent de mer (gris)}
\end{entrée}

\begin{entrée}{bubu}{}{ⓔbubu}
\région{BO PA}
(\domainesémantique{Couleurs})
\classe{v.stat.}
\begin{glose}
\pfra{vert, bleu}
\end{glose}
\end{entrée}

\begin{entrée}{bue}{}{ⓔbue}
\formephonétique{buɨ}
\région{PA}
(\domainesémantique{Mouvements ou actions avec la tête, les yeux, la bouche})
\classe{v}
\begin{glose}
\pfra{embrasser}
\end{glose}
\end{entrée}

\begin{entrée}{bu-êgu}{}{ⓔbu-êgu}
\région{PA}
(\domainesémantique{Cours de la vie})
\classe{nom}
\begin{glose}
\pfra{tombe}
\end{glose}
\end{entrée}

\begin{entrée}{bui}{}{ⓔbui}
\région{BO PA}
(\domainesémantique{Vêtements, parure})
\classe{nom}
\begin{glose}
\pfra{bracelet (formé d'un seul coquillage taillé ; Dubois ms)}
\end{glose}
\end{entrée}

\begin{entrée}{buji-}{}{ⓔbuji-}
\région{GOs PA}
(\domainesémantique{Corps humain})
\classe{nom}
\begin{glose}
\pfra{articulation}
\end{glose}
\newline
\begin{exemple}
\région{WEM}
\textbf{\pnua{buyini-hi}}
\pfra{coude}
\end{exemple}
\newline
\begin{exemple}
\région{WEM}
\textbf{\pnua{buyini-ko}}
\pfra{genou}
\end{exemple}
\end{entrée}

\begin{entrée}{bûkû}{}{ⓔbûkû}
\région{GOs}
\variante{%
bûûwû
\région{BO [BM]}}
(\domainesémantique{Fonctions naturelles humaines})
\classe{v}
\begin{glose}
\pfra{sentir mauvais (odeur de chair)}
\end{glose}
\newline
\begin{exemple}
\région{GO}
\textbf{\pnua{li bûkû hii-nu}}
\pfra{mes mains sentent le poisson}
\end{exemple}
\newline
\relationsémantique{Cf.}{\lien{}{bû nò}}
\glosecourte{odeur de poisson, sentir le poisson}
\end{entrée}

\begin{entrée}{bu-kui}{}{ⓔbu-kui}
\région{GOs PA}
(\domainesémantique{Ignames
, Cultures, techniques, boutures})
\classe{nom}
\begin{glose}
\pfra{butte d'igname}
\end{glose}
\end{entrée}

\begin{entrée}{bulago}{}{ⓔbulago}
\région{GOs}
(\domainesémantique{Processus liés aux plantes})
\classe{v}
\begin{glose}
\pfra{pourri; effriter (s')}
\end{glose}
\newline
\begin{exemple}
\région{GOs}
\textbf{\pnua{e bulago kòli phwee-mwa}}
\pfra{le linteau de la porte s'effrite (mangé par les termites), est pourri}
\end{exemple}
\end{entrée}

\begin{entrée}{bulaivi}{}{ⓔbulaivi}
\région{GOs PA BO}
(\domainesémantique{Armes})
\classe{nom}
\begin{glose}
\pfra{casse-tête (générique)}
\end{glose}
\end{entrée}

\begin{entrée}{bule}{}{ⓔbule}
\région{GOs PA}
(\domainesémantique{Verbes de mouvement})
\classe{v}
\begin{glose}
\pfra{toucher une cible}
\end{glose}
\newline
\begin{exemple}
\région{GO}
\textbf{\pnua{e bule mèni}}
\pfra{il a touché l'oiseau}
\end{exemple}
\newline
\begin{exemple}
\textbf{\pnua{li pe-bu-i-li}}
\pfra{ils sont entrés en collision}
\end{exemple}
\end{entrée}

\begin{entrée}{buleony}{}{ⓔbuleony}
\région{BO [Corne, BM]}
\variante{%
buleon
}
(\domainesémantique{Cocotiers})
\classe{nom}
\begin{glose}
\pfra{spathe de cocotier (feuille qui enveloppe l'inflorescence ou qui est à la base du pédoncule floral)}
\end{glose}
\newline
\note{utilisé pour envelopper la monnaie traditionnelle}{glose}{}
\end{entrée}

\begin{entrée}{buli}{}{ⓔbuli}
\région{GOs BO}
(\domainesémantique{Société})
\classe{v.stat.}
\begin{glose}
\pfra{décimé ; sans descendance}
\end{glose}
\end{entrée}

\begin{entrée}{bulu}{}{ⓔbulu}
\région{GOs PA BO}
\newline
\groupe{A}
(\domainesémantique{Quantificateurs})
\classe{nom}
\begin{glose}
\pfra{groupe [PA, BO]}
\end{glose}
\begin{glose}
\pfra{bande (oiseaux, enfants) [PA, BO]}
\end{glose}
\begin{glose}
\pfra{tas [PA, BO]}
\end{glose}
\begin{glose}
\pfra{assemblage , réunion}
\end{glose}
\newline
\begin{exemple}
\région{BO}
\textbf{\pnua{yu nòòle lana bulu paa ?}}
\pfra{tu vois ce tas de cailloux ? [BM]}
\end{exemple}
\newline
\begin{sous-entrée}{bulu ênõ}{ⓔbuluⓝbulu ênõ}
\begin{glose}
\pfra{bande d'enfants}
\end{glose}
\end{sous-entrée}
\newline
\begin{sous-entrée}{bulu mèni}{ⓔbuluⓝbulu mèni}
\begin{glose}
\pfra{bande d'oiseaux}
\end{glose}
\end{sous-entrée}
\newline
\begin{sous-entrée}{bulu paa}{ⓔbuluⓝbulu paa}
\région{BO}
\begin{glose}
\pfra{tas de pierre}
\end{glose}
\end{sous-entrée}
\newline
\groupe{B}
(\domainesémantique{Quantificateurs})
\classe{COLL}
\begin{glose}
\pfra{ensemble}
\end{glose}
\newline
\begin{exemple}
\textbf{\pnua{li a bulu}}
\pfra{ils partent ensemble}
\end{exemple}
\newline
\begin{sous-entrée}{pe-bulu-ni}{ⓔbuluⓝpe-bulu-ni}
\begin{glose}
\pfra{être ensemble}
\end{glose}
\end{sous-entrée}
\newline
\begin{sous-entrée}{naa bulu}{ⓔbuluⓝnaa bulu}
\begin{glose}
\pfra{mettre ensemble}
\end{glose}
\end{sous-entrée}
\newline
\begin{sous-entrée}{thaivwi bulu}{ⓔbuluⓝthaivwi bulu}
\région{GO}
\begin{glose}
\pfra{entasser, ramasser}
\end{glose}
\end{sous-entrée}
\newline
\begin{sous-entrée}{pa-bulu-ni}{ⓔbuluⓝpa-bulu-ni}
\begin{glose}
\pfra{entasser}
\end{glose}
\newline
\relationsémantique{Cf.}{\lien{}{a poxe}}
\glosecourte{partir ensemble (lit. un)}
\end{sous-entrée}
\end{entrée}

\begin{entrée}{bumi}{}{ⓔbumi}
\région{GOs PA BO}
\variante{%
bumîî
\région{BO}}
\classe{nom}
\newline
\sens{1}
(\domainesémantique{Arbre})
\begin{glose}
\pfra{banian}
\end{glose}
\newline
\note{les noeuds dans les bandes de tapa permettaient de transmettre des messages d'un groupe à un autre; les bagayou étaient faits en tapa}{glose}{}
\nomscientifique{Ficus obliqua}
\newline
\sens{2}
(\domainesémantique{Arbre})
\begin{glose}
\pfra{balassor ; arbre à tapa}
\end{glose}
\newline
\note{les noeuds dans les bandes de tapa permettaient de transmettre des messages d'un groupe à un autre; les bagayou étaient faits en tapa}{glose}{}
\nomscientifique{Broussonetia (Moracées)}
\newline
\relationsémantique{Cf.}{\lien{ⓔbe}{be}}
\glosecourte{banian}
\end{entrée}

\begin{entrée}{bumira}{}{ⓔbumira}
\région{PA BO}
(\domainesémantique{Corps humain})
\classe{nom}
\begin{glose}
\pfra{péritoine}
\end{glose}
\end{entrée}

\begin{entrée}{bu-mwa}{}{ⓔbu-mwa}
\région{GOs BO}
(\domainesémantique{Types de maison, architecture de la maison})
\classe{nom}
\begin{glose}
\pfra{tertre ; emplacement de la maison (trace d'une maison disparue)}
\end{glose}
\newline
\begin{exemple}
\textbf{\pnua{bwa bu-mwa}}
\pfra{à l'emplacement de la maison}
\end{exemple}
\newline
\relationsémantique{Cf.}{\lien{ⓔkêê-mwa}{kêê-mwa}}
\glosecourte{emplacement de la case, terrassement}
\end{entrée}

\begin{entrée}{bunu}{}{ⓔbunu}
\région{GOs}
(\domainesémantique{Pierre, roche})
\classe{nom}
\begin{glose}
\pfra{terre rouge ; latérite}
\end{glose}
\end{entrée}

\begin{entrée}{bunya}{}{ⓔbunya}
\région{GOs}
(\domainesémantique{Aliments, alimentation})
\classe{nom}
\begin{glose}
\pfra{bounia}
\end{glose}
\end{entrée}

\begin{entrée}{buo}{}{ⓔbuo}
\région{GOs WEM PA BO}
(\domainesémantique{Arbre})
\classe{nom}
\begin{glose}
\pfra{bois "pétrole"}
\end{glose}
\nomscientifique{Fagraea schlechteri (Loganiacées)}
\end{entrée}

\begin{entrée}{burali}{}{ⓔburali}
\région{BO}
(\domainesémantique{Sons, bruits})
\classe{v}
\begin{glose}
\pfra{détonner avec éclat ; tomber et se briser avec bruit [Corne]}
\end{glose}
\end{entrée}

\begin{entrée}{burey}{}{ⓔburey}
\région{PA}
(\domainesémantique{Ustensiles})
\classe{nom}
\begin{glose}
\pfra{bouteille}
\end{glose}
\newline
\emprunt{bouteille (FR)}
\end{entrée}

\begin{entrée}{burò}{1}{ⓔburòⓗ1}
\formephonétique{buɽɔ}
\région{GOs}
\variante{%
buròn
\formephonétique{burɔn}
\région{BO PA WEM}, 
bwòn
\formephonétique{bwɔn}
\région{BO}}
\newline
\groupe{A}
(\domainesémantique{Lumière et obscurité})
\classe{v.stat.}
\begin{glose}
\pfra{sombre ; obscur ; noir}
\end{glose}
\newline
\begin{exemple}
\textbf{\pnua{e burò-ni mee-je}}
\pfra{il est évanoui (lit.ses yeux sont dans l'obscurité)}
\end{exemple}
\newline
\groupe{B}
(\domainesémantique{Lumière et obscurité})
\classe{nom}
\begin{glose}
\pfra{obscurité}
\end{glose}
\newline
\begin{sous-entrée}{tòbwòn}{ⓔburòⓗ1ⓝtòbwòn}
\région{BO}
\begin{glose}
\pfra{soir}
\end{glose}
\end{sous-entrée}
\newline
\begin{sous-entrée}{gòbwòn}{ⓔburòⓗ1ⓝgòbwòn}
\région{BO}
\begin{glose}
\pfra{minuit}
\end{glose}
\end{sous-entrée}
\end{entrée}

\begin{entrée}{burò}{2}{ⓔburòⓗ2}
\région{GOs}
(\domainesémantique{Poissons})
\classe{nom}
\begin{glose}
\pfra{"planqueur"}
\end{glose}
\end{entrée}

\begin{entrée}{butrö}{}{ⓔbutrö}
\formephonétique{buɽω}
\région{GOs}
(\domainesémantique{Poissons})
\classe{nom}
\begin{glose}
\pfra{poisson "baleinier"}
\end{glose}
\nomscientifique{Sillago ciliata (Sillaginidés)}
\end{entrée}

\begin{entrée}{butrõ}{}{ⓔbutrõ}
\formephonétique{buɽõ}
\région{GOs}
\variante{%
burõ
\région{GO(s)}, 
buròm
\région{WEM PA BO}}
(\domainesémantique{Soins du corps})
\classe{v}
\begin{glose}
\pfra{baigner (se) ; laver (se)}
\end{glose}
\newline
\begin{exemple}
\région{GO}
\textbf{\pnua{e pe-burõ}}
\pfra{il se baigne}
\end{exemple}
\newline
\begin{exemple}
\région{PA}
\textbf{\pnua{e pe-buròm (e) Kaavo (e euphonique)}}
\pfra{Kaavo se baigne}
\end{exemple}
\newline
\begin{exemple}
\région{GO}
\textbf{\pnua{e burõ Kaavwo}}
\pfra{Kaavose baigne}
\end{exemple}
\newline
\begin{exemple}
\textbf{\pnua{pa-burõ-ni}}
\pfra{baigner qqn}
\end{exemple}
\newline
\relationsémantique{Cf.}{\lien{}{chavwoo, jamwe [GOs]}}
\glosecourte{se laver}
\end{entrée}

\begin{entrée}{buu}{1}{ⓔbuuⓗ1}
\région{GOs BO PA}
(\domainesémantique{Corps humain})
\classe{nom}
\begin{glose}
\pfra{épaule}
\end{glose}
\newline
\begin{exemple}
\région{GO}
\textbf{\pnua{buu-je}}
\pfra{son épaule}
\end{exemple}
\newline
\begin{exemple}
\région{PA}
\textbf{\pnua{buu-n}}
\pfra{son épaule}
\end{exemple}
\newline
\begin{exemple}
\textbf{\pnua{bu hii-n}}
\pfra{le haut du bras}
\end{exemple}
\end{entrée}

\begin{entrée}{buu}{2}{ⓔbuuⓗ2}
\région{PA}
(\domainesémantique{Noms des plantes})
\classe{nom}
\begin{glose}
\pfra{croton (protège les maisons et les êtres humains)}
\end{glose}
\nomscientifique{Codiaeum variegatum, Euphorbiacées}
\end{entrée}

\begin{entrée}{buubu}{}{ⓔbuubu}
(\domainesémantique{Santé, maladie})
\classe{nom}
\begin{glose}
\pfra{tache décolorée (sur peau)}
\end{glose}
\end{entrée}

\begin{entrée}{buu-dili}{}{ⓔbuu-dili}
\région{GOs WEM WE}
(\domainesémantique{Terre})
\classe{nom}
\begin{glose}
\pfra{tas de terre ; monticule de terre}
\end{glose}
\end{entrée}

\begin{entrée}{buuni}{}{ⓔbuuni}
\formephonétique{buːɳi}
\région{GOs}
\variante{%
buuni
\région{BO}}
(\domainesémantique{Cultures, techniques, boutures})
\classe{v}
\begin{glose}
\pfra{butter (les tubercules)}
\end{glose}
\end{entrée}

\begin{entrée}{buvaa}{}{ⓔbuvaa}
\région{GOs BO PA WEM}
\variante{%
bupaa
\région{GO vx}}
(\domainesémantique{Arbre})
\classe{nom}
\begin{glose}
\pfra{oranger sauvage ; faux-oranger}
\end{glose}
\newline
\note{(au pays des morts, un jeu de balle avec le fruit de l'oranger sauvage permet de distinguer l'esprit d'un vivant de celui d'un mort)}{glose}{}
\nomscientifique{Citrus macroptera (Rutacées)}
\newline
\begin{sous-entrée}{pu-buvaa}{ⓔbuvaaⓝpu-buvaa}
\begin{glose}
\pfra{pied d'oranger sauvage}
\end{glose}
\end{sous-entrée}
\end{entrée}

\begin{entrée}{buzo-kò}{}{ⓔbuzo-kò}
\région{GOs}
(\domainesémantique{Corps humain})
\classe{nom}
\begin{glose}
\pfra{mollet}
\end{glose}
\newline
\begin{exemple}
\région{PA}
\textbf{\pnua{bulo-kòò-ny}}
\pfra{mon mollet}
\end{exemple}
\end{entrée}

\newpage

\lettrine{bw}\begin{entrée}{bwa}{}{ⓔbwa}
\région{GOs BO}
\newline
\sens{1}
(\domainesémantique{Corps humain})
\classe{nom}
\begin{glose}
\pfra{tête}
\end{glose}
\newline
\begin{exemple}
\région{GO}
\textbf{\pnua{bwa-nu}}
\pfra{ma tête}
\end{exemple}
\newline
\begin{exemple}
\région{BO PA}
\textbf{\pnua{bwa-n}}
\pfra{sa tête}
\end{exemple}
\newline
\begin{exemple}
\textbf{\pnua{bwa chòvwa}}
\pfra{la tête du cheval}
\end{exemple}
\newline
\sens{2}
(\domainesémantique{Localisation})
\classe{LOC}
\begin{glose}
\pfra{dessus ; au-dessus de ; sur}
\end{glose}
\newline
\begin{exemple}
\textbf{\pnua{bwa ce}}
\pfra{dans l'arbre}
\end{exemple}
\newline
\begin{exemple}
\région{GO}
\textbf{\pnua{a bwa-wamwa !}}
\pfra{va au champ !}
\end{exemple}
\newline
\begin{exemple}
\région{GO}
\textbf{\pnua{nu a-mi na bwa wamwa}}
\pfra{je reviens du champ}
\end{exemple}
\newline
\begin{sous-entrée}{na bwa}{ⓔbwaⓢ2ⓝna bwa}
\région{BO}
\begin{glose}
\pfra{au-dessus de}
\end{glose}
\end{sous-entrée}
\newline
\étymologie{
\langue{POc}
\étymon{*bwatu}}
\end{entrée}

\begin{entrée}{bwa-}{}{ⓔbwa-}
\région{GOs PA}
(\domainesémantique{Préfixes classificateurs numériques})
\classe{CLF.NUM}
\begin{glose}
\pfra{bottes d'herbes et paquets de feuilles (pandanus, etc.)}
\end{glose}
\newline
\begin{sous-entrée}{bwa-xe, bwa-tru, bwa-ko, bwa-pa, bwa-ni, etc.}{ⓔbwa-ⓝbwa-xe, bwa-tru, bwa-ko, bwa-pa, bwa-ni, etc.}
\begin{glose}
\pfra{une botte, deux, trois, quatre, cinq, etc.}
\end{glose}
\newline
\begin{exemple}
\textbf{\pnua{bwa-xe bwalo-phoetc.}}
\pfra{une botte de pandanus, etc.}
\end{exemple}
\newline
\begin{exemple}
\textbf{\pnua{bwa-xe mae}}
\pfra{une botte de Imperata cylindrica}
\end{exemple}
\end{sous-entrée}
\end{entrée}

\begin{entrée}{bwaa}{1}{ⓔbwaaⓗ1}
\région{GOs PA BO}
\variante{%
bwaamwa
\région{GOs PA BO}}
(\domainesémantique{Interpellation})
\classe{adresse honorifique}
\begin{glose}
\pfra{ô vous !}
\end{glose}
\newline
\begin{exemple}
\textbf{\pnua{bwaa ! mo phweexu cana}}
\pfra{chers amis ! nous bavardons sans fin}
\end{exemple}
\end{entrée}

\begin{entrée}{bwaa}{2}{ⓔbwaaⓗ2}
\région{GOs}
\variante{%
bwaa
\région{PA}}
(\domainesémantique{Reptiles marins})
\classe{nom}
\begin{glose}
\pfra{serpent de mer (gris)}
\end{glose}
\newline
\relationsémantique{Cf.}{\lien{}{bwaò}}
\glosecourte{serpent "tricot rayé"}
\end{entrée}

\begin{entrée}{bwaaçu}{}{ⓔbwaaçu}
\région{GOs}
\variante{%
bwaayu
\région{WEM WE BO PA}}
(\domainesémantique{Caractéristiques et propriétés des personnes})
\classe{v.stat.}
\begin{glose}
\pfra{courageux ; travailleur}
\end{glose}
\newline
\begin{sous-entrée}{e a-bwaayu}{ⓔbwaaçuⓝe a-bwaayu}
\begin{glose}
\pfra{il est travailleur}
\end{glose}
\newline
\relationsémantique{Cf.}{\lien{}{bemãbe [GOs]}}
\glosecourte{courageux, travailleur}
\end{sous-entrée}
\end{entrée}

\begin{entrée}{bwaado}{}{ⓔbwaado}
\région{GOs PA BO}
(\domainesémantique{Oiseaux})
\classe{nom}
\begin{glose}
\pfra{martin-pêcheur}
\end{glose}
\nomscientifique{Halcyon Sanctus canacorum}
\end{entrée}

\begin{entrée}{bwaadu}{}{ⓔbwaadu}
\région{BO [BM]}
(\domainesémantique{Sentiments})
\classe{v}
\begin{glose}
\pfra{dégoûté}
\end{glose}
\newline
\begin{exemple}
\région{BO}
\textbf{\pnua{nu u bwaadu}}
\pfra{je suis dégoûté}
\end{exemple}
\end{entrée}

\begin{entrée}{bwaare}{}{ⓔbwaare}
\région{WEM WE BO}
\variante{%
bware
\région{PA BO}}
(\domainesémantique{Fonctions naturelles humaines})
\classe{v ; n}
\begin{glose}
\pfra{fatigué ; fatigue (grande)}
\end{glose}
\begin{glose}
\pfra{épuisé ; épuisement}
\end{glose}
\newline
\begin{exemple}
\textbf{\pnua{nu bwaare wa nyama}}
\pfra{je suis épuisé à cause du travail}
\end{exemple}
\newline
\begin{sous-entrée}{pa-bwaare}{ⓔbwaareⓝpa-bwaare}
\begin{glose}
\pfra{fatiguer, épuiser qqn}
\end{glose}
\newline
\relationsémantique{Cf.}{\lien{ⓔmora}{mora}}
\glosecourte{épuisé, éreinté}
\end{sous-entrée}
\end{entrée}

\begin{entrée}{bwaa-xe}{}{ⓔbwaa-xe}
\région{PA}
(\domainesémantique{Préfixes classificateurs numériques})
\classe{CLF.NUM}
\begin{glose}
\pfra{un fagot ; un paquet de feuilles, de paille}
\end{glose}
\end{entrée}

\begin{entrée}{bwabòzö}{}{ⓔbwabòzö}
\région{GOs}
(\domainesémantique{Cultures, techniques, boutures})
\classe{nom}
\begin{glose}
\pfra{mur de soutènement d'une cuvette}
\end{glose}
\newline
\note{destinée à laver l'igname "dimwa" pour la rendre comestible}{glose}{}
\end{entrée}

\begin{entrée}{bwabu}{}{ⓔbwabu}
\région{GOs PA BO}
(\domainesémantique{Localisation})
\classe{LOC}
\begin{glose}
\pfra{au-dessous ; par terre ; en bas}
\end{glose}
\newline
\begin{exemple}
\région{GO}
\textbf{\pnua{trabwa bwabu}}
\pfra{assieds-toi par terre}
\end{exemple}
\newline
\begin{exemple}
\région{PA}
\textbf{\pnua{i ã-du bwabu (Frâs)}}
\pfra{elle va en France}
\end{exemple}
\end{entrée}

\begin{entrée}{bwaçu}{}{ⓔbwaçu}
\région{GOs}
\variante{%
bwaju
\région{WEM WE}, 
bwayu
\région{PA}}
\classe{v}
\newline
\sens{1}
(\domainesémantique{Aliments, alimentation})
\begin{glose}
\pfra{manger (respectueux, en parlant d'un chef)}
\end{glose}
\newline
\sens{2}
(\domainesémantique{Coutumes, dons coutumiers})
\begin{glose}
\pfra{festoyer}
\end{glose}
\newline
\relationsémantique{Cf.}{\lien{ⓔhovwo}{hovwo}}
\glosecourte{manger (général)}
\newline
\relationsémantique{Cf.}{\lien{}{cèni, cani}}
\glosecourte{manger (féculents)}
\newline
\relationsémantique{Cf.}{\lien{ⓔhuu}{huu}}
\glosecourte{manger (nourriture carnée) + sucreries (PA]}
\newline
\relationsémantique{Cf.}{\lien{ⓔbiije}{biije}}
\glosecourte{mêcher des écorces ou du magnania}
\newline
\relationsémantique{Cf.}{\lien{}{whizi ê ; whal èm [PA] whili èm}}
\glosecourte{manger (canne à sucre)}
\newline
\relationsémantique{Cf.}{\lien{ⓔkûûńiⓗ1}{kûûńi}}
\glosecourte{manger (fruits + feuilles)}
\end{entrée}

\begin{entrée}{bwadraa}{}{ⓔbwadraa}
\formephonétique{bwaɖaː}
\région{GOs BO PA}
\variante{%
bwadaa
\région{PA}}
(\domainesémantique{Topographie})
\classe{nom}
\begin{glose}
\pfra{plaine cultivable}
\end{glose}
\begin{glose}
\pfra{plateau ; ouverture de la vallée}
\end{glose}
\end{entrée}

\begin{entrée}{bwa dre}{}{ⓔbwa dre}
\région{GOs PA BO}
\variante{%
bwa dèèn
\région{PA BO}}
(\domainesémantique{Navigation})
\classe{LOC}
\begin{glose}
\pfra{au vent ; vent debout}
\end{glose}
\end{entrée}

\begin{entrée}{bwadreo}{}{ⓔbwadreo}
\formephonétique{bwaɖeo}
\région{GOs}
\variante{%
bwadeo
\région{BO (Corne)}}
(\domainesémantique{Vêtements, parure})
\classe{nom}
\begin{glose}
\pfra{turban}
\end{glose}
\end{entrée}

\begin{entrée}{bwaè}{}{ⓔbwaè}
\région{GOs}
\variante{%
bwaalek
\région{PA}}
(\domainesémantique{Oiseaux})
\classe{nom}
\begin{glose}
\pfra{buse de mer}
\end{glose}
\nomscientifique{Pandion haliaetus melvillensis}
\end{entrée}

\begin{entrée}{bwagili}{}{ⓔbwagili}
\région{GOs PA WEM}
\variante{%
bwagil
\région{BO}}
(\domainesémantique{Corps humain})
\classe{nom}
\begin{glose}
\pfra{genou}
\end{glose}
\newline
\begin{sous-entrée}{du-bwagili}{ⓔbwagiliⓝdu-bwagili}
\région{GO}
\begin{glose}
\pfra{os du genou}
\end{glose}
\end{sous-entrée}
\newline
\begin{sous-entrée}{thi-bwagil}{ⓔbwagiliⓝthi-bwagil}
\région{BO}
\begin{glose}
\pfra{être à genoux}
\end{glose}
\newline
\begin{exemple}
\région{PA}
\textbf{\pnua{bwagili-m, bwagilii-m}}
\pfra{ton genou}
\end{exemple}
\newline
\begin{exemple}
\région{GO}
\textbf{\pnua{bwagili-je}}
\pfra{son genou}
\end{exemple}
\end{sous-entrée}
\newline
\étymologie{
\étymon{*bwaqu 'genou' (reflet de PSO (proto-S-Oceanic), *bwa-turu 'tête de genou' comme au S Vanuatu, venant de *tunu POc 'genou'}}
\end{entrée}

\begin{entrée}{bwagiloo}{}{ⓔbwagiloo}
\région{GOs PA}
\classe{v}
\newline
\sens{1}
(\domainesémantique{Mouvements ou actions faits avec le corps, les bras, les mains, les pieds})
\begin{glose}
\pfra{courber (se)}
\end{glose}
\begin{glose}
\pfra{baisser la tête}
\end{glose}
\newline
\sens{2}
(\domainesémantique{Religion, représentations religieuses})
\begin{glose}
\pfra{prosterner (se)}
\end{glose}
\begin{glose}
\pfra{incliner la tête}
\end{glose}
\newline
\relationsémantique{Cf.}{\lien{}{kiluu [GOs BO]}}
\glosecourte{courber (se)}
\end{entrée}

\begin{entrée}{bwa gu-hi}{}{ⓔbwa gu-hi}
\région{PA WE}
(\domainesémantique{Directions})
\classe{LOC}
\begin{glose}
\pfra{droite (à)}
\end{glose}
\end{entrée}

\begin{entrée}{bwaida}{}{ⓔbwaida}
\région{GOs}
(\domainesémantique{Mollusques})
\classe{nom}
\begin{glose}
\pfra{palourde}
\end{glose}
\end{entrée}

\begin{entrée}{bwa-kaça}{}{ⓔbwa-kaça}
\formephonétique{bwaɣacʰabwaɣaʒa}
\région{GOs}
\variante{%
bwa-xaça
\région{GO(s)}}
(\domainesémantique{Noms locatifs})
\classe{nom}
\begin{glose}
\pfra{dessus, dos (de la main, du pied)}
\end{glose}
\newline
\begin{exemple}
\textbf{\pnua{bwa-xaça hii-je}}
\pfra{le dessus de sa main}
\end{exemple}
\newline
\begin{exemple}
\textbf{\pnua{bwa-xaça kò-je}}
\pfra{le dessus de son pied}
\end{exemple}
\newline
\begin{exemple}
\textbf{\pnua{bwa-xaça za}}
\pfra{le dos de l'assiette}
\end{exemple}
\newline
\relationsémantique{Ant.}{\lien{ⓔkaça}{kaça}}
\glosecourte{arrière}
\end{entrée}

\begin{entrée}{bwa-kaça hi}{}{ⓔbwa-kaça hi}
\formephonétique{bwaɣacʰabwaɣaʒa}
\région{GOs}
\variante{%
bwa-xaça
\région{GO(s)}}
(\domainesémantique{Corps humain})
\classe{nom}
\begin{glose}
\pfra{dessus /dos (de la main)}
\end{glose}
\newline
\begin{exemple}
\textbf{\pnua{bwa-xaça hii-je}}
\pfra{le dessus de sa main}
\end{exemple}
\newline
\begin{exemple}
\textbf{\pnua{bwa-xaça kò-je}}
\pfra{le dessus de son pied}
\end{exemple}
\end{entrée}

\begin{entrée}{bwa-kazi-}{}{ⓔbwa-kazi-}
\région{GO}
\variante{%
kaji-n
\région{BO (Corne)}}
(\domainesémantique{Corps humain})
\classe{nom}
\begin{glose}
\pfra{pubis ; partie antérieure des os de la hanche}
\end{glose}
\newline
\begin{exemple}
\textbf{\pnua{kaji-n}}
\pfra{son pubis (Dubois)}
\end{exemple}
\end{entrée}

\begin{entrée}{bwa-kitra-hi}{}{ⓔbwa-kitra-hi}
\formephonétique{bwakiɽa-hi, bwaɣiɽa-hi}
\région{GOs}
\variante{%
bwagira-hi
\région{GO(s) BO}}
(\domainesémantique{Corps humain})
\classe{nom}
\begin{glose}
\pfra{coude}
\end{glose}
\newline
\begin{exemple}
\région{BO}
\textbf{\pnua{bwagira hi-n}}
\pfra{son coude}
\end{exemple}
\newline
\relationsémantique{Cf.}{\lien{}{böji hi}}
\glosecourte{poignet (lit. articulation de la main)}
\end{entrée}

\begin{entrée}{bwa-kitra-me}{}{ⓔbwa-kitra-me}
\formephonétique{bwaɣiɽame}
\région{GOs}
\variante{%
bwa-kira-me, bwa-gira-mè
\région{GO(s) PA BO}, 
bwagila-me
\région{BO}}
(\domainesémantique{Corps humain})
\classe{nom}
\begin{glose}
\pfra{arcade sourcilière}
\end{glose}
\newline
\begin{exemple}
\région{GO}
\textbf{\pnua{bwakira mee-nu}}
\pfra{mon arcade sourcilière}
\end{exemple}
\newline
\begin{exemple}
\région{BO}
\textbf{\pnua{bwagila me-ny}}
\pfra{mon arcade sourcilière}
\end{exemple}
\newline
\begin{sous-entrée}{pu-bwa-kira-me}{ⓔbwa-kitra-meⓝpu-bwa-kira-me}
\begin{glose}
\pfra{sourcils}
\end{glose}
\newline
\relationsémantique{Cf.}{\lien{ⓔmeⓗ1}{me}}
\glosecourte{oeil}
\end{sous-entrée}
\end{entrée}

\begin{entrée}{bwala}{}{ⓔbwala}
\région{BO PA}
(\domainesémantique{Types de champs})
\classe{nom}
\begin{glose}
\pfra{tarodière irriguée en terrasse}
\end{glose}
\newline
\note{(plane, les pieds de taros sont dans l'eau, et non plantés sur un billon 'aru', comme c'est le cas dans la tarodière sèche'penu' ; la tarodière 'bwala' est de taille supérieure à la tarodière 'penu')}{glose}{}
\newline
\begin{sous-entrée}{kêê-bwala}{ⓔbwalaⓝkêê-bwala}
\begin{glose}
\pfra{tarodière irriguée}
\end{glose}
\end{sous-entrée}
\end{entrée}

\begin{entrée}{bwalò}{}{ⓔbwalò}
\région{GOs PA BO}
\variante{%
pwalò
\région{GO(s)}}
\newline
\groupe{A}
\classe{nom}
\newline
\sens{1}
(\domainesémantique{Configuration des objets})
\begin{glose}
\pfra{fagot (bois, canne à sucre)}
\end{glose}
\begin{glose}
\pfra{tas (feuilles, coco)}
\end{glose}
\newline
\begin{sous-entrée}{bwalò-ce}{ⓔbwalòⓢ1ⓝbwalò-ce}
\région{GO PA}
\begin{glose}
\pfra{fagot de bois}
\end{glose}
\end{sous-entrée}
\newline
\begin{sous-entrée}{bwalòò-pa}{ⓔbwalòⓢ1ⓝbwalòò-pa}
\région{GO}
\begin{glose}
\pfra{tas de pierres}
\end{glose}
\end{sous-entrée}
\newline
\begin{sous-entrée}{bwalò-ê}{ⓔbwalòⓢ1ⓝbwalò-ê}
\région{GO}
\begin{glose}
\pfra{fagot de canne à sucre}
\end{glose}
\end{sous-entrée}
\newline
\begin{sous-entrée}{bwalò-pò-ce}{ⓔbwalòⓢ1ⓝbwalò-pò-ce}
\région{PA}
\begin{glose}
\pfra{tas de fruits}
\end{glose}
\end{sous-entrée}
\newline
\begin{sous-entrée}{bwalò-mae}{ⓔbwalòⓢ1ⓝbwalò-mae}
\région{PA}
\begin{glose}
\pfra{fagot de paille}
\end{glose}
\end{sous-entrée}
\newline
\groupe{B}
\newline
\sens{2}
(\domainesémantique{Préfixes classificateurs numériques})
\begin{glose}
\pfra{tas de}
\end{glose}
\newline
\begin{exemple}
\région{GO PA}
\textbf{\pnua{bwalò-xe, bwalò-tru, bwalò-kò}}
\pfra{un, deux, trois tas}
\end{exemple}
\end{entrée}

\begin{entrée}{bwa mè}{}{ⓔbwa mè}
\région{GOs BO}
(\domainesémantique{Localisation})
\classe{LOC}
\begin{glose}
\pfra{face à ; présence (en) de}
\end{glose}
\newline
\begin{exemple}
\région{GO}
\textbf{\pnua{nu kò bwa mèè-we}}
\pfra{je suis face à vous}
\end{exemple}
\end{entrée}

\begin{entrée}{bwa-mõ}{}{ⓔbwa-mõ}
\région{GOs}
\variante{%
bwa-mol
\région{BO}}
(\domainesémantique{Navigation
, Topographie})
\classe{LOC}
\begin{glose}
\pfra{à terre (lit. sec)}
\end{glose}
\newline
\begin{exemple}
\textbf{\pnua{li co-da bwa-mõ}}
\pfra{ils remontent à terre}
\end{exemple}
\end{entrée}

\begin{entrée}{bwa mhõ}{}{ⓔbwa mhõ}
\formephonétique{bwa mʰɔ̃}
\région{GOs WE}
\variante{%
mò
\formephonétique{mɔ̃}
\région{BO}}
(\domainesémantique{Directions})
\classe{n ; v.stat.}
\begin{glose}
\pfra{gauche (à)}
\end{glose}
\newline
\begin{sous-entrée}{bwa mhõ}{ⓔbwa mhõⓝbwa mhõ}
\begin{glose}
\pfra{à gauche}
\end{glose}
\end{sous-entrée}
\newline
\begin{sous-entrée}{alabo bwa mhõ}{ⓔbwa mhõⓝalabo bwa mhõ}
\région{GO WEM}
\begin{glose}
\pfra{côté gauche}
\end{glose}
\newline
\begin{exemple}
\région{PAWEM}
\textbf{\pnua{bwa gu-(h)i}}
\pfra{à droite}
\end{exemple}
\newline
\relationsémantique{Ant.}{\lien{}{bwa mhwã [GO]}}
\glosecourte{à droite}
\end{sous-entrée}
\newline
\étymologie{
\langue{POc}
\étymon{*mauRi}}
\end{entrée}

\begin{entrée}{bwa-mwa}{}{ⓔbwa-mwa}
\région{GOs}
\variante{%
bwa-mwa
\région{PA}}
(\domainesémantique{Types de maison, architecture de la maison})
\classe{nom}
\begin{glose}
\pfra{toit}
\end{glose}
\end{entrée}

\begin{entrée}{bwa mhwã}{}{ⓔbwa mhwã}
\formephonétique{bwa mʰwɛ̃}
\région{GOs}
(\domainesémantique{Directions})
\classe{DIR}
\begin{glose}
\pfra{droite (à)}
\end{glose}
\newline
\begin{sous-entrée}{hii-je bwa mhwã}{ⓔbwa mhwãⓝhii-je bwa mhwã}
\région{GO}
\begin{glose}
\pfra{à droite}
\end{glose}
\end{sous-entrée}
\newline
\begin{sous-entrée}{bwa gu-hi}{ⓔbwa mhwãⓝbwa gu-hi}
\région{PA}
\begin{glose}
\pfra{à droite}
\end{glose}
\end{sous-entrée}
\newline
\begin{sous-entrée}{bwa gu-i}{ⓔbwa mhwãⓝbwa gu-i}
\région{WEM}
\begin{glose}
\pfra{à droite}
\end{glose}
\newline
\relationsémantique{Ant.}{\lien{ⓔbwa mhõ}{bwa mhõ}}
\glosecourte{à gauche}
\end{sous-entrée}
\end{entrée}

\begin{entrée}{bwana}{}{ⓔbwana}
\région{GO}
(\domainesémantique{Ignames})
\classe{nom}
\begin{glose}
\pfra{côté mâle du massif d'ignames [Haudricourt]}
\end{glose}
\newline
\note{mot inconnu des locuteurs actuels}{général}{}
\end{entrée}

\begin{entrée}{bwanamwa}{}{ⓔbwanamwa}
\formephonétique{'bwaɳamwa}
\région{GOs}
\variante{%
bwaamwa
\région{GO(s) PA WEM}, 
bwaa
\région{PA}}
(\domainesémantique{Interjection})
\classe{INTJ (pitié, affection)}
\begin{glose}
\pfra{hélas ; pauvre ! ; cher !}
\end{glose}
\newline
\begin{exemple}
\textbf{\pnua{bwaa dony !}}
\pfra{pauvre buse !}
\end{exemple}
\newline
\relationsémantique{Cf.}{\lien{}{gaanamwa (jugé moins correct)}}
\end{entrée}

\begin{entrée}{bwane}{}{ⓔbwane}
\région{GOs}
\variante{%
bwea
\région{WE PA}, 
bwani
\région{BO}}
(\domainesémantique{Objets et meubles de la maison})
\classe{nom}
\begin{glose}
\pfra{oreiller ; appuie-tête}
\end{glose}
\newline
\begin{exemple}
\région{BO}
\textbf{\pnua{bwani-ny}}
\pfra{mon oreiller}
\end{exemple}
\end{entrée}

\begin{entrée}{bwange}{1}{ⓔbwangeⓗ1}
\région{GOs}
(\domainesémantique{Actions liées aux éléments (liquide, fumée)})
\classe{v}
\begin{glose}
\pfra{dévier (eau)}
\end{glose}
\end{entrée}

\begin{entrée}{bwange}{2}{ⓔbwangeⓗ2}
\région{GOs}
(\domainesémantique{Verbes de déplacement et moyens de déplacement})
\classe{v}
\begin{glose}
\pfra{tourner ; revenir; retourner (s'en)}
\end{glose}
\newline
\begin{exemple}
\textbf{\pnua{bwange-cö !}}
\pfra{retourne-toi !}
\end{exemple}
\newline
\relationsémantique{Cf.}{\lien{}{pwaa [PA]}}
\end{entrée}

\begin{entrée}{bwaô}{}{ⓔbwaô}
\région{GOs}
(\domainesémantique{Poissons})
\classe{nom}
\begin{glose}
\pfra{carangue noire (de très grosse taille)}
\end{glose}
\nomscientifique{Caranx lugubris ou Caranx ignobilis (Carangidées)}
\newline
\relationsémantique{Cf.}{\lien{}{kûxû (petit), dròò-kibö (moyen), putrakou (plus gros), bwaô}}
\glosecourte{les 4 noms de la même carangue à des tailles différentes}
\end{entrée}

\begin{entrée}{bwaole}{}{ⓔbwaole}
\région{GOs PA BO}
(\domainesémantique{Oiseaux})
\classe{nom}
\begin{glose}
\pfra{aigle pêcheur ; balbuzard pêcheur ; aigle siffleur}
\end{glose}
\nomscientifique{Haliastur sphenurus}
\end{entrée}

\begin{entrée}{bwaòle}{}{ⓔbwaòle}
\formephonétique{bwaɔle}
\région{PA BO WEM}
\classe{v ; n}
\newline
\sens{1}
(\domainesémantique{Verbes de mouvement})
\begin{glose}
\pfra{rouler ; tourner une roue}
\end{glose}
\begin{glose}
\pfra{tourner une roue}
\end{glose}
\newline
\sens{2}
(\domainesémantique{Instruments})
\begin{glose}
\pfra{roue}
\end{glose}
\newline
\note{bwaòle}{grammaire}{faire rouler qqch}
\end{entrée}

\begin{entrée}{bwarabo}{}{ⓔbwarabo}
\région{BO}
(\domainesémantique{Description des objets, formes, consistance, taille})
\classe{v.stat.}
\begin{glose}
\pfra{rond [BM]}
\end{glose}
\end{entrée}

\begin{entrée}{bwarao}{}{ⓔbwarao}
\formephonétique{bwaɽao}
\région{GOs}
\variante{%
bwaòl
\région{WEM}}
(\domainesémantique{Verbes de mouvement})
\classe{v}
\begin{glose}
\pfra{rouler}
\end{glose}
\begin{glose}
\pfra{faire des galipettes (sur le côté)}
\end{glose}
\end{entrée}

\begin{entrée}{bwarele}{}{ⓔbwarele}
\formephonétique{bwaɽele}
\région{GOs PA WEM BO}
\variante{%
bwatrele
\formephonétique{bwaɽele}
\région{GO(s)}}
(\domainesémantique{Oiseaux})
\classe{nom}
\begin{glose}
\pfra{"collier blanc" ; pigeon à gorge blanche}
\end{glose}
\nomscientifique{Columba vitiensis hypoenochroa}
\end{entrée}

\begin{entrée}{bwaroe}{}{ⓔbwaroe}
\formephonétique{bwaɽoe}
\région{GOs BO}
(\domainesémantique{Portage})
\classe{v}
\begin{glose}
\pfra{porter un enfant dans les bras (contre la poitrine)}
\end{glose}
\end{entrée}

\begin{entrée}{bwatra}{}{ⓔbwatra}
\formephonétique{bwaɽa}
\région{GOs}
\variante{%
bwara
\région{GO(s)}}
(\domainesémantique{Poissons})
\classe{nom}
\begin{glose}
\pfra{'crocro'}
\end{glose}
\nomscientifique{Pomadasys argenteus (Haemulidae)}
\end{entrée}

\begin{entrée}{bwatratra}{}{ⓔbwatratra}
\formephonétique{bwaɽaɽa}
\région{GOs}
(\domainesémantique{Instruments})
\classe{nom}
\begin{glose}
\pfra{maillet à poisson}
\end{glose}
\end{entrée}

\begin{entrée}{bwatrû}{}{ⓔbwatrû}
\formephonétique{bwaʈũ bwaɽũ}
\région{GOs}
\variante{%
bwarû
\région{PA}, 
bwarong, bwarô
\région{BO}}
(\domainesémantique{Oiseaux})
\classe{nom}
\begin{glose}
\pfra{nid (oiseau)}
\end{glose}
\newline
\begin{exemple}
\textbf{\pnua{bwatrûû meni}}
\pfra{nid d'oiseau}
\end{exemple}
\newline
\begin{exemple}
\région{BO}
\textbf{\pnua{bwarô mèèni (ou) bwaarong mèni}}
\pfra{nid d'oiseau}
\end{exemple}
\end{entrée}

\begin{entrée}{bwaû-bwara}{}{ⓔbwaû-bwara}
\région{GOs}
(\domainesémantique{Poissons})
\classe{nom}
\begin{glose}
\pfra{castex}
\end{glose}
\nomscientifique{Diagramma pictum (Hémulidés)}
\newline
\relationsémantique{Cf.}{\lien{ⓔbwaû-wããdri}{bwaû-wããdri}}
\glosecourte{castex}
\end{entrée}

\begin{entrée}{bwaû-wããdri}{}{ⓔbwaû-wããdri}
\région{GOs}
(\domainesémantique{Poissons})
\classe{nom}
\begin{glose}
\pfra{castex}
\end{glose}
\nomscientifique{Plectorhinchus gibbosus (Hémulidés)}
\newline
\relationsémantique{Cf.}{\lien{ⓔbwaû-bwara}{bwaû-bwara}}
\glosecourte{castex}
\end{entrée}

\begin{entrée}{bwavwa}{}{ⓔbwavwa}
\région{GOs BO}
\variante{%
bwapa
\région{GO(s)}}
(\domainesémantique{Mollusques})
\classe{nom}
\begin{glose}
\pfra{bénitier géant (coquille) (gastéropode)}
\end{glose}
\end{entrée}

\begin{entrée}{bwavwaida}{}{ⓔbwavwaida}
\région{GOs}
\variante{%
bwaivwada
\région{GO(s) BO}, 
bwapaida
\région{vx}}
(\domainesémantique{Oiseaux})
\classe{nom}
\begin{glose}
\pfra{oiseau de proie ; aigle}
\end{glose}
\end{entrée}

\begin{entrée}{bwavwòlo}{}{ⓔbwavwòlo}
\région{GOs}
\variante{%
bweevòlo
\région{PA BO}}
(\domainesémantique{Cultures, techniques, boutures})
\classe{nom}
\begin{glose}
\pfra{barrage sur une rivière où les femmes lavent le 'dimwa' (Charles)}
\end{glose}
\begin{glose}
\pfra{barrage (pour l'irrigation)}
\end{glose}
\end{entrée}

\begin{entrée}{bwavwu-we}{}{ⓔbwavwu-we}
\région{GOs BO PA}
\région{GOs}
\variante{%
bwevwu-we
, 
pwe-we
\région{PA}}
(\domainesémantique{Cultures, techniques, boutures})
\classe{nom}
\begin{glose}
\pfra{source (Charles)}
\end{glose}
\begin{glose}
\pfra{barrage ; vanne de canal de tarodière (Dubois)}
\end{glose}
\end{entrée}

\begin{entrée}{bwawe}{}{ⓔbwawe}
\région{GOs}
(\domainesémantique{Mouvements ou actions faits avec le corps, les bras, les mains, les pieds})
\classe{v}
\begin{glose}
\pfra{démêler}
\end{glose}
\newline
\begin{exemple}
\textbf{\pnua{e bwawe-zoo-ni}}
\pfra{il les démêle (tuyau, corde)}
\end{exemple}
\end{entrée}

\begin{entrée}{bwaxala}{}{ⓔbwaxala}
\région{BO}
(\domainesémantique{Navigation})
\classe{nom}
\begin{glose}
\pfra{pirogue (Corne)}
\end{glose}
\newline
\note{non vérifié}{général}{}
\end{entrée}

\begin{entrée}{bwaxe}{}{ⓔbwaxe}
\région{PA BO}
(\domainesémantique{Fonctions naturelles humaines})
\classe{v}
\begin{glose}
\pfra{fatigué; faible}
\end{glose}
\newline
\begin{exemple}
\textbf{\pnua{nu po bwaxe}}
\pfra{je suis un peu fatigué}
\end{exemple}
\end{entrée}

\begin{entrée}{bwaxeni}{}{ⓔbwaxeni}
\région{GOs WEMBO}
(\domainesémantique{Types de maison, architecture de la maison})
\classe{nom}
\begin{glose}
\pfra{tertre}
\end{glose}
\newline
\begin{exemple}
\textbf{\pnua{bwaxeni mwa}}
\pfra{tertre de la maison}
\end{exemple}
\end{entrée}

\begin{entrée}{bwaxixi}{}{ⓔbwaxixi}
\région{GOs}
\variante{%
bwaxii
}
(\domainesémantique{Fonctions naturelles humaines})
\classe{v}
\begin{glose}
\pfra{regarder (se) (dans un miroir)}
\end{glose}
\newline
\relationsémantique{Cf.}{\lien{}{zido [GOs], zhido [GA]}}
\glosecourte{se mirer}
\end{entrée}

\begin{entrée}{bwaxuli}{}{ⓔbwaxuli}
\région{WEM WE}
(\domainesémantique{Instruments})
\classe{nom}
\begin{glose}
\pfra{chaîne}
\end{glose}
\end{entrée}

\begin{entrée}{bwe}{1}{ⓔbweⓗ1}
\région{BO}
\variante{%
bwee
\région{BO}}
(\domainesémantique{Parties de plantes})
\classe{nom}
\begin{glose}
\pfra{tronc ; souche ; bout}
\end{glose}
\newline
\begin{exemple}
\région{BO}
\textbf{\pnua{bwe laloe}}
\pfra{tige d'aloès}
\end{exemple}
\newline
\begin{exemple}
\région{BO}
\textbf{\pnua{bwe ce}}
\pfra{bâton, bout de bois}
\end{exemple}
\newline
\relationsémantique{Cf.}{\lien{}{bwevwu}}
\glosecourte{tronc; souche; bout de}
\end{entrée}

\begin{entrée}{bwe-}{}{ⓔbwe-}
\région{GOs BO PA}
\classe{n (composition)}
\newline
\sens{1}
(\domainesémantique{Corps humain})
\begin{glose}
\pfra{tête de ;}
\end{glose}
\begin{glose}
\pfra{coiffure de}
\end{glose}
\newline
\begin{sous-entrée}{bwè-kui}{ⓔbwe-ⓢ1ⓝbwè-kui}
\begin{glose}
\pfra{tête d'igname}
\end{glose}
\end{sous-entrée}
\newline
\begin{sous-entrée}{bwè-mwa}{ⓔbwe-ⓢ1ⓝbwè-mwa}
\begin{glose}
\pfra{dernière rangée de paille sur le toit (forme un bourrelet qui ferme le faîtage)}
\end{glose}
\end{sous-entrée}
\newline
\begin{sous-entrée}{bwè-po}{ⓔbwe-ⓢ1ⓝbwè-po}
\région{BO}
\begin{glose}
\pfra{casse-tête en bec d'oiseau}
\end{glose}
\end{sous-entrée}
\newline
\sens{2}
(\domainesémantique{Topographie})
\begin{glose}
\pfra{sommet de ; dessus de}
\end{glose}
\newline
\begin{sous-entrée}{bwe-hogo}{ⓔbwe-ⓢ2ⓝbwe-hogo}
\région{GO BO}
\begin{glose}
\pfra{sommet de la montagne}
\end{glose}
\end{sous-entrée}
\end{entrée}

\begin{entrée}{bwèdò}{}{ⓔbwèdò}
\formephonétique{mbwɛndɔ}
\région{GOs PA}
(\domainesémantique{Parties du corps humain : doigts, orteil})
\classe{nom}
\begin{glose}
\pfra{doigt}
\end{glose}
\newline
\begin{exemple}
\région{GO}
\textbf{\pnua{bwèdò-hii-je}}
\pfra{son doigt}
\end{exemple}
\newline
\begin{exemple}
\textbf{\pnua{bwèdò kòò-n}}
\pfra{orteils}
\end{exemple}
\newline
\begin{sous-entrée}{êmwèn}{ⓔbwèdòⓝêmwèn}
\begin{glose}
\pfra{index}
\end{glose}
\end{sous-entrée}
\newline
\begin{sous-entrée}{bwèdò-hi-n êmwèn}{ⓔbwèdòⓝbwèdò-hi-n êmwèn}
\begin{glose}
\pfra{son index}
\end{glose}
\end{sous-entrée}
\newline
\begin{sous-entrée}{bwèdò-hi-n tòòmwa}{ⓔbwèdòⓝbwèdò-hi-n tòòmwa}
\begin{glose}
\pfra{son pouce}
\end{glose}
\end{sous-entrée}
\newline
\begin{sous-entrée}{bwèdò-kòò-n tòòmwa}{ⓔbwèdòⓝbwèdò-kòò-n tòòmwa}
\begin{glose}
\pfra{son gros orteil}
\end{glose}
\end{sous-entrée}
\end{entrée}

\begin{entrée}{bwèdo-hi thoomwa}{1}{ⓔbwèdo-hi thoomwaⓗ1}
\région{GOs PA}
\variante{%
bwèdo-hi thooma
\région{BO}}
(\domainesémantique{Parties du corps humain : doigts, orteil})
\classe{nom}
\begin{glose}
\pfra{pouce}
\end{glose}
\end{entrée}

\begin{entrée}{bwèdò-kò}{}{ⓔbwèdò-kò}
\région{GOs PA BO}
\variante{%
bwèdò-xò
\région{GO(s) PA BO}}
(\domainesémantique{Parties du corps humain : doigts, orteil})
\classe{nom}
\begin{glose}
\pfra{orteil}
\end{glose}
\end{entrée}

\begin{entrée}{bwèèdrò}{2}{ⓔbwèèdròⓗ2}
\formephonétique{mbwɛːnɖɔ}
\région{GOs}
\variante{%
bwèèdò
\région{PA BO}, 
bwèdòl
\région{BO}}
(\domainesémantique{Corps humain})
\classe{nom}
\begin{glose}
\pfra{front}
\end{glose}
\newline
\begin{exemple}
\région{GO}
\textbf{\pnua{bwèèdrò-nu}}
\pfra{mon front}
\end{exemple}
\newline
\begin{exemple}
\région{PA}
\textbf{\pnua{bwèdòò-n}}
\pfra{son front}
\end{exemple}
\end{entrée}

\begin{entrée}{bwèèdrö}{1}{ⓔbwèèdröⓗ1}
\formephonétique{mbwɛːnɖo}
\région{GOs}
\variante{%
bwèèdo
\région{PA BO}}
(\domainesémantique{Terre})
\classe{nom}
\begin{glose}
\pfra{terre ; pays ; sol ; Terre}
\end{glose}
\newline
\begin{exemple}
\textbf{\pnua{na bwèèdro}}
\pfra{par terre, au sol}
\end{exemple}
\end{entrée}

\begin{entrée}{bweena}{}{ⓔbweena}
\région{GOs PA BO}
(\domainesémantique{Reptiles})
\classe{nom}
\begin{glose}
\pfra{lézard}
\end{glose}
\end{entrée}

\begin{entrée}{bwèèra}{}{ⓔbwèèra}
\formephonétique{bwèːɽa}
\région{GOs PA}
\variante{%
bwèèrao
\région{BO (Corne)}}
(\domainesémantique{Feu : objets et actions liés au feu})
\classe{nom}
\begin{glose}
\pfra{pierres (utilisées comme support à marmite, souvent au nombre de trois)}
\end{glose}
\begin{glose}
\pfra{chenêts ; foyer ; rails du feu}
\end{glose}
\newline
\begin{sous-entrée}{bwèèra pa}{ⓔbwèèraⓝbwèèra pa}
\begin{glose}
\pfra{les pierres qui servent de support aux marmites}
\end{glose}
\end{sous-entrée}
\end{entrée}

\begin{entrée}{bweeravac}{}{ⓔbweeravac}
\région{PA BO}
(\domainesémantique{Vents})
\classe{nom}
\begin{glose}
\pfra{vent soufflant du sud au nord}
\end{glose}
\end{entrée}

\begin{entrée}{bweetroe}{}{ⓔbweetroe}
\formephonétique{bweːɽoe}
\région{GOs}
\variante{%
bweeroe
\région{GO(s)}}
(\domainesémantique{Poissons})
\classe{nom}
\begin{glose}
\pfra{loche (en général)}
\end{glose}
\nomscientifique{Epinephelus sp. (Serranidae)}
\newline
\relationsémantique{Cf.}{\lien{ⓔpoxa-zaaja}{poxa-zaaja}}
\glosecourte{loche (de grande taille)}
\end{entrée}

\begin{entrée}{bwèèvaça kò}{}{ⓔbwèèvaça kò}
\formephonétique{bweːvadʒa}
\région{GOs}
\variante{%
bwèèvao kò-ã
\région{PA}, 
bwèèva kò
\région{BO WEM}}
(\domainesémantique{Corps humain})
\classe{nom}
\begin{glose}
\pfra{talon}
\end{glose}
\newline
\begin{exemple}
\région{BO}
\textbf{\pnua{bwèèva kòò-n}}
\pfra{son talon}
\end{exemple}
\end{entrée}

\begin{entrée}{bweevwu}{}{ⓔbweevwu}
\région{GOs PA BO}
\classe{nom}
\newline
\sens{1}
(\domainesémantique{Parties de plantes})
\begin{glose}
\pfra{tronc ; souche}
\end{glose}
\newline
\begin{sous-entrée}{bwewu ci-pai}{ⓔbweevwuⓢ1ⓝbwewu ci-pai}
\begin{glose}
\pfra{pied d'arbre à pain}
\end{glose}
\end{sous-entrée}
\newline
\begin{sous-entrée}{bweewu nu}{ⓔbweevwuⓢ1ⓝbweewu nu}
\begin{glose}
\pfra{au pied du / sous le cocotier}
\end{glose}
\end{sous-entrée}
\newline
\begin{sous-entrée}{bwewu go}{ⓔbweevwuⓢ1ⓝbwewu go}
\région{BO}
\begin{glose}
\pfra{au pied du bambou [BM]}
\end{glose}
\end{sous-entrée}
\newline
\sens{2}
(\domainesémantique{Découpage du temps})
\begin{glose}
\pfra{origine}
\end{glose}
\newline
\begin{exemple}
\textbf{\pnua{bwewuu-n}}
\pfra{son origine}
\end{exemple}
\end{entrée}

\begin{entrée}{bwèè-xò}{}{ⓔbwèè-xò}
\région{GOs PA WEM BO}
(\domainesémantique{Verbes de déplacement et moyens de déplacement})
\classe{nom}
\begin{glose}
\pfra{traces ; empreintes (homme ou animal)}
\end{glose}
\newline
\begin{exemple}
\textbf{\pnua{bwèxòò-n}}
\pfra{ses traces}
\end{exemple}
\newline
\begin{exemple}
\textbf{\pnua{bwèxò lòtò}}
\pfra{les traces de voiture}
\end{exemple}
\end{entrée}

\begin{entrée}{bweeye}{}{ⓔbweeye}
\région{PA}
(\domainesémantique{Jours})
\classe{nom}
\begin{glose}
\pfra{saison chaude et sèche}
\end{glose}
\end{entrée}

\begin{entrée}{bwe-hogo}{}{ⓔbwe-hogo}
\région{GOs}
(\domainesémantique{Topographie})
\classe{nom}
\begin{glose}
\pfra{sommet de la montagne (lit. tête de la montagne) ; crête de la montagne}
\end{glose}
\end{entrée}

\begin{entrée}{bwehulo}{}{ⓔbwehulo}
\région{GOs}
(\domainesémantique{Corps humain})
\classe{v.stat.}
\begin{glose}
\pfra{amputé (d'une partie du corps)}
\end{glose}
\newline
\begin{exemple}
\région{GO}
\textbf{\pnua{bwehulo hii-je}}
\pfra{son bras est (partiellement) amputé}
\end{exemple}
\newline
\begin{exemple}
\région{GO}
\textbf{\pnua{bwehulo kòò-je}}
\pfra{sa jambe est amputé}
\end{exemple}
\newline
\begin{exemple}
\région{GO}
\textbf{\pnua{bwehulo bwedòò hii-je}}
\pfra{il lui manque un doigt}
\end{exemple}
\newline
\relationsémantique{Cf.}{\lien{ⓔpaxu}{paxu}}
\glosecourte{manquer}
\end{entrée}

\begin{entrée}{bwe-kui}{}{ⓔbwe-kui}
\région{GOs PA}
(\domainesémantique{Ignames
, Cultures, techniques, boutures})
\classe{nom}
\begin{glose}
\pfra{tête de l'igname (qui est replantée)}
\end{glose}
\begin{glose}
\pfra{bouture d'igname (à partir de l'extrémité inférieure de l'igname)}
\end{glose}
\newline
\relationsémantique{Ant.}{\lien{}{tho-kui}}
\glosecourte{bout inférieur de l'igname}
\end{entrée}

\begin{entrée}{bwè-mwa}{}{ⓔbwè-mwa}
\région{GOs PA BO}
(\domainesémantique{Types de maison, architecture de la maison})
\classe{nom}
\begin{glose}
\pfra{paille (dernière rangée de paille sur le toit, forme un bourrelet qui ferme le faîtage)}
\end{glose}
\newline
\relationsémantique{Cf.}{\lien{ⓔphu}{phu}}
\glosecourte{paille (première rangée de paille sur le toit)}
\end{entrée}

\begin{entrée}{bwe-no}{}{ⓔbwe-no}
\région{GOs}
(\domainesémantique{Santé, maladie})
\classe{v}
\begin{glose}
\pfra{torticolis (avoir le)}
\end{glose}
\newline
\begin{exemple}
\textbf{\pnua{e toonu bwe-no}}
\pfra{j'ai le torticolis}
\end{exemple}
\end{entrée}

\begin{entrée}{bweo}{}{ⓔbweo}
\région{PA BO}
(\domainesémantique{Vents})
\classe{nom}
\begin{glose}
\pfra{vent d'ouest}
\end{glose}
\end{entrée}

\begin{entrée}{bwe-pai}{}{ⓔbwe-pai}
\région{GOs PA}
\variante{%
bwe-vai
\région{GO(s) PA}}
(\domainesémantique{Fonctions intellectuelles})
\classe{nom}
\begin{glose}
\pfra{résultat ; conséquence (bénéfique) (lit. tête du tubercule)}
\end{glose}
\end{entrée}

\begin{entrée}{bwè-po}{}{ⓔbwè-po}
\région{GO}
(\domainesémantique{Armes})
\classe{nom}
\begin{glose}
\pfra{casse-tête à bout en bec d'oiseau}
\end{glose}
\end{entrée}

\begin{entrée}{bwe-phwamwa}{}{ⓔbwe-phwamwa}
\région{GOs}
(\domainesémantique{Cultures, techniques, boutures})
\classe{nom}
\begin{glose}
\pfra{extrémité du champ (lit. tête du champ)}
\end{glose}
\end{entrée}

\begin{entrée}{bwevwu-cee}{}{ⓔbwevwu-cee}
\région{GOs}
(\domainesémantique{Arbre
, Parties de plantes})
\classe{nom}
\begin{glose}
\pfra{souche ; base de l'arbre}
\end{glose}
\end{entrée}

\begin{entrée}{bwi}{}{ⓔbwi}
\région{GOsPA BO}
(\domainesémantique{Santé, maladie})
\classe{v.stat.}
\begin{glose}
\pfra{aveugle}
\end{glose}
\begin{glose}
\pfra{conjonctivite}
\end{glose}
\newline
\begin{exemple}
\région{BO}
\textbf{\pnua{bwi mèè-n}}
\pfra{il devient aveugle (lit. ses yeux sont aveugles)}
\end{exemple}
\end{entrée}

\begin{entrée}{bwihin}{}{ⓔbwihin}
\région{BO}
\variante{%
bwiin
\région{PA}}
(\domainesémantique{Ignames})
\classe{nom}
\begin{glose}
\pfra{igname mauve}
\end{glose}
\end{entrée}

\begin{entrée}{bwili}{}{ⓔbwili}
\région{GOs}
\variante{%
dixo-bwa-n
\région{PA}}
(\domainesémantique{Corps animal})
\classe{nom}
\begin{glose}
\pfra{corne}
\end{glose}
\newline
\begin{sous-entrée}{bwili dube}{ⓔbwiliⓝbwili dube}
\begin{glose}
\pfra{corne du cerf}
\end{glose}
\end{sous-entrée}
\newline
\begin{sous-entrée}{bwili kava}{ⓔbwiliⓝbwili kava}
\begin{glose}
\pfra{corne du dawa}
\end{glose}
\end{sous-entrée}
\newline
\étymologie{
\langue{POc}
\étymon{*mpuji}}
\end{entrée}

\begin{entrée}{bwi-nu}{}{ⓔbwi-nu}
\région{BO}
(\domainesémantique{Ustensiles})
\classe{nom}
\begin{glose}
\pfra{calebasse ; noix de coco vide}
\end{glose}
\newline
\étymologie{
\langue{POc}
\étymon{*bwilo}
\glosecourte{coconut shell used as liquid container}
\auteur{Blust}}
\end{entrée}

\begin{entrée}{bwiri}{}{ⓔbwiri}
\formephonétique{bwiɽi}
\région{GOs}
\variante{%
bwirik
\région{PA BO}}
(\domainesémantique{Cordes, cordages})
\classe{nom}
\begin{glose}
\pfra{bride}
\end{glose}
\newline
\begin{sous-entrée}{pò-bwiri}{ⓔbwiriⓝpò-bwiri}
\région{GO}
\begin{glose}
\pfra{mors}
\end{glose}
\end{sous-entrée}
\newline
\begin{sous-entrée}{khô-bwiri}{ⓔbwiriⓝkhô-bwiri}
\région{GO}
\begin{glose}
\pfra{rênes}
\end{glose}
\newline
\begin{exemple}
\région{BO}
\textbf{\pnua{pwò-bwirik}}
\pfra{mors}
\end{exemple}
\newline
\begin{exemple}
\région{BO}
\textbf{\pnua{khô-bwirik}}
\pfra{rênes}
\end{exemple}
\end{sous-entrée}
\end{entrée}

\begin{entrée}{bwixuu}{}{ⓔbwixuu}
\région{GOs}
\variante{%
bwixu
\région{PA}, 
bwivu
\région{BO (Corne)}}
(\domainesémantique{Mammifères})
\classe{nom}
\begin{glose}
\pfra{chauve-souris (petite)}
\end{glose}
\end{entrée}

\begin{entrée}{bwo}{}{ⓔbwo}
\région{GOs BO}
\variante{%
bo
\région{PA BO}}
(\domainesémantique{Aliments, alimentation
, Description des objets, formes, consistance, taille})
\classe{v}
\begin{glose}
\pfra{pourri (être) ; sentir}
\end{glose}
\end{entrée}

\begin{entrée}{bwò}{1}{ⓔbwòⓗ1}
\formephonétique{bwɔ}
\région{GOs PA BO}
\variante{%
bò, bo
\région{BO}}
(\domainesémantique{Mammifères})
\classe{nom}
\begin{glose}
\pfra{roussette}
\end{glose}
\nomscientifique{Pteropus sp.}
\newline
\étymologie{
\langue{POc}
\étymon{*mpeka}}
\end{entrée}

\begin{entrée}{bwò}{2}{ⓔbwòⓗ2}
\région{GOs}
(\domainesémantique{Poissons})
\classe{nom}
\begin{glose}
\pfra{poisson-papillon ; poisson-lune}
\end{glose}
\nomscientifique{Platax teira (Ephippidés)}
\newline
\relationsémantique{Cf.}{\lien{ⓔthrimavwo}{thrimavwo}}
\glosecourte{picot de palétuvier ; poisson-papillon}
\end{entrée}

\begin{entrée}{bwò}{3}{ⓔbwòⓗ3}
\région{GOs}
\variante{%
bò
\région{GO(s)}, 
bwòn
\région{BO (Corne, BM)}}
(\domainesémantique{Temps})
\classe{nom}
\begin{glose}
\pfra{jour fixé ; date convenue}
\end{glose}
\newline
\begin{exemple}
\région{GO}
\textbf{\pnua{poniza bò mhwããnu ?}}
\pfra{quelle est la date?}
\end{exemple}
\newline
\begin{exemple}
\région{GO}
\textbf{\pnua{bwò u-da ni chòmu}}
\pfra{le jour de la rentrée scolaire}
\end{exemple}
\newline
\begin{exemple}
\région{GO}
\textbf{\pnua{bò-phade-kui}}
\pfra{la date pour montrer l'igname}
\end{exemple}
\newline
\begin{exemple}
\région{GO}
\textbf{\pnua{bò-khî kui}}
\pfra{la date pour griller l'igname}
\end{exemple}
\newline
\begin{exemple}
\région{GO}
\textbf{\pnua{bòni-ã mõnõ}}
\pfra{le moment que nous nous sommes fixés (notre date) demain}
\end{exemple}
\newline
\étymologie{
\langue{POc}
\étymon{*mpoŋi}
\glosecourte{night}}
\end{entrée}

\begin{entrée}{bwòivhe}{}{ⓔbwòivhe}
\région{GOs BO}
\variante{%
boive
\région{PA}}
(\domainesémantique{Insectes})
\classe{nom}
\begin{glose}
\pfra{papillon (sorte de)}
\end{glose}
\end{entrée}

\begin{entrée}{bwò-kabu}{}{ⓔbwò-kabu}
\région{GOs}
\région{BO PA}
\variante{%
bò-kabun
}
(\domainesémantique{Jours})
\classe{nom}
\begin{glose}
\pfra{dimanche (lit. jour sacré)}
\end{glose}
\end{entrée}

\begin{entrée}{bwòò}{}{ⓔbwòò}
\région{GOs}
\variante{%
bool
\région{WE}}
(\domainesémantique{Jeux divers})
\classe{nom}
\begin{glose}
\pfra{ballon ; balle}
\end{glose}
\newline
\emprunt{ball (GB)}
\end{entrée}

\begin{entrée}{bwòòm}{}{ⓔbwòòm}
\région{BO [BM, Corne]}
\classe{nom}
\newline
\sens{1}
(\domainesémantique{Caractéristiques et propriétés des personnes})
\begin{glose}
\pfra{calme ; paisible ; humble (personne)}
\end{glose}
\newline
\sens{2}
(\domainesémantique{Description des objets, formes, consistance, taille})
\begin{glose}
\pfra{ombre ; endroit ombragé [ BO PA]}
\end{glose}
\end{entrée}

\begin{entrée}{bwòvwô}{}{ⓔbwòvwô}
\région{GOs BO}
\variante{%
bòòvwô, bòpô
\région{GO(s)}}
(\domainesémantique{Fonctions naturelles humaines})
\classe{v}
\begin{glose}
\pfra{fatigué (être)}
\end{glose}
\begin{glose}
\pfra{épuisé}
\end{glose}
\newline
\begin{exemple}
\textbf{\pnua{kawa e bwòvwô pune la mogo i je dròrò ?}}
\pfra{n'est-il pas fatigué de son travail hier ?}
\end{exemple}
\newline
\begin{exemple}
\textbf{\pnua{Ôô ! e bwòvwô !}}
\pfra{Oui ! il en est fatigué !}
\end{exemple}
\newline
\begin{exemple}
\textbf{\pnua{kawa e bwòvwô ui la ẽnõ ni mõõ-chomu ?}}
\pfra{n'est-il pas fatigué des enfants de l'école?}
\end{exemple}
\newline
\begin{exemple}
\textbf{\pnua{Ôô ! e bwòvwô ui la !}}
\pfra{Oui ! il en est fatigué !}
\end{exemple}
\end{entrée}

\newpage

\lettrine{
c
ç
}\begin{entrée}{ca}{1}{ⓔcaⓗ1}
\formephonétique{ca}
\région{GOs BO}
(\domainesémantique{Description des objets, formes, consistance, taille})
\classe{v}
\begin{glose}
\pfra{coupant ; bien aiguisé}
\end{glose}
\newline
\begin{exemple}
\textbf{\pnua{u ca hèlè}}
\pfra{le couteau est coupant}
\end{exemple}
\newline
\relationsémantique{Cf.}{\lien{ⓔyazoo}{yazoo}}
\glosecourte{affûter}
\end{entrée}

\begin{entrée}{ca}{2}{ⓔcaⓗ2}
\formephonétique{ca}
\région{GOs PA BO}
\classe{v}
\newline
\sens{1}
(\domainesémantique{Chasse})
\begin{glose}
\pfra{toucher (cible avec sagaie)}
\end{glose}
\newline
\begin{exemple}
\région{GO}
\textbf{\pnua{e a-pha-ca !}}
\pfra{il est adroit (car ne rate pas la cible)}
\end{exemple}
\newline
\begin{exemple}
\région{GO PA}
\textbf{\pnua{u ca !}}
\pfra{touché ! dans le mille ! ça s'est réalisé !}
\end{exemple}
\newline
\sens{2}
(\domainesémantique{Verbes d'action (en général)})
\begin{glose}
\pfra{avoir lieu (pour un événement fixé)}
\end{glose}
\newline
\note{cale (v.t.) [GOs, PA]}{grammaire}{}
\end{entrée}

\begin{entrée}{ca}{3}{ⓔcaⓗ3}
\formephonétique{ca}
\région{GOs}
\variante{%
ya, yaa, yai
\région{PA}}
\classe{PREP}
\newline
\sens{1}
(\domainesémantique{Prépositions})
\begin{glose}
\pfra{à ; vers}
\end{glose}
\newline
\begin{exemple}
\région{GO}
\textbf{\pnua{nu ne hivwine-ne ca-ni la-ã pòi-nu}}
\pfra{je n'ai jamais su le faire/raconter (lit. souvent ne pas savoir faire) à mes enfants}
\end{exemple}
\newline
\begin{exemple}
\région{GO}
\textbf{\pnua{xa e wãã mwa cai (i)je xo ã kani}}
\pfra{Et le canard lui répond}
\end{exemple}
\newline
\begin{exemple}
\région{PA}
\textbf{\pnua{i khôbwe ya i je}}
\pfra{il lui dit}
\end{exemple}
\newline
\begin{exemple}
\région{BO}
\textbf{\pnua{i kôbwe yaa i nu}}
\pfra{il me dit}
\end{exemple}
\newline
\begin{exemple}
\région{BO}
\textbf{\pnua{yu kôbwe da yaa ni daalèn ?}}
\pfra{qu'as-tu dit aux Européens ?}
\end{exemple}
\newline
\begin{exemple}
\région{BO}
\textbf{\pnua{nu kôbwe yai ije nye (ou) nu kôbwe nye yai ije}}
\pfra{je lui ai dit cela}
\end{exemple}
\newline
\begin{exemple}
\région{BO}
\textbf{\pnua{nu uvi yaa hi-n}}
\pfra{je le lui ai acheté}
\end{exemple}
\newline
\sens{2}
(\domainesémantique{Prépositions})
\begin{glose}
\pfra{jusqu'à}
\end{glose}
\newline
\begin{exemple}
\région{GO}
\textbf{\pnua{Kavwö jaxa-cö vwö cö zòò ca bwa drau}}
\pfra{Tu ne pourras pas nager jusqu'à l'île}
\end{exemple}
\newline
\note{ca-ni + nom pluriel}{grammaire}{aux}
\end{entrée}

\begin{entrée}{ça}{}{ⓔça}
\formephonétique{ʒa}
\région{GOs PA}
\variante{%
ka
\région{GO(s)}, 
ce, je, ye
\région{BO PA}}
(\domainesémantique{Structure informationnelle})
\classe{THEM}
\begin{glose}
\pfra{thématisation}
\end{glose}
\newline
\begin{exemple}
\région{GO}
\textbf{\pnua{yaza aponoo-va ça wêwêne}}
\pfra{le nom de chez nous, c'est W.}
\end{exemple}
\newline
\begin{exemple}
\région{PA}
\textbf{\pnua{novo hagana ca mi a-du mwa paawa}}
\pfra{mais aujourd'hui, nous allons désherber}
\end{exemple}
\newline
\relationsémantique{Cf.}{\lien{ⓔkaⓗ1}{ka}}
\glosecourte{et}
\end{entrée}

\begin{entrée}{caa-}{}{ⓔcaa-}
\région{GO PA}
\variante{%
cè-
\région{GO}, 
ca-
\région{BO [Corne, BM]}}
(\domainesémantique{Préfixes classificateurs de la nourriture
, Préfixes classificateurs possessifs de la nourriture})
\classe{nom}
\begin{glose}
\pfra{part (sa) de féculents}
\end{glose}
\newline
\begin{exemple}
\région{GO}
\textbf{\pnua{caa-xa da ? - caa-xa lai}}
\pfra{quel est l'accompagnement? - l'accompagnement c'est du riz}
\end{exemple}
\newline
\begin{exemple}
\région{GO}
\textbf{\pnua{caa-xa cee-ã}}
\pfra{l'accompagnement de nos féculents}
\end{exemple}
\newline
\begin{exemple}
\région{GO}
\textbf{\pnua{caa-xa kui}}
\pfra{l'accompagnement de l'igname}
\end{exemple}
\newline
\begin{exemple}
\région{BO}
\textbf{\pnua{na-mi ca-ã bwa tap !}}
\pfra{apporte la nourriture sur la table ! (BM)}
\end{exemple}
\newline
\étymologie{
\langue{POc}
\étymon{*kani}}
\end{entrée}

\begin{entrée}{caaça}{}{ⓔcaaça}
\formephonétique{caːʒa caːdʒa}
\région{GOs}
\variante{%
caaya, caya
\formephonétique{caːja}
\région{PA BO}}
(\domainesémantique{Alliance})
\classe{nom}
\begin{glose}
\pfra{papa ; tonton (oncle paternel)}
\end{glose}
\newline
\begin{exemple}
\région{GO}
\textbf{\pnua{caaça-nu}}
\pfra{mon père}
\end{exemple}
\newline
\begin{exemple}
\région{PA}
\textbf{\pnua{caayè-ny}}
\pfra{mon père}
\end{exemple}
\end{entrée}

\begin{entrée}{caai}{1}{ⓔcaaiⓗ1}
\région{GOs PA}
(\domainesémantique{Aliments, alimentation})
\classe{nom}
\begin{glose}
\pfra{condiment (pour accompagner les féculents)}
\end{glose}
\newline
\note{utilisé comme terme général à Gomen}{glose}{}
\newline
\begin{exemple}
\région{GO}
\textbf{\pnua{kixa caai}}
\pfra{il n'y a rien comme condiment}
\end{exemple}
\newline
\begin{sous-entrée}{caa-xa kui}{ⓔcaaiⓗ1ⓝcaa-xa kui}
\région{GO}
\begin{glose}
\pfra{l'accompagnement de l'igname}
\end{glose}
\end{sous-entrée}
\end{entrée}

\begin{entrée}{caai}{2}{ⓔcaaiⓗ2}
\région{GOs PA BO}
\variante{%
caak, caai
\région{BO}, 
samelõ
\région{GO(s)}}
(\domainesémantique{Arbre})
\classe{nom}
\begin{glose}
\pfra{jamelonier (sorte de prunier sauvage) ; jamblon}
\end{glose}
\begin{glose}
\pfra{pommier canaque}
\end{glose}
\begin{glose}
\pfra{pomme-rose [PA]}
\end{glose}
\nomscientifique{Syzygium cuminii (Myrtacées)}
\nomscientifique{Syzygium malaccense; Eugenia malaccensis}
\nomscientifique{Syzygium jambos (Myrtacées)}
\newline
\begin{exemple}
\textbf{\pnua{pò-cai (ou) pò-caai}}
\pfra{"pomme canaque", fruit du jamelonier}
\end{exemple}
\newline
\étymologie{
\langue{POc}
\étymon{*kapika}}
\end{entrée}

\begin{entrée}{caan}{}{ⓔcaan}
\région{PA}
(\domainesémantique{Poissons})
\classe{nom}
\begin{glose}
\pfra{carpe (grosse)}
\end{glose}
\end{entrée}

\begin{entrée}{caanô}{}{ⓔcaanô}
\région{PA BO (Corne)}
(\domainesémantique{Noms des plantes})
\classe{nom}
\begin{glose}
\pfra{liane ; salsepareille}
\end{glose}
\nomscientifique{Lygodium (ou) Smilax sp.}
\end{entrée}

\begin{entrée}{caave}{}{ⓔcaave}
\région{GOs}
(\domainesémantique{Oiseaux})
\classe{nom}
\begin{glose}
\pfra{frégate (petite)}
\end{glose}
\nomscientifique{Fregata Ariel Ariel}
\end{entrée}

\begin{entrée}{caaxai}{}{ⓔcaaxai}
\région{BO}
(\domainesémantique{Relations et interaction sociales})
\classe{v}
\begin{glose}
\pfra{menacer [Corne]}
\end{glose}
\newline
\note{non vérifié}{général}{}
\end{entrée}

\begin{entrée}{caaxo}{}{ⓔcaaxo}
\région{GOs BO PA}
\variante{%
kyaaxo
\région{BO [BM]}, 
caawo
\région{BO (Corne)}}
(\domainesémantique{Relations et interaction sociales})
\classe{v}
\begin{glose}
\pfra{surprendre ; cachette (faire en) ; faire doucement}
\end{glose}
\newline
\begin{exemple}
\région{GO}
\textbf{\pnua{i vhaa caaxo}}
\pfra{il parle doucement}
\end{exemple}
\newline
\begin{exemple}
\région{BO}
\textbf{\pnua{va kyaaxo !}}
\pfra{parlez doucement ! [BM]}
\end{exemple}
\newline
\begin{exemple}
\région{BO}
\textbf{\pnua{nu caaxo a-ò}}
\pfra{j'y suis allé en cachette [BM]}
\end{exemple}
\end{entrée}

\begin{entrée}{caaxö}{}{ⓔcaaxö}
\région{GOs PA}
(\domainesémantique{Discours, échanges verbaux})
\classe{v}
\begin{glose}
\pfra{acquiescer ; répondre}
\end{glose}
\end{entrée}

\begin{entrée}{caaxô}{}{ⓔcaaxô}
\formephonétique{caːɣõ}
\région{GOs}
\région{GOs PA}
\variante{%
caxõõl
\région{PA}, 
caxool
\région{BO}, 
cawhûûl
\région{BO}}
\classe{v}
\newline
\sens{1}
(\domainesémantique{Sons, bruits})
\begin{glose}
\pfra{grommeler ; gronder ; murmurer (de mécontentement)}
\end{glose}
\newline
\sens{2}
(\domainesémantique{Discours, échanges verbaux})
\begin{glose}
\pfra{plaindre (se) constamment}
\end{glose}
\newline
\begin{exemple}
\région{GO}
\textbf{\pnua{e vhaa-caxô, e vhaa-caxû}}
\pfra{il parle entre ses dents}
\end{exemple}
\end{entrée}

\begin{entrée}{caaya}{}{ⓔcaaya}
\région{PA}
(\domainesémantique{Parenté})
\classe{nom}
\begin{glose}
\pfra{père (appellation)}
\end{glose}
\end{entrée}

\begin{entrée}{caayo!}{}{ⓔcaayo!}
\région{BO (Corne)}
\variante{%
caayu!
}
(\domainesémantique{Vocatifs})
\classe{vocatif}
\begin{glose}
\pfra{papa !}
\end{glose}
\end{entrée}

\begin{entrée}{cabeng}{}{ⓔcabeng}
\région{PA BO}
(\domainesémantique{Bananiers et bananes})
\classe{nom}
\begin{glose}
\pfra{bananier (clone de) ; variété de banane-chef}
\end{glose}
\newline
\relationsémantique{Cf.}{\lien{}{pòdi, puyai}}
\end{entrée}

\begin{entrée}{cabi}{}{ⓔcabi}
\région{GOs PA BO}
\classe{v}
\newline
\sens{1}
(\domainesémantique{Mouvements ou actions faits avec le corps, les bras, les mains, les pieds})
\begin{glose}
\pfra{frapper}
\end{glose}
\begin{glose}
\pfra{heurter}
\end{glose}
\newline
\begin{exemple}
\textbf{\pnua{cabi phweemwa !}}
\pfra{tape à la porte}
\end{exemple}
\newline
\sens{2}
(\domainesémantique{Mouvements ou actions faits avec le corps, les bras, les mains, les pieds})
\begin{glose}
\pfra{cogner ; taper ; giffler}
\end{glose}
\newline
\begin{sous-entrée}{pha-cabi}{ⓔcabiⓢ2ⓝpha-cabi}
\begin{glose}
\pfra{taper pour enfoncer}
\end{glose}
\end{sous-entrée}
\newline
\begin{sous-entrée}{ba-cabi}{ⓔcabiⓢ2ⓝba-cabi}
\begin{glose}
\pfra{marteau}
\end{glose}
\end{sous-entrée}
\newline
\sens{3}
(\domainesémantique{Verbes de mouvement})
\begin{glose}
\pfra{buter sur qqch.}
\end{glose}
\newline
\sens{4}
(\domainesémantique{Musique, instruments de musique})
\begin{glose}
\pfra{battre en rythme ; battre (cloche)}
\end{glose}
\newline
\begin{sous-entrée}{ba-cabi khaa}{ⓔcabiⓢ4ⓝba-cabi khaa}
\begin{glose}
\pfra{tambour}
\end{glose}
\end{sous-entrée}
\newline
\étymologie{
\langue{POc}
\étymon{*tapi}
\glosecourte{frapper avec la main}}
\end{entrée}

\begin{entrée}{cabicabi}{}{ⓔcabicabi}
\région{BO}
(\domainesémantique{Saisons})
\classe{nom}
\begin{glose}
\pfra{époque où l'on choisit lesignames qu'on va consommer et semer (octobre à novembre). Dubois}
\end{glose}
\newline
\relationsémantique{Cf.}{\lien{ⓔpwebae}{pwebae}}
\glosecourte{époque où les ignames commencent à mûrir}
\newline
\relationsémantique{Cf.}{\lien{}{maxal}}
\end{entrée}

\begin{entrée}{cabo}{}{ⓔcabo}
\région{GOs}
\variante{%
cabwòl, cabòl
\région{PA BO WEM}}
\classe{v}
\newline
\sens{1}
(\domainesémantique{Fonctions naturelles humaines})
\begin{glose}
\pfra{lever (se)}
\end{glose}
\begin{glose}
\pfra{réveiller (se)}
\end{glose}
\newline
\sens{2}
(\domainesémantique{Verbes de mouvement})
\begin{glose}
\pfra{monter}
\end{glose}
\begin{glose}
\pfra{apparaître}
\end{glose}
\begin{glose}
\pfra{émerger}
\end{glose}
\newline
\begin{exemple}
\région{GO}
\textbf{\pnua{e cabo a}}
\pfra{le soleil se lève}
\end{exemple}
\newline
\begin{exemple}
\région{BO}
\textbf{\pnua{i cabòl al}}
\pfra{le soleil se lève}
\end{exemple}
\newline
\begin{sous-entrée}{a-cabo}{ⓔcaboⓢ2ⓝa-cabo}
\région{GO}
\begin{glose}
\pfra{maternels}
\end{glose}
\end{sous-entrée}
\newline
\begin{sous-entrée}{a-yabòl}{ⓔcaboⓢ2ⓝa-yabòl}
\région{PA}
\begin{glose}
\pfra{neveux côté maternel}
\end{glose}
\newline
\note{pa-cabo-ni, pa-jabo-ni}{grammaire}{révéler qqch.}
\end{sous-entrée}
\newline
\sens{3}
(\domainesémantique{Processus liés aux plantes})
\classe{v ; n}
\begin{glose}
\pfra{repousse des feuilles}
\end{glose}
\begin{glose}
\pfra{reverdir}
\end{glose}
\newline
\begin{exemple}
\région{GO}
\région{GO}
\textbf{\pnua{e cabo kibwòò-cee}}
\pfra{les bourgeons repoussent}
\end{exemple}
\end{entrée}

\begin{entrée}{cabo kò}{}{ⓔcabo kò}
\région{GOs}
\variante{%
cabwo kò
\région{GO(s)}, 
cabwò kòl
\région{BO}}
(\domainesémantique{Mouvements ou actions faits avec le corps, les bras, les mains, les pieds})
\classe{v}
\begin{glose}
\pfra{mettre debout (se)}
\end{glose}
\newline
\begin{exemple}
\textbf{\pnua{cabwo kò !}}
\pfra{lève-toi !}
\end{exemple}
\end{entrée}

\begin{entrée}{cabòl !}{}{ⓔcabòl !}
\région{PA BO}
(\domainesémantique{Verbes de déplacement et moyens de déplacement})
\classe{INTJ}
\begin{glose}
\pfra{sortir ; sors !}
\end{glose}
\newline
\note{appels prononcés par la personne qui est sur le toit à destination de celui qui est à l'intérieur de la maison pour qu'il pique l'alène et la pousse vers lui (cabòl!), ensuite on crie (a-ò! : pars) pour la récupérer ensuite.}{glose}{}
\newline
\relationsémantique{Cf.}{\lien{ⓔa-ò !}{a-ò !}}
\glosecourte{pars !}
\end{entrée}

\begin{entrée}{cabul}{}{ⓔcabul}
\région{PA BO [BM]}
(\domainesémantique{Actions liées aux éléments (liquide, fumée)})
\classe{v}
\begin{glose}
\pfra{déborder}
\end{glose}
\newline
\relationsémantique{Cf.}{\lien{}{phu [GOs]}}
\glosecourte{déborder}
\end{entrée}

\begin{entrée}{cabwau}{}{ⓔcabwau}
\région{BO}
(\domainesémantique{Ignames})
\classe{nom}
\begin{glose}
\pfra{igname violette (Dubois)}
\end{glose}
\end{entrée}

\begin{entrée}{caçai}{}{ⓔcaçai}
\formephonétique{caʒai}
\région{GOs}
\variante{%
cayaai
\région{PA}}
(\domainesémantique{Aliments, alimentation})
\classe{v}
\begin{glose}
\pfra{mâcher (en écrasant)}
\end{glose}
\newline
\étymologie{
\langue{POc}
\étymon{*kaRati}
\glosecourte{mordre, tenir entre les dents}}
\end{entrée}

\begin{entrée}{ça ea ?}{}{ⓔça ea ?}
\formephonétique{ʒa ea}
\région{GOs}
(\domainesémantique{Interrogatifs})
\classe{INT.LOC (dynamique)}
\begin{glose}
\pfra{jusqu'où ?}
\end{glose}
\newline
\begin{exemple}
\textbf{\pnua{e a ça ea lòtò-ã ?}}
\pfra{jusqu'où va cette voiture ?}
\end{exemple}
\end{entrée}

\begin{entrée}{cai}{}{ⓔcai}
\formephonétique{caiʒaɨ}
\région{GOs WEM}
\variante{%
çai
\région{GO(s)}, 
yai
\région{PA BO}}
(\domainesémantique{Prépositions})
\classe{PREP (objet indirect)}
\begin{glose}
\pfra{à (destinataire animé)}
\end{glose}
\newline
\begin{exemple}
\région{GO}
\textbf{\pnua{i khõbwe cai la}}
\pfra{il leur dit}
\end{exemple}
\newline
\begin{exemple}
\région{BO}
\textbf{\pnua{i khõbwe jai/yai la}}
\pfra{il leur dit}
\end{exemple}
\newline
\begin{exemple}
\région{WEM}
\textbf{\pnua{ole jai jö}}
\pfra{merci à toi}
\end{exemple}
\newline
\relationsémantique{Cf.}{\lien{ⓔcaⓗ1}{ca}}
\glosecourte{à , pour (+inanimés)}
\end{entrée}

\begin{entrée}{caigo}{}{ⓔcaigo}
\formephonétique{caiŋgo}
\région{GOs BO}
(\domainesémantique{Mouvements ou actions avec la tête, les yeux, la bouche})
\classe{v}
\begin{glose}
\pfra{attraper avec les dents}
\end{glose}
\begin{glose}
\pfra{tenir avec les dents}
\end{glose}
\begin{glose}
\pfra{couper (avec les dents)}
\end{glose}
\end{entrée}

\begin{entrée}{caivwo}{}{ⓔcaivwo}
\région{GOs PA BO}
\variante{%
caipo
\région{GO vx}}
(\domainesémantique{Mammifères})
\classe{nom}
\begin{glose}
\pfra{mulot ; souris}
\end{glose}
\newline
\relationsémantique{Cf.}{\lien{}{ciibwin, zine}}
\glosecourte{rat}
\end{entrée}

\begin{entrée}{ca-khã}{}{ⓔca-khã}
\région{GOs}
(\domainesémantique{Verbes de mouvement})
\classe{v}
\begin{glose}
\pfra{ricocher}
\end{glose}
\begin{glose}
\pfra{effleurer}
\end{glose}
\end{entrée}

\begin{entrée}{caladaa}{}{ⓔcaladaa}
\région{GOs}
(\domainesémantique{Société})
\classe{nom}
\begin{glose}
\pfra{gendarme}
\end{glose}
\newline
\emprunt{gendarme (FR)}
\end{entrée}

\begin{entrée}{calaru}{}{ⓔcalaru}
\région{PA}
(\domainesémantique{Insectes})
\classe{nom}
\begin{glose}
\pfra{sauterelle (marron, petite)}
\end{glose}
\end{entrée}

\begin{entrée}{cale}{}{ⓔcale}
\région{GOs BO}
(\domainesémantique{Feu : objets et actions liés au feu})
\classe{v}
\begin{glose}
\pfra{allumer (feu, lampe, cigarette, briquet)}
\end{glose}
\newline
\begin{sous-entrée}{cale pwaip}{ⓔcaleⓝcale pwaip}
\région{BO}
\begin{glose}
\pfra{allumer une pipe}
\end{glose}
\end{sous-entrée}
\newline
\begin{sous-entrée}{cale yaai}{ⓔcaleⓝcale yaai}
\région{GO}
\begin{glose}
\pfra{allumer le feu}
\end{glose}
\end{sous-entrée}
\newline
\begin{sous-entrée}{cale kibi}{ⓔcaleⓝcale kibi}
\région{BO}
\begin{glose}
\pfra{allumer le four}
\end{glose}
\end{sous-entrée}
\end{entrée}

\begin{entrée}{calii}{}{ⓔcalii}
\région{GOs PA BO}
(\domainesémantique{Noms des plantes})
\classe{nom}
\begin{glose}
\pfra{magnania (petit tubercule sauvage)}
\end{glose}
\nomscientifique{Pueraria sp.}
\end{entrée}

\begin{entrée}{calo}{}{ⓔcalo}
\région{GO}
(\domainesémantique{Corps humain})
\classe{nom}
\begin{glose}
\pfra{fontanelle [Haudricourt]}
\end{glose}
\newline
\note{non reconnu par les locuteurs actuels}{général}{}
\end{entrée}

\begin{entrée}{calò}{}{ⓔcalò}
\région{BO}
(\domainesémantique{Poissons})
\classe{nom}
\begin{glose}
\pfra{carangue jaune (à l'âge adulte) [Corne]}
\end{glose}
\nomscientifique{Carangoides gilberti et Gnathanodon speciosus (Carangidées)}
\newline
\note{non vérifié}{général}{}
\end{entrée}

\begin{entrée}{ca-ma}{}{ⓔca-ma}
\région{PA BO}
\classe{nom}
\newline
\sens{1}
(\domainesémantique{Fonctions naturelles humaines})
\begin{glose}
\pfra{bave de mort (lit. nourriture des morts)}
\end{glose}
\newline
\sens{2}
(\domainesémantique{Santé, maladie})
\begin{glose}
\pfra{épilepsie}
\end{glose}
\newline
\begin{exemple}
\textbf{\pnua{ca-ma-n}}
\pfra{bave (épilepsie)}
\end{exemple}
\end{entrée}

\begin{entrée}{camadi}{}{ⓔcamadi}
\formephonétique{camandi}
\région{GOs}
(\domainesémantique{Sentiments})
\classe{v}
\begin{glose}
\pfra{mauvaise conscience (avoir)}
\end{glose}
\newline
\begin{exemple}
\textbf{\pnua{e camadi}}
\pfra{il a mauvaiseconscience}
\end{exemple}
\newline
\begin{sous-entrée}{i a-camadi}{ⓔcamadiⓝi a-camadi}
\begin{glose}
\pfra{prédicateur}
\end{glose}
\end{sous-entrée}
\end{entrée}

\begin{entrée}{camhãã}{}{ⓔcamhãã}
\région{PA BO}
\variante{%
kyamhãã
\région{BO (Corne)}}
(\domainesémantique{Arbre})
\classe{nom}
\begin{glose}
\pfra{ficus}
\end{glose}
\nomscientifique{Ficus sp.ou Ficus scabra Forster (Moracées)}
\end{entrée}

\begin{entrée}{canabwe}{}{ⓔcanabwe}
\région{BO}
(\domainesémantique{Taros})
\classe{nom}
\begin{glose}
\pfra{taro (clone de) de terrain sec (Dubois)}
\end{glose}
\end{entrée}

\begin{entrée}{cani}{}{ⓔcani}
\région{BO}
(\domainesémantique{Aliments, alimentation})
\classe{v}
\begin{glose}
\pfra{manger (féculents)}
\end{glose}
\newline
\begin{exemple}
\région{BO}
\textbf{\pnua{i thiò dimwa ma wu i ra un cani}}
\pfra{il gratte son igname pour la manger}
\end{exemple}
\newline
\étymologie{
\langue{POc}
\étymon{*kani}}
\end{entrée}

\begin{entrée}{caro}{}{ⓔcaro}
\région{BO}
\variante{%
caaro
}
(\domainesémantique{Danses})
\classe{nom}
\begin{glose}
\pfra{danse des morts (effectuée par les femmes) [Corne]}
\end{glose}
\newline
\relationsémantique{Cf.}{\lien{ⓔcitèèn}{citèèn}}
\glosecourte{danse des morts (effectuée par les hommes)}
\newline
\note{non vérifié}{général}{}
\end{entrée}

\begin{entrée}{carû}{}{ⓔcarû}
\formephonétique{caɽû}
\région{GOs}
\variante{%
carun
\région{PA BO}}
(\domainesémantique{Feu : objets et actions liés au feu})
\classe{v}
\begin{glose}
\pfra{attiser ; pousser le feu (en ajoutant du bois)}
\end{glose}
\newline
\begin{exemple}
\région{PA}
\textbf{\pnua{carûni yaai !}}
\pfra{pousse le feu !}
\end{exemple}
\newline
\begin{exemple}
\région{BO}
\textbf{\pnua{pha carun}}
\pfra{pierres pour le four enterré}
\end{exemple}
\newline
\relationsémantique{Cf.}{\lien{}{tha-carûni [PA]}}
\glosecourte{pousser le feu}
\end{entrée}

\begin{entrée}{cauvala}{}{ⓔcauvala}
\région{PA}
(\domainesémantique{Sentiments})
\classe{v}
\begin{glose}
\pfra{exprimer son mécontentement (se dit de gens qu'on entend de loin, sans entendre le détail de ce qu'ils disent)}
\end{glose}
\newline
\begin{exemple}
\textbf{\pnua{la cauvala}}
\pfra{ils sont mécontents, expriment leur mécontentement}
\end{exemple}
\end{entrée}

\begin{entrée}{cavwe}{}{ⓔcavwe}
\formephonétique{caβe}
\région{GOs}
(\domainesémantique{Quantificateurs})
\classe{COLL ; QNT}
\begin{glose}
\pfra{ensemble}
\end{glose}
\newline
\begin{exemple}
\région{PA}
\textbf{\pnua{la cavwe khai wa}}
\pfra{ils tirent tous la corde}
\end{exemple}
\newline
\begin{exemple}
\région{GO}
\textbf{\pnua{mo cavwe a, kixa ne yu !}}
\pfra{partons ensemble, personne ne reste !}
\end{exemple}
\newline
\begin{exemple}
\région{GO}
\textbf{\pnua{mo cavwe wa !}}
\pfra{chantons ensemble !}
\end{exemple}
\newline
\begin{exemple}
\région{GO}
\textbf{\pnua{la cavwe wa !}}
\pfra{ils sont partis ensemble !}
\end{exemple}
\newline
\begin{exemple}
\région{GO}
\textbf{\pnua{lo cavwe pe-be-yaza}}
\pfra{ils(3) ont ensemble le même nom}
\end{exemple}
\end{entrée}

\begin{entrée}{cawane}{}{ⓔcawane}
\région{GOs}
\variante{%
cawan
\région{BO (Corne)}}
(\domainesémantique{Navigation})
\classe{nom}
\begin{glose}
\pfra{mât}
\end{glose}
\end{entrée}

\begin{entrée}{caxòò}{}{ⓔcaxòò}
\région{GOs}
(\domainesémantique{Verbes de mouvement})
\classe{v}
\begin{glose}
\pfra{monter aux arbres à quatre pattes, en écartant le corps du tronc}
\end{glose}
\end{entrée}

\begin{entrée}{cayae}{}{ⓔcayae}
\région{PA}
\variante{%
ceyai
\région{BO [BM]}}
(\domainesémantique{Aliments, alimentation})
\classe{v}
\begin{glose}
\pfra{mâcher (en général)}
\end{glose}
\end{entrée}

\begin{entrée}{cazae}{}{ⓔcazae}
\région{GOs}
(\domainesémantique{Verbes d'action (en général)})
\classe{v}
\begin{glose}
\pfra{esquinter ; abimer}
\end{glose}
\end{entrée}

\begin{entrée}{ce}{1}{ⓔceⓗ1}
\formephonétique{cɨ}
\région{GOs PA BO}
\variante{%
ce
\région{PA}, 
chee
\région{BO}}
(\domainesémantique{Matière, matériaux
, Arbre})
\classe{nom}
\begin{glose}
\pfra{bois}
\end{glose}
\begin{glose}
\pfra{arbre}
\end{glose}
\newline
\begin{sous-entrée}{ce do}{ⓔceⓗ1ⓝce do}
\région{PA BO}
\begin{glose}
\pfra{bois pour la cuisine (lit. bois marmite)}
\end{glose}
\end{sous-entrée}
\newline
\begin{sous-entrée}{ce bwòn}{ⓔceⓗ1ⓝce bwòn}
\région{BO}
\begin{glose}
\pfra{bûche pour la nuit (lit. bois nuit)}
\end{glose}
\end{sous-entrée}
\newline
\begin{sous-entrée}{cee bon}{ⓔceⓗ1ⓝcee bon}
\région{PA}
\begin{glose}
\pfra{bûche pour la nuit}
\end{glose}
\end{sous-entrée}
\newline
\begin{sous-entrée}{ce bwo}{ⓔceⓗ1ⓝce bwo}
\région{GO}
\begin{glose}
\pfra{bûche pour la nuit}
\end{glose}
\end{sous-entrée}
\newline
\begin{sous-entrée}{ce he}{ⓔceⓗ1ⓝce he}
\région{GO}
\begin{glose}
\pfra{bois pour allumer le feu (lit. bois frotté)}
\end{glose}
\end{sous-entrée}
\newline
\begin{sous-entrée}{ce cia}{ⓔceⓗ1ⓝce cia}
\région{GO}
\begin{glose}
\pfra{poteau de danse}
\end{glose}
\end{sous-entrée}
\newline
\begin{sous-entrée}{ce-kiai, ce-xiai}{ⓔceⓗ1ⓝce-kiai, ce-xiai}
\région{GO}
\begin{glose}
\pfra{bois pour le feu, pour la cuisine}
\end{glose}
\end{sous-entrée}
\newline
\begin{sous-entrée}{ce ko}{ⓔceⓗ1ⓝce ko}
\région{GO}
\begin{glose}
\pfra{perches (devant la porte) (lit. bois debout)}
\end{glose}
\end{sous-entrée}
\newline
\begin{sous-entrée}{ce kòl}{ⓔceⓗ1ⓝce kòl}
\région{PA}
\begin{glose}
\pfra{perches (devant la porte) (lit. bois debout)}
\end{glose}
\end{sous-entrée}
\newline
\begin{sous-entrée}{ce mwa}{ⓔceⓗ1ⓝce mwa}
\région{BO}
\begin{glose}
\pfra{solive}
\end{glose}
\end{sous-entrée}
\newline
\begin{sous-entrée}{ce nobu}{ⓔceⓗ1ⓝce nobu}
\région{GO}
\begin{glose}
\pfra{perche (signalant un interdit)}
\end{glose}
\end{sous-entrée}
\newline
\begin{sous-entrée}{ce kabun}{ⓔceⓗ1ⓝce kabun}
\région{PA}
\begin{glose}
\pfra{perche (signalant un interdit)}
\end{glose}
\end{sous-entrée}
\newline
\étymologie{
\langue{POc}
\étymon{*kai}}
\end{entrée}

\begin{entrée}{ce}{2}{ⓔceⓗ2}
\région{GOs}
\variante{%
je
\région{GOs}, 
ye
\région{PA}}
(\domainesémantique{Structure informationnelle})
\classe{THEM}
\begin{glose}
\pfra{thématisation}
\end{glose}
\newline
\begin{exemple}
\région{GO}
\textbf{\pnua{da yaaza-cu ? yaaza-nu Isabelle - yaaza-nu ce/je Isabelle}}
\pfra{quel est ton nom ? - je m'appelle Isabelle - mon nom c'est I.}
\end{exemple}
\end{entrée}

\begin{entrée}{cè-}{}{ⓔcè-}
\région{GOs}
\variante{%
caa-
\région{PA BO}}
(\domainesémantique{Préfixes classificateurs possessifs de la nourriture})
\classe{n ; CLF.POSS}
\begin{glose}
\pfra{part de féculents}
\end{glose}
\newline
\begin{exemple}
\textbf{\pnua{cè-nu}}
\pfra{ma nourriture}
\end{exemple}
\newline
\begin{exemple}
\région{GO}
\textbf{\pnua{cè-nu kui}}
\pfra{ma part d'igname}
\end{exemple}
\newline
\begin{exemple}
\région{GO}
\textbf{\pnua{cè-ã kuru}}
\pfra{notre part de taro}
\end{exemple}
\newline
\begin{exemple}
\région{PA}
\textbf{\pnua{caa-ny kuvi}}
\pfra{ma part d'igname}
\end{exemple}
\newline
\relationsémantique{Cf.}{\lien{}{kû-}}
\glosecourte{manger (fruits)}
\newline
\étymologie{
\langue{POc}
\étymon{*ka}}
\end{entrée}

\begin{entrée}{ce-baalu}{}{ⓔce-baalu}
\région{GOs}
(\domainesémantique{Ponts})
\classe{nom}
\begin{glose}
\pfra{passerelle ; planche servant de pont pour traverser une rivière (ou posé sur la boue)}
\end{glose}
\end{entrée}

\begin{entrée}{cebaèp}{}{ⓔcebaèp}
\région{BO}
(\domainesémantique{Vents})
\classe{nom}
\begin{glose}
\pfra{vent d'est}
\end{glose}
\end{entrée}

\begin{entrée}{ce ba-thi halelewa}{}{ⓔce ba-thi halelewa}
\région{GOs}
(\domainesémantique{Instruments
, Chasse})
\classe{nom}
\begin{glose}
\pfra{bâton pour attraper les cigales (avec la glu du gommier)}
\end{glose}
\newline
\note{jeu d'enfants}{glose}{}
\end{entrée}

\begin{entrée}{ce-bò}{}{ⓔce-bò}
\formephonétique{cɨ mbɔ}
\région{GOs}
\variante{%
ce-bòn
\région{WEM BO}, 
ci-bòn
\région{PA}}
(\domainesémantique{Bois
, Feu : objets et actions liés au feu})
\classe{nom}
\begin{glose}
\pfra{bûche (grosse, pour la nuit, portée par les hommes)}
\end{glose}
\newline
\begin{sous-entrée}{kô-pa-ce-bò}{ⓔce-bòⓝkô-pa-ce-bò}
\région{GO}
\begin{glose}
\pfra{couché près du feu}
\end{glose}
\end{sous-entrée}
\newline
\begin{sous-entrée}{ce-bò yaai}{ⓔce-bòⓝce-bò yaai}
\région{GO}
\begin{glose}
\pfra{bois pour le feu}
\end{glose}
\end{sous-entrée}
\end{entrée}

\begin{entrée}{cebòn}{}{ⓔcebòn}
\région{BO}
(\domainesémantique{Anguilles})
\classe{nom}
\begin{glose}
\pfra{anguille (variété d') [BM]}
\end{glose}
\end{entrée}

\begin{entrée}{ce-cia}{}{ⓔce-cia}
\région{BO}
(\domainesémantique{Danses})
\classe{nom}
\begin{glose}
\pfra{poteaux de danse [Corne]}
\end{glose}
\end{entrée}

\begin{entrée}{cêê}{}{ⓔcêê}
\région{GOs}
\variante{%
cê
\région{BO}}
(\domainesémantique{Corps humain})
\classe{nom}
\begin{glose}
\pfra{sexe (femme) ; vulve; vagin}
\end{glose}
\newline
\begin{sous-entrée}{kumèè-cêê}{ⓔcêêⓝkumèè-cêê}
\begin{glose}
\pfra{clitoris}
\end{glose}
\end{sous-entrée}
\newline
\begin{sous-entrée}{pu-cêê}{ⓔcêêⓝpu-cêê}
\begin{glose}
\pfra{poils de pubis}
\end{glose}
\end{sous-entrée}
\newline
\étymologie{
\langue{POc}
\étymon{*kala}
\glosecourte{parties génitales}}
\end{entrée}

\begin{entrée}{cee-xòò}{}{ⓔcee-xòò}
\région{PA}
\variante{%
ce-kòòl
\région{BO (Corne)}}
(\domainesémantique{Objets coutumiers})
\classe{nom}
\begin{glose}
\pfra{perches devant la porte}
\end{glose}
\end{entrée}

\begin{entrée}{cego}{}{ⓔcego}
\région{PA}
(\domainesémantique{Objets coutumiers})
\classe{nom}
\begin{glose}
\pfra{perches plantées devant les portes des cases (Dubois)}
\end{glose}
\end{entrée}

\begin{entrée}{ce he}{}{ⓔce he}
\formephonétique{cɨ he}
\région{GOs PA BO}
(\domainesémantique{Feu : objets et actions liés au feu})
\classe{v}
\begin{glose}
\pfra{bois sur lequel on frotte pour faire du feu}
\end{glose}
\newline
\relationsémantique{Cf.}{\lien{ⓔyaa-he}{yaa-he}}
\glosecourte{feu allumé par friction}
\newline
\relationsémantique{Cf.}{\lien{ⓔhe}{he}}
\glosecourte{frotter}
\end{entrée}

\begin{entrée}{ce-kabun}{}{ⓔce-kabun}
\région{BO}
(\domainesémantique{Objets coutumiers})
\classe{nom}
\begin{glose}
\pfra{perche signalant un interdit [Corne]}
\end{glose}
\newline
\relationsémantique{Cf.}{\lien{ⓔce-nôbu}{ce-nôbu}}
\end{entrée}

\begin{entrée}{ce-ka, ce-xa}{}{ⓔce-ka, ce-xa}
\région{GOs}
\variante{%
ce-kam
\région{BO (Corne)}}
(\domainesémantique{Noms des plantes})
\classe{nom}
\begin{glose}
\pfra{arbuste ; bagayou des vieux (voir le livre des plantes du chemin kanak)}
\end{glose}
\nomscientifique{Polyscias scutelaria (N.L. Burm.) Fosberg (Araliacées)}
\end{entrée}

\begin{entrée}{ce-kiyai}{}{ⓔce-kiyai}
\région{GOs BO}
\variante{%
ce-xiyai
}
(\domainesémantique{Feu : objets et actions liés au feu})
\classe{nom}
\begin{glose}
\pfra{bois pour la cuisine}
\end{glose}
\begin{glose}
\pfra{bois pour le feu allumé dehors pour se réchauffer}
\end{glose}
\newline
\relationsémantique{Cf.}{\lien{}{ce-bò ; ce bwo}}
\glosecourte{bois pour la nuit}
\end{entrée}

\begin{entrée}{ce-kui}{}{ⓔce-kui}
\région{GOs}
\variante{%
ce-xui
\région{GO(s)}}
(\domainesémantique{Noms des plantes})
\classe{nom}
\begin{glose}
\pfra{Mimusops parviflora}
\end{glose}
\nomscientifique{Mimusops parviflora}
\end{entrée}

\begin{entrée}{ce-kura}{}{ⓔce-kura}
\formephonétique{cɨ kura}
\formephonétique{cɨ kuʈa cɨ kuɽa}
\région{GOs PA}
\variante{%
ce-kutra
\région{GO(s)}}
(\domainesémantique{Noms des plantes})
\classe{nom}
\begin{glose}
\pfra{bois de sang ; "sang dragon", Euphorbiacée}
\end{glose}
\nomscientifique{Pterocarpus indicus Willdenow (Fabacées Caesalpinioidées)}
\end{entrée}

\begin{entrée}{celèng}{}{ⓔcelèng}
\formephonétique{celɛŋ}
\région{PA BO [BM]}
(\domainesémantique{Verbes d'action (en général)})
\classe{nom}
\begin{glose}
\pfra{effleurer}
\end{glose}
\end{entrée}

\begin{entrée}{ce-mããni}{}{ⓔce-mããni}
\région{GOs}
(\domainesémantique{Noms des plantes})
\classe{nom}
\begin{glose}
\pfra{sensitive}
\end{glose}
\newline
\note{(plante avec des piquants, une tige rouge , des feuilles vert-gris, elle se ferme quand on la touche, d'où "mããni" 'dormir')}{glose}{}
\nomscientifique{Mimosa pudica L.}
\end{entrée}

\begin{entrée}{ce-mwa}{}{ⓔce-mwa}
\région{PA BO}
(\domainesémantique{Types de maison, architecture de la maison})
\classe{nom}
\begin{glose}
\pfra{chevrons ; solives}
\end{glose}
\newline
\note{(les solives suivent la pente du toit et supportent les gaulettes qui tiennent la paille)}{glose}{}
\newline
\relationsémantique{Cf.}{\lien{ⓔce-nuda}{ce-nuda}}
\glosecourte{solives}
\end{entrée}

\begin{entrée}{cèni}{}{ⓔcèni}
\formephonétique{'cɛɳi}
\région{GOs}
\variante{%
cani
\région{BO PA}}
(\domainesémantique{Aliments, alimentation})
\classe{v}
\begin{glose}
\pfra{manger (des féculents)}
\end{glose}
\newline
\begin{exemple}
\textbf{\pnua{i cani pò-wha}}
\pfra{il mange des figues sauvages}
\end{exemple}
\newline
\begin{exemple}
\région{BO PA}
\textbf{\pnua{i aa-cani}}
\pfra{il est gourmand}
\end{exemple}
\newline
\étymologie{
\langue{POc}
\étymon{*kani}}
\end{entrée}

\begin{entrée}{ce-nôbu}{}{ⓔce-nôbu}
\région{BO}
(\domainesémantique{Objets coutumiers})
\classe{nom}
\begin{glose}
\pfra{perche signalant un interdit [Corne]}
\end{glose}
\newline
\relationsémantique{Cf.}{\lien{ⓔce-kabun}{ce-kabun}}
\glosecourte{perche signalant un interdit}
\end{entrée}

\begin{entrée}{ce-nuda}{}{ⓔce-nuda}
\région{PA BO}
(\domainesémantique{Types de maison, architecture de la maison})
\classe{nom}
\begin{glose}
\pfra{solives}
\end{glose}
\newline
\note{(les solives supportent les gaulettes qui retiennent les peaux de niaoulis et la paille couvrant la maison ; plus petit que ce-mwa ; Dubois)}{glose}{}
\newline
\relationsémantique{Cf.}{\lien{}{orèi, zabo}}
\end{entrée}

\begin{entrée}{cetil}{}{ⓔcetil}
\région{BO}
(\domainesémantique{Armes})
\classe{nom}
\begin{glose}
\pfra{plumet de fronde (en fibre d'aloès) [Corne]}
\end{glose}
\newline
\relationsémantique{Cf.}{\lien{ⓔthila}{thila}}
\glosecourte{plumet de fronde plus court}
\newline
\note{non vérifié}{général}{}
\end{entrée}

\begin{entrée}{ce-tha}{}{ⓔce-tha}
\région{GOs WEM}
(\domainesémantique{Types de maison, architecture de la maison})
\classe{nom}
\begin{glose}
\pfra{poutre faitîère (rejoint le "pwabwani" au sommet du toit)}
\end{glose}
\begin{glose}
\pfra{charpente (maison)}
\end{glose}
\end{entrée}

\begin{entrée}{ce-thîni}{}{ⓔce-thîni}
\région{GOs}
(\domainesémantique{Types de maison, architecture de la maison})
\classe{nom}
\begin{glose}
\pfra{poteaux de barrière}
\end{glose}
\end{entrée}

\begin{entrée}{ce-thri}{}{ⓔce-thri}
\région{GOs}
(\domainesémantique{Objets coutumiers})
\classe{nom}
\begin{glose}
\pfra{perche qui annonce le décès d'un chef}
\end{glose}
\end{entrée}

\begin{entrée}{ce-vada}{}{ⓔce-vada}
\région{GOs PA}
\variante{%
ce-pada
\région{GO(s)}}
(\domainesémantique{Arbre})
\classe{nom}
\begin{glose}
\pfra{flamboyant}
\end{glose}
\nomscientifique{Serianthes calycina Benth.}
\end{entrée}

\begin{entrée}{cèvèro}{}{ⓔcèvèro}
\région{BO PA BO}
(\domainesémantique{Mammifères})
\classe{nom}
\begin{glose}
\pfra{cerf}
\end{glose}
\end{entrée}

\begin{entrée}{ce-xou}{}{ⓔce-xou}
\région{GOs PA}
\variante{%
hup
\région{PA}}
(\domainesémantique{Arbre})
\classe{nom}
\begin{glose}
\pfra{houp}
\end{glose}
\nomscientifique{Montrouziera sp.}
\end{entrée}

\begin{entrée}{ci}{}{ⓔci}
\région{GOs}
\variante{%
cin
\région{WEM WE BO}}
(\domainesémantique{Arbre})
\classe{nom}
\newline
\sens{1}
\begin{glose}
\pfra{arbre à pain}
\end{glose}
\nomscientifique{Artocarpus altilis (Park.) Fosb.}
\newline
\sens{2}
\begin{glose}
\pfra{papayer}
\end{glose}
\nomscientifique{Carica papaya L.}
\newline
\begin{sous-entrée}{po-ci}{ⓔciⓢ2ⓝpo-ci}
\région{GO}
\begin{glose}
\pfra{papaye}
\end{glose}
\newline
\begin{exemple}
\région{BO}
\textbf{\pnua{po-cin}}
\pfra{papaye}
\end{exemple}
\newline
\begin{exemple}
\région{PA}
\textbf{\pnua{cin phai}}
\pfra{fruit de l'arbre à pain}
\end{exemple}
\end{sous-entrée}
\newline
\étymologie{
\langue{POc}
\étymon{*kulu(R)}}
\end{entrée}

\begin{entrée}{cî}{}{ⓔcî}
\région{GOs PA BO}
(\domainesémantique{Crustacés, crabes})
\classe{nom}
\begin{glose}
\pfra{crabe de palétuvier}
\end{glose}
\newline
\note{plus petit que "ji, jim", de couleur noire, il creuse la terre.}{glose}{}
\end{entrée}

\begin{entrée}{cia}{}{ⓔcia}
\formephonétique{cɨ.a}
\région{GOs BO PA}
(\domainesémantique{Danses})
\classe{v ; n}
\begin{glose}
\pfra{danser ; danse}
\end{glose}
\end{entrée}

\begin{entrée}{cibaalu}{}{ⓔcibaalu}
\région{GOs}
(\domainesémantique{Ponts})
\classe{nom}
\begin{glose}
\pfra{pont en bois}
\end{glose}
\end{entrée}

\begin{entrée}{cibò}{}{ⓔcibò}
\région{GOs}
(\domainesémantique{Corps animal})
\classe{nom}
\begin{glose}
\pfra{queue (poisson)}
\end{glose}
\newline
\begin{sous-entrée}{cibò nò}{ⓔcibòⓝcibò nò}
\begin{glose}
\pfra{queue de poisson}
\end{glose}
\end{sous-entrée}
\end{entrée}

\begin{entrée}{ci-ciò}{}{ⓔci-ciò}
\région{GOs}
(\domainesémantique{Instruments})
\classe{nom}
\begin{glose}
\pfra{tôle (lit. peau/couverture en tôle)}
\end{glose}
\newline
\relationsémantique{Cf.}{\lien{}{ci-}}
\glosecourte{peau}
\end{entrée}

\begin{entrée}{ci-chaamwa}{}{ⓔci-chaamwa}
\formephonétique{ci-cʰaːmwa}
\région{GOs}
(\domainesémantique{Bananiers et bananes})
\classe{nom}
\begin{glose}
\pfra{enveloppe de tronc de bananier}
\end{glose}
\end{entrée}

\begin{entrée}{cii}{1}{ⓔciiⓗ1}
\région{GOs PA BO}
\classe{nom}
\newline
\sens{1}
(\domainesémantique{Corps humain})
\begin{glose}
\pfra{peau}
\end{glose}
\newline
\begin{sous-entrée}{ci-phwa-n}{ⓔciiⓗ1ⓢ1ⓝci-phwa-n}
\begin{glose}
\pfra{ses lèvres}
\end{glose}
\end{sous-entrée}
\newline
\sens{2}
(\domainesémantique{Parties de plantes})
\begin{glose}
\pfra{écorce}
\end{glose}
\newline
\begin{sous-entrée}{ci-ce}{ⓔciiⓗ1ⓢ2ⓝci-ce}
\begin{glose}
\pfra{écorce}
\end{glose}
\end{sous-entrée}
\newline
\begin{sous-entrée}{ci-kui}{ⓔciiⓗ1ⓢ2ⓝci-kui}
\begin{glose}
\pfra{peau de l'igname}
\end{glose}
\end{sous-entrée}
\newline
\begin{sous-entrée}{ci-chaamwa}{ⓔciiⓗ1ⓢ2ⓝci-chaamwa}
\région{GO}
\begin{glose}
\pfra{enveloppe de tronc de bananier}
\end{glose}
\newline
\note{cii-n}{grammaire}{sa peau}
\end{sous-entrée}
\newline
\étymologie{
\langue{POc}
\étymon{*kuli(t)}}
\end{entrée}

\begin{entrée}{cii}{2}{ⓔciiⓗ2}
\région{GOs PA BO}
\variante{%
cee
\région{PA}}
(\domainesémantique{Marques de degré})
\classe{INTENS}
\begin{glose}
\pfra{très ; vraiment}
\end{glose}
\newline
\begin{sous-entrée}{cii êgu}{ⓔciiⓗ2ⓝcii êgu}
\région{GO}
\begin{glose}
\pfra{personne de grande taille}
\end{glose}
\newline
\begin{exemple}
\région{GO}
\textbf{\pnua{e cii êgu nai je}}
\pfra{il est plus grand qu'elle}
\end{exemple}
\newline
\begin{exemple}
\région{GO}
\textbf{\pnua{e cii gi}}
\pfra{il pleure vraiment}
\end{exemple}
\newline
\begin{exemple}
\région{PA}
\textbf{\pnua{nu cii vha}}
\pfra{je parle sérieusement}
\end{exemple}
\newline
\begin{exemple}
\région{PA}
\textbf{\pnua{i xau cii êgu !}}
\pfra{il est vraiment costaud}
\end{exemple}
\end{sous-entrée}
\end{entrée}

\begin{entrée}{ciia}{}{ⓔciia}
\formephonétique{ciːa}
\région{GOs BO PA}
\variante{%
ciiya
\région{BO}}
(\domainesémantique{Céphalopodes})
\classe{nom}
\begin{glose}
\pfra{poulpe ; pieuvre}
\end{glose}
\newline
\étymologie{
\langue{POc}
\étymon{*kuRita}}
\end{entrée}

\begin{entrée}{ciia hulò hailò hulò}{}{ⓔciia hulò hailò hulò}
\région{GOs}
(\domainesémantique{Poissons})
\classe{nom}
\begin{glose}
\pfra{seiche (lit. pieuvre sans bout)}
\end{glose}
\end{entrée}

\begin{entrée}{ciibwin}{}{ⓔciibwin}
\région{PA BO}
\variante{%
cibwi
\région{BO}}
(\domainesémantique{Mammifères})
\classe{nom}
\begin{glose}
\pfra{rat}
\end{glose}
\newline
\relationsémantique{Cf.}{\lien{}{zine [GOs]}}
\glosecourte{rat}
\newline
\étymologie{
\langue{POc}
\étymon{*ka(n)supe}}
\end{entrée}

\begin{entrée}{cii-ce}{}{ⓔcii-ce}
\région{GO PA}
\variante{%
ci-cee
\région{PA}}
(\domainesémantique{Parties de plantes})
\classe{nom}
\begin{glose}
\pfra{écorce}
\end{glose}
\newline
\étymologie{
\langue{POc}
\étymon{*kupit}
\glosecourte{bark, peelings}}
\end{entrée}

\begin{entrée}{cii-du}{}{ⓔcii-du}
\formephonétique{cɨːndu}
\région{GOs}
(\domainesémantique{Corps humain})
\classe{nom}
\begin{glose}
\pfra{colonne vertébrale}
\end{glose}
\newline
\begin{exemple}
\textbf{\pnua{cii-duu-nu}}
\pfra{ma colonne vertébrale}
\end{exemple}
\newline
\relationsémantique{Cf.}{\lien{ⓔduⓗ1}{du}}
\glosecourte{os}
\end{entrée}

\begin{entrée}{cii êgu}{}{ⓔcii êgu}
\région{GOs}
(\domainesémantique{Société})
\classe{nom}
\begin{glose}
\pfra{homme mûr ; dans la force de l'âge}
\end{glose}
\end{entrée}

\begin{entrée}{cii-hoogo}{}{ⓔcii-hoogo}
\formephonétique{cɨːhoːŋgo}
\région{PA}
(\domainesémantique{Topographie})
\classe{nom}
\begin{glose}
\pfra{crête de la montagne}
\end{glose}
\end{entrée}

\begin{entrée}{cii.i}{}{ⓔcii.i}
\formephonétique{ciːi}
\région{GOs WE PA BO}
\variante{%
chi:i
\région{BO}}
(\domainesémantique{Insectes})
\classe{nom}
\begin{glose}
\pfra{pou}
\end{glose}
\begin{glose}
\pfra{puce}
\end{glose}
\newline
\étymologie{
\langue{POc}
\étymon{*kutu}
\glosecourte{pou}}
\end{entrée}

\begin{entrée}{cii-kui}{}{ⓔcii-kui}
\région{GOs}
(\domainesémantique{Ignames})
\classe{nom}
\begin{glose}
\pfra{peau de l'igname}
\end{glose}
\end{entrée}

\begin{entrée}{cii-pòò}{}{ⓔcii-pòò}
\région{GOs}
(\domainesémantique{Parties de plantes})
\classe{nom}
\begin{glose}
\pfra{écorce de bourao (sert à la confection des jupes anciennes)}
\end{glose}
\end{entrée}

\begin{entrée}{cii-phagòò}{}{ⓔcii-phagòò}
\région{GOs PA}
(\domainesémantique{Corps humain})
\classe{nom}
\begin{glose}
\pfra{peau (humain); enveloppe corporelle}
\end{glose}
\newline
\begin{exemple}
\textbf{\pnua{cii-phagòò-je}}
\pfra{sa peau}
\end{exemple}
\newline
\begin{exemple}
\région{PA}
\textbf{\pnua{la phaade cii-phagò}}
\pfra{ils ont fait acte de présence}
\end{exemple}
\newline
\relationsémantique{Cf.}{\lien{}{cii-chova}}
\glosecourte{la peau du cheval}
\end{entrée}

\begin{entrée}{cii-vhaa}{}{ⓔcii-vhaa}
\région{GOs}
(\domainesémantique{Discours, échanges verbaux})
\classe{nom}
\begin{glose}
\pfra{vraie parole}
\end{glose}
\end{entrée}

\begin{entrée}{ciixo}{}{ⓔciixo}
\formephonétique{ciːɣo}
\région{PA}
(\domainesémantique{Objets coutumiers})
\classe{nom}
\begin{glose}
\pfra{perches plantées devant la maison (Charles)}
\end{glose}
\end{entrée}

\begin{entrée}{ciiza}{}{ⓔciiza}
\région{GOs}
\variante{%
ciilaa
\région{BO}}
(\domainesémantique{Insectes})
\classe{nom}
\begin{glose}
\pfra{pou de corps ; morpion}
\end{glose}
\end{entrée}

\begin{entrée}{ci-kãbwa}{}{ⓔci-kãbwa}
\région{GOs}
\variante{%
ci-xãbwa
\région{GO(s)}}
(\domainesémantique{Vêtements, parure})
\classe{nom}
\begin{glose}
\pfra{étoffe ; tissu ; vêtement (lit. peau du diable)}
\end{glose}
\end{entrée}

\begin{entrée}{cili}{}{ⓔcili}
\région{PA BO}
(\domainesémantique{Mouvements ou actions faits avec le corps, les bras, les mains, les pieds})
\classe{v}
\begin{glose}
\pfra{secouer}
\end{glose}
\end{entrée}

\begin{entrée}{ciluu}{}{ⓔciluu}
\région{BO}
(\domainesémantique{Verbes de mouvement})
\classe{v}
\begin{glose}
\pfra{incliner (s') [BM]}
\end{glose}
\end{entrée}

\begin{entrée}{ci-mee}{}{ⓔci-mee}
\région{GOs}
(\domainesémantique{Corps humain})
\classe{nom}
\begin{glose}
\pfra{paupière}
\end{glose}
\end{entrée}

\begin{entrée}{cimic}{}{ⓔcimic}
\région{PA}
(\domainesémantique{Vêtements, parure})
\classe{nom}
\begin{glose}
\pfra{chemise}
\end{glose}
\newline
\emprunt{chemise (FR)}
\end{entrée}

\begin{entrée}{cimwî}{}{ⓔcimwî}
\région{GOs PA}
\variante{%
khimwi
\région{BO [BM]}}
(\domainesémantique{Mouvements ou actions faits avec le corps, les bras, les mains, les pieds})
\classe{v}
\begin{glose}
\pfra{tenir ferme ; saisir}
\end{glose}
\newline
\begin{sous-entrée}{te-yimwî}{ⓔcimwîⓝte-yimwî}
\région{GO}
\begin{glose}
\pfra{attraper, saisir}
\end{glose}
\end{sous-entrée}
\newline
\étymologie{
\langue{POc}
\étymon{*kumi}}
\end{entrée}

\begin{entrée}{cin}{}{ⓔcin}
\région{BO}
(\domainesémantique{Arbre})
\classe{nom}
\newline
\sens{1}
\begin{glose}
\pfra{arbre à pain}
\end{glose}
\nomscientifique{Artocarpus altilis (Park.) Fosb.}
\newline
\sens{2}
\begin{glose}
\pfra{papayer}
\end{glose}
\nomscientifique{Carica papaya L.}
\newline
\begin{exemple}
\textbf{\pnua{po cin}}
\pfra{fruit de l'arbre à pain}
\end{exemple}
\newline
\étymologie{
\langue{POc}
\étymon{*kulu(R)}}
\end{entrée}

\begin{entrée}{cińevwö}{}{ⓔcińevwö}
\formephonétique{cineβω}
\région{GOs BO}
(\domainesémantique{Caractéristiques et propriétés des personnes})
\classe{v.stat.}
\begin{glose}
\pfra{important}
\end{glose}
\newline
\begin{exemple}
\région{BO}
\textbf{\pnua{ci nevo}}
\pfra{très important [BM]}
\end{exemple}
\end{entrée}

\begin{entrée}{ci-nu}{}{ⓔci-nu}
\formephonétique{cɨ-ɳu}
\région{GOs}
(\domainesémantique{Cocotiers})
\classe{nom}
\begin{glose}
\pfra{bourre de coco}
\end{glose}
\end{entrée}

\begin{entrée}{ci pojo}{}{ⓔci pojo}
\région{BO}
(\domainesémantique{Insectes})
\classe{nom}
\begin{glose}
\pfra{punaise (qui sent mauvais) (Corne)}
\end{glose}
\newline
\note{non vérifié}{général}{}
\end{entrée}

\begin{entrée}{ci-phai}{}{ⓔci-phai}
\région{GOs}
\variante{%
cin-phai
\région{PA BO}}
(\domainesémantique{Arbre})
\classe{nom}
\begin{glose}
\pfra{arbre à pain (lit. arbre à bouillir)}
\end{glose}
\nomscientifique{Artocarpus altilis (Park.) Fosb.}
\end{entrée}

\begin{entrée}{ci-phwa}{}{ⓔci-phwa}
\région{GOs}
\variante{%
ci-phwa-n
\région{BO PA}}
(\domainesémantique{Corps humain})
\classe{nom}
\begin{glose}
\pfra{lèvres}
\end{glose}
\end{entrée}

\begin{entrée}{cirara}{}{ⓔcirara}
\formephonétique{ciɽaɽacirara}
\région{GOs}
\variante{%
citratra
\région{GO(s)}}
(\domainesémantique{Description des objets, formes, consistance, taille})
\classe{v.stat.}
\begin{glose}
\pfra{fin ; mince}
\end{glose}
\end{entrée}

\begin{entrée}{ciròvwe}{}{ⓔciròvwe}
\formephonétique{cirɔβe}
\région{GOs}
\variante{%
ciròvhe
\région{PA BO}}
(\domainesémantique{Verbes d'action (en général)})
\classe{v}
\begin{glose}
\pfra{retourner (chemise, etc.) ; mettre à l'envers (linge)}
\end{glose}
\end{entrée}

\begin{entrée}{citèèn}{}{ⓔcitèèn}
\région{BO}
(\domainesémantique{Danses})
\classe{nom}
\begin{glose}
\pfra{danse des morts (effectuée par les hommes) (Corne)}
\end{glose}
\newline
\note{non vérifié}{général}{}
\end{entrée}

\begin{entrée}{civwi}{}{ⓔcivwi}
\formephonétique{ciβi}
\région{GOs}
(\domainesémantique{Fonctions naturelles humaines})
\classe{v}
\begin{glose}
\pfra{toucher (un bobo, une blessure)}
\end{glose}
\end{entrée}

\begin{entrée}{cixè}{}{ⓔcixè}
\région{GOs}
(\domainesémantique{Société})
\classe{nom}
\begin{glose}
\pfra{même génération ; même tranche d'âge}
\end{glose}
\newline
\begin{exemple}
\région{GO}
\textbf{\pnua{pe-cixèè-la}}
\pfra{ils sont d'une même tranche d'âge, d'une même génération}
\end{exemple}
\newline
\begin{exemple}
\région{PA}
\textbf{\pnua{pe-poxe cixèè-li}}
\pfra{ils sont d'une même tranche d'âge, d'une même génération}
\end{exemple}
\newline
\begin{exemple}
\région{GO}
\textbf{\pnua{cixè ẽnõ}}
\pfra{enfants d'une même tranche d'âge, d'une même génération}
\end{exemple}
\newline
\begin{exemple}
\région{GO}
\textbf{\pnua{cixè whamã}}
\pfra{vieilles personnes d'une même tranche d'âge, d'une même génération}
\end{exemple}
\end{entrée}

\begin{entrée}{cò}{}{ⓔcò}
\région{GOs BO}
(\domainesémantique{Corps humain})
\classe{nom}
\begin{glose}
\pfra{pénis (grossier) ; sexe (de l'homme)}
\end{glose}
\newline
\begin{exemple}
\région{GO}
\textbf{\pnua{wee co-je}}
\pfra{son sperme}
\end{exemple}
\newline
\begin{exemple}
\région{BO}
\textbf{\pnua{còò-n}}
\pfra{son sexe, pénis}
\end{exemple}
\newline
\relationsémantique{Cf.}{\lien{}{phi, pi}}
\glosecourte{testicules}
\newline
\étymologie{
\langue{POc}
\étymon{*kau, *kayu}
\glosecourte{'penis'}}
\end{entrée}

\begin{entrée}{cobo}{}{ⓔcobo}
\région{GOs}
(\domainesémantique{Mouvements ou actions faits avec le corps, les bras, les mains, les pieds})
\classe{v}
\begin{glose}
\pfra{tapoter}
\end{glose}
\end{entrée}

\begin{entrée}{coboe}{}{ⓔcoboe}
\région{GOs}
\variante{%
cobwoi
\région{BO (Corne, BM)}}
(\domainesémantique{Mouvements ou actions faits avec le corps, les bras, les mains, les pieds})
\classe{v}
\begin{glose}
\pfra{pincer entre les doigts}
\end{glose}
\end{entrée}

\begin{entrée}{cocovwa}{}{ⓔcocovwa}
\formephonétique{coʒoβa}
\région{GOs}
\variante{%
cocopa
\région{GO(s)}, 
cocova
\région{BO}}
\classe{COLL ; QNT}
\newline
\sens{1}
(\domainesémantique{Quantificateurs})
\begin{glose}
\pfra{tous ; chaque}
\end{glose}
\begin{glose}
\pfra{ensemble}
\end{glose}
\newline
\begin{sous-entrée}{coçovwa ègu}{ⓔcocovwaⓢ1ⓝcoçovwa ègu}
\begin{glose}
\pfra{les hommes ensemble}
\end{glose}
\end{sous-entrée}
\newline
\begin{sous-entrée}{coçovwa tree}{ⓔcocovwaⓢ1ⓝcoçovwa tree}
\begin{glose}
\pfra{tous les jours}
\end{glose}
\end{sous-entrée}
\newline
\sens{2}
(\domainesémantique{Quantificateurs})
\begin{glose}
\pfra{tout ; toutes sortes de}
\end{glose}
\newline
\begin{exemple}
\textbf{\pnua{haivwö la coçovwa pwaixe xa lò yue}}
\pfra{il y a beaucoup de sortes d'animaux qu'ils élèvent (lit. choses)}
\end{exemple}
\end{entrée}

\begin{entrée}{cö, çö, yö}{}{ⓔcö, çö, yö}
\formephonétique{ʒω}
\région{GOs}
\variante{%
co, yo
\région{PA}, 
cu, yu
\région{PABO}}
(\domainesémantique{Pronoms})
\classe{PRO 2° pers. SG (sujet ou OBJ)}
\begin{glose}
\pfra{tu ; te}
\end{glose}
\end{entrée}

\begin{entrée}{cò-chaamwa}{}{ⓔcò-chaamwa}
\formephonétique{cɔ-cʰaːmwa}
\région{GOs}
(\domainesémantique{Bananiers et bananes
, Parties de plantes})
\classe{nom}
\begin{glose}
\pfra{bout de l'inflorescence de bananier}
\end{glose}
\end{entrée}

\begin{entrée}{cö-da}{}{ⓔcö-da}
\formephonétique{cωnda}
\région{GOs WEM}
\variante{%
cu-da
\région{PA}}
(\domainesémantique{Verbes de mouvement})
\classe{v}
\begin{glose}
\pfra{monter}
\end{glose}
\begin{glose}
\pfra{grimper}
\end{glose}
\begin{glose}
\pfra{sauter vers le haut}
\end{glose}
\newline
\begin{exemple}
\région{PA}
\textbf{\pnua{nu cu-da bwa mol}}
\pfra{je débarque sur la terre ferme, je sors de l'eau}
\end{exemple}
\newline
\begin{exemple}
\région{GO}
\textbf{\pnua{nu cö-da bwa loto}}
\pfra{je monte dans la voiture}
\end{exemple}
\newline
\begin{exemple}
\région{GO}
\textbf{\pnua{cö-da}}
\pfra{monter au cocotier en se poussant des pieds}
\end{exemple}
\newline
\begin{exemple}
\région{GO}
\textbf{\pnua{cö-du}}
\pfra{sauter à bas}
\end{exemple}
\end{entrée}

\begin{entrée}{cö-du}{}{ⓔcö-du}
(\domainesémantique{Verbes de mouvement})
\classe{v}
\begin{glose}
\pfra{sauter en bas}
\end{glose}
\begin{glose}
\pfra{plonger}
\end{glose}
\newline
\begin{exemple}
\région{GOs WEM}
\région{PA}
\textbf{\pnua{nu cu-du ni we}}
\pfra{je saute à l'eau}
\end{exemple}
\newline
\relationsémantique{Cf.}{\lien{ⓔu-du}{u-du}}
\glosecourte{entrer (dans une maison)}
\end{entrée}

\begin{entrée}{cö-e}{}{ⓔcö-e}
\formephonétique{cωe}
\région{GOs BO}
\variante{%
cu-e
\région{BO}}
(\domainesémantique{Verbes de déplacement et moyens de déplacement})
\classe{v}
\begin{glose}
\pfra{traverser (rivière)}
\end{glose}
\newline
\begin{exemple}
\région{GO}
\textbf{\pnua{e cö-e nõgò}}
\pfra{il traverse la rivière}
\end{exemple}
\end{entrée}

\begin{entrée}{côî}{}{ⓔcôî}
\formephonétique{côî}
\région{GOs}
\variante{%
chôî
\région{BO (Corne)}}
(\domainesémantique{Mouvements ou actions faits avec le corps, les bras, les mains, les pieds})
\classe{v.t.}
\begin{glose}
\pfra{suspendre}
\end{glose}
\begin{glose}
\pfra{prendre au collet}
\end{glose}
\newline
\begin{exemple}
\région{GO}
\textbf{\pnua{e côî pu-noo-je}}
\pfra{il l'attrape par le collet}
\end{exemple}
\newline
\begin{exemple}
\région{GO}
\textbf{\pnua{nu kha-côî kee-nu}}
\pfra{je marche avec mon sac au bras ou à la main}
\end{exemple}
\end{entrée}

\begin{entrée}{cöi}{}{ⓔcöi}
\formephonétique{cωi}
\région{GOs PA BO}
(\domainesémantique{Mouvements ou actions faits avec le corps, les bras, les mains, les pieds})
\classe{v}
\begin{glose}
\pfra{creuser (trou)}
\end{glose}
\begin{glose}
\pfra{gratter (terre)}
\end{glose}
\begin{glose}
\pfra{creuser (pour contrôler l'état des tubercules)}
\end{glose}
\newline
\begin{exemple}
\région{GO}
\textbf{\pnua{e cöi phwe pwaji}}
\pfra{elle creuse le trou du crabe}
\end{exemple}
\newline
\begin{exemple}
\région{GO}
\textbf{\pnua{e cöi dili}}
\pfra{elle creuse la terre}
\end{exemple}
\newline
\étymologie{
\langue{POc}
\étymon{*k(a/e)li}}
\end{entrée}

\begin{entrée}{cò mhwedin}{}{ⓔcò mhwedin}
\région{BO}
(\domainesémantique{Corps humain})
\classe{nom}
\begin{glose}
\pfra{partie du nez entre les deux narines [Corne]}
\end{glose}
\newline
\note{non vérifié}{général}{}
\end{entrée}

\begin{entrée}{cöńi}{}{ⓔcöńi}
\formephonétique{cωni}
\région{GOs BO PA}
\variante{%
cööni
\région{PA}}
\classe{v}
\newline
\sens{1}
(\domainesémantique{Santé, maladie})
\begin{glose}
\pfra{souffrir}
\end{glose}
\begin{glose}
\pfra{endurer}
\end{glose}
\newline
\sens{2}
(\domainesémantique{Sentiments})
\begin{glose}
\pfra{affligé}
\end{glose}
\begin{glose}
\pfra{retenir ses larmes (enfant)}
\end{glose}
\begin{glose}
\pfra{retenir son souffle [PA]}
\end{glose}
\newline
\sens{3}
(\domainesémantique{Société})
\begin{glose}
\pfra{pleurer un mort}
\end{glose}
\begin{glose}
\pfra{deuil (être en)}
\end{glose}
\newline
\begin{exemple}
\région{BO}
\textbf{\pnua{i coni}}
\pfra{il est en deuil (Corne)}
\end{exemple}
\newline
\relationsémantique{Cf.}{\lien{}{mõõdim [WEM WE BO PA]}}
\glosecourte{deuil}
\newline
\relationsémantique{Cf.}{\lien{ⓔgiul}{giul}}
\glosecourte{pleurer un mort (pour les hommes)}
\end{entrée}

\begin{entrée}{cöńi-vwo}{}{ⓔcöńi-vwo}
\région{GOs}
(\domainesémantique{Santé, maladie})
\classe{nom}
\begin{glose}
\pfra{souffrance}
\end{glose}
\end{entrée}

\begin{entrée}{côô}{}{ⓔcôô}
\formephonétique{cõː}
\région{GOs}
\variante{%
cô
\région{BO (BM]}}
\classe{v.i.}
\newline
\sens{1}
(\domainesémantique{Préfixes et verbes de position})
\begin{glose}
\pfra{suspendu}
\end{glose}
\begin{glose}
\pfra{accroché}
\end{glose}
\newline
\begin{exemple}
\région{GO}
\textbf{\pnua{e kô-côô}}
\pfra{elle dort suspendue (roussette)}
\end{exemple}
\newline
\begin{sous-entrée}{pa-côôe}{ⓔcôôⓢ1ⓝpa-côôe}
\région{GO}
\begin{glose}
\pfra{suspendre, accrocher qqch.}
\end{glose}
\end{sous-entrée}
\newline
\begin{sous-entrée}{pa-yôôi}{ⓔcôôⓢ1ⓝpa-yôôi}
\région{BO}
\begin{glose}
\pfra{suspendre, accrocher qqch.}
\end{glose}
\end{sous-entrée}
\newline
\sens{2}
(\domainesémantique{Mouvements ou actions faits avec le corps, les bras, les mains, les pieds})
\begin{glose}
\pfra{soulever (un objet léger)}
\end{glose}
\newline
\begin{exemple}
\région{GO}
\textbf{\pnua{côô-e-da !}}
\pfra{soulève-le !}
\end{exemple}
\newline
\sens{3}
(\domainesémantique{Navigation})
\begin{glose}
\pfra{accoster (bateau)}
\end{glose}
\newline
\begin{exemple}
\région{GO}
\textbf{\pnua{e côô wô}}
\pfra{le bateau a accosté}
\end{exemple}
\end{entrée}

\begin{entrée}{cöö}{}{ⓔcöö}
\formephonétique{cωː}
\région{GOs}
\variante{%
còòl
\région{PA BO}, 
cul, cu
\région{PA}}
(\domainesémantique{Verbes de mouvement})
\classe{v}
\begin{glose}
\pfra{sauter}
\end{glose}
\begin{glose}
\pfra{débarquer}
\end{glose}
\begin{glose}
\pfra{traverser}
\end{glose}
\newline
\begin{exemple}
\région{GO}
\textbf{\pnua{novwo ã uvilu je cö-ò cö-mi na bwa ã-da yòò}}
\pfra{mais cet oiseau, il sautille de-ci de-là sur ce bois de fer}
\end{exemple}
\end{entrée}

\begin{entrée}{cö-ò cö-mi}{}{ⓔcö-ò cö-mi}
\formephonétique{cω-ɔcω-ɔmi}
\région{GOs}
\variante{%
cu-ò cu-mi
\région{BO}}
(\domainesémantique{Verbes de mouvement})
\classe{v}
\begin{glose}
\pfra{sauter de-ci de-là}
\end{glose}
\newline
\begin{exemple}
\région{BO}
\textbf{\pnua{i pe-cu-ò cu-mi}}
\pfra{il saute de-ci de-là}
\end{exemple}
\end{entrée}

\begin{entrée}{cooge}{}{ⓔcooge}
\région{GOs BO}
\variante{%
coge
\région{BO}}
(\domainesémantique{Mouvements ou actions faits avec le corps, les bras, les mains, les pieds})
\classe{v}
\begin{glose}
\pfra{lever ; soulever (des choses lourdes) ; élever}
\end{glose}
\begin{glose}
\pfra{ramasser}
\end{glose}
\begin{glose}
\pfra{soutenir (avec la paume de la main)}
\end{glose}
\newline
\begin{sous-entrée}{coge-da}{ⓔcoogeⓝcoge-da}
\begin{glose}
\pfra{soulever en l'air}
\end{glose}
\end{sous-entrée}
\end{entrée}

\begin{entrée}{côôni}{}{ⓔcôôni}
\région{PA}
(\domainesémantique{Mouvements ou actions faits avec le corps, les bras, les mains, les pieds})
\classe{v}
\begin{glose}
\pfra{toucher à qqch}
\end{glose}
\newline
\begin{exemple}
\textbf{\pnua{i côôni u ri ?}}
\pfra{qui l'a touché ? (et abimé)}
\end{exemple}
\newline
\begin{exemple}
\textbf{\pnua{kebwa côôni hèlè}}
\pfra{touche pas au couteau}
\end{exemple}
\end{entrée}

\begin{entrée}{cou}{}{ⓔcou}
\formephonétique{co.u}
\région{GOs PA WEM WE}
\classe{v}
\newline
\sens{1}
(\domainesémantique{Aliments, alimentation})
\begin{glose}
\pfra{dur (taro, igname et manioc)}
\end{glose}
\begin{glose}
\pfra{immangeable (taro et manioc, quand c'est mal cuit)}
\end{glose}
\newline
\sens{2}
(\domainesémantique{Fonctions intellectuelles})
\begin{glose}
\pfra{zozoter (sens figuré)}
\end{glose}
\newline
\begin{exemple}
\région{GO}
\textbf{\pnua{e cou phwa-je}}
\pfra{il zozote}
\end{exemple}
\end{entrée}

\begin{entrée}{coxada}{}{ⓔcoxada}
\région{PA}
(\domainesémantique{Navigation})
\classe{v}
\begin{glose}
\pfra{accoster}
\end{glose}
\newline
\begin{exemple}
\région{PA}
\textbf{\pnua{u coxada je wony}}
\pfra{ce bateau accoste}
\end{exemple}
\end{entrée}

\begin{entrée}{còxe}{}{ⓔcòxe}
\formephonétique{cɔɣe}
\région{GOs BO PA}
\variante{%
còge
\formephonétique{cɔːŋge}
\région{BO PA vx}}
(\domainesémantique{Soins du corps})
\classe{v}
\begin{glose}
\pfra{couper (par ex. cheveux, barbe,etc.avec des ciseaux) ;}
\end{glose}
\begin{glose}
\pfra{tailler (barbe) ;}
\end{glose}
\newline
\begin{exemple}
\région{PA}
\textbf{\pnua{i còòxe pi-hi-n}}
\pfra{il se coupe les ongles}
\end{exemple}
\newline
\begin{exemple}
\région{PA}
\textbf{\pnua{i còòxe pu-bwaa-n}}
\pfra{il s'est coupé les cheveux}
\end{exemple}
\newline
\begin{exemple}
\région{PA}
\textbf{\pnua{nu pe-còòxe pu}}
\pfra{je me rase les poils}
\end{exemple}
\newline
\begin{exemple}
\région{PA}
\textbf{\pnua{li pe-còòxe}}
\pfra{ils sont occupés à se couper les cheveux (pas forcément réciproque; 2 personnes impliquées dans le processus dont l'un est celui qui coupe)}
\end{exemple}
\newline
\begin{exemple}
\textbf{\pnua{ã-du mwa còòxe}}
\pfra{(ils) entrent le frapper (couper)}
\end{exemple}
\newline
\étymologie{
\langue{POc}
\étymon{*koti}}
\end{entrée}

\begin{entrée}{cò, yò}{}{ⓔcò, yò}
\région{GO PA}
(\domainesémantique{Pronoms})
\classe{PRO 2° pers. duel (sujet)}
\begin{glose}
\pfra{vous 2}
\end{glose}
\end{entrée}

\begin{entrée}{cu}{}{ⓔcu}
\région{BO}
\variante{%
cuu
}
(\domainesémantique{Marques de degré})
\classe{QNT}
\begin{glose}
\pfra{trop (Corne)}
\end{glose}
\newline
\begin{exemple}
\région{BO}
\textbf{\pnua{u mha cu mwa la peena e phe}}
\pfra{il a trop pris d'anguilles}
\end{exemple}
\end{entrée}

\begin{entrée}{cubii}{}{ⓔcubii}
\région{PA}
(\domainesémantique{Mouvements ou actions avec la tête, les yeux, la bouche})
\classe{v}
\begin{glose}
\pfra{déchirer avec les dents}
\end{glose}
\newline
\relationsémantique{Cf.}{\lien{ⓔkaobi}{kaobi}}
\glosecourte{casser avec les dents}
\end{entrée}

\begin{entrée}{cubu}{}{ⓔcubu}
\région{GOs}
(\domainesémantique{Actions liées aux éléments (liquide, fumée)})
\classe{v}
\begin{glose}
\pfra{déborder}
\end{glose}
\newline
\begin{exemple}
\textbf{\pnua{e cubu we}}
\pfra{l'eau déborde}
\end{exemple}
\end{entrée}

\begin{entrée}{cuk}{}{ⓔcuk}
\région{PA BO}
(\domainesémantique{Aliments, alimentation})
\classe{nom}
\begin{glose}
\pfra{sucre}
\end{glose}
\newline
\emprunt{sucre (FR)}
\end{entrée}

\begin{entrée}{cuka}{}{ⓔcuka}
\région{BO [BM, Corne]}
(\domainesémantique{Aliments, alimentation})
\classe{nom}
\begin{glose}
\pfra{pomme [BM]}
\end{glose}
\begin{glose}
\pfra{banane sucre (Corne)}
\end{glose}
\newline
\emprunt{sugar (GB)}
\newline
\note{non vérifié}{général}{}
\end{entrée}

\begin{entrée}{cul (a) kao}{}{ⓔcul (a) kao}
\région{PA}
\variante{%
col a kao
\région{BO}}
(\domainesémantique{Actions liées aux éléments (liquide, fumée)})
\classe{v}
\begin{glose}
\pfra{déborder (lit. se lever inondation)}
\end{glose}
\end{entrée}

\begin{entrée}{cura}{}{ⓔcura}
\région{BO}
(\domainesémantique{Habitat})
\classe{v}
\begin{glose}
\pfra{habiter (littéraire)[BM]}
\end{glose}
\newline
\begin{exemple}
\textbf{\pnua{li a cura bwa ènè-da Phaja}}
\pfra{ils habitent à cet endroit en haut à Phaja}
\end{exemple}
\end{entrée}

\begin{entrée}{cuxi}{}{ⓔcuxi}
\formephonétique{cuɣi}
\région{PA BO}
\variante{%
cuki
\région{GO(s)}, 
cugi
\région{BO}}
\classe{v.stat. ; n}
\newline
\sens{1}
(\domainesémantique{Santé, maladie})
\begin{glose}
\pfra{fort ; vigoureux ; résistant}
\end{glose}
\newline
\sens{2}
(\domainesémantique{Caractéristiques et propriétés des personnes})
\begin{glose}
\pfra{courage ; courageux}
\end{glose}
\end{entrée}

\newpage

\lettrine{ch}\begin{entrée}{cha}{}{ⓔcha}
\formephonétique{cʰa}
\région{PA}
(\domainesémantique{Description des objets, formes, consistance, taille})
\classe{v}
\begin{glose}
\pfra{aiguisé; coupant}
\end{glose}
\newline
\begin{exemple}
\textbf{\pnua{nooli cha hèlè !}}
\pfra{regarde les traces de machette (dans la brousse)}
\end{exemple}
\end{entrée}

\begin{entrée}{chaa}{}{ⓔchaa}
\formephonétique{cʰaː}
\région{GOs}
(\domainesémantique{Mer : topographie})
\classe{nom}
\begin{glose}
\pfra{récif}
\end{glose}
\newline
\étymologie{
\langue{POc}
\étymon{*sakaRu}}
\end{entrée}

\begin{entrée}{chaaçee}{}{ⓔchaaçee}
\formephonétique{cʰaːʒe}
\région{GOs PA}
(\domainesémantique{Préfixes et verbes de position})
\classe{ADV}
\begin{glose}
\pfra{travers (de) ; penché sur uncôté}
\end{glose}
\newline
\begin{exemple}
\textbf{\pnua{e no-chaaçee}}
\pfra{elle regarde la tête penchée sur le côté}
\end{exemple}
\newline
\begin{exemple}
\textbf{\pnua{ku-chaaçee nye ce mwa}}
\pfra{l'arbre est penchée sur le côté}
\end{exemple}
\newline
\begin{exemple}
\textbf{\pnua{kô-chaaçee loto}}
\pfra{la voiture est de travers (penchée sur le bas-côté)}
\end{exemple}
\end{entrée}

\begin{entrée}{chaamwa}{}{ⓔchaamwa}
\formephonétique{cʰaːmwa}
\région{GOs PA BO WE}
(\domainesémantique{Bananiers et bananes
, Fruits})
\classe{nom}
\begin{glose}
\pfra{banane (générique) ; bananier}
\end{glose}
\nomscientifique{Musa sp.(Musacées)}
\newline
\begin{sous-entrée}{cò-chaamwa}{ⓔchaamwaⓝcò-chaamwa}
\région{GO}
\begin{glose}
\pfra{bout de l'inflorescence de bananier}
\end{glose}
\end{sous-entrée}
\newline
\begin{sous-entrée}{zò-chaamwa}{ⓔchaamwaⓝzò-chaamwa}
\région{GO}
\begin{glose}
\pfra{rejet de bananier}
\end{glose}
\end{sous-entrée}
\newline
\begin{sous-entrée}{tò-chaamwa}{ⓔchaamwaⓝtò-chaamwa}
\région{GO}
\begin{glose}
\pfra{régime de banane}
\end{glose}
\end{sous-entrée}
\end{entrée}

\begin{entrée}{chaamwa we-ê}{}{ⓔchaamwa we-ê}
\région{GOs}
(\domainesémantique{Bananiers et bananes})
\classe{nom}
\begin{glose}
\pfra{banane sucrée (petite)}
\end{glose}
\end{entrée}

\begin{entrée}{chaavwa-dili}{}{ⓔchaavwa-dili}
\formephonétique{cʰaːβa}
\région{GOs}
\variante{%
cava dili
\région{BO}}
(\domainesémantique{Terre})
\classe{v.stat.}
\begin{glose}
\pfra{boue (lit. terre molle)}
\end{glose}
\newline
\begin{exemple}
\région{BO}
\textbf{\pnua{we cava dili}}
\pfra{eau boueuse}
\end{exemple}
\newline
\begin{exemple}
\région{BO}
\textbf{\pnua{tomènô ni cavaan dili}}
\pfra{marcher dans la boue}
\end{exemple}
\end{entrée}

\begin{entrée}{chan}{}{ⓔchan}
\région{BO}
(\domainesémantique{Discours, échanges verbaux})
\classe{v}
\begin{glose}
\pfra{bégayer ; bredouiller [BM]}
\end{glose}
\newline
\note{non vérifié}{général}{}
\end{entrée}

\begin{entrée}{chãnã}{1}{ⓔchãnãⓗ1}
\formephonétique{cʰɛ̃ɳɛ̃}
\formephonétique{cʰɛ̃nɛ̃}
\région{GOs WE}
\variante{%
chãnã
\région{PA BO}}
\newline
\sens{1}
(\domainesémantique{Fonctions naturelles humaines})
\classe{v ; n}
\begin{glose}
\pfra{respirer}
\end{glose}
\begin{glose}
\pfra{souffle}
\end{glose}
\begin{glose}
\pfra{respiration}
\end{glose}
\newline
\begin{exemple}
\région{PA}
\textbf{\pnua{kixa chãnã-n}}
\pfra{il est essouflé, à bout de souffle}
\end{exemple}
\newline
\begin{exemple}
\région{PA}
\textbf{\pnua{chãnãã-n}}
\pfra{son souffle}
\end{exemple}
\newline
\sens{2}
(\domainesémantique{Fonctions naturelles humaines})
\classe{v ; n}
\begin{glose}
\pfra{reposer (se) ; repos}
\end{glose}
\end{entrée}

\begin{entrée}{chãnã}{2}{ⓔchãnãⓗ2}
\formephonétique{chɛ̃ɳɛ̃}
\région{GOs}
(\domainesémantique{Aspect})
\classe{ASP persistif}
\begin{glose}
\pfra{sans cesse ; constamment ; toujours ; à répétition ; sans arrêt ; tout le temps}
\end{glose}
\newline
\begin{exemple}
\textbf{\pnua{e gi chãnã ẽnõ-ã}}
\pfra{cet enfant-làpleure sans cesse}
\end{exemple}
\newline
\begin{exemple}
\textbf{\pnua{e vhaa chãnã}}
\pfra{il parle sans cesse, il s'obstine à parler}
\end{exemple}
\newline
\begin{exemple}
\textbf{\pnua{e zòò chãnã xa za phee-je du xo we}}
\pfra{il nage tout le temps mais le courant l'emporte}
\end{exemple}
\newline
\begin{exemple}
\textbf{\pnua{e zòò da xa za phee-je du chãnã xo we}}
\pfra{il nage vers la rive, mais le courant l'emporte tout le temps vers le large}
\end{exemple}
\newline
\begin{exemple}
\textbf{\pnua{e cani chãnã kuru}}
\pfra{il ne mange que du taro}
\end{exemple}
\newline
\begin{exemple}
\textbf{\pnua{mo phweexu chãnã}}
\pfra{nous bavardons sans arrêt}
\end{exemple}
\newline
\relationsémantique{Cf.}{\lien{ⓔhaaⓗ1}{haa}}
\glosecourte{PA}
\end{entrée}

\begin{entrée}{chãnã waa}{}{ⓔchãnã waa}
\région{PA}
(\domainesémantique{Fonctions naturelles humaines})
\classe{v}
\begin{glose}
\pfra{respirer la bouche ouverte}
\end{glose}
\begin{glose}
\pfra{haleter}
\end{glose}
\end{entrée}

\begin{entrée}{chavwi}{}{ⓔchavwi}
\formephonétique{cʰaβi}
\région{GOs}
\variante{%
cabwi
\région{PA}}
(\domainesémantique{Aliments, alimentation})
\classe{nom}
\begin{glose}
\pfra{disette ; famine}
\end{glose}
\end{entrée}

\begin{entrée}{chavwo}{}{ⓔchavwo}
\formephonétique{cʰaβo}
\région{GOs}
\variante{%
chapo
\région{GO(s) vx}, 
cavo
\région{PA}, 
caavu
\région{BO (Corne)}}
(\domainesémantique{Soins du corps})
\classe{v ; n}
\begin{glose}
\pfra{savon}
\end{glose}
\newline
\begin{exemple}
\région{GO}
\textbf{\pnua{e chavwoo-ni mee-je xo chavwo}}
\pfra{elle se lave le visage avec du savon}
\end{exemple}
\newline
\begin{exemple}
\région{GO}
\textbf{\pnua{e chavwoo mee xo Kavwo}}
\pfra{K se lave le visage}
\end{exemple}
\newline
\begin{exemple}
\région{PA}
\textbf{\pnua{i cavo-ni mee-n u Kavwo}}
\pfra{K se lave le visage (ou) K lui lave le visage}
\end{exemple}
\newline
\begin{exemple}
\région{PA}
\textbf{\pnua{i cavo-ni mee Kavwo}}
\pfra{K se lave le visage (ou) elle lave le visage de Kaawo}
\end{exemple}
\newline
\begin{exemple}
\région{PA}
\textbf{\pnua{i cavo-ni mee-n}}
\pfra{elle se lave le visage (seule interprétation)}
\end{exemple}
\begin{glose}
\pfra{laver(avec du savon)}
\end{glose}
\newline
\relationsémantique{Cf.}{\lien{}{jaamwe}}
\glosecourte{laver}
\newline
\emprunt{savon (FR)}
\newline
\note{e chavwo-ni hõbwò}{grammaire}{elle lave les vêtements}
\end{entrée}

\begin{entrée}{chawe}{}{ⓔchawe}
\formephonétique{cʰawe}
\région{GOs}
(\domainesémantique{Préparation des aliments; modes de préparation et de cuisson})
\classe{v}
\begin{glose}
\pfra{faire du bouillon, de la soupe}
\end{glose}
\end{entrée}

\begin{entrée}{chaxe}{}{ⓔchaxe}
\formephonétique{cʰaɣe}
\région{PA BO [BM]}
(\domainesémantique{Mouvements ou actions faits avec le corps, les bras, les mains, les pieds})
\classe{v}
\begin{glose}
\pfra{étirer (s')}
\end{glose}
\end{entrée}

\begin{entrée}{chèèvwe}{}{ⓔchèèvwe}
\formephonétique{cʰɛːβe}
\région{GOs}
\variante{%
chèèbwe
, 
cewe
\région{BO}}
(\domainesémantique{Bananiers et bananes})
\classe{nom}
\begin{glose}
\pfra{banane (à peau grise, elle a la forme d'une pirogue, sert à la préparation de "wô" 'bateau')}
\end{glose}
\end{entrée}

\begin{entrée}{chele}{}{ⓔchele}
\formephonétique{cʰele}
\région{GOs PA}
(\domainesémantique{Mouvements ou actions faits avec le corps, les bras, les mains, les pieds})
\classe{v}
\begin{glose}
\pfra{toucher}
\end{glose}
\begin{glose}
\pfra{effleurer (qqch) ; frôler}
\end{glose}
\end{entrée}

\begin{entrée}{chèńi}{}{ⓔchèńi}
\formephonétique{'cʰɛni}
\région{GOs}
(\domainesémantique{Poissons})
\classe{nom}
\begin{glose}
\pfra{mulet de rivière}
\end{glose}
\end{entrée}

\begin{entrée}{chiçô}{}{ⓔchiçô}
\formephonétique{cʰiʒõ}
\région{GOs}
\variante{%
cicô
\région{PA}}
(\domainesémantique{Instruments})
\classe{nom}
\begin{glose}
\pfra{ciseaux}
\end{glose}
\newline
\emprunt{ciseaux (FR)}
\end{entrée}

\begin{entrée}{chińõ}{}{ⓔchińõ}
\formephonétique{cʰinɔ̃}
\région{GOs PA}
\variante{%
cinõ
\région{BO}}
\newline
\groupe{A}
(\domainesémantique{Description des objets, formes, consistance, taille})
\classe{nom}
\begin{glose}
\pfra{grosseur ; épaisseur}
\end{glose}
\begin{glose}
\pfra{taille ; circonférence}
\end{glose}
\newline
\begin{exemple}
\région{BO}
\textbf{\pnua{cinõ-n}}
\pfra{sa taille}
\end{exemple}
\newline
\begin{exemple}
\textbf{\pnua{cinõ-we}}
\pfra{le lit du fleuve}
\end{exemple}
\newline
\begin{exemple}
\textbf{\pnua{e whaya chinõ ?}}
\pfra{quelle taille, grandeur ?}
\end{exemple}
\newline
\begin{exemple}
\région{PA}
\textbf{\pnua{e whaya chinõ-n ?}}
\pfra{quelle taille, grandeur ?}
\end{exemple}
\newline
\begin{exemple}
\textbf{\pnua{pe chinõ je we}}
\pfra{elle faisait/ prenait toute la taille du trou d'eau (une anguille)}
\end{exemple}
\newline
\begin{exemple}
\textbf{\pnua{pe-poxe chinõ-li}}
\pfra{ils sont de même taille (en largeur)}
\end{exemple}
\newline
\relationsémantique{Ant.}{\lien{}{pònõ}}
\glosecourte{petit}
\newline
\groupe{B}
(\domainesémantique{Description des objets, formes, consistance, taille})
\classe{v}
\begin{glose}
\pfra{occuper tout l'espace}
\end{glose}
\newline
\groupe{C}
(\domainesémantique{Quantificateurs})
\begin{glose}
\pfra{tous}
\end{glose}
\newline
\begin{exemple}
\région{PA}
\textbf{\pnua{chinõ-mwa}}
\pfra{la totalité de la maison}
\end{exemple}
\newline
\begin{exemple}
\région{PA GO}
\textbf{\pnua{chinõ êgu}}
\pfra{tous les gens}
\end{exemple}
\end{entrée}

\begin{entrée}{chińõõ}{}{ⓔchińõõ}
\formephonétique{cʰinɔ̃ː}
\région{GOs PA}
(\domainesémantique{Mouvements ou actions faits avec le corps, les bras, les mains, les pieds})
\classe{v}
\begin{glose}
\pfra{essorer ;}
\end{glose}
\begin{glose}
\pfra{frotter (linge, mains, etc.) ;}
\end{glose}
\begin{glose}
\pfra{tripoter (un objet) ; toucher}
\end{glose}
\end{entrée}

\begin{entrée}{chiò}{}{ⓔchiò}
\formephonétique{cʰiɔ}
\région{GOs BO PA}
(\domainesémantique{Ustensiles})
\classe{nom}
\begin{glose}
\pfra{seau}
\end{glose}
\newline
\emprunt{seau (FR)}
\end{entrée}

\begin{entrée}{chira, chiira}{}{ⓔchira, chiira}
\formephonétique{cʰira, cʰiɽa}
\région{GOs PA}
(\domainesémantique{Pêche})
\classe{nom}
\begin{glose}
\pfra{encoche de l'hameçon}
\end{glose}
\begin{glose}
\pfra{barbeau de la sagaie [BO, BM]}
\end{glose}
\end{entrée}

\begin{entrée}{chivi}{}{ⓔchivi}
\formephonétique{cʰivi}
\région{BO}
\variante{%
chivi, civi
\région{BO}}
(\domainesémantique{Verbes de mouvement})
\classe{v}
\begin{glose}
\pfra{écarter ; chasser (animal)}
\end{glose}
\end{entrée}

\begin{entrée}{chiwe}{}{ⓔchiwe}
\formephonétique{cʰiwe}
\région{GOs PA BO}
(\domainesémantique{Fonctions naturelles humaines})
\classe{v}
\begin{glose}
\pfra{éternuer}
\end{glose}
\end{entrée}

\begin{entrée}{chîxi}{}{ⓔchîxi}
\formephonétique{cʰîɣî}
\région{GOs PA}
\variante{%
chîngi
\formephonétique{cʰiŋi}
\région{GO(s) BO}}
(\domainesémantique{Manière de faire l’action : verbes et adverbes de manière})
\classe{v}
\begin{glose}
\pfra{contraire ; contraire (faire le)}
\end{glose}
\begin{glose}
\pfra{envers (à l') ; faire à l'envers}
\end{glose}
\begin{glose}
\pfra{faire l'inverse (de tout le monde)}
\end{glose}
\newline
\begin{exemple}
\région{GO}
\textbf{\pnua{e a-chîngi}}
\pfra{il est gauche, il fait tout à l'envers}
\end{exemple}
\newline
\begin{exemple}
\région{GO}
\textbf{\pnua{e ne-chîngi-ni}}
\pfra{il l'a fait à l'envers}
\end{exemple}
\newline
\begin{exemple}
\région{GO}
\textbf{\pnua{chîngi mwêêje-je}}
\pfra{ses façons de faire sont à l'envers de tout le monde}
\end{exemple}
\newline
\begin{exemple}
\textbf{\pnua{nu a-chîngi-da}}
\pfra{je monte (en sens opposé de toi qui descend)}
\end{exemple}
\newline
\begin{exemple}
\textbf{\pnua{i a chîxî}}
\pfra{il part en sens inverse}
\end{exemple}
\newline
\begin{exemple}
\région{PA}
\textbf{\pnua{i aa-chîxî}}
\pfra{qqn qui fait tout à l'inverse}
\end{exemple}
\end{entrée}

\begin{entrée}{chö}{}{ⓔchö}
\formephonétique{cʰω}
\région{GOs}
\variante{%
coho
\région{BO (Corne)}}
(\domainesémantique{Poissons})
\classe{nom}
\begin{glose}
\pfra{baleine ; cachalot}
\end{glose}
\end{entrée}

\begin{entrée}{chôã}{}{ⓔchôã}
\formephonétique{cʰôɛ̃}
\région{GOs PA BO WEM WE}
(\domainesémantique{Jeux divers})
\classe{v ; n}
\begin{glose}
\pfra{jouer ; s'amuser ; jeu}
\end{glose}
\newline
\begin{exemple}
\région{GO}
\textbf{\pnua{la pe-chôã}}
\pfra{ils jouent ensemble, ils se se jouent des tours}
\end{exemple}
\newline
\begin{exemple}
\région{GO}
\textbf{\pnua{e chôã-ni loto}}
\pfra{il joue avec sa voiture}
\end{exemple}
\newline
\begin{exemple}
\région{GO}
\textbf{\pnua{kebwa chôãni nye-na}}
\pfra{ne joue pas avec ça}
\end{exemple}
\newline
\begin{sous-entrée}{chôã-raa}{ⓔchôãⓝchôã-raa}
\région{GO}
\begin{glose}
\pfra{jouer des mauvais tours}
\end{glose}
\end{sous-entrée}
\newline
\begin{sous-entrée}{pe-chôã}{ⓔchôãⓝpe-chôã}
\région{GO}
\begin{glose}
\pfra{jouer ensemble}
\end{glose}
\newline
\note{chôã-nu}{grammaire}{mes jeux}
\newline
\note{v.t. chôã-ni, chôô-ni}{grammaire}{jouer avec qqch}
\end{sous-entrée}
\end{entrée}

\begin{entrée}{chôãni}{}{ⓔchôãni}
\formephonétique{cʰôɛ̃ɳi}
\région{GOs WEM WE}
(\domainesémantique{Jeux divers})
\classe{v.t.}
\begin{glose}
\pfra{jouer à qqch.}
\end{glose}
\newline
\note{chôã (v.i.)}{grammaire}{}
\end{entrée}

\begin{entrée}{chomu}{}{ⓔchomu}
\formephonétique{cʰomu}
\région{GOs BO}
\variante{%
comu
\région{PA}}
(\domainesémantique{Fonctions intellectuelles})
\classe{v}
\begin{glose}
\pfra{apprendre ; étudier}
\end{glose}
\begin{glose}
\pfra{lire [PA, BO]}
\end{glose}
\newline
\begin{exemple}
\région{BO}
\textbf{\pnua{nu chomuu-ni yuanga}}
\pfra{j'apprends le yuanga}
\end{exemple}
\newline
\begin{sous-entrée}{pha-chomu-ni}{ⓔchomuⓝpha-chomu-ni}
\begin{glose}
\pfra{enseigner}
\end{glose}
\end{sous-entrée}
\newline
\begin{sous-entrée}{aa-chomu}{ⓔchomuⓝaa-chomu}
\begin{glose}
\pfra{enseignant}
\end{glose}
\end{sous-entrée}
\newline
\begin{sous-entrée}{ba-chomu, ba-chòmu}{ⓔchomuⓝba-chomu, ba-chòmu}
\begin{glose}
\pfra{livre}
\end{glose}
\end{sous-entrée}
\newline
\begin{sous-entrée}{mo-chomu}{ⓔchomuⓝmo-chomu}
\begin{glose}
\pfra{école}
\end{glose}
\newline
\note{chomu-ni (v.t.)}{grammaire}{apprendre}
\end{sous-entrée}
\newline
\relationsémantique{Cf.}{\lien{}{pinãã [GOs]}}
\glosecourte{lire, compter}
\end{entrée}

\begin{entrée}{chòvwa}{}{ⓔchòvwa}
\formephonétique{cʰɔβa cʰɔːβa cʰɔva}
\région{GOs}
\variante{%
còval
\région{WE BO}, 
cova
\région{PA}}
(\domainesémantique{Mammifères})
\classe{nom}
\begin{glose}
\pfra{cheval}
\end{glose}
\newline
\begin{exemple}
\textbf{\pnua{chòvwa i nu}}
\pfra{mon cheval}
\end{exemple}
\newline
\emprunt{cheval (FR)}
\end{entrée}

\begin{entrée}{chue}{}{ⓔchue}
\formephonétique{cʰue}
\région{GOs WEM WE}
(\domainesémantique{Musique, instruments de musique})
\classe{v}
\begin{glose}
\pfra{jouer (guitare, carte, jeu de balle, sport)}
\end{glose}
\newline
\emprunt{jouer (FR)}
\end{entrée}

\newpage

\lettrine{d}\begin{entrée}{da}{}{ⓔda}
\formephonétique{nda}
\région{GOs PA BO}
\classe{DIR}
\newline
\sens{1}
(\domainesémantique{Directionnels})
\begin{glose}
\pfra{en haut}
\end{glose}
\begin{glose}
\pfra{en amont}
\end{glose}
\newline
\begin{sous-entrée}{ã-da-mi !}{ⓔdaⓢ1ⓝã-da-mi !}
\begin{glose}
\pfra{monte ici !}
\end{glose}
\newline
\relationsémantique{Ant.}{\lien{ⓔduⓗ2ⓢ1ⓝã-du !}{ã-du !}}
\glosecourte{descends ! (ou va vers le nord)}
\end{sous-entrée}
\newline
\sens{2}
(\domainesémantique{Directionnels})
\begin{glose}
\pfra{vers le sud}
\end{glose}
\newline
\sens{3}
(\domainesémantique{Directionnels})
\begin{glose}
\pfra{vers l'est}
\end{glose}
\newline
\sens{4}
(\domainesémantique{Directionnels})
\begin{glose}
\pfra{vers la terre ; vers le fond de la vallée ou l'intérieur du pays}
\end{glose}
\newline
\sens{5}
(\domainesémantique{Directionnels})
\begin{glose}
\pfra{à l'intérieur de la maison, vers le fond de la maison}
\end{glose}
\newline
\étymologie{
\langue{POc}
\étymon{*sake}}
\end{entrée}

\begin{entrée}{da?}{}{ⓔda?}
\formephonétique{nda}
\région{GOs PA BO}
\variante{%
ta?
\région{GO(s)}}
(\domainesémantique{Interrogatifs})
\classe{INT}
\begin{glose}
\pfra{quoi ?}
\end{glose}
\newline
\begin{exemple}
\textbf{\pnua{da nye hu-jo?}}
\pfra{qu'est-ce qui t'a mordu ?}
\end{exemple}
\newline
\begin{exemple}
\textbf{\pnua{e hu-jo xo da ?, hu-jo da ?}}
\pfra{qu'est-ce qui t'a mordu ?}
\end{exemple}
\newline
\begin{exemple}
\textbf{\pnua{da ? thoomwa o èmwê ?}}
\pfra{qu'est-ce ? une fille ou un garçon ?}
\end{exemple}
\newline
\begin{exemple}
\textbf{\pnua{i môgu ni da ?}}
\pfra{que fait-il ? (quel travail?)}
\end{exemple}
\newline
\begin{exemple}
\textbf{\pnua{da i cu ?}}
\pfra{qu'est-il pour toi ? (dans la parenté)}
\end{exemple}
\newline
\begin{exemple}
\région{GO}
\textbf{\pnua{da yaaza-cu ?- yaaza-nu Isabelle - yaaza-nu ce/je Isabelle}}
\pfra{quel est ton nom ? - je m'appelle Isabelle - mon nom c'est I.}
\end{exemple}
\newline
\begin{exemple}
\région{WE}
\textbf{\pnua{da yaala-cu ?}}
\pfra{quel est ton nom ? - je m'appelle Isabelle - mon nom c'est I.}
\end{exemple}
\newline
\begin{exemple}
\textbf{\pnua{ba-thu da ?}}
\pfra{à quoi ça sert ?}
\end{exemple}
\newline
\begin{exemple}
\textbf{\pnua{cu throbo ni da ?}}
\pfra{quand es-tu né ?}
\end{exemple}
\newline
\begin{exemple}
\textbf{\pnua{da ê? da nyè ?}}
\pfra{qu'est-ce que c'est?}
\end{exemple}
\newline
\begin{exemple}
\région{PA}
\textbf{\pnua{da yaala-n ?}}
\pfra{quel est son nom?}
\end{exemple}
\newline
\begin{exemple}
\région{GO}
\textbf{\pnua{da yaaza-cö ?}}
\pfra{quel est ton nom?}
\end{exemple}
\newline
\begin{exemple}
\région{GO}
\textbf{\pnua{yaaza da 'chö' ?}}
\pfra{que signifie 'chö'? (lit. c'est le nom de quoi 'chö')?}
\end{exemple}
\newline
\begin{exemple}
\région{GO}
\textbf{\pnua{da pwê-mewu i cu ?}}
\pfra{quel est ton village ?}
\end{exemple}
\newline
\begin{exemple}
\région{GO}
\textbf{\pnua{pò da nye ?}}
\pfra{c'est un fruit de quoi ?}
\end{exemple}
\newline
\begin{exemple}
\textbf{\pnua{i nôôli da ?}}
\pfra{que regarde-t-il?}
\end{exemple}
\newline
\begin{exemple}
\textbf{\pnua{jo kiiga da ?}}
\pfra{de quoi ris-tu ?}
\end{exemple}
\newline
\begin{exemple}
\région{GO}
\textbf{\pnua{co po za ?}}
\pfra{que fais-tu ?}
\end{exemple}
\newline
\begin{exemple}
\région{WEM}
\textbf{\pnua{co po ra ?}}
\pfra{que fais-tu ?}
\end{exemple}
\newline
\relationsémantique{Cf.}{\lien{}{ra ?; za ?}}
\glosecourte{quoi ? (position postposée)}
\newline
\relationsémantique{Cf.}{\lien{}{dajâ?}}
\glosecourte{quoi ?}
\newline
\étymologie{
\langue{POc}
\étymon{*sapa}
\glosecourte{what?}}
\end{entrée}

\begin{entrée}{daal}{}{ⓔdaal}
\région{PA}
(\domainesémantique{Bananiers et bananes})
\classe{nom}
\begin{glose}
\pfra{banane (non comestible, dont la sève rouge foncé est utilisée comme peinture lors des danses)}
\end{glose}
\end{entrée}

\begin{entrée}{daawe}{}{ⓔdaawe}
\formephonétique{ndaːwe}
\région{PA BO}
(\domainesémantique{Vents})
\classe{nom}
\begin{glose}
\pfra{vent froid du sud-ouest ; alizés du sud-ouest}
\end{glose}
\end{entrée}

\begin{entrée}{dabò}{}{ⓔdabò}
\région{GOs}
(\domainesémantique{Navigation})
\classe{nom}
\begin{glose}
\pfra{flotteur de balancier}
\end{glose}
\end{entrée}

\begin{entrée}{dada}{}{ⓔdada}
\formephonétique{danda}
\région{GOs PA}
(\domainesémantique{Insectes})
\classe{nom}
\begin{glose}
\pfra{cigale (très petite)}
\end{glose}
\end{entrée}

\begin{entrée}{dagi}{}{ⓔdagi}
\région{GOs}
\variante{%
daginy
\région{WEM BO PA}}
\newline
\sens{1}
(\domainesémantique{Oiseaux})
\classe{nom}
\begin{glose}
\pfra{lève-queue ; passereau}
\end{glose}
\nomscientifique{Rhipidura spilodera verreauxi}
\newline
\sens{2}
(\domainesémantique{Organisation sociale})
\classe{nom}
\begin{glose}
\pfra{messager du grand-chef}
\end{glose}
\end{entrée}

\begin{entrée}{dagi pwemwa}{}{ⓔdagi pwemwa}
\région{GOs}
\variante{%
daginy pwemwa
\région{WEM}}
(\domainesémantique{Organisation sociale})
\classe{nom}
\begin{glose}
\pfra{médiateur de la chefferie}
\end{glose}
\end{entrée}

\begin{entrée}{dagony}{}{ⓔdagony}
\région{PA}
(\domainesémantique{Insectes})
\classe{nom}
\begin{glose}
\pfra{libellule}
\end{glose}
\end{entrée}

\begin{entrée}{dalaèèn}{}{ⓔdalaèèn}
\région{BO}
\variante{%
daalèn ; dalaèn
\région{BO}, 
dalaan
\région{PA}}
(\domainesémantique{Société})
\classe{nom}
\begin{glose}
\pfra{étranger ; blanc ; européen}
\end{glose}
\end{entrée}

\begin{entrée}{daluça mã}{}{ⓔdaluça mã}
\région{GOs}
\variante{%
ba-oginen
\région{PA BO}}
(\domainesémantique{Objets coutumiers})
\classe{nom}
\begin{glose}
\pfra{bouquet de plante contenant une monnaie et entouré d'un lien de paille}
\end{glose}
\newline
\note{il est lancé par terre, dernier geste coutumier pour le défunt}{glose}{}
\end{entrée}

\begin{entrée}{dame}{}{ⓔdame}
\région{GOs BO}
\variante{%
dam
\région{BO}}
(\domainesémantique{Mouvements ou actions faits avec le corps, les bras, les mains, les pieds})
\classe{v}
\begin{glose}
\pfra{tasser (en frappant)}
\end{glose}
\newline
\begin{sous-entrée}{ba-dam}{ⓔdameⓝba-dam}
\begin{glose}
\pfra{outil pour damer}
\end{glose}
\newline
\relationsémantique{Cf.}{\lien{ⓔkhaⓗ1}{kha}}
\glosecourte{écraser (avec le pied), appuyer}
\end{sous-entrée}
\newline
\emprunt{damer (FR)}
\end{entrée}

\begin{entrée}{dao}{}{ⓔdao}
\formephonétique{ndao}
\région{GOs}
\région{PA BO}
\variante{%
daòn
\formephonétique{ndaɔn}}
\classe{nom}
\newline
\sens{1}
(\domainesémantique{Phénomènes atmosphériques et naturels})
\begin{glose}
\pfra{vapeur}
\end{glose}
\begin{glose}
\pfra{brouillard}
\end{glose}
\newline
\sens{2}
(\domainesémantique{Fonctions naturelles humaines})
\begin{glose}
\pfra{haleine}
\end{glose}
\end{entrée}

\begin{entrée}{dauliõ}{}{ⓔdauliõ}
\région{PA}
\variante{%
dawuliõ
\région{BO}}
(\domainesémantique{Vents})
\classe{nom}
\begin{glose}
\pfra{tourbillon (d'air)}
\end{glose}
\end{entrée}

\begin{entrée}{de}{}{ⓔde}
\formephonétique{nde}
\région{GOs BO PA}
\classe{nom}
\newline
\sens{1}
(\domainesémantique{Instruments})
\begin{glose}
\pfra{fourchette}
\end{glose}
\newline
\sens{2}
(\domainesémantique{Pêche})
\begin{glose}
\pfra{sagaie de pêche}
\end{glose}
\end{entrée}

\begin{entrée}{de-}{}{ⓔde-}
\formephonétique{nde}
\région{GOs PA}
(\domainesémantique{Préfixes classificateurs numériques})
\classe{CLF.NUM (mains de banane)}
\begin{glose}
\pfra{main de bananes}
\end{glose}
\newline
\begin{exemple}
\textbf{\pnua{de-xe, de-tru, de-ko, etc.}}
\pfra{une, deux, trois main de bananes, etc.}
\end{exemple}
\newline
\relationsémantique{Cf.}{\lien{ⓔthò-chaamwaⓢ2ⓝthò-xe}{thò-xe}}
\glosecourte{(pour les régimes)}
\end{entrée}

\begin{entrée}{dè}{}{ⓔdè}
\formephonétique{dɛ}
\région{GOs}
\variante{%
dèn
\formephonétique{dɛn}
\région{BO PA}}
\classe{nom}
\newline
\sens{1}
(\domainesémantique{Verbes de déplacement et moyens de déplacement})
\begin{glose}
\pfra{chemin ; sentier ; route}
\end{glose}
\newline
\begin{exemple}
\région{GO}
\textbf{\pnua{dèè-nu}}
\pfra{mon chemin}
\end{exemple}
\newline
\begin{exemple}
\textbf{\pnua{dèè we gò}}
\pfra{chenal/aqueduc en bambou}
\end{exemple}
\newline
\begin{sous-entrée}{phe dè-kibwaa}{ⓔdèⓢ1ⓝphe dè-kibwaa}
\begin{glose}
\pfra{prendre un raccourci (lit. chemin coupé)}
\end{glose}
\end{sous-entrée}
\newline
\begin{sous-entrée}{ce dèn}{ⓔdèⓢ1ⓝce dèn}
\begin{glose}
\pfra{grande route}
\end{glose}
\end{sous-entrée}
\newline
\sens{2}
(\domainesémantique{Organisation sociale})
\begin{glose}
\pfra{chemin coutumier}
\end{glose}
\newline
\begin{exemple}
\région{GO}
\textbf{\pnua{pe-dè îbi}}
\pfra{notre chemin (relation de parenté duelle réciproque par alliance)}
\end{exemple}
\newline
\étymologie{
\langue{POc}
\étymon{*(n)sala(n), *njala(n)}}
\end{entrée}

\begin{entrée}{deang}{}{ⓔdeang}
\région{WEM PA BO}
(\domainesémantique{Pêche})
\classe{nom}
\begin{glose}
\pfra{épuisette ; haveneau ; nasse (à crevette)}
\end{glose}
\newline
\relationsémantique{Cf.}{\lien{ⓔkevalu}{kevalu}}
\end{entrée}

\begin{entrée}{de-du}{}{ⓔde-du}
\région{GOs}
\variante{%
degu, dego
\région{BO [Corne]}}
(\domainesémantique{Types de maison, architecture de la maison})
\classe{nom}
\begin{glose}
\pfra{toiture en paille ("racines dehors")}
\end{glose}
\newline
\begin{exemple}
\textbf{\pnua{e yaa de-du}}
\pfra{il fait la toiture en paille racines vers l'extérieur}
\end{exemple}
\end{entrée}

\begin{entrée}{dee}{1}{ⓔdeeⓗ1}
\région{PA BO}
(\domainesémantique{Corps humain})
\classe{nom}
\begin{glose}
\pfra{côtes}
\end{glose}
\newline
\begin{exemple}
\région{PA}
\textbf{\pnua{dee-n}}
\pfra{ses côtes}
\end{exemple}
\newline
\begin{exemple}
\région{BO}
\textbf{\pnua{deein}}
\pfra{ses côtes}
\end{exemple}
\end{entrée}

\begin{entrée}{dee}{2}{ⓔdeeⓗ2}
\région{GOs}
\variante{%
deeny
\région{PA BO}}
(\domainesémantique{Anguilles})
\classe{nom}
\begin{glose}
\pfra{anguille de creek (rouge)}
\end{glose}
\end{entrée}

\begin{entrée}{dèè}{}{ⓔdèè}
\région{GOs}
(\domainesémantique{Instruments})
\classe{nom}
\begin{glose}
\pfra{roue}
\end{glose}
\newline
\begin{exemple}
\textbf{\pnua{dèè-loto}}
\pfra{la roue de la voiture}
\end{exemple}
\end{entrée}

\begin{entrée}{dee-chaamwa}{}{ⓔdee-chaamwa}
\région{GOs BO}
(\domainesémantique{Bananiers et bananes})
\classe{nom}
\begin{glose}
\pfra{main de banane}
\end{glose}
\end{entrée}

\begin{entrée}{deeny}{1}{ⓔdeenyⓗ1}
\région{BO PA}
(\domainesémantique{Vents})
\classe{v}
\begin{glose}
\pfra{vent du sud}
\end{glose}
\end{entrée}

\begin{entrée}{deeny}{2}{ⓔdeenyⓗ2}
\région{PA BO}
(\domainesémantique{Anguilles})
\classe{nom}
\begin{glose}
\pfra{anguille de creek et de forêt}
\end{glose}
\end{entrée}

\begin{entrée}{dèè-we}{}{ⓔdèè-we}
\région{GOs BO PA}
(\domainesémantique{Cultures, techniques, boutures})
\classe{nom}
\begin{glose}
\pfra{conduite d'eau pour les cultures ; aqueduc d'irrigation}
\end{glose}
\begin{glose}
\pfra{fossé d'écoulement (sur le bord du champ d'igname) [PA]}
\end{glose}
\newline
\begin{sous-entrée}{dèè-we gò}{ⓔdèè-weⓝdèè-we gò}
\région{BO}
\begin{glose}
\pfra{canalisation en bambou}
\end{glose}
\newline
\relationsémantique{Cf.}{\lien{}{pwang [BO]}}
\end{sous-entrée}
\end{entrée}

\begin{entrée}{degam}{}{ⓔdegam}
\formephonétique{ndeŋgam}
\région{BO}
(\domainesémantique{Mollusques})
\classe{nom}
\begin{glose}
\pfra{huître [BM, Corne]}
\end{glose}
\newline
\note{non verifié}{général}{}
\end{entrée}

\begin{entrée}{dei}{}{ⓔdei}
\région{PA BO [Corne]}
\variante{%
deei
}
(\domainesémantique{Mouvements ou actions faits avec le corps, les bras, les mains, les pieds})
\classe{v}
\begin{glose}
\pfra{couper (faire)}
\end{glose}
\newline
\relationsémantique{Cf.}{\lien{ⓔthrei}{threi}}
\end{entrée}

\begin{entrée}{dèl}{}{ⓔdèl}
\région{PA}
(\domainesémantique{Arbre})
\classe{nom}
\begin{glose}
\pfra{arbre (et bois qui sent comme le santal)}
\end{glose}
\nomscientifique{Santalum austro-caledonicum, Santalacées}
\end{entrée}

\begin{entrée}{dèn kha-jöe}{}{ⓔdèn kha-jöe}
\région{PA}
(\domainesémantique{Verbes de déplacement et moyens de déplacement})
\classe{nom}
\begin{glose}
\pfra{raccourci}
\end{glose}
\end{entrée}

\begin{entrée}{dèxavi}{}{ⓔdèxavi}
\formephonétique{ndɛɣavi}
\région{GOs}
\variante{%
dèè-xavi
\région{BO [Corne]}}
(\domainesémantique{Vents})
\classe{nom}
\begin{glose}
\pfra{tourbillon}
\end{glose}
\end{entrée}

\begin{entrée}{di}{}{ⓔdi}
\formephonétique{ndi}
\région{GOs PA BO}
(\domainesémantique{Noms des plantes})
\classe{nom}
\begin{glose}
\pfra{cordyline (symbole masculin)}
\end{glose}
\nomscientifique{Cordyline fruticosa (L.) A. Chev. (Agavacées)}
\newline
\étymologie{
\langue{POc}
\étymon{*siRi, *jiRi}
\glosecourte{cordyline}
\auteur{Ross}}
\end{entrée}

\begin{entrée}{dibee}{}{ⓔdibee}
\formephonétique{ndibeː}
\région{GOs}
(\domainesémantique{Aliments, alimentation})
\classe{nom}
\begin{glose}
\pfra{beurre}
\end{glose}
\newline
\emprunt{du beurre (FR)}
\end{entrée}

\begin{entrée}{didi}{}{ⓔdidi}
\région{BO}
\classe{v.stat.}
\newline
\sens{1}
(\domainesémantique{Description des objets, formes, consistance, taille})
\begin{glose}
\pfra{profond [Corne]}
\end{glose}
\newline
\sens{2}
(\domainesémantique{Couleurs})
\begin{glose}
\pfra{vert}
\end{glose}
\newline
\note{non vérifié}{général}{}
\end{entrée}

\begin{entrée}{digo}{}{ⓔdigo}
\formephonétique{ndiŋgo}
\région{GOs WEM WE BO}
\classe{nom}
\newline
\sens{1}
(\domainesémantique{Configuration des objets})
\begin{glose}
\pfra{fourche (arbre)}
\end{glose}
\begin{glose}
\pfra{bananes jumelles (dans une même enveloppe)}
\end{glose}
\newline
\sens{2}
(\domainesémantique{Corps animal})
\begin{glose}
\pfra{cornes [WEM BO]}
\end{glose}
\newline
\begin{exemple}
\textbf{\pnua{digo bwaa-n}}
\pfra{ses cornes}
\end{exemple}
\newline
\étymologie{
\langue{POc}
\étymon{*saŋa}
\auteur{Blust}}
\end{entrée}

\begin{entrée}{digöö}{}{ⓔdigöö}
\formephonétique{ndiŋgωː}
\formephonétique{ndiŋgoːɲ}
\région{GOs}
\variante{%
digoony
\région{BO PA}}
(\domainesémantique{Noms des plantes})
\classe{nom}
\begin{glose}
\pfra{"cassis" (arbuste épineux) ; épine}
\end{glose}
\end{entrée}

\begin{entrée}{diia}{}{ⓔdiia}
\région{PA BO}
\variante{%
diva
\région{BO}}
\classe{nom}
\newline
\sens{1}
(\domainesémantique{Mollusques})
\begin{glose}
\pfra{coquillage servant à couper l'igname (Charles)}
\end{glose}
\newline
\sens{2}
(\domainesémantique{Instruments})
\begin{glose}
\pfra{couteau pour igname (Dubois)}
\end{glose}
\end{entrée}

\begin{entrée}{dii-nu}{}{ⓔdii-nu}
\formephonétique{diːɳu}
\région{GOs BO}
(\domainesémantique{Cocotiers})
\classe{nom}
\begin{glose}
\pfra{fibre de coco}
\end{glose}
\end{entrée}

\begin{entrée}{diiri}{}{ⓔdiiri}
\région{GOs WEM}
(\domainesémantique{Oiseaux})
\classe{nom}
\begin{glose}
\pfra{cagou}
\end{glose}
\end{entrée}

\begin{entrée}{diiru}{}{ⓔdiiru}
\région{BO}
(\domainesémantique{Anguilles})
\classe{nom}
\begin{glose}
\pfra{anguille (tachetée bleu et blanc) [Corne]}
\end{glose}
\end{entrée}

\begin{entrée}{diixe}{}{ⓔdiixe}
\région{GOs PA}
(\domainesémantique{Verbes d'action (en général)})
\classe{v}
\begin{glose}
\pfra{ceindre ; serrer ; attacher (avec une corde) ; tendre (corde)}
\end{glose}
\end{entrée}

\begin{entrée}{dilee}{}{ⓔdilee}
\région{PA}
(\domainesémantique{Types de maison, architecture de la maison})
\classe{v}
\begin{glose}
\pfra{crépir ; faire un mur en torchis}
\end{glose}
\newline
\begin{exemple}
\région{GO}
\textbf{\pnua{la dilee mwa}}
\pfra{ils ont faire le mur en torchis, ils ont crépi la maison}
\end{exemple}
\end{entrée}

\begin{entrée}{dili}{}{ⓔdili}
\formephonétique{ndili}
\région{GOs BO PA}
(\domainesémantique{Terre})
\classe{nom}
\begin{glose}
\pfra{terre ; torchis}
\end{glose}
\newline
\begin{sous-entrée}{bu dili}{ⓔdiliⓝbu dili}
\région{BO PA}
\begin{glose}
\pfra{butte de terre}
\end{glose}
\end{sous-entrée}
\newline
\begin{sous-entrée}{dili bang}{ⓔdiliⓝdili bang}
\région{BO PA}
\begin{glose}
\pfra{terre noire}
\end{glose}
\end{sous-entrée}
\newline
\begin{sous-entrée}{dili baa}{ⓔdiliⓝdili baa}
\région{GO}
\begin{glose}
\pfra{terre noire}
\end{glose}
\end{sous-entrée}
\newline
\begin{sous-entrée}{dili pulo}{ⓔdiliⓝdili pulo}
\région{BO}
\begin{glose}
\pfra{terre blanche, chaux}
\end{glose}
\end{sous-entrée}
\newline
\begin{sous-entrée}{dili phozo}{ⓔdiliⓝdili phozo}
\begin{glose}
\pfra{terre blanche, chaux}
\end{glose}
\end{sous-entrée}
\newline
\begin{sous-entrée}{pubu dili}{ⓔdiliⓝpubu dili}
\begin{glose}
\pfra{poussière}
\end{glose}
\end{sous-entrée}
\newline
\begin{sous-entrée}{no dili}{ⓔdiliⓝno dili}
\begin{glose}
\pfra{terre}
\end{glose}
\end{sous-entrée}
\newline
\begin{sous-entrée}{dili mii}{ⓔdiliⓝdili mii}
\begin{glose}
\pfra{terre rouge}
\end{glose}
\newline
\relationsémantique{Cf.}{\lien{ⓔdilee}{dilee}}
\glosecourte{crépir ; faire (un mur) en torchis}
\end{sous-entrée}
\end{entrée}

\begin{entrée}{dili baa}{}{ⓔdili baa}
\variante{%
dili baang
\région{PA}}
(\domainesémantique{Terre})
\classe{nom}
\begin{glose}
\pfra{terre noire}
\end{glose}
\end{entrée}

\begin{entrée}{dimòm}{}{ⓔdimòm}
\formephonétique{dimɔm}
\région{BO [BM]}
\variante{%
dimwò-n
\région{BO}}
(\domainesémantique{Fonctions naturelles humaines})
\classe{nom}
\begin{glose}
\pfra{morve}
\end{glose}
\newline
\begin{exemple}
\textbf{\pnua{dimwò-n}}
\pfra{sa morve}
\end{exemple}
\newline
\relationsémantique{Cf.}{\lien{}{têi [GOs]}}
\glosecourte{morve}
\end{entrée}

\begin{entrée}{dimwã}{1}{ⓔdimwãⓗ1}
\formephonétique{dimwɛ̃}
\région{GOs PA BO}
(\domainesémantique{Ignames})
\classe{nom}
\begin{glose}
\pfra{igname sauvage (variété de);}
\end{glose}
\newline
\note{pour préparer une purée de cette igname rapée, il faut raper le 'dimwa' dans l'eau de la rivière (voir bwevòlò) pour enlever l'amertume, puis on recueille la chair lavée dans un panier 'keruau', elle est ensuite séchée puis cuite. Se déguste de préférence sucré.}{glose}{}
\nomscientifique{Dioscorea bulbifera}
\end{entrée}

\begin{entrée}{dimwã}{2}{ⓔdimwãⓗ2}
\région{GOs}
(\domainesémantique{Poissons})
\classe{nom}
\begin{glose}
\pfra{poisson-perroquet}
\end{glose}
\end{entrée}

\begin{entrée}{dimwãã ko}{}{ⓔdimwãã ko}
\région{GOs}
\variante{%
dimwãã diwe-ko
\région{BO PA}}
(\domainesémantique{Corps animal})
\classe{nom}
\begin{glose}
\pfra{crête de coq}
\end{glose}
\end{entrée}

\begin{entrée}{diva}{}{ⓔdiva}
\région{GO}
\variante{%
diia
\région{GO}}
(\domainesémantique{Mollusques})
\classe{nom}
\begin{glose}
\pfra{Pinctada (Pélécypodes)}
\end{glose}
\nomscientifique{Pinctada}
\end{entrée}

\begin{entrée}{dive}{}{ⓔdive}
\région{BO}
(\domainesémantique{Corps humain})
\classe{nom}
\begin{glose}
\pfra{hanche (Haudricourt, Corne)}
\end{glose}
\newline
\begin{exemple}
\région{BO}
\textbf{\pnua{divè-n}}
\pfra{sa hanche}
\end{exemple}
\end{entrée}

\begin{entrée}{divhii}{}{ⓔdivhii}
\région{GOs}
(\domainesémantique{Oiseaux})
\classe{nom}
\begin{glose}
\pfra{bécassine ; courlis corlieu}
\end{glose}
\end{entrée}

\begin{entrée}{dixa-ce}{}{ⓔdixa-ce}
\région{GOs PA}
(\domainesémantique{Parties de plantes})
\classe{nom}
\begin{glose}
\pfra{résine ; sève}
\end{glose}
\newline
\begin{exemple}
\région{PA}
\textbf{\pnua{thu dixa-n}}
\pfra{y avoir de la sève}
\end{exemple}
\newline
\étymologie{
\langue{POC}
\étymon{*suRuq}}
\end{entrée}

\begin{entrée}{dixa-nu}{}{ⓔdixa-nu}
\région{GOs PA BO}
\variante{%
dika-nu
\région{GO(s) BO}}
(\domainesémantique{Cocotiers})
\classe{nom}
\begin{glose}
\pfra{lait de coco ; huile de coco}
\end{glose}
\newline
\begin{sous-entrée}{mini-nu}{ⓔdixa-nuⓝmini-nu}
\begin{glose}
\pfra{résidu de coco}
\end{glose}
\end{sous-entrée}
\end{entrée}

\begin{entrée}{dixoo}{}{ⓔdixoo}
\région{GOs BO}
\classe{nom}
(\domainesémantique{Corps animal})
\begin{glose}
\pfra{cornes}
\end{glose}
\newline
\begin{sous-entrée}{dixoo drube}{ⓔdixooⓝdixoo drube}
\région{GO}
\begin{glose}
\pfra{cornes de cerf}
\end{glose}
\newline
\begin{exemple}
\région{BO}
\textbf{\pnua{dixoo ceevero}}
\pfra{cornes de cerf}
\end{exemple}
\newline
\begin{exemple}
\textbf{\pnua{dixoo bwa ceevero}}
\pfra{cornes du cerf}
\end{exemple}
\newline
\begin{exemple}
\textbf{\pnua{dixoo bwa-n}}
\pfra{cornes}
\end{exemple}
\end{sous-entrée}
\end{entrée}

\begin{entrée}{do}{}{ⓔdo}
\formephonétique{ndo}
\région{GOs PA BO}
\classe{nom}
\newline
\sens{1}
(\domainesémantique{Armes})
\begin{glose}
\pfra{sagaie}
\end{glose}
\newline
\begin{sous-entrée}{do de}{ⓔdoⓢ1ⓝdo de}
\begin{glose}
\pfra{sagaie à 3 pointes (trident)}
\end{glose}
\end{sous-entrée}
\newline
\begin{sous-entrée}{do teò}{ⓔdoⓢ1ⓝdo teò}
\begin{glose}
\pfra{sagaie de pêche}
\end{glose}
\end{sous-entrée}
\newline
\begin{sous-entrée}{do wexe}{ⓔdoⓢ1ⓝdo wexe}
\begin{glose}
\pfra{1 sagaie}
\end{glose}
\newline
\begin{exemple}
\région{BO PA}
\textbf{\pnua{doo-n}}
\pfra{sa sagaie}
\end{exemple}
\end{sous-entrée}
\newline
\sens{2}
(\domainesémantique{Jeux divers})
\begin{glose}
\pfra{figure de jeu de ficelle "la sagaie"}
\end{glose}
\newline
\étymologie{
\langue{POc}
\étymon{*sao(t)}}
\end{entrée}

\begin{entrée}{dö}{}{ⓔdö}
\région{BO}
(\domainesémantique{Corps humain})
\classe{nom}
\begin{glose}
\pfra{côtes}
\end{glose}
\newline
\begin{exemple}
\textbf{\pnua{dö-n}}
\pfra{sa côte}
\end{exemple}
\end{entrée}

\begin{entrée}{do-a}{}{ⓔdo-a}
\formephonétique{ndo.a}
\région{GOs}
\variante{%
do-al
\région{PA}}
(\domainesémantique{Astres})
\classe{nom}
\begin{glose}
\pfra{rayon de soleil}
\end{glose}
\newline
\relationsémantique{Cf.}{\lien{ⓔdo}{do}}
\glosecourte{sagaie}
\end{entrée}

\begin{entrée}{doau}{}{ⓔdoau}
\région{PA}
(\domainesémantique{Crustacés, crabes})
\classe{nom}
\begin{glose}
\pfra{crabe (de creek et de forêt, tout petit, quelques cms)}
\end{glose}
\end{entrée}

\begin{entrée}{döbe}{}{ⓔdöbe}
\formephonétique{dωmbe}
\région{GOs WEM BO}
\variante{%
dube
\formephonétique{dumbe}
\région{GA BO PA}}
(\domainesémantique{Fonctions intellectuelles})
\classe{v}
\begin{glose}
\pfra{dire des bêtises ; dérailler ; faire qqch sans sérieux ; faire des bêtises}
\end{glose}
\end{entrée}

\begin{entrée}{do-bubu}{}{ⓔdo-bubu}
\région{BO}
(\domainesémantique{Topographie})
\classe{nom}
\begin{glose}
\pfra{plaine verte [Corne]}
\end{glose}
\end{entrée}

\begin{entrée}{dobwa}{}{ⓔdobwa}
\région{BO}
(\domainesémantique{Taros})
\classe{nom}
\begin{glose}
\pfra{taro d'eau (clone) (Dubois)}
\end{glose}
\end{entrée}

\begin{entrée}{do-de}{}{ⓔdo-de}
(\domainesémantique{Armes})
\classe{nom}
\begin{glose}
\pfra{sagaie à 3 pointes (trident) (lit. sagaie fourchette)}
\end{glose}
\end{entrée}

\begin{entrée}{do-jitrua}{}{ⓔdo-jitrua}
\région{GOs}
(\domainesémantique{Armes})
\classe{nom}
\begin{glose}
\pfra{flèche}
\end{glose}
\end{entrée}

\begin{entrée}{dòlògò}{}{ⓔdòlògò}
\formephonétique{dɔlɔŋgɔ}
\région{GOs}
\variante{%
dològòm
\région{PA BO}}
(\domainesémantique{Noms des plantes})
\classe{nom}
\begin{glose}
\pfra{épinard (sorte d') ; feuille d'Aramanthus ; brède pariétaire (herbe à feuilles comestibles)}
\end{glose}
\nomscientifique{Amaranthus interruptus R. Br., Amaranthacées}
\end{entrée}

\begin{entrée}{dom}{}{ⓔdom}
\région{PA}
\variante{%
dum
\région{WE WEM}}
(\domainesémantique{Description des objets, formes, consistance, taille})
\classe{v}
\begin{glose}
\pfra{pointu}
\end{glose}
\end{entrée}

\begin{entrée}{dòmã}{}{ⓔdòmã}
\formephonétique{ndɔmã}
\région{PA BO [BM]}
(\domainesémantique{Couleurs})
\classe{v}
\begin{glose}
\pfra{noir}
\end{glose}
\newline
\relationsémantique{Cf.}{\lien{}{baa [GOs]}}
\glosecourte{noir}
\end{entrée}

\begin{entrée}{dòmògèn}{}{ⓔdòmògèn}
\formephonétique{dɔmɔŋgɛn}
\région{BO [BM]}
(\domainesémantique{Oiseaux})
\classe{nom}
\begin{glose}
\pfra{ralle de forêt (gros oiseau)}
\end{glose}
\end{entrée}

\begin{entrée}{dõni}{}{ⓔdõni}
\région{GOs PA BO}
(\domainesémantique{Quantificateurs})
\classe{LOC}
\begin{glose}
\pfra{parmi ; entre}
\end{glose}
\newline
\begin{exemple}
\région{BO}
\textbf{\pnua{i phe aa-xe na ni dõni la-ã ko}}
\pfra{il a pris un des poulets}
\end{exemple}
\end{entrée}

\begin{entrée}{dònò}{}{ⓔdònò}
\formephonétique{dɔɳɔdɔɳɔ̃}
\région{GOs}
\variante{%
dòòn
\formephonétique{ndɔːn}
\région{PA BO}}
(\domainesémantique{Astres})
\classe{nom}
\begin{glose}
\pfra{ciel ; cieux (en PA sens religieux)}
\end{glose}
\newline
\begin{sous-entrée}{dòò-va}{ⓔdònòⓝdòò-va}
\région{BO}
\begin{glose}
\pfra{ciel clair, dégagé}
\end{glose}
\end{sous-entrée}
\newline
\begin{sous-entrée}{dòò ni pwa}{ⓔdònòⓝdòò ni pwa}
\région{BO}
\begin{glose}
\pfra{ciel nuageux}
\end{glose}
\newline
\relationsémantique{Cf.}{\lien{}{phwa [PA]}}
\glosecourte{ciel}
\end{sous-entrée}
\end{entrée}

\begin{entrée}{dõõgo}{}{ⓔdõõgo}
\formephonétique{dɔ̃ːŋgo}
\région{GOs}
(\domainesémantique{Crustacés, crabes})
\classe{nom}
\begin{glose}
\pfra{crabe vide}
\end{glose}
\end{entrée}

\begin{entrée}{dòòla}{}{ⓔdòòla}
\région{BO}
(\domainesémantique{Sons, bruits})
\classe{nom}
\begin{glose}
\pfra{bruissement [Corne]}
\end{glose}
\newline
\begin{exemple}
\région{BO}
\textbf{\pnua{dòòla dòò-ce}}
\pfra{le bruissement des feuilles}
\end{exemple}
\newline
\note{non vérifié}{général}{}
\end{entrée}

\begin{entrée}{döölia}{}{ⓔdöölia}
\formephonétique{dωːlia}
\région{GOs PA BO}
\variante{%
dolia
\région{PA BO}}
(\domainesémantique{Parties de plantes})
\classe{nom}
\begin{glose}
\pfra{épine de}
\end{glose}
\newline
\begin{exemple}
\textbf{\pnua{döölia orã}}
\pfra{épine d'oranger}
\end{exemple}
\newline
\étymologie{
\langue{POc}
\étymon{*suRi}}
\end{entrée}

\begin{entrée}{döölia thra}{}{ⓔdöölia thra}
\région{GOs}
(\domainesémantique{Parties de plantes})
\classe{nom}
\begin{glose}
\pfra{épines, piquants de la nervure centrale de pandanus}
\end{glose}
\end{entrée}

\begin{entrée}{doo-pe}{}{ⓔdoo-pe}
\formephonétique{doːpe, dωːpe}
\région{GOs PA}
(\domainesémantique{Corps animal})
\classe{nom}
\begin{glose}
\pfra{dard de la raie (lit. sagaie de la raie)}
\end{glose}
\end{entrée}

\begin{entrée}{dòò-phwa}{}{ⓔdòò-phwa}
\région{PA}
(\domainesémantique{Corps humain})
\classe{nom}
\begin{glose}
\pfra{lèvres (feuille-bouche)}
\end{glose}
\end{entrée}

\begin{entrée}{doori}{}{ⓔdoori}
\région{PA BO}
(\domainesémantique{Types de maison, architecture de la maison})
\classe{nom}
\begin{glose}
\pfra{bord inférieur de la toiture (dépasse de la sablière)}
\end{glose}
\begin{glose}
\pfra{premier rang de paille (dépasse de la sablière)}
\end{glose}
\end{entrée}

\begin{entrée}{dopweza}{}{ⓔdopweza}
\région{PA BO}
(\domainesémantique{Objets coutumiers})
\classe{nom}
\begin{glose}
\pfra{feuille de bananier qui enveloppe la monnaie 'weem' (Charles)}
\end{glose}
\begin{glose}
\pfra{monnaie (Dubois : 1 dopweza de 2,5 m vaut 20 francs)}
\end{glose}
\begin{glose}
\pfra{feuille de bananier (voir 'pweza')}
\end{glose}
\newline
\relationsémantique{Cf.}{\lien{}{pwãmwãnu ; weem; yòò}}
\end{entrée}

\begin{entrée}{dou}{1}{ⓔdouⓗ1}
\région{GOs}
(\domainesémantique{Dons, échanges, achat et vente, vol})
\classe{nom}
\begin{glose}
\pfra{don ; offrande}
\end{glose}
\newline
\begin{exemple}
\textbf{\pnua{dou-nu}}
\pfra{mes dons}
\end{exemple}
\end{entrée}

\begin{entrée}{dou}{2}{ⓔdouⓗ2}
\région{GOs PA}
\variante{%
deü
\région{BO [Corne]}}
\classe{nom}
\newline
\sens{1}
(\domainesémantique{Corps humain})
\begin{glose}
\pfra{enveloppe}
\end{glose}
\begin{glose}
\pfra{enveloppe (corporelle)}
\end{glose}
\newline
\begin{exemple}
\région{PA}
\textbf{\pnua{dowa hegi [dou-a hegi]}}
\pfra{l'enveloppe de la monnaie}
\end{exemple}
\newline
\sens{2}
(\domainesémantique{Fonctions naturelles des animaux})
\begin{glose}
\pfra{dépouille de mue}
\end{glose}
\newline
\begin{exemple}
\région{PA}
\textbf{\pnua{dou-n}}
\pfra{son enveloppe}
\end{exemple}
\newline
\begin{exemple}
\région{BO}
\textbf{\pnua{deü pwaaji}}
\pfra{carapace de crabe (vide)}
\end{exemple}
\newline
\begin{exemple}
\région{BO}
\textbf{\pnua{deü-n}}
\pfra{carapace de crabe (vide)}
\end{exemple}
\end{entrée}

\begin{entrée}{dö-vwiã}{}{ⓔdö-vwiã}
\formephonétique{dωβiã}
\région{GOs}
\variante{%
dö-piã
\région{GO(s)}, 
du-piã
\région{BO}}
(\domainesémantique{Oiseaux})
\classe{nom}
\begin{glose}
\pfra{'cardinal' ; rouge-gorge (Diamant psittaculaire)}
\end{glose}
\begin{glose}
\pfra{colibri (Sucrier écarlate)}
\end{glose}
\nomscientifique{Erythrura psittacea (Estrildidés)}
\nomscientifique{Myzomela cardinalis et Myzomela dibapha}
\end{entrée}

\begin{entrée}{du}{1}{ⓔduⓗ1}
\formephonétique{ndu}
\région{GOs BO PA}
\classe{nom}
\newline
\sens{1}
(\domainesémantique{Corps humain})
\begin{glose}
\pfra{os}
\end{glose}
\newline
\begin{sous-entrée}{du-ko-je}{ⓔduⓗ1ⓢ1ⓝdu-ko-je}
\région{GO}
\begin{glose}
\pfra{son tibia}
\end{glose}
\end{sous-entrée}
\newline
\begin{sous-entrée}{duu bwa-n}{ⓔduⓗ1ⓢ1ⓝduu bwa-n}
\begin{glose}
\pfra{crâne}
\end{glose}
\newline
\begin{exemple}
\région{PA}
\textbf{\pnua{duu-n}}
\pfra{BO}
\pfra{son os}
\end{exemple}
\end{sous-entrée}
\newline
\sens{2}
(\domainesémantique{Corps humain})
\begin{glose}
\pfra{dos}
\end{glose}
\newline
\begin{exemple}
\région{GO}
\textbf{\pnua{e khinu duu-nu}}
\pfra{j'ai mal au dos}
\end{exemple}
\newline
\begin{exemple}
\région{GO}
\textbf{\pnua{e ciia duu-je du mu ca la ẽnõ}}
\pfra{il danse dos (tourné) aux enfants}
\end{exemple}
\newline
\begin{sous-entrée}{ki-duu-ny}{ⓔduⓗ1ⓢ2ⓝki-duu-ny}
\région{PA}
\begin{glose}
\pfra{ma colonne vertébrale}
\end{glose}
\end{sous-entrée}
\newline
\sens{2}
(\domainesémantique{Poissons})
\begin{glose}
\pfra{arête (poisson)}
\end{glose}
\newline
\begin{sous-entrée}{du-no}{ⓔduⓗ1ⓢ2ⓝdu-no}
\begin{glose}
\pfra{arête de poisson}
\end{glose}
\end{sous-entrée}
\newline
\sens{3}
(\domainesémantique{Cultures, techniques, boutures})
\begin{glose}
\pfra{tuteur à igname (grand)}
\end{glose}
\newline
\begin{sous-entrée}{du-kui}{ⓔduⓗ1ⓢ3ⓝdu-kui}
\begin{glose}
\pfra{tuteur à igname}
\end{glose}
\newline
\note{duu (en composition)}{grammaire}{}
\end{sous-entrée}
\newline
\relationsémantique{Cf.}{\lien{ⓔthaⓗ1}{tha}}
\glosecourte{tuteur (petit)}
\newline
\étymologie{
\langue{POc}
\étymon{*suRi}}
\end{entrée}

\begin{entrée}{du}{2}{ⓔduⓗ2}
\formephonétique{ndu}
\région{GOs PA}
\classe{DIR}
\newline
\sens{1}
(\domainesémantique{Directionnels})
\begin{glose}
\pfra{en bas}
\end{glose}
\newline
\begin{sous-entrée}{ã-du !}{ⓔduⓗ2ⓢ1ⓝã-du !}
\begin{glose}
\pfra{descends !}
\end{glose}
\end{sous-entrée}
\newline
\sens{2}
(\domainesémantique{Directionnels})
\begin{glose}
\pfra{vers le nord}
\end{glose}
\newline
\sens{3}
(\domainesémantique{Directionnels})
\begin{glose}
\pfra{vers l'ouest}
\end{glose}
\newline
\sens{4}
(\domainesémantique{Directionnels})
\begin{glose}
\pfra{vers la mer ; en aval}
\end{glose}
\newline
\sens{5}
(\domainesémantique{Directionnels})
\begin{glose}
\pfra{à l'extérieur de la maison, vers la porte}
\end{glose}
\newline
\étymologie{
\langue{POc}
\étymon{*(n)sipo}}
\end{entrée}

\begin{entrée}{dua}{}{ⓔdua}
\région{BO PA}
(\domainesémantique{Topographie})
\classe{nom}
\begin{glose}
\pfra{grotte ; caverne}
\end{glose}
\begin{glose}
\pfra{ravinement sur les routes[PA]}
\end{glose}
\newline
\note{non verifié}{général}{}
\end{entrée}

\begin{entrée}{dubila}{}{ⓔdubila}
\région{GOs PA BO WEM WE}
\variante{%
döbela, dubela
\région{GA GO(s)}}
(\domainesémantique{Soins du corps})
\classe{nom}
\begin{glose}
\pfra{peigne}
\end{glose}
\newline
\begin{exemple}
\région{GO}
\textbf{\pnua{dubilaa-nu}}
\pfra{mon peigne}
\end{exemple}
\newline
\begin{exemple}
\région{PA}
\textbf{\pnua{dubila-n}}
\pfra{son peigne}
\end{exemple}
\newline
\begin{sous-entrée}{döbela gò}{ⓔdubilaⓝdöbela gò}
\begin{glose}
\pfra{peigne en bambou}
\end{glose}
\end{sous-entrée}
\end{entrée}

\begin{entrée}{du-bwò}{}{ⓔdu-bwò}
\formephonétique{ndumbwo}
\région{GOs PA BO}
\variante{%
duu-bò
\région{BO}}
\classe{nom}
\newline
\sens{1}
(\domainesémantique{Couture})
\begin{glose}
\pfra{aiguille (lit. os de roussette')}
\end{glose}
\newline
\sens{2}
(\domainesémantique{Pêche})
\begin{glose}
\pfra{navette à filet}
\end{glose}
\newline
\étymologie{
\langue{POc}
\étymon{*(n)saRu}}
\end{entrée}

\begin{entrée}{du dròò-chaamwa}{}{ⓔdu dròò-chaamwa}
\formephonétique{ndu ɖɔː}
\région{GOs}
(\domainesémantique{Bananiers et bananes})
\classe{nom}
\begin{glose}
\pfra{nervure dorsale de la feuille de bananier}
\end{glose}
\end{entrée}

\begin{entrée}{du-hegi}{}{ⓔdu-hegi}
\région{GOs}
(\domainesémantique{Objets coutumiers})
\classe{nom}
\begin{glose}
\pfra{ossature de la monnaie}
\end{glose}
\end{entrée}

\begin{entrée}{du-hogo}{}{ⓔdu-hogo}
\région{PA}
(\domainesémantique{Topographie})
\classe{nom}
\begin{glose}
\pfra{ligne de crête (lit. os de la montagne) ; pente de la montagne}
\end{glose}
\end{entrée}

\begin{entrée}{du-kai}{}{ⓔdu-kai}
\formephonétique{ndu kai}
\région{GOs}
\variante{%
du-kaè-n
\formephonétique{dukaɛn}
\région{PA BO}}
(\domainesémantique{Corps humain})
\classe{nom}
\begin{glose}
\pfra{colonne vertébrale}
\end{glose}
\newline
\begin{exemple}
\région{GO}
\textbf{\pnua{du-kai-nu}}
\pfra{ma colonne vertébrale}
\end{exemple}
\end{entrée}

\begin{entrée}{du-kò}{}{ⓔdu-kò}
\région{GOs}
(\domainesémantique{Corps humain})
\classe{nom}
\begin{glose}
\pfra{tibia}
\end{glose}
\end{entrée}

\begin{entrée}{du-mi}{}{ⓔdu-mi}
\région{GOs}
(\domainesémantique{Directionnels})
\classe{DIR}
\begin{glose}
\pfra{approcher (s') en descendant}
\end{glose}
\end{entrée}

\newpage

\lettrine{dr
(variante de GOs)}\begin{entrée}{dra}{1}{ⓔdraⓗ1}
\formephonétique{nɖa}
\région{GOs}
\région{BO PA}
\variante{%
da
\formephonétique{nda}}
(\domainesémantique{Feu : objets et actions liés au feu})
\classe{nom}
\begin{glose}
\pfra{cendres ; poudre}
\end{glose}
\begin{glose}
\pfra{suie}
\end{glose}
\newline
\begin{sous-entrée}{drawa yai}{ⓔdraⓗ1ⓝdrawa yai}
\région{GO}
\begin{glose}
\pfra{cendres du feu}
\end{glose}
\end{sous-entrée}
\newline
\begin{sous-entrée}{drawa dröö}{ⓔdraⓗ1ⓝdrawa dröö}
\région{GO}
\begin{glose}
\pfra{la suie sur la marmite (litt. suie de la marmite)}
\end{glose}
\end{sous-entrée}
\newline
\begin{sous-entrée}{dawa doo}{ⓔdraⓗ1ⓝdawa doo}
\région{BO PA}
\begin{glose}
\pfra{la suie sur la marmite (litt. suie de la marmite)}
\end{glose}
\newline
\note{drawa (forme déterminée)}{grammaire}{poudre de}
\end{sous-entrée}
\newline
\étymologie{
\langue{POc}
\étymon{*ɖapu}}
\end{entrée}

\begin{entrée}{dra}{2}{ⓔdraⓗ2}
\formephonétique{ɖa}
\région{GOs}
(\domainesémantique{Sons, bruits})
\classe{v}
\begin{glose}
\pfra{éclater}
\end{glose}
\newline
\begin{exemple}
\textbf{\pnua{e dra de loto}}
\pfra{la roue de la voiture a éclaté}
\end{exemple}
\newline
\note{drale (v.t.)}{grammaire}{}
\end{entrée}

\begin{entrée}{draa}{1}{ⓔdraaⓗ1}
\formephonétique{ɖaː}
\région{GOs}
\variante{%
daa
\région{PA BO}}
(\domainesémantique{Topographie})
\classe{nom}
\begin{glose}
\pfra{plaine}
\end{glose}
\newline
\relationsémantique{Cf.}{\lien{ⓔbwadraa}{bwadraa}}
\glosecourte{plaine; plateau}
\end{entrée}

\begin{entrée}{draa}{2}{ⓔdraaⓗ2}
\formephonétique{ɖaː}
\région{GOs}
\variante{%
daa
\région{BO}}
\newline
\sens{1}
(\domainesémantique{Modalité, verbes modaux})
\classe{MODIF}
\begin{glose}
\pfra{faire spontanément (sans savoir, sans penser au résultat)}
\end{glose}
\newline
\begin{exemple}
\région{GO}
\textbf{\pnua{e draa ne}}
\pfra{il l'a fait tout seul, il l'a fait sans penser au résultat}
\end{exemple}
\newline
\begin{exemple}
\région{BO}
\textbf{\pnua{i daa nee}}
\pfra{il l'a fait tout seul, il l'a fait sans penser au résultat}
\end{exemple}
\newline
\begin{exemple}
\textbf{\pnua{e draa khôbwe}}
\pfra{il a dit cela sans savoir, il a inventé}
\end{exemple}
\newline
\sens{2}
(\domainesémantique{Intensificateur})
\classe{INTENS ; RFLX}
\begin{glose}
\pfra{seul ; de/par soi-même}
\end{glose}
\newline
\begin{exemple}
\région{GO}
\textbf{\pnua{nu draa a khilaa-je}}
\pfra{je suis parti moi-même la chercher}
\end{exemple}
\newline
\begin{exemple}
\textbf{\pnua{e draa kaalu}}
\pfra{il est tombé tout seul}
\end{exemple}
\newline
\begin{exemple}
\textbf{\pnua{e draa uça !}}
\pfra{il est revenu tout seul}
\end{exemple}
\newline
\begin{exemple}
\région{BO}
\textbf{\pnua{nu daa nye}}
\pfra{je l'ai fait tout seul}
\end{exemple}
\newline
\relationsémantique{Cf.}{\lien{ⓔdraa pune}{draa pune}}
\glosecourte{volontairement}
\end{entrée}

\begin{entrée}{draa}{3}{ⓔdraaⓗ3}
\formephonétique{ɖa}
\région{GOs}
\variante{%
daa
\région{PA BO}}
\newline
\sens{1}
(\domainesémantique{Contraste})
\classe{PRE.VB de contraste}
\begin{glose}
\pfra{contraste (opposition entre des actions, des agents)}
\end{glose}
\newline
\begin{exemple}
\textbf{\pnua{jö yu (è)nè, ma nu dra ã-du bòli}}
\pfra{reste ici, car je vais descendre là-bas en bas}
\end{exemple}
\newline
\begin{exemple}
\textbf{\pnua{nòme kavwö huu pwe, jö dra ã-du}}
\pfra{si cela ne mort pas, tu vas plus bas}
\end{exemple}
\newline
\sens{2}
(\domainesémantique{Contraste})
\classe{PRE.VB}
\begin{glose}
\pfra{tour de (au)}
\end{glose}
\newline
\begin{exemple}
\textbf{\pnua{draa ijö !}}
\pfra{à ton tour !}
\end{exemple}
\newline
\begin{exemple}
\textbf{\pnua{jö dra môgu, ma nu tree-çãnã}}
\pfra{à ton tour de travailler, car je vais me reposer}
\end{exemple}
\newline
\begin{exemple}
\région{PA}
\textbf{\pnua{i daa khôbwe !}}
\pfra{il a fini par avouer, par le dire !}
\end{exemple}
\newline
\relationsémantique{Cf.}{\lien{}{mwaa ijö !}}
\glosecourte{à ton tour !}
\end{entrée}

\begin{entrée}{draaçi}{}{ⓔdraaçi}
\formephonétique{ɖaːdʒi}
\région{GOs}
\variante{%
dacim
\région{PA}}
(\domainesémantique{Vents})
\classe{nom}
\begin{glose}
\pfra{vent alizé du sud-ouest}
\end{glose}
\end{entrée}

\begin{entrée}{draadro}{}{ⓔdraadro}
\formephonétique{ɖaːɖo}
\région{GOs}
\variante{%
dadeng
\région{BO (Corne)}, 
daadòng
\région{BO (Dubois, Corne)}}
(\domainesémantique{Arbre})
\classe{nom}
\begin{glose}
\pfra{arbuste de bord de mer à feuilles grises (plante de guerre) ; Gattilier}
\end{glose}
\begin{glose}
\pfra{arbuste (utilisé comme produit anti-puce)}
\end{glose}
\nomscientifique{Vitex trifolia (Verbénacées)}
\nomscientifique{Vitex rotundifolia (Verbénacées)}
\end{entrée}

\begin{entrée}{draalai}{}{ⓔdraalai}
\formephonétique{ɖaːlai}
\région{GOs}
\variante{%
daalaèn, dalaèn
\formephonétique{ndaːlɛːn}
\région{BO}}
(\domainesémantique{Société})
\classe{nom}
\begin{glose}
\pfra{Blanc (lit. poudre riz) ; européen}
\end{glose}
\end{entrée}

\begin{entrée}{draańi}{}{ⓔdraańi}
\formephonétique{ɖaːni}
\région{GOs}
(\domainesémantique{Poissons})
\classe{nom}
\begin{glose}
\pfra{bec de cane}
\end{glose}
\nomscientifique{Lethrinus sp.}
\end{entrée}

\begin{entrée}{draa pune}{}{ⓔdraa pune}
\formephonétique{ɖaːpuɳe}
\région{GOs}
(\domainesémantique{Modalité, verbes modaux})
\classe{ADV}
\begin{glose}
\pfra{volontairement ; exprès}
\end{glose}
\newline
\begin{exemple}
\région{GO}
\textbf{\pnua{e draa pune vwo zòi hii-je}}
\pfra{elle s'est coupé le bras volontairement}
\end{exemple}
\newline
\begin{exemple}
\région{GO}
\textbf{\pnua{e za draa zòi}}
\pfra{elle s'est coupée volontairement}
\end{exemple}
\newline
\begin{exemple}
\région{GO}
\textbf{\pnua{e kaavwö draa pune}}
\pfra{c'était involontaire}
\end{exemple}
\newline
\begin{exemple}
\région{GO}
\textbf{\pnua{li (za) draa pune xo/vwo phãde-nu}}
\pfra{ils me l'ont montré exprès}
\end{exemple}
\newline
\relationsémantique{Cf.}{\lien{}{kavwö li draa pu-ne}}
\glosecourte{ils ne l'ont pas fait volontairement}
\end{entrée}

\begin{entrée}{draa-phwalawa}{}{ⓔdraa-phwalawa}
\région{GOs}
(\domainesémantique{Aliments, alimentation})
\classe{nom}
\begin{glose}
\pfra{farine (lit. poudre pain)}
\end{glose}
\end{entrée}

\begin{entrée}{draba}{}{ⓔdraba}
\formephonétique{ɖaba}
\région{GOs}
(\domainesémantique{Mer : topographie})
\classe{nom}
\begin{glose}
\pfra{îlot d'alluvion}
\end{glose}
\end{entrée}

\begin{entrée}{drabu}{}{ⓔdrabu}
\formephonétique{ɖabu}
\région{GOs}
(\domainesémantique{Reptiles marins})
\classe{nom}
\begin{glose}
\pfra{tortue de mer "grosse tête"}
\end{glose}
\end{entrée}

\begin{entrée}{drale}{1}{ⓔdraleⓗ1}
\formephonétique{ɖale}
\région{GOs}
\variante{%
dale, daale
\région{GO(s)}}
(\domainesémantique{Mouvements ou actions faits avec le corps, les bras, les mains, les pieds})
\classe{v}
\begin{glose}
\pfra{fendre ; casser}
\end{glose}
\newline
\begin{sous-entrée}{khi-drale}{ⓔdraleⓗ1ⓝkhi-drale}
\begin{glose}
\pfra{fendre}
\end{glose}
\end{sous-entrée}
\newline
\étymologie{
\langue{POc}
\étymon{*saRi}}
\end{entrée}

\begin{entrée}{drale}{2}{ⓔdraleⓗ2}
\formephonétique{ɖale}
\région{GOs}
\variante{%
daala
\région{BO [Corne]}}
(\domainesémantique{Crustacés, crabes})
\classe{nom}
\begin{glose}
\pfra{crabe de rivière}
\end{glose}
\end{entrée}

\begin{entrée}{drapwê}{}{ⓔdrapwê}
\région{GOs}
\variante{%
damwê
}
(\domainesémantique{Cours de la vie
, Parenté})
\classe{nom}
\begin{glose}
\pfra{veuf ; veuve}
\end{glose}
\newline
\begin{exemple}
\textbf{\pnua{drapwela Pwacili}}
\pfra{veuf du clan Pwacili}
\end{exemple}
\end{entrée}

\begin{entrée}{drau}{}{ⓔdrau}
\formephonétique{nɖa.u}
\région{GOs}
\variante{%
dau
\formephonétique{nda.u}
\région{PA BO}}
(\domainesémantique{Mer : topographie})
\classe{nom}
\begin{glose}
\pfra{île ; plâtier}
\end{glose}
\newline
\étymologie{
\langue{POc}
\étymon{*sakaRu}
\glosecourte{récif, banc de sable}}
\end{entrée}

\begin{entrée}{drava-dröö}{}{ⓔdrava-dröö}
\région{GOs}
(\domainesémantique{Préparation des aliments; modes de préparation et de cuisson})
\classe{nom}
\begin{glose}
\pfra{vapeur de la marmite}
\end{glose}
\end{entrée}

\begin{entrée}{drawa-dröö}{}{ⓔdrawa-dröö}
\région{GOs}
(\domainesémantique{Feu : objets et actions liés au feu})
\classe{nom}
\begin{glose}
\pfra{suie sur la marmite}
\end{glose}
\newline
\relationsémantique{Cf.}{\lien{ⓔdraⓗ1}{dra}}
\glosecourte{cendres}
\end{entrée}

\begin{entrée}{drawalu}{}{ⓔdrawalu}
\formephonétique{'nɖawalu}
\région{GOs}
(\domainesémantique{Noms des plantes})
\classe{nom}
\begin{glose}
\pfra{herbe ; pelouse}
\end{glose}
\end{entrée}

\begin{entrée}{dra-wawe}{}{ⓔdra-wawe}
\formephonétique{nɖawawe}
\région{GOs}
\variante{%
da-whaawe
\région{BO (Corne)}, 
dra-wapwe
\région{vx}}
(\domainesémantique{Arbre})
\classe{nom}
\begin{glose}
\pfra{erythrine "peuplier"}
\end{glose}
\nomscientifique{Erythrina pyramidalis}
\newline
\étymologie{
\langue{POc}
\étymon{*(n)ɖaɖap}}
\end{entrée}

\begin{entrée}{drawa yai}{}{ⓔdrawa yai}
\formephonétique{ɖawa yai}
\région{GOs}
(\domainesémantique{Feu : objets et actions liés au feu})
\classe{nom}
\begin{glose}
\pfra{cendres du feu}
\end{glose}
\newline
\relationsémantique{Cf.}{\lien{ⓔdraⓗ1}{dra}}
\glosecourte{cendres}
\end{entrée}

\begin{entrée}{dre}{1}{ⓔdreⓗ1}
\formephonétique{ɖe}
\région{GOs WEM}
(\domainesémantique{Types de maison, architecture de la maison})
\classe{nom}
\begin{glose}
\pfra{lianes attachent les bois sur la maison}
\end{glose}
\end{entrée}

\begin{entrée}{dre}{2}{ⓔdreⓗ2}
\formephonétique{ɖe}
\région{GOs}
(\domainesémantique{Poissons})
\classe{nom}
\begin{glose}
\pfra{poisson "cochon"}
\end{glose}
\nomscientifique{Gazza minuta et Leiognathus equulus (Leiognathidés)}
\end{entrée}

\begin{entrée}{dree}{}{ⓔdree}
\formephonétique{ɖeː}
\formephonétique{ndɛːn}
\région{GOs}
\variante{%
dèèn
\région{BO PA}}
(\domainesémantique{Vents})
\classe{nom}
\begin{glose}
\pfra{vent ; air ; atmosphère ; cyclone}
\end{glose}
\newline
\begin{exemple}
\région{PA}
\textbf{\pnua{i pha dèèn}}
\pfra{le vent souffle}
\end{exemple}
\newline
\begin{exemple}
\textbf{\pnua{dèèn nî we-za}}
\pfra{brise de mer}
\end{exemple}
\newline
\begin{sous-entrée}{dre-bwamò}{ⓔdreeⓝdre-bwamò}
\région{GO}
\begin{glose}
\pfra{brise de terre}
\end{glose}
\end{sous-entrée}
\newline
\begin{sous-entrée}{dree kaze}{ⓔdreeⓝdree kaze}
\région{GO}
\begin{glose}
\pfra{brise de mer}
\end{glose}
\end{sous-entrée}
\newline
\begin{sous-entrée}{dre xa wî}{ⓔdreeⓝdre xa wî}
\begin{glose}
\pfra{vent qui est fort}
\end{glose}
\end{sous-entrée}
\newline
\begin{sous-entrée}{pa-dre}{ⓔdreeⓝpa-dre}
\région{GO}
\begin{glose}
\pfra{coup de vent, retour de cyclone}
\end{glose}
\end{sous-entrée}
\newline
\begin{sous-entrée}{bwa dre}{ⓔdreeⓝbwa dre}
\région{GO}
\begin{glose}
\pfra{au vent, vent debout}
\end{glose}
\end{sous-entrée}
\end{entrée}

\begin{entrée}{dreebò}{}{ⓔdreebò}
\région{GOs}
(\domainesémantique{Phénomènes atmosphériques et naturels})
\classe{n ; v.stat.}
\begin{glose}
\pfra{nuageux ; gros nuage}
\end{glose}
\end{entrée}

\begin{entrée}{dree-bwamõ}{}{ⓔdree-bwamõ}
\formephonétique{ɖeː}
\région{GOs}
\variante{%
dèèn-bwa-mòl
\région{PA BO [Corne]}}
(\domainesémantique{Vents})
\classe{nom}
\begin{glose}
\pfra{vent de terre ; brise de terre ; vent d'est}
\end{glose}
\end{entrée}

\begin{entrée}{dree bwa pwa}{}{ⓔdree bwa pwa}
\formephonétique{ɖeː}
\région{GOs}
(\domainesémantique{Vents})
\classe{nom}
\begin{glose}
\pfra{vent annonciateur de pluie}
\end{glose}
\end{entrée}

\begin{entrée}{dree-bwava}{}{ⓔdree-bwava}
\formephonétique{ɖeː}
\région{GOs}
(\domainesémantique{Vents})
\classe{nom}
\begin{glose}
\pfra{vent alizé du S.E}
\end{glose}
\end{entrée}

\begin{entrée}{dree ni we}{}{ⓔdree ni we}
\formephonétique{ɖeː}
\région{GOs}
(\domainesémantique{Vents})
\classe{nom}
\begin{glose}
\pfra{vent de mer}
\end{glose}
\end{entrée}

\begin{entrée}{drele-ma-drele}{}{ⓔdrele-ma-drele}
\formephonétique{ɖele}
\région{GOs}
\variante{%
dele-ma-dele
\région{BO}}
\classe{nom}
\newline
\sens{1}
(\domainesémantique{Parenté})
\begin{glose}
\pfra{arrière-arrière petits-enfants [GOs]}
\end{glose}
\newline
\sens{2}
(\domainesémantique{Organisation sociale})
\begin{glose}
\pfra{ascendants (de la lignée) [BO, Corne]}
\end{glose}
\newline
\begin{exemple}
\région{BO}
\textbf{\pnua{dele-ma-dele a Mateo}}
\pfra{les ascendants de Mateo}
\end{exemple}
\end{entrée}

\begin{entrée}{drewaa}{}{ⓔdrewaa}
\formephonétique{ɖewaː}
\région{GOs}
\variante{%
dea
\région{WE BO}, 
deaang
\formephonétique{ndeaːŋ}
\région{PA}}
(\domainesémantique{Pêche})
\classe{nom}
\begin{glose}
\pfra{nasse (en forme de poche pour fouiller les berges)}
\end{glose}
\begin{glose}
\pfra{épuisette à crevettes}
\end{glose}
\newline
\begin{sous-entrée}{drewaa kula}{ⓔdrewaaⓝdrewaa kula}
\begin{glose}
\pfra{épuisette à crevette}
\end{glose}
\end{sous-entrée}
\newline
\begin{sous-entrée}{drewaa pwaji}{ⓔdrewaaⓝdrewaa pwaji}
\begin{glose}
\pfra{épuisette à crabe}
\end{glose}
\newline
\relationsémantique{Cf.}{\lien{ⓔkevalu}{kevalu}}
\glosecourte{épuisette (plus grande que "deaang")}
\end{sous-entrée}
\end{entrée}

\begin{entrée}{driluu}{}{ⓔdriluu}
\formephonétique{ɖiluː}
\région{GOs}
\variante{%
dilu, dilo
\région{PA BO}, 
diluuc
\région{BO (Corne)}}
(\domainesémantique{Noms des plantes})
\classe{nom}
\begin{glose}
\pfra{hibiscus}
\end{glose}
\nomscientifique{Hibiscus rosa sinensis L.}
\end{entrée}

\begin{entrée}{drò}{}{ⓔdrò}
\formephonétique{ɖɔ}
\région{GOs}
\variante{%
dòny
\formephonétique{dɔɲ}
\région{PA BO}}
(\domainesémantique{Oiseaux})
\classe{nom}
\begin{glose}
\pfra{buse ; émouchet ; faucon (petit, avec des panaches rougeâtres sous le ventre)}
\end{glose}
\nomscientifique{Accipiter fasciatus vigilax (Accipitridés)}
\nomscientifique{ou Circus approximans approximans}
\end{entrée}

\begin{entrée}{drõbö}{}{ⓔdrõbö}
\formephonétique{ɖɔ̃ːbω}
\région{GOs}
\variante{%
dòbo
\région{BO PA}}
\classe{v ; n}
\newline
\sens{1}
(\domainesémantique{Verbes de mouvement})
\begin{glose}
\pfra{éboulement ; écrouler (s')}
\end{glose}
\newline
\begin{exemple}
\région{GO}
\textbf{\pnua{e drõbö mwa}}
\pfra{la maison s'est écroulée}
\end{exemple}
\newline
\begin{exemple}
\région{GO}
\textbf{\pnua{e drõbö dili}}
\pfra{la terre s'éboule}
\end{exemple}
\newline
\begin{exemple}
\textbf{\pnua{nu drõbö-ni mwa}}
\pfra{j'ai démoli les murs de lamaison}
\end{exemple}
\newline
\sens{2}
(\domainesémantique{Topographie})
\begin{glose}
\pfra{érosion ; ravinement ; terre ravinée}
\end{glose}
\newline
\begin{exemple}
\région{GO}
\textbf{\pnua{e drõbö hogo}}
\pfra{la montagne se ravine, s'érode}
\end{exemple}
\end{entrée}

\begin{entrée}{drògò}{}{ⓔdrògò}
\formephonétique{ɖɔŋgɔ}
\région{GOs WEM}
\variante{%
dògò
\région{PA}}
(\domainesémantique{Objets coutumiers
, Types de maison, architecture de la maison})
\classe{nom}
\begin{glose}
\pfra{masque (comprenant l'habit qui accompagne le masque)}
\end{glose}
\begin{glose}
\pfra{chambranles}
\end{glose}
\begin{glose}
\pfra{sculpture faîtière}
\end{glose}
\end{entrée}

\begin{entrée}{dròò}{}{ⓔdròò}
\formephonétique{ɖɔː}
\région{GOs}
\variante{%
dòò
\région{PA BO}}
(\domainesémantique{Parties de plantes})
\classe{nom}
\begin{glose}
\pfra{feuille}
\end{glose}
\newline
\begin{sous-entrée}{dròò kò}{ⓔdròòⓝdròò kò}
\région{GO}
\begin{glose}
\pfra{petite plante à feuilles comestibles}
\end{glose}
\end{sous-entrée}
\newline
\begin{sous-entrée}{dròò nu}{ⓔdròòⓝdròò nu}
\begin{glose}
\pfra{palme de cocotier}
\end{glose}
\end{sous-entrée}
\newline
\begin{sous-entrée}{dròò-ce}{ⓔdròòⓝdròò-ce}
\région{GO}
\begin{glose}
\pfra{feuille d'arbre, rameaux}
\end{glose}
\end{sous-entrée}
\newline
\begin{sous-entrée}{drò-kenii-je}{ⓔdròòⓝdrò-kenii-je}
\région{GO}
\begin{glose}
\pfra{son pavillon d'oreille}
\end{glose}
\end{sous-entrée}
\newline
\begin{sous-entrée}{dòò ce}{ⓔdròòⓝdòò ce}
\région{PA}
\begin{glose}
\pfra{feuilles; "médicament"; "boucan"}
\end{glose}
\newline
\begin{exemple}
\région{PA BO}
\textbf{\pnua{dòò-n}}
\pfra{sa feuille}
\end{exemple}
\end{sous-entrée}
\newline
\étymologie{
\langue{POc}
\étymon{*nɖau(n)}}
\end{entrée}

\begin{entrée}{dröö}{}{ⓔdröö}
\formephonétique{ɖωː}
\région{GOs}
\variante{%
doo
\formephonétique{ndoː}
\région{BO PA}}
(\domainesémantique{Ustensiles})
\classe{nom}
\begin{glose}
\pfra{marmite (originellement en poterie)}
\end{glose}
\begin{glose}
\pfra{argile de poterie}
\end{glose}
\newline
\begin{exemple}
\région{GO}
\textbf{\pnua{phuu drööa-wa !}}
\pfra{retirez votre marmite de nourriture (du feu) !}
\end{exemple}
\newline
\begin{sous-entrée}{drööa kui}{ⓔdrööⓝdrööa kui}
\région{GO}
\begin{glose}
\pfra{une marmite d'ignames}
\end{glose}
\end{sous-entrée}
\newline
\begin{sous-entrée}{drööa nò}{ⓔdrööⓝdrööa nò}
\région{GO}
\begin{glose}
\pfra{une marmite de poisson}
\end{glose}
\end{sous-entrée}
\newline
\begin{sous-entrée}{kivha dooa-ny}{ⓔdrööⓝkivha dooa-ny}
\région{PA}
\begin{glose}
\pfra{le couvercle de ma marmite}
\end{glose}
\end{sous-entrée}
\newline
\begin{sous-entrée}{whaa dooa no}{ⓔdrööⓝwhaa dooa no}
\région{PA}
\begin{glose}
\pfra{une grande marmite de poisson}
\end{glose}
\newline
\begin{exemple}
\région{PA}
\textbf{\pnua{whaa dooa-ny (a) no}}
\pfra{ma grande marmite de poisson}
\end{exemple}
\end{sous-entrée}
\newline
\begin{sous-entrée}{doo-togi}{ⓔdrööⓝdoo-togi}
\région{BO}
\begin{glose}
\pfra{marmite en fonte}
\end{glose}
\newline
\note{dooa-n}{grammaire}{sa marmite}
\end{sous-entrée}
\newline
\étymologie{
\langue{POc}
\étymon{*daRoq}
\glosecourte{argile (clay)}}
\end{entrée}

\begin{entrée}{dròò-du}{}{ⓔdròò-du}
\formephonétique{ɖɔːdu}
\région{GOs}
(\domainesémantique{Types de maison, architecture de la maison})
\classe{nom}
\begin{glose}
\pfra{bord inférieur de la toiture}
\end{glose}
\end{entrée}

\begin{entrée}{dròò ê}{}{ⓔdròò ê}
\formephonétique{ɖɔː ê}
\région{GOs}
\variante{%
dòòèm
\région{PA}}
(\domainesémantique{Parties de plantes})
\classe{nom}
\begin{glose}
\pfra{feuille de canne à sucre}
\end{glose}
\end{entrée}

\begin{entrée}{dròò-keni}{}{ⓔdròò-keni}
\formephonétique{ɖɔː keɳi}
\région{GOs}
\variante{%
dròò-xeni
\région{GO(s)}, 
cò-keni
\région{GO(s) PA}, 
dòò-keni
\région{BO}, 
dòò-va-jeni
\région{WEM}}
(\domainesémantique{Corps humain})
\classe{nom}
\begin{glose}
\pfra{lobe (oreille)}
\end{glose}
\begin{glose}
\pfra{pavillon de l'oreille}
\end{glose}
\newline
\begin{exemple}
\région{GO}
\textbf{\pnua{drò-kenii-je}}
\pfra{le pavillon de son oreille}
\end{exemple}
\newline
\begin{exemple}
\région{BO}
\textbf{\pnua{dòò-kenii-n}}
\pfra{le pavillon de son oreille}
\end{exemple}
\newline
\begin{exemple}
\textbf{\pnua{còò kenii-n}}
\pfra{lobe de l'oreille}
\end{exemple}
\end{entrée}

\begin{entrée}{dròò-kibö}{}{ⓔdròò-kibö}
\formephonétique{ɖɔːkibω}
\région{GOs}
\variante{%
dròò-xibö
\formephonétique{ɖɔːɣibω}}
(\domainesémantique{Poissons})
\classe{nom}
\begin{glose}
\pfra{carangue (taille moyenne) (lit.feuille de palétuvier)}
\end{glose}
\newline
\relationsémantique{Cf.}{\lien{ⓔkûxûⓗ1}{kûxû}}
\glosecourte{carangue (même) de petite taille}
\end{entrée}

\begin{entrée}{dròò-ko}{}{ⓔdròò-ko}
\formephonétique{ɖɔːko}
\région{GOs}
\variante{%
dòò-ko
\région{BO}}
(\domainesémantique{Noms des plantes})
\classe{nom}
\begin{glose}
\pfra{brède à feuilles comestibles ; Morelle noire}
\end{glose}
\nomscientifique{Solanum nigrum L. (Solanacées)}
\end{entrée}

\begin{entrée}{droo-mwa}{}{ⓔdroo-mwa}
\formephonétique{ɖoːmwa}
\région{GOs}
\variante{%
doori
\région{PA BO}}
(\domainesémantique{Types de maison, architecture de la maison})
\classe{nom}
\begin{glose}
\pfra{bord inférieur de la toiture (qui dépasse de la sablière des maisons carrées)}
\end{glose}
\end{entrée}

\begin{entrée}{dròò-phê}{}{ⓔdròò-phê}
\formephonétique{ɖɔːphê}
\région{GOs}
\variante{%
dròò-phoã
\région{GOs}}
(\domainesémantique{Noms des plantes})
\classe{nom}
\begin{glose}
\pfra{laiteron ; feuille de "pissenlit"}
\end{glose}
\nomscientifique{Sonchus oleraceus L. (Euphorbiacées)}
\end{entrée}

\begin{entrée}{dròòrò}{}{ⓔdròòrò}
\formephonétique{ɖɔːɽɔɖɔːrɔ}
\région{GOs}
\variante{%
dròrò
\région{GO(s)}}
(\domainesémantique{Adverbes déictiques de temps})
\classe{LOC}
\begin{glose}
\pfra{hier}
\end{glose}
\newline
\begin{sous-entrée}{kabu dròòrò}{ⓔdròòròⓝkabu dròòrò}
\begin{glose}
\pfra{la semaine dernière}
\end{glose}
\end{sous-entrée}
\newline
\begin{sous-entrée}{kabu ni hêbu}{ⓔdròòròⓝkabu ni hêbu}
\begin{glose}
\pfra{la semaine d'avant (il y a longtemps), les semaines passées}
\end{glose}
\end{sous-entrée}
\end{entrée}

\begin{entrée}{dròò-xè}{}{ⓔdròò-xè}
\formephonétique{ɖɔːɣɛ}
\région{GO}
\variante{%
dòò-
\région{PA}}
(\domainesémantique{Préfixes classificateurs numériques})
\classe{CLF.NUM (feuilles)}
\begin{glose}
\pfra{une (feuille)}
\end{glose}
\newline
\begin{exemple}
\textbf{\pnua{dròò-xè; dro-tru}}
\pfra{un ; deux feuilles}
\end{exemple}
\end{entrée}

\begin{entrée}{dròòxi}{}{ⓔdròòxi}
\formephonétique{ɖɔːɣi}
\région{GOs}
\variante{%
dòki
\région{BO}}
(\domainesémantique{Religion, représentations religieuses})
\classe{nom}
\begin{glose}
\pfra{magie}
\end{glose}
\begin{glose}
\pfra{esprit (se manifestant par une boule de feu dans la nuit)}
\end{glose}
\end{entrée}

\begin{entrée}{dròò-yòò}{}{ⓔdròò-yòò}
\formephonétique{ɖɔːyɔː}
\région{GOs}
(\domainesémantique{Poissons})
\classe{nom}
\begin{glose}
\pfra{poisson (lit. feuille de bois de fer)}
\end{glose}
\end{entrée}

\begin{entrée}{drope}{}{ⓔdrope}
\formephonétique{ɖope}
\région{GOs}
(\domainesémantique{Couleurs})
\classe{v}
\begin{glose}
\pfra{multicolore}
\end{glose}
\end{entrée}

\begin{entrée}{drò-uva}{}{ⓔdrò-uva}
(\domainesémantique{Taros})
\classe{nom}
\begin{glose}
\pfra{feuille de taro d'eau}
\end{glose}
\newline
\relationsémantique{Cf.}{\lien{}{drò [GOs]}}
\glosecourte{feuille de taro d'eau}
\end{entrée}

\begin{entrée}{dròvivińi}{}{ⓔdròvivińi}
\formephonétique{ɖɔβiβini}
\région{GOs}
\variante{%
dopipini
\région{GO}, 
dovivini
\région{BO}}
(\domainesémantique{Découpage du temps})
\classe{nom}
\begin{glose}
\pfra{crépuscule}
\end{glose}
\end{entrée}

\begin{entrée}{drovwe}{}{ⓔdrovwe}
\formephonétique{ɖoβe}
\région{GOs PA BO}
\variante{%
do-phe
\région{PA BO}}
(\domainesémantique{Terre})
\classe{nom}
\begin{glose}
\pfra{terre laissée par l'inondation ; terre d'alluvions}
\end{glose}
\end{entrée}

\begin{entrée}{drovwiju}{}{ⓔdrovwiju}
\formephonétique{ɖoβidʒu}
\région{GOs}
\variante{%
dopicu, dovio, duvio
\région{BO PA}}
(\domainesémantique{Outils})
\classe{nom}
\begin{glose}
\pfra{clou}
\end{glose}
\end{entrée}

\begin{entrée}{drua-ko}{}{ⓔdrua-ko}
\région{GOs PA}
(\domainesémantique{Verbes de déplacement et moyens de déplacement})
\classe{v}
\begin{glose}
\pfra{frayer (se) un chemin dans la brousse (pour se sauver)}
\end{glose}
\begin{glose}
\pfra{sauver (se) dans la brousse}
\end{glose}
\end{entrée}

\begin{entrée}{druali}{}{ⓔdruali}
\région{GOs}
(\domainesémantique{Mollusques})
\classe{nom}
\begin{glose}
\pfra{coquillage long qui s'enfonce dans le sable}
\end{glose}
\end{entrée}

\begin{entrée}{drube}{}{ⓔdrube}
\région{GOs}
\variante{%
dube
\région{PA}, 
cèvèroo
\région{WEM}}
(\domainesémantique{Mammifères})
\classe{nom}
\begin{glose}
\pfra{cerf}
\end{glose}
\end{entrée}

\begin{entrée}{drudruu}{}{ⓔdrudruu}
\formephonétique{ɖuɖuː}
\région{GOs}
(\domainesémantique{Mollusques})
\classe{nom}
\begin{glose}
\pfra{porte-montre}
\end{glose}
\nomscientifique{Chicoreus ramosus}
\end{entrée}

\newpage

\lettrine{
e
è
ẽ
ê
}\begin{entrée}{e}{}{ⓔe}
\région{GOs}
(\domainesémantique{Modalité, verbes modaux})
\classe{v}
\begin{glose}
\pfra{convenir ; être bien}
\end{glose}
\newline
\begin{exemple}
\région{GO}
\textbf{\pnua{ezoma e, nòme zo tree mònõ}}
\pfra{ce serait bien s'il fait beau demain}
\end{exemple}
\newline
\begin{exemple}
\région{BO}
\textbf{\pnua{kavwo e}}
\pfra{ce n'est pas ainsi; cela ne convient pas}
\end{exemple}
\end{entrée}

\begin{entrée}{e-}{}{ⓔe-}
\région{BO [BM]}
(\domainesémantique{Couples de parenté})
\classe{PREF (couple PAR)}
\begin{glose}
\pfra{parenté duelle}
\end{glose}
\newline
\begin{sous-entrée}{e-peebu}{ⓔe-ⓝe-peebu}
\begin{glose}
\pfra{grand-père et petit-fils}
\end{glose}
\end{sous-entrée}
\newline
\begin{sous-entrée}{e-poi-n}{ⓔe-ⓝe-poi-n}
\région{BO}
\begin{glose}
\pfra{père et fils, mère et enfant (BM)}
\end{glose}
\end{sous-entrée}
\newline
\begin{sous-entrée}{epoi epè-toma kolo-li}{ⓔe-ⓝepoi epè-toma kolo-li}
\begin{glose}
\pfra{tante paternelle et neveu ou nièce}
\end{glose}
\end{sous-entrée}
\end{entrée}

\begin{entrée}{-e}{}{ⓔ-e}
\région{GOs PA}
(\domainesémantique{Directionnels})
\classe{DIR (transverse)}
\begin{glose}
\pfra{en s'éloignant du locuteur (axe transverse) (d'une vallée à l'autre, traversant un cours d'eau)}
\end{glose}
\newline
\begin{exemple}
\région{GO}
\textbf{\pnua{e na-e pòi-je ce-la lai}}
\pfra{elle donne à ses enfants du riz}
\end{exemple}
\end{entrée}

\begin{entrée}{-è}{}{ⓔ-è}
\région{GOs}
(\domainesémantique{Démonstratifs})
\classe{DEIC.1}
\begin{glose}
\pfra{ce-ci}
\end{glose}
\newline
\begin{exemple}
\textbf{\pnua{ẽnõ-è}}
\pfra{cet enfant-ci}
\end{exemple}
\newline
\begin{exemple}
\textbf{\pnua{aazo-è}}
\pfra{ce chef-ci}
\end{exemple}
\newline
\begin{exemple}
\région{GOs}
\textbf{\pnua{ni khabu-è}}
\pfra{cette semaine}
\end{exemple}
\end{entrée}

\begin{entrée}{ê}{1}{ⓔêⓗ1}
\région{GOs}
(\domainesémantique{Démonstratifs})
\classe{ANAPH}
\begin{glose}
\pfra{là ; là-bas (inanimés ; absent mais connu des interlocuteurs)}
\end{glose}
\newline
\relationsémantique{Cf.}{\lien{ⓔ-òⓗ1}{-ò}}
\glosecourte{là Dx2}
\newline
\relationsémantique{Cf.}{\lien{}{ã, òli}}
\glosecourte{Dx3}
\end{entrée}

\begin{entrée}{ê}{2}{ⓔêⓗ2}
\formephonétique{ê}
\région{GOs}
\variante{%
èm
\formephonétique{ɛ̃m}
\région{PA BO}}
(\domainesémantique{Noms des plantes})
\classe{nom}
\begin{glose}
\pfra{canne à sucre}
\end{glose}
\nomscientifique{Saccharum officinarum (Graminées)}
\newline
\begin{sous-entrée}{tha ê}{ⓔêⓗ2ⓝtha ê}
\région{GO}
\begin{glose}
\pfra{cueillirla canne à sucre}
\end{glose}
\end{sous-entrée}
\newline
\begin{sous-entrée}{whizi ê}{ⓔêⓗ2ⓝwhizi ê}
\région{GO}
\begin{glose}
\pfra{manger de la canne à sucre}
\end{glose}
\end{sous-entrée}
\newline
\begin{sous-entrée}{whili èm}{ⓔêⓗ2ⓝwhili èm}
\région{PA}
\begin{glose}
\pfra{manger de la canne à sucre}
\end{glose}
\end{sous-entrée}
\end{entrée}

\begin{entrée}{èa?}{}{ⓔèa?}
\région{GOs}
\variante{%
ia?
\région{GO(s)}, 
ia, ya
\région{BO}}
(\domainesémantique{Interrogatifs})
\classe{INT}
\begin{glose}
\pfra{où?}
\end{glose}
\newline
\begin{exemple}
\textbf{\pnua{kò èa ?}}
\pfra{dans quelle forêt ?}
\end{exemple}
\newline
\begin{exemple}
\textbf{\pnua{e yu èa mwã ?}}
\pfra{où vit-il ?}
\end{exemple}
\newline
\begin{exemple}
\textbf{\pnua{jö ã-da èa mwã ?}}
\pfra{où montes-tu ?}
\end{exemple}
\newline
\begin{exemple}
\textbf{\pnua{jö uja na èa mwã ?}}
\pfra{d'où arrives-tu ?}
\end{exemple}
\newline
\begin{exemple}
\textbf{\pnua{ge èa mwã nye ẽnõ ?}}
\pfra{où est cet enfant ?}
\end{exemple}
\newline
\begin{exemple}
\textbf{\pnua{ge èa loto i jö ?}}
\pfra{où est ta voiture ?}
\end{exemple}
\newline
\begin{exemple}
\textbf{\pnua{ge èa thoomwã ã ?}}
\pfra{où est cette femme ?}
\end{exemple}
\newline
\begin{exemple}
\textbf{\pnua{ge nu èa mwã ?}}
\pfra{où/ à quel endroit suis-je ?}
\end{exemple}
\newline
\begin{exemple}
\région{GOs}
\textbf{\pnua{ge èa hèlè ?}}
\pfra{où se trouve le couteau ?}
\end{exemple}
\newline
\begin{exemple}
\région{GOs}
\textbf{\pnua{ge èa mò-jö ?}}
\pfra{où se trouve ta maison ?}
\end{exemple}
\newline
\begin{exemple}
\région{GOs}
\textbf{\pnua{e yu èa mwã ?}}
\pfra{où vit-il ?}
\end{exemple}
\newline
\begin{exemple}
\région{GOs}
\textbf{\pnua{a-yu èa mwã ?}}
\pfra{c'est un habitant d'où ?}
\end{exemple}
\newline
\begin{exemple}
\textbf{\pnua{ge-je èa caaja ?}}
\pfra{où est Papa ?}
\end{exemple}
\newline
\begin{exemple}
\textbf{\pnua{ge-jö èa ?}}
\pfra{où es-tu ?}
\end{exemple}
\newline
\begin{exemple}
\textbf{\pnua{ge-la èa ?}}
\pfra{où sont-ils ?}
\end{exemple}
\newline
\begin{exemple}
\textbf{\pnua{e a ja/ca èa loto-ã ?}}
\pfra{jusqu'où va cette voiture ?}
\end{exemple}
\newline
\begin{exemple}
\textbf{\pnua{e trê ja/ca èa ?}}
\pfra{jusqu'où court-il ?}
\end{exemple}
\newline
\begin{exemple}
\textbf{\pnua{la trê na èa mwã ?}}
\pfra{où sont-ils allés courir ?}
\end{exemple}
\newline
\relationsémantique{Cf.}{\lien{ⓔia ?}{ia ?}}
\glosecourte{où ?}
\newline
\relationsémantique{Cf.}{\lien{}{ça èa ? ja èa?}}
\glosecourte{jusqu'où ?}
\end{entrée}

\begin{entrée}{êba}{}{ⓔêba}
\région{GOs PA}
(\domainesémantique{Localisation})
\classe{LOC}
\begin{glose}
\pfra{là-bas latéralement}
\end{glose}
\newline
\begin{exemple}
\région{PA}
\textbf{\pnua{ge êba |}}
\pfra{il est là-bas (latéralement)}
\end{exemple}
\newline
\note{forme courte de: ène-ba}{grammaire}{}
\end{entrée}

\begin{entrée}{ebe}{}{ⓔebe}
\région{GOs}
\variante{%
mõõ-n
\région{PA}}
(\domainesémantique{Alliance})
\classe{nom}
\begin{glose}
\pfra{beau-frère}
\end{glose}
\newline
\begin{exemple}
\région{GO}
\textbf{\pnua{ebee-nu}}
\pfra{mon beau-frère}
\end{exemple}
\end{entrée}

\begin{entrée}{ebe ba-êgu}{}{ⓔebe ba-êgu}
\région{GOs}
\variante{%
mõõ-n thoomwã, mõõ-n dòòmwã
\région{PA}}
(\domainesémantique{Alliance})
\classe{nom}
\begin{glose}
\pfra{belle-soeur}
\end{glose}
\end{entrée}

\begin{entrée}{ebe-thoomwã kòlò}{}{ⓔebe-thoomwã kòlò}
\région{GOs}
\variante{%
epe
\région{GO}}
(\domainesémantique{Couples de parenté})
\classe{couple PAR}
\begin{glose}
\pfra{tante paternelle (tantine) et nièce}
\end{glose}
\newline
\begin{exemple}
\région{GO}
\textbf{\pnua{ebe-thoomwã kòlò-nu}}
\pfra{ma tante paternelle}
\end{exemple}
\newline
\begin{sous-entrée}{epe-thoomwã kolo-li}{ⓔebe-thoomwã kòlòⓝepe-thoomwã kolo-li}
\begin{glose}
\pfra{ils sont en relation de paternelle et neveu ou nièce}
\end{glose}
\end{sous-entrée}
\end{entrée}

\begin{entrée}{ebiigi}{}{ⓔebiigi}
\région{GOs BO}
\variante{%
biigi
\région{PA BO}}
(\domainesémantique{Parenté})
\classe{nom}
\begin{glose}
\pfra{cousin croisé de même sexe (aîné ou cadet):fils/fille de soeur de père}
\end{glose}
\begin{glose}
\pfra{fils/fille de frère de mère; fils de la soeur du père}
\end{glose}
\newline
\begin{exemple}
\région{PA}
\textbf{\pnua{i biigi-ny}}
\pfra{c'est mon cousin}
\end{exemple}
\newline
\begin{exemple}
\région{GO}
\textbf{\pnua{ebiigi-nu}}
\pfra{c'est mon cousin}
\end{exemple}
\newline
\begin{exemple}
\région{GO}
\textbf{\pnua{la pe-ebiigi}}
\pfra{ils sont cousins}
\end{exemple}
\newline
\relationsémantique{Cf.}{\lien{ⓔbibi}{bibi}}
\glosecourte{terme d'appellation}
\end{entrée}

\begin{entrée}{ebòli}{}{ⓔebòli}
\région{GOs PA BO}
(\domainesémantique{Directionnels})
\classe{LOC.DIR}
\begin{glose}
\pfra{là-bas en bas ; là-bas vers la mer (visible ou non)}
\end{glose}
\newline
\begin{exemple}
\région{GO}
\textbf{\pnua{e a-du ebòli kòli we za}}
\pfra{elle est descendue à la mer}
\end{exemple}
\newline
\begin{exemple}
\région{GO}
\textbf{\pnua{e a-du ebòli kòli kaaze}}
\pfra{elle est descendue à la mer}
\end{exemple}
\newline
\begin{exemple}
\région{GO}
\textbf{\pnua{eniza nye çö thaavwu piina-du èbòli bwabu ?}}
\pfra{quand es-tu allée en France pour la première fois ?}
\end{exemple}
\newline
\begin{exemple}
\région{PA}
\textbf{\pnua{e a-du ebòli bwabu}}
\pfra{elle est partie en France}
\end{exemple}
\end{entrée}

\begin{entrée}{êda}{}{ⓔêda}
\région{GO}
(\domainesémantique{Directionnels})
\classe{DEIC.DIR}
\begin{glose}
\pfra{là-bas en haut ; devant}
\end{glose}
\end{entrée}

\begin{entrée}{êdime}{}{ⓔêdime}
\région{GOs}
(\domainesémantique{Mollusques})
\classe{nom}
\begin{glose}
\pfra{bigorneau}
\end{glose}
\nomscientifique{Turbo petholatus}
\end{entrée}

\begin{entrée}{êdi-me}{}{ⓔêdi-me}
\région{GOs}
(\domainesémantique{Corps humain})
\classe{nom}
\begin{glose}
\pfra{oeil}
\end{glose}
\end{entrée}

\begin{entrée}{êdoa}{}{ⓔêdoa}
\formephonétique{ê'ndo.a}
\région{GOs}
\variante{%
êdo
\région{GO(s)}}
(\domainesémantique{Parties de plantes})
\classe{nom}
\begin{glose}
\pfra{graine}
\end{glose}
\begin{glose}
\pfra{noyau ; pépin}
\end{glose}
\end{entrée}

\begin{entrée}{êdu}{}{ⓔêdu}
\région{GO}
(\domainesémantique{Directions})
\classe{DEIC.DIR}
\begin{glose}
\pfra{là-bas en bas ; derrière}
\end{glose}
\end{entrée}

\begin{entrée}{êê-}{}{ⓔêê-}
\région{GOs PA BO [BM]}
(\domainesémantique{Cultures, techniques, boutures})
\classe{nom}
\begin{glose}
\pfra{plants}
\end{glose}
\newline
\begin{exemple}
\région{GOs}
\textbf{\pnua{êê-nu}}
\pfra{mes plants}
\end{exemple}
\newline
\begin{exemple}
\région{BO}
\textbf{\pnua{êê-ny}}
\pfra{mes plants}
\end{exemple}
\end{entrée}

\begin{entrée}{-êê}{}{ⓔ-êê}
\région{GO}
(\domainesémantique{Pronoms})
\classe{PRO 1° pers. duel incl. (OBJ ou POSS)}
\begin{glose}
\pfra{nous 2 ; nos}
\end{glose}
\end{entrée}

\begin{entrée}{êgi}{}{ⓔêgi}
\région{GOs}
(\domainesémantique{Mouvements ou actions faits avec le corps, les bras, les mains, les pieds})
\classe{v}
\begin{glose}
\pfra{attraper (qqch en mouvement: ballon, à la pêche, des crabes, poissons)}
\end{glose}
\newline
\begin{exemple}
\textbf{\pnua{e ne la-ã pwaixe vwo kebwa na mi êgi-li mwã}}
\pfra{elle fait tout cela pour nous empêcher de les attraper}
\end{exemple}
\newline
\begin{exemple}
\textbf{\pnua{cö êgi kaamwale ? - hê-pwe; hê-pwiò; hê-do}}
\pfra{tu les as attrappés comment? - à la ligne; à la senne; à la sagaie}
\end{exemple}
\end{entrée}

\begin{entrée}{êgo}{}{ⓔêgo}
\formephonétique{̃êŋgo}
\région{GOs}
\variante{%
pi-ko
\région{PA}}
(\domainesémantique{Oiseaux
, Poissons})
\classe{nom}
\begin{glose}
\pfra{oeuf (poule, poisson, crustacé)}
\end{glose}
\newline
\begin{sous-entrée}{êgo ko}{ⓔêgoⓝêgo ko}
\région{GOs}
\begin{glose}
\pfra{oeuf de poule}
\end{glose}
\end{sous-entrée}
\newline
\begin{sous-entrée}{êgo mebu}{ⓔêgoⓝêgo mebu}
\région{GOs}
\begin{glose}
\pfra{nid de guêpe}
\end{glose}
\end{sous-entrée}
\newline
\begin{sous-entrée}{êgo ulo}{ⓔêgoⓝêgo ulo}
\région{GOs}
\begin{glose}
\pfra{larve de sauterelle}
\end{glose}
\end{sous-entrée}
\end{entrée}

\begin{entrée}{êgo dili}{}{ⓔêgo dili}
\région{GOs}
(\domainesémantique{Cultures, techniques, boutures})
\classe{nom}
\begin{glose}
\pfra{motte de terre}
\end{glose}
\end{entrée}

\begin{entrée}{ẽgõgò}{}{ⓔẽgõgò}
\formephonétique{̃ɛ̃ŋgɔ̃ŋgɔ}
\région{GOs}
\variante{%
êgòl
\formephonétique{̃ɛ̃ŋgɔl}
\région{PA BO WEM WE}, 
êgògòn
\région{BO}}
(\domainesémantique{Temps})
\classe{ADV}
\begin{glose}
\pfra{autrefois ; il y a longtemps ; avant}
\end{glose}
\newline
\begin{sous-entrée}{whamã ẽgõgò}{ⓔẽgõgòⓝwhamã ẽgõgò}
\begin{glose}
\pfra{les vieux d'avant}
\end{glose}
\end{sous-entrée}
\end{entrée}

\begin{entrée}{êgòl}{}{ⓔêgòl}
\formephonétique{̃êŋgɔl}
\région{PA BO WEM WE}
\variante{%
êgõgò
\région{GO}}
(\domainesémantique{Temps})
\classe{ADV}
\begin{glose}
\pfra{autrefois ; il y a longtemps ; avant}
\end{glose}
\newline
\begin{exemple}
\région{PA}
\textbf{\pnua{êgòl êgòl}}
\pfra{il y a très longtemps}
\end{exemple}
\end{entrée}

\begin{entrée}{êgo mebu}{}{ⓔêgo mebu}
\région{GOs}
(\domainesémantique{Insectes})
\classe{nom}
\begin{glose}
\pfra{nid de guêpe}
\end{glose}
\end{entrée}

\begin{entrée}{êgo ulò}{}{ⓔêgo ulò}
\région{GOs}
(\domainesémantique{Insectes})
\classe{nom}
\begin{glose}
\pfra{larve de sauterelle}
\end{glose}
\end{entrée}

\begin{entrée}{egu}{}{ⓔegu}
\région{GO}
\variante{%
eku
\région{GO}}
(\domainesémantique{Agent})
\classe{AGT}
\begin{glose}
\pfra{agent}
\end{glose}
\end{entrée}

\begin{entrée}{êgu}{1}{ⓔêguⓗ1}
\région{GOs BO}
(\domainesémantique{Société})
\classe{nom}
\begin{glose}
\pfra{homme ; personne}
\end{glose}
\newline
\begin{exemple}
\région{BO}
\textbf{\pnua{yo êgu va ?}}
\pfra{d'où es-tu?}
\end{exemple}
\newline
\begin{exemple}
\région{PA}
\textbf{\pnua{Haxe novwo êgu-n, ca ka mhwã nõõli}}
\pfra{mais quant à sa personne, on ne la voit pas}
\end{exemple}
\newline
\begin{sous-entrée}{êgu hayu}{ⓔêguⓗ1ⓝêgu hayu}
\begin{glose}
\pfra{un homme quelconque}
\end{glose}
\end{sous-entrée}
\newline
\begin{sous-entrée}{êgu polo [BO]}{ⓔêguⓗ1ⓝêgu polo [BO]}
\begin{glose}
\pfra{albinos}
\end{glose}
\end{sous-entrée}
\newline
\begin{sous-entrée}{êgu ni nipa}{ⓔêguⓗ1ⓝêgu ni nipa}
\begin{glose}
\pfra{un homme fou}
\end{glose}
\end{sous-entrée}
\end{entrée}

\begin{entrée}{êgu}{2}{ⓔêguⓗ2}
\région{GO}
(\domainesémantique{Fonctions naturelles humaines})
\classe{v}
\begin{glose}
\pfra{sobre}
\end{glose}
\newline
\begin{exemple}
\textbf{\pnua{la za peve êgu}}
\pfra{ils sont tous sobres}
\end{exemple}
\end{entrée}

\begin{entrée}{êgu-zo}{}{ⓔêgu-zo}
\région{GOs}
\variante{%
ayò
\région{BO PA}}
(\domainesémantique{Caractéristiques et propriétés des personnes})
\classe{v.stat.}
\begin{glose}
\pfra{joli ; beau (personne, chose)}
\end{glose}
\end{entrée}

\begin{entrée}{èjè-ã!}{}{ⓔèjè-ã!}
\formephonétique{̃̃̃̃̃̃̃ɛjɛ.ɛ̃}
\région{GOs}
\newline
\sens{1}
(\domainesémantique{Démonstratifs})
\classe{DEM}
\begin{glose}
\pfra{cette femme-ci}
\end{glose}
\newline
\begin{sous-entrée}{èjè-eni}{ⓔèjè-ã!ⓢ1ⓝèjè-eni}
\begin{glose}
\pfra{cette femme-là (DX2)}
\end{glose}
\end{sous-entrée}
\newline
\begin{sous-entrée}{èjè-ba}{ⓔèjè-ã!ⓢ1ⓝèjè-ba}
\begin{glose}
\pfra{cette femme-là (DX2 sur le côté)}
\end{glose}
\end{sous-entrée}
\newline
\begin{sous-entrée}{èjè-õli}{ⓔèjè-ã!ⓢ1ⓝèjè-õli}
\begin{glose}
\pfra{cette femme-là-bas (DX3)}
\end{glose}
\end{sous-entrée}
\newline
\begin{sous-entrée}{èjè-du mu}{ⓔèjè-ã!ⓢ1ⓝèjè-du mu}
\begin{glose}
\pfra{cette femme-là derrière}
\end{glose}
\end{sous-entrée}
\newline
\begin{sous-entrée}{èjè-èda}{ⓔèjè-ã!ⓢ1ⓝèjè-èda}
\begin{glose}
\pfra{cette femme-là-haut}
\end{glose}
\end{sous-entrée}
\newline
\begin{sous-entrée}{èjè-èdu}{ⓔèjè-ã!ⓢ1ⓝèjè-èdu}
\begin{glose}
\pfra{cette femme-là en bas}
\end{glose}
\end{sous-entrée}
\newline
\begin{sous-entrée}{èjè-bòli}{ⓔèjè-ã!ⓢ1ⓝèjè-bòli}
\begin{glose}
\pfra{cette femme-là loin en bas}
\end{glose}
\end{sous-entrée}
\newline
\sens{2}
(\domainesémantique{Interpellation})
\begin{glose}
\pfra{eh ! la femme !}
\end{glose}
\end{entrée}

\begin{entrée}{e-jeni}{}{ⓔe-jeni}
\région{GOs PA BO}
(\domainesémantique{Pronoms})
\classe{DEM}
\begin{glose}
\pfra{voilà (le) pas loin}
\end{glose}
\end{entrée}

\begin{entrée}{èla-ã}{}{ⓔèla-ã}
\région{GOs PA}
(\domainesémantique{Démonstratifs})
\classe{DEM.DEIC.1}
\begin{glose}
\pfra{ceux-ci}
\end{glose}
\newline
\begin{sous-entrée}{èla-èni}{ⓔèla-ãⓝèla-èni}
\begin{glose}
\pfra{ceux-là (DX2)}
\end{glose}
\end{sous-entrée}
\newline
\begin{sous-entrée}{èla-èba}{ⓔèla-ãⓝèla-èba}
\begin{glose}
\pfra{ceux-là (DX2 sur le côté)}
\end{glose}
\end{sous-entrée}
\newline
\begin{sous-entrée}{èla-õli}{ⓔèla-ãⓝèla-õli}
\begin{glose}
\pfra{ceux-là-bas (DX3)}
\end{glose}
\end{sous-entrée}
\newline
\begin{sous-entrée}{èla-èdu mu}{ⓔèla-ãⓝèla-èdu mu}
\begin{glose}
\pfra{ceux-là derrière}
\end{glose}
\end{sous-entrée}
\newline
\begin{sous-entrée}{èla-èda}{ⓔèla-ãⓝèla-èda}
\begin{glose}
\pfra{ceux-là-haut}
\end{glose}
\end{sous-entrée}
\newline
\begin{sous-entrée}{èla-èdu}{ⓔèla-ãⓝèla-èdu}
\begin{glose}
\pfra{ceux-là en bas}
\end{glose}
\end{sous-entrée}
\newline
\begin{sous-entrée}{èla-èbòli}{ⓔèla-ãⓝèla-èbòli}
\begin{glose}
\pfra{ceux-là en bas loin}
\end{glose}
\end{sous-entrée}
\end{entrée}

\begin{entrée}{èla-õli}{}{ⓔèla-õli}
\région{GOs PA}
(\domainesémantique{Démonstratifs})
\classe{DEM.DEIC.1}
\begin{glose}
\pfra{ceux-là-bas (DX3)}
\end{glose}
\end{entrée}

\begin{entrée}{ele}{}{ⓔele}
\région{GOs BO}
(\domainesémantique{Mouvements ou actions faits avec le corps, les bras, les mains, les pieds})
\classe{v}
\begin{glose}
\pfra{pousser (qqch avec la main)}
\end{glose}
\newline
\begin{exemple}
\région{GO}
\textbf{\pnua{po ele-mi}}
\pfra{pousse-le un peu vers moi}
\end{exemple}
\newline
\begin{exemple}
\région{GO}
\textbf{\pnua{po ele-ò}}
\pfra{pousse-le un peu vers là}
\end{exemple}
\end{entrée}

\begin{entrée}{èli-ã}{}{ⓔèli-ã}
\région{GOs}
(\domainesémantique{Démonstratifs})
\classe{DEM.DEIC.1}
\begin{glose}
\pfra{ces2-ci}
\end{glose}
\newline
\begin{sous-entrée}{èli-èni}{ⓔèli-ãⓝèli-èni}
\begin{glose}
\pfra{ces2-là (DX2)}
\end{glose}
\end{sous-entrée}
\newline
\begin{sous-entrée}{èli-èba}{ⓔèli-ãⓝèli-èba}
\begin{glose}
\pfra{ces2-là (DX2 sur le côté)}
\end{glose}
\end{sous-entrée}
\newline
\begin{sous-entrée}{èli-õli}{ⓔèli-ãⓝèli-õli}
\begin{glose}
\pfra{ces2-là-bas (DX3)}
\end{glose}
\end{sous-entrée}
\newline
\begin{sous-entrée}{èli-èdu mu}{ⓔèli-ãⓝèli-èdu mu}
\begin{glose}
\pfra{ces2-là derrière}
\end{glose}
\end{sous-entrée}
\newline
\begin{sous-entrée}{èli-èda,}{ⓔèli-ãⓝèli-èda,}
\begin{glose}
\pfra{ces2-là-haut}
\end{glose}
\end{sous-entrée}
\newline
\begin{sous-entrée}{èli-èdu}{ⓔèli-ãⓝèli-èdu}
\begin{glose}
\pfra{ces2-là en bas}
\end{glose}
\end{sous-entrée}
\newline
\begin{sous-entrée}{èli-ebòli}{ⓔèli-ãⓝèli-ebòli}
\begin{glose}
\pfra{ces2-là loin en bas}
\end{glose}
\end{sous-entrée}
\end{entrée}

\begin{entrée}{èlò}{}{ⓔèlò}
\région{GO PA BO}
\classe{INTJ ; v}
\newline
\sens{1}
(\domainesémantique{Marques assertives})
\begin{glose}
\pfra{oui}
\end{glose}
\newline
\sens{2}
(\domainesémantique{Fonctions intellectuelles})
\begin{glose}
\pfra{accepter ; dire oui}
\end{glose}
\newline
\begin{exemple}
\région{PA}
\textbf{\pnua{i eloge}}
\pfra{il l'a accepté}
\end{exemple}
\newline
\relationsémantique{Cf.}{\lien{}{eloge [PA]}}
\glosecourte{accepter, acquiescer}
\newline
\relationsémantique{Ant.}{\lien{}{ai}}
\glosecourte{non}
\end{entrée}

\begin{entrée}{èlò-ã}{}{ⓔèlò-ã}
\région{GOs}
(\domainesémantique{Démonstratifs})
\classe{DEM.DEIC.1}
\begin{glose}
\pfra{ces3-ci}
\end{glose}
\newline
\begin{sous-entrée}{èlò-èni}{ⓔèlò-ãⓝèlò-èni}
\begin{glose}
\pfra{ces3-là (DX2)}
\end{glose}
\end{sous-entrée}
\newline
\begin{sous-entrée}{èlò-èba}{ⓔèlò-ãⓝèlò-èba}
\begin{glose}
\pfra{ces3-là (DX2 sur le côté)}
\end{glose}
\end{sous-entrée}
\newline
\begin{sous-entrée}{èlò-õli}{ⓔèlò-ãⓝèlò-õli}
\begin{glose}
\pfra{ces3-là-bas (DX3)}
\end{glose}
\end{sous-entrée}
\newline
\begin{sous-entrée}{èlò-èdu mu}{ⓔèlò-ãⓝèlò-èdu mu}
\begin{glose}
\pfra{ces3-là derrière,}
\end{glose}
\end{sous-entrée}
\newline
\begin{sous-entrée}{èlò-èda}{ⓔèlò-ãⓝèlò-èda}
\begin{glose}
\pfra{ces3-là-haut}
\end{glose}
\end{sous-entrée}
\newline
\begin{sous-entrée}{èlò-èdu}{ⓔèlò-ãⓝèlò-èdu}
\begin{glose}
\pfra{ces3 là en bas}
\end{glose}
\end{sous-entrée}
\newline
\begin{sous-entrée}{èlò-èbòli}{ⓔèlò-ãⓝèlò-èbòli}
\begin{glose}
\pfra{ces3-là loin en bas}
\end{glose}
\end{sous-entrée}
\end{entrée}

\begin{entrée}{eloe}{}{ⓔeloe}
\région{GA}
(\domainesémantique{Préparation des aliments; modes de préparation et de cuisson})
\classe{v}
\begin{glose}
\pfra{couper en lamelle}
\end{glose}
\end{entrée}

\begin{entrée}{èlòè !}{}{ⓔèlòè !}
\région{GOs}
(\domainesémantique{Interpellation
, Démonstratifs})
\classe{INTJ}
\begin{glose}
\pfra{eh ! vous! (triel)}
\end{glose}
\end{entrée}

\begin{entrée}{emãli}{}{ⓔemãli}
\région{GOs}
\classe{DEM.duel}
\newline
\sens{1}
(\domainesémantique{Démonstratifs})
\begin{glose}
\pfra{voilà ces deux-là}
\end{glose}
\newline
\sens{2}
(\domainesémantique{Interpellation})
\begin{glose}
\pfra{eh vous deux !}
\end{glose}
\end{entrée}

\begin{entrée}{emãlo}{}{ⓔemãlo}
\région{GOs}
(\domainesémantique{Démonstratifs})
\classe{DEM.triel}
\begin{glose}
\pfra{voilà ces trois-là}
\end{glose}
\newline
\begin{exemple}
\région{GO}
\textbf{\pnua{i emãlo}}
\pfra{où sont les autres ?}
\end{exemple}
\end{entrée}

\begin{entrée}{è-mõõ}{}{ⓔè-mõõ}
\région{GOs}
\variante{%
mõõn
\région{PA}}
(\domainesémantique{Couples de parenté})
\classe{couple PAR}
\begin{glose}
\pfra{beaux-parents: beau-père (d'épouse ou de mari) ; belle-mère (d'épouse ou de mari)}
\end{glose}
\begin{glose}
\pfra{beau-père et beau-fils}
\end{glose}
\newline
\begin{exemple}
\région{PA}
\textbf{\pnua{mõõn i je}}
\pfra{ses beaux-parents}
\end{exemple}
\newline
\begin{exemple}
\région{PA}
\textbf{\pnua{li e-mõõn}}
\pfra{ils sont beaux-parents et beaux-enfants}
\end{exemple}
\end{entrée}

\begin{entrée}{e-mõû}{}{ⓔe-mõû}
\région{GOs}
\variante{%
e-mõû-n
\région{PA BO}}
(\domainesémantique{Couples de parenté})
\classe{couple PAR}
\begin{glose}
\pfra{époux (mari et femme)}
\end{glose}
\newline
\begin{exemple}
\région{PA}
\textbf{\pnua{li e-mõû-n}}
\pfra{ils sont mari et femme}
\end{exemple}
\newline
\begin{exemple}
\région{GO}
\textbf{\pnua{li e-mõû-li}}
\pfra{ils sont mari et femme}
\end{exemple}
\end{entrée}

\begin{entrée}{êmwê}{1}{ⓔêmwêⓗ1}
\région{GOs}
\variante{%
êmwèn
\région{PA BO}}
(\domainesémantique{Société})
\classe{nom}
\begin{glose}
\pfra{homme ; mâle}
\end{glose}
\newline
\begin{sous-entrée}{ẽnô tòòmwa}{ⓔêmwêⓗ1ⓝẽnô tòòmwa}
\begin{glose}
\pfra{fille}
\end{glose}
\end{sous-entrée}
\newline
\begin{sous-entrée}{ẽnò emwê}{ⓔêmwêⓗ1ⓝẽnò emwê}
\begin{glose}
\pfra{garçon}
\end{glose}
\end{sous-entrée}
\newline
\begin{sous-entrée}{ẽnô èmwèn [PA]}{ⓔêmwêⓗ1ⓝẽnô èmwèn [PA]}
\begin{glose}
\pfra{garçon}
\end{glose}
\newline
\begin{exemple}
\textbf{\pnua{abaa-ny èmwèn [PA]}}
\pfra{mon frère}
\end{exemple}
\newline
\begin{exemple}
\textbf{\pnua{abaa-ny tòòmwa [PA]}}
\pfra{ma soeur}
\end{exemple}
\end{sous-entrée}
\newline
\begin{sous-entrée}{hi-n tòòmwa [PA]}{ⓔêmwêⓗ1ⓝhi-n tòòmwa [PA]}
\begin{glose}
\pfra{pouce}
\end{glose}
\end{sous-entrée}
\newline
\begin{sous-entrée}{hi-n èmwèn [PA]}{ⓔêmwêⓗ1ⓝhi-n èmwèn [PA]}
\begin{glose}
\pfra{index}
\end{glose}
\end{sous-entrée}
\end{entrée}

\begin{entrée}{êmwê}{2}{ⓔêmwêⓗ2}
\région{GOs BO}
\variante{%
êmwèn
\région{PA}}
(\domainesémantique{Parties du corps humain : doigts, orteil})
\classe{nom}
\begin{glose}
\pfra{index (main)}
\end{glose}
\newline
\relationsémantique{Cf.}{\lien{}{thizii [GOs], thiri [PA], tiriai-n}}
\glosecourte{auriculaire}
\end{entrée}

\begin{entrée}{êmwen kòlò-}{}{ⓔêmwen kòlò-}
\région{PA BO}
(\domainesémantique{Parenté})
\classe{nom}
\begin{glose}
\pfra{fils de frère (lit. garçon de mon côté)}
\end{glose}
\end{entrée}

\begin{entrée}{è-...-n}{}{ⓔè-...-n}
\région{PA}
(\domainesémantique{Couples de parenté})
\classe{couple PAR}
\begin{glose}
\pfra{parenté réciproque}
\end{glose}
\newline
\begin{exemple}
\textbf{\pnua{è-pòi-n}}
\pfra{père et fils}
\end{exemple}
\newline
\begin{exemple}
\textbf{\pnua{è-mõû-n}}
\pfra{les époux, le couple}
\end{exemple}
\end{entrée}

\begin{entrée}{êńa}{}{ⓔêńa}
\formephonétique{êna}
\région{GOs PA BO}
(\domainesémantique{Localisation})
\classe{ADV.LOC (spatio-temporel)}
\begin{glose}
\pfra{ici}
\end{glose}
\newline
\begin{exemple}
\région{GOs}
\textbf{\pnua{na êńa}}
\pfra{à cet endroit-ci}
\end{exemple}
\newline
\begin{exemple}
\région{GOs}
\textbf{\pnua{jo yu êńa ?}}
\pfra{tu habites à cet endroit-ci ?}
\end{exemple}
\end{entrée}

\begin{entrée}{ênã}{}{ⓔênã}
\formephonétique{êɳɛ̃}
\région{GOs BO}
\variante{%
ènan
\région{BO [BM]}, 
kaalu
\région{GO(s)}}
(\domainesémantique{Santé, maladie})
\classe{v ; n}
\begin{glose}
\pfra{blessé}
\end{glose}
\newline
\begin{exemple}
\région{BO}
\textbf{\pnua{i thu ènã}}
\pfra{il a eu un accident [BM]}
\end{exemple}
\end{entrée}

\begin{entrée}{ênè}{1}{ⓔênèⓗ1}
\formephonétique{êɳɛ}
\région{GOs BO}
(\domainesémantique{Localisation})
\classe{n.LOC}
\begin{glose}
\pfra{endroit où}
\end{glose}
\newline
\begin{exemple}
\région{BO}
\textbf{\pnua{iru a ênè no kobwe}}
\pfra{il ira là où je dis}
\end{exemple}
\newline
\begin{exemple}
\région{GOs}
\textbf{\pnua{nu hivwine ènè e a le}}
\pfra{j'ignore l'endroit où il est allé}
\end{exemple}
\newline
\begin{exemple}
\région{GOs}
\textbf{\pnua{nu porome ènè e yu (le)}}
\pfra{j'ai oublié où il habite}
\end{exemple}
\end{entrée}

\begin{entrée}{ênè}{2}{ⓔênèⓗ2}
\formephonétique{êɳɛ}
\région{GOs BO}
\variante{%
enã
\région{PA}}
(\domainesémantique{Localisation})
\classe{ADV}
\begin{glose}
\pfra{ici}
\end{glose}
\end{entrée}

\begin{entrée}{êne-ba}{}{ⓔêne-ba}
\formephonétique{êɳɛba}
\région{GOs}
\variante{%
èba
\région{GO(s)}}
(\domainesémantique{Directionnels})
\classe{LOC.DEIC.2}
\begin{glose}
\pfra{là (sur le côté, latéralement)}
\end{glose}
\end{entrée}

\begin{entrée}{ênê-da}{}{ⓔênê-da}
\formephonétique{êɳêda}
\région{GOs}
\variante{%
ènîda
\région{WEM}}
(\domainesémantique{Directionnels})
\classe{DIR}
\begin{glose}
\pfra{vers le haut ; en amont ; vers la montagne}
\end{glose}
\begin{glose}
\pfra{vers la terre}
\end{glose}
\begin{glose}
\pfra{au sud ; à l'est}
\end{glose}
\end{entrée}

\begin{entrée}{ênê-du}{}{ⓔênê-du}
\formephonétique{êɳêdu}
\région{GOs}
(\domainesémantique{Directionnels})
\classe{LOC.DIR}
\begin{glose}
\pfra{bas (en) ; vers le bas ; en aval ; vers la mer}
\end{glose}
\begin{glose}
\pfra{au nord ; à l'ouest}
\end{glose}
\newline
\begin{exemple}
\textbf{\pnua{ênê-du mwã}}
\pfra{très loin en aval}
\end{exemple}
\end{entrée}

\begin{entrée}{ênè-ò}{}{ⓔênè-ò}
\formephonétique{êɳɛɔ}
\région{GOs}
(\domainesémantique{Localisation})
\classe{LOC.ANAPH}
\begin{glose}
\pfra{endroit-là (connu des locuteurs)}
\end{glose}
\end{entrée}

\begin{entrée}{êni}{}{ⓔêni}
\formephonétique{êɳi}
\région{GOs}
\variante{%
enim
\région{PA}}
(\domainesémantique{Localisation})
\classe{ADV.LOC.DEIC.2}
\begin{glose}
\pfra{là}
\end{glose}
\newline
\relationsémantique{Cf.}{\lien{ⓔênèⓗ1}{ênè}}
\glosecourte{ici}
\end{entrée}

\begin{entrée}{èńiza ?}{}{ⓔèńiza ?}
\formephonétique{ɛńiða (dental)}
\région{GOs}
\variante{%
ènira ?
\région{WEM WE PA}, 
inira ?
\région{BO}}
(\domainesémantique{Interrogatifs})
\classe{INT}
\begin{glose}
\pfra{quand? (passé et futur)}
\end{glose}
\newline
\begin{exemple}
\région{BO}
\textbf{\pnua{yo a-mi inira ?}}
\pfra{quand arrives-tu ?}
\end{exemple}
\newline
\étymologie{
\langue{POc}
\étymon{*ŋainsa, *pinsa}}
\end{entrée}

\begin{entrée}{eńiza-mwã}{}{ⓔeńiza-mwã}
\formephonétique{ɛńiða (dental)}
\région{GO}
\variante{%
inira-mwã
\région{BO}}
(\domainesémantique{Temps})
\classe{ADV}
\begin{glose}
\pfra{indéfiniment ; un jour}
\end{glose}
\end{entrée}

\begin{entrée}{ẽnõ}{1}{ⓔẽnõⓗ1}
\formephonétique{ɛ̃ɳɔ̃}
\région{GOs}
\variante{%
ênõ
\région{PA BO}}
\classe{nom}
\newline
\sens{1}
(\domainesémantique{Cours de la vie})
\begin{glose}
\pfra{jeune ; petit}
\end{glose}
\newline
\begin{exemple}
\textbf{\pnua{nu po ẽnõ nai çö}}
\pfra{je suis un peu plus jeune que toi}
\end{exemple}
\newline
\begin{sous-entrée}{ẽnõ-êmwê}{ⓔẽnõⓗ1ⓢ1ⓝẽnõ-êmwê}
\région{GO}
\begin{glose}
\pfra{garçon}
\end{glose}
\end{sous-entrée}
\newline
\begin{sous-entrée}{ẽnõ-emwèn}{ⓔẽnõⓗ1ⓢ1ⓝẽnõ-emwèn}
\région{PA}
\begin{glose}
\pfra{garçon}
\end{glose}
\end{sous-entrée}
\newline
\begin{sous-entrée}{ẽnõ-zoomwã}{ⓔẽnõⓗ1ⓢ1ⓝẽnõ-zoomwã}
\région{GO}
\begin{glose}
\pfra{fille}
\end{glose}
\end{sous-entrée}
\newline
\begin{sous-entrée}{ẽnõ-roomwã}{ⓔẽnõⓗ1ⓢ1ⓝẽnõ-roomwã}
\région{WEM}
\begin{glose}
\pfra{fille}
\end{glose}
\end{sous-entrée}
\newline
\begin{sous-entrée}{ẽnõ-ba-êgu}{ⓔẽnõⓗ1ⓢ1ⓝẽnõ-ba-êgu}
\région{GO}
\begin{glose}
\pfra{fille}
\end{glose}
\end{sous-entrée}
\newline
\sens{2}
(\domainesémantique{Parenté})
\begin{glose}
\pfra{enfant (âge)}
\end{glose}
\newline
\begin{exemple}
\textbf{\pnua{pòi-nu ẽnõ}}
\pfra{mon dernier enfant}
\end{exemple}
\newline
\begin{sous-entrée}{ẽnõ ni gò}{ⓔẽnõⓗ1ⓢ2ⓝẽnõ ni gò}
\begin{glose}
\pfra{le puîné (enfant du milieu)}
\end{glose}
\end{sous-entrée}
\end{entrée}

\begin{entrée}{ẽnõ}{2}{ⓔẽnõⓗ2}
\formephonétique{ɛ̃ɳɔ̃}
\région{GOs}
\variante{%
ènõ
\région{PA}}
(\domainesémantique{Parenté})
\classe{nom}
\begin{glose}
\pfra{soeur de père ; tante paternelle ("tantine")}
\end{glose}
\begin{glose}
\pfra{épouse du frère de mère}
\end{glose}
\begin{glose}
\pfra{cousine du père ;}
\end{glose}
\begin{glose}
\pfra{épouse des cousins de mère}
\end{glose}
\begin{glose}
\pfra{neveu ou nièce de la soeur du père}
\end{glose}
\newline
\begin{exemple}
\textbf{\pnua{ẽnõõ i nu}}
\pfra{ma tante paternelle}
\end{exemple}
\newline
\begin{exemple}
\textbf{\pnua{nõnõ !}}
\pfra{tantine (appellation)}
\end{exemple}
\end{entrée}

\begin{entrée}{ẽnõbau}{}{ⓔẽnõbau}
\formephonétique{ɛ̃ɳɔ̃bau}
\région{GOs}
(\domainesémantique{Description des objets, formes, consistance, taille})
\classe{v}
\begin{glose}
\pfra{agréable ; beau ; merveilleux (termes de l'église)}
\end{glose}
\end{entrée}

\begin{entrée}{ẽnõli}{}{ⓔẽnõli}
\formephonétique{ɛ̃ɳɔ̃li}
\région{GOs}
\variante{%
ènõli
\région{PA BO}}
(\domainesémantique{Directions})
\classe{LOC.3}
\begin{glose}
\pfra{là-bas ; au-delà (invisible)}
\end{glose}
\newline
\begin{exemple}
\textbf{\pnua{ge je ẽnõli}}
\pfra{il est là-bas}
\end{exemple}
\newline
\begin{exemple}
\textbf{\pnua{ge je ẽnõli mwã}}
\pfra{il est loin là-bas}
\end{exemple}
\end{entrée}

\begin{entrée}{èńoma}{}{ⓔèńoma}
\formephonétique{ɛńoma}
\région{PA BO [BM]}
(\domainesémantique{Pronoms})
\classe{DEM}
\begin{glose}
\pfra{les autres}
\end{glose}
\newline
\begin{exemple}
\textbf{\pnua{ge vhaa èńoma}}
\pfra{les autres parlent}
\end{exemple}
\end{entrée}

\begin{entrée}{ẽnõ ni gò}{}{ⓔẽnõ ni gò}
\formephonétique{ɛ̃ɳɔ̃ ni gɔ}
\région{GO PA}
(\domainesémantique{Parenté})
\classe{nom}
\begin{glose}
\pfra{puîné (lit. enfant du milieu)}
\end{glose}
\end{entrée}

\begin{entrée}{ẽnõ xa pogabe}{}{ⓔẽnõ xa pogabe}
\formephonétique{ɛ̃ɳɔ̃}
\région{GOs}
(\domainesémantique{Cours de la vie})
\classe{nom}
\begin{glose}
\pfra{nourrisson ; nouveau-né (lit. enfant qui est très petit)}
\end{glose}
\end{entrée}

\begin{entrée}{ênuda}{}{ⓔênuda}
\formephonétique{'êɳuda}
\région{GOs BO}
(\domainesémantique{Directions})
\classe{LOC}
\begin{glose}
\pfra{là-haut ; en haut (sur la montagne) ; au-dessus}
\end{glose}
\newline
\begin{exemple}
\région{GO}
\textbf{\pnua{mo za thrôbo-du mwã na (ê)nuda}}
\pfra{nous descendions de là-haut}
\end{exemple}
\newline
\begin{exemple}
\région{BO}
\textbf{\pnua{ênuda mwã}}
\pfra{là-bas en haut loin}
\end{exemple}
\end{entrée}

\begin{entrée}{e-nye}{}{ⓔe-nye}
\région{GOs PA BO}
(\domainesémantique{Présentatifs})
\classe{DEM}
\begin{glose}
\pfra{voici (le)}
\end{glose}
\end{entrée}

\begin{entrée}{e-nye-ba}{}{ⓔe-nye-ba}
\région{PA}
(\domainesémantique{Présentatifs})
\classe{DEM}
\begin{glose}
\pfra{voilà (le) là (on le montre et on regarde dans sa direction)}
\end{glose}
\end{entrée}

\begin{entrée}{e-nye-boli}{}{ⓔe-nye-boli}
\région{PA}
(\domainesémantique{Présentatifs})
\classe{DEM.DEIC.3}
\begin{glose}
\pfra{voilà (le) là-bas (à un endroit plus bas que là où on est)}
\end{glose}
\end{entrée}

\begin{entrée}{e-nyoli}{}{ⓔe-nyoli}
\région{GOs PA}
(\domainesémantique{Démonstratifs
, Présentatifs})
\classe{DEM}
\begin{glose}
\pfra{voilà (le) ; celui-là (visible)}
\end{glose}
\end{entrée}

\begin{entrée}{e-nyu-da}{}{ⓔe-nyu-da}
\région{PA}
(\domainesémantique{Présentatifs})
\classe{LOC}
\begin{glose}
\pfra{voilà (le) en haut}
\end{glose}
\newline
\relationsémantique{Cf.}{\lien{}{e-nyu-du}}
\glosecourte{voilà (le) en bas}
\end{entrée}

\begin{entrée}{epè-}{}{ⓔepè-}
\région{GO}
(\domainesémantique{Couples de parenté})
\classe{PREF (couple PAR)}
\begin{glose}
\pfra{couple de relation}
\end{glose}
\newline
\begin{exemple}
\textbf{\pnua{e pè-be-yaza}}
\pfra{ils sont deux homonymes}
\end{exemple}
\newline
\begin{exemple}
\textbf{\pnua{li e-peebu}}
\pfra{ils sont le grand-père et petit-fils}
\end{exemple}
\newline
\begin{exemple}
\textbf{\pnua{li epè-me}}
\pfra{ils sont frère et soeur}
\end{exemple}
\newline
\begin{exemple}
\textbf{\pnua{li epè-è-mõõ}}
\pfra{ils sont belle-mère et gendre}
\end{exemple}
\newline
\begin{exemple}
\textbf{\pnua{li epè-è-pööni}}
\pfra{oncle utérin et neveu}
\end{exemple}
\end{entrée}

\begin{entrée}{e-peebu}{}{ⓔe-peebu}
\formephonétique{evweːbu}
\région{GOs}
\variante{%
e-veebu
\région{GO(s)}, 
e-veebu-n
\région{PA}}
(\domainesémantique{Couples de parenté})
\classe{couple PAR}
\begin{glose}
\pfra{grand-père ou grand-mère et petit-fils ou petite-fille}
\end{glose}
\newline
\begin{exemple}
\région{GO}
\textbf{\pnua{bi e-veebu}}
\pfra{nous sommes grand-père (ou) grand-mère et petit-fils (ou) petite-fille}
\end{exemple}
\newline
\begin{exemple}
\région{PA}
\textbf{\pnua{li e-veebu-n}}
\pfra{ils sont grand-père (ou) grand-mère et petit-fils (ou) petite-fille}
\end{exemple}
\end{entrée}

\begin{entrée}{e-pòi}{}{ⓔe-pòi}
\région{GOs}
\variante{%
e-vwòi
\formephonétique{evwɔi}
\région{GO(s)}, 
e-pòi-n
\région{PA BO}}
\classe{couple PAR}
\newline
\sens{1}
(\domainesémantique{Couples de parenté})
\begin{glose}
\pfra{père et fils (ou) fille}
\end{glose}
\begin{glose}
\pfra{mère et fils (ou) fille}
\end{glose}
\newline
\begin{exemple}
\textbf{\pnua{li e-pòi}}
\pfra{ils sont père (ou) mère et fils (ou) fille}
\end{exemple}
\newline
\begin{exemple}
\région{PA}
\textbf{\pnua{li e-pòi-n}}
\pfra{ils sont père (ou) mère et fils (ou) fille}
\end{exemple}
\newline
\sens{2}
(\domainesémantique{Couples de parenté})
\begin{glose}
\pfra{soeur de la mère et neveu (ou) nièce}
\end{glose}
\begin{glose}
\pfra{frère du père et neveu (ou) nièce}
\end{glose}
\newline
\begin{exemple}
\région{PA}
\textbf{\pnua{li e-pòi-n}}
\pfra{ils sont tante et neveu/nièce}
\end{exemple}
\newline
\begin{exemple}
\région{GO}
\textbf{\pnua{li e-vwòi}}
\pfra{ils sont tante et neveu/nièce}
\end{exemple}
\end{entrée}

\begin{entrée}{è-pööni}{}{ⓔè-pööni}
\formephonétique{ɛvwωːɳi}
\région{GOs}
\variante{%
è-vwööni
\région{GO(s)}, 
è-pööni-n
\formephonétique{ɛpuːnin}
\région{PA}}
(\domainesémantique{Couples de parenté})
\classe{n ; couple PAR}
\begin{glose}
\pfra{oncle maternel et neveu/nièce maternel}
\end{glose}
\newline
\note{PA n'utilise "è-pööni-n" que pour la parenté réciproque; sinon "pööni-n" désigne l'oncle maternel.}{glose}{}
\newline
\begin{exemple}
\région{PA}
\textbf{\pnua{li è-pööni}}
\pfra{ils sont en relation d'oncle et neveu/nièce maternel}
\end{exemple}
\newline
\begin{exemple}
\textbf{\pnua{li pe-è-pööni}}
\pfra{ils sont en relation d'oncle et neveu/nièce maternel}
\end{exemple}
\end{entrée}

\begin{entrée}{eva?}{}{ⓔeva?}
\région{BO PA}
(\domainesémantique{Interrogatifs})
\classe{INT.LOC (statique)}
\begin{glose}
\pfra{où ?}
\end{glose}
\newline
\begin{exemple}
\textbf{\pnua{ge je eva ?}}
\pfra{où est-elle ?}
\end{exemple}
\newline
\begin{exemple}
\textbf{\pnua{i ivi ja eva ?}}
\pfra{où a-t-elle ramassé les saletés ?}
\end{exemple}
\end{entrée}

\begin{entrée}{evadan}{}{ⓔevadan}
\région{BO}
(\domainesémantique{Ignames})
\classe{nom}
\begin{glose}
\pfra{igname (clone) (Dubois)}
\end{glose}
\end{entrée}

\begin{entrée}{evhe}{}{ⓔevhe}
\région{PA BO}
\variante{%
epe
}
(\domainesémantique{Quantificateurs})
\classe{COLL ; QNT}
\begin{glose}
\pfra{ensemble}
\end{glose}
\newline
\begin{exemple}
\région{PA}
\textbf{\pnua{li evhe be-alaa-n}}
\pfra{ils portent tous le même nom}
\end{exemple}
\newline
\note{epè- indique un couple hétérogène (Haudricourt, Leenhardt)}{général}{}
\end{entrée}

\begin{entrée}{exa}{}{ⓔexa}
\région{PABO}
\variante{%
eka
\région{BO}}
(\domainesémantique{Conjonction})
\classe{CNJ}
\begin{glose}
\pfra{quand ; lorsque (référence passée)}
\end{glose}
\newline
\begin{exemple}
\région{PA}
\région{BO}
\région{BO}
\textbf{\pnua{eka goon-al}}
\pfra{à midi}
\end{exemple}
\newline
\relationsémantique{Cf.}{\lien{}{novw-exa}}
\glosecourte{quand, lorsque}
\newline
\relationsémantique{Cf.}{\lien{}{novwo exa}}
\glosecourte{quand, lorsque}
\end{entrée}

\begin{entrée}{e zo !}{}{ⓔe zo !}
\région{GOs}
(\domainesémantique{Interjection})
\classe{INTJ}
\begin{glose}
\pfra{bien fait !}
\end{glose}
\newline
\relationsémantique{Cf.}{\lien{}{i no-n !}}
\glosecourte{bien fait pour lui ! (= ta crotte) [PA]}
\end{entrée}

\begin{entrée}{ezoma}{}{ⓔezoma}
\région{GOs}
\variante{%
ruma
\région{PA}}
(\domainesémantique{Temps})
\classe{FUT}
\begin{glose}
\pfra{futur}
\end{glose}
\newline
\begin{exemple}
\textbf{\pnua{ezoma li ubò mònò ? - Hai ! kò (neg) li zoma ubò mònò, ezoma li yu avwônô}}
\pfra{vont-ils sortir demain ? - Non ! ils ne vont pas sortir demain, ils vont rester à la maison ?}
\end{exemple}
\end{entrée}

\newpage

\lettrine{f}\begin{entrée}{fari}{}{ⓔfari}
\région{GOs}
(\domainesémantique{Aliments, alimentation})
\classe{nom}
\begin{glose}
\pfra{farine}
\end{glose}
\newline
\emprunt{farine (FR)}
\end{entrée}

\begin{entrée}{fè}{}{ⓔfè}
\région{GOs}
(\domainesémantique{Relations et interaction sociales})
\classe{nom}
\begin{glose}
\pfra{fête}
\end{glose}
\newline
\emprunt{fête (FR)}
\end{entrée}

\newpage

\lettrine{g}\begin{entrée}{gaa}{1}{ⓔgaaⓗ1}
\région{GOs PA BO}
\variante{%
ga
\région{BO}}
(\domainesémantique{Aspect})
\classe{ASP duratif}
\begin{glose}
\pfra{encore en train de ; toujours en train de}
\end{glose}
\newline
\begin{exemple}
\région{GO}
\textbf{\pnua{e gaa vhaa gò}}
\pfra{il continue à parler (ça n'en finit plus)}
\end{exemple}
\newline
\begin{exemple}
\région{PA}
\textbf{\pnua{e gaa vhaa gòl}}
\pfra{il continue à parler}
\end{exemple}
\newline
\begin{exemple}
\région{PA}
\textbf{\pnua{gaa waang}}
\pfra{(c'est) encore tôt le matin}
\end{exemple}
\newline
\begin{exemple}
\région{PA}
\textbf{\pnua{gaa ge-la-daa-mi ni dèn}}
\pfra{ils sont en route}
\end{exemple}
\newline
\begin{exemple}
\région{PA}
\textbf{\pnua{li gaa ẽno}}
\pfra{ils sont encore jeunes, ce sont encore des enfants}
\end{exemple}
\newline
\begin{exemple}
\région{BO}
\textbf{\pnua{i gaa ẽno}}
\pfra{il est encore petit}
\end{exemple}
\newline
\begin{exemple}
\région{PA}
\textbf{\pnua{gaa mãã}}
\pfra{(c'est) encore cru (pas encore cuit)}
\end{exemple}
\newline
\relationsémantique{Cf.}{\lien{}{gò [GOs], gòl [PA]}}
\glosecourte{encore; continuer à}
\end{entrée}

\begin{entrée}{gaa}{2}{ⓔgaaⓗ2}
\formephonétique{gaː}
\région{GOs BO}
\variante{%
gee
\formephonétique{geː}
\région{BO}, 
gèèn
\région{BO [Corne]}}
\classe{nom}
\newline
\sens{1}
(\domainesémantique{Fonctions naturelles humaines})
\begin{glose}
\pfra{voix}
\end{glose}
\newline
\sens{2}
(\domainesémantique{Sons, bruits})
\begin{glose}
\pfra{son ; bruit (de paroles)}
\end{glose}
\newline
\sens{3}
(\domainesémantique{Musique, instruments de musique})
\begin{glose}
\pfra{musique ; air (d'une chanson) ; mélodie}
\end{glose}
\newline
\begin{sous-entrée}{gee-wa}{ⓔgaaⓗ2ⓢ3ⓝgee-wa}
\begin{glose}
\pfra{mélodie, air du chant}
\end{glose}
\end{sous-entrée}
\end{entrée}

\begin{entrée}{gaa ... gòl}{}{ⓔgaa ... gòl}
\région{PA}
(\domainesémantique{Aspect})
\classe{ASP}
\begin{glose}
\pfra{toujours en train de}
\end{glose}
\begin{glose}
\pfra{faire le premier ; faire d'abord}
\end{glose}
\newline
\begin{exemple}
\région{PA}
\textbf{\pnua{i gaa mããni gòl}}
\pfra{il est encore en train de dormir}
\end{exemple}
\newline
\begin{exemple}
\région{PA}
\textbf{\pnua{gaa waang gòl}}
\pfra{c'est encore tôt}
\end{exemple}
\newline
\begin{exemple}
\région{PA}
\textbf{\pnua{gaa meã gòl}}
\pfra{ce n'est toujours pas cuit, c'est encore cru}
\end{exemple}
\newline
\begin{exemple}
\région{PA}
\textbf{\pnua{na ru gaa vhaa gòl}}
\pfra{je vais parler en premier}
\end{exemple}
\newline
\begin{exemple}
\région{PA}
\textbf{\pnua{na ru gaa khobwe gòl jena nu nõnõmi}}
\pfra{je vais d'abord dire ce que je pense}
\end{exemple}
\newline
\begin{exemple}
\région{PA}
\textbf{\pnua{na ru gaa huu gòl nye nò}}
\pfra{je me réserve ces poissons (c'est moi qui vais en manger d'abord)}
\end{exemple}
\newline
\begin{exemple}
\région{PA}
\textbf{\pnua{yu gaa kòò gòl, ma nu gaa a kole we ni mûû}}
\pfra{attends un peu, je vais aller arroser les fleurs}
\end{exemple}
\end{entrée}

\begin{entrée}{gaa ... hô}{}{ⓔgaa ... hô}
\région{PA}
(\domainesémantique{Aspect})
\classe{ASP}
\begin{glose}
\pfra{venir juste de}
\end{glose}
\newline
\begin{exemple}
\région{PA}
\textbf{\pnua{i gaa hô uvhi loto}}
\pfra{il vient d'acheter une voiture}
\end{exemple}
\newline
\begin{exemple}
\région{PA}
\textbf{\pnua{i ra gaa hô a-du-ò}}
\pfra{il vient juste de partir}
\end{exemple}
\end{entrée}

\begin{entrée}{gaajò}{}{ⓔgaajò}
\formephonétique{gaːɲɟɔ}
\région{GOs}
\variante{%
gaajòn
\région{PA BO WEM}}
(\domainesémantique{Fonctions intellectuelles})
\classe{v.i.}
\begin{glose}
\pfra{surpris (être) ; sursauter ; étonner (s') ; étonné (être)}
\end{glose}
\newline
\begin{exemple}
\textbf{\pnua{e gaajò-ni êê}}
\pfra{il s'étonne de cela}
\end{exemple}
\newline
\begin{exemple}
\textbf{\pnua{nu pa-gaajò-ni je}}
\pfra{je l'ai fait sursauter (sans toucher)}
\end{exemple}
\newline
\relationsémantique{Cf.}{\lien{ⓔhããmal}{hããmal}}
\glosecourte{admirer}
\end{entrée}

\begin{entrée}{gaa kòò gò}{}{ⓔgaa kòò gò}
\région{GOs}
\variante{%
gaa kòòl
\région{PA BO}}
(\domainesémantique{Interjection})
\classe{v}
\begin{glose}
\pfra{attendez ! (respect) (lit. restez debout)}
\end{glose}
\end{entrée}

\begin{entrée}{gaa mara}{}{ⓔgaa mara}
\région{PA}
(\domainesémantique{Aspect})
\classe{ASP}
\begin{glose}
\pfra{venir tout juste de}
\end{glose}
\newline
\begin{exemple}
\région{PA}
\textbf{\pnua{nu ra gaa mara khobwe}}
\pfra{c'est ce que je viens de dire}
\end{exemple}
\newline
\begin{exemple}
\région{PA}
\textbf{\pnua{i ra gaa mara a-du-ò}}
\pfra{il vient juste de partir}
\end{exemple}
\newline
\begin{exemple}
\région{PA}
\textbf{\pnua{gaa mara mò-n}}
\pfra{c'est sa première maison}
\end{exemple}
\end{entrée}

\begin{entrée}{gaaò}{}{ⓔgaaò}
\région{PA BO WE}
\variante{%
ha
\région{GO(s)}}
\newline
\sens{1}
(\domainesémantique{Verbes d'action (en général)})
\classe{v}
\begin{glose}
\pfra{casser (verre) ; éclater}
\end{glose}
\newline
\begin{exemple}
\région{BO}
\textbf{\pnua{nu gaaò ver i nu}}
\pfra{j'ai cassé mon verre}
\end{exemple}
\newline
\begin{exemple}
\région{WEM}
\textbf{\pnua{e gaaò za}}
\pfra{il a cassé l'assiette}
\end{exemple}
\newline
\begin{exemple}
\région{GO}
\textbf{\pnua{e ha za}}
\pfra{il a cassé l'assiette}
\end{exemple}
\newline
\sens{2}
(\domainesémantique{Description des objets, formes, consistance, taille})
\begin{glose}
\pfra{félé ; cassé}
\end{glose}
\end{entrée}

\begin{entrée}{gaa-vhaa}{}{ⓔgaa-vhaa}
\région{GOs}
\variante{%
gaa-fhaa
\région{GA}}
(\domainesémantique{Sons, bruits})
\classe{nom}
\begin{glose}
\pfra{bruit des voix ; brouhaha}
\end{glose}
\newline
\begin{exemple}
\textbf{\pnua{gaa-vhaa i êgu}}
\pfra{le bruit des voix des gens}
\end{exemple}
\end{entrée}

\begin{entrée}{gaawe}{}{ⓔgaawe}
\région{GOs}
(\domainesémantique{Processus liés aux plantes})
\classe{v}
\begin{glose}
\pfra{fleurir}
\end{glose}
\end{entrée}

\begin{entrée}{gaa-we}{}{ⓔgaa-we}
\région{GOs PA BO}
\variante{%
thò-we
\région{GO(s)}}
(\domainesémantique{Eau})
\classe{nom}
\begin{glose}
\pfra{cascade (grosse et qui fait du bruit) (lit. mélodie de l'eau)}
\end{glose}
\end{entrée}

\begin{entrée}{gaayi}{}{ⓔgaayi}
\région{PA}
(\domainesémantique{Discours, échanges verbaux})
\classe{nom}
\begin{glose}
\pfra{cri (annonce l'interruption ou la fin d'une danse)}
\end{glose}
\end{entrée}

\begin{entrée}{gaga}{}{ⓔgaga}
\région{GOs}
(\domainesémantique{Mammifères})
\classe{nom}
\begin{glose}
\pfra{roussette (avec du blanc sur le cou)}
\end{glose}
\end{entrée}

\begin{entrée}{gala-mii}{}{ⓔgala-mii}
\région{GOs}
(\domainesémantique{Processus liés aux plantes})
\classe{v}
\begin{glose}
\pfra{mûr (fruits)}
\end{glose}
\end{entrée}

\begin{entrée}{galò}{}{ⓔgalò}
\région{GOs PA}
(\domainesémantique{Description des objets, formes, consistance, taille})
\classe{v}
\begin{glose}
\pfra{croquant (sous la dent ; se dit d'un fruit pas mûr comme la mangue ou la papaye)}
\end{glose}
\newline
\begin{exemple}
\région{PA}
\textbf{\pnua{e galò nye pò-mãã}}
\pfra{la mangue croque sous la dent}
\end{exemple}
\end{entrée}

\begin{entrée}{gana}{}{ⓔgana}
\région{GOs BO}
(\domainesémantique{Noms des plantes})
\classe{nom}
\begin{glose}
\pfra{pois d'angole ; Ambrevade}
\end{glose}
\nomscientifique{Cajanus indicus}
\end{entrée}

\begin{entrée}{gaò}{}{ⓔgaò}
\région{GOs PA}
\classe{v}
\newline
\sens{1}
(\domainesémantique{Processus liés aux plantes})
\begin{glose}
\pfra{éclore}
\end{glose}
\newline
\begin{exemple}
\région{PA}
\textbf{\pnua{u gaò muu}}
\pfra{la fleur a éclos}
\end{exemple}
\newline
\sens{2}
(\domainesémantique{Fonctions naturelles des animaux})
\begin{glose}
\pfra{sortir du cocon (papillon)}
\end{glose}
\end{entrée}

\begin{entrée}{gapavwu-}{}{ⓔgapavwu-}
\région{GOs}
\variante{%
gavwu-
\région{PA}}
(\domainesémantique{Préfixes classificateurs sémantiques})
\classe{CLF}
\begin{glose}
\pfra{préfixe des touffes de bambous ou de bananiers}
\end{glose}
\end{entrée}

\begin{entrée}{garuã}{}{ⓔgaruã}
\région{GOs PA BO}
(\domainesémantique{Caractéristiques et propriétés des personnes})
\classe{v}
\begin{glose}
\pfra{ruser}
\end{glose}
\newline
\begin{sous-entrée}{a-garuã}{ⓔgaruãⓝa-garuã}
\begin{glose}
\pfra{qqn qui ruse, un rusé}
\end{glose}
\newline
\begin{exemple}
\textbf{\pnua{i ra u garuã ! [PA]}}
\pfra{il est rusé !}
\end{exemple}
\newline
\relationsémantique{Cf.}{\lien{}{truã}}
\glosecourte{mentir, jouer des tours}
\end{sous-entrée}
\end{entrée}

\begin{entrée}{gase ?}{}{ⓔgase ?}
\région{GOs}
(\domainesémantique{Interjection})
\classe{LOCUT}
\begin{glose}
\pfra{on y va ? (2 personnes ou plus)}
\end{glose}
\end{entrée}

\begin{entrée}{ge}{1}{ⓔgeⓗ1}
\région{GOs BO PA}
\classe{v.LOC ; progressif}
\newline
\sens{1}
(\domainesémantique{Verbes locatifs})
\begin{glose}
\pfra{trouver (se) à ; être dans un endroit}
\end{glose}
\begin{glose}
\pfra{près de (être)}
\end{glose}
\newline
\begin{exemple}
\région{GO}
\textbf{\pnua{ge nu Gome}}
\pfra{je suis à Gomen}
\end{exemple}
\newline
\begin{exemple}
\région{GO}
\textbf{\pnua{ge çö Gome}}
\pfra{tu es à Gomen}
\end{exemple}
\newline
\begin{exemple}
\région{GO}
\textbf{\pnua{ge je Numia}}
\end{exemple}
\newline
\begin{exemple}
\région{GO}
\textbf{\pnua{ge je nòli}}
\pfra{il est là-bas}
\end{exemple}
\newline
\begin{exemple}
\région{GO}
\textbf{\pnua{ge le da ?}}
\pfra{qu'est-ce qu'il y a ?}
\end{exemple}
\newline
\begin{exemple}
\région{GO}
\textbf{\pnua{ge va hèlè i nu ?}}
\pfra{où est mon couteau?}
\end{exemple}
\newline
\begin{exemple}
\région{BO}
\textbf{\pnua{ge je va ?}}
\pfra{d'où est-il ?}
\end{exemple}
\newline
\begin{exemple}
\région{PA}
\textbf{\pnua{pwòmò-m ge kê-kui ?}}
\pfra{est-ce ton champ d'ignames?}
\end{exemple}
\newline
\begin{sous-entrée}{ge na}{ⓔgeⓗ1ⓢ1ⓝge na}
\région{BO PA}
\begin{glose}
\pfra{il est ici}
\end{glose}
\end{sous-entrée}
\newline
\begin{sous-entrée}{ge nòli}{ⓔgeⓗ1ⓢ1ⓝge nòli}
\région{BO PA}
\begin{glose}
\pfra{il est là-bas}
\end{glose}
\end{sous-entrée}
\newline
\sens{2}
(\domainesémantique{Aspect})
\begin{glose}
\pfra{être en train de}
\end{glose}
\newline
\begin{exemple}
\région{GO}
\textbf{\pnua{novwö ne ge je ne mhenõ mãni çe/ça me zoma baa-je}}
\pfra{quand elle sera en train de dormir, on la frappera}
\end{exemple}
\newline
\begin{exemple}
\textbf{\pnua{kavwo ge je pwaa ce, ma ge je mããni !}}
\pfra{il n'est pas en train de couper du bois, il est en train de dormir}
\end{exemple}
\end{entrée}

\begin{entrée}{ge}{2}{ⓔgeⓗ2}
\région{BO}
(\domainesémantique{Conjonction})
\classe{CNJ ; THEM}
\begin{glose}
\pfra{et alors}
\end{glose}
\newline
\begin{exemple}
\région{BO}
\textbf{\pnua{pwòmò-m ge kêê-kui ?}}
\pfra{ton champ est-il un champ d'ignames ?}
\end{exemple}
\end{entrée}

\begin{entrée}{gèa}{}{ⓔgèa}
\région{GOs BO}
\newline
\groupe{A}
(\domainesémantique{Santé, maladie})
\classe{v}
\begin{glose}
\pfra{loucher}
\end{glose}
\newline
\begin{exemple}
\textbf{\pnua{gèa mee-je}}
\pfra{il louche}
\end{exemple}
\newline
\begin{exemple}
\textbf{\pnua{egu xa gèa mee-je}}
\pfra{quelqu'un qui louche}
\end{exemple}
\newline
\groupe{B}
(\domainesémantique{Santé, maladie})
\classe{nom}
\begin{glose}
\pfra{voile blanc de la pupille (maladie de l'oeil)}
\end{glose}
\end{entrée}

\begin{entrée}{ge ... bwa}{}{ⓔge ... bwa}
\région{GOs}
(\domainesémantique{Comparaison})
\classe{v}
\begin{glose}
\pfra{plus haut que (être)}
\end{glose}
\newline
\begin{exemple}
\textbf{\pnua{ge je bwa na ni mwa-nu}}
\pfra{il est plus haut que ma maison}
\end{exemple}
\newline
\begin{exemple}
\textbf{\pnua{ge je bwa na i nu}}
\pfra{il est plus haut que moi (mon supérieur hiérarchique)}
\end{exemple}
\end{entrée}

\begin{entrée}{gee}{}{ⓔgee}
\région{GOs}
\variante{%
gèèng
\région{BO PA}, 
gèènk
\région{BO vx}}
\newline
\groupe{A}
(\domainesémantique{Description des objets, formes, consistance, taille})
\classe{v.stat.}
\begin{glose}
\pfra{sale}
\end{glose}
\begin{glose}
\pfra{taché}
\end{glose}
\newline
\begin{exemple}
\région{PA}
\textbf{\pnua{i gèèn}}
\pfra{il est sale}
\end{exemple}
\newline
\groupe{B}
(\domainesémantique{Description des objets, formes, consistance, taille})
\classe{nom}
\begin{glose}
\pfra{saleté ; détritus}
\end{glose}
\begin{glose}
\pfra{tache}
\end{glose}
\newline
\begin{exemple}
\région{PA}
\textbf{\pnua{geenga-n}}
\pfra{sa crasse}
\end{exemple}
\newline
\begin{exemple}
\région{BO}
\textbf{\pnua{gee-je}}
\pfra{sa crasse}
\end{exemple}
\newline
\relationsémantique{Ant.}{\lien{}{zo [GOs]}}
\glosecourte{propre, bien, bon}
\newline
\étymologie{
\langue{POc}
\étymon{*tanuq}
\glosecourte{poussiéreux, sale}}
\end{entrée}

\begin{entrée}{gèè}{}{ⓔgèè}
\formephonétique{gɛː}
\région{GOs PA BO}
(\domainesémantique{Parenté})
\classe{nom}
\begin{glose}
\pfra{grand-mère (maternelle ou paternelle, désignation et appellation)}
\end{glose}
\begin{glose}
\pfra{soeur de grand-mère}
\end{glose}
\begin{glose}
\pfra{cousine de grand-mère}
\end{glose}
\newline
\begin{exemple}
\région{GO PA}
\textbf{\pnua{gèè i nu}}
\pfra{ma grand-mère}
\end{exemple}
\end{entrée}

\begin{entrée}{gèè-thraa, gè-raa}{}{ⓔgèè-thraa, gè-raa}
\région{GOs PA}
\région{PA WEM WE}
\variante{%
gèè-mãmã, gè-raa
}
(\domainesémantique{Parenté})
\classe{nom}
\begin{glose}
\pfra{arrière-grand-mère}
\end{glose}
\end{entrée}

\begin{entrée}{gee-vha}{}{ⓔgee-vha}
\région{GOs PA BO}
(\domainesémantique{Musique, instruments de musique})
\classe{nom}
\begin{glose}
\pfra{son, mélodie de la voix}
\end{glose}
\end{entrée}

\begin{entrée}{gee-wal}{}{ⓔgee-wal}
\région{BO}
(\domainesémantique{Musique, instruments de musique})
\classe{nom}
\begin{glose}
\pfra{air (chant, musique)}
\end{glose}
\end{entrée}

\begin{entrée}{ge-le}{}{ⓔge-le}
\région{GOs PA BO}
(\domainesémantique{Prédicats existentiels})
\classe{v.LOC}
\begin{glose}
\pfra{il y (en) a}
\end{glose}
\newline
\begin{exemple}
\région{GO}
\textbf{\pnua{ge-le da ?}}
\pfra{qu'est-ce qu'il y a ?}
\end{exemple}
\newline
\begin{exemple}
\région{BO}
\textbf{\pnua{ge-le phwa}}
\pfra{il y a des trous}
\end{exemple}
\end{entrée}

\begin{entrée}{ge-le-xa}{}{ⓔge-le-xa}
\région{GOs BO}
(\domainesémantique{Prédicats existentiels})
\classe{v}
\begin{glose}
\pfra{il y a (indéfini nonspécifique)}
\end{glose}
\newline
\begin{exemple}
\région{GO}
\textbf{\pnua{ge-le da ?}}
\pfra{qu'est-ce qu'il y a ?}
\end{exemple}
\newline
\begin{exemple}
\région{BO}
\textbf{\pnua{ge-le-xa po ènèda}}
\pfra{il y a qqch là-haut}
\end{exemple}
\end{entrée}

\begin{entrée}{ge ... mhenõõ}{}{ⓔge ... mhenõõ}
\région{GOs PA}
(\domainesémantique{Aspect})
\classe{ASP}
\begin{glose}
\pfra{en train de}
\end{glose}
\newline
\begin{exemple}
\région{PA}
\textbf{\pnua{ge je mhenõõ burom}}
\pfra{il est en train de se baigner/laver}
\end{exemple}
\newline
\begin{exemple}
\région{PA}
\textbf{\pnua{gaa ge je mhenõõ burom}}
\pfra{il est encore en train de se baigner/laver}
\end{exemple}
\end{entrée}

\begin{entrée}{gènè}{}{ⓔgènè}
\région{GOs}
\variante{%
gèn
\région{PA BO}}
(\domainesémantique{Couleurs})
\classe{nom}
\begin{glose}
\pfra{couleur [GOs]}
\end{glose}
\begin{glose}
\pfra{couleur ; dessin ; maquillage pour danser [PA BO]}
\end{glose}
\newline
\begin{exemple}
\région{GO}
\textbf{\pnua{gènè-je}}
\pfra{sa couleur}
\end{exemple}
\newline
\begin{exemple}
\région{BO PA}
\textbf{\pnua{whaya gènè-n ?}}
\pfra{quelle couleur ?}
\end{exemple}
\newline
\begin{exemple}
\région{PA}
\textbf{\pnua{meevu gènè-n ?}}
\pfra{de toutes les couleurs}
\end{exemple}
\newline
\begin{sous-entrée}{gèène jopo}{ⓔgènèⓝgèène jopo}
\région{BO PA}
\begin{glose}
\pfra{dessins géométriques sur les chambranles}
\end{glose}
\end{sous-entrée}
\newline
\begin{sous-entrée}{gè thaa}{ⓔgènèⓝgè thaa}
\région{BO PA}
\begin{glose}
\pfra{tatouage}
\end{glose}
\newline
\note{gène: forme déterminée de gè}{grammaire}{}
\end{sous-entrée}
\end{entrée}

\begin{entrée}{gèny}{}{ⓔgèny}
\région{BO}
\classe{v.stat.}
(\domainesémantique{Santé, maladie})
\begin{glose}
\pfra{stérile [BM]}
\end{glose}
\end{entrée}

\begin{entrée}{gèrè}{}{ⓔgèrè}
\région{GOs PA BO}
(\domainesémantique{Aliments, alimentation})
\classe{nom}
\begin{glose}
\pfra{graisse; huile}
\end{glose}
\newline
\begin{exemple}
\textbf{\pnua{mini-gèrèè}}
\pfra{résidu de saindoux, graisse 'gratton'}
\end{exemple}
\newline
\emprunt{graisse (FR)}
\end{entrée}

\begin{entrée}{gè-thaa}{}{ⓔgè-thaa}
\région{GOs}
(\domainesémantique{Soins du corps})
\classe{v}
\begin{glose}
\pfra{tatouer}
\end{glose}
\newline
\relationsémantique{Cf.}{\lien{}{gè -thaa}}
\glosecourte{couleur-piquer}
\end{entrée}

\begin{entrée}{ge-yai}{}{ⓔge-yai}
\région{GOs BO PA}
(\domainesémantique{Soins du corps})
\classe{nom}
\begin{glose}
\pfra{tatouage (fait avec des pointes de feu, des côtes de feuilles de coco allumées)}
\end{glose}
\end{entrée}

\begin{entrée}{gi}{}{ⓔgi}
\formephonétique{ŋgi}
\région{GO PA BO}
(\domainesémantique{Fonctions naturelles humaines})
\classe{v}
\begin{glose}
\pfra{pleurer ; gémir}
\end{glose}
\newline
\begin{exemple}
\textbf{\pnua{e gi ã ẽnõ}}
\pfra{cet enfant pleure}
\end{exemple}
\newline
\begin{exemple}
\région{GO}
\textbf{\pnua{e gi mèni}}
\pfra{l'oiseau pleure (Doriane)}
\end{exemple}
\newline
\étymologie{
\langue{POc}
\étymon{*taŋis}}
\newline
\note{pa-gi}{grammaire}{faire pleurer}
\end{entrée}

\begin{entrée}{gibwa}{}{ⓔgibwa}
\région{GOs}
\newline
\sens{1}
(\domainesémantique{Mouvements ou actions faits avec le corps, les bras, les mains, les pieds})
\classe{v}
\begin{glose}
\pfra{tirer d'un coup sec}
\end{glose}
\newline
\sens{2}
(\domainesémantique{Pêche})
\classe{v}
\begin{glose}
\pfra{ferrer un poisson}
\end{glose}
\end{entrée}

\begin{entrée}{giçaò}{}{ⓔgiçaò}
\formephonétique{ŋgiʒaɔ ŋgidʒaɔ}
\région{GOs}
(\domainesémantique{Relations et interaction sociales})
\classe{v}
\begin{glose}
\pfra{repentir (se) ; demander pardon ; confesser (se)}
\end{glose}
\end{entrée}

\begin{entrée}{gii-we}{}{ⓔgii-we}
\région{GOs}
(\domainesémantique{Eau})
\classe{nom}
\begin{glose}
\pfra{courant de l'eau}
\end{glose}
\end{entrée}

\begin{entrée}{gi-mã}{}{ⓔgi-mã}
\région{GOs}
(\domainesémantique{Société})
\classe{v}
\begin{glose}
\pfra{pleurer un mort}
\end{glose}
\end{entrée}

\begin{entrée}{giner}{}{ⓔginer}
\région{PA}
(\domainesémantique{Noms des plantes})
\classe{nom}
\begin{glose}
\pfra{"herbe à éléphant"}
\end{glose}
\end{entrée}

\begin{entrée}{gita}{}{ⓔgita}
\région{GOs}
(\domainesémantique{Musique, instruments de musique})
\classe{nom}
\begin{glose}
\pfra{guitare}
\end{glose}
\newline
\emprunt{guitare (FR)}
\end{entrée}

\begin{entrée}{giul}{}{ⓔgiul}
\région{BO}
(\domainesémantique{Société})
\classe{v}
\begin{glose}
\pfra{pleurer un mort (pour les hommes) [Corne]}
\end{glose}
\newline
\relationsémantique{Cf.}{\lien{}{gi, cöńi}}
\glosecourte{pleurer}
\newline
\note{non vérifié}{général}{}
\end{entrée}

\begin{entrée}{gò}{1}{ⓔgòⓗ1}
\région{GOs BO PA}
\classe{nom}
\newline
\sens{1}
(\domainesémantique{Noms des plantes})
\begin{glose}
\pfra{bambou}
\end{glose}
\newline
\begin{sous-entrée}{gò-hûgu}{ⓔgòⓗ1ⓢ1ⓝgò-hûgu}
\région{PA}
\begin{glose}
\pfra{sorte de bambou (utilisé comme percussion)}
\end{glose}
\end{sous-entrée}
\newline
\begin{sous-entrée}{gò-pwêûp}{ⓔgòⓗ1ⓢ1ⓝgò-pwêûp}
\région{BO}
\begin{glose}
\pfra{sorte de bambou (utilisé comme percussion)}
\end{glose}
\end{sous-entrée}
\newline
\sens{2}
(\domainesémantique{Outils})
\begin{glose}
\pfra{couteau de bambou ; couteau à subincision}
\end{glose}
\newline
\sens{3}
(\domainesémantique{Musique, instruments de musique})
\begin{glose}
\pfra{musique (lit. son de la flûte en bambou) ; appareil de musique}
\end{glose}
\newline
\begin{exemple}
\région{GO}
\textbf{\pnua{e tho gò}}
\pfra{la musique joue}
\end{exemple}
\newline
\étymologie{
\langue{POc}
\étymon{*qauR}}
\end{entrée}

\begin{entrée}{gò}{2}{ⓔgòⓗ2}
\région{GOs}
\variante{%
gòl
\région{PA}}
(\domainesémantique{Aspect})
\classe{ASP (post-verbal)}
\begin{glose}
\pfra{continuatif ; sans interruption ; duratif}
\end{glose}
\newline
\begin{exemple}
\textbf{\pnua{e mããni gò!}}
\pfra{il continue à dormir}
\end{exemple}
\newline
\begin{exemple}
\textbf{\pnua{e mãyã gò!}}
\pfra{c'est encore cru (mal cuit) !}
\end{exemple}
\newline
\begin{exemple}
\textbf{\pnua{ge je mããni gò!}}
\pfra{il est encore en train de à dormir}
\end{exemple}
\newline
\begin{exemple}
\textbf{\pnua{wazaò me ẽnõ gò!}}
\pfra{quand nous étions encore enfant}
\end{exemple}
\newline
\begin{exemple}
\textbf{\pnua{e uja ? - Hai ! kavwö uja gò!}}
\pfra{il est arrivé ? - Non ! il n'est pas encore arrivé !}
\end{exemple}
\newline
\begin{exemple}
\textbf{\pnua{co hovwo gò!}}
\pfra{continue à manger}
\end{exemple}
\newline
\begin{exemple}
\région{GO}
\textbf{\pnua{e mha whaa gò}}
\pfra{c'est encore trop tôt}
\end{exemple}
\newline
\begin{sous-entrée}{kavwö ... gò}{ⓔgòⓗ2ⓝkavwö ... gò}
\begin{glose}
\pfra{pas encore}
\end{glose}
\end{sous-entrée}
\end{entrée}

\begin{entrée}{gò-êgu}{}{ⓔgò-êgu}
\région{GOs}
(\domainesémantique{Cours de la vie})
\classe{nom}
\begin{glose}
\pfra{personne d'âge moyen, mûr (ni enfant, ni vieillard)}
\end{glose}
\newline
\relationsémantique{Cf.}{\lien{ⓔẽnõⓗ1}{ẽnõ}}
\glosecourte{enfant}
\newline
\relationsémantique{Cf.}{\lien{}{whaamã}}
\glosecourte{vieillard}
\end{entrée}

\begin{entrée}{gò-hii}{}{ⓔgò-hii}
\région{PA}
\classe{nom}
\newline
\sens{1}
(\domainesémantique{Corps humain})
\begin{glose}
\pfra{avant-bras}
\end{glose}
\newline
\sens{2}
(\domainesémantique{Objets coutumiers})
\begin{glose}
\pfra{monnaie kanak (de la longueur d'un avant-bras, Charles)}
\end{glose}
\end{entrée}

\begin{entrée}{gò-khai}{}{ⓔgò-khai}
\région{GOs}
(\domainesémantique{Musique, instruments de musique})
\classe{nom}
\begin{glose}
\pfra{accordéon}
\end{glose}
\newline
\relationsémantique{Cf.}{\lien{ⓔkhaiⓗ1}{khai}}
\glosecourte{tirer}
\end{entrée}

\begin{entrée}{gò-niza ?}{}{ⓔgò-niza ?}
\région{GO}
\variante{%
gò-nira ?
\région{BO}}
(\domainesémantique{Interrogatifs})
\classe{INT}
\begin{glose}
\pfra{combien de morceaux (bois, poisson, anguilles) ?}
\end{glose}
\newline
\begin{exemple}
\région{PA BO}
\textbf{\pnua{gò-ruuji bwa gò-ru}}
\pfra{douze morceaux (Dubois)}
\end{exemple}
\end{entrée}

\begin{entrée}{goo}{}{ⓔgoo}
\région{GOs}
\variante{%
goony
\région{PA BO}}
(\domainesémantique{Cocotiers})
\classe{nom}
\begin{glose}
\pfra{nervure centrale des folioles de palmes de cocotier (sert à faire des petits balais)}
\end{glose}
\newline
\begin{sous-entrée}{goo-nu}{ⓔgooⓝgoo-nu}
\région{GO}
\begin{glose}
\pfra{nervure centrale des folioles de palmes de cocotier}
\end{glose}
\end{sous-entrée}
\newline
\begin{sous-entrée}{goo-a dròò-nu}{ⓔgooⓝgoo-a dròò-nu}
\région{GO}
\begin{glose}
\pfra{nervure centrale des folioles de palmes de cocotier}
\end{glose}
\end{sous-entrée}
\newline
\begin{sous-entrée}{goonya-nu}{ⓔgooⓝgoonya-nu}
\région{PA}
\begin{glose}
\pfra{nervure centrale des folioles de palmes de cocotier}
\end{glose}
\end{sous-entrée}
\end{entrée}

\begin{entrée}{gòò}{}{ⓔgòò}
\région{GOs PA BO}
\newline
\groupe{A}
\classe{nom}
\newline
\sens{1}
(\domainesémantique{Quantificateurs})
\begin{glose}
\pfra{morceau ; partie}
\end{glose}
\newline
\begin{exemple}
\région{BO}
\textbf{\pnua{gò maè}}
\pfra{bout de paille}
\end{exemple}
\newline
\sens{2}
(\domainesémantique{Parties de plantes})
\begin{glose}
\pfra{tronc}
\end{glose}
\newline
\begin{exemple}
\région{BO}
\textbf{\pnua{gòò ce}}
\pfra{tronc d'arbre, morceau de bois}
\end{exemple}
\newline
\sens{3}
(\domainesémantique{Corps humain})
\begin{glose}
\pfra{taille (lit. milieu)}
\end{glose}
\newline
\begin{exemple}
\région{PA}
\textbf{\pnua{gòò-n}}
\pfra{sa taille}
\end{exemple}
\newline
\sens{4}
(\domainesémantique{Noms locatifs})
\begin{glose}
\pfra{milieu}
\end{glose}
\newline
\begin{sous-entrée}{gòò êgu}{ⓔgòòⓢ4ⓝgòò êgu}
\région{GOs}
\begin{glose}
\pfra{homme d'âge mûr/moyen}
\end{glose}
\newline
\begin{exemple}
\région{GO}
\textbf{\pnua{ge ni gòò}}
\pfra{c'est au milieu, à moitié plein}
\end{exemple}
\newline
\begin{exemple}
\région{PA}
\textbf{\pnua{ẽnõ ni gò}}
\pfra{le puîné (enfant du milieu)}
\end{exemple}
\newline
\begin{exemple}
\région{BO}
\textbf{\pnua{gòò-n}}
\pfra{son milieu}
\end{exemple}
\newline
\begin{exemple}
\textbf{\pnua{ni gòò-n [PA]}}
\pfra{au milieu}
\end{exemple}
\end{sous-entrée}
\newline
\groupe{B}
(\domainesémantique{Préfixes classificateurs numériques})
\classe{CLF.NUM (morceaux) PA [pas à Gomen]}
\begin{glose}
\pfra{morceau (objet long)}
\end{glose}
\newline
\begin{sous-entrée}{gò-niza ?}{ⓔgòòⓝgò-niza ?}
\région{GO}
\begin{glose}
\pfra{combiende ?}
\end{glose}
\newline
\begin{exemple}
\région{BO}
\textbf{\pnua{gò-nira ?}}
\pfra{combiende ?}
\end{exemple}
\newline
\begin{exemple}
\textbf{\pnua{gò-xe, gò-ru, gò-kon, gò-ruji (= go-tujic)}}
\pfra{un, deux, trois, dix morceaux}
\end{exemple}
\end{sous-entrée}
\newline
\étymologie{
\langue{PNNC}
\étymon{*qau}
\glosecourte{milieu}
\auteur{Hollyman}}
\end{entrée}

\begin{entrée}{gòò-bwòn}{}{ⓔgòò-bwòn}
\région{PA BO}
\variante{%
gò-bòn
\région{BO}}
(\domainesémantique{Découpage du temps})
\classe{nom}
\begin{glose}
\pfra{minuit [BM, Corne]}
\end{glose}
\end{entrée}

\begin{entrée}{gòò-ce}{}{ⓔgòò-ce}
\région{GOs}
(\domainesémantique{Parties de plantes})
\classe{nom}
\begin{glose}
\pfra{tronc d'arbre}
\end{glose}
\end{entrée}

\begin{entrée}{gòò-kui}{}{ⓔgòò-kui}
\région{PA}
(\domainesémantique{Ignames})
\classe{nom}
\begin{glose}
\pfra{corps de l'igname}
\end{glose}
\end{entrée}

\begin{entrée}{gòò-mwa}{}{ⓔgòò-mwa}
\région{GOs}
\variante{%
gò-mwê
\région{GO}}
(\domainesémantique{Types de maison, architecture de la maison})
\classe{nom}
\begin{glose}
\pfra{mur}
\end{glose}
\newline
\begin{sous-entrée}{phwa ni gòò-mwa}{ⓔgòò-mwaⓝphwa ni gòò-mwa}
\begin{glose}
\pfra{fenêtre}
\end{glose}
\end{sous-entrée}
\end{entrée}

\begin{entrée}{gòòn-a}{}{ⓔgòòn-a}
\région{GOsWE}
\variante{%
gòòn-al
\région{BO PA WEM}}
(\domainesémantique{Phénomènes atmosphériques et naturels})
\classe{nom}
\begin{glose}
\pfra{soleil irradiant}
\end{glose}
\begin{glose}
\pfra{midi ; zénith}
\end{glose}
\newline
\begin{exemple}
\textbf{\pnua{gòn-a tobo}}
\pfra{le soleil décline}
\end{exemple}
\newline
\begin{exemple}
\textbf{\pnua{i mãni gòòn-al}}
\pfra{il a fait la sieste}
\end{exemple}
\end{entrée}

\begin{entrée}{gòò-ui}{}{ⓔgòò-ui}
\région{GOs BO}
(\domainesémantique{Musique, instruments de musique})
\classe{nom}
\begin{glose}
\pfra{flûte ; harmonica}
\end{glose}
\newline
\begin{exemple}
\région{GO}
\textbf{\pnua{la ui gò}}
\pfra{ils jouent de la flûte}
\end{exemple}
\newline
\relationsémantique{Cf.}{\lien{ⓔuiⓗ1}{ui}}
\glosecourte{souffler (gò-ui :lit. bambou souffler)}
\newline
\étymologie{
\langue{POc}
\étymon{*kopi}
\glosecourte{bamboo (flute)}
\auteur{Blust}}
\end{entrée}

\begin{entrée}{goovwû}{}{ⓔgoovwû}
\formephonétique{ŋgoːβû}
\région{GOs}
\variante{%
gobu
\région{GO(s)}}
(\domainesémantique{Temps})
\classe{v}
\begin{glose}
\pfra{retard (en)}
\end{glose}
\newline
\begin{exemple}
\région{GO}
\textbf{\pnua{e uça goovwû}}
\pfra{il est arrivé en retard}
\end{exemple}
\newline
\relationsémantique{Cf.}{\lien{}{uça hêbu}}
\glosecourte{arrivé en avance}
\end{entrée}

\begin{entrée}{gò-pwãu}{}{ⓔgò-pwãu}
\région{PA BO}
(\domainesémantique{Arbre})
\classe{nom}
\begin{glose}
\pfra{bambou qui sert de percussion (lors des danses)}
\end{glose}
\begin{glose}
\pfra{bambou (utilisé pour gratter les 'dimwa' dans l'eau (voir 'bwevòlò')}
\end{glose}
\end{entrée}

\begin{entrée}{gorolo}{}{ⓔgorolo}
\région{BO [BM, Corne]}
\variante{%
gòròlò
\région{BO [BM]}}
(\domainesémantique{Noms des plantes})
\classe{nom}
\begin{glose}
\pfra{plante}
\end{glose}
\nomscientifique{Triumphetta rhomboïdea}
\end{entrée}

\begin{entrée}{gòrui}{}{ⓔgòrui}
\région{GOs}
(\domainesémantique{Matière, matériaux})
\classe{nom}
\begin{glose}
\pfra{fer}
\end{glose}
\end{entrée}

\begin{entrée}{gò-wõ}{}{ⓔgò-wõ}
\région{GOs}
\variante{%
gò-wòny
\région{BO}}
(\domainesémantique{Navigation})
\classe{nom}
\begin{glose}
\pfra{milieu du bateau}
\end{glose}
\end{entrée}

\begin{entrée}{grenui}{}{ⓔgrenui}
\région{GOs}
\variante{%
goronui
\région{BO}}
(\domainesémantique{Reptiles})
\classe{nom}
\begin{glose}
\pfra{grenouille}
\end{glose}
\newline
\emprunt{grenouille (FR)}
\end{entrée}

\begin{entrée}{gu}{1}{ⓔguⓗ1}
\région{GOs}
\variante{%
gun
\région{PA WEM}}
(\domainesémantique{Oiseaux})
\classe{nom}
\begin{glose}
\pfra{pigeon vert}
\end{glose}
\newline
\note{(petit, à tête pourpre, vert du dos à la poitrine, jaune sous la queue, taches pourpres sur le ventre)}{glose}{}
\nomscientifique{Drepanoptila holosericea, Columbidés}
\end{entrée}

\begin{entrée}{gu}{2}{ⓔguⓗ2}
\région{GOs BO}
(\domainesémantique{Pêche})
\classe{nom}
\begin{glose}
\pfra{filoche}
\end{glose}
\begin{glose}
\pfra{lacet}
\end{glose}
\begin{glose}
\pfra{liane (servant à enfiler)}
\end{glose}
\begin{glose}
\pfra{brochette}
\end{glose}
\newline
\begin{sous-entrée}{gua-nò}{ⓔguⓗ2ⓝgua-nò}
\begin{glose}
\pfra{filoche de poisson}
\end{glose}
\end{sous-entrée}
\newline
\begin{sous-entrée}{gua-alaxò}{ⓔguⓗ2ⓝgua-alaxò}
\begin{glose}
\pfra{lacet de chaussure}
\end{glose}
\end{sous-entrée}
\newline
\begin{sous-entrée}{thi-gua no}{ⓔguⓗ2ⓝthi-gua no}
\begin{glose}
\pfra{filoche de poisson}
\end{glose}
\newline
\begin{exemple}
\région{PA}
\textbf{\pnua{gua-n}}
\pfra{sa filoche}
\end{exemple}
\newline
\note{forme déterminée: gua}{grammaire}{}
\end{sous-entrée}
\end{entrée}

\begin{entrée}{gu}{3}{ⓔguⓗ3}
\région{GOs}
\variante{%
gun
\région{BO PA}}
(\domainesémantique{Sons, bruits})
\classe{v ; n}
\begin{glose}
\pfra{bruit sourd}
\end{glose}
\begin{glose}
\pfra{faire du bruit}
\end{glose}
\newline
\begin{sous-entrée}{thu gu}{ⓔguⓗ3ⓝthu gu}
\begin{glose}
\pfra{faire du bruit}
\end{glose}
\end{sous-entrée}
\newline
\begin{sous-entrée}{guna loto}{ⓔguⓗ3ⓝguna loto}
\région{GO}
\begin{glose}
\pfra{le bruit de la voiture}
\end{glose}
\end{sous-entrée}
\newline
\begin{sous-entrée}{gune chiò}{ⓔguⓗ3ⓝgune chiò}
\région{GO}
\begin{glose}
\pfra{le bruit du seau}
\end{glose}
\newline
\begin{exemple}
\région{BO}
\textbf{\pnua{guna loto}}
\pfra{le bruit de la voiture}
\end{exemple}
\newline
\begin{exemple}
\région{PA}
\textbf{\pnua{guno wony}}
\pfra{le bruit du bateau}
\end{exemple}
\newline
\begin{exemple}
\région{PA}
\textbf{\pnua{guno nyo}}
\pfra{le bruit du tonnerre}
\end{exemple}
\newline
\begin{exemple}
\région{PA}
\textbf{\pnua{guno da ?}}
\pfra{le bruit de quoi ?}
\end{exemple}
\newline
\note{guno-n}{grammaire}{son bruit}
\newline
\note{gunè-n}{grammaire}{son bruit}
\end{sous-entrée}
\end{entrée}

\begin{entrée}{gu}{4}{ⓔguⓗ4}
\région{GOs BO [Corne]}
(\domainesémantique{Caractéristiques et propriétés des personnes})
\classe{v.stat.}
\begin{glose}
\pfra{vrai ; droit}
\end{glose}
\begin{glose}
\pfra{autochtone, du pays ; véritable [BO]}
\end{glose}
\newline
\begin{exemple}
\région{BO}
\textbf{\pnua{gu ma hangai}}
\pfra{c'est vraiment grand}
\end{exemple}
\newline
\begin{sous-entrée}{gu mwa}{ⓔguⓗ4ⓝgu mwa}
\région{GO}
\begin{glose}
\pfra{maison ronde}
\end{glose}
\end{sous-entrée}
\newline
\begin{sous-entrée}{gu hi}{ⓔguⓗ4ⓝgu hi}
\begin{glose}
\pfra{main droite}
\end{glose}
\end{sous-entrée}
\newline
\begin{sous-entrée}{gu we}{ⓔguⓗ4ⓝgu we}
\région{GO}
\begin{glose}
\pfra{eau potable}
\end{glose}
\end{sous-entrée}
\newline
\étymologie{
\langue{POc}
\étymon{*(n)tuqu}}
\end{entrée}

\begin{entrée}{gu}{5}{ⓔguⓗ5}
\région{GOs}
\variante{%
ku
\région{PA}}
(\domainesémantique{Modalité, verbes modaux})
\classe{INJ}
\begin{glose}
\pfra{ordre ; injonction}
\end{glose}
\newline
\begin{exemple}
\région{GOs}
\textbf{\pnua{gu jaxa}}
\pfra{ça suffit !}
\end{exemple}
\newline
\begin{exemple}
\région{GOs}
\textbf{\pnua{gu himi phwaa-jö}}
\pfra{tais-toi !}
\end{exemple}
\newline
\begin{exemple}
\région{GOs}
\textbf{\pnua{gu ne la nu kûjaa-jö}}
\pfra{fais ce que je te dis !}
\end{exemple}
\end{entrée}

\begin{entrée}{gu}{6}{ⓔguⓗ6}
\région{GOs}
\variante{%
ku
\région{PA}}
(\domainesémantique{Aspect})
\classe{RESTR}
\begin{glose}
\pfra{rester à ; ne faire que}
\end{glose}
\newline
\begin{exemple}
\textbf{\pnua{la hû, kavwö la vhaa mwa, la gu traabwa wãã na}}
\pfra{elles se taisent, elles ne parlent plus, elles restent assises comme ça}
\end{exemple}
\newline
\begin{sous-entrée}{gu-traabwa}{ⓔguⓗ6ⓝgu-traabwa}
\begin{glose}
\pfra{assis sans bouger}
\end{glose}
\end{sous-entrée}
\end{entrée}

\begin{entrée}{gu-}{}{ⓔgu-}
\région{GOs PA}
\newline
\sens{1}
(\domainesémantique{Préfixes classificateurs numériques})
\classe{CLF.NUM}
\begin{glose}
\pfra{filoche de poisson [GOs, PA]}
\end{glose}
\newline
\begin{exemple}
\textbf{\pnua{la pe-gu-xe}}
\pfra{ils font une filoche de poissons}
\end{exemple}
\newline
\sens{2}
(\domainesémantique{Préfixes classificateurs numériques})
\classe{CLF.NUM}
\begin{glose}
\pfra{files de voitures [PA]}
\end{glose}
\end{entrée}

\begin{entrée}{gu-hi-n}{}{ⓔgu-hi-n}
\région{PA BO}
\variante{%
gu-yin
\région{BO}}
(\domainesémantique{Corps humain})
\classe{nom}
\begin{glose}
\pfra{main droite}
\end{glose}
\newline
\begin{sous-entrée}{yi-ny gui}{ⓔgu-hi-nⓝyi-ny gui}
\région{BO}
\begin{glose}
\pfra{ma main droite}
\end{glose}
\end{sous-entrée}
\newline
\begin{sous-entrée}{keni-ny gui}{ⓔgu-hi-nⓝkeni-ny gui}
\région{BO}
\begin{glose}
\pfra{mon oreille droite}
\end{glose}
\end{sous-entrée}
\newline
\begin{sous-entrée}{gu-(y)i-ny}{ⓔgu-hi-nⓝgu-(y)i-ny}
\région{BO}
\begin{glose}
\pfra{ma main droite}
\end{glose}
\newline
\begin{exemple}
\région{BO}
\textbf{\pnua{cu tabwa bwa gu-yi-ny}}
\pfra{tu es assis à ma droite}
\end{exemple}
\end{sous-entrée}
\newline
\étymologie{
\langue{POc}
\étymon{*taqu}
\glosecourte{right}}
\end{entrée}

\begin{entrée}{gu-kui}{}{ⓔgu-kui}
\région{GOs BO}
\variante{%
gu-kui, gu-xui
\région{PA}}
(\domainesémantique{Ignames})
\classe{nom}
\begin{glose}
\pfra{igname du chef (comporte deux espèces: variété blanche "pwalamu" et violette "pwang"). Dubois}
\end{glose}
\newline
\relationsémantique{Cf.}{\lien{}{pwalamu, pwang}}
\end{entrée}

\begin{entrée}{gumãgu}{}{ⓔgumãgu}
\région{GOs PA BO}
(\domainesémantique{Relations et interaction sociales})
\classe{v ; n}
\begin{glose}
\pfra{vrai ; vérité}
\end{glose}
\begin{glose}
\pfra{dire la vérité ; avoir raison}
\end{glose}
\newline
\begin{sous-entrée}{gumãgu}{ⓔgumãguⓝgumãgu}
\begin{glose}
\pfra{c'est vrai}
\end{glose}
\newline
\begin{exemple}
\région{GO}
\textbf{\pnua{gumãguu-je}}
\pfra{il a raison}
\end{exemple}
\newline
\begin{exemple}
\région{BO}
\textbf{\pnua{ra gumãgu}}
\pfra{c'est vrai}
\end{exemple}
\newline
\begin{exemple}
\région{PA}
\textbf{\pnua{phaa-gumãgu !}}
\pfra{fais en sorte que cela soit vrai !}
\end{exemple}
\newline
\relationsémantique{Cf.}{\lien{}{gu-mã-gu}}
\glosecourte{très vrai}
\end{sous-entrée}
\end{entrée}

\begin{entrée}{gu-mwa}{}{ⓔgu-mwa}
\région{GOs PA BO}
(\domainesémantique{Types de maison, architecture de la maison})
\classe{nom}
\begin{glose}
\pfra{maison ronde}
\end{glose}
\end{entrée}

\begin{entrée}{gunebwa}{}{ⓔgunebwa}
\région{PA BO}
(\domainesémantique{Religion, représentations religieuses})
\classe{nom}
\begin{glose}
\pfra{lézard (être mythologique qui protège les cultures) [Corne]}
\end{glose}
\end{entrée}

\begin{entrée}{guni}{}{ⓔguni}
\région{GOs}
\variante{%
gunim
\région{BO PA}}
(\domainesémantique{Santé, maladie})
\classe{v}
\begin{glose}
\pfra{infecté ; infecter (s')}
\end{glose}
\newline
\begin{exemple}
\région{GO}
\textbf{\pnua{e guni kòò-nu}}
\pfra{j'ai la jambe infectée}
\end{exemple}
\end{entrée}

\begin{entrée}{gu-traabwa}{}{ⓔgu-traabwa}
\région{GOs}
(\domainesémantique{Préfixes et verbes de position})
\classe{v}
\begin{glose}
\pfra{assis sans bouger}
\end{glose}
\end{entrée}

\begin{entrée}{gu-we}{}{ⓔgu-we}
\région{GOs}
(\domainesémantique{Eau})
\classe{nom}
\begin{glose}
\pfra{eau potable}
\end{glose}
\end{entrée}

\begin{entrée}{guya}{1}{ⓔguyaⓗ1}
\région{GOs}
\variante{%
gwayal
\région{PA}}
(\domainesémantique{Arbre})
\classe{nom}
\begin{glose}
\pfra{goyavier}
\end{glose}
\nomscientifique{Psidium guajava L. (Myrtacées)}
\end{entrée}

\begin{entrée}{guya}{2}{ⓔguyaⓗ2}
\région{GOs}
(\domainesémantique{Insectes})
\classe{nom}
\begin{glose}
\pfra{ver de bancoul (long avec des pattes)}
\end{glose}
\end{entrée}

\newpage

\lettrine{h}\begin{entrée}{ha}{1}{ⓔhaⓗ1}
\région{GOs PA}
(\domainesémantique{Mammifères})
\classe{nom}
\begin{glose}
\pfra{roussette (petite, de rocher)}
\end{glose}
\end{entrée}

\begin{entrée}{ha}{2}{ⓔhaⓗ2}
\région{GOs}
(\domainesémantique{Description des objets, formes, consistance, taille})
\classe{v.i.}
\begin{glose}
\pfra{cassé (verre)}
\end{glose}
\newline
\begin{exemple}
\textbf{\pnua{e ha za}}
\pfra{l'assiette est cassée}
\end{exemple}
\newline
\note{v.t. hale}{grammaire}{casser qqch}
\end{entrée}

\begin{entrée}{hã}{1}{ⓔhãⓗ1}
\formephonétique{hɛ̃}
\région{GOs}
\variante{%
hãny
\région{BO}}
(\domainesémantique{Navigation})
\classe{v ; n}
\begin{glose}
\pfra{gouvernail}
\end{glose}
\begin{glose}
\pfra{pagaie}
\end{glose}
\begin{glose}
\pfra{diriger ; conduire}
\end{glose}
\newline
\note{hãnge (v.t.)}{grammaire}{diriger qqch}
\end{entrée}

\begin{entrée}{hã}{2}{ⓔhãⓗ2}
\région{PA BO}
\variante{%
hangai
\région{PA BO}}
(\domainesémantique{Description des objets, formes, consistance, taille})
\classe{v}
\begin{glose}
\pfra{gros ; grand}
\end{glose}
\end{entrée}

\begin{entrée}{hã}{3}{ⓔhãⓗ3}
\région{BO}
(\domainesémantique{Pronoms})
\classe{PRO 2° pers. PL}
\begin{glose}
\pfra{vous (plur.)}
\end{glose}
\newline
\begin{exemple}
\textbf{\pnua{hã u nòòl !}}
\pfra{réveillez-vous !}
\end{exemple}
\end{entrée}

\begin{entrée}{haa}{1}{ⓔhaaⓗ1}
\région{GOs PA BO}
(\domainesémantique{Aspect})
\classe{ASP}
\begin{glose}
\pfra{sans arrêt ; sans cesse ; tout le temps}
\end{glose}
\newline
\begin{exemple}
\textbf{\pnua{i ra haa kiga}}
\pfra{il ne fait que rire}
\end{exemple}
\newline
\begin{sous-entrée}{haa-nonòm}{ⓔhaaⓗ1ⓝhaa-nonòm}
\région{PA}
\begin{glose}
\pfra{penser sans arrêt}
\end{glose}
\end{sous-entrée}
\newline
\begin{sous-entrée}{i haa-vhaa}{ⓔhaaⓗ1ⓝi haa-vhaa}
\région{PA}
\begin{glose}
\pfra{il parle sans cesse}
\end{glose}
\end{sous-entrée}
\end{entrée}

\begin{entrée}{haa}{2}{ⓔhaaⓗ2}
\région{GOs PA}
(\domainesémantique{Préfixes classificateurs numériques})
\classe{CLF.NUM (tissus et étoffes végétales)}
\begin{glose}
\pfra{tissus et étoffes végétales}
\end{glose}
\newline
\begin{exemple}
\textbf{\pnua{haa-xe ci-xãbwa, thrô, murò}}
\pfra{une étoffe, natte, couverture}
\end{exemple}
\newline
\begin{exemple}
\textbf{\pnua{haa-xe, haa-tru, haa-ko, haa-pa, haa-ni ci-xãbwa, etc.}}
\pfra{un, deux, trois tissus, quatre, cinq étoffes, etc.}
\end{exemple}
\newline
\begin{exemple}
\région{GO PA}
\textbf{\pnua{haa-ru mhaû ci-xãbwa}}
\pfra{deux rouleaux d'étoffe}
\end{exemple}
\newline
\begin{exemple}
\région{GO}
\textbf{\pnua{e tree-kuzaò haa-tru mada}}
\pfra{il reste 2 pièces de tissu en plus}
\end{exemple}
\end{entrée}

\begin{entrée}{haal}{}{ⓔhaal}
\région{PA BO}
(\domainesémantique{Navigation})
\classe{v}
\begin{glose}
\pfra{ramer}
\end{glose}
\newline
\begin{exemple}
\région{BO}
\textbf{\pnua{nu haale wòny}}
\pfra{je fais avancer le bateau à la rame}
\end{exemple}
\newline
\begin{sous-entrée}{a-haal}{ⓔhaalⓝa-haal}
\région{BO}
\begin{glose}
\pfra{rameur}
\end{glose}
\end{sous-entrée}
\newline
\begin{sous-entrée}{ba-haal}{ⓔhaalⓝba-haal}
\région{BO}
\begin{glose}
\pfra{rame}
\end{glose}
\newline
\note{v.i. haale}{grammaire}{conduire qqch à la rame}
\end{sous-entrée}
\newline
\étymologie{
\langue{PSO}
\étymon{*xalo(r)}
\auteur{Geraghty}}
\end{entrée}

\begin{entrée}{hããmal}{}{ⓔhããmal}
\région{PA BO}
(\domainesémantique{Sentiments})
\classe{v}
\begin{glose}
\pfra{émerveiller (s') ; émerveillé (être) ; admiratif}
\end{glose}
\newline
\begin{exemple}
\région{PA}
\textbf{\pnua{nu hããmali-je}}
\pfra{je l'admire}
\end{exemple}
\newline
\relationsémantique{Cf.}{\lien{}{gaajò ; gaajòn}}
\glosecourte{s'étonner, être surpris}
\end{entrée}

\begin{entrée}{hããni}{}{ⓔhããni}
\région{BO}
(\domainesémantique{Société
, Fonctions naturelles humaines})
\classe{v}
\begin{glose}
\pfra{voir (registre respectueux pour le Grand Chef) [BM]}
\end{glose}
\newline
\begin{exemple}
\région{BO}
\textbf{\pnua{nu hãã-je}}
\pfra{je l'ai vu}
\end{exemple}
\end{entrée}

\begin{entrée}{haa-nu}{}{ⓔhaa-nu}
\région{PA BO}
(\domainesémantique{Cocotiers})
\classe{nom}
\begin{glose}
\pfra{gaine de l'inflorescence du cocotier (gaine sèche qui reste de l'inflorescence)}
\end{glose}
\newline
\relationsémantique{Cf.}{\lien{}{haa-n}}
\glosecourte{ce qui reste}
\end{entrée}

\begin{entrée}{hããny}{}{ⓔhããny}
\région{PA BO}
(\domainesémantique{Navigation})
\classe{v ; n}
\begin{glose}
\pfra{diriger (voiture)}
\end{glose}
\begin{glose}
\pfra{barrer (bateau)}
\end{glose}
\begin{glose}
\pfra{gouvernail ; volant}
\end{glose}
\newline
\begin{exemple}
\textbf{\pnua{i phe hããny}}
\pfra{il prend le gouvernail}
\end{exemple}
\end{entrée}

\begin{entrée}{haa-uva}{}{ⓔhaa-uva}
\région{GOs PA BO}
(\domainesémantique{Taros})
\classe{nom}
\begin{glose}
\pfra{extrémité inférieure du pétiole de taro d'eau}
\end{glose}
\newline
\note{il est pressé dans la bouche du nouveau-né par la mère.}{glose}{}
\end{entrée}

\begin{entrée}{haavwu}{}{ⓔhaavwu}
\formephonétique{'haːβu}
\région{GOs PA BO [Corne]}
\variante{%
haapu
}
(\domainesémantique{Types de champs})
\classe{nom}
\begin{glose}
\pfra{jardin (au bord de rivière) ; massif de culture humide (comme les taros)}
\end{glose}
\begin{glose}
\pfra{terre alluvionnaire (au bord de rivière et au pied des montagnes)}
\end{glose}
\end{entrée}

\begin{entrée}{haaxa}{}{ⓔhaaxa}
\région{BO}
\classe{nom}
\newline
\sens{1}
(\domainesémantique{Instruments})
\begin{glose}
\pfra{baguette ; canne [BO Corne]}
\end{glose}
\newline
\sens{2}
(\domainesémantique{Chasse})
\begin{glose}
\pfra{armature de piège [BO Corne]}
\end{glose}
\end{entrée}

\begin{entrée}{hããxa}{}{ⓔhããxa}
\formephonétique{hɛ̃ːɣa}
\région{GOs BO}
\variante{%
haaka
\région{BO}}
(\domainesémantique{Relations et interaction sociales})
\classe{v.i.}
\begin{glose}
\pfra{craintif ; peureux ; craindre ; avoir peur}
\end{glose}
\begin{glose}
\pfra{hésiter}
\end{glose}
\newline
\begin{exemple}
\textbf{\pnua{nu hããxa ne nu a-da-o}}
\pfra{j'ai peur de monter (vers toi)}
\end{exemple}
\newline
\begin{exemple}
\région{PA}
\textbf{\pnua{nu hããxa na i poxèè thûã}}
\pfra{j'ai peur qu'il ne joue des tours}
\end{exemple}
\newline
\begin{exemple}
\région{BO}
\textbf{\pnua{i hããxe-je}}
\pfra{il a peur de lui}
\end{exemple}
\newline
\note{hããxe (v.t.)}{grammaire}{craindre qqn}
\newline
\note{pha-hããxe}{grammaire}{effrayer, faire peur à}
\newline
\note{paza-hããxe}{grammaire}{menacer qqn}
\end{entrée}

\begin{entrée}{hããxe}{}{ⓔhããxe}
\formephonétique{hɛ̃ːɣe}
\région{GOs BO}
(\domainesémantique{Relations et interaction sociales})
\classe{v.t.}
\begin{glose}
\pfra{craindre (qqn, qqch)}
\end{glose}
\newline
\begin{exemple}
\textbf{\pnua{kebwa (jö) hããxe-bi mã ãbaa-nu}}
\pfra{n'aie pas peur de moi et de mes frères}
\end{exemple}
\newline
\note{pha-hããxe}{grammaire}{effrayer, faire peur à qqn}
\end{entrée}

\begin{entrée}{haaxo}{}{ⓔhaaxo}
\région{GOs PA BO}
(\domainesémantique{Insectes})
\classe{nom}
\begin{glose}
\pfra{ver de bancoul (gros, blanc et comestible)}
\end{glose}
\end{entrée}

\begin{entrée}{haba}{}{ⓔhaba}
\région{BO}
(\domainesémantique{Prépositions})
\classe{PREP.BENEF}
\begin{glose}
\pfra{pour}
\end{glose}
\newline
\begin{exemple}
\textbf{\pnua{ha-ò wòny ja ne-wo haba i jo}}
\pfra{ce bateau, nous l'avons fait pour toi}
\end{exemple}
\newline
\note{non vérifié}{général}{}
\end{entrée}

\begin{entrée}{hãbaö}{}{ⓔhãbaö}
\région{GOs BO}
(\domainesémantique{Modalité, verbes modaux})
\classe{v.MODAL}
\begin{glose}
\pfra{utile ; nécessaire ; falloir}
\end{glose}
\newline
\begin{exemple}
\région{GO}
\textbf{\pnua{kixa na hãbaö vwo jö mã phe-uja lã-na}}
\pfra{il était inutile d'apporter toutes ces choses}
\end{exemple}
\newline
\begin{exemple}
\région{GO}
\textbf{\pnua{kixa na hãbaö ẽnõ-ã !}}
\pfra{cet enfant est inutile (= il n'aide pas)}
\end{exemple}
\newline
\begin{exemple}
\région{BO}
\textbf{\pnua{ka phe-mi ce o hãbao yaai !}}
\pfra{apporte du bois pour le feu}
\end{exemple}
\end{entrée}

\begin{entrée}{hãbe}{}{ⓔhãbe}
\région{GOs BO}
(\domainesémantique{Corps humain})
\classe{n (inaliénable)}
\begin{glose}
\pfra{aisselle}
\end{glose}
\newline
\begin{exemple}
\région{GOs}
\textbf{\pnua{phwe-hãbee je}}
\pfra{son aisselle}
\end{exemple}
\newline
\begin{exemple}
\textbf{\pnua{hãbee-n}}
\pfra{son aisselle}
\end{exemple}
\newline
\begin{sous-entrée}{pu-hãbe}{ⓔhãbeⓝpu-hãbe}
\begin{glose}
\pfra{poils des aisselles}
\end{glose}
\end{sous-entrée}
\end{entrée}

\begin{entrée}{hãbira}{}{ⓔhãbira}
\région{GOs}
(\domainesémantique{Portage})
\classe{v}
\begin{glose}
\pfra{porter sous le bras}
\end{glose}
\newline
\begin{exemple}
\textbf{\pnua{nu phe ci-xãbwa po nu kha-hãbira}}
\pfra{je prends l'étoffe et je l'emporte sous le bras}
\end{exemple}
\end{entrée}

\begin{entrée}{hãbo}{}{ⓔhãbo}
\région{GOs}
(\domainesémantique{Discours, échanges verbaux})
\classe{v}
\begin{glose}
\pfra{jurer ; promettre}
\end{glose}
\begin{glose}
\pfra{prêter serment}
\end{glose}
\end{entrée}

\begin{entrée}{habwa}{}{ⓔhabwa}
\région{GOs}
(\domainesémantique{Mouvements ou actions faits avec le corps, les bras, les mains, les pieds})
\classe{v}
\begin{glose}
\pfra{soulever (lentement)}
\end{glose}
\newline
\begin{exemple}
\textbf{\pnua{habwa hi-jo}}
\pfra{lève ton bras}
\end{exemple}
\end{entrée}

\begin{entrée}{hãda}{}{ⓔhãda}
\région{GOs PA BO}
\variante{%
(h)ada
}
(\domainesémantique{Marques restrictives})
\classe{ADV ; QNT}
\begin{glose}
\pfra{seul ; tout seul ; seulement}
\end{glose}
\newline
\begin{exemple}
\textbf{\pnua{Hai ! za inu hãda ma ãbaa-nu nye bi a}}
\pfra{Non! c'est seulement ma soeur et moi qui sommes parties}
\end{exemple}
\newline
\begin{exemple}
\textbf{\pnua{Hai ! inu hãda ma ãbaa-nu nye bi a}}
\pfra{Non! c'est seulement ma soeur et moi qui sommes parties}
\end{exemple}
\newline
\begin{exemple}
\textbf{\pnua{Hai ! inu ma ãbaa-nu hãda nye bi a}}
\pfra{Non! c'est seulement ma soeur et moi qui sommes parties}
\end{exemple}
\newline
\begin{exemple}
\région{GO}
\textbf{\pnua{Axe za hine-hãdaa-ni xo kani}}
\pfra{mais seul le canard le savait}
\end{exemple}
\newline
\begin{exemple}
\textbf{\pnua{ili hãda}}
\pfra{eux deux seuls}
\end{exemple}
\newline
\begin{exemple}
\région{GO}
\textbf{\pnua{nu tròòli-adaa-ni thixèè ala-xòò-nu}}
\pfra{je n 'ai trouvé qu'une seule chaussure}
\end{exemple}
\newline
\begin{exemple}
\région{PA}
\textbf{\pnua{axe (h)ãda pòi-n}}
\pfra{il n'a qu'un enfant}
\end{exemple}
\end{entrée}

\begin{entrée}{hãgana}{}{ⓔhãgana}
\formephonétique{'hɛ̃ŋgaɳa, 'hɛ̃ŋgana}
\région{GOs BO PA}
(\domainesémantique{Adverbes déictiques de temps})
\classe{ADV}
\begin{glose}
\pfra{maintenant ; aujourd'hui ; tout-à-l'heure}
\end{glose}
\newline
\begin{exemple}
\région{GO}
\textbf{\pnua{hãgana na mi hovwo}}
\pfra{quand nous mangerons}
\end{exemple}
\newline
\begin{exemple}
\région{GO}
\textbf{\pnua{hãgana novwo kui zèèno}}
\pfra{maintenant que l'igname est mûre}
\end{exemple}
\newline
\begin{sous-entrée}{hãgana hâ}{ⓔhãganaⓝhãgana hâ}
\begin{glose}
\pfra{maintenant}
\end{glose}
\end{sous-entrée}
\newline
\begin{sous-entrée}{bala hãgana}{ⓔhãganaⓝbala hãgana}
\begin{glose}
\pfra{jusqu'à maintenant (Dubois)}
\end{glose}
\end{sous-entrée}
\end{entrée}

\begin{entrée}{hãgu}{}{ⓔhãgu}
\formephonétique{'hɛ̃ŋgu}
\région{GOs}
(\domainesémantique{Noms des plantes})
\classe{nom}
\begin{glose}
\pfra{roseau}
\end{glose}
\begin{glose}
\pfra{bambou (petit, pousse au bord de la rivière)}
\end{glose}
\end{entrée}

\begin{entrée}{hai}{1}{ⓔhaiⓗ1}
\région{PA BO}
\variante{%
ha, ai
\région{PA BO}, 
a
\région{GO(s)}}
(\domainesémantique{Conjonction})
\classe{CNJ}
\begin{glose}
\pfra{ou bien}
\end{glose}
\end{entrée}

\begin{entrée}{hai}{2}{ⓔhaiⓗ2}
\région{GOs PA BO}
\variante{%
hayai
\région{GO(s)}}
(\domainesémantique{Marques assertives})
\classe{NEG (en réponse à une question)}
\begin{glose}
\pfra{non}
\end{glose}
\newline
\relationsémantique{Ant.}{\lien{ⓔèlò}{èlò}}
\glosecourte{oui}
\end{entrée}

\begin{entrée}{hai}{3}{ⓔhaiⓗ3}
\région{GOs BO PA}
(\domainesémantique{Quantificateurs})
\classe{QNT}
\begin{glose}
\pfra{il y a beaucoup ; très (lié à "haivwo")}
\end{glose}
\newline
\begin{sous-entrée}{hai kãbu}{ⓔhaiⓗ3ⓝhai kãbu}
\région{GO}
\begin{glose}
\pfra{très froid}
\end{glose}
\newline
\begin{exemple}
\région{BO}
\textbf{\pnua{hai nèèng}}
\pfra{il y a des nuages, très nuageux}
\end{exemple}
\newline
\relationsémantique{Cf.}{\lien{}{haivo}}
\glosecourte{il y en a}
\newline
\relationsémantique{Cf.}{\lien{}{gele [GOs]}}
\glosecourte{il y a}
\end{sous-entrée}
\end{entrée}

\begin{entrée}{hai-kãbu}{}{ⓔhai-kãbu}
\région{GOs}
\variante{%
hai'xãbu
\région{GO(s)}}
(\domainesémantique{Température})
\classe{v.stat.}
\begin{glose}
\pfra{très froid}
\end{glose}
\end{entrée}

\begin{entrée}{hai-trirê}{}{ⓔhai-trirê}
\formephonétique{hai-'ʈiɽɛ̃}
\région{GOs}
\variante{%
haitritrê
\région{GO(s)}}
(\domainesémantique{Fonctions naturelles humaines})
\classe{v}
\begin{glose}
\pfra{suer ; transpirer (lit. très chaud)}
\end{glose}
\end{entrée}

\begin{entrée}{haivwö}{}{ⓔhaivwö}
\région{GOs BO PA}
\variante{%
hai
\région{BO}, 
haipo
\région{vx}}
(\domainesémantique{Quantificateurs})
\classe{QNT}
\begin{glose}
\pfra{beaucoup ; nombreux ; trop}
\end{glose}
\newline
\begin{exemple}
\région{GO}
\textbf{\pnua{haivwö chaamwa i nu}}
\pfra{j'ai beaucoup de bananes}
\end{exemple}
\newline
\begin{exemple}
\région{GO}
\textbf{\pnua{haivwö la mala phwe-meevwu Maluma xa la uça}}
\pfra{beaucoup de gens du clan Maluma sont venus}
\end{exemple}
\newline
\begin{exemple}
\textbf{\pnua{la ci haivwö}}
\pfra{ils sont très nombreux}
\end{exemple}
\newline
\begin{exemple}
\textbf{\pnua{la mã haivwö}}
\pfra{ils sont trop nombreux}
\end{exemple}
\newline
\begin{exemple}
\textbf{\pnua{la po haivwö nai la}}
\pfra{ils sont un peu plus nombreux qu'eux}
\end{exemple}
\newline
\relationsémantique{Ant.}{\lien{ⓔhoxèè}{hoxèè}}
\glosecourte{peu}
\end{entrée}

\begin{entrée}{hajaa-wõ}{}{ⓔhajaa-wõ}
\région{GOs}
(\domainesémantique{Navigation})
\classe{nom}
\begin{glose}
\pfra{balancier de pirogue}
\end{glose}
\end{entrée}

\begin{entrée}{halalaò}{}{ⓔhalalaò}
\région{PA}
(\domainesémantique{Description des objets, formes, consistance, taille})
\classe{v}
\begin{glose}
\pfra{lâche (être) ; grand}
\end{glose}
\newline
\begin{exemple}
\région{PA}
\textbf{\pnua{i (mha) halalaò jene patolõ}}
\pfra{il est lâche ce pantalon}
\end{exemple}
\end{entrée}

\begin{entrée}{halan}{}{ⓔhalan}
\région{PA BO [BM]}
(\domainesémantique{Description des objets, formes, consistance, taille})
\classe{v}
\begin{glose}
\pfra{gâté ; abîmé}
\end{glose}
\newline
\begin{exemple}
\région{PA BO}
\textbf{\pnua{i halan a kui}}
\pfra{cette igname est gâtée (trop vieux)}
\end{exemple}
\end{entrée}

\begin{entrée}{halaò}{}{ⓔhalaò}
\région{GOs}
(\domainesémantique{Santé, maladie})
\classe{v}
\begin{glose}
\pfra{peler (peau)}
\end{glose}
\newline
\begin{exemple}
\région{GO}
\textbf{\pnua{e halaò cii hii-nu ko/xo we tòò}}
\pfra{la peau de ma main pèle à cause de l'eau brûlante}
\end{exemple}
\newline
\begin{exemple}
\textbf{\pnua{e halaò cii hii-je ko we tòò}}
\pfra{la peau de ma main pèle à cause de l'eau brûlante (l'indice sujet 'e' réfère à 'cii hii-je')}
\end{exemple}
\end{entrée}

\begin{entrée}{hale}{}{ⓔhale}
\région{GOs}
(\domainesémantique{Verbes d'action (en général)})
\classe{v.t.}
\begin{glose}
\pfra{casser (verre, assiette)}
\end{glose}
\newline
\begin{exemple}
\textbf{\pnua{e hale za}}
\pfra{elle a cassé l'assiette}
\end{exemple}
\newline
\note{v.i. ha}{grammaire}{être cassé}
\end{entrée}

\begin{entrée}{halelewa}{}{ⓔhalelewa}
\formephonétique{'hale'lewa}
\région{GOs}
\variante{%
haleleang
\région{PA WE}, 
alelewa, halelea
\région{PA}}
(\domainesémantique{Insectes})
\classe{nom}
\begin{glose}
\pfra{cigale (petite et verte)}
\end{glose}
\newline
\begin{sous-entrée}{ba-thi-halelewa}{ⓔhalelewaⓝba-thi-halelewa}
\begin{glose}
\pfra{bâton à glu (sur lequel on colle les cigales)}
\end{glose}
\end{sous-entrée}
\end{entrée}

\begin{entrée}{hãnge}{1}{ⓔhãngeⓗ1}
\région{PA}
(\domainesémantique{Mouvements ou actions faits avec le corps, les bras, les mains, les pieds})
\classe{v}
\begin{glose}
\pfra{lever le bras en menaçant}
\end{glose}
\end{entrée}

\begin{entrée}{hãnge}{2}{ⓔhãngeⓗ2}
\région{GOs}
(\domainesémantique{Navigation})
\classe{v}
\begin{glose}
\pfra{diriger le bateau/voiture ; barrer ; conduire}
\end{glose}
\newline
\begin{exemple}
\textbf{\pnua{e hãnge wõ}}
\pfra{il dirige le bateau}
\end{exemple}
\newline
\begin{sous-entrée}{ba-hãnge wõ}{ⓔhãngeⓗ2ⓝba-hãnge wõ}
\begin{glose}
\pfra{gouvernail, barre}
\end{glose}
\newline
\note{hã (v.i.)}{grammaire}{barrer, diriger}
\end{sous-entrée}
\end{entrée}

\begin{entrée}{haòm}{}{ⓔhaòm}
\région{PA}
\variante{%
aom
\région{PA BO}}
(\domainesémantique{Description des objets, formes, consistance, taille})
\classe{v.stat.}
\begin{glose}
\pfra{léger}
\end{glose}
\newline
\relationsémantique{Ant.}{\lien{}{pwalu, pwaalu}}
\glosecourte{lourd}
\end{entrée}

\begin{entrée}{hau}{1}{ⓔhauⓗ1}
\région{PA}
(\domainesémantique{Vêtements, parure})
\classe{nom}
\begin{glose}
\pfra{coiffure (tout type)}
\end{glose}
\newline
\begin{exemple}
\région{PA}
\textbf{\pnua{hau-n}}
\pfra{son chapeau}
\end{exemple}
\newline
\begin{exemple}
\région{PA}
\textbf{\pnua{i phu hau-n}}
\pfra{il enlève son chapeau}
\end{exemple}
\end{entrée}

\begin{entrée}{hau}{2}{ⓔhauⓗ2}
\région{PA}
(\domainesémantique{Noms des plantes})
\classe{nom}
\begin{glose}
\pfra{liane utilisée pour attacher les gaulettes de la maison}
\end{glose}
\end{entrée}

\begin{entrée}{haulaa}{}{ⓔhaulaa}
\formephonétique{'haulaː}
\région{GOs BO PA}
(\domainesémantique{Caractéristiques et propriétés des personnes})
\classe{v}
\begin{glose}
\pfra{faire l'intéressant ; faire le malin ; hautain}
\end{glose}
\newline
\begin{exemple}
\textbf{\pnua{i haulaa}}
\pfra{il est hautain, orgueilleux}
\end{exemple}
\end{entrée}

\begin{entrée}{hauva}{}{ⓔhauva}
\région{GOs PA BO}
(\domainesémantique{Coutumes, dons coutumiers})
\classe{nom}
\begin{glose}
\pfra{coutume (cérémonie) de deuil: don du clan paternel au clan maternel de l'épouse ou de l'époux}
\end{glose}
\newline
\begin{exemple}
\textbf{\pnua{hovwo za ? - hovwo ponga hauva}}
\pfra{quel type de nourriture ? - de la nourriture pour la levée de deuil}
\end{exemple}
\newline
\begin{sous-entrée}{thîni hauva}{ⓔhauvaⓝthîni hauva}
\région{GO}
\begin{glose}
\pfra{enclos (ceint par une barrière où l'on met les dons coutumiers destinés au clan maternel; les oncles maternels démolissent l'enclos et emportent alors les dons ; ne se pratique que pour les personnes de rang important)}
\end{glose}
\newline
\relationsémantique{Cf.}{\lien{ⓔmãû bweera}{mãû bweera}}
\glosecourte{cérémonie coutumière de deuil pour une femme (don du mari au clan paternel de l'épouse)}
\end{sous-entrée}
\end{entrée}

\begin{entrée}{havaa-n}{}{ⓔhavaa-n}
\région{PA BO}
\variante{%
hapa
\région{PA}}
\classe{n.LOC}
\newline
\sens{1}
(\domainesémantique{Topographie})
\begin{glose}
\pfra{bord}
\end{glose}
\begin{glose}
\pfra{extrémité (champ, natte) (désigne les fibres qui dépassent du tressage)}
\end{glose}
\newline
\sens{2}
(\domainesémantique{Cultures, techniques, boutures})
\begin{glose}
\pfra{billon ; côté femelle du massif d'ignames}
\end{glose}
\end{entrée}

\begin{entrée}{havade}{}{ⓔhavade}
\région{GOs}
(\domainesémantique{Bananiers et bananes})
\classe{nom}
\begin{glose}
\pfra{feuille sèche de bananier}
\end{glose}
\newline
\begin{exemple}
\textbf{\pnua{havadi chaamwa}}
\pfra{feuille sèche de bananier}
\end{exemple}
\end{entrée}

\begin{entrée}{hava-hi}{}{ⓔhava-hi}
\formephonétique{'haβa-'hi}
\région{GOs}
\variante{%
hava-hi-n
\région{BO PA}}
(\domainesémantique{Corps animal})
\classe{nom}
\begin{glose}
\pfra{aile}
\end{glose}
\newline
\begin{exemple}
\textbf{\pnua{hava-hi-je}}
\pfra{son aile}
\end{exemple}
\newline
\étymologie{
\langue{POc}
\étymon{*kapa(k)}}
\end{entrée}

\begin{entrée}{havan}{}{ⓔhavan}
\région{BO}
(\domainesémantique{Processus liés aux plantes})
\classe{v ; n}
\begin{glose}
\pfra{flétri ; fané ; mort ; feuille fanée}
\end{glose}
\newline
\relationsémantique{Cf.}{\lien{}{mènõ [GOs]}}
\glosecourte{flétri ; fané}
\end{entrée}

\begin{entrée}{haveveno}{}{ⓔhaveveno}
\région{PA}
\variante{%
haveno
\région{BO}}
(\domainesémantique{Insectes})
\classe{nom}
\begin{glose}
\pfra{grillon ; criquet}
\end{glose}
\end{entrée}

\begin{entrée}{havwo}{}{ⓔhavwo}
\formephonétique{'haβo}
\région{GOs}
\classe{nom}
(\domainesémantique{Arbre})
\begin{glose}
\pfra{bois de rose (pousse au bord de mer ; fleurs jaunes semblables à celles du bourao)}
\end{glose}
\nomscientifique{Thespesia populnea (Malvacées)}
\newline
\étymologie{
\langue{POc}
\étymon{*qaqaparu}}
\end{entrée}

\begin{entrée}{havwò}{}{ⓔhavwò}
\formephonétique{'haːβɔ}
\région{GOs}
\variante{%
hawòl
\région{BO}, 
ha
\région{GA}}
(\domainesémantique{Fonctions naturelles des animaux})
\classe{v}
\begin{glose}
\pfra{aboyer}
\end{glose}
\newline
\begin{exemple}
\textbf{\pnua{e ha eda ? (ou) e havwò eda ?}}
\pfra{pourquoi aboie-t-il?}
\end{exemple}
\newline
\begin{exemple}
\textbf{\pnua{nhye kuau e huu ti nhye e ha (ou) e havwò ?}}
\pfra{c'est le chien qui a mordu qui aboie?}
\end{exemple}
\newline
\begin{exemple}
\textbf{\pnua{la ha i loto (ou) la havwò i loto}}
\pfra{ils aboient contre les voitures}
\end{exemple}
\end{entrée}

\begin{entrée}{hãwã}{}{ⓔhãwã}
\région{GOs}
\variante{%
haawa
\région{BO [Corne]}}
(\domainesémantique{Coutumes, dons coutumiers})
\classe{nom}
\begin{glose}
\pfra{effet (d'une parole, d'un médicament, etc.)}
\end{glose}
\begin{glose}
\pfra{conséquence ; marque}
\end{glose}
\end{entrée}

\begin{entrée}{haxa}{}{ⓔhaxa}
\région{GOs}
(\domainesémantique{Aspect})
\classe{ASP}
\begin{glose}
\pfra{ne plus (avec négation)}
\end{glose}
\newline
\begin{exemple}
\textbf{\pnua{haxa kòi-je mwã na êna}}
\pfra{il n'est plus là maintenant}
\end{exemple}
\end{entrée}

\begin{entrée}{hãxãî}{}{ⓔhãxãî}
\formephonétique{hɛ̃'ɣã.î}
\région{GOs PA BO}
\variante{%
hangai, angai
\région{PA BO}, 
hã
\région{PA}}
(\domainesémantique{Description des objets, formes, consistance, taille})
\classe{v.stat.}
\begin{glose}
\pfra{grand ; gros ; volumineux}
\end{glose}
\end{entrée}

\begin{entrée}{haxe}{}{ⓔhaxe}
\région{GOs PA BO}
\variante{%
axe
\région{GO(s) PA BO}, 
age
\région{BO}}
\classe{CNJ}
\newline
\sens{1}
(\domainesémantique{Conjonction})
\begin{glose}
\pfra{et ; mais ; mais (entre-temps ; pendant ce temps)}
\end{glose}
\newline
\begin{exemple}
\textbf{\pnua{la phweeku, hake/haxe gi â po-ka aazo-ã}}
\pfra{ils bavardaient, pendant ce temps, le fils du chef pleurait}
\end{exemple}
\newline
\begin{exemple}
\région{GO}
\textbf{\pnua{nu mõgu hayu, haxe nu tròòli mã}}
\pfra{je travaille quand même mais je suis malade, bien que je sois malade}
\end{exemple}
\newline
\begin{exemple}
\région{BO}
\textbf{\pnua{kain age}}
\pfra{ensuite}
\end{exemple}
\newline
\begin{exemple}
\région{BO}
\textbf{\pnua{age novu}}
\pfra{(ce)pendant}
\end{exemple}
\newline
\sens{2}
(\domainesémantique{Conjonction})
\begin{glose}
\pfra{d'une part ... d'autre part}
\end{glose}
\end{entrée}

\begin{entrée}{haxo}{}{ⓔhaxo}
\région{GOs}
(\domainesémantique{Poissons})
\classe{nom}
\begin{glose}
\pfra{hareng}
\end{glose}
\end{entrée}

\begin{entrée}{hayu}{}{ⓔhayu}
\région{GOs BO}
\newline
\sens{1}
(\domainesémantique{Modalité, verbes modaux})
\classe{ADV ; MODIF}
\begin{glose}
\pfra{au hasard ; de ci de là}
\end{glose}
\begin{glose}
\pfra{quelconque ; n'importe comment ; en vain}
\end{glose}
\begin{glose}
\pfra{indécis}
\end{glose}
\begin{glose}
\pfra{sans retour ; sans but ; sans limite}
\end{glose}
\newline
\begin{sous-entrée}{êgu hayu}{ⓔhayuⓢ1ⓝêgu hayu}
\begin{glose}
\pfra{un homme quelconque}
\end{glose}
\end{sous-entrée}
\newline
\begin{sous-entrée}{e pe-a hayu}{ⓔhayuⓢ1ⓝe pe-a hayu}
\begin{glose}
\pfra{il va sans but}
\end{glose}
\end{sous-entrée}
\newline
\sens{2}
(\domainesémantique{Modalité, verbes modaux})
\classe{ADV ; MODIF}
\begin{glose}
\pfra{quand même ; faire à contre-coeur [GOs]}
\end{glose}
\newline
\begin{exemple}
\textbf{\pnua{ne-hayu-ni !}}
\pfra{fais-le quand même !}
\end{exemple}
\newline
\begin{exemple}
\textbf{\pnua{nee hayu}}
\pfra{fais-le quand même !}
\end{exemple}
\newline
\begin{sous-entrée}{pe-wa hayu}{ⓔhayuⓢ2ⓝpe-wa hayu}
\begin{glose}
\pfra{chante quand même}
\end{glose}
\newline
\relationsémantique{Cf.}{\lien{}{lòlò [GOs]}}
\glosecourte{sans but; au hasard}
\end{sous-entrée}
\end{entrée}

\begin{entrée}{haza ce}{}{ⓔhaza ce}
\région{GOs}
(\domainesémantique{Types de maison, architecture de la maison})
\classe{nom}
\begin{glose}
\pfra{gaulettes qui tiennent la paille (posées sur la paille et qui sont attachées ensuite par un lien)}
\end{glose}
\end{entrée}

\begin{entrée}{haza nu}{}{ⓔhaza nu}
\formephonétique{haða ɳu}
\région{GOs}
\variante{%
hara-nu
\région{PA}}
(\domainesémantique{Cocotiers})
\classe{nom}
\begin{glose}
\pfra{partie inférieure de la palme de cocotier (base arrondie de la palme là où elle s'attache au tronc)}
\end{glose}
\begin{glose}
\pfra{fibre de feuille de cocotier}
\end{glose}
\end{entrée}

\begin{entrée}{haze}{1}{ⓔhazeⓗ1}
\région{GOs}
\variante{%
hale
\région{PA BO}}
\newline
\sens{1}
(\domainesémantique{Description des objets, formes, consistance, taille})
\classe{v}
\begin{glose}
\pfra{différent ; à part ; à l'écart ; bizarre}
\end{glose}
\newline
\begin{exemple}
\région{GO}
\textbf{\pnua{na-mi xa pwaixe ne haze}}
\pfra{donne-moi autre chose}
\end{exemple}
\newline
\begin{exemple}
\région{BO}
\textbf{\pnua{avwono la hê-hale-ã}}
\pfra{les clans ennemis}
\end{exemple}
\newline
\begin{exemple}
\région{BO}
\textbf{\pnua{ra pe-hale va i la}}
\pfra{leurs paroles sont dispersées}
\end{exemple}
\newline
\sens{2}
(\domainesémantique{Conjonction})
\classe{CNJ}
\begin{glose}
\pfra{si (contrefactuel)}
\end{glose}
\newline
\begin{exemple}
\région{PA}
\textbf{\pnua{hale na yo tilòò, ye nu ru na hi-m va i la}}
\pfra{si tu l'avais demandé, je te l'aurais donné}
\end{exemple}
\end{entrée}

\begin{entrée}{haze}{2}{ⓔhazeⓗ2}
\région{GOs}
\région{BO}
\variante{%
hale
}
(\domainesémantique{Verbes d'action (en général)})
\classe{v}
\begin{glose}
\pfra{sécher (au soleil)}
\end{glose}
\newline
\note{v.t. hazee-ni [GOs], halee-ni [PA, BO]}{grammaire}{étendre le linge; sécher qqch (au soleil)}
\end{entrée}

\begin{entrée}{haze na}{}{ⓔhaze na}
\formephonétique{haðe ɳa}
\région{PA}
\variante{%
hale na
\région{PA}}
(\domainesémantique{Conjonction})
\classe{CNJ}
\begin{glose}
\pfra{si (contrefactuel)}
\end{glose}
\newline
\begin{exemple}
\région{PA}
\textbf{\pnua{hale na yo tilòò, ye nu ru na hi-m va i la}}
\pfra{si tu l'avais demandé, je te l'aurais donné}
\end{exemple}
\end{entrée}

\begin{entrée}{hazo}{}{ⓔhazo}
\région{GOs}
\région{PA BO}
\variante{%
halòòn
}
(\domainesémantique{Parenté
, Société})
\classe{v}
\begin{glose}
\pfra{marier (se) (pour une femme, prendre époux)}
\end{glose}
\newline
\begin{exemple}
\textbf{\pnua{la halòòn}}
\pfra{ils se marient}
\end{exemple}
\newline
\relationsémantique{Cf.}{\lien{ⓔe-mõû}{e-mõû}}
\glosecourte{époux}
\newline
\relationsémantique{Cf.}{\lien{}{phe thòòmwa [GOs]}}
\glosecourte{prendre une femme}
\end{entrée}

\begin{entrée}{he}{}{ⓔhe}
\région{GOs PA BO}
(\domainesémantique{Mouvements ou actions faits avec le corps, les bras, les mains, les pieds})
\classe{v ; n}
\begin{glose}
\pfra{frotter}
\end{glose}
\begin{glose}
\pfra{procréer ; saillir (un animal) ; accouplement}
\end{glose}
\newline
\begin{exemple}
\textbf{\pnua{hêê-he-co}}
\pfra{le fruit de ton accouplement}
\end{exemple}
\newline
\begin{exemple}
\textbf{\pnua{pa-he-ni}}
\pfra{saillir (un animal)}
\end{exemple}
\newline
\begin{exemple}
\textbf{\pnua{ba-he}}
\pfra{bois avec lequel on allume le feu par friction}
\end{exemple}
\newline
\begin{sous-entrée}{ce-he}{ⓔheⓝce-he}
\begin{glose}
\pfra{bois frotté pour allumer le feu}
\end{glose}
\end{sous-entrée}
\newline
\begin{sous-entrée}{yaai-he, yaa-he}{ⓔheⓝyaai-he, yaa-he}
\begin{glose}
\pfra{feu obtenu par friction de morceaux de bois}
\end{glose}
\end{sous-entrée}
\end{entrée}

\begin{entrée}{hè}{}{ⓔhè}
\région{PA}
(\domainesémantique{Interjection})
\classe{INTJ}
\begin{glose}
\pfra{interpellation}
\end{glose}
\newline
\begin{exemple}
\région{PA}
\textbf{\pnua{hè-yò}}
\pfra{hé! vous deux}
\end{exemple}
\newline
\begin{exemple}
\région{PA}
\textbf{\pnua{hè-zò}}
\pfra{hé! vous (plur.)}
\end{exemple}
\newline
\begin{exemple}
\région{PA}
\textbf{\pnua{hè-m}}
\pfra{hé! toi}
\end{exemple}
\end{entrée}

\begin{entrée}{hê-}{}{ⓔhê-}
\région{GOs}
\variante{%
hêê-n
\formephonétique{hẽː}
\région{PA BO}}
\newline
\sens{1}
(\domainesémantique{Préfixes sémantiques divers})
\classe{nom}
\begin{glose}
\pfra{contenu de}
\end{glose}
\newline
\begin{exemple}
\région{GO}
\textbf{\pnua{hêê-he-co}}
\pfra{le fruit de ton accouplement}
\end{exemple}
\newline
\begin{exemple}
\région{GO}
\textbf{\pnua{a-niza hê-loto ?}}
\pfra{combien de personnes contient cette voiture ?}
\end{exemple}
\newline
\begin{exemple}
\région{GO}
\textbf{\pnua{a-niza hê ?}}
\pfra{combien de personnes y a-t-il ? (dans un contenant: maison, voiture)}
\end{exemple}
\newline
\begin{exemple}
\textbf{\pnua{poniza hê loto ?}}
\pfra{combien contient la voiture ?}
\end{exemple}
\newline
\begin{exemple}
\région{GO}
\textbf{\pnua{kixa hê loto}}
\pfra{il n'y a personne dans la voiture}
\end{exemple}
\newline
\begin{exemple}
\région{GO}
\textbf{\pnua{kixa hê}}
\pfra{il n'y a rien dedans}
\end{exemple}
\newline
\begin{exemple}
\région{PA}
\textbf{\pnua{hêê-n}}
\pfra{son contenu}
\end{exemple}
\newline
\begin{exemple}
\région{PA}
\textbf{\pnua{kixa/kiya hêê-n}}
\pfra{il n'est pas chargé (réfère à un fusil)}
\end{exemple}
\newline
\begin{exemple}
\région{PA}
\textbf{\pnua{kixa/kiya hê-mwa}}
\pfra{la maison est vide d'objet}
\end{exemple}
\newline
\begin{exemple}
\région{BO}
\textbf{\pnua{kixa hêê-n}}
\pfra{c'est vide, il n'y a rien dedans}
\end{exemple}
\newline
\begin{sous-entrée}{hê-kòlò, hê-xòlò}{ⓔhê-ⓢ1ⓝhê-kòlò, hê-xòlò}
\begin{glose}
\pfra{famille (lit. contenu de la maison)}
\end{glose}
\end{sous-entrée}
\newline
\begin{sous-entrée}{hê-jigel}{ⓔhê-ⓢ1ⓝhê-jigel}
\région{PA}
\begin{glose}
\pfra{cartouche}
\end{glose}
\end{sous-entrée}
\newline
\begin{sous-entrée}{hêê-kio-n}{ⓔhê-ⓢ1ⓝhêê-kio-n}
\région{PA}
\begin{glose}
\pfra{ses viscères}
\end{glose}
\end{sous-entrée}
\newline
\begin{sous-entrée}{hê-wony}{ⓔhê-ⓢ1ⓝhê-wony}
\région{PA}
\begin{glose}
\pfra{les marchandises ou les choses dans la cale du bateau}
\end{glose}
\end{sous-entrée}
\newline
\begin{sous-entrée}{hê-mwa}{ⓔhê-ⓢ1ⓝhê-mwa}
\région{BO}
\begin{glose}
\pfra{l'intérieur / le contenu de la maison}
\end{glose}
\end{sous-entrée}
\newline
\begin{sous-entrée}{hê-mwa-iyu}{ⓔhê-ⓢ1ⓝhê-mwa-iyu}
\begin{glose}
\pfra{marchandise}
\end{glose}
\end{sous-entrée}
\newline
\begin{sous-entrée}{hê-êgu}{ⓔhê-ⓢ1ⓝhê-êgu}
\région{BO}
\begin{glose}
\pfra{la personnalité d'un homme}
\end{glose}
\end{sous-entrée}
\newline
\sens{2}
(\domainesémantique{Dons, échanges, achat et vente, vol})
\classe{nom}
\begin{glose}
\pfra{don ; cadeau}
\end{glose}
\newline
\begin{sous-entrée}{hê-hnawo}{ⓔhê-ⓢ2ⓝhê-hnawo}
\région{PA}
\begin{glose}
\pfra{don coutumier}
\end{glose}
\newline
\relationsémantique{Cf.}{\lien{}{PPN *(ka)kano}}
\end{sous-entrée}
\end{entrée}

\begin{entrée}{hêbu}{}{ⓔhêbu}
\région{GOs}
\variante{%
hêbun
\région{PA BO}}
\newline
\groupe{A}
(\domainesémantique{Localisation})
\classe{v}
\begin{glose}
\pfra{devant (être)}
\end{glose}
\begin{glose}
\pfra{premier (être)}
\end{glose}
\newline
\begin{exemple}
\région{GO}
\textbf{\pnua{a hêbu !}}
\pfra{va/pars devant!}
\end{exemple}
\newline
\relationsémantique{Ant.}{\lien{}{mura [GOs], mun [PA]}}
\glosecourte{après}
\newline
\relationsémantique{Ant.}{\lien{ⓔmu}{mu}}
\glosecourte{derrière, après}
\newline
\groupe{B}
(\domainesémantique{Conjonction})
\classe{PREP ; ADV ; CNJ}
\begin{glose}
\pfra{devant ; avant , d'abord}
\end{glose}
\newline
\begin{exemple}
\région{GO}
\textbf{\pnua{hêbu na i nu}}
\pfra{devant moi}
\end{exemple}
\newline
\begin{exemple}
\région{PA}
\textbf{\pnua{hêbun i yu}}
\pfra{devant toi}
\end{exemple}
\newline
\begin{exemple}
\région{PA}
\textbf{\pnua{hêbun ni nye mwa}}
\pfra{devant cette maison}
\end{exemple}
\newline
\begin{sous-entrée}{hêbu na}{ⓔhêbuⓝhêbu na}
\begin{glose}
\pfra{avant de/que}
\end{glose}
\newline
\begin{exemple}
\textbf{\pnua{hêbu na ni hovo}}
\pfra{avant de manger}
\end{exemple}
\newline
\begin{exemple}
\textbf{\pnua{hêbu na nu hovwo}}
\pfra{avant que je mange}
\end{exemple}
\end{sous-entrée}
\end{entrée}

\begin{entrée}{hê-cò}{}{ⓔhê-cò}
\région{GO}
(\domainesémantique{Corps humain})
\classe{nom}
\begin{glose}
\pfra{gland}
\end{glose}
\end{entrée}

\begin{entrée}{hê-drewaa}{}{ⓔhê-drewaa}
\région{GOs}
(\domainesémantique{Pêche})
\classe{nom}
\begin{glose}
\pfra{prise à la nasse}
\end{glose}
\newline
\begin{exemple}
\région{GO}
\textbf{\pnua{hê-drewaa-nu la kula}}
\pfra{j'ai pêché ces crevettes à la nasse}
\end{exemple}
\end{entrée}

\begin{entrée}{hêêbwi}{}{ⓔhêêbwi}
\région{BO}
(\domainesémantique{Mouvements ou actions faits avec le corps, les bras, les mains, les pieds})
\classe{v}
\begin{glose}
\pfra{fermer la main [Corne]}
\end{glose}
\newline
\note{non vérifié}{général}{}
\end{entrée}

\begin{entrée}{hêêdo}{}{ⓔhêêdo}
\région{GOs WEM BO PA}
(\domainesémantique{Oiseaux})
\classe{nom}
\begin{glose}
\pfra{"grive" ; oiseau-moine}
\end{glose}
\nomscientifique{Philemon diemenensis (Méliphagidés)}
\end{entrée}

\begin{entrée}{heela}{}{ⓔheela}
\région{GOs BO}
\classe{v}
\newline
\sens{1}
(\domainesémantique{Verbes de mouvement})
\begin{glose}
\pfra{glisser}
\end{glose}
\newline
\sens{2}
(\domainesémantique{Description des objets, formes, consistance, taille})
\begin{glose}
\pfra{glissant}
\end{glose}
\begin{glose}
\pfra{lisse (cheveux)}
\end{glose}
\newline
\begin{exemple}
\textbf{\pnua{e heela dè}}
\pfra{la route est glissante}
\end{exemple}
\end{entrée}

\begin{entrée}{hêgi}{}{ⓔhêgi}
\formephonétique{hẽŋgi}
\région{GOs PA BO}
(\domainesémantique{Objets coutumiers})
\classe{nom}
\begin{glose}
\pfra{monnaie traditionnelle}
\end{glose}
\newline
\begin{sous-entrée}{hêgi pulo}{ⓔhêgiⓝhêgi pulo}
\begin{glose}
\pfra{monnaie blanche (faite de coquillages blancs, Charles)}
\end{glose}
\end{sous-entrée}
\newline
\begin{sous-entrée}{hegi mhõõng}{ⓔhêgiⓝhegi mhõõng}
\région{PA}
\begin{glose}
\pfra{monnaie offerte avec un rameau de n'importe quel arbre, des noeuds sur la cordelette séparent les coquillages blancs enfilés (Charles)}
\end{glose}
\newline
\relationsémantique{Cf.}{\lien{}{mwoni [empr. GB money]}}
\glosecourte{argent}
\end{sous-entrée}
\end{entrée}

\begin{entrée}{hêgo}{}{ⓔhêgo}
\formephonétique{hẽŋgo}
\région{GOs PA BO}
(\domainesémantique{Instruments})
\classe{nom}
\begin{glose}
\pfra{bâton ; canne (pour marcher ; symbole d'ancienneté)}
\end{glose}
\end{entrée}

\begin{entrée}{hê-jige}{}{ⓔhê-jige}
\région{GOs}
(\domainesémantique{Armes})
\classe{nom}
\begin{glose}
\pfra{cartouche (de fusil)}
\end{glose}
\end{entrée}

\begin{entrée}{hê-ka}{}{ⓔhê-ka}
\région{GO}
(\domainesémantique{Noms des plantes})
\classe{nom}
\begin{glose}
\pfra{brède}
\end{glose}
\nomscientifique{Solanum nigrum L. (Solanacées)}
\end{entrée}

\begin{entrée}{hê-kee}{}{ⓔhê-kee}
\région{GOs}
(\domainesémantique{Pêche
, Chasse})
\classe{nom}
\begin{glose}
\pfra{prise (générique: à la pêche ou la chasse ; lit. ce qu'on ramène de son panier, mais lexicalisé pour toute prise)}
\end{glose}
\newline
\begin{exemple}
\région{GO}
\textbf{\pnua{ge le xa hê-kee-jö ? - hû, ge le drube - ge le nò}}
\pfra{as-tu pris quelque chose ? - oui, un cerf, des poissons}
\end{exemple}
\newline
\begin{exemple}
\région{GO}
\textbf{\pnua{hê-kee-nu xa nu taaja xo dö}}
\pfra{j'ai pris cela en piquant à la sagaie}
\end{exemple}
\newline
\begin{exemple}
\région{GO}
\textbf{\pnua{hê-kee-nu la lapya nu khai pwio}}
\pfra{j'ai pris ces tilapia (lit. en tirant la senne)}
\end{exemple}
\newline
\begin{exemple}
\région{GO}
\textbf{\pnua{hê-kee-nu la no nu pawe pwiò}}
\pfra{j'ai pris ces poissons au filet épervier (lit. en lançant le filet épervier)}
\end{exemple}
\newline
\begin{exemple}
\région{GO}
\textbf{\pnua{hê-kee-jö nya lapya ?}}
\pfra{c'est toi qui as pris ces tilapia ? (ils ne sont pas toujours dans un panier)}
\end{exemple}
\end{entrée}

\begin{entrée}{hê-kênii}{}{ⓔhê-kênii}
\région{GOs}
(\domainesémantique{Vêtements, parure})
\classe{nom}
\begin{glose}
\pfra{boucle d'oreille}
\end{glose}
\begin{glose}
\pfra{parure d'oreille}
\end{glose}
\end{entrée}

\begin{entrée}{hê-kòlò-}{}{ⓔhê-kòlò-}
\région{GOsPA BO}
\variante{%
hê-xòlò
\région{GO(s)}}
\classe{nom}
\newline
\sens{1}
(\domainesémantique{Parenté})
\begin{glose}
\pfra{enfant de frère et de cousins (femme parlant)}
\end{glose}
\newline
\begin{exemple}
\région{PA}
\textbf{\pnua{hê-kòlò-ny}}
\pfra{mes neveux/nièces (enfants de frère, femme parlant)}
\end{exemple}
\newline
\begin{exemple}
\région{BO}
\textbf{\pnua{li a-pe-hê-kòlò-li}}
\pfra{ils sont tante et neveux/nièces (enfants de frère)}
\end{exemple}
\newline
\sens{2}
(\domainesémantique{Parenté})
\begin{glose}
\pfra{famille ; allié}
\end{glose}
\newline
\begin{exemple}
\région{BO}
\textbf{\pnua{avwono la hê-kòlò-ã}}
\pfra{les clans alliés}
\end{exemple}
\newline
\begin{exemple}
\région{GO}
\textbf{\pnua{pe-hê-xòlò-lò}}
\pfra{ils (3) sont de la même famille}
\end{exemple}
\newline
\note{terme d'appellation : enõ 'tantine'}{général}{}
\end{entrée}

\begin{entrée}{hèlè}{}{ⓔhèlè}
\formephonétique{hɛlɛ}
\région{GO PA BO}
\classe{v ; n}
\newline
\sens{1}
(\domainesémantique{Outils})
\begin{glose}
\pfra{couteau}
\end{glose}
\begin{glose}
\pfra{scie(r)}
\end{glose}
\begin{glose}
\pfra{couper}
\end{glose}
\newline
\begin{exemple}
\région{GO}
\textbf{\pnua{hèlè ã abaa-nu}}
\pfra{le couteau de mon frère}
\end{exemple}
\newline
\begin{exemple}
\région{GO}
\textbf{\pnua{hèlè-nu}}
\pfra{mon couteau}
\end{exemple}
\newline
\begin{exemple}
\région{BO}
\textbf{\pnua{hèlè-ny}}
\pfra{mon couteau}
\end{exemple}
\newline
\begin{exemple}
\région{PA}
\textbf{\pnua{hèlè-ny hèlè-ce}}
\pfra{ma tronçonneuse (couteau à bois)}
\end{exemple}
\newline
\begin{sous-entrée}{hèlè pono}{ⓔhèlèⓢ1ⓝhèlè pono}
\région{GO}
\begin{glose}
\pfra{petit couteau}
\end{glose}
\end{sous-entrée}
\newline
\begin{sous-entrée}{hèlè pobe}{ⓔhèlèⓢ1ⓝhèlè pobe}
\région{BO}
\begin{glose}
\pfra{petit couteau}
\end{glose}
\end{sous-entrée}
\newline
\begin{sous-entrée}{hèlè hangai}{ⓔhèlèⓢ1ⓝhèlè hangai}
\région{PA}
\begin{glose}
\pfra{sabre d'abatis}
\end{glose}
\end{sous-entrée}
\newline
\begin{sous-entrée}{hèlè kawali}{ⓔhèlèⓢ1ⓝhèlè kawali}
\région{GO}
\begin{glose}
\pfra{sabre d'abatis}
\end{glose}
\newline
\note{v.t. hèlèèni}{grammaire}{couper qqch.}
\end{sous-entrée}
\newline
\sens{2}
(\domainesémantique{Jeux divers})
\begin{glose}
\pfra{figure du jeu de ficelle (la scie)}
\end{glose}
\newline
\emprunt{hele (POLYN) (POc *sele)}
\end{entrée}

\begin{entrée}{hèlè kawali}{}{ⓔhèlè kawali}
\région{GOs}
\variante{%
hèlè hã
\région{WE}}
(\domainesémantique{Outils})
\classe{nom}
\begin{glose}
\pfra{sabre d'abatis}
\end{glose}
\end{entrée}

\begin{entrée}{hè-m !}{}{ⓔhè-m !}
\région{PA BO}
(\domainesémantique{Interpellation})
\classe{interpellation}
\begin{glose}
\pfra{eh toi !}
\end{glose}
\newline
\begin{exemple}
\textbf{\pnua{hè-yò !}}
\pfra{eh vous deux!}
\end{exemple}
\end{entrée}

\begin{entrée}{hê-me}{}{ⓔhê-me}
\région{GOs}
(\domainesémantique{Corps humain})
\classe{nom}
\begin{glose}
\pfra{globe oculaire}
\end{glose}
\end{entrée}

\begin{entrée}{henim}{}{ⓔhenim}
\région{PA BO [BM]}
\classe{v ; n}
\newline
\sens{1}
(\domainesémantique{Mouvements ou actions faits avec le corps, les bras, les mains, les pieds})
\begin{glose}
\pfra{taper avec un bâton (pour taper les roussettes ou gauler les grappes de coco)}
\end{glose}
\newline
\begin{exemple}
\textbf{\pnua{i hènime nu}}
\pfra{il tape le coco avec un bâton (pour le faire tomber)}
\end{exemple}
\newline
\sens{2}
(\domainesémantique{Armes})
\begin{glose}
\pfra{bâton pour lancer}
\end{glose}
\end{entrée}

\begin{entrée}{hênu}{}{ⓔhênu}
\formephonétique{hẽɳu}
\région{GOs}
\variante{%
hînu
\région{PA}, 
hênuul
\région{BO}}
\classe{nom}
\newline
\sens{1}
(\domainesémantique{Lumière et obscurité})
\begin{glose}
\pfra{ombre (portée d'un animé)}
\end{glose}
\begin{glose}
\pfra{reflet}
\end{glose}
\newline
\begin{exemple}
\région{GO BO}
\textbf{\pnua{hênuã ce}}
\pfra{l'ombre d'un arbre}
\end{exemple}
\newline
\sens{2}
(\domainesémantique{Caractéristiques et propriétés des personnes})
\begin{glose}
\pfra{image ; photo ; portrait (représentation)}
\end{glose}
\newline
\begin{exemple}
\région{GO}
\textbf{\pnua{e noole hênuã-nu na ni vea}}
\pfra{il a vu mon reflet dans la vitre}
\end{exemple}
\newline
\begin{exemple}
\région{GO}
\textbf{\pnua{e alö-le hênuã-je na ni we}}
\pfra{elle regarde son reflet dans l'eau}
\end{exemple}
\newline
\begin{exemple}
\région{PA}
\textbf{\pnua{hînua-n}}
\pfra{son ombre, son image, sa photo}
\end{exemple}
\newline
\begin{exemple}
\région{PA}
\textbf{\pnua{pòi-ny nye hînu}}
\pfra{c'est une photo que j'ai prise}
\end{exemple}
\newline
\begin{exemple}
\région{PA}
\textbf{\pnua{hînua-ny}}
\pfra{ma photo (me représentant)}
\end{exemple}
\newline
\étymologie{
\langue{POc}
\étymon{*(q)anunu}}
\newline
\note{hênuã : forme déterminée}{grammaire}{}
\end{entrée}

\begin{entrée}{hê-nu}{}{ⓔhê-nu}
\formephonétique{hẽ-ɳu}
\région{GOs}
(\domainesémantique{Cocotiers})
\classe{nom}
\begin{glose}
\pfra{chair de coco (lit. contenu de la noix de coco)}
\end{glose}
\end{entrée}

\begin{entrée}{hênuã-me}{}{ⓔhênuã-me}
\formephonétique{hẽɳuɛ̃ me}
\région{GOs}
(\domainesémantique{Vêtements, parure})
\classe{nom}
\begin{glose}
\pfra{visière (faite d'une feuille ; lit. ombre de l'oeil)}
\end{glose}
\end{entrée}

\begin{entrée}{hèò}{}{ⓔhèò}
\région{PA}
\variante{%
-èò
\région{PA}}
(\domainesémantique{Démonstratifs})
\classe{DEM SG}
\begin{glose}
\pfra{réfère au passé (plus loin dans le temps que -ò)}
\end{glose}
\newline
\begin{exemple}
\région{PA}
\textbf{\pnua{ia je loto hèò ?}}
\pfra{où est ta voiture d'avant ?}
\end{exemple}
\newline
\begin{exemple}
\région{GO}
\textbf{\pnua{ia loto i cö-ò ?}}
\pfra{où est ta voiture d'avant ?}
\end{exemple}
\newline
\begin{exemple}
\région{PA}
\textbf{\pnua{liè emõû-n mali-èò}}
\pfra{ces deux époux (référence à un passé plus lointain)}
\end{exemple}
\newline
\begin{exemple}
\région{PA}
\textbf{\pnua{liè emõû-n mali-ò}}
\pfra{ces deux époux (référence à un passé moins lointain)}
\end{exemple}
\newline
\relationsémantique{Cf.}{\lien{}{-la-èò}}
\glosecourte{pluriel}
\end{entrée}

\begin{entrée}{hê-pii-n}{}{ⓔhê-pii-n}
\région{BO}
(\domainesémantique{Corps humain})
\classe{nom}
\begin{glose}
\pfra{testicules}
\end{glose}
\end{entrée}

\begin{entrée}{hê-pwe}{}{ⓔhê-pwe}
\région{GOs}
(\domainesémantique{Pêche})
\classe{nom}
\begin{glose}
\pfra{prise à la ligne}
\end{glose}
\newline
\begin{exemple}
\région{GO}
\textbf{\pnua{hê-pwee-nu la lapya}}
\pfra{j'ai pêché ces tilapia à la ligne}
\end{exemple}
\newline
\begin{exemple}
\région{PA}
\textbf{\pnua{hê-pwee-ny (a) liã no}}
\pfra{j'ai pêché ces poissons à la ligne}
\end{exemple}
\end{entrée}

\begin{entrée}{hê-pwiò}{}{ⓔhê-pwiò}
\région{GOs}
(\domainesémantique{Pêche})
\classe{nom}
\begin{glose}
\pfra{prise au filet}
\end{glose}
\newline
\begin{exemple}
\région{GO}
\textbf{\pnua{hê-pwiò-nu la nõ}}
\pfra{j'ai pêché ces poissons au filet}
\end{exemple}
\end{entrée}

\begin{entrée}{hê-thini}{}{ⓔhê-thini}
\région{GOs BO}
(\domainesémantique{Organisation sociale})
\classe{nom}
\begin{glose}
\pfra{contenu de la barrière}
\end{glose}
\begin{glose}
\pfra{offrandes de nourriture dans les coutumes}
\end{glose}
\end{entrée}

\begin{entrée}{hevwe}{}{ⓔhevwe}
\formephonétique{heβe}
\région{GOs}
\variante{%
yaave
\région{PA BO}}
(\domainesémantique{Ignames})
\classe{nom}
\begin{glose}
\pfra{igname longue (Charles + Dubois)}
\end{glose}
\end{entrée}

\begin{entrée}{hi}{1}{ⓔhiⓗ1}
\région{PA}
(\domainesémantique{Travail bois})
\classe{v}
\begin{glose}
\pfra{écorcer}
\end{glose}
\newline
\begin{exemple}
\région{PA}
\textbf{\pnua{i hi zòòni}}
\pfra{il écorce des niaouli}
\end{exemple}
\newline
\relationsémantique{Cf.}{\lien{}{hili v.t}}
\glosecourte{écorcer}
\end{entrée}

\begin{entrée}{hi}{2}{ⓔhiⓗ2}
\région{GOs BO}
\classe{nom}
(\domainesémantique{Vêtements, parure})
\begin{glose}
\pfra{morceau d'étoffe (plus fine que "mada")}
\end{glose}
\end{entrée}

\begin{entrée}{hi}{3}{ⓔhiⓗ3}
\région{PA}
(\domainesémantique{Prépositions})
\classe{n.BENEF}
\begin{glose}
\pfra{à ; pour}
\end{glose}
\newline
\begin{exemple}
\région{PA}
\textbf{\pnua{nu tèè-na hi-m nye wõjò-ny}}
\pfra{je te prête mon bateau}
\end{exemple}
\newline
\begin{exemple}
\région{PA}
\textbf{\pnua{na hi-n}}
\pfra{donne-le lui}
\end{exemple}
\end{entrée}

\begin{entrée}{hî}{}{ⓔhî}
\région{PA BO [BM]}
(\domainesémantique{Démonstratifs})
\classe{DEIC.1}
\begin{glose}
\pfra{ceci}
\end{glose}
\newline
\begin{exemple}
\région{BO}
\textbf{\pnua{hî tèèn}}
\pfra{ce jour-ci}
\end{exemple}
\end{entrée}

\begin{entrée}{hia}{}{ⓔhia}
\région{PA BO}
(\domainesémantique{Danses})
\classe{nom}
\begin{glose}
\pfra{crier en dansant [Corne]}
\end{glose}
\end{entrée}

\begin{entrée}{hibil}{}{ⓔhibil}
\région{PA BO}
\classe{v}
(\domainesémantique{Caractéristiques et propriétés des personnes})
\begin{glose}
\pfra{vif ; agile ; dynamique}
\end{glose}
\newline
\relationsémantique{Cf.}{\lien{}{zibi GOs}}
\end{entrée}

\begin{entrée}{hi-hõbwo}{}{ⓔhi-hõbwo}
\région{GOs}
(\domainesémantique{Vêtements, parure})
\classe{nom}
\begin{glose}
\pfra{manche}
\end{glose}
\newline
\begin{exemple}
\textbf{\pnua{e zugi hi-hõbwo}}
\pfra{elle retrousse ses manches}
\end{exemple}
\end{entrée}

\begin{entrée}{hi-hôgo}{}{ⓔhi-hôgo}
\région{GOs PA}
(\domainesémantique{Topographie})
\classe{nom}
\begin{glose}
\pfra{ramification de montagne}
\end{glose}
\end{entrée}

\begin{entrée}{hii}{}{ⓔhii}
\région{GOs BO PA}
\variante{%
yi-n
}
\classe{nom}
\newline
\sens{1}
(\domainesémantique{Corps humain})
\begin{glose}
\pfra{main ;}
\end{glose}
\begin{glose}
\pfra{bras ;}
\end{glose}
\begin{glose}
\pfra{aile ;}
\end{glose}
\begin{glose}
\pfra{tentacule}
\end{glose}
\newline
\begin{exemple}
\région{GO}
\textbf{\pnua{hii-je}}
\pfra{sa main}
\end{exemple}
\newline
\begin{exemple}
\région{PA}
\textbf{\pnua{hi-n}}
\pfra{sa main}
\end{exemple}
\newline
\begin{sous-entrée}{gu hi-n}{ⓔhiiⓢ1ⓝgu hi-n}
\begin{glose}
\pfra{main droite ; droite}
\end{glose}
\end{sous-entrée}
\newline
\begin{sous-entrée}{hi-n mani kòò-n}{ⓔhiiⓢ1ⓝhi-n mani kòò-n}
\région{BO}
\begin{glose}
\pfra{les bras et les jambes; les membres}
\end{glose}
\end{sous-entrée}
\newline
\begin{sous-entrée}{hi-n tòòmwa}{ⓔhiiⓢ1ⓝhi-n tòòmwa}
\région{BO}
\begin{glose}
\pfra{pouce}
\end{glose}
\end{sous-entrée}
\newline
\begin{sous-entrée}{hi-n èmwèn}{ⓔhiiⓢ1ⓝhi-n èmwèn}
\région{BO}
\begin{glose}
\pfra{index}
\end{glose}
\end{sous-entrée}
\newline
\sens{2}
(\domainesémantique{Noms des plantes})
\begin{glose}
\pfra{algues de rivière}
\end{glose}
\newline
\sens{3}
(\domainesémantique{Parties de plantes})
\begin{glose}
\pfra{branche}
\end{glose}
\newline
\begin{sous-entrée}{hi-ce}{ⓔhiiⓢ3ⓝhi-ce}
\région{BO}
\begin{glose}
\pfra{branche}
\end{glose}
\end{sous-entrée}
\newline
\étymologie{
\langue{POc}
\étymon{*kuku}}
\end{entrée}

\begin{entrée}{hii-je bwa mhwã}{}{ⓔhii-je bwa mhwã}
\région{GOs}
(\domainesémantique{Corps humain})
\classe{nom}
\begin{glose}
\pfra{main droite (sa)}
\end{glose}
\end{entrée}

\begin{entrée}{hii-je mhõ}{}{ⓔhii-je mhõ}
\région{GOs}
(\domainesémantique{Corps humain})
\classe{nom}
\begin{glose}
\pfra{main gauche (sa)}
\end{glose}
\newline
\begin{sous-entrée}{mho-n}{ⓔhii-je mhõⓝmho-n}
\begin{glose}
\pfra{main gauche ; gauche}
\end{glose}
\end{sous-entrée}
\end{entrée}

\begin{entrée}{hiili}{}{ⓔhiili}
\région{PA BO}
\variante{%
iili
}
(\domainesémantique{Mouvements ou actions faits avec le corps, les bras, les mains, les pieds})
\classe{v}
\begin{glose}
\pfra{cacher (se)}
\end{glose}
\begin{glose}
\pfra{rétracter (se) (dans une coquille: escargot, coquillage)}
\end{glose}
\begin{glose}
\pfra{effacer}
\end{glose}
\newline
\relationsémantique{Cf.}{\lien{}{hiilini}}
\glosecourte{effacer}
\end{entrée}

\begin{entrée}{hijini}{}{ⓔhijini}
\formephonétique{hiɲɟini}
\région{GOs BO}
\variante{%
hiliini
\région{BO}}
(\domainesémantique{Mouvements ou actions faits avec le corps, les bras, les mains, les pieds})
\classe{v}
\begin{glose}
\pfra{effacer}
\end{glose}
\end{entrée}

\begin{entrée}{hili}{1}{ⓔhiliⓗ1}
\région{GOs BO}
\classe{v.t.}
\newline
\sens{1}
(\domainesémantique{Travail bois})
\begin{glose}
\pfra{écorcer (niaouli)}
\end{glose}
\newline
\begin{exemple}
\région{PA}
\textbf{\pnua{i hi zòòni}}
\pfra{ilécorce des niaouli}
\end{exemple}
\newline
\begin{exemple}
\textbf{\pnua{i hili cii yòòni}}
\pfra{il enlève l'écorce du niaouli}
\end{exemple}
\newline
\sens{2}
(\domainesémantique{Préparation des aliments; modes de préparation et de cuisson})
\begin{glose}
\pfra{écorcher (animal)}
\end{glose}
\begin{glose}
\pfra{dépecer}
\end{glose}
\end{entrée}

\begin{entrée}{hili}{2}{ⓔhiliⓗ2}
\région{GOs BO}
(\domainesémantique{Verbes d'action (en général)})
\classe{v}
\begin{glose}
\pfra{bouger ; secouer (branches)}
\end{glose}
\end{entrée}

\begin{entrée}{hili}{3}{ⓔhiliⓗ3}
\région{GOs BO}
(\domainesémantique{Fonctions naturelles humaines})
\classe{v}
\begin{glose}
\pfra{plaindre (se) ; gémir}
\end{glose}
\end{entrée}

\begin{entrée}{hili}{4}{ⓔhiliⓗ4}
\région{GOs PA}
\variante{%
ili
}
(\domainesémantique{Démonstratifs})
\classe{DEM}
\begin{glose}
\pfra{cela (audible, mais invisible)}
\end{glose}
\newline
\begin{exemple}
\région{PA}
\textbf{\pnua{nu tone je vhaa hili}}
\pfra{j'ai entendu ces propos-là}
\end{exemple}
\newline
\begin{exemple}
\région{PA}
\textbf{\pnua{da la-ò waal mala-(h)ili}}
\pfra{c'est quoi ces chansons ?}
\end{exemple}
\newline
\begin{exemple}
\région{PA}
\textbf{\pnua{vhaa hili}}
\pfra{ces paroles (qu'on entend)}
\end{exemple}
\newline
\begin{exemple}
\textbf{\pnua{la êgu malaa-ili}}
\pfra{ces gens (qu'on entend)}
\end{exemple}
\newline
\begin{exemple}
\textbf{\pnua{la baa-êgu malaa-ili}}
\pfra{ces femmes (qu'on entend)}
\end{exemple}
\newline
\begin{exemple}
\textbf{\pnua{ila lha-ili}}
\pfra{eux, ceux-là (qu'on entend)}
\end{exemple}
\newline
\begin{exemple}
\textbf{\pnua{ili lhi-ili}}
\pfra{eux 2, ces deux-là (qu'on entend)}
\end{exemple}
\end{entrée}

\begin{entrée}{hiliçôô}{}{ⓔhiliçôô}
\formephonétique{hiliʒõː}
\région{GOs}
\variante{%
yaoli
\région{WEM WE}}
(\domainesémantique{Mouvements ou actions faits avec le corps, les bras, les mains, les pieds
, Jeux divers})
\classe{nom}
\begin{glose}
\pfra{balançoire ; balancer (se)}
\end{glose}
\newline
\begin{exemple}
\textbf{\pnua{e hiliçôô}}
\pfra{il se balance}
\end{exemple}
\newline
\begin{exemple}
\textbf{\pnua{e pe-hiliçôô}}
\pfra{il se balance}
\end{exemple}
\end{entrée}

\begin{entrée}{hilò}{}{ⓔhilò}
\région{GO PA}
\variante{%
hiloo
}
(\domainesémantique{Arbre})
\classe{nom}
\begin{glose}
\pfra{morindier ; fromager (petit arbre, arbuste sert de médicament et écorce a des propriétés teinturales, jaune)}
\end{glose}
\nomscientifique{Morindia citrifolia}
\end{entrée}

\begin{entrée}{hîmi}{1}{ⓔhîmiⓗ1}
\région{GOs}
\classe{v}
\newline
\sens{1}
(\domainesémantique{Verbes d'action (en général)})
\begin{glose}
\pfra{serré ; coincé}
\end{glose}
\newline
\begin{exemple}
\textbf{\pnua{e hîmi hi-nu}}
\pfra{ma main est coincée}
\end{exemple}
\newline
\sens{2}
(\domainesémantique{Verbes d'action faite par des animaux})
\begin{glose}
\pfra{pincer (crabes, langoustes)}
\end{glose}
\end{entrée}

\begin{entrée}{hîmi}{2}{ⓔhîmiⓗ2}
\région{GOs}
\variante{%
hòmi
\région{PA}}
(\domainesémantique{Mouvements ou actions faits avec le corps, les bras, les mains, les pieds})
\classe{v}
\begin{glose}
\pfra{fermer (bouche, main)}
\end{glose}
\newline
\begin{exemple}
\textbf{\pnua{hîmi pwha-jo !}}
\pfra{ferme ta bouche !}
\end{exemple}
\newline
\relationsémantique{Cf.}{\lien{ⓔkivwi}{kivwi}}
\glosecourte{fermer (la porte)}
\end{entrée}

\begin{entrée}{hine}{}{ⓔhine}
\formephonétique{hĩɳe}
\région{GOs}
\variante{%
hine
\région{PA BO}}
(\domainesémantique{Fonctions intellectuelles})
\classe{v.t.}
\begin{glose}
\pfra{savoir ; connaître ; comprendre}
\end{glose}
\newline
\begin{exemple}
\textbf{\pnua{e hine}}
\pfra{il le sait}
\end{exemple}
\newline
\begin{exemple}
\région{GO}
\textbf{\pnua{kavu nu hine khõbwe e zoma a-mi haa/hai kòi-je}}
\pfra{je ne sais pas s'il viendra ou pas}
\end{exemple}
\newline
\begin{exemple}
\région{GO}
\textbf{\pnua{kavu nu hine traabwa bwa chovwa}}
\pfra{je ne suis jamais monté à cheval}
\end{exemple}
\newline
\begin{exemple}
\région{PA}
\textbf{\pnua{yo mara hine}}
\pfra{(maintenant) tu le sauras (après une punition)}
\end{exemple}
\newline
\begin{sous-entrée}{hine-wo}{ⓔhineⓝhine-wo}
\région{GO}
\begin{glose}
\pfra{intelligent, cultivé}
\end{glose}
\end{sous-entrée}
\newline
\begin{sous-entrée}{te-(h)ine}{ⓔhineⓝte-(h)ine}
\begin{glose}
\pfra{sûr, certain de (être)}
\end{glose}
\end{sous-entrée}
\newline
\begin{sous-entrée}{hine-kaamweni}{ⓔhineⓝhine-kaamweni}
\région{GO}
\begin{glose}
\pfra{comprendre}
\end{glose}
\end{sous-entrée}
\newline
\étymologie{
\langue{POc}
\étymon{*kilala}}
\end{entrée}

\begin{entrée}{hine-kaamweni}{}{ⓔhine-kaamweni}
\formephonétique{hiɳekaːmweɳi}
\région{GOs}
(\domainesémantique{Fonctions intellectuelles})
\classe{v}
\begin{glose}
\pfra{comprendre (bien) qqch.}
\end{glose}
\end{entrée}

\begin{entrée}{hinevwo}{}{ⓔhinevwo}
\formephonétique{'hiɳeβo}
\région{GOs PA}
(\domainesémantique{Fonctions intellectuelles})
\classe{v ; n}
\begin{glose}
\pfra{instruit ; intelligent ; connaissance ; intelligence}
\end{glose}
\newline
\begin{exemple}
\textbf{\pnua{e hinevwo}}
\pfra{il est intelligent}
\end{exemple}
\newline
\relationsémantique{Cf.}{\lien{ⓔhine}{hine}}
\glosecourte{savoir; connaître}
\end{entrée}

\begin{entrée}{hing}{}{ⓔhing}
\région{PA BO}
\classe{v}
\newline
\sens{1}
(\domainesémantique{Sentiments})
\begin{glose}
\pfra{haïr ; détester}
\end{glose}
\newline
\sens{2}
(\domainesémantique{Aliments, alimentation})
\begin{glose}
\pfra{dégoûté ; faire le difficile}
\end{glose}
\newline
\begin{exemple}
\région{PA}
\textbf{\pnua{i hinge-nu}}
\pfra{il me déteste}
\end{exemple}
\newline
\begin{exemple}
\région{PA}
\textbf{\pnua{i a-hing}}
\pfra{il fait le difficile}
\end{exemple}
\newline
\note{v.t. hige}{grammaire}{}
\end{entrée}

\begin{entrée}{hîńõ}{}{ⓔhîńõ}
\région{GOs PA}
\variante{%
hinõn
\région{BO}}
\newline
\groupe{A}
(\domainesémantique{Mouvements ou actions faits avec le corps, les bras, les mains, les pieds})
\classe{nom}
\begin{glose}
\pfra{signe}
\end{glose}
\newline
\begin{exemple}
\région{BO}
\textbf{\pnua{i thu pa-hinõn ja inu}}
\pfra{il me fait signe}
\end{exemple}
\newline
\begin{sous-entrée}{hinõ-al}{ⓔhîńõⓝhinõ-al}
\région{PA}
\begin{glose}
\pfra{heure (lit. signe du soleil)}
\end{glose}
\end{sous-entrée}
\newline
\begin{sous-entrée}{hińõ-a}{ⓔhîńõⓝhińõ-a}
\région{GO}
\begin{glose}
\pfra{heure (lit. signe du soleil)}
\end{glose}
\end{sous-entrée}
\newline
\begin{sous-entrée}{hinõõ-tèèn}{ⓔhîńõⓝhinõõ-tèèn}
\région{BO}
\begin{glose}
\pfra{étoile du matin (lit. signe du jour ; tèèn : jour)}
\end{glose}
\end{sous-entrée}
\newline
\groupe{B}
(\domainesémantique{Mouvements ou actions faits avec le corps, les bras, les mains, les pieds})
\classe{v}
\begin{glose}
\pfra{montrer}
\end{glose}
\newline
\étymologie{
\langue{PSO}
\étymon{*khina}
\glosecourte{savoir}
\auteur{Geraghty}}
\end{entrée}

\begin{entrée}{hińõ-a}{}{ⓔhińõ-a}
\formephonétique{'hinɔ̃-a}
\région{GOs}
\variante{%
hinõ-al
\région{PA BO}}
(\domainesémantique{Découpage du temps})
\classe{nom}
\begin{glose}
\pfra{heure}
\end{glose}
\begin{glose}
\pfra{montre ; pendule [PA]}
\end{glose}
\newline
\begin{exemple}
\région{GO}
\textbf{\pnua{poniza hińõ-a ?}}
\pfra{quelle heure est-il ?}
\end{exemple}
\newline
\begin{exemple}
\région{PA}
\textbf{\pnua{ponira hinõ-al ?}}
\pfra{quelle heure est-il ?}
\end{exemple}
\newline
\begin{exemple}
\région{BO}
\textbf{\pnua{na whaya hinõ-al ?}}
\pfra{à quelle heure ?}
\end{exemple}
\end{entrée}

\begin{entrée}{hińõ-tree}{1}{ⓔhińõ-treeⓗ1}
\région{GOs}
\variante{%
hinõ-tèèn
\région{PA BO}}
(\domainesémantique{Découpage du temps})
\classe{nom}
\begin{glose}
\pfra{aurore ; aube}
\end{glose}
\newline
\begin{exemple}
\région{GO}
\textbf{\pnua{bwa hińõ-tree}}
\pfra{à l'aurore}
\end{exemple}
\end{entrée}

\begin{entrée}{hińõ-tree}{2}{ⓔhińõ-treeⓗ2}
\formephonétique{hinɔ̃-ʈeː}
\région{GOs PA}
(\domainesémantique{Astres})
\classe{nom}
\begin{glose}
\pfra{étoile du matin (lit. signe du matin)}
\end{glose}
\end{entrée}

\begin{entrée}{hi-pwaji}{}{ⓔhi-pwaji}
\formephonétique{'hi-'pwaɲɟi}
\région{GOs}
(\domainesémantique{Crustacés, crabes})
\classe{nom}
\begin{glose}
\pfra{pinces du crabe}
\end{glose}
\end{entrée}

\begin{entrée}{hituvaè}{}{ⓔhituvaè}
\région{BO}
\variante{%
hiluvai
\région{GO}}
(\domainesémantique{Crustacés, crabes})
\classe{nom}
\begin{glose}
\pfra{langouste [Corne]}
\end{glose}
\newline
\note{non vérifié}{général}{}
\end{entrée}

\begin{entrée}{hivwa}{}{ⓔhivwa}
\région{GOs}
(\domainesémantique{Mollusques})
\classe{nom}
\begin{glose}
\pfra{moule ; coque (sert de grattoir à banane)}
\end{glose}
\newline
\relationsémantique{Cf.}{\lien{ⓔhizu}{hizu}}
\glosecourte{(plus petite)}
\end{entrée}

\begin{entrée}{hivwaje}{}{ⓔhivwaje}
\formephonétique{hiβaɲɟe}
\région{GOs}
(\domainesémantique{Fonctions naturelles humaines})
\classe{v ; n}
\begin{glose}
\pfra{sourire}
\end{glose}
\end{entrée}

\begin{entrée}{hivwi}{1}{ⓔhivwiⓗ1}
\formephonétique{hiβi}
\région{GOs}
(\domainesémantique{Arbre})
\classe{nom}
\begin{glose}
\pfra{palétuvier gris (court)}
\end{glose}
\nomscientifique{Avicennia marina (Verbénacées)}
\end{entrée}

\begin{entrée}{hivwi}{2}{ⓔhivwiⓗ2}
\formephonétique{hiβi}
\région{GOs PA}
\variante{%
hivi
\région{BO}, 
hipi
\région{BO vx}}
(\domainesémantique{Mouvements ou actions faits avec le corps, les bras, les mains, les pieds})
\classe{v}
\begin{glose}
\pfra{ramasser (nourriture, vêtement)}
\end{glose}
\begin{glose}
\pfra{entasser}
\end{glose}
\end{entrée}

\begin{entrée}{hivwine}{}{ⓔhivwine}
\formephonétique{hiβiɳe}
\région{GOs}
\variante{%
hipine
\région{GO(s)}, 
hivine
\région{BO}}
(\domainesémantique{Fonctions intellectuelles})
\classe{v.t.}
\begin{glose}
\pfra{ignorer ; ne pas savoir ; ignorer (s')}
\end{glose}
\newline
\begin{exemple}
\textbf{\pnua{i hivwine kõbwe da la jö caabi}}
\pfra{il ne sait pas quels objets tu as frappés}
\end{exemple}
\newline
\begin{exemple}
\textbf{\pnua{li pe-hivwine-li ã-mi na ni nye ba-egu}}
\pfra{ils s'ignorent à cause de cette femme (venir de cette femme)}
\end{exemple}
\newline
\begin{sous-entrée}{hivwinevwo}{ⓔhivwineⓝhivwinevwo}
\begin{glose}
\pfra{ignorant}
\end{glose}
\newline
\note{hivwine (v.t.)}{grammaire}{savoir qqch}
\end{sous-entrée}
\newline
\relationsémantique{Ant.}{\lien{ⓔhine}{hine}}
\glosecourte{savoir}
\end{entrée}

\begin{entrée}{hivwinevwo}{}{ⓔhivwinevwo}
\formephonétique{hiβiɳeβo}
\région{GOs}
(\domainesémantique{Fonctions intellectuelles})
\classe{v.i.}
\begin{glose}
\pfra{ignorant (être)}
\end{glose}
\newline
\begin{exemple}
\textbf{\pnua{e hivwinevwo}}
\pfra{il est ignorant}
\end{exemple}
\end{entrée}

\begin{entrée}{hivwivwu}{}{ⓔhivwivwu}
\formephonétique{hiβiβu}
\région{GO}
(\domainesémantique{Oiseaux})
\classe{nom}
\begin{glose}
\pfra{oiseau de terre (petit, marron, court et mange les graines semées dans les champs)}
\end{glose}
\begin{glose}
\pfra{ralle}
\end{glose}
\end{entrée}

\begin{entrée}{hi-xe}{}{ⓔhi-xe}
\région{GOs PA}
(\domainesémantique{Préfixes classificateurs numériques})
\classe{CLF.NUM}
\begin{glose}
\pfra{une branche (ne s'emploie plus)}
\end{glose}
\newline
\begin{exemple}
\textbf{\pnua{hi-ru hi-ce}}
\pfra{deux branches d'arbre}
\end{exemple}
\end{entrée}

\begin{entrée}{hizu}{}{ⓔhizu}
\région{GOs}
\classe{nom}
\newline
\sens{1}
(\domainesémantique{Mollusques})
\begin{glose}
\pfra{moule}
\end{glose}
\begin{glose}
\pfra{coquille de moule}
\end{glose}
\newline
\sens{2}
(\domainesémantique{Ustensiles})
\begin{glose}
\pfra{coquille de moule (sert de grattoir à banane)}
\end{glose}
\newline
\relationsémantique{Cf.}{\lien{ⓔhivwa}{hivwa}}
\glosecourte{moule (plus grosse)}
\end{entrée}

\begin{entrée}{ho}{}{ⓔho}
\région{PA}
(\domainesémantique{Relations et interaction sociales})
\classe{v}
\begin{glose}
\pfra{protéger}
\end{glose}
\newline
\begin{sous-entrée}{a-ho}{ⓔhoⓝa-ho}
\begin{glose}
\pfra{protecteur}
\end{glose}
\end{sous-entrée}
\newline
\begin{sous-entrée}{hova mwa}{ⓔhoⓝhova mwa}
\begin{glose}
\pfra{les protecteurs de la chefferie}
\end{glose}
\newline
\note{hova: forme déterminée}{grammaire}{}
\end{sous-entrée}
\end{entrée}

\begin{entrée}{ho-}{}{ⓔho-}
\région{GOs BO}
\variante{%
hò
\région{PA}, 
hu-
\région{BO}}
(\domainesémantique{Préfixes classificateurs possessifs de la nourriture})
\classe{CLF.POSS}
\classe{nom}
\begin{glose}
\pfra{nourriture carnée part (de poisson et viande),}
\end{glose}
\begin{glose}
\pfra{ration ; part de sucreries; médicaments(PA)}
\end{glose}
\newline
\begin{exemple}
\textbf{\pnua{ho-ã}}
\pfra{notre nourriture carnée}
\end{exemple}
\newline
\begin{exemple}
\textbf{\pnua{nu pii vwo ho-tru}}
\pfra{j'ai coupé en 2 morceaux}
\end{exemple}
\newline
\begin{exemple}
\textbf{\pnua{jaxa ho-m ène}}
\pfra{tu as assez de nourriture (lit. suffisant ta nourriture)}
\end{exemple}
\newline
\begin{exemple}
\région{PA}
\textbf{\pnua{bo ho-îî da ?}}
\pfra{quelle est cette odeur de viande ?}
\end{exemple}
\newline
\begin{exemple}
\textbf{\pnua{hò-n na nò}}
\pfra{sa ration de poisson}
\end{exemple}
\newline
\begin{exemple}
\textbf{\pnua{hò-m na cèèvèro}}
\pfra{ta ration de viande}
\end{exemple}
\end{entrée}

\begin{entrée}{hò}{1}{ⓔhòⓗ1}
\région{GOs PA BO}
(\domainesémantique{Insectes})
\classe{nom}
\begin{glose}
\pfra{cigale (grosse et verte)}
\end{glose}
\end{entrée}

\begin{entrée}{hò}{2}{ⓔhòⓗ2}
\région{GOs}
(\domainesémantique{Mouvements ou actions faits avec le corps, les bras, les mains, les pieds})
\classe{v}
\begin{glose}
\pfra{écarter ; sortir (de son habitacle)}
\end{glose}
\newline
\begin{exemple}
\région{GO}
\textbf{\pnua{e hò pi-mee}}
\pfra{il a les yeux exhorbités, écarquillés}
\end{exemple}
\newline
\begin{exemple}
\textbf{\pnua{e hò kumee-mee}}
\pfra{il a la langue pendante}
\end{exemple}
\end{entrée}

\begin{entrée}{hõ}{1}{ⓔhõⓗ1}
\région{GOs}
\variante{%
hom
\région{PABO}}
(\domainesémantique{Caractéristiques et propriétés des personnes})
\classe{v.stat.}
\begin{glose}
\pfra{muet}
\end{glose}
\newline
\relationsémantique{Cf.}{\lien{ⓔhûⓗ1}{hû}}
\glosecourte{silencieux}
\newline
\étymologie{
\langue{POc}
\étymon{*kumu}
\glosecourte{muet}, 
\langue{PSO}
\étymon{*xxumu}
\auteur{Geraghty}}
\end{entrée}

\begin{entrée}{hõ}{2}{ⓔhõⓗ2}
\région{GOs}
\variante{%
hô
\région{PA}, 
hò
\région{BO}}
\newline
\groupe{A}
(\domainesémantique{Description des objets, formes, consistance, taille})
\classe{v}
\begin{glose}
\pfra{nouveau ; neuf}
\end{glose}
\begin{glose}
\pfra{récent}
\end{glose}
\newline
\begin{sous-entrée}{hõ lòtò}{ⓔhõⓗ2ⓝhõ lòtò}
\région{GO}
\begin{glose}
\pfra{nouvelle voiture}
\end{glose}
\end{sous-entrée}
\newline
\begin{sous-entrée}{hõ mwa}{ⓔhõⓗ2ⓝhõ mwa}
\région{GO}
\begin{glose}
\pfra{nouvelle maison}
\end{glose}
\newline
\begin{exemple}
\région{BO}
\textbf{\pnua{i ho na le nûû po-xe}}
\pfra{il met une autre torche}
\end{exemple}
\end{sous-entrée}
\newline
\groupe{B}
\classe{ASP}
\newline
\sens{1}
(\domainesémantique{Aspect})
\begin{glose}
\pfra{encore ; à nouveau [BO]}
\end{glose}
\newline
\begin{exemple}
\région{BO}
\textbf{\pnua{hò kobwe õ-xe}}
\pfra{répète encore une fois}
\end{exemple}
\newline
\begin{exemple}
\région{BO}
\textbf{\pnua{hò kole òn}}
\pfra{mets encore du sel}
\end{exemple}
\newline
\begin{exemple}
\région{BO}
\textbf{\pnua{la ho kõbwe "u nòòl !"}}
\pfra{il redit " réveillez-vous !"}
\end{exemple}
\newline
\sens{2}
(\domainesémantique{Aspect})
\begin{glose}
\pfra{venir de}
\end{glose}
\newline
\begin{exemple}
\région{GO}
\textbf{\pnua{e hõ a nye !}}
\pfra{il vient juste de partir !}
\end{exemple}
\newline
\begin{exemple}
\région{GO}
\textbf{\pnua{e hõ pwe}}
\pfra{il vientde naître}
\end{exemple}
\newline
\begin{exemple}
\région{GO}
\textbf{\pnua{la hõ ne}}
\pfra{ils l'ont fait récemment}
\end{exemple}
\newline
\begin{exemple}
\région{PA}
\textbf{\pnua{nu ra gaa hô hovwo}}
\pfra{je viens de manger, j'ai déjà mangé}
\end{exemple}
\newline
\relationsémantique{Cf.}{\lien{}{hõ-xe}}
\end{entrée}

\begin{entrée}{hõ-ã}{}{ⓔhõ-ã}
\région{BO}
(\domainesémantique{Directionnels})
\classe{ADV}
\begin{glose}
\pfra{par ici ; de ce côté [BM, Corne]}
\end{glose}
\newline
\begin{exemple}
\textbf{\pnua{hõ-ã je i pe-kiga waa-na}}
\pfra{quant à lui il se rit ainsi}
\end{exemple}
\newline
\relationsémantique{Cf.}{\lien{ⓔhõõ-li}{hõõ-li}}
\glosecourte{de ce côté-là, de l'autre côté}
\end{entrée}

\begin{entrée}{hôbòl}{}{ⓔhôbòl}
\région{PA}
(\domainesémantique{Relations et interaction sociales})
\classe{v}
\begin{glose}
\pfra{jurer (que c'est vrai)}
\end{glose}
\end{entrée}

\begin{entrée}{hõbwò}{}{ⓔhõbwò}
\formephonétique{hɔ̃bwɔ}
\région{GOs}
\variante{%
hõbò
\région{GO(s)}, 
hãbwòn
\région{PA BO}, 
hõbwòn
\formephonétique{hɔ̃bwɔ̃n}
\région{BO}}
(\domainesémantique{Vêtements, parure})
\classe{nom}
\begin{glose}
\pfra{vêtements}
\end{glose}
\begin{glose}
\pfra{linge ; tissu}
\end{glose}
\newline
\begin{exemple}
\région{GO}
\textbf{\pnua{hõbwòli-nu, hõbòli-nu}}
\pfra{mes vêtements}
\end{exemple}
\newline
\begin{exemple}
\région{GO}
\textbf{\pnua{hõbwò ni ba-mogu}}
\pfra{des vêtements pour le travail}
\end{exemple}
\newline
\begin{sous-entrée}{hõbwò bwabu}{ⓔhõbwòⓝhõbwò bwabu}
\région{GO}
\begin{glose}
\pfra{sous-vêtements}
\end{glose}
\newline
\begin{exemple}
\région{GO}
\textbf{\pnua{hòbwò ba-thu-mwêê}}
\pfra{de beaux vêtements}
\end{exemple}
\newline
\begin{exemple}
\région{GO}
\textbf{\pnua{hõbwòra ba-êgu; hõbwòwa ba-êgu}}
\pfra{des vêtements de femme}
\end{exemple}
\newline
\begin{exemple}
\région{GO}
\textbf{\pnua{hõbwò za ? -hõbwòra pwa li-nu - hõbwòra choomu}}
\pfra{quel type devêtement ? - des vêtements pour la pluie - des vêtements pour l'école}
\end{exemple}
\newline
\begin{exemple}
\région{GO}
\textbf{\pnua{hõbwò xa whaya ? -hõbwò xa khawali- hõbwò ba-thu-mwêê}}
\pfra{desvêtements comment ? - des vêtements longs - de beaux vêtements}
\end{exemple}
\newline
\begin{exemple}
\région{PA}
\textbf{\pnua{hãbwòli-n}}
\pfra{ses vêtements}
\end{exemple}
\newline
\begin{exemple}
\région{BO PA}
\textbf{\pnua{hòbwòni-n}}
\pfra{ses vêtements}
\end{exemple}
\newline
\begin{exemple}
\région{BO PA}
\textbf{\pnua{i udale hòbwòni-n u Kaavo}}
\pfra{K met ses vêtements}
\end{exemple}
\newline
\begin{exemple}
\région{PA}
\textbf{\pnua{hãbwòra khabu}}
\pfra{vêtements pour le froid}
\end{exemple}
\newline
\begin{exemple}
\région{WEM}
\textbf{\pnua{kixa hõbwòli-la}}
\pfra{ils n'ont pas de vêtement}
\end{exemple}
\end{sous-entrée}
\end{entrée}

\begin{entrée}{hôbwo}{}{ⓔhôbwo}
\formephonétique{hõbwo}
\région{GOs}
\variante{%
hôbo
\région{GO(s) PA}}
(\domainesémantique{Relations et interaction sociales})
\classe{v}
\begin{glose}
\pfra{garder ; surveiller}
\end{glose}
\begin{glose}
\pfra{attendre}
\end{glose}
\newline
\begin{sous-entrée}{a-hôbwo}{ⓔhôbwoⓝa-hôbwo}
\begin{glose}
\pfra{gardien, surveillant}
\end{glose}
\end{sous-entrée}
\end{entrée}

\begin{entrée}{hõbwo-ko}{}{ⓔhõbwo-ko}
\région{GOs}
(\domainesémantique{Vêtements, parure})
\classe{nom}
\begin{glose}
\pfra{robe popinée}
\end{glose}
\end{entrée}

\begin{entrée}{hõbwo-tralago}{}{ⓔhõbwo-tralago}
\région{GOs}
(\domainesémantique{Vêtements, parure})
\classe{nom}
\begin{glose}
\pfra{robe}
\end{glose}
\end{entrée}

\begin{entrée}{hôdò}{}{ⓔhôdò}
\formephonétique{hõdɔ}
\région{GOs PA}
\variante{%
hôde
\région{BO}}
(\domainesémantique{Santé, maladie})
\classe{v}
\begin{glose}
\pfra{jeûner ; jeûne}
\end{glose}
\newline
\begin{exemple}
\région{GO}
\textbf{\pnua{we-kò tree xa nu hôdò}}
\pfra{cela fait 3 jours que je jeûne}
\end{exemple}
\newline
\begin{exemple}
\région{PA}
\textbf{\pnua{we-kòn na tèèn ka nu hôdò}}
\pfra{cela fait 3 jours que je jeûne}
\end{exemple}
\newline
\begin{exemple}
\région{BO}
\textbf{\pnua{bo hôde}}
\pfra{vendredi}
\end{exemple}
\end{entrée}

\begin{entrée}{hõge}{}{ⓔhõge}
(\domainesémantique{Armes})
\classe{nom}
\begin{glose}
\pfra{doigtier de la sagaie (propulseur) ; lance sagaie}
\end{glose}
\end{entrée}

\begin{entrée}{hõgo}{}{ⓔhõgo}
\formephonétique{hɔ̃ŋgo}
\région{GOs}
(\domainesémantique{Noms des plantes})
\classe{nom}
\begin{glose}
\pfra{roseau}
\end{glose}
\end{entrée}

\begin{entrée}{hõgõõne mhwããnu}{}{ⓔhõgõõne mhwããnu}
\région{PA BO}
\variante{%
hõgõõn
\région{PA BO}}
(\domainesémantique{Astres})
\classe{nom}
\begin{glose}
\pfra{premier ou deuxième quartier de lune}
\end{glose}
\newline
\begin{exemple}
\région{PA}
\textbf{\pnua{hõgõõne mhwããnu}}
\pfra{1er ou 2ème quartier de lune}
\end{exemple}
\end{entrée}

\begin{entrée}{homi}{}{ⓔhomi}
\région{PA BO}
(\domainesémantique{Mouvements ou actions faits avec le corps, les bras, les mains, les pieds})
\classe{v}
\begin{glose}
\pfra{fermer ; pincer}
\end{glose}
\newline
\begin{exemple}
\textbf{\pnua{homi phwaa-m !}}
\pfra{tais-toi (n'ouvre plus la bouche)}
\end{exemple}
\newline
\relationsémantique{Cf.}{\lien{}{kivwi phwaa-m !}}
\glosecourte{tais-toi (momentanément)}
\end{entrée}

\begin{entrée}{homwi}{}{ⓔhomwi}
\région{BO [Corne]}
(\domainesémantique{Feu : objets et actions liés au feu})
\classe{nom}
\begin{glose}
\pfra{pincettes}
\end{glose}
\newline
\étymologie{
\langue{POc}
\étymon{*(ŋ)kompi}
\glosecourte{serrer, pincer}}
\end{entrée}

\begin{entrée}{hõn}{}{ⓔhõn}
\formephonétique{hɔ̃n}
\région{PA}
(\domainesémantique{Eau})
\classe{nom}
\begin{glose}
\pfra{source}
\end{glose}
\end{entrée}

\begin{entrée}{hòne}{}{ⓔhòne}
\région{BO}
(\domainesémantique{Verbes d'action (en général)})
\classe{v}
\begin{glose}
\pfra{réparer}
\end{glose}
\newline
\relationsémantique{Cf.}{\lien{ⓔhòⓗ1}{hò}}
\glosecourte{nouveau, encore}
\end{entrée}

\begin{entrée}{hô-niza ?}{}{ⓔhô-niza ?}
\région{GOs}
(\domainesémantique{Interrogatifs})
\classe{INT}
\begin{glose}
\pfra{combien de morceaux (pastèque, igname)}
\end{glose}
\end{entrée}

\begin{entrée}{hôno}{}{ⓔhôno}
\région{GOs PA}
\classe{v.stat.}
(\domainesémantique{Santé, maladie})
\begin{glose}
\pfra{malade (gravement)}
\end{glose}
\begin{glose}
\pfra{allité}
\end{glose}
\newline
\begin{exemple}
\région{PA}
\textbf{\pnua{e gaa hôno}}
\pfra{il est très gravement malade, allité}
\end{exemple}
\end{entrée}

\begin{entrée}{hoo}{1}{ⓔhooⓗ1}
\région{PA}
(\domainesémantique{Interjection})
\classe{v}
\begin{glose}
\pfra{signaler sa présence par un appel}
\end{glose}
\newline
\begin{exemple}
\textbf{\pnua{i hoo}}
\pfra{il signale sa présence par un appel}
\end{exemple}
\end{entrée}

\begin{entrée}{hoo}{2}{ⓔhooⓗ2}
\région{GOs PA}
(\domainesémantique{Outils})
\classe{nom}
\begin{glose}
\pfra{coin (pour caler ; lit. nourriture)}
\end{glose}
\newline
\begin{exemple}
\région{GO}
\textbf{\pnua{e thu hoo kòò-piòò}}
\pfra{il cale le manche de la pioche}
\end{exemple}
\newline
\begin{exemple}
\région{PA}
\textbf{\pnua{ne vwo hoo-n !}}
\pfra{cale-le ! (lit. fais sa nourriture)}
\end{exemple}
\end{entrée}

\begin{entrée}{hoo}{3}{ⓔhooⓗ3}
\région{GOs}
\variante{%
tau
\région{BO PA}}
(\domainesémantique{Pêche})
\classe{v}
\begin{glose}
\pfra{pêcher à marée basse ou àmarée montante}
\end{glose}
\end{entrée}

\begin{entrée}{hòò}{}{ⓔhòò}
\région{GOs}
\variante{%
hòòl
\région{PA BO}}
(\domainesémantique{Localisation})
\classe{ADV ; LOC}
\begin{glose}
\pfra{loin ; éloigné ; lointain}
\end{glose}
\begin{glose}
\pfra{longtemps (d'il y a)}
\end{glose}
\newline
\begin{exemple}
\région{GO}
\textbf{\pnua{gee hòò}}
\pfra{grand-mère d'il y a longtemps}
\end{exemple}
\newline
\begin{exemple}
\région{GO}
\textbf{\pnua{jige bu-hòò / buxò}}
\pfra{un fusil à longue portée}
\end{exemple}
\newline
\begin{exemple}
\textbf{\pnua{kavwö hòò vwo nu baani !}}
\pfra{il s'en est fallu de peu que je ne le frappe !}
\end{exemple}
\newline
\begin{exemple}
\région{PA}
\textbf{\pnua{ge hòòl}}
\pfra{il est loin}
\end{exemple}
\newline
\begin{sous-entrée}{a-ò}{ⓔhòòⓝa-ò}
\begin{glose}
\pfra{éloigner (s')}
\end{glose}
\end{sous-entrée}
\newline
\étymologie{
\langue{POc}
\étymon{*saud, *sauq}
\auteur{Grace}}
\end{entrée}

\begin{entrée}{hõõ}{}{ⓔhõõ}
\région{PA}
(\domainesémantique{Quantificateurs})
\classe{v}
\begin{glose}
\pfra{morceau long et plat (réfère à une surface allongée)}
\end{glose}
\newline
\begin{sous-entrée}{hõõ-ce}{ⓔhõõⓝhõõ-ce}
\begin{glose}
\pfra{planche}
\end{glose}
\newline
\relationsémantique{Cf.}{\lien{}{hõõge}}
\glosecourte{allonger}
\end{sous-entrée}
\end{entrée}

\begin{entrée}{hòòa}{}{ⓔhòòa}
\région{GOs}
(\domainesémantique{Découpage du temps})
\classe{nom}
\begin{glose}
\pfra{matin}
\end{glose}
\end{entrée}

\begin{entrée}{hõõ-ce}{}{ⓔhõõ-ce}
\région{PA}
(\domainesémantique{Bois})
\classe{nom}
\begin{glose}
\pfra{planche}
\end{glose}
\end{entrée}

\begin{entrée}{hoogo}{}{ⓔhoogo}
\formephonétique{hoːŋgo}
\région{GOs WEM PA BO}
\variante{%
hoogo
\région{BO}}
(\domainesémantique{Topographie})
\classe{nom}
\begin{glose}
\pfra{montagne}
\end{glose}
\newline
\begin{sous-entrée}{bwe hogo}{ⓔhoogoⓝbwe hogo}
\begin{glose}
\pfra{pic, sommet de la montagne}
\end{glose}
\end{sous-entrée}
\newline
\begin{sous-entrée}{do hogo ; du hogo}{ⓔhoogoⓝdo hogo ; du hogo}
\begin{glose}
\pfra{crête de la montagne}
\end{glose}
\end{sous-entrée}
\newline
\begin{sous-entrée}{ci hogo}{ⓔhoogoⓝci hogo}
\begin{glose}
\pfra{versant}
\end{glose}
\end{sous-entrée}
\newline
\begin{sous-entrée}{hi-hogo/hi-hogo}{ⓔhoogoⓝhi-hogo/hi-hogo}
\begin{glose}
\pfra{ramification de la montagne}
\end{glose}
\end{sous-entrée}
\end{entrée}

\begin{entrée}{hòòl}{3}{ⓔhòòlⓗ3}
\région{GOs PA}
(\domainesémantique{Tradition orale})
\classe{v ; n}
\begin{glose}
\pfra{haranguer (dans les grandes cérémonies)}
\end{glose}
\begin{glose}
\pfra{discours sur le bois (discours rythmé sur le bambou)}
\end{glose}
\end{entrée}

\begin{entrée}{hooli}{}{ⓔhooli}
\région{PA BO}
\classe{v}
\newline
\sens{1}
(\domainesémantique{Mouvements ou actions faits avec le corps, les bras, les mains, les pieds})
\begin{glose}
\pfra{mélanger ; tourner (dans la marmite)}
\end{glose}
\newline
\sens{2}
(\domainesémantique{Verbes d'action (en général)})
\begin{glose}
\pfra{éparpiller ; semer la pagaille}
\end{glose}
\end{entrée}

\begin{entrée}{hõõ-li}{}{ⓔhõõ-li}
\formephonétique{hɔ̃ːli}
\région{BO [BM, Corne]}
(\domainesémantique{Localisation})
\classe{LOC}
\begin{glose}
\pfra{de l'autre côté}
\end{glose}
\newline
\relationsémantique{Cf.}{\lien{ⓔhõ-ã}{hõ-ã}}
\glosecourte{de ce côté-ci}
\end{entrée}

\begin{entrée}{hõõng}{}{ⓔhõõng}
\formephonétique{hɔ̃ːŋ}
\région{PA}
(\domainesémantique{Mouvements ou actions avec la tête, les yeux, la bouche
, Verbes de mouvement})
\classe{v}
\begin{glose}
\pfra{tirer (langue)}
\end{glose}
\begin{glose}
\pfra{sortir d'un trou (animal, reptile)}
\end{glose}
\newline
\begin{exemple}
\région{PA}
\textbf{\pnua{i hõõnge kumèè-n}}
\pfra{il tire la langue}
\end{exemple}
\newline
\begin{exemple}
\région{PA}
\textbf{\pnua{i hõõnge mèè-n}}
\pfra{il a les yeux exorbités}
\end{exemple}
\newline
\relationsémantique{Cf.}{\lien{}{phããde kumèè-n}}
\glosecourte{tirer la langue}
\end{entrée}

\begin{entrée}{hoońõ}{}{ⓔhoońõ}
\formephonétique{hoːnɔ̃}
\région{GOs PA BO}
(\domainesémantique{Corps humain})
\classe{nom}
\begin{glose}
\pfra{intestins ; boyaux ; entrailles}
\end{glose}
\newline
\begin{exemple}
\région{GO}
\textbf{\pnua{hoońõ-nu}}
\pfra{mes intestins}
\end{exemple}
\newline
\begin{exemple}
\textbf{\pnua{hoońõ ko}}
\pfra{les intestins du poulet}
\end{exemple}
\newline
\begin{exemple}
\région{PA}
\textbf{\pnua{hoonõ-m}}
\pfra{tes intestins}
\end{exemple}
\end{entrée}

\begin{entrée}{hõõ-tòl}{}{ⓔhõõ-tòl}
\formephonétique{hɔ̃ːtɔl}
\région{PA}
(\domainesémantique{Instruments})
\classe{nom}
\begin{glose}
\pfra{tôle}
\end{glose}
\end{entrée}

\begin{entrée}{hoova}{}{ⓔhoova}
\région{GOs}
(\domainesémantique{Mouvements ou actions avec la tête, les yeux, la bouche})
\classe{v}
\begin{glose}
\pfra{souffler}
\end{glose}
\end{entrée}

\begin{entrée}{hõõ-wony}{}{ⓔhõõ-wony}
\formephonétique{hɔ̃-wɔ̃ɲ}
\région{PA}
(\domainesémantique{Navigation})
\classe{nom}
\begin{glose}
\pfra{planche de bateau}
\end{glose}
\end{entrée}

\begin{entrée}{hò pii-me}{}{ⓔhò pii-me}
\région{GOs}
\variante{%
wò pii-me
\région{GO(s)}}
(\domainesémantique{Mouvements ou actions avec la tête, les yeux, la bouche})
\classe{v}
\begin{glose}
\pfra{écarquiller le yeux}
\end{glose}
\newline
\begin{exemple}
\région{PA GO}
\textbf{\pnua{i hò pii-mèè-n}}
\pfra{elle écarquille les yeux}
\end{exemple}
\newline
\begin{exemple}
\région{GO}
\textbf{\pnua{e wò pii-me}}
\pfra{elle écarquille les yeux}
\end{exemple}
\end{entrée}

\begin{entrée}{horayee}{}{ⓔhorayee}
\région{PA}
\région{PA BO}
\variante{%
orayee
}
(\domainesémantique{Types de maison, architecture de la maison})
\classe{nom}
\begin{glose}
\pfra{gaulettes circulaires du toit}
\end{glose}
\begin{glose}
\pfra{baguette}
\end{glose}
\newline
\note{perpendiculaires à la pente du toit, tiennent les peaux de niaoulis ou la paille ; plus petites que 'ce-mwa'}{glose}{}
\newline
\begin{sous-entrée}{orayee pwa}{ⓔhorayeeⓝorayee pwa}
\begin{glose}
\pfra{gaulettes extérieures}
\end{glose}
\end{sous-entrée}
\newline
\begin{sous-entrée}{orayee mwa}{ⓔhorayeeⓝorayee mwa}
\begin{glose}
\pfra{gaulettes intérieures}
\end{glose}
\newline
\relationsémantique{Cf.}{\lien{ⓔce-mwa}{ce-mwa}}
\glosecourte{solives}
\end{sous-entrée}
\end{entrée}

\begin{entrée}{hore}{}{ⓔhore}
\région{PA WEM BO}
(\domainesémantique{Verbes de déplacement et moyens de déplacement})
\classe{v}
\begin{glose}
\pfra{suivre ; longer}
\end{glose}
\newline
\begin{exemple}
\textbf{\pnua{i hore bwèè-xòò-n}}
\pfra{elle suit les traces de pas}
\end{exemple}
\newline
\begin{exemple}
\région{BO}
\textbf{\pnua{i piina hore jaaòl}}
\pfra{elle se promène le long du fleuve}
\end{exemple}
\newline
\begin{exemple}
\région{PA}
\textbf{\pnua{i a-hore wèdò}}
\pfra{il suit les coutumes}
\end{exemple}
\end{entrée}

\begin{entrée}{hovalek}{}{ⓔhovalek}
\région{PA}
\classe{nom}
\newline
\sens{1}
(\domainesémantique{Ponts})
\begin{glose}
\pfra{passerelle ; pont (tronc d'arbre ou planche pour traverser une rivière)}
\end{glose}
\newline
\sens{2}
(\domainesémantique{Relations et interaction sociales})
\begin{glose}
\pfra{personne servant de lien entre deux clans}
\end{glose}
\end{entrée}

\begin{entrée}{hova-mwa}{}{ⓔhova-mwa}
\région{GOs PA WEM}
(\domainesémantique{Organisation sociale})
\classe{nom}
\begin{glose}
\pfra{clans (tous les) qui composent la chefferie et qui la protègent en temps de guerre}
\end{glose}
\newline
\relationsémantique{Cf.}{\lien{ⓔho}{ho}}
\glosecourte{protéger}
\end{entrée}

\begin{entrée}{hovwa}{}{ⓔhovwa}
\formephonétique{hoβa}
\région{GOs WEM}
\variante{%
hova
\région{PA BO WEM}, 
hava
\région{PA BO}}
\newline
\sens{1}
(\domainesémantique{Verbes de déplacement et moyens de déplacement})
\classe{v}
\begin{glose}
\pfra{arriver ; arrivé (courrier)}
\end{glose}
\newline
\begin{sous-entrée}{hovwa-da}{ⓔhovwaⓢ1ⓝhovwa-da}
\région{GO}
\begin{glose}
\pfra{arrivé en haut}
\end{glose}
\end{sous-entrée}
\newline
\sens{2}
(\domainesémantique{Localisation})
\classe{PREP}
\begin{glose}
\pfra{jusqu'à}
\end{glose}
\newline
\begin{exemple}
\région{GO}
\textbf{\pnua{e a na Gome hovwa Kumwa}}
\pfra{il est allé de Gomen jusqu'à Koumac}
\end{exemple}
\newline
\sens{3}
(\domainesémantique{Conjonction})
\classe{CNJ}
\begin{glose}
\pfra{jusqu'à ce que}
\end{glose}
\newline
\begin{exemple}
\région{GO}
\textbf{\pnua{ne nye-na hovwa-da na jö ogi}}
\pfra{fais-le jusqu'à ce que tu aies fini}
\end{exemple}
\newline
\relationsémantique{Ant.}{\lien{}{li (h)ova-du}}
\glosecourte{ils arrivent en bas}
\end{entrée}

\begin{entrée}{hovwa-da}{}{ⓔhovwa-da}
\formephonétique{hoβanda}
\région{GOs}
\variante{%
havha-da
\région{PA}}
(\domainesémantique{Conjonction})
\classe{CNJ}
\begin{glose}
\pfra{jusqu'à (locatif)}
\end{glose}
\newline
\begin{exemple}
\textbf{\pnua{havha-da ni wôding}}
\pfra{jusqu'au col}
\end{exemple}
\end{entrée}

\begin{entrée}{hovwa-da xa}{}{ⓔhovwa-da xa}
\formephonétique{hoβada}
\région{GOs}
\variante{%
havha-da
\région{PA}}
(\domainesémantique{Conjonction})
\classe{CNJ}
\begin{glose}
\pfra{jusqu'à ce que}
\end{glose}
\newline
\begin{exemple}
\région{GO}
\textbf{\pnua{e yu kòlò kêê-je ma õã-je xa ẽnõ gò hovwa-da xa whamã mwã}}
\pfra{il vit chez son père et sa mère depuis qu'il est enfant jusqu'à ce qu'il soit devenu grand}
\end{exemple}
\newline
\begin{exemple}
\région{PA}
\textbf{\pnua{havha-da xa u whamã mwã}}
\pfra{jusqu'à ce qu'il soit devenu grand}
\end{exemple}
\end{entrée}

\begin{entrée}{hòvwi}{}{ⓔhòvwi}
\formephonétique{hɔβi}
\région{GOs}
\variante{%
hòvi
\région{BO PA}}
\classe{v}
\newline
\sens{1}
(\domainesémantique{Mouvements ou actions faits avec le corps, les bras, les mains, les pieds})
\begin{glose}
\pfra{soulever (des pierres, etc. pour chercher qqch.)}
\end{glose}
\newline
\begin{exemple}
\textbf{\pnua{e hòvwi paa}}
\pfra{elle soulève des pierres (pour chercher qqch.)}
\end{exemple}
\newline
\sens{2}
(\domainesémantique{Fonctions naturelles humaines})
\begin{glose}
\pfra{réveiller (en secouant, en retournant)}
\end{glose}
\end{entrée}

\begin{entrée}{hovwo}{}{ⓔhovwo}
\formephonétique{hoβo}
\région{GOs}
\variante{%
hovho
\région{PA BO}, 
hopo
\région{GO vx}}
\classe{v ; n}
\newline
\sens{1}
(\domainesémantique{Aliments, alimentation})
\begin{glose}
\pfra{manger (générique) ; nourriture (générique)}
\end{glose}
\newline
\begin{exemple}
\textbf{\pnua{e pa-hovwo-ni la ẽnõ}}
\pfra{elle fait manger les enfants}
\end{exemple}
\newline
\begin{exemple}
\textbf{\pnua{hovwo za ? - hovwo ponga hauva}}
\pfra{quel type de nourriture ? - de la nourriture pour la levée de deuil}
\end{exemple}
\newline
\begin{exemple}
\région{BO}
\textbf{\pnua{i a-hovho}}
\pfra{il est gourmand}
\end{exemple}
\newline
\begin{exemple}
\région{PA}
\textbf{\pnua{dòò-ce hovho}}
\pfra{feuilles comestibles}
\end{exemple}
\newline
\sens{2}
(\domainesémantique{Tabac, actions liées au tabac})
\begin{glose}
\pfra{chiquer (tabac) [BO]}
\end{glose}
\begin{glose}
\pfra{fumer (tabac) [BO]}
\end{glose}
\newline
\relationsémantique{Cf.}{\lien{}{cèni [GOs], cani}}
\glosecourte{manger (féculents)}
\newline
\relationsémantique{Cf.}{\lien{}{hu(u), hupo, huvo}}
\glosecourte{manger (nourriture carnée)}
\newline
\relationsémantique{Cf.}{\lien{ⓔbiije}{biije}}
\glosecourte{mâcher des écorces ou du magnania}
\newline
\relationsémantique{Cf.}{\lien{}{whizi, wili}}
\glosecourte{manger (canne à sucre)}
\newline
\relationsémantique{Cf.}{\lien{ⓔkûûńiⓗ1}{kûûńi}}
\glosecourte{manger (fruits)}
\end{entrée}

\begin{entrée}{hovwo thrôbo}{}{ⓔhovwo thrôbo}
\formephonétique{hoβo}
\région{GOs}
(\domainesémantique{Aliments, alimentation})
\classe{nom}
\begin{glose}
\pfra{dîner, souper}
\end{glose}
\end{entrée}

\begin{entrée}{hòwala}{}{ⓔhòwala}
\formephonétique{'hɔwala}
\région{GOs}
\variante{%
hòpwala
\région{GO(s)}, 
oala
\région{WE}}
(\domainesémantique{Fonctions naturelles humaines})
\classe{v}
\begin{glose}
\pfra{bâiller; éructer (?)}
\end{glose}
\end{entrée}

\begin{entrée}{hõxa}{}{ⓔhõxa}
\formephonétique{hɔ̃ɣa}
\région{GOs}
\variante{%
hoxa
\région{PA}}
(\domainesémantique{Quantificateurs})
\classe{nom}
\begin{glose}
\pfra{morceau ; part ; fragment}
\end{glose}
\newline
\begin{sous-entrée}{hõxa ce}{ⓔhõxaⓝhõxa ce}
\région{GO}
\begin{glose}
\pfra{une planche}
\end{glose}
\end{sous-entrée}
\newline
\begin{sous-entrée}{hoxa doo}{ⓔhõxaⓝhoxa doo}
\région{PA}
\begin{glose}
\pfra{un fragment de poterie}
\end{glose}
\end{sous-entrée}
\end{entrée}

\begin{entrée}{hoxaba}{}{ⓔhoxaba}
\région{PA BO}
\variante{%
hoyaba
\région{BO [Corne]}}
(\domainesémantique{Parties de plantes})
\classe{nom}
\begin{glose}
\pfra{fibres pour panier}
\end{glose}
\nomscientifique{Epipremnum pinnatum, Aracées}
\end{entrée}

\begin{entrée}{hõxa-ce}{}{ⓔhõxa-ce}
\région{GOs}
\variante{%
hõõ-ce
\région{PA}}
(\domainesémantique{Bois})
\classe{nom}
\begin{glose}
\pfra{planche}
\end{glose}
\end{entrée}

\begin{entrée}{hoxaxe}{}{ⓔhoxaxe}
\région{PA}
(\domainesémantique{Ustensiles})
\classe{nom}
\begin{glose}
\pfra{grattoir métallique}
\end{glose}
\end{entrée}

\begin{entrée}{hoxe}{}{ⓔhoxe}
\région{GOs PA}
(\domainesémantique{Aspect})
\classe{ITER}
\begin{glose}
\pfra{encore ; à nouveau}
\end{glose}
\newline
\begin{exemple}
\textbf{\pnua{i hoxe a-da}}
\pfra{il remonte encore}
\end{exemple}
\newline
\begin{exemple}
\textbf{\pnua{i cabol ta/ra i hoxe maani}}
\pfra{il se réveille pui se rendort encore}
\end{exemple}
\newline
\note{parfois abrégé en : o}{grammaire}{}
\end{entrée}

\begin{entrée}{hô-xe}{}{ⓔhô-xe}
\région{GOsPA}
(\domainesémantique{Préfixes classificateurs numériques})
\classe{CLF.NUM}
\begin{glose}
\pfra{un morceau (pastèque, fruit, igname, etc.)}
\end{glose}
\newline
\begin{exemple}
\textbf{\pnua{hô-xe, hô-tru, hõ-kò, hõ-pa, hõ-ni, etc.}}
\pfra{un, deux, trois, quatre, cinq morceaux}
\end{exemple}
\newline
\begin{exemple}
\région{PA}
\textbf{\pnua{hô-ru hõxa-ce}}
\pfra{deux planches}
\end{exemple}
\end{entrée}

\begin{entrée}{hoxèè}{}{ⓔhoxèè}
\formephonétique{hoɣɛː}
\région{GOs BO}
(\domainesémantique{Quantificateurs})
\classe{QNT}
\begin{glose}
\pfra{peu ; quelques ; quelques}
\end{glose}
\newline
\begin{exemple}
\région{GO}
\textbf{\pnua{hoxèè chaamwa i nu}}
\pfra{j'ai peu de bananes}
\end{exemple}
\newline
\begin{exemple}
\région{GO}
\textbf{\pnua{hoxèè la mala phwe-meevwu Maluma xa la uça}}
\pfra{peu de gens du clan Maluma sont venus}
\end{exemple}
\newline
\begin{exemple}
\région{GO}
\textbf{\pnua{la po hoxèè nai la}}
\pfra{ils sont un peu moins nombreux qu'eux}
\end{exemple}
\newline
\begin{exemple}
\région{GO}
\textbf{\pnua{e ceni hoxèè kui nai nu}}
\pfra{il mange moins d'igname que moi}
\end{exemple}
\newline
\begin{exemple}
\région{BO}
\textbf{\pnua{i cani hoxèè kwi nai nu}}
\pfra{il mange moins d'igname que moi}
\end{exemple}
\newline
\begin{sous-entrée}{po-hoxè}{ⓔhoxèèⓝpo-hoxè}
\begin{glose}
\pfra{peu}
\end{glose}
\newline
\relationsémantique{Ant.}{\lien{ⓔhaivwö}{haivwö}}
\glosecourte{beaucoup}
\end{sous-entrée}
\end{entrée}

\begin{entrée}{höze}{}{ⓔhöze}
\région{GOs}
\variante{%
hure
\région{WEM WE PA}, 
hore
\région{BO}}
\classe{v}
\newline
\sens{1}
(\domainesémantique{Verbes de déplacement et moyens de déplacement})
\begin{glose}
\pfra{suivre (berge, rivière, etc.)}
\end{glose}
\newline
\begin{exemple}
\région{GO}
\textbf{\pnua{e a höze koli we-za}}
\pfra{il suit le bord de la mer}
\end{exemple}
\newline
\begin{exemple}
\région{GO}
\textbf{\pnua{e a höze dè}}
\pfra{il suit la route}
\end{exemple}
\newline
\begin{exemple}
\région{GO}
\textbf{\pnua{e a höze koli we}}
\pfra{il suit la berge de la rivière}
\end{exemple}
\newline
\begin{exemple}
\région{BO}
\textbf{\pnua{a-hore}}
\pfra{aller en suivant}
\end{exemple}
\newline
\sens{2}
(\domainesémantique{Fonctions intellectuelles})
\begin{glose}
\pfra{raconter (histoire en suivant bien l'histoire)}
\end{glose}
\begin{glose}
\pfra{dire (prière) ; réciter}
\end{glose}
\newline
\begin{exemple}
\région{GO}
\textbf{\pnua{höze-zooni !}}
\pfra{dis-le bien !}
\end{exemple}
\end{entrée}

\begin{entrée}{hòzi}{}{ⓔhòzi}
\région{GOs}
(\domainesémantique{Mouvements ou actions faits avec le corps, les bras, les mains, les pieds})
\classe{v}
\begin{glose}
\pfra{mélanger ; tourner}
\end{glose}
\begin{glose}
\pfra{ratisser}
\end{glose}
\end{entrée}

\begin{entrée}{hu}{1}{ⓔhuⓗ1}
\région{GOs BO}
(\domainesémantique{Ignames})
\classe{nom}
\begin{glose}
\pfra{igname sp. (la plus dure)}
\end{glose}
\nomscientifique{Dioscorea alata L. (Dioscoréacées)}
\end{entrée}

\begin{entrée}{hu}{2}{ⓔhuⓗ2}
\région{BO (BM)}
(\domainesémantique{Agent})
\classe{AGT}
\begin{glose}
\pfra{agent}
\end{glose}
\newline
\begin{exemple}
\textbf{\pnua{i pa-toni go hu ri?}}
\pfra{qui fait sonner la musique?}
\end{exemple}
\end{entrée}

\begin{entrée}{hu}{3}{ⓔhuⓗ3}
\région{BO}
(\domainesémantique{Conjonction})
\classe{CNJ}
\begin{glose}
\pfra{jusqu'à (probablement < huli 'suivre')}
\end{glose}
\newline
\begin{exemple}
\textbf{\pnua{hu ra u kûûni}}
\pfra{jusqu'à ce qu'il ait fini}
\end{exemple}
\end{entrée}

\begin{entrée}{hû}{1}{ⓔhûⓗ1}
\région{GOs}
\variante{%
hûn
\région{PA BO}}
(\domainesémantique{Discours, échanges verbaux})
\classe{v}
\begin{glose}
\pfra{taire (se) ; faire le silence ; rester silencieux}
\end{glose}
\newline
\begin{sous-entrée}{ègu hû}{ⓔhûⓗ1ⓝègu hû}
\begin{glose}
\pfra{une personne renfermée, timide}
\end{glose}
\newline
\begin{exemple}
\textbf{\pnua{i ku-hûn}}
\pfra{il reste silencieux}
\end{exemple}
\newline
\relationsémantique{Cf.}{\lien{}{hô}}
\glosecourte{muet}
\end{sous-entrée}
\end{entrée}

\begin{entrée}{hû}{2}{ⓔhûⓗ2}
\région{GOs}
(\domainesémantique{Sons, bruits})
\classe{v}
\begin{glose}
\pfra{gronder}
\end{glose}
\newline
\begin{exemple}
\textbf{\pnua{hû niô}}
\pfra{le tonnerre gronde}
\end{exemple}
\end{entrée}

\begin{entrée}{hua}{}{ⓔhua}
\région{GOs}
\variante{%
ua
\région{GO(s)}, 
uany
\région{BO [BM]}, 
whany
\région{PA}}
(\domainesémantique{Relations et interaction sociales})
\classe{nom}
\begin{glose}
\pfra{malédiction}
\end{glose}
\end{entrée}

\begin{entrée}{hu-ãgu}{}{ⓔhu-ãgu}
\région{GOs}
(\domainesémantique{Santé, maladie})
\classe{nom}
\begin{glose}
\pfra{champignon (sur la peau ; lit. qui mange les gens)}
\end{glose}
\end{entrée}

\begin{entrée}{hûbale}{}{ⓔhûbale}
\région{PA BO}
(\domainesémantique{Aliments, alimentation})
\classe{v}
\begin{glose}
\pfra{manger sans dents [Corne]}
\end{glose}
\end{entrée}

\begin{entrée}{hubo}{}{ⓔhubo}
\région{GOs}
(\domainesémantique{Verbes d'action (en général)})
\classe{v}
\begin{glose}
\pfra{défaire (se) tout seul}
\end{glose}
\newline
\begin{exemple}
\textbf{\pnua{e hubo mhenõ-mhõ}}
\pfra{le noeud s'est défait}
\end{exemple}
\end{entrée}

\begin{entrée}{hubu}{}{ⓔhubu}
\région{GOs}
\variante{%
hubun, hubi
\région{PA}}
(\domainesémantique{Religion, représentations religieuses})
\classe{nom}
\begin{glose}
\pfra{puissance ; charisme}
\end{glose}
\begin{glose}
\pfra{mana}
\end{glose}
\newline
\begin{exemple}
\région{PA}
\textbf{\pnua{hubu-n}}
\pfra{sa puissance}
\end{exemple}
\newline
\begin{sous-entrée}{hubi-zo}{ⓔhubuⓝhubi-zo}
\begin{glose}
\pfra{la puissance bénéfique}
\end{glose}
\end{sous-entrée}
\newline
\begin{sous-entrée}{hubi-thraa}{ⓔhubuⓝhubi-thraa}
\begin{glose}
\pfra{la puissance maléfique}
\end{glose}
\end{sous-entrée}
\end{entrée}

\begin{entrée}{hu-da}{}{ⓔhu-da}
\région{GOs}
(\domainesémantique{Directionnels})
\classe{SUFF.DIR}
\begin{glose}
\pfra{en montant (suppose un déplacement)}
\end{glose}
\newline
\begin{exemple}
\textbf{\pnua{a-hu-mi-da}}
\pfra{viens ici en haut}
\end{exemple}
\end{entrée}

\begin{entrée}{hûda}{}{ⓔhûda}
\région{GOs PA BO}
(\domainesémantique{Noms des plantes})
\classe{nom}
\begin{glose}
\pfra{roseau}
\end{glose}
\end{entrée}

\begin{entrée}{hu-du}{}{ⓔhu-du}
\région{GOs}
(\domainesémantique{Directionnels})
\classe{SUFF.DIR}
\begin{glose}
\pfra{en descendant}
\end{glose}
\newline
\begin{exemple}
\textbf{\pnua{a-hu-mi-du}}
\pfra{viens ici en bas}
\end{exemple}
\end{entrée}

\begin{entrée}{hûgu}{}{ⓔhûgu}
\région{GOs}
\classe{v.stat.}
(\domainesémantique{Description des objets, formes, consistance, taille})
\begin{glose}
\pfra{épais}
\end{glose}
\end{entrée}

\begin{entrée}{hulò}{}{ⓔhulò}
\région{GOs BO PA}
\classe{nom}
\newline
\sens{1}
(\domainesémantique{Configuration des objets})
\begin{glose}
\pfra{bout (d'une chose longue) ; extrémité ; fin ; terme}
\end{glose}
\newline
\begin{sous-entrée}{hulò-de}{ⓔhulòⓢ1ⓝhulò-de}
\begin{glose}
\pfra{le bout de la route}
\end{glose}
\end{sous-entrée}
\newline
\begin{sous-entrée}{hulò-ta}{ⓔhulòⓢ1ⓝhulò-ta}
\begin{glose}
\pfra{le bout de la table}
\end{glose}
\end{sous-entrée}
\newline
\begin{sous-entrée}{hulò-menò}{ⓔhulòⓢ1ⓝhulò-menò}
\begin{glose}
\pfra{présent coutumier}
\end{glose}
\end{sous-entrée}
\newline
\begin{sous-entrée}{hulo ce}{ⓔhulòⓢ1ⓝhulo ce}
\région{PA}
\begin{glose}
\pfra{le faîte/la cime d'un arbre}
\end{glose}
\newline
\relationsémantique{Cf.}{\lien{}{ku- [BO, PA]}}
\glosecourte{bout, extrémité d'une surface ou d'une chose étendue}
\end{sous-entrée}
\newline
\sens{2}
(\domainesémantique{Fonctions intellectuelles})
\begin{glose}
\pfra{résultat (bon ou mauvais) ; conséquence}
\end{glose}
\newline
\begin{exemple}
\région{GO}
\textbf{\pnua{hulò vhaa i je}}
\pfra{la conséquence de ses paroles}
\end{exemple}
\newline
\begin{exemple}
\région{BO}
\textbf{\pnua{hulò-n}}
\pfra{son résultat}
\end{exemple}
\end{entrée}

\begin{entrée}{hulò-mhèńõ}{}{ⓔhulò-mhèńõ}
\formephonétique{'hulɔ-'mhɛnɔ̃}
\région{GOs}
\variante{%
hulò-mhee-n
\région{PA BO}}
(\domainesémantique{Coutumes, dons coutumiers})
\classe{nom}
\begin{glose}
\pfra{don pour saluer les hôtes quand on arrive quelque part (lit. fin du chemin)}
\end{glose}
\newline
\begin{exemple}
\région{BOPA}
\textbf{\pnua{hulò-mhee-ny}}
\pfra{ma coutume d'arrivée}
\end{exemple}
\newline
\begin{exemple}
\région{BOPA}
\textbf{\pnua{la na la hulò-mhee-la}}
\pfra{ils donnent leur coutume d'arrivée}
\end{exemple}
\end{entrée}

\begin{entrée}{hu-mi}{}{ⓔhu-mi}
\formephonétique{humi}
\région{GOs}
(\domainesémantique{Directions})
\classe{SUFF.DIR}
\begin{glose}
\pfra{près d'égo}
\end{glose}
\newline
\begin{exemple}
\textbf{\pnua{a-hu-mi}}
\pfra{approche !}
\end{exemple}
\newline
\begin{exemple}
\textbf{\pnua{a-hu-ò}}
\pfra{éloigne-toi !}
\end{exemple}
\end{entrée}

\begin{entrée}{hû-mudree}{}{ⓔhû-mudree}
\formephonétique{'hũ-'muɖe}
\région{GOs}
(\domainesémantique{Mouvements ou actions faits avec le corps, les bras, les mains, les pieds})
\classe{v}
\begin{glose}
\pfra{couper avec les dents ; déchirer avec les dents}
\end{glose}
\end{entrée}

\begin{entrée}{hûn}{}{ⓔhûn}
\région{PA BO}
\classe{v}
\newline
\sens{1}
(\domainesémantique{Phénomènes atmosphériques et naturels
, Sons, bruits})
\begin{glose}
\pfra{tonner ; gronder}
\end{glose}
\begin{glose}
\pfra{grondement (tonnerre)}
\end{glose}
\newline
\sens{2}
(\domainesémantique{Eau
, Sons, bruits})
\begin{glose}
\pfra{bruit de ruissellement de l'eau}
\end{glose}
\newline
\begin{exemple}
\région{PA}
\textbf{\pnua{i hûn (n)e nhyô}}
\pfra{le tonnerre gronde}
\end{exemple}
\newline
\begin{exemple}
\région{PA}
\textbf{\pnua{hûn-a we}}
\pfra{bruit de ruissellement de l'eau}
\end{exemple}
\newline
\relationsémantique{Cf.}{\lien{}{hûna (forme déterminée)}}
\glosecourte{grondement de qqch}
\end{entrée}

\begin{entrée}{hu-ò}{}{ⓔhu-ò}
\région{GOs}
(\domainesémantique{Directionnels})
\classe{SUFF.DIR}
\begin{glose}
\pfra{là-bas (en s'éloignant d'égo)}
\end{glose}
\newline
\begin{exemple}
\textbf{\pnua{a hu-ò}}
\pfra{pousse-toi!}
\end{exemple}
\end{entrée}

\begin{entrée}{huraò}{}{ⓔhuraò}
\formephonétique{huɽaɔ}
\région{GOs}
(\domainesémantique{Sentiments})
\classe{v}
\begin{glose}
\pfra{rester bouche-bée (bouche ouverte)}
\end{glose}
\newline
\begin{exemple}
\textbf{\pnua{e huraò}}
\pfra{il reste bouche-bée}
\end{exemple}
\end{entrée}

\begin{entrée}{huu}{}{ⓔhuu}
\région{GOs PA BO}
\variante{%
whuu
\région{GO(s)}}
\classe{v ; n}
\newline
\sens{1}
(\domainesémantique{Aliments, alimentation})
\begin{glose}
\pfra{manger (de la viande, du coco)}
\end{glose}
\newline
\begin{exemple}
\région{GO}
\textbf{\pnua{mã novwö bi baani vwo huu-bi}}
\pfra{et alors nous le tuerons pour le manger (pour qu'il soit notre nourriture)}
\end{exemple}
\newline
\begin{sous-entrée}{hu pilon}{ⓔhuuⓢ1ⓝhu pilon}
\région{PA}
\begin{glose}
\pfra{manger de la viande}
\end{glose}
\end{sous-entrée}
\newline
\sens{2}
(\domainesémantique{Mouvements ou actions avec la tête, les yeux, la bouche})
\begin{glose}
\pfra{mordre}
\end{glose}
\begin{glose}
\pfra{ronger}
\end{glose}
\newline
\begin{exemple}
\région{PA}
\textbf{\pnua{i whuu-nu u kuau}}
\pfra{le chien m'a mordu}
\end{exemple}
\newline
\begin{exemple}
\région{BO}
\textbf{\pnua{i huu pwe ?}}
\pfra{ça mord? (lit. il a mordu la ligne ?)}
\end{exemple}
\newline
\begin{exemple}
\région{BO}
\textbf{\pnua{i hu du u kuau}}
\pfra{le chien ronge l'os}
\end{exemple}
\newline
\begin{exemple}
\région{BO}
\textbf{\pnua{i hu-nu kuau}}
\pfra{le chien m'a mordu}
\end{exemple}
\newline
\sens{3}
(\domainesémantique{Mouvements ou actions faits avec le corps, les bras, les mains, les pieds})
\begin{glose}
\pfra{pincer}
\end{glose}
\begin{glose}
\pfra{piquer}
\end{glose}
\newline
\begin{exemple}
\région{PA}
\textbf{\pnua{i whuu-nu u neebu}}
\pfra{le moustique m'a piqué}
\end{exemple}
\newline
\begin{exemple}
\région{BO}
\textbf{\pnua{i hu-nu u paaji}}
\pfra{le crabe m'a pincé}
\end{exemple}
\newline
\note{hu-po, hu-vo, hu-wo}{grammaire}{manger qqch (marque d'objet indéfini)}
\end{entrée}

\begin{entrée}{huvado}{}{ⓔhuvado}
\région{GOs PA BO}
\variante{%
vado
\région{GO(s) BO}}
(\domainesémantique{Corps humain})
\classe{v.stat.}
\begin{glose}
\pfra{cheveux blancs (avoir les)}
\end{glose}
\newline
\begin{exemple}
\région{BO}
\textbf{\pnua{i huvado}}
\pfra{il a les cheveux blancs}
\end{exemple}
\end{entrée}

\begin{entrée}{hu-vo}{}{ⓔhu-vo}
\région{PA}
\variante{%
hu-po
\région{PA}}
\classe{v}
\newline
\sens{1}
(\domainesémantique{Aliments, alimentation})
\begin{glose}
\pfra{manger (en général)}
\end{glose}
\begin{glose}
\pfra{ronger}
\end{glose}
\newline
\relationsémantique{Cf.}{\lien{ⓔhovwo}{hovwo}}
\glosecourte{manger (général)}
\newline
\relationsémantique{Cf.}{\lien{}{cèni cani}}
\glosecourte{manger (féculents)}
\newline
\relationsémantique{Cf.}{\lien{ⓔhuu}{huu}}
\glosecourte{manger (nourriture carnée)}
\newline
\relationsémantique{Cf.}{\lien{ⓔbiije}{biije}}
\glosecourte{mâcher des écorces ou du magnania}
\newline
\relationsémantique{Cf.}{\lien{ⓔwiliⓗ1}{wili}}
\glosecourte{manger (canne à sucre)}
\newline
\relationsémantique{Cf.}{\lien{ⓔkûûniⓗ2}{kûûni}}
\glosecourte{manger (fruits)}
\newline
\sens{2}
(\domainesémantique{Mouvements ou actions avec la tête, les yeux, la bouche})
\begin{glose}
\pfra{mordre}
\end{glose}
\newline
\sens{3}
(\domainesémantique{Mouvements ou actions faits avec le corps, les bras, les mains, les pieds})
\begin{glose}
\pfra{pincer}
\end{glose}
\newline
\sens{4}
(\domainesémantique{Santé, maladie})
\begin{glose}
\pfra{démanger}
\end{glose}
\end{entrée}

\begin{entrée}{huzooni}{}{ⓔhuzooni}
\région{GO}
(\domainesémantique{Noms des plantes})
\classe{nom}
\begin{glose}
\pfra{plante}
\end{glose}
\nomscientifique{Jussieuae sp.}
\end{entrée}

\newpage

\lettrine{
i
î
}\begin{entrée}{i}{1}{ⓔiⓗ1}
\région{GOs}
\newline
\sens{1}
(\domainesémantique{Prépositions})
\classe{POSS.INDIR}
\begin{glose}
\pfra{à ; pour}
\end{glose}
\newline
\begin{exemple}
\textbf{\pnua{nu na loto i Abel}}
\pfra{je donne la voiture à/d'Abel}
\end{exemple}
\newline
\sens{2}
(\domainesémantique{Prépositions})
\classe{OBJ.INDIR}
\begin{glose}
\pfra{marque d'objet indirect}
\end{glose}
\end{entrée}

\begin{entrée}{i}{2}{ⓔiⓗ2}
\région{PA}
(\domainesémantique{Pronoms})
\classe{PRO 3° pers. SG (sujet)}
\begin{glose}
\pfra{il, elle}
\end{glose}
\end{entrée}

\begin{entrée}{i ?}{}{ⓔi ?}
\région{GOs BO}
\variante{%
wi?
\région{GO(s) BO}}
(\domainesémantique{Interrogatifs})
\classe{INT (statique : humains, PRO, n)}
\begin{glose}
\pfra{où ?}
\end{glose}
\newline
\begin{exemple}
\textbf{\pnua{i caaja ?}}
\end{exemple}
\newline
\begin{exemple}
\textbf{\pnua{i thoomwã-ò ?}}
\pfra{où est cette femme ?}
\pfra{où est Papa ?}
\end{exemple}
\newline
\begin{exemple}
\textbf{\pnua{ço a wi ?}}
\pfra{où vas-tu ?}
\end{exemple}
\newline
\begin{exemple}
\région{BO}
\textbf{\pnua{i Pol ?}}
\pfra{où est Paul ?}
\end{exemple}
\newline
\begin{exemple}
\textbf{\pnua{i je ?}}
\pfra{où est-il ?}
\end{exemple}
\newline
\begin{exemple}
\textbf{\pnua{i cö ?}}
\pfra{où es-tu ?}
\end{exemple}
\newline
\begin{exemple}
\textbf{\pnua{i la ?}}
\pfra{où sont-ils?}
\end{exemple}
\end{entrée}

\begin{entrée}{-î}{}{ⓔ-î}
\région{GO BO}
(\domainesémantique{Pronoms})
\classe{POSS 1° pers. duel incl.}
\begin{glose}
\pfra{nos (deux2incl.)}
\end{glose}
\end{entrée}

\begin{entrée}{ia ?}{}{ⓔia ?}
\région{GOs BO}
\variante{%
hia
\région{PA}}
(\domainesémantique{Interrogatifs})
\classe{INT (statique)}
\begin{glose}
\pfra{où ? ; quel ?}
\end{glose}
\newline
\begin{exemple}
\région{GO}
\textbf{\pnua{nu a kaze-du- ko ya ? - ko Waambi}}
\pfra{je vais à la pêche - où ? - à Waambi}
\end{exemple}
\newline
\begin{exemple}
\région{GO}
\textbf{\pnua{ia mõ-jö ?}}
\pfra{où est ta maison ?}
\end{exemple}
\newline
\begin{exemple}
\région{GO}
\textbf{\pnua{hèlè ia ? ma kixa !}}
\pfra{mais quel couteau? il n'y en a pas !}
\end{exemple}
\newline
\begin{exemple}
\région{GO}
\textbf{\pnua{ia nye ãbaa-je thoomwa ?}}
\pfra{où est sa soeur ?}
\end{exemple}
\newline
\begin{exemple}
\région{GO}
\textbf{\pnua{ia la ègu ?}}
\pfra{où sont les gens ?}
\end{exemple}
\newline
\begin{exemple}
\région{GO}
\textbf{\pnua{ia ègu-ò ?}}
\pfra{où est cet homme ?}
\end{exemple}
\newline
\begin{exemple}
\région{GO}
\textbf{\pnua{ia kuau ?}}
\pfra{où est ce chien ?}
\end{exemple}
\newline
\begin{exemple}
\région{BO}
\textbf{\pnua{ia mõ-m ?}}
\pfra{où est ta maison?}
\end{exemple}
\newline
\begin{exemple}
\textbf{\pnua{ia hinõ ?}}
\pfra{quel/où est le signe ?}
\end{exemple}
\newline
\relationsémantique{Cf.}{\lien{}{ge ea ?}}
\glosecourte{où se trouve ?}
\end{entrée}

\begin{entrée}{iã}{}{ⓔiã}
\formephonétique{i ̃ɛ̃}
\région{GOs PA BO}
(\domainesémantique{Pronoms})
\classe{PRO 1° pers. incl. PL}
\begin{glose}
\pfra{nous (plur. incl.)}
\end{glose}
\end{entrée}

\begin{entrée}{-iã}{}{ⓔ-iã}
\région{GOsPA BO}
(\domainesémantique{Pronoms})
\classe{POSS 1° pers. incl.}
\begin{glose}
\pfra{notre (plur. incl.)}
\end{glose}
\end{entrée}

\begin{entrée}{ibî}{}{ⓔibî}
\région{GOs}
\variante{%
ibîn
\région{PA}, 
iciibii
\région{vx}}
(\domainesémantique{Pronoms})
\classe{PRO.INDEP 1° pers. duel excl.}
\begin{glose}
\pfra{nous deux (excl.)}
\end{glose}
\end{entrée}

\begin{entrée}{içö}{}{ⓔiçö}
\formephonétique{iʒo}
\région{GOs}
\variante{%
iyo
\région{PA BO}, 
eyo
\région{BO}}
(\domainesémantique{Pronoms})
\classe{PRO.INDEP 2° pers. SG}
\begin{glose}
\pfra{toi}
\end{glose}
\newline
\begin{exemple}
\région{GO}
\textbf{\pnua{novwö içö, çö yuu wãã-na}}
\pfra{quant à toi, tu restes ainsi}
\end{exemple}
\end{entrée}

\begin{entrée}{içu}{}{ⓔiçu}
\formephonétique{iʒu}
\région{GOs}
\variante{%
iyu
\région{PA}}
(\domainesémantique{Dons, échanges, achat et vente, vol})
\classe{v}
\begin{glose}
\pfra{vendre ; commercer}
\end{glose}
\begin{glose}
\pfra{acheter [PA]}
\end{glose}
\newline
\begin{exemple}
\textbf{\pnua{e u içu-ni a-kò pwaji cai Kaawo xo õã-nu}}
\pfra{ma mère a vendu 3 crabes à Kaawo}
\end{exemple}
\newline
\begin{exemple}
\textbf{\pnua{e içu-ni cai Kaawo a-kò pwaji xo õã-nu}}
\pfra{ma mère a vendu à Kaawo 3 crabes}
\end{exemple}
\newline
\note{v.t. iyuni [PA]; ijuni [GOs]}{grammaire}{acheter qqch.}
\end{entrée}

\begin{entrée}{îdò-}{}{ⓔîdò-}
\région{GOs PA BO}
\newline
\sens{1}
(\domainesémantique{Configuration des objets})
\classe{nom}
\begin{glose}
\pfra{ligne ; alignement ; rangée (ignames, poteaux, etc.)}
\end{glose}
\newline
\begin{exemple}
\région{PA}
\textbf{\pnua{îdò-kui, îdoo-kui}}
\pfra{rangée d'ignames}
\end{exemple}
\newline
\begin{exemple}
\région{GO}
\textbf{\pnua{pe-îdò-bi}}
\pfra{nous sommes de la même lignée}
\end{exemple}
\newline
\begin{sous-entrée}{îdò-ègu}{ⓔîdò-ⓢ1ⓝîdò-ègu}
\begin{glose}
\pfra{génération, lignée}
\end{glose}
\end{sous-entrée}
\newline
\sens{2}
(\domainesémantique{Préfixes classificateurs numériques})
\classe{CLF.NUM}
\begin{glose}
\pfra{rangée (d'ignames, poteaux, etc.)}
\end{glose}
\newline
\begin{exemple}
\textbf{\pnua{îdò-xe, îdo-tru, îdo-ko, îdo-pa îdò-kui, etc.}}
\pfra{1, 2, 3, 4, 5 rangée(s) d'ignames}
\end{exemple}
\end{entrée}

\begin{entrée}{ii}{}{ⓔii}
\région{GOs PA BO}
\classe{v}
\newline
\sens{1}
(\domainesémantique{Préparation des aliments; modes de préparation et de cuisson})
\begin{glose}
\pfra{retirer qqch. de qqch. (marmite)}
\end{glose}
\begin{glose}
\pfra{vider (le four enterré) ; sortir du four}
\end{glose}
\begin{glose}
\pfra{servir (la nourriture)}
\end{glose}
\newline
\begin{exemple}
\région{GO}
\textbf{\pnua{e ii dröö}}
\pfra{elle sert (la nourriture de) la marmite (lit. elle sort de la marmite)}
\end{exemple}
\newline
\begin{exemple}
\région{GO}
\textbf{\pnua{e ii na ni kîbi}}
\pfra{elle sort (la nourriture du four}
\end{exemple}
\newline
\begin{exemple}
\région{GO}
\textbf{\pnua{è ii lai}}
\pfra{elle sert le riz}
\end{exemple}
\newline
\begin{exemple}
\région{GO}
\textbf{\pnua{ii dröö mwa !}}
\pfra{videz la marmite !}
\end{exemple}
\newline
\begin{exemple}
\région{PA}
\textbf{\pnua{ii mwa xo li doo}}
\pfra{ils se servent dans la marmite !}
\end{exemple}
\newline
\sens{2}
(\domainesémantique{Verbes d'action (en général)})
\begin{glose}
\pfra{sortir (de qqch, d'une poche, d'un panier)}
\end{glose}
\newline
\begin{exemple}
\région{GO}
\textbf{\pnua{è ii hõbwo na ni bese}}
\pfra{elle sert le linge de la cuvette}
\end{exemple}
\newline
\begin{exemple}
\région{GO}
\textbf{\pnua{la ii phò-kamyõ}}
\pfra{ils déchargent le camion}
\end{exemple}
\newline
\relationsémantique{Cf.}{\lien{ⓔudi}{udi}}
\glosecourte{retirer, enlever}
\newline
\relationsémantique{Cf.}{\lien{}{yatre, yare}}
\glosecourte{sortir}
\end{entrée}

\begin{entrée}{îî}{}{ⓔîî}
\région{GOsPA}
(\domainesémantique{Pronoms})
\classe{PRO 1° pers. duel incl. (sujet)}
\begin{glose}
\pfra{nous deux (inclusif)}
\end{glose}
\newline
\begin{exemple}
\région{GO}
\textbf{\pnua{xa îî, avwö mi baani}}
\pfra{et nous, nous avons voulu le tuer}
\end{exemple}
\end{entrée}

\begin{entrée}{-îî}{}{ⓔ-îî}
\région{GOs PA}
(\domainesémantique{Pronoms})
\classe{PRO (OBJ) ou POSS 1° pers. duel incl.}
\begin{glose}
\pfra{nous deux (incl.)}
\end{glose}
\end{entrée}

\begin{entrée}{iili}{}{ⓔiili}
\région{PA BO [Corne]}
(\domainesémantique{Verbes de mouvement})
\classe{v}
\begin{glose}
\pfra{rétracter (se)}
\end{glose}
\end{entrée}

\begin{entrée}{iing}{}{ⓔiing}
\région{PA}
(\domainesémantique{Aliments, alimentation})
\classe{v}
\begin{glose}
\pfra{dégoûté ; faire le difficile}
\end{glose}
\end{entrée}

\begin{entrée}{ije}{}{ⓔije}
\formephonétique{iɲɟe}
\région{GOs BO PA}
\région{PA}
\variante{%
iye
}
(\domainesémantique{Pronoms})
\classe{PRO.INDEP 3° pers.}
\begin{glose}
\pfra{elle ; lui}
\end{glose}
\newline
\begin{exemple}
\textbf{\pnua{axe ije ca, e phe jitua i je}}
\pfra{mais lui, il prend son arc}
\end{exemple}
\newline
\begin{exemple}
\textbf{\pnua{ije nye}}
\pfra{c'est elle, la voilà, elle est là}
\end{exemple}
\end{entrée}

\begin{entrée}{ijèè !}{}{ⓔijèè !}
\formephonétique{ijɛː}
\région{GOs}
(\domainesémantique{Interpellation})
\classe{INTJ}
\begin{glose}
\pfra{eh ! la femme !}
\end{glose}
\end{entrée}

\begin{entrée}{ijè-òli}{}{ⓔijè-òli}
\formephonétique{iɲɟɛ-ɔli}
\région{GOs}
(\domainesémantique{Pronoms})
\classe{PRO.DEIC.3 (3° pers.)}
\begin{glose}
\pfra{elle (au loin)}
\end{glose}
\newline
\begin{exemple}
\textbf{\pnua{nu nyãnume ijè-òli}}
\pfra{je lui ai fait signe à elle là-bas}
\end{exemple}
\newline
\relationsémantique{Cf.}{\lien{}{ã-òli}}
\glosecourte{lui là-bas}
\end{entrée}

\begin{entrée}{ijò}{}{ⓔijò}
\région{GOs}
(\domainesémantique{Pronoms})
\classe{PRO.INDEP 2° pers. duel}
\begin{glose}
\pfra{vous 2}
\end{glose}
\end{entrée}

\begin{entrée}{ilaa}{}{ⓔilaa}
\région{GOs PA}
(\domainesémantique{Pronoms})
\classe{PRO.INDEP 3° pers. PL}
\begin{glose}
\pfra{eux, elles}
\end{glose}
\newline
\note{quand ce pronom est employé pour référer à une seule personne, il a une valeur honorifique}{général}{}
\end{entrée}

\begin{entrée}{ila-e}{}{ⓔila-e}
\région{GOs}
(\domainesémantique{Démonstratifs})
\classe{PRO.DEIC 3° pers. fém. PL}
\begin{glose}
\pfra{elles-là}
\end{glose}
\end{entrée}

\begin{entrée}{ila-lhãã-ba}{}{ⓔila-lhãã-ba}
\région{GOs PA BO}
(\domainesémantique{Démonstratifs})
\classe{PRO.DEIC.2 (3° pers. PL)}
\begin{glose}
\pfra{voilà (les) là-bas (où je montre)}
\end{glose}
\end{entrée}

\begin{entrée}{ila-lhãã-òli}{}{ⓔila-lhãã-òli}
\région{GOs}
(\domainesémantique{Pronoms})
\classe{PRO.DEIC.3 (3° pers. fém. PL)}
\begin{glose}
\pfra{eux/elles là-bas}
\end{glose}
\end{entrée}

\begin{entrée}{ili}{}{ⓔili}
\région{GOs PA}
(\domainesémantique{Pronoms})
\classe{PRO.INDEP 3° pers. duel}
\begin{glose}
\pfra{eux deux}
\end{glose}
\end{entrée}

\begin{entrée}{ili-e}{}{ⓔili-e}
\région{GOs}
(\domainesémantique{Pronoms})
\classe{interpellation ou DEM}
\begin{glose}
\pfra{ces deux (personnes, proche) !}
\end{glose}
\newline
\begin{exemple}
\textbf{\pnua{ili-e !}}
\pfra{eh ! vous deux (proche) !}
\end{exemple}
\newline
\begin{exemple}
\textbf{\pnua{Li za a mwa ili-e}}
\pfra{les deux (autres) repartent}
\end{exemple}
\end{entrée}

\begin{entrée}{ilili}{}{ⓔilili}
\région{BO}
(\domainesémantique{Guerre})
\classe{v}
\begin{glose}
\pfra{pousser un cri de guerre [BM]}
\end{glose}
\end{entrée}

\begin{entrée}{ilò}{}{ⓔilò}
\région{GOs}
(\domainesémantique{Pronoms})
\classe{PRO.INDEP 3° pers. triel ou paucal}
\begin{glose}
\pfra{eux trois; eux (petit groupe)}
\end{glose}
\newline
\begin{sous-entrée}{ilò-è}{ⓔilòⓝilò-è}
\begin{glose}
\pfra{ceux-là (proche)}
\end{glose}
\end{sous-entrée}
\newline
\begin{sous-entrée}{ilò-ba}{ⓔilòⓝilò-ba}
\begin{glose}
\pfra{ceux-là (loin)}
\end{glose}
\end{sous-entrée}
\end{entrée}

\begin{entrée}{imã}{}{ⓔimã}
\région{GOs BO}
(\domainesémantique{Fonctions naturelles humaines})
\classe{v}
\begin{glose}
\pfra{uriner}
\end{glose}
\newline
\begin{sous-entrée}{we imã}{ⓔimãⓝwe imã}
\begin{glose}
\pfra{urine}
\end{glose}
\newline
\begin{exemple}
\région{GO}
\textbf{\pnua{we imòò-je}}
\pfra{son urine}
\end{exemple}
\end{sous-entrée}
\newline
\étymologie{
\langue{POc}
\étymon{*mimiR}}
\end{entrée}

\begin{entrée}{imaze}{}{ⓔimaze}
\formephonétique{i'maðe}
\région{GOs}
(\domainesémantique{Echinodermes})
\classe{nom}
\begin{glose}
\pfra{holothurie ; "bêche de mer"}
\end{glose}
\end{entrée}

\begin{entrée}{ime}{}{ⓔime}
\formephonétique{ime}
\région{GOs}
\variante{%
icòme
\région{vx}}
(\domainesémantique{Pronoms})
\classe{PRO.INDEP 1° pers. triel excl.}
\begin{glose}
\pfra{nous trois excl}
\end{glose}
\newline
\begin{exemple}
\textbf{\pnua{xa ime, me a yu kòlò-li}}
\pfra{et nous, nous restions chez eux}
\end{exemple}
\end{entrée}

\begin{entrée}{ine}{}{ⓔine}
\région{GOs BO}
(\domainesémantique{Oiseaux})
\classe{nom}
\begin{glose}
\pfra{pétrel (noir, sort la nuit)}
\end{glose}
\end{entrée}

\begin{entrée}{inîjeò}{}{ⓔinîjeò}
\région{PA}
(\domainesémantique{Temps})
\classe{ADV}
\begin{glose}
\pfra{auparavant ; la fois d'avant}
\end{glose}
\end{entrée}

\begin{entrée}{inu}{}{ⓔinu}
\formephonétique{iɳu}
\région{GOs PA}
(\domainesémantique{Pronoms})
\classe{PRO.INDEP 1° pers.}
\begin{glose}
\pfra{moi}
\end{glose}
\newline
\begin{exemple}
\textbf{\pnua{inu nye}}
\pfra{c'est moi, me voilà, je suis là}
\end{exemple}
\end{entrée}

\begin{entrée}{iò}{}{ⓔiò}
\région{GOs PA BO}
(\domainesémantique{Adverbes déictiques de temps})
\classe{ADV}
\begin{glose}
\pfra{bientôt ; tout à l'heure (à) (futur ou passé)}
\end{glose}
\newline
\begin{exemple}
\textbf{\pnua{pe-rooli iò !}}
\pfra{à tout à l'heure !}
\end{exemple}
\newline
\begin{exemple}
\textbf{\pnua{iò waa !}}
\pfra{ce matin (retrospectif)}
\end{exemple}
\newline
\begin{exemple}
\région{BO}
\textbf{\pnua{iò ni waang !}}
\pfra{ce matin (retrospectif)}
\end{exemple}
\newline
\begin{exemple}
\région{BO}
\textbf{\pnua{iò xa waang !}}
\pfra{ce matin (prospectif)}
\end{exemple}
\newline
\begin{exemple}
\région{GO}
\textbf{\pnua{ge le xa ãbaa-we ne zoma a iò ne thrõbo}}
\pfra{certains d'entre vous partiront ce soir (tout à l'heure au soir)}
\end{exemple}
\newline
\begin{exemple}
\région{GO}
\textbf{\pnua{ge le ãbaa wõ ma la kòi-ò}}
\pfra{certains bateaux ont disparu}
\end{exemple}
\end{entrée}

\begin{entrée}{iõ}{}{ⓔiõ}
\formephonétique{iɔ̃}
\région{GOs}
(\domainesémantique{Pronoms})
\classe{PRO.INDEP 1° pers. triel incl.}
\begin{glose}
\pfra{nous trois incl. ; nous paucal}
\end{glose}
\end{entrée}

\begin{entrée}{iò-gò}{}{ⓔiò-gò}
\région{GOs}
\variante{%
iò-gòl
\région{WEM WE PA BO}}
(\domainesémantique{Adverbes déictiques de temps})
\classe{ADV}
\begin{glose}
\pfra{tout-à-l'heure (passé)}
\end{glose}
\newline
\begin{exemple}
\textbf{\pnua{nu ogine iò-gò}}
\pfra{je l'ai déjà fait}
\end{exemple}
\end{entrée}

\begin{entrée}{ira}{}{ⓔira}
\région{BO}
(\domainesémantique{Types de champs})
\classe{nom}
\begin{glose}
\pfra{champ de cultures sur une pente [Corne]}
\end{glose}
\newline
\note{les sillons suivent la direction de la pente ; des trous captent l'eau et permettent son écoulement, évitant l'érosion}{glose}{}
\newline
\note{non vérifié}{général}{}
\end{entrée}

\begin{entrée}{irô}{}{ⓔirô}
\région{BO}
(\domainesémantique{Relations et interaction sociales})
\classe{v}
\begin{glose}
\pfra{déranger}
\end{glose}
\newline
\relationsémantique{Cf.}{\lien{}{irôni}}
\glosecourte{déranger qqn}
\end{entrée}

\begin{entrée}{iva}{}{ⓔiva}
\région{GOs}
(\domainesémantique{Pronoms})
\classe{PRO.INDEP 1° pers. excl. PL}
\begin{glose}
\pfra{nous (plur. excl.)}
\end{glose}
\newline
\note{(forme moderne de izava)}{général}{}
\end{entrée}

\begin{entrée}{ivwö}{}{ⓔivwö}
\formephonétique{iβo}
\région{GOs}
(\domainesémantique{Alliance})
\classe{nom}
\begin{glose}
\pfra{belle-soeur}
\end{glose}
\newline
\relationsémantique{Cf.}{\lien{ⓔbeeⓗ1}{bee}}
\glosecourte{beau-frère}
\end{entrée}

\begin{entrée}{iwa}{}{ⓔiwa}
\région{GOs PA}
(\domainesémantique{Pronoms})
\classe{PRO.INDEP 2° pers. PL}
\begin{glose}
\pfra{vous (pluriel)}
\end{glose}
\end{entrée}

\begin{entrée}{iwe}{}{ⓔiwe}
\région{GOs}
\variante{%
icòòwe
\région{vx}}
(\domainesémantique{Pronoms})
\classe{PRO.INDEP 2° pers. triel}
\begin{glose}
\pfra{vous 3}
\end{glose}
\end{entrée}

\begin{entrée}{iyo}{}{ⓔiyo}
\région{BO PA}
\variante{%
eyo
\région{BO}}
(\domainesémantique{Pronoms})
\classe{PRO.INDEP 2° pers. SG}
\begin{glose}
\pfra{toi}
\end{glose}
\end{entrée}

\begin{entrée}{iza}{}{ⓔiza}
\région{PA}
\variante{%
iyãã
\région{BO}}
(\domainesémantique{Pronoms})
\classe{PRO.INDEP 1° pers. excl. PL}
\begin{glose}
\pfra{nous (plur. excl.)}
\end{glose}
\end{entrée}

\begin{entrée}{izava}{}{ⓔizava}
\formephonétique{iðava}
\région{GOs}
\variante{%
zava
\formephonétique{ðava}
\région{GO(s)}, 
za
\région{PA}}
(\domainesémantique{Pronoms})
\classe{PRO.INDEP 1° pers. excl. PL}
\begin{glose}
\pfra{nous (excl) (forme ancienne)}
\end{glose}
\end{entrée}

\begin{entrée}{izawa}{}{ⓔizawa}
\formephonétique{iðawa}
\région{GOs}
(\domainesémantique{Pronoms})
\classe{PRO.INDEP 2° pers. PL}
\begin{glose}
\pfra{vous (forme ancienne)}
\end{glose}
\end{entrée}

\begin{entrée}{izòò}{}{ⓔizòò}
\région{PA BO}
(\domainesémantique{Pronoms})
\classe{PRO.INDEP 2° pers. PL}
\begin{glose}
\pfra{vous (plur.)}
\end{glose}
\end{entrée}

\newpage

\lettrine{j}\begin{entrée}{ja}{1}{ⓔjaⓗ1}
\formephonétique{ɲɟa}
\région{GOs BO}
(\domainesémantique{Description des objets, formes, consistance, taille})
\classe{nom}
\begin{glose}
\pfra{saletés ; ordures ; détritus ; déchets}
\end{glose}
\end{entrée}

\begin{entrée}{ja}{2}{ⓔjaⓗ2}
\région{BO [Corne]}
\variante{%
jan
\région{BO [BM]}}
\classe{nom}
\newline
\sens{1}
(\domainesémantique{Parties de plantes})
\begin{glose}
\pfra{paille ; brindille}
\end{glose}
\newline
\sens{2}
(\domainesémantique{Types de maison, architecture de la maison})
\begin{glose}
\pfra{paille à toiture}
\end{glose}
\end{entrée}

\begin{entrée}{ja}{3}{ⓔjaⓗ3}
\région{GOs}
\variante{%
jak
\région{BO PA}}
(\domainesémantique{Fonctions intellectuelles})
\classe{v}
\begin{glose}
\pfra{mesurer}
\end{glose}
\begin{glose}
\pfra{peser}
\end{glose}
\newline
\begin{sous-entrée}{baa-ja}{ⓔjaⓗ3ⓝbaa-ja}
\begin{glose}
\pfra{balance}
\end{glose}
\newline
\relationsémantique{Cf.}{\lien{ⓔjange}{jange}}
\glosecourte{mesurer qqch}
\end{sous-entrée}
\end{entrée}

\begin{entrée}{jaa}{1}{ⓔjaaⓗ1}
\région{GOs PA BO}
\variante{%
jaac
\région{BO (Corne)}}
(\domainesémantique{Noms des plantes})
\classe{nom}
\begin{glose}
\pfra{salsepareille}
\end{glose}
\newline
\note{(liane d'ornementation, sert à faire des guirlandes ; sert à faire l'armature des berceaux pour les nourrissons et l'armature des épuisettes à crevette)}{glose}{}
\nomscientifique{Smilax sp. (Liliacées)}
\end{entrée}

\begin{entrée}{jaa}{2}{ⓔjaaⓗ2}
\région{GOs}
(\domainesémantique{Discours, échanges verbaux})
\classe{nom}
\begin{glose}
\pfra{annonce ; forme courte de "jaale"}
\end{glose}
\newline
\begin{exemple}
\textbf{\pnua{la phããde jaa}}
\pfra{ils font une annonce}
\end{exemple}
\newline
\relationsémantique{Cf.}{\lien{ⓔjaale}{jaale}}
\glosecourte{annoncer}
\end{entrée}

\begin{entrée}{jaa-ce}{}{ⓔjaa-ce}
\région{GOs}
(\domainesémantique{Bois})
\classe{nom}
\begin{glose}
\pfra{copeaux de bois}
\end{glose}
\end{entrée}

\begin{entrée}{jaale}{}{ⓔjaale}
\région{GOs BO}
(\domainesémantique{Discours, échanges verbaux})
\classe{v}
\begin{glose}
\pfra{annoncer ; prévenir}
\end{glose}
\newline
\begin{exemple}
\textbf{\pnua{la jaale-xa vhaa}}
\pfra{ils ont annoncé une nouvelle (lit. parole)}
\end{exemple}
\newline
\begin{sous-entrée}{baa-jaale}{ⓔjaaleⓝbaa-jaale}
\begin{glose}
\pfra{geste destiné à annoncer}
\end{glose}
\newline
\relationsémantique{Cf.}{\lien{}{tre-khôbwe}}
\glosecourte{prévenir}
\newline
\relationsémantique{Cf.}{\lien{ⓔphããde}{phããde}}
\glosecourte{révéler, annoncer}
\end{sous-entrée}
\end{entrée}

\begin{entrée}{jaaò}{}{ⓔjaaò}
\région{GOs}
\variante{%
jaaòl, jaòòl
\région{BO PA}}
(\domainesémantique{Eau})
\classe{nom}
\begin{glose}
\pfra{fleuve ; Diahot (nom d'un fleuve)}
\end{glose}
\newline
\begin{sous-entrée}{kudo phwe jaaò}{ⓔjaaòⓝkudo phwe jaaò}
\begin{glose}
\pfra{l'embouchure de la rivière}
\end{glose}
\end{sous-entrée}
\newline
\begin{sous-entrée}{koli jaaò}{ⓔjaaòⓝkoli jaaò}
\begin{glose}
\pfra{rive de la rivière}
\end{glose}
\newline
\begin{exemple}
\région{BO}
\textbf{\pnua{ku jaòòl}}
\pfra{fond de la rivière}
\end{exemple}
\newline
\begin{exemple}
\région{BO}
\textbf{\pnua{phwè jaòòl}}
\pfra{embouchure de la rivière}
\end{exemple}
\end{sous-entrée}
\end{entrée}

\begin{entrée}{jaaxe}{}{ⓔjaaxe}
\région{BO}
(\domainesémantique{Fonctions intellectuelles})
\classe{v}
\begin{glose}
\pfra{mesurer (la hauteur avec son corps)}
\end{glose}
\newline
\begin{exemple}
\textbf{\pnua{i jaaxe we}}
\pfra{elle mesure la hauteur de l'eau sur son corps)}
\end{exemple}
\end{entrée}

\begin{entrée}{jago}{}{ⓔjago}
\région{PA}
(\domainesémantique{Aliments, alimentation})
\classe{v}
\begin{glose}
\pfra{manger sans rien laisser aux autres}
\end{glose}
\end{entrée}

\begin{entrée}{jak}{}{ⓔjak}
\région{BO}
(\domainesémantique{Fonctions intellectuelles})
\classe{v.i}
\begin{glose}
\pfra{mesurer}
\end{glose}
\newline
\begin{sous-entrée}{ba-jak}{ⓔjakⓝba-jak}
\begin{glose}
\pfra{instrument de mesure}
\end{glose}
\newline
\relationsémantique{Cf.}{\lien{}{jaxe (v.t.)}}
\glosecourte{mesurer qqch}
\end{sous-entrée}
\end{entrée}

\begin{entrée}{jali}{}{ⓔjali}
\région{BO}
(\domainesémantique{Taros})
\classe{nom}
\begin{glose}
\pfra{taro d'eau (clone ; Dubois)}
\end{glose}
\end{entrée}

\begin{entrée}{jamali}{}{ⓔjamali}
\région{GOs WEM BO PA}
\newline
\groupe{A}
(\domainesémantique{Bois})
\classe{nom}
\begin{glose}
\pfra{coeur de bois dur (par ex. de gaiac, bois de fer, bois pétrole)}
\end{glose}
\newline
\groupe{B}
\classe{MODIF}
(\domainesémantique{Description des objets, formes, consistance, taille})
\begin{glose}
\pfra{ferme ; dur ; solide}
\end{glose}
\end{entrée}

\begin{entrée}{jamo-hi}{}{ⓔjamo-hi}
\région{PA}
(\domainesémantique{Corps humain})
\classe{nom}
\begin{glose}
\pfra{poing}
\end{glose}
\newline
\begin{exemple}
\textbf{\pnua{jamo-hi-n}}
\pfra{son poing}
\end{exemple}
\end{entrée}

\begin{entrée}{jamwe}{}{ⓔjamwe}
\région{GOs WEM WE BO PA}
(\domainesémantique{Soins du corps})
\classe{v}
\begin{glose}
\pfra{laver (vaisselle, vêtement, cheveu)}
\end{glose}
\newline
\begin{exemple}
\région{GO}
\textbf{\pnua{nu jamwe inu}}
\pfra{je me lave}
\end{exemple}
\newline
\begin{exemple}
\région{GO}
\textbf{\pnua{e jamwe mee xo Kavwo}}
\pfra{K se lave le visage}
\end{exemple}
\newline
\begin{exemple}
\région{GO}
\textbf{\pnua{e jamwe mee-je xo chaavwõ}}
\pfra{K se lave le visage avec du savon}
\end{exemple}
\newline
\begin{exemple}
\région{GO}
\textbf{\pnua{e jamwe mee-je xo Kavwo}}
\pfra{K lave son visage (celui de qqn d'autre)}
\end{exemple}
\end{entrée}

\begin{entrée}{jana}{}{ⓔjana}
\région{GOs BO}
(\domainesémantique{Dons, échanges, achat et vente, vol})
\classe{nom}
\begin{glose}
\pfra{marché ; échange de marchandises}
\end{glose}
\end{entrée}

\begin{entrée}{jange}{}{ⓔjange}
\région{GOs}
\variante{%
jaxe
\région{BO PA}}
(\domainesémantique{Fonctions intellectuelles})
\classe{v}
\begin{glose}
\pfra{mesurer}
\end{glose}
\begin{glose}
\pfra{peser}
\end{glose}
\newline
\note{jak (v.i), jaxe (v.t.)}{grammaire}{}
\end{entrée}

\begin{entrée}{jara}{}{ⓔjara}
\région{GOs}
(\domainesémantique{Modalité, verbes modaux})
\classe{MODAL}
\begin{glose}
\pfra{pas du tout}
\end{glose}
\newline
\begin{exemple}
\textbf{\pnua{kavwö nu jara phe}}
\pfra{je ne veux pas du tout le prendre}
\end{exemple}
\newline
\begin{exemple}
\textbf{\pnua{kavwö nu jara nee}}
\pfra{je ne absolument pas le faire}
\end{exemple}
\end{entrée}

\begin{entrée}{jaxa}{}{ⓔjaxa}
\région{GOs BO}
\newline
\groupe{A}
(\domainesémantique{Modalité, verbes modaux})
\begin{glose}
\pfra{mesure ; assez ; juste ; suffisant}
\end{glose}
\newline
\begin{exemple}
\région{GO}
\textbf{\pnua{gu jaxa !}}
\pfra{ça suffit !}
\end{exemple}
\newline
\begin{exemple}
\région{BO}
\textbf{\pnua{jaxa-n}}
\pfra{ça suffit !}
\end{exemple}
\newline
\begin{exemple}
\textbf{\pnua{pe-jaxa-li}}
\pfra{de même taille}
\end{exemple}
\newline
\groupe{B}
\classe{MODAL}
\newline
\sens{1}
(\domainesémantique{Modalité, verbes modaux})
\begin{glose}
\pfra{capable}
\end{glose}
\newline
\begin{exemple}
\région{GO}
\textbf{\pnua{kavwö jaxa-je vwö e zòò}}
\pfra{il est incapable de nager}
\end{exemple}
\newline
\sens{2}
(\domainesémantique{Modalité, verbes modaux})
\begin{glose}
\pfra{faillir ; manquer de}
\end{glose}
\newline
\begin{exemple}
\région{GO}
\textbf{\pnua{ja(xa)-vwo kaalu ẽnõ-ni !}}
\pfra{cet enfant a failli tomber !}
\end{exemple}
\newline
\begin{exemple}
\région{GO}
\textbf{\pnua{za ja(xa)-vwo la za mã !}}
\pfra{ils ont failli mourir !}
\end{exemple}
\newline
\sens{3}
(\domainesémantique{Modalité, verbes modaux})
\begin{glose}
\pfra{devoir (épistémique)}
\end{glose}
\newline
\begin{exemple}
\région{GO}
\textbf{\pnua{jaxa 3 heures mwa}}
\pfra{il doit être 3h}
\end{exemple}
\classe{n.MODAL}
\end{entrée}

\begin{entrée}{jaxe}{}{ⓔjaxe}
\région{GOs BO}
(\domainesémantique{Fonctions intellectuelles})
\classe{v}
\begin{glose}
\pfra{mesurer ;}
\end{glose}
\begin{glose}
\pfra{peser}
\end{glose}
\end{entrée}

\begin{entrée}{je}{1}{ⓔjeⓗ1}
\région{GOs PA BO}
\variante{%
ji
\région{BO}}
(\domainesémantique{Chasse})
\classe{nom}
\begin{glose}
\pfra{piège (à oiseau, rat) ; lacet}
\end{glose}
\newline
\begin{exemple}
\région{BO}
\textbf{\pnua{wa je}}
\pfra{corde pour piège}
\end{exemple}
\end{entrée}

\begin{entrée}{je}{2}{ⓔjeⓗ2}
\région{PA BO}
(\domainesémantique{Démonstratifs})
\classe{DEM.ANAPH}
\begin{glose}
\pfra{ce-là}
\end{glose}
\newline
\begin{exemple}
\région{PA}
\textbf{\pnua{ti je i tho ?}}
\pfra{qui a appelé ?}
\end{exemple}
\newline
\begin{exemple}
\région{BO}
\textbf{\pnua{nu je nu nooli}}
\pfra{c'est moi qui l'ai vu}
\end{exemple}
\newline
\begin{exemple}
\région{PA}
\textbf{\pnua{ti je i a-daa-mi ?}}
\pfra{qui est monté ici ?}
\end{exemple}
\end{entrée}

\begin{entrée}{-je}{}{ⓔ-je}
\région{GO}
(\domainesémantique{Pronoms})
\classe{PRO 3° pers. SG (OBJ)}
\begin{glose}
\pfra{le, la}
\end{glose}
\end{entrée}

\begin{entrée}{je-ba}{}{ⓔje-ba}
\région{GOs PA}
(\domainesémantique{Démonstratifs})
\classe{PRO.DEIC.2 (latéral)}
\begin{glose}
\pfra{celle-là (sur le côté, latéralement)}
\end{glose}
\newline
\begin{exemple}
\textbf{\pnua{nòòle je-ba thoomwã ba}}
\pfra{regarde cette fille-là}
\end{exemple}
\end{entrée}

\begin{entrée}{jebo}{}{ⓔjebo}
\région{PA WEM}
(\domainesémantique{Types de maison, architecture de la maison})
\classe{nom}
\begin{glose}
\pfra{poutre qui soutient la toiture (PA, WEM)}
\end{glose}
\newline
\relationsémantique{Cf.}{\lien{}{pwabwani [GO] ; pwabwaning [PA]}}
\glosecourte{poutre maîtresse}
\end{entrée}

\begin{entrée}{jee}{}{ⓔjee}
\région{PA}
(\domainesémantique{Religion, représentations religieuses})
\classe{nom}
\begin{glose}
\pfra{lutins (petits êtres aux cheveux longs, qui jouent des tours aux humains)}
\end{glose}
\newline
\relationsémantique{Cf.}{\lien{}{uramõ [GOs]}}
\end{entrée}

\begin{entrée}{jego}{}{ⓔjego}
\région{PA}
(\domainesémantique{Types de maison, architecture de la maison})
\classe{v}
\begin{glose}
\pfra{couvrir de paille racines vers l'extérieur}
\end{glose}
\end{entrée}

\begin{entrée}{jèmaa}{}{ⓔjèmaa}
\région{BO PA}
\variante{%
dada
\région{GO(s) PA}}
\classe{nom}
\newline
\sens{1}
(\domainesémantique{Insectes})
\begin{glose}
\pfra{cigale (petite, à tête verte, vit en forêt)}
\end{glose}
\newline
\sens{2}
(\domainesémantique{Caractéristiques et propriétés des personnes})
\begin{glose}
\pfra{pleurnicheur (métaphoriquement)}
\end{glose}
\end{entrée}

\begin{entrée}{je-nã}{}{ⓔje-nã}
\formephonétique{ɲɟeɳa}
\région{GOs}
\variante{%
jene
\région{PA BO}}
(\domainesémantique{Démonstratifs})
\classe{DEIC.2 ; ANAPH ; ASS}
\begin{glose}
\pfra{ce-là ; voilà ; c'est cela !}
\end{glose}
\begin{glose}
\pfra{c'est cela ; exactement ; tout à fait}
\end{glose}
\newline
\begin{exemple}
\textbf{\pnua{jenã mwã !}}
\pfra{c'était cela !}
\end{exemple}
\newline
\begin{exemple}
\textbf{\pnua{je-nã thoomwã}}
\pfra{cette femme-là (dont on a parlé)}
\end{exemple}
\newline
\begin{exemple}
\textbf{\pnua{jö trône je-nã gu dròrò ?}}
\pfra{tu as entendu ce bruit hier ?}
\end{exemple}
\end{entrée}

\begin{entrée}{je-ò}{}{ⓔje-ò}
\région{GOs PA}
(\domainesémantique{Démonstratifs})
\classe{PRO.DEIC ou ANAPH DX3}
\begin{glose}
\pfra{celle-là}
\end{glose}
\end{entrée}

\begin{entrée}{jeworo}{}{ⓔjeworo}
\région{BO}
(\domainesémantique{Découpage du temps})
\classe{nom}
\begin{glose}
\pfra{chant du coq (à l'aube)}
\end{glose}
\newline
\begin{exemple}
\région{GO}
\textbf{\pnua{tho ko ne mõnu trèè}}
\pfra{le chant du coq avant le jour}
\end{exemple}
\end{entrée}

\begin{entrée}{jeyu}{}{ⓔjeyu}
\formephonétique{ɲɟeyu}
\région{GOs WE}
\variante{%
jeü
\région{PA BO [BM]}}
(\domainesémantique{Arbre})
\classe{nom}
\begin{glose}
\pfra{kaori}
\end{glose}
\newline
\note{(son fruit est l'image des clans autour de la chefferie)}{glose}{}
\nomscientifique{Agathis moorei, (Araucariacées)}
\end{entrée}

\begin{entrée}{ji}{}{ⓔji}
\région{GOs}
\variante{%
jim
\région{PA BO}}
(\domainesémantique{Crustacés, crabes})
\classe{nom}
\begin{glose}
\pfra{crabe de palétuvier}
\end{glose}
\end{entrée}

\begin{entrée}{ji-a}{}{ⓔji-a}
\région{GOs BO}
(\domainesémantique{Interrogatifs})
\classe{PRO.INT}
\begin{glose}
\pfra{le(s)quel(s) ?}
\end{glose}
\newline
\begin{exemple}
\textbf{\pnua{nu phe jia ?}}
\pfra{je prend le(s)quel(s) ?}
\end{exemple}
\newline
\begin{exemple}
\textbf{\pnua{nu thooma-ni jia ?}}
\pfra{j'appelle le(s)quel(s) ?}
\end{exemple}
\newline
\begin{exemple}
\textbf{\pnua{nu phe tiiwo ia ?}}
\pfra{je prend quel livre ?}
\end{exemple}
\newline
\begin{exemple}
\textbf{\pnua{pòi-m a jia ?}}
\pfra{lequel est ton enfant ?}
\end{exemple}
\newline
\relationsémantique{Cf.}{\lien{ⓔia ?}{ia ?}}
\glosecourte{quel, où ? (+ nom)}
\end{entrée}

\begin{entrée}{jibaale}{}{ⓔjibaale}
\région{GOs PA}
(\domainesémantique{Insectes})
\classe{nom}
\begin{glose}
\pfra{araignée (vénimeuse, grosse et noire, vit dans la terre)}
\end{glose}
\end{entrée}

\begin{entrée}{jibwa}{}{ⓔjibwa}
\région{GOs BO}
(\domainesémantique{Jeux divers})
\classe{v}
\begin{glose}
\pfra{faire des galipettes ; faire des tonneaux}
\end{glose}
\begin{glose}
\pfra{retourner en bousculant}
\end{glose}
\end{entrée}

\begin{entrée}{jige}{}{ⓔjige}
\région{GOs}
(\domainesémantique{Armes})
\classe{nom}
\begin{glose}
\pfra{fusil de chasse}
\end{glose}
\newline
\begin{exemple}
\région{BO}
\région{PA}
\textbf{\pnua{jige bu-ò}}
\pfra{un fusil qui frappe loin (à longue portée)}
\end{exemple}
\newline
\begin{sous-entrée}{phao jige}{ⓔjigeⓝphao jige}
\begin{glose}
\pfra{tirer (au fusil)}
\end{glose}
\newline
\note{jigali-n}{grammaire}{son fusil}
\newline
\note{jigèle-n}{grammaire}{son fusil}
\newline
\note{jiga i je}{grammaire}{son fusil}
\end{sous-entrée}
\end{entrée}

\begin{entrée}{jigo}{}{ⓔjigo}
\région{BO}
(\domainesémantique{Arbre})
\classe{nom}
\begin{glose}
\pfra{palétuvier [Corne]}
\end{glose}
\nomscientifique{Rhizophora sp}
\newline
\note{non vérifié}{général}{}
\end{entrée}

\begin{entrée}{jiilè}{}{ⓔjiilè}
\formephonétique{'jiːlɛ}
\région{GOs}
(\domainesémantique{Verbes de mouvement})
\classe{v}
\begin{glose}
\pfra{tituber}
\end{glose}
\end{entrée}

\begin{entrée}{ji-li}{}{ⓔji-li}
\région{GOs PA}
(\domainesémantique{Démonstratifs})
\classe{DEIC.3}
\begin{glose}
\pfra{cela (éloigné, invisible mais audible)}
\end{glose}
\newline
\begin{exemple}
\région{GO}
\textbf{\pnua{jö trône jili gu ?}}
\pfra{tu entends ce bruit ? (au présent obligatoirement)}
\end{exemple}
\newline
\begin{exemple}
\région{GO}
\textbf{\pnua{da jili gu ?}}
\pfra{qu'est-ce que ce bruit ?}
\end{exemple}
\newline
\begin{exemple}
\région{GO}
\textbf{\pnua{ti jili ?}}
\pfra{qui sont ceux qu'on entend ?}
\end{exemple}
\newline
\begin{exemple}
\région{GO}
\textbf{\pnua{ti jili waa ?}}
\pfra{qui est-ce qui chante ?}
\end{exemple}
\newline
\begin{exemple}
\région{GO}
\textbf{\pnua{ti jili kãã ?}}
\pfra{qui est-ce qui crie ?}
\end{exemple}
\newline
\morphologie{je-ili}
\end{entrée}

\begin{entrée}{jime}{}{ⓔjime}
\région{GOs}
(\domainesémantique{Discours, échanges verbaux})
\classe{v}
\begin{glose}
\pfra{assembler (faire la synthèse des paroles avant de conclure))}
\end{glose}
\end{entrée}

\begin{entrée}{jimòng}{}{ⓔjimòng}
\région{PA}
(\domainesémantique{Eau})
\classe{nom}
\begin{glose}
\pfra{mare (qui sèche au soleil)}
\end{glose}
\end{entrée}

\begin{entrée}{jingã}{}{ⓔjingã}
\région{GOs}
\variante{%
jing
\région{PA}, 
pajin
\région{BO [Corne]}}
(\domainesémantique{Corps humain})
\classe{nom}
\begin{glose}
\pfra{gencives}
\end{glose}
\newline
\begin{exemple}
\région{PA}
\textbf{\pnua{jigã-ny}}
\pfra{mes gencives (une fois les dents tombées)}
\end{exemple}
\end{entrée}

\begin{entrée}{ji-ni}{}{ⓔji-ni}
\formephonétique{jiɳi}
\région{GOs}
\variante{%
je-nim, ji-nim
\région{PA}}
(\domainesémantique{Démonstratifs})
\classe{DEM (médial ou proche)}
\begin{glose}
\pfra{cela ;}
\end{glose}
\begin{glose}
\pfra{cela (péjoratif, mis à distance) [PA]}
\end{glose}
\newline
\begin{exemple}
\textbf{\pnua{nòòle ji-ni thoomwã ni}}
\pfra{regarde cette fille-là}
\end{exemple}
\end{entrée}

\begin{entrée}{jińõ ce-bon}{}{ⓔjińõ ce-bon}
\formephonétique{jinɔ̃ cɨmbɔn}
\région{PA BO}
(\domainesémantique{Feu : objets et actions liés au feu})
\classe{nom}
\begin{glose}
\pfra{tison}
\end{glose}
\begin{glose}
\pfra{brandon de la bûche pour le feu de la nuit}
\end{glose}
\end{entrée}

\begin{entrée}{jinoji}{}{ⓔjinoji}
\région{BO}
(\domainesémantique{Ignames})
\classe{nom}
\begin{glose}
\pfra{igname violette, tendre (Dubois)}
\end{glose}
\end{entrée}

\begin{entrée}{jińõ yaai}{}{ⓔjińõ yaai}
\formephonétique{'jinɔ̃ 'yaːi}
\région{GOs PA BO}
(\domainesémantique{Feu : objets et actions liés au feu})
\classe{nom}
\begin{glose}
\pfra{tisons}
\end{glose}
\begin{glose}
\pfra{brandon (utilisé pour mettre le feu ou s'éclairer dans le noir)}
\end{glose}
\end{entrée}

\begin{entrée}{jińu}{1}{ⓔjińuⓗ1}
\formephonétique{'jinuː}
\région{GOs PA BO}
(\domainesémantique{Religion, représentations religieuses})
\classe{nom}
\begin{glose}
\pfra{puissance ; force spirituelle}
\end{glose}
\begin{glose}
\pfra{vertu (d'une plante, d'un sorcier)}
\end{glose}
\begin{glose}
\pfra{esprit}
\end{glose}
\end{entrée}

\begin{entrée}{jińu}{2}{ⓔjińuⓗ2}
\formephonétique{'jinu}
\région{GOs}
\variante{%
jinuu-n, jinoo-n
\région{BO}}
(\domainesémantique{Température})
\classe{nom}
\begin{glose}
\pfra{chaleur}
\end{glose}
\newline
\begin{sous-entrée}{jińu yaai}{ⓔjińuⓗ2ⓝjińu yaai}
\région{GO PA}
\begin{glose}
\pfra{chaleur du feu}
\end{glose}
\end{sous-entrée}
\newline
\begin{sous-entrée}{jińu a}{ⓔjińuⓗ2ⓝjińu a}
\région{GO}
\begin{glose}
\pfra{chaleur du soleil}
\end{glose}
\end{sous-entrée}
\newline
\begin{sous-entrée}{jinuu al}{ⓔjińuⓗ2ⓝjinuu al}
\région{BO}
\begin{glose}
\pfra{chaleur du soleil}
\end{glose}
\end{sous-entrée}
\end{entrée}

\begin{entrée}{jiò}{}{ⓔjiò}
\région{GOs}
\variante{%
jòò-n
\région{BO [Corne]}}
(\domainesémantique{Corps humain})
\classe{nom}
\begin{glose}
\pfra{tempe}
\end{glose}
\end{entrée}

\begin{entrée}{jitrua}{}{ⓔjitrua}
\formephonétique{'jiʈua, jiɽua}
\région{GOs}
(\domainesémantique{Armes})
\classe{nom}
\begin{glose}
\pfra{arc}
\end{glose}
\newline
\begin{exemple}
\région{PA}
\textbf{\pnua{jitua i je}}
\pfra{son arc}
\end{exemple}
\newline
\begin{sous-entrée}{mee-jitua}{ⓔjitruaⓝmee-jitua}
\begin{glose}
\pfra{flèche ; pointe de la flèche}
\end{glose}
\end{sous-entrée}
\end{entrée}

\begin{entrée}{jiu}{}{ⓔjiu}
\région{GOs}
\variante{%
jiu-n
\région{PA BO [BM]}}
(\domainesémantique{Quantificateurs})
\classe{QNT}
\begin{glose}
\pfra{complètement ; totalement ; tout ; ensemble}
\end{glose}
\newline
\begin{exemple}
\région{BO}
\textbf{\pnua{waya jiu-la ?}}
\pfra{combien sont-ils en tout ?}
\end{exemple}
\newline
\begin{sous-entrée}{jiu-ã, jiu-la}{ⓔjiuⓝjiu-ã, jiu-la}
\région{PA}
\begin{glose}
\pfra{nous ensemble, eux ensemble}
\end{glose}
\newline
\begin{exemple}
\région{PA}
\textbf{\pnua{phe-jiun}}
\pfra{prend tout}
\end{exemple}
\newline
\begin{exemple}
\région{PA}
\textbf{\pnua{kavwö nu khobwe jiun}}
\pfra{tu n'as pas tout dit}
\end{exemple}
\newline
\begin{exemple}
\région{BO}
\textbf{\pnua{nu nèè jiun}}
\pfra{je l'ai fait complètement}
\end{exemple}
\end{sous-entrée}
\end{entrée}

\begin{entrée}{jivwa}{}{ⓔjivwa}
\formephonétique{'jiβa}
\région{GOs PA BO}
\variante{%
jipwa
\région{GO(s)}}
(\domainesémantique{Quantificateurs})
\classe{QNT}
\begin{glose}
\pfra{tous ; tout le monde ; totalité}
\end{glose}
\begin{glose}
\pfra{quelques ; plusieurs}
\end{glose}
\newline
\begin{exemple}
\région{GO}
\textbf{\pnua{Li nhuã mwã ije jivwa la yada kêê-je ma õã-je}}
\pfra{Son père et sa mère lui lèguent tous leurs biens}
\end{exemple}
\newline
\begin{exemple}
\région{PA}
\textbf{\pnua{a-mi jivwa}}
\pfra{venez tous!}
\end{exemple}
\newline
\begin{exemple}
\région{GO}
\textbf{\pnua{pe-jiwa-li/-lò/-la}}
\pfra{ils sont en nombre égal (les 2/ 3/plur.)}
\end{exemple}
\newline
\begin{exemple}
\région{PA}
\textbf{\pnua{a-mi jivwa}}
\pfra{venez tous!}
\end{exemple}
\newline
\begin{exemple}
\région{PA}
\textbf{\pnua{jivwa meevwu}}
\pfra{toutes sortes}
\end{exemple}
\end{entrée}

\begin{entrée}{jivwa meewu}{}{ⓔjivwa meewu}
\région{PA}
(\domainesémantique{Quantificateurs})
\classe{nom}
\begin{glose}
\pfra{toutes sortes de}
\end{glose}
\end{entrée}

\begin{entrée}{jiwaa}{}{ⓔjiwaa}
\formephonétique{'jiwaː}
\région{GOs}
\variante{%
jiia
\région{WEM PA BO}}
(\domainesémantique{Oiseaux})
\classe{nom}
\begin{glose}
\pfra{"siffleur", échenilleur calédonien}
\end{glose}
\nomscientifique{Coracina caledonica caledonica (Campéphagidés)}
\end{entrée}

\begin{entrée}{jo}{1}{ⓔjoⓗ1}
\région{GOs}
(\domainesémantique{Adverbe})
\classe{ADV}
\begin{glose}
\pfra{soudain}
\end{glose}
\newline
\begin{exemple}
\région{GO}
\textbf{\pnua{i jo gaajò}}
\pfra{il est tout à coup surpris}
\end{exemple}
\end{entrée}

\begin{entrée}{jo}{2}{ⓔjoⓗ2}
\région{GOs BO}
\variante{%
ço
\région{GO(s)}}
(\domainesémantique{Conjonction})
\classe{CNJ}
\begin{glose}
\pfra{et après ; puis}
\end{glose}
\end{entrée}

\begin{entrée}{jò}{1}{ⓔjòⓗ1}
\région{GOs}
\variante{%
jòn
\région{BO}}
(\domainesémantique{Mouvements ou actions faits avec le corps, les bras, les mains, les pieds})
\classe{v}
\begin{glose}
\pfra{sursauter}
\end{glose}
\newline
\begin{exemple}
\textbf{\pnua{nu jò}}
\pfra{je sursaute}
\end{exemple}
\end{entrée}

\begin{entrée}{jò}{2}{ⓔjòⓗ2}
\région{GOs}
\variante{%
jòm
\région{PA BO}, 
jem
\région{BO (Corne)}}
(\domainesémantique{Arbre})
\classe{nom}
\begin{glose}
\pfra{bancoulier}
\end{glose}
\newline
\note{(on extrait une teinture noire du fruit, cette couleur est l'un des symboles des premiers occupants, des maîtres de la terre et des soutiens de la chefferie)}{glose}{}
\nomscientifique{Aleurites moluccana L., (Euphorbiacées)}
\newline
\begin{sous-entrée}{pò-jò}{ⓔjòⓗ2ⓝpò-jò}
\région{GO}
\begin{glose}
\pfra{fruit de bancoulier}
\end{glose}
\end{sous-entrée}
\newline
\begin{sous-entrée}{pò-jòm}{ⓔjòⓗ2ⓝpò-jòm}
\région{PA}
\begin{glose}
\pfra{fruit de bancoulier}
\end{glose}
\end{sous-entrée}
\end{entrée}

\begin{entrée}{-jò}{}{ⓔ-jò}
\région{GOs}
(\domainesémantique{Pronoms})
\classe{PRO 2° pers. duel (OBJ ou POSS)}
\begin{glose}
\pfra{vous2; vos}
\end{glose}
\end{entrée}

\begin{entrée}{jòjò}{}{ⓔjòjò}
\formephonétique{ɲɟɔɲɟɔ, ndjɔɲdjɔ}
\région{GOs}
\variante{%
jòjòn
\formephonétique{ɲɟɔɲɟɔn}
\région{PA BO}}
(\domainesémantique{Fonctions naturelles humaines})
\classe{v}
\begin{glose}
\pfra{trembler (de peur, de froid, de colère)}
\end{glose}
\begin{glose}
\pfra{vibrer ; réagir (à un bruit)}
\end{glose}
\newline
\begin{exemple}
\textbf{\pnua{nu jòjò}}
\pfra{je tremble (de froid)}
\end{exemple}
\newline
\begin{exemple}
\textbf{\pnua{e jò mwa}}
\pfra{la maison tremble (sous le vent)}
\end{exemple}
\newline
\begin{exemple}
\textbf{\pnua{e jò dili}}
\pfra{la terre tremble}
\end{exemple}
\newline
\begin{exemple}
\région{BO}
\textbf{\pnua{i pa-jòni nu xo kuau}}
\pfra{le chien m'a fait sursauter}
\end{exemple}
\end{entrée}

\begin{entrée}{jòme}{}{ⓔjòme}
\formephonétique{ɲɟɔme, djɔme}
\région{GOs}
(\domainesémantique{Mouvements ou actions faits avec le corps, les bras, les mains, les pieds})
\classe{v}
\begin{glose}
\pfra{secouer}
\end{glose}
\newline
\begin{exemple}
\textbf{\pnua{e jòme-nu xo loto}}
\pfra{la voiture m'a retourné, secoué}
\end{exemple}
\newline
\begin{exemple}
\textbf{\pnua{nu pha-jòme je}}
\pfra{je l'ai secoué}
\end{exemple}
\end{entrée}

\begin{entrée}{jomûgò}{}{ⓔjomûgò}
\formephonétique{ɲɟomûgɔ, ɲdjomûgɔ}
\région{GOs}
(\domainesémantique{Oiseaux})
\classe{nom}
\begin{glose}
\pfra{grive perlée (Méliphage barré)}
\end{glose}
\nomscientifique{Phylidonyris undulata}
\end{entrée}

\begin{entrée}{jòòwe}{}{ⓔjòòwe}
\formephonétique{ɲɟɔːwe, ndjɔːwe}
\région{GOs}
(\domainesémantique{Description des objets, formes, consistance, taille})
\classe{nom}
\begin{glose}
\pfra{objets ou débris flottés}
\end{glose}
\end{entrée}

\begin{entrée}{jua}{}{ⓔjua}
\région{GOs BO}
(\domainesémantique{Santé, maladie})
\classe{nom}
\begin{glose}
\pfra{verrue}
\end{glose}
\end{entrée}

\begin{entrée}{ju-la}{}{ⓔju-la}
\région{BO}
(\domainesémantique{Fonctions intellectuelles})
\classe{v}
\begin{glose}
\pfra{compter ; nombre [BM]}
\end{glose}
\newline
\begin{exemple}
\région{BO}
\textbf{\pnua{whaya ju-la ?}}
\pfra{combien sont-ils}
\end{exemple}
\end{entrée}

\begin{entrée}{jumeã}{}{ⓔjumeã}
\formephonétique{ɲɟumeɛ̃, ɲdjumeɛ̃}
\région{GOs}
(\domainesémantique{Poissons})
\classe{nom}
\begin{glose}
\pfra{mulet (le plus gros) ou maquereau}
\end{glose}
\nomscientifique{Mugil cephalus}
\newline
\relationsémantique{Cf.}{\lien{ⓔwhaiⓗ1}{whai}}
\glosecourte{mulet (de petite taille)}
\newline
\relationsémantique{Cf.}{\lien{ⓔnaxo}{naxo}}
\glosecourte{mulet noir de rivière}
\newline
\relationsémantique{Cf.}{\lien{ⓔmene}{mene}}
\glosecourte{mulet queue bleue}
\end{entrée}

\begin{entrée}{jumo}{}{ⓔjumo}
\région{BO}
(\domainesémantique{Poissons})
\classe{nom}
\begin{glose}
\pfra{rouget [BM]}
\end{glose}
\end{entrée}

\begin{entrée}{jutri}{}{ⓔjutri}
\formephonétique{ɲɟuʈi, ɲdjuʈi}
\région{GOs}
(\domainesémantique{Corps animal})
\classe{nom}
\begin{glose}
\pfra{nageoire}
\end{glose}
\end{entrée}

\begin{entrée}{juyu}{}{ⓔjuyu}
\région{BO}
(\domainesémantique{Noms des plantes})
\classe{nom}
\begin{glose}
\pfra{cycas (Corne)}
\end{glose}
\nomscientifique{Cycas circinalis, Cycadacées}
\newline
\note{non vérifié}{général}{}
\end{entrée}

\newpage

\lettrine{k}\begin{entrée}{ka}{1}{ⓔkaⓗ1}
\région{GOs}
\variante{%
ga, xa
\région{GO(s) PA}, 
ko
\région{GO(s)}}
\newline
\sens{1}
(\domainesémantique{Conjonction})
\classe{COORD}
\begin{glose}
\pfra{et alors ; et aussi ; et en même temps}
\end{glose}
\newline
\begin{exemple}
\région{GO}
\textbf{\pnua{e trêê ka hopo}}
\pfra{il court en mangeant (et aussi mange)}
\end{exemple}
\newline
\begin{exemple}
\région{GO}
\textbf{\pnua{e trêê ka wa}}
\pfra{il court en chantant (et aussi chante)}
\end{exemple}
\newline
\sens{2}
(\domainesémantique{Conjonction})
\classe{REL}
\begin{glose}
\pfra{qui, que}
\end{glose}
\newline
\begin{exemple}
\région{GO}
\textbf{\pnua{kòlò je-na wamã xa yazaa-je Mwe}}
\pfra{chez ce vieux qui s'appelle Chouette}
\end{exemple}
\newline
\begin{exemple}
\textbf{\pnua{kòlò je-na wamã ka yala-n ni mwèn}}
\pfra{chez ce vieux qui s'appelle chouette (lit. dont le nom est chouette)}
\end{exemple}
\end{entrée}

\begin{entrée}{ka}{2}{ⓔkaⓗ2}
\région{GOs PA BO}
\variante{%
kò
\région{GO(n)}}
\classe{nom}
\newline
\sens{1}
(\domainesémantique{Découpage du temps})
\begin{glose}
\pfra{année}
\end{glose}
\newline
\begin{exemple}
\textbf{\pnua{ni ka xa pò-xè}}
\pfra{la même année}
\end{exemple}
\newline
\begin{exemple}
\région{BO}
\textbf{\pnua{we-xe ka}}
\pfra{une année}
\end{exemple}
\newline
\begin{exemple}
\région{BO}
\textbf{\pnua{nye ka hèmbun}}
\pfra{l'année d'avant}
\end{exemple}
\newline
\begin{exemple}
\région{BO}
\textbf{\pnua{nye ka hã}}
\pfra{cette année}
\end{exemple}
\newline
\begin{exemple}
\région{BO}
\textbf{\pnua{je ka akònòbòn}}
\pfra{l'année dernière}
\end{exemple}
\newline
\sens{2}
(\domainesémantique{Cultures, techniques, boutures})
\begin{glose}
\pfra{récolte d'ignames}
\end{glose}
\begin{glose}
\pfra{plante annuelle}
\end{glose}
\newline
\étymologie{
\langue{POc}
\étymon{*taqu}}
\newline
\note{kau-n [PA], kau-je [GOs]}{grammaire}{son âge}
\end{entrée}

\begin{entrée}{kã}{}{ⓔkã}
\région{GOs}
\variante{%
kãm
\région{BO}, 
kham
\région{PA}}
\classe{v}
\newline
\sens{1}
(\domainesémantique{Verbes d'action (en général)})
\begin{glose}
\pfra{érafler}
\end{glose}
\begin{glose}
\pfra{effleurer ; frôler}
\end{glose}
\begin{glose}
\pfra{glisser}
\end{glose}
\newline
\sens{2}
(\domainesémantique{Verbes de déplacement et moyens de déplacement})
\begin{glose}
\pfra{passer à côté en frôlant}
\end{glose}
\newline
\sens{3}
(\domainesémantique{Relations et interaction sociales})
\begin{glose}
\pfra{éviter}
\end{glose}
\begin{glose}
\pfra{manquer}
\end{glose}
\end{entrée}

\begin{entrée}{ka ?}{}{ⓔka ?}
\région{GOs BO}
(\domainesémantique{Interrogatifs})
\classe{v}
\begin{glose}
\pfra{qu'est-ce qu'il y a ? ; qu'est-ce qui se passe?}
\end{glose}
\begin{glose}
\pfra{pourquoi ? ; comment ?}
\end{glose}
\newline
\begin{exemple}
\textbf{\pnua{e ka ?}}
\pfra{(alors) comment est-ce ?}
\end{exemple}
\newline
\begin{exemple}
\région{BO}
\textbf{\pnua{i ka ?}}
\pfra{(alors) comment est-ce ?}
\end{exemple}
\newline
\begin{exemple}
\région{BO}
\textbf{\pnua{i cabi inu ka ?}}
\pfra{pourquoi m'a-t-il frappé ?}
\end{exemple}
\newline
\begin{exemple}
\région{PA}
\textbf{\pnua{e ka phagoo-m ? - e zo}}
\pfra{comment vas-tu ? - ça va (se dit à quelqu'un de malade)}
\end{exemple}
\newline
\begin{exemple}
\région{GO}
\textbf{\pnua{co gi xa ço ka ?}}
\pfra{pourquoi pleures-tu ?}
\end{exemple}
\newline
\begin{exemple}
\textbf{\pnua{cö ka ?}}
\pfra{comment vas-tu ?, qu'as-tu ? que t'arrive-t-il ?}
\end{exemple}
\newline
\relationsémantique{Cf.}{\lien{}{kamwêlè ?}}
\glosecourte{faire comment?}
\end{entrée}

\begin{entrée}{kaa}{1}{ⓔkaaⓗ1}
\région{GOs}
(\domainesémantique{Danses})
\classe{nom}
\begin{glose}
\pfra{battoir (en écorce pour rythmer la danse) ; tambour}
\end{glose}
\newline
\begin{sous-entrée}{ba-cabi kaa}{ⓔkaaⓗ1ⓝba-cabi kaa}
\begin{glose}
\pfra{tambour}
\end{glose}
\end{sous-entrée}
\end{entrée}

\begin{entrée}{kaa}{2}{ⓔkaaⓗ2}
\région{GOs WE}
(\domainesémantique{Mer : topographie})
\classe{nom}
\begin{glose}
\pfra{récif coralien}
\end{glose}
\end{entrée}

\begin{entrée}{kãã}{}{ⓔkãã}
\formephonétique{kɛ̃ː}
\région{GOs PA BO}
\région{PA}
\variante{%
khêê
}
(\domainesémantique{Sons, bruits})
\classe{v}
\begin{glose}
\pfra{crier ; hurler ; pousser des cris ; crier de loin}
\end{glose}
\newline
\begin{exemple}
\textbf{\pnua{e kãã}}
\pfra{il pousse des cris}
\end{exemple}
\end{entrée}

\begin{entrée}{kaabilò}{}{ⓔkaabilò}
\formephonétique{'kaːbilɔ}
\région{GOs}
(\domainesémantique{Mouvements ou actions faits avec le corps, les bras, les mains, les pieds})
\classe{v}
\begin{glose}
\pfra{tordre (se) (doigt, cheville)}
\end{glose}
\begin{glose}
\pfra{fouler (se) (pied, cheville)}
\end{glose}
\newline
\begin{exemple}
\textbf{\pnua{e kaabilò kòò-je}}
\pfra{il s'est foulé le pied}
\end{exemple}
\end{entrée}

\begin{entrée}{kaabole}{}{ⓔkaabole}
\région{GOs}
\variante{%
khaabule
\région{PA BO}, 
kaabule
\région{BO [BM]}}
(\domainesémantique{Actions liées aux éléments (liquide, fumée)})
\classe{v}
\begin{glose}
\pfra{envahir d'eau ; inonder ; déborder ; passer par-dessus}
\end{glose}
\begin{glose}
\pfra{tremper (le linge dans l'eau) ; mouiller}
\end{glose}
\newline
\begin{exemple}
\région{PA}
\textbf{\pnua{i khaabule wony}}
\pfra{le bateau a coulé}
\end{exemple}
\end{entrée}

\begin{entrée}{kaaça}{}{ⓔkaaça}
\formephonétique{kaːʒa}
\région{GOs}
(\domainesémantique{Description des objets, formes, consistance, taille})
\classe{v}
\begin{glose}
\pfra{tendu (corde) ; raide}
\end{glose}
\end{entrée}

\begin{entrée}{kããge}{}{ⓔkããge}
\région{GOs}
\variante{%
kããgèn
\région{BO PA}}
(\domainesémantique{Fonctions intellectuelles})
\classe{v}
\begin{glose}
\pfra{croire ; espérer}
\end{glose}
\newline
\begin{sous-entrée}{me-kããgen}{ⓔkããgeⓝme-kããgen}
\région{BO}
\begin{glose}
\pfra{croyance; foi}
\end{glose}
\end{sous-entrée}
\end{entrée}

\begin{entrée}{kaageen}{}{ⓔkaageen}
\région{BO}
(\domainesémantique{Relations et interaction sociales})
\classe{v}
\begin{glose}
\pfra{obéir [BM]}
\end{glose}
\end{entrée}

\begin{entrée}{kaaje}{}{ⓔkaaje}
\région{BO}
\variante{%
kayè
}
(\domainesémantique{Parties de plantes})
\classe{nom}
\begin{glose}
\pfra{jonc à corbeille [Corne]}
\end{glose}
\newline
\note{non vérifié}{général}{}
\end{entrée}

\begin{entrée}{kaal}{}{ⓔkaal}
\région{PA}
(\domainesémantique{Remèdes, médecine
, Actions liées aux plantes})
\classe{v}
\begin{glose}
\pfra{cueillir des feuilles et herbes (pour faire des médicaments)}
\end{glose}
\end{entrée}

\begin{entrée}{kaale}{}{ⓔkaale}
\région{GOs PA BO}
(\domainesémantique{Modalité, verbes modaux})
\classe{v}
\begin{glose}
\pfra{laisser ; permettre}
\end{glose}
\newline
\begin{exemple}
\textbf{\pnua{kaale kee-jö ènè !}}
\pfra{laisse ton panier ici!}
\end{exemple}
\newline
\begin{exemple}
\textbf{\pnua{kaale pu nu khôbwe-ayuni mwã ne ogine teen i ã}}
\pfra{permettez-moi de dire comme cela que notre journée prend fin}
\end{exemple}
\newline
\note{kaalexa}{grammaire}{}
\end{entrée}

\begin{entrée}{kããle}{}{ⓔkããle}
\formephonétique{kɛ̃ːle}
\région{GOs BO PA}
\classe{v}
\newline
\sens{1}
(\domainesémantique{Remèdes, médecine})
\begin{glose}
\pfra{soigner ; prendre soin de}
\end{glose}
\begin{glose}
\pfra{guérir}
\end{glose}
\newline
\sens{2}
(\domainesémantique{Société})
\begin{glose}
\pfra{élever (enfants, animaux)}
\end{glose}
\begin{glose}
\pfra{s'occuper (de qqn)}
\end{glose}
\newline
\sens{3}
(\domainesémantique{Soins du corps})
\begin{glose}
\pfra{conserver}
\end{glose}
\newline
\relationsémantique{Cf.}{\lien{ⓔyue}{yue}}
\glosecourte{adopter, garder}
\end{entrée}

\begin{entrée}{kaalu}{}{ⓔkaalu}
\région{GOs BO PA}
\newline
\sens{1}
(\domainesémantique{Verbes de mouvement})
\classe{v}
\begin{glose}
\pfra{tomber (d'un animé)}
\end{glose}
\newline
\begin{exemple}
\région{GO}
\textbf{\pnua{e kaalu ã ẽnõ}}
\pfra{l'enfant est tombé !}
\end{exemple}
\newline
\begin{exemple}
\région{PA}
\textbf{\pnua{phaa-kaalu-je}}
\pfra{fais-le tomber !}
\end{exemple}
\newline
\sens{2}
(\domainesémantique{Verbes d'action (en général)})
\classe{v ; n}
\begin{glose}
\pfra{accident ; avoir un accident}
\end{glose}
\newline
\begin{exemple}
\région{GO}
\textbf{\pnua{e kaalu na bwa loto}}
\pfra{il a eu un accident}
\end{exemple}
\newline
\begin{exemple}
\région{BO}
\textbf{\pnua{i kaalu na bwa ce}}
\pfra{il est tombé de l'arbre}
\end{exemple}
\newline
\sens{3}
(\domainesémantique{Topographie})
\classe{nom}
\begin{glose}
\pfra{creux ; dépression sur un terrain [BO]}
\end{glose}
\newline
\relationsémantique{Cf.}{\lien{ⓔthrõbo}{thrõbo}}
\glosecourte{tomber (chose), naître}
\end{entrée}

\begin{entrée}{kaaluni}{}{ⓔkaaluni}
\région{GOs PA}
(\domainesémantique{Coutumes, dons coutumiers})
\classe{v}
\begin{glose}
\pfra{enlever les dons lors des cérémonies coutumières}
\end{glose}
\newline
\begin{exemple}
\textbf{\pnua{ba-kaaluni ena kui}}
\pfra{geste pour enlever les ignames}
\end{exemple}
\end{entrée}

\begin{entrée}{kaamuda}{}{ⓔkaamuda}
\région{PA}
(\domainesémantique{Outils})
\classe{nom}
\begin{glose}
\pfra{hache (type de)}
\end{glose}
\end{entrée}

\begin{entrée}{kaamwene ?}{}{ⓔkaamwene ?}
\formephonétique{kaː'mweɳe}
\région{GOs}
\variante{%
kamwelè ?
\région{GO(s) BO}, 
kamweli
\région{BO}}
(\domainesémantique{Interrogatifs})
\classe{v.INT}
\begin{glose}
\pfra{comment faire ? ; que faire ? (être comment?) ; faire ainsi}
\end{glose}
\newline
\begin{exemple}
\textbf{\pnua{jö kaamwe-ne ?}}
\pfra{comment as-tu fait ?}
\end{exemple}
\newline
\begin{exemple}
\textbf{\pnua{jö kamwelè tiiwo i nu ?}}
\pfra{qu'est-ce que tu as fait avec mon livre ?}
\end{exemple}
\newline
\begin{exemple}
\textbf{\pnua{jò kamweli me nee-xo ?}}
\pfra{comment avez-vous (deux) fait pour le faire ?}
\end{exemple}
\newline
\relationsémantique{Cf.}{\lien{}{ka?}}
\glosecourte{être comment ?}
\end{entrée}

\begin{entrée}{kaamweni}{}{ⓔkaamweni}
\formephonétique{kaː'mweɳi}
\région{GOs}
(\domainesémantique{Fonctions intellectuelles})
\classe{v}
\begin{glose}
\pfra{comprendre ; sage}
\end{glose}
\newline
\begin{exemple}
\région{GO}
\textbf{\pnua{jö trõne kamweni ?}}
\pfra{tu as bien compris ?}
\end{exemple}
\newline
\begin{exemple}
\région{GO}
\textbf{\pnua{jö hine kamweni ?}}
\pfra{tu as bien compris ?}
\end{exemple}
\newline
\begin{exemple}
\région{GO}
\textbf{\pnua{jö kamweni-zooni ?}}
\pfra{tu as bien compris ?}
\end{exemple}
\newline
\begin{exemple}
\région{GO}
\textbf{\pnua{kavwo nu trõne kaamweni me-vhaa i la}}
\pfra{je ne comprends pas leur façon de parler}
\end{exemple}
\end{entrée}

\begin{entrée}{kaano}{}{ⓔkaano}
\région{GOs WEM}
(\domainesémantique{Jeux divers})
\classe{nom}
\begin{glose}
\pfra{jeu de propulsion}
\end{glose}
\newline
\note{jeu avec deux bâtons: on tape sur une extrémité du bâton avec l'autre bâton et le gagnant est celui qui l'a propulsé le plus loin}{glose}{}
\end{entrée}

\begin{entrée}{ka-avè}{}{ⓔka-avè}
\région{BO}
(\domainesémantique{Société})
\classe{nom}
\begin{glose}
\pfra{compagnon de naissance [Corne]}
\end{glose}
\newline
\note{non vérifié}{général}{}
\end{entrée}

\begin{entrée}{kaavwu}{}{ⓔkaavwu}
\formephonétique{kaːβu}
\région{GOs}
\variante{%
kapu
\région{GO(s) vx}, 
kaawuun
\région{PA BO}}
(\domainesémantique{Organisation sociale})
\classe{nom}
\begin{glose}
\pfra{gardien ; maître ; propriétaire}
\end{glose}
\newline
\begin{exemple}
\région{PA}
\textbf{\pnua{kaawu-n}}
\pfra{son maître}
\end{exemple}
\newline
\begin{exemple}
\région{BO}
\textbf{\pnua{kaawun i nu nye go}}
\pfra{cette radio est à moi}
\end{exemple}
\newline
\begin{sous-entrée}{kavwu waiya}{ⓔkaavwuⓝkavwu waiya}
\begin{glose}
\pfra{chef de guerre}
\end{glose}
\end{sous-entrée}
\newline
\begin{sous-entrée}{kaawu dili}{ⓔkaavwuⓝkaawu dili}
\région{GO PA WEM}
\begin{glose}
\pfra{le maître du sol}
\end{glose}
\end{sous-entrée}
\end{entrée}

\begin{entrée}{kaavwu dili}{}{ⓔkaavwu dili}
\formephonétique{kaːβu dili}
\région{GOs}
(\domainesémantique{Organisation sociale})
\classe{nom}
\begin{glose}
\pfra{protecteur du sol (nom d'un clan)}
\end{glose}
\end{entrée}

\begin{entrée}{kaaxo}{}{ⓔkaaxo}
\région{GOsWE}
(\domainesémantique{Parenté})
\classe{nom}
\begin{glose}
\pfra{cousin (terme respectueux d'appellation ou désignation aux personnes plus agées)}
\end{glose}
\newline
\begin{exemple}
\textbf{\pnua{li pe-kaaxo}}
\pfra{ils sont cousins}
\end{exemple}
\end{entrée}

\begin{entrée}{kaayang}{}{ⓔkaayang}
\région{PA BO [Corne]}
(\domainesémantique{Description des objets, formes, consistance, taille})
\classe{v}
\begin{glose}
\pfra{raide (être)}
\end{glose}
\begin{glose}
\pfra{tendu}
\end{glose}
\begin{glose}
\pfra{raide mort}
\end{glose}
\end{entrée}

\begin{entrée}{kaba}{}{ⓔkaba}
\région{GOs}
(\domainesémantique{Verbes d'action (en général)})
\classe{v}
\begin{glose}
\pfra{protéger (se)}
\end{glose}
\newline
\begin{exemple}
\textbf{\pnua{nu kaba}}
\pfra{je me protège}
\end{exemple}
\end{entrée}

\begin{entrée}{kaba we}{}{ⓔkaba we}
\région{WEM WE BO}
(\domainesémantique{Actions liées aux éléments (liquide, fumée)})
\classe{v}
\begin{glose}
\pfra{puiser ; prendre de l'eau (avec un petit récipient)}
\end{glose}
\begin{glose}
\pfra{écoper ;}
\end{glose}
\begin{glose}
\pfra{retirer d'une marmite (surtout du liquide)}
\end{glose}
\newline
\begin{exemple}
\région{BO}
\textbf{\pnua{i kaba we}}
\pfra{elle puise de l'eau}
\end{exemple}
\newline
\relationsémantique{Cf.}{\lien{ⓔtröi}{tröi}}
\glosecourte{puiser}
\newline
\relationsémantique{Cf.}{\lien{ⓔii}{ii}}
\glosecourte{servir, sortir de la marmite (riz)}
\end{entrée}

\begin{entrée}{kabu}{}{ⓔkabu}
\formephonétique{'kabu}
\région{GOs}
\variante{%
kabun
\région{PA BO}}
(\domainesémantique{Religion, représentations religieuses})
\classe{v}
\begin{glose}
\pfra{interdit ; sacré}
\end{glose}
\newline
\begin{sous-entrée}{mo-xabu}{ⓔkabuⓝmo-xabu}
\région{GO}
\begin{glose}
\pfra{église, temple}
\end{glose}
\end{sous-entrée}
\newline
\begin{sous-entrée}{mwa-kabu}{ⓔkabuⓝmwa-kabu}
\begin{glose}
\pfra{église, temple}
\end{glose}
\end{sous-entrée}
\newline
\begin{sous-entrée}{pwe-kabun}{ⓔkabuⓝpwe-kabun}
\région{PA}
\begin{glose}
\pfra{sorcier (celui qui peut faire mourir)}
\end{glose}
\newline
\begin{exemple}
\région{GO}
\textbf{\pnua{e kabu na mõ pweeni nò-ni}}
\pfra{il nous est interdit de pêcher ce poisson-là}
\end{exemple}
\newline
\begin{exemple}
\région{GO}
\textbf{\pnua{õ-tru xa e kaò ni kabu-è}}
\pfra{cela fait deux fois que cela déborde (rivière) dans la semaine}
\end{exemple}
\end{sous-entrée}
\newline
\begin{sous-entrée}{camwa kabun}{ⓔkabuⓝcamwa kabun}
\région{BO}
\begin{glose}
\pfra{banane chef}
\end{glose}
\newline
\note{v.t. kaabuni [ka:'buni]}{grammaire}{}
\end{sous-entrée}
\newline
\étymologie{
\langue{POc}
\étymon{*tampu}
\glosecourte{interdit}}
\end{entrée}

\begin{entrée}{kãbwa}{}{ⓔkãbwa}
\formephonétique{kɛ̃bwa}
\région{GOs BO}
\variante{%
kabwa
\région{PA}}
(\domainesémantique{Religion, représentations religieuses})
\classe{nom}
\begin{glose}
\pfra{génie forestier ; esprit ; ancêtre}
\end{glose}
\begin{glose}
\pfra{totem ; esprit bienfaisant ; dieu}
\end{glose}
\newline
\relationsémantique{Cf.}{\lien{}{kãbwa-hili}}
\glosecourte{esprit (des forêts) (hili: se retirer comme un serpent dans un trou)}
\end{entrée}

\begin{entrée}{kãbwaço}{}{ⓔkãbwaço}
\formephonétique{kɛ̃'bwaʒo, kɛ̃'bwadʒo}
\région{GOs}
\variante{%
kabwayòl
\région{PA BO}, 
kabwoyòl, kaboyòl
\région{BO[BM]}}
(\domainesémantique{Poissons})
\classe{nom}
\begin{glose}
\pfra{requin}
\end{glose}
\end{entrée}

\begin{entrée}{kaça}{}{ⓔkaça}
\formephonétique{kaʒa}
\région{GOs}
\variante{%
kaya
\région{PA BO}}
(\domainesémantique{Localisation
, Noms locatifs})
\classe{n.LOC}
\begin{glose}
\pfra{derrière}
\end{glose}
\begin{glose}
\pfra{arrière (qqch, inanimé)}
\end{glose}
\begin{glose}
\pfra{après}
\end{glose}
\newline
\begin{exemple}
\région{GO}
\textbf{\pnua{e phu kaça-je}}
\pfra{il vole derrière lui}
\end{exemple}
\newline
\begin{sous-entrée}{kaca ko-n}{ⓔkaçaⓝkaca ko-n}
\région{WEM}
\begin{glose}
\pfra{talon (arrière du pied)}
\end{glose}
\end{sous-entrée}
\newline
\begin{sous-entrée}{kaça mwa}{ⓔkaçaⓝkaça mwa}
\région{GO}
\begin{glose}
\pfra{arrière de la maison ; le nord}
\end{glose}
\end{sous-entrée}
\newline
\begin{sous-entrée}{kaça wõ}{ⓔkaçaⓝkaça wõ}
\région{GO}
\begin{glose}
\pfra{la poupe du bateau}
\end{glose}
\end{sous-entrée}
\newline
\begin{sous-entrée}{kaça hovwo}{ⓔkaçaⓝkaça hovwo}
\région{GO}
\begin{glose}
\pfra{après-midi (lit. après le déjeuner)}
\end{glose}
\end{sous-entrée}
\newline
\begin{sous-entrée}{kaya hovho}{ⓔkaçaⓝkaya hovho}
\région{PA}
\begin{glose}
\pfra{après-midi (lit. après le déjeuner)}
\end{glose}
\end{sous-entrée}
\newline
\begin{sous-entrée}{kaça mõnõ}{ⓔkaçaⓝkaça mõnõ}
\begin{glose}
\pfra{après-demain}
\end{glose}
\newline
\relationsémantique{Ant.}{\lien{}{me-mwa [GOs]}}
\glosecourte{le devant de la maison ; le sud}
\end{sous-entrée}
\end{entrée}

\begin{entrée}{kaça-kabu}{}{ⓔkaça-kabu}
\région{GOs}
\variante{%
wa-kabun
\région{PA BO}}
(\domainesémantique{Jours})
\classe{nom}
\begin{glose}
\pfra{lundi (lit. après le sacré)}
\end{glose}
\end{entrée}

\begin{entrée}{kaça-kò}{}{ⓔkaça-kò}
\formephonétique{ka'ʒakɔ}
\région{GOs}
(\domainesémantique{Corps humain})
\classe{nom}
\begin{glose}
\pfra{talon (lit. arrière du pied)}
\end{glose}
\end{entrée}

\begin{entrée}{kaça mwa}{}{ⓔkaça mwa}
\région{GOs}
\classe{n.LOC}
\newline
\sens{1}
(\domainesémantique{Types de maison, architecture de la maison})
\begin{glose}
\pfra{arrière de la maison}
\end{glose}
\newline
\sens{2}
(\domainesémantique{Directions})
\begin{glose}
\pfra{nord du pays}
\end{glose}
\end{entrée}

\begin{entrée}{kaci}{}{ⓔkaci}
\région{PA BO}
(\domainesémantique{Oiseaux})
\classe{nom}
\begin{glose}
\pfra{émouchet bleu ; faucon ; buse blanche et noire}
\end{glose}
\nomscientifique{Accipiter haplochrous (Accipitridés)}
\end{entrée}

\begin{entrée}{kacia}{}{ⓔkacia}
\région{BO [BM]}
(\domainesémantique{Noms des plantes})
\classe{nom}
\begin{glose}
\pfra{lantana}
\end{glose}
\newline
\emprunt{acacia (FR)}
\end{entrée}

\begin{entrée}{kaço}{}{ⓔkaço}
\formephonétique{kadʒo}
\région{GOs}
\région{WEM WE PA BO}
\variante{%
kayòl
}
(\domainesémantique{Habitat})
\classe{nom}
\begin{glose}
\pfra{cimetière}
\end{glose}
\end{entrée}

\begin{entrée}{kae}{}{ⓔkae}
\région{GOs PA WEM WE BO}
\classe{v.t.}
\newline
\sens{1}
(\domainesémantique{Relations et interaction sociales})
\begin{glose}
\pfra{enlever ; ravir (femme)}
\end{glose}
\newline
\sens{2}
(\domainesémantique{Relations et interaction sociales})
\begin{glose}
\pfra{garder pour soi ; mettre en sûreté}
\end{glose}
\newline
\begin{exemple}
\région{GO}
\textbf{\pnua{li pe-kae nye loto}}
\pfra{ils se disputent cette voiture}
\end{exemple}
\newline
\begin{exemple}
\région{GO}
\textbf{\pnua{e kae je}}
\pfra{il l'a enlevée}
\end{exemple}
\newline
\begin{exemple}
\région{PA}
\textbf{\pnua{i kae je ku Teã-ma}}
\pfra{le chef l'a enlevée}
\end{exemple}
\newline
\begin{exemple}
\région{PA}
\textbf{\pnua{i kae na hii-n}}
\pfra{il le lui a enlevé des mains}
\end{exemple}
\end{entrée}

\begin{entrée}{kaè}{}{ⓔkaè}
\région{GOs PA BO}
(\domainesémantique{Cucurbitacées})
\classe{nom}
\begin{glose}
\pfra{citrouille ; gourde}
\end{glose}
\nomscientifique{Crescentia cujete L.}
\newline
\begin{sous-entrée}{kaè mii}{ⓔkaèⓝkaè mii}
\begin{glose}
\pfra{citrouille rouge}
\end{glose}
\end{sous-entrée}
\newline
\begin{sous-entrée}{kaè thaxilò}{ⓔkaèⓝkaè thaxilò}
\begin{glose}
\pfra{pastèque}
\end{glose}
\end{sous-entrée}
\newline
\begin{sous-entrée}{kaè phai}{ⓔkaèⓝkaè phai}
\begin{glose}
\pfra{citrouille}
\end{glose}
\end{sous-entrée}
\newline
\begin{sous-entrée}{kaè pulo}{ⓔkaèⓝkaè pulo}
\begin{glose}
\pfra{citrouille blanche}
\end{glose}
\end{sous-entrée}
\newline
\begin{sous-entrée}{kaè pwaalii}{ⓔkaèⓝkaè pwaalii}
\begin{glose}
\pfra{citrouille longue}
\end{glose}
\end{sous-entrée}
\end{entrée}

\begin{entrée}{kaè cèni}{}{ⓔkaè cèni}
\formephonétique{'ka.ɛ 'cɛni}
\région{GOs}
(\domainesémantique{Cucurbitacées})
\classe{nom}
\begin{glose}
\pfra{citrouille (lit. citrouille comestible)}
\end{glose}
\nomscientifique{Cucurbita pepo L. (Cucurbitacées)}
\end{entrée}

\begin{entrée}{kae mõõxi}{}{ⓔkae mõõxi}
\région{GOs PA}
(\domainesémantique{Verbes d'action (en général)})
\classe{v}
\begin{glose}
\pfra{sauver qqn (= ravir à la mort) ; sauver ; préserver (vie)}
\end{glose}
\newline
\begin{exemple}
\région{PA}
\textbf{\pnua{e kae moxi-ny}}
\pfra{il m'a sauvé}
\end{exemple}
\newline
\begin{exemple}
\région{GO}
\textbf{\pnua{e kae mõõxi-nu}}
\pfra{il m'a sauvé la vie}
\end{exemple}
\end{entrée}

\begin{entrée}{kaen}{}{ⓔkaen}
\région{PA}
(\domainesémantique{Mollusques})
\classe{nom}
\begin{glose}
\pfra{gastéropode d'eau douce}
\end{glose}
\end{entrée}

\begin{entrée}{kaè thraxilò}{}{ⓔkaè thraxilò}
\formephonétique{'ka.ɛ 'ʈhaɣilɔ}
\région{GOs}
\variante{%
kaè kuni
\région{GO(s)}}
(\domainesémantique{Cucurbitacées})
\classe{nom}
\begin{glose}
\pfra{pastèque}
\end{glose}
\nomscientifique{Citrullus vulgaris Schrad.}
\end{entrée}

\begin{entrée}{kae wòdo}{}{ⓔkae wòdo}
\région{GOs}
\variante{%
kae wedo
\région{PA}}
(\domainesémantique{Coutumes, dons coutumiers})
\classe{v}
\begin{glose}
\pfra{suivre les usages}
\end{glose}
\newline
\begin{exemple}
\région{PA}
\textbf{\pnua{mwa kae wedo}}
\pfra{nous suivons les usages}
\end{exemple}
\end{entrée}

\begin{entrée}{kafaa}{}{ⓔkafaa}
\région{GOs}
(\domainesémantique{Insectes})
\classe{nom}
\begin{glose}
\pfra{cafard}
\end{glose}
\newline
\emprunt{cafard (FR)}
\end{entrée}

\begin{entrée}{kafe}{}{ⓔkafe}
\région{GOs}
\variante{%
kape
\région{PA BO WEM WE}}
(\domainesémantique{Aliments, alimentation})
\classe{nom}
\begin{glose}
\pfra{café}
\end{glose}
\newline
\begin{exemple}
\région{WEM}
\textbf{\pnua{li thu kape}}
\pfra{elles font du café}
\end{exemple}
\newline
\emprunt{café (FR)}
\end{entrée}

\begin{entrée}{kãgoo}{}{ⓔkãgoo}
\région{GOs}
(\domainesémantique{Objets et meubles de la maison})
\classe{nom}
\begin{glose}
\pfra{berceau (pour bercer un bébé)}
\end{glose}
\end{entrée}

\begin{entrée}{kãgu}{}{ⓔkãgu}
\formephonétique{kɛ̃ŋgu}
\région{GOs}
\variante{%
kãgun
\région{BO PA}}
\classe{v ; n}
\newline
\sens{1}
(\domainesémantique{Religion, représentations religieuses})
\begin{glose}
\pfra{esprit ; âme}
\end{glose}
\newline
\sens{2}
(\domainesémantique{Sentiments})
\begin{glose}
\pfra{suivre qqn comme son ombre (qqn à qui on est très attaché)}
\end{glose}
\newline
\begin{exemple}
\textbf{\pnua{e kãgu-je}}
\pfra{il le suit partout (comme son ombre, lit. c'est son ombre)}
\end{exemple}
\newline
\étymologie{
\langue{POc}
\étymon{*tanu, *kanitu}}
\end{entrée}

\begin{entrée}{kãgu-hênû}{}{ⓔkãgu-hênû}
\région{GOs}
(\domainesémantique{Description des objets, formes, consistance, taille})
\classe{nom}
\begin{glose}
\pfra{négatif d'une photo}
\end{glose}
\newline
\relationsémantique{Cf.}{\lien{}{hênû}}
\glosecourte{photo, image}
\end{entrée}

\begin{entrée}{kãgu-mwã}{}{ⓔkãgu-mwã}
\région{GOs}
(\domainesémantique{Religion, représentations religieuses})
\classe{nom}
\begin{glose}
\pfra{nain (lit. esprit de la maison)}
\end{glose}
\end{entrée}

\begin{entrée}{kai}{1}{ⓔkaiⓗ1}
\région{GOs PA}
(\domainesémantique{Prépositions})
\classe{ASSOC}
\begin{glose}
\pfra{avec (lit. derrière + animé)}
\end{glose}
\newline
\begin{exemple}
\textbf{\pnua{a kai-nu}}
\pfra{viens avec moi}
\end{exemple}
\newline
\begin{exemple}
\région{PA}
\textbf{\pnua{i a kai-n}}
\pfra{il part avec lui}
\end{exemple}
\newline
\begin{exemple}
\textbf{\pnua{ti nye u-da/ã-da kai-je ni mwã ?}}
\pfra{avec qui est-il entré dans la maison ?}
\end{exemple}
\newline
\begin{exemple}
\textbf{\pnua{e a-xai kêê-je mã õã-je}}
\pfra{elle est partie avec son père et sa mère}
\end{exemple}
\end{entrée}

\begin{entrée}{kai}{2}{ⓔkaiⓗ2}
\région{GOsBO PA}
\classe{n ; n.LOC}
\newline
\sens{1}
(\domainesémantique{Corps humain})
\begin{glose}
\pfra{dos}
\end{glose}
\newline
\sens{2}
(\domainesémantique{Localisation
, Noms locatifs})
\begin{glose}
\pfra{derrière}
\end{glose}
\begin{glose}
\pfra{dans le dos}
\end{glose}
\begin{glose}
\pfra{après (qqn, animé)}
\end{glose}
\newline
\begin{exemple}
\région{GO}
\textbf{\pnua{ge je kai-nu}}
\pfra{il est juste derrière moi, dans mon dos; après moi}
\end{exemple}
\newline
\begin{exemple}
\région{PA BO}
\textbf{\pnua{kai-ny}}
\pfra{mon dos; après moi}
\end{exemple}
\newline
\begin{sous-entrée}{mwãju kai}{ⓔkaiⓗ2ⓢ2ⓝmwãju kai}
\région{GO PA}
\begin{glose}
\pfra{contre-don}
\end{glose}
\end{sous-entrée}
\newline
\étymologie{
\langue{POc}
\étymon{*taku}}
\end{entrée}

\begin{entrée}{kãjawa}{}{ⓔkãjawa}
\formephonétique{kɛ̃ɲɟawa, kɛ̃ɲdjawa}
\région{GOs}
\variante{%
kajaa
\région{BO (Dubois)}}
(\domainesémantique{Ignames})
\classe{nom}
\begin{glose}
\pfra{igname (clone)}
\end{glose}
\end{entrée}

\begin{entrée}{kaje}{}{ⓔkaje}
\région{BO}
(\domainesémantique{Taros})
\classe{nom}
\begin{glose}
\pfra{taro (clone) de terrain sec (Dubois)}
\end{glose}
\end{entrée}

\begin{entrée}{kake}{}{ⓔkake}
\région{GO}
(\domainesémantique{Conjonction})
\classe{CNJ}
\begin{glose}
\pfra{quoique}
\end{glose}
\end{entrée}

\begin{entrée}{kakorola}{}{ⓔkakorola}
\région{PA BO [Corne]}
(\domainesémantique{Insectes})
\classe{nom}
\begin{glose}
\pfra{cancrelat ; cafard}
\end{glose}
\newline
\emprunt{cancrelat (FR)}
\end{entrée}

\begin{entrée}{kakulinãgu}{}{ⓔkakulinãgu}
\région{GOs BO}
(\domainesémantique{Oiseaux})
\classe{nom}
\begin{glose}
\pfra{coucou à éventail, Gammier (noir, petit)}
\end{glose}
\nomscientifique{Cacomantis pyrrhophanus pyrrhophanus}
\end{entrée}

\begin{entrée}{kala}{}{ⓔkala}
(\domainesémantique{Verbes de déplacement et moyens de déplacement})
\classe{v}
\begin{glose}
\pfra{sauver (se)}
\end{glose}
\begin{glose}
\pfra{partir ; quitter}
\end{glose}
\begin{glose}
\pfra{fuir du mauvais côté ; prendre la fuite}
\end{glose}
\newline
\begin{exemple}
\région{GOs PA BO}
\textbf{\pnua{va kala-du mwa}}
\pfra{nous repartons en bas maintenant}
\end{exemple}
\newline
\begin{exemple}
\région{PA}
\textbf{\pnua{e kala gèè i je}}
\pfra{il laisse sa grand-mère}
\end{exemple}
\end{entrée}

\begin{entrée}{kalanden}{}{ⓔkalanden}
\région{PA BO}
(\domainesémantique{Caractéristiques et propriétés des personnes})
\classe{v}
\begin{glose}
\pfra{négligent ; insouciant}
\end{glose}
\end{entrée}

\begin{entrée}{kalãnge}{}{ⓔkalãnge}
\formephonétique{ka'lãŋe}
\région{GOs}
\variante{%
kala-kagee
\région{PA}}
(\domainesémantique{Verbes d'action (en général)})
\classe{v}
\begin{glose}
\pfra{laisser ; quitter}
\end{glose}
\begin{glose}
\pfra{abandonner}
\end{glose}
\newline
\begin{exemple}
\textbf{\pnua{nu kalãnge-we}}
\pfra{je vous quitte}
\end{exemple}
\end{entrée}

\begin{entrée}{kale kou}{}{ⓔkale kou}
\région{BO}
(\domainesémantique{Marées})
\classe{nom}
\begin{glose}
\pfra{marée d'équinoxe [Corne]}
\end{glose}
\newline
\note{non vérifié}{général}{}
\end{entrée}

\begin{entrée}{kaleva}{}{ⓔkaleva}
\région{PA BO [BM]}
\région{BO}
\variante{%
kaleba
}
(\domainesémantique{Description des objets, formes, consistance, taille})
\classe{v}
\begin{glose}
\pfra{plat}
\end{glose}
\begin{glose}
\pfra{aplatir ; aplati}
\end{glose}
\newline
\begin{exemple}
\région{BO}
\textbf{\pnua{i kaleva nye pa}}
\pfra{ce caillou est plat}
\end{exemple}
\end{entrée}

\begin{entrée}{kalò}{}{ⓔkalò}
\région{GOs}
(\domainesémantique{Jeux divers})
\classe{nom}
\begin{glose}
\pfra{grosse bille}
\end{glose}
\end{entrée}

\begin{entrée}{kalòya}{}{ⓔkalòya}
\région{GOs}
(\domainesémantique{Jeux divers})
\classe{v ; n}
\begin{glose}
\pfra{corde à sauter ; sauter à la corde}
\end{glose}
\newline
\begin{exemple}
\textbf{\pnua{e coo kalòya}}
\pfra{elle saute à la corde}
\end{exemple}
\newline
\begin{exemple}
\textbf{\pnua{e paò kalòya}}
\pfra{elle fait tourner la corde}
\end{exemple}
\end{entrée}

\begin{entrée}{kãmakã}{}{ⓔkãmakã}
\formephonétique{kɛ̃'makɛ̃}
\région{GOs}
(\domainesémantique{Caractéristiques et propriétés des personnes})
\classe{v}
\begin{glose}
\pfra{bizarre ; distrait}
\end{glose}
\newline
\begin{sous-entrée}{a-kamakã}{ⓔkãmakãⓝa-kamakã}
\begin{glose}
\pfra{un distrait}
\end{glose}
\end{sous-entrée}
\newline
\begin{sous-entrée}{e pe-kamakã}{ⓔkãmakãⓝe pe-kamakã}
\begin{glose}
\pfra{il est distrait}
\end{glose}
\end{sous-entrée}
\end{entrée}

\begin{entrée}{kamwe ?}{}{ⓔkamwe ?}
\région{GOs BO}
(\domainesémantique{Interrogatifs})
\classe{v.INT}
\begin{glose}
\pfra{que faire ?}
\end{glose}
\newline
\begin{exemple}
\textbf{\pnua{jò kamwe-je ?}}
\pfra{qu'est-ce que vous lui avez fait ?}
\end{exemple}
\newline
\relationsémantique{Cf.}{\lien{}{è ka?}}
\glosecourte{comment ça va ?, qu'est-ce qui se passe ?}
\end{entrée}

\begin{entrée}{kanang}{}{ⓔkanang}
\région{PA}
(\domainesémantique{Fonctions naturelles des animaux})
\classe{v}
\begin{glose}
\pfra{coïter (animaux)}
\end{glose}
\end{entrée}

\begin{entrée}{kanii}{}{ⓔkanii}
\formephonétique{kaɳiː}
\région{GOs}
(\domainesémantique{Oiseaux})
\classe{nom}
\begin{glose}
\pfra{canard (importé)}
\end{glose}
\end{entrée}

\begin{entrée}{kaò}{}{ⓔkaò}
\formephonétique{kaɔ}
\région{GOs PA BO}
(\domainesémantique{Eau})
\classe{nom}
\begin{glose}
\pfra{inondation}
\end{glose}
\newline
\begin{exemple}
\région{PA}
\textbf{\pnua{kò-mõnu kaò}}
\pfra{la saison des pluies est proche}
\end{exemple}
\end{entrée}

\begin{entrée}{kaobi}{}{ⓔkaobi}
\région{GOs PA}
(\domainesémantique{Mouvements ou actions avec la tête, les yeux, la bouche})
\classe{v}
\begin{glose}
\pfra{casser avec les dents ; écraser avec les dents (bonbon, qqch de dur)}
\end{glose}
\newline
\relationsémantique{Cf.}{\lien{ⓔcubii}{cubii}}
\glosecourte{déchirer avec les dents}
\end{entrée}

\begin{entrée}{ka-poxe}{}{ⓔka-poxe}
\région{GOs}
(\domainesémantique{Distributifs})
\classe{DISTR}
\begin{glose}
\pfra{un par un}
\end{glose}
\newline
\begin{exemple}
\textbf{\pnua{hivwi ka-poxe!}}
\pfra{ramasse-les un par un !}
\end{exemple}
\end{entrée}

\begin{entrée}{karava}{}{ⓔkarava}
\région{GO}
(\domainesémantique{Navigation})
\classe{nom}
\begin{glose}
\pfra{pirogue}
\end{glose}
\newline
\note{non vérifié}{général}{}
\end{entrée}

\begin{entrée}{kari}{}{ⓔkari}
\région{PA}
\classe{v}
\newline
\sens{1}
(\domainesémantique{Mouvements ou actions faits avec le corps, les bras, les mains, les pieds})
\begin{glose}
\pfra{cacher (se)}
\end{glose}
\begin{glose}
\pfra{rester à l'écart}
\end{glose}
\newline
\sens{3}
(\domainesémantique{Caractéristiques et propriétés des animaux})
\begin{glose}
\pfra{craintif (animaux surtout)}
\end{glose}
\newline
\begin{sous-entrée}{kari mèni}{ⓔkariⓢ3ⓝkari mèni}
\begin{glose}
\pfra{oiseau craintif}
\end{glose}
\end{sous-entrée}
\newline
\begin{sous-entrée}{kari ègu}{ⓔkariⓢ3ⓝkari ègu}
\begin{glose}
\pfra{personne craintive, qui reste à l'écart}
\end{glose}
\end{sous-entrée}
\end{entrée}

\begin{entrée}{karolia}{}{ⓔkarolia}
\région{BO}
(\domainesémantique{Jeux divers
, Verbes de mouvement})
\classe{v}
\begin{glose}
\pfra{sauter à la corde [Corne]}
\end{glose}
\end{entrée}

\begin{entrée}{karòò}{}{ⓔkaròò}
\région{GOs WEM}
\variante{%
kharo
\région{BO}}
(\domainesémantique{Coraux})
\classe{nom}
\begin{glose}
\pfra{corail ; chaux [BM]}
\end{glose}
\newline
\relationsémantique{Cf.}{\lien{ⓔvaxaròò}{vaxaròò}}
\glosecourte{patate de corail}
\end{entrée}

\begin{entrée}{katòli}{}{ⓔkatòli}
\région{GOs}
\variante{%
katòlik
\région{PA}}
(\domainesémantique{Religion, représentations religieuses})
\classe{nom}
\begin{glose}
\pfra{catholique}
\end{glose}
\newline
\relationsémantique{Cf.}{\lien{}{mõ-mãxi [GOs]}}
\glosecourte{protestants, prier}
\end{entrée}

\begin{entrée}{katre}{}{ⓔkatre}
\formephonétique{kaːʈe ; kaːɽe}
\région{GOs}
\variante{%
kaarèng
\région{PA BO}, 
katèng
\région{BO}, 
katèk
\région{BO [Corne]}}
(\domainesémantique{Phénomènes atmosphériques et naturels})
\classe{nom}
\begin{glose}
\pfra{brouillard}
\end{glose}
\end{entrée}

\begin{entrée}{katrepa}{}{ⓔkatrepa}
\formephonétique{kaʈepa}
\région{GOs}
(\domainesémantique{Navigation})
\classe{nom}
\begin{glose}
\pfra{pirogue à balancier}
\end{glose}
\end{entrée}

\begin{entrée}{katri}{}{ⓔkatri}
\formephonétique{kaʈi}
\région{GOs}
\variante{%
kari
\région{PA BO}}
\classe{nom}
\newline
\sens{1}
(\domainesémantique{Noms des plantes})
\begin{glose}
\pfra{gingembre (culinaire)}
\end{glose}
\newline
\sens{2}
(\domainesémantique{Couleurs})
\begin{glose}
\pfra{jaune ; orange ; curry}
\end{glose}
\end{entrée}

\begin{entrée}{katria}{}{ⓔkatria}
\formephonétique{'kaɽia}
\région{GOs}
\variante{%
karia
\région{GO(s)}, 
kathia, karia
\région{PA}, 
katia
\région{BO}}
(\domainesémantique{Santé, maladie})
\classe{v ; n}
\begin{glose}
\pfra{lèpre ; lépreux ; avoir la lèpre}
\end{glose}
\newline
\emprunt{katia (POLYN) (PPN *katia)}
\end{entrée}

\begin{entrée}{katrińi}{}{ⓔkatrińi}
\formephonétique{kaʈini}
\région{GOs}
\variante{%
karini, karili
\région{BO}}
(\domainesémantique{Insectes})
\classe{nom}
\begin{glose}
\pfra{mille-pattes}
\end{glose}
\end{entrée}

\begin{entrée}{kau-}{}{ⓔkau-}
\région{GOs BO}
\classe{nom}
\newline
\sens{1}
(\domainesémantique{Découpage du temps})
\begin{glose}
\pfra{année de}
\end{glose}
\newline
\sens{2}
(\domainesémantique{Cours de la vie})
\begin{glose}
\pfra{âge}
\end{glose}
\newline
\begin{exemple}
\région{GO}
\textbf{\pnua{pòniza kau-jö ?}}
\pfra{quel âge as-tu ?}
\end{exemple}
\newline
\begin{exemple}
\région{BO}
\textbf{\pnua{waya kau-m ?}}
\pfra{quel âge as-tu ?}
\end{exemple}
\newline
\begin{sous-entrée}{pe-kau-li}{ⓔkau-ⓢ2ⓝpe-kau-li}
\begin{glose}
\pfra{ils ont le même âge}
\end{glose}
\newline
\relationsémantique{Cf.}{\lien{ⓔkaⓗ1}{ka}}
\glosecourte{année}
\end{sous-entrée}
\end{entrée}

\begin{entrée}{kauda}{}{ⓔkauda}
\région{BO}
(\domainesémantique{Guerre})
\classe{v}
\begin{glose}
\pfra{réfugier (se) [Haudr., Corne]}
\end{glose}
\newline
\note{non vérifié}{général}{}
\end{entrée}

\begin{entrée}{kaureji}{}{ⓔkaureji}
\formephonétique{kauredji}
\région{GOs}
\variante{%
kaureim
\région{PA}}
(\domainesémantique{Insectes})
\classe{nom}
\begin{glose}
\pfra{luciole}
\end{glose}
\end{entrée}

\begin{entrée}{kausu}{}{ⓔkausu}
\région{GOs}
(\domainesémantique{Arbre})
\classe{nom}
\begin{glose}
\pfra{banian (avec la sève duquel on fait des balles de cricket)}
\end{glose}
\newline
\emprunt{caoutchouc (FR)}
\end{entrée}

\begin{entrée}{kava}{1}{ⓔkavaⓗ1}
\région{GOs}
(\domainesémantique{Poissons})
\classe{nom}
\begin{glose}
\pfra{dawa}
\end{glose}
\nomscientifique{Naso unicornis}
\newline
\relationsémantique{Cf.}{\lien{}{PEO taRa (Geraghty)}}
\end{entrée}

\begin{entrée}{kava}{2}{ⓔkavaⓗ2}
\région{GO}
(\domainesémantique{Actions liées aux plantes})
\classe{v}
\begin{glose}
\pfra{cueillir (des fleurs, des herbes magiques) [Haudricourt]}
\end{glose}
\newline
\note{non vérifié}{général}{}
\end{entrée}

\begin{entrée}{kavobe}{}{ⓔkavobe}
\région{GOs BO}
(\domainesémantique{Ignames})
\classe{nom}
\begin{glose}
\pfra{igname (petite)}
\end{glose}
\end{entrée}

\begin{entrée}{kavuxavwu}{}{ⓔkavuxavwu}
\formephonétique{kavuɣaβu}
\région{GOs}
\variante{%
kavu-avu
\région{PA}, 
kavu-kavu
\région{BO (Corne)}}
(\domainesémantique{Noms des plantes})
\classe{nom}
\begin{glose}
\pfra{fougère arborescente}
\end{glose}
\nomscientifique{Cyathea sp. (Cyathéacées)}
\end{entrée}

\begin{entrée}{kavwa hovwo}{}{ⓔkavwa hovwo}
\formephonétique{kaβa hoβo}
\région{GOs}
(\domainesémantique{Aliments, alimentation})
\classe{NEG}
\begin{glose}
\pfra{non comestible}
\end{glose}
\end{entrée}

\begin{entrée}{kavwègu}{}{ⓔkavwègu}
\formephonétique{kaβɛŋgu}
\région{GOs PA BO}
\variante{%
kapègu
\région{GO(s) vx}}
\classe{nom}
\newline
\sens{1}
(\domainesémantique{Organisation sociale})
\begin{glose}
\pfra{chefferie}
\end{glose}
\begin{glose}
\pfra{cour de la chefferie}
\end{glose}
\newline
\begin{exemple}
\région{PA}
\textbf{\pnua{ni kavègu-la}}
\pfra{dans leur chefferie}
\end{exemple}
\newline
\begin{exemple}
\région{BO}
\textbf{\pnua{kavègu i khiny}}
\pfra{le territoire de l'hirondelle (conte)}
\end{exemple}
\newline
\begin{exemple}
\région{GO}
\textbf{\pnua{kavègu i aazo}}
\pfra{la chefferie}
\end{exemple}
\newline
\sens{2}
(\domainesémantique{Habitat})
\begin{glose}
\pfra{village ; ensemble de maisons}
\end{glose}
\begin{glose}
\pfra{place du village}
\end{glose}
\end{entrée}

\begin{entrée}{kavwo}{}{ⓔkavwo}
\formephonétique{'kaβo}
\région{GOs}
\variante{%
kavwong
\région{PA BO}}
(\domainesémantique{Fonctions naturelles humaines})
\classe{v}
\begin{glose}
\pfra{rôter}
\end{glose}
\end{entrée}

\begin{entrée}{kavwö}{}{ⓔkavwö}
\formephonétique{kaβω}
\région{GOs PA BO}
\variante{%
kavwa
\région{GOs BO}, 
ka, kavwu
\région{PA}, 
kapoa
\région{vx}}
(\domainesémantique{Négation})
\classe{NEG}
\begin{glose}
\pfra{ne...pas}
\end{glose}
\newline
\begin{exemple}
\région{GO}
\textbf{\pnua{kavwö nu trònè}}
\pfra{je n'ai pas entendu}
\end{exemple}
\newline
\begin{exemple}
\région{GO}
\textbf{\pnua{kavwö pwa}}
\pfra{il ne pleut pas}
\end{exemple}
\newline
\begin{exemple}
\région{PA}
\textbf{\pnua{kavu pwal}}
\pfra{il ne pleut pas}
\end{exemple}
\newline
\begin{exemple}
\région{GO}
\textbf{\pnua{kavwu nu hine}}
\pfra{je ne sais pas}
\end{exemple}
\newline
\begin{exemple}
\région{BO}
\textbf{\pnua{kavwa ai-n}}
\pfra{il n'a pas envie}
\end{exemple}
\newline
\begin{exemple}
\région{BO}
\textbf{\pnua{kavwa li tone}}
\pfra{il n'entendent pas}
\end{exemple}
\newline
\begin{exemple}
\région{BO}
\textbf{\pnua{kavwo e}}
\pfra{ce n'est pas ainsi}
\end{exemple}
\newline
\begin{sous-entrée}{kavo ... ro}{ⓔkavwöⓝkavo ... ro}
\begin{glose}
\pfra{ne ... plus}
\end{glose}
\newline
\relationsémantique{Cf.}{\lien{ⓔkebwa}{kebwa}}
\glosecourte{ne...pas (prohibitif)}
\end{sous-entrée}
\end{entrée}

\begin{entrée}{kavwö ... gò}{}{ⓔkavwö ... gò}
\région{GOs}
(\domainesémantique{Aspect})
\classe{ASP}
\begin{glose}
\pfra{pas encore}
\end{glose}
\newline
\begin{exemple}
\région{GO}
\textbf{\pnua{kavwö e mããni gò}}
\pfra{il ne dort pas encore}
\end{exemple}
\newline
\begin{exemple}
\textbf{\pnua{kavwö ne ne gò lã-nã}}
\pfra{il n'a encore jamais fait ces choses-là}
\end{exemple}
\end{entrée}

\begin{entrée}{kavwö hine}{}{ⓔkavwö hine}
\formephonétique{kaβω hiɳe}
\région{GOs}
(\domainesémantique{Modalité, verbes modaux})
\classe{v}
\begin{glose}
\pfra{ne jamais faire qqch (= ne pas savoir)}
\end{glose}
\newline
\begin{exemple}
\textbf{\pnua{e kavwo hine thuâ}}
\pfra{il ne ment jamais}
\end{exemple}
\end{entrée}

\begin{entrée}{kavwö ... mwã}{}{ⓔkavwö ... mwã}
\région{GOs}
(\domainesémantique{Aspect})
\classe{ASP (révolu)}
\begin{glose}
\pfra{ne ... plus}
\end{glose}
\newline
\begin{exemple}
\textbf{\pnua{kavwö nu ẽnõ mwã}}
\pfra{je ne suis plus jeune}
\end{exemple}
\newline
\begin{exemple}
\textbf{\pnua{kavwö tròòli mã mwã}}
\pfra{il n'est plus malade}
\end{exemple}
\newline
\begin{exemple}
\région{GO}
\textbf{\pnua{kavwö mããni mwã}}
\pfra{il ne dort plus}
\end{exemple}
\end{entrée}

\begin{entrée}{kavwö ... nee ... taagin}{}{ⓔkavwö ... nee ... taagin}
\région{PA BO [Corne]}
(\domainesémantique{Aspect})
\classe{ASP}
\begin{glose}
\pfra{encore jamais ; jamais}
\end{glose}
\newline
\begin{exemple}
\région{BO}
\textbf{\pnua{kavwö nu hovo taagin}}
\pfra{je ne mange jamais}
\end{exemple}
\end{entrée}

\begin{entrée}{kavwö ... ne... gò}{}{ⓔkavwö ... ne... gò}
\formephonétique{kaβω ɳe}
\région{GOs}
(\domainesémantique{Aspect})
\classe{ASP}
\begin{glose}
\pfra{encore jamais ; jamais}
\end{glose}
\newline
\begin{exemple}
\région{GO}
\textbf{\pnua{kavwö nu ne ã-du gò Frans}}
\pfra{je ne suis encore jamais allé en France}
\end{exemple}
\newline
\begin{exemple}
\textbf{\pnua{kavwö nu ne nõõ-je}}
\pfra{je ne l'ai jamais vu(e)}
\end{exemple}
\newline
\begin{exemple}
\textbf{\pnua{kavwö ne ne gò lã-nã}}
\pfra{il n'a encore jamais fait ces choses-là}
\end{exemple}
\end{entrée}

\begin{entrée}{kawaang}{}{ⓔkawaang}
\région{PA BO [Corne]}
(\domainesémantique{Découpage du temps})
\classe{ADV}
\begin{glose}
\pfra{tôt le matin ; de bon matin}
\end{glose}
\end{entrée}

\begin{entrée}{kawê}{}{ⓔkawê}
\région{BO PA}
(\domainesémantique{Insectes})
\classe{nom}
\begin{glose}
\pfra{sauterelle (de cocotier, grosse, verte)}
\end{glose}
\end{entrée}

\begin{entrée}{kaweeng}{}{ⓔkaweeng}
\région{PA BO [BM]}
(\domainesémantique{Relations et interaction sociales})
\classe{v}
\begin{glose}
\pfra{imiter ; singer}
\end{glose}
\newline
\begin{exemple}
\région{PA}
\textbf{\pnua{i pe-kaweeng u me-va i nu}}
\pfra{il imite ma façon de parler}
\end{exemple}
\newline
\begin{exemple}
\région{BO}
\textbf{\pnua{i kaweeng u me-va i nu}}
\pfra{il imite ma façon de parler}
\end{exemple}
\newline
\begin{exemple}
\région{BO}
\textbf{\pnua{nu kaweeng u me-tèmèno i je}}
\pfra{j'imite sa façon de marcher}
\end{exemple}
\end{entrée}

\begin{entrée}{kaxe}{}{ⓔkaxe}
\région{GO}
(\domainesémantique{Conjonction})
\classe{CNJ}
\begin{glose}
\pfra{tandis que ; pendant}
\end{glose}
\begin{glose}
\pfra{pendant que}
\end{glose}
\end{entrée}

\begin{entrée}{kaya-ko}{}{ⓔkaya-ko}
\région{WEM}
(\domainesémantique{Corps humain})
\classe{nom}
\begin{glose}
\pfra{arrière de la jambe}
\end{glose}
\newline
\relationsémantique{Ant.}{\lien{ⓔala-meⓢ2ⓝala-me-ko}{ala-me-ko}}
\glosecourte{devant de la jambe}
\end{entrée}

\begin{entrée}{kaya-mwa}{}{ⓔkaya-mwa}
\région{BO}
\classe{nom}
\newline
\sens{1}
(\domainesémantique{Types de maison, architecture de la maison})
\begin{glose}
\pfra{arrière de la maison}
\end{glose}
\newline
\sens{2}
(\domainesémantique{Directions})
\begin{glose}
\pfra{nord du pays}
\end{glose}
\end{entrée}

\begin{entrée}{kayuna}{}{ⓔkayuna}
\région{GOs}
\variante{%
kauna
\région{PA BO}}
(\domainesémantique{Société})
\classe{nom}
\begin{glose}
\pfra{célibataire (homme ou femme)}
\end{glose}
\end{entrée}

\begin{entrée}{kaza}{}{ⓔkaza}
\région{GOs BO}
(\domainesémantique{Pêche})
\classe{v}
\begin{glose}
\pfra{pêcher (aller à la pêche à qqch)}
\end{glose}
\newline
\begin{exemple}
\région{GO}
\textbf{\pnua{co kaze kaza ? - nu kaze du cî}}
\pfra{tuvas faire la pêche à quoi ? -je vais à la pêche au crabe}
\end{exemple}
\end{entrée}

\begin{entrée}{kaze}{}{ⓔkaze}
\formephonétique{kaðe}
\région{GOs}
\variante{%
kale
\région{BO PA}}
\classe{n; v}
\newline
\sens{1}
(\domainesémantique{Marées})
\begin{glose}
\pfra{marée montante (être) ; marée haute}
\end{glose}
\begin{glose}
\pfra{remplir (se) (mer) [BO]}
\end{glose}
\newline
\begin{exemple}
\région{GO}
\textbf{\pnua{e wa-kaze}}
\pfra{c'est la marée haute}
\end{exemple}
\newline
\begin{sous-entrée}{me-kaze}{ⓔkazeⓢ1ⓝme-kaze}
\région{GO}
\begin{glose}
\pfra{marée montante}
\end{glose}
\end{sous-entrée}
\newline
\begin{sous-entrée}{kaze kou}{ⓔkazeⓢ1ⓝkaze kou}
\région{GO}
\begin{glose}
\pfra{marée d'équinoxe de mars}
\end{glose}
\end{sous-entrée}
\newline
\begin{sous-entrée}{me kale}{ⓔkazeⓢ1ⓝme kale}
\région{PA}
\begin{glose}
\pfra{marée montante (avant de la marée montante, limite de l'eau montante)}
\end{glose}
\end{sous-entrée}
\newline
\begin{sous-entrée}{kale kou}{ⓔkazeⓢ1ⓝkale kou}
\région{PA}
\begin{glose}
\pfra{marée d'équinoxe}
\end{glose}
\newline
\begin{exemple}
\région{PA}
\textbf{\pnua{i kale}}
\pfra{la marée monte}
\end{exemple}
\newline
\begin{exemple}
\région{PA}
\textbf{\pnua{i pwa kale}}
\pfra{la marée descend}
\end{exemple}
\end{sous-entrée}
\newline
\sens{2}
(\domainesémantique{Pêche})
\classe{v}
\begin{glose}
\pfra{aller à la pêche (à la mer) ; aller chercher de la nourriture à la mer}
\end{glose}
\newline
\begin{exemple}
\région{GO}
\textbf{\pnua{e a-kaze}}
\pfra{elle va à la pêche (à la mer à marée montante)}
\end{exemple}
\newline
\begin{exemple}
\région{PA}
\textbf{\pnua{e a kale}}
\pfra{elle va à la pêche (à la mer à marée montante)}
\end{exemple}
\newline
\begin{exemple}
\région{BO}
\textbf{\pnua{i a du kale}}
\pfra{elle va à la pêche (à la mer à marée montante)}
\end{exemple}
\newline
\étymologie{
\langue{POc}
\étymon{*ta(n)si(k)}
\glosecourte{mer}}
\end{entrée}

\begin{entrée}{kazi}{}{ⓔkazi}
\formephonétique{kaːði}
\région{GOs}
\variante{%
kali-
\région{PA BO}}
(\domainesémantique{Parenté})
\classe{nom}
\begin{glose}
\pfra{cadet ; frères ou soeurs plus jeunes qu'ego}
\end{glose}
\newline
\begin{exemple}
\région{GO}
\textbf{\pnua{kazi-nu}}
\pfra{mon frère (plus jeune)}
\end{exemple}
\newline
\begin{exemple}
\région{PA}
\textbf{\pnua{kali-ny}}
\pfra{mon cadet, mes frères ou soeurs (plus jeunes)}
\end{exemple}
\newline
\étymologie{
\langue{POc}
\étymon{*taji}}
\end{entrée}

\begin{entrée}{kazubi}{}{ⓔkazubi}
\formephonétique{ka'ðumbi}
\région{GOs}
\variante{%
karubi, katrubi
\formephonétique{ka'ɽubi}
\région{WEM WE}, 
kharubin
\région{PA BO}}
\newline
\groupe{A}
(\domainesémantique{Manière de faire l’action : verbes et adverbes de manière})
\classe{v}
\begin{glose}
\pfra{faire vite}
\end{glose}
\newline
\begin{exemple}
\région{GO}
\textbf{\pnua{kazubi-ni môgu i jö !}}
\pfra{fais vite ton travail}
\end{exemple}
\newline
\groupe{B}
(\domainesémantique{Manière de faire l’action : verbes et adverbes de manière})
\classe{ADV}
\begin{glose}
\pfra{vite ; rapidement ; tout de suite}
\end{glose}
\newline
\begin{exemple}
\région{GO}
\textbf{\pnua{ã-mi jö whili-da (è)nè kazubi}}
\pfra{viens ici et amène-le vite ici}
\end{exemple}
\newline
\relationsémantique{Cf.}{\lien{}{kha-maaçee, maaja, beloo}}
\glosecourte{lentement, lent}
\end{entrée}

\begin{entrée}{kazu kaza}{}{ⓔkazu kaza}
\région{GOs}
(\domainesémantique{Jeux divers})
\classe{nom}
\begin{glose}
\pfra{balancelle}
\end{glose}
\end{entrée}

\begin{entrée}{kâdi}{}{ⓔkâdi}
\région{GOs PA BO}
(\domainesémantique{Relations et interaction sociales})
\classe{nom}
\begin{glose}
\pfra{récompense ; rétribution ; remerciement}
\end{glose}
\newline
\begin{exemple}
\textbf{\pnua{na kâdi}}
\pfra{une récompense}
\end{exemple}
\newline
\begin{exemple}
\région{PA}
\textbf{\pnua{kâdi-n}}
\pfra{sa récompense}
\end{exemple}
\end{entrée}

\begin{entrée}{kâduk}{}{ⓔkâduk}
\région{BO}
\variante{%
kâdu
}
(\domainesémantique{Noms des plantes})
\classe{nom}
\begin{glose}
\pfra{liane de forêt (utilisée pour la construction de cases [Corne])}
\end{glose}
\newline
\note{non vérifié}{général}{}
\end{entrée}

\begin{entrée}{ke}{}{ⓔke}
\région{GOs}
\variante{%
keel
\région{PA BO}}
(\domainesémantique{Paniers})
\classe{nom}
\begin{glose}
\pfra{panier}
\end{glose}
\newline
\begin{exemple}
\région{GO}
\textbf{\pnua{kee-nu}}
\pfra{mon panier}
\end{exemple}
\newline
\begin{exemple}
\région{PA}
\textbf{\pnua{kee-ny}}
\pfra{mon panier}
\end{exemple}
\newline
\begin{sous-entrée}{ke simi}{ⓔkeⓝke simi}
\begin{glose}
\pfra{poche de chemise}
\end{glose}
\end{sous-entrée}
\newline
\begin{sous-entrée}{ke haba}{ⓔkeⓝke haba}
\begin{glose}
\pfra{panier (à tressage lâche)}
\end{glose}
\end{sous-entrée}
\newline
\begin{sous-entrée}{keala}{ⓔkeⓝkeala}
\begin{glose}
\pfra{panier à provisions (pour les champs)}
\end{glose}
\end{sous-entrée}
\newline
\begin{sous-entrée}{kehin}{ⓔkeⓝkehin}
\begin{glose}
\pfra{panier (à tressage serré)}
\end{glose}
\end{sous-entrée}
\newline
\begin{sous-entrée}{ke hegi}{ⓔkeⓝke hegi}
\begin{glose}
\pfra{panier à monnaie}
\end{glose}
\end{sous-entrée}
\newline
\begin{sous-entrée}{ke kabun}{ⓔkeⓝke kabun}
\begin{glose}
\pfra{panier sacré}
\end{glose}
\end{sous-entrée}
\newline
\begin{sous-entrée}{kee-paa}{ⓔkeⓝkee-paa}
\begin{glose}
\pfra{panier pour porter les pierres de fronde}
\end{glose}
\end{sous-entrée}
\newline
\begin{sous-entrée}{kerewòla, kerewala}{ⓔkeⓝkerewòla, kerewala}
\begin{glose}
\pfra{panier}
\end{glose}
\end{sous-entrée}
\newline
\begin{sous-entrée}{keruau}{ⓔkeⓝkeruau}
\begin{glose}
\pfra{grand panier en palmes de cocotier}
\end{glose}
\end{sous-entrée}
\newline
\begin{sous-entrée}{ke-thal}{ⓔkeⓝke-thal}
\région{PA}
\begin{glose}
\pfra{panier en feuilles de pandanus}
\end{glose}
\end{sous-entrée}
\newline
\begin{sous-entrée}{ke-thrõõbo}{ⓔkeⓝke-thrõõbo}
\begin{glose}
\pfra{panier de charge (porté sur le dos)}
\end{glose}
\end{sous-entrée}
\newline
\begin{sous-entrée}{kevaho}{ⓔkeⓝkevaho}
\begin{glose}
\pfra{panier à anse}
\end{glose}
\end{sous-entrée}
\newline
\étymologie{
\langue{POc}
\étymon{*taŋa}}
\end{entrée}

\begin{entrée}{ke-}{}{ⓔke-}
\région{PA}
\variante{%
kee
\région{PA}}
(\domainesémantique{Paniers
, Préfixes classificateurs sémantiques})
\classe{CLF}
\begin{glose}
\pfra{préfixe des paniers}
\end{glose}
\newline
\begin{sous-entrée}{ke-baxo}{ⓔke-ⓝke-baxo}
\begin{glose}
\pfra{panier à anse}
\end{glose}
\end{sous-entrée}
\newline
\begin{sous-entrée}{ke-bukaka}{ⓔke-ⓝke-bukaka}
\begin{glose}
\pfra{panier sacré}
\end{glose}
\end{sous-entrée}
\newline
\begin{sous-entrée}{ke-haxaba}{ⓔke-ⓝke-haxaba}
\begin{glose}
\pfra{panier de jonc}
\end{glose}
\end{sous-entrée}
\newline
\begin{sous-entrée}{ke-paa}{ⓔke-ⓝke-paa}
\begin{glose}
\pfra{giberne de fronde}
\end{glose}
\end{sous-entrée}
\newline
\begin{sous-entrée}{ke-thrõõbo}{ⓔke-ⓝke-thrõõbo}
\begin{glose}
\pfra{panier de charge}
\end{glose}
\end{sous-entrée}
\newline
\begin{sous-entrée}{ke-tru}{ⓔke-ⓝke-tru}
\begin{glose}
\pfra{panier de monnaie}
\end{glose}
\end{sous-entrée}
\newline
\begin{sous-entrée}{ke-vala}{ⓔke-ⓝke-vala}
\begin{glose}
\pfra{panier porté sur les reins}
\end{glose}
\end{sous-entrée}
\end{entrée}

\begin{entrée}{kê-}{}{ⓔkê-}
\région{GOs PA}
\variante{%
kêê
\région{GO}}
(\domainesémantique{Préfixes classificateurs sémantiques
, Types de champs})
\classe{nom}
\begin{glose}
\pfra{champ ; emplacement}
\end{glose}
\newline
\begin{sous-entrée}{kê-pomo-nu}{ⓔkê-ⓝkê-pomo-nu}
\région{GO}
\begin{glose}
\pfra{mon champ}
\end{glose}
\end{sous-entrée}
\newline
\begin{sous-entrée}{kê-kui}{ⓔkê-ⓝkê-kui}
\begin{glose}
\pfra{massif d'igname; billon d'igname}
\end{glose}
\end{sous-entrée}
\newline
\begin{sous-entrée}{kê-uva}{ⓔkê-ⓝkê-uva}
\begin{glose}
\pfra{champ de taro}
\end{glose}
\end{sous-entrée}
\newline
\begin{sous-entrée}{kê-chaamwa}{ⓔkê-ⓝkê-chaamwa}
\begin{glose}
\pfra{bananeraie}
\end{glose}
\end{sous-entrée}
\newline
\begin{sous-entrée}{kê-pwâ}{ⓔkê-ⓝkê-pwâ}
\begin{glose}
\pfra{jardin, plantation}
\end{glose}
\end{sous-entrée}
\newline
\begin{sous-entrée}{kê-mwa}{ⓔkê-ⓝkê-mwa}
\begin{glose}
\pfra{emplacement de la maison}
\end{glose}
\end{sous-entrée}
\newline
\begin{sous-entrée}{kê kui èmwèn}{ⓔkê-ⓝkê kui èmwèn}
\begin{glose}
\pfra{côté du massif d'igname travaillé par les hommes}
\end{glose}
\end{sous-entrée}
\newline
\begin{sous-entrée}{kê kui tòòmwa}{ⓔkê-ⓝkê kui tòòmwa}
\begin{glose}
\pfra{côté du massif d'igname travaillé par les femmes}
\end{glose}
\end{sous-entrée}
\end{entrée}

\begin{entrée}{kea}{}{ⓔkea}
\région{GOs PA BO}
(\domainesémantique{Préfixes et verbes de position})
\classe{v}
\begin{glose}
\pfra{incliné ; appuyé}
\end{glose}
\newline
\begin{exemple}
\textbf{\pnua{mhenõ-mwê mani mhenõ-kea}}
\pfra{nos soutiens et nos appuis}
\end{exemple}
\end{entrée}

\begin{entrée}{keala}{}{ⓔkeala}
\région{GOs PA WEM}
\variante{%
kevala
\région{WEM WE}}
(\domainesémantique{Paniers})
\classe{nom}
\begin{glose}
\pfra{panier}
\end{glose}
\newline
\note{en palme de cocotier, rectangulaire ou ovale, à fond plat : utilisé pour envelopper de la nourriture}{glose}{}
\end{entrée}

\begin{entrée}{kebwa}{}{ⓔkebwa}
\région{GOs}
(\domainesémantique{Armes})
\classe{nom}
\begin{glose}
\pfra{sarbacane}
\end{glose}
\end{entrée}

\begin{entrée}{kêbwa}{}{ⓔkêbwa}
\formephonétique{kêbwa}
\région{GOs PA BO}
\variante{%
kêbwa-n
\région{BO}}
(\domainesémantique{Modalité, verbes modaux})
\classe{PROH}
\begin{glose}
\pfra{ne pas falloir}
\end{glose}
\newline
\begin{exemple}
\région{GO}
\textbf{\pnua{kêbwa ubò pwaa !}}
\pfra{ne sors pas !}
\end{exemple}
\newline
\begin{exemple}
\région{BO}
\textbf{\pnua{kêbwa hovwo}}
\pfra{ne mange pas !}
\end{exemple}
\newline
\begin{exemple}
\région{BO}
\textbf{\pnua{kêbwa-n na jo po khõbwe mwã}}
\pfra{il ne faut pas que tu dises quoi que ce soit}
\end{exemple}
\newline
\relationsémantique{Cf.}{\lien{}{kavwa}}
\glosecourte{ne...pas}
\end{entrée}

\begin{entrée}{ke-bwaxo}{}{ⓔke-bwaxo}
\région{GOs}
(\domainesémantique{Paniers})
\classe{nom}
\begin{glose}
\pfra{panier tressé en pandanus (utilisé comme sac à main)}
\end{glose}
\end{entrée}

\begin{entrée}{ke-caadu}{}{ⓔke-caadu}
\formephonétique{ketjaːdu}
\région{GOs}
(\domainesémantique{Paniers})
\classe{nom}
\begin{glose}
\pfra{panier (en palme de cocotier, rond, plus grand que "keruau")}
\end{glose}
\end{entrée}

\begin{entrée}{kecak}{}{ⓔkecak}
\région{BO [Corne]}
(\domainesémantique{Paniers})
\classe{nom}
\begin{glose}
\pfra{sac}
\end{glose}
\newline
\emprunt{sac (FR)}
\newline
\note{composé : ke- (préfixe des paniers), FR sac}{général}{}
\end{entrée}

\begin{entrée}{ke-cee}{}{ⓔke-cee}
\région{GOs PA}
\variante{%
ke-çee
\région{GO(s)}}
(\domainesémantique{Cours de la vie})
\classe{nom}
\begin{glose}
\pfra{cercueil (lit. panier en bois)}
\end{glose}
\end{entrée}

\begin{entrée}{kêdi}{}{ⓔkêdi}
\région{GOs}
(\domainesémantique{Mouvements ou actions faits avec le corps, les bras, les mains, les pieds})
\classe{v}
\begin{glose}
\pfra{ouvrir une enveloppe (bougna)}
\end{glose}
\end{entrée}

\begin{entrée}{kee}{}{ⓔkee}
\région{GOs}
(\domainesémantique{Modalité, verbes modaux})
\classe{v}
\begin{glose}
\pfra{laisser ; permettre}
\end{glose}
\newline
\begin{exemple}
\région{GO}
\textbf{\pnua{kee je vwo yu ènè !}}
\pfra{laisse-le rester ici !}
\end{exemple}
\newline
\begin{exemple}
\textbf{\pnua{le kee-nu vwo nu pwaala}}
\pfra{ils m'ont laissé naviguer}
\end{exemple}
\end{entrée}

\begin{entrée}{kèè}{}{ⓔkèè}
\région{GOs}
(\domainesémantique{Arbre})
\classe{nom}
\begin{glose}
\pfra{palétuvier gris}
\end{glose}
\nomscientifique{Avicennia marina, Verbénacées}
\end{entrée}

\begin{entrée}{kêê}{1}{ⓔkêêⓗ1}
\région{GOs BO PA}
(\domainesémantique{Parenté})
\classe{nom}
\begin{glose}
\pfra{père (désignation)}
\end{glose}
\begin{glose}
\pfra{frère de père ; cousins du père}
\end{glose}
\begin{glose}
\pfra{époux de soeur de mère ; époux de cousine de mère}
\end{glose}
\newline
\begin{exemple}
\région{PA}
\textbf{\pnua{kêê-n}}
\pfra{son père}
\end{exemple}
\newline
\begin{exemple}
\région{PA}
\textbf{\pnua{kêê-n mani kibu-n}}
\pfra{les pères et grand-pères}
\end{exemple}
\newline
\étymologie{
\langue{POc}
\étymon{*tama}}
\end{entrée}

\begin{entrée}{kêê}{2}{ⓔkêêⓗ2}
\région{GOs PA}
(\domainesémantique{Types de champs})
\classe{nom}
\begin{glose}
\pfra{champ ; plantation}
\end{glose}
\newline
\begin{exemple}
\région{PA}
\textbf{\pnua{kêê-n}}
\pfra{son champ, ses plantations}
\end{exemple}
\newline
\begin{exemple}
\région{PA}
\textbf{\pnua{kêê-phwãã}}
\pfra{champ déjà récolté, laissé en jachère}
\end{exemple}
\end{entrée}

\begin{entrée}{kêê-chaamwa}{}{ⓔkêê-chaamwa}
\formephonétique{kêːcʰaːmwa}
\région{GOs}
(\domainesémantique{Types de champs})
\classe{nom}
\begin{glose}
\pfra{bananeraie ; champ de bananiers}
\end{glose}
\end{entrée}

\begin{entrée}{keeda}{}{ⓔkeeda}
\région{GOs BO PA}
\région{BO}
\variante{%
jeeda
}
(\domainesémantique{Arbre})
\classe{nom}
\begin{glose}
\pfra{badamier}
\end{glose}
\nomscientifique{Terminalia catapa L.}
\end{entrée}

\begin{entrée}{kee hõbo}{}{ⓔkee hõbo}
\formephonétique{'keː 'hɔ̃mbo}
\région{GOs}
\variante{%
kee-hãbwo
}
(\domainesémantique{Objets et meubles de la maison})
\classe{nom}
\begin{glose}
\pfra{armoire}
\end{glose}
\end{entrée}

\begin{entrée}{keejò}{}{ⓔkeejò}
\région{GOs}
(\domainesémantique{Arbre})
\classe{nom}
\begin{glose}
\pfra{"bois de lait" (on utilise les graines comme perles)}
\end{glose}
\end{entrée}

\begin{entrée}{kêê-kui}{}{ⓔkêê-kui}
\région{GOs PA}
(\domainesémantique{Types de champs})
\classe{nom}
\begin{glose}
\pfra{champ d'ignames}
\end{glose}
\end{entrée}

\begin{entrée}{kêê-khò}{}{ⓔkêê-khò}
\formephonétique{kêː kʰɔ}
\région{GOs PA}
(\domainesémantique{Types de champs})
\classe{nom}
\begin{glose}
\pfra{champ cultivé ; champ labouré}
\end{glose}
\end{entrée}

\begin{entrée}{kêê-mu-ce}{}{ⓔkêê-mu-ce}
\région{GOs}
(\domainesémantique{Types de champs})
\classe{nom}
\begin{glose}
\pfra{jardin de fleurs}
\end{glose}
\end{entrée}

\begin{entrée}{kêê-mwa}{}{ⓔkêê-mwa}
\région{GOs}
(\domainesémantique{Types de maison, architecture de la maison})
\classe{nom}
\begin{glose}
\pfra{terrassement de la maison}
\end{glose}
\end{entrée}

\begin{entrée}{kêê-nu}{}{ⓔkêê-nu}
\région{GOs PA BO}
(\domainesémantique{Types de champs})
\classe{nom}
\begin{glose}
\pfra{champ de cocotier ; plantation de cocotier}
\end{glose}
\end{entrée}

\begin{entrée}{kee-pazalô}{}{ⓔkee-pazalô}
\région{GOs}
(\domainesémantique{Vêtements, parure})
\classe{nom}
\begin{glose}
\pfra{poche de pantalon}
\end{glose}
\end{entrée}

\begin{entrée}{kêê-phwãã}{}{ⓔkêê-phwãã}
\région{GOs}
(\domainesémantique{Types de champs})
\classe{nom}
\begin{glose}
\pfra{champ en jachère(déjà récolté, dans lequel poussent des rejets)}
\end{glose}
\end{entrée}

\begin{entrée}{kêê-tre}{}{ⓔkêê-tre}
\région{GOs}
(\domainesémantique{Cultures, techniques, boutures})
\classe{nom}
\begin{glose}
\pfra{coin débroussé pour cultures}
\end{glose}
\end{entrée}

\begin{entrée}{kêê-yaai}{}{ⓔkêê-yaai}
\région{PA BO}
(\domainesémantique{Types de champs})
\classe{nom}
\begin{glose}
\pfra{champ après brûlis}
\end{glose}
\end{entrée}

\begin{entrée}{kègele}{}{ⓔkègele}
\formephonétique{kɛ'ŋgele}
\région{GOs PA}
\variante{%
kègel
\formephonétique{kɛ'gɛl}}
(\domainesémantique{Mouvements ou actions faits avec le corps, les bras, les mains, les pieds})
\classe{v}
\begin{glose}
\pfra{agiter un objet contenant qqch (fait un son)}
\end{glose}
\end{entrée}

\begin{entrée}{ke haba}{}{ⓔke haba}
\région{BO [Corne]}
(\domainesémantique{Paniers})
\classe{nom}
\begin{glose}
\pfra{panier (pour filtrer le dimwa, à tressage lâche)}
\end{glose}
\newline
\note{non vérifié}{général}{}
\end{entrée}

\begin{entrée}{kehin}{}{ⓔkehin}
\région{BO}
(\domainesémantique{Paniers})
\classe{nom}
\begin{glose}
\pfra{panier (à tressage serré) [Corne]}
\end{glose}
\newline
\note{non vérifié}{général}{}
\end{entrée}

\begin{entrée}{kela}{}{ⓔkela}
\région{GO}
\variante{%
kelo
\région{GO}}
(\domainesémantique{Paniers})
\classe{nom}
\begin{glose}
\pfra{panier (grand) en palme de cocotier}
\end{glose}
\end{entrée}

\begin{entrée}{kèlèrè}{}{ⓔkèlèrè}
\région{GOs}
(\domainesémantique{Ignames})
\classe{nom}
\begin{glose}
\pfra{igname (violette)}
\end{glose}
\end{entrée}

\begin{entrée}{kelò}{}{ⓔkelò}
\région{GOs}
(\domainesémantique{Paniers})
\classe{nom}
\begin{glose}
\pfra{panier en cocotier (pour porter les ignames)}
\end{glose}
\end{entrée}

\begin{entrée}{ke-mõã}{}{ⓔke-mõã}
\région{GOs WEM}
(\domainesémantique{Paniers})
\classe{nom}
\begin{glose}
\pfra{panier de restes (de nourriture)}
\end{glose}
\newline
\begin{exemple}
\textbf{\pnua{ke-mõã-nu}}
\pfra{mon panier de restes (de nourriture)}
\end{exemple}
\end{entrée}

\begin{entrée}{kemõ-hi}{}{ⓔkemõ-hi}
\région{GOs}
(\domainesémantique{Corps humain})
\classe{nom}
\begin{glose}
\pfra{poing}
\end{glose}
\newline
\begin{exemple}
\textbf{\pnua{kemõ-hi-nu}}
\pfra{mon poing}
\end{exemple}
\end{entrée}

\begin{entrée}{kê-muuc}{}{ⓔkê-muuc}
\région{PA}
(\domainesémantique{Types de champs})
\classe{nom}
\begin{glose}
\pfra{massif de fleurs}
\end{glose}
\end{entrée}

\begin{entrée}{kêni}{}{ⓔkêni}
\formephonétique{kêɳi, kîɳi}
\région{GOs WEM PA BO}
\variante{%
kîni
}
(\domainesémantique{Corps humain})
\classe{nom}
\begin{glose}
\pfra{oreille}
\end{glose}
\newline
\begin{exemple}
\région{GO}
\textbf{\pnua{kenii-je}}
\pfra{son oreille}
\end{exemple}
\newline
\begin{exemple}
\région{PA}
\textbf{\pnua{kenii-n}}
\pfra{son oreille}
\end{exemple}
\newline
\begin{sous-entrée}{coo-kenii-n}{ⓔkêniⓝcoo-kenii-n}
\région{PA}
\begin{glose}
\pfra{le lobe de son oreille}
\end{glose}
\end{sous-entrée}
\newline
\begin{sous-entrée}{dò-kenii-n}{ⓔkêniⓝdò-kenii-n}
\région{BO}
\begin{glose}
\pfra{pavillon de l'oreille}
\end{glose}
\end{sous-entrée}
\newline
\begin{sous-entrée}{no keni-n}{ⓔkêniⓝno keni-n}
\région{PA}
\begin{glose}
\pfra{cire de ses oreilles}
\end{glose}
\end{sous-entrée}
\newline
\begin{sous-entrée}{phwe-keni-n}{ⓔkêniⓝphwe-keni-n}
\région{PA}
\begin{glose}
\pfra{pavillon de ses oreilles}
\end{glose}
\end{sous-entrée}
\newline
\étymologie{
\langue{POc}
\étymon{*taliŋga}}
\end{entrée}

\begin{entrée}{kênii-döö}{}{ⓔkênii-döö}
\région{PA}
(\domainesémantique{Ustensiles})
\classe{nom}
\begin{glose}
\pfra{poignées de la marmite}
\end{glose}
\end{entrée}

\begin{entrée}{kênim}{}{ⓔkênim}
\région{PA}
(\domainesémantique{Coutumes, dons coutumiers})
\classe{nom}
\begin{glose}
\pfra{prestation de deuil aux maternels}
\end{glose}
\newline
\begin{exemple}
\textbf{\pnua{na ni kênime-n [PA]}}
\pfra{lors de leur prestation de deuil}
\end{exemple}
\end{entrée}

\begin{entrée}{kênõ}{}{ⓔkênõ}
\formephonétique{kêɳɔ̃}
\région{GOs}
\région{BO PA}
\variante{%
kênõng
\formephonétique{kɛ̃nɔ̃ŋ}}
\classe{v ; n}
\newline
\sens{1}
(\domainesémantique{Verbes d'action (en général)})
\begin{glose}
\pfra{tourner ; retourner ; retourner (se) ; poser à l'envers}
\end{glose}
\newline
\begin{exemple}
\textbf{\pnua{i kênõge dau}}
\pfra{il fait le tour de l'île}
\end{exemple}
\newline
\begin{exemple}
\région{GO}
\textbf{\pnua{e a-kênõ ni dau}}
\pfra{il fait le tour de l'île}
\end{exemple}
\newline
\begin{exemple}
\région{GO}
\textbf{\pnua{e kênõge vea pwa du bwabu}}
\pfra{elle a mis le verre à l'envers}
\end{exemple}
\newline
\begin{exemple}
\région{GO}
\textbf{\pnua{e ku kênõ}}
\pfra{il se retourne (debout)}
\end{exemple}
\newline
\begin{exemple}
\région{GO}
\textbf{\pnua{e tree kênõ}}
\pfra{il se retourne (assis)}
\end{exemple}
\newline
\begin{exemple}
\région{GO}
\textbf{\pnua{la ku kênõ}}
\pfra{ils sont debout en cercle}
\end{exemple}
\newline
\begin{exemple}
\région{GO}
\textbf{\pnua{la tree kênõ}}
\pfra{ils sont assis en cercle}
\end{exemple}
\newline
\sens{2}
(\domainesémantique{Verbes de mouvement})
\begin{glose}
\pfra{faire des moulinets du bras ; moulinet du bras (Ltd 1008)}
\end{glose}
\newline
\sens{3}
(\domainesémantique{Caractéristiques et propriétés des personnes})
\begin{glose}
\pfra{saoûl (être) (lit. la tête tourne)}
\end{glose}
\newline
\sens{4}
(\domainesémantique{Eau})
\begin{glose}
\pfra{tourbillon d'eau}
\end{glose}
\newline
\begin{sous-entrée}{we kênõ}{ⓔkênõⓢ4ⓝwe kênõ}
\région{GO}
\begin{glose}
\pfra{tourbillon d'eau}
\end{glose}
\end{sous-entrée}
\end{entrée}

\begin{entrée}{ke-paa}{}{ⓔke-paa}
\région{GOs BO}
\variante{%
ke-paa
\région{PA}}
(\domainesémantique{Paniers
, Armes})
\classe{nom}
\begin{glose}
\pfra{giberne de fronde}
\end{glose}
\end{entrée}

\begin{entrée}{ke-palu}{}{ⓔke-palu}
\région{GOs}
(\domainesémantique{Paniers})
\classe{nom}
\begin{glose}
\pfra{panier à richessess ; tirelire}
\end{glose}
\end{entrée}

\begin{entrée}{keraa}{}{ⓔkeraa}
\région{GOs}
\variante{%
keraò
\région{PA}}
\classe{nom}
\newline
\sens{1}
(\domainesémantique{Paniers})
\begin{glose}
\pfra{panier qui contient les dimwa grattées}
\end{glose}
\newline
\sens{2}
(\domainesémantique{Pêche})
\begin{glose}
\pfra{épuisette à crevettes [GO]}
\end{glose}
\end{entrée}

\begin{entrée}{kerao}{}{ⓔkerao}
\région{GOs BO}
(\domainesémantique{Instruments})
\classe{nom}
\begin{glose}
\pfra{instrument en paille pour gratter dans l'eau les bulbilles de "dimwa" (Dubois)}
\end{glose}
\end{entrée}

\begin{entrée}{kerewala}{}{ⓔkerewala}
\région{BO}
\variante{%
kerewòla
\région{BO}}
(\domainesémantique{Paniers})
\classe{nom}
\begin{glose}
\pfra{panier (tressé en palmes de cocotier)}
\end{glose}
\newline
\emprunt{ketewola (POLYN) (PPN *ketepola)}
\end{entrée}

\begin{entrée}{keruau}{}{ⓔkeruau}
\formephonétique{keɽuau}
\région{GOs}
\variante{%
ke-truau
\formephonétique{keʈuau}
\région{GO(s)}}
(\domainesémantique{Paniers})
\classe{nom}
\begin{glose}
\pfra{panier (circulaire en feuille de coco, petit à anses)}
\end{glose}
\end{entrée}

\begin{entrée}{ke-tilo}{}{ⓔke-tilo}
\région{BO}
(\domainesémantique{Coutumes, dons coutumiers})
\classe{nom}
\begin{glose}
\pfra{ossature de la monnaie [Corne]}
\end{glose}
\newline
\note{non vérifié}{général}{}
\end{entrée}

\begin{entrée}{ke-thal}{}{ⓔke-thal}
\région{PA BO}
(\domainesémantique{Paniers})
\classe{nom}
\begin{glose}
\pfra{panier en pandanus}
\end{glose}
\end{entrée}

\begin{entrée}{ke-thi}{}{ⓔke-thi}
\région{GOs}
(\domainesémantique{Paniers})
\classe{nom}
\begin{glose}
\pfra{porte-monnaie (claque quand se referme)}
\end{glose}
\end{entrée}

\begin{entrée}{ke-tru}{}{ⓔke-tru}
\région{GOs}
\variante{%
kèru
\région{BO}}
(\domainesémantique{Paniers})
\classe{nom}
\begin{glose}
\pfra{panier à monnaie ; enveloppe de monnaie coutumière}
\end{glose}
\end{entrée}

\begin{entrée}{ke-thrõõbo}{}{ⓔke-thrõõbo}
\région{GOs}
(\domainesémantique{Paniers})
\classe{nom}
\begin{glose}
\pfra{panier de charge ; panier porté sur le dos (avec une bandoulière ou comme un sac à dos)}
\end{glose}
\newline
\relationsémantique{Cf.}{\lien{ⓔthrõõbo}{thrõõbo}}
\glosecourte{porter sur le dos}
\end{entrée}

\begin{entrée}{kevaho}{}{ⓔkevaho}
\région{BO}
(\domainesémantique{Paniers})
\classe{nom}
\begin{glose}
\pfra{panier à anse [Corne]}
\end{glose}
\newline
\note{non vérifié}{général}{}
\end{entrée}

\begin{entrée}{kevalu}{}{ⓔkevalu}
\région{GOs BO}
\classe{nom}
\newline
\sens{1}
(\domainesémantique{Pêche})
\begin{glose}
\pfra{nasse (à anguilles ou poisson) ; filet}
\end{glose}
\newline
\note{filet dont la construction est identique à celle de "deaang", mais plus grand (Dubois, Corne).}{glose}{}
\newline
\sens{2}
(\domainesémantique{Types de maison, architecture de la maison})
\begin{glose}
\pfra{corbeille (maison, Charles)}
\end{glose}
\newline
\note{filet dont la construction est identique à celle de "deaang", mais plus grand (Dubois, Corne).}{glose}{}
\end{entrée}

\begin{entrée}{kevi}{}{ⓔkevi}
\région{PA}
(\domainesémantique{Préparation des aliments; modes de préparation et de cuisson})
\classe{v}
\begin{glose}
\pfra{évider avec une cuillère (noix de coco, papaye, avocat)}
\end{glose}
\end{entrée}

\begin{entrée}{kevwa}{}{ⓔkevwa}
\formephonétique{'keβa}
\région{GOs}
\variante{%
kewang
\région{BO PA}}
\classe{nom}
\newline
\sens{1}
(\domainesémantique{Topographie})
\begin{glose}
\pfra{vallée ; creux (terrain)}
\end{glose}
\begin{glose}
\pfra{ravin ; talweg}
\end{glose}
\newline
\begin{sous-entrée}{ku-kevwa}{ⓔkevwaⓢ1ⓝku-kevwa}
\région{GO}
\begin{glose}
\pfra{le haut de la vallée}
\end{glose}
\end{sous-entrée}
\newline
\begin{sous-entrée}{ku-kewang}{ⓔkevwaⓢ1ⓝku-kewang}
\région{PA}
\begin{glose}
\pfra{le haut de la vallée, talweg}
\end{glose}
\end{sous-entrée}
\newline
\sens{2}
(\domainesémantique{Corps humain})
\begin{glose}
\pfra{espace entre les côtes}
\end{glose}
\end{entrée}

\begin{entrée}{kêxê}{}{ⓔkêxê}
\région{GOs BO}
\variante{%
kèèngè
\région{BO [BM]}}
(\domainesémantique{Oiseaux})
\classe{nom}
\begin{glose}
\pfra{perroquet ; loriquet calédonien}
\end{glose}
\nomscientifique{Trichoglossus haematodus (Psittacidés)}
\end{entrée}

\begin{entrée}{kèxi}{}{ⓔkèxi}
\région{BO}
(\domainesémantique{Conjonction})
\classe{CNJ}
\begin{glose}
\pfra{et ensuite ; et alors [Corne]}
\end{glose}
\end{entrée}

\begin{entrée}{kâgao}{}{ⓔkâgao}
\région{GOs BO [Corne]}
(\domainesémantique{Noms des plantes})
\classe{nom}
\begin{glose}
\pfra{liane sauvage}
\end{glose}
\nomscientifique{Ipomoea cairica (L.) Sweet (Convolvulacées)}
\end{entrée}

\begin{entrée}{ki}{1}{ⓔkiⓗ1}
\région{GOs PA BO}
(\domainesémantique{Préparation des aliments; modes de préparation et de cuisson})
\classe{v}
\begin{glose}
\pfra{griller ; brûler}
\end{glose}
\newline
\begin{exemple}
\textbf{\pnua{e ki nô}}
\pfra{il grille le poisson}
\end{exemple}
\newline
\begin{exemple}
\textbf{\pnua{nu kini (v.t.)}}
\pfra{je l'ai grillé}
\end{exemple}
\end{entrée}

\begin{entrée}{ki}{2}{ⓔkiⓗ2}
\région{GOs WEM}
\variante{%
kim
\région{PA BO}}
(\domainesémantique{Processus liés aux plantes})
\classe{v}
\begin{glose}
\pfra{pousser ; grandir (plantes)}
\end{glose}
\begin{glose}
\pfra{croître ; germer}
\end{glose}
\newline
\begin{exemple}
\région{BO}
\textbf{\pnua{la kim ada}}
\pfra{ils poussent tout seuls}
\end{exemple}
\end{entrée}

\begin{entrée}{ki}{3}{ⓔkiⓗ3}
\région{GOs}
\variante{%
kil
\région{PA}}
(\domainesémantique{Santé, maladie})
\classe{v ; n}
\begin{glose}
\pfra{douleur ; faire mal}
\end{glose}
\end{entrée}

\begin{entrée}{kia}{}{ⓔkia}
\région{PA BO [BM]}
\variante{%
khia
\région{BO [BM]}}
(\domainesémantique{Remèdes, médecine
, Religion, représentations religieuses})
\classe{v ; n}
\begin{glose}
\pfra{appliquer (médicament) ; traiter}
\end{glose}
\begin{glose}
\pfra{médicament ; remède}
\end{glose}
\end{entrée}

\begin{entrée}{kia po}{}{ⓔkia po}
\région{BO}
(\domainesémantique{Prédicats existentiels})
\classe{v}
\begin{glose}
\pfra{il n'y a rien (lit. il n'y a pas chose)}
\end{glose}
\newline
\begin{exemple}
\textbf{\pnua{kia poo-le}}
\pfra{il n'y a rien là}
\end{exemple}
\newline
\étymologie{
\langue{POc}
\étymon{*tika}}
\end{entrée}

\begin{entrée}{kia we}{}{ⓔkia we}
\région{GOs PA BO}
(\domainesémantique{Pêche})
\classe{v}
\begin{glose}
\pfra{pêcher au poison (lit. empoisonner l'eau)}
\end{glose}
\newline
\étymologie{
\langue{POc}
\étymon{*tupa}}
\end{entrée}

\begin{entrée}{kibao}{}{ⓔkibao}
\formephonétique{kimbao}
\région{GOs PA BO}
\classe{v}
\newline
\sens{1}
(\domainesémantique{Verbes d'action (en général)})
\begin{glose}
\pfra{toucher (une cible avec un projectile et éventuellement tuer)}
\end{glose}
\newline
\begin{exemple}
\textbf{\pnua{co kibao ? hai, nu pa-tha !}}
\pfra{tu l'as touché ? non je l'ai raté}
\end{exemple}
\newline
\begin{exemple}
\textbf{\pnua{e kibao mèni}}
\pfra{il a touché l'oiseau}
\end{exemple}
\newline
\relationsémantique{Cf.}{\lien{ⓔcha}{cha}}
\glosecourte{toucher (cible)}
\newline
\sens{2}
(\domainesémantique{Coutumes, dons coutumiers})
\begin{glose}
\pfra{recevoir un geste coutumier (en remerciant par un geste en retour)}
\end{glose}
\end{entrée}

\begin{entrée}{kîbi}{}{ⓔkîbi}
\région{GOs PA BO}
(\domainesémantique{Préparation des aliments; modes de préparation et de cuisson})
\classe{nom}
\begin{glose}
\pfra{four enterré}
\end{glose}
\newline
\begin{exemple}
\région{BO}
\textbf{\pnua{i tu kibi mhwããnu}}
\pfra{la lune a un halo (lit. fait le four) (Dubois)}
\end{exemple}
\newline
\begin{sous-entrée}{na-du no kîbi}{ⓔkîbiⓝna-du no kîbi}
\begin{glose}
\pfra{faire cuire au four enterré}
\end{glose}
\end{sous-entrée}
\newline
\begin{sous-entrée}{cale kîbi}{ⓔkîbiⓝcale kîbi}
\région{BO}
\begin{glose}
\pfra{allumer le four}
\end{glose}
\end{sous-entrée}
\newline
\begin{sous-entrée}{thu kîbi}{ⓔkîbiⓝthu kîbi}
\begin{glose}
\pfra{faire le four}
\end{glose}
\end{sous-entrée}
\newline
\begin{sous-entrée}{paxa-kîbi}{ⓔkîbiⓝpaxa-kîbi}
\begin{glose}
\pfra{pierres du four}
\end{glose}
\end{sous-entrée}
\newline
\étymologie{
\langue{POc}
\étymon{*qumun}
\auteur{Blust}}
\end{entrée}

\begin{entrée}{kibii}{}{ⓔkibii}
\région{GOs PA BO}
\variante{%
khibii
\région{BO [BM]}}
(\domainesémantique{Mouvements ou actions faits avec le corps, les bras, les mains, les pieds})
\classe{v}
\begin{glose}
\pfra{casser en morceaux ; piler (verre, assiette, miroir)}
\end{glose}
\newline
\begin{sous-entrée}{kibii mwani}{ⓔkibiiⓝkibii mwani}
\région{PA}
\begin{glose}
\pfra{faire de la monnaie}
\end{glose}
\end{sous-entrée}
\end{entrée}

\begin{entrée}{kibo}{}{ⓔkibo}
\région{GOs PA BO}
\variante{%
kîbwòn
\région{PA}}
(\domainesémantique{Parties de plantes})
\classe{nom}
\begin{glose}
\pfra{bourgeons ; rejet (de plante) ; germe}
\end{glose}
\begin{glose}
\pfra{jeunes feuilles encore roulées qui sortent du coeur de la plante (taro)}
\end{glose}
\newline
\begin{sous-entrée}{kiboo ce}{ⓔkiboⓝkiboo ce}
\région{GO}
\begin{glose}
\pfra{bourgeons d'arbre}
\end{glose}
\end{sous-entrée}
\newline
\begin{sous-entrée}{kiboo nu}{ⓔkiboⓝkiboo nu}
\région{GO}
\begin{glose}
\pfra{germe de coco}
\end{glose}
\end{sous-entrée}
\newline
\begin{sous-entrée}{kibò phuleng}{ⓔkiboⓝkibò phuleng}
\région{PA}
\begin{glose}
\pfra{rejet de phuleng}
\end{glose}
\end{sous-entrée}
\end{entrée}

\begin{entrée}{kîbö}{}{ⓔkîbö}
\région{GOs PA BO}
\variante{%
keebo
\région{BO}}
(\domainesémantique{Arbre})
\classe{nom}
\begin{glose}
\pfra{palétuvier rouge (racines aériennes)}
\end{glose}
\nomscientifique{Rhizophora apiculata (Rhizophoracées)}
\newline
\begin{sous-entrée}{kîbö mii}{ⓔkîböⓝkîbö mii}
\begin{glose}
\pfra{palétuvier rouge}
\end{glose}
\end{sous-entrée}
\end{entrée}

\begin{entrée}{kibu}{}{ⓔkibu}
\région{GOs PA BO}
(\domainesémantique{Parenté})
\classe{nom}
\begin{glose}
\pfra{grand-père ; ancêtre}
\end{glose}
\newline
\begin{exemple}
\région{GO}
\textbf{\pnua{kibu-ã}}
\pfra{nos ancêtres}
\end{exemple}
\newline
\étymologie{
\langue{POc}
\étymon{*tumpu}}
\end{entrée}

\begin{entrée}{kibwa}{}{ⓔkibwa}
\région{BO}
(\domainesémantique{Mouvements ou actions faits avec le corps, les bras, les mains, les pieds})
\classe{v}
\begin{glose}
\pfra{pousser ; précipiter qqch dans}
\end{glose}
\newline
\note{v.t. kibwale}{grammaire}{}
\newline
\note{non vérifié}{général}{}
\end{entrée}

\begin{entrée}{kîbwò}{}{ⓔkîbwò}
\région{GOs}
\variante{%
kîbwòn
\région{PA}}
(\domainesémantique{Parties de plantes})
\classe{nom}
\begin{glose}
\pfra{rejet (de plante)}
\end{glose}
\newline
\begin{sous-entrée}{kîbwò phuleng}{ⓔkîbwòⓝkîbwò phuleng}
\begin{glose}
\pfra{rejet de 'phuleng'}
\end{glose}
\newline
\relationsémantique{Cf.}{\lien{ⓔkhoriing}{khoriing}}
\glosecourte{rejet d'arbuste taillé}
\end{sous-entrée}
\end{entrée}

\begin{entrée}{kîbwoo-nu}{}{ⓔkîbwoo-nu}
\formephonétique{kîbwoːɳu}
\région{GOs}
(\domainesémantique{Cocotiers})
\classe{nom}
\begin{glose}
\pfra{germe du coco}
\end{glose}
\end{entrée}

\begin{entrée}{kica}{}{ⓔkica}
\région{BO}
\classe{nom}
\begin{glose}
\pfra{paroi ; mur}
\end{glose}
\newline
\begin{exemple}
\textbf{\pnua{kica mwa}}
\pfra{les murs de la maison}
\end{exemple}
\newline
\note{non-vérifié}{général}{}
\end{entrée}

\begin{entrée}{kiga}{}{ⓔkiga}
\région{GOs WEM PA BO}
(\domainesémantique{Fonctions naturelles humaines})
\classe{v ; n}
\begin{glose}
\pfra{rire ; sourire}
\end{glose}
\newline
\begin{exemple}
\région{BO}
\textbf{\pnua{yu kigae da ?}}
\pfra{de quoi ris-tu ?}
\end{exemple}
\newline
\begin{exemple}
\région{BO}
\textbf{\pnua{i pe-kiga hada}}
\pfra{il rit tout seul}
\end{exemple}
\newline
\begin{sous-entrée}{pha-kiga}{ⓔkigaⓝpha-kiga}
\begin{glose}
\pfra{faire rire}
\end{glose}
\end{sous-entrée}
\end{entrée}

\begin{entrée}{kii}{1}{ⓔkiiⓗ1}
\région{GOs PA BO}
\variante{%
kivi
\région{BO}}
(\domainesémantique{Vêtements, parure})
\classe{nom}
\begin{glose}
\pfra{jupe ; jupon ; manou (femme)}
\end{glose}
\newline
\begin{sous-entrée}{kii tòòmwa}{ⓔkiiⓗ1ⓝkii tòòmwa}
\begin{glose}
\pfra{jupe}
\end{glose}
\newline
\begin{exemple}
\région{GO}
\textbf{\pnua{kii-je}}
\pfra{son manou}
\end{exemple}
\newline
\begin{exemple}
\région{PA BO}
\textbf{\pnua{kii-n}}
\pfra{son manou, sa jupe}
\end{exemple}
\newline
\begin{exemple}
\région{BO}
\textbf{\pnua{kivi-n}}
\pfra{sa jupe}
\end{exemple}
\newline
\begin{exemple}
\région{BO}
\textbf{\pnua{kia kii-n}}
\pfra{il est nu}
\end{exemple}
\end{sous-entrée}
\newline
\étymologie{
\langue{POc}
\étymon{*tipi}
\glosecourte{loincloth}
\auteur{Blust}}
\end{entrée}

\begin{entrée}{kii}{2}{ⓔkiiⓗ2}
\région{GOs PA}
(\domainesémantique{Actions liées aux plantes})
\classe{v}
\begin{glose}
\pfra{tailler (un arbre en cassant les branches à la main et vers le bas)}
\end{glose}
\newline
\begin{exemple}
\textbf{\pnua{e kii mûû-ce}}
\pfra{il casse une tige de fleur}
\end{exemple}
\end{entrée}

\begin{entrée}{kîî}{}{ⓔkîî}
\région{GOs}
(\domainesémantique{Sons, bruits})
\classe{v}
\begin{glose}
\pfra{bruit aigu (qui vrille les oreilles)}
\end{glose}
\newline
\begin{exemple}
\textbf{\pnua{e kîî kii-nu}}
\pfra{ça me vrille les oreilles}
\end{exemple}
\end{entrée}

\begin{entrée}{kiiça}{}{ⓔkiiça}
\formephonétique{kiːdʒa, kiːʒa}
\région{GOs}
\variante{%
kiia
\région{PA BO}, 
khia
\région{BO}}
(\domainesémantique{Sentiments})
\classe{v ; n}
\begin{glose}
\pfra{jaloux (être) ; jalouser ; jalousie}
\end{glose}
\newline
\begin{exemple}
\région{GO}
\textbf{\pnua{e kiiça i je}}
\pfra{il est jaloux de lui}
\end{exemple}
\newline
\begin{exemple}
\région{GO}
\textbf{\pnua{e kiiça ui je}}
\pfra{il est jaloux de / envers lui (Doriane)}
\end{exemple}
\newline
\begin{exemple}
\région{GO}
\textbf{\pnua{e kiiça ui ãbaa-je xo ã ãgu ã}}
\pfra{cet homme est jaloux de son frère (Doriane)}
\end{exemple}
\newline
\begin{exemple}
\région{GO}
\textbf{\pnua{e kiiça ui Pwayili pexa dilii je ?}}
\pfra{est -il jaloux de Pwayili à cause de ses terres ? (Doriane)}
\end{exemple}
\newline
\begin{exemple}
\région{GO}
\textbf{\pnua{e kiiça pexa dilii je ?}}
\pfra{il est jaloux de ses terres ? (Doriane)}
\end{exemple}
\newline
\begin{exemple}
\région{GO}
\textbf{\pnua{Ôô ! e kiiça !}}
\pfra{oui, il en est jaloux ! (Doriane)}
\end{exemple}
\end{entrée}

\begin{entrée}{kîîgaze}{}{ⓔkîîgaze}
\région{GOs}
(\domainesémantique{Noms des plantes})
\classe{nom}
\begin{glose}
\pfra{liane de bord de mer (utilisée pour la pêche au poison)}
\end{glose}
\nomscientifique{Derris trifoliata, Papilionacées}
\end{entrée}

\begin{entrée}{kiil}{}{ⓔkiil}
\région{BO}
(\domainesémantique{Paniers})
\classe{nom}
\begin{glose}
\pfra{panier [Corne]}
\end{glose}
\newline
\begin{sous-entrée}{ki-wèdal}{ⓔkiilⓝki-wèdal}
\begin{glose}
\pfra{panier de fronde}
\end{glose}
\end{sous-entrée}
\newline
\begin{sous-entrée}{ki-rau}{ⓔkiilⓝki-rau}
\begin{glose}
\pfra{panier en osier}
\end{glose}
\newline
\note{non vérifié}{général}{}
\end{sous-entrée}
\end{entrée}

\begin{entrée}{kilaavwi}{}{ⓔkilaavwi}
\formephonétique{kilaːβi}
\région{GOs}
(\domainesémantique{Fonctions intellectuelles})
\classe{v}
\begin{glose}
\pfra{tromper (se) (en parlant)}
\end{glose}
\newline
\begin{exemple}
\région{GO}
\textbf{\pnua{nu kilaavwi}}
\pfra{je me suis trompé}
\end{exemple}
\end{entrée}

\begin{entrée}{kilapuu}{}{ⓔkilapuu}
\région{GOs}
(\domainesémantique{Relations et interaction sociales})
\classe{v}
\begin{glose}
\pfra{provoquer (par la parole ou l'attitude)}
\end{glose}
\newline
\begin{exemple}
\textbf{\pnua{e kilapuu çai nu}}
\pfra{il m'a provoqué}
\end{exemple}
\newline
\begin{sous-entrée}{a-kilapu}{ⓔkilapuuⓝa-kilapu}
\begin{glose}
\pfra{un provocateur}
\end{glose}
\end{sous-entrée}
\end{entrée}

\begin{entrée}{kilè}{}{ⓔkilè}
\région{GOs}
(\domainesémantique{Instruments})
\classe{nom}
\begin{glose}
\pfra{clé}
\end{glose}
\newline
\emprunt{clé (FR)}
\end{entrée}

\begin{entrée}{kilee-ni}{}{ⓔkilee-ni}
\formephonétique{kileːɳi}
\région{GOs}
(\domainesémantique{Outils})
\classe{v.t.}
\begin{glose}
\pfra{fermer à clé}
\end{glose}
\newline
\begin{exemple}
\textbf{\pnua{kile-ni pweemwa !}}
\pfra{ferme la porte à clé !}
\end{exemple}
\newline
\emprunt{clé (FR)}
\end{entrée}

\begin{entrée}{kili}{}{ⓔkili}
\région{GOs BO PA}
(\domainesémantique{Types de maison, architecture de la maison})
\classe{nom}
\begin{glose}
\pfra{alène}
\end{glose}
\newline
\note{alène de bois percé au bout et utilisé pour attacher la paille sur les toits avec des lianes}{glose}{}
\end{entrée}

\begin{entrée}{kilooc}{}{ⓔkilooc}
\région{PA}
(\domainesémantique{Instruments})
\classe{nom}
\begin{glose}
\pfra{cloche}
\end{glose}
\newline
\emprunt{cloche (FR)}
\end{entrée}

\begin{entrée}{kiluu}{}{ⓔkiluu}
\région{GOs BO}
\variante{%
ciluu
\région{PA}}
\newline
\sens{1}
(\domainesémantique{Mouvements ou actions faits avec le corps, les bras, les mains, les pieds})
\classe{v}
\begin{glose}
\pfra{courber (se)}
\end{glose}
\begin{glose}
\pfra{baisser la tête}
\end{glose}
\newline
\sens{2}
(\domainesémantique{Religion, représentations religieuses})
\classe{v}
\begin{glose}
\pfra{prosterner (se)}
\end{glose}
\newline
\begin{exemple}
\textbf{\pnua{ku-kiluu, tre-kiluu}}
\pfra{se prosterner debout, se prosterner assis}
\end{exemple}
\newline
\relationsémantique{Cf.}{\lien{ⓔbwagiloo}{bwagiloo}}
\end{entrée}

\begin{entrée}{kimwado}{}{ⓔkimwado}
\région{GOs WEM WE BO PA}
(\domainesémantique{Insectes})
\classe{nom}
\begin{glose}
\pfra{sauterelle (petite et verte)}
\end{glose}
\end{entrée}

\begin{entrée}{kine}{1}{ⓔkineⓗ1}
\formephonétique{kiɳe}
\région{GOs PA BO}
\variante{%
kin
\région{BO}}
(\domainesémantique{Verbes d'action (en général)})
\classe{v}
\begin{glose}
\pfra{ajouter ; allonger ; assembler}
\end{glose}
\newline
\begin{sous-entrée}{pe-kine-ni}{ⓔkineⓗ1ⓝpe-kine-ni}
\begin{glose}
\pfra{mettre bout à bout}
\end{glose}
\newline
\note{kine-ni}{grammaire}{assembler qqch}
\end{sous-entrée}
\end{entrée}

\begin{entrée}{kine}{2}{ⓔkineⓗ2}
\formephonétique{kiɳe}
\région{GOs PA}
\variante{%
khine
\région{GO}}
(\domainesémantique{Chasse})
\classe{v}
\begin{glose}
\pfra{viser ; pointer}
\end{glose}
\newline
\begin{exemple}
\région{GO}
\textbf{\pnua{e kine ã meni, ço e pa-tha}}
\pfra{il a visé cet oiseau, et il l'a raté}
\end{exemple}
\newline
\begin{exemple}
\textbf{\pnua{khine-zooni drube !}}
\pfra{vise bien le cerf !}
\end{exemple}
\end{entrée}

\begin{entrée}{kîni}{}{ⓔkîni}
\formephonétique{kiɳi}
\région{GOs}
\variante{%
khînî
\région{PA BO}}
\classe{v.t.}
\newline
\sens{1}
(\domainesémantique{Feu : objets et actions liés au feu})
\begin{glose}
\pfra{brûler (brousse) ; incendier}
\end{glose}
\newline
\sens{2}
(\domainesémantique{Préparation des aliments; modes de préparation et de cuisson})
\begin{glose}
\pfra{cuire/griller sur le feu (sur les braises ou dans les cendres)}
\end{glose}
\newline
\begin{exemple}
\textbf{\pnua{co kî ko !}}
\pfra{fais griller le poulet !}
\end{exemple}
\newline
\begin{exemple}
\région{PA}
\textbf{\pnua{i khî kui}}
\pfra{elle fait griller l'igname}
\end{exemple}
\newline
\begin{sous-entrée}{ku kîni}{ⓔkîniⓢ2ⓝku kîni}
\begin{glose}
\pfra{igname grillée}
\end{glose}
\end{sous-entrée}
\newline
\étymologie{
\langue{POc}
\étymon{*tunu}}
\end{entrée}

\begin{entrée}{kinõ}{}{ⓔkinõ}
\région{PA BO}
\variante{%
khinõ
\région{WE}}
(\domainesémantique{Fonctions naturelles humaines})
\classe{v ; n}
\begin{glose}
\pfra{transpirer ; transpiration ; sueur}
\end{glose}
\newline
\begin{exemple}
\région{PA}
\textbf{\pnua{hai kinõ}}
\pfra{transpirer}
\end{exemple}
\newline
\begin{exemple}
\région{PA}
\textbf{\pnua{tho kinõ}}
\pfra{transpirer (lit. la chaleur coule)}
\end{exemple}
\newline
\étymologie{
\langue{POc}
\étymon{*tunu}
\glosecourte{chaud}}
\newline
\note{kinõ-n}{grammaire}{sa sueur}
\end{entrée}

\begin{entrée}{kînu}{}{ⓔkînu}
\région{GOs BO}
(\domainesémantique{Navigation})
\classe{nom}
\begin{glose}
\pfra{ancre}
\end{glose}
\end{entrée}

\begin{entrée}{kiò}{}{ⓔkiò}
\région{GOs PA BO}
(\domainesémantique{Corps humain})
\classe{nom}
\begin{glose}
\pfra{ventre}
\end{glose}
\newline
\begin{exemple}
\région{GO}
\textbf{\pnua{kiò-je}}
\pfra{son ventre}
\end{exemple}
\newline
\begin{exemple}
\région{PA BO}
\textbf{\pnua{kiò-n}}
\pfra{son ventre}
\end{exemple}
\newline
\étymologie{
\langue{POc}
\étymon{*tia(n)}}
\end{entrée}

\begin{entrée}{kiri}{}{ⓔkiri}
\région{PA BO}
(\domainesémantique{Mouvements ou actions faits avec le corps, les bras, les mains, les pieds})
\classe{v}
\begin{glose}
\pfra{ratisser}
\end{glose}
\begin{glose}
\pfra{gratter (la terre, comme les poules)}
\end{glose}
\newline
\begin{exemple}
\textbf{\pnua{i kiri ja}}
\pfra{elle ratisse les saletés}
\end{exemple}
\newline
\begin{exemple}
\textbf{\pnua{la kiri dili}}
\pfra{elles ratisse la terre}
\end{exemple}
\end{entrée}

\begin{entrée}{kiriket}{}{ⓔkiriket}
\région{GOs}
(\domainesémantique{Jeux divers})
\classe{nom}
\begin{glose}
\pfra{cricket}
\end{glose}
\newline
\emprunt{cricket (GB)}
\end{entrée}

\begin{entrée}{kitrabwi}{}{ⓔkitrabwi}
\région{GOs}
(\domainesémantique{Relations et interaction sociales})
\classe{v}
\begin{glose}
\pfra{accueillir ; recevoir}
\end{glose}
\end{entrée}

\begin{entrée}{kivwa}{}{ⓔkivwa}
\formephonétique{kiβa}
\région{GOs}
\variante{%
kipa
\région{GO}, 
kivha
\région{PA BO}}
(\domainesémantique{Instruments})
\classe{nom}
\begin{glose}
\pfra{fermeture ; couvercle}
\end{glose}
\newline
\begin{exemple}
\région{PA}
\textbf{\pnua{kivha döö}}
\pfra{couvercle de marmite}
\end{exemple}
\newline
\begin{exemple}
\région{BO}
\textbf{\pnua{kivha-n}}
\pfra{son couvercle}
\end{exemple}
\newline
\begin{sous-entrée}{kivwa burei}{ⓔkivwaⓝkivwa burei}
\begin{glose}
\pfra{bouchon de bouteille}
\end{glose}
\end{sous-entrée}
\newline
\begin{sous-entrée}{kivwa phweemwa}{ⓔkivwaⓝkivwa phweemwa}
\begin{glose}
\pfra{porte}
\end{glose}
\newline
\note{kivwi (v.t.)}{grammaire}{fermer}
\end{sous-entrée}
\newline
\relationsémantique{Cf.}{\lien{ⓔthôni}{thôni}}
\glosecourte{fermer à clé}
\newline
\relationsémantique{Ant.}{\lien{ⓔthala}{thala}}
\glosecourte{ouvrir}
\end{entrée}

\begin{entrée}{kivwi}{}{ⓔkivwi}
\formephonétique{kiβi}
\région{GOs}
\variante{%
kivhi
\région{PA BO}}
\classe{v}
\newline
\sens{1}
(\domainesémantique{Verbes d'action (en général)})
\begin{glose}
\pfra{fermer ; boucher ; couvrir (boîte, marmite)}
\end{glose}
\newline
\begin{exemple}
\région{GO}
\textbf{\pnua{khivwi pwaa-jö !}}
\pfra{ferme ta bouche ! tais-toi !}
\end{exemple}
\newline
\begin{exemple}
\région{PA}
\textbf{\pnua{kivhi phwaa-m !}}
\pfra{tais-toi (momentanément)}
\end{exemple}
\newline
\begin{exemple}
\région{PA}
\textbf{\pnua{kivhi döö !}}
\pfra{couvre la marmite !}
\end{exemple}
\newline
\sens{2}
\begin{glose}
\pfra{empêcher de (parler)}
\end{glose}
\newline
\begin{exemple}
\région{GO}
\textbf{\pnua{e kivwi-nu vwo kebwa ne nu vhaa !}}
\pfra{il m'a empêché de parler !}
\end{exemple}
\end{entrée}

\begin{entrée}{kixa}{}{ⓔkixa}
\région{GOs PA}
\variante{%
kiga
\région{GO(s)}, 
kia
\région{PA BO}, 
kiaxa, cixa
\région{PA}}
(\domainesémantique{Négation existentielle})
\classe{PRED.NEG}
\begin{glose}
\pfra{il n'y a pas ; rien ; sans}
\end{glose}
\newline
\begin{exemple}
\région{GO}
\textbf{\pnua{kixa na nu ceni}}
\pfra{je n'ai rien à manger}
\end{exemple}
\newline
\begin{exemple}
\région{PA}
\textbf{\pnua{e ru kixa mwa gèè i yu}}
\pfra{tu n'auras plus de grand-mère}
\end{exemple}
\newline
\begin{exemple}
\région{PA}
\textbf{\pnua{cixa hovo}}
\pfra{rien à manger}
\end{exemple}
\newline
\begin{exemple}
\région{BO}
\textbf{\pnua{kia pu-n}}
\pfra{sans raison (Dubois)}
\end{exemple}
\newline
\begin{exemple}
\région{BO}
\textbf{\pnua{kia poo le}}
\pfra{il n'y a rien (BM)}
\end{exemple}
\newline
\begin{sous-entrée}{kia gunan}{ⓔkixaⓝkia gunan}
\begin{glose}
\pfra{calme plat, vide}
\end{glose}
\end{sous-entrée}
\newline
\begin{sous-entrée}{kia kiin}{ⓔkixaⓝkia kiin}
\begin{glose}
\pfra{nudité sexuelle}
\end{glose}
\end{sous-entrée}
\newline
\begin{sous-entrée}{kia poin}{ⓔkixaⓝkia poin}
\begin{glose}
\pfra{stérilité (femme)}
\end{glose}
\end{sous-entrée}
\newline
\begin{sous-entrée}{kia puun}{ⓔkixaⓝkia puun}
\begin{glose}
\pfra{faire inutilement}
\end{glose}
\end{sous-entrée}
\newline
\begin{sous-entrée}{kia yalan}{ⓔkixaⓝkia yalan}
\begin{glose}
\pfra{sans nom}
\end{glose}
\newline
\relationsémantique{Ant.}{\lien{}{hai; haivo}}
\glosecourte{il y a}
\end{sous-entrée}
\end{entrée}

\begin{entrée}{kixa hê}{}{ⓔkixa hê}
\région{GOs}
\classe{LOCUT}
(\domainesémantique{Description des objets, formes, consistance, taille})
\begin{glose}
\pfra{vide (lit. pas de contenu)}
\end{glose}
\newline
\begin{exemple}
\textbf{\pnua{kixa hê kee-nu}}
\pfra{mon panier est vide}
\end{exemple}
\newline
\relationsémantique{Ant.}{\lien{ⓔpu hê}{pu hê}}
\glosecourte{qui a un contenuu}
\end{entrée}

\begin{entrée}{kixa khôôme}{}{ⓔkixa khôôme}
\région{GOs}
(\domainesémantique{Aspect
, Modalité, verbes modaux})
\classe{RESTR ; ASP}
\begin{glose}
\pfra{ne faire que ; n'avoir de cesse que}
\end{glose}
\begin{glose}
\pfra{n'avoir jamais assez de}
\end{glose}
\newline
\begin{exemple}
\région{GO}
\textbf{\pnua{kixa khôôme nye mããni jena !}}
\pfra{il ne fait que dormir ! il est toujours en train de dormir celui-là (jamais assez)}
\end{exemple}
\newline
\begin{exemple}
\région{GO}
\textbf{\pnua{kixa khôôme nye hovwo jena !}}
\pfra{il ne fait que manger ! il est toujours en train de manger celui-là (pas rassasié)}
\end{exemple}
\newline
\relationsémantique{Cf.}{\lien{ⓔkhôôme}{khôôme}}
\glosecourte{rassasié, comblé}
\end{entrée}

\begin{entrée}{kixa mwa}{}{ⓔkixa mwa}
\région{GOs}
(\domainesémantique{Négation existentielle})
\classe{v}
\begin{glose}
\pfra{il n'y a plus}
\end{glose}
\end{entrée}

\begin{entrée}{kixa na}{}{ⓔkixa na}
\formephonétique{kiɣa ɳa}
\région{GOs}
\variante{%
kiaxa ne
\région{PA}}
(\domainesémantique{Négation existentielle})
\classe{NEG}
\begin{glose}
\pfra{personne (il n'y a) ; rien (lit. il n'y a pas de x que ...)}
\end{glose}
\newline
\begin{exemple}
\région{GO}
\textbf{\pnua{kixa na nu huu}}
\pfra{je n'ai rien mangé}
\end{exemple}
\newline
\begin{exemple}
\région{GO}
\textbf{\pnua{kixa na nu noo-je}}
\pfra{je n'ai vu personne}
\end{exemple}
\newline
\begin{exemple}
\région{GO}
\textbf{\pnua{kixa na la yuu}}
\pfra{personne parmi eux ne reste}
\end{exemple}
\newline
\begin{exemple}
\région{GO}
\textbf{\pnua{kixa na iyô na tre-yuu}}
\pfra{personne parmi nous ne reste}
\end{exemple}
\newline
\begin{exemple}
\région{PA}
\textbf{\pnua{kawa li tone-xa ne e caxo}}
\pfra{ils n'entendent personne qui répond}
\end{exemple}
\newline
\begin{exemple}
\région{PA}
\textbf{\pnua{kixa ne i a}}
\pfra{personne ne part}
\end{exemple}
\end{entrée}

\begin{entrée}{kixa zòò}{}{ⓔkixa zòò}
\région{GOs}
(\domainesémantique{Description des objets, formes, consistance, taille})
\classe{v}
\begin{glose}
\pfra{facile}
\end{glose}
\newline
\relationsémantique{Ant.}{\lien{}{pu zòò [GOs]}}
\glosecourte{difficile}
\end{entrée}

\begin{entrée}{kixi}{}{ⓔkixi}
\formephonétique{kiɣi}
\région{GOs}
(\domainesémantique{Sons, bruits})
\classe{v}
\begin{glose}
\pfra{crier}
\end{glose}
\end{entrée}

\begin{entrée}{kiya ai-n}{}{ⓔkiya ai-n}
\région{PA}
(\domainesémantique{Caractéristiques et propriétés des animaux})
\classe{v}
\begin{glose}
\pfra{pas dressé}
\end{glose}
\newline
\begin{exemple}
\textbf{\pnua{kiya ai-n}}
\pfra{il n'est pas dressé}
\end{exemple}
\newline
\relationsémantique{Ant.}{\lien{ⓔthu ai-n}{thu ai-n}}
\glosecourte{dressé}
\end{entrée}

\begin{entrée}{ko}{1}{ⓔkoⓗ1}
\région{GOs BO PA}
\classe{nom}
(\domainesémantique{Oiseaux})
\newline
\sens{1}
\begin{glose}
\pfra{poule ; coq ; volaille}
\end{glose}
\newline
\begin{exemple}
\région{BO}
\textbf{\pnua{ko èmwèn}}
\pfra{coq}
\end{exemple}
\newline
\sens{2}
\begin{glose}
\pfra{cagou}
\end{glose}
\end{entrée}

\begin{entrée}{ko}{2}{ⓔkoⓗ2}
\région{GOs}
\variante{%
xo, ka
\région{GO(s)}}
(\domainesémantique{Conjonction})
\classe{COORD}
\begin{glose}
\pfra{et}
\end{glose}
\end{entrée}

\begin{entrée}{ko}{3}{ⓔkoⓗ3}
\région{PA GO}
\variante{%
xo, o, ku, u
\région{PA}}
(\domainesémantique{Agent})
\classe{sujet ; AGT}
\begin{glose}
\pfra{agent}
\end{glose}
\newline
\begin{exemple}
\textbf{\pnua{i kubu ije ko kâbwa}}
\pfra{le dieu le frappe}
\end{exemple}
\end{entrée}

\begin{entrée}{ko}{4}{ⓔkoⓗ4}
\région{GOs}
\variante{%
kavwö
}
(\domainesémantique{Négation})
\classe{NEG}
\begin{glose}
\pfra{ne ... pas}
\end{glose}
\newline
\begin{exemple}
\région{GO}
\textbf{\pnua{e wã mwã xo Kaawo : "ko (= kawa, kavwö) jö nooli poi-nu ?"}}
\pfra{Kaawo fait/dit : "tu n'as pas vu mon enfant ?"}
\end{exemple}
\newline
\note{forme courte de NEG kavwö, kawa}{grammaire}{}
\end{entrée}

\begin{entrée}{kò}{1}{ⓔkòⓗ1}
\formephonétique{kɔ}
\région{GOs BO}
(\domainesémantique{Arbre})
\classe{nom}
\begin{glose}
\pfra{palétuvier (à fruit comestible)}
\end{glose}
\nomscientifique{Bruguiera sp.}
\newline
\étymologie{
\langue{POc}
\étymon{*toŋoR}}
\end{entrée}

\begin{entrée}{kò}{2}{ⓔkòⓗ2}
\formephonétique{kɔ}
\région{GOs PA BO}
(\domainesémantique{Végétation})
\classe{nom}
\begin{glose}
\pfra{forêt ; brousse}
\end{glose}
\begin{glose}
\pfra{cimetière (dans la forêt)}
\end{glose}
\newline
\begin{exemple}
\région{GO}
\textbf{\pnua{nò kò}}
\pfra{dans la brousse, la forêt}
\end{exemple}
\newline
\étymologie{
\langue{POc}
\étymon{*quta(n)}}
\end{entrée}

\begin{entrée}{kò}{3}{ⓔkòⓗ3}
\formephonétique{kɔ}
\région{GOs BO PA}
\classe{nom}
\newline
\sens{1}
(\domainesémantique{Corps humain})
\begin{glose}
\pfra{pied ; jambe}
\end{glose}
\newline
\begin{exemple}
\région{GO}
\textbf{\pnua{kòò-nu}}
\pfra{mon pied}
\end{exemple}
\newline
\begin{exemple}
\région{PA}
\textbf{\pnua{kòò-n}}
\pfra{son pied}
\end{exemple}
\newline
\begin{exemple}
\région{PA}
\textbf{\pnua{kòò-m !}}
\pfra{debout !}
\end{exemple}
\newline
\begin{sous-entrée}{kò-pazalô}{ⓔkòⓗ3ⓢ1ⓝkò-pazalô}
\begin{glose}
\pfra{jambe du pantalon}
\end{glose}
\end{sous-entrée}
\newline
\begin{sous-entrée}{kò-uva}{ⓔkòⓗ3ⓢ1ⓝkò-uva}
\begin{glose}
\pfra{extrémité inférieure du taro}
\end{glose}
\end{sous-entrée}
\newline
\begin{sous-entrée}{kò-manyô}{ⓔkòⓗ3ⓢ1ⓝkò-manyô}
\begin{glose}
\pfra{pied, bouture de manioc}
\end{glose}
\end{sous-entrée}
\newline
\sens{2}
(\domainesémantique{Outils})
\begin{glose}
\pfra{manche}
\end{glose}
\newline
\begin{exemple}
\textbf{\pnua{kòò-piòò}}
\pfra{le manche de la hache}
\end{exemple}
\end{entrée}

\begin{entrée}{-kò}{}{ⓔ-kò}
\formephonétique{kɔ}
\région{GOs}
\variante{%
-kòn
\région{PA BO}}
(\domainesémantique{Numéraux cardinaux})
\classe{NUM}
\begin{glose}
\pfra{trois}
\end{glose}
\newline
\étymologie{
\langue{POc}
\étymon{*tolu}}
\end{entrée}

\begin{entrée}{kô-}{1}{ⓔkô-ⓗ1}
\région{GOs PA BO}
(\domainesémantique{Préfixes et verbes de position
, Préfixes sémantiques de position})
\classe{PREF (indiquant une position couchée)}
\begin{glose}
\pfra{couché}
\end{glose}
\newline
\begin{sous-entrée}{kô-xea}{ⓔkô-ⓗ1ⓝkô-xea}
\région{GOs PA}
\begin{glose}
\pfra{s'allonger, se reposer, faire un petit somme}
\end{glose}
\end{sous-entrée}
\newline
\begin{sous-entrée}{kô-wãga}{ⓔkô-ⓗ1ⓝkô-wãga}
\begin{glose}
\pfra{couché les jambes ou pattes écartées}
\end{glose}
\end{sous-entrée}
\newline
\begin{sous-entrée}{kô-pa-ce-bo}{ⓔkô-ⓗ1ⓝkô-pa-ce-bo}
\begin{glose}
\pfra{couché près du feu}
\end{glose}
\end{sous-entrée}
\newline
\begin{sous-entrée}{kô-phaa-ce-bon}{ⓔkô-ⓗ1ⓝkô-phaa-ce-bon}
\région{WEM}
\begin{glose}
\pfra{dormir auprès du feu (la nuit)}
\end{glose}
\end{sous-entrée}
\newline
\begin{sous-entrée}{e kô-alaxè}{ⓔkô-ⓗ1ⓝe kô-alaxè}
\région{GO}
\begin{glose}
\pfra{il est couché sur le côté}
\end{glose}
\end{sous-entrée}
\newline
\begin{sous-entrée}{e kô-bazòò}{ⓔkô-ⓗ1ⓝe kô-bazòò}
\région{GO}
\begin{glose}
\pfra{il est couché en travers de l'entrée}
\end{glose}
\end{sous-entrée}
\newline
\begin{sous-entrée}{la kô-goonã}{ⓔkô-ⓗ1ⓝla kô-goonã}
\région{GO}
\begin{glose}
\pfra{ils font la sieste (ou) la grasse matinée (allongé-midi)}
\end{glose}
\newline
\begin{exemple}
\région{GO}
\textbf{\pnua{e mani-da bwaa-je no mwa xo kòò-je du ni phwee-mwa}}
\pfra{il dort la tête vers l'intérieur de la maison et les pieds vers la porte}
\end{exemple}
\newline
\begin{exemple}
\région{GO}
\textbf{\pnua{e kô-zugi kòò-je}}
\pfra{il est couché les jambes repliées}
\end{exemple}
\newline
\begin{exemple}
\région{PA}
\textbf{\pnua{nu kô-nòòli tèèn}}
\pfra{je me suis réveillé de bonne heure}
\end{exemple}
\newline
\begin{exemple}
\textbf{\pnua{kô-baaxòl ??}}
\pfra{couché en long}
\end{exemple}
\end{sous-entrée}
\newline
\étymologie{
\langue{POc}
\étymon{*qenop}}
\end{entrée}

\begin{entrée}{kô-}{2}{ⓔkô-ⓗ2}
\région{GOs}
(\domainesémantique{Préfixes classificateurs sémantiques})
\classe{nom}
\begin{glose}
\pfra{préfixe des boutures de plante (lianes)}
\end{glose}
\newline
\begin{exemple}
\textbf{\pnua{êê-nu kô-kumwala}}
\pfra{mes boutures de patate douce}
\end{exemple}
\newline
\begin{sous-entrée}{kô-kumwala}{ⓔkô-ⓗ2ⓝkô-kumwala}
\begin{glose}
\pfra{bouture de patate douce}
\end{glose}
\end{sous-entrée}
\newline
\étymologie{
\langue{POc}
\étymon{*qulu}
\glosecourte{tête, sommet}}
\end{entrée}

\begin{entrée}{kôa}{}{ⓔkôa}
\région{GOs PA BO}
\variante{%
kôya, koeza
\région{GO(s)}}
(\domainesémantique{Parties du corps humain : doigts, orteil})
\classe{nom}
\begin{glose}
\pfra{majeur (doigt)}
\end{glose}
\end{entrée}

\begin{entrée}{ko-a dròò-nu}{}{ⓔko-a dròò-nu}
\région{GOs PA}
(\domainesémantique{Cocotiers})
\classe{nom}
\begin{glose}
\pfra{nervure centrale de la palme de cocotier}
\end{glose}
\end{entrée}

\begin{entrée}{kô-alaxe}{}{ⓔkô-alaxe}
\formephonétique{kõ-'alaɣe}
\région{GOs}
(\domainesémantique{Fonctions naturelles humaines})
\classe{v}
\begin{glose}
\pfra{dormir sur le côté ; couché sur le côté}
\end{glose}
\end{entrée}

\begin{entrée}{kô-bazòò}{}{ⓔkô-bazòò}
\région{GOs BO}
(\domainesémantique{Préfixes et verbes de position})
\classe{v}
\begin{glose}
\pfra{en travers; couché en travers (de l'entrée, d'un lit, etc.)}
\end{glose}
\newline
\begin{exemple}
\textbf{\pnua{e kô-bazòò loto bwa dè}}
\pfra{la voiture est de travers sur la route}
\end{exemple}
\newline
\begin{exemple}
\textbf{\pnua{ne pu kô-bazòò}}
\pfra{mets-le en travers}
\end{exemple}
\end{entrée}

\begin{entrée}{kobwako}{}{ⓔkobwako}
\région{GOs}
(\domainesémantique{Parenté})
\classe{nom}
\begin{glose}
\pfra{cadet (lit. debout sur le pied)}
\end{glose}
\newline
\begin{exemple}
\textbf{\pnua{e kobwakoo-je}}
\pfra{il est son cadet}
\end{exemple}
\end{entrée}

\begin{entrée}{kò bwa me}{}{ⓔkò bwa me}
\région{GOs}
\variante{%
a ni dòn i ègu
\région{PA}}
(\domainesémantique{Relations et interaction sociales})
\classe{v}
\begin{glose}
\pfra{faire remarquer (se) ; se mettre en évidence}
\end{glose}
\newline
\begin{exemple}
\région{GO}
\textbf{\pnua{e kò bwa me-ã}}
\pfra{il se met en évidence}
\end{exemple}
\newline
\begin{exemple}
\région{PA}
\textbf{\pnua{i a-da ni dòn i ègu}}
\pfra{il se fait remarquer, il passe au milieu des gens}
\end{exemple}
\end{entrée}

\begin{entrée}{kò-çãńã}{}{ⓔkò-çãńã}
\formephonétique{kɔ-dʒɛ̃nɛ̃}
\région{GOs}
\variante{%
kò-cãnã
\formephonétique{kɔ-cɛ̃nɛ̃}
\région{PA}}
(\domainesémantique{Relations et interaction sociales})
\classe{v}
\begin{glose}
\pfra{insister ; demander avec insistance ; persister à (sens négatif) ; s'entêter}
\end{glose}
\end{entrée}

\begin{entrée}{kô-chòvwa}{}{ⓔkô-chòvwa}
\région{GOs}
(\domainesémantique{Cordes, cordages})
\classe{nom}
\begin{glose}
\pfra{lanière du cheval}
\end{glose}
\end{entrée}

\begin{entrée}{koe}{1}{ⓔkoeⓗ1}
\région{GOs}
(\domainesémantique{Noms des plantes})
\classe{nom}
\begin{glose}
\pfra{pois d'angole ; Ambrevade}
\end{glose}
\nomscientifique{Cajanus indicus}
\end{entrée}

\begin{entrée}{koe}{2}{ⓔkoeⓗ2}
\région{GOs}
(\domainesémantique{Fonctions intellectuelles})
\classe{v}
\begin{glose}
\pfra{inventer (chant, histoire) ; créer}
\end{glose}
\newline
\begin{exemple}
\textbf{\pnua{wa xa e daa koe}}
\pfra{chanson qu'il a inventée lui-même}
\end{exemple}
\end{entrée}

\begin{entrée}{koè}{}{ⓔkoè}
\région{BO}
\classe{n ; v}
\begin{glose}
\pfra{venger (se); vengeance [Corne]}
\end{glose}
\newline
\note{non vérifié}{général}{}
\end{entrée}

\begin{entrée}{köe}{}{ⓔköe}
\région{GOs}
\variante{%
khoe
\région{BO [BM]}}
\classe{v}
\newline
\sens{1}
(\domainesémantique{Actions liées aux plantes})
\begin{glose}
\pfra{cercler (arbre) ; tailler (haie)}
\end{glose}
\newline
\sens{2}
(\domainesémantique{Verbes d'action faite par des animaux})
\begin{glose}
\pfra{castrer (animal)}
\end{glose}
\end{entrée}

\begin{entrée}{kô-e}{}{ⓔkô-e}
\région{GOs BO PA}
(\domainesémantique{Préfixes et verbes de position})
\classe{v}
\begin{glose}
\pfra{couché en tenant qqch dans les bras}
\end{glose}
\newline
\begin{exemple}
\région{GO}
\textbf{\pnua{e kô-e ẽnõ}}
\pfra{il est allongé avec l'enfant dans ses bras}
\end{exemple}
\newline
\begin{exemple}
\région{PA}
\textbf{\pnua{i kô-e pòi-n}}
\pfra{elle est allongée avec son enfant dans les bras}
\end{exemple}
\end{entrée}

\begin{entrée}{koèn}{}{ⓔkoèn}
\région{PA BO}
\variante{%
koèèn,kwèèn
\région{BO}, 
khoeo
\région{BO}}
(\domainesémantique{Verbes d'action (en général)})
\classe{v}
\begin{glose}
\pfra{disparaître ; perdre ; perdu ; absent}
\end{glose}
\newline
\begin{exemple}
\région{PA}
\textbf{\pnua{koèn-xa ãbaa wony}}
\pfra{certains bateaux ont disparu}
\end{exemple}
\newline
\begin{exemple}
\région{PA}
\textbf{\pnua{koèn-xa ãbaa êgu}}
\pfra{certaines personnes sont absentes}
\end{exemple}
\newline
\begin{exemple}
\région{PA}
\textbf{\pnua{koèn ala-kòò-ny}}
\pfra{mes souliers sont perdus}
\end{exemple}
\newline
\begin{exemple}
\région{PA}
\textbf{\pnua{i pha-koène ala-kòò-ny}}
\pfra{il a perdu mes souliers}
\end{exemple}
\newline
\begin{exemple}
\région{BO}
\textbf{\pnua{i koèn a nòòla-ny}}
\pfra{j'ai perdu mon argent}
\end{exemple}
\newline
\begin{exemple}
\région{BO}
\textbf{\pnua{u koèn}}
\pfra{c'est perdu}
\end{exemple}
\newline
\relationsémantique{Cf.}{\lien{}{kòyò, kòi [GOs]}}
\glosecourte{perdre}
\newline
\relationsémantique{Cf.}{\lien{ⓔkòiⓗ1}{kòi}}
\glosecourte{absent}
\end{entrée}

\begin{entrée}{koe-piça-ni}{}{ⓔkoe-piça-ni}
\formephonétique{kɔe-pidʒa-ɳi}
\région{GOs}
(\domainesémantique{Mouvements ou actions faits avec le corps, les bras, les mains, les pieds})
\classe{v}
\begin{glose}
\pfra{tenir fermement qqch}
\end{glose}
\end{entrée}

\begin{entrée}{kôgò}{}{ⓔkôgò}
\région{GOs}
(\domainesémantique{Verbes de mouvement})
\classe{v}
\begin{glose}
\pfra{émerger (poisson)}
\end{glose}
\end{entrée}

\begin{entrée}{kôgòò}{}{ⓔkôgòò}
\région{GOs}
\variante{%
kôgò-n
\région{BO PA}, 
kugo
\région{BO PA}}
(\domainesémantique{Quantificateurs})
\classe{nom}
\begin{glose}
\pfra{reste (le) ; restant ; surplus}
\end{glose}
\newline
\begin{sous-entrée}{kôgòò hovwo}{ⓔkôgòòⓝkôgòò hovwo}
\région{GO}
\begin{glose}
\pfra{restes de nourriture}
\end{glose}
\end{sous-entrée}
\newline
\begin{sous-entrée}{kôgòò mwani}{ⓔkôgòòⓝkôgòò mwani}
\région{PA}
\begin{glose}
\pfra{la monnaie (reste d'argent)}
\end{glose}
\end{sous-entrée}
\newline
\begin{sous-entrée}{kôgò-n}{ⓔkôgòòⓝkôgò-n}
\région{PA}
\begin{glose}
\pfra{ce qui lui reste (lit. son reste )}
\end{glose}
\newline
\begin{exemple}
\région{BO}
\textbf{\pnua{gaa mwèènò kôgò-n a hovo}}
\pfra{il reste encore de la nourriture à manger}
\end{exemple}
\end{sous-entrée}
\end{entrée}

\begin{entrée}{kô-goon-a}{}{ⓔkô-goon-a}
\région{GOs}
(\domainesémantique{Fonctions naturelles humaines})
\classe{v}
\begin{glose}
\pfra{faire la sieste}
\end{glose}
\end{entrée}

\begin{entrée}{kòi}{1}{ⓔkòiⓗ1}
\région{GOs BO}
\variante{%
kòe, koi
\région{PA}}
(\domainesémantique{Négation})
\classe{PRED.NEG (humain)}
\begin{glose}
\pfra{absent ; ne pas/plus être là ; manquer ; sans}
\end{glose}
\newline
\begin{exemple}
\région{GO}
\textbf{\pnua{bala kòi-la}}
\pfra{ils ont disparu (à tout jamais)}
\end{exemple}
\newline
\begin{exemple}
\région{GO}
\textbf{\pnua{kòi-je gò !}}
\pfra{il n'est toujours pas là!}
\end{exemple}
\newline
\begin{exemple}
\région{GO}
\textbf{\pnua{kòi-nu}}
\pfra{je suis absent}
\end{exemple}
\newline
\begin{exemple}
\région{GO}
\textbf{\pnua{ge le ãbaa wõ ma la kòi-ò}}
\pfra{certains bateaux ont disparu}
\end{exemple}
\newline
\begin{exemple}
\région{GO}
\textbf{\pnua{ge le la ãbaa ma la kòi-la}}
\pfra{certains d'entre eux ont disparu}
\end{exemple}
\newline
\begin{exemple}
\région{BO}
\textbf{\pnua{gaa kòe ri ?}}
\pfra{qui manque encore ?}
\end{exemple}
\newline
\begin{exemple}
\région{BO}
\textbf{\pnua{la gaa kòi-la ?}}
\pfra{qui manque encore ?}
\end{exemple}
\newline
\begin{exemple}
\région{BO}
\textbf{\pnua{u kòi-yo kònòbòn}}
\pfra{tu avais disparu hier}
\end{exemple}
\newline
\begin{exemple}
\région{GO PA}
\textbf{\pnua{kòe-je}}
\pfra{il est absent, il est mort}
\end{exemple}
\newline
\begin{exemple}
\région{PA}
\textbf{\pnua{koi-n}}
\pfra{il est absent}
\end{exemple}
\newline
\begin{exemple}
\région{BO}
\textbf{\pnua{u kwèi-je}}
\pfra{il est absent}
\end{exemple}
\newline
\relationsémantique{Cf.}{\lien{ⓔkòyò}{kòyò}}
\glosecourte{disparaître; perdu (non-humain)}
\end{entrée}

\begin{entrée}{kòi}{2}{ⓔkòiⓗ2}
\région{GOs PA BO}
(\domainesémantique{Corps humain})
\classe{nom}
\begin{glose}
\pfra{foie}
\end{glose}
\newline
\begin{exemple}
\région{GO}
\textbf{\pnua{kòi dube}}
\pfra{le foie du cerf}
\end{exemple}
\newline
\begin{exemple}
\région{GO}
\textbf{\pnua{kòi-nu}}
\pfra{son foie}
\end{exemple}
\newline
\begin{exemple}
\région{PA BO}
\textbf{\pnua{kòi-n}}
\pfra{son foie}
\end{exemple}
\newline
\étymologie{
\langue{POc}
\étymon{*qate}}
\end{entrée}

\begin{entrée}{ko ia ?}{}{ⓔko ia ?}
\région{GOs}
(\domainesémantique{Interrogatifs})
\classe{INT}
\begin{glose}
\pfra{à quel endroit ?}
\end{glose}
\newline
\begin{exemple}
\région{GO}
\textbf{\pnua{nu a kaze-du- ko ia ? - ko Waambi}}
\pfra{je vais à la pêche - où ? - à Waambi}
\end{exemple}
\end{entrée}

\begin{entrée}{kô-ii}{}{ⓔkô-ii}
\région{GOs}
(\domainesémantique{Fonctions naturelles humaines
, Verbes de mouvement})
\classe{v}
\begin{glose}
\pfra{agiter (s') en dormant}
\end{glose}
\newline
\begin{exemple}
\textbf{\pnua{a kô-ii ẽnõ ã}}
\pfra{cet enfant s'agite en dormant}
\end{exemple}
\newline
\begin{exemple}
\textbf{\pnua{a kô-ii òri ẽnõ ã}}
\pfra{cet enfant s'agite beaucoup en dormant}
\end{exemple}
\end{entrée}

\begin{entrée}{kô-kabòn}{}{ⓔkô-kabòn}
\région{PA}
(\domainesémantique{Corps humain})
\classe{nom}
\begin{glose}
\pfra{vulve}
\end{glose}
\end{entrée}

\begin{entrée}{kò-kai}{}{ⓔkò-kai}
\région{GOs}
\variante{%
kò-xai
\région{GOs}, 
kòò-kain
\région{PA}}
(\domainesémantique{Parenté})
\classe{nom}
\begin{glose}
\pfra{enfants suivant l'aîné (lit. debout derrière)}
\end{glose}
\newline
\begin{exemple}
\région{GO}
\textbf{\pnua{ẽnõ kò-xai}}
\pfra{le 2ème enfant (ou) les enfants suivant l'aîné}
\end{exemple}
\newline
\begin{exemple}
\région{PA}
\textbf{\pnua{ẽnõ kòò-kai-n}}
\pfra{le ou les enfants suivant l'aîné}
\end{exemple}
\newline
\relationsémantique{Cf.}{\lien{}{kòòl kai-n [PA]}}
\glosecourte{debout derrière lui}
\end{entrée}

\begin{entrée}{kô-kea}{}{ⓔkô-kea}
\région{GOs}
\variante{%
kô-kea, kô-xea
\région{PA BO}}
\classe{v}
\newline
\sens{1}
(\domainesémantique{Préfixes et verbes de position})
\begin{glose}
\pfra{incliné ; allonger un peu (s')}
\end{glose}
\newline
\sens{2}
(\domainesémantique{Fonctions naturelles humaines})
\begin{glose}
\pfra{faire un petit somme, faire la sieste}
\end{glose}
\newline
\relationsémantique{Cf.}{\lien{ⓔkea}{kea}}
\glosecourte{incliné}
\end{entrée}

\begin{entrée}{kô-kiai}{}{ⓔkô-kiai}
\région{GOs}
\variante{%
kô-xiai
\région{GO(s)}}
(\domainesémantique{Feu : objets et actions liés au feu})
\classe{v}
\begin{glose}
\pfra{réchauffer (se) couché auprès du feu (la nuit)}
\end{glose}
\newline
\morphologie{forme courte de: kô-khi(ni)-yaai (lit. allongé-chauffer-feu)}
\end{entrée}

\begin{entrée}{kô-kumala}{}{ⓔkô-kumala}
\région{GOs}
\variante{%
kô-kumwãla
\formephonétique{,kõku'm(w)ãla}
\région{GO(s)}}
(\domainesémantique{Cultures, techniques, boutures})
\classe{nom}
\begin{glose}
\pfra{bouture de patate douce}
\end{glose}
\end{entrée}

\begin{entrée}{kòlaao}{}{ⓔkòlaao}
\région{GOs BO}
\variante{%
kòlaao
\région{PA}, 
kòlao; kòlaho
\région{BO}}
\classe{nom}
\newline
\sens{1}
(\domainesémantique{Mollusques})
\begin{glose}
\pfra{conque (gastéropode)}
\end{glose}
\nomscientifique{Tritonis sp.}
\newline
\sens{2}
(\domainesémantique{Types de maison, architecture de la maison})
\begin{glose}
\pfra{conque (de la flèche faîtière)}
\end{glose}
\newline
\note{kawolok 'conque' (langue nêlêmwa)}{général}{}
\end{entrée}

\begin{entrée}{kòladuu}{}{ⓔkòladuu}
\région{GOs}
\variante{%
kòladuun
\région{BO}}
\classe{v.stat.}
(\domainesémantique{Santé, maladie})
\begin{glose}
\pfra{maigre ; maigrir}
\end{glose}
\begin{glose}
\pfra{dépérir}
\end{glose}
\newline
\begin{exemple}
\textbf{\pnua{e kòladuu pagòò je}}
\pfra{elle maigrit}
\end{exemple}
\newline
\relationsémantique{Ant.}{\lien{ⓔwaa}{waa}}
\glosecourte{gros, corpulent}
\end{entrée}

\begin{entrée}{kole}{1}{ⓔkoleⓗ1}
\région{GOs}
\variante{%
kule
\région{PA BO}}
(\domainesémantique{Verbes de mouvement})
\classe{v}
\begin{glose}
\pfra{tomber}
\end{glose}
\newline
\begin{exemple}
\région{GO}
\textbf{\pnua{kole pwa}}
\pfra{il pleut}
\end{exemple}
\newline
\begin{exemple}
\région{PA}
\textbf{\pnua{i kule pwal}}
\pfra{il pleut}
\end{exemple}
\newline
\begin{exemple}
\région{PA}
\textbf{\pnua{la daa pe-kule po-mãã}}
\pfra{les mangues tombent toutes d'elles-même (toutes seules)}
\end{exemple}
\newline
\relationsémantique{Cf.}{\lien{}{thrõbo ; kaalu}}
\end{entrée}

\begin{entrée}{kole}{2}{ⓔkoleⓗ2}
\région{GOs BO}
\classe{v}
\newline
\sens{1}
(\domainesémantique{Pêche})
\begin{glose}
\pfra{jeter (filet)}
\end{glose}
\newline
\begin{exemple}
\textbf{\pnua{kole pwiyo}}
\pfra{jeter un filet dans l'eau}
\end{exemple}
\newline
\sens{2}
(\domainesémantique{Actions liées aux éléments (liquide, fumée)})
\begin{glose}
\pfra{vider ; renverser (liquide)}
\end{glose}
\newline
\begin{exemple}
\textbf{\pnua{kole we}}
\pfra{arroser, verser l'eau}
\end{exemple}
\end{entrée}

\begin{entrée}{kõle}{}{ⓔkõle}
\région{BO}
(\domainesémantique{Santé, maladie})
\classe{v}
\begin{glose}
\pfra{avorter}
\end{glose}
\newline
\begin{exemple}
\textbf{\pnua{i kõle ẽno}}
\pfra{elle a avorté [BM]}
\end{exemple}
\newline
\note{non verifié}{général}{}
\end{entrée}

\begin{entrée}{kole-ò}{}{ⓔkole-ò}
\région{BO}
(\domainesémantique{Noms locatifs})
\classe{nom}
\begin{glose}
\pfra{rive de l'autre côté}
\end{glose}
\newline
\note{non vérifié}{général}{}
\end{entrée}

\begin{entrée}{kole pwa}{}{ⓔkole pwa}
\région{GOs}
\variante{%
pwal
\région{PA}}
(\domainesémantique{Phénomènes atmosphériques et naturels})
\classe{v}
\begin{glose}
\pfra{pleuvoir}
\end{glose}
\end{entrée}

\begin{entrée}{kòli}{}{ⓔkòli}
\région{GOs BO PA}
\newline
\sens{1}
(\domainesémantique{Noms locatifs})
\classe{n.LOC}
\begin{glose}
\pfra{bord ; côté ; près de ; au bord de}
\end{glose}
\newline
\begin{exemple}
\région{GO}
\textbf{\pnua{koli jaaò}}
\pfra{la rive de la rivière}
\end{exemple}
\newline
\begin{exemple}
\région{GO}
\textbf{\pnua{kòli we}}
\pfra{la rive, berge}
\end{exemple}
\newline
\begin{exemple}
\région{PA}
\textbf{\pnua{kòli jaòl}}
\pfra{le bord de la rivière}
\end{exemple}
\newline
\sens{2}
(\domainesémantique{Localisation})
\classe{LOC}
\begin{glose}
\pfra{près de ; bord de (au) (très proche)}
\end{glose}
\newline
\begin{exemple}
\textbf{\pnua{kòli we}}
\pfra{près de l'eau}
\end{exemple}
\newline
\begin{exemple}
\région{GO PA}
\textbf{\pnua{kòli we-za}}
\pfra{le bord de mer, au bord de la mer}
\end{exemple}
\newline
\begin{exemple}
\région{GO}
\textbf{\pnua{e a höze kòli we-za}}
\pfra{il suit le bord de la mer}
\end{exemple}
\newline
\begin{exemple}
\région{GO}
\textbf{\pnua{kòli kaze}}
\pfra{au bord de mer}
\end{exemple}
\newline
\relationsémantique{Cf.}{\lien{}{kòlo}}
\glosecourte{à côté (plus loin)}
\end{entrée}

\begin{entrée}{kòli kaze}{}{ⓔkòli kaze}
\région{GOs}
(\domainesémantique{Topographie})
\classe{n ; LOC}
\begin{glose}
\pfra{bord de mer}
\end{glose}
\newline
\begin{exemple}
\textbf{\pnua{nooli kòli kaze (ou) nooli kòli we-za}}
\pfra{regarde le bord de mer}
\end{exemple}
\end{entrée}

\begin{entrée}{kòli we-za}{}{ⓔkòli we-za}
\région{PA}
(\domainesémantique{Topographie})
\classe{LOC}
\begin{glose}
\pfra{bord de mer (au)}
\end{glose}
\end{entrée}

\begin{entrée}{kòlò}{}{ⓔkòlò}
\région{GOs PA BO}
\variante{%
kòlò-n
\région{BO}, 
kòli
\région{GO(s) PA}}
\newline
\sens{1}
(\domainesémantique{Configuration des objets})
\classe{nom}
\begin{glose}
\pfra{côté ; bord ; extrémité ; lisière}
\end{glose}
\begin{glose}
\pfra{flanc}
\end{glose}
\newline
\sens{2}
(\domainesémantique{Localisation})
\classe{n.LOC}
\begin{glose}
\pfra{près de ; à ; chez ; vers ; auprès de ; de l'autre côté de}
\end{glose}
\newline
\begin{exemple}
\région{GO}
\textbf{\pnua{kòlò-jo}}
\pfra{près de toi}
\end{exemple}
\newline
\begin{exemple}
\région{PA}
\textbf{\pnua{kòlò-m}}
\pfra{près de toi}
\end{exemple}
\newline
\begin{exemple}
\région{PA}
\textbf{\pnua{ge je kòlò-n èba ni we}}
\pfra{il est de l'autre côté de la rivière}
\end{exemple}
\newline
\begin{exemple}
\région{GO}
\textbf{\pnua{kòlò phwa}}
\pfra{au bord du trou}
\end{exemple}
\newline
\begin{exemple}
\région{GO}
\textbf{\pnua{e yu kòlò lò}}
\pfra{il vit chez eux}
\end{exemple}
\newline
\begin{sous-entrée}{kòlò-nu}{ⓔkòlòⓢ2ⓝkòlò-nu}
\région{GO}
\begin{glose}
\pfra{chez moi}
\end{glose}
\end{sous-entrée}
\newline
\begin{sous-entrée}{kòlò-ny}{ⓔkòlòⓢ2ⓝkòlò-ny}
\région{PA}
\begin{glose}
\pfra{chez moi}
\end{glose}
\end{sous-entrée}
\newline
\begin{sous-entrée}{kòlò kò}{ⓔkòlòⓢ2ⓝkòlò kò}
\begin{glose}
\pfra{près de la forêt}
\end{glose}
\end{sous-entrée}
\newline
\begin{sous-entrée}{kòlò bo}{ⓔkòlòⓢ2ⓝkòlò bo}
\begin{glose}
\pfra{le bord du ravin}
\end{glose}
\end{sous-entrée}
\newline
\begin{sous-entrée}{kòlò-òli}{ⓔkòlòⓢ2ⓝkòlò-òli}
\begin{glose}
\pfra{vers l'autre côté}
\end{glose}
\end{sous-entrée}
\newline
\begin{sous-entrée}{kòlò dèèn}{ⓔkòlòⓢ2ⓝkòlò dèèn}
\région{PA BO}
\begin{glose}
\pfra{exposé au vent}
\end{glose}
\end{sous-entrée}
\newline
\begin{sous-entrée}{kòlò de}{ⓔkòlòⓢ2ⓝkòlò de}
\région{GO}
\begin{glose}
\pfra{bordure du chemin ; petit talus en bordure du chemin}
\end{glose}
\end{sous-entrée}
\newline
\begin{sous-entrée}{kòlò den}{ⓔkòlòⓢ2ⓝkòlò den}
\région{PA}
\begin{glose}
\pfra{bordure du chemin ; petit talus en bordure du chemin}
\end{glose}
\end{sous-entrée}
\newline
\begin{sous-entrée}{kòlò pwèmwa}{ⓔkòlòⓢ2ⓝkòlò pwèmwa}
\région{PA}
\begin{glose}
\pfra{ouest (litt. côté de la porte)}
\end{glose}
\end{sous-entrée}
\newline
\begin{sous-entrée}{bwa kòlò}{ⓔkòlòⓢ2ⓝbwa kòlò}
\begin{glose}
\pfra{sur le côté}
\end{glose}
\end{sous-entrée}
\newline
\begin{sous-entrée}{kòlò gu-i}{ⓔkòlòⓢ2ⓝkòlò gu-i}
\région{PA}
\begin{glose}
\pfra{à droite}
\end{glose}
\end{sous-entrée}
\newline
\begin{sous-entrée}{kòlò mò}{ⓔkòlòⓢ2ⓝkòlò mò}
\région{PA}
\begin{glose}
\pfra{à gauche}
\end{glose}
\end{sous-entrée}
\newline
\begin{sous-entrée}{kòlò mwawa}{ⓔkòlòⓢ2ⓝkòlò mwawa}
\begin{glose}
\pfra{à droite}
\end{glose}
\end{sous-entrée}
\newline
\sens{3}
(\domainesémantique{Prépositions})
\classe{PREP.BENEF}
\begin{glose}
\pfra{à, pour}
\end{glose}
\newline
\begin{exemple}
\région{PA}
\textbf{\pnua{hulò kòlò Peiva}}
\pfra{mon présent coutumier à Peiva}
\end{exemple}
\newline
\begin{exemple}
\région{GO}
\textbf{\pnua{hulò mèè-nu kòlò Peiva}}
\pfra{mon présent coutumier pour/chez Peiva}
\end{exemple}
\end{entrée}

\begin{entrée}{kòlò ije}{}{ⓔkòlò ije}
\région{BO}
(\domainesémantique{Parenté})
\classe{nom}
\begin{glose}
\pfra{fils du frère (soeur du père parlant) (tantine) [Corne]}
\end{glose}
\end{entrée}

\begin{entrée}{kòlò-khia}{}{ⓔkòlò-khia}
\région{BO}
(\domainesémantique{Cultures, techniques, boutures})
\classe{nom}
\begin{glose}
\pfra{pente du massif d'ignames (Dubois)}
\end{glose}
\newline
\relationsémantique{Cf.}{\lien{ⓔkhiaⓗ1}{khia}}
\glosecourte{massif d'ignames}
\end{entrée}

\begin{entrée}{kòmaze}{}{ⓔkòmaze}
\formephonétique{kɔmaðe}
\région{GOs}
(\domainesémantique{Corps humain})
\classe{nom}
\begin{glose}
\pfra{poumons}
\end{glose}
\end{entrée}

\begin{entrée}{kôme}{}{ⓔkôme}
\région{GOs}
(\domainesémantique{Verbes d'action (en général)})
\classe{v}
\begin{glose}
\pfra{étouffer}
\end{glose}
\newline
\begin{exemple}
\textbf{\pnua{e kôme-nu xo laai}}
\pfra{je me suis étranglée avec le riz}
\end{exemple}
\end{entrée}

\begin{entrée}{kò-mõn}{}{ⓔkò-mõn}
\région{PA}
(\domainesémantique{Verbes de déplacement et moyens de déplacement})
\classe{v}
\begin{glose}
\pfra{approcher}
\end{glose}
\newline
\begin{exemple}
\région{PA}
\textbf{\pnua{kò-mõnu kaò}}
\pfra{la saison des pluies est proche}
\end{exemple}
\end{entrée}

\begin{entrée}{komwãcii}{}{ⓔkomwãcii}
\région{GOs}
(\domainesémantique{Mouvements ou actions faits avec le corps, les bras, les mains, les pieds})
\classe{v}
\begin{glose}
\pfra{chatouiller}
\end{glose}
\end{entrée}

\begin{entrée}{kõmwõgi}{}{ⓔkõmwõgi}
\région{GOs}
\variante{%
kõmõgin
\région{BO}}
(\domainesémantique{Quantificateurs})
\classe{QNT}
\begin{glose}
\pfra{entier}
\end{glose}
\begin{glose}
\pfra{entier ; rond (BO, Dubois)}
\end{glose}
\newline
\begin{exemple}
\région{GO}
\textbf{\pnua{kavwö nu hine kõmwõgi-ni}}
\pfra{I don't know it all (Schooling)}
\end{exemple}
\end{entrée}

\begin{entrée}{kò-na-mi}{}{ⓔkò-na-mi}
\région{PA}
(\domainesémantique{Relations et interaction sociales})
\classe{v}
\begin{glose}
\pfra{donne-moi un peu}
\end{glose}
\end{entrée}

\begin{entrée}{kõnõbwòn}{}{ⓔkõnõbwòn}
\formephonétique{kɔ̃nɔ̃mbwɔn}
\région{PA BO}
\variante{%
kõnõ-bòn
\région{PA BO}}
(\domainesémantique{Adverbes déictiques de temps})
\classe{ADV}
\begin{glose}
\pfra{hier}
\end{glose}
\newline
\begin{exemple}
\région{BO}
\textbf{\pnua{je ka kõnõbwòn}}
\pfra{l'an dernier}
\end{exemple}
\newline
\begin{sous-entrée}{kõnõbòn aeò}{ⓔkõnõbwònⓝkõnõbòn aeò}
\région{GO PA}
\begin{glose}
\pfra{avant-hier}
\end{glose}
\end{sous-entrée}
\newline
\begin{sous-entrée}{kõnõbwòn èhò}{ⓔkõnõbwònⓝkõnõbwòn èhò}
\région{PA}
\begin{glose}
\pfra{avant-hier}
\end{glose}
\newline
\relationsémantique{Cf.}{\lien{ⓔdròòrò}{dròòrò}}
\glosecourte{hier}
\end{sous-entrée}
\end{entrée}

\begin{entrée}{kõnõbwòn èò}{}{ⓔkõnõbwòn èò}
\région{PA BO}
(\domainesémantique{Découpage du temps})
\classe{ADV}
\begin{glose}
\pfra{avant-hier}
\end{glose}
\end{entrée}

\begin{entrée}{kônõ-da}{}{ⓔkônõ-da}
\formephonétique{kõɳɔ̃-nda}
\région{GOs BO}
(\domainesémantique{Préfixes et verbes de position})
\classe{v}
\begin{glose}
\pfra{couché sur le dos}
\end{glose}
\begin{glose}
\pfra{couché la tête vers l'intérieur de la maison}
\end{glose}
\newline
\begin{exemple}
\région{GO}
\textbf{\pnua{e kônõ-da (ni dònò)}}
\pfra{il est couché sur le dos (litt. couché regardant en haut (vers le ciel))}
\end{exemple}
\newline
\begin{exemple}
\région{GO}
\textbf{\pnua{e kônõ-da bwaa-je no mwa}}
\pfra{il est couché la tête vers l'intérieur de la maison}
\end{exemple}
\newline
\begin{exemple}
\région{BO}
\textbf{\pnua{i kônõ-da ni phwa}}
\pfra{il est couché sur le dos [Corne]}
\end{exemple}
\newline
\étymologie{
\langue{POc}
\étymon{*qenop}
\glosecourte{couché}}
\end{entrée}

\begin{entrée}{kônõ-du}{}{ⓔkônõ-du}
\formephonétique{kõɳɔ̃-ndu}
\région{GOs BO PA}
(\domainesémantique{Préfixes et verbes de position})
\classe{v}
\begin{glose}
\pfra{couché sur le ventre;}
\end{glose}
\begin{glose}
\pfra{couché (la tête) vers la porte}
\end{glose}
\newline
\begin{exemple}
\textbf{\pnua{e kônõ-du}}
\pfra{il est couché sur le ventre (litt. couché regardant en bas)}
\end{exemple}
\newline
\begin{exemple}
\région{GO}
\textbf{\pnua{e kônõ-du ni phwee-mwa}}
\pfra{il est couché (la tête) vers la porte}
\end{exemple}
\end{entrée}

\begin{entrée}{kõńõõ}{}{ⓔkõńõõ}
\formephonétique{kɔ̃nɔ̃ː}
\région{GOs}
\variante{%
kònôôl
\région{PA}}
\classe{v}
\newline
\sens{1}
(\domainesémantique{Caractéristiques et propriétés des personnes})
\begin{glose}
\pfra{paresseux (homme)}
\end{glose}
\newline
\begin{exemple}
\textbf{\pnua{êgu xa kõńõõ}}
\pfra{un paresseux}
\end{exemple}
\newline
\begin{exemple}
\textbf{\pnua{me-xõńõõ !}}
\pfra{paresseux !}
\end{exemple}
\newline
\sens{2}
(\domainesémantique{Caractéristiques et propriétés des animaux})
\begin{glose}
\pfra{doux (animal) ; apprivoisé (animal)}
\end{glose}
\newline
\note{pa-kòńòò-ni}{grammaire}{apprivoiser (animal)}
\end{entrée}

\begin{entrée}{kô-nòò}{}{ⓔkô-nòò}
\formephonétique{kõ-ɳɔː}
\région{GOs}
\variante{%
kô-nòi
\région{BO}}
(\domainesémantique{Fonctions naturelles humaines})
\classe{v}
\begin{glose}
\pfra{rêver}
\end{glose}
\newline
\begin{exemple}
\région{GO}
\textbf{\pnua{e kô-nòò}}
\pfra{il rêve}
\end{exemple}
\newline
\begin{exemple}
\région{GO}
\textbf{\pnua{nu kô-nòò}}
\pfra{j'ai fait un rêve}
\end{exemple}
\newline
\begin{exemple}
\région{GO}
\textbf{\pnua{kô-nòò-nu i nyanya}}
\pfra{j'ai rêvé de maman}
\end{exemple}
\newline
\begin{exemple}
\région{BO}
\textbf{\pnua{nu kô-nòi-ny i nyanya}}
\pfra{j'ai rêvé de maman}
\end{exemple}
\end{entrée}

\begin{entrée}{kônôô}{}{ⓔkônôô}
\formephonétique{kõnõː}
\région{WEM WE}
\variante{%
kònôôl
\région{BO}}
(\domainesémantique{Caractéristiques et propriétés des personnes})
\classe{v.stat.}
\begin{glose}
\pfra{maladroit ; gauche ; pas débrouillard}
\end{glose}
\end{entrée}

\begin{entrée}{kô-nòòl}{}{ⓔkô-nòòl}
\région{PA BO}
(\domainesémantique{Fonctions naturelles humaines})
\classe{v}
\begin{glose}
\pfra{rester éveillé}
\end{glose}
\newline
\begin{exemple}
\région{PA}
\textbf{\pnua{nu kô-nòòli tèèn}}
\pfra{je me suis réveillé de bonne heure}
\end{exemple}
\end{entrée}

\begin{entrée}{kô-nõõli tree}{}{ⓔkô-nõõli tree}
\formephonétique{kõ-ɳɔ̃ːli}
\région{GOs}
\variante{%
kô-nõõli tèèn
\région{PA}}
(\domainesémantique{Fonctions naturelles humaines})
\classe{v}
\begin{glose}
\pfra{réveiller (se) tôt}
\end{glose}
\newline
\begin{exemple}
\textbf{\pnua{bi kô-nõõli tree-monõ}}
\pfra{nous nous levions tôt le lendemain matin (lit. voir le jour se lever)}
\end{exemple}
\end{entrée}

\begin{entrée}{kônya}{}{ⓔkônya}
\région{GOs}
\variante{%
kônyal
\région{WEM WE PA}}
(\domainesémantique{Caractéristiques et propriétés des personnes})
\classe{v}
\begin{glose}
\pfra{drôle ; risible ; ridicule}
\end{glose}
\newline
\begin{exemple}
\textbf{\pnua{e a-thu kônya}}
\pfra{il est drôle, il fait rire, il fait le clown}
\end{exemple}
\end{entrée}

\begin{entrée}{kõ-nhyò}{}{ⓔkõ-nhyò}
\région{PA}
(\domainesémantique{Mammifères})
\classe{nom}
\begin{glose}
\pfra{grappe de roussettes}
\end{glose}
\newline
\relationsémantique{Cf.}{\lien{}{kõ}}
\glosecourte{corde}
\end{entrée}

\begin{entrée}{kòò}{}{ⓔkòò}
\formephonétique{kɔː}
\région{GOs}
\variante{%
kòòl,kòl
\région{PA BO WEM}}
\classe{v}
\newline
\sens{1}
(\domainesémantique{Préfixes et verbes de position})
\begin{glose}
\pfra{debout ; debout (être) ; dresser (se) ; mettre debout (se) ; debout (être) immobile}
\end{glose}
\newline
\begin{exemple}
\région{GO}
\textbf{\pnua{e kòò-da}}
\pfra{elle se lève}
\end{exemple}
\newline
\begin{sous-entrée}{pa-kòò-ni}{ⓔkòòⓢ1ⓝpa-kòò-ni}
\begin{glose}
\pfra{mettre debout, dresser}
\end{glose}
\newline
\note{v.t kòòli}{grammaire}{mettre debout, dresser}
\end{sous-entrée}
\newline
\sens{2}
(\domainesémantique{Verbes d'action (en général)})
\begin{glose}
\pfra{arrêter de marcher ; arrêter}
\end{glose}
\newline
\begin{exemple}
\région{GO}
\textbf{\pnua{co kòò ! jö ga kòò gò !}}
\pfra{attends!, arrête-toi !}
\end{exemple}
\newline
\begin{exemple}
\région{PA}
\textbf{\pnua{kòòl !}}
\pfra{arrête-toi !}
\end{exemple}
\newline
\sens{3}
(\domainesémantique{Verbes d'action (en général)})
\begin{glose}
\pfra{attendre}
\end{glose}
\newline
\begin{exemple}
\région{GO}
\textbf{\pnua{cö kòò vwö nu a threi cee}}
\pfra{attends, je vais aller couper du bois}
\end{exemple}
\newline
\étymologie{
\langue{POc}
\étymon{*tuqud}}
\end{entrée}

\begin{entrée}{kôô}{1}{ⓔkôôⓗ1}
\région{GOs}
\variante{%
kô
\région{BO PA}}
(\domainesémantique{Cordes, cordages})
\classe{nom}
\begin{glose}
\pfra{corde ; lien ; chaîne}
\end{glose}
\newline
\begin{sous-entrée}{kô-pwe}{ⓔkôôⓗ1ⓝkô-pwe}
\région{BO}
\begin{glose}
\pfra{ligne de pêche}
\end{glose}
\end{sous-entrée}
\newline
\begin{sous-entrée}{kô-wony}{ⓔkôôⓗ1ⓝkô-wony}
\région{BO}
\begin{glose}
\pfra{corde de bateau}
\end{glose}
\end{sous-entrée}
\end{entrée}

\begin{entrée}{kôô}{2}{ⓔkôôⓗ2}
\région{GOs}
(\domainesémantique{Insectes})
\classe{nom}
\begin{glose}
\pfra{mante religieuse}
\end{glose}
\end{entrée}

\begin{entrée}{kôô}{3}{ⓔkôôⓗ3}
\région{GOs}
\variante{%
kôông
\région{BO PA}}
(\domainesémantique{Oiseaux})
\classe{nom}
\begin{glose}
\pfra{long-cou ; héron à face blanche ; héron gris des rivières}
\end{glose}
\nomscientifique{Ardea novaehollandiae nana; Ardea sacra albolineata}
\end{entrée}

\begin{entrée}{kôôbua}{}{ⓔkôôbua}
\région{GOs}
\variante{%
kôôbwa
\région{GO}, 
meebwa
\région{PA WE WEM}}
\classe{v}
\newline
\sens{1}
(\domainesémantique{Caractéristiques et propriétés des personnes})
\begin{glose}
\pfra{dynamique ; en forme [GOs]}
\end{glose}
\newline
\begin{exemple}
\textbf{\pnua{nu kôôbwa}}
\pfra{je suis en forme}
\end{exemple}
\newline
\sens{2}
(\domainesémantique{Caractéristiques et propriétés des personnes})
\begin{glose}
\pfra{obéir ; obéissant ; docile ; serviable ; prêt à aider ; bien disposé}
\end{glose}
\newline
\begin{exemple}
\textbf{\pnua{e aa-kôôbwa}}
\pfra{il est serviable, prêt à aider}
\end{exemple}
\newline
\begin{exemple}
\textbf{\pnua{êgu a-kôôbua}}
\pfra{toujours bien disposé}
\end{exemple}
\newline
\relationsémantique{Ant.}{\lien{ⓔkueⓝaa-kue}{aa-kue}}
\glosecourte{pas serviable (qui refuse toujours)}
\end{entrée}

\begin{entrée}{kòò-dale}{}{ⓔkòò-dale}
\région{PA BO [BM, Corne]}
(\domainesémantique{Vêtements, parure})
\classe{v}
\begin{glose}
\pfra{vêtir (se) ; habiller (s') (plutôt les vêtements du bas et chaussures)}
\end{glose}
\newline
\begin{exemple}
\région{PA}
\textbf{\pnua{i kòò-da-le pazalõ}}
\pfra{il met son pantalon}
\end{exemple}
\newline
\begin{exemple}
\région{BO}
\textbf{\pnua{kò-da ni je hòmbòni-m}}
\pfra{habille-toi !}
\end{exemple}
\newline
\begin{exemple}
\région{BO}
\textbf{\pnua{a kò-dale hòmbwòni-m}}
\pfra{va t'habiller !}
\end{exemple}
\newline
\begin{exemple}
\région{BO}
\textbf{\pnua{a kò-dale taî-m}}
\pfra{va t'habiller !}
\end{exemple}
\end{entrée}

\begin{entrée}{kòò-dö}{}{ⓔkòò-dö}
(\domainesémantique{Armes})
\classe{nom}
\begin{glose}
\pfra{manche de sagaie}
\end{glose}
\end{entrée}

\begin{entrée}{kool}{}{ⓔkool}
\région{PA BO}
(\domainesémantique{Mer})
\classe{nom}
\begin{glose}
\pfra{vague}
\end{glose}
\newline
\étymologie{
\langue{POc}
\étymon{*qaRus, *ŋalu}
\glosecourte{couler}
\auteur{Grace}}
\end{entrée}

\begin{entrée}{kòòl}{}{ⓔkòòl}
\région{PA}
(\domainesémantique{Objets coutumiers})
\classe{nom}
\begin{glose}
\pfra{monnaie kanak}
\end{glose}
\newline
\note{dont la longueur est calculée debout, de la hauteur de la tête jusqu'au sol (Charles)}{glose}{}
\newline
\relationsémantique{Cf.}{\lien{}{gò-hii, tabwa}}
\end{entrée}

\begin{entrée}{kõõl}{}{ⓔkõõl}
\région{BO}
(\domainesémantique{Sons, bruits})
\classe{v}
\begin{glose}
\pfra{gargouiller [BM]}
\end{glose}
\newline
\begin{exemple}
\région{BO}
\textbf{\pnua{i kõõl a kiò-ny}}
\pfra{mon ventre gargouille}
\end{exemple}
\end{entrée}

\begin{entrée}{kòòli}{}{ⓔkòòli}
\région{GOs BO}
\classe{v}
\newline
\sens{1}
(\domainesémantique{Santé, maladie})
\begin{glose}
\pfra{blesser (se) (sur un objet piquant)}
\end{glose}
\begin{glose}
\pfra{piquer (se)}
\end{glose}
\newline
\begin{exemple}
\textbf{\pnua{e kòòli kòò-je}}
\pfra{il s'est piqué le pied}
\end{exemple}
\newline
\begin{exemple}
\textbf{\pnua{e thuvwu kòòli hii-je xo dubwo}}
\pfra{il s'est piqué la main avec l'aiguille}
\end{exemple}
\newline
\sens{2}
(\domainesémantique{Verbes d'action (en général)})
\begin{glose}
\pfra{gratter}
\end{glose}
\begin{glose}
\pfra{toucher avec une pointe}
\end{glose}
\newline
\begin{sous-entrée}{kòòli dili}{ⓔkòòliⓢ2ⓝkòòli dili}
\begin{glose}
\pfra{gratter la terre (comme les poules)}
\end{glose}
\end{sous-entrée}
\end{entrée}

\begin{entrée}{kòòl kòlò}{}{ⓔkòòl kòlò}
\région{PA}
(\domainesémantique{Relations et interaction sociales})
\classe{v}
\begin{glose}
\pfra{prendre le parti de qqn, défendre}
\end{glose}
\newline
\begin{exemple}
\textbf{\pnua{i kòòl kòlò-n}}
\pfra{il prend sa défense}
\end{exemple}
\end{entrée}

\begin{entrée}{kòò-manyô}{}{ⓔkòò-manyô}
\région{GOs}
(\domainesémantique{Cultures, techniques, boutures})
\classe{nom}
\begin{glose}
\pfra{bouture de manioc}
\end{glose}
\newline
\begin{exemple}
\textbf{\pnua{êê-nu kòò-manyô}}
\pfra{bouture de manioc}
\end{exemple}
\end{entrée}

\begin{entrée}{koone}{}{ⓔkoone}
\région{BO}
(\domainesémantique{Navigation})
\classe{v}
\begin{glose}
\pfra{virer de bord vent debout ; louvoyer}
\end{glose}
\newline
\begin{sous-entrée}{pe-koone}{ⓔkooneⓝpe-koone}
\begin{glose}
\pfra{tirer des bordées vent debout; louvoyer}
\end{glose}
\newline
\relationsémantique{Cf.}{\lien{}{pweweede nhe}}
\newline
\note{non vérifié}{général}{}
\end{sous-entrée}
\end{entrée}

\begin{entrée}{kööni}{}{ⓔkööni}
\formephonétique{kωːɳi}
\région{GOs}
\variante{%
kooni
\formephonétique{koːni}
\région{PA WEM}}
(\domainesémantique{Préparation des aliments; modes de préparation et de cuisson})
\classe{v}
\begin{glose}
\pfra{cuire à l'étouffée}
\end{glose}
\begin{glose}
\pfra{cuire au four enterré ; mettre au four enterré}
\end{glose}
\newline
\begin{exemple}
\textbf{\pnua{kööni po}}
\pfra{mets la tortue au four}
\end{exemple}
\newline
\relationsémantique{Cf.}{\lien{ⓔkîbi}{kîbi}}
\glosecourte{four enterré}
\end{entrée}

\begin{entrée}{kòò-pwe}{}{ⓔkòò-pwe}
\région{GOs BO}
(\domainesémantique{Pêche})
\classe{nom}
\begin{glose}
\pfra{canne à pêche (lit. pied de ligne)}
\end{glose}
\end{entrée}

\begin{entrée}{köö-vwölö}{}{ⓔköö-vwölö}
\formephonétique{kωː-'vwωlω}
\région{GOs}
\variante{%
kuuwulo, kuuwolo
\région{BO}}
(\domainesémantique{Corps humain})
\classe{v}
\begin{glose}
\pfra{albinos}
\end{glose}
\end{entrée}

\begin{entrée}{kòò-wamwa}{}{ⓔkòò-wamwa}
\région{GOs}
\variante{%
kòò-wamon
\région{BO}}
(\domainesémantique{Outils})
\classe{nom}
\begin{glose}
\pfra{manche de hache}
\end{glose}
\end{entrée}

\begin{entrée}{kòòwe}{}{ⓔkòòwe}
\région{BO}
(\domainesémantique{Sentiments})
\classe{v}
\begin{glose}
\pfra{tourmenté}
\end{glose}
\end{entrée}

\begin{entrée}{kôôxô}{}{ⓔkôôxô}
\région{GOs}
\variante{%
kôôhòl
\formephonétique{kõːhɔl}
\région{PABO}, 
kôôl
\formephonétique{kõːl}
\région{BO}}
(\domainesémantique{Fonctions naturelles humaines})
\classe{v}
\begin{glose}
\pfra{gargouiller (ventre)}
\end{glose}
\newline
\begin{exemple}
\région{GO}
\textbf{\pnua{e kôôxô kiò-nu}}
\pfra{j'ai le ventre qui gargouille}
\end{exemple}
\newline
\begin{exemple}
\région{BO}
\textbf{\pnua{i kôôhòl kio-ny}}
\pfra{j'ai le ventre qui gargouille}
\end{exemple}
\end{entrée}

\begin{entrée}{kô-pa-ce-bò}{}{ⓔkô-pa-ce-bò}
\région{GOs}
\variante{%
kô-pa-ce-bòn, kô-phaa-ce-bòn
\région{WEM}, 
kô-pha-ce-bòn
\formephonétique{kõpʰayebɔn}
\région{PA}}
(\domainesémantique{Préfixes et verbes de position})
\classe{v}
\begin{glose}
\pfra{couché près du feu ; dormir près du feu}
\end{glose}
\end{entrée}

\begin{entrée}{kò-pe-bulu}{}{ⓔkò-pe-bulu}
\formephonétique{kɔ-βe-bulu}
\région{GOs}
(\domainesémantique{Préfixes et verbes de position})
\classe{v}
\begin{glose}
\pfra{debout ensemble}
\end{glose}
\newline
\begin{exemple}
\région{GO}
\textbf{\pnua{mo uça vwo mwa ko-vwe bulu mwa hãgana}}
\pfra{vous êtes venus pour que nous soyons réunis aujourd'hui}
\end{exemple}
\newline
\note{kò-pe-bulu s'utilise pour des groupes qui s'associent; tandis que kò-bulu réfère à un seul groupe.}{grammaire}{}
\end{entrée}

\begin{entrée}{kô-pii}{}{ⓔkô-pii}
\région{GOs PA WE}
\variante{%
pii
\région{BO}}
(\domainesémantique{Corps humain})
\classe{nom}
\begin{glose}
\pfra{testicules}
\end{glose}
\newline
\begin{exemple}
\textbf{\pnua{kô-pii-n}}
\pfra{ses testicules}
\end{exemple}
\end{entrée}

\begin{entrée}{kò-pò-pwaale}{}{ⓔkò-pò-pwaale}
\région{GOs}
(\domainesémantique{Parties de plantes})
\classe{nom}
\begin{glose}
\pfra{tige de maïs}
\end{glose}
\begin{glose}
\pfra{maïs (pied de)}
\end{glose}
\newline
\begin{sous-entrée}{pò-pwaale}{ⓔkò-pò-pwaaleⓝpò-pwaale}
\begin{glose}
\pfra{épi de maïs}
\end{glose}
\end{sous-entrée}
\end{entrée}

\begin{entrée}{kô-phaaxe}{}{ⓔkô-phaaxe}
\région{GOs}
\variante{%
kô-phaaxen
\région{PA BO}}
(\domainesémantique{Préfixes et verbes de position})
\classe{v}
\begin{glose}
\pfra{allongé en écoutant}
\end{glose}
\end{entrée}

\begin{entrée}{kò-phaxeen}{}{ⓔkò-phaxeen}
\région{PA}
(\domainesémantique{Préfixes et verbes de position
, Aspect})
\classe{v}
\begin{glose}
\pfra{écouter un instant, un peu}
\end{glose}
\newline
\relationsémantique{Cf.}{\lien{}{kò-}}
\glosecourte{debout}
\end{entrée}

\begin{entrée}{kô-phoo}{}{ⓔkô-phoo}
\région{GOs}
(\domainesémantique{Préfixes et verbes de position})
\classe{v}
\begin{glose}
\pfra{dormir sur le ventre}
\end{glose}
\end{entrée}

\begin{entrée}{kô-raa}{}{ⓔkô-raa}
\formephonétique{kõ-ɽaː}
\région{GOs PA BO}
(\domainesémantique{Modalité, verbes modaux})
\classe{v.IMPERS}
\begin{glose}
\pfra{impossible (lit. couché mal) ; difficile}
\end{glose}
\begin{glose}
\pfra{jamais}
\end{glose}
\newline
\begin{exemple}
\région{GO}
\textbf{\pnua{kô-raa na nu mããni}}
\pfra{je ne peux pas dormir}
\end{exemple}
\newline
\begin{exemple}
\région{GO}
\textbf{\pnua{kô-raa vwö thoi lã-nã}}
\pfra{il ne pourra pas planter ces plants}
\end{exemple}
\newline
\relationsémantique{Ant.}{\lien{}{kô-zo [GO]}}
\glosecourte{possible, bon que}
\newline
\relationsémantique{Ant.}{\lien{}{kôô-jo [BO]}}
\glosecourte{possible, bon que}
\end{entrée}

\begin{entrée}{kòròò}{}{ⓔkòròò}
\région{GOs}
(\domainesémantique{Fonctions naturelles humaines})
\classe{v}
\begin{glose}
\pfra{étrangler (s') ; étouffer (s')}
\end{glose}
\newline
\begin{exemple}
\région{GO}
\textbf{\pnua{e kòròò-nu xo hovwo}}
\pfra{j'ai avalé de travers, je me suis étouffé (lit. la nourriture m'a étouffé)}
\end{exemple}
\end{entrée}

\begin{entrée}{kotô}{}{ⓔkotô}
\région{GOs}
(\domainesémantique{Noms des plantes})
\classe{nom}
\begin{glose}
\pfra{cotonnier}
\end{glose}
\newline
\emprunt{coton (FR)}
\end{entrée}

\begin{entrée}{kô-töö}{}{ⓔkô-töö}
\région{GOs BO}
(\domainesémantique{Préfixes et verbes de position})
\classe{v}
\begin{glose}
\pfra{incliné (arbre) (lit. couché-ramper)}
\end{glose}
\newline
\relationsémantique{Cf.}{\lien{ⓔkiluu}{kiluu}}
\glosecourte{se courber ; se prosterner}
\end{entrée}

\begin{entrée}{kotra}{}{ⓔkotra}
\formephonétique{koɽa}
\région{GOs}
\variante{%
kora
\région{BO}}
(\domainesémantique{Ignames})
\classe{nom}
\begin{glose}
\pfra{igname violette et grosse (Dubois)}
\end{glose}
\end{entrée}

\begin{entrée}{kòtrixê}{}{ⓔkòtrixê}
\formephonétique{kɔɽiɣɛ̃}
\région{GOs}
\variante{%
kòtrikê
\formephonétique{kɔʈikɛ̃}
\région{vx}, 
kòriê
\formephonétique{'kɔɽiɛ̃}
\région{GO(s)}, 
kòtrikã, kòriã
\région{GO(s)}, 
kòòlixê
\région{GO(s)}}
(\domainesémantique{Sentiments})
\classe{v ; n}
\begin{glose}
\pfra{colère ; se mettre en colère}
\end{glose}
\newline
\relationsémantique{Cf.}{\lien{}{pojai [GO]}}
\glosecourte{être en colère}
\end{entrée}

\begin{entrée}{kou}{1}{ⓔkouⓗ1}
\région{GOs PA}
(\domainesémantique{Arbre})
\classe{nom}
\begin{glose}
\pfra{arbre de bord de mer}
\end{glose}
\newline
\note{(arbre dont la sève est toxique, utilisé pour soigner les piqûres de raie}{glose}{}
\nomscientifique{Excoecaria agallocha L. (Euphorbiacées))}
\end{entrée}

\begin{entrée}{kou}{2}{ⓔkouⓗ2}
\région{PA}
(\domainesémantique{Saisons})
\classe{nom}
\begin{glose}
\pfra{saison de disette (entre décembre et avril)}
\end{glose}
\newline
\relationsémantique{Cf.}{\lien{}{maxal, bweeye}}
\end{entrée}

\begin{entrée}{kovanyi}{}{ⓔkovanyi}
\région{GOs}
(\domainesémantique{Relations et interaction sociales})
\classe{nom}
\begin{glose}
\pfra{compagnon ; ami}
\end{glose}
\newline
\begin{exemple}
\région{GO}
\textbf{\pnua{li kovanyi}}
\pfra{ils sont compagnons}
\end{exemple}
\newline
\begin{exemple}
\textbf{\pnua{kovanyi-nu}}
\pfra{mes amis, compagnons}
\end{exemple}
\newline
\emprunt{compagnie (FR)}
\end{entrée}

\begin{entrée}{kò-waayu}{}{ⓔkò-waayu}
\région{PA}
(\domainesémantique{Modalité, verbes modaux})
\classe{v}
\begin{glose}
\pfra{persister à (sens positif)}
\end{glose}
\newline
\relationsémantique{Cf.}{\lien{}{kò- < kòòl}}
\glosecourte{debout}
\end{entrée}

\begin{entrée}{kô-waga}{}{ⓔkô-waga}
\région{GOs}
(\domainesémantique{Préfixes et verbes de position})
\classe{v}
\begin{glose}
\pfra{allongé les jambes écartées en l'air}
\end{glose}
\end{entrée}

\begin{entrée}{kò-wiò}{}{ⓔkò-wiò}
\région{GOs}
\variante{%
kòyò
\région{GO(s)}, 
koeo
\région{GO(s)}}
(\domainesémantique{Parenté})
\classe{nom}
\begin{glose}
\pfra{frère cadet (tous les frères plus jeunes que l'aîné)}
\end{glose}
\begin{glose}
\pfra{cousin (fille de frère de père ; fils/fille de soeur de mère)}
\end{glose}
\newline
\begin{exemple}
\textbf{\pnua{kò-wiò-nu}}
\pfra{mon frère cadet (plus jeune)}
\end{exemple}
\newline
\relationsémantique{Cf.}{\lien{ⓔebiigi}{ebiigi}}
\glosecourte{cousins croisés}
\end{entrée}

\begin{entrée}{kô-wõ}{}{ⓔkô-wõ}
\région{GOs}
\variante{%
kô-wony
\région{BO PA}}
(\domainesémantique{Navigation})
\classe{nom}
\begin{glose}
\pfra{cordage de bateau}
\end{glose}
\end{entrée}

\begin{entrée}{kò-wony}{}{ⓔkò-wony}
\région{BO}
(\domainesémantique{Navigation})
\classe{nom}
\begin{glose}
\pfra{mât du bateau}
\end{glose}
\end{entrée}

\begin{entrée}{köxa}{}{ⓔköxa}
\formephonétique{kωɣa}
\région{GOs}
(\domainesémantique{Conjonction})
\classe{CNJ}
\begin{glose}
\pfra{alors, puisque}
\end{glose}
\newline
\begin{exemple}
\textbf{\pnua{köxa, na cö puxãnuu-je}}
\pfra{soit/alors puisque tu aimes celui-là}
\end{exemple}
\end{entrée}

\begin{entrée}{köxö}{}{ⓔköxö}
\formephonétique{kωɣω}
\région{GOs PA}
(\domainesémantique{Discours, échanges verbaux})
\classe{v}
\begin{glose}
\pfra{bégayer}
\end{glose}
\newline
\étymologie{
\langue{POc}
\étymon{*kakap}
\glosecourte{bégayer}
\auteur{Blust}}
\end{entrée}

\begin{entrée}{kòyò}{}{ⓔkòyò}
\région{GOs BO PA}
(\domainesémantique{Verbes d'action (en général)})
\classe{v (non-humains)}
\begin{glose}
\pfra{perdu ; disparaître}
\end{glose}
\newline
\begin{exemple}
\textbf{\pnua{e kòyò pòi-nu kuau}}
\pfra{mon chien est perdu}
\end{exemple}
\newline
\begin{exemple}
\textbf{\pnua{e kòyò}}
\pfra{c'est perdu}
\end{exemple}
\newline
\begin{sous-entrée}{pa-kòyò-ni}{ⓔkòyòⓝpa-kòyò-ni}
\begin{glose}
\pfra{supprimer, enlever qqch}
\end{glose}
\end{sous-entrée}
\end{entrée}

\begin{entrée}{kôzaxebi}{}{ⓔkôzaxebi}
\formephonétique{kõ'ðaɣembi}
\région{GOs BO}
\variante{%
kôzakebi
\région{GO}}
(\domainesémantique{Caractéristiques et propriétés des personnes})
\classe{v}
\begin{glose}
\pfra{avoir l'habitude ; sage}
\end{glose}
\newline
\begin{exemple}
\région{GO}
\textbf{\pnua{nu kôzaxebi nye chaabo ne waa gò}}
\pfra{j'ai l'habitude de me lever tôt}
\end{exemple}
\end{entrée}

\begin{entrée}{kô-zo}{}{ⓔkô-zo}
\région{GOsWEM}
\variante{%
kô-yo
\région{PA BO}}
(\domainesémantique{Modalité, verbes modaux})
\classe{v}
\begin{glose}
\pfra{possible de ; permis de}
\end{glose}
\newline
\begin{exemple}
\région{GO}
\textbf{\pnua{kô-zo na jö a-du traabwa ?}}
\pfra{peux-tu venir t'asseoir ?}
\end{exemple}
\newline
\begin{exemple}
\région{GO}
\textbf{\pnua{kô-zo na nu whili-cö}}
\pfra{je peux te conduire}
\end{exemple}
\newline
\begin{exemple}
\région{PA}
\textbf{\pnua{kô-zo na la pe-tòò-la monòn mãni bò-na}}
\pfra{ils pourront se retrouver demain et après-demain}
\end{exemple}
\newline
\relationsémantique{Ant.}{\lien{}{kô-raa na}}
\glosecourte{impossible que}
\newline
\relationsémantique{Cf.}{\lien{}{e zo}}
\glosecourte{il faut que}
\end{entrée}

\begin{entrée}{ku}{1}{ⓔkuⓗ1}
\région{BO GO}
\variante{%
ko
\région{BO}}
(\domainesémantique{Agent})
\classe{AGT}
\begin{glose}
\pfra{agent}
\end{glose}
\newline
\begin{exemple}
\région{Haudricourt}
\textbf{\pnua{i kobwi ku Tèma}}
\pfra{Le chef dit}
\end{exemple}
\end{entrée}

\begin{entrée}{ku}{2}{ⓔkuⓗ2}
\région{PA BO}
(\domainesémantique{Noms des plantes})
\classe{nom}
\begin{glose}
\pfra{liane}
\end{glose}
\newline
\note{(sorte de liane dont on mange les fruits légèrement grillés, a un goût de café)}{glose}{}
\end{entrée}

\begin{entrée}{ku}{3}{ⓔkuⓗ3}
\région{GOs}
\variante{%
kuul
\région{PA BO WE}}
(\domainesémantique{Verbes de mouvement})
\classe{v}
\begin{glose}
\pfra{tomber (de qqch, pour un inanimé)}
\end{glose}
\newline
\begin{exemple}
\région{PA}
\textbf{\pnua{la daa pe-kuule pò-maak}}
\pfra{les mangues tombent toutes seules, d'elles-même}
\end{exemple}
\newline
\begin{exemple}
\région{PA}
\textbf{\pnua{la daa pe-kuul}}
\pfra{elles tombent toutes seules, d'elles-même}
\end{exemple}
\newline
\relationsémantique{Cf.}{\lien{}{thrõbo ; kaalu}}
\end{entrée}

\begin{entrée}{ku}{4}{ⓔkuⓗ4}
\région{GOs PA}
\variante{%
kun
\région{PA}}
\classe{nom}
\newline
\sens{1}
(\domainesémantique{Noms locatifs})
\begin{glose}
\pfra{endroit ; place}
\end{glose}
\newline
\begin{exemple}
\région{PA}
\textbf{\pnua{kun-na}}
\pfra{cet endroit-ci}
\end{exemple}
\newline
\sens{2}
(\domainesémantique{Organisation sociale})
\begin{glose}
\pfra{ensemble des clans formant la tribu [PA]}
\end{glose}
\end{entrée}

\begin{entrée}{ku}{5}{ⓔkuⓗ5}
\région{GOs BO PA}
\classe{nom}
\newline
\sens{1}
(\domainesémantique{Topographie})
\begin{glose}
\pfra{fond de la vallée ; haut d'une vallée}
\end{glose}
\newline
\begin{sous-entrée}{ku-kepwa}{ⓔkuⓗ5ⓢ1ⓝku-kepwa}
\begin{glose}
\pfra{le haut de la vallée, talweg}
\end{glose}
\end{sous-entrée}
\newline
\begin{sous-entrée}{ku-kewang}{ⓔkuⓗ5ⓢ1ⓝku-kewang}
\région{BO}
\begin{glose}
\pfra{le haut de la vallée, talweg}
\end{glose}
\end{sous-entrée}
\newline
\begin{sous-entrée}{ku-nogo}{ⓔkuⓗ5ⓢ1ⓝku-nogo}
\région{BO}
\begin{glose}
\pfra{la source de la rivière}
\end{glose}
\end{sous-entrée}
\newline
\sens{2}
(\domainesémantique{Description des objets, formes, consistance, taille})
\begin{glose}
\pfra{bout, extrémité d'une surface ou d'une chose étendue}
\end{glose}
\newline
\étymologie{
\langue{POc}
\étymon{*qulu}
\glosecourte{tête, sommet}}
\end{entrée}

\begin{entrée}{ku}{6}{ⓔkuⓗ6}
\région{GOs}
(\domainesémantique{Aspect})
\classe{ASP.HAB}
\begin{glose}
\pfra{habitude}
\end{glose}
\newline
\begin{exemple}
\textbf{\pnua{e ku ne}}
\pfra{il le fait souvent}
\end{exemple}
\newline
\begin{exemple}
\textbf{\pnua{lò ku khûbu-bi je-na}}
\pfra{ils nous tapaient souvent}
\end{exemple}
\newline
\begin{exemple}
\textbf{\pnua{ezoma nu a ku nõ nyaanya, na nu uça awônô}}
\pfra{j'irai voir ma mère (régulièrement) quand je rentrerai chez moi}
\end{exemple}
\end{entrée}

\begin{entrée}{ku-}{1}{ⓔku-ⓗ1}
\région{BO}
(\domainesémantique{Ignames
, Préfixes classificateurs sémantiques})
\classe{nom}
\begin{glose}
\pfra{préfixe des ignames}
\end{glose}
\newline
\begin{sous-entrée}{ku-kò}{ⓔku-ⓗ1ⓝku-kò}
\begin{glose}
\pfra{igname poule}
\end{glose}
\end{sous-entrée}
\newline
\begin{sous-entrée}{ku-mòòlò}{ⓔku-ⓗ1ⓝku-mòòlò}
\begin{glose}
\pfra{igname blanche (la dernière à arriver à maturité)}
\end{glose}
\newline
\relationsémantique{Cf.}{\lien{}{ku-bè ; ku-peena ; ku-pe ; ku-jaa ; ku-cu ; ku-dimwa ; ku-bwaolè ; ku-bweena}}
\glosecourte{noms de clones d'igname}
\end{sous-entrée}
\end{entrée}

\begin{entrée}{ku-}{2}{ⓔku-ⓗ2}
\région{PA BO}
(\domainesémantique{Préfixes sémantiques de position})
\classe{v}
\classe{PREF (indiquant la position debout)}
\begin{glose}
\pfra{debout}
\end{glose}
\newline
\begin{exemple}
\textbf{\pnua{i ku nò-da}}
\pfra{il regarde en l'air debout}
\end{exemple}
\newline
\begin{exemple}
\textbf{\pnua{ku nenèm !}}
\pfra{tiens-toi (debout) tranquille !}
\end{exemple}
\newline
\étymologie{
\langue{POc}
\étymon{*tuqud}}
\end{entrée}

\begin{entrée}{kû}{1}{ⓔkûⓗ1}
\région{BO}
(\domainesémantique{Aliments, alimentation})
\classe{v}
\begin{glose}
\pfra{manger (fruits) ; croquer (fruits, légumes verts)}
\end{glose}
\newline
\begin{exemple}
\textbf{\pnua{nu kû orâ}}
\pfra{je mange des oranges}
\end{exemple}
\newline
\begin{exemple}
\région{BO}
\textbf{\pnua{kû-ny orâ}}
\pfra{ma part d'oranges}
\end{exemple}
\newline
\note{kûûni (v.t.)}{grammaire}{}
\end{entrée}

\begin{entrée}{kû}{2}{ⓔkûⓗ2}
\région{GOs}
\classe{v}
\newline
\sens{1}
(\domainesémantique{Aspect})
\begin{glose}
\pfra{terminer}
\end{glose}
\newline
\begin{exemple}
\textbf{\pnua{ba-kûûni-xo}}
\pfra{la fin}
\end{exemple}
\newline
\sens{2}
(\domainesémantique{Verbes d'action (en général)})
\begin{glose}
\pfra{(r)emplir}
\end{glose}
\begin{glose}
\pfra{envahir (de peur)}
\end{glose}
\newline
\begin{exemple}
\textbf{\pnua{e kûû je xo hããxa}}
\pfra{il est envahi par la peur}
\end{exemple}
\newline
\begin{exemple}
\textbf{\pnua{e kûû je xo nyaru}}
\pfra{il est envahi par la gale, les plaies}
\end{exemple}
\newline
\note{kûûni (v.t.)}{grammaire}{remplir; envahir}
\end{entrée}

\begin{entrée}{ku-ã}{}{ⓔku-ã}
\région{GOs}
\variante{%
kunã
}
(\domainesémantique{Localisation})
\classe{LOC ; n}
\begin{glose}
\pfra{de ce côté-ci}
\end{glose}
\begin{glose}
\pfra{place}
\end{glose}
\newline
\begin{exemple}
\textbf{\pnua{kunãã-je}}
\pfra{sa place}
\end{exemple}
\newline
\relationsémantique{Cf.}{\lien{ⓔkunⓝphe kunã}{phe kunã}}
\glosecourte{prendre la place de qqn}
\end{entrée}

\begin{entrée}{kuãgòòn}{}{ⓔkuãgòòn}
\région{BO}
\variante{%
kwãgòn
\région{BO}}
(\domainesémantique{Mollusques})
\classe{nom}
\begin{glose}
\pfra{moule ; coquillage rond [Corne]}
\end{glose}
\nomscientifique{Mytilus smaragdinus}
\end{entrée}

\begin{entrée}{ku-ãgu}{}{ⓔku-ãgu}
\région{GOs}
(\domainesémantique{Ignames})
\classe{nom}
\begin{glose}
\pfra{igname}
\end{glose}
\newline
\note{(a la forme générale d'une personne)}{glose}{}
\end{entrée}

\begin{entrée}{kuani}{1}{ⓔkuaniⓗ1}
\région{PA}
(\domainesémantique{Aliments, alimentation})
\classe{v}
\begin{glose}
\pfra{laisser fondre dans la bouche}
\end{glose}
\end{entrée}

\begin{entrée}{kuani}{2}{ⓔkuaniⓗ2}
\région{PA}
\classe{v}
(\domainesémantique{Pêche})
\begin{glose}
\pfra{mettre sur une filoche (poisson)}
\end{glose}
\newline
\begin{exemple}
\région{PA}
\textbf{\pnua{pe-kuani line no}}
\pfra{met ces poissons sur une filoche}
\end{exemple}
\end{entrée}

\begin{entrée}{kuau}{}{ⓔkuau}
\région{GOs PABO}
(\domainesémantique{Mammifères})
\classe{nom}
\begin{glose}
\pfra{chien}
\end{glose}
\newline
\begin{exemple}
\textbf{\pnua{pòi-nu kuau [GOs]}}
\pfra{mon chien}
\end{exemple}
\newline
\begin{exemple}
\textbf{\pnua{e kòyò pòi-nu kuau}}
\pfra{mon chien est perdu}
\end{exemple}
\newline
\emprunt{kuau (POLYN) 'young of an animal'}
\end{entrée}

\begin{entrée}{ku-baazo}{}{ⓔku-baazo}
\région{PA}
\variante{%
ku-baayo
\région{BO}}
(\domainesémantique{Directions})
\classe{v}
\begin{glose}
\pfra{travers (de) (utilisé pour tout, y compris pour le soleil : lorsqu'il descend vers l'horizon)}
\end{glose}
\end{entrée}

\begin{entrée}{ku-baazo al}{}{ⓔku-baazo al}
\région{PA BO}
(\domainesémantique{Découpage du temps})
\classe{nom}
\begin{glose}
\pfra{après-midi (lorsque le soleil descend vers l'horizon)}
\end{glose}
\end{entrée}

\begin{entrée}{ku-be}{}{ⓔku-be}
\région{GOs BO}
(\domainesémantique{Ignames})
\classe{nom}
\begin{glose}
\pfra{igname (variété)}
\end{glose}
\newline
\note{(igname à petites racines, plantées sur le bord du billon, elles poussent plus vite que celles à racines longues du centre du billon et donnent les premières récoltes ; Charles)}{glose}{}
\end{entrée}

\begin{entrée}{kubi}{}{ⓔkubi}
\région{GOs PA BO}
\classe{v ; n}
\newline
\sens{1}
(\domainesémantique{Verbes d'action (en général)})
\begin{glose}
\pfra{gratter}
\end{glose}
\newline
\sens{2}
(\domainesémantique{Préparation des aliments; modes de préparation et de cuisson})
\begin{glose}
\pfra{écailler le poisson ; écaille (de poisson)}
\end{glose}
\newline
\begin{exemple}
\région{GO}
\textbf{\pnua{nu kubi nò}}
\pfra{j'écaille un poisson}
\end{exemple}
\newline
\sens{3}
(\domainesémantique{Santé, maladie})
\begin{glose}
\pfra{croûtes sur la tête des bébés}
\end{glose}
\newline
\étymologie{
\langue{POc}
\étymon{*qunapi}
\glosecourte{écailler}}
\newline
\note{kubi-n}{grammaire}{ses écailles}
\end{entrée}

\begin{entrée}{kubo}{}{ⓔkubo}
\région{GOs PA}
(\domainesémantique{Verbes d'action (en général)})
\classe{v}
\begin{glose}
\pfra{attendre}
\end{glose}
\newline
\begin{exemple}
\textbf{\pnua{kubo-je !}}
\pfra{attends-le !}
\end{exemple}
\newline
\begin{exemple}
\textbf{\pnua{co tre-kubo ti ?}}
\pfra{qui attends-tu !}
\end{exemple}
\newline
\begin{exemple}
\textbf{\pnua{nu tre-kubo thoomwã hê ?}}
\pfra{j'attends cette femme-ci !}
\end{exemple}
\end{entrée}

\begin{entrée}{kûbu}{}{ⓔkûbu}
\région{GOs PA BO}
\variante{%
khûbu
\région{BO}}
(\domainesémantique{Mouvements ou actions faits avec le corps, les bras, les mains, les pieds})
\classe{v}
\begin{glose}
\pfra{frapper ; tuer}
\end{glose}
\newline
\begin{exemple}
\région{BO}
\textbf{\pnua{hla kûbu-ye}}
\pfra{ils l'ont tué}
\end{exemple}
\end{entrée}

\begin{entrée}{ku-bulu}{}{ⓔku-bulu}
\région{PA}
(\domainesémantique{Verbes de mouvement})
\classe{v}
\begin{glose}
\pfra{rassembler debout (se)}
\end{glose}
\end{entrée}

\begin{entrée}{ku-bwau}{}{ⓔku-bwau}
\région{GOs BO PA}
(\domainesémantique{Ignames})
\classe{nom}
\begin{glose}
\pfra{igname ronde}
\end{glose}
\newline
\note{(à petites racines, on les plante sur le bord du billon, elles poussent plus vite que celles à racines longues du centre du billon et donnent les premières récoltes; Charles)}{glose}{}
\newline
\relationsémantique{Cf.}{\lien{}{kube, zaòl}}
\end{entrée}

\begin{entrée}{ku-bweena}{}{ⓔku-bweena}
\région{GOs BO}
(\domainesémantique{Ignames})
\classe{nom}
\begin{glose}
\pfra{igname (la tige est de la couleur d'un serpent)}
\end{glose}
\end{entrée}

\begin{entrée}{ku-bwii}{}{ⓔku-bwii}
\région{GOs}
(\domainesémantique{Ignames})
\classe{nom}
\begin{glose}
\pfra{igname (clone, à chair mauve)}
\end{glose}
\end{entrée}

\begin{entrée}{ku-çaaxò}{}{ⓔku-çaaxò}
\formephonétique{ku-'ʒaːɣɔ}
\région{GOs}
\variante{%
ku-caaxò
\formephonétique{ku-'tjaːɣɔ}
\région{PA BO}}
\classe{v.i.}
\newline
\sens{1}
(\domainesémantique{Mouvements ou actions faits avec le corps, les bras, les mains, les pieds})
\begin{glose}
\pfra{cacher (se)}
\end{glose}
\newline
\sens{2}
(\domainesémantique{Guerre})
\begin{glose}
\pfra{réfugier (se)}
\end{glose}
\newline
\relationsémantique{Cf.}{\lien{}{caaxò}}
\glosecourte{se cacher}
\newline
\relationsémantique{Cf.}{\lien{}{v.t. thözoe}}
\glosecourte{cacher qqn}
\end{entrée}

\begin{entrée}{ku-cabo}{}{ⓔku-cabo}
\formephonétique{ku-cabo}
\région{GOs}
(\domainesémantique{Ignames})
\classe{nom}
\begin{glose}
\pfra{igname (qui ressort de terre)}
\end{glose}
\end{entrée}

\begin{entrée}{ku-cimwi}{}{ⓔku-cimwi}
\région{GOs PA}
(\domainesémantique{Préfixes et verbes de position})
\classe{v}
\begin{glose}
\pfra{debout en tenant qqch serré dans la main}
\end{glose}
\end{entrée}

\begin{entrée}{kudi}{}{ⓔkudi}
\formephonétique{'kundi}
\région{GOs PA}
(\domainesémantique{Configuration des objets})
\classe{nom}
\begin{glose}
\pfra{coin ; angle}
\end{glose}
\end{entrée}

\begin{entrée}{kudi-mwa}{}{ⓔkudi-mwa}
\formephonétique{ku'ndi-mwã}
\région{GOs PA}
(\domainesémantique{Types de maison, architecture de la maison})
\classe{nom}
\begin{glose}
\pfra{coin externe de la maison}
\end{glose}
\newline
\relationsémantique{Cf.}{\lien{}{puni [GOs], puning [PA]}}
\glosecourte{fond de lamaison ronde}
\end{entrée}

\begin{entrée}{kudò}{}{ⓔkudò}
\formephonétique{kundɔ}
\région{GOs PA}
\variante{%
kido
\région{PA BO WEM WE}}
\newline
\groupe{A}
(\domainesémantique{Aliments, alimentation})
\classe{nom}
\begin{glose}
\pfra{boisson}
\end{glose}
\newline
\begin{exemple}
\région{PA BO}
\textbf{\pnua{kido-n}}
\pfra{sa boisson}
\end{exemple}
\newline
\groupe{B}
(\domainesémantique{Aliments, alimentation})
\classe{v}
\begin{glose}
\pfra{boire}
\end{glose}
\newline
\begin{exemple}
\textbf{\pnua{nu kudò}}
\pfra{je bois}
\end{exemple}
\newline
\begin{exemple}
\textbf{\pnua{ku kafe ?}}
\pfra{tu bois/veux du café ?}
\end{exemple}
\newline
\begin{exemple}
\région{PA}
\textbf{\pnua{la pe-kido}}
\pfra{ils ont bu ensemble}
\end{exemple}
\newline
\begin{exemple}
\région{PA}
\textbf{\pnua{kido-too}}
\pfra{boire chaud}
\end{exemple}
\newline
\begin{exemple}
\région{PA}
\textbf{\pnua{kido-tuujo}}
\pfra{boire froid}
\end{exemple}
\newline
\relationsémantique{Cf.}{\lien{}{pha-kido-ni}}
\glosecourte{faire boire (qqn)}
\newline
\groupe{C}
(\domainesémantique{Préfixes classificateurs possessifs de la nourriture})
\classe{CLF.POSS}
\newline
\begin{exemple}
\région{PA}
\région{GO}
\textbf{\pnua{kudòò-nu we}}
\pfra{mon eau (lit. ma boisson eau)}
\end{exemple}
\newline
\begin{exemple}
\région{PA}
\textbf{\pnua{kidoo-ny (a) we}}
\pfra{mon eau (lit. ma boisson eau)}
\end{exemple}
\newline
\étymologie{
\langue{PNC (Proto-Neo- Caledonian)}
\étymon{*qanjauq}
\auteur{Haudricourt}}
\end{entrée}

\begin{entrée}{kûdo}{1}{ⓔkûdoⓗ1}
\région{GOs}
(\domainesémantique{Mer : topographie})
\classe{nom}
\begin{glose}
\pfra{baie}
\end{glose}
\end{entrée}

\begin{entrée}{kûdo}{2}{ⓔkûdoⓗ2}
\région{GOs}
(\domainesémantique{Poissons})
\classe{nom}
\begin{glose}
\pfra{rémora}
\end{glose}
\nomscientifique{Echeneis naucrates (Echeneidae)}
\end{entrée}

\begin{entrée}{kue}{}{ⓔkue}
\région{GOs}
\variante{%
kuel, kwel
\région{BO}}
(\domainesémantique{Relations et interaction sociales})
\classe{v}
\begin{glose}
\pfra{refuser ; rejeter ; désobéir}
\end{glose}
\begin{glose}
\pfra{détester}
\end{glose}
\newline
\begin{exemple}
\textbf{\pnua{ãgu xa nu kuene}}
\pfra{une personne que je déteste}
\end{exemple}
\newline
\begin{exemple}
\région{BO}
\textbf{\pnua{i kwèl}}
\pfra{il ne veut pas}
\end{exemple}
\newline
\begin{exemple}
\région{BO}
\textbf{\pnua{co kwèl inu}}
\pfra{tune veux pas de moi}
\end{exemple}
\newline
\begin{sous-entrée}{aa-kue}{ⓔkueⓝaa-kue}
\begin{glose}
\pfra{pas serviable (qui refuse toujours)}
\end{glose}
\end{sous-entrée}
\newline
\begin{sous-entrée}{tre-kue}{ⓔkueⓝtre-kue}
\begin{glose}
\pfra{jalouser; être jaloux}
\end{glose}
\newline
\note{kuele (v.t) kuene [GO]}{grammaire}{refuser qqch}
\end{sous-entrée}
\end{entrée}

\begin{entrée}{ku-e}{}{ⓔku-e}
\région{GOs}
(\domainesémantique{Préfixes et verbes de position})
\classe{v}
\begin{glose}
\pfra{debout en portant dans les bras}
\end{glose}
\newline
\begin{exemple}
\textbf{\pnua{e ku-e ẽnõ}}
\pfra{il est debout avec l'enfant dans les bras}
\end{exemple}
\end{entrée}

\begin{entrée}{kuee}{}{ⓔkuee}
\région{GOs}
(\domainesémantique{Verbes d'action (en général)})
\classe{v}
\begin{glose}
\pfra{résister (à une épreuve) ; tenir le coup}
\end{glose}
\end{entrée}

\begin{entrée}{kuel}{}{ⓔkuel}
\région{BO PA}
(\domainesémantique{Relations et interaction sociales})
\classe{v.i.}
\begin{glose}
\pfra{rejeter ; refuser ; aimer (ne pas)}
\end{glose}
\newline
\begin{exemple}
\région{BO}
\textbf{\pnua{i kuel}}
\pfra{il ne veut pas}
\end{exemple}
\newline
\note{kuele (v.t.)}{grammaire}{rejeter qqch/qqn}
\end{entrée}

\begin{entrée}{kuele}{}{ⓔkuele}
\région{GOs BO}
(\domainesémantique{Relations et interaction sociales})
\classe{v}
\begin{glose}
\pfra{rejeter ; refuser ; détester ; aimer (ne pas)}
\end{glose}
\newline
\begin{exemple}
\textbf{\pnua{nu kuele la nu ne}}
\pfra{je déteste ce que j'ai fait}
\end{exemple}
\newline
\begin{exemple}
\textbf{\pnua{nu kweli-jö}}
\pfra{je te déteste}
\end{exemple}
\newline
\note{v.t. kueli (+ animés)}{grammaire}{}
\newline
\note{v.i. kue [GOs], kuel [BO PA]}{grammaire}{refuser}
\end{entrée}

\begin{entrée}{kugoo}{}{ⓔkugoo}
\région{GOs}
(\domainesémantique{Description des objets, formes, consistance, taille})
\classe{v}
\begin{glose}
\pfra{mou ; flasque}
\end{glose}
\end{entrée}

\begin{entrée}{ku-gòò}{}{ⓔku-gòò}
\région{GOs}
\variante{%
kô-go
\région{BO}}
\newline
\groupe{A}
\classe{v.stat.}
\newline
\sens{1}
(\domainesémantique{Description des objets, formes, consistance, taille})
\begin{glose}
\pfra{droit ; rectiligne}
\end{glose}
\newline
\sens{2}
(\domainesémantique{Description des objets, formes, consistance, taille})
\begin{glose}
\pfra{vrai}
\end{glose}
\newline
\begin{exemple}
\région{GO}
\textbf{\pnua{fhaa ka ku-gòò, vhaa xa ku-gòò}}
\pfra{parole vraie}
\end{exemple}
\newline
\groupe{B}
\newline
\sens{3}
(\domainesémantique{Modalité, verbes modaux})
\classe{nom}
\begin{glose}
\pfra{droit ; autorisation}
\end{glose}
\newline
\begin{exemple}
\région{GO}
\textbf{\pnua{pu ku-gòò-nu vwo nu vhaa cai jö}}
\pfra{j'ai le droit de te parler}
\end{exemple}
\end{entrée}

\begin{entrée}{ku-gozi}{}{ⓔku-gozi}
\région{GOs}
(\domainesémantique{Ignames})
\classe{nom}
\begin{glose}
\pfra{igname blanche}
\end{glose}
\end{entrée}

\begin{entrée}{ku-hôboe}{}{ⓔku-hôboe}
\région{GOs}
\variante{%
ku-hôbwo
\région{GO(s) BO}}
\newline
\sens{1}
(\domainesémantique{Verbes d'action (en général)})
\classe{v}
\begin{glose}
\pfra{surveiller ; garder ; faire le guet}
\end{glose}
\newline
\sens{2}
(\domainesémantique{Guerre})
\classe{v}
\begin{glose}
\pfra{embuscade (faire une) ; surveiller la route}
\end{glose}
\newline
\begin{exemple}
\textbf{\pnua{la ku-hôboe phwee-de-la}}
\pfra{ils les ont attendu sur le chemin}
\end{exemple}
\newline
\relationsémantique{Cf.}{\lien{ⓔhôbwo}{hôbwo}}
\glosecourte{surveiller}
\end{entrée}

\begin{entrée}{kui}{}{ⓔkui}
\région{GOs BO PA}
(\domainesémantique{Ignames})
\classe{nom}
\begin{glose}
\pfra{igname}
\end{glose}
\nomscientifique{Dioscorea alata (Dioscoréacées)}
\newline
\begin{sous-entrée}{nò kui}{ⓔkuiⓝnò kui}
\begin{glose}
\pfra{extrémité inférieure de l'igname}
\end{glose}
\end{sous-entrée}
\newline
\begin{sous-entrée}{gu kui}{ⓔkuiⓝgu kui}
\begin{glose}
\pfra{l'igname du chef (offerte pour les prémices)}
\end{glose}
\end{sous-entrée}
\newline
\begin{sous-entrée}{caa-nu kui}{ⓔkuiⓝcaa-nu kui}
\région{GO}
\begin{glose}
\pfra{mon igname (cuite, à manger) (lit. nourriture-ma igname)}
\end{glose}
\newline
\begin{exemple}
\textbf{\pnua{kui-nu}}
\pfra{mon igname (tubercule)}
\end{exemple}
\newline
\relationsémantique{Cf.}{\lien{}{nom de différents clones : kui paao ; kui kamôve ; kui pòwa ; kui êpâdan/evadan ; yave ; kajaa ; kora ; kera ; kubwau ; djinodji ; bea ; mwacoa ; zara ; thiabwau ; ua ; papua ; hou ; dimwa}}
\end{sous-entrée}
\newline
\étymologie{
\langue{POc}
\étymon{*qupi}}
\end{entrée}

\begin{entrée}{ku-ido-xe}{}{ⓔku-ido-xe}
\région{GOs}
(\domainesémantique{Verbes d'action (en général)})
\classe{v}
\begin{glose}
\pfra{aligner (des choses)}
\end{glose}
\newline
\begin{exemple}
\textbf{\pnua{çö ne wu ku-ido-xe ce-thîni}}
\pfra{aligne les poteaux de la barrière}
\end{exemple}
\newline
\relationsémantique{Cf.}{\lien{}{-xe}}
\glosecourte{un}
\end{entrée}

\begin{entrée}{kuii}{}{ⓔkuii}
\région{GOs}
(\domainesémantique{Corps humain})
\classe{nom}
\begin{glose}
\pfra{rein ; rognon}
\end{glose}
\end{entrée}

\begin{entrée}{kû-jaa}{1}{ⓔkû-jaaⓗ1}
\région{GOs}
(\domainesémantique{Ignames})
\classe{nom}
\begin{glose}
\pfra{igname (2 sortes : blanche ou jaunâtre)}
\end{glose}
\end{entrée}

\begin{entrée}{kû-jaa}{2}{ⓔkû-jaaⓗ2}
\région{GOs}
(\domainesémantique{Discours, échanges verbaux})
\classe{v}
\begin{glose}
\pfra{dire ; avertir ; prévenir}
\end{glose}
\newline
\begin{exemple}
\textbf{\pnua{li kû-jaa nye ẽnõ}}
\pfra{ils disent à ce garçon}
\end{exemple}
\newline
\begin{exemple}
\région{GO}
\textbf{\pnua{lò xa kû-jaa-li}}
\pfra{ils leur disent (à eux deux)}
\end{exemple}
\newline
\note{forme courte de : khobwe jaa- 'dire à'}{grammaire}{}
\end{entrée}

\begin{entrée}{ku-kewang}{}{ⓔku-kewang}
\région{PA}
(\domainesémantique{Topographie})
\classe{nom}
\begin{glose}
\pfra{haut de la vallée}
\end{glose}
\end{entrée}

\begin{entrée}{ku-kiai}{}{ⓔku-kiai}
\région{GOs}
\variante{%
ku-xiai
\région{GO(s)}}
(\domainesémantique{Feu : objets et actions liés au feu})
\classe{v}
\begin{glose}
\pfra{réchauffer (se) debout près du feu}
\end{glose}
\newline
\morphologie{forme courte de ku-khi(ni)-yaai (lit. debout-chauffer-feu)}
\end{entrée}

\begin{entrée}{ku-ko}{}{ⓔku-ko}
\région{GOs PA BO}
(\domainesémantique{Ignames})
\classe{nom}
\begin{glose}
\pfra{igname (clone ; ressemble à une tête de poule)}
\end{glose}
\end{entrée}

\begin{entrée}{ku-kue}{}{ⓔku-kue}
\région{GOs}
(\domainesémantique{Préfixes et verbes de position})
\classe{v}
\begin{glose}
\pfra{debout en portant (bébé)}
\end{glose}
\newline
\begin{exemple}
\textbf{\pnua{e ku-kue ẽnõ}}
\pfra{elle est debout portant le bébé}
\end{exemple}
\end{entrée}

\begin{entrée}{kula}{1}{ⓔkulaⓗ1}
\région{GOs PA BO}
\classe{nom}
\newline
\sens{1}
(\domainesémantique{Crustacés, crabes})
\begin{glose}
\pfra{crevette}
\end{glose}
\newline
\begin{sous-entrée}{kula ni we-za}{ⓔkulaⓗ1ⓢ1ⓝkula ni we-za}
\région{GO BO}
\begin{glose}
\pfra{langouste}
\end{glose}
\end{sous-entrée}
\newline
\begin{sous-entrée}{kula nõgo}{ⓔkulaⓗ1ⓢ1ⓝkula nõgo}
\région{GO}
\begin{glose}
\pfra{crevette}
\end{glose}
\end{sous-entrée}
\newline
\begin{sous-entrée}{kula be}{ⓔkulaⓗ1ⓢ1ⓝkula be}
\région{GO BO}
\begin{glose}
\pfra{langouste noire}
\end{glose}
\end{sous-entrée}
\newline
\sens{2}
(\domainesémantique{Jeux divers})
\begin{glose}
\pfra{figure du jeu de ficelle (crevette) [BO]}
\end{glose}
\newline
\étymologie{
\langue{POc}
\étymon{*quɖa(ŋ), *quraŋ}}
\end{entrée}

\begin{entrée}{kula}{2}{ⓔkulaⓗ2}
\région{GOs BO}
(\domainesémantique{Actions liées aux éléments (liquide, fumée)})
\classe{v.i.}
\begin{glose}
\pfra{couler ; répandre (se) ; vider (se)}
\end{glose}
\newline
\begin{exemple}
\région{BO}
\textbf{\pnua{i kula we}}
\pfra{l'eau a débordé}
\end{exemple}
\newline
\note{kule, kole}{grammaire}{verser, répandre}
\end{entrée}

\begin{entrée}{kula-be}{}{ⓔkula-be}
\région{GOs BO}
(\domainesémantique{Crustacés, crabes})
\classe{nom}
\begin{glose}
\pfra{crevette (grosse et noire)}
\end{glose}
\end{entrée}

\begin{entrée}{kulaçe}{}{ⓔkulaçe}
\formephonétique{kuladʒe}
\région{GOs}
\variante{%
kulaye
\région{PA}}
\classe{v.stat.}
\newline
\sens{1}
(\domainesémantique{Santé, maladie})
\begin{glose}
\pfra{raide (être) ; courbaturé}
\end{glose}
\newline
\begin{exemple}
\région{GO}
\textbf{\pnua{kulaçe wa ne hi-nu}}
\pfra{mes muscles de bras sont raides, endoloris}
\end{exemple}
\newline
\sens{2}
(\domainesémantique{Description des objets, formes, consistance, taille})
\begin{glose}
\pfra{dur (pain)}
\end{glose}
\end{entrée}

\begin{entrée}{kula-kaze}{}{ⓔkula-kaze}
\région{GOs}
(\domainesémantique{Crustacés, crabes})
\classe{nom}
\begin{glose}
\pfra{langouste}
\end{glose}
\end{entrée}

\begin{entrée}{kula we ni ki}{}{ⓔkula we ni ki}
\région{GOs}
(\domainesémantique{Santé, maladie})
\classe{v}
\begin{glose}
\pfra{dysenterie (avoir la) ; diarhée (avoir la)}
\end{glose}
\end{entrée}

\begin{entrée}{kule}{}{ⓔkule}
\région{GOs PA}
\variante{%
kole, kula
}
\classe{v}
\newline
\sens{1}
(\domainesémantique{Actions liées aux éléments (liquide, fumée)})
\begin{glose}
\pfra{verser ; répandre ; vider}
\end{glose}
\newline
\begin{exemple}
\région{BO}
\textbf{\pnua{i kule pwal}}
\pfra{il pleut}
\end{exemple}
\newline
\sens{2}
(\domainesémantique{Pêche})
\begin{glose}
\pfra{déployer ; étendre (filet)}
\end{glose}
\newline
\begin{sous-entrée}{kule pwiò}{ⓔkuleⓢ2ⓝkule pwiò}
\région{GO}
\begin{glose}
\pfra{jeter, étendre le filet}
\end{glose}
\end{sous-entrée}
\end{entrée}

\begin{entrée}{kulèng}{}{ⓔkulèng}
\région{BO WE}
(\domainesémantique{Caractéristiques et propriétés des personnes})
\classe{v}
\begin{glose}
\pfra{fou ; imbécile}
\end{glose}
\begin{glose}
\pfra{saoûl}
\end{glose}
\newline
\relationsémantique{Cf.}{\lien{}{a-kulèng [BO]}}
\glosecourte{un fou}
\end{entrée}

\begin{entrée}{kuli}{}{ⓔkuli}
\région{GOs}
\variante{%
kule
\région{PA BO}}
\classe{v}
\newline
\sens{1}
(\domainesémantique{Processus liés aux plantes})
\begin{glose}
\pfra{tomber (tout seul: fruit, feuilles)}
\end{glose}
\newline
\begin{exemple}
\région{GO}
\textbf{\pnua{la ku}}
\pfra{ils sont tombés}
\end{exemple}
\newline
\begin{exemple}
\région{BO}
\textbf{\pnua{i kul}}
\pfra{il est tombé}
\end{exemple}
\newline
\sens{2}
(\domainesémantique{Processus liés aux plantes})
\begin{glose}
\pfra{perdre ses feuilles}
\end{glose}
\newline
\begin{exemple}
\région{GO}
\textbf{\pnua{la kuli dròò-ce}}
\pfra{les feuilles tombent}
\end{exemple}
\newline
\begin{exemple}
\région{GO}
\textbf{\pnua{yeewa kuli dròò-la}}
\pfra{c'est l'époque de la chute des feuilles}
\end{exemple}
\newline
\étymologie{
\langue{POc}
\étymon{*aqulu}
\glosecourte{tombé, détaché}}
\newline
\note{ku [GOs], kul [PA] (v.i.)}{grammaire}{tomber}
\end{entrée}

\begin{entrée}{kulò}{}{ⓔkulò}
\région{GOs BO}
(\domainesémantique{Préparation des aliments; modes de préparation et de cuisson})
\classe{nom}
\begin{glose}
\pfra{couverture du four (en peaux de niaoulis)}
\end{glose}
\end{entrée}

\begin{entrée}{kumala}{}{ⓔkumala}
\région{GOs BO PA}
\variante{%
kumwala
\région{BO PA}}
(\domainesémantique{Noms des plantes})
\classe{nom}
\begin{glose}
\pfra{patate douce}
\end{glose}
\nomscientifique{Ipomoea batatas}
\newline
\begin{sous-entrée}{kumala kari}{ⓔkumalaⓝkumala kari}
\begin{glose}
\pfra{patate douce jaune}
\end{glose}
\end{sous-entrée}
\newline
\begin{sous-entrée}{kumala èrona}{ⓔkumalaⓝkumala èrona}
\begin{glose}
\pfra{clone de patate douce}
\end{glose}
\end{sous-entrée}
\end{entrée}

\begin{entrée}{ku-manyõ}{}{ⓔku-manyõ}
\région{GOs}
(\domainesémantique{Ignames})
\classe{nom}
\begin{glose}
\pfra{igname (se ramifie comme le manioc)}
\end{glose}
\end{entrée}

\begin{entrée}{kumè}{}{ⓔkumè}
\région{GOs BO PA}
\classe{nom}
\newline
\sens{1}
(\domainesémantique{Corps humain})
\begin{glose}
\pfra{langue}
\end{glose}
\newline
\begin{exemple}
\région{GO}
\textbf{\pnua{kumèè-je}}
\pfra{sa langue}
\end{exemple}
\newline
\begin{exemple}
\région{PA}
\textbf{\pnua{kûmè-n, kûmèè-n}}
\pfra{sa langue}
\end{exemple}
\newline
\sens{2}
(\domainesémantique{Parties de plantes})
\begin{glose}
\pfra{bourgeon}
\end{glose}
\newline
\begin{exemple}
\région{GO}
\textbf{\pnua{kumèè-je}}
\pfra{son bourgeon}
\end{exemple}
\newline
\begin{exemple}
\région{PA}
\textbf{\pnua{kumèè-n}}
\pfra{son bourgeon}
\end{exemple}
\newline
\begin{sous-entrée}{kumè ê}{ⓔkumèⓢ2ⓝkumè ê}
\région{GO}
\begin{glose}
\pfra{bourgeon de canne à sucre}
\end{glose}
\end{sous-entrée}
\newline
\begin{sous-entrée}{kumèè chaamwa}{ⓔkumèⓢ2ⓝkumèè chaamwa}
\région{GO PA}
\begin{glose}
\pfra{le coeur du bananier}
\end{glose}
\end{sous-entrée}
\newline
\begin{sous-entrée}{kumèè nu}{ⓔkumèⓢ2ⓝkumèè nu}
\région{GO PA}
\begin{glose}
\pfra{le coeur de cocotier (comestible)}
\end{glose}
\end{sous-entrée}
\end{entrée}

\begin{entrée}{kumee}{}{ⓔkumee}
\région{GOs BO}
(\domainesémantique{Arbre
, Parties de plantes})
\classe{nom}
\begin{glose}
\pfra{cime (arbre)}
\end{glose}
\newline
\begin{sous-entrée}{kumee-ce}{ⓔkumeeⓝkumee-ce}
\begin{glose}
\pfra{la cime de l'arbre}
\end{glose}
\end{sous-entrée}
\end{entrée}

\begin{entrée}{kumèè chaamwa}{}{ⓔkumèè chaamwa}
\région{GOsPA}
(\domainesémantique{Bananiers et bananes})
\classe{nom}
\begin{glose}
\pfra{coeur du bananier}
\end{glose}
\end{entrée}

\begin{entrée}{kumèè nu}{}{ⓔkumèè nu}
\région{GO PA BO}
(\domainesémantique{Cocotiers})
\classe{nom}
\begin{glose}
\pfra{coeur de cocotier (comestible)}
\end{glose}
\end{entrée}

\begin{entrée}{kûmèè tòòmwa}{}{ⓔkûmèè tòòmwa}
\région{BO}
(\domainesémantique{Noms des plantes})
\classe{nom}
\begin{glose}
\pfra{"langue de femme"}
\end{glose}
\newline
\note{(petit buisson aux feuilles épaisses et succulentes [Corne])}{glose}{}
\nomscientifique{Kalanchoëe pinnata}
\end{entrée}

\begin{entrée}{kumõõ}{}{ⓔkumõõ}
\région{GOs}
(\domainesémantique{Poissons})
\classe{nom}
\begin{glose}
\pfra{rouget}
\end{glose}
\end{entrée}

\begin{entrée}{kun}{}{ⓔkun}
\région{GO PA BO}
\variante{%
kunõng
\région{BO}, 
ku(n)
\région{GO(s)}}
(\domainesémantique{Noms locatifs})
\classe{nom}
\begin{glose}
\pfra{endroit}
\end{glose}
\newline
\note{"kun" contient le "phwe-meewu"}{glose}{}
\newline
\begin{exemple}
\textbf{\pnua{na ni kun-òli}}
\pfra{dans cet endroit-là}
\end{exemple}
\newline
\begin{exemple}
\région{PA}
\textbf{\pnua{na ni kuni-m, na ni kuna-n}}
\pfra{à ta place}
\end{exemple}
\newline
\begin{exemple}
\région{BO}
\textbf{\pnua{na ni kunõ-ny}}
\pfra{dans mon endroit}
\end{exemple}
\newline
\begin{sous-entrée}{phe kunã}{ⓔkunⓝphe kunã}
\begin{glose}
\pfra{prendre la place de qqn}
\end{glose}
\end{sous-entrée}
\end{entrée}

\begin{entrée}{kun-êba}{}{ⓔkun-êba}
\région{GOs}
(\domainesémantique{Directions})
\classe{LOC}
\begin{glose}
\pfra{là sur le côté ; à cet endroit latéralement}
\end{glose}
\end{entrée}

\begin{entrée}{kun-êda}{}{ⓔkun-êda}
\région{GOs}
(\domainesémantique{Directions})
\classe{LOC}
\begin{glose}
\pfra{là en haut ; à cet endroit en haut}
\end{glose}
\end{entrée}

\begin{entrée}{kun-êdu}{}{ⓔkun-êdu}
\région{GOs}
(\domainesémantique{Directions})
\classe{LOC}
\begin{glose}
\pfra{là en bas ; à cet endroit en bas}
\end{glose}
\end{entrée}

\begin{entrée}{kuni-m}{}{ⓔkuni-m}
\région{BO}
(\domainesémantique{Directions})
\classe{LOC}
\begin{glose}
\pfra{là de ton côté [BM]}
\end{glose}
\end{entrée}

\begin{entrée}{kuńô}{}{ⓔkuńô}
\formephonétique{kunõ}
\région{GOs}
(\domainesémantique{Insectes})
\classe{nom}
\begin{glose}
\pfra{asticot}
\end{glose}
\newline
\étymologie{
\langue{POc}
\étymon{*qulos}}
\end{entrée}

\begin{entrée}{ku-nõgo}{}{ⓔku-nõgo}
\formephonétique{ku-ɳɔ̃ŋgo}
\région{GOs}
\variante{%
ku-nõgò
\région{BO PA}}
(\domainesémantique{Topographie})
\classe{nom}
\begin{glose}
\pfra{amont de la rivière (vers la source)}
\end{glose}
\newline
\relationsémantique{Cf.}{\lien{ⓔnõgòⓢ2ⓝphwe-nõgò}{phwe-nõgò}}
\glosecourte{source}
\end{entrée}

\begin{entrée}{kun-òli}{}{ⓔkun-òli}
\formephonétique{kuɳ-ɔli}
\région{GOs}
\variante{%
kun-òli
\région{PA}}
(\domainesémantique{Localisation})
\classe{ADV}
\begin{glose}
\pfra{de l'autre côté là-bas (de la montagne, de la rivière)}
\end{glose}
\newline
\begin{exemple}
\région{GO PA}
\textbf{\pnua{no je mwa ge èba kun-òli}}
\pfra{regarde la maison sur le côté de de l'autre côté là-bas}
\end{exemple}
\end{entrée}

\begin{entrée}{ku-pe-bala}{}{ⓔku-pe-bala}
\formephonétique{ku-βe-bala}
\région{GOs}
\variante{%
ku-vwe-bala
\région{GO(s)}}
(\domainesémantique{Préfixes et verbes de position})
\classe{v}
\begin{glose}
\pfra{aligné ; côte à côte}
\end{glose}
\newline
\begin{sous-entrée}{ku-vwebala}{ⓔku-pe-balaⓝku-vwebala}
\begin{glose}
\pfra{debout en ligne, côte à côte}
\end{glose}
\end{sous-entrée}
\newline
\begin{sous-entrée}{te-vwebala}{ⓔku-pe-balaⓝte-vwebala}
\begin{glose}
\pfra{assis en ligne}
\end{glose}
\newline
\relationsémantique{Cf.}{\lien{ⓔpe-bala}{pe-bala}}
\glosecourte{équipe}
\end{sous-entrée}
\end{entrée}

\begin{entrée}{ku-peena}{}{ⓔku-peena}
\région{GOs PA BO}
(\domainesémantique{Ignames})
\classe{nom}
\begin{glose}
\pfra{igname (pousse comme une anguille)}
\end{glose}
\end{entrée}

\begin{entrée}{ku-pô}{}{ⓔku-pô}
\région{GOs}
\variante{%
ku-pòng
\région{BO WEM WE}}
(\domainesémantique{Préfixes et verbes de position})
\classe{v}
\begin{glose}
\pfra{debout tordu ; de travers}
\end{glose}
\newline
\begin{exemple}
\textbf{\pnua{e ku-pô}}
\pfra{c'est tordu (d'un mur)}
\end{exemple}
\newline
\begin{exemple}
\textbf{\pnua{ku-pô duviju}}
\pfra{le clou est tordu}
\end{exemple}
\end{entrée}

\begin{entrée}{ku-poxe}{}{ⓔku-poxe}
\région{PA}
(\domainesémantique{Verbes de mouvement})
\classe{v}
\begin{glose}
\pfra{rassembler (se) (lit. debout-un)}
\end{glose}
\newline
\relationsémantique{Cf.}{\lien{ⓔtree-poxe}{tree-poxe}}
\glosecourte{assis ensemble}
\end{entrée}

\begin{entrée}{ku-phwa}{}{ⓔku-phwa}
\région{GOs}
(\domainesémantique{Topographie})
\classe{nom}
\begin{glose}
\pfra{haut de la vallée}
\end{glose}
\end{entrée}

\begin{entrée}{kurô}{}{ⓔkurô}
\région{BO}
(\domainesémantique{Corps humain})
\classe{nom}
\begin{glose}
\pfra{clitoris [Corne]}
\end{glose}
\end{entrée}

\begin{entrée}{ku-tibu}{}{ⓔku-tibu}
\région{GOs}
(\domainesémantique{Préfixes et verbes de position
, Mouvements ou actions faits avec le corps, les bras, les mains, les pieds})
\classe{v}
\begin{glose}
\pfra{accouder (s')}
\end{glose}
\end{entrée}

\begin{entrée}{ku-tua}{}{ⓔku-tua}
\région{GOs}
\variante{%
kutuwa
\région{GO(s)}}
(\domainesémantique{Ignames})
\classe{nom}
\begin{glose}
\pfra{igname sauvage}
\end{glose}
\nomscientifique{Dioscorea alata sauvage}
\end{entrée}

\begin{entrée}{kutra}{}{ⓔkutra}
\formephonétique{kuɽa, kuʈa}
\région{GOs}
\variante{%
kura
\région{BO PA}}
\newline
\groupe{A}
(\domainesémantique{Corps humain})
\classe{nom}
\begin{glose}
\pfra{sang}
\end{glose}
\newline
\begin{exemple}
\région{GO}
\textbf{\pnua{kutraa-je}}
\pfra{son sang}
\end{exemple}
\newline
\begin{exemple}
\région{PA}
\textbf{\pnua{kura-n}}
\pfra{son sang}
\end{exemple}
\newline
\groupe{B}
(\domainesémantique{Fonctions naturelles humaines})
\classe{v}
\begin{glose}
\pfra{saigner}
\end{glose}
\newline
\begin{exemple}
\région{PA}
\textbf{\pnua{e mõlò kuraa-je ; e mõlò kuraa-nu}}
\pfra{il est nerveux (lit. son sang est vivant) ; je suis nerveux}
\end{exemple}
\newline
\étymologie{
\langue{POc}
\étymon{*ɖaRaq}}
\end{entrée}

\begin{entrée}{kutru}{}{ⓔkutru}
\formephonétique{kuʈu, kuɽu}
\région{GOs}
\variante{%
kuru
\formephonétique{kuɽu}
\région{GO(s)}, 
kuru
\région{PA BO}}
(\domainesémantique{Parties de plantes})
\classe{nom}
\begin{glose}
\pfra{tubercule du taro d'eau (uvha) ; taro d'eau (terme employé dans le contexte coutumier)}
\end{glose}
\newline
\begin{exemple}
\textbf{\pnua{kuru mani kui}}
\pfra{taros et ignames}
\end{exemple}
\newline
\relationsémantique{Cf.}{\lien{ⓔuva}{uva}}
\glosecourte{pied de taro d'eau}
\end{entrée}

\begin{entrée}{ku-thralò}{}{ⓔku-thralò}
\région{GOs}
(\domainesémantique{Relations et interaction sociales})
\classe{v}
\begin{glose}
\pfra{provoquer (se)}
\end{glose}
\end{entrée}

\begin{entrée}{kuu}{}{ⓔkuu}
\région{GOs}
\variante{%
kuun
\région{BO}}
(\domainesémantique{Topographie})
\classe{nom}
\begin{glose}
\pfra{fond de la vallée}
\end{glose}
\begin{glose}
\pfra{amont du creek}
\end{glose}
\newline
\étymologie{
\langue{POc}
\étymon{*qulu}
\glosecourte{head, top, upper end of valley}
\auteur{Geraghty}}
\end{entrée}

\begin{entrée}{kûû}{}{ⓔkûû}
\région{GOs BO}
\classe{v}
\newline
\sens{1}
(\domainesémantique{Mouvements ou actions avec la tête, les yeux, la bouche
, Aliments, alimentation})
\begin{glose}
\pfra{croquer}
\end{glose}
\newline
\sens{2}
(\domainesémantique{Aliments, alimentation})
\begin{glose}
\pfra{manger (des végétaux, fruits)}
\end{glose}
\newline
\note{kûûńi (v.t.)}{grammaire}{}
\end{entrée}

\begin{entrée}{kûû-}{}{ⓔkûû-}
\région{GOs PA BO}
(\domainesémantique{Préfixes classificateurs possessifs de la nourriture})
\classe{CLF.POSS}
\begin{glose}
\pfra{part de fruits ou de feuilles}
\end{glose}
\newline
\begin{exemple}
\région{GO}
\textbf{\pnua{kûû-nu pò-mã}}
\pfra{ma mangue}
\end{exemple}
\newline
\begin{exemple}
\région{PA}
\textbf{\pnua{kû-n}}
\pfra{sa part}
\end{exemple}
\newline
\begin{exemple}
\région{PA}
\textbf{\pnua{kûû-m e cin}}
\pfra{ta part de papaye}
\end{exemple}
\newline
\relationsémantique{Cf.}{\lien{}{caa- ; ce- [GOs] ; ho-}}
\end{entrée}

\begin{entrée}{kuue}{}{ⓔkuue}
\région{GOs}
(\domainesémantique{Mouvements ou actions faits avec le corps, les bras, les mains, les pieds})
\classe{v}
\begin{glose}
\pfra{tenir, retenir (un animal par une bride ou une corde)}
\end{glose}
\end{entrée}

\begin{entrée}{kuu-jaaò}{}{ⓔkuu-jaaò}
\formephonétique{kuː-ɲɟaːɔ}
\région{GOs}
\région{BO}
\variante{%
kuu-jaaòl
}
(\domainesémantique{Topographie})
\classe{nom}
\begin{glose}
\pfra{amont du fleuve}
\end{glose}
\end{entrée}

\begin{entrée}{kûûni}{2}{ⓔkûûniⓗ2}
\formephonétique{kûːɳi}
\région{GOs BO PA}
(\domainesémantique{Aspect})
\classe{v.t.}
\begin{glose}
\pfra{finir ; terminer}
\end{glose}
\newline
\begin{exemple}
\région{BO}
\textbf{\pnua{nu kûûni nyâ thòm}}
\pfra{j'ai fini cette natte}
\end{exemple}
\newline
\begin{exemple}
\région{GO}
\textbf{\pnua{nu kûûni hõbwoli-nu}}
\pfra{j'ai fini ma natte}
\end{exemple}
\end{entrée}

\begin{entrée}{kûûńi}{1}{ⓔkûûńiⓗ1}
\formephonétique{kûːni}
\région{GOs PA}
\variante{%
kôôni
\région{BO}}
(\domainesémantique{Aliments, alimentation})
\classe{v}
\begin{glose}
\pfra{manger (des fruits, feuilles)}
\end{glose}
\newline
\relationsémantique{Cf.}{\lien{ⓔhovwo}{hovwo}}
\glosecourte{manger (général)}
\newline
\relationsémantique{Cf.}{\lien{}{cèni ; cani}}
\glosecourte{manger (féculents)}
\newline
\relationsémantique{Cf.}{\lien{ⓔhuu}{huu}}
\glosecourte{manger (nourriture carnée)}
\newline
\relationsémantique{Cf.}{\lien{ⓔbiije}{biije}}
\glosecourte{mâcher des écorces ou du magnania}
\newline
\relationsémantique{Cf.}{\lien{}{whizi ; wili}}
\glosecourte{manger (canne à sucre)}
\end{entrée}

\begin{entrée}{kuvêê}{}{ⓔkuvêê}
\région{GOs PA}
\variante{%
kuveen
\région{BO}}
(\domainesémantique{Parties de plantes})
\classe{nom}
\begin{glose}
\pfra{pousses (toutes plantes)}
\end{glose}
\begin{glose}
\pfra{bourgeon}
\end{glose}
\begin{glose}
\pfra{jeunes feuilles (par ex. de taro, quand la feuille commence à se déplier)}
\end{glose}
\end{entrée}

\begin{entrée}{kuvêê nu}{}{ⓔkuvêê nu}
\région{PA}
(\domainesémantique{Cocotiers})
\classe{nom}
\begin{glose}
\pfra{coco germé}
\end{glose}
\end{entrée}

\begin{entrée}{kuvêê-uvhia}{}{ⓔkuvêê-uvhia}
\région{GOs PA}
(\domainesémantique{Taros})
\classe{nom}
\begin{glose}
\pfra{jeunes feuilles de taro de montagne}
\end{glose}
\end{entrée}

\begin{entrée}{ku-wãga}{}{ⓔku-wãga}
\région{GOs}
(\domainesémantique{Préfixes et verbes de position})
\classe{v}
\begin{glose}
\pfra{debout jambes écartées}
\end{glose}
\end{entrée}

\begin{entrée}{kuwe}{}{ⓔkuwe}
\région{GOs BO}
(\domainesémantique{Ignames})
\classe{nom}
\begin{glose}
\pfra{igname mauve}
\end{glose}
\end{entrée}

\begin{entrée}{ku-wee}{}{ⓔku-wee}
\région{GOs}
(\domainesémantique{Ignames})
\classe{nom}
\begin{glose}
\pfra{igname (au goût sucré)}
\end{glose}
\end{entrée}

\begin{entrée}{ku-xea}{}{ⓔku-xea}
\région{GOs}
\variante{%
ku-xia
\région{GO(s)}}
(\domainesémantique{Préfixes et verbes de position
, Verbes de mouvement})
\classe{v}
\begin{glose}
\pfra{adosser (s') ; adossé}
\end{glose}
\newline
\begin{exemple}
\textbf{\pnua{nu ku-xea ni gòò-mwa}}
\pfra{je suis adossé au mur}
\end{exemple}
\end{entrée}

\begin{entrée}{kûxû}{1}{ⓔkûxûⓗ1}
\région{GOs}
\variante{%
kûû
\région{PA BO}, 
kûkû
\région{BO}}
(\domainesémantique{Fonctions naturelles humaines})
\classe{v}
\begin{glose}
\pfra{têter ; sucer}
\end{glose}
\newline
\begin{exemple}
\région{GO}
\textbf{\pnua{e pa-kûxû-ni pòi-je}}
\pfra{elle allaite son enfant}
\end{exemple}
\newline
\begin{exemple}
\région{PA}
\textbf{\pnua{e pa-kûûe pòi-n}}
\pfra{elle allaite son enfant}
\end{exemple}
\newline
\begin{sous-entrée}{pa-kûxû-ni}{ⓔkûxûⓗ1ⓝpa-kûxû-ni}
\begin{glose}
\pfra{allaiter (v.t.)}
\end{glose}
\end{sous-entrée}
\end{entrée}

\begin{entrée}{kûxû}{2}{ⓔkûxûⓗ2}
\région{GOs}
\variante{%
kûxûl
\région{BO PA}}
(\domainesémantique{Sons, bruits})
\classe{v}
\begin{glose}
\pfra{grogner ; gronder ; grommeler ; maugréer ; bougonner}
\end{glose}
\end{entrée}

\begin{entrée}{kûxû}{3}{ⓔkûxûⓗ3}
\région{GOs}
(\domainesémantique{Poissons})
\classe{nom}
\begin{glose}
\pfra{carangue (petite)}
\end{glose}
\newline
\note{(selon les locuteurs, elle est appelée ainsiparce qu'elle 'murmure' (cf kûxû) quand on la sort de l'eau)}{glose}{}
\newline
\relationsémantique{Cf.}{\lien{}{dròò-xibö}}
\glosecourte{carangue (la même) de taille moyenne}
\end{entrée}

\begin{entrée}{ku-yabo}{}{ⓔku-yabo}
\région{GOs}
\variante{%
ku-yòbo
\région{BO PA}}
(\domainesémantique{Verbes d'action (en général)})
\classe{v}
\begin{glose}
\pfra{attendre}
\end{glose}
\newline
\begin{exemple}
\région{GO}
\textbf{\pnua{çö ku-yòbo-nu ?}}
\pfra{tu m'attends ?}
\end{exemple}
\newline
\begin{exemple}
\région{PA}
\textbf{\pnua{ku-yòbo-nu !}}
\pfra{attends-moi !}
\end{exemple}
\newline
\begin{exemple}
\région{BO}
\textbf{\pnua{i yòboe egu}}
\pfra{il attend qqn.}
\end{exemple}
\end{entrée}

\begin{entrée}{kuzaò}{}{ⓔkuzaò}
\région{GOs}
\variante{%
kuraò
\région{WEM WEH}}
(\domainesémantique{Quantificateurs
, Marques de degré})
\classe{QNT}
\begin{glose}
\pfra{trop (en) ; en surplus}
\end{glose}
\newline
\begin{exemple}
\textbf{\pnua{e tree-kuzaò pò-ko}}
\pfra{il en reste 3 en plus}
\end{exemple}
\newline
\begin{exemple}
\textbf{\pnua{e tree-kuzaò haa-tru mada}}
\pfra{il reste 2 pièces de tissu en plus}
\end{exemple}
\end{entrée}

\newpage

\lettrine{kh}\begin{entrée}{kha}{1}{ⓔkhaⓗ1}
\région{GOs PA}
\variante{%
ka
\région{GOs PA}}
(\domainesémantique{Préfixes sémantiques divers})
\classe{PREF}
\begin{glose}
\pfra{attribue une propriété}
\end{glose}
\newline
\begin{exemple}
\région{GO}
\textbf{\pnua{bî gaa kha-gi}}
\pfra{nous on pleurait sans cesse}
\end{exemple}
\newline
\begin{exemple}
\textbf{\pnua{e a-kha-gi}}
\pfra{c'est un pleurnicheur, il pleure pour un rien, il est sensible}
\end{exemple}
\newline
\begin{exemple}
\textbf{\pnua{e a-kha thô}}
\pfra{elle est susceptible, se met en colère pour rien, colèreu}
\end{exemple}
\newline
\begin{exemple}
\textbf{\pnua{e a-kha-mani}}
\pfra{il dort sans cesse, facilement}
\end{exemple}
\newline
\begin{exemple}
\région{GO}
\textbf{\pnua{e a-kha-phorõ}}
\pfra{il perd la mémoire (= fort)}
\end{exemple}
\end{entrée}

\begin{entrée}{kha}{2}{ⓔkhaⓗ2}
\région{GOs}
\variante{%
khan
\région{PA BO}}
(\domainesémantique{Types de champs})
\classe{nom}
\begin{glose}
\pfra{champ ; culture en forêt défrichée}
\end{glose}
\end{entrée}

\begin{entrée}{kha}{3}{ⓔkhaⓗ3}
\région{GOs PA BO}
\variante{%
khaa
\région{PA}}
\newline
\groupe{A}
\classe{v}
(\domainesémantique{Mouvements ou actions faits avec le corps, les bras, les mains, les pieds})
\begin{glose}
\pfra{appuyer}
\end{glose}
\begin{glose}
\pfra{tasser (avec les mains ou les pieds)}
\end{glose}
\begin{glose}
\pfra{masser}
\end{glose}
\begin{glose}
\pfra{presser}
\end{glose}
\begin{glose}
\pfra{écraser}
\end{glose}
\newline
\begin{exemple}
\région{GO}
\textbf{\pnua{e khaa-du hõbwò}}
\pfra{elle fait tremper le linge (en appuyant)}
\end{exemple}
\newline
\begin{sous-entrée}{khaa !}{ⓔkhaⓗ3ⓝkhaa !}
\région{PA}
\begin{glose}
\pfra{accélère !}
\end{glose}
\end{sous-entrée}
\newline
\begin{sous-entrée}{yai-khaa}{ⓔkhaⓗ3ⓝyai-khaa}
\région{GO}
\begin{glose}
\pfra{lampe électrique (lit. feu-appuyer)}
\end{glose}
\end{sous-entrée}
\newline
\groupe{B}
\classe{PREF}
\newline
\sens{1}
(\domainesémantique{Préfixes sémantiques d’action})
\begin{glose}
\pfra{action faite en appuyant avec le pied ou la main}
\end{glose}
\newline
\begin{exemple}
\région{GO}
\textbf{\pnua{la khaa pu-mwa}}
\pfra{ils font le mur de la maison en torchis}
\end{exemple}
\newline
\sens{2}
(\domainesémantique{Préfixes sémantiques de déplacement})
\begin{glose}
\pfra{faire qqch en se déplaçant à pied ou en mouvement}
\end{glose}
\begin{glose}
\pfra{déplacer (se) à pied}
\end{glose}
\newline
\begin{exemple}
\région{PA}
\textbf{\pnua{i khaa kule kile}}
\pfra{il laissé tomber (perdu) sa clé en marchant}
\end{exemple}
\newline
\begin{exemple}
\région{PA}
\textbf{\pnua{i khaa wal}}
\pfra{il chante en marchant}
\end{exemple}
\newline
\begin{exemple}
\région{PA}
\textbf{\pnua{i khaa gi}}
\pfra{il pleure en marchant}
\end{exemple}
\newline
\begin{exemple}
\région{BO}
\textbf{\pnua{i kha co-da}}
\pfra{il continue à monter}
\end{exemple}
\end{entrée}

\begin{entrée}{kha}{4}{ⓔkhaⓗ4}
\région{PA BO}
(\domainesémantique{Description des objets, formes, consistance, taille})
\classe{nom}
\begin{glose}
\pfra{fente ; craquelure ; craquelé; fissuré (terre)}
\end{glose}
\end{entrée}

\begin{entrée}{kha}{5}{ⓔkhaⓗ5}
\région{GOs}
(\domainesémantique{Préfixes sémantiques divers})
\classe{PREF.}
\begin{glose}
\pfra{action faite en même temps}
\end{glose}
\newline
\begin{exemple}
\textbf{\pnua{e kha-tho}}
\pfra{elle appelle (en même temps)}
\end{exemple}
\newline
\begin{exemple}
\région{GO}
\textbf{\pnua{e u tree kha nõõ-li na-bòli}}
\pfra{elle les a déjà aperçus en se déplaçant de loin là-bas}
\end{exemple}
\newline
\begin{exemple}
\région{GO}
\textbf{\pnua{nu xa kha-thoe hii-nu}}
\pfra{j'ai tendu le bras en même temps}
\end{exemple}
\newline
\note{indique une action faite en se déplaçant, en même temps qu'une autre action (notion de simultanéité); s'y ajoute parfois une notion de médiativité (evidential) en association avec un verbe de perception: le préfixe indique alors une perception non voulue, indirecte, faite en se déplaçant}{grammaire}{}
\end{entrée}

\begin{entrée}{kha-}{}{ⓔkha-}
\région{GOs PA BO}
(\domainesémantique{Distributifs})
\classe{DISTR}
\begin{glose}
\pfra{chaque ; chacun (+ numéral)}
\end{glose}
\newline
\begin{exemple}
\région{GO}
\textbf{\pnua{na vwo kha-potru}}
\pfra{dispose-les par 2}
\end{exemple}
\newline
\begin{exemple}
\région{GO}
\textbf{\pnua{na pe-kha-poxe, pe-kha-potru, etc.}}
\pfra{mets les par 1, 2,}
\end{exemple}
\newline
\begin{exemple}
\région{GO}
\textbf{\pnua{na wo kha-potru po-mã cai êgu kha a-xe}}
\pfra{donne les mangues par deux à chacune des personnes}
\end{exemple}
\newline
\begin{exemple}
\région{GO}
\textbf{\pnua{na cai kha-axe êgu kha-potru po-mã}}
\pfra{donne à chaque personne deux mangues chacun}
\end{exemple}
\newline
\begin{exemple}
\région{GO}
\textbf{\pnua{li vara kha-we-xe loto}}
\pfra{chacun des deux a une voiture}
\end{exemple}
\newline
\begin{exemple}
\textbf{\pnua{mo pe-kha-a-niza na ni ba ?}}
\pfra{nous (paucal) sommes combien par /dans chaque équipe ?}
\end{exemple}
\newline
\begin{exemple}
\région{PA}
\textbf{\pnua{na vwo kha-potu}}
\pfra{dispose-les par 2}
\end{exemple}
\newline
\begin{sous-entrée}{ka-õxe}{ⓔkha-ⓝka-õxe}
\région{BO}
\begin{glose}
\pfra{quelquefois (Dubois)}
\end{glose}
\end{sous-entrée}
\newline
\begin{sous-entrée}{ka-põge}{ⓔkha-ⓝka-põge}
\région{BO}
\begin{glose}
\pfra{chaque, chacun (inanimé) (Dubois)}
\end{glose}
\end{sous-entrée}
\newline
\begin{sous-entrée}{ka-age}{ⓔkha-ⓝka-age}
\région{BO}
\begin{glose}
\pfra{chacun (Dubois)}
\end{glose}
\end{sous-entrée}
\newline
\begin{sous-entrée}{kau ka-wege}{ⓔkha-ⓝkau ka-wege}
\région{BO}
\begin{glose}
\pfra{chaque année (Dubois)}
\end{glose}
\end{sous-entrée}
\newline
\begin{sous-entrée}{õ-ge na ni kau ka-wege}{ⓔkha-ⓝõ-ge na ni kau ka-wege}
\région{BO}
\begin{glose}
\pfra{une fois par an (Dubois)}
\end{glose}
\end{sous-entrée}
\end{entrée}

\begin{entrée}{khaa}{1}{ⓔkhaaⓗ1}
\région{GOs PA BO}
(\domainesémantique{Eau})
\classe{nom}
\begin{glose}
\pfra{mare ; étang}
\end{glose}
\end{entrée}

\begin{entrée}{khaa}{2}{ⓔkhaaⓗ2}
\région{GOs}
(\domainesémantique{Interaction avec les animaux})
\classe{v}
\begin{glose}
\pfra{dresser (cheval)}
\end{glose}
\end{entrée}

\begin{entrée}{khaa-bîni}{}{ⓔkhaa-bîni}
\formephonétique{kʰaː-bîɳi}
\région{GOs}
\variante{%
khaa-bîni
\région{PA}}
(\domainesémantique{Mouvements ou actions faits avec le corps, les bras, les mains, les pieds
, Préfixes sémantiques d’action})
\classe{v}
\begin{glose}
\pfra{écraser (avec le pied)}
\end{glose}
\begin{glose}
\pfra{aplatir}
\end{glose}
\end{entrée}

\begin{entrée}{khaabu}{}{ⓔkhaabu}
\région{GOs BO}
(\domainesémantique{Fonctions naturelles humaines})
\classe{v.stat.}
\begin{glose}
\pfra{froid (avoir)}
\end{glose}
\newline
\begin{exemple}
\région{BO PA}
\textbf{\pnua{nu hai khaabu}}
\pfra{j'ai très froid}
\end{exemple}
\newline
\begin{exemple}
\région{GO}
\textbf{\pnua{nu hai-xaabu}}
\pfra{j'ai très froid}
\end{exemple}
\newline
\begin{sous-entrée}{wara khabu}{ⓔkhaabuⓝwara khabu}
\région{BO}
\begin{glose}
\pfra{hiver}
\end{glose}
\newline
\relationsémantique{Cf.}{\lien{}{tuuçò, tuyong}}
\end{sous-entrée}
\end{entrée}

\begin{entrée}{khaa-êgo}{}{ⓔkhaa-êgo}
\région{GOs}
\variante{%
khaa-pi
\région{WE WEM}, 
khaa-vwi
\région{GO(s)}}
(\domainesémantique{Fonctions naturelles des animaux})
\classe{v}
\begin{glose}
\pfra{couver (des oeufs)}
\end{glose}
\newline
\relationsémantique{Cf.}{\lien{}{khaa-pi ; khaa-vwi}}
\glosecourte{pondre, couver}
\end{entrée}

\begin{entrée}{khaagi}{}{ⓔkhaagi}
\région{GOs}
(\domainesémantique{Sentiments})
\classe{v}
\begin{glose}
\pfra{retenir (se) de pleurer}
\end{glose}
\end{entrée}

\begin{entrée}{khaaimudre}{}{ⓔkhaaimudre}
\formephonétique{kʰaːi'muɖe}
\région{GOs}
\variante{%
khaimode
\région{BO}}
(\domainesémantique{Manière de faire l’action : verbes et adverbes de manière})
\classe{n ; v.stat.}
\begin{glose}
\pfra{résolution ; résolu ; définitif (être) ; définitivement}
\end{glose}
\newline
\begin{exemple}
\textbf{\pnua{nu wedoni khaimode}}
\pfra{j'ai fermement/définitivement décidé}
\end{exemple}
\end{entrée}

\begin{entrée}{kha-alawe}{}{ⓔkha-alawe}
\région{GOs}
\variante{%
kha-olae
\région{PA}}
(\domainesémantique{Préfixes sémantiques de déplacement})
\classe{v}
\begin{glose}
\pfra{partir en disant au-revoir}
\end{glose}
\newline
\begin{exemple}
\région{PA}
\textbf{\pnua{i kha-olae-nu vwo gèè}}
\pfra{grand-mère est partie en me disant au-revoir}
\end{exemple}
\end{entrée}

\begin{entrée}{khaa-pi}{}{ⓔkhaa-pi}
\formephonétique{kʰaː-βi}
\région{GOs}
\variante{%
khaa-vwi
}
(\domainesémantique{Fonctions naturelles des animaux})
\classe{v}
\begin{glose}
\pfra{pondre}
\end{glose}
\newline
\relationsémantique{Cf.}{\lien{}{khaa-êgo ; thu êgo}}
\glosecourte{pondre, couver}
\end{entrée}

\begin{entrée}{khaa-tia}{}{ⓔkhaa-tia}
\région{GOs}
\variante{%
khaa-zia
\région{GOs}}
(\domainesémantique{Mouvements ou actions faits avec le corps, les bras, les mains, les pieds
, Préfixes sémantiques d’action})
\classe{v}
\begin{glose}
\pfra{bousculer qqn}
\end{glose}
\newline
\begin{exemple}
\textbf{\pnua{e khaa-zia nu}}
\pfra{il m'a poussé}
\end{exemple}
\newline
\relationsémantique{Cf.}{\lien{ⓔtia}{tia}}
\glosecourte{pousser}
\end{entrée}

\begin{entrée}{kha-axe}{}{ⓔkha-axe}
\région{GOs}
(\domainesémantique{Quantificateurs})
\classe{DISTR}
\begin{glose}
\pfra{chacun(e)}
\end{glose}
\newline
\begin{exemple}
\textbf{\pnua{li za kha-axe chovwa vwo li za u a}}
\pfra{ils ont chacun un cheval pour partir}
\end{exemple}
\end{entrée}

\begin{entrée}{kha-bazae}{}{ⓔkha-bazae}
\formephonétique{kʰa-'baðae}
\région{GOs}
(\domainesémantique{Préfixes sémantiques de déplacement})
\classe{v}
\begin{glose}
\pfra{dépasser en se déplaçant}
\end{glose}
\newline
\begin{exemple}
\textbf{\pnua{e kha-bazae-çö}}
\pfra{il t'a dépassé}
\end{exemple}
\end{entrée}

\begin{entrée}{khabe}{}{ⓔkhabe}
\formephonétique{kʰabe}
\région{GOs PA BO}
\classe{v}
\newline
\sens{1}
(\domainesémantique{Mouvements ou actions faits avec le corps, les bras, les mains, les pieds})
\begin{glose}
\pfra{dresser (poteau, etc.)}
\end{glose}
\begin{glose}
\pfra{taper pour enfoncer (poteau)}
\end{glose}
\begin{glose}
\pfra{enfoncer}
\end{glose}
\newline
\sens{2}
(\domainesémantique{Types de maison, architecture de la maison})
\begin{glose}
\pfra{construire (maison)}
\end{glose}
\newline
\begin{exemple}
\région{PA}
\textbf{\pnua{la pe-zage u la khabe nye mwa}}
\pfra{ils s'entraident pour construire cette maison}
\end{exemple}
\newline
\sens{3}
(\domainesémantique{Organisation sociale})
\begin{glose}
\pfra{établir, instituer}
\end{glose}
\end{entrée}

\begin{entrée}{khabe nõbu}{}{ⓔkhabe nõbu}
\formephonétique{kʰabe ɳɔ̃bu}
\région{GOs BO}
(\domainesémantique{Religion, représentations religieuses})
\classe{v}
\begin{glose}
\pfra{mettre le tabou}
\end{glose}
\newline
\begin{exemple}
\textbf{\pnua{e khabe nõbu}}
\pfra{il a planté une perche d'interdiction}
\end{exemple}
\newline
\begin{exemple}
\textbf{\pnua{nõbu-ã}}
\pfra{nos lois}
\end{exemple}
\newline
\relationsémantique{Cf.}{\lien{ⓔnõbuⓢ2ⓝphu nõbu}{phu nõbu}}
\glosecourte{enlever un interdit}
\end{entrée}

\begin{entrée}{khaboi}{}{ⓔkhaboi}
\région{BO}
(\domainesémantique{Mouvements ou actions faits avec le corps, les bras, les mains, les pieds})
\classe{v}
\begin{glose}
\pfra{étendre la main horizontalement (comme pour tapoter) [Corne]}
\end{glose}
\end{entrée}

\begin{entrée}{khabwa}{}{ⓔkhabwa}
\région{GOs}
(\domainesémantique{Marées})
\classe{nom}
\begin{glose}
\pfra{marée descendante}
\end{glose}
\end{entrée}

\begin{entrée}{kha-bwaroe}{}{ⓔkha-bwaroe}
\région{PA}
(\domainesémantique{Verbes de déplacement et moyens de déplacement
, Préfixes sémantiques de déplacement})
\classe{v}
\begin{glose}
\pfra{déplacer (se) en portant dans les bras}
\end{glose}
\end{entrée}

\begin{entrée}{kha-çaaxò}{}{ⓔkha-çaaxò}
\formephonétique{kʰa-'ʒaːɣɔ, kʰa-'dʒaːɣɔ}
\région{GOs}
\variante{%
kacaaò
\région{BO}}
(\domainesémantique{Verbes de déplacement et moyens de déplacement
, Préfixes sémantiques de déplacement})
\classe{v}
\begin{glose}
\pfra{marcher sans bruit ; déplacer (se) doucement}
\end{glose}
\newline
\begin{exemple}
\région{GO}
\textbf{\pnua{e kha-çaaxò da}}
\pfra{il monte sans bruit}
\end{exemple}
\newline
\begin{exemple}
\région{GOs BO}
\textbf{\pnua{kha-çaaxò ma mani}}
\pfra{doucement l'enfant dort}
\end{exemple}
\newline
\begin{exemple}
\région{GO}
\textbf{\pnua{thala çaaxo-ni pweemwa !}}
\pfra{ouvre la porte doucement !}
\end{exemple}
\newline
\relationsémantique{Cf.}{\lien{ⓔku-çaaxò}{ku-çaaxò}}
\glosecourte{en se cachant}
\end{entrée}

\begin{entrée}{khaçańi}{}{ⓔkhaçańi}
\formephonétique{kʰa'ʒani}
\région{GOs}
(\domainesémantique{Oiseaux})
\classe{nom}
\begin{glose}
\pfra{hirondelle des grottes}
\end{glose}
\nomscientifique{Collocalia}
\end{entrée}

\begin{entrée}{kha-cimwî}{}{ⓔkha-cimwî}
\région{PA}
(\domainesémantique{Mouvements ou actions faits avec le corps, les bras, les mains, les pieds
, Préfixes sémantiques de déplacement})
\classe{v}
\begin{glose}
\pfra{saisir en se déplaçant (en emportant ou amenant)}
\end{glose}
\newline
\begin{exemple}
\textbf{\pnua{kha-cimwî je-nã poxa kui !}}
\pfra{venez prendre cette petite igname}
\end{exemple}
\newline
\begin{exemple}
\textbf{\pnua{i kha-cimwî-mi je-nã kui !}}
\pfra{il arrive en apportant cette igname}
\end{exemple}
\newline
\begin{exemple}
\textbf{\pnua{i kha-cimwî-ò je-nã kui !}}
\pfra{il part en emportant cette igname}
\end{exemple}
\end{entrée}

\begin{entrée}{kha-çöe}{}{ⓔkha-çöe}
\formephonétique{kʰa-'ʒωe}
\région{GOs}
\variante{%
khaa-jöe
\région{PA}}
(\domainesémantique{Verbes de déplacement et moyens de déplacement
, Préfixes sémantiques de déplacement})
\classe{v}
\begin{glose}
\pfra{traverser à pied (une route)}
\end{glose}
\newline
\relationsémantique{Cf.}{\lien{}{khaa-cöe}}
\glosecourte{se déplacer-traverser}
\end{entrée}

\begin{entrée}{kha-da}{}{ⓔkha-da}
\région{GOs PA}
(\domainesémantique{Verbes de déplacement et moyens de déplacement
, Préfixes sémantiques de déplacement})
\classe{v}
\begin{glose}
\pfra{monter à pied}
\end{glose}
\begin{glose}
\pfra{grimper (en marchant)}
\end{glose}
\newline
\begin{sous-entrée}{ba-kha-da}{ⓔkha-daⓝba-kha-da}
\région{PA}
\begin{glose}
\pfra{échelle}
\end{glose}
\end{sous-entrée}
\end{entrée}

\begin{entrée}{khadra}{}{ⓔkhadra}
\région{GOs}
\variante{%
kadae
\région{BO}}
\classe{v}
\newline
\sens{1}
(\domainesémantique{Mouvements ou actions faits avec le corps, les bras, les mains, les pieds})
\begin{glose}
\pfra{bien appuyer les pieds pour marcher}
\end{glose}
\newline
\sens{2}
(\domainesémantique{Préfixes sémantiques d’action})
\begin{glose}
\pfra{fouler au pied}
\end{glose}
\newline
\begin{exemple}
\région{GO}
\textbf{\pnua{nu khadra bwa mu-ce}}
\pfra{il foule au pied cette fleur}
\end{exemple}
\newline
\begin{exemple}
\région{BO}
\textbf{\pnua{i kadae nye tòimwa}}
\pfra{il foule au pied cette vieille}
\end{exemple}
\end{entrée}

\begin{entrée}{khagebwa}{}{ⓔkhagebwa}
\région{GOs}
\variante{%
khagebwan
\région{PA}}
(\domainesémantique{Relations et interaction sociales})
\classe{v}
\begin{glose}
\pfra{laisser}
\end{glose}
\begin{glose}
\pfra{abandonner}
\end{glose}
\begin{glose}
\pfra{rejeter}
\end{glose}
\newline
\begin{exemple}
\région{PA}
\textbf{\pnua{nu u khagebwan kee-ny na Pum}}
\pfra{j'ai laissé mon panier à Poum}
\end{exemple}
\end{entrée}

\begin{entrée}{khagee}{}{ⓔkhagee}
\région{GOs}
\variante{%
keege
\région{BO [BM]}}
(\domainesémantique{Relations et interaction sociales})
\classe{v}
\begin{glose}
\pfra{laisser}
\end{glose}
\begin{glose}
\pfra{quitter}
\end{glose}
\newline
\begin{exemple}
\textbf{\pnua{va khagee-wa}}
\pfra{nous vous quittons}
\end{exemple}
\end{entrée}

\begin{entrée}{kha-hêgo}{}{ⓔkha-hêgo}
\région{GOs PA}
(\domainesémantique{Verbes de déplacement et moyens de déplacement
, Préfixes sémantiques de déplacement})
\classe{v}
\begin{glose}
\pfra{marcher avec une canne}
\end{glose}
\newline
\begin{exemple}
\textbf{\pnua{i kha-hêgo}}
\pfra{il marche (appuie) avec une canne}
\end{exemple}
\end{entrée}

\begin{entrée}{kha-hoze}{}{ⓔkha-hoze}
\formephonétique{kʰa-'hoze}
\région{GOs}
\variante{%
a-hoze
\région{GO(s)}}
(\domainesémantique{Verbes de déplacement et moyens de déplacement
, Préfixes sémantiques de déplacement})
\classe{v}
\begin{glose}
\pfra{suivre ; longer à pied}
\end{glose}
\newline
\begin{exemple}
\textbf{\pnua{e kha-hoze koli we-za}}
\pfra{il longe le bord de la mer}
\end{exemple}
\end{entrée}

\begin{entrée}{khai}{1}{ⓔkhaiⓗ1}
\formephonétique{kʰai}
\région{GOs PA}
(\domainesémantique{Mouvements ou actions faits avec le corps, les bras, les mains, les pieds
, Armes})
\classe{v}
\begin{glose}
\pfra{tirer à l'arc}
\end{glose}
\end{entrée}

\begin{entrée}{khai}{2}{ⓔkhaiⓗ2}
\région{GOs PA BO}
\classe{v}
\newline
\sens{1}
(\domainesémantique{Mouvements ou actions faits avec le corps, les bras, les mains, les pieds})
\begin{glose}
\pfra{tirer ;}
\end{glose}
\begin{glose}
\pfra{retirer qqch.}
\end{glose}
\newline
\begin{exemple}
\région{PA}
\textbf{\pnua{la pe-khai}}
\pfra{ils tirent chacun de leur côté (compétition)}
\end{exemple}
\newline
\begin{exemple}
\région{PA}
\textbf{\pnua{la khai-bulu-ni nye wony}}
\pfra{ils tirent tous ensemble ce bateau}
\end{exemple}
\newline
\begin{exemple}
\région{PA}
\textbf{\pnua{la kha chaamwa}}
\pfra{ils arrachent des bananiers}
\end{exemple}
\newline
\begin{sous-entrée}{khai muge}{ⓔkhaiⓗ2ⓢ1ⓝkhai muge}
\région{BO}
\begin{glose}
\pfra{casser en tirant}
\end{glose}
\end{sous-entrée}
\newline
\begin{sous-entrée}{khai pwiò}{ⓔkhaiⓗ2ⓢ1ⓝkhai pwiò}
\begin{glose}
\pfra{tirer le filet}
\end{glose}
\newline
\relationsémantique{Ant.}{\lien{ⓔtia}{tia}}
\glosecourte{pousser}
\end{sous-entrée}
\newline
\sens{2}
(\domainesémantique{Découpage du temps})
\begin{glose}
\pfra{fixer (date)}
\end{glose}
\newline
\begin{exemple}
\région{PA}
\textbf{\pnua{i khai tèèn}}
\pfra{il fixe une date}
\end{exemple}
\newline
\begin{exemple}
\région{BO}
\textbf{\pnua{i khai kai-je tèèn}}
\pfra{il fixe une date de retour}
\end{exemple}
\newline
\étymologie{
\langue{PSO}
\étymon{*thaki}
\auteur{Geraghty}}
\end{entrée}

\begin{entrée}{kha-kibwaa}{}{ⓔkha-kibwaa}
\région{GOs}
(\domainesémantique{Verbes de déplacement et moyens de déplacement
, Préfixes sémantiques de déplacement})
\classe{v}
\begin{glose}
\pfra{prendre un raccourci}
\end{glose}
\newline
\begin{exemple}
\textbf{\pnua{nu kha-kibwaa de}}
\pfra{j'ai pris un raccourci}
\end{exemple}
\newline
\begin{exemple}
\textbf{\pnua{nu phe dee-kibwaa}}
\pfra{j'ai pris un raccourci}
\end{exemple}
\end{entrée}

\begin{entrée}{kha-ku}{}{ⓔkha-ku}
\région{GOs}
\variante{%
kha-kule
\région{PA}}
(\domainesémantique{Verbes de déplacement et moyens de déplacement
, Préfixes sémantiques de déplacement})
\classe{v}
\begin{glose}
\pfra{faire tomber en marchant}
\end{glose}
\newline
\begin{exemple}
\région{GO}
\textbf{\pnua{e kha-ku kile}}
\pfra{il a laissé tomber sa clé en marchant (et l'a perdue)}
\end{exemple}
\newline
\begin{exemple}
\région{PA}
\textbf{\pnua{i khaa-kule kile}}
\pfra{il a laissé tomber sa clé en marchant (et l'a perdue)}
\end{exemple}
\end{entrée}

\begin{entrée}{kha-khoońe}{}{ⓔkha-khoońe}
\formephonétique{kʰa-'kʰoːne}
\région{GOs}
(\domainesémantique{Verbes de déplacement et moyens de déplacement
, Préfixes sémantiques de déplacement})
\classe{v}
\begin{glose}
\pfra{marcher avec une charge sur le dos}
\end{glose}
\end{entrée}

\begin{entrée}{khalu}{}{ⓔkhalu}
\région{GOs}
(\domainesémantique{Topographie})
\classe{nom}
\begin{glose}
\pfra{creux ; dépression sur un terrain}
\end{glose}
\newline
\begin{exemple}
\textbf{\pnua{çö thu-mhenõ, çö phwene ma po khalu}}
\pfra{quand tu te promènes fais attention, il y a un petit creux}
\end{exemple}
\end{entrée}

\begin{entrée}{kham}{}{ⓔkham}
\région{PA BO}
\variante{%
khã
\région{GO(s)}}
(\domainesémantique{Verbes de mouvement})
\classe{v}
\begin{glose}
\pfra{ricocher ;}
\end{glose}
\begin{glose}
\pfra{effleurer ;}
\end{glose}
\begin{glose}
\pfra{éviter ;}
\end{glose}
\begin{glose}
\pfra{rater ; manquer}
\end{glose}
\end{entrée}

\begin{entrée}{kha-maaçee}{}{ⓔkha-maaçee}
\formephonétique{kʰa-maːdʒeː}
\région{GOs}
\newline
\sens{1}
(\domainesémantique{Caractéristiques et propriétés des personnes})
\classe{v.stat.}
\begin{glose}
\pfra{lent}
\end{glose}
\newline
\begin{exemple}
\région{GO}
\textbf{\pnua{ègu xa a ka-maaçe}}
\pfra{c'est quelqu'un de lent}
\end{exemple}
\newline
\sens{2}
(\domainesémantique{Manière de faire l’action : verbes et adverbes de manière})
\classe{ADV}
\begin{glose}
\pfra{lentement}
\end{glose}
\newline
\begin{exemple}
\région{GO}
\textbf{\pnua{ne ka-maaçe-ni}}
\pfra{fais-le lentement}
\end{exemple}
\newline
\begin{exemple}
\région{GO}
\textbf{\pnua{cheni ka-maaçe-ni cee-jö lai}}
\pfra{mange ton riz lentement}
\end{exemple}
\newline
\relationsémantique{Cf.}{\lien{ⓔmaaja}{maaja}}
\end{entrée}

\begin{entrée}{kha-mudree}{}{ⓔkha-mudree}
\région{GOs}
\variante{%
kha-mude
\région{PA}}
(\domainesémantique{Mouvements ou actions faits avec le corps, les bras, les mains, les pieds
, Préfixes sémantiques d’action})
\classe{v}
\begin{glose}
\pfra{déchirer en marchant}
\end{glose}
\newline
\begin{exemple}
\région{GO}
\textbf{\pnua{e kha-mudree hôbwoli-je xo thini}}
\pfra{la barrière a déchiré sa robe}
\end{exemple}
\newline
\begin{exemple}
\région{PA}
\textbf{\pnua{i kha-mude hôbwoli-n na bwa thini}}
\pfra{elle a déchiré sa robe sur la barrière}
\end{exemple}
\end{entrée}

\begin{entrée}{kha-nhyale}{}{ⓔkha-nhyale}
\région{GOs}
(\domainesémantique{Mouvements ou actions faits avec le corps, les bras, les mains, les pieds
, Préfixes sémantiques d’action})
\classe{v}
\begin{glose}
\pfra{écraser avec le pied (en marchant)}
\end{glose}
\end{entrée}

\begin{entrée}{kha-phe}{}{ⓔkha-phe}
\formephonétique{'kʰa-pʰe}
\région{GOs PA BO}
\variante{%
kha-vwe
\formephonétique{kʰa-βe}
\région{GO(s)}}
\newline
\groupe{A}
(\domainesémantique{Mouvements ou actions faits avec le corps, les bras, les mains, les pieds
, Préfixes sémantiques d’action})
\classe{v}
\begin{glose}
\pfra{emporter}
\end{glose}
\begin{glose}
\pfra{prendre ; saisir (en partant)}
\end{glose}
\newline
\begin{exemple}
\textbf{\pnua{kha-phe kôgò hopo}}
\pfra{emporte les restes de nourriture}
\end{exemple}
\newline
\begin{exemple}
\région{GO}
\textbf{\pnua{kha-phe-mi !}}
\pfra{apporte-le !}
\end{exemple}
\newline
\begin{sous-entrée}{trêê-kha-vwe-mi}{ⓔkha-pheⓝtrêê-kha-vwe-mi}
\région{GO}
\begin{glose}
\pfra{apporter ici en courant}
\end{glose}
\end{sous-entrée}
\newline
\groupe{B}
(\domainesémantique{Prépositions})
\classe{PREP}
\begin{glose}
\pfra{avec ; ensemble}
\end{glose}
\end{entrée}

\begin{entrée}{kha pwiò}{}{ⓔkha pwiò}
\région{GOs}
(\domainesémantique{Pêche})
\classe{v}
\begin{glose}
\pfra{pêcher au filet}
\end{glose}
\newline
\relationsémantique{Cf.}{\lien{ⓔkhaiⓗ2ⓢ1ⓝkhai pwiò}{khai pwiò}}
\glosecourte{tirer le filet}
\end{entrée}

\begin{entrée}{khara-a nu}{}{ⓔkhara-a nu}
\région{PA BO}
(\domainesémantique{Cocotiers})
\classe{nom}
\begin{glose}
\pfra{fibre prise sur la nervure centrale de la palme de cocotier (sert de lien)}
\end{glose}
\end{entrée}

\begin{entrée}{kha-ru-heela}{}{ⓔkha-ru-heela}
\région{PA BO}
\variante{%
kha-thu-heela
}
(\domainesémantique{Verbes de mouvement})
\classe{v ; n}
\begin{glose}
\pfra{déraper}
\end{glose}
\begin{glose}
\pfra{glisser sur une glissoire}
\end{glose}
\begin{glose}
\pfra{glissoire aménagée par les enfants sur une pente mouillée}
\end{glose}
\newline
\relationsémantique{Cf.}{\lien{ⓔheela}{heela}}
\glosecourte{glisser}
\newline
\relationsémantique{Cf.}{\lien{}{kha-thu-heela}}
\glosecourte{glisser}
\end{entrée}

\begin{entrée}{kharu-mhween}{}{ⓔkharu-mhween}
\région{GOs PA BO}
(\domainesémantique{Actions liées aux éléments (liquide, fumée)})
\classe{v}
\begin{glose}
\pfra{flotter (emporté par l'eau) ; échouer (sur la grève)}
\end{glose}
\end{entrée}

\begin{entrée}{kha-thi}{}{ⓔkha-thi}
\formephonétique{kʰa-tʰi}
\région{GOs}
(\domainesémantique{Santé, maladie
, Préfixes sémantiques d’action})
\classe{v}
\begin{glose}
\pfra{boîter ; boiteux}
\end{glose}
\end{entrée}

\begin{entrée}{kha-thixò}{}{ⓔkha-thixò}
\formephonétique{kʰa-'tʰiɣɔ}
\région{GOs}
\variante{%
kha-thixò
\région{PA}}
\classe{v}
\newline
\sens{1}
(\domainesémantique{Verbes de déplacement et moyens de déplacement
, Préfixes sémantiques de déplacement})
\begin{glose}
\pfra{marcher sur la pointe des pieds [GOs]}
\end{glose}
\newline
\relationsémantique{Cf.}{\lien{ⓔkha-thixò}{kha-thixò}}
\glosecourte{(lit.) appuyer piquer pied}
\newline
\sens{2}
(\domainesémantique{Santé, maladie})
\begin{glose}
\pfra{boiteux ; boîter [PA]}
\end{glose}
\newline
\begin{exemple}
\textbf{\pnua{i kha-thixò}}
\pfra{il marche en boîtant}
\end{exemple}
\end{entrée}

\begin{entrée}{kha-tho}{}{ⓔkha-tho}
\formephonétique{kʰa-tʰo}
\région{GOs}
(\domainesémantique{Relations et interaction sociales
, Préfixes sémantiques de déplacement})
\classe{v}
\begin{glose}
\pfra{appeler en marchant}
\end{glose}
\newline
\begin{exemple}
\région{GO}
\textbf{\pnua{e kha-tho}}
\pfra{elle appelle en marchant}
\end{exemple}
\end{entrée}

\begin{entrée}{kha-tree-çimwi}{}{ⓔkha-tree-çimwi}
\formephonétique{kʰaː-ʈeː-ʒimwi}
\région{GOs}
(\domainesémantique{Mouvements ou actions faits avec le corps, les bras, les mains, les pieds
, Préfixes sémantiques de déplacement})
\classe{v}
\begin{glose}
\pfra{attraper (en déplacement)}
\end{glose}
\begin{glose}
\pfra{rejoindre ; rattraper qqn}
\end{glose}
\newline
\begin{exemple}
\textbf{\pnua{la thrêê kha-tree-çimwi}}
\pfra{ils courent pour les attraper}
\end{exemple}
\end{entrée}

\begin{entrée}{kha-trilòò}{}{ⓔkha-trilòò}
\formephonétique{kʰa-'ʈilɔː}
\région{GOs}
(\domainesémantique{Relations et interaction sociales})
\classe{v}
\begin{glose}
\pfra{demander la permission (intensif)}
\end{glose}
\newline
\begin{exemple}
\textbf{\pnua{çö kha-trilòò ?}}
\pfra{tu as bien/vraiment demandé la permission ?}
\end{exemple}
\end{entrée}

\begin{entrée}{kha-trivwi}{}{ⓔkha-trivwi}
\région{GO}
(\domainesémantique{Verbes de déplacement et moyens de déplacement
, Préfixes sémantiques de déplacement})
\classe{v}
\begin{glose}
\pfra{tirer en se déplaçant}
\end{glose}
\newline
\begin{exemple}
\région{GO}
\textbf{\pnua{bî threi-xa ce vwö e kha-trivwi xo je}}
\pfra{nous coupons un bout de bois pour qu'il (me) tire en se déplaçant}
\end{exemple}
\end{entrée}

\begin{entrée}{kha-tròòli}{}{ⓔkha-tròòli}
\formephonétique{kʰa-ʈɔːli}
\région{GOs}
(\domainesémantique{Relations et interaction sociales
, Préfixes sémantiques d’action})
\classe{v}
\begin{glose}
\pfra{rencontrer par hasard ; rattraper}
\end{glose}
\newline
\begin{exemple}
\région{GO}
\textbf{\pnua{bî za xa kha-tròò-li}}
\pfra{nous les avons rattrapés aussi en courant}
\end{exemple}
\end{entrée}

\begin{entrée}{kha-thrõbo}{}{ⓔkha-thrõbo}
\formephonétique{kʰa-'ʈʰɔ̃bo}
\région{GOs}
(\domainesémantique{Verbes de déplacement et moyens de déplacement
, Préfixes sémantiques de déplacement})
\classe{v}
\begin{glose}
\pfra{descendre en marchant}
\end{glose}
\end{entrée}

\begin{entrée}{khau}{}{ⓔkhau}
\région{GOs PA BO}
\classe{v}
\newline
\sens{1}
(\domainesémantique{Verbes de mouvement})
\begin{glose}
\pfra{passer par dessus}
\end{glose}
\begin{glose}
\pfra{enjamber}
\end{glose}
\newline
\begin{exemple}
\textbf{\pnua{e khau bwa thîni}}
\pfra{elle a enjambé la barrière}
\end{exemple}
\newline
\begin{exemple}
\région{GO}
\textbf{\pnua{xa pe-phu khau wãã}}
\pfra{il volait au-dessus (de moi) ainsi}
\end{exemple}
\newline
\sens{2}
(\domainesémantique{Verbes de déplacement et moyens de déplacement})
\begin{glose}
\pfra{franchir (une montagne, un col)}
\end{glose}
\newline
\sens{3}
(\domainesémantique{Relations et interaction sociales})
\begin{glose}
\pfra{transgresser (interdit)}
\end{glose}
\newline
\note{v.t. khaule}{grammaire}{}
\end{entrée}

\begin{entrée}{khaû}{}{ⓔkhaû}
\région{PA BO [BM]}
(\domainesémantique{Portage})
\classe{v}
\begin{glose}
\pfra{transporter}
\end{glose}
\newline
\begin{exemple}
\textbf{\pnua{i khaû pa}}
\pfra{il transporte des pierres}
\end{exemple}
\newline
\note{khaûne (v.t.)}{grammaire}{transporter qqch}
\end{entrée}

\begin{entrée}{khau-da}{}{ⓔkhau-da}
\formephonétique{kʰau-nda}
\région{GOs PA BO}
(\domainesémantique{Verbes de déplacement et moyens de déplacement})
\classe{v}
\begin{glose}
\pfra{passer par dessus (montagne) ; passer d'une vallée à l'autre}
\end{glose}
\begin{glose}
\pfra{enjamber}
\end{glose}
\end{entrée}

\begin{entrée}{khau-ni}{}{ⓔkhau-ni}
\formephonétique{kʰauɳi (GO)}
\classe{v.t.}
\newline
\sens{1}
(\domainesémantique{Verbes de déplacement et moyens de déplacement})
\begin{glose}
\pfra{passer par dessus (montagne, interdit)}
\end{glose}
\newline
\sens{2}
(\domainesémantique{Relations et interaction sociales})
\begin{glose}
\pfra{transgresser (règle, interdit)}
\end{glose}
\newline
\begin{exemple}
\région{GOs PA BO}
\textbf{\pnua{la khau-ni nõbu}}
\pfra{ils ont transgressé l'interdit}
\end{exemple}
\newline
\begin{exemple}
\textbf{\pnua{e khauni nya kobwe xo tagaza}}
\pfra{il a transgressé ce qu'a dit le docteur}
\end{exemple}
\end{entrée}

\begin{entrée}{khawali}{}{ⓔkhawali}
\région{GOs PA}
(\domainesémantique{Description des objets, formes, consistance, taille})
\classe{v.stat.}
\begin{glose}
\pfra{long (verticalement : arbre, personne)}
\end{glose}
\newline
\begin{exemple}
\région{GO}
\textbf{\pnua{e khawali na ni mwa}}
\pfra{il est plus haut que la maison}
\end{exemple}
\newline
\begin{exemple}
\région{GO}
\textbf{\pnua{e khawali nai nu}}
\pfra{il est plus grand que moi}
\end{exemple}
\newline
\begin{exemple}
\région{GO}
\textbf{\pnua{e po khawali thrûã nai çö}}
\pfra{il est un peu plus grand que toi}
\end{exemple}
\newline
\relationsémantique{Cf.}{\lien{ⓔphwawali}{phwawali}}
\glosecourte{long (horizontalement)}
\end{entrée}

\begin{entrée}{kha-whili}{}{ⓔkha-whili}
\formephonétique{kʰa-'wʰili}
\région{GOs BO}
(\domainesémantique{Verbes d'action (en général)
, Préfixes sémantiques d’action})
\classe{v}
\begin{glose}
\pfra{traîner (un cheval) ; emmener (personne)}
\end{glose}
\end{entrée}

\begin{entrée}{khazia}{}{ⓔkhazia}
\région{GOs}
\variante{%
karia
\région{PA BO}, 
khatia
\région{BO}}
(\domainesémantique{Localisation})
\classe{PREP.LOC}
\begin{glose}
\pfra{près (être) ; auprès ; à côté de}
\end{glose}
\newline
\begin{exemple}
\région{GO}
\textbf{\pnua{ã-mi khazia nu}}
\pfra{viens à côté de moi}
\end{exemple}
\newline
\begin{exemple}
\région{GO}
\textbf{\pnua{trabwa khazia nu !}}
\pfra{assieds-toi à côté de moi}
\end{exemple}
\newline
\begin{exemple}
\région{GO}
\textbf{\pnua{ge je khazia nu}}
\pfra{assieds-toi à côté de moi}
\end{exemple}
\newline
\begin{exemple}
\région{GO}
\textbf{\pnua{khazia-jö}}
\pfra{auprès de toi}
\end{exemple}
\newline
\begin{exemple}
\région{GO}
\textbf{\pnua{nu bala khazia nye ce}}
\pfra{j'ai trouvé cet arbre par hasard}
\end{exemple}
\end{entrée}

\begin{entrée}{khee}{1}{ⓔkheeⓗ1}
\région{GOs}
\classe{v}
\newline
\sens{1}
(\domainesémantique{Actions liées aux éléments (liquide, fumée)})
\begin{glose}
\pfra{écoper}
\end{glose}
\newline
\begin{exemple}
\textbf{\pnua{e khee we}}
\pfra{il écope}
\end{exemple}
\newline
\sens{2}
(\domainesémantique{Mouvements ou actions faits avec le corps, les bras, les mains, les pieds})
\begin{glose}
\pfra{pelleter}
\end{glose}
\end{entrée}

\begin{entrée}{khee}{2}{ⓔkheeⓗ2}
\région{BO}
\classe{v}
(\domainesémantique{Pêche})
\begin{glose}
\pfra{attraper (des crevettes avec une épuisette) [BM]}
\end{glose}
\newline
\begin{exemple}
\région{BO}
\textbf{\pnua{i khee kula}}
\pfra{il attraper des crevettes (avec une épuisette)}
\end{exemple}
\end{entrée}

\begin{entrée}{khee}{3}{ⓔkheeⓗ3}
\région{BO}
(\domainesémantique{Verbes de mouvement})
\classe{v}
\begin{glose}
\pfra{chavirer sur le côté ; gîter ; penché ; couché [Corne]}
\end{glose}
\newline
\note{non vérifié}{général}{}
\end{entrée}

\begin{entrée}{khee we}{}{ⓔkhee we}
\région{GOs}
(\domainesémantique{Actions liées aux éléments (liquide, fumée)})
\classe{v}
\begin{glose}
\pfra{prendre de l'eau (avec un petit récipient)}
\end{glose}
\begin{glose}
\pfra{écoper}
\end{glose}
\newline
\begin{sous-entrée}{ba-khee we}{ⓔkhee weⓝba-khee we}
\begin{glose}
\pfra{écope}
\end{glose}
\newline
\relationsémantique{Cf.}{\lien{ⓔtröi}{tröi}}
\glosecourte{puiser}
\end{sous-entrée}
\end{entrée}

\begin{entrée}{khêmèni}{}{ⓔkhêmèni}
\région{GOs}
\variante{%
khemèn
\région{PA BO}}
(\domainesémantique{Verbes d'action (en général)})
\classe{v}
\begin{glose}
\pfra{choisir ; trier}
\end{glose}
\end{entrée}

\begin{entrée}{khêmi}{}{ⓔkhêmi}
\région{GOs PA}
\variante{%
kêmi
\région{PA BO}}
\classe{v}
\newline
\sens{1}
(\domainesémantique{Mouvements ou actions faits avec le corps, les bras, les mains, les pieds})
\begin{glose}
\pfra{enterrer qqch ; mettre en terre}
\end{glose}
\newline
\begin{exemple}
\région{GO}
\textbf{\pnua{la khêmi jaa}}
\pfra{ils ont enterré les ordures}
\end{exemple}
\newline
\sens{2}
(\domainesémantique{Préparation des aliments; modes de préparation et de cuisson})
\begin{glose}
\pfra{mettre (à cuire) dans la cendre}
\end{glose}
\newline
\begin{exemple}
\région{GO}
\textbf{\pnua{la khêmi kîbi}}
\pfra{ils ont enterré le four}
\end{exemple}
\newline
\begin{exemple}
\région{GO}
\textbf{\pnua{la khêmi chaamwa na ni draa}}
\pfra{ils ont mis les bananes à cuire sous la cendre}
\end{exemple}
\newline
\étymologie{
\langue{POc}
\étymon{*tanum}
\glosecourte{enterrer, mettre en terre}}
\end{entrée}

\begin{entrée}{khêni}{}{ⓔkhêni}
\formephonétique{kʰêɳi}
\région{GOs}
\variante{%
kheni
\région{PA BO}}
\classe{v}
(\domainesémantique{Relations et interaction sociales})
\begin{glose}
\pfra{envoyer qqn faire qqch. ; ordonner}
\end{glose}
\newline
\begin{exemple}
\région{GO}
\textbf{\pnua{e khêni-ni hlãã yabwe i je vwo lha a vwo lha a tree cii}}
\pfra{il envoie ses sujets pour aller couper du bois}
\end{exemple}
\end{entrée}

\begin{entrée}{khi}{1}{ⓔkhiⓗ1}
\région{GOs PA BO WEM}
\newline
\sens{1}
(\domainesémantique{Modalité, verbes modaux})
\classe{QNT}
\classe{ATTEN atténuatif (ordre) (forme de politesse)}
\begin{glose}
\pfra{un peu [GOs]}
\end{glose}
\newline
\begin{exemple}
\région{GO}
\textbf{\pnua{khi ã-mi nõõli !}}
\pfra{viens voir un peu !}
\end{exemple}
\newline
\begin{exemple}
\région{GO}
\textbf{\pnua{ã-mi vwo jö khi yawe duu-nu !}}
\pfra{viens un peu me gratter le dos !}
\end{exemple}
\newline
\begin{exemple}
\région{GO}
\textbf{\pnua{jö gaa khi hovwo !}}
\pfra{mange un peu !}
\end{exemple}
\newline
\begin{exemple}
\région{PA}
\textbf{\pnua{khi na-mihèlè !}}
\pfra{passe-moi un peu le couteau !}
\end{exemple}
\newline
\sens{2}
(\domainesémantique{Modalité, verbes modaux})
\classe{ponctuel}
\begin{glose}
\pfra{un coup}
\end{glose}
\begin{glose}
\pfra{bref}
\end{glose}
\newline
\begin{exemple}
\textbf{\pnua{e khi-phu}}
\pfra{il vole un coup (brièvement)}
\end{exemple}
\end{entrée}

\begin{entrée}{khi}{2}{ⓔkhiⓗ2}
\région{GOs BO}
\variante{%
khibi
\région{PA}}
(\domainesémantique{Mouvements ou actions faits avec le corps, les bras, les mains, les pieds})
\classe{v}
\begin{glose}
\pfra{casser ; fendre (coco)}
\end{glose}
\newline
\begin{sous-entrée}{khi nu}{ⓔkhiⓗ2ⓝkhi nu}
\begin{glose}
\pfra{fendre un coco}
\end{glose}
\end{sous-entrée}
\end{entrée}

\begin{entrée}{khî}{}{ⓔkhî}
\région{GOs}
(\domainesémantique{Mollusques})
\classe{nom}
\begin{glose}
\pfra{huître de palétuvier}
\end{glose}
\nomscientifique{Crassostrea cucullata}
\end{entrée}

\begin{entrée}{khia}{1}{ⓔkhiaⓗ1}
\région{PA BO}
(\domainesémantique{Vêtements, parure})
\classe{v}
\begin{glose}
\pfra{mettre (chapeau)}
\end{glose}
\newline
\begin{exemple}
\région{PA}
\textbf{\pnua{i khia hau-n}}
\pfra{il met son chapeau}
\end{exemple}
\newline
\begin{exemple}
\région{PA}
\textbf{\pnua{i khia mwêêga-n}}
\pfra{il met son chapeau}
\end{exemple}
\end{entrée}

\begin{entrée}{khia}{2}{ⓔkhiaⓗ2}
\région{GOs BO PA}
(\domainesémantique{Types de champs})
\classe{nom}
\begin{glose}
\pfra{champ/massif d'igname du chef}
\end{glose}
\begin{glose}
\pfra{billon (avant le stade du billon cultivé : kêê)}
\end{glose}
\newline
\begin{sous-entrée}{thu khia}{ⓔkhiaⓗ2ⓝthu khia}
\begin{glose}
\pfra{planter le billon du chef}
\end{glose}
\end{sous-entrée}
\newline
\begin{sous-entrée}{kê khia}{ⓔkhiaⓗ2ⓝkê khia}
\begin{glose}
\pfra{le champ du chef}
\end{glose}
\end{sous-entrée}
\end{entrée}

\begin{entrée}{khiai}{}{ⓔkhiai}
\région{GOs PA BO}
(\domainesémantique{Feu : objets et actions liés au feu})
\classe{v}
\begin{glose}
\pfra{réchauffer (se) auprès du feu}
\end{glose}
\newline
\begin{sous-entrée}{i te-khiai}{ⓔkhiaiⓝi te-khiai}
\begin{glose}
\pfra{il est assis auprès du feu}
\end{glose}
\newline
\relationsémantique{Cf.}{\lien{}{ce-khiai}}
\glosecourte{bois pour se réchauffer auprès du feu}
\end{sous-entrée}
\newline
\morphologie{khi(ni)-yaai}
\end{entrée}

\begin{entrée}{khiba}{}{ⓔkhiba}
\région{GOs PA}
(\domainesémantique{Sentiments})
\classe{v}
\begin{glose}
\pfra{refuser ; rejeter}
\end{glose}
\end{entrée}

\begin{entrée}{khibii}{}{ⓔkhibii}
\région{GOs}
(\domainesémantique{Processus liés aux plantes})
\classe{v}
\begin{glose}
\pfra{éclore}
\end{glose}
\end{entrée}

\begin{entrée}{khi-bö}{}{ⓔkhi-bö}
\région{GOs}
(\domainesémantique{Feu : objets et actions liés au feu})
\classe{v}
\begin{glose}
\pfra{éteindre (le feu)}
\end{glose}
\newline
\begin{exemple}
\textbf{\pnua{khi-bö yai}}
\pfra{éteindre le feu}
\end{exemple}
\end{entrée}

\begin{entrée}{khîbö}{}{ⓔkhîbö}
\région{GOs}
(\domainesémantique{Relations et interaction sociales})
\classe{v}
\begin{glose}
\pfra{montrer ses fesses}
\end{glose}
\end{entrée}

\begin{entrée}{khibu}{}{ⓔkhibu}
\région{GOs BO}
(\domainesémantique{Santé, maladie})
\classe{v ; n}
\begin{glose}
\pfra{gonfler ; enfler (membre)}
\end{glose}
\begin{glose}
\pfra{ganglion}
\end{glose}
\begin{glose}
\pfra{bosse}
\end{glose}
\newline
\begin{exemple}
\région{GO}
\textbf{\pnua{khibu na ni piçanga-ã}}
\pfra{ganglion de l'aine (lit. de notre aine)}
\end{exemple}
\newline
\begin{exemple}
\région{GO}
\textbf{\pnua{khibu bwèèdrò-nu}}
\pfra{j'ai une bosse sur le front (lit. mon front est enflé)}
\end{exemple}
\newline
\begin{exemple}
\région{GO}
\textbf{\pnua{khibu na ni no-ã}}
\pfra{ganglion du cou}
\end{exemple}
\newline
\étymologie{
\langue{POc}
\étymon{*tumpuq, *tutumpu(q)}
\glosecourte{pousser}}
\newline
\note{pha-khibu-ni}{grammaire}{gonfler qqch}
\end{entrée}

\begin{entrée}{khibu bwèèdrò}{}{ⓔkhibu bwèèdrò}
\région{GOs}
(\domainesémantique{Santé, maladie})
\classe{v}
\begin{glose}
\pfra{enflé (front) ; avoir une bosse sur le front}
\end{glose}
\end{entrée}

\begin{entrée}{khibu me}{}{ⓔkhibu me}
\région{GOs}
(\domainesémantique{Santé, maladie})
\classe{v}
\begin{glose}
\pfra{gonflé (yeux)}
\end{glose}
\begin{glose}
\pfra{enflé}
\end{glose}
\newline
\begin{exemple}
\textbf{\pnua{e za khibu me ã-e}}
\pfra{il a les yeux vraiment gonflés}
\end{exemple}
\end{entrée}

\begin{entrée}{khibwaa}{}{ⓔkhibwaa}
\région{GOs PA BO}
\classe{v}
\newline
\sens{1}
(\domainesémantique{Verbes d'action (en général)})
\begin{glose}
\pfra{couper ; barrer}
\end{glose}
\newline
\begin{exemple}
\région{GO}
\textbf{\pnua{e khibwaa dee-nu}}
\pfra{il m'a barré la route, coupé le chemin}
\end{exemple}
\newline
\begin{exemple}
\région{PA}
\textbf{\pnua{e khibwaa dèn}}
\pfra{il a barré la route, coupé le chemin}
\end{exemple}
\newline
\begin{exemple}
\région{GO}
\textbf{\pnua{e kô khibwaa e xo ã ce}}
\pfra{cet arbre a barré la route}
\end{exemple}
\newline
\sens{2}
(\domainesémantique{Verbes de déplacement et moyens de déplacement})
\begin{glose}
\pfra{traverser ; passer à travers}
\end{glose}
\newline
\begin{exemple}
\région{GO}
\textbf{\pnua{e khibwaa ko}}
\pfra{il a traversé la forêt}
\end{exemple}
\newline
\begin{sous-entrée}{de-khibwaa}{ⓔkhibwaaⓢ2ⓝde-khibwaa}
\begin{glose}
\pfra{un raccourci (lit. chemin coupé)}
\end{glose}
\end{sous-entrée}
\end{entrée}

\begin{entrée}{khi-drale}{}{ⓔkhi-drale}
\région{GOs}
\variante{%
khi-dale
\région{PA}}
(\domainesémantique{Mouvements ou actions faits avec le corps, les bras, les mains, les pieds})
\classe{v}
\begin{glose}
\pfra{couper (en deux) ; fendre (coprah)}
\end{glose}
\newline
\begin{exemple}
\textbf{\pnua{e khi-drale nu}}
\pfra{il fend le coco}
\end{exemple}
\end{entrée}

\begin{entrée}{khi-kò}{}{ⓔkhi-kò}
\région{GOs}
(\domainesémantique{Feu : objets et actions liés au feu})
\classe{v}
\begin{glose}
\pfra{allumer un feu de brousse ; brûler (pour préparer un champ)}
\end{glose}
\end{entrée}

\begin{entrée}{khî-kui}{}{ⓔkhî-kui}
\région{GOs PA BO}
\newline
\groupe{A}
\classe{v}
\newline
\sens{1}
(\domainesémantique{Préparation des aliments; modes de préparation et de cuisson})
\begin{glose}
\pfra{griller, cuire au feu les prémices des ignames}
\end{glose}
\newline
\begin{sous-entrée}{ku khîni}{ⓔkhî-kuiⓢ1ⓝku khîni}
\begin{glose}
\pfra{igname grillée}
\end{glose}
\newline
\relationsémantique{Cf.}{\lien{ⓔkhîni}{khîni}}
\glosecourte{griller}
\end{sous-entrée}
\newline
\groupe{B}
\classe{nom}
\newline
\sens{2}
(\domainesémantique{Coutumes, dons coutumiers})
\begin{glose}
\pfra{fête des prémices des ignames}
\end{glose}
\end{entrée}

\begin{entrée}{khi-kha}{}{ⓔkhi-kha}
\région{GOs}
\variante{%
khi-ga
\région{GOs}, 
ki kha
\région{GO(s)}, 
ki khan
\région{PA}}
(\domainesémantique{Cultures, techniques, boutures
, Feu : objets et actions liés au feu})
\classe{v}
\begin{glose}
\pfra{brûler (les champs) ; pratiquer le brûlis}
\end{glose}
\newline
\relationsémantique{Cf.}{\lien{}{kîni, khîni}}
\glosecourte{brûler}
\end{entrée}

\begin{entrée}{khila}{}{ⓔkhila}
\région{GOs BO PA}
(\domainesémantique{Mouvements ou actions faits avec le corps, les bras, les mains, les pieds})
\classe{v}
\begin{glose}
\pfra{fouiller ; chercher ; tenter}
\end{glose}
\newline
\begin{exemple}
\textbf{\pnua{e khila-vwo kibao mèni, axe e bala tha}}
\pfra{il cherchait à tuer l'oiseau, mais il l'a raté}
\end{exemple}
\newline
\relationsémantique{Ant.}{\lien{}{tròò [GO]}}
\glosecourte{trouver}
\newline
\relationsémantique{Ant.}{\lien{}{too, tooli [PA]}}
\glosecourte{trouver}
\end{entrée}

\begin{entrée}{khilò}{}{ⓔkhilò}
\région{GOs BO}
(\domainesémantique{Actions liées aux éléments (liquide, fumée)})
\classe{v}
\begin{glose}
\pfra{couler goutte à goutte ; fuir}
\end{glose}
\end{entrée}

\begin{entrée}{khimò}{}{ⓔkhimò}
\formephonétique{kʰimɔ}
\région{GOs}
\variante{%
khimòn
\formephonétique{kʰimɔn}
\région{BO PA}}
(\domainesémantique{Caractéristiques et propriétés des personnes})
\classe{v.stat.}
\begin{glose}
\pfra{sourd}
\end{glose}
\newline
\begin{exemple}
\textbf{\pnua{e khimò}}
\pfra{il est sourd}
\end{exemple}
\end{entrée}

\begin{entrée}{khîni}{}{ⓔkhîni}
\région{GOs PA BO}
(\domainesémantique{Préparation des aliments; modes de préparation et de cuisson})
\classe{v}
\begin{glose}
\pfra{griller ; rôtir}
\end{glose}
\end{entrée}

\begin{entrée}{khinu}{1}{ⓔkhinuⓗ1}
\formephonétique{kʰiɳu}
\région{GOs PA BO}
\classe{v ; n}
\newline
\sens{1}
(\domainesémantique{Santé, maladie})
\begin{glose}
\pfra{malade ; maladie}
\end{glose}
\begin{glose}
\pfra{faire mal ; être douloureux}
\end{glose}
\newline
\begin{exemple}
\région{GO}
\textbf{\pnua{e khinu hii-nu}}
\pfra{j'ai mal au bras}
\end{exemple}
\newline
\begin{exemple}
\région{GO}
\textbf{\pnua{e khinu koo-nu}}
\pfra{j'ai mal au pied}
\end{exemple}
\newline
\begin{exemple}
\région{PA}
\textbf{\pnua{i khinu koo-ny}}
\pfra{j'ai mal au pied}
\end{exemple}
\newline
\begin{exemple}
\région{PA}
\textbf{\pnua{i khinu kio-ny}}
\pfra{j'ai mal au ventre}
\end{exemple}
\newline
\begin{exemple}
\région{BO}
\textbf{\pnua{i khinu hii-ny}}
\pfra{j'ai mal au bras}
\end{exemple}
\newline
\begin{exemple}
\région{PA}
\textbf{\pnua{khinu ai-ny}}
\pfra{je suis malheureux}
\end{exemple}
\newline
\begin{exemple}
\région{PA}
\textbf{\pnua{nu tòòli khinu}}
\pfra{je suis tombé malade}
\end{exemple}
\newline
\sens{2}
(\domainesémantique{Cours de la vie})
\begin{glose}
\pfra{mort (terme d'évitement et de respect)}
\end{glose}
\end{entrée}

\begin{entrée}{khinu}{2}{ⓔkhinuⓗ2}
\formephonétique{kʰiɳu}
\région{GOs PA}
(\domainesémantique{Température})
\classe{v}
\begin{glose}
\pfra{chaud (être) (atmosphère, dans la maison)}
\end{glose}
\newline
\relationsémantique{Cf.}{\lien{}{tòò [GOs]}}
\glosecourte{avoir chaud}
\end{entrée}

\begin{entrée}{khi-trilòò}{}{ⓔkhi-trilòò}
\formephonétique{kʰi-'ʈilɔː}
\région{GOs}
\variante{%
ki-tilò, khi-cilo
\région{PA}, 
khi-tilòò
\région{BO}}
(\domainesémantique{Fonctions intellectuelles})
\classe{v}
\begin{glose}
\pfra{demander un peu qqch.}
\end{glose}
\begin{glose}
\pfra{emprunter (lit. demander un peu, pour un moment) [PA]}
\end{glose}
\end{entrée}

\begin{entrée}{khò}{1}{ⓔkhòⓗ1}
\formephonétique{kʰɔ}
\région{PA}
(\domainesémantique{Cultures, techniques, boutures})
\classe{v}
\begin{glose}
\pfra{labourer}
\end{glose}
\newline
\étymologie{
\langue{POc}
\étymon{*quma}}
\end{entrée}

\begin{entrée}{khò}{2}{ⓔkhòⓗ2}
\formephonétique{kʰɔ}
\région{GOs}
(\domainesémantique{Anguilles})
\classe{nom}
\begin{glose}
\pfra{anguille de mer (sorte d')}
\end{glose}
\end{entrée}

\begin{entrée}{khõ}{}{ⓔkhõ}
\région{GOs}
\variante{%
khò, kò-
\région{PA}}
(\domainesémantique{Quantificateurs})
\classe{QNT ; atténuatif}
\begin{glose}
\pfra{un peu ; un instant}
\end{glose}
\newline
\begin{sous-entrée}{khõ 'na-mi !}{ⓔkhõⓝkhõ 'na-mi !}
\région{GOs PA}
\begin{glose}
\pfra{donne un peu !}
\end{glose}
\newline
\begin{exemple}
\région{GOs}
\textbf{\pnua{khõ 'phaxe !}}
\pfra{écoute un peu !}
\end{exemple}
\newline
\begin{exemple}
\région{PA}
\textbf{\pnua{kò 'phaxeen !}}
\pfra{écoute un peu !}
\end{exemple}
\newline
\begin{exemple}
\région{PA}
\textbf{\pnua{khò 'tia-mi !}}
\pfra{pousse un peu vers moi !}
\end{exemple}
\newline
\relationsémantique{Cf.}{\lien{ⓔkòòl}{kòòl}}
\glosecourte{debout (être)}
\end{sous-entrée}
\end{entrée}

\begin{entrée}{khô}{1}{ⓔkhôⓗ1}
\région{GOsPA}
\variante{%
khô
\région{BO}}
(\domainesémantique{Cordes, cordages})
\classe{nom}
\begin{glose}
\pfra{liane ; corde ; courroie ; longe}
\end{glose}
\newline
\begin{sous-entrée}{khôô-keala}{ⓔkhôⓗ1ⓝkhôô-keala}
\begin{glose}
\pfra{anse de panier; bretelles de portage}
\end{glose}
\end{sous-entrée}
\newline
\begin{sous-entrée}{khôô-wô}{ⓔkhôⓗ1ⓝkhôô-wô}
\région{GO}
\begin{glose}
\pfra{l'amarre du bateau}
\end{glose}
\end{sous-entrée}
\newline
\begin{sous-entrée}{khôô-kumwala}{ⓔkhôⓗ1ⓝkhôô-kumwala}
\begin{glose}
\pfra{bouture de patate douce}
\end{glose}
\end{sous-entrée}
\newline
\begin{sous-entrée}{khô-wany}{ⓔkhôⓗ1ⓝkhô-wany}
\région{PA}
\begin{glose}
\pfra{corde de bateau}
\end{glose}
\newline
\begin{exemple}
\région{BO}
\textbf{\pnua{khôô-n}}
\pfra{son anse de panier; bretelles de portage}
\end{exemple}
\newline
\relationsémantique{Cf.}{\lien{ⓔwa-}{wa-}}
\glosecourte{lien}
\end{sous-entrée}
\end{entrée}

\begin{entrée}{khô}{2}{ⓔkhôⓗ2}
\région{GOs}
(\domainesémantique{Description des objets, formes, consistance, taille})
\classe{v.stat.}
\begin{glose}
\pfra{calme ; paisible (temps, atmosphère, personne)}
\end{glose}
\newline
\begin{exemple}
\textbf{\pnua{e khô}}
\pfra{il est calme}
\end{exemple}
\end{entrée}

\begin{entrée}{khõbo}{}{ⓔkhõbo}
\région{BO}
(\domainesémantique{Mouvements ou actions faits avec le corps, les bras, les mains, les pieds})
\classe{v}
\begin{glose}
\pfra{taper}
\end{glose}
\newline
\begin{exemple}
\textbf{\pnua{nu khõbo-je xo ce}}
\pfra{je l'ai frappé avec un bout de bois [BM]}
\end{exemple}
\end{entrée}

\begin{entrée}{khõbwe}{}{ⓔkhõbwe}
\formephonétique{kʰɔ̃bwe}
\région{GOs}
\région{GOs BO}
\variante{%
kõbwe
\formephonétique{kɔ̃bwe}}
\newline
\groupe{A}
(\domainesémantique{Discours, échanges verbaux})
\classe{v}
\begin{glose}
\pfra{dire ; penser ; croire}
\end{glose}
\newline
\begin{exemple}
\textbf{\pnua{la xa khõbwe na mhã Jae}}
\pfra{ils croyaient que Jae était mort}
\end{exemple}
\newline
\groupe{B}
(\domainesémantique{Conjonction})
\classe{CNJ ; COMP}
\begin{glose}
\pfra{que ; quotatif ; marque d'incertitude}
\end{glose}
\newline
\begin{exemple}
\textbf{\pnua{i khõbwi cai nu khõbwe a kila Kavo}}
\pfra{il dit d'aller chercher Kavo}
\end{exemple}
\end{entrée}

\begin{entrée}{khõbwe-raa}{}{ⓔkhõbwe-raa}
\région{GOs}
(\domainesémantique{Relations et interaction sociales})
\classe{v}
\begin{glose}
\pfra{dire du mal de qqn}
\end{glose}
\newline
\begin{exemple}
\textbf{\pnua{e khõbwe-raa-ini ãgu}}
\pfra{elle a dit du mal de qqn}
\end{exemple}
\newline
\begin{exemple}
\textbf{\pnua{e khõbwe-raa-i-nu}}
\pfra{elle a dit du mal de moi}
\end{exemple}
\end{entrée}

\begin{entrée}{khô-bwiri}{}{ⓔkhô-bwiri}
\région{GOs}
(\domainesémantique{Cordes, cordages})
\classe{nom}
\begin{glose}
\pfra{rênes}
\end{glose}
\end{entrée}

\begin{entrée}{khoe}{}{ⓔkhoe}
\région{PA}
\variante{%
khoè-n
\région{BO [Corne]}}
(\domainesémantique{Parenté})
\classe{n (référence)}
\begin{glose}
\pfra{frère ou soeur aîné(e) ; grand-frère ; grande-soeur}
\end{glose}
\begin{glose}
\pfra{aîné (frère, soeur) ; deuxième frère aîné [BO]}
\end{glose}
\newline
\begin{exemple}
\région{PA}
\textbf{\pnua{khoe-n thòòmwa}}
\pfra{son deuxième frère aîné}
\end{exemple}
\newline
\étymologie{
\langue{POc}
\étymon{*tuqa}}
\end{entrée}

\begin{entrée}{khô-jitrua}{}{ⓔkhô-jitrua}
\région{GOs}
(\domainesémantique{Armes})
\classe{nom}
\begin{glose}
\pfra{corde de l'arc}
\end{glose}
\end{entrée}

\begin{entrée}{khô-kaze}{}{ⓔkhô-kaze}
\région{GOs}
(\domainesémantique{Marées})
\classe{nom}
\begin{glose}
\pfra{marée étale (calme)}
\end{glose}
\end{entrée}

\begin{entrée}{khòle}{}{ⓔkhòle}
\région{GOs BO PA}
(\domainesémantique{Verbes de mouvement
, Mouvements ou actions faits avec le corps, les bras, les mains, les pieds})
\classe{v}
\begin{glose}
\pfra{toucher (avec la main)}
\end{glose}
\begin{glose}
\pfra{serrer}
\end{glose}
\newline
\begin{exemple}
\textbf{\pnua{mi pe-khòle hî-î}}
\pfra{nous nous sommes serrés la main}
\end{exemple}
\newline
\begin{exemple}
\textbf{\pnua{khòle-mã !}}
\pfra{serrez fort !, accrochez-vous}
\end{exemple}
\end{entrée}

\begin{entrée}{khòlima}{}{ⓔkhòlima}
\région{GOs}
(\domainesémantique{Mouvements ou actions faits avec le corps, les bras, les mains, les pieds})
\classe{v}
\begin{glose}
\pfra{agripper (s')}
\end{glose}
\begin{glose}
\pfra{cramponner}
\end{glose}
\begin{glose}
\pfra{retenir (se)}
\end{glose}
\newline
\begin{exemple}
\textbf{\pnua{e ka khòlima}}
\pfra{il s'agrippe}
\end{exemple}
\end{entrée}

\begin{entrée}{khône}{}{ⓔkhône}
\région{GOs}
(\domainesémantique{Description des objets, formes, consistance, taille})
\classe{v}
\begin{glose}
\pfra{immobile}
\end{glose}
\newline
\begin{exemple}
\textbf{\pnua{khône hava-hii-je}}
\pfra{il a les ailes immobiles (il plane)}
\end{exemple}
\end{entrée}

\begin{entrée}{khõnò}{}{ⓔkhõnò}
\formephonétique{kʰɔ̃ɳɔ̃}
\région{GOs}
(\domainesémantique{Insectes})
\classe{nom}
\begin{glose}
\pfra{cocon\_d'insecte}
\end{glose}
\end{entrée}

\begin{entrée}{khòò}{}{ⓔkhòò}
\formephonétique{kʰɔː}
\région{GOs}
\variante{%
khòòl
\région{PA BO}}
(\domainesémantique{Reptiles})
\classe{nom}
\begin{glose}
\pfra{caméléon}
\end{glose}
\end{entrée}

\begin{entrée}{khooje}{}{ⓔkhooje}
\région{GOs}
(\domainesémantique{Oiseaux})
\classe{nom}
\begin{glose}
\pfra{émouchet bleu (à ventre blanc)}
\end{glose}
\nomscientifique{Accipiter haplochrous}
\end{entrée}

\begin{entrée}{khööjo}{}{ⓔkhööjo}
\formephonétique{kʰωːɲɟo}
\région{GOs}
\variante{%
koojòng, khoojòng
\région{PA BO}}
(\domainesémantique{Arbre})
\classe{nom}
\begin{glose}
\pfra{arbre à latex (dont la sève est utilisée comme poison pour la pêche stupéfiante)}
\end{glose}
\begin{glose}
\pfra{faux manguier (dont le fruit contient un noyau très toxique, utilisé comme poison pour la pêche)}
\end{glose}
\nomscientifique{Cerbera odollam}
\nomscientifique{Cerbera manghas (Apocynacées)}
\newline
\begin{sous-entrée}{khööjo pozo}{ⓔkhööjoⓝkhööjo pozo}
\begin{glose}
\pfra{arbre à latex blanc}
\end{glose}
\end{sous-entrée}
\newline
\begin{sous-entrée}{khööjo mii}{ⓔkhööjoⓝkhööjo mii}
\begin{glose}
\pfra{arbre à latex rouge}
\end{glose}
\end{sous-entrée}
\end{entrée}

\begin{entrée}{khôô-keala}{}{ⓔkhôô-keala}
\formephonétique{'kʰõː-'keala}
\région{GOs}
\variante{%
khôô-ke
\région{GO}}
(\domainesémantique{Portage})
\classe{nom}
\begin{glose}
\pfra{bretelles de portage}
\end{glose}
\end{entrée}

\begin{entrée}{khôôme}{}{ⓔkhôôme}
\région{GOs PA}
\newline
\sens{1}
(\domainesémantique{Aliments, alimentation})
\classe{v}
\begin{glose}
\pfra{rassasié (être)}
\end{glose}
\newline
\begin{exemple}
\textbf{\pnua{e khôôme-nu xo bunya}}
\pfra{le bougna m'a rassasié (rempli)}
\end{exemple}
\newline
\sens{2}
(\domainesémantique{Modalité, verbes modaux})
\classe{nom}
\begin{glose}
\pfra{ne faire que ; n'avoir de cesse que}
\end{glose}
\newline
\begin{exemple}
\textbf{\pnua{kixa khôôme nye mããni}}
\pfra{il ne fait que dormir (il n'est pas rassasié de sommeil)}
\end{exemple}
\newline
\begin{exemple}
\textbf{\pnua{kixa khôôme nye hovwo jena}}
\pfra{celui-là est toujours en train de manger}
\end{exemple}
\newline
\relationsémantique{Cf.}{\lien{ⓔkixa khôôme}{kixa khôôme}}
\glosecourte{ne jamais avoir assez, être toujours en train de}
\end{entrée}

\begin{entrée}{khoońe}{}{ⓔkhoońe}
\formephonétique{kʰoːne}
\région{GOsPA BO}
(\domainesémantique{Portage})
\classe{v}
\begin{glose}
\pfra{porter sur l'épaule ; chargé}
\end{glose}
\newline
\begin{exemple}
\région{GO}
\textbf{\pnua{e kha-khoońe cee}}
\pfra{il marche avec un fagot sur le dos}
\end{exemple}
\newline
\relationsémantique{Cf.}{\lien{}{khoon [BO]}}
\glosecourte{porter sur l'épaule}
\end{entrée}

\begin{entrée}{khòòni}{}{ⓔkhòòni}
\formephonétique{kʰɔːɳi}
\région{GOs}
\variante{%
kòòni
\région{BO PA}}
(\domainesémantique{Cultures, techniques, boutures
, Actions avec un instrument, un outil})
\classe{v}
\begin{glose}
\pfra{piocher ; bêcher (champ)}
\end{glose}
\newline
\begin{exemple}
\région{PA}
\textbf{\pnua{whara ò kòòni ya-wòlò}}
\pfra{le temps de labourer le champ du chef}
\end{exemple}
\end{entrée}

\begin{entrée}{khooni-n}{}{ⓔkhooni-n}
\région{PA}
(\domainesémantique{Portage})
\classe{nom}
\begin{glose}
\pfra{fardeau-ton}
\end{glose}
\end{entrée}

\begin{entrée}{khô-pwe}{}{ⓔkhô-pwe}
\région{GOs BO}
(\domainesémantique{Pêche})
\classe{nom}
\begin{glose}
\pfra{ligne (à hameçon)}
\end{glose}
\end{entrée}

\begin{entrée}{khòraa}{}{ⓔkhòraa}
\région{BO}
(\domainesémantique{Verbes d'action (en général)})
\classe{v}
\begin{glose}
\pfra{gaspiller}
\end{glose}
\newline
\begin{exemple}
\textbf{\pnua{i khòrale nòòla-n}}
\pfra{il gaspille ses richesses [BM]}
\end{exemple}
\newline
\note{khòrale (v.t.)}{grammaire}{gaspiller qqch}
\end{entrée}

\begin{entrée}{khoriing}{}{ⓔkhoriing}
\région{PA}
(\domainesémantique{Parties de plantes})
\classe{nom}
\begin{glose}
\pfra{rejet d'arbuste ou d'arbre taillé}
\end{glose}
\newline
\relationsémantique{Cf.}{\lien{}{kîbwò ; kîbwòn}}
\glosecourte{rejet (qui part du pied)}
\end{entrée}

\begin{entrée}{khòzole}{}{ⓔkhòzole}
\région{GOs}
\variante{%
khòzoole
}
(\domainesémantique{Verbes d'action (en général)})
\classe{v}
\begin{glose}
\pfra{gaspiller (argent, nourriture) ; ne pas mériter}
\end{glose}
\newline
\begin{exemple}
\textbf{\pnua{e khòzole hovwo}}
\pfra{il gaspille de la nourriture}
\end{exemple}
\newline
\begin{exemple}
\textbf{\pnua{e khòzole thoomwã xo êmwê ênã}}
\pfra{ce garçon ne mérite pas cette fille}
\end{exemple}
\end{entrée}

\newpage

\lettrine{l}\begin{entrée}{la}{}{ⓔla}
\région{BO}
(\domainesémantique{})
\classe{ART.PL}
\begin{glose}
\pfra{les}
\end{glose}
\newline
\begin{exemple}
\région{BO}
\textbf{\pnua{i hibine da la yala-ã}}
\pfra{il ne connaît pas nos noms [Corne]}
\end{exemple}
\newline
\begin{exemple}
\région{BO}
\textbf{\pnua{i hivine (kôbwe) da la yu caabi}}
\pfra{il ne sait pas quelles choses tu as frappées [Corne]}
\end{exemple}
\end{entrée}

\begin{entrée}{-laa}{}{ⓔ-laa}
\région{GOs PA}
(\domainesémantique{Pronoms})
\classe{PRO 3° pers. PL (obj. ou poss.)}
\begin{glose}
\pfra{les ; leur(s)}
\end{glose}
\end{entrée}

\begin{entrée}{lã-ã}{}{ⓔlã-ã}
\région{GOs PA BO}
\variante{%
lãã, lã
}
(\domainesémantique{Démonstratifs})
\classe{DEIC.PL.1 ou ANAPH}
\begin{glose}
\pfra{ces ... ci}
\end{glose}
\newline
\begin{exemple}
\textbf{\pnua{la thoomwa lãã}}
\pfra{ces femmes-ci}
\end{exemple}
\newline
\begin{exemple}
\textbf{\pnua{lãã ba-êgu}}
\pfra{ces femmes-ci}
\end{exemple}
\newline
\begin{exemple}
\textbf{\pnua{ila lãã}}
\pfra{les voici ceux-ci}
\end{exemple}
\end{entrée}

\begin{entrée}{laa-ba}{}{ⓔlaa-ba}
\région{GOs}
(\domainesémantique{Démonstratifs})
\classe{PRO.DEIC PL}
\begin{glose}
\pfra{ceux-là là-bas}
\end{glose}
\end{entrée}

\begin{entrée}{laa-du}{}{ⓔlaa-du}
\région{GOs PA}
(\domainesémantique{Démonstratifs})
\classe{PRO.DEIC.PL}
\begin{glose}
\pfra{ceux-là en bas}
\end{glose}
\end{entrée}

\begin{entrée}{-laeo}{}{ⓔ-laeo}
\région{GO}
(\domainesémantique{Démonstratifs})
\classe{DEM PL}
\begin{glose}
\pfra{-ci (proche du locuteur et loin de l'interlocuteur) Dubois}
\end{glose}
\newline
\relationsémantique{Cf.}{\lien{}{-eo}}
\glosecourte{démonstratif sg}
\end{entrée}

\begin{entrée}{lai}{}{ⓔlai}
\région{GOs}
(\domainesémantique{Aliments, alimentation})
\classe{nom}
\begin{glose}
\pfra{riz}
\end{glose}
\newline
\begin{sous-entrée}{pwo-lai}{ⓔlaiⓝpwo-lai}
\begin{glose}
\pfra{grain de riz}
\end{glose}
\end{sous-entrée}
\end{entrée}

\begin{entrée}{la-ida}{}{ⓔla-ida}
\région{GOs PA}
(\domainesémantique{Démonstratifs})
\classe{DEIC.PL}
\begin{glose}
\pfra{ceux-là en haut}
\end{glose}
\end{entrée}

\begin{entrée}{lalue}{}{ⓔlalue}
\région{GOs PA}
(\domainesémantique{Noms des plantes})
\classe{nom}
\begin{glose}
\pfra{aloes}
\end{glose}
\nomscientifique{Cordyline fruticosa (L.) Agavacées}
\end{entrée}

\begin{entrée}{lã-nã}{}{ⓔlã-nã}
\région{PA BO}
(\domainesémantique{Démonstratifs})
\classe{DEM.DEIC.2 duel et ANAPH}
\begin{glose}
\pfra{ceux-là}
\end{glose}
\newline
\begin{exemple}
\textbf{\pnua{la thoomwã mãla êba}}
\pfra{ces femmes là-bas}
\end{exemple}
\newline
\begin{exemple}
\textbf{\pnua{pwawa ne jö whili-mi lãnã pòi-m ?}}
\pfra{est-il possible que tu amènes ici tes petits enfants ?}
\end{exemple}
\newline
\begin{sous-entrée}{la-nã}{ⓔlã-nãⓝla-nã}
\begin{glose}
\pfra{ces choses (Dubois)}
\end{glose}
\end{sous-entrée}
\end{entrée}

\begin{entrée}{la-nim}{}{ⓔla-nim}
\région{PA}
(\domainesémantique{Démonstratifs})
\classe{DEM.DEIC.3 PL}
\begin{glose}
\pfra{ces ... là (péjoratif pour les humains, mis à distance)}
\end{glose}
\end{entrée}

\begin{entrée}{lao}{}{ⓔlao}
\région{PA BO}
\variante{%
lau
\région{BO}}
(\domainesémantique{Verbes d'action (en général)})
\classe{v}
\begin{glose}
\pfra{rater (cible) ; louper ; manquer}
\end{glose}
\newline
\relationsémantique{Cf.}{\lien{}{pa-tha [GOs]}}
\glosecourte{rater}
\end{entrée}

\begin{entrée}{la-òli}{}{ⓔla-òli}
\région{GOs PA}
(\domainesémantique{Démonstratifs})
\classe{DEM.DEIC.3 ou ANAPH}
\begin{glose}
\pfra{là ; là-bas (visible)}
\end{glose}
\newline
\begin{exemple}
\textbf{\pnua{la-òli}}
\pfra{ceux-là là-bas}
\end{exemple}
\newline
\begin{exemple}
\textbf{\pnua{li-òli mwa}}
\pfra{ceux-là loin là-bas}
\end{exemple}
\end{entrée}

\begin{entrée}{lapya}{}{ⓔlapya}
\région{GOs}
(\domainesémantique{Poissons})
\classe{nom}
\begin{glose}
\pfra{tilapia}
\end{glose}
\nomscientifique{Oreochromis mossambica (Cichlidae)}
\newline
\emprunt{tilapia (FR)}
\end{entrée}

\begin{entrée}{layô}{}{ⓔlayô}
\région{GOs}
\variante{%
laviã
\région{WE}}
(\domainesémantique{Aliments, alimentation})
\classe{nom}
\begin{glose}
\pfra{viande rouge}
\end{glose}
\newline
\emprunt{la viande (FR)}
\end{entrée}

\begin{entrée}{-le}{}{ⓔ-le}
\classe{LOC.ANAPH}
(\domainesémantique{Localisation})
\begin{glose}
\pfra{là}
\end{glose}
\end{entrée}

\begin{entrée}{lè}{}{ⓔlè}
\région{GOs}
(\domainesémantique{Aliments, alimentation})
\classe{nom}
\begin{glose}
\pfra{lait}
\end{glose}
\newline
\emprunt{lait (FR)}
\end{entrée}

\begin{entrée}{-li}{}{ⓔ-li}
\région{GO PA}
(\domainesémantique{Pronoms})
\classe{PRO 3° pers. duel (OBJ ou POSS)}
\begin{glose}
\pfra{les ; leur}
\end{glose}
\newline
\begin{exemple}
\textbf{\pnua{lhi ã-da mã poi-li Numia ? - Hai ! lhi za ã-da hãda Numia}}
\pfra{ils sont partis avec leur fils à Nouméa ?- Non ! ils sont partis seulement tous les deux à Nouméa}
\end{exemple}
\end{entrée}

\begin{entrée}{li-ã}{}{ⓔli-ã}
\région{GOs PA}
(\domainesémantique{Démonstratifs})
\classe{DEIC.1 duel ou ANAPH}
\begin{glose}
\pfra{ces deux ... ci}
\end{glose}
\newline
\begin{exemple}
\textbf{\pnua{li-ã ba-êgu}}
\pfra{ces deux femmes-ci}
\end{exemple}
\newline
\begin{sous-entrée}{li-ã-du}{ⓔli-ãⓝli-ã-du}
\begin{glose}
\pfra{ces deux-ci en bas}
\end{glose}
\end{sous-entrée}
\newline
\begin{sous-entrée}{li-ã-da}{ⓔli-ãⓝli-ã-da}
\begin{glose}
\pfra{ces deux-ci en haut}
\end{glose}
\end{sous-entrée}
\end{entrée}

\begin{entrée}{liè}{}{ⓔliè}
\région{GOs}
\variante{%
li-ã
\région{BO}}
(\domainesémantique{Démonstratifs})
\classe{DEM duel}
\begin{glose}
\pfra{ces deux-ci}
\end{glose}
\newline
\begin{exemple}
\région{GO}
\textbf{\pnua{liè kòò-nu}}
\pfra{mes deux pieds}
\end{exemple}
\newline
\begin{exemple}
\région{BO}
\textbf{\pnua{li-ã koo-ny}}
\pfra{mes deux pieds}
\end{exemple}
\end{entrée}

\begin{entrée}{liè-ba}{}{ⓔliè-ba}
\région{GOs PA}
(\domainesémantique{Démonstratifs})
\classe{DEIC.3 duel (latéralement)}
\begin{glose}
\pfra{ces deux là-bas sur le côté}
\end{glose}
\newline
\begin{exemple}
\textbf{\pnua{cö hine thoomwã ma-liè-ba ?}}
\pfra{tu connais ces deux filles là-bas ?}
\end{exemple}
\end{entrée}

\begin{entrée}{liè-nã}{}{ⓔliè-nã}
\région{GOs PA}
\variante{%
li-nã
\région{GO(s)}, 
li-ne
\région{PA}}
(\domainesémantique{Démonstratifs})
\classe{DEM.DEIC.2 duel et ANAPH}
\begin{glose}
\pfra{ces deux-là}
\end{glose}
\newline
\begin{exemple}
\textbf{\pnua{thoomwã mã-linã nye jö kôbwe ? - Ô ! li-nã}}
\pfra{ce sont ces deux femmes là dont tu parles ? - oui, c'est ces deux-là}
\end{exemple}
\end{entrée}

\begin{entrée}{li-nim}{}{ⓔli-nim}
\région{PA}
(\domainesémantique{Démonstratifs})
\classe{DEM.DEIC.3 duel}
\begin{glose}
\pfra{ces deux... là (péjoratif pour les humains, mis à distance)}
\end{glose}
\end{entrée}

\begin{entrée}{li-òli}{}{ⓔli-òli}
\région{GOs PA}
(\domainesémantique{Démonstratifs})
\classe{DEM.DEIC.3 ou ANAPH}
\begin{glose}
\pfra{là ; là-bas (visible)}
\end{glose}
\newline
\begin{exemple}
\textbf{\pnua{li-òli}}
\pfra{ces deux-là là-bas}
\end{exemple}
\newline
\begin{exemple}
\textbf{\pnua{li-òli mwã}}
\pfra{ces deux-là loin là-bas}
\end{exemple}
\end{entrée}

\begin{entrée}{lòlò}{}{ⓔlòlò}
\région{GOs}
(\domainesémantique{Modalité, verbes modaux})
\classe{ADV}
\begin{glose}
\pfra{au hasard ; sans but}
\end{glose}
\newline
\begin{exemple}
\textbf{\pnua{e pe-thu-mhenõ lòlò}}
\pfra{il marche sans but}
\end{exemple}
\newline
\begin{exemple}
\textbf{\pnua{e traabwa lòlò}}
\pfra{il est assis sans rien faire}
\end{exemple}
\newline
\begin{exemple}
\textbf{\pnua{jo za hine lòlò òri !}}
\pfra{tu dis des bêtises ! (lit. tu sais sans savoir)}
\end{exemple}
\newline
\begin{exemple}
\textbf{\pnua{jo za hine ãbe !}}
\pfra{tu crois tout savoir ! tu sais toujours tout !}
\end{exemple}
\newline
\relationsémantique{Cf.}{\lien{ⓔãbe}{ãbe}}
\end{entrée}

\begin{entrée}{lòn}{}{ⓔlòn}
\région{BO}
(\domainesémantique{Relations et interaction sociales})
\classe{v}
\begin{glose}
\pfra{mentir ; mensonge [Corne]}
\end{glose}
\newline
\begin{exemple}
\région{BO}
\textbf{\pnua{a-lòn}}
\pfra{menteur}
\end{exemple}
\newline
\note{non vérifié}{général}{}
\end{entrée}

\begin{entrée}{lòò}{}{ⓔlòò}
\région{GO}
\variante{%
lò
, 
mhõ
}
(\domainesémantique{Pronoms})
\classe{PRO 3° pers. triel (sujet, OBJ ou POSS)}
\begin{glose}
\pfra{eux trois}
\end{glose}
\end{entrée}

\begin{entrée}{-lòò}{}{ⓔ-lòò}
\région{GO}
\variante{%
-lò
}
(\domainesémantique{Pronoms})
\classe{PRO 3° pers. triel (OBJ ou POSS)}
\begin{glose}
\pfra{eux trois ; leurs (à eux trois)}
\end{glose}
\end{entrée}

\begin{entrée}{loto}{}{ⓔloto}
\région{GOs}
\variante{%
lòto
\région{BO}, 
wathuu
\région{GO(s)}}
(\domainesémantique{Moyens de locomotion et chemins})
\classe{nom}
\begin{glose}
\pfra{voiture ; auto}
\end{glose}
\newline
\emprunt{l'auto (FR)}
\end{entrée}

\newpage

\lettrine{lh}\begin{entrée}{lha}{}{ⓔlha}
\région{GOs PA}
\variante{%
le
\région{BO}}
(\domainesémantique{Pronoms})
\classe{PRO 3° pers. PL (sujet)}
\begin{glose}
\pfra{ils}
\end{glose}
\newline
\begin{exemple}
\textbf{\pnua{lha za yaawa dròrò}}
\pfra{ils étaient tristes hier}
\end{exemple}
\newline
\begin{exemple}
\textbf{\pnua{za ilha nye la a dròrò}}
\pfra{c'est seulement eux qui sont partis hier}
\end{exemple}
\end{entrée}

\begin{entrée}{lhaa}{}{ⓔlhaa}
\région{BO}
(\domainesémantique{Pronoms})
\classe{PRO.INDEP 3° pers. PL}
\begin{glose}
\pfra{eux}
\end{glose}
\end{entrée}

\begin{entrée}{lhi}{}{ⓔlhi}
\région{GOs PA}
(\domainesémantique{Pronoms})
\classe{PRO 3° pers. duel (sujet)}
\begin{glose}
\pfra{ils}
\end{glose}
\newline
\begin{exemple}
\région{GO}
\textbf{\pnua{lhi za ubò dròrò ? - Hai ! lhi za yu avwônô}}
\pfra{ils sont sortis hier ?- Non ! ils sont restés à la maison}
\end{exemple}
\newline
\begin{exemple}
\textbf{\pnua{li za a-da bulu Numia ?}}
\pfra{ils sont partis ensemble à Nouméa ?}
\end{exemple}
\newline
\begin{exemple}
\textbf{\pnua{za ili nye lhi a-da Numia}}
\pfra{c'est seulement eux-deux qui sont partis à Nouméa}
\end{exemple}
\end{entrée}

\begin{entrée}{lhò}{}{ⓔlhò}
\région{GOs}
\variante{%
zò
\région{WE}}
(\domainesémantique{Pronoms})
\classe{PRO (sujet)}
\begin{glose}
\pfra{eux trois ; eux (paucal)}
\end{glose}
\newline
\begin{exemple}
\région{GO}
\textbf{\pnua{me nõõli druube ma lhò kovanye-nu dròrò}}
\pfra{mes amis et moi avons vu des cerfs hier}
\end{exemple}
\newline
\begin{exemple}
\région{GO}
\textbf{\pnua{inu ma lhò kovanye-nu me nõõli druube dròrò}}
\pfra{mes amis et moi avons vu des cerfs hier}
\end{exemple}
\newline
\begin{sous-entrée}{lhò-na}{ⓔlhòⓝlhò-na}
\begin{glose}
\pfra{eux trois (dx 3)}
\end{glose}
\end{sous-entrée}
\newline
\begin{sous-entrée}{lhò-ba}{ⓔlhòⓝlhò-ba}
\begin{glose}
\pfra{eux trois là sur le côté}
\end{glose}
\end{sous-entrée}
\newline
\begin{sous-entrée}{lhò-è}{ⓔlhòⓝlhò-è}
\begin{glose}
\pfra{ces trois-ci}
\end{glose}
\newline
\note{aspiré}{général}{}
\end{sous-entrée}
\end{entrée}

\begin{entrée}{lhòlòi}{}{ⓔlhòlòi}
\région{GOs}
\variante{%
lòloi
\région{BO PA}}
\newline
\sens{1}
(\domainesémantique{Préparation des aliments; modes de préparation et de cuisson})
\classe{v}
\begin{glose}
\pfra{couper en lamelles}
\end{glose}
\newline
\sens{2}
(\domainesémantique{Préparation des aliments; modes de préparation et de cuisson})
\classe{nom}
\begin{glose}
\pfra{préparation à base d'igname coupées en fines rondelles}
\end{glose}
\newline
\begin{exemple}
\région{GO}
\textbf{\pnua{e thu lhòlòi}}
\pfra{elle coupe l'igname en fines rondelles pour faire un bounia}
\end{exemple}
\newline
\relationsémantique{Cf.}{\lien{ⓔbunya}{bunya}}
\glosecourte{bounia}
\newline
\relationsémantique{Cf.}{\lien{ⓔeloe}{eloe}}
\glosecourte{couper en lamelles}
\end{entrée}

\newpage

\lettrine{m}\begin{entrée}{-m}{}{ⓔ-m}
\région{BO PA}
(\domainesémantique{Pronoms})
\classe{SUFF.POSS 2° pers.}
\begin{glose}
\pfra{ton ; ta ; tes}
\end{glose}
\end{entrée}

\begin{entrée}{ma}{}{ⓔma}
\région{BO}
(\domainesémantique{Mouvements ou actions faits avec le corps, les bras, les mains, les pieds})
\classe{v}
\begin{glose}
\pfra{embrasser}
\end{glose}
\begin{glose}
\pfra{prendre dans ses bras [Corne]}
\end{glose}
\newline
\note{non vérifié}{général}{}
\end{entrée}

\begin{entrée}{mã}{1}{ⓔmãⓗ1}
\formephonétique{ṃæ̃}
\région{GOs PABO}
\variante{%
mhã
}
\classe{v.stat. ; n}
\newline
\sens{1}
(\domainesémantique{Cours de la vie})
\begin{glose}
\pfra{mort ; mourir}
\end{glose}
\newline
\begin{sous-entrée}{me-mhã}{ⓔmãⓗ1ⓢ1ⓝme-mhã}
\région{PA}
\begin{glose}
\pfra{la mort}
\end{glose}
\end{sous-entrée}
\newline
\sens{2}
(\domainesémantique{Santé, maladie})
\begin{glose}
\pfra{paralysé ; engourdi}
\end{glose}
\newline
\begin{exemple}
\région{GO}
\textbf{\pnua{e mã na le nhe}}
\pfra{il est apathique (lit. les mouches meurent là)}
\end{exemple}
\newline
\begin{exemple}
\région{GO}
\textbf{\pnua{e mã hii-je}}
\pfra{il est handicapé du bras}
\end{exemple}
\newline
\begin{exemple}
\région{GO}
\textbf{\pnua{e mã kòlò-je}}
\pfra{il est hémiplégique}
\end{exemple}
\newline
\begin{exemple}
\région{GO}
\textbf{\pnua{mã phagöö-je}}
\pfra{il est paralysé, impotent, invalide}
\end{exemple}
\newline
\begin{exemple}
\région{PA}
\textbf{\pnua{mhã ai-ny}}
\pfra{je suis malheureux}
\end{exemple}
\newline
\sens{3}
(\domainesémantique{Santé, maladie})
\begin{glose}
\pfra{maladie ; malade}
\end{glose}
\newline
\begin{exemple}
\région{GO}
\textbf{\pnua{nu tròòli mã}}
\pfra{je suis malade}
\end{exemple}
\newline
\begin{exemple}
\région{PA BO}
\textbf{\pnua{nu tòòli mhã}}
\pfra{je suis malade, tombé malade}
\end{exemple}
\newline
\begin{exemple}
\textbf{\pnua{kavwö nu tròòli mã mwa, nu zo mwa}}
\pfra{je ne suis plus malade, je suis guéri}
\end{exemple}
\newline
\begin{exemple}
\région{GO}
\textbf{\pnua{mã ije}}
\pfra{sa maladie}
\end{exemple}
\newline
\begin{exemple}
\région{BO}
\textbf{\pnua{mhãi-n}}
\pfra{sa maladie}
\end{exemple}
\newline
\begin{sous-entrée}{we mhã}{ⓔmãⓗ1ⓢ3ⓝwe mhã}
\begin{glose}
\pfra{médicament}
\end{glose}
\end{sous-entrée}
\newline
\étymologie{
\langue{POc}
\étymon{*mate}}
\end{entrée}

\begin{entrée}{mã}{2}{ⓔmãⓗ2}
\région{GOs PA BO}
\classe{CNJ}
\newline
\sens{1}
(\domainesémantique{Conjonction})
\begin{glose}
\pfra{et ; aussi}
\end{glose}
\begin{glose}
\pfra{avec}
\end{glose}
\newline
\begin{exemple}
\région{GO}
\textbf{\pnua{xa nowö kêê Jae, ça za li za gi jena mã õã-je}}
\pfra{et le père de Jae, il pleure toujours là avec sa femme (i.e. la mère de Jae)}
\end{exemple}
\newline
\begin{exemple}
\région{GO}
\textbf{\pnua{e yu kòlò kêê-je ma õã-je}}
\pfra{il vit chez son père et sa mère}
\end{exemple}
\newline
\begin{exemple}
\région{GO}
\textbf{\pnua{ça novwö ibî ma Paola, ça bî uça na ni choomu}}
\pfra{Paola et moi, nous revenions de l'école}
\end{exemple}
\newline
\begin{exemple}
\région{GO}
\textbf{\pnua{Xa nowö kêê Jae, ça za li za gi jena mã õã-je}}
\pfra{Et le père de Jae, il pleure toujours là avec sa femme (i.e. la mère de Jae)}
\end{exemple}
\newline
\sens{2}
(\domainesémantique{Conjonction})
\begin{glose}
\pfra{car ; parce que}
\end{glose}
\newline
\begin{exemple}
\région{PA}
\textbf{\pnua{nòòl mã bi a pwe}}
\pfra{réveillez-vous car nous allons à la pêche}
\end{exemple}
\newline
\begin{exemple}
\région{BO}
\textbf{\pnua{nòòl mã u tèèn}}
\pfra{réveillez-vous car il fait jour}
\end{exemple}
\newline
\sens{3}
(\domainesémantique{Conjonction})
\begin{glose}
\pfra{pour que}
\end{glose}
\newline
\relationsémantique{Cf.}{\lien{ⓔmaniⓗ1}{mani}}
\glosecourte{et avec}
\end{entrée}

\begin{entrée}{mã-}{}{ⓔmã-}
\région{GOs PA}
(\domainesémantique{Marque de nombre})
\classe{nombre}
\begin{glose}
\pfra{marque de non singulier}
\end{glose}
\newline
\begin{exemple}
\textbf{\pnua{ẽnò mãl-ò !}}
\pfra{vous ! (paucal)}
\end{exemple}
\newline
\begin{exemple}
\textbf{\pnua{êgu mãli-ã, mãli-èni, mãli-òli}}
\pfra{(duel) ces 2-là, ces 2-là (dx2), ces 2 là-bas}
\end{exemple}
\newline
\begin{exemple}
\textbf{\pnua{êgu mãlò-ã, mãlò-èni, mãlò-òli}}
\pfra{(paucal) ces gens-là, ces gens-là (dx2), ces gens là-bas}
\end{exemple}
\newline
\begin{exemple}
\textbf{\pnua{êgu mãla-ã, mãla-èni, mãla-òli}}
\pfra{ces gens ! (pluriel)}
\end{exemple}
\newline
\begin{sous-entrée}{mãli-}{ⓔmã-ⓝmãli-}
\begin{glose}
\pfra{duel}
\end{glose}
\end{sous-entrée}
\newline
\begin{sous-entrée}{mãlò-}{ⓔmã-ⓝmãlò-}
\begin{glose}
\pfra{triel}
\end{glose}
\end{sous-entrée}
\newline
\begin{sous-entrée}{mãla-}{ⓔmã-ⓝmãla-}
\begin{glose}
\pfra{pluriel}
\end{glose}
\end{sous-entrée}
\end{entrée}

\begin{entrée}{maü}{}{ⓔmaü}
\région{BO}
\classe{v}
\newline
\sens{1}
(\domainesémantique{Santé, maladie})
\begin{glose}
\pfra{démanger [Corne, BM]}
\end{glose}
\newline
\sens{2}
(\domainesémantique{Description des objets, formes, consistance, taille})
\begin{glose}
\pfra{piquant (piment)}
\end{glose}
\end{entrée}

\begin{entrée}{maa-}{}{ⓔmaa-}
\région{PA}
(\domainesémantique{Préfixes classificateurs possessifs de la nourriture})
\classe{nom}
\begin{glose}
\pfra{part de nourriture qui se mastique}
\end{glose}
\newline
\begin{exemple}
\région{PA}
\textbf{\pnua{maa-ny (a) caali}}
\pfra{ma part de magnania}
\end{exemple}
\end{entrée}

\begin{entrée}{mãã}{1}{ⓔmããⓗ1}
\région{GOs PA BO}
(\domainesémantique{Aliments, alimentation})
\classe{v}
\begin{glose}
\pfra{mâcher (de la nourriture pour un bébé)}
\end{glose}
\newline
\begin{exemple}
\textbf{\pnua{e mãã ce-ẽnõ}}
\pfra{elle mâche la nourriture pour l'enfant}
\end{exemple}
\newline
\note{mããni (v.t.)}{grammaire}{}
\end{entrée}

\begin{entrée}{mãã}{2}{ⓔmããⓗ2}
\région{GOs}
\variante{%
maak
\région{PA BO}}
(\domainesémantique{Arbre})
\classe{nom}
\begin{glose}
\pfra{manguier}
\end{glose}
\nomscientifique{Mangifera indica L.}
\newline
\begin{exemple}
\textbf{\pnua{po-xè pò-mã}}
\pfra{une mangue}
\end{exemple}
\end{entrée}

\begin{entrée}{mãã}{3}{ⓔmããⓗ3}
\région{GOs WEM BO}
\variante{%
mãã
\région{PA}}
\classe{nom}
(\domainesémantique{Oiseaux})
\newline
\sens{1}
\begin{glose}
\pfra{oiseau 'lunette'}
\end{glose}
\nomscientifique{Zosterops sp.}
\newline
\sens{2}
\begin{glose}
\pfra{fauvette à ventre jaune ; fauvette gobe-mouche}
\end{glose}
\nomscientifique{Gerygone Flavolateralis Flavolateralis}
\end{entrée}

\begin{entrée}{mãã}{4}{ⓔmããⓗ4}
\région{GOs BO PA}
(\domainesémantique{Mollusques})
\classe{nom}
\begin{glose}
\pfra{troca (gastéropode)}
\end{glose}
\end{entrée}

\begin{entrée}{mãã-ce-bwòn}{}{ⓔmãã-ce-bwòn}
\région{BO}
(\domainesémantique{Feu : objets et actions liés au feu})
\classe{nom}
\begin{glose}
\pfra{foyer[BM]}
\end{glose}
\end{entrée}

\begin{entrée}{maadre}{}{ⓔmaadre}
\région{GOs}
\variante{%
maadraò
\région{GO(s)}, 
maada
\région{BO}}
(\domainesémantique{Feu : objets et actions liés au feu})
\classe{nom}
\begin{glose}
\pfra{foyer ; endroit où l'on fait le feu}
\end{glose}
\end{entrée}

\begin{entrée}{maagò}{1}{ⓔmaagòⓗ1}
\région{GOs}
(\domainesémantique{Caractéristiques et propriétés des personnes})
\classe{v}
\begin{glose}
\pfra{apathique (être)}
\end{glose}
\end{entrée}

\begin{entrée}{maagò}{2}{ⓔmaagòⓗ2}
\région{WEM BO PA}
(\domainesémantique{Types de maison, architecture de la maison})
\classe{nom}
\begin{glose}
\pfra{poutre maîtresse des maisons carrées}
\end{glose}
\begin{glose}
\pfra{poutre faîtière}
\end{glose}
\end{entrée}

\begin{entrée}{maaja}{}{ⓔmaaja}
\région{GOs}
(\domainesémantique{Manière de faire l’action : verbes et adverbes de manière})
\classe{MODIF}
\begin{glose}
\pfra{ralentir ; calmer (se)}
\end{glose}
\newline
\begin{exemple}
\textbf{\pnua{e kô-maaja}}
\pfra{il se calme (cyclone, mer)}
\end{exemple}
\newline
\begin{exemple}
\textbf{\pnua{a maaja !}}
\pfra{va plus lentement ! ralentis !}
\end{exemple}
\newline
\relationsémantique{Cf.}{\lien{ⓔkhôⓗ1}{khô}}
\glosecourte{calme (mer, vent)}
\end{entrée}

\begin{entrée}{mããle}{}{ⓔmããle}
\région{GOs}
\variante{%
maalèm
\région{BO PA}}
(\domainesémantique{Noms des plantes})
\classe{nom}
\begin{glose}
\pfra{mimosa de forêt (fleur à pompon jaune)}
\end{glose}
\nomscientifique{Pithecolobium schlechteri Guillaum.}
\end{entrée}

\begin{entrée}{maalò}{}{ⓔmaalò}
\région{BO}
(\domainesémantique{Anguilles})
\classe{nom}
\begin{glose}
\pfra{anguille jaune [Corne]}
\end{glose}
\newline
\note{non vérifié}{général}{}
\end{entrée}

\begin{entrée}{maalu}{}{ⓔmaalu}
\région{BO PA}
(\domainesémantique{Fonctions naturelles humaines})
\classe{v}
\begin{glose}
\pfra{soif (avoir)}
\end{glose}
\newline
\begin{exemple}
\région{PA}
\textbf{\pnua{nu maalu}}
\pfra{j'ai soif}
\end{exemple}
\newline
\begin{exemple}
\région{BO}
\textbf{\pnua{malu nu}}
\pfra{j'ai soif (Dubois)}
\end{exemple}
\end{entrée}

\begin{entrée}{mããni}{}{ⓔmããni}
\formephonétique{mɛ̃ːɳi}
\région{GOs WEM BO PA}
\variante{%
mãni
\région{PA}}
\newline
\sens{1}
(\domainesémantique{Fonctions naturelles humaines})
\classe{v ; n}
\begin{glose}
\pfra{dormir ; couché (être) ; allongé (être)}
\end{glose}
\begin{glose}
\pfra{sommeil}
\end{glose}
\newline
\begin{exemple}
\textbf{\pnua{e paa-mãńi-bî}}
\pfra{elle nous endormait}
\end{exemple}
\newline
\begin{exemple}
\textbf{\pnua{i mããni gòòn-a}}
\pfra{il a fait la sieste}
\end{exemple}
\newline
\begin{exemple}
\région{BO}
\textbf{\pnua{i mããni gòòn-al}}
\pfra{il a fait la sieste}
\end{exemple}
\newline
\begin{sous-entrée}{mõ-mããni}{ⓔmããniⓢ1ⓝmõ-mããni}
\région{BO}
\begin{glose}
\pfra{chambre à coucher}
\end{glose}
\end{sous-entrée}
\newline
\begin{sous-entrée}{paa-mããni-ni}{ⓔmããniⓢ1ⓝpaa-mããni-ni}
\begin{glose}
\pfra{coucher qqn, endormir}
\end{glose}
\end{sous-entrée}
\newline
\sens{2}
(\domainesémantique{Préfixes et verbes de position})
\classe{v}
\begin{glose}
\pfra{traîner par terre ; éparpiller (PA)}
\end{glose}
\end{entrée}

\begin{entrée}{mããni-mhã}{}{ⓔmããni-mhã}
\région{PA}
(\domainesémantique{Fonctions naturelles humaines})
\classe{v}
\begin{glose}
\pfra{dormir trop (faire la grasse matinée)}
\end{glose}
\end{entrée}

\begin{entrée}{maara}{}{ⓔmaara}
\région{PA}
(\domainesémantique{Eau})
\classe{nom}
\begin{glose}
\pfra{marécage}
\end{glose}
\end{entrée}

\begin{entrée}{maari}{}{ⓔmaari}
\région{GOs BO}
(\domainesémantique{Sentiments})
\classe{v}
\begin{glose}
\pfra{admirer}
\end{glose}
\end{entrée}

\begin{entrée}{mãã-trele}{}{ⓔmãã-trele}
\région{GOs}
\variante{%
mãã-rele
\région{GO(s)}}
(\domainesémantique{Oiseaux})
\classe{nom}
\begin{glose}
\pfra{gobe-mouche}
\end{glose}
\nomscientifique{Myiagra caledonica (Monarchidés)}
\end{entrée}

\begin{entrée}{maayèè}{}{ⓔmaayèè}
\région{BO}
(\domainesémantique{Caractéristiques et propriétés des personnes})
\classe{v}
\begin{glose}
\pfra{paresseux [Corne]}
\end{glose}
\newline
\begin{exemple}
\textbf{\pnua{a-maayèè}}
\pfra{un paresseux}
\end{exemple}
\end{entrée}

\begin{entrée}{mabu}{}{ⓔmabu}
\région{GOs}
(\domainesémantique{Fonctions naturelles humaines})
\classe{v}
\begin{glose}
\pfra{fatigué (après une nuit courte)}
\end{glose}
\newline
\begin{exemple}
\textbf{\pnua{e mabu}}
\pfra{il est fatigué}
\end{exemple}
\end{entrée}

\begin{entrée}{mada}{}{ⓔmada}
\région{BO PA}
\variante{%
mara
\région{PA BO}}
(\domainesémantique{Aspect})
\classe{INCH, en cours}
\begin{glose}
\pfra{venir de ; commencer ; depuis}
\end{glose}
\newline
\begin{exemple}
\textbf{\pnua{nu mada hup}}
\pfra{je suis en train de manger}
\end{exemple}
\newline
\begin{exemple}
\région{PA}
\textbf{\pnua{cu mada kol}}
\pfra{lève-toi !}
\end{exemple}
\newline
\begin{exemple}
\région{BO}
\textbf{\pnua{eka i mara tavune molo}}
\pfra{quand commence la vie (Dubois)}
\end{exemple}
\end{entrée}

\begin{entrée}{mãdra}{}{ⓔmãdra}
\région{GOs}
(\domainesémantique{Vêtements, parure})
\classe{nom}
\begin{glose}
\pfra{morceau d'étoffe (fait avec la racine du banian) ;}
\end{glose}
\begin{glose}
\pfra{jupe de femme (petite) formée de 'wepooe' et de 'pobil' (Dubois ms) ;}
\end{glose}
\begin{glose}
\pfra{manou ; pagne (des hommes)}
\end{glose}
\newline
\begin{sous-entrée}{mhadra bwabu}{ⓔmãdraⓝmhadra bwabu}
\begin{glose}
\pfra{jupon}
\end{glose}
\end{sous-entrée}
\end{entrée}

\begin{entrée}{mãdra bwabu}{}{ⓔmãdra bwabu}
\région{GOs}
(\domainesémantique{Vêtements, parure})
\classe{nom}
\begin{glose}
\pfra{jupon}
\end{glose}
\end{entrée}

\begin{entrée}{mãe}{}{ⓔmãe}
\région{GOs}
\variante{%
mae
\région{BO PA}, 
mai
\région{BO}}
(\domainesémantique{Noms des plantes})
\classe{nom}
\begin{glose}
\pfra{herbe ; paille ; chaume}
\end{glose}
\begin{glose}
\pfra{vétiver [BO]}
\end{glose}
\newline
\note{(cette herbacée est utilisée pour retenir le talus ; ses racines séchées peuvent être utilisées pour parfumer)}{glose}{}
\nomscientifique{Imperata arundinacea ou Imperata cylindrica (graminées)}
\end{entrée}

\begin{entrée}{mãè-}{}{ⓔmãè-}
\région{GOs}
\variante{%
mãi-
\région{PA BO}}
(\domainesémantique{Préfixes classificateurs numériques})
\classe{CLF.NUM (lots : fête de la nouvelle igname, contexte cérémoniel)}
\begin{glose}
\pfra{lot de 3 ignames}
\end{glose}
\begin{glose}
\pfra{lot de 4 taros ou de 4 noix de coco}
\end{glose}
\newline
\begin{exemple}
\région{PA}
\textbf{\pnua{mãi-ru}}
\pfra{2 lots de 3 ignames}
\end{exemple}
\newline
\begin{exemple}
\textbf{\pnua{mãe-xe ; mãe-tru ; mãe-ni ma-xe, etc.}}
\pfra{1 paquet de 3, 2 paquets de 3 ; 6 paquets de 3 , etc.}
\end{exemple}
\end{entrée}

\begin{entrée}{mãebo}{}{ⓔmãebo}
\région{GOs PA BO}
(\domainesémantique{Noms des plantes})
\classe{nom}
\begin{glose}
\pfra{citronnelle}
\end{glose}
\nomscientifique{Cymbopogon citratus (L.) Stapf (graminées)}
\newline
\begin{sous-entrée}{dròò-mãebo}{ⓔmãeboⓝdròò-mãebo}
\begin{glose}
\pfra{feuilles de citronnelle}
\end{glose}
\end{sous-entrée}
\end{entrée}

\begin{entrée}{mãè-ni}{}{ⓔmãè-ni}
\région{GOs}
(\domainesémantique{Préfixes classificateurs numériques})
\classe{nom}
\begin{glose}
\pfra{cinq paquets de 3 ignames}
\end{glose}
\end{entrée}

\begin{entrée}{mãè-nira ?}{}{ⓔmãè-nira ?}
\région{BO}
(\domainesémantique{Interrogatifs})
\classe{INT}
\begin{glose}
\pfra{combien de paquets de 3 (ignames, etc.) ?}
\end{glose}
\end{entrée}

\begin{entrée}{mãè-xè}{}{ⓔmãè-xè}
\région{GOsBO}
(\domainesémantique{Préfixes classificateurs numériques})
\classe{CLF.NUM (pour compter des paquets d'ignames constitués de trois ignames)}
\begin{glose}
\pfra{un paquet (de trois ignames)}
\end{glose}
\newline
\begin{exemple}
\textbf{\pnua{mãè-nira ?}}
\pfra{combien de paquets de 3 ignames?}
\end{exemple}
\newline
\relationsémantique{Cf.}{\lien{}{mãè-tru, mãè-kò, etc.}}
\glosecourte{deux paquets (de trois ignames), deux paquets, etc;}
\end{entrée}

\begin{entrée}{mãgi}{}{ⓔmãgi}
\région{GOs}
\variante{%
mwãgi
\région{PA BO}}
\classe{nom}
\newline
\sens{1}
(\domainesémantique{Coutumes, dons coutumiers})
\begin{glose}
\pfra{hache ostensoir}
\end{glose}
\newline
\sens{2}
(\domainesémantique{Outils})
\begin{glose}
\pfra{hache à double tranchant}
\end{glose}
\end{entrée}

\begin{entrée}{mãgiça}{}{ⓔmãgiça}
\formephonétique{mɛ̃giʒa}
\région{GOs}
(\domainesémantique{Poissons})
\classe{nom}
\begin{glose}
\pfra{poisson sauteur de palétuviers}
\end{glose}
\nomscientifique{Périophtalme sp. (Gobiidae)}
\end{entrée}

\begin{entrée}{magira}{}{ⓔmagira}
\formephonétique{magiɽa}
\région{GOs}
(\domainesémantique{Préparation des aliments; modes de préparation et de cuisson})
\classe{v}
\begin{glose}
\pfra{cuit (à moitié) ; pas assez cuit}
\end{glose}
\newline
\begin{exemple}
\textbf{\pnua{magira dröö}}
\pfra{la nourriture de la marmite est à moitié cuite}
\end{exemple}
\end{entrée}

\begin{entrée}{magòòny}{}{ⓔmagòòny}
\région{BO}
(\domainesémantique{Types de maison, architecture de la maison})
\classe{nom}
\begin{glose}
\pfra{traverses (charpente)[Corne]}
\end{glose}
\newline
\note{non vérifié}{général}{}
\end{entrée}

\begin{entrée}{magu}{}{ⓔmagu}
\région{GOs}
\variante{%
maguny
\région{BO PA}, 
maguc
\région{BO}}
(\domainesémantique{Insectes})
\classe{nom}
\begin{glose}
\pfra{guêpe noire}
\end{glose}
\newline
\begin{exemple}
\région{BO}
\textbf{\pnua{pi maguc}}
\pfra{nid de guêpe}
\end{exemple}
\end{entrée}

\begin{entrée}{maic}{}{ⓔmaic}
\région{PA}
(\domainesémantique{Noms des plantes})
\classe{nom}
\begin{glose}
\pfra{maïs}
\end{glose}
\newline
\emprunt{maïs (FR)}
\end{entrée}

\begin{entrée}{mãiyã}{}{ⓔmãiyã}
\région{GOs}
\variante{%
mãiã
\région{GO(s)}, 
meã
\région{BO [BM]}, 
mee
\région{PA}}
(\domainesémantique{Aliments, alimentation
, Préparation des aliments; modes de préparation et de cuisson})
\classe{v.stat.}
\begin{glose}
\pfra{cuit (mal) ; pas assez cuit (riz, viande)}
\end{glose}
\newline
\begin{exemple}
\textbf{\pnua{mãiyã dröö}}
\pfra{la nourriture de la marmite n'est pas cuite}
\end{exemple}
\newline
\begin{exemple}
\textbf{\pnua{mãiyã layô}}
\pfra{la viande n'est pas cuite}
\end{exemple}
\newline
\begin{exemple}
\textbf{\pnua{mãiyã lai}}
\pfra{le riz n'est pas cuit}
\end{exemple}
\newline
\begin{exemple}
\textbf{\pnua{mãiyã gò}}
\pfra{c'est encore cru}
\end{exemple}
\newline
\relationsémantique{Cf.}{\lien{}{thraxilo ; thaxilo}}
\newline
\relationsémantique{Ant.}{\lien{}{minõ ; minong (PA)}}
\glosecourte{cuit}
\end{entrée}

\begin{entrée}{maja}{1}{ⓔmajaⓗ1}
\formephonétique{maɲɟa}
\région{GOs}
(\domainesémantique{Aliments, alimentation})
\classe{v}
\begin{glose}
\pfra{altéré ; pas frais (nourriture)}
\end{glose}
\newline
\begin{exemple}
\région{GO}
\textbf{\pnua{e maja nò}}
\pfra{le poisson n'est pas frais}
\end{exemple}
\end{entrée}

\begin{entrée}{maja}{2}{ⓔmajaⓗ2}
\formephonétique{maɲɟa}
\région{GOs}
\variante{%
maya
\région{BO}}
(\domainesémantique{Aliments, alimentation})
\classe{nom}
\begin{glose}
\pfra{miettes ; résidus}
\end{glose}
\newline
\begin{exemple}
\textbf{\pnua{maja-ce}}
\pfra{copeaux de bois}
\end{exemple}
\newline
\begin{exemple}
\textbf{\pnua{maja-hovwo}}
\pfra{miettes de nourriture}
\end{exemple}
\end{entrée}

\begin{entrée}{maja-ce}{}{ⓔmaja-ce}
\formephonétique{maɲɟacɨ}
\région{GOs}
\variante{%
ja-ce
\région{GO(s)}, 
maya-ce, meya-ce
\région{BO}}
(\domainesémantique{Bois})
\classe{nom}
\begin{glose}
\pfra{copeaux de bois}
\end{glose}
\begin{glose}
\pfra{sciure}
\end{glose}
\end{entrée}

\begin{entrée}{maja-hovwo}{}{ⓔmaja-hovwo}
\région{GOs}
(\domainesémantique{Aliments, alimentation})
\classe{nom}
\begin{glose}
\pfra{miettes de nourriture ; reliefs de nourriture}
\end{glose}
\end{entrée}

\begin{entrée}{majiwe}{}{ⓔmajiwe}
\région{BO}
(\domainesémantique{Vêtements, parure})
\classe{nom}
\begin{glose}
\pfra{collier (de jade) ; pendentif [Corne]}
\end{glose}
\newline
\note{non vérifié}{général}{}
\end{entrée}

\begin{entrée}{majo}{}{ⓔmajo}
\formephonétique{maɲɟo}
\région{GOs BO}
(\domainesémantique{Reptiles})
\classe{nom}
\begin{glose}
\pfra{gecko ; margouillat ; tarente}
\end{glose}
\newline
\relationsémantique{Cf.}{\lien{}{toro, tròtrò}}
\end{entrée}

\begin{entrée}{makoyoo}{}{ⓔmakoyoo}
\région{GOs}
\variante{%
maxoyoo
\région{GO(s)}}
(\domainesémantique{Aliments, alimentation})
\classe{v}
\begin{glose}
\pfra{avoir envie de manger qqch}
\end{glose}
\newline
\begin{exemple}
\région{GO}
\textbf{\pnua{nu maxoyoo nò}}
\pfra{j'ai envie de poisson}
\end{exemple}
\end{entrée}

\begin{entrée}{mala}{}{ⓔmala}
\région{GOs}
(\domainesémantique{Lumière et obscurité})
\classe{nom}
\begin{glose}
\pfra{lumière du jour ; lumière}
\end{glose}
\newline
\begin{sous-entrée}{mala-a}{ⓔmalaⓝmala-a}
\région{GO}
\begin{glose}
\pfra{lumière du soleil}
\end{glose}
\end{sous-entrée}
\newline
\begin{sous-entrée}{mala-yaai}{ⓔmalaⓝmala-yaai}
\région{GO PA}
\begin{glose}
\pfra{lumière du feu, lumière électrique}
\end{glose}
\end{sous-entrée}
\newline
\begin{sous-entrée}{mala-mhwããnu}{ⓔmalaⓝmala-mhwããnu}
\begin{glose}
\pfra{lumière de la lune}
\end{glose}
\end{sous-entrée}
\end{entrée}

\begin{entrée}{mãla-}{}{ⓔmãla-}
\région{GOs PA}
(\domainesémantique{Marque de nombre})
\classe{DEM PL (post-nom)}
\begin{glose}
\pfra{marque de pluriel (des déterminants)}
\end{glose}
\newline
\begin{exemple}
\région{PA}
\textbf{\pnua{li ãgu mãla-ã}}
\pfra{ces personnes-ci}
\end{exemple}
\newline
\begin{sous-entrée}{êgu mãla-ã}{ⓔmãla-ⓝêgu mãla-ã}
\begin{glose}
\pfra{ces gens-là (pluriel)}
\end{glose}
\end{sous-entrée}
\newline
\begin{sous-entrée}{êgu mãla-na}{ⓔmãla-ⓝêgu mãla-na}
\begin{glose}
\pfra{ces gens-là (dx2)}
\end{glose}
\end{sous-entrée}
\newline
\begin{sous-entrée}{êgu mãla-òli}{ⓔmãla-ⓝêgu mãla-òli}
\begin{glose}
\pfra{ces gens-là-bas}
\end{glose}
\end{sous-entrée}
\newline
\begin{sous-entrée}{êgu mãla-ã-du}{ⓔmãla-ⓝêgu mãla-ã-du}
\begin{glose}
\pfra{ces gens-là en bas}
\end{glose}
\end{sous-entrée}
\newline
\begin{sous-entrée}{êgu mãla-ã-da}{ⓔmãla-ⓝêgu mãla-ã-da}
\begin{glose}
\pfra{ces gens-là en haut}
\end{glose}
\end{sous-entrée}
\newline
\begin{sous-entrée}{êgu mãla-ò}{ⓔmãla-ⓝêgu mãla-ò}
\begin{glose}
\pfra{ces gens-là (anaphorique)}
\end{glose}
\end{sous-entrée}
\newline
\relationsémantique{Cf.}{\lien{ⓔmã-ⓝmãlò-}{mãlò-}}
\glosecourte{triel}
\newline
\relationsémantique{Cf.}{\lien{ⓔmã-ⓝmãli-}{mãli-}}
\glosecourte{duel}
\end{entrée}

\begin{entrée}{malèmwi}{}{ⓔmalèmwi}
\région{BO}
\classe{v}
\begin{glose}
\pfra{plaire; séduire [Corne]}
\end{glose}
\newline
\note{non vérifié}{général}{}
\end{entrée}

\begin{entrée}{malevwi}{}{ⓔmalevwi}
\formephonétique{'maleβi}
\région{GOs}
\variante{%
maalemi
\région{BO [BM]}}
(\domainesémantique{Mouvements ou actions avec la tête, les yeux, la bouche})
\classe{v}
\begin{glose}
\pfra{lécher}
\end{glose}
\end{entrée}

\begin{entrée}{mãli}{1}{ⓔmãliⓗ1}
\région{BO [Corne]}
(\domainesémantique{Marque de nombre})
\classe{nombre}
\begin{glose}
\pfra{marque de duel (des déterminants)}
\end{glose}
\newline
\begin{exemple}
\région{BO}
\textbf{\pnua{ènõõ-mali !}}
\pfra{vous2 autres}
\end{exemple}
\newline
\begin{exemple}
\région{BO}
\textbf{\pnua{ènõõ-ma !}}
\pfra{vous autres}
\end{exemple}
\end{entrée}

\begin{entrée}{mãli-}{2}{ⓔmãli-ⓗ2}
\région{GOs PA}
(\domainesémantique{Marque de nombre})
\classe{DEM duel (post-nom)}
\begin{glose}
\pfra{marque de duel (des déterminants)}
\end{glose}
\newline
\begin{exemple}
\région{PA}
\textbf{\pnua{li ãgu mãli-ã}}
\pfra{ces deux personnes-ci}
\end{exemple}
\newline
\begin{sous-entrée}{êgu mãli-ã}{ⓔmãli-ⓗ2ⓝêgu mãli-ã}
\begin{glose}
\pfra{ces 2 gens-là}
\end{glose}
\end{sous-entrée}
\newline
\begin{sous-entrée}{êgu mãli-èni}{ⓔmãli-ⓗ2ⓝêgu mãli-èni}
\begin{glose}
\pfra{ces 2 gens-là (dx2)}
\end{glose}
\end{sous-entrée}
\newline
\begin{sous-entrée}{êgu mãli-òli}{ⓔmãli-ⓗ2ⓝêgu mãli-òli}
\begin{glose}
\pfra{ces 2 gens-là-bas}
\end{glose}
\end{sous-entrée}
\newline
\begin{sous-entrée}{êgu mãli-ã-du}{ⓔmãli-ⓗ2ⓝêgu mãli-ã-du}
\begin{glose}
\pfra{ces 2 gens-là en bas}
\end{glose}
\end{sous-entrée}
\newline
\begin{sous-entrée}{êgu mãla-ã-da}{ⓔmãli-ⓗ2ⓝêgu mãla-ã-da}
\begin{glose}
\pfra{ces 2 gens-là en haut}
\end{glose}
\end{sous-entrée}
\newline
\begin{sous-entrée}{êgu mãli-ò}{ⓔmãli-ⓗ2ⓝêgu mãli-ò}
\begin{glose}
\pfra{ces 2 gens-là (anaphorique)}
\end{glose}
\newline
\relationsémantique{Cf.}{\lien{ⓔmã-ⓝmãlò-}{mãlò-}}
\glosecourte{triel}
\newline
\relationsémantique{Cf.}{\lien{ⓔmã-ⓝmãla-}{mãla-}}
\glosecourte{pluriel}
\end{sous-entrée}
\end{entrée}

\begin{entrée}{mãlò-}{}{ⓔmãlò-}
\région{GOs}
(\domainesémantique{Marque de nombre})
\classe{DEM triel (post-nom)}
\begin{glose}
\pfra{marque de triel (des déterminants)}
\end{glose}
\newline
\begin{sous-entrée}{êgu mãlo-ã}{ⓔmãlò-ⓝêgu mãlo-ã}
\begin{glose}
\pfra{ces 3 personnes-là}
\end{glose}
\end{sous-entrée}
\newline
\begin{sous-entrée}{êgu mãlo-na}{ⓔmãlò-ⓝêgu mãlo-na}
\begin{glose}
\pfra{ces 3 personnes-là (dx2)}
\end{glose}
\end{sous-entrée}
\newline
\begin{sous-entrée}{êgu mãlo-òli}{ⓔmãlò-ⓝêgu mãlo-òli}
\begin{glose}
\pfra{ces 3 personnes là-bas}
\end{glose}
\end{sous-entrée}
\newline
\begin{sous-entrée}{êgu mãlo-ã-du}{ⓔmãlò-ⓝêgu mãlo-ã-du}
\begin{glose}
\pfra{ces3 personnes-là en bas}
\end{glose}
\end{sous-entrée}
\newline
\begin{sous-entrée}{êgu mãlo-ã-da}{ⓔmãlò-ⓝêgu mãlo-ã-da}
\begin{glose}
\pfra{ces 3 personnes-là en haut}
\end{glose}
\end{sous-entrée}
\newline
\begin{sous-entrée}{êgu mãlo-ò}{ⓔmãlò-ⓝêgu mãlo-ò}
\begin{glose}
\pfra{ces 3 personnes-là (anaphorique)}
\end{glose}
\end{sous-entrée}
\end{entrée}

\begin{entrée}{maloom}{}{ⓔmaloom}
\région{WEM WE BO}
\variante{%
malum
\région{PA}}
(\domainesémantique{Soins du corps})
\classe{v}
\begin{glose}
\pfra{propre ; neuf}
\end{glose}
\newline
\begin{exemple}
\région{PA}
\textbf{\pnua{thu malum}}
\pfra{nettoyer; rénover}
\end{exemple}
\end{entrée}

\begin{entrée}{mangane}{}{ⓔmangane}
\région{BO}
(\domainesémantique{Taros})
\classe{nom}
\begin{glose}
\pfra{taro (clone) de terrain sec (Dubois)}
\end{glose}
\end{entrée}

\begin{entrée}{mani}{1}{ⓔmaniⓗ1}
\région{GOs BO WEM}
\variante{%
meni
\région{PA}}
\classe{CNJ}
\newline
\sens{1}
(\domainesémantique{Prépositions})
\begin{glose}
\pfra{avec (en compagnie)}
\end{glose}
\newline
\sens{2}
(\domainesémantique{Conjonction})
\begin{glose}
\pfra{et}
\end{glose}
\newline
\begin{exemple}
\région{GO}
\textbf{\pnua{bî waaçu mãni chòòva}}
\pfra{le cheval et moi avons persévéré}
\end{exemple}
\newline
\begin{exemple}
\région{GO}
\textbf{\pnua{bi/nu ã-du kaaze mãni/mã õã-nu (bi est meilleur que nu)}}
\pfra{je suis allé pêcher avec ma mère}
\end{exemple}
\newline
\begin{exemple}
\région{GO}
\textbf{\pnua{mwõ-bi mãni ãbaa-nu}}
\pfra{c'est notre maison à moi et mon frère}
\end{exemple}
\newline
\begin{exemple}
\région{PA}
\textbf{\pnua{ili mãni nya gèè i je}}
\pfra{lui et sa grand-mère}
\end{exemple}
\newline
\begin{exemple}
\région{GO}
\textbf{\pnua{e ciia mãni wa}}
\pfra{il danse et chante (en même temps)}
\end{exemple}
\newline
\begin{exemple}
\région{PA}
\textbf{\pnua{i ciia mãni wal}}
\pfra{il danse et chante (en même temps)}
\end{exemple}
\newline
\begin{exemple}
\textbf{\pnua{zaa mãni poi-je ka li a-yuu Wêwêne}}
\pfra{une poule sultane et son enfant vivaient à Wêwêne}
\end{exemple}
\newline
\begin{exemple}
\région{PA}
\textbf{\pnua{enoli duu-n mãni canaa-n}}
\pfra{il y a là l'os et le souffle}
\end{exemple}
\newline
\begin{exemple}
\région{PA}
\textbf{\pnua{pu na gi le kêê-ê mãni kibu-ê}}
\pfra{c'est pour cela que sont présents nos pères et grand-pères}
\end{exemple}
\end{entrée}

\begin{entrée}{ma nye}{}{ⓔma nye}
\région{GOs}
(\domainesémantique{Conjonction})
\classe{CNJ}
\begin{glose}
\pfra{parce que ; du fait que}
\end{glose}
\newline
\begin{exemple}
\textbf{\pnua{bi hoxè pe-ada nõgò ma nye va a thraabu}}
\pfra{nous2 allons remonter à la rivière parce que nous allons faire la}
\end{exemple}
\newline
\begin{exemple}
\région{GO}
\textbf{\pnua{tho-raa ma nye pònõ mwa nye}}
\pfra{il répond mal car il est faible}
\end{exemple}
\newline
\begin{exemple}
\région{GO}
\textbf{\pnua{ma nye za kòi-nu na ênè avwõnõ, pu-nye da a khilaa-je}}
\pfra{la raison pour laquelle je n'étais pas ici à la maison, c'était parce que je suis parti la chercher}
\end{exemple}
\end{entrée}

\begin{entrée}{manyô}{}{ⓔmanyô}
\région{GOs PA}
(\domainesémantique{Noms des plantes})
\classe{nom}
\begin{glose}
\pfra{manioc}
\end{glose}
\end{entrée}

\begin{entrée}{mararâ}{}{ⓔmararâ}
\région{BO}
(\domainesémantique{Oiseaux})
\classe{nom}
\begin{glose}
\pfra{fauvette calédonienne [Corne]}
\end{glose}
\nomscientifique{Megalurulus mariei}
\newline
\note{non vérifié}{général}{}
\end{entrée}

\begin{entrée}{mara-tèèn}{}{ⓔmara-tèèn}
\région{PA}
(\domainesémantique{Découpage du temps})
\classe{nom}
\begin{glose}
\pfra{aube}
\end{glose}
\newline
\relationsémantique{Cf.}{\lien{ⓔmala}{mala}}
\glosecourte{éclat, lueur}
\end{entrée}

\begin{entrée}{masi}{}{ⓔmasi}
\région{GOs}
(\domainesémantique{Instruments})
\classe{nom}
\begin{glose}
\pfra{machine}
\end{glose}
\newline
\emprunt{machine (FR)}
\end{entrée}

\begin{entrée}{maû}{1}{ⓔmaûⓗ1}
\formephonétique{maû}
\région{GOs PA BO}
\variante{%
maûng
\région{BO}}
(\domainesémantique{Description des objets, formes, consistance, taille})
\classe{v.stat.}
\begin{glose}
\pfra{sec (linge, feuille) ; asséché (rivière)}
\end{glose}
\newline
\begin{exemple}
\région{GO}
\textbf{\pnua{e mãû noo-nu}}
\pfra{j'ai la gorge sèche}
\end{exemple}
\newline
\begin{exemple}
\région{GO}
\textbf{\pnua{dròò-chaamwa mãû}}
\pfra{feuilles sèches de bananier}
\end{exemple}
\newline
\begin{sous-entrée}{nu mãû}{ⓔmaûⓗ1ⓝnu mãû}
\région{GO}
\begin{glose}
\pfra{coco sec}
\end{glose}
\end{sous-entrée}
\newline
\begin{sous-entrée}{ce mãûng}{ⓔmaûⓗ1ⓝce mãûng}
\région{BO}
\begin{glose}
\pfra{bois sec}
\end{glose}
\end{sous-entrée}
\end{entrée}

\begin{entrée}{maû}{2}{ⓔmaûⓗ2}
\région{GOs PA}
(\domainesémantique{Vêtements, parure
, Objets coutumiers})
\classe{nom}
\begin{glose}
\pfra{rouleau d'étoffe}
\end{glose}
\newline
\begin{exemple}
\région{GO PA}
\textbf{\pnua{ha-ru maû ci-xãbwa}}
\pfra{deux rouleaux d'étoffe}
\end{exemple}
\end{entrée}

\begin{entrée}{mãû}{}{ⓔmãû}
\région{GOs PA}
(\domainesémantique{Coutumes, dons coutumiers})
\classe{nom}
\begin{glose}
\pfra{coutume (cérémonie) ou don coutumier}
\end{glose}
\newline
\note{don coutumier qui accompagne les signes de déclin de la personne jusqu'à sa mort et qui sont offerts au clan maternel (à l'oncle maternel)}{glose}{}
\newline
\begin{sous-entrée}{mãû kura}{ⓔmãûⓝmãû kura}
\région{PA}
\begin{glose}
\pfra{don pour le sang (versé quand on s'est blessé ou que l'on a subi une opération)}
\end{glose}
\end{sous-entrée}
\newline
\begin{sous-entrée}{mãû pu-bwaa-n}{ⓔmãûⓝmãû pu-bwaa-n}
\région{PA}
\begin{glose}
\pfra{don (pour le blanchissement des) cheveux}
\end{glose}
\end{sous-entrée}
\newline
\begin{sous-entrée}{mãû parõ}{ⓔmãûⓝmãû parõ}
\région{PA}
\begin{glose}
\pfra{don (pour la chute des) dents}
\end{glose}
\end{sous-entrée}
\newline
\begin{sous-entrée}{mãû mèè-n}{ⓔmãûⓝmãû mèè-n}
\région{PA}
\begin{glose}
\pfra{don (pour la baisse de la vue) des yeux}
\end{glose}
\end{sous-entrée}
\end{entrée}

\begin{entrée}{mãû bweera}{}{ⓔmãû bweera}
\région{GOs}
\variante{%
mhãû
\région{PA}}
(\domainesémantique{Coutumes, dons coutumiers})
\classe{nom}
\begin{glose}
\pfra{coutume de deuil pour une femme}
\end{glose}
\newline
\note{don du mari au clan paternel de l'épouse) [lit. sec foyer]}{glose}{}
\newline
\relationsémantique{Cf.}{\lien{ⓔhauva}{hauva}}
\glosecourte{don du clan paternel au clan maternel de l'épouse ou de l'époux}
\end{entrée}

\begin{entrée}{mã-wãge}{}{ⓔmã-wãge}
\région{GOs}
(\domainesémantique{Santé, maladie})
\classe{nom}
\begin{glose}
\pfra{tuberculose ; tuberculeux}
\end{glose}
\newline
\begin{exemple}
\textbf{\pnua{e trooli mã-wãge}}
\pfra{il a attrapé la tuberculose}
\end{exemple}
\newline
\begin{exemple}
\textbf{\pnua{e mã wãge}}
\pfra{il a la tuberculose}
\end{exemple}
\end{entrée}

\begin{entrée}{ma-we}{}{ⓔma-we}
\région{GOs BO}
(\domainesémantique{Eau})
\classe{nom}
\begin{glose}
\pfra{puits (lit. eau morte) [GO]}
\end{glose}
\begin{glose}
\pfra{source occasionnelle (qui ne coule qu'après de grosses pluies) [BO]}
\end{glose}
\begin{glose}
\pfra{trou d'eau [BO]}
\end{glose}
\end{entrée}

\begin{entrée}{maxa}{}{ⓔmaxa}
\région{GOs}
\variante{%
maxal
\région{PA BO}}
\classe{nom}
\newline
\sens{1}
(\domainesémantique{Phénomènes atmosphériques et naturels})
\begin{glose}
\pfra{rosée ; brouillard de rivière}
\end{glose}
\newline
\begin{sous-entrée}{pò-maxa}{ⓔmaxaⓢ1ⓝpò-maxa}
\begin{glose}
\pfra{buée (fruit de la saison froide)}
\end{glose}
\end{sous-entrée}
\newline
\sens{2}
(\domainesémantique{Saisons})
\begin{glose}
\pfra{saison froide}
\end{glose}
\begin{glose}
\pfra{époque de maturité de l'igname (mars à avril)}
\end{glose}
\newline
\relationsémantique{Cf.}{\lien{ⓔpwebae}{pwebae}}
\end{entrée}

\begin{entrée}{maxewa}{}{ⓔmaxewa}
\région{GOs}
(\domainesémantique{Oiseaux})
\classe{nom}
\begin{glose}
\pfra{canard (à gorge blanche)}
\end{glose}
\end{entrée}

\begin{entrée}{mãxi}{}{ⓔmãxi}
\région{GOs}
\variante{%
mãxim
\région{WEM WE}, 
mhãkim, mhãxim
\région{BO}, 
mããxim
\région{PA}}
\classe{v}
\newline
\sens{1}
(\domainesémantique{Mouvements ou actions avec la tête, les yeux, la bouche})
\begin{glose}
\pfra{fermer les yeux}
\end{glose}
\newline
\begin{sous-entrée}{ba-mãxim}{ⓔmãxiⓢ1ⓝba-mãxim}
\begin{glose}
\pfra{geste pour fermer les yeux (coutumes de deuil)}
\end{glose}
\end{sous-entrée}
\newline
\sens{2}
(\domainesémantique{Religion, représentations religieuses})
\begin{glose}
\pfra{prier}
\end{glose}
\begin{glose}
\pfra{recueillir\_(se)}
\end{glose}
\newline
\sens{3}
(\domainesémantique{Santé, maladie})
\begin{glose}
\pfra{évanoui [BO]}
\end{glose}
\end{entrée}

\begin{entrée}{maxuã}{}{ⓔmaxuã}
\région{PA}
(\domainesémantique{Cultures, techniques, boutures})
\classe{v}
\begin{glose}
\pfra{glâner de la canne à sucre}
\end{glose}
\newline
\relationsémantique{Cf.}{\lien{}{zagaò ; zagaòl}}
\glosecourte{glâner (ignames, taros, bananes)}
\end{entrée}

\begin{entrée}{maya}{1}{ⓔmayaⓗ1}
\région{GOs}
\variante{%
mãã
\région{PA}, 
mhaa
\région{WEM}, 
maca
\région{BO (Corne)}}
(\domainesémantique{Arbre})
\classe{nom}
\begin{glose}
\pfra{faux gaïac}
\end{glose}
\nomscientifique{Acacia spirorbis Labil. ?}
\end{entrée}

\begin{entrée}{maya}{2}{ⓔmayaⓗ2}
\région{PA BO}
\variante{%
meya
\région{PA}}
(\domainesémantique{Manière de faire l’action : verbes et adverbes de manière})
\classe{ADV}
\begin{glose}
\pfra{lent ; lentement}
\end{glose}
\end{entrée}

\begin{entrée}{mãyo}{}{ⓔmãyo}
\région{GOs}
\variante{%
maoe
\région{BO [BM]}, 
maü
\région{BO}}
(\domainesémantique{Santé, maladie})
\classe{v}
\begin{glose}
\pfra{démanger ; gratter ; piquant (sur la peau)}
\end{glose}
\newline
\begin{exemple}
\textbf{\pnua{e mãyo bwa-nu}}
\pfra{ma tête me démange}
\end{exemple}
\end{entrée}

\begin{entrée}{mazao}{}{ⓔmazao}
\région{GOs}
(\domainesémantique{Coutumes, dons coutumiers})
\classe{nom}
\begin{glose}
\pfra{coutume (cérémonie) de deuil des femmes issues de la chefferie}
\end{glose}
\newline
\note{une femme de chacun des clans de la chefferie portait deux robes l'une sur l'autre, elles en enlevaient une qu'elles posaient sur la femme assise (la soeur la défunte ou celle qui a pris le nom de la défunte).}{glose}{}
\end{entrée}

\begin{entrée}{mazido}{}{ⓔmazido}
\région{GOs}
(\domainesémantique{Description des objets, formes, consistance, taille
, Lumière et obscurité})
\classe{v}
\begin{glose}
\pfra{brillant ; scintillant}
\end{glose}
\end{entrée}

\begin{entrée}{mazii}{}{ⓔmazii}
\région{GOs}
(\domainesémantique{Parties de plantes
, Tressage})
\classe{nom}
\begin{glose}
\pfra{feuilles de pandanus de creek}
\end{glose}
\newline
\note{feuilles de pandanus "phivwâi" ou de "thra": utilisées pour tresser des nattes}{glose}{}
\newline
\relationsémantique{Cf.}{\lien{}{thra ; thral [PA]}}
\glosecourte{pandanus}
\end{entrée}

\begin{entrée}{mazilo}{}{ⓔmazilo}
\région{GOs}
(\domainesémantique{Poissons})
\classe{nom}
\begin{glose}
\pfra{poisson "balabio"}
\end{glose}
\nomscientifique{Gerres acinaces (Gerridés)}
\end{entrée}

\begin{entrée}{me}{1}{ⓔmeⓗ1}
\région{GOs BO}
\variante{%
pii-me
\région{GO(s) PA}, 
pimèè-n ; mèè-n
\région{BO}}
(\domainesémantique{Corps humain})
\classe{nom}
\begin{glose}
\pfra{oeil}
\end{glose}
\newline
\begin{exemple}
\région{GO}
\textbf{\pnua{mee-je}}
\pfra{son oeil}
\end{exemple}
\newline
\begin{exemple}
\région{PA}
\textbf{\pnua{mè-n mali}}
\pfra{ses deux yeux}
\end{exemple}
\newline
\begin{exemple}
\région{GO}
\textbf{\pnua{e kò bwa mee-nu}}
\pfra{il me bloque la vue (il est debout devant mes yeux)}
\end{exemple}
\newline
\begin{exemple}
\région{GO}
\textbf{\pnua{e kò bwa mee-ã}}
\pfra{il se fait remarquer (il est debout devant nos yeux)}
\end{exemple}
\newline
\begin{sous-entrée}{pu-me-n}{ⓔmeⓗ1ⓝpu-me-n}
\région{PA}
\begin{glose}
\pfra{sourcils (ses)}
\end{glose}
\end{sous-entrée}
\newline
\begin{sous-entrée}{pii-me}{ⓔmeⓗ1ⓝpii-me}
\begin{glose}
\pfra{lunettes}
\end{glose}
\end{sous-entrée}
\newline
\begin{sous-entrée}{mwo-pii-me}{ⓔmeⓗ1ⓝmwo-pii-me}
\begin{glose}
\pfra{étui à lunettes (lit. maison, contenant de l'oeil)}
\end{glose}
\end{sous-entrée}
\newline
\begin{sous-entrée}{hê-me}{ⓔmeⓗ1ⓝhê-me}
\begin{glose}
\pfra{globe oculaire (lit. contenu de l'oeil)}
\end{glose}
\end{sous-entrée}
\newline
\begin{sous-entrée}{thala me}{ⓔmeⓗ1ⓝthala me}
\begin{glose}
\pfra{ouvrir les yeux}
\end{glose}
\end{sous-entrée}
\newline
\étymologie{
\langue{POc}
\étymon{*mata}}
\end{entrée}

\begin{entrée}{me}{2}{ⓔmeⓗ2}
\région{GO}
(\domainesémantique{Pronoms})
\classe{PRO 1° pers. triel excl. (sujet, OBJ ou POSS)}
\begin{glose}
\pfra{nous trois ; à nous trois}
\end{glose}
\end{entrée}

\begin{entrée}{me}{3}{ⓔmeⓗ3}
\région{GOs BO}
\variante{%
mèè-n
\région{BO}}
(\domainesémantique{Corps humain})
\classe{nom}
\begin{glose}
\pfra{figure; visage}
\end{glose}
\begin{glose}
\pfra{apparence}
\end{glose}
\newline
\begin{exemple}
\région{GO}
\textbf{\pnua{nu ko bwa me-we}}
\pfra{je suis debout devant vous}
\end{exemple}
\newline
\begin{sous-entrée}{me-mwa}{ⓔmeⓗ3ⓝme-mwa}
\begin{glose}
\pfra{le devant de la maison ; le sud}
\end{glose}
\newline
\relationsémantique{Ant.}{\lien{ⓔkaçaⓝkaça mwa}{kaça mwa}}
\glosecourte{l'arrière de la maison ; le nord}
\newline
\relationsémantique{Ant.}{\lien{}{kaya mwa}}
\glosecourte{l'arrière de la maison ; le nord}
\end{sous-entrée}
\end{entrée}

\begin{entrée}{me-}{1}{ⓔme-ⓗ1}
\région{GOs PA BO}
\variante{%
mèè-n
\région{BO}}
(\domainesémantique{Description des objets, formes, consistance, taille})
\classe{nom}
\begin{glose}
\pfra{pointe ; avant}
\end{glose}
\newline
\begin{exemple}
\textbf{\pnua{bwa me-õn}}
\pfra{sur la pointe de sable}
\end{exemple}
\newline
\begin{sous-entrée}{me-wô}{ⓔme-ⓗ1ⓝme-wô}
\région{GO}
\begin{glose}
\pfra{la proue du bateau}
\end{glose}
\end{sous-entrée}
\newline
\begin{sous-entrée}{me-hele}{ⓔme-ⓗ1ⓝme-hele}
\begin{glose}
\pfra{la pointe du couteau}
\end{glose}
\end{sous-entrée}
\newline
\begin{sous-entrée}{me-do}{ⓔme-ⓗ1ⓝme-do}
\région{GO}
\begin{glose}
\pfra{la pointe de la sagaie}
\end{glose}
\end{sous-entrée}
\newline
\begin{sous-entrée}{me-lòto}{ⓔme-ⓗ1ⓝme-lòto}
\région{BO}
\begin{glose}
\pfra{l'avant de la voiture}
\end{glose}
\end{sous-entrée}
\newline
\begin{sous-entrée}{me-ti}{ⓔme-ⓗ1ⓝme-ti}
\begin{glose}
\pfra{mamelon (Dubois)}
\end{glose}
\end{sous-entrée}
\newline
\étymologie{
\langue{POc}
\étymon{*mata}}
\end{entrée}

\begin{entrée}{me-}{2}{ⓔme-ⓗ2}
\région{GOs BO PA}
(\domainesémantique{Dérivation})
\classe{PREF.NMLZ}
\begin{glose}
\pfra{action de ; façon de ; fait de}
\end{glose}
\newline
\begin{exemple}
\région{GO}
\textbf{\pnua{nooli me-phû-wa dònò !}}
\pfra{regarde le bleu du ciel !}
\end{exemple}
\newline
\begin{exemple}
\région{GO}
\textbf{\pnua{nooli me-mii-wa hõbwòli-je !}}
\pfra{regarde le rouge de sa robe !}
\end{exemple}
\newline
\begin{exemple}
\région{GO}
\textbf{\pnua{... pune me-piça xo/wo/o bwaa-je}}
\pfra{... à cause de son entêtement (de la dureté de sa tête)}
\end{exemple}
\newline
\begin{exemple}
\région{GO}
\textbf{\pnua{... pune nye piça bwaa-je}}
\pfra{... à cause du fait qu'il a la tête dure}
\end{exemple}
\newline
\begin{exemple}
\région{GO}
\textbf{\pnua{kebwa ai-nu nye me-vhaa i je}}
\pfra{je n'aime pas sa façon de parler}
\end{exemple}
\newline
\begin{exemple}
\région{GO}
\textbf{\pnua{kebwa ai-nu nye me-nee-vwo i je}}
\pfra{je n'aime pas ses manières, ses façons de faire}
\end{exemple}
\newline
\begin{exemple}
\région{GO}
\textbf{\pnua{kavwo nu trõne kaamweni me-vhaa i la}}
\pfra{je ne comprends pas leur façon de parler}
\end{exemple}
\newline
\begin{exemple}
\région{GO}
\textbf{\pnua{kavwo nu trõne kaamweni nye me-pe-thô i li}}
\pfra{je ne comprends pas comment ils se sont fâchés}
\end{exemple}
\newline
\begin{exemple}
\région{GO}
\textbf{\pnua{kavwo nu trõne kaamweni nye pe-thô i li}}
\pfra{je ne comprends pas leur dispute}
\end{exemple}
\newline
\begin{exemple}
\région{BO}
\textbf{\pnua{waya me-temwi pwaji ?}}
\pfra{comment attrape-t-on les crabes ? [BM]}
\end{exemple}
\newline
\begin{exemple}
\région{BO}
\textbf{\pnua{yo nooli nye me-yu i yã}}
\pfra{vous voyez notre façon de vivre [BM]}
\end{exemple}
\newline
\begin{sous-entrée}{me-yuu i je}{ⓔme-ⓗ2ⓝme-yuu i je}
\begin{glose}
\pfra{sa façon de vivre}
\end{glose}
\end{sous-entrée}
\newline
\begin{sous-entrée}{me-kinu-a no mwa}{ⓔme-ⓗ2ⓝme-kinu-a no mwa}
\begin{glose}
\pfra{la chaleur de la maison}
\end{glose}
\end{sous-entrée}
\newline
\begin{sous-entrée}{me-khinu-je}{ⓔme-ⓗ2ⓝme-khinu-je}
\begin{glose}
\pfra{sa souffrance (malade)}
\end{glose}
\end{sous-entrée}
\newline
\begin{sous-entrée}{me-bubu i je}{ⓔme-ⓗ2ⓝme-bubu i je}
\begin{glose}
\pfra{sa verdure}
\end{glose}
\end{sous-entrée}
\newline
\begin{sous-entrée}{me-mani}{ⓔme-ⓗ2ⓝme-mani}
\begin{glose}
\pfra{sa façon de dormir}
\end{glose}
\end{sous-entrée}
\end{entrée}

\begin{entrée}{mebo}{}{ⓔmebo}
\région{GOs}
(\domainesémantique{Insectes})
\classe{nom}
\begin{glose}
\pfra{guêpe ; abeille}
\end{glose}
\newline
\begin{sous-entrée}{mebo hu-ã}{ⓔmeboⓝmebo hu-ã}
\région{GO}
\begin{glose}
\pfra{abeille}
\end{glose}
\end{sous-entrée}
\newline
\begin{sous-entrée}{mebo sapone}{ⓔmeboⓝmebo sapone}
\région{GO}
\begin{glose}
\pfra{guêpe jaune}
\end{glose}
\end{sous-entrée}
\end{entrée}

\begin{entrée}{mebo hu-ã}{}{ⓔmebo hu-ã}
\région{GOs}
(\domainesémantique{Insectes})
\classe{nom}
\begin{glose}
\pfra{abeille (hu-ã: notre viande)}
\end{glose}
\end{entrée}

\begin{entrée}{me-cöni}{}{ⓔme-cöni}
\région{BO}
(\domainesémantique{Manière de faire l’action : verbes et adverbes de manière})
\classe{nom}
\begin{glose}
\pfra{peine ; difficulté [Corne]}
\end{glose}
\newline
\begin{exemple}
\région{BO}
\textbf{\pnua{i a-da mani me-cöni}}
\pfra{il monte avec peine}
\end{exemple}
\end{entrée}

\begin{entrée}{medatri}{}{ⓔmedatri}
\formephonétique{me'daɽi}
\région{GOs}
(\domainesémantique{Arbre})
\classe{nom}
\begin{glose}
\pfra{mandarinier}
\end{glose}
\end{entrée}

\begin{entrée}{me-de}{}{ⓔme-de}
\région{GO BO}
\variante{%
mide
\région{BO}}
(\domainesémantique{Types de maison, architecture de la maison})
\classe{nom}
\begin{glose}
\pfra{gaulettes verticales (pointe des)}
\end{glose}
\newline
\note{pointe des gaulettes qui forment la corbeille du poteau central de la case (Dubois)}{glose}{}
\newline
\relationsémantique{Cf.}{\lien{ⓔmoko}{moko}}
\glosecourte{gaulettes horizontales enroulées autour des "mede" (Dubois)}
\end{entrée}

\begin{entrée}{mee}{}{ⓔmee}
\région{GOs}
\variante{%
mèèn
\région{PA}}
(\domainesémantique{Goût des aliments})
\classe{v}
\begin{glose}
\pfra{salé (cuisine)}
\end{glose}
\newline
\begin{exemple}
\région{GO}
\textbf{\pnua{u mee}}
\pfra{c'est assez salé (quand on goûte dans la marmite)}
\end{exemple}
\newline
\begin{exemple}
\région{PA}
\textbf{\pnua{e mèèn}}
\pfra{c'est salé (quand on goûte dans la marmite)}
\end{exemple}
\newline
\relationsémantique{Cf.}{\lien{}{za ! [GO, PA]}}
\glosecourte{c'est trop salé au goût}
\newline
\relationsémantique{Cf.}{\lien{}{zani [GO]}}
\glosecourte{sel}
\newline
\relationsémantique{Cf.}{\lien{}{òn [PA BO]}}
\glosecourte{sel}
\newline
\relationsémantique{Cf.}{\lien{}{õne [BO]}}
\glosecourte{saler}
\end{entrée}

\begin{entrée}{mèè}{}{ⓔmèè}
\région{BO}
(\domainesémantique{Aspect})
\classe{v}
\begin{glose}
\pfra{commencer [Corne]}
\end{glose}
\newline
\begin{exemple}
\région{BO}
\textbf{\pnua{nu mèè (na) hovo}}
\pfra{je commence à manger}
\end{exemple}
\newline
\note{non vérifié}{général}{}
\end{entrée}

\begin{entrée}{mèèdi}{}{ⓔmèèdi}
\région{BO PA}
\variante{%
meedi
\région{BO}}
(\domainesémantique{Sentiments})
\classe{v}
\begin{glose}
\pfra{compatir ; pitié (avoir) ; prendre en pitié}
\end{glose}
\newline
\begin{exemple}
\région{BO}
\textbf{\pnua{nu mèèdi-je}}
\pfra{j'ai pitié de lui}
\end{exemple}
\end{entrée}

\begin{entrée}{meedree}{}{ⓔmeedree}
\région{GOs}
\variante{%
meedèèn
\région{PA BO}}
(\domainesémantique{Phénomènes atmosphériques et naturels})
\classe{nom}
\begin{glose}
\pfra{temps atmosphérique}
\end{glose}
\newline
\begin{exemple}
\textbf{\pnua{cabo meedree}}
\pfra{il fait lourd}
\end{exemple}
\end{entrée}

\begin{entrée}{meeji-thre}{}{ⓔmeeji-thre}
\formephonétique{meːɲɟi ʈe}
\région{GOs}
(\domainesémantique{Poissons})
\classe{nom}
\begin{glose}
\pfra{bossu d'herbe}
\end{glose}
\nomscientifique{Lethrinus harak (Lethrinidés)}
\end{entrée}

\begin{entrée}{meeli}{}{ⓔmeeli}
\région{GOs}
\variante{%
melin
\région{PA}, 
mèèlin
\région{BO}}
(\domainesémantique{Insectes})
\classe{nom}
\begin{glose}
\pfra{chenille (nom générique)}
\end{glose}
\newline
\relationsémantique{Cf.}{\lien{ⓔmèra-ⓝmèra-mòmò}{mèra-mòmò}}
\glosecourte{chenille du peuplier}
\end{entrée}

\begin{entrée}{mèèni}{}{ⓔmèèni}
\région{GOs}
(\domainesémantique{Santé, maladie})
\classe{nom}
\begin{glose}
\pfra{muguet (boutons sur la langues des bébés)}
\end{glose}
\end{entrée}

\begin{entrée}{mee-phwa}{}{ⓔmee-phwa}
\région{GOs}
\variante{%
dò-phwa-n
\région{PA}}
(\domainesémantique{Corps animal})
\classe{nom}
\begin{glose}
\pfra{bec (plat de canard)}
\end{glose}
\newline
\begin{sous-entrée}{mee-wha meni}{ⓔmee-phwaⓝmee-wha meni}
\begin{glose}
\pfra{bec d'oiseau}
\end{glose}
\end{sous-entrée}
\end{entrée}

\begin{entrée}{mee-vwha mãni}{}{ⓔmee-vwha mãni}
\région{GOs}
\variante{%
mee-phwa mãni
\région{GO(s)}}
(\domainesémantique{Oiseaux})
\classe{nom}
\begin{glose}
\pfra{bec d'oiseau}
\end{glose}
\end{entrée}

\begin{entrée}{mèèvwu}{}{ⓔmèèvwu}
\région{GO PA BO}
\variante{%
mèèpu
\région{vx}}
(\domainesémantique{Parenté})
\classe{nom}
\begin{glose}
\pfra{frères/soeurs ; phratrie ; sororité (général, désigne la relation de frère ou soeur)}
\end{glose}
\end{entrée}

\begin{entrée}{me-hele}{}{ⓔme-hele}
\région{GOs}
(\domainesémantique{Outils})
\classe{nom}
\begin{glose}
\pfra{lame du couteau}
\end{glose}
\end{entrée}

\begin{entrée}{me-hele ba-pe-thra}{}{ⓔme-hele ba-pe-thra}
\région{GOs}
(\domainesémantique{Outils})
\classe{nom}
\begin{glose}
\pfra{lame de rasoir}
\end{glose}
\end{entrée}

\begin{entrée}{me-jitrua}{}{ⓔme-jitrua}
(\domainesémantique{Armes})
\classe{nom}
\begin{glose}
\pfra{flèche ; pointe de la flèche}
\end{glose}
\end{entrée}

\begin{entrée}{me-kaze}{}{ⓔme-kaze}
\région{GOs}
\variante{%
me-kale
\région{BO}}
(\domainesémantique{Marées})
\classe{nom}
\begin{glose}
\pfra{marée montante}
\end{glose}
\end{entrée}

\begin{entrée}{me-kônôô}{}{ⓔme-kônôô}
\région{GOs}
\classe{v}
\newline
\sens{1}
(\domainesémantique{Caractéristiques et propriétés des personnes})
\begin{glose}
\pfra{nonchalant ; mou}
\end{glose}
\newline
\sens{2}
(\domainesémantique{Caractéristiques et propriétés des animaux})
\begin{glose}
\pfra{doux (animal) ; apprivoisé}
\end{glose}
\end{entrée}

\begin{entrée}{meli}{}{ⓔmeli}
\région{GOs}
(\domainesémantique{Marque de nombre})
\classe{DET duel}
\begin{glose}
\pfra{marque de duel (des déterminants)}
\end{glose}
\newline
\begin{exemple}
\textbf{\pnua{êmwê melî-ã}}
\pfra{ces deux hommes-ci}
\end{exemple}
\newline
\begin{exemple}
\textbf{\pnua{êmwê meli-ôli}}
\pfra{ces deux hommes-là}
\end{exemple}
\end{entrée}

\begin{entrée}{mèloo}{}{ⓔmèloo}
\région{GOs}
\variante{%
mèloom
\région{PA BO WEM WE}, 
malòm
\région{BO PA}}
(\domainesémantique{Description des objets, formes, consistance, taille})
\classe{v.stat.}
\begin{glose}
\pfra{transparent ; limpide (eau) ; clair}
\end{glose}
\end{entrée}

\begin{entrée}{memee}{1}{ⓔmemeeⓗ1}
\région{GOs}
\variante{%
mêmê
\région{BO}}
(\domainesémantique{Mer : topographie})
\classe{nom}
\begin{glose}
\pfra{cap}
\end{glose}
\end{entrée}

\begin{entrée}{memee}{2}{ⓔmemeeⓗ2}
\région{PA}
\variante{%
wamee ne
\région{GO}}
(\domainesémantique{Modalité, verbes modaux})
\classe{MODAL}
\begin{glose}
\pfra{ça doit être ; ce serait bien}
\end{glose}
\newline
\begin{exemple}
\région{PA}
\textbf{\pnua{memee zo na mi tee-a}}
\pfra{ce serait bien qu'on prenne la route, qu'on parte avant (les autres)}
\end{exemple}
\newline
\begin{exemple}
\région{PA}
\textbf{\pnua{i memee na le xo wha}}
\pfra{c'est peut-être grand-père qui l'a mis là}
\end{exemple}
\newline
\begin{exemple}
\région{PA}
\textbf{\pnua{memee je}}
\pfra{ce doit être elle}
\end{exemple}
\end{entrée}

\begin{entrée}{memexãi}{}{ⓔmemexãi}
\région{GOs}
(\domainesémantique{Oiseaux})
\classe{nom}
\begin{glose}
\pfra{rossignol à ventre jaune}
\end{glose}
\nomscientifique{Eopsaltria flaviventris (Eopsaltriidés)}
\end{entrée}

\begin{entrée}{me-mwa}{}{ⓔme-mwa}
\région{GOs}
\variante{%
mee-mwa
\région{PA BO}}
(\domainesémantique{Types de maison, architecture de la maison})
\classe{n.LOC}
\begin{glose}
\pfra{devant (le) de la maison}
\end{glose}
\begin{glose}
\pfra{sud (le)}
\end{glose}
\newline
\relationsémantique{Cf.}{\lien{}{kaça mwa [GO] ; kaya mwa [PA BO]}}
\glosecourte{arrière ; nord}
\end{entrée}

\begin{entrée}{men-a-me}{}{ⓔmen-a-me}
\région{GOs}
(\domainesémantique{Comparaison})
\classe{ADV}
\begin{glose}
\pfra{tel quel (c'est la même apparence)}
\end{glose}
\newline
\begin{exemple}
\textbf{\pnua{za men-a-me}}
\pfra{c'est resté tel quel}
\end{exemple}
\newline
\relationsémantique{Cf.}{\lien{ⓔmee}{mee}}
\glosecourte{yeux, apparence}
\end{entrée}

\begin{entrée}{mene}{}{ⓔmene}
\région{GOs}
(\domainesémantique{Poissons})
\classe{nom}
\begin{glose}
\pfra{mulet queue bleue}
\end{glose}
\nomscientifique{Crenimugil crenilabis}
\nomscientifique{Cestraeus plicatilis (Mugilidae)}
\newline
\relationsémantique{Cf.}{\lien{ⓔnaxo}{naxo}}
\glosecourte{mulet noir}
\newline
\relationsémantique{Cf.}{\lien{ⓔwhaiⓗ1}{whai}}
\glosecourte{mulet (de petite taille)}
\newline
\relationsémantique{Cf.}{\lien{ⓔjumeã}{jumeã}}
\glosecourte{mulet (le plus gros)}
\end{entrée}

\begin{entrée}{me-nee-vwo}{}{ⓔme-nee-vwo}
\formephonétique{meɳeːβo ; meɳeːβɷ}
\région{GOs WEM}
\variante{%
mèneevwò,mèneevwu-n, mèneexu-n
\région{BO PA}}
(\domainesémantique{Relations et interaction sociales})
\classe{nom}
\begin{glose}
\pfra{attitude ; comportement ; manière de faire}
\end{glose}
\newline
\begin{exemple}
\région{GO}
\textbf{\pnua{whaya me-nee-vwö bami ?}}
\pfra{comment fait-on le bami ?}
\end{exemple}
\newline
\begin{exemple}
\région{GO}
\textbf{\pnua{nu kuele la me-nee-vwö i je}}
\pfra{je n'aime pas ses manières}
\end{exemple}
\newline
\begin{exemple}
\région{GO}
\textbf{\pnua{kavwo nu tõõne kaamweni me-nee-vwö i çö}}
\pfra{je ne comprends pas ta façon de procéder}
\end{exemple}
\end{entrée}

\begin{entrée}{mèng}{}{ⓔmèng}
\région{BO}
(\domainesémantique{Aliments, alimentation})
\classe{v}
\begin{glose}
\pfra{difficile (qui fait le) [Corne]}
\end{glose}
\end{entrée}

\begin{entrée}{mèni}{}{ⓔmèni}
\région{GOs PA BO}
\variante{%
mèèni
\région{BO}}
(\domainesémantique{Oiseaux})
\classe{nom}
\begin{glose}
\pfra{oiseau}
\end{glose}
\newline
\étymologie{
\langue{POc}
\étymon{*manuk}}
\end{entrée}

\begin{entrée}{mèni bwa bolomakau}{}{ⓔmèni bwa bolomakau}
\région{GOs}
\variante{%
mèni bwa bòlòxau
\région{GO(s)}}
(\domainesémantique{Oiseaux})
\classe{nom}
\begin{glose}
\pfra{merle des Moluques}
\end{glose}
\nomscientifique{Acridotheres tristis}
\end{entrée}

\begin{entrée}{menixe}{}{ⓔmenixe}
\région{GOs PA BO}
(\domainesémantique{Comparaison})
\classe{v.COMPAR}
\begin{glose}
\pfra{pareil (être) ; semblable}
\end{glose}
\newline
\begin{exemple}
\région{PA}
\textbf{\pnua{menixe vhaa kòlò-n}}
\pfra{la parole de son côté est la même}
\end{exemple}
\end{entrée}

\begin{entrée}{mènõ}{}{ⓔmènõ}
\région{GOs}
\variante{%
mènõng
\région{PA BO}}
(\domainesémantique{Processus liés aux plantes})
\classe{v}
\begin{glose}
\pfra{fané}
\end{glose}
\begin{glose}
\pfra{séché ; desséché (plantes)}
\end{glose}
\newline
\begin{exemple}
\région{GO}
\textbf{\pnua{mènõ kô-kui}}
\pfra{la tige de l'igname est fanée}
\end{exemple}
\newline
\begin{exemple}
\région{BO}
\textbf{\pnua{i mènõng a muu-n}}
\pfra{la fleur est fanée}
\end{exemple}
\end{entrée}

\begin{entrée}{me-pwamwa}{}{ⓔme-pwamwa}
\région{GOs}
(\domainesémantique{Directions})
\classe{nom}
\begin{glose}
\pfra{sud (lit. visage du pays)}
\end{glose}
\end{entrée}

\begin{entrée}{me-phwa}{}{ⓔme-phwa}
\formephonétique{meβa}
\région{GOs}
\région{GOs}
\variante{%
me-pwa
}
(\domainesémantique{Corps humain})
\classe{nom}
\begin{glose}
\pfra{bouche}
\end{glose}
\newline
\begin{exemple}
\région{PA}
\textbf{\pnua{me-phwa-n}}
\pfra{sa bouche}
\end{exemple}
\end{entrée}

\begin{entrée}{mèra-}{}{ⓔmèra-}
\région{GOs BO}
(\domainesémantique{Insectes
, Préfixes classificateurs sémantiques})
\classe{PREF}
\begin{glose}
\pfra{préfixe des noms de chenilles}
\end{glose}
\newline
\begin{sous-entrée}{mèra-mòmò}{ⓔmèra-ⓝmèra-mòmò}
\begin{glose}
\pfra{chenille de peuplier}
\end{glose}
\end{sous-entrée}
\newline
\begin{sous-entrée}{mèra-uva}{ⓔmèra-ⓝmèra-uva}
\begin{glose}
\pfra{chenille verte}
\end{glose}
\end{sous-entrée}
\newline
\begin{sous-entrée}{mèra-chaamwa}{ⓔmèra-ⓝmèra-chaamwa}
\begin{glose}
\pfra{chenille du bananier}
\end{glose}
\newline
\relationsémantique{Cf.}{\lien{ⓔmeeli}{meeli}}
\glosecourte{chenille}
\end{sous-entrée}
\end{entrée}

\begin{entrée}{mero}{}{ⓔmero}
\région{BO}
(\domainesémantique{Armes})
\classe{nom}
\begin{glose}
\pfra{casse-tête à bout phallique (aussi une espèce d'arbre)}
\end{glose}
\newline
\note{non vérifié}{général}{}
\end{entrée}

\begin{entrée}{mero-dö}{}{ⓔmero-dö}
\formephonétique{me'ro-ndɷ}
\région{GOs}
(\domainesémantique{Description des objets, formes, consistance, taille})
\classe{v}
\begin{glose}
\pfra{pointu}
\end{glose}
\newline
\begin{exemple}
\textbf{\pnua{mero-dö bwe-mwa}}
\pfra{le toit de la maison est pointu}
\end{exemple}
\end{entrée}

\begin{entrée}{mèròò}{}{ⓔmèròò}
\région{PA BO WEM WE}
\variante{%
mããro
\région{BO [BM]}}
(\domainesémantique{Noms des plantes})
\classe{nom}
\begin{glose}
\pfra{herbe ; pelouse}
\end{glose}
\newline
\begin{exemple}
\région{PA}
\textbf{\pnua{bwa mèròò}}
\pfra{sur l'herbe}
\end{exemple}
\end{entrée}

\begin{entrée}{mè-thi}{}{ⓔmè-thi}
\région{GOs BO PA}
(\domainesémantique{Corps humain})
\classe{nom}
\begin{glose}
\pfra{têton (du sein)}
\end{glose}
\newline
\begin{exemple}
\région{BO}
\textbf{\pnua{mè-thi-n}}
\pfra{son têton}
\end{exemple}
\end{entrée}

\begin{entrée}{me-trabwa}{}{ⓔme-trabwa}
\région{GOs}
(\domainesémantique{Description des objets, formes, consistance, taille})
\classe{NMLZ}
\begin{glose}
\pfra{forme}
\end{glose}
\newline
\begin{exemple}
\textbf{\pnua{haze me-trabwa mõ-xabu (mõ-kabu), tretrabwau, khawali mee !}}
\pfra{la forme du temple est bizarre, elle est ronde et la flèche est très haute !}
\end{exemple}
\end{entrée}

\begin{entrée}{mètrô}{}{ⓔmètrô}
\région{GO}
(\domainesémantique{Instruments})
\classe{nom}
\begin{glose}
\pfra{battoir pour écorce}
\end{glose}
\newline
\note{non vérifié}{général}{}
\end{entrée}

\begin{entrée}{mevwuu}{}{ⓔmevwuu}
\région{GOs}
(\domainesémantique{Oiseaux})
\classe{nom}
\begin{glose}
\pfra{bengali à bec rouge (ainsi nommé car est toujours en groupe)}
\end{glose}
\nomscientifique{Estrilda astrild (Estrildidés)}
\end{entrée}

\begin{entrée}{mè-wõ}{}{ⓔmè-wõ}
\région{GOs}
\variante{%
mee-wony
\région{PA BO}}
(\domainesémantique{Navigation})
\classe{nom}
\begin{glose}
\pfra{proue}
\end{glose}
\newline
\relationsémantique{Cf.}{\lien{}{gò-wõ; gò-wony [PA]}}
\glosecourte{milieu du bateau}
\newline
\relationsémantique{Cf.}{\lien{ⓔpòbwinõ-wõ}{pòbwinõ-wõ}}
\glosecourte{poupe du bateau}
\newline
\relationsémantique{Cf.}{\lien{ⓔmura-wõ}{mura-wõ}}
\glosecourte{proue}
\end{entrée}

\begin{entrée}{mèxèè}{}{ⓔmèxèè}
\région{BO}
(\domainesémantique{Préparation des aliments; modes de préparation et de cuisson})
\classe{v}
\begin{glose}
\pfra{cailler [BM]}
\end{glose}
\end{entrée}

\begin{entrée}{mexò}{}{ⓔmexò}
\région{GOs}
(\domainesémantique{Santé, maladie})
\classe{nom}
\begin{glose}
\pfra{varicelle}
\end{glose}
\begin{glose}
\pfra{rougeole}
\end{glose}
\end{entrée}

\begin{entrée}{meyaam}{}{ⓔmeyaam}
\région{PA}
(\domainesémantique{Types de champs})
\classe{n ; v}
\begin{glose}
\pfra{tarodière sèche}
\end{glose}
\begin{glose}
\pfra{planter un champ de taro aux abords d'une source, sans conduite d'eau}
\end{glose}
\end{entrée}

\begin{entrée}{mâge}{}{ⓔmâge}
\région{BO}
\variante{%
mhâge
}
\classe{n ; v}
\newline
\sens{1}
(\domainesémantique{Noeuds})
\begin{glose}
\pfra{nouer ; noeud (du filet)}
\end{glose}
\newline
\sens{2}
(\domainesémantique{Pêche})
\begin{glose}
\pfra{maille ; faire un filet}
\end{glose}
\end{entrée}

\begin{entrée}{mi}{}{ⓔmi}
\région{GOs PA BO}
(\domainesémantique{Pronoms})
\classe{PRO 1° pers. duel incl. (sujet)}
\begin{glose}
\pfra{nous deux}
\end{glose}
\end{entrée}

\begin{entrée}{-mi}{}{ⓔ-mi}
\région{GOsPA BO}
(\domainesémantique{Directionnels})
\classe{DIR (centripète)}
\begin{glose}
\pfra{vers ici; vers ego}
\end{glose}
\newline
\begin{exemple}
\textbf{\pnua{ã-mi !}}
\pfra{viens ici !}
\end{exemple}
\newline
\relationsémantique{Cf.}{\lien{ⓔdu-mi}{du-mi}}
\glosecourte{vers ici en bas}
\newline
\relationsémantique{Cf.}{\lien{}{da-mi}}
\glosecourte{vers ici en haut}
\end{entrée}

\begin{entrée}{mibwa}{}{ⓔmibwa}
\région{GOs}
\variante{%
miibwan
\région{BO}}
(\domainesémantique{Religion, représentations religieuses})
\classe{nom}
\begin{glose}
\pfra{être aquatique (nom d'un)}
\end{glose}
\newline
\note{(être craint par les femmes enceintes car il est responsable des déformations physiques, des albinos, etc.)}{glose}{}
\end{entrée}

\begin{entrée}{mii}{}{ⓔmii}
\région{GOs PA BO}
\classe{v.stat.}
\newline
\sens{1}
(\domainesémantique{Couleurs})
\begin{glose}
\pfra{rouge ; violet}
\end{glose}
\newline
\begin{sous-entrée}{dili mii}{ⓔmiiⓢ1ⓝdili mii}
\begin{glose}
\pfra{terre rouge}
\end{glose}
\end{sous-entrée}
\newline
\begin{sous-entrée}{we mii}{ⓔmiiⓢ1ⓝwe mii}
\begin{glose}
\pfra{vin (lit. eau rouge)}
\end{glose}
\end{sous-entrée}
\newline
\sens{2}
(\domainesémantique{Processus liés aux plantes})
\begin{glose}
\pfra{mûr}
\end{glose}
\newline
\begin{sous-entrée}{chaamwa mii}{ⓔmiiⓢ2ⓝchaamwa mii}
\begin{glose}
\pfra{banane mûre}
\end{glose}
\end{sous-entrée}
\newline
\étymologie{
\langue{POc}
\étymon{*maiRa, *meRaq}
\glosecourte{rouge}}
\end{entrée}

\begin{entrée}{mîjo}{}{ⓔmîjo}
\région{BO JAWE}
(\domainesémantique{Arbre})
\classe{nom}
\begin{glose}
\pfra{gaïac}
\end{glose}
\end{entrée}

\begin{entrée}{mimaalu}{}{ⓔmimaalu}
\région{BO}
(\domainesémantique{Description des objets, formes, consistance, taille})
\classe{v}
\begin{glose}
\pfra{invisible [Corne]}
\end{glose}
\newline
\begin{exemple}
\textbf{\pnua{a-mimaalu}}
\pfra{un être invisible}
\end{exemple}
\end{entrée}

\begin{entrée}{mimi}{}{ⓔmimi}
\région{GOs}
\variante{%
minòn
\région{PA BO}}
(\domainesémantique{Mammifères})
\classe{nom}
\begin{glose}
\pfra{chat}
\end{glose}
\end{entrée}

\begin{entrée}{mini-}{}{ⓔmini-}
\région{GOs}
(\domainesémantique{Aliments, alimentation})
\classe{nom}
\begin{glose}
\pfra{débris ; restes}
\end{glose}
\newline
\begin{sous-entrée}{mini-ce}{ⓔmini-ⓝmini-ce}
\begin{glose}
\pfra{sciure}
\end{glose}
\end{sous-entrée}
\newline
\begin{sous-entrée}{mini-mã-bò}{ⓔmini-ⓝmini-mã-bò}
\begin{glose}
\pfra{les restes de nourriture mâchée par les roussettes}
\end{glose}
\newline
\relationsémantique{Cf.}{\lien{ⓔjaa-ce}{jaa-ce}}
\glosecourte{copeaux}
\end{sous-entrée}
\end{entrée}

\begin{entrée}{minõ}{}{ⓔminõ}
\région{GOs}
\région{PA BO}
\variante{%
minòng
}
(\domainesémantique{Préparation des aliments; modes de préparation et de cuisson})
\classe{v}
\begin{glose}
\pfra{cuit}
\end{glose}
\newline
\begin{exemple}
\région{GO}
\textbf{\pnua{u minõ dröö}}
\pfra{les marmites sont prêtes (= cuites)}
\end{exemple}
\newline
\begin{exemple}
\région{PA}
\textbf{\pnua{u minong doo}}
\pfra{la marmite est prête}
\end{exemple}
\newline
\relationsémantique{Cf.}{\lien{ⓔphuu}{phuu}}
\glosecourte{cuit}
\newline
\relationsémantique{Ant.}{\lien{}{mãiyã [GOs], meã [PA]}}
\glosecourte{mal cuit}
\end{entrée}

\begin{entrée}{minyõ}{}{ⓔminyõ}
\région{GOs}
(\domainesémantique{Bananiers et bananes})
\classe{nom}
\begin{glose}
\pfra{banane (espèce de petite taille qui se mange bien mûre)}
\end{glose}
\end{entrée}

\begin{entrée}{mõ}{1}{ⓔmõⓗ1}
\région{GOs}
\région{PA BO}
\variante{%
mòl
}
\newline
\sens{1}
(\domainesémantique{Description des objets, formes, consistance, taille})
\classe{v.statif}
\begin{glose}
\pfra{asséché ; sec (rivière, etc.)}
\end{glose}
\newline
\begin{exemple}
\textbf{\pnua{u mõ}}
\pfra{c'est sec}
\end{exemple}
\newline
\begin{exemple}
\textbf{\pnua{e mõ-du we}}
\pfra{l'eau a baissé, s'est retirée}
\end{exemple}
\newline
\sens{2}
(\domainesémantique{Topographie})
\begin{glose}
\pfra{à terre}
\end{glose}
\newline
\begin{sous-entrée}{bwa mõ}{ⓔmõⓗ1ⓢ2ⓝbwa mõ}
\région{GO}
\begin{glose}
\pfra{à terre (par rapport à la mer)}
\end{glose}
\newline
\begin{exemple}
\région{PA}
\textbf{\pnua{bwa mòl}}
\pfra{à terre (par rapport à la mer)}
\end{exemple}
\newline
\note{pha-mòle}{grammaire}{vider qqch}
\end{sous-entrée}
\newline
\étymologie{
\langue{POc}
\étymon{*ma-masa}
\glosecourte{sec, marée basse}}
\end{entrée}

\begin{entrée}{mõ}{2}{ⓔmõⓗ2}
\formephonétique{mɔ̃}
\région{GOs WEM}
(\domainesémantique{Pronoms})
\classe{PRO 1° pers. triel incl. (sujet)}
\begin{glose}
\pfra{nous trois}
\end{glose}
\end{entrée}

\begin{entrée}{mõ-}{}{ⓔmõ-}
\formephonétique{mɔ̃}
\région{GOs PA BO}
\variante{%
mwõ-
\région{PA}}
(\domainesémantique{Types de maison, architecture de la maison})
\classe{PREF (désignant un contenant)}
\begin{glose}
\pfra{maison ; maison (grande chefferie )}
\end{glose}
\begin{glose}
\pfra{contenant de qqch ; manche (vêtement)}
\end{glose}
\newline
\begin{exemple}
\région{GO}
\textbf{\pnua{mõ-jö}}
\pfra{ta maison}
\end{exemple}
\newline
\begin{exemple}
\région{GO}
\textbf{\pnua{mwa-dili mõ-nu}}
\pfra{ma maison en terre}
\end{exemple}
\newline
\begin{exemple}
\région{GO}
\textbf{\pnua{mõ-nu ca mwa-dili}}
\pfra{ma maison est en terre}
\end{exemple}
\newline
\begin{exemple}
\région{PA}
\textbf{\pnua{whaa mõ-ny (a) mwa mãe}}
\pfra{ma grande maison en paille}
\end{exemple}
\newline
\begin{exemple}
\région{BO}
\textbf{\pnua{ia mõ-m ?}}
\pfra{ou est ta maison?}
\end{exemple}
\newline
\begin{sous-entrée}{mõ-ima}{ⓔmõ-ⓝmõ-ima}
\begin{glose}
\pfra{urine}
\end{glose}
\end{sous-entrée}
\newline
\begin{sous-entrée}{mõ-puyol}{ⓔmõ-ⓝmõ-puyol}
\begin{glose}
\pfra{cuisine}
\end{glose}
\end{sous-entrée}
\newline
\begin{sous-entrée}{mõ-kabun}{ⓔmõ-ⓝmõ-kabun}
\begin{glose}
\pfra{église}
\end{glose}
\end{sous-entrée}
\newline
\begin{sous-entrée}{mõ-wetin}{ⓔmõ-ⓝmõ-wetin}
\région{PA BO}
\begin{glose}
\pfra{encrier}
\end{glose}
\newline
\note{forme déterminée de mwa}{grammaire}{}
\end{sous-entrée}
\newline
\relationsémantique{Cf.}{\lien{ⓔhê-}{hê-}}
\glosecourte{préfixe désignant un contenu}
\end{entrée}

\begin{entrée}{mõã}{}{ⓔmõã}
\région{GOs WEM}
\variante{%
mhõ
\région{PA WEM}, 
mõ
\région{BO}}
(\domainesémantique{Aliments, alimentation})
\classe{nom}
\begin{glose}
\pfra{restes de nourriture}
\end{glose}
\begin{glose}
\pfra{provisions de route (froid)}
\end{glose}
\newline
\begin{exemple}
\région{GO}
\textbf{\pnua{mõã-nu}}
\pfra{mes restes de nourriture}
\end{exemple}
\newline
\begin{exemple}
\région{WEM}
\textbf{\pnua{co kaale xa mhõ-ny}}
\pfra{laisse- moi de la nourriture}
\end{exemple}
\newline
\begin{exemple}
\région{PA BO}
\textbf{\pnua{mõõ-la}}
\pfra{leurs vivres}
\end{exemple}
\newline
\begin{exemple}
\région{PA BO}
\textbf{\pnua{la thu mõ}}
\pfra{ils préparent des vivres}
\end{exemple}
\newline
\begin{sous-entrée}{ke-mõã}{ⓔmõãⓝke-mõã}
\région{GO}
\begin{glose}
\pfra{panier à restes}
\end{glose}
\newline
\note{forme déterminée : mhõã, mõã}{grammaire}{}
\end{sous-entrée}
\end{entrée}

\begin{entrée}{mõ-butro}{}{ⓔmõ-butro}
\formephonétique{mɔ̃'buɽo}
\région{GOs}
(\domainesémantique{Types de maison, architecture de la maison})
\classe{nom}
\begin{glose}
\pfra{douche (lieu)}
\end{glose}
\end{entrée}

\begin{entrée}{mõ-caaxò}{}{ⓔmõ-caaxò}
\région{GO}
\variante{%
mõ-caao
\région{GO}}
(\domainesémantique{Types de maison, architecture de la maison})
\classe{nom}
\begin{glose}
\pfra{maison où se retirent les femmes (pendant les règles) (lit. où l'on se cache)}
\end{glose}
\end{entrée}

\begin{entrée}{mõ-cãna}{}{ⓔmõ-cãna}
\région{GOs}
\variante{%
mõ-cana
\région{BO}}
(\domainesémantique{Corps humain})
\classe{nom}
\begin{glose}
\pfra{poumon (lit. contenant-respiration)}
\end{glose}
\newline
\begin{exemple}
\région{BO}
\textbf{\pnua{mõ-cana-n}}
\pfra{ses poumons}
\end{exemple}
\end{entrée}

\begin{entrée}{mõ-do}{}{ⓔmõ-do}
\région{BO}
(\domainesémantique{Armes})
\classe{nom}
\begin{glose}
\pfra{manche de la sagaie}
\end{glose}
\end{entrée}

\begin{entrée}{mõ-ẽnõ}{}{ⓔmõ-ẽnõ}
\formephonétique{mɔ̃-ɛ̃ɳɔ̃}
\région{GOs}
\variante{%
mõ-ènõ
\région{BO}}
(\domainesémantique{Corps humain})
\classe{nom}
\begin{glose}
\pfra{utérus ;}
\end{glose}
\begin{glose}
\pfra{placenta}
\end{glose}
\end{entrée}

\begin{entrée}{mõgavwo}{}{ⓔmõgavwo}
\formephonétique{mɔ̃'gaβo}
\région{GOs}
\variante{%
mûgavo
\région{PA}}
(\domainesémantique{Portage})
\classe{nom}
\begin{glose}
\pfra{corde de portage (des fagots)}
\end{glose}
\end{entrée}

\begin{entrée}{môgo}{}{ⓔmôgo}
\région{GOs}
\variante{%
mogòn
\région{BO PA}, 
mûgòn
\région{BO}}
(\domainesémantique{Santé, maladie})
\classe{v}
\begin{glose}
\pfra{migraine (avoir la) ; mal de tête (avoir un)}
\end{glose}
\newline
\begin{exemple}
\région{GO}
\textbf{\pnua{nu môgo}}
\pfra{j'ai mal à la tête}
\end{exemple}
\newline
\begin{exemple}
\région{PA}
\textbf{\pnua{nu mogon}}
\pfra{j'ai mal à la tête}
\end{exemple}
\end{entrée}

\begin{entrée}{mõgu}{}{ⓔmõgu}
\région{GOs BO}
\variante{%
mwõgu
\région{GO(s)}}
\classe{v ; n}
\newline
\sens{1}
(\domainesémantique{Verbes d'action (en général)})
\begin{glose}
\pfra{travailler ; travail}
\end{glose}
\newline
\begin{sous-entrée}{mhènõ-mõgu}{ⓔmõguⓢ1ⓝmhènõ-mõgu}
\begin{glose}
\pfra{lieu de travail}
\end{glose}
\end{sous-entrée}
\newline
\begin{sous-entrée}{ba-mõgu}{ⓔmõguⓢ1ⓝba-mõgu}
\begin{glose}
\pfra{outils}
\end{glose}
\end{sous-entrée}
\newline
\begin{sous-entrée}{mhènõ ba-mõgu}{ⓔmõguⓢ1ⓝmhènõ ba-mõgu}
\begin{glose}
\pfra{appentis à outils}
\end{glose}
\end{sous-entrée}
\newline
\begin{sous-entrée}{mõgu-raa}{ⓔmõguⓢ1ⓝmõgu-raa}
\begin{glose}
\pfra{travailler mal, faire de travers}
\end{glose}
\end{sous-entrée}
\newline
\sens{2}
(\domainesémantique{Fonctions naturelles humaines})
\begin{glose}
\pfra{sueur [BO]}
\end{glose}
\end{entrée}

\begin{entrée}{mõ-hèlè}{}{ⓔmõ-hèlè}
\région{GOs BO PA}
(\domainesémantique{Instruments})
\classe{nom}
\begin{glose}
\pfra{étui ; fourreau de couteau ; manche}
\end{glose}
\end{entrée}

\begin{entrée}{mõ-hovwo}{}{ⓔmõ-hovwo}
\région{GOs}
\variante{%
mõ-(h)ovwo
\région{GO(s)}, 
mõ-hopo
\région{GO vx}}
(\domainesémantique{Corps humain})
\classe{nom}
\begin{glose}
\pfra{estomac}
\end{glose}
\end{entrée}

\begin{entrée}{mõ-ima}{}{ⓔmõ-ima}
\région{GOs WEM BO}
(\domainesémantique{Corps humain})
\classe{nom}
\begin{glose}
\pfra{vessie}
\end{glose}
\newline
\begin{exemple}
\région{BO}
\textbf{\pnua{mõ-ima-n}}
\pfra{sa vessie}
\end{exemple}
\end{entrée}

\begin{entrée}{mõ-iyu}{}{ⓔmõ-iyu}
\région{BO}
(\domainesémantique{Types de maison, architecture de la maison})
\classe{nom}
\begin{glose}
\pfra{magasin ; boutique}
\end{glose}
\end{entrée}

\begin{entrée}{mõ-ja}{}{ⓔmõ-ja}
\région{GOs}
(\domainesémantique{Objets et meubles de la maison})
\classe{nom}
\begin{glose}
\pfra{poubelle}
\end{glose}
\end{entrée}

\begin{entrée}{mõ-kabun}{}{ⓔmõ-kabun}
\région{BO}
(\domainesémantique{Religion, représentations religieuses})
\classe{nom}
\begin{glose}
\pfra{église ; temple}
\end{glose}
\end{entrée}

\begin{entrée}{moko}{}{ⓔmoko}
\région{BO}
(\domainesémantique{Types de maison, architecture de la maison})
\classe{nom}
\begin{glose}
\pfra{gaulettes horizontales}
\end{glose}
\newline
\note{enroulée autour des "me-de" pour former la corbeille du poteau central de la case (Dubois)}{glose}{}
\newline
\note{non vérifié}{général}{}
\end{entrée}

\begin{entrée}{mõlò}{}{ⓔmõlò}
\formephonétique{mɔ̃lɔ}
\région{GOs}
\variante{%
mòlò
\région{PA BO}, 
mòòlè
\région{BO}}
\classe{v ; n}
\newline
\sens{1}
(\domainesémantique{Cours de la vie})
\begin{glose}
\pfra{vivre ; vivant; vie}
\end{glose}
\begin{glose}
\pfra{coutumes}
\end{glose}
\newline
\begin{exemple}
\région{GO}
\textbf{\pnua{mõlò-wa kêê-ã mani kibu-ã}}
\pfra{les coutumes de nos pères et grand-pères}
\end{exemple}
\newline
\begin{exemple}
\région{PA}
\textbf{\pnua{mòlò-n}}
\pfra{sa vie}
\end{exemple}
\newline
\begin{exemple}
\région{PA}
\textbf{\pnua{li pe-mòlò bulu}}
\pfra{ils vivent ensemble}
\end{exemple}
\newline
\begin{exemple}
\région{PA}
\textbf{\pnua{i mòlò kuraa-li}}
\pfra{ils sont nerveux, agités (lit. leur sang vit)}
\end{exemple}
\newline
\begin{exemple}
\région{BO}
\textbf{\pnua{i mòlò phagòò-n}}
\pfra{il est vif, courageux}
\end{exemple}
\newline
\begin{sous-entrée}{me-mòlò}{ⓔmõlòⓢ1ⓝme-mòlò}
\région{PA BO}
\begin{glose}
\pfra{vie}
\end{glose}
\end{sous-entrée}
\newline
\sens{2}
(\domainesémantique{Fonctions naturelles humaines})
\begin{glose}
\pfra{rassasié}
\end{glose}
\newline
\begin{exemple}
\région{GO}
\textbf{\pnua{nu mõlò}}
\pfra{je suis rassasié, j'ai assez mangé}
\end{exemple}
\newline
\étymologie{
\langue{POc}
\étymon{*madip, *maqudi(p)}
\glosecourte{vie, vivre}}
\end{entrée}

\begin{entrée}{mõ-mãxi}{}{ⓔmõ-mãxi}
\région{GOs}
(\domainesémantique{Religion, représentations religieuses})
\classe{nom}
\begin{glose}
\pfra{protestants (les)}
\end{glose}
\end{entrée}

\begin{entrée}{mõnõ}{1}{ⓔmõnõⓗ1}
\formephonétique{mɔ̃ɳɔ̃}
\région{GOs}
\variante{%
mènòòn
\formephonétique{mɛnɔːn}
\région{PA WEM WE BO}}
(\domainesémantique{Adverbes déictiques de temps})
\classe{ADV}
\begin{glose}
\pfra{demain ; lendemain (le) ; prochain}
\end{glose}
\newline
\begin{exemple}
\région{BO}
\textbf{\pnua{mènòòn mani bwòna}}
\pfra{demain et après-demain}
\end{exemple}
\newline
\begin{sous-entrée}{bwona, kaça mõnõ}{ⓔmõnõⓗ1ⓝbwona, kaça mõnõ}
\région{GO}
\begin{glose}
\pfra{après-demain}
\end{glose}
\end{sous-entrée}
\newline
\begin{sous-entrée}{mõnõ ne waa}{ⓔmõnõⓗ1ⓝmõnõ ne waa}
\région{GO}
\begin{glose}
\pfra{demain matin}
\end{glose}
\end{sous-entrée}
\newline
\begin{sous-entrée}{mõnõ ne thròbò}{ⓔmõnõⓗ1ⓝmõnõ ne thròbò}
\région{GO}
\begin{glose}
\pfra{demain soir}
\end{glose}
\end{sous-entrée}
\newline
\begin{sous-entrée}{kaa mõnõ}{ⓔmõnõⓗ1ⓝkaa mõnõ}
\région{GO}
\begin{glose}
\pfra{l'an prochain}
\end{glose}
\end{sous-entrée}
\end{entrée}

\begin{entrée}{mõnõ}{2}{ⓔmõnõⓗ2}
\formephonétique{mɔ̃ɳɔ̃}
\région{GOs}
\variante{%
mõnô, mwõnô
\région{BO}}
(\domainesémantique{Aliments, alimentation})
\classe{nom}
\begin{glose}
\pfra{graisse (de tortue uniquement)}
\end{glose}
\begin{glose}
\pfra{huileux ; graisseux}
\end{glose}
\newline
\begin{sous-entrée}{mõõnõ pwòe}{ⓔmõnõⓗ2ⓝmõõnõ pwòe}
\begin{glose}
\pfra{graisse de tortue}
\end{glose}
\newline
\begin{exemple}
\région{BO}
\textbf{\pnua{i mõnô cii-ny}}
\pfra{ma peau est grasse}
\end{exemple}
\end{sous-entrée}
\newline
\étymologie{
\langue{POc}
\étymon{*moɲak}
\glosecourte{graisse}}
\end{entrée}

\begin{entrée}{mõnu}{}{ⓔmõnu}
\formephonétique{mɔ̃ɳu}
\région{GOsPA}
\variante{%
mõnu
\région{PA BO}, 
mwonu
\région{BO}}
\classe{LOC}
\newline
\sens{1}
(\domainesémantique{Prépositions})
\begin{glose}
\pfra{proche ; près}
\end{glose}
\newline
\begin{exemple}
\région{PA}
\textbf{\pnua{ge mõnu}}
\pfra{il est proche}
\end{exemple}
\newline
\begin{exemple}
\région{PA}
\textbf{\pnua{kò-mõnu kaò}}
\pfra{la saison des pluies est proche}
\end{exemple}
\newline
\sens{2}
(\domainesémantique{Aspect})
\begin{glose}
\pfra{sur le point de ; bientôt}
\end{glose}
\begin{glose}
\pfra{presque}
\end{glose}
\newline
\begin{exemple}
\région{GO}
\textbf{\pnua{mõnu vwo la a}}
\pfra{ils sont sur le point de partir}
\end{exemple}
\newline
\begin{exemple}
\région{GO}
\textbf{\pnua{mõnu tree}}
\pfra{il va faire jour}
\end{exemple}
\newline
\begin{exemple}
\région{BO}
\textbf{\pnua{u mõnu u tèèn}}
\pfra{il va faire jour}
\end{exemple}
\newline
\relationsémantique{Ant.}{\lien{ⓔhòò}{hòò}}
\glosecourte{loin}
\end{entrée}

\begin{entrée}{mõõ-}{}{ⓔmõõ-}
\région{GOs}
\variante{%
mõõ-
\région{PA}, 
mwòòn
\région{BO}}
(\domainesémantique{Alliance})
\classe{nom}
\begin{glose}
\pfra{beau-père ; belle-mère (père/mère d'épouse ; père/mère du mari)}
\end{glose}
\begin{glose}
\pfra{gendre (mari de fille) ; belle-fille (épouse de fils)}
\end{glose}
\begin{glose}
\pfra{beau-père et beau-fils (le terme duel tombe en désuétude)}
\end{glose}
\newline
\begin{exemple}
\région{PA}
\textbf{\pnua{mõõ-n}}
\pfra{son beau-fils}
\end{exemple}
\newline
\begin{exemple}
\région{PA}
\textbf{\pnua{mõõ-n thòòmwa, mõõ-n dòòmwa}}
\pfra{sa belle-fille}
\end{exemple}
\end{entrée}

\begin{entrée}{mõõdi}{}{ⓔmõõdi}
\région{GOs}
\variante{%
mõõdim
\région{WEM WE PA BO}}
\newline
\sens{1}
(\domainesémantique{Sentiments})
\classe{v}
\begin{glose}
\pfra{honte (avoir)(lié au deuil car on n'a pas su conserver la vie)}
\end{glose}
\newline
\begin{exemple}
\textbf{\pnua{e mõõdi pexa la kobwe ?}}
\pfra{a-t-il honte de ce qu'il a dit ?}
\end{exemple}
\newline
\begin{exemple}
\région{BO}
\textbf{\pnua{nu mõõdim na nu va}}
\pfra{j'aipeur de parler}
\end{exemple}
\newline
\begin{exemple}
\textbf{\pnua{Ôô ! e mõõdi !}}
\pfra{Oui, il en a honte !}
\end{exemple}
\newline
\sens{2}
(\domainesémantique{Coutumes, dons coutumiers})
\classe{nom}
\begin{glose}
\pfra{deuil ; coutume de deuil}
\end{glose}
\newline
\relationsémantique{Cf.}{\lien{}{cöńi, giul, thiin}}
\glosecourte{pleurer un mort; être en deuil [BO, PA]}
\end{entrée}

\begin{entrée}{mõõdim}{}{ⓔmõõdim}
\région{PA BO}
\classe{v ; n}
\newline
\sens{1}
(\domainesémantique{Coutumes, dons coutumiers})
\begin{glose}
\pfra{cérémonie de deuil aux maternels}
\end{glose}
\newline
\sens{2}
(\domainesémantique{Sentiments})
\begin{glose}
\pfra{honte (avoir) (lié au deuil car on n'a pas su conserver la vie)}
\end{glose}
\end{entrée}

\begin{entrée}{mõõmõ}{}{ⓔmõõmõ}
\formephonétique{mɔ̃ːmɔ̃ (légère nasalité)}
\région{GOs PA BO}
(\domainesémantique{Arbre})
\classe{nom}
\begin{glose}
\pfra{peuplier kanak (représente la terre et la femme)}
\end{glose}
\begin{glose}
\pfra{erythrine à épines}
\end{glose}
\nomscientifique{Erythrina indica Lam. (Leguminacées)}
\nomscientifique{Erythrina fusca Lour.}
\end{entrée}

\begin{entrée}{mõõxi}{}{ⓔmõõxi}
\région{GOs PA}
\variante{%
mòòle
\région{BO}}
\classe{nom}
\newline
\sens{1}
(\domainesémantique{Cours de la vie})
\begin{glose}
\pfra{vie (principe de vie)}
\end{glose}
\begin{glose}
\pfra{salut (le) (religion)}
\end{glose}
\newline
\sens{2}
(\domainesémantique{Parenté})
\begin{glose}
\pfra{descendance}
\end{glose}
\newline
\begin{exemple}
\région{GO}
\textbf{\pnua{e kae mõõxi-nu}}
\pfra{il m'a sauvé la vie}
\end{exemple}
\newline
\begin{exemple}
\région{PA}
\textbf{\pnua{e kae mõõxi-ny}}
\pfra{il m'a sauvé la vie}
\end{exemple}
\newline
\begin{exemple}
\région{GO}
\textbf{\pnua{nu thu mõõxi}}
\pfra{je fais des boutures (lié à la notion de vie)}
\end{exemple}
\newline
\begin{exemple}
\région{GO}
\textbf{\pnua{kixa mõõxi}}
\pfra{il est sans descendance (hommes) ; il est mort (plantes)}
\end{exemple}
\newline
\étymologie{
\langue{POc}
\étymon{*maqudip}}
\end{entrée}

\begin{entrée}{mõ-puyòl}{}{ⓔmõ-puyòl}
\région{BO}
\variante{%
mõ-wuyòl
\région{BO}}
(\domainesémantique{Types de maison, architecture de la maison})
\classe{nom}
\begin{glose}
\pfra{cuisine}
\end{glose}
\end{entrée}

\begin{entrée}{mõ-phaa-ce-bo}{}{ⓔmõ-phaa-ce-bo}
\région{GOs}
\région{WEM PA}
\variante{%
mõ-phaa-ce-bòn
}
(\domainesémantique{Feu : objets et actions liés au feu})
\classe{nom}
\begin{glose}
\pfra{foyer ; maison où l'on fait le feu pour dormir}
\end{glose}
\end{entrée}

\begin{entrée}{mõ-pha-yai}{}{ⓔmõ-pha-yai}
\région{GOs WEM}
(\domainesémantique{Feu : objets et actions liés au feu})
\classe{nom}
\begin{glose}
\pfra{foyer}
\end{glose}
\end{entrée}

\begin{entrée}{mõ-phòò}{}{ⓔmõ-phòò}
\région{GOs}
(\domainesémantique{Types de maison, architecture de la maison})
\classe{nom}
\begin{glose}
\pfra{toilettes ; cabinet}
\end{glose}
\end{entrée}

\begin{entrée}{mõ-pwal}{}{ⓔmõ-pwal}
\région{BO}
\classe{nom}
\newline
\sens{1}
(\domainesémantique{Instruments})
\begin{glose}
\pfra{parapluie}
\end{glose}
\newline
\sens{2}
(\domainesémantique{Noms des plantes})
\begin{glose}
\pfra{champignon}
\end{glose}
\end{entrée}

\begin{entrée}{mõ-phwayuu}{}{ⓔmõ-phwayuu}
\région{GOs}
(\domainesémantique{Types de maison, architecture de la maison})
\classe{nom}
\begin{glose}
\pfra{maison où se retirent les femmes (pendant les règles) (lit. où l'on se cache)}
\end{glose}
\end{entrée}

\begin{entrée}{mora}{}{ⓔmora}
\région{BO PA}
(\domainesémantique{Santé, maladie})
\classe{v}
\begin{glose}
\pfra{épuisé ; éreinté}
\end{glose}
\end{entrée}

\begin{entrée}{môre}{}{ⓔmôre}
\formephonétique{mõɽe}
\région{GOs}
(\domainesémantique{Discours, échanges verbaux})
\classe{v}
\begin{glose}
\pfra{parler du nez}
\end{glose}
\end{entrée}

\begin{entrée}{môtra ênõ}{}{ⓔmôtra ênõ}
\formephonétique{mõʈa ẽɳɔ̃}
\région{GOs}
\variante{%
möra ènõ
\région{PA}}
(\domainesémantique{Parenté})
\classe{nom}
\begin{glose}
\pfra{benjamin}
\end{glose}
\end{entrée}

\begin{entrée}{mõ-tri}{}{ⓔmõ-tri}
\formephonétique{mõʈi}
\région{GOs}
(\domainesémantique{Ustensiles})
\classe{nom}
\begin{glose}
\pfra{bouilloire (lit. contenant-thé)}
\end{glose}
\end{entrée}

\begin{entrée}{motrò}{}{ⓔmotrò}
\région{GOs}
(\domainesémantique{Marées})
\classe{nom}
\begin{glose}
\pfra{marée basse du soir}
\end{glose}
\end{entrée}

\begin{entrée}{mõû-}{}{ⓔmõû-}
\région{GOs}
\variante{%
mõû-, maû
\région{PA BO}}
(\domainesémantique{Parenté})
\classe{nom}
\begin{glose}
\pfra{épouse ; soeur de l'épouse ; épouse du frère}
\end{glose}
\newline
\begin{exemple}
\région{GO}
\textbf{\pnua{la mõû-je}}
\pfra{elles sont ses épouses}
\end{exemple}
\newline
\begin{exemple}
\région{GO}
\textbf{\pnua{mõû-nu}}
\pfra{mon épouse; ma belle-soeur (épouse du frère)}
\end{exemple}
\newline
\begin{exemple}
\région{PA BO}
\textbf{\pnua{mõû-ny}}
\pfra{mon épouse}
\end{exemple}
\end{entrée}

\begin{entrée}{mõ-vhaa}{}{ⓔmõ-vhaa}
\région{GOs}
(\domainesémantique{Noms locatifs})
\classe{nom}
\begin{glose}
\pfra{lieu de discussion}
\end{glose}
\end{entrée}

\begin{entrée}{mõ-wae}{}{ⓔmõ-wae}
\région{WEM BO}
(\domainesémantique{Types de maison, architecture de la maison})
\classe{nom}
\begin{glose}
\pfra{maison à toit à deux pentes}
\end{glose}
\end{entrée}

\begin{entrée}{mõ-we}{}{ⓔmõ-we}
\région{GOs}
\classe{nom}
\newline
\sens{1}
(\domainesémantique{Instruments})
\begin{glose}
\pfra{contenant à liquide ; calebasse}
\end{glose}
\newline
\sens{2}
(\domainesémantique{Noms des plantes})
\begin{glose}
\pfra{lys d'eau [BO]}
\end{glose}
\end{entrée}

\begin{entrée}{mõxõ}{}{ⓔmõxõ}
\région{GOs}
\variante{%
mòòxòm
\région{BO}}
(\domainesémantique{Verbes d'action (en général)})
\classe{v}
\begin{glose}
\pfra{noyer (se)}
\end{glose}
\end{entrée}

\begin{entrée}{mõ-yai}{}{ⓔmõ-yai}
\région{GOs WEH WEM PA}
(\domainesémantique{Feu : objets et actions liés au feu})
\classe{nom}
\begin{glose}
\pfra{allumettes ; boîte d'allumettes (lit. boîte à feu)}
\end{glose}
\end{entrée}

\begin{entrée}{mòza}{}{ⓔmòza}
\région{GOs}
\variante{%
mora
\région{BO PA}}
(\domainesémantique{Caractéristiques et propriétés des personnes})
\classe{v}
\begin{glose}
\pfra{las ; fatigué ; en avoir assez}
\end{glose}
\end{entrée}

\begin{entrée}{mozi}{}{ⓔmozi}
\région{GOs}
\variante{%
mhõril
\région{PA}}
(\domainesémantique{Fonctions naturelles humaines})
\classe{v}
\begin{glose}
\pfra{pleurnicher (bébé) ; sangloter}
\end{glose}
\end{entrée}

\begin{entrée}{mu}{}{ⓔmu}
\région{GOs}
\variante{%
mun
\région{PA BO}}
\classe{v}
\newline
\sens{1}
(\domainesémantique{Préfixes et verbes de position})
\begin{glose}
\pfra{derrière (être)}
\end{glose}
\newline
\begin{sous-entrée}{yuu mu}{ⓔmuⓢ1ⓝyuu mu}
\région{GO}
\begin{glose}
\pfra{rester derrière, en arrière}
\end{glose}
\end{sous-entrée}
\newline
\sens{2}
(\domainesémantique{Localisation})
\begin{glose}
\pfra{après ; ensuite}
\end{glose}
\newline
\begin{exemple}
\région{GO}
\textbf{\pnua{mu nai jo}}
\pfra{derrière toi}
\end{exemple}
\newline
\begin{exemple}
\région{PA GO}
\textbf{\pnua{mun i yu}}
\pfra{derrière toi}
\end{exemple}
\newline
\begin{exemple}
\région{GO}
\textbf{\pnua{ge mu nye loto}}
\pfra{la voiture est derrière}
\end{exemple}
\newline
\begin{exemple}
\région{PA}
\textbf{\pnua{ge mun nye loto}}
\pfra{la voiture est derrière}
\end{exemple}
\newline
\relationsémantique{Cf.}{\lien{}{kai-nu}}
\glosecourte{juste dans mon dos}
\end{entrée}

\begin{entrée}{mû}{1}{ⓔmûⓗ1}
\région{GO}
\variante{%
muu
\région{PA BO}, 
muuc
\région{BO}}
(\domainesémantique{Processus liés aux plantes})
\classe{v ; n}
\begin{glose}
\pfra{fleur ; fleurir}
\end{glose}
\newline
\begin{exemple}
\textbf{\pnua{e mûû mûû-cee}}
\pfra{la fleur fleurit}
\end{exemple}
\newline
\begin{sous-entrée}{mûû-cee}{ⓔmûⓗ1ⓝmûû-cee}
\région{GO PA BO}
\begin{glose}
\pfra{fleur}
\end{glose}
\end{sous-entrée}
\end{entrée}

\begin{entrée}{mû}{2}{ⓔmûⓗ2}
\région{GOs BO}
\variante{%
mûû
\région{BO}}
\classe{v}
\newline
\sens{1}
(\domainesémantique{Fonctions naturelles humaines})
\begin{glose}
\pfra{ronfler [BO]}
\end{glose}
\newline
\sens{2}
(\domainesémantique{Sons, bruits})
\begin{glose}
\pfra{bourdonner ; faire un bruit de bourdon}
\end{glose}
\end{entrée}

\begin{entrée}{mû-cee}{}{ⓔmû-cee}
\région{GOs}
(\domainesémantique{Parties de plantes})
\classe{nom}
\begin{glose}
\pfra{fleur}
\end{glose}
\end{entrée}

\begin{entrée}{mû-cee-dròò}{}{ⓔmû-cee-dròò}
\région{GOs}
(\domainesémantique{Noms des plantes})
\classe{nom}
\begin{glose}
\pfra{nom des plantes dont les feuilles sont multicolores (comme les crotons)}
\end{glose}
\begin{glose}
\pfra{croton (lit. fleur-arbre-feuille)}
\end{glose}
\end{entrée}

\begin{entrée}{mû-chaamwa}{}{ⓔmû-chaamwa}
\région{GOs}
(\domainesémantique{Parties de plantes})
\classe{nom}
\begin{glose}
\pfra{inflorescence de bananier}
\end{glose}
\end{entrée}

\begin{entrée}{mudim}{}{ⓔmudim}
\région{BO}
(\domainesémantique{Mammifères marins})
\classe{nom}
\begin{glose}
\pfra{dugong [Corne]}
\end{glose}
\newline
\note{non vérifié}{général}{}
\end{entrée}

\begin{entrée}{mudra}{}{ⓔmudra}
\région{GOs WEM}
\variante{%
muda, môda
\région{BO [BM]}}
(\domainesémantique{Verbes d'action (en général)})
\classe{v.stat.}
\begin{glose}
\pfra{déchiré (être) ; cassé}
\end{glose}
\newline
\begin{exemple}
\région{BO}
\textbf{\pnua{i muda}}
\pfra{c'est déchiré}
\end{exemple}
\newline
\note{mudre, mude (v.t.)}{grammaire}{casser qqch}
\end{entrée}

\begin{entrée}{mudree}{}{ⓔmudree}
\région{GOs}
\variante{%
mudee, môdee
\région{BO [BM]}}
(\domainesémantique{Mouvements ou actions faits avec le corps, les bras, les mains, les pieds})
\classe{v}
\begin{glose}
\pfra{casser ; rompre (corde) ; déchirer}
\end{glose}
\newline
\begin{sous-entrée}{hû-mudree}{ⓔmudreeⓝhû-mudree}
\begin{glose}
\pfra{déchirer avec les dents}
\end{glose}
\newline
\note{mudra}{grammaire}{déchiré (statif)}
\end{sous-entrée}
\end{entrée}

\begin{entrée}{mudro}{}{ⓔmudro}
\région{GOs}
\variante{%
mudrã
\région{GO(s)}, 
mudo
\région{PA}, 
muda, mudo
\région{BO}}
(\domainesémantique{Description des objets, formes, consistance, taille})
\classe{v}
\begin{glose}
\pfra{vieux ; usé (linge)}
\end{glose}
\begin{glose}
\pfra{haillons ; loques}
\end{glose}
\begin{glose}
\pfra{délabré}
\end{glose}
\newline
\begin{sous-entrée}{mudro mwa}{ⓔmudroⓝmudro mwa}
\région{GO}
\begin{glose}
\pfra{maison délabrée}
\end{glose}
\newline
\begin{exemple}
\région{PA}
\textbf{\pnua{mudo hõbwòn}}
\pfra{vêtements usés}
\end{exemple}
\newline
\begin{exemple}
\région{BO}
\textbf{\pnua{mudo-n}}
\pfra{c'est vieux, usé}
\end{exemple}
\end{sous-entrée}
\end{entrée}

\begin{entrée}{muga}{}{ⓔmuga}
\région{GOs BO PA}
(\domainesémantique{Fonctions naturelles humaines})
\classe{v ; n}
\begin{glose}
\pfra{vomir ; vomissure}
\end{glose}
\newline
\étymologie{
\langue{POc}
\étymon{*muta(q)}
\glosecourte{vomir}}
\end{entrée}

\begin{entrée}{mugo}{}{ⓔmugo}
\région{GOs BO}
(\domainesémantique{Bananiers et bananes})
\classe{nom}
\begin{glose}
\pfra{banane ; bananier de la chefferie (on ne peut que la bouillir, il est interdit de la griller)}
\end{glose}
\newline
\begin{sous-entrée}{mugo mii}{ⓔmugoⓝmugo mii}
\begin{glose}
\pfra{banane mure}
\end{glose}
\end{sous-entrée}
\newline
\begin{sous-entrée}{pò-muge ni kò-ny}{ⓔmugoⓝpò-muge ni kò-ny}
\région{PA}
\begin{glose}
\pfra{mon mollet (lit. le fruit banane de ma jambe)}
\end{glose}
\end{sous-entrée}
\end{entrée}

\begin{entrée}{muna-le}{}{ⓔmuna-le}
\région{GOs PA}
(\domainesémantique{Conjonction})
\classe{ADV}
\begin{glose}
\pfra{ensuite ; après}
\end{glose}
\end{entrée}

\begin{entrée}{munõ}{}{ⓔmunõ}
\formephonétique{muɳɔ̃}
\région{GOs}
\variante{%
muunõ
\région{BO}}
(\domainesémantique{Actions liées aux éléments (liquide, fumée)})
\classe{v}
\begin{glose}
\pfra{arroser (fleurs)}
\end{glose}
\begin{glose}
\pfra{éteindre (feu)}
\end{glose}
\newline
\begin{sous-entrée}{munõ yai}{ⓔmunõⓝmunõ yai}
\begin{glose}
\pfra{éteindre le feu}
\end{glose}
\end{sous-entrée}
\end{entrée}

\begin{entrée}{mura}{}{ⓔmura}
\région{PA BO}
\variante{%
murò
\région{BO}, 
mun
\région{PA}}
(\domainesémantique{Localisation})
\classe{ADV.TEMPS}
\begin{glose}
\pfra{après ; derrière}
\end{glose}
\newline
\begin{exemple}
\région{PA}
\textbf{\pnua{mura hovwo}}
\pfra{après manger}
\end{exemple}
\newline
\begin{exemple}
\région{PA}
\textbf{\pnua{mura na nu hovwo}}
\pfra{après que j'ai mangé}
\end{exemple}
\newline
\begin{exemple}
\région{BO}
\textbf{\pnua{murò hovo}}
\pfra{après manger}
\end{exemple}
\newline
\begin{exemple}
\région{BO}
\textbf{\pnua{murò-n}}
\pfra{après lui}
\end{exemple}
\newline
\relationsémantique{Ant.}{\lien{}{hêbu ; hêbun}}
\glosecourte{avant, devant}
\newline
\étymologie{
\langue{POc}
\étymon{*muri-}
\glosecourte{rear, stern}
\auteur{Blust}}
\end{entrée}

\begin{entrée}{murae}{}{ⓔmurae}
\région{GOs BO}
(\domainesémantique{Parenté})
\classe{nom}
\begin{glose}
\pfra{benjamin}
\end{glose}
\end{entrée}

\begin{entrée}{mura-hovwo}{}{ⓔmura-hovwo}
\région{PA}
(\domainesémantique{Découpage du temps})
\classe{nom}
\begin{glose}
\pfra{après-midi}
\end{glose}
\end{entrée}

\begin{entrée}{mura-wõ}{}{ⓔmura-wõ}
\région{GOs}
(\domainesémantique{Navigation})
\classe{nom}
\begin{glose}
\pfra{poupe}
\end{glose}
\newline
\relationsémantique{Cf.}{\lien{}{pòbwinõ wõ}}
\glosecourte{poupe}
\newline
\relationsémantique{Cf.}{\lien{ⓔmè-wõ}{mè-wõ}}
\glosecourte{proue}
\newline
\étymologie{
\langue{POc}
\étymon{*muri-}
\glosecourte{rear, stern}
\auteur{Blust}}
\end{entrée}

\begin{entrée}{murò}{}{ⓔmurò}
\région{GOs}
(\domainesémantique{Vêtements, parure})
\classe{nom}
\begin{glose}
\pfra{couverture (pour dormir)}
\end{glose}
\newline
\begin{exemple}
\textbf{\pnua{muròò-nu}}
\pfra{ma couverture}
\end{exemple}
\end{entrée}

\begin{entrée}{muswa}{}{ⓔmuswa}
\région{GOs}
(\domainesémantique{Vêtements, parure})
\classe{nom}
\begin{glose}
\pfra{mouchoir}
\end{glose}
\newline
\emprunt{mouchoir (FR)}
\end{entrée}

\begin{entrée}{muzi}{}{ⓔmuzi}
\région{GOs}
\variante{%
mulin
\région{PA}}
(\domainesémantique{Objets coutumiers})
\classe{nom}
\begin{glose}
\pfra{plumet de monnaie}
\end{glose}
\end{entrée}

\newpage

\lettrine{mh}\begin{entrée}{mhã}{1}{ⓔmhãⓗ1}
\région{GOs}
\variante{%
mhãm
\région{BO}}
(\domainesémantique{Description des objets, formes, consistance, taille})
\classe{v}
\begin{glose}
\pfra{humide}
\end{glose}
\newline
\begin{exemple}
\région{GO}
\textbf{\pnua{mhã ce}}
\pfra{le bois est humide}
\end{exemple}
\newline
\begin{exemple}
\région{BO}
\textbf{\pnua{i mhãm a pên}}
\pfra{le pain est humide [BM]}
\end{exemple}
\end{entrée}

\begin{entrée}{mhã}{2}{ⓔmhãⓗ2}
\région{GOs BO}
\variante{%
mhãng
\région{PA}}
(\domainesémantique{Types de maison, architecture de la maison})
\newline
\sens{1}
\classe{nom}
\begin{glose}
\pfra{ligature ; liane (pour toiture)}
\end{glose}
\begin{glose}
\pfra{attache de paille (toit)}
\end{glose}
\newline
\begin{exemple}
\textbf{\pnua{mhã-wa mwa}}
\pfra{ligature pour la maison}
\end{exemple}
\newline
\sens{2}
\classe{v}
\begin{glose}
\pfra{attacher (la paille aux chevrons)}
\end{glose}
\end{entrée}

\begin{entrée}{mhãã}{1}{ⓔmhããⓗ1}
\région{PA WEM BO}
\variante{%
maak, mheek
\région{BO}}
(\domainesémantique{Arbre})
\classe{nom}
\begin{glose}
\pfra{gaïac}
\end{glose}
\nomscientifique{Acacia spirorbis Labil.}
\end{entrée}

\begin{entrée}{mhãã}{2}{ⓔmhããⓗ2}
\région{GOs}
\variante{%
mhããng
\région{BO}}
(\domainesémantique{Fonctions naturelles humaines})
\classe{v}
\begin{glose}
\pfra{somnambule ; agiter (s') en dormant}
\end{glose}
\begin{glose}
\pfra{faire un cauchemar [BO]}
\end{glose}
\newline
\begin{exemple}
\textbf{\pnua{e mhãã}}
\pfra{il s'agiter en dormant}
\end{exemple}
\end{entrée}

\begin{entrée}{mhãã}{3}{ⓔmhããⓗ3}
\formephonétique{ṃʰɛ̃ː}
\région{GOs PA BO}
(\domainesémantique{Quantificateurs})
\classe{INTENS ; QNT}
\begin{glose}
\pfra{plus ; beaucoup}
\end{glose}
\begin{glose}
\pfra{très ; trop}
\end{glose}
\newline
\begin{exemple}
\région{PA}
\textbf{\pnua{i mhã geen}}
\pfra{c'est très sale}
\end{exemple}
\newline
\begin{exemple}
\région{GO}
\textbf{\pnua{e mhã mii, pu, baa, kari}}
\pfra{c'est très rouge, bleu, noir, jaune (objet)}
\end{exemple}
\newline
\begin{exemple}
\région{BO}
\textbf{\pnua{i mhã kole pwal}}
\pfra{c'est très sale}
\end{exemple}
\newline
\begin{exemple}
\région{BO}
\textbf{\pnua{i mhã hovo nai inu}}
\pfra{il mange plus que moi}
\end{exemple}
\newline
\begin{exemple}
\région{GO}
\textbf{\pnua{la mhã haivwö}}
\pfra{ils sont trop/très nombreux}
\end{exemple}
\newline
\begin{exemple}
\région{GO}
\textbf{\pnua{e mhã gi}}
\pfra{il pleure fort}
\end{exemple}
\newline
\relationsémantique{Cf.}{\lien{}{e ci gi}}
\glosecourte{il pleure vraiment}
\newline
\relationsémantique{Cf.}{\lien{}{pa, para}}
\glosecourte{utilisé pour les animés}
\end{entrée}

\begin{entrée}{mhaaloo}{}{ⓔmhaaloo}
\région{GOs}
(\domainesémantique{Eau})
\classe{nom}
\begin{glose}
\pfra{marais salant ; marais ; terrain marécageux}
\end{glose}
\newline
\begin{exemple}
\textbf{\pnua{bwa mhaaloo}}
\pfra{dans un marais}
\end{exemple}
\end{entrée}

\begin{entrée}{mhããm}{}{ⓔmhããm}
\région{BO}
(\domainesémantique{Eau})
\classe{nom}
\begin{glose}
\pfra{source}
\end{glose}
\newline
\note{non vérifié}{général}{}
\end{entrée}

\begin{entrée}{mhããni}{}{ⓔmhããni}
\région{PA BO}
(\domainesémantique{Aliments, alimentation})
\classe{v}
\begin{glose}
\pfra{mâcher ; mastiquer (des fibres de magnania par ex.)}
\end{glose}
\newline
\étymologie{
\langue{POc}
\étymon{*mama}
\glosecourte{mâcher}}
\end{entrée}

\begin{entrée}{mhã ẽnõ}{}{ⓔmhã ẽnõ}
\formephonétique{mʰɛ̃ ɛ̃ɳɔ̃}
\région{GOs}
(\domainesémantique{Comparaison})
\classe{COMPAR}
\begin{glose}
\pfra{plus jeune}
\end{glose}
\newline
\begin{exemple}
\région{PA}
\textbf{\pnua{aba-ny mha ẽnõ}}
\pfra{petit-frère, petite-soeur}
\end{exemple}
\newline
\begin{exemple}
\région{GO}
\textbf{\pnua{abaa-nu xa mha ẽnõ}}
\pfra{mon plus jeune petit-frère, petite-soeur}
\end{exemple}
\end{entrée}

\begin{entrée}{mhãi-}{1}{ⓔmhãi-ⓗ1}
\région{GOs PA BO}
\newline
\groupe{A}
(\domainesémantique{Quantificateurs})
\classe{nom}
\begin{glose}
\pfra{morceau (de viande, igname coupée) ; part ; fraction}
\end{glose}
\newline
\begin{exemple}
\région{PA}
\textbf{\pnua{mhãi-ny}}
\pfra{mon morceau}
\end{exemple}
\newline
\begin{exemple}
\région{PA}
\textbf{\pnua{mhãi ti/ri ?}}
\pfra{c'est la part de qui ?}
\end{exemple}
\newline
\begin{exemple}
\région{GO}
\textbf{\pnua{mwêêno mhãi vwo nu kibao-je !}}
\pfra{c'était moins une que je le touche}
\end{exemple}
\newline
\groupe{B}
(\domainesémantique{Préfixes classificateurs numériques})
\classe{CLF.NUM}
\begin{glose}
\pfra{morceau ; part}
\end{glose}
\newline
\begin{exemple}
\région{PA}
\textbf{\pnua{mhãi-xe, mhãi-ru}}
\pfra{1, 2 morceaux}
\end{exemple}
\newline
\begin{exemple}
\région{PA}
\textbf{\pnua{mhãi-xe mhava kui}}
\pfra{un morceau d'igname}
\end{exemple}
\end{entrée}

\begin{entrée}{mhãi-}{2}{ⓔmhãi-ⓗ2}
\région{PA}
(\domainesémantique{Quantificateurs})
\classe{nom}
\begin{glose}
\pfra{morceau}
\end{glose}
\newline
\begin{exemple}
\textbf{\pnua{mhãi ti/ri ? - mhãi-n}}
\pfra{un morceau pour qui ? - c'est son morceau}
\end{exemple}
\end{entrée}

\begin{entrée}{mha minõ}{}{ⓔmha minõ}
\région{GOs}
(\domainesémantique{Préparation des aliments; modes de préparation et de cuisson})
\classe{v.stat.}
\begin{glose}
\pfra{trop cuit}
\end{glose}
\end{entrée}

\begin{entrée}{mha-mhwããnu}{}{ⓔmha-mhwããnu}
\région{GOs}
(\domainesémantique{Fonctions naturelles humaines})
\classe{v}
\begin{glose}
\pfra{menstruations (avoir ses)}
\end{glose}
\newline
\begin{sous-entrée}{tròòli mha-mhwããnu}{ⓔmha-mhwããnuⓝtròòli mha-mhwããnu}
\begin{glose}
\pfra{avoir ses menstruations}
\end{glose}
\end{sous-entrée}
\end{entrée}

\begin{entrée}{mharii}{}{ⓔmharii}
\région{BO}
(\domainesémantique{Pêche})
\classe{v}
\begin{glose}
\pfra{chasser les poissons vers le filet [Corne]}
\end{glose}
\newline
\note{non vérifié}{général}{}
\end{entrée}

\begin{entrée}{mhava}{}{ⓔmhava}
\région{GOs PA}
\variante{%
mhava-n
\région{PA BO}}
(\domainesémantique{Quantificateurs})
\classe{nom}
\begin{glose}
\pfra{morceau ; bout de qqch}
\end{glose}
\newline
\begin{sous-entrée}{mhava gò}{ⓔmhavaⓝmhava gò}
\begin{glose}
\pfra{un bout de bambou}
\end{glose}
\end{sous-entrée}
\newline
\begin{sous-entrée}{mhava kui}{ⓔmhavaⓝmhava kui}
\région{PA}
\begin{glose}
\pfra{un morceau d'igname}
\end{glose}
\newline
\begin{exemple}
\région{PA}
\textbf{\pnua{mhava-n}}
\pfra{son extrémité}
\end{exemple}
\end{sous-entrée}
\end{entrée}

\begin{entrée}{mhava mhwããnu}{}{ⓔmhava mhwããnu}
\région{GOs BO}
(\domainesémantique{Astres})
\classe{nom}
\begin{glose}
\pfra{premier quartier de lune}
\end{glose}
\end{entrée}

\begin{entrée}{mhavwa}{}{ⓔmhavwa}
\formephonétique{mʰaβa}
\région{GOs}
\variante{%
mhava
\région{PA BO}}
(\domainesémantique{Quantificateurs})
\classe{n.QNT}
\begin{glose}
\pfra{morceau}
\end{glose}
\newline
\begin{exemple}
\région{GO}
\textbf{\pnua{e na-e pòi-je xa mhavwa lai}}
\pfra{elle donne à ses enfants un peu de riz}
\end{exemple}
\newline
\begin{exemple}
\région{PA}
\textbf{\pnua{mhava kui}}
\pfra{un morceau d'igname}
\end{exemple}
\end{entrée}

\begin{entrée}{mhã whama}{}{ⓔmhã whama}
\région{GOs}
(\domainesémantique{Comparaison})
\classe{COMPAR}
\begin{glose}
\pfra{plus agé ; plus vieux}
\end{glose}
\newline
\begin{exemple}
\textbf{\pnua{e mhã whama}}
\pfra{il est plus grand}
\end{exemple}
\end{entrée}

\begin{entrée}{mhayu}{}{ⓔmhayu}
\région{BO}
(\domainesémantique{Fonctions intellectuelles})
\classe{nom}
\begin{glose}
\pfra{souvenir ; héritage des vieux [BM]}
\end{glose}
\newline
\note{non verifié}{général}{}
\end{entrée}

\begin{entrée}{mhaza}{}{ⓔmhaza}
\région{GOs}
\variante{%
maza
, 
mhara
\région{PA WEM}}
\classe{INCH}
\newline
\sens{1}
(\domainesémantique{Aspect})
\begin{glose}
\pfra{commencer à ; se mettre à ; être sur le point de}
\end{glose}
\newline
\begin{exemple}
\région{GO}
\textbf{\pnua{nu mhaza hovwo}}
\pfra{je commence tout juste à manger}
\end{exemple}
\newline
\begin{exemple}
\région{GO}
\textbf{\pnua{mi za u mhaza a khiila hãgana vwö mi baani}}
\pfra{mettons-nous à leur recherche maintenant pour les tuer}
\end{exemple}
\newline
\begin{exemple}
\région{WEM}
\textbf{\pnua{la mhara buròm ẽnõ}}
\pfra{les enfants se sont mis à se baigner, les enfants viennent de se baigner}
\end{exemple}
\newline
\begin{exemple}
\région{PA}
\textbf{\pnua{i ra gaa mhara a}}
\pfra{il vient juste de partir}
\end{exemple}
\newline
\begin{exemple}
\région{BO}
\textbf{\pnua{nu mhara õgin a hovo}}
\pfra{je viens de finir de manger}
\end{exemple}
\newline
\begin{exemple}
\région{BO}
\textbf{\pnua{i mhara a-du we}}
\pfra{la marée commence juste à descendre}
\end{exemple}
\newline
\sens{2}
(\domainesémantique{Aspect})
\begin{glose}
\pfra{être la première fois que}
\end{glose}
\newline
\begin{exemple}
\textbf{\pnua{za mhaza nòòli xo je nye èmwê}}
\pfra{c'est la 1° fois qu'elle voit unhomme (elle n'en a jamais vu avant)}
\end{exemple}
\newline
\begin{exemple}
\région{GO}
\textbf{\pnua{po nye za mhaza nõõli xo je nye êmwê}}
\pfra{parce que c'est la première fois qu'elle voit un homme}
\end{exemple}
\newline
\sens{3}
(\domainesémantique{Aspect})
\begin{glose}
\pfra{venir tout juste de}
\end{glose}
\newline
\begin{exemple}
\région{GO}
\textbf{\pnua{e mhaza a nye !}}
\pfra{il vient juste de partir !}
\end{exemple}
\newline
\begin{exemple}
\région{GO}
\textbf{\pnua{e mhaza uca !}}
\pfra{il vient juste d'arriver !}
\end{exemple}
\newline
\begin{exemple}
\région{GO}
\textbf{\pnua{e mhaza pwe}}
\pfra{il vient de naître; nouveau-né (il n'y a pas d'autre mot pour nouveau-né)}
\end{exemple}
\newline
\begin{exemple}
\région{PA}
\textbf{\pnua{i mhara a}}
\pfra{il vient juste de partir}
\end{exemple}
\newline
\begin{exemple}
\région{PA}
\textbf{\pnua{mhara tèèn}}
\pfra{le jour se lève}
\end{exemple}
\newline
\begin{exemple}
\région{PA}
\textbf{\pnua{i ra gaa mhara pwal}}
\pfra{il vient juste de se mettre à pleuvoir}
\end{exemple}
\newline
\sens{4}
(\domainesémantique{Aspect})
\begin{glose}
\pfra{juste au moment où}
\end{glose}
\newline
\begin{exemple}
\région{GO}
\textbf{\pnua{mhaza u ka thoe hi-je}}
\pfra{juste au moment où elle se penche (u) et en même temps tend son bras}
\end{exemple}
\newline
\sens{5}
(\domainesémantique{Aspect})
\begin{glose}
\pfra{en train de [BO]}
\end{glose}
\newline
\sens{6}
(\domainesémantique{Aspect})
\begin{glose}
\pfra{enfin}
\end{glose}
\newline
\begin{exemple}
\région{GO}
\textbf{\pnua{nu mhaza nõõ-je}}
\pfra{je vais enfin la voir}
\end{exemple}
\newline
\begin{exemple}
\région{PA}
\textbf{\pnua{nu mhara hine}}
\pfra{je sais maintenant}
\end{exemple}
\end{entrée}

\begin{entrée}{mhe}{}{ⓔmhe}
\région{GOs}
(\domainesémantique{Arbre})
\classe{nom}
\begin{glose}
\pfra{tamanou (faux)}
\end{glose}
\nomscientifique{Gessois montana}
\end{entrée}

\begin{entrée}{mhè}{}{ⓔmhè}
\région{GOs BO}
(\domainesémantique{Verbes de déplacement et moyens de déplacement})
\classe{v ; n}
\begin{glose}
\pfra{promener ; promenade}
\end{glose}
\newline
\begin{exemple}
\région{BO}
\textbf{\pnua{mhèè-n}}
\pfra{sa promenade}
\end{exemple}
\end{entrée}

\begin{entrée}{mhedrame}{}{ⓔmhedrame}
\région{GOs}
\variante{%
mhedam
\région{PA}}
(\domainesémantique{Vents})
\classe{n ; LOC}
\begin{glose}
\pfra{vent du nord ; nord}
\end{glose}
\end{entrée}

\begin{entrée}{mhenõ}{}{ⓔmhenõ}
\formephonétique{mʰeɳɔ̃}
\région{GOs}
(\domainesémantique{Verbes de déplacement et moyens de déplacement})
\classe{nom}
\begin{glose}
\pfra{voyage en groupe ; déplacement en groupe}
\end{glose}
\newline
\begin{sous-entrée}{koe xa mhenõ}{ⓔmhenõⓝkoe xa mhenõ}
\begin{glose}
\pfra{préparer un voyage}
\end{glose}
\end{sous-entrée}
\newline
\begin{sous-entrée}{thu-menõ}{ⓔmhenõⓝthu-menõ}
\begin{glose}
\pfra{voyager}
\end{glose}
\end{sous-entrée}
\end{entrée}

\begin{entrée}{mhenõ-a-pe-aze dè}{}{ⓔmhenõ-a-pe-aze dè}
\région{GOs}
\région{BO PA (Dubois)}
\variante{%
menõ-aveale-dèn
}
(\domainesémantique{Topographie})
\classe{nom}
\begin{glose}
\pfra{carrefour divergent}
\end{glose}
\end{entrée}

\begin{entrée}{mhe-nõbo}{}{ⓔmhe-nõbo}
\formephonétique{mʰe-'ɳɔ̃bo}
\région{GOs}
\variante{%
mhe-nõvwo
\formephonétique{mʰe-'ɳɔ̃βo}
\région{GOs}}
(\domainesémantique{Santé, maladie})
\classe{nom}
\begin{glose}
\pfra{cicatrice}
\end{glose}
\begin{glose}
\pfra{blessure}
\end{glose}
\newline
\begin{exemple}
\textbf{\pnua{u mha mhe-nõvwo}}
\pfra{c'est cicatrisé (lit. la blessure est morte)}
\end{exemple}
\end{entrée}

\begin{entrée}{mhenõ-içu}{}{ⓔmhenõ-içu}
\formephonétique{'mʰeɳɔ̃-'iʒu}
\région{GOs}
\variante{%
mhenõ-iyu
\région{WEM BO}, 
mõ-iyu
\région{BO}}
(\domainesémantique{Noms locatifs})
\classe{nom}
\begin{glose}
\pfra{emplacement de la vente}
\end{glose}
\begin{glose}
\pfra{magasin ; boutique}
\end{glose}
\end{entrée}

\begin{entrée}{mhenõ-kole ja}{}{ⓔmhenõ-kole ja}
\formephonétique{'mʰeɳɔ̃-ko'le-ja}
\région{GOs}
\variante{%
mhenõô-kole ja
\région{PA}}
(\domainesémantique{Types de maison, architecture de la maison})
\classe{nom}
\begin{glose}
\pfra{dépotoir}
\end{glose}
\end{entrée}

\begin{entrée}{mhenõ-mhõ}{}{ⓔmhenõ-mhõ}
\région{GOs}
\variante{%
mhenõ-mhõng
\région{PA BO}}
(\domainesémantique{Noeuds})
\classe{nom}
\begin{glose}
\pfra{noeud (y compris celui des filets)}
\end{glose}
\end{entrée}

\begin{entrée}{mhenõ na pwaawe}{}{ⓔmhenõ na pwaawe}
\formephonétique{mʰe'ɳɔ̃ na 'pwaːwe}
\région{GOs}
(\domainesémantique{Types de maison, architecture de la maison})
\classe{nom}
\begin{glose}
\pfra{étagère ; claie pour fumer}
\end{glose}
\end{entrée}

\begin{entrée}{mhenõõ}{1}{ⓔmhenõõⓗ1}
\région{GOs BO PA}
\variante{%
mènõ
\région{BO}}
\classe{nom}
\newline
\sens{1}
(\domainesémantique{Noms locatifs})
\begin{glose}
\pfra{endroit ; place ; passage (col, gué, passe dans un récif)}
\end{glose}
\newline
\begin{sous-entrée}{mhenõõ-traabwa}{ⓔmhenõõⓗ1ⓢ1ⓝmhenõõ-traabwa}
\begin{glose}
\pfra{lieu où l'on s'assoit}
\end{glose}
\end{sous-entrée}
\newline
\begin{sous-entrée}{mhenõõ-cöö}{ⓔmhenõõⓗ1ⓢ1ⓝmhenõõ-cöö}
\région{GO}
\begin{glose}
\pfra{gué}
\end{glose}
\end{sous-entrée}
\newline
\begin{sous-entrée}{mhenõ-còòl}{ⓔmhenõõⓗ1ⓢ1ⓝmhenõ-còòl}
\région{PA BO}
\begin{glose}
\pfra{gué}
\end{glose}
\end{sous-entrée}
\newline
\begin{sous-entrée}{mhenõõ-köle ja}{ⓔmhenõõⓗ1ⓢ1ⓝmhenõõ-köle ja}
\région{GO}
\begin{glose}
\pfra{dépotoir}
\end{glose}
\end{sous-entrée}
\newline
\sens{2}
(\domainesémantique{Aspect})
\begin{glose}
\pfra{en train de}
\end{glose}
\newline
\begin{exemple}
\région{GO}
\textbf{\pnua{novwö ne ge je ne mhenõ mãni, çe/ça me zoma baa-je}}
\pfra{quand elle sera en train de dormir, on la frappera}
\end{exemple}
\end{entrée}

\begin{entrée}{mhenõõ}{2}{ⓔmhenõõⓗ2}
\région{GOs WEM PA}
\variante{%
mhenõõ
\région{GOs WEM}}
(\domainesémantique{Verbes de déplacement et moyens de déplacement})
\classe{nom}
\begin{glose}
\pfra{trace ; marque}
\end{glose}
\newline
\begin{exemple}
\textbf{\pnua{mhenõ va mhenõ ?}}
\pfra{où vont les traces de pas ?}
\end{exemple}
\newline
\begin{sous-entrée}{mhenõõ-hèlè}{ⓔmhenõõⓗ2ⓝmhenõõ-hèlè}
\région{GO}
\begin{glose}
\pfra{trace du couteau}
\end{glose}
\end{sous-entrée}
\newline
\begin{sous-entrée}{mhenõõ-parô-kuau}{ⓔmhenõõⓗ2ⓝmhenõõ-parô-kuau}
\région{GO}
\begin{glose}
\pfra{trace de dent du chien}
\end{glose}
\end{sous-entrée}
\end{entrée}

\begin{entrée}{mhenõ-paxe we}{}{ⓔmhenõ-paxe we}
\région{PA}
(\domainesémantique{Cultures, techniques, boutures})
\classe{nom}
\begin{glose}
\pfra{vanne de canal}
\end{glose}
\newline
\relationsémantique{Cf.}{\lien{}{paxe [PA]}}
\glosecourte{dévier}
\end{entrée}

\begin{entrée}{mhenõ-pe-ki}{}{ⓔmhenõ-pe-ki}
\région{GOs}
(\domainesémantique{Configuration des objets})
\classe{nom}
\begin{glose}
\pfra{soudure (par ex. des 2 parties du crâne) ; raccord}
\end{glose}
\end{entrée}

\begin{entrée}{mhenõ-pe-vhi de}{}{ⓔmhenõ-pe-vhi de}
\région{GOs}
\variante{%
mhenõ-piga-dèn
\région{BO (Dubois)}}
(\domainesémantique{Topographie})
\classe{nom}
\begin{glose}
\pfra{carrefour convergent ; lieu de rencontre sur un chemin}
\end{glose}
\newline
\note{non vérifié}{général}{}
\end{entrée}

\begin{entrée}{mhenõ-tabwa}{}{ⓔmhenõ-tabwa}
\région{PA}
(\domainesémantique{Objets et meubles de la maison})
\classe{nom}
\begin{glose}
\pfra{chaise [PA] ; lieu où l'on s'assoit [GO]}
\end{glose}
\end{entrée}

\begin{entrée}{mhenõ-tivwo}{}{ⓔmhenõ-tivwo}
\région{GOs}
(\domainesémantique{Objets et meubles de la maison})
\classe{nom}
\begin{glose}
\pfra{tableau}
\end{glose}
\end{entrée}

\begin{entrée}{mhenõ-yu}{}{ⓔmhenõ-yu}
\formephonétique{mʰeɳɔ̃ yuː ; meɳɔ̃ yuː}
\région{GOs}
\variante{%
menõ-yuu
\région{GOs}}
(\domainesémantique{Habitat})
\classe{nom}
\begin{glose}
\pfra{demeure}
\end{glose}
\begin{glose}
\pfra{séjour}
\end{glose}
\begin{glose}
\pfra{lieu de résidence}
\end{glose}
\newline
\begin{exemple}
\région{GO}
\textbf{\pnua{mhenõ-yu-w-a kãgu êgu ma la mã}}
\pfra{la demeure des esprits des morts}
\end{exemple}
\end{entrée}

\begin{entrée}{mhõ}{1}{ⓔmhõⓗ1}
\formephonétique{mʰɔ̃}
\région{GOs PA BO}
(\domainesémantique{Insectes})
\classe{nom}
\begin{glose}
\pfra{fourmi noire (petite)}
\end{glose}
\end{entrée}

\begin{entrée}{mhõ}{2}{ⓔmhõⓗ2}
\région{GOs WE BO}
\variante{%
mõ
\région{GO}, 
bwa mhõ
\région{PA}}
(\domainesémantique{Localisation})
\classe{nom}
\begin{glose}
\pfra{gauche (côté) ; gaucher}
\end{glose}
\newline
\begin{sous-entrée}{kolo-je mõ}{ⓔmhõⓗ2ⓝkolo-je mõ}
\région{GO}
\begin{glose}
\pfra{son côté gauche}
\end{glose}
\newline
\begin{exemple}
\région{GO}
\textbf{\pnua{e mõ}}
\pfra{il est gaucher}
\end{exemple}
\newline
\begin{exemple}
\région{BO}
\textbf{\pnua{yi-ny a mhõ}}
\pfra{ma main gauche}
\end{exemple}
\newline
\relationsémantique{Ant.}{\lien{ⓔhiiⓢ1ⓝgu hi-n}{gu hi-n}}
\glosecourte{droit (côté)}
\end{sous-entrée}
\newline
\étymologie{
\langue{POc}
\étymon{*mauRi}
\glosecourte{gauche}}
\end{entrée}

\begin{entrée}{mhõ}{3}{ⓔmhõⓗ3}
\formephonétique{mʰɔ̃}
\région{GOs}
\variante{%
mhwòl
\région{BO}}
(\domainesémantique{Noeuds})
\classe{v ; n}
\begin{glose}
\pfra{pli ; plisser}
\end{glose}
\begin{glose}
\pfra{noeud}
\end{glose}
\newline
\begin{exemple}
\textbf{\pnua{thu mhõ}}
\pfra{faire des plis}
\end{exemple}
\end{entrée}

\begin{entrée}{mhõdòni}{}{ⓔmhõdòni}
\région{GOs}
(\domainesémantique{Quantificateurs})
\classe{n.QNT}
\begin{glose}
\pfra{ce qui s'ajoute à un tas (de dons coutumiers, mais ne peut constituer un tas complet)}
\end{glose}
\newline
\begin{exemple}
\région{GO}
\textbf{\pnua{mãè-xe xo mhõdòni}}
\pfra{1 tas de 3 ignames et 1 ou 2 en plus}
\end{exemple}
\end{entrée}

\begin{entrée}{mhodro}{}{ⓔmhodro}
\région{GOs}
\variante{%
mhòdo
\région{BO}}
(\domainesémantique{Aliments, alimentation})
\classe{v}
\begin{glose}
\pfra{avarié ; tourné ; sûr}
\end{glose}
\begin{glose}
\pfra{moisi (nourriture)}
\end{glose}
\newline
\begin{exemple}
\textbf{\pnua{e mhodro hovwo}}
\pfra{la nourriture est avariée}
\end{exemple}
\end{entrée}

\begin{entrée}{mhõge}{}{ⓔmhõge}
\formephonétique{mʰɔ̃ŋge}
\région{GO PA BO}
\classe{v ; n}
\newline
\sens{1}
(\domainesémantique{Noeuds})
\begin{glose}
\pfra{nouer ; attacher avec un noeud ; faire un noeud}
\end{glose}
\newline
\begin{exemple}
\région{BO}
\textbf{\pnua{mhòge pwio}}
\pfra{faire un filet}
\end{exemple}
\newline
\sens{2}
(\domainesémantique{Coutumes, dons coutumiers})
\begin{glose}
\pfra{unir (s'), rassembler (se) (contexte coutumier)}
\end{glose}
\end{entrée}

\begin{entrée}{mhômõwe}{}{ⓔmhômõwe}
\formephonétique{mʰõmɔ̃we}
\région{GOs}
(\domainesémantique{Terre})
\classe{nom}
\begin{glose}
\pfra{boue}
\end{glose}
\end{entrée}

\begin{entrée}{mhõril}{}{ⓔmhõril}
\région{PA BO}
(\domainesémantique{Fonctions naturelles humaines})
\classe{v}
\begin{glose}
\pfra{pleurnicher ; sangloter ; hoqueter}
\end{glose}
\end{entrée}

\begin{entrée}{mhôwe}{}{ⓔmhôwe}
\région{GOs}
(\domainesémantique{Mouvements ou actions faits avec le corps, les bras, les mains, les pieds})
\classe{v}
\begin{glose}
\pfra{essorer ; presser (fruit)}
\end{glose}
\begin{glose}
\pfra{tordre}
\end{glose}
\newline
\begin{sous-entrée}{mhôwe nu}{ⓔmhôweⓝmhôwe nu}
\begin{glose}
\pfra{presser pour extraire le lait de coco}
\end{glose}
\end{sous-entrée}
\newline
\étymologie{
\langue{POc}
\étymon{*momos}
\glosecourte{squeeze}
\auteur{Blust}}
\end{entrée}

\begin{entrée}{mhõ-zixe}{}{ⓔmhõ-zixe}
\région{GOs}
(\domainesémantique{Noeuds})
\classe{nom}
\begin{glose}
\pfra{noeud coulant}
\end{glose}
\end{entrée}

\begin{entrée}{mhûûzi}{}{ⓔmhûûzi}
\région{GOs}
(\domainesémantique{Crustacés, crabes})
\classe{nom}
\begin{glose}
\pfra{crabe de palétuvier (plus gros que "ji")}
\end{glose}
\end{entrée}

\newpage

\lettrine{mw}\begin{entrée}{mwa}{}{ⓔmwa}
\région{GOs}
(\domainesémantique{Types de maison, architecture de la maison})
\classe{nom}
\begin{glose}
\pfra{maison}
\end{glose}
\newline
\begin{sous-entrée}{mwa pho}{ⓔmwaⓝmwa pho}
\région{GO}
\begin{glose}
\pfra{la maisonnée (lit. maison-tressage), les femmes et enfants}
\end{glose}
\newline
\begin{exemple}
\région{GO}
\textbf{\pnua{mwa za ? - mwa dili}}
\pfra{quelle sorte de maison ? - une maison en terre}
\end{exemple}
\newline
\begin{exemple}
\région{GO}
\textbf{\pnua{mwa xa whaiya ? - mwa xa tretrabwau}}
\pfra{quelle sorte de maison ? - une maison ronde}
\end{exemple}
\newline
\begin{exemple}
\région{GO}
\textbf{\pnua{mõõ-nu}}
\pfra{ma maison}
\end{exemple}
\newline
\begin{exemple}
\région{BO}
\textbf{\pnua{mwa mõ-hovo}}
\pfra{garde-manger (maison pour nourriture) (store-house for food)}
\end{exemple}
\newline
\begin{exemple}
\textbf{\pnua{mõ-da ? - mõ-pe-rooli - mõ-thia}}
\pfra{une maison pour quoi? qui sert à quoi ? - une maison de réunion, une maison de danse}
\end{exemple}
\newline
\begin{exemple}
\textbf{\pnua{mõ-ti ? - mõ-ãbaa-nu (*mõ-ri)}}
\pfra{la maison de qui? - la maison de mon frère}
\end{exemple}
\end{sous-entrée}
\newline
\begin{sous-entrée}{mwa puco, mwa pujo, mwa-wujo}{ⓔmwaⓝmwa puco, mwa pujo, mwa-wujo}
\région{GO}
\begin{glose}
\pfra{cuisine}
\end{glose}
\end{sous-entrée}
\newline
\begin{sous-entrée}{mwa vèle}{ⓔmwaⓝmwa vèle}
\région{GO}
\begin{glose}
\pfra{garage à bateau}
\end{glose}
\end{sous-entrée}
\newline
\begin{sous-entrée}{mwa-dinyo}{ⓔmwaⓝmwa-dinyo}
\région{BO}
\begin{glose}
\pfra{case des plantations (Corne)}
\end{glose}
\end{sous-entrée}
\newline
\begin{sous-entrée}{mwa-gol}{ⓔmwaⓝmwa-gol}
\région{BO}
\begin{glose}
\pfra{abri de fortune (Dubois)}
\end{glose}
\end{sous-entrée}
\newline
\begin{sous-entrée}{bu mwa}{ⓔmwaⓝbu mwa}
\begin{glose}
\pfra{emplacement de maison}
\end{glose}
\end{sous-entrée}
\newline
\begin{sous-entrée}{ce mwa}{ⓔmwaⓝce mwa}
\begin{glose}
\pfra{solives}
\end{glose}
\end{sous-entrée}
\newline
\begin{sous-entrée}{bwaxeni mwa}{ⓔmwaⓝbwaxeni mwa}
\begin{glose}
\pfra{tertre}
\end{glose}
\end{sous-entrée}
\newline
\begin{sous-entrée}{kica mwa}{ⓔmwaⓝkica mwa}
\begin{glose}
\pfra{paroi de maison}
\end{glose}
\newline
\note{forme déterminée : mõ(õ)}{grammaire}{}
\end{sous-entrée}
\newline
\étymologie{
\langue{POc}
\étymon{*Rumaq}}
\end{entrée}

\begin{entrée}{mwã}{1}{ⓔmwãⓗ1}
\région{GOs}
(\domainesémantique{Conjonction
, Aspect})
\classe{ADV.SEQ (continuatif)}
\begin{glose}
\pfra{alors ; continuer à}
\end{glose}
\newline
\begin{exemple}
\textbf{\pnua{mwã ico !}}
\pfra{à ton tour !}
\end{exemple}
\newline
\relationsémantique{Cf.}{\lien{}{draa ico !}}
\glosecourte{à ton tour !}
\end{entrée}

\begin{entrée}{mwã}{2}{ⓔmwãⓗ2}
\région{GO PA}
\newline
\sens{1}
(\domainesémantique{Verbes de déplacement et moyens de déplacement})
\classe{v}
\begin{glose}
\pfra{revenir}
\end{glose}
\newline
\begin{exemple}
\région{GO}
\textbf{\pnua{li a mwaa=mi}}
\pfra{ils sont revenus}
\end{exemple}
\newline
\sens{2}
(\domainesémantique{Aspect})
\classe{ASP transition, répétition, réversif}
\begin{glose}
\pfra{re- ; à nouveau}
\end{glose}
\newline
\begin{sous-entrée}{a-mwã-mi}{ⓔmwãⓗ2ⓢ2ⓝa-mwã-mi}
\begin{glose}
\pfra{revenir}
\end{glose}
\end{sous-entrée}
\newline
\begin{sous-entrée}{a-mwã-e}{ⓔmwãⓗ2ⓢ2ⓝa-mwã-e}
\begin{glose}
\pfra{repartir}
\end{glose}
\end{sous-entrée}
\newline
\begin{sous-entrée}{a-da mwã}{ⓔmwãⓗ2ⓢ2ⓝa-da mwã}
\begin{glose}
\pfra{remonter}
\end{glose}
\end{sous-entrée}
\newline
\begin{sous-entrée}{na mwã}{ⓔmwãⓗ2ⓢ2ⓝna mwã}
\begin{glose}
\pfra{rendre}
\end{glose}
\end{sous-entrée}
\newline
\sens{3}
(\domainesémantique{Aspect})
\classe{ASP changement d'état(u ... mwã)}
\begin{glose}
\pfra{enfin}
\end{glose}
\newline
\begin{exemple}
\région{PA}
\textbf{\pnua{i u mã mwã}}
\pfra{il est mort}
\end{exemple}
\newline
\begin{exemple}
\région{BO}
\textbf{\pnua{i mwã hangai}}
\pfra{il grossit}
\end{exemple}
\end{entrée}

\begin{entrée}{mwã}{3}{ⓔmwãⓗ3}
\région{GOs}
\variante{%
mhwã
\région{PA}}
(\domainesémantique{Pronoms})
\classe{PRO 1° pers. incl. PL (sujet)}
\begin{glose}
\pfra{nous (incl.)}
\end{glose}
\end{entrée}

\begin{entrée}{mwãã}{}{ⓔmwãã}
\région{GOs}
(\domainesémantique{Relations et interaction sociales})
\classe{nom}
\begin{glose}
\pfra{soutien ; appui}
\end{glose}
\newline
\begin{exemple}
\textbf{\pnua{mhenõ-mwã mani mhenõ-kea}}
\pfra{nos soutiens et nos appuis}
\end{exemple}
\end{entrée}

\begin{entrée}{mwa-araba}{}{ⓔmwa-araba}
\région{GOs WEM}
\variante{%
mwa-alaba (mwa-halapa Corne)
\région{BO}}
(\domainesémantique{Types de maison, architecture de la maison})
\classe{nom}
\begin{glose}
\pfra{maison à toit plat ; maison à toit à deux pentes}
\end{glose}
\end{entrée}

\begin{entrée}{mwããxe}{1}{ⓔmwããxeⓗ1}
\région{GOs PA}
(\domainesémantique{Verbes d'action (en général)})
\classe{v}
\begin{glose}
\pfra{redresser (fer)}
\end{glose}
\newline
\relationsémantique{Cf.}{\lien{}{variante : mwhãnge}}
\end{entrée}

\begin{entrée}{mwããxe}{2}{ⓔmwããxeⓗ2}
\région{BO [BM]}
(\domainesémantique{Verbes d'action (en général)})
\classe{v}
\begin{glose}
\pfra{tordre (du fer)}
\end{glose}
\end{entrée}

\begin{entrée}{mwa-çii}{}{ⓔmwa-çii}
\région{GO}
(\domainesémantique{Préparation des aliments; modes de préparation et de cuisson})
\classe{MODIF}
\begin{glose}
\pfra{peau (cuire avec la)}
\end{glose}
\newline
\begin{sous-entrée}{phai walei mwacii}{ⓔmwa-çiiⓝphai walei mwacii}
\begin{glose}
\pfra{cuire l'igname (sucrée) avec la peau}
\end{glose}
\end{sous-entrée}
\end{entrée}

\begin{entrée}{mwacoa}{}{ⓔmwacoa}
\région{PA}
(\domainesémantique{Ignames})
\classe{nom}
\begin{glose}
\pfra{igname ronde (clone) (Dubois + Charles)}
\end{glose}
\end{entrée}

\begin{entrée}{mwa-draeca}{}{ⓔmwa-draeca}
\région{GOs}
(\domainesémantique{Navigation})
\classe{nom}
\begin{glose}
\pfra{pirogue double}
\end{glose}
\end{entrée}

\begin{entrée}{mwagi}{}{ⓔmwagi}
\région{GOs}
\variante{%
mwagin
\région{PA}}
(\domainesémantique{Oiseaux})
\classe{nom}
\begin{glose}
\pfra{cagou (sorte de)}
\end{glose}
\nomscientifique{Rhynochetos jubatus}
\end{entrée}

\begin{entrée}{mwãgi}{}{ⓔmwãgi}
\région{GOs}
(\domainesémantique{Noms des plantes})
\classe{nom}
\begin{glose}
\pfra{cactus sauvage}
\end{glose}
\end{entrée}

\begin{entrée}{mwa-gol}{}{ⓔmwa-gol}
\région{BO}
\variante{%
mwa-ol
\région{BO [BM]}}
(\domainesémantique{Types de maison, architecture de la maison})
\classe{nom}
\begin{glose}
\pfra{abri de fortune [Dubois, BM]}
\end{glose}
\end{entrée}

\begin{entrée}{mwa-huvo}{}{ⓔmwa-huvo}
\région{GOs}
(\domainesémantique{Types de maison, architecture de la maison})
\classe{nom}
\begin{glose}
\pfra{maison où l'on garde la nourriture}
\end{glose}
\end{entrée}

\begin{entrée}{mwaitri}{}{ⓔmwaitri}
\formephonétique{mwaiʈi}
\région{GO}
(\domainesémantique{Arbre})
\classe{nom}
\begin{glose}
\pfra{mûrier}
\end{glose}
\end{entrée}

\begin{entrée}{mwaje}{}{ⓔmwaje}
\région{GOs PA}
(\domainesémantique{Manière de faire l’action : verbes et adverbes de manière})
\classe{nom}
\begin{glose}
\pfra{manière de faire ; façon de faire ; procédure}
\end{glose}
\end{entrée}

\begin{entrée}{mwajin}{}{ⓔmwajin}
\région{PA}
(\domainesémantique{Temps})
\classe{nom}
\begin{glose}
\pfra{temps}
\end{glose}
\newline
\begin{exemple}
\région{PA}
\textbf{\pnua{pwali mwajin ?}}
\pfra{combien de temps ?}
\end{exemple}
\newline
\begin{exemple}
\région{PA}
\textbf{\pnua{au mwaji-n}}
\pfra{il est en retard}
\end{exemple}
\newline
\begin{exemple}
\région{PA}
\textbf{\pnua{au mwaji-m}}
\pfra{ton retard}
\end{exemple}
\end{entrée}

\begin{entrée}{mwãju}{}{ⓔmwãju}
\formephonétique{mwɛ̃ɲju}
\région{GOs PA BO}
\région{GOs}
\variante{%
mwèju
, 
mwaji
\région{BO}}
\classe{v ; n}
\newline
\sens{1}
(\domainesémantique{Verbes de déplacement et moyens de déplacement})
\begin{glose}
\pfra{retourner ; retourner (s'en)}
\end{glose}
\begin{glose}
\pfra{demi-tour (faire) ; revenir sur ses pas ; retourner (s'en)}
\end{glose}
\begin{glose}
\pfra{revenir sur ses pas}
\end{glose}
\newline
\begin{exemple}
\région{GO}
\textbf{\pnua{i ra u mwãju mwã a-ò}}
\pfra{l'homme s'en retourne}
\end{exemple}
\newline
\begin{exemple}
\région{BO}
\textbf{\pnua{mwaji-a}}
\pfra{votre retour (BM)}
\end{exemple}
\newline
\relationsémantique{Cf.}{\lien{}{pwaa [PA]}}
\glosecourte{retourner (s'en) ; faire demi-tour ; revenir sur ses pas}
\newline
\sens{2}
(\domainesémantique{Coutumes, dons coutumiers})
\begin{glose}
\pfra{contre-don (dans les dons coutumiers)}
\end{glose}
\begin{glose}
\pfra{rendre (la monnaie)}
\end{glose}
\newline
\begin{exemple}
\région{GO}
\textbf{\pnua{mwãju kaja ãgu}}
\pfra{contre-don (lit. don après la personne)}
\end{exemple}
\newline
\begin{exemple}
\région{GO PA}
\textbf{\pnua{mwãju kai}}
\pfra{contre-don}
\end{exemple}
\end{entrée}

\begin{entrée}{mwa-kabu}{}{ⓔmwa-kabu}
\région{GO}
(\domainesémantique{Religion, représentations religieuses})
\classe{nom}
\begin{glose}
\pfra{église ; temple}
\end{glose}
\end{entrée}

\begin{entrée}{mwang}{}{ⓔmwang}
\région{PA BO}
(\domainesémantique{Caractéristiques et propriétés des personnes})
\classe{v.stat.}
\begin{glose}
\pfra{mauvais ; mal}
\end{glose}
\newline
\begin{exemple}
\région{BO}
\textbf{\pnua{i wal mwang}}
\pfra{il chante mal}
\end{exemple}
\end{entrée}

\begin{entrée}{mwani}{}{ⓔmwani}
\formephonétique{mwaɳi}
\région{GOs PA BO}
(\domainesémantique{Richesses, monnaies traditionnelles})
\classe{nom}
\begin{glose}
\pfra{argent}
\end{glose}
\newline
\emprunt{money (GB)}
\end{entrée}

\begin{entrée}{mwa-paa}{}{ⓔmwa-paa}
\région{PA}
(\domainesémantique{Habitat})
\classe{nom}
\begin{glose}
\pfra{abri dans/sous un rocher}
\end{glose}
\end{entrée}

\begin{entrée}{mwa-pe-cinô}{}{ⓔmwa-pe-cinô}
\région{BO}
(\domainesémantique{Types de maison, architecture de la maison})
\classe{nom}
\begin{glose}
\pfra{maison carrée}
\end{glose}
\end{entrée}

\begin{entrée}{mwa-puçò}{}{ⓔmwa-puçò}
\région{GOs}
(\domainesémantique{Types de maison, architecture de la maison})
\classe{nom}
\begin{glose}
\pfra{cuisine (lieu)}
\end{glose}
\end{entrée}

\begin{entrée}{mwa-pho}{}{ⓔmwa-pho}
\région{GOs WEM}
\variante{%
mwa-wo
\région{GO(s) WEM}}
(\domainesémantique{Types de maison, architecture de la maison})
\classe{nom}
\begin{glose}
\pfra{maison où dorment femmes et enfants (lit. maison du tressage)}
\end{glose}
\newline
\relationsémantique{Cf.}{\lien{ⓔpho}{pho}}
\glosecourte{tressage}
\end{entrée}

\begin{entrée}{mwa-pwayu}{}{ⓔmwa-pwayu}
\région{GOs BO}
\variante{%
mwa-pwaeu
\région{BO}}
(\domainesémantique{Types de maison, architecture de la maison})
\classe{nom}
\begin{glose}
\pfra{maison où se retirent les femmes (pendant les règles)}
\end{glose}
\end{entrée}

\begin{entrée}{mwa-phwamwêêgu}{}{ⓔmwa-phwamwêêgu}
\région{GO BO PA}
\variante{%
mwa-phwamwãgu
\région{PA}}
(\domainesémantique{Types de maison, architecture de la maison})
\classe{nom}
\begin{glose}
\pfra{maison des hommes (grande maison servant de lieu de réunion)}
\end{glose}
\newline
\relationsémantique{Cf.}{\lien{ⓔgu-mwa}{gu-mwa}}
\end{entrée}

\begin{entrée}{mwathra}{}{ⓔmwathra}
\région{GOs}
\variante{%
mwara
\région{GO(s)}, 
mwarang
\région{BO}}
\classe{nom}
\newline
\sens{1}
(\domainesémantique{Guerre})
\begin{glose}
\pfra{noeud de guerre (annonce coutumière);}
\end{glose}
\begin{glose}
\pfra{message de guerre (transmis de chef en chef pour chercher des alliés)}
\end{glose}
\newline
\sens{2}
(\domainesémantique{Discours, échanges verbaux})
\begin{glose}
\pfra{annonce coutumière}
\end{glose}
\end{entrée}

\begin{entrée}{mwa-vèle}{}{ⓔmwa-vèle}
\région{GOs PA}
(\domainesémantique{Types de maison, architecture de la maison})
\classe{nom}
\begin{glose}
\pfra{abri (dans les champs comportant une plate-forme sur laquelle on entrepose les récoltes)}
\end{glose}
\end{entrée}

\begin{entrée}{mwê}{}{ⓔmwê}
\région{GOs}
\variante{%
mwèn
\région{PA}, 
mwãulò
\région{WE}, 
mwãulòn
\région{BO}}
(\domainesémantique{Oiseaux})
\classe{nom}
\begin{glose}
\pfra{chouette}
\end{glose}
\nomscientifique{Tyto alba lifuensis}
\end{entrée}

\begin{entrée}{mweau}{}{ⓔmweau}
\région{GOs WEM PA}
(\domainesémantique{Organisation sociale})
\classe{nom}
\begin{glose}
\pfra{fils cadet de chef}
\end{glose}
\end{entrée}

\begin{entrée}{mwêê}{}{ⓔmwêê}
\formephonétique{mwêː}
\région{GOs}
\variante{%
mwêêng
\formephonétique{mwêːŋ}
\région{PA BO}}
(\domainesémantique{Vêtements, parure})
\classe{nom}
\begin{glose}
\pfra{chapeau}
\end{glose}
\begin{glose}
\pfra{coiffe}
\end{glose}
\newline
\begin{exemple}
\région{GO}
\textbf{\pnua{mwêêga-nu, mwêêxa-nu}}
\pfra{mon chapeau}
\end{exemple}
\newline
\begin{exemple}
\région{PA}
\textbf{\pnua{i khia mwêêga-n}}
\pfra{il met son chapeau}
\end{exemple}
\newline
\begin{exemple}
\région{BO}
\textbf{\pnua{mwêêga-n}}
\pfra{son chapeau}
\end{exemple}
\end{entrée}

\begin{entrée}{mweeja}{}{ⓔmweeja}
\formephonétique{mweːɲɟa}
\région{GOs}
(\domainesémantique{Caractéristiques et propriétés des personnes})
\classe{v}
\begin{glose}
\pfra{vigoureux ; costaud ; courageux (qualifie le corps)}
\end{glose}
\end{entrée}

\begin{entrée}{mwêêje}{}{ⓔmwêêje}
\formephonétique{mwêːɲɟe}
\région{GOs PA}
(\domainesémantique{Coutumes, dons coutumiers})
\classe{nom}
\begin{glose}
\pfra{us et coutumes ; comportement ; attitude ; manière}
\end{glose}
\newline
\begin{exemple}
\région{GO}
\textbf{\pnua{mwêêje-je}}
\pfra{son comportement}
\end{exemple}
\newline
\begin{exemple}
\région{PA}
\textbf{\pnua{zo mwêêje-n}}
\pfra{il a un comportement agréable}
\end{exemple}
\end{entrée}

\begin{entrée}{mwèèn}{}{ⓔmwèèn}
\région{BO [BM]}
(\domainesémantique{Noms des plantes})
\classe{nom}
\begin{glose}
\pfra{cycas}
\end{glose}
\end{entrée}

\begin{entrée}{mwêêno}{}{ⓔmwêêno}
\formephonétique{mwêːɳo}
\région{GOs BO}
\classe{v.IMPERS}
\newline
\sens{1}
(\domainesémantique{Quantificateurs})
\begin{glose}
\pfra{manque (il) ; rester}
\end{glose}
\newline
\begin{exemple}
\région{GO}
\textbf{\pnua{mwêêno pò-ko tiivwo !}}
\pfra{il manque 3 livres !}
\end{exemple}
\newline
\begin{exemple}
\région{GO}
\textbf{\pnua{mwêêno gò pò-tru phwe-meevwu !}}
\pfra{il manque 2 clans !}
\end{exemple}
\newline
\begin{exemple}
\région{GO}
\textbf{\pnua{mwêêno ijö mwa !}}
\pfra{il ne reste plus que toi (à jouer)}
\end{exemple}
\newline
\begin{exemple}
\région{GO}
\textbf{\pnua{mwêêno ijö mwa !}}
\pfra{il ne reste plus que toi (lit. il manque toi qui dois encore jouer)}
\end{exemple}
\newline
\begin{exemple}
\région{GO}
\textbf{\pnua{mwêêno mwa lò !}}
\pfra{il ne reste plus qu'eux (lit. il manque eux qui doivent encore jouer)}
\end{exemple}
\newline
\relationsémantique{Ant.}{\lien{}{kuzaò, kuraò}}
\glosecourte{être en surplus, de trop}
\newline
\sens{2}
(\domainesémantique{Modalité, verbes modaux})
\begin{glose}
\pfra{faillir ; s'en falloir de peu que}
\end{glose}
\newline
\begin{exemple}
\région{GO}
\textbf{\pnua{mwêêno pònõ vwo la za mã !}}
\pfra{il s'en est fallu de peu qu'ils ne meurent !}
\end{exemple}
\newline
\begin{exemple}
\région{GO}
\textbf{\pnua{mwêêno mhãi vwo nu kibao-je !}}
\pfra{j'ai failli le toucher}
\end{exemple}
\newline
\begin{sous-entrée}{mwêêno pònõ vwo}{ⓔmwêênoⓢ2ⓝmwêêno pònõ vwo}
\région{GO}
\begin{glose}
\pfra{s'en falloir de peu que}
\end{glose}
\end{sous-entrée}
\end{entrée}

\begin{entrée}{mweling}{}{ⓔmweling}
\région{PA BO}
(\domainesémantique{Goût des aliments})
\classe{v.stat.}
\begin{glose}
\pfra{acide}
\end{glose}
\end{entrée}

\begin{entrée}{mwetre}{}{ⓔmwetre}
\formephonétique{mweɽe}
\région{GOs}
\variante{%
mwata
\région{BO PA}}
(\domainesémantique{Aliments, alimentation})
\classe{v}
\begin{glose}
\pfra{préparation de manioc, de banane rapée}
\end{glose}
\newline
\note{cette préparation est enveloppée dans des feuilles de canne à sucre ou de cordyline sauvage (di) et cuit)}{glose}{}
\newline
\begin{sous-entrée}{mwetre chaamwa}{ⓔmwetreⓝmwetre chaamwa}
\région{GO}
\begin{glose}
\pfra{banane rapée}
\end{glose}
\end{sous-entrée}
\newline
\begin{sous-entrée}{mwetre nu}{ⓔmwetreⓝmwetre nu}
\région{GO}
\begin{glose}
\pfra{coco rapé}
\end{glose}
\end{sous-entrée}
\newline
\begin{sous-entrée}{mwata manyõ}{ⓔmwetreⓝmwata manyõ}
\région{BO}
\begin{glose}
\pfra{manioc rapé}
\end{glose}
\end{sous-entrée}
\end{entrée}

\begin{entrée}{mwâô}{}{ⓔmwâô}
\région{BO}
(\domainesémantique{Verbes de mouvement})
\classe{v}
\begin{glose}
\pfra{glisser[Corne]}
\end{glose}
\newline
\note{non vérifié}{général}{}
\end{entrée}

\begin{entrée}{mwömö}{}{ⓔmwömö}
\région{GOs}
(\domainesémantique{Eau})
\classe{nom}
\begin{glose}
\pfra{bulles d'air (qui remontent à la surface de l'eau)}
\end{glose}
\end{entrée}

\begin{entrée}{mwòni}{}{ⓔmwòni}
\région{BO [BM]}
\variante{%
mòni
\région{BO}, 
mone
\région{PA}}
(\domainesémantique{Cultures, techniques, boutures})
\classe{v}
\begin{glose}
\pfra{arracher (la paille)}
\end{glose}
\newline
\begin{exemple}
\textbf{\pnua{i mò mhae [BO, PA]}}
\pfra{il arrache de la paille}
\end{exemple}
\end{entrée}

\begin{entrée}{mwozi}{}{ⓔmwozi}
\région{GOs}
(\domainesémantique{Fonctions naturelles humaines})
\classe{v ; n}
\begin{glose}
\pfra{hoquet ; hoquet (avoir le) ; hoqueter (en pleurs)}
\end{glose}
\end{entrée}

\newpage

\lettrine{mhw}\begin{entrée}{mhwããnu}{}{ⓔmhwããnu}
\formephonétique{mwʰɛ̃ɛ̃nu, mʰwɛ̃ɳu}
\région{GOs PA BO}
\classe{nom}
\newline
\sens{1}
(\domainesémantique{Astres})
\begin{glose}
\pfra{lune}
\end{glose}
\newline
\begin{sous-entrée}{mhava mhwããnu}{ⓔmhwããnuⓢ1ⓝmhava mhwããnu}
\begin{glose}
\pfra{premier quartier de lune}
\end{glose}
\end{sous-entrée}
\newline
\begin{sous-entrée}{hõgõõne mhwããnu}{ⓔmhwããnuⓢ1ⓝhõgõõne mhwããnu}
\région{PA}
\begin{glose}
\pfra{2ème quartier de lune, demi-lune}
\end{glose}
\end{sous-entrée}
\newline
\begin{sous-entrée}{cabòl mhwããnu}{ⓔmhwããnuⓢ1ⓝcabòl mhwããnu}
\région{PA}
\begin{glose}
\pfra{pleine lune}
\end{glose}
\end{sous-entrée}
\newline
\begin{sous-entrée}{phaa-mè mhwããnu}{ⓔmhwããnuⓢ1ⓝphaa-mè mhwããnu}
\begin{glose}
\pfra{pleine lune}
\end{glose}
\end{sous-entrée}
\newline
\begin{sous-entrée}{mhwããnu ni trabwa}{ⓔmhwããnuⓢ1ⓝmhwããnu ni trabwa}
\begin{glose}
\pfra{demi-lune}
\end{glose}
\end{sous-entrée}
\newline
\begin{sous-entrée}{we ni mè mhwããnu}{ⓔmhwããnuⓢ1ⓝwe ni mè mhwããnu}
\begin{glose}
\pfra{pluie qui accompagne la pleine lune}
\end{glose}
\end{sous-entrée}
\newline
\begin{sous-entrée}{u tabwa mhwããnu}{ⓔmhwããnuⓢ1ⓝu tabwa mhwããnu}
\région{PA}
\begin{glose}
\pfra{nouvelle lune}
\end{glose}
\end{sous-entrée}
\newline
\sens{2}
(\domainesémantique{Découpage du temps})
\begin{glose}
\pfra{mois}
\end{glose}
\end{entrée}

\begin{entrée}{mhwacidro}{}{ⓔmhwacidro}
\formephonétique{mʰwaciɖo}
\région{GOs}
(\domainesémantique{Insectes})
\classe{nom}
\begin{glose}
\pfra{fourmi noire}
\end{glose}
\end{entrée}

\begin{entrée}{mhwãnge}{}{ⓔmhwãnge}
\formephonétique{mwʰɛ̃ŋe}
\région{GOs}
(\domainesémantique{Verbes d'action (en général)})
\classe{v}
\begin{glose}
\pfra{redresser (fer)}
\end{glose}
\end{entrée}

\begin{entrée}{mhwêê}{}{ⓔmhwêê}
\formephonétique{mʰwêː}
\région{GOs}
\variante{%
mhwèèn
\formephonétique{mʰwɛːn}
\région{PA BO}}
(\domainesémantique{Navigation})
\classe{v}
\begin{glose}
\pfra{flotter, dériver}
\end{glose}
\newline
\étymologie{
\langue{POc}
\étymon{*ma-qanu}}
\end{entrée}

\begin{entrée}{mhwêêdi}{}{ⓔmhwêêdi}
\région{GOs BO}
\variante{%
mweedi
\région{PA BO}}
(\domainesémantique{Corps humain})
\classe{nom}
\begin{glose}
\pfra{nez}
\end{glose}
\newline
\begin{exemple}
\textbf{\pnua{mhweedi-n}}
\pfra{son nez}
\end{exemple}
\newline
\begin{sous-entrée}{drò-mhwêêdi}{ⓔmhwêêdiⓝdrò-mhwêêdi}
\begin{glose}
\pfra{ailes du nez}
\end{glose}
\end{sous-entrée}
\end{entrée}

\begin{entrée}{mhwi}{}{ⓔmhwi}
\région{BO}
(\domainesémantique{Caractéristiques et propriétés des personnes})
\classe{v.stat.}
\begin{glose}
\pfra{timide ; doux}
\end{glose}
\end{entrée}

\newpage

\lettrine{
ń
n
}\begin{entrée}{-n}{}{ⓔ-n}
\région{BO PA}
(\domainesémantique{Pronoms})
\classe{SUFF.POSS 3° pers. SG}
\begin{glose}
\pfra{son ; sa ; ses}
\end{glose}
\end{entrée}

\begin{entrée}{na}{1}{ⓔnaⓗ1}
\formephonétique{ɳa}
\région{GOs}
(\domainesémantique{Localisation})
\classe{PREP.LOC (spatio-temporel)}
\newline
\sens{1}
\begin{glose}
\pfra{sur ; à}
\end{glose}
\newline
\begin{exemple}
\textbf{\pnua{na le}}
\pfra{à cet endroit-là}
\end{exemple}
\newline
\begin{exemple}
\région{GO}
\textbf{\pnua{ê na Kolikò}}
\pfra{les gens de Koligo}
\end{exemple}
\newline
\sens{2}
\begin{glose}
\pfra{de (ablatif)}
\end{glose}
\newline
\begin{exemple}
\région{GO}
\textbf{\pnua{bi adaa-mi na Kumak}}
\pfra{nous venons de Koumac}
\end{exemple}
\newline
\sens{3}
\begin{glose}
\pfra{par rapport à ; parmi}
\end{glose}
\newline
\begin{exemple}
\région{PA}
\textbf{\pnua{i ẽnò na inu}}
\pfra{il est mon cadet / il est plus jeune que moi}
\end{exemple}
\newline
\begin{exemple}
\région{GO}
\textbf{\pnua{a-tru na iwe nye li zuma a iò ne thrôbo}}
\pfra{2 d'entre vous partiront ce soir}
\end{exemple}
\end{entrée}

\begin{entrée}{na}{2}{ⓔnaⓗ2}
\région{PA BO}
\newline
\groupe{A}
(\domainesémantique{Localisation})
\classe{PREP.LOC}
\begin{glose}
\pfra{dans ; sur ; à}
\end{glose}
\newline
\begin{exemple}
\textbf{\pnua{na bwa}}
\pfra{au-dessus}
\end{exemple}
\newline
\groupe{B}
(\domainesémantique{Verbes d'action (en général)})
\classe{v}
\begin{glose}
\pfra{poser}
\end{glose}
\end{entrée}

\begin{entrée}{na}{3}{ⓔnaⓗ3}
\formephonétique{ɳa, ɳe}
\région{GOs}
\variante{%
ne
\région{GO(s)}, 
na, ne
\région{BO PA}}
(\domainesémantique{Conjonction})
\classe{CNJ}
\begin{glose}
\pfra{si ; quand}
\end{glose}
\begin{glose}
\pfra{que}
\end{glose}
\newline
\begin{sous-entrée}{mõnõ na waa}{ⓔnaⓗ3ⓝmõnõ na waa}
\région{GO}
\begin{glose}
\pfra{demain matin}
\end{glose}
\newline
\begin{exemple}
\région{GO}
\région{GO}
\textbf{\pnua{nee nye-na hovwa-da na çö ogi}}
\pfra{fais-le jusqu'à ce que tu aies fini}
\end{exemple}
\newline
\begin{exemple}
\région{GO}
\textbf{\pnua{nu ne hivwine ne ça-ni la-ã pòi-nu}}
\pfra{je n'ai jamais su le faire/raconter (lit. souvent ne pas savoir faire) à mes enfants}
\end{exemple}
\newline
\begin{exemple}
\région{GO}
\textbf{\pnua{e kaabu na mõ pweeni nò-ni}}
\pfra{il nous est interdit de pêcher ce poisson-là}
\end{exemple}
\newline
\begin{exemple}
\région{GO}
\textbf{\pnua{e zo na çö wa zo}}
\pfra{il faut que tu chantes bien}
\end{exemple}
\newline
\begin{exemple}
\région{PA}
\textbf{\pnua{la khobwe na nu a-du-ò}}
\pfra{ils m'ont demandé d'aller là-bas en bas}
\end{exemple}
\newline
\begin{exemple}
\région{GO}
\textbf{\pnua{e zoma uça na we-tru tre}}
\pfra{elle arrivera dans 2 jours}
\end{exemple}
\newline
\begin{exemple}
\région{GO}
\textbf{\pnua{nu hããxa ne nu a-da-ò}}
\pfra{j'ai peur d'aller là-haut là-bas}
\end{exemple}
\newline
\begin{exemple}
\région{BO}
\textbf{\pnua{ne yu a-mi}}
\pfra{si tu viens}
\end{exemple}
\newline
\begin{exemple}
\région{BO}
\textbf{\pnua{egu ne thu-mhenõ}}
\pfra{l'homme qui marche}
\end{exemple}
\newline
\begin{exemple}
\région{BO}
\textbf{\pnua{thoomwa ne nu nõõli}}
\pfra{la femme que j'ai vue}
\end{exemple}
\newline
\relationsémantique{Ant.}{\lien{ⓔxaⓗ3ⓝdròrò xa waa}{dròrò xa waa}}
\glosecourte{hier matin}
\newline
\relationsémantique{Cf.}{\lien{}{novwö-na [GOs]}}
\glosecourte{si}
\end{sous-entrée}
\end{entrée}

\begin{entrée}{-na}{}{ⓔ-na}
\formephonétique{ɳa}
\région{GOs PA BO}
(\domainesémantique{Démonstratifs})
\classe{DEM.DEIC.2 ; ANAPH ; assertif}
\begin{glose}
\pfra{là (visible ou non)}
\end{glose}
\newline
\begin{exemple}
\textbf{\pnua{nye-na}}
\pfra{cela}
\end{exemple}
\newline
\begin{exemple}
\textbf{\pnua{we a eniza ? - Me a nye-na !}}
\pfra{quand partez-vous ? - Nous partons tout de suite !}
\end{exemple}
\newline
\begin{exemple}
\région{BO PA}
\textbf{\pnua{ge na}}
\pfra{il est ici}
\end{exemple}
\newline
\begin{sous-entrée}{bon na}{ⓔ-naⓝbon na}
\région{GO}
\begin{glose}
\pfra{après-demain}
\end{glose}
\end{sous-entrée}
\end{entrée}

\begin{entrée}{naa}{}{ⓔnaa}
\formephonétique{ɳaː}
\région{GOs}
\variante{%
na, ne
\région{PA BO}}
\classe{v}
\newline
\sens{1}
(\domainesémantique{Dons, échanges, achat et vente, vol})
\begin{glose}
\pfra{donner}
\end{glose}
\newline
\begin{exemple}
\région{GO}
\textbf{\pnua{e na lai çai la pòi-je xo õ-ẽnõ ã}}
\pfra{leur mère donne du riz à ses enfants}
\end{exemple}
\newline
\begin{exemple}
\région{GO}
\textbf{\pnua{e na lai xo õ-ẽnõ ã çai la pòi-je}}
\pfra{leur mère donne du riz à ses enfants}
\end{exemple}
\newline
\begin{exemple}
\région{GO}
\textbf{\pnua{e na-e pòi-je xa mhavwa lai}}
\pfra{elle donne à ses enfants un peu de riz}
\end{exemple}
\newline
\begin{exemple}
\région{GO}
\textbf{\pnua{e na-e pòi-je ce-la lai}}
\pfra{elle donne à ses enfants leur part de riz}
\end{exemple}
\newline
\begin{exemple}
\région{GO}
\textbf{\pnua{e na çai la ce-la lai}}
\pfra{elle leur donne du riz}
\end{exemple}
\newline
\begin{exemple}
\région{PA}
\textbf{\pnua{nee hi-n !}}
\pfra{donne-le lui (poser main-sa)}
\end{exemple}
\newline
\begin{exemple}
\région{BO}
\textbf{\pnua{nu na î Kaawo}}
\pfra{je l'ai donné à K.}
\end{exemple}
\newline
\begin{exemple}
\région{PA}
\textbf{\pnua{i na hi-m}}
\pfra{il te l'a donné}
\end{exemple}
\newline
\begin{exemple}
\région{BO}
\textbf{\pnua{i na yi-ny}}
\pfra{il me l'a donné}
\end{exemple}
\newline
\sens{2}
(\domainesémantique{Mouvements ou actions faits avec le corps, les bras, les mains, les pieds})
\begin{glose}
\pfra{mettre}
\end{glose}
\begin{glose}
\pfra{poser}
\end{glose}
\newline
\begin{exemple}
\textbf{\pnua{naa-du}}
\pfra{poser par terre}
\end{exemple}
\end{entrée}

\begin{entrée}{nãã-n}{}{ⓔnãã-n}
\région{PA BO}
(\domainesémantique{Relations et interaction sociales})
\classe{nom}
\begin{glose}
\pfra{injure ; offense ; affront ; calomnie ; mauvais sort}
\end{glose}
\begin{glose}
\pfra{leçon donnée pour faire réfléchir qqn}
\end{glose}
\newline
\begin{exemple}
\région{PA}
\textbf{\pnua{i pe-thoele nãã-n}}
\pfra{il lui a lancé des insultes}
\end{exemple}
\newline
\begin{exemple}
\région{PA}
\textbf{\pnua{i pe-thoele nãã-la}}
\pfra{il leur a lancé des insultes}
\end{exemple}
\newline
\begin{exemple}
\région{BO}
\textbf{\pnua{yo thu nãã-ny}}
\pfra{tu m'as insulté}
\end{exemple}
\newline
\begin{exemple}
\région{PA}
\textbf{\pnua{li pe-thoeli nããn}}
\pfra{ils se lancent des insultes}
\end{exemple}
\newline
\begin{exemple}
\région{BO}
\textbf{\pnua{nu pe-thoele nããn}}
\pfra{je l'ai insulté}
\end{exemple}
\newline
\begin{exemple}
\région{PA}
\textbf{\pnua{la pe-khôbwe nããn}}
\pfra{ils s'injurient}
\end{exemple}
\newline
\begin{exemple}
\région{PA}
\textbf{\pnua{i khôbwe nãã-ny}}
\pfra{il m'ainsulté}
\end{exemple}
\newline
\begin{exemple}
\région{PA}
\textbf{\pnua{nu khôbwe nãã-n}}
\pfra{je l'aiinsulté}
\end{exemple}
\newline
\begin{exemple}
\région{PA}
\textbf{\pnua{co khôbwe la-ili ma vwu nãã-ri ?}}
\pfra{à qui as-tu adressé ces insultes ?}
\end{exemple}
\newline
\begin{sous-entrée}{thu nããn}{ⓔnãã-nⓝthu nããn}
\begin{glose}
\pfra{offenser qqun, donner à penser à qqn}
\end{glose}
\newline
\relationsémantique{Cf.}{\lien{}{paxa-nããn, pawa-nããn}}
\glosecourte{injure, offense, affront}
\end{sous-entrée}
\end{entrée}

\begin{entrée}{naa yaai}{}{ⓔnaa yaai}
\formephonétique{ɳaː}
\région{GOs}
(\domainesémantique{Coutumes, dons coutumiers})
\classe{v}
\begin{glose}
\pfra{faire une demande coutumière (lit. donner le feu)}
\end{glose}
\end{entrée}

\begin{entrée}{na bòli}{}{ⓔna bòli}
\formephonétique{ɳa mbɔli}
\région{GOs BO PA}
(\domainesémantique{Directions})
\classe{LOC}
\begin{glose}
\pfra{là-bas}
\end{glose}
\newline
\begin{exemple}
\région{BO}
\textbf{\pnua{da yala ce na bòli ?}}
\pfra{comment s'appelle cet arbre là-bas ?}
\end{exemple}
\end{entrée}

\begin{entrée}{na-bulu-ni}{}{ⓔna-bulu-ni}
\formephonétique{ɳa-bu'lu-ɳi}
\région{GOs}
(\domainesémantique{Relations et interaction sociales})
\classe{v}
\begin{glose}
\pfra{assembler ; rassembler}
\end{glose}
\end{entrée}

\begin{entrée}{na ènòli}{}{ⓔna ènòli}
\formephonétique{ɳa ɛɳɔli}
\région{GOs}
(\domainesémantique{Localisation})
\classe{PREP.LOC}
\begin{glose}
\pfra{de là-bas (ablatif)}
\end{glose}
\end{entrée}

\begin{entrée}{na-hû-mi}{}{ⓔna-hû-mi}
\formephonétique{ɳa}
\région{GOs}
(\domainesémantique{Verbes de mouvement})
\classe{v}
\begin{glose}
\pfra{poser près d'ego}
\end{glose}
\newline
\begin{exemple}
\textbf{\pnua{na-hû-mi !}}
\pfra{pose le près de moi !}
\end{exemple}
\end{entrée}

\begin{entrée}{nai}{}{ⓔnai}
\formephonétique{ɳai}
\région{GOs}
\région{BO}
\variante{%
nai
}
(\domainesémantique{Prépositions})
\classe{PREP}
\begin{glose}
\pfra{de (ablatif) ; par rapport à ; envers}
\end{glose}
\newline
\begin{exemple}
\région{GO}
\textbf{\pnua{e pwawali nai je}}
\pfra{il est plus grand qu'elle}
\end{exemple}
\newline
\begin{exemple}
\région{GO}
\textbf{\pnua{e povwonu nai poi-nu}}
\pfra{il est plus petit que mon enfant}
\end{exemple}
\newline
\begin{exemple}
\région{GO}
\textbf{\pnua{e ẽnò nai nu}}
\pfra{il est mon cadet / il est plus jeune que moi}
\end{exemple}
\newline
\begin{exemple}
\région{BO}
\textbf{\pnua{i vaa nai ti ?}}
\pfra{de qui parle-t-il ?}
\end{exemple}
\newline
\begin{exemple}
\région{BO}
\textbf{\pnua{i vaa nai da ?}}
\pfra{de quoi parle-t-il ?}
\end{exemple}
\end{entrée}

\begin{entrée}{na kòlò}{}{ⓔna kòlò}
\formephonétique{ɳa}
\région{GOs}
(\domainesémantique{Localisation})
\classe{PREP (ablatif)}
\begin{glose}
\pfra{de; à}
\end{glose}
\newline
\begin{exemple}
\textbf{\pnua{e uvwi ogine-ni axe êgu po-mã na kòlò Pwayili}}
\pfra{il a acheté 20 mangues à Pwayili}
\end{exemple}
\end{entrée}

\begin{entrée}{na mwã}{}{ⓔna mwã}
\région{PA}
(\domainesémantique{Relations et interaction sociales})
\classe{v}
\begin{glose}
\pfra{rendre (qqch)}
\end{glose}
\newline
\relationsémantique{Cf.}{\lien{ⓔtèè-na}{tèè-na}}
\glosecourte{prêter}
\end{entrée}

\begin{entrée}{na ni}{}{ⓔna ni}
\formephonétique{ɳa ɳi}
\région{GOs}
(\domainesémantique{Localisation})
\classe{PREP.LOC}
\begin{glose}
\pfra{pendant ; dans ; sur}
\end{glose}
\end{entrée}

\begin{entrée}{nani}{}{ⓔnani}
\formephonétique{ɳaɳi}
\région{GOs}
\variante{%
nani
\région{PA BO}}
(\domainesémantique{Mammifères})
\classe{nom}
\begin{glose}
\pfra{chèvre}
\end{glose}
\end{entrée}

\begin{entrée}{napoine}{}{ⓔnapoine}
\région{BO}
(\domainesémantique{Coutumes, dons coutumiers})
\classe{nom}
\begin{glose}
\pfra{feuilles enfouies pour la fécondité des femmes (Dubois)}
\end{glose}
\newline
\note{non vérifié}{général}{}
\end{entrée}

\begin{entrée}{na va}{}{ⓔna va}
\formephonétique{ɳa va}
\région{GOs BO PA}
(\domainesémantique{Prépositions})
\classe{PREP}
\begin{glose}
\pfra{d'où}
\end{glose}
\newline
\begin{exemple}
\région{BO}
\textbf{\pnua{nu hivwine za a-mi na va}}
\pfra{je ne sais pas d'où ils viennent}
\end{exemple}
\end{entrée}

\begin{entrée}{na-vwo}{}{ⓔna-vwo}
\formephonétique{ɳa-βo}
\région{GOs PA BO}
(\domainesémantique{Coutumes, dons coutumiers})
\classe{nom}
\begin{glose}
\pfra{cérémonie coutumière}
\end{glose}
\begin{glose}
\pfra{dons coutumiers}
\end{glose}
\end{entrée}

\begin{entrée}{nawêni}{}{ⓔnawêni}
\formephonétique{ɳawêɳi}
\région{GOs}
\variante{%
naõni
\formephonétique{ɳaɔ̃ɳi}
\région{GO(s)}, 
naõnil
\région{PA BO}, 
naõnin
\région{BO}}
(\domainesémantique{Noms des plantes})
\classe{nom}
\begin{glose}
\pfra{chou kanak [GOs]}
\end{glose}
\begin{glose}
\pfra{'épinard' [PA]}
\end{glose}
\nomscientifique{Hibiscus manihot L. (Malvacées)}
\end{entrée}

\begin{entrée}{naxo}{}{ⓔnaxo}
\formephonétique{ɳa}
\région{GOs}
\variante{%
naxo, nago
\région{BO PA}}
(\domainesémantique{Poissons})
\classe{nom}
\begin{glose}
\pfra{mulet noir (de cascade)}
\end{glose}
\nomscientifique{Cestraeus plicatilis (Mugilidae)}
\nomscientifique{Crenimugil crenilabis}
\newline
\relationsémantique{Cf.}{\lien{ⓔwhaiⓗ1}{whai}}
\glosecourte{mulet (de petite taille)}
\newline
\relationsémantique{Cf.}{\lien{ⓔmene}{mene}}
\glosecourte{mulet queue bleue}
\end{entrée}

\begin{entrée}{ne}{4}{ⓔneⓗ4}
\formephonétique{ɳe}
\région{GOs PA}
(\domainesémantique{Aspect})
\classe{FREQ}
\begin{glose}
\pfra{souvent}
\end{glose}
\newline
\begin{exemple}
\région{GO}
\textbf{\pnua{ezoma nu ne a nõ nyaanya, na nu uça avwônô}}
\pfra{j'irai voir ma mère (souvent) quand je rentrerai chez moi}
\end{exemple}
\newline
\begin{exemple}
\région{GO}
\textbf{\pnua{e ne thruã-me}}
\pfra{il nous ment souvent}
\end{exemple}
\newline
\begin{exemple}
\région{GO}
\textbf{\pnua{lò ne kûbu-bi xo ãbaa-bi êmwê-e}}
\pfra{nos(3) frères nous frappaient souvent}
\end{exemple}
\newline
\begin{sous-entrée}{kavwö ne ...}{ⓔneⓗ4ⓝkavwö ne ...}
\begin{glose}
\pfra{ne pas souvent, jamais}
\end{glose}
\newline
\begin{exemple}
\région{GO}
\textbf{\pnua{kavwö ne môgu ègu ã !}}
\pfra{il ne travaille pas souvent celui-là !}
\end{exemple}
\newline
\begin{exemple}
\région{GO PA}
\textbf{\pnua{kavwö ne e hine khõbwe gele-xa êmwê}}
\pfra{elle n'avait jamais su qu'il existait des hommes}
\end{exemple}
\newline
\begin{exemple}
\région{GO}
\textbf{\pnua{nu ne hivwine ne ça-ni la-ã pòi-nu}}
\pfra{je n'ai jamais su le faire/raconter (lit. souvent ne pas savoir faire) à mes enfants}
\end{exemple}
\newline
\begin{exemple}
\région{PA}
\textbf{\pnua{kavwö li ne vhaa yai yo}}
\pfra{ils ne te parlent jamais}
\end{exemple}
\end{sous-entrée}
\end{entrée}

\begin{entrée}{ne}{5}{ⓔneⓗ5}
(\domainesémantique{Relateurs et relateurs possessifs})
\classe{REL}
\begin{glose}
\pfra{de}
\end{glose}
\newline
\begin{exemple}
\textbf{\pnua{pò-mugè ne kòò-n}}
\pfra{son mollet}
\end{exemple}
\end{entrée}

\begin{entrée}{nee}{2}{ⓔneeⓗ2}
\formephonétique{ɳeː}
\région{GOs}
\variante{%
ne, nee
\région{BO PA}}
(\domainesémantique{Verbes d'action (en général)})
\classe{v}
\begin{glose}
\pfra{faire ; effectuer}
\end{glose}
\newline
\begin{exemple}
\textbf{\pnua{ne hayu ma i mayo !}}
\pfra{fais-le quand même parce que cela démange !}
\end{exemple}
\newline
\begin{exemple}
\textbf{\pnua{mi ne a choomu}}
\pfra{nous allions à l'école ensemble ?}
\end{exemple}
\newline
\begin{exemple}
\région{PA}
\textbf{\pnua{e ra ne}}
\pfra{il l'a fait}
\end{exemple}
\newline
\begin{exemple}
\région{BO}
\textbf{\pnua{nee-yo-ni dili !}}
\pfra{mets bien la terre}
\end{exemple}
\newline
\begin{sous-entrée}{ne nhye thraa}{ⓔneeⓗ2ⓝne nhye thraa}
\région{GO}
\begin{glose}
\pfra{lui faire du mal, lui nuire}
\end{glose}
\end{sous-entrée}
\newline
\begin{sous-entrée}{ne ...vwo}{ⓔneeⓗ2ⓝne ...vwo}
\begin{glose}
\pfra{se mettre à, entreprendre de}
\end{glose}
\newline
\relationsémantique{Cf.}{\lien{}{po [GOs]}}
\glosecourte{faire}
\newline
\relationsémantique{Cf.}{\lien{}{pu [GOs]}}
\glosecourte{il y a}
\end{sous-entrée}
\end{entrée}

\begin{entrée}{neebu, ńeebu}{}{ⓔneebu, ńeebu}
\formephonétique{ɳɛ̃ːbu ; nɛ̃ːbu}
\région{GOs}
\variante{%
neebu
\région{PA BO}}
(\domainesémantique{Insectes})
\classe{nom}
\begin{glose}
\pfra{moustique}
\end{glose}
\newline
\étymologie{
\langue{POc}
\étymon{*ɲamu(k)}
\glosecourte{moustique}}
\end{entrée}

\begin{entrée}{nee, ńee}{1}{ⓔnee, ńeeⓗ1}
\formephonétique{ɳeː ; neː}
\région{GOs}
\variante{%
nèèng
\région{PA BO}}
(\domainesémantique{Phénomènes atmosphériques et naturels})
\classe{nom}
\begin{glose}
\pfra{nuage}
\end{glose}
\end{entrée}

\begin{entrée}{nee-wo mani phwe-wedevwo}{}{ⓔnee-wo mani phwe-wedevwo}
\formephonétique{ɳeːβo}
\région{GOs}
(\domainesémantique{Coutumes, dons coutumiers})
\classe{nom}
\begin{glose}
\pfra{us et coutumes}
\end{glose}
\end{entrée}

\begin{entrée}{nee-zo}{}{ⓔnee-zo}
\formephonétique{ɳeːzo}
\région{GOs}
(\domainesémantique{Verbes d'action (en général)})
\classe{v}
\begin{glose}
\pfra{ranger (faire bien)}
\end{glose}
\newline
\note{ne-zoo-ni (v.t.)}{grammaire}{ranger qqch}
\end{entrée}

\begin{entrée}{nèm}{}{ⓔnèm}
\région{PA}
(\domainesémantique{Goût des aliments})
\classe{v.stat.}
\begin{glose}
\pfra{doux ; sucré}
\end{glose}
\begin{glose}
\pfra{insipide [BO]}
\end{glose}
\newline
\begin{sous-entrée}{we nèm}{ⓔnèmⓝwe nèm}
\région{PA}
\begin{glose}
\pfra{eau douce}
\end{glose}
\end{sous-entrée}
\newline
\begin{sous-entrée}{we ne}{ⓔnèmⓝwe ne}
\région{GO}
\begin{glose}
\pfra{eau douce}
\end{glose}
\newline
\begin{exemple}
\région{BO}
\textbf{\pnua{i ne zo}}
\pfra{il a bon goût}
\end{exemple}
\newline
\relationsémantique{Ant.}{\lien{}{we za}}
\glosecourte{eau salée}
\end{sous-entrée}
\end{entrée}

\begin{entrée}{neme}{}{ⓔneme}
\formephonétique{ɳeme}
\région{GOs}
\variante{%
nemee-n
\région{PA BO}}
(\domainesémantique{Goût des aliments})
\classe{nom}
\begin{glose}
\pfra{goût ; saveur}
\end{glose}
\newline
\begin{exemple}
\région{GO}
\textbf{\pnua{kixa neme}}
\pfra{sans goût, fade}
\end{exemple}
\newline
\begin{exemple}
\région{PA}
\textbf{\pnua{ke neme}}
\pfra{sans goût, fade}
\end{exemple}
\newline
\begin{exemple}
\textbf{\pnua{e waya neme ?}}
\pfra{quel goût ça a ?}
\end{exemple}
\newline
\begin{exemple}
\région{BO}
\textbf{\pnua{i zo neme-n}}
\pfra{ça a bon goût}
\end{exemple}
\newline
\begin{sous-entrée}{ne-zo}{ⓔnemeⓝne-zo}
\begin{glose}
\pfra{délicieux, bon au goût}
\end{glose}
\end{sous-entrée}
\newline
\begin{sous-entrée}{ne-raa}{ⓔnemeⓝne-raa}
\begin{glose}
\pfra{mauvais (au goût)}
\end{glose}
\end{sous-entrée}
\newline
\étymologie{
\langue{POc}
\étymon{*ɲami}
\glosecourte{goûter}}
\end{entrée}

\begin{entrée}{ne-mu}{}{ⓔne-mu}
\formephonétique{ɳe-mu}
\région{GOs}
(\domainesémantique{Verbes d'action (en général)})
\classe{v}
\begin{glose}
\pfra{faire ensuite ; faire après}
\end{glose}
\end{entrée}

\begin{entrée}{ne, na}{}{ⓔne, na}
\formephonétique{ɳe ; ɳa}
\région{GOs BO}
(\domainesémantique{Modalité, verbes modaux})
\classe{OPT}
\begin{glose}
\pfra{optatif (à l'initiale de l'énoncé ; exprime: ordre, conseil, souhait)}
\end{glose}
\newline
\begin{exemple}
\textbf{\pnua{Poinyena ne ço phaaxe !}}
\pfra{Poinyena, écoute !}
\end{exemple}
\newline
\begin{exemple}
\région{BO}
\textbf{\pnua{na jo !}}
\pfra{attention à toi ! (Dubois)}
\end{exemple}
\end{entrée}

\begin{entrée}{ne, ńe}{1}{ⓔne, ńeⓗ1}
\formephonétique{ɳe ; ne}
\région{GOs}
(\domainesémantique{Outils})
\classe{nom}
\begin{glose}
\pfra{bout de verre utilisé pour couper}
\end{glose}
\end{entrée}

\begin{entrée}{ne, ńe}{2}{ⓔne, ńeⓗ2}
\formephonétique{ɳe ; ne}
\région{GOs}
\variante{%
nèn
\région{PA}}
\classe{v.stat.}
(\domainesémantique{Description des objets, formes, consistance, taille})
\begin{glose}
\pfra{émoussé [GOs]}
\end{glose}
\end{entrée}

\begin{entrée}{ne, ńe}{3}{ⓔne, ńeⓗ3}
\formephonétique{ɳe ; ne}
\région{GOs}
\variante{%
nèn
\région{PA BO}}
(\domainesémantique{Insectes})
\classe{nom}
\begin{glose}
\pfra{mouche ; moucheron}
\end{glose}
\newline
\étymologie{
\langue{POc}
\étymon{*laŋo, *lalo}
\glosecourte{mouche}}
\end{entrée}

\begin{entrée}{nenèm}{}{ⓔnenèm}
\formephonétique{nenɛm}
\région{PA}
\variante{%
nenèèm
\région{BO}}
(\domainesémantique{Caractéristiques et propriétés des personnes})
\classe{v}
\begin{glose}
\pfra{tranquille ; sage ; immobile}
\end{glose}
\newline
\begin{sous-entrée}{tee-nenèm !}{ⓔnenèmⓝtee-nenèm !}
\begin{glose}
\pfra{reste assis tranquille !}
\end{glose}
\end{sous-entrée}
\newline
\begin{sous-entrée}{ku-nenèm}{ⓔnenèmⓝku-nenèm}
\begin{glose}
\pfra{rester debout tranquille}
\end{glose}
\end{sous-entrée}
\end{entrée}

\begin{entrée}{ne-phû}{}{ⓔne-phû}
\formephonétique{ɳe-pʰû}
\région{GOs}
\variante{%
nèn phûny
\région{PA}}
(\domainesémantique{Insectes})
\classe{nom}
\begin{glose}
\pfra{mouche bleue}
\end{glose}
\end{entrée}

\begin{entrée}{ne-ra}{}{ⓔne-ra}
\formephonétique{ɳe-ra}
\région{WEM}
(\domainesémantique{Aspect})
\classe{v}
\begin{glose}
\pfra{faire en même temps}
\end{glose}
\newline
\begin{exemple}
\textbf{\pnua{e ne ra ka "caca"}}
\pfra{elle fait cela et en même temps, elle fait caca}
\end{exemple}
\newline
\begin{exemple}
\textbf{\pnua{e ne ra phao yaoli}}
\pfra{en même temps/ensuite, elle lance la balançoire}
\end{exemple}
\end{entrée}

\begin{entrée}{ne-raa}{}{ⓔne-raa}
\formephonétique{ɳe-ɽaː}
\région{GOs BO}
(\domainesémantique{Goût des aliments})
\classe{v.stat.}
\begin{glose}
\pfra{mauvais au goût}
\end{glose}
\begin{glose}
\pfra{acide [BO]}
\end{glose}
\newline
\relationsémantique{Cf.}{\lien{}{thraa [GO]}}
\glosecourte{mauvais}
\newline
\relationsémantique{Cf.}{\lien{ⓔneme}{neme}}
\glosecourte{goût}
\end{entrée}

\begin{entrée}{neule}{}{ⓔneule}
\formephonétique{ɳeule}
\région{GOs}
\variante{%
neule-gat
\région{BO}}
(\domainesémantique{Eau})
\classe{nom}
\begin{glose}
\pfra{eau saumâtre}
\end{glose}
\end{entrée}

\begin{entrée}{ne-vwo}{}{ⓔne-vwo}
\formephonétique{ɳe-βo}
\région{GOs}
\variante{%
nee-vwo
\région{BO}}
(\domainesémantique{Verbes d'action (en général)})
\classe{v ; n}
\begin{glose}
\pfra{faire qqch ; actions ; actes}
\end{glose}
\newline
\begin{exemple}
\textbf{\pnua{nee-vwo mani phwe-wedevwo}}
\pfra{les us et coutumes}
\end{exemple}
\end{entrée}

\begin{entrée}{ne-wã le}{}{ⓔne-wã le}
\formephonétique{ɳe-wã-le}
\région{GOs}
(\domainesémantique{Verbes d'action (en général)})
\classe{v}
\begin{glose}
\pfra{faire comme ceci}
\end{glose}
\newline
\begin{sous-entrée}{ne-wã na le}{ⓔne-wã leⓝne-wã na le}
\begin{glose}
\pfra{faire comme cela}
\end{glose}
\end{sous-entrée}
\end{entrée}

\begin{entrée}{nè-zo}{}{ⓔnè-zo}
\formephonétique{ɳɛ-ðo}
\région{GOs}
(\domainesémantique{Goût des aliments})
\classe{v.stat.}
\begin{glose}
\pfra{sucré ; bon au goût ; succulent}
\end{glose}
\newline
\relationsémantique{Ant.}{\lien{}{nè-raa}}
\glosecourte{mauvais au goût}
\end{entrée}

\begin{entrée}{ngãmãã}{}{ⓔngãmãã}
\formephonétique{ŋãmãː}
\région{GOs}
\variante{%
ngamããm
\région{PA BO [BM]}}
(\domainesémantique{Aliments, alimentation})
\classe{nom}
\begin{glose}
\pfra{restes de nourriture}
\end{glose}
\end{entrée}

\begin{entrée}{ngãngi}{}{ⓔngãngi}
\formephonétique{ŋãŋi}
\région{GOs}
(\domainesémantique{Goût des aliments})
\classe{v.stat.}
\begin{glose}
\pfra{acide ; amer}
\end{glose}
\end{entrée}

\begin{entrée}{ni}{1}{ⓔniⓗ1}
\formephonétique{ɳi}
\région{GOs}
\variante{%
nhim
\région{PA WEM WE BO}}
(\domainesémantique{Description des objets, formes, consistance, taille})
\classe{v ; MODIF}
\begin{glose}
\pfra{profond (mer ou rivière)}
\end{glose}
\newline
\begin{exemple}
\région{GO}
\textbf{\pnua{we ni}}
\pfra{eau profonde}
\end{exemple}
\end{entrée}

\begin{entrée}{ni}{2}{ⓔniⓗ2}
\formephonétique{ɳi}
\région{GOs}
\variante{%
ni
\région{PA BO}}
(\domainesémantique{Prépositions})
\classe{PREP (spatio-temporelle)}
\begin{glose}
\pfra{dans (contenant) ; à ; vers}
\end{glose}
\newline
\begin{exemple}
\région{GO}
\textbf{\pnua{e ã-da ni ko}}
\pfra{il monte vers la forêt}
\end{exemple}
\newline
\begin{exemple}
\région{GO}
\textbf{\pnua{e ã-da ni nò ko}}
\pfra{il s'enfonce dans la forêt}
\end{exemple}
\newline
\begin{exemple}
\région{BO}
\textbf{\pnua{na-du ni do}}
\pfra{mets-le dans la marmite}
\end{exemple}
\newline
\begin{exemple}
\textbf{\pnua{ni don}}
\pfra{dans le ciel}
\end{exemple}
\newline
\begin{exemple}
\textbf{\pnua{ni ka}}
\pfra{dans l'année}
\end{exemple}
\end{entrée}

\begin{entrée}{ni}{3}{ⓔniⓗ3}
\formephonétique{ɳi}
\région{GOs}
\variante{%
ning
\région{WEM PA BO}}
(\domainesémantique{Types de maison, architecture de la maison})
\classe{nom}
\begin{glose}
\pfra{poteaux (petits) de maison}
\end{glose}
\newline
\begin{exemple}
\région{PA}
\textbf{\pnua{ninga-mwa}}
\pfra{poteau (de maison)}
\end{exemple}
\newline
\relationsémantique{Cf.}{\lien{}{nixò mwa [GO], nixòòl [PA]}}
\glosecourte{poteau central}
\end{entrée}

\begin{entrée}{-ni}{1}{ⓔ-niⓗ1}
\formephonétique{ɳi}
\région{GOs}
\variante{%
-nim
\région{PA BO}}
(\domainesémantique{Numéraux cardinaux})
\classe{NUM}
\begin{glose}
\pfra{cinq}
\end{glose}
\newline
\étymologie{
\langue{POc}
\étymon{*lima}
\glosecourte{cinq}}
\end{entrée}

\begin{entrée}{-ni}{2}{ⓔ-niⓗ2}
\formephonétique{ɳi}
\région{GOs}
\variante{%
-nim
\région{PA}}
(\domainesémantique{Démonstratifs})
\classe{DEM.DEIC.3 ou ANAPH}
\begin{glose}
\pfra{ce ... là (mis à distance, peut être péjoratif)}
\end{glose}
\newline
\begin{exemple}
\région{PA}
\textbf{\pnua{enim}}
\pfra{là (à distance)}
\end{exemple}
\newline
\begin{exemple}
\région{PA}
\textbf{\pnua{ã-nim}}
\pfra{celui-là (péjoratif)}
\end{exemple}
\newline
\begin{exemple}
\région{PA}
\textbf{\pnua{je-ne ãgu-nim}}
\pfra{cette personne-là (péjoratif)}
\end{exemple}
\newline
\begin{exemple}
\région{PA}
\textbf{\pnua{thoomwã-nim}}
\pfra{cette femme-là (péjoratif)}
\end{exemple}
\newline
\begin{exemple}
\région{PA}
\textbf{\pnua{êgu-nim}}
\pfra{cette personne-là (péjoratif)}
\end{exemple}
\newline
\begin{exemple}
\région{PA}
\textbf{\pnua{loto-nim}}
\pfra{cette voiture-là}
\end{exemple}
\end{entrée}

\begin{entrée}{-ni}{3}{ⓔ-niⓗ3}
\formephonétique{ɳi}
\région{GOs}
\variante{%
-ni
\région{PA BO}}
(\domainesémantique{Suffixes transitifs})
\classe{SUFF (transitif)}
\begin{glose}
\pfra{suffixe transitif}
\end{glose}
\end{entrée}

\begin{entrée}{ni gò}{}{ⓔni gò}
\formephonétique{ɳĩ ŋgɔ}
\région{GOs}
(\domainesémantique{Noms locatifs})
\classe{n.LOC}
\begin{glose}
\pfra{au milieu}
\end{glose}
\newline
\begin{sous-entrée}{chaamwa ni gò}{ⓔni gòⓝchaamwa ni gò}
\région{GOs}
\begin{glose}
\pfra{bananier de taille moyenne}
\end{glose}
\end{sous-entrée}
\end{entrée}

\begin{entrée}{nii}{}{ⓔnii}
\formephonétique{ɳiː}
\région{GOs}
\variante{%
nii
\région{PA BO}}
(\domainesémantique{Oiseaux})
\classe{nom}
\begin{glose}
\pfra{canard (autochtone) ; canard à sourcils}
\end{glose}
\nomscientifique{Anas superciliosa pelewensis}
\end{entrée}

\begin{entrée}{niila}{}{ⓔniila}
\formephonétique{ɳiːla}
\région{GOs}
\variante{%
niila
\région{PA}}
(\domainesémantique{Parenté})
\classe{nom}
\begin{glose}
\pfra{arrière-petit-fils}
\end{glose}
\newline
\begin{exemple}
\région{GO}
\textbf{\pnua{niila-nu}}
\pfra{mes petits-enfants}
\end{exemple}
\newline
\begin{exemple}
\région{PA}
\textbf{\pnua{niila-ny}}
\pfra{mes arrière- petits-enfants}
\end{exemple}
\end{entrée}

\begin{entrée}{niilöö}{}{ⓔniilöö}
\formephonétique{ɳiːlωː}
\région{GOs}
\variante{%
nhiilö
\région{PA BO}}
(\domainesémantique{Eau})
\classe{nom}
\begin{glose}
\pfra{remous ; reflux}
\end{glose}
\begin{glose}
\pfra{tourbillon (dans l'eau, grand et lent)}
\end{glose}
\newline
\relationsémantique{Cf.}{\lien{ⓔpomõnim}{pomõnim}}
\glosecourte{petit tourbillon rapide}
\end{entrée}

\begin{entrée}{niivwa}{}{ⓔniivwa}
\formephonétique{ɳiːβa}
\région{GOs}
\variante{%
niivha
\région{PA BO}, 
nipa
\région{vx}}
(\domainesémantique{Fonctions intellectuelles})
\classe{v ; n}
\begin{glose}
\pfra{erreur; tromper (se)}
\end{glose}
\begin{glose}
\pfra{tromper (se); perdre (se)}
\end{glose}
\newline
\begin{exemple}
\région{GO}
\textbf{\pnua{e niivwa na ni kò}}
\pfra{il s'est perdu dans la forêt}
\end{exemple}
\end{entrée}

\begin{entrée}{-ni-ma-ba}{}{ⓔ-ni-ma-ba}
\formephonétique{ɳi-ma-mba}
\région{GOs PA BO}
(\domainesémantique{Numéraux cardinaux})
\classe{NUM}
\begin{glose}
\pfra{neuf (=5 et 3)}
\end{glose}
\end{entrée}

\begin{entrée}{-ni-ma-dru}{}{ⓔ-ni-ma-dru}
\formephonétique{ɳi-ma-nɖu}
\région{GOs}
\variante{%
-ni ma-du
\région{BO PA}}
(\domainesémantique{Numéraux cardinaux})
\classe{NUM}
\begin{glose}
\pfra{sept (=5 et 2)}
\end{glose}
\end{entrée}

\begin{entrée}{-ni-ma-gò}{}{ⓔ-ni-ma-gò}
\formephonétique{ɳi-ma-ŋgɔ}
\région{GOs}
\variante{%
ni-ma-kòn
\formephonétique{ni-ma-kɔn}
\région{PA BO}}
(\domainesémantique{Numéraux cardinaux})
\classe{NUM}
\begin{glose}
\pfra{huit (=5 et 3)}
\end{glose}
\end{entrée}

\begin{entrée}{ni-malu}{}{ⓔni-malu}
\région{GO}
(\domainesémantique{Description des objets, formes, consistance, taille})
\classe{v.stat.}
\begin{glose}
\pfra{profond ; invisible}
\end{glose}
\newline
\note{non vérifié}{général}{}
\end{entrée}

\begin{entrée}{-ni-ma-xè}{}{ⓔ-ni-ma-xè}
\formephonétique{ɳi-ma-ɣɛ}
\région{GOs PA}
\variante{%
nim-a-xe
\région{BO}}
(\domainesémantique{Numéraux cardinaux})
\classe{NUM}
\begin{glose}
\pfra{six (=5 et 1)}
\end{glose}
\newline
\begin{exemple}
\textbf{\pnua{nima-dru-ni(m) ma-dru}}
\pfra{sept}
\end{exemple}
\newline
\begin{exemple}
\textbf{\pnua{nima-gon-ni(m) ma-gon}}
\pfra{huit}
\end{exemple}
\newline
\begin{exemple}
\textbf{\pnua{nima-ba-ni(m) ma-ba}}
\pfra{neuf}
\end{exemple}
\end{entrée}

\begin{entrée}{ni mhenõ}{}{ⓔni mhenõ}
\région{GO}
(\domainesémantique{Aspect})
\classe{ASP}
\begin{glose}
\pfra{sans cesse ; sans arrêt}
\end{glose}
\newline
\begin{exemple}
\région{GO}
\textbf{\pnua{ge li gò ni mhenõ gi}}
\pfra{ils sont toujours en train de pleurer}
\end{exemple}
\end{entrée}

\begin{entrée}{ninigin}{}{ⓔninigin}
\région{BO}
(\domainesémantique{Armes})
\classe{nom}
\begin{glose}
\pfra{casse-tête à bout dentelé (aussi une espèce d'arbre dont on utilisait une partie du tronc et le début des racines coupées en pointe. Dubois)}
\end{glose}
\newline
\note{non vérifié}{général}{}
\end{entrée}

\begin{entrée}{ni nõ}{}{ⓔni nõ}
\formephonétique{ɳĩ ɳɔ̃}
\région{GOs}
(\domainesémantique{Localisation})
\classe{LOC}
\begin{glose}
\pfra{dans ; dedans ; à l'intérieur de}
\end{glose}
\newline
\begin{exemple}
\région{GOs}
\textbf{\pnua{e ã-da ni nõ ko}}
\pfra{il s'enfonce dans la forêt}
\end{exemple}
\newline
\begin{exemple}
\région{GOs}
\textbf{\pnua{e u-da ni nõ mwa}}
\pfra{il entre à l'intérieur de la maison}
\end{exemple}
\newline
\begin{exemple}
\région{GOs}
\textbf{\pnua{e u-du ni nõ we}}
\pfra{il plonge sous l'eau}
\end{exemple}
\end{entrée}

\begin{entrée}{niô}{}{ⓔniô}
\formephonétique{ɳiõ}
\région{GOs BO}
\variante{%
nhyô
\région{PA}}
(\domainesémantique{Phénomènes atmosphériques et naturels})
\classe{nom}
\begin{glose}
\pfra{tonnerre}
\end{glose}
\newline
\begin{exemple}
\région{PA}
\textbf{\pnua{i hûn ne nhyô}}
\pfra{le tonnerre gronde}
\end{exemple}
\end{entrée}

\begin{entrée}{niû}{}{ⓔniû}
\région{BO}
(\domainesémantique{Navigation})
\classe{nom}
\begin{glose}
\pfra{ancre [BM, Corne]}
\end{glose}
\newline
\begin{exemple}
\région{BO}
\textbf{\pnua{niû wony}}
\pfra{l'ancre du bateau}
\end{exemple}
\newline
\note{niûni (v.t.)}{grammaire}{ancrer}
\end{entrée}

\begin{entrée}{niûni}{}{ⓔniûni}
\région{BO [BM]}
(\domainesémantique{Navigation})
\classe{v}
\begin{glose}
\pfra{ancrer}
\end{glose}
\newline
\begin{exemple}
\région{BO}
\textbf{\pnua{i niûni wòny}}
\pfra{il a ancré le bateau}
\end{exemple}
\newline
\relationsémantique{Cf.}{\lien{ⓔniû}{niû}}
\glosecourte{ancre}
\end{entrée}

\begin{entrée}{ni-xa}{}{ⓔni-xa}
\formephonétique{ɳiɣa}
\région{GOs}
(\domainesémantique{Localisation})
\classe{LOC}
\begin{glose}
\pfra{quelque part}
\end{glose}
\newline
\begin{exemple}
\textbf{\pnua{ge ni-xa}}
\pfra{il est quelque part}
\end{exemple}
\end{entrée}

\begin{entrée}{nixò}{}{ⓔnixò}
\formephonétique{ɳiɣɔ}
\région{GOs}
\variante{%
nixòòl
\région{PA BO WEM}, 
nigòòl
\région{BO}}
(\domainesémantique{Types de maison, architecture de la maison})
\classe{nom}
\begin{glose}
\pfra{poteau central de la case}
\end{glose}
\begin{glose}
\pfra{mât (bateau)}
\end{glose}
\newline
\begin{sous-entrée}{nixò-mwa}{ⓔnixòⓝnixò-mwa}
\région{GO}
\begin{glose}
\pfra{poteau central de la case}
\end{glose}
\end{sous-entrée}
\newline
\relationsémantique{Cf.}{\lien{ⓔńodo}{ńodo}}
\glosecourte{gaulettes}
\end{entrée}

\begin{entrée}{-niza ?}{}{ⓔ-niza ?}
\formephonétique{ɳiða}
\région{GOs}
\variante{%
-nira ?
\région{PA BO}}
(\domainesémantique{Interrogatifs})
\classe{INT}
\begin{glose}
\pfra{combien?}
\end{glose}
\newline
\begin{exemple}
\région{GO}
\textbf{\pnua{a-niza ?}}
\pfra{combien ? (d'êtres animés)}
\end{exemple}
\newline
\begin{exemple}
\région{GO}
\textbf{\pnua{pu-niza}}
\pfra{combien (de pieds d'arbres)}
\end{exemple}
\newline
\begin{exemple}
\région{PA BO}
\textbf{\pnua{pwò-nira ?}}
\pfra{combien ? (de choses rondes)}
\end{exemple}
\newline
\begin{exemple}
\textbf{\pnua{wè-nira}}
\pfra{combien (de choses longues)}
\end{exemple}
\newline
\begin{exemple}
\textbf{\pnua{wa(n)-nira}}
\pfra{combien (de lots de 2 roussettes ou notous)}
\end{exemple}
\newline
\begin{exemple}
\textbf{\pnua{maè-nira, etc.}}
\pfra{combien (de bottes de paille)}
\end{exemple}
\newline
\étymologie{
\langue{POc}
\étymon{*pinsa, *pija}
\glosecourte{combien?}}
\end{entrée}

\begin{entrée}{ńõ}{1}{ⓔńõⓗ1}
\formephonétique{nɔ̃}
\région{GOs}
\variante{%
nòl
\région{BO PA}}
\newline
\sens{1}
(\domainesémantique{Fonctions naturelles humaines})
\classe{v}
\begin{glose}
\pfra{éveiller (s') ; réveiller (se)}
\end{glose}
\newline
\begin{exemple}
\région{GO}
\textbf{\pnua{ńõ a}}
\pfra{le soleil monte (lit. s'éveille)}
\end{exemple}
\newline
\begin{exemple}
\région{PA}
\textbf{\pnua{pa-nòòli-je}}
\pfra{réveille-le}
\end{exemple}
\newline
\sens{2}
(\domainesémantique{Mouvements ou actions avec la tête, les yeux, la bouche})
\classe{v}
\begin{glose}
\pfra{ouvrir les yeux}
\end{glose}
\end{entrée}

\begin{entrée}{ńõ}{2}{ⓔńõⓗ2}
\formephonétique{nɔ̃}
\région{GOs}
(\domainesémantique{Marques restrictives})
\classe{RESTR}
\begin{glose}
\pfra{seul(ement) (on attend plus)}
\end{glose}
\newline
\begin{exemple}
\région{GOs}
\textbf{\pnua{weniza wõ ? - ca we-xe ńõ wõ - ca we-tru}}
\pfra{combien de bateaux? - un seul bateau - seulement deux}
\end{exemple}
\newline
\begin{exemple}
\région{GOs}
\textbf{\pnua{nu nõõli wõ xa we-tru}}
\pfra{j'ai vu deux bateaux}
\end{exemple}
\end{entrée}

\begin{entrée}{no}{}{ⓔno}
\formephonétique{ɳo}
\région{GOs}
\région{PA BO}
\variante{%
nòòl
, 
nòò
\région{PA}}
(\domainesémantique{Fonctions naturelles humaines})
\classe{v}
\begin{glose}
\pfra{voir}
\end{glose}
\newline
\begin{exemple}
\région{GO}
\textbf{\pnua{nu noo-jo}}
\pfra{je t'ai vu}
\end{exemple}
\newline
\begin{exemple}
\région{PA}
\textbf{\pnua{nu noo-du noo-da nai je}}
\pfra{je l'ai regardé de haut en bas}
\end{exemple}
\newline
\begin{exemple}
\région{GO}
\textbf{\pnua{e no ciia xo zine}}
\pfra{le rat voit le poulpe}
\end{exemple}
\newline
\begin{exemple}
\région{GO}
\textbf{\pnua{e no-du ni phwa ni gò-mwa}}
\pfra{elle regarde par la fenêtre}
\end{exemple}
\newline
\begin{exemple}
\région{GO}
\textbf{\pnua{e no-wã-du}}
\pfra{elle regarde vers le bas}
\end{exemple}
\newline
\begin{exemple}
\région{GO}
\textbf{\pnua{nole wô !}}
\pfra{regarde le bateau !}
\end{exemple}
\newline
\begin{exemple}
\région{GO}
\textbf{\pnua{e no-phènô}}
\pfra{il regarde furtivement}
\end{exemple}
\newline
\note{v.t. nòòli (+ inanimé)}{grammaire}{}
\end{entrée}

\begin{entrée}{nõ}{1}{ⓔnõⓗ1}
\formephonétique{ɳɔ̃}
\région{GOs}
\région{PA BO WEM WE}
(\domainesémantique{Noms locatifs})
\classe{N.LOC}
\begin{glose}
\pfra{lieu ; endroit}
\end{glose}
\newline
\begin{sous-entrée}{nõ-ko}{ⓔnõⓗ1ⓝnõ-ko}
\begin{glose}
\pfra{forêt}
\end{glose}
\end{sous-entrée}
\newline
\begin{sous-entrée}{nõ-avwono}{ⓔnõⓗ1ⓝnõ-avwono}
\begin{glose}
\pfra{cour de la maison}
\end{glose}
\end{sous-entrée}
\newline
\begin{sous-entrée}{nõ-chaamwa}{ⓔnõⓗ1ⓝnõ-chaamwa}
\begin{glose}
\pfra{bananeraie}
\end{glose}
\end{sous-entrée}
\newline
\begin{sous-entrée}{nõ-gò}{ⓔnõⓗ1ⓝnõ-gò}
\begin{glose}
\pfra{bambouseraie}
\end{glose}
\end{sous-entrée}
\newline
\begin{sous-entrée}{nõ-mu-ce}{ⓔnõⓗ1ⓝnõ-mu-ce}
\begin{glose}
\pfra{un massif de fleurs}
\end{glose}
\end{sous-entrée}
\newline
\begin{sous-entrée}{nõ-lalue}{ⓔnõⓗ1ⓝnõ-lalue}
\begin{glose}
\pfra{une touffe d'aloès}
\end{glose}
\newline
\begin{exemple}
\région{PA}
\textbf{\pnua{u-da ni nõ-kui}}
\pfra{monter au champ d'igname}
\end{exemple}
\newline
\relationsémantique{Cf.}{\lien{ⓔkêê-mu-ce}{kêê-mu-ce}}
\glosecourte{un jardin de fleurs}
\end{sous-entrée}
\end{entrée}

\begin{entrée}{nõ}{2}{ⓔnõⓗ2}
\formephonétique{ɳɔ̃}
\région{GOs}
\variante{%
nõ
\région{PA BO}}
(\domainesémantique{Noms locatifs})
\classe{n.LOC}
\begin{glose}
\pfra{intérieur (à l') ; dans}
\end{glose}
\newline
\begin{exemple}
\région{GO}
\textbf{\pnua{e u-da mwa}}
\pfra{il entre dans la maison}
\end{exemple}
\newline
\begin{exemple}
\région{GO}
\textbf{\pnua{e u-du ni nõ-we}}
\pfra{il plonge sous l'eau}
\end{exemple}
\newline
\begin{exemple}
\région{GO}
\textbf{\pnua{ge je (ni) nõ mwa}}
\pfra{il est dans la maison}
\end{exemple}
\newline
\begin{sous-entrée}{nõ-dili}{ⓔnõⓗ2ⓝnõ-dili}
\begin{glose}
\pfra{sous la terre}
\end{glose}
\end{sous-entrée}
\newline
\begin{sous-entrée}{nõ-weza}{ⓔnõⓗ2ⓝnõ-weza}
\begin{glose}
\pfra{sous la mer}
\end{glose}
\end{sous-entrée}
\newline
\begin{sous-entrée}{nõ-pwamwa}{ⓔnõⓗ2ⓝnõ-pwamwa}
\begin{glose}
\pfra{tout le pays}
\end{glose}
\newline
\relationsémantique{Cf.}{\lien{ⓔniⓗ1}{ni}}
\glosecourte{vers}
\end{sous-entrée}
\newline
\étymologie{
\langue{POc}
\étymon{*lalo}
\glosecourte{intérieur}}
\end{entrée}

\begin{entrée}{nõ}{3}{ⓔnõⓗ3}
\formephonétique{ɳɔ̃}
\région{GOs}
\variante{%
nõ
\région{PA BO}}
(\domainesémantique{Poissons})
\classe{nom}
\begin{glose}
\pfra{poisson}
\end{glose}
\newline
\étymologie{
\langue{POc}
\étymon{*lau(k)}
\glosecourte{fish}}
\end{entrée}

\begin{entrée}{nõ-}{}{ⓔnõ-}
\formephonétique{ɳɔ̃}
\région{GOs}
\variante{%
nõ-
\région{PA}}
(\domainesémantique{Préfixes classificateurs numériques})
\classe{CLF.NUM}
\begin{glose}
\pfra{champ (d'ignames, etc.)}
\end{glose}
\newline
\begin{exemple}
\textbf{\pnua{nõ-xe, nõ-tru, nõ-ko nõ-kui, etc.}}
\pfra{un, deux, trois billons d'ignames}
\end{exemple}
\newline
\begin{exemple}
\textbf{\pnua{nõ-xe kêê kui, etc.}}
\pfra{un champ d'ignames}
\end{exemple}
\end{entrée}

\begin{entrée}{nobe}{}{ⓔnobe}
\formephonétique{ɳombe}
\région{GOs}
(\domainesémantique{Caractéristiques et propriétés des personnes})
\classe{v}
\begin{glose}
\pfra{voyeur (être)}
\end{glose}
\newline
\begin{exemple}
\textbf{\pnua{e nobe}}
\pfra{il est voyeur}
\end{exemple}
\end{entrée}

\begin{entrée}{nõbo}{}{ⓔnõbo}
\formephonétique{ɳɔ̃bo}
\région{GOs}
\variante{%
nõbo, nõbwo
\région{WEM WE PA BO}}
\classe{nom}
\newline
\sens{1}
(\domainesémantique{Noms locatifs})
\begin{glose}
\pfra{emplacement ; trace ; marque}
\end{glose}
\newline
\begin{exemple}
\région{BO}
\textbf{\pnua{la nõbo-î mwang}}
\pfra{nos mauvaises actions (Dubois)}
\end{exemple}
\newline
\sens{2}
(\domainesémantique{Santé, maladie})
\begin{glose}
\pfra{cicatrice ; blessure}
\end{glose}
\newline
\begin{exemple}
\textbf{\pnua{e thu nõbo hele na ènêda}}
\pfra{elle fait une entaille avec un couteau en haut}
\end{exemple}
\newline
\begin{sous-entrée}{nõbo hele}{ⓔnõboⓢ2ⓝnõbo hele}
\begin{glose}
\pfra{blessure du couteau}
\end{glose}
\newline
\begin{exemple}
\région{BO}
\textbf{\pnua{nõbo jigal}}
\pfra{trou fait par une balle de fusil}
\end{exemple}
\end{sous-entrée}
\newline
\begin{sous-entrée}{nõbo yai [PA]}{ⓔnõboⓢ2ⓝnõbo yai [PA]}
\begin{glose}
\pfra{brûlure, trace de feu}
\end{glose}
\end{sous-entrée}
\newline
\begin{sous-entrée}{nõbo paro-n}{ⓔnõboⓢ2ⓝnõbo paro-n}
\begin{glose}
\pfra{morsure, trace de dent}
\end{glose}
\newline
\relationsémantique{Cf.}{\lien{}{mhe-nõbo [GOs]}}
\glosecourte{blessure}
\end{sous-entrée}
\end{entrée}

\begin{entrée}{nõbu}{}{ⓔnõbu}
\formephonétique{ɳɔ̃bu}
\région{GOs}
\variante{%
nõbu
\région{BO PA}}
\classe{nom}
\newline
\sens{1}
(\domainesémantique{Objets coutumiers})
\begin{glose}
\pfra{perche avec un paquet (signale un interdit)}
\end{glose}
\begin{glose}
\pfra{signe (interdisant de toucher à qqch.)}
\end{glose}
\newline
\sens{2}
(\domainesémantique{Organisation sociale})
\begin{glose}
\pfra{interdit}
\end{glose}
\begin{glose}
\pfra{règle ; loi}
\end{glose}
\begin{glose}
\pfra{protection}
\end{glose}
\newline
\begin{exemple}
\textbf{\pnua{e na nõbu nye ce}}
\pfra{il a mit un interdit sur un arbre}
\end{exemple}
\newline
\begin{exemple}
\région{BO}
\textbf{\pnua{i khabe nõbu}}
\pfra{il a planté une perche d'interdiction}
\end{exemple}
\newline
\begin{exemple}
\textbf{\pnua{nõbu-ã}}
\pfra{nos lois}
\end{exemple}
\newline
\begin{sous-entrée}{ce-nõbu}{ⓔnõbuⓢ2ⓝce-nõbu}
\begin{glose}
\pfra{perche signalant un interdit}
\end{glose}
\end{sous-entrée}
\newline
\begin{sous-entrée}{nõbu thoomwã}{ⓔnõbuⓢ2ⓝnõbu thoomwã}
\begin{glose}
\pfra{geste donné en signe de fiançaille d'une jeune-femme}
\end{glose}
\end{sous-entrée}
\newline
\begin{sous-entrée}{na nõbu}{ⓔnõbuⓢ2ⓝna nõbu}
\begin{glose}
\pfra{poser un interdit}
\end{glose}
\end{sous-entrée}
\newline
\begin{sous-entrée}{phu nõbu}{ⓔnõbuⓢ2ⓝphu nõbu}
\begin{glose}
\pfra{enlever un interdit}
\end{glose}
\end{sous-entrée}
\end{entrée}

\begin{entrée}{nobwò}{}{ⓔnobwò}
\formephonétique{ɳobwɔ}
\région{GOs}
\variante{%
nòbu
\région{BO PA}}
(\domainesémantique{Verbes d'action (en général)})
\classe{nom}
\begin{glose}
\pfra{devoir ; tâche}
\end{glose}
\begin{glose}
\pfra{actes ; actions ; occupations}
\end{glose}
\newline
\begin{exemple}
\région{GO}
\textbf{\pnua{nobwò-jö}}
\pfra{ta tâche}
\end{exemple}
\newline
\begin{exemple}
\région{GO}
\textbf{\pnua{nobwò-nu vwo nu na cee-je mõnõ}}
\pfra{je dois (ma tâche) lui donner à manger demain}
\end{exemple}
\newline
\begin{exemple}
\région{BO}
\textbf{\pnua{tu nòbu}}
\pfra{faire une tâche}
\end{exemple}
\newline
\relationsémantique{Cf.}{\lien{}{e zo}}
\glosecourte{il faut que}
\end{entrée}

\begin{entrée}{nõbwo wha}{}{ⓔnõbwo wha}
\formephonétique{nɔ̃bwo}
\région{BO}
(\domainesémantique{Corps humain})
\classe{nom}
\begin{glose}
\pfra{fontanelle [Corne]}
\end{glose}
\newline
\note{non vérifié}{général}{}
\end{entrée}

\begin{entrée}{ńodo}{}{ⓔńodo}
\formephonétique{nondo}
\région{WEM BO PA}
(\domainesémantique{Types de maison, architecture de la maison})
\classe{nom}
\begin{glose}
\pfra{gaulettes servant d'appui aux solives}
\end{glose}
\newline
\note{sorte de sablière tenant les gaulettes verticales "me-de" et les chevrons "ce-mwa" du sommet du poteau central "nigol" de la maison ronde (Dubois); lit. cou de la marmite}{glose}{}
\end{entrée}

\begin{entrée}{nòe, ne}{}{ⓔnòe, ne}
\formephonétique{ɳɔe, ɳe}
\région{GOs}
\variante{%
nòe
\région{BO}, 
ne
\région{PA}}
(\domainesémantique{Verbes d'action (en général)})
\classe{v}
\begin{glose}
\pfra{faire ; agir}
\end{glose}
\newline
\begin{sous-entrée}{nòe hayu}{ⓔnòe, neⓝnòe hayu}
\begin{glose}
\pfra{faire au hasard}
\end{glose}
\newline
\begin{exemple}
\région{BO}
\textbf{\pnua{me-nòe-wo}}
\pfra{action}
\end{exemple}
\end{sous-entrée}
\end{entrée}

\begin{entrée}{noga}{}{ⓔnoga}
\formephonétique{ɳoŋga}
\région{GOs}
\variante{%
noga
\région{BO}}
(\domainesémantique{Religion, représentations religieuses})
\classe{nom}
\begin{glose}
\pfra{voyant ; devin}
\end{glose}
\end{entrée}

\begin{entrée}{nõgò}{}{ⓔnõgò}
\formephonétique{ɳɔ̃ŋgɔ}
\région{GOs}
\variante{%
nõgò
\région{BO PA}}
\classe{nom}
\newline
\sens{1}
(\domainesémantique{Eau})
\begin{glose}
\pfra{rivière ; creek [PA BO] ; ruisseau}
\end{glose}
\newline
\sens{2}
(\domainesémantique{Topographie})
\begin{glose}
\pfra{ravin ; vallée}
\end{glose}
\newline
\begin{sous-entrée}{ku-nõgò}{ⓔnõgòⓢ2ⓝku-nõgò}
\région{GO}
\begin{glose}
\pfra{source de rivière}
\end{glose}
\end{sous-entrée}
\newline
\begin{sous-entrée}{phwe-nõgò}{ⓔnõgòⓢ2ⓝphwe-nõgò}
\région{GO}
\begin{glose}
\pfra{embouchure de la rivière}
\end{glose}
\end{sous-entrée}
\newline
\begin{sous-entrée}{phwe-nõgò}{ⓔnõgòⓢ2ⓝphwe-nõgò}
\région{PA}
\begin{glose}
\pfra{confluent d'un creek dans un fleuve}
\end{glose}
\end{sous-entrée}
\newline
\begin{sous-entrée}{pwò-nõgo}{ⓔnõgòⓢ2ⓝpwò-nõgo}
\région{BO}
\begin{glose}
\pfra{petite vallée}
\end{glose}
\newline
\begin{exemple}
\région{BO}
\textbf{\pnua{ni nõgo jaaòl}}
\pfra{dans la vallée du Diahot}
\end{exemple}
\end{sous-entrée}
\end{entrée}

\begin{entrée}{no kaö}{}{ⓔno kaö}
\formephonétique{ɳo}
\région{GOs}
(\domainesémantique{Fonctions naturelles humaines})
\classe{v}
\begin{glose}
\pfra{regarder par dessus}
\end{glose}
\newline
\begin{exemple}
\textbf{\pnua{nu no kaö-je}}
\pfra{je regarde par dessus elle}
\end{exemple}
\end{entrée}

\begin{entrée}{nõ-kò}{}{ⓔnõ-kò}
\formephonétique{ɳɔ̃kɔ}
\région{GOs}
\variante{%
nõ-ko, nõ-xo
\région{PA}}
(\domainesémantique{Végétation})
\classe{nom}
\begin{glose}
\pfra{forêt ; brousse ; maquis}
\end{glose}
\end{entrée}

\begin{entrée}{nò-khia}{}{ⓔnò-khia}
\région{BO}
(\domainesémantique{Ignames})
\classe{nom}
\begin{glose}
\pfra{côté mâle du massif d'ignames (Dubois)}
\end{glose}
\begin{glose}
\pfra{billon}
\end{glose}
\newline
\note{non vérifié}{général}{}
\end{entrée}

\begin{entrée}{no-maari}{}{ⓔno-maari}
\formephonétique{ɳo-maːri}
\région{GOs}
(\domainesémantique{Fonctions naturelles humaines})
\classe{v}
\begin{glose}
\pfra{regarder avec envie, avec admiration}
\end{glose}
\end{entrée}

\begin{entrée}{no-me}{}{ⓔno-me}
\formephonétique{ɳɔ-me}
\région{GOs}
(\domainesémantique{Conjonction})
\classe{CNJ}
\begin{glose}
\pfra{si ; hypothétique}
\end{glose}
\newline
\begin{exemple}
\région{GOs}
\textbf{\pnua{no-me [=novwo na khõbwe] çö bala a, çö thomã-nu}}
\pfra{si jamais tu t'en vas, tu m'appelles}
\end{exemple}
\newline
\begin{exemple}
\région{GOs}
\textbf{\pnua{ezoma e, no-me [= novwo na khõbwe] zo tree mònõ}}
\pfra{ce serait bien s'il fait beau demain}
\end{exemple}
\newline
\note{forme courte de : novwo na khobwe}{grammaire}{}
\end{entrée}

\begin{entrée}{nòme}{}{ⓔnòme}
\formephonétique{ɳɔme}
\région{GOs}
\variante{%
nòme
\région{PA BO}}
(\domainesémantique{Fonctions naturelles humaines})
\classe{v}
\begin{glose}
\pfra{avaler}
\end{glose}
\newline
\begin{sous-entrée}{nòme hô}{ⓔnòmeⓝnòme hô}
\begin{glose}
\pfra{avaler sans mâcher, tout rond}
\end{glose}
\end{sous-entrée}
\newline
\étymologie{
\langue{POc}
\étymon{*konom}}
\end{entrée}

\begin{entrée}{nòmo}{}{ⓔnòmo}
\région{PA}
(\domainesémantique{Fonctions naturelles humaines})
\classe{v}
\begin{glose}
\pfra{avaler sans mâcher}
\end{glose}
\end{entrée}

\begin{entrée}{nõ-na}{}{ⓔnõ-na}
\formephonétique{ɳɔ̃-ɳa}
\région{GO PA}
\variante{%
novwö na
, 
nõ-ne
\région{GO PA}}
(\domainesémantique{Conjonction})
\classe{CNJ}
\begin{glose}
\pfra{quand ; si}
\end{glose}
\newline
\begin{exemple}
\textbf{\pnua{nò-na [=novwö na] uça}}
\pfra{quand elle arrivera}
\end{exemple}
\newline
\note{forme courte de : novwö na}{grammaire}{}
\end{entrée}

\begin{entrée}{nõnõ}{}{ⓔnõnõ}
\formephonétique{ɳɔ̃ɳɔ̃}
\région{GOs}
\variante{%
nõnõm
\région{BO}}
(\domainesémantique{Fonctions intellectuelles})
\classe{v ; n}
\begin{glose}
\pfra{penser ; rappeler (se) ; rêvasser}
\end{glose}
\newline
\begin{exemple}
\région{GO}
\textbf{\pnua{e nõnõ oã-je}}
\pfra{il pense à sa mère}
\end{exemple}
\newline
\begin{exemple}
\région{BO}
\textbf{\pnua{nõnõ-m}}
\pfra{tes pensées}
\end{exemple}
\newline
\begin{exemple}
\région{BO}
\textbf{\pnua{yo ra nõnõ-nu ?}}
\pfra{tu te souviens de moi ?}
\end{exemple}
\end{entrée}

\begin{entrée}{nõnõmi}{}{ⓔnõnõmi}
\formephonétique{ɳɔ̃ɳɔ̃mi}
\région{GOs}
\variante{%
nõnõmi
\région{PA BO}}
(\domainesémantique{Fonctions intellectuelles})
\classe{v ; n}
\begin{glose}
\pfra{penser ; souvenir de (se)}
\end{glose}
\newline
\begin{exemple}
\région{GO}
\textbf{\pnua{e nõnõmi pòmò-je}}
\pfra{il pense à son pays}
\end{exemple}
\newline
\begin{exemple}
\région{GO}
\textbf{\pnua{jo nõnõmi da ?}}
\pfra{à quoi penses-tu ?}
\end{exemple}
\newline
\begin{exemple}
\région{PA}
\textbf{\pnua{kavwu jaxa nõnõm i la}}
\pfra{ils n'ont pas assez réfléchi}
\end{exemple}
\newline
\begin{exemple}
\région{BO}
\textbf{\pnua{yo ra nõnõmi ?}}
\pfra{tu te souviens ?}
\end{exemple}
\newline
\begin{sous-entrée}{pha-nõnõmi}{ⓔnõnõmiⓝpha-nõnõmi}
\begin{glose}
\pfra{rappeler (à qqn); faire se souvenir}
\end{glose}
\end{sous-entrée}
\newline
\begin{sous-entrée}{tre-nõnõmi}{ⓔnõnõmiⓝtre-nõnõmi}
\région{GO}
\begin{glose}
\pfra{réfléchir}
\end{glose}
\end{sous-entrée}
\end{entrée}

\begin{entrée}{nòò}{}{ⓔnòò}
\formephonétique{ɳɔː}
\région{GOs}
\variante{%
nòi-n
\région{BO}}
(\domainesémantique{Fonctions naturelles humaines})
\classe{v ; n}
\begin{glose}
\pfra{rêve ; rêver}
\end{glose}
\newline
\begin{exemple}
\région{GO}
\textbf{\pnua{e kô-nòò}}
\pfra{il rêve}
\end{exemple}
\newline
\begin{exemple}
\région{GO}
\textbf{\pnua{da nòò ijo dròrò ?}}
\pfra{qu'as-tu rêvé hier ?}
\end{exemple}
\newline
\begin{exemple}
\région{GO}
\textbf{\pnua{nu kô-nòò}}
\pfra{j'ai fait un rêve}
\end{exemple}
\newline
\begin{exemple}
\région{GO}
\textbf{\pnua{kô-nòò-nu nyanya}}
\pfra{j'ai rêvé de maman}
\end{exemple}
\newline
\begin{exemple}
\région{GO}
\textbf{\pnua{nòò-nu}}
\pfra{mon rêve}
\end{exemple}
\newline
\begin{exemple}
\région{BO}
\textbf{\pnua{nòi-m}}
\pfra{ton rêve}
\end{exemple}
\newline
\begin{exemple}
\région{BO}
\textbf{\pnua{nu kò nòi-ny i nyanya}}
\pfra{j'ai rêvé de maman}
\end{exemple}
\newline
\begin{exemple}
\région{BO}
\textbf{\pnua{ju kò nòi-m i ri ?}}
\pfra{de qui as-tu rêvé?}
\end{exemple}
\end{entrée}

\begin{entrée}{nõõ}{1}{ⓔnõõⓗ1}
\formephonétique{ɳɔ̃ː}
\région{GOs}
\variante{%
nõõ
\formephonétique{nɔ̃ː}
\région{BO PA}}
(\domainesémantique{Corps humain})
\classe{nom}
\begin{glose}
\pfra{cou ; gorge}
\end{glose}
\newline
\begin{exemple}
\région{PA}
\textbf{\pnua{nõõ-n}}
\pfra{son cou}
\end{exemple}
\newline
\begin{exemple}
\région{BO}
\textbf{\pnua{pu-nõõ-n}}
\pfra{sa crinière}
\end{exemple}
\end{entrée}

\begin{entrée}{nõõ}{2}{ⓔnõõⓗ2}
\formephonétique{̃̃ɳɔ̃ː}
\région{GOs}
\variante{%
nõõ
\région{BO PA}}
(\domainesémantique{Fonctions naturelles humaines})
\classe{v}
\begin{glose}
\pfra{regarder}
\end{glose}
\newline
\begin{exemple}
\région{BO}
\textbf{\pnua{nu nõõ-jo}}
\pfra{je te regarde, te vois}
\end{exemple}
\end{entrée}

\begin{entrée}{nôô-ba}{}{ⓔnôô-ba}
\formephonétique{ɳôː-ba}
\région{PA}
(\domainesémantique{Fonctions naturelles humaines})
\classe{v}
\begin{glose}
\pfra{regarder dans le noir}
\end{glose}
\end{entrée}

\begin{entrée}{nõõ-do}{}{ⓔnõõ-do}
\région{BO}
(\domainesémantique{Description des objets, formes, consistance, taille})
\classe{nom}
\begin{glose}
\pfra{col de la marmite}
\end{glose}
\end{entrée}

\begin{entrée}{nõõ-hi}{}{ⓔnõõ-hi}
\formephonétique{ɳɔ̃ː-hi}
\région{GOs BO PA}
(\domainesémantique{Corps humain})
\classe{nom}
\begin{glose}
\pfra{poignet}
\end{glose}
\newline
\begin{exemple}
\textbf{\pnua{nõõ-hi-n}}
\pfra{son poignet}
\end{exemple}
\end{entrée}

\begin{entrée}{nõõ-kò}{}{ⓔnõõ-kò}
\formephonétique{ɳɔ̃ː-kɔ}
\région{GOs}
(\domainesémantique{Corps humain})
\classe{nom}
\begin{glose}
\pfra{cheville}
\end{glose}
\newline
\begin{exemple}
\textbf{\pnua{nõõ-kò-je}}
\pfra{sa cheville}
\end{exemple}
\end{entrée}

\begin{entrée}{nõõli}{}{ⓔnõõli}
\formephonétique{̃̃ɳɔ̃ːli}
\région{GOs}
\variante{%
nõõli
\région{BO PA}}
(\domainesémantique{Fonctions naturelles humaines})
\classe{v}
\begin{glose}
\pfra{scruter ; regarder}
\end{glose}
\newline
\begin{exemple}
\région{BO}
\textbf{\pnua{nõõli nye ẽno}}
\pfra{regarde cet enfant}
\end{exemple}
\end{entrée}

\begin{entrée}{noo-za}{}{ⓔnoo-za}
\formephonétique{ɳoː-ða}
\région{GOs}
(\domainesémantique{Poissons})
\classe{nom}
\begin{glose}
\pfra{hareng des marais salants (lit. poisson du sel)}
\end{glose}
\end{entrée}

\begin{entrée}{nò-paa}{}{ⓔnò-paa}
\formephonétique{ɳɔ̃-paː}
\région{GOs}
(\domainesémantique{Poissons})
\classe{nom}
\begin{glose}
\pfra{poisson-pierre}
\end{glose}
\nomscientifique{Synanceja verrucosa}
\newline
\étymologie{
\langue{POc}
\étymon{*ɲopuq}}
\end{entrée}

\begin{entrée}{no phèńô}{}{ⓔno phèńô}
\formephonétique{ɳo pʰɛɳô}
\région{GOs}
(\domainesémantique{Fonctions naturelles humaines})
\classe{v}
\begin{glose}
\pfra{regarder furtivement}
\end{glose}
\newline
\relationsémantique{Cf.}{\lien{ⓔphèńô}{phèńô}}
\glosecourte{voler, dérober}
\end{entrée}

\begin{entrée}{nò-tòn}{}{ⓔnò-tòn}
\région{BO [BM]}
\classe{nom}
\newline
\sens{1}
(\domainesémantique{Végétation})
\begin{glose}
\pfra{broussailles ; maquis ; brousse}
\end{glose}
\newline
\begin{exemple}
\région{BO}
\textbf{\pnua{poxa nò-ton}}
\pfra{enfant illégitime [BM]}
\end{exemple}
\newline
\sens{2}
(\domainesémantique{Types de champs})
\begin{glose}
\pfra{jachère}
\end{glose}
\newline
\begin{exemple}
\région{BO}
\textbf{\pnua{i nò-tòn na mhenõ thu-poã}}
\pfra{le champ est laissé en jachère [BM]}
\end{exemple}
\newline
\étymologie{
\langue{POc}
\étymon{*talun}
\glosecourte{fallow land, land returning to 2ary growth}
\auteur{Blust}}
\end{entrée}

\begin{entrée}{no thiraò}{}{ⓔno thiraò}
\formephonétique{ɳo}
\région{GOs}
(\domainesémantique{Description des objets, formes, consistance, taille})
\classe{v}
\begin{glose}
\pfra{transparent (voir à travers)}
\end{glose}
\end{entrée}

\begin{entrée}{novwö ... ça}{}{ⓔnovwö ... ça}
\formephonétique{ɳoβω}
\région{GOs}
\variante{%
novwö ... ye
\région{BO PA}, 
nopo, novwu
\région{vx}}
(\domainesémantique{Structure informationnelle})
\classe{THEM}
\begin{glose}
\pfra{quant à ... alors}
\end{glose}
\newline
\begin{exemple}
\région{GO}
\textbf{\pnua{novwö lie meewu, ça li pe-kweli-li}}
\pfra{quant aux deux frères, ils se détestent}
\end{exemple}
\newline
\begin{exemple}
\région{PA}
\textbf{\pnua{novwö nyanya, ye i a-da Numia}}
\pfra{quant à maman, elle est partie àNouméa}
\end{exemple}
\end{entrée}

\begin{entrée}{novwö exa}{}{ⓔnovwö exa}
\formephonétique{̃̃ɳoβω}
\région{GOs}
\variante{%
novw-exa
\région{PA}}
(\domainesémantique{Conjonction})
\classe{CNJ}
\begin{glose}
\pfra{quand, lorsque (passé)}
\end{glose}
\newline
\begin{exemple}
\région{GO}
\textbf{\pnua{novwö (e)xa tho-da ilie (parents), axe novwö nye nye Mwani-mi, a-du}}
\pfra{alors qu'ils appellent (les parents), alors elle, Mwani-mi, elle sort de la maison}
\end{exemple}
\newline
\begin{exemple}
\région{GO}
\textbf{\pnua{xa novwö (e)xa whamã mwa ã ẽnõ-ò}}
\pfra{et lorsque ce garçon-là est devenu grand}
\end{exemple}
\newline
\relationsémantique{Cf.}{\lien{}{nou-na, novwö na}}
\glosecourte{quand, lorsque (futur)}
\end{entrée}

\begin{entrée}{novwö me}{}{ⓔnovwö me}
\formephonétique{̃̃ɳoβω}
\région{GO}
\variante{%
novwu
\région{PA}}
(\domainesémantique{Conjonction})
\classe{CNJ}
\begin{glose}
\pfra{quand}
\end{glose}
\newline
\begin{exemple}
\région{GO}
\textbf{\pnua{novwö me nu xa kha thoe hii-nu, axe balaa-ce za ne mwã kò}}
\pfra{Quand j'ai aussi tendu le bras en me déplaçant, alors (elle a lancé) un bout de bois et il a surgi une forêt}
\end{exemple}
\newline
\begin{exemple}
\textbf{\pnua{Novwu jena tèèn na yu ruma a-da}}
\pfra{le jour où tu viendras}
\end{exemple}
\end{entrée}

\begin{entrée}{novwö na .... ça}{}{ⓔnovwö na .... ça}
\formephonétique{ɳoβω ɳa}
\variante{%
novwu-na ... ye
\région{PA}, 
nou-na ... ye
\région{PA}, 
nopu
\région{vx}}
(\domainesémantique{Conjonction})
\classe{CNJ}
\begin{glose}
\pfra{quand ; lorsque (référence au futur)}
\end{glose}
\begin{glose}
\pfra{si (hypothétique)}
\end{glose}
\newline
\begin{exemple}
\région{GO}
\textbf{\pnua{novwö na mwã huu koi gèè, ca e ru mhê ?}}
\pfra{si nous mangeons le foie de grand-mère, est-ce qu'elle meurt ?}
\end{exemple}
\newline
\begin{exemple}
\région{GO}
\textbf{\pnua{novwö na ge je na-mhenõõ mani, ça/çe me zoma baa-je}}
\pfra{quand elle sera en train de dormir, nous la frapperons}
\end{exemple}
\newline
\begin{exemple}
\textbf{\pnua{novwö ni xa teen mwa na yu whili thoomwa, yu ra mwaju mwã ya avwònò}}
\pfra{si un jour tu trouves une épouse, tu reviendras alors à la maison}
\end{exemple}
\newline
\begin{exemple}
\région{PA}
\textbf{\pnua{nou na nu u nooli-xa kaareng na bwa ko, ye nu khôbwe "Kaavwo mwã jene"}}
\pfra{si je vois du brouillard dans la forêt, je dirai " ça c'est Kaavwo"}
\end{exemple}
\newline
\begin{exemple}
\région{PA}
\textbf{\pnua{nou na nu ruma nooli-xa kaareng na bwa ko, ye nu ruma khôbwe "Kaavwo mwã jene"}}
\pfra{si je vois du brouillard dans la forêt, je dirai " ça c'est Kaavwo"}
\end{exemple}
\newline
\begin{exemple}
\région{PA}
\textbf{\pnua{nowu na nu ruma hoxe havha, mi ruma phe bala-n}}
\pfra{quand je reviendrai, nous reprendrons la suite (la limite, là où nous nous sommes arrêtés)}
\end{exemple}
\newline
\begin{exemple}
\région{PA}
\textbf{\pnua{nowu na koi-nu, ye nu a-da Numia}}
\pfra{si je suis absent, c'est que je suis parti à Nouméa}
\end{exemple}
\newline
\relationsémantique{Cf.}{\lien{}{exa [PA]}}
\glosecourte{quand, lorsque (passé)}
\newline
\relationsémantique{Cf.}{\lien{}{novw(u)-exa [PA]}}
\glosecourte{quand, lorsque (référence au passé)}
\newline
\relationsémantique{Cf.}{\lien{}{(k)age novu}}
\glosecourte{pendant (Dubois)}
\newline
\relationsémantique{Cf.}{\lien{}{novu êga}}
\glosecourte{pendant (Dubois)}
\end{entrée}

\begin{entrée}{novwö na khõbwe}{}{ⓔnovwö na khõbwe}
\formephonétique{ɳoβω ɳa}
\région{GOs}
(\domainesémantique{Conjonction})
\classe{CNJ}
\begin{glose}
\pfra{si jamais}
\end{glose}
\newline
\begin{exemple}
\région{GO}
\textbf{\pnua{novwö na khõbwe çö trõne khõbwe ge-le-xa thoomwa xa Mwani-mii}}
\pfra{si jamais tu entends dire qu'il y a une femme du nom de Mwani-mi}
\end{exemple}
\end{entrée}

\begin{entrée}{nõ-wame}{}{ⓔnõ-wame}
\région{BO}
(\domainesémantique{Fonctions naturelles humaines})
\classe{v}
\begin{glose}
\pfra{voir mal [Corne]}
\end{glose}
\end{entrée}

\begin{entrée}{noyo}{}{ⓔnoyo}
\région{BO}
\classe{nom}
(\domainesémantique{Corps humain})
\begin{glose}
\pfra{sperme (Dubois)}
\end{glose}
\newline
\note{non vérifié}{général}{}
\end{entrée}

\begin{entrée}{nu}{1}{ⓔnuⓗ1}
\formephonétique{ɳũ}
\région{GOs}
\variante{%
nu
\région{BO PA}}
(\domainesémantique{Cocotiers})
\classe{nom}
\begin{glose}
\pfra{coco (noix de)}
\end{glose}
\begin{glose}
\pfra{cocotier}
\end{glose}
\nomscientifique{Cocos nucifera L.}
\newline
\begin{sous-entrée}{nu maû}{ⓔnuⓗ1ⓝnu maû}
\begin{glose}
\pfra{coco sec}
\end{glose}
\end{sous-entrée}
\newline
\begin{sous-entrée}{nu we}{ⓔnuⓗ1ⓝnu we}
\begin{glose}
\pfra{coco vert (contenant de l'eau)}
\end{glose}
\end{sous-entrée}
\newline
\begin{sous-entrée}{we nu}{ⓔnuⓗ1ⓝwe nu}
\begin{glose}
\pfra{eau de coco}
\end{glose}
\end{sous-entrée}
\newline
\begin{sous-entrée}{dixa nu}{ⓔnuⓗ1ⓝdixa nu}
\begin{glose}
\pfra{lait de coco}
\end{glose}
\end{sous-entrée}
\newline
\étymologie{
\langue{POc}
\étymon{*niuR}}
\end{entrée}

\begin{entrée}{nu}{2}{ⓔnuⓗ2}
\formephonétique{ɳu}
\région{GOs}
\variante{%
nu
\région{PA BO}}
(\domainesémantique{Pronoms})
\classe{PRO 1° pers. SG (sujet ou OBJ)}
\begin{glose}
\pfra{je}
\end{glose}
\end{entrée}

\begin{entrée}{nu}{3}{ⓔnuⓗ3}
\formephonétique{ɳũ}
\région{GOs PA}
\classe{nom}
\newline
\sens{1}
(\domainesémantique{Tressage})
\begin{glose}
\pfra{fibres longues ; lanières}
\end{glose}
\begin{glose}
\pfra{tresses de fibres de pandanus}
\end{glose}
\newline
\begin{exemple}
\textbf{\pnua{e töö nu-phoo bee}}
\pfra{elle coupe des fibres fraîches de pandanus}
\end{exemple}
\newline
\sens{2}
(\domainesémantique{Corps humain})
\begin{glose}
\pfra{mèches de cheveux}
\end{glose}
\newline
\begin{exemple}
\région{GO}
\textbf{\pnua{nu-pu-bwaa-je}}
\pfra{sa mèche de cheveux}
\end{exemple}
\newline
\begin{exemple}
\région{PA}
\textbf{\pnua{nu-pu-bwaa-n}}
\pfra{sa mèche de cheveux}
\end{exemple}
\end{entrée}

\begin{entrée}{-nu}{}{ⓔ-nu}
\formephonétique{ɳu}
\région{GOs}
(\domainesémantique{Pronoms})
\classe{PRO 1° pers. SG (OBJ ou POSS)}
\begin{glose}
\pfra{me ; mon ; ma ; mes}
\end{glose}
\newline
\begin{exemple}
\textbf{\pnua{hii-nu}}
\pfra{mon bras}
\end{exemple}
\end{entrée}

\begin{entrée}{nuãda}{}{ⓔnuãda}
\formephonétique{ɳuãnda}
\région{GOs PA}
(\domainesémantique{Arbre})
\classe{nom}
\begin{glose}
\pfra{palmier calédonien (à croissance rapide)}
\end{glose}
\end{entrée}

\begin{entrée}{nube}{}{ⓔnube}
\formephonétique{ɳumbe}
\région{GOs}
(\domainesémantique{Verbes de mouvement})
\classe{v}
\begin{glose}
\pfra{faufiler (se)}
\end{glose}
\begin{glose}
\pfra{glisser (se)}
\end{glose}
\newline
\begin{exemple}
\textbf{\pnua{e nube}}
\pfra{il s'éclipse sans se faire remarquer, il prend la tangeante}
\end{exemple}
\end{entrée}

\begin{entrée}{nu hêgi}{}{ⓔnu hêgi}
\formephonétique{ɳu hêŋgi}
\région{GOs PA}
(\domainesémantique{Objets coutumiers})
\classe{nom}
\begin{glose}
\pfra{longueur de monnaie}
\end{glose}
\newline
\begin{exemple}
\région{PA}
\textbf{\pnua{we-xe nu hêgi}}
\pfra{une longueur de monnaie}
\end{exemple}
\end{entrée}

\begin{entrée}{nu-ki}{}{ⓔnu-ki}
\formephonétique{ɳu-ki}
\région{GOs}
\variante{%
nu-kim
\région{PA}}
(\domainesémantique{Cocotiers})
\classe{nom}
\begin{glose}
\pfra{coco germé}
\end{glose}
\end{entrée}

\begin{entrée}{nu-mãû}{}{ⓔnu-mãû}
\formephonétique{ɳu-mãû}
\région{GOs}
(\domainesémantique{Cocotiers})
\classe{nom}
\begin{glose}
\pfra{coco sec}
\end{glose}
\newline
\relationsémantique{Cf.}{\lien{ⓔnu-wee}{nu-wee}}
\glosecourte{coco vert}
\end{entrée}

\begin{entrée}{nuu}{}{ⓔnuu}
\région{BO}
(\domainesémantique{Parenté})
\classe{nom}
\begin{glose}
\pfra{famille (sans doute lié au cocotier) [BM]}
\end{glose}
\newline
\note{non verifié}{général}{}
\end{entrée}

\begin{entrée}{nûû}{}{ⓔnûû}
\formephonétique{ɳûː}
\région{GOs BO PA}
\newline
\groupe{A}
\classe{v}
\newline
\sens{1}
(\domainesémantique{Lumière et obscurité})
\begin{glose}
\pfra{éclairer (avec une lampe)}
\end{glose}
\newline
\begin{sous-entrée}{nûû xo ya-ka}{ⓔnûûⓢ1ⓝnûû xo ya-ka}
\région{GO}
\begin{glose}
\pfra{éclairer avec une lampe électrique (litlumière-appuyer)}
\end{glose}
\newline
\begin{exemple}
\région{BO}
\textbf{\pnua{nûûe mwa !}}
\pfra{éclaire !}
\end{exemple}
\newline
\relationsémantique{Cf.}{\lien{}{nûûe}}
\glosecourte{éclairer qqch}
\end{sous-entrée}
\newline
\sens{2}
(\domainesémantique{Pêche})
\begin{glose}
\pfra{pêcher à la torche}
\end{glose}
\newline
\begin{exemple}
\région{BO}
\textbf{\pnua{li thu nûû}}
\pfra{ils pêchent à la torche}
\end{exemple}
\newline
\begin{exemple}
\région{BO}
\textbf{\pnua{li tu nûû}}
\pfra{ils descendent pêcher à la torche}
\end{exemple}
\newline
\groupe{B}
\newline
\sens{3}
(\domainesémantique{Lumière et obscurité})
\classe{nom}
\begin{glose}
\pfra{torche}
\end{glose}
\newline
\begin{exemple}
\région{BO}
\textbf{\pnua{nûûa-n}}
\pfra{sa torche}
\end{exemple}
\newline
\étymologie{
\langue{PWMP}
\étymon{*(me-)ɲuluq}}
\end{entrée}

\begin{entrée}{nuu-ce}{}{ⓔnuu-ce}
\formephonétique{ɳuː-ce}
\région{GOs PA}
(\domainesémantique{Bois})
\classe{nom}
\begin{glose}
\pfra{écharde}
\end{glose}
\end{entrée}

\begin{entrée}{nuu-phò}{}{ⓔnuu-phò}
\formephonétique{ɳuː-pʰɔ}
\région{GOs}
(\domainesémantique{Tressage})
\classe{nom}
\begin{glose}
\pfra{lanières de pandanus (pour le tressage)}
\end{glose}
\begin{glose}
\pfra{fibres}
\end{glose}
\end{entrée}

\begin{entrée}{nu-wee}{}{ⓔnu-wee}
\formephonétique{ɳu-weː}
\région{GOs}
(\domainesémantique{Cocotiers})
\classe{nom}
\begin{glose}
\pfra{coco vert}
\end{glose}
\end{entrée}

\newpage

\lettrine{
ńh
nh
}\begin{entrée}{nhã}{}{ⓔnhã}
\région{GOs WEM WE BO PA}
(\domainesémantique{Fonctions naturelles humaines})
\classe{nom}
\begin{glose}
\pfra{crotte ; excréments}
\end{glose}
\newline
\begin{exemple}
\région{GO}
\textbf{\pnua{nho-chòva}}
\pfra{crotin de cheval}
\end{exemple}
\newline
\begin{exemple}
\région{GO}
\textbf{\pnua{nho-ãgu}}
\pfra{crotte}
\end{exemple}
\newline
\begin{exemple}
\région{GO}
\textbf{\pnua{nhò-je}}
\pfra{ses excréments}
\end{exemple}
\newline
\note{nho- en composition}{grammaire}{}
\end{entrée}

\begin{entrée}{nhe}{1}{ⓔnheⓗ1}
\formephonétique{nʰe}
\région{PA}
(\domainesémantique{Feu : objets et actions liés au feu})
\classe{nom}
\begin{glose}
\pfra{bûche}
\end{glose}
\end{entrée}

\begin{entrée}{nhe}{2}{ⓔnheⓗ2}
\formephonétique{ɳʰe}
\région{GOs}
\variante{%
nhe
\région{PA BO}}
(\domainesémantique{Navigation})
\classe{nom}
\begin{glose}
\pfra{voile (bateau) ; bâche}
\end{glose}
\newline
\begin{sous-entrée}{pwepwede nhe}{ⓔnheⓗ2ⓝpwepwede nhe}
\begin{glose}
\pfra{virer de bord par vent arrière}
\end{glose}
\end{sous-entrée}
\newline
\begin{sous-entrée}{nhewa-wô}{ⓔnheⓗ2ⓝnhewa-wô}
\région{GO}
\begin{glose}
\pfra{la voile du bateau}
\end{glose}
\end{sous-entrée}
\newline
\begin{sous-entrée}{nhe-a wòny}{ⓔnheⓗ2ⓝnhe-a wòny}
\région{BO}
\begin{glose}
\pfra{la voile du bateau}
\end{glose}
\end{sous-entrée}
\newline
\begin{sous-entrée}{wòny nhe}{ⓔnheⓗ2ⓝwòny nhe}
\région{BO}
\begin{glose}
\pfra{bateau à voile}
\end{glose}
\end{sous-entrée}
\newline
\étymologie{
\langue{POc}
\étymon{*layaR}
\glosecourte{voile}}
\end{entrée}

\begin{entrée}{nhei}{}{ⓔnhei}
\formephonétique{ɳʰei}
\région{GOs GA}
\variante{%
nhei
\région{PA}}
(\domainesémantique{Fonctions naturelles humaines})
\classe{v}
\begin{glose}
\pfra{crampe (avoir une)}
\end{glose}
\newline
\begin{exemple}
\région{GO}
\textbf{\pnua{e nhei kòò-nu}}
\pfra{j'ai une crampe à la jambe}
\end{exemple}
\newline
\begin{exemple}
\textbf{\pnua{mã nhei}}
\pfra{(lit. maladie crampe)}
\end{exemple}
\end{entrée}

\begin{entrée}{nhi}{1}{ⓔnhiⓗ1}
\région{BO}
(\domainesémantique{Description des objets, formes, consistance, taille})
\classe{v}
\begin{glose}
\pfra{crisser (sous la dent) ; abrasif}
\end{glose}
\newline
\begin{exemple}
\région{BO}
\textbf{\pnua{i hni la hovo}}
\pfra{la nourriture crisse sous la dent}
\end{exemple}
\end{entrée}

\begin{entrée}{nhi}{2}{ⓔnhiⓗ2}
\formephonétique{nʰi}
\région{BO [BM]}
(\domainesémantique{Pierre, roche})
\classe{nom}
\begin{glose}
\pfra{roc ; rocher calcaire}
\end{glose}
\newline
\begin{sous-entrée}{pa-hni}{ⓔnhiⓗ2ⓝpa-hni}
\begin{glose}
\pfra{rocher calcaire (Dubois)}
\end{glose}
\end{sous-entrée}
\end{entrée}

\begin{entrée}{nhi}{3}{ⓔnhiⓗ3}
\formephonétique{ɳʰi}
\région{GOs}
\variante{%
nhil
\région{PA BO}}
(\domainesémantique{Fonctions naturelles humaines})
\classe{v}
\begin{glose}
\pfra{moucher (se)}
\end{glose}
\begin{glose}
\pfra{renifler [PA]}
\end{glose}
\begin{glose}
\pfra{renâcler (cheval)}
\end{glose}
\newline
\begin{exemple}
\région{GO}
\textbf{\pnua{nu nhi}}
\pfra{je me mouche !}
\end{exemple}
\newline
\begin{exemple}
\région{GO}
\textbf{\pnua{nhile têi-jö}}
\pfra{mouche-toi ! (lit. mouche ta morve)}
\end{exemple}
\newline
\begin{exemple}
\région{GO}
\textbf{\pnua{nu nhile têi-nu}}
\pfra{je me mouche ! (lit. je mouche ma morve)}
\end{exemple}
\newline
\note{nhile (v.t.)}{grammaire}{moucher qqch}
\end{entrée}

\begin{entrée}{nhii}{}{ⓔnhii}
\formephonétique{ɳʰiː}
\région{GOs}
\variante{%
nhii
\région{PA BO}}
(\domainesémantique{Actions liées aux plantes})
\classe{v}
\begin{glose}
\pfra{cueillir à la main (fruit, baies)}
\end{glose}
\newline
\begin{exemple}
\région{GO}
\textbf{\pnua{mo a nhii-vwo}}
\pfra{nous allons faire la cueillette}
\end{exemple}
\end{entrée}

\begin{entrée}{nhiida}{}{ⓔnhiida}
\formephonétique{ɳʰiːnda}
\région{GOs}
\variante{%
niida
\région{BO}}
(\domainesémantique{Insectes})
\classe{nom}
\begin{glose}
\pfra{lentes}
\end{glose}
\newline
\étymologie{
\langue{POc}
\étymon{*li(n)sa}
\glosecourte{lentes}}
\end{entrée}

\begin{entrée}{nhiiji}{}{ⓔnhiiji}
\région{BO PA}
\variante{%
nhiije
\région{PA BO}}
(\domainesémantique{Chasse})
\classe{nom}
\begin{glose}
\pfra{lacet (pour prendre les oiseaux dans les arbres)}
\end{glose}
\begin{glose}
\pfra{piège à oiseau ; collet}
\end{glose}
\begin{glose}
\pfra{collet}
\end{glose}
\end{entrée}

\begin{entrée}{ńho}{}{ⓔńho}
\formephonétique{nʰo}
\région{GOs}
(\domainesémantique{Ignames})
\classe{nom}
\begin{glose}
\pfra{partie supérieure du tubercule d'igname}
\end{glose}
\newline
\note{(cette partie supérieure avec les lianes est replantée après prélèvement du bas du tubercule)}{glose}{}
\newline
\note{ńhome (v.t.)}{grammaire}{couper et prélever}
\end{entrée}

\begin{entrée}{ńhõ-}{}{ⓔńhõ-}
\formephonétique{nʰɔ̃}
\région{GOs PA BO}
(\domainesémantique{Fonctions naturelles humaines})
\classe{nom}
\begin{glose}
\pfra{crotte (forme déterminé)}
\end{glose}
\newline
\begin{exemple}
\région{GO}
\textbf{\pnua{ńhõ-chòvwa}}
\pfra{crotin de cheval}
\end{exemple}
\newline
\begin{exemple}
\région{PA BO}
\textbf{\pnua{nhõõ-m}}
\pfra{ta crotte}
\end{exemple}
\newline
\begin{exemple}
\région{WEM WE}
\textbf{\pnua{e nhõõ-m}}
\pfra{bien fait pour toi ! (= ta crotte)}
\end{exemple}
\newline
\begin{exemple}
\région{PA BO}
\textbf{\pnua{i nhõ-n !}}
\pfra{bien fait pour lui !}
\end{exemple}
\end{entrée}

\begin{entrée}{ńhôã}{}{ⓔńhôã}
\formephonétique{nʰɔ̃ã}
\région{GOs}
(\domainesémantique{Santé, maladie})
\classe{v}
\begin{glose}
\pfra{constipé}
\end{glose}
\newline
\begin{exemple}
\textbf{\pnua{e ńhôã}}
\pfra{il est constipé}
\end{exemple}
\end{entrée}

\begin{entrée}{ńhõbe}{}{ⓔńhõbe}
\formephonétique{nʰɔ̃mbe}
\région{GOs}
(\domainesémantique{Fonctions naturelles humaines})
\classe{v}
\begin{glose}
\pfra{crampe (avoir une)}
\end{glose}
\end{entrée}

\begin{entrée}{nhõginy}{}{ⓔnhõginy}
\formephonétique{nʰɔ̃ŋgiɲ}
\région{PA}
\variante{%
nõginy
\région{BO}}
(\domainesémantique{Insectes})
\classe{nom}
\begin{glose}
\pfra{araignée (de terre, noire, grosse)}
\end{glose}
\end{entrée}

\begin{entrée}{nhõî}{}{ⓔnhõî}
\formephonétique{ɳʰɔ̃î}
\région{GOs PA}
\variante{%
nhõî, nhõõî
\région{BO}}
(\domainesémantique{Cultures, techniques, boutures
, Mouvements ou actions faits avec le corps, les bras, les mains, les pieds})
\classe{v}
\begin{glose}
\pfra{lier ; ligoter ; attacher}
\end{glose}
\begin{glose}
\pfra{attacher la tige d'igname}
\end{glose}
\newline
\begin{exemple}
\région{GO}
\textbf{\pnua{e pe-nhõi-le wa}}
\pfra{il a attaché les cordes l'une à l'autre}
\end{exemple}
\end{entrée}

\begin{entrée}{ńhòme kui}{}{ⓔńhòme kui}
\formephonétique{nʰɔme}
\région{GOs}
(\domainesémantique{Cultures, techniques, boutures})
\classe{v}
\begin{glose}
\pfra{couper et prélever le bas du tubercule d'igname et replanter la partie supérieure avec ses lianes}
\end{glose}
\newline
\note{cette opération se fait quand l'igname est encore verte}{glose}{}
\end{entrée}

\begin{entrée}{nhõõl}{}{ⓔnhõõl}
\région{BO PA}
\variante{%
nõõl
\région{BO PA}}
(\domainesémantique{Richesses, monnaies traditionnelles})
\classe{v}
\begin{glose}
\pfra{argent ; monnaie}
\end{glose}
\begin{glose}
\pfra{richesses ; biens}
\end{glose}
\newline
\begin{sous-entrée}{nhõõ mii}{ⓔnhõõlⓝnhõõ mii}
\begin{glose}
\pfra{argent européen}
\end{glose}
\newline
\begin{exemple}
\textbf{\pnua{nhõõla-ny}}
\pfra{mes biens}
\end{exemple}
\newline
\begin{exemple}
\textbf{\pnua{nhõõla-ri ?}}
\pfra{à qui sont ces biens ?}
\end{exemple}
\newline
\begin{exemple}
\textbf{\pnua{nhõla-n}}
\pfra{ses affaires}
\end{exemple}
\end{sous-entrée}
\newline
\begin{sous-entrée}{nho pujo/pulo}{ⓔnhõõlⓝnho pujo/pulo}
\région{BO}
\begin{glose}
\pfra{monnaie blanche (argent)}
\end{glose}
\end{sous-entrée}
\newline
\begin{sous-entrée}{nho mii}{ⓔnhõõlⓝnho mii}
\région{BO}
\begin{glose}
\pfra{monnaie d'Europe (cuivre)}
\end{glose}
\end{sous-entrée}
\end{entrée}

\begin{entrée}{nhõõxi}{}{ⓔnhõõxi}
\formephonétique{ɳʰɔ̃ːɣi}
\région{GOs}
(\domainesémantique{Préparation des aliments; modes de préparation et de cuisson})
\classe{v}
\begin{glose}
\pfra{envelopper et attacher (nourriture)}
\end{glose}
\newline
\begin{exemple}
\région{BO}
\textbf{\pnua{nhõõxi mwata}}
\pfra{envelopper du 'mwata'}
\end{exemple}
\end{entrée}

\begin{entrée}{nhu}{}{ⓔnhu}
\formephonétique{ɳʰu}
\région{GOs}
\variante{%
nhu
\région{BO}}
(\domainesémantique{Verbes de mouvement})
\classe{v}
\begin{glose}
\pfra{écrouler (s')}
\end{glose}
\begin{glose}
\pfra{dégringoler}
\end{glose}
\begin{glose}
\pfra{glisser}
\end{glose}
\newline
\begin{exemple}
\région{GO}
\textbf{\pnua{e nhu dili na èna}}
\pfra{le terrain a glissé à cet endroit-là}
\end{exemple}
\newline
\begin{exemple}
\région{BO}
\textbf{\pnua{i nhu dili na èna}}
\pfra{le terrain a glissé à cet endroit-là}
\end{exemple}
\end{entrée}

\begin{entrée}{nhuã}{}{ⓔnhuã}
\formephonétique{ɳʰuɛ̃}
\formephonétique{ɳuɛ̃}
\région{GOs}
\variante{%
nuã
, 
nhuã, nuã
\région{BO PA}}
(\domainesémantique{Mouvements ou actions faits avec le corps, les bras, les mains, les pieds})
\classe{v}
\begin{glose}
\pfra{lâcher ; relâcher}
\end{glose}
\begin{glose}
\pfra{laisser tomber}
\end{glose}
\newline
\begin{exemple}
\région{GO}
\textbf{\pnua{nhuã-nu !}}
\pfra{lâche-moi !}
\end{exemple}
\newline
\begin{exemple}
\textbf{\pnua{e nhuã cova}}
\pfra{il a relâché le cheval}
\end{exemple}
\newline
\begin{exemple}
\textbf{\pnua{nhuã kô-choova !}}
\pfra{lâche la longe du cheval !}
\end{exemple}
\newline
\begin{sous-entrée}{paa-n(h)uã}{ⓔnhuãⓝpaa-n(h)uã}
\région{PA BO}
\begin{glose}
\pfra{laisser partir, relâcher}
\end{glose}
\end{sous-entrée}
\end{entrée}

\newpage

\lettrine{ny}\begin{entrée}{-ny}{}{ⓔ-ny}
\région{BO PA}
(\domainesémantique{Pronoms})
\classe{SUFF.POSS 1° pers.}
\begin{glose}
\pfra{mon ; ma ; mes}
\end{glose}
\end{entrée}

\begin{entrée}{nyãã}{}{ⓔnyãã}
\région{GOs BO}
(\domainesémantique{Terre})
\classe{nom}
\begin{glose}
\pfra{terre d'alluvion}
\end{glose}
\end{entrée}

\begin{entrée}{nyaanya}{}{ⓔnyaanya}
\région{BO}
(\domainesémantique{Verbes de mouvement})
\classe{v}
\begin{glose}
\pfra{dandiner (se); se balancer (en marchant) [Corne]}
\end{glose}
\end{entrée}

\begin{entrée}{nyamã}{}{ⓔnyamã}
\région{GOs PA BO}
\classe{v ; n}
\newline
\sens{1}
(\domainesémantique{Verbes de mouvement})
\begin{glose}
\pfra{bouger ; remuer}
\end{glose}
\newline
\begin{exemple}
\région{GO PA}
\textbf{\pnua{kêbwa nyamã !}}
\pfra{ne bouge pas !}
\end{exemple}
\newline
\sens{2}
(\domainesémantique{Verbes d'action (en général)})
\begin{glose}
\pfra{travailler ; travail}
\end{glose}
\newline
\begin{exemple}
\région{PA BO}
\région{BO}
\textbf{\pnua{la me-nyamã i nu}}
\pfra{mes travaux, oeuvres (Dubois)}
\end{exemple}
\newline
\begin{exemple}
\région{BO}
\textbf{\pnua{thu mhèno-nyamã i nu}}
\pfra{j'ai du travail}
\end{exemple}
\newline
\begin{sous-entrée}{a-nyama}{ⓔnyamãⓢ2ⓝa-nyama}
\begin{glose}
\pfra{travailleur}
\end{glose}
\newline
\note{nyamãle (v.t.)}{grammaire}{}
\end{sous-entrée}
\newline
\note{"nyamã" en GO n'a que le sens de 'bouger'}{général}{}
\end{entrée}

\begin{entrée}{nyamãle}{}{ⓔnyamãle}
\région{GOs}
(\domainesémantique{Verbes de mouvement})
\classe{v}
\begin{glose}
\pfra{bouger (un objet)}
\end{glose}
\end{entrée}

\begin{entrée}{nyãnume}{}{ⓔnyãnume}
\région{GOs}
(\domainesémantique{Mouvements ou actions faits avec le corps, les bras, les mains, les pieds})
\classe{v}
\begin{glose}
\pfra{signe de la main (faire un)}
\end{glose}
\newline
\begin{exemple}
\textbf{\pnua{nu nyãnume ije vwo/pu a-mi}}
\pfra{je lui ai fait signe de s'approcher}
\end{exemple}
\end{entrée}

\begin{entrée}{nyãnyã}{}{ⓔnyãnyã}
\région{GOs PA BO}
(\domainesémantique{Appellation parenté})
\classe{nom}
\begin{glose}
\pfra{maman ; tante maternelle}
\end{glose}
\newline
\begin{exemple}
\textbf{\pnua{nyãnyã i je}}
\pfra{sa mère}
\end{exemple}
\newline
\begin{exemple}
\textbf{\pnua{nyãnyã whamã}}
\pfra{la tante maternelle la plus agée}
\end{exemple}
\end{entrée}

\begin{entrée}{nye}{}{ⓔnye}
\formephonétique{ɲɛ}
\région{GOs}
\variante{%
nyèn
\formephonétique{ɲɛn}
\région{PA BO}, 
nhyèn
\région{WEM}}
(\domainesémantique{Noms des plantes})
\classe{nom}
\begin{glose}
\pfra{curcuma ; gingembre (comestible)}
\end{glose}
\newline
\étymologie{
\langue{POc}
\étymon{*yaŋo}
\glosecourte{turmeric}
\auteur{Blust}}
\end{entrée}

\begin{entrée}{nyejo!}{}{ⓔnyejo!}
\région{PA}
(\domainesémantique{Appellation parenté
, Vocatifs})
\classe{vocatif}
\begin{glose}
\pfra{maman!}
\end{glose}
\end{entrée}

\begin{entrée}{nye-na}{}{ⓔnye-na}
\formephonétique{ɲe-ɳa}
\région{GOs}
(\domainesémantique{Démonstratifs})
\classe{ANAPH}
\begin{glose}
\pfra{cela}
\end{glose}
\newline
\begin{exemple}
\textbf{\pnua{we a eniza ? - Me a nye-na !}}
\pfra{quand partez-vous ? - Nous partons maintenant !}
\end{exemple}
\end{entrée}

\begin{entrée}{nyèn êmwen}{}{ⓔnyèn êmwen}
\région{PA}
(\domainesémantique{Noms des plantes})
\classe{nom}
\begin{glose}
\pfra{gingembre mâle non comestible}
\end{glose}
\newline
\note{(il a une sorte de bourgeon qui fleurit rouge)}{glose}{}
\end{entrée}

\begin{entrée}{nye, nyã}{}{ⓔnye, nyã}
\région{GOs PA}
\variante{%
nyã ; nye-ã
\région{GO(s) PA}}
(\domainesémantique{Démonstratifs})
\classe{DET; PRO.DEIC.1}
\begin{glose}
\pfra{ce ... -ci; cela}
\end{glose}
\newline
\begin{exemple}
\région{GO}
\textbf{\pnua{nye kuau}}
\pfra{ce chien}
\end{exemple}
\newline
\begin{exemple}
\région{GO}
\textbf{\pnua{da nye pwaixe ba ?}}
\pfra{quelle est cette chose là-bas ?}
\end{exemple}
\newline
\begin{exemple}
\région{GO}
\textbf{\pnua{da nye ku ?}}
\pfra{qu'est-ce qui est tombé?}
\end{exemple}
\newline
\begin{exemple}
\région{GO}
\textbf{\pnua{ti nye a ? - za inu nye a}}
\pfra{qui est-ce qui part ? - c'est moi qui pars}
\end{exemple}
\newline
\begin{exemple}
\région{PA}
\textbf{\pnua{phe nyã}}
\pfra{prends ceci ! (en le tendant vers l'interlocuteur)}
\end{exemple}
\newline
\begin{exemple}
\région{GO}
\textbf{\pnua{da nyã ku ?}}
\pfra{qu'est-ce qui est tombé?}
\end{exemple}
\newline
\begin{exemple}
\textbf{\pnua{da nyã i kul ?}}
\pfra{qu'est-ce qui est tombé?}
\end{exemple}
\newline
\begin{exemple}
\textbf{\pnua{we a eniza ? - Me a nye !}}
\pfra{quand partez-vous ? - Nous partons tout de suite !}
\end{exemple}
\newline
\begin{sous-entrée}{nye-ni}{ⓔnye, nyãⓝnye-ni}
\begin{glose}
\pfra{cette chose-là (DX2)}
\end{glose}
\end{sous-entrée}
\newline
\begin{sous-entrée}{nye-ba}{ⓔnye, nyãⓝnye-ba}
\begin{glose}
\pfra{cette chose-là sur le côté (DX2)}
\end{glose}
\end{sous-entrée}
\newline
\begin{sous-entrée}{nye-õli}{ⓔnye, nyãⓝnye-õli}
\begin{glose}
\pfra{cette chose-là-bas (DX3)}
\end{glose}
\end{sous-entrée}
\newline
\begin{sous-entrée}{nye-du mu}{ⓔnye, nyãⓝnye-du mu}
\begin{glose}
\pfra{cette chose derrière}
\end{glose}
\end{sous-entrée}
\newline
\begin{sous-entrée}{nye-du}{ⓔnye, nyãⓝnye-du}
\begin{glose}
\pfra{cette chose en bas,}
\end{glose}
\end{sous-entrée}
\newline
\begin{sous-entrée}{nye-bòli}{ⓔnye, nyãⓝnye-bòli}
\begin{glose}
\pfra{cette chose loin en bas}
\end{glose}
\end{sous-entrée}
\newline
\begin{sous-entrée}{nye-da}{ⓔnye, nyãⓝnye-da}
\begin{glose}
\pfra{cette chose en haut}
\end{glose}
\end{sous-entrée}
\end{entrée}

\begin{entrée}{nyiwã}{}{ⓔnyiwã}
\région{GOs}
(\domainesémantique{Préfixes et verbes de position})
\classe{v.stat.}
\begin{glose}
\pfra{recroquevillé}
\end{glose}
\newline
\begin{exemple}
\textbf{\pnua{e kô-nyiwã}}
\pfra{il dort recroquevillé}
\end{exemple}
\end{entrée}

\begin{entrée}{nyò}{}{ⓔnyò}
\région{GOs}
(\domainesémantique{Démonstratifs})
\classe{DEM.ANAPH}
\begin{glose}
\pfra{cela (en question)}
\end{glose}
\newline
\morphologie{nye-ò}
\end{entrée}

\begin{entrée}{nyõli}{}{ⓔnyõli}
\région{GOs WEM WE PA BO}
(\domainesémantique{Démonstratifs})
\classe{DEM.DEIC.3 (distal)}
\begin{glose}
\pfra{cela là-bas (loin des interlocuteurs)}
\end{glose}
\newline
\begin{exemple}
\textbf{\pnua{kwau-õli !}}
\pfra{le chien là-bas}
\end{exemple}
\newline
\begin{exemple}
\textbf{\pnua{èmwê-õli !}}
\pfra{l'homme là-bas}
\end{exemple}
\newline
\begin{exemple}
\région{BO}
\textbf{\pnua{ti nyõli ègu-õli !}}
\pfra{qui est-ce là-bas ?}
\end{exemple}
\newline
\relationsémantique{Cf.}{\lien{ⓔnyõli mwa}{nyõli mwa}}
\glosecourte{cela loin là-bas}
\newline
\morphologie{nye-õli}
\end{entrée}

\begin{entrée}{nyõli mwa}{}{ⓔnyõli mwa}
\région{PA}
(\domainesémantique{Démonstratifs})
\classe{DEM.DEIC.4}
\begin{glose}
\pfra{là-bas très loin}
\end{glose}
\end{entrée}

\begin{entrée}{nyòlò}{}{ⓔnyòlò}
\région{PA}
\variante{%
nyolõng
\région{PA}}
(\domainesémantique{Mouvements ou actions avec la tête, les yeux, la bouche})
\classe{v}
\begin{glose}
\pfra{grimacer ; faire des grimaces}
\end{glose}
\newline
\begin{exemple}
\textbf{\pnua{i nyolõng}}
\pfra{il fait des grimaces}
\end{exemple}
\end{entrée}

\newpage

\lettrine{nhy}\begin{entrée}{nhya}{}{ⓔnhya}
\formephonétique{ɲʰa}
\région{GOs}
\variante{%
nhyal
\région{PA}}
(\domainesémantique{Description des objets, formes, consistance, taille})
\classe{v.stat.}
\begin{glose}
\pfra{écrasé ; mou (une fois écrasé)}
\end{glose}
\newline
\begin{exemple}
\région{PA}
\textbf{\pnua{nhyal ma nhyal}}
\pfra{totalement mou}
\end{exemple}
\end{entrée}

\begin{entrée}{nhyã}{}{ⓔnhyã}
(\domainesémantique{Pêche})
\classe{nom}
\begin{glose}
\pfra{appâts ; amorces (pêche)}
\end{glose}
\newline
\begin{exemple}
\région{GOs PA BO}
\région{BO}
\textbf{\pnua{nhyã-m}}
\pfra{ton amorce}
\end{exemple}
\end{entrée}

\begin{entrée}{nhyãã}{}{ⓔnhyãã}
\région{GOs}
\variante{%
nhyang
\région{PA BO}}
(\domainesémantique{Coutumes, dons coutumiers})
\classe{n ; v}
\begin{glose}
\pfra{coutume (cérémonie coutumière) ; fête ; occupations}
\end{glose}
\begin{glose}
\pfra{occuper (s')}
\end{glose}
\newline
\begin{exemple}
\région{PA}
\textbf{\pnua{pe-nhyang}}
\pfra{s'occuper avec qqch}
\end{exemple}
\newline
\note{nhyaga}{grammaire}{fête de}
\newline
\note{nhya ponga / poxa ...}{grammaire}{fête pour / de qqch}
\end{entrée}

\begin{entrée}{nhyal}{}{ⓔnhyal}
\région{PA BO}
\classe{v.stat.}
\newline
\sens{1}
(\domainesémantique{Description des objets, formes, consistance, taille})
\begin{glose}
\pfra{mou}
\end{glose}
\begin{glose}
\pfra{gâté (fruit)}
\end{glose}
\newline
\sens{2}
(\domainesémantique{Santé, maladie})
\begin{glose}
\pfra{purulent (bobo)}
\end{glose}
\newline
\relationsémantique{Cf.}{\lien{ⓔnhyale}{nhyale}}
\glosecourte{ramollir}
\end{entrée}

\begin{entrée}{nhyale}{}{ⓔnhyale}
\région{GOs BO PA}
\classe{v}
\newline
\sens{1}
(\domainesémantique{Mouvements ou actions faits avec le corps, les bras, les mains, les pieds})
\begin{glose}
\pfra{écraser (dans la main) ; ramollir}
\end{glose}
\newline
\begin{sous-entrée}{nhyal}{ⓔnhyaleⓢ1ⓝnhyal}
\begin{glose}
\pfra{mou}
\end{glose}
\end{sous-entrée}
\newline
\sens{2}
(\domainesémantique{Cultures, techniques, boutures})
\begin{glose}
\pfra{émotter}
\end{glose}
\newline
\begin{sous-entrée}{nhyale dili}{ⓔnhyaleⓢ2ⓝnhyale dili}
\begin{glose}
\pfra{émotter}
\end{glose}
\end{sous-entrée}
\end{entrée}

\begin{entrée}{nhyanga-n}{}{ⓔnhyanga-n}
\formephonétique{ɲʰaŋgan}
\région{PA BO [BM]}
(\domainesémantique{Coutumes, dons coutumiers})
\classe{nom}
\begin{glose}
\pfra{don à la mère pour avoir élevé un garçon (se fait vers 7-8 ans)}
\end{glose}
\end{entrée}

\begin{entrée}{nhyang mòlò}{}{ⓔnhyang mòlò}
\région{PA}
(\domainesémantique{Coutumes, dons coutumiers})
\classe{nom}
\begin{glose}
\pfra{coutumes (cérémonies)}
\end{glose}
\newline
\note{(ensemble de) ou de dons coutumiers qui accompagnent le développement de l'enfant de la conception jusqu'au moment où il porte le manou et qui sont offerts au clan maternel (à l'oncle maternel). Ces gestes coutumiers ne sont faits que pour le premier né.}{glose}{}
\end{entrée}

\begin{entrée}{nhyang mhã}{}{ⓔnhyang mhã}
\formephonétique{ɲʰaŋ}
\région{PA}
(\domainesémantique{Coutumes, dons coutumiers})
\classe{nom}
\begin{glose}
\pfra{coutumes (cérémonies)}
\end{glose}
\newline
\note{(ensemble de) ou de dons coutumiers qui accompagnent les signes de déclin de la personne jusqu'à sa mort et qui sont offerts au clan maternel (à l'oncle maternel)}{glose}{}
\newline
\relationsémantique{Cf.}{\lien{}{mhãu}}
\end{entrée}

\begin{entrée}{nhya, nhye}{}{ⓔnhya, nhye}
\région{GOs PA BO}
(\domainesémantique{Démonstratifs})
\classe{ART}
\begin{glose}
\pfra{le, la, les ; ceci}
\end{glose}
\newline
\begin{exemple}
\région{GO}
\textbf{\pnua{nòòli nhya nò !}}
\pfra{regarde ce poisson !}
\end{exemple}
\end{entrée}

\begin{entrée}{nhyatru}{}{ⓔnhyatru}
\formephonétique{ɲʰaʈu}
\région{GOs}
\variante{%
nhyaru
\formephonétique{ɲʰaɽu}
\région{GO(s)}}
\classe{v.stat.}
\newline
\sens{1}
(\domainesémantique{Description des objets, formes, consistance, taille})
\begin{glose}
\pfra{mou ; trop mûr}
\end{glose}
\newline
\begin{sous-entrée}{pwaji nhyatru, nhyaru [GOs]}{ⓔnhyatruⓢ1ⓝpwaji nhyatru, nhyaru [GOs]}
\begin{glose}
\pfra{crabe mou}
\end{glose}
\newline
\relationsémantique{Cf.}{\lien{}{pitrêê ; wadolò}}
\end{sous-entrée}
\newline
\sens{2}
(\domainesémantique{Santé, maladie})
\begin{glose}
\pfra{couvert de bobos, d'ulcères ; couvert de gale}
\end{glose}
\end{entrée}

\begin{entrée}{nhyatru dili}{}{ⓔnhyatru dili}
\région{GOs}
(\domainesémantique{Terre})
\classe{nom}
\begin{glose}
\pfra{bonne terre (molle)}
\end{glose}
\end{entrée}

\begin{entrée}{nhye}{}{ⓔnhye}
\région{GOs PA BO}
(\domainesémantique{Démonstratifs})
\classe{DEM.DEIC.1}
\begin{glose}
\pfra{ce...ci}
\end{glose}
\newline
\begin{exemple}
\textbf{\pnua{ti nhye mãni ?}}
\pfra{qui est celui qui dort ici ?}
\end{exemple}
\end{entrée}

\begin{entrée}{nhye-ba}{}{ⓔnhye-ba}
\région{GOs WEM WE}
(\domainesémantique{Démonstratifs})
\classe{DEIC}
\begin{glose}
\pfra{cela (inanimé ; distance moyenne, mais visible)}
\end{glose}
\newline
\relationsémantique{Cf.}{\lien{}{ã jè-ba}}
\glosecourte{celle-là (femme; éloigné mais visible)}
\newline
\relationsémantique{Cf.}{\lien{}{ã ê-ba}}
\glosecourte{celui-là (homme; éloigné mais visible)}
\end{entrée}

\begin{entrée}{nhye-da}{}{ⓔnhye-da}
\région{GOs}
(\domainesémantique{Démonstratifs})
\classe{DEM.DIR}
\begin{glose}
\pfra{cela en haut}
\end{glose}
\newline
\relationsémantique{Ant.}{\lien{ⓔnye, nyãⓝnye-du}{nye-du}}
\glosecourte{cela en bas}
\end{entrée}

\begin{entrée}{nhyò}{}{ⓔnhyò}
\formephonétique{ɲʰɔ}
\région{GOs PA}
\variante{%
nyò, nyõ
\région{BO}}
(\domainesémantique{Mammifères})
\classe{nom}
\begin{glose}
\pfra{nid de roussette (endroit où les roussettes se mettent le jour)}
\end{glose}
\begin{glose}
\pfra{essaim de roussette}
\end{glose}
\end{entrée}

\begin{entrée}{nhyô}{}{ⓔnhyô}
\formephonétique{ɲʰõ}
\région{GOs}
\variante{%
nyong, nyô
\région{PA}}
(\domainesémantique{Actions liées aux éléments (liquide, fumée)})
\classe{v ; n}
\begin{glose}
\pfra{baisser (niveau d'eau) ; descendre (niveau d'eau)}
\end{glose}
\begin{glose}
\pfra{décrue}
\end{glose}
\newline
\begin{exemple}
\région{GO}
\textbf{\pnua{e nhyô we}}
\pfra{l'eau baisse}
\end{exemple}
\newline
\begin{exemple}
\région{PA}
\textbf{\pnua{i nyông (a) kale}}
\pfra{la marée descend}
\end{exemple}
\newline
\relationsémantique{Cf.}{\lien{ⓔnhyôgo we}{nhyôgo we}}
\glosecourte{niveau d'eau}
\newline
\relationsémantique{Cf.}{\lien{}{nyôgo we}}
\glosecourte{niveau d'eau}
\end{entrée}

\begin{entrée}{nhyôgo we}{}{ⓔnhyôgo we}
\formephonétique{ɲʰõŋgo}
\région{GOs}
\variante{%
nyongo we
\région{BO}}
(\domainesémantique{Description des objets, formes, consistance, taille})
\classe{nom}
\begin{glose}
\pfra{niveau (d'eau) ; profondeur}
\end{glose}
\newline
\begin{exemple}
\région{GO}
\textbf{\pnua{e whaya nhyôgoo we ? - e nhyô}}
\pfra{comment est le niveau d'eau ? - il a baissé}
\end{exemple}
\newline
\begin{exemple}
\textbf{\pnua{a-ho-du-ò vwo po nhyôgo-çö êne kòlò wããge-çö}}
\pfra{avance un peu pour que le niveau de l'eau t'arrive à la poitrine (lit. ton niveau d'eau soit à la poitrine)}
\end{exemple}
\newline
\begin{exemple}
\textbf{\pnua{e tha-nhyôgo}}
\pfra{elle sonde la profondeur}
\end{exemple}
\newline
\relationsémantique{Cf.}{\lien{ⓔnhyô}{nhyô}}
\glosecourte{baisser (niveau d'eau)}
\end{entrée}

\begin{entrée}{nhyôni}{}{ⓔnhyôni}
\formephonétique{ɲʰõɳi}
\région{GOs}
\variante{%
nyõnim
\région{PA BO}}
(\domainesémantique{Phénomènes atmosphériques et naturels})
\classe{nom}
\begin{glose}
\pfra{éclair}
\end{glose}
\end{entrée}

\newpage

\lettrine{
o
ò
õ
ô
ö
}\begin{entrée}{-ò}{1}{ⓔ-òⓗ1}
\région{GOs}
\variante{%
-wò
\région{GO(s)}, 
-ò
\région{PA}}
(\domainesémantique{Directionnels})
\classe{DIR (centrifuge)}
\begin{glose}
\pfra{éloigner (s') du locuteur dans une direction transverse}
\end{glose}
\newline
\begin{sous-entrée}{phe-wò}{ⓔ-òⓗ1ⓝphe-wò}
\begin{glose}
\pfra{emporte-le}
\end{glose}
\end{sous-entrée}
\newline
\begin{sous-entrée}{phe-da-ò}{ⓔ-òⓗ1ⓝphe-da-ò}
\begin{glose}
\pfra{emporte-le en haut}
\end{glose}
\end{sous-entrée}
\newline
\begin{sous-entrée}{phe-du-ò}{ⓔ-òⓗ1ⓝphe-du-ò}
\begin{glose}
\pfra{emporte-le en bas}
\end{glose}
\end{sous-entrée}
\newline
\begin{sous-entrée}{e trêê-ò}{ⓔ-òⓗ1ⓝe trêê-ò}
\région{GO}
\begin{glose}
\pfra{il partit en courant}
\end{glose}
\end{sous-entrée}
\newline
\begin{sous-entrée}{ã-ò !}{ⓔ-òⓗ1ⓝã-ò !}
\begin{glose}
\pfra{va-t-en !}
\end{glose}
\end{sous-entrée}
\newline
\begin{sous-entrée}{ã-du-ò !}{ⓔ-òⓗ1ⓝã-du-ò !}
\begin{glose}
\pfra{descend (en s'éloignant) !}
\end{glose}
\newline
\begin{exemple}
\textbf{\pnua{tia-ò !}}
\pfra{pousse!}
\end{exemple}
\newline
\begin{exemple}
\textbf{\pnua{tia-da-ò !}}
\pfra{pousse vers le haut !}
\end{exemple}
\end{sous-entrée}
\end{entrée}

\begin{entrée}{-ò}{2}{ⓔ-òⓗ2}
\région{GOs}
\région{PA WEM WE BO}
\variante{%
-o
}
(\domainesémantique{Vocatifs})
\classe{vocatif}
\begin{glose}
\pfra{vocatif ; exclamatif}
\end{glose}
\newline
\begin{sous-entrée}{nyanya-ò !}{ⓔ-òⓗ2ⓝnyanya-ò !}
\région{GOs}
\begin{glose}
\pfra{ô maman !}
\end{glose}
\end{sous-entrée}
\newline
\begin{sous-entrée}{caaya-o !}{ⓔ-òⓗ2ⓝcaaya-o !}
\région{WEM BO}
\begin{glose}
\pfra{papa !}
\end{glose}
\end{sous-entrée}
\newline
\begin{sous-entrée}{caay-o !}{ⓔ-òⓗ2ⓝcaay-o !}
\région{GO PA}
\begin{glose}
\pfra{papa !}
\end{glose}
\end{sous-entrée}
\newline
\begin{sous-entrée}{nyaj-o ! nyej-o}{ⓔ-òⓗ2ⓝnyaj-o ! nyej-o}
\région{PA}
\begin{glose}
\pfra{ô maman !}
\end{glose}
\end{sous-entrée}
\newline
\étymologie{
\langue{PAN}
\étymon{*-aw}}
\end{entrée}

\begin{entrée}{-ò}{3}{ⓔ-òⓗ3}
\région{GOs}
(\domainesémantique{Démonstratifs})
\classe{ANAPH (discours)}
\begin{glose}
\pfra{cela (anaphorique) ; celui ; celle}
\end{glose}
\newline
\begin{exemple}
\textbf{\pnua{thoomwa-ò}}
\pfra{cette femme-là}
\end{exemple}
\newline
\begin{exemple}
\textbf{\pnua{ije-ò thoomwa-ò}}
\pfra{c'est la femme en question}
\end{exemple}
\newline
\begin{exemple}
\textbf{\pnua{axe novwö nye õã-lò Poimenya, ò yue Paola, ça fami õã-lò}}
\pfra{et quant à leur mère à eux 3 Poymegna, celle qui a adopté Paola, elle est liée à eux}
\end{exemple}
\end{entrée}

\begin{entrée}{õ}{1}{ⓔõⓗ1}
\région{GOs BO PA}
(\domainesémantique{Parenté})
\classe{nom}
\begin{glose}
\pfra{mère ; soeur de mère}
\end{glose}
\begin{glose}
\pfra{cousines de mère}
\end{glose}
\begin{glose}
\pfra{épouse du frère de père ; épouse des cousins de père}
\end{glose}
\newline
\begin{exemple}
\région{GOs}
\textbf{\pnua{õã-nu}}
\pfra{ma mère, ma tante maternelle}
\end{exemple}
\newline
\begin{exemple}
\région{PA}
\textbf{\pnua{õõ-n}}
\pfra{sa mère}
\end{exemple}
\newline
\begin{exemple}
\région{BO}
\textbf{\pnua{õõ-ny}}
\pfra{ma mère}
\end{exemple}
\newline
\begin{exemple}
\région{BO}
\textbf{\pnua{õ Pol}}
\pfra{la mère de Paul}
\end{exemple}
\newline
\note{õã- forme déterminée}{grammaire}{}
\end{entrée}

\begin{entrée}{õ}{2}{ⓔõⓗ2}
\formephonétique{ɔ̃}
\région{GOs}
\variante{%
òn
\formephonétique{ɔn}
\région{BO PA}}
\classe{nom}
\newline
\sens{1}
(\domainesémantique{Mer})
\begin{glose}
\pfra{sable [GOs]}
\end{glose}
\newline
\begin{exemple}
\textbf{\pnua{bwa òn}}
\pfra{sur la grève, sur le rivage}
\end{exemple}
\newline
\sens{2}
\begin{glose}
\pfra{sable ; sel [PA BO]}
\end{glose}
\newline
\étymologie{
\langue{POc}
\étymon{*qone}
\glosecourte{sable}}
\end{entrée}

\begin{entrée}{õ}{3}{ⓔõⓗ3}
\formephonétique{ɔ̃}
\région{GOs}
\variante{%
ô
\région{BO PA}}
(\domainesémantique{Quantificateurs})
\classe{QNT}
\begin{glose}
\pfra{tout ; tous}
\end{glose}
\newline
\begin{sous-entrée}{õ-tree}{ⓔõⓗ3ⓝõ-tree}
\région{GOs}
\begin{glose}
\pfra{tous les jours}
\end{glose}
\end{sous-entrée}
\newline
\begin{sous-entrée}{õ-mhwanu}{ⓔõⓗ3ⓝõ-mhwanu}
\région{GOs}
\begin{glose}
\pfra{tous les mois}
\end{glose}
\end{sous-entrée}
\newline
\begin{sous-entrée}{ô tèèn [PA, BO]}{ⓔõⓗ3ⓝô tèèn [PA, BO]}
\begin{glose}
\pfra{tous les jours}
\end{glose}
\end{sous-entrée}
\newline
\begin{sous-entrée}{ô-taagin}{ⓔõⓗ3ⓝô-taagin}
\région{PA}
\begin{glose}
\pfra{souvent}
\end{glose}
\end{sous-entrée}
\end{entrée}

\begin{entrée}{õ-}{}{ⓔõ-}
\région{GOs PA BO}
\variante{%
on
\région{PA BO}}
(\domainesémantique{Quantificateurs
, Préfixes classificateurs numériques})
\classe{CLF.NUM (multiples)}
\begin{glose}
\pfra{n-fois}
\end{glose}
\newline
\begin{sous-entrée}{õ-xè}{ⓔõ-ⓝõ-xè}
\begin{glose}
\pfra{une fois}
\end{glose}
\end{sous-entrée}
\newline
\begin{sous-entrée}{õ-tru, õ-ru}{ⓔõ-ⓝõ-tru, õ-ru}
\begin{glose}
\pfra{deux fois}
\end{glose}
\newline
\begin{exemple}
\région{GOs}
\textbf{\pnua{õ-tru xa e kaò ni kabu-è}}
\pfra{cela fait deux fois que cela déborde (rivière) dans la semaine}
\end{exemple}
\end{sous-entrée}
\end{entrée}

\begin{entrée}{ôã-muu-ce}{}{ⓔôã-muu-ce}
\région{GOs}
\variante{%
ô-muuc
\région{PA}}
(\domainesémantique{Parties de plantes})
\classe{nom}
\begin{glose}
\pfra{pied principal d'une plante (lit. mère des fleurs)}
\end{glose}
\end{entrée}

\begin{entrée}{ôdri}{}{ⓔôdri}
\région{GOs}
(\domainesémantique{Armes})
\classe{nom}
\begin{glose}
\pfra{pierre de fronde (générique)}
\end{glose}
\end{entrée}

\begin{entrée}{ògi}{}{ⓔògi}
\formephonétique{ɔŋgi}
\région{GOs}
\variante{%
ògin
\formephonétique{ɔŋgin}
\région{BO PA WEM}}
\newline
\sens{1}
(\domainesémantique{Aspect})
\classe{v.TERM}
\begin{glose}
\pfra{finir ; terminer ; être prêt}
\end{glose}
\newline
\begin{exemple}
\région{GO}
\textbf{\pnua{e ogine mogu xo ã ẽnõ}}
\pfra{l'enfant a fini son travail}
\end{exemple}
\newline
\begin{exemple}
\région{PA}
\textbf{\pnua{u ogin}}
\pfra{c'est fini}
\end{exemple}
\newline
\begin{exemple}
\région{PA}
\textbf{\pnua{nu pavage ogin}}
\pfra{j'ai tout préparé}
\end{exemple}
\newline
\begin{exemple}
\région{BO}
\textbf{\pnua{ogine vha !}}
\pfra{arrête de parler}
\end{exemple}
\newline
\begin{sous-entrée}{ba-ogine [BO]}{ⓔògiⓢ1ⓝba-ogine [BO]}
\begin{glose}
\pfra{fin}
\end{glose}
\newline
\note{v.t. ogine [GOs]}{grammaire}{}
\end{sous-entrée}
\newline
\relationsémantique{Cf.}{\lien{}{kûûni [GOs]}}
\glosecourte{terminer complètement}
\newline
\sens{2}
(\domainesémantique{Aspect})
\classe{ASP.ACC}
\begin{glose}
\pfra{déjà}
\end{glose}
\newline
\begin{exemple}
\région{GO}
\textbf{\pnua{nu ogi thözoe}}
\pfra{je l'ai déjà caché}
\end{exemple}
\newline
\begin{exemple}
\région{GO}
\textbf{\pnua{jö ogi ã-du Frans ?}}
\pfra{es-tu déjà allé en France ?}
\end{exemple}
\newline
\begin{exemple}
\région{GO}
\textbf{\pnua{e uvwi-ogine-ni a-xe êgu xè pò-mã}}
\pfra{il a acheté 20 mangues}
\end{exemple}
\newline
\begin{exemple}
\région{PA}
\textbf{\pnua{nu ogi hovwo}}
\pfra{j'ai déjà mangé}
\end{exemple}
\newline
\sens{3}
(\domainesémantique{Aspect})
\classe{SEQ}
\begin{glose}
\pfra{puis ; et puis ; ensuite}
\end{glose}
\newline
\begin{exemple}
\textbf{\pnua{e pwe, ogi e thuvwu phai nò}}
\pfra{il a pêché et ensuite il a cuit le poisson}
\end{exemple}
\newline
\begin{exemple}
\textbf{\pnua{e pweni nò, ogi e thuvwu phai}}
\pfra{il a pêché le poisson et ensuite il l'a cuit}
\end{exemple}
\newline
\étymologie{
\langue{POc}
\étymon{*qoti}
\glosecourte{finir}}
\end{entrée}

\begin{entrée}{ohaim}{}{ⓔohaim}
\région{PA}
\variante{%
ohahèm
\région{BO [Corne]}}
(\domainesémantique{Fonctions naturelles humaines})
\classe{v}
\begin{glose}
\pfra{bâiller}
\end{glose}
\end{entrée}

\begin{entrée}{ohe}{}{ⓔohe}
\région{PA}
\variante{%
whe
\région{PA}}
(\domainesémantique{Localisation})
\classe{ADV}
\begin{glose}
\pfra{côté (sur le) ; côté (à)}
\end{glose}
\newline
\begin{exemple}
\région{PA}
\textbf{\pnua{po tee-ohe}}
\pfra{assieds-toi un peu plus loin sur le côté}
\end{exemple}
\end{entrée}

\begin{entrée}{ô-jitua}{}{ⓔô-jitua}
\région{GOs BO [Corne]}
(\domainesémantique{Armes})
\classe{nom}
\begin{glose}
\pfra{arc}
\end{glose}
\end{entrée}

\begin{entrée}{ole}{}{ⓔole}
\région{GOs PA BO}
(\domainesémantique{Relations et interaction sociales})
\classe{v ; INTJ}
\begin{glose}
\pfra{remercier ; merci}
\end{glose}
\newline
\begin{exemple}
\textbf{\pnua{nu ole nai jö}}
\pfra{je te remercie}
\end{exemple}
\newline
\begin{exemple}
\textbf{\pnua{ole nai jö}}
\pfra{merci à toi, je te remercie}
\end{exemple}
\end{entrée}

\begin{entrée}{-òli}{}{ⓔ-òli}
\région{GOs}
\variante{%
hòli
}
(\domainesémantique{Démonstratifs})
\classe{DEIC.3 (visible)}
\begin{glose}
\pfra{là ; là-bas}
\end{glose}
\newline
\begin{exemple}
\textbf{\pnua{êgu-òli}}
\pfra{cette personne-là}
\end{exemple}
\newline
\begin{exemple}
\textbf{\pnua{êgu-mãli-òli}}
\pfra{ces deux personnes-là}
\end{exemple}
\newline
\begin{exemple}
\textbf{\pnua{êgu-mãla-òli}}
\pfra{ces personnes-là}
\end{exemple}
\end{entrée}

\begin{entrée}{olo}{}{ⓔolo}
\région{GOs BO}
(\domainesémantique{Feu : objets et actions liés au feu})
\classe{v}
\begin{glose}
\pfra{flamber ; brûler}
\end{glose}
\newline
\begin{exemple}
\région{BO}
\textbf{\pnua{i olo}}
\pfra{ça flambe}
\end{exemple}
\end{entrée}

\begin{entrée}{oloomã}{}{ⓔoloomã}
\région{GOs BO}
(\domainesémantique{Feu : objets et actions liés au feu})
\classe{nom}
\begin{glose}
\pfra{flamme}
\end{glose}
\newline
\begin{sous-entrée}{oloomã yai}{ⓔoloomãⓝoloomã yai}
\begin{glose}
\pfra{flamme du feu}
\end{glose}
\end{sous-entrée}
\end{entrée}

\begin{entrée}{-ò ... -mi}{}{ⓔ-ò ... -mi}
\région{GOs BO}
(\domainesémantique{Directionnels})
\classe{DIR}
\begin{glose}
\pfra{de-ci de-là ; en allant et venant ; partout}
\end{glose}
\newline
\begin{exemple}
\textbf{\pnua{i töö-ò töö-mi}}
\pfra{il rampe par-ci par-là}
\end{exemple}
\newline
\begin{exemple}
\région{GOs}
\textbf{\pnua{e thumenõ-ò thumenõ-mi}}
\pfra{il fait des allées et venues}
\end{exemple}
\newline
\begin{exemple}
\textbf{\pnua{la pa(o)-ò pao-mi boo (ou) la pa-ò vao-mi boo}}
\pfra{ils lancent le ballon d'un côté et de l'autre}
\end{exemple}
\end{entrée}

\begin{entrée}{õmwã}{}{ⓔõmwã}
\région{GOs}
(\domainesémantique{Mollusques})
\classe{nom}
\begin{glose}
\pfra{bernard-l'ermite (gastéropode)}
\end{glose}
\newline
\étymologie{
\langue{POc}
\étymon{*qumaŋ}
\glosecourte{bernard-l'ermite}}
\end{entrée}

\begin{entrée}{-on}{}{ⓔ-on}
\région{BO}
(\domainesémantique{Marques restrictives})
\classe{RESTR + NUM}
\begin{glose}
\pfra{seul ; seulement}
\end{glose}
\newline
\begin{exemple}
\textbf{\pnua{poi-n da axe-on}}
\pfra{son seul enfant (Dubois)}
\end{exemple}
\end{entrée}

\begin{entrée}{õn}{}{ⓔõn}
\région{PA BO}
\classe{nom}
\newline
\sens{1}
(\domainesémantique{Aliments, alimentation})
\begin{glose}
\pfra{sel}
\end{glose}
\newline
\relationsémantique{Cf.}{\lien{}{õne [BO]}}
\glosecourte{saler}
\newline
\sens{2}
(\domainesémantique{Mer})
\begin{glose}
\pfra{sable ; plage}
\end{glose}
\newline
\begin{exemple}
\textbf{\pnua{bwa õn}}
\pfra{sur le sable}
\end{exemple}
\newline
\étymologie{
\langue{POc}
\étymon{*qone}
\glosecourte{sable}}
\end{entrée}

\begin{entrée}{õ-niza ?}{}{ⓔõ-niza ?}
\région{GOs}
\variante{%
õ-nira ?
\région{BO}}
(\domainesémantique{Interrogatifs})
\classe{INT}
\begin{glose}
\pfra{combien de fois?}
\end{glose}
\newline
\begin{exemple}
\région{GOs}
\textbf{\pnua{õ-niza çö a-da Numea ?}}
\pfra{combien de fois es-tu allé à Nouméa ?}
\end{exemple}
\end{entrée}

\begin{entrée}{õn na}{}{ⓔõn na}
\région{PA}
(\domainesémantique{Aspect})
\classe{n.CNJ}
\begin{glose}
\pfra{chaque fois que}
\end{glose}
\newline
\begin{exemple}
\région{PA}
\textbf{\pnua{õn na yu havha, ye kòi-yu}}
\pfra{chaque fois que je viens, tu n'es pas là}
\end{exemple}
\end{entrée}

\begin{entrée}{ono-n}{}{ⓔono-n}
\région{BO}
\classe{nom}
(\domainesémantique{Corps humain})
\begin{glose}
\pfra{estomac [Coyaud]}
\end{glose}
\newline
\note{non vérifié}{général}{}
\end{entrée}

\begin{entrée}{ôô}{}{ⓔôô}
\région{BO}
(\domainesémantique{Fonctions naturelles des animaux})
\classe{nom}
\begin{glose}
\pfra{fumier ; crotte (animal) [Corne]}
\end{glose}
\newline
\begin{sous-entrée}{ôô vaci}{ⓔôôⓝôô vaci}
\région{BO}
\begin{glose}
\pfra{bouse de vache}
\end{glose}
\end{sous-entrée}
\newline
\begin{sous-entrée}{ôô choval}{ⓔôôⓝôô choval}
\région{BO}
\begin{glose}
\pfra{crottin de cheval}
\end{glose}
\end{sous-entrée}
\end{entrée}

\begin{entrée}{-ôô}{}{ⓔ-ôô}
\formephonétique{õː}
\région{GO}
(\domainesémantique{Pronoms})
\classe{PRO 1° pers. triel incl. (OBJ ou POSS)}
\begin{glose}
\pfra{nous-trois}
\end{glose}
\end{entrée}

\begin{entrée}{òòl}{}{ⓔòòl}
\région{PA BO [Corne]}
(\domainesémantique{Description des objets, formes, consistance, taille})
\classe{v}
\begin{glose}
\pfra{ramollir ; étirer (s') (comme de la gomme)}
\end{glose}
\newline
\begin{exemple}
\textbf{\pnua{òòl (a) we-pwaa-n}}
\pfra{il bave (se dit d'un bébé)}
\end{exemple}
\end{entrée}

\begin{entrée}{oole}{}{ⓔoole}
\région{PA BO}
\classe{v}
\newline
\sens{1}
(\domainesémantique{Pêche})
\begin{glose}
\pfra{barrage à la pêche (faire un) (avec cailloux, pierres, branches)}
\end{glose}
\newline
\sens{2}
(\domainesémantique{Verbes d'action (en général)})
\begin{glose}
\pfra{barrer ; empêcher}
\end{glose}
\newline
\begin{exemple}
\région{PA}
\textbf{\pnua{i oole dèèn u je ce}}
\pfra{cet arbre a barré la route}
\end{exemple}
\newline
\begin{exemple}
\région{PA}
\textbf{\pnua{i oole-vwo}}
\pfra{il a fait barrage, il a bloqué}
\end{exemple}
\newline
\begin{exemple}
\région{PA}
\textbf{\pnua{i oole dèn}}
\pfra{il a bloqué la route}
\end{exemple}
\newline
\begin{exemple}
\région{BO}
\textbf{\pnua{la oole-nu na nu a-è}}
\pfra{ils m'ont empêché de partir}
\end{exemple}
\newline
\relationsémantique{Cf.}{\lien{}{khibwaa dèn}}
\glosecourte{barrer, couper la route}
\end{entrée}

\begin{entrée}{oole we}{}{ⓔoole we}
\région{GOs}
(\domainesémantique{Actions liées aux éléments (liquide, fumée)})
\classe{v}
\begin{glose}
\pfra{barrer l'eau}
\end{glose}
\end{entrée}

\begin{entrée}{õ-pengan}{}{ⓔõ-pengan}
\région{PA WE}
\variante{%
haivwo
\région{GO(s)}}
(\domainesémantique{Quantificateurs})
\classe{ADV}
\begin{glose}
\pfra{plusieurs fois}
\end{glose}
\end{entrée}

\begin{entrée}{ô-phwalawa}{}{ⓔô-phwalawa}
\région{GOs}
\variante{%
ô-palawa
\région{BO}}
(\domainesémantique{Aliments, alimentation})
\classe{nom}
\begin{glose}
\pfra{levain (lit. mère du pain)}
\end{glose}
\end{entrée}

\begin{entrée}{orã}{}{ⓔorã}
\région{GOs BO}
(\domainesémantique{Arbre})
\classe{nom}
\begin{glose}
\pfra{orange(r)}
\end{glose}
\newline
\emprunt{orange (FR)}
\end{entrée}

\begin{entrée}{orèyi}{}{ⓔorèyi}
\région{GO WEM}
\newline
\sens{1}
(\domainesémantique{Types de maison, architecture de la maison})
\classe{nom}
\begin{glose}
\pfra{gaulettes circulaires du toit}
\end{glose}
\newline
\sens{2}
(\domainesémantique{Types de maison, architecture de la maison})
\classe{v}
\begin{glose}
\pfra{couper le bois qui sert de gaulettes (retenant la couverture du toit, i.e. les écorces de niaouli et la paille)}
\end{glose}
\newline
\relationsémantique{Cf.}{\lien{ⓔzabò}{zabò}}
\end{entrée}

\begin{entrée}{òri}{1}{ⓔòriⓗ1}
\formephonétique{ɔri}
\formephonétique{ɔʈi}
\région{GOs BO}
\variante{%
òtri
\région{GO(s) WE}}
(\domainesémantique{Caractéristiques et propriétés des personnes})
\classe{v.stat.}
\begin{glose}
\pfra{fou (être)}
\end{glose}
\begin{glose}
\pfra{saoûl ; ivre}
\end{glose}
\newline
\begin{exemple}
\région{GOs}
\textbf{\pnua{e òri bwawe}}
\pfra{il est fou}
\end{exemple}
\newline
\relationsémantique{Cf.}{\lien{ⓔkulèng}{kulèng}}
\end{entrée}

\begin{entrée}{òri}{2}{ⓔòriⓗ2}
\région{GOs}
(\domainesémantique{Adverbe})
\classe{ADV}
\begin{glose}
\pfra{adverbe péjoratif}
\end{glose}
\newline
\begin{exemple}
\textbf{\pnua{e za ãbe òri !}}
\pfra{il croit tout savoir !}
\end{exemple}
\newline
\begin{exemple}
\textbf{\pnua{jö za hine lòlò òri !}}
\pfra{tu dis des bêtises ! (lit. tu sais sans savoir)}
\end{exemple}
\newline
\begin{exemple}
\textbf{\pnua{a-kô-ii òri ẽnõ ã !}}
\pfra{cet enfant s'agite beaucoup en dormant}
\end{exemple}
\newline
\begin{exemple}
\textbf{\pnua{a-pejöli òri ègu ba !}}
\pfra{qu'est-ce qu'il est râleur cet homme là-bas !}
\end{exemple}
\newline
\begin{exemple}
\région{GOs}
\textbf{\pnua{e ala òri ègu ba !}}
\pfra{ce qu'il est maladroit cet homme !}
\end{exemple}
\end{entrée}

\begin{entrée}{õ-taagi}{}{ⓔõ-taagi}
\région{GOs}
\variante{%
õ-taagin
\région{PA BO}}
(\domainesémantique{Aspect})
\classe{ADV}
\begin{glose}
\pfra{souvent ; toujours}
\end{glose}
\newline
\begin{exemple}
\région{PA}
\textbf{\pnua{i khõbwe õ-taagin}}
\pfra{il dit souvent}
\end{exemple}
\end{entrée}

\begin{entrée}{õ-thõm}{}{ⓔõ-thõm}
\région{BO}
(\domainesémantique{Nattes})
\classe{nom}
\begin{glose}
\pfra{natte-manteau à longues pailles [Corne]}
\end{glose}
\end{entrée}

\begin{entrée}{otròtròya}{}{ⓔotròtròya}
\formephonétique{oɽɔ'ɽɔya}
\région{GOs WEM}
\variante{%
oroyai
\région{GO(s) PA BO}, 
yüe
\région{BO}}
(\domainesémantique{Relations et interaction sociales})
\classe{v ; n}
\begin{glose}
\pfra{bercer (un enfant) ; berceuse (chant)}
\end{glose}
\end{entrée}

\begin{entrée}{ô-uvia}{}{ⓔô-uvia}
\région{GOs}
(\domainesémantique{Parties de plantes})
\classe{nom}
\begin{glose}
\pfra{tige de taro (tige principale) (lit. mère du taro)}
\end{glose}
\end{entrée}

\begin{entrée}{ovwee}{}{ⓔovwee}
\formephonétique{oβeː}
\région{GOs}
\variante{%
ovee
\région{BO [BM]}}
(\domainesémantique{Verbes d'action (en général)})
\classe{v}
\begin{glose}
\pfra{ôter ; enlever}
\end{glose}
\newline
\begin{exemple}
\région{GOs}
\textbf{\pnua{ovwee hõbwõli-ço}}
\pfra{enlève ta chemise}
\end{exemple}
\newline
\begin{exemple}
\région{BO}
\textbf{\pnua{ove hõbwõni-m}}
\pfra{enlève ta chemise}
\end{exemple}
\end{entrée}

\begin{entrée}{õ-waran}{}{ⓔõ-waran}
\région{PA}
(\domainesémantique{Conjonction})
\classe{CNJ}
\begin{glose}
\pfra{chaque fois que ; souvent ; tout le temps}
\end{glose}
\newline
\begin{exemple}
\textbf{\pnua{õ waran ne yu havha ...}}
\pfra{chaque fois que tu arrives}
\end{exemple}
\newline
\begin{exemple}
\textbf{\pnua{õ waran ne pwal ...}}
\pfra{chaque fois qu'il pleut}
\end{exemple}
\newline
\begin{exemple}
\région{PA}
\textbf{\pnua{õ-waran na yö havha}}
\pfra{chaque fois que tu arrives}
\end{exemple}
\newline
\begin{exemple}
\région{PA}
\textbf{\pnua{õ-waran na i pwal}}
\pfra{chaque fois qu'il pleut}
\end{exemple}
\newline
\begin{exemple}
\région{PA}
\textbf{\pnua{nu ra khõbwe õ-waran}}
\pfra{chaque fois qu'il pleut}
\end{exemple}
\end{entrée}

\begin{entrée}{õ-xe}{}{ⓔõ-xe}
\formephonétique{õɣe}
\région{GOs BO PA}
\variante{%
hokè, hogè
\région{BO}}
(\domainesémantique{Numéraux cardinaux})
\classe{n-fois}
\begin{glose}
\pfra{une fois}
\end{glose}
\begin{glose}
\pfra{un autre ; un nouveau}
\end{glose}
\newline
\begin{exemple}
\textbf{\pnua{õ-tru, õ-ko}}
\pfra{deux autres, trois autres}
\end{exemple}
\newline
\begin{exemple}
\région{GOs}
\textbf{\pnua{nu trõne õ-tru thixa jige}}
\pfra{j'ai entendu deux autres coups de fusil}
\end{exemple}
\end{entrée}

\begin{entrée}{õxè}{}{ⓔõxè}
\région{GOs}
\variante{%
oxa
\région{BO [BM]}}
(\domainesémantique{Aspect})
\classe{REV}
\begin{glose}
\pfra{encore ; de nouveau}
\end{glose}
\newline
\begin{exemple}
\région{GOs}
\textbf{\pnua{e õxe mòlò mwa}}
\pfra{il est revenu à la vie}
\end{exemple}
\newline
\begin{exemple}
\région{GOs}
\textbf{\pnua{ne õxè}}
\pfra{refais le !}
\end{exemple}
\newline
\begin{exemple}
\région{BO}
\textbf{\pnua{oxa na}}
\pfra{donne encore}
\end{exemple}
\newline
\begin{sous-entrée}{õxe nòe}{ⓔõxèⓝõxe nòe}
\begin{glose}
\pfra{faire à nouveau}
\end{glose}
\end{sous-entrée}
\end{entrée}

\begin{entrée}{õ-xe-nò}{}{ⓔõ-xe-nò}
\formephonétique{ɔ̃ɣeɳɔ}
\région{GOs}
(\domainesémantique{Numéraux cardinaux})
\classe{n-fois}
\begin{glose}
\pfra{une seule fois}
\end{glose}
\newline
\relationsémantique{Cf.}{\lien{}{pòxè-nò}}
\end{entrée}

\begin{entrée}{õ-xe-on}{}{ⓔõ-xe-on}
\région{BO PA}
(\domainesémantique{Numéraux cardinaux})
\classe{n-fois}
\begin{glose}
\pfra{une seule fois (Dubois)}
\end{glose}
\newline
\note{non vérifié}{général}{}
\end{entrée}

\begin{entrée}{oya}{}{ⓔoya}
\région{PA}
(\domainesémantique{Parties de plantes})
\classe{nom}
\begin{glose}
\pfra{résine}
\end{glose}
\newline
\begin{sous-entrée}{oya jeü}{ⓔoyaⓝoya jeü}
\begin{glose}
\pfra{résine de kaori}
\end{glose}
\end{sous-entrée}
\newline
\begin{sous-entrée}{oya waawe}{ⓔoyaⓝoya waawe}
\begin{glose}
\pfra{résine de sapin}
\end{glose}
\end{sous-entrée}
\end{entrée}

\newpage

\lettrine{p}\begin{entrée}{pa}{}{ⓔpa}
\région{GOs PA}
(\domainesémantique{Tressage})
\classe{v}
\begin{glose}
\pfra{tresser}
\end{glose}
\newline
\begin{exemple}
\région{GO}
\textbf{\pnua{e pa thrô}}
\pfra{elle fait du tressage de natte}
\end{exemple}
\newline
\begin{exemple}
\région{GO}
\textbf{\pnua{ijö nye jö pae thrô ã ?}}
\pfra{est-ce toi qui a tressé cette natte ?}
\end{exemple}
\newline
\begin{exemple}
\région{PA}
\textbf{\pnua{i pa keel}}
\pfra{elle tresse un panier}
\end{exemple}
\newline
\étymologie{
\langue{POc}
\étymon{*patu}
\glosecourte{tresser}}
\newline
\note{pae [GOs], pai [PA]}{grammaire}{tresser qqch.}
\end{entrée}

\begin{entrée}{pa-}{}{ⓔpa-}
\région{GOs PA}
(\domainesémantique{Marques de degré})
\classe{QNT}
\begin{glose}
\pfra{très (+ animés)}
\end{glose}
\newline
\begin{exemple}
\région{GO}
\textbf{\pnua{e za u cii pa-baa}}
\pfra{il est vraiment très noir de peau}
\end{exemple}
\newline
\begin{sous-entrée}{pa-baan}{ⓔpa-ⓝpa-baan}
\région{PA}
\begin{glose}
\pfra{très noir; noirâtre}
\end{glose}
\end{sous-entrée}
\newline
\begin{sous-entrée}{i pa-tha}{ⓔpa-ⓝi pa-tha}
\région{PA}
\begin{glose}
\pfra{il est totalement chauve}
\end{glose}
\end{sous-entrée}
\newline
\begin{sous-entrée}{pa-whaa}{ⓔpa-ⓝpa-whaa}
\région{GO}
\begin{glose}
\pfra{très gros}
\end{glose}
\end{sous-entrée}
\newline
\relationsémantique{Cf.}{\lien{ⓔparaⓗ1}{para}}
\glosecourte{très}
\newline
\note{mhaa : 'très'}{grammaire}{}
\end{entrée}

\begin{entrée}{-pa}{}{ⓔ-pa}
\région{GOs BO PA}
(\domainesémantique{Numéraux cardinaux})
\classe{NUM}
\begin{glose}
\pfra{quatre}
\end{glose}
\newline
\étymologie{
\langue{POc}
\étymon{*pat}
\glosecourte{4}}
\end{entrée}

\begin{entrée}{pã}{}{ⓔpã}
\région{WE}
(\domainesémantique{Aliments, alimentation})
\classe{nom}
\begin{glose}
\pfra{pain}
\end{glose}
\newline
\emprunt{pain (FR)}
\end{entrée}

\begin{entrée}{paa}{1}{ⓔpaaⓗ1}
\région{GOs}
\variante{%
paa
\région{PA BO}}
(\domainesémantique{Guerre})
\classe{nom}
\begin{glose}
\pfra{guerre}
\end{glose}
\newline
\begin{exemple}
\région{GO}
\textbf{\pnua{li pe-thu-paa lie pwe-meevwu}}
\pfra{les deux clans se font la guerre}
\end{exemple}
\newline
\begin{sous-entrée}{kapu paa, kawu paa}{ⓔpaaⓗ1ⓝkapu paa, kawu paa}
\région{GO}
\begin{glose}
\pfra{chef de guerre}
\end{glose}
\end{sous-entrée}
\newline
\begin{sous-entrée}{thi-paa, to-paa}{ⓔpaaⓗ1ⓝthi-paa, to-paa}
\région{BO}
\begin{glose}
\pfra{embûches, embuscade}
\end{glose}
\end{sous-entrée}
\newline
\begin{sous-entrée}{paa tu}{ⓔpaaⓗ1ⓝpaa tu}
\begin{glose}
\pfra{mettre en fuite}
\end{glose}
\end{sous-entrée}
\end{entrée}

\begin{entrée}{paa}{2}{ⓔpaaⓗ2}
\région{GOs PA BO}
(\domainesémantique{Pierre, roche})
\classe{nom}
\begin{glose}
\pfra{pierre ; caillou}
\end{glose}
\newline
\begin{sous-entrée}{paa temi}{ⓔpaaⓗ2ⓝpaa temi}
\begin{glose}
\pfra{schistes}
\end{glose}
\end{sous-entrée}
\newline
\begin{sous-entrée}{mwa-paa}{ⓔpaaⓗ2ⓝmwa-paa}
\begin{glose}
\pfra{grotte}
\end{glose}
\end{sous-entrée}
\newline
\begin{sous-entrée}{phwee-paa}{ⓔpaaⓗ2ⓝphwee-paa}
\région{GO}
\begin{glose}
\pfra{grotte}
\end{glose}
\end{sous-entrée}
\newline
\begin{sous-entrée}{po paa}{ⓔpaaⓗ2ⓝpo paa}
\région{BO}
\begin{glose}
\pfra{petit caillou}
\end{glose}
\end{sous-entrée}
\newline
\begin{sous-entrée}{paa ni gò}{ⓔpaaⓗ2ⓝpaa ni gò}
\région{GO}
\begin{glose}
\pfra{pile (caillou pour la musique)}
\end{glose}
\end{sous-entrée}
\newline
\begin{sous-entrée}{paa ni ya-khaa}{ⓔpaaⓗ2ⓝpaa ni ya-khaa}
\région{GO}
\begin{glose}
\pfra{pile pour la torche}
\end{glose}
\end{sous-entrée}
\newline
\étymologie{
\langue{POc}
\étymon{*patu}
\glosecourte{pierre}}
\end{entrée}

\begin{entrée}{paaba}{}{ⓔpaaba}
\région{GOs}
(\domainesémantique{Navigation})
\classe{v}
\begin{glose}
\pfra{ramer}
\end{glose}
\newline
\begin{sous-entrée}{ba-paaba}{ⓔpaabaⓝba-paaba}
\begin{glose}
\pfra{rame}
\end{glose}
\newline
\relationsémantique{Cf.}{\lien{ⓔhaal}{haal}}
\glosecourte{ramer}
\end{sous-entrée}
\end{entrée}

\begin{entrée}{paa-bule}{}{ⓔpaa-bule}
\région{PA}
(\domainesémantique{Actions liées aux éléments (liquide, fumée)})
\classe{v}
\begin{glose}
\pfra{tremper dans l'eau}
\end{glose}
\end{entrée}

\begin{entrée}{paaçae}{}{ⓔpaaçae}
\formephonétique{paːʒae}
\région{GOs}
(\domainesémantique{Mer})
\classe{nom}
\begin{glose}
\pfra{vague (mer)}
\end{glose}
\end{entrée}

\begin{entrée}{paa-çôe}{}{ⓔpaa-çôe}
\formephonétique{paː-ʒõe}
\région{GOs WEM}
\variante{%
pa-nyoî
\région{BO}}
(\domainesémantique{Mouvements ou actions faits avec le corps, les bras, les mains, les pieds})
\classe{v}
\begin{glose}
\pfra{accrocher ; suspendre qqch}
\end{glose}
\newline
\begin{exemple}
\région{GO}
\textbf{\pnua{nu paa-çôe ke}}
\pfra{j'accroche le panier}
\end{exemple}
\newline
\relationsémantique{Cf.}{\lien{ⓔcôô}{côô}}
\glosecourte{suspendu, accroché}
\end{entrée}

\begin{entrée}{paa-du}{}{ⓔpaa-du}
\région{PA}
(\domainesémantique{Cours de la vie})
\classe{v}
\begin{glose}
\pfra{adulte (être) ; mature (lit. pierre-os)}
\end{glose}
\newline
\begin{exemple}
\textbf{\pnua{i paa-duu-n}}
\pfra{il est adulte}
\end{exemple}
\newline
\begin{exemple}
\textbf{\pnua{ra u paa-duu-n}}
\pfra{il est adulte (il a le dos dur)}
\end{exemple}
\end{entrée}

\begin{entrée}{paaje}{}{ⓔpaaje}
\formephonétique{paːɲɟe}
\région{GOs PA}
\variante{%
paaye
\région{PA BO}}
(\domainesémantique{Topographie})
\classe{nom}
\begin{glose}
\pfra{chaîne centrale}
\end{glose}
\end{entrée}

\begin{entrée}{paa-majing}{}{ⓔpaa-majing}
\région{PA BO}
(\domainesémantique{Pierre, roche})
\classe{nom}
\begin{glose}
\pfra{quartz (pierre blanche \& dure, utilisée pour les fours)}
\end{glose}
\end{entrée}

\begin{entrée}{paa-moze}{}{ⓔpaa-moze}
\formephonétique{paːmoze}
\région{GOs}
\variante{%
paa-moze
\région{GO(s)}}
(\domainesémantique{Actions liées aux éléments (liquide, fumée)})
\classe{v}
\begin{glose}
\pfra{vider (un liquide: au sens de boire)}
\end{glose}
\end{entrée}

\begin{entrée}{paa-neng}{}{ⓔpaa-neng}
\région{BO}
(\domainesémantique{Coutumes, dons coutumiers})
\classe{nom}
\begin{glose}
\pfra{pierre ; couteau fait dans cette pierre et utilisée pour la circoncision [Corne]}
\end{glose}
\newline
\relationsémantique{Cf.}{\lien{}{tragòò ; tagòò [PA]}}
\glosecourte{circoncire}
\newline
\note{non vérifié}{général}{}
\end{entrée}

\begin{entrée}{paang}{}{ⓔpaang}
\région{PA BO}
(\domainesémantique{Cultures, techniques, boutures})
\classe{v ; n}
\begin{glose}
\pfra{désherber ; faucher ;désherbage}
\end{glose}
\begin{glose}
\pfra{arracher l'herbe (à la main)}
\end{glose}
\newline
\begin{exemple}
\textbf{\pnua{ni bala paang i bin}}
\pfra{dans la partie de notre désherbage}
\end{exemple}
\newline
\begin{sous-entrée}{ba-paang}{ⓔpaangⓝba-paang}
\begin{glose}
\pfra{faucille}
\end{glose}
\newline
\note{v.t. paage}{grammaire}{}
\end{sous-entrée}
\newline
\relationsémantique{Cf.}{\lien{}{phaawa [GOs]}}
\glosecourte{désherber}
\end{entrée}

\begin{entrée}{paa-nhi}{}{ⓔpaa-nhi}
\région{GOs}
(\domainesémantique{Pierre, roche})
\classe{nom}
\begin{glose}
\pfra{rocher}
\end{glose}
\end{entrée}

\begin{entrée}{paara}{}{ⓔpaara}
\région{BO}
(\domainesémantique{Fonctions intellectuelles})
\classe{nom}
\begin{glose}
\pfra{information, nouvelle [Corne]}
\end{glose}
\end{entrée}

\begin{entrée}{paa-trèè-mii}{}{ⓔpaa-trèè-mii}
\région{GOs}
\variante{%
pa-tèè-mii
\région{BO}}
(\domainesémantique{Feu : objets et actions liés au feu})
\classe{nom}
\begin{glose}
\pfra{pierre rouge (qu'on frappe pour faire des étincelles) [Corne]}
\end{glose}
\begin{glose}
\pfra{schistes}
\end{glose}
\end{entrée}

\begin{entrée}{paawa}{}{ⓔpaawa}
\région{GOs}
(\domainesémantique{Ignames})
\classe{nom}
\begin{glose}
\pfra{igname (clone)}
\end{glose}
\end{entrée}

\begin{entrée}{paazò}{}{ⓔpaazò}
\région{GOs}
\variante{%
paarò
\région{PA}}
(\domainesémantique{Terre})
\classe{nom}
\begin{glose}
\pfra{terre damée, dure}
\end{glose}
\end{entrée}

\begin{entrée}{paça}{}{ⓔpaça}
\formephonétique{paʒa}
\région{GOs}
\variante{%
payang, paya, peyeng
\région{WEM PA BO}}
(\domainesémantique{Verbes de déplacement et moyens de déplacement})
\classe{nom}
\begin{glose}
\pfra{route ; chemin frayé à l'outil}
\end{glose}
\end{entrée}

\begin{entrée}{pa-cabi}{}{ⓔpa-cabi}
\formephonétique{pa-cambi}
\région{GOs}
(\domainesémantique{Mouvements ou actions faits avec le corps, les bras, les mains, les pieds})
\classe{v}
\begin{glose}
\pfra{taper pour enfoncer}
\end{glose}
\end{entrée}

\begin{entrée}{paçagò}{}{ⓔpaçagò}
\formephonétique{paʒaŋgɔ}
\région{GOs}
\variante{%
payagòl
\région{PA}, 
peyego
\région{BO}}
(\domainesémantique{Santé, maladie})
\classe{nom}
\begin{glose}
\pfra{hernie}
\end{glose}
\end{entrée}

\begin{entrée}{paçô}{}{ⓔpaçô}
\formephonétique{paʒõ}
\région{GOs}
(\domainesémantique{Relations et interaction sociales})
\classe{MODIF}
\begin{glose}
\pfra{gâter (enfant) ; vanter (se)}
\end{glose}
\newline
\begin{exemple}
\région{GOs}
\textbf{\pnua{e vhaa paçô-je}}
\pfra{il se vante, il est vaniteux}
\end{exemple}
\end{entrée}

\begin{entrée}{pada}{}{ⓔpada}
\région{GOs}
\variante{%
phada
\région{PA}}
(\domainesémantique{Sons, bruits})
\classe{v}
\begin{glose}
\pfra{faire un bruit de percussion (comme des maracas) ;}
\end{glose}
\begin{glose}
\pfra{cliqueter ; faire un bruit de cliquetis}
\end{glose}
\newline
\begin{exemple}
\textbf{\pnua{da jene i pada na ni kèè-m ?}}
\pfra{qu'est ce qui racasse dans ton sac ?}
\end{exemple}
\end{entrée}

\begin{entrée}{page}{}{ⓔpage}
\région{GOs BO}
\variante{%
pwaxe
\région{BO}}
(\domainesémantique{Coutumes, dons coutumiers})
\classe{nom}
\begin{glose}
\pfra{tas de vivres}
\end{glose}
\newline
\begin{sous-entrée}{pa-page, pa-vwage}{ⓔpageⓝpa-page, pa-vwage}
\begin{glose}
\pfra{préparer, prendre soin}
\end{glose}
\end{sous-entrée}
\end{entrée}

\begin{entrée}{pa-hinõ}{}{ⓔpa-hinõ}
\région{GOs PA}
(\domainesémantique{Vêtements, parure})
\classe{v}
\begin{glose}
\pfra{nu (être) (lit. montrer)}
\end{glose}
\newline
\begin{exemple}
\textbf{\pnua{e pa-hinõ}}
\pfra{il est nu (il montre?)}
\end{exemple}
\end{entrée}

\begin{entrée}{pa-hovwo-ni}{}{ⓔpa-hovwo-ni}
\formephonétique{pa-hoβo-ɳi}
\région{GOs}
(\domainesémantique{Aliments, alimentation})
\classe{v}
\begin{glose}
\pfra{faire manger}
\end{glose}
\newline
\begin{exemple}
\textbf{\pnua{e pa-hovwo-ni la ẽnõ}}
\pfra{elle fait manger les enfants}
\end{exemple}
\end{entrée}

\begin{entrée}{pai}{1}{ⓔpaiⓗ1}
\région{GOs BO}
\variante{%
pae
\région{GO(s)}, 
pele
\région{PA}}
\classe{v ; n}
\newline
\sens{1}
(\domainesémantique{Tressage})
\begin{glose}
\pfra{tresser ; faire une tresse}
\end{glose}
\begin{glose}
\pfra{tresse}
\end{glose}
\begin{glose}
\pfra{natte (faire une)}
\end{glose}
\newline
\sens{2}
(\domainesémantique{Cordes, cordages})
\begin{glose}
\pfra{torsader, tresser une corde à plusieurs brins}
\end{glose}
\begin{glose}
\pfra{corde à plusieurs brins (faire une)}
\end{glose}
\newline
\begin{sous-entrée}{pai wa [GOs]}{ⓔpaiⓗ1ⓢ2ⓝpai wa [GOs]}
\begin{glose}
\pfra{tresser ou torsader une corde}
\end{glose}
\end{sous-entrée}
\newline
\étymologie{
\langue{POc}
\étymon{*patu}}
\end{entrée}

\begin{entrée}{pai}{2}{ⓔpaiⓗ2}
\région{GOs}
\variante{%
pain
\région{BO PA}}
\classe{nom}
\newline
\sens{1}
(\domainesémantique{Parties de plantes})
\begin{glose}
\pfra{tubercule comestible}
\end{glose}
\newline
\begin{exemple}
\région{PA}
\textbf{\pnua{kia pain}}
\pfra{il n'y a pas de tubercule}
\end{exemple}
\newline
\begin{exemple}
\région{BO}
\textbf{\pnua{pain manyok}}
\pfra{tubercule de manioc}
\end{exemple}
\newline
\sens{2}
(\domainesémantique{Discours, échanges verbaux})
\begin{glose}
\pfra{parole (chanson)}
\end{glose}
\newline
\begin{exemple}
\région{GOs}
\textbf{\pnua{wa xa ge le pai ?}}
\pfra{cette chanson a-t-elle des paroles ?}
\end{exemple}
\newline
\relationsémantique{Cf.}{\lien{ⓔpaxa-ⓝpaxa-vhaa}{paxa-vhaa}}
\glosecourte{mots, paroles de chanson (en composition)}
\end{entrée}

\begin{entrée}{pai}{3}{ⓔpaiⓗ3}
\région{GOs}
\variante{%
phai
\région{PA BO}}
(\domainesémantique{Préfixes classificateurs possessifs})
\classe{CLF.POSS (armes)}
\begin{glose}
\pfra{arme (de jet ou assimilé)}
\end{glose}
\newline
\begin{exemple}
\région{GO}
\textbf{\pnua{pai-nu jige}}
\pfra{mon fusil}
\end{exemple}
\newline
\begin{exemple}
\région{PA}
\textbf{\pnua{phai-ny bulaivi}}
\pfra{mon casse-tête}
\end{exemple}
\newline
\begin{exemple}
\région{PA}
\textbf{\pnua{phai-ny nye bulaivi}}
\pfra{c'est à moi ce casse-tête}
\end{exemple}
\newline
\begin{exemple}
\région{PA}
\textbf{\pnua{phai-ny ôdim}}
\pfra{ma pierre de fronde}
\end{exemple}
\newline
\begin{exemple}
\région{PA BO}
\textbf{\pnua{phai-ny wamòn}}
\pfra{ma hache}
\end{exemple}
\newline
\begin{exemple}
\région{PA}
\textbf{\pnua{phai-ny jigèl}}
\pfra{mon fusil}
\end{exemple}
\newline
\relationsémantique{Cf.}{\lien{ⓔphaⓗ1}{pha}}
\glosecourte{tirer, lancer}
\end{entrée}

\begin{entrée}{pa-kamaze}{}{ⓔpa-kamaze}
\région{GOs}
(\domainesémantique{Vêtements, parure})
\classe{v}
\begin{glose}
\pfra{mettre à l'envers (vêtements)}
\end{glose}
\end{entrée}

\begin{entrée}{pa-ku-gòò-ni}{}{ⓔpa-ku-gòò-ni}
\formephonétique{pa-ku-ŋgɔː-ɳi}
\région{GOs}
(\domainesémantique{Fonctions intellectuelles})
\classe{v}
\begin{glose}
\pfra{approuver (dire que c'est droit, correct)}
\end{glose}
\newline
\begin{sous-entrée}{ku-gòò [GOs]}{ⓔpa-ku-gòò-niⓝku-gòò [GOs]}
\begin{glose}
\pfra{vrai, droit, juste}
\end{glose}
\end{sous-entrée}
\end{entrée}

\begin{entrée}{pa-khêmi}{}{ⓔpa-khêmi}
\formephonétique{pa-kʰêmi}
\région{GOs}
(\domainesémantique{Cours de la vie})
\classe{v}
\begin{glose}
\pfra{enterrer}
\end{glose}
\newline
\begin{exemple}
\textbf{\pnua{la pa-khêmi-du je}}
\pfra{ils l'ont enterré}
\end{exemple}
\end{entrée}

\begin{entrée}{pala}{}{ⓔpala}
\région{BO}
(\domainesémantique{Discours, échanges verbaux})
\classe{nom}
\begin{glose}
\pfra{parole (ancien) [BM]}
\end{glose}
\newline
\begin{exemple}
\textbf{\pnua{pala-ny}}
\pfra{mes paroles}
\end{exemple}
\newline
\begin{exemple}
\textbf{\pnua{nu kobwe pala}}
\pfra{j'ai dit des paroles}
\end{exemple}
\end{entrée}

\begin{entrée}{palao}{}{ⓔpalao}
\région{BO}
(\domainesémantique{Paniers})
\classe{nom}
\begin{glose}
\pfra{panier de réserves (autour du mur) [BM]}
\end{glose}
\end{entrée}

\begin{entrée}{palawu}{}{ⓔpalawu}
\région{BO}
(\domainesémantique{Mer : topographie})
\classe{nom}
\begin{glose}
\pfra{récif de corail [BM]}
\end{glose}
\end{entrée}

\begin{entrée}{pale}{}{ⓔpale}
\région{GOs BO PA}
\variante{%
palee
\région{BO}}
\classe{v}
\newline
\sens{1}
(\domainesémantique{Mouvements ou actions faits avec le corps, les bras, les mains, les pieds})
\begin{glose}
\pfra{tâtonner}
\end{glose}
\begin{glose}
\pfra{chercher à tâtons}
\end{glose}
\begin{glose}
\pfra{palper}
\end{glose}
\begin{glose}
\pfra{effleurer}
\end{glose}
\newline
\begin{exemple}
\région{PA}
\textbf{\pnua{la u pale pale na mwã kae wèdò}}
\pfra{ils ont bien tâtonné pour que nous suivions les usages}
\end{exemple}
\newline
\sens{2}
(\domainesémantique{Pêche})
\begin{glose}
\pfra{fouiller dans un trou (sans y voir) ; plonger le bras dans qqch.}
\end{glose}
\begin{glose}
\pfra{pêcher à la main (sans y voir)}
\end{glose}
\newline
\begin{exemple}
\région{GO}
\textbf{\pnua{e pale khila hõbwòli-je}}
\pfra{elle cherche sa robe à tâtons}
\end{exemple}
\newline
\note{pale vs palee}{grammaire}{Les formes définie et indéfinie se distinguent par la longueur}
\end{entrée}

\begin{entrée}{palu}{}{ⓔpalu}
\région{GOs BO}
(\domainesémantique{Dons, échanges, achat et vente, vol})
\classe{v.stat.}
\begin{glose}
\pfra{avare (être) ; refuser de donner}
\end{glose}
\newline
\begin{exemple}
\région{GO}
\textbf{\pnua{i paluu-nu}}
\pfra{il a refusé de me donner qqch}
\end{exemple}
\newline
\begin{exemple}
\région{BO}
\textbf{\pnua{i paluu}}
\pfra{il est mesquin}
\end{exemple}
\newline
\begin{sous-entrée}{a-palu}{ⓔpaluⓝa-palu}
\begin{glose}
\pfra{un avare}
\end{glose}
\newline
\note{paluni (v.t.)}{grammaire}{priver qqn de qqch.}
\end{sous-entrée}
\end{entrée}

\begin{entrée}{pa-modee}{}{ⓔpa-modee}
\formephonétique{pa-mondeː}
\région{GOs}
(\domainesémantique{Mouvements ou actions faits avec le corps, les bras, les mains, les pieds})
\classe{v}
\begin{glose}
\pfra{déchirer ; trouer (linge)}
\end{glose}
\newline
\begin{exemple}
\région{GOs}
\textbf{\pnua{e pa-modee hõbwòli-je}}
\pfra{cela a déchiré son vêtement}
\end{exemple}
\end{entrée}

\begin{entrée}{pana}{}{ⓔpana}
\région{GOs BO}
(\domainesémantique{Relations et interaction sociales})
\classe{v}
\begin{glose}
\pfra{gronder ; disputer}
\end{glose}
\newline
\begin{exemple}
\textbf{\pnua{e pana-nu}}
\pfra{il m'a grondé}
\end{exemple}
\end{entrée}

\begin{entrée}{panooli}{}{ⓔpanooli}
\région{BO [Corne]}
(\domainesémantique{Noms des plantes})
\classe{nom}
\begin{glose}
\pfra{plante qui pousse au ras du sol et fait des petites fleurs bleues}
\end{glose}
\newline
\note{(la décoction des feuilles et tiges sert de purge pour les bébés ; pour annoncer la naissance d'un enfant à son oncle maternel on offre une tige de cette plante liée à une sagaie (garçon) ou à une écorce de niaouli (fille))}{glose}{}
\nomscientifique{Plectranthus parviflorus (Labiées)}
\newline
\note{hmahmulim (nemi) mitre (f. uvea)}{général}{}
\newline
\note{non vérifié}{général}{}
\end{entrée}

\begin{entrée}{pa-nuã}{}{ⓔpa-nuã}
\formephonétique{pa-ɳuɛ̃}
\région{GOs}
\variante{%
pa-nhuã
\région{PA BO}}
\classe{v}
\newline
\sens{1}
(\domainesémantique{Mouvements ou actions faits avec le corps, les bras, les mains, les pieds})
\begin{glose}
\pfra{lâcher ; laisser sortir (prisonnier)}
\end{glose}
\begin{glose}
\pfra{laisser tomber}
\end{glose}
\begin{glose}
\pfra{déposer [BO]}
\end{glose}
\newline
\sens{2}
(\domainesémantique{Modalité, verbes modaux})
\begin{glose}
\pfra{autoriser [BO]}
\end{glose}
\newline
\begin{exemple}
\région{GO}
\textbf{\pnua{la pa-nuã-nu dròrò}}
\pfra{ils m'ont laissé sortir hier}
\end{exemple}
\end{entrée}

\begin{entrée}{pa-nue}{}{ⓔpa-nue}
\formephonétique{pa-ɳue}
\région{GOs}
(\domainesémantique{Relations et interaction sociales})
\classe{v}
\begin{glose}
\pfra{apaiser}
\end{glose}
\end{entrée}

\begin{entrée}{pa-nûû}{}{ⓔpa-nûû}
\région{GOs}
(\domainesémantique{Lumière et obscurité})
\classe{nom}
\begin{glose}
\pfra{torche}
\end{glose}
\end{entrée}

\begin{entrée}{pa-nûûe}{}{ⓔpa-nûûe}
\formephonétique{pa-ɳûːe}
\région{GOs}
(\domainesémantique{Lumière et obscurité})
\classe{v}
\begin{glose}
\pfra{éclairer}
\end{glose}
\newline
\begin{exemple}
\région{GOs}
\textbf{\pnua{nu pa-nûûe-je}}
\pfra{je l'ai éclairé}
\end{exemple}
\end{entrée}

\begin{entrée}{pao}{}{ⓔpao}
\région{GOsPA BO}
\classe{v}
\newline
\sens{1}
(\domainesémantique{Mouvements ou actions faits avec le corps, les bras, les mains, les pieds})
\begin{glose}
\pfra{jeter en l'air ; lancer}
\end{glose}
\begin{glose}
\pfra{agiter}
\end{glose}
\newline
\begin{sous-entrée}{pao-ò phao-mi}{ⓔpaoⓢ1ⓝpao-ò phao-mi}
\begin{glose}
\pfra{lancer / jeter en tout sens}
\end{glose}
\end{sous-entrée}
\newline
\begin{sous-entrée}{pao pwio}{ⓔpaoⓢ1ⓝpao pwio}
\begin{glose}
\pfra{lancer le filet}
\end{glose}
\end{sous-entrée}
\newline
\begin{sous-entrée}{pao pwe}{ⓔpaoⓢ1ⓝpao pwe}
\begin{glose}
\pfra{lancer la ligne}
\end{glose}
\end{sous-entrée}
\newline
\begin{sous-entrée}{pao-du piò}{ⓔpaoⓢ1ⓝpao-du piò}
\begin{glose}
\pfra{frapper avec la pioche (lancer la pioche en bas)}
\end{glose}
\end{sous-entrée}
\newline
\begin{sous-entrée}{pao jige}{ⓔpaoⓢ1ⓝpao jige}
\begin{glose}
\pfra{tirer (au fusil)}
\end{glose}
\end{sous-entrée}
\newline
\begin{sous-entrée}{pao paa}{ⓔpaoⓢ1ⓝpao paa}
\begin{glose}
\pfra{jeter des cailloux}
\end{glose}
\newline
\begin{exemple}
\région{BO}
\textbf{\pnua{i pao dèn}}
\pfra{le vent souffle fort}
\end{exemple}
\newline
\relationsémantique{Cf.}{\lien{ⓔpawe}{pawe}}
\glosecourte{lancer à côté}
\end{sous-entrée}
\newline
\sens{2}
(\domainesémantique{Verbes de déplacement et moyens de déplacement})
\begin{glose}
\pfra{prendre la route ; mettre (se) en route}
\end{glose}
\end{entrée}

\begin{entrée}{pao-biini}{}{ⓔpao-biini}
\région{PA}
(\domainesémantique{Verbes d'action (en général)})
\classe{v}
\begin{glose}
\pfra{cabosser}
\end{glose}
\end{entrée}

\begin{entrée}{pao-cabi}{}{ⓔpao-cabi}
\région{PA}
(\domainesémantique{Mouvements ou actions faits avec le corps, les bras, les mains, les pieds})
\classe{v}
\begin{glose}
\pfra{tambouriner}
\end{glose}
\end{entrée}

\begin{entrée}{pao-dale}{}{ⓔpao-dale}
\région{PA}
(\domainesémantique{Mouvements ou actions faits avec le corps, les bras, les mains, les pieds})
\classe{v}
\begin{glose}
\pfra{casser en jetant}
\end{glose}
\end{entrée}

\begin{entrée}{pao-gaaò}{}{ⓔpao-gaaò}
\région{PA}
(\domainesémantique{Mouvements ou actions faits avec le corps, les bras, les mains, les pieds})
\classe{v}
\begin{glose}
\pfra{casser (du verre en jetant)}
\end{glose}
\end{entrée}

\begin{entrée}{pao-kaö}{}{ⓔpao-kaö}
\région{PA}
(\domainesémantique{Mouvements ou actions faits avec le corps, les bras, les mains, les pieds})
\classe{v}
\begin{glose}
\pfra{jeter par dessus ; lancer par dessus}
\end{glose}
\end{entrée}

\begin{entrée}{pao-kibi}{}{ⓔpao-kibi}
\région{PA}
(\domainesémantique{Verbes d'action (en général)})
\classe{v}
\begin{glose}
\pfra{faire éclater (en jetant)}
\end{glose}
\newline
\begin{exemple}
\région{PA}
\textbf{\pnua{la pao-kibi bwaa-n}}
\pfra{ils lui ont fracassé la tête (avec un casse-tête)}
\end{exemple}
\end{entrée}

\begin{entrée}{pao kinu}{}{ⓔpao kinu}
\région{GOs}
(\domainesémantique{Navigation})
\classe{v}
\begin{glose}
\pfra{ancrer ; jeter l'ancre}
\end{glose}
\end{entrée}

\begin{entrée}{pao-khi-dale}{}{ⓔpao-khi-dale}
\région{PA}
(\domainesémantique{Mouvements ou actions faits avec le corps, les bras, les mains, les pieds})
\classe{v}
\begin{glose}
\pfra{fracasser (en jetant)}
\end{glose}
\end{entrée}

\begin{entrée}{paophi}{}{ⓔpaophi}
\région{GOs}
(\domainesémantique{Pêche})
\classe{v}
\begin{glose}
\pfra{faire du bruit dans l'eau avec les mains pour effrayer les poissons (comme pour le jeu "thathibul")}
\end{glose}
\end{entrée}

\begin{entrée}{pao-thali}{}{ⓔpao-thali}
\région{PA}
(\domainesémantique{Mouvements ou actions faits avec le corps, les bras, les mains, les pieds})
\classe{v}
\begin{glose}
\pfra{frapper (enfant)}
\end{glose}
\end{entrée}

\begin{entrée}{pa-pii-ni}{}{ⓔpa-pii-ni}
\formephonétique{pa-piː-ɳi}
\région{GOs}
(\domainesémantique{Préparation des aliments; modes de préparation et de cuisson})
\classe{v}
\begin{glose}
\pfra{vider qqch (marmite)}
\end{glose}
\end{entrée}

\begin{entrée}{pa-pònume}{}{ⓔpa-pònume}
\région{PA BO}
(\domainesémantique{Verbes d'action (en général)})
\classe{v.t.}
\begin{glose}
\pfra{raccourcir}
\end{glose}
\end{entrée}

\begin{entrée}{papua}{}{ⓔpapua}
\région{GOs BO}
(\domainesémantique{Ignames})
\classe{nom}
\begin{glose}
\pfra{igname longue et dure}
\end{glose}
\end{entrée}

\begin{entrée}{pa-pho}{}{ⓔpa-pho}
\formephonétique{pa-pʰo}
\région{GOs PA}
(\domainesémantique{Tressage})
\classe{v ; n}
\begin{glose}
\pfra{tresser ; tressage ; faire de la vannerie}
\end{glose}
\newline
\begin{exemple}
\région{GOs}
\textbf{\pnua{novwö pa-pho ca/ça mogu i baa-êgu}}
\pfra{la vannerie, c'est le travail des femmes}
\end{exemple}
\newline
\relationsémantique{Cf.}{\lien{}{pai-pho}}
\end{entrée}

\begin{entrée}{pa-phuu-ni}{}{ⓔpa-phuu-ni}
\formephonétique{p'a-pʰuː-ɳi (accent sur initiale)}
\région{GOs}
(\domainesémantique{Verbes d'action (en général)})
\classe{v}
\begin{glose}
\pfra{gonfler qqch}
\end{glose}
\newline
\relationsémantique{Cf.}{\lien{ⓔphuu}{phuu}}
\glosecourte{enflé}
\end{entrée}

\begin{entrée}{pa-phwa-ni}{}{ⓔpa-phwa-ni}
\formephonétique{pa-pʰwa-ɳi}
\région{GOs BO}
(\domainesémantique{Verbes d'action (en général)})
\classe{v}
\begin{glose}
\pfra{trouer (ballon)}
\end{glose}
\begin{glose}
\pfra{déchirer (tissu)}
\end{glose}
\newline
\begin{exemple}
\région{GOs}
\textbf{\pnua{e paa-phwa-ni hõbwòli-nu}}
\pfra{il a déchiré ma robe}
\end{exemple}
\end{entrée}

\begin{entrée}{para}{1}{ⓔparaⓗ1}
\région{BO [Corne]}
(\domainesémantique{Noms des plantes})
\classe{nom}
\begin{glose}
\pfra{herbe (poussant le long des rivières)}
\end{glose}
\nomscientifique{Brachiara mutica}
\end{entrée}

\begin{entrée}{para}{2}{ⓔparaⓗ2}
\région{GOs}
(\domainesémantique{Marques de degré})
\classe{QNT}
\begin{glose}
\pfra{très}
\end{glose}
\newline
\begin{exemple}
\région{GOs}
\textbf{\pnua{e para-poonu}}
\pfra{il est très petit}
\end{exemple}
\newline
\note{ne s'utilise qu'en référence à des humains/animés; mhaa s'utilise pour des inanimés}{grammaire}{}
\end{entrée}

\begin{entrée}{parang}{}{ⓔparang}
\région{BO}
(\domainesémantique{Préparation des aliments; modes de préparation et de cuisson})
\classe{v}
\begin{glose}
\pfra{rôtir}
\end{glose}
\newline
\begin{sous-entrée}{ba-parang}{ⓔparangⓝba-parang}
\begin{glose}
\pfra{poêle à frire}
\end{glose}
\end{sous-entrée}
\end{entrée}

\begin{entrée}{paree}{}{ⓔparee}
\région{PA BO}
(\domainesémantique{Relations et interaction sociales})
\classe{v}
\begin{glose}
\pfra{négliger ; délaisser ; abandonner}
\end{glose}
\newline
\relationsémantique{Cf.}{\lien{ⓔthaxiba}{thaxiba}}
\glosecourte{rejeter}
\end{entrée}

\begin{entrée}{parèma}{}{ⓔparèma}
\région{GOs}
\variante{%
parèman
\région{PA}}
(\domainesémantique{Noms locatifs})
\classe{nom}
\begin{glose}
\pfra{côte est ; rivage ; bord de mer}
\end{glose}
\newline
\begin{sous-entrée}{parèma pomwòli}{ⓔparèmaⓝparèma pomwòli}
\région{GOs}
\begin{glose}
\pfra{côte est}
\end{glose}
\end{sous-entrée}
\newline
\begin{sous-entrée}{parèma pomwã}{ⓔparèmaⓝparèma pomwã}
\région{GOs}
\begin{glose}
\pfra{côte ouest}
\end{glose}
\end{sous-entrée}
\end{entrée}

\begin{entrée}{parixo}{}{ⓔparixo}
\région{GOs}
(\domainesémantique{Verbes de mouvement})
\classe{v}
\begin{glose}
\pfra{faire des culbutes ; faire des tonneaux (voiture)}
\end{glose}
\end{entrée}

\begin{entrée}{pa-tòè}{}{ⓔpa-tòè}
\formephonétique{pa-tɔɛ}
\région{GOs}
(\domainesémantique{Préparation des aliments; modes de préparation et de cuisson})
\classe{v}
\begin{glose}
\pfra{réchauffer (nourriture)}
\end{glose}
\newline
\begin{exemple}
\textbf{\pnua{la pa-tòè dro}}
\pfra{elles réchauffent les marmites}
\end{exemple}
\end{entrée}

\begin{entrée}{pa-trevwaò}{}{ⓔpa-trevwaò}
\formephonétique{pa-ʈeβaɔ}
\région{GOs}
\variante{%
pa-tevwaò
\formephonétique{pateβaɔ}
\région{PA BO}}
(\domainesémantique{Mouvements ou actions faits avec le corps, les bras, les mains, les pieds})
\classe{v}
\begin{glose}
\pfra{jeter}
\end{glose}
\newline
\begin{exemple}
\région{GOs}
\textbf{\pnua{e pa-trevwaò ja-je bwabu}}
\pfra{elle a jeté ses saletés par terre}
\end{exemple}
\end{entrée}

\begin{entrée}{patrô}{}{ⓔpatrô}
\formephonétique{paʈõ}
\région{GOs}
\variante{%
parô-, parôô-n
\région{BO PA}}
(\domainesémantique{Corps humain})
\classe{nom}
\begin{glose}
\pfra{dent}
\end{glose}
\newline
\begin{exemple}
\région{PA}
\textbf{\pnua{parôô-n}}
\pfra{ses dents}
\end{exemple}
\newline
\relationsémantique{Cf.}{\lien{ⓔwhau}{whau}}
\glosecourte{édenté}
\end{entrée}

\begin{entrée}{pa-thrô}{}{ⓔpa-thrô}
\formephonétique{pa-ʈʰõ}
\région{GOs}
(\domainesémantique{Tressage})
\classe{v}
\begin{glose}
\pfra{tresser une natte ; faire une natte}
\end{glose}
\end{entrée}

\begin{entrée}{pavada}{}{ⓔpavada}
\région{GOs}
\variante{%
pavade
\région{PA WE}}
(\domainesémantique{Sons, bruits})
\classe{v}
\begin{glose}
\pfra{bruit (faire du)}
\end{glose}
\newline
\begin{exemple}
\textbf{\pnua{pavade kiva do}}
\pfra{le couvercle de la marmite fait du bruit}
\end{exemple}
\end{entrée}

\begin{entrée}{pavo-romwa}{}{ⓔpavo-romwa}
\région{BO}
(\domainesémantique{Organisation sociale})
\classe{nom}
\begin{glose}
\pfra{emplacement des femmes dans une réunion (Dubois)}
\end{glose}
\newline
\note{non vérifié}{général}{}
\end{entrée}

\begin{entrée}{pavwa}{}{ⓔpavwa}
\formephonétique{paβa}
\région{GOs}
(\domainesémantique{Verbes d'action (en général)})
\classe{v.i.}
\begin{glose}
\pfra{préparatifs (faire les)}
\end{glose}
\newline
\begin{exemple}
\région{GOs}
\textbf{\pnua{tree pavwa ma iwa mwa a}}
\pfra{préparez-vous à partir}
\end{exemple}
\newline
\begin{exemple}
\région{GOs}
\textbf{\pnua{la pavwa ponga nyã-mòlò}}
\pfra{ils font les préparatifs pour les fêtes coutumières}
\end{exemple}
\newline
\begin{exemple}
\région{GOs}
\textbf{\pnua{la pavwa ponga pe-navwo}}
\pfra{ils font les préparatifs pour les fêtes coutumières}
\end{exemple}
\end{entrée}

\begin{entrée}{pavwã}{}{ⓔpavwã}
\formephonétique{paβɛ̃}
\région{GOs}
(\domainesémantique{Fonctions intellectuelles})
\classe{v ; n}
\begin{glose}
\pfra{confiance (avoir) ; espoir ; espérer}
\end{glose}
\end{entrée}

\begin{entrée}{pavwange}{}{ⓔpavwange}
\formephonétique{paβaŋe}
\région{GOs}
\variante{%
pavang, pavange
\région{BO}}
(\domainesémantique{Verbes d'action (en général)})
\classe{v.t.}
\begin{glose}
\pfra{préparer (en général) ; faire des parts (vivres)}
\end{glose}
\newline
\relationsémantique{Cf.}{\lien{ⓔpavwange}{pavwange}}
\glosecourte{préparer, prendre soin}
\end{entrée}

\begin{entrée}{pavwaze}{}{ⓔpavwaze}
\formephonétique{paβaze}
\région{GOs}
(\domainesémantique{Tradition orale})
\classe{v}
\begin{glose}
\pfra{réciter les généalogies}
\end{glose}
\newline
\begin{exemple}
\textbf{\pnua{êgu e aa-pavwaze}}
\pfra{celui qui fait le discours sur les généalogies}
\end{exemple}
\end{entrée}

\begin{entrée}{pawe}{}{ⓔpawe}
\région{GOs BO}
\variante{%
paawe
\région{BO}}
(\domainesémantique{Mouvements ou actions faits avec le corps, les bras, les mains, les pieds})
\classe{v}
\begin{glose}
\pfra{jeter ; lancer ; frapper (de haut en bas)}
\end{glose}
\newline
\begin{exemple}
\région{GOs}
\textbf{\pnua{hê-kee-nu la no nu pawe pwiò}}
\pfra{j'ai pris ces poissons au filet épervier (lit. en lançant le filet épervier)}
\end{exemple}
\end{entrée}

\begin{entrée}{pawe hii-je}{}{ⓔpawe hii-je}
\région{GOs}
(\domainesémantique{Relations et interaction sociales})
\classe{v}
\begin{glose}
\pfra{faire un signe de la main}
\end{glose}
\newline
\begin{exemple}
\textbf{\pnua{e pawe hii-je}}
\pfra{il lui a fait un signe de la main}
\end{exemple}
\newline
\relationsémantique{Cf.}{\lien{ⓔpao}{pao}}
\glosecourte{jeter}
\end{entrée}

\begin{entrée}{paxa}{}{ⓔpaxa}
\région{PA}
(\domainesémantique{Noms des plantes})
\classe{nom}
\begin{glose}
\pfra{herbes vertes (poussant dans les rivières ou terrains marécageux)}
\end{glose}
\end{entrée}

\begin{entrée}{paxa-}{}{ⓔpaxa-}
\région{GOs}
(\domainesémantique{Préfixes sémantiques divers})
\classe{PREF}
\begin{glose}
\pfra{contenant ; produit de}
\end{glose}
\newline
\begin{sous-entrée}{paxa-no-n}{ⓔpaxa-ⓝpaxa-no-n}
\begin{glose}
\pfra{estomac}
\end{glose}
\end{sous-entrée}
\newline
\begin{sous-entrée}{paxa-zume i je}{ⓔpaxa-ⓝpaxa-zume i je}
\begin{glose}
\pfra{son crachat}
\end{glose}
\end{sous-entrée}
\newline
\begin{sous-entrée}{paxa-wa}{ⓔpaxa-ⓝpaxa-wa}
\begin{glose}
\pfra{thème de chant}
\end{glose}
\end{sous-entrée}
\newline
\begin{sous-entrée}{paxa-kîbi}{ⓔpaxa-ⓝpaxa-kîbi}
\begin{glose}
\pfra{pierres de four}
\end{glose}
\end{sous-entrée}
\newline
\begin{sous-entrée}{paxa-vhaa}{ⓔpaxa-ⓝpaxa-vhaa}
\begin{glose}
\pfra{mots}
\end{glose}
\end{sous-entrée}
\newline
\begin{sous-entrée}{paxa-ze [GOs]}{ⓔpaxa-ⓝpaxa-ze [GOs]}
\begin{glose}
\pfra{expectorations}
\end{glose}
\newline
\relationsémantique{Cf.}{\lien{}{hê}}
\end{sous-entrée}
\end{entrée}

\begin{entrée}{paxaa}{}{ⓔpaxaa}
\région{GOs PA}
(\domainesémantique{Santé, maladie})
\classe{nom}
\begin{glose}
\pfra{maladie contagieuse (grippe, etc.)}
\end{glose}
\newline
\begin{exemple}
\textbf{\pnua{nu tòòli paxaa}}
\pfra{j'ai attrapé la grippe}
\end{exemple}
\end{entrée}

\begin{entrée}{paxa-dili}{}{ⓔpaxa-dili}
\région{PA BO [Corne]}
(\domainesémantique{Cultures, techniques, boutures})
\classe{nom}
\begin{glose}
\pfra{motte de terre}
\end{glose}
\end{entrée}

\begin{entrée}{paxadraa}{}{ⓔpaxadraa}
\région{GOs}
\variante{%
paxadaa
\région{WEM}}
(\domainesémantique{Préparation des aliments; modes de préparation et de cuisson})
\classe{v}
\begin{glose}
\pfra{frire}
\end{glose}
\newline
\begin{sous-entrée}{ba-paxadraa}{ⓔpaxadraaⓝba-paxadraa}
\begin{glose}
\pfra{poêle à frire}
\end{glose}
\end{sous-entrée}
\end{entrée}

\begin{entrée}{paxa-ka}{}{ⓔpaxa-ka}
\région{GOs}
(\domainesémantique{Ignames})
\classe{nom}
\begin{glose}
\pfra{prémices (la première igname récoltée de l'année)}
\end{glose}
\newline
\relationsémantique{Cf.}{\lien{ⓔkaⓗ1}{ka}}
\glosecourte{an}
\end{entrée}

\begin{entrée}{paxa-kîbi}{}{ⓔpaxa-kîbi}
\région{GOs BO PA}
\variante{%
paxawa kîbi
\région{GO(s)}}
(\domainesémantique{Préparation des aliments; modes de préparation et de cuisson})
\classe{nom}
\begin{glose}
\pfra{pierres du four enterré}
\end{glose}
\newline
\begin{sous-entrée}{paxa-kîbi [GOs]}{ⓔpaxa-kîbiⓝpaxa-kîbi [GOs]}
\begin{glose}
\pfra{pierres du four}
\end{glose}
\newline
\relationsémantique{Cf.}{\lien{ⓔpaaⓗ1}{paa}}
\glosecourte{pierre}
\newline
\relationsémantique{Cf.}{\lien{ⓔkîbi}{kîbi}}
\glosecourte{four}
\end{sous-entrée}
\end{entrée}

\begin{entrée}{paxa-kui}{}{ⓔpaxa-kui}
\région{GOs BO PA}
(\domainesémantique{Parties de plantes})
\classe{nom}
\begin{glose}
\pfra{tubercule d'igname}
\end{glose}
\end{entrée}

\begin{entrée}{paxa-kumwala}{}{ⓔpaxa-kumwala}
\région{GOs PA}
(\domainesémantique{Parties de plantes})
\classe{nom}
\begin{glose}
\pfra{tubercule de patate douce}
\end{glose}
\newline
\begin{exemple}
\textbf{\pnua{kia paxa-kumwala}}
\pfra{il n'y a pas de patate douce}
\end{exemple}
\end{entrée}

\begin{entrée}{paxa-nãã-n}{}{ⓔpaxa-nãã-n}
\région{PA BO}
(\domainesémantique{Relations et interaction sociales})
\classe{nom}
\begin{glose}
\pfra{injure ; insulte ; offense}
\end{glose}
\begin{glose}
\pfra{mauvais sort}
\end{glose}
\newline
\relationsémantique{Cf.}{\lien{ⓔnãã-n}{nãã-n}}
\glosecourte{injure}
\end{entrée}

\begin{entrée}{paxa-nò}{}{ⓔpaxa-nò}
\formephonétique{pa'ɣa-nɔ}
\région{GOs PA BO}
\variante{%
paga-nò
\région{BO}}
(\domainesémantique{Corps humain})
\classe{nom}
\begin{glose}
\pfra{gésier; estomac}
\end{glose}
\newline
\begin{exemple}
\région{GO}
\textbf{\pnua{paxa-nò-nu}}
\pfra{mon estomac}
\end{exemple}
\newline
\begin{exemple}
\textbf{\pnua{paxa-nò-n [PA, BO]}}
\pfra{son estomac}
\end{exemple}
\end{entrée}

\begin{entrée}{paxa-nô-döö}{}{ⓔpaxa-nô-döö}
\région{PA}
(\domainesémantique{Description des objets, formes, consistance, taille})
\classe{nom}
\begin{glose}
\pfra{fond de la marmite (face externe)}
\end{glose}
\end{entrée}

\begin{entrée}{paxa-nõ-pu}{}{ⓔpaxa-nõ-pu}
\région{BO}
(\domainesémantique{Fonctions naturelles humaines})
\classe{nom}
\begin{glose}
\pfra{crachat (de grippe) [Corne]}
\end{glose}
\end{entrée}

\begin{entrée}{paxa-pwe}{}{ⓔpaxa-pwe}
\région{GOs}
(\domainesémantique{Pêche})
\classe{nom}
\begin{glose}
\pfra{plomb de la ligne}
\end{glose}
\end{entrée}

\begin{entrée}{paxa-pwiò}{}{ⓔpaxa-pwiò}
\région{GOs}
(\domainesémantique{Pêche})
\classe{nom}
\begin{glose}
\pfra{plomb du filet}
\end{glose}
\end{entrée}

\begin{entrée}{paxa-uva}{}{ⓔpaxa-uva}
\région{GOs BO PA}
(\domainesémantique{Parties de plantes})
\classe{nom}
\begin{glose}
\pfra{tubercule du taro d'eau}
\end{glose}
\end{entrée}

\begin{entrée}{paxa-vha}{}{ⓔpaxa-vha}
\formephonétique{pa'ɣa-va}
\région{GOs}
(\domainesémantique{Discours, échanges verbaux})
\classe{nom}
\begin{glose}
\pfra{mot (lit. produit de la parole)}
\end{glose}
\end{entrée}

\begin{entrée}{paxawa}{}{ⓔpaxawa}
\région{GOs PA}
(\domainesémantique{Noms des plantes})
\classe{nom}
\begin{glose}
\pfra{coléus (symbole de vie)}
\end{glose}
\newline
\note{(un bouquet de coleus accompagne la monnaie traditionnelle donnée aux oncles maternels à la naissance de l'enfant)}{glose}{}
\nomscientifique{Lamiaceae Plectranthus scutellarioides (L.) R.Br. (Caballion); Solenostemon scutellarioides (Labiacées)}
\end{entrée}

\begin{entrée}{paxa-wa}{}{ⓔpaxa-wa}
\région{GOs BO}
\variante{%
paga-wal
\région{PA}}
(\domainesémantique{Musique, instruments de musique})
\classe{nom}
\begin{glose}
\pfra{paroles de la chanson ; thème d'un chant}
\end{glose}
\newline
\begin{exemple}
\région{GOs}
\textbf{\pnua{la paxa wa}}
\pfra{les paroles de la chanson}
\end{exemple}
\newline
\begin{exemple}
\région{PA}
\textbf{\pnua{la paga wal}}
\pfra{les paroles de la chanson}
\end{exemple}
\end{entrée}

\begin{entrée}{paxa-we}{}{ⓔpaxa-we}
\région{GOs}
(\domainesémantique{Pêche
, Configuration des objets})
\classe{nom}
\begin{glose}
\pfra{trou d'eau (dans une rivière, dans la mer)}
\end{glose}
\end{entrée}

\begin{entrée}{paxe}{}{ⓔpaxe}
\région{PA BO}
(\domainesémantique{Actions liées aux éléments (liquide, fumée)})
\classe{v}
\begin{glose}
\pfra{dévier (eau)}
\end{glose}
\newline
\begin{exemple}
\textbf{\pnua{paxe we}}
\pfra{dévier l'eau}
\end{exemple}
\newline
\begin{exemple}
\textbf{\pnua{nu paxe hale we}}
\pfra{j'ai dévié l'eau}
\end{exemple}
\end{entrée}

\begin{entrée}{paxeze}{}{ⓔpaxeze}
\région{GOs}
(\domainesémantique{Relations et interaction sociales})
\classe{v}
\begin{glose}
\pfra{interdire qqch (momentanément) ; empêcher (qqn de faire qqch)}
\end{glose}
\newline
\begin{exemple}
\textbf{\pnua{e paxeze-nu wo nu a-mi kòlò-je}}
\pfra{il m'a empêché de venir chez lui}
\end{exemple}
\end{entrée}

\begin{entrée}{paxu}{}{ⓔpaxu}
\région{GOs PA}
\classe{v}
\newline
\sens{1}
(\domainesémantique{Vêtements, parure})
\begin{glose}
\pfra{nu (être)}
\end{glose}
\newline
\begin{exemple}
\textbf{\pnua{i paxu}}
\pfra{il est nu}
\end{exemple}
\newline
\sens{2}
(\domainesémantique{Quantificateurs})
\begin{glose}
\pfra{manquer}
\end{glose}
\newline
\begin{exemple}
\région{GOs}
\textbf{\pnua{paxu bwedòò hii-je}}
\pfra{il lui manque un doigt}
\end{exemple}
\newline
\relationsémantique{Cf.}{\lien{}{bwehulo ; bwexulo}}
\glosecourte{manquer}
\end{entrée}

\begin{entrée}{payang}{}{ⓔpayang}
\région{BO}
(\domainesémantique{Verbes de déplacement et moyens de déplacement})
\classe{nom}
\begin{glose}
\pfra{grande route}
\end{glose}
\end{entrée}

\begin{entrée}{pazalõ}{}{ⓔpazalõ}
(\domainesémantique{Vêtements, parure})
\classe{nom}
\begin{glose}
\pfra{pantalon}
\end{glose}
\newline
\begin{sous-entrée}{pajalõ pagoo [GOs]}{ⓔpazalõⓝpajalõ pagoo [GOs]}
\begin{glose}
\pfra{short}
\end{glose}
\end{sous-entrée}
\end{entrée}

\begin{entrée}{pa-zo}{}{ⓔpa-zo}
\région{PA}
\variante{%
payo
\région{BO}}
(\domainesémantique{Cultures, techniques, boutures})
\classe{v}
\begin{glose}
\pfra{qui donne/produit bien (champ)}
\end{glose}
\end{entrée}

\begin{entrée}{pe}{1}{ⓔpeⓗ1}
\région{GOs BOPA}
(\domainesémantique{Poissons})
\classe{nom}
\begin{glose}
\pfra{raie}
\end{glose}
\newline
\étymologie{
\langue{POc}
\étymon{*paRi}
\glosecourte{raie}}
\end{entrée}

\begin{entrée}{pe}{2}{ⓔpeⓗ2}
\région{PA}
(\domainesémantique{Chasse})
\classe{nom}
\begin{glose}
\pfra{plateforme posée dans les arbres situés sur le passage des roussettes}
\end{glose}
\newline
\note{(les chasseurs s'y postaient et interceptaient les roussettes avec des branches pour les faire tomber au sol)}{glose}{}
\newline
\étymologie{
\langue{POc}
\étymon{*pataR}
\glosecourte{platform of any kind}
\auteur{Blust}}
\end{entrée}

\begin{entrée}{pe}{3}{ⓔpeⓗ3}
\région{PA BO}
(\domainesémantique{Types de maison, architecture de la maison})
\classe{nom}
\begin{glose}
\pfra{gaulette formant la corbeille}
\end{glose}
\end{entrée}

\begin{entrée}{pe-}{1}{ⓔpe-ⓗ1}
\région{BO}
(\domainesémantique{Aspect})
\classe{ASP}
\begin{glose}
\pfra{en train de (marque la durée)}
\end{glose}
\newline
\begin{exemple}
\textbf{\pnua{novo dony i pe-na-hin bwa dèèn}}
\pfra{la buse quant à elle plane avec le vent (Joachim)}
\end{exemple}
\end{entrée}

\begin{entrée}{pe-}{2}{ⓔpe-ⓗ2}
\région{GOs PA}
\newline
\sens{1}
(\domainesémantique{Réciproque})
\classe{PREF.REC}
\begin{glose}
\pfra{mutuellement}
\end{glose}
\newline
\begin{exemple}
\textbf{\pnua{la pe-khõbwi}}
\pfra{ils discutent}
\end{exemple}
\newline
\begin{exemple}
\textbf{\pnua{pe-abaa-li}}
\pfra{ils sont frères}
\end{exemple}
\newline
\begin{exemple}
\région{PA, GO}
\textbf{\pnua{pe-dilo-la}}
\pfra{ils sont égaux (Dubois)}
\end{exemple}
\newline
\sens{2}
(\domainesémantique{Réciproque collectif})
\classe{COLL}
\begin{glose}
\pfra{ensemble (devant un nombre: marque un tout)}
\end{glose}
\newline
\begin{exemple}
\textbf{\pnua{la pe-taivi}}
\pfra{ils se réunissent}
\end{exemple}
\newline
\begin{exemple}
\région{GOs}
\textbf{\pnua{pe-poniza?}}
\pfra{combien y en a-t-il en tout?}
\end{exemple}
\newline
\begin{exemple}
\région{PA, GO}
\textbf{\pnua{pe-ponira?}}
\pfra{combien y en a-t-il en tout? (Dubois)}
\end{exemple}
\newline
\begin{exemple}
\région{PA, GO}
\textbf{\pnua{pe-ponimadu}}
\pfra{il y en a 7 en tout (Dubois)}
\end{exemple}
\newline
\begin{sous-entrée}{pepe}{ⓔpe-ⓗ2ⓢ2ⓝpepe}
\begin{glose}
\pfra{tous ensemble (Dubois)}
\end{glose}
\end{sous-entrée}
\newline
\sens{3}
(\domainesémantique{Dispersifs})
\begin{glose}
\pfra{dispersif ; sans but ; comme ça (ou activité non bornée)}
\end{glose}
\newline
\begin{exemple}
\région{PA}
\textbf{\pnua{li u a-da mwa a-ve-hale na lina pwamwa}}
\pfra{ils s'en retournent dans leur terroir respectif}
\end{exemple}
\newline
\begin{exemple}
\région{BO}
\textbf{\pnua{nu ga pe-kû pò-puleng}}
\pfra{je suis en train de manger un fruit de Pipturus (atélique)}
\end{exemple}
\end{entrée}

\begin{entrée}{pè}{}{ⓔpè}
\région{GOs BO PA}
(\domainesémantique{Corps humain})
\classe{nom}
\begin{glose}
\pfra{cuisse ; fesse}
\end{glose}
\newline
\begin{exemple}
\textbf{\pnua{pèè-nu [GO]}}
\pfra{ma cuisse, mes fesses}
\end{exemple}
\newline
\begin{exemple}
\textbf{\pnua{pèè-n [PA]}}
\pfra{sa cuisse, ses fesses}
\end{exemple}
\newline
\begin{sous-entrée}{po-pee}{ⓔpèⓝpo-pee}
\begin{glose}
\pfra{petites cuisses}
\end{glose}
\end{sous-entrée}
\newline
\étymologie{
\langue{POc}
\étymon{*paqa(l)}}
\end{entrée}

\begin{entrée}{pe-a}{}{ⓔpe-a}
\région{GOs}
(\domainesémantique{Verbes de déplacement et moyens de déplacement})
\classe{v}
\begin{glose}
\pfra{aller (sans but)}
\end{glose}
\newline
\begin{exemple}
\textbf{\pnua{e pe-a hoze koli we}}
\pfra{il suit la berge de la rivière}
\end{exemple}
\newline
\begin{exemple}
\textbf{\pnua{e pe-a ayu}}
\pfra{il va sans but}
\end{exemple}
\end{entrée}

\begin{entrée}{pe-a-bala}{}{ⓔpe-a-bala}
\région{GO PA}
(\domainesémantique{Réciproque collectif})
\classe{v ; n}
\begin{glose}
\pfra{même équipe}
\end{glose}
\newline
\begin{exemple}
\textbf{\pnua{pe-a-bala-bi}}
\pfra{nous sommes tous deux dans la même équipe}
\end{exemple}
\newline
\begin{exemple}
\région{GO}
\textbf{\pnua{mõ pe-a-bala ?}}
\pfra{sommes-nous tous trois dans la même équipe?}
\end{exemple}
\end{entrée}

\begin{entrée}{pe-a-da}{}{ⓔpe-a-da}
\région{GO}
(\domainesémantique{Réciproque collectif})
\classe{COLL}
\begin{glose}
\pfra{monter ensemble}
\end{glose}
\newline
\begin{exemple}
\textbf{\pnua{bi pe-a-da nògò}}
\pfra{nous deux allons monter à la rivière ensemble}
\end{exemple}
\end{entrée}

\begin{entrée}{pe-a-hoze}{}{ⓔpe-a-hoze}
\région{GOs}
(\domainesémantique{Verbes de déplacement et moyens de déplacement})
\classe{v}
\begin{glose}
\pfra{longer qqch (activité)}
\end{glose}
\newline
\begin{exemple}
\région{GOs}
\textbf{\pnua{e pe-a-hoze koli we-za}}
\pfra{il longe le bord de la mer}
\end{exemple}
\end{entrée}

\begin{entrée}{pe-a-kai-n}{}{ⓔpe-a-kai-n}
\région{GOs PA}
(\domainesémantique{Verbes de déplacement et moyens de déplacement})
\classe{v}
\begin{glose}
\pfra{suivre (se) ; suivre ; marcher en file indienne}
\end{glose}
\newline
\begin{exemple}
\région{GOs}
\textbf{\pnua{la pe-a-kai-la xo la ègu}}
\pfra{les gens marchent en file}
\end{exemple}
\newline
\begin{exemple}
\région{GOs}
\textbf{\pnua{la a-kai lhaaba êgu xo lhaaba êmwê}}
\pfra{ces hommes suivent ces gens}
\end{exemple}
\end{entrée}

\begin{entrée}{peale}{}{ⓔpeale}
\région{PA BO}
\classe{v ; n}
\newline
\sens{1}
(\domainesémantique{Dispersifs})
\begin{glose}
\pfra{à chacun ; à part}
\end{glose}
\newline
\sens{2}
(\domainesémantique{Dispersifs})
\begin{glose}
\pfra{différent ; distinct}
\end{glose}
\newline
\begin{exemple}
\textbf{\pnua{ge-ã ni peale}}
\pfra{nous sommes dans (des endroits) différents}
\end{exemple}
\newline
\begin{exemple}
\textbf{\pnua{peale vhaa i la}}
\pfra{leur langage est distinct}
\end{exemple}
\end{entrée}

\begin{entrée}{pe-alo}{}{ⓔpe-alo}
\région{BO}
(\domainesémantique{Aspect})
\classe{ASP}
\begin{glose}
\pfra{en train de regarder}
\end{glose}
\newline
\begin{exemple}
\région{BO}
\textbf{\pnua{nu hâ pe-alo}}
\pfra{j'admire la belle vue}
\end{exemple}
\end{entrée}

\begin{entrée}{pe-a-vwe}{}{ⓔpe-a-vwe}
\région{GOs}
(\domainesémantique{Réciproque collectif})
\classe{v}
\begin{glose}
\pfra{ensemble (être)}
\end{glose}
\newline
\begin{exemple}
\textbf{\pnua{la pe-a-vwe bulu}}
\pfra{ils sont ensemble en groupe}
\end{exemple}
\newline
\begin{exemple}
\textbf{\pnua{la pe-a bulu}}
\pfra{ils y vont ensemble}
\end{exemple}
\end{entrée}

\begin{entrée}{pe-bala}{}{ⓔpe-bala}
\région{GOs}
(\domainesémantique{Société
, Réciproque collectif})
\classe{nom}
\begin{glose}
\pfra{même équipe (être dans la)}
\end{glose}
\newline
\begin{exemple}
\textbf{\pnua{pe-bala-la}}
\pfra{ils sont dans la même équipe}
\end{exemple}
\end{entrée}

\begin{entrée}{pe-balan}{}{ⓔpe-balan}
\région{BO}
(\domainesémantique{Comparaison})
\classe{v}
\begin{glose}
\pfra{ressembler (se) ; pareil (être)}
\end{glose}
\newline
\begin{exemple}
\textbf{\pnua{la pe-balan}}
\pfra{ils se ressemblent}
\end{exemple}
\end{entrée}

\begin{entrée}{pe-bu}{}{ⓔpe-bu}
\région{GOs WEM}
\classe{v}
\newline
\sens{1}
(\domainesémantique{Guerre
, Réciproque collectif})
\begin{glose}
\pfra{bagarer (se) ; battre (se)}
\end{glose}
\newline
\sens{2}
(\domainesémantique{Verbes d'action (en général)
, Réciproque collectif})
\begin{glose}
\pfra{cogner (se) ; entrer en collision}
\end{glose}
\newline
\begin{exemple}
\région{GOs}
\textbf{\pnua{li pe-bu-i-li}}
\pfra{ils sont entrés en collision}
\end{exemple}
\newline
\note{bule (v.t.)}{grammaire}{cogner qqch ou qqn}
\end{entrée}

\begin{entrée}{pe-bulu}{}{ⓔpe-bulu}
\région{GOs BO}
(\domainesémantique{Relations et interaction sociales
, Réciproque collectif})
\classe{COLL}
\begin{glose}
\pfra{ensemble}
\end{glose}
\newline
\begin{exemple}
\région{GOs}
\textbf{\pnua{mi ne pe-bulu-ni}}
\pfra{nous le faisons ensemble}
\end{exemple}
\end{entrée}

\begin{entrée}{pecabi}{}{ⓔpecabi}
\région{BO}
(\domainesémantique{Santé, maladie})
\classe{v.stat.}
\begin{glose}
\pfra{endolori [BO, Corne, Haudricourt]}
\end{glose}
\end{entrée}

\begin{entrée}{pe-çabi}{}{ⓔpe-çabi}
\formephonétique{pe-ʒambi}
\région{GOs}
\variante{%
pe-çabi
\formephonétique{pe-dʒambi}
\région{GO(s)}, 
pe-cabi
\région{BO [Corne]}}
(\domainesémantique{Réciproque collectif
, Guerre})
\classe{v}
\begin{glose}
\pfra{battre (se)}
\end{glose}
\end{entrée}

\begin{entrée}{peçi}{}{ⓔpeçi}
\région{GOs}
\variante{%
peyi
\région{PA}}
(\domainesémantique{Mollusques})
\classe{nom}
\begin{glose}
\pfra{"savonnette" (coquillage)}
\end{glose}
\end{entrée}

\begin{entrée}{pe-cimwi hi}{}{ⓔpe-cimwi hi}
\région{GOs PA}
(\domainesémantique{Mouvements ou actions faits avec le corps, les bras, les mains, les pieds
, Réciproque})
\classe{v}
\begin{glose}
\pfra{serrer (se) la main}
\end{glose}
\newline
\begin{exemple}
\région{PA}
\textbf{\pnua{li pe-cimwi hi, li pe-cimwi hi-li}}
\pfra{ils se serrent la main}
\end{exemple}
\end{entrée}

\begin{entrée}{pe-co-ò coo-mi}{}{ⓔpe-co-ò coo-mi}
\formephonétique{pe-coːɔ coːmi}
\région{GOs BO}
\variante{%
pe-cool
\région{PA}}
(\domainesémantique{Verbes de mouvement})
\classe{v}
\begin{glose}
\pfra{sauter de-ci de-là}
\end{glose}
\end{entrée}

\begin{entrée}{peçööli}{}{ⓔpeçööli}
\formephonétique{peʒωːli, pedʒωːli}
\région{GOs}
\variante{%
pejooli
\formephonétique{peɲɟoːli}
\région{PA}}
(\domainesémantique{Relations et interaction sociales})
\classe{v}
\begin{glose}
\pfra{engueuler ; tancer}
\end{glose}
\newline
\begin{sous-entrée}{a-pecöli}{ⓔpeçööliⓝa-pecöli}
\begin{glose}
\pfra{râleur}
\end{glose}
\newline
\begin{exemple}
\région{PA}
\textbf{\pnua{la pe-pejooli}}
\pfra{se disputer}
\end{exemple}
\end{sous-entrée}
\end{entrée}

\begin{entrée}{pe-chinõõ tro mani trèè}{}{ⓔpe-chinõõ tro mani trèè}
\région{GOs}
(\domainesémantique{Découpage du temps})
\classe{nom}
\begin{glose}
\pfra{minuit (lit. le juste milieu entre la nuit et le jour)}
\end{glose}
\end{entrée}

\begin{entrée}{pe-chôã}{}{ⓔpe-chôã}
\région{GOs}
(\domainesémantique{Relations et interaction sociales
, Réciproque})
\classe{v}
\begin{glose}
\pfra{jouer (se) des tours ; taquiner (se)}
\end{glose}
\begin{glose}
\pfra{chercher querelle (se)}
\end{glose}
\end{entrée}

\begin{entrée}{pe-dodobe}{}{ⓔpe-dodobe}
\région{BO}
(\domainesémantique{Fonctions intellectuelles})
\classe{v}
\begin{glose}
\pfra{feindre ; faire semblant [Corne]}
\end{glose}
\newline
\begin{exemple}
\textbf{\pnua{nu pe-dodobe thiidin}}
\pfra{je feins d'être en colère}
\end{exemple}
\end{entrée}

\begin{entrée}{pe-dõńi}{}{ⓔpe-dõńi}
\formephonétique{pe-dɔ̃ni}
\région{GOs BO}
(\domainesémantique{Prépositions})
\classe{PREP}
\begin{glose}
\pfra{parmi ; entre}
\end{glose}
\newline
\begin{exemple}
\région{GO}
\textbf{\pnua{pe-dõni-la}}
\pfra{entre eux}
\end{exemple}
\newline
\begin{exemple}
\région{GO}
\textbf{\pnua{ge je ni pe-dõni X ma Y}}
\pfra{il est entre X et Y}
\end{exemple}
\newline
\begin{exemple}
\région{BO}
\textbf{\pnua{i phe aa-xe na ni dõni la-ã ko}}
\pfra{il a pris un des poulets}
\end{exemple}
\end{entrée}

\begin{entrée}{pedõńõ}{}{ⓔpedõńõ}
\formephonétique{pedɔ̃nɔ̃}
\région{GOs}
\variante{%
penõnõ
\formephonétique{penɔ̃nɔ̃}}
(\domainesémantique{Noms locatifs})
\classe{LOC}
\begin{glose}
\pfra{milieu (au)}
\end{glose}
\newline
\begin{sous-entrée}{penõnõ mwa}{ⓔpedõńõⓝpenõnõ mwa}
\begin{glose}
\pfra{au milieu de la maison}
\end{glose}
\end{sous-entrée}
\newline
\begin{sous-entrée}{penõni ce}{ⓔpedõńõⓝpenõni ce}
\begin{glose}
\pfra{au milieu des arbres}
\end{glose}
\end{sous-entrée}
\end{entrée}

\begin{entrée}{pee-}{}{ⓔpee-}
\région{BO}
(\domainesémantique{Anguilles
, Préfixes classificateurs sémantiques})
\classe{nom}
\begin{glose}
\pfra{préfixe des anguilles}
\end{glose}
\newline
\begin{exemple}
\région{BO}
\textbf{\pnua{pee-nõ}}
\pfra{anguille (dont les yeux ressemblent à ceux d'un poisson)}
\end{exemple}
\newline
\relationsémantique{Cf.}{\lien{ⓔpeeńã}{peeńã}}
\glosecourte{anguille}
\end{entrée}

\begin{entrée}{pèèbu}{}{ⓔpèèbu}
\région{GOs WEM PA BO}
(\domainesémantique{Parenté})
\classe{nom}
\begin{glose}
\pfra{petit-fils ; petite-fille ; descendant}
\end{glose}
\newline
\begin{exemple}
\région{GO}
\textbf{\pnua{pèèbu-nu}}
\pfra{mon petit-fils ; ma petite-fille}
\end{exemple}
\newline
\begin{exemple}
\région{WEM}
\textbf{\pnua{pèèbuu-m}}
\pfra{ton petit-fils / ta petite-fille}
\end{exemple}
\newline
\begin{exemple}
\région{PA}
\textbf{\pnua{peebu-n}}
\pfra{son petit-fils / sa petite-fille}
\end{exemple}
\end{entrée}

\begin{entrée}{peeńã}{}{ⓔpeeńã}
\région{GOs PA BO}
(\domainesémantique{Anguilles})
\classe{nom}
\begin{glose}
\pfra{anguille (de rivière)}
\end{glose}
\end{entrée}

\begin{entrée}{peeńã-nõgò}{}{ⓔpeeńã-nõgò}
\formephonétique{peːnɛ̃ ɳɔ̃ŋgɔ}
\région{GOs}
(\domainesémantique{Anguilles})
\classe{nom}
\begin{glose}
\pfra{anguille de creek}
\end{glose}
\end{entrée}

\begin{entrée}{peeni}{}{ⓔpeeni}
\région{GOs}
(\domainesémantique{Mouvements ou actions faits avec le corps, les bras, les mains, les pieds
, Préparation des aliments; modes de préparation et de cuisson})
\classe{v}
\begin{glose}
\pfra{pétrir (pain)}
\end{glose}
\newline
\begin{exemple}
\textbf{\pnua{e peeni draa phalawa}}
\pfra{elle pétrit la farine}
\end{exemple}
\end{entrée}

\begin{entrée}{peenu}{}{ⓔpeenu}
\région{PA BO}
(\domainesémantique{Types de champs})
\classe{nom}
\begin{glose}
\pfra{tarodière}
\end{glose}
\begin{glose}
\pfra{butte de terre au bord du canal de la tarodière, à terre bien remuée (Charles)}
\end{glose}
\newline
\relationsémantique{Cf.}{\lien{}{bwala ; aru}}
\end{entrée}

\begin{entrée}{peera}{}{ⓔpeera}
\région{BO}
(\domainesémantique{Parties de plantes})
\classe{v}
\begin{glose}
\pfra{bouton (en) ; non éclos [Corne]}
\end{glose}
\newline
\note{non vérifié}{général}{}
\end{entrée}

\begin{entrée}{pegaabe}{}{ⓔpegaabe}
\région{GOs}
(\domainesémantique{Relations et interaction sociales})
\classe{v.stat.}
\begin{glose}
\pfra{humble (se faire) ; petit (se faire) ; abaisser (s') (marque de respect dans les discours coutumiers)}
\end{glose}
\newline
\begin{exemple}
\textbf{\pnua{nu pegaabe i nu bwa xai-wa}}
\pfra{je me fais petit devant vous}
\end{exemple}
\end{entrée}

\begin{entrée}{pe-gaixe}{}{ⓔpe-gaixe}
\formephonétique{pe-'ŋgaiɣe}
\région{GOs PA BO [Corne]}
(\domainesémantique{Comparaison})
\classe{ADV}
\begin{glose}
\pfra{parallèle ; dans le même alignement ; de même niveau ; ex aequo}
\end{glose}
\begin{glose}
\pfra{même alignement ; parallèle ; ex aequo}
\end{glose}
\begin{glose}
\pfra{ajuster (pour arriver sur la même ligne)}
\end{glose}
\newline
\begin{exemple}
\textbf{\pnua{li pe-gaixe}}
\pfra{ils sont sur la même ligne (au même niveau)}
\end{exemple}
\newline
\begin{exemple}
\région{GO}
\textbf{\pnua{li hovwa pe-gaixe}}
\pfra{ils sont arrivés ex-aequo}
\end{exemple}
\newline
\begin{exemple}
\région{GO}
\textbf{\pnua{cö ne vwo li pe-gaixe}}
\pfra{ajuste les pour qu'ils soientsur la même ligne}
\end{exemple}
\newline
\begin{exemple}
\région{GO}
\textbf{\pnua{li uça gaixe}}
\pfra{ils arrivent en même temps}
\end{exemple}
\end{entrée}

\begin{entrée}{pègalò}{}{ⓔpègalò}
\région{GOs}
(\domainesémantique{Noms des plantes})
\classe{nom}
\begin{glose}
\pfra{pomme cythère}
\end{glose}
\end{entrée}

\begin{entrée}{pe-gan}{}{ⓔpe-gan}
\région{BO}
(\domainesémantique{Quantificateurs})
\classe{QNT}
\begin{glose}
\pfra{beaucoup ; nombreux (lit. 3 fois) (Dubois)}
\end{glose}
\newline
\begin{exemple}
\textbf{\pnua{õ-pe-gan}}
\pfra{souvent; plusieurs fois}
\end{exemple}
\end{entrée}

\begin{entrée}{pègòò}{}{ⓔpègòò}
\région{GOs WE}
\variante{%
pègòm
\région{BO}}
(\domainesémantique{Cours de la vie})
\classe{v.stat.}
\begin{glose}
\pfra{dans la force de l'âge (être) ; bonne santé (être en)}
\end{glose}
\end{entrée}

\begin{entrée}{pe-gu-xe}{}{ⓔpe-gu-xe}
\région{PA}
(\domainesémantique{Verbes de déplacement et moyens de déplacement
, Réciproque collectif})
\classe{v}
\begin{glose}
\pfra{marcher en file indienne}
\end{glose}
\newline
\begin{exemple}
\région{PA}
\textbf{\pnua{la pe-gu-xe}}
\pfra{ils marchent en file indienne}
\end{exemple}
\end{entrée}

\begin{entrée}{pe-hatra}{}{ⓔpe-hatra}
\région{GOs}
\variante{%
pe-hara
\région{GO(s) PA WEM}, 
haram
\région{BO}}
(\domainesémantique{Mouvements ou actions faits avec le corps, les bras, les mains, les pieds})
\classe{v}
\begin{glose}
\pfra{jongler (par ex. avec des oranges sauvages, occupation dans le pays des morts)}
\end{glose}
\newline
\note{pe-harame (v.t.)}{grammaire}{}
\end{entrée}

\begin{entrée}{pe-haze}{}{ⓔpe-haze}
\région{GOs}
\variante{%
pe-aze
\région{GO(s)}, 
pe-hale, pe-ale
\région{BO [BM]}, 
ve-ale
\région{BO}}
(\domainesémantique{Dispersifs})
\classe{dispersif ; DISTR}
\begin{glose}
\pfra{séparément ; chacun de son côté}
\end{glose}
\begin{glose}
\pfra{différent l'un de l'autre}
\end{glose}
\newline
\begin{exemple}
\textbf{\pnua{li pe-haze}}
\pfra{ils sont différents l'un de l'autre}
\end{exemple}
\newline
\begin{exemple}
\textbf{\pnua{li a pe-haze ãbaa-nu ma nata}}
\pfra{mon frère et le pasteur sont partis chacun de leur côté}
\end{exemple}
\newline
\begin{exemple}
\région{BO}
\textbf{\pnua{thiu ve-ale}}
\pfra{dispersez-vous !}
\end{exemple}
\newline
\begin{exemple}
\région{BO}
\textbf{\pnua{mènõ ve-ale}}
\pfra{embranchement}
\end{exemple}
\newline
\begin{exemple}
\région{BO}
\textbf{\pnua{ra pe-hale va i la}}
\pfra{leurs paroles sont dispersées}
\end{exemple}
\newline
\begin{exemple}
\région{BO}
\textbf{\pnua{a-ve-hale}}
\pfra{se disperser}
\end{exemple}
\newline
\note{pe-haze-ni (v.t.)}{grammaire}{}
\end{entrée}

\begin{entrée}{pe-hê-xòlò}{}{ⓔpe-hê-xòlò}
\région{GOs PA}
(\domainesémantique{Parenté})
\classe{nom}
\begin{glose}
\pfra{même famille (de la)}
\end{glose}
\newline
\begin{exemple}
\région{GO}
\textbf{\pnua{pe-hê-xòlò-lò}}
\pfra{ils sont de la même famille}
\end{exemple}
\newline
\relationsémantique{Cf.}{\lien{}{hê-kòlò, hê xòlò}}
\glosecourte{de la même famille}
\end{entrée}

\begin{entrée}{pe-hiliçôô}{}{ⓔpe-hiliçôô}
\formephonétique{pe-hiliʒôː ; pe-hilidʒôː}
\région{GOs}
(\domainesémantique{Verbes de mouvement})
\classe{v}
\begin{glose}
\pfra{balancer (se)}
\end{glose}
\newline
\begin{exemple}
\textbf{\pnua{è pe-hiliçôô}}
\pfra{il se balance}
\end{exemple}
\end{entrée}

\begin{entrée}{pe-hine}{}{ⓔpe-hine}
\région{GOs}
(\domainesémantique{Fonctions intellectuelles
, Réciproque})
\classe{v.REC}
\begin{glose}
\pfra{connaître (se)}
\end{glose}
\newline
\begin{exemple}
\région{BO}
\textbf{\pnua{pe-hine-ã}}
\pfra{nous nous connaissons mutuellement}
\end{exemple}
\end{entrée}

\begin{entrée}{pe-hò}{}{ⓔpe-hò}
\région{GOs PA WEM}
(\domainesémantique{Jeux divers})
\classe{v}
\begin{glose}
\pfra{jouer à chat perché}
\end{glose}
\end{entrée}

\begin{entrée}{pe-hòze}{}{ⓔpe-hòze}
\région{GOs}
\variante{%
pe-hòre
\région{BO}}
(\domainesémantique{Verbes de déplacement et moyens de déplacement})
\classe{v}
\begin{glose}
\pfra{suivre ; suivre (se)}
\end{glose}
\newline
\begin{exemple}
\textbf{\pnua{nu pe-(h)òre nyè jaòl}}
\pfra{je suis ce fleuve Diahot}
\end{exemple}
\end{entrée}

\begin{entrée}{pe-îdò}{}{ⓔpe-îdò}
\région{GOs}
(\domainesémantique{Comparaison})
\classe{nom}
\begin{glose}
\pfra{même lignée}
\end{glose}
\newline
\begin{exemple}
\région{GOs}
\textbf{\pnua{pe-îdò-bi}}
\pfra{nous deux sommes de la même lignée}
\end{exemple}
\end{entrée}

\begin{entrée}{pe-iô}{}{ⓔpe-iô}
\région{GOs}
(\domainesémantique{Pronoms})
\classe{COLL + PRO}
\begin{glose}
\pfra{c'est à nous trois}
\end{glose}
\end{entrée}

\begin{entrée}{pe-jaxa}{}{ⓔpe-jaxa}
\région{GOs}
(\domainesémantique{Comparaison})
\classe{nom}
\begin{glose}
\pfra{même taille ; même mesure}
\end{glose}
\newline
\begin{exemple}
\région{GOs}
\textbf{\pnua{pe-jaxa-li}}
\pfra{ils sont de même taille}
\end{exemple}
\end{entrée}

\begin{entrée}{pe-jivwaa}{}{ⓔpe-jivwaa}
\formephonétique{pe-ɲɟiwaː}
\région{GOs}
(\domainesémantique{Comparaison})
\classe{NUM.COMPAR}
\begin{glose}
\pfra{nombre égal (être en)}
\end{glose}
\newline
\begin{exemple}
\région{GO}
\textbf{\pnua{pe-jivwa-li/-lò/-la}}
\pfra{ils sont en nombre égal (les 2/les 3/plur.)}
\end{exemple}
\end{entrée}

\begin{entrée}{pe-jölö}{}{ⓔpe-jölö}
\région{GOs}
(\domainesémantique{Fonctions naturelles humaines})
\classe{v}
\begin{glose}
\pfra{accoupler (s') ; faire l'amour}
\end{glose}
\end{entrée}

\begin{entrée}{pe-ka-}{}{ⓔpe-ka-}
\région{GOs}
(\domainesémantique{Quantificateurs})
\classe{QNT.DISTR}
\begin{glose}
\pfra{plusieurs (à)}
\end{glose}
\newline
\begin{exemple}
\région{GO}
\textbf{\pnua{la pe-ka-a-tru ni no loto}}
\pfra{ils étaient par deux dans chaque voiture}
\end{exemple}
\newline
\begin{exemple}
\région{GO}
\textbf{\pnua{la pe-a-ni ni no loto}}
\pfra{ils étaient à cinq dans la (même) voiture}
\end{exemple}
\end{entrée}

\begin{entrée}{pe-ka-aniza ?}{}{ⓔpe-ka-aniza ?}
\région{GOs}
\variante{%
pe-aniza ?
\région{GO(s)}}
(\domainesémantique{Interrogatifs})
\classe{DISTR}
\begin{glose}
\pfra{à combien dans chaque ?}
\end{glose}
\newline
\begin{exemple}
\région{GOs}
\textbf{\pnua{ka-a-niza na ni ba ?}}
\pfra{combien de personnes y a-t-il dans chaque équipe ?}
\end{exemple}
\newline
\begin{exemple}
\région{GOs}
\textbf{\pnua{mô pe-ka-a-niza na ni ba ?}}
\pfra{nous sommes combien de personnes dans chaque équipe ?}
\end{exemple}
\newline
\begin{exemple}
\région{GOs}
\textbf{\pnua{pe-ka-a-niza we ni no loto ?}}
\pfra{à combien vous (étiez) dans chaque voiture ?}
\end{exemple}
\newline
\begin{exemple}
\région{GOs}
\textbf{\pnua{pe-a-niza we ni no loto ?}}
\pfra{à combien vous (étiez) dans la (même) voiture ?}
\end{exemple}
\end{entrée}

\begin{entrée}{pe-kae}{}{ⓔpe-kae}
\région{GOs}
\variante{%
pe-kaeny
\région{PA}}
(\domainesémantique{Relations et interaction sociales
, Réciproque})
\classe{v}
\begin{glose}
\pfra{disputer (se)(jeu, compétition) ; garder pour soi}
\end{glose}
\newline
\begin{exemple}
\région{GO}
\textbf{\pnua{li pe-kae nye loto}}
\pfra{ils se disputent cette voiture}
\end{exemple}
\newline
\begin{exemple}
\région{GO}
\textbf{\pnua{li pe-kae je thoomwã}}
\pfra{ils se disputent cette femme}
\end{exemple}
\newline
\begin{exemple}
\région{PA}
\textbf{\pnua{i pe-kae balô}}
\pfra{il garde le ballon pour lui (et ne le passe pas)}
\end{exemple}
\newline
\begin{exemple}
\région{PA}
\textbf{\pnua{li pe-kaeny u je loto}}
\pfra{ils se disputent pour cette voiture}
\end{exemple}
\newline
\begin{exemple}
\région{PA}
\textbf{\pnua{kôbwa/kêbwa pe-kaeny !}}
\pfra{il ne faut pas être égoïste (garder pour soi)}
\end{exemple}
\end{entrée}

\begin{entrée}{pe-ka-poniza ?}{}{ⓔpe-ka-poniza ?}
\région{GOs}
(\domainesémantique{Interrogatifs
, Distributifs})
\classe{DISTR}
\begin{glose}
\pfra{à combien dans chaque ?}
\end{glose}
\end{entrée}

\begin{entrée}{pe-ka-poxe}{}{ⓔpe-ka-poxe}
\région{GOs}
(\domainesémantique{Distributifs})
\classe{DISTR}
\begin{glose}
\pfra{un par un (mettre)}
\end{glose}
\newline
\begin{sous-entrée}{pe-ka-po-tru, pe-ka-po-ko}{ⓔpe-ka-poxeⓝpe-ka-po-tru, pe-ka-po-ko}
\begin{glose}
\pfra{deux par deux, trois par trois}
\end{glose}
\newline
\begin{exemple}
\textbf{\pnua{mo pe-ka-poxe ba-tiiwo}}
\pfra{nous avons reçu chacun un stylo}
\end{exemple}
\newline
\begin{exemple}
\textbf{\pnua{li thu-menõ pe-ka-po-tru}}
\pfra{ils marchent deux par deux}
\end{exemple}
\newline
\begin{exemple}
\textbf{\pnua{mo thu-menõ pe-ka-po-ko}}
\pfra{ils marchent trois par trois}
\end{exemple}
\newline
\begin{exemple}
\textbf{\pnua{mo cue ? - pe-ka-aniza ?}}
\pfra{on joue ? - à combien ?}
\end{exemple}
\end{sous-entrée}
\end{entrée}

\begin{entrée}{pe-ki}{}{ⓔpe-ki}
\région{GOs}
(\domainesémantique{Verbes d'action (en général)
, Réciproque})
\classe{v}
\begin{glose}
\pfra{raccorder}
\end{glose}
\newline
\note{pe-kine (v.t.)}{grammaire}{}
\end{entrée}

\begin{entrée}{pe-kiga}{}{ⓔpe-kiga}
\formephonétique{pe-kiŋga}
\région{GOs BO}
(\domainesémantique{Fonctions naturelles humaines})
\classe{v}
\begin{glose}
\pfra{rire (se)}
\end{glose}
\newline
\begin{exemple}
\textbf{\pnua{hoã je, i pe-kiga wãã-na}}
\pfra{quant à lui, il se rit ainsi}
\end{exemple}
\newline
\begin{exemple}
\textbf{\pnua{çö pe-kiga hãda}}
\pfra{tu ris tout seul}
\end{exemple}
\end{entrée}

\begin{entrée}{pe-kine}{}{ⓔpe-kine}
\formephonétique{pe-kine}
\région{GOs}
(\domainesémantique{Verbes d'action (en général)
, Réciproque collectif})
\classe{v}
\begin{glose}
\pfra{mettre bout à bout (et allonger)}
\end{glose}
\newline
\begin{exemple}
\région{GOs}
\textbf{\pnua{e pe-kine ôxa-ci}}
\pfra{il a mis les planches bout à bout}
\end{exemple}
\end{entrée}

\begin{entrée}{pe-kô-alö}{}{ⓔpe-kô-alö}
\région{GOs}
(\domainesémantique{Préfixes et verbes de position
, Réciproque collectif})
\classe{v}
\begin{glose}
\pfra{couché (être) face à face}
\end{glose}
\newline
\begin{exemple}
\textbf{\pnua{lò pe-kô-alö-i-lò}}
\pfra{ils(3) sont couchés face à face}
\end{exemple}
\end{entrée}

\begin{entrée}{pe-koone}{}{ⓔpe-koone}
\région{BO}
(\domainesémantique{Navigation})
\classe{v}
\begin{glose}
\pfra{louvoyer ; tirer des bords vent debout}
\end{glose}
\newline
\note{non vérifié}{général}{}
\end{entrée}

\begin{entrée}{pe-kû}{}{ⓔpe-kû}
\région{BO}
(\domainesémantique{Aspect})
\classe{v}
\begin{glose}
\pfra{en train de manger}
\end{glose}
\newline
\begin{exemple}
\textbf{\pnua{nu ga pe-kû pò-puleng}}
\pfra{je suis en train de manger ce fruit}
\end{exemple}
\end{entrée}

\begin{entrée}{pe-ku-alö}{}{ⓔpe-ku-alö}
\région{GOs}
(\domainesémantique{Préfixes et verbes de position
, Réciproque})
\classe{v}
\begin{glose}
\pfra{debout (être) face à face}
\end{glose}
\newline
\begin{exemple}
\région{GOs}
\textbf{\pnua{lò pe-ku-alö-i-lò}}
\pfra{ils(3) sont debout face à face}
\end{exemple}
\newline
\begin{exemple}
\région{GOs}
\textbf{\pnua{lò pe-ku-alö}}
\pfra{ils(3) sont debout face à qqch d'autre}
\end{exemple}
\end{entrée}

\begin{entrée}{pe-kubu}{}{ⓔpe-kubu}
\formephonétique{pe-kubu}
\région{GOs PA}
(\domainesémantique{Guerre
, Réciproque})
\classe{v}
\begin{glose}
\pfra{battre (se) (avec armes)}
\end{glose}
\newline
\begin{exemple}
\région{GO}
\textbf{\pnua{li pe-kubu-li}}
\pfra{ils se battent}
\end{exemple}
\end{entrée}

\begin{entrée}{pe-ku-çaaxo}{}{ⓔpe-ku-çaaxo}
\formephonétique{pe-ku-ʒaːɣo}
\région{GOs}
(\domainesémantique{Jeux divers})
\classe{v}
\begin{glose}
\pfra{jouer à cache-cache}
\end{glose}
\end{entrée}

\begin{entrée}{pe-kueli}{}{ⓔpe-kueli}
\région{GOs}
(\domainesémantique{Sentiments
, Réciproque})
\classe{v}
\begin{glose}
\pfra{détester (se) ; rejeter (se)}
\end{glose}
\newline
\begin{exemple}
\textbf{\pnua{li pe-kueli-li ui nye thoomwã xo lie meewu xa êmwe}}
\pfra{les deux frères se rejettent à cause de cette femme (Doriane)}
\end{exemple}
\newline
\begin{exemple}
\textbf{\pnua{la pe-kueli-la pexa dili / pune dili xo la êgu}}
\pfra{les gens se rejettent à cause de la terre (Doriane)}
\end{exemple}
\newline
\begin{exemple}
\textbf{\pnua{li pe-kueli-li lie meewu}}
\pfra{les deux frères se rejettent (Mario)}
\end{exemple}
\newline
\begin{exemple}
\textbf{\pnua{liã meewu ça li pe-kueli-li}}
\pfra{les deux frères se rejettent (Mario)}
\end{exemple}
\newline
\begin{exemple}
\textbf{\pnua{lie meewu xa êmwê, li pe-kueli-li pexa nye thoomwã}}
\pfra{les deux frères, ils se rejettent à cause de cette femme}
\end{exemple}
\newline
\begin{exemple}
\textbf{\pnua{li pe-kueli lie meewu xa êmwê ui nye thoomwã}}
\pfra{les deux frères se rejettent à cause de cette femme}
\end{exemple}
\newline
\begin{exemple}
\textbf{\pnua{li pe-kueli-li lie meewu xa êmwê}}
\pfra{les deux frères se rejettent}
\end{exemple}
\end{entrée}

\begin{entrée}{pe-khabeçö}{}{ⓔpe-khabeçö}
\région{GOs}
(\domainesémantique{Actions liées aux éléments (liquide, fumée)})
\classe{v}
\begin{glose}
\pfra{plonger (pour s'amuser, se dit d'enfants)}
\end{glose}
\end{entrée}

\begin{entrée}{pe-khînu}{}{ⓔpe-khînu}
\région{PA}
(\domainesémantique{Sentiments})
\classe{v}
\begin{glose}
\pfra{triste ; malheureux}
\end{glose}
\newline
\begin{exemple}
\région{PA}
\textbf{\pnua{pe-khînu ai-ny}}
\pfra{je suis triste, malheureux}
\end{exemple}
\end{entrée}

\begin{entrée}{pe-khõbwe}{}{ⓔpe-khõbwe}
\région{GOs}
(\domainesémantique{Discours, échanges verbaux
, Réciproque collectif})
\classe{v}
\begin{glose}
\pfra{discuter ; mettre d'accord (se)}
\end{glose}
\newline
\begin{exemple}
\textbf{\pnua{bi pe-khõbwe po bi pe-tòò-bi mõnõ}}
\pfra{on a décidé de se retrouver demain}
\end{exemple}
\end{entrée}

\begin{entrée}{pe-menixe}{}{ⓔpe-menixe}
\région{GOs PA BO}
(\domainesémantique{Comparaison})
\classe{v.COMPAR}
\begin{glose}
\pfra{pareil (être) ; semblable}
\end{glose}
\newline
\relationsémantique{Cf.}{\lien{ⓔpe-poxe}{pe-poxe}}
\glosecourte{pareil}
\end{entrée}

\begin{entrée}{pe-mhe}{}{ⓔpe-mhe}
\région{GOs BO}
(\domainesémantique{Verbes de déplacement et moyens de déplacement
, Réciproque collectif})
\classe{v}
\begin{glose}
\pfra{aller ensemble ; accompagner}
\end{glose}
\newline
\begin{exemple}
\région{GO}
\textbf{\pnua{pe-mhe-ã}}
\pfra{nous faisons route ensemble}
\end{exemple}
\newline
\begin{exemple}
\région{BO}
\textbf{\pnua{pe-mhe-la}}
\pfra{ils font route ensemble}
\end{exemple}
\newline
\begin{exemple}
\région{GO}
\textbf{\pnua{ai-nu wo pe-mhe-î}}
\pfra{je veux t'accompagner / que nous 2 y allions ensemble}
\end{exemple}
\newline
\begin{exemple}
\région{GO}
\textbf{\pnua{ai-nu wo pe-mhe-õ}}
\pfra{je veux vous accompagner vous2 / que nous 3 y allions ensemble}
\end{exemple}
\newline
\begin{exemple}
\région{BO}
\textbf{\pnua{ciibwin je ai-n (p)u pe-mhe-la}}
\pfra{le rat veut les accompagner (BM)}
\end{exemple}
\newline
\begin{exemple}
\région{BO}
\textbf{\pnua{ai-ny (w)u ra pe-mhe-ã}}
\pfra{je veux vraiment vous accompagner (BM)}
\end{exemple}
\newline
\begin{exemple}
\région{BO}
\textbf{\pnua{ai-nu (w)u pe-mhe-õ}}
\pfra{je veux vraiment que nous3 y allions ensemble = je veux vous accompagner (vous 2)}
\end{exemple}
\newline
\begin{exemple}
\région{BO}
\textbf{\pnua{a-mi ho pe-mhe-î}}
\pfra{viens avec moi (BM)}
\end{exemple}
\newline
\begin{exemple}
\région{BO}
\textbf{\pnua{nu a kaè-n ho pe-mhe-yò}}
\pfra{va avec lui (BM)}
\end{exemple}
\newline
\begin{sous-entrée}{pe-mhe a}{ⓔpe-mheⓝpe-mhe a}
\région{BO}
\begin{glose}
\pfra{accompagner}
\end{glose}
\newline
\relationsémantique{Cf.}{\lien{}{mhe, mhenõ}}
\glosecourte{déplacement}
\end{sous-entrée}
\end{entrée}

\begin{entrée}{pe-na}{}{ⓔpe-na}
\région{GOs}
(\domainesémantique{Relations et interaction sociales
, Réciproque collectif})
\classe{v}
\begin{glose}
\pfra{envoyer (s') mutuellement}
\end{glose}
\newline
\begin{exemple}
\textbf{\pnua{li vara pe-na a-pe-vhaa i li}}
\pfra{ils s'envoient mutuellement un messager}
\end{exemple}
\end{entrée}

\begin{entrée}{pe-na-bulu-ni}{}{ⓔpe-na-bulu-ni}
\formephonétique{pe-ɳa-bu'lu-ɳi}
\région{GOs}
(\domainesémantique{Verbes de mouvement
, Réciproque collectif})
\classe{v.t.}
\begin{glose}
\pfra{rassembler ; assembler}
\end{glose}
\end{entrée}

\begin{entrée}{pe-na bwa}{}{ⓔpe-na bwa}
\région{GOs}
(\domainesémantique{Mouvements ou actions faits avec le corps, les bras, les mains, les pieds
, Réciproque collectif})
\classe{v}
\begin{glose}
\pfra{empiler}
\end{glose}
\end{entrée}

\begin{entrée}{pe-na-bwa na kui}{}{ⓔpe-na-bwa na kui}
\région{GOs}
(\domainesémantique{Coutumes, dons coutumiers})
\classe{nom}
\begin{glose}
\pfra{empilement d'ignames}
\end{glose}
\end{entrée}

\begin{entrée}{pe-na-hi-n}{}{ⓔpe-na-hi-n}
\région{PA}
(\domainesémantique{Caractéristiques et propriétés des personnes})
\classe{v}
\begin{glose}
\pfra{planer (métaphoriquement réfère à qqn qui paresse)}
\end{glose}
\newline
\begin{exemple}
\textbf{\pnua{novwu ã i bwayu, e yu mwã mòn;}}
\pfra{quant à celui qui est travailleur, il reste là ;}
\end{exemple}
\newline
\begin{exemple}
\textbf{\pnua{novwu ã i baro, e pe-nahi-n bwa dèèn}}
\pfra{quant à celui qui est paresseux, il plane dans le vent}
\end{exemple}
\end{entrée}

\begin{entrée}{pe-navwo}{}{ⓔpe-navwo}
\région{GOs PA}
(\domainesémantique{Coutumes, dons coutumiers
, Réciproque collectif})
\classe{v}
\begin{glose}
\pfra{faire une cérémonie coutumière ; faire les échanges coutumiers}
\end{glose}
\end{entrée}

\begin{entrée}{pe-ne-zo-ni}{}{ⓔpe-ne-zo-ni}
\formephonétique{pe-ɳe-ðo-ɳi}
\région{GOs}
(\domainesémantique{Relations et interaction sociales
, Réciproque collectif})
\classe{v}
\begin{glose}
\pfra{pardonner (se)}
\end{glose}
\newline
\begin{exemple}
\textbf{\pnua{mi pe-ne-zo-ni}}
\pfra{nous nous sommes pardonnés}
\end{exemple}
\newline
\begin{exemple}
\textbf{\pnua{la pe-ne-zo-ni vha-raa xo la ãgu}}
\pfra{les gens se sont pardonnés les insultes}
\end{exemple}
\newline
\begin{exemple}
\région{GO}
\textbf{\pnua{mi pe-ne-zo-ni}}
\pfra{nous nous sommes pardonnés}
\end{exemple}
\newline
\relationsémantique{Cf.}{\lien{}{pe-tha thria [GOs]}}
\glosecourte{faire une coutume de pardon}
\end{entrée}

\begin{entrée}{pe-nò-du}{}{ⓔpe-nò-du}
\région{GOs BO}
(\domainesémantique{Fonctions naturelles humaines})
\classe{v}
\begin{glose}
\pfra{regarder en bas (sans but)}
\end{glose}
\newline
\begin{exemple}
\textbf{\pnua{e kawö pe-nò-du mwa bwabu}}
\pfra{elle ne regarde pas par terre}
\end{exemple}
\end{entrée}

\begin{entrée}{penõõ}{}{ⓔpenõõ}
\région{PA}
(\domainesémantique{Aspect})
\classe{ASP}
\begin{glose}
\pfra{pas encore}
\end{glose}
\newline
\begin{exemple}
\région{PA}
\textbf{\pnua{kavwo nu penõõ ku nõõ-je}}
\pfra{je ne l'ai jamais vu}
\end{exemple}
\newline
\begin{exemple}
\région{GO}
\textbf{\pnua{kavwo nu ku nõõ-je go}}
\pfra{je ne l'ai jamais vu}
\end{exemple}
\newline
\begin{exemple}
\région{PA}
\textbf{\pnua{kavwo i penõõ ku cabol}}
\pfra{il ne dort pas encore}
\end{exemple}
\newline
\begin{exemple}
\région{GO}
\textbf{\pnua{kavwo e ku cabo go}}
\pfra{il ne dort pas encore}
\end{exemple}
\newline
\begin{exemple}
\région{PA}
\textbf{\pnua{kavwo nu penõõ ku hovwo}}
\pfra{jen 'ai pas encore mangé}
\end{exemple}
\newline
\begin{exemple}
\région{GO}
\textbf{\pnua{kavwo nu ku hovwo go}}
\pfra{jen 'ai pas encore mangé}
\end{exemple}
\end{entrée}

\begin{entrée}{penuu}{1}{ⓔpenuuⓗ1}
\région{GOs}
(\domainesémantique{Organisation sociale})
\classe{nom}
\begin{glose}
\pfra{liens de famille}
\end{glose}
\newline
\begin{exemple}
\région{GOs}
\textbf{\pnua{mõ penuu}}
\pfra{on a des liens de famille}
\end{exemple}
\end{entrée}

\begin{entrée}{pe-nhoi thria}{}{ⓔpe-nhoi thria}
\région{GOs}
(\domainesémantique{Relations et interaction sociales
, Réciproque collectif})
\classe{v}
\begin{glose}
\pfra{pardonner (se) (lit. attacher le pardon)}
\end{glose}
\newline
\begin{exemple}
\textbf{\pnua{mo pe-nhoi thria}}
\pfra{nous avons fait la coutume de pardon}
\end{exemple}
\end{entrée}

\begin{entrée}{pe-paree}{}{ⓔpe-paree}
\région{PA BO [BM]}
(\domainesémantique{Fonctions intellectuelles})
\classe{v}
\begin{glose}
\pfra{distrait (être) ; ne pas faire attention}
\end{glose}
\newline
\begin{exemple}
\textbf{\pnua{la paree nai je wha-ma}}
\pfra{ils ont délaissé, ignoré le vieux, ils ne s'en sont pas occupés}
\end{exemple}
\end{entrée}

\begin{entrée}{pe-piina}{}{ⓔpe-piina}
\région{GOs BO}
(\domainesémantique{Aspect})
\classe{ASP}
\begin{glose}
\pfra{en train de se promener}
\end{glose}
\newline
\begin{exemple}
\textbf{\pnua{nu pe-piina, pe-hòre nyè jaaòl}}
\pfra{je me promène, je suis le fleuve comme cela}
\end{exemple}
\end{entrée}

\begin{entrée}{pe-po}{}{ⓔpe-po}
\région{BO}
(\domainesémantique{Quantificateurs})
\classe{QNT}
\begin{glose}
\pfra{toutes les choses}
\end{glose}
\newline
\begin{exemple}
\textbf{\pnua{la pepe-po}}
\pfra{toutes les choses (Dubois ms)}
\end{exemple}
\newline
\note{non vérifié}{général}{}
\end{entrée}

\begin{entrée}{pe-po-niza ?}{}{ⓔpe-po-niza ?}
\région{GOs}
(\domainesémantique{Interrogatifs})
\classe{INT}
\begin{glose}
\pfra{combien en tout (avons nous en commun)?}
\end{glose}
\newline
\begin{exemple}
\textbf{\pnua{mo pe-ka poniza ?}}
\pfra{combien avons-nous chacun ? (chaque groupe ou équipe dans un jeu)}
\end{exemple}
\newline
\begin{sous-entrée}{pe-po-ru}{ⓔpe-po-niza ?ⓝpe-po-ru}
\begin{glose}
\pfra{deux en tout}
\end{glose}
\end{sous-entrée}
\newline
\begin{sous-entrée}{pe-po-kone}{ⓔpe-po-niza ?ⓝpe-po-kone}
\begin{glose}
\pfra{trois en tout}
\end{glose}
\end{sous-entrée}
\newline
\begin{sous-entrée}{pe-po-nima-du}{ⓔpe-po-niza ?ⓝpe-po-nima-du}
\begin{glose}
\pfra{sept en tout}
\end{glose}
\end{sous-entrée}
\end{entrée}

\begin{entrée}{pe-poxe}{}{ⓔpe-poxe}
\région{GOs}
(\domainesémantique{Comparaison})
\classe{v}
\begin{glose}
\pfra{semblable ; pareil ; autant}
\end{glose}
\newline
\begin{exemple}
\textbf{\pnua{pe-poxee-li}}
\pfra{ils 2 sont semblables}
\end{exemple}
\newline
\begin{exemple}
\textbf{\pnua{pe-poxee-la}}
\pfra{ils sont semblables, autant}
\end{exemple}
\newline
\begin{exemple}
\textbf{\pnua{la pe-poxee}}
\pfra{ils sont semblables, se ressemblent}
\end{exemple}
\newline
\begin{exemple}
\textbf{\pnua{pe-poxee pwaxi-li}}
\pfra{ils sont de même taille (hauteur)}
\end{exemple}
\newline
\begin{exemple}
\textbf{\pnua{pe-poxee chinõ-li}}
\pfra{ils sont de même taille (largeur)}
\end{exemple}
\newline
\begin{exemple}
\textbf{\pnua{pe-poxee ala-mee-li}}
\pfra{ils se ressemblent}
\end{exemple}
\newline
\relationsémantique{Cf.}{\lien{}{pe-poxee-lò}}
\newline
\relationsémantique{Cf.}{\lien{ⓔpe-menixe}{pe-menixe}}
\glosecourte{pareil}
\end{entrée}

\begin{entrée}{pe-phããde kui}{}{ⓔpe-phããde kui}
\région{GOs}
(\domainesémantique{Coutumes, dons coutumiers})
\classe{nom}
\begin{glose}
\pfra{cérémonie de la nouvelle igname (lit. se présenter les ignames)}
\end{glose}
\end{entrée}

\begin{entrée}{pe-pha-nõnõmi}{}{ⓔpe-pha-nõnõmi}
\formephonétique{pe-pʰa-ɳɔ̃ɳɔ̃mi}
\région{GOs}
(\domainesémantique{Relations et interaction sociales
, Réciproque collectif})
\classe{v}
\begin{glose}
\pfra{rappeller (se) mutuellement}
\end{glose}
\newline
\begin{exemple}
\région{GOs}
\textbf{\pnua{li pe-pha-nonomi xo Kaavwo ma Hiixe la mwêêje êgõgò}}
\pfra{Kaavwo et Hiixe se rappellent mutuellement les coutumes d'antan}
\end{exemple}
\newline
\begin{exemple}
\région{GOs}
\textbf{\pnua{li pe-pha-nonomi çai li la mwêêje êgõgò}}
\pfra{elles se rappellent mutuellement les coutumes d'antan}
\end{exemple}
\newline
\begin{exemple}
\région{GOs}
\textbf{\pnua{li pha-nonom la mwêêje êgõgò xo Kaavwo ma Hiixe}}
\pfra{Kaavwo et Hiixe rappellent (à d'autres) les coutumes d'antan}
\end{exemple}
\end{entrée}

\begin{entrée}{pe-phao}{}{ⓔpe-phao}
\région{PA}
(\domainesémantique{Verbes de déplacement et moyens de déplacement})
\classe{v}
\begin{glose}
\pfra{prendre la route}
\end{glose}
\newline
\begin{exemple}
\région{PA}
\textbf{\pnua{pe-phao-bin ni deen}}
\pfra{nous prenons la route}
\end{exemple}
\end{entrée}

\begin{entrée}{pe-phaza-hããxa}{}{ⓔpe-phaza-hããxa}
\région{GOs}
(\domainesémantique{Relations et interaction sociales
, Réciproque collectif})
\classe{v}
\begin{glose}
\pfra{faire (se) peur mutuellement}
\end{glose}
\newline
\begin{exemple}
\région{GOs}
\textbf{\pnua{hã pe-phaza-hããxa}}
\pfra{on joue à se faire peur mutuellement}
\end{exemple}
\end{entrée}

\begin{entrée}{pe-phoo}{}{ⓔpe-phoo}
\région{GOs}
(\domainesémantique{Préfixes et verbes de position})
\classe{v}
\begin{glose}
\pfra{couché sur le ventre (étape pour un bébé)}
\end{glose}
\end{entrée}

\begin{entrée}{pe-phumõ}{}{ⓔpe-phumõ}
\formephonétique{pe-'pʰumɔ̃}
\région{GOs WEM BO PA}
(\domainesémantique{Discours, échanges verbaux})
\classe{v ; n}
\begin{glose}
\pfra{discourir ; prêcher ; enseigner ; discours}
\end{glose}
\end{entrée}

\begin{entrée}{pe-pwaixe}{}{ⓔpe-pwaixe}
\région{GOs}
(\domainesémantique{Quantificateurs
, Réciproque collectif})
\classe{QNT}
\begin{glose}
\pfra{toutes les choses possédées ensemble}
\end{glose}
\newline
\begin{exemple}
\textbf{\pnua{pe-pwaixe-î}}
\pfra{toutes les choses que nous possédons ensemble, qui nous sont communes}
\end{exemple}
\end{entrée}

\begin{entrée}{pe-pwali}{}{ⓔpe-pwali}
\région{GOs}
\variante{%
pe-pwawali
\région{PA}}
(\domainesémantique{Comparaison})
\classe{COMPAR}
\begin{glose}
\pfra{même hauteur}
\end{glose}
\newline
\begin{exemple}
\région{GOs}
\textbf{\pnua{pe-pwali-li}}
\pfra{ils ont la même taille (hauteur)}
\end{exemple}
\newline
\relationsémantique{Cf.}{\lien{}{pwali, pwawali}}
\end{entrée}

\begin{entrée}{pe-pwawè}{}{ⓔpe-pwawè}
\région{GOs}
\variante{%
pe-pwawèn
\région{PA}}
(\domainesémantique{Relations et interaction sociales})
\classe{v}
\begin{glose}
\pfra{imiter}
\end{glose}
\newline
\begin{exemple}
\textbf{\pnua{e pe-pwawè kêê-je}}
\pfra{il imite son père}
\end{exemple}
\end{entrée}

\begin{entrée}{pe-pwò-tru meewu}{}{ⓔpe-pwò-tru meewu}
\région{GOs}
(\domainesémantique{Numéraux cardinaux})
\classe{NUM}
\begin{glose}
\pfra{il y a deux variétés}
\end{glose}
\end{entrée}

\begin{entrée}{pe-phweexu}{}{ⓔpe-phweexu}
\région{GOs PA BO}
(\domainesémantique{Discours, échanges verbaux
, Réciproque collectif})
\classe{v}
\begin{glose}
\pfra{discuter}
\end{glose}
\newline
\relationsémantique{Cf.}{\lien{ⓔphweexu}{phweexu}}
\glosecourte{discuter}
\end{entrée}

\begin{entrée}{pe-ravhi}{}{ⓔpe-ravhi}
\région{BO PA}
(\domainesémantique{Soins du corps})
\classe{v}
\begin{glose}
\pfra{raser (barbe)}
\end{glose}
\newline
\begin{exemple}
\textbf{\pnua{i pe-ravhi}}
\pfra{il se rase}
\end{exemple}
\newline
\relationsémantique{Cf.}{\lien{ⓔthraⓗ1}{thra}}
\glosecourte{[GOs]}
\newline
\relationsémantique{Cf.}{\lien{ⓔthaⓗ1}{tha}}
\glosecourte{[BO PA]}
\end{entrée}

\begin{entrée}{pe-ravwi}{}{ⓔpe-ravwi}
\région{GOs}
(\domainesémantique{Verbes de mouvement})
\classe{v}
\begin{glose}
\pfra{faire la course (de vitesse) ; poursuivre (se)}
\end{glose}
\newline
\relationsémantique{Cf.}{\lien{ⓔthravwi}{thravwi}}
\glosecourte{poursuivre (à la chasse); chasser; écarter (chiens, volailles)}
\end{entrée}

\begin{entrée}{pe-rulai}{}{ⓔpe-rulai}
\région{PA BO [Corne]}
(\domainesémantique{Relations et interaction sociales})
\classe{v}
\begin{glose}
\pfra{moquer (se)}
\end{glose}
\newline
\begin{exemple}
\textbf{\pnua{a-perulai}}
\pfra{moqueur}
\end{exemple}
\newline
\relationsémantique{Cf.}{\lien{}{uza [GOs], ula [BO]}}
\glosecourte{moquer (se)}
\end{entrée}

\begin{entrée}{pe-tizi mabu}{}{ⓔpe-tizi mabu}
\région{GOs}
(\domainesémantique{Jeux divers})
\classe{LOCUT}
\begin{glose}
\pfra{jeu consistant à faire rire l'autre (le 1er qui rit a perdu)}
\end{glose}
\end{entrée}

\begin{entrée}{pe-tò do}{}{ⓔpe-tò do}
\région{PA BO}
(\domainesémantique{Relations et interaction sociales
, Réciproque collectif})
\classe{v}
\begin{glose}
\pfra{pardonner (se) mutuellement}
\end{glose}
\newline
\relationsémantique{Cf.}{\lien{}{tòè do}}
\glosecourte{lancer la sagaie}
\end{entrée}

\begin{entrée}{pe-tòe}{}{ⓔpe-tòe}
\région{BO}
(\domainesémantique{Aspect})
\classe{ASP.INACC}
\begin{glose}
\pfra{en train de planter}
\end{glose}
\newline
\begin{exemple}
\région{BO}
\textbf{\pnua{nu ga pe-tòe êê-ny ã èm}}
\pfra{je suis en train de planter mes plants de canne à sucre}
\end{exemple}
\end{entrée}

\begin{entrée}{pe-tou}{}{ⓔpe-tou}
\région{GOs BO}
(\domainesémantique{Coutumes, dons coutumiers})
\classe{v}
\begin{glose}
\pfra{partager (partage coutumier des ignames)}
\end{glose}
\end{entrée}

\begin{entrée}{pe-tha}{}{ⓔpe-tha}
\région{GOs WEM}
(\domainesémantique{Jeux divers})
\classe{nom}
\begin{glose}
\pfra{jeu de lancer de sagaie (lancer-ricochet)}
\end{glose}
\end{entrée}

\begin{entrée}{pe-thahî}{}{ⓔpe-thahî}
\région{GOs}
\variante{%
pe-tha-hînõ
\région{GO(s)}, 
pe-thahîn
\région{PA BO}}
(\domainesémantique{Jeux divers})
\classe{v ; n}
\begin{glose}
\pfra{jouer aux devinette ; deviner ; concours de devinettes}
\end{glose}
\newline
\begin{exemple}
\région{PA}
\textbf{\pnua{la pe-thahîn}}
\pfra{ils jouent aux devinettes, ils se posent des devinettes}
\end{exemple}
\newline
\relationsémantique{Cf.}{\lien{}{thahîn [PA, BO]}}
\glosecourte{devinettes}
\end{entrée}

\begin{entrée}{pe-thaivwi}{}{ⓔpe-thaivwi}
\formephonétique{pe-tʰaiβi}
\région{GOs PA}
(\domainesémantique{Relations et interaction sociales
, Réciproque collectif})
\classe{v}
\begin{glose}
\pfra{réunir (se)}
\end{glose}
\end{entrée}

\begin{entrée}{pe-tha thria}{}{ⓔpe-tha thria}
\région{GOs}
\variante{%
pe-tha thia
\région{PA}}
(\domainesémantique{Coutumes, dons coutumiers
, Relations et interaction sociales
, Réciproque collectif})
\classe{v}
\begin{glose}
\pfra{pardonner (se) (pardon coutumier)}
\end{glose}
\newline
\begin{exemple}
\région{GO}
\textbf{\pnua{li pe-tha-thria li-e ãbaa-nu êmwê}}
\pfra{mes deux frères se sont pardonnés}
\end{exemple}
\newline
\relationsémantique{Cf.}{\lien{ⓔpe-ne-zo-ni}{pe-ne-zo-ni}}
\glosecourte{pardonner}
\end{entrée}

\begin{entrée}{pe-thauvweni}{}{ⓔpe-thauvweni}
\formephonétique{pe-tʰauβeni}
\région{GOs}
(\domainesémantique{Fonctions intellectuelles
, Relations et interaction sociales})
\classe{v}
\begin{glose}
\pfra{tromper (se) en prenant qqch}
\end{glose}
\begin{glose}
\pfra{prendre qqch. par erreur}
\end{glose}
\end{entrée}

\begin{entrée}{pe-thi}{}{ⓔpe-thi}
\région{GOs}
(\domainesémantique{Feu : objets et actions liés au feu})
\classe{v}
\begin{glose}
\pfra{incendier; mettre le feu}
\end{glose}
\newline
\begin{sous-entrée}{egu xa a-pe-thi}{ⓔpe-thiⓝegu xa a-pe-thi}
\begin{glose}
\pfra{un pyromane}
\end{glose}
\newline
\begin{exemple}
\textbf{\pnua{e pe-thi}}
\pfra{il a mis le feu}
\end{exemple}
\end{sous-entrée}
\end{entrée}

\begin{entrée}{pe-thilò}{}{ⓔpe-thilò}
\région{GOs}
\variante{%
pe-dilo
\région{BO}}
\classe{v}
\newline
\sens{1}
(\domainesémantique{Réciproque collectif})
\begin{glose}
\pfra{paire ; faire équipe ; être en binôme avec}
\end{glose}
\newline
\begin{exemple}
\région{GO}
\textbf{\pnua{pe-thilò-bi ma Mario}}
\pfra{je fais équipe avec M.}
\end{exemple}
\newline
\sens{2}
(\domainesémantique{Relations et interaction sociales})
\begin{glose}
\pfra{pareil (être) ; égal (Dubois)}
\end{glose}
\newline
\begin{sous-entrée}{pe-dilo-li}{ⓔpe-thilòⓢ2ⓝpe-dilo-li}
\région{BO PA}
\begin{glose}
\pfra{les deux (frère et soeur)}
\end{glose}
\end{sous-entrée}
\newline
\begin{sous-entrée}{pe-dilo-la}{ⓔpe-thilòⓢ2ⓝpe-dilo-la}
\région{BO PA}
\begin{glose}
\pfra{ils sont pareils}
\end{glose}
\end{sous-entrée}
\end{entrée}

\begin{entrée}{pe-thirãgo}{}{ⓔpe-thirãgo}
\région{GOs}
(\domainesémantique{Préfixes et verbes de position
, Réciproque collectif})
\classe{v}
\begin{glose}
\pfra{superposer (se)}
\end{glose}
\newline
\begin{exemple}
\textbf{\pnua{ne vwo la pe-thirãgo}}
\pfra{fais en sorte qu'ils se superposent}
\end{exemple}
\end{entrée}

\begin{entrée}{pe-thi thô}{}{ⓔpe-thi thô}
\région{GOs}
(\domainesémantique{Relations et interaction sociales
, Réciproque collectif})
\classe{v}
\begin{glose}
\pfra{provoquer (se)}
\end{glose}
\newline
\begin{exemple}
\région{GOs}
\textbf{\pnua{la pe-thi thô nai la}}
\pfra{ils se provoquent (piquent la colère)}
\end{exemple}
\end{entrée}

\begin{entrée}{pe-thu-ba}{}{ⓔpe-thu-ba}
\région{GOs}
(\domainesémantique{Société
, Réciproque collectif})
\classe{v}
\begin{glose}
\pfra{faire équipe}
\end{glose}
\newline
\begin{exemple}
\textbf{\pnua{mo pe-thu-ba}}
\pfra{nous faisons équipe}
\end{exemple}
\newline
\relationsémantique{Cf.}{\lien{}{bala-nu}}
\glosecourte{mon équipier}
\end{entrée}

\begin{entrée}{pe-thu-menõ}{}{ⓔpe-thu-menõ}
\région{GOs}
\variante{%
pe-tumenõ
\région{GO(s)}}
(\domainesémantique{Verbes de déplacement et moyens de déplacement})
\classe{v}
\begin{glose}
\pfra{déplacer (se) ; promener (se)}
\end{glose}
\newline
\begin{exemple}
\textbf{\pnua{e pe-thumenõ}}
\pfra{il se promène}
\end{exemple}
\newline
\begin{exemple}
\textbf{\pnua{nu pe-thumenõ-du kolo-je}}
\pfra{je vais vers chez lui (sans but)}
\end{exemple}
\newline
\begin{exemple}
\région{GOs}
\textbf{\pnua{bi pe-thumenõ mãni ãbaa-nu xa thõõmwa}}
\pfra{j'ai fait le chemin avec ma soeur}
\end{exemple}
\newline
\begin{exemple}
\région{GOs}
\textbf{\pnua{li pe-thumenõ (bulu) xo Kaavo ma Hiixe}}
\pfra{Kaavo et Hiixe ont fait le chemin ensemble}
\end{exemple}
\newline
\relationsémantique{Cf.}{\lien{}{e pe-piina}}
\glosecourte{il se promène}
\end{entrée}

\begin{entrée}{pe-thu-paa}{}{ⓔpe-thu-paa}
\région{GOs}
(\domainesémantique{Guerre
, Réciproque collectif})
\classe{v}
\begin{glose}
\pfra{faire la guerre (se)}
\end{glose}
\newline
\begin{exemple}
\textbf{\pnua{li pe-thu-paa (xo) lie pweemwa}}
\pfra{les deux clans se font la guerre}
\end{exemple}
\end{entrée}

\begin{entrée}{pe-tre-alö}{}{ⓔpe-tre-alö}
\région{GOs}
(\domainesémantique{Préfixes et verbes de position
, Réciproque collectif})
\classe{v}
\begin{glose}
\pfra{assis (être) face à face}
\end{glose}
\newline
\begin{exemple}
\textbf{\pnua{hî pe-tre-alö-i-î}}
\pfra{nous sommes assis face à face}
\end{exemple}
\newline
\begin{exemple}
\textbf{\pnua{me pe-tre-alö-i-me}}
\pfra{nous sommes assis face à face}
\end{exemple}
\end{entrée}

\begin{entrée}{pe-trò}{}{ⓔpe-trò}
\région{GOs}
(\domainesémantique{Relations et interaction sociales
, Réciproque collectif})
\classe{v}
\begin{glose}
\pfra{rencontrer (se)}
\end{glose}
\newline
\begin{exemple}
\textbf{\pnua{bi pe-trò-bi ma Jan dròrò}}
\pfra{Jean et moi nous sommes rencontréshier}
\end{exemple}
\newline
\begin{exemple}
\région{PA}
\textbf{\pnua{kô-zo na la pe-tòò-la monon mãni bona}}
\pfra{ils pourront se retrouver demain et après-demain}
\end{exemple}
\end{entrée}

\begin{entrée}{pe-tròòli}{}{ⓔpe-tròòli}
\région{GOs PA}
\variante{%
pe-ròòli
\région{GO(s)}}
\classe{v ; n}
\newline
\sens{1}
(\domainesémantique{Réciproque collectif})
\begin{glose}
\pfra{rencontrer (se)}
\end{glose}
\begin{glose}
\pfra{réunion ; réunir (se)}
\end{glose}
\newline
\begin{exemple}
\textbf{\pnua{li pe-ròòli bwa de}}
\pfra{ils se sont rencontrés sur la route}
\end{exemple}
\newline
\sens{2}
(\domainesémantique{Relations et interaction sociales})
\classe{LOCUT}
\begin{glose}
\pfra{à bientôt}
\end{glose}
\newline
\begin{sous-entrée}{pe-ròòli !}{ⓔpe-tròòliⓢ2ⓝpe-ròòli !}
\begin{glose}
\pfra{à bientôt}
\end{glose}
\end{sous-entrée}
\newline
\begin{sous-entrée}{pe-ròòli io !}{ⓔpe-tròòliⓢ2ⓝpe-ròòli io !}
\begin{glose}
\pfra{à tout à l'heure !}
\end{glose}
\end{sous-entrée}
\newline
\begin{sous-entrée}{pe-ròòli mõnõ!}{ⓔpe-tròòliⓢ2ⓝpe-ròòli mõnõ!}
\begin{glose}
\pfra{à demain !}
\end{glose}
\end{sous-entrée}
\newline
\begin{sous-entrée}{pe-ròòli ni tree !}{ⓔpe-tròòliⓢ2ⓝpe-ròòli ni tree !}
\begin{glose}
\pfra{à un de ces jours !}
\end{glose}
\end{sous-entrée}
\end{entrée}

\begin{entrée}{pe-tròòli èò}{}{ⓔpe-tròòli èò}
\région{GOs}
(\domainesémantique{Relations et interaction sociales
, Réciproque collectif})
\classe{LOCUT}
\begin{glose}
\pfra{à tout à l'heure ; à tout de suite}
\end{glose}
\end{entrée}

\begin{entrée}{pe-tròòli mõnõ}{}{ⓔpe-tròòli mõnõ}
\région{GOs}
\variante{%
pe-ròòli mõnõ
\région{GO(s)}, 
pe-tòòli menon
\région{PA}}
\newline
\sens{1}
(\domainesémantique{Réciproque collectif})
\classe{v}
\begin{glose}
\pfra{dire au revoir (se) ; à demain !}
\end{glose}
\newline
\sens{2}
(\domainesémantique{Relations et interaction sociales})
\classe{LOCUT}
\begin{glose}
\pfra{à demain !}
\end{glose}
\end{entrée}

\begin{entrée}{pe-tròòli ni tree}{}{ⓔpe-tròòli ni tree}
\région{GOs}
(\domainesémantique{Relations et interaction sociales
, Réciproque collectif})
\classe{LOCUT}
\begin{glose}
\pfra{à un de ces jours !}
\end{glose}
\newline
\begin{exemple}
\textbf{\pnua{pe-trooli ni xa tree !}}
\pfra{à un de ces jours !}
\end{exemple}
\end{entrée}

\begin{entrée}{pe-thra}{}{ⓔpe-thra}
\formephonétique{pe-ʈʰa}
\région{GOs}
\variante{%
pe-thaa
\région{BO PA}}
(\domainesémantique{Soins du corps})
\classe{v}
\begin{glose}
\pfra{raser (se)}
\end{glose}
\newline
\begin{exemple}
\textbf{\pnua{e pe-thra}}
\pfra{il se rase (en train de, activité)}
\end{exemple}
\newline
\begin{exemple}
\textbf{\pnua{ge je pe-thra}}
\pfra{il est en train de se raser}
\end{exemple}
\newline
\begin{exemple}
\région{PA}
\textbf{\pnua{i pe-tha Pwayili}}
\pfra{P. se rase (en train de, activité)}
\end{exemple}
\newline
\begin{exemple}
\région{PA}
\textbf{\pnua{i còòxe pu-n wu ra u thaa}}
\pfra{il se coupe les poils àras}
\end{exemple}
\newline
\begin{sous-entrée}{ba-pe-thra}{ⓔpe-thraⓝba-pe-thra}
\begin{glose}
\pfra{rasoir}
\end{glose}
\newline
\relationsémantique{Cf.}{\lien{ⓔthraⓗ1}{thra}}
\glosecourte{raser}
\end{sous-entrée}
\end{entrée}

\begin{entrée}{peu}{1}{ⓔpeuⓗ1}
\formephonétique{pe.u}
\région{GOs}
\variante{%
peul
\région{PA BO}}
(\domainesémantique{Modalité, verbes modaux})
\classe{MODAL}
\begin{glose}
\pfra{en vain ; sans résultat ; tant pis ! ; ce n'est pas grave}
\end{glose}
\newline
\begin{exemple}
\région{GO}
\textbf{\pnua{nu môgu ni peu}}
\pfra{je travaille pour rien, sans résultat}
\end{exemple}
\newline
\begin{exemple}
\textbf{\pnua{mo vhaa ni peu}}
\pfra{nous parlons pour rien, dans le vide}
\end{exemple}
\end{entrée}

\begin{entrée}{peu}{2}{ⓔpeuⓗ2}
\région{BO}
(\domainesémantique{Mouvements ou actions avec la tête, les yeux, la bouche})
\classe{v}
\begin{glose}
\pfra{claquer des lèvres en signe de mépris [Corne]}
\end{glose}
\end{entrée}

\begin{entrée}{peve jun}{}{ⓔpeve jun}
\région{PA BO}
(\domainesémantique{Quantificateurs})
\classe{QNT}
\begin{glose}
\pfra{tout ; l'ensemble}
\end{glose}
\end{entrée}

\begin{entrée}{peven}{}{ⓔpeven}
\région{PA}
(\domainesémantique{Instruments})
\classe{v}
\begin{glose}
\pfra{servir à}
\end{glose}
\newline
\begin{exemple}
\textbf{\pnua{peve da jene ?}}
\pfra{ça sert à quoi ?}
\end{exemple}
\newline
\begin{exemple}
\textbf{\pnua{kixa pune peven}}
\pfra{ça ne sert à rien}
\end{exemple}
\end{entrée}

\begin{entrée}{pevera}{}{ⓔpevera}
\région{WEM WE}
(\domainesémantique{Caractéristiques et propriétés des personnes})
\classe{v}
\begin{glose}
\pfra{débrouillard}
\end{glose}
\end{entrée}

\begin{entrée}{pe-vii}{}{ⓔpe-vii}
\région{GOs}
(\domainesémantique{Relations et interaction sociales
, Réciproque collectif})
\classe{v}
\begin{glose}
\pfra{éviter (s')}
\end{glose}
\newline
\begin{exemple}
\région{GOs}
\textbf{\pnua{li pe-vii}}
\pfra{ils s'évitent}
\end{exemple}
\newline
\relationsémantique{Cf.}{\lien{ⓔpiiⓗ1}{pii}}
\glosecourte{éviter, esquiver}
\end{entrée}

\begin{entrée}{pevheańo}{}{ⓔpevheańo}
\formephonétique{peβeano}
\région{GOs PA}
\variante{%
pepeano
\région{GO(s)}}
(\domainesémantique{Insectes})
\classe{nom}
\begin{glose}
\pfra{grillon}
\end{glose}
\end{entrée}

\begin{entrée}{pevwe}{}{ⓔpevwe}
\région{GOs}
\variante{%
pevhe
\région{PA BO}, 
pepe
\région{vx}}
(\domainesémantique{Quantificateurs})
\classe{QNT}
\begin{glose}
\pfra{tous}
\end{glose}
\newline
\begin{exemple}
\région{GO}
\textbf{\pnua{la pevwe a bulu}}
\pfra{ils partent tous ensemble (en même temps)}
\end{exemple}
\newline
\begin{exemple}
\région{GO}
\textbf{\pnua{lò pevwe pe-be-yaza}}
\pfra{ils(3) ont tous le même nom}
\end{exemple}
\newline
\begin{exemple}
\région{BO}
\textbf{\pnua{hã ra pevhe ãgu}}
\pfra{nous sommes tous des hommes}
\end{exemple}
\newline
\begin{exemple}
\région{BO}
\textbf{\pnua{la pepe-po}}
\pfra{toutes les choses (Dubois)}
\end{exemple}
\newline
\begin{exemple}
\textbf{\pnua{peve ge la èna}}
\pfra{ils sont tous ici}
\end{exemple}
\newline
\relationsémantique{Cf.}{\lien{}{li pe-avwe bulu}}
\glosecourte{ils sont tous ensemble}
\newline
\relationsémantique{Cf.}{\lien{}{la pe-avwe bulu [GOs]}}
\glosecourte{ils sont tous ensemble}
\newline
\relationsémantique{Cf.}{\lien{}{cavwe [GOs]}}
\glosecourte{tous ensemble}
\end{entrée}

\begin{entrée}{pe-vwii}{}{ⓔpe-vwii}
\formephonétique{pe-βiː}
\région{GOs}
\variante{%
pe-viing
\région{PA WEM}, 
phiing
\région{PA BO}}
(\domainesémantique{Verbes de déplacement et moyens de déplacement
, Réciproque collectif})
\classe{v}
\begin{glose}
\pfra{rejoindre (se)}
\end{glose}
\begin{glose}
\pfra{joindre (se)}
\end{glose}
\newline
\begin{exemple}
\textbf{\pnua{la pe-vhii}}
\pfra{ils se rejoignent}
\end{exemple}
\newline
\begin{sous-entrée}{pe-phiing, pe-vhiing [BO]}{ⓔpe-vwiiⓝpe-phiing, pe-vhiing [BO]}
\begin{glose}
\pfra{se rejoindre}
\end{glose}
\end{sous-entrée}
\end{entrée}

\begin{entrée}{pevwö}{}{ⓔpevwö}
\formephonétique{peβω}
\région{GOs}
\classe{v}
\newline
\sens{1}
(\domainesémantique{Fonctions intellectuelles})
\begin{glose}
\pfra{attention (faire) à}
\end{glose}
\newline
\begin{exemple}
\région{GOs}
\textbf{\pnua{kavwö çö pevwö nai je}}
\pfra{ne fais pas attention à lui}
\end{exemple}
\newline
\sens{2}
(\domainesémantique{Soins du corps})
\begin{glose}
\pfra{s'occuper de (enfant, qqn)}
\end{glose}
\end{entrée}

\begin{entrée}{pe-wame}{}{ⓔpe-wame}
\région{GOs WEM}
(\domainesémantique{Comparaison})
\classe{v}
\begin{glose}
\pfra{ressembler (se) ; semblable (être)}
\end{glose}
\newline
\begin{exemple}
\textbf{\pnua{pe-wame-li}}
\pfra{elles se ressemblent}
\end{exemple}
\end{entrée}

\begin{entrée}{pe-wèle}{}{ⓔpe-wèle}
\région{BO PA}
(\domainesémantique{Relations et interaction sociales
, Réciproque collectif})
\classe{v}
\begin{glose}
\pfra{disputer (se) ; battre (se) (avec ou sans armes)}
\end{glose}
\end{entrée}

\begin{entrée}{pewi}{}{ⓔpewi}
\région{GOs PA}
(\domainesémantique{Verbes d'action faite par des animaux})
\classe{v}
\begin{glose}
\pfra{mordre (en parlant d'animaux)}
\end{glose}
\newline
\begin{exemple}
\textbf{\pnua{kuau a-pewi}}
\pfra{un chien qui mord}
\end{exemple}
\newline
\relationsémantique{Cf.}{\lien{}{huu [PA]}}
\glosecourte{manger (protéines, sucre), piquer, mordre}
\end{entrée}

\begin{entrée}{pewoo}{}{ⓔpewoo}
\région{BO [BM]}
(\domainesémantique{Relations et interaction sociales})
\classe{v}
\begin{glose}
\pfra{faire attention}
\end{glose}
\newline
\begin{exemple}
\région{BO}
\textbf{\pnua{kawu i pewo nani poi-n}}
\pfra{elle ne s'occupe pas de ses enfants}
\end{exemple}
\end{entrée}

\begin{entrée}{pe-wova}{}{ⓔpe-wova}
\région{GOs PA}
\variante{%
pe-woza
\région{WEM}}
(\domainesémantique{Guerre
, Réciproque collectif})
\classe{v}
\begin{glose}
\pfra{battre (se) (avec ou sans armes)}
\end{glose}
\newline
\relationsémantique{Cf.}{\lien{ⓔpe-wèle}{pe-wèle}}
\end{entrée}

\begin{entrée}{pe-whaguzai}{}{ⓔpe-whaguzai}
\région{GOs}
(\domainesémantique{Discours, échanges verbaux
, Réciproque collectif})
\classe{v ; n}
\begin{glose}
\pfra{débattre ; discuter ; débat ; discussion}
\end{glose}
\end{entrée}

\begin{entrée}{pe-whili}{}{ⓔpe-whili}
\région{GOs}
(\domainesémantique{Verbes de déplacement et moyens de déplacement
, Réciproque collectif})
\classe{v}
\begin{glose}
\pfra{suivre (se) ; marcher l'un derrière l'autre}
\end{glose}
\newline
\begin{exemple}
\textbf{\pnua{li pe-whili}}
\pfra{ils se suivent}
\end{exemple}
\end{entrée}

\begin{entrée}{pexa}{}{ⓔpexa}
\région{GOs PA}
(\domainesémantique{Prépositions})
\classe{PREP}
\begin{glose}
\pfra{au sujet de ; à propos de (sens maléfactif)}
\end{glose}
\newline
\begin{exemple}
\région{GOs}
\textbf{\pnua{lie mèèwu, li pe-kweli-li pexa nye thoomwã}}
\pfra{les deux frères, ils se détestent à cause de cette femme}
\end{exemple}
\newline
\begin{exemple}
\région{GOs}
\textbf{\pnua{la pe-vhaa pexa nye-na}}
\pfra{ils ont discuté de cela}
\end{exemple}
\newline
\begin{exemple}
\région{GOs}
\textbf{\pnua{la pe-vhaa pexa nu}}
\pfra{ils ont discuté de moi}
\end{exemple}
\newline
\relationsémantique{Cf.}{\lien{ⓔpune}{pune}}
\glosecourte{à cause de}
\end{entrée}

\begin{entrée}{pexu}{}{ⓔpexu}
\région{GOs PA BO}
(\domainesémantique{Relations et interaction sociales})
\classe{v ; n}
\begin{glose}
\pfra{médire ; médisance}
\end{glose}
\newline
\begin{exemple}
\textbf{\pnua{li pexu i-nu}}
\pfra{ils disent du mal de moi}
\end{exemple}
\newline
\begin{exemple}
\textbf{\pnua{e a-pe-hexu}}
\pfra{il est médisant}
\end{exemple}
\end{entrée}

\begin{entrée}{pe-yu}{}{ⓔpe-yu}
\région{GOs}
(\domainesémantique{Société})
\classe{v}
\begin{glose}
\pfra{célibataire (être)}
\end{glose}
\newline
\begin{exemple}
\textbf{\pnua{e pe-yu}}
\pfra{elle est célibataire}
\end{exemple}
\newline
\begin{exemple}
\textbf{\pnua{za pe-yu ni phãgoo-nu}}
\pfra{c'est resté dans mon corps}
\end{exemple}
\end{entrée}

\begin{entrée}{pezii}{}{ⓔpezii}
\région{GOs}
(\domainesémantique{Jeux divers})
\classe{v}
\begin{glose}
\pfra{jouer}
\end{glose}
\newline
\note{noté peti, pezi par Grace}{général}{}
\end{entrée}

\begin{entrée}{pezoli}{}{ⓔpezoli}
\région{GOs}
\variante{%
peroli
\région{PA}}
(\domainesémantique{Description des objets, formes, consistance, taille})
\classe{v.stat.}
\begin{glose}
\pfra{rugueux}
\end{glose}
\end{entrée}

\begin{entrée}{pe-zööni}{}{ⓔpe-zööni}
\région{GOs}
(\domainesémantique{Relations et interaction sociales})
\classe{v}
\begin{glose}
\pfra{maudire}
\end{glose}
\end{entrée}

\begin{entrée}{pi}{}{ⓔpi}
\région{GOs PA BO WE WEM GA}
\newline
\groupe{A}
\classe{nom}
\newline
\sens{1}
(\domainesémantique{Oiseaux})
\begin{glose}
\pfra{oeuf}
\end{glose}
\newline
\begin{sous-entrée}{pi-ko}{ⓔpiⓢ1ⓝpi-ko}
\région{GO}
\begin{glose}
\pfra{oeuf de poule}
\end{glose}
\end{sous-entrée}
\newline
\sens{2}
(\domainesémantique{Poissons})
\begin{glose}
\pfra{frai ; oeufs (crustacés) ; laitance (poisson)}
\end{glose}
\newline
\begin{sous-entrée}{pi-a no}{ⓔpiⓢ2ⓝpi-a no}
\région{GO}
\begin{glose}
\pfra{oeufs de poisson}
\end{glose}
\end{sous-entrée}
\newline
\begin{sous-entrée}{pi-pwon}{ⓔpiⓢ2ⓝpi-pwon}
\région{PA}
\begin{glose}
\pfra{oeuf de tortue}
\end{glose}
\newline
\relationsémantique{Cf.}{\lien{ⓔêgo}{êgo}}
\glosecourte{oeuf}
\end{sous-entrée}
\newline
\sens{3}
(\domainesémantique{Corps humain})
\begin{glose}
\pfra{testicule}
\end{glose}
\newline
\begin{exemple}
\région{PA}
\textbf{\pnua{pi-n}}
\pfra{testicule}
\end{exemple}
\newline
\groupe{B}
\classe{v}
\newline
\sens{4}
(\domainesémantique{Fonctions naturelles des animaux})
\begin{glose}
\pfra{oeuf (avoir des)}
\end{glose}
\newline
\begin{exemple}
\textbf{\pnua{e pi}}
\pfra{il a des oeufs}
\end{exemple}
\newline
\étymologie{
\langue{POc}
\étymon{*mpiRa}
\glosecourte{oeuf}}
\end{entrée}

\begin{entrée}{pia}{}{ⓔpia}
\région{GOs PA}
(\domainesémantique{Taros})
\classe{nom}
\begin{glose}
\pfra{taro géant (sauvage, à larges feuilles)}
\end{glose}
\nomscientifique{Alocasia sp.(Aracées)}
\newline
\étymologie{
\langue{POc}
\étymon{*(m)piraq}}
\end{entrée}

\begin{entrée}{pi-ai}{}{ⓔpi-ai}
\région{GOs BO}
(\domainesémantique{Corps humain})
\classe{nom}
\begin{glose}
\pfra{omoplate (lit. la carapace du coeur)}
\end{glose}
\newline
\begin{exemple}
\textbf{\pnua{pi-ai-nu}}
\pfra{mon omoplate}
\end{exemple}
\end{entrée}

\begin{entrée}{pi-ãmu}{}{ⓔpi-ãmu}
\région{BO PA}
(\domainesémantique{Aliments, alimentation})
\classe{nom}
\begin{glose}
\pfra{miel en rayon}
\end{glose}
\end{entrée}

\begin{entrée}{pi-bwa}{}{ⓔpi-bwa}
\région{GOs PA}
\variante{%
du-bwaa-n
\région{BO PA}}
(\domainesémantique{Corps humain})
\classe{nom}
\begin{glose}
\pfra{crâne}
\end{glose}
\newline
\begin{exemple}
\région{GOs}
\textbf{\pnua{pi-bwaa-nu}}
\pfra{mon crâne}
\end{exemple}
\end{entrée}

\begin{entrée}{pi-bwèdò}{}{ⓔpi-bwèdò}
\région{GOs BO}
(\domainesémantique{Corps humain})
\classe{nom}
\begin{glose}
\pfra{ongle}
\end{glose}
\newline
\begin{exemple}
\région{PA}
\textbf{\pnua{pi-bwèdò-hi-n}}
\pfra{ses ongles de doigt de main}
\end{exemple}
\newline
\begin{exemple}
\région{PA}
\textbf{\pnua{pi-bwèdò-kò-n}}
\pfra{ses ongles de doigt de main}
\end{exemple}
\newline
\begin{exemple}
\région{GO}
\textbf{\pnua{pi-kòò-je}}
\pfra{ses ongles de pied}
\end{exemple}
\newline
\begin{exemple}
\région{GO}
\textbf{\pnua{pi-hii-je}}
\pfra{ses ongles de main}
\end{exemple}
\newline
\begin{exemple}
\région{PA}
\textbf{\pnua{pi-hii-n}}
\pfra{ses ongles des doigts}
\end{exemple}
\newline
\relationsémantique{Cf.}{\lien{}{bwèdò [GOs, PA]}}
\glosecourte{doigt}
\end{entrée}

\begin{entrée}{pibwena}{}{ⓔpibwena}
\région{PA}
(\domainesémantique{Taros})
\classe{nom}
\begin{glose}
\pfra{taro (clone)}
\end{glose}
\end{entrée}

\begin{entrée}{piça}{}{ⓔpiça}
\formephonétique{piʒa, pidʒa}
\région{GOs}
\variante{%
piya
\région{BO}}
(\domainesémantique{Description des objets, formes, consistance, taille})
\classe{v}
\begin{glose}
\pfra{dur}
\end{glose}
\begin{glose}
\pfra{résistant}
\end{glose}
\begin{glose}
\pfra{fort}
\end{glose}
\begin{glose}
\pfra{solide}
\end{glose}
\newline
\begin{exemple}
\région{GO}
\textbf{\pnua{e piça phwe-mwa}}
\pfra{la porte est dure (à ouvrir)}
\end{exemple}
\newline
\begin{exemple}
\région{GO}
\textbf{\pnua{e piça bwaa-je}}
\pfra{il est têtu}
\end{exemple}
\newline
\begin{exemple}
\région{BO}
\textbf{\pnua{cimwi piyaa-ni}}
\pfra{serre-le fort}
\end{exemple}
\newline
\relationsémantique{Cf.}{\lien{}{nhyatru ; hnyaru}}
\glosecourte{mou (substance, terre)}
\end{entrée}

\begin{entrée}{piçanga}{}{ⓔpiçanga}
\formephonétique{pidʒaŋa}
\région{GOs}
\variante{%
pijanga
\région{BO}}
(\domainesémantique{Corps humain})
\classe{nom}
\begin{glose}
\pfra{aine}
\end{glose}
\newline
\begin{exemple}
\région{BO}
\textbf{\pnua{pijanga-n}}
\pfra{son aine}
\end{exemple}
\end{entrée}

\begin{entrée}{piçilè}{}{ⓔpiçilè}
\formephonétique{pidʒilɛ}
\région{GOs}
\variante{%
pivileng
\région{PA}}
(\domainesémantique{Insectes})
\classe{nom}
\begin{glose}
\pfra{guêpe maçonne}
\end{glose}
\newline
\begin{exemple}
\région{BO}
\textbf{\pnua{pi-pivileng}}
\pfra{nid de la guêpe maçonne}
\end{exemple}
\end{entrée}

\begin{entrée}{pidi}{}{ⓔpidi}
\région{GO}
(\domainesémantique{Corps humain})
\classe{nom}
\begin{glose}
\pfra{clitoris}
\end{glose}
\end{entrée}

\begin{entrée}{pi-dili}{}{ⓔpi-dili}
\région{GOs PA}
\variante{%
pigo dili
\région{PA}}
(\domainesémantique{Cultures, techniques, boutures})
\classe{nom}
\begin{glose}
\pfra{motte de terre}
\end{glose}
\end{entrée}

\begin{entrée}{pidru}{}{ⓔpidru}
\région{GOs}
\variante{%
pidu
\région{PA BO}}
(\domainesémantique{Parenté})
\classe{nom}
\begin{glose}
\pfra{jumeaux}
\end{glose}
\end{entrée}

\begin{entrée}{piga}{}{ⓔpiga}
\région{PA}
(\domainesémantique{Verbes d'action (en général)})
\classe{v}
\begin{glose}
\pfra{craquer}
\end{glose}
\end{entrée}

\begin{entrée}{pigi yaai}{}{ⓔpigi yaai}
\région{GOs}
(\domainesémantique{Feu : objets et actions liés au feu})
\classe{v}
\begin{glose}
\pfra{pousser le feu (sous la marmite, ou dans la maison, moins fort que 'carûni')}
\end{glose}
\newline
\relationsémantique{Cf.}{\lien{ⓔtha-carûni}{tha-carûni}}
\glosecourte{pousser le feu}
\end{entrée}

\begin{entrée}{pi-hi}{}{ⓔpi-hi}
\région{GOs PA BO}
\variante{%
pi-yi
\région{BO}}
(\domainesémantique{Corps humain})
\classe{nom}
\begin{glose}
\pfra{ongle}
\end{glose}
\newline
\begin{exemple}
\région{GOs}
\textbf{\pnua{pi-hii je}}
\pfra{ses ongles de main}
\end{exemple}
\newline
\begin{exemple}
\textbf{\pnua{pi-hii-n [PA, BO]}}
\pfra{ses ongles des doigts}
\end{exemple}
\newline
\begin{exemple}
\région{PA}
\textbf{\pnua{pi-yii-n}}
\pfra{ses ongles des doigts}
\end{exemple}
\end{entrée}

\begin{entrée}{pii}{1}{ⓔpiiⓗ1}
\région{GOs BO PA}
(\domainesémantique{Mouvements ou actions faits avec le corps, les bras, les mains, les pieds})
\classe{v}
\begin{glose}
\pfra{couper en deux ; partager}
\end{glose}
\begin{glose}
\pfra{rompre (un bout de pain, un bout d'igname cuite)}
\end{glose}
\newline
\begin{exemple}
\textbf{\pnua{e pii phwalawa}}
\pfra{il a coupé le pain}
\end{exemple}
\end{entrée}

\begin{entrée}{pii}{2}{ⓔpiiⓗ2}
\région{GOs PA BO}
\classe{v}
\newline
\sens{1}
(\domainesémantique{Relations et interaction sociales})
\begin{glose}
\pfra{éviter (qqn ou qqch) ; esquiver}
\end{glose}
\newline
\begin{exemple}
\région{GOs}
\textbf{\pnua{e pii do}}
\pfra{il évite la sagaie}
\end{exemple}
\newline
\begin{exemple}
\région{GOs}
\textbf{\pnua{nu pii pa}}
\pfra{j'ai évité la pierre}
\end{exemple}
\newline
\begin{sous-entrée}{pe-vii, pe-pii}{ⓔpiiⓗ2ⓢ1ⓝpe-vii, pe-pii}
\région{GOs}
\begin{glose}
\pfra{s'éviter}
\end{glose}
\end{sous-entrée}
\newline
\sens{2}
(\domainesémantique{Guerre})
\begin{glose}
\pfra{parer (un coup)}
\end{glose}
\end{entrée}

\begin{entrée}{pii}{3}{ⓔpiiⓗ3}
\région{GOs PA BO}
\variante{%
pii-n
\région{BO}}
\classe{v.stat. ; n}
\newline
\sens{1}
(\domainesémantique{Description des objets, formes, consistance, taille})
\begin{glose}
\pfra{vide}
\end{glose}
\newline
\begin{exemple}
\région{GOs}
\textbf{\pnua{e pii kamyõ}}
\pfra{le camion est vide}
\end{exemple}
\newline
\begin{exemple}
\région{GOs}
\textbf{\pnua{e pii dröö}}
\pfra{la marmite est vide}
\end{exemple}
\newline
\begin{sous-entrée}{paa-pii-ni [GOs]}{ⓔpiiⓗ3ⓢ1ⓝpaa-pii-ni [GOs]}
\begin{glose}
\pfra{vider qqch}
\end{glose}
\newline
\begin{exemple}
\région{BO}
\textbf{\pnua{pii-n}}
\pfra{c'est vide}
\end{exemple}
\newline
\begin{exemple}
\région{PA}
\textbf{\pnua{pii-bwat}}
\pfra{une boîte vide}
\end{exemple}
\newline
\begin{exemple}
\région{PA}
\textbf{\pnua{piin (a) mwa}}
\pfra{la maison est vide (d'objets)}
\end{exemple}
\end{sous-entrée}
\newline
\sens{2}
(\domainesémantique{Crustacés, crabes})
\begin{glose}
\pfra{carapace vide}
\end{glose}
\newline
\begin{exemple}
\région{BO}
\textbf{\pnua{pi-pwaji}}
\pfra{carapace vide de crabe}
\end{exemple}
\newline
\sens{3}
(\domainesémantique{Mollusques})
\begin{glose}
\pfra{coquille vide (de coquillage)}
\end{glose}
\newline
\begin{exemple}
\région{PA}
\textbf{\pnua{pii-n}}
\pfra{sa coquille}
\end{exemple}
\newline
\begin{exemple}
\région{BO}
\textbf{\pnua{pi-tagiliã}}
\pfra{coquille vide de bénitier}
\end{exemple}
\end{entrée}

\begin{entrée}{pii}{4}{ⓔpiiⓗ4}
\région{GOs PA BO}
(\domainesémantique{Corps animal})
\classe{nom}
\begin{glose}
\pfra{carapace ; écaille de tortue}
\end{glose}
\newline
\begin{exemple}
\textbf{\pnua{pii-n}}
\pfra{sa carapace}
\end{exemple}
\end{entrée}

\begin{entrée}{piia}{}{ⓔpiia}
\région{GOs BO PA}
(\domainesémantique{Relations et interaction sociales})
\classe{v}
\begin{glose}
\pfra{disputer (se) (verbalement) ; chamailler (se)}
\end{glose}
\newline
\begin{exemple}
\région{GOs}
\textbf{\pnua{pe-piia, pe-vhiia}}
\pfra{se combattre, se faire la guerre}
\end{exemple}
\newline
\begin{exemple}
\région{PA}
\textbf{\pnua{la piia pexa dili}}
\pfra{ils se disputent à propos des terres}
\end{exemple}
\newline
\begin{exemple}
\région{PA}
\textbf{\pnua{li pe-pii-li}}
\pfra{ils se disputent}
\end{exemple}
\end{entrée}

\begin{entrée}{pii-gu}{}{ⓔpii-gu}
\région{GOs PA BO}
\classe{nom}
\newline
\sens{1}
(\domainesémantique{Mollusques})
\begin{glose}
\pfra{valve de coquillage}
\end{glose}
\newline
\sens{2}
(\domainesémantique{Instruments})
\begin{glose}
\pfra{râpe (faite d'une valve de coquillage, utilisée pour gratter le coco, banane, etc.)}
\end{glose}
\end{entrée}

\begin{entrée}{pii-me}{}{ⓔpii-me}
\région{GOs BO PA}
(\domainesémantique{Corps humain})
\classe{nom}
\begin{glose}
\pfra{oeil}
\end{glose}
\newline
\begin{exemple}
\région{PA}
\textbf{\pnua{pii-mee-m}}
\pfra{ton oeil}
\end{exemple}
\end{entrée}

\begin{entrée}{pîînã}{}{ⓔpîînã}
\région{GOs BO PA}
(\domainesémantique{Verbes de déplacement et moyens de déplacement})
\classe{v}
\begin{glose}
\pfra{voyager ; promener (se) ;}
\end{glose}
\begin{glose}
\pfra{rendre visite}
\end{glose}
\newline
\begin{exemple}
\textbf{\pnua{eniza nye çö thaavwu pîîna-du èbòli bwaabu ?}}
\pfra{quand es-tu allée en France pour la première fois ?}
\end{exemple}
\newline
\begin{sous-entrée}{pe-pîînã}{ⓔpîînãⓝpe-pîînã}
\begin{glose}
\pfra{se promener}
\end{glose}
\end{sous-entrée}
\end{entrée}

\begin{entrée}{pii-nu}{}{ⓔpii-nu}
\formephonétique{piː-ɳu}
\région{GOs}
\variante{%
pi-nu
\région{GO(s)}}
(\domainesémantique{Cocotiers})
\classe{nom}
\begin{glose}
\pfra{coquille vide de noix de coco}
\end{glose}
\end{entrée}

\begin{entrée}{pii-peçi}{}{ⓔpii-peçi}
\formephonétique{piː-peʒi}
\région{GOs}
\variante{%
pi-peyi, pi-peji
\région{PA}}
(\domainesémantique{Corps humain})
\classe{nom}
\begin{glose}
\pfra{rotule (lit. carapace de 'savonnette')}
\end{glose}
\begin{glose}
\pfra{malléole}
\end{glose}
\newline
\begin{exemple}
\textbf{\pnua{pi peya ko-ny [BO]}}
\pfra{ma rotule}
\end{exemple}
\newline
\begin{sous-entrée}{we-peji}{ⓔpii-peçiⓝwe-peji}
\begin{glose}
\pfra{synovie}
\end{glose}
\end{sous-entrée}
\end{entrée}

\begin{entrée}{pii-pò}{}{ⓔpii-pò}
\formephonétique{piː-pɔ}
\région{GOs}
\variante{%
pii-pwòn
}
(\domainesémantique{Reptiles marins})
\classe{nom}
\begin{glose}
\pfra{carapace de tortue}
\end{glose}
\end{entrée}

\begin{entrée}{pii-pwaji}{}{ⓔpii-pwaji}
\région{GOs PA}
(\domainesémantique{Crustacés, crabes})
\classe{nom}
\begin{glose}
\pfra{carapace de crabe}
\end{glose}
\end{entrée}

\begin{entrée}{pii-ragooni}{}{ⓔpii-ragooni}
\région{GOs}
(\domainesémantique{Mollusques})
\classe{nom}
\begin{glose}
\pfra{coquille de coquille saint-jacques (sert de grattoir à coco et à papaye)}
\end{glose}
\end{entrée}

\begin{entrée}{piixã}{}{ⓔpiixã}
\région{GOs}
(\domainesémantique{Poissons})
\classe{nom}
\begin{glose}
\pfra{picot noir ; picot (en général)}
\end{glose}
\end{entrée}

\begin{entrée}{piixã ni paa}{}{ⓔpiixã ni paa}
\région{GOs}
(\domainesémantique{Poissons})
\classe{nom}
\begin{glose}
\pfra{picot rayé (qui se cache entre les pierres)}
\end{glose}
\end{entrée}

\begin{entrée}{pijeva}{}{ⓔpijeva}
\région{GO}
\variante{%
pijopa
\région{GO}}
(\domainesémantique{Danses})
\classe{nom}
\begin{glose}
\pfra{danse (type de)}
\end{glose}
\newline
\note{non vérifié}{général}{}
\end{entrée}

\begin{entrée}{pijoo}{}{ⓔpijoo}
\région{BO}
(\domainesémantique{Verbes d'action (en général)})
\classe{v}
\begin{glose}
\pfra{briser ; casser à moitié [BM]}
\end{glose}
\end{entrée}

\begin{entrée}{pijopa}{}{ⓔpijopa}
\région{GO}
(\domainesémantique{Religion, représentations religieuses})
\classe{nom}
\begin{glose}
\pfra{dieu des enfers}
\end{glose}
\newline
\note{non vérifié}{général}{}
\end{entrée}

\begin{entrée}{pi-kò}{}{ⓔpi-kò}
\région{GOs PA}
(\domainesémantique{Corps humain})
\classe{nom}
\begin{glose}
\pfra{griffe ; ongle de pied}
\end{glose}
\newline
\begin{sous-entrée}{pi-kò kuau}{ⓔpi-kòⓝpi-kò kuau}
\begin{glose}
\pfra{griffe de chien}
\end{glose}
\newline
\begin{exemple}
\région{PA}
\textbf{\pnua{pi-kòò-n}}
\pfra{ongle de pied}
\end{exemple}
\end{sous-entrée}
\end{entrée}

\begin{entrée}{pi-mèni}{}{ⓔpi-mèni}
\région{WE WEM GA}
(\domainesémantique{Oiseaux})
\classe{nom}
\begin{glose}
\pfra{oeuf d'oiseau}
\end{glose}
\end{entrée}

\begin{entrée}{pi-nõ}{}{ⓔpi-nõ}
\formephonétique{pi-nɔ̃}
\région{GOs PA BO}
(\domainesémantique{Vêtements, parure})
\classe{nom}
\begin{glose}
\pfra{collier}
\end{glose}
\begin{glose}
\pfra{pendentif}
\end{glose}
\newline
\begin{exemple}
\région{GO}
\textbf{\pnua{pi-nõõ-je}}
\pfra{son collier}
\end{exemple}
\newline
\begin{exemple}
\région{PA BO}
\textbf{\pnua{pi-nõõ-n}}
\pfra{son collier}
\end{exemple}
\end{entrée}

\begin{entrée}{pio}{}{ⓔpio}
\région{GOs PA BO}
\classe{nom}
\newline
\sens{1}
(\domainesémantique{Astres})
\begin{glose}
\pfra{étoile}
\end{glose}
\newline
\sens{2}
(\domainesémantique{Echinodermes})
\begin{glose}
\pfra{étoile de mer [GOs]}
\end{glose}
\newline
\étymologie{
\langue{POc}
\étymon{*pituqun}}
\end{entrée}

\begin{entrée}{piòò}{}{ⓔpiòò}
\région{GOs}
\variante{%
piyòc
\région{BO}}
(\domainesémantique{Outils})
\classe{nom}
\begin{glose}
\pfra{pioche}
\end{glose}
\newline
\emprunt{pioche (FR)}
\end{entrée}

\begin{entrée}{pio yaai}{}{ⓔpio yaai}
\région{BO}
(\domainesémantique{Astres})
\classe{nom}
\begin{glose}
\pfra{Mars (lit. étoile feu)}
\end{glose}
\end{entrée}

\begin{entrée}{pira}{}{ⓔpira}
\région{GOs}
(\domainesémantique{Localisation})
\classe{LOC}
\begin{glose}
\pfra{dessous (en) ; sous (d'une surface, d'un point)}
\end{glose}
\newline
\begin{exemple}
\textbf{\pnua{a-du pira ta}}
\pfra{va sous la table}
\end{exemple}
\newline
\begin{exemple}
\textbf{\pnua{a-e traabwa ni pira ce}}
\pfra{va t'asseoir sous l'arbre}
\end{exemple}
\newline
\begin{exemple}
\textbf{\pnua{e ti da ni pira khii-n}}
\pfra{elle le met sous sa jupe}
\end{exemple}
\end{entrée}

\begin{entrée}{pitre}{}{ⓔpitre}
\formephonétique{piʈe}
\région{GOs}
\variante{%
pire
\région{BO}}
(\domainesémantique{Tressage})
\classe{v ; n}
\begin{glose}
\pfra{tresser (corde) ; corde tressée}
\end{glose}
\end{entrée}

\begin{entrée}{pitrêê}{}{ⓔpitrêê}
\région{GOs}
(\domainesémantique{Crustacés, crabes})
\classe{nom}
\begin{glose}
\pfra{crabe plein ('double peau', lorsque sa carapace dure se détache)}
\end{glose}
\end{entrée}

\begin{entrée}{pivida}{}{ⓔpivida}
\région{PA}
(\domainesémantique{Insectes})
\classe{nom}
\begin{glose}
\pfra{sauterelle (cette sauterelle à pattes rouges et au corps jaune fait un bruit de crécelle)}
\end{glose}
\end{entrée}

\begin{entrée}{piviige}{}{ⓔpiviige}
\région{BO}
(\domainesémantique{Mouvements ou actions faits avec le corps, les bras, les mains, les pieds})
\classe{v}
\begin{glose}
\pfra{saisir avec le bras ; embrasser [Corne]}
\end{glose}
\newline
\note{non vérifié}{général}{}
\end{entrée}

\begin{entrée}{pivivu}{}{ⓔpivivu}
\région{BO}
(\domainesémantique{Mammifères})
\classe{nom}
\begin{glose}
\pfra{chauve-souris [Corne]}
\end{glose}
\end{entrée}

\begin{entrée}{pivwia}{}{ⓔpivwia}
\formephonétique{piβia}
\région{GOs}
\variante{%
pivia
\région{PA}, 
pevia
\région{BO}}
(\domainesémantique{Relations et interaction sociales})
\classe{v}
\begin{glose}
\pfra{quereller (se) ; disputer (se) ; gronder}
\end{glose}
\end{entrée}

\begin{entrée}{pivwilo}{}{ⓔpivwilo}
\formephonétique{piβilo}
\région{GOs}
\variante{%
pwivwilö
\région{GO(s)}, 
pivhilö
\région{PA}, 
pevelo
\région{BO}}
(\domainesémantique{Oiseaux})
\classe{nom}
\begin{glose}
\pfra{hirondelle (à dos noir) ; martinet}
\end{glose}
\nomscientifique{Collocolia spodopygia leucopygia (ou)?? Artamus leucorhynchus melaleucus}
\end{entrée}

\begin{entrée}{pivwizai}{}{ⓔpivwizai}
\région{GOs}
(\domainesémantique{Description des objets, formes, consistance, taille})
\classe{v}
\begin{glose}
\pfra{étroit (passage en mer, sur terre)}
\end{glose}
\newline
\begin{exemple}
\région{GOs}
\textbf{\pnua{nu thu-menõ bwa dè ka/xa e pivwizai}}
\pfra{je me promène sur un chemin étroit}
\end{exemple}
\newline
\relationsémantique{Ant.}{\lien{ⓔwala}{wala}}
\glosecourte{large}
\end{entrée}

\begin{entrée}{pi-wãge}{}{ⓔpi-wãge}
\région{GOs}
(\domainesémantique{Corps humain})
\classe{nom}
\begin{glose}
\pfra{clavicule}
\end{glose}
\newline
\begin{exemple}
\textbf{\pnua{pi-wãge-nu}}
\pfra{ma clavicule}
\end{exemple}
\end{entrée}

\begin{entrée}{pixãge}{}{ⓔpixãge}
\région{GOs}
(\domainesémantique{Religion, représentations religieuses})
\classe{nom}
\begin{glose}
\pfra{croix}
\end{glose}
\end{entrée}

\begin{entrée}{pixu}{}{ⓔpixu}
\région{GOs}
(\domainesémantique{Sons, bruits})
\classe{v}
\begin{glose}
\pfra{grincer}
\end{glose}
\newline
\begin{exemple}
\région{GOs}
\textbf{\pnua{e pixu phwee-mwa}}
\pfra{la porte grince}
\end{exemple}
\newline
\begin{exemple}
\région{GOs}
\textbf{\pnua{e pixu vele}}
\pfra{le lit grince}
\end{exemple}
\newline
\begin{exemple}
\région{GOs}
\textbf{\pnua{la pixu ce}}
\pfra{les arbres ploient en faisant du bruit (sous l'effet du vent)}
\end{exemple}
\end{entrée}

\begin{entrée}{piyu}{}{ⓔpiyu}
\région{PA BO}
(\domainesémantique{Armes})
\classe{nom}
\begin{glose}
\pfra{pierre de fronde (petite, noire, dure)}
\end{glose}
\newline
\begin{exemple}
\textbf{\pnua{pa-piyu}}
\pfra{serpentine (avec laquelle on faisait les pierres de fronde)}
\end{exemple}
\end{entrée}

\begin{entrée}{piyuli}{}{ⓔpiyuli}
\région{PA BO [BM]}
(\domainesémantique{Relations et interaction sociales})
\classe{v}
\begin{glose}
\pfra{disputer (se) ; chercher querelle ; reprocher}
\end{glose}
\newline
\begin{sous-entrée}{a-piyuli}{ⓔpiyuliⓝa-piyuli}
\begin{glose}
\pfra{qui cherche toujours des querelles, qqn reproche toujours qqch}
\end{glose}
\end{sous-entrée}
\newline
\begin{sous-entrée}{pe-piyuli}{ⓔpiyuliⓝpe-piyuli}
\begin{glose}
\pfra{se disputer, se chamailler}
\end{glose}
\end{sous-entrée}
\end{entrée}

\begin{entrée}{pizò}{}{ⓔpizò}
\formephonétique{piðɔ}
\région{GOs}
\variante{%
pilo-n, pilòò, pilò
\région{PA BO}, 
pila
\région{BO}}
(\domainesémantique{Aliments, alimentation})
\classe{nom}
\begin{glose}
\pfra{chair (en général) ; viande ; muscle}
\end{glose}
\begin{glose}
\pfra{chair (d'igname, taro)}
\end{glose}
\newline
\begin{sous-entrée}{pizò kui [GOs]}{ⓔpizòⓝpizò kui [GOs]}
\begin{glose}
\pfra{chair de l'igname}
\end{glose}
\newline
\begin{exemple}
\région{BO}
\textbf{\pnua{pila kuru}}
\pfra{chair du taro (Dubois)}
\end{exemple}
\newline
\relationsémantique{Cf.}{\lien{ⓔlayô}{layô}}
\glosecourte{viande}
\end{sous-entrée}
\end{entrée}

\begin{entrée}{po}{1}{ⓔpoⓗ1}
\région{GOs BO}
\variante{%
poo
\région{BO}}
(\domainesémantique{Pronoms})
\classe{n ; PRO}
\begin{glose}
\pfra{chose ; quelque chose}
\end{glose}
\newline
\begin{exemple}
\région{BO}
\textbf{\pnua{kawu nu nooli-xa po}}
\pfra{je ne vois rien}
\end{exemple}
\newline
\begin{exemple}
\région{BO}
\textbf{\pnua{i khila-xa po}}
\pfra{il cherche qqch.}
\end{exemple}
\end{entrée}

\begin{entrée}{po}{2}{ⓔpoⓗ2}
\région{GOs BO}
\variante{%
pwò
\région{BO}, 
thu
\région{PA}}
(\domainesémantique{Verbes d'action (en général)})
\classe{v}
\begin{glose}
\pfra{faire}
\end{glose}
\begin{glose}
\pfra{il y a}
\end{glose}
\newline
\begin{exemple}
\région{GOs}
\textbf{\pnua{e po za ?}}
\pfra{qu'a-t-elle fait ?}
\end{exemple}
\newline
\begin{exemple}
\région{BO}
\textbf{\pnua{i pwò ra ?}}
\pfra{que fait-elle ?}
\end{exemple}
\newline
\begin{exemple}
\textbf{\pnua{i po na ?}}
\pfra{qu'a-t-elle fait ?}
\end{exemple}
\newline
\begin{sous-entrée}{po za ? ; po ra ?}{ⓔpoⓗ2ⓝpo za ? ; po ra ?}
\begin{glose}
\pfra{faire comment ?}
\end{glose}
\end{sous-entrée}
\newline
\étymologie{
\langue{POc}
\étymon{*pua(t)}}
\end{entrée}

\begin{entrée}{po}{3}{ⓔpoⓗ3}
\région{GOs}
\variante{%
vwo
\région{GO(s)}, 
pu
\région{PA BO}}
(\domainesémantique{Conjonction})
\classe{CNJ}
\begin{glose}
\pfra{que ; pour que ; afin que ; si}
\end{glose}
\newline
\begin{exemple}
\textbf{\pnua{e khõbwe wö mõ a}}
\pfra{il nous a dit de partir}
\end{exemple}
\end{entrée}

\begin{entrée}{po-}{}{ⓔpo-}
\région{GOs PA}
(\domainesémantique{Préfixes classificateurs numériques})
\classe{CLF.NUM (générique et des objets ronds)}
\begin{glose}
\pfra{objets ronds (fruits, heure, etc.)}
\end{glose}
\newline
\begin{exemple}
\textbf{\pnua{po-xe (1); po-tru (2); po-ko, po-pa, etc.}}
\pfra{un, deux, trois, quatre objets ronds}
\end{exemple}
\newline
\begin{exemple}
\textbf{\pnua{po-xè pò-mã}}
\pfra{une mangue}
\end{exemple}
\newline
\note{a(a)- (animés), go- , we- , pepo-}{général}{}
\end{entrée}

\begin{entrée}{pò}{1}{ⓔpòⓗ1}
\région{GOs}
\variante{%
pò-n
\région{PA BO}, 
pwò
\région{BO}}
\newline
\sens{1}
(\domainesémantique{Parties de plantes})
\classe{nom}
\begin{glose}
\pfra{fruit ; graine}
\end{glose}
\newline
\begin{exemple}
\région{PA}
\textbf{\pnua{pòò-n ; pò-n}}
\pfra{son fruit}
\end{exemple}
\newline
\begin{sous-entrée}{pò-caai}{ⓔpòⓗ1ⓢ1ⓝpò-caai}
\begin{glose}
\pfra{fruit du jamelonier}
\end{glose}
\end{sous-entrée}
\newline
\begin{sous-entrée}{pwò-mãã}{ⓔpòⓗ1ⓢ1ⓝpwò-mãã}
\région{GO}
\begin{glose}
\pfra{mangue}
\end{glose}
\end{sous-entrée}
\newline
\begin{sous-entrée}{pò-pwe}{ⓔpòⓗ1ⓢ1ⓝpò-pwe}
\région{BO}
\begin{glose}
\pfra{hameçon}
\end{glose}
\end{sous-entrée}
\newline
\sens{2}
(\domainesémantique{Préfixes classificateurs numériques})
\begin{glose}
\pfra{fruit (1, 2, etc.)}
\end{glose}
\newline
\begin{exemple}
\textbf{\pnua{po-xè pò-mãã}}
\pfra{une mangue}
\end{exemple}
\newline
\begin{exemple}
\région{GO}
\textbf{\pnua{pwò-kò, pò-kò pò-mãã}}
\pfra{trois, quatre mangues}
\end{exemple}
\newline
\étymologie{
\langue{POc}
\étymon{*pua(q)}}
\end{entrée}

\begin{entrée}{pò}{2}{ⓔpòⓗ2}
\région{GOs}
\variante{%
pòl
\région{BO}, 
pul
\région{PA}}
(\domainesémantique{Noms des plantes})
\classe{nom}
\begin{glose}
\pfra{fougère (petite)}
\end{glose}
\nomscientifique{Pteridium aquilinum}
\end{entrée}

\begin{entrée}{pò}{3}{ⓔpòⓗ3}
\région{GOs PA BO}
\variante{%
pwò
\région{BO}}
(\domainesémantique{Quantificateurs})
\classe{QNT ; atténuatif}
\begin{glose}
\pfra{peu (un)}
\end{glose}
\newline
\begin{exemple}
\région{GOs}
\textbf{\pnua{nu pò thûã-çö}}
\pfra{je t'ai un peu menti}
\end{exemple}
\newline
\begin{exemple}
\région{GOs}
\textbf{\pnua{pò na-mi}}
\pfra{donne un peu}
\end{exemple}
\newline
\begin{exemple}
\textbf{\pnua{nu pò thûã-yu}}
\pfra{je t'ai un peu menti}
\end{exemple}
\newline
\begin{exemple}
\textbf{\pnua{e pò ẽnõ nai nu}}
\pfra{il est un peu plus jeune que moi}
\end{exemple}
\end{entrée}

\begin{entrée}{pò}{4}{ⓔpòⓗ4}
\région{GOs}
(\domainesémantique{Poissons})
\classe{nom}
\begin{glose}
\pfra{poisson "ruban"}
\end{glose}
\nomscientifique{Trichiurus lepturus (Trichiuridés)}
\end{entrée}

\begin{entrée}{pô}{}{ⓔpô}
\formephonétique{põ}
\région{GOs}
\variante{%
pôm
\région{PA BO}}
(\domainesémantique{Insectes})
\classe{nom}
\begin{glose}
\pfra{papillon (de nuit, marron et duveteux qui se nourrit de fruit)}
\end{glose}
\newline
\étymologie{
\langue{POc}
\étymon{*mpompoŋ}}
\end{entrée}

\begin{entrée}{pò a-hu-ò}{}{ⓔpò a-hu-ò}
\région{GOs}
\variante{%
pwo a-wò
\région{GOs}}
\classe{v}
\newline
\sens{1}
(\domainesémantique{Verbes de déplacement et moyens de déplacement})
\begin{glose}
\pfra{avancer un peu}
\end{glose}
\newline
\sens{2}
(\domainesémantique{Verbes de mouvement})
\begin{glose}
\pfra{pousser (se) un peu ; faire un peu de place}
\end{glose}
\newline
\note{-(w)ò (directionnel centripète)}{grammaire}{}
\end{entrée}

\begin{entrée}{pò-baa}{}{ⓔpò-baa}
\région{BO}
(\domainesémantique{Richesses, monnaies traditionnelles})
\classe{nom}
\begin{glose}
\pfra{perles de verre [BM]}
\end{glose}
\newline
\note{non verifié}{général}{}
\end{entrée}

\begin{entrée}{pobe}{}{ⓔpobe}
\région{PA BO}
\variante{%
pwobe
\région{BO}}
\classe{v.stat.}
\newline
\sens{1}
(\domainesémantique{Description des objets, formes, consistance, taille})
\begin{glose}
\pfra{petit}
\end{glose}
\newline
\sens{2}
(\domainesémantique{Quantificateurs})
\begin{glose}
\pfra{un peu}
\end{glose}
\newline
\begin{exemple}
\région{PA}
\textbf{\pnua{cooxe-xa pã na popobe}}
\pfra{coupe un peu de pain}
\end{exemple}
\newline
\relationsémantique{Cf.}{\lien{ⓔpopobe}{popobe}}
\glosecourte{un petit peu}
\end{entrée}

\begin{entrée}{pobil}{}{ⓔpobil}
\région{PA BO}
(\domainesémantique{Objets coutumiers})
\classe{nom}
\begin{glose}
\pfra{ceinture de femme finement tressée}
\end{glose}
\newline
\note{sert de monnaie, elle est faite avec les racines debourao et a plus de valeur que 'wepoo'. Elle faisait plusieurs fois le tour du bassinet se portait au-dessus de 'wepoo'. (Dubois ms + Charles )}{glose}{}
\newline
\relationsémantique{Cf.}{\lien{}{thabil [PA]}}
\glosecourte{ceinture de femme (monnaie)}
\end{entrée}

\begin{entrée}{pobo}{}{ⓔpobo}
\région{GOs PA BO}
(\domainesémantique{Verbes de déplacement et moyens de déplacement})
\classe{nom}
\begin{glose}
\pfra{trace (laissée à un endroit par un animal ou une personne qui y a dormi)}
\end{glose}
\newline
\begin{exemple}
\textbf{\pnua{pobo-n}}
\pfra{ses traces}
\end{exemple}
\newline
\relationsémantique{Cf.}{\lien{}{nõbo hele [GOs]}}
\glosecourte{marques, traces de couteau}
\end{entrée}

\begin{entrée}{pobo-poin}{}{ⓔpobo-poin}
\région{GO WE}
(\domainesémantique{Types de champs})
\classe{nom}
\begin{glose}
\pfra{champ ; trace de champ abandonné (Dubois)}
\end{glose}
\newline
\relationsémantique{Cf.}{\lien{}{poin [BO]}}
\glosecourte{champ}
\newline
\note{non vérifié}{général}{}
\end{entrée}

\begin{entrée}{pòbwinõ}{}{ⓔpòbwinõ}
\région{PA BO}
(\domainesémantique{Corps humain})
\classe{nom}
\begin{glose}
\pfra{fesses ; derrière ; postérieur}
\end{glose}
\newline
\begin{exemple}
\textbf{\pnua{pòbwinõ-n}}
\pfra{ses fesses}
\end{exemple}
\end{entrée}

\begin{entrée}{pòbwinõ-wõ}{}{ⓔpòbwinõ-wõ}
\région{GOs}
\variante{%
pòbwinõ-wòny
\région{PA BO}}
(\domainesémantique{Navigation})
\classe{nom}
\begin{glose}
\pfra{poupe}
\end{glose}
\newline
\relationsémantique{Cf.}{\lien{ⓔmura-wõ}{mura-wõ}}
\glosecourte{poupe}
\end{entrée}

\begin{entrée}{pò-bwiri}{}{ⓔpò-bwiri}
\région{GOs}
(\domainesémantique{Instruments})
\classe{nom}
\begin{glose}
\pfra{mors}
\end{glose}
\end{entrée}

\begin{entrée}{pôbwi-we}{}{ⓔpôbwi-we}
\formephonétique{põbwi we}
\région{GOs}
(\domainesémantique{Eau})
\classe{nom}
\begin{glose}
\pfra{lac}
\end{glose}
\end{entrée}

\begin{entrée}{pò-caai}{}{ⓔpò-caai}
\formephonétique{pɔ-caːi}
\région{GOsPA}
(\domainesémantique{Fruits})
\classe{nom}
\begin{glose}
\pfra{pomme canaque}
\end{glose}
\begin{glose}
\pfra{fruit de pomme rose}
\end{glose}
\nomscientifique{Syzygium malaccense (Myrtacées); Eugenia malaccensis}
\nomscientifique{Syzygium jambos (Myrtacées)}
\end{entrée}

\begin{entrée}{pò-ci}{}{ⓔpò-ci}
\formephonétique{pɔ-cɨ}
\région{GOs}
\variante{%
po-cin
\formephonétique{pɔ-tjin}
\région{PA WE}}
(\domainesémantique{Fruits})
\classe{nom}
\begin{glose}
\pfra{papaye}
\end{glose}
\end{entrée}

\begin{entrée}{po-da ?}{}{ⓔpo-da ?}
\région{GO}
(\domainesémantique{Interrogatifs})
\classe{INT}
\begin{glose}
\pfra{pour quoi ?}
\end{glose}
\end{entrée}

\begin{entrée}{podi}{}{ⓔpodi}
\région{GOs PA BO}
\variante{%
pwodi
\région{BO}}
(\domainesémantique{Bananiers et bananes})
\classe{nom}
\begin{glose}
\pfra{bananier ('banane chef')}
\end{glose}
\newline
\note{il est interdit de la cuire sur la braise, elle ne peut être que bouillie}{glose}{}
\nomscientifique{Musa sp.}
\newline
\étymologie{
\langue{POc}
\étymon{*pundi}}
\end{entrée}

\begin{entrée}{pòdòu}{}{ⓔpòdòu}
\région{GOs}
(\domainesémantique{Description des objets, formes, consistance, taille})
\classe{nom}
\begin{glose}
\pfra{objet emballé}
\end{glose}
\end{entrée}

\begin{entrée}{pò-draado}{}{ⓔpò-draado}
\région{GOs}
(\domainesémantique{Corps humain})
\classe{nom}
\begin{glose}
\pfra{iris (oeil) (lit. fruit de Vitex trifoliata)}
\end{glose}
\end{entrée}

\begin{entrée}{pogabe}{}{ⓔpogabe}
\région{GOs BO}
(\domainesémantique{Description des objets, formes, consistance, taille})
\classe{v.stat.}
\begin{glose}
\pfra{mince ; étroit}
\end{glose}
\begin{glose}
\pfra{imperceptible}
\end{glose}
\begin{glose}
\pfra{très petit}
\end{glose}
\end{entrée}

\begin{entrée}{poge kui}{}{ⓔpoge kui}
\région{GOs}
(\domainesémantique{Coutumes, dons coutumiers})
\classe{nom}
\begin{glose}
\pfra{tas d'ignames}
\end{glose}
\end{entrée}

\begin{entrée}{po-hoxè}{}{ⓔpo-hoxè}
\région{GOs}
(\domainesémantique{Quantificateurs})
\classe{QNT}
\begin{glose}
\pfra{moins}
\end{glose}
\end{entrée}

\begin{entrée}{poi}{}{ⓔpoi}
\région{GO}
(\domainesémantique{Modalité, verbes modaux})
\classe{HORT}
\begin{glose}
\pfra{que !}
\end{glose}
\end{entrée}

\begin{entrée}{pòi}{}{ⓔpòi}
\région{GOs PA WEM BO}
\variante{%
pwe
\région{BO}}
(\domainesémantique{Parenté})
\classe{nom}
\begin{glose}
\pfra{enfant (fille/fils) ; enfant de frère et de cousins (homme parlant)}
\end{glose}
\begin{glose}
\pfra{enfant de fils de frère ou soeur du père (= petits cousins, homme parlant)}
\end{glose}
\begin{glose}
\pfra{enfant de fils de frère ou de soeur de mère (homme parlant) ;}
\end{glose}
\begin{glose}
\pfra{enfant de soeur et de cousines (femme parlant)}
\end{glose}
\newline
\begin{sous-entrée}{pòi-nu wha-mã}{ⓔpòiⓝpòi-nu wha-mã}
\begin{glose}
\pfra{mon aîné}
\end{glose}
\end{sous-entrée}
\newline
\begin{sous-entrée}{pòi-nu ẽnõ}{ⓔpòiⓝpòi-nu ẽnõ}
\begin{glose}
\pfra{mon dernier enfant}
\end{glose}
\newline
\begin{exemple}
\région{GOs}
\textbf{\pnua{pòi-je}}
\pfra{son enfant}
\end{exemple}
\newline
\begin{exemple}
\région{WEM}
\textbf{\pnua{pòi-m}}
\pfra{ton enfant}
\end{exemple}
\newline
\begin{exemple}
\région{PA}
\textbf{\pnua{pòi-ã vaaci}}
\pfra{notre bétail}
\end{exemple}
\newline
\begin{exemple}
\région{BO}
\textbf{\pnua{gele-xa pwe-m/poi-m ?}}
\pfra{as-tu des enfants ?}
\end{exemple}
\newline
\relationsémantique{Cf.}{\lien{ⓔpööni}{pööni}}
\glosecourte{enfant de soeur (homme parlant)}
\newline
\relationsémantique{Cf.}{\lien{}{hê-kòlò}}
\glosecourte{enfant de frère et de cousin (femme parlant)}
\end{sous-entrée}
\end{entrée}

\begin{entrée}{poka}{}{ⓔpoka}
\région{WE PA BO}
(\domainesémantique{Mammifères})
\classe{nom}
\begin{glose}
\pfra{cochon ; porc}
\end{glose}
\end{entrée}

\begin{entrée}{pò-kênii}{}{ⓔpò-kênii}
(\domainesémantique{Vêtements, parure})
\classe{nom}
\begin{glose}
\pfra{boucle d'oreilles (lit. fruit des oreilles)}
\end{glose}
\end{entrée}

\begin{entrée}{po-ki}{}{ⓔpo-ki}
\région{PA BO}
(\domainesémantique{Fonctions naturelles humaines})
\classe{v}
\begin{glose}
\pfra{enceinte (être) (lit. po kiò 'petit ventre')}
\end{glose}
\end{entrée}

\begin{entrée}{pò-kiga}{}{ⓔpò-kiga}
\formephonétique{pɔ-kiŋga}
\région{GOs}
(\domainesémantique{Fonctions naturelles humaines})
\classe{v}
\begin{glose}
\pfra{rire un peu}
\end{glose}
\end{entrée}

\begin{entrée}{po-kiò}{}{ⓔpo-kiò}
\région{GOs PA BO}
(\domainesémantique{Caractéristiques et propriétés des personnes})
\classe{v.stat.}
\begin{glose}
\pfra{ventru ; corpulent}
\end{glose}
\end{entrée}

\begin{entrée}{pò-ko}{}{ⓔpò-ko}
\région{GOs}
(\domainesémantique{Numéraux cardinaux})
\classe{NUM}
\begin{glose}
\pfra{trois}
\end{glose}
\end{entrée}

\begin{entrée}{pò-kô-hu-ò}{}{ⓔpò-kô-hu-ò}
\région{GOs}
\variante{%
pwo kô-wò
\région{GOs}}
(\domainesémantique{Mouvements ou actions faits avec le corps, les bras, les mains, les pieds})
\classe{v}
\begin{glose}
\pfra{pousser (se) un peu ; faire un peu de place (quand on est couché)}
\end{glose}
\end{entrée}

\begin{entrée}{pòko-kabu}{}{ⓔpòko-kabu}
\région{GOs}
\variante{%
pòko-xabu
\région{GO(s)}, 
poko-kabun
\région{PA BO}}
(\domainesémantique{Jours})
\classe{nom}
\begin{glose}
\pfra{jeudi (lit. 3° sacré)}
\end{glose}
\end{entrée}

\begin{entrée}{pò-ku-hu-ò}{}{ⓔpò-ku-hu-ò}
\région{GOs}
\variante{%
pwo ku-wò
\région{GOs}}
(\domainesémantique{Mouvements ou actions faits avec le corps, les bras, les mains, les pieds})
\classe{v}
\begin{glose}
\pfra{pousser (se) un peu (debout) ; faire un peu de place (quand on est debout)}
\end{glose}
\end{entrée}

\begin{entrée}{pola}{}{ⓔpola}
\région{PA BO}
\variante{%
pwola
\région{BO}, 
thrale
\région{GO(s)}}
(\domainesémantique{Nattes})
\classe{nom}
\begin{glose}
\pfra{natte (faite avec deux demi palmes de cocotier pour couvrir le toit)}
\end{glose}
\newline
\étymologie{
\langue{POc}
\étymon{*mpola}}
\end{entrée}

\begin{entrée}{pò-maxa}{}{ⓔpò-maxa}
\région{GOs}
(\domainesémantique{Phénomènes atmosphériques et naturels})
\classe{nom}
\begin{glose}
\pfra{buée (fruit de la saison froide)}
\end{glose}
\end{entrée}

\begin{entrée}{pòmi-nò}{}{ⓔpòmi-nò}
\région{GOs}
\variante{%
pòbwi-nò-n
\région{BO}}
(\domainesémantique{Corps humain})
\classe{nom}
\begin{glose}
\pfra{derrière ; postérieur}
\end{glose}
\begin{glose}
\pfra{fesses}
\end{glose}
\newline
\begin{exemple}
\région{BO}
\textbf{\pnua{pòbwinòò-n}}
\pfra{ses fesses}
\end{exemple}
\end{entrée}

\begin{entrée}{pòminõ pwamwa}{}{ⓔpòminõ pwamwa}
\région{GOs}
(\domainesémantique{Directions})
\classe{nom}
\begin{glose}
\pfra{nord du pays}
\end{glose}
\end{entrée}

\begin{entrée}{pomitee}{}{ⓔpomitee}
\région{GOs}
\variante{%
'pomtee
\région{GO(s)}}
(\domainesémantique{Aliments, alimentation})
\classe{nom}
\begin{glose}
\pfra{pomme de terre}
\end{glose}
\newline
\emprunt{pomme de terre (FR)}
\end{entrée}

\begin{entrée}{pomõ}{}{ⓔpomõ}
\région{GOs PA BO}
\variante{%
pwòmò
\région{WE PA}}
\classe{n.LOC (forme POSS de pwamwa)}
\newline
\sens{1}
(\domainesémantique{Types de champs})
\begin{glose}
\pfra{champ}
\end{glose}
\newline
\begin{exemple}
\région{GOs}
\textbf{\pnua{kê-pomõ-nu}}
\pfra{mon champ}
\end{exemple}
\newline
\sens{2}
(\domainesémantique{Habitat})
\begin{glose}
\pfra{pays}
\end{glose}
\newline
\begin{exemple}
\région{GOs}
\textbf{\pnua{pomõ-nu}}
\pfra{mon pays}
\end{exemple}
\newline
\relationsémantique{Cf.}{\lien{ⓔpwamwa}{pwamwa}}
\glosecourte{pays}
\newline
\sens{3}
(\domainesémantique{Noms locatifs})
\begin{glose}
\pfra{chez}
\end{glose}
\newline
\begin{exemple}
\région{BO}
\textbf{\pnua{nu a-da pomwõ-ny}}
\pfra{je monte chez moi}
\end{exemple}
\newline
\begin{exemple}
\région{PA}
\textbf{\pnua{pomõ-n}}
\pfra{chez lui (lit. demeure-sa)}
\end{exemple}
\end{entrée}

\begin{entrée}{pomõ-da}{}{ⓔpomõ-da}
\région{GOs}
\variante{%
pomwa-da, poma-da
\région{BO}}
(\domainesémantique{Topographie})
\classe{DIR}
\begin{glose}
\pfra{en amont ; en haut}
\end{glose}
\end{entrée}

\begin{entrée}{pomõ-du}{}{ⓔpomõ-du}
\région{GOs}
\variante{%
pomwõdu, pomõ-du
\région{BO}}
(\domainesémantique{Topographie})
\classe{DIR}
\begin{glose}
\pfra{en aval ; en bas}
\end{glose}
\end{entrée}

\begin{entrée}{pomõ-li}{}{ⓔpomõ-li}
\région{GOs BO}
(\domainesémantique{Noms locatifs})
\classe{DIR}
\begin{glose}
\pfra{de l'autre côté ; au-delà}
\end{glose}
\newline
\begin{exemple}
\région{GOs}
\textbf{\pnua{a-e pomõ-li !}}
\pfra{va de l'autre côté}
\end{exemple}
\newline
\begin{exemple}
\région{BO}
\textbf{\pnua{a mwa-e pomõ-li !}}
\pfra{va de l'autre côté}
\end{exemple}
\end{entrée}

\begin{entrée}{pomõ-mi}{}{ⓔpomõ-mi}
\région{GOs}
(\domainesémantique{Directionnels})
\classe{LOC}
\begin{glose}
\pfra{par ici}
\end{glose}
\newline
\begin{exemple}
\textbf{\pnua{pomõ-mi ne}}
\pfra{approche-toi vers ici}
\end{exemple}
\newline
\begin{exemple}
\textbf{\pnua{pomõ-mi xòlò nu}}
\pfra{approche-toi de moi}
\end{exemple}
\end{entrée}

\begin{entrée}{pomõnim}{}{ⓔpomõnim}
\région{PA}
(\domainesémantique{Eau})
\classe{nom}
\begin{glose}
\pfra{tourbillon (petit et rapide dans l'eau)}
\end{glose}
\newline
\relationsémantique{Cf.}{\lien{ⓔniilöö}{niilöö}}
\glosecourte{grand tourbillon lent}
\end{entrée}

\begin{entrée}{pò-mugo ni hi}{}{ⓔpò-mugo ni hi}
\région{WEMBO}
\variante{%
pò-mugo ne hii-n
\région{BO}}
(\domainesémantique{Corps humain})
\classe{nom}
\begin{glose}
\pfra{biceps}
\end{glose}
\end{entrée}

\begin{entrée}{pò-mugo ni kò}{}{ⓔpò-mugo ni kò}
\région{WEM BO}
\variante{%
pò-mugo ne kòò-n
\région{BO}}
(\domainesémantique{Corps humain})
\classe{nom}
\begin{glose}
\pfra{mollet}
\end{glose}
\newline
\begin{exemple}
\région{BO}
\textbf{\pnua{pò-mugo ni/ne kòò-n}}
\pfra{son mollet}
\end{exemple}
\end{entrée}

\begin{entrée}{põng}{}{ⓔpõng}
\formephonétique{pɔ̃ŋ}
\région{BO}
(\domainesémantique{Cocotiers})
\classe{nom}
\begin{glose}
\pfra{gaine de l'inflorescence du cocotier [Corne]}
\end{glose}
\newline
\note{non vérifié}{général}{}
\end{entrée}

\begin{entrée}{ponga}{}{ⓔponga}
\formephonétique{poŋa}
\région{GOs}
(\domainesémantique{Prépositions})
\classe{PREP.BENEF}
\begin{glose}
\pfra{pour}
\end{glose}
\newline
\begin{exemple}
\textbf{\pnua{ponga-jo}}
\pfra{pour toi}
\end{exemple}
\end{entrée}

\begin{entrée}{ponga da ?}{}{ⓔponga da ?}
\formephonétique{poŋa da}
\région{GOs BO}
\variante{%
puxã da ?
\région{GO(s)}}
(\domainesémantique{Interrogatifs})
\classe{INT}
\begin{glose}
\pfra{pourquoi faire ? ; à quoi sert ?}
\end{glose}
\newline
\begin{exemple}
\textbf{\pnua{ponga da ?}}
\pfra{pourquoi faire ?}
\end{exemple}
\end{entrée}

\begin{entrée}{pò-ni}{}{ⓔpò-ni}
\formephonétique{pɔɳi, pɔni}
\région{GO}
(\domainesémantique{Numéraux cardinaux})
\classe{NUM}
\begin{glose}
\pfra{cinq}
\end{glose}
\newline
\begin{sous-entrée}{ponimaxe}{ⓔpò-niⓝponimaxe}
\begin{glose}
\pfra{six}
\end{glose}
\end{sous-entrée}
\newline
\begin{sous-entrée}{ponimadru}{ⓔpò-niⓝponimadru}
\begin{glose}
\pfra{sept}
\end{glose}
\end{sous-entrée}
\newline
\begin{sous-entrée}{ponimagò}{ⓔpò-niⓝponimagò}
\begin{glose}
\pfra{huit}
\end{glose}
\end{sous-entrée}
\newline
\begin{sous-entrée}{ponimaba}{ⓔpò-niⓝponimaba}
\begin{glose}
\pfra{neuf}
\end{glose}
\end{sous-entrée}
\newline
\begin{sous-entrée}{ponita ?}{ⓔpò-niⓝponita ?}
\begin{glose}
\pfra{combien ?}
\end{glose}
\end{sous-entrée}
\end{entrée}

\begin{entrée}{pòni-kabu}{}{ⓔpòni-kabu}
\région{GOs}
\variante{%
pòni-kabun
\région{PA BO}}
(\domainesémantique{Jours})
\classe{nom}
\begin{glose}
\pfra{mardi (lit. 5 ème [jour] sacré)}
\end{glose}
\end{entrée}

\begin{entrée}{pò-ni-ma-ba}{}{ⓔpò-ni-ma-ba}
\région{GOs}
(\domainesémantique{Numéraux cardinaux})
\classe{NUM}
\begin{glose}
\pfra{neuf}
\end{glose}
\end{entrée}

\begin{entrée}{pò-ni-ma-dru}{}{ⓔpò-ni-ma-dru}
\région{GOs}
(\domainesémantique{Numéraux cardinaux})
\classe{NUM}
\begin{glose}
\pfra{sept}
\end{glose}
\end{entrée}

\begin{entrée}{pò-ni-ma-gò}{}{ⓔpò-ni-ma-gò}
\région{GOs}
(\domainesémantique{Numéraux cardinaux})
\classe{NUM}
\begin{glose}
\pfra{huit}
\end{glose}
\end{entrée}

\begin{entrée}{pòni-ma-wee}{}{ⓔpòni-ma-wee}
\région{GOs}
\variante{%
pòdi-ma-pwèèl
\région{BO [Corne]}}
(\domainesémantique{Corps humain})
\classe{nom}
\begin{glose}
\pfra{cheville ; malléole de la cheville ? (Haudricourt)}
\end{glose}
\end{entrée}

\begin{entrée}{pò-ni-ma-xe}{}{ⓔpò-ni-ma-xe}
\région{GOs}
(\domainesémantique{Numéraux cardinaux})
\classe{NUM}
\begin{glose}
\pfra{six}
\end{glose}
\end{entrée}

\begin{entrée}{pò-niza ?}{}{ⓔpò-niza ?}
\formephonétique{pɔ-ɳiða}
\région{GOs}
\variante{%
pònita?pònira?
\région{PA BO}, 
pwònira?
\région{BO}}
(\domainesémantique{Interrogatifs})
\classe{INT}
\begin{glose}
\pfra{combien ? (inanimés)}
\end{glose}
\newline
\begin{exemple}
\région{GO}
\textbf{\pnua{pò-niza hino-a ?}}
\pfra{quelle heure est-il ?}
\end{exemple}
\newline
\begin{exemple}
\région{BO}
\textbf{\pnua{pwò-nira hino-al ?}}
\pfra{quelle heure est-il ?}
\end{exemple}
\newline
\begin{exemple}
\région{GO}
\textbf{\pnua{pò-niza kau çö ?}}
\pfra{quel âge as-tu ?}
\end{exemple}
\newline
\begin{exemple}
\région{GO}
\textbf{\pnua{we-niza ka ?}}
\pfra{combien d'années ?}
\end{exemple}
\newline
\étymologie{
\langue{POc}
\étymon{*pinsa, *pija}}
\end{entrée}

\begin{entrée}{pòńõ}{}{ⓔpòńõ}
\formephonétique{pɔnɔ̃}
\région{GOs}
(\domainesémantique{Quantificateurs})
\classe{QNT}
\begin{glose}
\pfra{peu ; un peu (quantité)}
\end{glose}
\newline
\begin{exemple}
\région{GO}
\textbf{\pnua{îî cèè-çö xo pònõ}}
\pfra{sers ta nourriture (féculents) en petite quantité (n'en prends pas trop)}
\end{exemple}
\newline
\begin{exemple}
\textbf{\pnua{mwêêno pònõ vwo la za mã !}}
\pfra{il s'en est fallu de peu qu'il ne meurent !}
\end{exemple}
\newline
\begin{exemple}
\région{GO}
\textbf{\pnua{cooxe phwalawa na pònõ}}
\pfra{coupe un peu de pain}
\end{exemple}
\newline
\begin{sous-entrée}{mwêêno pònõ vwo}{ⓔpòńõⓝmwêêno pònõ vwo}
\begin{glose}
\pfra{s'en falloir de peu que}
\end{glose}
\newline
\relationsémantique{Cf.}{\lien{}{pobe, popobe [PA]}}
\glosecourte{un peu}
\end{sous-entrée}
\end{entrée}

\begin{entrée}{po-nõgò}{}{ⓔpo-nõgò}
\formephonétique{po-ɳɔ̃ŋgɔ}
\région{GOs WEM}
\variante{%
po-nogo
\région{PA}, 
nogo
\région{BO}}
(\domainesémantique{Eau})
\classe{nom}
\begin{glose}
\pfra{creek ; rivière ; ruisseau}
\end{glose}
\end{entrée}

\begin{entrée}{pònu}{}{ⓔpònu}
\formephonétique{pɔɳu}
\région{GOs BO}
\variante{%
pwònu
\région{BO}}
\classe{v}
\newline
\sens{1}
(\domainesémantique{Description des objets, formes, consistance, taille})
\begin{glose}
\pfra{plein (être)}
\end{glose}
\newline
\begin{exemple}
\textbf{\pnua{e pònu xo we}}
\pfra{c'est plein d'eau}
\end{exemple}
\newline
\sens{2}
(\domainesémantique{Caractéristiques et propriétés des animaux})
\begin{glose}
\pfra{plein (crabe)}
\end{glose}
\newline
\étymologie{
\langue{POc}
\étymon{*ponuq}
\glosecourte{full}}
\end{entrée}

\begin{entrée}{pô-nu}{}{ⓔpô-nu}
\formephonétique{põ-ɳu}
\région{GOs}
\variante{%
pon
\région{BO}}
(\domainesémantique{Cocotiers})
\classe{nom}
\begin{glose}
\pfra{rachis de coco}
\end{glose}
\end{entrée}

\begin{entrée}{pònum}{}{ⓔpònum}
\région{PA BO}
\variante{%
pònèn, pwanèn
\région{BO}}
(\domainesémantique{Description des objets, formes, consistance, taille})
\classe{v.stat.}
\begin{glose}
\pfra{court ; petit}
\end{glose}
\newline
\begin{exemple}
\région{PA}
\textbf{\pnua{i pònum ço hangai}}
\pfra{il est petit et gros}
\end{exemple}
\newline
\begin{exemple}
\région{BO}
\textbf{\pnua{pònum a mada}}
\pfra{le tissu est court}
\end{exemple}
\newline
\begin{sous-entrée}{pa-pònume hôxa ce}{ⓔpònumⓝpa-pònume hôxa ce}
\begin{glose}
\pfra{raccourcir un bout de bois}
\end{glose}
\end{sous-entrée}
\end{entrée}

\begin{entrée}{ponyãã}{}{ⓔponyãã}
\région{GOs}
\variante{%
ponyam
\région{PA BO}}
(\domainesémantique{Caractéristiques et propriétés des personnes})
\classe{v.stat.}
\begin{glose}
\pfra{aimable ; doux ; gentil}
\end{glose}
\newline
\begin{exemple}
\région{GOs}
\textbf{\pnua{e ponyãã}}
\pfra{il est doux}
\end{exemple}
\newline
\begin{exemple}
\région{PA}
\textbf{\pnua{i a-ponyam}}
\pfra{il est gentil}
\end{exemple}
\end{entrée}

\begin{entrée}{po nye}{}{ⓔpo nye}
\région{GO}
(\domainesémantique{Conjonction})
\classe{CNJ}
\begin{glose}
\pfra{du fait que}
\end{glose}
\end{entrée}

\begin{entrée}{pòò}{}{ⓔpòò}
\région{GOsPA BO}
(\domainesémantique{Arbre})
\classe{nom}
\begin{glose}
\pfra{bourao (générique)}
\end{glose}
\nomscientifique{Hibiscus tiliaceus L.}
\newline
\étymologie{
\langue{POc}
\étymon{*paRu}}
\end{entrée}

\begin{entrée}{pòòdra}{}{ⓔpòòdra}
\région{GOs}
\variante{%
pòòda
\région{BO}, 
pòòdaang
\région{BO}}
(\domainesémantique{Arbre})
\classe{nom}
\begin{glose}
\pfra{bourao (de bord de mer, à écorce comestible, nourriture de disette)}
\end{glose}
\newline
\note{son écorce sert à fabriquer les jupes de femme et les cordages.}{glose}{}
\nomscientifique{Hibiscus tiliaceus (Malvacées)}
\end{entrée}

\begin{entrée}{pòò-hovwo}{}{ⓔpòò-hovwo}
\région{GOs}
(\domainesémantique{Arbre})
\classe{nom}
\begin{glose}
\pfra{bourao (sorte à écorce comestible)}
\end{glose}
\end{entrée}

\begin{entrée}{pööni}{}{ⓔpööni}
\région{GOs}
\variante{%
puuni
\région{PA BO}}
(\domainesémantique{Parenté})
\classe{nom}
\begin{glose}
\pfra{oncle maternel}
\end{glose}
\begin{glose}
\pfra{cousins de mère}
\end{glose}
\begin{glose}
\pfra{époux de soeur de père}
\end{glose}
\begin{glose}
\pfra{enfant de soeur (homme parlant)}
\end{glose}
\begin{glose}
\pfra{enfant de fille de frère ou soeur du père (= petits cousins)}
\end{glose}
\begin{glose}
\pfra{enfant de fille de frère ou soeur de mère}
\end{glose}
\newline
\begin{exemple}
\région{GO}
\textbf{\pnua{pööni-je}}
\pfra{son neveu, sa nièce (utérin)}
\end{exemple}
\newline
\begin{exemple}
\région{BO}
\textbf{\pnua{pööni-n}}
\pfra{son neveu, sa nièce (utérin)}
\end{exemple}
\newline
\begin{exemple}
\textbf{\pnua{pöpö}}
\pfra{tonton (utérin) (langage des enfants)}
\end{exemple}
\newline
\relationsémantique{Cf.}{\lien{ⓔpòi}{pòi}}
\glosecourte{enfant de frère et de cousins (homme parlant); enfant de soeur et de cousines (femme parlant)}
\end{entrée}

\begin{entrée}{poo-yaai}{}{ⓔpoo-yaai}
\région{PA BO}
(\domainesémantique{Feu : objets et actions liés au feu})
\classe{nom}
\begin{glose}
\pfra{étincelle (du feu)}
\end{glose}
\end{entrée}

\begin{entrée}{pò-pa}{}{ⓔpò-pa}
\région{GO}
(\domainesémantique{Numéraux cardinaux})
\classe{NUM}
\begin{glose}
\pfra{quatre}
\end{glose}
\end{entrée}

\begin{entrée}{pò-pa-kabu}{}{ⓔpò-pa-kabu}
\région{GOs}
\variante{%
pò-pa-xabu
\région{GO(s)}, 
pò-pa-kabun
\région{PA BO}}
(\domainesémantique{Jours})
\classe{nom}
\begin{glose}
\pfra{mercredi (lit. le 4°([jour] sacré)}
\end{glose}
\end{entrée}

\begin{entrée}{pò-pa-kudi}{}{ⓔpò-pa-kudi}
\région{GOs}
(\domainesémantique{Configuration des objets})
\classe{v}
\begin{glose}
\pfra{carré (lit. 4 coins)}
\end{glose}
\end{entrée}

\begin{entrée}{popobe}{}{ⓔpopobe}
\région{PA BO [Dubois]}
\variante{%
pobe
\région{WE}}
\classe{v.stat.}
\newline
\sens{1}
(\domainesémantique{Description des objets, formes, consistance, taille})
\begin{glose}
\pfra{petit ; court}
\end{glose}
\begin{glose}
\pfra{menu}
\end{glose}
\newline
\sens{2}
(\domainesémantique{Quantificateurs})
\begin{glose}
\pfra{un peu}
\end{glose}
\newline
\begin{exemple}
\région{PA}
\textbf{\pnua{cooxe-xa pã na popobe}}
\pfra{coupe un peu de pain}
\end{exemple}
\end{entrée}

\begin{entrée}{pò-pwa}{}{ⓔpò-pwa}
\région{GOs}
\variante{%
pòò-pwal
\région{PA}}
(\domainesémantique{Phénomènes atmosphériques et naturels})
\classe{nom}
\begin{glose}
\pfra{goutte de pluie}
\end{glose}
\end{entrée}

\begin{entrée}{pò-pwaale}{}{ⓔpò-pwaale}
\région{GOs}
\variante{%
pò-vwale
\région{GO(s)}}
(\domainesémantique{Noms des plantes})
\classe{nom}
\begin{glose}
\pfra{maïs (épi de)}
\end{glose}
\newline
\begin{sous-entrée}{kò-pò-pwaale}{ⓔpò-pwaaleⓝkò-pò-pwaale}
\begin{glose}
\pfra{tige de maïs, maïs (plante)}
\end{glose}
\end{sous-entrée}
\end{entrée}

\begin{entrée}{pò-pwe}{}{ⓔpò-pwe}
\région{GOs PA BO}
\variante{%
pwò-pwe
\région{GO(s) BO}}
(\domainesémantique{Pêche})
\classe{nom}
\begin{glose}
\pfra{hameçon}
\end{glose}
\end{entrée}

\begin{entrée}{po-ra ?}{}{ⓔpo-ra ?}
\région{BO PA}
\variante{%
po-za?
\région{GO}}
(\domainesémantique{Interrogatifs})
\classe{v}
\begin{glose}
\pfra{quoi faire ? (lit. faire quoi ?)}
\end{glose}
\newline
\begin{exemple}
\textbf{\pnua{la po-ra ?}}
\pfra{qu'ont-ils fait ?}
\end{exemple}
\end{entrée}

\begin{entrée}{pò sitrô}{}{ⓔpò sitrô}
\région{GOs}
(\domainesémantique{Fruits})
\classe{nom}
\begin{glose}
\pfra{citron (lit. fruit citron)}
\end{glose}
\newline
\emprunt{citron (FR)}
\end{entrée}

\begin{entrée}{pò-thala}{}{ⓔpò-thala}
\région{GOs}
(\domainesémantique{Description des objets, formes, consistance, taille})
\classe{v}
\begin{glose}
\pfra{entrouvert}
\end{glose}
\newline
\begin{exemple}
\textbf{\pnua{po-thala pweemwa}}
\pfra{la porte est entrouvert}
\end{exemple}
\end{entrée}

\begin{entrée}{pò-thi}{}{ⓔpò-thi}
\région{GOs}
\variante{%
pò-ti-n, pwò-thi-n
\région{BO}}
(\domainesémantique{Corps humain})
\classe{nom}
\begin{glose}
\pfra{mamelon du sein}
\end{glose}
\newline
\begin{exemple}
\région{GO}
\textbf{\pnua{pò-thi nu}}
\pfra{mon sein}
\end{exemple}
\end{entrée}

\begin{entrée}{pò-tree-hu-ò}{}{ⓔpò-tree-hu-ò}
\région{GOs}
\variante{%
pwo tree-wò
\région{GOs}}
(\domainesémantique{Verbes de mouvement
, Mouvements ou actions faits avec le corps, les bras, les mains, les pieds})
\classe{v}
\begin{glose}
\pfra{pousser (se) un peu}
\end{glose}
\begin{glose}
\pfra{faire un peu de place (quand on est assis)}
\end{glose}
\end{entrée}

\begin{entrée}{pò-tru}{}{ⓔpò-tru}
\formephonétique{pɔ-ʈu}
\région{GOs}
\variante{%
pò-ru
\région{PA}, 
pò-du, pò-ru
\région{BO}}
(\domainesémantique{Numéraux cardinaux})
\classe{NUM}
\begin{glose}
\pfra{deux (générique)}
\end{glose}
\newline
\begin{sous-entrée}{ba-poru}{ⓔpò-truⓝba-poru}
\begin{glose}
\pfra{deuxième}
\end{glose}
\end{sous-entrée}
\end{entrée}

\begin{entrée}{pò-tru kabu}{}{ⓔpò-tru kabu}
\formephonétique{pɔ-ʈu kambu}
\région{GOs}
\variante{%
po-ru-kabun
\région{PA BO}, 
bo-hode
\région{PA}}
(\domainesémantique{Jours})
\classe{nom}
\begin{glose}
\pfra{vendredi(lit. jour-jeûne)}
\end{glose}
\end{entrée}

\begin{entrée}{po-vhaa}{}{ⓔpo-vhaa}
\région{GOs}
(\domainesémantique{Discours, échanges verbaux})
\classe{v}
\begin{glose}
\pfra{parler doucement (lit. un peu parler)}
\end{glose}
\end{entrée}

\begin{entrée}{pòvwònò}{}{ⓔpòvwònò}
\région{GOs}
(\domainesémantique{Description des objets, formes, consistance, taille})
\classe{v.stat.}
\begin{glose}
\pfra{petit (épaisseur ; par opposition à gros)}
\end{glose}
\newline
\note{réduplication}{général}{}
\end{entrée}

\begin{entrée}{povwonû}{}{ⓔpovwonû}
\région{GOs}
(\domainesémantique{Description des objets, formes, consistance, taille})
\classe{v.stat.}
\begin{glose}
\pfra{court (taille, hauteur)}
\end{glose}
\newline
\begin{exemple}
\région{GOs}
\textbf{\pnua{e povwonû nai pòi-nu}}
\pfra{il est plus petit que mon enfant}
\end{exemple}
\newline
\begin{exemple}
\textbf{\pnua{nu povwonû thuã nai jo}}
\pfra{je suis un tout petit peu plus petit que toi}
\end{exemple}
\newline
\relationsémantique{Ant.}{\lien{}{waa [GOs]}}
\glosecourte{grand}
\end{entrée}

\begin{entrée}{pòvwòtru}{}{ⓔpòvwòtru}
\région{GOs}
\variante{%
pwòvwòtru
\région{GO(s)}, 
pòpòru
\région{PA}}
(\domainesémantique{Fonctions intellectuelles})
\classe{v}
\begin{glose}
\pfra{douter; hésiter}
\end{glose}
\newline
\begin{exemple}
\région{PA}
\textbf{\pnua{la pòpòru}}
\pfra{ils ont hésité}
\end{exemple}
\newline
\note{réduplication de pò-tru "deux" > pò-pò-tru 'être entre deux'}{général}{}
\end{entrée}

\begin{entrée}{pôwe}{}{ⓔpôwe}
\formephonétique{põwe}
\région{GOs}
(\domainesémantique{Eau})
\classe{v}
\begin{glose}
\pfra{mouillé (par la pluie, la rosée)}
\end{glose}
\end{entrée}

\begin{entrée}{poweede}{}{ⓔpoweede}
\région{GOs PA BO}
(\domainesémantique{Verbes de mouvement})
\classe{v}
\begin{glose}
\pfra{retourner (verre, seau, etc.)}
\end{glose}
\begin{glose}
\pfra{tourner (page)}
\end{glose}
\begin{glose}
\pfra{changer; corriger [BO]}
\end{glose}
\end{entrée}

\begin{entrée}{pò-wha}{}{ⓔpò-wha}
\région{GOs}
(\domainesémantique{Ignames})
\classe{nom}
\begin{glose}
\pfra{igname (ressemble au fruit du figuier)}
\end{glose}
\end{entrée}

\begin{entrée}{poxa}{1}{ⓔpoxaⓗ1}
\région{GOs PA GO}
\variante{%
poga
\région{PA}}
\classe{nom}
\newline
\sens{1}
(\domainesémantique{Caractéristiques et propriétés des personnes})
\begin{glose}
\pfra{jeune}
\end{glose}
\newline
\begin{sous-entrée}{poxa kò}{ⓔpoxaⓗ1ⓢ1ⓝpoxa kò}
\région{GOs}
\begin{glose}
\pfra{bâtard (lit. enfant de la forêt)}
\end{glose}
\end{sous-entrée}
\newline
\begin{sous-entrée}{poxa no mae}{ⓔpoxaⓗ1ⓢ1ⓝpoxa no mae}
\région{GOs}
\begin{glose}
\pfra{bâtard (lit. enfant dans la paille)}
\end{glose}
\end{sous-entrée}
\newline
\sens{2}
(\domainesémantique{Caractéristiques et propriétés des animaux})
\begin{glose}
\pfra{petit (le petit d'un animal)}
\end{glose}
\newline
\begin{sous-entrée}{poxa ko}{ⓔpoxaⓗ1ⓢ2ⓝpoxa ko}
\begin{glose}
\pfra{poussin}
\end{glose}
\end{sous-entrée}
\newline
\begin{sous-entrée}{poxa poxa}{ⓔpoxaⓗ1ⓢ2ⓝpoxa poxa}
\begin{glose}
\pfra{porcelet}
\end{glose}
\end{sous-entrée}
\newline
\sens{3}
(\domainesémantique{Description des objets, formes, consistance, taille})
\newline
\begin{sous-entrée}{poxa mwa}{ⓔpoxaⓗ1ⓢ3ⓝpoxa mwa}
\begin{glose}
\pfra{petite maison}
\end{glose}
\end{sous-entrée}
\newline
\begin{sous-entrée}{poxa wô}{ⓔpoxaⓗ1ⓢ3ⓝpoxa wô}
\begin{glose}
\pfra{petit bateau}
\end{glose}
\newline
\note{souvent abrégé en : po}{grammaire}{}
\end{sous-entrée}
\end{entrée}

\begin{entrée}{poxa}{2}{ⓔpoxaⓗ2}
\région{GOs}
\variante{%
pwaxa
\région{GA}, 
poka
\région{PA}, 
pwaka, pwòka
}
(\domainesémantique{Mammifères})
\classe{nom}
\begin{glose}
\pfra{porc ; cochon}
\end{glose}
\newline
\begin{exemple}
\textbf{\pnua{poxa i nu}}
\pfra{mon cochon}
\end{exemple}
\end{entrée}

\begin{entrée}{poxa aazo}{}{ⓔpoxa aazo}
\région{GOs}
\variante{%
poga aao
\région{BO}}
(\domainesémantique{Organisation sociale})
\classe{nom}
\begin{glose}
\pfra{fils du chef (lit. le petit)}
\end{glose}
\end{entrée}

\begin{entrée}{poxabu}{}{ⓔpoxabu}
\région{BO}
(\domainesémantique{Aliments, alimentation})
\classe{nom}
\begin{glose}
\pfra{gateau de taro [Corne]}
\end{glose}
\newline
\note{non vérifié}{général}{}
\end{entrée}

\begin{entrée}{poxabwa}{}{ⓔpoxabwa}
\région{GOs}
\variante{%
paxâbwa
\région{PA BO}}
(\domainesémantique{Ignames})
\classe{nom}
\begin{glose}
\pfra{igname (grosse et longue)}
\end{glose}
\newline
\note{à racine profonde plantée au centre du billon (Charles)}{glose}{}
\end{entrée}

\begin{entrée}{poxa-du hogo}{}{ⓔpoxa-du hogo}
\région{GOs PA}
(\domainesémantique{Topographie})
\classe{nom}
\begin{glose}
\pfra{ramification dans la montagne}
\end{glose}
\end{entrée}

\begin{entrée}{poxa-he}{}{ⓔpoxa-he}
\région{GOs}
(\domainesémantique{Feu : objets et actions liés au feu})
\classe{nom}
\begin{glose}
\pfra{tisonnier}
\end{glose}
\end{entrée}

\begin{entrée}{poxa Teã}{}{ⓔpoxa Teã}
\région{PA}
\variante{%
poxo Teã
}
(\domainesémantique{Organisation sociale})
\classe{nom}
\begin{glose}
\pfra{petit chef}
\end{glose}
\end{entrée}

\begin{entrée}{poxawèo}{}{ⓔpoxawèo}
\région{BO}
\variante{%
pwawèo
\région{BO}, 
pokaweo
\région{BO}}
(\domainesémantique{Relations et interaction sociales})
\classe{v ; n}
\begin{glose}
\pfra{imiter ; imitation[Corne]}
\end{glose}
\newline
\note{non vérifié}{général}{}
\end{entrée}

\begin{entrée}{poxa-zaaja}{}{ⓔpoxa-zaaja}
\formephonétique{poxa-zaːdja}
\région{GOs}
(\domainesémantique{Poissons})
\classe{nom}
\begin{glose}
\pfra{loche (de grande taille)}
\end{glose}
\end{entrée}

\begin{entrée}{pò-xè}{}{ⓔpò-xè}
\région{GO PA}
(\domainesémantique{Numéraux cardinaux})
\classe{CLF.NUM (objets ronds)}
\begin{glose}
\pfra{un (objet rond) ; un jour}
\end{glose}
\newline
\begin{exemple}
\textbf{\pnua{pò-xè, pò-tru}}
\pfra{un , deux , etc;}
\end{exemple}
\newline
\begin{exemple}
\région{PA}
\textbf{\pnua{li u a pò-xe}}
\pfra{ils partent ensemble}
\end{exemple}
\newline
\begin{exemple}
\région{BO}
\textbf{\pnua{ni tèn xa pò-xe}}
\pfra{un jour}
\end{exemple}
\newline
\begin{exemple}
\textbf{\pnua{ni ka xa pò-xè}}
\pfra{la même année}
\end{exemple}
\newline
\begin{exemple}
\région{GOs}
\textbf{\pnua{li ẽno ni ka xa pò-xè}}
\pfra{ils sont nés la même année}
\end{exemple}
\newline
\begin{exemple}
\région{GOs}
\textbf{\pnua{pò-xè kenii-je}}
\pfra{il n'a qu'une oreille}
\end{exemple}
\end{entrée}

\begin{entrée}{poxee ma}{}{ⓔpoxee ma}
\région{GOs}
(\domainesémantique{Conjonction})
\classe{ADV}
\begin{glose}
\pfra{en revanche ; pourtant}
\end{glose}
\newline
\begin{exemple}
\textbf{\pnua{kavwö li yue li, axe poxee ma li zalae-me}}
\pfra{ils ne les ont pas adoptés, mais en revanche, ils nous ont élevés}
\end{exemple}
\end{entrée}

\begin{entrée}{poxèè na}{}{ⓔpoxèè na}
\formephonétique{pɔɣɛːɳa}
\région{GOs PA}
\variante{%
poxè
\région{PA BO}}
(\domainesémantique{Modalité, verbes modaux})
\classe{MODAL}
\begin{glose}
\pfra{peut-être que}
\end{glose}
\newline
\begin{exemple}
\région{GOs}
\textbf{\pnua{poxè na ezoma uca mõnõ}}
\pfra{il viendra peut-être demain}
\end{exemple}
\newline
\begin{exemple}
\région{PA}
\textbf{\pnua{poxèè na u ruma havha menon}}
\pfra{il viendra peut-être demain}
\end{exemple}
\newline
\begin{exemple}
\région{PA}
\textbf{\pnua{novwo na i havha na poxèè kòi-nu, ye yu thala mwa}}
\pfra{s'il vient demain quand je serai peut-être absent, alors ouvre la maison}
\end{exemple}
\newline
\begin{exemple}
\région{PA}
\textbf{\pnua{poxèè na e kha-phe xo Kaavo}}
\pfra{peut-être que Kaavo l'a pris}
\end{exemple}
\newline
\begin{exemple}
\textbf{\pnua{poxè na e}}
\pfra{c'est peut-être ainsi/ ça}
\end{exemple}
\newline
\begin{exemple}
\textbf{\pnua{poxè na kavwö trõõne}}
\pfra{il n'a peut-être pas entendu}
\end{exemple}
\end{entrée}

\begin{entrée}{po-xe kabu}{}{ⓔpo-xe kabu}
\région{GOs}
\variante{%
po-xe kabun
\région{BO}, 
cavato
\région{PA BO}}
(\domainesémantique{Jours})
\classe{nom}
\begin{glose}
\pfra{samedi}
\end{glose}
\end{entrée}

\begin{entrée}{pò-xè ńõ}{}{ⓔpò-xè ńõ}
\formephonétique{pɔɣɛ nɔ}
\région{GOs}
(\domainesémantique{Quantificateurs})
\classe{RESTR}
\begin{glose}
\pfra{un seul(ement)}
\end{glose}
\newline
\begin{exemple}
\région{GOs}
\textbf{\pnua{nu nõõle pò-xè ńõ wõ}}
\pfra{j'ai vu un seul bateau}
\end{exemple}
\end{entrée}

\begin{entrée}{poxi}{}{ⓔpoxi}
\région{GOs}
\variante{%
poki
\région{PA}}
(\domainesémantique{Fonctions naturelles humaines})
\classe{v}
\begin{glose}
\pfra{enceinte (être)}
\end{glose}
\newline
\begin{exemple}
\textbf{\pnua{e poxi}}
\pfra{elle est enceinte}
\end{exemple}
\newline
\relationsémantique{Cf.}{\lien{ⓔpuⓗ1}{pu}}
\glosecourte{enfler ; gonflé ; grossir}
\end{entrée}

\begin{entrée}{po za ?}{}{ⓔpo za ?}
\région{GOs}
\variante{%
po ra ?
\région{PA}}
(\domainesémantique{Interrogatifs})
\classe{v}
\begin{glose}
\pfra{que fais-tu ?}
\end{glose}
\newline
\begin{exemple}
\textbf{\pnua{çö po za ?}}
\pfra{que fais-tu ?}
\end{exemple}
\end{entrée}

\begin{entrée}{po-za ?}{}{ⓔpo-za ?}
\région{GOs}
\variante{%
po-ra?
\région{PA BO}}
(\domainesémantique{Interrogatifs})
\classe{CNJ}
\begin{glose}
\pfra{faire comment ?}
\end{glose}
\end{entrée}

\begin{entrée}{pò-zaalo}{}{ⓔpò-zaalo}
\région{GOs}
(\domainesémantique{Fruits})
\classe{nom}
\begin{glose}
\pfra{fruit du "gommier"}
\end{glose}
\end{entrée}

\begin{entrée}{pozo}{}{ⓔpozo}
\région{GOs}
(\domainesémantique{Vêtements, parure})
\classe{nom}
\begin{glose}
\pfra{bague}
\end{glose}
\newline
\begin{sous-entrée}{pozo-hi}{ⓔpozoⓝpozo-hi}
\begin{glose}
\pfra{bague; alliance}
\end{glose}
\end{sous-entrée}
\end{entrée}

\begin{entrée}{pu}{1}{ⓔpuⓗ1}
\région{GOs PA BO}
\variante{%
phu
}
(\domainesémantique{Corps humain})
\classe{nom}
\begin{glose}
\pfra{cheveux}
\end{glose}
\begin{glose}
\pfra{poil}
\end{glose}
\begin{glose}
\pfra{fourrure}
\end{glose}
\begin{glose}
\pfra{plume}
\end{glose}
\newline
\begin{sous-entrée}{pu bwaa-je}{ⓔpuⓗ1ⓝpu bwaa-je}
\région{GOs}
\begin{glose}
\pfra{ses cheveux (il faut spécifier la partie du corps)}
\end{glose}
\newline
\begin{exemple}
\région{PA}
\textbf{\pnua{pu-n}}
\pfra{ses cheveux, poils}
\end{exemple}
\end{sous-entrée}
\newline
\begin{sous-entrée}{pu-bwa-ny}{ⓔpuⓗ1ⓝpu-bwa-ny}
\région{PA}
\begin{glose}
\pfra{mes cheveux}
\end{glose}
\end{sous-entrée}
\newline
\begin{sous-entrée}{pu-mhwêêdi-n}{ⓔpuⓗ1ⓝpu-mhwêêdi-n}
\région{PA}
\begin{glose}
\pfra{sa moustache}
\end{glose}
\end{sous-entrée}
\newline
\begin{sous-entrée}{pu-me-n}{ⓔpuⓗ1ⓝpu-me-n}
\région{PA}
\begin{glose}
\pfra{cils (poils-yeux)}
\end{glose}
\end{sous-entrée}
\newline
\begin{sous-entrée}{pu-bo}{ⓔpuⓗ1ⓝpu-bo}
\begin{glose}
\pfra{poils de roussette}
\end{glose}
\end{sous-entrée}
\newline
\begin{sous-entrée}{pu ko}{ⓔpuⓗ1ⓝpu ko}
\begin{glose}
\pfra{plume de poule}
\end{glose}
\end{sous-entrée}
\newline
\étymologie{
\langue{POc}
\étymon{*pulu}
\glosecourte{poil}}
\end{entrée}

\begin{entrée}{pu}{2}{ⓔpuⓗ2}
\région{GOs}
\variante{%
pum
\région{PA BO}, 
bu, bo
\région{BO}}
\classe{nom}
\newline
\sens{1}
(\domainesémantique{Feu : objets et actions liés au feu})
\begin{glose}
\pfra{fumée}
\end{glose}
\newline
\sens{2}
(\domainesémantique{Terre})
\begin{glose}
\pfra{poussière}
\end{glose}
\newline
\begin{exemple}
\région{BO}
\textbf{\pnua{i pum a mèè-n}}
\pfra{il est inconscient}
\end{exemple}
\newline
\begin{sous-entrée}{pubu dili [BO]}{ⓔpuⓗ2ⓢ2ⓝpubu dili [BO]}
\begin{glose}
\pfra{poussière}
\end{glose}
\end{sous-entrée}
\newline
\note{pubu yaai [BO]}{grammaire}{fumée du feu}
\end{entrée}

\begin{entrée}{pu}{3}{ⓔpuⓗ3}
\région{PA BO [Corne]}
\variante{%
wu
}
(\domainesémantique{Conjonction})
\classe{CNJ}
\begin{glose}
\pfra{pour ; afin de}
\end{glose}
\end{entrée}

\begin{entrée}{pu}{4}{ⓔpuⓗ4}
\région{GOs PA}
(\domainesémantique{Prédicats existentiels})
\classe{v}
\begin{glose}
\pfra{il y a}
\end{glose}
\newline
\begin{exemple}
\région{GOs}
\textbf{\pnua{mô vara pu phò-ã}}
\pfra{nous avons chacunnotre charge/mission}
\end{exemple}
\newline
\begin{exemple}
\région{GO}
\textbf{\pnua{pu mõû pòi-je}}
\pfra{leur fils a une épouse}
\end{exemple}
\newline
\begin{exemple}
\textbf{\pnua{pu mwanii-lò, pu lòtò lò}}
\pfra{ils ont de l'argent, ils ont une voiture}
\end{exemple}
\newline
\begin{exemple}
\région{PA}
\textbf{\pnua{pu ẽnõ i nu}}
\pfra{j'ai des enfants}
\end{exemple}
\end{entrée}

\begin{entrée}{pu-}{}{ⓔpu-}
\région{GOs PA}
(\domainesémantique{Préfixes classificateurs numériques})
\classe{CLF.NUM}
\begin{glose}
\pfra{pieds d'arbre, feuilles, tubercules, racines}
\end{glose}
\newline
\begin{exemple}
\textbf{\pnua{pu-xe, pu-tru, pu-ko chaamwa, etc.}}
\pfra{un, deux, trois pieds de bananier}
\end{exemple}
\end{entrée}

\begin{entrée}{pû}{}{ⓔpû}
\région{GOs PA BO}
(\domainesémantique{Noms des plantes})
\classe{nom}
\begin{glose}
\pfra{haricot}
\end{glose}
\nomscientifique{Phasoleus sp}
\newline
\begin{sous-entrée}{pû-mii}{ⓔpûⓝpû-mii}
\begin{glose}
\pfra{haricot violet}
\end{glose}
\end{sous-entrée}
\newline
\begin{sous-entrée}{pû-zol}{ⓔpûⓝpû-zol}
\begin{glose}
\pfra{haricot vert et violet sauvage (non comestible)}
\end{glose}
\end{sous-entrée}
\newline
\begin{sous-entrée}{pò-pû}{ⓔpûⓝpò-pû}
\begin{glose}
\pfra{gousse de haricot}
\end{glose}
\end{sous-entrée}
\newline
\begin{sous-entrée}{pû-bilò}{ⓔpûⓝpû-bilò}
\begin{glose}
\pfra{haricot sauvage (à gousse biscornue)}
\end{glose}
\end{sous-entrée}
\end{entrée}

\begin{entrée}{pua}{}{ⓔpua}
\région{PA}
(\domainesémantique{Processus liés aux plantes})
\classe{v}
\begin{glose}
\pfra{produire ; donner (des fruits, tubercules)}
\end{glose}
\newline
\begin{exemple}
\région{PA}
\textbf{\pnua{la u pua mã}}
\pfra{les manguiers sont chargés de fruits, donnent en abondance}
\end{exemple}
\end{entrée}

\begin{entrée}{puãgo}{}{ⓔpuãgo}
\région{GOs}
\variante{%
pwãgo
\région{GO(s)}}
(\domainesémantique{Mollusques})
\classe{nom}
\begin{glose}
\pfra{coquillage}
\end{glose}
\end{entrée}

\begin{entrée}{pubu dili}{}{ⓔpubu dili}
\région{GOs PA BO}
(\domainesémantique{Terre})
\classe{nom}
\begin{glose}
\pfra{poussière de la terre}
\end{glose}
\end{entrée}

\begin{entrée}{pu-bwakira-me}{}{ⓔpu-bwakira-me}
\région{GOs}
\variante{%
pu-bwakitra-me
}
(\domainesémantique{Corps humain})
\classe{nom}
\begin{glose}
\pfra{sourcils}
\end{glose}
\end{entrée}

\begin{entrée}{pu-bwa-n}{}{ⓔpu-bwa-n}
\région{BO PA}
(\domainesémantique{Corps humain})
\classe{nom}
\begin{glose}
\pfra{barbe}
\end{glose}
\begin{glose}
\pfra{moustache}
\end{glose}
\begin{glose}
\pfra{cheveux}
\end{glose}
\end{entrée}

\begin{entrée}{pu-bwò}{}{ⓔpu-bwò}
\région{GOs}
(\domainesémantique{Corps animal})
\classe{nom}
\begin{glose}
\pfra{poils de roussette}
\end{glose}
\end{entrée}

\begin{entrée}{puçee}{}{ⓔpuçee}
\région{GOs}
\variante{%
puuye, puuce
\région{BO [Corne]}}
(\domainesémantique{Danses})
\classe{nom}
\begin{glose}
\pfra{pilou (avec percussion et ae, ae!)}
\end{glose}
\begin{glose}
\pfra{lieu de danse [BO]}
\end{glose}
\end{entrée}

\begin{entrée}{puçiu}{}{ⓔpuçiu}
\formephonétique{pudʒiu}
\région{GOs}
(\domainesémantique{Objets coutumiers
, Vêtements, parure})
\classe{nom}
\begin{glose}
\pfra{affaires (vêtements de qqn)}
\end{glose}
\begin{glose}
\pfra{biens personnels d'un défunt qu'on remet à ses maternels}
\end{glose}
\end{entrée}

\begin{entrée}{puçò}{}{ⓔpuçò}
\formephonétique{puʒɔ}
\région{GOs}
\variante{%
puyòl, puçòl
\formephonétique{pujɔl, puʒɔl}
\région{WE PA BO}}
(\domainesémantique{Préparation des aliments; modes de préparation et de cuisson})
\classe{v}
\begin{glose}
\pfra{cuisine (faire la) ; cuisiner}
\end{glose}
\newline
\begin{exemple}
\région{GOs}
\textbf{\pnua{kavwö e puço gèè}}
\pfra{la grand-mère n'a pas fait la cuisine}
\end{exemple}
\newline
\begin{sous-entrée}{mwa-puyòl}{ⓔpuçòⓝmwa-puyòl}
\région{PA}
\begin{glose}
\pfra{cuisine}
\end{glose}
\end{sous-entrée}
\newline
\begin{sous-entrée}{aa-puyòl}{ⓔpuçòⓝaa-puyòl}
\région{PA}
\begin{glose}
\pfra{cuisinier}
\end{glose}
\end{sous-entrée}
\end{entrée}

\begin{entrée}{pudi-go}{}{ⓔpudi-go}
\région{GOs}
\variante{%
puding-go
\région{PA BO}}
(\domainesémantique{Parties de plantes})
\classe{nom}
\begin{glose}
\pfra{noeud de bambou}
\end{glose}
\end{entrée}

\begin{entrée}{pu-döölia}{}{ⓔpu-döölia}
\région{GOs}
(\domainesémantique{Noms des plantes})
\classe{nom}
\begin{glose}
\pfra{épineux}
\end{glose}
\newline
\begin{exemple}
\textbf{\pnua{e pu-döölia ce}}
\pfra{c'est un arbre épineux}
\end{exemple}
\end{entrée}

\begin{entrée}{pu-drõgo}{}{ⓔpu-drõgo}
\région{GOs}
\variante{%
pu-dõgo
\région{PA}}
(\domainesémantique{Types de maison, architecture de la maison})
\classe{nom}
\begin{glose}
\pfra{masque ; chambranles sculptés}
\end{glose}
\newline
\begin{exemple}
\région{GOs}
\textbf{\pnua{pu-drõgo ni phwee-mwa Teã-ma}}
\pfra{les gardiens de la porte du grand chef}
\end{exemple}
\end{entrée}

\begin{entrée}{pue}{}{ⓔpue}
\région{GOs}
(\domainesémantique{Relations et interaction sociales})
\classe{v}
\begin{glose}
\pfra{respecter ; honorer}
\end{glose}
\newline
\begin{sous-entrée}{thu puu}{ⓔpueⓝthu puu}
\begin{glose}
\pfra{montrer du respect}
\end{glose}
\newline
\begin{exemple}
\textbf{\pnua{e vhaa pue-je}}
\pfra{il se vante, il parle respectueusement de lui-même, en s'accordant de l'importance}
\end{exemple}
\newline
\begin{exemple}
\région{GOs}
\textbf{\pnua{e zo na çö pue ewööni-çö}}
\pfra{il faut respecter ton oncle maternel}
\end{exemple}
\newline
\note{pueli, puuli (v.t.)}{grammaire}{}
\end{sous-entrée}
\end{entrée}

\begin{entrée}{pu gènè}{}{ⓔpu gènè}
\région{GOs}
(\domainesémantique{Description des objets, formes, consistance, taille})
\classe{n ; v}
\begin{glose}
\pfra{couleur}
\end{glose}
\begin{glose}
\pfra{coloré}
\end{glose}
\newline
\begin{exemple}
\textbf{\pnua{pu gènè hõbwoli-jö !}}
\pfra{ta robe a beaucoup de couleurs !}
\end{exemple}
\end{entrée}

\begin{entrée}{pu hê}{}{ⓔpu hê}
\région{GOs}
(\domainesémantique{Description des objets, formes, consistance, taille})
\classe{v}
\begin{glose}
\pfra{avoir un contenu}
\end{glose}
\newline
\begin{exemple}
\textbf{\pnua{pu hê dröö ? - Elo, pu hê dröö !}}
\pfra{il y a qqch dans la marmite ? - Oui, il y a qqch dans la marmite}
\end{exemple}
\newline
\relationsémantique{Ant.}{\lien{}{kixa hê dröö}}
\glosecourte{la marmite est vide}
\end{entrée}

\begin{entrée}{pui}{}{ⓔpui}
\région{GOs PA}
\variante{%
phui-n
\région{BO [Corne]}}
\classe{nom}
\newline
\sens{1}
(\domainesémantique{Topographie})
\begin{glose}
\pfra{hauteur}
\end{glose}
\newline
\begin{exemple}
\région{GOs}
\textbf{\pnua{mwa na bwa pui}}
\pfra{une maison sur une hauteur}
\end{exemple}
\newline
\begin{exemple}
\région{PA}
\textbf{\pnua{ge ni pui}}
\pfra{elle est sur une hauteur (maison)}
\end{exemple}
\newline
\sens{2}
(\domainesémantique{Corps humain})
\begin{glose}
\pfra{bosse (sur la tête)}
\end{glose}
\end{entrée}

\begin{entrée}{pû-kai-ã}{}{ⓔpû-kai-ã}
\région{GOs}
\classe{nom}
(\domainesémantique{Corps humain})
\begin{glose}
\pfra{rein}
\end{glose}
\newline
\begin{exemple}
\région{GOs}
\textbf{\pnua{pû-kai-nu}}
\pfra{mon rein}
\end{exemple}
\end{entrée}

\begin{entrée}{pu-ko}{}{ⓔpu-ko}
\région{GOs}
(\domainesémantique{Oiseaux})
\classe{nom}
\begin{glose}
\pfra{plume de poule}
\end{glose}
\end{entrée}

\begin{entrée}{pu-kòò-n}{}{ⓔpu-kòò-n}
\région{BO}
(\domainesémantique{Corps humain})
\classe{nom}
\begin{glose}
\pfra{hanche (lit. la base des jambes) [Corne]}
\end{glose}
\end{entrée}

\begin{entrée}{pulòn}{}{ⓔpulòn}
\région{PA}
(\domainesémantique{Insectes})
\classe{nom}
\begin{glose}
\pfra{punaise (de lit, bois)}
\end{glose}
\end{entrée}

\begin{entrée}{pum-a mèè}{}{ⓔpum-a mèè}
\région{GOs WEM WE BO PA}
(\domainesémantique{Santé, maladie})
\classe{nom}
\begin{glose}
\pfra{évanouissement ; évanoui}
\end{glose}
\newline
\begin{exemple}
\région{PA}
\textbf{\pnua{pum a mèè-n}}
\pfra{il est inconscient (a de la fumée dans les yeux)}
\end{exemple}
\newline
\relationsémantique{Cf.}{\lien{}{burò la mèè-je [GOs]}}
\glosecourte{il est évanoui (lit.ses yeux sont dans l'obscurité)}
\end{entrée}

\begin{entrée}{pu-mee}{}{ⓔpu-mee}
\région{GOs}
(\domainesémantique{Corps humain})
\classe{nom}
\begin{glose}
\pfra{cils}
\end{glose}
\newline
\begin{exemple}
\textbf{\pnua{pu-mee-ã}}
\pfra{nos cils}
\end{exemple}
\end{entrée}

\begin{entrée}{pu-mèni}{}{ⓔpu-mèni}
\région{GOs PA}
(\domainesémantique{Oiseaux})
\classe{nom}
\begin{glose}
\pfra{plume}
\end{glose}
\newline
\relationsémantique{Cf.}{\lien{}{*bulu POc}}
\end{entrée}

\begin{entrée}{pumhãmã}{}{ⓔpumhãmã}
\région{GOs}
(\domainesémantique{Description des objets, formes, consistance, taille})
\classe{v}
\begin{glose}
\pfra{léger}
\end{glose}
\end{entrée}

\begin{entrée}{pu-mwa}{}{ⓔpu-mwa}
\région{GOs}
(\domainesémantique{Types de maison, architecture de la maison})
\classe{nom}
\begin{glose}
\pfra{arrière de la maison}
\end{glose}
\newline
\relationsémantique{Cf.}{\lien{ⓔkaçaⓝkaça mwa}{kaça mwa}}
\glosecourte{arrière de la maison}
\end{entrée}

\begin{entrée}{pune}{}{ⓔpune}
\région{GOs}
\variante{%
puni
\région{PA}}
(\domainesémantique{Conjonction})
\classe{CNJ}
\begin{glose}
\pfra{car ; parce que ; à cause de}
\end{glose}
\newline
\begin{exemple}
\textbf{\pnua{kôra mo a pune dree}}
\pfra{on ne peut pas partir à cause du vent}
\end{exemple}
\newline
\begin{exemple}
\région{GO}
\textbf{\pnua{mii mwa dili, puni nye e mwani mii (mwa) jena}}
\pfra{la terre devient rouge, du fait que c'est de l'argent rouge cela}
\end{exemple}
\newline
\begin{exemple}
\région{GOs}
\textbf{\pnua{e thô pune nu}}
\pfra{il est en colère (lit. fermé) à cause de moi}
\end{exemple}
\newline
\begin{sous-entrée}{pune lò}{ⓔpuneⓝpune lò}
\région{GOs}
\begin{glose}
\pfra{à cause d'eux}
\end{glose}
\newline
\begin{exemple}
\région{PA}
\textbf{\pnua{puni yo}}
\pfra{à cause de toi}
\end{exemple}
\newline
\relationsémantique{Cf.}{\lien{ⓔuiⓗ1}{ui}}
\glosecourte{envers}
\end{sous-entrée}
\end{entrée}

\begin{entrée}{puneda ?}{}{ⓔpuneda ?}
\région{GOs BO}
\variante{%
punanda?
\région{PA}}
(\domainesémantique{Interrogatifs})
\classe{INT}
\begin{glose}
\pfra{pourquoi ?}
\end{glose}
\newline
\begin{exemple}
\région{BO}
\textbf{\pnua{puneda u yu gi ?}}
\pfra{pourquoi pleures-tu ?}
\end{exemple}
\end{entrée}

\begin{entrée}{pu nee}{}{ⓔpu nee}
\région{PA BO [BM]}
(\domainesémantique{Manière de faire l’action : verbes et adverbes de manière})
\classe{v}
\begin{glose}
\pfra{faire exprès}
\end{glose}
\newline
\begin{exemple}
\région{PA}
\textbf{\pnua{kavwö nu pu nee}}
\pfra{je n'ai pas fait exprès}
\end{exemple}
\end{entrée}

\begin{entrée}{pu-ni}{}{ⓔpu-ni}
\formephonétique{pu-ɳi}
\région{GOs}
\variante{%
pu-ning
\région{PA}}
(\domainesémantique{Types de maison, architecture de la maison})
\classe{nom}
\begin{glose}
\pfra{fond de la maison ronde (aussi la base du 'ning')}
\end{glose}
\end{entrée}

\begin{entrée}{pu-n ma}{}{ⓔpu-n ma}
\région{BO}
(\domainesémantique{Prépositions})
\classe{PREP}
\begin{glose}
\pfra{à cause de ; parce que}
\end{glose}
\newline
\begin{exemple}
\textbf{\pnua{pu-n ma kia po eneda po huda}}
\pfra{parce qu'il n'y a rien une peu plus haut}
\end{exemple}
\end{entrée}

\begin{entrée}{punõ}{}{ⓔpunõ}
\formephonétique{puɳɔ̃}
\région{GOs BO PA}
(\domainesémantique{Description des objets, formes, consistance, taille})
\classe{nom}
\begin{glose}
\pfra{fond}
\end{glose}
\newline
\begin{exemple}
\région{GOs}
\textbf{\pnua{punõõ we}}
\pfra{le fond de l'eau}
\end{exemple}
\newline
\begin{exemple}
\région{PA}
\textbf{\pnua{punõõ keel}}
\pfra{le fond du panier}
\end{exemple}
\end{entrée}

\begin{entrée}{pu-noo}{}{ⓔpu-noo}
\formephonétique{pu-ɳoː}
\région{GOs}
\variante{%
pu-noo-n
\région{PA}}
(\domainesémantique{Corps humain})
\classe{nom}
\begin{glose}
\pfra{nuque}
\end{glose}
\end{entrée}

\begin{entrée}{punõõ-dröö}{}{ⓔpunõõ-dröö}
\formephonétique{puɳɔ̃ː-ɖωː}
\région{GOs}
\variante{%
punõõ-döö
\région{PA}}
(\domainesémantique{Description des objets, formes, consistance, taille})
\classe{nom}
\begin{glose}
\pfra{fond de la marmite}
\end{glose}
\end{entrée}

\begin{entrée}{punõõ-we}{}{ⓔpunõõ-we}
\formephonétique{puɳɔ̃ː-we}
\région{GOs BO PA}
(\domainesémantique{Description des objets, formes, consistance, taille})
\classe{nom}
\begin{glose}
\pfra{fond de l'eau}
\end{glose}
\begin{glose}
\pfra{lit de la rivière [BO]}
\end{glose}
\end{entrée}

\begin{entrée}{pu nye}{}{ⓔpu nye}
\région{GO}
\variante{%
po nye
\région{GO}}
(\domainesémantique{Conjonction})
\classe{CNJ}
\begin{glose}
\pfra{du fait que ; parce que}
\end{glose}
\newline
\begin{exemple}
\textbf{\pnua{pu nye za maza nõõli xo je nye êmwê}}
\pfra{parce que c'est la première fois qu'elle voit un homme}
\end{exemple}
\end{entrée}

\begin{entrée}{punyeda}{}{ⓔpunyeda}
\région{GOs}
(\domainesémantique{Conjonction})
\classe{CNJ}
\begin{glose}
\pfra{parce que ; à cause de}
\end{glose}
\newline
\begin{exemple}
\région{GO}
\textbf{\pnua{ma nye za kòi-nu na ênè avwõnõ, pu-nye da a khilaa-je}}
\pfra{la raison pour laquelle je n'étais pas ici à la maison, c'était parce que je suis parti la chercher}
\end{exemple}
\end{entrée}

\begin{entrée}{pu pai}{}{ⓔpu pai}
\région{GOs}
(\domainesémantique{Description des objets, formes, consistance, taille})
\classe{v}
\begin{glose}
\pfra{avoir des tubercules}
\end{glose}
\end{entrée}

\begin{entrée}{pu-pee-ã}{}{ⓔpu-pee-ã}
\région{GOs}
\variante{%
pu-vee
\région{GO(s)}}
(\domainesémantique{Corps humain})
\classe{nom}
\begin{glose}
\pfra{hanche}
\end{glose}
\newline
\begin{exemple}
\région{GOs}
\textbf{\pnua{pu-pee-nu}}
\pfra{ma hanche (lit. la basedes cuisses)}
\end{exemple}
\newline
\begin{exemple}
\région{BO}
\textbf{\pnua{pu-pee-n}}
\pfra{sa hanche (lit. la base des cuisses)}
\end{exemple}
\end{entrée}

\begin{entrée}{pu-pii}{}{ⓔpu-pii}
\région{GOs}
(\domainesémantique{Crustacés, crabes})
\classe{nom}
\begin{glose}
\pfra{crabe double peau}
\end{glose}
\newline
\note{(cette carapace est décollée mais n'est pas encore tombée)}{glose}{}
\newline
\relationsémantique{Cf.}{\lien{ⓔa-pii}{a-pii}}
\glosecourte{crabe mou (dont la carapace est tombée)}
\end{entrée}

\begin{entrée}{pu punõ}{}{ⓔpu punõ}
\région{GOs}
(\domainesémantique{Description des objets, formes, consistance, taille})
\classe{v.stat.}
\begin{glose}
\pfra{avoir un fond}
\end{glose}
\end{entrée}

\begin{entrée}{pu pwò}{}{ⓔpu pwò}
\région{GOs}
(\domainesémantique{Description des objets, formes, consistance, taille})
\classe{v}
\begin{glose}
\pfra{avoir des fruits}
\end{glose}
\newline
\begin{exemple}
\textbf{\pnua{e za pu pwò, mãã ni ? - Ô, e za pu pwò}}
\pfra{est-ce qu'il y a des fruits, (dans) ce manguier ? -Oui, il y a des fruits.}
\end{exemple}
\end{entrée}

\begin{entrée}{pu-phwa}{}{ⓔpu-phwa}
\région{GOs}
\variante{%
pu-phwa
\région{BO PA}}
(\domainesémantique{Corps humain})
\classe{nom}
\begin{glose}
\pfra{barbe ; moustache}
\end{glose}
\newline
\begin{exemple}
\région{PA}
\textbf{\pnua{pu-phwa-n}}
\pfra{sa barbe/moustache}
\end{exemple}
\newline
\begin{exemple}
\région{BO}
\textbf{\pnua{pu-phwa-ny}}
\pfra{ma barbe}
\end{exemple}
\end{entrée}

\begin{entrée}{puradimwã}{}{ⓔpuradimwã}
\formephonétique{puɽa'dimwã}
\région{GOs PA}
\classe{nom}
\newline
\sens{1}
(\domainesémantique{Oiseaux})
\begin{glose}
\pfra{tourterelle verte}
\end{glose}
\nomscientifique{Chalcophaps indica (Columbidés)}
\newline
\sens{2}
(\domainesémantique{Organisation sociale})
\begin{glose}
\pfra{messager de la chefferie}
\end{glose}
\end{entrée}

\begin{entrée}{pure}{}{ⓔpure}
\région{GOs BO}
\variante{%
pyèro
\région{PA BO}}
(\domainesémantique{Oiseaux})
\classe{nom}
\begin{glose}
\pfra{"merle noir", stourne calédonien}
\end{glose}
\nomscientifique{Aplonis striatus striatus, Sturnidés}
\end{entrée}

\begin{entrée}{puri}{}{ⓔpuri}
\région{BO}
(\domainesémantique{Reptiles marins})
\classe{nom}
\begin{glose}
\pfra{serpent de mer}
\end{glose}
\end{entrée}

\begin{entrée}{putrakou}{}{ⓔputrakou}
\formephonétique{pu'ɽakou}
\région{GOs}
\variante{%
purakou
\région{GO(s)}}
(\domainesémantique{Poissons})
\classe{nom}
\begin{glose}
\pfra{carangue (grosse)}
\end{glose}
\end{entrée}

\begin{entrée}{putrumi}{}{ⓔputrumi}
\formephonétique{pu'ɽumi}
\région{GOs}
\région{PA BO}
\variante{%
purumi
\formephonétique{pu'rumi}}
(\domainesémantique{Insectes})
\classe{nom}
\begin{glose}
\pfra{fourmi (petite et rouge)}
\end{glose}
\newline
\emprunt{fourmi (FR)}
\end{entrée}

\begin{entrée}{putruna}{}{ⓔputruna}
\formephonétique{puɽuɳa}
\région{GOs}
(\domainesémantique{Vêtements, parure})
\classe{nom}
\begin{glose}
\pfra{manou (hommes)}
\end{glose}
\end{entrée}

\begin{entrée}{puu}{1}{ⓔpuuⓗ1}
\région{GOs PA BO}
\variante{%
pu
\région{GO(s)}, 
puu-n
\région{BO PA}, 
puxu-n
\région{BO}}
\classe{nom}
\newline
\sens{1}
(\domainesémantique{Parties de plantes})
\begin{glose}
\pfra{pied ; tronc}
\end{glose}
\begin{glose}
\pfra{base}
\end{glose}
\newline
\begin{sous-entrée}{pu-no-je [GO]}{ⓔpuuⓗ1ⓢ1ⓝpu-no-je [GO]}
\begin{glose}
\pfra{la base de son cou}
\end{glose}
\end{sous-entrée}
\newline
\begin{sous-entrée}{pu-no-n [PA]}{ⓔpuuⓗ1ⓢ1ⓝpu-no-n [PA]}
\begin{glose}
\pfra{la base de son cou}
\end{glose}
\end{sous-entrée}
\newline
\sens{2}
(\domainesémantique{Organisation sociale})
\begin{glose}
\pfra{ancêtres}
\end{glose}
\newline
\begin{exemple}
\textbf{\pnua{puu-ã}}
\pfra{notre origine,Dieu}
\end{exemple}
\newline
\sens{3}
(\domainesémantique{Conjonction})
\begin{glose}
\pfra{origine ; source ; cause}
\end{glose}
\newline
\begin{exemple}
\région{GOs}
\textbf{\pnua{kixa puu}}
\pfra{il n'y a pas de raison, sans cause}
\end{exemple}
\newline
\begin{exemple}
\région{GOs}
\textbf{\pnua{puu xo la woovwa}}
\pfra{la raison de leur dispute}
\end{exemple}
\newline
\begin{exemple}
\textbf{\pnua{kia puu-n}}
\pfra{sans cause}
\end{exemple}
\newline
\begin{exemple}
\région{BO}
\textbf{\pnua{puu-xe puu-go}}
\end{exemple}
\newline
\begin{sous-entrée}{puu-neda ?}{ⓔpuuⓗ1ⓢ3ⓝpuu-neda ?}
\begin{glose}
\pfra{pourquoi ?}
\end{glose}
\end{sous-entrée}
\newline
\begin{sous-entrée}{kia puu-n}{ⓔpuuⓗ1ⓢ3ⓝkia puu-n}
\begin{glose}
\pfra{sans raison}
\end{glose}
\end{sous-entrée}
\newline
\begin{sous-entrée}{pu fa}{ⓔpuuⓗ1ⓢ3ⓝpu fa}
\begin{glose}
\pfra{l'origine d'une affaire}
\end{glose}
\newline
\relationsémantique{Cf.}{\lien{ⓔpune}{pune}}
\glosecourte{parce que, car}
\end{sous-entrée}
\newline
\étymologie{
\langue{POc}
\étymon{*puqu(n)}}
\end{entrée}

\begin{entrée}{puu}{2}{ⓔpuuⓗ2}
\région{GOs}
\variante{%
pulu
\région{BO}}
(\domainesémantique{Navigation})
\classe{nom}
\begin{glose}
\pfra{perche pour pousser la pirogue}
\end{glose}
\begin{glose}
\pfra{gaffe ; barre}
\end{glose}
\end{entrée}

\begin{entrée}{puu-ce}{}{ⓔpuu-ce}
\région{GOs PA BO}
(\domainesémantique{Arbre})
\classe{nom}
\begin{glose}
\pfra{collet de l'arbre (sa base)}
\end{glose}
\end{entrée}

\begin{entrée}{puu-mwa}{}{ⓔpuu-mwa}
\région{GOs PA BO}
\variante{%
pu-mwa
\région{PA BO}}
(\domainesémantique{Types de maison, architecture de la maison})
\classe{nom}
\begin{glose}
\pfra{mur de la maison (tous les murs)}
\end{glose}
\end{entrée}

\begin{entrée}{puuńô}{}{ⓔpuuńô}
(\domainesémantique{Discours, échanges verbaux})
\classe{v}
\begin{glose}
\pfra{discourir ; faire le discours de coutume ; haranguer}
\end{glose}
\begin{glose}
\pfra{parler pour se réconcilier ; paix (faire la) ; faire un discours coutumier}
\end{glose}
\end{entrée}

\begin{entrée}{puuvwoo}{}{ⓔpuuvwoo}
\région{GOs}
(\domainesémantique{Poissons})
\classe{nom}
\begin{glose}
\pfra{tarpon à filament}
\end{glose}
\nomscientifique{Megalops cyprinoides (Elopidés)}
\end{entrée}

\begin{entrée}{puvwu}{}{ⓔpuvwu}
\région{GOs}
\variante{%
pupu-n, puvwu-n
\région{BO}}
(\domainesémantique{Corps humain})
\classe{nom}
\begin{glose}
\pfra{nuque}
\end{glose}
\newline
\begin{exemple}
\région{GOs}
\textbf{\pnua{puvwu-nu}}
\pfra{ma nuque}
\end{exemple}
\newline
\begin{exemple}
\région{BO PA}
\textbf{\pnua{puvwu-ny}}
\pfra{ma nuque}
\end{exemple}
\end{entrée}

\begin{entrée}{puxãnu}{}{ⓔpuxãnu}
\formephonétique{puɣɛ̃ɳu}
\région{GOs}
\variante{%
poxãnu, pwããnu
\région{GO(s)}, 
poxònu, poonu
\région{BO}}
(\domainesémantique{Sentiments
, Relations et interaction sociales})
\classe{v ; n}
\begin{glose}
\pfra{amour ; compatir ; avoir pitié de}
\end{glose}
\begin{glose}
\pfra{aimer ; affectionner}
\end{glose}
\newline
\begin{exemple}
\textbf{\pnua{e pwãnuu ẽnõ}}
\pfra{elle aime cet enfant}
\end{exemple}
\newline
\begin{exemple}
\textbf{\pnua{puxãnu -ayu}}
\pfra{la grâce}
\end{exemple}
\newline
\begin{exemple}
\région{BO}
\textbf{\pnua{nu ma poxònu-m}}
\pfra{j'ai beaucoup pitié de toi}
\end{exemple}
\end{entrée}

\begin{entrée}{puxa-teã}{}{ⓔpuxa-teã}
\région{GOs}
(\domainesémantique{Organisation sociale})
\classe{nom}
\begin{glose}
\pfra{clan cadet ; cadet de la chefferie}
\end{glose}
\end{entrée}

\begin{entrée}{pu-xè}{}{ⓔpu-xè}
\région{PA BO}
(\domainesémantique{Préfixes classificateurs numériques})
\classe{CLF.NUM (maisons)}
\begin{glose}
\pfra{un (bâtiment, arbre, qqch qui a une souche et qui est haut)}
\end{glose}
\newline
\begin{sous-entrée}{pu-xè ; pu-tru}{ⓔpu-xèⓝpu-xè ; pu-tru}
\begin{glose}
\pfra{un ; deux}
\end{glose}
\end{sous-entrée}
\end{entrée}

\begin{entrée}{puxu-n}{}{ⓔpuxu-n}
\région{PA}
\variante{%
puvwu-n
\région{PA}}
(\domainesémantique{Types de maison, architecture de la maison})
\classe{nom}
\begin{glose}
\pfra{fond (de la maison)}
\end{glose}
\end{entrée}

\begin{entrée}{puya}{}{ⓔpuya}
\région{BO}
(\domainesémantique{Préparation des aliments; modes de préparation et de cuisson})
\classe{v}
\begin{glose}
\pfra{vider le four enterré [Corne]}
\end{glose}
\end{entrée}

\begin{entrée}{puyai}{}{ⓔpuyai}
\région{PA}
(\domainesémantique{Bananiers et bananes})
\classe{nom}
\begin{glose}
\pfra{bananier ('banane chef')}
\end{glose}
\end{entrée}

\begin{entrée}{pu zòò}{}{ⓔpu zòò}
\région{GOs}
(\domainesémantique{Caractéristiques et propriétés des personnes})
\classe{v}
\begin{glose}
\pfra{difficile}
\end{glose}
\newline
\relationsémantique{Ant.}{\lien{ⓔkixa zòò}{kixa zòò}}
\glosecourte{facile}
\end{entrée}

\newpage

\lettrine{ph}\begin{entrée}{pha}{1}{ⓔphaⓗ1}
\région{GOs BO}
\classe{v}
\newline
\sens{1}
(\domainesémantique{Mouvements ou actions faits avec le corps, les bras, les mains, les pieds})
\begin{glose}
\pfra{lancer}
\end{glose}
\newline
\begin{exemple}
\textbf{\pnua{e pha-u pa, e pha-ò pa}}
\pfra{il lance une pierre}
\end{exemple}
\newline
\sens{2}
(\domainesémantique{Armes})
\begin{glose}
\pfra{tirer (au fusil)}
\end{glose}
\end{entrée}

\begin{entrée}{pha}{2}{ⓔphaⓗ2}
\région{PA BO}
(\domainesémantique{Vents})
\classe{v}
\begin{glose}
\pfra{souffler (vent)}
\end{glose}
\end{entrée}

\begin{entrée}{pha-}{}{ⓔpha-}
\région{GO BO PA}
\variante{%
phaa-
}
(\domainesémantique{Causatif})
\classe{PREF.CAUS}
\begin{glose}
\pfra{faire (faire)}
\end{glose}
\newline
\begin{exemple}
\région{GO}
\textbf{\pnua{phaa-vhaa-je !}}
\pfra{incite le à parler !}
\end{exemple}
\newline
\begin{exemple}
\région{PA}
\textbf{\pnua{phaa-gumãgu !}}
\pfra{fais en sorte que cela soit vrai !}
\end{exemple}
\newline
\note{variante pa-}{grammaire}{}
\end{entrée}

\begin{entrée}{phaa}{}{ⓔphaa}
\région{GOs BO PA}
\classe{nom}
\newline
\sens{1}
(\domainesémantique{Corps humain})
\begin{glose}
\pfra{poumon}
\end{glose}
\newline
\begin{exemple}
\région{PA}
\textbf{\pnua{phaa-n}}
\pfra{ses poumons}
\end{exemple}
\newline
\sens{2}
(\domainesémantique{Navigation})
\begin{glose}
\pfra{radeau ; flotteur}
\end{glose}
\newline
\begin{exemple}
\région{GO}
\textbf{\pnua{phaa ne xo gò}}
\pfra{un radeau fait en bambou}
\end{exemple}
\newline
\begin{exemple}
\région{PA}
\textbf{\pnua{phaa-m}}
\pfra{ton radeau}
\end{exemple}
\newline
\étymologie{
\langue{POc}
\étymon{*paRaq}
\glosecourte{poumon}}
\end{entrée}

\begin{entrée}{phãã}{}{ⓔphãã}
\région{GOs}
(\domainesémantique{Poissons})
\classe{nom}
\begin{glose}
\pfra{"aiguillette"}
\end{glose}
\nomscientifique{Tylosorus crocodilus crocodilus (Belonidés)}
\end{entrée}

\begin{entrée}{phaa-bini}{}{ⓔphaa-bini}
\formephonétique{pʰaː-biɳi}
\région{GOs}
(\domainesémantique{Verbes d'action (en général)})
\classe{v}
\begin{glose}
\pfra{dégonfler}
\end{glose}
\end{entrée}

\begin{entrée}{phaa-butrõ}{}{ⓔphaa-butrõ}
\formephonétique{,pʰa-'buɽõ}
\région{GOs}
\variante{%
pa-burõ
\région{GO(s)}}
(\domainesémantique{Soins du corps})
\classe{v}
\begin{glose}
\pfra{baigner (enfant)}
\end{glose}
\newline
\begin{exemple}
\région{GOs}
\textbf{\pnua{e phaa-butrõ-ni ẽnõ}}
\pfra{il fait baigner l'enfant}
\end{exemple}
\end{entrée}

\begin{entrée}{phaa-cebwo}{}{ⓔphaa-cebwo}
\formephonétique{pʰaː-tjebwo}
\région{GOs}
\variante{%
pu-cibwo
\région{GO(s)}}
(\domainesémantique{Fonctions naturelles humaines})
\classe{v}
\begin{glose}
\pfra{dormir auprès du feu (la nuit)}
\end{glose}
\newline
\begin{sous-entrée}{phaai cebwo}{ⓔphaa-cebwoⓝphaai cebwo}
\begin{glose}
\pfra{allumer le feu pour la nuit}
\end{glose}
\end{sous-entrée}
\newline
\begin{sous-entrée}{kô-pa-ce-bò ; kô-phaa-cebòn}{ⓔphaa-cebwoⓝkô-pa-ce-bò ; kô-phaa-cebòn}
\région{WEM}
\begin{glose}
\pfra{dormir auprès du feu (la nuit)}
\end{glose}
\end{sous-entrée}
\end{entrée}

\begin{entrée}{phaa-cii}{}{ⓔphaa-cii}
\région{GOs}
\variante{%
phaa-çi
\région{GO(s)}}
(\domainesémantique{Mouvements ou actions faits avec le corps, les bras, les mains, les pieds})
\classe{v}
\begin{glose}
\pfra{épouiller ; chercher les poux}
\end{glose}
\newline
\begin{exemple}
\région{GOs}
\textbf{\pnua{e phaa-cii-ni bwaa-nu}}
\pfra{viens chercher les poux de ma tête}
\end{exemple}
\newline
\begin{exemple}
\textbf{\pnua{co a-da-mi phaa-cii-nu}}
\pfra{viens m'épouiller}
\end{exemple}
\end{entrée}

\begin{entrée}{phããde}{}{ⓔphããde}
\formephonétique{pʰɛ̃ːde}
\région{GOs BO PA}
\variante{%
phãde
\région{GO(s) PA}}
(\domainesémantique{Mouvements ou actions faits avec le corps, les bras, les mains, les pieds})
\classe{v}
\begin{glose}
\pfra{montrer}
\end{glose}
\begin{glose}
\pfra{présenter}
\end{glose}
\begin{glose}
\pfra{révéler}
\end{glose}
\newline
\begin{exemple}
\textbf{\pnua{la phããde fim}}
\pfra{ils montrent un film}
\end{exemple}
\newline
\begin{exemple}
\textbf{\pnua{la phããde vhaa}}
\pfra{ils annoncent une nouvelle}
\end{exemple}
\newline
\begin{exemple}
\textbf{\pnua{la phããde jaa}}
\pfra{ils annoncent ce qui va arriver}
\end{exemple}
\newline
\begin{exemple}
\région{PA}
\textbf{\pnua{la phããde cii-phagò}}
\pfra{ils ont fait acte de présence}
\end{exemple}
\end{entrée}

\begin{entrée}{phããde cii-phagò}{}{ⓔphããde cii-phagò}
\région{PA}
(\domainesémantique{Relations et interaction sociales})
\classe{v}
\begin{glose}
\pfra{faire acte de présence}
\end{glose}
\newline
\begin{exemple}
\région{PA}
\textbf{\pnua{la phããde cii-phagò}}
\pfra{ils ont fait acte de présence}
\end{exemple}
\end{entrée}

\begin{entrée}{phããde kui}{}{ⓔphããde kui}
\région{GOs}
(\domainesémantique{Coutumes, dons coutumiers})
\classe{nom}
\begin{glose}
\pfra{fête des prémices des ignames (lit. montrer l'igname)}
\end{glose}
\end{entrée}

\begin{entrée}{phaa-gò}{}{ⓔphaa-gò}
\formephonétique{pʰaː-ŋgɔ}
\région{GOs PA BO}
\variante{%
phaa
\région{GO(s)}}
(\domainesémantique{Navigation})
\classe{nom}
\begin{glose}
\pfra{radeau (en bambou)}
\end{glose}
\end{entrée}

\begin{entrée}{phaamee}{}{ⓔphaamee}
\région{GOs PA}
(\domainesémantique{Parenté})
\classe{nom}
\begin{glose}
\pfra{aîné (des enfants)}
\end{glose}
\newline
\begin{exemple}
\région{GOs}
\textbf{\pnua{phaamee pòi-nu}}
\pfra{l'aîné de mes enfants}
\end{exemple}
\newline
\begin{exemple}
\région{PA}
\textbf{\pnua{phaamee ẽnõ}}
\pfra{l'aîné des enfants}
\end{exemple}
\end{entrée}

\begin{entrée}{phaa-puio}{}{ⓔphaa-puio}
\région{BO [BM]}
(\domainesémantique{Pêche})
\classe{nom}
\begin{glose}
\pfra{flotteur du filet}
\end{glose}
\end{entrée}

\begin{entrée}{phaa-tuuçò}{}{ⓔphaa-tuuçò}
\région{PA BO}
\variante{%
pha-ruyòng, pha-thuyòng
\région{PA}, 
phaxayuk
\région{BO}}
(\domainesémantique{Noms des plantes})
\classe{nom}
\begin{glose}
\pfra{plante ; graminée}
\end{glose}
\begin{glose}
\pfra{nom de la purge à base de Polygonum subsessile (Corne)}
\end{glose}
\nomscientifique{Polygonum subsessile}
\end{entrée}

\begin{entrée}{phaa-udale}{}{ⓔphaa-udale}
\région{GOs}
(\domainesémantique{Vêtements, parure})
\classe{v}
\begin{glose}
\pfra{vêtir ; habiller qqn}
\end{glose}
\newline
\begin{exemple}
\région{GOs}
\textbf{\pnua{e phaa-udale hõbwò i Kaavwo}}
\pfra{elle a habillé Kaavwo}
\end{exemple}
\end{entrée}

\begin{entrée}{phaavã}{}{ⓔphaavã}
\région{PA BO}
(\domainesémantique{Feu : objets et actions liés au feu})
\classe{nom}
\begin{glose}
\pfra{suie ; noir de fumée}
\end{glose}
\end{entrée}

\begin{entrée}{phaavwi}{}{ⓔphaavwi}
\formephonétique{pʰaːβi}
\région{GOs}
(\domainesémantique{Cultures, techniques, boutures})
\classe{v.t.}
\begin{glose}
\pfra{désherber ; couper l'herbe ; débrousser}
\end{glose}
\newline
\begin{exemple}
\textbf{\pnua{nu phaavwi pomõ-je}}
\pfra{je débrousse son champ}
\end{exemple}
\end{entrée}

\begin{entrée}{phaawa}{}{ⓔphaawa}
\région{GOs}
(\domainesémantique{Cultures, techniques, boutures})
\classe{v.i.}
\begin{glose}
\pfra{désherber ; couper l'herbe}
\end{glose}
\begin{glose}
\pfra{faucher}
\end{glose}
\newline
\begin{exemple}
\textbf{\pnua{e phaawange havu-je}}
\pfra{il désherbe son jardin}
\end{exemple}
\newline
\note{phaawange (v.t.)}{grammaire}{}
\end{entrée}

\begin{entrée}{phabuno}{}{ⓔphabuno}
\région{GOs}
(\domainesémantique{Insectes})
\classe{nom}
\begin{glose}
\pfra{ver de bancoul (au stade juvénile)}
\end{glose}
\end{entrée}

\begin{entrée}{pha-ca}{}{ⓔpha-ca}
(\domainesémantique{Verbes de mouvement})
\classe{v}
\begin{glose}
\pfra{toucher (cible)}
\end{glose}
\newline
\begin{exemple}
\région{GOs}
\textbf{\pnua{e aa-pha-ca}}
\pfra{il est adroit (il a touché la cible)}
\end{exemple}
\newline
\relationsémantique{Cf.}{\lien{ⓔphaⓗ1}{pha}}
\glosecourte{lancer}
\end{entrée}

\begin{entrée}{phaçegai}{}{ⓔphaçegai}
\formephonétique{pʰaʒeŋgai}
\région{GOs}
\variante{%
pha-caxai
\région{PA}}
(\domainesémantique{Relations et interaction sociales})
\classe{v}
\begin{glose}
\pfra{taquiner qqn ; embêter}
\end{glose}
\newline
\begin{exemple}
\textbf{\pnua{e a-pe-phaçegai}}
\pfra{il est taquin}
\end{exemple}
\end{entrée}

\begin{entrée}{phachaani}{}{ⓔphachaani}
\formephonétique{pʰacʰaːɳi}
\région{GOs}
(\domainesémantique{Fonctions intellectuelles})
\classe{v}
\begin{glose}
\pfra{approuver}
\end{glose}
\end{entrée}

\begin{entrée}{pha-choomu-ni}{}{ⓔpha-choomu-ni}
\formephonétique{pʰa-'cʰoːmu-ɳi}
\région{GOs}
(\domainesémantique{Fonctions intellectuelles})
\classe{v}
\begin{glose}
\pfra{enseigner}
\end{glose}
\end{entrée}

\begin{entrée}{phago}{}{ⓔphago}
\région{GOs}
(\domainesémantique{Couleurs})
\classe{nom}
\begin{glose}
\pfra{couleur}
\end{glose}
\newline
\begin{exemple}
\textbf{\pnua{e whaya phago ?}}
\pfra{quelle est sa couleur ?}
\end{exemple}
\end{entrée}

\begin{entrée}{phãgoo}{}{ⓔphãgoo}
\formephonétique{pʰɛ̃ŋgɔː}
\région{GOs}
\variante{%
phagoo
\région{PA BO}}
(\domainesémantique{Corps humain})
\classe{nom}
\begin{glose}
\pfra{corps ; enveloppe corporelle}
\end{glose}
\newline
\begin{exemple}
\région{GO}
\textbf{\pnua{waya phãgoo loto ?}}
\pfra{quelle est la couleur de la voiture ? (comment est le corps de la voiture?)}
\end{exemple}
\newline
\begin{exemple}
\région{PA}
\textbf{\pnua{phagoo-n}}
\pfra{son corps}
\end{exemple}
\newline
\étymologie{
\langue{PNNC (proto nord neo-calédonien)}
\étymon{*papa-qau, *qau}
\glosecourte{milieu}
\auteur{Hollyman}}
\end{entrée}

\begin{entrée}{phagoo-mwa}{}{ⓔphagoo-mwa}
\région{PA}
(\domainesémantique{Types de maison, architecture de la maison})
\classe{nom}
\begin{glose}
\pfra{toit (pente du) vu de l'intérieur}
\end{glose}
\newline
\relationsémantique{Cf.}{\lien{ⓔbwa-mwa}{bwa-mwa}}
\glosecourte{toit}
\end{entrée}

\begin{entrée}{pha-gumãgu-ni}{}{ⓔpha-gumãgu-ni}
\région{GOs}
(\domainesémantique{Fonctions intellectuelles})
\classe{v.t.}
\begin{glose}
\pfra{approuver qqn ; acquiescer}
\end{glose}
\end{entrée}

\begin{entrée}{phai}{}{ⓔphai}
\région{GOs}
\région{BO PA}
\variante{%
phaai
}
(\domainesémantique{Préparation des aliments; modes de préparation et de cuisson})
\classe{v}
\begin{glose}
\pfra{bouillir ; faire cuire}
\end{glose}
\newline
\begin{exemple}
\région{GOs}
\textbf{\pnua{kavwö e phai lai xo gèè}}
\pfra{la grand-mère n'a pas fait cuire le riz}
\end{exemple}
\newline
\begin{exemple}
\textbf{\pnua{e phai kui}}
\pfra{elle fait cuire l'igname}
\end{exemple}
\newline
\begin{exemple}
\textbf{\pnua{co phai da ?}}
\pfra{qu'as-tu fait cuire ?}
\end{exemple}
\end{entrée}

\begin{entrée}{phai cii}{}{ⓔphai cii}
\région{GOs}
(\domainesémantique{Préparation des aliments; modes de préparation et de cuisson})
\classe{v}
\begin{glose}
\pfra{bouillir, cuire (les ignames) avec la peau}
\end{glose}
\newline
\begin{exemple}
\région{GOs}
\textbf{\pnua{e phai cii kui}}
\pfra{elle a cuit les ignames avec la peau}
\end{exemple}
\end{entrée}

\begin{entrée}{phai mwa-çii}{}{ⓔphai mwa-çii}
\région{GOs}
(\domainesémantique{Préparation des aliments; modes de préparation et de cuisson})
\classe{v}
\begin{glose}
\pfra{cuire avec la peau (bananes)}
\end{glose}
\newline
\begin{exemple}
\région{GOs}
\textbf{\pnua{nu phai mwa-çii chaamwa}}
\pfra{j'ai fait cuire les bananes avec leur peau}
\end{exemple}
\end{entrée}

\begin{entrée}{phai-yaai}{}{ⓔphai-yaai}
\région{GOs BO}
\variante{%
phai-yaai, pha-yaai
\région{PA}}
(\domainesémantique{Feu : objets et actions liés au feu})
\classe{v}
\begin{glose}
\pfra{allumer un feu}
\end{glose}
\newline
\relationsémantique{Cf.}{\lien{ⓔthi yaai}{thi yaai}}
\end{entrée}

\begin{entrée}{phaja}{}{ⓔphaja}
\région{WEM WE PA BO}
(\domainesémantique{Fonctions intellectuelles})
\classe{v}
\begin{glose}
\pfra{demander à qqn ; interroger qqn}
\end{glose}
\begin{glose}
\pfra{demander (se)}
\end{glose}
\newline
\begin{exemple}
\région{BO}
\textbf{\pnua{nu phaja i je}}
\pfra{je lui ai demandé}
\end{exemple}
\newline
\relationsémantique{Cf.}{\lien{}{zala, zhala [GOs]}}
\glosecourte{demander à qqn}
\end{entrée}

\begin{entrée}{phajoo}{}{ⓔphajoo}
\formephonétique{pʰaɲɟoː}
\région{GOs PA BO}
\classe{nom}
\newline
\sens{1}
(\domainesémantique{Parties de plantes})
\begin{glose}
\pfra{entre-noeuds (bambou, canne à sucre)}
\end{glose}
\newline
\begin{sous-entrée}{phajo-ê}{ⓔphajooⓢ1ⓝphajo-ê}
\begin{glose}
\pfra{entre-noeuds de canne à sucre}
\end{glose}
\newline
\begin{exemple}
\région{BO}
\textbf{\pnua{pwajoo-n}}
\pfra{son entre-noeud}
\end{exemple}
\end{sous-entrée}
\newline
\sens{2}
(\domainesémantique{Corps humain})
\begin{glose}
\pfra{phalange (doigt)}
\end{glose}
\end{entrée}

\begin{entrée}{phajoo-hii-n}{}{ⓔphajoo-hii-n}
\région{BO}
(\domainesémantique{Corps humain})
\classe{nom}
\begin{glose}
\pfra{avant-bras [Corne]}
\end{glose}
\newline
\note{non vérifié}{général}{}
\end{entrée}

\begin{entrée}{phajoo-kòò-n}{}{ⓔphajoo-kòò-n}
\région{BO}
(\domainesémantique{Corps humain})
\classe{nom}
\begin{glose}
\pfra{bas de la jambe [Corne]}
\end{glose}
\newline
\note{non vérifié}{général}{}
\end{entrée}

\begin{entrée}{pha-kôôńô-ni}{}{ⓔpha-kôôńô-ni}
\formephonétique{,pʰa-kõː'nõ-ɳi}
\région{GOs}
(\domainesémantique{Interaction avec les animaux})
\classe{v}
\begin{glose}
\pfra{domestiquer (un animal) ; apprivoiser}
\end{glose}
\end{entrée}

\begin{entrée}{phalawe}{1}{ⓔphalaweⓗ1}
\région{GOs}
(\domainesémantique{Corps humain})
\classe{nom}
\begin{glose}
\pfra{thorax ; cage thoracique ; côtes}
\end{glose}
\newline
\begin{exemple}
\textbf{\pnua{e bi phalawe}}
\pfra{il s'est coupé le souffle}
\end{exemple}
\newline
\relationsémantique{Cf.}{\lien{ⓔzabò}{zabò}}
\glosecourte{côte}
\end{entrée}

\begin{entrée}{phalawe}{2}{ⓔphalaweⓗ2}
\région{PA}
(\domainesémantique{Types de maison, architecture de la maison})
\classe{nom}
\begin{glose}
\pfra{étagère, claie sur laquelle on suspendait les paniers dans la maison}
\end{glose}
\begin{glose}
\pfra{claie sur laquelle on fumait la nourriture}
\end{glose}
\end{entrée}

\begin{entrée}{phalawu}{}{ⓔphalawu}
\région{BO [BM]}
\variante{%
phalau
\région{BO}}
(\domainesémantique{Alliance})
\classe{nom}
\begin{glose}
\pfra{belle-soeur (soeur d'épouse ; épouse du frère)}
\end{glose}
\newline
\note{terme tombé en désuétude, maintenant on utilise bee-}{glose}{}
\newline
\begin{exemple}
\région{BO}
\textbf{\pnua{phalawu-ny}}
\pfra{ma belle-soeur}
\end{exemple}
\end{entrée}

\begin{entrée}{pha- ...-ni}{}{ⓔpha- ...-ni}
\formephonétique{pʰa-...-ɳi}
\région{GOs}
(\domainesémantique{Causatif})
\classe{PREF.CAUS}
\begin{glose}
\pfra{faire faire qqch. ; laisser}
\end{glose}
\newline
\begin{exemple}
\textbf{\pnua{e pha-thròbo-ni}}
\pfra{il l'a fait tomber}
\end{exemple}
\end{entrée}

\begin{entrée}{pha-nõnõ}{}{ⓔpha-nõnõ}
\formephonétique{pʰa-ɳɔɳɔ}
\région{GOs}
(\domainesémantique{Fonctions intellectuelles})
\classe{v}
\begin{glose}
\pfra{faire se souvenir ; rappeler qqch à qqn}
\end{glose}
\newline
\begin{exemple}
\textbf{\pnua{pha-nõnõ çai je xo a-du-mi !}}
\pfra{rappelle-lui de descendre !}
\end{exemple}
\newline
\begin{exemple}
\textbf{\pnua{pha-nõnõmi-je xo a-du-mi !}}
\pfra{rappelle-lui de descendre !}
\end{exemple}
\newline
\relationsémantique{Cf.}{\lien{ⓔnõnõmiⓝpha-nõnõmi}{pha-nõnõmi}}
\glosecourte{rappeler à qqn}
\end{entrée}

\begin{entrée}{phaò}{}{ⓔphaò}
\région{GOs}
(\domainesémantique{Mouvements ou actions faits avec le corps, les bras, les mains, les pieds})
\classe{v}
\begin{glose}
\pfra{entasser}
\end{glose}
\end{entrée}

\begin{entrée}{phao nã}{}{ⓔphao nã}
\formephonétique{pʰao ɳɛ̃}
\région{GOs}
\variante{%
phao nããn
\région{PA}}
(\domainesémantique{Relations et interaction sociales})
\classe{v}
\begin{glose}
\pfra{injurier ; offenser (lit. jeter offense)}
\end{glose}
\newline
\begin{exemple}
\région{GOs}
\textbf{\pnua{e phao nãã-nu}}
\pfra{il m'a offensé}
\end{exemple}
\newline
\begin{exemple}
\région{PA}
\textbf{\pnua{i phao paxa-nãã-ny}}
\pfra{il m'a offensé}
\end{exemple}
\newline
\relationsémantique{Cf.}{\lien{}{tòè nãã-n [PA]}}
\glosecourte{injurier}
\end{entrée}

\begin{entrée}{phaöö}{}{ⓔphaöö}
\région{GOs}
\variante{%
phauvwo
\région{BO [Corne]}}
(\domainesémantique{Vents})
\classe{nom}
\begin{glose}
\pfra{tourbillon de vent ; trombe de vent}
\end{glose}
\end{entrée}

\begin{entrée}{phao paxa-nãn}{}{ⓔphao paxa-nãn}
\région{PA}
(\domainesémantique{Relations et interaction sociales})
\classe{nom}
\begin{glose}
\pfra{offense ; injure}
\end{glose}
\newline
\begin{exemple}
\région{PA}
\textbf{\pnua{i phao paxa-nãã-ny}}
\pfra{il m'a offensé}
\end{exemple}
\end{entrée}

\begin{entrée}{pharu}{}{ⓔpharu}
\région{BO}
(\domainesémantique{Aliments, alimentation})
\classe{v}
\begin{glose}
\pfra{boire chaud [Corne]}
\end{glose}
\end{entrée}

\begin{entrée}{pha-tuya}{}{ⓔpha-tuya}
\région{BO}
(\domainesémantique{Coutumes, dons coutumiers})
\classe{v}
\begin{glose}
\pfra{enlever un tabou ; rendre profane [Corne]}
\end{glose}
\newline
\note{non vérifié}{général}{}
\end{entrée}

\begin{entrée}{pha-tha}{}{ⓔpha-tha}
\région{GOs}
\variante{%
pha-tha
\région{PA}}
(\domainesémantique{Verbes d'action (en général)})
\classe{v}
\begin{glose}
\pfra{rater ; louper}
\end{glose}
\newline
\begin{exemple}
\région{GO}
\textbf{\pnua{e pha-tha xo Pwayili na ni mèni}}
\pfra{P a raté l'oiseau}
\end{exemple}
\newline
\begin{exemple}
\région{GO}
\textbf{\pnua{nu pha-tha na ni kar}}
\pfra{j'ai raté le car}
\end{exemple}
\newline
\begin{exemple}
\région{GO}
\textbf{\pnua{nu pha-tha}}
\pfra{j'ai raté (la cible)}
\end{exemple}
\newline
\relationsémantique{Cf.}{\lien{}{pao-tha}}
\glosecourte{lancer-rater}
\end{entrée}

\begin{entrée}{pha-tre-çaxoo-ni}{}{ⓔpha-tre-çaxoo-ni}
\formephonétique{,pʰa-ʈe-'ʒaɣoː-ɳi}
\région{GOs}
(\domainesémantique{Relations et interaction sociales})
\classe{v}
\begin{glose}
\pfra{apaiser ; calmer (qqn)}
\end{glose}
\end{entrée}

\begin{entrée}{phau}{}{ⓔphau}
\région{GO}
\variante{%
paup
\région{PA BO}}
(\domainesémantique{Vêtements, parure})
\classe{nom}
\begin{glose}
\pfra{coiffure en sparterie ; turban (des anciens) (Dubois ms)}
\end{glose}
\newline
\note{non vérifié}{général}{}
\end{entrée}

\begin{entrée}{phavwu}{}{ⓔphavwu}
\formephonétique{pʰaβu}
\région{GOs}
\variante{%
phavwuun
\région{PA}}
(\domainesémantique{Quantificateurs})
\classe{QNT}
\begin{glose}
\pfra{groupe de personnes (nombreuses) ; foule}
\end{glose}
\begin{glose}
\pfra{beaucoup}
\end{glose}
\newline
\begin{sous-entrée}{phavwu êgu}{ⓔphavwuⓝphavwu êgu}
\région{GO}
\begin{glose}
\pfra{une foule de gens}
\end{glose}
\newline
\begin{exemple}
\région{GO}
\textbf{\pnua{phavwu i la}}
\pfra{ils sont très nombreux}
\end{exemple}
\newline
\begin{exemple}
\région{PA}
\textbf{\pnua{phavwuu êgu}}
\pfra{il y a beaucoup de gens}
\end{exemple}
\newline
\begin{exemple}
\région{PA}
\textbf{\pnua{nu nooli êgu axe phavwuun}}
\pfra{j'ai vu beaucoup de gens (lit. j'ai vu beaucoup de gens et ils sont nombreux)}
\end{exemple}
\end{sous-entrée}
\end{entrée}

\begin{entrée}{phavwuun}{}{ⓔphavwuun}
\région{PA BO [BM]}
(\domainesémantique{Quantificateurs})
\classe{QNT}
\begin{glose}
\pfra{beaucoup}
\end{glose}
\newline
\begin{exemple}
\textbf{\pnua{phawuun nyama}}
\pfra{c'est beaucoup de travail}
\end{exemple}
\newline
\begin{exemple}
\textbf{\pnua{phawuu mwa}}
\pfra{il y a beaucoup de travail}
\end{exemple}
\end{entrée}

\begin{entrée}{phawe}{}{ⓔphawe}
\région{GOs BO}
(\domainesémantique{Coutumes, dons coutumiers})
\classe{v ; n}
\begin{glose}
\pfra{tas (d'ignames, de pierres)}
\end{glose}
\begin{glose}
\pfra{aligner les ignames ; faire des tas d'ignames (pour les cérémonies)}
\end{glose}
\newline
\begin{sous-entrée}{phawe kui}{ⓔphaweⓝphawe kui}
\begin{glose}
\pfra{tas d'ignames}
\end{glose}
\end{sous-entrée}
\newline
\begin{sous-entrée}{phawe paa}{ⓔphaweⓝphawe paa}
\begin{glose}
\pfra{tas de pierres}
\end{glose}
\end{sous-entrée}
\end{entrée}

\begin{entrée}{phaxee}{}{ⓔphaxee}
\formephonétique{pʰaɣeː}
\région{GOs}
\variante{%
phaxeen, phakeen
\région{BO PA}}
(\domainesémantique{Fonctions naturelles humaines})
\classe{v}
\begin{glose}
\pfra{écouter ; prêter l'oreille}
\end{glose}
\begin{glose}
\pfra{scruter}
\end{glose}
\end{entrée}

\begin{entrée}{phaza-}{}{ⓔphaza-}
\région{GOs PA}
\variante{%
para
\région{BO}}
(\domainesémantique{Causatif})
\classe{PREF.CAUS}
\begin{glose}
\pfra{faire (causatif)}
\end{glose}
\end{entrée}

\begin{entrée}{phaza-hããxe}{}{ⓔphaza-hããxe}
\région{GOs}
\variante{%
paza-hããxa, para-hããxe
\région{PA}}
(\domainesémantique{Relations et interaction sociales})
\classe{v}
\begin{glose}
\pfra{effrayer ; faire peur à qqn}
\end{glose}
\newline
\begin{exemple}
\région{GO}
\textbf{\pnua{hã pe-phaza-hããxa}}
\pfra{on joue à se faire peur}
\end{exemple}
\newline
\begin{exemple}
\région{GO}
\textbf{\pnua{e phaza-hããxe-nu}}
\pfra{il m'a fait peur}
\end{exemple}
\newline
\begin{exemple}
\région{GO}
\textbf{\pnua{e phaza-hããxe-ni ẽnõ}}
\pfra{il a fait peur à l'enfant}
\end{exemple}
\newline
\begin{exemple}
\région{PA}
\textbf{\pnua{la phaza-hããxee ẽnõ}}
\pfra{ils effrayent les enfants}
\end{exemple}
\end{entrée}

\begin{entrée}{phe}{}{ⓔphe}
\région{BO}
(\domainesémantique{Instruments})
\classe{nom}
\begin{glose}
\pfra{lime ; pierre à aiguiser [BM]}
\end{glose}
\end{entrée}

\begin{entrée}{phè}{}{ⓔphè}
\région{GOs PA BO}
\variante{%
phe
}
\classe{v}
\newline
\sens{1}
(\domainesémantique{Mouvements ou actions faits avec le corps, les bras, les mains, les pieds})
\begin{glose}
\pfra{prendre; enlever}
\end{glose}
\begin{glose}
\pfra{emporter}
\end{glose}
\newline
\begin{exemple}
\région{PA}
\textbf{\pnua{phe-vwo !}}
\pfra{sers-toi ! (emporte)}
\end{exemple}
\newline
\begin{sous-entrée}{phè-da-ò}{ⓔphèⓢ1ⓝphè-da-ò}
\région{GO}
\begin{glose}
\pfra{emporte-le en haut}
\end{glose}
\end{sous-entrée}
\newline
\begin{sous-entrée}{phè-du-ò}{ⓔphèⓢ1ⓝphè-du-ò}
\région{GO}
\begin{glose}
\pfra{emporte-le en bas}
\end{glose}
\end{sous-entrée}
\newline
\begin{sous-entrée}{phè-da-e}{ⓔphèⓢ1ⓝphè-da-e}
\région{GO}
\begin{glose}
\pfra{emporte-le (transverse)}
\end{glose}
\end{sous-entrée}
\newline
\sens{2}
(\domainesémantique{Portage})
\begin{glose}
\pfra{porter}
\end{glose}
\begin{glose}
\pfra{apporter}
\end{glose}
\newline
\begin{exemple}
\région{GOs}
\textbf{\pnua{e phe-du xo kêê-nu we-kò kui kòlò oã-nu}}
\pfra{mon père a apporté 3 ignames à ma mère (Doriane)}
\end{exemple}
\newline
\begin{exemple}
\région{GOs}
\textbf{\pnua{e phe-du kòlò oã-nu we-kò kui xo kêê-nu}}
\pfra{mon père a apporté 3 ignames à ma mère (Doriane)}
\end{exemple}
\newline
\begin{exemple}
\région{GOs}
\textbf{\pnua{e phe-da xo kêê-nu kòlò ewuuni la ci-kãbwa ponga nawo}}
\pfra{mon père a apporté des tissus à mon oncle pour la fête (Doriane)}
\end{exemple}
\newline
\begin{exemple}
\région{GOs}
\textbf{\pnua{e phe-da ci-kãbwa xo kêê-nu kòlò ewuuni ponga nawo}}
\pfra{mon père a apporté des tissus à mon oncle pour la fête (Doriane)}
\end{exemple}
\newline
\begin{exemple}
\région{PA}
\textbf{\pnua{chooval phe-vwo !}}
\pfra{cheval de somme}
\end{exemple}
\newline
\sens{3}
(\domainesémantique{Verbes d'action (en général)})
\begin{glose}
\pfra{recevoir}
\end{glose}
\begin{glose}
\pfra{obtenir}
\end{glose}
\newline
\sens{4}
(\domainesémantique{Relations et interaction sociales})
\begin{glose}
\pfra{accepter}
\end{glose}
\end{entrée}

\begin{entrée}{phê}{}{ⓔphê}
\région{GOs}
\variante{%
phêng
\région{PA}, 
pang, phang
\région{BO [BM]}}
(\domainesémantique{Noms des plantes})
\classe{nom}
\begin{glose}
\pfra{brède (sorte de pissenlit) ; laiteron}
\end{glose}
\nomscientifique{Sonchus oleraceus L. (Euphorbiacées)}
\newline
\begin{sous-entrée}{dròò-phê [GOs]}{ⓔphêⓝdròò-phê [GOs]}
\begin{glose}
\pfra{pissenlit}
\end{glose}
\end{sous-entrée}
\end{entrée}

\begin{entrée}{phè bala}{}{ⓔphè bala}
\région{GOs}
(\domainesémantique{Verbes d'action (en général)})
\classe{v}
\begin{glose}
\pfra{prendre la suite}
\end{glose}
\end{entrée}

\begin{entrée}{phè-caaxo}{}{ⓔphè-caaxo}
\région{GOs}
(\domainesémantique{Mouvements ou actions faits avec le corps, les bras, les mains, les pieds})
\classe{v}
\begin{glose}
\pfra{prendre en cachette}
\end{glose}
\newline
\begin{exemple}
\textbf{\pnua{e phè-caaxo-ni hèlè}}
\pfra{il a pris le couteau en cachette}
\end{exemple}
\end{entrée}

\begin{entrée}{pheege}{}{ⓔpheege}
\région{BO}
(\domainesémantique{Portage})
\classe{v}
\begin{glose}
\pfra{tenir dans ses bras [BM]}
\end{glose}
\newline
\begin{exemple}
\textbf{\pnua{i pheege hi-n}}
\pfra{il le tient dans ses bras}
\end{exemple}
\newline
\note{non verifié}{général}{}
\end{entrée}

\begin{entrée}{phele}{}{ⓔphele}
\région{PA BO}
(\domainesémantique{Mouvements ou actions faits avec le corps, les bras, les mains, les pieds})
\classe{v}
\begin{glose}
\pfra{attacher ; ceindre (manou)}
\end{glose}
\begin{glose}
\pfra{rouler (des torons)}
\end{glose}
\newline
\begin{exemple}
\région{PA}
\textbf{\pnua{i phele mãda i ye}}
\pfra{il met son manou}
\end{exemple}
\newline
\begin{exemple}
\région{BO}
\textbf{\pnua{nu phele kii-ny}}
\pfra{je mets mon manou}
\end{exemple}
\end{entrée}

\begin{entrée}{phè-mi}{}{ⓔphè-mi}
\région{GOs}
(\domainesémantique{Mouvements ou actions faits avec le corps, les bras, les mains, les pieds})
\classe{v}
\begin{glose}
\pfra{apporter}
\end{glose}
\newline
\relationsémantique{Cf.}{\lien{}{phe-ho}}
\glosecourte{emporte là-bas}
\end{entrée}

\begin{entrée}{phèńô}{}{ⓔphèńô}
\formephonétique{pʰɛnõ}
\région{GOs}
\variante{%
pènô
\région{PA BO}}
(\domainesémantique{Dons, échanges, achat et vente, vol})
\classe{v}
\begin{glose}
\pfra{voler ; dérober}
\end{glose}
\newline
\begin{exemple}
\région{GOs}
\textbf{\pnua{e no-phèńô}}
\pfra{il regarde furtivement}
\end{exemple}
\newline
\begin{exemple}
\région{BO}
\textbf{\pnua{a-pèno}}
\pfra{voleur}
\end{exemple}
\newline
\étymologie{
\langue{POc}
\étymon{*penako, *panako}
\auteur{Blust}}
\end{entrée}

\begin{entrée}{phènòng}{}{ⓔphènòng}
\région{BO}
(\domainesémantique{Religion, représentations religieuses})
\classe{v ; n}
\begin{glose}
\pfra{ensorceller [BM, Corne]}
\end{glose}
\newline
\begin{exemple}
\textbf{\pnua{i phènòng i nu}}
\pfra{il m'a jeté un sort}
\end{exemple}
\newline
\begin{sous-entrée}{a-phènòng}{ⓔphènòngⓝa-phènòng}
\begin{glose}
\pfra{sorcier}
\end{glose}
\end{sous-entrée}
\newline
\begin{sous-entrée}{pha-phènòng-e}{ⓔphènòngⓝpha-phènòng-e}
\begin{glose}
\pfra{ensorceller qqn}
\end{glose}
\end{sous-entrée}
\end{entrée}

\begin{entrée}{phe-vwo}{}{ⓔphe-vwo}
\région{GOs PA}
(\domainesémantique{Verbes d'action (en général)})
\classe{v}
\begin{glose}
\pfra{emporte-le ! ; sers-toi !}
\end{glose}
\newline
\begin{exemple}
\région{PA}
\textbf{\pnua{chooval phe-vwo}}
\pfra{cheval de somme}
\end{exemple}
\newline
\begin{exemple}
\région{PA}
\textbf{\pnua{phe-vwo !}}
\pfra{sers-toi !}
\end{exemple}
\end{entrée}

\begin{entrée}{phi}{1}{ⓔphiⓗ1}
\région{GOs}
\variante{%
phii-n
\région{PA BO}}
(\domainesémantique{Corps humain})
\classe{nom}
\begin{glose}
\pfra{sexe (de l'homme) ; pénis}
\end{glose}
\newline
\begin{exemple}
\région{BO}
\textbf{\pnua{hê-phii-n}}
\pfra{testicules}
\end{exemple}
\newline
\relationsémantique{Cf.}{\lien{ⓔcò}{cò}}
\glosecourte{sexe (de l'homme)}
\end{entrée}

\begin{entrée}{phi}{2}{ⓔphiⓗ2}
\région{GOs}
(\domainesémantique{Préparation des aliments; modes de préparation et de cuisson})
\classe{v}
\begin{glose}
\pfra{cuit (mal) ; pas cuit}
\end{glose}
\end{entrée}

\begin{entrée}{phidru}{}{ⓔphidru}
\région{GOs}
\variante{%
phiju
\région{BO}}
(\domainesémantique{Types de maison, architecture de la maison})
\classe{nom}
\begin{glose}
\pfra{paille du faîtage (formant un bourrelet)}
\end{glose}
\end{entrée}

\begin{entrée}{phiige}{}{ⓔphiige}
\région{GOs}
\variante{%
peenge
\région{BO [BM]}}
\classe{v}
\newline
\sens{1}
(\domainesémantique{Coutumes, dons coutumiers})
\begin{glose}
\pfra{rassembler de la nourriture (pour une coutume)}
\end{glose}
\newline
\sens{2}
(\domainesémantique{Feu : objets et actions liés au feu})
\begin{glose}
\pfra{rassembler/entasser les braises}
\end{glose}
\newline
\note{phiing (v.i), phii}{grammaire}{}
\end{entrée}

\begin{entrée}{phiing}{}{ⓔphiing}
\région{PA}
(\domainesémantique{Relations et interaction sociales})
\classe{v ; n}
\begin{glose}
\pfra{assemblée; rassembler}
\end{glose}
\end{entrée}

\begin{entrée}{phinãã}{}{ⓔphinãã}
\région{GOs PA BO}
(\domainesémantique{Fonctions intellectuelles})
\classe{v ; n}
\begin{glose}
\pfra{lire}
\end{glose}
\begin{glose}
\pfra{compter ; compter sur qqn ; nombre}
\end{glose}
\newline
\begin{exemple}
\région{GOs}
\textbf{\pnua{i phinãã bwa xai poi-li}}
\pfra{ils comptent sur leur enfant}
\end{exemple}
\end{entrée}

\begin{entrée}{phiri kûû}{}{ⓔphiri kûû}
\formephonétique{pʰiri kũː}
\région{GOs}
\variante{%
phili kûû
\formephonétique{pʰili kũː}
\région{GO(s)}}
(\domainesémantique{Sons, bruits})
\classe{v ; n}
\begin{glose}
\pfra{son d'une vibration (comme un boomerang) ; plaque vibrante (musique)}
\end{glose}
\begin{glose}
\pfra{vibrer ; faire du bruit}
\end{glose}
\newline
\begin{exemple}
\textbf{\pnua{e phiri kûû ã loto}}
\pfra{cette voiture vibre}
\end{exemple}
\end{entrée}

\begin{entrée}{phiu}{}{ⓔphiu}
\région{GOs}
(\domainesémantique{Arbre})
\classe{nom}
\begin{glose}
\pfra{tamanou}
\end{glose}
\nomscientifique{Calophyllum inophyllum L. (Guttifères)}
\end{entrée}

\begin{entrée}{phivwâi}{}{ⓔphivwâi}
\région{GOs}
(\domainesémantique{Noms des plantes})
\classe{nom}
\begin{glose}
\pfra{pandanus sauvage (bord de creek, sert à tresser, au bord des creek, les roussettes mangent les fruits)}
\end{glose}
\end{entrée}

\begin{entrée}{pho}{}{ⓔpho}
\région{GOs BO}
(\domainesémantique{Tressage})
\classe{nom}
\begin{glose}
\pfra{feuille de pandanus sèche pour le tressage (dont on a enlevé les piquants)}
\end{glose}
\newline
\begin{exemple}
\textbf{\pnua{la thu pho}}
\pfra{elles font de la vannerie}
\end{exemple}
\end{entrée}

\begin{entrée}{phò}{1}{ⓔphòⓗ1}
\région{GOs BO}
\variante{%
pho-
\région{PA}}
\classe{nom}
\newline
\sens{1}
(\domainesémantique{Préfixes classificateurs possessifs})
\begin{glose}
\pfra{charge ; mission ; fardeau (métaph. se mêler de)}
\end{glose}
\newline
\begin{exemple}
\textbf{\pnua{phò-ce [GOs, BO]}}
\pfra{fagot / tas de bois}
\end{exemple}
\newline
\begin{exemple}
\région{BO}
\textbf{\pnua{phò-la nò}}
\pfra{leur charge de poisson}
\end{exemple}
\newline
\begin{exemple}
\région{PA}
\textbf{\pnua{phò-m (a) nò}}
\pfra{tu es chargé d'apporter du poisson}
\end{exemple}
\newline
\begin{exemple}
\région{PA}
\textbf{\pnua{phò-m (a) tha kui}}
\pfra{tu es chargé de déterrer le ignames}
\end{exemple}
\newline
\begin{exemple}
\région{GOs}
\textbf{\pnua{phò-nu phò-ce}}
\pfra{ma charge de fagot de bois}
\end{exemple}
\newline
\begin{exemple}
\région{PA}
\textbf{\pnua{pho-m}}
\pfra{ta charge, ton fardeau}
\end{exemple}
\newline
\begin{exemple}
\région{GOs}
\textbf{\pnua{mô vara pu phò-ã}}
\pfra{nous avons chacunnotre charge/mission}
\end{exemple}
\newline
\begin{exemple}
\région{GOs}
\textbf{\pnua{da nye phò-nu ?}}
\pfra{que dois-je apporter?}
\end{exemple}
\newline
\begin{exemple}
\région{GOs}
\textbf{\pnua{phò-nu lai ?}}
\pfra{je dois apporter du riz}
\end{exemple}
\newline
\begin{exemple}
\région{GOs}
\textbf{\pnua{phò-kamyô}}
\pfra{la charge du camion}
\end{exemple}
\newline
\begin{exemple}
\région{GOs}
\textbf{\pnua{da nye phò-çö na ni jena ?}}
\pfra{de quoi tu te mêles dans cette affaire?}
\end{exemple}
\newline
\begin{exemple}
\région{GOs}
\textbf{\pnua{kixa phò-çö na le?}}
\pfra{tu n'as rien à voir avec cela}
\end{exemple}
\newline
\begin{exemple}
\région{GOs}
\textbf{\pnua{novwö phò-je na xòlò Tea-ma ça/ca a-phe-fhaa?}}
\pfra{quant à sa mission auprès du chef c'est d'être le porte-parole}
\end{exemple}
\newline
\sens{2}
(\domainesémantique{Préfixes classificateurs possessifs})
\begin{glose}
\pfra{part}
\end{glose}
\newline
\begin{exemple}
\région{GOs}
\textbf{\pnua{phòò-nu lai}}
\pfra{ma part de riz qu'on doit apporter dans une coutume}
\end{exemple}
\newline
\étymologie{
\langue{POc}
\étymon{*papa}
\glosecourte{porter sur le dos}}
\end{entrée}

\begin{entrée}{phò}{2}{ⓔphòⓗ2}
\région{GOs}
(\domainesémantique{Quantificateurs})
\classe{QNT ; DISTR}
\begin{glose}
\pfra{tous}
\end{glose}
\begin{glose}
\pfra{chacun}
\end{glose}
\newline
\begin{exemple}
\textbf{\pnua{nu phò zala-la}}
\pfra{je leur ai demandé à chacun}
\end{exemple}
\newline
\begin{exemple}
\textbf{\pnua{nu phò zala-la kûûni}}
\pfra{je leur ai demandé à tous}
\end{exemple}
\newline
\begin{exemple}
\textbf{\pnua{nu phò-phe kûûni}}
\pfra{j'ai tout pris}
\end{exemple}
\end{entrée}

\begin{entrée}{phò-}{}{ⓔphò-}
\région{GOs}
(\domainesémantique{Préfixes classificateurs numériques})
\classe{CLF.NUM}
\begin{glose}
\pfra{charge ; fardeau}
\end{glose}
\newline
\note{compter ce qu'on apporte (dans une coutume)}{glose}{}
\newline
\begin{exemple}
\textbf{\pnua{phò-xe, phò-tru, etc.}}
\pfra{un tas, deux tas, etc.}
\end{exemple}
\newline
\begin{exemple}
\région{PA}
\textbf{\pnua{phò-xe phò-ce}}
\pfra{un fagot de bois à porter}
\end{exemple}
\end{entrée}

\begin{entrée}{phõ}{}{ⓔphõ}
\région{GOs}
\variante{%
phõng
\région{PA BO WE WEM}}
(\domainesémantique{Description des objets, formes, consistance, taille})
\classe{v.i.}
\begin{glose}
\pfra{tordu ; tors}
\end{glose}
\newline
\begin{exemple}
\région{PA}
\textbf{\pnua{i a phõng-ò phõng-mi}}
\pfra{il va en zigzag}
\end{exemple}
\newline
\begin{exemple}
\région{BO}
\textbf{\pnua{i baxòòl ha i phõng ?}}
\pfra{il est droit ou tordu ? (BM)}
\end{exemple}
\newline
\relationsémantique{Cf.}{\lien{ⓔphõge}{phõge}}
\glosecourte{tordre qqch}
\end{entrée}

\begin{entrée}{phò-ce}{}{ⓔphò-ce}
\région{GOs}
(\domainesémantique{Feu : objets et actions liés au feu
, Bois
, Configuration des objets})
\classe{nom}
\begin{glose}
\pfra{fagot de bois}
\end{glose}
\newline
\begin{exemple}
\région{GOs}
\textbf{\pnua{phò-nu phò-ce}}
\pfra{mon fagot de bois}
\end{exemple}
\end{entrée}

\begin{entrée}{phoê}{}{ⓔphoê}
\région{GOs PA}
(\domainesémantique{Noms des plantes})
\classe{nom}
\begin{glose}
\pfra{plante (générique)}
\end{glose}
\begin{glose}
\pfra{cultures}
\end{glose}
\newline
\begin{exemple}
\région{PA}
\textbf{\pnua{e wara thu phoê}}
\pfra{c'est l'époque de faire les champs}
\end{exemple}
\end{entrée}

\begin{entrée}{phò-ê}{}{ⓔphò-ê}
\région{GOs}
(\domainesémantique{Configuration des objets})
\classe{nom}
\begin{glose}
\pfra{fagot de canne à sucre (dans les cérémonies)}
\end{glose}
\end{entrée}

\begin{entrée}{phöge}{}{ⓔphöge}
\région{GOs}
(\domainesémantique{Mouvements ou actions faits avec le corps, les bras, les mains, les pieds})
\classe{v}
\begin{glose}
\pfra{entasser ; faire des tas}
\end{glose}
\newline
\begin{exemple}
\textbf{\pnua{phöge hopo}}
\pfra{faire des tas de nourriture}
\end{exemple}
\end{entrée}

\begin{entrée}{phõge}{}{ⓔphõge}
\région{GOs PA}
\variante{%
phõng
\région{BO}}
\classe{v}
\newline
\sens{1}
(\domainesémantique{Mouvements ou actions faits avec le corps, les bras, les mains, les pieds})
\begin{glose}
\pfra{tordre (en revenant vers l'arrière)}
\end{glose}
\newline
\sens{2}
(\domainesémantique{Description des objets, formes, consistance, taille})
\begin{glose}
\pfra{tordu}
\end{glose}
\newline
\relationsémantique{Cf.}{\lien{ⓔphõ}{phõ}}
\glosecourte{tordu}
\end{entrée}

\begin{entrée}{phò-kui}{}{ⓔphò-kui}
\région{GOs}
(\domainesémantique{Coutumes, dons coutumiers
, Configuration des objets})
\classe{nom}
\begin{glose}
\pfra{tas d'igname (dans les cérémonies)}
\end{glose}
\end{entrée}

\begin{entrée}{phole}{}{ⓔphole}
\région{GOs}
(\domainesémantique{Actions liées aux éléments (liquide, fumée)})
\classe{v}
\begin{glose}
\pfra{tremper dans l'eau}
\end{glose}
\newline
\étymologie{
\langue{POc}
\étymon{*puraq}
\glosecourte{immerse}}
\newline
\note{pole (en nêlêmwa)}{général}{}
\end{entrée}

\begin{entrée}{pholo}{}{ⓔpholo}
\région{GOs BO}
(\domainesémantique{Pêche
, Mouvements ou actions faits avec le corps, les bras, les mains, les pieds})
\classe{v}
\begin{glose}
\pfra{taper dans l'eau pour faire du bruit et effrayer le poisson et l'amener dans le filet}
\end{glose}
\newline
\begin{sous-entrée}{thi-pholo}{ⓔpholoⓝthi-pholo}
\begin{glose}
\pfra{taper dans l'eau}
\end{glose}
\end{sous-entrée}
\end{entrée}

\begin{entrée}{phòlò ja}{}{ⓔphòlò ja}
\région{GOs}
\variante{%
phòlò jang
\région{PA}}
(\domainesémantique{Description des objets, formes, consistance, taille})
\classe{nom}
\begin{glose}
\pfra{objet ou bois flotté (apporté par la mer ou la rivière)}
\end{glose}
\begin{glose}
\pfra{débris laissés par l'inondation}
\end{glose}
\end{entrée}

\begin{entrée}{phöloo}{}{ⓔphöloo}
\région{GOs PA}
\variante{%
phuloo
\région{BO}, 
phuluu
\région{WEM}}
(\domainesémantique{Eau})
\classe{v}
\begin{glose}
\pfra{trouble (eau) ; sale (eau) ; limoneux}
\end{glose}
\newline
\begin{sous-entrée}{thu phulu}{ⓔphölooⓝthu phulu}
\begin{glose}
\pfra{troubler (l'eau)}
\end{glose}
\newline
\begin{exemple}
\région{WEM}
\textbf{\pnua{la ra u mara thu-wuluu (thu phöloo)}}
\pfra{ils ont encore plus troublé l'eau}
\end{exemple}
\end{sous-entrée}
\end{entrée}

\begin{entrée}{phöloo we}{}{ⓔphöloo we}
\région{GOs}
(\domainesémantique{Eau})
\classe{nom}
\begin{glose}
\pfra{eau trouble}
\end{glose}
\end{entrée}

\begin{entrée}{phònõ}{}{ⓔphònõ}
\formephonétique{pʰɔɳɔ̃}
\région{GOs}
\variante{%
phònõng
\région{PA}}
(\domainesémantique{Religion, représentations religieuses})
\classe{v}
\begin{glose}
\pfra{ensorceler ; jeter des sorts}
\end{glose}
\newline
\begin{exemple}
\textbf{\pnua{e phònõnge-nu}}
\pfra{il m'a ensorcelé}
\end{exemple}
\newline
\begin{exemple}
\textbf{\pnua{e a-phònõng}}
\pfra{c'est un ensorceleur}
\end{exemple}
\newline
\note{phònõge (v.t.)}{grammaire}{}
\end{entrée}

\begin{entrée}{phoo}{}{ⓔphoo}
\région{GOs BO}
\variante{%
phwoo
\région{GO(s)}}
(\domainesémantique{Verbes d'action (en général)})
\classe{v}
\begin{glose}
\pfra{remplir ; charger}
\end{glose}
\newline
\begin{exemple}
\textbf{\pnua{la phoo-da kamyõ}}
\pfra{ils chargent le camion}
\end{exemple}
\newline
\begin{exemple}
\textbf{\pnua{e phoo-du nye keruau}}
\pfra{elle remplit les paniers}
\end{exemple}
\newline
\begin{exemple}
\région{BO}
\textbf{\pnua{pho-du a we}}
\pfra{remplis la marmite d'eau}
\end{exemple}
\end{entrée}

\begin{entrée}{phòò}{}{ⓔphòò}
\formephonétique{pʰɔ}
\région{GOs}
\variante{%
phòòl
\formephonétique{pʰɔːl}
\région{BO PA}, 
pòòl, pwòl
\région{BO}}
(\domainesémantique{Fonctions naturelles humaines})
\classe{v}
\begin{glose}
\pfra{déféquer}
\end{glose}
\end{entrée}

\begin{entrée}{phöö}{}{ⓔphöö}
\région{GOs}
(\domainesémantique{Préfixes et verbes de position})
\classe{v}
\begin{glose}
\pfra{tourner sur le ventre (se)}
\end{glose}
\newline
\begin{exemple}
\région{GOs}
\textbf{\pnua{e phöö ẽnõ}}
\pfra{l'enfant s'est mis sur le ventre}
\end{exemple}
\end{entrée}

\begin{entrée}{phõõme}{}{ⓔphõõme}
\région{GOs}
\variante{%
phõõm
\région{BO}}
(\domainesémantique{Préparation des aliments; modes de préparation et de cuisson})
\classe{v}
\begin{glose}
\pfra{envelopper de la nourriture dans des feuilles pour les faire cuire}
\end{glose}
\end{entrée}

\begin{entrée}{phòtrõ}{}{ⓔphòtrõ}
\formephonétique{pʰɔʈɔ̃, pʰɔ'ɽɔ̃}
\région{GOs}
\variante{%
phòrõ
\région{GO(s)}, 
pòròm
\région{PA BO}}
(\domainesémantique{Fonctions intellectuelles})
\classe{v ; n}
\begin{glose}
\pfra{oublier ; pardonner ; pardon}
\end{glose}
\newline
\begin{exemple}
\région{GO}
\textbf{\pnua{phòrõ-nu !}}
\pfra{pardonne moi}
\end{exemple}
\newline
\begin{exemple}
\région{GO}
\textbf{\pnua{e phòrõ-nu}}
\pfra{il m'a pardonné}
\end{exemple}
\newline
\begin{exemple}
\région{GO}
\textbf{\pnua{e pe-phòrõme ne a-mi}}
\pfra{il a oublié de venir}
\end{exemple}
\newline
\begin{exemple}
\région{GO}
\textbf{\pnua{e a-phòrõ}}
\pfra{il oublie souvent}
\end{exemple}
\newline
\begin{exemple}
\région{GO}
\textbf{\pnua{e a-ka-phòrõ}}
\pfra{il perd la mémoire (maladie)}
\end{exemple}
\newline
\begin{exemple}
\région{BO}
\textbf{\pnua{nu tilò pòròm yai caaya-m}}
\pfra{j'ai demandé pardon à ton père}
\end{exemple}
\newline
\note{phòrõme (v.t.)}{grammaire}{oublier qqch}
\end{entrée}

\begin{entrée}{phòtrõme}{}{ⓔphòtrõme}
\formephonétique{pʰɔ'ʈɔ̃me, pʰɔ'ɽɔ̃me}
\région{GOs}
\variante{%
phòrõme
\région{PA}, 
pòrõme
\région{BO}}
(\domainesémantique{Fonctions intellectuelles})
\classe{v}
\begin{glose}
\pfra{oublier ; pardonner}
\end{glose}
\newline
\begin{exemple}
\région{BO}
\textbf{\pnua{i pòrò-nu}}
\pfra{il m'a pardonné}
\end{exemple}
\newline
\begin{exemple}
\région{BO}
\textbf{\pnua{nu tilò pòròm yai caaya-m}}
\pfra{j'ai demandé pardon à ton père}
\end{exemple}
\end{entrée}

\begin{entrée}{phowôgo}{}{ⓔphowôgo}
\région{GOs}
(\domainesémantique{Eau})
\classe{nom}
\begin{glose}
\pfra{écume}
\end{glose}
\end{entrée}

\begin{entrée}{phozo}{}{ⓔphozo}
\formephonétique{pʰoðo}
\région{GOs}
\variante{%
polo, pulo
\région{PA BO}}
\classe{v.stat.}
\newline
\sens{1}
(\domainesémantique{Couleurs})
\begin{glose}
\pfra{blanc}
\end{glose}
\newline
\begin{sous-entrée}{êgu polo}{ⓔphozoⓢ1ⓝêgu polo}
\begin{glose}
\pfra{personne albinos}
\end{glose}
\end{sous-entrée}
\newline
\sens{2}
(\domainesémantique{Description des objets, formes, consistance, taille})
\begin{glose}
\pfra{propre ; neuf}
\end{glose}
\newline
\begin{exemple}
\textbf{\pnua{e phozo zo}}
\pfra{il est propre (personne)}
\end{exemple}
\newline
\begin{exemple}
\textbf{\pnua{e phozo}}
\pfra{il est propre, neuf (objet)}
\end{exemple}
\end{entrée}

\begin{entrée}{phòzo}{}{ⓔphòzo}
\formephonétique{pʰɔðo}
\région{GOs}
\variante{%
phòlo
\région{PA BO}}
(\domainesémantique{Aliments, alimentation})
\classe{v}
\begin{glose}
\pfra{boire chaud (soupe)}
\end{glose}
\begin{glose}
\pfra{manger du non solide (bouillie, purée, soupe)}
\end{glose}
\end{entrée}

\begin{entrée}{phòzò}{}{ⓔphòzò}
\formephonétique{pʰɔðɔ}
\région{GOs}
\variante{%
phòlò
\région{PA}}
(\domainesémantique{Verbes d'action (en général)})
\classe{v}
\begin{glose}
\pfra{casser (se)}
\end{glose}
\end{entrée}

\begin{entrée}{phu}{}{ⓔphu}
\région{BO}
(\domainesémantique{Santé, maladie})
\classe{v ; n}
\begin{glose}
\pfra{tousser ; grippe[Corne]}
\end{glose}
\newline
\begin{exemple}
\région{BO}
\textbf{\pnua{i tooli phu}}
\pfra{il a attrapé la grippe}
\end{exemple}
\newline
\note{non vérifié}{général}{}
\end{entrée}

\begin{entrée}{phu}{1}{ⓔphuⓗ1}
\région{GOs}
\variante{%
phul
\région{PA BO}}
(\domainesémantique{Préparation des aliments; modes de préparation et de cuisson})
\classe{v}
\begin{glose}
\pfra{bouillir (eau)}
\end{glose}
\begin{glose}
\pfra{cuit (être)}
\end{glose}
\begin{glose}
\pfra{prêt (être)}
\end{glose}
\newline
\begin{exemple}
\région{GOs}
\textbf{\pnua{la phu dröö !}}
\pfra{les marmites sont prêtes (on peut les retirer du feu) !}
\end{exemple}
\newline
\begin{exemple}
\région{GOs}
\textbf{\pnua{e phu (we) !}}
\pfra{ça bout !}
\end{exemple}
\newline
\begin{exemple}
\région{BO}
\textbf{\pnua{i phul a we !}}
\pfra{l'eaubout !}
\end{exemple}
\newline
\étymologie{
\langue{POc}
\étymon{*puro}}
\end{entrée}

\begin{entrée}{phu}{2}{ⓔphuⓗ2}
\région{GOs PA BO}
\classe{nom}
\newline
\sens{1}
(\domainesémantique{Noms des plantes})
\begin{glose}
\pfra{herbe à paille à tige longue (pousse en touffe, utilisée pour le torchis)}
\end{glose}
\nomscientifique{Themeda triandra (graminées)}
\newline
\sens{2}
(\domainesémantique{Types de maison, architecture de la maison})
\begin{glose}
\pfra{premier rang de paille au bord du toit ; rebord du toit de chaume}
\end{glose}
\newline
\begin{exemple}
\région{GOs}
\textbf{\pnua{e thu phu}}
\pfra{il met la première rangée de paille}
\end{exemple}
\end{entrée}

\begin{entrée}{phu}{3}{ⓔphuⓗ3}
\région{GOs}
\variante{%
phuxa
\région{PA}}
\classe{v}
\newline
\sens{1}
(\domainesémantique{Vêtements, parure})
\begin{glose}
\pfra{enlever (chapeau)}
\end{glose}
\newline
\begin{exemple}
\région{GOs}
\textbf{\pnua{e phu mwêêxa-je}}
\pfra{il enlève son chapeau}
\end{exemple}
\newline
\begin{exemple}
\région{PA}
\textbf{\pnua{i phu hau-n}}
\pfra{il enlève son chapeau}
\end{exemple}
\newline
\sens{2}
(\domainesémantique{Préparation des aliments; modes de préparation et de cuisson})
\begin{glose}
\pfra{retirer (marmite du feu)}
\end{glose}
\newline
\begin{exemple}
\région{GOs}
\textbf{\pnua{phuu drööa-wa !}}
\pfra{retirez votre marmite (du feu) !}
\end{exemple}
\newline
\relationsémantique{Cf.}{\lien{}{phuxa [PA]}}
\glosecourte{enlever}
\end{entrée}

\begin{entrée}{phu}{4}{ⓔphuⓗ4}
\région{GOs PA}
\variante{%
phwu
\région{BO}}
(\domainesémantique{Cultures, techniques, boutures})
\classe{v}
\begin{glose}
\pfra{arracher (paille, taro d'eau)}
\end{glose}
\newline
\begin{exemple}
\textbf{\pnua{i phu kuru}}
\pfra{elle arrache le taro d'eau}
\end{exemple}
\newline
\note{phugi (v.t.)}{grammaire}{}
\end{entrée}

\begin{entrée}{phu}{5}{ⓔphuⓗ5}
\région{GOs BO PA}
(\domainesémantique{Verbes d'action faite par des animaux})
\classe{v}
\begin{glose}
\pfra{voler (oiseau)}
\end{glose}
\newline
\étymologie{
\langue{POc}
\étymon{*Ropok}}
\end{entrée}

\begin{entrée}{phû}{}{ⓔphû}
\région{GOs}
\variante{%
phûny
\région{WEM BO PA}}
(\domainesémantique{Couleurs})
\classe{v.stat.}
\begin{glose}
\pfra{bleu ; vert}
\end{glose}
\newline
\begin{exemple}
\région{GOs}
\textbf{\pnua{e phû phãgoo mèni}}
\pfra{cet oiseau est vert (lit. le corps de l'oiseau est vert)}
\end{exemple}
\newline
\begin{exemple}
\région{GOs}
\textbf{\pnua{nooli me-phû-wa dònò !}}
\pfra{regarde le bleu du ciel !}
\end{exemple}
\newline
\begin{exemple}
\région{PA}
\textbf{\pnua{nèn phûny}}
\pfra{mouche verte}
\end{exemple}
\end{entrée}

\begin{entrée}{phubwe}{}{ⓔphubwe}
\région{GOs BO}
\variante{%
phöbwe
\région{GO(s)}, 
phobwe
\région{BO}}
(\domainesémantique{Préparation des aliments; modes de préparation et de cuisson})
\classe{v}
\begin{glose}
\pfra{passer sur la flamme pour assouplir les feuilles}
\end{glose}
\begin{glose}
\pfra{passer sur la flamme des anguilles pour enlever la couche gluante}
\end{glose}
\newline
\begin{exemple}
\région{BO}
\textbf{\pnua{la phubwe dòò-chaamwa}}
\pfra{ils passer les feuilles de bananier sur la flamme}
\end{exemple}
\newline
\relationsémantique{Cf.}{\lien{ⓔûne}{ûne}}
\glosecourte{passer sur la flamme}
\end{entrée}

\begin{entrée}{phue}{}{ⓔphue}
\région{GOs}
(\domainesémantique{Instruments})
\classe{nom}
\begin{glose}
\pfra{fouet}
\end{glose}
\newline
\begin{exemple}
\textbf{\pnua{thila phue}}
\pfra{le bout/mèche du fouet}
\end{exemple}
\newline
\emprunt{fouet (FR)}
\end{entrée}

\begin{entrée}{phugi}{}{ⓔphugi}
\région{GOs PA BO}
\classe{v.t.}
\newline
\sens{1}
(\domainesémantique{Actions liées aux plantes})
\begin{glose}
\pfra{arracher (à la main des herbes ou des plantes à racine : kumala, kuru, etc., sur une petite surface)}
\end{glose}
\begin{glose}
\pfra{décoller ; enlever [BO]}
\end{glose}
\newline
\begin{exemple}
\région{GOs}
\textbf{\pnua{phugi drawalö !}}
\pfra{arrache l'herbe !}
\end{exemple}
\newline
\begin{exemple}
\région{PA}
\textbf{\pnua{i phugi mae !}}
\pfra{elle arrache la paille}
\end{exemple}
\newline
\sens{2}
(\domainesémantique{Remèdes, médecine})
\begin{glose}
\pfra{cueillir des herbes magiques [BO]}
\end{glose}
\newline
\note{phu (v.i.)}{grammaire}{}
\end{entrée}

\begin{entrée}{phu ko}{}{ⓔphu ko}
(\domainesémantique{Santé, maladie})
\classe{nom}
\begin{glose}
\pfra{elephantiasis (lit. jambe qui enfle)}
\end{glose}
\end{entrée}

\begin{entrée}{phu kuru}{}{ⓔphu kuru}
\région{PA}
(\domainesémantique{Cultures, techniques, boutures})
\classe{v}
\begin{glose}
\pfra{arracher les taros d'eau}
\end{glose}
\newline
\note{phugi (v.t.)}{grammaire}{}
\end{entrée}

\begin{entrée}{phulè}{}{ⓔphulè}
\région{PA BO}
\variante{%
phulèng
, 
phulek
\région{BO (Corne)}}
(\domainesémantique{Noms des plantes})
\classe{nom}
\begin{glose}
\pfra{Urticacée (les fibres servent à faire des nasses, cordes de fronde)}
\end{glose}
\nomscientifique{Pipturus argenteus, Urticacée}
\newline
\begin{sous-entrée}{pò-puleng}{ⓔphulèⓝpò-puleng}
\begin{glose}
\pfra{fruit de Pipturus}
\end{glose}
\newline
\relationsémantique{Cf.}{\lien{ⓔalamwi}{alamwi}}
\end{sous-entrée}
\end{entrée}

\begin{entrée}{phumõ}{}{ⓔphumõ}
\région{GOs}
\variante{%
pumõ
\région{PA}}
(\domainesémantique{Fonctions intellectuelles
, Discours, échanges verbaux})
\classe{v}
\begin{glose}
\pfra{discourir ; faire un discours (coutumier)}
\end{glose}
\begin{glose}
\pfra{conseiller}
\end{glose}
\begin{glose}
\pfra{prêcher ; sermonner}
\end{glose}
\newline
\begin{exemple}
\région{GOs}
\textbf{\pnua{e phumõ nu}}
\pfra{il m'a conseillé}
\end{exemple}
\newline
\begin{exemple}
\région{PA}
\textbf{\pnua{e pumõ nu}}
\pfra{il m'a conseillé}
\end{exemple}
\newline
\begin{exemple}
\région{GOs}
\textbf{\pnua{e phumõ-ni pòi-nu}}
\pfra{il a conseillé mon enfant}
\end{exemple}
\newline
\begin{exemple}
\région{GOs}
\textbf{\pnua{e pe-phumõ}}
\pfra{il fait son prêche}
\end{exemple}
\newline
\begin{exemple}
\région{GOs}
\textbf{\pnua{e a-phumõ}}
\pfra{il prêche}
\end{exemple}
\newline
\begin{exemple}
\région{PA}
\textbf{\pnua{li pe-pumõ}}
\pfra{ils se sont consultés, ont pris conseil l'un auprès de l'autre}
\end{exemple}
\newline
\note{phumõ-ni (v.t)}{grammaire}{}
\end{entrée}

\begin{entrée}{phumwêê}{}{ⓔphumwêê}
\région{GOs}
(\domainesémantique{Navigation})
\classe{v}
\begin{glose}
\pfra{flotter}
\end{glose}
\end{entrée}

\begin{entrée}{phunò}{}{ⓔphunò}
\formephonétique{pʰuɳɔ}
\région{GOs}
(\domainesémantique{Coutumes, dons coutumiers})
\classe{nom}
\begin{glose}
\pfra{coutume (de deuil ou de mariage)}
\end{glose}
\begin{glose}
\pfra{ornement de deuil}
\end{glose}
\newline
\begin{exemple}
\région{GOs}
\textbf{\pnua{phunò mã}}
\pfra{coutume de deuil}
\end{exemple}
\end{entrée}

\begin{entrée}{phu nõbu}{}{ⓔphu nõbu}
\formephonétique{pʰu ɳɔ̃bu}
\région{GOs}
(\domainesémantique{Coutumes, dons coutumiers})
\classe{v}
\begin{glose}
\pfra{enlever un interdit}
\end{glose}
\newline
\relationsémantique{Cf.}{\lien{ⓔkhabe nõbu}{khabe nõbu}}
\glosecourte{mettre un interdit}
\end{entrée}

\begin{entrée}{phu-nõgo}{}{ⓔphu-nõgo}
\région{GOs}
(\domainesémantique{Santé, maladie})
\classe{v}
\begin{glose}
\pfra{tousser ; avoir la grippe}
\end{glose}
\end{entrée}

\begin{entrée}{phunò mã}{}{ⓔphunò mã}
\formephonétique{pʰuɳɔ}
\région{GOs}
(\domainesémantique{Coutumes, dons coutumiers})
\classe{nom}
\begin{glose}
\pfra{coutume de deuil}
\end{glose}
\end{entrée}

\begin{entrée}{phuri}{}{ⓔphuri}
\région{BO}
(\domainesémantique{Santé, maladie})
\classe{v}
\begin{glose}
\pfra{grossir [BM]}
\end{glose}
\end{entrée}

\begin{entrée}{phuu}{}{ⓔphuu}
\région{GOs PA BO}
\variante{%
phuuxu
\région{GO}}
\classe{v.i.}
\newline
\sens{1}
(\domainesémantique{Verbes de mouvement})
\begin{glose}
\pfra{monter (eau) ; déborder}
\end{glose}
\newline
\begin{exemple}
\région{GOs}
\textbf{\pnua{e phuu we}}
\pfra{l'eau monte (rivière)}
\end{exemple}
\newline
\begin{exemple}
\région{GOs}
\textbf{\pnua{e phuuxu we}}
\pfra{l'eau déborde (rivière)}
\end{exemple}
\newline
\begin{exemple}
\région{PA}
\textbf{\pnua{i phuu we}}
\pfra{l'eau monte (rivière)}
\end{exemple}
\newline
\sens{2}
(\domainesémantique{Santé, maladie})
\begin{glose}
\pfra{gonfler ; enfler (membre, partie du corps)}
\end{glose}
\newline
\begin{exemple}
\région{GOs}
\textbf{\pnua{phuu mee-je}}
\pfra{ses yeux enflent}
\end{exemple}
\newline
\note{pha-phuu-ni}{grammaire}{faire gonfler qqch.}
\end{entrée}

\begin{entrée}{phuvwu}{}{ⓔphuvwu}
\formephonétique{pʰuβu}
\région{GOs}
\variante{%
phuul
\région{BO PA}}
(\domainesémantique{Fonctions naturelles humaines})
\classe{v}
\begin{glose}
\pfra{ronfler}
\end{glose}
\end{entrée}

\begin{entrée}{phuxa}{}{ⓔphuxa}
\région{PA}
(\domainesémantique{Verbes d'action (en général)})
\classe{v}
\begin{glose}
\pfra{enlever}
\end{glose}
\newline
\begin{exemple}
\textbf{\pnua{i phuxa hau-n}}
\pfra{il enlève son chapeau}
\end{exemple}
\end{entrée}

\begin{entrée}{phuzi}{}{ⓔphuzi}
\formephonétique{pʰuði}
\région{GOs}
(\domainesémantique{Caractéristiques et propriétés des personnes})
\classe{v.stat.}
\begin{glose}
\pfra{gras}
\end{glose}
\end{entrée}

\newpage

\lettrine{pw}\begin{entrée}{pwa}{1}{ⓔpwaⓗ1}
\région{GOs}
(\domainesémantique{Poissons})
\classe{nom}
\begin{glose}
\pfra{barracuda}
\end{glose}
\end{entrée}

\begin{entrée}{pwa}{2}{ⓔpwaⓗ2}
\région{GOs}
\variante{%
pwal
\région{PA}}
(\domainesémantique{Phénomènes atmosphériques et naturels})
\classe{nom}
\begin{glose}
\pfra{pluie}
\end{glose}
\newline
\begin{sous-entrée}{kule pwal}{ⓔpwaⓗ2ⓝkule pwal}
\région{PA}
\begin{glose}
\pfra{pleuvoir}
\end{glose}
\end{sous-entrée}
\newline
\begin{sous-entrée}{we pwal}{ⓔpwaⓗ2ⓝwe pwal}
\région{PA}
\begin{glose}
\pfra{eau de pluie}
\end{glose}
\end{sous-entrée}
\newline
\begin{sous-entrée}{po(o) pwal}{ⓔpwaⓗ2ⓝpo(o) pwal}
\région{PA}
\begin{glose}
\pfra{goutte de pluie}
\end{glose}
\end{sous-entrée}
\end{entrée}

\begin{entrée}{pwa}{3}{ⓔpwaⓗ3}
\région{GOs PA BO}
\variante{%
pwau
\région{BO}}
(\domainesémantique{Localisation})
\classe{LOC}
\begin{glose}
\pfra{dehors ; extérieur}
\end{glose}
\newline
\begin{exemple}
\textbf{\pnua{u-pwa !}}
\pfra{sors !}
\end{exemple}
\newline
\begin{exemple}
\textbf{\pnua{na pwa}}
\pfra{au dehors}
\end{exemple}
\newline
\begin{exemple}
\textbf{\pnua{e a-du pwa}}
\pfra{il est sorti dehors (de la maison)}
\end{exemple}
\end{entrée}

\begin{entrée}{pwaa}{1}{ⓔpwaaⓗ1}
\région{GOs PABO}
\classe{v}
\newline
\sens{1}
(\domainesémantique{Actions liées aux plantes})
\begin{glose}
\pfra{casser (bois en pliant) ; couper (en cassant)}
\end{glose}
\begin{glose}
\pfra{cueillir (en cassant, les fleurs)}
\end{glose}
\begin{glose}
\pfra{replier (la tige de l'igname sur elle-même quand elle dépasse la hauteur du tuteur)}
\end{glose}
\newline
\begin{sous-entrée}{pwaa muu-cee}{ⓔpwaaⓗ1ⓢ1ⓝpwaa muu-cee}
\begin{glose}
\pfra{cueillir des fleurs}
\end{glose}
\newline
\begin{exemple}
\région{GOs}
\textbf{\pnua{e pwaa kui}}
\pfra{elle replie la tige de l'igname}
\end{exemple}
\newline
\begin{exemple}
\région{BO}
\textbf{\pnua{u ru pwaale cee}}
\pfra{il va casser le bois}
\end{exemple}
\end{sous-entrée}
\newline
\sens{2}
(\domainesémantique{Mouvements ou actions faits avec le corps, les bras, les mains, les pieds})
\begin{glose}
\pfra{presser dans la main}
\end{glose}
\begin{glose}
\pfra{recueillir (le miel sur les arbres, en cassant la branche sur laquelle se trouve la ruche) [PA]}
\end{glose}
\newline
\begin{sous-entrée}{pwaa aamu}{ⓔpwaaⓗ1ⓢ2ⓝpwaa aamu}
\begin{glose}
\pfra{recueillir le miel}
\end{glose}
\newline
\relationsémantique{Cf.}{\lien{ⓔpwaale}{pwaale}}
\glosecourte{casser qqch}
\end{sous-entrée}
\newline
\étymologie{
\langue{POc}
\étymon{*posi}}
\end{entrée}

\begin{entrée}{pwaa}{2}{ⓔpwaaⓗ2}
\région{PA}
\variante{%
phwala
\région{BO}}
\classe{v}
(\domainesémantique{Verbes de déplacement et moyens de déplacement})
\begin{glose}
\pfra{faire demi-tour ; revenir de ; rentrer de}
\end{glose}
\begin{glose}
\pfra{tourner}
\end{glose}
\newline
\begin{exemple}
\région{PA}
\textbf{\pnua{bi pwaa-bin Numia}}
\pfra{nous rentrons / sommes rentrés de Nouméa}
\end{exemple}
\newline
\begin{exemple}
\région{BO}
\textbf{\pnua{phwaal-ò phwaal-mi}}
\pfra{il tourne par ci par là}
\end{exemple}
\end{entrée}

\begin{entrée}{pwaa}{3}{ⓔpwaaⓗ3}
\région{GOs}
(\domainesémantique{Mollusques})
\classe{nom}
\begin{glose}
\pfra{grisette}
\end{glose}
\end{entrée}

\begin{entrée}{pwaaci}{}{ⓔpwaaci}
\formephonétique{pwaːcɨ}
\région{GOs}
\variante{%
pwayi
\région{PA BO}}
(\domainesémantique{Préparation des aliments; modes de préparation et de cuisson})
\classe{v}
\begin{glose}
\pfra{éplucher (manioc) ; peler (avec les doigts, banane, tubercule cuit)}
\end{glose}
\newline
\relationsémantique{Cf.}{\lien{ⓔthebe}{thebe}}
\glosecourte{éplucher, peler (avec un couteau, igname ou taro cru)}
\end{entrée}

\begin{entrée}{pwaade}{}{ⓔpwaade}
\région{BO}
(\domainesémantique{Poissons})
\classe{nom}
\begin{glose}
\pfra{hareng [Corne]}
\end{glose}
\newline
\note{non vérifié}{général}{}
\end{entrée}

\begin{entrée}{pwaadi}{}{ⓔpwaadi}
\région{GOs}
(\domainesémantique{Verbes d'action (en général)})
\classe{v}
\begin{glose}
\pfra{arrêter ; interrompre}
\end{glose}
\newline
\begin{exemple}
\textbf{\pnua{e u pwaadi phwa-je}}
\pfra{elle l'a sevré}
\end{exemple}
\newline
\begin{exemple}
\textbf{\pnua{ba-pwaadi phwa-je}}
\pfra{plante qui permet de sevrer un nourrisson (arrête l'allaitement)}
\end{exemple}
\end{entrée}

\begin{entrée}{pwaala}{}{ⓔpwaala}
\région{GOs PA BO}
(\domainesémantique{Navigation})
\classe{v}
\begin{glose}
\pfra{naviguer (avec un bateau à voile) ; voguer}
\end{glose}
\newline
\begin{sous-entrée}{wô pwaala}{ⓔpwaalaⓝwô pwaala}
\begin{glose}
\pfra{bateau à voile}
\end{glose}
\end{sous-entrée}
\newline
\étymologie{
\langue{POc}
\étymon{*parau, *palauR}
\glosecourte{flotte, voyager, bateau}
\auteur{Blust}}
\end{entrée}

\begin{entrée}{pwaale}{}{ⓔpwaale}
\région{GOs PABO}
(\domainesémantique{Actions liées aux plantes})
\classe{v}
\begin{glose}
\pfra{casser (bois en pliant); couper (en cassant)}
\end{glose}
\begin{glose}
\pfra{cueillir (en cassant, les fleurs)}
\end{glose}
\newline
\begin{exemple}
\région{BO}
\textbf{\pnua{u ru pwaale cee}}
\pfra{il va casser le bois}
\end{exemple}
\end{entrée}

\begin{entrée}{pwaang}{}{ⓔpwaang}
\région{PA}
(\domainesémantique{Corps humain})
\classe{nom}
\begin{glose}
\pfra{joue}
\end{glose}
\end{entrée}

\begin{entrée}{pwããnu}{}{ⓔpwããnu}
\formephonétique{pwɛ̃ːɳu}
\région{GOs}
\variante{%
poxããnu, puxãnu
\région{GO(s)}}
(\domainesémantique{Sentiments})
\classe{v ; n}
\begin{glose}
\pfra{aimer ; amour}
\end{glose}
\newline
\begin{exemple}
\textbf{\pnua{nu mã pwããnuu-je}}
\pfra{je l'aime beaucoup}
\end{exemple}
\end{entrée}

\begin{entrée}{pwabaluni}{}{ⓔpwabaluni}
\formephonétique{pwabaluɳi}
\région{GOs}
(\domainesémantique{Fonctions intellectuelles})
\classe{v}
\begin{glose}
\pfra{tromper (se) ; faire une erreur (dans ses gestes)}
\end{glose}
\newline
\begin{exemple}
\région{GOs}
\textbf{\pnua{nu pwabaluni}}
\pfra{je me suis trompé}
\end{exemple}
\end{entrée}

\begin{entrée}{pwabwani}{}{ⓔpwabwani}
\formephonétique{pwabwaɳi}
\région{GOs WEM}
\variante{%
pwabwaning, pabwaning
\région{PA BO}}
(\domainesémantique{Types de maison, architecture de la maison})
\classe{nom}
\begin{glose}
\pfra{poutre maîtresse (supporte la toiture des maisons carrées ou rondes)}
\end{glose}
\begin{glose}
\pfra{panne sablière (supporte la toiture des maisons carrées ou rondes)}
\end{glose}
\newline
\relationsémantique{Cf.}{\lien{}{jebo, horayee}}
\end{entrée}

\begin{entrée}{pwai}{}{ⓔpwai}
\région{GOs}
(\domainesémantique{Tabac, actions liées au tabac})
\classe{nom}
\begin{glose}
\pfra{cigarette}
\end{glose}
\end{entrée}

\begin{entrée}{pwai-ce}{}{ⓔpwai-ce}
\formephonétique{pwai-cɨ}
\région{GOs}
(\domainesémantique{Tabac, actions liées au tabac})
\classe{nom}
\begin{glose}
\pfra{pipe}
\end{glose}
\end{entrée}

\begin{entrée}{pwaiòng}{}{ⓔpwaiòng}
\région{BO PA}
\variante{%
pwayòng
\région{BO PA}}
(\domainesémantique{Ignames})
\classe{nom}
\begin{glose}
\pfra{sillon du massif d'ignames}
\end{glose}
\newline
\relationsémantique{Cf.}{\lien{}{no-kia}}
\glosecourte{billon}
\newline
\relationsémantique{Cf.}{\lien{}{kê}}
\glosecourte{billon, champ}
\end{entrée}

\begin{entrée}{pwaixe}{}{ⓔpwaixe}
\région{GOs PA}
\variante{%
pwaike
\région{GO}}
(\domainesémantique{Description des objets, formes, consistance, taille})
\classe{nom}
\begin{glose}
\pfra{chose}
\end{glose}
\newline
\begin{exemple}
\textbf{\pnua{po-tru pwaixe}}
\pfra{2 choses}
\end{exemple}
\newline
\begin{exemple}
\textbf{\pnua{bi pweexu pwaixe nani vhaa zuanga}}
\pfra{nous 2 discutons de choses sur la langue zuanga}
\end{exemple}
\newline
\begin{sous-entrée}{pwaixe haze}{ⓔpwaixeⓝpwaixe haze}
\begin{glose}
\pfra{autre chose}
\end{glose}
\end{sous-entrée}
\end{entrée}

\begin{entrée}{pwajã}{}{ⓔpwajã}
\région{PA BO [Corne]}
(\domainesémantique{Arbre})
\classe{nom}
\begin{glose}
\pfra{arbre (dont les graines sont utilisées comme teinture rouge pour les poils des masques)}
\end{glose}
\nomscientifique{Macaranga}
\end{entrée}

\begin{entrée}{pwaji}{}{ⓔpwaji}
\formephonétique{pwaɲɟɨ}
\région{GOs PA BO}
\variante{%
pwaaje
\région{PA BO}}
(\domainesémantique{Crustacés, crabes})
\classe{nom}
\begin{glose}
\pfra{crabe de vase de palétuvier}
\end{glose}
\newline
\begin{sous-entrée}{pwaaji nyaaru}{ⓔpwajiⓝpwaaji nyaaru}
\begin{glose}
\pfra{crabe mou}
\end{glose}
\newline
\relationsémantique{Cf.}{\lien{}{cî, jim}}
\end{sous-entrée}
\end{entrée}

\begin{entrée}{pwajiò}{}{ⓔpwajiò}
\région{PA BO [Corne]}
(\domainesémantique{Oiseaux})
\classe{nom}
\begin{glose}
\pfra{rossignol à ventre jaune}
\end{glose}
\nomscientifique{Eopsaltria flaviventris (Eopsaltriidés)}
\end{entrée}

\begin{entrée}{pwal}{}{ⓔpwal}
\région{PA BO}
\variante{%
kole pwa
\région{GO}}
(\domainesémantique{Phénomènes atmosphériques et naturels})
\classe{v ; n}
\begin{glose}
\pfra{pleuvoir ; pluie}
\end{glose}
\newline
\begin{exemple}
\région{BO}
\textbf{\pnua{i ogin a pwal}}
\pfra{il a cessé de pleuvoir}
\end{exemple}
\newline
\begin{sous-entrée}{we-pwal}{ⓔpwalⓝwe-pwal}
\begin{glose}
\pfra{eau de pluie}
\end{glose}
\end{sous-entrée}
\newline
\begin{sous-entrée}{pwal ni kòu}{ⓔpwalⓝpwal ni kòu}
\begin{glose}
\pfra{grandes pluies}
\end{glose}
\end{sous-entrée}
\end{entrée}

\begin{entrée}{pwalamu}{}{ⓔpwalamu}
\région{BO}
(\domainesémantique{Ignames})
\classe{nom}
\begin{glose}
\pfra{igname du chef (variété blanche). Dubois}
\end{glose}
\newline
\relationsémantique{Cf.}{\lien{ⓔgu-kui}{gu-kui}}
\glosecourte{igname du chef}
\newline
\relationsémantique{Cf.}{\lien{ⓔpwangⓗ1}{pwang}}
\glosecourte{igname du chef, variété violette (Dubois)}
\newline
\note{non vérifié}{général}{}
\end{entrée}

\begin{entrée}{pwala-mwaji}{}{ⓔpwala-mwaji}
\formephonétique{,pwala-'mwãɲɟi}
\région{GOs}
\variante{%
pwali-mwajin
\région{PA}}
\classe{v}
\newline
\sens{1}
(\domainesémantique{Temps})
\begin{glose}
\pfra{longtemps (mettre)}
\end{glose}
\newline
\sens{2}
(\domainesémantique{Caractéristiques et propriétés des personnes})
\begin{glose}
\pfra{long (être) à faire qqch}
\end{glose}
\begin{glose}
\pfra{lent}
\end{glose}
\newline
\begin{exemple}
\région{GOs}
\textbf{\pnua{pwala-mwajii-nu vwo nu mõgu}}
\pfra{j'ai mis longtemps à travailler}
\end{exemple}
\newline
\begin{exemple}
\région{GOs}
\textbf{\pnua{pwala-mwajii-je vwo e ã-mi}}
\pfra{il a mis du temps pour venir}
\end{exemple}
\newline
\begin{exemple}
\textbf{\pnua{e za mõgu (vwo) pwala-mwajii-je èna}}
\pfra{il a longtemps travaillé là}
\end{exemple}
\end{entrée}

\begin{entrée}{pwalee}{}{ⓔpwalee}
\région{GOs}
(\domainesémantique{Mouvements ou actions faits avec le corps, les bras, les mains, les pieds})
\classe{v}
\begin{glose}
\pfra{étaler (natte)}
\end{glose}
\begin{glose}
\pfra{étendre (natte)}
\end{glose}
\begin{glose}
\pfra{déployer (membre)}
\end{glose}
\begin{glose}
\pfra{ouvrir (livre)}
\end{glose}
\newline
\begin{exemple}
\textbf{\pnua{e pwalee thrõ}}
\pfra{elle étale la natte}
\end{exemple}
\newline
\begin{exemple}
\textbf{\pnua{e kô-pwalee kò-je}}
\pfra{elle est allongée les jambes étendues}
\end{exemple}
\newline
\begin{exemple}
\textbf{\pnua{e tree-pwalee kò-je}}
\pfra{elle est assise les jambes étendues}
\end{exemple}
\newline
\relationsémantique{Cf.}{\lien{}{zhugi, zugi}}
\glosecourte{replier}
\end{entrée}

\begin{entrée}{pwali}{2}{ⓔpwaliⓗ2}
\région{BO PA}
(\domainesémantique{Description des objets, formes, consistance, taille})
\classe{v.stat. ; ADV}
\begin{glose}
\pfra{long ; haut}
\end{glose}
\begin{glose}
\pfra{grand}
\end{glose}
\begin{glose}
\pfra{longtemps}
\end{glose}
\newline
\begin{exemple}
\région{BO}
\textbf{\pnua{ra u pwali yala-m}}
\pfra{ton nom est long}
\end{exemple}
\end{entrée}

\begin{entrée}{pwali ?}{1}{ⓔpwali ?ⓗ1}
\région{PA}
(\domainesémantique{Interrogatifs})
\classe{INT}
\begin{glose}
\pfra{quelle longueur ?}
\end{glose}
\newline
\begin{exemple}
\région{PA}
\textbf{\pnua{pwali mwajin ?}}
\pfra{combien de temps ?}
\end{exemple}
\end{entrée}

\begin{entrée}{pwali mwajin}{}{ⓔpwali mwajin}
\région{PA}
(\domainesémantique{Temps})
\classe{LOCUT}
\begin{glose}
\pfra{il y a longtemps}
\end{glose}
\end{entrée}

\begin{entrée}{pwalu}{}{ⓔpwalu}
\région{GOs}
\variante{%
pwaalu
\région{PA BO}}
(\domainesémantique{Description des objets, formes, consistance, taille})
\classe{v.stat.}
\begin{glose}
\pfra{lourd ; grave}
\end{glose}
\begin{glose}
\pfra{cher [PA, BO]}
\end{glose}
\newline
\begin{exemple}
\région{PA}
\textbf{\pnua{ole pwaalu}}
\pfra{merci beaucoup}
\end{exemple}
\newline
\begin{exemple}
\région{PA}
\textbf{\pnua{la noo pwaalu-ni}}
\pfra{considérer qqch avec respect}
\end{exemple}
\newline
\begin{sous-entrée}{thu pwaalu [GOs, PA]}{ⓔpwaluⓝthu pwaalu [GOs, PA]}
\begin{glose}
\pfra{respecter; honorer}
\end{glose}
\newline
\note{pwaalu-ni}{grammaire}{}
\end{sous-entrée}
\newline
\relationsémantique{Ant.}{\lien{}{(h)aom}}
\glosecourte{léger}
\end{entrée}

\begin{entrée}{pwamwa}{}{ⓔpwamwa}
\région{GOs PA BO}
\variante{%
pwamwò-n
\région{PA}, 
phwamwa
\région{BO}}
\newline
\sens{1}
(\domainesémantique{Habitat})
\classe{nom}
\begin{glose}
\pfra{pays ; tribu ; contrée}
\end{glose}
\newline
\begin{sous-entrée}{me-pwamwa}{ⓔpwamwaⓢ1ⓝme-pwamwa}
\région{GOs}
\begin{glose}
\pfra{le sud du pays}
\end{glose}
\end{sous-entrée}
\newline
\begin{sous-entrée}{pòminõ pwamwa}{ⓔpwamwaⓢ1ⓝpòminõ pwamwa}
\région{GOs}
\begin{glose}
\pfra{le nord du pays}
\end{glose}
\end{sous-entrée}
\newline
\sens{2}
(\domainesémantique{Types de champs})
\classe{nom}
\begin{glose}
\pfra{champ ; plantation}
\end{glose}
\newline
\begin{exemple}
\région{GOs}
\textbf{\pnua{a bwa-wamwa !}}
\pfra{va au champ !}
\end{exemple}
\newline
\begin{exemple}
\région{GOs}
\textbf{\pnua{pòmò-nu}}
\pfra{mon pays, chez moi}
\end{exemple}
\newline
\begin{exemple}
\région{PA}
\textbf{\pnua{pwamò-ny}}
\pfra{mon pays, chez moi}
\end{exemple}
\newline
\étymologie{
\langue{POc}
\étymon{*panua}}
\newline
\note{forme déterminée : pòmò, pwamò}{grammaire}{}
\end{entrée}

\begin{entrée}{pwamwãgu}{}{ⓔpwamwãgu}
\région{PA BO}
\variante{%
pwamwègu
\région{BO}}
\classe{nom}
\newline
\sens{1}
(\domainesémantique{Organisation sociale})
\begin{glose}
\pfra{place de réunion dans le village ; emplacement des hommes (dans une réunion)}
\end{glose}
\begin{glose}
\pfra{assemblée}
\end{glose}
\newline
\relationsémantique{Cf.}{\lien{ⓔpwamwã-ròòmwa}{pwamwã-ròòmwa}}
\glosecourte{emplacement des femmes}
\newline
\sens{2}
(\domainesémantique{Discours, échanges verbaux})
\begin{glose}
\pfra{discuter}
\end{glose}
\newline
\begin{exemple}
\région{PA}
\textbf{\pnua{la pwamwããgu ni da ?}}
\pfra{de quoi parlent-ils ?}
\end{exemple}
\end{entrée}

\begin{entrée}{pwãmwãnu}{}{ⓔpwãmwãnu}
\région{PA}
\variante{%
pobwanu
\région{BO}}
(\domainesémantique{Objets coutumiers})
\classe{nom}
\begin{glose}
\pfra{monnaie}
\end{glose}
\newline
\note{de valeur moindre que 'yòò' et 'weem', mais de valeur équivalente à 'yhalo' (selon Charles). Dubois : 1 pobwanu de 5 m vaut 10 fr). Hiérarchiedes valeurs : yòò > weem > yhalo}{glose}{}
\newline
\relationsémantique{Cf.}{\lien{}{dopweza ; weem;yòò ; yhalo}}
\end{entrée}

\begin{entrée}{pwamwã-ròòmwa}{}{ⓔpwamwã-ròòmwa}
\région{PA BO}
(\domainesémantique{Organisation sociale})
\classe{nom}
\begin{glose}
\pfra{emplacement des femmes}
\end{glose}
\end{entrée}

\begin{entrée}{pwang}{1}{ⓔpwangⓗ1}
\région{BO}
(\domainesémantique{Cultures, techniques, boutures})
\classe{nom}
\begin{glose}
\pfra{fossé d'écoulement entre les massifs de culture (sur le bord du massif d'igname)[Corne]}
\end{glose}
\end{entrée}

\begin{entrée}{pwang}{2}{ⓔpwangⓗ2}
\région{BO}
(\domainesémantique{Ignames})
\classe{nom}
\begin{glose}
\pfra{igname du chef (variété violette). (Dubois)}
\end{glose}
\newline
\relationsémantique{Cf.}{\lien{ⓔgu-kui}{gu-kui}}
\glosecourte{igname du chef}
\newline
\relationsémantique{Cf.}{\lien{ⓔpwalamu}{pwalamu}}
\glosecourte{igname du chef, variété blanche (Dubois)}
\end{entrée}

\begin{entrée}{pwãû}{}{ⓔpwãû}
\région{GOs BO}
\variante{%
gò-pwãû
\région{PA}}
(\domainesémantique{Musique, instruments de musique})
\classe{nom}
\begin{glose}
\pfra{bambou (qui sert de percussion)}
\end{glose}
\newline
\begin{sous-entrée}{gò-pwãû}{ⓔpwãûⓝgò-pwãû}
\begin{glose}
\pfra{sorte de bambou}
\end{glose}
\end{sous-entrée}
\end{entrée}

\begin{entrée}{pwawa}{}{ⓔpwawa}
\région{WEM WE BO PA}
(\domainesémantique{Modalité, verbes modaux})
\classe{QNT}
\begin{glose}
\pfra{difficile ; impossible}
\end{glose}
\newline
\begin{exemple}
\région{PA}
\textbf{\pnua{e pwawa-xa me-tooli kun jena}}
\pfra{il est impossible de trouver cette place (façon de faire)}
\end{exemple}
\newline
\begin{exemple}
\région{BO}
\textbf{\pnua{pwawa na i ru pe-me-ã}}
\pfra{il ne peut pas nous suivre (BM)}
\end{exemple}
\newline
\begin{exemple}
\textbf{\pnua{pwawa ne çö whili-mi lãnã pòi-m ? [WEM]}}
\pfra{peux-tu amener tesenfants ?}
\end{exemple}
\end{entrée}

\begin{entrée}{pwawaa}{}{ⓔpwawaa}
\région{GOs}
\variante{%
pwawa
\région{PA BO}, 
pwaaò-n
\région{BO}}
(\domainesémantique{Corps humain})
\classe{nom}
\begin{glose}
\pfra{joue}
\end{glose}
\newline
\begin{exemple}
\région{PA}
\textbf{\pnua{pwawa-n}}
\pfra{sa joue}
\end{exemple}
\newline
\begin{exemple}
\région{BO}
\textbf{\pnua{pwaaò-n}}
\pfra{sa joue}
\end{exemple}
\newline
\begin{sous-entrée}{cabi pwawa-n}{ⓔpwawaaⓝcabi pwawa-n}
\begin{glose}
\pfra{gifler}
\end{glose}
\end{sous-entrée}
\end{entrée}

\begin{entrée}{pwawaale}{}{ⓔpwawaale}
\région{GOs}
\variante{%
pwapale
\région{GO(s)}}
(\domainesémantique{Cultures, techniques, boutures})
\classe{nom}
\begin{glose}
\pfra{maïs}
\end{glose}
\end{entrée}

\begin{entrée}{pwawalèng}{}{ⓔpwawalèng}
\région{BO [Corne]}
(\domainesémantique{Arbre})
\classe{nom}
\begin{glose}
\pfra{arbre}
\end{glose}
\nomscientifique{Ricinus communis}
\newline
\relationsémantique{Cf.}{\lien{}{jom}}
\glosecourte{Ricinus communis}
\end{entrée}

\begin{entrée}{pwawe}{}{ⓔpwawe}
\région{BO [BM]}
(\domainesémantique{Dons, échanges, achat et vente, vol})
\classe{v}
\begin{glose}
\pfra{récompenser ; payer ; rétribuer}
\end{glose}
\newline
\begin{exemple}
\région{BO}
\textbf{\pnua{yu ne wu pwawe i nu ?}}
\pfra{tu l'as fait pour me récompenser}
\end{exemple}
\end{entrée}

\begin{entrée}{pwaxa tree}{}{ⓔpwaxa tree}
\région{GOs}
(\domainesémantique{Découpage du temps})
\classe{LOCUT}
\begin{glose}
\pfra{toute la journée}
\end{glose}
\end{entrée}

\begin{entrée}{pwaxilo}{}{ⓔpwaxilo}
\région{PA BO}
\variante{%
pwagilo
\région{PA BO}, 
pwagilo
\région{vx}}
(\domainesémantique{Types de maison, architecture de la maison})
\classe{nom}
\begin{glose}
\pfra{porte (de maison)}
\end{glose}
\newline
\relationsémantique{Cf.}{\lien{}{phwe-mwa [GOs], phwee-mwa [PA]}}
\end{entrée}

\begin{entrée}{pwayòò}{}{ⓔpwayòò}
\région{GOs BO}
\variante{%
pwajòl
\région{BO}}
(\domainesémantique{Religion, représentations religieuses})
\classe{nom}
\begin{glose}
\pfra{lieu sacré}
\end{glose}
\begin{glose}
\pfra{présent déposé devant l'autel (Corne)}
\end{glose}
\begin{glose}
\pfra{autel près de la case (Corne, Dubois)}
\end{glose}
\newline
\begin{exemple}
\textbf{\pnua{pwajo-ã}}
\pfra{notre lieu sacré}
\end{exemple}
\end{entrée}

\begin{entrée}{pwazi}{}{ⓔpwazi}
\région{GOs}
(\domainesémantique{Préparation des aliments; modes de préparation et de cuisson})
\classe{v}
\begin{glose}
\pfra{écorcher (s')}
\end{glose}
\end{entrée}

\begin{entrée}{pwe}{1}{ⓔpweⓗ1}
\région{GOs BO}
(\domainesémantique{Pêche})
\classe{v ; n}
\begin{glose}
\pfra{pêcher à la ligne ; ligne (de pêche)}
\end{glose}
\newline
\begin{sous-entrée}{pao pwe}{ⓔpweⓗ1ⓝpao pwe}
\région{BO}
\begin{glose}
\pfra{lancer la ligne}
\end{glose}
\end{sous-entrée}
\newline
\begin{sous-entrée}{khô-pwe}{ⓔpweⓗ1ⓝkhô-pwe}
\région{BO}
\begin{glose}
\pfra{ligne}
\end{glose}
\end{sous-entrée}
\newline
\begin{sous-entrée}{pwò-pwe}{ⓔpweⓗ1ⓝpwò-pwe}
\région{BO}
\begin{glose}
\pfra{hameçon}
\end{glose}
\newline
\note{pweni (v.t.)}{grammaire}{}
\end{sous-entrée}
\end{entrée}

\begin{entrée}{pwe}{2}{ⓔpweⓗ2}
\région{GOs}
\classe{v}
\newline
\sens{1}
(\domainesémantique{Fonctions naturelles humaines})
\begin{glose}
\pfra{accoucher ; enfanter}
\end{glose}
\newline
\sens{2}
(\domainesémantique{Cours de la vie})
\begin{glose}
\pfra{naître}
\end{glose}
\newline
\begin{exemple}
\textbf{\pnua{e mhaza pwe}}
\pfra{il vient de naître; nouveau-né (il n'y a pas d'autre mot pour nouveau-né)}
\end{exemple}
\newline
\étymologie{
\langue{POc}
\étymon{*potu}
\glosecourte{apparaître}}
\end{entrée}

\begin{entrée}{pwê}{}{ⓔpwê}
\région{PA}
(\domainesémantique{Insectes})
\classe{nom}
\begin{glose}
\pfra{libellule}
\end{glose}
\end{entrée}

\begin{entrée}{pwebae}{}{ⓔpwebae}
\région{BO}
(\domainesémantique{Saisons})
\classe{nom}
\begin{glose}
\pfra{époque où les ignames commencent à mûrir (février à mars) (Dubois)}
\end{glose}
\newline
\note{non vérifié}{général}{}
\end{entrée}

\begin{entrée}{pwebwe}{}{ⓔpwebwe}
\région{BO}
(\domainesémantique{Verbes de mouvement})
\classe{v}
\begin{glose}
\pfra{tourner autour}
\end{glose}
\newline
\begin{exemple}
\région{BO}
\textbf{\pnua{la pwebwe xo aamu}}
\pfra{les mouches tournent autour [BM]}
\end{exemple}
\end{entrée}

\begin{entrée}{pwedaou}{}{ⓔpwedaou}
\région{BO}
(\domainesémantique{Bananiers et bananes})
\classe{nom}
\begin{glose}
\pfra{bananier (clone ; Dubois)}
\end{glose}
\newline
\note{non vérifié}{général}{}
\end{entrée}

\begin{entrée}{pwede}{}{ⓔpwede}
\région{GOs}
(\domainesémantique{Guerre})
\classe{v}
\begin{glose}
\pfra{cerner ; encercler}
\end{glose}
\newline
\begin{exemple}
\textbf{\pnua{la pwede-je}}
\pfra{ils l'ont cerné}
\end{exemple}
\end{entrée}

\begin{entrée}{pwedee}{}{ⓔpwedee}
\région{GOs}
(\domainesémantique{Verbes de déplacement et moyens de déplacement})
\classe{nom}
\begin{glose}
\pfra{traces de bois trainés}
\end{glose}
\end{entrée}

\begin{entrée}{pwe-ẽnõ}{}{ⓔpwe-ẽnõ}
\région{BO}
(\domainesémantique{Cours de la vie})
\classe{nom}
\begin{glose}
\pfra{nourrisson [Corne]}
\end{glose}
\end{entrée}

\begin{entrée}{pwe-kewang}{}{ⓔpwe-kewang}
\région{BO}
(\domainesémantique{Topographie})
\classe{nom}
\begin{glose}
\pfra{ravin ; précipice}
\end{glose}
\end{entrée}

\begin{entrée}{pwen}{}{ⓔpwen}
\région{BO}
(\domainesémantique{Corps humain})
\classe{nom}
\begin{glose}
\pfra{abdomen}
\end{glose}
\end{entrée}

\begin{entrée}{pwêng}{}{ⓔpwêng}
\région{PA BO}
(\domainesémantique{Noms des plantes})
\classe{nom}
\begin{glose}
\pfra{herbe}
\end{glose}
\newline
\note{(herbe dont la décoction des feuilles sert de 1° purge pour les bébés juste après la naissance, avant l'allaitement)}{glose}{}
\end{entrée}

\begin{entrée}{pweralo}{}{ⓔpweralo}
\région{BO}
(\domainesémantique{Saisons})
\classe{nom}
\begin{glose}
\pfra{mois où l'on débrousse les plantations [Dubois]}
\end{glose}
\newline
\note{non vérifié}{général}{}
\end{entrée}

\begin{entrée}{pweuna}{}{ⓔpweuna}
\région{PA BO}
(\domainesémantique{Caractéristiques et propriétés des personnes})
\classe{v}
\begin{glose}
\pfra{gourmand ; glouton}
\end{glose}
\end{entrée}

\begin{entrée}{pwevwenu}{}{ⓔpwevwenu}
\région{GOs}
\variante{%
pwemwêning, pwepwêni (Dubois)
\région{BO}}
(\domainesémantique{Eau})
\classe{nom}
\begin{glose}
\pfra{tourbillon (sur un cours d'eau)}
\end{glose}
\end{entrée}

\begin{entrée}{pwewee}{}{ⓔpwewee}
\région{GOs}
\variante{%
pweween
\région{PA}, 
phweween
\région{BO}}
(\domainesémantique{Verbes de déplacement et moyens de déplacement})
\classe{v}
\begin{glose}
\pfra{revenir (en sens inverse) ; se retourner}
\end{glose}
\begin{glose}
\pfra{contourner}
\end{glose}
\begin{glose}
\pfra{fouler (se) ; tordre [BO]}
\end{glose}
\newline
\begin{sous-entrée}{a-pwewee}{ⓔpweweeⓝa-pwewee}
\begin{glose}
\pfra{revenir, retourner en sens inverse}
\end{glose}
\end{sous-entrée}
\newline
\begin{sous-entrée}{pweweede}{ⓔpweweeⓝpweweede}
\begin{glose}
\pfra{retourner devant derrière}
\end{glose}
\newline
\begin{exemple}
\région{BO}
\textbf{\pnua{i phweween xo Kaavo}}
\pfra{Kaavo se retourne}
\end{exemple}
\end{sous-entrée}
\end{entrée}

\begin{entrée}{pweweede}{}{ⓔpweweede}
\région{GOs BO PA}
(\domainesémantique{Verbes de mouvement})
\classe{v}
\begin{glose}
\pfra{tourner ; retourner ; poser à l'envers ; retourner un objet (sur le même plan)}
\end{glose}
\begin{glose}
\pfra{changer ; traduire}
\end{glose}
\newline
\begin{sous-entrée}{pwepweede nhe}{ⓔpweweedeⓝpwepweede nhe}
\région{GOs}
\begin{glose}
\pfra{virer de bord par vent arrière}
\end{glose}
\newline
\note{pweede 'tourner'}{général}{}
\end{sous-entrée}
\end{entrée}

\begin{entrée}{pwe-yal}{}{ⓔpwe-yal}
\région{BO}
(\domainesémantique{Types de maison, architecture de la maison})
\classe{nom}
\begin{glose}
\pfra{rangée de paille sur un toit [Corne]}
\end{glose}
\newline
\note{non vérifié}{général}{}
\end{entrée}

\begin{entrée}{pweza}{}{ⓔpweza}
\région{GO PA}
(\domainesémantique{Bananiers et bananes})
\classe{nom}
\begin{glose}
\pfra{bananier (clone ancien)}
\end{glose}
\begin{glose}
\pfra{'banane de la chefferie' (Charles, PA)}
\end{glose}
\newline
\begin{sous-entrée}{dròò-pweza}{ⓔpwezaⓝdròò-pweza}
\begin{glose}
\pfra{feuille de bananier}
\end{glose}
\end{sous-entrée}
\newline
\begin{sous-entrée}{nõ-pweza}{ⓔpwezaⓝnõ-pweza}
\begin{glose}
\pfra{champ de bananier}
\end{glose}
\newline
\relationsémantique{Cf.}{\lien{}{cabeng, pòdi [PA]}}
\glosecourte{clone de bananiers-chefs}
\end{sous-entrée}
\end{entrée}

\begin{entrée}{pwii-khai}{}{ⓔpwii-khai}
\région{GOs BO}
(\domainesémantique{Pêche})
\classe{nom}
\begin{glose}
\pfra{senne (filet que l'on déploie en tirant) ; filet pour cerner}
\end{glose}
\newline
\relationsémantique{Cf.}{\lien{ⓔkhaiⓗ1}{khai}}
\glosecourte{tirer}
\end{entrée}

\begin{entrée}{pwiò}{}{ⓔpwiò}
\région{GOs PA}
\variante{%
puio, puiyo, pwiyo
\région{BO}}
(\domainesémantique{Pêche})
\classe{nom}
\begin{glose}
\pfra{filet ; senne}
\end{glose}
\newline
\begin{sous-entrée}{kha pwiò}{ⓔpwiòⓝkha pwiò}
\begin{glose}
\pfra{pêcher au filet (en se déplaçant ?)}
\end{glose}
\end{sous-entrée}
\newline
\étymologie{
\langue{POc}
\étymon{*pukot}
\glosecourte{filet}}
\end{entrée}

\begin{entrée}{pwi-phawe}{}{ⓔpwi-phawe}
\région{GOs PA}
\variante{%
pwi-phaò
\région{GO(s)}}
(\domainesémantique{Pêche})
\classe{nom}
\begin{glose}
\pfra{filet épervier}
\end{glose}
\end{entrée}

\begin{entrée}{pwiri}{}{ⓔpwiri}
\formephonétique{pwiɽi}
\région{GOs PA}
\région{GOs}
\variante{%
pwitri
, 
pwiiri
\région{BO}}
(\domainesémantique{Oiseaux})
\classe{nom}
\begin{glose}
\pfra{perruche ; loriquet calédonien}
\end{glose}
\nomscientifique{Erythrura trichoa cyaneifrons}
\end{entrée}

\begin{entrée}{pwiwii}{}{ⓔpwiwii}
\région{GOs WEM PA BO}
(\domainesémantique{Oiseaux})
\classe{nom}
\begin{glose}
\pfra{notou}
\end{glose}
\nomscientifique{Ducula goliath}
\end{entrée}

\begin{entrée}{pwiya}{}{ⓔpwiya}
\région{BO}
(\domainesémantique{Préparation des aliments; modes de préparation et de cuisson})
\classe{v}
\begin{glose}
\pfra{déterrer (le four)}
\end{glose}
\end{entrée}

\begin{entrée}{pwò}{}{ⓔpwò}
\région{GOs}
\variante{%
pò
\région{GO(s)}, 
pwòn
\région{PA BO}, 
pòn, pô
\région{BO PA}}
(\domainesémantique{Reptiles marins})
\classe{nom}
\begin{glose}
\pfra{tortue de mer}
\end{glose}
\newline
\étymologie{
\langue{POc}
\étymon{*poɲu}
\glosecourte{tortue}}
\end{entrée}

\begin{entrée}{pwòk}{}{ⓔpwòk}
\région{BO}
\classe{nom}
\begin{glose}
\pfra{voile (petite) de pirogue [Corne]}
\end{glose}
\newline
\note{non vérifié}{général}{}
\end{entrée}

\begin{entrée}{pwòmò-n}{}{ⓔpwòmò-n}
\région{WE PA}
\variante{%
pòmò-n
\région{BO PA}}
(\domainesémantique{Types de champs})
\classe{nom}
\begin{glose}
\pfra{plantation ; champ}
\end{glose}
\end{entrée}

\begin{entrée}{pwò-mhãã}{}{ⓔpwò-mhãã}
\région{GOs}
(\domainesémantique{Reptiles marins})
\classe{nom}
\begin{glose}
\pfra{tortue verte}
\end{glose}
\end{entrée}

\begin{entrée}{pwònèn}{}{ⓔpwònèn}
\région{BO [Corne]}
\variante{%
ponèn
\région{PA}}
(\domainesémantique{Quantificateurs})
\classe{v.stat. ; QNT}
\begin{glose}
\pfra{petit ; mince ; un peu}
\end{glose}
\newline
\begin{exemple}
\région{BO}
\textbf{\pnua{i hovo pwònèn nai inu}}
\pfra{il mange moins que moi}
\end{exemple}
\end{entrée}

\begin{entrée}{pwono}{}{ⓔpwono}
\région{GOs}
(\domainesémantique{Pêche})
\classe{nom}
\begin{glose}
\pfra{barrage (sur la rivière pour la pêche)}
\end{glose}
\newline
\begin{sous-entrée}{thu pwono}{ⓔpwonoⓝthu pwono}
\begin{glose}
\pfra{faire un barrage}
\end{glose}
\end{sous-entrée}
\end{entrée}

\begin{entrée}{pwõ-o pwõ-mi}{}{ⓔpwõ-o pwõ-mi}
\région{GOs}
(\domainesémantique{Verbes de déplacement et moyens de déplacement})
\classe{v}
\begin{glose}
\pfra{faire des zigzags}
\end{glose}
\end{entrée}

\newpage

\lettrine{phw}\begin{entrée}{phwa}{1}{ⓔphwaⓗ1}
\région{GOs PA BO}
\newline
\groupe{A}
\classe{nom}
\newline
\sens{1}
(\domainesémantique{Corps humain})
\begin{glose}
\pfra{bouche}
\end{glose}
\newline
\begin{sous-entrée}{ci-phwa}{ⓔphwaⓗ1ⓢ1ⓝci-phwa}
\begin{glose}
\pfra{lèvre}
\end{glose}
\newline
\note{phwa-n [PA]}{grammaire}{sa bouche}
\end{sous-entrée}
\newline
\sens{2}
(\domainesémantique{Configuration des objets})
\begin{glose}
\pfra{ouverture}
\end{glose}
\begin{glose}
\pfra{trou}
\end{glose}
\begin{glose}
\pfra{orifice}
\end{glose}
\begin{glose}
\pfra{creux}
\end{glose}
\newline
\begin{sous-entrée}{phwe-paa}{ⓔphwaⓗ1ⓢ2ⓝphwe-paa}
\begin{glose}
\pfra{entrée de la grotte}
\end{glose}
\newline
\note{phwe-}{grammaire}{forme déterminée de phwa}
\end{sous-entrée}
\newline
\sens{3}
(\domainesémantique{Discours, échanges verbaux})
\classe{v ; n}
\begin{glose}
\pfra{langue (parlée)}
\end{glose}
\newline
\begin{sous-entrée}{phwa dalaan}{ⓔphwaⓗ1ⓢ3ⓝphwa dalaan}
\begin{glose}
\pfra{langue des étrangers}
\end{glose}
\end{sous-entrée}
\newline
\groupe{B}
\classe{v.stat.}
\newline
\sens{1}
(\domainesémantique{Configuration des objets})
\begin{glose}
\pfra{percé}
\end{glose}
\begin{glose}
\pfra{troué}
\end{glose}
\newline
\begin{exemple}
\textbf{\pnua{i phwa}}
\pfra{c'est troué}
\end{exemple}
\newline
\étymologie{
\langue{POc}
\étymon{*papa}
\glosecourte{trou}}
\end{entrée}

\begin{entrée}{phwa}{2}{ⓔphwaⓗ2}
\région{PA BO}
\variante{%
phwal
\région{BO}}
(\domainesémantique{Astres})
\classe{nom}
\begin{glose}
\pfra{ciel}
\end{glose}
\newline
\begin{exemple}
\région{PA}
\textbf{\pnua{i paò ni phwa}}
\pfra{il le jette en l'air}
\end{exemple}
\newline
\relationsémantique{Cf.}{\lien{}{dòòn [PA]}}
\glosecourte{cieux}
\end{entrée}

\begin{entrée}{phwa-}{1}{ⓔphwa-ⓗ1}
\région{PA}
\newline
\groupe{A}
(\domainesémantique{Topographie})
\classe{nom}
\begin{glose}
\pfra{trou}
\end{glose}
\newline
\groupe{B}
(\domainesémantique{Préfixes classificateurs numériques})
\classe{CLF.NUM}
\begin{glose}
\pfra{trous et tas de prestations cérémonielles}
\end{glose}
\end{entrée}

\begin{entrée}{phwa-}{2}{ⓔphwa-ⓗ2}
\région{GOs PA}
(\domainesémantique{Préfixes classificateurs numériques})
\classe{CLF.NUM}
\begin{glose}
\pfra{coup ; détonation}
\end{glose}
\newline
\begin{sous-entrée}{phwa-xe, phwa-tru}{ⓔphwa-ⓗ2ⓝphwa-xe, phwa-tru}
\begin{glose}
\pfra{1, 2 détonation(s)}
\end{glose}
\newline
\begin{exemple}
\textbf{\pnua{i thi je phwa-xe}}
\pfra{il y a eu une détonation}
\end{exemple}
\end{sous-entrée}
\end{entrée}

\begin{entrée}{phwaa}{}{ⓔphwaa}
\région{GOs}
\région{PA BO}
\variante{%
phwaal
}
\classe{v}
\newline
\sens{1}
(\domainesémantique{Lumière et obscurité})
\begin{glose}
\pfra{clair (être) ; faire jour ; dégagé (ciel) ; faire beau}
\end{glose}
\newline
\begin{exemple}
\région{GOs}
\textbf{\pnua{e thaavwu phwaa}}
\pfra{il commence à faire jour}
\end{exemple}
\newline
\begin{exemple}
\région{GOs}
\textbf{\pnua{e phwaa}}
\pfra{il fait jour}
\end{exemple}
\newline
\begin{exemple}
\région{PA}
\textbf{\pnua{phwaale teen}}
\pfra{un jour clair}
\end{exemple}
\newline
\begin{exemple}
\région{BO}
\textbf{\pnua{i phwaal a dòn}}
\pfra{le ciel est dégagé}
\end{exemple}
\newline
\sens{2}
(\domainesémantique{Relations et interaction sociales})
\begin{glose}
\pfra{clairement ; public}
\end{glose}
\newline
\begin{exemple}
\région{GOs}
\textbf{\pnua{e kô-phwaa vhaa-je}}
\pfra{sa parole est claire}
\end{exemple}
\newline
\begin{exemple}
\région{GOs}
\textbf{\pnua{khôbwe-phwaaze}}
\pfra{parler clairement}
\end{exemple}
\newline
\begin{exemple}
\région{PA}
\textbf{\pnua{phwaal nai yo ?}}
\pfra{est-ce clair pour toi ?}
\end{exemple}
\newline
\begin{exemple}
\région{BO}
\textbf{\pnua{vaa phwal}}
\pfra{parler clairement, à voix haute}
\end{exemple}
\newline
\sens{3}
(\domainesémantique{Cultures, techniques, boutures})
\begin{glose}
\pfra{défriché ; dégagé}
\end{glose}
\newline
\begin{exemple}
\région{GOs}
\textbf{\pnua{pwaamwa kavwö phwaa}}
\pfra{le champ n'est pas débroussé/dégagé}
\end{exemple}
\newline
\note{v.t. phwaale [BO, PA], pwaaze [GOs]}{grammaire}{}
\end{entrée}

\begin{entrée}{phwããgo}{}{ⓔphwããgo}
\région{GOs BO}
\variante{%
pwããgo
\région{PA BO}}
(\domainesémantique{Bananiers et bananes})
\classe{nom}
\begin{glose}
\pfra{bananier (clone de)}
\end{glose}
\nomscientifique{Musa sp.}
\end{entrée}

\begin{entrée}{phwaa khoojòng}{}{ⓔphwaa khoojòng}
\région{PA}
(\domainesémantique{Relations et interaction sociales})
\classe{v}
\begin{glose}
\pfra{faire des plaisanteries grivoises}
\end{glose}
\newline
\relationsémantique{Cf.}{\lien{}{khoojòng}}
\glosecourte{faux manguier}
\end{entrée}

\begin{entrée}{phwaal}{}{ⓔphwaal}
\région{PA BO}
(\domainesémantique{Verbes de mouvement})
\classe{v}
\begin{glose}
\pfra{rouler (se)}
\end{glose}
\begin{glose}
\pfra{tourner}
\end{glose}
\newline
\begin{exemple}
\région{BO}
\textbf{\pnua{phwaal-ò phwaal-mi}}
\pfra{il tourne par ci par là}
\end{exemple}
\end{entrée}

\begin{entrée}{phwaala tèèn}{}{ⓔphwaala tèèn}
\région{BO}
(\domainesémantique{Lumière et obscurité})
\classe{nom}
\begin{glose}
\pfra{lumière du jour ; lumière}
\end{glose}
\end{entrée}

\begin{entrée}{phwaa-me mhwããnu}{}{ⓔphwaa-me mhwããnu}
\région{GOs}
(\domainesémantique{Astres})
\classe{nom}
\begin{glose}
\pfra{pleine lune}
\end{glose}
\end{entrée}

\begin{entrée}{phwaawe}{}{ⓔphwaawe}
\région{GOs PA WEM BO}
(\domainesémantique{Préparation des aliments; modes de préparation et de cuisson})
\classe{v}
\begin{glose}
\pfra{fumer (poisson)}
\end{glose}
\newline
\begin{exemple}
\textbf{\pnua{li thu phwaawe nò}}
\pfra{ils fument (font le fumage) le poisson}
\end{exemple}
\newline
\begin{sous-entrée}{ba-phwaawe}{ⓔphwaaweⓝba-phwaawe}
\begin{glose}
\pfra{fumoir}
\end{glose}
\end{sous-entrée}
\end{entrée}

\begin{entrée}{phwaaza-tree}{}{ⓔphwaaza-tree}
\région{GOs}
\variante{%
phwaala-tèèn, phwara-tèèn
\région{PA BO}}
(\domainesémantique{Découpage du temps})
\classe{nom}
\begin{glose}
\pfra{aurore ; premières lueurs du jour}
\end{glose}
\newline
\begin{exemple}
\région{GOs}
\textbf{\pnua{phwaaza tree}}
\pfra{le jour se lève}
\end{exemple}
\end{entrée}

\begin{entrée}{phwa draalae}{}{ⓔphwa draalae}
\région{GOs}
(\domainesémantique{Discours, échanges verbaux})
\classe{v}
\begin{glose}
\pfra{parler français}
\end{glose}
\end{entrée}

\begin{entrée}{phwalawa}{}{ⓔphwalawa}
\région{GOs}
\variante{%
pwalaa, palawa
\région{PA}}
(\domainesémantique{Aliments, alimentation})
\classe{nom}
\begin{glose}
\pfra{pain}
\end{glose}
\newline
\emprunt{flour (GB)}
\end{entrée}

\begin{entrée}{phwamwã-roomwã}{}{ⓔphwamwã-roomwã}
\région{PA}
(\domainesémantique{Types de maison, architecture de la maison})
\classe{nom}
\begin{glose}
\pfra{maison des femmes}
\end{glose}
\end{entrée}

\begin{entrée}{phwa ni gòò-mwa}{}{ⓔphwa ni gòò-mwa}
\région{GOs WEM WEH}
\variante{%
phwe-kuracee
\région{PA}}
(\domainesémantique{Types de maison, architecture de la maison})
\classe{nom}
\begin{glose}
\pfra{fenêtre}
\end{glose}
\end{entrée}

\begin{entrée}{phwau-n}{}{ⓔphwau-n}
\région{PA BO [BM]}
(\domainesémantique{Coutumes, dons coutumiers})
\classe{nom}
\begin{glose}
\pfra{don à la mère pour la naissance d'un enfant}
\end{glose}
\newline
\begin{exemple}
\textbf{\pnua{thu phwau-n}}
\pfra{faire un don pour une naissance}
\end{exemple}
\end{entrée}

\begin{entrée}{phwawali}{}{ⓔphwawali}
\région{GOs BO PA}
\variante{%
pwapwali
\région{BO PA}, 
pwali
\région{BO}}
(\domainesémantique{Description des objets, formes, consistance, taille})
\classe{v.stat.}
\begin{glose}
\pfra{long ; haut}
\end{glose}
\begin{glose}
\pfra{grand}
\end{glose}
\begin{glose}
\pfra{longtemps}
\end{glose}
\newline
\begin{exemple}
\région{GO}
\textbf{\pnua{e phwawali / khawali}}
\pfra{il est grand}
\end{exemple}
\newline
\begin{exemple}
\région{GO}
\textbf{\pnua{phwawali je xo burô}}
\pfra{il met du temps à se laver}
\end{exemple}
\newline
\begin{exemple}
\région{GO}
\textbf{\pnua{e phwawali nai je}}
\pfra{il est plus grand qu'elle}
\end{exemple}
\newline
\begin{exemple}
\région{PA}
\textbf{\pnua{pe-pwali-li}}
\pfra{ils ont la même taille}
\end{exemple}
\newline
\begin{exemple}
\textbf{\pnua{pwawali mwa tree}}
\pfra{les jours passent}
\end{exemple}
\newline
\begin{sous-entrée}{pwali-mwajin}{ⓔphwawaliⓝpwali-mwajin}
\région{PA}
\begin{glose}
\pfra{lentement}
\end{glose}
\newline
\relationsémantique{Cf.}{\lien{ⓔkhawali}{khawali}}
\glosecourte{long; haut}
\end{sous-entrée}
\end{entrée}

\begin{entrée}{phwaxa}{}{ⓔphwaxa}
\région{GOs PA}
(\domainesémantique{Description des objets, formes, consistance, taille})
\classe{nom}
\begin{glose}
\pfra{longueur}
\end{glose}
\begin{glose}
\pfra{durée}
\end{glose}
\begin{glose}
\pfra{hauteur}
\end{glose}
\newline
\begin{exemple}
\région{GO}
\textbf{\pnua{phwaxa tree}}
\pfra{la durée de la journée, toute la journée}
\end{exemple}
\newline
\begin{exemple}
\région{PA}
\textbf{\pnua{phwaxa teen}}
\pfra{la durée de la journée, toute la journée}
\end{exemple}
\newline
\begin{exemple}
\région{PA}
\textbf{\pnua{phwaxa deen}}
\pfra{la longueur du chemin}
\end{exemple}
\newline
\begin{exemple}
\région{GO}
\textbf{\pnua{phwaxa wô}}
\pfra{la longueur du bateau}
\end{exemple}
\end{entrée}

\begin{entrée}{phwaxaa}{}{ⓔphwaxaa}
\région{GOs}
(\domainesémantique{Crustacés, crabes})
\classe{nom}
\begin{glose}
\pfra{crabe de sable (sert d'amorce)}
\end{glose}
\end{entrée}

\begin{entrée}{phwaxi}{}{ⓔphwaxi}
\région{GOs PA BO}
\variante{%
phwaxa
\région{GO(s) PA}}
(\domainesémantique{Caractéristiques et propriétés des personnes})
\classe{nom}
\begin{glose}
\pfra{hauteur ; taille (en hauteur)}
\end{glose}
\newline
\begin{exemple}
\textbf{\pnua{pe-phwaxi-li [GOs, PA]}}
\pfra{ils sont de même taille}
\end{exemple}
\newline
\begin{exemple}
\région{PA}
\textbf{\pnua{waya phwaxi-m ?}}
\pfra{tu mesures combien ? (quelle est ta hauteur ?)}
\end{exemple}
\newline
\begin{exemple}
\région{BO}
\textbf{\pnua{phwaxi-n}}
\pfra{sa taille}
\end{exemple}
\end{entrée}

\begin{entrée}{phwaxoi-n}{}{ⓔphwaxoi-n}
\région{BO}
\classe{nom}
(\domainesémantique{Corps humain})
\begin{glose}
\pfra{diaphragme [BM]}
\end{glose}
\end{entrée}

\begin{entrée}{phwa-xudo}{}{ⓔphwa-xudo}
\région{GOs}
(\domainesémantique{Mer : topographie})
\classe{nom}
\begin{glose}
\pfra{chenal}
\end{glose}
\end{entrée}

\begin{entrée}{phwayuu}{}{ⓔphwayuu}
\région{GOs WEM WE BO}
\variante{%
whayuu
\région{GO(s)}}
(\domainesémantique{Fonctions naturelles humaines})
\classe{nom}
\begin{glose}
\pfra{sang menstruel}
\end{glose}
\newline
\begin{exemple}
\région{BO}
\textbf{\pnua{i tòòli phwayuu}}
\pfra{elle a ses règles}
\end{exemple}
\newline
\begin{sous-entrée}{mwa phwayuu}{ⓔphwayuuⓝmwa phwayuu}
\région{GO}
\begin{glose}
\pfra{case pour les femmes}
\end{glose}
\end{sous-entrée}
\newline
\begin{sous-entrée}{mõ-wayuu}{ⓔphwayuuⓝmõ-wayuu}
\région{GO}
\begin{glose}
\pfra{case pour les femmes}
\end{glose}
\end{sous-entrée}
\end{entrée}

\begin{entrée}{phwa-zua zuanga}{}{ⓔphwa-zua zuanga}
\région{GOs}
(\domainesémantique{Discours, échanges verbaux})
\classe{nom}
\begin{glose}
\pfra{langue parlée zuanga/yuanga}
\end{glose}
\end{entrée}

\begin{entrée}{phwè-}{}{ⓔphwè-}
\région{GOs}
\classe{nom}
\newline
\sens{1}
(\domainesémantique{})
\begin{glose}
\pfra{trou de (en composition)}
\end{glose}
\begin{glose}
\pfra{trou}
\end{glose}
\newline
\begin{sous-entrée}{phwè-pwè-n}{ⓔphwè-ⓢ1ⓝphwè-pwè-n}
\begin{glose}
\pfra{trou de crabe}
\end{glose}
\end{sous-entrée}
\newline
\relationsémantique{Cf.}{\lien{ⓔphwaⓗ1}{phwa}}
\newline
\sens{2}
(\domainesémantique{Corps humain})
\begin{glose}
\pfra{orifice}
\end{glose}
\newline
\begin{sous-entrée}{phwè-bozo}{ⓔphwè-ⓢ2ⓝphwè-bozo}
\région{GO}
\end{sous-entrée}
\newline
\begin{sous-entrée}{phwè-bulò-n}{ⓔphwè-ⓢ2ⓝphwè-bulò-n}
\région{PA BO}
\end{sous-entrée}
\newline
\begin{sous-entrée}{phwè-nò-n}{ⓔphwè-ⓢ2ⓝphwè-nò-n}
\région{PA BO}
\end{sous-entrée}
\newline
\begin{sous-entrée}{phwè-bwino-n}{ⓔphwè-ⓢ2ⓝphwè-bwino-n}
\région{PA BO}
\begin{glose}
\pfra{son anus}
\end{glose}
\end{sous-entrée}
\newline
\sens{3}
(\domainesémantique{Topographie})
\begin{glose}
\pfra{tranchée}
\end{glose}
\newline
\begin{sous-entrée}{phwè-wano}{ⓔphwè-ⓢ3ⓝphwè-wano}
\begin{glose}
\pfra{tranchée sur crête forestière}
\end{glose}
\end{sous-entrée}
\end{entrée}

\begin{entrée}{phwe-ai}{}{ⓔphwe-ai}
\région{GOs PA BO}
\classe{nom}
\newline
\sens{1}
(\domainesémantique{Corps humain})
\begin{glose}
\pfra{coeur}
\end{glose}
\newline
\begin{exemple}
\région{GOs}
\textbf{\pnua{phwe-ai je}}
\pfra{son coeur}
\end{exemple}
\newline
\begin{exemple}
\région{PA BO}
\textbf{\pnua{phwe-ai-n}}
\pfra{son coeur}
\end{exemple}
\newline
\sens{2}
(\domainesémantique{Sentiments})
\begin{glose}
\pfra{amour ; sentiment ; affection ; désir ; volonté}
\end{glose}
\begin{glose}
\pfra{désirer}
\end{glose}
\newline
\relationsémantique{Cf.}{\lien{ⓔphwè-}{phwè-}}
\glosecourte{trou, emplacement}
\end{entrée}

\begin{entrée}{phwè-bozo}{}{ⓔphwè-bozo}
\région{GOs}
\variante{%
pwe-bulo-n
\région{PA BO}}
(\domainesémantique{Corps humain})
\classe{nom}
\begin{glose}
\pfra{nombril}
\end{glose}
\newline
\begin{exemple}
\région{GO}
\textbf{\pnua{phwè-bozoo je}}
\pfra{son nombril}
\end{exemple}
\newline
\begin{exemple}
\région{BO}
\textbf{\pnua{phwè-bujo-n}}
\pfra{son nombril}
\end{exemple}
\newline
\étymologie{
\langue{POc}
\étymon{*mpusos}
\glosecourte{nombril}}
\end{entrée}

\begin{entrée}{phwe-chãnã}{}{ⓔphwe-chãnã}
\formephonétique{pʰwe-cʰɛ̃ɳɛ̃}
\région{GOs}
(\domainesémantique{Corps humain})
\classe{nom}
\begin{glose}
\pfra{diaphragme (lit. orifice du souffle) ; sternum}
\end{glose}
\end{entrée}

\begin{entrée}{phwè-dèn}{}{ⓔphwè-dèn}
\région{PA BO}
\classe{nom}
\newline
\sens{1}
(\domainesémantique{Verbes de déplacement et moyens de déplacement})
\begin{glose}
\pfra{chemin}
\end{glose}
\newline
\sens{2}
(\domainesémantique{Organisation sociale})
\begin{glose}
\pfra{chemin coutumier}
\end{glose}
\newline
\begin{exemple}
\région{BO}
\textbf{\pnua{i a va nye phwè-dèn ?}}
\pfra{où va ce chemin ?}
\end{exemple}
\end{entrée}

\begin{entrée}{phwe-döö}{}{ⓔphwe-döö}
\région{PA}
(\domainesémantique{Ustensiles})
\classe{nom}
\begin{glose}
\pfra{ouverture de la marmite}
\end{glose}
\end{entrée}

\begin{entrée}{phwee}{}{ⓔphwee}
\région{GOs}
(\domainesémantique{Santé, maladie
, Fonctions naturelles humaines})
\classe{v}
\begin{glose}
\pfra{expectorer}
\end{glose}
\begin{glose}
\pfra{expulser par la bouche}
\end{glose}
\begin{glose}
\pfra{cracher}
\end{glose}
\end{entrée}

\begin{entrée}{phweegi}{}{ⓔphweegi}
\région{PA}
(\domainesémantique{Bananiers et bananes})
\classe{nom}
\begin{glose}
\pfra{banane-chef}
\end{glose}
\newline
\note{(il est interdit de la cuire sur la braise, ne peut être que bouillie)}{glose}{}
\end{entrée}

\begin{entrée}{phwè-êgu}{}{ⓔphwè-êgu}
\région{GOs}
(\domainesémantique{Corps humain})
\classe{nom}
\begin{glose}
\pfra{vagin}
\end{glose}
\end{entrée}

\begin{entrée}{phwee-mee pwiò}{}{ⓔphwee-mee pwiò}
\région{GOs PA BO}
(\domainesémantique{Pêche})
\classe{nom}
\begin{glose}
\pfra{maille de filet}
\end{glose}
\end{entrée}

\begin{entrée}{phwee-mwãgi}{}{ⓔphwee-mwãgi}
\région{GOs}
(\domainesémantique{Manière de faire l’action : verbes et adverbes de manière})
\classe{ADV}
\begin{glose}
\pfra{inutile ; vainement ; en vain}
\end{glose}
\newline
\begin{exemple}
\textbf{\pnua{phwe-dee xa phwee-mwãgi}}
\pfra{un chemin coutumier qui est mal fait}
\end{exemple}
\newline
\begin{exemple}
\textbf{\pnua{mogu xo phwee-mwãgi}}
\pfra{un travail qui ne sert à rien}
\end{exemple}
\newline
\relationsémantique{Cf.}{\lien{ⓔmwãgi}{mwãgi}}
\glosecourte{cactus, épineux}
\end{entrée}

\begin{entrée}{phweewe}{}{ⓔphweewe}
\région{BO PA}
(\domainesémantique{Discours, échanges verbaux})
\classe{v}
\begin{glose}
\pfra{raconter}
\end{glose}
\end{entrée}

\begin{entrée}{phwèè-we}{}{ⓔphwèè-we}
\région{GOs PA BO}
\région{GO PA}
\variante{%
phwè-we
}
\classe{nom}
\newline
\sens{1}
(\domainesémantique{Mer : topographie})
\begin{glose}
\pfra{passe (dans la mer)}
\end{glose}
\newline
\sens{2}
(\domainesémantique{Eau})
\begin{glose}
\pfra{trou d'eau ; mare}
\end{glose}
\end{entrée}

\begin{entrée}{phweexoe}{}{ⓔphweexoe}
\région{GOs}
(\domainesémantique{Discours, échanges verbaux})
\classe{v}
\begin{glose}
\pfra{annoncer (avec un geste coutumier)}
\end{glose}
\newline
\begin{exemple}
\région{GOs}
\textbf{\pnua{la a phweexoe mha}}
\pfra{ils sont allés annoncer un deuil}
\end{exemple}
\end{entrée}

\begin{entrée}{phweexu}{}{ⓔphweexu}
\région{GOs PA BO}
\variante{%
phweeku
\région{GO(s)}, 
phweewu
\région{BO}}
\newline
\sens{1}
(\domainesémantique{Discours, échanges verbaux})
\classe{v ; n}
\begin{glose}
\pfra{annoncer ; raconter}
\end{glose}
\begin{glose}
\pfra{annonce ; information ; nouvelle (souvent mauvaise)}
\end{glose}
\newline
\begin{exemple}
\textbf{\pnua{ge le phweexu}}
\pfra{il y a une (mauvaise) nouvelle}
\end{exemple}
\newline
\begin{exemple}
\région{BO}
\textbf{\pnua{nu ru phweewu-yu pu yu tòne}}
\pfra{je vais te le raconter pour que tu entendes}
\end{exemple}
\newline
\begin{exemple}
\région{BO PA}
\textbf{\pnua{tòne je phweewu}}
\pfra{écoute cette nouvelle}
\end{exemple}
\newline
\begin{exemple}
\région{PA}
\textbf{\pnua{phweexua-ny}}
\pfra{mon histoire que je raconte}
\end{exemple}
\newline
\begin{exemple}
\région{PA}
\textbf{\pnua{phweexu na inu}}
\pfra{mon histoire à propos de moi}
\end{exemple}
\newline
\sens{2}
(\domainesémantique{Relations et interaction sociales})
\classe{v ; n}
\begin{glose}
\pfra{bavarder ; converser ; discuter}
\end{glose}
\newline
\begin{exemple}
\région{GO PA}
\textbf{\pnua{la pe-phweexu}}
\pfra{ils bavardent, se racontent des histoires}
\end{exemple}
\end{entrée}

\begin{entrée}{phwè-gòògoni}{}{ⓔphwè-gòògoni}
\formephonétique{pʰwe-}
\région{PA}
(\domainesémantique{Corps humain})
\classe{nom}
\begin{glose}
\pfra{sternum}
\end{glose}
\end{entrée}

\begin{entrée}{phwè-keni}{}{ⓔphwè-keni}
\région{GOs}
(\domainesémantique{Corps humain})
\classe{nom}
\begin{glose}
\pfra{trou de l'oreille}
\end{glose}
\newline
\begin{exemple}
\textbf{\pnua{phwè-kenii-je}}
\pfra{le trou de son oreille}
\end{exemple}
\end{entrée}

\begin{entrée}{phwe-kepwa}{}{ⓔphwe-kepwa}
\région{GO}
(\domainesémantique{Topographie})
\classe{nom}
\begin{glose}
\pfra{ouverture de la vallée}
\end{glose}
\end{entrée}

\begin{entrée}{phwè-kui}{}{ⓔphwè-kui}
\région{GOs WEM PA}
(\domainesémantique{Cultures, techniques, boutures})
\classe{nom}
\begin{glose}
\pfra{trou (préparé pour planter l'igname)}
\end{glose}
\newline
\relationsémantique{Cf.}{\lien{ⓔzaa phwa}{zaa phwa}}
\glosecourte{faire le trou pour planter l'igname}
\end{entrée}

\begin{entrée}{phwe-kurace}{}{ⓔphwe-kurace}
\région{PA BO}
(\domainesémantique{Types de maison, architecture de la maison})
\classe{nom}
\begin{glose}
\pfra{fenêtre}
\end{glose}
\end{entrée}

\begin{entrée}{phwe-meewu}{}{ⓔphwe-meewu}
\région{GOs PA BO}
(\domainesémantique{Organisation sociale})
\classe{nom}
\begin{glose}
\pfra{clan ; famille}
\end{glose}
\end{entrée}

\begin{entrée}{phwè-mwa}{}{ⓔphwè-mwa}
\région{GOs}
\variante{%
phwee-mwa
\région{PA BO}}
\classe{nom}
\newline
\sens{1}
(\domainesémantique{Types de maison, architecture de la maison})
\begin{glose}
\pfra{porte (de maison)}
\end{glose}
\newline
\sens{2}
(\domainesémantique{Organisation sociale})
\begin{glose}
\pfra{personne qui sert de 'chemin' pour entrer dans la chefferie [PA]}
\end{glose}
\newline
\relationsémantique{Cf.}{\lien{}{pwaxilo [BO, PA]}}
\glosecourte{porte}
\end{entrée}

\begin{entrée}{phwè-mwêêdi}{}{ⓔphwè-mwêêdi}
\région{GOs PA BO}
(\domainesémantique{Corps humain})
\classe{nom}
\begin{glose}
\pfra{narine}
\end{glose}
\newline
\begin{exemple}
\région{BO}
\textbf{\pnua{phwè-mwêêdi-n}}
\pfra{son nez}
\end{exemple}
\end{entrée}

\begin{entrée}{phwè-na}{}{ⓔphwè-na}
\région{GOs}
\variante{%
phwe-nò
\région{BO}}
(\domainesémantique{Corps humain})
\classe{nom}
\begin{glose}
\pfra{anus (grossier, Dubois)}
\end{glose}
\newline
\begin{exemple}
\région{BO}
\textbf{\pnua{phwè-nõ-n}}
\pfra{son anus}
\end{exemple}
\newline
\relationsémantique{Cf.}{\lien{ⓔpwaⓗ1}{pwa}}
\glosecourte{trou (terme poli)}
\end{entrée}

\begin{entrée}{phwêne}{}{ⓔphwêne}
\région{GOs}
\variante{%
phwêneng
\région{PA}}
(\domainesémantique{Relations et interaction sociales})
\classe{INTJ ; v}
\begin{glose}
\pfra{attention! ; faire attention}
\end{glose}
\newline
\begin{exemple}
\textbf{\pnua{ra u mha phwêneng !}}
\pfra{faites bien attention!}
\end{exemple}
\end{entrée}

\begin{entrée}{phwè-nògò}{}{ⓔphwè-nògò}
\formephonétique{pʰwɛ-ɳɔŋgɔ}
\région{GOs}
\variante{%
pwè-nògò
\région{BO PA}}
(\domainesémantique{Mer : topographie})
\classe{nom}
\begin{glose}
\pfra{embouchure de la rivière}
\end{glose}
\begin{glose}
\pfra{confluent d'un creek dans un fleuve}
\end{glose}
\end{entrée}

\begin{entrée}{phweõ}{}{ⓔphweõ}
\région{GOs}
\variante{%
phwee-hòn, phweòn
\région{BO [Corne]}}
(\domainesémantique{Eau})
\classe{nom}
\begin{glose}
\pfra{source (endroit où l'eau sourd)}
\end{glose}
\end{entrée}

\begin{entrée}{phwè-paa}{}{ⓔphwè-paa}
\région{GOs}
(\domainesémantique{Topographie})
\classe{nom}
\begin{glose}
\pfra{grotte}
\end{glose}
\end{entrée}

\begin{entrée}{phwè-peeńã}{}{ⓔphwè-peeńã}
\région{GOs}
(\domainesémantique{Anguilles})
\classe{nom}
\begin{glose}
\pfra{trou d'anguille ; anfractuosité dans rocher (où se cachent les anguilles)}
\end{glose}
\end{entrée}

\begin{entrée}{phwè-po-xaò}{}{ⓔphwè-po-xaò}
\région{GOs}
\variante{%
phwè-vwo-xaò
\région{GO(s)}}
(\domainesémantique{Cultures, techniques, boutures})
\classe{nom}
\begin{glose}
\pfra{caniveau}
\end{glose}
\end{entrée}

\begin{entrée}{phwevwöu}{}{ⓔphwevwöu}
\région{GOs PA BO}
\classe{n.LOC}
\newline
\sens{1}
(\domainesémantique{Noms locatifs})
\begin{glose}
\pfra{intervalle entre deux rangées d'un champ d'ignames}
\end{glose}
\newline
\begin{sous-entrée}{phwevwöu mwa}{ⓔphwevwöuⓢ1ⓝphwevwöu mwa}
\région{PA}
\begin{glose}
\pfra{intervalle entre les maisons}
\end{glose}
\end{sous-entrée}
\newline
\begin{sous-entrée}{na ni phwevwöu}{ⓔphwevwöuⓢ1ⓝna ni phwevwöu}
\région{BO}
\begin{glose}
\pfra{dans l'intervalle}
\end{glose}
\newline
\begin{exemple}
\région{BO}
\textbf{\pnua{nee-je ni phwevwöu daalaèn}}
\pfra{mets-le entre les deux blancs}
\end{exemple}
\end{sous-entrée}
\newline
\sens{2}
(\domainesémantique{Noms locatifs})
\begin{glose}
\pfra{entre}
\end{glose}
\newline
\begin{exemple}
\région{PA}
\textbf{\pnua{i a-da phwevwöu ègu}}
\pfra{il marche entre les gens}
\end{exemple}
\end{entrée}

\begin{entrée}{phwè-wado}{}{ⓔphwè-wado}
\région{BO}
\classe{nom}
\newline
\sens{1}
(\domainesémantique{Santé, maladie})
\begin{glose}
\pfra{carie dentaire}
\end{glose}
\newline
\sens{2}
(\domainesémantique{Corps humain})
\begin{glose}
\pfra{interstice entre les dents}
\end{glose}
\begin{glose}
\pfra{trou dans la dentition}
\end{glose}
\end{entrée}

\begin{entrée}{phwè-wõõ}{}{ⓔphwè-wõõ}
\région{GOs}
(\domainesémantique{Cordes, cordages})
\classe{nom}
\begin{glose}
\pfra{lasso (pour attraper un cheval)}
\end{glose}
\begin{glose}
\pfra{noeud coulant}
\end{glose}
\end{entrée}

\begin{entrée}{phwè-whaa}{}{ⓔphwè-whaa}
\région{GOs}
\variante{%
phwè-whara
\région{PA}}
(\domainesémantique{Corps humain})
\classe{nom}
\begin{glose}
\pfra{fontanelle}
\end{glose}
\newline
\begin{exemple}
\textbf{\pnua{phwè-whaa i je}}
\pfra{sa fontanelle}
\end{exemple}
\newline
\relationsémantique{Cf.}{\lien{ⓔwhaⓗ1}{wha}}
\glosecourte{grandir}
\end{entrée}

\begin{entrée}{phwe-zatri}{}{ⓔphwe-zatri}
\formephonétique{pʰwe-ðaɽi}
\région{GOs PA BO}
\variante{%
pwe-yari
\région{BO}}
(\domainesémantique{Remèdes, médecine})
\classe{nom}
\begin{glose}
\pfra{guérisseur (lit. origine du médicament)}
\end{glose}
\end{entrée}

\begin{entrée}{phwè-zini}{}{ⓔphwè-zini}
\région{GOs}
\variante{%
phwe-thîni
\région{PA}}
\classe{nom}
\newline
\sens{1}
(\domainesémantique{Types de maison, architecture de la maison})
\begin{glose}
\pfra{porte de clôture}
\end{glose}
\begin{glose}
\pfra{portail}
\end{glose}
\newline
\sens{2}
(\domainesémantique{Organisation sociale})
\begin{glose}
\pfra{entrée de la chefferie}
\end{glose}
\end{entrée}

\begin{entrée}{phwia, phwiça}{}{ⓔphwia, phwiça}
\région{GOs PA WEM WE BO}
\variante{%
phuça
\formephonétique{pʰudʒa}
\région{GO(s)}, 
phuya
\région{BO}}
\classe{v}
\newline
\sens{1}
(\domainesémantique{Fonctions naturelles humaines})
\begin{glose}
\pfra{réveiller qqn (en secouant, en retournant)}
\end{glose}
\newline
\begin{exemple}
\région{GO}
\textbf{\pnua{phuya meni wu la u nòòl !}}
\pfra{secoue les oiseaux pour qu'ils se réveillent !}
\end{exemple}
\newline
\begin{exemple}
\région{GO}
\textbf{\pnua{phwia-je !}}
\pfra{réveille-le !}
\end{exemple}
\newline
\sens{2}
(\domainesémantique{Verbes de mouvement})
\begin{glose}
\pfra{retourner, soulever une pierre pour voir s'il y a qqch en dessous [PA]}
\end{glose}
\end{entrée}

\begin{entrée}{phwi-n}{}{ⓔphwi-n}
\région{BO [Corne]}
\variante{%
phui-n
\région{BO}}
(\domainesémantique{Santé, maladie})
\classe{nom}
\begin{glose}
\pfra{bosse [Corne]}
\end{glose}
\newline
\begin{exemple}
\textbf{\pnua{phwi-n}}
\pfra{sa bosse}
\end{exemple}
\newline
\note{non vérifié}{général}{}
\end{entrée}

\begin{entrée}{phwioo}{}{ⓔphwioo}
\région{GOs}
\variante{%
phoyo
\région{GO(s)}}
(\domainesémantique{Parenté})
\classe{nom}
\begin{glose}
\pfra{aîné (fils)}
\end{glose}
\newline
\begin{exemple}
\région{GO}
\textbf{\pnua{phwioo-nu}}
\pfra{mon aîné}
\end{exemple}
\newline
\begin{exemple}
\textbf{\pnua{phwioo-ã}}
\pfra{notre fils aîné}
\end{exemple}
\newline
\begin{exemple}
\textbf{\pnua{ẽnõ-ã ça e thrôbo phwioo-iço}}
\pfra{cet enfant-là est ton fils aîné (lit. est né ton aîné)}
\end{exemple}
\end{entrée}

\begin{entrée}{phwô}{}{ⓔphwô}
\région{GOs}
(\domainesémantique{Mouvements ou actions faits avec le corps, les bras, les mains, les pieds})
\classe{v}
\begin{glose}
\pfra{courber ; tordre}
\end{glose}
\newline
\begin{exemple}
\région{GOs}
\textbf{\pnua{e phwô-ò phwô-mi}}
\pfra{il zigzague}
\end{exemple}
\end{entrée}

\begin{entrée}{phwòli}{}{ⓔphwòli}
\région{BO [BM, Corne]}
\classe{v}
\newline
\sens{1}
(\domainesémantique{Mouvements ou actions faits avec le corps, les bras, les mains, les pieds})
\begin{glose}
\pfra{presser ; écraser (la pulpe de coco)}
\end{glose}
\newline
\sens{2}
(\domainesémantique{Soins du corps})
\begin{glose}
\pfra{masser}
\end{glose}
\newline
\étymologie{
\langue{POc}
\étymon{*posi}}
\end{entrée}

\begin{entrée}{phwòòn}{}{ⓔphwòòn}
\région{BO}
\variante{%
fwòòn
\région{BO}}
(\domainesémantique{Poissons})
\classe{nom}
\begin{glose}
\pfra{loche (petite) ; lochon}
\end{glose}
\end{entrée}

\newpage

\lettrine{r}\begin{entrée}{ra}{}{ⓔra}
\région{PA}
(\domainesémantique{Marques assertives})
\classe{PTCL (assertive)}
\begin{glose}
\pfra{assertif}
\end{glose}
\newline
\begin{exemple}
\textbf{\pnua{e ra yu pwamwa}}
\pfra{il reste sur ses terres}
\end{exemple}
\end{entrée}

\begin{entrée}{ra ?}{}{ⓔra ?}
\région{PA WE}
\variante{%
za ?
\région{GO(s)}}
(\domainesémantique{Interrogatifs})
\classe{INT}
\begin{glose}
\pfra{quoi?}
\end{glose}
\newline
\begin{exemple}
\région{WE}
\textbf{\pnua{co pwò ra ?; co pò ra ?}}
\pfra{que fais-tu?}
\end{exemple}
\newline
\begin{exemple}
\région{GOs}
\textbf{\pnua{co pò-za ?}}
\pfra{que fais-tu?}
\end{exemple}
\end{entrée}

\begin{entrée}{ra-u}{}{ⓔra-u}
\région{PA BO}
(\domainesémantique{Marques assertives})
\classe{MODAL}
\begin{glose}
\pfra{vraiment ; tout à fait}
\end{glose}
\newline
\begin{exemple}
\textbf{\pnua{i ra-u / ra-o thoo phao-i-je jigel}}
\pfra{il s'est tiré un coup de fusil à lui-même}
\end{exemple}
\newline
\begin{exemple}
\région{BO}
\textbf{\pnua{kabun ra-u è}}
\pfra{c'est tout à fait défendu}
\end{exemple}
\end{entrée}

\begin{entrée}{ri ?}{}{ⓔri ?}
\région{PA WE}
(\domainesémantique{Interrogatifs})
\classe{SUFF.POSS.INT}
\begin{glose}
\pfra{qui (de) ?}
\end{glose}
\newline
\begin{exemple}
\textbf{\pnua{hèlè ri nye ?}}
\pfra{à qui est ce couteau ? (lit. couteau de qui ?)}
\end{exemple}
\newline
\relationsémantique{Cf.}{\lien{ⓔti ?}{ti ?}}
\glosecourte{qui ?}
\newline
\note{la forme "ri" n'existe pas en GO(s)}{grammaire}{}
\end{entrée}

\begin{entrée}{ru}{}{ⓔru}
\région{PA BO}
\variante{%
to, ro
\région{BO}}
(\domainesémantique{Temps})
\classe{FUT}
\begin{glose}
\pfra{futur ; prospectif}
\end{glose}
\newline
\begin{exemple}
\région{PA}
\textbf{\pnua{i ru pwal}}
\pfra{il va pleuvoir}
\end{exemple}
\newline
\begin{exemple}
\textbf{\pnua{nu ru a}}
\pfra{je pars, je vais partir}
\end{exemple}
\newline
\relationsémantique{Cf.}{\lien{ⓔu ru}{u ru}}
\glosecourte{généralement postposée au pronom sujet}
\end{entrée}

\begin{entrée}{ruma}{}{ⓔruma}
\région{PA BO}
\variante{%
toma
\région{BO}}
(\domainesémantique{Temps})
\classe{FUT}
\begin{glose}
\pfra{futur}
\end{glose}
\newline
\begin{exemple}
\région{PA}
\textbf{\pnua{nu ruma khôbwe}}
\pfra{je le dirai}
\end{exemple}
\newline
\begin{exemple}
\région{BO}
\textbf{\pnua{mwa ruma õ-xe pe-ròòli}}
\pfra{on se reverra une fois}
\end{exemple}
\end{entrée}

\newpage

\lettrine{s}\begin{entrée}{salaa}{}{ⓔsalaa}
\région{GOs}
(\domainesémantique{Aliments, alimentation})
\classe{nom}
\begin{glose}
\pfra{salade}
\end{glose}
\newline
\emprunt{salade (FR)}
\end{entrée}

\begin{entrée}{sapone}{}{ⓔsapone}
\région{GOs}
(\domainesémantique{Société})
\classe{nom}
\begin{glose}
\pfra{japonais}
\end{glose}
\newline
\emprunt{japonais (FR)}
\end{entrée}

\begin{entrée}{simi}{}{ⓔsimi}
\région{GOs}
\variante{%
cimic
\région{PA}}
(\domainesémantique{Vêtements, parure})
\classe{nom}
\begin{glose}
\pfra{chemise}
\end{glose}
\newline
\begin{exemple}
\région{GOs}
\textbf{\pnua{e udale simi}}
\pfra{il met sa chemise}
\end{exemple}
\newline
\begin{exemple}
\région{PA}
\textbf{\pnua{i udale cimic}}
\pfra{il met sa chemise}
\end{exemple}
\newline
\begin{exemple}
\région{PA}
\textbf{\pnua{i udale cimiy-i ye}}
\pfra{il met sa chemise}
\end{exemple}
\newline
\emprunt{chemise (FR)}
\end{entrée}

\begin{entrée}{siro}{}{ⓔsiro}
\région{GOs}
(\domainesémantique{Aliments, alimentation})
\classe{nom}
\begin{glose}
\pfra{sirop}
\end{glose}
\newline
\emprunt{sirop (FR)}
\end{entrée}

\newpage

\lettrine{t}\begin{entrée}{ta}{1}{ⓔtaⓗ1}
(\domainesémantique{Conjonction})
\classe{ADV}
\begin{glose}
\pfra{alors}
\end{glose}
\end{entrée}

\begin{entrée}{ta}{2}{ⓔtaⓗ2}
\région{PA}
(\domainesémantique{Verbes de déplacement et moyens de déplacement})
\classe{v}
\begin{glose}
\pfra{marcher}
\end{glose}
\newline
\begin{sous-entrée}{i ta-o ta-mi}{ⓔtaⓗ2ⓝi ta-o ta-mi}
\région{PA}
\begin{glose}
\pfra{il va et vient, il fait les 100 pas}
\end{glose}
\end{sous-entrée}
\end{entrée}

\begin{entrée}{ta}{3}{ⓔtaⓗ3}
\région{PA}
\variante{%
ra
\région{PA}}
(\domainesémantique{Marques assertives})
\classe{PTCL (assertive)}
\begin{glose}
\pfra{c'est}
\end{glose}
\end{entrée}

\begin{entrée}{ta}{4}{ⓔtaⓗ4}
\région{GOs}
\variante{%
taam
\région{WE}, 
taavw
\région{PA}}
(\domainesémantique{Objets et meubles de la maison})
\classe{nom}
\begin{glose}
\pfra{table}
\end{glose}
\newline
\emprunt{table (FR)}
\end{entrée}

\begin{entrée}{ta-}{}{ⓔta-}
\région{BO}
(\domainesémantique{Interpellation})
\classe{n.INTJ}
\begin{glose}
\pfra{eh !}
\end{glose}
\newline
\begin{exemple}
\textbf{\pnua{ta-m ! (BM)}}
\pfra{eh toi !}
\end{exemple}
\end{entrée}

\begin{entrée}{taa}{}{ⓔtaa}
\région{GOs PA BO}
\classe{v}
\newline
\sens{1}
(\domainesémantique{Mouvements ou actions faits avec le corps, les bras, les mains, les pieds})
\begin{glose}
\pfra{creuser}
\end{glose}
\begin{glose}
\pfra{plonger la main pour creuser}
\end{glose}
\newline
\begin{sous-entrée}{taa phwa}{ⓔtaaⓢ1ⓝtaa phwa}
\begin{glose}
\pfra{creuser un trou, une tombe}
\end{glose}
\end{sous-entrée}
\newline
\sens{2}
(\domainesémantique{Cultures, techniques, boutures})
\begin{glose}
\pfra{déterrer les tubercules (ignames, taro de montagne)}
\end{glose}
\newline
\begin{sous-entrée}{taa kumwala}{ⓔtaaⓢ2ⓝtaa kumwala}
\begin{glose}
\pfra{déterrer des patates douces}
\end{glose}
\end{sous-entrée}
\newline
\begin{sous-entrée}{taa kui}{ⓔtaaⓢ2ⓝtaa kui}
\begin{glose}
\pfra{déterrer les ignames}
\end{glose}
\end{sous-entrée}
\newline
\begin{sous-entrée}{taa uvhia}{ⓔtaaⓢ2ⓝtaa uvhia}
\begin{glose}
\pfra{déterrer les taros de montagne}
\end{glose}
\end{sous-entrée}
\newline
\sens{3}
(\domainesémantique{Actions avec un instrument, un outil})
\begin{glose}
\pfra{piocher}
\end{glose}
\end{entrée}

\begin{entrée}{tãã}{}{ⓔtãã}
\région{GOs}
\classe{v}
\newline
\sens{1}
(\domainesémantique{Verbes de mouvement})
\begin{glose}
\pfra{surgir}
\end{glose}
\newline
\sens{2}
(\domainesémantique{Actions liées aux éléments (liquide, fumée)})
\begin{glose}
\pfra{gicler}
\end{glose}
\newline
\begin{sous-entrée}{pa-tãã}{ⓔtããⓢ2ⓝpa-tãã}
\begin{glose}
\pfra{faire gicler}
\end{glose}
\end{sous-entrée}
\end{entrée}

\begin{entrée}{taabö}{}{ⓔtaabö}
\région{GOs}
(\domainesémantique{Configuration des objets})
\classe{v ; n}
\begin{glose}
\pfra{séparer ; séparation ; frontière}
\end{glose}
\newline
\begin{sous-entrée}{taabö-wo mwa}{ⓔtaaböⓝtaabö-wo mwa}
\begin{glose}
\pfra{cloison}
\end{glose}
\end{sous-entrée}
\newline
\begin{sous-entrée}{menõ taabö-wo}{ⓔtaaböⓝmenõ taabö-wo}
\begin{glose}
\pfra{frontière (endroit de la séparation)}
\end{glose}
\end{sous-entrée}
\end{entrée}

\begin{entrée}{taabunõõ}{}{ⓔtaabunõõ}
\formephonétique{taːbuɳɔ̃ː}
\région{GOs}
(\domainesémantique{Actions liées aux éléments (liquide, fumée)})
\classe{v}
\begin{glose}
\pfra{ajouter du liquide}
\end{glose}
\end{entrée}

\begin{entrée}{taabwa pi}{}{ⓔtaabwa pi}
\région{PA}
(\domainesémantique{Fonctions naturelles des animaux})
\classe{v}
\begin{glose}
\pfra{couver des oeufs}
\end{glose}
\end{entrée}

\begin{entrée}{taagi kui}{}{ⓔtaagi kui}
\région{BO}
(\domainesémantique{Cultures, techniques, boutures})
\classe{v}
\begin{glose}
\pfra{replier la tigne d'igname sur elle-même (quand elle dépasse la hauteur du tuteur) [Corne]}
\end{glose}
\newline
\note{non vérifié}{général}{}
\end{entrée}

\begin{entrée}{taagine}{}{ⓔtaagine}
\formephonétique{taːŋgiɳe}
\région{GOs}
\variante{%
taagin
\région{PA BO}, 
tagin
\région{BO}, 
taagi
\région{WEM}}
\newline
\sens{1}
(\domainesémantique{Aspect})
\classe{v}
\begin{glose}
\pfra{continuer ; perpétuer}
\end{glose}
\newline
\begin{exemple}
\région{PA}
\textbf{\pnua{kêbwa co taagin !}}
\pfra{ne recommence pas ! (à un enfant)}
\end{exemple}
\newline
\sens{2}
(\domainesémantique{Aspect})
\classe{ADV}
\begin{glose}
\pfra{toujours ; tout le temps}
\end{glose}
\newline
\begin{exemple}
\région{PA}
\textbf{\pnua{i ra u nee-taagine}}
\pfra{il le fait continuellement}
\end{exemple}
\newline
\begin{exemple}
\région{PA}
\textbf{\pnua{i pajaa-je taagine}}
\pfra{il lui demande constamment}
\end{exemple}
\newline
\begin{exemple}
\région{PA}
\textbf{\pnua{i pajaa taagine yai je}}
\pfra{il lui demande constamment}
\end{exemple}
\newline
\begin{exemple}
\région{PA}
\textbf{\pnua{i ra ku-thiloo-je taagin u mwani}}
\pfra{il lui demande souvent de l'argent}
\end{exemple}
\newline
\begin{exemple}
\région{BO}
\textbf{\pnua{i pwal taagin}}
\pfra{il pleut tout le temps}
\end{exemple}
\newline
\begin{sous-entrée}{ô-taagin}{ⓔtaagineⓢ2ⓝô-taagin}
\begin{glose}
\pfra{constamment}
\end{glose}
\end{sous-entrée}
\newline
\note{taagine (v.t.)}{grammaire}{}
\end{entrée}

\begin{entrée}{taaja}{}{ⓔtaaja}
\région{GOs BO}
(\domainesémantique{Chasse
, Pêche})
\classe{v}
\begin{glose}
\pfra{piquer (à la sagaie)}
\end{glose}
\end{entrée}

\begin{entrée}{taa phwa}{}{ⓔtaa phwa}
\région{GOs}
\variante{%
taa-vwa
}
(\domainesémantique{Mouvements ou actions faits avec le corps, les bras, les mains, les pieds})
\classe{v}
\begin{glose}
\pfra{creuser un trou pour planter l'igname}
\end{glose}
\newline
\begin{exemple}
\textbf{\pnua{mi a taa-vwa-ni pomõ-ã}}
\pfra{allons préparer (les trous pour les plants dans) notre champ}
\end{exemple}
\newline
\begin{exemple}
\textbf{\pnua{e taa phwe kui}}
\pfra{il creuse les trous pour l'igname}
\end{exemple}
\newline
\note{taa-vwa-ni (v.t.)}{grammaire}{}
\end{entrée}

\begin{entrée}{taare}{}{ⓔtaare}
\région{BO}
(\domainesémantique{Feu : objets et actions liés au feu})
\classe{v}
\begin{glose}
\pfra{séparer ; dégager}
\end{glose}
\begin{glose}
\pfra{étaler (le feu du four, enlever le bois et éparpiller les braises) [BM, Corne]}
\end{glose}
\newline
\note{non verifié}{général}{}
\end{entrée}

\begin{entrée}{tabila}{}{ⓔtabila}
\région{BO PA WE}
(\domainesémantique{Actions avec un instrument, un outil
, Outils})
\classe{v}
\begin{glose}
\pfra{clouer}
\end{glose}
\end{entrée}

\begin{entrée}{tabwa}{}{ⓔtabwa}
\région{PA}
(\domainesémantique{Objets coutumiers})
\classe{nom}
\begin{glose}
\pfra{monnaie kanak}
\end{glose}
\newline
\note{monnaie dont la longueur est calculée en position assise, de la hauteur de la tête jusqu'au sol (Charles)}{glose}{}
\newline
\relationsémantique{Cf.}{\lien{}{kòòl, gò-hii}}
\glosecourte{monnaies}
\end{entrée}

\begin{entrée}{taçinô}{}{ⓔtaçinô}
\formephonétique{taʒiɳõ}
\région{GOs}
(\domainesémantique{Relations et interaction sociales})
\classe{v}
\begin{glose}
\pfra{embêter, empêcher, déranger}
\end{glose}
\newline
\begin{exemple}
\textbf{\pnua{taçinô-je !}}
\pfra{embête-le !}
\end{exemple}
\newline
\begin{exemple}
\textbf{\pnua{e pe-taçinô}}
\pfra{il est casse-pieds, embêtant}
\end{exemple}
\end{entrée}

\begin{entrée}{taçuuni}{}{ⓔtaçuuni}
\formephonétique{taʒuːɳi}
\région{GOs}
(\domainesémantique{Relations et interaction sociales})
\classe{v ; n}
\begin{glose}
\pfra{refuser la demande de pardon}
\end{glose}
\newline
\relationsémantique{Ant.}{\lien{ⓔuvwa}{uvwa}}
\glosecourte{accepter la demande de pardon}
\end{entrée}

\begin{entrée}{tae}{}{ⓔtae}
\région{PA BO [BM]}
(\domainesémantique{Mouvements ou actions faits avec le corps, les bras, les mains, les pieds})
\classe{v}
\begin{glose}
\pfra{allonger (le pas ou le bras pour saisir qqch)}
\end{glose}
\begin{glose}
\pfra{étendre (bras)}
\end{glose}
\newline
\begin{exemple}
\région{BOPA}
\textbf{\pnua{i tae kòò-n}}
\pfra{elle étend ses jambes}
\end{exemple}
\end{entrée}

\begin{entrée}{ta-ecâna}{}{ⓔta-ecâna}
\région{GO}
(\domainesémantique{Temps})
\classe{ADV}
\begin{glose}
\pfra{toujours (Haudricourt)}
\end{glose}
\newline
\note{non vérifié}{général}{}
\end{entrée}

\begin{entrée}{tagi}{}{ⓔtagi}
\région{PA BO}
\newline
\sens{1}
(\domainesémantique{Cordes, cordages})
\classe{nom}
\begin{glose}
\pfra{toron de corde}
\end{glose}
\newline
\sens{2}
(\domainesémantique{Mouvements ou actions faits avec le corps, les bras, les mains, les pieds})
\classe{v}
\begin{glose}
\pfra{enrouler (s') (une liane autour d'un arbre, la tige d'igname autour du tuteur) ;}
\end{glose}
\begin{glose}
\pfra{emmêler (s') dans une corde ;}
\end{glose}
\begin{glose}
\pfra{prendre (se) dans un filet (en s'enroulant)}
\end{glose}
\end{entrée}

\begin{entrée}{tagiliã}{}{ⓔtagiliã}
\région{GOs PA BO}
\variante{%
tãgilijã
\région{GO(s)}}
(\domainesémantique{Mollusques})
\classe{nom}
\begin{glose}
\pfra{bénitier (gastéropode)}
\end{glose}
\nomscientifique{Tridacna elongata}
\end{entrée}

\begin{entrée}{tago}{}{ⓔtago}
\région{GOs BO}
(\domainesémantique{Verbes d'action (en général)})
\classe{v}
\begin{glose}
\pfra{arrêter}
\end{glose}
\newline
\begin{exemple}
\textbf{\pnua{nu cabòl yai jo wu nu tago je}}
\pfra{je t'apparaîs pour t'arrêter (de faire qqch)}
\end{exemple}
\end{entrée}

\begin{entrée}{tagooni}{}{ⓔtagooni}
\région{GOs}
(\domainesémantique{Mollusques})
\classe{nom}
\begin{glose}
\pfra{coquille saint-jacques}
\end{glose}
\end{entrée}

\begin{entrée}{tãî}{}{ⓔtãî}
\région{PA BO}
\classe{nom}
\newline
\sens{1}
(\domainesémantique{Vêtements, parure})
\begin{glose}
\pfra{vêtements}
\end{glose}
\newline
\begin{exemple}
\région{PA}
\textbf{\pnua{i thu tãî kaavo}}
\pfra{kaavo s'habille}
\end{exemple}
\newline
\begin{exemple}
\région{BO}
\textbf{\pnua{tãî-m}}
\pfra{tes vêtements [BM]}
\end{exemple}
\newline
\sens{2}
(\domainesémantique{Cours de la vie})
\begin{glose}
\pfra{ornement de deuil}
\end{glose}
\end{entrée}

\begin{entrée}{tali}{}{ⓔtali}
\région{GOs PA BO}
(\domainesémantique{Cultures, techniques, boutures})
\classe{v}
\begin{glose}
\pfra{arracher (paille, herbes, lianes)}
\end{glose}
\newline
\begin{exemple}
\textbf{\pnua{i tali mèròò}}
\pfra{il arrache}
\end{exemple}
\newline
\begin{exemple}
\région{PA}
\textbf{\pnua{i tali wal}}
\pfra{il arrache des lianes}
\end{exemple}
\newline
\begin{exemple}
\région{PA}
\textbf{\pnua{i tali mae}}
\pfra{il arrache des herbes (à paille)}
\end{exemple}
\end{entrée}

\begin{entrée}{taluang}{}{ⓔtaluang}
\région{PA}
(\domainesémantique{Cultures, techniques, boutures})
\classe{v}
\begin{glose}
\pfra{ravager les champs (pour des animaux)}
\end{glose}
\begin{glose}
\pfra{ne pas respecter les principes des cultures ou de chasse (ou de la nature en général)}
\end{glose}
\newline
\begin{exemple}
\région{PA}
\textbf{\pnua{la taluang êê-nu xu choval}}
\pfra{les chevaux ont ravagé mes plants}
\end{exemple}
\end{entrée}

\begin{entrée}{tãnim}{}{ⓔtãnim}
\région{BO}
(\domainesémantique{Armes})
\classe{nom}
\begin{glose}
\pfra{pierre noire (qui sert à faire des pierres de fronde) [Corne]}
\end{glose}
\newline
\note{non vérifié}{général}{}
\end{entrée}

\begin{entrée}{taru}{}{ⓔtaru}
\région{GOs}
(\domainesémantique{Ignames})
\classe{nom}
\begin{glose}
\pfra{igname (violette)}
\end{glose}
\end{entrée}

\begin{entrée}{taue}{}{ⓔtaue}
\région{BO}
\classe{v}
\begin{glose}
\pfra{juste ; droit [Corne]}
\end{glose}
\newline
\note{non vérifié}{général}{}
\end{entrée}

\begin{entrée}{ta-uuni}{}{ⓔta-uuni}
\formephonétique{tauːɳi}
\région{GOs}
(\domainesémantique{Mouvements ou actions faits avec le corps, les bras, les mains, les pieds})
\classe{v}
\begin{glose}
\pfra{renverser ; faire tomber}
\end{glose}
\newline
\begin{exemple}
\région{GOs}
\textbf{\pnua{la ta-uuni la-ã ba-raabwa}}
\pfra{ils ont renversé ces chaises}
\end{exemple}
\end{entrée}

\begin{entrée}{tav}{}{ⓔtav}
\région{PA}
(\domainesémantique{Objets et meubles de la maison})
\classe{nom}
\begin{glose}
\pfra{table}
\end{glose}
\newline
\emprunt{table (FR)}
\end{entrée}

\begin{entrée}{taxaza}{}{ⓔtaxaza}
\région{GOs}
(\domainesémantique{Remèdes, médecine})
\classe{nom}
\begin{glose}
\pfra{docteur}
\end{glose}
\end{entrée}

\begin{entrée}{taye}{}{ⓔtaye}
\région{BO}
(\domainesémantique{Corps humain})
\classe{nom}
\begin{glose}
\pfra{tibia (Dubois)}
\end{glose}
\newline
\note{non vérifié}{général}{}
\end{entrée}

\begin{entrée}{ta-zo}{}{ⓔta-zo}
\région{PA BO}
(\domainesémantique{Caractéristiques et propriétés des personnes})
\classe{v}
\begin{glose}
\pfra{agile ; prompt ; vif}
\end{glose}
\newline
\begin{exemple}
\région{BO}
\textbf{\pnua{mha ta-zo kòò-n}}
\pfra{il a les jambes très agiles}
\end{exemple}
\end{entrée}

\begin{entrée}{te}{}{ⓔte}
\région{PA}
(\domainesémantique{Processus liés aux plantes})
\classe{v}
\begin{glose}
\pfra{commencer à mûrir (tubercules, fruits)}
\end{glose}
\end{entrée}

\begin{entrée}{tee}{}{ⓔtee}
\région{PA}
(\domainesémantique{Marques restrictives})
\classe{RESTR}
\begin{glose}
\pfra{seulement}
\end{glose}
\newline
\begin{exemple}
\région{PA}
\textbf{\pnua{tee axe pòi-n}}
\pfra{il a un seul enfant}
\end{exemple}
\newline
\begin{exemple}
\région{PA}
\textbf{\pnua{tee aru pòi-n}}
\pfra{il a seulement deux enfants}
\end{exemple}
\newline
\begin{exemple}
\région{PA}
\textbf{\pnua{ra tee akòn (a) pòi-n}}
\pfra{il n'a que trois enfants}
\end{exemple}
\newline
\begin{exemple}
\région{PA}
\textbf{\pnua{ra tee we-kòn (a) wony a nu nooli}}
\pfra{je n'ai vu que trois bateaux}
\end{exemple}
\end{entrée}

\begin{entrée}{tee-axe}{}{ⓔtee-axe}
\région{PA}
(\domainesémantique{Quantificateurs})
\classe{QNT}
\begin{glose}
\pfra{unique ; seul}
\end{glose}
\newline
\begin{exemple}
\région{PA}
\textbf{\pnua{tee-axe pòi-n}}
\pfra{son unique enfant}
\end{exemple}
\newline
\relationsémantique{Cf.}{\lien{}{(h)ada}}
\glosecourte{seul}
\end{entrée}

\begin{entrée}{teele}{}{ⓔteele}
\région{PA}
\classe{v}
(\domainesémantique{Fonctions naturelles humaines})
\begin{glose}
\pfra{fatigué}
\end{glose}
\end{entrée}

\begin{entrée}{tèè-na}{}{ⓔtèè-na}
\région{PA BO}
(\domainesémantique{Dons, échanges, achat et vente, vol})
\classe{v}
\begin{glose}
\pfra{prêter}
\end{glose}
\end{entrée}

\begin{entrée}{tèèn hãgana}{}{ⓔtèèn hãgana}
\région{PA}
(\domainesémantique{Adverbes déictiques de temps})
\classe{LOCUT}
\begin{glose}
\pfra{aujourd'hui}
\end{glose}
\end{entrée}

\begin{entrée}{tèèn ne hêbuun}{}{ⓔtèèn ne hêbuun}
\région{PA}
(\domainesémantique{Adverbes déictiques de temps})
\classe{LOCUT}
\begin{glose}
\pfra{il y a 3 jours}
\end{glose}
\end{entrée}

\begin{entrée}{teevaò}{}{ⓔteevaò}
\région{BO}
(\domainesémantique{Actions liées aux éléments (liquide, fumée)})
\classe{v}
\begin{glose}
\pfra{jeter (dans l'eau) [BM, Coyaud]}
\end{glose}
\end{entrée}

\begin{entrée}{teevwuun}{}{ⓔteevwuun}
\région{PA}
\variante{%
tee-puu-n
}
(\domainesémantique{Verbes d'action (en général)})
\classe{v}
\begin{glose}
\pfra{faire avant ; commencer avant}
\end{glose}
\newline
\begin{exemple}
\textbf{\pnua{co teevwuun !}}
\pfra{commence avant !}
\end{exemple}
\end{entrée}

\begin{entrée}{têi}{}{ⓔtêi}
\région{GOs BO PA}
(\domainesémantique{Corps humain})
\classe{nom}
\begin{glose}
\pfra{morve}
\end{glose}
\newline
\begin{exemple}
\région{GOs}
\textbf{\pnua{e tho têi-jö}}
\pfra{tu as le nez qui coule (lit. ta morve coule)}
\end{exemple}
\newline
\begin{exemple}
\région{GOs}
\textbf{\pnua{nhrile têi-jö}}
\pfra{mouche ta morve}
\end{exemple}
\newline
\begin{exemple}
\région{PA}
\textbf{\pnua{têi-m}}
\pfra{ta morve}
\end{exemple}
\newline
\begin{exemple}
\textbf{\pnua{e tûûne têi-n}}
\pfra{il se mouche (lit. essuie sa morve)}
\end{exemple}
\end{entrée}

\begin{entrée}{tèn}{}{ⓔtèn}
\région{BO}
(\domainesémantique{Navigation})
\classe{nom}
\begin{glose}
\pfra{pirogue double [Corne]}
\end{glose}
\newline
\note{non vérifié}{général}{}
\end{entrée}

\begin{entrée}{tene}{}{ⓔtene}
\région{BO}
(\domainesémantique{Religion, représentations religieuses})
\classe{v ; n}
\begin{glose}
\pfra{blasphémer ; blasphème (Dubois)}
\end{glose}
\newline
\note{non vérifié}{général}{}
\end{entrée}

\begin{entrée}{tèng}{}{ⓔtèng}
\région{BO}
(\domainesémantique{Configuration des objets})
\classe{v ; n}
\begin{glose}
\pfra{fourche [BM]}
\end{glose}
\begin{glose}
\pfra{brancher}
\end{glose}
\newline
\étymologie{
\langue{POc}
\étymon{*saŋa}
\auteur{Blust}}
\end{entrée}

\begin{entrée}{teol}{}{ⓔteol}
\région{BO}
\variante{%
teo
\région{BO}}
(\domainesémantique{Mouvements ou actions faits avec le corps, les bras, les mains, les pieds})
\classe{v}
\begin{glose}
\pfra{déchirer (tissu) [BM]}
\end{glose}
\newline
\begin{exemple}
\région{BO}
\textbf{\pnua{i teo kii-n}}
\pfra{il a déchiré son manou}
\end{exemple}
\end{entrée}

\begin{entrée}{terè}{}{ⓔterè}
\région{PA BO}
(\domainesémantique{Description des objets, formes, consistance, taille})
\classe{v.stat.}
\begin{glose}
\pfra{mince ; fin}
\end{glose}
\end{entrée}

\begin{entrée}{tewai}{}{ⓔtewai}
\région{BO PA}
(\domainesémantique{Armes})
\classe{nom}
\begin{glose}
\pfra{casse-tête à bout rond}
\end{glose}
\end{entrée}

\begin{entrée}{texee}{}{ⓔtexee}
\région{BO}
(\domainesémantique{Verbes d'action (en général)})
\classe{v}
\begin{glose}
\pfra{serrer (noeud) [Corne]}
\end{glose}
\newline
\note{non vérifié}{général}{}
\end{entrée}

\begin{entrée}{ti ?}{}{ⓔti ?}
\région{GO PA BO}
(\domainesémantique{Interrogatifs})
\classe{INT}
\begin{glose}
\pfra{qui ?}
\end{glose}
\newline
\sens{1}
\classe{pronom interrogatif}
\newline
\begin{exemple}
\région{GO}
\textbf{\pnua{ti nye uça ne ?}}
\pfra{qui est arrivé ici ?}
\end{exemple}
\newline
\begin{exemple}
\région{GO}
\textbf{\pnua{ti thoomwã ne jö kha-whili-je na}}
\pfra{qui est la femme que tu as amené là ?}
\end{exemple}
\newline
\begin{exemple}
\région{PA}
\textbf{\pnua{ti-xa na i a ?}}
\pfra{qui est-ce qui estparti ? (xa : indéfini)}
\end{exemple}
\newline
\begin{exemple}
\région{PA}
\textbf{\pnua{ti je i a ?}}
\pfra{qui est parti ?}
\end{exemple}
\newline
\begin{exemple}
\région{GO}
\textbf{\pnua{ti nye ?}}
\pfra{qui est-ce ?}
\end{exemple}
\newline
\begin{exemple}
\textbf{\pnua{ti jo ?}}
\pfra{qui es-tu ?}
\end{exemple}
\newline
\begin{exemple}
\région{PA}
\textbf{\pnua{ti gi ne ?}}
\pfra{qui est là ?}
\end{exemple}
\newline
\begin{exemple}
\région{PA}
\textbf{\pnua{ti nyoli ?}}
\pfra{qui est là ?}
\end{exemple}
\newline
\sens{2}
\classe{interrogatif possessif}
\newline
\begin{exemple}
\région{WE PA}
\textbf{\pnua{hèlèè ri ?}}
\pfra{à qui est le couteau ? c'est le couteau de qui ?}
\end{exemple}
\newline
\begin{exemple}
\région{GOs (*ri?)}
\textbf{\pnua{hèlèè ti nye ?}}
\pfra{à qui est le couteau ? c'est le couteau de qui ?}
\end{exemple}
\newline
\begin{exemple}
\textbf{\pnua{ôa ti nye ?}}
\pfra{c'est la mère de qui celle-là ?}
\end{exemple}
\newline
\begin{exemple}
\région{PA}
\textbf{\pnua{kêê ti / ri ?}}
\pfra{le père de qui ?}
\end{exemple}
\newline
\étymologie{
\langue{POc}
\étymon{*(n)sai}}
\newline
\note{ri ?}{grammaire}{qui ?}
\end{entrée}

\begin{entrée}{tia}{}{ⓔtia}
\région{GOs BO}
\variante{%
tiia
\région{GO(s)}}
(\domainesémantique{Mouvements ou actions faits avec le corps, les bras, les mains, les pieds})
\classe{v}
\begin{glose}
\pfra{pousser horizontalement ; pousser qqn à faire qqch}
\end{glose}
\newline
\begin{exemple}
\textbf{\pnua{tia loto}}
\pfra{pousser une voiture}
\end{exemple}
\end{entrée}

\begin{entrée}{tibö}{}{ⓔtibö}
\région{PA}
\variante{%
tibu
\région{PA}}
(\domainesémantique{Préfixes classificateurs numériques})
\classe{CLF.NUM}
\begin{glose}
\pfra{grappes (par ex. de tomates, orange)}
\end{glose}
\newline
\begin{exemple}
\textbf{\pnua{tibö-xe tibö toomwa}}
\pfra{une grappe de tomates (ne s'emploie pas au-delà de 'un')}
\end{exemple}
\end{entrée}

\begin{entrée}{tigi}{1}{ⓔtigiⓗ1}
\région{GOs PA BO}
\variante{%
tigin
\région{WE}}
\classe{v}
\newline
\sens{1}
(\domainesémantique{Description des objets, formes, consistance, taille})
\begin{glose}
\pfra{épais ; emmêlé ; inextricable}
\end{glose}
\begin{glose}
\pfra{pris (dans un filet)}
\end{glose}
\begin{glose}
\pfra{coincé ; bloqué}
\end{glose}
\newline
\begin{exemple}
\textbf{\pnua{e tigi pu-bwaa-jö}}
\pfra{tes cheveux sont emmêlés}
\end{exemple}
\newline
\sens{2}
(\domainesémantique{Verbes d'action (en général)})
\begin{glose}
\pfra{mettre des obstacles ; entraver}
\end{glose}
\newline
\begin{exemple}
\région{GOs}
\textbf{\pnua{la pe-tigi}}
\pfra{ils sont en conflit (bloqués par des problèmes)}
\end{exemple}
\newline
\begin{exemple}
\région{WE}
\textbf{\pnua{li pe-tigin}}
\pfra{ils sont en conflit}
\end{exemple}
\newline
\note{pa-tigi-ni}{grammaire}{mettre des obstacles}
\end{entrée}

\begin{entrée}{tigi}{2}{ⓔtigiⓗ2}
\région{GOs PA BO}
\classe{v ; n}
\newline
\sens{1}
(\domainesémantique{Description des objets, formes, consistance, taille})
\begin{glose}
\pfra{englué ; collé ; emmêlé ; embrouillé ; enchevêtré}
\end{glose}
\newline
\sens{2}
(\domainesémantique{Instruments
, Chasse})
\begin{glose}
\pfra{bâton à glu (enduit de colle de fruit du gommier, utilisé pour attraper les cigales en leur collant les ailes)}
\end{glose}
\newline
\sens{3}
(\domainesémantique{Coutumes, dons coutumiers})
\begin{glose}
\pfra{réfère aux cérémonie funéraires (perturbation de l'ordre social: D. Bretteville)}
\end{glose}
\end{entrée}

\begin{entrée}{tigi}{3}{ⓔtigiⓗ3}
\région{GOs BO PA}
\classe{nom}
\newline
\sens{1}
(\domainesémantique{Végétation})
\begin{glose}
\pfra{forêt impraticable ; fourré}
\end{glose}
\newline
\sens{2}
(\domainesémantique{Description des objets, formes, consistance, taille})
\begin{glose}
\pfra{dense}
\end{glose}
\newline
\begin{exemple}
\région{BO}
\textbf{\pnua{ko tigi}}
\pfra{forêt dense}
\end{exemple}
\end{entrée}

\begin{entrée}{tii}{}{ⓔtii}
\région{GOs}
\variante{%
tii-n
\région{PA BO}}
(\domainesémantique{Fonctions intellectuelles})
\classe{v}
\begin{glose}
\pfra{écrire ; écriture}
\end{glose}
\begin{glose}
\pfra{marquer ; graver}
\end{glose}
\newline
\begin{exemple}
\région{BO}
\textbf{\pnua{tii-n}}
\pfra{son écriture}
\end{exemple}
\newline
\begin{exemple}
\région{BO}
\textbf{\pnua{i tii yaal na bwa ce}}
\pfra{il écrit/grave son nom sur l'arbre}
\end{exemple}
\newline
\begin{sous-entrée}{we-tii [GOs], we-tiin [PA, BO]}{ⓔtiiⓝwe-tii [GOs], we-tiin [PA, BO]}
\begin{glose}
\pfra{encre}
\end{glose}
\end{sous-entrée}
\end{entrée}

\begin{entrée}{tii-hênu}{}{ⓔtii-hênu}
\formephonétique{tiː hêɳu}
\région{GOs}
(\domainesémantique{Fonctions intellectuelles})
\classe{v}
\begin{glose}
\pfra{dessiner}
\end{glose}
\end{entrée}

\begin{entrée}{tiiu}{}{ⓔtiiu}
\région{BO}
(\domainesémantique{Santé, maladie})
\classe{v}
\begin{glose}
\pfra{piquer ; démanger (comme une plaie sous l'effet de l'alcool) [BM]}
\end{glose}
\newline
\begin{exemple}
\textbf{\pnua{i tiiu}}
\pfra{ça pique}
\end{exemple}
\end{entrée}

\begin{entrée}{tiivwo}{}{ⓔtiivwo}
\région{GOs PA}
(\domainesémantique{Fonctions intellectuelles})
\classe{nom}
\begin{glose}
\pfra{lettre ; livre}
\end{glose}
\end{entrée}

\begin{entrée}{tiivwo kabu}{}{ⓔtiivwo kabu}
\région{GOs}
(\domainesémantique{Religion, représentations religieuses})
\classe{nom}
\begin{glose}
\pfra{Bible}
\end{glose}
\end{entrée}

\begin{entrée}{tîî-yaai}{}{ⓔtîî-yaai}
\région{WEM}
(\domainesémantique{Feu : objets et actions liés au feu})
\classe{v ; n}
\begin{glose}
\pfra{étincelle du feu ; crépiter (feu)}
\end{glose}
\end{entrée}

\begin{entrée}{tikudi}{}{ⓔtikudi}
\région{GOs}
(\domainesémantique{Mer : topographie})
\classe{nom}
\begin{glose}
\pfra{havre}
\end{glose}
\end{entrée}

\begin{entrée}{tili}{}{ⓔtili}
\région{GOs}
(\domainesémantique{Processus liés aux plantes})
\classe{v}
\begin{glose}
\pfra{multiplier (se) ; faire des feuilles (arbres)}
\end{glose}
\newline
\begin{exemple}
\textbf{\pnua{e tili ce}}
\pfra{l'arbre se couvre de feuilles}
\end{exemple}
\newline
\begin{exemple}
\textbf{\pnua{e tili kui}}
\pfra{l'igname fait des feuilles}
\end{exemple}
\end{entrée}

\begin{entrée}{tio}{}{ⓔtio}
\région{PA}
\variante{%
tioo
\région{PA}}
\classe{v}
\newline
\sens{1}
(\domainesémantique{Mouvements ou actions faits avec le corps, les bras, les mains, les pieds})
\begin{glose}
\pfra{déchirer ; faire un accroc}
\end{glose}
\newline
\begin{exemple}
\région{PA}
\textbf{\pnua{i tio patalõ}}
\pfra{il a fait un accroc à son pantalon}
\end{exemple}
\newline
\begin{exemple}
\région{PA}
\textbf{\pnua{i tio kii-n}}
\pfra{il a déchiré son manou}
\end{exemple}
\newline
\sens{2}
(\domainesémantique{Mouvements ou actions faits avec le corps, les bras, les mains, les pieds})
\begin{glose}
\pfra{érafler ; écorcher (s')}
\end{glose}
\newline
\begin{exemple}
\région{PA}
\textbf{\pnua{tio kòò-n}}
\pfra{éraflure, écorchure sur sa jambe}
\end{exemple}
\newline
\begin{exemple}
\région{PA}
\textbf{\pnua{i kha-tio kòò-n}}
\pfra{il s'est écorché la jambe (en marchant)}
\end{exemple}
\end{entrée}

\begin{entrée}{tithaa}{}{ⓔtithaa}
\région{PA}
(\domainesémantique{Verbes d'action (en général)})
\classe{v}
\begin{glose}
\pfra{effleurer ; ricocher}
\end{glose}
\end{entrée}

\begin{entrée}{ti-xa ?}{}{ⓔti-xa ?}
\région{PA}
(\domainesémantique{Interrogatifs})
\classe{INT (indéfini)}
\begin{glose}
\pfra{qui donc ?}
\end{glose}
\newline
\begin{exemple}
\région{PA}
\textbf{\pnua{ti xa egu na i cabi mwa ?}}
\pfra{qui a bien pu frapper à la maison ?}
\end{exemple}
\end{entrée}

\begin{entrée}{tixãã}{}{ⓔtixãã}
\région{WE}
\variante{%
tilixãi
\région{BO}}
(\domainesémantique{Sentiments})
\classe{v}
\begin{glose}
\pfra{colère (être en grande)}
\end{glose}
\end{entrée}

\begin{entrée}{ti-yaai}{}{ⓔti-yaai}
\région{GOs}
(\domainesémantique{Feu : objets et actions liés au feu})
\classe{nom}
\begin{glose}
\pfra{suie (du feu) (sur le toit ou les marmites)}
\end{glose}
\newline
\begin{exemple}
\textbf{\pnua{bö ti-yaai}}
\pfra{l'odeur du feu l}
\end{exemple}
\end{entrée}

\begin{entrée}{tiyôô}{}{ⓔtiyôô}
\région{PA}
(\domainesémantique{Verbes de déplacement et moyens de déplacement})
\classe{v}
\begin{glose}
\pfra{passer sous une barrière}
\end{glose}
\end{entrée}

\begin{entrée}{tò}{}{ⓔtò}
\région{GO}
(\domainesémantique{Description des objets, formes, consistance, taille})
\classe{v.stat.}
\begin{glose}
\pfra{mou ; lisse (cheveux)}
\end{glose}
\end{entrée}

\begin{entrée}{tò do}{}{ⓔtò do}
\région{GOs PA BO}
(\domainesémantique{Guerre})
\classe{v}
\begin{glose}
\pfra{paix (faire la) (lit. lancer la sagaie)}
\end{glose}
\newline
\relationsémantique{Cf.}{\lien{ⓔtòè}{tòè}}
\glosecourte{lancer}
\end{entrée}

\begin{entrée}{tòè}{}{ⓔtòè}
\région{GOs PA BO}
\classe{v}
\newline
\sens{1}
(\domainesémantique{Armes})
\begin{glose}
\pfra{lancer (sagaie, etc.)}
\end{glose}
\newline
\begin{exemple}
\région{GOs}
\textbf{\pnua{nu tò do}}
\pfra{je lance la sagaie, fais la paix}
\end{exemple}
\newline
\sens{2}
(\domainesémantique{Mouvements ou actions faits avec le corps, les bras, les mains, les pieds})
\begin{glose}
\pfra{piquer avec la sagaie}
\end{glose}
\begin{glose}
\pfra{donner un coup}
\end{glose}
\newline
\begin{exemple}
\région{PA}
\textbf{\pnua{nu tòè}}
\pfra{je l'ai lancé}
\end{exemple}
\newline
\begin{exemple}
\textbf{\pnua{mo a-tò do}}
\pfra{on va faire du lancer de sagaie}
\end{exemple}
\newline
\begin{exemple}
\région{BO}
\textbf{\pnua{i tò je doo-n, to-kòòl ni wèngè-n}}
\pfra{il lance sa sagaie, qui se fiche dans sa poitrine [Coyaud]}
\end{exemple}
\newline
\begin{sous-entrée}{to-puu}{ⓔtòèⓢ2ⓝto-puu}
\begin{glose}
\pfra{pousser avec la perche (bateau)}
\end{glose}
\end{sous-entrée}
\newline
\begin{sous-entrée}{to-vaò (< tòè-phaò)}{ⓔtòèⓢ2ⓝto-vaò (< tòè-phaò)}
\begin{glose}
\pfra{piquer avec la sagaie}
\end{glose}
\end{sous-entrée}
\newline
\sens{3}
(\domainesémantique{Verbes de déplacement et moyens de déplacement})
\begin{glose}
\pfra{filer comme une flèche ; courir à toute allure}
\end{glose}
\newline
\begin{exemple}
\région{GOs}
\textbf{\pnua{e za tòè ã ẽnõ ba}}
\pfra{l'enfant là-bas court à toute vitesse}
\end{exemple}
\newline
\sens{4}
(\domainesémantique{Jeux divers})
\begin{glose}
\pfra{figure du jeu de ficelle "la sagaie"}
\end{glose}
\end{entrée}

\begin{entrée}{tòè-nããn}{}{ⓔtòè-nããn}
\région{PA BO}
(\domainesémantique{Relations et interaction sociales})
\classe{v}
\begin{glose}
\pfra{lancer des injures, des offenses (piquer)}
\end{glose}
\begin{glose}
\pfra{injurier ; offenser}
\end{glose}
\newline
\begin{exemple}
\région{PA}
\textbf{\pnua{li pe-tòè-li nããn}}
\pfra{ils se lancent des insultes}
\end{exemple}
\newline
\relationsémantique{Cf.}{\lien{}{phao paxa-nããn [PA]}}
\glosecourte{lancer des injures, des offenses}
\end{entrée}

\begin{entrée}{tòò}{1}{ⓔtòòⓗ1}
\région{GOs PABO}
\classe{v.stat.}
\newline
\sens{1}
(\domainesémantique{Fonctions naturelles humaines})
\begin{glose}
\pfra{chaud (avoir)}
\end{glose}
\newline
\begin{exemple}
\région{GO}
\textbf{\pnua{nu tòò}}
\pfra{j'ai chaud}
\end{exemple}
\newline
\begin{exemple}
\région{GO}
\textbf{\pnua{tòò a}}
\pfra{le soleil chauffe}
\end{exemple}
\newline
\begin{exemple}
\textbf{\pnua{mã tòò-raa !}}
\pfra{il fait trop chaud !}
\end{exemple}
\newline
\begin{sous-entrée}{mã tòò}{ⓔtòòⓗ1ⓢ1ⓝmã tòò}
\begin{glose}
\pfra{très chaud}
\end{glose}
\end{sous-entrée}
\newline
\begin{sous-entrée}{kavwu mã tòò}{ⓔtòòⓗ1ⓢ1ⓝkavwu mã tòò}
\begin{glose}
\pfra{pas trop chaud}
\end{glose}
\end{sous-entrée}
\newline
\begin{sous-entrée}{we tòò; we-ròò}{ⓔtòòⓗ1ⓢ1ⓝwe tòò; we-ròò}
\begin{glose}
\pfra{eau chaude ; eau brûlante}
\end{glose}
\end{sous-entrée}
\newline
\begin{sous-entrée}{pa-toè droo}{ⓔtòòⓗ1ⓢ1ⓝpa-toè droo}
\begin{glose}
\pfra{réchauffer la marmite}
\end{glose}
\end{sous-entrée}
\newline
\sens{2}
(\domainesémantique{Feu : objets et actions liés au feu})
\begin{glose}
\pfra{rougi ; enflammé [BO]}
\end{glose}
\begin{glose}
\pfra{brûler ; brûlant}
\end{glose}
\newline
\begin{exemple}
\région{BO}
\textbf{\pnua{i tòò hi-m ?}}
\pfra{tu t'es brûlé la main ?}
\end{exemple}
\end{entrée}

\begin{entrée}{tòò}{2}{ⓔtòòⓗ2}
\région{PA BO}
(\domainesémantique{Relations et interaction sociales})
\classe{v}
\begin{glose}
\pfra{rencontrer, trouver qqn}
\end{glose}
\newline
\begin{exemple}
\région{BO}
\textbf{\pnua{yò tha tòò-nu mwa pe-pwexu}}
\pfra{vous(2) êtes venus (tha) me trouver pour que nous (mwa) discutions}
\end{exemple}
\newline
\note{tòòli}{grammaire}{trouver qqch., qqn}
\end{entrée}

\begin{entrée}{töö}{1}{ⓔtööⓗ1}
\région{GOs BO}
\classe{v}
\newline
\sens{1}
(\domainesémantique{Tressage})
\begin{glose}
\pfra{couper en lanières (des fibres de pandanus pour le tressage)}
\end{glose}
\newline
\begin{exemple}
\textbf{\pnua{e töö nu-pho}}
\pfra{elle coupe en lanières des fibres à tresser}
\end{exemple}
\newline
\sens{2}
(\domainesémantique{Actions liées aux plantes})
\begin{glose}
\pfra{écorcer (enlever les épines)}
\end{glose}
\end{entrée}

\begin{entrée}{töö}{2}{ⓔtööⓗ2}
\région{GOs PA WEM}
\variante{%
too
\région{PA BO}}
(\domainesémantique{Verbes de déplacement et moyens de déplacement})
\classe{v}
\begin{glose}
\pfra{ramper (enfant)}
\end{glose}
\begin{glose}
\pfra{marcher à 4 pattes}
\end{glose}
\begin{glose}
\pfra{hisser (se)}
\end{glose}
\begin{glose}
\pfra{déplacer (se) sans bruit}
\end{glose}
\begin{glose}
\pfra{aller de nuit taper à la fenêtre d'une fille}
\end{glose}
\newline
\begin{exemple}
\région{GO}
\textbf{\pnua{e thumenõ ka-töö}}
\pfra{il marche courbé (vieillard)}
\end{exemple}
\newline
\begin{sous-entrée}{kô-töö}{ⓔtööⓗ2ⓝkô-töö}
\begin{glose}
\pfra{incliné}
\end{glose}
\newline
\relationsémantique{Cf.}{\lien{}{beela [GOs]}}
\glosecourte{ramper (reptile)}
\end{sous-entrée}
\end{entrée}

\begin{entrée}{töö-pho}{}{ⓔtöö-pho}
\région{GOs}
(\domainesémantique{Mouvements ou actions faits avec le corps, les bras, les mains, les pieds})
\classe{nom}
\begin{glose}
\pfra{faire des lamelles de feuilles de pandanus}
\end{glose}
\end{entrée}

\begin{entrée}{tòòri}{}{ⓔtòòri}
\région{GOs}
\variante{%
toorim
\région{PA}}
(\domainesémantique{Préparation des aliments; modes de préparation et de cuisson})
\classe{v}
\begin{glose}
\pfra{brûlé, carbonisé (au fond de la marmite)}
\end{glose}
\newline
\begin{exemple}
\région{GOs}
\textbf{\pnua{e tòòri lai}}
\pfra{le riz est brûlé}
\end{exemple}
\newline
\begin{exemple}
\région{PA}
\textbf{\pnua{tõõne bö toorim}}
\pfra{ça sent le (l'odeur de) brûlé}
\end{exemple}
\end{entrée}

\begin{entrée}{tòòwu}{}{ⓔtòòwu}
\région{BO [BM]}
\variante{%
thòòwu
}
(\domainesémantique{Modalité, verbes modaux})
\classe{v}
\begin{glose}
\pfra{vouloir}
\end{glose}
\newline
\begin{exemple}
\région{BO}
\textbf{\pnua{i tòòwu-ny na nu phe}}
\pfra{je veux le prendre}
\end{exemple}
\newline
\note{non verifié}{général}{}
\end{entrée}

\begin{entrée}{toro}{}{ⓔtoro}
\région{BO}
(\domainesémantique{Reptiles})
\classe{nom}
\begin{glose}
\pfra{gecko ; margouillat}
\end{glose}
\newline
\relationsémantique{Cf.}{\lien{ⓔmajo}{majo}}
\end{entrée}

\begin{entrée}{tou}{}{ⓔtou}
\région{GOs PA BO}
\newline
\groupe{A}
(\domainesémantique{Dons, échanges, achat et vente, vol})
\classe{v ; n}
\begin{glose}
\pfra{partager ; distribuer}
\end{glose}
\begin{glose}
\pfra{partage (dans les fêtes coutumières)}
\end{glose}
\newline
\begin{sous-entrée}{thi-tou [GOs, PA]}{ⓔtouⓝthi-tou [GOs, PA]}
\begin{glose}
\pfra{distribuer en partage}
\end{glose}
\end{sous-entrée}
\newline
\begin{sous-entrée}{phwe-tou}{ⓔtouⓝphwe-tou}
\région{PA}
\begin{glose}
\pfra{recevoir en partage}
\end{glose}
\end{sous-entrée}
\newline
\begin{sous-entrée}{pe-tou}{ⓔtouⓝpe-tou}
\région{PA}
\begin{glose}
\pfra{se partager}
\end{glose}
\newline
\begin{exemple}
\région{BO}
\textbf{\pnua{mi pe-toe la abaan}}
\pfra{nous nous sommes partagés les parts}
\end{exemple}
\newline
\relationsémantique{Cf.}{\lien{}{toe}}
\glosecourte{partager qqch.}
\end{sous-entrée}
\newline
\groupe{B}
(\domainesémantique{Préfixes classificateurs numériques})
\classe{CLF.NUM}
\begin{glose}
\pfra{aliments enveloppés dans des feuilles}
\end{glose}
\begin{glose}
\pfra{tas (distribué dans des cérémonies coutumières)}
\end{glose}
\newline
\begin{exemple}
\textbf{\pnua{tou-xe, tou-tru}}
\pfra{un, deux tas}
\end{exemple}
\newline
\begin{exemple}
\région{GOs}
\textbf{\pnua{içö tou-xe}}
\pfra{un tas pour toi}
\end{exemple}
\end{entrée}

\begin{entrée}{tourou}{}{ⓔtourou}
\région{GOs WEM}
(\domainesémantique{Oiseaux})
\classe{nom}
\begin{glose}
\pfra{tourou ; Grive perlée}
\end{glose}
\nomscientifique{Phylidonyris undulata (Méliphagidés)}
\end{entrée}

\begin{entrée}{tò-vhaò}{}{ⓔtò-vhaò}
\région{GOs}
(\domainesémantique{Pêche})
\classe{v}
\begin{glose}
\pfra{piquer à la sagaie}
\end{glose}
\newline
\morphologie{tòe-phaò 'piquer-jeter'}
\end{entrée}

\begin{entrée}{tua}{}{ⓔtua}
\région{GOs BO PA}
(\domainesémantique{Mouvements ou actions faits avec le corps, les bras, les mains, les pieds})
\classe{v}
\begin{glose}
\pfra{dénouer ; défaire (noeud) ; détacher}
\end{glose}
\newline
\begin{exemple}
\textbf{\pnua{e pe-tua hada}}
\pfra{il s'est défait tout seul}
\end{exemple}
\newline
\relationsémantique{Cf.}{\lien{ⓔthaò}{thaò}}
\glosecourte{dénouer}
\end{entrée}

\begin{entrée}{tua pwaala}{}{ⓔtua pwaala}
\région{BO}
(\domainesémantique{Navigation})
\classe{v}
\begin{glose}
\pfra{tirer des bords [Corne]}
\end{glose}
\newline
\begin{exemple}
\textbf{\pnua{i tua pwaala wony}}
\pfra{le bateau tire des bords}
\end{exemple}
\newline
\relationsémantique{Cf.}{\lien{ⓔtua}{tua}}
\glosecourte{zigzagger}
\newline
\note{non vérifié}{général}{}
\end{entrée}

\begin{entrée}{tubun}{}{ⓔtubun}
\région{PA BO [Corne]}
(\domainesémantique{Discours, échanges verbaux})
\classe{v}
\begin{glose}
\pfra{grommeler ; grogner}
\end{glose}
\end{entrée}

\begin{entrée}{tuu}{}{ⓔtuu}
\région{GOs}
(\domainesémantique{Habitat})
\classe{v}
\begin{glose}
\pfra{déménager}
\end{glose}
\begin{glose}
\pfra{partir avec toutes ses affaires}
\end{glose}
\end{entrée}

\begin{entrée}{tûû}{}{ⓔtûû}
\région{GOs}
(\domainesémantique{Mouvements ou actions faits avec le corps, les bras, les mains, les pieds})
\classe{v}
\begin{glose}
\pfra{érafler (s') ; égratigner (s')}
\end{glose}
\end{entrée}

\begin{entrée}{tuuçò}{}{ⓔtuuçò}
\formephonétique{tuːʒɔ}
\région{GOs}
\variante{%
tuujong tuuyòng
\région{PA BO [BM]}}
\classe{v ; n}
\newline
\sens{1}
(\domainesémantique{Fonctions naturelles humaines})
\begin{glose}
\pfra{froid ; fièvre}
\end{glose}
\newline
\begin{exemple}
\région{GOs}
\textbf{\pnua{mã tuuçò}}
\pfra{fièvre (lit. affection froid)}
\end{exemple}
\newline
\begin{exemple}
\région{GOs}
\textbf{\pnua{e phee-je xo tuuçò}}
\pfra{il a de la fièvre (lit. le froid l'a pris)}
\end{exemple}
\newline
\begin{exemple}
\région{BO}
\textbf{\pnua{nu tòòli tuujong}}
\pfra{j'ai attrapé froid}
\end{exemple}
\newline
\sens{2}
(\domainesémantique{Température})
\begin{glose}
\pfra{faire froid}
\end{glose}
\begin{glose}
\pfra{refroidir ; refroidi}
\end{glose}
\newline
\begin{exemple}
\région{BO}
\textbf{\pnua{i tuujong}}
\pfra{il fait froid}
\end{exemple}
\newline
\begin{exemple}
\région{BO}
\textbf{\pnua{i tuujong aari}}
\pfra{le riz est froid}
\end{exemple}
\newline
\note{pha-tuuço-ni}{grammaire}{refroidir}
\end{entrée}

\begin{entrée}{tûûne}{}{ⓔtûûne}
\région{GOs PA BO}
(\domainesémantique{Mouvements ou actions faits avec le corps, les bras, les mains, les pieds})
\classe{v}
\begin{glose}
\pfra{essuyer}
\end{glose}
\begin{glose}
\pfra{nettoyer}
\end{glose}
\begin{glose}
\pfra{ratisser}
\end{glose}
\newline
\begin{exemple}
\textbf{\pnua{e tûûni-je}}
\pfra{il s'essuie}
\end{exemple}
\newline
\begin{exemple}
\textbf{\pnua{nu tûûni-nu}}
\pfra{je m'essuie}
\end{exemple}
\end{entrée}

\begin{entrée}{tuuwa}{}{ⓔtuuwa}
\région{BO}
(\domainesémantique{Verbes de mouvement})
\classe{v}
\begin{glose}
\pfra{glisser (se) ; se faufiler[Corne]}
\end{glose}
\newline
\note{non vérifié}{général}{}
\end{entrée}

\newpage

\lettrine{th}\begin{entrée}{tha}{1}{ⓔthaⓗ1}
\région{GOs}
(\domainesémantique{Verbes d'action (en général)})
\classe{v}
\begin{glose}
\pfra{dévier ; rater (une cible, qqch) ; déraper}
\end{glose}
\newline
\begin{exemple}
\textbf{\pnua{e tha na ni de}}
\pfra{il s'est trompé de chemin, il s'est égaré}
\end{exemple}
\newline
\begin{sous-entrée}{pa-tha}{ⓔthaⓗ1ⓝpa-tha}
\begin{glose}
\pfra{rater}
\end{glose}
\end{sous-entrée}
\end{entrée}

\begin{entrée}{tha}{2}{ⓔthaⓗ2}
\région{GOs BO PA}
(\domainesémantique{Actions liées aux plantes
, Préparation des aliments; modes de préparation et de cuisson})
\classe{v}
\begin{glose}
\pfra{écorcer (du coco sur un épieu) ; éplucher (coco)}
\end{glose}
\newline
\begin{sous-entrée}{tha nu}{ⓔthaⓗ2ⓝtha nu}
\begin{glose}
\pfra{éplucher coco}
\end{glose}
\end{sous-entrée}
\end{entrée}

\begin{entrée}{tha}{3}{ⓔthaⓗ3}
\région{GOs}
(\domainesémantique{Poissons})
\classe{nom}
\begin{glose}
\pfra{mulet "queue bleue" de petite taille}
\end{glose}
\nomscientifique{Crenimugil}
\end{entrée}

\begin{entrée}{tha}{4}{ⓔthaⓗ4}
\formephonétique{tʰa}
\région{GOs}
(\domainesémantique{Navigation})
\classe{nom}
\begin{glose}
\pfra{pagaie ; perche}
\end{glose}
\end{entrée}

\begin{entrée}{thaa}{1}{ⓔthaaⓗ1}
\région{GOs}
(\domainesémantique{Aliments, alimentation})
\newline
\sens{1}
\classe{v}
\begin{glose}
\pfra{piquer (nourriture) ; piquant}
\end{glose}
\begin{glose}
\pfra{pimenté}
\end{glose}
\newline
\sens{2}
\classe{nom}
\begin{glose}
\pfra{piment rouge}
\end{glose}
\end{entrée}

\begin{entrée}{thaa}{2}{ⓔthaaⓗ2}
\région{PA BO}
(\domainesémantique{Verbes de déplacement et moyens de déplacement})
\classe{v}
\begin{glose}
\pfra{arriver ; aller}
\end{glose}
\newline
\begin{exemple}
\région{PA}
\textbf{\pnua{thaa-du bwa pô}}
\pfra{arriver sur le pont}
\end{exemple}
\newline
\begin{exemple}
\textbf{\pnua{thaa-da mi}}
\pfra{il monte vers ici}
\end{exemple}
\newline
\begin{sous-entrée}{i thaa-ò tha-mi}{ⓔthaaⓗ2ⓝi thaa-ò tha-mi}
\région{PA}
\begin{glose}
\pfra{il va de ci de là}
\end{glose}
\end{sous-entrée}
\newline
\begin{sous-entrée}{thaa-da/thaa-du}{ⓔthaaⓗ2ⓝthaa-da/thaa-du}
\région{PA BO}
\begin{glose}
\pfra{jusqu'à}
\end{glose}
\end{sous-entrée}
\newline
\begin{sous-entrée}{thaa-e}{ⓔthaaⓗ2ⓝthaa-e}
\région{PA}
\begin{glose}
\pfra{arriver d'une direction transverse}
\end{glose}
\end{sous-entrée}
\end{entrée}

\begin{entrée}{thaa}{3}{ⓔthaaⓗ3}
\région{GOs BO PA}
(\domainesémantique{Cultures, techniques, boutures})
\classe{nom}
\begin{glose}
\pfra{tuteur d'ignames (petit)}
\end{glose}
\newline
\relationsémantique{Cf.}{\lien{ⓔduⓗ1}{du}}
\glosecourte{tuteur d'ignames (grand)}
\end{entrée}

\begin{entrée}{thãã}{}{ⓔthãã}
\région{PA BO}
(\domainesémantique{Oiseaux})
\classe{nom}
\begin{glose}
\pfra{long-cou}
\end{glose}
\end{entrée}

\begin{entrée}{thaa-bi}{}{ⓔthaa-bi}
\région{GOs}
(\domainesémantique{Mouvements ou actions faits avec le corps, les bras, les mains, les pieds})
\classe{v}
\begin{glose}
\pfra{piler ; écraser}
\end{glose}
\newline
\begin{sous-entrée}{ba-thaabi}{ⓔthaa-biⓝba-thaabi}
\begin{glose}
\pfra{pilon}
\end{glose}
\end{sous-entrée}
\end{entrée}

\begin{entrée}{thaabwe}{}{ⓔthaabwe}
\région{GOs BO}
\variante{%
thaaboi, thabui, thabwi
\région{GO(s) WEM BO}}
\classe{v}
\newline
\sens{1}
(\domainesémantique{Types de maison, architecture de la maison})
\begin{glose}
\pfra{couvrir (maison)}
\end{glose}
\newline
\begin{exemple}
\région{BO}
\textbf{\pnua{li u thaabwe mwa}}
\pfra{ils couvrent la maison}
\end{exemple}
\newline
\sens{2}
(\domainesémantique{Mouvements ou actions faits avec le corps, les bras, les mains, les pieds})
\begin{glose}
\pfra{envelopper}
\end{glose}
\end{entrée}

\begin{entrée}{thaa kui}{}{ⓔthaa kui}
\région{GOs PA}
(\domainesémantique{Cultures, techniques, boutures})
\classe{v}
\begin{glose}
\pfra{creuser (pour récolter des ignames) ; récolter les ignames}
\end{glose}
\newline
\begin{sous-entrée}{wara thaa kui}{ⓔthaa kuiⓝwara thaa kui}
\begin{glose}
\pfra{époque de la récolte des ignames}
\end{glose}
\end{sous-entrée}
\end{entrée}

\begin{entrée}{thaam !}{}{ⓔthaam !}
\région{PA}
\variante{%
thãã
\région{BO}}
(\domainesémantique{})
\classe{INTJ ; appel respectueux à une pers.}
\begin{glose}
\pfra{hé dites! (à des personnes)}
\end{glose}
\newline
\begin{exemple}
\textbf{\pnua{thaam !}}
\pfra{dites-moi !, écoutez !}
\end{exemple}
\end{entrée}

\begin{entrée}{thaao}{}{ⓔthaao}
\région{BO}
(\domainesémantique{Description des objets, formes, consistance, taille})
\classe{v.stat.}
\begin{glose}
\pfra{rugueux[Corne]}
\end{glose}
\newline
\note{non vérifié}{général}{}
\end{entrée}

\begin{entrée}{thaavi}{}{ⓔthaavi}
\région{GOs}
\variante{%
thaaviing
\région{BO [BM]}}
(\domainesémantique{Relations et interaction sociales})
\classe{v}
\begin{glose}
\pfra{réunir (se) pour délibérer}
\end{glose}
\end{entrée}

\begin{entrée}{thaavwan}{}{ⓔthaavwan}
\région{PA BO}
\variante{%
thaavan
}
(\domainesémantique{Marées})
\classe{nom}
\begin{glose}
\pfra{marée basse ; le rivage à marée basse}
\end{glose}
\end{entrée}

\begin{entrée}{thaavwo}{}{ⓔthaavwo}
\région{PA}
(\domainesémantique{Verbes d'action (en général)})
\classe{v}
\begin{glose}
\pfra{créer ; faire qqch ; façonner}
\end{glose}
\newline
\note{thaavwoni (v.t.)}{grammaire}{}
\end{entrée}

\begin{entrée}{thaavwu}{}{ⓔthaavwu}
\région{GOs}
\variante{%
thaapu
\région{GO(s)}, 
thaavwun, thaapun, taapun
\région{PA BO}, 
taavwu(n)
\région{BO}, 
teewu
\région{WEM}}
(\domainesémantique{Aspect})
\classe{v}
\begin{glose}
\pfra{commencer ; mettre à (se) ; créer}
\end{glose}
\newline
\begin{exemple}
\région{GO}
\textbf{\pnua{e thaavwu pujo}}
\pfra{elle se met à faire la cuisine}
\end{exemple}
\newline
\begin{exemple}
\région{GO}
\textbf{\pnua{eniza nye çö thaavwu piina-du ebòli bwaabu ?}}
\pfra{quand es-tu allée en France pour la première fois ?}
\end{exemple}
\newline
\begin{exemple}
\région{GO}
\textbf{\pnua{li thaapu u wa}}
\pfra{ils commencent à chanter}
\end{exemple}
\newline
\begin{exemple}
\région{BO}
\textbf{\pnua{nu taawu pwexu}}
\pfra{je commence à parler}
\end{exemple}
\newline
\begin{exemple}
\région{BO}
\textbf{\pnua{nu taawu pwexu}}
\pfra{je commence à parler}
\end{exemple}
\newline
\note{thaavwuni, tavwune (v.t.)}{grammaire}{}
\end{entrée}

\begin{entrée}{thaawe}{}{ⓔthaawe}
\région{PA}
(\domainesémantique{Verbes d'action (en général)})
\classe{v}
\begin{glose}
\pfra{garder précieusement ; conserver}
\end{glose}
\end{entrée}

\begin{entrée}{thaaxô}{}{ⓔthaaxô}
\région{GOs}
\variante{%
thaxõõ
\région{GO(s)}}
(\domainesémantique{Mouvements ou actions faits avec le corps, les bras, les mains, les pieds})
\classe{v}
\begin{glose}
\pfra{empêcher (bagarre) ; arrêter (qqn)}
\end{glose}
\begin{glose}
\pfra{bloquer ; barrer (route) ; empêcher (de se déplacer)}
\end{glose}
\newline
\begin{exemple}
\textbf{\pnua{e thaaxô-nu}}
\pfra{il m'a arrêté}
\end{exemple}
\newline
\begin{exemple}
\textbf{\pnua{e thaaxô-ni loto}}
\pfra{il a arrêté la voiture}
\end{exemple}
\newline
\begin{exemple}
\textbf{\pnua{la thaxõõ-ni de}}
\pfra{ils ont bloqué la route}
\end{exemple}
\newline
\begin{exemple}
\région{GOs}
\textbf{\pnua{la thaxõõ-la vwo/pu kebwa ne la a}}
\pfra{ils les ont empêché de partir}
\end{exemple}
\newline
\begin{exemple}
\région{BO}
\textbf{\pnua{i thaxõ-nu}}
\pfra{il m'a arrêté}
\end{exemple}
\newline
\note{thaxõõ-ni (v.t.)}{grammaire}{}
\end{entrée}

\begin{entrée}{thaaxôni}{}{ⓔthaaxôni}
\formephonétique{tʰaː'ɣõɳi}
\région{GOs}
\variante{%
thaaxõni, thaxõni
\région{BO}, 
tagòni
\région{BO vx}}
(\domainesémantique{Verbes d'action (en général)})
\classe{v}
\begin{glose}
\pfra{arrêter ; garder ; retenir}
\end{glose}
\end{entrée}

\begin{entrée}{thaa-zia}{}{ⓔthaa-zia}
\région{GOs}
\variante{%
tha-tia
\région{BO}}
(\domainesémantique{Mouvements ou actions faits avec le corps, les bras, les mains, les pieds})
\classe{v}
\begin{glose}
\pfra{pousser (qqn)}
\end{glose}
\newline
\relationsémantique{Cf.}{\lien{ⓔtia}{tia}}
\glosecourte{pousser (voiture)}
\end{entrée}

\begin{entrée}{thaba}{}{ⓔthaba}
\région{GOs PA BO}
(\domainesémantique{Portage
, Mouvements ou actions faits avec le corps, les bras, les mains, les pieds})
\classe{v}
\begin{glose}
\pfra{porter qqch de lourd dans les bras}
\end{glose}
\begin{glose}
\pfra{soulever}
\end{glose}
\newline
\relationsémantique{Cf.}{\lien{ⓔköe}{köe}}
\glosecourte{porter dans les bras}
\end{entrée}

\begin{entrée}{thabaò}{}{ⓔthabaò}
\région{PA}
(\domainesémantique{Verbes de mouvement})
\classe{v}
\begin{glose}
\pfra{toucher avec la sagaie ; piquer}
\end{glose}
\end{entrée}

\begin{entrée}{thabil}{}{ⓔthabil}
\région{PA}
(\domainesémantique{Objets coutumiers})
\classe{nom}
\begin{glose}
\pfra{ceinture de femme tressée (monnaie)}
\end{glose}
\newline
\note{ceinture faite à partir des racines de bourao tapées, lavées, séchées (Charles)}{glose}{}
\newline
\relationsémantique{Cf.}{\lien{}{pobil, wepòò [PA]}}
\glosecourte{ceinture de femme (monnaie)}
\end{entrée}

\begin{entrée}{thabila}{}{ⓔthabila}
\région{GOs PA}
(\domainesémantique{Verbes d'action (en général)})
\classe{v}
\begin{glose}
\pfra{protéger ; préserver ; garder}
\end{glose}
\newline
\begin{exemple}
\région{GOs}
\textbf{\pnua{thabila-zooni}}
\pfra{protège-le bien !}
\end{exemple}
\end{entrée}

\begin{entrée}{thabòe}{}{ⓔthabòe}
\région{GOs PA}
(\domainesémantique{Actions liées aux plantes})
\classe{v}
\begin{glose}
\pfra{effeuiller (en pinçant et cassant la tige des feuilles avec le pouce et l'index)}
\end{glose}
\newline
\begin{exemple}
\région{PA}
\textbf{\pnua{i thabòe dòò naõnil}}
\pfra{elle effeuille le chou Kanak}
\end{exemple}
\end{entrée}

\begin{entrée}{tha-bulu-ni}{}{ⓔtha-bulu-ni}
\formephonétique{tʰa-bulu-ɳi}
\région{GOs}
(\domainesémantique{Mouvements ou actions faits avec le corps, les bras, les mains, les pieds})
\classe{v}
\begin{glose}
\pfra{entasser ; rassembler}
\end{glose}
\newline
\begin{exemple}
\textbf{\pnua{ta-bulu-ni ja !}}
\pfra{rassembles les saletés !}
\end{exemple}
\newline
\relationsémantique{Cf.}{\lien{ⓔna-bulu-ni}{na-bulu-ni}}
\glosecourte{rassembler}
\end{entrée}

\begin{entrée}{thabwi}{}{ⓔthabwi}
\région{GOs}
\variante{%
thaabwi
\région{PA BO}}
(\domainesémantique{Soins du corps})
\classe{v}
\begin{glose}
\pfra{nettoyer}
\end{glose}
\begin{glose}
\pfra{soigner}
\end{glose}
\begin{glose}
\pfra{réparer}
\end{glose}
\end{entrée}

\begin{entrée}{tha-carûni}{}{ⓔtha-carûni}
\région{PA}
\variante{%
tha-yarûni
\région{PA}}
(\domainesémantique{Feu : objets et actions liés au feu})
\classe{v}
\begin{glose}
\pfra{pousser le feu}
\end{glose}
\newline
\begin{exemple}
\textbf{\pnua{tha-yarûni yaai !}}
\pfra{pousse le feu !}
\end{exemple}
\newline
\relationsémantique{Cf.}{\lien{}{carû ; carûn}}
\glosecourte{pousser le feu}
\newline
\relationsémantique{Cf.}{\lien{ⓔpigi yaai}{pigi yaai}}
\glosecourte{pousser le feu}
\end{entrée}

\begin{entrée}{thaçe}{}{ⓔthaçe}
\formephonétique{tʰadʒe}
\région{GOs}
(\domainesémantique{Mouvements ou actions faits avec le corps, les bras, les mains, les pieds})
\classe{v}
\begin{glose}
\pfra{cogner ; frapper fort}
\end{glose}
\begin{glose}
\pfra{lancer fort}
\end{glose}
\end{entrée}

\begin{entrée}{thadi}{}{ⓔthadi}
\région{GOs BO}
(\domainesémantique{Verbes d'action (en général)})
\classe{v}
\begin{glose}
\pfra{démolir (toiture dela maison, etc.) ; enlever (la paille du toit)}
\end{glose}
\begin{glose}
\pfra{saccager}
\end{glose}
\newline
\begin{exemple}
\région{BO}
\textbf{\pnua{noli je mwa, ma je i thadi u dèèn}}
\pfra{regarde cette maison, c'est celle que le vent a abîmée}
\end{exemple}
\end{entrée}

\begin{entrée}{tha ê}{}{ⓔtha ê}
\région{GOs}
\variante{%
tho êm
\région{PA}}
(\domainesémantique{Actions liées aux plantes
, Cultures, techniques, boutures})
\classe{v}
\begin{glose}
\pfra{cueillir la canne à sucre}
\end{glose}
\end{entrée}

\begin{entrée}{thae}{}{ⓔthae}
\formephonétique{tʰae}
\région{GOs PA BO}
(\domainesémantique{Mouvements ou actions faits avec le corps, les bras, les mains, les pieds})
\classe{v}
\begin{glose}
\pfra{attacher (qqch avec une corde temporairement)}
\end{glose}
\newline
\begin{exemple}
\région{PA}
\textbf{\pnua{i thae còval}}
\pfra{il attache le cheval}
\end{exemple}
\newline
\begin{exemple}
\région{PA}
\textbf{\pnua{i thae pu-bwaa-n}}
\pfra{elle s'attache les cheveux}
\end{exemple}
\end{entrée}

\begin{entrée}{thaeza}{}{ⓔthaeza}
\région{GOs}
\variante{%
thaila
\région{GO}, 
taela
\région{WE}}
(\domainesémantique{Outils})
\classe{v}
\begin{glose}
\pfra{clouer ; fixer (avec un clou)}
\end{glose}
\end{entrée}

\begin{entrée}{thagi}{}{ⓔthagi}
\région{GOs WEM WE PA BO}
\variante{%
t(h)agi
\région{BO}}
(\domainesémantique{Mouvements ou actions faits avec le corps, les bras, les mains, les pieds})
\classe{v}
\begin{glose}
\pfra{plumer (volaille)}
\end{glose}
\begin{glose}
\pfra{arracher les poils}
\end{glose}
\begin{glose}
\pfra{arracher (feuilles, lianes)}
\end{glose}
\newline
\begin{exemple}
\région{BO}
\textbf{\pnua{i thagi ko}}
\pfra{elle plume la poule}
\end{exemple}
\end{entrée}

\begin{entrée}{thahîn}{}{ⓔthahîn}
\région{PA BO}
\variante{%
thaî
\région{GO(s)}}
(\domainesémantique{Tradition orale})
\classe{nom}
\begin{glose}
\pfra{devinette}
\end{glose}
\newline
\begin{exemple}
\région{BO}
\textbf{\pnua{i thahîn yaa inu}}
\pfra{il me pose une devinette [BM]}
\end{exemple}
\end{entrée}

\begin{entrée}{thai}{}{ⓔthai}
\région{GOs}
(\domainesémantique{Vêtements, parure})
\classe{v}
\begin{glose}
\pfra{enfiler (vêtement)}
\end{glose}
\newline
\begin{exemple}
\textbf{\pnua{e thai kii-je}}
\pfra{il a mis son manou}
\end{exemple}
\end{entrée}

\begin{entrée}{thãi}{}{ⓔthãi}
\région{GOs PA BO}
(\domainesémantique{Poissons})
\classe{nom}
\begin{glose}
\pfra{carpe}
\end{glose}
\nomscientifique{Kuhlia sp.(Kuhliidae)}
\newline
\relationsémantique{Cf.}{\lien{}{zòxu [GOs]}}
\glosecourte{carpe}
\end{entrée}

\begin{entrée}{thai hõbò}{}{ⓔthai hõbò}
\région{GOs}
(\domainesémantique{Vêtements, parure})
\classe{v}
\begin{glose}
\pfra{mettre ; enfiler (vêtement)}
\end{glose}
\newline
\relationsémantique{Cf.}{\lien{}{udale hõbò}}
\glosecourte{mettre; enfiler (vêtement)}
\end{entrée}

\begin{entrée}{thaila}{}{ⓔthaila}
\région{GOs}
\variante{%
thaeza
\région{GO}, 
taela
\région{WE}, 
taabila, tabila
\région{PA}}
(\domainesémantique{Actions avec un instrument, un outil})
\classe{v}
\begin{glose}
\pfra{clouer}
\end{glose}
\end{entrée}

\begin{entrée}{thaivwi}{}{ⓔthaivwi}
\formephonétique{tʰaiβi}
\région{GOs PA}
\variante{%
thaiving
\région{PA BO}}
\classe{v}
\newline
\sens{1}
(\domainesémantique{Mouvements ou actions faits avec le corps, les bras, les mains, les pieds})
\begin{glose}
\pfra{entasser}
\end{glose}
\newline
\begin{exemple}
\textbf{\pnua{thaivwi ja !}}
\pfra{ramasse/entasse les détritus !}
\end{exemple}
\newline
\sens{2}
(\domainesémantique{Relations et interaction sociales})
\begin{glose}
\pfra{réunir ; rassembler (gens)}
\end{glose}
\newline
\begin{sous-entrée}{pe-thaivwi}{ⓔthaivwiⓢ2ⓝpe-thaivwi}
\begin{glose}
\pfra{se réunir}
\end{glose}
\newline
\note{thaivwinge (v.t.)}{grammaire}{}
\end{sous-entrée}
\end{entrée}

\begin{entrée}{thaji}{}{ⓔthaji}
\région{GOs}
(\domainesémantique{Mouvements ou actions faits avec le corps, les bras, les mains, les pieds})
\classe{v}
\begin{glose}
\pfra{frapper fort}
\end{glose}
\end{entrée}

\begin{entrée}{thala}{}{ⓔthala}
\formephonétique{tʰala}
\région{GOs BO PA}
(\domainesémantique{Mouvements ou actions faits avec le corps, les bras, les mains, les pieds})
\classe{v}
\begin{glose}
\pfra{ouvrir (maison, boîte, marmite, etc.) ; déboucher (bouteille) ; enlever (couvercle)}
\end{glose}
\newline
\begin{exemple}
\région{GO}
\textbf{\pnua{e thala mee-je}}
\pfra{il ouvre les yeux}
\end{exemple}
\newline
\begin{exemple}
\région{GO}
\textbf{\pnua{thala phweemwa !}}
\pfra{ouvre la porte !}
\end{exemple}
\newline
\begin{exemple}
\région{PA}
\textbf{\pnua{e thala-da pweemwa}}
\pfra{elle ouvre la porte (pour entrer)}
\end{exemple}
\newline
\begin{exemple}
\région{PA}
\textbf{\pnua{e thala-dupweemwa}}
\pfra{elle ouvre la porte (pour sortir)}
\end{exemple}
\newline
\begin{exemple}
\région{PA}
\textbf{\pnua{thala döö !}}
\pfra{ouvre la marmite !}
\end{exemple}
\newline
\begin{exemple}
\textbf{\pnua{thala we !}}
\pfra{ouvre l'eau !}
\end{exemple}
\newline
\begin{exemple}
\région{GO}
\textbf{\pnua{thala go !}}
\pfra{mets la musique !}
\end{exemple}
\newline
\begin{exemple}
\textbf{\pnua{thala bwat !}}
\pfra{ouvre la boîte}
\end{exemple}
\newline
\begin{exemple}
\textbf{\pnua{thala phwa !}}
\pfra{ouvre la bouche !}
\end{exemple}
\end{entrée}

\begin{entrée}{thalei}{}{ⓔthalei}
\région{WEM PA BO}
(\domainesémantique{Types de maison, architecture de la maison})
\classe{nom}
\begin{glose}
\pfra{chambranles sculptées de porte}
\end{glose}
\newline
\relationsémantique{Cf.}{\lien{ⓔdrògò}{drògò}}
\glosecourte{masque}
\newline
\relationsémantique{Cf.}{\lien{}{throo-mwa [GOs] ;thoo-mwa [PA BO]}}
\glosecourte{flèche faitière}
\end{entrée}

\begin{entrée}{thali}{}{ⓔthali}
\région{PA BO}
\classe{v}
\newline
\sens{1}
(\domainesémantique{Verbes de mouvement})
\begin{glose}
\pfra{buter (sur qqch) ; trébucher}
\end{glose}
\newline
\begin{exemple}
\région{PA}
\textbf{\pnua{i thali kò na ni we-ce}}
\pfra{il s'est pris les pieds dans les racines}
\end{exemple}
\newline
\begin{exemple}
\région{PA}
\textbf{\pnua{i thali kò bwa pa}}
\pfra{il a trébuché sur une pierre}
\end{exemple}
\newline
\sens{2}
(\domainesémantique{Actions liées aux plantes})
\begin{glose}
\pfra{gauler (fruit)}
\end{glose}
\newline
\sens{3}
(\domainesémantique{Mouvements ou actions faits avec le corps, les bras, les mains, les pieds})
\begin{glose}
\pfra{taper}
\end{glose}
\end{entrée}

\begin{entrée}{thaliang}{}{ⓔthaliang}
\région{PA}
\variante{%
thaliwa
\région{WEM}, 
thraliwa
\région{GO(s)}}
(\domainesémantique{Fonctions naturelles humaines})
\classe{v}
\begin{glose}
\pfra{suicider (se)}
\end{glose}
\end{entrée}

\begin{entrée}{thane}{}{ⓔthane}
\région{BO}
(\domainesémantique{Préparation des aliments; modes de préparation et de cuisson})
\classe{v}
\begin{glose}
\pfra{chauffer [BM]}
\end{glose}
\newline
\étymologie{
\langue{POc}
\étymon{*raraŋ}
\glosecourte{heat sthg or o.s by fire}
\auteur{Blust}}
\newline
\note{mot non verifié}{général}{}
\end{entrée}

\begin{entrée}{thãne}{}{ⓔthãne}
\région{PA}
(\domainesémantique{Armes})
\classe{nom}
\begin{glose}
\pfra{pierre de fronde (grosse)}
\end{glose}
\end{entrée}

\begin{entrée}{thani}{}{ⓔthani}
\formephonétique{tʰaɳi}
\région{GOs WEM WE}
(\domainesémantique{Caractéristiques et propriétés des personnes})
\classe{v}
\begin{glose}
\pfra{vif ; dynamique ; en forme}
\end{glose}
\end{entrée}

\begin{entrée}{thano}{}{ⓔthano}
\région{BO}
(\domainesémantique{Noms des plantes})
\classe{nom}
\begin{glose}
\pfra{fougère lianescente [Corne]}
\end{glose}
\nomscientifique{Schizéacée}
\newline
\note{non vérifié}{général}{}
\end{entrée}

\begin{entrée}{tha nu}{}{ⓔtha nu}
\formephonétique{tʰa ɳu}
\région{GOs PA BO}
(\domainesémantique{Actions liées aux plantes
, Préparation des aliments; modes de préparation et de cuisson})
\classe{v}
\begin{glose}
\pfra{écorcer le coco (sur un épieu)}
\end{glose}
\end{entrée}

\begin{entrée}{tha-nyale}{}{ⓔtha-nyale}
\région{GOs}
(\domainesémantique{Mouvements ou actions faits avec le corps, les bras, les mains, les pieds})
\classe{v}
\begin{glose}
\pfra{fendre et couper en morceaux (citrouille)}
\end{glose}
\newline
\relationsémantique{Cf.}{\lien{}{tha-}}
\glosecourte{taper ou piquer avec qqch.}
\end{entrée}

\begin{entrée}{tha nhyôgo we}{}{ⓔtha nhyôgo we}
\région{GOs}
(\domainesémantique{Actions liées aux éléments (liquide, fumée)})
\classe{v}
\begin{glose}
\pfra{sonder la profondeur de l'eau}
\end{glose}
\end{entrée}

\begin{entrée}{thaò}{}{ⓔthaò}
\région{GOs BO}
\variante{%
thawa
\région{GO(s)}}
(\domainesémantique{Mouvements ou actions faits avec le corps, les bras, les mains, les pieds})
\classe{v}
\begin{glose}
\pfra{dénouer (corde) ; défaire (noeud) ; relâcher un peu}
\end{glose}
\newline
\relationsémantique{Cf.}{\lien{ⓔtua}{tua}}
\glosecourte{nouer}
\end{entrée}

\begin{entrée}{tha-pwe}{}{ⓔtha-pwe}
\région{GOs PA BO}
(\domainesémantique{Pêche})
\classe{v}
\begin{glose}
\pfra{pêcher à la ligne}
\end{glose}
\end{entrée}

\begin{entrée}{thathibul}{}{ⓔthathibul}
\région{PA}
(\domainesémantique{Jeux divers})
\classe{v}
\begin{glose}
\pfra{jeu}
\end{glose}
\newline
\note{on provoque une explosion en emprisonnant de l'air entre les mains et en les croisant sous l'eau l'une sous l'autre}{glose}{}
\end{entrée}

\begin{entrée}{thatra-hi}{}{ⓔthatra-hi}
\formephonétique{tʰaʈa-hi}
\région{GOs}
\classe{v ; n}
\newline
\sens{1}
(\domainesémantique{Configuration des objets})
\begin{glose}
\pfra{bouquet de paille à la main}
\end{glose}
\newline
\sens{2}
(\domainesémantique{Mouvements ou actions faits avec le corps, les bras, les mains, les pieds})
\begin{glose}
\pfra{brandir dans la main}
\end{glose}
\end{entrée}

\begin{entrée}{tha-truãrôô}{}{ⓔtha-truãrôô}
\formephonétique{'tʰa'ʈuɛ̃ɽõː ; tʰaʈʰuɛ̃ɽõː}
\région{GOs PA}
\variante{%
thruatrôô
\région{vx (Haudricourt)}, 
ta-tuarô
\région{BO [BM]}}
(\domainesémantique{Insectes})
\classe{v ; n}
\begin{glose}
\pfra{toile d'araignée ; faire une toile (araignée)}
\end{glose}
\newline
\note{sans doute liée au jeu de ficelle : tha-thrûã}{général}{}
\end{entrée}

\begin{entrée}{tha-thrûã}{}{ⓔtha-thrûã}
\région{GOs BO}
\variante{%
tha-thûã, ta-thûã
\région{BO}}
(\domainesémantique{Jeux divers})
\classe{v ; n}
\begin{glose}
\pfra{jeu de ficelle}
\end{glose}
\end{entrée}

\begin{entrée}{thau}{}{ⓔthau}
\région{GOs}
\variante{%
thaul
\région{PA BO}}
(\domainesémantique{Armes})
\classe{nom}
\begin{glose}
\pfra{pierre de fronde allongée et polie aux deux bouts}
\end{glose}
\newline
\begin{exemple}
\région{BO PA}
\textbf{\pnua{pa-thaul}}
\pfra{pierre de fronde}
\end{exemple}
\end{entrée}

\begin{entrée}{tha-uji}{}{ⓔtha-uji}
\région{BO}
\variante{%
taauji
\région{BO}}
(\domainesémantique{Mouvements ou actions faits avec le corps, les bras, les mains, les pieds})
\classe{v}
\begin{glose}
\pfra{pousser et faire tomber à la renverse [BM]}
\end{glose}
\end{entrée}

\begin{entrée}{thawa-da}{}{ⓔthawa-da}
\région{BO}
\variante{%
thaa-da
}
(\domainesémantique{Temps})
\classe{CNJ}
\begin{glose}
\pfra{jusqu'à ce que}
\end{glose}
\newline
\begin{exemple}
\région{BO}
\textbf{\pnua{nu waayu thaa-da i phããde je wèdali-n}}
\pfra{j'ai insisté jusqu'à ce qu'il montre sa fronde}
\end{exemple}
\end{entrée}

\begin{entrée}{thaxe}{}{ⓔthaxe}
\région{GOs BO}
\classe{nom}
\newline
\sens{1}
(\domainesémantique{Feu : objets et actions liés au feu})
\begin{glose}
\pfra{bois qui encadre le foyer}
\end{glose}
\newline
\begin{exemple}
\textbf{\pnua{thaxe-ã}}
\pfra{notre foyer}
\end{exemple}
\newline
\sens{2}
(\domainesémantique{Types de maison, architecture de la maison})
\begin{glose}
\pfra{pierre de seuil [BO]}
\end{glose}
\end{entrée}

\begin{entrée}{thaxebi}{}{ⓔthaxebi}
\formephonétique{'tʰaːɣebi}
\région{GOs BO}
(\domainesémantique{Relations et interaction sociales})
\classe{v}
\begin{glose}
\pfra{accuser ; calomnier ; diffamer}
\end{glose}
\newline
\begin{exemple}
\textbf{\pnua{e thaxebi-ni èmwê}}
\pfra{il a accusé l'homme}
\end{exemple}
\end{entrée}

\begin{entrée}{thaxee-phweemwa}{}{ⓔthaxee-phweemwa}
\région{GOs BO}
\variante{%
taage, thaaxe
\région{BO}}
\classe{nom}
\newline
\sens{1}
(\domainesémantique{Types de maison, architecture de la maison})
\begin{glose}
\pfra{seuil de la porte}
\end{glose}
\begin{glose}
\pfra{pierre de seuil}
\end{glose}
\newline
\sens{2}
(\domainesémantique{Feu : objets et actions liés au feu})
\begin{glose}
\pfra{pierre autour du foyer [BO, [BM]]}
\end{glose}
\end{entrée}

\begin{entrée}{thaxiba}{}{ⓔthaxiba}
\région{GOs PA}
\variante{%
thakiba
\région{GO(s)}, 
taxiba
\région{BO PA}}
\newline
\sens{1}
(\domainesémantique{Relations et interaction sociales})
\classe{v}
\begin{glose}
\pfra{refuser ; rejeter ; chasser [PA] ; congédier}
\end{glose}
\begin{glose}
\pfra{abandonner ; délaisser}
\end{glose}
\newline
\sens{2}
(\domainesémantique{Relations et interaction sociales})
\classe{v}
\begin{glose}
\pfra{moquer (se) ; mépriser}
\end{glose}
\newline
\sens{3}
(\domainesémantique{Coutumes, dons coutumiers})
\classe{v}
\begin{glose}
\pfra{répudier (femme)}
\end{glose}
\newline
\relationsémantique{Cf.}{\lien{ⓔparee}{paree}}
\glosecourte{délaisser}
\end{entrée}

\begin{entrée}{thaxim}{}{ⓔthaxim}
\région{PA BO [BM]}
(\domainesémantique{Verbes de mouvement})
\classe{v}
\begin{glose}
\pfra{rouler (se) par terre ;}
\end{glose}
\begin{glose}
\pfra{frotter (se) par terre ;}
\end{glose}
\begin{glose}
\pfra{vautrer (se)}
\end{glose}
\end{entrée}

\begin{entrée}{thazabi}{}{ⓔthazabi}
\région{GOs}
\variante{%
tharabil
\région{PA}}
\classe{v}
\newline
\sens{1}
(\domainesémantique{Verbes d'action faite par des animaux})
\begin{glose}
\pfra{frétiller (poisson) ;}
\end{glose}
\begin{glose}
\pfra{battre des ailes}
\end{glose}
\newline
\sens{2}
(\domainesémantique{Verbes de mouvement})
\begin{glose}
\pfra{rouler (se) par terre (animal, enfant)}
\end{glose}
\end{entrée}

\begin{entrée}{the}{}{ⓔthe}
\région{GOs}
\variante{%
tioo
\région{PA}}
(\domainesémantique{Mouvements ou actions faits avec le corps, les bras, les mains, les pieds})
\classe{v}
\begin{glose}
\pfra{déchirer en long}
\end{glose}
\end{entrée}

\begin{entrée}{thebe}{}{ⓔthebe}
\région{GOs}
\variante{%
tebe
\région{BO [Corne]}}
(\domainesémantique{Actions liées aux plantes})
\classe{v}
\begin{glose}
\pfra{tailler (arbre, plante)}
\end{glose}
\begin{glose}
\pfra{épointer}
\end{glose}
\begin{glose}
\pfra{éplucher (en taillant avec un geste vers l'extérieur)}
\end{glose}
\end{entrée}

\begin{entrée}{theepwa}{}{ⓔtheepwa}
\région{GOs}
(\domainesémantique{Verbes de déplacement et moyens de déplacement})
\classe{v}
\begin{glose}
\pfra{galoper}
\end{glose}
\end{entrée}

\begin{entrée}{thèl}{}{ⓔthèl}
\région{PA BO}
(\domainesémantique{Cultures, techniques, boutures})
\classe{v}
\begin{glose}
\pfra{débroussailler (champ, chemin au sabre d'abatis)}
\end{glose}
\begin{glose}
\pfra{défricher}
\end{glose}
\newline
\begin{sous-entrée}{wara-a ; wara-ò thèl}{ⓔthèlⓝwara-a ; wara-ò thèl}
\begin{glose}
\pfra{époque du débroussaillage (mai)}
\end{glose}
\newline
\note{thèli (v.t.)}{grammaire}{}
\end{sous-entrée}
\end{entrée}

\begin{entrée}{thele}{}{ⓔthele}
\région{GOs}
(\domainesémantique{Chasse
, Pêche
, Guerre})
\classe{v}
\begin{glose}
\pfra{planter (sagaie) ; frapper}
\end{glose}
\end{entrée}

\begin{entrée}{thele paa}{}{ⓔthele paa}
\région{GOs}
(\domainesémantique{Verbes de mouvement})
\classe{v}
\begin{glose}
\pfra{ricocher (faire) un caillou}
\end{glose}
\end{entrée}

\begin{entrée}{thèmi}{}{ⓔthèmi}
\région{BO}
(\domainesémantique{Mouvements ou actions avec la tête, les yeux, la bouche})
\classe{v}
\begin{glose}
\pfra{lécher [BM]}
\end{glose}
\newline
\relationsémantique{Cf.}{\lien{}{maalemi [BO]}}
\glosecourte{lécher}
\newline
\relationsémantique{Cf.}{\lien{}{thami (caac)}}
\glosecourte{lécher}
\newline
\étymologie{
\langue{POc}
\étymon{*samu(k)}}
\newline
\note{mot non vérifié}{général}{}
\end{entrée}

\begin{entrée}{theul}{}{ⓔtheul}
\région{PA BO}
(\domainesémantique{Poissons})
\classe{nom}
\begin{glose}
\pfra{mulet de rivière (juvénile, il remonte le cours des rivières, puis devient "naxo" à l'âge adulte)}
\end{glose}
\newline
\étymologie{
\langue{POc}
\étymon{*(s,d)aud}}
\end{entrée}

\begin{entrée}{thi}{1}{ⓔthiⓗ1}
\formephonétique{tʰi}
\région{GOs BO PA}
(\domainesémantique{Corps humain})
\classe{nom}
\begin{glose}
\pfra{sein}
\end{glose}
\begin{glose}
\pfra{mamelle}
\end{glose}
\newline
\begin{sous-entrée}{we-thi}{ⓔthiⓗ1ⓝwe-thi}
\begin{glose}
\pfra{lait maternel}
\end{glose}
\end{sous-entrée}
\newline
\begin{sous-entrée}{me-thi}{ⓔthiⓗ1ⓝme-thi}
\begin{glose}
\pfra{mamelon}
\end{glose}
\newline
\begin{exemple}
\région{PA}
\textbf{\pnua{thi-n}}
\pfra{son sein}
\end{exemple}
\newline
\begin{exemple}
\région{BO}
\textbf{\pnua{pwò-thi-n}}
\pfra{son mamelon}
\end{exemple}
\end{sous-entrée}
\newline
\étymologie{
\langue{POc}
\étymon{*susu}
\glosecourte{sein}}
\end{entrée}

\begin{entrée}{thi}{2}{ⓔthiⓗ2}
\région{GOs}
(\domainesémantique{Santé, maladie})
\classe{nom}
\begin{glose}
\pfra{bouton ; acné}
\end{glose}
\end{entrée}

\begin{entrée}{thi}{3}{ⓔthiⓗ3}
\région{GOs PA BO}
\classe{v ; n}
\newline
\sens{1}
(\domainesémantique{Verbes d'action (en général)})
\begin{glose}
\pfra{exploser ; éclater}
\end{glose}
\newline
\begin{sous-entrée}{thi-pholo}{ⓔthiⓗ3ⓢ1ⓝthi-pholo}
\begin{glose}
\pfra{taper dans l'eau}
\end{glose}
\newline
\begin{exemple}
\région{GOs}
\textbf{\pnua{e thi niô}}
\pfra{le tonnerre tonne}
\end{exemple}
\newline
\relationsémantique{Cf.}{\lien{}{hû niô [GOs]}}
\glosecourte{le tonnerre gronde}
\newline
\relationsémantique{Cf.}{\lien{}{i hûn ne nhyô [PA]}}
\glosecourte{le tonnerre gronde}
\end{sous-entrée}
\newline
\sens{2}
(\domainesémantique{Sons, bruits})
\begin{glose}
\pfra{détonation ; coup (de fusil)}
\end{glose}
\newline
\begin{sous-entrée}{thixa jige}{ⓔthiⓗ3ⓢ2ⓝthixa jige}
\région{GOs}
\begin{glose}
\pfra{un coup de fusil}
\end{glose}
\newline
\begin{exemple}
\région{GOs}
\textbf{\pnua{nu trõne thi ne phwa-xe, phwa-tru}}
\pfra{j'ai entendu un/deux coups de fusil}
\end{exemple}
\newline
\begin{exemple}
\région{GOs}
\textbf{\pnua{nu trõne thi xa phwa-tru}}
\pfra{j'ai entendu un/deux coups de fusil}
\end{exemple}
\newline
\begin{exemple}
\région{GOs}
\textbf{\pnua{nu trõne thi xa õ-xe}}
\pfra{j'ai entendu un coup de fusil (lit. une fois)}
\end{exemple}
\newline
\begin{exemple}
\région{GOs}
\textbf{\pnua{nu trõne thixa jige xa phwa-kò}}
\pfra{j'ai entendu 3 coups de fusil}
\end{exemple}
\newline
\begin{exemple}
\région{PA}
\textbf{\pnua{i thi je phwa-xe}}
\pfra{il y a eu une détonation}
\end{exemple}
\newline
\begin{exemple}
\région{BO}
\textbf{\pnua{i thi jigèl}}
\pfra{il y a eu un coup de fusil}
\end{exemple}
\end{sous-entrée}
\end{entrée}

\begin{entrée}{thibe}{}{ⓔthibe}
\formephonétique{tʰibe}
\région{GOs PA BO}
\variante{%
thebe
\région{GO(s) BO}}
(\domainesémantique{Préparation des aliments; modes de préparation et de cuisson})
\classe{v}
\begin{glose}
\pfra{peler (avec un couteau, fruits, igname ou taro cru) ; éplucher (légumes)}
\end{glose}
\begin{glose}
\pfra{gratter avec un couteau}
\end{glose}
\newline
\begin{exemple}
\région{BO}
\textbf{\pnua{nu thebe kumwala o hèlè-m}}
\pfra{j'ai pelé la patate douce avec ton couteau}
\end{exemple}
\newline
\relationsémantique{Cf.}{\lien{}{pwayi [PA], pwaji [GO]}}
\glosecourte{éplucher, peler (avec les doigts, banane, manioc, tubercule cuit)}
\newline
\note{thibi (v.t.)}{grammaire}{}
\end{entrée}

\begin{entrée}{thibi}{}{ⓔthibi}
\région{GOs}
(\domainesémantique{Couture})
\classe{v}
\begin{glose}
\pfra{coudre}
\end{glose}
\end{entrée}

\begin{entrée}{thi-bö}{}{ⓔthi-bö}
\région{GOs}
\variante{%
bwo
\région{BO}}
(\domainesémantique{Feu : objets et actions liés au feu})
\classe{v.t.}
\begin{glose}
\pfra{éteindre (petit feu, lumière)}
\end{glose}
\newline
\begin{exemple}
\textbf{\pnua{thi-böö-ni yaai}}
\pfra{éteindre le feu}
\end{exemple}
\newline
\note{thi-böö-ni (v.t.)}{grammaire}{}
\end{entrée}

\begin{entrée}{thi-bööni}{}{ⓔthi-bööni}
\formephonétique{tʰi-'bɷːɳi}
\région{GOs}
\région{BO}
\variante{%
bwo, bo
}
(\domainesémantique{Feu : objets et actions liés au feu})
\classe{v}
\begin{glose}
\pfra{éteint (être) ; éteindre (un feu)}
\end{glose}
\newline
\begin{sous-entrée}{thi-bööni yai}{ⓔthi-bööniⓝthi-bööni yai}
\région{GO}
\begin{glose}
\pfra{éteindre le feu en frappant}
\end{glose}
\newline
\begin{exemple}
\région{BO}
\textbf{\pnua{a bo nuua-n}}
\pfra{il éteint sa torche}
\end{exemple}
\newline
\note{bwooni [BO] (v.t.)}{grammaire}{éteindre qqch.}
\end{sous-entrée}
\end{entrée}

\begin{entrée}{thibuyul}{}{ⓔthibuyul}
\région{PA}
(\domainesémantique{Verbes de mouvement})
\classe{v}
\begin{glose}
\pfra{rebondir et revenir en sens inverse}
\end{glose}
\end{entrée}

\begin{entrée}{thi-bwagil}{}{ⓔthi-bwagil}
\région{PA BO}
(\domainesémantique{Préfixes et verbes de position
, Verbes de mouvement})
\classe{v}
\begin{glose}
\pfra{agenouiller (s')}
\end{glose}
\begin{glose}
\pfra{genoux (être à)}
\end{glose}
\end{entrée}

\begin{entrée}{thiçe}{}{ⓔthiçe}
\formephonétique{tʰiʒe}
\région{GOs}
\classe{v}
\newline
\sens{1}
(\domainesémantique{Feu : objets et actions liés au feu})
\begin{glose}
\pfra{remuer les braises}
\end{glose}
\newline
\sens{2}
(\domainesémantique{Préparation des aliments; modes de préparation et de cuisson})
\begin{glose}
\pfra{cuire sous la cendre (en remuant la nourriture) ; mettre à cuire sous les braises}
\end{glose}
\newline
\begin{exemple}
\textbf{\pnua{e thiçe pò-wale}}
\pfra{il grille un épi de maïs sous la cendre}
\end{exemple}
\end{entrée}

\begin{entrée}{thiçoo}{}{ⓔthiçoo}
(\domainesémantique{Verbes de mouvement})
\classe{v}
\begin{glose}
\pfra{rebondir}
\end{glose}
\newline
\relationsémantique{Cf.}{\lien{ⓔcöö}{cöö}}
\glosecourte{sauter}
\end{entrée}

\begin{entrée}{thidin}{}{ⓔthidin}
\région{PA BO WEM}
(\domainesémantique{Sentiments})
\classe{v.stat.}
\begin{glose}
\pfra{coléreux ; en colère ; irrité}
\end{glose}
\newline
\begin{exemple}
\région{PA}
\textbf{\pnua{i thidin i nu}}
\pfra{il est en colère contre moi}
\end{exemple}
\newline
\begin{exemple}
\région{PA}
\textbf{\pnua{i thidin puni nu}}
\pfra{il est en colère contre à cause de moi}
\end{exemple}
\end{entrée}

\begin{entrée}{thi-du}{}{ⓔthi-du}
\région{GOs BO[Corne]}
(\domainesémantique{Mouvements ou actions faits avec le corps, les bras, les mains, les pieds})
\classe{v}
\begin{glose}
\pfra{plonger le bras (dans une cavité, dans l'obscurité)}
\end{glose}
\end{entrée}

\begin{entrée}{thige}{}{ⓔthige}
\région{PA WE}
\variante{%
thege
\région{BO [BM]}}
(\domainesémantique{Couture})
\classe{v}
\begin{glose}
\pfra{coudre}
\end{glose}
\newline
\begin{sous-entrée}{ba-thige}{ⓔthigeⓝba-thige}
\région{PA}
\begin{glose}
\pfra{fil à coudre}
\end{glose}
\end{sous-entrée}
\newline
\begin{sous-entrée}{ba-thege}{ⓔthigeⓝba-thege}
\région{BO}
\begin{glose}
\pfra{fil à coudre}
\end{glose}
\end{sous-entrée}
\newline
\étymologie{
\langue{POc}
\étymon{*saqit}
\glosecourte{sew}
\auteur{Blust}}
\end{entrée}

\begin{entrée}{thi-gu}{}{ⓔthi-gu}
\région{GOs BO}
\variante{%
thi-gu(a)
\région{PA}}
(\domainesémantique{Pêche})
\classe{v}
\begin{glose}
\pfra{enfiler (sur une filoche)}
\end{glose}
\newline
\begin{exemple}
\région{GOs}
\textbf{\pnua{e thi gua nò}}
\pfra{il met les poissons sur la filoche}
\end{exemple}
\newline
\begin{exemple}
\région{GOs}
\textbf{\pnua{e thi-gua-ni line nò}}
\pfra{il a enfilé les 2 poissons sur la filoche}
\end{exemple}
\newline
\begin{exemple}
\région{BO}
\textbf{\pnua{i thi gua nò}}
\pfra{il enfile les poissons sur la filoche}
\end{exemple}
\newline
\begin{exemple}
\textbf{\pnua{gua nò}}
\pfra{filoche de poissons}
\end{exemple}
\newline
\note{forme déterminée de "gu" > gua}{grammaire}{}
\newline
\note{thi-gua-ni (v.t.)}{grammaire}{}
\end{entrée}

\begin{entrée}{thii}{1}{ⓔthiiⓗ1}
\région{GOs PA BO WEM}
\classe{v}
\newline
\sens{1}
(\domainesémantique{Mouvements ou actions faits avec le corps, les bras, les mains, les pieds})
\begin{glose}
\pfra{fouiller (dans un trou avec un bâton)}
\end{glose}
\begin{glose}
\pfra{piquer ; faire une piqûre}
\end{glose}
\begin{glose}
\pfra{extraire (épine)}
\end{glose}
\newline
\begin{sous-entrée}{thi kili}{ⓔthiiⓗ1ⓢ1ⓝthi kili}
\région{PA}
\begin{glose}
\pfra{piquer la navette}
\end{glose}
\end{sous-entrée}
\newline
\begin{sous-entrée}{thii du}{ⓔthiiⓗ1ⓢ1ⓝthii du}
\begin{glose}
\pfra{piquer vers le bas}
\end{glose}
\end{sous-entrée}
\newline
\begin{sous-entrée}{thii phwe-pwaaji}{ⓔthiiⓗ1ⓢ1ⓝthii phwe-pwaaji}
\begin{glose}
\pfra{piquer/fouiller dans un trou (à la recherche de crabe)}
\end{glose}
\newline
\begin{exemple}
\région{PA}
\textbf{\pnua{i ra u phe deang, thii ne pwe-keel ra a.}}
\pfra{elle prend l'épuisette, la met en travers du panier et part}
\end{exemple}
\end{sous-entrée}
\newline
\sens{2}
(\domainesémantique{Mouvements ou actions faits avec le corps, les bras, les mains, les pieds})
\begin{glose}
\pfra{provoquer}
\end{glose}
\newline
\begin{exemple}
\région{GOs}
\textbf{\pnua{la pe-thii thô nai la}}
\pfra{ils se provoquent (piquent la colère)}
\end{exemple}
\newline
\relationsémantique{Cf.}{\lien{}{thaa, thòzò}}
\glosecourte{piquer}
\newline
\étymologie{
\langue{POc}
\étymon{*susuRi}
\glosecourte{coudre}
\auteur{Blust}}
\end{entrée}

\begin{entrée}{thii}{2}{ⓔthiiⓗ2}
\région{GOs BO}
(\domainesémantique{Soins du corps})
\classe{v}
\begin{glose}
\pfra{peigner ; peigner (se)}
\end{glose}
\newline
\begin{exemple}
\région{GOs}
\textbf{\pnua{nu thii-vwo}}
\pfra{je me peigne}
\end{exemple}
\newline
\begin{exemple}
\région{GOs}
\textbf{\pnua{nu thii pu-bwaa-nu}}
\pfra{je peigne mes cheveux}
\end{exemple}
\newline
\begin{exemple}
\région{GOs}
\textbf{\pnua{e thii-vwo ẽnõ ã}}
\pfra{l'enfant se peigne}
\end{exemple}
\newline
\begin{exemple}
\région{GOs}
\textbf{\pnua{e thuvwu thii pu-bwaa-je}}
\pfra{il se peigne}
\end{exemple}
\newline
\begin{exemple}
\région{GOs}
\textbf{\pnua{e thuvwu thii pu-bwaa ẽnõ ã}}
\pfra{l'enfant se peigne}
\end{exemple}
\newline
\begin{exemple}
\région{GOs}
\textbf{\pnua{e thii pu-bwaa-je xo ẽnõ ã}}
\pfra{l'enfant le peigne}
\end{exemple}
\newline
\begin{exemple}
\région{GOs}
\textbf{\pnua{e thii pu-bwaa ẽnõ ã}}
\pfra{il peigne les cheveux de l'enfant}
\end{exemple}
\newline
\étymologie{
\langue{POc}
\étymon{*saRu, *s(ae)du}
\glosecourte{comb}
\auteur{Blust}}
\newline
\note{agramm. : *nu pe-thi-vwo}{grammaire}{}
\end{entrée}

\begin{entrée}{thii}{3}{ⓔthiiⓗ3}
\région{BO [BM]}
\variante{%
thiin
\région{BO}}
(\domainesémantique{Feu : objets et actions liés au feu})
\classe{v}
\begin{glose}
\pfra{incendier ; mettre le feu}
\end{glose}
\newline
\begin{exemple}
\région{BO}
\textbf{\pnua{i thii nò-tòn}}
\pfra{il a mis le feu aux broussailles}
\end{exemple}
\end{entrée}

\begin{entrée}{thii}{4}{ⓔthiiⓗ4}
\formephonétique{θiː}
\région{GO PA WEM}
(\domainesémantique{Corps animal})
\classe{nom}
\begin{glose}
\pfra{touffe de poils de roussette et fil de coton (Dubois ms)}
\end{glose}
\newline
\note{tout ce qu'on utilise pour faire la monnaie (coquillage, poils de roussette, etc.)}{glose}{}
\end{entrée}

\begin{entrée}{thiibu}{}{ⓔthiibu}
\région{BO}
(\domainesémantique{Préfixes et verbes de position})
\classe{v}
\begin{glose}
\pfra{appuyer (s') [Corne]}
\end{glose}
\end{entrée}

\begin{entrée}{thii kura}{}{ⓔthii kura}
\région{GOs PA}
(\domainesémantique{Relations et interaction sociales})
\classe{v}
\begin{glose}
\pfra{énerver ; agacer (lit. piquer le sang)}
\end{glose}
\end{entrée}

\begin{entrée}{thii-no}{}{ⓔthii-no}
\région{BO}
(\domainesémantique{Corps humain})
\classe{nom}
\begin{glose}
\pfra{fesses [Dubois]}
\end{glose}
\newline
\note{non vérifié}{général}{}
\end{entrée}

\begin{entrée}{thiinyûû}{}{ⓔthiinyûû}
\région{GOs}
\variante{%
thiyu
\région{BO}}
(\domainesémantique{Poissons})
\classe{nom}
\begin{glose}
\pfra{marsouin [Corne]}
\end{glose}
\end{entrée}

\begin{entrée}{thii-puu}{}{ⓔthii-puu}
\région{GOs PA}
(\domainesémantique{Cultures, techniques, boutures})
\classe{v}
\begin{glose}
\pfra{planter des palmes de cocotier ou des branches d'autres arbres dans le sol}
\end{glose}
\newline
\note{(ces branchages sont plantés dans le sol pour faire des barrières, un abri de champ ou un abri temporaire)}{glose}{}
\newline
\begin{exemple}
\textbf{\pnua{i thii-puu mwa}}
\pfra{il pique des palmes ou branches comme protection}
\end{exemple}
\end{entrée}

\begin{entrée}{thiipuun}{}{ⓔthiipuun}
\région{PA}
(\domainesémantique{Relations et interaction sociales})
\classe{v}
\begin{glose}
\pfra{piquer ; provoquer}
\end{glose}
\end{entrée}

\begin{entrée}{thiipuun ka thae pwaxilo, kuu mwa xa doon ku ijö}{}{ⓔthiipuun ka thae pwaxilo, kuu mwa xa doon ku ijö}
\région{PA}
\variante{%
thiiphuu thahulò
\région{BO}}
(\domainesémantique{Tradition orale})
\classe{LOCUT}
\begin{glose}
\pfra{formule de fin de conte}
\end{glose}
\newline
\begin{exemple}
\textbf{\pnua{thiiphuu thahulò}}
\pfra{à ton tour d'en ajouter un autre (conte)}
\end{exemple}
\end{entrée}

\begin{entrée}{thii-vwo}{}{ⓔthii-vwo}
\région{GOs}
(\domainesémantique{Soins du corps})
\classe{v}
\begin{glose}
\pfra{coiffer (se)}
\end{glose}
\begin{glose}
\pfra{peigner (se)}
\end{glose}
\end{entrée}

\begin{entrée}{thi-kham}{}{ⓔthi-kham}
\région{PA}
(\domainesémantique{Verbes de mouvement})
\classe{v}
\begin{glose}
\pfra{ricocher}
\end{glose}
\end{entrée}

\begin{entrée}{thila}{}{ⓔthila}
\région{GOs PA BO [BM, Corne]}
\variante{%
thira
\région{BO [Corne]}}
\classe{nom}
\newline
\sens{1}
(\domainesémantique{Armes})
\begin{glose}
\pfra{plumet de fronde}
\end{glose}
\begin{glose}
\pfra{mèche (du fouet)}
\end{glose}
\newline
\begin{sous-entrée}{thila weda}{ⓔthilaⓢ1ⓝthila weda}
\région{PA}
\begin{glose}
\pfra{plumet de fronde}
\end{glose}
\end{sous-entrée}
\newline
\begin{sous-entrée}{thila wedal}{ⓔthilaⓢ1ⓝthila wedal}
\région{BO}
\begin{glose}
\pfra{plumet de fronde}
\end{glose}
\end{sous-entrée}
\newline
\begin{sous-entrée}{thila phue}{ⓔthilaⓢ1ⓝthila phue}
\begin{glose}
\pfra{le bout/mèche du fouet}
\end{glose}
\newline
\relationsémantique{Cf.}{\lien{ⓔcetil}{cetil}}
\glosecourte{plumet de fronde}
\end{sous-entrée}
\newline
\sens{2}
(\domainesémantique{Danses})
\begin{glose}
\pfra{bouquet de paille pour la danse}
\end{glose}
\end{entrée}

\begin{entrée}{thile}{}{ⓔthile}
\région{PA}
(\domainesémantique{Chasse})
\classe{v}
\begin{glose}
\pfra{toucher (avec une balle de fusil)}
\end{glose}
\end{entrée}

\begin{entrée}{thilò}{}{ⓔthilò}
\région{GOs}
(\domainesémantique{Quantificateurs})
\classe{nom}
\begin{glose}
\pfra{paire ; l'autre d'une paire}
\end{glose}
\newline
\begin{sous-entrée}{thilò-nu}{ⓔthilòⓝthilò-nu}
\région{GOs}
\begin{glose}
\pfra{mon binôme, la personne qui fait équipe avec moi}
\end{glose}
\newline
\begin{exemple}
\région{GOs}
\textbf{\pnua{pe-thilò-bi ma Mario}}
\pfra{je fais équipe avec M.}
\end{exemple}
\newline
\begin{exemple}
\région{GOs}
\textbf{\pnua{pe-thilò-li}}
\pfra{ils vont ensemble}
\end{exemple}
\newline
\begin{exemple}
\région{GOs}
\textbf{\pnua{pe-thilò ala-xò}}
\pfra{la paire de chaussures, les deux chaussures}
\end{exemple}
\newline
\begin{exemple}
\région{GOs}
\textbf{\pnua{a khila thilò ala-kòò-çö}}
\pfra{va chercher ton autre chaussure}
\end{exemple}
\newline
\begin{exemple}
\région{GOs}
\textbf{\pnua{ia thilò ala-xòò-çö ?}}
\pfra{où est ton autre chaussure ?}
\end{exemple}
\newline
\relationsémantique{Cf.}{\lien{}{thixèè ala-xò [GOs]}}
\glosecourte{une seule chaussure}
\end{sous-entrée}
\end{entrée}

\begin{entrée}{thi-ma-thi}{}{ⓔthi-ma-thi}
\région{GOs}
(\domainesémantique{Sons, bruits})
\classe{v}
\begin{glose}
\pfra{pétarader}
\end{glose}
\newline
\begin{exemple}
\textbf{\pnua{e thi-ma-thi loto}}
\pfra{la voiture pétarade}
\end{exemple}
\end{entrée}

\begin{entrée}{thîni}{}{ⓔthîni}
\formephonétique{tʰiɳi}
\région{GOs PA BO}
(\domainesémantique{Habitat})
\classe{nom}
\begin{glose}
\pfra{barrière ; clôture}
\end{glose}
\begin{glose}
\pfra{enclos}
\end{glose}
\newline
\begin{exemple}
\région{GO}
\textbf{\pnua{thîniva-nu, thînia-nu}}
\pfra{ma barrière}
\end{exemple}
\newline
\begin{sous-entrée}{thu thîni}{ⓔthîniⓝthu thîni}
\région{GO}
\begin{glose}
\pfra{faire une barrière}
\end{glose}
\end{sous-entrée}
\newline
\begin{sous-entrée}{phwe-thîni}{ⓔthîniⓝphwe-thîni}
\région{GO PA BO}
\begin{glose}
\pfra{porte de clôture, barrière}
\end{glose}
\end{sous-entrée}
\newline
\begin{sous-entrée}{phwe-zini}{ⓔthîniⓝphwe-zini}
\région{GO}
\begin{glose}
\pfra{porte de clôture, barrière}
\end{glose}
\end{sous-entrée}
\newline
\begin{sous-entrée}{cee-thîni}{ⓔthîniⓝcee-thîni}
\begin{glose}
\pfra{poteaux de barrière}
\end{glose}
\end{sous-entrée}
\newline
\begin{sous-entrée}{thîni-ce}{ⓔthîniⓝthîni-ce}
\région{GO}
\begin{glose}
\pfra{une barrière en bois}
\end{glose}
\end{sous-entrée}
\newline
\begin{sous-entrée}{thîni-pòl}{ⓔthîniⓝthîni-pòl}
\région{BO}
\begin{glose}
\pfra{une barrière en fougère}
\end{glose}
\end{sous-entrée}
\newline
\begin{sous-entrée}{thîni hauva}{ⓔthîniⓝthîni hauva}
\région{GO}
\begin{glose}
\pfra{enclos}
\end{glose}
\newline
\note{forme déterminée : thini(v)a}{grammaire}{}
\end{sous-entrée}
\end{entrée}

\begin{entrée}{thîni-a kavegu}{}{ⓔthîni-a kavegu}
\formephonétique{tʰîɳi}
\région{GOs}
(\domainesémantique{Habitat
, Organisation sociale})
\classe{nom}
\begin{glose}
\pfra{palissade de la chefferie}
\end{glose}
\end{entrée}

\begin{entrée}{thîni hauva}{}{ⓔthîni hauva}
\formephonétique{tʰiɳi hauva}
\région{GOs}
(\domainesémantique{Organisation sociale})
\classe{nom}
\begin{glose}
\pfra{enclos pour les dons coutumiers}
\end{glose}
\newline
\note{(ceint par une barrière où l'on met les dons coutumiers destinés au clan maternel; les oncles maternels démolissent l'enclos et emportent alors les dons ; ne se pratique que pour les personnes de rang important)}{glose}{}
\end{entrée}

\begin{entrée}{thiò dimwa}{}{ⓔthiò dimwa}
\région{GOs PA BO}
\variante{%
thixò, thiò
\région{PA BO}}
(\domainesémantique{Préparation des aliments; modes de préparation et de cuisson})
\classe{v}
\begin{glose}
\pfra{gratter l'igname sauvage (dimwa) ; râper}
\end{glose}
\newline
\begin{exemple}
\région{BO}
\textbf{\pnua{la thi(x)ò la ca-la dimwa}}
\pfra{elles rapent leur igname sauvage à manger}
\end{exemple}
\end{entrée}

\begin{entrée}{thiò nu}{}{ⓔthiò nu}
\formephonétique{tʰiɔ ɳu}
\région{GOs PA}
\variante{%
thixò
\région{GO(s) PA BO}}
(\domainesémantique{Cocotiers})
\classe{v}
\begin{glose}
\pfra{décortiquer le coprah (avec un couteau)}
\end{glose}
\end{entrée}

\begin{entrée}{thi-paa}{}{ⓔthi-paa}
\région{BO}
\variante{%
tho-paa
}
(\domainesémantique{Guerre})
\classe{nom}
\begin{glose}
\pfra{embuscade (guerre) ; embûches [Corne]}
\end{glose}
\newline
\note{non vérifié}{général}{}
\end{entrée}

\begin{entrée}{thi-parô}{}{ⓔthi-parô}
\région{GOs}
(\domainesémantique{Soins du corps})
\classe{v}
\begin{glose}
\pfra{curer (se) les dents}
\end{glose}
\end{entrée}

\begin{entrée}{thi-pöloo}{}{ⓔthi-pöloo}
\formephonétique{'tʰiβωloː}
\région{GOs}
\variante{%
thuvwuloo
\région{WEM WEH}}
(\domainesémantique{Pêche})
\classe{v}
\begin{glose}
\pfra{troubler (l'eau par exemple pour pêcher dans la rivière)}
\end{glose}
\newline
\relationsémantique{Cf.}{\lien{ⓔphöloo}{phöloo}}
\glosecourte{trouble}
\end{entrée}

\begin{entrée}{thi-pu}{}{ⓔthi-pu}
\région{PA}
(\domainesémantique{Soins du corps})
\classe{v}
\begin{glose}
\pfra{peigner (se)}
\end{glose}
\newline
\begin{exemple}
\région{PA}
\textbf{\pnua{i thi pu-bwaa-n}}
\pfra{il se peigne}
\end{exemple}
\newline
\begin{exemple}
\région{PA}
\textbf{\pnua{i thi pu Kaavo}}
\pfra{Kaavo se peigne}
\end{exemple}
\newline
\begin{exemple}
\région{PA}
\textbf{\pnua{i thi puu-n}}
\pfra{il se peigne}
\end{exemple}
\end{entrée}

\begin{entrée}{thi-puu}{}{ⓔthi-puu}
\région{BO PA}
\variante{%
tho-puu
\région{GO(s) BO}}
(\domainesémantique{Navigation})
\classe{v}
\begin{glose}
\pfra{pousser (bateau) avec la perche (lit. piquer)}
\end{glose}
\newline
\begin{exemple}
\région{BO}
\textbf{\pnua{i tho-pue phaa-gò}}
\pfra{il pousse le radeau avec une perche}
\end{exemple}
\newline
\begin{sous-entrée}{ba-thi-puu}{ⓔthi-puuⓝba-thi-puu}
\région{PA BO}
\begin{glose}
\pfra{perche}
\end{glose}
\newline
\relationsémantique{Cf.}{\lien{}{thoe, thi}}
\glosecourte{piquer}
\end{sous-entrée}
\end{entrée}

\begin{entrée}{thi-pholo}{}{ⓔthi-pholo}
\région{GOs BO}
(\domainesémantique{Pêche
, Mouvements ou actions faits avec le corps, les bras, les mains, les pieds})
\classe{v}
\begin{glose}
\pfra{taper dans l'eau pour faire du bruit (et effrayer le poisson afin de le pousser dans le filet)}
\end{glose}
\end{entrée}

\begin{entrée}{thiram}{}{ⓔthiram}
\région{PA}
\variante{%
thiam
\région{BO}}
(\domainesémantique{Danses})
\classe{nom}
\begin{glose}
\pfra{danse d'accueil des hommes}
\end{glose}
\end{entrée}

\begin{entrée}{thiraò}{}{ⓔthiraò}
\région{GOs PA BO}
(\domainesémantique{Verbes de déplacement et moyens de déplacement})
\classe{v}
\begin{glose}
\pfra{traverser ; passer à travers; transpercer}
\end{glose}
\newline
\begin{exemple}
\région{GOs}
\textbf{\pnua{hã no thiraò hõbwòli-çö}}
\pfra{on voit à travers ta robe}
\end{exemple}
\end{entrée}

\begin{entrée}{thirawa}{}{ⓔthirawa}
\région{GOs}
(\domainesémantique{Verbes de déplacement et moyens de déplacement})
\classe{v}
\begin{glose}
\pfra{traverser ; passer à travers}
\end{glose}
\newline
\begin{exemple}
\textbf{\pnua{e thirawa do}}
\pfra{la sagaie est passée à travers}
\end{exemple}
\newline
\begin{exemple}
\textbf{\pnua{nu no thirawa pweemwa}}
\pfra{je vois à travers la porte}
\end{exemple}
\newline
\begin{exemple}
\textbf{\pnua{nu no thirawa ni hõbwòli-je}}
\pfra{je vois à travers ses vêtements}
\end{exemple}
\end{entrée}

\begin{entrée}{thirûû}{}{ⓔthirûû}
\région{GOs}
(\domainesémantique{Relations et interaction sociales})
\classe{v}
\begin{glose}
\pfra{ennuyer (s')}
\end{glose}
\newline
\begin{exemple}
\textbf{\pnua{e thirûû}}
\pfra{elle s'ennuie}
\end{exemple}
\end{entrée}

\begin{entrée}{thi-tou}{}{ⓔthi-tou}
\région{GOs PA}
(\domainesémantique{Dons, échanges, achat et vente, vol})
\classe{v}
\begin{glose}
\pfra{distribuer en partage}
\end{glose}
\end{entrée}

\begin{entrée}{thi-ula}{}{ⓔthi-ula}
\région{PA}
(\domainesémantique{Mouvements ou actions faits avec le corps, les bras, les mains, les pieds})
\classe{v}
\begin{glose}
\pfra{décrocher en piquant avec un bâton (qqch qui se trouve en hauteur)}
\end{glose}
\end{entrée}

\begin{entrée}{thivwaa}{}{ⓔthivwaa}
\formephonétique{tʰiβaː}
\région{GOs BO}
\classe{nom}
\newline
\sens{1}
(\domainesémantique{Vêtements, parure})
\begin{glose}
\pfra{bagayou ; étui pénien}
\end{glose}
\newline
\begin{exemple}
\région{GO}
\textbf{\pnua{thipwa-n}}
\pfra{son étui pénien}
\end{exemple}
\newline
\sens{2}
(\domainesémantique{Pêche})
\begin{glose}
\pfra{flotteur de filet [BO]}
\end{glose}
\newline
\begin{exemple}
\région{BO}
\textbf{\pnua{thivwaa pwio}}
\pfra{flotteur de filet}
\end{exemple}
\end{entrée}

\begin{entrée}{thivwi}{}{ⓔthivwi}
\formephonétique{tʰiβi}
\région{GOs BO}
\variante{%
thipi
\formephonétique{tʰipi}
\région{GO(s)}}
(\domainesémantique{Fonctions naturelles humaines})
\classe{v}
\begin{glose}
\pfra{aspirer (liquide avec une paille)}
\end{glose}
\begin{glose}
\pfra{sucer (bonbon)}
\end{glose}
\end{entrée}

\begin{entrée}{thivwöloo}{}{ⓔthivwöloo}
\formephonétique{'tʰiβωloː}
\région{GOs BO}
\variante{%
tivwoloo
\région{BO}}
(\domainesémantique{Santé, maladie})
\classe{v ; n}
\begin{glose}
\pfra{chancre}
\end{glose}
\begin{glose}
\pfra{pustule}
\end{glose}
\begin{glose}
\pfra{boutons (sur le corps)}
\end{glose}
\begin{glose}
\pfra{pus}
\end{glose}
\end{entrée}

\begin{entrée}{thixa jige}{}{ⓔthixa jige}
\région{GOs}
(\domainesémantique{Armes})
\classe{nom}
\begin{glose}
\pfra{coup de fusil}
\end{glose}
\end{entrée}

\begin{entrée}{thixèè}{}{ⓔthixèè}
\région{GOs}
\variante{%
thaxee
\région{PA}}
(\domainesémantique{Quantificateurs})
\classe{QNT}
\begin{glose}
\pfra{un d'une paire ; un seul (d'une paire)}
\end{glose}
\newline
\begin{exemple}
\région{GOs}
\textbf{\pnua{thixèè mee-je}}
\pfra{il est borgne (il a un seul oeil)}
\end{exemple}
\newline
\begin{exemple}
\région{GOs}
\textbf{\pnua{thixèè ala-xòò-je}}
\pfra{il n'a qu'une seule chaussure}
\end{exemple}
\newline
\begin{exemple}
\région{PA}
\textbf{\pnua{ala-kò thaxee}}
\pfra{une seule chaussure}
\end{exemple}
\newline
\begin{exemple}
\région{GOs}
\textbf{\pnua{e mhoge ala-xòò-je thixèè}}
\pfra{il n'a attaché qu'une chaussure}
\end{exemple}
\newline
\begin{exemple}
\région{GOs}
\textbf{\pnua{nu tròòli thixèè ala-xòò-nu}}
\pfra{je n 'ai trouvé qu'une chaussure}
\end{exemple}
\newline
\begin{exemple}
\région{GOs}
\textbf{\pnua{nu tròòli-adaa-ni thixèè ala-xòò-nu}}
\pfra{je n 'ai trouvé qu'une seule chaussure}
\end{exemple}
\newline
\relationsémantique{Cf.}{\lien{}{wè-xè hii-je [GOs]}}
\glosecourte{il n'a qu'un seul bras}
\newline
\relationsémantique{Cf.}{\lien{}{thilò [GOs]}}
\glosecourte{l'autre d'une paire}
\end{entrée}

\begin{entrée}{thixèè mee-je}{}{ⓔthixèè mee-je}
\région{GOs}
(\domainesémantique{Santé, maladie})
\classe{v}
\begin{glose}
\pfra{borgne (lit. il a un seul oeil)}
\end{glose}
\newline
\begin{exemple}
\textbf{\pnua{e thixèè mee-nu}}
\pfra{je suis borgne}
\end{exemple}
\end{entrée}

\begin{entrée}{thixò}{}{ⓔthixò}
\région{GOs}
\classe{v}
(\domainesémantique{Préfixes et verbes de position})
\begin{glose}
\pfra{mettre (se) sur la pointe des pieds}
\end{glose}
\newline
\relationsémantique{Cf.}{\lien{}{ala-kò-thixò, ala-xò-thixò}}
\glosecourte{chaussure à talons}
\newline
\morphologie{thi kò}
\end{entrée}

\begin{entrée}{thixudi}{}{ⓔthixudi}
\région{PA BO}
\variante{%
thivwudi
\région{PA BO}}
(\domainesémantique{Configuration des objets})
\classe{nom}
\begin{glose}
\pfra{coin ; angle}
\end{glose}
\newline
\begin{exemple}
\textbf{\pnua{thivwudi thîni}}
\pfra{coin de l'enclos}
\end{exemple}
\newline
\relationsémantique{Cf.}{\lien{ⓔkudi}{kudi}}
\glosecourte{coin}
\end{entrée}

\begin{entrée}{thi yaai}{}{ⓔthi yaai}
\région{GOs PA WEM BO}
(\domainesémantique{Feu : objets et actions liés au feu})
\classe{v}
\begin{glose}
\pfra{allumer un feu de brousse (lit. piquer le feu) ; mettre le feu}
\end{glose}
\begin{glose}
\pfra{attiser le feu en remuant les braises avec un bâton [PA]}
\end{glose}
\newline
\note{allumer un feu de brousse avec des bouts de feuilles de coco secs ou des brindilles enflammées}{glose}{}
\newline
\begin{exemple}
\région{GO}
\textbf{\pnua{e pe-thi}}
\pfra{il a mis le feu}
\end{exemple}
\newline
\begin{exemple}
\région{GO}
\textbf{\pnua{egu xa a-pe-thi}}
\pfra{un pyromane}
\end{exemple}
\newline
\begin{sous-entrée}{a-pe-thi}{ⓔthi yaaiⓝa-pe-thi}
\begin{glose}
\pfra{pyromane}
\end{glose}
\end{sous-entrée}
\newline
\begin{sous-entrée}{nobo-yai}{ⓔthi yaaiⓝnobo-yai}
\begin{glose}
\pfra{brûlure; traces de feu}
\end{glose}
\end{sous-entrée}
\end{entrée}

\begin{entrée}{thiza}{}{ⓔthiza}
\région{GOs}
(\domainesémantique{Armes})
\classe{nom}
\begin{glose}
\pfra{barbeau de la sagaie}
\end{glose}
\end{entrée}

\begin{entrée}{thizi}{}{ⓔthizi}
\région{GOs}
(\domainesémantique{Mouvements ou actions faits avec le corps, les bras, les mains, les pieds})
\classe{v}
\begin{glose}
\pfra{essuyer (s') les fesses}
\end{glose}
\newline
\begin{exemple}
\textbf{\pnua{e thizi buxè}}
\pfra{elle s'essuie (ce qui reste sur l'anus)}
\end{exemple}
\end{entrée}

\begin{entrée}{thizii}{}{ⓔthizii}
\région{GOs}
\variante{%
thiri
\région{PA}, 
tiri
\région{BO WEM}}
(\domainesémantique{Parties du corps humain : doigts, orteil})
\classe{nom}
\begin{glose}
\pfra{auriculaire}
\end{glose}
\newline
\begin{sous-entrée}{thiri-a kò-n}{ⓔthiziiⓝthiri-a kò-n}
\région{PA}
\begin{glose}
\pfra{petit orteil}
\end{glose}
\end{sous-entrée}
\end{entrée}

\begin{entrée}{tho}{1}{ⓔthoⓗ1}
\formephonétique{tʰo}
\région{GOs PA BO}
\classe{v.i. ; n}
\newline
\sens{1}
(\domainesémantique{Sons, bruits})
\begin{glose}
\pfra{cri ; appel ; son}
\end{glose}
\newline
\sens{2}
(\domainesémantique{Relations et interaction sociales})
\begin{glose}
\pfra{appeler ; interpeller}
\end{glose}
\newline
\begin{exemple}
\région{BO}
\textbf{\pnua{i thomã kaawo}}
\pfra{il appelle Kaavo}
\end{exemple}
\newline
\begin{exemple}
\région{BO}
\textbf{\pnua{i thomã-nu}}
\pfra{il m'appelle}
\end{exemple}
\newline
\sens{3}
(\domainesémantique{Musique, instruments de musique})
\begin{glose}
\pfra{chanter (oiseau) ; chant}
\end{glose}
\begin{glose}
\pfra{musique}
\end{glose}
\newline
\begin{exemple}
\textbf{\pnua{e tho mèni}}
\pfra{l'oiseau chante}
\end{exemple}
\newline
\begin{sous-entrée}{tho-mèni}{ⓔthoⓗ1ⓢ3ⓝtho-mèni}
\begin{glose}
\pfra{le chant de l'oiseau}
\end{glose}
\end{sous-entrée}
\newline
\begin{sous-entrée}{tho-ko}{ⓔthoⓗ1ⓢ3ⓝtho-ko}
\begin{glose}
\pfra{le chant du coq}
\end{glose}
\newline
\begin{exemple}
\région{BO}
\textbf{\pnua{tho gò}}
\pfra{il y a de la musique}
\end{exemple}
\newline
\begin{exemple}
\région{BO}
\textbf{\pnua{i pa-tho-ni gò hu ri ?}}
\pfra{qui fait jouer de la musique ?}
\end{exemple}
\newline
\note{thoni (v.t.)}{grammaire}{}
\end{sous-entrée}
\newline
\étymologie{
\langue{POc}
\étymon{*soRov(i)}
\auteur{*cho}
\auteur{Geraghty}}
\end{entrée}

\begin{entrée}{tho}{2}{ⓔthoⓗ2}
\formephonétique{tʰo}
\région{GOs PA BO}
(\domainesémantique{Actions liées aux éléments (liquide, fumée)})
\classe{v}
\begin{glose}
\pfra{couler (eau, sang) ; écouler (s')}
\end{glose}
\newline
\begin{exemple}
\textbf{\pnua{tho na va ?}}
\pfra{d'où prend il sa source ? (fleuve)}
\end{exemple}
\newline
\begin{exemple}
\région{BO}
\textbf{\pnua{thoo-we}}
\pfra{courant d'eau, fuite d'eau, cascade}
\end{exemple}
\end{entrée}

\begin{entrée}{thò}{}{ⓔthò}
\région{GOs}
(\domainesémantique{Arbre})
\classe{nom}
\begin{glose}
\pfra{banian ; caoutchouc}
\end{glose}
\end{entrée}

\begin{entrée}{thò-}{}{ⓔthò-}
\région{GOs PA BO}
(\domainesémantique{Préfixes classificateurs numériques})
\classe{CLF.NUM}
\begin{glose}
\pfra{régimes de banane}
\end{glose}
\newline
\begin{exemple}
\textbf{\pnua{thò-xe, thò-tru, thò-ko, thò-pa, thò-ni thò-chaamwa etc.}}
\pfra{un, deux, trois, quatre, cinq régimes de banane}
\end{exemple}
\end{entrée}

\begin{entrée}{thô}{}{ⓔthô}
\région{GOs}
\variante{%
thõn
\région{BO}}
\classe{v ; n}
\newline
\sens{1}
(\domainesémantique{Verbes d'action (en général)})
\begin{glose}
\pfra{fermer}
\end{glose}
\newline
\sens{2}
(\domainesémantique{Description des objets, formes, consistance, taille})
\begin{glose}
\pfra{étanche}
\end{glose}
\newline
\begin{exemple}
\région{PA}
\textbf{\pnua{ba-kham thô}}
\pfra{louche}
\end{exemple}
\newline
\begin{exemple}
\région{BO}
\textbf{\pnua{ba-kam thõn}}
\pfra{louche}
\end{exemple}
\newline
\sens{3}
(\domainesémantique{Relations et interaction sociales})
\begin{glose}
\pfra{colère contre qqn (être en) ; dispute}
\end{glose}
\newline
\begin{exemple}
\région{GO}
\textbf{\pnua{e thôe-nu}}
\pfra{elle est en colère contre moi}
\end{exemple}
\newline
\begin{exemple}
\région{GO}
\textbf{\pnua{e ka thô}}
\pfra{elle est susceptible, se met en colère pour rien}
\end{exemple}
\newline
\begin{exemple}
\région{GO}
\textbf{\pnua{li pe-thô}}
\pfra{ils sont en conflit}
\end{exemple}
\newline
\begin{exemple}
\région{GO}
\textbf{\pnua{kavwo nu tõõne kaamweni nye pe-thô i li}}
\pfra{je ne comprends pas leur dispute}
\end{exemple}
\end{entrée}

\begin{entrée}{thò-chaamwa}{}{ⓔthò-chaamwa}
\formephonétique{,tʰɔ-'tʃʰaːmwa, tʰɔ-'cʰaːmwa}
\région{GOs PA BO}
\newline
\sens{1}
(\domainesémantique{Bananiers et bananes})
\classe{nom}
\begin{glose}
\pfra{régime de bananes}
\end{glose}
\newline
\sens{2}
(\domainesémantique{Préfixes classificateurs numériques})
\classe{CLF.NUM}
\begin{glose}
\pfra{régimede bananes}
\end{glose}
\newline
\begin{sous-entrée}{thò-xe}{ⓔthò-chaamwaⓢ2ⓝthò-xe}
\begin{glose}
\pfra{un régime de bananes}
\end{glose}
\end{sous-entrée}
\end{entrée}

\begin{entrée}{thoda}{}{ⓔthoda}
\région{PA}
(\domainesémantique{Relations et interaction sociales})
\classe{nom}
\begin{glose}
\pfra{malédiction ; mauvais sort}
\end{glose}
\newline
\begin{exemple}
\textbf{\pnua{we whany mãni thoda}}
\pfra{potion/antidote pour les malédictions et mauvais sorts}
\end{exemple}
\end{entrée}

\begin{entrée}{thodia}{}{ⓔthodia}
\formephonétique{'tʰodia}
\région{GOs}
\variante{%
thidya, tidya
\région{BO}}
(\domainesémantique{Description des objets, formes, consistance, taille})
\classe{v.stat.}
\begin{glose}
\pfra{rouillé}
\end{glose}
\end{entrée}

\begin{entrée}{thoè}{}{ⓔthoè}
\région{GOs}
\variante{%
thöe, toe
\région{BO PA}}
(\domainesémantique{Cultures, techniques, boutures})
\classe{v}
\begin{glose}
\pfra{planter ; mettre en terre (ignames, taro)}
\end{glose}
\end{entrée}

\begin{entrée}{thogaavwi}{}{ⓔthogaavwi}
\formephonétique{tʰogaːβi}
\région{GOs PA BO}
\variante{%
tho'gapi
\région{GO(s) vx}, 
tho'gavhi
\région{PA BO}}
(\domainesémantique{Phénomènes atmosphériques et naturels})
\classe{nom}
\begin{glose}
\pfra{echo}
\end{glose}
\newline
\begin{exemple}
\région{GOs}
\textbf{\pnua{nu trõne thogaavwi}}
\pfra{j'ai entendu l'écho}
\end{exemple}
\newline
\begin{exemple}
\région{PA}
\textbf{\pnua{pha-thogavwi}}
\pfra{faire écho}
\end{exemple}
\end{entrée}

\begin{entrée}{thôge}{}{ⓔthôge}
\région{GOs}
(\domainesémantique{Santé, maladie})
\classe{v}
\begin{glose}
\pfra{stérile (femme)}
\end{glose}
\newline
\begin{exemple}
\textbf{\pnua{e thôge}}
\pfra{elle est stérile}
\end{exemple}
\end{entrée}

\begin{entrée}{thoi haa}{}{ⓔthoi haa}
\région{GOs}
(\domainesémantique{Cultures, techniques, boutures})
\classe{v}
\begin{glose}
\pfra{planter des taros au bord de l'eau (rivière, etc.) sans système d'irrigation}
\end{glose}
\newline
\begin{exemple}
\région{GOs}
\textbf{\pnua{a thoi haa !}}
\pfra{va planter les taros au bord de l'eau !}
\end{exemple}
\end{entrée}

\begin{entrée}{thoimwã}{}{ⓔthoimwã}
\région{GOs}
\variante{%
toimwa
\région{PA BO}}
(\domainesémantique{Cours de la vie})
\classe{nom}
\begin{glose}
\pfra{vieille-femme}
\end{glose}
\newline
\relationsémantique{Cf.}{\lien{}{wha-mã [GOs], hua-mã}}
\glosecourte{vieil-homme}
\end{entrée}

\begin{entrée}{tho-khõbwe}{}{ⓔtho-khõbwe}
\région{GOs PA}
(\domainesémantique{Relations et interaction sociales})
\classe{v}
\begin{glose}
\pfra{annoncer publiquement ; faire une annonce}
\end{glose}
\end{entrée}

\begin{entrée}{thòlòe}{}{ⓔthòlòe}
\région{BO [Corne]}
\variante{%
tholee tolee
\région{BO [Corne]}}
(\domainesémantique{Actions liées aux éléments (liquide, fumée)})
\classe{v}
\begin{glose}
\pfra{répandre (se) (eau, fumée)}
\end{glose}
\end{entrée}

\begin{entrée}{thomã}{}{ⓔthomã}
\région{GOs WEM PA BO}
(\domainesémantique{Relations et interaction sociales})
\classe{v.t.}
\begin{glose}
\pfra{appeler}
\end{glose}
\newline
\begin{exemple}
\région{GOs}
\textbf{\pnua{thomã-je !}}
\pfra{appelle-le !}
\end{exemple}
\newline
\begin{exemple}
\région{GOs}
\textbf{\pnua{e thomã-çö !}}
\pfra{il/elle t'appelle !}
\end{exemple}
\newline
\begin{exemple}
\région{WEM}
\textbf{\pnua{tho-mi xo Honorine}}
\pfra{Honorine appelle (vers elle)}
\end{exemple}
\newline
\begin{exemple}
\région{WEM}
\textbf{\pnua{thomã-je xo õ-Milen}}
\pfra{Milène l'appelle}
\end{exemple}
\newline
\begin{sous-entrée}{tho-mi}{ⓔthomãⓝtho-mi}
\begin{glose}
\pfra{appeler vers soi}
\end{glose}
\newline
\note{la forme thoma a un objet pronominal; thomani a un objet nominal}{grammaire}{}
\end{sous-entrée}
\end{entrée}

\begin{entrée}{thomãni}{}{ⓔthomãni}
\formephonétique{tʰomɛ̃ɳi}
\région{GOs PA}
(\domainesémantique{Relations et interaction sociales})
\classe{v.t.}
\begin{glose}
\pfra{appeler qqn}
\end{glose}
\newline
\begin{exemple}
\textbf{\pnua{e thomãni êgu na èboli}}
\pfra{il appelle les gens d'en bas}
\end{exemple}
\end{entrée}

\begin{entrée}{tho-me}{}{ⓔtho-me}
\région{PA}
(\domainesémantique{Caractéristiques et propriétés des personnes})
\classe{v}
\begin{glose}
\pfra{crâner ; faire le malin}
\end{glose}
\newline
\begin{exemple}
\textbf{\pnua{i pe-tho-me}}
\pfra{il fait le malin}
\end{exemple}
\end{entrée}

\begin{entrée}{thonga}{}{ⓔthonga}
\formephonétique{tʰoŋa}
\région{GOs}
(\domainesémantique{Santé, maladie})
\classe{nom}
\begin{glose}
\pfra{plaie sur les pieds}
\end{glose}
\newline
\emprunt{tonga (POLYN) (PPN *tonga)}
\end{entrée}

\begin{entrée}{thôni}{}{ⓔthôni}
\formephonétique{tʰôɳi}
\région{GOs WEM WE PA BO}
\classe{v.t.}
\newline
\sens{1}
(\domainesémantique{Verbes d'action (en général)})
\begin{glose}
\pfra{fermer qqch (avec un objet, un couvercle)}
\end{glose}
\newline
\sens{2}
(\domainesémantique{Relations et interaction sociales})
\begin{glose}
\pfra{interdire}
\end{glose}
\begin{glose}
\pfra{empêcher (obstacle)}
\end{glose}
\newline
\begin{exemple}
\région{GOs}
\textbf{\pnua{tee thô}}
\pfra{c'est fermé}
\end{exemple}
\newline
\begin{exemple}
\région{GOs}
\textbf{\pnua{la thôni dè ko/xo la êgu}}
\pfra{les hommes ont barré/bloqué la route}
\end{exemple}
\newline
\begin{exemple}
\région{GOs}
\textbf{\pnua{nu thôni-çö na ni nye kudo}}
\pfra{je t'interdis la boisson (je te barre l'accès à cette boisson)}
\end{exemple}
\newline
\begin{exemple}
\région{GOs}
\textbf{\pnua{e thô kinii-nu}}
\pfra{j'ai les oreilles bouchées}
\end{exemple}
\newline
\begin{exemple}
\région{BO}
\textbf{\pnua{thôni phwee-mwa}}
\pfra{ferme la porte}
\end{exemple}
\newline
\begin{sous-entrée}{thôni we}{ⓔthôniⓢ2ⓝthôni we}
\begin{glose}
\pfra{fermer le robinet d'eau, vanne}
\end{glose}
\newline
\note{v.i. thô}{grammaire}{}
\end{sous-entrée}
\end{entrée}

\begin{entrée}{tho nõ}{}{ⓔtho nõ}
\formephonétique{tʰo ɳɔ̃}
\région{GOs BO}
(\domainesémantique{Fonctions naturelles humaines})
\classe{v}
\begin{glose}
\pfra{péter ; avoir des vents}
\end{glose}
\newline
\begin{exemple}
\région{BO}
\textbf{\pnua{i tho nõ-n}}
\pfra{il pète}
\end{exemple}
\newline
\relationsémantique{Cf.}{\lien{}{vhii [GOs]}}
\glosecourte{péter}
\end{entrée}

\begin{entrée}{thoo}{1}{ⓔthooⓗ1}
\région{GOs PA BO}
(\domainesémantique{Cultures, techniques, boutures})
\classe{v}
\begin{glose}
\pfra{arracher la canne à sucre ; récolter}
\end{glose}
\newline
\begin{exemple}
\région{GOs}
\textbf{\pnua{e thoo ê}}
\pfra{il arrache, récolte la canne à sucre}
\end{exemple}
\newline
\begin{exemple}
\région{PA}
\textbf{\pnua{i thoo èm}}
\pfra{il arrache, récolte la canne à sucre}
\end{exemple}
\newline
\begin{exemple}
\région{PA}
\textbf{\pnua{thoo-xa whala-m èm !}}
\pfra{va récolter de la canne à sucre pour toi !}
\end{exemple}
\end{entrée}

\begin{entrée}{thoo}{2}{ⓔthooⓗ2}
\région{PA BO}
\variante{%
throo
\région{GO(s)}}
(\domainesémantique{Vêtements, parure})
\classe{nom}
\begin{glose}
\pfra{aigrette (coiffure)}
\end{glose}
\begin{glose}
\pfra{plume ou fleur plantée sur le sommet de la tête}
\end{glose}
\newline
\begin{exemple}
\région{PA}
\textbf{\pnua{thoo-ny}}
\pfra{mon plumet}
\end{exemple}
\end{entrée}

\begin{entrée}{thòò}{}{ⓔthòò}
\région{GOs BO}
(\domainesémantique{Fonctions naturelles humaines})
\classe{v}
\begin{glose}
\pfra{accoupler (s') ; avoir des relations sexuelles}
\end{glose}
\begin{glose}
\pfra{violer}
\end{glose}
\newline
\begin{exemple}
\région{BO}
\textbf{\pnua{i thòò-e}}
\pfra{il l'a violée}
\end{exemple}
\end{entrée}

\begin{entrée}{thöö}{1}{ⓔthööⓗ1}
\formephonétique{tʰωː}
\région{GOs PA}
\variante{%
thoo
\région{BO}}
(\domainesémantique{Relations et interaction sociales})
\classe{v}
\begin{glose}
\pfra{maudire}
\end{glose}
\newline
\begin{exemple}
\région{BO}
\textbf{\pnua{i thoo-je}}
\pfra{il l'a maudite}
\end{exemple}
\end{entrée}

\begin{entrée}{thöö}{2}{ⓔthööⓗ2}
\région{PA}
\variante{%
tuu
\région{BO}}
(\domainesémantique{Aspect})
\classe{ASP}
\begin{glose}
\pfra{en même temps}
\end{glose}
\end{entrée}

\begin{entrée}{thooli}{}{ⓔthooli}
\région{GOs PA}
(\domainesémantique{Mollusques})
\classe{nom}
\begin{glose}
\pfra{"coquilon"(à coquille longue)}
\end{glose}
\begin{glose}
\pfra{bernard-l'ermite (se met dans la coquille du 'thooli')}
\end{glose}
\nomscientifique{Coenobita ollivieri}
\end{entrée}

\begin{entrée}{thoomwã}{}{ⓔthoomwã}
\formephonétique{tʰoːmwɛ̃}
\région{GOs PA}
(\domainesémantique{Société})
\classe{nom}
\begin{glose}
\pfra{femme ; féminin}
\end{glose}
\newline
\begin{sous-entrée}{phe thoomwã}{ⓔthoomwãⓝphe thoomwã}
\begin{glose}
\pfra{se marier (prendre une épouse )}
\end{glose}
\end{sous-entrée}
\newline
\begin{sous-entrée}{ẽnõ thoomwã}{ⓔthoomwãⓝẽnõ thoomwã}
\begin{glose}
\pfra{fille}
\end{glose}
\end{sous-entrée}
\newline
\begin{sous-entrée}{thoimwã}{ⓔthoomwãⓝthoimwã}
\begin{glose}
\pfra{vieille-femme}
\end{glose}
\end{sous-entrée}
\end{entrée}

\begin{entrée}{thoomwã kòlò}{}{ⓔthoomwã kòlò}
\région{PA BO}
(\domainesémantique{Parenté})
\classe{nom}
\begin{glose}
\pfra{nièce (fille de frère et cousins)}
\end{glose}
\newline
\begin{exemple}
\région{PA}
\textbf{\pnua{thoomwã kòlò-ny}}
\pfra{fille de frère: nièce}
\end{exemple}
\end{entrée}

\begin{entrée}{thooni}{}{ⓔthooni}
\formephonétique{tʰoːɳi}
\région{GOs PA}
\classe{v}
\newline
\sens{1}
(\domainesémantique{Relations et interaction sociales})
\begin{glose}
\pfra{annoncer des informations ; présenter}
\end{glose}
\newline
\sens{2}
(\domainesémantique{Coutumes, dons coutumiers})
\begin{glose}
\pfra{désigner un tas pour un clan (coutume)}
\end{glose}
\end{entrée}

\begin{entrée}{thööni}{}{ⓔthööni}
\formephonétique{tʰωːɳi}
\région{GOs PA}
(\domainesémantique{Relations et interaction sociales})
\classe{v}
\begin{glose}
\pfra{maudire}
\end{glose}
\end{entrée}

\begin{entrée}{thòòzò}{}{ⓔthòòzò}
\région{GOs}
\variante{%
thòròe
\région{PA}}
(\domainesémantique{Pêche})
\classe{v}
\begin{glose}
\pfra{piquer (dans un trou avec une sagaie pour chercher des anguilles, poissons)}
\end{glose}
\newline
\begin{exemple}
\région{GOs}
\textbf{\pnua{e thòzò peena}}
\pfra{il pique pour trouver une anguille}
\end{exemple}
\newline
\begin{exemple}
\région{PA}
\textbf{\pnua{e thòròe para}}
\pfra{il pique les herbes (pour trouver des anguilles)}
\end{exemple}
\newline
\begin{exemple}
\région{PA}
\textbf{\pnua{e thòròe paa}}
\pfra{il pique (sous) les pierres (pour trouver des anguilles, poissons)}
\end{exemple}
\newline
\relationsémantique{Cf.}{\lien{}{thaa, thi}}
\glosecourte{piquer}
\end{entrée}

\begin{entrée}{tho-puu}{}{ⓔtho-puu}
\région{GOs BO}
\variante{%
thi-puu
\région{PA BO}}
(\domainesémantique{Navigation})
\classe{v}
\begin{glose}
\pfra{pousser (bateau) avec la perche (lit. piquer)}
\end{glose}
\newline
\begin{exemple}
\région{BO}
\textbf{\pnua{i tho-pue phaa-gò}}
\pfra{il pousse le radeau avec une perche}
\end{exemple}
\newline
\begin{sous-entrée}{ba-thi-puu}{ⓔtho-puuⓝba-thi-puu}
\région{PA BO}
\begin{glose}
\pfra{perche}
\end{glose}
\newline
\relationsémantique{Cf.}{\lien{}{thoe [GOs], thi}}
\glosecourte{piquer}
\end{sous-entrée}
\end{entrée}

\begin{entrée}{thou}{}{ⓔthou}
\région{GOs}
\variante{%
theun
\région{BO [Corne]}}
(\domainesémantique{Fonctions naturelles des animaux})
\classe{v}
\begin{glose}
\pfra{muer (lézard, serpent)}
\end{glose}
\newline
\begin{exemple}
\textbf{\pnua{e thou pwaji}}
\pfra{le crabe mue}
\end{exemple}
\end{entrée}

\begin{entrée}{thovwa}{}{ⓔthovwa}
\formephonétique{tʰoβa}
\région{GOs}
\variante{%
thòva
\région{BO [BM]}}
(\domainesémantique{Description des objets, formes, consistance, taille})
\classe{v}
\begin{glose}
\pfra{plat ; aplati}
\end{glose}
\newline
\begin{exemple}
\textbf{\pnua{ne wo thovwa !}}
\pfra{aplatis-le (lit. fais pour que ce soit plat)}
\end{exemple}
\end{entrée}

\begin{entrée}{thô-vwaa-ni}{}{ⓔthô-vwaa-ni}
\région{GOs}
(\domainesémantique{Mouvements ou actions faits avec le corps, les bras, les mains, les pieds})
\classe{v}
\begin{glose}
\pfra{rapiécer (lit. fermer le trou)}
\end{glose}
\newline
\begin{sous-entrée}{thô-vwaa}{ⓔthô-vwaa-niⓝthô-vwaa}
\begin{glose}
\pfra{pièce (pour rapiécer)}
\end{glose}
\newline
\relationsémantique{Cf.}{\lien{ⓔphwaⓗ1}{phwa}}
\glosecourte{trou}
\end{sous-entrée}
\end{entrée}

\begin{entrée}{thòxe}{}{ⓔthòxe}
\région{PA}
\variante{%
thòxe, thòòge
\région{BO [BM]}}
(\domainesémantique{Description des objets, formes, consistance, taille})
\classe{v}
\begin{glose}
\pfra{collé (par ex. au fond de la marmite)}
\end{glose}
\newline
\begin{exemple}
\région{BO}
\textbf{\pnua{i thòxe na inu u ciia}}
\pfra{le poulpe s'est collé à moi}
\end{exemple}
\newline
\begin{exemple}
\région{BO}
\textbf{\pnua{i thòòge ari}}
\pfra{le riz est collé}
\end{exemple}
\end{entrée}

\begin{entrée}{thozoe}{}{ⓔthozoe}
\région{GOs}
\variante{%
toroe
\région{PA BO}}
\classe{v}
\newline
\sens{1}
(\domainesémantique{Mouvements ou actions faits avec le corps, les bras, les mains, les pieds})
\begin{glose}
\pfra{cacher ; dissimuler qqch}
\end{glose}
\newline
\begin{exemple}
\textbf{\pnua{na mi mwani pu nu tree thozoe}}
\pfra{donne-moi l'argent pour que je le cache (tree: pendant que tu fais autre chose)}
\end{exemple}
\newline
\begin{sous-entrée}{kô-thozoe}{ⓔthozoeⓢ1ⓝkô-thozoe}
\begin{glose}
\pfra{caché}
\end{glose}
\newline
\relationsémantique{Cf.}{\lien{ⓔku-çaaxò}{ku-çaaxò}}
\glosecourte{se cacher}
\end{sous-entrée}
\newline
\sens{2}
(\domainesémantique{Relations et interaction sociales})
\begin{glose}
\pfra{garder secret}
\end{glose}
\newline
\sens{3}
(\domainesémantique{Cours de la vie})
\begin{glose}
\pfra{enterrer (qqn)}
\end{glose}
\newline
\relationsémantique{Cf.}{\lien{}{khêmi [GOs]}}
\glosecourte{enterrer qqch.}
\end{entrée}

\begin{entrée}{thu}{}{ⓔthu}
\région{GOs WEM BO PA}
\variante{%
tho
\région{BO}}
(\domainesémantique{Prédicats existentiels})
\classe{v}
\begin{glose}
\pfra{faire}
\end{glose}
\begin{glose}
\pfra{il y a ; c'est}
\end{glose}
\newline
\begin{sous-entrée}{thu-paa}{ⓔthuⓝthu-paa}
\région{GO}
\begin{glose}
\pfra{faire la guerre}
\end{glose}
\end{sous-entrée}
\newline
\begin{sous-entrée}{thu-pwalu}{ⓔthuⓝthu-pwalu}
\begin{glose}
\pfra{respecter}
\end{glose}
\end{sous-entrée}
\newline
\begin{sous-entrée}{thu mhenõ}{ⓔthuⓝthu mhenõ}
\région{GO}
\begin{glose}
\pfra{voyager}
\end{glose}
\end{sous-entrée}
\newline
\begin{sous-entrée}{thu kibi}{ⓔthuⓝthu kibi}
\begin{glose}
\pfra{faire le four}
\end{glose}
\end{sous-entrée}
\newline
\begin{sous-entrée}{ba thu da ?}{ⓔthuⓝba thu da ?}
\begin{glose}
\pfra{ca sert à quoi ?}
\end{glose}
\newline
\begin{exemple}
\région{PA}
\textbf{\pnua{u thu ai-n}}
\pfra{il a de la maturité}
\end{exemple}
\newline
\begin{exemple}
\région{PA}
\textbf{\pnua{u thu pòi-n mwã}}
\pfra{il a des enfants maintenant}
\end{exemple}
\newline
\begin{exemple}
\région{BO}
\textbf{\pnua{i thu angai kale}}
\pfra{c'est la haute marée d'équinoxe (Dubois)}
\end{exemple}
\newline
\begin{exemple}
\région{BO}
\textbf{\pnua{i thu paga tavane}}
\pfra{c'est la basse marée d'équinoxe (Dubois)}
\end{exemple}
\newline
\begin{exemple}
\région{BO}
\textbf{\pnua{thu hava-hi-la}}
\pfra{ils ont des ailes (Boyd)}
\end{exemple}
\newline
\begin{exemple}
\région{BO}
\textbf{\pnua{thu-xa radio i yo ?}}
\pfra{as-tu une radio ? (Boyd)}
\end{exemple}
\newline
\begin{exemple}
\région{BO}
\textbf{\pnua{õ, thu radio i nu}}
\pfra{oui, j'ai une radio (Boyd)}
\end{exemple}
\end{sous-entrée}
\end{entrée}

\begin{entrée}{thua}{}{ⓔthua}
\formephonétique{tʰwa, tʰua}
\région{GOs PA BO}
(\domainesémantique{Caractéristiques et propriétés des animaux})
\classe{v.stat.}
\begin{glose}
\pfra{sauvage ; non domestiqué}
\end{glose}
\newline
\begin{exemple}
\région{GOs}
\textbf{\pnua{kwau thua}}
\pfra{chien sauvage}
\end{exemple}
\newline
\begin{exemple}
\région{PA}
\textbf{\pnua{poka thua}}
\pfra{cochon sauvage}
\end{exemple}
\newline
\begin{exemple}
\textbf{\pnua{di thua}}
\pfra{cordyline sauvage}
\end{exemple}
\end{entrée}

\begin{entrée}{thu ãbaa}{}{ⓔthu ãbaa}
\région{GOs}
\variante{%
tho ãbaa-n
\région{BO}}
(\domainesémantique{Verbes d'action (en général)})
\classe{v}
\begin{glose}
\pfra{ajouter ; mettre plus ; compléter}
\end{glose}
\newline
\begin{exemple}
\région{GOs}
\textbf{\pnua{thu ãbaa mwani}}
\pfra{ajoute de l'argent}
\end{exemple}
\newline
\begin{exemple}
\région{GOs}
\textbf{\pnua{na ãbaa mwani}}
\pfra{donne plus d'argent}
\end{exemple}
\end{entrée}

\begin{entrée}{thuada}{}{ⓔthuada}
\région{GOs WEM BO}
(\domainesémantique{Armes})
\classe{nom}
\begin{glose}
\pfra{armes de guerre (lance, sagaie)}
\end{glose}
\begin{glose}
\pfra{sagaie (grande) de guerre [BO]}
\end{glose}
\end{entrée}

\begin{entrée}{thu-ada}{}{ⓔthu-ada}
\région{GOs PA}
(\domainesémantique{Caractéristiques et propriétés des personnes})
\classe{v ; n}
\begin{glose}
\pfra{orgueil ; vouloir surpasser}
\end{glose}
\begin{glose}
\pfra{entêter (s') ; entêté [PA, GOs]}
\end{glose}
\newline
\begin{exemple}
\textbf{\pnua{li pe-thu-ada}}
\pfra{ils sont en compétition, ils font de la surenchère}
\end{exemple}
\end{entrée}

\begin{entrée}{thu ai-n}{}{ⓔthu ai-n}
\région{PA}
\classe{v}
\newline
\sens{1}
(\domainesémantique{Caractéristiques et propriétés des animaux})
\begin{glose}
\pfra{dressé (cheval, animal)}
\end{glose}
\newline
\begin{exemple}
\région{PA}
\textbf{\pnua{thu ai-n}}
\pfra{il est dressé}
\end{exemple}
\newline
\sens{2}
(\domainesémantique{Caractéristiques et propriétés des personnes})
\begin{glose}
\pfra{raisonnable ; mature (personne)}
\end{glose}
\newline
\begin{exemple}
\région{PA}
\textbf{\pnua{u thu ai-n}}
\pfra{il a une conscience, il est mûr (lit. il y a son coeur)}
\end{exemple}
\newline
\relationsémantique{Cf.}{\lien{ⓔkiya ai-n}{kiya ai-n}}
\glosecourte{pas dressé}
\end{entrée}

\begin{entrée}{thu bwa}{}{ⓔthu bwa}
\région{GOs}
(\domainesémantique{Caractéristiques et propriétés des personnes})
\classe{v}
\begin{glose}
\pfra{entêté}
\end{glose}
\newline
\begin{exemple}
\textbf{\pnua{e a-thu-bwa}}
\pfra{il est entêté}
\end{exemple}
\end{entrée}

\begin{entrée}{thu bwahî}{}{ⓔthu bwahî}
\région{GOs}
(\domainesémantique{Relations et interaction sociales})
\classe{v}
\begin{glose}
\pfra{prêter serment}
\end{glose}
\newline
\begin{exemple}
\textbf{\pnua{nu thu bwahî na bwa ala-mè pu-ã}}
\pfra{je prête serment devant Dieu}
\end{exemple}
\end{entrée}

\begin{entrée}{thu êgo}{}{ⓔthu êgo}
\région{GOs}
(\domainesémantique{Fonctions naturelles des animaux})
\classe{v}
\begin{glose}
\pfra{pondre}
\end{glose}
\newline
\relationsémantique{Cf.}{\lien{}{khaa-pi, khaa-vwi}}
\glosecourte{pondre}
\end{entrée}

\begin{entrée}{thu haal}{}{ⓔthu haal}
\région{PA}
(\domainesémantique{Verbes d'action (en général)})
\classe{v}
\begin{glose}
\pfra{mettre de côté ; réserver}
\end{glose}
\end{entrée}

\begin{entrée}{thu hoo}{}{ⓔthu hoo}
\région{GOs PA}
\variante{%
nee-vwo hoo-n
\région{PA}}
(\domainesémantique{Outils})
\classe{v}
\begin{glose}
\pfra{caler le manche}
\end{glose}
\newline
\begin{exemple}
\région{GOs}
\textbf{\pnua{e thu hoo kòò-piòò}}
\pfra{il cale le manche de la pioche}
\end{exemple}
\newline
\begin{exemple}
\région{PA}
\textbf{\pnua{nee-vwo-n}}
\pfra{cale-le !}
\end{exemple}
\end{entrée}

\begin{entrée}{thu hubu}{}{ⓔthu hubu}
\région{GOs}
\variante{%
thu hubun
\région{PA}}
(\domainesémantique{Caractéristiques et propriétés des personnes})
\classe{v}
\begin{glose}
\pfra{orgueilleux ; faire le fier; manquer d'humilité}
\end{glose}
\newline
\begin{exemple}
\textbf{\pnua{e thu hubu}}
\pfra{il est orgueilleux}
\end{exemple}
\newline
\begin{exemple}
\région{PA}
\textbf{\pnua{i thu hubun}}
\pfra{il est orgueilleux}
\end{exemple}
\end{entrée}

\begin{entrée}{thu-me}{}{ⓔthu-me}
\formephonétique{tʰuṃe}
\région{GOs PA}
(\domainesémantique{Caractéristiques et propriétés des personnes})
\classe{v}
\begin{glose}
\pfra{fier ; se faire remarquer faire (se) remarquer ; faire le malin}
\end{glose}
\newline
\begin{sous-entrée}{pa-thu-me-ni}{ⓔthu-meⓝpa-thu-me-ni}
\begin{glose}
\pfra{arborer qqch, faire remarquer qqch}
\end{glose}
\end{sous-entrée}
\end{entrée}

\begin{entrée}{thu-menõ}{}{ⓔthu-menõ}
\formephonétique{tʰumeɳɔ̃}
\région{GOs PA}
\variante{%
tomèno
\région{BO}}
(\domainesémantique{Verbes de déplacement et moyens de déplacement})
\classe{v}
\begin{glose}
\pfra{marcher (faire des traces)}
\end{glose}
\begin{glose}
\pfra{aller}
\end{glose}
\begin{glose}
\pfra{promener (se)}
\end{glose}
\begin{glose}
\pfra{fonctionner (machine)}
\end{glose}
\newline
\begin{exemple}
\région{GOs}
\textbf{\pnua{bi pe-thumenõ mãni ãbaa-nu xa thõõmwa}}
\pfra{j'ai fait le chemin avec ma soeur}
\end{exemple}
\newline
\begin{sous-entrée}{thumènõ-hayu}{ⓔthu-menõⓝthumènõ-hayu}
\région{GOs}
\begin{glose}
\pfra{aller sans savoir où, sans direction}
\end{glose}
\end{sous-entrée}
\newline
\begin{sous-entrée}{thumènõ-hayu}{ⓔthu-menõⓝthumènõ-hayu}
\région{GOs}
\begin{glose}
\pfra{aller à contrecoeur, en se forçant}
\end{glose}
\end{sous-entrée}
\newline
\begin{sous-entrée}{thumènõ lòlò}{ⓔthu-menõⓝthumènõ lòlò}
\région{GOs}
\begin{glose}
\pfra{marcher au hasard, sans but}
\end{glose}
\end{sous-entrée}
\newline
\begin{sous-entrée}{thumènõ ka udu}{ⓔthu-menõⓝthumènõ ka udu}
\région{GOs}
\begin{glose}
\pfra{marcher en se baissant}
\end{glose}
\end{sous-entrée}
\newline
\begin{sous-entrée}{thumènõ du mu}{ⓔthu-menõⓝthumènõ du mu}
\région{GOs}
\begin{glose}
\pfra{marcher à reculons (lit. en bas derrière)}
\end{glose}
\end{sous-entrée}
\end{entrée}

\begin{entrée}{thu mõõxi}{}{ⓔthu mõõxi}
\région{GO}
(\domainesémantique{Actions liées aux plantes})
\classe{v}
\begin{glose}
\pfra{faire des boutures (lié à la notion de vie)}
\end{glose}
\newline
\begin{exemple}
\région{GO}
\textbf{\pnua{nu thu mõõxi}}
\pfra{je fais des boutures}
\end{exemple}
\end{entrée}

\begin{entrée}{thu mwêê}{}{ⓔthu mwêê}
\région{GOs BO}
(\domainesémantique{Vêtements, parure})
\classe{v}
\begin{glose}
\pfra{vêtir (se) ; habiller (s') ; apprêter (s')}
\end{glose}
\newline
\begin{exemple}
\région{GOs}
\textbf{\pnua{e thu mwêê Kaavwo}}
\pfra{Kaavwo s'habille}
\end{exemple}
\newline
\begin{exemple}
\région{GOs}
\textbf{\pnua{hõbwò ba-thu-mwêê}}
\pfra{de beaux vêtements}
\end{exemple}
\newline
\relationsémantique{Cf.}{\lien{ⓔmwêê}{mwêê}}
\glosecourte{chapeau}
\newline
\relationsémantique{Cf.}{\lien{}{udale [GOs]}}
\glosecourte{s'habiller}
\end{entrée}

\begin{entrée}{thu mwêêxa}{}{ⓔthu mwêêxa}
\région{GOs}
(\domainesémantique{Vêtements, parure})
\classe{v}
\begin{glose}
\pfra{apprêter (s'); préparer (se) (corps : habits et maquillage)}
\end{glose}
\begin{glose}
\pfra{parer (se); se vêtir}
\end{glose}
\newline
\begin{exemple}
\région{GOs}
\textbf{\pnua{la thu mwêêxa}}
\pfra{ils s'apprêtent}
\end{exemple}
\newline
\relationsémantique{Cf.}{\lien{ⓔthu mwêê}{thu mwêê}}
\end{entrée}

\begin{entrée}{thu paa}{}{ⓔthu paa}
\région{GOs}
(\domainesémantique{Guerre})
\classe{v}
\begin{glose}
\pfra{faire la guerre}
\end{glose}
\newline
\begin{exemple}
\région{GOs}
\textbf{\pnua{li pe-thu-paa lie pwe-meevwu}}
\pfra{les deux clans se font la guerre}
\end{exemple}
\end{entrée}

\begin{entrée}{thu pi}{}{ⓔthu pi}
\région{PA BO}
(\domainesémantique{Fonctions naturelles des animaux})
\classe{v}
\begin{glose}
\pfra{pondre (lit. faire la coquille) [Corne]}
\end{glose}
\end{entrée}

\begin{entrée}{thu phu}{}{ⓔthu phu}
\région{GOs}
(\domainesémantique{Types de maison, architecture de la maison})
\classe{nom}
\begin{glose}
\pfra{mettre la première rangée de paille au bord du toit}
\end{glose}
\newline
\begin{exemple}
\textbf{\pnua{e thu phu}}
\pfra{il met la première rangée de paille}
\end{exemple}
\end{entrée}

\begin{entrée}{thu pwaalu}{}{ⓔthu pwaalu}
\région{GOs BO [Corne]}
(\domainesémantique{Relations et interaction sociales})
\classe{v ; n}
\begin{glose}
\pfra{respect ; respecter}
\end{glose}
\end{entrée}

\begin{entrée}{thu pwalu}{}{ⓔthu pwalu}
\région{PA BO [Corne]}
(\domainesémantique{Relations et interaction sociales})
\classe{v}
\begin{glose}
\pfra{faire une fête}
\end{glose}
\end{entrée}

\begin{entrée}{thu phwa}{}{ⓔthu phwa}
\région{GOs PA BO}
(\domainesémantique{Cultures, techniques, boutures})
\classe{v}
\begin{glose}
\pfra{faire un trou (pour les cultures, etc.)}
\end{glose}
\end{entrée}

\begin{entrée}{thu-phwãã}{}{ⓔthu-phwãã}
\région{GOs PA BO}
(\domainesémantique{Cultures, techniques, boutures})
\classe{v}
\begin{glose}
\pfra{cultiver ; faire un champ}
\end{glose}
\newline
\begin{sous-entrée}{mhenõ thu-phwãã}{ⓔthu-phwããⓝmhenõ thu-phwãã}
\région{GOs}
\begin{glose}
\pfra{un champ cultivé}
\end{glose}
\end{sous-entrée}
\newline
\begin{sous-entrée}{mhenõ thu-phwã}{ⓔthu-phwããⓝmhenõ thu-phwã}
\région{BO}
\begin{glose}
\pfra{un champ cultivé [BM]}
\end{glose}
\end{sous-entrée}
\end{entrée}

\begin{entrée}{thu-tãî}{}{ⓔthu-tãî}
\région{PA}
(\domainesémantique{Vêtements, parure})
\classe{v}
\begin{glose}
\pfra{habiller (s') ; vêtir (se)}
\end{glose}
\newline
\begin{exemple}
\textbf{\pnua{i thu-tãî Kaavo}}
\pfra{Kaavo s'habille}
\end{exemple}
\end{entrée}

\begin{entrée}{thuu}{1}{ⓔthuuⓗ1}
\région{GOs BO}
(\domainesémantique{Santé, maladie})
\classe{v}
\begin{glose}
\pfra{gale}
\end{glose}
\begin{glose}
\pfra{boutons sur la figure (avoir des)}
\end{glose}
\begin{glose}
\pfra{maladie des féculents (avoir une) (plaques sur la tête)}
\end{glose}
\newline
\begin{exemple}
\textbf{\pnua{e thuu kui}}
\pfra{l'igname a une maladie}
\end{exemple}
\newline
\begin{exemple}
\textbf{\pnua{e thuu bwaa-je}}
\pfra{il a des plaques sur la tête}
\end{exemple}
\end{entrée}

\begin{entrée}{thuu}{2}{ⓔthuuⓗ2}
\région{PA}
\variante{%
thuvwu
\région{GO(s)}}
(\domainesémantique{Réfléchi})
\classe{RFLX}
\begin{glose}
\pfra{réfléchi (du sujet, agent)}
\end{glose}
\newline
\begin{exemple}
\région{PA}
\textbf{\pnua{i rau thuu theei-je (x)o hele}}
\pfra{il s'est frappé avec un couteau}
\end{exemple}
\newline
\begin{exemple}
\région{PA}
\textbf{\pnua{i theei-je xo hele}}
\pfra{il l'a frappé avec un couteau, il lui a donné un coup de couteau}
\end{exemple}
\newline
\begin{exemple}
\région{PA}
\textbf{\pnua{nu thuu hããxa}}
\pfra{je me fais peur}
\end{exemple}
\end{entrée}

\begin{entrée}{thuvwu}{}{ⓔthuvwu}
\région{GOs}
\variante{%
thuu
\région{PA}, 
tuu
\région{BO}}
(\domainesémantique{Réfléchi})
\classe{RFLX}
\begin{glose}
\pfra{réfléchi (du sujet, agent)}
\end{glose}
\newline
\begin{exemple}
\région{GOs}
\textbf{\pnua{e thuvwu thrae pu-bwaa-je}}
\pfra{il s'est rasé les cheveux}
\end{exemple}
\newline
\begin{exemple}
\région{GOs}
\textbf{\pnua{e thuvwu kòòli hii-je xo dubwo}}
\pfra{il s'est piqué la main avec l'aiguille}
\end{exemple}
\newline
\begin{exemple}
\région{GOs}
\textbf{\pnua{e thuvwu zòi hii-je}}
\pfra{il s'est coupé le bras ((in)volontairement, mais voir draa pune pour 'volontairement')}
\end{exemple}
\newline
\begin{exemple}
\région{GOs}
\textbf{\pnua{e thuvwu phai nò}}
\pfra{il a cuit le poisson pour lui-même}
\end{exemple}
\newline
\begin{exemple}
\région{GOs}
\textbf{\pnua{e thuvwu vhaa ui-je cai Kaavo}}
\pfra{elle parle d'elle-même à Kaavo}
\end{exemple}
\newline
\begin{exemple}
\région{GOs}
\textbf{\pnua{e thuvwu vhaa ui-je xo Hiixe cai Kaavo}}
\pfra{Hiixe parle d'elle-même à Kaavo}
\end{exemple}
\newline
\begin{exemple}
\région{GOs [*e thuvwu tròròwuu-je]}
\textbf{\pnua{e thuvwu tròròwuu ui je (ou) e thuvwu tròròwuu}}
\pfra{il est content de lui-même}
\end{exemple}
\newline
\begin{exemple}
\région{GOs}
\textbf{\pnua{e thuvwu tròròwuu-je ui nye ẽnõ ã}}
\pfra{il est content de cet enfant}
\end{exemple}
\newline
\begin{exemple}
\région{PA}
\textbf{\pnua{i ra u/ra o thuu phao i je jigel}}
\pfra{il s'est tiré un coup de fusil sur lui-même}
\end{exemple}
\end{entrée}

\begin{entrée}{thuvwu zòò}{}{ⓔthuvwu zòò}
\formephonétique{'tʰuvwu 'ðɔː}
\région{GOs}
(\domainesémantique{Réfléchi})
\classe{v.RFLX}
\begin{glose}
\pfra{couper (se)}
\end{glose}
\newline
\begin{exemple}
\textbf{\pnua{nu thuvwu zòi-nu}}
\pfra{je me suis coupé}
\end{exemple}
\newline
\note{zòi (v.t.)}{grammaire}{}
\end{entrée}

\newpage

\lettrine{tr
(variante de GOs)}\begin{entrée}{trabwa}{}{ⓔtrabwa}
\formephonétique{ʈabwa}
\région{GOs}
\variante{%
tabwa
\région{BO PA}}
\classe{v}
\newline
\sens{1}
(\domainesémantique{Préfixes et verbes de position})
\begin{glose}
\pfra{asseoir (s') ; assis}
\end{glose}
\begin{glose}
\pfra{percher (se) ; perché}
\end{glose}
\newline
\begin{sous-entrée}{mhènõ-trabwa}{ⓔtrabwaⓢ1ⓝmhènõ-trabwa}
\région{GOs}
\begin{glose}
\pfra{chaise, siège}
\end{glose}
\end{sous-entrée}
\newline
\begin{sous-entrée}{mhènõ-tabwa}{ⓔtrabwaⓢ1ⓝmhènõ-tabwa}
\région{BO}
\begin{glose}
\pfra{chaise, siège}
\end{glose}
\end{sous-entrée}
\newline
\begin{sous-entrée}{we-tabwa [BO]}{ⓔtrabwaⓢ1ⓝwe-tabwa [BO]}
\begin{glose}
\pfra{eau stagnante}
\end{glose}
\end{sous-entrée}
\newline
\sens{2}
(\domainesémantique{Verbes de mouvement})
\begin{glose}
\pfra{poser (se)}
\end{glose}
\begin{glose}
\pfra{atterrir ; toucher terre}
\end{glose}
\newline
\sens{3}
(\domainesémantique{Fonctions naturelles des animaux})
\begin{glose}
\pfra{couver (oeufs)}
\end{glose}
\newline
\note{tre-}{grammaire}{en composition}
\end{entrée}

\begin{entrée}{trabwa mhwããnu}{}{ⓔtrabwa mhwããnu}
\région{GOs}
\variante{%
tabwa mhwããnu
\région{BO}}
(\domainesémantique{Découpage du temps})
\classe{nom}
\begin{glose}
\pfra{nouvelle lune}
\end{glose}
\end{entrée}

\begin{entrée}{tragòò}{}{ⓔtragòò}
\région{GOs}
\variante{%
taagò
\région{BO}}
(\domainesémantique{Corps humain
, Cours de la vie})
\classe{nom}
\begin{glose}
\pfra{circoncision ; subincision (moment où l'on donnait le 'bagayou' au garçon)}
\end{glose}
\newline
\relationsémantique{Cf.}{\lien{ⓔgòⓗ1}{gò}}
\glosecourte{couteau à subincision}
\end{entrée}

\begin{entrée}{trale}{}{ⓔtrale}
\formephonétique{'ʈale}
\région{GOs}
\variante{%
tali
\région{PA}}
(\domainesémantique{Actions liées aux plantes})
\classe{v}
\begin{glose}
\pfra{cueillir un fruit qui n'est pas mûr}
\end{glose}
\begin{glose}
\pfra{gauler ; taper pour faire tomber (fruit)}
\end{glose}
\newline
\relationsémantique{Cf.}{\lien{ⓔthraⓗ1}{thra}}
\end{entrée}

\begin{entrée}{travwa}{}{ⓔtravwa}
\formephonétique{ʈaβa}
\région{GOs}
\variante{%
trapa
\région{GO(s)}, 
tavang
\région{PA}}
(\domainesémantique{Tabac, actions liées au tabac})
\classe{nom}
\begin{glose}
\pfra{tabac ; cigarette}
\end{glose}
\end{entrée}

\begin{entrée}{tre}{}{ⓔtre}
\région{GOs}
\variante{%
tee, te
\région{PA BO}}
(\domainesémantique{Contraste})
\classe{PRE-VB.}
\begin{glose}
\pfra{contraste (entre deux agents, deux moments, deux actions dont l'une est antérieure)}
\end{glose}
\begin{glose}
\pfra{déjà ; état résultant}
\end{glose}
\newline
\begin{exemple}
\région{GOs}
\textbf{\pnua{li za u tre kha nooli}}
\pfra{ils l'ont déjà vue en se déplaçant}
\end{exemple}
\newline
\begin{exemple}
\région{GOs}
\textbf{\pnua{la tree threi mãe hãbu vwo la yaaze mwã}}
\pfra{ils ont coupé la paille avant de couvrir la maison}
\end{exemple}
\newline
\begin{exemple}
\région{GOs}
\textbf{\pnua{nu tree ne}}
\pfra{je vais le faire (pendant que tu fais autre chose)}
\end{exemple}
\newline
\begin{exemple}
\région{GOs}
\textbf{\pnua{jö tree yu (è)nè, ma nu a pwe}}
\pfra{reste ici car je vais à la pêche}
\end{exemple}
\newline
\begin{exemple}
\région{GOs}
\textbf{\pnua{ezoma nu tree pujo, na jö a pwe}}
\pfra{je vais faire la cuisine, quand tu iras pêcher}
\end{exemple}
\newline
\begin{exemple}
\région{PA}
\textbf{\pnua{co tee a}}
\pfra{pars avant}
\end{exemple}
\newline
\begin{exemple}
\région{PA}
\textbf{\pnua{co tee-vwun}}
\pfra{commence avant}
\end{exemple}
\newline
\begin{exemple}
\région{PA}
\textbf{\pnua{tee-hovwo}}
\pfra{mange avant}
\end{exemple}
\newline
\begin{exemple}
\région{BO}
\textbf{\pnua{na ju, te ju èna ma ja a khila xa ho-ã}}
\pfra{toi, tu vas rester là et nous allons chercher notre nourriture [BM]}
\end{exemple}
\newline
\relationsémantique{Cf.}{\lien{}{thaavwu ; teevwuun, thaavwun [PA]}}
\glosecourte{faire avant, commencer avant}
\end{entrée}

\begin{entrée}{tre-bwaalu}{}{ⓔtre-bwaalu}
\région{GOs PA}
(\domainesémantique{Préfixes et verbes de position})
\classe{v}
\begin{glose}
\pfra{assis en tailleur}
\end{glose}
\end{entrée}

\begin{entrée}{trebwalu}{}{ⓔtrebwalu}
\région{GOs}
\variante{%
tebwalu
\région{BO PA}}
(\domainesémantique{Topographie})
\classe{nom}
\begin{glose}
\pfra{petite élevation ; colline}
\end{glose}
\end{entrée}

\begin{entrée}{treçaaxo}{}{ⓔtreçaaxo}
\formephonétique{ʈe-'ʒaːɣo}
\région{GOs}
(\domainesémantique{Caractéristiques et propriétés des personnes})
\classe{v}
\begin{glose}
\pfra{calme ; paisible}
\end{glose}
\end{entrée}

\begin{entrée}{tre-çôô}{}{ⓔtre-çôô}
\formephonétique{ʈe-ʒõː}
\région{GOs}
(\domainesémantique{Préfixes et verbes de position})
\classe{v}
\begin{glose}
\pfra{pendu ; accroché (tre : marque d'état < déjà fait)}
\end{glose}
\newline
\relationsémantique{Cf.}{\lien{}{chôô}}
\glosecourte{suspendre, accrocher}
\end{entrée}

\begin{entrée}{tre-chaaçee}{}{ⓔtre-chaaçee}
\formephonétique{ʈe-cʰaːʒeː}
\région{GOs}
(\domainesémantique{Préfixes et verbes de position})
\classe{ADV}
\begin{glose}
\pfra{travers (de) ; bancal ; penché sur uncôté ; pas à niveau}
\end{glose}
\newline
\begin{exemple}
\textbf{\pnua{e tre-chaaçee mwa}}
\pfra{la maison penche d'un côté}
\end{exemple}
\end{entrée}

\begin{entrée}{tre-chamadi}{}{ⓔtre-chamadi}
\formephonétique{ʈe-cʰamandi}
\région{GOs}
(\domainesémantique{Discours, échanges verbaux})
\classe{v}
\begin{glose}
\pfra{prédire (l'avenir)}
\end{glose}
\end{entrée}

\begin{entrée}{tree}{1}{ⓔtreeⓗ1}
\formephonétique{ʈeː}
\région{GOs}
\variante{%
tèèn, tèn
\formephonétique{tɛːn, tɛn}
\région{PA BO}}
(\domainesémantique{Découpage du temps})
\classe{nom}
\begin{glose}
\pfra{jour ; journée}
\end{glose}
\newline
\begin{exemple}
\région{GOs}
\textbf{\pnua{e zo tree}}
\pfra{il fait beau}
\end{exemple}
\newline
\begin{exemple}
\région{GOs}
\textbf{\pnua{e thraa tre}}
\pfra{il fait mauvais temps}
\end{exemple}
\newline
\begin{exemple}
\région{GOs}
\textbf{\pnua{tree-monõ}}
\pfra{le jour du lendemain}
\end{exemple}
\newline
\begin{exemple}
\région{GOs}
\textbf{\pnua{õ tree}}
\pfra{tous les jours}
\end{exemple}
\newline
\begin{exemple}
\région{PA}
\textbf{\pnua{õ tèèn}}
\pfra{tous les jours}
\end{exemple}
\newline
\begin{exemple}
\région{BO}
\textbf{\pnua{we-ru tèèn}}
\pfra{2 jours}
\end{exemple}
\newline
\begin{exemple}
\région{BO}
\textbf{\pnua{u tèèn}}
\pfra{il fait jour}
\end{exemple}
\newline
\begin{exemple}
\région{PA}
\textbf{\pnua{we-nira tèèn}}
\pfra{combien de jours ?}
\end{exemple}
\newline
\étymologie{
\langue{POc}
\étymon{*ɖan(i), *daqani}}
\end{entrée}

\begin{entrée}{tree}{2}{ⓔtreeⓗ2}
\région{GOs}
\variante{%
tee
\région{PA}}
(\domainesémantique{Verbes de déplacement et moyens de déplacement})
\classe{v}
\begin{glose}
\pfra{conduire ; diriger (voiture, bateau, etc.)}
\end{glose}
\newline
\begin{exemple}
\région{PA}
\textbf{\pnua{i tee loto}}
\pfra{il conduit une voiture}
\end{exemple}
\end{entrée}

\begin{entrée}{tree}{3}{ⓔtreeⓗ3}
\région{GOs}
\variante{%
tee
\région{PA}}
(\domainesémantique{Préfixes sémantiques de position
, Préfixes et verbes de position})
\classe{PREF (position assise)}
\begin{glose}
\pfra{assis (faire)}
\end{glose}
\newline
\begin{exemple}
\région{GOs}
\textbf{\pnua{mi tree-bulu}}
\pfra{nous sommes assis ensemble}
\end{exemple}
\newline
\begin{exemple}
\région{GOs}
\textbf{\pnua{e tree-raa mwa !}}
\pfra{la maison est mal située}
\end{exemple}
\newline
\begin{exemple}
\région{PA}
\textbf{\pnua{co tee-nenèm !}}
\pfra{reste assis tranquille (à un enfant)}
\end{exemple}
\end{entrée}

\begin{entrée}{tre-e}{}{ⓔtre-e}
\formephonétique{ʈe.e}
\région{GOs}
(\domainesémantique{Préfixes et verbes de position})
\classe{v}
\begin{glose}
\pfra{assis en tenant qqch dans les bras}
\end{glose}
\newline
\begin{exemple}
\textbf{\pnua{e tre-e ẽnõ}}
\pfra{il est assise avec l'enfant dans ses bras}
\end{exemple}
\newline
\relationsémantique{Cf.}{\lien{}{trabwa [GO], tabwa [PA]}}
\end{entrée}

\begin{entrée}{tree-çãnã}{}{ⓔtree-çãnã}
\formephonétique{ʈeː-'ʒɛ̃ɳɛ̃}
\région{GOs}
\variante{%
teecãnã
\région{WEM}}
(\domainesémantique{Fonctions naturelles humaines})
\classe{v}
\begin{glose}
\pfra{reprendre son souffle ; reprendre haleine}
\end{glose}
\begin{glose}
\pfra{reposer (se)}
\end{glose}
\begin{glose}
\pfra{souffler}
\end{glose}
\newline
\relationsémantique{Cf.}{\lien{ⓔchãnãⓗ1}{chãnã}}
\glosecourte{se reposer, respirer}
\end{entrée}

\begin{entrée}{tree-çimwî}{}{ⓔtree-çimwî}
\formephonétique{ʈeː-'ʒimwi}
\région{GOs}
\variante{%
tee-cimwî, tee-jimwî, tee-yimwî
\région{PABO}}
(\domainesémantique{Mouvements ou actions faits avec le corps, les bras, les mains, les pieds})
\classe{v}
\begin{glose}
\pfra{attraper ; saisir}
\end{glose}
\begin{glose}
\pfra{retenir ; serrer}
\end{glose}
\begin{glose}
\pfra{mémoriser ; retenir}
\end{glose}
\newline
\begin{exemple}
\région{PA}
\textbf{\pnua{li pe-cimwî hi, li pe-cimwî hi-li}}
\pfra{ils se serrent la main}
\end{exemple}
\newline
\begin{exemple}
\région{BO}
\textbf{\pnua{waya me tee-yimwî pwaji ?}}
\pfra{comment attrape-t-on les crabes ?}
\end{exemple}
\newline
\relationsémantique{Cf.}{\lien{ⓔcimwî}{cimwî}}
\glosecourte{tenir, saisir}
\end{entrée}

\begin{entrée}{tree hêbu}{}{ⓔtree hêbu}
\région{GOs}
(\domainesémantique{Adverbes déictiques de temps})
\classe{ADV}
\begin{glose}
\pfra{avant-hier (lit. le jour d'avant)}
\end{glose}
\end{entrée}

\begin{entrée}{tree-kiyai}{}{ⓔtree-kiyai}
\région{GOs}
\variante{%
tre-xiyai
\région{GOs}, 
tee-kiyai
\région{PA}}
(\domainesémantique{Préfixes et verbes de position})
\classe{nom}
\begin{glose}
\pfra{asseoir (s') auprès du feu de bois pour se réchauffer}
\end{glose}
\newline
\begin{exemple}
\textbf{\pnua{a tree kiyai !}}
\pfra{va t'asseoir près du feu (pour te réchauffer)}
\end{exemple}
\newline
\morphologie{tre-khini-yaai (assis-chauffer-feu)}
\end{entrée}

\begin{entrée}{tree-ku}{}{ⓔtree-ku}
\région{GO}
(\domainesémantique{Préfixes sémantiques de position})
\classe{v}
\begin{glose}
\pfra{rester}
\end{glose}
\newline
\begin{exemple}
\région{GO}
\textbf{\pnua{na jö tree-ku hõbwò-nu}}
\pfra{et toi tu resteras là à m'attendre}
\end{exemple}
\end{entrée}

\begin{entrée}{tree-poxe}{}{ⓔtree-poxe}
\région{PA}
(\domainesémantique{Verbes de mouvement})
\classe{v}
\begin{glose}
\pfra{assis ensemble}
\end{glose}
\end{entrée}

\begin{entrée}{trèèzu}{}{ⓔtrèèzu}
\région{GOs}
\variante{%
tèèzo
\région{PA}}
(\domainesémantique{Topographie})
\classe{v.stat. ; n}
\begin{glose}
\pfra{aplani ; plat ; plaine}
\end{glose}
\newline
\begin{exemple}
\région{PA BO}
\textbf{\pnua{ni tèèzo}}
\pfra{dans la plaine}
\end{exemple}
\end{entrée}

\begin{entrée}{tre-go}{}{ⓔtre-go}
\région{GOs}
(\domainesémantique{Préfixes et verbes de position})
\classe{v}
\begin{glose}
\pfra{assis en rêvant}
\end{glose}
\newline
\begin{exemple}
\textbf{\pnua{e tre-go}}
\pfra{il est assis en étant dans les nuages}
\end{exemple}
\end{entrée}

\begin{entrée}{tre-hû}{}{ⓔtre-hû}
\région{GOs}
(\domainesémantique{Préfixes et verbes de position})
\classe{v}
\begin{glose}
\pfra{assis sans parler (asseoir-muet)}
\end{glose}
\end{entrée}

\begin{entrée}{trehuni}{}{ⓔtrehuni}
\formephonétique{ʈehuɳi}
\région{GOs}
(\domainesémantique{Religion, représentations religieuses})
\classe{nom}
\begin{glose}
\pfra{foi}
\end{glose}
\end{entrée}

\begin{entrée}{tre-kea}{}{ⓔtre-kea}
\région{GOs}
\variante{%
tre-xea
\région{GO(s)}}
(\domainesémantique{Préfixes et verbes de position})
\classe{v}
\begin{glose}
\pfra{assis adossé à qqch}
\end{glose}
\end{entrée}

\begin{entrée}{tre-kuçaaxo}{}{ⓔtre-kuçaaxo}
\région{GOs}
\variante{%
tee-kujaaxo
\région{PA}}
(\domainesémantique{Préfixes et verbes de position})
\classe{v}
\begin{glose}
\pfra{assis en cachette, à l'affût}
\end{glose}
\end{entrée}

\begin{entrée}{tre-kue}{}{ⓔtre-kue}
\région{GOs}
(\domainesémantique{Sentiments})
\classe{v}
\begin{glose}
\pfra{jaloux (être) ; jalouser}
\end{glose}
\newline
\begin{exemple}
\textbf{\pnua{e tre-kuele loto i nu}}
\pfra{il est jaloux de ma voiture}
\end{exemple}
\newline
\begin{sous-entrée}{a-tre-kue}{ⓔtre-kueⓝa-tre-kue}
\begin{glose}
\pfra{un jaloux}
\end{glose}
\end{sous-entrée}
\end{entrée}

\begin{entrée}{tre-khõbwe}{}{ⓔtre-khõbwe}
\région{GOs}
\variante{%
te-kôbwe
\région{BO}}
(\domainesémantique{Relations et interaction sociales
, Discours, échanges verbaux})
\classe{v}
\begin{glose}
\pfra{annoncer ; prévenir ; promettre ; déclarer ; aviser}
\end{glose}
\newline
\relationsémantique{Cf.}{\lien{ⓔtre-chamadi}{tre-chamadi}}
\glosecourte{prédire}
\end{entrée}

\begin{entrée}{trenene}{}{ⓔtrenene}
\formephonétique{ʈeɳeɳe}
\région{GOs}
\variante{%
tee-nenem
\région{PA BO}}
(\domainesémantique{Manière de faire l’action : verbes et adverbes de manière})
\classe{v ; n}
\begin{glose}
\pfra{paix ; tranquille ; paisible}
\end{glose}
\begin{glose}
\pfra{rester tranquille ; reposer en paix}
\end{glose}
\newline
\begin{exemple}
\textbf{\pnua{mo za trenene na koi ẽnõ mãlò}}
\pfra{on est tranquille quand les enfants ne sont pas là}
\end{exemple}
\end{entrée}

\begin{entrée}{tre-nõnõmi}{}{ⓔtre-nõnõmi}
\formephonétique{ʈeɳɔ̃ɳɔ̃mi}
\région{GOs}
\variante{%
tee-nònòmi
\région{PA BO}}
(\domainesémantique{Fonctions intellectuelles})
\classe{v}
\begin{glose}
\pfra{prévoir}
\end{glose}
\begin{glose}
\pfra{penser à ; réfléchir}
\end{glose}
\newline
\begin{exemple}
\région{PA}
\textbf{\pnua{co tee-nònòmi da ?}}
\pfra{à quoi penses-tu assis ? (sans rien dire)}
\end{exemple}
\end{entrée}

\begin{entrée}{tre-pavwã}{}{ⓔtre-pavwã}
\formephonétique{ʈepaβa}
\région{GOs}
\variante{%
tre-pavhã
\région{GO(s)BO PA}}
(\domainesémantique{Fonctions intellectuelles})
\classe{v ; n}
\begin{glose}
\pfra{espoir ; confiance (avoir) ; espérer}
\end{glose}
\newline
\relationsémantique{Cf.}{\lien{ⓔpavwã}{pavwã}}
\glosecourte{confiance}
\end{entrée}

\begin{entrée}{tre-paxo}{}{ⓔtre-paxo}
\région{GOs}
\variante{%
tee-vhaxol, tee-waxol
\région{PA BO}}
(\domainesémantique{Préfixes et verbes de position})
\classe{v}
\begin{glose}
\pfra{accroupi ; s'accroupir}
\end{glose}
\end{entrée}

\begin{entrée}{tre-pwalee kò-ã}{}{ⓔtre-pwalee kò-ã}
\région{GOs}
(\domainesémantique{Préfixes et verbes de position})
\classe{v}
\begin{glose}
\pfra{assis les jambes allongées (lit. assis-étaler jambes-nos)}
\end{glose}
\end{entrée}

\begin{entrée}{tre-raa}{}{ⓔtre-raa}
\région{GOs}
(\domainesémantique{Description des objets, formes, consistance, taille})
\classe{v}
\begin{glose}
\pfra{pas stable (route, chemin)}
\end{glose}
\begin{glose}
\pfra{mauvais ; dangereux}
\end{glose}
\begin{glose}
\pfra{dangereux}
\end{glose}
\newline
\morphologie{tre-thraa : assis mauvais}
\end{entrée}

\begin{entrée}{tre-thibu}{}{ⓔtre-thibu}
\formephonétique{ʈe-tʰibu}
\région{GOs}
\variante{%
tre-zibu
\région{GO(s)}}
(\domainesémantique{Verbes de mouvement
, Préfixes et verbes de position})
\classe{v}
\begin{glose}
\pfra{agenouiller (s') ; agenouillé}
\end{glose}
\end{entrée}

\begin{entrée}{tretrabwau}{}{ⓔtretrabwau}
\formephonétique{ʈeɽa'bwa.u}
\région{GOs}
\variante{%
terebwau
\région{PA BO}}
(\domainesémantique{Description des objets, formes, consistance, taille})
\classe{v.stat.}
\begin{glose}
\pfra{rond}
\end{glose}
\end{entrée}

\begin{entrée}{tre-xinãã}{}{ⓔtre-xinãã}
\formephonétique{ʈe-'ɣiɳɛ̃ː}
\région{GOs}
\variante{%
tre-khini-a
\région{GO(s)}, 
tee-khini-al
\région{PA}}
(\domainesémantique{Soins du corps})
\classe{v}
\begin{glose}
\pfra{chauffer assis au soleil (se)}
\end{glose}
\begin{glose}
\pfra{sécher au soleil (se)}
\end{glose}
\end{entrée}

\begin{entrée}{tre-xiyai}{}{ⓔtre-xiyai}
\région{GOs PA}
\variante{%
tre-kiyai, te-'xiyai
\région{GO(s)}, 
tee-kiyai
\région{PA}}
(\domainesémantique{Feu : objets et actions liés au feu})
\classe{v}
\begin{glose}
\pfra{chauffer (se) assis près du feu}
\end{glose}
\newline
\morphologie{tre-khini-yaai (assis-chauffer-feu)}
\end{entrée}

\begin{entrée}{tre-yuu}{}{ⓔtre-yuu}
\région{GOs}
\variante{%
te-yu
\région{PA BO}}
(\domainesémantique{Préfixes et verbes de position})
\classe{v}
\begin{glose}
\pfra{rester ; demeurer}
\end{glose}
\newline
\begin{exemple}
\région{BO}
\textbf{\pnua{te-yu bulu}}
\pfra{rester ensemble}
\end{exemple}
\end{entrée}

\begin{entrée}{trezô}{}{ⓔtrezô}
\région{GOs}
(\domainesémantique{Description des objets, formes, consistance, taille})
\classe{v}
\begin{glose}
\pfra{fermé (tout seul: porte, fenêtre, etc.)}
\end{glose}
\newline
\relationsémantique{Cf.}{\lien{ⓔthôni}{thôni}}
\glosecourte{fermer}
\end{entrée}

\begin{entrée}{trèzoo}{}{ⓔtrèzoo}
\région{GOs}
(\domainesémantique{Verbes de mouvement})
\classe{v}
\begin{glose}
\pfra{échouer (s')}
\end{glose}
\newline
\begin{exemple}
\région{GOs}
\textbf{\pnua{e trèzoo wô na bwa yaò, na bwa ô}}
\pfra{le bateau s'est échoué sur le corail, sur le sable}
\end{exemple}
\newline
\begin{exemple}
\région{BO}
\textbf{\pnua{i too wòny na bwa kharo}}
\pfra{le bateau s'est échoué sur le corail}
\end{exemple}
\end{entrée}

\begin{entrée}{tri}{}{ⓔtri}
\formephonétique{ʈi}
\région{GOs}
(\domainesémantique{Aliments, alimentation})
\classe{nom}
\begin{glose}
\pfra{thé}
\end{glose}
\newline
\emprunt{tea (GB)}
\end{entrée}

\begin{entrée}{trile}{}{ⓔtrile}
\région{GOs WEM}
\variante{%
tilèèng
\région{PA BO}}
(\domainesémantique{Oiseaux})
\classe{nom}
\begin{glose}
\pfra{"suceur" oiseau ; Méliphage à oreillon gris}
\end{glose}
\nomscientifique{Lichmera incana (Méliphagidés)}
\end{entrée}

\begin{entrée}{trilòò}{}{ⓔtrilòò}
\région{GOs}
\variante{%
tilòò
\région{PA BO}}
(\domainesémantique{Fonctions intellectuelles})
\classe{v ; n}
\begin{glose}
\pfra{demander qqch ; demande ; requête}
\end{glose}
\newline
\begin{exemple}
\région{GOs}
\textbf{\pnua{e trilòò kêê-je xa dili}}
\pfra{il demande à son père de la terre [xa : marque d'objet indirect]}
\end{exemple}
\newline
\begin{exemple}
\textbf{\pnua{e trilòò dili xo Pwayili ui kêê-je [GOs - courant]}}
\pfra{Pwayili demande de la terre à son père}
\end{exemple}
\newline
\begin{exemple}
\textbf{\pnua{e trilò-nu}}
\pfra{il m'a demandé}
\end{exemple}
\newline
\begin{exemple}
\région{BO}
\textbf{\pnua{i tilòò kôbwe nu ru a}}
\pfra{il a demandé si je partirai}
\end{exemple}
\newline
\begin{exemple}
\région{PA}
\textbf{\pnua{tilòò-m}}
\pfra{ta demande}
\end{exemple}
\newline
\begin{sous-entrée}{trilò porô}{ⓔtrilòòⓝtrilò porô}
\région{GOs}
\begin{glose}
\pfra{demander pardon}
\end{glose}
\end{sous-entrée}
\newline
\begin{sous-entrée}{tilòò poròm}{ⓔtrilòòⓝtilòò poròm}
\région{BO}
\begin{glose}
\pfra{demander pardon}
\end{glose}
\end{sous-entrée}
\newline
\begin{sous-entrée}{ki-tilò}{ⓔtrilòòⓝki-tilò}
\région{PA}
\begin{glose}
\pfra{emprunter (lit. demander un peu, pour un moment)}
\end{glose}
\end{sous-entrée}
\end{entrée}

\begin{entrée}{trirê}{}{ⓔtrirê}
\formephonétique{ʈiɽɛ̃, ʈiʈɛ̃}
\région{GOs}
(\domainesémantique{Fonctions naturelles humaines})
\classe{nom}
\begin{glose}
\pfra{sueur ; transpiration}
\end{glose}
\newline
\begin{exemple}
\textbf{\pnua{nu a-trirê}}
\pfra{je suis en sueur}
\end{exemple}
\newline
\begin{sous-entrée}{hai trirê}{ⓔtrirêⓝhai trirê}
\begin{glose}
\pfra{transpirer}
\end{glose}
\end{sous-entrée}
\end{entrée}

\begin{entrée}{trivwi}{}{ⓔtrivwi}
\formephonétique{ʈivwi}
\région{GOs}
\variante{%
tripwi
\formephonétique{ʈipwi}, 
tiwi
\région{BO [BM]}, 
tipui
\région{BO (Corne)}}
(\domainesémantique{Mouvements ou actions faits avec le corps, les bras, les mains, les pieds})
\classe{v}
\begin{glose}
\pfra{traîner par terre ; tirer}
\end{glose}
\end{entrée}

\begin{entrée}{trò}{1}{ⓔtròⓗ1}
\formephonétique{ʈɔ}
\région{GOs}
\variante{%
tòn, thòn
\région{WEM PA BO}}
(\domainesémantique{Lumière et obscurité
, Découpage du temps})
\classe{v ; n}
\begin{glose}
\pfra{nuit ; obscurité}
\end{glose}
\begin{glose}
\pfra{nuit (faire)}
\end{glose}
\newline
\begin{exemple}
\région{BO}
\textbf{\pnua{u ru tòn}}
\pfra{il va faire nuit}
\end{exemple}
\newline
\begin{exemple}
\région{BO}
\textbf{\pnua{u tòn}}
\pfra{il fait nuit}
\end{exemple}
\end{entrée}

\begin{entrée}{trò}{2}{ⓔtròⓗ2}
\formephonétique{ʈɔ}
\région{GOs}
\variante{%
tò
\région{BO}}
(\domainesémantique{Fonctions naturelles humaines})
\classe{v.i.}
\begin{glose}
\pfra{entendre}
\end{glose}
\begin{glose}
\pfra{sentir}
\end{glose}
\newline
\étymologie{
\langue{POc}
\étymon{*ɖoŋoR}}
\end{entrée}

\begin{entrée}{trö}{}{ⓔtrö}
\région{GOs}
(\domainesémantique{Description des objets, formes, consistance, taille})
\classe{v}
\begin{glose}
\pfra{fissuré ; crevassé ; crevasser (se) ; fissurer (se)}
\end{glose}
\newline
\begin{exemple}
\textbf{\pnua{e trö dili}}
\pfra{la terre est crevassée, se crevasse}
\end{exemple}
\end{entrée}

\begin{entrée}{tröi}{}{ⓔtröi}
\région{GOs}
\région{GOs}
\variante{%
trööi
, 
tui
\région{WEM WE PA BO}}
(\domainesémantique{Actions liées aux éléments (liquide, fumée)})
\classe{v}
\begin{glose}
\pfra{puiser de l'eau}
\end{glose}
\newline
\begin{exemple}
\région{GOs}
\textbf{\pnua{e tröi we}}
\pfra{il puise de l'eau}
\end{exemple}
\newline
\begin{exemple}
\région{BO}
\textbf{\pnua{a tui-ni we}}
\pfra{va chercher de l'eau}
\end{exemple}
\newline
\begin{sous-entrée}{ba-tröi}{ⓔtröiⓝba-tröi}
\région{GOs}
\begin{glose}
\pfra{écope}
\end{glose}
\end{sous-entrée}
\newline
\begin{sous-entrée}{ba-tui; ba-rui}{ⓔtröiⓝba-tui; ba-rui}
\région{PA BO}
\begin{glose}
\pfra{cuiller}
\end{glose}
\newline
\relationsémantique{Cf.}{\lien{}{khee we [GOs]}}
\glosecourte{écoper}
\end{sous-entrée}
\end{entrée}

\begin{entrée}{tromwa}{}{ⓔtromwa}
\région{GOs}
\variante{%
toma
\région{GO(s)}}
(\domainesémantique{Aliments, alimentation})
\classe{nom}
\begin{glose}
\pfra{tomate}
\end{glose}
\newline
\emprunt{tomate (FR)}
\end{entrée}

\begin{entrée}{trõne}{}{ⓔtrõne}
\formephonétique{ʈɔ̃ɳe}
\région{GOs}
\variante{%
tõne
\région{BO PA}}
\classe{v.t.}
\newline
\sens{1}
(\domainesémantique{Fonctions naturelles humaines})
\begin{glose}
\pfra{entendre ; sentir}
\end{glose}
\newline
\begin{exemple}
\région{BO}
\textbf{\pnua{ka u nu tõne}}
\pfra{je n'ai pas entendu}
\end{exemple}
\newline
\sens{2}
(\domainesémantique{Fonctions naturelles humaines})
\begin{glose}
\pfra{sentir (odeur ou toucher) ; entendre}
\end{glose}
\newline
\begin{exemple}
\région{GO}
\textbf{\pnua{e trõne böö-je}}
\pfra{il sent son odeur}
\end{exemple}
\newline
\begin{exemple}
\région{PA}
\textbf{\pnua{tõne bo-n}}
\pfra{sentir son odeur}
\end{exemple}
\newline
\sens{3}
(\domainesémantique{Fonctions intellectuelles})
\begin{glose}
\pfra{comprendre}
\end{glose}
\newline
\begin{exemple}
\région{GO}
\textbf{\pnua{jö trõne kaamweni ?}}
\pfra{tu as compris ?}
\end{exemple}
\end{entrée}

\begin{entrée}{trõne bö}{}{ⓔtrõne bö}
\formephonétique{ʈɔ̃ɳe}
\région{GOs}
\variante{%
tone bo-n
\région{PA BO [Corne]}}
(\domainesémantique{Fonctions naturelles humaines})
\classe{v}
\begin{glose}
\pfra{sentir (une odeur)}
\end{glose}
\newline
\begin{exemple}
\région{PA}
\textbf{\pnua{i to bwon}}
\pfra{il sent une odeur}
\end{exemple}
\end{entrée}

\begin{entrée}{trõne kaamweni}{}{ⓔtrõne kaamweni}
\formephonétique{ʈɔ̃ɳe}
\région{GOs}
(\domainesémantique{Fonctions intellectuelles})
\classe{v}
\begin{glose}
\pfra{comprendre}
\end{glose}
\newline
\begin{exemple}
\région{GOs}
\textbf{\pnua{kavwo nu trõne kaamweni me-vhaa i la}}
\pfra{je ne comprends pas leur façon de parler}
\end{exemple}
\end{entrée}

\begin{entrée}{tròò}{}{ⓔtròò}
\formephonétique{ʈɔː}
\région{GOs WEM}
\variante{%
tòò
\région{BO}}
(\domainesémantique{Verbes d'action (en général)})
\classe{v.t.}
\begin{glose}
\pfra{trouver ; trouver (se) dans un état ; rencontrer}
\end{glose}
\newline
\begin{exemple}
\région{GO}
\textbf{\pnua{li tròò-la}}
\pfra{ils les trouvent}
\end{exemple}
\newline
\begin{exemple}
\région{GO}
\textbf{\pnua{jò ta tròò-nu}}
\pfra{vous êtes venu me voir}
\end{exemple}
\newline
\note{la forme tròò a un objet pronominal uniquement; v.t. tròòli + objet nominal}{grammaire}{}
\end{entrée}

\begin{entrée}{tròòli}{}{ⓔtròòli}
\formephonétique{ʈɔːli}
\région{GOs}
\variante{%
tooli
\région{PA}}
(\domainesémantique{Verbes d'action (en général)})
\classe{v.t.}
\begin{glose}
\pfra{trouver qqch}
\end{glose}
\newline
\begin{sous-entrée}{ka-tròòli}{ⓔtròòliⓝka-tròòli}
\région{GO}
\begin{glose}
\pfra{rencontrer par}
\end{glose}
\newline
\begin{exemple}
\région{GO}
\textbf{\pnua{nu tròòli mha}}
\pfra{je suis tombé malade}
\end{exemple}
\newline
\begin{exemple}
\région{GO}
\textbf{\pnua{e tròòli khabu}}
\pfra{il a attrapé froid}
\end{exemple}
\newline
\begin{exemple}
\région{GO}
\textbf{\pnua{nòme çö tròòli-xa, çö thomã-nu}}
\pfra{si tu en trouves, tu m'appelles}
\end{exemple}
\newline
\begin{exemple}
\région{GO}
\textbf{\pnua{nòme çö tròòli-vwo, çö thomã-nu}}
\pfra{si tu en trouves, tu m'appelles}
\end{exemple}
\newline
\begin{exemple}
\région{PA}
\textbf{\pnua{hã ra tòò-je}}
\pfra{nous l'avons trouvé/rencontré}
\end{exemple}
\end{sous-entrée}
\end{entrée}

\begin{entrée}{tròòli mha-mhwããnu}{}{ⓔtròòli mha-mhwããnu}
\région{GOs}
(\domainesémantique{Fonctions naturelles humaines})
\classe{v}
\begin{glose}
\pfra{menstruations (avoir ses)}
\end{glose}
\end{entrée}

\begin{entrée}{tròòli phwayuu}{}{ⓔtròòli phwayuu}
\région{GOs}
(\domainesémantique{Fonctions naturelles humaines})
\classe{v}
\begin{glose}
\pfra{menstruations (avoir ses)}
\end{glose}
\newline
\relationsémantique{Cf.}{\lien{ⓔmha-mhwããnuⓝtròòli mha-mhwããnu}{tròòli mha-mhwããnu}}
\glosecourte{avoir ses menstruations}
\end{entrée}

\begin{entrée}{trörö}{}{ⓔtrörö}
\formephonétique{ʈωʈω, ʈωɽω}
\région{GOs}
\variante{%
toro
\région{BO}}
(\domainesémantique{Reptiles})
\classe{nom}
\begin{glose}
\pfra{lézard (petit et marron, vit dans l'herbe)}
\end{glose}
\newline
\relationsémantique{Cf.}{\lien{ⓔmajo}{majo}}
\glosecourte{lézard}
\end{entrée}

\begin{entrée}{tròròvwuu}{}{ⓔtròròvwuu}
\formephonétique{ʈɔɽɔwuː}
\région{GOs}
\variante{%
tròròwuu
\région{GO(s)}}
(\domainesémantique{Sentiments})
\classe{v ; n}
\begin{glose}
\pfra{joie ; joyeux ; réjouir (se) ; content (être)}
\end{glose}
\newline
\begin{exemple}
\région{GOs}
\textbf{\pnua{e tròròwuu-je}}
\pfra{il est content de lui-même}
\end{exemple}
\newline
\begin{exemple}
\région{GOs}
\textbf{\pnua{e tròròwuu ui nye ẽnõ ã}}
\pfra{il est content de cet enfant}
\end{exemple}
\newline
\begin{exemple}
\région{GO}
\textbf{\pnua{e tròròwuu ãbaa-nu pexa pòi-je}}
\pfra{mon frère est content de son enfant (Doriane)}
\end{exemple}
\newline
\begin{exemple}
\région{GOs}
\textbf{\pnua{e tròròwuu pexa nye ẽnõ ã}}
\pfra{il est content de cet enfant}
\end{exemple}
\newline
\begin{exemple}
\région{GOs}
\textbf{\pnua{e tròròwuu pexa wôjô-je}}
\pfra{il est content de son bateau}
\end{exemple}
\newline
\begin{exemple}
\région{GOs}
\textbf{\pnua{e tròròwuu pexa mõõ-je}}
\pfra{il est content de sa maison}
\end{exemple}
\newline
\begin{exemple}
\région{GOs}
\textbf{\pnua{e za tròròwuu}}
\pfra{il en est content (il est content de cela)}
\end{exemple}
\end{entrée}

\begin{entrée}{tröxi}{}{ⓔtröxi}
\région{GOsWE}
\variante{%
tööxi
\région{PA Paita}}
(\domainesémantique{Outils
, Armes})
\classe{nom}
\begin{glose}
\pfra{hache (petite, en fer)}
\end{glose}
\begin{glose}
\pfra{tamioc ; fer}
\end{glose}
\newline
\begin{sous-entrée}{go-tröxi}{ⓔtröxiⓝgo-tröxi}
\begin{glose}
\pfra{bout de métal (lit. tronc de métal)}
\end{glose}
\end{sous-entrée}
\newline
\begin{sous-entrée}{jige tröxi [WE]}{ⓔtröxiⓝjige tröxi [WE]}
\begin{glose}
\pfra{fusil à longue portée}
\end{glose}
\end{sous-entrée}
\newline
\emprunt{toki (POLYN) (PPN *toki)}
\end{entrée}

\begin{entrée}{-tru}{}{ⓔ-tru}
\formephonétique{ʈu}
\région{GOs}
\variante{%
-ru
\région{PA BO}, 
-lu
}
(\domainesémantique{Numéraux cardinaux})
\classe{NUM}
\begin{glose}
\pfra{deux}
\end{glose}
\newline
\étymologie{
\langue{POc}
\étymon{*ɖua}}
\end{entrée}

\begin{entrée}{trûã}{}{ⓔtrûã}
\formephonétique{ʈũ.ɛ̃}
\région{GOs WEM}
\variante{%
thûã
\région{PA}, 
tûãn
\région{BO}}
\classe{v ; n}
\newline
\sens{1}
(\domainesémantique{Relations et interaction sociales})
\begin{glose}
\pfra{mentir ; mensonge}
\end{glose}
\begin{glose}
\pfra{jouer des tours}
\end{glose}
\newline
\begin{exemple}
\région{GOs}
\textbf{\pnua{nu trûã i je}}
\pfra{je lui ai joué un mauvais tour}
\end{exemple}
\newline
\sens{2}
(\domainesémantique{Comparaison})
\begin{glose}
\pfra{comparatif}
\end{glose}
\newline
\begin{exemple}
\région{GOs}
\textbf{\pnua{nu powonu trûã nai çö}}
\pfra{je suis un tout petit peu plus petit que toi}
\end{exemple}
\newline
\begin{exemple}
\région{GOs}
\textbf{\pnua{nu po pwawali trûã nai çö}}
\pfra{je suis un peu plus grand que toi}
\end{exemple}
\newline
\begin{exemple}
\région{GOs}
\textbf{\pnua{nu po whamã trûã nai çö}}
\pfra{je suis un peu plus vieux que toi}
\end{exemple}
\end{entrée}

\begin{entrée}{truãrôô}{}{ⓔtruãrôô}
\formephonétique{ʈu'ɛ̃ɽõː}
\région{GOs}
\variante{%
truãrô
\région{BO}, 
tuarôn
\région{PA}}
(\domainesémantique{Insectes})
\classe{nom}
\begin{glose}
\pfra{araignée}
\end{glose}
\newline
\begin{sous-entrée}{tha-truãrôô}{ⓔtruãrôôⓝtha-truãrôô}
\begin{glose}
\pfra{toile d'araignée}
\end{glose}
\end{sous-entrée}
\end{entrée}

\begin{entrée}{truçaabèlè}{}{ⓔtruçaabèlè}
\formephonétique{ʈu'dʒaːbɛlɛ, ʈu'ʒaːbɛlɛ}
\région{GOs}
\variante{%
thruucaabèlè
\région{vx (Haudricourt)}, 
tuyabèlèp
\région{PA}, 
tuyabèlè
\région{BO}}
(\domainesémantique{Phénomènes atmosphériques et naturels})
\classe{nom}
\begin{glose}
\pfra{arc-en-ciel}
\end{glose}
\end{entrée}

\begin{entrée}{truu}{}{ⓔtruu}
\région{GOs}
\variante{%
thuu
\région{PA}, 
tuu
\région{BO}}
(\domainesémantique{Pêche})
\classe{v}
\begin{glose}
\pfra{plonger (pour pêcher) ; faire de la plongée}
\end{glose}
\end{entrée}

\begin{entrée}{truuçi}{}{ⓔtruuçi}
\formephonétique{ʈuːdʒi, ʈuːʒi}
\région{GOs}
\variante{%
tuuyi
\formephonétique{tuːyi}
\région{PA}, 
truji
\région{BO}}
(\domainesémantique{Numéraux cardinaux})
\classe{NUM}
\begin{glose}
\pfra{dix}
\end{glose}
\newline
\begin{sous-entrée}{truuçi bwa pòxè}{ⓔtruuçiⓝtruuçi bwa pòxè}
\begin{glose}
\pfra{onze}
\end{glose}
\end{sous-entrée}
\newline
\begin{sous-entrée}{truuçi bwa pòtru / poru}{ⓔtruuçiⓝtruuçi bwa pòtru / poru}
\begin{glose}
\pfra{douze}
\end{glose}
\end{sous-entrée}
\newline
\begin{sous-entrée}{truuçi bwa pòko /pokon}{ⓔtruuçiⓝtruuçi bwa pòko /pokon}
\begin{glose}
\pfra{treize}
\end{glose}
\end{sous-entrée}
\newline
\begin{sous-entrée}{truuçi bwa pòpa /}{ⓔtruuçiⓝtruuçi bwa pòpa /}
\begin{glose}
\pfra{quatorze}
\end{glose}
\end{sous-entrée}
\newline
\begin{sous-entrée}{truuçi bwa pòni /ponim}{ⓔtruuçiⓝtruuçi bwa pòni /ponim}
\begin{glose}
\pfra{quinze}
\end{glose}
\end{sous-entrée}
\newline
\begin{sous-entrée}{truuçi bwa pòni-ma-xe}{ⓔtruuçiⓝtruuçi bwa pòni-ma-xe}
\begin{glose}
\pfra{seize}
\end{glose}
\end{sous-entrée}
\newline
\begin{sous-entrée}{truuçi bwa pòni-ma-dru / ma du}{ⓔtruuçiⓝtruuçi bwa pòni-ma-dru / ma du}
\begin{glose}
\pfra{dix-sept}
\end{glose}
\end{sous-entrée}
\newline
\begin{sous-entrée}{truuçi bwa pòni-ma-gò}{ⓔtruuçiⓝtruuçi bwa pòni-ma-gò}
\begin{glose}
\pfra{dix-huit}
\end{glose}
\end{sous-entrée}
\newline
\begin{sous-entrée}{truuçi bwa pòni-ma-ba}{ⓔtruuçiⓝtruuçi bwa pòni-ma-ba}
\begin{glose}
\pfra{dix-neuf}
\end{glose}
\end{sous-entrée}
\newline
\begin{sous-entrée}{tuuyi bwa pòxè}{ⓔtruuçiⓝtuuyi bwa pòxè}
\begin{glose}
\pfra{onze}
\end{glose}
\end{sous-entrée}
\newline
\begin{sous-entrée}{tuuyi bwa pòru}{ⓔtruuçiⓝtuuyi bwa pòru}
\begin{glose}
\pfra{douze}
\end{glose}
\end{sous-entrée}
\newline
\begin{sous-entrée}{tuuyi bwa pòkòn}{ⓔtruuçiⓝtuuyi bwa pòkòn}
\begin{glose}
\pfra{treize}
\end{glose}
\end{sous-entrée}
\newline
\begin{sous-entrée}{tuuyi bwa pòpa}{ⓔtruuçiⓝtuuyi bwa pòpa}
\begin{glose}
\pfra{quatorze}
\end{glose}
\end{sous-entrée}
\newline
\begin{sous-entrée}{tuuyi bwa pònim}{ⓔtruuçiⓝtuuyi bwa pònim}
\begin{glose}
\pfra{quinze}
\end{glose}
\end{sous-entrée}
\newline
\begin{sous-entrée}{tuuyi bwa pòni-ma-xe}{ⓔtruuçiⓝtuuyi bwa pòni-ma-xe}
\begin{glose}
\pfra{seize}
\end{glose}
\end{sous-entrée}
\newline
\begin{sous-entrée}{tuuyi bwa pòni-ma-du}{ⓔtruuçiⓝtuuyi bwa pòni-ma-du}
\begin{glose}
\pfra{dix-sept}
\end{glose}
\end{sous-entrée}
\newline
\begin{sous-entrée}{tuuyi bwa pòni-ma-gòn}{ⓔtruuçiⓝtuuyi bwa pòni-ma-gòn}
\begin{glose}
\pfra{dix-huit}
\end{glose}
\end{sous-entrée}
\newline
\begin{sous-entrée}{tuuyi bwa pòni-ma-ba}{ⓔtruuçiⓝtuuyi bwa pòni-ma-ba}
\begin{glose}
\pfra{dix-neuf}
\end{glose}
\end{sous-entrée}
\end{entrée}

\begin{entrée}{truuçi bwa pò-tru}{}{ⓔtruuçi bwa pò-tru}
\région{GO}
\variante{%
tuuyi bwa pòru
\région{PA}}
(\domainesémantique{Numéraux cardinaux})
\classe{NUM}
\begin{glose}
\pfra{douze (10 et 2)}
\end{glose}
\newline
\begin{exemple}
\région{GOs}
\textbf{\pnua{truuçi bwa we-tru balaa-ce}}
\pfra{12 morceaux de bois}
\end{exemple}
\end{entrée}

\begin{entrée}{truuçi bwa pòxe}{}{ⓔtruuçi bwa pòxe}
\région{GO}
\variante{%
tuuyi bwa pòxè
\région{PA}}
(\domainesémantique{Numéraux cardinaux})
\classe{NUM}
\begin{glose}
\pfra{onze (10 et 1)}
\end{glose}
\end{entrée}

\newpage

\lettrine{thr
(variante de GOs)}\begin{entrée}{thra}{1}{ⓔthraⓗ1}
\formephonétique{ʈʰa}
\région{GOs}
\variante{%
tha, pa-tha
\région{PA BO}}
(\domainesémantique{Corps humain})
\classe{v.stat.}
\begin{glose}
\pfra{chauve}
\end{glose}
\newline
\begin{exemple}
\région{BO}
\textbf{\pnua{tha bwaa-n}}
\pfra{il est chauve}
\end{exemple}
\newline
\begin{exemple}
\région{PA}
\textbf{\pnua{i phaa tha}}
\pfra{il est totalement chauve}
\end{exemple}
\newline
\begin{exemple}
\région{PA}
\textbf{\pnua{i tha}}
\pfra{il est chauve}
\end{exemple}
\end{entrée}

\begin{entrée}{thra}{2}{ⓔthraⓗ2}
\formephonétique{ʈʰa}
\région{GOs}
\variante{%
tha, thaa
\région{BO PA}}
\classe{v}
\newline
\sens{1}
(\domainesémantique{Soins du corps})
\begin{glose}
\pfra{raser}
\end{glose}
\begin{glose}
\pfra{tondre (poils)}
\end{glose}
\newline
\begin{exemple}
\région{GOs}
\textbf{\pnua{e pe-thra}}
\pfra{il se rase}
\end{exemple}
\newline
\begin{exemple}
\région{PA}
\textbf{\pnua{i pe-tha}}
\pfra{il se rase}
\end{exemple}
\newline
\begin{exemple}
\région{PA}
\textbf{\pnua{i coxe pu-n wu ra u thaa}}
\pfra{il se coupe les cheveux pour qu'ils soient courts}
\end{exemple}
\newline
\begin{exemple}
\textbf{\pnua{i pe-ravhi [PA, BO]}}
\pfra{il se rase}
\end{exemple}
\newline
\begin{exemple}
\région{GOs}
\textbf{\pnua{ge je pe-thra}}
\pfra{il est en train de se raser (la barbe)}
\end{exemple}
\newline
\begin{exemple}
\région{GOs}
\textbf{\pnua{e pe-thra Pwayili}}
\pfra{P.se rase}
\end{exemple}
\newline
\begin{exemple}
\région{GOs}
\textbf{\pnua{e thrae pu-phwa-je}}
\pfra{il s'est rasé la barbe (ou) il a rasé la barbe de qqn d'autre}
\end{exemple}
\newline
\begin{exemple}
\région{GOs}
\textbf{\pnua{e thrae pu-phwa xo Pwayili}}
\pfra{P. s'est rasé la barbe}
\end{exemple}
\newline
\begin{exemple}
\région{GOs}
\textbf{\pnua{e thrae pu-phwa Pwayili}}
\pfra{P. s'est rasé la barbe}
\end{exemple}
\newline
\begin{exemple}
\région{GOs}
\textbf{\pnua{nu thrae pu-phwa-nu}}
\pfra{je me suis rasé la barbe}
\end{exemple}
\newline
\begin{exemple}
\région{GOs}
\textbf{\pnua{nu thrae pu-phwa Pwayili}}
\pfra{j'ai rasé la barbe de P.}
\end{exemple}
\newline
\sens{2}
(\domainesémantique{Description des objets, formes, consistance, taille})
\begin{glose}
\pfra{court ; ras}
\end{glose}
\newline
\note{v.t. thrae}{grammaire}{}
\end{entrée}

\begin{entrée}{thra}{3}{ⓔthraⓗ3}
\formephonétique{ʈʰa}
\région{GOs}
\variante{%
thal
\formephonétique{tʰal}
\région{PA BO}}
(\domainesémantique{Noms des plantes})
\classe{nom}
\begin{glose}
\pfra{pandanus}
\end{glose}
\nomscientifique{Pandanus sp.(Pandanacées)}
\end{entrée}

\begin{entrée}{thraa}{}{ⓔthraa}
\formephonétique{ʈʰaː}
\région{GOs}
\variante{%
thaa
\région{PA BO}, 
mwang
\région{BO}}
(\domainesémantique{Description des objets, formes, consistance, taille})
\classe{v.stat. ; MODIF}
\begin{glose}
\pfra{mauvais ; dangereux ; mal}
\end{glose}
\newline
\begin{exemple}
\région{GOs}
\textbf{\pnua{i nè-raa}}
\pfra{cela a mauvais goût}
\end{exemple}
\newline
\begin{exemple}
\textbf{\pnua{ne-raa}}
\pfra{mauvais au goût}
\end{exemple}
\newline
\begin{exemple}
\textbf{\pnua{i khõbwe-raa-ini}}
\pfra{il l'a mal dit}
\end{exemple}
\newline
\begin{exemple}
\région{PA}
\textbf{\pnua{i nè-raa-ini ye}}
\pfra{il lui a fait du tort, du mal}
\end{exemple}
\newline
\begin{exemple}
\région{PA}
\textbf{\pnua{la ne-raa-ini}}
\pfra{ils l'ont mal fait}
\end{exemple}
\newline
\begin{exemple}
\région{BO}
\textbf{\pnua{thaa mõû-n}}
\pfra{sa femme est méchante}
\end{exemple}
\newline
\begin{exemple}
\région{BO}
\textbf{\pnua{nu kõbwe-raa-ini}}
\pfra{je l'aimal raconté}
\end{exemple}
\newline
\begin{exemple}
\textbf{\pnua{ãgu-raa}}
\pfra{une personne méchante}
\end{exemple}
\newline
\begin{exemple}
\région{BO}
\textbf{\pnua{i bo-raa}}
\pfra{il sent mauvais}
\end{exemple}
\newline
\begin{exemple}
\textbf{\pnua{i vhaa-raa nai yo}}
\pfra{il médit, dit du mal de lui}
\end{exemple}
\newline
\begin{exemple}
\région{PA}
\textbf{\pnua{i kibwa-raa-ini}}
\pfra{il l'a mal coupé}
\end{exemple}
\newline
\étymologie{
\langue{POc}
\étymon{*(n)saqa(t)}}
\newline
\note{forme transitive : -raa-ini}{grammaire}{}
\end{entrée}

\begin{entrée}{thraabu}{}{ⓔthraabu}
\région{GOs}
\région{PA BO}
\variante{%
thaabu
}
(\domainesémantique{Pêche})
\classe{v}
\begin{glose}
\pfra{pêcher en rivière (avec un filet ou à la main)}
\end{glose}
\newline
\begin{sous-entrée}{thrabu lapia}{ⓔthraabuⓝthrabu lapia}
\begin{glose}
\pfra{pêcher le tilapia}
\end{glose}
\end{sous-entrée}
\newline
\begin{sous-entrée}{thrabu kula}{ⓔthraabuⓝthrabu kula}
\begin{glose}
\pfra{pêcher la crevette}
\end{glose}
\end{sous-entrée}
\end{entrée}

\begin{entrée}{thraalo}{}{ⓔthraalo}
\région{GO}
\variante{%
thaalo
\région{BO (Corne)}}
(\domainesémantique{Noms des plantes})
\classe{nom}
\begin{glose}
\pfra{plante}
\end{glose}
\nomscientifique{Cordia myxa}
\end{entrée}

\begin{entrée}{thraanõ}{}{ⓔthraanõ}
\formephonétique{ʈʰaːɳɔ̃}
\région{GOs}
(\domainesémantique{Poissons})
\classe{nom}
\begin{glose}
\pfra{loche (de rivière, grande taille)}
\end{glose}
\nomscientifique{Eleotris melanosoma (Eleotridés)}
\newline
\relationsémantique{Cf.}{\lien{}{baatro ; baaro}}
\glosecourte{lochon (petite taille)}
\end{entrée}

\begin{entrée}{thraavwã}{}{ⓔthraavwã}
\formephonétique{ʈʰaːβɛ̃}
\région{GOs}
\région{GOs}
\variante{%
thraapã
\formephonétique{ʈʰaːpɛ̃}, 
thaavan
\région{BO PA}}
(\domainesémantique{Marées})
\classe{nom}
\begin{glose}
\pfra{marée descendante ; marée basse}
\end{glose}
\newline
\begin{exemple}
\région{BO}
\textbf{\pnua{thaavan gòòn-al}}
\pfra{marée basse de l'après-midi}
\end{exemple}
\newline
\begin{exemple}
\région{BO}
\textbf{\pnua{thaavan tòbwòn}}
\pfra{marée basse du soir}
\end{exemple}
\end{entrée}

\begin{entrée}{thrãbo}{}{ⓔthrãbo}
\formephonétique{ʈʰɛ̃bo}
\région{GOs}
(\domainesémantique{Description des objets, formes, consistance, taille})
\classe{v.stat.}
\begin{glose}
\pfra{épais}
\end{glose}
\end{entrée}

\begin{entrée}{thrae}{}{ⓔthrae}
\formephonétique{ʈʰae}
\région{GOs}
\variante{%
thaèè
\région{BO [BM, Corne]}}
(\domainesémantique{Matière, matériaux})
\classe{nom}
\begin{glose}
\pfra{argile à pot ; glaise}
\end{glose}
\end{entrée}

\begin{entrée}{thrale}{}{ⓔthrale}
\formephonétique{'ʈʰale}
\région{GOs}
\variante{%
thalei
\région{PA}}
(\domainesémantique{Nattes})
\classe{nom}
\begin{glose}
\pfra{natte (faite de deux palmes de cocotier tressées et posées par terre)}
\end{glose}
\newline
\relationsémantique{Cf.}{\lien{ⓔbenõõ}{benõõ}}
\end{entrée}

\begin{entrée}{thraliwa}{}{ⓔthraliwa}
\formephonétique{'ʈʰaliwa}
\région{GOs}
\variante{%
thaliang
\région{PA}}
(\domainesémantique{Cours de la vie})
\classe{v}
\begin{glose}
\pfra{tuer (se) ; suicider (se)}
\end{glose}
\end{entrée}

\begin{entrée}{thralo}{}{ⓔthralo}
\région{GOs}
\variante{%
thalo
\région{PA BO}}
(\domainesémantique{Arbre})
\classe{nom}
\begin{glose}
\pfra{"cerisier bleu" ; "bois bleu"}
\end{glose}
\nomscientifique{Eleocarpus Elaeocarpacées}
\end{entrée}

\begin{entrée}{thramwenge}{}{ⓔthramwenge}
\formephonétique{ʈʰa'mweŋe}
\région{GOs}
(\domainesémantique{Verbes d'action (en général)})
\classe{v}
\begin{glose}
\pfra{fabriquer ; créer}
\end{glose}
\end{entrée}

\begin{entrée}{thrao}{}{ⓔthrao}
\région{GOs}
\variante{%
thao
\région{PABO}}
(\domainesémantique{Noms des plantes})
\classe{nom}
\begin{glose}
\pfra{champignon (terme générique)}
\end{glose}
\end{entrée}

\begin{entrée}{thrao-da}{}{ⓔthrao-da}
\région{GOs WEM}
\variante{%
thrawa-da
\région{GO(s)}, 
thaa-da
\région{PA BO}}
(\domainesémantique{Verbes de déplacement et moyens de déplacement})
\classe{v}
\begin{glose}
\pfra{arriver en haut}
\end{glose}
\newline
\begin{exemple}
\région{BO}
\textbf{\pnua{thaa-da}}
\pfra{jusqu'à}
\end{exemple}
\newline
\relationsémantique{Ant.}{\lien{}{thrao-du}}
\glosecourte{arriver en bas}
\end{entrée}

\begin{entrée}{thrava}{}{ⓔthrava}
\formephonétique{ʈʰava}
\région{GOs}
(\domainesémantique{Caractéristiques et propriétés des personnes
, Modalité, verbes modaux})
\classe{v.stat. ; n}
\begin{glose}
\pfra{mauvais ; le mal}
\end{glose}
\end{entrée}

\begin{entrée}{thravayu}{}{ⓔthravayu}
\région{GOs}
\variante{%
thayu
\région{GO(s)}}
(\domainesémantique{Fonctions naturelles humaines})
\classe{nom}
\begin{glose}
\pfra{bave de mort}
\end{glose}
\newline
\begin{exemple}
\textbf{\pnua{tayu-n}}
\pfra{sabave (d'un mourant)}
\end{exemple}
\end{entrée}

\begin{entrée}{thravwi}{}{ⓔthravwi}
\formephonétique{ʈʰawɨ}
\région{GOs}
\variante{%
thawi, tawi
\région{PA BO}}
(\domainesémantique{Chasse})
\classe{v}
\begin{glose}
\pfra{poursuivre (à la chasse)}
\end{glose}
\begin{glose}
\pfra{chasser ; écarter (chiens, volailles)}
\end{glose}
\newline
\begin{exemple}
\région{PA}
\textbf{\pnua{e thawi poxa meni}}
\pfra{il chasse les petits oiseaux}
\end{exemple}
\end{entrée}

\begin{entrée}{thrawa}{}{ⓔthrawa}
\formephonétique{ʈʰawa}
\région{GOs}
\variante{%
thrao
\région{GO(s)}, 
thawa, thaa
\région{BO}}
(\domainesémantique{Verbes de déplacement et moyens de déplacement})
\classe{v}
\begin{glose}
\pfra{arriver ; survenir}
\end{glose}
\newline
\begin{exemple}
\textbf{\pnua{e u thrawa}}
\pfra{il est arrivé}
\end{exemple}
\newline
\begin{exemple}
\région{BO}
\textbf{\pnua{thaa-da}}
\pfra{jusqu'à ce que}
\end{exemple}
\end{entrée}

\begin{entrée}{thraxe}{}{ⓔthraxe}
\formephonétique{ʈʰaɣe}
\région{GOs}
(\domainesémantique{Sentiments})
\classe{v}
\begin{glose}
\pfra{gémir (de douleur) ; hurler}
\end{glose}
\end{entrée}

\begin{entrée}{thraxilo}{}{ⓔthraxilo}
\région{GOs}
\variante{%
thaxilò
\région{PA}, 
taxilò
\région{BO}}
(\domainesémantique{Aliments, alimentation})
\classe{v}
\begin{glose}
\pfra{manger vert et cru (des fruits)}
\end{glose}
\newline
\relationsémantique{Cf.}{\lien{}{mãiyã [GOs], meâ [BO]}}
\glosecourte{cru ; pas cuit}
\end{entrée}

\begin{entrée}{thre}{}{ⓔthre}
\région{GOs}
\variante{%
the
\région{BO}}
(\domainesémantique{Noms des plantes})
\classe{nom}
\begin{glose}
\pfra{mousse verte de rivière}
\end{glose}
\end{entrée}

\begin{entrée}{thrê}{}{ⓔthrê}
\région{GOs}
\variante{%
thê, thã
\région{BO}}
(\domainesémantique{Oiseaux})
\classe{nom}
\begin{glose}
\pfra{héron de nuit ; aigrette}
\end{glose}
\nomscientifique{Nycticorax caledonicus caledonicus (Ardéidés)}
\end{entrée}

\begin{entrée}{three}{}{ⓔthree}
\région{GOs}
\variante{%
thee
\région{BO [Corne]}}
(\domainesémantique{Actions liées aux plantes})
\classe{v}
\begin{glose}
\pfra{rapporter de nouvelles boutures chez soi}
\end{glose}
\end{entrée}

\begin{entrée}{thrêê}{}{ⓔthrêê}
\formephonétique{ʈʰêː}
\région{GOs}
\variante{%
têên
\formephonétique{tɛ̃ɛ̃n}
\région{PA BO WEM}}
(\domainesémantique{Verbes de déplacement et moyens de déplacement})
\classe{v}
\begin{glose}
\pfra{courir}
\end{glose}
\newline
\begin{exemple}
\région{GO}
\textbf{\pnua{e thrêê òò}}
\pfra{il s'éloigne en courant}
\end{exemple}
\newline
\begin{exemple}
\région{GO}
\textbf{\pnua{e thrêê upwa}}
\pfra{il sort en courant}
\end{exemple}
\end{entrée}

\begin{entrée}{thrêê-kai}{}{ⓔthrêê-kai}
\région{GOs}
(\domainesémantique{Chasse})
\classe{v}
\begin{glose}
\pfra{poursuivre ; courir derrière}
\end{glose}
\newline
\note{thrêê (courir) + kai (dos)}{général}{}
\end{entrée}

\begin{entrée}{thrêê kha-ve}{}{ⓔthrêê kha-ve}
\région{GOs}
\variante{%
thêên
\région{PA}}
(\domainesémantique{Verbes de déplacement et moyens de déplacement})
\classe{v}
\begin{glose}
\pfra{courir en emportant qqch}
\end{glose}
\newline
\relationsémantique{Cf.}{\lien{}{kha-phe, kha-ve}}
\glosecourte{apporter}
\end{entrée}

\begin{entrée}{threi}{}{ⓔthrei}
\formephonétique{ʈʰei}
\région{GOs WEM}
\variante{%
thei, thèi
\région{BO PA}}
\classe{v}
\newline
\sens{1}
(\domainesémantique{Mouvements ou actions faits avec le corps, les bras, les mains, les pieds})
\begin{glose}
\pfra{couper d'un coup}
\end{glose}
\begin{glose}
\pfra{abattre (arbre)}
\end{glose}
\begin{glose}
\pfra{frapper (qqn pour le tuer)}
\end{glose}
\newline
\begin{exemple}
\région{PA}
\textbf{\pnua{i thei u/xo/vwo wamòn}}
\pfra{il l'a abattu avec une hache}
\end{exemple}
\newline
\sens{2}
(\domainesémantique{Travail bois})
\begin{glose}
\pfra{tailler du bois}
\end{glose}
\begin{glose}
\pfra{sculpter}
\end{glose}
\newline
\begin{exemple}
\région{PA}
\textbf{\pnua{a-thei ce}}
\pfra{un sculpteur}
\end{exemple}
\newline
\begin{sous-entrée}{the ce}{ⓔthreiⓢ2ⓝthe ce}
\région{PA}
\begin{glose}
\pfra{sculpter le bois}
\end{glose}
\newline
\relationsémantique{Cf.}{\lien{}{dei, thibe}}
\glosecourte{couper}
\end{sous-entrée}
\end{entrée}

\begin{entrée}{thre kha}{}{ⓔthre kha}
\région{GOs}
\variante{%
thèl
\région{BO PA}}
(\domainesémantique{Cultures, techniques, boutures})
\classe{v}
\begin{glose}
\pfra{préparer les champs (ignames) ; débroussailler leschamps (ignames)}
\end{glose}
\newline
\relationsémantique{Cf.}{\lien{ⓔthrei}{threi}}
\glosecourte{couper}
\end{entrée}

\begin{entrée}{thre-mii}{}{ⓔthre-mii}
\formephonétique{ʈʰemiː}
\région{GOs}
\variante{%
the-mii
\région{PA BO}}
(\domainesémantique{Noms des plantes})
\classe{nom}
\begin{glose}
\pfra{erythrinier à piquants (à fleurs rouges)}
\end{glose}
\newline
\begin{sous-entrée}{mu thre-mii}{ⓔthre-miiⓝmu thre-mii}
\région{GOs}
\begin{glose}
\pfra{fleur d'erythrinier}
\end{glose}
\end{sous-entrée}
\end{entrée}

\begin{entrée}{thri}{}{ⓔthri}
\formephonétique{ʈʰi}
\région{GOs WEM}
\variante{%
thiing, thing
\région{BO [BM}, 
thiin
\région{BO [Corne]}}
(\domainesémantique{Coutumes, dons coutumiers})
\classe{nom}
\begin{glose}
\pfra{deuil de chef}
\end{glose}
\begin{glose}
\pfra{coutumes de deuil du chef : levée de deuil d'un grand chef}
\end{glose}
\newline
\begin{sous-entrée}{ce-thrii}{ⓔthriⓝce-thrii}
\begin{glose}
\pfra{perche qui annonce le décès d'un chef}
\end{glose}
\newline
\relationsémantique{Cf.}{\lien{ⓔhauva}{hauva}}
\glosecourte{levée de deuil pour le commun des mortels}
\newline
\relationsémantique{Cf.}{\lien{ⓔmõõdim}{mõõdim}}
\glosecourte{coutumes de deuil (générique)}
\end{sous-entrée}
\end{entrée}

\begin{entrée}{thria}{}{ⓔthria}
\région{GOs}
\variante{%
thia
\région{PA}}
(\domainesémantique{Coutumes, dons coutumiers})
\classe{nom}
\begin{glose}
\pfra{pardon coutumier}
\end{glose}
\newline
\begin{exemple}
\région{GOs}
\textbf{\pnua{li pe-tha-thria li-e ãbaa-nu èmwê}}
\pfra{mes deux frères se sont pardonnés (tha : attacher)}
\end{exemple}
\newline
\begin{exemple}
\région{GOs}
\textbf{\pnua{mô pe-tha-thria}}
\pfra{nous avons fait un pardon coutumier}
\end{exemple}
\newline
\begin{exemple}
\région{GOs}
\textbf{\pnua{mô pe-nhõi-thria}}
\pfra{nous avons fait un pardon coutumier (lit. nous avons attaché le pardon coutumier)}
\end{exemple}
\newline
\begin{sous-entrée}{tha thria}{ⓔthriaⓝtha thria}
\begin{glose}
\pfra{demander pardon}
\end{glose}
\end{sous-entrée}
\end{entrée}

\begin{entrée}{thridoo}{}{ⓔthridoo}
\région{GOs}
(\domainesémantique{Poissons})
\classe{nom}
\begin{glose}
\pfra{murène}
\end{glose}
\end{entrée}

\begin{entrée}{thrimavwo}{}{ⓔthrimavwo}
\région{GOs}
\variante{%
thrimapwo
\région{GO(s)}}
(\domainesémantique{Poissons})
\classe{nom}
\begin{glose}
\pfra{picot (de palétuvier qui remonte les rivières) ; poisson-papillon}
\end{glose}
\end{entrée}

\begin{entrée}{thrîmi}{}{ⓔthrîmi}
\formephonétique{ʈʰîmi}
\région{GOs}
\variante{%
thimi
\région{PA BO}}
(\domainesémantique{Soins du corps
, Mouvements ou actions faits avec le corps, les bras, les mains, les pieds})
\classe{v}
\begin{glose}
\pfra{peindre}
\end{glose}
\begin{glose}
\pfra{enduire}
\end{glose}
\begin{glose}
\pfra{teindre (cheveux)}
\end{glose}
\newline
\begin{exemple}
\région{BO}
\textbf{\pnua{ju thimi mèè-n}}
\pfra{étale-la (crème) sur ton oeil}
\end{exemple}
\newline
\begin{exemple}
\région{BO}
\textbf{\pnua{nu ru thimi mõ-ny u mii}}
\pfra{je vais peindre ma maison en rouge}
\end{exemple}
\newline
\begin{sous-entrée}{ba-thimi}{ⓔthrîmiⓝba-thimi}
\begin{glose}
\pfra{pinceau}
\end{glose}
\end{sous-entrée}
\end{entrée}

\begin{entrée}{thringã}{}{ⓔthringã}
\formephonétique{ʈʰiŋɛ̃}
\région{GOs}
\variante{%
thrixã
\région{GO(s)}, 
thingã
\région{BO}}
(\domainesémantique{Corps animal})
\classe{nom}
\begin{glose}
\pfra{queue (oiseau, animal)}
\end{glose}
\newline
\begin{exemple}
\région{BO}
\textbf{\pnua{thingã-n}}
\pfra{sa queue}
\end{exemple}
\end{entrée}

\begin{entrée}{thriu}{}{ⓔthriu}
(\domainesémantique{Verbes de déplacement et moyens de déplacement
, Verbes de mouvement})
\classe{v}
\begin{glose}
\pfra{partir ;}
\end{glose}
\begin{glose}
\pfra{disperser (se)}
\end{glose}
\newline
\begin{exemple}
\région{GOs}
\région{GOs}
\textbf{\pnua{mô thriu !}}
\pfra{partons !}
\end{exemple}
\newline
\begin{exemple}
\région{GOs}
\textbf{\pnua{la u thriu pe-haze !}}
\pfra{ils partent chacun de leur côté !}
\end{exemple}
\newline
\begin{exemple}
\textbf{\pnua{la u tiu veale !}}
\pfra{ils se sont dispersés !}
\end{exemple}
\end{entrée}

\begin{entrée}{thrivwaja}{}{ⓔthrivwaja}
\formephonétique{ʈʰivwaja}
\région{GOs}
\variante{%
thripwaja
\région{vx (Haudricourt)}}
(\domainesémantique{Vents})
\classe{nom}
\begin{glose}
\pfra{vent d'ouest}
\end{glose}
\end{entrée}

\begin{entrée}{thrô}{}{ⓔthrô}
\formephonétique{ʈʰõ}
\région{GOs}
\région{PA BO}
\variante{%
thôm
}
(\domainesémantique{Nattes})
\classe{nom}
\begin{glose}
\pfra{natte (de pandanus) ; noeud}
\end{glose}
\newline
\begin{sous-entrée}{pa thrô}{ⓔthrôⓝpa thrô}
\begin{glose}
\pfra{faire une natte ; faire un noeud}
\end{glose}
\newline
\begin{exemple}
\région{GOs}
\textbf{\pnua{thrô za ?}}
\pfra{une natte pour quel usage ?}
\end{exemple}
\newline
\begin{exemple}
\région{GOs}
\textbf{\pnua{thrôvwo-nu}}
\pfra{ma natte}
\end{exemple}
\newline
\begin{exemple}
\région{BO}
\textbf{\pnua{thrombò-n}}
\pfra{sa natte}
\end{exemple}
\end{sous-entrée}
\newline
\begin{sous-entrée}{thrôvwo covwa}{ⓔthrôⓝthrôvwo covwa}
\région{GOs}
\begin{glose}
\pfra{selle de cheval (une natte)}
\end{glose}
\end{sous-entrée}
\newline
\begin{sous-entrée}{thrôvwo loto}{ⓔthrôⓝthrôvwo loto}
\région{GOs}
\begin{glose}
\pfra{housse de siège de voiture (une natte)}
\end{glose}
\end{sous-entrée}
\newline
\begin{sous-entrée}{thrôvwo ta}{ⓔthrôⓝthrôvwo ta}
\région{GOs}
\begin{glose}
\pfra{nappe}
\end{glose}
\end{sous-entrée}
\newline
\begin{sous-entrée}{thrôvwo vele s}{ⓔthrôⓝthrôvwo vele s}
\région{GOs}
\begin{glose}
\pfra{drap}
\end{glose}
\end{sous-entrée}
\end{entrée}

\begin{entrée}{thrõbo}{}{ⓔthrõbo}
\formephonétique{ʈʰɔ̃bo}
\région{GOs WEM}
\variante{%
thôbo
\formephonétique{tʰɔ̃bo}
\région{BO PA}}
\classe{v}
\newline
\sens{1}
(\domainesémantique{Verbes de mouvement})
\begin{glose}
\pfra{descendre ; tomber}
\end{glose}
\newline
\sens{2}
(\domainesémantique{Cours de la vie})
\begin{glose}
\pfra{naître}
\end{glose}
\newline
\begin{exemple}
\région{GOs}
\textbf{\pnua{e thrôbo-du}}
\pfra{il descend}
\end{exemple}
\newline
\begin{exemple}
\région{PA}
\textbf{\pnua{thöbo-du-mi !}}
\pfra{descends ici !}
\end{exemple}
\newline
\begin{exemple}
\région{BO}
\textbf{\pnua{i thrôbo al}}
\pfra{le soleil décline}
\end{exemple}
\newline
\begin{exemple}
\textbf{\pnua{nu thrôbo ni ka 1992}}
\pfra{je suis né en 1992}
\end{exemple}
\newline
\begin{exemple}
\région{GOs}
\textbf{\pnua{e ka-thrôbo pu trabwa}}
\pfra{il descend pour se poser (oiseau)}
\end{exemple}
\newline
\begin{exemple}
\région{GOs}
\textbf{\pnua{e thrõbo-gò}}
\pfra{il ne fait pas encore nuit}
\end{exemple}
\newline
\begin{sous-entrée}{ke-thrôbo}{ⓔthrõboⓢ2ⓝke-thrôbo}
\begin{glose}
\pfra{panier de charge}
\end{glose}
\end{sous-entrée}
\newline
\begin{sous-entrée}{we-thrôbo}{ⓔthrõboⓢ2ⓝwe-thrôbo}
\begin{glose}
\pfra{prise / ouverture d'eau}
\end{glose}
\newline
\note{pa-thrôbo-ni}{grammaire}{faire tomber}
\end{sous-entrée}
\newline
\relationsémantique{Cf.}{\lien{ⓔpweⓗ1}{pwe}}
\glosecourte{naître}
\newline
\relationsémantique{Cf.}{\lien{}{co-du}}
\glosecourte{descendre de (cheval)}
\newline
\relationsémantique{Cf.}{\lien{}{co-da}}
\glosecourte{débarquer}
\end{entrée}

\begin{entrée}{thrõbo trò}{}{ⓔthrõbo trò}
\région{GOs}
\variante{%
thòbwò tòn
\région{PA}}
(\domainesémantique{Découpage du temps})
\classe{nom}
\begin{glose}
\pfra{tombée de la nuit}
\end{glose}
\newline
\begin{exemple}
\région{BO}
\textbf{\pnua{thòbwòn}}
\pfra{soir}
\end{exemple}
\newline
\relationsémantique{Cf.}{\lien{}{tòn (PA)}}
\glosecourte{nuit}
\end{entrée}

\begin{entrée}{thrõbwò}{}{ⓔthrõbwò}
\formephonétique{ʈʰɔ̃bwɔ}
\région{GOs}
\région{PA BO}
\variante{%
thõbwòn, thõbòn
}
(\domainesémantique{Découpage du temps})
\classe{nom}
\begin{glose}
\pfra{soir}
\end{glose}
\newline
\begin{exemple}
\région{BO}
\textbf{\pnua{ra u thõbwòn}}
\pfra{il fait nuit}
\end{exemple}
\end{entrée}

\begin{entrée}{thrõgo}{}{ⓔthrõgo}
\formephonétique{ʈʰɔ̃ŋgo}
\région{GOs}
\variante{%
thõgo
\région{BO}}
(\domainesémantique{Actions liées aux plantes})
\classe{v}
\begin{glose}
\pfra{cueillir (feuilles, herbes, jeunes pousses)}
\end{glose}
\newline
\begin{exemple}
\textbf{\pnua{thrõgo dròò-cee}}
\pfra{cueillir des feuilles}
\end{exemple}
\end{entrée}

\begin{entrée}{thrô-kui}{}{ⓔthrô-kui}
\région{GOs}
\variante{%
thô-kui
\région{PA BO}, 
pwe-nô kui
\région{WEM}}
(\domainesémantique{Ignames})
\classe{nom}
\begin{glose}
\pfra{extrêmité inférieure de l'igname}
\end{glose}
\newline
\relationsémantique{Cf.}{\lien{}{bwe-kui [GOs, PA]}}
\glosecourte{tête de l'igname}
\end{entrée}

\begin{entrée}{thrô-kuru}{}{ⓔthrô-kuru}
\région{GOs}
(\domainesémantique{Taros})
\classe{nom}
\begin{glose}
\pfra{extrêmité inférieure du taro}
\end{glose}
\end{entrée}

\begin{entrée}{throli}{}{ⓔthroli}
\formephonétique{ʈʰoli}
\région{GOs}
(\domainesémantique{Cocotiers})
\classe{nom}
\begin{glose}
\pfra{rachis de coco}
\end{glose}
\end{entrée}

\begin{entrée}{thròlò}{}{ⓔthròlò}
\région{GOs}
\variante{%
thòlò
\région{PA}}
(\domainesémantique{Types de maison, architecture de la maison})
\classe{v ; n}
\begin{glose}
\pfra{couvrir (une maison) ; couverture (en général)}
\end{glose}
\begin{glose}
\pfra{faitage sculpté (Dubois)}
\end{glose}
\end{entrée}

\begin{entrée}{thròlòe}{}{ⓔthròlòe}
\région{GOs}
\variante{%
thòlòe
\région{PA}, 
tòloè
\région{BO (Corne)}, 
tholòè, tòlee
\région{BO (BM)}, 
throleitholei
\région{WEM}}
(\domainesémantique{Mouvements ou actions faits avec le corps, les bras, les mains, les pieds
, Verbes de mouvement})
\classe{v}
\begin{glose}
\pfra{déplier}
\end{glose}
\begin{glose}
\pfra{étendre (bras, etc.)}
\end{glose}
\begin{glose}
\pfra{étaler (natte, etc.)}
\end{glose}
\begin{glose}
\pfra{étaler (s') (pour des plantes)}
\end{glose}
\begin{glose}
\pfra{ramper (lianes)}
\end{glose}
\newline
\begin{exemple}
\région{GOs}
\textbf{\pnua{e throlòè thrô}}
\pfra{elle étale la natte}
\end{exemple}
\newline
\begin{exemple}
\région{GOs}
\textbf{\pnua{ce xa thròlòè}}
\pfra{un arbre qui déploie ses branches}
\end{exemple}
\newline
\begin{exemple}
\région{BO}
\textbf{\pnua{tòlè thom bwabu}}
\pfra{étale une natte au sol}
\end{exemple}
\end{entrée}

\begin{entrée}{throloo}{}{ⓔthroloo}
\région{WEM}
(\domainesémantique{Types de maison, architecture de la maison})
\classe{nom}
\begin{glose}
\pfra{décorations sur la flèche faitière de la maison 'throo-mwa' (conques etc)}
\end{glose}
\end{entrée}

\begin{entrée}{throo}{}{ⓔthroo}
\formephonétique{ʈʰoː}
\région{GOs WEM}
\variante{%
thoo
\région{PA BO}}
\classe{nom}
\newline
\sens{1}
(\domainesémantique{Corps animal})
\begin{glose}
\pfra{crête ; aigrette}
\end{glose}
\newline
\begin{exemple}
\région{GOs}
\textbf{\pnua{throo-je}}
\pfra{son aigrette}
\end{exemple}
\newline
\sens{2}
(\domainesémantique{Vêtements, parure})
\begin{glose}
\pfra{plumet (dans les cheveux ou la coiffure)}
\end{glose}
\end{entrée}

\begin{entrée}{thrõõbo}{}{ⓔthrõõbo}
\région{GOs}
\variante{%
thõõbon
\région{BO}, 
thõba, thõõbwa
\région{PA WEM}}
(\domainesémantique{Portage})
\classe{v}
\begin{glose}
\pfra{porter sur le dos}
\end{glose}
\newline
\note{porter sur le dos ou les épaules avec une corde de portage ; ce sont les femmes qui portent ainsi les paniers)}{glose}{}
\newline
\begin{exemple}
\région{GOs}
\textbf{\pnua{e thrõõbo cee}}
\pfra{il porte du bois sur le dos}
\end{exemple}
\newline
\begin{exemple}
\région{GOs}
\textbf{\pnua{i thrõõbo-ni xo/xu õõ-li}}
\pfra{leur mère le porte sur le dos}
\end{exemple}
\newline
\begin{sous-entrée}{ke-thrõõbo}{ⓔthrõõboⓝke-thrõõbo}
\begin{glose}
\pfra{panier porté sur le dos}
\end{glose}
\newline
\note{thrõõbone [GOs, BO]}{grammaire}{porter qqch}
\end{sous-entrée}
\end{entrée}

\begin{entrée}{throobwa}{}{ⓔthroobwa}
\région{GOs}
\variante{%
thobwang
\région{PA}}
(\domainesémantique{Noms des plantes})
\classe{nom}
\begin{glose}
\pfra{liane (des endroits humides, à fleur blanche et à grosses feuilles, parasite des arbres)}
\end{glose}
\nomscientifique{Ipomoea macrantha, Convolvulacées}
\end{entrée}

\begin{entrée}{throo-mwa}{}{ⓔthroo-mwa}
\région{GOs WEM}
\variante{%
thoo-mwa
\région{PA BO}}
(\domainesémantique{Types de maison, architecture de la maison})
\classe{nom}
\begin{glose}
\pfra{flèche faîtière}
\end{glose}
\end{entrée}

\begin{entrée}{throonye}{}{ⓔthroonye}
\région{GOs}
(\domainesémantique{Poissons})
\classe{nom}
\begin{glose}
\pfra{loche géante, carite}
\end{glose}
\nomscientifique{Epinephelus lanceolatus (Serranidés)}
\end{entrée}

\begin{entrée}{throvwa}{}{ⓔthrovwa}
\région{GOs}
\variante{%
thropwa
\région{vx}}
(\domainesémantique{Anguilles})
\classe{nom}
\begin{glose}
\pfra{anguille de creek (noire et verte, à grosse tête)}
\end{glose}
\end{entrée}

\begin{entrée}{thrûã}{}{ⓔthrûã}
\formephonétique{ʈʰũɛ̃}
\région{GO}
(\domainesémantique{Jeux divers})
\classe{nom}
\begin{glose}
\pfra{jeu de ficelle (figure du)}
\end{glose}
\end{entrée}

\begin{entrée}{thrugò}{}{ⓔthrugò}
\formephonétique{ʈʰuŋgɔ}
\région{GOs}
\variante{%
thugò
\région{PA}}
(\domainesémantique{Outils})
\classe{nom}
\begin{glose}
\pfra{herminette}
\end{glose}
\end{entrée}

\begin{entrée}{thruumã}{}{ⓔthruumã}
\région{GOs}
\variante{%
thuumã
\région{PA BO}, 
tuumã
\région{PA BO (Corne)}}
(\domainesémantique{Sentiments})
\classe{v.stat. ; n}
\begin{glose}
\pfra{heureux ; joyeux ; content ; joie ; joyeux ; content}
\end{glose}
\newline
\begin{exemple}
\région{BO}
\textbf{\pnua{i thûûmã i a buram}}
\pfra{il est content d'aller se baigner}
\end{exemple}
\newline
\begin{sous-entrée}{pa-tûma}{ⓔthruumãⓝpa-tûma}
\begin{glose}
\pfra{rendre heureux}
\end{glose}
\end{sous-entrée}
\end{entrée}

\newpage

\lettrine{
u
û
}\begin{entrée}{u}{1}{ⓔuⓗ1}
\région{GOs PA}
(\domainesémantique{Aspect})
\classe{ASP.ACC}
\begin{glose}
\pfra{accompli}
\end{glose}
\newline
\begin{exemple}
\région{GOs}
\textbf{\pnua{u uja}}
\pfra{il est déjà arrivé}
\end{exemple}
\newline
\begin{exemple}
\région{GOs}
\textbf{\pnua{u ògi}}
\pfra{c'est fini}
\end{exemple}
\newline
\begin{exemple}
\région{PA}
\textbf{\pnua{u ògin}}
\pfra{c'est fini}
\end{exemple}
\end{entrée}

\begin{entrée}{u}{2}{ⓔuⓗ2}
\région{BO [BM, Corne]}
(\domainesémantique{Prépositions})
\classe{PREP (instrumental)}
\begin{glose}
\pfra{avec ; par ; grâce à ; du fait de ; à cause de}
\end{glose}
\newline
\begin{exemple}
\textbf{\pnua{teemwi u yi â}}
\pfra{on les attrape avec la main (crabes)}
\end{exemple}
\end{entrée}

\begin{entrée}{u}{3}{ⓔuⓗ3}
\région{GOs BO}
(\domainesémantique{Verbes de mouvement
, Mouvements ou actions faits avec le corps, les bras, les mains, les pieds})
\classe{v}
\begin{glose}
\pfra{pencher (se) ;}
\end{glose}
\begin{glose}
\pfra{baisser (se) ;}
\end{glose}
\begin{glose}
\pfra{courber (se) (par ex. pour entrer dans une maison)}
\end{glose}
\newline
\begin{exemple}
\région{BO}
\textbf{\pnua{i u-du mwa al}}
\pfra{le soleil s'est couché}
\end{exemple}
\newline
\begin{sous-entrée}{u-du !}{ⓔuⓗ3ⓝu-du !}
\begin{glose}
\pfra{baisse-toi !}
\end{glose}
\end{sous-entrée}
\newline
\begin{sous-entrée}{u-da !}{ⓔuⓗ3ⓝu-da !}
\begin{glose}
\pfra{entre ! (dans la maison)}
\end{glose}
\end{sous-entrée}
\newline
\begin{sous-entrée}{u-e !}{ⓔuⓗ3ⓝu-e !}
\begin{glose}
\pfra{entre ! (latéralement)}
\end{glose}
\end{sous-entrée}
\end{entrée}

\begin{entrée}{u}{4}{ⓔuⓗ4}
\région{PA BO}
\variante{%
ku, xu, xo
}
(\domainesémantique{Agent})
\classe{AGT}
\begin{glose}
\pfra{agent}
\end{glose}
\newline
\begin{exemple}
\textbf{\pnua{i whili u ti/ri ?}}
\pfra{qui l'a mangé (la canne à sucre) ?}
\end{exemple}
\newline
\begin{exemple}
\textbf{\pnua{i pa-toni go u ti/ri ?}}
\pfra{qui joue de la musique ?}
\end{exemple}
\end{entrée}

\begin{entrée}{u}{5}{ⓔuⓗ5}
\région{GOs PA}
\classe{v}
(\domainesémantique{Verbes de mouvement})
\begin{glose}
\pfra{tomber}
\end{glose}
\newline
\begin{exemple}
\textbf{\pnua{u je ce}}
\pfra{cet arbre est tombé}
\end{exemple}
\newline
\begin{exemple}
\région{GOs}
\textbf{\pnua{e u ce xa we-xe}}
\pfra{un arbre est tombé}
\end{exemple}
\newline
\begin{exemple}
\région{GOs}
\textbf{\pnua{la u la-ã ce}}
\pfra{ces arbres sont tombés}
\end{exemple}
\newline
\begin{exemple}
\région{GOs}
\textbf{\pnua{la tha-uu-ni la-ã ce}}
\pfra{ils ont fait tomber ces arbres (en les poussant)}
\end{exemple}
\end{entrée}

\begin{entrée}{û}{}{ⓔû}
\région{GOs}
(\domainesémantique{Cocotiers})
\classe{nom}
\begin{glose}
\pfra{spathe de cocotier}
\end{glose}
\begin{glose}
\pfra{étoffe}
\end{glose}
\newline
\begin{sous-entrée}{û-vwa nu}{ⓔûⓝû-vwa nu}
\begin{glose}
\pfra{fibre du cocotier}
\end{glose}
\end{sous-entrée}
\end{entrée}

\begin{entrée}{ua}{}{ⓔua}
\région{BO}
(\domainesémantique{Ignames})
\classe{nom}
\begin{glose}
\pfra{igname violette (Dubois)}
\end{glose}
\end{entrée}

\begin{entrée}{ubò}{}{ⓔubò}
\formephonétique{ubɔ}
\région{GOs}
\variante{%
u-vwa = u phwaa (différent)
\région{GO(s)}, 
uva
\région{WEM}}
(\domainesémantique{Verbes de déplacement et moyens de déplacement})
\classe{v}
\begin{glose}
\pfra{sortir (de la maison)}
\end{glose}
\newline
\begin{exemple}
\région{GOs}
\textbf{\pnua{e ubò na ni mwa}}
\pfra{il est sorti de la maison}
\end{exemple}
\newline
\begin{exemple}
\région{GOs}
\textbf{\pnua{ubò-du pwa !}}
\pfra{va dehors !}
\end{exemple}
\newline
\begin{exemple}
\région{GOs}
\textbf{\pnua{e ubò pwa}}
\pfra{il sort dehors}
\end{exemple}
\end{entrée}

\begin{entrée}{ubòl}{}{ⓔubòl}
\région{PA BO [Corne]}
(\domainesémantique{Mouvements ou actions faits avec le corps, les bras, les mains, les pieds})
\classe{nom}
\begin{glose}
\pfra{détacher}
\end{glose}
\begin{glose}
\pfra{lâcher}
\end{glose}
\begin{glose}
\pfra{laisser}
\end{glose}
\end{entrée}

\begin{entrée}{uça}{}{ⓔuça}
\formephonétique{uʒa}
\région{GOs}
(\domainesémantique{Verbes de déplacement et moyens de déplacement})
\classe{v}
\begin{glose}
\pfra{arriver}
\end{glose}
\newline
\begin{exemple}
\textbf{\pnua{ti nye uça xòlò i ju ?}}
\pfra{qui t'a rendu visite hier ?}
\end{exemple}
\newline
\begin{exemple}
\textbf{\pnua{e uça xòlò i çö xo ti ?}}
\pfra{qui t'a rendu visite hier ?}
\end{exemple}
\end{entrée}

\begin{entrée}{uda}{}{ⓔuda}
\région{BO}
(\domainesémantique{Types de maison, architecture de la maison})
\classe{nom}
\begin{glose}
\pfra{poutre sablière (poutre circulaire supportant la charpente des cases rondes ; Dubois)}
\end{glose}
\newline
\note{mot non vérifié}{général}{}
\end{entrée}

\begin{entrée}{u-da}{}{ⓔu-da}
\région{GOs}
(\domainesémantique{Verbes de déplacement et moyens de déplacement})
\classe{v}
\begin{glose}
\pfra{entrer (se baisser pour entrer dans une maison)}
\end{glose}
\begin{glose}
\pfra{monter}
\end{glose}
\newline
\begin{exemple}
\textbf{\pnua{e u-da-mi}}
\pfra{il est entré ici}
\end{exemple}
\newline
\begin{exemple}
\région{GOs}
\textbf{\pnua{e u-da mwa}}
\pfra{il est entré dans la maison}
\end{exemple}
\newline
\relationsémantique{Ant.}{\lien{ⓔu-du}{u-du}}
\glosecourte{sortir}
\end{entrée}

\begin{entrée}{udale}{}{ⓔudale}
\région{GOs PA}
(\domainesémantique{Vêtements, parure})
\classe{v}
\begin{glose}
\pfra{mettre (un vêtement) (lit. monter dedans)}
\end{glose}
\newline
\begin{exemple}
\région{GOs}
\textbf{\pnua{e udale simi}}
\pfra{il met sa chemise}
\end{exemple}
\newline
\begin{exemple}
\région{PA}
\textbf{\pnua{i udale cimic}}
\pfra{il met sa chemise}
\end{exemple}
\newline
\begin{exemple}
\région{PA}
\textbf{\pnua{i udale orop u Kaavo}}
\pfra{K met sa robe}
\end{exemple}
\newline
\begin{exemple}
\région{GOs}
\textbf{\pnua{e udale hõbòli-je}}
\pfra{il met son vêtement}
\end{exemple}
\newline
\begin{exemple}
\région{GOs}
\textbf{\pnua{e udale hõbwò xo Kaavwo}}
\pfra{K met son vêtement}
\end{exemple}
\newline
\begin{exemple}
\région{GOs}
\textbf{\pnua{e udale hõbwòli Kaavwo xo Hiixe}}
\pfra{H a enfilé les vêtements de Kaavo}
\end{exemple}
\newline
\begin{exemple}
\région{PA}
\textbf{\pnua{i udale orop u Kaavwo}}
\pfra{Kaavo a mis sa robe}
\end{exemple}
\newline
\relationsémantique{Cf.}{\lien{ⓔthai hõbò}{thai hõbò}}
\glosecourte{enfiler, mettre}
\newline
\relationsémantique{Ant.}{\lien{ⓔudi}{udi}}
\glosecourte{enlever}
\end{entrée}

\begin{entrée}{udang}{}{ⓔudang}
\région{BO}
(\domainesémantique{Taros})
\classe{nom}
\begin{glose}
\pfra{taro (clone) de terrain sec (Dubois)}
\end{glose}
\newline
\note{non vérifié}{général}{}
\end{entrée}

\begin{entrée}{udi}{}{ⓔudi}
\région{GOs BO}
\classe{v}
\newline
\sens{1}
(\domainesémantique{Vêtements, parure})
\begin{glose}
\pfra{enlever (en général, vêtement, etc.)}
\end{glose}
\begin{glose}
\pfra{ôter}
\end{glose}
\newline
\begin{exemple}
\textbf{\pnua{e udi simi i je}}
\pfra{il enlève sa chemise}
\end{exemple}
\newline
\begin{exemple}
\région{GOs}
\textbf{\pnua{e udi mwêêga-je}}
\pfra{il enlève son chapeau}
\end{exemple}
\newline
\relationsémantique{Ant.}{\lien{ⓔudale}{udale}}
\glosecourte{mettre (vêtement)}
\newline
\sens{2}
(\domainesémantique{Mouvements ou actions faits avec le corps, les bras, les mains, les pieds})
\begin{glose}
\pfra{extraire (épine, etc.) ; sortir}
\end{glose}
\begin{glose}
\pfra{déclouer}
\end{glose}
\end{entrée}

\begin{entrée}{ûdo}{}{ⓔûdo}
\région{GOs}
(\domainesémantique{Arbre})
\classe{nom}
\begin{glose}
\pfra{palétuvier (à feuilles comestibles)}
\end{glose}
\end{entrée}

\begin{entrée}{u-du}{}{ⓔu-du}
\région{GOs BO WEM}
(\domainesémantique{Verbes de mouvement
, Verbes de déplacement et moyens de déplacement})
\classe{v}
\begin{glose}
\pfra{baisser (se) ; pencher (se) [GOs]}
\end{glose}
\begin{glose}
\pfra{coucher (se) (soleil) [GOs]}
\end{glose}
\begin{glose}
\pfra{enfoncer (s') (dans la boue, le sable ) [GOs, BO]}
\end{glose}
\begin{glose}
\pfra{rentrer (dans un trou, dans l'eau) [GOs, BO]}
\end{glose}
\begin{glose}
\pfra{plonger [GOs, BO]}
\end{glose}
\begin{glose}
\pfra{entrer (se baisser pour entrer ou sortir d'une maison) [GOs, BO]}
\end{glose}
\begin{glose}
\pfra{disparaitre [GOs, BO]}
\end{glose}
\newline
\begin{exemple}
\région{BO}
\textbf{\pnua{u-du al}}
\pfra{le soleil se couche}
\end{exemple}
\newline
\begin{exemple}
\région{GOs}
\textbf{\pnua{e u-du a}}
\pfra{le soleil se couche}
\end{exemple}
\newline
\begin{exemple}
\région{BO}
\textbf{\pnua{u-du ni dili}}
\pfra{s'enfoncer dans la boue}
\end{exemple}
\newline
\begin{exemple}
\région{BO}
\textbf{\pnua{i u-du ni pira hoogo}}
\pfra{il disparaît derrière la montagne}
\end{exemple}
\newline
\relationsémantique{Cf.}{\lien{}{u-pwa, u-du-pwa [GOs]}}
\glosecourte{se baisser pour sortir (pwa: dehors)}
\end{entrée}

\begin{entrée}{ui}{1}{ⓔuiⓗ1}
\région{GOs PA BO}
\classe{v}
\newline
\sens{1}
(\domainesémantique{Feu : objets et actions liés au feu})
\begin{glose}
\pfra{souffler (sur le feu)}
\end{glose}
\newline
\begin{exemple}
\région{BO PA}
\textbf{\pnua{i ui yaai}}
\pfra{elle ravive / attise le feu en soufflant}
\end{exemple}
\newline
\begin{sous-entrée}{ba-u}{ⓔuiⓗ1ⓢ1ⓝba-u}
\begin{glose}
\pfra{éventail pour attiser le feu}
\end{glose}
\end{sous-entrée}
\newline
\begin{sous-entrée}{u-pwai [GOs], u-pwaim [PA]}{ⓔuiⓗ1ⓢ1ⓝu-pwai [GOs], u-pwaim [PA]}
\begin{glose}
\pfra{fumer}
\end{glose}
\end{sous-entrée}
\newline
\sens{2}
(\domainesémantique{Remèdes, médecine})
\begin{glose}
\pfra{souffler des feuilles médicinales (pour guérir)}
\end{glose}
\newline
\begin{sous-entrée}{ba-u}{ⓔuiⓗ1ⓢ2ⓝba-u}
\begin{glose}
\pfra{feuilles médicinales (qu'on souffle pour guérir)}
\end{glose}
\end{sous-entrée}
\end{entrée}

\begin{entrée}{ui}{2}{ⓔuiⓗ2}
\région{GOs}
\variante{%
wi
}
(\domainesémantique{Prépositions})
\classe{PREP}
\begin{glose}
\pfra{au sujet de ; envers ; à propos de (+ animés préférentiellement); à cause de}
\end{glose}
\newline
\begin{exemple}
\textbf{\pnua{e kiija ui ãbaa-je xo êmwê e}}
\pfra{cet homme est jaloux de son frère}
\end{exemple}
\newline
\begin{exemple}
\textbf{\pnua{e pana-nu ui çö}}
\pfra{il m'a grondé à cause de toi}
\end{exemple}
\newline
\begin{exemple}
\textbf{\pnua{e thuvwu vhaa ui je xo Hiixe çai Kaavo}}
\pfra{Hiixe parle d'elle-même à Kaavo}
\end{exemple}
\newline
\begin{exemple}
\textbf{\pnua{e thô ui nu}}
\pfra{il en colère envers / après moi}
\end{exemple}
\newline
\begin{exemple}
\textbf{\pnua{e thô pune nu}}
\pfra{il en colère à cause de moi}
\end{exemple}
\newline
\begin{exemple}
\textbf{\pnua{pune mee-dree (mais *ui mee-dree)}}
\pfra{à cause du temps}
\end{exemple}
\newline
\relationsémantique{Cf.}{\lien{}{pexa, pune}}
\glosecourte{à cause de}
\end{entrée}

\begin{entrée}{ui-bööni}{}{ⓔui-bööni}
\région{GOs PA BO}
(\domainesémantique{Feu : objets et actions liés au feu})
\classe{v}
\begin{glose}
\pfra{éteindre en soufflant (bougie, allumette)}
\end{glose}
\newline
\begin{exemple}
\région{GO}
\textbf{\pnua{i ui-bööni ya-phao}}
\pfra{elle éteint l'allumette en soufflant}
\end{exemple}
\end{entrée}

\begin{entrée}{ui gò}{}{ⓔui gò}
\région{GOs}
(\domainesémantique{Musique, instruments de musique})
\classe{v}
\begin{glose}
\pfra{jouer de la flûte}
\end{glose}
\newline
\begin{sous-entrée}{gòò-ui}{ⓔui gòⓝgòò-ui}
\begin{glose}
\pfra{flûte}
\end{glose}
\end{sous-entrée}
\end{entrée}

\begin{entrée}{uilu}{}{ⓔuilu}
\région{GOs WEM}
\variante{%
uvilu
\région{PA}}
(\domainesémantique{Oiseaux})
\newline
\sens{1}
\classe{nom}
\begin{glose}
\pfra{échenilleur Pie (donne des nouvelles)}
\end{glose}
\nomscientifique{Lalage leucopygia simillima (Campéphagidés)}
\newline
\sens{2}
\classe{nom}
\begin{glose}
\pfra{fauvette à ventre jaune (Wapipi)}
\end{glose}
\end{entrée}

\begin{entrée}{ul}{}{ⓔul}
\région{PA BO}
\classe{v}
\newline
\sens{1}
(\domainesémantique{Caractéristiques et propriétés des personnes})
\begin{glose}
\pfra{pauvre ; indigent}
\end{glose}
\newline
\begin{exemple}
\région{BO}
\textbf{\pnua{êgu ul}}
\pfra{un pauvre}
\end{exemple}
\newline
\sens{2}
(\domainesémantique{Modalité, verbes modaux})
\begin{glose}
\pfra{inutile}
\end{glose}
\begin{glose}
\pfra{gaspiller [BO]}
\end{glose}
\newline
\begin{exemple}
\textbf{\pnua{pe-ul}}
\pfra{ce n'est rien, ce n'est pas grave}
\end{exemple}
\newline
\begin{exemple}
\région{BO}
\textbf{\pnua{i ul}}
\pfra{il n'est bon à rien}
\end{exemple}
\newline
\begin{exemple}
\région{BO}
\textbf{\pnua{dili xa i ul}}
\pfra{une mauvaise terre, terre laissée à l'abandon}
\end{exemple}
\end{entrée}

\begin{entrée}{ula}{1}{ⓔulaⓗ1}
\région{GOs BO}
\classe{v.t.}
\newline
\sens{1}
(\domainesémantique{Verbes d'action (en général)})
\begin{glose}
\pfra{chasser ; éloigner (des insectes)}
\end{glose}
\newline
\sens{2}
(\domainesémantique{Mouvements ou actions faits avec le corps, les bras, les mains, les pieds})
\begin{glose}
\pfra{éventer (s')}
\end{glose}
\newline
\begin{exemple}
\région{GOs}
\textbf{\pnua{nu ula-nu}}
\pfra{je m'évente}
\end{exemple}
\newline
\begin{sous-entrée}{ba-ula}{ⓔulaⓗ1ⓢ2ⓝba-ula}
\région{GOs}
\begin{glose}
\pfra{éventail}
\end{glose}
\end{sous-entrée}
\newline
\begin{sous-entrée}{ba-ul}{ⓔulaⓗ1ⓢ2ⓝba-ul}
\région{PA BO}
\begin{glose}
\pfra{éventail}
\end{glose}
\end{sous-entrée}
\newline
\sens{3}
(\domainesémantique{Feu : objets et actions liés au feu})
\begin{glose}
\pfra{attiser (en faisant du vent)}
\end{glose}
\newline
\begin{exemple}
\région{PA BO}
\textbf{\pnua{i ula yaai}}
\pfra{elle attise le feu}
\end{exemple}
\newline
\étymologie{
\langue{PSO}
\étymon{*ura}
\auteur{Geraghty}}
\end{entrée}

\begin{entrée}{ula}{2}{ⓔulaⓗ2}
\région{GOs PA}
(\domainesémantique{Préfixes classificateurs numériques})
\classe{CLF.NUM}
\begin{glose}
\pfra{grappes de noix de coco}
\end{glose}
\newline
\begin{exemple}
\région{PA}
\textbf{\pnua{ula-xe ula-nu}}
\pfra{une grappe de noix de coco}
\end{exemple}
\end{entrée}

\begin{entrée}{ulavwi}{}{ⓔulavwi}
\région{GOs}
(\domainesémantique{Verbes d'action (en général)})
\classe{v}
\begin{glose}
\pfra{dégager ; faire place nette ; nettoyer (champ)}
\end{glose}
\newline
\begin{exemple}
\textbf{\pnua{e ulavwi dònò}}
\pfra{le ciel est totalement dégagé}
\end{exemple}
\end{entrée}

\begin{entrée}{ula yai}{}{ⓔula yai}
\région{GOs PA}
(\domainesémantique{Feu : objets et actions liés au feu})
\classe{v}
\begin{glose}
\pfra{attiser le feu avec éventail}
\end{glose}
\newline
\relationsémantique{Cf.}{\lien{ⓔuiⓗ1}{ui}}
\glosecourte{souffler}
\end{entrée}

\begin{entrée}{uli}{}{ⓔuli}
\région{PA WE BO}
\variante{%
uzi
\région{GOs}}
\classe{v}
(\domainesémantique{Actions liées aux plantes})
\begin{glose}
\pfra{lisser ; rendre lisse (tige)}
\end{glose}
\begin{glose}
\pfra{biseauter}
\end{glose}
\newline
\begin{exemple}
\région{PA}
\textbf{\pnua{uli u dom}}
\pfra{tailler en pointe}
\end{exemple}
\end{entrée}

\begin{entrée}{ulo}{}{ⓔulo}
\région{BO}
(\domainesémantique{Feu : objets et actions liés au feu})
\classe{v}
\begin{glose}
\pfra{brûler ; flamber}
\end{glose}
\newline
\note{mot non vérifié}{général}{}
\end{entrée}

\begin{entrée}{ulò}{1}{ⓔulòⓗ1}
\région{GOs PA BO}
(\domainesémantique{Insectes})
\classe{nom}
\begin{glose}
\pfra{sauterelle}
\end{glose}
\newline
\begin{sous-entrée}{egoo ulò}{ⓔulòⓗ1ⓝegoo ulò}
\begin{glose}
\pfra{larve de sauterelle}
\end{glose}
\end{sous-entrée}
\end{entrée}

\begin{entrée}{ulò}{2}{ⓔulòⓗ2}
\région{GOs}
(\domainesémantique{Poissons})
\classe{nom}
\begin{glose}
\pfra{poisson-volant}
\end{glose}
\end{entrée}

\begin{entrée}{ulu}{}{ⓔulu}
\région{GOs}
(\domainesémantique{Verbes de mouvement})
\classe{v}
\begin{glose}
\pfra{enfoncer (s')}
\end{glose}
\begin{glose}
\pfra{embourber (s')}
\end{glose}
\begin{glose}
\pfra{descendre verticalement}
\end{glose}
\newline
\begin{exemple}
\région{GOs}
\textbf{\pnua{e ulu a}}
\pfra{le soleil se couche}
\end{exemple}
\newline
\begin{exemple}
\textbf{\pnua{ulu loto}}
\pfra{la voiture s'est embourbée}
\end{exemple}
\end{entrée}

\begin{entrée}{umau}{}{ⓔumau}
\région{BO}
(\domainesémantique{Mouvements ou actions faits avec le corps, les bras, les mains, les pieds})
\classe{v}
\begin{glose}
\pfra{tituber [Corne]}
\end{glose}
\newline
\note{mot non vérifié}{général}{}
\end{entrée}

\begin{entrée}{uâme}{}{ⓔuâme}
\région{PA}
(\domainesémantique{Fonctions intellectuelles})
\classe{v}
\begin{glose}
\pfra{ne pas voir (ce qui était évident ; comment a-t-on fait pour ne pas voir)}
\end{glose}
\end{entrée}

\begin{entrée}{ûne}{}{ⓔûne}
\région{GOs}
(\domainesémantique{Préparation des aliments; modes de préparation et de cuisson})
\classe{v}
\begin{glose}
\pfra{passer sur la flamme pour assouplir (feuille de bananier)}
\end{glose}
\newline
\begin{exemple}
\textbf{\pnua{nu ûne dròò-chaamwa}}
\pfra{je passe la feuille de bananier sur la flamme}
\end{exemple}
\end{entrée}

\begin{entrée}{u-pwa !}{}{ⓔu-pwa !}
\région{GOs PA}
\variante{%
u-vwa
\région{GO PA}}
(\domainesémantique{Injonction})
\classe{v}
\begin{glose}
\pfra{sors !; dehors !}
\end{glose}
\end{entrée}

\begin{entrée}{u-pwai}{}{ⓔu-pwai}
\région{GOs}
\variante{%
u-pwaim
\région{WEM}, 
pwai
\région{PA}}
(\domainesémantique{Tabac, actions liées au tabac})
\classe{v}
\begin{glose}
\pfra{fumer (tabac)}
\end{glose}
\newline
\begin{exemple}
\textbf{\pnua{u pwai}}
\pfra{il fume}
\end{exemple}
\newline
\relationsémantique{Cf.}{\lien{ⓔuⓗ1}{u}}
\glosecourte{souffler, fumer}
\newline
\relationsémantique{Cf.}{\lien{ⓔhovwo}{hovwo}}
\glosecourte{chiquer, manger}
\end{entrée}

\begin{entrée}{u-pwa, u-vwa}{}{ⓔu-pwa, u-vwa}
\région{GO}
(\domainesémantique{Verbes de déplacement et moyens de déplacement})
\classe{DIR}
\begin{glose}
\pfra{aller vers l'extérieur}
\end{glose}
\end{entrée}

\begin{entrée}{uramõ}{}{ⓔuramõ}
\formephonétique{u'ramɔ̃}
\région{GOs}
(\domainesémantique{Religion, représentations religieuses})
\classe{nom}
\begin{glose}
\pfra{lutins ; petits génies}
\end{glose}
\newline
\relationsémantique{Cf.}{\lien{}{jee [PA]}}
\glosecourte{lutins}
\end{entrée}

\begin{entrée}{urîni}{}{ⓔurîni}
\région{PA}
(\domainesémantique{Soins du corps
, Mouvements ou actions faits avec le corps, les bras, les mains, les pieds})
\classe{v}
\begin{glose}
\pfra{frotter}
\end{glose}
\begin{glose}
\pfra{masser}
\end{glose}
\begin{glose}
\pfra{frictionner (avec des plantes ou des pommades)}
\end{glose}
\end{entrée}

\begin{entrée}{u ru}{}{ⓔu ru}
\région{WE PA}
\variante{%
u ru, ro
}
(\domainesémantique{Temps})
\classe{FUT}
\begin{glose}
\pfra{futur proche}
\end{glose}
\newline
\begin{exemple}
\région{PA}
\textbf{\pnua{u ru tòn}}
\pfra{il va faire nuit}
\end{exemple}
\newline
\begin{exemple}
\textbf{\pnua{u ru pwal ha ka uru pwal?}}
\pfra{va-t-il pleuvoir ou non?}
\end{exemple}
\newline
\begin{exemple}
\région{BO}
\textbf{\pnua{u ru còge yo !}}
\pfra{tu vas te couper ! [BM]}
\end{exemple}
\newline
\begin{exemple}
\région{PA}
\textbf{\pnua{u ru pwal !}}
\pfra{il va pleuvoir}
\end{exemple}
\newline
\relationsémantique{Cf.}{\lien{ⓔru}{ru}}
\end{entrée}

\begin{entrée}{uu}{}{ⓔuu}
\région{PA BO}
(\domainesémantique{Actions liées aux plantes})
\classe{v}
\begin{glose}
\pfra{cueillir (fleur, feuilles, bourgeons)}
\end{glose}
\newline
\begin{exemple}
\région{PA}
\textbf{\pnua{la uu do-ce}}
\pfra{elles cueillent des feuilles}
\end{exemple}
\newline
\begin{exemple}
\région{BO}
\textbf{\pnua{la uu mu-ce}}
\pfra{elles cueillent des fleurs}
\end{exemple}
\end{entrée}

\begin{entrée}{ûû}{}{ⓔûû}
\région{GO}
\variante{%
ôô
}
(\domainesémantique{Interjection})
\classe{INTJ}
\begin{glose}
\pfra{hmm (rythme le discours de quelqu'un)}
\end{glose}
\end{entrée}

\begin{entrée}{ûûni}{}{ⓔûûni}
\région{PA}
(\domainesémantique{Verbes d'action (en général)})
\classe{v}
\begin{glose}
\pfra{faire osciller les arbres (vent)}
\end{glose}
\begin{glose}
\pfra{faire tomber (qqch sur pied)}
\end{glose}
\newline
\begin{exemple}
\région{PA}
\textbf{\pnua{i ûûni ce u dèèn}}
\pfra{le vent a fait tomber l'arbre}
\end{exemple}
\end{entrée}

\begin{entrée}{uva}{}{ⓔuva}
\région{GOs BO}
(\domainesémantique{Taros})
\classe{nom}
\begin{glose}
\pfra{taro (pied de) d'eau (nom générique)}
\end{glose}
\newline
\begin{sous-entrée}{dròò-uva}{ⓔuvaⓝdròò-uva}
\région{GOs}
\begin{glose}
\pfra{feuille de taro d'eau}
\end{glose}
\end{sous-entrée}
\newline
\begin{sous-entrée}{kê-uva}{ⓔuvaⓝkê-uva}
\begin{glose}
\pfra{champ de taro}
\end{glose}
\end{sous-entrée}
\newline
\begin{sous-entrée}{zò-uva}{ⓔuvaⓝzò-uva}
\begin{glose}
\pfra{rejet de taro annexes (poussent autour de la tige principale)}
\end{glose}
\end{sous-entrée}
\end{entrée}

\begin{entrée}{uvilu}{}{ⓔuvilu}
\région{PA}
(\domainesémantique{Oiseaux})
\classe{nom}
\begin{glose}
\pfra{petit lève-queue [PA]}
\end{glose}
\nomscientifique{Rhipidura fuliginosa (Rhipiduridés)}
\end{entrée}

\begin{entrée}{uvo-ê}{}{ⓔuvo-ê}
\région{GOs PA}
(\domainesémantique{Cultures, techniques, boutures})
\classe{nom}
\begin{glose}
\pfra{bouture de canne à sucre}
\end{glose}
\end{entrée}

\begin{entrée}{uvo-uva}{}{ⓔuvo-uva}
\région{GOs}
(\domainesémantique{Taros
, Cultures, techniques, boutures})
\classe{nom}
\begin{glose}
\pfra{bouture de taro (pédoncule de taro muni d'une tige)}
\end{glose}
\begin{glose}
\pfra{pied de taro}
\end{glose}
\newline
\étymologie{
\langue{POc}
\étymon{*upe}
\glosecourte{plante}}
\end{entrée}

\begin{entrée}{uvwa}{}{ⓔuvwa}
\formephonétique{uβa}
\région{GOs}
(\domainesémantique{Relations et interaction sociales})
\classe{v ; n}
\begin{glose}
\pfra{accepter la demande de pardon}
\end{glose}
\begin{glose}
\pfra{nom du don pour la la demande de pardon}
\end{glose}
\newline
\relationsémantique{Cf.}{\lien{ⓔtaçuuni}{taçuuni}}
\glosecourte{refuser la demande de pardon}
\end{entrée}

\begin{entrée}{uvwi}{}{ⓔuvwi}
\formephonétique{uβi}
\région{GOs}
\variante{%
upi
\région{GO(s)}, 
uvi
\région{PA BO}}
(\domainesémantique{Dons, échanges, achat et vente, vol})
\classe{v ; n}
\begin{glose}
\pfra{acheter ; payer}
\end{glose}
\begin{glose}
\pfra{prix ; salaire}
\end{glose}
\newline
\begin{exemple}
\textbf{\pnua{e u uvwi xo kêê-nu axe-êgu po-mã na kòlò Pwayili}}
\pfra{mon père a acheté 20 mangues à Pwayili}
\end{exemple}
\newline
\begin{exemple}
\textbf{\pnua{e u uvwi xo kêê-nu na kòlò Pwayili axe-êgu po-mã}}
\pfra{mon père a acheté 20 mangues à Pwayili}
\end{exemple}
\newline
\begin{exemple}
\région{BO}
\textbf{\pnua{nu uvi yaa hi-n}}
\pfra{je le lui ai acheté}
\end{exemple}
\newline
\begin{exemple}
\région{BO}
\textbf{\pnua{pwaalu uvi-n}}
\pfra{c'est cher}
\end{exemple}
\newline
\relationsémantique{Ant.}{\lien{}{içu, iyu}}
\glosecourte{vendre}
\end{entrée}

\begin{entrée}{uvwia}{}{ⓔuvwia}
\formephonétique{uβia}
\région{GOs PA BO}
\variante{%
upia, uvhia
\région{PA BO}}
(\domainesémantique{Taros})
\classe{nom}
\begin{glose}
\pfra{taro (de montagne, nom du tubercule ou du pied de taro)}
\end{glose}
\nomscientifique{Colocasia esculenta (Aracées)}
\newline
\begin{sous-entrée}{uvwia kari}{ⓔuvwiaⓝuvwia kari}
\begin{glose}
\pfra{taro jaune}
\end{glose}
\newline
\relationsémantique{Cf.}{\lien{ⓔuvwa}{uvwa}}
\glosecourte{pied de taro d'eau}
\newline
\relationsémantique{Cf.}{\lien{}{kutru, kuru}}
\glosecourte{tubercule de taro d'eau}
\end{sous-entrée}
\end{entrée}

\begin{entrée}{uxi}{}{ⓔuxi}
\région{GOs}
(\domainesémantique{Mouvements ou actions faits avec le corps, les bras, les mains, les pieds})
\classe{v}
\begin{glose}
\pfra{percer ; faire des trous; ronger}
\end{glose}
\newline
\begin{sous-entrée}{ba-uxi ce}{ⓔuxiⓝba-uxi ce}
\begin{glose}
\pfra{perceuse à bois}
\end{glose}
\end{sous-entrée}
\end{entrée}

\begin{entrée}{uxo}{}{ⓔuxo}
\région{GO}
(\domainesémantique{Noms des plantes})
\classe{nom}
\begin{glose}
\pfra{Crinum sp.}
\end{glose}
\nomscientifique{Crinum sp.}
\end{entrée}

\begin{entrée}{uza}{}{ⓔuza}
\région{GOs}
\variante{%
ula
\région{PA BO}}
(\domainesémantique{Relations et interaction sociales})
\classe{v}
\begin{glose}
\pfra{moquer de (se)}
\end{glose}
\newline
\begin{exemple}
\région{GOs}
\textbf{\pnua{la uza-nu}}
\pfra{ils se moquent de moi}
\end{exemple}
\newline
\begin{exemple}
\région{PA}
\textbf{\pnua{la uza-je}}
\pfra{ils se moquent de lui}
\end{exemple}
\newline
\begin{exemple}
\région{BO}
\textbf{\pnua{la ula-je}}
\pfra{ils se moquent de lui}
\end{exemple}
\end{entrée}

\begin{entrée}{uzi}{}{ⓔuzi}
\région{GOs}
\variante{%
uli
\région{PA WE BO}}
\classe{v}
\newline
\sens{1}
(\domainesémantique{Tressage
, Actions liées aux plantes})
\begin{glose}
\pfra{assouplir les fibres en les lissant}
\end{glose}
\begin{glose}
\pfra{lisser (la tige)}
\end{glose}
\newline
\begin{exemple}
\textbf{\pnua{e uzi ce}}
\pfra{il enlève les fibres en lissant la tige}
\end{exemple}
\newline
\sens{2}
(\domainesémantique{Travail bois
, Actions avec un instrument, un outil})
\begin{glose}
\pfra{raboter ; lisser}
\end{glose}
\newline
\begin{exemple}
\textbf{\pnua{e uzi ce}}
\pfra{il rabote le bout de bois}
\end{exemple}
\end{entrée}

\begin{entrée}{uzi cii}{}{ⓔuzi cii}
\région{GOs}
(\domainesémantique{Actions liées aux plantes})
\classe{v}
\begin{glose}
\pfra{enlever les fibres (en lissant avec un couteau)}
\end{glose}
\end{entrée}

\begin{entrée}{uzi dròò}{}{ⓔuzi dròò}
\région{GOs}
\variante{%
uli doo
\région{PA}}
(\domainesémantique{Actions liées aux plantes})
\classe{v}
\begin{glose}
\pfra{effeuiller une tige}
\end{glose}
\end{entrée}

\newpage

\lettrine{v}\begin{entrée}{va}{}{ⓔva}
\région{GOs}
(\domainesémantique{Préfixes classificateurs de la nourriture
, Préfixes classificateurs possessifs de la nourriture})
\classe{nom}
\begin{glose}
\pfra{part de nourriture donnée dans les coutumes}
\end{glose}
\newline
\note{vangi-nu}{grammaire}{ma part d'igname}
\end{entrée}

\begin{entrée}{va ?}{}{ⓔva ?}
\région{WEM BO}
(\domainesémantique{Interrogatifs})
\classe{INT (statique)}
\begin{glose}
\pfra{où?}
\end{glose}
\newline
\begin{exemple}
\région{BO}
\textbf{\pnua{yo ègu va ?}}
\pfra{d'où es-tu ?}
\end{exemple}
\newline
\begin{exemple}
\région{BO}
\textbf{\pnua{ge va la mu-ce ?}}
\pfra{où sont les fleurs ?}
\end{exemple}
\newline
\relationsémantique{Cf.}{\lien{}{iva?}}
\glosecourte{où?}
\newline
\étymologie{
\langue{POc}
\étymon{*pe}
\glosecourte{où?}}
\end{entrée}

\begin{entrée}{vaaci}{}{ⓔvaaci}
\région{PA BO WEM}
(\domainesémantique{Mammifères})
\classe{nom}
\begin{glose}
\pfra{bétail ; vache}
\end{glose}
\newline
\emprunt{vache (FR)}
\end{entrée}

\begin{entrée}{vajama}{}{ⓔvajama}
\formephonétique{va'ɲɟama}
\région{GOs PA BO}
\variante{%
fhajama
\région{GO(s)}}
(\domainesémantique{Tradition orale})
\classe{nom}
\begin{glose}
\pfra{mythe d'origine ; conte}
\end{glose}
\end{entrée}

\begin{entrée}{vala}{}{ⓔvala}
\région{PA}
(\domainesémantique{Verbes d'action (en général)})
\classe{v}
\begin{glose}
\pfra{ajouter}
\end{glose}
\newline
\begin{exemple}
\région{PA}
\textbf{\pnua{mwa ruma vala-ni mwa jena}}
\pfra{nous ajouterons cela}
\end{exemple}
\end{entrée}

\begin{entrée}{valèèma}{}{ⓔvalèèma}
\région{GOs}
(\domainesémantique{Description des objets, formes, consistance, taille})
\classe{v.stat.}
\begin{glose}
\pfra{lisse}
\end{glose}
\end{entrée}

\begin{entrée}{vali}{}{ⓔvali}
\région{GOs PA}
(\domainesémantique{Cultures, techniques, boutures})
\classe{nom}
\begin{glose}
\pfra{pierre constituant un passage/pont}
\end{glose}
\newline
\note{ce pont couvre une conduite d'eau (de we) amenant l'eau d'une tarodière à l'autre}{glose}{}
\end{entrée}

\begin{entrée}{vana}{}{ⓔvana}
\région{GOs}
(\domainesémantique{Types de champs
, Taros})
\classe{nom}
\begin{glose}
\pfra{tarodière en terrasse et irriguée (de taro d'eau)}
\end{glose}
\newline
\begin{exemple}
\textbf{\pnua{thu vana}}
\pfra{faire la tarodière en terrasse}
\end{exemple}
\end{entrée}

\begin{entrée}{vara}{}{ⓔvara}
\région{GOs PA}
(\domainesémantique{Distributifs})
\classe{dispersif ; DISTR}
\begin{glose}
\pfra{séparément ; chacun}
\end{glose}
\newline
\begin{exemple}
\région{GOs}
\textbf{\pnua{la vara a}}
\pfra{ils partent chacun de leur côté}
\end{exemple}
\newline
\begin{exemple}
\région{GOs}
\textbf{\pnua{mô vara pu phò-ã}}
\pfra{nous avons chacunnotre charge}
\end{exemple}
\newline
\begin{exemple}
\région{GOs}
\textbf{\pnua{li vara a ni dee-li}}
\pfra{ils partent chacun sur leur chemin}
\end{exemple}
\newline
\begin{exemple}
\région{GOs}
\textbf{\pnua{li vara phe 1000F xo ãbaa-nu mã nata}}
\pfra{mon frère et le pasteur ont pris 1000F chacun}
\end{exemple}
\newline
\begin{exemple}
\région{GOs}
\textbf{\pnua{ãbaa-nu mã nata, li vara phe 1000F}}
\pfra{mon frère et le pasteur, ils ont pris 1000F chacun}
\end{exemple}
\newline
\begin{exemple}
\région{GOs}
\textbf{\pnua{li vara uvwi loto ka we-xe}}
\pfra{ils ont chacun acheté une voiture}
\end{exemple}
\newline
\begin{exemple}
\région{GOs}
\textbf{\pnua{bi vara kibaò a-kò bò mã ãbaa-nu}}
\pfra{mes frères et moi avons tué 3 roussettes chacun}
\end{exemple}
\newline
\begin{exemple}
\région{GO}
\textbf{\pnua{vara Poimenya õã-lò}}
\pfra{leurs mères sont chacune des Poymegna}
\end{exemple}
\end{entrée}

\begin{entrée}{varan}{}{ⓔvaran}
\région{PA}
(\domainesémantique{Types de maison, architecture de la maison})
\classe{nom}
\begin{glose}
\pfra{véranda}
\end{glose}
\newline
\emprunt{véranda (FR)}
\end{entrée}

\begin{entrée}{varû}{}{ⓔvarû}
\région{BO}
(\domainesémantique{Mouvements ou actions faits avec le corps, les bras, les mains, les pieds})
\classe{v}
\begin{glose}
\pfra{tripoter [BM]}
\end{glose}
\end{entrée}

\begin{entrée}{vauma}{}{ⓔvauma}
\région{GOs BO}
(\domainesémantique{Pierre, roche})
\classe{nom}
\begin{glose}
\pfra{pierre-ponce}
\end{glose}
\end{entrée}

\begin{entrée}{vaxaròò}{}{ⓔvaxaròò}
\région{GOs PA BO}
(\domainesémantique{Coraux})
\classe{nom}
\begin{glose}
\pfra{patate de corail (blanc)}
\end{glose}
\begin{glose}
\pfra{caillou ressemblant à une patate de corail}
\end{glose}
\newline
\relationsémantique{Cf.}{\lien{ⓔkaròò}{karòò}}
\glosecourte{corail}
\end{entrée}

\begin{entrée}{vea}{}{ⓔvea}
\région{GOs}
(\domainesémantique{Matière, matériaux})
\classe{nom}
\begin{glose}
\pfra{verre ; vitre}
\end{glose}
\newline
\emprunt{verre (FR)}
\end{entrée}

\begin{entrée}{vee}{}{ⓔvee}
\région{GOs WEM}
(\domainesémantique{Discours, échanges verbaux})
\classe{nom}
\begin{glose}
\pfra{discussion pour savoir comment procéder pour la coutume}
\end{glose}
\newline
\begin{sous-entrée}{mo-vee}{ⓔveeⓝmo-vee}
\begin{glose}
\pfra{maison pour discuter comment procéder pour la coutume}
\end{glose}
\end{sous-entrée}
\end{entrée}

\begin{entrée}{vèlè}{}{ⓔvèlè}
\région{GOs PA}
\variante{%
baalaba
\région{PA BO}}
(\domainesémantique{Objets et meubles de la maison})
\classe{nom}
\begin{glose}
\pfra{lit}
\end{glose}
\newline
\begin{exemple}
\textbf{\pnua{thô bwa vèlè}}
\pfra{drap de lit}
\end{exemple}
\end{entrée}

\begin{entrée}{vi}{}{ⓔvi}
\région{GOs}
(\domainesémantique{Poissons})
\classe{nom}
\begin{glose}
\pfra{carangue}
\end{glose}
\end{entrée}

\begin{entrée}{vijang}{}{ⓔvijang}
\région{PA BO}
\classe{nom}
\newline
\sens{1}
(\domainesémantique{Objets coutumiers})
\begin{glose}
\pfra{bouquet de fibres (accrochés à un piquet pour marquer un interdit ou montrer qu'une plante est réservée)}
\end{glose}
\newline
\begin{exemple}
\région{PA}
\textbf{\pnua{tha vijang}}
\pfra{attacher un bouquet de fibres}
\end{exemple}
\newline
\begin{exemple}
\région{BO}
\textbf{\pnua{i thu vijang}}
\pfra{attacher un bouquet de fibres}
\end{exemple}
\newline
\sens{2}
(\domainesémantique{Objets coutumiers})
\begin{glose}
\pfra{noeud (fait sur une herbe, signale un interdit, protège des magies)}
\end{glose}
\end{entrée}

\newpage

\lettrine{vh}\begin{entrée}{vhaa}{}{ⓔvhaa}
\région{GOs PA BO}
\variante{%
fhaa
\région{GA}}
(\domainesémantique{Discours, échanges verbaux})
\classe{v ; n}
\begin{glose}
\pfra{parler ; parole ; voix}
\end{glose}
\newline
\begin{exemple}
\région{GO}
\textbf{\pnua{vhaa i je}}
\pfra{sa voix}
\end{exemple}
\newline
\begin{exemple}
\région{PA}
\textbf{\pnua{ge le vhaa}}
\pfra{il y a une nouvelle, information}
\end{exemple}
\newline
\begin{exemple}
\région{BO}
\textbf{\pnua{la vhaa ni yuanga}}
\pfra{ils parlent yuanga}
\end{exemple}
\newline
\begin{exemple}
\région{BO}
\textbf{\pnua{la vhaa na dòò-ce}}
\pfra{ils parlent de ces feuilles}
\end{exemple}
\newline
\begin{exemple}
\région{PA}
\textbf{\pnua{vhaa i la}}
\pfra{leurs voix}
\end{exemple}
\newline
\begin{exemple}
\région{BO}
\textbf{\pnua{vhaa caaxò}}
\pfra{parler doucement, murmurer}
\end{exemple}
\newline
\begin{exemple}
\région{GA}
\textbf{\pnua{fhaa whamã}}
\pfra{la parole des anciens}
\end{exemple}
\newline
\begin{exemple}
\région{GA}
\textbf{\pnua{nu fa cai-co}}
\pfra{je te parle}
\end{exemple}
\newline
\begin{sous-entrée}{vhaa baxo}{ⓔvhaaⓝvhaa baxo}
\begin{glose}
\pfra{parler juste}
\end{glose}
\end{sous-entrée}
\newline
\begin{sous-entrée}{fhaa ka kugo}{ⓔvhaaⓝfhaa ka kugo}
\begin{glose}
\pfra{parole véritable}
\end{glose}
\end{sous-entrée}
\newline
\begin{sous-entrée}{khôbwe vhaa-raa}{ⓔvhaaⓝkhôbwe vhaa-raa}
\région{PA}
\begin{glose}
\pfra{dire des grossièretés}
\end{glose}
\end{sous-entrée}
\newline
\begin{sous-entrée}{paxa-vhaa}{ⓔvhaaⓝpaxa-vhaa}
\begin{glose}
\pfra{mot}
\end{glose}
\end{sous-entrée}
\newline
\begin{sous-entrée}{pu-fhaa}{ⓔvhaaⓝpu-fhaa}
\begin{glose}
\pfra{cause, origine du discours}
\end{glose}
\end{sous-entrée}
\end{entrée}

\begin{entrée}{vhaa caaxò}{}{ⓔvhaa caaxò}
\région{BO}
(\domainesémantique{Discours, échanges verbaux})
\classe{v}
\begin{glose}
\pfra{parler doucement, murmurer}
\end{glose}
\end{entrée}

\begin{entrée}{vhaa draalae}{}{ⓔvhaa draalae}
\région{GOs}
\variante{%
vhaa daleen
\région{PA}}
(\domainesémantique{Discours, échanges verbaux})
\classe{nom}
\begin{glose}
\pfra{français}
\end{glose}
\end{entrée}

\begin{entrée}{vhaa-raa}{}{ⓔvhaa-raa}
\région{GOs PA}
(\domainesémantique{Relations et interaction sociales})
\classe{v}
\begin{glose}
\pfra{jurer (lit. parler mauvais)}
\end{glose}
\begin{glose}
\pfra{dire des gros mots ; médire}
\end{glose}
\end{entrée}

\begin{entrée}{vhaa-zo}{}{ⓔvhaa-zo}
\région{GOs}
(\domainesémantique{Discours, échanges verbaux})
\classe{v}
\begin{glose}
\pfra{parler clairement}
\end{glose}
\newline
\relationsémantique{Ant.}{\lien{}{vhaa-ra}}
\glosecourte{parler mal}
\end{entrée}

\begin{entrée}{vhaò}{}{ⓔvhaò}
\région{GOs}
\variante{%
fhaò
\région{GA}}
(\domainesémantique{Instruments})
\classe{nom}
\begin{glose}
\pfra{fil de fer (barbelé)}
\end{glose}
\end{entrée}

\begin{entrée}{vhii}{}{ⓔvhii}
\région{GOs BO}
\variante{%
vii
\région{GO(s) BO}, 
fhi
\région{GO(s)}}
(\domainesémantique{Fonctions naturelles humaines})
\classe{v}
\begin{glose}
\pfra{péter (grossier)}
\end{glose}
\newline
\relationsémantique{Cf.}{\lien{}{tabawi (+ poli)}}
\glosecourte{avoir des vents}
\end{entrée}

\begin{entrée}{vhiliçô}{}{ⓔvhiliçô}
\formephonétique{vilidʒõ}
\région{GOs}
(\domainesémantique{Santé, maladie})
\classe{nom}
\begin{glose}
\pfra{bourbouille}
\end{glose}
\end{entrée}

\newpage

\lettrine{vw}\begin{entrée}{vwo}{1}{ⓔvwoⓗ1}
\région{BO}
(\domainesémantique{Prépositions})
\classe{PREP (objet indirect)}
\begin{glose}
\pfra{objet indirect}
\end{glose}
\newline
\begin{exemple}
\textbf{\pnua{i nonomi vwo Bwode}}
\pfra{il pense à Bondé}
\end{exemple}
\newline
\note{non vérifié}{général}{}
\end{entrée}

\begin{entrée}{vwo}{2}{ⓔvwoⓗ2}
\région{GOs PA}
\variante{%
vo
\région{BO}}
(\domainesémantique{Structure informationnelle})
\classe{FOC}
\begin{glose}
\pfra{focus}
\end{glose}
\newline
\begin{exemple}
\région{GOs}
\textbf{\pnua{nu nõõli vwo/xo inu}}
\pfra{c'est moi qui l'ai vu}
\end{exemple}
\newline
\begin{exemple}
\région{GOs}
\textbf{\pnua{nu nõõli xo inu}}
\pfra{c'est moi qui le vois (inanimé)}
\end{exemple}
\newline
\begin{exemple}
\région{GOs}
\textbf{\pnua{nu nõõ-je xo inu}}
\pfra{c'est moi qui le vois (animé)}
\end{exemple}
\newline
\begin{exemple}
\région{BO}
\textbf{\pnua{ti je i a ? - nu a vo nu}}
\pfra{qui part ? - c'est moi qui pars}
\end{exemple}
\newline
\begin{exemple}
\région{PA}
\textbf{\pnua{ti-xa na i a ? - nu a u/vwo inu}}
\pfra{qui est-ce qui estparti ? - c'est moi qui suis parti (xa : indéfini)}
\end{exemple}
\newline
\begin{exemple}
\région{PA}
\textbf{\pnua{ti je i a ? - nu a u/vwo inu}}
\pfra{qui estparti ?- c'est moi qui suis parti}
\end{exemple}
\newline
\note{xo (est parfois utilisé avec la même fonction)}{grammaire}{}
\end{entrée}

\begin{entrée}{-vwò}{}{ⓔ-vwò}
\région{GOs PA BO}
(\domainesémantique{Dérivation})
\classe{SUFF.NMLZ}
\begin{glose}
\pfra{nominalisateur ; saturateur de transitivité}
\end{glose}
\newline
\begin{exemple}
\région{GOs}
\textbf{\pnua{mo a nhii-vwò}}
\pfra{nous allons faire la cueillette}
\end{exemple}
\newline
\begin{exemple}
\région{PA}
\textbf{\pnua{phee-vwò !}}
\pfra{servez-vous !}
\end{exemple}
\newline
\begin{sous-entrée}{na-vwò}{ⓔ-vwòⓝna-vwò}
\begin{glose}
\pfra{cadeau (donner) ; coutumes}
\end{glose}
\end{sous-entrée}
\newline
\begin{sous-entrée}{nèè-vwò}{ⓔ-vwòⓝnèè-vwò}
\begin{glose}
\pfra{acte ; action (faire)}
\end{glose}
\end{sous-entrée}
\newline
\begin{sous-entrée}{hine-vwò}{ⓔ-vwòⓝhine-vwò}
\begin{glose}
\pfra{intelligent}
\end{glose}
\end{sous-entrée}
\newline
\begin{sous-entrée}{e tii-vwò}{ⓔ-vwòⓝe tii-vwò}
\begin{glose}
\pfra{il écrit}
\end{glose}
\end{sous-entrée}
\newline
\begin{sous-entrée}{la na-vwò}{ⓔ-vwòⓝla na-vwò}
\begin{glose}
\pfra{ils font des dons, ils font la coutume}
\end{glose}
\newline
\relationsémantique{Cf.}{\lien{ⓔpoⓗ1}{po}}
\glosecourte{choses}
\end{sous-entrée}
\end{entrée}

\begin{entrée}{vwö}{1}{ⓔvwöⓗ1}
\région{GOs}
\variante{%
vwu
\région{BO PA}, 
wi
\région{BO}}
(\domainesémantique{Conjonction})
\classe{CNJ}
\begin{glose}
\pfra{pour que}
\end{glose}
\newline
\begin{exemple}
\région{GO}
\textbf{\pnua{kô-zo na cö whili-nu vwö nu nõõ-je ?}}
\pfra{Serait-il possible que tu m'emmènes, pour que ja la voie ?}
\end{exemple}
\end{entrée}

\begin{entrée}{vwö}{2}{ⓔvwöⓗ2}
\région{GOs}
(\domainesémantique{Structure informationnelle})
\classe{THEM}
\begin{glose}
\pfra{quant à}
\end{glose}
\newline
\note{forme courte de novwö}{grammaire}{}
\end{entrée}

\newpage

\lettrine{w}\begin{entrée}{wa}{1}{ⓔwaⓗ1}
\région{GOs}
\variante{%
wal
\région{PA BO}, 
wòl
\région{WEM}}
\classe{nom}
\newline
\sens{1}
(\domainesémantique{Cordes, cordages})
\begin{glose}
\pfra{corde (général) ; ficelle}
\end{glose}
\begin{glose}
\pfra{liane ; lien}
\end{glose}
\newline
\begin{exemple}
\région{GO}
\textbf{\pnua{wa-bwa-nu}}
\pfra{mon serre-tête}
\end{exemple}
\newline
\begin{exemple}
\région{PA}
\textbf{\pnua{wa-bwa-ny}}
\pfra{mon serre-tête}
\end{exemple}
\newline
\begin{exemple}
\région{PA}
\textbf{\pnua{wa-gòò-hii-n}}
\pfra{son brassard}
\end{exemple}
\newline
\begin{sous-entrée}{wa-rama}{ⓔwaⓗ1ⓢ1ⓝwa-rama}
\région{GO}
\begin{glose}
\pfra{ceinture de soutien}
\end{glose}
\end{sous-entrée}
\newline
\begin{sous-entrée}{wa-kiò, wa-ki}{ⓔwaⓗ1ⓢ1ⓝwa-kiò, wa-ki}
\région{GO}
\begin{glose}
\pfra{ceinture de soutien ventral}
\end{glose}
\end{sous-entrée}
\newline
\begin{sous-entrée}{gu-wa-alaxò}{ⓔwaⓗ1ⓢ1ⓝgu-wa-alaxò}
\région{GOs}
\begin{glose}
\pfra{lacet (gu: piquer comme une filoche)}
\end{glose}
\newline
\note{wazi-je [GO], wali-n [PA, BO]}{grammaire}{sa corde}
\newline
\note{pha-wal [PA]}{grammaire}{attacher}
\end{sous-entrée}
\newline
\sens{2}
(\domainesémantique{Corps humain})
\begin{glose}
\pfra{veine ; artère}
\end{glose}
\begin{glose}
\pfra{tendon}
\end{glose}
\newline
\begin{sous-entrée}{wal kamun}{ⓔwaⓗ1ⓢ2ⓝwal kamun}
\begin{glose}
\pfra{artères}
\end{glose}
\end{sous-entrée}
\newline
\begin{sous-entrée}{wa ni bwèèdrò}{ⓔwaⓗ1ⓢ2ⓝwa ni bwèèdrò}
\begin{glose}
\pfra{la veine du front (il faut toujours spécifier la partie du corps où se trouve la veine)}
\end{glose}
\end{sous-entrée}
\newline
\étymologie{
\langue{POc}
\étymon{*uRat}}
\end{entrée}

\begin{entrée}{wa}{2}{ⓔwaⓗ2}
\région{GO}
\région{PABO}
\variante{%
wal
}
(\domainesémantique{Parties de plantes})
\classe{nom}
\begin{glose}
\pfra{racine}
\end{glose}
\newline
\relationsémantique{Cf.}{\lien{ⓔweⓗ1}{we}}
\glosecourte{forme en composition ou détermination}
\newline
\étymologie{
\langue{POc}
\étymon{*wakaR}
\glosecourte{racine}}
\end{entrée}

\begin{entrée}{wa}{3}{ⓔwaⓗ3}
\région{BO [Corne]}
(\domainesémantique{Conjonction})
\classe{PREP ; CNJ}
\begin{glose}
\pfra{à cause de ; parce que}
\end{glose}
\newline
\begin{exemple}
\région{BO}
\textbf{\pnua{nu bware wa nyama}}
\pfra{je suis fatigué à cause de mon travail}
\end{exemple}
\newline
\note{mot non vérifié}{général}{}
\end{entrée}

\begin{entrée}{wa}{4}{ⓔwaⓗ4}
\région{GOs}
\variante{%
wal
\région{PA BO}}
(\domainesémantique{Fonctions naturelles humaines
, Musique, instruments de musique})
\classe{v ; n}
\newline
\begin{exemple}
\région{GOs}
\textbf{\pnua{la wa bulu}}
\pfra{ils chantent ensemble}
\end{exemple}
\newline
\begin{exemple}
\région{GOs}
\textbf{\pnua{e waze-zoo-ni wa nye}}
\pfra{il a bien chanté cette chanson}
\end{exemple}
\newline
\begin{exemple}
\région{GOs}
\textbf{\pnua{e wa-zo}}
\pfra{il chante bien}
\end{exemple}
\begin{glose}
\pfra{chanter ; chant}
\end{glose}
\newline
\note{v.t. waze}{grammaire}{}
\end{entrée}

\begin{entrée}{wa-}{}{ⓔwa-}
\région{GOs PA BO}
\variante{%
wan-
\région{BO PA}}
(\domainesémantique{Préfixes classificateurs numériques})
\classe{CLF.NUM}
\begin{glose}
\pfra{lot de 2 roussettes ou notous (lots cérémoniels)}
\end{glose}
\newline
\begin{exemple}
\région{GO}
\textbf{\pnua{wa-xe b(w)ò ko ido}}
\pfra{1 lot de 3 roussettes (une paire de roussettes et une demi paire)}
\end{exemple}
\newline
\begin{exemple}
\textbf{\pnua{wa-tru, wan-ru bwò}}
\pfra{2 lots de 2 roussettes}
\end{exemple}
\newline
\begin{exemple}
\textbf{\pnua{wa-tru bò ko ido}}
\pfra{2 lots de 2 roussettes plus une (deux paires de roussettes et une demi paire)}
\end{exemple}
\newline
\begin{exemple}
\textbf{\pnua{wa-ko bò}}
\pfra{3 lot de 2 roussettes}
\end{exemple}
\newline
\begin{exemple}
\textbf{\pnua{wa-pa bò}}
\pfra{4 lots de 2 roussettes}
\end{exemple}
\newline
\begin{exemple}
\textbf{\pnua{wa-truci bwò}}
\pfra{10 lots de 2 roussettes}
\end{exemple}
\newline
\begin{exemple}
\région{BO PA}
\textbf{\pnua{wan-xe, wa-tu, wa-kon, wa-pa, wa-nim}}
\pfra{1, 2 , 3, 4, 5 lots de 2 roussettes ou notous}
\end{exemple}
\newline
\relationsémantique{Cf.}{\lien{}{wa(l)}}
\glosecourte{lien}
\end{entrée}

\begin{entrée}{wã}{}{ⓔwã}
\région{GOs}
\variante{%
wãã
}
\classe{LOC.DIR}
(\domainesémantique{Prépositions})
\begin{glose}
\pfra{vers}
\end{glose}
\newline
\begin{exemple}
\région{GOs}
\textbf{\pnua{nu weena kôbwe e a-wãã-du ènè}}
\pfra{je pense qu'il est passé par là (en le montrant du doigt)}
\end{exemple}
\newline
\begin{exemple}
\région{GOs}
\textbf{\pnua{nu weena kôbwe e a-du ènè}}
\pfra{je pense qu'il est passé par là}
\end{exemple}
\newline
\begin{sous-entrée}{e a wãã-mi}{ⓔwãⓝe a wãã-mi}
\begin{glose}
\pfra{il est allé vers ici}
\end{glose}
\end{sous-entrée}
\newline
\begin{sous-entrée}{e a wãã-ò}{ⓔwãⓝe a wãã-ò}
\begin{glose}
\pfra{il est allé vers là-bas}
\end{glose}
\end{sous-entrée}
\newline
\begin{sous-entrée}{e a wãã-è}{ⓔwãⓝe a wãã-è}
\begin{glose}
\pfra{il est allé dans une direction transverse}
\end{glose}
\end{sous-entrée}
\newline
\begin{sous-entrée}{e a wãã-da}{ⓔwãⓝe a wãã-da}
\begin{glose}
\pfra{il est allé vers le haut}
\end{glose}
\end{sous-entrée}
\newline
\begin{sous-entrée}{e a wãã-du èbòli[* e a wãã-du bòli]}{ⓔwãⓝe a wãã-du èbòli[* e a wãã-du bòli]}
\begin{glose}
\pfra{il est allé vers le bas là-bas}
\end{glose}
\end{sous-entrée}
\newline
\begin{sous-entrée}{e nò-wã-du}{ⓔwãⓝe nò-wã-du}
\begin{glose}
\pfra{elle regarde vers le bas}
\end{glose}
\end{sous-entrée}
\newline
\begin{sous-entrée}{e nò-du}{ⓔwãⓝe nò-du}
\begin{glose}
\pfra{elle regarde en bas}
\end{glose}
\end{sous-entrée}
\end{entrée}

\begin{entrée}{waa}{}{ⓔwaa}
\région{BO [BM, Corne]}
\variante{%
wò
\région{GO(s)}}
(\domainesémantique{Mouvements ou actions avec la tête, les yeux, la bouche})
\classe{v}
\begin{glose}
\pfra{ouvrir la bouche}
\end{glose}
\end{entrée}

\begin{entrée}{wa-aazo}{}{ⓔwa-aazo}
\région{GOs PA}
\variante{%
wa-aazo ni kò
\région{PA WEM}, 
wa-ayo
\région{BO}}
(\domainesémantique{Corps humain})
\classe{nom}
\begin{glose}
\pfra{tendon d'achille (lit. le tendon du chef)}
\end{glose}
\end{entrée}

\begin{entrée}{waaçu}{}{ⓔwaaçu}
\formephonétique{waːdʒu waːʒu}
\région{GOs WEM WE}
\variante{%
waçuçu
\région{GO(s) WEM WE}, 
waayu, waaju
\région{PA BO}}
(\domainesémantique{Relations et interaction sociales})
\classe{v}
\begin{glose}
\pfra{efforcer de (s') ; persister à ; insister}
\end{glose}
\begin{glose}
\pfra{perséverer ; persévérant}
\end{glose}
\begin{glose}
\pfra{obliger}
\end{glose}
\newline
\begin{exemple}
\région{GOs}
\textbf{\pnua{me waaçu vwo me kila-xa whaya me tròòli xo mwani}}
\pfra{nous nous efforçons de chercher comment gagner de l'argent}
\end{exemple}
\newline
\begin{exemple}
\région{PA}
\textbf{\pnua{kò-waayu}}
\pfra{persister à}
\end{exemple}
\newline
\begin{exemple}
\textbf{\pnua{ha peve waaju kôbwe ha tooli de-ra mwani}}
\pfra{nous tous nous efforçons de trouver comment gagner de l'argent (lit. le chemin de l'argent)}
\end{exemple}
\end{entrée}

\begin{entrée}{wããdri}{}{ⓔwããdri}
\région{GOs}
(\domainesémantique{Poissons})
\classe{nom}
\begin{glose}
\pfra{brème bleue}
\end{glose}
\nomscientifique{Acanthopagrus berda (Sparidés)}
\end{entrée}

\begin{entrée}{waajô}{}{ⓔwaajô}
\région{PA}
(\domainesémantique{Relations et interaction sociales})
\classe{v}
\begin{glose}
\pfra{accroché à sa mère ; 'collant' (enfant)}
\end{glose}
\newline
\begin{exemple}
\région{PA}
\textbf{\pnua{i waajô kai jo}}
\pfra{il est collé à toi}
\end{exemple}
\end{entrée}

\begin{entrée}{waal}{}{ⓔwaal}
\région{PA BO}
\classe{v ; n}
\newline
\sens{1}
(\domainesémantique{Religion, représentations religieuses})
\begin{glose}
\pfra{religion}
\end{glose}
\newline
\begin{sous-entrée}{waala-â}{ⓔwaalⓢ1ⓝwaala-â}
\begin{glose}
\pfra{notre religion}
\end{glose}
\end{sous-entrée}
\newline
\sens{2}
(\domainesémantique{Fonctions naturelles humaines})
\begin{glose}
\pfra{chanter (personne, oiseau)}
\end{glose}
\end{entrée}

\begin{entrée}{wãã-na}{}{ⓔwãã-na}
\région{GOs PA BO}
(\domainesémantique{Manière de faire l’action : verbes et adverbes de manière})
\classe{v.COMPAR}
\begin{glose}
\pfra{faire ainsi ; être ainsi}
\end{glose}
\end{entrée}

\begin{entrée}{waang}{}{ⓔwaang}
\région{PA WEM BO}
(\domainesémantique{Découpage du temps})
\classe{nom}
\begin{glose}
\pfra{matin ; aube ; premières lueurs du jour}
\end{glose}
\newline
\begin{exemple}
\région{WEM}
\textbf{\pnua{u waang}}
\pfra{c'est l'aube}
\end{exemple}
\end{entrée}

\begin{entrée}{waara}{}{ⓔwaara}
\région{GOs PA BO}
\variante{%
waatra
\région{GO(s)}}
(\domainesémantique{Fonctions naturelles humaines})
\classe{v}
\begin{glose}
\pfra{grossir ; grandir ; croître (plantes)}
\end{glose}
\begin{glose}
\pfra{pousser (en longueur et en largeur)}
\end{glose}
\newline
\begin{sous-entrée}{pa-waara}{ⓔwaaraⓝpa-waara}
\begin{glose}
\pfra{faire grandir}
\end{glose}
\end{sous-entrée}
\end{entrée}

\begin{entrée}{waat}{}{ⓔwaat}
\région{BO [Corne]}
(\domainesémantique{Noms des plantes})
\classe{nom}
\begin{glose}
\pfra{Triumfetta rhomboïdea}
\end{glose}
\nomscientifique{Triumfetta rhomboïdea}
\end{entrée}

\begin{entrée}{waawè}{}{ⓔwaawè}
\région{GOs PA BO}
\variante{%
waapwè
\région{GO(s) vx BO (Corne)}}
(\domainesémantique{Arbre})
\classe{nom}
\begin{glose}
\pfra{pin colonnaire (représente le côté mâle)}
\end{glose}
\newline
\note{symbole de la durée du lignage}{glose}{}
\nomscientifique{Araucaria columnaris (Araucariacées)}
\newline
\relationsémantique{Cf.}{\lien{ⓔjeyu}{jeyu}}
\glosecourte{kaori}
\end{entrée}

\begin{entrée}{wa-bile}{}{ⓔwa-bile}
\région{GOs BO PA}
(\domainesémantique{Cordes, cordages})
\classe{v ; n}
\begin{glose}
\pfra{toron (roulé sur la cuisse) ; fil de filet}
\end{glose}
\begin{glose}
\pfra{faire un toron (roulé sur la cuisse)}
\end{glose}
\end{entrée}

\begin{entrée}{wabwa}{}{ⓔwabwa}
\région{GOs BO PA}
\variante{%
wabwa pwojo
\région{GO(s)}}
(\domainesémantique{Arbre})
\classe{nom}
\begin{glose}
\pfra{"bois de chou"}
\end{glose}
\nomscientifique{Gyropcarpus americanus (Hernandiacées)}
\end{entrée}

\begin{entrée}{wa-bwanu}{}{ⓔwa-bwanu}
\région{PA BO}
(\domainesémantique{Objets coutumiers})
\classe{nom}
\begin{glose}
\pfra{ceinture d'écorce (tressée de 3 brins de paille ou de feuilles de pandanus ; signe de départ en guerre. Dubois ms)}
\end{glose}
\begin{glose}
\pfra{ceinture faite en fibre d'écorce de banian ou en 'alamwi', dans laquelle on mettait les pierres et les paquets magiques pour la guerre}
\end{glose}
\end{entrée}

\begin{entrée}{wa-bwèèdrò}{}{ⓔwa-bwèèdrò}
\région{GOs}
(\domainesémantique{Vêtements, parure})
\classe{nom}
\begin{glose}
\pfra{bandeau ; turban}
\end{glose}
\end{entrée}

\begin{entrée}{wa-cama}{}{ⓔwa-cama}
\formephonétique{wacama}
\région{GOs}
(\domainesémantique{Noms des plantes})
\classe{nom}
\begin{glose}
\pfra{liane (à pétiole subéreux)}
\end{glose}
\end{entrée}

\begin{entrée}{wado}{}{ⓔwado}
\région{GOs BO}
\variante{%
waado
\région{BO}}
\classe{v.stat.}
\newline
\sens{1}
(\domainesémantique{Corps humain})
\begin{glose}
\pfra{édenté ; interstice laissé par des dents tombées}
\end{glose}
\newline
\begin{exemple}
\région{BO}
\textbf{\pnua{wado-paro-n}}
\pfra{l'interstice laissé par ses dents manquantes (Dubois)}
\end{exemple}
\newline
\sens{2}
(\domainesémantique{Description des objets, formes, consistance, taille})
\begin{glose}
\pfra{ébréché (pour une lame de couteau)}
\end{glose}
\end{entrée}

\begin{entrée}{wa-do}{}{ⓔwa-do}
\région{GOs PABO}
(\domainesémantique{Armes})
\classe{nom}
\begin{glose}
\pfra{ligature de la sagaie}
\end{glose}
\newline
\relationsémantique{Cf.}{\lien{}{wa [GO]}}
\glosecourte{corde, lien}
\newline
\relationsémantique{Cf.}{\lien{}{wal [BO, PA]}}
\glosecourte{corde, lien}
\end{entrée}

\begin{entrée}{wadolò}{}{ⓔwadolò}
\région{GOs}
(\domainesémantique{Crustacés, crabes})
\classe{nom}
\begin{glose}
\pfra{crabe qui vient d'avoir une nouvelle carapace}
\end{glose}
\newline
\relationsémantique{Cf.}{\lien{}{pitrêê, a-pii, nhyatru}}
\end{entrée}

\begin{entrée}{waga}{}{ⓔwaga}
\région{BO}
(\domainesémantique{Taros})
\classe{nom}
\begin{glose}
\pfra{taro d'eau (clone) (Dubois)}
\end{glose}
\newline
\note{mot non vérifié}{général}{}
\end{entrée}

\begin{entrée}{wãga}{}{ⓔwãga}
\région{GOs BO}
(\domainesémantique{Préfixes et verbes de position})
\classe{v}
\begin{glose}
\pfra{jambes écartées (être)}
\end{glose}
\newline
\begin{exemple}
\textbf{\pnua{ti nye thumenõ wãga ?}}
\pfra{qui marche les jambes écartées ?}
\end{exemple}
\newline
\begin{sous-entrée}{ku-wãga}{ⓔwãgaⓝku-wãga}
\begin{glose}
\pfra{debout jambes écartées}
\end{glose}
\end{sous-entrée}
\newline
\begin{sous-entrée}{kô-wãga}{ⓔwãgaⓝkô-wãga}
\begin{glose}
\pfra{couché jambes écartées}
\end{glose}
\end{sous-entrée}
\newline
\begin{sous-entrée}{te-wãga}{ⓔwãgaⓝte-wãga}
\begin{glose}
\pfra{assis jambes écartées}
\end{glose}
\end{sous-entrée}
\end{entrée}

\begin{entrée}{wãge}{}{ⓔwãge}
\région{GOs BO PA}
(\domainesémantique{Corps humain})
\classe{nom}
\begin{glose}
\pfra{poitrine}
\end{glose}
\newline
\begin{exemple}
\région{GOs}
\textbf{\pnua{wãgee-nu}}
\pfra{ma poitrine}
\end{exemple}
\newline
\begin{exemple}
\textbf{\pnua{wãgee-n [PA, BO]}}
\pfra{sa poitrine}
\end{exemple}
\end{entrée}

\begin{entrée}{wagiça}{}{ⓔwagiça}
\formephonétique{wagiʒa}
\région{GOs}
(\domainesémantique{Description des objets, formes, consistance, taille})
\classe{v}
\begin{glose}
\pfra{résistant (corde, personne, viande)}
\end{glose}
\newline
\relationsémantique{Cf.}{\lien{ⓔkulaçe}{kulaçe}}
\glosecourte{dur (viande)}
\newline
\relationsémantique{Cf.}{\lien{ⓔpiça}{piça}}
\glosecourte{dur}
\end{entrée}

\begin{entrée}{wa-hi}{}{ⓔwa-hi}
\région{GOs PA BO}
\variante{%
wa-yi
\région{BO}}
(\domainesémantique{Vêtements, parure})
\classe{nom}
\begin{glose}
\pfra{bracelet (lit. lien-bras)}
\end{glose}
\begin{glose}
\pfra{brassard}
\end{glose}
\newline
\begin{exemple}
\région{GO}
\textbf{\pnua{wa-hii-je}}
\pfra{son bracelet}
\end{exemple}
\newline
\begin{exemple}
\région{PA BO}
\textbf{\pnua{wa-hii-n}}
\pfra{son bracelet}
\end{exemple}
\end{entrée}

\begin{entrée}{wa-kabun}{}{ⓔwa-kabun}
\région{GOs}
\variante{%
kaça-kabu
\région{GO(s)}}
(\domainesémantique{Jours})
\classe{nom}
\begin{glose}
\pfra{lundi (lit. après le sacré)}
\end{glose}
\end{entrée}

\begin{entrée}{wakagume}{}{ⓔwakagume}
\région{GOs}
(\domainesémantique{Préfixes et verbes de position})
\classe{v}
\begin{glose}
\pfra{renversé ; chaviré (voiture, bateau)}
\end{glose}
\end{entrée}

\begin{entrée}{wa-kaze}{}{ⓔwa-kaze}
\région{GOs}
\variante{%
wa-kale
\région{GO(s)}}
(\domainesémantique{Marées})
\classe{nom}
\begin{glose}
\pfra{marée haute}
\end{glose}
\newline
\étymologie{
\langue{POc}
\étymon{*Rua(p)}
\glosecourte{marée}}
\end{entrée}

\begin{entrée}{wa-kiò}{}{ⓔwa-kiò}
\région{GOs PA BO}
\variante{%
waki, wa-kiò
\région{PA}}
(\domainesémantique{Vêtements, parure})
\classe{nom}
\begin{glose}
\pfra{ceinture}
\end{glose}
\newline
\begin{exemple}
\région{GOs}
\textbf{\pnua{wa-kiò-nu}}
\pfra{ma ceinture}
\end{exemple}
\newline
\begin{exemple}
\région{PABO}
\textbf{\pnua{wa-kiò-n}}
\pfra{sa ceinture}
\end{exemple}
\end{entrée}

\begin{entrée}{wala}{}{ⓔwala}
\région{GOs}
(\domainesémantique{Description des objets, formes, consistance, taille})
\classe{v.stat.}
\begin{glose}
\pfra{large}
\end{glose}
\newline
\relationsémantique{Ant.}{\lien{ⓔpivwizai}{pivwizai}}
\glosecourte{étroit}
\end{entrée}

\begin{entrée}{walaga}{}{ⓔwalaga}
\formephonétique{wa'laŋga}
\région{GOs}
(\domainesémantique{Caractéristiques et propriétés des personnes})
\classe{v}
\begin{glose}
\pfra{rusé ; malin ; qui joue des tours}
\end{glose}
\end{entrée}

\begin{entrée}{wala-me}{}{ⓔwala-me}
\région{BO}
(\domainesémantique{Comparaison})
\classe{COMPAR}
\begin{glose}
\pfra{comme ; semblable ; pareil}
\end{glose}
\newline
\note{mot non vérifié (Dubois)}{général}{}
\end{entrée}

\begin{entrée}{walei}{}{ⓔwalei}
\région{GOs BO}
(\domainesémantique{Ignames})
\classe{nom}
\begin{glose}
\pfra{igname}
\end{glose}
\nomscientifique{Dioscorea esculenta Burk}
\newline
\emprunt{walei (POLYN) (PPN *walei)}
\end{entrée}

\begin{entrée}{waluvwi}{}{ⓔwaluvwi}
\région{GOs}
\variante{%
waluvi
\région{BO [BM]}}
(\domainesémantique{Aliments, alimentation})
\classe{nom}
\begin{glose}
\pfra{chair de coco contenant des vers [GO]}
\end{glose}
\begin{glose}
\pfra{jus de coco tourné [BO]}
\end{glose}
\end{entrée}

\begin{entrée}{wa-mãgiça}{}{ⓔwa-mãgiça}
\formephonétique{wa-mɛ̃ŋgiʒa}
\région{GOs}
(\domainesémantique{Cordes, cordages})
\classe{nom}
\begin{glose}
\pfra{chaîne (lit. attache dure)}
\end{glose}
\end{entrée}

\begin{entrée}{wa-me}{}{ⓔwa-me}
\région{GOs WEM}
\variante{%
wããme
\région{GO(s)}, 
wamèèn
\région{BO}}
(\domainesémantique{Comparaison})
\classe{COMPAR}
\begin{glose}
\pfra{comme ; pareil ; semblable}
\end{glose}
\newline
\begin{exemple}
\région{GOs}
\textbf{\pnua{e ne wa-me nu}}
\pfra{il fait comme moi}
\end{exemple}
\newline
\begin{exemple}
\région{GOs}
\textbf{\pnua{e ne wa-me kêê-je}}
\pfra{il fait comme son père}
\end{exemple}
\newline
\begin{exemple}
\région{BO}
\textbf{\pnua{i vhaa wa-mèèn Tèè-ma}}
\pfra{il parle comme un chef}
\end{exemple}
\newline
\begin{exemple}
\région{BO}
\textbf{\pnua{i vhaa wa-mèèn}}
\pfra{il parle comme lui}
\end{exemple}
\newline
\begin{exemple}
\région{BO}
\textbf{\pnua{kavu la wa-mè la yu nee ?}}
\pfra{n'est-ce pas ainsi que tu as fait ?}
\end{exemple}
\end{entrée}

\begin{entrée}{wa-me da?}{}{ⓔwa-me da?}
\région{GOs PA}
\variante{%
wa da?
\région{PA}}
(\domainesémantique{Interrogatifs})
\classe{INT}
\begin{glose}
\pfra{comme quoi? ; qui ressemble à quoi?}
\end{glose}
\end{entrée}

\begin{entrée}{wa-mèèn exa}{}{ⓔwa-mèèn exa}
\région{BO [Corne]}
(\domainesémantique{Comparaison})
\classe{CNJ}
\begin{glose}
\pfra{comme si}
\end{glose}
\newline
\begin{exemple}
\région{BO}
\textbf{\pnua{i nee hye wamèèn exa yo phããde}}
\pfra{il l'a fait comme tu l'as montré}
\end{exemple}
\end{entrée}

\begin{entrée}{wa-me ne}{}{ⓔwa-me ne}
\formephonétique{wa-me ɳe}
\région{GOs}
(\domainesémantique{Conjonction})
\classe{v}
\begin{glose}
\pfra{faire en sorte que}
\end{glose}
\newline
\begin{exemple}
\région{GO}
\textbf{\pnua{e wa-me ne khõbwe (e) novwö na li tho ilie-bòli, çaa e chaaxö ã we-zumee-je}}
\pfra{Elle a fait en sorte que, quand ils appellent les deux (parents) en bas, sa salive réponde}
\end{exemple}
\newline
\begin{exemple}
\région{GO}
\textbf{\pnua{wame ne e tia-wãã-le da}}
\pfra{il fait en sorte de pousser comme cela vers le haut}
\end{exemple}
\end{entrée}

\begin{entrée}{wame ni}{}{ⓔwame ni}
\région{GOs}
(\domainesémantique{Directions})
\classe{LOC}
\begin{glose}
\pfra{environs (aux) de ; vers}
\end{glose}
\end{entrée}

\begin{entrée}{wa-me ti?}{}{ⓔwa-me ti?}
\région{GOs WEM}
(\domainesémantique{Interrogatifs})
\classe{INT.COMPAR}
\begin{glose}
\pfra{comme qui ?}
\end{glose}
\end{entrée}

\begin{entrée}{wamòn}{}{ⓔwamòn}
\région{PA WEH BO}
\variante{%
wamwa, wamò
\région{GO(s)}}
\classe{nom}
\newline
\sens{1}
(\domainesémantique{Armes
, Outils})
\begin{glose}
\pfra{hache}
\end{glose}
\begin{glose}
\pfra{tamioc}
\end{glose}
\newline
\sens{2}
(\domainesémantique{Outils})
\begin{glose}
\pfra{herminette}
\end{glose}
\newline
\begin{exemple}
\région{BO}
\textbf{\pnua{pai-ny wamòn}}
\pfra{ma hache}
\end{exemple}
\newline
\begin{exemple}
\région{BO}
\textbf{\pnua{wamòn paa}}
\pfra{herminette (lit. hache pierre)}
\end{exemple}
\end{entrée}

\begin{entrée}{wamwa}{}{ⓔwamwa}
\région{GOs WEM}
\variante{%
wamòn
\région{PA}}
\classe{nom}
\newline
\sens{1}
(\domainesémantique{Outils
, Armes})
\begin{glose}
\pfra{hache}
\end{glose}
\begin{glose}
\pfra{tamioc}
\end{glose}
\newline
\sens{2}
(\domainesémantique{Outils})
\begin{glose}
\pfra{herminette}
\end{glose}
\end{entrée}

\begin{entrée}{wa-na}{}{ⓔwa-na}
\région{PA}
(\domainesémantique{Modalité, verbes modaux})
\classe{MODAL}
\begin{glose}
\pfra{peut-être ; et si ?}
\end{glose}
\begin{glose}
\pfra{faillir}
\end{glose}
\newline
\begin{exemple}
\région{PA}
\textbf{\pnua{wa-na nu thuã}}
\pfra{je me trompe peut-être}
\end{exemple}
\newline
\begin{exemple}
\région{PA}
\textbf{\pnua{ra u wa-na i mòròm}}
\pfra{il a failli se noyer}
\end{exemple}
\newline
\begin{exemple}
\région{PA}
\textbf{\pnua{wa-na mi tee-a ?}}
\pfra{et si on partait d'abord ?}
\end{exemple}
\newline
\begin{exemple}
\région{PA}
\textbf{\pnua{wa-na ti ?}}
\pfra{qui cela peut-il bien être ?}
\end{exemple}
\end{entrée}

\begin{entrée}{wanga}{}{ⓔwanga}
\formephonétique{waŋa}
\région{GOs}
\variante{%
whaga-n
\région{BO [BM]}}
(\domainesémantique{Fonctions intellectuelles})
\classe{nom}
\begin{glose}
\pfra{sens ; signification ; raison}
\end{glose}
\newline
\begin{exemple}
\région{GOs}
\textbf{\pnua{kixa wanga}}
\pfra{cela n'a aucun sens}
\end{exemple}
\end{entrée}

\begin{entrée}{waniri}{}{ⓔwaniri}
\région{PA BO WE}
(\domainesémantique{Alliance})
\classe{nom}
\begin{glose}
\pfra{maternels (clan des)}
\end{glose}
\newline
\relationsémantique{Cf.}{\lien{}{a-kaalu}}
\glosecourte{maternels (dans les cérémonie de deuil)}
\end{entrée}

\begin{entrée}{wa-pwe}{}{ⓔwa-pwe}
\région{GOs}
(\domainesémantique{Pêche})
\classe{nom}
\begin{glose}
\pfra{fil de la ligne}
\end{glose}
\end{entrée}

\begin{entrée}{wara}{}{ⓔwara}
\région{GO}
\variante{%
wawa
\région{GO}, 
whara-n
\région{BO PA}}
(\domainesémantique{Temps})
\classe{nom}
\begin{glose}
\pfra{moment ; époque ; période}
\end{glose}
\begin{glose}
\pfra{saison}
\end{glose}
\begin{glose}
\pfra{heure ; temps}
\end{glose}
\newline
\begin{exemple}
\région{GOs}
\textbf{\pnua{wara u mo a}}
\pfra{c'est le moment de notre départ}
\end{exemple}
\newline
\begin{exemple}
\région{GOs}
\textbf{\pnua{wara u xa za e trabwa ni nõ-ma-òli}}
\pfra{quand il était encore assis à cet endroit là-bas}
\end{exemple}
\newline
\begin{exemple}
\région{PA}
\textbf{\pnua{ni wara da ?}}
\pfra{à quelle époque ?}
\end{exemple}
\newline
\begin{exemple}
\région{PA}
\textbf{\pnua{e wara thu phoê}}
\pfra{c'est l'époque de faire nos cultures}
\end{exemple}
\newline
\begin{exemple}
\région{PA BO}
\textbf{\pnua{ni wara al}}
\pfra{période de soleil, saison sèche}
\end{exemple}
\newline
\begin{exemple}
\région{BO}
\textbf{\pnua{ni wara khabu}}
\pfra{saison froide}
\end{exemple}
\newline
\begin{exemple}
\région{BO}
\textbf{\pnua{u wara hovwo}}
\pfra{c'est l'heure de manger}
\end{exemple}
\newline
\begin{exemple}
\région{BO}
\textbf{\pnua{u ta wara}}
\pfra{le moment est arrivé}
\end{exemple}
\newline
\begin{exemple}
\région{BO PA}
\textbf{\pnua{ra thu wara-ny}}
\pfra{j'ai le temps}
\end{exemple}
\newline
\begin{exemple}
\région{PA}
\textbf{\pnua{ra gaa thu whara-ny}}
\pfra{j'ai encore le temps}
\end{exemple}
\newline
\begin{sous-entrée}{whara ò thèl, wharaa thèl}{ⓔwaraⓝwhara ò thèl, wharaa thèl}
\région{PA}
\begin{glose}
\pfra{époque du débroussage (mai)}
\end{glose}
\end{sous-entrée}
\newline
\begin{sous-entrée}{õ wara ne}{ⓔwaraⓝõ wara ne}
\région{PA}
\begin{glose}
\pfra{chaque fois que}
\end{glose}
\end{sous-entrée}
\end{entrée}

\begin{entrée}{waramã}{}{ⓔwaramã}
\région{GOs}
\variante{%
wara
\région{GO(s)}}
(\domainesémantique{Vêtements, parure})
\classe{nom}
\begin{glose}
\pfra{ceinture}
\end{glose}
\newline
\begin{exemple}
\textbf{\pnua{waramã i nu}}
\pfra{ma ceinture}
\end{exemple}
\end{entrée}

\begin{entrée}{warô}{}{ⓔwarô}
\région{BO PA}
(\domainesémantique{Cultures, techniques, boutures})
\classe{nom}
\begin{glose}
\pfra{graine ; semence}
\end{glose}
\newline
\begin{sous-entrée}{warô kae}{ⓔwarôⓝwarô kae}
\begin{glose}
\pfra{des graines de pastèque}
\end{glose}
\newline
\begin{exemple}
\textbf{\pnua{waro-n}}
\pfra{sa graine}
\end{exemple}
\end{sous-entrée}
\end{entrée}

\begin{entrée}{wa-tru}{}{ⓔwa-tru}
\formephonétique{waʈu}
\région{GOs}
(\domainesémantique{Préfixes classificateurs numériques})
\classe{CLF.NUM}
\begin{glose}
\pfra{deux paires (de roussette ou notous dans les dons coutumiers)}
\end{glose}
\end{entrée}

\begin{entrée}{wa-tru ko ido}{}{ⓔwa-tru ko ido}
\région{GO}
(\domainesémantique{Préfixes classificateurs numériques})
\classe{CLF.NUM}
\begin{glose}
\pfra{deux paires et une demi-paire (de roussette ou notous dans les dons coutumiers)}
\end{glose}
\end{entrée}

\begin{entrée}{wa-truuji}{}{ⓔwa-truuji}
\région{GO}
(\domainesémantique{Préfixes classificateurs numériques})
\classe{CLF.NUM}
\begin{glose}
\pfra{dix paires (de roussette ou notous dans les dons coutumiers)}
\end{glose}
\end{entrée}

\begin{entrée}{wathrã}{}{ⓔwathrã}
\région{GOs}
(\domainesémantique{Coutumes, dons coutumiers})
\classe{nom}
\begin{glose}
\pfra{lien (coutumier)}
\end{glose}
\newline
\begin{exemple}
\textbf{\pnua{wathrã-mã}}
\pfra{notre lien (coutumier)}
\end{exemple}
\end{entrée}

\begin{entrée}{wa-vwo}{}{ⓔwa-vwo}
\région{GOs}
(\domainesémantique{Modalité, verbes modaux})
\classe{MODAL}
\begin{glose}
\pfra{peut-être que oui (réponse)}
\end{glose}
\end{entrée}

\begin{entrée}{wãwã}{}{ⓔwãwã}
\région{GOs}
\variante{%
wââng
\région{PA WE}}
(\domainesémantique{Oiseaux})
\classe{nom}
\begin{glose}
\pfra{corbeau}
\end{glose}
\nomscientifique{Corvus moneduloides (Corvidés)}
\end{entrée}

\begin{entrée}{wa-xè}{}{ⓔwa-xè}
\région{GOs PA}
(\domainesémantique{Préfixes classificateurs numériques})
\classe{CLF.NUM}
\begin{glose}
\pfra{une paire (de roussettes ou notous dans les dons coutumiers)}
\end{glose}
\newline
\begin{exemple}
\région{GOs}
\textbf{\pnua{wa-xè bwò}}
\pfra{une paire de roussettes}
\end{exemple}
\end{entrée}

\begin{entrée}{waza}{1}{ⓔwazaⓗ1}
\région{GOs}
\variante{%
wala, whala
\région{PA WE}}
(\domainesémantique{Préfixes classificateurs possessifs de la nourriture})
\classe{CLF.POSS}
\begin{glose}
\pfra{canne à sucre (à manger)}
\end{glose}
\newline
\begin{sous-entrée}{waza-nu ê}{ⓔwazaⓗ1ⓝwaza-nu ê}
\région{GO}
\begin{glose}
\pfra{ma canne à sucre (à manger)}
\end{glose}
\newline
\relationsémantique{Cf.}{\lien{}{whizi ; wizi ; wa}}
\glosecourte{manger de la canne à sucre}
\end{sous-entrée}
\end{entrée}

\begin{entrée}{waza}{2}{ⓔwazaⓗ2}
\variante{%
wara
\région{PA}}
(\domainesémantique{Temps})
\classe{nom}
\begin{glose}
\pfra{moment ; époque ; heure}
\end{glose}
\newline
\begin{exemple}
\région{GOs}
\textbf{\pnua{egôgo waza ò nu ẽnõ}}
\pfra{avant quand j'étais enfant}
\end{exemple}
\newline
\begin{exemple}
\région{GOs}
\textbf{\pnua{co a-mi ni waza ò ia ?}}
\pfra{à quelle époque es-tu arrivé ?}
\end{exemple}
\end{entrée}

\begin{entrée}{wazale kòò}{}{ⓔwazale kòò}
\région{GOs}
\variante{%
wè-rali kòò
\région{BO}}
(\domainesémantique{Mouvements ou actions faits avec le corps, les bras, les mains, les pieds})
\classe{v}
\begin{glose}
\pfra{faire un croc-en-jambes}
\end{glose}
\newline
\begin{exemple}
\région{BO}
\textbf{\pnua{i wè-rali kòò-n}}
\pfra{il lui a fait un croc-en-jambes [Corne]}
\end{exemple}
\newline
\relationsémantique{Cf.}{\lien{ⓔthali}{thali}}
\end{entrée}

\begin{entrée}{waza o kole pwa}{}{ⓔwaza o kole pwa}
\région{GOs}
\variante{%
waza ò kole pwal
\région{PA}}
(\domainesémantique{Saisons})
\classe{nom}
\begin{glose}
\pfra{saison des pluies, des cyclones, des grandes marées (mars, avril)}
\end{glose}
\end{entrée}

\begin{entrée}{waza o thaa kui}{}{ⓔwaza o thaa kui}
\région{GOs}
(\domainesémantique{Saisons})
\classe{nom}
\begin{glose}
\pfra{saison de la récolte des ignames}
\end{glose}
\end{entrée}

\begin{entrée}{waza o thòe kui}{}{ⓔwaza o thòe kui}
\région{GOs}
(\domainesémantique{Saisons})
\classe{nom}
\begin{glose}
\pfra{saison de la plantation des ignames (juillet à septembre)}
\end{glose}
\end{entrée}

\begin{entrée}{wazizibu}{}{ⓔwazizibu}
\région{GOs}
(\domainesémantique{Verbes de mouvement})
\classe{v}
\begin{glose}
\pfra{trébucher}
\end{glose}
\newline
\begin{exemple}
\région{GOs}
\textbf{\pnua{e wazizibu bwa paa}}
\pfra{il a trébuché sur une pierre}
\end{exemple}
\end{entrée}

\begin{entrée}{we}{1}{ⓔweⓗ1}
\région{GOs PA BO}
(\domainesémantique{Eau})
\classe{nom}
\begin{glose}
\pfra{eau}
\end{glose}
\newline
\begin{sous-entrée}{gu-we}{ⓔweⓗ1ⓝgu-we}
\begin{glose}
\pfra{eau potable}
\end{glose}
\end{sous-entrée}
\newline
\begin{sous-entrée}{we-ne(m)}{ⓔweⓗ1ⓝwe-ne(m)}
\begin{glose}
\pfra{eau douce}
\end{glose}
\end{sous-entrée}
\newline
\begin{sous-entrée}{we-za}{ⓔweⓗ1ⓝwe-za}
\région{GOs}
\begin{glose}
\pfra{mer, eau salée}
\end{glose}
\end{sous-entrée}
\newline
\begin{sous-entrée}{we-pwa}{ⓔweⓗ1ⓝwe-pwa}
\région{GOs}
\begin{glose}
\pfra{eau de pluie}
\end{glose}
\end{sous-entrée}
\newline
\begin{sous-entrée}{gaa-we}{ⓔweⓗ1ⓝgaa-we}
\begin{glose}
\pfra{cascade}
\end{glose}
\end{sous-entrée}
\newline
\begin{sous-entrée}{pwe-we}{ⓔweⓗ1ⓝpwe-we}
\begin{glose}
\pfra{trou d'eau}
\end{glose}
\end{sous-entrée}
\newline
\begin{sous-entrée}{we-cabol}{ⓔweⓗ1ⓝwe-cabol}
\région{PA}
\begin{glose}
\pfra{source}
\end{glose}
\end{sous-entrée}
\newline
\begin{sous-entrée}{we-pò-ce}{ⓔweⓗ1ⓝwe-pò-ce}
\begin{glose}
\pfra{jus de fruit (d'arbre)}
\end{glose}
\end{sous-entrée}
\newline
\begin{sous-entrée}{we-ce}{ⓔweⓗ1ⓝwe-ce}
\begin{glose}
\pfra{sève}
\end{glose}
\end{sous-entrée}
\newline
\begin{sous-entrée}{we-fa}{ⓔweⓗ1ⓝwe-fa}
\begin{glose}
\pfra{suc de la parole}
\end{glose}
\end{sous-entrée}
\newline
\begin{sous-entrée}{we-ima}{ⓔweⓗ1ⓝwe-ima}
\begin{glose}
\pfra{urine}
\end{glose}
\end{sous-entrée}
\newline
\begin{sous-entrée}{we-mebo}{ⓔweⓗ1ⓝwe-mebo}
\begin{glose}
\pfra{miel}
\end{glose}
\end{sous-entrée}
\newline
\begin{sous-entrée}{we-nu}{ⓔweⓗ1ⓝwe-nu}
\begin{glose}
\pfra{lait de coco}
\end{glose}
\end{sous-entrée}
\newline
\begin{sous-entrée}{we ni mèè-n}{ⓔweⓗ1ⓝwe ni mèè-n}
\région{PA}
\begin{glose}
\pfra{ses larmes}
\end{glose}
\end{sous-entrée}
\newline
\begin{sous-entrée}{we-thi}{ⓔweⓗ1ⓝwe-thi}
\begin{glose}
\pfra{lait maternel}
\end{glose}
\end{sous-entrée}
\newline
\begin{sous-entrée}{we-zume}{ⓔweⓗ1ⓝwe-zume}
\begin{glose}
\pfra{salive}
\end{glose}
\end{sous-entrée}
\newline
\begin{sous-entrée}{we-wa}{ⓔweⓗ1ⓝwe-wa}
\begin{glose}
\pfra{bave}
\end{glose}
\end{sous-entrée}
\newline
\begin{sous-entrée}{koli we}{ⓔweⓗ1ⓝkoli we}
\begin{glose}
\pfra{au bord de l'eau}
\end{glose}
\end{sous-entrée}
\newline
\étymologie{
\langue{POc}
\étymon{*wai(R)}}
\end{entrée}

\begin{entrée}{we}{2}{ⓔweⓗ2}
\région{GOs PA}
\région{PA BO}
\variante{%
wèè-n
}
(\domainesémantique{Parties de plantes})
\classe{nom}
\begin{glose}
\pfra{racine de (forme en composition ou détermination de wa(l) 'racine')}
\end{glose}
\newline
\begin{sous-entrée}{we bumi}{ⓔweⓗ2ⓝwe bumi}
\begin{glose}
\pfra{racines de banian}
\end{glose}
\end{sous-entrée}
\newline
\begin{sous-entrée}{we ce}{ⓔweⓗ2ⓝwe ce}
\région{GO BO}
\begin{glose}
\pfra{racines d'arbre}
\end{glose}
\newline
\begin{exemple}
\région{PA}
\textbf{\pnua{wèè-n}}
\pfra{sa racine}
\end{exemple}
\newline
\relationsémantique{Cf.}{\lien{}{wa ; wal}}
\glosecourte{racine}
\end{sous-entrée}
\newline
\étymologie{
\langue{POc}
\étymon{*wakaR}
\glosecourte{racine}}
\end{entrée}

\begin{entrée}{we-}{}{ⓔwe-}
\région{GOs PA BO}
(\domainesémantique{Préfixes classificateurs numériques})
\classe{CLF.NUM général}
\begin{glose}
\pfra{objets longs CLF (voiture, bateau, arbre couché, poteau)}
\end{glose}
\begin{glose}
\pfra{année, mois CLF}
\end{glose}
\begin{glose}
\pfra{chants CLF}
\end{glose}
\newline
\begin{exemple}
\textbf{\pnua{we-xe ka; we-tru mhwããnu; we-ko ka; we-pa, etc.}}
\pfra{un an, deux mois, trois ans; quatre}
\end{exemple}
\newline
\begin{exemple}
\textbf{\pnua{we-ru tèèn}}
\pfra{deux jours}
\end{exemple}
\end{entrée}

\begin{entrée}{we ãmu}{}{ⓔwe ãmu}
\région{BO PA}
(\domainesémantique{Aliments, alimentation})
\classe{nom}
\begin{glose}
\pfra{miel}
\end{glose}
\newline
\relationsémantique{Cf.}{\lien{}{pi ãmu}}
\glosecourte{miel}
\end{entrée}

\begin{entrée}{we bumi}{}{ⓔwe bumi}
\région{BO [Corne]}
(\domainesémantique{Vêtements, parure})
\classe{nom}
\begin{glose}
\pfra{étoffe d'écorce de banian}
\end{glose}
\newline
\note{non vérifié}{général}{}
\end{entrée}

\begin{entrée}{we-bwaxixi}{}{ⓔwe-bwaxixi}
\région{GOs}
\variante{%
we-bwaxii
\région{GO(s)}}
(\domainesémantique{Instruments})
\classe{nom}
\begin{glose}
\pfra{miroir}
\end{glose}
\newline
\relationsémantique{Cf.}{\lien{}{we-zhido [GOs]}}
\glosecourte{miroir}
\end{entrée}

\begin{entrée}{we-cabo}{}{ⓔwe-cabo}
\région{GOs}
\variante{%
we-cabòl
\région{PA}}
(\domainesémantique{Eau})
\classe{nom}
\begin{glose}
\pfra{source d'eau}
\end{glose}
\end{entrée}

\begin{entrée}{we-ce}{}{ⓔwe-ce}
\région{GOs BO PA}
(\domainesémantique{Parties de plantes})
\classe{nom}
\begin{glose}
\pfra{sève}
\end{glose}
\end{entrée}

\begin{entrée}{we-co}{}{ⓔwe-co}
\région{GOs}
(\domainesémantique{Corps humain})
\classe{nom}
\begin{glose}
\pfra{sperme}
\end{glose}
\end{entrée}

\begin{entrée}{wèda}{}{ⓔwèda}
\région{GOs}
(\domainesémantique{Armes})
\classe{nom}
\begin{glose}
\pfra{fronde}
\end{glose}
\newline
\begin{exemple}
\région{BO}
\textbf{\pnua{hi-wèdal}}
\pfra{le doigtier de la fronde}
\end{exemple}
\newline
\note{wèdali-n [PA, BO]}{grammaire}{sa fronde}
\end{entrée}

\begin{entrée}{wèdò}{}{ⓔwèdò}
\région{BO PA}
\variante{%
wôdo
\région{PA}}
(\domainesémantique{Coutumes, dons coutumiers})
\classe{v ; n}
\begin{glose}
\pfra{actes coutumiers ; coutumes}
\end{glose}
\begin{glose}
\pfra{usages ; manières ; moeurs}
\end{glose}
\newline
\begin{exemple}
\région{PA}
\textbf{\pnua{wèdòò-ni bulu ni la wèdòò-la}}
\pfra{faisons nos coutumes pour leur cérémonie coutumière}
\end{exemple}
\newline
\begin{exemple}
\région{BO}
\textbf{\pnua{wèdòò-n}}
\pfra{ses coutumes, manières}
\end{exemple}
\newline
\begin{exemple}
\région{PA}
\textbf{\pnua{wôdòò-n}}
\pfra{ses coutumes, manières}
\end{exemple}
\newline
\begin{exemple}
\région{BO}
\textbf{\pnua{wèdo daaleèn}}
\pfra{les coutumes des européens}
\end{exemple}
\end{entrée}

\begin{entrée}{we-ê}{}{ⓔwe-ê}
\région{GOs}
\variante{%
we-èm
\région{WE}, 
cuk
\région{PA}}
(\domainesémantique{Aliments, alimentation})
\classe{nom}
\begin{glose}
\pfra{sucre (lit. jus de canne à sucre)}
\end{glose}
\end{entrée}

\begin{entrée}{weem}{}{ⓔweem}
\région{PA BO}
(\domainesémantique{Objets coutumiers})
\classe{nom}
\begin{glose}
\pfra{monnaie (de moins grande valeur que 'yòò')}
\end{glose}
\newline
\note{monnaie faite de coquillages blancs, offerte attachée à un rameau de niaoulis ou de bananier. L'autre nom de cette monnaie est 'hègi pulo'). (Dubois :1 weem de 5 m vaut 100 fr). Hiérarchiedes valeurs : yòò > weem > yhalo.}{glose}{}
\newline
\relationsémantique{Cf.}{\lien{}{pwãmwãnu ; dopweza; yòò}}
\end{entrée}

\begin{entrée}{weena}{}{ⓔweena}
\formephonétique{weːɳa}
\région{GOs PA BO}
(\domainesémantique{Fonctions intellectuelles})
\classe{v}
\begin{glose}
\pfra{penser (incertain) ; croire (sans être sûr)}
\end{glose}
\newline
\begin{exemple}
\région{GOs}
\textbf{\pnua{nu weena khõbwe e a-wãã-du ènè}}
\pfra{je pense qu'il est passé par là (en le montrant du doigt)}
\end{exemple}
\newline
\begin{exemple}
\région{GOs}
\textbf{\pnua{nu weena khõbwe e a-du ènè}}
\pfra{je pense qu'il est passé par là}
\end{exemple}
\newline
\begin{exemple}
\région{GOs}
\textbf{\pnua{nu weena khõbwe e a-è ènòli}}
\pfra{je pense qu'il est passé par là}
\end{exemple}
\newline
\begin{exemple}
\région{GOs}
\textbf{\pnua{kavwö me wero, pu nye me weena me ezoma la bòzi-me}}
\pfra{nous n'avons pas fait de bruit, parce qu'ils nous auraient punis à coup sûr}
\end{exemple}
\newline
\begin{exemple}
\région{GOs}
\textbf{\pnua{nu pe-weena}}
\pfra{je pense, il me semble (incertain)}
\end{exemple}
\newline
\note{weena-ni}{grammaire}{}
\end{entrée}

\begin{entrée}{wèè-uva}{}{ⓔwèè-uva}
\région{GOs BO}
(\domainesémantique{Taros})
\classe{nom}
\begin{glose}
\pfra{racines du taro d'eau}
\end{glose}
\newline
\begin{sous-entrée}{wèè-uvhia}{ⓔwèè-uvaⓝwèè-uvhia}
\begin{glose}
\pfra{racines du taro de montagne}
\end{glose}
\end{sous-entrée}
\end{entrée}

\begin{entrée}{we hêê-du}{}{ⓔwe hêê-du}
\région{GOs}
\variante{%
bòyil, böyil
\région{PA}}
(\domainesémantique{Corps humain})
\classe{nom}
\begin{glose}
\pfra{moëlle des os}
\end{glose}
\end{entrée}

\begin{entrée}{we-hogo}{}{ⓔwe-hogo}
\région{PA}
(\domainesémantique{Topographie})
\classe{nom}
\begin{glose}
\pfra{versant ; pente de la montagne (lit. racines de la montagne < wal)}
\end{glose}
\end{entrée}

\begin{entrée}{we imã}{}{ⓔwe imã}
\région{GOs BO PA}
(\domainesémantique{Fonctions naturelles humaines})
\classe{nom}
\begin{glose}
\pfra{urine}
\end{glose}
\end{entrée}

\begin{entrée}{we-kae}{}{ⓔwe-kae}
\région{GOs}
(\domainesémantique{Ustensiles})
\classe{nom}
\begin{glose}
\pfra{calebasse (servant à porter l'eau)}
\end{glose}
\end{entrée}

\begin{entrée}{we-kênõng}{}{ⓔwe-kênõng}
\région{BO}
(\domainesémantique{Eau})
\classe{nom}
\begin{glose}
\pfra{tourbillon ; contre-courant}
\end{glose}
\end{entrée}

\begin{entrée}{wèle}{}{ⓔwèle}
\région{BO [BM, Corne]}
(\domainesémantique{Guerre})
\classe{v ; n}
\begin{glose}
\pfra{battre (se) ; bagarre}
\end{glose}
\newline
\begin{exemple}
\région{BO}
\textbf{\pnua{yo wèle ma ri ?}}
\pfra{avec qui t'es-tu battu ?}
\end{exemple}
\end{entrée}

\begin{entrée}{we-mebo, mebo}{}{ⓔwe-mebo, mebo}
\région{GOs}
(\domainesémantique{Aliments, alimentation})
\classe{nom}
\begin{glose}
\pfra{miel}
\end{glose}
\end{entrée}

\begin{entrée}{we-mèèn}{}{ⓔwe-mèèn}
\région{PA}
(\domainesémantique{Eau})
\classe{nom}
\begin{glose}
\pfra{eau légèrement salée ; eau saumâtre}
\end{glose}
\newline
\relationsémantique{Cf.}{\lien{}{mèèn}}
\glosecourte{salé (cuisine)}
\end{entrée}

\begin{entrée}{we mii}{}{ⓔwe mii}
\région{GOs}
(\domainesémantique{Aliments, alimentation})
\classe{nom}
\begin{glose}
\pfra{vin (lit. eau rouge)}
\end{glose}
\end{entrée}

\begin{entrée}{we-ne}{}{ⓔwe-ne}
\formephonétique{weɳe}
\région{GOs}
\variante{%
we-nèm
\région{BO PA}}
(\domainesémantique{Eau})
\classe{nom}
\begin{glose}
\pfra{eau douce}
\end{glose}
\end{entrée}

\begin{entrée}{wène}{}{ⓔwène}
\région{BO PA}
(\domainesémantique{Vêtements, parure})
\classe{nom}
\begin{glose}
\pfra{étoffe d'écorce de banian (écorce des racines ; Dubois ms)}
\end{glose}
\newline
\note{non vérifié}{général}{}
\end{entrée}

\begin{entrée}{wêne}{}{ⓔwêne}
\région{GOs PA BO}
\variante{%
wône
}
(\domainesémantique{Relations et interaction sociales})
\classe{v}
\begin{glose}
\pfra{changer ; échanger ; remplacer}
\end{glose}
\newline
\begin{exemple}
\région{BO}
\textbf{\pnua{nu wene mõ-ny}}
\pfra{je déménage}
\end{exemple}
\newline
\begin{exemple}
\région{PA}
\textbf{\pnua{i tee wene-nu}}
\pfra{il me remplace}
\end{exemple}
\end{entrée}

\begin{entrée}{wè-ni}{}{ⓔwè-ni}
\formephonétique{weɳi}
\région{GOs}
(\domainesémantique{Numéraux cardinaux})
\classe{NUM.ORD}
\begin{glose}
\pfra{cinq (objets longs)}
\end{glose}
\newline
\begin{exemple}
\région{GOs}
\textbf{\pnua{nye za u we-ni ?}}
\pfra{cela fait combien ?}
\end{exemple}
\newline
\begin{sous-entrée}{wè-ni ma xe}{ⓔwè-niⓝwè-ni ma xe}
\begin{glose}
\pfra{sixième}
\end{glose}
\end{sous-entrée}
\newline
\begin{sous-entrée}{wè-ni ma dru}{ⓔwè-niⓝwè-ni ma dru}
\begin{glose}
\pfra{septième}
\end{glose}
\end{sous-entrée}
\end{entrée}

\begin{entrée}{weni-do}{}{ⓔweni-do}
\région{BO}
\classe{nom}
\begin{glose}
\pfra{serpentine verte (Dubois)}
\end{glose}
\newline
\note{non vérifié}{général}{}
\end{entrée}

\begin{entrée}{wè-ni ma du}{}{ⓔwè-ni ma du}
\région{BO}
(\domainesémantique{Numéraux cardinaux})
\classe{NUM}
\begin{glose}
\pfra{sept (choses longues)}
\end{glose}
\newline
\begin{exemple}
\région{BO}
\textbf{\pnua{wè-ni ma du kau-ny}}
\pfra{j'ai 7 ans}
\end{exemple}
\end{entrée}

\begin{entrée}{we ni mè}{}{ⓔwe ni mè}
\région{GOs}
\variante{%
we ni mèè-n
\région{PA BO}}
(\domainesémantique{Corps humain})
\classe{nom}
\begin{glose}
\pfra{larmes}
\end{glose}
\newline
\begin{exemple}
\région{PA}
\textbf{\pnua{we ni mèè-ny}}
\pfra{mes larmes}
\end{exemple}
\end{entrée}

\begin{entrée}{wè-niza ?}{}{ⓔwè-niza ?}
\région{GOs}
\variante{%
we-nira ?
\région{PA BO}}
(\domainesémantique{Interrogatifs})
\classe{INT}
\begin{glose}
\pfra{combien (de choses longues, jours, an) ?}
\end{glose}
\newline
\begin{exemple}
\textbf{\pnua{wè-niza wô ?}}
\pfra{combien de bateaux ?}
\end{exemple}
\newline
\begin{exemple}
\textbf{\pnua{wè-niza ka ?}}
\pfra{combien d'années ?}
\end{exemple}
\newline
\begin{exemple}
\textbf{\pnua{po-niza kau jö ?}}
\pfra{quel âge as-tu ?}
\end{exemple}
\newline
\begin{exemple}
\région{PA}
\textbf{\pnua{we-niza tèèn ?}}
\pfra{combien de jours ?}
\end{exemple}
\newline
\étymologie{
\langue{POc}
\étymon{*pinsa, *pija}}
\end{entrée}

\begin{entrée}{we-no}{}{ⓔwe-no}
\région{PA}
(\domainesémantique{Cours de la vie})
\classe{v}
\begin{glose}
\pfra{pendre (se)}
\end{glose}
\end{entrée}

\begin{entrée}{we-nu}{}{ⓔwe-nu}
\formephonétique{weɳu}
\région{GOs BO}
(\domainesémantique{Cocotiers})
\classe{nom}
\begin{glose}
\pfra{eau de coco ; coco à boire}
\end{glose}
\newline
\relationsémantique{Cf.}{\lien{ⓔnu-wee}{nu-wee}}
\glosecourte{coco vert}
\end{entrée}

\begin{entrée}{we-pò-ce}{}{ⓔwe-pò-ce}
\région{GOs}
(\domainesémantique{Aliments, alimentation})
\classe{nom}
\begin{glose}
\pfra{jus de fruit}
\end{glose}
\end{entrée}

\begin{entrée}{wepòò}{}{ⓔwepòò}
\région{PA BO}
(\domainesémantique{Objets coutumiers})
\classe{nom}
\begin{glose}
\pfra{ceinture de femme (litt. we-pòò 'racine de bourao')}
\end{glose}
\newline
\note{en racine de bourao ; sert de monnaie à bas prix, offertes roulées (Charles).(Dubois ms : faisait plusieurs fois le tour du bassin.)}{glose}{}
\newline
\relationsémantique{Cf.}{\lien{}{pobil, thabil}}
\glosecourte{ceinture de femme (monnaie)}
\end{entrée}

\begin{entrée}{we-phölo}{}{ⓔwe-phölo}
\formephonétique{'we-'pʰωlo, 'we-'vwωlo}
\région{GOs}
\variante{%
we-vwölo
\région{PA}}
(\domainesémantique{Eau})
\classe{nom}
\begin{glose}
\pfra{eau boueuse ; marais}
\end{glose}
\newline
\relationsémantique{Cf.}{\lien{ⓔphöloo}{phöloo}}
\glosecourte{sale, trouble}
\end{entrée}

\begin{entrée}{we-phwa}{}{ⓔwe-phwa}
\région{GOs}
\variante{%
we-vwa
\région{GO(s) BO PA}}
(\domainesémantique{Fonctions naturelles humaines})
\classe{nom}
\begin{glose}
\pfra{salive ; bave (lit. eau-bouche)}
\end{glose}
\newline
\begin{exemple}
\région{GOs}
\textbf{\pnua{we-phwaa-je (ou) we-vwaa-je}}
\pfra{sa salive}
\end{exemple}
\newline
\begin{exemple}
\région{PA}
\textbf{\pnua{we-wa-n}}
\pfra{sa salive}
\end{exemple}
\end{entrée}

\begin{entrée}{wero}{}{ⓔwero}
\région{GOs}
(\domainesémantique{Sons, bruits})
\classe{v}
\begin{glose}
\pfra{faire du bruit}
\end{glose}
\newline
\begin{exemple}
\textbf{\pnua{ko (kawo) jö wero !}}
\pfra{arrête de faire du bruit !}
\end{exemple}
\end{entrée}

\begin{entrée}{wè-ru-mô}{}{ⓔwè-ru-mô}
\région{GOs}
\variante{%
wòruumò
\région{BO [BM]}}
(\domainesémantique{Caractéristiques et propriétés des personnes})
\classe{nom}
\begin{glose}
\pfra{ambidextre (lit. deux gauche)}
\end{glose}
\end{entrée}

\begin{entrée}{we-tiivwo}{}{ⓔwe-tiivwo}
\région{GOs}
\variante{%
we-tiin
\région{PA}}
(\domainesémantique{Matière, matériaux})
\classe{nom}
\begin{glose}
\pfra{encre (lit. liquide écrire)}
\end{glose}
\end{entrée}

\begin{entrée}{we-thi}{}{ⓔwe-thi}
\région{GOs BO PA}
(\domainesémantique{Corps humain})
\classe{nom}
\begin{glose}
\pfra{lait maternel (lit. liquide de sein)}
\end{glose}
\end{entrée}

\begin{entrée}{we-tho}{}{ⓔwe-tho}
\région{GOs BO}
(\domainesémantique{Eau})
\classe{nom}
\begin{glose}
\pfra{eau vive (lit. eau qui coule)}
\end{glose}
\end{entrée}

\begin{entrée}{we-trabwa}{}{ⓔwe-trabwa}
\région{GOs}
\variante{%
we-tabwa
\région{BO}}
(\domainesémantique{Eau})
\classe{nom}
\begin{glose}
\pfra{eau morte ; eau stagnante (lit. eau assise, eau qui ne coule plus)}
\end{glose}
\end{entrée}

\begin{entrée}{we-tru}{}{ⓔwe-tru}
\région{GOs}
(\domainesémantique{Actions liées aux éléments (liquide, fumée)})
\classe{nom}
\begin{glose}
\pfra{eau monte et coule (après une forte pluie)}
\end{glose}
\end{entrée}

\begin{entrée}{we-thrôbo}{}{ⓔwe-thrôbo}
\région{GOs}
\variante{%
we-thôbo
\région{BO}}
(\domainesémantique{Eau})
\classe{nom}
\begin{glose}
\pfra{chute d'eau ; cascade}
\end{glose}
\begin{glose}
\pfra{ouverture d'eau ; prise d'eau}
\end{glose}
\newline
\note{ouverture d'eau des talus externes de tarodière (peut être fermée de terre, de pierre pour laisser passer l'eau de l'étage supérieur à l'étage inférieur). Dubois}{glose}{}
\end{entrée}

\begin{entrée}{we-vhaa}{}{ⓔwe-vhaa}
\région{BO}
(\domainesémantique{Discours, échanges verbaux})
\classe{nom}
\begin{glose}
\pfra{substance de la parole}
\end{glose}
\end{entrée}

\begin{entrée}{wewele}{}{ⓔwewele}
\région{GOs}
(\domainesémantique{Mouvements ou actions faits avec le corps, les bras, les mains, les pieds})
\classe{v}
\begin{glose}
\pfra{secouer (arbre)}
\end{glose}
\end{entrée}

\begin{entrée}{wè-xèè}{}{ⓔwè-xèè}
\région{GOs PA BO}
(\domainesémantique{Préfixes classificateurs numériques})
\classe{CLF.NUM (bois, classificateur général et classificateur des objets longs, sagaies, arbres, certaines racines comestibles, cordes, doigts)}
\begin{glose}
\pfra{un (objet long)}
\end{glose}
\newline
\begin{exemple}
\région{GOs}
\textbf{\pnua{wè-xèè; wè-tru; etc.}}
\pfra{un , deux , etc;}
\end{exemple}
\newline
\begin{exemple}
\région{PA}
\textbf{\pnua{wè-xe, wè-ru, wè-kòn, wè-p(h)a, wè-nim, wè-ni ma-xèè, etc.}}
\pfra{un, deux, trois, quatre, cinq, six, etc.}
\end{exemple}
\newline
\begin{exemple}
\région{GOs}
\textbf{\pnua{ni wè-xèè mhwããnu}}
\pfra{dans un mois}
\end{exemple}
\newline
\begin{exemple}
\région{GOs}
\textbf{\pnua{nu nooli wõ xa wè-xèè}}
\pfra{j'ai vu un bateau}
\end{exemple}
\newline
\begin{exemple}
\région{GOs}
\textbf{\pnua{wèniza wõ xa çö nõõli ? - ca wè-xèè nõ wõ - ca wè-tru wõ xa nu nõõli}}
\pfra{combien de bateaux as-tu vus ?- juste un seul bateau - j'ai vu 2 bateaux}
\end{exemple}
\end{entrée}

\begin{entrée}{wè-xè hii-je}{}{ⓔwè-xè hii-je}
\région{GOs}
(\domainesémantique{Santé, maladie})
\begin{glose}
\pfra{manchot ; qui n'a qu'un seul bras}
\end{glose}
\end{entrée}

\begin{entrée}{wè-xè kòò-je}{}{ⓔwè-xè kòò-je}
\région{GOs}
(\domainesémantique{Santé, maladie})
\begin{glose}
\pfra{qui n'a qu'une seule jambe}
\end{glose}
\end{entrée}

\begin{entrée}{weya}{}{ⓔweya}
\région{BO}
(\domainesémantique{Mouvements ou actions faits avec le corps, les bras, les mains, les pieds})
\classe{v}
\begin{glose}
\pfra{fouiller (dans les affaires des autres) ; se mêler (de ce qui ne vous regarde pas) [Corne]}
\end{glose}
\newline
\begin{sous-entrée}{a-weya}{ⓔweyaⓝa-weya}
\begin{glose}
\pfra{indiscret}
\end{glose}
\newline
\note{mot non vérifié}{général}{}
\end{sous-entrée}
\end{entrée}

\begin{entrée}{we-yaai}{}{ⓔwe-yaai}
\région{GOs}
(\domainesémantique{Matière, matériaux})
\classe{nom}
\begin{glose}
\pfra{pétrole (de la lampe)}
\end{glose}
\end{entrée}

\begin{entrée}{we-za}{}{ⓔwe-za}
\région{GOs BO}
\variante{%
we-ya
}
(\domainesémantique{Mer})
\classe{nom}
\begin{glose}
\pfra{mer ; eau salée}
\end{glose}
\newline
\relationsémantique{Cf.}{\lien{ⓔkaze}{kaze}}
\end{entrée}

\begin{entrée}{weze}{}{ⓔweze}
\région{GOs}
\variante{%
wei
\région{BO}}
(\domainesémantique{Mouvements ou actions faits avec le corps, les bras, les mains, les pieds})
\classe{v}
\begin{glose}
\pfra{serrer}
\end{glose}
\newline
\begin{sous-entrée}{weze nõ}{ⓔwezeⓝweze nõ}
\begin{glose}
\pfra{étrangler}
\end{glose}
\end{sous-entrée}
\newline
\begin{sous-entrée}{wei nõ}{ⓔwezeⓝwei nõ}
\région{BO}
\begin{glose}
\pfra{étrangler, se pendre}
\end{glose}
\end{sous-entrée}
\end{entrée}

\begin{entrée}{weze nõ}{}{ⓔweze nõ}
\formephonétique{weðe ɳɔ̃}
\région{GOs}
(\domainesémantique{Mouvements ou actions faits avec le corps, les bras, les mains, les pieds})
\classe{v}
\begin{glose}
\pfra{étrangler (serrer le cou)}
\end{glose}
\end{entrée}

\begin{entrée}{we-zume}{}{ⓔwe-zume}
\région{GOs}
(\domainesémantique{Fonctions naturelles humaines})
\classe{nom}
\begin{glose}
\pfra{crachat}
\end{glose}
\newline
\begin{exemple}
\région{GOs}
\textbf{\pnua{we-zumee-je}}
\pfra{son crachat}
\end{exemple}
\newline
\relationsémantique{Cf.}{\lien{}{we-phwa ; we-vwa}}
\glosecourte{salive}
\end{entrée}

\begin{entrée}{we-zhido}{}{ⓔwe-zhido}
\région{GA}
\variante{%
we-zedo
\région{GO(s) BO}}
(\domainesémantique{Instruments})
\classe{nom}
\begin{glose}
\pfra{miroir}
\end{glose}
\end{entrée}

\begin{entrée}{wi}{}{ⓔwi}
\variante{%
we
\région{BO}, 
vwo
\région{GO}}
(\domainesémantique{Conjonction})
\classe{CNJ}
\begin{glose}
\pfra{que ; pour}
\end{glose}
\newline
\begin{exemple}
\textbf{\pnua{wi po-ra ?}}
\pfra{pourquoi faire ? (lit. pour faire quoi ?)}
\end{exemple}
\newline
\begin{exemple}
\textbf{\pnua{i nami wi êdu Kuma}}
\pfra{il pense qu'il va descendre à Koumac}
\end{exemple}
\end{entrée}

\begin{entrée}{wîî}{}{ⓔwîî}
\région{GOs PA BO}
\variante{%
wîî-n
\région{BO}, 
wêê-n
\région{BO [Corne]}}
(\domainesémantique{Caractéristiques et propriétés des personnes})
\classe{n ; v.stat.}
\begin{glose}
\pfra{puissance ; force}
\end{glose}
\begin{glose}
\pfra{fort}
\end{glose}
\newline
\begin{exemple}
\région{GOs}
\textbf{\pnua{e wîî}}
\pfra{il est fort}
\end{exemple}
\newline
\begin{exemple}
\région{GOs}
\textbf{\pnua{wîî-je}}
\pfra{sa force}
\end{exemple}
\newline
\begin{exemple}
\région{GOs}
\textbf{\pnua{ge je ni wîî mããni}}
\pfra{il est dans un sommeil profond}
\end{exemple}
\newline
\begin{exemple}
\région{PA}
\textbf{\pnua{wîî-n}}
\pfra{sa force}
\end{exemple}
\newline
\begin{sous-entrée}{wîî mani hubu}{ⓔwîîⓝwîî mani hubu}
\begin{glose}
\pfra{la force physique et le charisme/ et la force spirituelle}
\end{glose}
\end{sous-entrée}
\end{entrée}

\begin{entrée}{wîî-kale}{}{ⓔwîî-kale}
\région{BO}
(\domainesémantique{Marées})
\classe{nom}
\begin{glose}
\pfra{marée étale [Corne]}
\end{glose}
\newline
\note{non vérifié}{général}{}
\end{entrée}

\begin{entrée}{wili}{1}{ⓔwiliⓗ1}
\région{PA BO}
\variante{%
whizi
\région{GO}}
(\domainesémantique{Aliments, alimentation})
\classe{v.t.}
\begin{glose}
\pfra{manger (la canne à sucre)}
\end{glose}
\newline
\relationsémantique{Cf.}{\lien{}{whal èm [PA], whal ê [GOs]}}
\glosecourte{manger de la canne à sucre}
\end{entrée}

\begin{entrée}{wili}{2}{ⓔwiliⓗ2}
\région{PA BO}
\variante{%
huli
\région{BO PA}, 
wele
\région{BO}}
(\domainesémantique{Verbes de déplacement et moyens de déplacement})
\classe{v}
\begin{glose}
\pfra{suivre (se) ; suivre qqn}
\end{glose}
\newline
\begin{exemple}
\région{PA}
\textbf{\pnua{li ra pe-wili du bwa havu}}
\pfra{ils se suivent pour descendre au jardin}
\end{exemple}
\newline
\begin{sous-entrée}{pe-huli [BO PA]}{ⓔwiliⓗ2ⓝpe-huli [BO PA]}
\begin{glose}
\pfra{marcher en file indienne}
\end{glose}
\end{sous-entrée}
\end{entrée}

\begin{entrée}{wizi}{}{ⓔwizi}
\région{GOs}
\variante{%
wili
\région{PA}}
(\domainesémantique{Mouvements ou actions faits avec le corps, les bras, les mains, les pieds})
\classe{v}
\begin{glose}
\pfra{étrangler (avec une corde ou une liane)}
\end{glose}
\newline
\begin{exemple}
\région{GOs}
\textbf{\pnua{wizi nõõ-je}}
\pfra{il s'est étranglé}
\end{exemple}
\newline
\begin{exemple}
\région{GOs}
\textbf{\pnua{nu wizi nõõ-je}}
\pfra{je l'ai étranglé}
\end{exemple}
\end{entrée}

\begin{entrée}{wââng}{}{ⓔwââng}
\région{PA WE}
\variante{%
wãwã
\région{GO(s)}}
(\domainesémantique{Oiseaux})
\classe{nom}
\begin{glose}
\pfra{corbeau}
\end{glose}
\nomscientifique{Corvus moneduloides (Corvidés)}
\end{entrée}

\begin{entrée}{wõ}{}{ⓔwõ}
\formephonétique{wɔ̃}
\région{GOs}
\variante{%
wony
\formephonétique{wɔ̃ɲ}
\région{PA WEM BO}}
\classe{nom}
\newline
\sens{1}
(\domainesémantique{Navigation
, Moyens de locomotion et chemins})
\begin{glose}
\pfra{bateau ; embarcation}
\end{glose}
\newline
\begin{exemple}
\région{GOs}
\textbf{\pnua{wõ-ce wõjo-nu}}
\pfra{mon bateau en bois}
\end{exemple}
\newline
\begin{exemple}
\région{GOs}
\textbf{\pnua{wõjo-nu ca wõ-ce}}
\pfra{mon bateau est en bois}
\end{exemple}
\newline
\sens{2}
(\domainesémantique{Objets et meubles de la maison})
\begin{glose}
\pfra{berceau en fibre de cocotier [WEM, PA, BO]}
\end{glose}
\newline
\sens{3}
(\domainesémantique{Jeux divers})
\begin{glose}
\pfra{figure du jeu de ficelle 'le bateau' [BO]}
\end{glose}
\newline
\étymologie{
\langue{POc}
\étymon{*waŋka(ŋ)}
\glosecourte{bateau}}
\newline
\note{wõjò-nu}{grammaire}{mon bateau}
\end{entrée}

\begin{entrée}{wo-ce}{}{ⓔwo-ce}
\région{GOs}
(\domainesémantique{Objets, outils})
\classe{nom}
\begin{glose}
\pfra{épieu}
\end{glose}
\end{entrée}

\begin{entrée}{wõ-ce}{}{ⓔwõ-ce}
\région{GOs}
(\domainesémantique{Navigation})
\classe{nom}
\begin{glose}
\pfra{pirogue à balancier}
\end{glose}
\end{entrée}

\begin{entrée}{wôdrî}{}{ⓔwôdrî}
\région{GOs}
\variante{%
wôding
\région{PA BO}, 
wèding
\région{BO}}
(\domainesémantique{Topographie})
\classe{nom}
\begin{glose}
\pfra{col de montagne}
\end{glose}
\begin{glose}
\pfra{creux ; dépression (terrain)}
\end{glose}
\newline
\begin{exemple}
\textbf{\pnua{ni wôdi Pwagen}}
\pfra{au col de Pwagen}
\end{exemple}
\end{entrée}

\begin{entrée}{wòdro}{}{ⓔwòdro}
\région{GOs}
\variante{%
wèdò, vòdòòn
\région{BO}}
\classe{v}
\newline
\sens{1}
(\domainesémantique{Discours, échanges verbaux})
\begin{glose}
\pfra{discuter ; palabrer ; disposer de}
\end{glose}
\begin{glose}
\pfra{juger ; jugement}
\end{glose}
\begin{glose}
\pfra{discussions}
\end{glose}
\newline
\sens{2}
(\domainesémantique{Coutumes, dons coutumiers})
\begin{glose}
\pfra{actes coutumiers ; us et coutumes ; usages ; manières ; moeurs}
\end{glose}
\newline
\begin{exemple}
\textbf{\pnua{e wòdroo-ni zòò-ni}}
\pfra{il a porté un bon jugement}
\end{exemple}
\newline
\begin{sous-entrée}{vhaa mani wõdrò}{ⓔwòdroⓢ2ⓝvhaa mani wõdrò}
\begin{glose}
\pfra{paroles et jugements}
\end{glose}
\end{sous-entrée}
\newline
\begin{sous-entrée}{aa-wõdro}{ⓔwòdroⓢ2ⓝaa-wõdro}
\begin{glose}
\pfra{celui qui a bon jugement}
\end{glose}
\newline
\note{wõdro-ni, wèdo-ni (v.t.)}{grammaire}{}
\end{sous-entrée}
\end{entrée}

\begin{entrée}{wogama}{}{ⓔwogama}
\région{BO}
(\domainesémantique{Saisons})
\classe{nom}
\begin{glose}
\pfra{mois où on laboure les champs d'ignames et où on les plante (août-septembre) [Dubois]}
\end{glose}
\newline
\relationsémantique{Cf.}{\lien{}{pweralo, pwebae}}
\newline
\note{non vérifié}{général}{}
\end{entrée}

\begin{entrée}{wõ-go}{}{ⓔwõ-go}
\région{GOs}
(\domainesémantique{Navigation})
\classe{nom}
\begin{glose}
\pfra{radeau en bambou}
\end{glose}
\end{entrée}

\begin{entrée}{wône}{}{ⓔwône}
\formephonétique{wõɳe}
\région{GOs}
\variante{%
wene
\région{PA}}
\classe{v}
\newline
\sens{1}
(\domainesémantique{Verbes d'action (en général)})
\begin{glose}
\pfra{remplacer ; changer (vêtements)}
\end{glose}
\begin{glose}
\pfra{faire de la monnaie}
\end{glose}
\newline
\begin{exemple}
\région{GOs}
\textbf{\pnua{e wône nu}}
\pfra{il me remplace}
\end{exemple}
\newline
\begin{exemple}
\région{GOs}
\textbf{\pnua{e wône ã a-chomu}}
\pfra{il a remplacé cet enseignant}
\end{exemple}
\newline
\begin{sous-entrée}{a-wône-vwo}{ⓔwôneⓢ1ⓝa-wône-vwo}
\région{GOs}
\begin{glose}
\pfra{un remplaçant}
\end{glose}
\end{sous-entrée}
\newline
\begin{sous-entrée}{wône hôbwò}{ⓔwôneⓢ1ⓝwône hôbwò}
\begin{glose}
\pfra{changer de vêtement}
\end{glose}
\end{sous-entrée}
\newline
\begin{sous-entrée}{pe-wône}{ⓔwôneⓢ1ⓝpe-wône}
\begin{glose}
\pfra{échanger}
\end{glose}
\end{sous-entrée}
\newline
\sens{2}
(\domainesémantique{Verbes de déplacement et moyens de déplacement})
\begin{glose}
\pfra{déplacer ; changer de place}
\end{glose}
\newline
\begin{exemple}
\région{GOs}
\textbf{\pnua{e wône dè}}
\pfra{il change de route}
\end{exemple}
\newline
\begin{exemple}
\région{GOs}
\textbf{\pnua{wône choova}}
\pfra{changer un cheval de place}
\end{exemple}
\newline
\begin{sous-entrée}{wône ku}{ⓔwôneⓢ2ⓝwône ku}
\begin{glose}
\pfra{changer de place}
\end{glose}
\end{sous-entrée}
\newline
\begin{sous-entrée}{wône kuna}{ⓔwôneⓢ2ⓝwône kuna}
\begin{glose}
\pfra{changer de lieu}
\end{glose}
\end{sous-entrée}
\end{entrée}

\begin{entrée}{woo}{}{ⓔwoo}
\région{GOs}
(\domainesémantique{Chasse})
\classe{nom}
\begin{glose}
\pfra{lacet}
\end{glose}
\begin{glose}
\pfra{piège}
\end{glose}
\newline
\begin{sous-entrée}{pwe-woo}{ⓔwooⓝpwe-woo}
\begin{glose}
\pfra{noeud coulant}
\end{glose}
\end{sous-entrée}
\end{entrée}

\begin{entrée}{wööe}{}{ⓔwööe}
\région{GOs}
(\domainesémantique{Verbes d'action (en général)})
\classe{nom}
\begin{glose}
\pfra{serrer}
\end{glose}
\newline
\begin{sous-entrée}{phwe-wööe}{ⓔwööeⓝphwe-wööe}
\begin{glose}
\pfra{lasso}
\end{glose}
\end{sous-entrée}
\end{entrée}

\begin{entrée}{wòòzi}{}{ⓔwòòzi}
\formephonétique{wɔːði}
\région{GOs}
\variante{%
wooli
\région{PA BO}, 
woji
\région{BO}}
(\domainesémantique{Arbre})
\classe{nom}
\begin{glose}
\pfra{acajou}
\end{glose}
\nomscientifique{Semecarpus atra}
\end{entrée}

\begin{entrée}{wò phii-me}{}{ⓔwò phii-me}
\région{GOs}
(\domainesémantique{Mouvements ou actions avec la tête, les yeux, la bouche})
\classe{v}
\begin{glose}
\pfra{écarquiller les yeux}
\end{glose}
\newline
\begin{exemple}
\textbf{\pnua{e wò phii-me}}
\pfra{elle écarquille les yeux}
\end{exemple}
\end{entrée}

\begin{entrée}{wõ-phu}{}{ⓔwõ-phu}
\région{GOs}
(\domainesémantique{Moyens de locomotion et chemins})
\classe{nom}
\begin{glose}
\pfra{avion (lit. bateau volant)}
\end{glose}
\newline
\begin{sous-entrée}{mhenõ phe wõ-phu}{ⓔwõ-phuⓝmhenõ phe wõ-phu}
\begin{glose}
\pfra{aérodrome}
\end{glose}
\end{sous-entrée}
\end{entrée}

\begin{entrée}{wõ-pwaala}{}{ⓔwõ-pwaala}
\formephonétique{wõ-pwaːla}
\région{GOs}
\variante{%
wô-waala
\région{GO(s)}}
(\domainesémantique{Navigation})
\classe{nom}
\begin{glose}
\pfra{bateau à voile}
\end{glose}
\end{entrée}

\begin{entrée}{wòvwa}{}{ⓔwòvwa}
\formephonétique{wɔβa}
\région{GOs}
\variante{%
wòpa
\région{vx}, 
wovha
\région{PA}, 
woza
\région{WEM}}
(\domainesémantique{Guerre})
\classe{v ; n}
\begin{glose}
\pfra{bagarre ; bagarrer (se) ; affronter (s')}
\end{glose}
\newline
\begin{exemple}
\textbf{\pnua{li pe-wòvwa}}
\pfra{ils se battent}
\end{exemple}
\end{entrée}

\begin{entrée}{woxa}{}{ⓔwoxa}
\région{GOs WEM BO}
\variante{%
voxa
\région{BO}}
(\domainesémantique{Relations et interaction sociales})
\classe{v}
\begin{glose}
\pfra{nier}
\end{glose}
\newline
\note{woxe (v.t)}{grammaire}{cacher, ne pas dire, tromper}
\end{entrée}

\begin{entrée}{wòzò}{}{ⓔwòzò}
\région{GOs}
\variante{%
wòlò
\région{PA BO}, 
wojò
\région{BO}}
\classe{nom}
\newline
\sens{1}
(\domainesémantique{Objets, outils})
\begin{glose}
\pfra{épieu de culture ; bâton à fouir ; "barre-à-mine"}
\end{glose}
\newline
\begin{exemple}
\région{PA}
\textbf{\pnua{wòlò-ny wò-ce}}
\pfra{mon épieu en bois (par opposition à la barre à mine)}
\end{exemple}
\newline
\sens{2}
(\domainesémantique{Objets coutumiers})
\begin{glose}
\pfra{perche sacrée du champ d'igname (destinée à favoriser la récolte et protéger les plantations)}
\end{glose}
\newline
\étymologie{
\langue{PCEMP}
\étymon{*pasek}}
\end{entrée}

\begin{entrée}{wòzõõ}{}{ⓔwòzõõ}
\région{GOs}
\variante{%
wolõ
\région{PA}}
(\domainesémantique{Santé, maladie})
\classe{nom}
\begin{glose}
\pfra{furoncle}
\end{glose}
\end{entrée}

\newpage

\lettrine{wh}\begin{entrée}{wha}{1}{ⓔwhaⓗ1}
\région{GOs PA BO}
\classe{nom}
(\domainesémantique{Noms des plantes})
\begin{glose}
\pfra{figuier sauvage}
\end{glose}
\nomscientifique{Ficus habrophylla; Ficus edulis}
\newline
\begin{sous-entrée}{pò-wha}{ⓔwhaⓗ1ⓝpò-wha}
\begin{glose}
\pfra{fruit de figuier}
\end{glose}
\end{sous-entrée}
\newline
\begin{sous-entrée}{ci-wha}{ⓔwhaⓗ1ⓝci-wha}
\begin{glose}
\pfra{l'écorce du figuier}
\end{glose}
\end{sous-entrée}
\end{entrée}

\begin{entrée}{wha}{2}{ⓔwhaⓗ2}
\région{GOs}
\variante{%
hua
\région{PA BO}}
(\domainesémantique{Parenté})
\classe{nom}
\begin{glose}
\pfra{grand-père (maternel ou paternel, désignation et appellation)}
\end{glose}
\begin{glose}
\pfra{frère de grand-père}
\end{glose}
\begin{glose}
\pfra{cousin de grand-père}
\end{glose}
\begin{glose}
\pfra{vieux}
\end{glose}
\newline
\begin{exemple}
\région{GO}
\textbf{\pnua{wha i nu}}
\pfra{mon grand-père}
\end{exemple}
\newline
\begin{exemple}
\région{PA}
\textbf{\pnua{hua i nu}}
\pfra{mon grand-père}
\end{exemple}
\newline
\begin{exemple}
\textbf{\pnua{wawa}}
\pfra{grand-papa}
\end{exemple}
\newline
\begin{sous-entrée}{wha-mã}{ⓔwhaⓗ2ⓝwha-mã}
\begin{glose}
\pfra{vieil-homme, ancêtre}
\end{glose}
\end{sous-entrée}
\end{entrée}

\begin{entrée}{wha}{3}{ⓔwhaⓗ3}
\région{GOs}
\variante{%
whal
\région{PA}}
(\domainesémantique{Aliments, alimentation})
\classe{v ; n}
\begin{glose}
\pfra{manger (la canne à sucre)}
\end{glose}
\begin{glose}
\pfra{part de canne à sucre}
\end{glose}
\newline
\begin{exemple}
\région{GOs}
\textbf{\pnua{nu wha ê}}
\pfra{je mange de la canne à sucre}
\end{exemple}
\newline
\begin{exemple}
\région{PA}
\textbf{\pnua{i whal èm}}
\pfra{manger de la canne à sucre}
\end{exemple}
\newline
\begin{sous-entrée}{waza-nu ê}{ⓔwhaⓗ3ⓝwaza-nu ê}
\région{GOs}
\begin{glose}
\pfra{ma part de canne à sucre}
\end{glose}
\end{sous-entrée}
\newline
\begin{sous-entrée}{whala-n [PA], waza-je [GOs]}{ⓔwhaⓗ3ⓝwhala-n [PA], waza-je [GOs]}
\begin{glose}
\pfra{sa part de canne à sucre}
\end{glose}
\newline
\note{classificateur possessif : whala-n [PA], waza-je [GOs]}{grammaire}{}
\end{sous-entrée}
\newline
\relationsémantique{Cf.}{\lien{}{w(h)ili [PA], wizi [GOs]}}
\glosecourte{manger de la canne à sucre}
\newline
\relationsémantique{Cf.}{\lien{}{i whili èm [PA]}}
\glosecourte{il mange de la canne à sucre}
\end{entrée}

\begin{entrée}{whã}{}{ⓔwhã}
\région{GOs WEM}
\variante{%
wã
\région{GOs}, 
wa
\région{PA BO}, 
wame
\région{GO(s)}}
(\domainesémantique{Comparaison})
\classe{v.COMPAR}
\begin{glose}
\pfra{comme (être)}
\end{glose}
\begin{glose}
\pfra{faire comme ; faire ainsi (= dire ainsi)}
\end{glose}
\newline
\begin{exemple}
\région{GOs}
\textbf{\pnua{e ne wãã-na}}
\pfra{il l'a fait ainsi (anaphorique)}
\end{exemple}
\newline
\begin{exemple}
\région{GOs}
\textbf{\pnua{ne xo wã}}
\pfra{fais-le ainsi !}
\end{exemple}
\newline
\begin{exemple}
\région{GOs}
\textbf{\pnua{e wã mwa xo Kaawo : "ko (= kawa, kawö) ço nooli poi-nu ?"}}
\pfra{Kaawo fait/dit : "tu n'as pas vu mon enfant ?"}
\end{exemple}
\newline
\begin{exemple}
\région{PA}
\textbf{\pnua{i wã : "kawu jo nooli ja poi-ny?"}}
\pfra{elle fait/dit : "tu n'as pas vu mon enfant?"}
\end{exemple}
\newline
\begin{exemple}
\textbf{\pnua{axe wã xo dony}}
\pfra{mais la buse fait ainsi}
\end{exemple}
\newline
\begin{exemple}
\textbf{\pnua{wã ai-ne}}
\pfra{comme il veut (Dubois)}
\end{exemple}
\newline
\begin{exemple}
\textbf{\pnua{wã ãgu}}
\pfra{en tant qu'homme (Dubois)}
\end{exemple}
\newline
\begin{sous-entrée}{i wã-na}{ⓔwhãⓝi wã-na}
\begin{glose}
\pfra{il (est resté) comme cela (BO, Dubois)}
\end{glose}
\newline
\relationsémantique{Cf.}{\lien{ⓔwhaya ?}{whaya ?}}
\glosecourte{être comment ?}
\end{sous-entrée}
\end{entrée}

\begin{entrée}{whaa}{1}{ⓔwhaaⓗ1}
\région{GOs BO}
(\domainesémantique{Description des objets, formes, consistance, taille})
\classe{v}
\begin{glose}
\pfra{grand ; gros}
\end{glose}
\begin{glose}
\pfra{grandir ; croître; pousser (en long)}
\end{glose}
\newline
\begin{exemple}
\textbf{\pnua{e whaa}}
\pfra{il grandit}
\end{exemple}
\newline
\relationsémantique{Ant.}{\lien{}{pònõ}}
\glosecourte{petit}
\end{entrée}

\begin{entrée}{whaa}{2}{ⓔwhaaⓗ2}
\formephonétique{wʰaː}
\région{GOs}
\variante{%
waa
\région{WEM}, 
waang
\formephonétique{waːŋ}
\région{PA BO}, 
waak
\région{PA}}
(\domainesémantique{Découpage du temps})
\classe{n.LOC ; v}
\begin{glose}
\pfra{matin ; faire jour}
\end{glose}
\newline
\begin{exemple}
\région{GOs}
\textbf{\pnua{e gaa whaa}}
\pfra{c'est encore très tôt le matin}
\end{exemple}
\newline
\begin{exemple}
\région{GOs}
\textbf{\pnua{e mha whaa gò}}
\pfra{c'est encore trop tôt}
\end{exemple}
\newline
\begin{exemple}
\région{WEM}
\textbf{\pnua{e thau xa gaa waa gò}}
\pfra{il est arrivé très tôt ce matin}
\end{exemple}
\newline
\begin{sous-entrée}{gaa waa}{ⓔwhaaⓗ2ⓝgaa waa}
\begin{glose}
\pfra{très tôt le matin}
\end{glose}
\newline
\relationsémantique{Cf.}{\lien{}{goona, kaça-huvwo, thrõbwo}}
\end{sous-entrée}
\end{entrée}

\begin{entrée}{whaadrangi}{}{ⓔwhaadrangi}
\région{GOs}
\variante{%
wadaga
\région{BO}}
(\domainesémantique{Guerre})
\classe{nom}
\begin{glose}
\pfra{ennemi}
\end{glose}
\end{entrée}

\begin{entrée}{whaa-gò}{}{ⓔwhaa-gò}
\région{GOs}
(\domainesémantique{Découpage du temps})
\classe{LOC}
\begin{glose}
\pfra{aube ; matin de bonne heure}
\end{glose}
\newline
\begin{exemple}
\textbf{\pnua{e phe-mogu ne whaa-gò}}
\pfra{elle commence le travail tôt le matin}
\end{exemple}
\end{entrée}

\begin{entrée}{whaça}{}{ⓔwhaça}
\formephonétique{wʰadʒa}
\région{GOs}
\variante{%
waaya
\région{PA}, 
waiza
\région{BO}, 
whayap
\région{BO}}
(\domainesémantique{Guerre})
\classe{v ; n}
\begin{glose}
\pfra{guerre ; lutte ;}
\end{glose}
\begin{glose}
\pfra{combattre ; lutter (pour avoir qqch) ;}
\end{glose}
\begin{glose}
\pfra{assaillir (pour obtenir qqch) [GOs]}
\end{glose}
\newline
\begin{sous-entrée}{pe-waaça}{ⓔwhaçaⓝpe-waaça}
\région{GO}
\begin{glose}
\pfra{combattre (se); lutter l'un contre l'autre}
\end{glose}
\end{sous-entrée}
\newline
\begin{sous-entrée}{pe-waaya}{ⓔwhaçaⓝpe-waaya}
\région{PA}
\begin{glose}
\pfra{s'affronter}
\end{glose}
\end{sous-entrée}
\end{entrée}

\begin{entrée}{wha ê}{}{ⓔwha ê}
\région{GOs}
(\domainesémantique{Aliments, alimentation})
\classe{v}
\begin{glose}
\pfra{manger de la canne à sucre}
\end{glose}
\newline
\begin{exemple}
\textbf{\pnua{nu wha ê}}
\pfra{je mange de la canne à sucre}
\end{exemple}
\newline
\relationsémantique{Cf.}{\lien{}{whizi, w(h)ili}}
\glosecourte{manger (canne à sucre)}
\newline
\relationsémantique{Cf.}{\lien{}{whal èm [PA]}}
\glosecourte{manger de la canne à sucre}
\end{entrée}

\begin{entrée}{whai}{1}{ⓔwhaiⓗ1}
\région{GOs}
(\domainesémantique{Poissons})
\classe{nom}
\begin{glose}
\pfra{mulet (de mer, de petite taille)}
\end{glose}
\newline
\relationsémantique{Cf.}{\lien{ⓔnaxo}{naxo}}
\glosecourte{mulet noir (de mer)}
\newline
\relationsémantique{Cf.}{\lien{ⓔmene}{mene}}
\glosecourte{mulet queue bleue de mer)}
\end{entrée}

\begin{entrée}{whai}{2}{ⓔwhaiⓗ2}
\région{GOs}
\variante{%
wha
\région{PA}}
(\domainesémantique{Cultures, techniques, boutures})
\classe{v}
\begin{glose}
\pfra{récolter (manioc en arrachant)}
\end{glose}
\newline
\begin{exemple}
\région{PA}
\textbf{\pnua{i wha manyô}}
\pfra{il arrache le manioc}
\end{exemple}
\newline
\begin{exemple}
\région{PA}
\textbf{\pnua{i wha manyô u ri ?}}
\pfra{qui a arraché le manioc}
\end{exemple}
\end{entrée}

\begin{entrée}{whãi}{}{ⓔwhãi}
\région{GOs}
(\domainesémantique{Danses})
\classe{nom}
\begin{glose}
\pfra{danse (pour le deuil du petit chef, dansé par les oncles maternels)}
\end{glose}
\newline
\begin{exemple}
\textbf{\pnua{ciia whãi}}
\pfra{danse pour le deuil du petit chef}
\end{exemple}
\end{entrée}

\begin{entrée}{whala-}{}{ⓔwhala-}
\région{PA}
\variante{%
wala-
\région{PA}, 
waza
\région{GO(s)}}
(\domainesémantique{Préfixes classificateurs de la nourriture
, Préfixes classificateurs possessifs de la nourriture})
\classe{nom}
\begin{glose}
\pfra{part de canne à sucre}
\end{glose}
\newline
\begin{exemple}
\textbf{\pnua{ai-xa whala-m ?}}
\pfra{veux-tu ta part de canne à sucre ?}
\end{exemple}
\newline
\begin{exemple}
\textbf{\pnua{wili whala-m !}}
\pfra{mange ta canne à sucre !}
\end{exemple}
\newline
\relationsémantique{Cf.}{\lien{}{w(h)ili, wizi}}
\glosecourte{manger (la canne à sucre)}
\end{entrée}

\begin{entrée}{wha-mã}{}{ⓔwha-mã}
\région{GOs PA BO}
\variante{%
hua-mã
\région{vx}}
(\domainesémantique{Cours de la vie})
\classe{v ; n}
\begin{glose}
\pfra{vieux (les) ; vieil homme ; parents}
\end{glose}
\begin{glose}
\pfra{grandir ; vieillir (animés)}
\end{glose}
\newline
\begin{exemple}
\région{GOs}
\textbf{\pnua{wha-mã i nu}}
\pfra{mes parents (lit. mes vieux)}
\end{exemple}
\newline
\begin{exemple}
\région{GOs}
\textbf{\pnua{nu pò wha-mã nai jö}}
\pfra{je suis un peu plus vieille que toi}
\end{exemple}
\newline
\begin{exemple}
\région{GOs}
\textbf{\pnua{whamã i ã ègõgo}}
\pfra{nos vieux d'avant (lit. grand-père de nous avant)}
\end{exemple}
\newline
\begin{exemple}
\région{GOs}
\textbf{\pnua{e za u wha-mã ẽnõ-ni}}
\pfra{cet enfant a bien grandi}
\end{exemple}
\newline
\begin{exemple}
\région{GOs}
\textbf{\pnua{i mha wha-mã}}
\pfra{c'est l'aîné}
\end{exemple}
\newline
\begin{exemple}
\région{GOs}
\textbf{\pnua{pòi-nu wha-mã}}
\pfra{mon aîné}
\end{exemple}
\newline
\relationsémantique{Cf.}{\lien{}{hua-mã [PA]}}
\glosecourte{ancêtre}
\newline
\relationsémantique{Cf.}{\lien{ⓔthoimwã}{thoimwã}}
\glosecourte{vieille femme}
\newline
\relationsémantique{Ant.}{\lien{}{pò ẽnõ}}
\glosecourte{plus jeune}
\end{entrée}

\begin{entrée}{wha-maama}{}{ⓔwha-maama}
\région{WEM WE}
(\domainesémantique{Parenté})
\classe{nom}
\begin{glose}
\pfra{arrière-grand-père}
\end{glose}
\end{entrée}

\begin{entrée}{whany}{}{ⓔwhany}
\région{PA BO}
\variante{%
wany
\région{PA BO}}
(\domainesémantique{Religion, représentations religieuses
, Relations et interaction sociales})
\classe{v ; n}
\begin{glose}
\pfra{malédiction ; punition ; punir}
\end{glose}
\begin{glose}
\pfra{esprits des vieux du clan}
\end{glose}
\end{entrée}

\begin{entrée}{wha-thraa}{}{ⓔwha-thraa}
\région{GOs}
\variante{%
wha-rhaa
\région{GO(s)}}
(\domainesémantique{Parenté})
\classe{nom}
\begin{glose}
\pfra{arrière-grand-père}
\end{glose}
\end{entrée}

\begin{entrée}{whau}{}{ⓔwhau}
\région{GOs}
\variante{%
whãup
\région{PA}}
(\domainesémantique{Corps humain})
\classe{v}
\begin{glose}
\pfra{édenté}
\end{glose}
\end{entrée}

\begin{entrée}{whaya ?}{}{ⓔwhaya ?}
\région{GOs BO PA}
\classe{INT}
\newline
\sens{1}
(\domainesémantique{Interrogatifs})
\begin{glose}
\pfra{comment (être, faire) ? (aussi pour les propriétés physiques)}
\end{glose}
\newline
\begin{exemple}
\textbf{\pnua{e whaya phago ?}}
\pfra{quelle est sa couleur ?}
\end{exemple}
\newline
\begin{exemple}
\région{GOs}
\textbf{\pnua{êmwê xa whaya ?}}
\pfra{un homme comment (physiquement) ?}
\end{exemple}
\newline
\begin{exemple}
\région{GOs}
\textbf{\pnua{pwaji xa whaya ? - pwaji xa baa}}
\pfra{un crabe comment? - un crabe noir}
\end{exemple}
\newline
\begin{exemple}
\région{GOs}
\textbf{\pnua{hèlè xa whaya ? - hèlè xa ca - hèlè xa khawali}}
\pfra{un couteau comment? - un couteau affûté - un grand couteau}
\end{exemple}
\newline
\begin{exemple}
\région{GOs}
\textbf{\pnua{e whaya mwê-jeêmwê-e ?}}
\pfra{quelle sorte d'homme est-ce ? (lit. comment sont ses manières ?)}
\end{exemple}
\newline
\begin{exemple}
\région{GOs}
\textbf{\pnua{la ne whaya-le kibi ?}}
\pfra{comment ont-ils fait le four ?}
\end{exemple}
\newline
\begin{exemple}
\région{BO}
\textbf{\pnua{whaya me teyimwi pwaji ?}}
\pfra{comment attrape-t-on des crabes ?}
\end{exemple}
\newline
\begin{exemple}
\région{BO}
\textbf{\pnua{i hivine (kôbwe) yu nee whaya-le}}
\pfra{il se sait pas comment tu as fait}
\end{exemple}
\newline
\begin{exemple}
\textbf{\pnua{i mha whaya ?}}
\pfra{comment est-il malade ? (Dubois)}
\end{exemple}
\newline
\begin{exemple}
\textbf{\pnua{ka i whaya-le ?}}
\pfra{comment l'a-t-il fait ? (Dubois)}
\end{exemple}
\newline
\sens{2}
(\domainesémantique{Interrogatifs})
\begin{glose}
\pfra{combien?}
\end{glose}
\newline
\begin{exemple}
\textbf{\pnua{i hova na whaya hinõ-al?}}
\pfra{à quelle heure rentre-t-il?}
\end{exemple}
\newline
\begin{exemple}
\textbf{\pnua{whaya kau-m?}}
\pfra{quel âge as-tu ?}
\end{exemple}
\newline
\note{whayale (v.t.)}{grammaire}{}
\end{entrée}

\begin{entrée}{whayu}{}{ⓔwhayu}
\région{GOs PA}
\variante{%
wayu
\région{BO}}
(\domainesémantique{Mouvements ou actions avec la tête, les yeux, la bouche})
\classe{v}
\begin{glose}
\pfra{siffler avec les lèvres}
\end{glose}
\end{entrée}

\begin{entrée}{whili}{}{ⓔwhili}
\formephonétique{wʰɨlɨ}
\région{GOs WEM BO PA}
(\domainesémantique{Verbes de déplacement et moyens de déplacement
, Mouvements ou actions faits avec le corps, les bras, les mains, les pieds})
\classe{v}
\begin{glose}
\pfra{conduire ; guider}
\end{glose}
\begin{glose}
\pfra{amener ; emmener}
\end{glose}
\begin{glose}
\pfra{conduire (voiture) [PA]}
\end{glose}
\begin{glose}
\pfra{prendre par la main (enfant)}
\end{glose}
\begin{glose}
\pfra{chercher (épouse)}
\end{glose}
\begin{glose}
\pfra{mener (travail)}
\end{glose}
\newline
\begin{exemple}
\région{GOs}
\textbf{\pnua{nu whili dree}}
\pfra{je montre le chemin}
\end{exemple}
\newline
\begin{exemple}
\région{GOs}
\textbf{\pnua{nu whili-jö da-mi}}
\pfra{je t'ai amené ici en haut}
\end{exemple}
\newline
\begin{exemple}
\région{WEM}
\textbf{\pnua{pwawa ne jö whili-mi lana pòi-m ?}}
\pfra{peux-tu amener tes enfants ici ?}
\end{exemple}
\newline
\begin{exemple}
\région{PA}
\textbf{\pnua{i whili nye nyama}}
\pfra{il a mené ce travail}
\end{exemple}
\end{entrée}

\begin{entrée}{whili thòòmwa}{}{ⓔwhili thòòmwa}
\région{GOs}
\variante{%
huli thòòmwa
\région{WE}}
(\domainesémantique{Coutumes, dons coutumiers})
\classe{nom}
\begin{glose}
\pfra{coutume (cérémonie) de mariage (lit. chercher la femme)}
\end{glose}
\end{entrée}

\begin{entrée}{whizi}{}{ⓔwhizi}
\région{GOs}
\variante{%
wili, whili
\région{PA BO WE}}
(\domainesémantique{Aliments, alimentation})
\classe{v}
\begin{glose}
\pfra{manger (canne à sucre)}
\end{glose}
\begin{glose}
\pfra{mâcher de la canne à sucre}
\end{glose}
\newline
\begin{exemple}
\région{GOs}
\textbf{\pnua{a-vwö nu whizi ê}}
\pfra{j'ai envie de manger de la canne à sucre}
\end{exemple}
\newline
\begin{exemple}
\région{PA}
\textbf{\pnua{wili whala-m}}
\pfra{mange ta canne à sucre}
\end{exemple}
\newline
\begin{exemple}
\région{PA WE}
\textbf{\pnua{i wili whala-n èm}}
\pfra{il mange la canne à sucre}
\end{exemple}
\newline
\begin{exemple}
\région{PA}
\textbf{\pnua{i wili-ogine whala-n}}
\pfra{il a fini de manger sa canne à sucre}
\end{exemple}
\newline
\begin{exemple}
\région{PA WE}
\textbf{\pnua{i w(h)ili èm}}
\pfra{il mange la canne à sucre}
\end{exemple}
\newline
\relationsémantique{Cf.}{\lien{ⓔhovwo}{hovwo}}
\glosecourte{manger (général)}
\newline
\relationsémantique{Cf.}{\lien{}{cèni [GOs], cani}}
\glosecourte{manger (féculents)}
\newline
\relationsémantique{Cf.}{\lien{}{huu, huli}}
\glosecourte{manger (nourriture carnée)}
\newline
\relationsémantique{Cf.}{\lien{ⓔbiije}{biije}}
\glosecourte{mêcher des écorces ou du magnania}
\newline
\relationsémantique{Cf.}{\lien{ⓔkûûńiⓗ1}{kûûńi}}
\glosecourte{manger (fruits)}
\newline
\relationsémantique{Cf.}{\lien{}{whizi, w(h)ili}}
\glosecourte{manger (canne à sucre)}
\newline
\relationsémantique{Cf.}{\lien{}{wha ê [GOs] ; whal èm [PA],}}
\glosecourte{manger de la canne à sucre}
\end{entrée}

\begin{entrée}{whòi}{}{ⓔwhòi}
\région{GOs PA}
(\domainesémantique{Mouvements ou actions faits avec le corps, les bras, les mains, les pieds})
\classe{v}
\begin{glose}
\pfra{fouiller (dans un sac, une poche)}
\end{glose}
\newline
\begin{exemple}
\textbf{\pnua{whòi du ni kee !}}
\pfra{fouille dans le sac !}
\end{exemple}
\end{entrée}

\begin{entrée}{whun}{}{ⓔwhun}
\région{BO}
(\domainesémantique{Caractéristiques et propriétés des personnes})
\classe{v ; n}
\begin{glose}
\pfra{silencieux ; silence [Corne]}
\end{glose}
\end{entrée}

\newpage

\lettrine{x}\begin{entrée}{xa}{1}{ⓔxaⓗ1}
\région{GOs PA WE BO}
\variante{%
ga
\région{PA}}
(\domainesémantique{})
\classe{INDEF}
\begin{glose}
\pfra{un(e) certaine}
\end{glose}
\newline
\begin{exemple}
\région{GOs}
\textbf{\pnua{e trilòò kêê-je xa dili}}
\pfra{il demande à son père de la terre}
\end{exemple}
\newline
\begin{exemple}
\région{PA}
\textbf{\pnua{haivwo xa yaala-la}}
\pfra{ils ont beaucoup de noms}
\end{exemple}
\newline
\begin{exemple}
\région{PA BO}
\textbf{\pnua{ni xa tèèn}}
\pfra{un jour}
\end{exemple}
\newline
\begin{exemple}
\région{PA}
\textbf{\pnua{kavwu nu vha mwa na ni-xa kun mwa}}
\pfra{je ne vais pas parler d'un clan/endroit quelconque}
\end{exemple}
\newline
\begin{exemple}
\région{PA}
\textbf{\pnua{koen-xa ãbaa wony}}
\pfra{certains bateaux ont disparu}
\end{exemple}
\newline
\begin{exemple}
\région{GOs}
\textbf{\pnua{e na pòi-je xa mhavwa lai}}
\pfra{elle donne à son enfant un peu de riz}
\end{exemple}
\newline
\begin{exemple}
\région{GOs}
\textbf{\pnua{da-xa na co nõõli?}}
\pfra{que regardes-tu?}
\end{exemple}
\newline
\begin{exemple}
\région{GOs}
\textbf{\pnua{ti-xa na co nõõli?}}
\pfra{qui regardes-tu?}
\end{exemple}
\newline
\begin{exemple}
\région{GOs}
\textbf{\pnua{kia-xa na co nõõli}}
\pfra{tu ne regardes rien}
\end{exemple}
\newline
\begin{exemple}
\textbf{\pnua{ge ni-xa}}
\pfra{il est quelque part}
\end{exemple}
\newline
\begin{exemple}
\région{GOs}
\textbf{\pnua{kixa mwa xa kui}}
\pfra{il n'y a plus d'igname}
\end{exemple}
\newline
\begin{exemple}
\région{GOs}
\textbf{\pnua{e trõne kõbwe me ge-le-xa thoomwa xa Mwani-mi}}
\pfra{il a entendu dire qu'il y avait une femme qui s'appelait Mwani-mi}
\end{exemple}
\newline
\begin{exemple}
\région{GOs}
\textbf{\pnua{kavwö jö trône nye ẽnõ-zòòmwa ge-je (ni) xa bwa drau}}
\pfra{n'as tu pas entendu (parler) d'une jeune-fille qui serait sur une île}
\end{exemple}
\newline
\begin{exemple}
\région{GOs}
\textbf{\pnua{kavwö jö nõõli-xa thòòmwa na kõbwe Mwani-mi ?}}
\pfra{n'as tu pas vu une certaine jeune-fille qu'on appelle Mwani-mi ?}
\end{exemple}
\newline
\begin{exemple}
\région{GOs}
\textbf{\pnua{Na khõbwe cö trõne khõbwe ge-le-xa thoomwã xa Mwani-mii}}
\pfra{Si jamais tu entends dire qu'il y a une femme du nom de Mwańi-mi}
\end{exemple}
\newline
\begin{exemple}
\région{GOs}
\textbf{\pnua{me waaju vwo me kila-xa whaya me tròòli xo mwani}}
\pfra{nous nous efforçons de chercher comment gagner de l'argent}
\end{exemple}
\newline
\begin{exemple}
\région{GOs}
\textbf{\pnua{a khila xa hèlè na ca}}
\pfra{va chercher un couteau qui coupe}
\end{exemple}
\newline
\begin{exemple}
\textbf{\pnua{i khila xa poo}}
\pfra{il cherche qqch.}
\end{exemple}
\newline
\begin{exemple}
\textbf{\pnua{nu nõõlî xa poo}}
\pfra{je vois qqch.}
\end{exemple}
\newline
\begin{exemple}
\région{PA}
\textbf{\pnua{i toone xa egu}}
\pfra{il entend qqn}
\end{exemple}
\newline
\begin{exemple}
\région{PA}
\textbf{\pnua{ti xa egu na i cabi mwa ?}}
\pfra{qui a frappé à la maison ?}
\end{exemple}
\end{entrée}

\begin{entrée}{xa}{2}{ⓔxaⓗ2}
\région{GOs BO PA}
\variante{%
ga
\région{PA}, 
ra
\région{WEM}}
\newline
\groupe{A}
\classe{COORD}
\newline
\sens{1}
(\domainesémantique{Conjonction})
\begin{glose}
\pfra{et, aussi [PA]}
\end{glose}
\newline
\begin{exemple}
\région{PA}
\textbf{\pnua{puyol xo je, xa u minong doo}}
\pfra{il cuisine et la marmite est prête}
\end{exemple}
\newline
\begin{exemple}
\région{PA}
\textbf{\pnua{i têên xa (i) wal}}
\pfra{il court et (il) chante}
\end{exemple}
\newline
\sens{2}
(\domainesémantique{Conjonction})
\begin{glose}
\pfra{mais [GOs]}
\end{glose}
\newline
\begin{exemple}
\région{GOs}
\textbf{\pnua{e kolaadu-je xa e phu wî-je}}
\pfra{il est maigre, mais il a de la force}
\end{exemple}
\newline
\begin{exemple}
\région{GOs}
\textbf{\pnua{e kawali-je xa e kolaadu-je}}
\pfra{il est grand, mais il est maigre}
\end{exemple}
\newline
\groupe{B}
(\domainesémantique{Conjonction})
\classe{FOCUS}
\begin{glose}
\pfra{assertion, focus}
\end{glose}
\newline
\begin{exemple}
\région{PA}
\textbf{\pnua{e ra molo ? - ôô xa, e ta molo}}
\pfra{elle vit encore ? - oui bien sûr! elle vit encore}
\end{exemple}
\end{entrée}

\begin{entrée}{xa}{3}{ⓔxaⓗ3}
\région{GOs}
(\domainesémantique{Conjonction})
\classe{CONJ}
\begin{glose}
\pfra{quand (passé)}
\end{glose}
\newline
\begin{sous-entrée}{dròrò xa waa}{ⓔxaⓗ3ⓝdròrò xa waa}
\région{GOs}
\begin{glose}
\pfra{hier matin}
\end{glose}
\newline
\begin{exemple}
\région{GOs}
\textbf{\pnua{e uja gò xa jaxa we-tru ka}}
\pfra{elle est arrivée il y a 2 ans}
\end{exemple}
\newline
\relationsémantique{Cf.}{\lien{}{mõnõ na waa [GOs]}}
\glosecourte{demain matin}
\end{sous-entrée}
\end{entrée}

\begin{entrée}{xa}{4}{ⓔxaⓗ4}
\région{GOs PA BO}
\variante{%
ka
\région{vx}}
(\domainesémantique{Conjonction})
\classe{CONJ.REL}
\begin{glose}
\pfra{que ; qui}
\end{glose}
\newline
\begin{exemple}
\textbf{\pnua{ègu xa li du-mi nyama vuli}}
\pfra{personnes qui viennent traduire}
\end{exemple}
\newline
\begin{exemple}
\région{GOs}
\textbf{\pnua{ègu xa la ã-mi mògu bwa vhaa i ã}}
\pfra{des gens qui sont venus travailler sur notre langue}
\end{exemple}
\newline
\begin{exemple}
\textbf{\pnua{êmwê xa ti ?}}
\pfra{quel homme ?}
\end{exemple}
\newline
\begin{exemple}
\région{PA}
\textbf{\pnua{i egu xa i aa-vhaa}}
\pfra{c'est un bavard}
\end{exemple}
\newline
\begin{exemple}
\région{GO}
\textbf{\pnua{zixôô-nu ẽnõ êmwê xa a-xe xa yaaza-je Jae}}
\pfra{mon conte d'un jeune garçon qui s'appelle Jae}
\end{exemple}
\newline
\begin{exemple}
\textbf{\pnua{mèni xa whaya ?}}
\pfra{un oiseau comment ? de quelle sorte ?}
\end{exemple}
\newline
\begin{exemple}
\textbf{\pnua{loto xa whaya ?}}
\pfra{une voiture de quelle sorte ?}
\end{exemple}
\newline
\begin{exemple}
\région{GOs}
\textbf{\pnua{pwaji xa whaya ? - pwaji xa baa}}
\pfra{un crabe comment? - un crabe noir}
\end{exemple}
\end{entrée}

\begin{entrée}{xa}{5}{ⓔxaⓗ5}
\région{GOs}
(\domainesémantique{Aspect})
\classe{REV ; ITER GO(s)}
\begin{glose}
\pfra{encore ; de nouveau}
\end{glose}
\newline
\begin{exemple}
\textbf{\pnua{kaavwö nu xa kûûni pwaamwa-e}}
\pfra{je n'ai pas encore fini ce champ}
\end{exemple}
\newline
\begin{exemple}
\textbf{\pnua{e xa ã-da}}
\pfra{il est remonté}
\end{exemple}
\newline
\begin{exemple}
\textbf{\pnua{e xa thoma-jo iò}}
\pfra{il t'a rappelé tout à l'heure}
\end{exemple}
\newline
\begin{exemple}
\textbf{\pnua{e xa mããni}}
\pfra{il s'est rendormi}
\end{exemple}
\newline
\begin{exemple}
\textbf{\pnua{a we xa a kaze}}
\pfra{repartez à la pêche !}
\end{exemple}
\newline
\begin{exemple}
\textbf{\pnua{e za xa ã-da}}
\pfra{il est déjà/vraiment remonté}
\end{exemple}
\newline
\begin{exemple}
\textbf{\pnua{Na za xa bwovwô ã-è Jae : "Kani, nu bwovwô !"}}
\pfra{quand Jae est à nouveau fatigué : "canard, je suis fatigué"}
\end{exemple}
\end{entrée}

\begin{entrée}{xaatra}{}{ⓔxaatra}
\région{GOs}
(\domainesémantique{Poissons})
\classe{nom}
\begin{glose}
\pfra{aiguillette, demi-bec à taches noires}
\end{glose}
\nomscientifique{Hemirhamphus far (Hemiramphidae)}
\end{entrée}

\begin{entrée}{xaatròe}{}{ⓔxaatròe}
\formephonétique{'ɣaːɽɔe}
\région{GOs}
\variante{%
xaaròe
\région{PA}}
(\domainesémantique{Poissons})
\classe{nom}
\begin{glose}
\pfra{"napoléon" (poisson du chef)}
\end{glose}
\nomscientifique{Cheilinus undulatus (Labridés)}
\end{entrée}

\begin{entrée}{-xè}{}{ⓔ-xè}
\variante{%
po-xe
\région{GO PA BO}}
(\domainesémantique{Numéraux cardinaux})
\classe{NUM}
\begin{glose}
\pfra{un}
\end{glose}
\end{entrée}

\begin{entrée}{xhii}{}{ⓔxhii}
\région{GOs}
\variante{%
khi
\région{GO(s)}, 
khiny
\région{PA BO WEM WE}}
\classe{nom}
\newline
\sens{1}
(\domainesémantique{Oiseaux})
\begin{glose}
\pfra{hirondelle busière ; langrayen à ventre blanc (PA)}
\end{glose}
\nomscientifique{Artamus leucorhynchus melanoleucus, Artamidés}
\newline
\sens{2}
(\domainesémantique{Oiseaux})
\begin{glose}
\pfra{hirondelle du Pacifique (à dos blanc)}
\end{glose}
\nomscientifique{Hirundo tahitica subfusca}
\end{entrée}

\begin{entrée}{xo}{1}{ⓔxoⓗ1}
\région{GOs PA}
\variante{%
ko, go
\région{GO(s)}, 
vwo, o, u
\région{PA BO}}
(\domainesémantique{Agent})
\classe{AGT}
\begin{glose}
\pfra{sujet (marque de sujet des verbes actifs)}
\end{glose}
\newline
\begin{exemple}
\région{PA}
\textbf{\pnua{e gi xo khiny}}
\pfra{l'hirondelle pleure}
\end{exemple}
\newline
\begin{exemple}
\région{PA}
\textbf{\pnua{i phweween xo Kaavo}}
\pfra{Kaavo se retourne}
\end{exemple}
\newline
\begin{exemple}
\région{GO}
\textbf{\pnua{e khõbwe xo je}}
\pfra{il/elle dit}
\end{exemple}
\newline
\begin{exemple}
\région{GO}
\textbf{\pnua{e trabwa xo thoomwã xo puyol}}
\pfra{la femme est assise en train de faire la cuisine}
\end{exemple}
\newline
\begin{exemple}
\région{PA}
\textbf{\pnua{burom mwa xo je}}
\pfra{elle se baigne encore, elle est en train de se baigner}
\end{exemple}
\newline
\begin{exemple}
\textbf{\pnua{e õgine mõgu i ã xo ẽnõ ã}}
\pfra{cet enfant a fini notre travail (Doriane)}
\end{exemple}
\newline
\begin{exemple}
\textbf{\pnua{e thuvwu-õgine mõgu i je xo ẽnõ ã}}
\pfra{cet enfant à fini son travail (Doriane)}
\end{exemple}
\newline
\begin{exemple}
\textbf{\pnua{e kõbwe xo/ko kêê-nu kõbwe e zo na jö cuxi na ni mõlõõ-jö}}
\pfra{mon père dit qu'il faut que tu sois courageux dans ta vie}
\end{exemple}
\newline
\begin{exemple}
\région{GOs}
\textbf{\pnua{e alöe ciia xo zine}}
\pfra{le rat regarde le poulpe}
\end{exemple}
\newline
\begin{exemple}
\textbf{\pnua{e pa-tha na ni mèni xo T. Paak}}
\pfra{T. Paak a raté l'oiseau}
\end{exemple}
\newline
\begin{exemple}
\textbf{\pnua{e kibao mèni xo T. Paak}}
\pfra{T. Paak a tué l'oiseau}
\end{exemple}
\newline
\begin{exemple}
\textbf{\pnua{e pe-thumenõ bwa de xo ti ?}}
\pfra{quimarche sur le chemin ? (Doriane)}
\end{exemple}
\newline
\begin{exemple}
\textbf{\pnua{li pe-thumenõ bulu bwa de xo Kaavo ma Hiixe}}
\pfra{Kaavo et Hiixe marchent ensemble sur le chemin (Doriane)}
\end{exemple}
\newline
\begin{exemple}
\région{GOs}
\textbf{\pnua{e kòròò-nu xo hovwo}}
\pfra{j'ai avalé de travers, je me suis étouffé (lit. la nourriture m'a étouffé)}
\end{exemple}
\newline
\begin{exemple}
\textbf{\pnua{hèlè xa i uvwi xo Jan}}
\pfra{le couteau que Jean a acheté}
\end{exemple}
\end{entrée}

\begin{entrée}{xo}{2}{ⓔxoⓗ2}
\région{GOs PA}
(\domainesémantique{Conjonction})
\classe{CNJ}
\begin{glose}
\pfra{aussi ; et aussi}
\end{glose}
\newline
\begin{exemple}
\textbf{\pnua{inu xo}}
\pfra{moi aussi}
\end{exemple}
\newline
\begin{exemple}
\textbf{\pnua{ijo xo}}
\pfra{toi aussi}
\end{exemple}
\newline
\begin{exemple}
\région{GO}
\textbf{\pnua{kavwö nu trõńe, xo kavwö nu nõõ-je}}
\pfra{je n'en n'ai ni entendu parler, ni ne l'ai vu(e)}
\end{exemple}
\newline
\begin{exemple}
\textbf{\pnua{e kawali-je xo kolaadu-je}}
\pfra{il est grand et maigre}
\end{exemple}
\newline
\begin{exemple}
\textbf{\pnua{e wî xo kawali-je}}
\pfra{il est fort et grand}
\end{exemple}
\end{entrée}

\begin{entrée}{xo}{3}{ⓔxoⓗ3}
\région{GOs BO PA}
\variante{%
o
\région{BO}}
\newline
\sens{1}
(\domainesémantique{Prépositions})
\classe{PREP (instrument)}
\begin{glose}
\pfra{avec (instrumental)}
\end{glose}
\newline
\begin{exemple}
\textbf{\pnua{thei xo hèle}}
\pfra{couper avec un couteau}
\end{exemple}
\newline
\sens{2}
(\domainesémantique{Prépositions})
\classe{PREP (bénéficiaire, destinataire)}
\begin{glose}
\pfra{pour}
\end{glose}
\newline
\begin{exemple}
\région{GO}
\textbf{\pnua{e zo xo nu !, axe icö, cö zoma po-za mwa ?}}
\pfra{pour moi, ça va, mais toi comment feras-tu donc ?}
\end{exemple}
\end{entrée}

\begin{entrée}{xo}{4}{ⓔxoⓗ4}
\région{GOs}
\variante{%
vwo
\région{GO(s)}}
(\domainesémantique{Suffixes transitifs})
\classe{SUFF}
\begin{glose}
\pfra{saturateur transitif}
\end{glose}
\newline
\begin{exemple}
\textbf{\pnua{Xa jò (vous2) kamweli me nee-xo}}
\pfra{Et comment avez-vous2 fait pour le faire ?}
\end{exemple}
\newline
\begin{exemple}
\textbf{\pnua{e ne-xo wãã-na dròrò}}
\pfra{il l'a fait comme cela / ainsi hier}
\end{exemple}
\newline
\begin{exemple}
\textbf{\pnua{ne-xo wãã !}}
\pfra{fais-le comme cela / ainsi}
\end{exemple}
\newline
\begin{exemple}
\textbf{\pnua{ne-xo egu-zo ! (ou) ne-vwo egu-zo !}}
\pfra{fais-le joli !}
\end{exemple}
\newline
\begin{exemple}
\région{GOs}
\textbf{\pnua{me waaju vwo me khila-xa whaya me tròòli xo mwani}}
\pfra{nous nous efforçons de chercher comment gagner de l'argent}
\end{exemple}
\end{entrée}

\newpage

\lettrine{y}\begin{entrée}{ya}{}{ⓔya}
\région{GOs}
\variante{%
yhal
\région{PA}}
(\domainesémantique{Discours, échanges verbaux})
\classe{nom}
\begin{glose}
\pfra{nom ; mot}
\end{glose}
\newline
\begin{exemple}
\textbf{\pnua{da ya jö tii ?}}
\pfra{quel est le mot que tu as écrit ?}
\end{exemple}
\end{entrée}

\begin{entrée}{ya-}{}{ⓔya-}
\région{GOs}
(\domainesémantique{Lumière et obscurité})
\classe{nom}
\begin{glose}
\pfra{lumière (en composition)}
\end{glose}
\newline
\begin{sous-entrée}{ya-ka}{ⓔya-ⓝya-ka}
\begin{glose}
\pfra{lampe électrique}
\end{glose}
\end{sous-entrée}
\newline
\begin{sous-entrée}{ya-traabwa}{ⓔya-ⓝya-traabwa}
\begin{glose}
\pfra{lampe à pétrole (lampe assise)}
\end{glose}
\end{sous-entrée}
\newline
\begin{sous-entrée}{ya-côe, ya-çôe}{ⓔya-ⓝya-côe, ya-çôe}
\begin{glose}
\pfra{lampe-fanal}
\end{glose}
\newline
\relationsémantique{Cf.}{\lien{ⓔyaai}{yaai}}
\glosecourte{feu, lumière}
\end{sous-entrée}
\end{entrée}

\begin{entrée}{yaa}{}{ⓔyaa}
\région{GOs}
\région{WEM WE PA}
\variante{%
yaal, yaale
, 
yaali
\région{BO}}
(\domainesémantique{Types de maison, architecture de la maison})
\classe{v.i.}
\begin{glose}
\pfra{couvrir (un toit, originellement avec de la paille)}
\end{glose}
\newline
\note{v.t.yaaze [GOs], yaale [BO]}{grammaire}{}
\end{entrée}

\begin{entrée}{yaa-bweevwu}{}{ⓔyaa-bweevwu}
\région{GOs}
(\domainesémantique{Types de maison, architecture de la maison})
\classe{v}
\begin{glose}
\pfra{couvrir de paille racines vers l'extérieur}
\end{glose}
\newline
\relationsémantique{Cf.}{\lien{}{yaa-de-du}}
\glosecourte{couvrir de paille racines vers l'extérieur}
\end{entrée}

\begin{entrée}{yaa-do mae}{}{ⓔyaa-do mae}
\région{GOs}
(\domainesémantique{Types de maison, architecture de la maison})
\classe{v}
\begin{glose}
\pfra{couvrir de paille racines vers l'intérieur}
\end{glose}
\newline
\relationsémantique{Cf.}{\lien{}{yaa-bwevwu, yaa-de-du}}
\glosecourte{couvrir de paille racines vers l'extérieur}
\end{entrée}

\begin{entrée}{yaa-gòò}{}{ⓔyaa-gòò}
\région{GOs}
\variante{%
yagòòn
\région{BO}}
(\domainesémantique{Types de maison, architecture de la maison})
\newline
\sens{1}
\classe{nom}
\begin{glose}
\pfra{tapisserie (en écorce de niaoulis ou de palmes de cocotier tressées)}
\end{glose}
\newline
\begin{exemple}
\textbf{\pnua{yagòò-ra ? - yagòò mwa}}
\pfra{c'est la couverture de quoi ? - la couverture de la maison}
\end{exemple}
\newline
\sens{2}
\classe{v}
\begin{glose}
\pfra{couvrir (le toit d'écorce de niaoulis, ou de palmes de cocotier tressées)}
\end{glose}
\newline
\relationsémantique{Cf.}{\lien{ⓔya-mwaⓝyaa de-du}{yaa de-du}}
\glosecourte{couvrir de paille racines vers l'extérieur}
\newline
\relationsémantique{Cf.}{\lien{ⓔyaa-do mae}{yaa-do mae}}
\glosecourte{couvrir de paille racines vers l'intérieur}
\end{entrée}

\begin{entrée}{yaa-he}{}{ⓔyaa-he}
\région{GOs PA}
\variante{%
yaai-he
\région{GO PA}}
\classe{nom}
(\domainesémantique{Feu : objets et actions liés au feu})
\begin{glose}
\pfra{feu allumé par friction}
\end{glose}
\end{entrée}

\begin{entrée}{yaai}{}{ⓔyaai}
\région{GOs BO}
\variante{%
yai
\région{PA}}
(\domainesémantique{Feu : objets et actions liés au feu})
\classe{nom}
\begin{glose}
\pfra{feu}
\end{glose}
\newline
\begin{sous-entrée}{pha yaai [GOs]}{ⓔyaaiⓝpha yaai [GOs]}
\begin{glose}
\pfra{allumer le feu}
\end{glose}
\end{sous-entrée}
\newline
\begin{sous-entrée}{phai yai [PA]}{ⓔyaaiⓝphai yai [PA]}
\begin{glose}
\pfra{allumer le feu}
\end{glose}
\end{sous-entrée}
\newline
\étymologie{
\langue{POc}
\étymon{*api}}
\end{entrée}

\begin{entrée}{yaali}{}{ⓔyaali}
\région{BO}
(\domainesémantique{Caractéristiques et propriétés des personnes})
\classe{v}
\begin{glose}
\pfra{nerveux[BM]}
\end{glose}
\end{entrée}

\begin{entrée}{yaawa}{}{ⓔyaawa}
\région{GOs PA}
(\domainesémantique{Sentiments})
\classe{v}
\begin{glose}
\pfra{malheureux ; triste ; nostalgique}
\end{glose}
\newline
\begin{exemple}
\textbf{\pnua{e za yaawa ui pexa pomõ-je ? (ou) e za yaawa pexa pomõ-je ?}}
\pfra{est-il nostalgique de son pays ?}
\end{exemple}
\newline
\begin{exemple}
\textbf{\pnua{Ôô ! e za yaawa !}}
\pfra{oui ! il en est nostalgique}
\end{exemple}
\newline
\begin{exemple}
\textbf{\pnua{e za yaawa ui/pexa lie waama ?}}
\pfra{est-il nostalgique de ses parents ?}
\end{exemple}
\newline
\begin{exemple}
\textbf{\pnua{e za yaawa ui/pexa li}}
\pfra{oui ! il a la nostalgie d'eux}
\end{exemple}
\end{entrée}

\begin{entrée}{yaawe}{}{ⓔyaawe}
\région{BO}
(\domainesémantique{Cours de la vie})
\classe{nom}
\begin{glose}
\pfra{nouveau-né [Corne]}
\end{glose}
\newline
\note{non vérifié}{général}{}
\end{entrée}

\begin{entrée}{yaa-wòlò}{}{ⓔyaa-wòlò}
\région{PA BO}
(\domainesémantique{Coutumes, dons coutumiers})
\classe{nom}
\begin{glose}
\pfra{champ d'igname du chef}
\end{glose}
\newline
\begin{exemple}
\région{PA}
\textbf{\pnua{whara ò khòòni ya-wòlò}}
\pfra{le temps de labourer le champ du chef}
\end{exemple}
\newline
\relationsémantique{Cf.}{\lien{ⓔyaa-wòzò}{yaa-wòzò}}
\glosecourte{champ du chef}
\newline
\relationsémantique{Cf.}{\lien{}{thèl [PA]}}
\glosecourte{débrousser}
\newline
\relationsémantique{Cf.}{\lien{}{kîni, khîni}}
\glosecourte{brûler}
\newline
\relationsémantique{Cf.}{\lien{ⓔkhòòni}{khòòni}}
\glosecourte{labourer}
\newline
\relationsémantique{Cf.}{\lien{}{thoè ; thöe}}
\glosecourte{planter}
\end{entrée}

\begin{entrée}{yaa-wòzò}{}{ⓔyaa-wòzò}
\région{GOs}
\variante{%
yaa-wòlò
\région{WEM WE}, 
ya-wòjò
\région{BO}}
(\domainesémantique{Coutumes, dons coutumiers
, Types de champs})
\classe{nom}
\begin{glose}
\pfra{champ d'igname sacré du chef (que l'on défriche)}
\end{glose}
\begin{glose}
\pfra{massif calendrier}
\end{glose}
\newline
\begin{exemple}
\textbf{\pnua{ne yaa-wòzò}}
\pfra{préparer le champ d'igname du chef}
\end{exemple}
\newline
\relationsémantique{Cf.}{\lien{ⓔwòzò}{wòzò}}
\glosecourte{épieu, barre-à-mine}
\newline
\relationsémantique{Cf.}{\lien{ⓔyaa}{yaa}}
\glosecourte{couvrir de paille}
\end{entrée}

\begin{entrée}{yaaza}{}{ⓔyaaza}
\formephonétique{jaːða}
\région{GOs}
\variante{%
yaala-n, yala-n
\formephonétique{jaːla}
\région{PA WEM BO}, 
yhaala-n, yara-n
\région{PA}}
(\domainesémantique{Société})
\classe{nom}
\begin{glose}
\pfra{nom}
\end{glose}
\newline
\begin{exemple}
\région{GOs}
\textbf{\pnua{yaaza-je}}
\pfra{son nom}
\end{exemple}
\newline
\begin{exemple}
\région{BO}
\textbf{\pnua{da yhala ? [PA], da yala-n ?}}
\pfra{comment cela se dit-il, comment cela s'appelle-t-il ?}
\end{exemple}
\newline
\begin{exemple}
\région{PA}
\textbf{\pnua{yhala-n}}
\pfra{son nom}
\end{exemple}
\newline
\begin{sous-entrée}{na yhala-n}{ⓔyaazaⓝna yhala-n}
\begin{glose}
\pfra{donner un nom}
\end{glose}
\end{sous-entrée}
\newline
\étymologie{
\langue{POc}
\étymon{*qacan, *asan}}
\end{entrée}

\begin{entrée}{yaaza da?}{}{ⓔyaaza da?}
\région{GOs}
\variante{%
yhaala da ?
\région{PA}, 
yaala da ?
\région{PA}}
(\domainesémantique{Interrogatifs})
\classe{LOCUT}
\begin{glose}
\pfra{quel est le sens de ? ; quel est le nom de ?}
\end{glose}
\newline
\begin{exemple}
\textbf{\pnua{yaaza da 'chö' ?}}
\pfra{que signifie 'chö' ?}
\end{exemple}
\end{entrée}

\begin{entrée}{yaaze}{}{ⓔyaaze}
\région{GOs}
\région{WEM WE PA}
\variante{%
yaal, yaale
, 
yaali
\région{BO}}
(\domainesémantique{Types de maison, architecture de la maison})
\classe{v.t.}
\begin{glose}
\pfra{couvrir (un toit, originellement avec de la paille)}
\end{glose}
\newline
\note{v.i.yaa [GOs], yaal [BO]}{grammaire}{}
\end{entrée}

\begin{entrée}{yabo}{}{ⓔyabo}
\région{GOs}
(\domainesémantique{Noms des plantes})
\classe{nom}
\begin{glose}
\pfra{plante à fruits rouges (attire les roussettes)}
\end{glose}
\end{entrée}

\begin{entrée}{yabwe}{}{ⓔyabwe}
\région{GOs WEM PA BO}
(\domainesémantique{Organisation sociale})
\classe{nom}
\begin{glose}
\pfra{sujet ; serviteur}
\end{glose}
\newline
\begin{exemple}
\textbf{\pnua{yabwe i Teã-ma}}
\pfra{le serviteur du grand chef}
\end{exemple}
\end{entrée}

\begin{entrée}{ya-cõê}{}{ⓔya-cõê}
\formephonétique{jaʒõê}
\région{GOs}
\variante{%
ya-çôê
\région{GO(s)}}
(\domainesémantique{Lumière et obscurité})
\classe{nom}
\begin{glose}
\pfra{lampe tempête (qui s'accroche)}
\end{glose}
\end{entrée}

\begin{entrée}{yada}{}{ⓔyada}
\région{GOs PA BO}
\classe{nom}
\newline
\sens{1}
(\domainesémantique{Richesses, monnaies traditionnelles})
\begin{glose}
\pfra{affaires ; objets ; biens ; choses}
\end{glose}
\newline
\sens{2}
(\domainesémantique{Coutumes, dons coutumiers})
\begin{glose}
\pfra{fêtes coutumières [BO]}
\end{glose}
\newline
\begin{exemple}
\région{GOs}
\textbf{\pnua{yadaa-nu}}
\pfra{mes biens}
\end{exemple}
\newline
\begin{exemple}
\région{PA BO}
\textbf{\pnua{yada-ny}}
\pfra{mes biens}
\end{exemple}
\newline
\begin{exemple}
\région{BO}
\textbf{\pnua{yada-n jèna}}
\pfra{c'est à lui}
\end{exemple}
\newline
\begin{exemple}
\région{PA BO}
\textbf{\pnua{yada ki-kui}}
\pfra{la fête des nouvelles ignames}
\end{exemple}
\end{entrée}

\begin{entrée}{yage}{}{ⓔyage}
\région{GOs BO}
(\domainesémantique{Mouvements ou actions faits avec le corps, les bras, les mains, les pieds})
\classe{v}
\begin{glose}
\pfra{ramasser (sable, feuilles, etc.) ; enlever}
\end{glose}
\newline
\begin{exemple}
\textbf{\pnua{nu yage mwa da na mwa}}
\pfra{j'enlève les cendres de la maison}
\end{exemple}
\end{entrée}

\begin{entrée}{yago}{}{ⓔyago}
\région{GOs}
(\domainesémantique{Crustacés, crabes})
\classe{nom}
\begin{glose}
\pfra{araignée de mer}
\end{glose}
\end{entrée}

\begin{entrée}{yai-khaa}{}{ⓔyai-khaa}
\région{GOs PA BO}
(\domainesémantique{Lumière et obscurité})
\classe{nom}
\begin{glose}
\pfra{lampe-torche (électrique)}
\end{glose}
\newline
\relationsémantique{Cf.}{\lien{ⓔkhaⓗ3ⓝyai-khaa}{yai-khaa}}
\glosecourte{feu-appuyer}
\end{entrée}

\begin{entrée}{yala}{}{ⓔyala}
\région{GOs PA}
(\domainesémantique{Actions liées aux éléments (liquide, fumée)})
\classe{v}
\begin{glose}
\pfra{rincer (vaisselle, un récipient)}
\end{glose}
\end{entrée}

\begin{entrée}{yali}{}{ⓔyali}
\région{PA BO [BM]}
(\domainesémantique{Actions liées aux éléments (liquide, fumée)})
\classe{v}
\begin{glose}
\pfra{écoper}
\end{glose}
\begin{glose}
\pfra{vider}
\end{glose}
\begin{glose}
\pfra{éclabousser (avec les mains)}
\end{glose}
\newline
\begin{exemple}
\région{BO}
\textbf{\pnua{i yali we}}
\pfra{il vide/écope l'eau}
\end{exemple}
\end{entrée}

\begin{entrée}{yaloxa}{}{ⓔyaloxa}
\région{GOs}
(\domainesémantique{Aliments, alimentation})
\classe{v}
\begin{glose}
\pfra{manger goulûment, trop vite}
\end{glose}
\end{entrée}

\begin{entrée}{yamevwu}{}{ⓔyamevwu}
\région{GOs}
\variante{%
yamepu
}
(\domainesémantique{Organisation sociale})
\classe{nom}
\begin{glose}
\pfra{clan}
\end{glose}
\end{entrée}

\begin{entrée}{ya-mwa}{}{ⓔya-mwa}
\région{GOs PA BO}
(\domainesémantique{Types de maison, architecture de la maison})
\classe{v}
\begin{glose}
\pfra{couvrir une maison}
\end{glose}
\newline
\begin{exemple}
\textbf{\pnua{ge li ya-mwa}}
\pfra{ils sont en train de couvrir la maison}
\end{exemple}
\newline
\begin{sous-entrée}{yaa de-du}{ⓔya-mwaⓝyaa de-du}
\begin{glose}
\pfra{couvrir de paille racines vers l'extérieur}
\end{glose}
\end{sous-entrée}
\newline
\begin{sous-entrée}{yaado mae}{ⓔya-mwaⓝyaado mae}
\begin{glose}
\pfra{couvrir de paille racines vers l'intérieur}
\end{glose}
\end{sous-entrée}
\newline
\begin{sous-entrée}{yagòò}{ⓔya-mwaⓝyagòò}
\begin{glose}
\pfra{couvrir le toit de peaux de niaoulis ou de palmes de cocotier tressées}
\end{glose}
\newline
\relationsémantique{Cf.}{\lien{ⓔphu}{phu}}
\glosecourte{première rangée de paille au bord du toit}
\newline
\relationsémantique{Cf.}{\lien{}{yaaze, yaale}}
\glosecourte{couvrir la maison}
\end{sous-entrée}
\end{entrée}

\begin{entrée}{ya-nu}{}{ⓔya-nu}
\région{BO PA}
(\domainesémantique{Lumière et obscurité})
\classe{nom}
\begin{glose}
\pfra{torche}
\end{glose}
\end{entrée}

\begin{entrée}{yaò}{1}{ⓔyaòⓗ1}
\région{GOs}
(\domainesémantique{Coraux})
\classe{nom}
\begin{glose}
\pfra{corail}
\end{glose}
\end{entrée}

\begin{entrée}{yaò}{2}{ⓔyaòⓗ2}
\région{GOs}
(\domainesémantique{Relations et interaction sociales})
\classe{v}
\begin{glose}
\pfra{quémander}
\end{glose}
\newline
\begin{exemple}
\textbf{\pnua{e aa-yaò}}
\pfra{quémandeur}
\end{exemple}
\end{entrée}

\begin{entrée}{yaoli}{}{ⓔyaoli}
\région{PA WEM}
\variante{%
yauli
\région{BO}, 
hiliçôô
\région{GO(s)}}
(\domainesémantique{Mouvements ou actions faits avec le corps, les bras, les mains, les pieds
, Jeux divers})
\classe{nom}
\begin{glose}
\pfra{balançoire ; balancer (se)}
\end{glose}
\newline
\begin{exemple}
\textbf{\pnua{yaoli-ny}}
\pfra{ma balançoire}
\end{exemple}
\newline
\begin{exemple}
\région{WE}
\textbf{\pnua{e pe-yaoli}}
\pfra{il se balance}
\end{exemple}
\end{entrée}

\begin{entrée}{ya-paò}{}{ⓔya-paò}
\région{GOs}
(\domainesémantique{Feu : objets et actions liés au feu})
\classe{nom}
\begin{glose}
\pfra{allumette (lit. feu-frapper)}
\end{glose}
\end{entrée}

\begin{entrée}{ya-phwaa}{}{ⓔya-phwaa}
\région{GOs}
(\domainesémantique{Lumière et obscurité})
\classe{nom}
\begin{glose}
\pfra{lampe coleman (fait une lumière vive)}
\end{glose}
\end{entrée}

\begin{entrée}{yaro}{}{ⓔyaro}
\région{BO PA}
\variante{%
zaro
\région{GO(s)}}
(\domainesémantique{Objets, outils})
\classe{nom}
\begin{glose}
\pfra{pelle à fouir les ignames (en bois ou fer)}
\end{glose}
\begin{glose}
\pfra{bêche}
\end{glose}
\end{entrée}

\begin{entrée}{yatre}{}{ⓔyatre}
\région{GOs}
\variante{%
yare, yaare
\région{GO BO}}
\classe{v}
\newline
\sens{1}
(\domainesémantique{Mouvements ou actions faits avec le corps, les bras, les mains, les pieds})
\begin{glose}
\pfra{extraire}
\end{glose}
\begin{glose}
\pfra{ôter}
\end{glose}
\newline
\sens{2}
(\domainesémantique{Verbes d'action (en général)})
\begin{glose}
\pfra{sortir (d'un sac, etc.)}
\end{glose}
\newline
\sens{3}
(\domainesémantique{Préparation des aliments; modes de préparation et de cuisson})
\begin{glose}
\pfra{sortir (d'une marmite) ; servir}
\end{glose}
\newline
\begin{exemple}
\région{BO}
\textbf{\pnua{i yare lavian na ni do}}
\pfra{il sort la viande de la marmite}
\end{exemple}
\newline
\relationsémantique{Cf.}{\lien{ⓔii}{ii}}
\glosecourte{sortir, servir les aliments (d'une marmite)}
\end{entrée}

\begin{entrée}{yaweeni}{}{ⓔyaweeni}
\région{GOs}
(\domainesémantique{Mouvements ou actions faits avec le corps, les bras, les mains, les pieds})
\classe{v}
\begin{glose}
\pfra{étaler (sable)}
\end{glose}
\end{entrée}

\begin{entrée}{yawi}{}{ⓔyawi}
\région{GOs PA BO}
\variante{%
yawe
\région{GO(s)}}
(\domainesémantique{Mouvements ou actions faits avec le corps, les bras, les mains, les pieds})
\classe{v}
\begin{glose}
\pfra{gratter (se) ; gratter ; griffer}
\end{glose}
\newline
\begin{exemple}
\région{GO}
\textbf{\pnua{e yaawi duu-nu}}
\pfra{elle me gratte le dos}
\end{exemple}
\newline
\begin{exemple}
\région{BO}
\textbf{\pnua{nu yau i nu}}
\pfra{je me gratte}
\end{exemple}
\end{entrée}

\begin{entrée}{yaxi}{}{ⓔyaxi}
\région{BO}
(\domainesémantique{Relations et interaction sociales})
\classe{v}
\begin{glose}
\pfra{saluer [BM]}
\end{glose}
\newline
\begin{exemple}
\région{BO}
\textbf{\pnua{ma pe-yaxi}}
\pfra{nous nous saluons}
\end{exemple}
\end{entrée}

\begin{entrée}{yaze}{}{ⓔyaze}
\région{GOs}
(\domainesémantique{Actions liées aux éléments (liquide, fumée)})
\classe{v}
\begin{glose}
\pfra{asperger ; projeter (eau, boue avec les mains) ; arroser (avec la main)}
\end{glose}
\newline
\relationsémantique{Cf.}{\lien{}{tãã, pa-tãã}}
\glosecourte{gicler, faire gicler}
\end{entrée}

\begin{entrée}{yazoo}{}{ⓔyazoo}
\formephonétique{yaðoː}
\région{GOs}
\variante{%
yaloo, yalo
\région{PA BO}}
(\domainesémantique{Mouvements ou actions faits avec le corps, les bras, les mains, les pieds})
\classe{v}
\begin{glose}
\pfra{frotter}
\end{glose}
\begin{glose}
\pfra{limer}
\end{glose}
\begin{glose}
\pfra{polir}
\end{glose}
\begin{glose}
\pfra{affûter ; affûté}
\end{glose}
\begin{glose}
\pfra{aiguiser}
\end{glose}
\begin{glose}
\pfra{tranchant}
\end{glose}
\newline
\begin{sous-entrée}{ba-yazo}{ⓔyazooⓝba-yazo}
\begin{glose}
\pfra{pierre à affûter, adze-blade}
\end{glose}
\newline
\begin{exemple}
\textbf{\pnua{e yazo hèlè}}
\pfra{elle affûte le couteau}
\end{exemple}
\newline
\begin{exemple}
\textbf{\pnua{nu yazoo-ni}}
\pfra{je l'ai affûté}
\end{exemple}
\newline
\note{yazo-ni (v.t.)}{grammaire}{affûter qqch}
\end{sous-entrée}
\newline
\relationsémantique{Cf.}{\lien{}{caa}}
\glosecourte{coupant, aigu}
\newline
\étymologie{
\langue{POc}
\étymon{*asa(q), *i-asa(q)}
\glosecourte{râpe(r)}
\auteur{Blust}}
\end{entrée}

\begin{entrée}{ye}{}{ⓔye}
\région{PA}
(\domainesémantique{Structure informationnelle})
\classe{THEM}
\begin{glose}
\pfra{thématisation}
\end{glose}
\end{entrée}

\begin{entrée}{-ye}{}{ⓔ-ye}
\région{PA BO}
(\domainesémantique{Pronoms})
\classe{PRO 3° pers. SG (OBJ ou POSS)}
\begin{glose}
\pfra{le ; la ; son ; sa ; ses}
\end{glose}
\end{entrée}

\begin{entrée}{yüe}{}{ⓔyüe}
\région{BO}
(\domainesémantique{Relations et interaction sociales})
\classe{v}
\begin{glose}
\pfra{bercer (enfant) [BM]}
\end{glose}
\newline
\begin{exemple}
\région{BO}
\textbf{\pnua{nu yü ẽnõ}}
\pfra{je berce l'enfant}
\end{exemple}
\end{entrée}

\begin{entrée}{yeege}{}{ⓔyeege}
\région{PA}
(\domainesémantique{Mouvements ou actions faits avec le corps, les bras, les mains, les pieds})
\classe{v}
\begin{glose}
\pfra{prendre (sable, terre)}
\end{glose}
\begin{glose}
\pfra{ramasser dans le creux de la main}
\end{glose}
\newline
\begin{exemple}
\région{PA}
\textbf{\pnua{i yeege on}}
\pfra{il prend du sable (avec les mains, pelle)}
\end{exemple}
\end{entrée}

\begin{entrée}{yeevwa bwee-ce}{}{ⓔyeevwa bwee-ce}
\région{GOs}
(\domainesémantique{Saisons})
\classe{nom}
\begin{glose}
\pfra{saison chaude (novembre à février)}
\end{glose}
\end{entrée}

\begin{entrée}{yeevwa kou}{}{ⓔyeevwa kou}
\région{GOs PA}
(\domainesémantique{Saisons})
\classe{nom}
\begin{glose}
\pfra{saison sèche, froide (mai à août)}
\end{glose}
\end{entrée}

\begin{entrée}{yeevwa zeenô}{}{ⓔyeevwa zeenô}
\région{GOs}
(\domainesémantique{Saisons})
\classe{nom}
\begin{glose}
\pfra{époque de maturité des ignames}
\end{glose}
\newline
\begin{exemple}
\région{GOs}
\textbf{\pnua{zeenô kui}}
\pfra{l'igname est arrivée à maturité}
\end{exemple}
\end{entrée}

\begin{entrée}{yevwa}{}{ⓔyevwa}
\région{GOs}
\variante{%
yepwan, yebwa
}
(\domainesémantique{Temps})
\classe{nom}
\begin{glose}
\pfra{moment où ; quand}
\end{glose}
\newline
\begin{exemple}
\région{GOs}
\textbf{\pnua{kixa nêêbu ni yevwa tòò}}
\pfra{il n'y a pas de moustique à la saison chaude}
\end{exemple}
\end{entrée}

\begin{entrée}{yo}{}{ⓔyo}
\région{PA BO}
\variante{%
yu
\région{GO}}
(\domainesémantique{Pronoms})
\classe{PRO 2° pers. SG (sujet, OBJ ou POSS)}
\begin{glose}
\pfra{toi, tu}
\end{glose}
\end{entrée}

\begin{entrée}{yoi}{}{ⓔyoi}
\région{GOs}
(\domainesémantique{Mouvements ou actions faits avec le corps, les bras, les mains, les pieds})
\classe{v}
\begin{glose}
\pfra{ramasser (des objets qui traînent)}
\end{glose}
\end{entrée}

\begin{entrée}{yölae}{}{ⓔyölae}
\région{PA}
(\domainesémantique{Mouvements ou actions faits avec le corps, les bras, les mains, les pieds})
\classe{v}
\begin{glose}
\pfra{enlever les branches latérales d'un tronc (avec un couteau, tamioc)}
\end{glose}
\end{entrée}

\begin{entrée}{yomaeo}{}{ⓔyomaeo}
\région{BO}
(\domainesémantique{Taros})
\classe{nom}
\begin{glose}
\pfra{taro (clone) de terrain sec (Dubois)}
\end{glose}
\end{entrée}

\begin{entrée}{yòò}{1}{ⓔyòòⓗ1}
\région{GOs PA BO}
\variante{%
yhòò
\région{GO(s)}, 
yòòk
\région{PA}}
(\domainesémantique{Arbre})
\classe{nom}
\begin{glose}
\pfra{bois de fer (de plaine ou de montagne)}
\end{glose}
\newline
\note{on rend l'esprit du défunt aux oncles maternels en attachant et en portant les monnaies traditionnelles sur une branche de bois de fer}{glose}{}
\nomscientifique{Casuarina equisetefolia L. (Casuarinacées)}
\newline
\begin{sous-entrée}{yòò-ma}{ⓔyòòⓗ1ⓝyòò-ma}
\begin{glose}
\pfra{bois de fer (petit et situé en bordure de cours d'eau)}
\end{glose}
\end{sous-entrée}
\newline
\étymologie{
\langue{POc}
\étymon{*(y)aRu}
\glosecourte{Casuarina equisetefolia}}
\end{entrée}

\begin{entrée}{yòò}{2}{ⓔyòòⓗ2}
\région{PA BO}
(\domainesémantique{Objets coutumiers})
\classe{nom}
\begin{glose}
\pfra{monnaie}
\end{glose}
\newline
\note{d'après Charles : monnaie du chef, fine et noire, de haute valeur, offerte attachée à un rameau de bois de fer, d'où le nom; Dubois : 1 yòò de 5O cm vaut 100 fr). Hiérarchiedes valeurs : yòò > weem > yhalo.}{glose}{}
\newline
\relationsémantique{Cf.}{\lien{}{pwãmwãnu; weem; dopweza}}
\end{entrée}

\begin{entrée}{yöö}{}{ⓔyöö}
\région{GOs PA}
\classe{v}
\newline
\sens{1}
(\domainesémantique{Verbes d'action faite par des animaux})
\begin{glose}
\pfra{ramper (serpent, lézard, lianes sur les arbres ou sur les tuteurs)}
\end{glose}
\newline
\sens{2}
(\domainesémantique{Mouvements ou actions faits avec le corps, les bras, les mains, les pieds})
\begin{glose}
\pfra{faufiler (se)}
\end{glose}
\newline
\étymologie{
\langue{POc}
\étymon{*kawaR}}
\end{entrée}

\begin{entrée}{yo-vhaa}{}{ⓔyo-vhaa}
\région{GOs}
\variante{%
zo-vhaa-le
\région{BO}}
(\domainesémantique{Discours, échanges verbaux})
\classe{v}
\begin{glose}
\pfra{murmurer ; parler doucement ; parler à voix basse}
\end{glose}
\begin{glose}
\pfra{rapporter}
\end{glose}
\end{entrée}

\begin{entrée}{yue}{}{ⓔyue}
\région{GOs PA BO}
(\domainesémantique{Parenté})
\classe{v}
\begin{glose}
\pfra{adopter ; élever (enfant)}
\end{glose}
\begin{glose}
\pfra{garder (enfant)}
\end{glose}
\end{entrée}

\begin{entrée}{yuu}{}{ⓔyuu}
\région{GOs}
\variante{%
yu, yuu
\région{BO PA}}
\classe{v}
\newline
\sens{1}
(\domainesémantique{Habitat})
\begin{glose}
\pfra{demeurer}
\end{glose}
\begin{glose}
\pfra{rester}
\end{glose}
\begin{glose}
\pfra{résider}
\end{glose}
\newline
\begin{exemple}
\région{BO}
\textbf{\pnua{i yu kolo-n}}
\pfra{il demeure chez lui}
\end{exemple}
\newline
\begin{sous-entrée}{a-yuu [GOs, BO, PA]}{ⓔyuuⓢ1ⓝa-yuu [GOs, BO, PA]}
\begin{glose}
\pfra{habitant}
\end{glose}
\end{sous-entrée}
\newline
\begin{sous-entrée}{bala-yu}{ⓔyuuⓢ1ⓝbala-yu}
\begin{glose}
\pfra{compagnon de résidence, serviteur}
\end{glose}
\end{sous-entrée}
\newline
\sens{2}
(\domainesémantique{Verbes locatifs})
\begin{glose}
\pfra{trouver (se)}
\end{glose}
\begin{glose}
\pfra{être (loc.)}
\end{glose}
\end{entrée}

\begin{entrée}{yuu bulu}{}{ⓔyuu bulu}
\région{PA}
(\domainesémantique{Société})
\classe{v}
\begin{glose}
\pfra{mariés (être) (lit. rester ensemble)}
\end{glose}
\newline
\begin{exemple}
\textbf{\pnua{li yuu bulu}}
\pfra{ils sont en couple}
\end{exemple}
\end{entrée}

\newpage

\lettrine{yh}\begin{entrée}{yhaamwa}{}{ⓔyhaamwa}
\région{GOs BO}
(\domainesémantique{Fonctions intellectuelles})
\classe{v.IMPERS}
\begin{glose}
\pfra{on ne sait pas}
\end{glose}
\newline
\begin{exemple}
\région{GOs}
\textbf{\pnua{yhaamwa ! kavwö nu hine me e trõõne}}
\pfra{je n'en sais rien ! je ne sais pas s'il a entendu}
\end{exemple}
\newline
\begin{exemple}
\région{GOs}
\textbf{\pnua{yhaamwa me da la lò trõõne}}
\pfra{on ne sait pas ce qu'ils ont entendu}
\end{exemple}
\newline
\begin{exemple}
\région{GOs}
\textbf{\pnua{yhaamwa me ezoma lò uja}}
\pfra{on ne sait pas quand ils arriveront}
\end{exemple}
\newline
\begin{exemple}
\région{GOs}
\textbf{\pnua{nu a thraabu, nowu iju ca yhaamwa iju}}
\pfra{je vais à la pêche, quant à toi, je ne sais pas (ce qu'il en est) pour toi}
\end{exemple}
\end{entrée}

\begin{entrée}{yhal}{}{ⓔyhal}
\région{PA}
(\domainesémantique{Discours, échanges verbaux})
\classe{v}
\begin{glose}
\pfra{nommer}
\end{glose}
\newline
\begin{sous-entrée}{yhal paxa [BO]}{ⓔyhalⓝyhal paxa [BO]}
\begin{glose}
\pfra{surnom}
\end{glose}
\end{sous-entrée}
\newline
\begin{sous-entrée}{na yhala-n [PA]}{ⓔyhalⓝna yhala-n [PA]}
\begin{glose}
\pfra{donner un nom}
\end{glose}
\end{sous-entrée}
\end{entrée}

\begin{entrée}{yhala}{}{ⓔyhala}
\région{BO}
(\domainesémantique{Aliments, alimentation})
\classe{v}
\begin{glose}
\pfra{chercher de la nourriture ; aller à la pêche ; aller à la chasse [BM]}
\end{glose}
\end{entrée}

\begin{entrée}{yhalo}{}{ⓔyhalo}
\région{PA}
(\domainesémantique{Objets coutumiers})
\classe{nom}
\begin{glose}
\pfra{monnaie kanak}
\end{glose}
\newline
\note{(de valeur moindre que yòò et weem ; mais de valeur équivalente à pwãmwãnu). Hiérarchie des valeurs : yòò > weem > yhalo.}{glose}{}
\end{entrée}

\begin{entrée}{yhò}{}{ⓔyhò}
\région{GOs PA BO}
(\domainesémantique{Parenté})
\classe{n (terme d'appellation ou référence)}
\begin{glose}
\pfra{frère/soeur aîné(e)}
\end{glose}
\begin{glose}
\pfra{cousin(e) parallèle et aîné(e) ; cousin croisé de sexe opposé et aîné (enfants de la soeur du père)}
\end{glose}
\end{entrée}

\newpage

\lettrine{z}\begin{entrée}{za}{1}{ⓔzaⓗ1}
\formephonétique{ða}
\région{GOs}
\variante{%
zha
\formephonétique{θa}
\région{GA}, 
zam
\région{PA}, 
yam
\région{BO}}
(\domainesémantique{Ustensiles})
\classe{nom}
\begin{glose}
\pfra{assiette ; plat}
\end{glose}
\begin{glose}
\pfra{corbeille}
\end{glose}
\newline
\begin{sous-entrée}{bwa-xaça za}{ⓔzaⓗ1ⓝbwa-xaça za}
\région{GOs}
\begin{glose}
\pfra{le dos de l'assiette}
\end{glose}
\end{sous-entrée}
\newline
\begin{sous-entrée}{nò za}{ⓔzaⓗ1ⓝnò za}
\région{GOs}
\begin{glose}
\pfra{l'intérieur, le creux de l'assiette}
\end{glose}
\newline
\begin{exemple}
\région{GO}
\textbf{\pnua{zabo-jö}}
\pfra{ton assiette}
\end{exemple}
\newline
\begin{exemple}
\région{GOs}
\textbf{\pnua{zabo-nu}}
\pfra{mon assiette}
\end{exemple}
\newline
\begin{exemple}
\région{PA}
\textbf{\pnua{zabo-ny}}
\pfra{mon assiette}
\end{exemple}
\newline
\begin{exemple}
\région{BO}
\textbf{\pnua{yabo-m}}
\pfra{ton assiette}
\end{exemple}
\end{sous-entrée}
\end{entrée}

\begin{entrée}{za}{2}{ⓔzaⓗ2}
\formephonétique{ða}
\région{GOs PA}
\variante{%
zha
\région{GA}, 
ya
\région{BO}}
(\domainesémantique{Goût des aliments})
\classe{v.stat.}
\begin{glose}
\pfra{salé ; trop salé}
\end{glose}
\newline
\begin{sous-entrée}{we-za}{ⓔzaⓗ2ⓝwe-za}
\begin{glose}
\pfra{eau salée}
\end{glose}
\end{sous-entrée}
\end{entrée}

\begin{entrée}{za}{3}{ⓔzaⓗ3}
\formephonétique{ða}
\région{GOs}
\variante{%
ra
\région{WE WEM}}
(\domainesémantique{Marques assertives})
\classe{FOC ; RESTR (antéposé au GN) (za ... nye ...)}
\begin{glose}
\pfra{c'est vraiment ... que}
\end{glose}
\newline
\begin{exemple}
\textbf{\pnua{Ô ! za ije nye penõ dròrò}}
\pfra{Oui ! c'est bien lui qui a volé hier}
\end{exemple}
\newline
\begin{exemple}
\textbf{\pnua{Ô ! za ilò nye lò penõ dròrò}}
\pfra{Oui ! c'est bien euxqui ont volé hier}
\end{exemple}
\newline
\begin{exemple}
\textbf{\pnua{Hai ! za inu ma ãbaa-nu nye bi a}}
\pfra{Non! c'est bien ma soeur et moi qui sommes parties}
\end{exemple}
\newline
\begin{exemple}
\textbf{\pnua{Hai ! za inu ma ãbaa-nu hãda nye bi a}}
\pfra{Non! c'est seulement ma soeur et moi qui sommes parties}
\end{exemple}
\newline
\begin{exemple}
\textbf{\pnua{Hai ! za caaja hãda ma ãbaa-nu nye li a}}
\pfra{Non! c'est seulement mon père et ma soeur qui sont partis}
\end{exemple}
\newline
\begin{exemple}
\textbf{\pnua{e za gi dròrò ?}}
\pfra{il a vraiment pleuré hier ?}
\end{exemple}
\newline
\begin{exemple}
\textbf{\pnua{za ije nye gi dròrò}}
\pfra{c'est vraiment lui qui a pleuré hier}
\end{exemple}
\end{entrée}

\begin{entrée}{za}{4}{ⓔzaⓗ4}
\formephonétique{ða}
\région{GOs}
\variante{%
ra
\région{PABO}}
(\domainesémantique{Marques assertives})
\classe{ADV INTENS ; assertif (devant le prédicat)}
\begin{glose}
\pfra{vraiment ; tout à fait}
\end{glose}
\newline
\begin{exemple}
\région{GO}
\textbf{\pnua{nu za u trêê}}
\pfra{j'ai vraiment couru}
\end{exemple}
\newline
\begin{exemple}
\région{GO}
\textbf{\pnua{nu za nee xo nu}}
\pfra{c'est moi qui l'ai fait}
\end{exemple}
\newline
\begin{exemple}
\région{GO}
\textbf{\pnua{la za yaawa dròrò}}
\pfra{ils étaient vraiment tristes hier}
\end{exemple}
\newline
\begin{exemple}
\région{GO}
\textbf{\pnua{e za mõõlo gò ? - Ô, e za mõõlo gò}}
\pfra{elle vit vraiment encore ? - oui, elle vit toujours}
\end{exemple}
\newline
\begin{exemple}
\région{GO}
\textbf{\pnua{e zaa ne xo je}}
\pfra{il l'a vraiment fait}
\end{exemple}
\newline
\begin{exemple}
\région{GO}
\textbf{\pnua{e zaa uvi tiiwo ponga-je}}
\pfra{il s'est acheté un livre pour lui-même (=ponga-je)}
\end{exemple}
\newline
\begin{exemple}
\région{PA}
\textbf{\pnua{i khôbwe vwo raa ije}}
\pfra{il parle en son nom propre (pour lui-même)}
\end{exemple}
\end{entrée}

\begin{entrée}{-za}{}{ⓔ-za}
\région{PA BO}
\variante{%
-ya
\région{BO}}
(\domainesémantique{Pronoms})
\classe{PRO 1° pers. incl. (OBJ ou POSS)}
\begin{glose}
\pfra{nous, notre}
\end{glose}
\end{entrée}

\begin{entrée}{za ?}{}{ⓔza ?}
\région{GOs}
\variante{%
ra?
\région{PA BO}}
(\domainesémantique{Interrogatifs})
\classe{INT}
\begin{glose}
\pfra{quoi ? ; qu'est ce que ?}
\end{glose}
\begin{glose}
\pfra{quelle sorte (de) ?}
\end{glose}
\newline
\begin{exemple}
\région{GOs}
\textbf{\pnua{co po za ?}}
\pfra{que fais-tu ?}
\end{exemple}
\newline
\begin{exemple}
\région{GOs}
\textbf{\pnua{hèlè za ? - hèlè ba-cooxe layô}}
\pfra{quelle sorte de couteau? - un couteau pour couper la viande}
\end{exemple}
\newline
\begin{exemple}
\région{GOs}
\textbf{\pnua{pò-ce za ?}}
\pfra{quelle sorte de fruit est-ce ?}
\end{exemple}
\newline
\begin{exemple}
\région{GOs}
\textbf{\pnua{mwa za ? - mwa dili}}
\pfra{quelle sorte de maison ? - une maison en terre}
\end{exemple}
\newline
\begin{exemple}
\région{GOs}
\textbf{\pnua{mõ-da ? - mõ-pe-rooli - mõ-thia}}
\pfra{une maison pour quoi? qui sert à quoi ? - une maison de réunion, une maison de danse}
\end{exemple}
\newline
\begin{exemple}
\région{GOs}
\textbf{\pnua{ce za nye ?}}
\pfra{qu'est-ce que cet arbre ? comment s'appelle-t-il ?}
\end{exemple}
\newline
\begin{exemple}
\textbf{\pnua{no za ?}}
\pfra{quelle sorte de poisson ?}
\end{exemple}
\newline
\begin{exemple}
\textbf{\pnua{mwa za ?}}
\pfra{quelle sorte de maison ?}
\end{exemple}
\newline
\begin{exemple}
\textbf{\pnua{pwaji za ?}}
\pfra{quelle sorte de banane ?}
\end{exemple}
\end{entrée}

\begin{entrée}{zaa}{1}{ⓔzaaⓗ1}
\région{GOs PA}
\variante{%
zhaa
\région{GA}, 
yaa
\région{BO}}
(\domainesémantique{Oiseaux})
\classe{nom}
\begin{glose}
\pfra{poule sultane}
\end{glose}
\newline
\note{(symbole de l'abondance des cultures ; son contraire est le rat qui saccage les cultures)}{glose}{}
\nomscientifique{Porphyrio porphyrio caledonicus}
\end{entrée}

\begin{entrée}{zaa}{2}{ⓔzaaⓗ2}
\région{GOs PA}
\variante{%
zhaa
\région{GA}}
(\domainesémantique{Santé, maladie})
\classe{v}
\begin{glose}
\pfra{engourdi ; avoir des fourmis (dans les membres)}
\end{glose}
\newline
\begin{exemple}
\région{GOs}
\textbf{\pnua{e zaa kòò-nu}}
\pfra{j'ai la jambe engourdie}
\end{exemple}
\end{entrée}

\begin{entrée}{zaadu}{}{ⓔzaadu}
\région{GOs BO}
(\domainesémantique{Aliments, alimentation})
\classe{v}
\begin{glose}
\pfra{maigre ; non-gras (viande, poisson) [Corne]}
\end{glose}
\end{entrée}

\begin{entrée}{zaae}{}{ⓔzaae}
\région{GOs PA}
\variante{%
zhae
\région{GA}}
(\domainesémantique{Actions liées aux plantes})
\classe{v}
\begin{glose}
\pfra{faire mûrir (fruits)}
\end{glose}
\begin{glose}
\pfra{conserver pour faire mûrir}
\end{glose}
\end{entrée}

\begin{entrée}{zaalae}{}{ⓔzaalae}
\région{GOs}
\variante{%
zhaalae
\région{GA}}
(\domainesémantique{Relations et interaction sociales})
\classe{v}
\begin{glose}
\pfra{donner de la nourriture (à des animés)}
\end{glose}
\begin{glose}
\pfra{subvenir aux besoins}
\end{glose}
\begin{glose}
\pfra{élever}
\end{glose}
\newline
\relationsémantique{Cf.}{\lien{ⓔa-zaala}{a-zaala}}
\glosecourte{chercher de la nourriture}
\end{entrée}

\begin{entrée}{zaalo}{}{ⓔzaalo}
\formephonétique{ðaːlo}
\région{GOs PA}
\variante{%
zhaalo
\formephonétique{θaːlo}
\région{GA}, 
yaalo
\région{BO}}
(\domainesémantique{Arbre})
\classe{nom}
\begin{glose}
\pfra{"gommier"}
\end{glose}
\nomscientifique{Cordia dichotoma (Borraginacées)}
\end{entrée}

\begin{entrée}{zaaloè}{}{ⓔzaaloè}
\région{GOs PA}
(\domainesémantique{Verbes d'action (en général)})
\classe{v}
\begin{glose}
\pfra{coller}
\end{glose}
\newline
\begin{sous-entrée}{pe-zaaloè}{ⓔzaaloèⓝpe-zaaloè}
\begin{glose}
\pfra{coller ensemble}
\end{glose}
\end{sous-entrée}
\end{entrée}

\begin{entrée}{zaa pweza}{}{ⓔzaa pweza}
\région{PA}
(\domainesémantique{Bananiers et bananes})
\classe{v}
\begin{glose}
\pfra{prélever les rejets d'un bananier 'pweza' pour les replanter}
\end{glose}
\newline
\note{banane noble de la chefferie, entrant dans les échanges coutumiers}{glose}{}
\end{entrée}

\begin{entrée}{zaa phwa}{}{ⓔzaa phwa}
\région{PA WEM}
(\domainesémantique{Cultures, techniques, boutures})
\classe{v}
\begin{glose}
\pfra{préparer les champs et le trou pour planter les ignames}
\end{glose}
\newline
\note{retourner la terre avec un bâton, la rendre meuble, préparer la butte et enfin faire le trou en soulevant la terre)}{glose}{}
\newline
\relationsémantique{Cf.}{\lien{ⓔthu phwa}{thu phwa}}
\glosecourte{faire des trous}
\end{entrée}

\begin{entrée}{zaawane}{1}{ⓔzaawaneⓗ1}
\formephonétique{ðaːwaɳe}
\région{GOs}
(\domainesémantique{Poissons})
\classe{nom}
\begin{glose}
\pfra{"aiguillette" (de grande taille)}
\end{glose}
\nomscientifique{Tylosorus crocodilus crocodilus (Belonidés)}
\end{entrée}

\begin{entrée}{zaawane}{2}{ⓔzaawaneⓗ2}
\formephonétique{ðaːwaɳe}
\région{GOs}
(\domainesémantique{Processus liés aux plantes})
\classe{v}
\begin{glose}
\pfra{presque mûr}
\end{glose}
\end{entrée}

\begin{entrée}{zaba}{}{ⓔzaba}
\région{GOs PA}
\variante{%
zhaba
\région{GA}, 
yhaba
\région{BO}}
(\domainesémantique{Discours, échanges verbaux})
\classe{v}
\begin{glose}
\pfra{encourager ; soutenir}
\end{glose}
\begin{glose}
\pfra{répondre ; donner la réplique}
\end{glose}
\newline
\begin{exemple}
\textbf{\pnua{e zaba cai nu}}
\pfra{il m'a répondu}
\end{exemple}
\end{entrée}

\begin{entrée}{zabajo}{}{ⓔzabajo}
\région{GOs}
(\domainesémantique{Verbes de déplacement et moyens de déplacement})
\classe{v}
\begin{glose}
\pfra{passer à toute allure}
\end{glose}
\begin{glose}
\pfra{courir vite}
\end{glose}
\end{entrée}

\begin{entrée}{zabo}{}{ⓔzabo}
\région{BO}
(\domainesémantique{Feu : objets et actions liés au feu})
\classe{nom}
\begin{glose}
\pfra{suie noire de la fumée dans les maisons [Corne]}
\end{glose}
\newline
\note{non vérifié}{général}{}
\end{entrée}

\begin{entrée}{zabò}{}{ⓔzabò}
\région{GOs BO PAWEM}
\variante{%
zhabò
\région{GA}}
\classe{nom}
\newline
\sens{1}
(\domainesémantique{Corps humain})
\begin{glose}
\pfra{côtes}
\end{glose}
\newline
\begin{exemple}
\région{PA}
\textbf{\pnua{zabò-n}}
\pfra{sa côte}
\end{exemple}
\newline
\sens{2}
(\domainesémantique{Types de maison, architecture de la maison})
\begin{glose}
\pfra{gaulettes qui retiennent la couverture du toit (faite d'écorce de niaouli et de paille)}
\end{glose}
\newline
\begin{sous-entrée}{zabò mwa}{ⓔzabòⓢ2ⓝzabò mwa}
\begin{glose}
\pfra{gaulettes}
\end{glose}
\newline
\relationsémantique{Cf.}{\lien{}{orèi}}
\glosecourte{gaulettes circulaires}
\end{sous-entrée}
\end{entrée}

\begin{entrée}{zabò mwa}{}{ⓔzabò mwa}
\région{GOs}
(\domainesémantique{Types de maison, architecture de la maison})
\classe{nom}
\begin{glose}
\pfra{gaulettes (qui retiennent la couverture du toit faite d'écorce de niaouli et de paille)}
\end{glose}
\end{entrée}

\begin{entrée}{zaboriã}{}{ⓔzaboriã}
\région{BO}
(\domainesémantique{Organisation sociale})
\classe{nom}
\begin{glose}
\pfra{porte-parole du chef [Corne]}
\end{glose}
\newline
\note{non vérifié}{général}{}
\end{entrée}

\begin{entrée}{zaçixõõni}{}{ⓔzaçixõõni}
\région{GOs}
(\domainesémantique{Relations et interaction sociales})
\classe{v}
\begin{glose}
\pfra{rapporter ; dénoncer}
\end{glose}
\end{entrée}

\begin{entrée}{zagaò}{}{ⓔzagaò}
\région{GOs}
\variante{%
zagaòl
\région{PA}}
(\domainesémantique{Cultures, techniques, boutures})
\classe{v}
\begin{glose}
\pfra{récolter les ignames ; époque où l'on récolte les ignames}
\end{glose}
\begin{glose}
\pfra{glâner (des ignames, bananes, taros dans des champs laissés en jachère ou à l'abandon) (repousse spontanée des plants)}
\end{glose}
\newline
\begin{exemple}
\région{GOs}
\textbf{\pnua{la zagaò-ni kui}}
\pfra{ils récoltent les ignames}
\end{exemple}
\newline
\relationsémantique{Cf.}{\lien{ⓔmaxuã}{maxuã}}
\glosecourte{glâner de la canne à sucre}
\end{entrée}

\begin{entrée}{zagawe}{}{ⓔzagawe}
\région{GOs}
\variante{%
zhagawe
\région{GA}}
(\domainesémantique{Cultures, techniques, boutures})
\classe{v}
\begin{glose}
\pfra{réserver des tubercules (pour les replanter)}
\end{glose}
\newline
\begin{exemple}
\textbf{\pnua{la zhagawe kui}}
\pfra{ils réservent des ignames}
\end{exemple}
\newline
\begin{exemple}
\textbf{\pnua{la p(h)e-zhagawe-ni êê-la kui}}
\pfra{ils réservent leurs plants d'igname}
\end{exemple}
\end{entrée}

\begin{entrée}{zage}{}{ⓔzage}
\région{GOs PA}
\variante{%
zhage, zaage
\région{GO(s)}, 
yhage
\région{BO}}
(\domainesémantique{Relations et interaction sociales})
\classe{v}
\begin{glose}
\pfra{aider}
\end{glose}
\newline
\begin{exemple}
\région{PA}
\textbf{\pnua{la pe-zage u la khabe nye mwa}}
\pfra{ils s'entraident pour construire cette maison}
\end{exemple}
\newline
\begin{sous-entrée}{pe-yhage (BO)}{ⓔzageⓝpe-yhage (BO)}
\begin{glose}
\pfra{s'entraider}
\end{glose}
\end{sous-entrée}
\end{entrée}

\begin{entrée}{zageeni}{}{ⓔzageeni}
\région{GOs}
\variante{%
zhageeni
\région{GA}}
(\domainesémantique{Relations et interaction sociales})
\classe{v}
\begin{glose}
\pfra{ajouter ; abonder dans le sens de qqn}
\end{glose}
\end{entrée}

\begin{entrée}{zagi}{}{ⓔzagi}
\formephonétique{ðaŋgi}
\région{GOs PA}
\variante{%
zhagia
\formephonétique{θaːŋgia}
\région{GO(s)}, 
yagi
\région{PA BO}}
(\domainesémantique{Corps humain})
\classe{nom}
\begin{glose}
\pfra{cerveau ; cervelle}
\end{glose}
\end{entrée}

\begin{entrée}{zagia ciia}{}{ⓔzagia ciia}
\région{GOs}
(\domainesémantique{Céphalopodes})
\classe{nom}
\begin{glose}
\pfra{encre de poulpe}
\end{glose}
\end{entrée}

\begin{entrée}{zagu}{}{ⓔzagu}
\région{BO}
(\domainesémantique{Types de champs})
\classe{nom}
\begin{glose}
\pfra{champ de taro débroussé mais non labouré [Dubois]}
\end{glose}
\newline
\note{non vérifié}{général}{}
\end{entrée}

\begin{entrée}{zai}{}{ⓔzai}
\région{GOs BO}
\variante{%
zhai
\région{GA}}
(\domainesémantique{Musique, instruments de musique})
\classe{v}
\begin{glose}
\pfra{composer un chant}
\end{glose}
\newline
\begin{exemple}
\textbf{\pnua{wa xa zai xo ti ?}}
\pfra{c'est un chant composé par qui ?}
\end{exemple}
\newline
\begin{sous-entrée}{za wal [BO]}{ⓔzaiⓝza wal [BO]}
\begin{glose}
\pfra{trouver le thème d'un chant}
\end{glose}
\end{sous-entrée}
\end{entrée}

\begin{entrée}{zakèbi}{}{ⓔzakèbi}
\région{GOs}
\variante{%
zaxèbi
\région{GO(s)}, 
zhaxèbi
\région{GA}}
(\domainesémantique{Caractéristiques et propriétés des personnes})
\classe{nom}
\begin{glose}
\pfra{habile ; qui a du savoir faire}
\end{glose}
\begin{glose}
\pfra{habitué à faire qqch}
\end{glose}
\newline
\begin{exemple}
\textbf{\pnua{e kô-zaxèbi}}
\pfra{il a du savoir-faire}
\end{exemple}
\end{entrée}

\begin{entrée}{zala}{1}{ⓔzalaⓗ1}
\région{GOs PA}
\variante{%
yala
\région{BO}}
(\domainesémantique{Aliments, alimentation})
\classe{v}
\begin{glose}
\pfra{chercher de la nourriture}
\end{glose}
\begin{glose}
\pfra{glâner}
\end{glose}
\newline
\begin{sous-entrée}{a-zala}{ⓔzalaⓗ1ⓝa-zala}
\région{GOs}
\begin{glose}
\pfra{chercher de la nourriture}
\end{glose}
\end{sous-entrée}
\end{entrée}

\begin{entrée}{zala}{2}{ⓔzalaⓗ2}
\formephonétique{ðala}
\région{GOs PA}
\variante{%
zhala
\formephonétique{θala}
\région{GA}}
(\domainesémantique{Fonctions intellectuelles})
\classe{v ; n}
\begin{glose}
\pfra{demander à qqn ; interroger ; question(ner)}
\end{glose}
\newline
\begin{exemple}
\région{GOs}
\textbf{\pnua{e zala nu khõbeçö a mõnõ}}
\pfra{il me demande si tu pars demain}
\end{exemple}
\newline
\begin{exemple}
\région{GOs}
\textbf{\pnua{e zala khõbe la minõ dröö}}
\pfra{il demande si les marmites sont prêtes}
\end{exemple}
\newline
\relationsémantique{Cf.}{\lien{}{phaja [WEM]}}
\glosecourte{demander à qqn}
\newline
\relationsémantique{Cf.}{\lien{}{zaba [GOs]}}
\glosecourte{répondre}
\end{entrée}

\begin{entrée}{zali}{}{ⓔzali}
\formephonétique{ðali}
\région{GOs}
\variante{%
zhali
\formephonétique{θali}
\région{GA}}
\classe{v}
\newline
\sens{1}
(\domainesémantique{Mouvements ou actions faits avec le corps, les bras, les mains, les pieds})
\begin{glose}
\pfra{enlever (natte)}
\end{glose}
\begin{glose}
\pfra{soulever (des pierres, herbes)}
\end{glose}
\newline
\begin{exemple}
\région{GOs}
\textbf{\pnua{e zali thrô}}
\pfra{elle enlève les nattes}
\end{exemple}
\newline
\begin{exemple}
\région{GOs}
\textbf{\pnua{e zali dròò-chaamwa}}
\pfra{elle enlève les palmes de bananier (couvrant le four enterré)}
\end{exemple}
\newline
\begin{exemple}
\région{PA}
\textbf{\pnua{e zali paa}}
\pfra{il soulève les pierres}
\end{exemple}
\newline
\begin{exemple}
\région{PA}
\textbf{\pnua{e zali paxa}}
\pfra{il soulève les herbes}
\end{exemple}
\newline
\sens{2}
(\domainesémantique{Pêche})
\begin{glose}
\pfra{ramasser (filet)}
\end{glose}
\newline
\begin{exemple}
\région{PA}
\textbf{\pnua{e zali pwiò}}
\pfra{il ramasse le filet}
\end{exemple}
\newline
\sens{3}
(\domainesémantique{Cultures, techniques, boutures})
\begin{glose}
\pfra{retourner (la terre)}
\end{glose}
\newline
\relationsémantique{Cf.}{\lien{}{zaa phwa [PA]}}
\glosecourte{préparer les champs d'igname}
\newline
\relationsémantique{Cf.}{\lien{ⓔthu phwa}{thu phwa}}
\glosecourte{faire des trous (en retournant la terre)}
\end{entrée}

\begin{entrée}{zamadra}{}{ⓔzamadra}
\région{GOs}
(\domainesémantique{Armes})
\classe{nom}
\begin{glose}
\pfra{sparterie de sagaie}
\end{glose}
\end{entrée}

\begin{entrée}{za mee}{}{ⓔza mee}
\région{GOs}
\variante{%
tha-mee
\région{GO(s)}}
(\domainesémantique{Verbes d'action (en général)})
\classe{v}
\begin{glose}
\pfra{tailler en pointe}
\end{glose}
\newline
\begin{exemple}
\textbf{\pnua{e za mee ce}}
\pfra{il épointe un bout de bois}
\end{exemple}
\end{entrée}

\begin{entrée}{zamee}{}{ⓔzamee}
\région{GOs}
(\domainesémantique{Mouvements ou actions faits avec le corps, les bras, les mains, les pieds})
\classe{v}
\begin{glose}
\pfra{épointer (un bout de bois)}
\end{glose}
\end{entrée}

\begin{entrée}{zano}{}{ⓔzano}
\région{PA}
(\domainesémantique{Coutumes, dons coutumiers})
\classe{nom}
\begin{glose}
\pfra{herbe (contexte cérémoniel uniquement)}
\end{glose}
\newline
\begin{exemple}
\textbf{\pnua{ni zano}}
\pfra{dans l'herbe}
\end{exemple}
\end{entrée}

\begin{entrée}{zanyi}{}{ⓔzanyi}
\formephonétique{ðaɲi,θaɲi}
\région{GOs}
\variante{%
zhanyi
\région{GA}}
\newline
\sens{1}
(\domainesémantique{Aliments, alimentation})
\classe{nom}
\begin{glose}
\pfra{sel}
\end{glose}
\newline
\sens{2}
(\domainesémantique{Préparation des aliments; modes de préparation et de cuisson})
\classe{v}
\begin{glose}
\pfra{saler la nourriture ; mettre du sel}
\end{glose}
\newline
\begin{exemple}
\textbf{\pnua{zanyii ni dröö !}}
\pfra{met du sel dans la marmite !}
\end{exemple}
\end{entrée}

\begin{entrée}{zanyii}{}{ⓔzanyii}
\formephonétique{ðaɲiː}
\région{GOs}
\variante{%
zhanyii
\région{GA}, 
hing
\région{PA}}
(\domainesémantique{Aliments, alimentation})
\classe{v}
\begin{glose}
\pfra{dégoûté ; faire le difficile}
\end{glose}
\newline
\begin{exemple}
\textbf{\pnua{egu xa nu zanyii}}
\pfra{qqn qui me dégoûte}
\end{exemple}
\newline
\begin{exemple}
\textbf{\pnua{e a-zanyii}}
\pfra{c'est un maniaque (dégoûté par tout ce qui est sale)}
\end{exemple}
\end{entrée}

\begin{entrée}{zao}{}{ⓔzao}
\région{GOs}
\variante{%
cao
\région{GO}}
(\domainesémantique{Coutumes, dons coutumiers
, Religion, représentations religieuses})
\classe{v}
\begin{glose}
\pfra{invoquer ; parler aux esprits}
\end{glose}
\end{entrée}

\begin{entrée}{zaòl}{}{ⓔzaòl}
\région{PA}
(\domainesémantique{Ignames})
\classe{nom}
\begin{glose}
\pfra{igname}
\end{glose}
\newline
\note{(clone à petites racines, planté sur le bord du billon, elles poussent plus vite que celles à racines longues du centre du billon et donnent les premières récoltes ; Charles)}{glose}{}
\end{entrée}

\begin{entrée}{zara}{}{ⓔzara}
\région{BO}
(\domainesémantique{Ignames})
\classe{nom}
\begin{glose}
\pfra{igname blanche, tendre (Dubois)}
\end{glose}
\newline
\note{non vérifié}{général}{}
\end{entrée}

\begin{entrée}{zaro}{}{ⓔzaro}
\région{GOs PA}
\variante{%
zharo
\région{GA}, 
zaatro
\région{vx (Haudricourt)}, 
yaro
\région{BO}}
\classe{v ; n}
\newline
\sens{1}
(\domainesémantique{Outils})
\begin{glose}
\pfra{fourche (trident) [GOs]}
\end{glose}
\newline
\sens{2}
(\domainesémantique{Outils})
\begin{glose}
\pfra{pelle à fouir les ignames (en bois ou fer) ; bêche [PA, BO]}
\end{glose}
\newline
\sens{3}
(\domainesémantique{Cultures, techniques, boutures})
\begin{glose}
\pfra{labourer avec une pelle à fouir}
\end{glose}
\end{entrée}

\begin{entrée}{zatri}{}{ⓔzatri}
\formephonétique{ðaɽi}
\région{GOs}
\variante{%
zari
\région{GO(s) PA}, 
zhari
\région{GA}, 
yari
\région{BO}}
(\domainesémantique{Remèdes, médecine})
\classe{nom}
\begin{glose}
\pfra{remède ; médicaments}
\end{glose}
\newline
\begin{exemple}
\textbf{\pnua{zatria-je [GOs ]}}
\pfra{son médicament}
\end{exemple}
\newline
\begin{exemple}
\région{PA}
\textbf{\pnua{i a thu-zari}}
\pfra{elle va chercher des plantes médicinales}
\end{exemple}
\newline
\begin{sous-entrée}{zari-raa}{ⓔzatriⓝzari-raa}
\begin{glose}
\pfra{les mauvais médicaments}
\end{glose}
\end{sous-entrée}
\newline
\begin{sous-entrée}{zari-zo}{ⓔzatriⓝzari-zo}
\begin{glose}
\pfra{les bons médicaments}
\end{glose}
\end{sous-entrée}
\newline
\begin{sous-entrée}{pwe-zari [GOs PA]}{ⓔzatriⓝpwe-zari [GOs PA]}
\begin{glose}
\pfra{guérisseur}
\end{glose}
\end{sous-entrée}
\newline
\begin{sous-entrée}{zari-alo [PA]}{ⓔzatriⓝzari-alo [PA]}
\begin{glose}
\pfra{devin, voyant}
\end{glose}
\end{sous-entrée}
\end{entrée}

\begin{entrée}{zava}{}{ⓔzava}
\formephonétique{ðava}
\région{GOs}
\variante{%
za
\région{PA}}
(\domainesémantique{Pronoms})
\classe{PRO 1° pers. excl. PL (sujet)}
\begin{glose}
\pfra{nous (excl)}
\end{glose}
\newline
\note{pronom indépendant}{général}{}
\end{entrée}

\begin{entrée}{-zava}{}{ⓔ-zava}
\formephonétique{ðava}
\région{GOs}
(\domainesémantique{Pronoms})
\classe{PRO 1° pers. excl. PL (OBJ ou POSS)}
\begin{glose}
\pfra{nous ; nos}
\end{glose}
\end{entrée}

\begin{entrée}{zawa}{}{ⓔzawa}
\formephonétique{ðawa}
\région{GOs}
(\domainesémantique{Pronoms})
\classe{PRO 2° pers. PL (sujet)}
\begin{glose}
\pfra{vous (pl.)}
\end{glose}
\end{entrée}

\begin{entrée}{-zawa}{}{ⓔ-zawa}
\formephonétique{ðawa}
\région{GO}
(\domainesémantique{Pronoms})
\classe{PRO 2° pers. PL (OBJ ou POSS)}
\begin{glose}
\pfra{vous ; vos}
\end{glose}
\end{entrée}

\begin{entrée}{zawe}{}{ⓔzawe}
\région{GOs}
(\domainesémantique{Ignames})
\classe{nom}
\begin{glose}
\pfra{igname (violette)}
\end{glose}
\end{entrée}

\begin{entrée}{zawexan}{}{ⓔzawexan}
\région{PA BO}
\variante{%
yawegan
\région{BO}}
(\domainesémantique{Description des objets, formes, consistance, taille})
\classe{nom}
\begin{glose}
\pfra{rouille}
\end{glose}
\end{entrée}

\begin{entrée}{za xa}{}{ⓔza xa}
\région{GOs}
\variante{%
xa
\région{GO(s)}}
(\domainesémantique{Aspect})
\classe{ITER}
\begin{glose}
\pfra{re-}
\end{glose}
\newline
\begin{exemple}
\région{GOs}
\textbf{\pnua{e za xa thoma-jo iò}}
\pfra{il t'a déjà/vraiment rappelé tout à l'heure}
\end{exemple}
\newline
\begin{exemple}
\région{GOs}
\textbf{\pnua{na za xa bwovwô ã-è Jae : "Kani, nu bwovwô !"}}
\pfra{quand Jae est à nouveau fatigué : "canard, je suis fatigué"}
\end{exemple}
\end{entrée}

\begin{entrée}{zaxòe}{}{ⓔzaxòe}
\formephonétique{ða'xo.e}
\région{GOsPA}
\variante{%
zhaxòe
\formephonétique{θa'xo.e}
\région{GA}, 
zakòe
\région{GO(s)}, 
yaxòe, yagoe
\région{BO}}
\newline
\sens{1}
(\domainesémantique{Aliments, alimentation})
\classe{v}
\begin{glose}
\pfra{goûter}
\end{glose}
\newline
\begin{exemple}
\région{PA}
\textbf{\pnua{la zaxò kui}}
\pfra{ils goûtent les prémisses de l'igname}
\end{exemple}
\newline
\sens{2}
(\domainesémantique{Modalité, verbes modaux})
\classe{v}
\begin{glose}
\pfra{essayer ; à l'essai ; à tout hasard}
\end{glose}
\newline
\begin{exemple}
\région{GOs}
\textbf{\pnua{e zaxòe ne}}
\pfra{il a essayé de le faire}
\end{exemple}
\end{entrée}

\begin{entrée}{zee}{}{ⓔzee}
\formephonétique{ðe}
\région{GOs TRE}
(\domainesémantique{Fonctions naturelles humaines})
\classe{v ; n}
\begin{glose}
\pfra{cracher}
\end{glose}
\newline
\begin{sous-entrée}{we-zee}{ⓔzeeⓝwe-zee}
\begin{glose}
\pfra{crachat}
\end{glose}
\end{sous-entrée}
\newline
\begin{sous-entrée}{paxa-zee}{ⓔzeeⓝpaxa-zee}
\begin{glose}
\pfra{expectorations}
\end{glose}
\end{sous-entrée}
\end{entrée}

\begin{entrée}{zeede}{}{ⓔzeede}
\région{GOs}
\variante{%
zedil
\région{PA}}
(\domainesémantique{Mouvements ou actions avec la tête, les yeux, la bouche})
\classe{v}
\begin{glose}
\pfra{siffler avec les doigts pour héler qqn.}
\end{glose}
\end{entrée}

\begin{entrée}{zeele}{}{ⓔzeele}
\région{GOs}
(\domainesémantique{Poissons})
\classe{nom}
\begin{glose}
\pfra{requin marteau}
\end{glose}
\nomscientifique{Sphyrna lewini (Sphyrnidae)}
\end{entrée}

\begin{entrée}{zeenô}{}{ⓔzeenô}
\formephonétique{ðeːɳõ}
\région{GOs PA}
\variante{%
zheenô
\région{GA}, 
yhèno, zeno
\région{BO}}
(\domainesémantique{Processus liés aux plantes})
\classe{v}
\begin{glose}
\pfra{mûr ; arrivé à maturité}
\end{glose}
\begin{glose}
\pfra{bien formé}
\end{glose}
\newline
\begin{exemple}
\région{GOs BO}
\textbf{\pnua{zeenô kui}}
\pfra{l'igname est arrivée à maturité}
\end{exemple}
\newline
\begin{exemple}
\région{GOs}
\textbf{\pnua{hãgana novwo e zeenô mwa kui}}
\pfra{maintenant que l'igname est mûre}
\end{exemple}
\newline
\begin{exemple}
\région{GOs}
\textbf{\pnua{e zeenô phagoo-je (enfant)}}
\pfra{son corps est arrivé à maturité}
\end{exemple}
\newline
\relationsémantique{Cf.}{\lien{ⓔmii}{mii}}
\glosecourte{mûr; rouge}
\newline
\relationsémantique{Ant.}{\lien{}{aava, aa}}
\glosecourte{pas encore mûr (contraire de zeenô)}
\end{entrée}

\begin{entrée}{zido}{}{ⓔzido}
\formephonétique{θindo}
\région{GOs}
\variante{%
zhido
\région{GA}}
(\domainesémantique{Fonctions naturelles humaines})
\classe{v}
\begin{glose}
\pfra{regarder (se) (dans un miroir)}
\end{glose}
\newline
\begin{exemple}
\textbf{\pnua{e zido Kaavwo ni we}}
\pfra{Kaavwo se regarde/se mire dans l'eau}
\end{exemple}
\end{entrée}

\begin{entrée}{zii}{}{ⓔzii}
\formephonétique{ðiː}
\région{GOs PA BO}
(\domainesémantique{Vêtements, parure})
\classe{nom}
\begin{glose}
\pfra{étoffe d'écorce de banian ;}
\end{glose}
\begin{glose}
\pfra{balassor}
\end{glose}
\end{entrée}

\begin{entrée}{zine}{}{ⓔzine}
\formephonétique{ðiɳe}
\région{GOs}
\variante{%
zhine
\région{GA}, 
jine
\région{GO(s)}}
(\domainesémantique{Mammifères})
\classe{nom}
\begin{glose}
\pfra{rat}
\end{glose}
\newline
\relationsémantique{Cf.}{\lien{}{ciibwin [BO PA]}}
\glosecourte{rat}
\end{entrée}

\begin{entrée}{zixô}{}{ⓔzixô}
\région{GOs PA}
\variante{%
zhixô
\région{GO(s)}, 
zikô, zhikô
\région{GO(s) vx}, 
hixò, hingõn
\région{BO [BM]}}
(\domainesémantique{Tradition orale})
\classe{v ; n}
\begin{glose}
\pfra{histoire ; fable}
\end{glose}
\begin{glose}
\pfra{raconter une histoire}
\end{glose}
\newline
\begin{exemple}
\région{GOs}
\textbf{\pnua{zixôô-nu}}
\pfra{ma fable ; mon conte}
\end{exemple}
\newline
\begin{exemple}
\région{GOs}
\textbf{\pnua{e zixô cai la pòi-je xo õ ẽnõ ã}}
\pfra{elle raconte une histoire à ses enfants (Doriane)}
\end{exemple}
\newline
\begin{exemple}
\région{GOs}
\textbf{\pnua{e zixô õ ẽnõ ã cai la pòi-je}}
\pfra{elle raconte une histoire à ses enfants (Doriane)}
\end{exemple}
\newline
\begin{exemple}
\région{PA}
\textbf{\pnua{zixò-ny}}
\pfra{ma fable ; mon conte}
\end{exemple}
\newline
\begin{exemple}
\région{BO}
\textbf{\pnua{hixòò-ny}}
\pfra{ma fable ; mon conte}
\end{exemple}
\newline
\begin{exemple}
\région{BO}
\textbf{\pnua{higõ-ny ã ciibwin ma amala-ò mèèni}}
\pfra{mon conte sur le rat et les autres oiseaux}
\end{exemple}
\end{entrée}

\begin{entrée}{zo}{1}{ⓔzoⓗ1}
\formephonétique{ðo}
\région{GOs PA}
\variante{%
zho
\région{GO(s)}, 
yo
\région{BO}}
\classe{v.stat.}
\newline
\sens{1}
(\domainesémantique{Caractéristiques et propriétés des personnes})
\begin{glose}
\pfra{bien ; bon}
\end{glose}
\newline
\begin{exemple}
\région{GO}
\textbf{\pnua{e wa zo}}
\pfra{il chante bien}
\end{exemple}
\newline
\begin{exemple}
\région{PA}
\textbf{\pnua{i wa zo}}
\pfra{il chante bien}
\end{exemple}
\newline
\begin{exemple}
\textbf{\pnua{e waze-zoo-ni wa}}
\pfra{il a bien chanté la chanson}
\end{exemple}
\newline
\begin{exemple}
\région{BO}
\textbf{\pnua{ne-yoone}}
\pfra{fais-le bien}
\end{exemple}
\newline
\begin{exemple}
\région{BO}
\textbf{\pnua{i khobwe-yoo-ni}}
\pfra{il a bien parlé}
\end{exemple}
\newline
\sens{2}
(\domainesémantique{Caractéristiques et propriétés des personnes})
\begin{glose}
\pfra{propre}
\end{glose}
\newline
\sens{3}
(\domainesémantique{Modalité, verbes modaux})
\begin{glose}
\pfra{pouvoir ; falloir ; devoir}
\end{glose}
\newline
\begin{exemple}
\région{GO}
\textbf{\pnua{e zo na jö wa zo}}
\pfra{tu dois bien chanter}
\end{exemple}
\newline
\begin{exemple}
\région{GO}
\textbf{\pnua{Axe jena, e zo na e a-du ni we-za}}
\pfra{Mais voilà, il faudra traverser la mer}
\end{exemple}
\newline
\note{forme transitive en composition : v.t.-zoo-ni, v.t.-yoo-ni}{grammaire}{}
\end{entrée}

\begin{entrée}{zo}{2}{ⓔzoⓗ2}
\région{GO}
\variante{%
ro
\région{WEMWE}}
(\domainesémantique{Temps})
\classe{FUT}
\begin{glose}
\pfra{futur}
\end{glose}
\newline
\begin{exemple}
\région{GO}
\textbf{\pnua{kavwö mi zo nõõ-je}}
\pfra{nous ne la reverrons plus}
\end{exemple}
\end{entrée}

\begin{entrée}{zo}{3}{ⓔzoⓗ3}
\région{GOs}
\variante{%
zho
\région{GO(s)}}
(\domainesémantique{Richesses, monnaies traditionnelles})
\classe{nom}
\begin{glose}
\pfra{biens ; affaires}
\end{glose}
\newline
\begin{exemple}
\textbf{\pnua{zoo i je}}
\pfra{ses affaires}
\end{exemple}
\end{entrée}

\begin{entrée}{zò}{1}{ⓔzòⓗ1}
\région{PA}
(\domainesémantique{Pronoms})
\classe{PRO 2° pers. PL (sujet, OBJ ou POSS)}
\begin{glose}
\pfra{vous, votre (plur)}
\end{glose}
\end{entrée}

\begin{entrée}{zò}{2}{ⓔzòⓗ2}
\région{GOs}
\variante{%
zhò
\région{GO(s)}, 
zòn
\région{PA}}
(\domainesémantique{Santé, maladie})
\classe{v}
\begin{glose}
\pfra{gratteux (être)}
\end{glose}
\nomscientifique{Ciguaterra}
\newline
\begin{sous-entrée}{no zò}{ⓔzòⓗ2ⓝno zò}
\begin{glose}
\pfra{poisson gratteux}
\end{glose}
\newline
\begin{exemple}
\région{GOs}
\textbf{\pnua{nu tròòli ma no zò}}
\pfra{j'ai attrapé la gratte/ la Ciguaterra}
\end{exemple}
\end{sous-entrée}
\end{entrée}

\begin{entrée}{zoa}{}{ⓔzoa}
\région{GOs}
(\domainesémantique{Chasse})
\classe{nom}
\begin{glose}
\pfra{lacet (chasse)}
\end{glose}
\end{entrée}

\begin{entrée}{zò-chaamwa}{}{ⓔzò-chaamwa}
\région{GOs}
(\domainesémantique{Bananiers et bananes})
\classe{nom}
\begin{glose}
\pfra{pousse (ou) rejet de bananier}
\end{glose}
\begin{glose}
\pfra{bouture de bananier}
\end{glose}
\end{entrée}

\begin{entrée}{zoe}{}{ⓔzoe}
\région{GOs}
(\domainesémantique{Mouvements ou actions faits avec le corps, les bras, les mains, les pieds})
\classe{v}
\begin{glose}
\pfra{attacher (lacet, vêtement)}
\end{glose}
\end{entrée}

\begin{entrée}{zòi}{}{ⓔzòi}
\formephonétique{ðɔɨ}
\région{GOs PA BO}
\variante{%
zhòi
\région{GO(s)}}
\classe{v.t.}
\newline
\sens{1}
(\domainesémantique{Travail bois
, Actions avec un instrument, un outil})
\begin{glose}
\pfra{scier}
\end{glose}
\newline
\begin{sous-entrée}{ba-zò-ce}{ⓔzòiⓢ1ⓝba-zò-ce}
\begin{glose}
\pfra{scie (à bois), tronçonneuse}
\end{glose}
\end{sous-entrée}
\newline
\begin{sous-entrée}{ba-zò go-tuxi}{ⓔzòiⓢ1ⓝba-zò go-tuxi}
\begin{glose}
\pfra{scie (à métaux)}
\end{glose}
\end{sous-entrée}
\newline
\sens{2}
(\domainesémantique{Mouvements ou actions faits avec le corps, les bras, les mains, les pieds})
\begin{glose}
\pfra{couper (se)}
\end{glose}
\begin{glose}
\pfra{couper (viande)}
\end{glose}
\newline
\begin{exemple}
\région{GOs}
\textbf{\pnua{e za draa zòi}}
\pfra{il s'est coupé volontairement}
\end{exemple}
\newline
\begin{exemple}
\région{GOs}
\textbf{\pnua{e zòi hi-je xo hèlè}}
\pfra{il s'est coupé la main avec le couteau}
\end{exemple}
\end{entrée}

\begin{entrée}{zòli}{}{ⓔzòli}
\région{GOs PA}
\variante{%
zhòli
\région{GO(s)}, 
yòli, yòòli
\région{BO}}
\classe{v.t.}
\newline
\sens{1}
(\domainesémantique{Préparation des aliments; modes de préparation et de cuisson})
\begin{glose}
\pfra{râper (coco)}
\end{glose}
\begin{glose}
\pfra{gratter (l'igname cuite ou la peau de l'igname, patates, taro)}
\end{glose}
\newline
\begin{exemple}
\région{BO}
\textbf{\pnua{zò-nu}}
\pfra{râper du coco}
\end{exemple}
\newline
\begin{sous-entrée}{ba-zò-nu}{ⓔzòliⓢ1ⓝba-zò-nu}
\région{BO}
\begin{glose}
\pfra{râpe à coco}
\end{glose}
\end{sous-entrée}
\newline
\sens{2}
(\domainesémantique{Mouvements ou actions faits avec le corps, les bras, les mains, les pieds})
\begin{glose}
\pfra{récurer (le dos de la marmite avec de la cendre)}
\end{glose}
\newline
\sens{3}
(\domainesémantique{Mouvements ou actions faits avec le corps, les bras, les mains, les pieds})
\begin{glose}
\pfra{griffer}
\end{glose}
\begin{glose}
\pfra{écorcher (s') la peau}
\end{glose}
\newline
\begin{exemple}
\région{BO}
\textbf{\pnua{i yòli-nu ho minòn}}
\pfra{le chat m'a griffé}
\end{exemple}
\newline
\relationsémantique{Cf.}{\lien{}{zòl (v.i.) [PA]}}
\newline
\étymologie{
\langue{POc}
\étymon{*kodi}}
\end{entrée}

\begin{entrée}{zoma}{}{ⓔzoma}
\région{GOs}
(\domainesémantique{Temps})
\classe{FUT}
\begin{glose}
\pfra{futur}
\end{glose}
\newline
\begin{exemple}
\région{GO}
\textbf{\pnua{nu zoma ne mo-jö}}
\pfra{je construirai ta maison}
\end{exemple}
\newline
\begin{exemple}
\région{GO}
\textbf{\pnua{e zoma li ubò mònò ? - Hai ! kò (neg) li zoma ubò mònò, e zoma li yuu avwônô}}
\pfra{vont-ils sortir demain ? - Non ! ils ne vont pas sortir,ils vont rester à la maison}
\end{exemple}
\newline
\begin{exemple}
\textbf{\pnua{e zoma a nye !}}
\pfra{il part tout de suite (imminent)}
\end{exemple}
\newline
\relationsémantique{Cf.}{\lien{}{e zoma}}
\glosecourte{futur proche}
\end{entrée}

\begin{entrée}{zòn}{}{ⓔzòn}
\région{PA}
\variante{%
yòn, yhòn
\région{BO}}
\classe{v ; n}
\newline
\sens{1}
(\domainesémantique{Description des objets, formes, consistance, taille})
\begin{glose}
\pfra{toxique ; non-comestible ; poison}
\end{glose}
\newline
\begin{exemple}
\région{PA}
\textbf{\pnua{dòò-ce zòn}}
\pfra{feuilles non comestibles, toxiques}
\end{exemple}
\newline
\begin{exemple}
\région{BO}
\textbf{\pnua{i yòn a nye nò}}
\pfra{ce poisson n'est pas comestible, est "gratteux" (maladie)}
\end{exemple}
\newline
\relationsémantique{Ant.}{\lien{}{dòò-ce hovho [PA]}}
\glosecourte{feuilles comestibles}
\newline
\sens{2}
(\domainesémantique{Santé, maladie})
\begin{glose}
\pfra{intoxiqué par la Ciguaterra}
\end{glose}
\newline
\begin{exemple}
\région{PA}
\textbf{\pnua{nu zòn}}
\pfra{je suis intoxiqué par la Ciguaterra}
\end{exemple}
\newline
\begin{exemple}
\région{PA}
\textbf{\pnua{ma zòn}}
\pfra{maladie de la "gratte"}
\end{exemple}
\end{entrée}

\begin{entrée}{zòò}{1}{ⓔzòòⓗ1}
\région{GOs}
(\domainesémantique{Description des objets, formes, consistance, taille})
\classe{nom}
\begin{glose}
\pfra{difficulté; embûche}
\end{glose}
\newline
\begin{exemple}
\textbf{\pnua{kixa zòò}}
\pfra{facile}
\end{exemple}
\newline
\begin{exemple}
\textbf{\pnua{pu zòò}}
\pfra{difficile}
\end{exemple}
\end{entrée}

\begin{entrée}{zòò}{2}{ⓔzòòⓗ2}
\formephonétique{ðɔː}
\région{GOs}
\variante{%
zhòò
\région{GO(s)}, 
zòòm
\formephonétique{ðɔːm}
\région{PA}, 
yhòòm, yòòm, yòò
\région{BO}}
\newline
\sens{1}
(\domainesémantique{Mouvements ou actions faits avec le corps, les bras, les mains, les pieds})
\classe{v}
\begin{glose}
\pfra{nager}
\end{glose}
\newline
\sens{2}
(\domainesémantique{Actions liées aux plantes})
\classe{v}
\begin{glose}
\pfra{ramper (lianes, sur les arbres)}
\end{glose}
\newline
\étymologie{
\langue{POc}
\étymon{*kaRu}
\glosecourte{nager}, 
\étymon{*kakau (*kk > θ, ð)}
\glosecourte{PNC}
\auteur{Haudricourt}}
\end{entrée}

\begin{entrée}{zòò}{3}{ⓔzòòⓗ3}
\région{GOs PA}
\variante{%
zhò
\région{GO(s)}}
(\domainesémantique{Cultures, techniques, boutures})
\classe{nom}
\begin{glose}
\pfra{rejet (de plante servant à bouturer)}
\end{glose}
\newline
\begin{sous-entrée}{zòò-uva, zòò kuru}{ⓔzòòⓗ3ⓝzòò-uva, zòò kuru}
\région{GOs}
\begin{glose}
\pfra{rejet de taro}
\end{glose}
\end{sous-entrée}
\newline
\begin{sous-entrée}{zòò-chaamwa}{ⓔzòòⓗ3ⓝzòò-chaamwa}
\région{GOs}
\begin{glose}
\pfra{rejet de bananier}
\end{glose}
\end{sous-entrée}
\newline
\begin{sous-entrée}{zòò-ê}{ⓔzòòⓗ3ⓝzòò-ê}
\région{GOs}
\begin{glose}
\pfra{rejet/bouture de canne à sucre}
\end{glose}
\end{sous-entrée}
\end{entrée}

\begin{entrée}{zòòni}{}{ⓔzòòni}
\formephonétique{ðɔːɳi}
\région{GOs PA}
\variante{%
zhòòni
\région{GO(s)}, 
yooni
\région{BO (BM, Corne)}, 
yhoonik
\région{BO (Corne)}}
(\domainesémantique{Arbre})
\classe{nom}
\begin{glose}
\pfra{niaoulis}
\end{glose}
\newline
\note{Lors des naissances, on enveloppait le nourrisson dans l'écorce de niaoulis pour le protéger et lui donner de la force. Lors des décès, les feuilles et branches enveloppent la monnaie blanche qui autorise l'échange entre deux clans.}{glose}{}
\nomscientifique{Melaleuca leucadendron (Myrtacées)}
\newline
\begin{sous-entrée}{mû-zòòni}{ⓔzòòniⓝmû-zòòni}
\begin{glose}
\pfra{fleur de niaoulis}
\end{glose}
\end{sous-entrée}
\end{entrée}

\begin{entrée}{zòò-uva}{}{ⓔzòò-uva}
\région{GOs PA}
(\domainesémantique{Taros})
\classe{nom}
\begin{glose}
\pfra{jeunes pousses; repousse (de taro, bananier, déraciné puis transplanté)}
\end{glose}
\newline
\begin{exemple}
\région{BO}
\textbf{\pnua{zòò-n}}
\pfra{ses jeunes pousses}
\end{exemple}
\end{entrée}

\begin{entrée}{zòòwa}{}{ⓔzòòwa}
\région{GOs}
(\domainesémantique{Poissons})
\classe{nom}
\begin{glose}
\pfra{relégué}
\end{glose}
\nomscientifique{Terapon jarbua (Teraponidae)}
\end{entrée}

\begin{entrée}{zovaale}{}{ⓔzovaale}
\région{GOs}
\variante{%
zovaale
\région{PA}, 
yo-vhaale
\région{BO [Corne]}}
(\domainesémantique{Relations et interaction sociales})
\classe{v}
\begin{glose}
\pfra{rapporter ; dénoncer}
\end{glose}
\end{entrée}

\begin{entrée}{zò-xabu}{}{ⓔzò-xabu}
\région{GOs}
\variante{%
yo-xabu
\région{BO [Corne]}}
(\domainesémantique{Corps humain})
\classe{nom}
\begin{glose}
\pfra{chair de poule (froid)}
\end{glose}
\newline
\relationsémantique{Cf.}{\lien{}{zò khaabu}}
\glosecourte{réaction cutanée au froid}
\end{entrée}

\begin{entrée}{zoxãî}{}{ⓔzoxãî}
\région{WEM WE}
(\domainesémantique{Caractéristiques et propriétés des personnes})
\classe{v.stat.}
\begin{glose}
\pfra{joli; bien}
\end{glose}
\end{entrée}

\begin{entrée}{zòxu}{}{ⓔzòxu}
\région{GOs}
\variante{%
zhòxu
\région{GO(s)}}
(\domainesémantique{Poissons})
\classe{nom}
\begin{glose}
\pfra{carpe}
\end{glose}
\nomscientifique{Kuhlia sp. (Kuhliidae)}
\newline
\relationsémantique{Cf.}{\lien{ⓔthãi}{thãi}}
\glosecourte{carpe}
\end{entrée}

\begin{entrée}{zu}{}{ⓔzu}
\région{GOs PA}
\variante{%
zhu
\région{GO(s)}, 
yu
\région{BO}}
(\domainesémantique{Poissons})
\classe{nom}
\begin{glose}
\pfra{mulet de rivière (noir, pond vers la mer puis remonte dans la rivière)}
\end{glose}
\nomscientifique{Myxes elong ?}
\newline
\étymologie{
\langue{PEOc}
\étymon{*zau}
\glosecourte{mulet de rivière}}
\end{entrée}

\begin{entrée}{zuanga}{}{ⓔzuanga}
\formephonétique{ðu'aŋa}
\région{GOs PA}
\variante{%
phwa-zua
\région{GO(s)}, 
zhuanga
\région{GO(s)}, 
yuanga
\région{BO}}
(\domainesémantique{Discours, échanges verbaux})
\classe{nom}
\begin{glose}
\pfra{zuanga (nom de la langue)}
\end{glose}
\end{entrée}

\begin{entrée}{zugi}{}{ⓔzugi}
\formephonétique{ðuŋgi}
\région{GOs PA}
\variante{%
zhugi
\formephonétique{θuŋgi}
\région{GA}, 
yugi
\région{BO}}
(\domainesémantique{Mouvements ou actions faits avec le corps, les bras, les mains, les pieds})
\classe{v}
\begin{glose}
\pfra{tordre}
\end{glose}
\begin{glose}
\pfra{courber}
\end{glose}
\begin{glose}
\pfra{ployer}
\end{glose}
\begin{glose}
\pfra{arquer}
\end{glose}
\begin{glose}
\pfra{ramasser (en roulant)}
\end{glose}
\begin{glose}
\pfra{(re)tirer}
\end{glose}
\begin{glose}
\pfra{rembobiner (ligne de pêche)}
\end{glose}
\begin{glose}
\pfra{retrousser (robe, pantalon)}
\end{glose}
\newline
\begin{exemple}
\textbf{\pnua{e zugi pwiò}}
\pfra{il tire le filet}
\end{exemple}
\newline
\begin{exemple}
\textbf{\pnua{e zugi pwe}}
\pfra{il rembobine la ligne (de pêche)}
\end{exemple}
\newline
\begin{exemple}
\région{GO}
\textbf{\pnua{e zugi hõboli-je}}
\pfra{elle a retroussé sa robe}
\end{exemple}
\newline
\begin{exemple}
\région{GO}
\textbf{\pnua{e zugi hi-hõboli-je}}
\pfra{elle a retroussé les manches de son vêtement}
\end{exemple}
\newline
\begin{exemple}
\région{BO}
\textbf{\pnua{i yugi wal}}
\pfra{il enroule la corde}
\end{exemple}
\end{entrée}

\begin{entrée}{zume}{}{ⓔzume}
\région{GOs}
\variante{%
zhume
\région{GO(s)}, 
zome
\région{PA}, 
zume-n, yume-n
\région{BO}}
(\domainesémantique{Fonctions naturelles humaines})
\classe{v}
\begin{glose}
\pfra{cracher}
\end{glose}
\newline
\begin{exemple}
\région{GOs}
\textbf{\pnua{zume du pwa !}}
\pfra{crache dehors !}
\end{exemple}
\newline
\begin{sous-entrée}{we ni zume}{ⓔzumeⓝwe ni zume}
\région{PA}
\begin{glose}
\pfra{crachat}
\end{glose}
\end{sous-entrée}
\newline
\begin{sous-entrée}{we-zume}{ⓔzumeⓝwe-zume}
\région{GOs}
\begin{glose}
\pfra{crachat}
\end{glose}
\end{sous-entrée}
\newline
\begin{sous-entrée}{paxa-zume}{ⓔzumeⓝpaxa-zume}
\begin{glose}
\pfra{expectorations}
\end{glose}
\end{sous-entrée}
\end{entrée}

\begin{entrée}{zuzuu}{}{ⓔzuzuu}
\région{GOs}
\variante{%
zhuzuu
\région{GO(s)}}
(\domainesémantique{Crustacés, crabes})
\classe{nom}
\begin{glose}
\pfra{crabe de sable}
\end{glose}
\end{entrée}

\end{multicols}
\end{document}