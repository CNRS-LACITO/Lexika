
\documentclass[twoside,11pt]{article}
\title{Essai d’un dictionnaire thématique inverse}
\author{Isabelle Bril}
\usepackage[paperwidth=185mm,paperheight=260mm,top=16mm,bottom=16mm,left=15mm,right=20mm]{geometry}
\usepackage{multicol}
\setlength{\columnseprule}{1pt}
\setlength{\columnsep}{1.5cm}
\usepackage{titlesec}
\usepackage{changepage}
\usepackage[dvipsnames,table]{xcolor}
\usepackage{fancyhdr}
\pagestyle{fancy}
\fancyheadoffset{3.4em}
\fancyhead[LE,LO]{\rightmark}
\fancyhead[RE,RO]{\leftmark}
\usepackage{hyperref}
\hypersetup{pdftex,bookmarks=true,bookmarksnumbered,bookmarksopenlevel=5,bookmarksdepth=5,xetex,colorlinks=true,linkcolor=blue,citecolor=blue}
\usepackage[all]{hypcap}
\usepackage{fontspec}
\usepackage{natbib}
\usepackage{booktabs}
\usepackage{polyglossia}
\setdefaultlanguage{french}
\setmainfont{Liberation Serif}
\newfontfamily{\déf}[Mapping=tex-text,Ligatures=Common,Scale=MatchUppercase]{Liberation Serif}
\newfontfamily{\nua}[Mapping=tex-text,Ligatures=Common,Scale=MatchUppercase]{Charis SIL}
\newfontfamily{\fra}[Mapping=tex-text,Ligatures=Common,Scale=MatchUppercase]{EB Garamond}
\newcommand{\pdéf}[1]{\déf #1}
\newcommand{\pfra}[1]{\fra #1}
\newcommand{\pnua}[1]{\nua #1}
\newcommand{\cerclé}[1]{\raisebox{0pt}{\textcircled{\raisebox{-0.5pt} {\footnotesize{\pdéf{#1}}}}}}
\newenvironment{entrée}[1]{\addcontentsline{toc}{subsection}{#1}\hspace*{-1cm}\textbf{\pfra{\textcolor{OliveGreen}{#1}}}\markright{#1}}{}
\newenvironment{sous-entrée}[1]{\addcontentsline{toc}{subsubsection}{#1}\pdéf{■}~\textbf{\pfra{\textcolor{OliveGreen}{#1}}}}{}
\newenvironment{exemple}{\pdéf{¶}}{}
\newcommand{\vedette}[1]{\textbf{#1}}
\newcommand{\homonyme}[1]{\textcolor{Red}{\textsuperscript{#1}}}
\newcommand{\région}[1]{\textcolor{Gray}{[#1]}}
\newcommand{\variante}[1]{\textcolor{Sepia}{(#1)}}
\newcommand{\groupe}[1]{\cerclé{#1}}
\newcommand{\classe}[1]{\pfra{\textcolor{Blue}{\emph{#1}}}}
\newcommand{\sens}[1]{\cerclé{#1}}
\newcommand{\relationsémantique}[2]{\emph{#1}~:~\pnua{\textcolor{Sepia}{#2}}}
\newcommand{\lien}[2]{\hyperlink{#1}{\pnua{#2}}}
\setcounter{secnumdepth}{4}
\titleformat{\subsubsection}{\large\bfseries}{\thesubsubsection}{1em}{}
\titleformat{\paragraph}{\large\bfseries}{\theparagraph}{1em}{}
\titlespacing*{\paragraph}
{\normalfont\normalsize\bfseries}{\theparagraph}{1em}{}
\titlespacing*{\paragraph}
{0pt}{3.25ex plus 1ex minus .2ex}{1.5ex plus .2ex}\begin{document}
�introduction
\begin{multicols}{2}
\lhead{\firstmark}
\rhead{\botmark}
\section{Le corps : humains et animaux}

\subsection{Anatomie}

\subsubsection{Corps humain}

\begin{entrée}
{abdomen}
\vedette{pwen}
\région{BO}
\end{entrée}

\begin{entrée}
{aile ;}
\classe{nom}
\vedette{hii}
\sens{1}
\région{GOs BO PA}
\variante{%
\vedette{yi-n}}
\end{entrée}

\begin{entrée}
{aine}
\vedette{piçanga}
\région{GOs}
\variante{%
\vedette{pijanga}
\région{BO}}
\end{entrée}

\begin{entrée}
{aisselle}
\vedette{hãbe}
\région{GOs BO}
\end{entrée}

\begin{entrée}
{albinos}
\vedette{köö-vwölö}
\région{GOs}
\variante{%
\vedette{kuuwulo, kuuwolo}
\région{BO}}
\end{entrée}

\begin{entrée}
{amputé (d'une partie du corps)}
\vedette{bwehulo}
\région{GOs}
\end{entrée}

\begin{entrée}
{anus (grossier, Dubois)}
\vedette{phwè-na}
\région{GOs}
\variante{%
\vedette{phwe-nò}
\région{BO}}
\end{entrée}

\begin{entrée}
{apparence}
\vedette{me}\homonyme{3}
\région{GOs BO}
\variante{%
\vedette{mèè-n}
\région{BO}}
\end{entrée}

\begin{entrée}
{arcade sourcilière}
\vedette{bwa-kitra-me}
\région{GOs}
\variante{%
\vedette{bwa-kira-me, bwa-gira-mè}
\région{GO(s) PA BO}}
\variante{%
\vedette{bwagila-me}
\région{BO}}
\end{entrée}

\begin{entrée}
{arrière de la jambe}
\vedette{kaya-ko}
\région{WEM}
\end{entrée}

\begin{entrée}
{articulation}
\vedette{buji-}
\région{GOs PA}
\end{entrée}

\begin{entrée}
{articulation de (toute) la jambe}
\vedette{böji-kò}
\région{GOs}
\end{entrée}

\begin{entrée}
{avant-bras}
\classe{nom}
\vedette{gò-hii}
\sens{1}
\région{PA}
\end{entrée}

\begin{entrée}
{avant-bras [Corne]}
\vedette{phajoo-hii-n}
\région{BO}
\end{entrée}

\begin{entrée}
{barbe}
\vedette{pu-bwa-n}
\région{BO PA}
\end{entrée}

\begin{entrée}
{barbe ; moustache}
\vedette{pu-phwa}
\région{GOs}
\variante{%
\vedette{pu-phwa}
\région{BO PA}}
\end{entrée}

\begin{entrée}
{bas de la jambe [Corne]}
\vedette{phajoo-kòò-n}
\région{BO}
\end{entrée}

\begin{entrée}
{biceps}
\vedette{pò-mugo ni hi}
\région{WEMBO}
\variante{%
\vedette{pò-mugo ne hii-n}
\région{BO}}
\end{entrée}

\begin{entrée}
{bile ; fiel}
\vedette{azi}
\région{GOs}
\région{PA}
\variante{%
\vedette{kââli}}
\variante{%
\vedette{kâli}
\région{BO}}
\variante{%
\vedette{we-khâli}
\région{PA BO}}
\end{entrée}

\begin{entrée}
{bosse (sur la tête)}
\classe{nom}
\vedette{pui}
\sens{2}
\région{GOs PA}
\variante{%
\vedette{phui-n}
\région{BO [Corne]}}
\end{entrée}

\begin{entrée}
{bouche}
\vedette{me-phwa}
\région{GOs}
\région{GOs}
\variante{%
\vedette{me-pwa}}
\end{entrée}

\begin{entrée}
{bouche}
\classe{nom}
\vedette{phwa}\homonyme{1}
\groupe{A}
\sens{1}
\région{GOs PA BO}
\end{entrée}

\begin{entrée}
{bras ;}
\classe{nom}
\vedette{hii}
\sens{1}
\région{GOs BO PA}
\variante{%
\vedette{yi-n}}
\end{entrée}

\begin{entrée}
{cerveau ; cervelle}
\vedette{zagi}
\région{GOs PA}
\variante{%
\vedette{zhagia}
\région{GO(s)}}
\variante{%
\vedette{yagi}
\région{PA BO}}
\end{entrée}

\begin{entrée}
{chair de poule (froid)}
\vedette{zò-xabu}
\région{GOs}
\variante{%
\vedette{yo-xabu}
\région{BO [Corne]}}
\end{entrée}

\begin{entrée}
{chauve}
\vedette{thra}\homonyme{1}
\région{GOs}
\variante{%
\vedette{tha, pa-tha}
\région{PA BO}}
\end{entrée}

\begin{entrée}
{cheveux}
\vedette{pu}\homonyme{1}
\région{GOs PA BO}
\variante{%
\vedette{phu}}
\end{entrée}

\begin{entrée}
{cheveux}
\vedette{pu-bwa-n}
\région{BO PA}
\end{entrée}

\begin{entrée}
{cheveux blancs (avoir les)}
\vedette{huvado}
\région{GOs PA BO}
\variante{%
\vedette{vado}
\région{GO(s) BO}}
\end{entrée}

\begin{entrée}
{cheville}
\vedette{nõõ-kò}
\région{GOs}
\end{entrée}

\begin{entrée}
{cheville ; malléole de la cheville ? (Haudricourt)}
\vedette{pòni-ma-wee}
\région{GOs}
\variante{%
\vedette{pòdi-ma-pwèèl}
\région{BO [Corne]}}
\end{entrée}

\begin{entrée}
{cils}
\vedette{pu-mee}
\région{GOs}
\end{entrée}

\begin{entrée}
{circoncision ; subincision (moment où on donnait le bagayou au garçon)}
\vedette{tragòò}\homonyme{1}
\région{GOs}
\variante{%
\vedette{taagò}
\région{BO}}
\end{entrée}

\begin{entrée}
{clavicule}
\vedette{pi-wãge}
\région{GOs}
\end{entrée}

\begin{entrée}
{clitoris}
\vedette{pidi}
\région{GO}
\end{entrée}

\begin{entrée}
{clitoris [Corne]}
\vedette{kurô}
\région{BO}
\end{entrée}

\begin{entrée}
{coeur}
\classe{nom}
\vedette{phwe-ai}
\sens{1}
\région{GOs PA BO}
\end{entrée}

\begin{entrée}
{coiffure de}
\classe{n (composition)}
\vedette{bwe-}
\sens{1}
\région{GOs BO PA}
\end{entrée}

\begin{entrée}
{colonne vertébrale}
\vedette{cii-du}
\région{GOs}
\end{entrée}

\begin{entrée}
{colonne vertébrale}
\vedette{du-kai}
\région{GOs}
\variante{%
\vedette{du-kaè-n}
\région{PA BO}}
\end{entrée}

\begin{entrée}
{cordon ombilical ; nom de l'arbre planté à la naissance (et sous lequel on enterrait le cordon ombilical)}
\vedette{bozo}
\région{GOs}
\variante{%
\vedette{bolo}
\région{PA WE}}
\end{entrée}

\begin{entrée}
{corps ; enveloppe corporelle}
\vedette{phãgoo}
\région{GOs}
\variante{%
\vedette{phagoo}
\région{PA BO}}
\end{entrée}

\begin{entrée}
{côté (du corps)}
\vedette{alabo}
\région{GOs WEM}
\end{entrée}

\begin{entrée}
{côtes}
\vedette{dee}\homonyme{1}
\région{PA BO}
\end{entrée}

\begin{entrée}
{côtes}
\vedette{dö}
\région{BO}
\end{entrée}

\begin{entrée}
{côtes}
\classe{nom}
\vedette{zabò}
\sens{1}
\région{GOs BO PAWEM}
\variante{%
\vedette{zhabò}
\région{GA}}
\end{entrée}

\begin{entrée}
{coude}
\vedette{bwa-kitra-hi}
\région{GOs}
\variante{%
\vedette{bwagira-hi}
\région{GO(s) BO}}
\end{entrée}

\begin{entrée}
{coude ; l'articulation de (tout) le bras}
\vedette{böji-hi}
\région{GOs}
\end{entrée}

\begin{entrée}
{cou ; gorge}
\vedette{nõõ}\homonyme{1}
\région{GOs}
\variante{%
\vedette{nõõ}
\région{BO PA}}
\end{entrée}

\begin{entrée}
{crâne}
\vedette{pi-bwa}
\région{GOs PA}
\variante{%
\vedette{du-bwaa-n}
\région{BO PA}}
\end{entrée}

\begin{entrée}
{cuisse ; fesse}
\vedette{pè}
\région{GOs BO PA}
\end{entrée}

\begin{entrée}
{dent}
\vedette{patrô}
\région{GOs}
\variante{%
\vedette{parô-, parôô-n}
\région{BO PA}}
\end{entrée}

\begin{entrée}
{derrière ; postérieur}
\vedette{pòmi-nò}
\région{GOs}
\variante{%
\vedette{pòbwi-nò-n}
\région{BO}}
\end{entrée}

\begin{entrée}
{dessus /dos (de la main)}
\vedette{bwa-kaça hi}
\région{GOs}
\variante{%
\vedette{bwa-xaça}
\région{GO(s)}}
\end{entrée}

\begin{entrée}
{diaphragme[BM]}
\classe{nom}
\vedette{phwaxoi-n}
\région{BO}
\end{entrée}

\begin{entrée}
{diaphragme (lit. orifice du souffle) ; sternum}
\vedette{phwe-chãnã}
\région{GOs}
\end{entrée}

\begin{entrée}
{dos}
\classe{nom}
\vedette{du}\homonyme{1}
\sens{2}
\région{GOs BO PA}
\end{entrée}

\begin{entrée}
{dos}
\classe{n ; n.LOC}
\vedette{kai}\homonyme{2}
\sens{1}
\région{GOsBO PA}
\end{entrée}

\begin{entrée}
{édenté}
\vedette{whau}
\région{GOs}
\variante{%
\vedette{whãup}
\région{PA}}
\end{entrée}

\begin{entrée}
{édenté ; interstice laissé par des dents tombées}
\classe{v.stat.}
\vedette{wado}
\sens{1}
\région{GOs BO}
\variante{%
\vedette{waado}
\région{BO}}
\end{entrée}

\begin{entrée}
{enveloppe}
\classe{nom}
\vedette{dou}\homonyme{2}
\sens{1}
\région{GOs PA}
\variante{%
\vedette{deü}
\région{BO [Corne]}}
\end{entrée}

\begin{entrée}
{enveloppe (corporelle)}
\classe{nom}
\vedette{dou}\homonyme{2}
\sens{1}
\région{GOs PA}
\variante{%
\vedette{deü}
\région{BO [Corne]}}
\end{entrée}

\begin{entrée}
{épaule}
\vedette{buu}\homonyme{1}
\région{GOs BO PA}
\end{entrée}

\begin{entrée}
{espace entre les côtes}
\classe{nom}
\vedette{kevwa}
\sens{2}
\région{GOs}
\variante{%
\vedette{kewang}
\région{BO PA}}
\end{entrée}

\begin{entrée}
{estomac}
\vedette{mõ-hovwo}
\région{GOs}
\variante{%
\vedette{mõ-(h)ovwo}
\région{GO(s)}}
\variante{%
\vedette{mõ-hopo}
\région{GO vx}}
\end{entrée}

\begin{entrée}
{estomac [Coyaud]}
\classe{nom}
\vedette{ono-n}
\région{BO}
\end{entrée}

\begin{entrée}
{face ; visage ; avant}
\classe{n ; PREF. sémantique (référant à une surface extérieure)}
\vedette{ala-}
\sens{1}
\région{GOs PA BO}
\end{entrée}

\begin{entrée}
{fesses}
\vedette{pòmi-nò}
\région{GOs}
\variante{%
\vedette{pòbwi-nò-n}
\région{BO}}
\end{entrée}

\begin{entrée}
{fesses ; derrière ; postérieur}
\vedette{pòbwinõ}
\région{PA BO}
\end{entrée}

\begin{entrée}
{fesses[Dubois]}
\vedette{thii-no}
\région{BO}
\end{entrée}

\begin{entrée}
{figure; visage}
\vedette{me}\homonyme{3}
\région{GOs BO}
\variante{%
\vedette{mèè-n}
\région{BO}}
\end{entrée}

\begin{entrée}
{foie}
\vedette{kòi}\homonyme{2}
\région{GOs PA BO}
\end{entrée}

\begin{entrée}
{fontanelle}
\vedette{phwè-whaa}
\région{GOs}
\variante{%
\vedette{phwè-whara}
\région{PA}}
\end{entrée}

\begin{entrée}
{fontanelle [Corne]}
\vedette{nõbwo wha}
\région{BO}
\end{entrée}

\begin{entrée}
{fontanelle [Haudricourt]}
\vedette{calo}
\région{GO}
\end{entrée}

\begin{entrée}
{fourrure}
\vedette{pu}\homonyme{1}
\région{GOs PA BO}
\variante{%
\vedette{phu}}
\end{entrée}

\begin{entrée}
{front}
\vedette{bwèèdrò}\homonyme{2}
\région{GOs}
\variante{%
\vedette{bwèèdò}
\région{PA BO}}
\variante{%
\vedette{bwèdòl}
\région{BO}}
\end{entrée}

\begin{entrée}
{gencives}
\vedette{jingã}
\région{GOs}
\variante{%
\vedette{jing}
\région{PA}}
\variante{%
\vedette{pajin}
\région{BO [Corne]}}
\end{entrée}

\begin{entrée}
{genou}
\vedette{bwagili}
\région{GOs PA WEM}
\variante{%
\vedette{bwagil}
\région{BO}}
\end{entrée}

\begin{entrée}
{gésier; estomac}
\vedette{paxa-nò}
\région{GOs PA BO}
\variante{%
\vedette{paga-nò}
\région{BO}}
\end{entrée}

\begin{entrée}
{gland}
\vedette{hê-cò}
\région{GO}
\end{entrée}

\begin{entrée}
{globe oculaire}
\vedette{hê-me}
\région{GOs}
\end{entrée}

\begin{entrée}
{glotte}
\vedette{atrilò}
\région{GOs BO PA}
\end{entrée}

\begin{entrée}
{griffe ; ongle de pied}
\vedette{pi-kò}
\région{GOs PA}
\end{entrée}

\begin{entrée}
{hanche}
\vedette{pu-pee-ã}
\région{GOs}
\variante{%
\vedette{pu-vee}
\région{GO(s)}}
\end{entrée}

\begin{entrée}
{hanche (Haudricourt, Corne)}
\vedette{dive}
\région{BO}
\end{entrée}

\begin{entrée}
{hanche (lit. la base des jambes) [Corne]}
\vedette{pu-kòò-n}
\région{BO}
\end{entrée}

\begin{entrée}
{interstice entre les dents}
\classe{nom}
\vedette{phwè-wado}
\sens{2}
\région{BO}
\end{entrée}

\begin{entrée}
{intestins ; boyaux ; entrailles}
\vedette{hoońõ}
\région{GOs PA BO}
\end{entrée}

\begin{entrée}
{iris (oeil) (lit. fruit de Vitex trifoliata)}
\vedette{pò-draado}
\région{GOs}
\end{entrée}

\begin{entrée}
{joue}
\vedette{pwaang}
\région{PA}
\end{entrée}

\begin{entrée}
{joue}
\vedette{pwawaa}
\région{GOs}
\variante{%
\vedette{pwawa}
\région{PA BO}}
\variante{%
\vedette{pwaaò-n}
\région{BO}}
\end{entrée}

\begin{entrée}
{lait maternel (lit. liquide de sein)}
\vedette{we-thi}
\région{GOs BO PA}
\end{entrée}

\begin{entrée}
{langue}
\classe{nom}
\vedette{kumè}
\sens{1}
\région{GOs BO PA}
\end{entrée}

\begin{entrée}
{larmes}
\vedette{we ni mè}
\région{GOs}
\variante{%
\vedette{we ni mèè-n}
\région{PA BO}}
\end{entrée}

\begin{entrée}
{lèvres}
\vedette{ci-phwa}
\région{GOs}
\variante{%
\vedette{ci-phwa-n}
\région{BO PA}}
\end{entrée}

\begin{entrée}
{lèvres (feuille-bouche)}
\vedette{dòò-phwa}
\région{PA}
\end{entrée}

\begin{entrée}
{lobe (oreille)}
\vedette{dròò-keni}
\région{GOs}
\variante{%
\vedette{dròò-xeni}
\région{GO(s)}}
\variante{%
\vedette{cò-keni}
\région{GO(s) PA}}
\variante{%
\vedette{dòò-keni}
\région{BO}}
\variante{%
\vedette{dòò-va-jeni}
\région{WEM}}
\end{entrée}

\begin{entrée}
{main ;}
\classe{nom}
\vedette{hii}
\sens{1}
\région{GOs BO PA}
\variante{%
\vedette{yi-n}}
\end{entrée}

\begin{entrée}
{main droite}
\vedette{gu-hi-n}
\région{PA BO}
\variante{%
\vedette{gu-yin}
\région{BO}}
\end{entrée}

\begin{entrée}
{main droite (sa)}
\vedette{hii-je bwa mhwã}
\région{GOs}
\end{entrée}

\begin{entrée}
{main gauche (sa)}
\vedette{hii-je mhõ}
\région{GOs}
\end{entrée}

\begin{entrée}
{malléole}
\vedette{pii-peçi}
\région{GOs}
\variante{%
\vedette{pi-peyi, pi-peji}
\région{PA}}
\end{entrée}

\begin{entrée}
{mamelle}
\vedette{thi}\homonyme{1}
\région{GOs BO PA}
\end{entrée}

\begin{entrée}
{mamelon du sein}
\vedette{pò-thi}
\région{GOs}
\variante{%
\vedette{pò-ti-n, pwò-thi-n}
\région{BO}}
\end{entrée}

\begin{entrée}
{mèches de cheveux}
\classe{nom}
\vedette{nu}\homonyme{3}
\sens{2}
\région{GOs PA}
\end{entrée}

\begin{entrée}
{menton}
\vedette{baado}
\région{GOs PA}
\end{entrée}

\begin{entrée}
{moëlle des os}
\vedette{we hêê-du}
\région{GOs}
\variante{%
\vedette{bòyil, böyil}
\région{PA}}
\end{entrée}

\begin{entrée}
{mollet}
\vedette{buzo-kò}
\région{GOs}
\end{entrée}

\begin{entrée}
{mollet}
\vedette{pò-mugo ni kò}
\région{WEM BO}
\variante{%
\vedette{pò-mugo ne kòò-n}
\région{BO}}
\end{entrée}

\begin{entrée}
{morve}
\vedette{têi}\homonyme{2}
\région{GOs BO PA}
\end{entrée}

\begin{entrée}
{moustache}
\vedette{pu-bwa-n}
\région{BO PA}
\end{entrée}

\begin{entrée}
{narine}
\vedette{phwè-mwêêdi}
\région{GOs PA BO}
\end{entrée}

\begin{entrée}
{nez}
\vedette{mhwêêdi}
\région{GOs BO}
\variante{%
\vedette{mweedi}
\région{PA BO}}
\end{entrée}

\begin{entrée}
{nombril}
\vedette{phwè-bozo}
\région{GOs}
\variante{%
\vedette{pwe-bulo-n}
\région{PA BO}}
\end{entrée}

\begin{entrée}
{nuque}
\vedette{pu-noo}
\région{GOs}
\variante{%
\vedette{pu-noo-n}
\région{PA}}
\end{entrée}

\begin{entrée}
{nuque}
\vedette{puvwu}
\région{GOs}
\variante{%
\vedette{pupu-n, puvwu-n}
\région{BO}}
\end{entrée}

\begin{entrée}
{oeil}
\vedette{êdi-me}
\région{GOs}
\end{entrée}

\begin{entrée}
{oeil}
\vedette{me}\homonyme{1}
\région{GOs BO}
\variante{%
\vedette{pii-me}
\région{GO(s) PA}}
\variante{%
\vedette{pimèè-n ; mèè-n}
\région{BO}}
\end{entrée}

\begin{entrée}
{oeil}
\vedette{pii-me}
\région{GOs BO PA}
\end{entrée}

\begin{entrée}
{omoplate (lit. la carapace du coeur)}
\vedette{pi-ai}
\région{GOs BO}
\end{entrée}

\begin{entrée}
{ongle}
\vedette{pi-bwèdò}
\région{GOs BO}
\end{entrée}

\begin{entrée}
{ongle}
\vedette{pi-hi}
\région{GOs PA BO}
\variante{%
\vedette{pi-yi}
\région{BO}}
\end{entrée}

\begin{entrée}
{oreille}
\vedette{kêni}
\région{GOs WEM PA BO}
\variante{%
\vedette{kîni}}
\end{entrée}

\begin{entrée}
{orifice}
\classe{nom}
\vedette{phwè-}
\sens{2}
\région{GOs}
\end{entrée}

\begin{entrée}
{os}
\classe{nom}
\vedette{du}\homonyme{1}
\sens{1}
\région{GOs BO PA}
\end{entrée}

\begin{entrée}
{palais (bouche)}
\classe{nom}
\vedette{ãmã}
\sens{1}
\région{GOs}
\end{entrée}

\begin{entrée}
{partie du nez entre les deux narines [Corne]}
\vedette{cò mhwedin}
\région{BO}
\end{entrée}

\begin{entrée}
{paume}
\vedette{ala-hi}
\groupe{A}
\région{GOs BO PA}
\end{entrée}

\begin{entrée}
{paupière}
\vedette{ci-mee}
\région{GOs}
\end{entrée}

\begin{entrée}
{pavillon de l'oreille}
\vedette{dròò-keni}
\région{GOs}
\variante{%
\vedette{dròò-xeni}
\région{GO(s)}}
\variante{%
\vedette{cò-keni}
\région{GO(s) PA}}
\variante{%
\vedette{dòò-keni}
\région{BO}}
\variante{%
\vedette{dòò-va-jeni}
\région{WEM}}
\end{entrée}

\begin{entrée}
{peau}
\classe{nom}
\vedette{cii}\homonyme{1}
\sens{1}
\région{GOs PA BO}
\end{entrée}

\begin{entrée}
{peau (humain); enveloppe corporelle}
\vedette{cii-phagòò}
\région{GOs PA}
\end{entrée}

\begin{entrée}
{pénis (grossier) ; sexe (de l'homme)}
\vedette{cò}\homonyme{2}
\région{GOs BO}
\end{entrée}

\begin{entrée}
{péritoine}
\vedette{bumira}
\région{PA BO}
\end{entrée}

\begin{entrée}
{phalange (doigt)}
\classe{nom}
\vedette{phajoo}
\sens{2}
\région{GOs PA BO}
\end{entrée}

\begin{entrée}
{pied ; jambe}
\classe{nom}
\vedette{kò}\homonyme{3}
\sens{1}
\région{GOs BO PA}
\end{entrée}

\begin{entrée}
{placenta}
\vedette{mõ-ẽnõ}
\région{GOs}
\variante{%
\vedette{mõ-ènõ}
\région{BO}}
\end{entrée}

\begin{entrée}
{plante de pied}
\classe{nom}
\vedette{ala-kò}
\sens{1}
\région{GOs BO}
\variante{%
\vedette{ala-xò}
\région{GO(s)}}
\variante{%
\vedette{ala-kò}
\région{PA}}
\end{entrée}

\begin{entrée}
{plume}
\vedette{pu}\homonyme{1}
\région{GOs PA BO}
\variante{%
\vedette{phu}}
\end{entrée}

\begin{entrée}
{poignet}
\vedette{böji-hi}
\région{GOs}
\end{entrée}

\begin{entrée}
{poignet}
\vedette{nõõ-hi}
\région{GOs BO PA}
\end{entrée}

\begin{entrée}
{poil}
\vedette{pu}\homonyme{1}
\région{GOs PA BO}
\variante{%
\vedette{phu}}
\end{entrée}

\begin{entrée}
{poing}
\vedette{jamo-hi}
\région{PA}
\end{entrée}

\begin{entrée}
{poing}
\vedette{kemõ-hi}
\région{GOs}
\end{entrée}

\begin{entrée}
{poitrine}
\vedette{wãge}
\région{GOs BO PA}
\end{entrée}

\begin{entrée}
{poumon}
\classe{nom}
\vedette{phaa}
\sens{1}
\région{GOs BO PA}
\end{entrée}

\begin{entrée}
{poumon (lit. contenant-respiration)}
\vedette{mõ-cãna}
\région{GOs}
\variante{%
\vedette{mõ-cana}
\région{BO}}
\end{entrée}

\begin{entrée}
{poumons}
\vedette{kòmaze}
\région{GOs}
\end{entrée}

\begin{entrée}
{pubis ; partie antérieure des os de la hanche}
\vedette{bwa-kazi-}
\région{GO}
\variante{%
\vedette{kaji-n}
\région{BO (Corne)}}
\end{entrée}

\begin{entrée}
{rein}
\classe{nom}
\vedette{pû-kai-ã}
\région{GOs}
\end{entrée}

\begin{entrée}
{rein ; rognon}
\vedette{kuii}
\région{GOs}
\end{entrée}

\begin{entrée}
{rotule (lit. carapace de 'savonnette')}
\vedette{pii-peçi}
\région{GOs}
\variante{%
\vedette{pi-peyi, pi-peji}
\région{PA}}
\end{entrée}

\begin{entrée}
{sang}
\vedette{kutra}
\groupe{A}
\région{GOs}
\variante{%
\vedette{kura}
\région{BO PA}}
\end{entrée}

\begin{entrée}
{sein}
\vedette{thi}\homonyme{1}
\région{GOs BO PA}
\end{entrée}

\begin{entrée}
{sexe (de l'homme) ; pénis}
\vedette{phi}\homonyme{1}
\région{GOs}
\variante{%
\vedette{phii-n}
\région{PA BO}}
\end{entrée}

\begin{entrée}
{sexe (femme) ; vulve; vagin}
\vedette{cêê}
\région{GOs}
\variante{%
\vedette{cê}
\région{BO}}
\end{entrée}

\begin{entrée}
{sourcils}
\vedette{pu-bwakira-me}
\région{GOs}
\variante{%
\vedette{pu-bwakitra-me}}
\end{entrée}

\begin{entrée}
{sperme}
\vedette{we-co}
\région{GOs}
\end{entrée}

\begin{entrée}
{sperme (Dubois)}
\classe{nom}
\vedette{noyo}
\région{BO}
\end{entrée}

\begin{entrée}
{sternum}
\vedette{phwè-gòògoni}
\région{PA}
\end{entrée}

\begin{entrée}
{taille (lit. milieu)}
\classe{nom}
\vedette{gòò}
\groupe{A}
\sens{3}
\région{GOs PA BO}
\end{entrée}

\begin{entrée}
{talon}
\vedette{bwèèvaça kò}
\région{GOs}
\variante{%
\vedette{bwèèvao kò-ã}
\région{PA}}
\variante{%
\vedette{bwèèva kò}
\région{BO WEM}}
\end{entrée}

\begin{entrée}
{talon (lit. arrière du pied)}
\vedette{kaça-kò}
\région{GOs}
\end{entrée}

\begin{entrée}
{tempe}
\vedette{jiò}
\région{GOs}
\variante{%
\vedette{jòò-n}
\région{BO [Corne]}}
\end{entrée}

\begin{entrée}
{tendon}
\classe{nom}
\vedette{wa}\homonyme{1}
\sens{2}
\région{GOs}
\variante{%
\vedette{wal}
\région{PA BO}}
\variante{%
\vedette{wòl}
\région{WEM}}
\end{entrée}

\begin{entrée}
{tendon d'achille (lit. le tendon du chef)}
\vedette{wa-aazo}
\région{GOs PA}
\variante{%
\vedette{wa-aazo ni kò}
\région{PA WEM}}
\variante{%
\vedette{wa-ayo}
\région{BO}}
\end{entrée}

\begin{entrée}
{tentacule}
\classe{nom}
\vedette{hii}
\sens{1}
\région{GOs BO PA}
\variante{%
\vedette{yi-n}}
\end{entrée}

\begin{entrée}
{testicule}
\classe{nom}
\vedette{pi}\homonyme{1}
\groupe{A}
\sens{3}
\région{GOs PA BO WE WEM GA}
\end{entrée}

\begin{entrée}
{testicules}
\vedette{hê-pii-n}
\région{BO}
\end{entrée}

\begin{entrée}
{testicules}
\vedette{kô-pii}
\région{GOs PA WE}
\variante{%
\vedette{pii}
\région{BO}}
\end{entrée}

\begin{entrée}
{tête}
\vedette{bwa}
\sens{1}
\région{GOs BO}
\end{entrée}

\begin{entrée}
{tête de ;}
\classe{n (composition)}
\vedette{bwe-}
\sens{1}
\région{GOs BO PA}
\end{entrée}

\begin{entrée}
{têton (du sein)}
\vedette{mè-thi}
\région{GOs BO PA}
\end{entrée}

\begin{entrée}
{thorax ; cage thoracique ; côtes}
\vedette{phalawe}\homonyme{1}
\région{GOs}
\end{entrée}

\begin{entrée}
{tibia}
\vedette{du-kò}
\région{GOs}
\end{entrée}

\begin{entrée}
{tibia (Dubois)}
\vedette{taye}
\région{BO}
\end{entrée}

\begin{entrée}
{trou dans la dentition}
\classe{nom}
\vedette{phwè-wado}
\sens{2}
\région{BO}
\end{entrée}

\begin{entrée}
{trou de l'oreille}
\vedette{phwè-keni}
\région{GOs}
\end{entrée}

\begin{entrée}
{utérus ;}
\vedette{mõ-ẽnõ}
\région{GOs}
\variante{%
\vedette{mõ-ènõ}
\région{BO}}
\end{entrée}

\begin{entrée}
{vagin}
\vedette{phwè-êgu}
\région{GOs}
\end{entrée}

\begin{entrée}
{veine ; artère}
\classe{nom}
\vedette{wa}\homonyme{1}
\sens{2}
\région{GOs}
\variante{%
\vedette{wal}
\région{PA BO}}
\variante{%
\vedette{wòl}
\région{WEM}}
\end{entrée}

\begin{entrée}
{ventre}
\vedette{kiò}
\région{GOs PA BO}
\end{entrée}

\begin{entrée}
{vésicule biliaire}
\vedette{azi}
\région{GOs}
\région{PA}
\variante{%
\vedette{kââli}}
\variante{%
\vedette{kâli}
\région{BO}}
\variante{%
\vedette{we-khâli}
\région{PA BO}}
\end{entrée}

\begin{entrée}
{vessie}
\vedette{mõ-ima}
\région{GOs WEM BO}
\end{entrée}

\begin{entrée}
{visage ; face}
\classe{nom}
\vedette{ala-me}
\sens{1}
\région{GOs PA BO}
\end{entrée}

\begin{entrée}
{vulve}
\vedette{kô-kabòn}
\région{PA}
\end{entrée}

\paragraph{Parties du corps humain : doigts, orteil}

\begin{entrée}
{annulaire}
\vedette{baaba}
\région{GOs BO PA}
\end{entrée}

\begin{entrée}
{auriculaire}
\vedette{thizii}
\région{GOs}
\variante{%
\vedette{thiri}
\région{PA}}
\variante{%
\vedette{tiri}
\région{BO WEM}}
\end{entrée}

\begin{entrée}
{doigt}
\vedette{bwèdò}
\région{GOs PA}
\end{entrée}

\begin{entrée}
{index (main)}
\vedette{êmwê}\homonyme{2}
\région{GOs BO}
\variante{%
\vedette{êmwèn}
\région{PA}}
\end{entrée}

\begin{entrée}
{majeur (doigt)}
\vedette{kôa}
\région{GOs PA BO}
\variante{%
\vedette{kôya, koeza}
\région{GO(s)}}
\end{entrée}

\begin{entrée}
{orteil}
\vedette{bwèdò-kò}
\région{GOs PA BO}
\variante{%
\vedette{bwèdò-xò}
\région{GO(s) PA BO}}
\end{entrée}

\begin{entrée}
{pouce}
\vedette{bwèdo-hi thoomwa}\homonyme{1}
\région{GOs PA}
\variante{%
\vedette{bwèdo-hi thooma}
\région{BO}}
\end{entrée}

\subsubsection{Corps animal}

\begin{entrée}
{aile}
\vedette{hava-hi}
\région{GOs}
\variante{%
\vedette{hava-hi-n}
\région{BO PA}}
\end{entrée}

\begin{entrée}
{bec (plat de canard)}
\vedette{mee-phwa}
\région{GOs}
\variante{%
\vedette{dò-phwa-n}
\région{PA}}
\end{entrée}

\begin{entrée}
{carapace ; écaille de tortue}
\vedette{pii}\homonyme{4}
\région{GOs PA BO}
\end{entrée}

\begin{entrée}
{corne}
\vedette{bwili}
\région{GOs}
\variante{%
\vedette{dixo-bwa-n}
\région{PA}}
\end{entrée}

\begin{entrée}
{cornes}
\classe{nom}
\vedette{dixoo}
\région{GOs BO}
\end{entrée}

\begin{entrée}
{cornes [WEM BO]}
\classe{nom}
\vedette{digo}
\sens{2}
\région{GOs WEM WE BO}
\end{entrée}

\begin{entrée}
{crête ; aigrette}
\classe{nom}
\vedette{throo}
\sens{1}
\région{GOs WEM}
\variante{%
\vedette{thoo}
\région{PA BO}}
\end{entrée}

\begin{entrée}
{crête de coq}
\vedette{dimwãã ko}
\région{GOs}
\variante{%
\vedette{dimwãã diwe-ko}
\région{BO PA}}
\end{entrée}

\begin{entrée}
{dard de la raie (lit. sagaie de la raie)}
\vedette{doo-pe}
\région{GOs PA}
\end{entrée}

\begin{entrée}
{nageoire}
\vedette{jutri}
\région{GOs}
\end{entrée}

\begin{entrée}
{poils de roussette}
\vedette{pu-bwò}
\région{GOs}
\end{entrée}

\begin{entrée}
{queue (oiseau, animal)}
\vedette{thringã}
\région{GOs}
\variante{%
\vedette{thrixã}
\région{GO(s)}}
\variante{%
\vedette{thingã}
\région{BO}}
\end{entrée}

\begin{entrée}
{queue (poisson)}
\vedette{cibò}
\région{GOs}
\end{entrée}

\begin{entrée}
{touffe de poils de roussette et fil de coton (Dubois ms)}
\vedette{thii}\homonyme{4}
\région{GO PA WEM}
\end{entrée}

\subsection{Fonctions naturelles}

\subsubsection{Fonctions naturelles humaines}

\begin{entrée}
{accoucher ; enfanter}
\classe{v}
\vedette{pwe}\homonyme{2}
\sens{1}
\région{GOs}
\end{entrée}

\begin{entrée}
{accoupler (s') ; avoir des relations sexuelles}
\vedette{thòò}
\région{GOs BO}
\end{entrée}

\begin{entrée}
{accoupler (s') ; faire l'amour}
\vedette{pe-jölö}
\région{GOs}
\end{entrée}

\begin{entrée}
{agiter (s') en dormant}
\vedette{kô-ii}
\région{GOs}
\end{entrée}

\begin{entrée}
{aspirer (liquide avec une paille)}
\vedette{thivwi}
\région{GOs BO}
\variante{%
\vedette{thipi}
\région{GO(s)}}
\end{entrée}

\begin{entrée}
{avaler}
\vedette{nòme}\homonyme{1}
\région{GOs}
\variante{%
\vedette{nòme}
\région{PA BO}}
\end{entrée}

\begin{entrée}
{avaler sans mâcher}
\vedette{nòmo}
\région{PA}
\end{entrée}

\begin{entrée}
{bâiller}
\vedette{ohaim}
\région{PA}
\variante{%
\vedette{ohahèm}
\région{BO [Corne]}}
\end{entrée}

\begin{entrée}
{bâiller; éructer (?)}
\vedette{hòwala}
\région{GOs}
\variante{%
\vedette{hòpwala}
\région{GO(s)}}
\variante{%
\vedette{oala}
\région{WE}}
\end{entrée}

\begin{entrée}
{bave de mort}
\vedette{thravayu}
\région{GOs}
\variante{%
\vedette{thayu}
\région{GO(s)}}
\end{entrée}

\begin{entrée}
{bave de mort (lit. nourriture des morts)}
\classe{nom}
\vedette{ca-ma}
\sens{1}
\région{PA BO}
\end{entrée}

\begin{entrée}
{chanter (personne, oiseau)}
\classe{v ; n}
\vedette{waal}
\sens{2}
\région{PA BO}
\end{entrée}

\begin{entrée}
{chaud (avoir)}
\classe{v.stat.}
\vedette{tòò}\homonyme{1}
\sens{1}
\région{GOs PABO}
\end{entrée}

\begin{entrée}
{crachat}
\vedette{we-zume}
\région{GOs}
\end{entrée}

\begin{entrée}
{crachat (de grippe) [Corne]}
\vedette{paxa-nõ-pu}
\région{BO}
\end{entrée}

\begin{entrée}
{cracher}
\vedette{phwee}
\région{GOs}
\end{entrée}

\begin{entrée}
{cracher}
\vedette{zee}
\région{GOs TRE}
\end{entrée}

\begin{entrée}
{cracher}
\vedette{zume}
\région{GOs}
\variante{%
\vedette{zhume}
\région{GO(s)}}
\variante{%
\vedette{zome}
\région{PA}}
\variante{%
\vedette{zume-n, yume-n}
\région{BO}}
\end{entrée}

\begin{entrée}
{crampe (avoir une)}
\vedette{ńhõbe}
\région{GOs}
\end{entrée}

\begin{entrée}
{crampe (avoir une)}
\vedette{nhei}
\région{GOs GA}
\variante{%
\vedette{nhei}
\région{PA}}
\end{entrée}

\begin{entrée}
{crotte ; excréments}
\vedette{nhã}
\région{GOs WEM WE BO PA}
\end{entrée}

\begin{entrée}
{crotte (forme déterminé)}
\vedette{ńhõ-}
\région{GOs PA BO}
\end{entrée}

\begin{entrée}
{déféquer}
\vedette{phòò}
\région{GOs}
\variante{%
\vedette{phòòl}
\région{BO PA}}
\variante{%
\vedette{pòòl, pwòl}
\région{BO}}
\end{entrée}

\begin{entrée}
{dormir auprès du feu (la nuit)}
\vedette{phaa-cebwo}
\région{GOs}
\variante{%
\vedette{pu-cibwo}
\région{GO(s)}}
\end{entrée}

\begin{entrée}
{dormir ; couché (être) ; allongé (être)}
\vedette{mããni}
\sens{1}
\région{GOs WEM BO PA}
\variante{%
\vedette{mãni}
\région{PA}}
\end{entrée}

\begin{entrée}
{dormir sur le côté ; couché sur le côté}
\vedette{kô-alaxe}
\région{GOs}
\end{entrée}

\begin{entrée}
{dormir trop (faire la grasse matinée)}
\vedette{mããni-mhã}
\région{PA}
\end{entrée}

\begin{entrée}
{écouter ; prêter l'oreille}
\vedette{phaxee}
\région{GOs}
\variante{%
\vedette{phaxeen, phakeen}
\région{BO PA}}
\end{entrée}

\begin{entrée}
{enceinte (être)}
\vedette{poxi}
\région{GOs}
\variante{%
\vedette{poki}
\région{PA}}
\end{entrée}

\begin{entrée}
{enceinte (être) (lit. po kiò 'petit ventre')}
\vedette{po-ki}
\région{PA BO}
\end{entrée}

\begin{entrée}
{entendre}
\vedette{trò}\homonyme{2}
\région{GOs}
\variante{%
\vedette{tò}
\région{BO}}
\end{entrée}

\begin{entrée}
{entendre ; sentir}
\classe{v.t.}
\vedette{trõne}
\sens{1}
\région{GOs}
\variante{%
\vedette{tõne}
\région{BO PA}}
\end{entrée}

\begin{entrée}
{épuisé}
\vedette{bwòvwô}
\région{GOs BO}
\variante{%
\vedette{bòòvwô, bòpô}
\région{GO(s)}}
\end{entrée}

\begin{entrée}
{épuisé ; épuisement}
\vedette{bwaare}
\région{WEM WE BO}
\variante{%
\vedette{bware}
\région{PA BO}}
\end{entrée}

\begin{entrée}
{éternuer}
\vedette{chiwe}
\région{GOs PA BO}
\end{entrée}

\begin{entrée}
{étrangler (s') ; étouffer (s')}
\vedette{kòròò}
\région{GOs}
\end{entrée}

\begin{entrée}
{éveiller (s') ; réveiller (se)}
\vedette{ńõ}\homonyme{1}
\sens{1}
\région{GOs}
\variante{%
\vedette{nòl}
\région{BO PA}}
\end{entrée}

\begin{entrée}
{expectorer}
\vedette{phwee}
\région{GOs}
\end{entrée}

\begin{entrée}
{expulser par la bouche}
\vedette{phwee}
\région{GOs}
\end{entrée}

\begin{entrée}
{faim (avoir)}
\vedette{alavwu}
\région{GOs WEM PA BO}
\variante{%
\vedette{alapu}
\région{vx}}
\end{entrée}

\begin{entrée}
{faire la sieste}
\vedette{kô-goon-a}
\région{GOs}
\end{entrée}

\begin{entrée}
{faire un cauchemar [BO]}
\vedette{mhãã}\homonyme{2}
\région{GOs}
\variante{%
\vedette{mhããng}
\région{BO}}
\end{entrée}

\begin{entrée}
{faire un petit somme, faire la sieste}
\classe{v}
\vedette{kô-kea}
\sens{2}
\région{GOs}
\variante{%
\vedette{kô-kea, kô-xea}
\région{PA BO}}
\end{entrée}

\begin{entrée}
{fatigué}
\classe{v}
\vedette{teele}
\région{PA}
\end{entrée}

\begin{entrée}
{fatigué (après une nuit courte)}
\vedette{mabu}
\région{GOs}
\end{entrée}

\begin{entrée}
{fatigué (être)}
\vedette{bwòvwô}
\région{GOs BO}
\variante{%
\vedette{bòòvwô, bòpô}
\région{GO(s)}}
\end{entrée}

\begin{entrée}
{fatigué; faible}
\vedette{bwaxe}
\région{PA BO}
\end{entrée}

\begin{entrée}
{fatigué ; fatigue (grande)}
\vedette{bwaare}
\région{WEM WE BO}
\variante{%
\vedette{bware}
\région{PA BO}}
\end{entrée}

\begin{entrée}
{fixer du regard ; dévisager}
\vedette{alobo}
\région{GOs PA}
\end{entrée}

\begin{entrée}
{froid (avoir)}
\vedette{khaabu}
\région{GOs BO}
\end{entrée}

\begin{entrée}
{froid ; fièvre}
\classe{v ; n}
\vedette{tuuçò}
\sens{1}
\région{GOs}
\variante{%
\vedette{tuujong tuuyòng}
\région{PA BO [BM]}}
\end{entrée}

\begin{entrée}
{gargouiller (ventre)}
\vedette{kôôxô}
\région{GOs}
\variante{%
\vedette{kôôhòl}
\région{PABO}}
\variante{%
\vedette{kôôl}
\région{BO}}
\end{entrée}

\begin{entrée}
{grossir ; grandir ; croître (plantes)}
\vedette{waara}
\région{GOs PA BO}
\variante{%
\vedette{waatra}
\région{GO(s)}}
\end{entrée}

\begin{entrée}
{haleine}
\classe{nom}
\vedette{dao}\homonyme{1}
\sens{2}
\région{GOs}
\région{PA BO}
\variante{%
\vedette{daòn}}
\end{entrée}

\begin{entrée}
{haleter}
\vedette{chãnã waa}
\région{PA}
\end{entrée}

\begin{entrée}
{hoquet ; hoquet (avoir le) ; hoqueter (en pleurs)}
\vedette{mwozi}
\région{GOs}
\end{entrée}

\begin{entrée}
{humer}
\vedette{böleõne bö}
\région{GOs BO}
\variante{%
\vedette{bööle}
\région{PA}}
\end{entrée}

\begin{entrée}
{lever (se)}
\classe{v}
\vedette{cabo}
\sens{1}
\région{GOs}
\variante{%
\vedette{cabwòl, cabòl}
\région{PA BO WEM}}
\end{entrée}

\begin{entrée}
{menstruations (avoir ses)}
\vedette{mha-mhwããnu}
\région{GOs}
\end{entrée}

\begin{entrée}
{menstruations (avoir ses)}
\vedette{tròòli mha-mhwããnu}
\région{GOs}
\end{entrée}

\begin{entrée}
{menstruations (avoir ses)}
\vedette{tròòli phwayuu}
\région{GOs}
\end{entrée}

\begin{entrée}
{morve}
\vedette{dimòm}
\région{BO [BM]}
\variante{%
\vedette{dimwò-n}
\région{BO}}
\end{entrée}

\begin{entrée}
{moucher (se)}
\vedette{nhi}\homonyme{3}
\région{GOs}
\variante{%
\vedette{nhil}
\région{PA BO}}
\end{entrée}

\begin{entrée}
{péter ; avoir des vents}
\vedette{tho nõ}
\région{GOs BO}
\end{entrée}

\begin{entrée}
{péter (grossier)}
\vedette{vhii}
\région{GOs BO}
\variante{%
\vedette{vii}
\région{GO(s) BO}}
\variante{%
\vedette{fhi}
\région{GO(s)}}
\end{entrée}

\begin{entrée}
{plaindre (se) ; gémir}
\vedette{hili}\homonyme{3}
\région{GOs BO}
\end{entrée}

\begin{entrée}
{pleurer ; gémir}
\vedette{gi}
\région{GO PA BO}
\end{entrée}

\begin{entrée}
{pleurnicher (bébé) ; sangloter}
\vedette{mozi}
\région{GOs}
\variante{%
\vedette{mhõril}
\région{PA}}
\end{entrée}

\begin{entrée}
{pleurnicher ; sangloter ; hoqueter}
\vedette{mhõril}
\région{PA BO}
\end{entrée}

\begin{entrée}
{pousser (en longueur et en largeur)}
\vedette{waara}
\région{GOs PA BO}
\variante{%
\vedette{waatra}
\région{GO(s)}}
\end{entrée}

\begin{entrée}
{rassasié}
\classe{v ; n}
\vedette{mõlò}
\sens{2}
\région{GOs}
\variante{%
\vedette{mòlò}
\région{PA BO}}
\variante{%
\vedette{mòòlè}
\région{BO}}
\end{entrée}

\begin{entrée}
{regarder}
\vedette{nõõ}\homonyme{2}
\région{GOs}
\variante{%
\vedette{nõõ}
\région{BO PA}}
\end{entrée}

\begin{entrée}
{regarder avec envie, avec admiration}
\vedette{no-maari}
\région{GOs}
\end{entrée}

\begin{entrée}
{regarder dans le noir}
\vedette{nôô-ba}
\région{PA}
\end{entrée}

\begin{entrée}
{regarder en bas (sans but)}
\vedette{pe-nò-du}
\région{GOs BO}
\end{entrée}

\begin{entrée}
{regarder furtivement}
\vedette{no phèńô}
\région{GOs}
\end{entrée}

\begin{entrée}
{regarder ; observer ; guetter}
\vedette{alö}
\région{GOs BO}
\variante{%
\vedette{alu}
\région{PA}}
\end{entrée}

\begin{entrée}
{regarder par dessus}
\vedette{no kaö}
\région{GOs}
\end{entrée}

\begin{entrée}
{regarder (se) (dans un miroir)}
\vedette{bwaxixi}
\région{GOs}
\variante{%
\vedette{bwaxii}}
\end{entrée}

\begin{entrée}
{regarder (se) (dans un miroir)}
\vedette{zido}
\région{GOs}
\variante{%
\vedette{zhido}
\région{GA}}
\end{entrée}

\begin{entrée}
{renâcler (cheval)}
\vedette{nhi}\homonyme{3}
\région{GOs}
\variante{%
\vedette{nhil}
\région{PA BO}}
\end{entrée}

\begin{entrée}
{renifler [PA]}
\vedette{nhi}\homonyme{3}
\région{GOs}
\variante{%
\vedette{nhil}
\région{PA BO}}
\end{entrée}

\begin{entrée}
{reposer (se)}
\vedette{tree-çãnã}
\région{GOs}
\variante{%
\vedette{teecãnã}
\région{WEM}}
\end{entrée}

\begin{entrée}
{reposer (se) ; repos}
\vedette{chãnã}\homonyme{1}
\sens{2}
\région{GOs WE}
\variante{%
\vedette{chãnã}
\région{PA BO}}
\end{entrée}

\begin{entrée}
{reprendre son souffle ; reprendre haleine}
\vedette{tree-çãnã}
\région{GOs}
\variante{%
\vedette{teecãnã}
\région{WEM}}
\end{entrée}

\begin{entrée}
{respiration}
\vedette{chãnã}\homonyme{1}
\sens{1}
\région{GOs WE}
\variante{%
\vedette{chãnã}
\région{PA BO}}
\end{entrée}

\begin{entrée}
{respirer}
\vedette{chãnã}\homonyme{1}
\sens{1}
\région{GOs WE}
\variante{%
\vedette{chãnã}
\région{PA BO}}
\end{entrée}

\begin{entrée}
{respirer la bouche ouverte}
\vedette{chãnã waa}
\région{PA}
\end{entrée}

\begin{entrée}
{rester\_éveillé}
\vedette{kô-nòòl}
\région{PA BO}
\end{entrée}

\begin{entrée}
{réveiller (en secouant, en retournant)}
\classe{v}
\vedette{hòvwi}
\sens{2}
\région{GOs}
\variante{%
\vedette{hòvi}
\région{BO PA}}
\end{entrée}

\begin{entrée}
{réveiller qqn (en secouant, en retournant)}
\classe{v}
\vedette{phwia, phwiça}
\sens{1}
\région{GOs PA WEM WE BO}
\variante{%
\vedette{phuça}
\région{GO(s)}}
\variante{%
\vedette{phuya}
\région{BO}}
\end{entrée}

\begin{entrée}
{réveiller (se)}
\classe{v}
\vedette{cabo}
\sens{1}
\région{GOs}
\variante{%
\vedette{cabwòl, cabòl}
\région{PA BO WEM}}
\end{entrée}

\begin{entrée}
{réveiller (se) tôt}
\vedette{kô-nõõli tree}
\région{GOs}
\variante{%
\vedette{kô-nõõli tèèn}
\région{PA}}
\end{entrée}

\begin{entrée}
{rêver}
\vedette{kô-nòò}
\région{GOs}
\variante{%
\vedette{kô-nòi}
\région{BO}}
\end{entrée}

\begin{entrée}
{rêve ; rêver}
\vedette{nòò}
\région{GOs}
\variante{%
\vedette{nòi-n}
\région{BO}}
\end{entrée}

\begin{entrée}
{rire (se)}
\vedette{pe-kiga}
\région{GOs BO}
\end{entrée}

\begin{entrée}
{rire; sourire}
\vedette{kiga}\homonyme{2}
\région{GOs WEM PA BO}
\end{entrée}

\begin{entrée}
{rire un peu}
\vedette{pò-kiga}
\région{GOs}
\end{entrée}

\begin{entrée}
{ronfler}
\vedette{phuvwu}
\région{GOs}
\variante{%
\vedette{phuul}
\région{BO PA}}
\end{entrée}

\begin{entrée}
{ronfler [BO]}
\classe{v}
\vedette{mû}\homonyme{2}
\sens{1}
\région{GOs BO}
\variante{%
\vedette{mûû}
\région{BO}}
\end{entrée}

\begin{entrée}
{rôter}
\vedette{kavwo}
\région{GOs}
\variante{%
\vedette{kavwong}
\région{PA BO}}
\end{entrée}

\begin{entrée}
{saigner}
\vedette{kutra}
\groupe{B}
\région{GOs}
\variante{%
\vedette{kura}
\région{BO PA}}
\end{entrée}

\begin{entrée}
{salive ; bave (lit. eau-bouche)}
\vedette{we-phwa}
\région{GOs}
\variante{%
\vedette{we-vwa}
\région{GO(s) BO PA}}
\end{entrée}

\begin{entrée}
{sang menstruel}
\vedette{phwayuu}
\région{GOs WEM WE BO}
\variante{%
\vedette{whayuu}
\région{GO(s)}}
\end{entrée}

\begin{entrée}
{scruter}
\vedette{phaxee}
\région{GOs}
\variante{%
\vedette{phaxeen, phakeen}
\région{BO PA}}
\end{entrée}

\begin{entrée}
{scruter ; regarder}
\vedette{nõõli}
\région{GOs}
\variante{%
\vedette{nõõli}
\région{BO PA}}
\end{entrée}

\begin{entrée}
{sentir}
\vedette{trò}\homonyme{2}
\région{GOs}
\variante{%
\vedette{tò}
\région{BO}}
\end{entrée}

\begin{entrée}
{sentir mauvais (odeur de chair)}
\vedette{bûkû}
\région{GOs}
\variante{%
\vedette{bûûwû}
\région{BO [BM]}}
\end{entrée}

\begin{entrée}
{sentir (odeur) ; odeur ; avoir une odeur}
\vedette{bo}
\région{GOs PA BO}
\variante{%
\vedette{bwo}
\région{GO(s) BO}}
\variante{%
\vedette{bon, bwon}
\région{BO}}
\end{entrée}

\begin{entrée}
{sentir (odeur ou toucher) ; entendre}
\classe{v.t.}
\vedette{trõne}
\sens{2}
\région{GOs}
\variante{%
\vedette{tõne}
\région{BO PA}}
\end{entrée}

\begin{entrée}
{sentir (une odeur)}
\vedette{trõne bö}
\région{GOs}
\variante{%
\vedette{tone bo-n}
\région{PA BO [Corne]}}
\end{entrée}

\begin{entrée}
{sentir (volontairement)}
\vedette{böleõne bö}
\région{GOs BO}
\variante{%
\vedette{bööle}
\région{PA}}
\end{entrée}

\begin{entrée}
{sobre}
\vedette{êgu}\homonyme{2}
\région{GO}
\end{entrée}

\begin{entrée}
{soif (avoir)}
\vedette{maalu}
\région{BO PA}
\end{entrée}

\begin{entrée}
{soif (avoir) (lit. vouloir boire)}
\vedette{a-vwö kudo}
\région{GOs}
\variante{%
\vedette{aapo kudo}
\région{GO(s) vx}}
\end{entrée}

\begin{entrée}
{sommeil}
\vedette{mããni}
\sens{1}
\région{GOs WEM BO PA}
\variante{%
\vedette{mãni}
\région{PA}}
\end{entrée}

\begin{entrée}
{somnambule ; agiter (s') en dormant}
\vedette{mhãã}\homonyme{2}
\région{GOs}
\variante{%
\vedette{mhããng}
\région{BO}}
\end{entrée}

\begin{entrée}
{souffle}
\vedette{chãnã}\homonyme{1}
\sens{1}
\région{GOs WE}
\variante{%
\vedette{chãnã}
\région{PA BO}}
\end{entrée}

\begin{entrée}
{souffler}
\vedette{tree-çãnã}
\région{GOs}
\variante{%
\vedette{teecãnã}
\région{WEM}}
\end{entrée}

\begin{entrée}
{sourire}
\vedette{hivwaje}
\région{GOs}
\end{entrée}

\begin{entrée}
{sucer (bonbon)}
\vedette{thivwi}
\région{GOs BO}
\variante{%
\vedette{thipi}
\région{GO(s)}}
\end{entrée}

\begin{entrée}
{suer ; transpirer (lit. très chaud)}
\vedette{hai-trirê}
\région{GOs}
\variante{%
\vedette{haitritrê}
\région{GO(s)}}
\end{entrée}

\begin{entrée}
{sueur [BO]}
\classe{v ; n}
\vedette{mõgu}
\sens{2}
\région{GOs BO}
\variante{%
\vedette{mwõgu}
\région{GO(s)}}
\end{entrée}

\begin{entrée}
{sueur ; transpiration}
\vedette{trirê}
\région{GOs}
\end{entrée}

\begin{entrée}
{suicider (se)}
\vedette{thaliang}
\région{PA}
\variante{%
\vedette{thaliwa}
\région{WEM}}
\variante{%
\vedette{thraliwa}
\région{GO(s)}}
\end{entrée}

\begin{entrée}
{têter ; sucer}
\vedette{kûxû}\homonyme{1}
\région{GOs}
\variante{%
\vedette{kûû}
\région{PA BO}}
\variante{%
\vedette{kûkû}
\région{BO}}
\end{entrée}

\begin{entrée}
{toucher (un bobo, une blessure)}
\vedette{civwi}
\région{GOs}
\end{entrée}

\begin{entrée}
{transpirer ; transpiration ; sueur}
\vedette{kinõ}
\région{PA BO}
\variante{%
\vedette{khinõ}
\région{WE}}
\end{entrée}

\begin{entrée}
{trembler (de peur, de froid, de colère)}
\vedette{jòjò}
\région{GOs}
\variante{%
\vedette{jòjòn}
\région{PA BO}}
\end{entrée}

\begin{entrée}
{urine}
\vedette{we imã}
\région{GOs BO PA}
\end{entrée}

\begin{entrée}
{uriner}
\vedette{imã}
\région{GOs BO}
\end{entrée}

\begin{entrée}
{vibrer ; réagir (à un bruit)}
\vedette{jòjò}
\région{GOs}
\variante{%
\vedette{jòjòn}
\région{PA BO}}
\end{entrée}

\begin{entrée}
{violer}
\vedette{thòò}
\région{GOs BO}
\end{entrée}

\begin{entrée}
{voir}
\vedette{no}
\région{GOs}
\région{PA BO}
\variante{%
\vedette{nòòl}}
\variante{%
\vedette{nòò}
\région{PA}}
\end{entrée}

\begin{entrée}
{voir mal [Corne]}
\vedette{nõ-wame}
\région{BO}
\end{entrée}

\begin{entrée}
{voir (registre respectueux pour le Grand Chef) [BM]}
\vedette{hããni}
\région{BO}
\end{entrée}

\begin{entrée}
{voix}
\classe{nom}
\vedette{gaa}\homonyme{2}
\sens{1}
\région{GOs BO}
\variante{%
\vedette{gee}
\région{BO}}
\variante{%
\vedette{gèèn}
\région{BO [Corne]}}
\end{entrée}

\begin{entrée}
{vomir ; vomissure}
\vedette{muga}
\région{GOs BO PA}
\end{entrée}

\subsubsection{Fonctions naturelles des animaux}

\begin{entrée}
{aboyer}
\vedette{havwò}
\région{GOs}
\variante{%
\vedette{hawòl}
\région{BO}}
\variante{%
\vedette{ha}
\région{GA}}
\end{entrée}

\begin{entrée}
{coïter (animaux)}
\vedette{kanang}
\région{PA}
\end{entrée}

\begin{entrée}
{couver (des oeufs)}
\vedette{khaa-êgo}
\région{GOs}
\variante{%
\vedette{khaa-pi}
\région{WE WEM}}
\variante{%
\vedette{khaa-vwi}
\région{GO(s)}}
\end{entrée}

\begin{entrée}
{couver des oeufs}
\vedette{taabwa pi}
\région{PA}
\end{entrée}

\begin{entrée}
{couver (oeufs)}
\classe{v}
\vedette{trabwa}
\sens{3}
\région{GOs}
\variante{%
\vedette{tabwa}
\région{BO PA}}
\end{entrée}

\begin{entrée}
{dépouille de mue}
\classe{nom}
\vedette{dou}\homonyme{2}
\sens{2}
\région{GOs PA}
\variante{%
\vedette{deü}
\région{BO [Corne]}}
\end{entrée}

\begin{entrée}
{en rut [Corne]}
\vedette{bèèxu}
\région{BO}
\end{entrée}

\begin{entrée}
{fumier ; crotte (animal) [Corne]}
\vedette{ôô}
\région{BO}
\end{entrée}

\begin{entrée}
{muer (lézard, serpent)}
\vedette{thou}
\région{GOs}
\variante{%
\vedette{theun}
\région{BO [Corne]}}
\end{entrée}

\begin{entrée}
{oeuf (avoir des)}
\classe{v}
\vedette{pi}\homonyme{1}
\groupe{B}
\sens{4}
\région{GOs PA BO WE WEM GA}
\end{entrée}

\begin{entrée}
{pondre}
\vedette{khaa-pi}
\région{GOs}
\variante{%
\vedette{khaa-vwi}}
\end{entrée}

\begin{entrée}
{pondre}
\vedette{thu êgo}
\région{GOs}
\end{entrée}

\begin{entrée}
{pondre (lit. faire la coquille)[Corne]}
\vedette{thu pi}\homonyme{3}
\région{PA BO}
\end{entrée}

\begin{entrée}
{sortir du cocon (papillon)}
\classe{v}
\vedette{gaò}
\sens{2}
\région{GOs PA}
\end{entrée}

\subsection{Santé, maladie, médecine}

\subsubsection{Santé, maladie}

\begin{entrée}
{allité}
\classe{v.stat.}
\vedette{hôno}
\région{GOs PA}
\end{entrée}

\begin{entrée}
{aveugle}
\vedette{bwi}
\région{GOsPA BO}
\end{entrée}

\begin{entrée}
{avorter}
\vedette{kõle}
\région{BO}
\end{entrée}

\begin{entrée}
{blessé}
\vedette{ênã}
\région{GOs BO}
\variante{%
\vedette{ènan}
\région{BO [BM]}}
\variante{%
\vedette{kaalu}
\région{GO(s)}}
\end{entrée}

\begin{entrée}
{blesser (se) (sur un objet piquant)}
\classe{v}
\vedette{kòòli}
\sens{1}
\région{GOs BO}
\end{entrée}

\begin{entrée}
{blessure}
\vedette{mhe-nõbo}
\région{GOs}
\variante{%
\vedette{mhe-nõvwo}
\région{GOs}}
\end{entrée}

\begin{entrée}
{boîter ; boiteux}
\vedette{kha-thi}
\région{GOs}
\end{entrée}

\begin{entrée}
{boiteux ; boîter [PA]}
\classe{v}
\vedette{kha-thixò}
\sens{2}
\région{GOs}
\variante{%
\vedette{kha-thixò}
\région{PA}}
\end{entrée}

\begin{entrée}
{borgne (lit. il a un seul oeil)}
\vedette{thixèè mee-je}
\région{GOs}
\end{entrée}

\begin{entrée}
{bosse}
\vedette{khibu}
\région{GOs BO}
\end{entrée}

\begin{entrée}
{bosse [Corne]}
\vedette{phwi-n}
\région{BO [Corne]}
\variante{%
\vedette{phui-n}
\région{BO}}
\end{entrée}

\begin{entrée}
{bourbouille}
\vedette{bèxè}
\région{GOs}
\end{entrée}

\begin{entrée}
{bourbouille}
\vedette{vhiliçô}
\région{GOs}
\end{entrée}

\begin{entrée}
{bouton ; acné}
\vedette{thi}\homonyme{2}
\région{GOs}
\end{entrée}

\begin{entrée}
{boutons sur la figure (avoir des)}
\vedette{thuu}\homonyme{1}
\région{GOs BO}
\end{entrée}

\begin{entrée}
{boutons (sur le corps)}
\vedette{thivwöloo}
\région{GOs BO}
\variante{%
\vedette{tivwoloo}
\région{BO}}
\end{entrée}

\begin{entrée}
{carie dentaire}
\classe{nom}
\vedette{phwè-wado}
\sens{1}
\région{BO}
\end{entrée}

\begin{entrée}
{champignon (sur la peau ; lit. qui mange les gens)}
\vedette{hu-ãgu}
\région{GOs}
\end{entrée}

\begin{entrée}
{chancre}
\vedette{thivwöloo}
\région{GOs BO}
\variante{%
\vedette{tivwoloo}
\région{BO}}
\end{entrée}

\begin{entrée}
{cicatrice}
\vedette{mhe-nõbo}
\région{GOs}
\variante{%
\vedette{mhe-nõvwo}
\région{GOs}}
\end{entrée}

\begin{entrée}
{cicatrice ; blessure}
\classe{nom}
\vedette{nõbo}
\sens{2}
\région{GOs}
\variante{%
\vedette{nõbo, nõbwo}
\région{WEM WE PA BO}}
\end{entrée}

\begin{entrée}
{conjonctivite}
\vedette{bwi}
\région{GOsPA BO}
\end{entrée}

\begin{entrée}
{constipé}
\vedette{ńhôã}
\région{GOs}
\end{entrée}

\begin{entrée}
{coupé (souffle)}
\vedette{bi}\homonyme{1}
\région{GOs}
\variante{%
\vedette{biny}
\région{PA BO WE}}
\end{entrée}

\begin{entrée}
{couvert de bobos, d'ulcères ; couvert de gale}
\classe{v.stat.}
\vedette{nhyatru}
\sens{2}
\région{GOs}
\variante{%
\vedette{nhyaru}
\région{GO(s)}}
\end{entrée}

\begin{entrée}
{cracher}
\vedette{phwee}
\région{GOs}
\end{entrée}

\begin{entrée}
{croûtes sur la tête des bébés}
\classe{v ; n}
\vedette{kubi}
\sens{3}
\région{GOs PA BO}
\end{entrée}

\begin{entrée}
{dégonflé}
\vedette{bi}\homonyme{1}
\région{GOs}
\variante{%
\vedette{biny}
\région{PA BO WE}}
\end{entrée}

\begin{entrée}
{démanger}
\classe{v}
\vedette{hu-vo}\homonyme{1}
\sens{4}
\région{PA}
\variante{%
\vedette{hu-po}
\région{PA}}
\end{entrée}

\begin{entrée}
{démanger[Corne, BM]}
\classe{v}
\vedette{maü}
\sens{1}
\région{BO}
\end{entrée}

\begin{entrée}
{démanger ; gratter ; piquant (sur la peau)}
\vedette{mãyo}
\région{GOs}
\variante{%
\vedette{maoe}
\région{BO [BM]}}
\variante{%
\vedette{maü}
\région{BO}}
\end{entrée}

\begin{entrée}
{dépérir}
\classe{v.stat.}
\vedette{kòladuu}
\région{GOs}
\variante{%
\vedette{kòladuun}
\région{BO}}
\end{entrée}

\begin{entrée}
{douleur ; faire mal}
\vedette{ki}\homonyme{3}
\région{GOs}
\variante{%
\vedette{kil}
\région{PA}}
\end{entrée}

\begin{entrée}
{dysenterie (avoir la) ; diarhée (avoir la)}
\vedette{kula we ni ki}
\région{GOs}
\end{entrée}

\begin{entrée}
{elephantiasis (lit. jambe qui enfle)}
\vedette{phu ko}
\end{entrée}

\begin{entrée}
{endolori [BO, Corne, Haudricourt]}
\vedette{pecabi}
\région{BO}
\end{entrée}

\begin{entrée}
{endurer}
\classe{v}
\vedette{cöńi}
\sens{1}
\région{GOs BO PA}
\variante{%
\vedette{cööni}
\région{PA}}
\end{entrée}

\begin{entrée}
{enflé}
\vedette{khibu me}
\région{GOs}
\end{entrée}

\begin{entrée}
{enflé (front) ; avoir une bosse sur le front}
\vedette{khibu bwèèdrò}
\région{GOs}
\end{entrée}

\begin{entrée}
{engourdi ; avoir des fourmis (dans les membres)}
\vedette{zaa}\homonyme{2}
\région{GOs PA}
\variante{%
\vedette{zhaa}
\région{GA}}
\end{entrée}

\begin{entrée}
{épilepsie}
\classe{nom}
\vedette{ca-ma}
\sens{2}
\région{PA BO}
\end{entrée}

\begin{entrée}
{épuisé ; éreinté}
\vedette{mora}
\région{BO PA}
\end{entrée}

\begin{entrée}
{évanoui [BO]}
\classe{v}
\vedette{mãxi}
\sens{3}
\région{GOs}
\variante{%
\vedette{mãxim}
\région{WEM WE}}
\variante{%
\vedette{mhãkim, mhãxim}
\région{BO}}
\variante{%
\vedette{mããxim}
\région{PA}}
\end{entrée}

\begin{entrée}
{évanouissement ; évanoui}
\vedette{pum-a mèè}
\région{GOs WEM WE BO PA}
\end{entrée}

\begin{entrée}
{expectorer}
\vedette{phwee}
\région{GOs}
\end{entrée}

\begin{entrée}
{expulser par la bouche}
\vedette{phwee}
\région{GOs}
\end{entrée}

\begin{entrée}
{faible ; fragile}
\classe{v.stat.}
\vedette{beloo}
\sens{1}
\région{GOs PA}
\end{entrée}

\begin{entrée}
{faire mal ; être douloureux}
\classe{v ; n}
\vedette{khinu}\homonyme{1}
\sens{1}
\région{GOs PA BO}
\end{entrée}

\begin{entrée}
{fort ; vigoureux ; résistant}
\classe{v.stat. ; n}
\vedette{cuxi}
\sens{1}
\région{PA BO}
\variante{%
\vedette{cuki}
\région{GO(s)}}
\variante{%
\vedette{cugi}
\région{BO}}
\end{entrée}

\begin{entrée}
{furoncle}
\vedette{wòzõõ}
\région{GOs}
\variante{%
\vedette{wolõ}
\région{PA}}
\end{entrée}

\begin{entrée}
{gale}
\vedette{thuu}\homonyme{1}
\région{GOs BO}
\end{entrée}

\begin{entrée}
{ganglion}
\vedette{khibu}
\région{GOs BO}
\end{entrée}

\begin{entrée}
{gonfler ; enfler (membre)}
\vedette{khibu}
\région{GOs BO}
\end{entrée}

\begin{entrée}
{gonfler ; enfler (membre, partie du corps)}
\classe{v.i.}
\vedette{phuu}
\sens{2}
\région{GOs PA BO}
\variante{%
\vedette{phuuxu}
\région{GO}}
\end{entrée}

\begin{entrée}
{gonflé (yeux)}
\vedette{khibu me}
\région{GOs}
\end{entrée}

\begin{entrée}
{gratteux (être)}
\vedette{zò}\homonyme{2}
\région{GOs}
\variante{%
\vedette{zhò}
\région{GO(s)}}
\variante{%
\vedette{zòn}
\région{PA}}
\end{entrée}

\begin{entrée}
{grossir [BM]}
\vedette{phuri}
\région{BO}
\end{entrée}

\begin{entrée}
{hernie}
\vedette{paçagò}
\région{GOs}
\variante{%
\vedette{payagòl}
\région{PA}}
\variante{%
\vedette{peyego}
\région{BO}}
\end{entrée}

\begin{entrée}
{infecté ; infecter (s')}
\vedette{guni}
\région{GOs}
\variante{%
\vedette{gunim}
\région{BO PA}}
\end{entrée}

\begin{entrée}
{intoxiqué par la Ciguaterra}
\classe{v ; n}
\vedette{zòn}
\sens{2}
\région{PA}
\variante{%
\vedette{yòn, yhòn}
\région{BO}}
\end{entrée}

\begin{entrée}
{jeûner ; jeûne}
\vedette{hôdò}
\région{GOs PA}
\variante{%
\vedette{hôde}
\région{BO}}
\end{entrée}

\begin{entrée}
{lèpre ; lépreux ; avoir la lèpre}
\vedette{katria}
\région{GOs}
\variante{%
\vedette{karia}
\région{GO(s)}}
\variante{%
\vedette{kathia, karia}
\région{PA}}
\variante{%
\vedette{katia}
\région{BO}}
\end{entrée}

\begin{entrée}
{loucher}
\vedette{gèa}
\groupe{A}
\région{GOs BO}
\end{entrée}

\begin{entrée}
{maigre}
\vedette{bi}\homonyme{1}
\région{GOs}
\variante{%
\vedette{biny}
\région{PA BO WE}}
\end{entrée}

\begin{entrée}
{maigre ; maigrir}
\classe{v.stat.}
\vedette{kòladuu}
\région{GOs}
\variante{%
\vedette{kòladuun}
\région{BO}}
\end{entrée}

\begin{entrée}
{malade (gravement)}
\classe{v.stat.}
\vedette{hôno}
\région{GOs PA}
\end{entrée}

\begin{entrée}
{malade ; maladie}
\classe{v ; n}
\vedette{khinu}\homonyme{1}
\sens{1}
\région{GOs PA BO}
\end{entrée}

\begin{entrée}
{maladie contagieuse (grippe, etc.)}
\vedette{paxaa}
\région{GOs PA}
\end{entrée}

\begin{entrée}
{maladie des féculents (avoir une) (plaques sur la tête)}
\vedette{thuu}\homonyme{1}
\région{GOs BO}
\end{entrée}

\begin{entrée}
{maladie ; malade}
\classe{v.stat. ; n}
\vedette{mã}\homonyme{1}
\sens{3}
\région{GOs PABO}
\variante{%
\vedette{mhã}}
\end{entrée}

\begin{entrée}
{manchot ; qui n'a qu'un seul bras}
\vedette{wè-xè hii-je}
\région{GOs}
\end{entrée}

\begin{entrée}
{migraine (avoir la) ; mal de tête (avoir un)}
\vedette{môgo}
\région{GOs}
\variante{%
\vedette{mogòn}
\région{BO PA}}
\variante{%
\vedette{mûgòn}
\région{BO}}
\end{entrée}

\begin{entrée}
{muguet (boutons sur la langues des bébés)}
\vedette{mèèni}
\région{GOs}
\end{entrée}

\begin{entrée}
{paralysé ; engourdi}
\classe{v.stat. ; n}
\vedette{mã}\homonyme{1}
\sens{2}
\région{GOs PABO}
\variante{%
\vedette{mhã}}
\end{entrée}

\begin{entrée}
{peler (peau)}
\vedette{halaò}
\région{GOs}
\end{entrée}

\begin{entrée}
{piquer ; démanger (comme une plaie sous l'effet de l'alcool) [BM]}
\vedette{tiiu}
\région{BO}
\end{entrée}

\begin{entrée}
{piquer (se)}
\classe{v}
\vedette{kòòli}
\sens{1}
\région{GOs BO}
\end{entrée}

\begin{entrée}
{plaie sur les pieds}
\vedette{thonga}
\région{GOs}
\end{entrée}

\begin{entrée}
{purulent (bobo)}
\classe{v.stat.}
\vedette{nhyal}
\sens{2}
\région{PA BO}
\end{entrée}

\begin{entrée}
{pus}
\vedette{thivwöloo}
\région{GOs BO}
\variante{%
\vedette{tivwoloo}
\région{BO}}
\end{entrée}

\begin{entrée}
{pustule}
\vedette{thivwöloo}
\région{GOs BO}
\variante{%
\vedette{tivwoloo}
\région{BO}}
\end{entrée}

\begin{entrée}
{qui n'a qu'une seule jambe}
\vedette{wè-xè kòò-je}
\région{GOs}
\end{entrée}

\begin{entrée}
{raide (être) ; courbaturé}
\classe{v.stat.}
\vedette{kulaçe}
\sens{1}
\région{GOs}
\variante{%
\vedette{kulaye}
\région{PA}}
\end{entrée}

\begin{entrée}
{réaction cutanée (avoir une) à qqch ; s'infecter avec la sève de l'acajou}
\vedette{böö}
\région{GOs}
\end{entrée}

\begin{entrée}
{rougeole}
\vedette{mexò}
\région{GOs}
\end{entrée}

\begin{entrée}
{souffrance}
\vedette{cöńi-vwo}
\région{GOs}
\end{entrée}

\begin{entrée}
{souffrir}
\classe{v}
\vedette{cöńi}
\sens{1}
\région{GOs BO PA}
\variante{%
\vedette{cööni}
\région{PA}}
\end{entrée}

\begin{entrée}
{stérile [BM]}
\classe{v.stat.}
\vedette{gèny}
\région{BO}
\end{entrée}

\begin{entrée}
{stérile (femme)}
\vedette{thôge}
\région{GOs}
\end{entrée}

\begin{entrée}
{tache décolorée (sur peau)}
\vedette{buubu}
\end{entrée}

\begin{entrée}
{torticolis (avoir le)}
\vedette{bwe-no}
\région{GOs}
\end{entrée}

\begin{entrée}
{tousser ; avoir la grippe}
\vedette{phu-nõgo}
\région{GOs}
\end{entrée}

\begin{entrée}
{tousser ; grippe[Corne]}
\vedette{phu}
\région{BO}
\end{entrée}

\begin{entrée}
{tuberculose ; tuberculeux}
\vedette{mã-wãge}
\région{GOs}
\end{entrée}

\begin{entrée}
{varicelle}
\vedette{mexò}
\région{GOs}
\end{entrée}

\begin{entrée}
{verrue}
\vedette{jua}
\région{GOs BO}
\end{entrée}

\begin{entrée}
{voile blanc de la pupille (maladie de l'oeil)}
\vedette{gèa}
\groupe{B}
\région{GOs BO}
\end{entrée}

\subsubsection{Remèdes, médecine}

\begin{entrée}
{appliquer (médicament) ; traiter}
\vedette{kia}
\région{PA BO [BM]}
\variante{%
\vedette{khia}
\région{BO [BM]}}
\end{entrée}

\begin{entrée}
{cueillir des feuilles et herbes (pour faire des médicaments)}
\vedette{kaal}
\région{PA}
\end{entrée}

\begin{entrée}
{cueillir des herbes magiques [BO]}
\classe{v.t.}
\vedette{phugi}
\sens{2}
\région{GOs PA BO}
\end{entrée}

\begin{entrée}
{docteur}
\vedette{taxaza}
\région{GOs}
\end{entrée}

\begin{entrée}
{guérir}
\classe{v}
\vedette{kããle}
\sens{1}
\région{GOs BO PA}
\end{entrée}

\begin{entrée}
{guérisseur (lit. origine du médicament)}
\vedette{phwe-zatri}
\région{GOs PA BO}
\variante{%
\vedette{pwe-yari}
\région{BO}}
\end{entrée}

\begin{entrée}
{médecin (celui qui soigne)}
\vedette{a-kããle}
\région{GOs PA}
\end{entrée}

\begin{entrée}
{médicament ; remède}
\vedette{kia}
\région{PA BO [BM]}
\variante{%
\vedette{khia}
\région{BO [BM]}}
\end{entrée}

\begin{entrée}
{remède ; médicaments}
\vedette{zatri}
\région{GOs}
\variante{%
\vedette{zari}
\région{GO(s) PA}}
\variante{%
\vedette{zhari}
\région{GA}}
\variante{%
\vedette{yari}
\région{BO}}
\end{entrée}

\begin{entrée}
{soigner ; prendre soin de}
\classe{v}
\vedette{kããle}
\sens{1}
\région{GOs BO PA}
\end{entrée}

\begin{entrée}
{souffler des feuilles médicinales (pour guérir)}
\classe{v}
\vedette{ui}\homonyme{1}
\sens{2}
\région{GOs PA BO}
\end{entrée}

\subsection{Vêtements, parure, soins du corps}

\subsubsection{Vêtements, parure}

\begin{entrée}
{affaires (vêtements de qqn)}
\vedette{puçiu}
\région{GOs}
\end{entrée}

\begin{entrée}
{aigrette (coiffure)}
\vedette{thoo}\homonyme{2}
\région{PA BO}
\variante{%
\vedette{throo}
\région{GO(s)}}
\end{entrée}

\begin{entrée}
{apprêter (s'); préparer (se) (corps : habits et maquillage)}
\vedette{thu mwêêxa}
\région{GOs}
\end{entrée}

\begin{entrée}
{bagayou ; étui pénien}
\classe{nom}
\vedette{thivwaa}
\sens{1}
\région{GOs BO}
\end{entrée}

\begin{entrée}
{bague}
\vedette{pozo}
\région{GOs}
\end{entrée}

\begin{entrée}
{balassor}
\vedette{zii}
\région{GOs PA BO}
\end{entrée}

\begin{entrée}
{bandeau ; turban}
\vedette{wa-bwèèdrò}
\région{GOs}
\end{entrée}

\begin{entrée}
{biens personnels d'un défunt qu'on remet à ses maternels}
\vedette{puçiu}
\région{GOs}
\end{entrée}

\begin{entrée}
{boucle d'oreille}
\vedette{hê-kênii}
\région{GOs}
\end{entrée}

\begin{entrée}
{boucle d'oreilles (lit. fruit des oreilles)}
\vedette{pò-kênii}
\end{entrée}

\begin{entrée}
{bracelet (formé d'un seul coquillage taillé ; Dubois ms)}
\vedette{bui}
\région{BO PA}
\end{entrée}

\begin{entrée}
{bracelet (lit. lien-bras)}
\vedette{wa-hi}
\région{GOs PA BO}
\variante{%
\vedette{wa-yi}
\région{BO}}
\end{entrée}

\begin{entrée}
{brassard}
\vedette{wa-hi}
\région{GOs PA BO}
\variante{%
\vedette{wa-yi}
\région{BO}}
\end{entrée}

\begin{entrée}
{ceinture}
\vedette{wa-kiò}
\région{GOs PA BO}
\variante{%
\vedette{waki, wa-kiò}
\région{PA}}
\end{entrée}

\begin{entrée}
{ceinture}
\vedette{waramã}
\région{GOs}
\variante{%
\vedette{wara}
\région{GO(s)}}
\end{entrée}

\begin{entrée}
{chapeau}
\vedette{mwêê}
\région{GOs}
\variante{%
\vedette{mwêêng}
\région{PA BO}}
\end{entrée}

\begin{entrée}
{chaussure}
\classe{nom}
\vedette{ala-kò}
\sens{2}
\région{GOs BO}
\variante{%
\vedette{ala-xò}
\région{GO(s)}}
\variante{%
\vedette{ala-kò}
\région{PA}}
\end{entrée}

\begin{entrée}
{chemise}
\vedette{cimic}
\région{PA}
\end{entrée}

\begin{entrée}
{chemise}
\vedette{simi}
\région{GOs}
\variante{%
\vedette{cimic}
\région{PA}}
\end{entrée}

\begin{entrée}
{coiffe}
\vedette{mwêê}
\région{GOs}
\variante{%
\vedette{mwêêng}
\région{PA BO}}
\end{entrée}

\begin{entrée}
{coiffure en sparterie ; turban (des anciens) (Dubois ms)}
\vedette{phau}
\région{GO}
\variante{%
\vedette{paup}
\région{PA BO}}
\end{entrée}

\begin{entrée}
{coiffure (tout type)}
\vedette{hau}\homonyme{1}
\région{PA}
\end{entrée}

\begin{entrée}
{collier}
\vedette{pi-nõ}
\région{GOs PA BO}
\end{entrée}

\begin{entrée}
{collier (de jade) ; pendentif [Corne]}
\vedette{majiwe}
\région{BO}
\end{entrée}

\begin{entrée}
{couverture (pour dormir)}
\vedette{murò}
\région{GOs}
\end{entrée}

\begin{entrée}
{enfiler (vêtement)}
\vedette{thai}
\région{GOs}
\end{entrée}

\begin{entrée}
{enlever (chapeau)}
\classe{v}
\vedette{phu}\homonyme{3}
\sens{1}
\région{GOs}
\variante{%
\vedette{phuxa}
\région{PA}}
\end{entrée}

\begin{entrée}
{enlever (en général, vêtement, etc.)}
\classe{v}
\vedette{udi}
\sens{1}
\région{GOs BO}
\end{entrée}

\begin{entrée}
{étoffe d'écorce de banian}
\vedette{we bumi}
\région{BO [Corne]}
\end{entrée}

\begin{entrée}
{étoffe d'écorce de banian ;}
\vedette{zii}
\région{GOs PA BO}
\end{entrée}

\begin{entrée}
{étoffe d'écorce de banian (écorce des racines ; Dubois ms)}
\vedette{wène}
\région{BO PA}
\end{entrée}

\begin{entrée}
{étoffe ; tissu ; vêtement (lit. peau du diable)}
\vedette{ci-kãbwa}
\région{GOs}
\variante{%
\vedette{ci-xãbwa}
\région{GO(s)}}
\end{entrée}

\begin{entrée}
{habiller (s') ; vêtir (se)}
\vedette{thu-tãî}
\région{PA}
\end{entrée}

\begin{entrée}
{jupe de femme (petite) formée de 'wepooe' et de 'pobil' (Dubois ms) ;}
\vedette{mãdra}
\région{GOs}
\end{entrée}

\begin{entrée}
{jupe ; jupon ; manou (femme)}
\vedette{kii}\homonyme{1}
\région{GOs PA BO}
\variante{%
\vedette{kivi}
\région{BO}}
\end{entrée}

\begin{entrée}
{jupon}
\vedette{mãdra bwabu}
\région{GOs}
\end{entrée}

\begin{entrée}
{linge ; tissu}
\vedette{hõbwò}
\région{GOs}
\variante{%
\vedette{hõbò}
\région{GO(s)}}
\variante{%
\vedette{hãbwòn}
\région{PA BO}}
\variante{%
\vedette{hõbwòn}
\région{BO}}
\end{entrée}

\begin{entrée}
{manche}
\vedette{hi-hõbwo}
\région{GOs}
\end{entrée}

\begin{entrée}
{manou (hommes)}
\vedette{putruna}
\région{GOs}
\end{entrée}

\begin{entrée}
{manou ; pagne (des hommes)}
\vedette{mãdra}
\région{GOs}
\end{entrée}

\begin{entrée}
{mettre à l'envers (vêtements)}
\vedette{pa-kamaze}
\région{GOs}
\end{entrée}

\begin{entrée}
{mettre (chapeau)}
\vedette{khia}\homonyme{1}
\région{PA BO}
\end{entrée}

\begin{entrée}
{mettre ; enfiler (vêtement)}
\vedette{thai hõbò}
\région{GOs}
\end{entrée}

\begin{entrée}
{mettre (un vêtement) (lit. monter dedans)}
\vedette{udale}
\région{GOs PA}
\end{entrée}

\begin{entrée}
{morceau d'étoffe (fait avec la racine du banian) ;}
\vedette{mãdra}
\région{GOs}
\end{entrée}

\begin{entrée}
{morceau d'étoffe (plus fine que "mada")}
\classe{nom}
\vedette{hi}\homonyme{2}
\région{GOs BO}
\end{entrée}

\begin{entrée}
{mouchoir}
\vedette{muswa}
\région{GOs}
\end{entrée}

\begin{entrée}
{nu}
\vedette{a-hõ}
\région{GOs}
\variante{%
\vedette{a-õ}
\région{GO(s) BO}}
\end{entrée}

\begin{entrée}
{nu (être)}
\classe{v}
\vedette{paxu}
\sens{1}
\région{GOs PA}
\end{entrée}

\begin{entrée}
{nu (être) (lit. montrer)}
\vedette{pa-hinõ}
\région{GOs PA}
\end{entrée}

\begin{entrée}
{ôter}
\classe{v}
\vedette{udi}
\sens{1}
\région{GOs BO}
\end{entrée}

\begin{entrée}
{pantalon}
\vedette{pazalõ}
\end{entrée}

\begin{entrée}
{parer (se); se vêtir}
\vedette{thu mwêêxa}
\région{GOs}
\end{entrée}

\begin{entrée}
{parure d'oreille}
\vedette{hê-kênii}
\région{GOs}
\end{entrée}

\begin{entrée}
{pendentif}
\vedette{pi-nõ}
\région{GOs PA BO}
\end{entrée}

\begin{entrée}
{plume ou fleur plantée sur le sommet de la tête}
\vedette{thoo}\homonyme{2}
\région{PA BO}
\variante{%
\vedette{throo}
\région{GO(s)}}
\end{entrée}

\begin{entrée}
{plumet (dans les cheveux ou la coiffure)}
\classe{nom}
\vedette{throo}
\sens{2}
\région{GOs WEM}
\variante{%
\vedette{thoo}
\région{PA BO}}
\end{entrée}

\begin{entrée}
{poche de pantalon}
\vedette{kee-pazalô}
\région{GOs}
\end{entrée}

\begin{entrée}
{robe}
\vedette{hõbwo-tralago}
\région{GOs}
\end{entrée}

\begin{entrée}
{robe popinée}
\vedette{hõbwo-ko}
\région{GOs}
\end{entrée}

\begin{entrée}
{rouleau d'étoffe}
\vedette{maû}\homonyme{2}
\région{GOs PA}
\end{entrée}

\begin{entrée}
{turban}
\vedette{bwadreo}
\région{GOs}
\variante{%
\vedette{bwadeo}
\région{BO (Corne)}}
\end{entrée}

\begin{entrée}
{vêtements}
\vedette{hõbwò}
\région{GOs}
\variante{%
\vedette{hõbò}
\région{GO(s)}}
\variante{%
\vedette{hãbwòn}
\région{PA BO}}
\variante{%
\vedette{hõbwòn}
\région{BO}}
\end{entrée}

\begin{entrée}
{vêtements}
\classe{nom}
\vedette{tãî}
\sens{1}
\région{PA BO}
\end{entrée}

\begin{entrée}
{vêtir ; habiller qqn}
\vedette{phaa-udale}
\région{GOs}
\end{entrée}

\begin{entrée}
{vêtir (se) ; habiller (s') ; apprêter (s')}
\vedette{thu mwêê}
\région{GOs BO}
\end{entrée}

\begin{entrée}
{vêtir (se) ; habiller (s') (plutôt les vêtements du bas et chaussures)}
\vedette{kòò-dale}
\région{PA BO [BM, Corne]}
\end{entrée}

\begin{entrée}
{visière (faite d'une feuille ; lit. ombre de l'oeil)}
\vedette{hênuã-me}
\région{GOs}
\end{entrée}

\subsubsection{Soins du corps}

\begin{entrée}
{baigner (enfant)}
\vedette{phaa-butrõ}
\région{GOs}
\variante{%
\vedette{pa-burõ}
\région{GO(s)}}
\end{entrée}

\begin{entrée}
{baigner (se) ; laver (se)}
\vedette{butrõ}
\région{GOs}
\variante{%
\vedette{burõ}
\région{GO(s)}}
\variante{%
\vedette{buròm}
\région{WEM PA BO}}
\end{entrée}

\begin{entrée}
{chauffer assis au soleil (se)}
\vedette{tre-xinãã}
\région{GOs}
\variante{%
\vedette{tre-khini-a}
\région{GO(s)}}
\variante{%
\vedette{tee-khini-al}
\région{PA}}
\end{entrée}

\begin{entrée}
{coiffer (se)}
\vedette{thii-vwo}
\région{GOs}
\end{entrée}

\begin{entrée}
{conserver}
\classe{v}
\vedette{kããle}
\sens{3}
\région{GOs BO PA}
\end{entrée}

\begin{entrée}
{couper (par ex. cheveux, barbe,etc.avec des ciseaux) ;}
\vedette{còxe}
\région{GOs BO PA}
\variante{%
\vedette{còge}
\région{BO PA vx}}
\end{entrée}

\begin{entrée}
{curer (se) les dents}
\vedette{thi-parô}
\région{GOs}
\end{entrée}

\begin{entrée}
{enduire}
\vedette{thrîmi}
\région{GOs}
\variante{%
\vedette{thimi}
\région{PA BO}}
\end{entrée}

\begin{entrée}
{frictionner (avec des plantes ou des pommades)}
\vedette{urîni}
\région{PA}
\end{entrée}

\begin{entrée}
{frotter}
\vedette{urîni}
\région{PA}
\end{entrée}

\begin{entrée}
{laver (vaisselle, vêtement, cheveu)}
\vedette{jamwe}
\région{GOs WEM WE BO PA}
\end{entrée}

\begin{entrée}
{masser}
\classe{v}
\vedette{phwòli}
\sens{2}
\région{BO [BM, Corne]}
\end{entrée}

\begin{entrée}
{masser}
\vedette{urîni}
\région{PA}
\end{entrée}

\begin{entrée}
{nettoyer}
\vedette{thabwi}
\région{GOs}
\variante{%
\vedette{thaabwi}
\région{PA BO}}
\end{entrée}

\begin{entrée}
{peigne}
\vedette{dubila}
\région{GOs PA BO WEM WE}
\variante{%
\vedette{döbela, dubela}
\région{GA GO(s)}}
\end{entrée}

\begin{entrée}
{peigner ; peigner (se)}
\vedette{thii}\homonyme{2}
\région{GOs BO}
\end{entrée}

\begin{entrée}
{peigner (se)}
\vedette{thii-vwo}
\région{GOs}
\end{entrée}

\begin{entrée}
{peigner (se)}
\vedette{thi-pu}
\région{PA}
\end{entrée}

\begin{entrée}
{peindre}
\vedette{thrîmi}
\région{GOs}
\variante{%
\vedette{thimi}
\région{PA BO}}
\end{entrée}

\begin{entrée}
{propre ; neuf}
\vedette{maloom}
\région{WEM WE BO}
\variante{%
\vedette{malum}
\région{PA}}
\end{entrée}

\begin{entrée}
{raser}
\classe{v}
\vedette{thra}\homonyme{2}
\sens{1}
\région{GOs}
\variante{%
\vedette{tha, thaa}
\région{BO PA}}
\end{entrée}

\begin{entrée}
{raser (barbe)}
\vedette{pe-ravhi}
\région{BO PA}
\end{entrée}

\begin{entrée}
{raser (se)}
\vedette{pe-thra}
\région{GOs}
\variante{%
\vedette{pe-thaa}
\région{BO PA}}
\end{entrée}

\begin{entrée}
{rasoir}
\vedette{ba-pe-ravhi}
\région{PA}
\end{entrée}

\begin{entrée}
{rasoir}
\classe{nom}
\vedette{ba-pe-thra}
\région{GOs}
\end{entrée}

\begin{entrée}
{réparer}
\vedette{thabwi}
\région{GOs}
\variante{%
\vedette{thaabwi}
\région{PA BO}}
\end{entrée}

\begin{entrée}
{savon}
\vedette{chavwo}
\région{GOs}
\variante{%
\vedette{chapo}
\région{GO(s) vx}}
\variante{%
\vedette{cavo}
\région{PA}}
\variante{%
\vedette{caavu}
\région{BO (Corne)}}
\end{entrée}

\begin{entrée}
{sécher au soleil (se)}
\vedette{tre-xinãã}
\région{GOs}
\variante{%
\vedette{tre-khini-a}
\région{GO(s)}}
\variante{%
\vedette{tee-khini-al}
\région{PA}}
\end{entrée}

\begin{entrée}
{s'occuper de (enfant, qqn)}
\classe{v}
\vedette{pevwö}
\sens{2}
\région{GOs}
\end{entrée}

\begin{entrée}
{soigner}
\vedette{thabwi}
\région{GOs}
\variante{%
\vedette{thaabwi}
\région{PA BO}}
\end{entrée}

\begin{entrée}
{tailler (barbe) ;}
\vedette{còxe}
\région{GOs BO PA}
\variante{%
\vedette{còge}
\région{BO PA vx}}
\end{entrée}

\begin{entrée}
{tatouage (fait avec des pointes de feu, des côtes de feuilles de coco allumées)}
\vedette{ge-yai}
\région{GOs BO PA}
\end{entrée}

\begin{entrée}
{tatouer}
\vedette{gè-thaa}
\région{GOs}
\end{entrée}

\begin{entrée}
{teindre (cheveux)}
\vedette{thrîmi}
\région{GOs}
\variante{%
\vedette{thimi}
\région{PA BO}}
\end{entrée}

\begin{entrée}
{tondre (poils)}
\classe{v}
\vedette{thra}\homonyme{2}
\sens{1}
\région{GOs}
\variante{%
\vedette{tha, thaa}
\région{BO PA}}
\end{entrée}

\subsection{Positions, déplacements, mouvements, actions}

\subsubsection{Préfixes et verbes de position}

\begin{entrée}
{accouder (s')}
\vedette{ku-tibu}
\région{GOs}
\end{entrée}

\begin{entrée}
{accroché}
\classe{v.i.}
\vedette{côô}
\sens{1}
\région{GOs}
\variante{%
\vedette{cô}
\région{BO (BM]}}
\end{entrée}

\begin{entrée}
{accroché ; suspendu}
\classe{v}
\vedette{biçô}
\sens{2}
\région{GOs}
\end{entrée}

\begin{entrée}
{accroupi ; s'accroupir}
\vedette{tre-paxo}
\région{GOs}
\variante{%
\vedette{tee-vhaxol, tee-waxol}
\région{PA BO}}
\end{entrée}

\begin{entrée}
{adosser (s') ; adossé}
\vedette{ku-xea}
\région{GOs}
\variante{%
\vedette{ku-xia}
\région{GO(s)}}
\end{entrée}

\begin{entrée}
{agenouiller (s')}
\vedette{thi-bwagil}
\région{PA BO}
\end{entrée}

\begin{entrée}
{agenouiller (s') ; agenouillé}
\vedette{tre-thibu}
\région{GOs}
\variante{%
\vedette{tre-zibu}
\région{GO(s)}}
\end{entrée}

\begin{entrée}
{aligné ;}
\vedette{bavala}
\région{PA BO}
\end{entrée}

\begin{entrée}
{aligné ; côte à côte}
\vedette{ku-pe-bala}
\région{GOs}
\variante{%
\vedette{ku-vwe-bala}
\région{GO(s)}}
\end{entrée}

\begin{entrée}
{allongé en écoutant}
\vedette{kô-phaaxe}
\région{GOs}
\variante{%
\vedette{kô-phaaxen}
\région{PA BO}}
\end{entrée}

\begin{entrée}
{allongé les jambes écartées en l'air}
\vedette{kô-waga}
\région{GOs}
\end{entrée}

\begin{entrée}
{appuyer (s') [Corne]}
\vedette{thiibu}
\région{BO}
\end{entrée}

\begin{entrée}
{asseoir (s') ; assis}
\classe{v}
\vedette{trabwa}
\sens{1}
\région{GOs}
\variante{%
\vedette{tabwa}
\région{BO PA}}
\end{entrée}

\begin{entrée}
{asseoir (s') auprès du feu de bois pour se réchauffer}
\vedette{tree-kiyai}
\région{GOs}
\variante{%
\vedette{tre-xiyai}
\région{GOs}}
\variante{%
\vedette{tee-kiyai}
\région{PA}}
\end{entrée}

\begin{entrée}
{assis adossé à qqch}
\vedette{tre-kea}
\région{GOs}
\variante{%
\vedette{tre-xea}
\région{GO(s)}}
\end{entrée}

\begin{entrée}
{assis en cachette, à l'affût}
\vedette{tre-kuçaaxo}
\région{GOs}
\variante{%
\vedette{tee-kujaaxo}
\région{PA}}
\end{entrée}

\begin{entrée}
{assis en rêvant}
\vedette{tre-go}
\région{GOs}
\end{entrée}

\begin{entrée}
{assis en tailleur}
\vedette{tre-bwaalu}
\région{GOs PA}
\end{entrée}

\begin{entrée}
{assis en tenant qqch dans les bras}
\vedette{tre-e}
\région{GOs}
\end{entrée}

\begin{entrée}
{assis (être) face à face}
\vedette{pe-tre-alö}
\région{GOs}
\end{entrée}

\begin{entrée}
{assis (faire)}
\vedette{tree}\homonyme{3}
\région{GOs}
\variante{%
\vedette{tee}
\région{PA}}
\end{entrée}

\begin{entrée}
{assis les jambes allongées (lit. assis-étaler jambes-nos)}
\vedette{tre-pwalee kò-ã}
\région{GOs}
\end{entrée}

\begin{entrée}
{assis sans bouger}
\vedette{gu-traabwa}
\région{GOs}
\end{entrée}

\begin{entrée}
{assis sans parler (asseoir-muet)}
\vedette{tre-hû}
\région{GOs}
\end{entrée}

\begin{entrée}
{côte à côte}
\vedette{bavala}
\région{PA BO}
\end{entrée}

\begin{entrée}
{couché}
\vedette{kô-}\homonyme{1}
\région{GOs PA BO}
\end{entrée}

\begin{entrée}
{couché en tenant qqch dans les bras}
\vedette{kô-e}
\région{GOs BO PA}
\end{entrée}

\begin{entrée}
{couché (être) face à face}
\vedette{pe-kô-alö}
\région{GOs}
\end{entrée}

\begin{entrée}
{couché (la tête) vers la porte}
\vedette{kônõ-du}
\région{GOs BO PA}
\end{entrée}

\begin{entrée}
{couché la tête vers l'intérieur de la maison}
\vedette{kônõ-da}
\région{GOs BO}
\end{entrée}

\begin{entrée}
{couché près du feu ; dormir près du feu}
\vedette{kô-pa-ce-bò}
\région{GOs}
\variante{%
\vedette{kô-pa-ce-bòn, kô-phaa-ce-bòn}
\région{WEM}}
\variante{%
\vedette{kô-pha-ce-bòn}
\région{PA}}
\end{entrée}

\begin{entrée}
{couché sur le dos}
\vedette{kônõ-da}
\région{GOs BO}
\end{entrée}

\begin{entrée}
{couché sur le ventre;}
\vedette{kônõ-du}
\région{GOs BO PA}
\end{entrée}

\begin{entrée}
{couché sur le ventre (étape pour un bébé)}
\vedette{pe-phoo}
\région{GOs}
\end{entrée}

\begin{entrée}
{debout ; debout (être) ; dresser (se) ; mettre debout (se) ; debout (être) immobile}
\classe{v}
\vedette{kòò}\homonyme{2}
\sens{1}
\région{GOs}
\variante{%
\vedette{kòòl,kòl}
\région{PA BO WEM}}
\end{entrée}

\begin{entrée}
{debout en portant (bébé)}
\vedette{ku-kue}
\région{GOs}
\end{entrée}

\begin{entrée}
{debout en portant dans les bras}
\vedette{ku-e}
\région{GOs}
\end{entrée}

\begin{entrée}
{debout ensemble}
\vedette{kò-pe-bulu}
\région{GOs}
\end{entrée}

\begin{entrée}
{debout en tenant qqch serré dans la main}
\vedette{ku-cimwi}
\région{GOs PA}
\end{entrée}

\begin{entrée}
{debout (être) face à face}
\vedette{pe-ku-alö}
\région{GOs}
\end{entrée}

\begin{entrée}
{debout jambes écartées}
\vedette{ku-wãga}
\région{GOs}
\end{entrée}

\begin{entrée}
{debout tordu ; de travers}
\vedette{ku-pô}
\région{GOs}
\variante{%
\vedette{ku-pòng}
\région{BO WEM WE}}
\end{entrée}

\begin{entrée}
{derrière (être)}
\classe{v}
\vedette{mu}
\sens{1}
\région{GOs}
\variante{%
\vedette{mun}
\région{PA BO}}
\end{entrée}

\begin{entrée}
{dormir sur le ventre}
\vedette{kô-phoo}
\région{GOs}
\end{entrée}

\begin{entrée}
{droit (être) ; vertical ; d'aplomb}
\vedette{baaxò}
\groupe{B}
\sens{2}
\région{GOs WEM WE}
\région{BO}
\variante{%
\vedette{baaxòl}}
\variante{%
\vedette{baxòòl}
\région{PA}}
\end{entrée}

\begin{entrée}
{écouter un instant, un peu}
\vedette{kò-phaxeen}
\région{PA}
\end{entrée}

\begin{entrée}
{en travers}
\vedette{baazò}
\région{GOs}
\end{entrée}

\begin{entrée}
{en travers; couché en travers (de l'entrée, d'un lit, etc.)}
\vedette{kô-bazòò}
\région{GOs BO}
\end{entrée}

\begin{entrée}
{genoux (être à)}
\vedette{thi-bwagil}
\région{PA BO}
\end{entrée}

\begin{entrée}
{incliné ; allonger un peu (s')}
\classe{v}
\vedette{kô-kea}
\sens{1}
\région{GOs}
\variante{%
\vedette{kô-kea, kô-xea}
\région{PA BO}}
\end{entrée}

\begin{entrée}
{incliné ; appuyé}
\vedette{kea}
\région{GOs PA BO}
\end{entrée}

\begin{entrée}
{incliné (arbre) (lit. couché-ramper)}
\vedette{kô-töö}
\région{GOs BO}
\end{entrée}

\begin{entrée}
{jambes écartées (être)}
\vedette{wãga}
\région{GOs BO}
\end{entrée}

\begin{entrée}
{mettre (se) sur la pointe des pieds}
\classe{v}
\vedette{thixò}
\région{GOs}
\end{entrée}

\begin{entrée}
{pendu ; accroché (tre : marque d'état < déjà fait)}
\vedette{tre-çôô}
\région{GOs}
\end{entrée}

\begin{entrée}
{percher (se) ; perché}
\classe{v}
\vedette{trabwa}
\sens{1}
\région{GOs}
\variante{%
\vedette{tabwa}
\région{BO PA}}
\end{entrée}

\begin{entrée}
{recroquevillé}
\vedette{nyiwã}
\région{GOs}
\end{entrée}

\begin{entrée}
{renversé ; chaviré (voiture, bateau)}
\vedette{wakagume}
\région{GOs}
\end{entrée}

\begin{entrée}
{rester ; demeurer}
\vedette{tre-yuu}
\région{GOs}
\variante{%
\vedette{te-yu}
\région{PA BO}}
\end{entrée}

\begin{entrée}
{superposer (se)}
\vedette{pe-thirãgo}
\région{GOs}
\end{entrée}

\begin{entrée}
{suspendu}
\classe{v.i.}
\vedette{côô}
\sens{1}
\région{GOs}
\variante{%
\vedette{cô}
\région{BO (BM]}}
\end{entrée}

\begin{entrée}
{tourner sur le ventre (se)}
\vedette{phöö}
\région{GOs}
\end{entrée}

\begin{entrée}
{traîner par terre ; éparpiller (PA)}
\vedette{mããni}
\sens{2}
\région{GOs WEM BO PA}
\variante{%
\vedette{mãni}
\région{PA}}
\end{entrée}

\begin{entrée}
{travers (de) ; bancal ; penché sur uncôté ; pas à niveau}
\vedette{tre-chaaçee}
\région{GOs}
\end{entrée}

\begin{entrée}
{travers (de) ; penché sur uncôté}
\vedette{chaaçee}
\région{GOs PA}
\end{entrée}

\subsubsection{Verbes de déplacement et de mouvement}

\paragraph{Verbes de déplacement et moyens de déplacement}

\begin{entrée}
{accompagner (lit. aller suivre)}
\vedette{a-kai}
\région{GOs BO}
\end{entrée}

\begin{entrée}
{aller}
\vedette{thu-menõ}
\région{GOs PA}
\variante{%
\vedette{tomèno}
\région{BO}}
\end{entrée}

\begin{entrée}
{aller à flanc de montagne}
\vedette{a-lixee}
\région{GOs}
\end{entrée}

\begin{entrée}
{aller à la pêche (à la mer) ;}
\vedette{a-kaze}
\région{GOs}
\variante{%
\vedette{a-kale}
\région{PA}}
\end{entrée}

\begin{entrée}
{aller à la pêche ou la chasse à la torche}
\vedette{a-nûû}
\région{GOs}
\end{entrée}

\begin{entrée}
{aller chacun de son côté ; divorcer}
\vedette{a-pe-haze}
\région{GOs}
\région{PA BO}
\variante{%
\vedette{a-ve-hale}}
\end{entrée}

\begin{entrée}
{aller dans une direction transverse ; passer}
\vedette{a-è}
\région{GOs}
\end{entrée}

\begin{entrée}
{aller de nuit taper à la fenêtre d'une fille}
\vedette{töö}\homonyme{2}
\région{GOs PA WEM}
\variante{%
\vedette{too}
\région{PA BO}}
\end{entrée}

\begin{entrée}
{aller\_en amont d'un cours d'eau; sortir de l'eau, etc.}
\vedette{ã-da}
\région{GOs}
\end{entrée}

\begin{entrée}
{aller en descendant (sans destination précise)}
\vedette{a-vwe-du}
\région{GOs}
\end{entrée}

\begin{entrée}
{aller en montant (sans destination précise)}
\vedette{a-vwe-da}
\région{GOs}
\end{entrée}

\begin{entrée}
{aller en s'éloignant (sans destination précise)}
\vedette{a-wãã-ò}
\région{GOs}
\end{entrée}

\begin{entrée}
{aller ensemble ; accompagner}
\vedette{pe-mhe}
\région{GOs BO}
\end{entrée}

\begin{entrée}
{aller ensemble (lit. aller un)}
\vedette{a-poxe}
\région{GOs PA}
\end{entrée}

\begin{entrée}
{aller pêcher à la senne (lit. aller lancer le filet)}
\vedette{a-kha-pwiò}
\région{GOs}
\end{entrée}

\begin{entrée}
{aller près de}
\vedette{a-mõnu}
\région{GOs BO}
\end{entrée}

\begin{entrée}
{aller (sans but)}
\vedette{pe-a}
\région{GOs}
\end{entrée}

\begin{entrée}
{aller\_(s'en)}
\vedette{a-hò}
\région{GOs}
\variante{%
\vedette{a-ò}
\région{GO(s)}}
\end{entrée}

\begin{entrée}
{aller sur le côté ; aller dans une direction transverse}
\vedette{a-vwe-e}
\région{GOs}
\end{entrée}

\begin{entrée}
{aller sur le côté (sans destination précise)}
\vedette{a-wãã-e}
\région{GOs}
\end{entrée}

\begin{entrée}
{aller tout droit}
\vedette{a-baxoo}
\région{GOs}
\end{entrée}

\begin{entrée}
{aller vers l'extérieur}
\vedette{u-pwa, u-vwa}
\région{GO}
\end{entrée}

\begin{entrée}
{aller vers l'intérieur du pays}
\vedette{ã-da}
\région{GOs}
\end{entrée}

\begin{entrée}
{aller vers où ?}
\vedette{a waya ?}
\région{GOs}
\end{entrée}

\begin{entrée}
{amener ; emmener}
\vedette{whili}
\région{GOs WEM BO PA}
\end{entrée}

\begin{entrée}
{approcher}
\vedette{a-mõnu}
\région{GOs BO}
\end{entrée}

\begin{entrée}
{approcher}
\vedette{kò-mõn}
\région{PA}
\end{entrée}

\begin{entrée}
{approcher (s')}
\vedette{a-hû-mi}
\région{GOs}
\end{entrée}

\begin{entrée}
{arriver}
\vedette{uça}
\région{GOs}
\end{entrée}

\begin{entrée}
{arriver ; aller}
\vedette{thaa}\homonyme{2}
\région{PA BO}
\end{entrée}

\begin{entrée}
{arriver ; arrivé (courrier)}
\vedette{hovwa}
\sens{1}
\région{GOs WEM}
\variante{%
\vedette{hova}
\région{PA BO WEM}}
\variante{%
\vedette{hava}
\région{PA BO}}
\end{entrée}

\begin{entrée}
{arriver en haut}
\vedette{thrao-da}
\région{GOs WEM}
\variante{%
\vedette{thrawa-da}
\région{GO(s)}}
\variante{%
\vedette{thaa-da}
\région{PA BO}}
\end{entrée}

\begin{entrée}
{arriver ; survenir}
\vedette{thrawa}
\région{GOs}
\variante{%
\vedette{thrao}
\région{GO(s)}}
\variante{%
\vedette{thawa, thaa}
\région{BO}}
\end{entrée}

\begin{entrée}
{avancer un peu}
\classe{v}
\vedette{pò a-hu-ò}
\sens{1}
\région{GOs}
\variante{%
\vedette{pwo a-wò}
\région{GOs}}
\end{entrée}

\begin{entrée}
{baisser (se) ; pencher (se) [GOs]}
\vedette{u-du}
\région{GOs BO WEM}
\end{entrée}

\begin{entrée}
{chemin}
\classe{nom}
\vedette{phwè-dèn}
\sens{1}
\région{PA BO}
\end{entrée}

\begin{entrée}
{chemin ; sentier ; route}
\classe{nom}
\vedette{dè}
\sens{1}
\région{GOs}
\variante{%
\vedette{dèn}
\région{BO PA}}
\end{entrée}

\begin{entrée}
{chercher (épouse)}
\vedette{whili}
\région{GOs WEM BO PA}
\end{entrée}

\begin{entrée}
{conduire ; diriger (voiture, bateau, etc.)}
\vedette{tree}\homonyme{2}
\région{GOs}
\variante{%
\vedette{tee}
\région{PA}}
\end{entrée}

\begin{entrée}
{conduire ; guider}
\vedette{whili}
\région{GOs WEM BO PA}
\end{entrée}

\begin{entrée}
{conduire (voiture) [PA]}
\vedette{whili}
\région{GOs WEM BO PA}
\end{entrée}

\begin{entrée}
{contourner ;}
\vedette{pwewee}
\région{GOs}
\variante{%
\vedette{pweween}
\région{PA}}
\variante{%
\vedette{phweween}
\région{BO}}
\end{entrée}

\begin{entrée}
{contourner ; faire le tour}
\vedette{a-kênô}
\région{GOs}
\end{entrée}

\begin{entrée}
{coucher (se) (soleil) [GOs]}
\vedette{u-du}
\région{GOs BO WEM}
\end{entrée}

\begin{entrée}
{courir}
\vedette{thrêê}
\région{GOs}
\variante{%
\vedette{têên}
\région{PA BO WEM}}
\end{entrée}

\begin{entrée}
{courir en emportant qqch}
\vedette{thrêê kha-ve}
\région{GOs}
\variante{%
\vedette{thêên}
\région{PA}}
\end{entrée}

\begin{entrée}
{courir vite}
\vedette{zabajo}
\région{GOs}
\end{entrée}

\begin{entrée}
{demi-tour (faire) ; revenir sur ses pas ; retourner (s'en)}
\classe{v ; n}
\vedette{mwãju}
\sens{1}
\région{GOs PA BO}
\région{GOs}
\variante{%
\vedette{mwèju}}
\variante{%
\vedette{mwaji}
\région{BO}}
\end{entrée}

\begin{entrée}
{dépasser}
\classe{v ; QNT}
\vedette{biça}
\sens{1}
\région{WEM}
\end{entrée}

\begin{entrée}
{déplacer ; changer de place}
\classe{v}
\vedette{wône}
\sens{2}
\région{GOs}
\variante{%
\vedette{wene}
\région{PA}}
\end{entrée}

\begin{entrée}
{déplacer (se) en portant dans les bras}
\vedette{kha-bwaroe}
\région{PA}
\end{entrée}

\begin{entrée}
{déplacer (se) ; promener (se)}
\vedette{pe-thu-menõ}
\région{GOs}
\variante{%
\vedette{pe-tumenõ}
\région{GO(s)}}
\end{entrée}

\begin{entrée}
{déplacer (se) sans bruit}
\vedette{töö}\homonyme{2}
\région{GOs PA WEM}
\variante{%
\vedette{too}
\région{PA BO}}
\end{entrée}

\begin{entrée}
{descendre}
\vedette{ã-du}
\région{GOs PA}
\end{entrée}

\begin{entrée}
{descendre en marchant}
\vedette{kha-thrõbo}
\région{GOs}
\end{entrée}

\begin{entrée}
{descendre (sans destination précise)}
\vedette{a-wãã-du}
\région{GOs}
\end{entrée}

\begin{entrée}
{disparaitre [GOs, BO]}
\vedette{u-du}
\région{GOs BO WEM}
\end{entrée}

\begin{entrée}
{disperser (se)}
\vedette{thriu}
\end{entrée}

\begin{entrée}
{éloigner\_(s')}
\vedette{a-hò}
\région{GOs}
\variante{%
\vedette{a-ò}
\région{GO(s)}}
\end{entrée}

\begin{entrée}
{encercler}
\vedette{a-kênõge}
\région{GOs}
\end{entrée}

\begin{entrée}
{enfoncer (s') (dans la boue, le sable ) [GOs, BO]}
\vedette{u-du}
\région{GOs BO WEM}
\end{entrée}

\begin{entrée}
{enjamber}
\vedette{khau-da}
\région{GOs PA BO}
\end{entrée}

\begin{entrée}
{entourer}
\vedette{a-kênõge}
\région{GOs}
\end{entrée}

\begin{entrée}
{entrer dans une maison}
\vedette{ã-da}
\région{GOs}
\end{entrée}

\begin{entrée}
{entrer (se baisser pour entrer dans une maison)}
\vedette{u-da}
\région{GOs}
\end{entrée}

\begin{entrée}
{entrer (se baisser pour entrer ou sortir d'une maison) [GOs, BO]}
\vedette{u-du}
\région{GOs BO WEM}
\end{entrée}

\begin{entrée}
{faire demi-tour ; revenir de ; rentrer de}
\classe{v}
\vedette{pwaa}\homonyme{2}
\région{PA}
\variante{%
\vedette{phwala}
\région{BO}}
\end{entrée}

\begin{entrée}
{faire des zigzags}
\vedette{pwõ-o pwõ-mi}
\région{GOs}
\end{entrée}

\begin{entrée}
{faire tomber en marchant}
\vedette{kha-ku}
\région{GOs}
\variante{%
\vedette{kha-kule}
\région{PA}}
\end{entrée}

\begin{entrée}
{filer comme une flèche ; courir à toute allure}
\classe{v}
\vedette{tòè}
\sens{3}
\région{GOs PA BO}
\end{entrée}

\begin{entrée}
{fonctionner (machine)}
\vedette{thu-menõ}
\région{GOs PA}
\variante{%
\vedette{tomèno}
\région{BO}}
\end{entrée}

\begin{entrée}
{fouler (se) ; tordre [BO]}
\vedette{pwewee}
\région{GOs}
\variante{%
\vedette{pweween}
\région{PA}}
\variante{%
\vedette{phweween}
\région{BO}}
\end{entrée}

\begin{entrée}
{franchir (une montagne, un col)}
\classe{v}
\vedette{khau}
\sens{2}
\région{GOs PA BO}
\end{entrée}

\begin{entrée}
{frayer (se) un chemin dans la brousse (pour se sauver)}
\vedette{drua-ko}
\région{GOs PA}
\end{entrée}

\begin{entrée}
{fuir du mauvais côté ; prendre la fuite}
\classe{v}
\vedette{kala}\homonyme{1}
\région{GOs PA BO}
\end{entrée}

\begin{entrée}
{galoper}
\vedette{theepwa}
\région{GOs}
\end{entrée}

\begin{entrée}
{grande route}
\vedette{payang}
\région{BO}
\end{entrée}

\begin{entrée}
{grimper (en marchant)}
\vedette{kha-da}
\région{GOs PA}
\end{entrée}

\begin{entrée}
{hisser (se)}
\vedette{töö}\homonyme{2}
\région{GOs PA WEM}
\variante{%
\vedette{too}
\région{PA BO}}
\end{entrée}

\begin{entrée}
{joindre (se)}
\vedette{pe-vwii}
\région{GOs}
\variante{%
\vedette{pe-viing}
\région{PA WEM}}
\variante{%
\vedette{phiing}
\région{PA BO}}
\end{entrée}

\begin{entrée}
{longer qqch (activité)}
\vedette{pe-a-hoze}
\région{GOs}
\end{entrée}

\begin{entrée}
{marcher}
\vedette{ta}\homonyme{2}
\région{PA}
\end{entrée}

\begin{entrée}
{marcher à 4 pattes}
\vedette{töö}\homonyme{2}
\région{GOs PA WEM}
\variante{%
\vedette{too}
\région{PA BO}}
\end{entrée}

\begin{entrée}
{marcher avec une canne}
\vedette{kha-hêgo}
\région{GOs PA}
\end{entrée}

\begin{entrée}
{marcher avec une charge sur le dos}
\vedette{kha-khoońe}
\région{GOs}
\end{entrée}

\begin{entrée}
{marcher en file indienne}
\vedette{pe-gu-xe}
\région{PA}
\end{entrée}

\begin{entrée}
{marcher en tête ; passer devant}
\vedette{a-hãbu}
\région{GOs}
\end{entrée}

\begin{entrée}
{marcher (faire des traces)}
\vedette{thu-menõ}
\région{GOs PA}
\variante{%
\vedette{tomèno}
\région{BO}}
\end{entrée}

\begin{entrée}
{marcher sans bruit ; déplacer (se) doucement}
\vedette{kha-çaaxò}
\région{GOs}
\variante{%
\vedette{kacaaò}
\région{BO}}
\end{entrée}

\begin{entrée}
{marcher sur la pointe des pieds [GOs]}
\classe{v}
\vedette{kha-thixò}
\sens{1}
\région{GOs}
\variante{%
\vedette{kha-thixò}
\région{PA}}
\end{entrée}

\begin{entrée}
{mener (travail)}
\vedette{whili}
\région{GOs WEM BO PA}
\end{entrée}

\begin{entrée}
{monter}
\vedette{ã-da}
\région{GOs}
\end{entrée}

\begin{entrée}
{monter}
\vedette{u-da}
\région{GOs}
\end{entrée}

\begin{entrée}
{monter à pied}
\vedette{kha-da}
\région{GOs PA}
\end{entrée}

\begin{entrée}
{monter (en s'éloignantdu locuteur)}
\vedette{ã-da-ò}
\région{GOs}
\end{entrée}

\begin{entrée}
{monter (sans destination précise)}
\vedette{a-wãã-da}
\région{GOs}
\end{entrée}

\begin{entrée}
{où (aller) ? ; aller où ?}
\vedette{a-wi?}
\région{GOs PA}
\end{entrée}

\begin{entrée}
{partir}
\vedette{a-hò}
\région{GOs}
\variante{%
\vedette{a-ò}
\région{GO(s)}}
\end{entrée}

\begin{entrée}
{partir ;}
\vedette{thriu}
\end{entrée}

\begin{entrée}
{partir ; aller}
\vedette{a}\homonyme{1}
\région{GOs}
\variante{%
\vedette{ò}
\région{BO PA}}
\end{entrée}

\begin{entrée}
{partir ; quitter}
\classe{v}
\vedette{kala}\homonyme{1}
\région{GOs PA BO}
\end{entrée}

\begin{entrée}
{passer à côté en frôlant}
\classe{v}
\vedette{kã}
\sens{2}
\région{GOs}
\variante{%
\vedette{kãm}
\région{BO}}
\variante{%
\vedette{kham}
\région{PA}}
\end{entrée}

\begin{entrée}
{passer à toute allure}
\vedette{zabajo}
\région{GOs}
\end{entrée}

\begin{entrée}
{passer par dessus (montagne, interdit)}
\classe{v.t.}
\vedette{khau-ni}
\sens{1}
\end{entrée}

\begin{entrée}
{passer par dessus (montagne) ; passer d'une vallée à l'autre}
\vedette{khau-da}
\région{GOs PA BO}
\end{entrée}

\begin{entrée}
{passer près de}
\vedette{a-khazia}
\région{GOs}
\end{entrée}

\begin{entrée}
{passer sous une barrière}
\vedette{tiyôô}
\région{PA}
\end{entrée}

\begin{entrée}
{pêcher à la mer}
\vedette{a-kaze}
\région{GOs}
\variante{%
\vedette{a-kale}
\région{PA}}
\end{entrée}

\begin{entrée}
{plonger [GOs, BO]}
\vedette{u-du}
\région{GOs BO WEM}
\end{entrée}

\begin{entrée}
{prendre la route}
\vedette{pe-phao}
\région{PA}
\end{entrée}

\begin{entrée}
{prendre la route ; mettre (se) en route}
\classe{v}
\vedette{pao}
\sens{2}
\région{GOsPA BO}
\end{entrée}

\begin{entrée}
{prendre par la main (enfant)}
\vedette{whili}
\région{GOs WEM BO PA}
\end{entrée}

\begin{entrée}
{prendre un raccourci}
\vedette{kha-kibwaa}
\région{GOs}
\end{entrée}

\begin{entrée}
{promener ; promenade}
\vedette{mhè}\homonyme{1}
\région{GOs BO}
\end{entrée}

\begin{entrée}
{promener (se)}
\vedette{thu-menõ}
\région{GOs PA}
\variante{%
\vedette{tomèno}
\région{BO}}
\end{entrée}

\begin{entrée}
{raccourci}
\vedette{dèn kha-jöe}
\région{PA}
\end{entrée}

\begin{entrée}
{ramper (enfant)}
\vedette{töö}\homonyme{2}
\région{GOs PA WEM}
\variante{%
\vedette{too}
\région{PA BO}}
\end{entrée}

\begin{entrée}
{rejoindre (se)}
\vedette{pe-vwii}
\région{GOs}
\variante{%
\vedette{pe-viing}
\région{PA WEM}}
\variante{%
\vedette{phiing}
\région{PA BO}}
\end{entrée}

\begin{entrée}
{rendre visite}
\vedette{pîînã}
\région{GOs BO PA}
\end{entrée}

\begin{entrée}
{rendre visite à un malade}
\vedette{a-kha-kujaxo}
\région{GOs}
\end{entrée}

\begin{entrée}
{rentrer (dans un trou, dans l'eau) [GOs, BO]}
\vedette{u-du}
\région{GOs BO WEM}
\end{entrée}

\begin{entrée}
{retourner ; retourner (s'en)}
\classe{v ; n}
\vedette{mwãju}
\sens{1}
\région{GOs PA BO}
\région{GOs}
\variante{%
\vedette{mwèju}}
\variante{%
\vedette{mwaji}
\région{BO}}
\end{entrée}

\begin{entrée}
{revenir}
\vedette{mwã}\homonyme{2}
\sens{1}
\région{GO PA}
\end{entrée}

\begin{entrée}
{revenir (en sens inverse); se retourner;}
\vedette{pwewee}
\région{GOs}
\variante{%
\vedette{pweween}
\région{PA}}
\variante{%
\vedette{phweween}
\région{BO}}
\end{entrée}

\begin{entrée}
{revenir sur ses pas}
\classe{v ; n}
\vedette{mwãju}
\sens{1}
\région{GOs PA BO}
\région{GOs}
\variante{%
\vedette{mwèju}}
\variante{%
\vedette{mwaji}
\région{BO}}
\end{entrée}

\begin{entrée}
{route ; chemin frayé à l'outil}
\vedette{paça}
\région{GOs}
\variante{%
\vedette{payang, paya, peyeng}
\région{WEM PA BO}}
\end{entrée}

\begin{entrée}
{sauver (se)}
\classe{v}
\vedette{kala}\homonyme{1}
\région{GOs PA BO}
\end{entrée}

\begin{entrée}
{sauver (se) dans la brousse}
\vedette{drua-ko}
\région{GOs PA}
\end{entrée}

\begin{entrée}
{sortir (de la maison)}
\vedette{ã-du}
\région{GOs PA}
\end{entrée}

\begin{entrée}
{sortir (de la maison)}
\vedette{ubò}
\région{GOs}
\variante{%
\vedette{u-vwa = u phwaa (différent)}
\région{GO(s)}}
\variante{%
\vedette{uva}
\région{WEM}}
\end{entrée}

\begin{entrée}
{sortir ; sors !}
\vedette{cabòl !}
\région{PA BO}
\end{entrée}

\begin{entrée}
{suivre}
\vedette{a-kai}
\région{GOs BO}
\end{entrée}

\begin{entrée}
{suivre (berge, rivière, etc.)}
\classe{v}
\vedette{höze}
\sens{1}
\région{GOs}
\variante{%
\vedette{hure}
\région{WEM WE PA}}
\variante{%
\vedette{hore}
\région{BO}}
\end{entrée}

\begin{entrée}
{suivre ; longer}
\vedette{hore}
\région{PA WEM BO}
\end{entrée}

\begin{entrée}
{suivre ; longer à pied}
\vedette{kha-hoze}
\région{GOs}
\variante{%
\vedette{a-hoze}
\région{GO(s)}}
\end{entrée}

\begin{entrée}
{suivre (se) ; marcher l'un derrière l'autre}
\vedette{pe-whili}
\région{GOs}
\end{entrée}

\begin{entrée}
{suivre (se) ; suivre ; marcher en file indienne}
\vedette{pe-a-kai-n}
\région{GOs PA}
\end{entrée}

\begin{entrée}
{suivre (se) ; suivre qqn}
\vedette{wili}\homonyme{2}
\région{PA BO}
\variante{%
\vedette{huli}
\région{BO PA}}
\variante{%
\vedette{wele}
\région{BO}}
\end{entrée}

\begin{entrée}
{suivre ; suivre (se)}
\vedette{pe-hòze}
\région{GOs}
\variante{%
\vedette{pe-hòre}
\région{BO}}
\end{entrée}

\begin{entrée}
{tirer en se déplaçant}
\vedette{kha-trivwi}
\région{GO}
\end{entrée}

\begin{entrée}
{tourner}
\classe{v}
\vedette{pwaa}\homonyme{2}
\région{PA}
\variante{%
\vedette{phwala}
\région{BO}}
\end{entrée}

\begin{entrée}
{tourner ; revenir; retourner (s'en)}
\vedette{bwange}\homonyme{2}
\région{GOs}
\end{entrée}

\begin{entrée}
{trace (laissée à un endroit par un animal ou une personne qui y a dormi)}
\vedette{pobo}
\région{GOs PA BO}
\end{entrée}

\begin{entrée}
{trace ; marque}
\vedette{mhenõõ}\homonyme{2}
\région{GOs WEM PA}
\variante{%
\vedette{mhenõõ}
\région{GOs WEM}}
\end{entrée}

\begin{entrée}
{traces de bois trainés}
\vedette{pwedee}
\région{GOs}
\end{entrée}

\begin{entrée}
{traces ; empreintes (homme ou animal)}
\vedette{bwèè-xò}
\région{GOs PA WEM BO}
\end{entrée}

\begin{entrée}
{traverser à pied (une route)}
\vedette{kha-çöe}
\région{GOs}
\variante{%
\vedette{khaa-jöe}
\région{PA}}
\end{entrée}

\begin{entrée}
{traverser ;passer à travers}
\classe{v}
\vedette{khibwaa}
\sens{2}
\région{GOs PA BO}
\end{entrée}

\begin{entrée}
{traverser ; passer à travers}
\vedette{thirawa}
\région{GOs}
\end{entrée}

\begin{entrée}
{traverser ; passer à travers; transpercer}
\vedette{thiraò}
\région{GOs PA BO}
\end{entrée}

\begin{entrée}
{traverser (rivière)}
\vedette{cö-e}
\région{GOs BO}
\variante{%
\vedette{cu-e}
\région{BO}}
\end{entrée}

\begin{entrée}
{venir vers ego}
\vedette{ã-mi}
\région{GOs}
\variante{%
\vedette{ô-mi}
\région{BO}}
\end{entrée}

\begin{entrée}
{voyage en groupe ; déplacement en groupe}
\vedette{mhenõ}\homonyme{1}
\région{GOs}
\end{entrée}

\begin{entrée}
{voyager ; promener (se) ;}
\vedette{pîînã}
\région{GOs BO PA}
\end{entrée}

\paragraph{Verbes de mouvement}

\begin{entrée}
{adosser (s') ; adossé}
\vedette{ku-xea}
\région{GOs}
\variante{%
\vedette{ku-xia}
\région{GO(s)}}
\end{entrée}

\begin{entrée}
{agenouiller (s')}
\vedette{thi-bwagil}
\région{PA BO}
\end{entrée}

\begin{entrée}
{agenouiller (s') ; agenouillé}
\vedette{tre-thibu}
\région{GOs}
\variante{%
\vedette{tre-zibu}
\région{GO(s)}}
\end{entrée}

\begin{entrée}
{agiter (s') en dormant}
\vedette{kô-ii}
\région{GOs}
\end{entrée}

\begin{entrée}
{apparaître}
\classe{v}
\vedette{cabo}
\sens{2}
\région{GOs}
\variante{%
\vedette{cabwòl, cabòl}
\région{PA BO WEM}}
\end{entrée}

\begin{entrée}
{assis ensemble}
\vedette{tree-poxe}
\région{PA}
\end{entrée}

\begin{entrée}
{atterrir ; toucher terre}
\classe{v}
\vedette{trabwa}
\sens{2}
\région{GOs}
\variante{%
\vedette{tabwa}
\région{BO PA}}
\end{entrée}

\begin{entrée}
{baisser (se) ;}
\vedette{u}\homonyme{3}
\région{GOs BO}
\end{entrée}

\begin{entrée}
{baisser (se) ; pencher (se) [GOs]}
\vedette{u-du}
\région{GOs BO WEM}
\end{entrée}

\begin{entrée}
{balancer (se)}
\vedette{pe-hiliçôô}
\région{GOs}
\end{entrée}

\begin{entrée}
{bouger ; remuer}
\classe{v ; n}
\vedette{nyamã}
\sens{1}
\région{GOs PA BO}
\end{entrée}

\begin{entrée}
{bouger (un objet)}
\vedette{nyamãle}
\région{GOs}
\end{entrée}

\begin{entrée}
{buter sur qqch.}
\classe{v}
\vedette{cabi}
\sens{3}
\région{GOs PA BO}
\end{entrée}

\begin{entrée}
{buter (sur qqch) ; trébucher}
\classe{v}
\vedette{thali}
\sens{1}
\région{PA BO}
\end{entrée}

\begin{entrée}
{changer; corriger [BO]}
\vedette{poweede}
\région{GOs PA BO}
\end{entrée}

\begin{entrée}
{changer ; traduire}
\vedette{pweweede}
\région{GOs BO PA}
\end{entrée}

\begin{entrée}
{chavirer ; retourner (se)}
\vedette{bu}\homonyme{3}
\région{GOs}
\variante{%
\vedette{bul}
\région{BO PA}}
\end{entrée}

\begin{entrée}
{chavirer sur le côté ; gîter ; penché ; couché [Corne]}
\vedette{khee}\homonyme{3}
\région{BO}
\end{entrée}

\begin{entrée}
{coucher (se) (soleil) [GOs]}
\vedette{u-du}
\région{GOs BO WEM}
\end{entrée}

\begin{entrée}
{courber (se) (par ex. pour entrer dans une maison)}
\vedette{u}\homonyme{3}
\région{GOs BO}
\end{entrée}

\begin{entrée}
{dandiner (se); se balancer (en marchant) [Corne]}
\vedette{nyaanya}
\région{BO}
\end{entrée}

\begin{entrée}
{débarquer}
\vedette{cöö}
\région{GOs}
\variante{%
\vedette{còòl}
\région{PA BO}}
\variante{%
\vedette{cul, cu}
\région{PA}}
\end{entrée}

\begin{entrée}
{dégringoler}
\vedette{nhu}
\région{GOs}
\variante{%
\vedette{nhu}
\région{BO}}
\end{entrée}

\begin{entrée}
{déplier}
\vedette{thròlòe}
\région{GOs}
\variante{%
\vedette{thòlòe}
\région{PA}}
\variante{%
\vedette{tòloè}
\région{BO (Corne)}}
\variante{%
\vedette{tholòè, tòlee}
\région{BO (BM)}}
\variante{%
\vedette{throleitholei}
\région{WEM}}
\end{entrée}

\begin{entrée}
{déraper}
\vedette{kha-ru-heela}
\région{PA BO}
\variante{%
\vedette{kha-thu-heela}}
\end{entrée}

\begin{entrée}
{descendre ; tomber}
\classe{v}
\vedette{thrõbo}
\sens{1}
\région{GOs WEM}
\variante{%
\vedette{thôbo}
\région{BO PA}}
\end{entrée}

\begin{entrée}
{descendre verticalement}
\vedette{ulu}
\région{GOs}
\end{entrée}

\begin{entrée}
{disparaitre [GOs, BO]}
\vedette{u-du}
\région{GOs BO WEM}
\end{entrée}

\begin{entrée}
{disperser (se)}
\vedette{thriu}
\end{entrée}

\begin{entrée}
{éboulement ; écrouler (s')}
\classe{v ; n}
\vedette{drõbö}
\sens{1}
\région{GOs}
\variante{%
\vedette{dòbo}
\région{BO PA}}
\end{entrée}

\begin{entrée}
{écarter ; chasser (animal)}
\vedette{chivi}
\région{BO}
\variante{%
\vedette{chivi, civi}
\région{BO}}
\end{entrée}

\begin{entrée}
{échouer (s')}
\vedette{trèzoo}
\région{GOs}
\end{entrée}

\begin{entrée}
{écrouler (s')}
\vedette{nhu}
\région{GOs}
\variante{%
\vedette{nhu}
\région{BO}}
\end{entrée}

\begin{entrée}
{effleurer}
\vedette{ca-khã}
\région{GOs}
\end{entrée}

\begin{entrée}
{effleurer ;}
\vedette{kham}
\région{PA BO}
\variante{%
\vedette{khã}
\région{GO(s)}}
\end{entrée}

\begin{entrée}
{embourber\_(s')}
\vedette{ulu}
\région{GOs}
\end{entrée}

\begin{entrée}
{émerger}
\classe{v}
\vedette{cabo}
\sens{2}
\région{GOs}
\variante{%
\vedette{cabwòl, cabòl}
\région{PA BO WEM}}
\end{entrée}

\begin{entrée}
{émerger (poisson)}
\vedette{kôgò}
\région{GOs}
\end{entrée}

\begin{entrée}
{enfoncer\_(s')}
\vedette{ulu}
\région{GOs}
\end{entrée}

\begin{entrée}
{enfoncer (s') (dans la boue, le sable ) [GOs, BO]}
\vedette{u-du}
\région{GOs BO WEM}
\end{entrée}

\begin{entrée}
{enjamber}
\classe{v}
\vedette{khau}
\sens{1}
\région{GOs PA BO}
\end{entrée}

\begin{entrée}
{entrer (se baisser pour entrer ou sortir d'une maison) [GOs, BO]}
\vedette{u-du}
\région{GOs BO WEM}
\end{entrée}

\begin{entrée}
{étaler (natte, etc.)}
\vedette{thròlòe}
\région{GOs}
\variante{%
\vedette{thòlòe}
\région{PA}}
\variante{%
\vedette{tòloè}
\région{BO (Corne)}}
\variante{%
\vedette{tholòè, tòlee}
\région{BO (BM)}}
\variante{%
\vedette{throleitholei}
\région{WEM}}
\end{entrée}

\begin{entrée}
{étaler (s') (pour des plantes)}
\vedette{thròlòe}
\région{GOs}
\variante{%
\vedette{thòlòe}
\région{PA}}
\variante{%
\vedette{tòloè}
\région{BO (Corne)}}
\variante{%
\vedette{tholòè, tòlee}
\région{BO (BM)}}
\variante{%
\vedette{throleitholei}
\région{WEM}}
\end{entrée}

\begin{entrée}
{étendre (bras, etc.)}
\vedette{thròlòe}
\région{GOs}
\variante{%
\vedette{thòlòe}
\région{PA}}
\variante{%
\vedette{tòloè}
\région{BO (Corne)}}
\variante{%
\vedette{tholòè, tòlee}
\région{BO (BM)}}
\variante{%
\vedette{throleitholei}
\région{WEM}}
\end{entrée}

\begin{entrée}
{éviter ;}
\vedette{kham}
\région{PA BO}
\variante{%
\vedette{khã}
\région{GO(s)}}
\end{entrée}

\begin{entrée}
{faire des culbutes ; faire des tonneaux (voiture)}
\vedette{parixo}
\région{GOs}
\end{entrée}

\begin{entrée}
{faire des galipettes (sur le côté)}
\vedette{bwarao}
\région{GOs}
\variante{%
\vedette{bwaòl}
\région{WEM}}
\end{entrée}

\begin{entrée}
{faire des moulinets du bras ; moulinet du bras (Ltd 1008)}
\classe{v ; n}
\vedette{kênõ}
\sens{2}
\région{GOs}
\région{BO PA}
\variante{%
\vedette{kênõng}}
\end{entrée}

\begin{entrée}
{faire la course (de vitesse) ; poursuivre (se)}
\vedette{pe-ravwi}
\région{GOs}
\end{entrée}

\begin{entrée}
{faire un peu de place (quand on est assis)}
\vedette{pò-tree-hu-ò}
\région{GOs}
\variante{%
\vedette{pwo tree-wò}
\région{GOs}}
\end{entrée}

\begin{entrée}
{faufiler (se)}
\vedette{nube}
\région{GOs}
\end{entrée}

\begin{entrée}
{frotter (se) par terre ;}
\vedette{thaxim}
\région{PA BO [BM]}
\end{entrée}

\begin{entrée}
{genoux (être à)}
\vedette{thi-bwagil}
\région{PA BO}
\end{entrée}

\begin{entrée}
{glisser}
\classe{v}
\vedette{heela}
\sens{1}
\région{GOs BO}
\end{entrée}

\begin{entrée}
{glisser}
\vedette{nhu}
\région{GOs}
\variante{%
\vedette{nhu}
\région{BO}}
\end{entrée}

\begin{entrée}
{glisser[Corne]}
\vedette{mwâô}
\région{BO}
\end{entrée}

\begin{entrée}
{glisser (se)}
\vedette{nube}
\région{GOs}
\end{entrée}

\begin{entrée}
{glisser (se) ; se faufiler[Corne]}
\vedette{tuuwa}
\région{BO}
\end{entrée}

\begin{entrée}
{glisser sur une glissoire}
\vedette{kha-ru-heela}
\région{PA BO}
\variante{%
\vedette{kha-thu-heela}}
\end{entrée}

\begin{entrée}
{glissoire aménagée par les enfants sur une pente mouillée}
\vedette{kha-ru-heela}
\région{PA BO}
\variante{%
\vedette{kha-thu-heela}}
\end{entrée}

\begin{entrée}
{grimper}
\vedette{cö-da}
\région{GOs WEM}
\variante{%
\vedette{cu-da}
\région{PA}}
\end{entrée}

\begin{entrée}
{incliner (s') [BM]}
\vedette{ciluu}
\région{BO}
\end{entrée}

\begin{entrée}
{monter}
\classe{v}
\vedette{cabo}
\sens{2}
\région{GOs}
\variante{%
\vedette{cabwòl, cabòl}
\région{PA BO WEM}}
\end{entrée}

\begin{entrée}
{monter}
\vedette{cö-da}
\région{GOs WEM}
\variante{%
\vedette{cu-da}
\région{PA}}
\end{entrée}

\begin{entrée}
{monter aux arbres à quatre pattes, en écartant le corps du tronc}
\vedette{caxòò}
\région{GOs}
\end{entrée}

\begin{entrée}
{monter aux arbres (en serrant le tronc)}
\classe{v}
\vedette{beela}
\sens{2}
\région{GOs PA BO}
\end{entrée}

\begin{entrée}
{monter (eau) ; déborder}
\classe{v.i.}
\vedette{phuu}
\sens{1}
\région{GOs PA BO}
\variante{%
\vedette{phuuxu}
\région{GO}}
\end{entrée}

\begin{entrée}
{partir ;}
\vedette{thriu}
\end{entrée}

\begin{entrée}
{passer par dessus}
\classe{v}
\vedette{khau}
\sens{1}
\région{GOs PA BO}
\end{entrée}

\begin{entrée}
{pencher (se) ;}
\vedette{u}\homonyme{3}
\région{GOs BO}
\end{entrée}

\begin{entrée}
{plonger}
\vedette{cö-du}
\end{entrée}

\begin{entrée}
{plonger [GOs, BO]}
\vedette{u-du}
\région{GOs BO WEM}
\end{entrée}

\begin{entrée}
{poser près d'ego}
\vedette{na-hû-mi}
\région{GOs}
\end{entrée}

\begin{entrée}
{poser (se)}
\classe{v}
\vedette{trabwa}
\sens{2}
\région{GOs}
\variante{%
\vedette{tabwa}
\région{BO PA}}
\end{entrée}

\begin{entrée}
{pousser (se) un peu}
\vedette{pò-tree-hu-ò}
\région{GOs}
\variante{%
\vedette{pwo tree-wò}
\région{GOs}}
\end{entrée}

\begin{entrée}
{pousser (se) un peu ; faire un peu de place}
\classe{v}
\vedette{pò a-hu-ò}
\sens{2}
\région{GOs}
\variante{%
\vedette{pwo a-wò}
\région{GOs}}
\end{entrée}

\begin{entrée}
{ramper (enfant) ; marcher à 4 pattes (enfant)}
\classe{v}
\vedette{beela}
\sens{1}
\région{GOs PA BO}
\end{entrée}

\begin{entrée}
{ramper (lianes)}
\vedette{thròlòe}
\région{GOs}
\variante{%
\vedette{thòlòe}
\région{PA}}
\variante{%
\vedette{tòloè}
\région{BO (Corne)}}
\variante{%
\vedette{tholòè, tòlee}
\région{BO (BM)}}
\variante{%
\vedette{throleitholei}
\région{WEM}}
\end{entrée}

\begin{entrée}
{rassembler ; assembler}
\vedette{pe-na-bulu-ni}
\région{GOs}
\end{entrée}

\begin{entrée}
{rassembler debout (se)}
\vedette{ku-bulu}
\région{PA}
\end{entrée}

\begin{entrée}
{rassembler (se) (lit. debout-un)}
\vedette{ku-poxe}
\région{PA}
\end{entrée}

\begin{entrée}
{rater ; manquer}
\vedette{kham}
\région{PA BO}
\variante{%
\vedette{khã}
\région{GO(s)}}
\end{entrée}

\begin{entrée}
{rebondir}
\vedette{thiçoo}
\end{entrée}

\begin{entrée}
{rebondir et revenir en sens inverse}
\vedette{thibuyul}
\région{PA}
\end{entrée}

\begin{entrée}
{rentrer (dans un trou, dans l'eau) [GOs, BO]}
\vedette{u-du}
\région{GOs BO WEM}
\end{entrée}

\begin{entrée}
{retourner, soulever une pierre pour voir s'il y a qqch en dessous [PA]}
\classe{v}
\vedette{phwia, phwiça}
\sens{2}
\région{GOs PA WEM WE BO}
\variante{%
\vedette{phuça}
\région{GO(s)}}
\variante{%
\vedette{phuya}
\région{BO}}
\end{entrée}

\begin{entrée}
{retourner (verre, seau, etc.)}
\vedette{poweede}
\région{GOs PA BO}
\end{entrée}

\begin{entrée}
{rétracter\_(se)}
\vedette{iili}
\région{PA BO [Corne]}
\end{entrée}

\begin{entrée}
{ricocher}
\vedette{ca-khã}
\région{GOs}
\end{entrée}

\begin{entrée}
{ricocher}
\vedette{thi-kham}
\région{PA}
\end{entrée}

\begin{entrée}
{ricocher ;}
\vedette{kham}
\région{PA BO}
\variante{%
\vedette{khã}
\région{GO(s)}}
\end{entrée}

\begin{entrée}
{ricocher (faire) un caillou}
\vedette{thele paa}
\région{GOs}
\end{entrée}

\begin{entrée}
{rouler}
\vedette{bwarao}
\région{GOs}
\variante{%
\vedette{bwaòl}
\région{WEM}}
\end{entrée}

\begin{entrée}
{rouler (se)}
\vedette{phwaal}
\région{PA BO}
\end{entrée}

\begin{entrée}
{rouler (se) par terre ;}
\vedette{thaxim}
\région{PA BO [BM]}
\end{entrée}

\begin{entrée}
{rouler (se) par terre (animal, enfant)}
\classe{v}
\vedette{thazabi}
\sens{2}
\région{GOs}
\variante{%
\vedette{tharabil}
\région{PA}}
\end{entrée}

\begin{entrée}
{rouler ; tourner une roue}
\classe{v ; n}
\vedette{bwaòle}
\sens{1}
\région{PA BO WEM}
\end{entrée}

\begin{entrée}
{sauter}
\vedette{cöö}
\région{GOs}
\variante{%
\vedette{còòl}
\région{PA BO}}
\variante{%
\vedette{cul, cu}
\région{PA}}
\end{entrée}

\begin{entrée}
{sauter à la corde [Corne]}
\vedette{karolia}
\région{BO}
\end{entrée}

\begin{entrée}
{sauter de-ci de-là}
\vedette{cö-ò cö-mi}
\région{GOs}
\variante{%
\vedette{cu-ò cu-mi}
\région{BO}}
\end{entrée}

\begin{entrée}
{sauter de-ci de-là}
\vedette{pe-co-ò coo-mi}
\région{GOs BO}
\variante{%
\vedette{pe-cool}
\région{PA}}
\end{entrée}

\begin{entrée}
{sauter en bas}
\vedette{cö-du}
\end{entrée}

\begin{entrée}
{sauter vers le haut}
\vedette{cö-da}
\région{GOs WEM}
\variante{%
\vedette{cu-da}
\région{PA}}
\end{entrée}

\begin{entrée}
{serrer}
\vedette{khòle}
\région{GOs BO PA}
\end{entrée}

\begin{entrée}
{sombrer ; couler ; noyer (se)}
\vedette{bu}\homonyme{3}
\région{GOs}
\variante{%
\vedette{bul}
\région{BO PA}}
\end{entrée}

\begin{entrée}
{sortir d'un trou (animal, reptile)}
\vedette{hõõng}
\région{PA}
\end{entrée}

\begin{entrée}
{surgir}
\classe{v}
\vedette{tãã}
\sens{1}
\région{GOs}
\end{entrée}

\begin{entrée}
{tirer (langue)}
\vedette{hõõng}
\région{PA}
\end{entrée}

\begin{entrée}
{tituber}
\vedette{jiilè}
\région{GOs}
\end{entrée}

\begin{entrée}
{tomber}
\vedette{kole}\homonyme{1}
\région{GOs}
\variante{%
\vedette{kule}
\région{PA BO}}
\end{entrée}

\begin{entrée}
{tomber}
\classe{v}
\vedette{u}\homonyme{5}
\région{GOs PA}
\end{entrée}

\begin{entrée}
{tomber (de qqch, pour un inanimé)}
\vedette{ku}\homonyme{3}
\région{GOs}
\variante{%
\vedette{kuul}
\région{PA BO WE}}
\end{entrée}

\begin{entrée}
{tomber (d'un animé)}
\vedette{kaalu}
\sens{1}
\région{GOs BO PA}
\end{entrée}

\begin{entrée}
{toucher (avec la main)}
\vedette{khòle}
\région{GOs BO PA}
\end{entrée}

\begin{entrée}
{toucher avec la sagaie ; piquer}
\vedette{thabaò}
\région{PA}
\end{entrée}

\begin{entrée}
{toucher (cible)}
\vedette{pha-ca}
\end{entrée}

\begin{entrée}
{toucher une cible}
\vedette{bule}
\région{GOs PA}
\end{entrée}

\begin{entrée}
{tourner}
\vedette{phwaal}
\région{PA BO}
\end{entrée}

\begin{entrée}
{tourner autour}
\vedette{pwebwe}
\région{BO}
\end{entrée}

\begin{entrée}
{tourner (page)}
\vedette{poweede}
\région{GOs PA BO}
\end{entrée}

\begin{entrée}
{tourner ; retourner ; poser à l'envers ; retourner un objet (sur le même plan)}
\vedette{pweweede}
\région{GOs BO PA}
\end{entrée}

\begin{entrée}
{tourner une roue}
\classe{v ; n}
\vedette{bwaòle}
\sens{1}
\région{PA BO WEM}
\end{entrée}

\begin{entrée}
{traverser}
\vedette{cöö}
\région{GOs}
\variante{%
\vedette{còòl}
\région{PA BO}}
\variante{%
\vedette{cul, cu}
\région{PA}}
\end{entrée}

\begin{entrée}
{trébucher}
\vedette{wazizibu}
\région{GOs}
\end{entrée}

\begin{entrée}
{vautrer (se)}
\vedette{thaxim}
\région{PA BO [BM]}
\end{entrée}

\paragraph{Moyens de locomotion et chemins}

\begin{entrée}
{avion (lit. bateau volant)}
\vedette{wõ-phu}
\région{GOs}
\end{entrée}

\begin{entrée}
{bateau ; embarcation}
\classe{nom}
\vedette{wõ}
\sens{1}
\région{GOs}
\variante{%
\vedette{wony}
\région{PA WEM BO}}
\end{entrée}

\begin{entrée}
{voiture ; auto}
\vedette{loto}
\région{GOs}
\variante{%
\vedette{lòto}
\région{BO}}
\variante{%
\vedette{wathuu}
\région{GO(s)}}
\end{entrée}

\subsubsection{Verbes d'action (en général)}

\begin{entrée}
{abandonner}
\vedette{kalãnge}
\région{GOs}
\variante{%
\vedette{kala-kagee}
\région{PA}}
\end{entrée}

\begin{entrée}
{accident ; avoir un accident}
\vedette{kaalu}
\sens{2}
\région{GOs BO PA}
\end{entrée}

\begin{entrée}
{actes ; actions ; occupations}
\vedette{nobwò}
\région{GOs}
\variante{%
\vedette{nòbu}
\région{BO PA}}
\end{entrée}

\begin{entrée}
{ajouter}
\vedette{vala}
\région{PA}
\end{entrée}

\begin{entrée}
{ajouter ; allonger ; assembler}
\vedette{kine}\homonyme{1}
\région{GOs PA BO}
\variante{%
\vedette{kin}
\région{BO}}
\end{entrée}

\begin{entrée}
{ajouter ; mettre plus ; compléter}
\vedette{thu ãbaa}
\région{GOs}
\variante{%
\vedette{tho ãbaa-n}
\région{BO}}
\end{entrée}

\begin{entrée}
{aligner (des choses)}
\vedette{ku-ido-xe}
\région{GOs}
\end{entrée}

\begin{entrée}
{arrêter}
\vedette{tago}
\région{GOs BO}
\end{entrée}

\begin{entrée}
{arrêter de marcher ; arrêter}
\classe{v}
\vedette{kòò}\homonyme{2}
\sens{2}
\région{GOs}
\variante{%
\vedette{kòòl,kòl}
\région{PA BO WEM}}
\end{entrée}

\begin{entrée}
{arrêter ; garder ; retenir}
\vedette{thaaxôni}
\région{GOs}
\variante{%
\vedette{thaaxõni, thaxõni}
\région{BO}}
\variante{%
\vedette{tagòni}
\région{BO vx}}
\end{entrée}

\begin{entrée}
{arrêter ; interrompre}
\vedette{pwaadi}
\région{GOs}
\end{entrée}

\begin{entrée}
{attendre}
\classe{v}
\vedette{kòò}\homonyme{2}
\sens{3}
\région{GOs}
\variante{%
\vedette{kòòl,kòl}
\région{PA BO WEM}}
\end{entrée}

\begin{entrée}
{attendre}
\vedette{kubo}
\région{GOs PA}
\end{entrée}

\begin{entrée}
{attendre}
\vedette{ku-yabo}
\région{GOs}
\variante{%
\vedette{ku-yòbo}
\région{BO PA}}
\end{entrée}

\begin{entrée}
{avoir lieu (pour un événement fixé)}
\classe{v}
\vedette{ca}\homonyme{2}
\sens{2}
\région{GOs PA BO}
\end{entrée}

\begin{entrée}
{barrer ; empêcher}
\classe{v}
\vedette{oole}
\sens{2}
\région{PA BO}
\end{entrée}

\begin{entrée}
{bouger ; secouer (branches)}
\vedette{hili}\homonyme{2}
\région{GOs BO}
\end{entrée}

\begin{entrée}
{briser ; casser à moitié [BM]}
\vedette{pijoo}
\région{BO}
\end{entrée}

\begin{entrée}
{cabosser}
\vedette{pao-biini}
\région{PA}
\end{entrée}

\begin{entrée}
{casser (se)}
\vedette{phòzò}
\région{GOs}
\variante{%
\vedette{phòlò}
\région{PA}}
\end{entrée}

\begin{entrée}
{casser (verre, assiette)}
\vedette{hale}
\région{GOs}
\end{entrée}

\begin{entrée}
{casser (verre) ; éclater}
\vedette{gaaò}
\sens{1}
\région{PA BO WE}
\variante{%
\vedette{ha}
\région{GO(s)}}
\end{entrée}

\begin{entrée}
{ceindre ; serrer ; attacher (avec une corde) ; tendre (corde)}
\vedette{diixe}
\région{GOs PA}
\end{entrée}

\begin{entrée}
{chasser ; éloigner (des insectes)}
\classe{v.t.}
\vedette{ula}\homonyme{1}
\sens{1}
\région{GOs BO}
\end{entrée}

\begin{entrée}
{choisir ; trier}
\vedette{khêmèni}
\région{GOs}
\variante{%
\vedette{khemèn}
\région{PA BO}}
\end{entrée}

\begin{entrée}
{cogner (se) ; entrer en collision}
\classe{v}
\vedette{pe-bu}
\sens{2}
\région{GOs WEM}
\end{entrée}

\begin{entrée}
{coller}
\vedette{zaaloè}
\région{GOs PA}
\end{entrée}

\begin{entrée}
{couper ; barrer}
\classe{v}
\vedette{khibwaa}
\sens{1}
\région{GOs PA BO}
\end{entrée}

\begin{entrée}
{craquer}
\vedette{piga}
\région{PA}
\end{entrée}

\begin{entrée}
{créer ; faire qqch ; façonner}
\vedette{thaavwo}
\région{PA}
\end{entrée}

\begin{entrée}
{déchiré (être) ; cassé}
\vedette{mudra}
\région{GOs WEM}
\variante{%
\vedette{muda, môda}
\région{BO [BM]}}
\end{entrée}

\begin{entrée}
{déchirer (tissu)}
\vedette{pa-phwa-ni}
\région{GOs BO}
\end{entrée}

\begin{entrée}
{défaire (se) tout seul}
\vedette{hubo}
\région{GOs}
\end{entrée}

\begin{entrée}
{dégager ; faire place nette ; nettoyer (champ)}
\vedette{ulavwi}
\région{GOs}
\end{entrée}

\begin{entrée}
{dégonfler}
\vedette{phaa-bini}
\région{GOs}
\end{entrée}

\begin{entrée}
{démolir (toiture dela maison, etc.) ; enlever (la paille du toit)}
\vedette{thadi}
\région{GOs BO}
\end{entrée}

\begin{entrée}
{dévier qqch.}
\vedette{aaleni}
\région{PA}
\variante{%
\vedette{halèèni}
\région{BO [Corne]}}
\end{entrée}

\begin{entrée}
{dévier ; rater (une cible, qqch) ; déraper}
\vedette{tha}\homonyme{1}
\région{GOs}
\end{entrée}

\begin{entrée}
{devoir ; tâche}
\vedette{nobwò}
\région{GOs}
\variante{%
\vedette{nòbu}
\région{BO PA}}
\end{entrée}

\begin{entrée}
{disparaître ; perdre ; perdu ; absent}
\vedette{koèn}
\région{PA BO}
\variante{%
\vedette{koèèn,kwèèn}
\région{BO}}
\variante{%
\vedette{khoeo}
\région{BO}}
\end{entrée}

\begin{entrée}
{effleurer}
\vedette{celèng}
\région{PA BO [BM]}
\end{entrée}

\begin{entrée}
{effleurer ; frôler}
\classe{v}
\vedette{kã}
\sens{1}
\région{GOs}
\variante{%
\vedette{kãm}
\région{BO}}
\variante{%
\vedette{kham}
\région{PA}}
\end{entrée}

\begin{entrée}
{effleurer ; ricocher}
\vedette{tithaa}
\région{PA}
\end{entrée}

\begin{entrée}
{emporte-le ! ; sers-toi !}
\vedette{phe-vwo}
\région{GOs PA}
\end{entrée}

\begin{entrée}
{enlever}
\vedette{phuxa}
\région{PA}
\end{entrée}

\begin{entrée}
{envahir (de peur)}
\classe{v}
\vedette{kû}\homonyme{2}
\sens{2}
\région{GOs}
\end{entrée}

\begin{entrée}
{éparpiller ; semer la pagaille}
\classe{v}
\vedette{hooli}
\sens{2}
\région{PA BO}
\end{entrée}

\begin{entrée}
{érafler}
\classe{v}
\vedette{kã}
\sens{1}
\région{GOs}
\variante{%
\vedette{kãm}
\région{BO}}
\variante{%
\vedette{kham}
\région{PA}}
\end{entrée}

\begin{entrée}
{esquinter ; abimer}
\vedette{cazae}
\région{GOs}
\end{entrée}

\begin{entrée}
{étouffer}
\vedette{kôme}
\région{GOs}
\end{entrée}

\begin{entrée}
{exploser ; éclater}
\classe{v ; n}
\vedette{thi}\homonyme{3}
\sens{1}
\région{GOs PA BO}
\end{entrée}

\begin{entrée}
{fabriquer ; créer}
\vedette{thramwenge}
\région{GOs}
\end{entrée}

\begin{entrée}
{faire}
\vedette{po}\homonyme{2}
\région{GOs BO}
\variante{%
\vedette{pwò}
\région{BO}}
\variante{%
\vedette{thu}
\région{PA}}
\end{entrée}

\begin{entrée}
{faire ; agir}
\vedette{nòe, ne}
\région{GOs}
\variante{%
\vedette{nòe}
\région{BO}}
\variante{%
\vedette{ne}
\région{PA}}
\end{entrée}

\begin{entrée}
{faire avant ; commencer avant}
\vedette{teevwuun}
\région{PA}
\variante{%
\vedette{tee-puu-n}}
\end{entrée}

\begin{entrée}
{faire comme ceci}
\vedette{ne-wã le}
\région{GOs}
\end{entrée}

\begin{entrée}
{faire de la monnaie}
\classe{v}
\vedette{wône}
\sens{1}
\région{GOs}
\variante{%
\vedette{wene}
\région{PA}}
\end{entrée}

\begin{entrée}
{faire éclater (en jetant)}
\vedette{pao-kibi}
\région{PA}
\end{entrée}

\begin{entrée}
{faire ; effectuer}
\vedette{nee}\homonyme{2}
\région{GOs}
\variante{%
\vedette{ne, nee}
\région{BO PA}}
\end{entrée}

\begin{entrée}
{faire ensuite ; faire après}
\vedette{ne-mu}
\région{GOs}
\end{entrée}

\begin{entrée}
{faire osciller les arbres (vent)}
\vedette{ûûni}
\région{PA}
\end{entrée}

\begin{entrée}
{faire qqch ; actions ; actes}
\vedette{ne-vwo}
\région{GOs}
\variante{%
\vedette{nee-vwo}
\région{BO}}
\end{entrée}

\begin{entrée}
{faire tomber (qqch sur pied)}
\vedette{ûûni}
\région{PA}
\end{entrée}

\begin{entrée}
{fermer}
\classe{v ; n}
\vedette{thô}
\sens{1}
\région{GOs}
\variante{%
\vedette{thõn}
\région{BO}}
\end{entrée}

\begin{entrée}
{fermer ; boucher ; couvrir (boîte, marmite)}
\classe{v}
\vedette{kivwi}\homonyme{1}
\sens{1}
\région{GOs}
\variante{%
\vedette{kivhi}
\région{PA BO}}
\end{entrée}

\begin{entrée}
{fermer qqch (avec un objet, un couvercle)}
\classe{v.t.}
\vedette{thôni}
\sens{1}
\région{GOs WEM WE PA BO}
\end{entrée}

\begin{entrée}
{garder précieusement ; conserver}
\vedette{thaawe}
\région{PA}
\end{entrée}

\begin{entrée}
{gaspiller}
\vedette{khòraa}
\région{BO}
\end{entrée}

\begin{entrée}
{gaspiller (argent, nourriture) ; ne pas mériter}
\vedette{khòzole}
\région{GOs}
\variante{%
\vedette{khòzoole}}
\end{entrée}

\begin{entrée}
{glisser}
\classe{v}
\vedette{kã}
\sens{1}
\région{GOs}
\variante{%
\vedette{kãm}
\région{BO}}
\variante{%
\vedette{kham}
\région{PA}}
\end{entrée}

\begin{entrée}
{gonfler qqch}
\vedette{pa-phuu-ni}
\région{GOs}
\end{entrée}

\begin{entrée}
{gratter}
\classe{v}
\vedette{kòòli}
\sens{2}
\région{GOs BO}
\end{entrée}

\begin{entrée}
{gratter}
\classe{v ; n}
\vedette{kubi}
\sens{1}
\région{GOs PA BO}
\end{entrée}

\begin{entrée}
{il y a}
\vedette{po}\homonyme{2}
\région{GOs BO}
\variante{%
\vedette{pwò}
\région{BO}}
\variante{%
\vedette{thu}
\région{PA}}
\end{entrée}

\begin{entrée}
{laisser ; quitter}
\vedette{kalãnge}
\région{GOs}
\variante{%
\vedette{kala-kagee}
\région{PA}}
\end{entrée}

\begin{entrée}
{mettre bout à bout (et allonger)}
\vedette{pe-kine}
\région{GOs}
\end{entrée}

\begin{entrée}
{mettre de côté ; réserver}
\vedette{thu haal}
\région{PA}
\end{entrée}

\begin{entrée}
{mettre des obstacles ; entraver}
\classe{v}
\vedette{tigi}\homonyme{1}
\sens{2}
\région{GOs PA BO}
\variante{%
\vedette{tigin}
\région{WE}}
\end{entrée}

\begin{entrée}
{noyer (se)}
\vedette{mõxõ}
\région{GOs}
\variante{%
\vedette{mòòxòm}
\région{BO}}
\end{entrée}

\begin{entrée}
{obtenir}
\classe{v}
\vedette{phè}
\sens{3}
\région{GOs PA BO}
\variante{%
\vedette{phe}}
\end{entrée}

\begin{entrée}
{ôter ; enlever}
\vedette{ovwee}
\région{GOs}
\variante{%
\vedette{ovee}
\région{BO [BM]}}
\end{entrée}

\begin{entrée}
{perdu ; disparaître}
\vedette{kòyò}\homonyme{1}
\région{GOs BO PA}
\end{entrée}

\begin{entrée}
{poser}
\vedette{na}\homonyme{2}
\groupe{B}
\région{PA BO}
\end{entrée}

\begin{entrée}
{prendre la suite}
\vedette{phè bala}
\région{GOs}
\end{entrée}

\begin{entrée}
{préparatifs (faire les)}
\vedette{pavwa}
\région{GOs}
\end{entrée}

\begin{entrée}
{préparer (en général) ; faire des parts (vivres)}
\vedette{pavwange}
\région{GOs}
\variante{%
\vedette{pavang, pavange}
\région{BO}}
\end{entrée}

\begin{entrée}
{protéger ; préserver ; garder}
\vedette{thabila}
\région{GOs PA}
\end{entrée}

\begin{entrée}
{protéger (se)}
\vedette{kaba}\homonyme{1}
\région{GOs}
\end{entrée}

\begin{entrée}
{raccorder}
\vedette{pe-ki}
\région{GOs}
\end{entrée}

\begin{entrée}
{raccourcir}
\vedette{pa-pònume}
\région{PA BO}
\end{entrée}

\begin{entrée}
{ranger (faire bien)}
\vedette{nee-zo}
\région{GOs}
\end{entrée}

\begin{entrée}
{rater (cible) ; louper ; manquer}
\vedette{lao}
\région{PA BO}
\variante{%
\vedette{lau}
\région{BO}}
\end{entrée}

\begin{entrée}
{rater ; louper}
\vedette{pha-tha}
\région{GOs}
\variante{%
\vedette{pha-tha}
\région{PA}}
\end{entrée}

\begin{entrée}
{recevoir}
\classe{v}
\vedette{phè}
\sens{3}
\région{GOs PA BO}
\variante{%
\vedette{phe}}
\end{entrée}

\begin{entrée}
{redresser (fer)}
\vedette{mwããxe}\homonyme{1}
\région{GOs PA}
\end{entrée}

\begin{entrée}
{redresser (fer)}
\vedette{mwhãnge}
\région{GOs}
\end{entrée}

\begin{entrée}
{remplacer ; changer (vêtements)}
\classe{v}
\vedette{wône}
\sens{1}
\région{GOs}
\variante{%
\vedette{wene}
\région{PA}}
\end{entrée}

\begin{entrée}
{(r)emplir}
\classe{v}
\vedette{kû}\homonyme{2}
\sens{2}
\région{GOs}
\end{entrée}

\begin{entrée}
{remplir ; charger}
\vedette{phoo}
\région{GOs BO}
\variante{%
\vedette{phwoo}
\région{GO(s)}}
\end{entrée}

\begin{entrée}
{réparer}
\vedette{hòne}
\région{BO}
\end{entrée}

\begin{entrée}
{résister (à une épreuve) ; tenir le coup}
\vedette{kuee}
\région{GOs}
\end{entrée}

\begin{entrée}
{retourner (chemise, etc.) ; mettre à l'envers (linge)}
\vedette{ciròvwe}
\région{GOs}
\variante{%
\vedette{ciròvhe}
\région{PA BO}}
\end{entrée}

\begin{entrée}
{saccager}
\vedette{thadi}
\région{GOs BO}
\end{entrée}

\begin{entrée}
{sauver qqn (= ravir à la mort) ; sauver ; préserver (vie)}
\vedette{kae mõõxi}
\région{GOs PA}
\end{entrée}

\begin{entrée}
{sécher (au soleil)}
\vedette{haze}\homonyme{2}
\région{GOs}
\région{BO}
\variante{%
\vedette{hale}}
\end{entrée}

\begin{entrée}
{serré ; coincé}
\classe{v}
\vedette{hîmi}\homonyme{1}
\sens{1}
\région{GOs}
\end{entrée}

\begin{entrée}
{serrer}
\vedette{wööe}
\région{GOs}
\end{entrée}

\begin{entrée}
{sortir (de qqch, d'une poche, d'un panier)}
\classe{v}
\vedette{ii}\homonyme{1}
\sens{2}
\région{GOs PA BO}
\end{entrée}

\begin{entrée}
{sortir (d'un sac, etc.)}
\classe{v}
\vedette{yatre}
\sens{2}
\région{GOs}
\variante{%
\vedette{yare, yaare}
\région{GO BO}}
\end{entrée}

\begin{entrée}
{surveiller ; garder ; faire le guet}
\vedette{ku-hôboe}
\sens{1}
\région{GOs}
\variante{%
\vedette{ku-hôbwo}
\région{GO(s) BO}}
\end{entrée}

\begin{entrée}
{tailler en pointe}
\vedette{za mee}
\région{GOs}
\variante{%
\vedette{tha-mee}
\région{GO(s)}}
\end{entrée}

\begin{entrée}
{tordre (du fer)}
\vedette{mwããxe}\homonyme{2}
\région{BO [BM]}
\end{entrée}

\begin{entrée}
{toucher avec une pointe}
\classe{v}
\vedette{kòòli}
\sens{2}
\région{GOs BO}
\end{entrée}

\begin{entrée}
{toucher (une cible avec un projectile et éventuellement tuer)}
\classe{v}
\vedette{kibao}
\sens{1}
\région{GOs PA BO}
\end{entrée}

\begin{entrée}
{tourner ; retourner ; retourner (se) ; poser à l'envers}
\classe{v ; n}
\vedette{kênõ}
\sens{1}
\région{GOs}
\région{BO PA}
\variante{%
\vedette{kênõng}}
\end{entrée}

\begin{entrée}
{traîner (un cheval) ; emmener (personne)}
\vedette{kha-whili}
\région{GOs BO}
\end{entrée}

\begin{entrée}
{travail interrompu, non fini}
\vedette{bala-nyama}
\région{PA}
\end{entrée}

\begin{entrée}
{travailler ; travail}
\classe{v ; n}
\vedette{mõgu}
\sens{1}
\région{GOs BO}
\variante{%
\vedette{mwõgu}
\région{GO(s)}}
\end{entrée}

\begin{entrée}
{travailler ; travail}
\classe{v ; n}
\vedette{nyamã}
\sens{2}
\région{GOs PA BO}
\end{entrée}

\begin{entrée}
{trouer (ballon)}
\vedette{pa-phwa-ni}
\région{GOs BO}
\end{entrée}

\begin{entrée}
{trouver qqch}
\vedette{tròòli}
\région{GOs}
\variante{%
\vedette{tooli}
\région{PA}}
\end{entrée}

\begin{entrée}
{trouver qqch par hasard}
\vedette{bala-khazia}
\région{GOs}
\end{entrée}

\begin{entrée}
{trouver ; trouver (se) dans un état ; rencontrer}
\vedette{tròò}
\région{GOs WEM}
\variante{%
\vedette{tòò}
\région{BO}}
\end{entrée}

\subsubsection{Portage}

\begin{entrée}
{apporter}
\classe{v}
\vedette{phè}
\sens{2}
\région{GOs PA BO}
\variante{%
\vedette{phe}}
\end{entrée}

\begin{entrée}
{bât (cheval)}
\vedette{ba-phe-vwo}
\région{GOs}
\end{entrée}

\begin{entrée}
{bretelles de portage}
\vedette{khôô-keala}
\région{GOs}
\variante{%
\vedette{khôô-ke}
\région{GO}}
\end{entrée}

\begin{entrée}
{corde de portage (des fagots)}
\vedette{mõgavwo}
\région{GOs}
\variante{%
\vedette{mûgavo}
\région{PA}}
\end{entrée}

\begin{entrée}
{fardeau-ton}
\vedette{khooni-n}
\région{PA}
\end{entrée}

\begin{entrée}
{porte-bagages}
\vedette{ba-phe-vwo}
\région{GOs}
\end{entrée}

\begin{entrée}
{porter}
\classe{v}
\vedette{phè}
\sens{2}
\région{GOs PA BO}
\variante{%
\vedette{phe}}
\end{entrée}

\begin{entrée}
{porter qqch de lourd dans les bras}
\vedette{thaba}
\région{GOs PA BO}
\end{entrée}

\begin{entrée}
{porter sous le bras}
\vedette{hãbira}
\région{GOs}
\end{entrée}

\begin{entrée}
{porter sur le dos}
\vedette{thrõõbo}
\région{GOs}
\variante{%
\vedette{thõõbon}
\région{BO}}
\variante{%
\vedette{thõba, thõõbwa}
\région{PA WEM}}
\end{entrée}

\begin{entrée}
{porter sur l'épaule ; chargé}
\vedette{khoońe}\homonyme{1}
\région{GOsPA BO}
\end{entrée}

\begin{entrée}
{porter un enfant dans les bras (contre la poitrine)}
\vedette{bwaroe}
\région{GOs BO}
\end{entrée}

\begin{entrée}
{soulever}
\vedette{thaba}
\région{GOs PA BO}
\end{entrée}

\begin{entrée}
{tenir dans ses bras [BM]}
\vedette{pheege}
\région{BO}
\end{entrée}

\begin{entrée}
{transporter}
\vedette{khaû}
\région{PA BO [BM]}
\end{entrée}

\subsubsection{Mouvements ou actions faits avec le corps, les bras, les mains, les pieds}

\begin{entrée}
{abattre (animal)}
\vedette{baani}
\région{GOs PA BO}
\end{entrée}

\begin{entrée}
{abattre (arbre)}
\classe{v}
\vedette{threi}
\sens{1}
\région{GOs WEM}
\variante{%
\vedette{thei, thèi}
\région{BO PA}}
\end{entrée}

\begin{entrée}
{accouder (s')}
\vedette{ku-tibu}
\région{GOs}
\end{entrée}

\begin{entrée}
{accrocher (s') ; suspendre (se)}
\classe{v}
\vedette{biçô}
\sens{1}
\région{GOs}
\end{entrée}

\begin{entrée}
{accrocher ; suspendre qqch}
\vedette{paa-çôe}
\région{GOs WEM}
\variante{%
\vedette{pa-nyoî}
\région{BO}}
\end{entrée}

\begin{entrée}
{affûter ; affûté ;}
\vedette{yazoo}
\end{entrée}

\begin{entrée}
{agiter}
\classe{v}
\vedette{pao}
\sens{1}
\région{GOsPA BO}
\end{entrée}

\begin{entrée}
{agiter un objet contenant qqch (fait un son)}
\vedette{kègele}
\région{GOs PA}
\variante{%
\vedette{kègel}}
\end{entrée}

\begin{entrée}
{agripper (s')}
\vedette{khòlima}
\région{GOs}
\end{entrée}

\begin{entrée}
{aiguiser ;}
\vedette{yazoo}
\end{entrée}

\begin{entrée}
{allonger (le pas ou le bras pour saisir qqch)}
\vedette{tae}
\région{PA BO [BM]}
\end{entrée}

\begin{entrée}
{amener ; emmener}
\vedette{whili}
\région{GOs WEM BO PA}
\end{entrée}

\begin{entrée}
{aplatir}
\vedette{khaa-bîni}
\région{GOs}
\variante{%
\vedette{khaa-bîni}
\région{PA}}
\end{entrée}

\begin{entrée}
{apporter}
\vedette{phè-mi}
\région{GOs}
\end{entrée}

\begin{entrée}
{appuyer}
\classe{v}
\vedette{kha}\homonyme{3}
\groupe{A}
\région{GOs PA BO}
\variante{%
\vedette{khaa}
\région{PA}}
\end{entrée}

\begin{entrée}
{arquer}
\vedette{zugi}
\région{GOs PA}
\variante{%
\vedette{zhugi}
\région{GA}}
\variante{%
\vedette{yugi}
\région{BO}}
\end{entrée}

\begin{entrée}
{arracher (feuilles, lianes)}
\vedette{thagi}
\région{GOs WEM WE PA BO}
\variante{%
\vedette{t(h)agi}
\région{BO}}
\end{entrée}

\begin{entrée}
{arracher les poils}
\vedette{thagi}
\région{GOs WEM WE PA BO}
\variante{%
\vedette{t(h)agi}
\région{BO}}
\end{entrée}

\begin{entrée}
{attacher ; ceindre (manou)}
\vedette{phele}
\région{PA BO}
\end{entrée}

\begin{entrée}
{attacher (lacet, vêtement)}
\vedette{zoe}
\région{GOs}
\end{entrée}

\begin{entrée}
{attacher la tige d'igname}
\vedette{nhõî}
\région{GOs PA}
\variante{%
\vedette{nhõî, nhõõî}
\région{BO}}
\end{entrée}

\begin{entrée}
{attacher (qqch avec une corde temporairement)}
\vedette{thae}
\région{GOs PA BO}
\end{entrée}

\begin{entrée}
{attraper (en déplacement)}
\vedette{kha-tree-çimwi}
\région{GOs}
\end{entrée}

\begin{entrée}
{attraper (qqch en mouvement: ballon, à la pêche, des crabes, poissons)}
\vedette{êgi}
\région{GOs}
\end{entrée}

\begin{entrée}
{attraper ; saisir}
\vedette{tree-çimwî}
\région{GOs}
\variante{%
\vedette{tee-cimwî, tee-jimwî, tee-yimwî}
\région{PABO}}
\end{entrée}

\begin{entrée}
{baisser la tête}
\classe{v}
\vedette{bwagiloo}
\sens{1}
\région{GOs PA}
\end{entrée}

\begin{entrée}
{baisser la tête}
\vedette{kiluu}
\sens{1}
\région{GOs BO}
\variante{%
\vedette{ciluu}
\région{PA}}
\end{entrée}

\begin{entrée}
{baisser (se) ;}
\vedette{u}\homonyme{3}
\région{GOs BO}
\end{entrée}

\begin{entrée}
{balançoire ; balancer (se)}
\vedette{hiliçôô}
\région{GOs}
\variante{%
\vedette{yaoli}
\région{WEM WE}}
\end{entrée}

\begin{entrée}
{balançoire ; balancer (se)}
\vedette{yaoli}
\région{PA WEM}
\variante{%
\vedette{yauli}
\région{BO}}
\variante{%
\vedette{hiliçôô}
\région{GO(s)}}
\end{entrée}

\begin{entrée}
{bien appuyer les pieds pour marcher}
\classe{v}
\vedette{khadra}
\sens{1}
\région{GOs}
\variante{%
\vedette{kadae}
\région{BO}}
\end{entrée}

\begin{entrée}
{bloquer ; barrer (route) ; empêcher (de se déplacer)}
\vedette{thaaxô}
\région{GOs}
\variante{%
\vedette{thaxõõ}
\région{GO(s)}}
\end{entrée}

\begin{entrée}
{bousculer qqn}
\vedette{khaa-tia}
\région{GOs}
\variante{%
\vedette{khaa-zia}
\région{GOs}}
\end{entrée}

\begin{entrée}
{brandir dans la main}
\classe{v ; n}
\vedette{thatra-hi}
\sens{2}
\région{GOs}
\end{entrée}

\begin{entrée}
{cacher ; dissimuler qqch}
\classe{v}
\vedette{thozoe}
\sens{1}
\région{GOs}
\variante{%
\vedette{toroe}
\région{PA BO}}
\end{entrée}

\begin{entrée}
{cacher (se)}
\vedette{hiili}
\région{PA BO}
\variante{%
\vedette{iili}}
\end{entrée}

\begin{entrée}
{cacher (se)}
\classe{v}
\vedette{kari}
\sens{1}
\région{PA}
\end{entrée}

\begin{entrée}
{cacher (se)}
\classe{v.i.}
\vedette{ku-çaaxò}
\sens{1}
\région{GOs}
\variante{%
\vedette{ku-caaxò}
\région{PA BO}}
\end{entrée}

\begin{entrée}
{casser (du verre en jetant)}
\vedette{pao-gaaò}
\région{PA}
\end{entrée}

\begin{entrée}
{casser en jetant}
\vedette{pao-dale}
\région{PA}
\end{entrée}

\begin{entrée}
{casser en morceaux ; piler (verre, assiette, miroir)}
\vedette{kibii}
\région{GOs PA BO}
\variante{%
\vedette{khibii}
\région{BO [BM]}}
\end{entrée}

\begin{entrée}
{casser ; fendre (coco)}
\vedette{khi}\homonyme{2}
\région{GOs BO}
\variante{%
\vedette{khibi}
\région{PA}}
\end{entrée}

\begin{entrée}
{casser ; rompre (corde) ; déchirer}
\vedette{mudree}
\région{GOs}
\variante{%
\vedette{mudee, môdee}
\région{BO [BM]}}
\end{entrée}

\begin{entrée}
{chatouiller}
\vedette{komwãcii}
\région{GOs}
\end{entrée}

\begin{entrée}
{chercher à tâtons}
\classe{v}
\vedette{pale}
\sens{1}
\région{GOs BO PA}
\variante{%
\vedette{palee}
\région{BO}}
\end{entrée}

\begin{entrée}
{chercher (épouse)}
\vedette{whili}
\région{GOs WEM BO PA}
\end{entrée}

\begin{entrée}
{cogner ; frapper fort}
\vedette{thaçe}
\région{GOs}
\end{entrée}

\begin{entrée}
{cogner ; taper ; giffler}
\classe{v}
\vedette{cabi}
\sens{2}
\région{GOs PA BO}
\end{entrée}

\begin{entrée}
{conduire ; guider}
\vedette{whili}
\région{GOs WEM BO PA}
\end{entrée}

\begin{entrée}
{conduire (voiture) [PA]}
\vedette{whili}
\région{GOs WEM BO PA}
\end{entrée}

\begin{entrée}
{corde à torons tordus}
\vedette{bile}
\région{GOs BO}
\variante{%
\vedette{bire}
\région{GOs}}
\end{entrée}

\begin{entrée}
{corriger ; châtier}
\vedette{bòzi}
\région{GOs}
\variante{%
\vedette{bòli}
\région{BO PA}}
\end{entrée}

\begin{entrée}
{couper avec les dents ; déchirer avec les dents}
\vedette{hû-mudree}
\région{GOs}
\end{entrée}

\begin{entrée}
{couper d'un coup}
\classe{v}
\vedette{threi}
\sens{1}
\région{GOs WEM}
\variante{%
\vedette{thei, thèi}
\région{BO PA}}
\end{entrée}

\begin{entrée}
{couper (en deux) ; fendre (coprah)}
\vedette{khi-drale}
\région{GOs}
\variante{%
\vedette{khi-dale}
\région{PA}}
\end{entrée}

\begin{entrée}
{couper en deux ; partager}
\vedette{pii}\homonyme{1}
\région{GOs BO PA}
\end{entrée}

\begin{entrée}
{couper (faire)}
\vedette{dei}
\région{PA BO [Corne]}
\variante{%
\vedette{deei}}
\end{entrée}

\begin{entrée}
{couper (se)}
\classe{v.t.}
\vedette{zòi}
\sens{2}
\région{GOs PA BO}
\variante{%
\vedette{zhòi}
\région{GO(s)}}
\end{entrée}

\begin{entrée}
{couper (viande)}
\classe{v.t.}
\vedette{zòi}
\sens{2}
\région{GOs PA BO}
\variante{%
\vedette{zhòi}
\région{GO(s)}}
\end{entrée}

\begin{entrée}
{courber}
\vedette{zugi}
\région{GOs PA}
\variante{%
\vedette{zhugi}
\région{GA}}
\variante{%
\vedette{yugi}
\région{BO}}
\end{entrée}

\begin{entrée}
{courber (se)}
\classe{v}
\vedette{bwagiloo}
\sens{1}
\région{GOs PA}
\end{entrée}

\begin{entrée}
{courber (se)}
\vedette{kiluu}
\sens{1}
\région{GOs BO}
\variante{%
\vedette{ciluu}
\région{PA}}
\end{entrée}

\begin{entrée}
{courber (se) (par ex. pour entrer dans une maison)}
\vedette{u}\homonyme{3}
\région{GOs BO}
\end{entrée}

\begin{entrée}
{courber ; tordre}
\vedette{phwô}
\région{GOs}
\end{entrée}

\begin{entrée}
{cramponner}
\vedette{khòlima}
\région{GOs}
\end{entrée}

\begin{entrée}
{creuser ;}
\classe{v}
\vedette{taa}
\sens{1}
\région{GOs PA BO}
\end{entrée}

\begin{entrée}
{creuser (pour contrôler l'état des tubercules)}
\vedette{cöi}
\région{GOs PA BO}
\end{entrée}

\begin{entrée}
{creuser (trou)}
\vedette{cöi}
\région{GOs PA BO}
\end{entrée}

\begin{entrée}
{creuser un trou pour planter l'igname}
\vedette{taa phwa}
\région{GOs}
\variante{%
\vedette{taa-vwa}}
\end{entrée}

\begin{entrée}
{déchirer en long}
\vedette{the}
\région{GOs}
\variante{%
\vedette{tioo}
\région{PA}}
\end{entrée}

\begin{entrée}
{déchirer en marchant}
\vedette{kha-mudree}
\région{GOs}
\variante{%
\vedette{kha-mude}
\région{PA}}
\end{entrée}

\begin{entrée}
{déchirer ; faire un accroc}
\classe{v}
\vedette{tio}
\sens{1}
\région{PA}
\variante{%
\vedette{tioo}
\région{PA}}
\end{entrée}

\begin{entrée}
{déchirer (tissu) [BM]}
\vedette{teol}
\région{BO}
\variante{%
\vedette{teo}
\région{BO}}
\end{entrée}

\begin{entrée}
{déchirer ; trouer (linge)}
\vedette{pa-modee}
\région{GOs}
\end{entrée}

\begin{entrée}
{déclouer}
\classe{v}
\vedette{udi}
\sens{2}
\région{GOs BO}
\end{entrée}

\begin{entrée}
{décrocher en piquant avec un bâton (qqch qui se trouve en hauteur)}
\vedette{thi-ula}
\région{PA}
\end{entrée}

\begin{entrée}
{démêler}
\vedette{bwawe}
\région{GOs}
\end{entrée}

\begin{entrée}
{dénouer (corde) ; défaire (noeud) ; relâcher un peu}
\vedette{thaò}
\région{GOs BO}
\variante{%
\vedette{thawa}
\région{GO(s)}}
\end{entrée}

\begin{entrée}
{dénouer ; défaire (noeud) ; détacher}
\vedette{tua}
\région{GOs BO PA}
\end{entrée}

\begin{entrée}
{déplier}
\vedette{thròlòe}
\région{GOs}
\variante{%
\vedette{thòlòe}
\région{PA}}
\variante{%
\vedette{tòloè}
\région{BO (Corne)}}
\variante{%
\vedette{tholòè, tòlee}
\région{BO (BM)}}
\variante{%
\vedette{throleitholei}
\région{WEM}}
\end{entrée}

\begin{entrée}
{déployer (membre)}
\vedette{pwalee}
\région{GOs}
\end{entrée}

\begin{entrée}
{déposer [BO]}
\classe{v}
\vedette{pa-nuã}
\sens{1}
\région{GOs}
\variante{%
\vedette{pa-nhuã}
\région{PA BO}}
\end{entrée}

\begin{entrée}
{détacher}
\vedette{ubòl}
\région{PA BO [Corne]}
\end{entrée}

\begin{entrée}
{donner un coup}
\classe{v}
\vedette{tòè}
\sens{2}
\région{GOs PA BO}
\end{entrée}

\begin{entrée}
{dresser (poteau, etc.)}
\classe{v}
\vedette{khabe}
\sens{1}
\région{GOs PA BO}
\end{entrée}

\begin{entrée}
{écarter ; sortir (de son habitacle)}
\vedette{hò}\homonyme{2}
\région{GOs}
\end{entrée}

\begin{entrée}
{écorcher (s') la peau}
\classe{v.t.}
\vedette{zòli}
\sens{3}
\région{GOs PA}
\variante{%
\vedette{zhòli}
\région{GO(s)}}
\variante{%
\vedette{yòli, yòòli}
\région{BO}}
\end{entrée}

\begin{entrée}
{écraser}
\classe{v}
\vedette{kha}\homonyme{3}
\groupe{A}
\région{GOs PA BO}
\variante{%
\vedette{khaa}
\région{PA}}
\end{entrée}

\begin{entrée}
{écraser (avec le pied)}
\vedette{khaa-bîni}
\région{GOs}
\variante{%
\vedette{khaa-bîni}
\région{PA}}
\end{entrée}

\begin{entrée}
{écraser avec le pied (en marchant)}
\vedette{kha-nhyale}
\région{GOs}
\end{entrée}

\begin{entrée}
{écraser (dans la main) ; ramollir}
\classe{v}
\vedette{nhyale}
\sens{1}
\région{GOs BO PA}
\end{entrée}

\begin{entrée}
{effacer}
\vedette{hiili}
\région{PA BO}
\variante{%
\vedette{iili}}
\end{entrée}

\begin{entrée}
{effacer}
\vedette{hijini}
\région{GOs BO}
\variante{%
\vedette{hiliini}
\région{BO}}
\end{entrée}

\begin{entrée}
{effleurer}
\classe{v}
\vedette{pale}
\sens{1}
\région{GOs BO PA}
\variante{%
\vedette{palee}
\région{BO}}
\end{entrée}

\begin{entrée}
{effleurer (qqch) ; frôler}
\vedette{chele}\homonyme{1}
\région{GOs PA}
\end{entrée}

\begin{entrée}
{embrasser}
\vedette{ma}
\région{BO}
\end{entrée}

\begin{entrée}
{emmêler (s')dans une corde ;}
\vedette{tagi}
\sens{2}
\région{PA BO}
\end{entrée}

\begin{entrée}
{empêcher (bagarre) ; arrêter (qqn)}
\vedette{thaaxô}
\région{GOs}
\variante{%
\vedette{thaxõõ}
\région{GO(s)}}
\end{entrée}

\begin{entrée}
{empiler}
\vedette{pe-na bwa}
\région{GOs}
\end{entrée}

\begin{entrée}
{emporter}
\vedette{kha-phe}
\groupe{A}
\région{GOs PA BO}
\variante{%
\vedette{kha-vwe}
\région{GO(s)}}
\end{entrée}

\begin{entrée}
{emporter}
\classe{v}
\vedette{phè}
\sens{1}
\région{GOs PA BO}
\variante{%
\vedette{phe}}
\end{entrée}

\begin{entrée}
{enduire}
\vedette{thrîmi}
\région{GOs}
\variante{%
\vedette{thimi}
\région{PA BO}}
\end{entrée}

\begin{entrée}
{enfoncer}
\classe{v}
\vedette{khabe}
\sens{1}
\région{GOs PA BO}
\end{entrée}

\begin{entrée}
{enlever les branches latérales d'un tronc (avec un couteau, tamioc)}
\vedette{yölae}
\région{PA}
\end{entrée}

\begin{entrée}
{enlever (natte) ;}
\classe{v}
\vedette{zali}
\sens{1}
\région{GOs}
\variante{%
\vedette{zhali}
\région{GA}}
\end{entrée}

\begin{entrée}
{enrouler}
\vedette{bile}
\région{GOs BO}
\variante{%
\vedette{bire}
\région{GOs}}
\end{entrée}

\begin{entrée}
{enrouler (s') (une liane autour d'un arbre, la tige d'igname autour du tuteur) ;}
\vedette{tagi}
\sens{2}
\région{PA BO}
\end{entrée}

\begin{entrée}
{enrouler (tissu, natte)}
\vedette{bîni}
\région{GOs}
\variante{%
\vedette{biini}
\région{PA}}
\variante{%
\vedette{bîni}
\région{BO}}
\end{entrée}

\begin{entrée}
{entasser}
\vedette{hivwi}\homonyme{2}
\région{GOs PA}
\variante{%
\vedette{hivi}
\région{BO}}
\variante{%
\vedette{hipi}
\région{BO vx}}
\end{entrée}

\begin{entrée}
{entasser}
\vedette{phaò}
\région{GOs}
\end{entrée}

\begin{entrée}
{entasser}
\classe{v}
\vedette{tha-ivwi}
\sens{1}
\région{GOs PA}
\variante{%
\vedette{thaiving}
\région{PA BO}}
\end{entrée}

\begin{entrée}
{entasser ; faire des tas}
\vedette{phöge}
\région{GOs}
\end{entrée}

\begin{entrée}
{entasser ; rassembler}
\vedette{tha-bulu-ni}
\région{GOs}
\end{entrée}

\begin{entrée}
{enterrer qqch ; mettre en terre}
\classe{v}
\vedette{khêmi}
\sens{1}
\région{GOs PA}
\variante{%
\vedette{kêmi}
\région{PA BO}}
\end{entrée}

\begin{entrée}
{envelopper}
\classe{v}
\vedette{thaabwe}
\sens{2}
\région{GOs BO}
\variante{%
\vedette{thaaboi, thabui, thabwi}
\région{GO(s) WEM BO}}
\end{entrée}

\begin{entrée}
{épointer (un bout de bois)}
\vedette{zamee}
\région{GOs}
\end{entrée}

\begin{entrée}
{épouiller ; chercher les poux}
\vedette{phaa-cii}
\région{GOs}
\variante{%
\vedette{phaa-çi}
\région{GO(s)}}
\end{entrée}

\begin{entrée}
{érafler ; écorcher (s')}
\classe{v}
\vedette{tio}
\sens{2}
\région{PA}
\variante{%
\vedette{tioo}
\région{PA}}
\end{entrée}

\begin{entrée}
{érafler (s') ; égratigner (s')}
\vedette{tûû}
\région{GOs}
\end{entrée}

\begin{entrée}
{essorer}
\vedette{bilòò}
\région{GOs BO PA}
\end{entrée}

\begin{entrée}
{essorer ;}
\vedette{chińõõ}
\région{GOs PA}
\end{entrée}

\begin{entrée}
{essorer ; presser (fruit)}
\vedette{mhôwe}
\région{GOs}
\end{entrée}

\begin{entrée}
{essuyer}
\vedette{tûûne}
\région{GOs PA BO}
\end{entrée}

\begin{entrée}
{essuyer (s') les fesses}
\vedette{thizi}
\région{GOs}
\end{entrée}

\begin{entrée}
{étaler (natte)}
\vedette{pwalee}
\région{GOs}
\end{entrée}

\begin{entrée}
{étaler (natte, etc.)}
\vedette{thròlòe}
\région{GOs}
\variante{%
\vedette{thòlòe}
\région{PA}}
\variante{%
\vedette{tòloè}
\région{BO (Corne)}}
\variante{%
\vedette{tholòè, tòlee}
\région{BO (BM)}}
\variante{%
\vedette{throleitholei}
\région{WEM}}
\end{entrée}

\begin{entrée}
{étaler (sable)}
\vedette{yaweeni}
\région{GOs}
\end{entrée}

\begin{entrée}
{étaler (s') (pour des plantes)}
\vedette{thròlòe}
\région{GOs}
\variante{%
\vedette{thòlòe}
\région{PA}}
\variante{%
\vedette{tòloè}
\région{BO (Corne)}}
\variante{%
\vedette{tholòè, tòlee}
\région{BO (BM)}}
\variante{%
\vedette{throleitholei}
\région{WEM}}
\end{entrée}

\begin{entrée}
{étendre (bras)}
\vedette{tae}
\région{PA BO [BM]}
\end{entrée}

\begin{entrée}
{étendre (bras, etc.)}
\vedette{thròlòe}
\région{GOs}
\variante{%
\vedette{thòlòe}
\région{PA}}
\variante{%
\vedette{tòloè}
\région{BO (Corne)}}
\variante{%
\vedette{tholòè, tòlee}
\région{BO (BM)}}
\variante{%
\vedette{throleitholei}
\région{WEM}}
\end{entrée}

\begin{entrée}
{étendre la main horizontalement (comme pour tapoter) [Corne]}
\vedette{khaboi}
\région{BO}
\end{entrée}

\begin{entrée}
{étendre (natte)}
\vedette{pwalee}
\région{GOs}
\end{entrée}

\begin{entrée}
{étirer (s')}
\vedette{chaxe}
\région{PA BO [BM]}
\end{entrée}

\begin{entrée}
{étrangler (avec les mains)}
\vedette{bizi}
\région{GOs}
\variante{%
\vedette{biri}
\région{BO}}
\end{entrée}

\begin{entrée}
{étrangler (avec une corde ou une liane)}
\vedette{wizi}
\région{GOs}
\variante{%
\vedette{wili}
\région{PA}}
\end{entrée}

\begin{entrée}
{étrangler (serrer le cou)}
\vedette{weze nõ}
\région{GOs}
\end{entrée}

\begin{entrée}
{éventer (s')}
\classe{v.t.}
\vedette{ula}\homonyme{1}
\sens{2}
\région{GOs BO}
\end{entrée}

\begin{entrée}
{extraire}
\classe{v}
\vedette{yatre}
\sens{1}
\région{GOs}
\variante{%
\vedette{yare, yaare}
\région{GO BO}}
\end{entrée}

\begin{entrée}
{extraire (épine)}
\classe{v}
\vedette{thii}\homonyme{1}
\sens{1}
\région{GOs PA BO WEM}
\end{entrée}

\begin{entrée}
{extraire (épine, etc.) ; sortir}
\classe{v}
\vedette{udi}
\sens{2}
\région{GOs BO}
\end{entrée}

\begin{entrée}
{faire des lamelles de feuilles de pandanus}
\vedette{töö-pho}
\région{GOs}
\end{entrée}

\begin{entrée}
{faire fuir (animal)}
\vedette{baani}
\région{GOs PA BO}
\end{entrée}

\begin{entrée}
{faire un croc-en-jambes}
\vedette{wazale kòò}
\région{GOs}
\variante{%
\vedette{wè-rali kòò}
\région{BO}}
\end{entrée}

\begin{entrée}
{faire un peu de place (quand on est assis)}
\vedette{pò-tree-hu-ò}
\région{GOs}
\variante{%
\vedette{pwo tree-wò}
\région{GOs}}
\end{entrée}

\begin{entrée}
{faufiler (se)}
\classe{v}
\vedette{yöö}
\sens{2}
\région{GOs PA}
\end{entrée}

\begin{entrée}
{fendre ; casser}
\vedette{drale}\homonyme{1}
\région{GOs}
\variante{%
\vedette{dale, daale}
\région{GO(s)}}
\end{entrée}

\begin{entrée}
{fendre et couper en morceaux (citrouille)}
\vedette{tha-nyale}
\région{GOs}
\end{entrée}

\begin{entrée}
{fermer (bouche, main)}
\vedette{hîmi}\homonyme{2}
\région{GOs}
\variante{%
\vedette{hòmi}
\région{PA}}
\end{entrée}

\begin{entrée}
{fermer la main [Corne]}
\vedette{hêêbwi}
\région{BO}
\end{entrée}

\begin{entrée}
{fermer ; pincer}
\vedette{homi}
\région{PA BO}
\end{entrée}

\begin{entrée}
{filer}
\vedette{bile}
\région{GOs BO}
\variante{%
\vedette{bire}
\région{GOs}}
\end{entrée}

\begin{entrée}
{fouetter}
\vedette{bòzi}
\région{GOs}
\variante{%
\vedette{bòli}
\région{BO PA}}
\end{entrée}

\begin{entrée}
{fouiller ; chercher ; tenter}
\vedette{khila}
\région{GOs BO PA}
\end{entrée}

\begin{entrée}
{fouiller (dans les affaires des autres) ; se mêler (de ce qui ne vous regarde pas)[Corne]}
\vedette{weya}
\région{BO}
\end{entrée}

\begin{entrée}
{fouiller (dans un sac, une poche)}
\vedette{whòi}
\région{GOs PA}
\end{entrée}

\begin{entrée}
{fouiller (dans un trou avec un bâton)}
\classe{v}
\vedette{thii}\homonyme{1}
\sens{1}
\région{GOs PA BO WEM}
\end{entrée}

\begin{entrée}
{fouler (se) (pied, cheville)}
\vedette{kaabilò}
\région{GOs}
\end{entrée}

\begin{entrée}
{fracasser (en jetant)}
\vedette{pao-khi-dale}
\région{PA}
\end{entrée}

\begin{entrée}
{frapper}
\vedette{baa}\homonyme{3}
\région{GOs}
\end{entrée}

\begin{entrée}
{frapper}
\classe{v}
\vedette{cabi}
\sens{1}
\région{GOs PA BO}
\end{entrée}

\begin{entrée}
{frapper (de haut en bas)}
\vedette{baani}
\région{GOs PA BO}
\end{entrée}

\begin{entrée}
{frapper (enfant)}
\vedette{pao-thali}
\région{PA}
\end{entrée}

\begin{entrée}
{frapper fort}
\vedette{thaji}
\région{GOs}
\end{entrée}

\begin{entrée}
{frapper (qqn pour le tuer)}
\classe{v}
\vedette{threi}
\sens{1}
\région{GOs WEM}
\variante{%
\vedette{thei, thèi}
\région{BO PA}}
\end{entrée}

\begin{entrée}
{frapper ; taper (une cible)}
\vedette{bu}\homonyme{4}
\région{GOs PA}
\end{entrée}

\begin{entrée}
{frapper ; tuer}
\vedette{kûbu}
\région{GOs PA BO}
\variante{%
\vedette{khûbu}
\région{BO}}
\end{entrée}

\begin{entrée}
{frictionner (avec des plantes ou des pommades)}
\vedette{urîni}
\région{PA}
\end{entrée}

\begin{entrée}
{frotter}
\vedette{he}
\région{GOs PA BO}
\end{entrée}

\begin{entrée}
{frotter}
\vedette{urîni}
\région{PA}
\end{entrée}

\begin{entrée}
{frotter ;}
\vedette{yazoo}
\end{entrée}

\begin{entrée}
{frotter (linge, mains, etc.) ;}
\vedette{chińõõ}
\région{GOs PA}
\end{entrée}

\begin{entrée}
{gratter (la terre, comme les poules)}
\vedette{kiri}
\région{PA BO}
\end{entrée}

\begin{entrée}
{gratter (se) ; gratter ; griffer}
\vedette{yawi}
\région{GOs PA BO}
\variante{%
\vedette{yawe}
\région{GO(s)}}
\end{entrée}

\begin{entrée}
{gratter (terre)}
\vedette{cöi}
\région{GOs PA BO}
\end{entrée}

\begin{entrée}
{griffer}
\classe{v.t.}
\vedette{zòli}
\sens{3}
\région{GOs PA}
\variante{%
\vedette{zhòli}
\région{GO(s)}}
\variante{%
\vedette{yòli, yòòli}
\région{BO}}
\end{entrée}

\begin{entrée}
{heurter}
\classe{v}
\vedette{cabi}
\sens{1}
\région{GOs PA BO}
\end{entrée}

\begin{entrée}
{jeter}
\vedette{pa-trevwaò}
\région{GOs}
\variante{%
\vedette{pa-tevwaò}
\région{PA BO}}
\end{entrée}

\begin{entrée}
{jeter en l'air ; lancer}
\classe{v}
\vedette{pao}
\sens{1}
\région{GOsPA BO}
\end{entrée}

\begin{entrée}
{jeter ; lancer ; frapper (de haut en bas)}
\vedette{pawe}\homonyme{1}
\région{GOs BO}
\variante{%
\vedette{paawe}
\région{BO}}
\end{entrée}

\begin{entrée}
{jeter par dessus ; lancer par dessus}
\vedette{pao-kaö}
\région{PA}
\end{entrée}

\begin{entrée}
{jongler (par ex. avec des oranges sauvages, occupation dans le pays des morts)}
\vedette{pe-hatra}
\région{GOs}
\variante{%
\vedette{pe-hara}
\région{GO(s) PA WEM}}
\variante{%
\vedette{haram}
\région{BO}}
\end{entrée}

\begin{entrée}
{lâcher}
\vedette{ubòl}
\région{PA BO [Corne]}
\end{entrée}

\begin{entrée}
{lâcher ; laisser sortir (prisonnier)}
\classe{v}
\vedette{pa-nuã}
\sens{1}
\région{GOs}
\variante{%
\vedette{pa-nhuã}
\région{PA BO}}
\end{entrée}

\begin{entrée}
{lâcher ; relâcher}
\vedette{nhuã}
\région{GOs}
\variante{%
\vedette{nuã}}
\variante{%
\vedette{nhuã, nuã}
\région{BO PA}}
\end{entrée}

\begin{entrée}
{laisser}
\vedette{ubòl}
\région{PA BO [Corne]}
\end{entrée}

\begin{entrée}
{laisser tomber}
\vedette{nhuã}
\région{GOs}
\variante{%
\vedette{nuã}}
\variante{%
\vedette{nhuã, nuã}
\région{BO PA}}
\end{entrée}

\begin{entrée}
{laisser tomber}
\classe{v}
\vedette{pa-nuã}
\sens{1}
\région{GOs}
\variante{%
\vedette{pa-nhuã}
\région{PA BO}}
\end{entrée}

\begin{entrée}
{lancer}
\classe{v}
\vedette{pha}\homonyme{1}
\sens{1}
\région{GOs BO}
\end{entrée}

\begin{entrée}
{lancer fort}
\vedette{thaçe}
\région{GOs}
\end{entrée}

\begin{entrée}
{lever le bras en menaçant}
\vedette{hãnge}\homonyme{1}
\région{PA}
\end{entrée}

\begin{entrée}
{lever ; soulever (des choses lourdes) ; élever}
\vedette{cooge}
\région{GOs BO}
\variante{%
\vedette{coge}
\région{BO}}
\end{entrée}

\begin{entrée}
{lier ; ligoter ; attacher}
\vedette{nhõî}
\région{GOs PA}
\variante{%
\vedette{nhõî, nhõõî}
\région{BO}}
\end{entrée}

\begin{entrée}
{limer ;}
\vedette{yazoo}
\end{entrée}

\begin{entrée}
{masser}
\classe{v}
\vedette{kha}\homonyme{3}
\groupe{A}
\région{GOs PA BO}
\variante{%
\vedette{khaa}
\région{PA}}
\end{entrée}

\begin{entrée}
{masser}
\vedette{urîni}
\région{PA}
\end{entrée}

\begin{entrée}
{mélanger ; tourner}
\vedette{hòzi}
\région{GOs}
\end{entrée}

\begin{entrée}
{mélanger ; tourner (dans la marmite)}
\classe{v}
\vedette{hooli}
\sens{1}
\région{PA BO}
\end{entrée}

\begin{entrée}
{mémoriser ; retenir}
\vedette{tree-çimwî}
\région{GOs}
\variante{%
\vedette{tee-cimwî, tee-jimwî, tee-yimwî}
\région{PABO}}
\end{entrée}

\begin{entrée}
{mener (travail)}
\vedette{whili}
\région{GOs WEM BO PA}
\end{entrée}

\begin{entrée}
{mettre}
\classe{v}
\vedette{naa}\homonyme{3}
\sens{2}
\région{GOs}
\variante{%
\vedette{na, ne}
\région{PA BO}}
\end{entrée}

\begin{entrée}
{mettre debout (se)}
\vedette{cabo kò}
\région{GOs}
\variante{%
\vedette{cabwo kò}
\région{GO(s)}}
\variante{%
\vedette{cabwò kòl}
\région{BO}}
\end{entrée}

\begin{entrée}
{montrer}
\vedette{hîńõ}
\groupe{B}
\région{GOs PA}
\variante{%
\vedette{hinõn}
\région{BO}}
\end{entrée}

\begin{entrée}
{montrer}
\vedette{phããde}
\région{GOs BO PA}
\variante{%
\vedette{phãde}
\région{GO(s) PA}}
\end{entrée}

\begin{entrée}
{moudre}
\vedette{bilòò}
\région{GOs BO PA}
\end{entrée}

\begin{entrée}
{nager}
\vedette{zòò}\homonyme{2}
\sens{1}
\région{GOs}
\variante{%
\vedette{zhòò}
\région{GO(s)}}
\variante{%
\vedette{zòòm}
\région{PA}}
\variante{%
\vedette{yhòòm, yòòm, yòò}
\région{BO}}
\end{entrée}

\begin{entrée}
{nettoyer}
\vedette{tûûne}
\région{GOs PA BO}
\end{entrée}

\begin{entrée}
{ôter}
\classe{v}
\vedette{yatre}
\sens{1}
\région{GOs}
\variante{%
\vedette{yare, yaare}
\région{GO BO}}
\end{entrée}

\begin{entrée}
{ouvrir (livre)}
\vedette{pwalee}
\région{GOs}
\end{entrée}

\begin{entrée}
{ouvrir (maison, boîte, marmite, etc.) ; déboucher (bouteille) ; enlever (couvercle)}
\vedette{thala}
\région{GOs BO PA}
\end{entrée}

\begin{entrée}
{ouvrir une enveloppe (bougna)}
\vedette{kêdi}
\région{GOs}
\end{entrée}

\begin{entrée}
{palper}
\classe{v}
\vedette{pale}
\sens{1}
\région{GOs BO PA}
\variante{%
\vedette{palee}
\région{BO}}
\end{entrée}

\begin{entrée}
{peindre}
\vedette{thrîmi}
\région{GOs}
\variante{%
\vedette{thimi}
\région{PA BO}}
\end{entrée}

\begin{entrée}
{pelleter}
\classe{v}
\vedette{khee}\homonyme{1}
\sens{2}
\région{GOs}
\end{entrée}

\begin{entrée}
{pencher (se) ;}
\vedette{u}\homonyme{3}
\région{GOs BO}
\end{entrée}

\begin{entrée}
{percer ; faire des trous; ronger}
\vedette{uxi}
\région{GOs}
\end{entrée}

\begin{entrée}
{pétrir (pain)}
\vedette{peeni}
\région{GOs}
\end{entrée}

\begin{entrée}
{piler ; écraser}
\vedette{thaa-bi}
\région{GOs}
\end{entrée}

\begin{entrée}
{pincer}
\classe{v ; n}
\vedette{huu}\homonyme{1}
\sens{3}
\région{GOs PA BO}
\variante{%
\vedette{whuu}
\région{GO(s)}}
\end{entrée}

\begin{entrée}
{pincer}
\classe{v}
\vedette{hu-vo}\homonyme{1}
\sens{3}
\région{PA}
\variante{%
\vedette{hu-po}
\région{PA}}
\end{entrée}

\begin{entrée}
{pincer entre les doigts}
\vedette{coboe}
\région{GOs}
\variante{%
\vedette{cobwoi}
\région{BO (Corne, BM)}}
\end{entrée}

\begin{entrée}
{piquer}
\classe{v ; n}
\vedette{huu}\homonyme{1}
\sens{3}
\région{GOs PA BO}
\variante{%
\vedette{whuu}
\région{GO(s)}}
\end{entrée}

\begin{entrée}
{piquer avec la sagaie}
\classe{v}
\vedette{tòè}
\sens{2}
\région{GOs PA BO}
\end{entrée}

\begin{entrée}
{piquer ; faire une piqûre}
\classe{v}
\vedette{thii}\homonyme{1}
\sens{1}
\région{GOs PA BO WEM}
\end{entrée}

\begin{entrée}
{plier (linge)}
\vedette{bîni}
\région{GOs}
\variante{%
\vedette{biini}
\région{PA}}
\variante{%
\vedette{bîni}
\région{BO}}
\end{entrée}

\begin{entrée}
{plonger la main pour creuser}
\classe{v}
\vedette{taa}
\sens{1}
\région{GOs PA BO}
\end{entrée}

\begin{entrée}
{plonger le bras (dans une cavité, dans l'obscurité)}
\vedette{thi-du}
\région{GOs BO[Corne]}
\end{entrée}

\begin{entrée}
{ployer}
\vedette{zugi}
\région{GOs PA}
\variante{%
\vedette{zhugi}
\région{GA}}
\variante{%
\vedette{yugi}
\région{BO}}
\end{entrée}

\begin{entrée}
{plumer (volaille)}
\vedette{thagi}
\région{GOs WEM WE PA BO}
\variante{%
\vedette{t(h)agi}
\région{BO}}
\end{entrée}

\begin{entrée}
{polir ;}
\vedette{yazoo}
\end{entrée}

\begin{entrée}
{porter qqch de lourd dans les bras}
\vedette{thaba}
\région{GOs PA BO}
\end{entrée}

\begin{entrée}
{poser}
\classe{v}
\vedette{naa}\homonyme{3}
\sens{2}
\région{GOs}
\variante{%
\vedette{na, ne}
\région{PA BO}}
\end{entrée}

\begin{entrée}
{pousser et faire tomber à la renverse [BM]}
\vedette{tha-uji}
\région{BO}
\variante{%
\vedette{taauji}
\région{BO}}
\end{entrée}

\begin{entrée}
{pousser horizontalement ; pousser qqn à faire qqch}
\vedette{tia}
\région{GOs BO}
\variante{%
\vedette{tiia}
\région{GO(s)}}
\end{entrée}

\begin{entrée}
{pousser ; précipiter qqch dans}
\vedette{kibwa}
\région{BO}
\end{entrée}

\begin{entrée}
{pousser (qqch avec la main)}
\vedette{ele}
\région{GOs BO}
\end{entrée}

\begin{entrée}
{pousser (qqn)}
\vedette{thaa-zia}
\région{GOs}
\variante{%
\vedette{tha-tia}
\région{BO}}
\end{entrée}

\begin{entrée}
{pousser (se) un peu}
\vedette{pò-tree-hu-ò}
\région{GOs}
\variante{%
\vedette{pwo tree-wò}
\région{GOs}}
\end{entrée}

\begin{entrée}
{pousser (se) un peu (debout) ; faire un peu de place (quand on est debout)}
\vedette{pò-ku-hu-ò}
\région{GOs}
\variante{%
\vedette{pwo ku-wò}
\région{GOs}}
\end{entrée}

\begin{entrée}
{pousser (se) un peu ; faire un peu de place (quand on est couché)}
\vedette{pò-kô-hu-ò}
\région{GOs}
\variante{%
\vedette{pwo kô-wò}
\région{GOs}}
\end{entrée}

\begin{entrée}
{prendre au collet}
\vedette{côî}
\région{GOs}
\variante{%
\vedette{chôî}
\région{BO (Corne)}}
\end{entrée}

\begin{entrée}
{prendre dans ses bras [Corne]}
\vedette{ma}
\région{BO}
\end{entrée}

\begin{entrée}
{prendre en cachette}
\vedette{phè-caaxo}
\région{GOs}
\end{entrée}

\begin{entrée}
{prendre; enlever}
\classe{v}
\vedette{phè}
\sens{1}
\région{GOs PA BO}
\variante{%
\vedette{phe}}
\end{entrée}

\begin{entrée}
{prendre par la main (enfant)}
\vedette{whili}
\région{GOs WEM BO PA}
\end{entrée}

\begin{entrée}
{prendre (sable, terre)}
\vedette{yeege}
\région{PA}
\end{entrée}

\begin{entrée}
{prendre ; saisir (en partant)}
\vedette{kha-phe}
\groupe{A}
\région{GOs PA BO}
\variante{%
\vedette{kha-vwe}
\région{GO(s)}}
\end{entrée}

\begin{entrée}
{prendre (se) dans un filet (en s'enroulant)}
\vedette{tagi}
\sens{2}
\région{PA BO}
\end{entrée}

\begin{entrée}
{présenter}
\vedette{phããde}
\région{GOs BO PA}
\variante{%
\vedette{phãde}
\région{GO(s) PA}}
\end{entrée}

\begin{entrée}
{presser}
\classe{v}
\vedette{kha}\homonyme{3}
\groupe{A}
\région{GOs PA BO}
\variante{%
\vedette{khaa}
\région{PA}}
\end{entrée}

\begin{entrée}
{presser dans la main}
\classe{v}
\vedette{pwaa}\homonyme{1}
\sens{2}
\région{GOs PABO}
\end{entrée}

\begin{entrée}
{presser ; écraser (la pulpe de coco)}
\classe{v}
\vedette{phwòli}
\sens{1}
\région{BO [BM, Corne]}
\end{entrée}

\begin{entrée}
{procréer ; saillir (un animal) ; accouplement}
\vedette{he}
\région{GOs PA BO}
\end{entrée}

\begin{entrée}
{provoquer}
\classe{v}
\vedette{thii}\homonyme{1}
\sens{2}
\région{GOs PA BO WEM}
\end{entrée}

\begin{entrée}
{ramasser}
\vedette{cooge}
\région{GOs BO}
\variante{%
\vedette{coge}
\région{BO}}
\end{entrée}

\begin{entrée}
{ramasser dans le creux de la main}
\vedette{yeege}
\région{PA}
\end{entrée}

\begin{entrée}
{ramasser (des objets qui traînent)}
\vedette{yoi}
\région{GOs}
\end{entrée}

\begin{entrée}
{ramasser (en roulant)}
\vedette{zugi}
\région{GOs PA}
\variante{%
\vedette{zhugi}
\région{GA}}
\variante{%
\vedette{yugi}
\région{BO}}
\end{entrée}

\begin{entrée}
{ramasser (nourriture, vêtement)}
\vedette{hivwi}\homonyme{2}
\région{GOs PA}
\variante{%
\vedette{hivi}
\région{BO}}
\variante{%
\vedette{hipi}
\région{BO vx}}
\end{entrée}

\begin{entrée}
{ramasser (sable, feuilles, etc.) ; enlever}
\vedette{yage}
\région{GOs BO}
\end{entrée}

\begin{entrée}
{ramper (lianes)}
\vedette{thròlòe}
\région{GOs}
\variante{%
\vedette{thòlòe}
\région{PA}}
\variante{%
\vedette{tòloè}
\région{BO (Corne)}}
\variante{%
\vedette{tholòè, tòlee}
\région{BO (BM)}}
\variante{%
\vedette{throleitholei}
\région{WEM}}
\end{entrée}

\begin{entrée}
{rapiécer (lit. fermer le trou)}
\vedette{thô-vwaa-ni}
\région{GOs}
\end{entrée}

\begin{entrée}
{ratisser}
\vedette{hòzi}
\région{GOs}
\end{entrée}

\begin{entrée}
{ratisser}
\vedette{kiri}
\région{PA BO}
\end{entrée}

\begin{entrée}
{ratisser}
\vedette{tûûne}
\région{GOs PA BO}
\end{entrée}

\begin{entrée}
{recueillir (le miel sur les arbres, en cassant la branche sur laquelle se trouve la ruche) [PA]}
\classe{v}
\vedette{pwaa}\homonyme{1}
\sens{2}
\région{GOs PABO}
\end{entrée}

\begin{entrée}
{récurer (le dos de la marmite avec de la cendre)}
\classe{v.t.}
\vedette{zòli}
\sens{2}
\région{GOs PA}
\variante{%
\vedette{zhòli}
\région{GO(s)}}
\variante{%
\vedette{yòli, yòòli}
\région{BO}}
\end{entrée}

\begin{entrée}
{rejoindre ; rattraper qqn}
\vedette{kha-tree-çimwi}
\région{GOs}
\end{entrée}

\begin{entrée}
{rembobiner (ligne de pêche)}
\vedette{zugi}
\région{GOs PA}
\variante{%
\vedette{zhugi}
\région{GA}}
\variante{%
\vedette{yugi}
\région{BO}}
\end{entrée}

\begin{entrée}
{renverser ; faire tomber}
\vedette{ta-uuni}
\région{GOs}
\end{entrée}

\begin{entrée}
{rester à l'écart}
\classe{v}
\vedette{kari}
\sens{1}
\région{PA}
\end{entrée}

\begin{entrée}
{retenir (se)}
\vedette{khòlima}
\région{GOs}
\end{entrée}

\begin{entrée}
{retenir ; serrer}
\vedette{tree-çimwî}
\région{GOs}
\variante{%
\vedette{tee-cimwî, tee-jimwî, tee-yimwî}
\région{PABO}}
\end{entrée}

\begin{entrée}
{(re)tirer}
\vedette{zugi}
\région{GOs PA}
\variante{%
\vedette{zhugi}
\région{GA}}
\variante{%
\vedette{yugi}
\région{BO}}
\end{entrée}

\begin{entrée}
{retirer qqch.}
\classe{v}
\vedette{khai}\homonyme{2}
\sens{1}
\région{GOs PA BO}
\end{entrée}

\begin{entrée}
{rétracter (se) (dans une coquille: escargot, coquillage)}
\vedette{hiili}
\région{PA BO}
\variante{%
\vedette{iili}}
\end{entrée}

\begin{entrée}
{retrousser (robe, pantalon)}
\vedette{zugi}
\région{GOs PA}
\variante{%
\vedette{zhugi}
\région{GA}}
\variante{%
\vedette{yugi}
\région{BO}}
\end{entrée}

\begin{entrée}
{révéler}
\vedette{phããde}
\région{GOs BO PA}
\variante{%
\vedette{phãde}
\région{GO(s) PA}}
\end{entrée}

\begin{entrée}
{rompre (un bout de pain, un bout d'igname cuite)}
\vedette{pii}\homonyme{1}
\région{GOs BO PA}
\end{entrée}

\begin{entrée}
{rouler (des torons)}
\vedette{phele}
\région{PA BO}
\end{entrée}

\begin{entrée}
{rouler (un toron de corde sur la cuisse)}
\vedette{bile}
\région{GOs BO}
\variante{%
\vedette{bire}
\région{GOs}}
\end{entrée}

\begin{entrée}
{saisir avec le bras ; embrasser [Corne]}
\vedette{piviige}
\région{BO}
\end{entrée}

\begin{entrée}
{saisir en se déplaçant (en emportant ou amenant)}
\vedette{kha-cimwî}
\région{PA}
\end{entrée}

\begin{entrée}
{secouer}
\vedette{cili}
\région{PA BO}
\end{entrée}

\begin{entrée}
{secouer}
\vedette{jòme}
\région{GOs}
\end{entrée}

\begin{entrée}
{secouer (arbre)}
\vedette{wewele}
\région{GOs}
\end{entrée}

\begin{entrée}
{serrer}
\vedette{khòle}
\région{GOs BO PA}
\end{entrée}

\begin{entrée}
{serrer}
\vedette{weze}
\région{GOs}
\variante{%
\vedette{wei}
\région{BO}}
\end{entrée}

\begin{entrée}
{serrer (se) la main}
\vedette{pe-cimwi hi}
\région{GOs PA}
\end{entrée}

\begin{entrée}
{signe}
\vedette{hîńõ}
\groupe{A}
\région{GOs PA}
\variante{%
\vedette{hinõn}
\région{BO}}
\end{entrée}

\begin{entrée}
{signe de la main (faire un)}
\vedette{nyãnume}
\région{GOs}
\end{entrée}

\begin{entrée}
{soulever}
\vedette{thaba}
\région{GOs PA BO}
\end{entrée}

\begin{entrée}
{soulever (des pierres, etc. pour chercher qqch.)}
\classe{v}
\vedette{hòvwi}
\sens{1}
\région{GOs}
\variante{%
\vedette{hòvi}
\région{BO PA}}
\end{entrée}

\begin{entrée}
{soulever (des pierres, herbes)}
\classe{v}
\vedette{zali}
\sens{1}
\région{GOs}
\variante{%
\vedette{zhali}
\région{GA}}
\end{entrée}

\begin{entrée}
{soulever (lentement)}
\vedette{habwa}
\région{GOs}
\end{entrée}

\begin{entrée}
{soulever (un objet léger)}
\classe{v.i.}
\vedette{côô}
\sens{2}
\région{GOs}
\variante{%
\vedette{cô}
\région{BO (BM]}}
\end{entrée}

\begin{entrée}
{soutenir (avec la paume de la main)}
\vedette{cooge}
\région{GOs BO}
\variante{%
\vedette{coge}
\région{BO}}
\end{entrée}

\begin{entrée}
{sursauter}
\vedette{jò}\homonyme{1}
\région{GOs}
\variante{%
\vedette{jòn}
\région{BO}}
\end{entrée}

\begin{entrée}
{suspendre}
\vedette{côî}
\région{GOs}
\variante{%
\vedette{chôî}
\région{BO (Corne)}}
\end{entrée}

\begin{entrée}
{tambouriner}
\vedette{pao-cabi}
\région{PA}
\end{entrée}

\begin{entrée}
{taper}
\vedette{khõbo}
\région{BO}
\end{entrée}

\begin{entrée}
{taper}
\classe{v}
\vedette{thali}
\sens{3}
\région{PA BO}
\end{entrée}

\begin{entrée}
{taper avec un bâton}
\vedette{bò}\homonyme{1}
\région{GOs}
\variante{%
\vedette{bòl}
\région{BO}}
\end{entrée}

\begin{entrée}
{taper avec un bâton (pour taper les roussettes ou gauler les grappes de coco)}
\classe{v ; n}
\vedette{henim}
\sens{1}
\région{PA BO [BM]}
\end{entrée}

\begin{entrée}
{taper dans l'eau pour faire du bruit (et effrayer le poisson afin de le pousser dans le filet)}
\vedette{thi-pholo}
\région{GOs BO}
\end{entrée}

\begin{entrée}
{taper dans l'eau pour faire du bruit et effrayer le poisson et l'amener dans le filet}
\vedette{pholo}
\région{GOs BO}
\end{entrée}

\begin{entrée}
{taper pour enfoncer}
\vedette{pa-cabi}
\région{GOs}
\end{entrée}

\begin{entrée}
{taper pour enfoncer (poteau)}
\classe{v}
\vedette{khabe}
\sens{1}
\région{GOs PA BO}
\end{entrée}

\begin{entrée}
{tapoter}
\vedette{cobo}
\région{GOs}
\end{entrée}

\begin{entrée}
{tasser (avec les mains ou les pieds)}
\classe{v}
\vedette{kha}\homonyme{3}
\groupe{A}
\région{GOs PA BO}
\variante{%
\vedette{khaa}
\région{PA}}
\end{entrée}

\begin{entrée}
{tasser (en frappant)}
\vedette{dame}
\région{GOs BO}
\variante{%
\vedette{dam}
\région{BO}}
\end{entrée}

\begin{entrée}
{tâtonner}
\classe{v}
\vedette{pale}
\sens{1}
\région{GOs BO PA}
\variante{%
\vedette{palee}
\région{BO}}
\end{entrée}

\begin{entrée}
{teindre (cheveux)}
\vedette{thrîmi}
\région{GOs}
\variante{%
\vedette{thimi}
\région{PA BO}}
\end{entrée}

\begin{entrée}
{tenir fermement qqch}
\vedette{koe-piça-ni}
\région{GOs}
\end{entrée}

\begin{entrée}
{tenir ferme ; saisir}
\vedette{cimwî}
\région{GOs PA}
\variante{%
\vedette{khimwi}
\région{BO [BM]}}
\end{entrée}

\begin{entrée}
{tenir, retenir (un animal par une bride ou une corde)}
\vedette{kuue}
\région{GOs}
\end{entrée}

\begin{entrée}
{tirer ;}
\classe{v}
\vedette{khai}\homonyme{2}
\sens{1}
\région{GOs PA BO}
\end{entrée}

\begin{entrée}
{tirer à l'arc}
\vedette{khai}\homonyme{1}
\région{GOs PA}
\end{entrée}

\begin{entrée}
{tirer d'un coup sec}
\vedette{gibwa}
\sens{1}
\région{GOs}
\end{entrée}

\begin{entrée}
{tituber [Corne]}
\vedette{umau}
\région{BO}
\end{entrée}

\begin{entrée}
{tordre}
\vedette{mhôwe}
\région{GOs}
\end{entrée}

\begin{entrée}
{tordre}
\vedette{zugi}
\région{GOs PA}
\variante{%
\vedette{zhugi}
\région{GA}}
\variante{%
\vedette{yugi}
\région{BO}}
\end{entrée}

\begin{entrée}
{tordre (en revenant vers l'arrière)}
\classe{v}
\vedette{phõge}
\sens{1}
\région{GOs PA}
\variante{%
\vedette{phõng}
\région{BO}}
\end{entrée}

\begin{entrée}
{tordre; enrouler (une corde)}
\vedette{bilòò}
\région{GOs BO PA}
\end{entrée}

\begin{entrée}
{tordre (se) (doigt, cheville)}
\vedette{kaabilò}
\région{GOs}
\end{entrée}

\begin{entrée}
{toucher ;}
\vedette{chele}\homonyme{1}
\région{GOs PA}
\end{entrée}

\begin{entrée}
{toucher à qqch}
\vedette{côôni}
\région{PA}
\end{entrée}

\begin{entrée}
{toucher (avec la main)}
\vedette{khòle}
\région{GOs BO PA}
\end{entrée}

\begin{entrée}
{tourner}
\vedette{bilòò}
\région{GOs BO PA}
\end{entrée}

\begin{entrée}
{traîner par terre ; tirer}
\vedette{trivwi}
\région{GOs}
\variante{%
\vedette{tripwi}}
\variante{%
\vedette{tiwi}
\région{BO [BM]}}
\variante{%
\vedette{tipui}
\région{BO (Corne)}}
\end{entrée}

\begin{entrée}
{tranchant}
\vedette{yazoo}
\end{entrée}

\begin{entrée}
{tripoter [BM]}
\vedette{varû}
\région{BO}
\end{entrée}

\begin{entrée}
{tripoter (un objet) ; toucher}
\vedette{chińõõ}
\région{GOs PA}
\end{entrée}

\begin{entrée}
{tuer}
\vedette{baani}
\région{GOs PA BO}
\end{entrée}

\begin{entrée}
{tuer (les moustiques en tapant)}
\vedette{baani}
\région{GOs PA BO}
\end{entrée}

\subsubsection{Mouvements ou actions avec la tête, les yeux, la bouche}

\begin{entrée}
{attraper avec les dents}
\vedette{caigo}
\région{GOs BO}
\end{entrée}

\begin{entrée}
{casser avec les dents ; écraser avec les dents (bonbon, qqch de dur)}
\vedette{kaobi}
\région{GOs PA}
\end{entrée}

\begin{entrée}
{claquer des lèvres en signe de mépris [Corne]}
\vedette{peu}\homonyme{2}
\région{BO}
\end{entrée}

\begin{entrée}
{couper (avec les dents)}
\vedette{caigo}
\région{GOs BO}
\end{entrée}

\begin{entrée}
{croquer}
\classe{v}
\vedette{kûû}
\sens{1}
\région{GOs BO}
\end{entrée}

\begin{entrée}
{déchirer avec les dents}
\vedette{cubii}
\région{PA}
\end{entrée}

\begin{entrée}
{écarquiller les yeux}
\vedette{wò phii-me}
\région{GOs}
\end{entrée}

\begin{entrée}
{écarquiller le yeux}
\vedette{hò pii-me}
\région{GOs}
\variante{%
\vedette{wò pii-me}
\région{GO(s)}}
\end{entrée}

\begin{entrée}
{embrasser}
\vedette{bue}
\région{PA}
\end{entrée}

\begin{entrée}
{embrasser ; faire un baiser}
\vedette{biibu}
\région{GOs}
\variante{%
\vedette{bibwo}
\région{GO(s)}}
\end{entrée}

\begin{entrée}
{fermer les yeux}
\classe{v}
\vedette{mãxi}
\sens{1}
\région{GOs}
\variante{%
\vedette{mãxim}
\région{WEM WE}}
\variante{%
\vedette{mhãkim, mhãxim}
\région{BO}}
\variante{%
\vedette{mããxim}
\région{PA}}
\end{entrée}

\begin{entrée}
{grimacer ; faire des grimaces}
\vedette{nyòlò}
\région{PA}
\variante{%
\vedette{nyolõng}
\région{PA}}
\end{entrée}

\begin{entrée}
{lécher}
\vedette{malevwi}
\région{GOs}
\variante{%
\vedette{maalemi}
\région{BO [BM]}}
\end{entrée}

\begin{entrée}
{lécher [BM]}
\vedette{thèmi}
\région{BO}
\end{entrée}

\begin{entrée}
{mordre}
\classe{v ; n}
\vedette{huu}\homonyme{1}
\sens{2}
\région{GOs PA BO}
\variante{%
\vedette{whuu}
\région{GO(s)}}
\end{entrée}

\begin{entrée}
{mordre}
\classe{v}
\vedette{hu-vo}\homonyme{1}
\sens{2}
\région{PA}
\variante{%
\vedette{hu-po}
\région{PA}}
\end{entrée}

\begin{entrée}
{ouvrir la bouche}
\vedette{waa}
\région{BO [BM, Corne]}
\variante{%
\vedette{wò}
\région{GO(s)}}
\end{entrée}

\begin{entrée}
{ouvrir les yeux}
\vedette{ńõ}\homonyme{1}
\sens{2}
\région{GOs}
\variante{%
\vedette{nòl}
\région{BO PA}}
\end{entrée}

\begin{entrée}
{ronger}
\classe{v ; n}
\vedette{huu}\homonyme{1}
\sens{2}
\région{GOs PA BO}
\variante{%
\vedette{whuu}
\région{GO(s)}}
\end{entrée}

\begin{entrée}
{siffler avec les doigts pour héler qqn.}
\vedette{zeede}
\région{GOs}
\variante{%
\vedette{zedil}
\région{PA}}
\end{entrée}

\begin{entrée}
{siffler avec les lèvres}
\vedette{whayu}
\région{GOs PA}
\variante{%
\vedette{wayu}
\région{BO}}
\end{entrée}

\begin{entrée}
{sortir d'un trou (animal, reptile)}
\vedette{hõõng}
\région{PA}
\end{entrée}

\begin{entrée}
{souffler}
\vedette{hoova}
\région{GOs}
\end{entrée}

\begin{entrée}
{tenir avec les dents}
\vedette{caigo}
\région{GOs BO}
\end{entrée}

\begin{entrée}
{tirer (langue)}
\vedette{hõõng}
\région{PA}
\end{entrée}

\subsubsection{Verbes d'action faite par des animaux}

\begin{entrée}
{battre des ailes}
\classe{v}
\vedette{thazabi}
\sens{1}
\région{GOs}
\variante{%
\vedette{tharabil}
\région{PA}}
\end{entrée}

\begin{entrée}
{castrer (animal)}
\classe{v}
\vedette{köe}
\sens{2}
\région{GOs}
\variante{%
\vedette{khoe}
\région{BO [BM]}}
\end{entrée}

\begin{entrée}
{frétiller (poisson) ;}
\classe{v}
\vedette{thazabi}
\sens{1}
\région{GOs}
\variante{%
\vedette{tharabil}
\région{PA}}
\end{entrée}

\begin{entrée}
{mordre (en parlant d'animaux)}
\vedette{pewi}
\région{GOs PA}
\end{entrée}

\begin{entrée}
{pincer (crabes, langoustes)}
\classe{v}
\vedette{hîmi}\homonyme{1}
\sens{2}
\région{GOs}
\end{entrée}

\begin{entrée}
{ramper (serpent, lézard, lianes sur les arbres ou sur les tuteurs)}
\classe{v}
\vedette{yöö}
\sens{1}
\région{GOs PA}
\end{entrée}

\begin{entrée}
{voler (oiseau)}
\vedette{phu}\homonyme{5}
\région{GOs BO PA}
\end{entrée}

\subsubsection{Interaction avec les animaux}

\begin{entrée}
{domestiquer (un animal) ; apprivoiser}
\vedette{pha-kôôńô-ni}
\région{GOs}
\end{entrée}

\begin{entrée}
{dresser (cheval)}
\vedette{khaa}\homonyme{2}
\région{GOs}
\end{entrée}

\subsubsection{Actions liées aux éléments (liquide, fumée)}

\begin{entrée}
{ajouter du liquide}
\vedette{taabunõõ}
\région{GOs}
\end{entrée}

\begin{entrée}
{arroser (fleurs)}
\vedette{munõ}
\région{GOs}
\variante{%
\vedette{muunõ}
\région{BO}}
\end{entrée}

\begin{entrée}
{asperger ; projeter (eau, boue avec les mains) ; arroser (avec la main)}
\vedette{yaze}
\région{GOs}
\end{entrée}

\begin{entrée}
{baisser (niveau d'eau) ; descendre (niveau d'eau)}
\vedette{nhyô}
\région{GOs}
\variante{%
\vedette{nyong, nyô}
\région{PA}}
\end{entrée}

\begin{entrée}
{barrer l'eau}
\vedette{oole we}
\région{GOs}
\end{entrée}

\begin{entrée}
{couler (eau, sang) ; écouler (s')}
\vedette{tho}\homonyme{2}
\région{GOs PA BO}
\end{entrée}

\begin{entrée}
{couler goutte à goutte ; fuir}
\vedette{khilò}
\région{GOs BO}
\end{entrée}

\begin{entrée}
{couler ; répandre (se) ; vider (se)}
\vedette{kula}\homonyme{2}
\région{GOs BO}
\end{entrée}

\begin{entrée}
{déborder}
\vedette{cabul}
\région{PA BO [BM]}
\end{entrée}

\begin{entrée}
{déborder}
\vedette{cubu}
\région{GOs}
\end{entrée}

\begin{entrée}
{déborder (lit. se lever inondation)}
\vedette{cul (a) kao}
\région{PA}
\variante{%
\vedette{col a kao}
\région{BO}}
\end{entrée}

\begin{entrée}
{décrue}
\vedette{nhyô}
\région{GOs}
\variante{%
\vedette{nyong, nyô}
\région{PA}}
\end{entrée}

\begin{entrée}
{dévier (eau)}
\vedette{bwange}\homonyme{1}
\région{GOs}
\end{entrée}

\begin{entrée}
{dévier (eau)}
\vedette{paxe}
\région{PA BO}
\end{entrée}

\begin{entrée}
{eau monte et coule (après une forte pluie)}
\vedette{we-tru}
\région{GOs}
\end{entrée}

\begin{entrée}
{éclabousser (avec les mains)}
\vedette{yali}
\région{PA BO [BM]}
\end{entrée}

\begin{entrée}
{écoper}
\classe{v}
\vedette{khee}\homonyme{1}
\sens{1}
\région{GOs}
\end{entrée}

\begin{entrée}
{écoper}
\vedette{khee we}
\région{GOs}
\end{entrée}

\begin{entrée}
{écoper}
\vedette{yali}
\région{PA BO [BM]}
\end{entrée}

\begin{entrée}
{écoper ;}
\vedette{kaba we}\homonyme{2}
\région{WEM WE BO}
\end{entrée}

\begin{entrée}
{envahir d'eau ; inonder ; déborder ; passer par-dessus}
\vedette{kaabole}
\région{GOs}
\variante{%
\vedette{khaabule}
\région{PA BO}}
\variante{%
\vedette{kaabule}
\région{BO [BM]}}
\end{entrée}

\begin{entrée}
{éteindre (feu)}
\vedette{munõ}
\région{GOs}
\variante{%
\vedette{muunõ}
\région{BO}}
\end{entrée}

\begin{entrée}
{flotter (emporté par l'eau) ; échouer (sur la grève)}
\vedette{kharu-mhween}
\région{GOs PA BO}
\end{entrée}

\begin{entrée}
{gicler}
\classe{v}
\vedette{tãã}
\sens{2}
\région{GOs}
\end{entrée}

\begin{entrée}
{jeter (dans l'eau)[BM, Coyaud]}
\vedette{teevaò}
\région{BO}
\end{entrée}

\begin{entrée}
{plonger (pour s'amuser, se dit d'enfants)}
\vedette{pe-khabeçö}
\région{GOs}
\end{entrée}

\begin{entrée}
{prendre de l'eau (avec un petit récipient)}
\vedette{khee we}
\région{GOs}
\end{entrée}

\begin{entrée}
{puiser de l'eau}
\vedette{tröi}
\région{GOs}
\région{GOs}
\variante{%
\vedette{trööi}}
\variante{%
\vedette{tui}
\région{WEM WE PA BO}}
\end{entrée}

\begin{entrée}
{puiser ; prendre de l'eau (avec un petit récipient)}
\vedette{kaba we}\homonyme{2}
\région{WEM WE BO}
\end{entrée}

\begin{entrée}
{répandre (se) (eau, fumée)}
\vedette{thòlòe}
\région{BO [Corne]}
\variante{%
\vedette{tholee tolee}
\région{BO [Corne]}}
\end{entrée}

\begin{entrée}
{retirer d'une marmite (surtout du liquide)}
\vedette{kaba we}\homonyme{2}
\région{WEM WE BO}
\end{entrée}

\begin{entrée}
{rincer (vaisselle, un récipient)}
\vedette{yala}\homonyme{1}
\région{GOs PA}
\end{entrée}

\begin{entrée}
{sonder la profondeur de l'eau}
\vedette{tha nhyôgo we}
\région{GOs}
\end{entrée}

\begin{entrée}
{tremper dans l'eau}
\vedette{paa-bule}
\région{PA}
\end{entrée}

\begin{entrée}
{tremper dans l'eau}
\vedette{phole}
\région{GOs}
\end{entrée}

\begin{entrée}
{tremper (le linge dans l'eau) ; mouiller}
\vedette{kaabole}
\région{GOs}
\variante{%
\vedette{khaabule}
\région{PA BO}}
\variante{%
\vedette{kaabule}
\région{BO [BM]}}
\end{entrée}

\begin{entrée}
{verser ; répandre ; vider}
\classe{v}
\vedette{kule}
\sens{1}
\région{GOs PA}
\variante{%
\vedette{kole, kula}}
\end{entrée}

\begin{entrée}
{vider}
\vedette{yali}
\région{PA BO [BM]}
\end{entrée}

\begin{entrée}
{vider ; renverser (liquide)}
\classe{v}
\vedette{kole}\homonyme{2}
\sens{2}
\région{GOs BO}
\end{entrée}

\begin{entrée}
{vider (un liquide: au sens de boire)}
\vedette{paa-moze}
\région{GOs}
\variante{%
\vedette{paa-moze}
\région{GO(s)}}
\end{entrée}

\subsubsection{Actions liées aux plantes}

\begin{entrée}
{arracher (à la main des herbes ou des plantes à racine : kumala, kuru, etc., sur une petite surface)}
\classe{v.t.}
\vedette{phugi}
\sens{1}
\région{GOs PA BO}
\end{entrée}

\begin{entrée}
{assouplir les fibres en les lissant}
\classe{v}
\vedette{uzi}
\sens{1}
\région{GOs}
\variante{%
\vedette{uli}
\région{PA WE BO}}
\end{entrée}

\begin{entrée}
{biseauter}
\classe{v}
\vedette{uli}
\région{PA WE BO}
\variante{%
\vedette{uzi}
\région{GOs}}
\end{entrée}

\begin{entrée}
{casser (bois en pliant); couper (en cassant)}
\vedette{pwaale}
\région{GOs PABO}
\end{entrée}

\begin{entrée}
{casser (bois en pliant) ; couper (en cassant)}
\classe{v}
\vedette{pwaa}\homonyme{1}
\sens{1}
\région{GOs PABO}
\end{entrée}

\begin{entrée}
{cercler (arbre) ; tailler (haie)}
\classe{v}
\vedette{köe}
\sens{1}
\région{GOs}
\variante{%
\vedette{khoe}
\région{BO [BM]}}
\end{entrée}

\begin{entrée}
{conserver pour faire mûrir}
\vedette{zaae}
\région{GOs PA}
\variante{%
\vedette{zhae}
\région{GA}}
\end{entrée}

\begin{entrée}
{cueillir à la main (fruit, baies)}
\vedette{nhii}
\région{GOs}
\variante{%
\vedette{nhii}
\région{PA BO}}
\end{entrée}

\begin{entrée}
{cueillir des feuilles et herbes (pour faire des médicaments)}
\vedette{kaal}
\région{PA}
\end{entrée}

\begin{entrée}
{cueillir (des fleurs, des herbes magiques) [Haudricourt]}
\vedette{kava}\homonyme{2}
\région{GO}
\end{entrée}

\begin{entrée}
{cueillir (en cassant, les fleurs)}
\classe{v}
\vedette{pwaa}\homonyme{1}
\sens{1}
\région{GOs PABO}
\end{entrée}

\begin{entrée}
{cueillir (en cassant, les fleurs)}
\vedette{pwaale}
\région{GOs PABO}
\end{entrée}

\begin{entrée}
{cueillir (feuilles, herbes, jeunes pousses)}
\vedette{thrõgo}
\région{GOs}
\variante{%
\vedette{thõgo}
\région{BO}}
\end{entrée}

\begin{entrée}
{cueillir (fleur, feuilles, bourgeons)}
\vedette{uu}
\région{PA BO}
\end{entrée}

\begin{entrée}
{cueillir la canne à sucre}
\vedette{tha ê}
\région{GOs}
\variante{%
\vedette{tho êm}
\région{PA}}
\end{entrée}

\begin{entrée}
{cueillir un fruit qui n'est pas mûr}
\vedette{trale}
\région{GOs}
\variante{%
\vedette{tali}
\région{PA}}
\end{entrée}

\begin{entrée}
{décoller ; enlever [BO]}
\classe{v.t.}
\vedette{phugi}
\sens{1}
\région{GOs PA BO}
\end{entrée}

\begin{entrée}
{écorcer (du coco sur un épieu) ; éplucher (coco)}
\vedette{tha}\homonyme{2}
\région{GOs BO PA}
\end{entrée}

\begin{entrée}
{écorcer (enlever les épines)}
\classe{v}
\vedette{töö}\homonyme{1}
\sens{2}
\région{GOs BO}
\end{entrée}

\begin{entrée}
{écorcer le coco (sur un épieu)}
\vedette{tha nu}
\région{GOs PA BO}
\end{entrée}

\begin{entrée}
{effeuiller (en pinçant et cassant la tige des feuilles avec le pouce et l'index)}
\vedette{thabòe}
\région{GOs PA}
\end{entrée}

\begin{entrée}
{effeuiller une tige}
\vedette{uzi dròò}
\région{GOs}
\variante{%
\vedette{uli doo}
\région{PA}}
\end{entrée}

\begin{entrée}
{enlever les fibres (en lissant avec un couteau)}
\vedette{uzi cii}
\région{GOs}
\end{entrée}

\begin{entrée}
{éplucher (en taillant avec un geste vers l'extérieur)}
\vedette{thebe}
\région{GOs}
\variante{%
\vedette{tebe}
\région{BO [Corne]}}
\end{entrée}

\begin{entrée}
{épointer}
\vedette{thebe}
\région{GOs}
\variante{%
\vedette{tebe}
\région{BO [Corne]}}
\end{entrée}

\begin{entrée}
{faire des boutures (lié à la notion de vie)}
\vedette{thu mõõxi}
\région{GO}
\end{entrée}

\begin{entrée}
{faire mûrir (fruits)}
\vedette{zaae}
\région{GOs PA}
\variante{%
\vedette{zhae}
\région{GA}}
\end{entrée}

\begin{entrée}
{gauler (fruit)}
\classe{v}
\vedette{thali}
\sens{2}
\région{PA BO}
\end{entrée}

\begin{entrée}
{gauler ; taper pour faire tomber (fruit)}
\vedette{trale}
\région{GOs}
\variante{%
\vedette{tali}
\région{PA}}
\end{entrée}

\begin{entrée}
{lisser (la tige)}
\classe{v}
\vedette{uzi}
\sens{1}
\région{GOs}
\variante{%
\vedette{uli}
\région{PA WE BO}}
\end{entrée}

\begin{entrée}
{lisser; rendre lisse (tige)}
\classe{v}
\vedette{uli}
\région{PA WE BO}
\variante{%
\vedette{uzi}
\région{GOs}}
\end{entrée}

\begin{entrée}
{ramper (lianes, sur les arbres)}
\vedette{zòò}\homonyme{2}
\sens{2}
\région{GOs}
\variante{%
\vedette{zhòò}
\région{GO(s)}}
\variante{%
\vedette{zòòm}
\région{PA}}
\variante{%
\vedette{yhòòm, yòòm, yòò}
\région{BO}}
\end{entrée}

\begin{entrée}
{rapporter de nouvelles boutures chez soi}
\vedette{three}
\région{GOs}
\variante{%
\vedette{thee}
\région{BO [Corne]}}
\end{entrée}

\begin{entrée}
{replier (la tige de l'igname sur elle-même quand elle dépasse la hauteur du tuteur)}
\classe{v}
\vedette{pwaa}\homonyme{1}
\sens{1}
\région{GOs PABO}
\end{entrée}

\begin{entrée}
{tailler (arbre, plante)}
\vedette{thebe}
\région{GOs}
\variante{%
\vedette{tebe}
\région{BO [Corne]}}
\end{entrée}

\begin{entrée}
{tailler (un arbre en cassant les branches à la main et vers le bas)}
\vedette{kii}\homonyme{2}
\région{GOs PA}
\end{entrée}

\subsubsection{Manière de faire l’action : verbes et adverbes de manière}

\begin{entrée}
{contraire ; contraire (faire le)}
\vedette{chîxi}
\région{GOs PA}
\variante{%
\vedette{chîngi}
\région{GO(s) BO}}
\end{entrée}

\begin{entrée}
{envers (à l') ; faire à l'envers}
\vedette{chîxi}
\région{GOs PA}
\variante{%
\vedette{chîngi}
\région{GO(s) BO}}
\end{entrée}

\begin{entrée}
{faire ainsi ; être ainsi}
\vedette{wãã-na}
\région{GOs PA BO}
\end{entrée}

\begin{entrée}
{faire exprès}
\vedette{pu nee}
\région{PA BO [BM]}
\end{entrée}

\begin{entrée}
{faire l'inverse (de tout le monde)}
\vedette{chîxi}
\région{GOs PA}
\variante{%
\vedette{chîngi}
\région{GO(s) BO}}
\end{entrée}

\begin{entrée}
{faire vite}
\vedette{kazubi}
\groupe{A}
\région{GOs}
\variante{%
\vedette{karubi, katrubi}
\région{WEM WE}}
\variante{%
\vedette{kharubin}
\région{PA BO}}
\end{entrée}

\begin{entrée}
{inutile ; vainement ; en vain}
\vedette{phwee-mwãgi}
\région{GOs}
\end{entrée}

\begin{entrée}
{lentement}
\vedette{kha-maaçee}
\sens{2}
\région{GOs}
\end{entrée}

\begin{entrée}
{lent ; lentement}
\vedette{maya}\homonyme{2}
\région{PA BO}
\variante{%
\vedette{meya}
\région{PA}}
\end{entrée}

\begin{entrée}
{manière de faire ; façon de faire ; procédure}
\vedette{mwaje}
\région{GOs PA}
\end{entrée}

\begin{entrée}
{paix ; tranquille ; paisible}
\vedette{trenene}
\région{GOs}
\variante{%
\vedette{tee-nenem}
\région{PA BO}}
\end{entrée}

\begin{entrée}
{peine ; difficulté [Corne]}
\vedette{me-cöni}
\région{BO}
\end{entrée}

\begin{entrée}
{ralentir ; calmer (se)}
\vedette{maaja}
\région{GOs}
\end{entrée}

\begin{entrée}
{résolution ; résolu ; définitif (être) ; définitivement}
\vedette{khaaimudre}
\région{GOs}
\variante{%
\vedette{khaimode}
\région{BO}}
\end{entrée}

\begin{entrée}
{rester tranquille ; reposer en paix}
\vedette{trenene}
\région{GOs}
\variante{%
\vedette{tee-nenem}
\région{PA BO}}
\end{entrée}

\begin{entrée}
{vite ; rapidement ; tout de suite}
\vedette{kazubi}
\groupe{B}
\région{GOs}
\variante{%
\vedette{karubi, katrubi}
\région{WEM WE}}
\variante{%
\vedette{kharubin}
\région{PA BO}}
\end{entrée}

\subsubsection{Actions avec un instrument, un outil}

\begin{entrée}
{clouer}
\vedette{tabila}
\région{BO PA WE}
\end{entrée}

\begin{entrée}
{clouer}
\vedette{thaila}
\région{GOs}
\variante{%
\vedette{thaeza}
\région{GO}}
\variante{%
\vedette{taela}
\région{WE}}
\variante{%
\vedette{taabila, tabila}
\région{PA}}
\end{entrée}

\begin{entrée}
{piocher}
\classe{v}
\vedette{taa}
\sens{3}
\région{GOs PA BO}
\end{entrée}

\begin{entrée}
{piocher ; bêcher (champ)}
\vedette{khòòni}
\région{GOs}
\variante{%
\vedette{kòòni}
\région{BO PA}}
\end{entrée}

\begin{entrée}
{raboter ; lisser}
\classe{v}
\vedette{uzi}
\sens{2}
\région{GOs}
\variante{%
\vedette{uli}
\région{PA WE BO}}
\end{entrée}

\begin{entrée}
{scier}
\classe{v.t.}
\vedette{zòi}
\sens{1}
\région{GOs PA BO}
\variante{%
\vedette{zhòi}
\région{GO(s)}}
\end{entrée}

\section{Caractéristiques et propriétés}

\subsection{Sons, bruits}

\begin{entrée}
{bourdonner ; faire un bruit de bourdon}
\classe{v}
\vedette{mû}\homonyme{2}
\sens{2}
\région{GOs BO}
\variante{%
\vedette{mûû}
\région{BO}}
\end{entrée}

\begin{entrée}
{brouhaha des voix}
\classe{nom}
\vedette{ãmã}
\sens{2}
\région{GOs}
\end{entrée}

\begin{entrée}
{bruissement [Corne]}
\vedette{dòòla}
\région{BO}
\end{entrée}

\begin{entrée}
{bruit aigu (qui vrille les oreilles)}
\vedette{kîî}
\région{GOs}
\end{entrée}

\begin{entrée}
{bruit de ruissellement de l'eau}
\classe{v}
\vedette{hûn}
\sens{2}
\région{PA BO}
\end{entrée}

\begin{entrée}
{bruit des voix ; brouhaha}
\vedette{gaa-vhaa}
\région{GOs}
\variante{%
\vedette{gaa-fhaa}
\région{GA}}
\end{entrée}

\begin{entrée}
{bruit (faire du)}
\vedette{pavada}
\région{GOs}
\variante{%
\vedette{pavade}
\région{PA WE}}
\end{entrée}

\begin{entrée}
{bruit sourd}
\vedette{gu}\homonyme{3}
\région{GOs}
\variante{%
\vedette{gun}
\région{BO PA}}
\end{entrée}

\begin{entrée}
{cliqueter ; faire un bruit de cliquetis}
\vedette{pada}
\région{GOs}
\variante{%
\vedette{phada}
\région{PA}}
\end{entrée}

\begin{entrée}
{cri ; appel ; son}
\classe{v.i. ; n}
\vedette{tho}\homonyme{1}
\sens{1}
\région{GOs PA BO}
\end{entrée}

\begin{entrée}
{crier}
\vedette{kixi}
\région{GOs}
\end{entrée}

\begin{entrée}
{crier ; hurler ; pousser des cris ; crier de loin}
\vedette{kãã}
\région{GOs PA BO}
\région{PA}
\variante{%
\vedette{khêê}}
\end{entrée}

\begin{entrée}
{détonation ; coup (de fusil)}
\classe{v ; n}
\vedette{thi}\homonyme{3}
\sens{2}
\région{GOs PA BO}
\end{entrée}

\begin{entrée}
{détonner avec éclat ; tomber et se briser avec bruit [Corne]}
\vedette{burali}
\région{BO}
\end{entrée}

\begin{entrée}
{éclater}
\vedette{dra}\homonyme{2}
\région{GOs}
\end{entrée}

\begin{entrée}
{faire du bruit}
\vedette{gu}\homonyme{3}
\région{GOs}
\variante{%
\vedette{gun}
\région{BO PA}}
\end{entrée}

\begin{entrée}
{faire du bruit}
\vedette{wero}
\région{GOs}
\end{entrée}

\begin{entrée}
{faire un bruit de percussion (comme des maracas) ;}
\vedette{pada}
\région{GOs}
\variante{%
\vedette{phada}
\région{PA}}
\end{entrée}

\begin{entrée}
{gargouiller [BM]}
\vedette{kõõl}
\région{BO}
\end{entrée}

\begin{entrée}
{grincer}
\vedette{pixu}
\région{GOs}
\end{entrée}

\begin{entrée}
{grogner ; gronder ; grommeler ; maugréer ; bougonner}
\vedette{kûxû}\homonyme{2}
\région{GOs}
\variante{%
\vedette{kûxûl}
\région{BO PA}}
\end{entrée}

\begin{entrée}
{grommeler ; gronder ; murmurer (de mécontentement)}
\classe{v}
\vedette{caaxô}
\sens{1}
\région{GOs}
\région{GOs PA}
\variante{%
\vedette{caxõõl}
\région{PA}}
\variante{%
\vedette{caxool}
\région{BO}}
\variante{%
\vedette{cawhûûl}
\région{BO}}
\end{entrée}

\begin{entrée}
{grondement (tonnerre)}
\classe{v}
\vedette{hûn}
\sens{1}
\région{PA BO}
\end{entrée}

\begin{entrée}
{gronder}
\vedette{hû}\homonyme{2}
\région{GOs}
\end{entrée}

\begin{entrée}
{pétarader}
\vedette{thi-ma-thi}
\région{GOs}
\end{entrée}

\begin{entrée}
{son ; bruit (de paroles)}
\classe{nom}
\vedette{gaa}\homonyme{2}
\sens{2}
\région{GOs BO}
\variante{%
\vedette{gee}
\région{BO}}
\variante{%
\vedette{gèèn}
\région{BO [Corne]}}
\end{entrée}

\begin{entrée}
{son d'une vibration (comme un boomerang) ; plaque vibrante (musique)}
\vedette{phiri kûû}
\région{GOs}
\variante{%
\vedette{phili kûû}
\région{GO(s)}}
\end{entrée}

\begin{entrée}
{tonner ; gronder}
\classe{v}
\vedette{hûn}
\sens{1}
\région{PA BO}
\end{entrée}

\begin{entrée}
{vibrer ; faire du bruit}
\vedette{phiri kûû}
\région{GOs}
\variante{%
\vedette{phili kûû}
\région{GO(s)}}
\end{entrée}

\subsection{Couleurs}

\begin{entrée}
{blanc}
\classe{v.stat.}
\vedette{phozo}
\sens{1}
\région{GOs}
\variante{%
\vedette{polo, pulo}
\région{PA BO}}
\end{entrée}

\begin{entrée}
{bleu ; vert}
\vedette{phû}
\région{GOs}
\variante{%
\vedette{phûny}
\région{WEM BO PA}}
\end{entrée}

\begin{entrée}
{couleur}
\vedette{phago}
\région{GOs}
\end{entrée}

\begin{entrée}
{couleur ; dessin ; maquillage pour danser [PA BO]}
\vedette{gènè}
\région{GOs}
\variante{%
\vedette{gèn}
\région{PA BO}}
\end{entrée}

\begin{entrée}
{couleur [GOs]}
\vedette{gènè}
\région{GOs}
\variante{%
\vedette{gèn}
\région{PA BO}}
\end{entrée}

\begin{entrée}
{jaune ; orange ; curry}
\classe{nom}
\vedette{katri}
\sens{2}
\région{GOs}
\variante{%
\vedette{kari}
\région{PA BO}}
\end{entrée}

\begin{entrée}
{multicolore}
\vedette{drope}
\région{GOs}
\end{entrée}

\begin{entrée}
{noir}
\vedette{dòmã}
\région{PA BO [BM]}
\end{entrée}

\begin{entrée}
{noir ; noirci}
\vedette{baa}\homonyme{2}
\région{GOs}
\variante{%
\vedette{baang}
\région{PA}}
\variante{%
\vedette{baan}
\région{BO}}
\end{entrée}

\begin{entrée}
{rouge ; violet}
\classe{v.stat.}
\vedette{mii}
\sens{1}
\région{GOs PA BO}
\end{entrée}

\begin{entrée}
{teinture noire}
\classe{nom}
\vedette{bòdra}
\sens{1}
\région{GOs}
\variante{%
\vedette{bòda}
\région{BO}}
\end{entrée}

\begin{entrée}
{vert}
\classe{v.stat.}
\vedette{didi}
\sens{2}
\région{BO}
\end{entrée}

\begin{entrée}
{vert, bleu}
\vedette{bubu}
\région{BO PA}
\end{entrée}

\subsection{Caractéristiques et propriétés des personnes}

\begin{entrée}
{agile ; prompt ; vif}
\vedette{ta-zo}
\région{PA BO}
\end{entrée}

\begin{entrée}
{aimable ; doux ; gentil}
\vedette{ponyãã}
\région{GOs}
\variante{%
\vedette{ponyam}
\région{PA BO}}
\end{entrée}

\begin{entrée}
{ambidextre (lit. deux gauche)}
\vedette{wè-ru-mô}
\région{GOs}
\variante{%
\vedette{wòruumò}
\région{BO [BM]}}
\end{entrée}

\begin{entrée}
{apathique (être)}
\vedette{maagò}\homonyme{1}
\région{GOs}
\end{entrée}

\begin{entrée}
{apparence ; aspect}
\classe{n ; PREF. sémantique (référant à une surface extérieure)}
\vedette{ala-}
\sens{4}
\région{GOs PA BO}
\end{entrée}

\begin{entrée}
{autochtone, du pays ; véritable [BO]}
\vedette{gu}\homonyme{4}
\région{GOs BO [Corne]}
\end{entrée}

\begin{entrée}
{avoir l'habitude ; sage}
\vedette{kôzaxebi}
\région{GOs BO}
\variante{%
\vedette{kôzakebi}
\région{GO}}
\end{entrée}

\begin{entrée}
{bien ; bon}
\classe{v.stat.}
\vedette{zo}\homonyme{1}
\sens{1}
\région{GOs PA}
\variante{%
\vedette{zho}
\région{GO(s)}}
\variante{%
\vedette{yo}
\région{BO}}
\end{entrée}

\begin{entrée}
{bizarre ; distrait}
\vedette{kãmakã}
\région{GOs}
\end{entrée}

\begin{entrée}
{calme ; paisible}
\vedette{treçaaxo}
\région{GOs}
\end{entrée}

\begin{entrée}
{calme ; paisible ; humble (personne)}
\classe{nom}
\vedette{bwòòm}
\sens{1}
\région{BO [BM, Corne]}
\end{entrée}

\begin{entrée}
{courage ; courageux}
\classe{v.stat. ; n}
\vedette{cuxi}
\sens{2}
\région{PA BO}
\variante{%
\vedette{cuki}
\région{GO(s)}}
\variante{%
\vedette{cugi}
\région{BO}}
\end{entrée}

\begin{entrée}
{courageux ; travailleur}
\vedette{bwaaçu}
\région{GOs}
\variante{%
\vedette{bwaayu}
\région{WEM WE BO PA}}
\end{entrée}

\begin{entrée}
{courageux ; travailleur ; débrouillard}
\vedette{bemãbe}
\région{GOs}
\end{entrée}

\begin{entrée}
{crâner ; faire le malin}
\vedette{tho-me}
\région{PA}
\end{entrée}

\begin{entrée}
{débrouillard}
\vedette{pevera}
\région{WEM WE}
\end{entrée}

\begin{entrée}
{difficile}
\vedette{pu zòò}
\région{GOs}
\end{entrée}

\begin{entrée}
{drôle ; risible ; ridicule}
\vedette{kônya}
\région{GOs}
\variante{%
\vedette{kônyal}
\région{WEM WE PA}}
\end{entrée}

\begin{entrée}
{dynamique ; en forme [GOs]}
\classe{v}
\vedette{kôôbua}
\sens{1}
\région{GOs}
\variante{%
\vedette{kôôbwa}
\région{GO}}
\variante{%
\vedette{meebwa}
\région{PA WE WEM}}
\end{entrée}

\begin{entrée}
{entêté}
\vedette{thu bwa}
\région{GOs}
\end{entrée}

\begin{entrée}
{entêter (s') ; entêté [PA, GOs]}
\vedette{thu-ada}
\région{GOs PA}
\end{entrée}

\begin{entrée}
{faire l'intéressant ; faire le malin ; hautain}
\vedette{haulaa}
\région{GOs BO PA}
\end{entrée}

\begin{entrée}
{farceur ; qui joue des tours ; turbulent}
\vedette{a-pe-paçaxai}
\région{GOs}
\variante{%
\vedette{a-pe-pha-caaxai}
\région{PA}}
\end{entrée}

\begin{entrée}
{fier ; se faire remarquer faire (se) remarquer ; faire le malin}
\vedette{thu-me}
\région{GOs PA}
\end{entrée}

\begin{entrée}
{fort}
\vedette{wîî}
\région{GOs PA BO}
\variante{%
\vedette{wîî-n}
\région{BO}}
\variante{%
\vedette{wêê-n}
\région{BO [Corne]}}
\end{entrée}

\begin{entrée}
{fou (être)}
\vedette{òri}\homonyme{1}
\région{GOs BO}
\variante{%
\vedette{òtri}
\région{GO(s) WE}}
\end{entrée}

\begin{entrée}
{fou ; imbécile}
\vedette{kulèng}
\région{BO WE}
\end{entrée}

\begin{entrée}
{fragile (nourrisson)}
\classe{v.stat.}
\vedette{aava}
\sens{2}
\région{GOs}
\end{entrée}

\begin{entrée}
{gourmand ; glouton}
\vedette{pweuna}
\région{PA BO}
\end{entrée}

\begin{entrée}
{gras}
\vedette{phuzi}
\région{GOs}
\end{entrée}

\begin{entrée}
{habile}
\vedette{ala-hi}
\groupe{B}
\région{GOs BO PA}
\end{entrée}

\begin{entrée}
{habile ; qui a du savoir faire}
\vedette{zakèbi}
\région{GOs}
\variante{%
\vedette{zaxèbi}
\région{GO(s)}}
\variante{%
\vedette{zhaxèbi}
\région{GA}}
\end{entrée}

\begin{entrée}
{habitué à faire qqch}
\vedette{zakèbi}
\région{GOs}
\variante{%
\vedette{zaxèbi}
\région{GO(s)}}
\variante{%
\vedette{zhaxèbi}
\région{GA}}
\end{entrée}

\begin{entrée}
{hauteur ; taille (en hauteur)}
\vedette{phwaxi}
\région{GOs PA BO}
\variante{%
\vedette{phwaxa}
\région{GO(s) PA}}
\end{entrée}

\begin{entrée}
{image ; photo ; portrait (représentation)}
\classe{nom}
\vedette{hênu}
\sens{2}
\région{GOs}
\variante{%
\vedette{hînu}
\région{PA}}
\variante{%
\vedette{hênuul}
\région{BO}}
\end{entrée}

\begin{entrée}
{important}
\vedette{cińevwö}
\région{GOs BO}
\end{entrée}

\begin{entrée}
{jeune}
\classe{nom}
\vedette{poxa}\homonyme{1}
\sens{1}
\région{GOs PA GO}
\variante{%
\vedette{poga}
\région{PA}}
\end{entrée}

\begin{entrée}
{joli ; beau (personne, chose)}
\vedette{êgu-zo}
\région{GOs}
\variante{%
\vedette{ayò}
\région{BO PA}}
\end{entrée}

\begin{entrée}
{joli; bien}
\vedette{zoxãî}
\région{WEM WE}
\end{entrée}

\begin{entrée}
{las ; fatigué ; en avoir assez}
\vedette{mòza}
\région{GOs}
\variante{%
\vedette{mora}
\région{BO PA}}
\end{entrée}

\begin{entrée}
{lent}
\vedette{kha-maaçee}
\sens{1}
\région{GOs}
\end{entrée}

\begin{entrée}
{lent}
\classe{v}
\vedette{pwala-mwaji}
\sens{2}
\région{GOs}
\variante{%
\vedette{pwali-mwajin}
\région{PA}}
\end{entrée}

\begin{entrée}
{lent ; indolent}
\classe{v.stat.}
\vedette{beloo}
\sens{2}
\région{GOs PA}
\end{entrée}

\begin{entrée}
{long (être) à faire qqch}
\classe{v}
\vedette{pwala-mwaji}
\sens{2}
\région{GOs}
\variante{%
\vedette{pwali-mwajin}
\région{PA}}
\end{entrée}

\begin{entrée}
{maladroit ; gauche}
\vedette{ala}
\région{GOs}
\end{entrée}

\begin{entrée}
{maladroit ; gauche ; pas débrouillard}
\vedette{kônôô}
\région{WEM WE}
\variante{%
\vedette{kònôôl}
\région{BO}}
\end{entrée}

\begin{entrée}
{mauvais ; le mal}
\vedette{thrava}
\région{GOs}
\end{entrée}

\begin{entrée}
{mauvais ; mal}
\vedette{mwang}
\région{PA BO}
\end{entrée}

\begin{entrée}
{mou ; sans énergie}
\classe{v.stat.}
\vedette{beloo}
\sens{2}
\région{GOs PA}
\end{entrée}

\begin{entrée}
{muet}
\vedette{hõ}\homonyme{1}
\région{GOs}
\variante{%
\vedette{hom}
\région{PABO}}
\end{entrée}

\begin{entrée}
{négligent ; insouciant}
\vedette{kalanden}
\région{PA BO}
\end{entrée}

\begin{entrée}
{nerveux[BM]}
\vedette{yaali}
\région{BO}
\end{entrée}

\begin{entrée}
{nonchalant ; mou}
\classe{v}
\vedette{me-kônôô}
\sens{1}
\région{GOs}
\end{entrée}

\begin{entrée}
{obéir ; obéissant ; docile ; serviable ; prêt à aider ; bien disposé}
\classe{v}
\vedette{kôôbua}
\sens{2}
\région{GOs}
\variante{%
\vedette{kôôbwa}
\région{GO}}
\variante{%
\vedette{meebwa}
\région{PA WE WEM}}
\end{entrée}

\begin{entrée}
{orgueilleux ; faire le fier; manquer d'humilité}
\vedette{thu hubu}
\région{GOs}
\variante{%
\vedette{thu hubun}
\région{PA}}
\end{entrée}

\begin{entrée}
{orgueil ; vouloir surpasser}
\vedette{thu-ada}
\région{GOs PA}
\end{entrée}

\begin{entrée}
{paresseux [Corne]}
\vedette{maayèè}
\région{BO}
\end{entrée}

\begin{entrée}
{paresseux ; fainéant}
\vedette{aa-baatro}
\région{GOs}
\end{entrée}

\begin{entrée}
{paresseux (homme)}
\classe{v}
\vedette{kõńõõ}
\sens{1}
\région{GOs}
\variante{%
\vedette{kònôôl}
\région{PA}}
\end{entrée}

\begin{entrée}
{paresseux (humain)}
\vedette{baaro}
\région{GOs WEM BO PA}
\variante{%
\vedette{baatro}
\région{GO(s)}}
\end{entrée}

\begin{entrée}
{pauvre ; indigent}
\classe{v}
\vedette{ul}
\sens{1}
\région{PA BO}
\end{entrée}

\begin{entrée}
{planer (métaphoriquement réfère à qqn qui paresse)}
\vedette{pe-na-hi-n}
\région{PA}
\end{entrée}

\begin{entrée}
{pleurnicheur (métaphoriquement)}
\classe{nom}
\vedette{jèmaa}
\sens{2}
\région{BO PA}
\variante{%
\vedette{dada}
\région{GO(s) PA}}
\end{entrée}

\begin{entrée}
{propre}
\classe{v.stat.}
\vedette{zo}\homonyme{1}
\sens{2}
\région{GOs PA}
\variante{%
\vedette{zho}
\région{GO(s)}}
\variante{%
\vedette{yo}
\région{BO}}
\end{entrée}

\begin{entrée}
{puissance ; force}
\vedette{wîî}
\région{GOs PA BO}
\variante{%
\vedette{wîî-n}
\région{BO}}
\variante{%
\vedette{wêê-n}
\région{BO [Corne]}}
\end{entrée}

\begin{entrée}
{raisonnable ; mature (personne)}
\classe{v}
\vedette{thu ai-n}
\sens{2}
\région{PA}
\end{entrée}

\begin{entrée}
{rusé ; malin ; qui joue des tours}
\vedette{walaga}
\région{GOs}
\end{entrée}

\begin{entrée}
{ruser}
\vedette{garuã}
\région{GOs PA BO}
\end{entrée}

\begin{entrée}
{saoûl}
\vedette{kulèng}
\région{BO WE}
\end{entrée}

\begin{entrée}
{saoûl (être) (lit. la tête tourne)}
\classe{v ; n}
\vedette{kênõ}
\sens{3}
\région{GOs}
\région{BO PA}
\variante{%
\vedette{kênõng}}
\end{entrée}

\begin{entrée}
{saoûl ; ivre}
\vedette{òri}\homonyme{1}
\région{GOs BO}
\variante{%
\vedette{òtri}
\région{GO(s) WE}}
\end{entrée}

\begin{entrée}
{silencieux ; silence [Corne]}
\vedette{whun}
\région{BO}
\end{entrée}

\begin{entrée}
{sourd}
\vedette{khimò}
\région{GOs}
\variante{%
\vedette{khimòn}
\région{BO PA}}
\end{entrée}

\begin{entrée}
{timide ; doux}
\vedette{mhwi}
\région{BO}
\end{entrée}

\begin{entrée}
{tranquille ; sage ; immobile}
\vedette{nenèm}
\région{PA}
\variante{%
\vedette{nenèèm}
\région{BO}}
\end{entrée}

\begin{entrée}
{travailleur ; courageux}
\vedette{a-bwaayu}
\région{WEM}
\end{entrée}

\begin{entrée}
{ventru ; corpulent}
\vedette{po-kiò}
\région{GOs PA BO}
\end{entrée}

\begin{entrée}
{vif ; agile ; dynamique}
\classe{v}
\vedette{hibil}
\région{PA BO}
\end{entrée}

\begin{entrée}
{vif ; dynamique ; en forme}
\vedette{thani}
\région{GOs WEM WE}
\end{entrée}

\begin{entrée}
{vigoureux ; costaud ; courageux (qualifie le corps)}
\vedette{mweeja}
\région{GOs}
\end{entrée}

\begin{entrée}
{voyeur (être)}
\vedette{nobe}
\région{GOs}
\end{entrée}

\begin{entrée}
{vrai ; droit}
\vedette{gu}\homonyme{4}
\région{GOs BO [Corne]}
\end{entrée}

\subsection{Caractéristiques et propriétés des animaux}

\begin{entrée}
{craintif (animaux surtout)}
\classe{v}
\vedette{kari}
\sens{3}
\région{PA}
\end{entrée}

\begin{entrée}
{doux (animal) ; apprivoisé}
\classe{v}
\vedette{me-kônôô}
\sens{2}
\région{GOs}
\end{entrée}

\begin{entrée}
{doux (animal) ; apprivoisé (animal)}
\classe{v}
\vedette{kõńõõ}
\sens{2}
\région{GOs}
\variante{%
\vedette{kònôôl}
\région{PA}}
\end{entrée}

\begin{entrée}
{dressé (cheval, animal)}
\classe{v}
\vedette{thu ai-n}
\sens{1}
\région{PA}
\end{entrée}

\begin{entrée}
{pas dressé}
\vedette{kiya ai-n}
\région{PA}
\end{entrée}

\begin{entrée}
{petit (le petit d'un animal)}
\classe{nom}
\vedette{poxa}\homonyme{1}
\sens{2}
\région{GOs PA GO}
\variante{%
\vedette{poga}
\région{PA}}
\end{entrée}

\begin{entrée}
{plein (crabe)}
\classe{v}
\vedette{pònu}
\sens{2}
\région{GOs BO}
\variante{%
\vedette{pwònu}
\région{BO}}
\end{entrée}

\begin{entrée}
{sauvage ; non domestiqué}
\vedette{thua}
\région{GOs PA BO}
\end{entrée}

\subsection{Caractéristiques et propriétés des objets}

\subsubsection{Description des objets, formes, consistance, taille}

\begin{entrée}
{agréable ; beau ; merveilleux (termes de l'église)}
\vedette{ẽnõbau}
\région{GOs}
\end{entrée}

\begin{entrée}
{aiguisé; coupant}
\vedette{cha}
\région{PA}
\end{entrée}

\begin{entrée}
{aplatir ; aplati}
\vedette{kaleva}
\région{PA BO [BM]}
\région{BO}
\variante{%
\vedette{kaleba}}
\end{entrée}

\begin{entrée}
{asséché ; sec (rivière, etc.)}
\vedette{mõ}\homonyme{1}
\sens{1}
\région{GOs}
\région{PA BO}
\variante{%
\vedette{mòl}}
\end{entrée}

\begin{entrée}
{avoir des fruits}
\vedette{pu pwò}
\région{GOs}
\end{entrée}

\begin{entrée}
{avoir des tubercules}
\vedette{pu pai}
\région{GOs}
\end{entrée}

\begin{entrée}
{avoir un contenu}
\vedette{pu hê}
\région{GOs}
\end{entrée}

\begin{entrée}
{avoir un fond}
\vedette{pu punõ}
\région{GOs}
\end{entrée}

\begin{entrée}
{bout, extrémité d'une surface ou d'une chose étendue}
\classe{nom}
\vedette{ku}\homonyme{5}
\sens{2}
\région{GOs BO PA}
\end{entrée}

\begin{entrée}
{brillant ; scintillant}
\vedette{mazido}
\région{GOs}
\end{entrée}

\begin{entrée}
{calme ; paisible (temps, atmosphère, personne)}
\vedette{khô}\homonyme{2}
\région{GOs}
\end{entrée}

\begin{entrée}
{cassé (verre)}
\vedette{ha}\homonyme{2}
\région{GOs}
\end{entrée}

\begin{entrée}
{cher [PA, BO]}
\vedette{pwalu}
\région{GOs}
\variante{%
\vedette{pwaalu}
\région{PA BO}}
\end{entrée}

\begin{entrée}
{choc ; impact}
\vedette{bii}\homonyme{1}
\région{GOs}
\end{entrée}

\begin{entrée}
{chose}
\vedette{pwaixe}
\région{GOs PA}
\variante{%
\vedette{pwaike}
\région{GO}}
\end{entrée}

\begin{entrée}
{coincé ; bloqué}
\classe{v}
\vedette{tigi}\homonyme{1}
\sens{1}
\région{GOs PA BO}
\variante{%
\vedette{tigin}
\région{WE}}
\end{entrée}

\begin{entrée}
{col de la marmite}
\vedette{nõõ-do}
\région{BO}
\end{entrée}

\begin{entrée}
{collant (sous la dent) [PA]}
\classe{v.stat.}
\vedette{bizigi}
\sens{1}
\région{GOs}
\variante{%
\vedette{birigi}
\région{GO(s) WE}}
\variante{%
\vedette{bitigi}
\région{PA BO}}
\end{entrée}

\begin{entrée}
{collé}
\classe{v.stat.}
\vedette{bizigi}
\sens{1}
\région{GOs}
\variante{%
\vedette{birigi}
\région{GO(s) WE}}
\variante{%
\vedette{bitigi}
\région{PA BO}}
\end{entrée}

\begin{entrée}
{collé (par ex. au fond de la marmite)}
\vedette{thòxe}
\région{PA}
\variante{%
\vedette{thòxe, thòòge}
\région{BO [BM]}}
\end{entrée}

\begin{entrée}
{coloré}
\vedette{pu gènè}
\région{GOs}
\end{entrée}

\begin{entrée}
{couleur}
\vedette{pu gènè}
\région{GOs}
\end{entrée}

\begin{entrée}
{coupant ; bien aiguisé}
\vedette{ca}\homonyme{1}
\région{GOs BO}
\end{entrée}

\begin{entrée}
{court ; petit}
\vedette{pònum}
\région{PA BO}
\variante{%
\vedette{pònèn, pwanèn}
\région{BO}}
\end{entrée}

\begin{entrée}
{court ; ras}
\classe{v}
\vedette{thra}\homonyme{2}
\sens{2}
\région{GOs}
\variante{%
\vedette{tha, thaa}
\région{BO PA}}
\end{entrée}

\begin{entrée}
{court (taille, hauteur)}
\vedette{povwonû}
\région{GOs}
\end{entrée}

\begin{entrée}
{crisser (sous la dent) ; abrasif}
\vedette{nhi}\homonyme{1}
\région{BO}
\end{entrée}

\begin{entrée}
{croquant (sous la dent ; se dit d'un fruit pas mûr comme la mangue ou la papaye)}
\vedette{galò}
\région{GOs PA}
\end{entrée}

\begin{entrée}
{dangereux}
\vedette{tre-raa}
\région{GOs}
\end{entrée}

\begin{entrée}
{débris laissés par l'inondation}
\vedette{phòlò ja}
\région{GOs}
\variante{%
\vedette{phòlò jang}
\région{PA}}
\end{entrée}

\begin{entrée}
{délabré}
\vedette{mudro}
\région{GOs}
\variante{%
\vedette{mudrã}
\région{GO(s)}}
\variante{%
\vedette{mudo}
\région{PA}}
\variante{%
\vedette{muda, mudo}
\région{BO}}
\end{entrée}

\begin{entrée}
{dense}
\classe{nom}
\vedette{tigi}\homonyme{3}
\sens{2}
\région{GOs BO PA}
\end{entrée}

\begin{entrée}
{différent ; à part ; à l'écart ; bizarre}
\vedette{haze}\homonyme{1}
\sens{1}
\région{GOs}
\variante{%
\vedette{hale}
\région{PA BO}}
\end{entrée}

\begin{entrée}
{difficulté; embûche}
\vedette{zòò}\homonyme{1}
\région{GOs}
\end{entrée}

\begin{entrée}
{droit ; rectiligne}
\classe{v.stat.}
\vedette{ku-gòò}
\groupe{A}
\sens{1}
\région{GOs}
\variante{%
\vedette{kô-go}
\région{BO}}
\end{entrée}

\begin{entrée}
{dur}
\vedette{piça}
\région{GOs}
\variante{%
\vedette{piya}
\région{BO}}
\end{entrée}

\begin{entrée}
{durée}
\vedette{phwaxa}
\région{GOs PA}
\end{entrée}

\begin{entrée}
{dur (pain)}
\classe{v.stat.}
\vedette{kulaçe}
\sens{2}
\région{GOs}
\variante{%
\vedette{kulaye}
\région{PA}}
\end{entrée}

\begin{entrée}
{ébréché (pour une lame de couteau)}
\classe{v.stat.}
\vedette{wado}
\sens{2}
\région{GOs BO}
\variante{%
\vedette{waado}
\région{BO}}
\end{entrée}

\begin{entrée}
{écrasé ; mou (une fois écrasé)}
\vedette{nhya}
\région{GOs}
\variante{%
\vedette{nhyal}
\région{PA}}
\end{entrée}

\begin{entrée}
{émoussé [GOs]}
\classe{v.stat.}
\vedette{neńe}\homonyme{2}
\région{GOs}
\variante{%
\vedette{nèn}
\région{PA}}
\end{entrée}

\begin{entrée}
{enfoncé ; cabossé}
\vedette{bii}\homonyme{1}
\région{GOs}
\end{entrée}

\begin{entrée}
{englué ; collé ; emmêlé ; embrouillé ; enchevêtré}
\classe{v ; n}
\vedette{tigi}\homonyme{2}
\sens{1}
\région{GOs PA BO}
\end{entrée}

\begin{entrée}
{entrouvert}
\vedette{pò-thala}
\région{GOs}
\end{entrée}

\begin{entrée}
{épais}
\classe{v.stat.}
\vedette{hûgu}\homonyme{1}
\région{GOs}
\end{entrée}

\begin{entrée}
{épais}
\vedette{thrãbo}
\région{GOs}
\end{entrée}

\begin{entrée}
{épais ; emmêlé ; inextricable}
\classe{v}
\vedette{tigi}\homonyme{1}
\sens{1}
\région{GOs PA BO}
\variante{%
\vedette{tigin}
\région{WE}}
\end{entrée}

\begin{entrée}
{étanche}
\classe{v ; n}
\vedette{thô}
\sens{2}
\région{GOs}
\variante{%
\vedette{thõn}
\région{BO}}
\end{entrée}

\begin{entrée}
{étroit (passage en mer, sur terre)}
\vedette{pivwizai}
\région{GOs}
\end{entrée}

\begin{entrée}
{facile}
\vedette{kixa zòò}
\région{GOs}
\end{entrée}

\begin{entrée}
{félé ; cassé}
\vedette{gaaò}
\sens{2}
\région{PA BO WE}
\variante{%
\vedette{ha}
\région{GO(s)}}
\end{entrée}

\begin{entrée}
{fente ; craquelure ; craquelé; fissuré (terre)}
\vedette{kha}\homonyme{4}
\région{PA BO}
\end{entrée}

\begin{entrée}
{ferme ; dur ; solide}
\classe{MODIF}
\vedette{jamali}
\groupe{B}
\région{GOs WEM BO PA}
\end{entrée}

\begin{entrée}
{fermé (tout seul: porte, fenêtre, etc.)}
\vedette{trezô}
\région{GOs}
\end{entrée}

\begin{entrée}
{fin ; mince}
\vedette{cirara}
\région{GOs}
\variante{%
\vedette{citratra}
\région{GO(s)}}
\end{entrée}

\begin{entrée}
{fissuré ; crevassé ; crevasser (se) ; fissurer (se)}
\vedette{trö}
\région{GOs}
\end{entrée}

\begin{entrée}
{fond}
\vedette{punõ}\homonyme{1}
\région{GOs BO PA}
\end{entrée}

\begin{entrée}
{fond de la marmite}
\vedette{punõõ-dröö}
\région{GOs}
\variante{%
\vedette{punõõ-döö}
\région{PA}}
\end{entrée}

\begin{entrée}
{fond de la marmite (face externe)}
\vedette{paxa-nô-döö}
\région{PA}
\end{entrée}

\begin{entrée}
{fond de l'eau}
\vedette{punõõ-we}
\région{GOs BO PA}
\end{entrée}

\begin{entrée}
{forme}
\vedette{me-trabwa}
\région{GOs}
\end{entrée}

\begin{entrée}
{fort}
\vedette{piça}
\région{GOs}
\variante{%
\vedette{piya}
\région{BO}}
\end{entrée}

\begin{entrée}
{gâté ; abîmé}
\vedette{halan}
\région{PA BO [BM]}
\end{entrée}

\begin{entrée}
{gâté (fruit)}
\classe{v.stat.}
\vedette{nhyal}
\sens{1}
\région{PA BO}
\end{entrée}

\begin{entrée}
{glissant}
\classe{v}
\vedette{heela}
\sens{2}
\région{GOs BO}
\end{entrée}

\begin{entrée}
{grand}
\vedette{pwali}\homonyme{2}
\région{BO PA}
\end{entrée}

\begin{entrée}
{grand}
\vedette{phwawali}
\région{GOs BO PA}
\variante{%
\vedette{pwapwali}
\région{BO PA}}
\variante{%
\vedette{pwali}
\région{BO}}
\end{entrée}

\begin{entrée}
{grand ; gros}
\vedette{whaa}\homonyme{1}
\région{GOs BO}
\end{entrée}

\begin{entrée}
{grand ; gros ; volumineux}
\vedette{hãxãî}
\région{GOs PA BO}
\variante{%
\vedette{hangai, angai}
\région{PA BO}}
\variante{%
\vedette{hã}
\région{PA}}
\end{entrée}

\begin{entrée}
{grandir ; croître; pousser (en long)}
\vedette{whaa}\homonyme{1}
\région{GOs BO}
\end{entrée}

\begin{entrée}
{gros ; grand}
\vedette{hã}\homonyme{2}
\région{PA BO}
\variante{%
\vedette{hangai}
\région{PA BO}}
\end{entrée}

\begin{entrée}
{grosseur ; épaisseur}
\vedette{chińõ}
\groupe{A}
\région{GOs PA}
\variante{%
\vedette{cinõ}
\région{BO}}
\end{entrée}

\begin{entrée}
{haillons ; loques}
\vedette{mudro}
\région{GOs}
\variante{%
\vedette{mudrã}
\région{GO(s)}}
\variante{%
\vedette{mudo}
\région{PA}}
\variante{%
\vedette{muda, mudo}
\région{BO}}
\end{entrée}

\begin{entrée}
{hauteur}
\vedette{phwaxa}
\région{GOs PA}
\end{entrée}

\begin{entrée}
{humide}
\vedette{mhã}\homonyme{1}
\région{GOs}
\variante{%
\vedette{mhãm}
\région{BO}}
\end{entrée}

\begin{entrée}
{immobile}
\vedette{khône}
\région{GOs}
\end{entrée}

\begin{entrée}
{imperceptible}
\vedette{pogabe}
\région{GOs BO}
\end{entrée}

\begin{entrée}
{invisible [Corne]}
\vedette{mimaalu}
\région{BO}
\end{entrée}

\begin{entrée}
{lâche (être) ; grand}
\vedette{halalaò}
\région{PA}
\end{entrée}

\begin{entrée}
{large}
\vedette{wala}
\région{GOs}
\end{entrée}

\begin{entrée}
{léger}
\vedette{haòm}
\région{PA}
\variante{%
\vedette{aom}
\région{PA BO}}
\end{entrée}

\begin{entrée}
{léger}
\vedette{pumhãmã}
\région{GOs}
\end{entrée}

\begin{entrée}
{lisse}
\vedette{valèèma}
\région{GOs}
\end{entrée}

\begin{entrée}
{lisse (cheveux)}
\classe{v}
\vedette{heela}
\sens{2}
\région{GOs BO}
\end{entrée}

\begin{entrée}
{lit de la rivière [BO]}
\vedette{punõõ-we}
\région{GOs BO PA}
\end{entrée}

\begin{entrée}
{long ; haut}
\vedette{pwali}\homonyme{2}
\région{BO PA}
\end{entrée}

\begin{entrée}
{long ; haut}
\vedette{phwawali}
\région{GOs BO PA}
\variante{%
\vedette{pwapwali}
\région{BO PA}}
\variante{%
\vedette{pwali}
\région{BO}}
\end{entrée}

\begin{entrée}
{longtemps}
\vedette{pwali}\homonyme{2}
\région{BO PA}
\end{entrée}

\begin{entrée}
{longtemps}
\vedette{phwawali}
\région{GOs BO PA}
\variante{%
\vedette{pwapwali}
\région{BO PA}}
\variante{%
\vedette{pwali}
\région{BO}}
\end{entrée}

\begin{entrée}
{longueur}
\vedette{phwaxa}
\région{GOs PA}
\end{entrée}

\begin{entrée}
{long (verticalement : arbre, personne)}
\vedette{khawali}
\région{GOs PA}
\end{entrée}

\begin{entrée}
{lourd ; grave}
\vedette{pwalu}
\région{GOs}
\variante{%
\vedette{pwaalu}
\région{PA BO}}
\end{entrée}

\begin{entrée}
{mauvais ; dangereux}
\vedette{tre-raa}
\région{GOs}
\end{entrée}

\begin{entrée}
{mauvais ; dangereux ; mal}
\vedette{thraa}
\région{GOs}
\variante{%
\vedette{thaa}
\région{PA BO}}
\variante{%
\vedette{mwang}
\région{BO}}
\end{entrée}

\begin{entrée}
{menu}
\classe{v.stat.}
\vedette{popobe}
\sens{1}
\région{PA BO [Dubois]}
\variante{%
\vedette{pobe}
\région{WE}}
\end{entrée}

\begin{entrée}
{mince ; étroit}
\vedette{pogabe}
\région{GOs BO}
\end{entrée}

\begin{entrée}
{mince ; fin}
\vedette{terè}
\région{PA BO}
\end{entrée}

\begin{entrée}
{mou}
\classe{v.stat.}
\vedette{nhyal}
\sens{1}
\région{PA BO}
\end{entrée}

\begin{entrée}
{mou ; flasque}
\vedette{kugoo}
\région{GOs}
\end{entrée}

\begin{entrée}
{mouillé ; humide}
\vedette{bee}\homonyme{1}
\région{GOs}
\variante{%
\vedette{been}
\région{PA BO}}
\end{entrée}

\begin{entrée}
{mou ; lisse (cheveux)}
\vedette{tò}
\région{GO}
\end{entrée}

\begin{entrée}
{mou ; trop mûr}
\classe{v.stat.}
\vedette{nhyatru}
\sens{1}
\région{GOs}
\variante{%
\vedette{nhyaru}
\région{GO(s)}}
\end{entrée}

\begin{entrée}
{négatif d'une photo}
\vedette{kãgu-hênû}
\région{GOs}
\end{entrée}

\begin{entrée}
{niveau (d'eau) ; profondeur}
\vedette{nhyôgo we}
\région{GOs}
\variante{%
\vedette{nyongo we}
\région{BO}}
\end{entrée}

\begin{entrée}
{nouveau ; neuf}
\vedette{hõ}\homonyme{2}
\groupe{A}
\région{GOs}
\variante{%
\vedette{hô}
\région{PA}}
\variante{%
\vedette{hò}
\région{BO}}
\end{entrée}

\begin{entrée}
{objet emballé}
\vedette{pòdòu}
\région{GOs}
\end{entrée}

\begin{entrée}
{objet ou bois flotté (apporté par la mer ou la rivière)}
\vedette{phòlò ja}
\région{GOs}
\variante{%
\vedette{phòlò jang}
\région{PA}}
\end{entrée}

\begin{entrée}
{objets ou débris flottés}
\vedette{jòòwe}
\région{GOs}
\end{entrée}

\begin{entrée}
{occuper tout l'espace}
\vedette{chińõ}
\groupe{B}
\région{GOs PA}
\variante{%
\vedette{cinõ}
\région{BO}}
\end{entrée}

\begin{entrée}
{ombre ; endroit ombragé [ BO PA]}
\classe{nom}
\vedette{bwòòm}
\sens{2}
\région{BO [BM, Corne]}
\end{entrée}

\begin{entrée}
{pas stable (route, chemin)}
\vedette{tre-raa}
\région{GOs}
\end{entrée}

\begin{entrée}
{petit}
\classe{v.stat.}
\vedette{pobe}
\sens{1}
\région{PA BO}
\variante{%
\vedette{pwobe}
\région{BO}}
\end{entrée}

\begin{entrée}
{petit ; court}
\classe{v.stat.}
\vedette{popobe}
\sens{1}
\région{PA BO [Dubois]}
\variante{%
\vedette{pobe}
\région{WE}}
\end{entrée}

\begin{entrée}
{petit (épaisseur ; par opposition à gros)}
\vedette{pòvwònò}
\région{GOs}
\end{entrée}

\begin{entrée}
{piquant (piment)}
\classe{v}
\vedette{maü}
\sens{2}
\région{BO}
\end{entrée}

\begin{entrée}
{plat}
\vedette{kaleva}
\région{PA BO [BM]}
\région{BO}
\variante{%
\vedette{kaleba}}
\end{entrée}

\begin{entrée}
{plat ; aplati}
\vedette{thovwa}
\région{GOs}
\variante{%
\vedette{thòva}
\région{BO [BM]}}
\end{entrée}

\begin{entrée}
{plein (être)}
\classe{v}
\vedette{pònu}
\sens{1}
\région{GOs BO}
\variante{%
\vedette{pwònu}
\région{BO}}
\end{entrée}

\begin{entrée}
{pointe ; avant}
\vedette{me-}\homonyme{1}
\région{GOs PA BO}
\variante{%
\vedette{mèè-n}
\région{BO}}
\end{entrée}

\begin{entrée}
{pointu}
\vedette{dom}
\région{PA}
\variante{%
\vedette{dum}
\région{WE WEM}}
\end{entrée}

\begin{entrée}
{pointu}
\vedette{mero-dö}
\région{GOs}
\end{entrée}

\begin{entrée}
{pourri (être) ; sentir}
\vedette{bwo}\homonyme{1}
\région{GOs BO}
\variante{%
\vedette{bo}
\région{PA BO}}
\end{entrée}

\begin{entrée}
{pris (dans un filet)}
\classe{v}
\vedette{tigi}\homonyme{1}
\sens{1}
\région{GOs PA BO}
\variante{%
\vedette{tigin}
\région{WE}}
\end{entrée}

\begin{entrée}
{profond [Corne]}
\classe{v.stat.}
\vedette{didi}
\sens{1}
\région{BO}
\end{entrée}

\begin{entrée}
{profond ; invisible}
\vedette{ni-malu}
\région{GO}
\end{entrée}

\begin{entrée}
{profond (mer ou rivière)}
\vedette{ni}\homonyme{1}
\région{GOs}
\variante{%
\vedette{nhim}
\région{PA WEM WE BO}}
\end{entrée}

\begin{entrée}
{propre ; neuf}
\classe{v.stat.}
\vedette{phozo}
\sens{2}
\région{GOs}
\variante{%
\vedette{polo, pulo}
\région{PA BO}}
\end{entrée}

\begin{entrée}
{raide (être)}
\vedette{kaayang}
\région{PA BO [Corne]}
\end{entrée}

\begin{entrée}
{raide mort}
\vedette{kaayang}
\région{PA BO [Corne]}
\end{entrée}

\begin{entrée}
{ramollir ; étirer (s') (comme de la gomme)}
\vedette{òòl}
\région{PA BO [Corne]}
\end{entrée}

\begin{entrée}
{récent}
\vedette{hõ}\homonyme{2}
\groupe{A}
\région{GOs}
\variante{%
\vedette{hô}
\région{PA}}
\variante{%
\vedette{hò}
\région{BO}}
\end{entrée}

\begin{entrée}
{résistant}
\vedette{piça}
\région{GOs}
\variante{%
\vedette{piya}
\région{BO}}
\end{entrée}

\begin{entrée}
{résistant (corde, personne, viande)}
\vedette{wagiça}
\région{GOs}
\end{entrée}

\begin{entrée}
{rond}
\vedette{tretrabwau}
\région{GOs}
\variante{%
\vedette{terebwau}
\région{PA BO}}
\end{entrée}

\begin{entrée}
{rond [BM]}
\vedette{bwarabo}
\région{BO}
\end{entrée}

\begin{entrée}
{rouille}
\vedette{zawexan}
\région{PA BO}
\variante{%
\vedette{yawegan}
\région{BO}}
\end{entrée}

\begin{entrée}
{rouillé}
\vedette{thodia}
\région{GOs}
\variante{%
\vedette{thidya, tidya}
\région{BO}}
\end{entrée}

\begin{entrée}
{rugueux}
\vedette{pezoli}
\région{GOs}
\variante{%
\vedette{peroli}
\région{PA}}
\end{entrée}

\begin{entrée}
{rugueux[Corne]}
\vedette{thaao}\homonyme{2}
\région{BO}
\end{entrée}

\begin{entrée}
{sale}
\vedette{gee}
\groupe{A}
\région{GOs}
\variante{%
\vedette{gèèng}
\région{BO PA}}
\variante{%
\vedette{gèènk}
\région{BO vx}}
\end{entrée}

\begin{entrée}
{saleté ; détritus}
\vedette{gee}
\groupe{B}
\région{GOs}
\variante{%
\vedette{gèèng}
\région{BO PA}}
\variante{%
\vedette{gèènk}
\région{BO vx}}
\end{entrée}

\begin{entrée}
{saletés ; ordures ; détritus ; déchets}
\vedette{ja}\homonyme{1}
\région{GOs BO}
\end{entrée}

\begin{entrée}
{sec (linge, feuille) ; asséché (rivière)}
\vedette{maû}\homonyme{1}
\région{GOs PA BO}
\variante{%
\vedette{maûng}
\région{BO}}
\end{entrée}

\begin{entrée}
{solide}
\vedette{piça}
\région{GOs}
\variante{%
\vedette{piya}
\région{BO}}
\end{entrée}

\begin{entrée}
{tache}
\vedette{gee}
\groupe{B}
\région{GOs}
\variante{%
\vedette{gèèng}
\région{BO PA}}
\variante{%
\vedette{gèènk}
\région{BO vx}}
\end{entrée}

\begin{entrée}
{taché}
\vedette{gee}
\groupe{A}
\région{GOs}
\variante{%
\vedette{gèèng}
\région{BO PA}}
\variante{%
\vedette{gèènk}
\région{BO vx}}
\end{entrée}

\begin{entrée}
{taille ; circonférence}
\vedette{chińõ}
\groupe{A}
\région{GOs PA}
\variante{%
\vedette{cinõ}
\région{BO}}
\end{entrée}

\begin{entrée}
{talus}
\vedette{ba-thrôbo}
\région{GOs}
\end{entrée}

\begin{entrée}
{tendu}
\vedette{kaayang}
\région{PA BO [Corne]}
\end{entrée}

\begin{entrée}
{tendu (corde) ; raide}
\vedette{kaaça}
\région{GOs}
\end{entrée}

\begin{entrée}
{tordu}
\classe{v}
\vedette{phõge}
\sens{2}
\région{GOs PA}
\variante{%
\vedette{phõng}
\région{BO}}
\end{entrée}

\begin{entrée}
{tordu ; tors}
\vedette{phõ}
\région{GOs}
\variante{%
\vedette{phõng}
\région{PA BO WE WEM}}
\end{entrée}

\begin{entrée}
{toxique ; non-comestible ; poison}
\classe{v ; n}
\vedette{zòn}
\sens{1}
\région{PA}
\variante{%
\vedette{yòn, yhòn}
\région{BO}}
\end{entrée}

\begin{entrée}
{transparent ; limpide (eau) ; clair}
\vedette{mèloo}
\région{GOs}
\variante{%
\vedette{mèloom}
\région{PA BO WEM WE}}
\variante{%
\vedette{malòm}
\région{BO PA}}
\end{entrée}

\begin{entrée}
{transparent (voir à travers)}
\vedette{no thiraò}
\région{GOs}
\end{entrée}

\begin{entrée}
{très petit}
\vedette{pogabe}
\région{GOs BO}
\end{entrée}

\begin{entrée}
{vide}
\classe{v.stat. ; n}
\vedette{pii}\homonyme{3}
\sens{1}
\région{GOs PA BO}
\variante{%
\vedette{pii-n}
\région{BO}}
\end{entrée}

\begin{entrée}
{vide (lit. pas de contenu)}
\classe{LOCUT}
\vedette{kixa hê}
\région{GOs}
\end{entrée}

\begin{entrée}
{vieux ; usé (linge)}
\vedette{mudro}
\région{GOs}
\variante{%
\vedette{mudrã}
\région{GO(s)}}
\variante{%
\vedette{mudo}
\région{PA}}
\variante{%
\vedette{muda, mudo}
\région{BO}}
\end{entrée}

\begin{entrée}
{vrai}
\classe{v.stat.}
\vedette{ku-gòò}
\groupe{A}
\sens{2}
\région{GOs}
\variante{%
\vedette{kô-go}
\région{BO}}
\end{entrée}

\subsubsection{Configuration des objets}

\begin{entrée}
{bananes jumelles (dans une même enveloppe)}
\classe{nom}
\vedette{digo}
\sens{1}
\région{GOs WEM WE BO}
\end{entrée}

\begin{entrée}
{bouquet de paille à la main}
\classe{v ; n}
\vedette{thatra-hi}
\sens{1}
\région{GOs}
\end{entrée}

\begin{entrée}
{bout (d'une chose longue) ; extrémité ; fin ; terme}
\classe{nom}
\vedette{hulò}\homonyme{1}
\sens{1}
\région{GOs BO PA}
\end{entrée}

\begin{entrée}
{brancher}
\vedette{tèng}
\région{BO}
\end{entrée}

\begin{entrée}
{carré (lit. 4 coins)}
\vedette{pò-pa-kudi}
\région{GOs}
\end{entrée}

\begin{entrée}
{coin ; angle}
\vedette{kudi}
\région{GOs PA}
\end{entrée}

\begin{entrée}
{coin ; angle}
\vedette{thixudi}
\région{PA BO}
\variante{%
\vedette{thivwudi}
\région{PA BO}}
\end{entrée}

\begin{entrée}
{côté ; bord ; extrémité ; lisière}
\vedette{kòlò}
\sens{1}
\région{GOs PA BO}
\variante{%
\vedette{kòlò-n}
\région{BO}}
\variante{%
\vedette{kòli}
\région{GO(s) PA}}
\end{entrée}

\begin{entrée}
{creux}
\classe{nom}
\vedette{phwa}\homonyme{1}
\groupe{A}
\sens{2}
\région{GOs PA BO}
\end{entrée}

\begin{entrée}
{façade ; surface}
\classe{n ; PREF. sémantique (référant à une surface extérieure)}
\vedette{ala-}
\sens{2}
\région{GOs PA BO}
\end{entrée}

\begin{entrée}
{fagot (bois, canne à sucre)}
\classe{nom}
\vedette{bwalò}
\groupe{A}
\sens{1}
\région{GOs PA BO}
\variante{%
\vedette{pwalò}
\région{GO(s)}}
\end{entrée}

\begin{entrée}
{fagot de bois}
\vedette{phò-ce}
\région{GOs}
\end{entrée}

\begin{entrée}
{fagot de canne à sucre (dans les cérémonies)}
\vedette{phò-ê}
\région{GOs}
\end{entrée}

\begin{entrée}
{flanc}
\vedette{kòlò}
\sens{1}
\région{GOs PA BO}
\variante{%
\vedette{kòlò-n}
\région{BO}}
\variante{%
\vedette{kòli}
\région{GO(s) PA}}
\end{entrée}

\begin{entrée}
{fourche (arbre)}
\classe{nom}
\vedette{digo}
\sens{1}
\région{GOs WEM WE BO}
\end{entrée}

\begin{entrée}
{fourche [BM]}
\vedette{tèng}
\région{BO}
\end{entrée}

\begin{entrée}
{ligne ; alignement ; rangée (ignames, poteaux, etc.)}
\vedette{îdò-}
\sens{1}
\région{GOs PA BO}
\end{entrée}

\begin{entrée}
{limite ; bout ; fin}
\vedette{bala}\homonyme{3}
\région{GOs}
\variante{%
\vedette{bala-n}
\région{PA BO}}
\end{entrée}

\begin{entrée}
{limite du labour (là où on s'est arrêté)}
\vedette{bala-khò}
\région{PA}
\end{entrée}

\begin{entrée}
{orifice}
\classe{nom}
\vedette{phwa}\homonyme{1}
\groupe{A}
\sens{2}
\région{GOs PA BO}
\end{entrée}

\begin{entrée}
{ouverture}
\classe{nom}
\vedette{phwa}\homonyme{1}
\groupe{A}
\sens{2}
\région{GOs PA BO}
\end{entrée}

\begin{entrée}
{percé}
\classe{v.stat.}
\vedette{phwa}\homonyme{1}
\groupe{B}
\sens{1}
\région{GOs PA BO}
\end{entrée}

\begin{entrée}
{séparer ; séparation ; frontière}
\vedette{taabö}
\région{GOs}
\end{entrée}

\begin{entrée}
{soudure (par ex. des 2 parties du crâne) ; raccord}
\vedette{mhenõ-pe-ki}
\région{GOs}
\end{entrée}

\begin{entrée}
{tas d'igname (dans les cérémonies)}
\vedette{phò-kui}
\région{GOs}
\end{entrée}

\begin{entrée}
{tas (feuilles, coco)}
\classe{nom}
\vedette{bwalò}
\groupe{A}
\sens{1}
\région{GOs PA BO}
\variante{%
\vedette{pwalò}
\région{GO(s)}}
\end{entrée}

\begin{entrée}
{trou}
\classe{nom}
\vedette{phwa}\homonyme{1}
\groupe{A}
\sens{2}
\région{GOs PA BO}
\end{entrée}

\begin{entrée}
{trou d'eau (dans une rivière, dans la mer)}
\vedette{paxa-we}
\région{GOs}
\end{entrée}

\begin{entrée}
{troué}
\classe{v.stat.}
\vedette{phwa}\homonyme{1}
\groupe{B}
\sens{1}
\région{GOs PA BO}
\end{entrée}

\section{Techniques}

\subsection{Habitat}

\subsubsection{Habitat}

\begin{entrée}
{abri dans/sous un rocher}
\vedette{mwa-paa}
\région{PA}
\end{entrée}

\begin{entrée}
{barrière ; clôture}
\vedette{thîni}
\région{GOs PA BO}
\end{entrée}

\begin{entrée}
{cimetière}
\vedette{kaço}
\région{GOs}
\région{WEM WE PA BO}
\variante{%
\vedette{kayòl}}
\end{entrée}

\begin{entrée}
{déménager}
\vedette{tuu}
\région{GOs}
\end{entrée}

\begin{entrée}
{demeure}
\vedette{mhenõ-yu}
\région{GOs}
\variante{%
\vedette{menõ-yuu}
\région{GOs}}
\end{entrée}

\begin{entrée}
{demeurer}
\classe{v}
\vedette{yuu}
\sens{1}
\région{GOs}
\variante{%
\vedette{yu, yuu}
\région{BO PA}}
\end{entrée}

\begin{entrée}
{enclos}
\vedette{thîni}
\région{GOs PA BO}
\end{entrée}

\begin{entrée}
{habiter (littéraire)[BM]}
\vedette{cura}
\région{BO}
\end{entrée}

\begin{entrée}
{lieu de résidence}
\vedette{mhenõ-yu}
\région{GOs}
\variante{%
\vedette{menõ-yuu}
\région{GOs}}
\end{entrée}

\begin{entrée}
{palissade de la chefferie}
\vedette{thîni-a kavegu}
\région{GOs}
\end{entrée}

\begin{entrée}
{partir avec toutes ses affaires}
\vedette{tuu}
\région{GOs}
\end{entrée}

\begin{entrée}
{pays}
\classe{n.LOC (forme POSS de pwamwa)}
\vedette{pomõ}
\sens{2}
\région{GOs PA BO}
\variante{%
\vedette{pwòmò}
\région{WE PA}}
\end{entrée}

\begin{entrée}
{pays ; tribu ; contrée}
\vedette{pwamwa}
\sens{1}
\région{GOs PA BO}
\variante{%
\vedette{pwamwò-n}
\région{PA}}
\variante{%
\vedette{phwamwa}
\région{BO}}
\end{entrée}

\begin{entrée}
{place du village}
\classe{nom}
\vedette{kavwègu}
\sens{2}
\région{GOs PA BO}
\variante{%
\vedette{kapègu}
\région{GO(s) vx}}
\end{entrée}

\begin{entrée}
{résider}
\classe{v}
\vedette{yuu}
\sens{1}
\région{GOs}
\variante{%
\vedette{yu, yuu}
\région{BO PA}}
\end{entrée}

\begin{entrée}
{rester}
\classe{v}
\vedette{yuu}
\sens{1}
\région{GOs}
\variante{%
\vedette{yu, yuu}
\région{BO PA}}
\end{entrée}

\begin{entrée}
{séjour}
\vedette{mhenõ-yu}
\région{GOs}
\variante{%
\vedette{menõ-yuu}
\région{GOs}}
\end{entrée}

\begin{entrée}
{village ; ensemble de maisons}
\classe{nom}
\vedette{avwònò}
\sens{2}
\région{GOs}
\variante{%
\vedette{avwònò}
\région{PA BO}}
\variante{%
\vedette{apono}
\région{vx}}
\end{entrée}

\begin{entrée}
{village ; ensemble de maisons}
\classe{nom}
\vedette{kavwègu}
\sens{2}
\région{GOs PA BO}
\variante{%
\vedette{kapègu}
\région{GO(s) vx}}
\end{entrée}

\subsubsection{Types de maison, architecture de la maison}

\begin{entrée}
{abri (dans les champs comportant une plate-forme sur laquelle on entrepose les récoltes)}
\vedette{mwa-vèle}
\région{GOs PA}
\end{entrée}

\begin{entrée}
{abri de fortune [Dubois, BM]}
\vedette{mwa-gol}
\région{BO}
\variante{%
\vedette{mwa-ol}
\région{BO [BM]}}
\end{entrée}

\begin{entrée}
{alène}
\vedette{kili}
\région{GOs BO PA}
\end{entrée}

\begin{entrée}
{arrière de la maison}
\classe{n.LOC}
\vedette{kaça mwa}
\sens{1}
\région{GOs}
\end{entrée}

\begin{entrée}
{arrière de la maison}
\classe{nom}
\vedette{kaya-mwa}
\sens{1}
\région{BO}
\end{entrée}

\begin{entrée}
{arrière de la maison}
\vedette{pu-mwa}
\région{GOs}
\end{entrée}

\begin{entrée}
{baguette}
\vedette{horayee}
\région{PA}
\région{PA BO}
\variante{%
\vedette{orayee}}
\end{entrée}

\begin{entrée}
{bord inférieur de la toiture}
\vedette{dròò-du}
\région{GOs}
\end{entrée}

\begin{entrée}
{bord inférieur de la toiture (dépasse de la sablière)}
\vedette{doori}
\région{PA BO}
\end{entrée}

\begin{entrée}
{bord inférieur de la toiture (qui dépasse de la sablière des maisons carrées)}
\vedette{droo-mwa}
\région{GOs}
\variante{%
\vedette{doori}
\région{PA BO}}
\end{entrée}

\begin{entrée}
{chambranles}
\vedette{drògò}
\région{GOs WEM}
\variante{%
\vedette{dògò}
\région{PA}}
\end{entrée}

\begin{entrée}
{chambranles sculptées de porte}
\vedette{thalei}
\région{WEM PA BO}
\end{entrée}

\begin{entrée}
{charpente (maison)}
\vedette{ce-tha}
\région{GOs WEM}
\end{entrée}

\begin{entrée}
{chevrons ; solives}
\vedette{ce-mwa}
\région{PA BO}
\end{entrée}

\begin{entrée}
{claie sur laquelle on fumait la nourriture}
\vedette{phalawe}\homonyme{2}
\région{PA}
\end{entrée}

\begin{entrée}
{coin externe de la maison}
\vedette{kudi-mwa}
\région{GOs PA}
\end{entrée}

\begin{entrée}
{conque (de la flèche faîtière)}
\classe{nom}
\vedette{kòlaao}
\sens{2}
\région{GOs BO}
\variante{%
\vedette{kòlaao}
\région{PA}}
\variante{%
\vedette{kòlao; kòlaho}
\région{BO}}
\end{entrée}

\begin{entrée}
{construire (maison)}
\classe{v}
\vedette{khabe}
\sens{2}
\région{GOs PA BO}
\end{entrée}

\begin{entrée}
{construire (un mur) ; faire un mur}
\vedette{bööni}\homonyme{1}
\région{GOs PA BO}
\end{entrée}

\begin{entrée}
{contenant de qqch ; manche (vêtement)}
\vedette{mõ-}
\région{GOs PA BO}
\variante{%
\vedette{mwõ-}
\région{PA}}
\end{entrée}

\begin{entrée}
{corbeille (maison, Charles)}
\classe{nom}
\vedette{kevalu}
\sens{2}
\région{GOs BO}
\end{entrée}

\begin{entrée}
{couper le bois qui sert de gaulettes (retenant la couverture du toit, i.e. les écorces de niaouli et la paille)}
\vedette{orèyi}
\sens{2}
\région{GO WEM}
\end{entrée}

\begin{entrée}
{couvrir de paille racines vers l'extérieur}
\vedette{jego}
\région{PA}
\end{entrée}

\begin{entrée}
{couvrir de paille racines vers l'extérieur}
\vedette{yaa-bweevwu}
\région{GOs}
\end{entrée}

\begin{entrée}
{couvrir de paille racines vers l'intérieur}
\vedette{yaa-do mae}
\région{GOs}
\end{entrée}

\begin{entrée}
{couvrir (maison)}
\classe{v}
\vedette{thaabwe}
\sens{1}
\région{GOs BO}
\variante{%
\vedette{thaaboi, thabui, thabwi}
\région{GO(s) WEM BO}}
\end{entrée}

\begin{entrée}
{couvrir une maison}
\vedette{ya-mwa}
\région{GOs PA BO}
\end{entrée}

\begin{entrée}
{couvrir (une maison) ; couverture (en général)}
\vedette{thròlò}
\région{GOs}
\variante{%
\vedette{thòlò}
\région{PA}}
\end{entrée}

\begin{entrée}
{couvrir (un toit, originellement avec de la paille)}
\vedette{yaa}
\région{GOs}
\région{WEM WE PA}
\variante{%
\vedette{yaal, yaale}}
\variante{%
\vedette{yaali}
\région{BO}}
\end{entrée}

\begin{entrée}
{couvrir (un toit, originellement avec de la paille)}
\vedette{yaaze}
\région{GOs}
\région{WEM WE PA}
\variante{%
\vedette{yaal, yaale}}
\variante{%
\vedette{yaali}
\région{BO}}
\end{entrée}

\begin{entrée}
{crépir ; faire un mur en torchis}
\vedette{dilee}
\région{PA}
\end{entrée}

\begin{entrée}
{cuisine}
\vedette{mõ-puyòl}
\région{BO}
\variante{%
\vedette{mõ-wuyòl}
\région{BO}}
\end{entrée}

\begin{entrée}
{cuisine (lieu)}
\vedette{mwa-puçò}
\région{GOs}
\end{entrée}

\begin{entrée}
{décorations sur la flèche faitière de la maison 'throo-mwa' (conques etc)}
\vedette{throloo}
\région{WEM}
\end{entrée}

\begin{entrée}
{dépotoir}
\vedette{mhenõ-kole ja}
\région{GOs}
\variante{%
\vedette{mhenõô-kole ja}
\région{PA}}
\end{entrée}

\begin{entrée}
{devant (le) de la maison}
\vedette{me-mwa}
\région{GOs}
\variante{%
\vedette{mee-mwa}
\région{PA BO}}
\end{entrée}

\begin{entrée}
{douche (lieu)}
\vedette{mõ-butro}
\région{GOs}
\end{entrée}

\begin{entrée}
{étagère ; claie pour fumer}
\vedette{mhenõ na pwaawe}
\région{GOs}
\end{entrée}

\begin{entrée}
{étagère, claie sur laquelle on suspendait les paniers dans la maison}
\vedette{phalawe}\homonyme{2}
\région{PA}
\end{entrée}

\begin{entrée}
{faitage sculpté (Dubois)}
\vedette{thròlò}
\région{GOs}
\variante{%
\vedette{thòlò}
\région{PA}}
\end{entrée}

\begin{entrée}
{fenêtre}
\vedette{phwa ni gòò-mwa}
\région{GOs WEM WEH}
\variante{%
\vedette{phwe-kuracee}
\région{PA}}
\end{entrée}

\begin{entrée}
{fenêtre}
\vedette{phwe-kurace}
\région{PA BO}
\end{entrée}

\begin{entrée}
{flèche faîtière}
\vedette{throo-mwa}
\région{GOs WEM}
\variante{%
\vedette{thoo-mwa}
\région{PA BO}}
\end{entrée}

\begin{entrée}
{fond (de la maison)}
\vedette{puxu-n}
\région{PA}
\variante{%
\vedette{puvwu-n}
\région{PA}}
\end{entrée}

\begin{entrée}
{fond de la maison ronde (aussi la base du 'ning')}
\vedette{pu-ni}
\région{GOs}
\variante{%
\vedette{pu-ning}
\région{PA}}
\end{entrée}

\begin{entrée}
{gaulette formant la corbeille}
\vedette{pe}\homonyme{3}
\région{PA BO}
\end{entrée}

\begin{entrée}
{gaulettes circulaires du toit}
\vedette{horayee}
\région{PA}
\région{PA BO}
\variante{%
\vedette{orayee}}
\end{entrée}

\begin{entrée}
{gaulettes circulaires du toit}
\vedette{orèyi}
\sens{1}
\région{GO WEM}
\end{entrée}

\begin{entrée}
{gaulettes horizontales}
\vedette{moko}
\région{BO}
\end{entrée}

\begin{entrée}
{gaulettes (qui retiennent la couverture du toit faite d'écorce de niaouli et de paille)}
\vedette{zabò mwa}
\région{GOs}
\end{entrée}

\begin{entrée}
{gaulettes qui retiennent la couverture du toit (faite d'écorce de niaouli et de paille)}
\classe{nom}
\vedette{zabò}
\sens{2}
\région{GOs BO PAWEM}
\variante{%
\vedette{zhabò}
\région{GA}}
\end{entrée}

\begin{entrée}
{gaulettes qui tiennent la paille (posées sur la paille et qui sont attachées ensuite par un lien)}
\vedette{haza ce}
\région{GOs}
\end{entrée}

\begin{entrée}
{gaulettes servant d'appui aux solives}
\vedette{ńodo}
\région{WEM BO PA}
\end{entrée}

\begin{entrée}
{gaulettes verticales (pointe des)}
\vedette{me-de}
\région{GO BO}
\variante{%
\vedette{mide}
\région{BO}}
\end{entrée}

\begin{entrée}
{lianes attachent les bois sur la maison}
\vedette{dre}\homonyme{1}
\région{GOs WEM}
\end{entrée}

\begin{entrée}
{magasin ; boutique}
\vedette{mõ-iyu}
\région{BO}
\end{entrée}

\begin{entrée}
{maison}
\vedette{mwa}
\région{GOs}
\end{entrée}

\begin{entrée}
{maison à toit à deux pentes}
\vedette{mõ-wae}
\région{WEM BO}
\end{entrée}

\begin{entrée}
{maison à toit plat ; maison à toit à deux pentes}
\vedette{mwa-araba}
\région{GOs WEM}
\variante{%
\vedette{mwa-alaba (mwa-halapa Corne)}
\région{BO}}
\end{entrée}

\begin{entrée}
{maison carrée}
\vedette{mwa-pe-cinô}
\région{BO}
\end{entrée}

\begin{entrée}
{maison ; demeure}
\classe{nom}
\vedette{avwònò}
\sens{1}
\région{GOs}
\variante{%
\vedette{avwònò}
\région{PA BO}}
\variante{%
\vedette{apono}
\région{vx}}
\end{entrée}

\begin{entrée}
{maison des femmes}
\vedette{phwamwã-roomwã}
\région{PA}
\end{entrée}

\begin{entrée}
{maison des hommes (grande maison servant de lieu de réunion)}
\vedette{mwa-phwamwêêgu}
\région{GO BO PA}
\variante{%
\vedette{mwa-phwamwãgu}
\région{PA}}
\end{entrée}

\begin{entrée}
{maison ; maison (grande chefferie )}
\vedette{mõ-}
\région{GOs PA BO}
\variante{%
\vedette{mwõ-}
\région{PA}}
\end{entrée}

\begin{entrée}
{maison où dorment femmes et enfants (lit. maison du tressage)}
\vedette{mwa-pho}
\région{GOs WEM}
\variante{%
\vedette{mwa-wo}
\région{GO(s) WEM}}
\end{entrée}

\begin{entrée}
{maison où l'on garde la nourriture}
\vedette{mwa-huvo}
\région{GOs}
\end{entrée}

\begin{entrée}
{maison où se retirent les femmes (pendant les règles)}
\vedette{mwa-pwayu}
\région{GOs BO}
\variante{%
\vedette{mwa-pwaeu}
\région{BO}}
\end{entrée}

\begin{entrée}
{maison où se retirent les femmes (pendant les règles) (lit. où l'on se cache)}
\vedette{mõ-caaxò}
\région{GO}
\variante{%
\vedette{mõ-caao}
\région{GO}}
\end{entrée}

\begin{entrée}
{maison où se retirent les femmes (pendant les règles) (lit. où l'on se cache)}
\vedette{mõ-phwayuu}
\région{GOs}
\end{entrée}

\begin{entrée}
{maison ronde}
\vedette{gu-mwa}
\région{GOs PA BO}
\end{entrée}

\begin{entrée}
{masque ; chambranles sculptés}
\vedette{pu-drõgo}
\région{GOs}
\variante{%
\vedette{pu-dõgo}
\région{PA}}
\end{entrée}

\begin{entrée}
{masque (comprenant l'habit qui accompagne le masque)}
\vedette{drògò}
\région{GOs WEM}
\variante{%
\vedette{dògò}
\région{PA}}
\end{entrée}

\begin{entrée}
{mât (bateau)}
\vedette{nixò}
\région{GOs}
\variante{%
\vedette{nixòòl}
\région{PA BO WEM}}
\variante{%
\vedette{nigòòl}
\région{BO}}
\end{entrée}

\begin{entrée}
{mettre la première rangée de paille au bord du toit}
\vedette{thu phu}
\région{GOs}
\end{entrée}

\begin{entrée}
{mur}
\vedette{gòò-mwa}
\région{GOs}
\variante{%
\vedette{gò-mwê}
\région{GO}}
\end{entrée}

\begin{entrée}
{mur de la maison (tous les murs)}
\vedette{puu-mwa}
\région{GOs PA BO}
\variante{%
\vedette{pu-mwa}
\région{PA BO}}
\end{entrée}

\begin{entrée}
{paille à toiture}
\classe{nom}
\vedette{ja}\homonyme{2}
\sens{2}
\région{BO [Corne]}
\variante{%
\vedette{jan}
\région{BO [BM]}}
\end{entrée}

\begin{entrée}
{paille (dernière rangée de paille sur le toit, forme un bourrelet qui ferme le faîtage)}
\vedette{bwè-mwa}
\région{GOs PA BO}
\end{entrée}

\begin{entrée}
{paille du faîtage (formant un bourrelet)}
\vedette{phidru}
\région{GOs}
\variante{%
\vedette{phiju}
\région{BO}}
\end{entrée}

\begin{entrée}
{panne sablière (supporte la toiture des maisons carrées ou rondes)}
\vedette{pwabwani}
\région{GOs WEM}
\variante{%
\vedette{pwabwaning, pabwaning}
\région{PA BO}}
\end{entrée}

\begin{entrée}
{pierre de seuil}
\classe{nom}
\vedette{thaxee-phweemwa}
\sens{1}
\région{GOs BO}
\variante{%
\vedette{taage, thaaxe}
\région{BO}}
\end{entrée}

\begin{entrée}
{pierre de seuil [BO]}
\classe{nom}
\vedette{thaxe}
\sens{2}
\région{GOs BO}
\end{entrée}

\begin{entrée}
{plateforme ; place aménagée devant la case}
\vedette{ba}\homonyme{2}
\région{BO (Corne)}
\end{entrée}

\begin{entrée}
{portail}
\classe{nom}
\vedette{phwè-zini}
\sens{1}
\région{GOs}
\variante{%
\vedette{phwe-thîni}
\région{PA}}
\end{entrée}

\begin{entrée}
{porte de clôture}
\classe{nom}
\vedette{phwè-zini}
\sens{1}
\région{GOs}
\variante{%
\vedette{phwe-thîni}
\région{PA}}
\end{entrée}

\begin{entrée}
{porte (de maison)}
\vedette{pwaxilo}
\région{PA BO}
\variante{%
\vedette{pwagilo}
\région{PA BO}}
\variante{%
\vedette{pwagilo}
\région{vx}}
\end{entrée}

\begin{entrée}
{porte (de maison)}
\classe{nom}
\vedette{phwè-mwa}
\sens{1}
\région{GOs}
\variante{%
\vedette{phwee-mwa}
\région{PA BO}}
\end{entrée}

\begin{entrée}
{poteau central de la case}
\vedette{nixò}
\région{GOs}
\variante{%
\vedette{nixòòl}
\région{PA BO WEM}}
\variante{%
\vedette{nigòòl}
\région{BO}}
\end{entrée}

\begin{entrée}
{poteaux de barrière}
\vedette{ce-thîni}
\région{GOs}
\end{entrée}

\begin{entrée}
{poteaux (petits) de maison}
\vedette{ni}\homonyme{3}
\région{GOs}
\variante{%
\vedette{ning}
\région{WEM PA BO}}
\end{entrée}

\begin{entrée}
{poutre faîtière}
\vedette{maagò}\homonyme{2}
\région{WEM BO PA}
\end{entrée}

\begin{entrée}
{poutre faitîère (rejoint le "pwabwani" au sommet du toit)}
\vedette{ce-tha}
\région{GOs WEM}
\end{entrée}

\begin{entrée}
{poutre maîtresse des maisons carrées}
\vedette{maagò}\homonyme{2}
\région{WEM BO PA}
\end{entrée}

\begin{entrée}
{poutre maîtresse (supporte la toiture des maisons carrées ou rondes)}
\vedette{pwabwani}
\région{GOs WEM}
\variante{%
\vedette{pwabwaning, pabwaning}
\région{PA BO}}
\end{entrée}

\begin{entrée}
{poutre qui soutient la toiture (PA, WEM)}
\vedette{jebo}
\région{PA WEM}
\end{entrée}

\begin{entrée}
{poutre sablière (poutre circulaire supportant la charpente des cases rondes ;Dubois)}
\vedette{uda}
\région{BO}
\end{entrée}

\begin{entrée}
{premier rang de paille au bord du toit ; rebord du toit de chaume}
\classe{nom}
\vedette{phu}\homonyme{2}
\sens{2}
\région{GOs PA BO}
\end{entrée}

\begin{entrée}
{premier rang de paille (dépasse de la sablière)}
\vedette{doori}
\région{PA BO}
\end{entrée}

\begin{entrée}
{rangée de paille sur un toit [Corne]}
\vedette{pwe-yal}
\région{BO}
\end{entrée}

\begin{entrée}
{sculpture faîtière}
\vedette{drògò}
\région{GOs WEM}
\variante{%
\vedette{dògò}
\région{PA}}
\end{entrée}

\begin{entrée}
{seuil de la porte}
\classe{nom}
\vedette{thaxee-phweemwa}
\sens{1}
\région{GOs BO}
\variante{%
\vedette{taage, thaaxe}
\région{BO}}
\end{entrée}

\begin{entrée}
{solives}
\vedette{ce-nuda}
\région{PA BO}
\end{entrée}

\begin{entrée}
{solive verticale}
\vedette{aguko}
\région{BO PA}
\end{entrée}

\begin{entrée}
{sud (le)}
\vedette{me-mwa}
\région{GOs}
\variante{%
\vedette{mee-mwa}
\région{PA BO}}
\end{entrée}

\begin{entrée}
{terrassement de la maison}
\vedette{kêê-mwa}
\région{GOs}
\end{entrée}

\begin{entrée}
{tertre}
\vedette{bwaxeni}
\région{GOs WEMBO}
\end{entrée}

\begin{entrée}
{tertre ; emplacement de la maison (trace d'une maison disparue)}
\vedette{bu-mwa}
\région{GOs BO}
\end{entrée}

\begin{entrée}
{toilettes ; cabinet}
\vedette{mõ-phòò}
\région{GOs}
\end{entrée}

\begin{entrée}
{toit}
\vedette{bwa-mwa}
\région{GOs}
\variante{%
\vedette{bwa-mwa}
\région{PA}}
\end{entrée}

\begin{entrée}
{toit (pente du) vu de l'intérieur}
\vedette{phagoo-mwa}
\région{PA}
\end{entrée}

\begin{entrée}
{toiture en paille ("racines dehors")}
\vedette{de-du}
\région{GOs}
\variante{%
\vedette{degu, dego}
\région{BO [Corne]}}
\end{entrée}

\begin{entrée}
{traverses (charpente)[Corne]}
\vedette{magòòny}
\région{BO}
\end{entrée}

\begin{entrée}
{véranda}
\vedette{varan}
\région{PA}
\end{entrée}

\subsubsection{Objets et meubles de la maison}

\begin{entrée}
{armoire}
\vedette{kee hõbo}
\région{GOs}
\variante{%
\vedette{kee-hãbwo}}
\end{entrée}

\begin{entrée}
{balance}
\classe{nom}
\vedette{baa-ja}
\sens{2}
\région{GOs}
\end{entrée}

\begin{entrée}
{berceau en fibre de cocotier [WEM, PA, BO]}
\classe{nom}
\vedette{wõ}
\sens{2}
\région{GOs}
\variante{%
\vedette{wony}
\région{PA WEM BO}}
\end{entrée}

\begin{entrée}
{berceau (pour bercer un bébé)}
\vedette{kãgoo}
\région{GOs}
\end{entrée}

\begin{entrée}
{chaise ; banc ; chaise ; siège}
\vedette{ba-trabwa}
\région{GOs BO PA}
\variante{%
\vedette{ba-rabwa}
\région{GO(s)}}
\end{entrée}

\begin{entrée}
{chaise [PA] ; lieu où l'on s'assoit [GO]}
\vedette{mhenõ-tabwa}
\région{PA}
\end{entrée}

\begin{entrée}
{lit}
\vedette{baalaba}
\région{PA BO}
\end{entrée}

\begin{entrée}
{lit}
\vedette{vèlè}
\région{GOs PA}
\variante{%
\vedette{baalaba}
\région{PA BO}}
\end{entrée}

\begin{entrée}
{oreiller ; appuie-tête}
\vedette{bwane}
\région{GOs}
\variante{%
\vedette{bwea}
\région{WE PA}}
\variante{%
\vedette{bwani}
\région{BO}}
\end{entrée}

\begin{entrée}
{poubelle}
\vedette{mõ-ja}
\région{GOs}
\end{entrée}

\begin{entrée}
{table}
\vedette{ta}\homonyme{4}
\région{GOs}
\variante{%
\vedette{taam}
\région{WE}}
\variante{%
\vedette{taavw}
\région{PA}}
\end{entrée}

\begin{entrée}
{table}
\vedette{tav}
\région{PA}
\end{entrée}

\begin{entrée}
{tableau}
\vedette{mhenõ-tivwo}
\région{GOs}
\end{entrée}

\subsection{Cultures, plantations, récoltes, végétation}

\subsubsection{Cultures, techniques, boutures}

\begin{entrée}
{arracher la canne à sucre ; récolter}
\vedette{thoo}\homonyme{1}
\région{GOs PA BO}
\end{entrée}

\begin{entrée}
{arracher (la paille)}
\vedette{mwòni}
\région{BO [BM]}
\variante{%
\vedette{mòni}
\région{BO}}
\variante{%
\vedette{mone}
\région{PA}}
\end{entrée}

\begin{entrée}
{arracher les taros d'eau}
\vedette{phu kuru}
\région{PA}
\end{entrée}

\begin{entrée}
{arracher l'herbe (à la main)}
\vedette{paang}
\région{PA BO}
\end{entrée}

\begin{entrée}
{arracher (paille, herbes, lianes)}
\vedette{tali}
\région{GOs PA BO}
\end{entrée}

\begin{entrée}
{arracher (paille, taro d'eau)}
\vedette{phu}\homonyme{4}
\région{GOs PA}
\variante{%
\vedette{phwu}
\région{BO}}
\end{entrée}

\begin{entrée}
{attacher la tige d'igname}
\vedette{nhõî}
\région{GOs PA}
\variante{%
\vedette{nhõî, nhõõî}
\région{BO}}
\end{entrée}

\begin{entrée}
{barrage pour dévier l'eau vers la tarodière [WEM]}
\vedette{ba}\homonyme{3}
\région{WEM PA}
\end{entrée}

\begin{entrée}
{barrage (pour l'irrigation)}
\vedette{bwavwòlo}
\région{GOs}
\variante{%
\vedette{bweevòlo}
\région{PA BO}}
\end{entrée}

\begin{entrée}
{barrage sur une rivière où les femmes lavent le 'dimwa' (Charles)}
\vedette{bwavwòlo}
\région{GOs}
\variante{%
\vedette{bweevòlo}
\région{PA BO}}
\end{entrée}

\begin{entrée}
{barrage ; vanne de canal de tarodière (Dubois)}
\vedette{bwavwu-we}
\région{GOs BO PA}
\région{GOs}
\variante{%
\vedette{bwevwu-we}}
\variante{%
\vedette{pwe-we}
\région{PA}}
\end{entrée}

\begin{entrée}
{billon}
\vedette{bu}\homonyme{2}
\région{GOs PA}
\end{entrée}

\begin{entrée}
{billon ; côté femelle du massif d'ignames}
\classe{n.LOC}
\vedette{havaa-n}
\sens{2}
\région{PA BO}
\variante{%
\vedette{hapa}
\région{PA}}
\end{entrée}

\begin{entrée}
{bouture de canne à sucre}
\vedette{uvo-ê}
\région{GOs PA}
\end{entrée}

\begin{entrée}
{bouture de manioc}
\vedette{kòò-manyô}
\région{GOs}
\end{entrée}

\begin{entrée}
{bouture de patate douce}
\vedette{kô-kumala}
\région{GOs}
\variante{%
\vedette{kô-kumwãla}
\région{GO(s)}}
\end{entrée}

\begin{entrée}
{bouture de taro (pédoncule de taro muni d'une tige)}
\vedette{uvo-uva}
\région{GOs}
\end{entrée}

\begin{entrée}
{bouture d'igname (à partir de l'extrémité inférieure de l'igname)}
\vedette{bwe-kui}
\région{GOs PA}
\end{entrée}

\begin{entrée}
{brûler (les champs) ; pratiquer le brûlis}
\vedette{khi-kha}
\région{GOs}
\variante{%
\vedette{khi-ga}
\région{GOs}}
\variante{%
\vedette{ki kha}
\région{GO(s)}}
\variante{%
\vedette{ki khan}
\région{PA}}
\end{entrée}

\begin{entrée}
{butte de terre de la tarodière}
\vedette{aru}
\région{PA BO}
\end{entrée}

\begin{entrée}
{butte d'igname}
\vedette{bu-kui}
\région{GOs PA}
\end{entrée}

\begin{entrée}
{butter (les tubercules)}
\vedette{buuni}
\région{GOs}
\variante{%
\vedette{buuni}
\région{BO}}
\end{entrée}

\begin{entrée}
{caniveau}
\vedette{phwè-po-xaò}
\région{GOs}
\variante{%
\vedette{phwè-vwo-xaò}
\région{GO(s)}}
\end{entrée}

\begin{entrée}
{coin débroussé pour cultures}
\vedette{kêê-tre}
\région{GOs}
\end{entrée}

\begin{entrée}
{conduite d'eau en bambou (amène l'eau de la rivière à la tarodière) ; canalisation}
\vedette{bii}\homonyme{2}
\région{GOs PA}
\end{entrée}

\begin{entrée}
{conduite d'eau pour les cultures ; aqueduc d'irrigation}
\vedette{dèè-we}
\région{GOs BO PA}
\end{entrée}

\begin{entrée}
{couper et prélever le bas du tubercule d'igname et replanter la partie supérieure avec ses lianes}
\vedette{ńhòme kui}
\région{GOs}
\end{entrée}

\begin{entrée}
{creuser (pour récolter des ignames) ; récolter les ignames}
\vedette{thaa kui}
\région{GOs PA}
\end{entrée}

\begin{entrée}
{cueillir la canne à sucre}
\vedette{tha ê}
\région{GOs}
\variante{%
\vedette{tho êm}
\région{PA}}
\end{entrée}

\begin{entrée}
{cultiver ; faire un champ}
\vedette{thu-phwãã}
\région{GOs PA BO}
\end{entrée}

\begin{entrée}
{débroussailler (champ, chemin au sabre d'abatis)}
\vedette{thèl}
\région{PA BO}
\end{entrée}

\begin{entrée}
{défriché ; dégagé}
\classe{v}
\vedette{phwaa}
\sens{3}
\région{GOs}
\région{PA BO}
\variante{%
\vedette{phwaal}}
\end{entrée}

\begin{entrée}
{défricher}
\vedette{thèl}
\région{PA BO}
\end{entrée}

\begin{entrée}
{désherber ; couper l'herbe}
\vedette{phaawa}
\région{GOs}
\end{entrée}

\begin{entrée}
{désherber ; couper l'herbe ; débrousser}
\vedette{phaavwi}
\région{GOs}
\end{entrée}

\begin{entrée}
{désherber ; faucher ;désherbage}
\vedette{paang}
\région{PA BO}
\end{entrée}

\begin{entrée}
{déterrer les tubercules (ignames, taro de montagne)}
\classe{v}
\vedette{taa}
\sens{2}
\région{GOs PA BO}
\end{entrée}

\begin{entrée}
{émotter}
\classe{v}
\vedette{nhyale}
\sens{2}
\région{GOs BO PA}
\end{entrée}

\begin{entrée}
{extrémité du champ (lit. tête du champ)}
\vedette{bwe-phwamwa}
\région{GOs}
\end{entrée}

\begin{entrée}
{faire un trou (pour les cultures, etc.)}
\vedette{thu phwa}
\région{GOs PA BO}
\end{entrée}

\begin{entrée}
{faucher}
\vedette{phaawa}
\région{GOs}
\end{entrée}

\begin{entrée}
{fossé d'écoulement entre les massifs de culture (sur le bord du massif d'igname)[Corne]}
\vedette{pwang}\homonyme{1}
\région{BO}
\end{entrée}

\begin{entrée}
{fossé d'écoulement (sur le bord du champ d'igname) [PA]}
\vedette{dèè-we}
\région{GOs BO PA}
\end{entrée}

\begin{entrée}
{glâner de la canne à sucre}
\vedette{maxuã}
\région{PA}
\end{entrée}

\begin{entrée}
{glâner (des ignames, bananes, taros dans des champs laissés en jachère ou à l'abandon) (repousse spontanée des plants)}
\vedette{zagaò}
\région{GOs}
\variante{%
\vedette{zagaòl}
\région{PA}}
\end{entrée}

\begin{entrée}
{graine ; semence}
\vedette{warô}
\région{BO PA}
\end{entrée}

\begin{entrée}
{labourer}
\vedette{khò}\homonyme{1}
\région{PA}
\end{entrée}

\begin{entrée}
{labourer avec une pelle à fouir}
\classe{v ; n}
\vedette{zaro}
\sens{3}
\région{GOs PA}
\variante{%
\vedette{zharo}
\région{GA}}
\variante{%
\vedette{zaatro}
\région{vx (Haudricourt)}}
\variante{%
\vedette{yaro}
\région{BO}}
\end{entrée}

\begin{entrée}
{labourer le champ d'igname du chef}
\vedette{ba-thu-khia}
\région{PA BO}
\end{entrée}

\begin{entrée}
{lier ; ligoter ; attacher}
\vedette{nhõî}
\région{GOs PA}
\variante{%
\vedette{nhõî, nhõõî}
\région{BO}}
\end{entrée}

\begin{entrée}
{maïs}
\vedette{pwawaale}
\région{GOs}
\variante{%
\vedette{pwapale}
\région{GO(s)}}
\end{entrée}

\begin{entrée}
{motte de terre}
\vedette{êgo dili}
\région{GOs}
\end{entrée}

\begin{entrée}
{motte de terre}
\vedette{paxa-dili}
\région{PA BO [Corne]}
\end{entrée}

\begin{entrée}
{motte de terre}
\vedette{pi-dili}
\région{GOs PA}
\variante{%
\vedette{pigo dili}
\région{PA}}
\end{entrée}

\begin{entrée}
{mur de soutènement de la tarodière [PA]}
\vedette{ba}\homonyme{3}
\région{WEM PA}
\end{entrée}

\begin{entrée}
{mur de soutènement d'une cuvette}
\vedette{bwabòzö}
\région{GOs}
\end{entrée}

\begin{entrée}
{ne pas respecter les principes des cultures ou de chasse (ou de la nature en général)}
\vedette{taluang}
\région{PA}
\end{entrée}

\begin{entrée}
{ouverture du champ d'igname du chef}
\vedette{ba-thu-khia}
\région{PA BO}
\end{entrée}

\begin{entrée}
{pente du massif d'ignames (Dubois)}
\vedette{kòlò-khia}
\région{BO}
\end{entrée}

\begin{entrée}
{pied de taro}
\vedette{uvo-uva}
\région{GOs}
\end{entrée}

\begin{entrée}
{pierre constituant un passage/pont}
\vedette{vali}\homonyme{1}
\région{GOs}
\variante{%
\vedette{vali}
\région{PA}}
\end{entrée}

\begin{entrée}
{piocher ; bêcher (champ)}
\vedette{khòòni}
\région{GOs}
\variante{%
\vedette{kòòni}
\région{BO PA}}
\end{entrée}

\begin{entrée}
{plante annuelle}
\classe{nom}
\vedette{ka}\homonyme{2}
\sens{2}
\région{GOs PA BO}
\variante{%
\vedette{kò}
\région{GO(n)}}
\end{entrée}

\begin{entrée}
{planter des palmes de cocotier oudes branches d'autres arbres dans le sol (po)}
\vedette{thii-puu}
\région{GOs PA}
\end{entrée}

\begin{entrée}
{planter des taros au bord de l'eau (rivière, etc.) sans système d'irrigation}
\vedette{thoi haa}
\région{GOs}
\end{entrée}

\begin{entrée}
{planter ; mettre en terre (ignames, taro)}
\vedette{thoè}
\région{GOs}
\variante{%
\vedette{thöe, toe}
\région{BO PA}}
\end{entrée}

\begin{entrée}
{plants}
\vedette{êê-}
\région{GOs PA BO [BM]}
\end{entrée}

\begin{entrée}
{préparer les champs et le trou pour planter les ignames}
\vedette{zaa phwa}
\région{PA WEM}
\end{entrée}

\begin{entrée}
{préparer les champs (ignames) ; débroussailler leschamps (ignames)}
\vedette{thre kha}
\région{GOs}
\variante{%
\vedette{thèl}
\région{BO PA}}
\end{entrée}

\begin{entrée}
{qui donne/produit bien (champ)}
\vedette{pa-zo}
\région{PA}
\variante{%
\vedette{payo}
\région{BO}}
\end{entrée}

\begin{entrée}
{ravager les champs (pour des animaux)}
\vedette{taluang}
\région{PA}
\end{entrée}

\begin{entrée}
{récolte d'ignames}
\classe{nom}
\vedette{ka}\homonyme{2}
\sens{2}
\région{GOs PA BO}
\variante{%
\vedette{kò}
\région{GO(n)}}
\end{entrée}

\begin{entrée}
{récolter les ignames ; époque où l'on récolte les ignames}
\vedette{zagaò}
\région{GOs}
\variante{%
\vedette{zagaòl}
\région{PA}}
\end{entrée}

\begin{entrée}
{récolter (manioc en arrachant)}
\vedette{whai}\homonyme{2}
\région{GOs}
\variante{%
\vedette{wha}
\région{PA}}
\end{entrée}

\begin{entrée}
{rejet (de plante servant à bouturer)}
\vedette{zòò}\homonyme{3}
\région{GOs PA}
\variante{%
\vedette{zhò}
\région{GO(s)}}
\end{entrée}

\begin{entrée}
{replier la tigne d'igname sur elle-même (quand elle dépasse la hauteur du tuteur) [Corne]}
\vedette{taagi kui}
\région{BO}
\end{entrée}

\begin{entrée}
{réserver des tubercules (pour les replanter)}
\vedette{zagawe}
\région{GOs}
\variante{%
\vedette{zhagawe}
\région{GA}}
\end{entrée}

\begin{entrée}
{retourner (la terre)}
\classe{v}
\vedette{zali}
\sens{3}
\région{GOs}
\variante{%
\vedette{zhali}
\région{GA}}
\end{entrée}

\begin{entrée}
{source (Charles)}
\vedette{bwavwu-we}
\région{GOs BO PA}
\région{GOs}
\variante{%
\vedette{bwevwu-we}}
\variante{%
\vedette{pwe-we}
\région{PA}}
\end{entrée}

\begin{entrée}
{talus}
\vedette{bu}\homonyme{2}
\région{GOs PA}
\end{entrée}

\begin{entrée}
{tas}
\vedette{bu}\homonyme{2}
\région{GOs PA}
\end{entrée}

\begin{entrée}
{tertre}
\vedette{bu}\homonyme{2}
\région{GOs PA}
\end{entrée}

\begin{entrée}
{tête de l'igname (qui est replantée)}
\vedette{bwe-kui}
\région{GOs PA}
\end{entrée}

\begin{entrée}
{trou (préparé pour planter l'igname)}
\vedette{phwè-kui}
\région{GOs WEM PA}
\end{entrée}

\begin{entrée}
{tuteur à igname (grand)}
\classe{nom}
\vedette{du}\homonyme{1}
\sens{3}
\région{GOs BO PA}
\end{entrée}

\begin{entrée}
{tuteur d'ignames (petit)}
\vedette{thaa}\homonyme{3}
\région{GOs BO PA}
\end{entrée}

\begin{entrée}
{vanne de canal}
\vedette{mhenõ-paxe we}
\région{PA}
\end{entrée}

\subsubsection{Objets, outils}

\begin{entrée}
{bêche}
\vedette{yaro}
\région{BO PA}
\variante{%
\vedette{zaro}
\région{GO(s)}}
\end{entrée}

\begin{entrée}
{épieu}
\vedette{wo-ce}
\région{GOs}
\end{entrée}

\begin{entrée}
{épieu de culture ; bâton à fouir ; "barre-à-mine"}
\classe{nom}
\vedette{wòzò}
\sens{1}
\région{GOs}
\variante{%
\vedette{wòlò}
\région{PA BO}}
\variante{%
\vedette{wojò}
\région{BO}}
\end{entrée}

\begin{entrée}
{pelle à fouir les ignames (en bois ou fer)}
\vedette{yaro}
\région{BO PA}
\variante{%
\vedette{zaro}
\région{GO(s)}}
\end{entrée}

\subsubsection{Types de champs}

\begin{entrée}
{bananeraie ; champ de bananiers}
\vedette{kêê-chaamwa}
\région{GOs}
\end{entrée}

\begin{entrée}
{billon (avant le stade du billon cultivé : kêê)}
\vedette{khia}\homonyme{2}
\région{GOs BO PA}
\end{entrée}

\begin{entrée}
{butte de terre au bord du canal de la tarodière, à terre bien remuée (Charles)}
\vedette{peenu}\homonyme{2}
\région{PA BO}
\end{entrée}

\begin{entrée}
{champ}
\classe{n.LOC (forme POSS de pwamwa)}
\vedette{pomõ}
\sens{1}
\région{GOs PA BO}
\variante{%
\vedette{pwòmò}
\région{WE PA}}
\end{entrée}

\begin{entrée}
{champ après brûlis}
\vedette{kêê-yaai}
\région{PA BO}
\end{entrée}

\begin{entrée}
{champ cultivé ; champ labouré}
\vedette{kêê-khò}
\région{GOs PA}
\end{entrée}

\begin{entrée}
{champ ; culture en forêt défrichée}
\vedette{kha}\homonyme{2}
\région{GOs}
\variante{%
\vedette{khan}
\région{PA BO}}
\end{entrée}

\begin{entrée}
{champ de cocotier ; plantation de cocotier}
\vedette{kêê-nu}
\région{GOs PA BO}
\end{entrée}

\begin{entrée}
{champ de cultures sur une pente[Corne]}
\vedette{ira}
\région{BO}
\end{entrée}

\begin{entrée}
{champ de taro débroussé mais non labouré [Dubois]}
\vedette{zagu}
\région{BO}
\end{entrée}

\begin{entrée}
{champ d'ignames}
\vedette{kêê-kui}
\région{GOs PA}
\end{entrée}

\begin{entrée}
{champ d'igname sacré du chef (que l'on défriche)}
\vedette{yaa-wòzò}
\région{GOs}
\variante{%
\vedette{yaa-wòlò}
\région{WEM WE}}
\variante{%
\vedette{ya-wòjò}
\région{BO}}
\end{entrée}

\begin{entrée}
{champ ; emplacement}
\vedette{kê-}
\région{GOs PA}
\variante{%
\vedette{kêê}
\région{GO}}
\end{entrée}

\begin{entrée}
{champ en jachère(déjà récolté, dans lequel poussent des rejets)}
\vedette{kêê-phwãã}
\région{GOs}
\end{entrée}

\begin{entrée}
{champ/massif d'igname du chef}
\vedette{khia}\homonyme{2}
\région{GOs BO PA}
\end{entrée}

\begin{entrée}
{champ ; plantation}
\vedette{kêê}\homonyme{2}
\région{GOs PA}
\end{entrée}

\begin{entrée}
{champ ; plantation}
\vedette{pwamwa}
\sens{2}
\région{GOs PA BO}
\variante{%
\vedette{pwamwò-n}
\région{PA}}
\variante{%
\vedette{phwamwa}
\région{BO}}
\end{entrée}

\begin{entrée}
{champ ; trace de champ abandonné (Dubois)}
\vedette{pobo-poin}
\région{GO WE}
\end{entrée}

\begin{entrée}
{jachère}
\classe{nom}
\vedette{nò-tòn}
\sens{2}
\région{BO [BM]}
\end{entrée}

\begin{entrée}
{jardin (au bord de rivière) ; massif de culture humide (comme les taros)}
\vedette{haavwu}
\région{GOsPA BO [Corne]}
\variante{%
\vedette{haapu}}
\end{entrée}

\begin{entrée}
{jardin de fleurs}
\vedette{kêê-mu-ce}
\région{GOs}
\end{entrée}

\begin{entrée}
{massif calendrier}
\vedette{yaa-wòzò}
\région{GOs}
\variante{%
\vedette{yaa-wòlò}
\région{WEM WE}}
\variante{%
\vedette{ya-wòjò}
\région{BO}}
\end{entrée}

\begin{entrée}
{massif de fleurs}
\vedette{kê-muuc}
\région{PA}
\end{entrée}

\begin{entrée}
{plantation ; champ}
\vedette{pwòmò-n}
\région{WE PA}
\variante{%
\vedette{pòmò-n}
\région{BO PA}}
\end{entrée}

\begin{entrée}
{planter un champ de taro aux abords d'une source, sans conduite d'eau}
\vedette{meyaam}
\région{PA}
\end{entrée}

\begin{entrée}
{tarodière}
\vedette{peenu}\homonyme{2}
\région{PA BO}
\end{entrée}

\begin{entrée}
{tarodière en terrasse et irriguée (de taro d'eau)}
\vedette{vana}
\région{GOs}
\end{entrée}

\begin{entrée}
{tarodière irriguée en terrasse}
\vedette{bwala}
\région{BO PA}
\end{entrée}

\begin{entrée}
{tarodière sèche}
\vedette{meyaam}
\région{PA}
\end{entrée}

\begin{entrée}
{terre alluvionnaire (au bord de rivière et au pied des montagnes)}
\vedette{haavwu}
\région{GOsPA BO [Corne]}
\variante{%
\vedette{haapu}}
\end{entrée}

\subsubsection{Végétation}

\begin{entrée}
{broussailles ; maquis ; brousse}
\classe{nom}
\vedette{nò-tòn}
\sens{1}
\région{BO [BM]}
\end{entrée}

\begin{entrée}
{cimetière (dans la forêt)}
\vedette{kò}\homonyme{2}
\région{GOs PA BO}
\end{entrée}

\begin{entrée}
{forêt ; brousse}
\vedette{kò}\homonyme{2}
\région{GOs PA BO}
\end{entrée}

\begin{entrée}
{forêt ; brousse ; maquis}
\vedette{nõ-kò}
\région{GOs}
\variante{%
\vedette{nõ-ko, nõ-xo}
\région{PA}}
\end{entrée}

\begin{entrée}
{forêt impraticable ; fourré}
\classe{nom}
\vedette{tigi}\homonyme{3}
\sens{1}
\région{GOs BO PA}
\end{entrée}

\subsection{Chasse guerre}

\subsubsection{Armes}

\begin{entrée}
{arc}
\vedette{jitrua}
\région{GOs}
\end{entrée}

\begin{entrée}
{arc}
\vedette{ô-jitua}
\région{GOs BO [Corne]}
\end{entrée}

\begin{entrée}
{armes de guerre (lance, sagaie)}
\vedette{thuada}\homonyme{2}
\région{GOs WEM BO}
\end{entrée}

\begin{entrée}
{barbeau de la sagaie}
\vedette{thiza}
\région{GOs}
\end{entrée}

\begin{entrée}
{bâton pour lancer}
\classe{v ; n}
\vedette{henim}
\sens{2}
\région{PA BO [BM]}
\end{entrée}

\begin{entrée}
{cartouche (de fusil)}
\vedette{hê-jige}
\région{GOs}
\end{entrée}

\begin{entrée}
{casse-tête à bout dentelé (aussi une espèce d'arbre dont on utilisait une partie du tronc et le début des racines coupées en pointe. Dubois)}
\vedette{ninigin}
\région{BO}
\end{entrée}

\begin{entrée}
{casse-tête à bout en bec d'oiseau}
\vedette{bwè-po}
\région{GO}
\end{entrée}

\begin{entrée}
{casse-tête à bout phallique (aussi une espèce d'arbre)}
\vedette{mero}
\région{BO}
\end{entrée}

\begin{entrée}
{casse-tête à bout rond}
\vedette{tewai}
\région{BO PA}
\end{entrée}

\begin{entrée}
{casse-tête (générique)}
\vedette{bulaivi}
\région{GOs PA BO}
\end{entrée}

\begin{entrée}
{corde de l'arc}
\vedette{khô-jitrua}
\région{GOs}
\end{entrée}

\begin{entrée}
{coup de fusil}
\vedette{thixa jige}
\région{GOs}
\end{entrée}

\begin{entrée}
{doigtier de la sagaie (propulseur) ; lance sagaie}
\vedette{hõge}
\end{entrée}

\begin{entrée}
{flèche}
\vedette{do-jitrua}
\région{GOs}
\end{entrée}

\begin{entrée}
{flèche ; pointe de la flèche}
\vedette{me-jitrua}
\end{entrée}

\begin{entrée}
{fronde}
\vedette{wèda}
\région{GOs}
\end{entrée}

\begin{entrée}
{fusil de chasse}
\vedette{jige}
\région{GOs}
\end{entrée}

\begin{entrée}
{giberne de fronde}
\vedette{ke-paa}
\région{GOs BO}
\variante{%
\vedette{ke-paa}
\région{PA}}
\end{entrée}

\begin{entrée}
{hache}
\classe{nom}
\vedette{wamòn}
\sens{1}
\région{PA WEH BO}
\variante{%
\vedette{wamwa, wamò}
\région{GO(s)}}
\end{entrée}

\begin{entrée}
{hache}
\classe{nom}
\vedette{wamwa}
\sens{1}
\région{GOs WEM}
\variante{%
\vedette{wamòn}
\région{PA}}
\end{entrée}

\begin{entrée}
{hache (petite, en fer)}
\vedette{tröxi}
\région{GOsWE}
\variante{%
\vedette{tööxi}
\région{PA Paita}}
\end{entrée}

\begin{entrée}
{lancer (sagaie, etc.)}
\classe{v}
\vedette{tòè}
\sens{1}
\région{GOs PA BO}
\end{entrée}

\begin{entrée}
{ligature de la sagaie}
\vedette{wa-do}
\région{GOs PABO}
\end{entrée}

\begin{entrée}
{manche de la sagaie}
\vedette{mõ-do}
\région{BO}
\end{entrée}

\begin{entrée}
{manche de sagaie}
\vedette{kòò-dö}
\end{entrée}

\begin{entrée}
{mèche (du fouet)}
\classe{nom}
\vedette{thila}
\sens{1}
\région{GOs PA BO [BM, Corne]}
\variante{%
\vedette{thira}
\région{BO [Corne]}}
\end{entrée}

\begin{entrée}
{pierre de fronde allongée et polie aux deux bouts}
\vedette{thau}
\région{GOs}
\end{entrée}

\begin{entrée}
{pierre de fronde (générique)}
\vedette{ôdri}
\région{GOs}
\end{entrée}

\begin{entrée}
{pierre de fronde (grosse)}
\vedette{thãne}
\région{PA}
\end{entrée}

\begin{entrée}
{pierre de fronde (petite, noire, dure)}
\vedette{piyu}
\région{PA BO}
\end{entrée}

\begin{entrée}
{pierre noire (qui sert à faire des pierres de fronde) [Corne]}
\vedette{tãnim}
\région{BO}
\end{entrée}

\begin{entrée}
{plumet de fronde}
\classe{nom}
\vedette{thila}
\sens{1}
\région{GOs PA BO [BM, Corne]}
\variante{%
\vedette{thira}
\région{BO [Corne]}}
\end{entrée}

\begin{entrée}
{plumet de fronde (en fibre d'aloès) [Corne]}
\vedette{cetil}
\région{BO}
\end{entrée}

\begin{entrée}
{sagaie}
\classe{nom}
\vedette{do}\homonyme{1}
\sens{1}
\région{GOs PA BO}
\end{entrée}

\begin{entrée}
{sagaie à 3 pointes (trident) (lit. sagaie fourchette)}
\vedette{do-de}
\end{entrée}

\begin{entrée}
{sagaie (grande) de guerre [BO]}
\vedette{thuada}\homonyme{2}
\région{GOs WEM BO}
\end{entrée}

\begin{entrée}
{sarbacane}
\vedette{kebwa}\homonyme{2}
\région{GOs}
\end{entrée}

\begin{entrée}
{sparterie de sagaie}
\vedette{zamadra}
\région{GOs}
\end{entrée}

\begin{entrée}
{tamioc}
\classe{nom}
\vedette{wamòn}
\sens{1}
\région{PA WEH BO}
\variante{%
\vedette{wamwa, wamò}
\région{GO(s)}}
\end{entrée}

\begin{entrée}
{tamioc}
\classe{nom}
\vedette{wamwa}
\sens{1}
\région{GOs WEM}
\variante{%
\vedette{wamòn}
\région{PA}}
\end{entrée}

\begin{entrée}
{tamioc ; fer}
\vedette{tröxi}
\région{GOsWE}
\variante{%
\vedette{tööxi}
\région{PA Paita}}
\end{entrée}

\begin{entrée}
{tirer à l'arc}
\vedette{khai}\homonyme{1}
\région{GOs PA}
\end{entrée}

\begin{entrée}
{tirer (au fusil)}
\classe{v}
\vedette{pha}\homonyme{1}
\sens{2}
\région{GOs BO}
\end{entrée}

\subsubsection{Chasse}

\begin{entrée}
{aller à la chasse (cerf)}
\classe{v}
\vedette{a-zaala}
\sens{2}
\région{GOs}
\end{entrée}

\begin{entrée}
{armature de piège [BO Corne]}
\classe{nom}
\vedette{haaxa}
\sens{2}
\région{BO}
\end{entrée}

\begin{entrée}
{bâton à glu (enduit de colle de fruit du gommier, utilisé pour attraper les cigales en leur collant les ailes)}
\classe{v ; n}
\vedette{tigi}\homonyme{2}
\sens{2}
\région{GOs PA BO}
\end{entrée}

\begin{entrée}
{bâton pour attraper les cigales (avec la glu du gommier)}
\vedette{ce ba-thi halelewa}
\région{GOs}
\end{entrée}

\begin{entrée}
{chasser ; écarter (chiens, volailles)}
\vedette{thravwi}
\région{GOs}
\variante{%
\vedette{thawi, tawi}
\région{PA BO}}
\end{entrée}

\begin{entrée}
{chasser la roussette}
\vedette{baa-bo}
\région{GOs}
\end{entrée}

\begin{entrée}
{collet}
\vedette{nhiiji}
\région{BO PA}
\variante{%
\vedette{nhiije}
\région{PA BO}}
\end{entrée}

\begin{entrée}
{lacet}
\vedette{woo}
\région{GOs}
\end{entrée}

\begin{entrée}
{lacet (chasse)}
\vedette{zoa}
\région{GOs}
\end{entrée}

\begin{entrée}
{lacet (pour prendre les oiseaux dans les arbres)}
\vedette{nhiiji}
\région{BO PA}
\variante{%
\vedette{nhiije}
\région{PA BO}}
\end{entrée}

\begin{entrée}
{piège}
\vedette{woo}
\région{GOs}
\end{entrée}

\begin{entrée}
{piège à oiseau ; collet}
\vedette{nhiiji}
\région{BO PA}
\variante{%
\vedette{nhiije}
\région{PA BO}}
\end{entrée}

\begin{entrée}
{piège (à oiseau, rat) ; lacet}
\vedette{je}\homonyme{1}
\région{GOs PA BO}
\variante{%
\vedette{ji}
\région{BO}}
\end{entrée}

\begin{entrée}
{piquer (à la sagaie)}
\vedette{taaja}
\région{GOs BO}
\end{entrée}

\begin{entrée}
{planter (sagaie) ; frapper}
\vedette{thele}
\région{GOs}
\end{entrée}

\begin{entrée}
{plateforme posée dans les arbres situés sur le passage des roussettes}
\vedette{pe}\homonyme{2}
\région{PA}
\end{entrée}

\begin{entrée}
{poursuivre (à la chasse)}
\vedette{thravwi}
\région{GOs}
\variante{%
\vedette{thawi, tawi}
\région{PA BO}}
\end{entrée}

\begin{entrée}
{poursuivre ; courir derrière}
\vedette{thrêê-kai}
\région{GOs}
\end{entrée}

\begin{entrée}
{prise (générique: à la pêche ou la chasse ; lit. ce qu'on ramène de son panier, mais lexicalisé pour toute prise)}
\vedette{hê-kee}
\région{GOs}
\end{entrée}

\begin{entrée}
{toucher (avec une balle de fusil)}
\vedette{thile}
\région{PA}
\end{entrée}

\begin{entrée}
{toucher (cible avec sagaie)}
\classe{v}
\vedette{ca}\homonyme{2}
\sens{1}
\région{GOs PA BO}
\end{entrée}

\begin{entrée}
{viser ; pointer}
\vedette{kine}\homonyme{2}
\région{GOs PA}
\variante{%
\vedette{khine}
\région{GO}}
\end{entrée}

\subsubsection{Guerre}

\begin{entrée}
{assaillir (pour obtenir qqch) [GOs]}
\vedette{whaça}
\région{GOs}
\variante{%
\vedette{waaya}
\région{PA}}
\variante{%
\vedette{waiza}
\région{BO}}
\variante{%
\vedette{whayap}
\région{BO}}
\end{entrée}

\begin{entrée}
{bagarer (se) ; battre (se)}
\classe{v}
\vedette{pe-bu}
\sens{1}
\région{GOs WEM}
\end{entrée}

\begin{entrée}
{bagarre ; bagarrer (se) ; affronter (s')}
\vedette{wòvwa}
\région{GOs}
\variante{%
\vedette{wòpa}
\région{vx}}
\variante{%
\vedette{wovha}
\région{PA}}
\variante{%
\vedette{woza}
\région{WEM}}
\end{entrée}

\begin{entrée}
{battre (se)}
\vedette{pe-çabi}
\région{GOs}
\variante{%
\vedette{pe-çabi}
\région{GO(s)}}
\variante{%
\vedette{pe-cabi}
\région{BO [Corne]}}
\end{entrée}

\begin{entrée}
{battre (se) (avec armes)}
\vedette{pe-kubu}
\région{GOs PA}
\end{entrée}

\begin{entrée}
{battre (se) (avec ou sans armes)}
\vedette{pe-wova}
\région{GOs PA}
\variante{%
\vedette{pe-woza}
\région{WEM}}
\end{entrée}

\begin{entrée}
{battre (se) ; bagarre}
\vedette{wèle}
\région{BO [BM, Corne]}
\end{entrée}

\begin{entrée}
{cerner ; encercler}
\vedette{pwede}
\région{GOs}
\end{entrée}

\begin{entrée}
{chef de guerre}
\vedette{a whili paa}
\région{GOs PA}
\end{entrée}

\begin{entrée}
{combattre ; lutter (pour avoir qqch) ;}
\vedette{whaça}
\région{GOs}
\variante{%
\vedette{waaya}
\région{PA}}
\variante{%
\vedette{waiza}
\région{BO}}
\variante{%
\vedette{whayap}
\région{BO}}
\end{entrée}

\begin{entrée}
{embuscade (faire une) ; surveiller la route}
\vedette{ku-hôboe}
\sens{2}
\région{GOs}
\variante{%
\vedette{ku-hôbwo}
\région{GO(s) BO}}
\end{entrée}

\begin{entrée}
{embuscade (guerre) ; embûches [Corne]}
\vedette{thi-paa}
\région{BO}
\variante{%
\vedette{tho-paa}}
\end{entrée}

\begin{entrée}
{ennemi}
\vedette{whaadrangi}
\région{GOs}
\variante{%
\vedette{wadaga}
\région{BO}}
\end{entrée}

\begin{entrée}
{faire la guerre}
\vedette{thu paa}
\région{GOs}
\end{entrée}

\begin{entrée}
{faire la guerre (se)}
\vedette{pe-thu-paa}
\région{GOs}
\end{entrée}

\begin{entrée}
{guerre}
\vedette{paa}\homonyme{1}
\région{GOs}
\variante{%
\vedette{paa}
\région{PA BO}}
\end{entrée}

\begin{entrée}
{guerre ; lutte ;}
\vedette{whaça}
\région{GOs}
\variante{%
\vedette{waaya}
\région{PA}}
\variante{%
\vedette{waiza}
\région{BO}}
\variante{%
\vedette{whayap}
\région{BO}}
\end{entrée}

\begin{entrée}
{message de guerre (transmis de chef en chef pour chercher des alliés)}
\classe{nom}
\vedette{mwathra}
\sens{1}
\région{GOs}
\variante{%
\vedette{mwara}
\région{GO(s)}}
\variante{%
\vedette{mwarang}
\région{BO}}
\end{entrée}

\begin{entrée}
{noeud de guerre (annonce coutumière);}
\classe{nom}
\vedette{mwathra}
\sens{1}
\région{GOs}
\variante{%
\vedette{mwara}
\région{GO(s)}}
\variante{%
\vedette{mwarang}
\région{BO}}
\end{entrée}

\begin{entrée}
{paix (faire la) (lit. lancer la sagaie)}
\vedette{tò do}
\région{GOs PA BO}
\end{entrée}

\begin{entrée}
{parer (un coup)}
\classe{v}
\vedette{pii}\homonyme{2}
\sens{2}
\région{GOs PA BO}
\end{entrée}

\begin{entrée}
{planter (sagaie) ; frapper}
\vedette{thele}
\région{GOs}
\end{entrée}

\begin{entrée}
{pousser un cri de guerre [BM]}
\vedette{ilili}
\région{BO}
\end{entrée}

\begin{entrée}
{réfugier (se)}
\classe{v.i.}
\vedette{ku-çaaxò}
\sens{2}
\région{GOs}
\variante{%
\vedette{ku-caaxò}
\région{PA BO}}
\end{entrée}

\begin{entrée}
{réfugier (se) [Haudr., Corne]}
\vedette{kauda}
\région{BO}
\end{entrée}

\subsection{Pêche}

\begin{entrée}
{aller à la pêche (à la mer) ;}
\vedette{a-kaze}
\région{GOs}
\variante{%
\vedette{a-kale}
\région{PA}}
\end{entrée}

\begin{entrée}
{aller à la pêche (à la mer) ; aller chercher de la nourriture à la mer}
\classe{n; v}
\vedette{kaze}\homonyme{2}
\sens{2}
\région{GOs}
\variante{%
\vedette{kale}
\région{BO PA}}
\end{entrée}

\begin{entrée}
{aller à la pêche (sur le plâtier)}
\classe{v}
\vedette{a-zaala}
\sens{1}
\région{GOs}
\end{entrée}

\begin{entrée}
{aller pêcher à la senne (lit. aller lancer le filet)}
\vedette{a-kha-pwiò}
\région{GOs}
\end{entrée}

\begin{entrée}
{appâts ; amorces (pêche)}
\vedette{nhyã}
\end{entrée}

\begin{entrée}
{attraper (des crevettes avec une épuisette) [BM]}
\classe{v}
\vedette{khee}\homonyme{2}
\région{BO}
\end{entrée}

\begin{entrée}
{barbeau de la sagaie [BO, BM]}
\vedette{chira, chiira}
\région{GOs PA}
\end{entrée}

\begin{entrée}
{barrage à la pêche (faire un) (avec cailloux, pierres, branches)}
\classe{v}
\vedette{oole}
\sens{1}
\région{PA BO}
\end{entrée}

\begin{entrée}
{barrage (sur la rivière pour la pêche)}
\vedette{pwono}
\région{GOs}
\end{entrée}

\begin{entrée}
{brochette}
\vedette{gu}\homonyme{2}
\région{GOs BO}
\end{entrée}

\begin{entrée}
{canne à pêche (lit. pied de ligne)}
\vedette{kòò-pwe}
\région{GOs BO}
\end{entrée}

\begin{entrée}
{chasser les poissons vers le filet [Corne]}
\vedette{mharii}
\région{BO}
\end{entrée}

\begin{entrée}
{déployer ; étendre (filet)}
\classe{v}
\vedette{kule}
\sens{2}
\région{GOs PA}
\variante{%
\vedette{kole, kula}}
\end{entrée}

\begin{entrée}
{encoche de l'hameçon}
\vedette{chira, chiira}
\région{GOs PA}
\end{entrée}

\begin{entrée}
{enfiler (sur une filoche)}
\vedette{thi-gu}
\région{GOs BO}
\variante{%
\vedette{thi-gu(a)}
\région{PA}}
\end{entrée}

\begin{entrée}
{épuisette à crevettes}
\vedette{drewaa}
\région{GOs}
\variante{%
\vedette{dea}
\région{WE BO}}
\variante{%
\vedette{deaang}
\région{PA}}
\end{entrée}

\begin{entrée}
{épuisette à crevettes [GO]}
\classe{nom}
\vedette{keraa}
\sens{2}
\région{GOs}
\variante{%
\vedette{keraò}
\région{PA}}
\end{entrée}

\begin{entrée}
{épuisette ; haveneau ; nasse (à crevette)}
\vedette{deang}
\région{WEM PA BO}
\end{entrée}

\begin{entrée}
{faire du bruit dans l'eau avec les mains pour effrayer les poissons (comme pour le jeu "thathibul")}
\vedette{paophi}
\région{GOs}
\end{entrée}

\begin{entrée}
{ferrer un poisson}
\vedette{gibwa}
\sens{2}
\région{GOs}
\end{entrée}

\begin{entrée}
{fil de la ligne}
\vedette{wa-pwe}
\région{GOs}
\end{entrée}

\begin{entrée}
{filet épervier}
\vedette{pwi-phawe}
\région{GOs PA}
\variante{%
\vedette{pwi-phaò}
\région{GO(s)}}
\end{entrée}

\begin{entrée}
{filet ; senne}
\vedette{pwiò}
\région{GOs PA}
\variante{%
\vedette{puio, puiyo, pwiyo}
\région{BO}}
\end{entrée}

\begin{entrée}
{filoche}
\vedette{gu}\homonyme{2}
\région{GOs BO}
\end{entrée}

\begin{entrée}
{flotteur de filet [BO]}
\classe{nom}
\vedette{thivwaa}
\sens{2}
\région{GOs BO}
\end{entrée}

\begin{entrée}
{flotteur du filet}
\vedette{phaa-puio}
\région{BO [BM]}
\end{entrée}

\begin{entrée}
{fouiller dans un trou (sans y voir) ; plonger le bras dans qqch.}
\classe{v}
\vedette{pale}
\sens{2}
\région{GOs BO PA}
\variante{%
\vedette{palee}
\région{BO}}
\end{entrée}

\begin{entrée}
{hameçon}
\vedette{pò-pwe}
\région{GOs PA BO}
\variante{%
\vedette{pwò-pwe}
\région{GO(s) BO}}
\end{entrée}

\begin{entrée}
{jeter (filet)}
\classe{v}
\vedette{kole}\homonyme{2}
\sens{1}
\région{GOs BO}
\end{entrée}

\begin{entrée}
{lacet}
\vedette{gu}\homonyme{2}
\région{GOs BO}
\end{entrée}

\begin{entrée}
{liane (servant à enfiler)}
\vedette{gu}\homonyme{2}
\région{GOs BO}
\end{entrée}

\begin{entrée}
{ligne (à hameçon)}
\vedette{khô-pwe}
\région{GOs BO}
\end{entrée}

\begin{entrée}
{maille de filet}
\vedette{phwee-mee pwiò}
\région{GOs PA BO}
\end{entrée}

\begin{entrée}
{maille ; faire un filet}
\classe{n ; v}
\vedette{mâge}
\sens{2}
\région{BO}
\variante{%
\vedette{mhâge}}
\end{entrée}

\begin{entrée}
{mettre sur une filoche (poisson)}
\classe{v}
\vedette{kuani}\homonyme{2}
\région{PA}
\end{entrée}

\begin{entrée}
{nasse (à anguilles ou poisson) ; filet}
\classe{nom}
\vedette{kevalu}
\sens{1}
\région{GOs BO}
\end{entrée}

\begin{entrée}
{nasse (en forme de poche pour fouiller les berges)}
\vedette{drewaa}
\région{GOs}
\variante{%
\vedette{dea}
\région{WE BO}}
\variante{%
\vedette{deaang}
\région{PA}}
\end{entrée}

\begin{entrée}
{navette à filet}
\classe{nom}
\vedette{du-bwò}
\sens{2}
\région{GOs PA BO}
\variante{%
\vedette{duu-bò}
\région{BO}}
\end{entrée}

\begin{entrée}
{pêcher à la ligne}
\vedette{tha-pwe}
\région{GOs PA BO}
\end{entrée}

\begin{entrée}
{pêcher à la ligne ; ligne (de pêche)}
\vedette{pwe}\homonyme{1}
\région{GOs BO}
\end{entrée}

\begin{entrée}
{pêcher à la main (sans y voir)}
\classe{v}
\vedette{pale}
\sens{2}
\région{GOs BO PA}
\variante{%
\vedette{palee}
\région{BO}}
\end{entrée}

\begin{entrée}
{pêcher à la mer}
\vedette{a-kaze}
\région{GOs}
\variante{%
\vedette{a-kale}
\région{PA}}
\end{entrée}

\begin{entrée}
{pêcher à la torche}
\classe{v}
\vedette{nûû}
\groupe{A}
\sens{2}
\région{GOs BO PA}
\end{entrée}

\begin{entrée}
{pêcher (aller à la pêche à qqch)}
\vedette{kaza}\homonyme{2}
\région{GOs BO}
\end{entrée}

\begin{entrée}
{pêcher à marée basse ou àmarée montante}
\vedette{hoo}\homonyme{3}
\région{GOs}
\variante{%
\vedette{tau}
\région{BO PA}}
\end{entrée}

\begin{entrée}
{pêcher au filet}
\vedette{kha pwiò}
\région{GOs}
\end{entrée}

\begin{entrée}
{pêcher au poison (lit. empoisonner l'eau)}
\vedette{kia we}
\région{GOs PA BO}
\end{entrée}

\begin{entrée}
{pêcher en rivière (avec un filet ou à la main)}
\vedette{thraabu}
\région{GOs}
\région{PA BO}
\variante{%
\vedette{thaabu}}
\end{entrée}

\begin{entrée}
{piquer (à la sagaie)}
\vedette{taaja}
\région{GOs BO}
\end{entrée}

\begin{entrée}
{piquer à la sagaie}
\vedette{tò-vhaò}
\région{GOs}
\end{entrée}

\begin{entrée}
{piquer (dans un trou avec une sagaie pour chercher des anguilles, poissons)}
\vedette{thòòzò}
\région{GOs}
\variante{%
\vedette{thòròe}
\région{PA}}
\end{entrée}

\begin{entrée}
{planter (sagaie) ; frapper}
\vedette{thele}
\région{GOs}
\end{entrée}

\begin{entrée}
{plomb de la ligne}
\vedette{paxa-pwe}
\région{GOs}
\end{entrée}

\begin{entrée}
{plomb du filet}
\vedette{paxa-pwiò}
\région{GOs}
\end{entrée}

\begin{entrée}
{plonger (pour pêcher) ; faire de la plongée}
\vedette{truu}
\région{GOs}
\variante{%
\vedette{thuu}
\région{PA}}
\variante{%
\vedette{tuu}
\région{BO}}
\end{entrée}

\begin{entrée}
{prise à la ligne}
\vedette{hê-pwe}
\région{GOs}
\end{entrée}

\begin{entrée}
{prise à la nasse}
\vedette{hê-drewaa}
\région{GOs}
\end{entrée}

\begin{entrée}
{prise au filet}
\vedette{hê-pwiò}
\région{GOs}
\end{entrée}

\begin{entrée}
{prise (générique: à la pêche ou la chasse ; lit. ce qu'on ramène de son panier, mais lexicalisé pour toute prise)}
\vedette{hê-kee}
\région{GOs}
\end{entrée}

\begin{entrée}
{ramasser (filet)}
\classe{v}
\vedette{zali}
\sens{2}
\région{GOs}
\variante{%
\vedette{zhali}
\région{GA}}
\end{entrée}

\begin{entrée}
{sagaie de pêche}
\classe{nom}
\vedette{de}\homonyme{1}
\sens{2}
\région{GOs BO PA}
\end{entrée}

\begin{entrée}
{senne (filet que l'on déploie en tirant) ; filetpour cerner}
\vedette{pwii-khai}
\région{GOs BO}
\end{entrée}

\begin{entrée}
{taper dans l'eau pour faire du bruit (et effrayer le poisson afin de le pousser dans le filet)}
\vedette{thi-pholo}
\région{GOs BO}
\end{entrée}

\begin{entrée}
{taper dans l'eau pour faire du bruit et effrayer le poisson et l'amener dans le filet}
\vedette{pholo}
\région{GOs BO}
\end{entrée}

\begin{entrée}
{troubler (l'eau par exemple pour pêcher dans la rivière)}
\vedette{thi-pöloo}
\région{GOs}
\variante{%
\vedette{thuvwuloo}
\région{WEM WEH}}
\end{entrée}

\begin{entrée}
{trou d'eau (dans une rivière, dans la mer)}
\vedette{paxa-we}
\région{GOs}
\end{entrée}

\subsection{Navigation}

\begin{entrée}
{accoster}
\vedette{coxada}
\région{PA}
\end{entrée}

\begin{entrée}
{accoster (bateau)}
\classe{v.i.}
\vedette{côô}
\sens{3}
\région{GOs}
\variante{%
\vedette{cô}
\région{BO (BM]}}
\end{entrée}

\begin{entrée}
{ancre}
\vedette{kînu}
\région{GOs BO}
\end{entrée}

\begin{entrée}
{ancre [BM, Corne]}
\vedette{niû}
\région{BO}
\end{entrée}

\begin{entrée}
{ancrer}
\vedette{niûni}
\région{BO [BM]}
\end{entrée}

\begin{entrée}
{ancrer ; jeter l'ancre}
\vedette{pao kinu}
\région{GOs}
\end{entrée}

\begin{entrée}
{à terre (lit. sec)}
\vedette{bwa-mõ}
\région{GOs}
\variante{%
\vedette{bwa-mol}
\région{BO}}
\end{entrée}

\begin{entrée}
{au vent ; vent debout}
\vedette{bwa dre}
\région{GOs PA BO}
\variante{%
\vedette{bwa dèèn}
\région{PA BO}}
\end{entrée}

\begin{entrée}
{balancier de pirogue}
\vedette{hajaa-wõ}
\région{GOs}
\end{entrée}

\begin{entrée}
{barrer (bateau)}
\vedette{hããny}
\région{PA BO}
\end{entrée}

\begin{entrée}
{bateau à voile}
\vedette{wõ-pwaala}
\région{GOs}
\variante{%
\vedette{wô-waala}
\région{GO(s)}}
\end{entrée}

\begin{entrée}
{bateau ; embarcation}
\classe{nom}
\vedette{wõ}
\sens{1}
\région{GOs}
\variante{%
\vedette{wony}
\région{PA WEM BO}}
\end{entrée}

\begin{entrée}
{chavirer ; retourner (se)}
\vedette{bu}\homonyme{3}
\région{GOs}
\variante{%
\vedette{bul}
\région{BO PA}}
\end{entrée}

\begin{entrée}
{cordage de bateau}
\vedette{kô-wõ}
\région{GOs}
\variante{%
\vedette{kô-wony}
\région{BO PA}}
\end{entrée}

\begin{entrée}
{diriger ; conduire}
\vedette{hã}\homonyme{1}
\région{GOs}
\variante{%
\vedette{hãny}
\région{BO}}
\end{entrée}

\begin{entrée}
{diriger le bateau/voiture ; barrer ; conduire}
\vedette{hãnge}\homonyme{2}
\région{GOs}
\end{entrée}

\begin{entrée}
{diriger (voiture)}
\vedette{hããny}
\région{PA BO}
\end{entrée}

\begin{entrée}
{flotter}
\vedette{phumwêê}
\région{GOs}
\end{entrée}

\begin{entrée}
{flotter, dériver}
\vedette{mhwêê}
\région{GOs}
\variante{%
\vedette{mhwèèn}
\région{PA BO}}
\end{entrée}

\begin{entrée}
{flotteur de balancier}
\vedette{dabò}
\région{GOs}
\end{entrée}

\begin{entrée}
{gaffe ; barre}
\vedette{puu}\homonyme{2}
\région{GOs}
\variante{%
\vedette{pulu}
\région{BO}}
\end{entrée}

\begin{entrée}
{gouvernail}
\vedette{hã}\homonyme{1}
\région{GOs}
\variante{%
\vedette{hãny}
\région{BO}}
\end{entrée}

\begin{entrée}
{gouvernail ; volant}
\vedette{hããny}
\région{PA BO}
\end{entrée}

\begin{entrée}
{louvoyer ; tirer des bords vent debout}
\vedette{pe-koone}
\région{BO}
\end{entrée}

\begin{entrée}
{mât}
\vedette{cawane}
\région{GOs}
\variante{%
\vedette{cawan}
\région{BO (Corne)}}
\end{entrée}

\begin{entrée}
{mât du bateau}
\vedette{kò-wony}
\région{BO}
\end{entrée}

\begin{entrée}
{milieu du bateau}
\vedette{gò-wõ}
\région{GOs}
\variante{%
\vedette{gò-wòny}
\région{BO}}
\end{entrée}

\begin{entrée}
{naviguer (avec un bateau à voile) ; voguer}
\vedette{pwaala}
\région{GOs PA BO}
\end{entrée}

\begin{entrée}
{pagaie}
\vedette{hã}\homonyme{1}
\région{GOs}
\variante{%
\vedette{hãny}
\région{BO}}
\end{entrée}

\begin{entrée}
{pagaie ; perche}
\vedette{tha}\homonyme{4}
\région{GOs}
\end{entrée}

\begin{entrée}
{perche pour pousser la pirogue}
\vedette{puu}\homonyme{2}
\région{GOs}
\variante{%
\vedette{pulu}
\région{BO}}
\end{entrée}

\begin{entrée}
{pirogue}
\vedette{karava}
\région{GO}
\end{entrée}

\begin{entrée}
{pirogue à balancier}
\vedette{katrepa}
\région{GOs}
\end{entrée}

\begin{entrée}
{pirogue à balancier}
\vedette{wõ-ce}
\région{GOs}
\end{entrée}

\begin{entrée}
{pirogue (Corne)}
\vedette{bwaxala}
\région{BO}
\end{entrée}

\begin{entrée}
{pirogue double}
\vedette{mwa-draeca}
\région{GOs}
\end{entrée}

\begin{entrée}
{pirogue double [Corne]}
\vedette{tèn}
\région{BO}
\end{entrée}

\begin{entrée}
{planche de bateau}
\vedette{hõõ-wony}
\région{PA}
\end{entrée}

\begin{entrée}
{poupe}
\vedette{mura-wõ}
\région{GOs}
\end{entrée}

\begin{entrée}
{poupe}
\vedette{pòbwinõ-wõ}
\région{GOs}
\variante{%
\vedette{pòbwinõ-wòny}
\région{PA BO}}
\end{entrée}

\begin{entrée}
{pousser (bateau) avec la perche (lit. piquer)}
\vedette{thi-puu}
\région{BO PA}
\variante{%
\vedette{tho-puu}
\région{GO(s) BO}}
\end{entrée}

\begin{entrée}
{pousser (bateau) avec la perche (lit. piquer)}
\vedette{tho-puu}
\région{GOs BO}
\variante{%
\vedette{thi-puu}
\région{PA BO}}
\end{entrée}

\begin{entrée}
{proue}
\vedette{mè-wõ}
\région{GOs}
\variante{%
\vedette{mee-wony}
\région{PA BO}}
\end{entrée}

\begin{entrée}
{radeau (en bambou)}
\vedette{phaa-gò}
\région{GOs PA BO}
\variante{%
\vedette{phaa}
\région{GO(s)}}
\end{entrée}

\begin{entrée}
{radeau en bambou}
\vedette{wõ-go}
\région{GOs}
\end{entrée}

\begin{entrée}
{radeau ; flotteur}
\classe{nom}
\vedette{phaa}
\sens{2}
\région{GOs BO PA}
\end{entrée}

\begin{entrée}
{rame}
\vedette{ba-paaba}
\région{GOs}
\end{entrée}

\begin{entrée}
{ramer}
\vedette{haal}
\région{PA BO}
\end{entrée}

\begin{entrée}
{ramer}
\vedette{paaba}
\région{GOs}
\end{entrée}

\begin{entrée}
{sombrer ; couler ; noyer (se)}
\vedette{bu}\homonyme{3}
\région{GOs}
\variante{%
\vedette{bul}
\région{BO PA}}
\end{entrée}

\begin{entrée}
{tirer des bords [Corne]}
\vedette{tua pwaala}
\région{BO}
\end{entrée}

\begin{entrée}
{virer de bord vent debout ; louvoyer}
\vedette{koone}
\région{BO}
\end{entrée}

\begin{entrée}
{voile (bateau) ; bâche}
\vedette{nhe}\homonyme{2}
\région{GOs}
\variante{%
\vedette{nhe}
\région{PA BO}}
\end{entrée}

\subsection{Feu : objets et actions liés au feu}

\begin{entrée}
{allumer (feu, lampe, cigarette, briquet)}
\vedette{cale}
\région{GOs BO}
\end{entrée}

\begin{entrée}
{allumer un feu}
\vedette{phai-yaai}
\région{GOs BO}
\variante{%
\vedette{phai-yaai, pha-yaai}
\région{PA}}
\end{entrée}

\begin{entrée}
{allumer un feu de brousse ; brûler (pour préparer un champ)}
\vedette{khi-kò}
\région{GOs}
\end{entrée}

\begin{entrée}
{allumer un feu de brousse (lit. piquer le feu)}
\vedette{thi yaai}
\région{GOs WEM BO}
\end{entrée}

\begin{entrée}
{allumette (lit. feu-frapper)}
\vedette{ya-paò}
\région{GOs}
\end{entrée}

\begin{entrée}
{allumettes ; boîte d'allumettes (lit. boîte à feu)}
\vedette{mõ-yai}
\région{GOs WEH WEM PA}
\end{entrée}

\begin{entrée}
{attiser (en faisant du vent)}
\classe{v.t.}
\vedette{ula}\homonyme{1}
\sens{3}
\région{GOs BO}
\end{entrée}

\begin{entrée}
{attiser le feu avec éventail}
\vedette{ula yai}
\région{GOs PA}
\end{entrée}

\begin{entrée}
{attiser le feu en remuant les braises avec un bâton [PA]}
\vedette{thi yaai}
\région{GOs WEM BO}
\end{entrée}

\begin{entrée}
{attiser ; pousser le feu (en ajoutant du bois)}
\vedette{carû}
\région{GOs}
\variante{%
\vedette{carun}
\région{PA BO}}
\end{entrée}

\begin{entrée}
{bois pour la cuisine}
\vedette{ce-kiyai}
\région{GOs BO}
\variante{%
\vedette{ce-xiyai}}
\end{entrée}

\begin{entrée}
{bois pour le feu allumé dehors pour se réchauffer}
\vedette{ce-kiyai}
\région{GOs BO}
\variante{%
\vedette{ce-xiyai}}
\end{entrée}

\begin{entrée}
{bois qui encadre le foyer}
\classe{nom}
\vedette{thaxe}
\sens{1}
\région{GOs BO}
\end{entrée}

\begin{entrée}
{bois sur lequel on frotte pour faire du feu}
\vedette{ce he}\homonyme{1}
\région{GOs PA BO}
\end{entrée}

\begin{entrée}
{braise}
\vedette{alaaba}
\région{GOs BO PA}
\end{entrée}

\begin{entrée}
{brandon de la bûche pour le feu de la nuit}
\vedette{jińõ ce-bon}
\région{PA BO}
\end{entrée}

\begin{entrée}
{brandon (utilisé pour mettre le feu ou s'éclairer dans le noir)}
\vedette{jińõ yaai}
\région{GOs PA BO}
\end{entrée}

\begin{entrée}
{brûler (brousse) ; incendier}
\classe{v.t.}
\vedette{kîni}
\sens{1}
\région{GOs}
\variante{%
\vedette{khînî}
\région{PA BO}}
\end{entrée}

\begin{entrée}
{brûler ; brûlant}
\classe{v.stat.}
\vedette{tòò}\homonyme{1}
\sens{2}
\région{GOs PABO}
\end{entrée}

\begin{entrée}
{brûler ; flamber}
\vedette{ulo}
\région{BO}
\end{entrée}

\begin{entrée}
{brûler (les champs) ; pratiquer le brûlis}
\vedette{khi-kha}
\région{GOs}
\variante{%
\vedette{khi-ga}
\région{GOs}}
\variante{%
\vedette{ki kha}
\région{GO(s)}}
\variante{%
\vedette{ki khan}
\région{PA}}
\end{entrée}

\begin{entrée}
{bûche}
\vedette{nhe}\homonyme{1}
\région{PA}
\end{entrée}

\begin{entrée}
{bûche (grosse, pour la nuit, portée par les hommes)}
\vedette{ce-bò}
\région{GOs}
\variante{%
\vedette{ce-bòn}
\région{WEM BO}}
\variante{%
\vedette{ci-bòn}
\région{PA}}
\end{entrée}

\begin{entrée}
{cendres du feu}
\vedette{drawa yai}
\région{GOs}
\end{entrée}

\begin{entrée}
{cendres ; poudre}
\vedette{dra}\homonyme{1}
\région{GOs}
\région{BO PA}
\variante{%
\vedette{da}}
\end{entrée}

\begin{entrée}
{chauffer (se) assis près du feu}
\vedette{tre-xiyai}
\région{GOs PA}
\variante{%
\vedette{tre-kiyai, te-'xiyai}
\région{GO(s)}}
\variante{%
\vedette{tee-kiyai}
\région{PA}}
\end{entrée}

\begin{entrée}
{chenêts ; foyer ; rails du feu}
\vedette{bwèèra}
\région{GOs PA}
\variante{%
\vedette{bwèèrao}
\région{BO (Corne)}}
\end{entrée}

\begin{entrée}
{étaler (le feu du four, enlever le bois et éparpiller les braises) [BM, Corne]}
\vedette{taare}
\région{BO}
\end{entrée}

\begin{entrée}
{éteindre en soufflant (bougie, allumette)}
\vedette{ui-bööni}
\région{GOs PA BO}
\end{entrée}

\begin{entrée}
{éteindre (le feu)}
\vedette{khi-bö}
\région{GOs}
\end{entrée}

\begin{entrée}
{éteindre (petit feu, lumière)}
\vedette{thi-bö}\homonyme{1}
\région{GOs}
\variante{%
\vedette{bwo}
\région{BO}}
\end{entrée}

\begin{entrée}
{éteindre un feu}
\vedette{bö}
\région{WE BO}
\variante{%
\vedette{bwö}
\région{BO}}
\end{entrée}

\begin{entrée}
{éteint (être) ; éteindre (un feu)}
\vedette{thi-bööni}\homonyme{2}
\région{GOs}
\région{BO}
\variante{%
\vedette{bwo, bo}}
\end{entrée}

\begin{entrée}
{étincelle (du feu)}
\vedette{poo-yaai}
\région{PA BO}
\end{entrée}

\begin{entrée}
{étincelle du feu ; crépiter (feu)}
\vedette{tîî-yaai}
\région{WEM}
\end{entrée}

\begin{entrée}
{éventail (en feuille de cocotier pour le feu)}
\vedette{ba-u}
\région{GOs}
\variante{%
\vedette{ba-ul}
\région{WEM WE BO PA}}
\end{entrée}

\begin{entrée}
{fagot de bois}
\vedette{phò-ce}
\région{GOs}
\end{entrée}

\begin{entrée}
{feu}
\vedette{yaai}
\région{GOs BO}
\variante{%
\vedette{yai}
\région{PA}}
\end{entrée}

\begin{entrée}
{feu allumé par friction}
\classe{nom}
\vedette{yaa-he}
\région{GOs PA}
\variante{%
\vedette{yaai-he}
\région{GO PA}}
\end{entrée}

\begin{entrée}
{flamber ; brûler}
\vedette{olo}
\région{GOs BO}
\end{entrée}

\begin{entrée}
{flamme}
\vedette{oloomã}
\région{GOs BO}
\end{entrée}

\begin{entrée}
{foyer}
\vedette{mõ-pha-yai}
\région{GOs WEM}
\end{entrée}

\begin{entrée}
{foyer[BM]}
\vedette{mãã-ce-bwòn}
\région{BO}
\end{entrée}

\begin{entrée}
{foyer ; endroit où l'on fait le feu}
\vedette{maadre}
\région{GOs}
\variante{%
\vedette{maadraò}
\région{GO(s)}}
\variante{%
\vedette{maada}
\région{BO}}
\end{entrée}

\begin{entrée}
{foyer ; maison où l'on fait le feu pour dormir}
\vedette{mõ-phaa-ce-bo}
\région{GOs}
\région{WEM PA}
\variante{%
\vedette{mõ-phaa-ce-bòn}}
\end{entrée}

\begin{entrée}
{fumée}
\classe{nom}
\vedette{pu}\homonyme{2}
\sens{1}
\région{GOs}
\variante{%
\vedette{pum}
\région{PA BO}}
\variante{%
\vedette{bu, bo}
\région{BO}}
\end{entrée}

\begin{entrée}
{incendier; mettre le feu}
\vedette{pe-thi}
\région{GOs}
\end{entrée}

\begin{entrée}
{incendier ; mettre le feu}
\vedette{thii}\homonyme{3}
\région{BO [BM]}
\variante{%
\vedette{thiin}
\région{BO}}
\end{entrée}

\begin{entrée}
{mettre le feu}
\vedette{thi-yaai}
\région{GOs BO PA}
\end{entrée}

\begin{entrée}
{pierre autour du foyer [BO, [BM]]}
\classe{nom}
\vedette{thaxee-phweemwa}
\sens{2}
\région{GOs BO}
\variante{%
\vedette{taage, thaaxe}
\région{BO}}
\end{entrée}

\begin{entrée}
{pierre rouge (qu'on frappe pour faire des étincelles) [Corne]}
\vedette{paa-trèè-mii}
\région{GOs}
\variante{%
\vedette{pa-tèè-mii}
\région{BO}}
\end{entrée}

\begin{entrée}
{pierres (utilisées comme support à marmite, souvent au nombre de trois)}
\vedette{bwèèra}
\région{GOs PA}
\variante{%
\vedette{bwèèrao}
\région{BO (Corne)}}
\end{entrée}

\begin{entrée}
{pincettes}
\vedette{homwi}
\région{BO [Corne]}
\end{entrée}

\begin{entrée}
{pousser le feu}
\vedette{tha-carûni}
\région{PA}
\variante{%
\vedette{tha-yarûni}
\région{PA}}
\end{entrée}

\begin{entrée}
{pousser le feu (sous la marmite, ou dans la maison, moins fort que 'carûni')}
\vedette{pigi yaai}
\région{GOs}
\end{entrée}

\begin{entrée}
{rassembler/entasser les braises}
\classe{v}
\vedette{phiige}
\sens{2}
\région{GOs}
\variante{%
\vedette{peenge}
\région{BO [BM]}}
\end{entrée}

\begin{entrée}
{réchauffer (se) auprès du feu}
\vedette{khiai}
\région{GOs PA BO}
\end{entrée}

\begin{entrée}
{réchauffer (se) couché auprès du feu (la nuit)}
\vedette{kô-kiai}
\région{GOs}
\variante{%
\vedette{kô-xiai}
\région{GO(s)}}
\end{entrée}

\begin{entrée}
{réchauffer (se) debout près du feu}
\vedette{ku-kiai}
\région{GOs}
\variante{%
\vedette{ku-xiai}
\région{GO(s)}}
\end{entrée}

\begin{entrée}
{remuer les braises}
\classe{v}
\vedette{thiçe}
\sens{1}
\région{GOs}
\end{entrée}

\begin{entrée}
{rougi ; enflammé [BO]}
\classe{v.stat.}
\vedette{tòò}\homonyme{1}
\sens{2}
\région{GOs PABO}
\end{entrée}

\begin{entrée}
{schistes}
\vedette{paa-trèè-mii}
\région{GOs}
\variante{%
\vedette{pa-tèè-mii}
\région{BO}}
\end{entrée}

\begin{entrée}
{séparer ; dégager}
\vedette{taare}
\région{BO}
\end{entrée}

\begin{entrée}
{souffler (sur le feu)}
\classe{v}
\vedette{ui}\homonyme{1}
\sens{1}
\région{GOs PA BO}
\end{entrée}

\begin{entrée}
{suie}
\vedette{dra}\homonyme{1}
\région{GOs}
\région{BO PA}
\variante{%
\vedette{da}}
\end{entrée}

\begin{entrée}
{suie (du feu) (sur le toit ou les marmites)}
\vedette{ti-yaai}
\région{GOs}
\end{entrée}

\begin{entrée}
{suie ; noir de fumée}
\vedette{phaavã}
\région{PA BO}
\end{entrée}

\begin{entrée}
{suie noire de la fumée dans les maisons [Corne]}
\vedette{zabo}
\région{BO}
\end{entrée}

\begin{entrée}
{suie sur la marmite}
\vedette{drawa-dröö}
\région{GOs}
\end{entrée}

\begin{entrée}
{tison}
\vedette{jińõ ce-bon}
\région{PA BO}
\end{entrée}

\begin{entrée}
{tisonnier}
\vedette{ba-thiçe}
\région{GOs}
\end{entrée}

\begin{entrée}
{tisonnier}
\vedette{poxa-he}
\région{GOs}
\end{entrée}

\begin{entrée}
{tisons}
\vedette{jińõ yaai}
\région{GOs PA BO}
\end{entrée}

\subsection{Cuisine, alimentation}

\subsubsection{Ustensiles}

\begin{entrée}
{argile de poterie}
\vedette{dröö}
\région{GOs}
\variante{%
\vedette{doo}
\région{BO PA}}
\end{entrée}

\begin{entrée}
{assiette ; plat}
\vedette{za}\homonyme{1}
\région{GOs}
\variante{%
\vedette{zha}
\région{GA}}
\variante{%
\vedette{zam}
\région{PA}}
\variante{%
\vedette{yam}
\région{BO}}
\end{entrée}

\begin{entrée}
{bassine ; cuvette}
\vedette{bèsè}
\région{GOs}
\end{entrée}

\begin{entrée}
{bouilloire (lit. contenant-thé)}
\vedette{mõ-tri}
\région{GOs}
\end{entrée}

\begin{entrée}
{bouteille}
\vedette{burey}
\région{PA}
\end{entrée}

\begin{entrée}
{calebasse ; noix de coco vide}
\vedette{bwi-nu}
\région{BO}
\end{entrée}

\begin{entrée}
{calebasse (servant à porter l'eau)}
\vedette{we-kae}
\région{GOs}
\end{entrée}

\begin{entrée}
{coquille de moule (sert de grattoir à banane)}
\classe{nom}
\vedette{hizu}
\sens{2}
\région{GOs}
\end{entrée}

\begin{entrée}
{corbeille}
\vedette{za}\homonyme{1}
\région{GOs}
\variante{%
\vedette{zha}
\région{GA}}
\variante{%
\vedette{zam}
\région{PA}}
\variante{%
\vedette{yam}
\région{BO}}
\end{entrée}

\begin{entrée}
{cuillère}
\vedette{ba-tröi}
\région{GOs}
\variante{%
\vedette{ba-rui}
\région{PA}}
\end{entrée}

\begin{entrée}
{écumoire (lit. écope à trou)}
\vedette{ba-kheevwo phwa}
\région{GOs}
\variante{%
\vedette{ba-kham phwa}
\région{PA BO}}
\end{entrée}

\begin{entrée}
{grattoir}
\vedette{ba-zo}
\région{GOs}
\variante{%
\vedette{ba-zòl}
\région{PA}}
\variante{%
\vedette{ba-yòl}
\région{BO}}
\end{entrée}

\begin{entrée}
{grattoir métallique}
\vedette{hoxaxe}
\région{PA}
\end{entrée}

\begin{entrée}
{louche (lit. écope fermée)}
\vedette{ba-kheevwo thô}
\région{GOs}
\variante{%
\vedette{ba-kham thô}
\région{PA}}
\variante{%
\vedette{ba-kham thõn}
\région{BO}}
\end{entrée}

\begin{entrée}
{marmite (originellement en poterie)}
\vedette{dröö}
\région{GOs}
\variante{%
\vedette{doo}
\région{BO PA}}
\end{entrée}

\begin{entrée}
{ouverture de la marmite}
\vedette{phwe-döö}
\région{PA}
\end{entrée}

\begin{entrée}
{poignées de la marmite}
\vedette{kênii-döö}
\région{PA}
\end{entrée}

\begin{entrée}
{seau}
\vedette{chiò}
\région{GOs BO PA}
\end{entrée}

\begin{entrée}
{verre ; bol}
\vedette{ba-kudo}
\région{GOs}
\variante{%
\vedette{ba-xudo}
\région{GO(s)}}
\variante{%
\vedette{ba-kido}
\région{PA BO}}
\end{entrée}

\subsubsection{Préparation des aliments; modes de préparation et de cuisson}

\begin{entrée}
{bouillir, cuire (les ignames) avec la peau}
\vedette{phai cii}
\région{GOs}
\end{entrée}

\begin{entrée}
{bouillir (eau)}
\vedette{phu}\homonyme{1}
\région{GOs}
\variante{%
\vedette{phul}
\région{PA BO}}
\end{entrée}

\begin{entrée}
{bouillir ; faire cuire}
\vedette{phai}\homonyme{1}
\région{GOs}
\région{BO PA}
\variante{%
\vedette{phaai}}
\end{entrée}

\begin{entrée}
{brûlé, carbonisé (au fond de la marmite)}
\vedette{tòòri}
\région{GOs}
\variante{%
\vedette{toorim}
\région{PA}}
\end{entrée}

\begin{entrée}
{cailler [BM]}
\vedette{mèxèè}
\région{BO}
\end{entrée}

\begin{entrée}
{chauffer [BM]}
\vedette{thane}
\région{BO}
\end{entrée}

\begin{entrée}
{couper en lamelle}
\vedette{eloe}
\région{GA}
\end{entrée}

\begin{entrée}
{couper en lamelles}
\vedette{lhòlòi}
\sens{1}
\région{GOs}
\variante{%
\vedette{lòloi}
\région{BO PA}}
\end{entrée}

\begin{entrée}
{couverture du four (en peaux de niaoulis)}
\vedette{kulò}
\région{GOs BO}
\end{entrée}

\begin{entrée}
{cuire à l'étouffée}
\vedette{kööni}
\région{GOs}
\variante{%
\vedette{kooni}
\région{PA WEM}}
\end{entrée}

\begin{entrée}
{cuire au four enterré ; mettre au four enterré}
\vedette{kööni}
\région{GOs}
\variante{%
\vedette{kooni}
\région{PA WEM}}
\end{entrée}

\begin{entrée}
{cuire avec la peau (bananes)}
\vedette{phai mwa-çii}
\région{GOs}
\end{entrée}

\begin{entrée}
{cuire/griller sur le feu (sur les braises ou dans les cendres)}
\classe{v.t.}
\vedette{kîni}
\sens{2}
\région{GOs}
\variante{%
\vedette{khînî}
\région{PA BO}}
\end{entrée}

\begin{entrée}
{cuire sous la cendre (en remuant la nourriture) ; mettre à cuire sous les braises}
\classe{v}
\vedette{thiçe}
\sens{2}
\région{GOs}
\end{entrée}

\begin{entrée}
{cuisine (faire la) ; cuisiner}
\vedette{puçò}
\région{GOs}
\variante{%
\vedette{puyòl, puçòl}
\région{WE PA BO}}
\end{entrée}

\begin{entrée}
{cuit}
\vedette{minõ}
\région{GOs}
\région{PA BO}
\variante{%
\vedette{minòng}}
\end{entrée}

\begin{entrée}
{cuit (à moitié) ; pas assez cuit}
\vedette{magira}
\région{GOs}
\end{entrée}

\begin{entrée}
{cuit (être)}
\vedette{phu}\homonyme{1}
\région{GOs}
\variante{%
\vedette{phul}
\région{PA BO}}
\end{entrée}

\begin{entrée}
{cuit (mal) ; pas assez cuit (riz, viande)}
\vedette{mãiyã}
\région{GOs}
\variante{%
\vedette{mãiã}
\région{GO(s)}}
\variante{%
\vedette{meã}
\région{BO [BM]}}
\variante{%
\vedette{mee}
\région{PA}}
\end{entrée}

\begin{entrée}
{cuit (mal) ; pas cuit}
\vedette{phi}\homonyme{2}
\région{GOs}
\end{entrée}

\begin{entrée}
{dépecer}
\classe{v.t.}
\vedette{hili}\homonyme{1}
\sens{2}
\région{GOs BO}
\end{entrée}

\begin{entrée}
{déterrer (le four)}
\vedette{pwiya}
\région{BO}
\end{entrée}

\begin{entrée}
{écailler le poisson ; écaille (de poisson)}
\classe{v ; n}
\vedette{kubi}
\sens{2}
\région{GOs PA BO}
\end{entrée}

\begin{entrée}
{écorcer (du coco sur un épieu) ; éplucher (coco)}
\vedette{tha}\homonyme{2}
\région{GOs BO PA}
\end{entrée}

\begin{entrée}
{écorcer le coco (sur un épieu)}
\vedette{tha nu}
\région{GOs PA BO}
\end{entrée}

\begin{entrée}
{écorcher (animal)}
\classe{v.t.}
\vedette{hili}\homonyme{1}
\sens{2}
\région{GOs BO}
\end{entrée}

\begin{entrée}
{écorcher (s')}
\vedette{pwazi}
\région{GOs}
\end{entrée}

\begin{entrée}
{envelopper de la nourriture dans des feuilles pour les faire cuire}
\vedette{phõõme}
\région{GOs}
\variante{%
\vedette{phõõm}
\région{BO}}
\end{entrée}

\begin{entrée}
{envelopper et attacher (nourriture)}
\vedette{nhõõxi}
\région{GOs}
\end{entrée}

\begin{entrée}
{éplucher (manioc) ; peler (avec les doigts, banane, tubercule cuit)}
\vedette{pwaaci}
\région{GOs}
\variante{%
\vedette{pwayi}
\région{PA BO}}
\end{entrée}

\begin{entrée}
{évider avec une cuillère (noix de coco, papaye, avocat)}
\vedette{kevi}
\région{PA}
\end{entrée}

\begin{entrée}
{faire du bouillon, de la soupe}
\vedette{chawe}
\région{GOs}
\end{entrée}

\begin{entrée}
{four enterré}
\vedette{kîbi}
\région{GOs PA BO}
\end{entrée}

\begin{entrée}
{frire}
\vedette{paxadraa}
\région{GOs}
\variante{%
\vedette{paxadaa}
\région{WEM}}
\end{entrée}

\begin{entrée}
{fumer (poisson)}
\vedette{phwaawe}
\région{GOs PA WEM BO}
\end{entrée}

\begin{entrée}
{gratter avec un couteau}
\vedette{thibe}
\région{GOs PA BO}
\variante{%
\vedette{thebe}
\région{GO(s) BO}}
\end{entrée}

\begin{entrée}
{gratter (l'igname cuite ou la peau de l'igname, patates, taro)}
\classe{v.t.}
\vedette{zòli}
\sens{1}
\région{GOs PA}
\variante{%
\vedette{zhòli}
\région{GO(s)}}
\variante{%
\vedette{yòli, yòòli}
\région{BO}}
\end{entrée}

\begin{entrée}
{gratter l'igname sauvage (dimwa) ; râper}
\vedette{thiò dimwa}
\région{GOs PA BO}
\variante{%
\vedette{thixò, thiò}
\région{PA BO}}
\end{entrée}

\begin{entrée}
{griller ; brûler}
\vedette{ki}\homonyme{1}
\région{GOs PA BO}
\end{entrée}

\begin{entrée}
{griller, cuire au feu les prémices des ignames}
\classe{v}
\vedette{khî-kui}
\groupe{A}
\sens{1}
\région{GOs PA BO}
\end{entrée}

\begin{entrée}
{griller ; rôtir}
\vedette{khîni}
\région{GOs PA BO}
\end{entrée}

\begin{entrée}
{mettre (à cuire) dans la cendre}
\classe{v}
\vedette{khêmi}
\sens{2}
\région{GOs PA}
\variante{%
\vedette{kêmi}
\région{PA BO}}
\end{entrée}

\begin{entrée}
{passer sur la flamme des anguilles pour enlever la couche gluante}
\vedette{phubwe}
\région{GOs BO}
\variante{%
\vedette{phöbwe}
\région{GO(s)}}
\variante{%
\vedette{phobwe}
\région{BO}}
\end{entrée}

\begin{entrée}
{passer sur la flamme pour assouplir (feuille de bananier)}
\vedette{ûne}
\région{GOs}
\end{entrée}

\begin{entrée}
{passer sur la flamme pour assouplir les feuilles}
\vedette{phubwe}
\région{GOs BO}
\variante{%
\vedette{phöbwe}
\région{GO(s)}}
\variante{%
\vedette{phobwe}
\région{BO}}
\end{entrée}

\begin{entrée}
{peau (cuire avec la)}
\vedette{mwa-çii}
\région{GO}
\end{entrée}

\begin{entrée}
{peler (avec un couteau, fruits, igname ou taro cru) ; éplucher (légumes)}
\vedette{thibe}
\région{GOs PA BO}
\variante{%
\vedette{thebe}
\région{GO(s) BO}}
\end{entrée}

\begin{entrée}
{pétrir (pain)}
\vedette{peeni}
\région{GOs}
\end{entrée}

\begin{entrée}
{pierres du four enterré}
\vedette{paxa-kîbi}
\région{GOs BO PA}
\variante{%
\vedette{paxawa kîbi}
\région{GO(s)}}
\end{entrée}

\begin{entrée}
{poêle à frire}
\vedette{paxadraa}
\région{GOs}
\variante{%
\vedette{paxadaa}
\région{WEM}}
\end{entrée}

\begin{entrée}
{préparation à base d'igname coupées en fines rondelles}
\vedette{lhòlòi}
\sens{2}
\région{GOs}
\variante{%
\vedette{lòloi}
\région{BO PA}}
\end{entrée}

\begin{entrée}
{prêt (être)}
\vedette{phu}\homonyme{1}
\région{GOs}
\variante{%
\vedette{phul}
\région{PA BO}}
\end{entrée}

\begin{entrée}
{râper (coco)}
\classe{v.t.}
\vedette{zòli}
\sens{1}
\région{GOs PA}
\variante{%
\vedette{zhòli}
\région{GO(s)}}
\variante{%
\vedette{yòli, yòòli}
\région{BO}}
\end{entrée}

\begin{entrée}
{réchauffer (nourriture)}
\vedette{pa-tòè}
\région{GOs}
\end{entrée}

\begin{entrée}
{retirer (marmite du feu)}
\classe{v}
\vedette{phu}\homonyme{3}
\sens{2}
\région{GOs}
\variante{%
\vedette{phuxa}
\région{PA}}
\end{entrée}

\begin{entrée}
{retirer qqch. de qqch. (marmite)}
\classe{v}
\vedette{ii}\homonyme{1}
\sens{1}
\région{GOs PA BO}
\end{entrée}

\begin{entrée}
{rôtir}
\vedette{parang}
\région{BO}
\end{entrée}

\begin{entrée}
{saler la nourriture ; mettre du sel}
\vedette{zanyi}
\sens{2}
\région{GOs}
\variante{%
\vedette{zhanyi}
\région{GA}}
\end{entrée}

\begin{entrée}
{servir (la nourriture)}
\classe{v}
\vedette{ii}\homonyme{1}
\sens{1}
\région{GOs PA BO}
\end{entrée}

\begin{entrée}
{sortir (d'une marmite) ; servir}
\classe{v}
\vedette{yatre}
\sens{3}
\région{GOs}
\variante{%
\vedette{yare, yaare}
\région{GO BO}}
\end{entrée}

\begin{entrée}
{trop cuit}
\vedette{mha minõ}
\région{GOs}
\end{entrée}

\begin{entrée}
{vapeur de la marmite}
\vedette{drava-dröö}
\région{GOs}
\end{entrée}

\begin{entrée}
{vider le four enterré [Corne]}
\vedette{puya}
\région{BO}
\end{entrée}

\begin{entrée}
{vider (le four enterré); sortir du four}
\classe{v}
\vedette{ii}\homonyme{1}
\sens{1}
\région{GOs PA BO}
\end{entrée}

\begin{entrée}
{vider qqch (marmite)}
\vedette{pa-pii-ni}
\région{GOs}
\end{entrée}

\subsubsection{Aliments, alimentation}

\begin{entrée}
{altéré ; pas frais (nourriture)}
\vedette{maja}\homonyme{1}
\région{GOs}
\end{entrée}

\begin{entrée}
{appétissant ; beau à voir (légumes)}
\vedette{a-vwö-nu kûûni}
\région{GOs}
\end{entrée}

\begin{entrée}
{avarié ; tourné ; sûr}
\vedette{mhodro}
\région{GOs}
\variante{%
\vedette{mhòdo}
\région{BO}}
\end{entrée}

\begin{entrée}
{avoir envie de manger qqch}
\vedette{makoyoo}
\région{GOs}
\variante{%
\vedette{maxoyoo}
\région{GO(s)}}
\end{entrée}

\begin{entrée}
{banane sucre (Corne)}
\vedette{cuka}
\région{BO [BM, Corne]}
\end{entrée}

\begin{entrée}
{beurre}
\vedette{dibee}
\région{GOs}
\end{entrée}

\begin{entrée}
{boire}
\vedette{kudò}
\groupe{B}
\région{GOs PA}
\variante{%
\vedette{kido}
\région{PA BO WEM WE}}
\end{entrée}

\begin{entrée}
{boire chaud [Corne]}
\vedette{pharu}
\région{BO}
\end{entrée}

\begin{entrée}
{boire chaud (soupe)}
\vedette{phòzo}
\région{GOs}
\variante{%
\vedette{phòlo}
\région{PA BO}}
\end{entrée}

\begin{entrée}
{boisson}
\vedette{kudò}
\groupe{A}
\région{GOs PA}
\variante{%
\vedette{kido}
\région{PA BO WEM WE}}
\end{entrée}

\begin{entrée}
{bounia}
\vedette{bunya}
\région{GOs}
\end{entrée}

\begin{entrée}
{café}
\vedette{kafe}
\région{GOs}
\variante{%
\vedette{kape}
\région{PA BO WEM WE}}
\end{entrée}

\begin{entrée}
{chair de coco contenant des vers [GO]}
\vedette{waluvwi}
\région{GOs}
\variante{%
\vedette{waluvi}
\région{BO [BM]}}
\end{entrée}

\begin{entrée}
{chair (d'igname, taro)}
\vedette{pizò}
\région{GOs}
\variante{%
\vedette{pilo-n, pilòò, pilò}
\région{PA BO}}
\variante{%
\vedette{pila}
\région{BO}}
\end{entrée}

\begin{entrée}
{chair (en général) ; viande ; muscle}
\vedette{pizò}
\région{GOs}
\variante{%
\vedette{pilo-n, pilòò, pilò}
\région{PA BO}}
\variante{%
\vedette{pila}
\région{BO}}
\end{entrée}

\begin{entrée}
{chercher de la nourriture}
\vedette{zala}\homonyme{1}
\région{GOs PA}
\variante{%
\vedette{yala}
\région{BO}}
\end{entrée}

\begin{entrée}
{chercher de la nourriture ; aller à la pêche ; aller à la chasse [BM]}
\vedette{yhala}
\région{BO}
\end{entrée}

\begin{entrée}
{condiment (pour accompagner les féculents)}
\vedette{caai}\homonyme{1}
\région{GOs}
\variante{%
\vedette{ca-}
\région{BO}}
\end{entrée}

\begin{entrée}
{croquer}
\classe{v}
\vedette{kûû}
\sens{1}
\région{GOs BO}
\end{entrée}

\begin{entrée}
{cuit (mal) ; pas assez cuit (riz, viande)}
\vedette{mãiyã}
\région{GOs}
\variante{%
\vedette{mãiã}
\région{GO(s)}}
\variante{%
\vedette{meã}
\région{BO [BM]}}
\variante{%
\vedette{mee}
\région{PA}}
\end{entrée}

\begin{entrée}
{débris ; restes}
\vedette{mini-}
\région{GOs}
\end{entrée}

\begin{entrée}
{dégoûté ; faire le difficile}
\classe{v}
\vedette{hing}
\sens{2}
\région{PA BO}
\end{entrée}

\begin{entrée}
{dégoûté ; faire le difficile}
\vedette{iing}
\région{PA}
\end{entrée}

\begin{entrée}
{dégoûté ; faire le difficile}
\vedette{zanyii}
\région{GOs}
\variante{%
\vedette{zhanyii}
\région{GA}}
\variante{%
\vedette{hing}
\région{PA}}
\end{entrée}

\begin{entrée}
{difficile (qui fait le) [Corne]}
\vedette{mèng}
\région{BO}
\end{entrée}

\begin{entrée}
{dîner, souper}
\vedette{hovwo thrôbo}
\région{GOs}
\end{entrée}

\begin{entrée}
{disette ; famine}
\vedette{chavwi}
\région{GOs}
\variante{%
\vedette{cabwi}
\région{PA}}
\end{entrée}

\begin{entrée}
{dur (taro, igname et manioc)}
\classe{v}
\vedette{cou}
\sens{1}
\région{GOs PA WEM WE}
\end{entrée}

\begin{entrée}
{faire manger}
\vedette{pa-hovwo-ni}
\région{GOs}
\end{entrée}

\begin{entrée}
{farine}
\vedette{fari}
\région{GOs}
\end{entrée}

\begin{entrée}
{farine (lit. poudre pain)}
\vedette{draa-phwalawa}
\région{GOs}
\end{entrée}

\begin{entrée}
{gateau de taro [Corne]}
\vedette{poxabu}
\région{BO}
\end{entrée}

\begin{entrée}
{glâner}
\vedette{zala}\homonyme{1}
\région{GOs PA}
\variante{%
\vedette{yala}
\région{BO}}
\end{entrée}

\begin{entrée}
{gourmand ; vorace}
\vedette{a-hovwo}
\région{GOs PA}
\end{entrée}

\begin{entrée}
{goûter}
\vedette{zaxòe}
\sens{1}
\région{GOsPA}
\variante{%
\vedette{zhaxòe}
\région{GA}}
\variante{%
\vedette{zakòe}
\région{GO(s)}}
\variante{%
\vedette{yaxòe, yagoe}
\région{BO}}
\end{entrée}

\begin{entrée}
{graisse (de tortue uniquement)}
\vedette{mõnõ}\homonyme{2}
\région{GOs}
\variante{%
\vedette{mõnô, mwõnô}
\région{BO}}
\end{entrée}

\begin{entrée}
{graisse; huile}
\vedette{gèrè}
\région{GOs PA BO}
\end{entrée}

\begin{entrée}
{huileux ; graisseux}
\vedette{mõnõ}\homonyme{2}
\région{GOs}
\variante{%
\vedette{mõnô, mwõnô}
\région{BO}}
\end{entrée}

\begin{entrée}
{immangeable (taro et manioc, quand c'est mal cuit)}
\classe{v}
\vedette{cou}
\sens{1}
\région{GOs PA WEM WE}
\end{entrée}

\begin{entrée}
{j'ai envie de le manger}
\vedette{a-vwö-nu kûûni}
\région{GOs}
\end{entrée}

\begin{entrée}
{jus de coco tourné [BO]}
\vedette{waluvwi}
\région{GOs}
\variante{%
\vedette{waluvi}
\région{BO [BM]}}
\end{entrée}

\begin{entrée}
{jus de fruit}
\vedette{we-pò-ce}
\région{GOs}
\end{entrée}

\begin{entrée}
{laisser fondre dans la bouche}
\vedette{kuani}\homonyme{1}
\région{PA}
\end{entrée}

\begin{entrée}
{lait}
\vedette{lè}
\région{GOs}
\end{entrée}

\begin{entrée}
{levain (lit. mère du pain)}
\vedette{ô-phwalawa}
\région{GOs}
\variante{%
\vedette{ô-palawa}
\région{BO}}
\end{entrée}

\begin{entrée}
{mâcher de la canne à sucre}
\vedette{whizi}\homonyme{2}
\région{GOs}
\variante{%
\vedette{wili, whili}
\région{PA BO WE}}
\end{entrée}

\begin{entrée}
{mâcher (de la nourriture pour un bébé)}
\vedette{mãã}\homonyme{1}
\région{GOs PA BO}
\end{entrée}

\begin{entrée}
{mâcher (en écrasant)}
\vedette{caçai}
\région{GOs}
\variante{%
\vedette{cayaai}
\région{PA}}
\end{entrée}

\begin{entrée}
{mâcher (en général)}
\vedette{cayae}
\région{PA}
\variante{%
\vedette{ceyai}
\région{BO [BM]}}
\end{entrée}

\begin{entrée}
{mâcher ; mastiquer (des fibres de magnania par ex.)}
\vedette{mhããni}
\région{PA BO}
\end{entrée}

\begin{entrée}
{mâcher (pour extraire le jus) ; mastiquer (des fibres, l'écorce du bourao, du magnania)}
\vedette{biije}
\région{GOs PA BO}
\end{entrée}

\begin{entrée}
{maigre ; non-gras (viande, poisson) [Corne]}
\vedette{zaadu}
\région{GOs BO}
\end{entrée}

\begin{entrée}
{manger (canne à sucre)}
\vedette{whizi}\homonyme{2}
\région{GOs}
\variante{%
\vedette{wili, whili}
\région{PA BO WE}}
\end{entrée}

\begin{entrée}
{manger de la canne à sucre}
\vedette{wha ê}
\région{GOs}
\end{entrée}

\begin{entrée}
{manger (de la viande, du coco)}
\classe{v ; n}
\vedette{huu}\homonyme{1}
\sens{1}
\région{GOs PA BO}
\variante{%
\vedette{whuu}
\région{GO(s)}}
\end{entrée}

\begin{entrée}
{manger (des féculents)}
\vedette{cèni}
\région{GOs}
\variante{%
\vedette{cani}
\région{BO PA}}
\end{entrée}

\begin{entrée}
{manger (des fruits, feuilles)}
\vedette{kûûńi}\homonyme{1}
\région{GOs PA}
\variante{%
\vedette{kôôni}
\région{BO}}
\end{entrée}

\begin{entrée}
{manger (des végétaux, fruits)}
\classe{v}
\vedette{kûû}
\sens{2}
\région{GOs BO}
\end{entrée}

\begin{entrée}
{manger du non solide (bouillie, purée, soupe)}
\vedette{phòzo}
\région{GOs}
\variante{%
\vedette{phòlo}
\région{PA BO}}
\end{entrée}

\begin{entrée}
{manger (en général)}
\classe{v}
\vedette{hu-vo}\homonyme{1}
\sens{1}
\région{PA}
\variante{%
\vedette{hu-po}
\région{PA}}
\end{entrée}

\begin{entrée}
{manger (féculents)}
\vedette{cani}
\région{BO}
\end{entrée}

\begin{entrée}
{manger (fruits) ; croquer (fruits, légumes verts)}
\vedette{kû}\homonyme{1}
\région{BO}
\end{entrée}

\begin{entrée}
{manger (générique) ; nourriture (générique)}
\classe{v ; n}
\vedette{hovwo}
\sens{1}
\région{GOs}
\variante{%
\vedette{hovho}
\région{PA BO}}
\variante{%
\vedette{hopo}
\région{GO vx}}
\end{entrée}

\begin{entrée}
{manger goulûment, trop vite}
\vedette{yaloxa}
\région{GOs}
\end{entrée}

\begin{entrée}
{manger (la canne à sucre)}
\vedette{wili}\homonyme{1}
\région{PA BO}
\variante{%
\vedette{whizi}
\région{GO}}
\end{entrée}

\begin{entrée}
{manger (la canne à sucre)}
\vedette{wha}\homonyme{3}
\région{GOs}
\variante{%
\vedette{whal}
\région{PA}}
\end{entrée}

\begin{entrée}
{manger (respectueux, en parlant d'un chef)}
\classe{v}
\vedette{bwaçu}
\sens{1}
\région{GOs}
\variante{%
\vedette{bwaju}
\région{WEM WE}}
\variante{%
\vedette{bwayu}
\région{PA}}
\end{entrée}

\begin{entrée}
{manger sans dents [Corne]}
\vedette{hûbale}
\région{PA BO}
\end{entrée}

\begin{entrée}
{manger sans rien laisser aux autres}
\vedette{jago}
\région{PA}
\end{entrée}

\begin{entrée}
{manger vert et cru (des fruits)}
\vedette{thraxilo}
\région{GOs}
\variante{%
\vedette{thaxilò}
\région{PA}}
\variante{%
\vedette{taxilò}
\région{BO}}
\end{entrée}

\begin{entrée}
{miel}
\vedette{we ãmu}
\région{BOPA}
\end{entrée}

\begin{entrée}
{miel}
\vedette{we-mebo, mebo}
\région{GOs}
\end{entrée}

\begin{entrée}
{miel en rayon}
\vedette{pi-ãmu}
\région{BO PA}
\end{entrée}

\begin{entrée}
{miettes de nourriture ; reliefs de nourriture}
\vedette{maja-hovwo}
\région{GOs}
\end{entrée}

\begin{entrée}
{miettes ; résidus}
\vedette{maja}\homonyme{2}
\région{GOs}
\variante{%
\vedette{maya}
\région{BO}}
\end{entrée}

\begin{entrée}
{moisi (nourriture)}
\vedette{mhodro}
\région{GOs}
\variante{%
\vedette{mhòdo}
\région{BO}}
\end{entrée}

\begin{entrée}
{non comestible}
\vedette{kavwa hovwo}
\région{GOs}
\end{entrée}

\begin{entrée}
{pain}
\vedette{pã}
\région{WE}
\end{entrée}

\begin{entrée}
{pain}
\vedette{phwalawa}
\région{GOs}
\variante{%
\vedette{pwalaa, palawa}
\région{PA}}
\end{entrée}

\begin{entrée}
{part de canne à sucre}
\vedette{wha}\homonyme{3}
\région{GOs}
\variante{%
\vedette{whal}
\région{PA}}
\end{entrée}

\begin{entrée}
{pomme [BM]}
\vedette{cuka}
\région{BO [BM, Corne]}
\end{entrée}

\begin{entrée}
{pomme de terre}
\vedette{pomitee}
\région{GOs}
\variante{%
\vedette{'pomtee}
\région{GO(s)}}
\end{entrée}

\begin{entrée}
{pourri (être) ; sentir}
\vedette{bwo}\homonyme{1}
\région{GOs BO}
\variante{%
\vedette{bo}
\région{PA BO}}
\end{entrée}

\begin{entrée}
{préparation de manioc, de banane rapée}
\vedette{mwetre}
\région{GOs}
\variante{%
\vedette{mwata}
\région{BO PA}}
\end{entrée}

\begin{entrée}
{provisions de route (froid)}
\vedette{mõã}
\région{GOs WEM}
\variante{%
\vedette{mhõ}
\région{PA WEM}}
\variante{%
\vedette{mõ}
\région{BO}}
\end{entrée}

\begin{entrée}
{rassasié (être)}
\vedette{khôôme}
\sens{1}
\région{GOs PA}
\end{entrée}

\begin{entrée}
{restes de nourriture}
\vedette{mõã}
\région{GOs WEM}
\variante{%
\vedette{mhõ}
\région{PA WEM}}
\variante{%
\vedette{mõ}
\région{BO}}
\end{entrée}

\begin{entrée}
{restes de nourriture}
\vedette{ngãmãã}
\région{GOs}
\variante{%
\vedette{ngamããm}
\région{PA BO [BM]}}
\end{entrée}

\begin{entrée}
{riz}
\vedette{aari}
\région{PA}
\variante{%
\vedette{hari}
\région{PA}}
\end{entrée}

\begin{entrée}
{riz}
\vedette{lai}
\région{GOs}
\end{entrée}

\begin{entrée}
{ronger}
\classe{v}
\vedette{hu-vo}\homonyme{1}
\sens{1}
\région{PA}
\variante{%
\vedette{hu-po}
\région{PA}}
\end{entrée}

\begin{entrée}
{salade}
\vedette{salaa}
\région{GOs}
\end{entrée}

\begin{entrée}
{sel}
\classe{nom}
\vedette{õn}
\sens{1}
\région{PA BO}
\end{entrée}

\begin{entrée}
{sel}
\vedette{zanyi}
\sens{1}
\région{GOs}
\variante{%
\vedette{zhanyi}
\région{GA}}
\end{entrée}

\begin{entrée}
{sirop}
\vedette{siro}
\région{GOs}
\end{entrée}

\begin{entrée}
{sucre}
\vedette{cuk}
\région{PA BO}
\end{entrée}

\begin{entrée}
{sucre (lit. jus de canne à sucre)}
\vedette{we-ê}
\région{GOs}
\variante{%
\vedette{we-èm}
\région{WE}}
\variante{%
\vedette{cuk}
\région{PA}}
\end{entrée}

\begin{entrée}
{thé}
\vedette{tri}
\région{GOs}
\end{entrée}

\begin{entrée}
{tomate}
\vedette{tromwa}
\région{GOs}
\variante{%
\vedette{toma}
\région{GO(s)}}
\end{entrée}

\begin{entrée}
{vert (fruit) ; pas mûr}
\vedette{bu}\homonyme{1}
\région{GOs PA BO}
\end{entrée}

\begin{entrée}
{viande rouge}
\vedette{layô}
\région{GOs}
\variante{%
\vedette{laviã}
\région{WE}}
\end{entrée}

\begin{entrée}
{vin (lit. eau rouge)}
\vedette{we mii}
\région{GOs}
\end{entrée}

\subsubsection{Goût des aliments}

\begin{entrée}
{acide}
\vedette{mweling}
\région{PA BO}
\end{entrée}

\begin{entrée}
{acide ; amer}
\vedette{ngãngi}
\région{GOs}
\end{entrée}

\begin{entrée}
{acide [BO]}
\vedette{ne-raa}
\région{GOs BO}
\end{entrée}

\begin{entrée}
{doux ; sucré}
\vedette{nèm}
\région{PA}
\end{entrée}

\begin{entrée}
{goût ; saveur}
\vedette{neme}
\région{GOs}
\variante{%
\vedette{nemee-n}
\région{PA BO}}
\end{entrée}

\begin{entrée}
{insipide [BO]}
\vedette{nèm}
\région{PA}
\end{entrée}

\begin{entrée}
{mauvais au goût}
\vedette{ne-raa}
\région{GOs BO}
\end{entrée}

\begin{entrée}
{salé (cuisine)}
\vedette{mee}
\région{GOs}
\variante{%
\vedette{mèèn}
\région{PA}}
\end{entrée}

\begin{entrée}
{salé ; trop salé}
\vedette{za}\homonyme{2}
\région{GOs PA}
\variante{%
\vedette{zha}
\région{GA}}
\variante{%
\vedette{ya}
\région{BO}}
\end{entrée}

\begin{entrée}
{sucré ; bon au goût ; succulent}
\vedette{nè-zo}
\région{GOs}
\end{entrée}

\subsubsection{Tabac, actions liées au tabac}

\begin{entrée}
{chiquer (tabac) [BO]}
\classe{v ; n}
\vedette{hovwo}
\sens{2}
\région{GOs}
\variante{%
\vedette{hovho}
\région{PA BO}}
\variante{%
\vedette{hopo}
\région{GO vx}}
\end{entrée}

\begin{entrée}
{cigarette}
\vedette{pwai}
\région{GOs}
\end{entrée}

\begin{entrée}
{fumer (tabac)}
\vedette{u-pwai}
\région{GOs}
\variante{%
\vedette{u-pwaim}
\région{WEM}}
\variante{%
\vedette{pwai}
\région{PA}}
\end{entrée}

\begin{entrée}
{fumer (tabac) [BO]}
\classe{v ; n}
\vedette{hovwo}
\sens{2}
\région{GOs}
\variante{%
\vedette{hovho}
\région{PA BO}}
\variante{%
\vedette{hopo}
\région{GO vx}}
\end{entrée}

\begin{entrée}
{pipe}
\vedette{pwai-ce}
\région{GOs}
\end{entrée}

\begin{entrée}
{tabac ; cigarette}
\vedette{travwa}
\région{GOs}
\variante{%
\vedette{trapa}
\région{GO(s)}}
\variante{%
\vedette{tavang}
\région{PA}}
\end{entrée}

\subsection{Tressage (nattes, paniers), cordes, noeuds, paquets}

\subsubsection{Tressage}

\begin{entrée}
{assouplir les fibres en les lissant}
\classe{v}
\vedette{uzi}
\sens{1}
\région{GOs}
\variante{%
\vedette{uli}
\région{PA WE BO}}
\end{entrée}

\begin{entrée}
{couper en lanières (des fibres de pandanus pour le tressage)}
\classe{v}
\vedette{töö}\homonyme{1}
\sens{1}
\région{GOs BO}
\end{entrée}

\begin{entrée}
{feuille de pandanus sèche pour le tressage (dont on a enlevé les piquants)}
\vedette{pho}
\région{GOs BO}
\end{entrée}

\begin{entrée}
{feuilles de pandanus de creek}
\vedette{mazii}
\région{GOs}
\end{entrée}

\begin{entrée}
{fibres}
\vedette{nuu-phò}
\région{GOs}
\end{entrée}

\begin{entrée}
{fibres longues ; lanières}
\classe{nom}
\vedette{nu}\homonyme{3}
\sens{1}
\région{GOs PA}
\end{entrée}

\begin{entrée}
{lanières de pandanus (pour le tressage)}
\vedette{nuu-phò}
\région{GOs}
\end{entrée}

\begin{entrée}
{limite du tressage (là où on s'est arrêté)}
\vedette{bala-pho}
\région{GOs}
\end{entrée}

\begin{entrée}
{lisser (la tige)}
\classe{v}
\vedette{uzi}
\sens{1}
\région{GOs}
\variante{%
\vedette{uli}
\région{PA WE BO}}
\end{entrée}

\begin{entrée}
{natte (faire une)}
\classe{v ; n}
\vedette{pai}\homonyme{1}
\sens{1}
\région{GOs BO}
\variante{%
\vedette{pae}
\région{GO(s)}}
\variante{%
\vedette{pele}
\région{PA}}
\end{entrée}

\begin{entrée}
{tresse}
\classe{v ; n}
\vedette{pai}\homonyme{1}
\sens{1}
\région{GOs BO}
\variante{%
\vedette{pae}
\région{GO(s)}}
\variante{%
\vedette{pele}
\région{PA}}
\end{entrée}

\begin{entrée}
{tresser}
\vedette{pa}\homonyme{1}
\région{GOs PA}
\end{entrée}

\begin{entrée}
{tresser (corde) ; corde tressée}
\vedette{pitre}
\région{GOs}
\variante{%
\vedette{pire}
\région{BO}}
\end{entrée}

\begin{entrée}
{tresser ; faire une tresse}
\classe{v ; n}
\vedette{pai}\homonyme{1}
\sens{1}
\région{GOs BO}
\variante{%
\vedette{pae}
\région{GO(s)}}
\variante{%
\vedette{pele}
\région{PA}}
\end{entrée}

\begin{entrée}
{tresser ; tressage ; faire de la vannerie}
\vedette{pa-pho}
\région{GOs PA}
\end{entrée}

\begin{entrée}
{tresser une natte ; faire une natte}
\vedette{pa-thrô}
\région{GOs}
\end{entrée}

\begin{entrée}
{tresses de fibres de pandanus}
\classe{nom}
\vedette{nu}\homonyme{3}
\sens{1}
\région{GOs PA}
\end{entrée}

\subsubsection{Noeuds}

\begin{entrée}
{noeud}
\vedette{mhõ}\homonyme{3}
\région{GOs}
\variante{%
\vedette{mhwòl}
\région{BO}}
\end{entrée}

\begin{entrée}
{noeud coulant}
\vedette{mhõ-zixe}
\région{GOs}
\end{entrée}

\begin{entrée}
{noeud (y compris celui des filets)}
\vedette{mhenõ-mhõ}
\région{GOs}
\variante{%
\vedette{mhenõ-mhõng}
\région{PA BO}}
\end{entrée}

\begin{entrée}
{nouer ; attacher avec un noeud ; faire un noeud}
\classe{v ; n}
\vedette{mhõge}
\sens{1}
\région{GO PA BO}
\end{entrée}

\begin{entrée}
{nouer ; noeud (du filet)}
\classe{n ; v}
\vedette{mâge}
\sens{1}
\région{BO}
\variante{%
\vedette{mhâge}}
\end{entrée}

\begin{entrée}
{pli ; plisser}
\vedette{mhõ}\homonyme{3}
\région{GOs}
\variante{%
\vedette{mhwòl}
\région{BO}}
\end{entrée}

\subsubsection{Nattes}

\begin{entrée}
{natte (de palmes de cocotier tressées)}
\vedette{benõõ}
\région{PA BO WEM}
\end{entrée}

\begin{entrée}
{natte (de pandanus) ; noeud}
\vedette{thrô}
\région{GOs}
\région{PA BO}
\variante{%
\vedette{thôm}}
\end{entrée}

\begin{entrée}
{natte (faite avec deux demi palmes de cocotier pour couvrir le toit)}
\vedette{pola}\homonyme{1}
\région{PA BO}
\variante{%
\vedette{pwola}
\région{BO}}
\variante{%
\vedette{thrale}
\région{GO(s)}}
\end{entrée}

\begin{entrée}
{natte (faite de deux palmes de cocotier tressées et posées par terre)}
\vedette{thrale}
\région{GOs}
\variante{%
\vedette{thalei}
\région{PA}}
\end{entrée}

\begin{entrée}
{natte-manteau à longues pailles [Corne]}
\vedette{õ-thõm}
\région{BO}
\end{entrée}

\subsubsection{Paniers}

\begin{entrée}
{giberne de fronde}
\vedette{ke-paa}
\région{GOs BO}
\variante{%
\vedette{ke-paa}
\région{PA}}
\end{entrée}

\begin{entrée}
{panier}
\vedette{ke}
\région{GOs}
\variante{%
\vedette{keel}
\région{PA BO}}
\end{entrée}

\begin{entrée}
{panier}
\vedette{keala}
\région{GOs PA WEM}
\variante{%
\vedette{kevala}
\région{WEM WE}}
\end{entrée}

\begin{entrée}
{panier à anse [Corne]}
\vedette{kevaho}
\région{BO}
\end{entrée}

\begin{entrée}
{panier à monnaie ; enveloppe de monnaie coutumière}
\vedette{ke-tru}
\région{GOs}
\variante{%
\vedette{kèru}
\région{BO}}
\end{entrée}

\begin{entrée}
{panier à richessess ; tirelire}
\vedette{ke-palu}
\région{GOs}
\end{entrée}

\begin{entrée}
{panier (à tressage serré) [Corne]}
\vedette{kehin}
\région{BO}
\end{entrée}

\begin{entrée}
{panier (circulaire en feuille de coco, petit à anses)}
\vedette{keruau}
\région{GOs}
\variante{%
\vedette{ke-truau}
\région{GO(s)}}
\end{entrée}

\begin{entrée}
{panier [Corne]}
\vedette{kiil}
\région{BO}
\end{entrée}

\begin{entrée}
{panier de charge ; panier porté sur le dos (avec une bandoulière ou comme un sac à dos)}
\vedette{ke-thrõõbo}
\région{GOs}
\end{entrée}

\begin{entrée}
{panier de réserves (autour du mur) [BM]}
\vedette{palao}
\région{BO}
\end{entrée}

\begin{entrée}
{panier de restes (de nourriture)}
\vedette{ke-mõã}
\région{GOs WEM}
\end{entrée}

\begin{entrée}
{panier en cocotier (pour porter les ignames)}
\vedette{kelò}
\région{GOs}
\end{entrée}

\begin{entrée}
{panier (en palme de cocotier, rond, plus grand que "keruau")}
\vedette{ke-caadu}
\région{GOs}
\end{entrée}

\begin{entrée}
{panier en pandanus}
\vedette{ke-thal}
\région{PA BO}
\end{entrée}

\begin{entrée}
{panier (grand) en palme de cocotier}
\vedette{kela}
\région{GO}
\variante{%
\vedette{kelo}
\région{GO}}
\end{entrée}

\begin{entrée}
{panier (pour filtrer le dimwa, à tressage lâche)}
\vedette{ke haba}
\région{BO [Corne]}
\end{entrée}

\begin{entrée}
{panier qui contient les dimwa grattées}
\classe{nom}
\vedette{keraa}
\sens{1}
\région{GOs}
\variante{%
\vedette{keraò}
\région{PA}}
\end{entrée}

\begin{entrée}
{panier (tressé en palmes de cocotier)}
\vedette{kerewala}
\région{BO}
\variante{%
\vedette{kerewòla}
\région{BO}}
\end{entrée}

\begin{entrée}
{panier tressé en pandanus (utilisé comme sac à main)}
\vedette{ke-bwaxo}
\région{GOs}
\end{entrée}

\begin{entrée}
{porte-monnaie (claque quand se referme)}
\vedette{ke-thi}
\région{GOs}
\end{entrée}

\begin{entrée}
{préfixe des paniers}
\vedette{ke-}
\région{PA}
\variante{%
\vedette{kee}
\région{PA}}
\end{entrée}

\begin{entrée}
{sac}
\vedette{kecak}
\région{BO [Corne]}
\end{entrée}

\subsubsection{Cordes, cordages}

\begin{entrée}
{bride}
\vedette{bwiri}
\région{GOs}
\variante{%
\vedette{bwirik}
\région{PA BO}}
\end{entrée}

\begin{entrée}
{chaîne (lit. attache dure)}
\vedette{wa-mãgiça}
\région{GOs}
\end{entrée}

\begin{entrée}
{corde à plusieurs brins (faire une)}
\classe{v ; n}
\vedette{pai}\homonyme{1}
\sens{2}
\région{GOs BO}
\variante{%
\vedette{pae}
\région{GO(s)}}
\variante{%
\vedette{pele}
\région{PA}}
\end{entrée}

\begin{entrée}
{corde à torons tordus}
\vedette{bile}
\région{GOs BO}
\variante{%
\vedette{bire}
\région{GOs}}
\end{entrée}

\begin{entrée}
{corde (général) ; ficelle}
\classe{nom}
\vedette{wa}\homonyme{1}
\sens{1}
\région{GOs}
\variante{%
\vedette{wal}
\région{PA BO}}
\variante{%
\vedette{wòl}
\région{WEM}}
\end{entrée}

\begin{entrée}
{corde ; lien ; chaîne}
\vedette{kôô}\homonyme{1}
\région{GOs}
\variante{%
\vedette{kô}
\région{BO PA}}
\end{entrée}

\begin{entrée}
{enrouler}
\vedette{bile}
\région{GOs BO}
\variante{%
\vedette{bire}
\région{GOs}}
\end{entrée}

\begin{entrée}
{faire un toron (roulé sur la cuisse)}
\vedette{wa-bile}
\région{GOs BO PA}
\end{entrée}

\begin{entrée}
{fibre de jeune rejet de'phuleng'}
\classe{nom}
\vedette{alamwi}
\sens{2}
\région{BO PA}
\end{entrée}

\begin{entrée}
{filer}
\vedette{bile}
\région{GOs BO}
\variante{%
\vedette{bire}
\région{GOs}}
\end{entrée}

\begin{entrée}
{lanière du cheval}
\vedette{kô-chòvwa}
\région{GOs}
\end{entrée}

\begin{entrée}
{lasso (pour attraper un cheval)}
\vedette{phwè-wõõ}
\région{GOs}
\end{entrée}

\begin{entrée}
{liane ; corde ; courroie ; longe}
\vedette{khô}\homonyme{1}
\région{GOsPA}
\variante{%
\vedette{khô}
\région{BO}}
\end{entrée}

\begin{entrée}
{liane ; lien}
\classe{nom}
\vedette{wa}\homonyme{1}
\sens{1}
\région{GOs}
\variante{%
\vedette{wal}
\région{PA BO}}
\variante{%
\vedette{wòl}
\région{WEM}}
\end{entrée}

\begin{entrée}
{lien (lit. qui sert à attacher)}
\vedette{ba-nhõî}
\région{GOs PA}
\end{entrée}

\begin{entrée}
{noeud coulant}
\vedette{phwè-wõõ}
\région{GOs}
\end{entrée}

\begin{entrée}
{rênes}
\vedette{khô-bwiri}
\région{GOs}
\end{entrée}

\begin{entrée}
{rouler (un toron de corde sur la cuisse)}
\vedette{bile}
\région{GOs BO}
\variante{%
\vedette{bire}
\région{GOs}}
\end{entrée}

\begin{entrée}
{toron de corde}
\vedette{tagi}
\sens{1}
\région{PA BO}
\end{entrée}

\begin{entrée}
{toron (roulé sur la cuisse) ; fil de filet}
\vedette{wa-bile}
\région{GOs BO PA}
\end{entrée}

\begin{entrée}
{torsader, tresser une corde à plusieurs brins}
\classe{v ; n}
\vedette{pai}\homonyme{1}
\sens{2}
\région{GOs BO}
\variante{%
\vedette{pae}
\région{GO(s)}}
\variante{%
\vedette{pele}
\région{PA}}
\end{entrée}

\subsubsection{Couture}

\begin{entrée}
{aiguille (lit. os de roussette')}
\classe{nom}
\vedette{du-bwò}
\sens{1}
\région{GOs PA BO}
\variante{%
\vedette{duu-bò}
\région{BO}}
\end{entrée}

\begin{entrée}
{coudre}
\vedette{thibi}
\région{GOs}
\end{entrée}

\begin{entrée}
{coudre}
\vedette{thige}
\région{PA WE}
\variante{%
\vedette{thege}
\région{BO [BM]}}
\end{entrée}

\subsection{Bois et travail du bois, outils}

\subsubsection{Bois}

\begin{entrée}
{bûche (grosse, pour la nuit, portée par les hommes)}
\vedette{ce-bò}
\région{GOs}
\variante{%
\vedette{ce-bòn}
\région{WEM BO}}
\variante{%
\vedette{ci-bòn}
\région{PA}}
\end{entrée}

\begin{entrée}
{coeur de bois dur (par ex. de gaiac, bois de fer, bois pétrole)}
\vedette{jamali}
\groupe{A}
\région{GOs WEM BO PA}
\end{entrée}

\begin{entrée}
{copeaux de bois}
\vedette{jaa-ce}
\région{GOs}
\end{entrée}

\begin{entrée}
{copeaux de bois}
\vedette{maja-ce}
\région{GOs}
\variante{%
\vedette{ja-ce}
\région{GO(s)}}
\variante{%
\vedette{maya-ce, meya-ce}
\région{BO}}
\end{entrée}

\begin{entrée}
{écharde}
\vedette{nuu-ce}
\région{GOs PA}
\end{entrée}

\begin{entrée}
{fagot de bois}
\vedette{phò-ce}
\région{GOs}
\end{entrée}

\begin{entrée}
{planche}
\vedette{hõõ-ce}
\région{PA}
\end{entrée}

\begin{entrée}
{planche}
\vedette{hõxa-ce}
\région{GOs}
\variante{%
\vedette{hõõ-ce}
\région{PA}}
\end{entrée}

\begin{entrée}
{sciure}
\vedette{maja-ce}
\région{GOs}
\variante{%
\vedette{ja-ce}
\région{GO(s)}}
\variante{%
\vedette{maya-ce, meya-ce}
\région{BO}}
\end{entrée}

\subsubsection{Travail bois}

\begin{entrée}
{écorcer}
\vedette{hi}\homonyme{1}
\région{PA}
\end{entrée}

\begin{entrée}
{écorcer(niaouli)}
\classe{v.t.}
\vedette{hili}\homonyme{1}
\sens{1}
\région{GOs BO}
\end{entrée}

\begin{entrée}
{raboter ; lisser}
\classe{v}
\vedette{uzi}
\sens{2}
\région{GOs}
\variante{%
\vedette{uli}
\région{PA WE BO}}
\end{entrée}

\begin{entrée}
{scier}
\classe{v.t.}
\vedette{zòi}
\sens{1}
\région{GOs PA BO}
\variante{%
\vedette{zhòi}
\région{GO(s)}}
\end{entrée}

\begin{entrée}
{sculpter}
\classe{v}
\vedette{threi}
\sens{2}
\région{GOs WEM}
\variante{%
\vedette{thei, thèi}
\région{BO PA}}
\end{entrée}

\begin{entrée}
{sculpteur}
\vedette{a-thu drogò}
\région{GOs}
\end{entrée}

\begin{entrée}
{tailler du bois}
\classe{v}
\vedette{threi}
\sens{2}
\région{GOs WEM}
\variante{%
\vedette{thei, thèi}
\région{BO PA}}
\end{entrée}

\subsection{Outils, instruments, matériaux, pont}

\subsubsection{Outils}

\begin{entrée}
{bout de verre utilisé pour couper}
\vedette{ne ńe}\homonyme{1}
\région{GOs}
\end{entrée}

\begin{entrée}
{caler le manche}
\vedette{thu hoo}
\région{GOs PA}
\variante{%
\vedette{nee-vwo hoo-n}
\région{PA}}
\end{entrée}

\begin{entrée}
{clou}
\vedette{drovwiju}
\région{GOs}
\variante{%
\vedette{dopicu, dovio, duvio}
\région{BO PA}}
\end{entrée}

\begin{entrée}
{clouer}
\vedette{tabila}
\région{BO PA WE}
\end{entrée}

\begin{entrée}
{clouer ; fixer (avec un clou)}
\vedette{thaeza}
\région{GOs}
\variante{%
\vedette{thaila}
\région{GO}}
\variante{%
\vedette{taela}
\région{WE}}
\end{entrée}

\begin{entrée}
{coin (pour caler ; lit. nourriture)}
\vedette{hoo}\homonyme{2}
\région{GOs PA}
\end{entrée}

\begin{entrée}
{couper}
\classe{v ; n}
\vedette{hèlè}
\sens{1}
\région{GO PA BO}
\end{entrée}

\begin{entrée}
{couteau}
\classe{v ; n}
\vedette{hèlè}
\sens{1}
\région{GO PA BO}
\end{entrée}

\begin{entrée}
{couteau de bambou ; couteau à subincision}
\classe{nom}
\vedette{gò}\homonyme{1}
\sens{2}
\région{GOs BO PA}
\end{entrée}

\begin{entrée}
{fermer à clé}
\vedette{kilee-ni}
\région{GOs}
\end{entrée}

\begin{entrée}
{fourche (trident) [GOs]}
\classe{v ; n}
\vedette{zaro}
\sens{1}
\région{GOs PA}
\variante{%
\vedette{zharo}
\région{GA}}
\variante{%
\vedette{zaatro}
\région{vx (Haudricourt)}}
\variante{%
\vedette{yaro}
\région{BO}}
\end{entrée}

\begin{entrée}
{hache}
\classe{nom}
\vedette{wamòn}
\sens{1}
\région{PA WEH BO}
\variante{%
\vedette{wamwa, wamò}
\région{GO(s)}}
\end{entrée}

\begin{entrée}
{hache}
\classe{nom}
\vedette{wamwa}
\sens{1}
\région{GOs WEM}
\variante{%
\vedette{wamòn}
\région{PA}}
\end{entrée}

\begin{entrée}
{hache à double tranchant}
\classe{nom}
\vedette{mãgi}\homonyme{2}
\sens{2}
\région{GOs}
\variante{%
\vedette{mwãgi}
\région{PA BO}}
\end{entrée}

\begin{entrée}
{hache (petite, en fer)}
\vedette{tröxi}
\région{GOsWE}
\variante{%
\vedette{tööxi}
\région{PA Paita}}
\end{entrée}

\begin{entrée}
{hache (type de)}
\vedette{kaamuda}
\région{PA}
\end{entrée}

\begin{entrée}
{herminette}
\vedette{thrugò}
\région{GOs}
\variante{%
\vedette{thugò}
\région{PA}}
\end{entrée}

\begin{entrée}
{herminette}
\classe{nom}
\vedette{wamòn}
\sens{2}
\région{PA WEH BO}
\variante{%
\vedette{wamwa, wamò}
\région{GO(s)}}
\end{entrée}

\begin{entrée}
{herminette}
\classe{nom}
\vedette{wamwa}
\sens{2}
\région{GOs WEM}
\variante{%
\vedette{wamòn}
\région{PA}}
\end{entrée}

\begin{entrée}
{lame de rasoir}
\vedette{me-hele ba-pe-thra}
\région{GOs}
\end{entrée}

\begin{entrée}
{lame du couteau}
\vedette{me-hele}
\région{GOs}
\end{entrée}

\begin{entrée}
{manche}
\classe{nom}
\vedette{kò}\homonyme{3}
\sens{2}
\région{GOs BO PA}
\end{entrée}

\begin{entrée}
{manche de hache}
\vedette{kòò-wamwa}
\région{GOs}
\variante{%
\vedette{kòò-wamon}
\région{BO}}
\end{entrée}

\begin{entrée}
{marteau ; instrument pour taper}
\vedette{ba-cabi}
\région{GOs PA}
\variante{%
\vedette{marto}
\région{GOs}}
\end{entrée}

\begin{entrée}
{pelle à fouir les ignames (en bois ou fer) ; bêche [PA, BO]}
\classe{v ; n}
\vedette{zaro}
\sens{2}
\région{GOs PA}
\variante{%
\vedette{zharo}
\région{GA}}
\variante{%
\vedette{zaatro}
\région{vx (Haudricourt)}}
\variante{%
\vedette{yaro}
\région{BO}}
\end{entrée}

\begin{entrée}
{perceuse}
\vedette{ba-uxi-cee}
\région{GOs}
\end{entrée}

\begin{entrée}
{pioche}
\vedette{piòò}
\région{GOs}
\variante{%
\vedette{piyòc}
\région{BO}}
\end{entrée}

\begin{entrée}
{sabre d'abatis}
\vedette{hèlè kawali}
\région{GOs}
\variante{%
\vedette{hèlè hã}
\région{WE}}
\end{entrée}

\begin{entrée}
{scie}
\vedette{ba-zo cee}
\région{GOs PA}
\variante{%
\vedette{ba-yo}
\région{BO}}
\end{entrée}

\begin{entrée}
{scie(r)}
\classe{v ; n}
\vedette{hèlè}
\sens{1}
\région{GO PA BO}
\end{entrée}

\begin{entrée}
{tamioc}
\classe{nom}
\vedette{wamòn}
\sens{1}
\région{PA WEH BO}
\variante{%
\vedette{wamwa, wamò}
\région{GO(s)}}
\end{entrée}

\begin{entrée}
{tamioc}
\classe{nom}
\vedette{wamwa}
\sens{1}
\région{GOs WEM}
\variante{%
\vedette{wamòn}
\région{PA}}
\end{entrée}

\begin{entrée}
{tamioc ; fer}
\vedette{tröxi}
\région{GOsWE}
\variante{%
\vedette{tööxi}
\région{PA Paita}}
\end{entrée}

\subsubsection{Instruments et ponts}

\paragraph{Instruments}

\begin{entrée}
{baguette ; canne [BO Corne]}
\classe{nom}
\vedette{haaxa}
\sens{1}
\région{BO}
\end{entrée}

\begin{entrée}
{balayer ; balai}
\vedette{bale}
\région{GOs}
\end{entrée}

\begin{entrée}
{bâton à glu (enduit de colle de fruit du gommier, utilisé pour attraper les cigales en leur collant les ailes)}
\classe{v ; n}
\vedette{tigi}\homonyme{2}
\sens{2}
\région{GOs PA BO}
\end{entrée}

\begin{entrée}
{bâton à glu (sur lequel on colle les cigales)}
\vedette{ba-thi-halelewa}
\région{GO PA BO}
\end{entrée}

\begin{entrée}
{bâton ; canne (pour marcher ; symbole d'ancienneté)}
\vedette{hêgo}
\région{GOs PA BO}
\end{entrée}

\begin{entrée}
{bâton pour attraper les cigales (avec la glu du gommier)}
\vedette{ce ba-thi halelewa}
\région{GOs}
\end{entrée}

\begin{entrée}
{battoir pour écorce}
\vedette{mètrô}
\région{GO}
\end{entrée}

\begin{entrée}
{chaîne}
\vedette{bwaxuli}
\région{WEM WE}
\end{entrée}

\begin{entrée}
{ciseaux}
\vedette{ba-còòxe}
\région{GOs}
\end{entrée}

\begin{entrée}
{ciseaux}
\vedette{chiçô}
\région{GOs}
\variante{%
\vedette{cicô}
\région{PA}}
\end{entrée}

\begin{entrée}
{clé}
\vedette{kilè}
\région{GOs}
\end{entrée}

\begin{entrée}
{cloche}
\vedette{kilooc}
\région{PA}
\end{entrée}

\begin{entrée}
{contenant à liquide ; calebasse}
\classe{nom}
\vedette{mõ-we}
\sens{1}
\région{GOs}
\end{entrée}

\begin{entrée}
{couteau pour igname (Dubois)}
\classe{nom}
\vedette{diia}
\sens{2}
\région{PA BO}
\variante{%
\vedette{diva}
\région{BO}}
\end{entrée}

\begin{entrée}
{échelle}
\vedette{ba-kha-da}
\région{PA}
\end{entrée}

\begin{entrée}
{étui ; fourreau de couteau ; manche}
\vedette{mõ-hèlè}
\région{GOs BO PA}
\end{entrée}

\begin{entrée}
{éventail (en feuille de cocotier pour le feu)}
\vedette{ba-u}
\région{GOs}
\variante{%
\vedette{ba-ul}
\région{WEM WE BO PA}}
\end{entrée}

\begin{entrée}
{fermeture ; couvercle}
\vedette{kivwa}
\région{GOs}
\variante{%
\vedette{kipa}
\région{GO}}
\variante{%
\vedette{kivha}
\région{PA BO}}
\end{entrée}

\begin{entrée}
{fil de fer (barbelé)}
\vedette{vhaò}
\région{GOs}
\variante{%
\vedette{fhaò}
\région{GA}}
\end{entrée}

\begin{entrée}
{fouet}
\vedette{phue}
\région{GOs}
\end{entrée}

\begin{entrée}
{fourchette}
\classe{nom}
\vedette{de}\homonyme{1}
\sens{1}
\région{GOs BO PA}
\end{entrée}

\begin{entrée}
{instrument en paille pour gratter dans l'eau les bulbilles de "dimwa" (Dubois)}
\vedette{kerao}\homonyme{2}
\région{GOs BO}
\end{entrée}

\begin{entrée}
{lime ; pierre à aiguiser [BM]}
\vedette{phe}\homonyme{3}
\région{BO}
\end{entrée}

\begin{entrée}
{machine}
\vedette{masi}
\région{GOs}
\end{entrée}

\begin{entrée}
{maillet à poisson}
\vedette{bwatratra}
\région{GOs}
\end{entrée}

\begin{entrée}
{miroir}
\vedette{we-bwaxixi}
\région{GOs}
\variante{%
\vedette{we-bwaxii}
\région{GO(s)}}
\end{entrée}

\begin{entrée}
{miroir}
\vedette{we-zhido}
\région{GA}
\variante{%
\vedette{we-zedo}
\région{GO(s) BO}}
\end{entrée}

\begin{entrée}
{mors}
\vedette{pò-bwiri}
\région{GOs}
\end{entrée}

\begin{entrée}
{parapluie}
\classe{nom}
\vedette{mõ-pwal}
\sens{1}
\région{BO}
\end{entrée}

\begin{entrée}
{râpe (faite d'une valve de coquillage, utilisée pour gratter le coco, banane, etc.)}
\classe{nom}
\vedette{pii-gu}
\sens{2}
\région{GOs PA BO}
\end{entrée}

\begin{entrée}
{rasoir}
\vedette{ba-pe-ravhi}
\région{PA}
\end{entrée}

\begin{entrée}
{rasoir}
\classe{nom}
\vedette{ba-pe-thra}
\région{GOs}
\end{entrée}

\begin{entrée}
{roue}
\classe{v ; n}
\vedette{bwaòle}
\sens{2}
\région{PA BO WEM}
\end{entrée}

\begin{entrée}
{roue}
\vedette{dèè}
\région{GOs}
\end{entrée}

\begin{entrée}
{servir à}
\vedette{peven}
\région{PA}
\end{entrée}

\begin{entrée}
{tamis}
\vedette{ba-pase}
\région{GOs}
\end{entrée}

\begin{entrée}
{tisonnier}
\vedette{ba-thiçe}
\région{GOs}
\end{entrée}

\begin{entrée}
{tôle}
\vedette{hõõ-tòl}
\région{PA}
\end{entrée}

\begin{entrée}
{tôle (lit. peau/couverture en tôle)}
\vedette{ci-ciò}
\région{GOs}
\end{entrée}

\begin{entrée}
{volant (voiture)}
\vedette{ba-hã}
\région{GOs}
\end{entrée}

\paragraph{Ponts}

\begin{entrée}
{passerelle ; planche servant de pont pour traverser une rivière (ou posé sur la boue)}
\vedette{ce-baalu}
\région{GOs}
\end{entrée}

\begin{entrée}
{passerelle ; pont (tronc d'arbre ou planche pour traverser une rivière)}
\classe{nom}
\vedette{hovalek}
\sens{1}
\région{PA}
\end{entrée}

\begin{entrée}
{pont en bois}
\vedette{cibaalu}
\région{GOs}
\end{entrée}

\begin{entrée}
{pont en bois (sur une rivière)}
\vedette{bali-cee}
\région{GOs}
\end{entrée}

\section{Individu - société}

\subsection{Cours de la vie}

\begin{entrée}
{adulte (être) ; mature (lit. pierre-os)}
\vedette{paa-du}
\région{PA}
\end{entrée}

\begin{entrée}
{âge}
\classe{nom}
\vedette{kau-}
\sens{2}
\région{GOs BO}
\end{entrée}

\begin{entrée}
{cercueil (lit. panier en bois)}
\vedette{ke-cee}
\région{GOs PA}
\variante{%
\vedette{ke-çee}
\région{GO(s)}}
\end{entrée}

\begin{entrée}
{circoncision ; subincision (moment où on donnait le bagayou au garçon)}
\vedette{tragòò}\homonyme{1}
\région{GOs}
\variante{%
\vedette{taagò}
\région{BO}}
\end{entrée}

\begin{entrée}
{coutumes}
\classe{v ; n}
\vedette{mõlò}
\sens{1}
\région{GOs}
\variante{%
\vedette{mòlò}
\région{PA BO}}
\variante{%
\vedette{mòòlè}
\région{BO}}
\end{entrée}

\begin{entrée}
{dans la force de l'âge (être) ; bonne santé (être en)}
\vedette{pègòò}
\région{GOs WE}
\variante{%
\vedette{pègòm}
\région{BO}}
\end{entrée}

\begin{entrée}
{enterrer}
\vedette{pa-khêmi}
\région{GOs}
\end{entrée}

\begin{entrée}
{enterrer (qqn)}
\classe{v}
\vedette{thozoe}
\sens{3}
\région{GOs}
\variante{%
\vedette{toroe}
\région{PA BO}}
\end{entrée}

\begin{entrée}
{grandir ; vieillir (animés)}
\vedette{wha-mã}
\région{GOs PA BO}
\variante{%
\vedette{hua-mã}
\région{vx}}
\end{entrée}

\begin{entrée}
{jeune ; petit}
\classe{nom}
\vedette{ẽnõ}\homonyme{1}
\sens{1}
\région{GOs}
\variante{%
\vedette{ênõ}
\région{PA BO}}
\end{entrée}

\begin{entrée}
{mort ; mourir}
\classe{v.stat. ; n}
\vedette{mã}\homonyme{1}
\sens{1}
\région{GOs PABO}
\variante{%
\vedette{mhã}}
\end{entrée}

\begin{entrée}
{mort (terme d'évitement et de respect)}
\classe{v ; n}
\vedette{khinu}\homonyme{1}
\sens{2}
\région{GOs PA BO}
\end{entrée}

\begin{entrée}
{naître}
\classe{v}
\vedette{pwe}\homonyme{2}
\sens{2}
\région{GOs}
\end{entrée}

\begin{entrée}
{naître}
\classe{v}
\vedette{thrõbo}
\sens{2}
\région{GOs WEM}
\variante{%
\vedette{thôbo}
\région{BO PA}}
\end{entrée}

\begin{entrée}
{nourrisson [Corne]}
\vedette{pwe-ẽnõ}
\région{BO}
\end{entrée}

\begin{entrée}
{nourrisson ; nouveau-né (lit. enfant qui est très petit)}
\vedette{ẽnõ xa pogabe}
\région{GOs}
\end{entrée}

\begin{entrée}
{nouveau-né [Corne]}
\vedette{yaawe}
\région{BO}
\end{entrée}

\begin{entrée}
{ornement de deuil}
\classe{nom}
\vedette{tãî}
\sens{2}
\région{PA BO}
\end{entrée}

\begin{entrée}
{pendre (se)}
\vedette{we-no}
\région{PA}
\end{entrée}

\begin{entrée}
{personne d'âge moyen, mûr (ni enfant, ni vieillard)}
\vedette{gò-êgu}
\région{GOs}
\end{entrée}

\begin{entrée}
{salut (le) (religion)}
\classe{nom}
\vedette{mõõxi}
\sens{1}
\région{GOs PA}
\variante{%
\vedette{mòòle}
\région{BO}}
\end{entrée}

\begin{entrée}
{tombe}
\vedette{bu-êgu}
\région{PA}
\end{entrée}

\begin{entrée}
{tuer (se) ; suicider (se)}
\vedette{thraliwa}
\région{GOs}
\variante{%
\vedette{thaliang}
\région{PA}}
\end{entrée}

\begin{entrée}
{veuf ; veuve}
\vedette{drapwê}
\région{GOs}
\variante{%
\vedette{damwê}}
\end{entrée}

\begin{entrée}
{vieille-femme}
\vedette{thoimwã}
\région{GOs}
\variante{%
\vedette{toimwa}
\région{PA BO}}
\end{entrée}

\begin{entrée}
{vie (principe de vie)}
\classe{nom}
\vedette{mõõxi}
\sens{1}
\région{GOs PA}
\variante{%
\vedette{mòòle}
\région{BO}}
\end{entrée}

\begin{entrée}
{vieux (les) ; vieil homme ; parents}
\vedette{wha-mã}
\région{GOs PA BO}
\variante{%
\vedette{hua-mã}
\région{vx}}
\end{entrée}

\begin{entrée}
{vivre ; vivant; vie}
\classe{v ; n}
\vedette{mõlò}
\sens{1}
\région{GOs}
\variante{%
\vedette{mòlò}
\région{PA BO}}
\variante{%
\vedette{mòòlè}
\région{BO}}
\end{entrée}

\subsection{Fonctions intellectuelles, sentiments}

\subsubsection{Fonctions intellectuelles}

\begin{entrée}
{accepter ; dire oui}
\classe{INTJ ; v}
\vedette{èlò}
\sens{2}
\région{GO PA BO}
\end{entrée}

\begin{entrée}
{apprendre ; étudier}
\vedette{chomu}
\région{GOs BO}
\variante{%
\vedette{comu}
\région{PA}}
\end{entrée}

\begin{entrée}
{approuver}
\vedette{phachaani}
\région{GOs}
\end{entrée}

\begin{entrée}
{approuver (dire que c'est droit, correct)}
\vedette{pa-ku-gòò-ni}
\région{GOs}
\end{entrée}

\begin{entrée}
{approuver qqn ; acquiescer}
\vedette{pha-gumãgu-ni}
\région{GOs}
\end{entrée}

\begin{entrée}
{attention (faire) à}
\classe{v}
\vedette{pevwö}
\sens{1}
\région{GOs}
\end{entrée}

\begin{entrée}
{comprendre}
\classe{v.t.}
\vedette{trõne}
\sens{3}
\région{GOs}
\variante{%
\vedette{tõne}
\région{BO PA}}
\end{entrée}

\begin{entrée}
{comprendre}
\vedette{trõne kaamweni}
\région{GOs}
\end{entrée}

\begin{entrée}
{comprendre (bien) qqch.}
\vedette{hine-kaamweni}
\région{GOs}
\end{entrée}

\begin{entrée}
{comprendre ; sage}
\vedette{kaamweni}
\région{GOs}
\end{entrée}

\begin{entrée}
{compter ; compter sur qqn ; nombre}
\vedette{phinãã}\homonyme{1}
\région{GOs PA BO}
\end{entrée}

\begin{entrée}
{compter ; nombre [BM]}
\vedette{ju-la}
\région{BO}
\end{entrée}

\begin{entrée}
{confiance (avoir) ; espoir ; espérer}
\vedette{pavwã}
\région{GOs}
\end{entrée}

\begin{entrée}
{connaître (se)}
\vedette{pe-hine}
\région{GOs}
\end{entrée}

\begin{entrée}
{conseiller}
\vedette{phumõ}
\région{GOs}
\variante{%
\vedette{pumõ}
\région{PA}}
\end{entrée}

\begin{entrée}
{croire ; espérer}
\vedette{kããge}
\région{GOs}
\variante{%
\vedette{kããgèn}
\région{BO PA}}
\end{entrée}

\begin{entrée}
{croire tout savoir}
\vedette{ãbe}
\région{GOs}
\end{entrée}

\begin{entrée}
{demander à qqn ; interroger qqn}
\vedette{phaja}
\région{WEM WE PA BO}
\end{entrée}

\begin{entrée}
{demander à qqn ; interroger ; question(ner)}
\vedette{zala}\homonyme{2}
\région{GOs PA}
\variante{%
\vedette{zhala}
\région{GA}}
\end{entrée}

\begin{entrée}
{demander qqch ; demande ; requête}
\vedette{trilòò}
\région{GOs}
\variante{%
\vedette{tilòò}
\région{PA BO}}
\end{entrée}

\begin{entrée}
{demander (se)}
\vedette{phaja}
\région{WEM WE PA BO}
\end{entrée}

\begin{entrée}
{demander un peu qqch.}
\vedette{khi-trilòò}
\région{GOs}
\variante{%
\vedette{ki-tilò, khi-cilo}
\région{PA}}
\variante{%
\vedette{khi-tilòò}
\région{BO}}
\end{entrée}

\begin{entrée}
{dessiner}
\vedette{tii-hênu}
\région{GOs}
\end{entrée}

\begin{entrée}
{dire des bêtises ; dérailler ; faire qqch sans sérieux ; faire des bêtises}
\vedette{döbe}\homonyme{1}
\région{GOs WEM BO}
\variante{%
\vedette{dube}
\région{GA BO PA}}
\end{entrée}

\begin{entrée}
{dire (prière) ; réciter}
\classe{v}
\vedette{höze}
\sens{2}
\région{GOs}
\variante{%
\vedette{hure}
\région{WEM WE PA}}
\variante{%
\vedette{hore}
\région{BO}}
\end{entrée}

\begin{entrée}
{discourir ; faire un discours (coutumier)}
\vedette{phumõ}
\région{GOs}
\variante{%
\vedette{pumõ}
\région{PA}}
\end{entrée}

\begin{entrée}
{distrait (être) ; ne pas faire attention}
\vedette{pe-paree}
\région{PA BO [BM]}
\end{entrée}

\begin{entrée}
{douter; hésiter}
\vedette{pòvwòtru}
\région{GOs}
\variante{%
\vedette{pwòvwòtru}
\région{GO(s)}}
\variante{%
\vedette{pòpòru}
\région{PA}}
\end{entrée}

\begin{entrée}
{droit ; droiture}
\vedette{baaxò}
\groupe{A}
\sens{1}
\région{GOs WEM WE}
\région{BO}
\variante{%
\vedette{baaxòl}}
\variante{%
\vedette{baxòòl}
\région{PA}}
\end{entrée}

\begin{entrée}
{écrire ; écriture}
\vedette{tii}
\région{GOs}
\variante{%
\vedette{tii-n}
\région{PA BO}}
\end{entrée}

\begin{entrée}
{emprunter (lit. demander un peu, pour un moment) [PA]}
\vedette{khi-trilòò}
\région{GOs}
\variante{%
\vedette{ki-tilò, khi-cilo}
\région{PA}}
\variante{%
\vedette{khi-tilòò}
\région{BO}}
\end{entrée}

\begin{entrée}
{enseignant}
\vedette{a-choomu}
\région{GOs}
\end{entrée}

\begin{entrée}
{enseigner}
\vedette{pha-choomu-ni}
\région{GOs}
\end{entrée}

\begin{entrée}
{erreur; tromper (se)}
\vedette{niivwa}
\région{GOs}
\variante{%
\vedette{niivha}
\région{PA BO}}
\variante{%
\vedette{nipa}
\région{vx}}
\end{entrée}

\begin{entrée}
{espoir ; confiance (avoir) ; espérer}
\vedette{tre-pavwã}
\région{GOs}
\variante{%
\vedette{tre-pavhã}
\région{GO(s)BO PA}}
\end{entrée}

\begin{entrée}
{faire se souvenir ; rappeler qqch à qqn}
\vedette{pha-nõnõ}
\région{GOs}
\end{entrée}

\begin{entrée}
{feindre ; faire semblant [Corne]}
\vedette{pe-dodobe}
\région{BO}
\end{entrée}

\begin{entrée}
{ignorant (être)}
\vedette{hivwinevwo}
\région{GOs}
\end{entrée}

\begin{entrée}
{ignorer ; ne pas savoir ; ignorer (s')}
\vedette{hivwine}
\région{GOs}
\variante{%
\vedette{hipine}
\région{GO(s)}}
\variante{%
\vedette{hivine}
\région{BO}}
\end{entrée}

\begin{entrée}
{information, nouvelle [Corne]}
\vedette{paara}
\région{BO}
\end{entrée}

\begin{entrée}
{instruit ; intelligent ; connaissance ; intelligence}
\vedette{hinevwo}
\région{GOs PA}
\end{entrée}

\begin{entrée}
{inventer (chant, histoire) ; créer}
\vedette{koe}\homonyme{2}
\région{GOs}
\end{entrée}

\begin{entrée}
{lettre ; livre}
\vedette{tiivwo}
\région{GOs PA}
\end{entrée}

\begin{entrée}
{lire}
\vedette{phinãã}\homonyme{1}
\région{GOs PA BO}
\end{entrée}

\begin{entrée}
{lire [PA, BO]}
\vedette{chomu}
\région{GOs BO}
\variante{%
\vedette{comu}
\région{PA}}
\end{entrée}

\begin{entrée}
{marquer ; graver}
\vedette{tii}
\région{GOs}
\variante{%
\vedette{tii-n}
\région{PA BO}}
\end{entrée}

\begin{entrée}
{mesurer}
\vedette{ja}\homonyme{3}
\région{GOs}
\variante{%
\vedette{jak}
\région{BO PA}}
\end{entrée}

\begin{entrée}
{mesurer}
\vedette{jak}
\région{BO}
\end{entrée}

\begin{entrée}
{mesurer}
\vedette{jange}
\région{GOs}
\variante{%
\vedette{jaxe}
\région{BO PA}}
\end{entrée}

\begin{entrée}
{mesurer ;}
\vedette{jaxe}
\région{GOs BO}
\end{entrée}

\begin{entrée}
{mesurer (la hauteur avec son corps)}
\vedette{jaaxe}
\région{BO}
\end{entrée}

\begin{entrée}
{ne pas voir (ce qui était évident ; comment a-t-on fait pour ne pas voir)}
\vedette{uâme}
\région{PA}
\end{entrée}

\begin{entrée}
{on ne sait pas}
\vedette{yhaamwa}
\région{GOs BO}
\end{entrée}

\begin{entrée}
{oublier ; pardonner}
\vedette{phòtrõme}
\région{GOs}
\variante{%
\vedette{phòrõme}
\région{PA}}
\variante{%
\vedette{pòrõme}
\région{BO}}
\end{entrée}

\begin{entrée}
{oublier ; pardonner ; pardon}
\vedette{phòtrõ}
\région{GOs}
\variante{%
\vedette{phòrõ}
\région{GO(s)}}
\variante{%
\vedette{pòròm}
\région{PA BO}}
\end{entrée}

\begin{entrée}
{penser à ; réfléchir}
\vedette{tre-nõnõmi}
\région{GOs}
\variante{%
\vedette{tee-nònòmi}
\région{PA BO}}
\end{entrée}

\begin{entrée}
{penser (incertain) ; croire (sans être sûr)}
\vedette{weena}
\région{GOs PA BO}
\end{entrée}

\begin{entrée}
{penser ; rappeler (se) ; rêvasser}
\vedette{nõnõ}
\région{GOs}
\variante{%
\vedette{nõnõm}
\région{BO}}
\end{entrée}

\begin{entrée}
{penser ; souvenir de (se)}
\vedette{nõnõmi}
\région{GOs}
\variante{%
\vedette{nõnõmi}
\région{PA BO}}
\end{entrée}

\begin{entrée}
{peser}
\vedette{ja}\homonyme{3}
\région{GOs}
\variante{%
\vedette{jak}
\région{BO PA}}
\end{entrée}

\begin{entrée}
{peser}
\vedette{jange}
\région{GOs}
\variante{%
\vedette{jaxe}
\région{BO PA}}
\end{entrée}

\begin{entrée}
{peser}
\vedette{jaxe}
\région{GOs BO}
\end{entrée}

\begin{entrée}
{prêcher ; sermonner}
\vedette{phumõ}
\région{GOs}
\variante{%
\vedette{pumõ}
\région{PA}}
\end{entrée}

\begin{entrée}
{prendre qqch. par erreur}
\vedette{pe-thauvweni}
\région{GOs}
\end{entrée}

\begin{entrée}
{prévoir}
\vedette{tre-nõnõmi}
\région{GOs}
\variante{%
\vedette{tee-nònòmi}
\région{PA BO}}
\end{entrée}

\begin{entrée}
{raconter (histoire en suivant bien l'histoire)}
\classe{v}
\vedette{höze}
\sens{2}
\région{GOs}
\variante{%
\vedette{hure}
\région{WEM WE PA}}
\variante{%
\vedette{hore}
\région{BO}}
\end{entrée}

\begin{entrée}
{résultat (bon ou mauvais) ; conséquence}
\classe{nom}
\vedette{hulò}\homonyme{1}
\sens{2}
\région{GOs BO PA}
\end{entrée}

\begin{entrée}
{résultat ; conséquence (bénéfique) (lit. tête du tubercule)}
\vedette{bwe-pai}
\région{GOs PA}
\variante{%
\vedette{bwe-vai}
\région{GO(s) PA}}
\end{entrée}

\begin{entrée}
{savoir ; connaître ; comprendre}
\vedette{hine}
\région{GOs}
\variante{%
\vedette{hine}
\région{PA BO}}
\end{entrée}

\begin{entrée}
{sens ; signification ; raison}
\vedette{wanga}
\région{GOs}
\variante{%
\vedette{whaga-n}
\région{BO [BM]}}
\end{entrée}

\begin{entrée}
{souvenir ; héritage des vieux [BM]}
\vedette{mhayu}
\région{BO}
\end{entrée}

\begin{entrée}
{surpris (être) ; sursauter ; étonner (s') ; étonné (être)}
\vedette{gaajò}
\région{GOs}
\variante{%
\vedette{gaajòn}
\région{PA BO WEM}}
\end{entrée}

\begin{entrée}
{tromper (se) (en parlant)}
\vedette{kilaavwi}
\région{GOs}
\end{entrée}

\begin{entrée}
{tromper (se) en prenant qqch}
\vedette{pe-thauvweni}
\région{GOs}
\end{entrée}

\begin{entrée}
{tromper (se) ; faire une erreur (dans ses gestes)}
\vedette{pwabaluni}
\région{GOs}
\end{entrée}

\begin{entrée}
{tromper (se); perdre (se)}
\vedette{niivwa}
\région{GOs}
\variante{%
\vedette{niivha}
\région{PA BO}}
\variante{%
\vedette{nipa}
\région{vx}}
\end{entrée}

\begin{entrée}
{zozoter (sens figuré)}
\classe{v}
\vedette{cou}
\sens{2}
\région{GOs PA WEM WE}
\end{entrée}

\subsubsection{Sentiments}

\begin{entrée}
{admirer}
\vedette{maari}
\région{GOs BO}
\end{entrée}

\begin{entrée}
{affligé}
\classe{v}
\vedette{cöńi}
\sens{2}
\région{GOs BO PA}
\variante{%
\vedette{cööni}
\région{PA}}
\end{entrée}

\begin{entrée}
{aimer ; affectionner}
\vedette{puxãnu}
\région{GOs}
\variante{%
\vedette{poxãnu, pwããnu}
\région{GO(s)}}
\variante{%
\vedette{poxònu, poonu}
\région{BO}}
\end{entrée}

\begin{entrée}
{aimer ; amour}
\vedette{pwããnu}
\région{GOs}
\variante{%
\vedette{poxããnu, puxãnu}
\région{GO(s)}}
\end{entrée}

\begin{entrée}
{amour ; compatir ; avoir pitié de}
\vedette{puxãnu}
\région{GOs}
\variante{%
\vedette{poxãnu, pwããnu}
\région{GO(s)}}
\variante{%
\vedette{poxònu, poonu}
\région{BO}}
\end{entrée}

\begin{entrée}
{amour ; sentiment ; affection ; désir ; volonté}
\classe{nom}
\vedette{phwe-ai}
\sens{2}
\région{GOs PA BO}
\end{entrée}

\begin{entrée}
{coeur ; amour ; volonté ; envie de}
\vedette{ai-}
\région{GOs PA BO}
\variante{%
\vedette{awi-}}
\end{entrée}

\begin{entrée}
{colère (être en grande)}
\vedette{tixãã}
\région{WE}
\variante{%
\vedette{tilixãi}
\région{BO}}
\end{entrée}

\begin{entrée}
{colère ; se mettre en colère}
\vedette{kòtrixê}
\région{GOs}
\variante{%
\vedette{kòtrikê}
\région{vx}}
\variante{%
\vedette{kòriê}
\région{GO(s)}}
\variante{%
\vedette{kòtrikã, kòriã}
\région{GO(s)}}
\variante{%
\vedette{kòòlixê}
\région{GO(s)}}
\end{entrée}

\begin{entrée}
{coléreux ; en colère ; irrité}
\vedette{thidin}
\région{PA BO WEM}
\end{entrée}

\begin{entrée}
{compatir ; pitié (avoir) ; prendre en pitié}
\vedette{mèèdi}
\région{BO PA}
\variante{%
\vedette{meedi}
\région{BO}}
\end{entrée}

\begin{entrée}
{dégoûté}
\vedette{bwaadu}
\région{BO [BM]}
\end{entrée}

\begin{entrée}
{désirer}
\classe{nom}
\vedette{phwe-ai}
\sens{2}
\région{GOs PA BO}
\end{entrée}

\begin{entrée}
{détester (se) ; rejeter (se)}
\vedette{pe-kueli}
\région{GOs}
\end{entrée}

\begin{entrée}
{émerveiller (s') ; émerveillé (être) ; admiratif}
\vedette{hããmal}
\région{PA BO}
\end{entrée}

\begin{entrée}
{envie de (avoir)}
\vedette{a-vwö}\homonyme{2}
\région{GOs}
\variante{%
\vedette{apo}
\région{GO(s)}}
\variante{%
\vedette{a- (forme courte)}
\région{GO(s)}}
\variante{%
\vedette{a-wu-; avo-}
\région{PA}}
\end{entrée}

\begin{entrée}
{exprimer son mécontentement (se dit de gens qu'on entend de loin, sans entendre le détail de ce qu'ils disent)}
\vedette{cauvala}
\région{PA}
\end{entrée}

\begin{entrée}
{gémir (de douleur) ; hurler}
\vedette{thraxe}
\région{GOs}
\end{entrée}

\begin{entrée}
{haïr ; détester}
\classe{v}
\vedette{hing}
\sens{1}
\région{PA BO}
\end{entrée}

\begin{entrée}
{heureux ; joyeux ; content ; joie ; joyeux ; content}
\vedette{thruumã}
\région{GOs}
\variante{%
\vedette{thuumã}
\région{PA BO}}
\variante{%
\vedette{tuumã}
\région{PA BO (Corne)}}
\end{entrée}

\begin{entrée}
{honte (avoir)(lié au deuil car on n'a pas su conserver la vie)}
\vedette{mõõdi}
\sens{1}
\région{GOs}
\variante{%
\vedette{mõõdim}
\région{WEM WE PA BO}}
\end{entrée}

\begin{entrée}
{honte (avoir) (lié au deuil car on n'a pas su conserver la vie)}
\classe{v ; n}
\vedette{mõõdim}
\sens{2}
\région{PA BO}
\end{entrée}

\begin{entrée}
{jaloux (être) ; jalouser}
\vedette{tre-kue}
\région{GOs}
\end{entrée}

\begin{entrée}
{jaloux (être) ; jalouser ; jalousie}
\vedette{kiiça}
\région{GOs}
\variante{%
\vedette{kiia}
\région{PA BO}}
\variante{%
\vedette{khia}
\région{BO}}
\end{entrée}

\begin{entrée}
{joie ; joyeux ; réjouir (se) ; content (être)}
\vedette{tròròvwuu}
\région{GOs}
\variante{%
\vedette{tròròwuu}
\région{GO(s)}}
\end{entrée}

\begin{entrée}
{malheureux ; triste ; nostalgique}
\vedette{yaawa}
\région{GOs PA}
\end{entrée}

\begin{entrée}
{mauvaise conscience (avoir)}
\vedette{camadi}
\région{GOs}
\end{entrée}

\begin{entrée}
{refuser ; ne pas vouloir}
\vedette{bu}\homonyme{6}
\région{PA}
\end{entrée}

\begin{entrée}
{refuser ; rejeter}
\vedette{khiba}
\région{GOs PA}
\end{entrée}

\begin{entrée}
{rester bouche-bée (bouche ouverte)}
\vedette{huraò}
\région{GOs}
\end{entrée}

\begin{entrée}
{retenir (se) de pleurer}
\vedette{khaagi}
\région{GOs}
\end{entrée}

\begin{entrée}
{retenir ses larmes (enfant)}
\classe{v}
\vedette{cöńi}
\sens{2}
\région{GOs BO PA}
\variante{%
\vedette{cööni}
\région{PA}}
\end{entrée}

\begin{entrée}
{retenir son souffle [PA]}
\classe{v}
\vedette{cöńi}
\sens{2}
\région{GOs BO PA}
\variante{%
\vedette{cööni}
\région{PA}}
\end{entrée}

\begin{entrée}
{suivre qqn comme son ombre (qqn à qui on est très attaché)}
\classe{v ; n}
\vedette{kãgu}
\sens{2}
\région{GOs}
\variante{%
\vedette{kãgun}
\région{BO PA}}
\end{entrée}

\begin{entrée}
{tourmenté}
\vedette{kòòwe}
\région{BO}
\end{entrée}

\begin{entrée}
{triste ; malheureux}
\vedette{pe-khînu}
\région{PA}
\end{entrée}

\begin{entrée}
{vouloir}
\vedette{a-vwö}\homonyme{2}
\région{GOs}
\variante{%
\vedette{apo}
\région{GO(s)}}
\variante{%
\vedette{a- (forme courte)}
\région{GO(s)}}
\variante{%
\vedette{a-wu-; avo-}
\région{PA}}
\end{entrée}

\begin{entrée}
{vouloir ; envie de (avoir)}
\vedette{ai-xa}
\région{GOs}
\variante{%
\vedette{ai xa}
\région{PA}}
\end{entrée}

\subsection{Parenté}

\subsubsection{Parenté}

\begin{entrée}
{adopter ; élever (enfant)}
\vedette{yue}
\région{GOs PA BO}
\end{entrée}

\begin{entrée}
{affins du côté maternel (maternels parlant)}
\vedette{avwi-}
\région{GOs}
\variante{%
\vedette{avhi-}
\région{PA}}
\variante{%
\vedette{api}
\région{GO(s)}}
\end{entrée}

\begin{entrée}
{aîné (des enfants)}
\vedette{phaamee}
\région{GOs PA}
\end{entrée}

\begin{entrée}
{aîné (fils)}
\vedette{phwioo}
\région{GOs}
\variante{%
\vedette{phoyo}
\région{GO(s)}}
\end{entrée}

\begin{entrée}
{aîné (frère, soeur) ; deuxième frère aîné [BO]}
\vedette{khoe}
\région{PA}
\variante{%
\vedette{khoè-n}
\région{BO [Corne]}}
\end{entrée}

\begin{entrée}
{arrière-arrière petits-enfants [GOs]}
\classe{nom}
\vedette{drele-ma-drele}
\sens{1}
\région{GOs}
\variante{%
\vedette{dele-ma-dele}
\région{BO}}
\end{entrée}

\begin{entrée}
{arrière-grand-mère}
\vedette{gèè-thraa, gè-raa}
\région{GOs PA}
\région{PA WEM WE}
\variante{%
\vedette{gèè-mãmã, gè-raa}}
\end{entrée}

\begin{entrée}
{arrière-grand-père}
\vedette{wha-maama}
\région{WEM WE}
\end{entrée}

\begin{entrée}
{arrière-grand-père}
\vedette{wha-thraa}
\région{GOs}
\variante{%
\vedette{wha-rhaa}
\région{GO(s)}}
\end{entrée}

\begin{entrée}
{arrière-petit-fils}
\vedette{niila}
\région{GOs}
\variante{%
\vedette{niila}
\région{PA}}
\end{entrée}

\begin{entrée}
{benjamin}
\vedette{môtra ênõ}
\région{GOs}
\variante{%
\vedette{möra ènõ}
\région{PA}}
\end{entrée}

\begin{entrée}
{benjamin}
\vedette{murae}
\région{GOs BO}
\end{entrée}

\begin{entrée}
{cadet ; frères ou soeurs plus jeunes qu'ego}
\vedette{kazi}
\région{GOs}
\variante{%
\vedette{kali-}
\région{PA BO}}
\end{entrée}

\begin{entrée}
{cadet (lit. debout sur le pied)}
\vedette{kobwako}
\région{GOs}
\end{entrée}

\begin{entrée}
{cousin croisé de même sexe (aîné ou cadet):fils/fille de soeur de père}
\vedette{ebiigi}
\région{GOs BO}
\variante{%
\vedette{biigi}
\région{PA BO}}
\end{entrée}

\begin{entrée}
{cousin croisé de même sexe (terme d'appellation)}
\vedette{bibi}
\région{GOs BO}
\end{entrée}

\begin{entrée}
{cousin de grand-père}
\vedette{wha}\homonyme{2}
\région{GOs}
\variante{%
\vedette{hua}
\région{PA BO}}
\end{entrée}

\begin{entrée}
{cousine de grand-mère}
\vedette{gèè}
\région{GOs PA BO}
\end{entrée}

\begin{entrée}
{cousine du père ;}
\vedette{ẽnõ}\homonyme{2}
\région{GOs}
\variante{%
\vedette{ènõ}
\région{PA}}
\end{entrée}

\begin{entrée}
{cousin(e) parallèle et aîné(e) ; cousin croisé de sexe opposé et aîné (enfants de la soeur du père)}
\vedette{yhò}
\région{GOs PA BO}
\end{entrée}

\begin{entrée}
{cousines de mère}
\vedette{õ}\homonyme{1}
\région{GOs BO PA}
\end{entrée}

\begin{entrée}
{cousin (fille de frère de père ; fils/fille de soeur de mère)}
\vedette{kò-wiò}
\région{GOs}
\variante{%
\vedette{kòyò}
\région{GO(s)}}
\variante{%
\vedette{koeo}
\région{GO(s)}}
\end{entrée}

\begin{entrée}
{cousins de mère}
\vedette{pööni}
\région{GOs}
\variante{%
\vedette{puuni}
\région{PA BO}}
\end{entrée}

\begin{entrée}
{cousins parallèles (enfants de soeur de mère, enfants de frère de père)}
\vedette{ãbaa-}\homonyme{2}
\région{GOs PA BO}
\end{entrée}

\begin{entrée}
{cousin (terme respectueux d'appellation ou désignation aux personnes plus agées)}
\vedette{kaaxo}
\région{GOsWE}
\end{entrée}

\begin{entrée}
{descendance}
\classe{nom}
\vedette{mõõxi}
\sens{2}
\région{GOs PA}
\variante{%
\vedette{mòòle}
\région{BO}}
\end{entrée}

\begin{entrée}
{enfant (âge)}
\classe{nom}
\vedette{ẽnõ}\homonyme{1}
\sens{2}
\région{GOs}
\variante{%
\vedette{ênõ}
\région{PA BO}}
\end{entrée}

\begin{entrée}
{enfant de fille de frère ou soeur de mère}
\vedette{pööni}
\région{GOs}
\variante{%
\vedette{puuni}
\région{PA BO}}
\end{entrée}

\begin{entrée}
{enfant de fille de frère ou soeur du père (= petits cousins)}
\vedette{pööni}
\région{GOs}
\variante{%
\vedette{puuni}
\région{PA BO}}
\end{entrée}

\begin{entrée}
{enfant de fils de frère ou de soeur de mère (homme parlant) ;}
\vedette{pòi}
\région{GOs PA WEM BO}
\variante{%
\vedette{pwe}
\région{BO}}
\end{entrée}

\begin{entrée}
{enfant de fils de frère ou soeur du père (= petits cousins, homme parlant)}
\vedette{pòi}
\région{GOs PA WEM BO}
\variante{%
\vedette{pwe}
\région{BO}}
\end{entrée}

\begin{entrée}
{enfant de frère et de cousins (femme parlant)}
\classe{nom}
\vedette{hê-kòlò-}
\sens{1}
\région{GOsPA BO}
\variante{%
\vedette{hê-xòlò}
\région{GO(s)}}
\end{entrée}

\begin{entrée}
{enfant de frère et de cousins mutuels (femme parlant)}
\vedette{a-pe-hê-kòlò}
\région{BO}
\end{entrée}

\begin{entrée}
{enfant de soeur et de cousines (femme parlant)}
\vedette{pòi}
\région{GOs PA WEM BO}
\variante{%
\vedette{pwe}
\région{BO}}
\end{entrée}

\begin{entrée}
{enfant de soeur (homme parlant)}
\vedette{pööni}
\région{GOs}
\variante{%
\vedette{puuni}
\région{PA BO}}
\end{entrée}

\begin{entrée}
{enfant (fille/fils) ; enfant de frère et de cousins (homme parlant)}
\vedette{pòi}
\région{GOs PA WEM BO}
\variante{%
\vedette{pwe}
\région{BO}}
\end{entrée}

\begin{entrée}
{enfants suivant l'aîné (lit. debout derrière)}
\vedette{kò-kai}
\région{GOs}
\variante{%
\vedette{kò-xai}
\région{GOs}}
\variante{%
\vedette{kòò-kain}
\région{PA}}
\end{entrée}

\begin{entrée}
{épouse des cousins de mère}
\vedette{ẽnõ}\homonyme{2}
\région{GOs}
\variante{%
\vedette{ènõ}
\région{PA}}
\end{entrée}

\begin{entrée}
{épouse du frère de mère}
\vedette{ẽnõ}\homonyme{2}
\région{GOs}
\variante{%
\vedette{ènõ}
\région{PA}}
\end{entrée}

\begin{entrée}
{épouse du frère de père ; épouse des cousins de père}
\vedette{õ}\homonyme{1}
\région{GOs BO PA}
\end{entrée}

\begin{entrée}
{épouse ; soeur de l'épouse ; épouse du frère}
\vedette{mõû-}
\région{GOs}
\variante{%
\vedette{mõû-, maû}
\région{PA BO}}
\end{entrée}

\begin{entrée}
{époux de soeur de mère ; époux de cousine de mère}
\vedette{kêê}\homonyme{1}
\région{GOs BO PA}
\end{entrée}

\begin{entrée}
{époux de soeur de père}
\vedette{pööni}
\région{GOs}
\variante{%
\vedette{puuni}
\région{PA BO}}
\end{entrée}

\begin{entrée}
{époux ; mari}
\vedette{azoo}
\région{GOs}
\variante{%
\vedette{alòò- ; alu}
\région{PA BO}}
\end{entrée}

\begin{entrée}
{famille; allié}
\vedette{a-pe-hê-kòlò}
\région{BO}
\end{entrée}

\begin{entrée}
{famille ; allié}
\classe{nom}
\vedette{hê-kòlò-}
\sens{2}
\région{GOsPA BO}
\variante{%
\vedette{hê-xòlò}
\région{GO(s)}}
\end{entrée}

\begin{entrée}
{famille (sans doute lié au cocotier) [BM]}
\vedette{nuu}
\région{BO}
\end{entrée}

\begin{entrée}
{fils de frère (lit. garçon de mon côté)}
\vedette{êmwen kòlò-}
\région{PA BO}
\end{entrée}

\begin{entrée}
{fils du frère (soeur du père parlant) (tantine) [Corne]}
\vedette{kòlò ije}
\région{BO}
\end{entrée}

\begin{entrée}
{fils/fille de frère de mère; fils de la soeur du père}
\vedette{ebiigi}
\région{GOs BO}
\variante{%
\vedette{biigi}
\région{PA BO}}
\end{entrée}

\begin{entrée}
{frère}
\vedette{ãbaa-}\homonyme{2}
\région{GOs PA BO}
\end{entrée}

\begin{entrée}
{frère}
\vedette{ãbaa êmwê}
\région{GOs PA}
\variante{%
\vedette{ãbaa êmwên whamã}
\région{PA}}
\end{entrée}

\begin{entrée}
{frère aîné}
\vedette{ãbaa êmwê whamã}
\région{GOs}
\variante{%
\vedette{ãbaa êmwên whamã}
\région{PA BO}}
\end{entrée}

\begin{entrée}
{frère cadet (tous les frères plus jeunes que l'aîné)}
\vedette{kò-wiò}
\région{GOs}
\variante{%
\vedette{kòyò}
\région{GO(s)}}
\variante{%
\vedette{koeo}
\région{GO(s)}}
\end{entrée}

\begin{entrée}
{frère de grand-père}
\vedette{wha}\homonyme{2}
\région{GOs}
\variante{%
\vedette{hua}
\région{PA BO}}
\end{entrée}

\begin{entrée}
{frère de père ; cousins du père}
\vedette{kêê}\homonyme{1}
\région{GOs BO PA}
\end{entrée}

\begin{entrée}
{frère ou soeur aîné(e) ; grand-frère ; grande-soeur}
\vedette{khoe}
\région{PA}
\variante{%
\vedette{khoè-n}
\région{BO [Corne]}}
\end{entrée}

\begin{entrée}
{frère/soeur aîné(e)}
\vedette{yhò}
\région{GOs PA BO}
\end{entrée}

\begin{entrée}
{frères/soeurs ; phratrie ; sororité (général, désigne la relation de frère ou soeur)}
\vedette{mèèvwu}
\région{GO PA BO}
\variante{%
\vedette{mèèpu}
\région{vx}}
\end{entrée}

\begin{entrée}
{garder (enfant)}
\vedette{yue}
\région{GOs PA BO}
\end{entrée}

\begin{entrée}
{grand-mère (maternelle ou paternelle, désignation et appellation)}
\vedette{gèè}
\région{GOs PA BO}
\end{entrée}

\begin{entrée}
{grand-père ; ancêtre}
\vedette{kibu}\homonyme{1}
\région{GOs PA BO}
\end{entrée}

\begin{entrée}
{grand-père (maternel ou paternel, désignation et appellation)}
\vedette{wha}\homonyme{2}
\région{GOs}
\variante{%
\vedette{hua}
\région{PA BO}}
\end{entrée}

\begin{entrée}
{jumeaux}
\vedette{pidru}
\région{GOs}
\variante{%
\vedette{pidu}
\région{PA BO}}
\end{entrée}

\begin{entrée}
{marier (se) (pour une femme, prendre époux)}
\vedette{hazo}
\région{GOs}
\région{PA BO}
\variante{%
\vedette{halòòn}}
\end{entrée}

\begin{entrée}
{maternels (ceux qui reçoivent, qui s'inclinent pour recevoir?)}
\vedette{a-kalu}
\région{GOs}
\end{entrée}

\begin{entrée}
{maternels ; parenté ou famille côté maternel}
\vedette{a-çaabò}
\région{GOs}
\variante{%
\vedette{ayabòl}
\région{PA}}
\variante{%
\vedette{ayabwòl}
\région{BO (Corne)}}
\end{entrée}

\begin{entrée}
{même famille (de la)}
\vedette{pe-hê-xòlò}
\région{GOs PA}
\end{entrée}

\begin{entrée}
{mère ; soeur de mère}
\vedette{õ}\homonyme{1}
\région{GOs BO PA}
\end{entrée}

\begin{entrée}
{neveu ou nièce de la soeur du père}
\vedette{ẽnõ}\homonyme{2}
\région{GOs}
\variante{%
\vedette{ènõ}
\région{PA}}
\end{entrée}

\begin{entrée}
{nièce (fille de frère et cousins)}
\vedette{thoomwã kòlò}
\région{PA BO}
\end{entrée}

\begin{entrée}
{oncle maternel}
\vedette{pööni}
\région{GOs}
\variante{%
\vedette{puuni}
\région{PA BO}}
\end{entrée}

\begin{entrée}
{père (appellation)}
\vedette{caaya}
\région{PA}
\end{entrée}

\begin{entrée}
{père (désignation)}
\vedette{kêê}\homonyme{1}
\région{GOs BO PA}
\end{entrée}

\begin{entrée}
{petit-fils ; petite-fille ; descendant}
\vedette{pèèbu}
\région{GOs WEM PA BO}
\end{entrée}

\begin{entrée}
{puîné (lit. enfant du milieu)}
\vedette{ẽnõ ni gò}
\région{GO PA}
\end{entrée}

\begin{entrée}
{soeur}
\vedette{ãbaa-}\homonyme{2}
\région{GOs PA BO}
\end{entrée}

\begin{entrée}
{soeur}
\vedette{ãbaa thoomwã}
\région{GOsPA}
\variante{%
\vedette{ãbaa thòòmwa, ãbaa-dòòmwa}
\région{PA}}
\end{entrée}

\begin{entrée}
{soeur aînée}
\vedette{ãbaa thoomwã whamã}
\région{GOsPA}
\variante{%
\vedette{ãbaa thòòmwa, ãbaa-dòòmwa whamã}
\région{PA}}
\end{entrée}

\begin{entrée}
{soeur de grand-mère}
\vedette{gèè}
\région{GOs PA BO}
\end{entrée}

\begin{entrée}
{soeur de père ; tante paternelle ("tantine")}
\vedette{ẽnõ}\homonyme{2}
\région{GOs}
\variante{%
\vedette{ènõ}
\région{PA}}
\end{entrée}

\begin{entrée}
{veuf ; veuve}
\vedette{drapwê}
\région{GOs}
\variante{%
\vedette{damwê}}
\end{entrée}

\begin{entrée}
{vieux}
\vedette{wha}\homonyme{2}
\région{GOs}
\variante{%
\vedette{hua}
\région{PA BO}}
\end{entrée}

\subsubsection{Appellation parenté}

\begin{entrée}
{maman!}
\vedette{nyejo!}
\région{PA}
\end{entrée}

\begin{entrée}
{maman ; tante maternelle}
\vedette{nyãnyã}
\région{GOs PA BO}
\end{entrée}

\subsubsection{Alliance}

\begin{entrée}
{beau-frère}
\vedette{ebe}
\région{GOs}
\variante{%
\vedette{mõõ-n}
\région{PA}}
\end{entrée}

\begin{entrée}
{beau-père ; belle-mère (père/mère d'épouse ; père/mère du mari)}
\vedette{mõõ-}
\région{GOs}
\variante{%
\vedette{mõõ-}
\région{PA}}
\variante{%
\vedette{mwòòn}
\région{BO}}
\end{entrée}

\begin{entrée}
{beau-père et beau-fils (le terme duel tombe en désuétude)}
\vedette{mõõ-}
\région{GOs}
\variante{%
\vedette{mõõ-}
\région{PA}}
\variante{%
\vedette{mwòòn}
\région{BO}}
\end{entrée}

\begin{entrée}
{belle-soeur}
\vedette{ebe ba-êgu}
\région{GOs}
\variante{%
\vedette{mõõ-n thoomwã, mõõ-n dòòmwã}
\région{PA}}
\end{entrée}

\begin{entrée}
{belle-soeur}
\vedette{ivwö}
\région{GOs}
\end{entrée}

\begin{entrée}
{belle-soeur (soeur d'épouse ; épouse du frère)}
\vedette{phalawu}
\région{BO [BM]}
\variante{%
\vedette{phalau}
\région{BO}}
\end{entrée}

\begin{entrée}
{clan maternel (dans les cérémonies de deuil)}
\vedette{a-xalu}
\région{GOs}
\variante{%
\vedette{a-kalu}
\région{GO(s)}}
\end{entrée}

\begin{entrée}
{gendre (mari de fille) ; belle-fille (épouse de fils)}
\vedette{mõõ-}
\région{GOs}
\variante{%
\vedette{mõõ-}
\région{PA}}
\variante{%
\vedette{mwòòn}
\région{BO}}
\end{entrée}

\begin{entrée}
{maternels (clan des)}
\vedette{waniri}
\région{PA BO WE}
\end{entrée}

\begin{entrée}
{papa ; tonton (oncle paternel)}
\vedette{caaça}
\région{GOs}
\variante{%
\vedette{caaya, caya}
\région{PA BO}}
\end{entrée}

\begin{entrée}
{parenté par alliance}
\vedette{bee-}
\région{GOs PA BO}
\end{entrée}

\begin{entrée}
{soeur du mari ; frère d'épouse ; mari de soeur}
\vedette{bee-}
\région{GOs PA BO}
\end{entrée}

\begin{entrée}
{soeur ou frère du beau-frère ; soeur ou frère de la belle-soeur (désigne aussi 'homme parlant' les cousins parallèles de l'épouse: fils de frère de père, fils de soeur de mère et les cousins croisés de l'épouse: fils de frère de mère, fils de soeur de père)}
\vedette{bee-}
\région{GOs PA BO}
\end{entrée}

\subsubsection{Couples de parenté}

\begin{entrée}
{beau-père et beau-fils}
\vedette{è-mõõ}
\région{GOs}
\variante{%
\vedette{mõõn}
\région{PA}}
\end{entrée}

\begin{entrée}
{beaux-parents: beau-père (d'épouse ou de mari) ; belle-mère (d'épouse ou de mari)}
\vedette{è-mõõ}
\région{GOs}
\variante{%
\vedette{mõõn}
\région{PA}}
\end{entrée}

\begin{entrée}
{couple de relation}
\vedette{epè-}
\région{GO}
\end{entrée}

\begin{entrée}
{enfant de frère et de cousins mutuels (femme parlant)}
\vedette{a-pe-hê-kòlò}
\région{BO}
\end{entrée}

\begin{entrée}
{époux (mari et femme)}
\vedette{e-mõû}
\région{GOs}
\variante{%
\vedette{e-mõû-n}
\région{PA BO}}
\end{entrée}

\begin{entrée}
{famille; allié}
\vedette{a-pe-hê-kòlò}
\région{BO}
\end{entrée}

\begin{entrée}
{frère du père et neveu (ou) nièce}
\classe{couple PAR}
\vedette{e-pòi}
\sens{2}
\région{GOs}
\variante{%
\vedette{e-vwòi}
\région{GO(s)}}
\variante{%
\vedette{e-pòi-n}
\région{PA BO}}
\end{entrée}

\begin{entrée}
{grand-père ou grand-mère et petit-fils ou petite-fille}
\vedette{e-peebu}
\région{GOs}
\variante{%
\vedette{e-veebu}
\région{GO(s)}}
\variante{%
\vedette{e-veebu-n}
\région{PA}}
\end{entrée}

\begin{entrée}
{mère et fils (ou) fille}
\classe{couple PAR}
\vedette{e-pòi}
\sens{1}
\région{GOs}
\variante{%
\vedette{e-vwòi}
\région{GO(s)}}
\variante{%
\vedette{e-pòi-n}
\région{PA BO}}
\end{entrée}

\begin{entrée}
{oncle maternel et neveu/nièce maternel}
\vedette{è-pööni}
\région{GOs}
\variante{%
\vedette{è-vwööni}
\région{GO(s)}}
\variante{%
\vedette{è-pööni-n}
\région{PA}}
\end{entrée}

\begin{entrée}
{parenté duelle}
\vedette{e-}
\région{BO [BM]}
\end{entrée}

\begin{entrée}
{parenté réciproque}
\vedette{è-...-n}
\région{PA}
\end{entrée}

\begin{entrée}
{père et fils (ou) fille}
\classe{couple PAR}
\vedette{e-pòi}
\sens{1}
\région{GOs}
\variante{%
\vedette{e-vwòi}
\région{GO(s)}}
\variante{%
\vedette{e-pòi-n}
\région{PA BO}}
\end{entrée}

\begin{entrée}
{soeur de la mère et neveu (ou) nièce}
\classe{couple PAR}
\vedette{e-pòi}
\sens{2}
\région{GOs}
\variante{%
\vedette{e-vwòi}
\région{GO(s)}}
\variante{%
\vedette{e-pòi-n}
\région{PA BO}}
\end{entrée}

\begin{entrée}
{tante paternelle (tantine) et nièce}
\vedette{ebe-thoomwã kòlò}
\région{GOs}
\variante{%
\vedette{epe}
\région{GO}}
\end{entrée}

\subsection{Organisation sociale, richesses, dons, échanges}

\subsubsection{Société et organisation sociale}

\paragraph{Société}

\begin{entrée}
{Blanc (lit. poudre riz) ; européen}
\vedette{draalai}
\région{GOs}
\variante{%
\vedette{daalaèn, dalaèn}
\région{BO}}
\end{entrée}

\begin{entrée}
{célibataire (être)}
\vedette{pe-yu}
\région{GOs}
\end{entrée}

\begin{entrée}
{célibataire (homme ou femme)}
\vedette{kayuna}
\région{GOs}
\variante{%
\vedette{kauna}
\région{PA BO}}
\end{entrée}

\begin{entrée}
{clan venus de l'extérieur(pour des cérémonies)}
\classe{nom}
\vedette{aavhe}
\sens{2}
\région{PA BO}
\variante{%
\vedette{avwe}
\région{GO(s)}}
\end{entrée}

\begin{entrée}
{compagnon}
\vedette{amee}
\région{GOs}
\end{entrée}

\begin{entrée}
{compagnon de naissance[Corne]}
\vedette{ka-avè}
\région{BO}
\end{entrée}

\begin{entrée}
{décimé ; sans descendance}
\vedette{buli}
\région{GOs BO}
\end{entrée}

\begin{entrée}
{deuil (être en)}
\classe{v}
\vedette{cöńi}
\sens{3}
\région{GOs BO PA}
\variante{%
\vedette{cööni}
\région{PA}}
\end{entrée}

\begin{entrée}
{élever (enfants, animaux)}
\classe{v}
\vedette{kããle}
\sens{2}
\région{GOs BO PA}
\end{entrée}

\begin{entrée}
{étranger ; blanc ; européen}
\vedette{dalaèèn}
\région{BO}
\variante{%
\vedette{daalèn ; dalaèn}
\région{BO}}
\variante{%
\vedette{dalaan}
\région{PA}}
\end{entrée}

\begin{entrée}
{étranger ; inconnu (personnes, objets)}
\classe{nom}
\vedette{aavhe}
\sens{1}
\région{PA BO}
\variante{%
\vedette{avwe}
\région{GO(s)}}
\end{entrée}

\begin{entrée}
{faire équipe}
\vedette{pe-thu-ba}
\région{GOs}
\end{entrée}

\begin{entrée}
{femme ; féminin}
\vedette{thoomwã}
\région{GOs PA}
\end{entrée}

\begin{entrée}
{femme (terme respectueux)}
\vedette{ba-êgu}
\région{GOs}
\end{entrée}

\begin{entrée}
{gendarme}
\vedette{caladaa}
\région{GOs}
\end{entrée}

\begin{entrée}
{habitant (être) de}
\vedette{a-yu}
\région{GOs}
\end{entrée}

\begin{entrée}
{homme ; mâle}
\vedette{êmwê}\homonyme{1}
\région{GOs}
\variante{%
\vedette{êmwèn}
\région{PA BO}}
\end{entrée}

\begin{entrée}
{homme mûr ; dans la force de l'âge}
\vedette{cii êgu}
\région{GOs}
\end{entrée}

\begin{entrée}
{homme ; personne}
\vedette{êgu}\homonyme{1}
\région{GOs BO}
\end{entrée}

\begin{entrée}
{homonyme (être) (qui porte le même nom)}
\vedette{be-yaza}
\région{GOs}
\variante{%
\vedette{be-ala-}
\région{PA}}
\variante{%
\vedette{be-(y)ala}
\région{WEM WE}}
\end{entrée}

\begin{entrée}
{japonais}
\vedette{sapone}
\région{GOs}
\end{entrée}

\begin{entrée}
{marier (se) (pour une femme, prendre époux)}
\vedette{hazo}
\région{GOs}
\région{PA BO}
\variante{%
\vedette{halòòn}}
\end{entrée}

\begin{entrée}
{mariés (être) (lit. rester ensemble)}
\vedette{yuu bulu}
\région{PA}
\end{entrée}

\begin{entrée}
{même équipe (être dans la)}
\vedette{pe-bala}
\région{GOs}
\end{entrée}

\begin{entrée}
{même génération ; même tranche d'âge}
\vedette{cixè}
\région{GOs}
\end{entrée}

\begin{entrée}
{nom}
\vedette{yaaza}
\région{GOs}
\variante{%
\vedette{yaala-n, yala-n}
\région{PA WEM BO}}
\variante{%
\vedette{yhaala-n, yara-n}
\région{PA}}
\end{entrée}

\begin{entrée}
{pleurer un mort}
\classe{v}
\vedette{cöńi}
\sens{3}
\région{GOs BO PA}
\variante{%
\vedette{cööni}
\région{PA}}
\end{entrée}

\begin{entrée}
{pleurer un mort}
\vedette{gi-mã}
\région{GOs}
\end{entrée}

\begin{entrée}
{pleurer un mort (pour les hommes) [Corne]}
\vedette{giul}
\région{BO}
\end{entrée}

\begin{entrée}
{sculpteur}
\vedette{a-thu drogò}
\région{GOs}
\end{entrée}

\begin{entrée}
{s'occuper (de qqn)}
\classe{v}
\vedette{kããle}
\sens{2}
\région{GOs BO PA}
\end{entrée}

\begin{entrée}
{voir (registre respectueux pour le Grand Chef) [BM]}
\vedette{hããni}
\région{BO}
\end{entrée}

\paragraph{Organisation sociale}

\begin{entrée}
{ancêtres}
\classe{nom}
\vedette{puu}\homonyme{1}
\sens{2}
\région{GOs PA BO}
\variante{%
\vedette{pu}
\région{GO(s)}}
\variante{%
\vedette{puu-n}
\région{BO PA}}
\variante{%
\vedette{puxu-n}
\région{BO}}
\end{entrée}

\begin{entrée}
{ascendants (de la lignée) [BO, Corne]}
\classe{nom}
\vedette{drele-ma-drele}
\sens{2}
\région{GOs}
\variante{%
\vedette{dele-ma-dele}
\région{BO}}
\end{entrée}

\begin{entrée}
{assemblée}
\classe{nom}
\vedette{pwamwãgu}
\sens{1}
\région{PA BO}
\variante{%
\vedette{pwamwègu}
\région{BO}}
\end{entrée}

\begin{entrée}
{chef de guerre}
\vedette{a whili paa}
\région{GOs PA}
\end{entrée}

\begin{entrée}
{chefferie}
\classe{nom}
\vedette{kavwègu}
\sens{1}
\région{GOs PA BO}
\variante{%
\vedette{kapègu}
\région{GO(s) vx}}
\end{entrée}

\begin{entrée}
{chef ; grand-chef}
\vedette{aazo}
\région{GOs PA}
\end{entrée}

\begin{entrée}
{chemin coutumier}
\classe{nom}
\vedette{dè}
\sens{2}
\région{GOs}
\variante{%
\vedette{dèn}
\région{BO PA}}
\end{entrée}

\begin{entrée}
{chemin coutumier}
\classe{nom}
\vedette{phwè-dèn}
\sens{2}
\région{PA BO}
\end{entrée}

\begin{entrée}
{clan}
\vedette{yamevwu}
\région{GOs}
\variante{%
\vedette{yamepu}}
\end{entrée}

\begin{entrée}
{clan allié du clan maternel}
\vedette{apoxapenu}
\région{GOs}
\variante{%
\vedette{avhoxavhenu}
\région{PA}}
\end{entrée}

\begin{entrée}
{clan cadet ; cadet de la chefferie}
\vedette{puxa-teã}
\région{GOs}
\end{entrée}

\begin{entrée}
{clan ; famille}
\vedette{phwe-meewu}
\région{GOs PA BO}
\end{entrée}

\begin{entrée}
{clans (tous les) qui composent la chefferie et qui la protègent en temps de guerre}
\vedette{hova-mwa}
\région{GOs PA WEM}
\end{entrée}

\begin{entrée}
{contenu de la barrière}
\vedette{hê-thini}
\région{GOs BO}
\end{entrée}

\begin{entrée}
{cour de la chefferie}
\classe{nom}
\vedette{kavwègu}
\sens{1}
\région{GOs PA BO}
\variante{%
\vedette{kapègu}
\région{GO(s) vx}}
\end{entrée}

\begin{entrée}
{emplacement des femmes}
\vedette{pwamwã-ròòmwa}
\région{PA BO}
\end{entrée}

\begin{entrée}
{emplacement des femmes dans une réunion (Dubois)}
\vedette{pavo-romwa}
\région{BO}
\end{entrée}

\begin{entrée}
{enclos pour les dons coutumiers}
\vedette{thîni hauva}
\région{GOs}
\end{entrée}

\begin{entrée}
{ensemble des clans formant la tribu [PA]}
\classe{nom}
\vedette{ku}\homonyme{4}
\sens{2}
\région{GOs PA}
\variante{%
\vedette{kun}
\région{PA}}
\end{entrée}

\begin{entrée}
{entrée de la chefferie}
\classe{nom}
\vedette{phwè-zini}
\sens{2}
\région{GOs}
\variante{%
\vedette{phwe-thîni}
\région{PA}}
\end{entrée}

\begin{entrée}
{établir, instituer}
\classe{v}
\vedette{khabe}
\sens{3}
\région{GOs PA BO}
\end{entrée}

\begin{entrée}
{fille aînée de chef}
\vedette{Kaavwo}
\région{PA}
\end{entrée}

\begin{entrée}
{fille cadette de chef}
\vedette{Hixe}
\région{PA}
\end{entrée}

\begin{entrée}
{fils aîné de chef}
\vedette{Têã}
\région{BO PA}
\end{entrée}

\begin{entrée}
{fils cadet de chef}
\vedette{mweau}
\région{GOs WEM PA}
\end{entrée}

\begin{entrée}
{fils du chef (lit. le petit)}
\vedette{poxa aazo}
\région{GOs}
\variante{%
\vedette{poga aao}
\région{BO}}
\end{entrée}

\begin{entrée}
{gardien ; maître ; propriétaire}
\vedette{kaavwu}
\région{GOs}
\variante{%
\vedette{kapu}
\région{GO(s) vx}}
\variante{%
\vedette{kaawuun}
\région{PA BO}}
\end{entrée}

\begin{entrée}
{grand-chef}
\vedette{Treã-ma}
\région{GOs}
\variante{%
\vedette{Têã-ma}
\région{PA BO}}
\end{entrée}

\begin{entrée}
{interdit}
\classe{nom}
\vedette{nõbu}
\sens{2}
\région{GOs}
\variante{%
\vedette{nõbu}
\région{BO PA}}
\end{entrée}

\begin{entrée}
{labourer le champ d'igname du chef}
\vedette{ba-thu-khia}
\région{PA BO}
\end{entrée}

\begin{entrée}
{liens de famille}
\vedette{penuu}\homonyme{1}
\région{GOs}
\end{entrée}

\begin{entrée}
{loi ; règle}
\classe{nom}
\vedette{baa-ja}
\sens{1}
\région{GOs}
\end{entrée}

\begin{entrée}
{médiateur de la chefferie}
\vedette{dagi pwemwa}
\région{GOs}
\variante{%
\vedette{daginy pwemwa}
\région{WEM}}
\end{entrée}

\begin{entrée}
{messager de la chefferie}
\classe{nom}
\vedette{puradimwã}
\sens{2}
\région{GOs PA}
\end{entrée}

\begin{entrée}
{messager du grand-chef}
\vedette{dagi}
\sens{2}
\région{GOs}
\variante{%
\vedette{daginy}
\région{WEM BO PA}}
\end{entrée}

\begin{entrée}
{messager (lit. celui qui apporte le message)}
\vedette{a-phe-vhaa}
\région{GOs PA}
\end{entrée}

\begin{entrée}
{offrandes de nourriture dans les coutumes}
\vedette{hê-thini}
\région{GOs BO}
\end{entrée}

\begin{entrée}
{ouverture du champ d'igname du chef}
\vedette{ba-thu-khia}
\région{PA BO}
\end{entrée}

\begin{entrée}
{palissade de la chefferie}
\vedette{thîni-a kavegu}
\région{GOs}
\end{entrée}

\begin{entrée}
{personne qui sert de 'chemin' pour entrer dans la chefferie [PA]}
\classe{nom}
\vedette{phwè-mwa}
\sens{2}
\région{GOs}
\variante{%
\vedette{phwee-mwa}
\région{PA BO}}
\end{entrée}

\begin{entrée}
{personnes connues [PA]}
\vedette{apoxapenu}
\région{GOs}
\variante{%
\vedette{avhoxavhenu}
\région{PA}}
\end{entrée}

\begin{entrée}
{petit chef}
\vedette{poxa Teã}
\région{PA}
\variante{%
\vedette{poxo Teã}}
\end{entrée}

\begin{entrée}
{place de réunion dans le village ; emplacement des hommes (dans une réunion)}
\classe{nom}
\vedette{pwamwãgu}
\sens{1}
\région{PA BO}
\variante{%
\vedette{pwamwègu}
\région{BO}}
\end{entrée}

\begin{entrée}
{porte-parole [BM]}
\vedette{ã-mani}
\région{BO}
\end{entrée}

\begin{entrée}
{porte-parole du chef [Corne]}
\vedette{zaboriã}
\région{BO}
\end{entrée}

\begin{entrée}
{porte-parole (du chef) ; émissaire}
\vedette{a-maxo}
\région{GOs}
\end{entrée}

\begin{entrée}
{protecteur du sol (nom d'un clan)}
\vedette{kaavwu dili}
\région{GOs}
\end{entrée}

\begin{entrée}
{protection}
\classe{nom}
\vedette{nõbu}
\sens{2}
\région{GOs}
\variante{%
\vedette{nõbu}
\région{BO PA}}
\end{entrée}

\begin{entrée}
{règle ; loi}
\classe{nom}
\vedette{nõbu}
\sens{2}
\région{GOs}
\variante{%
\vedette{nõbu}
\région{BO PA}}
\end{entrée}

\begin{entrée}
{sorcier ('emboucaneur')}
\vedette{a-phònò}
\région{GOs}
\variante{%
\vedette{a-phònòng}
\région{PA}}
\end{entrée}

\begin{entrée}
{sorcier ; emboucaneur}
\vedette{aò}
\région{BO PA}
\end{entrée}

\begin{entrée}
{sujet ; serviteur}
\vedette{yabwe}
\région{GOs WEM PA BO}
\end{entrée}

\subsubsection{Relations et interaction sociales}

\begin{entrée}
{abandonner}
\vedette{khagebwa}
\région{GOs}
\variante{%
\vedette{khagebwan}
\région{PA}}
\end{entrée}

\begin{entrée}
{abandonner ; délaisser}
\vedette{thaxiba}
\sens{1}
\région{GOs PA}
\variante{%
\vedette{thakiba}
\région{GO(s)}}
\variante{%
\vedette{taxiba}
\région{BO PA}}
\end{entrée}

\begin{entrée}
{à bientôt}
\classe{v ; n}
\vedette{pe-tròòli}
\sens{2}
\région{GOs PA}
\variante{%
\vedette{pe-ròòli}
\région{GO(s)}}
\end{entrée}

\begin{entrée}
{accepter}
\classe{v}
\vedette{phè}
\sens{4}
\région{GOs PA BO}
\variante{%
\vedette{phe}}
\end{entrée}

\begin{entrée}
{accepter la demande de pardon}
\vedette{uvwa}
\région{GOs}
\end{entrée}

\begin{entrée}
{accroché à sa mère ; 'collant' (enfant)}
\vedette{waajô}
\région{PA}
\end{entrée}

\begin{entrée}
{accueillir ; recevoir}
\vedette{kitrabwi}
\région{GOs}
\end{entrée}

\begin{entrée}
{accuser ; calomnier ; diffamer}
\vedette{thaxebi}
\région{GOs BO}
\end{entrée}

\begin{entrée}
{à demain !}
\vedette{pe-tròòli mõnõ}
\sens{2}
\région{GOs}
\variante{%
\vedette{pe-ròòli mõnõ}
\région{GO(s)}}
\variante{%
\vedette{pe-tòòli menon}
\région{PA}}
\end{entrée}

\begin{entrée}
{aider}
\vedette{zage}
\région{GOs PA}
\variante{%
\vedette{zhage, zaage}
\région{GO(s)}}
\variante{%
\vedette{yhage}
\région{BO}}
\end{entrée}

\begin{entrée}
{aimer ; affectionner}
\vedette{puxãnu}
\région{GOs}
\variante{%
\vedette{poxãnu, pwããnu}
\région{GO(s)}}
\variante{%
\vedette{poxònu, poonu}
\région{BO}}
\end{entrée}

\begin{entrée}
{ajouter ; abonder dans le sens de qqn}
\vedette{zageeni}
\région{GOs}
\variante{%
\vedette{zhageeni}
\région{GA}}
\end{entrée}

\begin{entrée}
{amour ; compatir ; avoir pitié de}
\vedette{puxãnu}
\région{GOs}
\variante{%
\vedette{poxãnu, pwããnu}
\région{GO(s)}}
\variante{%
\vedette{poxònu, poonu}
\région{BO}}
\end{entrée}

\begin{entrée}
{annoncer des informations ; présenter}
\classe{v}
\vedette{thooni}
\sens{1}
\région{GOs PA}
\end{entrée}

\begin{entrée}
{annoncer ; prévenir ; promettre ; déclarer ; aviser}
\vedette{tre-khõbwe}
\région{GOs}
\variante{%
\vedette{te-kôbwe}
\région{BO}}
\end{entrée}

\begin{entrée}
{annoncer publiquement ; faire une annonce}
\vedette{tho-khõbwe}
\région{GOs PA}
\end{entrée}

\begin{entrée}
{apaiser}
\vedette{pa-nue}
\région{GOs}
\end{entrée}

\begin{entrée}
{apaiser ; calmer (qqn)}
\vedette{pha-tre-çaxoo-ni}
\région{GOs}
\end{entrée}

\begin{entrée}
{appeler}
\vedette{thomã}
\région{GOs WEM PA BO}
\end{entrée}

\begin{entrée}
{appeler en marchant}
\vedette{kha-tho}
\région{GOs}
\end{entrée}

\begin{entrée}
{appeler ; interpeller}
\classe{v.i. ; n}
\vedette{tho}\homonyme{1}
\sens{2}
\région{GOs PA BO}
\end{entrée}

\begin{entrée}
{appeler qqn}
\vedette{thomãni}
\région{GOs PA}
\end{entrée}

\begin{entrée}
{assemblée; rassembler}
\vedette{phiing}
\région{PA}
\end{entrée}

\begin{entrée}
{assembler ; rassembler}
\vedette{na-bulu-ni}
\région{GOs}
\end{entrée}

\begin{entrée}
{à tout à l'heure ; à tout de suite}
\vedette{pe-tròòli èò}
\région{GOs}
\end{entrée}

\begin{entrée}
{attendre}
\vedette{hôbwo}
\région{GOs}
\variante{%
\vedette{hôbo}
\région{GO(s) PA}}
\end{entrée}

\begin{entrée}
{attention! ; faire attention}
\vedette{phwêne}
\région{GOs}
\variante{%
\vedette{phwêneng}
\région{PA}}
\end{entrée}

\begin{entrée}
{attitude ; comportement ; manière de faire}
\vedette{me-nee-vwo}
\région{GOs WEM}
\variante{%
\vedette{mèneevwò,mèneevwu-n, mèneexu-n}
\région{BO PA}}
\end{entrée}

\begin{entrée}
{à un de ces jours !}
\vedette{pe-tròòli ni tree}
\région{GOs}
\end{entrée}

\begin{entrée}
{bavarder ; converser ; discuter}
\vedette{phweexu}
\sens{2}
\région{GOs PA BO}
\variante{%
\vedette{phweeku}
\région{GO(s)}}
\variante{%
\vedette{phweewu}
\région{BO}}
\end{entrée}

\begin{entrée}
{bercer (enfant) [BM]}
\vedette{yüe}
\région{BO}
\end{entrée}

\begin{entrée}
{bercer (un enfant) ; berceuse (chant)}
\vedette{otròtròya}
\région{GOs WEM}
\variante{%
\vedette{oroyai}
\région{GO(s) PA BO}}
\variante{%
\vedette{yüe}
\région{BO}}
\end{entrée}

\begin{entrée}
{changer ; échanger ; remplacer}
\vedette{wêne}
\région{GOs PA BO}
\variante{%
\vedette{wône}}
\end{entrée}

\begin{entrée}
{chercher querelle (se)}
\vedette{pe-chôã}
\région{GOs}
\end{entrée}

\begin{entrée}
{clairement ; public}
\classe{v}
\vedette{phwaa}
\sens{2}
\région{GOs}
\région{PA BO}
\variante{%
\vedette{phwaal}}
\end{entrée}

\begin{entrée}
{colère contre qqn (être en) ; dispute}
\classe{v ; n}
\vedette{thô}
\sens{3}
\région{GOs}
\variante{%
\vedette{thõn}
\région{BO}}
\end{entrée}

\begin{entrée}
{compagnon ; ami}
\vedette{kovanyi}
\région{GOs}
\end{entrée}

\begin{entrée}
{craindre (qqn, qqch)}
\vedette{hããxe}
\région{GOs BO}
\end{entrée}

\begin{entrée}
{craintif ; peureux ; craindre ; avoir peur}
\vedette{hããxa}
\région{GOs BO}
\variante{%
\vedette{haaka}
\région{BO}}
\end{entrée}

\begin{entrée}
{crier (pour annoncer sa présence)}
\vedette{bua}
\région{GOs PA BO}
\end{entrée}

\begin{entrée}
{demander la permission (intensif)}
\vedette{kha-trilòò}
\région{GOs}
\end{entrée}

\begin{entrée}
{déranger}
\vedette{irô}
\région{BO}
\end{entrée}

\begin{entrée}
{détester}
\vedette{kue}
\région{GOs}
\variante{%
\vedette{kuel, kwel}
\région{BO}}
\end{entrée}

\begin{entrée}
{dire des gros mots ; médire}
\vedette{vhaa-raa}
\région{GOs PA}
\end{entrée}

\begin{entrée}
{dire du mal de qqn}
\vedette{khõbwe-raa}
\région{GOs}
\end{entrée}

\begin{entrée}
{dire la vérité ; avoir raison}
\vedette{gumãgu}
\région{GOs PA BO}
\end{entrée}

\begin{entrée}
{disputer (se) ; battre (se) (avec ou sans armes)}
\vedette{pe-wèle}
\région{BO PA}
\end{entrée}

\begin{entrée}
{disputer (se) ; chercher querelle ; reprocher}
\vedette{piyuli}
\région{PA BO [BM]}
\end{entrée}

\begin{entrée}
{disputer (se)(jeu, compétition) ; garder pour soi}
\vedette{pe-kae}
\région{GOs}
\variante{%
\vedette{pe-kaeny}
\région{PA}}
\end{entrée}

\begin{entrée}
{disputer (se) (verbalement) ; chamailler (se)}
\vedette{piia}\homonyme{1}
\région{GOs BO PA}
\end{entrée}

\begin{entrée}
{donne-moi un peu}
\vedette{kò-na-mi}
\région{PA}
\end{entrée}

\begin{entrée}
{donner de la nourriture (à des animés)}
\vedette{zaalae}
\région{GOs}
\variante{%
\vedette{zhaalae}
\région{GA}}
\end{entrée}

\begin{entrée}
{efforcer de (s') ; persister à ; insister}
\vedette{waaçu}
\région{GOs WEM WE}
\variante{%
\vedette{waçuçu}
\région{GO(s) WEM WE}}
\variante{%
\vedette{waayu, waaju}
\région{PA BO}}
\end{entrée}

\begin{entrée}
{effrayer ; faire peur à qqn}
\vedette{phaza-hããxe}
\région{GOs}
\variante{%
\vedette{paza-hããxa, para-hããxe}
\région{PA}}
\end{entrée}

\begin{entrée}
{élever}
\vedette{zaalae}
\région{GOs}
\variante{%
\vedette{zhaalae}
\région{GA}}
\end{entrée}

\begin{entrée}
{embêter, empêcher, déranger}
\vedette{taçinô}
\région{GOs}
\end{entrée}

\begin{entrée}
{empêcher (obstacle)}
\classe{v.t.}
\vedette{thôni}
\sens{2}
\région{GOs WEM WE PA BO}
\end{entrée}

\begin{entrée}
{énerver ; agacer (lit. piquer le sang)}
\vedette{thii kura}
\région{GOs PA}
\end{entrée}

\begin{entrée}
{engueuler ; tancer}
\vedette{peçööli}
\région{GOs}
\variante{%
\vedette{pejooli}
\région{PA}}
\end{entrée}

\begin{entrée}
{enlever ; ravir (femme)}
\classe{v.t.}
\vedette{kae}\homonyme{1}
\sens{1}
\région{GOs PA WEM WE BO}
\end{entrée}

\begin{entrée}
{ennuyer (s')}
\vedette{thirûû}
\région{GOs}
\end{entrée}

\begin{entrée}
{ensemble}
\vedette{pe-bulu}
\région{GOs BO}
\end{entrée}

\begin{entrée}
{ensemble (être)}
\vedette{a-vwe bulu}
\région{GOs}
\end{entrée}

\begin{entrée}
{envoyer qqn faire qqch. ; ordonner}
\classe{v}
\vedette{khêni}
\région{GOs}
\variante{%
\vedette{kheni}
\région{PA BO}}
\end{entrée}

\begin{entrée}
{envoyer (s') mutuellement}
\vedette{pe-na}
\région{GOs}
\end{entrée}

\begin{entrée}
{esprits des vieux du clan}
\vedette{whany}
\région{PA BO}
\variante{%
\vedette{wany}
\région{PA BO}}
\end{entrée}

\begin{entrée}
{éviter}
\classe{v}
\vedette{kã}
\sens{3}
\région{GOs}
\variante{%
\vedette{kãm}
\région{BO}}
\variante{%
\vedette{kham}
\région{PA}}
\end{entrée}

\begin{entrée}
{éviter (qqn ou qqch) ; esquiver}
\classe{v}
\vedette{pii}\homonyme{2}
\sens{1}
\région{GOs PA BO}
\end{entrée}

\begin{entrée}
{éviter (s')}
\vedette{pe-vii}
\région{GOs}
\end{entrée}

\begin{entrée}
{faire acte de présence}
\vedette{phããde cii-phagò}
\région{PA}
\end{entrée}

\begin{entrée}
{faire attention}
\vedette{pewoo}
\région{BO [BM]}
\end{entrée}

\begin{entrée}
{faire des plaisanteries grivoises}
\vedette{phwaa khoojòng}
\région{PA}
\end{entrée}

\begin{entrée}
{faire remarquer (se) ; se mettre en évidence}
\vedette{kò bwa me}
\région{GOs}
\variante{%
\vedette{a ni dòn i ègu}
\région{PA}}
\end{entrée}

\begin{entrée}
{faire (se) peur mutuellement}
\vedette{pe-phaza-hããxa}
\région{GOs}
\end{entrée}

\begin{entrée}
{faire une fête}
\vedette{thu pwalu}
\région{PA BO [Corne]}
\end{entrée}

\begin{entrée}
{faire un signe de la main}
\vedette{pawe hii-je}
\région{GOs}
\end{entrée}

\begin{entrée}
{fête}
\vedette{fè}
\région{GOs}
\end{entrée}

\begin{entrée}
{garder pour soi; mettre en sûreté}
\classe{v.t.}
\vedette{kae}\homonyme{1}
\sens{2}
\région{GOs PA WEM WE BO}
\end{entrée}

\begin{entrée}
{garder secret}
\classe{v}
\vedette{thozoe}
\sens{2}
\région{GOs}
\variante{%
\vedette{toroe}
\région{PA BO}}
\end{entrée}

\begin{entrée}
{garder ; surveiller}
\vedette{hôbwo}
\région{GOs}
\variante{%
\vedette{hôbo}
\région{GO(s) PA}}
\end{entrée}

\begin{entrée}
{gâter (enfant) ; vanter (se)}
\vedette{paçô}
\région{GOs}
\end{entrée}

\begin{entrée}
{gronder ; disputer}
\vedette{pana}
\région{GOs BO}
\end{entrée}

\begin{entrée}
{guide}
\vedette{a-whili}
\région{GOs}
\variante{%
\vedette{a-huli}
\région{PA}}
\end{entrée}

\begin{entrée}
{héler ; faire signe}
\vedette{bua}
\région{GOs PA BO}
\end{entrée}

\begin{entrée}
{hésiter}
\vedette{hããxa}
\région{GOs BO}
\variante{%
\vedette{haaka}
\région{BO}}
\end{entrée}

\begin{entrée}
{humble (se faire) ; petit (se faire) ; abaisser (s') (marque de respect dans les discours coutumiers)}
\vedette{pegaabe}
\région{GOs}
\end{entrée}

\begin{entrée}
{imiter}
\vedette{pe-pwawè}
\région{GOs}
\variante{%
\vedette{pe-pwawèn}
\région{PA}}
\end{entrée}

\begin{entrée}
{imiter ; imitation[Corne]}
\vedette{poxawèo}
\région{BO}
\variante{%
\vedette{pwawèo}
\région{BO}}
\variante{%
\vedette{pokaweo}
\région{BO}}
\end{entrée}

\begin{entrée}
{imiter ; singer}
\vedette{kaweeng}
\région{PA BO [BM]}
\end{entrée}

\begin{entrée}
{injure ; insulte ; offense}
\vedette{paxa-nãã-n}
\région{PA BO}
\end{entrée}

\begin{entrée}
{injure ; offense ; affront ; calomnie ; mauvais sort}
\vedette{nãã-n}
\région{PA BO}
\end{entrée}

\begin{entrée}
{injurier ; offenser}
\vedette{tòè-nããn}
\région{PA BO}
\end{entrée}

\begin{entrée}
{injurier ; offenser (lit. jeter offense)}
\vedette{phao nã}
\région{GOs}
\variante{%
\vedette{phao nããn}
\région{PA}}
\end{entrée}

\begin{entrée}
{insister ; demander avec insistance ; persister à (sens négatif) ; s'entêter}
\vedette{kò-çãńã}
\région{GOs}
\variante{%
\vedette{kò-cãnã}
\région{PA}}
\end{entrée}

\begin{entrée}
{interdire}
\classe{v.t.}
\vedette{thôni}
\sens{2}
\région{GOs WEM WE PA BO}
\end{entrée}

\begin{entrée}
{interdire qqch (momentanément) ; empêcher (qqn de faire qqch)}
\vedette{paxeze}
\région{GOs}
\end{entrée}

\begin{entrée}
{jouer des tours}
\classe{v ; n}
\vedette{trûã}\homonyme{2}
\sens{1}
\région{GOs WEM}
\variante{%
\vedette{thûã}
\région{PA}}
\variante{%
\vedette{tûãn}
\région{BO}}
\end{entrée}

\begin{entrée}
{jouer (se) des tours ; taquiner (se)}
\vedette{pe-chôã}
\région{GOs}
\end{entrée}

\begin{entrée}
{jurer (lit. parler mauvais)}
\vedette{vhaa-raa}
\région{GOs PA}
\end{entrée}

\begin{entrée}
{jurer (que c'est vrai)}
\vedette{hôbòl}
\région{PA}
\end{entrée}

\begin{entrée}
{laisser}
\vedette{khagebwa}
\région{GOs}
\variante{%
\vedette{khagebwan}
\région{PA}}
\end{entrée}

\begin{entrée}
{laisser}
\vedette{khagee}
\région{GOs}
\variante{%
\vedette{keege}
\région{BO [BM]}}
\end{entrée}

\begin{entrée}
{lancer des injures, des offenses (piquer)}
\vedette{tòè-nããn}
\région{PA BO}
\end{entrée}

\begin{entrée}
{leçon donnée pour faire réfléchir qqn}
\vedette{nãã-n}
\région{PA BO}
\end{entrée}

\begin{entrée}
{malédiction}
\vedette{hua}
\région{GOs}
\variante{%
\vedette{ua}
\région{GO(s)}}
\variante{%
\vedette{uany}
\région{BO [BM]}}
\variante{%
\vedette{whany}
\région{PA}}
\end{entrée}

\begin{entrée}
{malédiction ; mauvais sort}
\vedette{thoda}
\région{PA}
\end{entrée}

\begin{entrée}
{malédiction ; punition ; punir}
\vedette{whany}
\région{PA BO}
\variante{%
\vedette{wany}
\région{PA BO}}
\end{entrée}

\begin{entrée}
{manquer}
\classe{v}
\vedette{kã}
\sens{3}
\région{GOs}
\variante{%
\vedette{kãm}
\région{BO}}
\variante{%
\vedette{kham}
\région{PA}}
\end{entrée}

\begin{entrée}
{maudire}
\vedette{pe-zööni}
\région{GOs}
\end{entrée}

\begin{entrée}
{maudire}
\vedette{thöö}\homonyme{1}
\région{GOs PA}
\variante{%
\vedette{thoo}
\région{BO}}
\end{entrée}

\begin{entrée}
{maudire}
\vedette{thööni}
\région{GOs PA}
\end{entrée}

\begin{entrée}
{mauvais sort}
\vedette{paxa-nãã-n}
\région{PA BO}
\end{entrée}

\begin{entrée}
{médire ; médisance}
\vedette{pexu}
\région{GOs PA BO}
\end{entrée}

\begin{entrée}
{menacer [Corne]}
\vedette{caaxai}
\région{BO}
\end{entrée}

\begin{entrée}
{mentir ; mensonge}
\classe{v ; n}
\vedette{trûã}\homonyme{2}
\sens{1}
\région{GOs WEM}
\variante{%
\vedette{thûã}
\région{PA}}
\variante{%
\vedette{tûãn}
\région{BO}}
\end{entrée}

\begin{entrée}
{mentir ; mensonge [Corne]}
\vedette{lòn}
\région{BO}
\end{entrée}

\begin{entrée}
{montrer ses fesses}
\vedette{khîbö}
\région{GOs}
\end{entrée}

\begin{entrée}
{moquer de (se)}
\vedette{uza}
\région{GOs}
\variante{%
\vedette{ula}
\région{PA BO}}
\end{entrée}

\begin{entrée}
{moquer (se)}
\vedette{pe-rulai}
\région{PA BO [Corne]}
\end{entrée}

\begin{entrée}
{moquer (se) ; mépriser}
\vedette{thaxiba}
\sens{2}
\région{GOs PA}
\variante{%
\vedette{thakiba}
\région{GO(s)}}
\variante{%
\vedette{taxiba}
\région{BO PA}}
\end{entrée}

\begin{entrée}
{négliger ; délaisser ; abandonner}
\vedette{paree}
\région{PA BO}
\end{entrée}

\begin{entrée}
{nier}
\vedette{woxa}
\région{GOs WEM BO}
\variante{%
\vedette{voxa}
\région{BO}}
\end{entrée}

\begin{entrée}
{nom du don pour la la demande de pardon}
\vedette{uvwa}
\région{GOs}
\end{entrée}

\begin{entrée}
{obéir [BM]}
\vedette{kaageen}
\région{BO}
\end{entrée}

\begin{entrée}
{obliger}
\vedette{waaçu}
\région{GOs WEM WE}
\variante{%
\vedette{waçuçu}
\région{GO(s) WEM WE}}
\variante{%
\vedette{waayu, waaju}
\région{PA BO}}
\end{entrée}

\begin{entrée}
{offense ; injure}
\vedette{phao paxa-nãn}
\région{PA}
\end{entrée}

\begin{entrée}
{pardonner (se)}
\vedette{pe-ne-zo-ni}
\région{GOs}
\end{entrée}

\begin{entrée}
{pardonner (se) (lit. attacher le pardon)}
\vedette{pe-nhoi thria}
\région{GOs}
\end{entrée}

\begin{entrée}
{pardonner (se) mutuellement}
\vedette{pe-tò do}
\région{PA BO}
\end{entrée}

\begin{entrée}
{pardonner (se) (pardon coutumier)}
\vedette{pe-tha thria}
\région{GOs}
\variante{%
\vedette{pe-tha thia}
\région{PA}}
\end{entrée}

\begin{entrée}
{pareil (être) ; égal (Dubois)}
\classe{v}
\vedette{pe-thilò}
\sens{2}
\région{GOs}
\variante{%
\vedette{pe-dilo}
\région{BO}}
\end{entrée}

\begin{entrée}
{partenaire ; co-équipier ; complice (groupe de plus de 2 personnes)}
\vedette{bala}\homonyme{2}
\région{GOs BO}
\end{entrée}

\begin{entrée}
{perséverer ; persévérant}
\vedette{waaçu}
\région{GOs WEM WE}
\variante{%
\vedette{waçuçu}
\région{GO(s) WEM WE}}
\variante{%
\vedette{waayu, waaju}
\région{PA BO}}
\end{entrée}

\begin{entrée}
{personne servant de lien entre deux clans}
\classe{nom}
\vedette{hovalek}
\sens{2}
\région{PA}
\end{entrée}

\begin{entrée}
{piquer ; provoquer}
\vedette{thiipuun}
\région{PA}
\end{entrée}

\begin{entrée}
{pitre (qqn qui fait le)}
\vedette{a-thu kônya}
\région{GOs}
\end{entrée}

\begin{entrée}
{prendre le parti de qqn, défendre}
\vedette{kòòl kòlò}
\région{PA}
\end{entrée}

\begin{entrée}
{prendre qqch. par erreur}
\vedette{pe-thauvweni}
\région{GOs}
\end{entrée}

\begin{entrée}
{prêter serment}
\vedette{thu bwahî}
\région{GOs}
\end{entrée}

\begin{entrée}
{protéger}
\vedette{ho}
\région{PA}
\end{entrée}

\begin{entrée}
{provoquer (par la parole ou l'attitude)}
\vedette{kilapuu}
\région{GOs}
\end{entrée}

\begin{entrée}
{provoquer (se)}
\vedette{ku-thralò}
\région{GOs}
\end{entrée}

\begin{entrée}
{provoquer (se)}
\vedette{pe-thi thô}
\région{GOs}
\end{entrée}

\begin{entrée}
{quémander}
\vedette{yaò}\homonyme{2}
\région{GOs}
\end{entrée}

\begin{entrée}
{quereller (se) ; disputer (se) ; gronder}
\vedette{pivwia}
\région{GOs}
\variante{%
\vedette{pivia}
\région{PA}}
\variante{%
\vedette{pevia}
\région{BO}}
\end{entrée}

\begin{entrée}
{quitter}
\vedette{khagee}
\région{GOs}
\variante{%
\vedette{keege}
\région{BO [BM]}}
\end{entrée}

\begin{entrée}
{rappeller (se) mutuellement}
\vedette{pe-pha-nõnõmi}
\région{GOs}
\end{entrée}

\begin{entrée}
{rapporter ; dénoncer}
\vedette{zaçixõõni}
\région{GOs}
\end{entrée}

\begin{entrée}
{rapporter ; dénoncer}
\vedette{zovaale}
\région{GOs}
\variante{%
\vedette{zovaale}
\région{PA}}
\variante{%
\vedette{yo-vhaale}
\région{BO [Corne]}}
\end{entrée}

\begin{entrée}
{récompense ; rétribution ; remerciement}
\vedette{kâdi}
\région{GOs PA BO}
\end{entrée}

\begin{entrée}
{refuser la demande de pardon}
\vedette{taçuuni}
\région{GOs}
\end{entrée}

\begin{entrée}
{refuser ; rejeter ; chasser [PA] ; congédier}
\vedette{thaxiba}
\sens{1}
\région{GOs PA}
\variante{%
\vedette{thakiba}
\région{GO(s)}}
\variante{%
\vedette{taxiba}
\région{BO PA}}
\end{entrée}

\begin{entrée}
{refuser ; rejeter ; désobéir}
\vedette{kue}
\région{GOs}
\variante{%
\vedette{kuel, kwel}
\région{BO}}
\end{entrée}

\begin{entrée}
{rejeter}
\vedette{khagebwa}
\région{GOs}
\variante{%
\vedette{khagebwan}
\région{PA}}
\end{entrée}

\begin{entrée}
{rejeter ; refuser ; aimer (ne pas)}
\vedette{kuel}
\région{BO PA}
\end{entrée}

\begin{entrée}
{rejeter ; refuser ; détester ; aimer (ne pas)}
\vedette{kuele}
\région{GOs BO}
\end{entrée}

\begin{entrée}
{relation réciproque}
\vedette{a-vwe bulu}
\région{GOs}
\end{entrée}

\begin{entrée}
{remercier ; merci}
\vedette{ole}
\région{GOs PA BO}
\end{entrée}

\begin{entrée}
{remplaçant}
\vedette{a-wône-vwo}
\région{GOs}
\end{entrée}

\begin{entrée}
{rencontrer par hasard ; rattraper}
\vedette{kha-tròòli}
\région{GOs}
\end{entrée}

\begin{entrée}
{rencontrer (se)}
\vedette{pe-trò}
\région{GOs}
\end{entrée}

\begin{entrée}
{rencontrer, trouver qqn}
\vedette{tòò}\homonyme{2}
\région{PA BO}
\end{entrée}

\begin{entrée}
{rendre (qqch)}
\vedette{na mwã}
\région{PA}
\end{entrée}

\begin{entrée}
{repentir (se) ; demander pardon ; confesser (se)}
\vedette{giçaò}
\région{GOs}
\end{entrée}

\begin{entrée}
{respecter ; honorer}
\vedette{pue}
\région{GOs}
\end{entrée}

\begin{entrée}
{respect ; respecter}
\vedette{thu pwaalu}
\région{GOs BO [Corne]}
\end{entrée}

\begin{entrée}
{réunir ; rassembler (gens)}
\classe{v}
\vedette{tha-ivwi}
\sens{2}
\région{GOs PA}
\variante{%
\vedette{thaiving}
\région{PA BO}}
\end{entrée}

\begin{entrée}
{réunir (se)}
\vedette{pe-thaivwi}
\région{GOs PA}
\end{entrée}

\begin{entrée}
{réunir (se) pour délibérer}
\vedette{thaavi}
\région{GOs}
\variante{%
\vedette{thaaviing}
\région{BO [BM]}}
\end{entrée}

\begin{entrée}
{saluer [BM]}
\vedette{yaxi}
\région{BO}
\end{entrée}

\begin{entrée}
{saluer (se)}
\vedette{bosu}
\région{GOs}
\end{entrée}

\begin{entrée}
{soutien ; appui}
\vedette{mwãã}
\région{GOs}
\end{entrée}

\begin{entrée}
{subvenir aux besoins}
\vedette{zaalae}
\région{GOs}
\variante{%
\vedette{zhaalae}
\région{GA}}
\end{entrée}

\begin{entrée}
{surprendre ; cachette (faire en) ; faire doucement}
\vedette{caaxo}
\région{GOs BO PA}
\variante{%
\vedette{kyaaxo}
\région{BO [BM]}}
\variante{%
\vedette{caawo}
\région{BO (Corne)}}
\end{entrée}

\begin{entrée}
{taquiner qqn ; embêter}
\vedette{phaçegai}
\région{GOs}
\variante{%
\vedette{pha-caxai}
\région{PA}}
\end{entrée}

\begin{entrée}
{transgresser (interdit)}
\classe{v}
\vedette{khau}
\sens{3}
\région{GOs PA BO}
\end{entrée}

\begin{entrée}
{transgresser (règle, interdit)}
\classe{v.t.}
\vedette{khau-ni}
\sens{2}
\end{entrée}

\begin{entrée}
{tromper (se) en prenant qqch}
\vedette{pe-thauvweni}
\région{GOs}
\end{entrée}

\begin{entrée}
{vrai ; vérité}
\vedette{gumãgu}
\région{GOs PA BO}
\end{entrée}

\subsubsection{Richesses, monnaies traditionnelles}

\begin{entrée}
{affaires ; objets ; biens ; choses}
\classe{nom}
\vedette{yada}
\sens{1}
\région{GOs PA BO}
\end{entrée}

\begin{entrée}
{argent}
\vedette{mwani}
\région{GOs PA BO}
\end{entrée}

\begin{entrée}
{argent ; monnaie}
\vedette{nhõõl}
\région{BO PA}
\variante{%
\vedette{nõõl}
\région{BO PA}}
\end{entrée}

\begin{entrée}
{biens ; affaires}
\vedette{zo}\homonyme{3}
\région{GOs}
\variante{%
\vedette{zho}
\région{GO(s)}}
\end{entrée}

\begin{entrée}
{perles de verre [BM]}
\vedette{pò-baa}
\région{BO}
\end{entrée}

\begin{entrée}
{richesses ; biens}
\vedette{nhõõl}
\région{BO PA}
\variante{%
\vedette{nõõl}
\région{BO PA}}
\end{entrée}

\subsubsection{Dons, échanges, achat et vente, vol}

\begin{entrée}
{acheter [PA]}
\vedette{içu}\homonyme{1}
\région{GOs}
\variante{%
\vedette{iyu}
\région{PA}}
\end{entrée}

\begin{entrée}
{acheter ; payer}
\vedette{uvwi}
\région{GOs}
\variante{%
\vedette{upi}
\région{GO(s)}}
\variante{%
\vedette{uvi}
\région{PA BO}}
\end{entrée}

\begin{entrée}
{avare, égoïste, qui ne partage pas}
\vedette{aa-palu}
\région{GOs}
\end{entrée}

\begin{entrée}
{avare (être) ; refuser de donner}
\vedette{palu}
\région{GOs BO}
\end{entrée}

\begin{entrée}
{distribuer en partage}
\vedette{thi-tou}
\région{GOs PA}
\end{entrée}

\begin{entrée}
{don ; cadeau}
\vedette{hê-}
\sens{2}
\région{GOs}
\variante{%
\vedette{hêê-n}
\région{PA BO}}
\end{entrée}

\begin{entrée}
{donner}
\classe{v}
\vedette{naa}\homonyme{3}
\sens{1}
\région{GOs}
\variante{%
\vedette{na, ne}
\région{PA BO}}
\end{entrée}

\begin{entrée}
{don ; offrande}
\vedette{dou}\homonyme{1}
\région{GOs}
\end{entrée}

\begin{entrée}
{marché ; échange de marchandises}
\vedette{jana}
\région{GOs BO}
\end{entrée}

\begin{entrée}
{partage (dans les fêtes coutumières)}
\vedette{tou}
\groupe{A}
\région{GOs PA BO}
\end{entrée}

\begin{entrée}
{partager ; distribuer}
\vedette{tou}
\groupe{A}
\région{GOs PA BO}
\end{entrée}

\begin{entrée}
{prêter}
\vedette{tèè-na}
\région{PA BO}
\end{entrée}

\begin{entrée}
{prix ; salaire}
\vedette{uvwi}
\région{GOs}
\variante{%
\vedette{upi}
\région{GO(s)}}
\variante{%
\vedette{uvi}
\région{PA BO}}
\end{entrée}

\begin{entrée}
{récompenser ; payer ; rétribuer}
\vedette{pwawe}
\région{BO [BM]}
\end{entrée}

\begin{entrée}
{vendre ; commercer}
\vedette{içu}\homonyme{1}
\région{GOs}
\variante{%
\vedette{iyu}
\région{PA}}
\end{entrée}

\begin{entrée}
{voler ; dérober}
\vedette{phèńô}
\région{GOs}
\variante{%
\vedette{pènô}
\région{PA BO}}
\end{entrée}

\begin{entrée}
{voleur}
\vedette{a-phènô}
\région{GOs}
\end{entrée}

\subsubsection{Objets coutumiers}

\begin{entrée}
{affaires (vêtements de qqn)}
\vedette{puçiu}
\région{GOs}
\end{entrée}

\begin{entrée}
{biens personnels d'un défunt qu'on remet à ses maternels}
\vedette{puçiu}
\région{GOs}
\end{entrée}

\begin{entrée}
{bouquet de fibres (accrochés à un piquet pour marquer un interdit ou montrer qu'une plante est réservée)}
\classe{nom}
\vedette{vijang}
\sens{1}
\région{PA BO}
\end{entrée}

\begin{entrée}
{bouquet de plante contenant une monnaie et entouré d'un lien de paille}
\vedette{daluça mã}
\région{GOs}
\variante{%
\vedette{ba-oginen}
\région{PA BO}}
\end{entrée}

\begin{entrée}
{ceinture d'écorce (tressée de 3 brins de paille ou de feuilles de pandanus ; signe de départ en guerre. Dubois ms)}
\vedette{wa-bwanu}
\région{PA BO}
\end{entrée}

\begin{entrée}
{ceinture de femme finement tressée}
\vedette{pobil}
\région{PA BO}
\end{entrée}

\begin{entrée}
{ceinture de femme (litt. we-pòò 'racine de bourao')}
\vedette{wepòò}
\région{PA BO}
\end{entrée}

\begin{entrée}
{ceinture de femme tressée (monnaie)}
\vedette{thabil}
\région{PA}
\end{entrée}

\begin{entrée}
{ceinture faite en fibre d'écorce de banian ou en 'alamwi', dans laquelle on mettait les pierres et les paquets magiques pour la guerre}
\vedette{wa-bwanu}
\région{PA BO}
\end{entrée}

\begin{entrée}
{chambranles}
\vedette{drògò}
\région{GOs WEM}
\variante{%
\vedette{dògò}
\région{PA}}
\end{entrée}

\begin{entrée}
{corde de monnaie de coquillage (Dubois)}
\classe{nom}
\vedette{alamwi}
\sens{1}
\région{BO PA}
\end{entrée}

\begin{entrée}
{feuille de bananier qui enveloppe la monnaie 'weem' (Charles)}
\vedette{dopweza}
\région{PA BO}
\end{entrée}

\begin{entrée}
{feuille de bananier (voir 'pweza')}
\vedette{dopweza}
\région{PA BO}
\end{entrée}

\begin{entrée}
{longueur de monnaie}
\vedette{nu hêgi}
\région{GOs PA}
\end{entrée}

\begin{entrée}
{masque (comprenant l'habit qui accompagne le masque)}
\vedette{drògò}
\région{GOs WEM}
\variante{%
\vedette{dògò}
\région{PA}}
\end{entrée}

\begin{entrée}
{monnaie}
\vedette{pwãmwãnu}
\région{PA}
\variante{%
\vedette{pobwanu}
\région{BO}}
\end{entrée}

\begin{entrée}
{monnaie}
\vedette{yòò}\homonyme{2}
\région{PA BO}
\end{entrée}

\begin{entrée}
{monnaie (de moins grande valeur que 'yòò')}
\vedette{weem}
\région{PA BO}
\end{entrée}

\begin{entrée}
{monnaie (Dubois : 1 dopweza de 2,5 m vaut 20 francs)}
\vedette{dopweza}
\région{PA BO}
\end{entrée}

\begin{entrée}
{monnaie kanak}
\vedette{kòòl}
\région{PA}
\end{entrée}

\begin{entrée}
{monnaie kanak}
\vedette{tabwa}
\région{PA}
\end{entrée}

\begin{entrée}
{monnaie kanak}
\vedette{yhalo}
\région{PA}
\end{entrée}

\begin{entrée}
{monnaie kanak (de la longueur d'un avant-bras, Charles)}
\classe{nom}
\vedette{gò-hii}
\sens{2}
\région{PA}
\end{entrée}

\begin{entrée}
{monnaie traditionnelle}
\vedette{hêgi}
\région{GOs PA BO}
\end{entrée}

\begin{entrée}
{noeud (fait sur une herbe, signale un interdit, protège des magies)}
\classe{nom}
\vedette{vijang}
\sens{2}
\région{PA BO}
\end{entrée}

\begin{entrée}
{ossature de la monnaie}
\vedette{du-hegi}
\région{GOs}
\end{entrée}

\begin{entrée}
{perche avec un paquet (signale un interdit)}
\classe{nom}
\vedette{nõbu}
\sens{1}
\région{GOs}
\variante{%
\vedette{nõbu}
\région{BO PA}}
\end{entrée}

\begin{entrée}
{perche qui annonce le décès d'un chef}
\vedette{ce-thri}
\région{GOs}
\end{entrée}

\begin{entrée}
{perche sacrée du champ d'igname (destinée à favoriser la récolte et protéger les plantations)}
\classe{nom}
\vedette{wòzò}
\sens{2}
\région{GOs}
\variante{%
\vedette{wòlò}
\région{PA BO}}
\variante{%
\vedette{wojò}
\région{BO}}
\end{entrée}

\begin{entrée}
{perches devant la porte}
\vedette{cee-xòò}
\région{PA}
\variante{%
\vedette{ce-kòòl}
\région{BO (Corne)}}
\end{entrée}

\begin{entrée}
{perche signalant un interdit [Corne]}
\vedette{ce-kabun}
\région{BO}
\end{entrée}

\begin{entrée}
{perche signalant un interdit [Corne]}
\vedette{ce-nôbu}
\région{BO}
\end{entrée}

\begin{entrée}
{perches plantées devant la maison (Charles)}
\vedette{ciixo}
\région{PA}
\end{entrée}

\begin{entrée}
{perches plantées devant les portes des cases (Dubois)}
\vedette{cego}
\région{PA}
\end{entrée}

\begin{entrée}
{plumet de monnaie}
\vedette{muzi}
\région{GOs}
\variante{%
\vedette{mulin}
\région{PA}}
\end{entrée}

\begin{entrée}
{rouleau d'étoffe}
\vedette{maû}\homonyme{2}
\région{GOs PA}
\end{entrée}

\begin{entrée}
{sculpture faîtière}
\vedette{drògò}
\région{GOs WEM}
\variante{%
\vedette{dògò}
\région{PA}}
\end{entrée}

\begin{entrée}
{signe (interdisant de toucher à qqch.)}
\classe{nom}
\vedette{nõbu}
\sens{1}
\région{GOs}
\variante{%
\vedette{nõbu}
\région{BO PA}}
\end{entrée}

\subsubsection{Coutumes, dons coutumiers}

\begin{entrée}
{actes coutumiers ; coutumes}
\vedette{wèdò}
\région{BO PA}
\variante{%
\vedette{wôdo}
\région{PA}}
\end{entrée}

\begin{entrée}
{actes coutumiers ; us et coutumes ; usages ; manières ; moeurs}
\classe{v}
\vedette{wòdro}
\sens{2}
\région{GOs}
\variante{%
\vedette{wèdò, vòdòòn}
\région{BO}}
\end{entrée}

\begin{entrée}
{aligner les ignames ; faire des tas d'ignames (pour les cérémonies)}
\vedette{phawe}
\région{GOs BO}
\end{entrée}

\begin{entrée}
{cérémonie coutumière}
\vedette{na-vwo}
\région{GOs PA BO}
\end{entrée}

\begin{entrée}
{cérémonie de deuil aux maternels}
\classe{v ; n}
\vedette{mõõdim}
\sens{1}
\région{PA BO}
\end{entrée}

\begin{entrée}
{cérémonie de la nouvelle igname (lit. se présenter les ignames)}
\vedette{pe-phããde kui}
\région{GOs}
\end{entrée}

\begin{entrée}
{champ d'igname du chef}
\vedette{yaa-wòlò}
\région{PA BO}
\end{entrée}

\begin{entrée}
{champ d'igname sacré du chef (que l'on défriche)}
\vedette{yaa-wòzò}
\région{GOs}
\variante{%
\vedette{yaa-wòlò}
\région{WEM WE}}
\variante{%
\vedette{ya-wòjò}
\région{BO}}
\end{entrée}

\begin{entrée}
{conséquence ; marque}
\vedette{hâwâ}
\région{GOs}
\variante{%
\vedette{haawa}
\région{BO [Corne]}}
\end{entrée}

\begin{entrée}
{contre-don (dans les dons coutumiers)}
\classe{v ; n}
\vedette{mwãju}
\sens{2}
\région{GOs PA BO}
\région{GOs}
\variante{%
\vedette{mwèju}}
\variante{%
\vedette{mwaji}
\région{BO}}
\end{entrée}

\begin{entrée}
{coutume (cérémonie coutumière) ; fête ; occupations}
\vedette{nhyãã}
\région{GOs}
\variante{%
\vedette{nhyang}
\région{PA BO}}
\end{entrée}

\begin{entrée}
{coutume (cérémonie) de deuil des femmes issues de la chefferie}
\vedette{mazao}
\région{GOs}
\end{entrée}

\begin{entrée}
{coutume (cérémonie) de deuil: don du clan paternel au clan maternel de l'épouse ou de l'époux}
\vedette{hauva}
\région{GOs PA BO}
\end{entrée}

\begin{entrée}
{coutume (cérémonie) de mariage (lit. chercher la femme)}
\vedette{whili thòòmwa}
\région{GOs}
\variante{%
\vedette{huli thòòmwa}
\région{WE}}
\end{entrée}

\begin{entrée}
{coutume (cérémonie) ou don coutumier}
\vedette{mãû}
\région{GOs PA}
\end{entrée}

\begin{entrée}
{coutume de départ}
\vedette{ba-alawe}
\région{GOs}
\end{entrée}

\begin{entrée}
{coutume de deuil}
\vedette{phunò mã}
\région{GOs}
\end{entrée}

\begin{entrée}
{coutume (de deuil ou de mariage)}
\vedette{phunò}
\région{GOs}
\end{entrée}

\begin{entrée}
{coutume de deuil pour une femme}
\vedette{mãû bweera}
\région{GOs}
\variante{%
\vedette{mhãû}
\région{PA}}
\end{entrée}

\begin{entrée}
{coutumes (cérémonies)}
\vedette{nhyang mòlò}
\région{PA}
\end{entrée}

\begin{entrée}
{coutumes (cérémonies)}
\vedette{nhyang mhã}
\région{PA}
\end{entrée}

\begin{entrée}
{coutumes de deuil du chef : levée de deuil d'un grand chef}
\vedette{thri}
\région{GOs WEM}
\variante{%
\vedette{thiing, thing}
\région{BO [BM}}
\variante{%
\vedette{thiin}
\région{BO [Corne]}}
\end{entrée}

\begin{entrée}
{désigner un tas pour un clan (coutume)}
\classe{v}
\vedette{thooni}
\sens{2}
\région{GOs PA}
\end{entrée}

\begin{entrée}
{deuil ; coutume de deuil}
\vedette{mõõdi}
\sens{2}
\région{GOs}
\variante{%
\vedette{mõõdim}
\région{WEM WE PA BO}}
\end{entrée}

\begin{entrée}
{deuil de chef}
\vedette{thri}
\région{GOs WEM}
\variante{%
\vedette{thiing, thing}
\région{BO [BM}}
\variante{%
\vedette{thiin}
\région{BO [Corne]}}
\end{entrée}

\begin{entrée}
{don à la mère pour avoir élevé un garçon (se fait vers 7-8 ans)}
\vedette{nhyanga-n}
\région{PA BO [BM]}
\end{entrée}

\begin{entrée}
{don à la mère pour la naissance d'un enfant}
\vedette{phwau-n}
\région{PA BO [BM]}
\end{entrée}

\begin{entrée}
{don lors d'une adoption (pour prendre l'enfant)}
\vedette{ba-phe-ẽnõ}
\région{PA}
\end{entrée}

\begin{entrée}
{don pour saluer les hôtes quand on arrive quelque part (lit. fin du chemin)}
\vedette{hulò-mhèńõ}
\région{GOs}
\variante{%
\vedette{hulò-mhee-n}
\région{PA BO}}
\end{entrée}

\begin{entrée}
{dons coutumiers}
\vedette{na-vwo}
\région{GOs PA BO}
\end{entrée}

\begin{entrée}
{effet (d'une parole, d'un médicament, etc.)}
\vedette{hâwâ}
\région{GOs}
\variante{%
\vedette{haawa}
\région{BO [Corne]}}
\end{entrée}

\begin{entrée}
{empilement d'ignames}
\vedette{pe-na-bwa na kui}
\région{GOs}
\end{entrée}

\begin{entrée}
{enlever les dons lors descérémonies coutumières}
\vedette{kaaluni}
\région{GOs PA}
\end{entrée}

\begin{entrée}
{enlever un interdit}
\vedette{phu nõbu}
\région{GOs}
\end{entrée}

\begin{entrée}
{enlever un tabou ; rendre profane [Corne]}
\vedette{pha-tuya}
\région{BO}
\end{entrée}

\begin{entrée}
{faire une cérémonie coutumière ; faire les échanges coutumiers}
\vedette{pe-navwo}
\région{GOs PA}
\end{entrée}

\begin{entrée}
{faire une demande coutumière (lit. donner le feu)}
\vedette{naa yaai}
\région{GOs}
\end{entrée}

\begin{entrée}
{festoyer}
\classe{v}
\vedette{bwaçu}
\sens{2}
\région{GOs}
\variante{%
\vedette{bwaju}
\région{WEM WE}}
\variante{%
\vedette{bwayu}
\région{PA}}
\end{entrée}

\begin{entrée}
{fête des prémices des ignames}
\classe{nom}
\vedette{khî-kui}
\groupe{B}
\sens{2}
\région{GOs PA BO}
\end{entrée}

\begin{entrée}
{fête des prémices des ignames (lit. montrer l'igname)}
\vedette{phããde kui}
\région{GOs}
\end{entrée}

\begin{entrée}
{fêtes coutumières [BO]}
\classe{nom}
\vedette{yada}
\sens{2}
\région{GOs PA BO}
\end{entrée}

\begin{entrée}
{feuilles enfouies pour la fécondité des femmes (Dubois)}
\vedette{napoine}
\région{BO}
\end{entrée}

\begin{entrée}
{geste de respect}
\vedette{ba-thu-pwaalu}
\région{GOs}
\end{entrée}

\begin{entrée}
{hache ostensoir}
\classe{nom}
\vedette{mãgi}\homonyme{2}
\sens{1}
\région{GOs}
\variante{%
\vedette{mwãgi}
\région{PA BO}}
\end{entrée}

\begin{entrée}
{herbe (contexte cérémoniel uniquement)}
\vedette{zano}
\région{PA}
\end{entrée}

\begin{entrée}
{invoquer ; parler aux esprits}
\vedette{zao}
\région{GOs}
\variante{%
\vedette{cao}
\région{GO}}
\end{entrée}

\begin{entrée}
{lien (coutumier)}
\vedette{wathrã}
\région{GOs}
\end{entrée}

\begin{entrée}
{massif calendrier}
\vedette{yaa-wòzò}
\région{GOs}
\variante{%
\vedette{yaa-wòlò}
\région{WEM WE}}
\variante{%
\vedette{ya-wòjò}
\région{BO}}
\end{entrée}

\begin{entrée}
{occuper (s')}
\vedette{nhyãã}
\région{GOs}
\variante{%
\vedette{nhyang}
\région{PA BO}}
\end{entrée}

\begin{entrée}
{ornement de deuil}
\vedette{phunò}
\région{GOs}
\end{entrée}

\begin{entrée}
{pardon coutumier}
\vedette{thria}
\région{GOs}
\variante{%
\vedette{thia}
\région{PA}}
\end{entrée}

\begin{entrée}
{pardonner (se) (pardon coutumier)}
\vedette{pe-tha thria}
\région{GOs}
\variante{%
\vedette{pe-tha thia}
\région{PA}}
\end{entrée}

\begin{entrée}
{partager (partage coutumier des ignames)}
\vedette{pe-tou}
\région{GOs BO}
\end{entrée}

\begin{entrée}
{pierre ; couteau fait dans cette pierre et utilisée pour la circoncision [Corne]}
\vedette{paa-neng}
\région{BO}
\end{entrée}

\begin{entrée}
{prestation de deuil aux maternels}
\vedette{kênim}
\région{PA}
\end{entrée}

\begin{entrée}
{rassembler de la nourriture (pour une coutume)}
\classe{v}
\vedette{phiige}
\sens{1}
\région{GOs}
\variante{%
\vedette{peenge}
\région{BO [BM]}}
\end{entrée}

\begin{entrée}
{recevoir un geste coutumier (en remerciant par un geste en retour)}
\classe{v}
\vedette{kibao}
\sens{2}
\région{GOs PA BO}
\end{entrée}

\begin{entrée}
{réfère aux cérémonie funéraires (perturbation de l'ordre social: D. Bretteville)}
\classe{v ; n}
\vedette{tigi}\homonyme{2}
\sens{3}
\région{GOs PA BO}
\end{entrée}

\begin{entrée}
{rendre (la monnaie)}
\classe{v ; n}
\vedette{mwãju}
\sens{2}
\région{GOs PA BO}
\région{GOs}
\variante{%
\vedette{mwèju}}
\variante{%
\vedette{mwaji}
\région{BO}}
\end{entrée}

\begin{entrée}
{répudier (femme)}
\vedette{thaxiba}
\sens{3}
\région{GOs PA}
\variante{%
\vedette{thakiba}
\région{GO(s)}}
\variante{%
\vedette{taxiba}
\région{BO PA}}
\end{entrée}

\begin{entrée}
{suivre les usages}
\vedette{kae wòdo}
\région{GOs}
\variante{%
\vedette{kae wedo}
\région{PA}}
\end{entrée}

\begin{entrée}
{tas de vivres}
\vedette{page}
\région{GOs BO}
\variante{%
\vedette{pwaxe}
\région{BO}}
\end{entrée}

\begin{entrée}
{tas d'igname (dans les cérémonies)}
\vedette{phò-kui}
\région{GOs}
\end{entrée}

\begin{entrée}
{tas d'ignames}
\vedette{poge kui}
\région{GOs}
\end{entrée}

\begin{entrée}
{tas (d'ignames, de pierres)}
\vedette{phawe}
\région{GOs BO}
\end{entrée}

\begin{entrée}
{unir (s'), rassembler (se) (contexte coutumier)}
\classe{v ; n}
\vedette{mhõge}
\sens{2}
\région{GO PA BO}
\end{entrée}

\begin{entrée}
{usages ; manières ; moeurs}
\vedette{wèdò}
\région{BO PA}
\variante{%
\vedette{wôdo}
\région{PA}}
\end{entrée}

\begin{entrée}
{us et coutumes}
\vedette{nee-wo mani phwe-wedevwo}
\région{GOs}
\end{entrée}

\begin{entrée}
{us et coutumes ; comportement ; attitude ; manière}
\vedette{mwêêje}
\région{GOs PA}
\end{entrée}

\subsection{Religion, représentations religieuses}

\begin{entrée}
{appliquer (médicament) ; traiter}
\vedette{kia}
\région{PA BO [BM]}
\variante{%
\vedette{khia}
\région{BO [BM]}}
\end{entrée}

\begin{entrée}
{autel près de la case (Corne, Dubois)}
\vedette{pwayòò}
\région{GOs BO}
\variante{%
\vedette{pwajòl}
\région{BO}}
\end{entrée}

\begin{entrée}
{Bible}
\vedette{tiivwo kabu}
\région{GOs}
\end{entrée}

\begin{entrée}
{blasphémer ; blasphème (Dubois)}
\vedette{tene}
\région{BO}
\end{entrée}

\begin{entrée}
{catholique}
\vedette{katòli}
\région{GOs}
\variante{%
\vedette{katòlik}
\région{PA}}
\end{entrée}

\begin{entrée}
{croix}
\vedette{pixãge}
\région{GOs}
\end{entrée}

\begin{entrée}
{dieu des enfers}
\vedette{pijopa}
\région{GO}
\end{entrée}

\begin{entrée}
{divinité qui favorise la pluie}
\vedette{Thulixâân}
\région{PA}
\end{entrée}

\begin{entrée}
{église ; temple}
\vedette{mõ-kabun}
\région{BO}
\end{entrée}

\begin{entrée}
{église ; temple}
\vedette{mwa-kabu}
\région{GO}
\end{entrée}

\begin{entrée}
{ensorceler ; jeter des sorts}
\vedette{phònõ}
\région{GOs}
\variante{%
\vedette{phònõng}
\région{PA}}
\end{entrée}

\begin{entrée}
{ensorceller [BM, Corne]}
\vedette{phènòng}
\région{BO}
\end{entrée}

\begin{entrée}
{esprit}
\vedette{jińu}\homonyme{1}
\région{GOs PA BO}
\end{entrée}

\begin{entrée}
{esprit ; âme}
\classe{v ; n}
\vedette{kãgu}
\sens{1}
\région{GOs}
\variante{%
\vedette{kãgun}
\région{BO PA}}
\end{entrée}

\begin{entrée}
{esprits des vieux du clan}
\vedette{whany}
\région{PA BO}
\variante{%
\vedette{wany}
\région{PA BO}}
\end{entrée}

\begin{entrée}
{esprit (se manifestant par une boule de feu dans la nuit)}
\vedette{dròòxi}
\région{GOs}
\variante{%
\vedette{dòki}
\région{BO}}
\end{entrée}

\begin{entrée}
{être aquatique (nom d'un)}
\vedette{mibwa}
\région{GOs}
\variante{%
\vedette{miibwan}
\région{BO}}
\end{entrée}

\begin{entrée}
{foi}
\vedette{trehuni}
\région{GOs}
\end{entrée}

\begin{entrée}
{génie forestier ; esprit ; ancêtre}
\vedette{kãbwa}
\région{GOs BO}
\variante{%
\vedette{kabwa}
\région{PA}}
\end{entrée}

\begin{entrée}
{incliner la tête}
\classe{v}
\vedette{bwagiloo}
\sens{2}
\région{GOs PA}
\end{entrée}

\begin{entrée}
{interdit ; sacré}
\vedette{kabu}
\région{GOs}
\variante{%
\vedette{kabun}
\région{PA BO}}
\end{entrée}

\begin{entrée}
{invoquer ; parler aux esprits}
\vedette{zao}
\région{GOs}
\variante{%
\vedette{cao}
\région{GO}}
\end{entrée}

\begin{entrée}
{lézard (être mythologique qui protège les cultures) [Corne]}
\vedette{gunebwa}
\région{PA BO}
\end{entrée}

\begin{entrée}
{lieu sacré}
\vedette{pwayòò}
\région{GOs BO}
\variante{%
\vedette{pwajòl}
\région{BO}}
\end{entrée}

\begin{entrée}
{lutins (petits êtres aux cheveux longs, qui jouent des tours aux humains)}
\vedette{jee}
\région{PA}
\end{entrée}

\begin{entrée}
{lutins ; petits génies}
\vedette{uramõ}
\région{GOs}
\end{entrée}

\begin{entrée}
{magie}
\vedette{dròòxi}
\région{GOs}
\variante{%
\vedette{dòki}
\région{BO}}
\end{entrée}

\begin{entrée}
{malédiction ; punition ; punir}
\vedette{whany}
\région{PA BO}
\variante{%
\vedette{wany}
\région{PA BO}}
\end{entrée}

\begin{entrée}
{mana}
\vedette{hubu}
\région{GOs}
\variante{%
\vedette{hubun, hubi}
\région{PA}}
\end{entrée}

\begin{entrée}
{médicament ; remède}
\vedette{kia}
\région{PA BO [BM]}
\variante{%
\vedette{khia}
\région{BO [BM]}}
\end{entrée}

\begin{entrée}
{mettre le tabou}
\vedette{khabe nõbu}
\région{GOs BO}
\end{entrée}

\begin{entrée}
{nain (lit. esprit de la maison)}
\vedette{kãgu-mwã}
\région{GOs}
\end{entrée}

\begin{entrée}
{présent déposé devant l'autel (Corne)}
\vedette{pwayòò}
\région{GOs BO}
\variante{%
\vedette{pwajòl}
\région{BO}}
\end{entrée}

\begin{entrée}
{prier}
\classe{v}
\vedette{mãxi}
\sens{2}
\région{GOs}
\variante{%
\vedette{mãxim}
\région{WEM WE}}
\variante{%
\vedette{mhãkim, mhãxim}
\région{BO}}
\variante{%
\vedette{mããxim}
\région{PA}}
\end{entrée}

\begin{entrée}
{prosterner (se)}
\classe{v}
\vedette{bwagiloo}
\sens{2}
\région{GOs PA}
\end{entrée}

\begin{entrée}
{prosterner (se)}
\vedette{kiluu}
\sens{2}
\région{GOs BO}
\variante{%
\vedette{ciluu}
\région{PA}}
\end{entrée}

\begin{entrée}
{protestants (les)}
\vedette{mõ-mãxi}
\région{GOs}
\end{entrée}

\begin{entrée}
{puissance ; charisme}
\vedette{hubu}
\région{GOs}
\variante{%
\vedette{hubun, hubi}
\région{PA}}
\end{entrée}

\begin{entrée}
{puissance ; force spirituelle}
\vedette{jińu}\homonyme{1}
\région{GOs PA BO}
\end{entrée}

\begin{entrée}
{recueillir\_(se)}
\classe{v}
\vedette{mãxi}
\sens{2}
\région{GOs}
\variante{%
\vedette{mãxim}
\région{WEM WE}}
\variante{%
\vedette{mhãkim, mhãxim}
\région{BO}}
\variante{%
\vedette{mããxim}
\région{PA}}
\end{entrée}

\begin{entrée}
{religion}
\classe{v ; n}
\vedette{waal}
\sens{1}
\région{PA BO}
\end{entrée}

\begin{entrée}
{totem ; esprit bienfaisant ; dieu}
\vedette{kãbwa}
\région{GOs BO}
\variante{%
\vedette{kabwa}
\région{PA}}
\end{entrée}

\begin{entrée}
{vertu (d'une plante, d'un sorcier)}
\vedette{jińu}\homonyme{1}
\région{GOs PA BO}
\end{entrée}

\begin{entrée}
{voyant ; devin}
\vedette{noga}
\région{GOs}
\variante{%
\vedette{noga}
\région{BO}}
\end{entrée}

\subsection{Fêtes, danse, chant, jeux}

\subsubsection{Danses}

\begin{entrée}
{battoir (en écorce pour rythmer la danse) ; tambour}
\vedette{kaa}\homonyme{1}
\région{GOs}
\end{entrée}

\begin{entrée}
{bouquet de paille pour la danse}
\classe{nom}
\vedette{thila}
\sens{2}
\région{GOs PA BO [BM, Corne]}
\variante{%
\vedette{thira}
\région{BO [Corne]}}
\end{entrée}

\begin{entrée}
{crier en dansant [Corne]}
\vedette{hia}
\région{PA BO}
\end{entrée}

\begin{entrée}
{danse d'accueil des hommes}
\vedette{thiram}
\région{PA}
\variante{%
\vedette{thiam}
\région{BO}}
\end{entrée}

\begin{entrée}
{danse des morts (effectuée par les femmes) [Corne]}
\vedette{caro}
\région{BO}
\variante{%
\vedette{caaro}}
\end{entrée}

\begin{entrée}
{danse des morts (effectuée par les hommes) (Corne)}
\vedette{citèèn}
\région{BO}
\end{entrée}

\begin{entrée}
{danse (pour le deuil du petit chef, dansé par les oncles maternels)}
\vedette{whãi}
\région{GOs}
\end{entrée}

\begin{entrée}
{danser ; danse}
\vedette{cia}\homonyme{1}
\région{GOs BO PA}
\end{entrée}

\begin{entrée}
{danse (type de)}
\vedette{pijeva}
\région{GO}
\variante{%
\vedette{pijopa}
\région{GO}}
\end{entrée}

\begin{entrée}
{lieu de danse [BO]}
\vedette{puçee}
\région{GOs}
\variante{%
\vedette{puuye, puuce}
\région{BO [Corne]}}
\end{entrée}

\begin{entrée}
{pilou (avec percussion et ae, ae!)}
\vedette{puçee}
\région{GOs}
\variante{%
\vedette{puuye, puuce}
\région{BO [Corne]}}
\end{entrée}

\begin{entrée}
{poteaux de danse [Corne]}
\vedette{ce-cia}
\région{BO}
\end{entrée}

\subsubsection{Musique, instruments de musique}

\begin{entrée}
{accordéon}
\vedette{gò-khai}
\région{GOs}
\end{entrée}

\begin{entrée}
{air (chant, musique)}
\vedette{gee-wal}
\région{BO}
\end{entrée}

\begin{entrée}
{bambou (qui sert de percussion)}
\vedette{pwãû}
\région{GOs BO}
\variante{%
\vedette{gò-pwãû}
\région{PA}}
\end{entrée}

\begin{entrée}
{battre en rythme ; battre (cloche)}
\classe{v}
\vedette{cabi}
\sens{4}
\région{GOs PA BO}
\end{entrée}

\begin{entrée}
{chanter (oiseau) ; chant}
\classe{v.i. ; n}
\vedette{tho}\homonyme{1}
\sens{3}
\région{GOs PA BO}
\end{entrée}

\begin{entrée}
{composer un chant}
\vedette{zai}
\région{GOs BO}
\variante{%
\vedette{zhai}
\région{GA}}
\end{entrée}

\begin{entrée}
{flûte ; harmonica}
\vedette{gòò-ui}
\région{GOs BO}
\end{entrée}

\begin{entrée}
{guitare}
\vedette{gita}
\région{GOs}
\end{entrée}

\begin{entrée}
{jouer de la flûte}
\vedette{ui gò}
\région{GOs}
\end{entrée}

\begin{entrée}
{jouer (guitare, carte, jeu de balle, sport)}
\vedette{chue}
\région{GOs WEM WE}
\end{entrée}

\begin{entrée}
{musique}
\classe{v.i. ; n}
\vedette{tho}\homonyme{1}
\sens{3}
\région{GOs PA BO}
\end{entrée}

\begin{entrée}
{musique ; air (d'une chanson) ; mélodie}
\classe{nom}
\vedette{gaa}\homonyme{2}
\sens{3}
\région{GOs BO}
\variante{%
\vedette{gee}
\région{BO}}
\variante{%
\vedette{gèèn}
\région{BO [Corne]}}
\end{entrée}

\begin{entrée}
{musique (lit. son de la flûte en bambou) ; appareil de musique}
\classe{nom}
\vedette{gò}\homonyme{1}
\sens{3}
\région{GOs BO PA}
\end{entrée}

\begin{entrée}
{paroles de la chanson ; thème d'un chant}
\vedette{paxa-wa}
\région{GOs BO}
\variante{%
\vedette{paga-wal}
\région{PA}}
\end{entrée}

\begin{entrée}
{son, mélodie de la voix}
\vedette{gee-vha}
\région{GOs PA BO}
\end{entrée}

\subsubsection{Jeux divers}

\begin{entrée}
{balancelle}
\vedette{kazu kaza}
\région{GOs}
\end{entrée}

\begin{entrée}
{balançoire ; balancer (se)}
\vedette{hiliçôô}
\région{GOs}
\variante{%
\vedette{yaoli}
\région{WEM WE}}
\end{entrée}

\begin{entrée}
{balançoire ; balancer (se)}
\vedette{yaoli}
\région{PA WEM}
\variante{%
\vedette{yauli}
\région{BO}}
\variante{%
\vedette{hiliçôô}
\région{GO(s)}}
\end{entrée}

\begin{entrée}
{ballon ; balle}
\vedette{bwòò}
\région{GOs}
\variante{%
\vedette{bool}
\région{WE}}
\end{entrée}

\begin{entrée}
{bille (de petite taille ; jeu)}
\vedette{bi}\homonyme{2}
\région{GOs}
\end{entrée}

\begin{entrée}
{corde à sauter ; sauter à la corde}
\vedette{kalòya}
\région{GOs}
\end{entrée}

\begin{entrée}
{cricket}
\vedette{kiriket}
\région{GOs}
\end{entrée}

\begin{entrée}
{faire des galipettes ; faire des tonneaux}
\vedette{jibwa}
\région{GOs BO}
\end{entrée}

\begin{entrée}
{figure de jeu de ficelle "la sagaie"}
\classe{nom}
\vedette{do}\homonyme{1}
\sens{2}
\région{GOs PA BO}
\end{entrée}

\begin{entrée}
{figure du jeu de ficelle (crevette) [BO]}
\classe{nom}
\vedette{kula}\homonyme{1}
\sens{2}
\région{GOs PA BO}
\end{entrée}

\begin{entrée}
{figure du jeu de ficelle "la sagaie"}
\classe{v}
\vedette{tòè}
\sens{4}
\région{GOs PA BO}
\end{entrée}

\begin{entrée}
{figure du jeu de ficelle (la scie)}
\classe{v ; n}
\vedette{hèlè}
\sens{2}
\région{GO PA BO}
\end{entrée}

\begin{entrée}
{figure du jeu de ficelle 'le bateau' [BO]}
\classe{nom}
\vedette{wõ}
\sens{3}
\région{GOs}
\variante{%
\vedette{wony}
\région{PA WEM BO}}
\end{entrée}

\begin{entrée}
{grosse bille}
\vedette{kalò}
\région{GOs}
\end{entrée}

\begin{entrée}
{jeu}
\vedette{thathibul}
\région{PA}
\end{entrée}

\begin{entrée}
{jeu consistant à faire rire l'autre (le 1er qui rit a perdu)}
\vedette{pe-tizi mabu}
\région{GOs}
\end{entrée}

\begin{entrée}
{jeu de ficelle}
\vedette{tha-thrûã}
\région{GOs BO}
\variante{%
\vedette{tha-thûã, ta-thûã}
\région{BO}}
\end{entrée}

\begin{entrée}
{jeu de ficelle (figure du)}
\vedette{thrûã}
\région{GO}
\end{entrée}

\begin{entrée}
{jeu de lancer de sagaie (lancer-ricochet)}
\vedette{pe-tha}
\région{GOs WEM}
\end{entrée}

\begin{entrée}
{jeu de propulsion}
\vedette{kaano}
\région{GOs WEM}
\end{entrée}

\begin{entrée}
{jouer}
\vedette{pezii}
\région{GOs}
\end{entrée}

\begin{entrée}
{jouer à cache-cache}
\vedette{pe-ku-çaaxo}
\région{GOs}
\end{entrée}

\begin{entrée}
{jouer à chat perché}
\vedette{pe-hò}
\région{GOs PA WEM}
\end{entrée}

\begin{entrée}
{jouer à qqch.}
\vedette{chôãni}
\région{GOs WEM WE}
\end{entrée}

\begin{entrée}
{jouer aux devinette ; deviner ; concours de devinettes}
\vedette{pe-thahî}
\région{GOs}
\variante{%
\vedette{pe-tha-hînõ}
\région{GO(s)}}
\variante{%
\vedette{pe-thahîn}
\région{PA BO}}
\end{entrée}

\begin{entrée}
{jouer ; s'amuser ; jeu}
\vedette{chôã}
\région{GOs PA BO WEM WE}
\end{entrée}

\begin{entrée}
{retourner en bousculant}
\vedette{jibwa}
\région{GOs BO}
\end{entrée}

\begin{entrée}
{sauter à la corde [Corne]}
\vedette{karolia}
\région{BO}
\end{entrée}

\subsection{Traditions orales, relations inter-individuelles}

\subsubsection{Tradition orale}

\begin{entrée}
{devinette}
\vedette{thahîn}
\région{PA BO}
\variante{%
\vedette{thaî}
\région{GO(s)}}
\end{entrée}

\begin{entrée}
{discours sur le bois (discours rythmé sur le bambou)}
\vedette{hòòl}\homonyme{3}
\région{GOs PA}
\end{entrée}

\begin{entrée}
{formule de fin de conte}
\vedette{thiipuun ka thae pwaxilo, kuu mwa xa doon ku ijö}
\région{PA}
\variante{%
\vedette{thiiphuu thahulò}
\région{BO}}
\end{entrée}

\begin{entrée}
{haranguer (dans les grandes cérémonies)}
\vedette{hòòl}\homonyme{3}
\région{GOs PA}
\end{entrée}

\begin{entrée}
{histoire ; fable}
\vedette{zixô}
\région{GOs PA}
\variante{%
\vedette{zhixô}
\région{GO(s)}}
\variante{%
\vedette{zikô, zhikô}
\région{GO(s) vx}}
\variante{%
\vedette{hixò, hingõn}
\région{BO [BM]}}
\end{entrée}

\begin{entrée}
{mythe d'origine ; conte}
\vedette{vajama}
\région{GOs PA BO}
\variante{%
\vedette{fhajama}
\région{GO(s)}}
\end{entrée}

\begin{entrée}
{raconter une histoire}
\vedette{zixô}
\région{GOs PA}
\variante{%
\vedette{zhixô}
\région{GO(s)}}
\variante{%
\vedette{zikô, zhikô}
\région{GO(s) vx}}
\variante{%
\vedette{hixò, hingõn}
\région{BO [BM]}}
\end{entrée}

\begin{entrée}
{réciter les généalogies}
\vedette{pavwaze}
\région{GOs}
\end{entrée}

\subsubsection{Discours, échanges verbaux}

\begin{entrée}
{acquiescer ; répondre}
\vedette{caaxö}
\région{GOs PA}
\end{entrée}

\begin{entrée}
{annonce coutumière}
\classe{nom}
\vedette{mwathra}
\sens{2}
\région{GOs}
\variante{%
\vedette{mwara}
\région{GO(s)}}
\variante{%
\vedette{mwarang}
\région{BO}}
\end{entrée}

\begin{entrée}
{annonce ; forme courte de "jaale"}
\vedette{jaa}\homonyme{2}
\région{GOs}
\end{entrée}

\begin{entrée}
{annonce ; information ; nouvelle (souvent mauvaise)}
\vedette{phweexu}
\sens{1}
\région{GOs PA BO}
\variante{%
\vedette{phweeku}
\région{GO(s)}}
\variante{%
\vedette{phweewu}
\région{BO}}
\end{entrée}

\begin{entrée}
{annoncer (avec un geste coutumier)}
\vedette{phweexoe}
\région{GOs}
\end{entrée}

\begin{entrée}
{annoncer ; prévenir}
\vedette{jaale}
\région{GOs BO}
\end{entrée}

\begin{entrée}
{annoncer ; prévenir ; promettre ; déclarer ; aviser}
\vedette{tre-khõbwe}
\région{GOs}
\variante{%
\vedette{te-kôbwe}
\région{BO}}
\end{entrée}

\begin{entrée}
{annoncer ; raconter}
\vedette{phweexu}
\sens{1}
\région{GOs PA BO}
\variante{%
\vedette{phweeku}
\région{GO(s)}}
\variante{%
\vedette{phweewu}
\région{BO}}
\end{entrée}

\begin{entrée}
{assembler (faire la synthèse des paroles avant de conclure))}
\vedette{jime}
\région{GOs}
\end{entrée}

\begin{entrée}
{bégayer}
\vedette{köxö}
\région{GOs PA}
\end{entrée}

\begin{entrée}
{bégayer ; bredouiller [BM]}
\vedette{chan}
\région{BO}
\end{entrée}

\begin{entrée}
{conseiller}
\vedette{phumõ}
\région{GOs}
\variante{%
\vedette{pumõ}
\région{PA}}
\end{entrée}

\begin{entrée}
{cri (annonce l'interruption ou la fin d'une danse)}
\vedette{gaayi}
\région{PA}
\end{entrée}

\begin{entrée}
{débattre ; discuter ; débat ; discussion}
\vedette{pe-whaguzai}
\région{GOs}
\end{entrée}

\begin{entrée}
{dire ; avertir ; prévenir}
\vedette{kû-jaa}\homonyme{2}
\région{GOs}
\end{entrée}

\begin{entrée}
{dire ; penser ; croire}
\vedette{khõbwe}
\groupe{A}
\région{GOs}
\région{GOs BO}
\variante{%
\vedette{kõbwe}}
\end{entrée}

\begin{entrée}
{discourir ; faire le discours de coutume ; haranguer}
\vedette{puuńô}
\end{entrée}

\begin{entrée}
{discourir ; faire un discours (coutumier)}
\vedette{phumõ}
\région{GOs}
\variante{%
\vedette{pumõ}
\région{PA}}
\end{entrée}

\begin{entrée}
{discourir ; prêcher ; enseigner ; discours}
\vedette{pe-phumõ}
\région{GOs WEM BO PA}
\end{entrée}

\begin{entrée}
{discussion pour savoir comment procéder pour la coutume}
\vedette{vee}
\région{GOs WEM}
\end{entrée}

\begin{entrée}
{discussions}
\classe{v}
\vedette{wòdro}
\sens{1}
\région{GOs}
\variante{%
\vedette{wèdò, vòdòòn}
\région{BO}}
\end{entrée}

\begin{entrée}
{discuter}
\vedette{pe-phweexu}
\région{GOs PA BO}
\end{entrée}

\begin{entrée}
{discuter}
\classe{nom}
\vedette{pwamwãgu}
\sens{2}
\région{PA BO}
\variante{%
\vedette{pwamwègu}
\région{BO}}
\end{entrée}

\begin{entrée}
{discuter ; mettre d'accord (se)}
\vedette{pe-khõbwe}
\région{GOs}
\end{entrée}

\begin{entrée}
{discuter ; palabrer ; disposer de}
\classe{v}
\vedette{wòdro}
\sens{1}
\région{GOs}
\variante{%
\vedette{wèdò, vòdòòn}
\région{BO}}
\end{entrée}

\begin{entrée}
{encourager ; soutenir}
\vedette{zaba}
\région{GOs PA}
\variante{%
\vedette{zhaba}
\région{GA}}
\variante{%
\vedette{yhaba}
\région{BO}}
\end{entrée}

\begin{entrée}
{français}
\vedette{vhaa draalae}
\région{GOs}
\variante{%
\vedette{vhaa daleen}
\région{PA}}
\end{entrée}

\begin{entrée}
{grommeler ; grogner}
\vedette{tubun}
\région{PA BO [Corne]}
\end{entrée}

\begin{entrée}
{juger ; jugement}
\classe{v}
\vedette{wòdro}
\sens{1}
\région{GOs}
\variante{%
\vedette{wèdò, vòdòòn}
\région{BO}}
\end{entrée}

\begin{entrée}
{jurer ; promettre}
\vedette{hãbo}
\région{GOs}
\end{entrée}

\begin{entrée}
{langue (parlée)}
\classe{nom}
\vedette{phwa}\homonyme{1}
\groupe{A}
\sens{3}
\région{GOs PA BO}
\end{entrée}

\begin{entrée}
{langue parlée zuanga/yuanga}
\vedette{phwa-zua zuanga}
\région{GOs}
\end{entrée}

\begin{entrée}
{mot (lit. produit de la parole)}
\vedette{paxa-vha}
\région{GOs}
\end{entrée}

\begin{entrée}
{murmurer ; parler doucement ; parler à voix basse}
\vedette{yo-vhaa}
\région{GOs}
\variante{%
\vedette{zo-vhaa-le}
\région{BO}}
\end{entrée}

\begin{entrée}
{nommer}
\vedette{yhal}
\région{PA}
\end{entrée}

\begin{entrée}
{nom ; mot}
\vedette{ya}
\région{GOs}
\variante{%
\vedette{yhal}
\région{PA}}
\end{entrée}

\begin{entrée}
{orateur}
\vedette{a-puunõ}
\région{GOs}
\variante{%
\vedette{a-puunol}
\région{PA}}
\end{entrée}

\begin{entrée}
{parler clairement}
\vedette{vhaa-zo}
\région{GOs}
\end{entrée}

\begin{entrée}
{parler doucement (lit. un peu parler)}
\vedette{po-vhaa}
\région{GOs}
\end{entrée}

\begin{entrée}
{parler doucement, murmurer}
\vedette{vhaa caaxò}
\région{BO}
\end{entrée}

\begin{entrée}
{parler du nez}
\vedette{môre}
\région{GOs}
\end{entrée}

\begin{entrée}
{parler français}
\vedette{phwa draalae}
\région{GOs}
\end{entrée}

\begin{entrée}
{parler ; parole ; voix}
\vedette{vhaa}
\région{GOs PA BO}
\variante{%
\vedette{fhaa}
\région{GA}}
\end{entrée}

\begin{entrée}
{parler pour se réconcilier ; paix (faire la) ; faire un discours coutumier}
\vedette{puuńô}
\end{entrée}

\begin{entrée}
{parole (ancien) [BM]}
\vedette{pala}
\région{BO}
\end{entrée}

\begin{entrée}
{parole (chanson)}
\classe{nom}
\vedette{pai}\homonyme{2}
\sens{2}
\région{GOs}
\variante{%
\vedette{pain}
\région{BO PA}}
\end{entrée}

\begin{entrée}
{plaindre (se) constamment}
\classe{v}
\vedette{caaxô}
\sens{2}
\région{GOs}
\région{GOs PA}
\variante{%
\vedette{caxõõl}
\région{PA}}
\variante{%
\vedette{caxool}
\région{BO}}
\variante{%
\vedette{cawhûûl}
\région{BO}}
\end{entrée}

\begin{entrée}
{prêcher ; sermonner}
\vedette{phumõ}
\région{GOs}
\variante{%
\vedette{pumõ}
\région{PA}}
\end{entrée}

\begin{entrée}
{prédire (l'avenir)}
\vedette{tre-chamadi}
\région{GOs}
\end{entrée}

\begin{entrée}
{prêter serment}
\vedette{hãbo}
\région{GOs}
\end{entrée}

\begin{entrée}
{raconter}
\vedette{phweewe}
\région{BO PA}
\end{entrée}

\begin{entrée}
{rapporter}
\vedette{yo-vhaa}
\région{GOs}
\variante{%
\vedette{zo-vhaa-le}
\région{BO}}
\end{entrée}

\begin{entrée}
{répondre ; donner la réplique}
\vedette{zaba}
\région{GOs PA}
\variante{%
\vedette{zhaba}
\région{GA}}
\variante{%
\vedette{yhaba}
\région{BO}}
\end{entrée}

\begin{entrée}
{substance de la parole}
\vedette{we-vhaa}
\région{BO}
\end{entrée}

\begin{entrée}
{taire (se) ; faire le silence ; rester silencieux}
\vedette{hû}\homonyme{1}
\région{GOs}
\variante{%
\vedette{hûn}
\région{PA BO}}
\end{entrée}

\begin{entrée}
{vraie parole}
\vedette{cii-vhaa}
\région{GOs}
\end{entrée}

\begin{entrée}
{zuanga (nom de la langue)}
\vedette{zuanga}
\région{GOs PA}
\variante{%
\vedette{phwa-zua}
\région{GO(s)}}
\variante{%
\vedette{zhuanga}
\région{GO(s)}}
\variante{%
\vedette{yuanga}
\région{BO}}
\end{entrée}

\subsection{Découpage du temps, jours, saisons}

\subsubsection{Temps}

\begin{entrée}
{auparavant ; la fois d'avant}
\vedette{inîjeò}
\région{PA}
\end{entrée}

\begin{entrée}
{heure ; temps}
\vedette{wara}
\région{GO}
\variante{%
\vedette{wawa}
\région{GO}}
\variante{%
\vedette{whara-n}
\région{BO PA}}
\end{entrée}

\begin{entrée}
{jour fixé ; date convenue}
\vedette{bwò}\homonyme{3}
\région{GOs}
\variante{%
\vedette{bò}
\région{GO(s)}}
\variante{%
\vedette{bwòn}
\région{BO (Corne, BM)}}
\end{entrée}

\begin{entrée}
{jusqu'à ce que}
\vedette{thawa-da}
\région{BO}
\variante{%
\vedette{thaa-da}}
\end{entrée}

\begin{entrée}
{longtemps (mettre)}
\classe{v}
\vedette{pwala-mwaji}
\sens{1}
\région{GOs}
\variante{%
\vedette{pwali-mwajin}
\région{PA}}
\end{entrée}

\begin{entrée}
{moment ; époque ; heure}
\vedette{waza}
\région{GO}
\variante{%
\vedette{waza}
\région{GO}}
\variante{%
\vedette{wara}
\région{PA}}
\end{entrée}

\begin{entrée}
{moment ; époque ; période}
\vedette{wara}
\région{GO}
\variante{%
\vedette{wawa}
\région{GO}}
\variante{%
\vedette{whara-n}
\région{BO PA}}
\end{entrée}

\begin{entrée}
{moment où ; quand}
\vedette{yevwa}
\région{GOs}
\variante{%
\vedette{yepwan, yebwa}}
\end{entrée}

\begin{entrée}
{passé ; dernier}
\vedette{akònòbòn}
\région{BO [BM]}
\end{entrée}

\begin{entrée}
{retard (en)}
\vedette{goovwû}
\région{GOs}
\variante{%
\vedette{gobu}
\région{GO(s)}}
\end{entrée}

\begin{entrée}
{retard (son) ; il est en retard}
\vedette{au mwaji-n}
\région{PA}
\end{entrée}

\begin{entrée}
{saison}
\vedette{wara}
\région{GO}
\variante{%
\vedette{wawa}
\région{GO}}
\variante{%
\vedette{whara-n}
\région{BO PA}}
\end{entrée}

\begin{entrée}
{temps}
\vedette{mwajin}
\région{PA}
\end{entrée}

\subsubsection{Adverbes déictiques de temps}

\begin{entrée}
{à l'avenir}
\vedette{bò-na}
\région{GOs}
\variante{%
\vedette{bwòn-na}
\région{PA BO}}
\variante{%
\vedette{bòna}
\région{BO}}
\end{entrée}

\begin{entrée}
{après-demain ; le lendemain}
\vedette{bò-na}
\région{GOs}
\variante{%
\vedette{bwòn-na}
\région{PA BO}}
\variante{%
\vedette{bòna}
\région{BO}}
\end{entrée}

\begin{entrée}
{aujourd'hui}
\vedette{tèèn hãgana}
\région{PA}
\end{entrée}

\begin{entrée}
{avant-hier (lit. le jour d'avant)}
\vedette{tree hêbu}
\région{GOs}
\end{entrée}

\begin{entrée}
{bientôt ; tout à l'heure (à) (futur ou passé)}
\vedette{iò}
\région{GOs PA BO}
\end{entrée}

\begin{entrée}
{demain ; lendemain (le) ; prochain}
\vedette{mõnõ}\homonyme{1}
\région{GOs}
\variante{%
\vedette{mènòòn}
\région{PA WEM WE BO}}
\end{entrée}

\begin{entrée}
{hier}
\vedette{dròòrò}
\région{GOs}
\variante{%
\vedette{dròrò}
\région{GO(s)}}
\end{entrée}

\begin{entrée}
{hier}
\vedette{kõnõbwòn}
\région{PA BO}
\variante{%
\vedette{kõnõ-bòn}
\région{PA BO}}
\end{entrée}

\begin{entrée}
{il y a 3 jours}
\vedette{tèèn ne hêbuun}
\région{PA}
\end{entrée}

\begin{entrée}
{maintenant ; aujourd'hui ; tout-à-l'heure}
\vedette{hãgana}
\région{GOsBO PA}
\end{entrée}

\begin{entrée}
{tout-à-l'heure (passé)}
\vedette{iò-gò}
\région{GOs}
\variante{%
\vedette{iò-gòl}
\région{WEM WE PA BO}}
\end{entrée}

\subsubsection{Découpage du temps}

\begin{entrée}
{année}
\classe{nom}
\vedette{ka}\homonyme{2}
\sens{1}
\région{GOs PA BO}
\variante{%
\vedette{kò}
\région{GO(n)}}
\end{entrée}

\begin{entrée}
{année de}
\classe{nom}
\vedette{kau-}
\sens{1}
\région{GOs BO}
\end{entrée}

\begin{entrée}
{après-midi}
\vedette{mura-hovwo}
\région{PA}
\end{entrée}

\begin{entrée}
{après-midi (lorsque le soleil descend vers l'horizon)}
\vedette{ku-baazo al}
\région{PA BO}
\end{entrée}

\begin{entrée}
{aube}
\vedette{mara-tèèn}
\région{PA}
\end{entrée}

\begin{entrée}
{aube ; matin de bonne heure}
\vedette{whaa-gò}
\région{GOs}
\end{entrée}

\begin{entrée}
{aurore ; aube}
\vedette{hińõ-tree}\homonyme{1}
\région{GOs}
\variante{%
\vedette{hinõ-tèèn}
\région{PA BO}}
\end{entrée}

\begin{entrée}
{aurore ; premières lueurs du jour}
\vedette{phwaaza-tree}
\région{GOs}
\variante{%
\vedette{phwaala-tèèn, phwara-tèèn}
\région{PA BO}}
\end{entrée}

\begin{entrée}
{avant-hier}
\vedette{kõnõbwòn èò}
\région{PA BO}
\end{entrée}

\begin{entrée}
{chant du coq (à l'aube)}
\vedette{jeworo}
\région{BO}
\end{entrée}

\begin{entrée}
{crépuscule}
\vedette{dròvivińi}
\région{GOs}
\variante{%
\vedette{dopipini}
\région{GO}}
\variante{%
\vedette{dovivini}
\région{BO}}
\end{entrée}

\begin{entrée}
{fixer (date)}
\classe{v}
\vedette{khai}\homonyme{2}
\sens{2}
\région{GOs PA BO}
\end{entrée}

\begin{entrée}
{heure}
\vedette{hińõ-a}
\région{GOs}
\variante{%
\vedette{hinõ-al}
\région{PA BO}}
\end{entrée}

\begin{entrée}
{jour ; journée}
\vedette{tree}\homonyme{1}
\région{GOs}
\variante{%
\vedette{tèèn, tèn}
\région{PA BO}}
\end{entrée}

\begin{entrée}
{matin}
\vedette{hòòa}
\région{GOs}
\end{entrée}

\begin{entrée}
{matin ; aube ; premières lueurs du jour}
\vedette{waang}
\région{PA WEM BO}
\end{entrée}

\begin{entrée}
{matin ; faire jour}
\vedette{whaa}\homonyme{2}
\région{GOs}
\variante{%
\vedette{waa}
\région{WEM}}
\variante{%
\vedette{waang}
\région{PA BO}}
\variante{%
\vedette{waak}
\région{PA}}
\end{entrée}

\begin{entrée}
{minuit [BM, Corne]}
\vedette{gòò-bwòn}
\région{PA BO}
\variante{%
\vedette{gò-bòn}
\région{BO}}
\end{entrée}

\begin{entrée}
{minuit (lit. le juste milieu entre la nuit et le jour)}
\vedette{pe-chinõõ tro mani trèè}
\région{GOs}
\end{entrée}

\begin{entrée}
{mois}
\classe{nom}
\vedette{mhwããnu}
\sens{2}
\région{GOs PA BO}
\end{entrée}

\begin{entrée}
{montre ; pendule [PA]}
\vedette{hińõ-a}
\région{GOs}
\variante{%
\vedette{hinõ-al}
\région{PA BO}}
\end{entrée}

\begin{entrée}
{nouvelle lune}
\vedette{trabwa mhwããnu}
\région{GOs}
\variante{%
\vedette{tabwa mhwããnu}
\région{BO}}
\end{entrée}

\begin{entrée}
{nuit (faire)}
\vedette{trò}\homonyme{1}
\région{GOs}
\variante{%
\vedette{tòn, thòn}
\région{WEM PA BO}}
\end{entrée}

\begin{entrée}
{nuit ; obscurité}
\vedette{trò}\homonyme{1}
\région{GOs}
\variante{%
\vedette{tòn, thòn}
\région{WEM PA BO}}
\end{entrée}

\begin{entrée}
{origine}
\classe{nom}
\vedette{bweevwu}
\sens{2}
\région{GOs PA BO}
\end{entrée}

\begin{entrée}
{soir}
\vedette{thrõbwò}
\région{GOs}
\région{PA BO}
\variante{%
\vedette{thõbwòn, thõbòn}}
\end{entrée}

\begin{entrée}
{tombée de la nuit}
\vedette{thrõbo trò}
\région{GOs}
\variante{%
\vedette{thòbwò tòn}
\région{PA}}
\end{entrée}

\begin{entrée}
{tôt le matin ; de bon matin}
\vedette{kawaang}
\région{PA BO [Corne]}
\end{entrée}

\begin{entrée}
{toute la journée}
\vedette{pwaxa tree}
\région{GOs}
\end{entrée}

\subsubsection{Jours}

\begin{entrée}
{dimanche (lit. jour sacré)}
\vedette{bwò-kabu}
\région{GOs}
\région{BO PA}
\variante{%
\vedette{bò-kabun}}
\end{entrée}

\begin{entrée}
{jeudi (lit. 3° sacré)}
\vedette{pòko-kabu}
\région{GOs}
\variante{%
\vedette{pòko-xabu}
\région{GO(s)}}
\variante{%
\vedette{poko-kabun}
\région{PA BO}}
\end{entrée}

\begin{entrée}
{lundi (lit. après le sacré)}
\vedette{kaça-kabu}
\région{GOs}
\variante{%
\vedette{wa-kabun}
\région{PA BO}}
\end{entrée}

\begin{entrée}
{lundi (lit. après le sacré)}
\vedette{wa-kabun}
\région{GOs}
\variante{%
\vedette{kaça-kabu}
\région{GO(s)}}
\end{entrée}

\begin{entrée}
{mardi (lit. 5 ème [jour] sacré)}
\vedette{pòni-kabu}
\région{GOs}
\variante{%
\vedette{pòni-kabun}
\région{PA BO}}
\end{entrée}

\begin{entrée}
{mercredi (lit. le 4°([jour] sacré)}
\vedette{pò-pa-kabu}
\région{GOs}
\variante{%
\vedette{pò-pa-xabu}
\région{GO(s)}}
\variante{%
\vedette{pò-pa-kabun}
\région{PA BO}}
\end{entrée}

\begin{entrée}
{saison chaude et sèche}
\vedette{bweeye}
\région{PA}
\end{entrée}

\begin{entrée}
{samedi}
\vedette{po-xe kabu}
\région{GOs}
\variante{%
\vedette{po-xe kabun}
\région{BO}}
\variante{%
\vedette{cavato}
\région{PA BO}}
\end{entrée}

\begin{entrée}
{vendredi(lit. jour-jeûne)}
\vedette{pò-tru kabu}
\région{GOs}
\variante{%
\vedette{po-ru-kabun}
\région{PA BO}}
\variante{%
\vedette{bo-hode}
\région{PA}}
\end{entrée}

\subsubsection{Saisons}

\begin{entrée}
{époque de maturité de l'igname (mars à avril)}
\classe{nom}
\vedette{maxa}
\sens{2}
\région{GOs}
\variante{%
\vedette{maxal}
\région{PA BO}}
\end{entrée}

\begin{entrée}
{époque de maturité des ignames}
\vedette{yeevwa zeenô}
\région{GOs}
\end{entrée}

\begin{entrée}
{époque où les ignames commencent à mûrir (février à mars) (Dubois)}
\vedette{pwebae}
\région{BO}
\end{entrée}

\begin{entrée}
{époque où l'on choisit lesignames qu'on va consommer et semer (octobre à novembre). Dubois}
\vedette{cabicabi}
\région{BO}
\end{entrée}

\begin{entrée}
{mois où l'on débrousse les plantations [Dubois]}
\vedette{pweralo}
\région{BO}
\end{entrée}

\begin{entrée}
{mois où on laboure les champs d'ignames et où on les plante (août-septembre) [Dubois]}
\vedette{wogama}
\région{BO}
\end{entrée}

\begin{entrée}
{saison chaude (novembre à février)}
\vedette{yeevwa bwee-ce}
\région{GOs}
\end{entrée}

\begin{entrée}
{saison de disette (entre décembre et avril)}
\vedette{kou}\homonyme{2}
\région{PA}
\end{entrée}

\begin{entrée}
{saison de la plantation des ignames (juillet à septembre)}
\vedette{waza o thòe kui}
\région{GOs}
\end{entrée}

\begin{entrée}
{saison de la récolte des ignames}
\vedette{waza o thaa kui}
\région{GOs}
\end{entrée}

\begin{entrée}
{saison des pluies, des cyclones, des grandes marées (mars, avril)}
\vedette{waza o kole pwa}
\région{GOs}
\variante{%
\vedette{waza ò kole pwal}
\région{PA}}
\end{entrée}

\begin{entrée}
{saison froide}
\classe{nom}
\vedette{maxa}
\sens{2}
\région{GOs}
\variante{%
\vedette{maxal}
\région{PA BO}}
\end{entrée}

\begin{entrée}
{saison sèche, froide (mai à août)}
\vedette{yeevwa kou}
\région{GOs PA}
\end{entrée}

\subsection{Orientation, direction, localisation}

\subsubsection{Directions}

\begin{entrée}
{droite (à)}
\vedette{bwa gu-hi}
\région{PA WE}
\end{entrée}

\begin{entrée}
{droite (à)}
\vedette{bwa mhwã}
\région{GOs}
\end{entrée}

\begin{entrée}
{environs (aux) de ; vers}
\vedette{wame ni}
\région{GOs}
\end{entrée}

\begin{entrée}
{gauche (à)}
\vedette{bwa mhõ}
\région{GOs WE}
\variante{%
\vedette{mò}
\région{BO}}
\end{entrée}

\begin{entrée}
{là-bas}
\vedette{na bòli}
\région{GOs BO PA}
\end{entrée}

\begin{entrée}
{là-bas ; au-delà (invisible)}
\vedette{ẽnõli}
\région{GOs}
\variante{%
\vedette{ènõli}
\région{PA BO}}
\end{entrée}

\begin{entrée}
{là-bas en bas}
\vedette{bòli}
\région{GOs}
\end{entrée}

\begin{entrée}
{là-bas en bas ; derrière}
\vedette{êdu}
\région{GO}
\end{entrée}

\begin{entrée}
{là de ton côté [BM]}
\vedette{kuni-m}
\région{BO}
\end{entrée}

\begin{entrée}
{là en bas ; à cet endroit en bas}
\vedette{kun-êdu}
\région{GOs}
\end{entrée}

\begin{entrée}
{là en haut ; à cet endroit en haut}
\vedette{kun-êda}
\région{GOs}
\end{entrée}

\begin{entrée}
{là-haut ; en haut (sur la montagne) ; au-dessus}
\vedette{ênuda}
\région{GOs BO}
\end{entrée}

\begin{entrée}
{là sur le côté ; à cet endroit latéralement}
\vedette{kun-êba}
\région{GOs}
\end{entrée}

\begin{entrée}
{nord du pays}
\classe{n.LOC}
\vedette{kaça mwa}
\sens{2}
\région{GOs}
\end{entrée}

\begin{entrée}
{nord du pays}
\classe{nom}
\vedette{kaya-mwa}
\sens{2}
\région{BO}
\end{entrée}

\begin{entrée}
{nord du pays}
\vedette{pòminõ pwamwa}
\région{GOs}
\end{entrée}

\begin{entrée}
{près d'égo}
\vedette{hu-mi}
\région{GOs}
\end{entrée}

\begin{entrée}
{sud (lit. visage du pays)}
\vedette{me-pwamwa}
\région{GOs}
\end{entrée}

\begin{entrée}
{travers (de) (utilisé pour tout, y compris pour le soleil : lorsqu'il descend vers l'horizon)}
\vedette{ku-baazo}
\région{PA}
\variante{%
\vedette{ku-baayo}
\région{BO}}
\end{entrée}

\subsubsection{Directionnels}

\begin{entrée}
{à l'extérieur de la maison, vers la porte}
\classe{DIR}
\vedette{du}\homonyme{2}
\sens{5}
\région{GOs PA}
\end{entrée}

\begin{entrée}
{à l'intérieur de la maison, vers le fond de la maison}
\classe{DIR}
\vedette{da}\homonyme{1}
\sens{5}
\région{GOs PA BO}
\end{entrée}

\begin{entrée}
{approcher (s') en descendant}
\vedette{du-mi}
\région{GOs}
\end{entrée}

\begin{entrée}
{au nord ; à l'ouest}
\vedette{ênê-du}
\région{GOs}
\end{entrée}

\begin{entrée}
{au sud ; à l'est}
\vedette{ênê-da}
\région{GOs}
\variante{%
\vedette{ènîda}
\région{WEM}}
\end{entrée}

\begin{entrée}
{bas (en) ; vers le bas ; en aval ; vers la mer}
\vedette{ênê-du}
\région{GOs}
\end{entrée}

\begin{entrée}
{de-ci de-là ; en allant et venant ; partout}
\vedette{-ò ... -mi}
\région{GOs BO}
\end{entrée}

\begin{entrée}
{éloigner (s') du locuteur dans une direction transverse}
\vedette{-ò}\homonyme{1}
\région{GOs}
\variante{%
\vedette{-wò}
\région{GO(s)}}
\variante{%
\vedette{-ò}
\région{PA}}
\end{entrée}

\begin{entrée}
{en amont}
\classe{DIR}
\vedette{da}\homonyme{1}
\sens{1}
\région{GOs PA BO}
\end{entrée}

\begin{entrée}
{en bas}
\classe{DIR}
\vedette{du}\homonyme{2}
\sens{1}
\région{GOs PA}
\end{entrée}

\begin{entrée}
{en descendant}
\vedette{hu-du}
\région{GOs}
\end{entrée}

\begin{entrée}
{en haut}
\classe{DIR}
\vedette{da}\homonyme{1}
\sens{1}
\région{GOs PA BO}
\end{entrée}

\begin{entrée}
{en montant (suppose un déplacement)}
\vedette{hu-da}
\région{GOs}
\end{entrée}

\begin{entrée}
{en s'éloignant du locuteur (axe transverse) (d'une vallée à l'autre, traversant un cours d'eau)}
\vedette{-e}\homonyme{1}
\région{GOs PA}
\end{entrée}

\begin{entrée}
{là-bas en bas ; là-bas vers la mer (visible ou non)}
\vedette{ebòli}
\région{GOs PA BO}
\end{entrée}

\begin{entrée}
{là-bas en haut ; devant}
\vedette{êda}
\région{GO}
\end{entrée}

\begin{entrée}
{là-bas (en s'éloignant d'égo)}
\vedette{hu-ò}
\région{GOs}
\end{entrée}

\begin{entrée}
{là (sur le côté, latéralement)}
\vedette{êne-ba}
\région{GOs}
\variante{%
\vedette{èba}
\région{GO(s)}}
\end{entrée}

\begin{entrée}
{par ici}
\vedette{pomõ-mi}
\région{GOs}
\end{entrée}

\begin{entrée}
{par ici ; de ce côté [BM, Corne]}
\vedette{hõ-ã}
\région{BO}
\end{entrée}

\begin{entrée}
{vers ici; vers ego}
\vedette{-mi}
\région{GOsPA BO}
\end{entrée}

\begin{entrée}
{vers la mer ; en aval}
\classe{DIR}
\vedette{du}\homonyme{2}
\sens{4}
\région{GOs PA}
\end{entrée}

\begin{entrée}
{vers la terre}
\vedette{ênê-da}
\région{GOs}
\variante{%
\vedette{ènîda}
\région{WEM}}
\end{entrée}

\begin{entrée}
{vers la terre ; vers le fond de la vallée ou l'intérieur du pays}
\classe{DIR}
\vedette{da}\homonyme{1}
\sens{4}
\région{GOs PA BO}
\end{entrée}

\begin{entrée}
{vers le haut ; en amont ; vers la montagne}
\vedette{ênê-da}
\région{GOs}
\variante{%
\vedette{ènîda}
\région{WEM}}
\end{entrée}

\begin{entrée}
{vers le nord}
\classe{DIR}
\vedette{du}\homonyme{2}
\sens{2}
\région{GOs PA}
\end{entrée}

\begin{entrée}
{vers l'est}
\classe{DIR}
\vedette{da}\homonyme{1}
\sens{3}
\région{GOs PA BO}
\end{entrée}

\begin{entrée}
{vers le sud}
\classe{DIR}
\vedette{da}\homonyme{1}
\sens{2}
\région{GOs PA BO}
\end{entrée}

\begin{entrée}
{vers l'ouest}
\classe{DIR}
\vedette{du}\homonyme{2}
\sens{3}
\région{GOs PA}
\end{entrée}

\subsubsection{Localisation}

\begin{entrée}
{après}
\vedette{kaça}
\région{GOs}
\variante{%
\vedette{kaya}
\région{PA BO}}
\end{entrée}

\begin{entrée}
{après ; derrière}
\vedette{mura}
\région{PA BO}
\variante{%
\vedette{murò}
\région{BO}}
\variante{%
\vedette{mun}
\région{PA}}
\end{entrée}

\begin{entrée}
{après ; ensuite}
\classe{v}
\vedette{mu}
\sens{2}
\région{GOs}
\variante{%
\vedette{mun}
\région{PA BO}}
\end{entrée}

\begin{entrée}
{après (qqn, animé)}
\classe{n ; n.LOC}
\vedette{kai}\homonyme{2}
\sens{2}
\région{GOsBO PA}
\end{entrée}

\begin{entrée}
{arrière (qqch, inanimé)}
\vedette{kaça}
\région{GOs}
\variante{%
\vedette{kaya}
\région{PA BO}}
\end{entrée}

\begin{entrée}
{au-dessous ; par terre ; en bas}
\vedette{bwabu}
\région{GOs PA BO}
\end{entrée}

\begin{entrée}
{côté (sur le) ; côté (à)}
\vedette{ohe}
\région{PA}
\variante{%
\vedette{whe}
\région{PA}}
\end{entrée}

\begin{entrée}
{dans ; dedans ; à l'intérieur de}
\vedette{ni nõ}
\région{GOs}
\end{entrée}

\begin{entrée}
{dans le dos}
\classe{n ; n.LOC}
\vedette{kai}\homonyme{2}
\sens{2}
\région{GOsBO PA}
\end{entrée}

\begin{entrée}
{dans ; sur ; à}
\vedette{na}\homonyme{2}
\groupe{A}
\région{PA BO}
\end{entrée}

\begin{entrée}
{de; à}
\vedette{na kòlò}
\région{GOs}
\end{entrée}

\begin{entrée}
{de ce côté-ci}
\vedette{ku-ã}
\région{GOs}
\variante{%
\vedette{kunã}}
\end{entrée}

\begin{entrée}
{dehors ; extérieur}
\vedette{pwa}\homonyme{3}
\région{GOs PA BO}
\variante{%
\vedette{pwau}
\région{BO}}
\end{entrée}

\begin{entrée}
{de là-bas (ablatif)}
\vedette{na ènòli}
\région{GOs}
\end{entrée}

\begin{entrée}
{de l'autre côté}
\vedette{hõõ-li}
\région{BO [BM, Corne]}
\end{entrée}

\begin{entrée}
{de l'autre côté là-bas (de la montagne, de la rivière)}
\vedette{kun-òli}
\région{GOs}
\variante{%
\vedette{kun-òli}
\région{PA}}
\end{entrée}

\begin{entrée}
{derrière}
\vedette{kaça}
\région{GOs}
\variante{%
\vedette{kaya}
\région{PA BO}}
\end{entrée}

\begin{entrée}
{derrière}
\classe{n ; n.LOC}
\vedette{kai}\homonyme{2}
\sens{2}
\région{GOsBO PA}
\end{entrée}

\begin{entrée}
{dessous (en) ; sous (d'une surface, d'un point)}
\vedette{pira}
\région{GOs}
\end{entrée}

\begin{entrée}
{dessus ; au-dessus de ; sur}
\vedette{bwa}
\sens{2}
\région{GOs BO}
\end{entrée}

\begin{entrée}
{devant (être)}
\vedette{hêbu}
\groupe{A}
\région{GOs}
\variante{%
\vedette{hêbun}
\région{PA BO}}
\end{entrée}

\begin{entrée}
{endroit-là (connu des locuteurs)}
\vedette{ênè-ò}
\région{GOs}
\end{entrée}

\begin{entrée}
{endroit où}
\vedette{ênè}\homonyme{1}
\région{GOs BO}
\end{entrée}

\begin{entrée}
{face à ; présence (en) de}
\vedette{bwa mè}
\région{GOs BO}
\end{entrée}

\begin{entrée}
{gauche (côté) ; gaucher}
\vedette{mhõ}\homonyme{2}
\région{GOs WE BO}
\variante{%
\vedette{mõ}
\région{GO}}
\variante{%
\vedette{bwa mhõ}
\région{PA}}
\end{entrée}

\begin{entrée}
{ici}
\vedette{êńa}
\région{GOs PA BO}
\end{entrée}

\begin{entrée}
{ici}
\vedette{ênè}\homonyme{2}
\région{GOs BO}
\variante{%
\vedette{enã}
\région{PA}}
\end{entrée}

\begin{entrée}
{jusqu'à}
\vedette{hovwa}
\sens{2}
\région{GOs WEM}
\variante{%
\vedette{hova}
\région{PA BO WEM}}
\variante{%
\vedette{hava}
\région{PA BO}}
\end{entrée}

\begin{entrée}
{là}
\vedette{êni}
\région{GOs}
\variante{%
\vedette{enim}
\région{PA}}
\end{entrée}

\begin{entrée}
{là}
\classe{LOC.ANAPH}
\vedette{-le}
\end{entrée}

\begin{entrée}
{là-bas latéralement}
\vedette{êba}
\région{GOs PA}
\end{entrée}

\begin{entrée}
{loin ; éloigné ; lointain}
\vedette{hòò}
\région{GOs}
\variante{%
\vedette{hòòl}
\région{PA BO}}
\end{entrée}

\begin{entrée}
{longtemps (d'il y a)}
\vedette{hòò}
\région{GOs}
\variante{%
\vedette{hòòl}
\région{PA BO}}
\end{entrée}

\begin{entrée}
{pendant ; dans ; sur}
\vedette{na ni}
\région{GOs}
\end{entrée}

\begin{entrée}
{place}
\vedette{ku-ã}
\région{GOs}
\variante{%
\vedette{kunã}}
\end{entrée}

\begin{entrée}
{premier (être)}
\vedette{hêbu}
\groupe{A}
\région{GOs}
\variante{%
\vedette{hêbun}
\région{PA BO}}
\end{entrée}

\begin{entrée}
{près de ; à ; chez ; vers ; auprès de ; de l'autre côté de}
\vedette{kòlò}
\sens{2}
\région{GOs PA BO}
\variante{%
\vedette{kòlò-n}
\région{BO}}
\variante{%
\vedette{kòli}
\région{GO(s) PA}}
\end{entrée}

\begin{entrée}
{près de ; bord de (au) (très proche)}
\vedette{kòli}
\sens{2}
\région{GOs BO PA}
\end{entrée}

\begin{entrée}
{près (être) ; auprès ; à côté de}
\vedette{khazia}
\région{GOs}
\variante{%
\vedette{karia}
\région{PA BO}}
\variante{%
\vedette{khatia}
\région{BO}}
\end{entrée}

\begin{entrée}
{quelque part}
\vedette{ni-xa}
\région{GOs}
\end{entrée}

\paragraph{Noms locatifs}

\begin{entrée}
{après}
\vedette{kaça}
\région{GOs}
\variante{%
\vedette{kaya}
\région{PA BO}}
\end{entrée}

\begin{entrée}
{après (qqn, animé)}
\classe{n ; n.LOC}
\vedette{kai}\homonyme{2}
\sens{2}
\région{GOsBO PA}
\end{entrée}

\begin{entrée}
{arrière (qqch, inanimé)}
\vedette{kaça}
\région{GOs}
\variante{%
\vedette{kaya}
\région{PA BO}}
\end{entrée}

\begin{entrée}
{au milieu}
\vedette{ni gò}
\région{GOs}
\end{entrée}

\begin{entrée}
{bord ; côté ; près de ; au bord de}
\vedette{kòli}
\sens{1}
\région{GOs BO PA}
\end{entrée}

\begin{entrée}
{chez}
\classe{nom}
\vedette{avwònò}
\sens{3}
\région{GOs}
\variante{%
\vedette{avwònò}
\région{PA BO}}
\variante{%
\vedette{apono}
\région{vx}}
\end{entrée}

\begin{entrée}
{chez}
\classe{n.LOC (forme POSS de pwamwa)}
\vedette{pomõ}
\sens{3}
\région{GOs PA BO}
\variante{%
\vedette{pwòmò}
\région{WE PA}}
\end{entrée}

\begin{entrée}
{côte est ; rivage ; bord de mer}
\vedette{parèma}
\région{GOs}
\variante{%
\vedette{parèman}
\région{PA}}
\end{entrée}

\begin{entrée}
{côté (sur le)}
\classe{v}
\vedette{alaxe}
\sens{1}
\région{GOs PA BO}
\end{entrée}

\begin{entrée}
{dans le dos}
\classe{n ; n.LOC}
\vedette{kai}\homonyme{2}
\sens{2}
\région{GOsBO PA}
\end{entrée}

\begin{entrée}
{de l'autre côté ; au-delà}
\vedette{pomõ-li}
\région{GOs BO}
\end{entrée}

\begin{entrée}
{derrière}
\vedette{kaça}
\région{GOs}
\variante{%
\vedette{kaya}
\région{PA BO}}
\end{entrée}

\begin{entrée}
{derrière}
\classe{n ; n.LOC}
\vedette{kai}\homonyme{2}
\sens{2}
\région{GOsBO PA}
\end{entrée}

\begin{entrée}
{dessus, dos (de la main, du pied)}
\vedette{bwa-kaça}
\région{GOs}
\variante{%
\vedette{bwa-xaça}
\région{GO(s)}}
\end{entrée}

\begin{entrée}
{devant}
\classe{nom}
\vedette{ala-me}
\sens{2}
\région{GOs PA BO}
\end{entrée}

\begin{entrée}
{emplacement de la vente}
\vedette{mhenõ-içu}
\région{GOs}
\variante{%
\vedette{mhenõ-iyu}
\région{WEM BO}}
\variante{%
\vedette{mõ-iyu}
\région{BO}}
\end{entrée}

\begin{entrée}
{emplacement ; trace ; marque}
\classe{nom}
\vedette{nõbo}
\sens{1}
\région{GOs}
\variante{%
\vedette{nõbo, nõbwo}
\région{WEM WE PA BO}}
\end{entrée}

\begin{entrée}
{endroit}
\vedette{kun}
\région{GO PA BO}
\variante{%
\vedette{kunõng}
\région{BO}}
\variante{%
\vedette{ku(n)}
\région{GO(s)}}
\end{entrée}

\begin{entrée}
{endroit ; place}
\classe{nom}
\vedette{ku}\homonyme{4}
\sens{1}
\région{GOs PA}
\variante{%
\vedette{kun}
\région{PA}}
\end{entrée}

\begin{entrée}
{endroit ; place ; passage (col, gué, passe dans un récif)}
\classe{nom}
\vedette{mhenõõ}\homonyme{1}
\sens{1}
\région{GOs BO PA}
\variante{%
\vedette{mènõ}
\région{BO}}
\end{entrée}

\begin{entrée}
{entre}
\classe{n.LOC}
\vedette{phwevwöu}
\sens{2}
\région{GOs PA BO}
\end{entrée}

\begin{entrée}
{intérieur (à l') ; dans}
\vedette{nõ}\homonyme{2}
\région{GOs}
\variante{%
\vedette{nõ}
\région{PA BO}}
\end{entrée}

\begin{entrée}
{intervalle entre deux rangées d'un champ d'ignames}
\classe{n.LOC}
\vedette{phwevwöu}
\sens{1}
\région{GOs PA BO}
\end{entrée}

\begin{entrée}
{lieu de discussion}
\vedette{mõ-vhaa}
\région{GOs}
\end{entrée}

\begin{entrée}
{lieu ; endroit}
\vedette{nõ}\homonyme{1}
\région{GOs}
\région{PA BO WEM WE}
\end{entrée}

\begin{entrée}
{magasin ; boutique}
\vedette{mhenõ-içu}
\région{GOs}
\variante{%
\vedette{mhenõ-iyu}
\région{WEM BO}}
\variante{%
\vedette{mõ-iyu}
\région{BO}}
\end{entrée}

\begin{entrée}
{milieu}
\classe{nom}
\vedette{gòò}
\groupe{A}
\sens{4}
\région{GOs PA BO}
\end{entrée}

\begin{entrée}
{milieu (au)}
\vedette{pedõńõ}
\région{GOs}
\variante{%
\vedette{penõnõ}}
\end{entrée}

\begin{entrée}
{rive de l'autre côté}
\vedette{kole-ò}
\région{BO}
\end{entrée}

\begin{entrée}
{travers (de) ; pas droit}
\classe{v}
\vedette{alaxe}
\sens{1}
\région{GOs PA BO}
\end{entrée}

\paragraph{Verbes locatifs}

\begin{entrée}
{être (loc.)}
\classe{v}
\vedette{yuu}
\sens{2}
\région{GOs}
\variante{%
\vedette{yu, yuu}
\région{BO PA}}
\end{entrée}

\begin{entrée}
{près de (être)}
\classe{v.LOC ; progressif}
\vedette{ge}\homonyme{1}
\sens{1}
\région{GOs BO PA}
\end{entrée}

\begin{entrée}
{trouver (se)}
\classe{v}
\vedette{yuu}
\sens{2}
\région{GOs}
\variante{%
\vedette{yu, yuu}
\région{BO PA}}
\end{entrée}

\begin{entrée}
{trouver (se) à ; être dans un endroit}
\classe{v.LOC ; progressif}
\vedette{ge}\homonyme{1}
\sens{1}
\région{GOs BO PA}
\end{entrée}

\section{Nature}

\subsection{Ciel}

\subsubsection{Astres}

\begin{entrée}
{ciel}
\vedette{phwa}\homonyme{2}
\région{PA BO}
\variante{%
\vedette{phwal}
\région{BO}}
\end{entrée}

\begin{entrée}
{ciel ; cieux (en PA sens religieux)}
\vedette{dònò}
\région{GOs}
\variante{%
\vedette{dòòn}
\région{PA BO}}
\end{entrée}

\begin{entrée}
{étoile}
\classe{nom}
\vedette{pio}
\sens{1}
\région{GOs PA BO}
\end{entrée}

\begin{entrée}
{étoile du matin (lit. signe du matin)}
\vedette{hińõ-tree}\homonyme{2}
\région{GOs PA}
\end{entrée}

\begin{entrée}
{étoile du soir ; Vénus}
\vedette{atre thrõbõ}
\région{GOs}
\variante{%
\vedette{te-thõbwõn, te-a thõbwõn}
\région{BO}}
\end{entrée}

\begin{entrée}
{lune}
\classe{nom}
\vedette{mhwããnu}
\sens{1}
\région{GOs PA BO}
\end{entrée}

\begin{entrée}
{Mars (lit. étoile feu)}
\vedette{pio yaai}
\région{BO}
\end{entrée}

\begin{entrée}
{pleine lune}
\vedette{phwaa-me mhwããnu}
\région{GOs}
\end{entrée}

\begin{entrée}
{premier ou deuxième quartier de lune}
\vedette{hõgõõne mhwããnu}
\région{PA BO}
\variante{%
\vedette{hõgõõn}
\région{PA BO}}
\end{entrée}

\begin{entrée}
{premier quartier de lune}
\vedette{mhava mhwããnu}
\région{GOs BO}
\end{entrée}

\begin{entrée}
{rayon de soleil}
\vedette{do-a}
\région{GOs}
\variante{%
\vedette{do-al}
\région{PA}}
\end{entrée}

\begin{entrée}
{soleil ; beau temps}
\vedette{a}\homonyme{4}
\région{GOs}
\variante{%
\vedette{al}
\région{PA BO}}
\end{entrée}

\subsubsection{Vents}

\begin{entrée}
{souffler (vent)}
\vedette{pha}\homonyme{2}
\région{PA BO}
\end{entrée}

\begin{entrée}
{tourbillon}
\vedette{dèxavi}
\région{GOs}
\variante{%
\vedette{dèè-xavi}
\région{BO [Corne]}}
\end{entrée}

\begin{entrée}
{tourbillon (d'air)}
\vedette{dauliõ}
\région{PA}
\variante{%
\vedette{dawuliõ}
\région{BO}}
\end{entrée}

\begin{entrée}
{tourbillon de vent ; trombe de vent}
\vedette{phaöö}
\région{GOs}
\variante{%
\vedette{phauvwo}
\région{BO [Corne]}}
\end{entrée}

\begin{entrée}
{vent ; air ; atmosphère ; cyclone}
\vedette{dree}
\région{GOs}
\variante{%
\vedette{dèèn}
\région{BO PA}}
\end{entrée}

\begin{entrée}
{vent alizé du S.E}
\vedette{dree-bwava}
\région{GOs}
\end{entrée}

\begin{entrée}
{vent alizé du sud-ouest}
\vedette{draaçi}
\région{GOs}
\variante{%
\vedette{dacim}
\région{PA}}
\end{entrée}

\begin{entrée}
{vent annonciateur de pluie}
\vedette{dree bwa pwa}
\région{GOs}
\end{entrée}

\begin{entrée}
{vent de mer}
\vedette{dree ni we}
\région{GOs}
\end{entrée}

\begin{entrée}
{vent d'est}
\vedette{cebaèp}
\région{BO}
\end{entrée}

\begin{entrée}
{vent de terre ; brise de terre ; vent d'est}
\vedette{dree-bwamõ}
\région{GOs}
\variante{%
\vedette{dèèn-bwa-mòl}
\région{PA BO [Corne]}}
\end{entrée}

\begin{entrée}
{vent d'ouest}
\vedette{bweo}
\région{PA BO}
\end{entrée}

\begin{entrée}
{vent d'ouest}
\vedette{thrivwaja}
\région{GOs}
\variante{%
\vedette{thripwaja}
\région{vx (Haudricourt)}}
\end{entrée}

\begin{entrée}
{vent du nord ; nord}
\vedette{mhedrame}
\région{GOs}
\variante{%
\vedette{mhedam}
\région{PA}}
\end{entrée}

\begin{entrée}
{vent du sud}
\vedette{deeny}\homonyme{1}
\région{BO PA}
\end{entrée}

\begin{entrée}
{vent froid du sud-ouest ; alizés du sud-ouest}
\vedette{daawe}\homonyme{2}
\région{PA BO}
\end{entrée}

\begin{entrée}
{vent soufflant du sud au nord}
\vedette{bweeravac}
\région{PA BO}
\end{entrée}

\subsubsection{Phénomènes atmosphériques et naturels}

\begin{entrée}
{arc-en-ciel}
\vedette{truçaabèlè}
\région{GOs}
\variante{%
\vedette{thruucaabèlè}
\région{vx (Haudricourt)}}
\variante{%
\vedette{tuyabèlèp}
\région{PA}}
\variante{%
\vedette{tuyabèlè}
\région{BO}}
\end{entrée}

\begin{entrée}
{brouillard}
\classe{nom}
\vedette{dao}\homonyme{1}
\sens{1}
\région{GOs}
\région{PA BO}
\variante{%
\vedette{daòn}}
\end{entrée}

\begin{entrée}
{brouillard}
\vedette{katre}
\région{GOs}
\variante{%
\vedette{kaarèng}
\région{PA BO}}
\variante{%
\vedette{katèng}
\région{BO}}
\variante{%
\vedette{katèk}
\région{BO [Corne]}}
\end{entrée}

\begin{entrée}
{buée (fruit de la saison froide)}
\vedette{pò-maxa}
\région{GOs}
\end{entrée}

\begin{entrée}
{echo}
\vedette{thogaavwi}
\région{GOs PA BO}
\variante{%
\vedette{tho'gapi}
\région{GO(s) vx}}
\variante{%
\vedette{tho'gavhi}
\région{PA BO}}
\end{entrée}

\begin{entrée}
{éclair}
\vedette{nhyôni}
\région{GOs}
\variante{%
\vedette{nyõnim}
\région{PA BO}}
\end{entrée}

\begin{entrée}
{goutte de pluie}
\vedette{pò-pwa}
\région{GOs}
\variante{%
\vedette{pòò-pwal}
\région{PA}}
\end{entrée}

\begin{entrée}
{grondement (tonnerre)}
\classe{v}
\vedette{hûn}
\sens{1}
\région{PA BO}
\end{entrée}

\begin{entrée}
{midi ; zénith}
\vedette{gòòn-a}
\région{GOsWE}
\variante{%
\vedette{gòòn-al}
\région{BO PA WEM}}
\end{entrée}

\begin{entrée}
{nuage}
\vedette{nee ńee}\homonyme{1}
\région{GOs}
\variante{%
\vedette{nèèng}
\région{PA BO}}
\end{entrée}

\begin{entrée}
{nuageux ; gros nuage}
\vedette{dreebò}
\région{GOs}
\end{entrée}

\begin{entrée}
{pleuvoir}
\vedette{kole pwa}
\région{GOs}
\variante{%
\vedette{pwal}
\région{PA}}
\end{entrée}

\begin{entrée}
{pleuvoir ; pluie}
\vedette{pwal}
\région{PA BO}
\variante{%
\vedette{kole pwa}
\région{GO}}
\end{entrée}

\begin{entrée}
{pluie}
\vedette{pwa}\homonyme{2}
\région{GOs}
\variante{%
\vedette{pwal}
\région{PA}}
\end{entrée}

\begin{entrée}
{rosée ; brouillard de rivière}
\classe{nom}
\vedette{maxa}
\sens{1}
\région{GOs}
\variante{%
\vedette{maxal}
\région{PA BO}}
\end{entrée}

\begin{entrée}
{soleil irradiant}
\vedette{gòòn-a}
\région{GOsWE}
\variante{%
\vedette{gòòn-al}
\région{BO PA WEM}}
\end{entrée}

\begin{entrée}
{temps atmosphérique}
\vedette{meedree}
\région{GOs}
\variante{%
\vedette{meedèèn}
\région{PA BO}}
\end{entrée}

\begin{entrée}
{tonner ; gronder}
\classe{v}
\vedette{hûn}
\sens{1}
\région{PA BO}
\end{entrée}

\begin{entrée}
{tonnerre}
\vedette{niô}
\région{GOs BO}
\variante{%
\vedette{nhyô}
\région{PA}}
\end{entrée}

\begin{entrée}
{vapeur}
\classe{nom}
\vedette{dao}\homonyme{1}
\sens{1}
\région{GOs}
\région{PA BO}
\variante{%
\vedette{daòn}}
\end{entrée}

\subsubsection{Température}

\begin{entrée}
{chaleur}
\vedette{jińu}\homonyme{2}
\région{GOs}
\variante{%
\vedette{jinuu-n, jinoo-n}
\région{BO}}
\end{entrée}

\begin{entrée}
{chaud (être) (atmosphère, dans la maison)}
\vedette{khinu}\homonyme{2}
\région{GOs PA}
\end{entrée}

\begin{entrée}
{faire froid}
\classe{v ; n}
\vedette{tuuçò}
\sens{2}
\région{GOs}
\variante{%
\vedette{tuujong tuuyòng}
\région{PA BO [BM]}}
\end{entrée}

\begin{entrée}
{refroidir ; refroidi}
\classe{v ; n}
\vedette{tuuçò}
\sens{2}
\région{GOs}
\variante{%
\vedette{tuujong tuuyòng}
\région{PA BO [BM]}}
\end{entrée}

\begin{entrée}
{très froid}
\vedette{hai-kãbu}
\région{GOs}
\variante{%
\vedette{hai'xãbu}
\région{GO(s)}}
\end{entrée}

\subsection{Terre; les terrains et leur constitution}

\subsubsection{Pierre, roche}

\begin{entrée}
{pierre ; caillou}
\vedette{paa}\homonyme{2}
\région{GOs PA BO}
\end{entrée}

\begin{entrée}
{pierre-ponce}
\vedette{vauma}
\région{GOs BO}
\end{entrée}

\begin{entrée}
{quartz (pierre blanche \& dure, utilisée pour les fours)}
\vedette{paa-majing}
\région{PA BO}
\end{entrée}

\begin{entrée}
{rocher}
\vedette{paa-nhi}
\région{GOs}
\end{entrée}

\begin{entrée}
{roc ; rocher calcaire}
\vedette{nhi}\homonyme{2}
\région{BO [BM]}
\end{entrée}

\begin{entrée}
{terre rouge ; latérite}
\vedette{bunu}
\région{GOs}
\end{entrée}

\subsubsection{Terre}

\begin{entrée}
{bonne terre (molle)}
\vedette{nhyatru dili}
\région{GOs}
\end{entrée}

\begin{entrée}
{boue}
\vedette{mhômõwe}
\région{GOs}
\end{entrée}

\begin{entrée}
{boue (lit. terre molle)}
\vedette{chaavwa-dili}
\région{GOs}
\variante{%
\vedette{cava dili}
\région{BO}}
\end{entrée}

\begin{entrée}
{poussière}
\classe{nom}
\vedette{pu}\homonyme{2}
\sens{2}
\région{GOs}
\variante{%
\vedette{pum}
\région{PA BO}}
\variante{%
\vedette{bu, bo}
\région{BO}}
\end{entrée}

\begin{entrée}
{poussière de la terre}
\vedette{pubu dili}
\région{GOs PA BO}
\end{entrée}

\begin{entrée}
{tas de terre ; monticule de terre}
\vedette{buu-dili}
\région{GOs WEM WE}
\end{entrée}

\begin{entrée}
{terre d'alluvion}
\vedette{nyãã}
\région{GOs BO}
\end{entrée}

\begin{entrée}
{terre damée, dure}
\vedette{paazò}
\région{GOs}
\variante{%
\vedette{paarò}
\région{PA}}
\end{entrée}

\begin{entrée}
{terre laissée par l'inondation ; terre d'alluvions}
\vedette{drovwe}
\région{GOs PA BO}
\variante{%
\vedette{do-phe}
\région{PA BO}}
\end{entrée}

\begin{entrée}
{terre noire}
\vedette{dili baa}
\variante{%
\vedette{dili baang}
\région{PA}}
\end{entrée}

\begin{entrée}
{terre ; pays ; sol ; Terre}
\vedette{bwèèdrö}\homonyme{1}
\région{GOs}
\variante{%
\vedette{bwèèdo}
\région{PA BO}}
\end{entrée}

\begin{entrée}
{terre ; torchis}
\vedette{dili}
\région{GOs BO PA}
\end{entrée}

\subsubsection{Topographie}

\begin{entrée}
{amont de la rivière (vers la source)}
\vedette{ku-nõgo}
\région{GOs}
\variante{%
\vedette{ku-nõgò}
\région{BO PA}}
\end{entrée}

\begin{entrée}
{amont du creek}
\vedette{kuu}
\région{GOs}
\variante{%
\vedette{kuun}
\région{BO}}
\end{entrée}

\begin{entrée}
{amont du fleuve}
\vedette{kuu-jaaò}
\région{GOs}
\région{BO}
\variante{%
\vedette{kuu-jaaòl}}
\end{entrée}

\begin{entrée}
{aplani ; plat ; plaine}
\vedette{trèèzu}
\région{GOs}
\variante{%
\vedette{tèèzo}
\région{PA}}
\end{entrée}

\begin{entrée}
{à terre}
\vedette{mõ}\homonyme{1}
\sens{2}
\région{GOs}
\région{PA BO}
\variante{%
\vedette{mòl}}
\end{entrée}

\begin{entrée}
{à terre (lit. sec)}
\vedette{bwa-mõ}
\région{GOs}
\variante{%
\vedette{bwa-mol}
\région{BO}}
\end{entrée}

\begin{entrée}
{bord}
\classe{n.LOC}
\vedette{havaa-n}
\sens{1}
\région{PA BO}
\variante{%
\vedette{hapa}
\région{PA}}
\end{entrée}

\begin{entrée}
{bord de mer}
\vedette{kòli kaze}
\région{GOs}
\end{entrée}

\begin{entrée}
{bord de mer (au)}
\vedette{kòli we-za}
\région{PA}
\end{entrée}

\begin{entrée}
{carrefour convergent ; lieu de rencontre sur un chemin}
\vedette{mhenõ-pe-vhi de}
\région{GOs}
\variante{%
\vedette{mhenõ-piga-dèn}
\région{BO (Dubois)}}
\end{entrée}

\begin{entrée}
{carrefour divergent}
\vedette{mhenõ-a-pe-aze dè}
\région{GOs}
\région{BO PA (Dubois)}
\variante{%
\vedette{menõ-aveale-dèn}}
\end{entrée}

\begin{entrée}
{chaîne centrale}
\vedette{paaje}
\région{GOs PA}
\variante{%
\vedette{paaye}
\région{PA BO}}
\end{entrée}

\begin{entrée}
{col de montagne}
\vedette{wôdrî}
\région{GOs}
\variante{%
\vedette{wôding}
\région{PA BO}}
\variante{%
\vedette{wèding}
\région{BO}}
\end{entrée}

\begin{entrée}
{crête de la montagne}
\vedette{cii-hoogo}
\région{PA}
\end{entrée}

\begin{entrée}
{creux ; dépression sur un terrain}
\vedette{khalu}
\région{GOs}
\end{entrée}

\begin{entrée}
{creux ; dépression sur un terrain [BO]}
\vedette{kaalu}
\sens{3}
\région{GOs BO PA}
\end{entrée}

\begin{entrée}
{creux ; dépression (terrain)}
\vedette{wôdrî}
\région{GOs}
\variante{%
\vedette{wôding}
\région{PA BO}}
\variante{%
\vedette{wèding}
\région{BO}}
\end{entrée}

\begin{entrée}
{en amont ; en haut}
\vedette{pomõ-da}
\région{GOs}
\variante{%
\vedette{pomwa-da, poma-da}
\région{BO}}
\end{entrée}

\begin{entrée}
{en aval ; en bas}
\vedette{pomõ-du}
\région{GOs}
\variante{%
\vedette{pomwõdu, pomõ-du}
\région{BO}}
\end{entrée}

\begin{entrée}
{érosion ; ravinement ; terre ravinée}
\classe{v ; n}
\vedette{drõbö}
\sens{2}
\région{GOs}
\variante{%
\vedette{dòbo}
\région{BO PA}}
\end{entrée}

\begin{entrée}
{extrémité (champ, natte) (désigne les fibres qui dépassent du tressage)}
\classe{n.LOC}
\vedette{havaa-n}
\sens{1}
\région{PA BO}
\variante{%
\vedette{hapa}
\région{PA}}
\end{entrée}

\begin{entrée}
{flanc de la montagne}
\vedette{ara-hogo}
\région{PA}
\end{entrée}

\begin{entrée}
{fond de la vallée}
\vedette{kuu}
\région{GOs}
\variante{%
\vedette{kuun}
\région{BO}}
\end{entrée}

\begin{entrée}
{fond de la vallée ; haut d'une vallée}
\classe{nom}
\vedette{ku}\homonyme{5}
\sens{1}
\région{GOs BO PA}
\end{entrée}

\begin{entrée}
{grotte}
\vedette{phwè-paa}
\région{GOs}
\end{entrée}

\begin{entrée}
{grotte ; caverne}
\vedette{dua}
\région{BO PA}
\end{entrée}

\begin{entrée}
{haut de la vallée}
\vedette{ku-kewang}
\région{PA}
\end{entrée}

\begin{entrée}
{haut de la vallée}
\vedette{ku-phwa}
\région{GOs}
\end{entrée}

\begin{entrée}
{hauteur}
\classe{nom}
\vedette{pui}
\sens{1}
\région{GOs PA}
\variante{%
\vedette{phui-n}
\région{BO [Corne]}}
\end{entrée}

\begin{entrée}
{ligne de crête (lit. os de la montagne) ; pente de la montagne}
\vedette{du-hogo}
\région{PA}
\end{entrée}

\begin{entrée}
{montagne}
\vedette{hoogo}
\région{GOs WEM PA BO}
\variante{%
\vedette{hoogo}
\région{BO}}
\end{entrée}

\begin{entrée}
{ouverture de la vallée}
\vedette{phwe-kepwa}
\région{GO}
\end{entrée}

\begin{entrée}
{pente (de montagne) ; ravin}
\vedette{bò}\homonyme{2}
\région{GOs}
\variante{%
\vedette{bòng}
\région{BO PA}}
\end{entrée}

\begin{entrée}
{petite élevation ; colline}
\vedette{trebwalu}
\région{GOs}
\variante{%
\vedette{tebwalu}
\région{BO PA}}
\end{entrée}

\begin{entrée}
{plaine}
\vedette{draa}\homonyme{1}
\région{GOs}
\variante{%
\vedette{daa}
\région{PA BO}}
\end{entrée}

\begin{entrée}
{plaine cultivable}
\vedette{bwadraa}
\région{GOs BO PA}
\variante{%
\vedette{bwadaa}
\région{PA}}
\end{entrée}

\begin{entrée}
{plaine verte [Corne]}
\vedette{do-bubu}
\région{BO}
\end{entrée}

\begin{entrée}
{plateau ; ouverture de la vallée}
\vedette{bwadraa}
\région{GOs BO PA}
\variante{%
\vedette{bwadaa}
\région{PA}}
\end{entrée}

\begin{entrée}
{ramification dans la montagne}
\vedette{poxa-du hogo}
\région{GOs PA}
\end{entrée}

\begin{entrée}
{ramification de montagne}
\vedette{hi-hôgo}
\région{GOs PA}
\end{entrée}

\begin{entrée}
{ravinement sur les routes[PA]}
\vedette{dua}
\région{BO PA}
\end{entrée}

\begin{entrée}
{ravin ; précipice}
\vedette{pwe-kewang}
\région{BO}
\end{entrée}

\begin{entrée}
{ravin ; talweg}
\classe{nom}
\vedette{kevwa}
\sens{1}
\région{GOs}
\variante{%
\vedette{kewang}
\région{BO PA}}
\end{entrée}

\begin{entrée}
{ravin ; vallée}
\classe{nom}
\vedette{nõgò}
\sens{2}
\région{GOs}
\variante{%
\vedette{nõgò}
\région{BO PA}}
\end{entrée}

\begin{entrée}
{sommet de ; dessus de}
\classe{n (composition)}
\vedette{bwe-}
\sens{2}
\région{GOs BO PA}
\end{entrée}

\begin{entrée}
{sommet de la montagne (lit. tête de la montagne) ; crête de la montagne}
\vedette{bwe-hogo}
\région{GOs}
\end{entrée}

\begin{entrée}
{talus}
\vedette{balevhi}
\région{GA}
\end{entrée}

\begin{entrée}
{tranchée}
\classe{nom}
\vedette{phwè-}
\sens{3}
\région{GOs}
\end{entrée}

\begin{entrée}
{trou}
\vedette{phwa-}\homonyme{1}
\groupe{A}
\région{PA}
\end{entrée}

\begin{entrée}
{vallée ; creux (terrain)}
\classe{nom}
\vedette{kevwa}
\sens{1}
\région{GOs}
\variante{%
\vedette{kewang}
\région{BO PA}}
\end{entrée}

\begin{entrée}
{versant ; pente de la montagne (lit. racines de la montagne < wal)}
\vedette{we-hogo}
\région{PA}
\end{entrée}

\subsection{Eau (eau douce, mer)}

\subsubsection{Eau}

\begin{entrée}
{bruit de ruissellement de l'eau}
\classe{v}
\vedette{hûn}
\sens{2}
\région{PA BO}
\end{entrée}

\begin{entrée}
{bulles d'air (qui remontent à la surface de l'eau)}
\vedette{mwömö}
\région{GOs}
\end{entrée}

\begin{entrée}
{cascade (grosse et qui fait du bruit) (lit. mélodie de l'eau)}
\vedette{gaa-we}
\région{GOs PA BO}
\variante{%
\vedette{thò-we}
\région{GO(s)}}
\end{entrée}

\begin{entrée}
{chute d'eau ; cascade}
\vedette{we-thrôbo}
\région{GOs}
\variante{%
\vedette{we-thôbo}
\région{BO}}
\end{entrée}

\begin{entrée}
{courant de l'eau}
\vedette{gii-we}
\région{GOs}
\end{entrée}

\begin{entrée}
{creek ; rivière ; ruisseau}
\vedette{po-nõgò}
\région{GOs WEM}
\variante{%
\vedette{po-nogo}
\région{PA}}
\variante{%
\vedette{nogo}
\région{BO}}
\end{entrée}

\begin{entrée}
{eau}
\vedette{we}\homonyme{1}
\région{GOs PA BO}
\end{entrée}

\begin{entrée}
{eau boueuse ; marais}
\vedette{we-phölo}
\région{GOs}
\variante{%
\vedette{we-vwölo}
\région{PA}}
\end{entrée}

\begin{entrée}
{eau douce}
\vedette{we-ne}
\région{GOs}
\variante{%
\vedette{we-nèm}
\région{BO PA}}
\end{entrée}

\begin{entrée}
{eau légèrement salée ; eau saumâtre}
\vedette{we-mèèn}
\région{PA}
\end{entrée}

\begin{entrée}
{eau morte ; eau stagnante (lit. eau assise, eau qui ne coule plus)}
\vedette{we-trabwa}
\région{GOs}
\variante{%
\vedette{we-tabwa}
\région{BO}}
\end{entrée}

\begin{entrée}
{eau potable}
\vedette{gu-we}
\région{GOs}
\end{entrée}

\begin{entrée}
{eau saumâtre}
\vedette{neule}
\région{GOs}
\variante{%
\vedette{neule-gat}
\région{BO}}
\end{entrée}

\begin{entrée}
{eau trouble}
\vedette{phöloo we}
\région{GOs}
\end{entrée}

\begin{entrée}
{eau vive (lit. eau qui coule)}
\vedette{we-tho}
\région{GOs BO}
\end{entrée}

\begin{entrée}
{écume}
\vedette{phowôgo}
\région{GOs}
\end{entrée}

\begin{entrée}
{fleuve ; Diahot (nom d'un fleuve)}
\vedette{jaaò}
\région{GOs}
\variante{%
\vedette{jaaòl, jaòòl}
\région{BO PA}}
\end{entrée}

\begin{entrée}
{inondation}
\vedette{kaò}
\région{GOs PA BO}
\end{entrée}

\begin{entrée}
{lac}
\vedette{pôbwi-we}
\région{GOs}
\end{entrée}

\begin{entrée}
{marais salant ; marais ; terrain marécageux}
\vedette{mhaaloo}
\région{GOs}
\end{entrée}

\begin{entrée}
{marécage}
\vedette{maara}
\région{PA}
\end{entrée}

\begin{entrée}
{mare ; étang}
\vedette{khaa}\homonyme{1}
\région{GOs PA BO}
\end{entrée}

\begin{entrée}
{mare (qui sèche au soleil)}
\vedette{jimòng}
\région{PA}
\end{entrée}

\begin{entrée}
{mouillé (par la pluie, la rosée)}
\vedette{pôwe}
\région{GOs}
\end{entrée}

\begin{entrée}
{ouverture d'eau ; prise d'eau}
\vedette{we-thrôbo}
\région{GOs}
\variante{%
\vedette{we-thôbo}
\région{BO}}
\end{entrée}

\begin{entrée}
{puits (lit. eau morte) [GO]}
\vedette{ma-we}
\région{GOs BO}
\end{entrée}

\begin{entrée}
{remous ; reflux}
\vedette{niilöö}
\région{GOs}
\variante{%
\vedette{nhiilö}
\région{PA BO}}
\end{entrée}

\begin{entrée}
{rivière ; creek [PA BO] ; ruisseau}
\classe{nom}
\vedette{nõgò}
\sens{1}
\région{GOs}
\variante{%
\vedette{nõgò}
\région{BO PA}}
\end{entrée}

\begin{entrée}
{source}
\vedette{hõn}
\région{PA}
\end{entrée}

\begin{entrée}
{source}
\vedette{mhããm}
\région{BO}
\end{entrée}

\begin{entrée}
{source d'eau}
\vedette{we-cabo}
\région{GOs}
\variante{%
\vedette{we-cabòl}
\région{PA}}
\end{entrée}

\begin{entrée}
{source (endroit où l'eau sourd)}
\vedette{phweõ}
\région{GOs}
\variante{%
\vedette{phwee-hòn, phweòn}
\région{BO [Corne]}}
\end{entrée}

\begin{entrée}
{source occasionnelle (qui ne coule qu'après de grosses pluies) [BO]}
\vedette{ma-we}
\région{GOs BO}
\end{entrée}

\begin{entrée}
{tourbillon ; contre-courant}
\vedette{we-kênõng}
\région{BO}
\end{entrée}

\begin{entrée}
{tourbillon (dans l'eau, grand et lent)}
\vedette{niilöö}
\région{GOs}
\variante{%
\vedette{nhiilö}
\région{PA BO}}
\end{entrée}

\begin{entrée}
{tourbillon d'eau}
\classe{v ; n}
\vedette{kênõ}
\sens{4}
\région{GOs}
\région{BO PA}
\variante{%
\vedette{kênõng}}
\end{entrée}

\begin{entrée}
{tourbillon (petit et rapide dans l'eau)}
\vedette{pomõnim}
\région{PA}
\end{entrée}

\begin{entrée}
{tourbillon (sur un cours d'eau)}
\vedette{pwevwenu}
\région{GOs}
\variante{%
\vedette{pwemwêning, pwepwêni (Dubois)}
\région{BO}}
\end{entrée}

\begin{entrée}
{trouble (eau) ; sale (eau) ; limoneux}
\vedette{phöloo}
\région{GOs PA}
\variante{%
\vedette{phuloo}
\région{BO}}
\variante{%
\vedette{phuluu}
\région{WEM}}
\end{entrée}

\begin{entrée}
{trou d'eau [BO]}
\vedette{ma-we}
\région{GOs BO}
\end{entrée}

\begin{entrée}
{trou d'eau ; mare}
\classe{nom}
\vedette{phwèè-we}
\sens{2}
\région{GOs PA BO}
\région{GO PA}
\variante{%
\vedette{phwè-we}}
\end{entrée}

\subsubsection{Marées}

\begin{entrée}
{marée basse du soir}
\vedette{motrò}
\région{GOs}
\end{entrée}

\begin{entrée}
{marée basse ; le rivage à marée basse}
\vedette{thaavwan}
\région{PA BO}
\variante{%
\vedette{thaavan}}
\end{entrée}

\begin{entrée}
{marée d'équinoxe [Corne]}
\vedette{kale kou}
\région{BO}
\end{entrée}

\begin{entrée}
{marée descendante}
\vedette{khabwa}
\région{GOs}
\end{entrée}

\begin{entrée}
{marée descendante ; marée basse}
\vedette{thraavwã}
\région{GOs}
\région{GOs}
\variante{%
\vedette{thraapã}}
\variante{%
\vedette{thaavan}
\région{BO PA}}
\end{entrée}

\begin{entrée}
{marée étale (calme)}
\vedette{khô-kaze}
\région{GOs}
\end{entrée}

\begin{entrée}
{marée étale [Corne]}
\vedette{wîî-kale}
\région{BO}
\end{entrée}

\begin{entrée}
{marée haute}
\vedette{wa-kaze}
\région{GOs}
\variante{%
\vedette{wa-kale}
\région{GO(s)}}
\end{entrée}

\begin{entrée}
{marée haute à l'aurore}
\vedette{balaa-mãã}
\région{GOs}
\end{entrée}

\begin{entrée}
{marée montante}
\vedette{me-kaze}
\région{GOs}
\variante{%
\vedette{me-kale}
\région{BO}}
\end{entrée}

\begin{entrée}
{marée montante (être) ; marée haute}
\classe{n; v}
\vedette{kaze}\homonyme{2}
\sens{1}
\région{GOs}
\variante{%
\vedette{kale}
\région{BO PA}}
\end{entrée}

\begin{entrée}
{remplir (se) (mer) [BO]}
\classe{n; v}
\vedette{kaze}\homonyme{2}
\sens{1}
\région{GOs}
\variante{%
\vedette{kale}
\région{BO PA}}
\end{entrée}

\subsubsection{Mer}

\begin{entrée}
{mer ; eau salée}
\vedette{we-za}
\région{GOs BO}
\variante{%
\vedette{we-ya}}
\end{entrée}

\begin{entrée}
{sable [GOs]}
\classe{nom}
\vedette{õ}\homonyme{2}
\sens{1}
\région{GOs}
\variante{%
\vedette{òn}
\région{BO PA}}
\end{entrée}

\begin{entrée}
{sable ; plage}
\classe{nom}
\vedette{õn}
\sens{2}
\région{PA BO}
\end{entrée}

\begin{entrée}
{vague}
\vedette{kool}
\région{PA BO}
\end{entrée}

\begin{entrée}
{vague (mer)}
\vedette{paaçae}
\région{GOs}
\end{entrée}

\subsubsection{Coraux}

\begin{entrée}
{caillou ressemblant à une patate de corail}
\vedette{vaxaròò}
\région{GOs PA BO}
\end{entrée}

\begin{entrée}
{corail}
\vedette{yaò}\homonyme{1}
\région{GOs}
\end{entrée}

\begin{entrée}
{corail ; chaux [BM]}
\vedette{karòò}
\région{GOs WEM}
\variante{%
\vedette{kharo}
\région{BO}}
\end{entrée}

\begin{entrée}
{patate de corail (blanc)}
\vedette{vaxaròò}
\région{GOs PA BO}
\end{entrée}

\subsubsection{Mer : topographie}

\begin{entrée}
{baie}
\vedette{kûdo}\homonyme{1}
\région{GOs}
\end{entrée}

\begin{entrée}
{cap}
\vedette{memee}\homonyme{1}
\région{GOs}
\variante{%
\vedette{mêmê}
\région{BO}}
\end{entrée}

\begin{entrée}
{chenal}
\vedette{phwa-xudo}
\région{GOs}
\end{entrée}

\begin{entrée}
{confluent d'un creek dans un fleuve}
\vedette{phwè-nògò}
\région{GOs}
\variante{%
\vedette{pwè-nògò}
\région{BO PA}}
\end{entrée}

\begin{entrée}
{embouchure de la rivière}
\vedette{phwè-nògò}
\région{GOs}
\variante{%
\vedette{pwè-nògò}
\région{BO PA}}
\end{entrée}

\begin{entrée}
{havre}
\vedette{tikudi}
\région{GOs}
\end{entrée}

\begin{entrée}
{île ; plâtier}
\vedette{drau}\homonyme{1}
\région{GOs}
\variante{%
\vedette{dau}
\région{PA BO}}
\end{entrée}

\begin{entrée}
{îlot d'alluvion}
\vedette{draba}
\région{GOs}
\end{entrée}

\begin{entrée}
{passe (dans la mer)}
\classe{nom}
\vedette{phwèè-we}
\sens{1}
\région{GOs PA BO}
\région{GO PA}
\variante{%
\vedette{phwè-we}}
\end{entrée}

\begin{entrée}
{récif}
\vedette{chaa}
\région{GOs}
\end{entrée}

\begin{entrée}
{récif coralien}
\vedette{kaa}\homonyme{2}
\région{GOs WE}
\end{entrée}

\begin{entrée}
{récif de corail [BM]}
\vedette{palawu}
\région{BO}
\end{entrée}

\subsection{Matière, matériaux}

\begin{entrée}
{arbre}
\vedette{ce}\homonyme{1}
\région{GOs PA BO}
\variante{%
\vedette{ce}
\région{PA}}
\variante{%
\vedette{chee}
\région{BO}}
\end{entrée}

\begin{entrée}
{argile à pot ; glaise}
\vedette{thrae}
\région{GOs}
\variante{%
\vedette{thaèè}
\région{BO [BM, Corne]}}
\end{entrée}

\begin{entrée}
{bois}
\vedette{ce}\homonyme{1}
\région{GOs PA BO}
\variante{%
\vedette{ce}
\région{PA}}
\variante{%
\vedette{chee}
\région{BO}}
\end{entrée}

\begin{entrée}
{encre (lit. liquide écrire)}
\vedette{we-tiivwo}
\région{GOs}
\variante{%
\vedette{we-tiin}
\région{PA}}
\end{entrée}

\begin{entrée}
{fer}
\vedette{gòrui}
\région{GOs}
\end{entrée}

\begin{entrée}
{pétrole (de la lampe)}
\vedette{we-yaai}
\région{GOs}
\end{entrée}

\begin{entrée}
{résine [PA] (de sapin, kaori) ; collant comme de la résine}
\classe{v.stat.}
\vedette{bizigi}
\sens{2}
\région{GOs}
\variante{%
\vedette{birigi}
\région{GO(s) WE}}
\variante{%
\vedette{bitigi}
\région{PA BO}}
\end{entrée}

\begin{entrée}
{verre ; vitre}
\vedette{vea}
\région{GOs}
\end{entrée}

\subsection{Lumière et obscurité}

\begin{entrée}
{brillant ; scintillant}
\vedette{mazido}
\région{GOs}
\end{entrée}

\begin{entrée}
{clair (être) ; faire jour ; dégagé (ciel) ; faire beau}
\classe{v}
\vedette{phwaa}
\sens{1}
\région{GOs}
\région{PA BO}
\variante{%
\vedette{phwaal}}
\end{entrée}

\begin{entrée}
{éclairer}
\vedette{pa-nûûe}
\région{GOs}
\end{entrée}

\begin{entrée}
{éclairer (avec une lampe)}
\classe{v}
\vedette{nûû}
\groupe{A}
\sens{1}
\région{GOs BO PA}
\end{entrée}

\begin{entrée}
{lampe coleman (fait une lumière vive)}
\vedette{ya-phwaa}
\région{GOs}
\end{entrée}

\begin{entrée}
{lampe tempête (qui s'accroche)}
\vedette{ya-cõê}
\région{GOs}
\variante{%
\vedette{ya-çôê}
\région{GO(s)}}
\end{entrée}

\begin{entrée}
{lampe-torche (électrique)}
\vedette{yai-khaa}
\région{GOs PA BO}
\end{entrée}

\begin{entrée}
{lumière du jour ; lumière}
\vedette{mala}
\région{GOs}
\end{entrée}

\begin{entrée}
{lumière du jour ; lumière}
\vedette{phwaala tèèn}
\région{BO}
\end{entrée}

\begin{entrée}
{lumière (en composition)}
\vedette{ya-}
\région{GOs}
\end{entrée}

\begin{entrée}
{nuit (faire)}
\vedette{trò}\homonyme{1}
\région{GOs}
\variante{%
\vedette{tòn, thòn}
\région{WEM PA BO}}
\end{entrée}

\begin{entrée}
{nuit ; obscurité}
\vedette{trò}\homonyme{1}
\région{GOs}
\variante{%
\vedette{tòn, thòn}
\région{WEM PA BO}}
\end{entrée}

\begin{entrée}
{obscurité}
\vedette{burò}\homonyme{1}
\groupe{B}
\région{GOs}
\variante{%
\vedette{buròn}
\région{BO PA WEM}}
\variante{%
\vedette{bwòn}
\région{BO}}
\end{entrée}

\begin{entrée}
{ombre (portée d'un animé)}
\classe{nom}
\vedette{hênu}
\sens{1}
\région{GOs}
\variante{%
\vedette{hînu}
\région{PA}}
\variante{%
\vedette{hênuul}
\région{BO}}
\end{entrée}

\begin{entrée}
{reflet}
\classe{nom}
\vedette{hênu}
\sens{1}
\région{GOs}
\variante{%
\vedette{hînu}
\région{PA}}
\variante{%
\vedette{hênuul}
\région{BO}}
\end{entrée}

\begin{entrée}
{sombre ; obscur ; noir}
\vedette{burò}\homonyme{1}
\groupe{A}
\région{GOs}
\variante{%
\vedette{buròn}
\région{BO PA WEM}}
\variante{%
\vedette{bwòn}
\région{BO}}
\end{entrée}

\begin{entrée}
{torche}
\vedette{nûû}
\groupe{B}
\sens{3}
\région{GOs BO PA}
\end{entrée}

\begin{entrée}
{torche}
\vedette{pa-nûû}
\région{GOs}
\end{entrée}

\begin{entrée}
{torche}
\vedette{ya-nu}
\région{BO PA}
\end{entrée}

\section{Zoologie}

\subsection{Oiseaux}

\begin{entrée}
{aigle pêcheur ; balbuzard pêcheur ; aigle siffleur}
\vedette{bwaole}
\région{GOs PA BO}
\end{entrée}

\begin{entrée}
{bécasse}
\vedette{bèn}
\région{PA BO}
\end{entrée}

\begin{entrée}
{bécassine ; courlis corlieu}
\vedette{divhii}
\région{GOs}
\end{entrée}

\begin{entrée}
{bec d'oiseau}
\vedette{mee-vwha mãni}
\région{GOs}
\variante{%
\vedette{mee-phwa mãni}
\région{GO(s)}}
\end{entrée}

\begin{entrée}
{bengali à bec rouge (ainsi nommé car est toujours en groupe)}
\vedette{mevwuu}
\région{GOs}
\end{entrée}

\begin{entrée}
{buse de mer}
\vedette{bwaè}
\région{GOs}
\variante{%
\vedette{bwaalek}
\région{PA}}
\end{entrée}

\begin{entrée}
{buse ; émouchet ; faucon (petit, avec des panaches rougeâtres sous le ventre)}
\vedette{drò}
\région{GOs}
\variante{%
\vedette{dòny}
\région{PA BO}}
\end{entrée}

\begin{entrée}
{cagou}
\vedette{diiri}
\région{GOs WEM}
\end{entrée}

\begin{entrée}
{cagou (sorte de)}
\vedette{mwagi}
\région{GOs}
\variante{%
\vedette{mwagin}
\région{PA}}
\end{entrée}

\begin{entrée}
{canard (à gorge blanche)}
\vedette{maxewa}
\région{GOs}
\end{entrée}

\begin{entrée}
{canard (autochtone) ; canard à sourcils}
\vedette{nii}
\région{GOs}
\variante{%
\vedette{nii}
\région{PA BO}}
\end{entrée}

\begin{entrée}
{canard (importé)}
\vedette{kanii}
\région{GOs}
\end{entrée}

\begin{entrée}
{'cardinal' ; rouge-gorge (Diamant psittaculaire)}
\vedette{dö-vwiã}
\région{GOs}
\variante{%
\vedette{dö-piã}
\région{GO(s)}}
\variante{%
\vedette{du-piã}
\région{BO}}
\end{entrée}

\begin{entrée}
{chouette}
\vedette{mwê}
\région{GOs}
\variante{%
\vedette{mwèn}
\région{PA}}
\variante{%
\vedette{mwãulò}
\région{WE}}
\variante{%
\vedette{mwãulòn}
\région{BO}}
\end{entrée}

\begin{entrée}
{colibri (Sucrier écarlate)}
\vedette{dö-vwiã}
\région{GOs}
\variante{%
\vedette{dö-piã}
\région{GO(s)}}
\variante{%
\vedette{du-piã}
\région{BO}}
\end{entrée}

\begin{entrée}
{"collier blanc" ; pigeon à gorge blanche}
\vedette{bwarele}
\région{GOs PA WEM BO}
\variante{%
\vedette{bwatrele}
\région{GO(s)}}
\end{entrée}

\begin{entrée}
{corbeau}
\vedette{wãwã}
\région{GOs}
\variante{%
\vedette{wââng}
\région{PA WE}}
\end{entrée}

\begin{entrée}
{corbeau}
\vedette{wââng}
\région{PA WE}
\variante{%
\vedette{wãwã}
\région{GO(s)}}
\end{entrée}

\begin{entrée}
{coucou à éventail, Gammier (noir, petit)}
\vedette{kakulinãgu}
\région{GOs BO}
\end{entrée}

\begin{entrée}
{émouchet bleu (à ventre blanc)}
\vedette{khooje}
\région{GOs}
\end{entrée}

\begin{entrée}
{émouchet bleu ; faucon ; buse blanche et noire}
\vedette{kaci}
\région{PA BO}
\end{entrée}

\begin{entrée}
{fauvette calédonienne [Corne]}
\vedette{mararâ}
\région{BO}
\end{entrée}

\begin{entrée}
{frégate (petite)}
\vedette{caave}
\région{GOs}
\end{entrée}

\begin{entrée}
{gobe-mouche}
\vedette{mãã-trele}
\région{GOs}
\variante{%
\vedette{mãã-rele}
\région{GO(s)}}
\end{entrée}

\begin{entrée}
{"grive" ; oiseau-moine}
\vedette{hêêdo}
\région{GOs WEM BO PA}
\end{entrée}

\begin{entrée}
{grive perlée (Méliphage barré)}
\vedette{jomûgò}
\région{GOs}
\end{entrée}

\begin{entrée}
{héron de nuit ; aigrette}
\vedette{thrê}
\région{GOs}
\variante{%
\vedette{thê, thã}
\région{BO}}
\end{entrée}

\begin{entrée}
{hirondelle (à dos noir) ; martinet}
\vedette{pivwilo}
\région{GOs}
\variante{%
\vedette{pwivwilö}
\région{GO(s)}}
\variante{%
\vedette{pivhilö}
\région{PA}}
\variante{%
\vedette{pevelo}
\région{BO}}
\end{entrée}

\begin{entrée}
{hirondelle busière ; langrayen à ventre blanc (PA)}
\classe{nom}
\vedette{xhii}
\sens{1}
\région{GOs}
\variante{%
\vedette{khi}
\région{GO(s)}}
\variante{%
\vedette{khiny}
\région{PA BO WEM WE}}
\end{entrée}

\begin{entrée}
{hirondelle des grottes}
\vedette{khaçańi}
\région{GOs}
\end{entrée}

\begin{entrée}
{hirondelle du Pacifique (à dos blanc)}
\classe{nom}
\vedette{xhii}
\sens{2}
\région{GOs}
\variante{%
\vedette{khi}
\région{GO(s)}}
\variante{%
\vedette{khiny}
\région{PA BO WEM WE}}
\end{entrée}

\begin{entrée}
{lève-queue ; passereau}
\vedette{dagi}
\sens{1}
\région{GOs}
\variante{%
\vedette{daginy}
\région{WEM BO PA}}
\end{entrée}

\begin{entrée}
{long-cou}
\vedette{thãã}
\région{PA BO}
\end{entrée}

\begin{entrée}
{long-cou ; héron à face blanche ; héron gris des rivières}
\vedette{kôô}\homonyme{3}
\région{GOs}
\variante{%
\vedette{kôông}
\région{BO PA}}
\end{entrée}

\begin{entrée}
{martin-pêcheur}
\vedette{bwaado}
\région{GOs PA BO}
\end{entrée}

\begin{entrée}
{merle des Moluques}
\vedette{mèni bwa bolomakau}
\région{GOs}
\variante{%
\vedette{mèni bwa bòlòxau}
\région{GO(s)}}
\end{entrée}

\begin{entrée}
{"merle noir", stourne calédonien}
\vedette{pure}
\région{GOs BO}
\variante{%
\vedette{pyèro}
\région{PA BO}}
\end{entrée}

\begin{entrée}
{nid (oiseau)}
\vedette{bwatrû}
\région{GOs}
\variante{%
\vedette{bwarû}
\région{PA}}
\variante{%
\vedette{bwarong, bwarô}
\région{BO}}
\end{entrée}

\begin{entrée}
{notou}
\vedette{pwiwii}
\région{GOs WEM PA BO}
\end{entrée}

\begin{entrée}
{oeuf}
\classe{nom}
\vedette{pi}\homonyme{1}
\groupe{A}
\sens{1}
\région{GOs PA BO WE WEM GA}
\end{entrée}

\begin{entrée}
{oeuf d'oiseau}
\vedette{pi-mèni}
\région{WE WEM GA}
\end{entrée}

\begin{entrée}
{oeuf (poule, poisson, crustacé)}
\vedette{êgo}
\région{GOs}
\variante{%
\vedette{pi-ko}
\région{PA}}
\end{entrée}

\begin{entrée}
{oiseau}
\vedette{mèni}
\région{GOs PA BO}
\variante{%
\vedette{mèèni}
\région{BO}}
\end{entrée}

\begin{entrée}
{oiseau de proie ; aigle}
\vedette{bwavwaida}
\région{GOs}
\variante{%
\vedette{bwaivwada}
\région{GO(s) BO}}
\variante{%
\vedette{bwapaida}
\région{vx}}
\end{entrée}

\begin{entrée}
{oiseau de terre (petit, marron, court et mange les graines semées dans les champs)}
\vedette{hivwivwu}
\région{GO}
\end{entrée}

\begin{entrée}
{oiseau (petit, noir)}
\vedette{atibuda}
\région{GOs}
\end{entrée}

\begin{entrée}
{perroquet ; loriquet calédonien}
\vedette{kêxê}
\région{GOs BO}
\variante{%
\vedette{kèèngè}
\région{BO [BM]}}
\end{entrée}

\begin{entrée}
{perruche ; loriquet calédonien}
\vedette{pwiri}
\région{GOs PA}
\région{GOs}
\variante{%
\vedette{pwitri}}
\variante{%
\vedette{pwiiri}
\région{BO}}
\end{entrée}

\begin{entrée}
{petit lève-queue [PA]}
\vedette{uvilu}
\région{PA}
\end{entrée}

\begin{entrée}
{pétrel (noir, sort la nuit)}
\vedette{ine}
\région{GOs BO}
\end{entrée}

\begin{entrée}
{pigeon vert}
\vedette{gu}\homonyme{1}
\région{GOs}
\variante{%
\vedette{gun}
\région{PA WEM}}
\end{entrée}

\begin{entrée}
{plume}
\vedette{pu-mèni}
\région{GOs PA}
\end{entrée}

\begin{entrée}
{plume de poule}
\vedette{pu-ko}
\région{GOs}
\end{entrée}

\begin{entrée}
{poule sultane}
\vedette{zaa}\homonyme{1}
\région{GOs PA}
\variante{%
\vedette{zhaa}
\région{GA}}
\variante{%
\vedette{yaa}
\région{BO}}
\end{entrée}

\begin{entrée}
{ralle}
\vedette{hivwivwu}
\région{GO}
\end{entrée}

\begin{entrée}
{ralle de forêt (gros oiseau)}
\vedette{dòmògèn}
\région{BO [BM]}
\end{entrée}

\begin{entrée}
{ralle (oiseau) ; bécassine}
\vedette{bè}\homonyme{2}
\région{GOs}
\variante{%
\vedette{bèn}
\région{BO PA}}
\end{entrée}

\begin{entrée}
{rossignol à ventre jaune}
\vedette{memexãi}
\région{GOs}
\end{entrée}

\begin{entrée}
{rossignol à ventre jaune}
\vedette{pwajiò}
\région{PA BO [Corne]}
\end{entrée}

\begin{entrée}
{"siffleur", échenilleur calédonien}
\vedette{jiwaa}
\région{GOs}
\variante{%
\vedette{jiia}
\région{WEM PA BO}}
\end{entrée}

\begin{entrée}
{"suceur" oiseau ; Méliphage à oreillon gris}
\vedette{trile}
\région{GOs WEM}
\variante{%
\vedette{tilèèng}
\région{PA BO}}
\end{entrée}

\begin{entrée}
{tourou ; Grive perlée}
\vedette{tourou}
\région{GOs WEM}
\end{entrée}

\begin{entrée}
{tourterelle verte}
\classe{nom}
\vedette{puradimwã}
\sens{1}
\région{GOs PA}
\end{entrée}

\subsection{Mammifères}

\subsubsection{Mammifères}

\begin{entrée}
{bétail}
\vedette{bòlòmaxaò}
\région{GOs}
\variante{%
\vedette{boloxao}
\région{GO(s)}}
\variante{%
\vedette{vaaci}
\région{WE}}
\variante{%
\vedette{vaci}
\région{PA}}
\end{entrée}

\begin{entrée}
{bétail ; vache}
\vedette{vaaci}
\région{PA BO WEM}
\end{entrée}

\begin{entrée}
{cerf}
\vedette{cèvèro}
\région{BO PA BO}
\end{entrée}

\begin{entrée}
{cerf}
\vedette{drube}
\région{GOs}
\variante{%
\vedette{dube}
\région{PA}}
\variante{%
\vedette{cèvèroo}
\région{WEM}}
\end{entrée}

\begin{entrée}
{chat}
\vedette{mimi}
\région{GOs}
\variante{%
\vedette{minòn}
\région{PA BO}}
\end{entrée}

\begin{entrée}
{chauve-souris [Corne]}
\vedette{pivivu}
\région{BO}
\end{entrée}

\begin{entrée}
{chauve-souris (petite)}
\vedette{bwixuu}
\région{GOs}
\variante{%
\vedette{bwixu}
\région{PA}}
\variante{%
\vedette{bwivu}
\région{BO (Corne)}}
\end{entrée}

\begin{entrée}
{cheval}
\vedette{chòvwa}
\région{GOs}
\variante{%
\vedette{còval}
\région{WE BO}}
\variante{%
\vedette{cova}
\région{PA}}
\end{entrée}

\begin{entrée}
{chèvre}
\vedette{nani}
\région{GOs}
\variante{%
\vedette{nani}
\région{PA BO}}
\end{entrée}

\begin{entrée}
{chien}
\vedette{kuau}
\région{GOs PABO}
\end{entrée}

\begin{entrée}
{cochon ; porc}
\vedette{poka}
\région{WE PA BO}
\end{entrée}

\begin{entrée}
{essaim de roussette}
\vedette{nhyò}
\région{GOs PA}
\variante{%
\vedette{nyò, nyõ}
\région{BO}}
\end{entrée}

\begin{entrée}
{grappe de roussettes}
\vedette{kõ-nhyò}
\région{PA}
\end{entrée}

\begin{entrée}
{mulot ; souris}
\vedette{caivwo}
\région{GOs PA BO}
\variante{%
\vedette{caipo}
\région{GO vx}}
\end{entrée}

\begin{entrée}
{nid de roussette (endroit où les roussettes se mettent le jour)}
\vedette{nhyò}
\région{GOs PA}
\variante{%
\vedette{nyò, nyõ}
\région{BO}}
\end{entrée}

\begin{entrée}
{porc ; cochon}
\vedette{poxa}\homonyme{2}
\région{GOs}
\variante{%
\vedette{pwaxa}
\région{GA}}
\variante{%
\vedette{poka}
\région{PA}}
\variante{%
\vedette{pwaka, pwòka}}
\end{entrée}

\begin{entrée}
{rat}
\vedette{ciibwin}
\région{PA BO}
\variante{%
\vedette{cibwi}
\région{BO}}
\end{entrée}

\begin{entrée}
{rat}
\vedette{zine}
\région{GOs}
\variante{%
\vedette{zhine}
\région{GA}}
\variante{%
\vedette{jine}
\région{GO(s)}}
\end{entrée}

\begin{entrée}
{roussette}
\vedette{bwò}\homonyme{1}
\région{GOs PA BO}
\variante{%
\vedette{bò, bo}
\région{BO}}
\end{entrée}

\begin{entrée}
{roussette (avec du blanc sur le cou)}
\vedette{gaga}
\région{GOs}
\end{entrée}

\begin{entrée}
{roussette (petite, de rocher)}
\vedette{ha}\homonyme{1}
\région{GOs PA}
\end{entrée}

\subsubsection{Mammifères marins}

\begin{entrée}
{dugong [Corne]}
\vedette{mudim}
\région{BO}
\end{entrée}

\begin{entrée}
{dugong ; vache marine}
\vedette{auva}
\région{GOs BO}
\end{entrée}

\subsection{Reptiles}

\subsubsection{Reptiles}

\begin{entrée}
{caméléon}
\vedette{khòò}
\région{GOs}
\variante{%
\vedette{khòòl}
\région{PA BO}}
\end{entrée}

\begin{entrée}
{gecko ; margouillat}
\vedette{toro}
\région{BO}
\end{entrée}

\begin{entrée}
{gecko ; margouillat ; tarente}
\vedette{majo}
\région{GOs BO}
\end{entrée}

\begin{entrée}
{grenouille}
\vedette{grenui}
\région{GOs}
\variante{%
\vedette{goronui}
\région{BO}}
\end{entrée}

\begin{entrée}
{lézard}
\vedette{bweena}
\région{GOs PA BO}
\end{entrée}

\begin{entrée}
{lézard (petit et marron, vit dans l'herbe)}
\vedette{trörö}
\région{GOs}
\variante{%
\vedette{toro}
\région{BO}}
\end{entrée}

\subsubsection{Reptiles marins}

\begin{entrée}
{carapace de tortue}
\vedette{pii-pò}
\région{GOs}
\variante{%
\vedette{pii-pwòn}}
\end{entrée}

\begin{entrée}
{serpent de mer}
\vedette{puri}
\région{BO}
\end{entrée}

\begin{entrée}
{serpent de mer (gris)}
\vedette{bwaa}\homonyme{2}
\région{GOs}
\variante{%
\vedette{bwaa}
\région{PA}}
\end{entrée}

\begin{entrée}
{serpent de mer (tricot rayé) ; plature}
\vedette{buaõ}
\région{GOs}
\variante{%
\vedette{buaôn}
\région{BO}}
\variante{%
\vedette{bwaô}
\région{PA}}
\end{entrée}

\begin{entrée}
{tortue "bonne écaille"}
\vedette{bu}\homonyme{5}
\région{GOs}
\end{entrée}

\begin{entrée}
{tortue de mer}
\vedette{pwò}
\région{GOs}
\variante{%
\vedette{pò}
\région{GO(s)}}
\variante{%
\vedette{pwòn}
\région{PA BO}}
\variante{%
\vedette{pòn, pô}
\région{BO PA}}
\end{entrée}

\begin{entrée}
{tortue de mer "grosse tête"}
\vedette{drabu}
\région{GOs}
\end{entrée}

\begin{entrée}
{tortue verte}
\vedette{pwò-mhãã}
\région{GOs}
\end{entrée}

\subsection{Crustacés, crabes}

\begin{entrée}
{araignée de mer}
\vedette{yago}
\région{GOs}
\end{entrée}

\begin{entrée}
{carapace de crabe}
\vedette{pii-pwaji}
\région{GOs PA}
\end{entrée}

\begin{entrée}
{carapace vide}
\classe{v.stat. ; n}
\vedette{pii}\homonyme{3}
\sens{2}
\région{GOs PA BO}
\variante{%
\vedette{pii-n}
\région{BO}}
\end{entrée}

\begin{entrée}
{crabe (de creek et de forêt, tout petit, quelques cms)}
\vedette{doau}
\région{PA}
\end{entrée}

\begin{entrée}
{crabe de palétuvier}
\vedette{cî}
\région{GOs PA BO}
\end{entrée}

\begin{entrée}
{crabe de palétuvier}
\vedette{ji}
\région{GOs}
\variante{%
\vedette{jim}
\région{PA BO}}
\end{entrée}

\begin{entrée}
{crabe de palétuvier (plus gros que "ji")}
\vedette{mhûûzi}
\région{GOs}
\end{entrée}

\begin{entrée}
{crabe de rivière}
\vedette{drale}\homonyme{3}
\région{GOs}
\variante{%
\vedette{daala}
\région{BO [Corne]}}
\end{entrée}

\begin{entrée}
{crabe de sable}
\vedette{zuzuu}
\région{GOs}
\variante{%
\vedette{zhuzuu}
\région{GO(s)}}
\end{entrée}

\begin{entrée}
{crabe de sable (sert d'amorce)}
\vedette{phwaxaa}
\région{GOs}
\end{entrée}

\begin{entrée}
{crabe de vase de palétuvier}
\vedette{pwaji}
\région{GOs PA BO}
\variante{%
\vedette{pwaaje}
\région{PA BO}}
\end{entrée}

\begin{entrée}
{crabe double peau}
\vedette{pu-pii}
\région{GOs}
\end{entrée}

\begin{entrée}
{crabe en train de muer (et de jeter sa carapace)}
\vedette{a-pii}
\région{GOs}
\end{entrée}

\begin{entrée}
{crabe plein ('double peau', lorsque sa carapace dure se détache)}
\vedette{pitrêê}
\région{GOs}
\end{entrée}

\begin{entrée}
{crabe qui vient d'avoir une nouvelle carapace}
\vedette{wadolò}
\région{GOs}
\end{entrée}

\begin{entrée}
{crabe vide}
\vedette{dõõgo}
\région{GOs}
\end{entrée}

\begin{entrée}
{crevette}
\classe{nom}
\vedette{kula}\homonyme{1}
\sens{1}
\région{GOs PA BO}
\end{entrée}

\begin{entrée}
{crevette (grosse et noire)}
\vedette{kula-be}
\région{GOs BO}
\end{entrée}

\begin{entrée}
{langouste}
\vedette{kula-kaze}
\région{GOs}
\end{entrée}

\begin{entrée}
{langouste [Corne]}
\vedette{hituvaè}
\région{BO}
\variante{%
\vedette{hiluvai}
\région{GO}}
\end{entrée}

\begin{entrée}
{pinces du crabe}
\vedette{hi-pwaji}
\région{GOs}
\end{entrée}

\subsection{Echinodermes, céphalopodes}

\subsubsection{Echinodermes}

\begin{entrée}
{étoile de mer [GOs]}
\classe{nom}
\vedette{pio}
\sens{2}
\région{GOs PA BO}
\end{entrée}

\begin{entrée}
{holothurie ; "bêche de mer"}
\vedette{imaze}
\région{GOs}
\end{entrée}

\subsubsection{Céphalopodes}

\begin{entrée}
{encre de poulpe}
\vedette{zagia ciia}
\région{GOs}
\end{entrée}

\begin{entrée}
{poulpe ; pieuvre}
\vedette{ciia}\homonyme{2}
\région{GOs BO PA}
\variante{%
\vedette{ciiya}
\région{BO}}
\end{entrée}

\subsection{Mollusques}

\begin{entrée}
{bénitier (gastéropode)}
\vedette{tagiliã}
\région{GOs PA BO}
\variante{%
\vedette{tãgilijã}
\région{GO(s)}}
\end{entrée}

\begin{entrée}
{bénitier géant (coquille) (gastéropode)}
\vedette{bwavwa}
\région{GOs BO}
\variante{%
\vedette{bwapa}
\région{GO(s)}}
\end{entrée}

\begin{entrée}
{bernacle ; anatife (chapeau chinois, clovisse)}
\vedette{amwidra}
\région{GOs}
\variante{%
\vedette{aamwida}
\région{BO}}
\end{entrée}

\begin{entrée}
{bernard-l'ermite (gastéropode)}
\vedette{õmwã}
\région{GOs}
\end{entrée}

\begin{entrée}
{bernard-l'ermite (se met dans la coquille du 'thooli')}
\vedette{thooli}
\région{GOs PA}
\end{entrée}

\begin{entrée}
{bigorneau}
\vedette{êdime}
\région{GOs}
\end{entrée}

\begin{entrée}
{conque (gastéropode)}
\classe{nom}
\vedette{kòlaao}
\sens{1}
\région{GOs BO}
\variante{%
\vedette{kòlaao}
\région{PA}}
\variante{%
\vedette{kòlao; kòlaho}
\région{BO}}
\end{entrée}

\begin{entrée}
{coquillage}
\vedette{puãgo}
\région{GOs}
\variante{%
\vedette{pwãgo}
\région{GO(s)}}
\end{entrée}

\begin{entrée}
{coquillage long qui s'enfonce dans le sable}
\vedette{druali}
\région{GOs}
\end{entrée}

\begin{entrée}
{coquillage servant à couper l'igname (Charles)}
\classe{nom}
\vedette{diia}
\sens{1}
\région{PA BO}
\variante{%
\vedette{diva}
\région{BO}}
\end{entrée}

\begin{entrée}
{coquille de coquille saint-jacques (sert de grattoir à coco et à papaye)}
\vedette{pii-ragooni}
\région{GOs}
\end{entrée}

\begin{entrée}
{coquille de moule}
\classe{nom}
\vedette{hizu}
\sens{1}
\région{GOs}
\end{entrée}

\begin{entrée}
{coquille saint-jacques}
\vedette{tagooni}
\région{GOs}
\end{entrée}

\begin{entrée}
{coquille vide (de coquillage)}
\classe{v.stat. ; n}
\vedette{pii}\homonyme{3}
\sens{3}
\région{GOs PA BO}
\variante{%
\vedette{pii-n}
\région{BO}}
\end{entrée}

\begin{entrée}
{"coquilon"(à coquille longue)}
\vedette{thooli}
\région{GOs PA}
\end{entrée}

\begin{entrée}
{escargot de terre ; bulime}
\vedette{biluu}
\région{GOs PA BO}
\end{entrée}

\begin{entrée}
{gastéropode d'eau douce}
\vedette{kaen}
\région{PA}
\end{entrée}

\begin{entrée}
{grisette}
\vedette{pwaa}\homonyme{3}
\région{GOs}
\end{entrée}

\begin{entrée}
{huître [BM, Corne]}
\vedette{degam}
\région{BO}
\end{entrée}

\begin{entrée}
{huître de palétuvier}
\vedette{khî}
\région{GOs}
\end{entrée}

\begin{entrée}
{moule}
\classe{nom}
\vedette{hizu}
\sens{1}
\région{GOs}
\end{entrée}

\begin{entrée}
{moule ; coque (sert de grattoir à banane)}
\vedette{hivwa}
\région{GOs}
\end{entrée}

\begin{entrée}
{moule ; coquillage rond [Corne]}
\vedette{kuãgòòn}
\région{BO}
\variante{%
\vedette{kwãgòn}
\région{BO}}
\end{entrée}

\begin{entrée}
{palourde}
\vedette{bwaida}
\région{GOs}
\end{entrée}

\begin{entrée}
{Pinctada (Pélécypodes)}
\vedette{diva}
\région{GO}
\variante{%
\vedette{diia}
\région{GO}}
\end{entrée}

\begin{entrée}
{porte-montre}
\vedette{drudruu}
\région{GOs}
\end{entrée}

\begin{entrée}
{"savonnette" (coquillage)}
\vedette{peçi}
\région{GOs}
\variante{%
\vedette{peyi}
\région{PA}}
\end{entrée}

\begin{entrée}
{troca (gastéropode)}
\vedette{mãã}\homonyme{4}
\région{GOs BO PA}
\end{entrée}

\begin{entrée}
{valve de coquillage}
\classe{nom}
\vedette{pii-gu}
\sens{1}
\région{GOs PA BO}
\end{entrée}

\subsection{Poissons}

\begin{entrée}
{"aiguillette"}
\vedette{phãã}
\région{GOs}
\end{entrée}

\begin{entrée}
{"aiguillette" (de grande taille)}
\vedette{zaawane}\homonyme{1}
\région{GOs}
\end{entrée}

\begin{entrée}
{aiguillette, demi-bec à taches noires}
\vedette{xaatra}
\région{GOs}
\end{entrée}

\begin{entrée}
{arête (poisson)}
\classe{nom}
\vedette{du}\homonyme{1}
\sens{2}
\région{GOs BO PA}
\end{entrée}

\begin{entrée}
{baleine ; cachalot}
\vedette{chö}
\région{GOs}
\variante{%
\vedette{coho}
\région{BO (Corne)}}
\end{entrée}

\begin{entrée}
{barracuda}
\vedette{pwa}\homonyme{1}
\région{GOs}
\end{entrée}

\begin{entrée}
{bec de cane}
\vedette{draańi}
\région{GOs}
\end{entrée}

\begin{entrée}
{bossu d'herbe}
\vedette{meeji-thre}
\région{GOs}
\end{entrée}

\begin{entrée}
{"bossu doré"}
\vedette{aazi}
\région{GOs}
\end{entrée}

\begin{entrée}
{brème bleue}
\vedette{wããdri}
\région{GOs}
\end{entrée}

\begin{entrée}
{carangue}
\vedette{vi}
\région{GOs}
\end{entrée}

\begin{entrée}
{carangue (grosse)}
\vedette{putrakou}
\région{GOs}
\variante{%
\vedette{purakou}
\région{GO(s)}}
\end{entrée}

\begin{entrée}
{carangue jaune (à l'âge adulte) [Corne]}
\vedette{calò}
\région{BO}
\end{entrée}

\begin{entrée}
{carangue noire (de très grosse taille)}
\vedette{bwaô}
\région{GOs}
\end{entrée}

\begin{entrée}
{carangue (petite)}
\vedette{kûxû}\homonyme{3}
\région{GOs}
\end{entrée}

\begin{entrée}
{carangue (taille moyenne) (lit.feuille de palétuvier)}
\vedette{dròò-kibö}
\région{GOs}
\variante{%
\vedette{dròò-xibö}}
\end{entrée}

\begin{entrée}
{carpe}
\vedette{thãi}
\région{GOs PA BO}
\end{entrée}

\begin{entrée}
{carpe}
\vedette{zòxu}
\région{GOs}
\variante{%
\vedette{zhòxu}
\région{GO(s)}}
\end{entrée}

\begin{entrée}
{carpe (grosse)}
\vedette{caan}
\région{PA}
\end{entrée}

\begin{entrée}
{castex}
\vedette{bwaû-bwara}
\région{GOs}
\end{entrée}

\begin{entrée}
{castex}
\vedette{bwaû-wããdri}
\région{GOs}
\end{entrée}

\begin{entrée}
{'crocro'}
\vedette{bwatra}
\région{GOs}
\variante{%
\vedette{bwara}
\région{GO(s)}}
\end{entrée}

\begin{entrée}
{dawa}
\vedette{kava}\homonyme{1}
\région{GOs}
\end{entrée}

\begin{entrée}
{frai ; oeufs (crustacés) ; laitance (poisson)}
\classe{nom}
\vedette{pi}\homonyme{1}
\groupe{A}
\sens{2}
\région{GOs PA BO WE WEM GA}
\end{entrée}

\begin{entrée}
{gobie}
\vedette{ba}\homonyme{1}
\région{GOs BO PA}
\end{entrée}

\begin{entrée}
{hareng}
\vedette{haxo}
\région{GOs}
\end{entrée}

\begin{entrée}
{hareng [Corne]}
\vedette{pwaade}
\région{BO}
\end{entrée}

\begin{entrée}
{hareng des marais salants (lit. poisson du sel)}
\vedette{noo-za}
\région{GOs}
\end{entrée}

\begin{entrée}
{loche blanche de rivière}
\vedette{baaròl}
\région{PA BO}
\end{entrée}

\begin{entrée}
{loche (de grande taille)}
\vedette{poxa-zaaja}
\région{GOs}
\end{entrée}

\begin{entrée}
{loche (de rivière, grande taille)}
\vedette{thraanõ}
\région{GOs}
\end{entrée}

\begin{entrée}
{loche (en général)}
\vedette{bweetroe}
\région{GOs}
\variante{%
\vedette{bweeroe}
\région{GO(s)}}
\end{entrée}

\begin{entrée}
{loche géante, carite}
\vedette{throonye}
\région{GOs}
\end{entrée}

\begin{entrée}
{loche (petite) ; lochon}
\vedette{phwòòn}
\région{BO}
\variante{%
\vedette{fwòòn}
\région{BO}}
\end{entrée}

\begin{entrée}
{loche ; sardine}
\vedette{ba}\homonyme{1}
\région{GOs BO PA}
\end{entrée}

\begin{entrée}
{lochon (petit poisson d'eau douce)}
\vedette{baatro}
\région{GOs}
\variante{%
\vedette{baaro}
\région{GO(s)}}
\end{entrée}

\begin{entrée}
{marsouin [Corne]}
\vedette{thiinyûû}
\région{GOs}
\variante{%
\vedette{thiyu}
\région{BO}}
\end{entrée}

\begin{entrée}
{mulet (de mer, de petite taille)}
\vedette{whai}\homonyme{1}
\région{GOs}
\end{entrée}

\begin{entrée}
{mulet de rivière}
\vedette{chèńi}
\région{GOs}
\end{entrée}

\begin{entrée}
{mulet de rivière (juvénile, il remonte le cours des rivières, puis devient "naxo" à l'âge adulte)}
\vedette{theul}
\région{PA BO}
\end{entrée}

\begin{entrée}
{mulet de rivière (noir, pond vers la mer puis remonte dans la rivière)}
\vedette{zu}
\région{GOs PA}
\variante{%
\vedette{zhu}
\région{GO(s)}}
\variante{%
\vedette{yu}
\région{BO}}
\end{entrée}

\begin{entrée}
{mulet (le plus gros) ou maquereau}
\vedette{jumeã}
\région{GOs}
\end{entrée}

\begin{entrée}
{mulet noir (de cascade)}
\vedette{naxo}
\région{GOs}
\variante{%
\vedette{naxo, nago}
\région{BO PA}}
\end{entrée}

\begin{entrée}
{mulet queue bleue}
\vedette{mene}
\région{GOs}
\end{entrée}

\begin{entrée}
{mulet "queue bleue" de petite taille}
\vedette{tha}\homonyme{3}
\région{GOs}
\end{entrée}

\begin{entrée}
{murène}
\vedette{thridoo}
\région{GOs}
\end{entrée}

\begin{entrée}
{"napoléon" (poisson du chef)}
\vedette{xaatròe}
\région{GOs}
\variante{%
\vedette{xaaròe}
\région{PA}}
\end{entrée}

\begin{entrée}
{oeuf (poule, poisson, crustacé)}
\vedette{êgo}
\région{GOs}
\variante{%
\vedette{pi-ko}
\région{PA}}
\end{entrée}

\begin{entrée}
{picot (de palétuvier qui remonte les rivières) ; poisson-papillon}
\vedette{thrimavwo}
\région{GOs}
\variante{%
\vedette{thrimapwo}
\région{GO(s)}}
\end{entrée}

\begin{entrée}
{picot noir ; picot (en général)}
\vedette{piixã}
\région{GOs}
\end{entrée}

\begin{entrée}
{picot rayé[Corne]}
\vedette{alaal}
\région{BO}
\end{entrée}

\begin{entrée}
{picot rayé (qui se cache entre les pierres)}
\vedette{piixã ni paa}
\région{GOs}
\end{entrée}

\begin{entrée}
{"planqueur"}
\vedette{burò}\homonyme{2}
\région{GOs}
\end{entrée}

\begin{entrée}
{poisson}
\vedette{nõ}\homonyme{3}
\région{GOs}
\variante{%
\vedette{nõ}
\région{PA BO}}
\end{entrée}

\begin{entrée}
{poisson "balabio"}
\vedette{mazilo}
\région{GOs}
\end{entrée}

\begin{entrée}
{poisson "baleinier"}
\vedette{butrö}
\région{GOs}
\end{entrée}

\begin{entrée}
{poisson "cochon"}
\vedette{dre}\homonyme{2}
\région{GOs}
\end{entrée}

\begin{entrée}
{poisson (lit. feuille de bois de fer)}
\vedette{dròò-yòò}
\région{GOs}
\end{entrée}

\begin{entrée}
{poisson "million"}
\vedette{aa-pubwe wè-ce}
\région{GOs}
\end{entrée}

\begin{entrée}
{poisson-papillon ; poisson-lune}
\vedette{bwò}\homonyme{2}
\région{GOs}
\end{entrée}

\begin{entrée}
{poisson-perroquet}
\vedette{dimwã}\homonyme{2}
\région{GOs}
\end{entrée}

\begin{entrée}
{poisson-pierre}
\vedette{nò-paa}
\région{GOs}
\end{entrée}

\begin{entrée}
{poisson "ruban"}
\vedette{pò}\homonyme{4}
\région{GOs}
\end{entrée}

\begin{entrée}
{poisson "sabre"}
\vedette{a-whili pò}
\région{GOs}
\end{entrée}

\begin{entrée}
{poisson sauteur de palétuviers}
\vedette{mãgiça}\homonyme{2}
\région{GOs}
\end{entrée}

\begin{entrée}
{poisson-volant}
\vedette{ulò}\homonyme{2}
\région{GOs}
\end{entrée}

\begin{entrée}
{raie}
\vedette{pe}\homonyme{1}
\région{GOs BOPA}
\end{entrée}

\begin{entrée}
{relégué}
\vedette{zòòwa}
\région{GOs}
\end{entrée}

\begin{entrée}
{rémora}
\vedette{kûdo}\homonyme{2}
\région{GOs}
\end{entrée}

\begin{entrée}
{requin}
\vedette{kãbwaço}
\région{GOs}
\variante{%
\vedette{kabwayòl}
\région{PA BO}}
\variante{%
\vedette{kabwoyòl, kaboyòl}
\région{BO [BM]}}
\end{entrée}

\begin{entrée}
{requin marteau}
\vedette{zeele}
\région{GOs}
\end{entrée}

\begin{entrée}
{rouget}
\vedette{kumõõ}
\région{GOs}
\end{entrée}

\begin{entrée}
{rouget [BM]}
\vedette{jumo}
\région{BO}
\end{entrée}

\begin{entrée}
{seiche (lit. pieuvre sans bout)}
\vedette{ciia hulò hailò hulò}
\région{GOs}
\end{entrée}

\begin{entrée}
{silure (de rivière) [Corne]}
\vedette{bãã}
\région{BO}
\end{entrée}

\begin{entrée}
{tarpon à filament}
\vedette{puuvwoo}
\région{GOs}
\end{entrée}

\begin{entrée}
{tilapia}
\vedette{lapya}
\région{GOs}
\end{entrée}

\subsection{Anguilles}

\begin{entrée}
{anguille de creek}
\vedette{peeńã-nõgò}
\région{GOs}
\end{entrée}

\begin{entrée}
{anguille de creek et de forêt}
\vedette{deeny}\homonyme{2}
\région{PA BO}
\end{entrée}

\begin{entrée}
{anguille de creek (noire et verte, à grosse tête)}
\vedette{throvwa}
\région{GOs}
\variante{%
\vedette{thropwa}
\région{vx}}
\end{entrée}

\begin{entrée}
{anguille de creek (rouge)}
\vedette{dee}\homonyme{2}
\région{GOs}
\variante{%
\vedette{deeny}
\région{PA BO}}
\end{entrée}

\begin{entrée}
{anguille de mer (sorte d')}
\vedette{khò}\homonyme{2}
\région{GOs}
\end{entrée}

\begin{entrée}
{anguille (de rivière)}
\vedette{peeńã}
\région{GOs PA BO}
\end{entrée}

\begin{entrée}
{anguille jaune [Corne]}
\vedette{maalò}
\région{BO}
\end{entrée}

\begin{entrée}
{anguille (tachetée bleu et blanc) [Corne]}
\vedette{diiru}
\région{BO}
\end{entrée}

\begin{entrée}
{anguille (variété d') [BM]}
\vedette{cebòn}
\région{BO}
\end{entrée}

\begin{entrée}
{préfixe des anguilles}
\vedette{pee-}
\région{BO}
\end{entrée}

\begin{entrée}
{trou d'anguille ; anfractuosité dans rocher (où se cachent les anguilles)}
\vedette{phwè-peeńã}
\région{GOs}
\end{entrée}

\subsection{Insectes}

\begin{entrée}
{abeille (hu-ã: notre viande)}
\vedette{mebo hu-ã}
\région{GOs}
\end{entrée}

\begin{entrée}
{abeille [PA, BO]}
\vedette{aamu}
\sens{2}
\région{GOs}
\variante{%
\vedette{amu}
\région{PA}}
\variante{%
\vedette{ããmu}
\région{BO}}
\end{entrée}

\begin{entrée}
{araignée}
\vedette{truãrôô}
\région{GOs}
\variante{%
\vedette{truãrô}
\région{BO}}
\variante{%
\vedette{tuarôn}
\région{PA}}
\end{entrée}

\begin{entrée}
{araignée (de terre, noire, grosse)}
\vedette{nhõginy}
\région{PA}
\variante{%
\vedette{nõginy}
\région{BO}}
\end{entrée}

\begin{entrée}
{araignée (vénimeuse, grosse et noire, vit dans la terre)}
\vedette{jibaale}\homonyme{1}
\région{GOs PA}
\end{entrée}

\begin{entrée}
{asticot}
\vedette{kuńô}
\région{GOs}
\end{entrée}

\begin{entrée}
{cafard}
\vedette{kafaa}
\région{GOs}
\end{entrée}

\begin{entrée}
{cancrelat ; cafard}
\vedette{kakorola}
\région{PA BO [Corne]}
\end{entrée}

\begin{entrée}
{chenille (nom générique)}
\vedette{meeli}
\région{GOs}
\variante{%
\vedette{melin}
\région{PA}}
\variante{%
\vedette{mèèlin}
\région{BO}}
\end{entrée}

\begin{entrée}
{cigale (grosse et verte)}
\vedette{hò}\homonyme{1}
\région{GOs PA BO}
\end{entrée}

\begin{entrée}
{cigale (petite, à tête verte de la forêt)}
\classe{nom}
\vedette{jèmaa}
\sens{1}
\région{BO PA}
\variante{%
\vedette{dada}
\région{GO(s) PA}}
\end{entrée}

\begin{entrée}
{cigale (petite et rouge)}
\vedette{aleleang}
\région{PA}
\end{entrée}

\begin{entrée}
{cigale (petite et verte)}
\vedette{halelewa}
\région{GOs}
\variante{%
\vedette{haleleang}
\région{PA WE}}
\variante{%
\vedette{alelewa, halelea}
\région{PA}}
\end{entrée}

\begin{entrée}
{cigale (très petite)}
\vedette{dada}
\région{GOs PA}
\end{entrée}

\begin{entrée}
{cocon\_d'insecte}
\vedette{khõnò}
\région{GOs}
\end{entrée}

\begin{entrée}
{fourmi noire}
\vedette{mhwacidro}
\région{GOs}
\end{entrée}

\begin{entrée}
{fourmi noire (petite)}
\vedette{mhõ}\homonyme{1}
\région{GOs PA BO}
\end{entrée}

\begin{entrée}
{fourmi (petite et rouge)}
\vedette{putrumi}
\région{GOs}
\région{PA BO}
\variante{%
\vedette{purumi}}
\end{entrée}

\begin{entrée}
{grillon}
\vedette{pevheańo}
\région{GOs PA}
\variante{%
\vedette{pepeano}
\région{GO(s)}}
\end{entrée}

\begin{entrée}
{grillon ; criquet}
\vedette{haveveno}
\région{PA}
\variante{%
\vedette{haveno}
\région{BO}}
\end{entrée}

\begin{entrée}
{guêpe ; abeille}
\vedette{mebo}
\région{GOs}
\end{entrée}

\begin{entrée}
{guêpe maçonne}
\vedette{piçilè}
\région{GOs}
\variante{%
\vedette{pivileng}
\région{PA}}
\end{entrée}

\begin{entrée}
{guêpe noire}
\vedette{magu}
\région{GOs}
\variante{%
\vedette{maguny}
\région{BO PA}}
\variante{%
\vedette{maguc}
\région{BO}}
\end{entrée}

\begin{entrée}
{larve de sauterelle}
\vedette{êgo ulò}
\région{GOs}
\end{entrée}

\begin{entrée}
{lentes}
\vedette{nhiida}
\région{GOs}
\variante{%
\vedette{niida}
\région{BO}}
\end{entrée}

\begin{entrée}
{libellule}
\vedette{dagony}
\région{PA}
\end{entrée}

\begin{entrée}
{libellule}
\vedette{pwê}
\région{PA}
\end{entrée}

\begin{entrée}
{luciole}
\vedette{kaureji}
\région{GOs}
\variante{%
\vedette{kaureim}
\région{PA}}
\end{entrée}

\begin{entrée}
{mante religieuse}
\vedette{kôô}\homonyme{2}
\région{GOs}
\end{entrée}

\begin{entrée}
{mille-pattes}
\vedette{katrińi}
\région{GOs}
\variante{%
\vedette{karini, karili}
\région{BO}}
\end{entrée}

\begin{entrée}
{mouche bleue}
\vedette{ne-phû}
\région{GOs}
\variante{%
\vedette{nèn phûny}
\région{PA}}
\end{entrée}

\begin{entrée}
{mouche (grosse) [GOs]}
\vedette{aamu}
\sens{1}
\région{GOs}
\variante{%
\vedette{amu}
\région{PA}}
\variante{%
\vedette{ããmu}
\région{BO}}
\end{entrée}

\begin{entrée}
{mouche ; moucheron}
\vedette{ne ńe}\homonyme{4}
\région{GOs}
\variante{%
\vedette{nèn}
\région{PA BO}}
\end{entrée}

\begin{entrée}
{moustique}
\vedette{neebu ńeebu}
\région{GOs}
\variante{%
\vedette{neebu}
\région{GO(s) PA BO}}
\end{entrée}

\begin{entrée}
{nid de guêpe}
\vedette{êgo mebu}
\région{GOs}
\end{entrée}

\begin{entrée}
{papillon (de nuit, marron et duveteux qui se nourrit de fruit)}
\vedette{pô}
\région{GOs}
\variante{%
\vedette{pôm}
\région{PA BO}}
\end{entrée}

\begin{entrée}
{papillon (sorte de)}
\vedette{bwòivhe}
\région{GOs BO}
\variante{%
\vedette{boive}
\région{PA}}
\end{entrée}

\begin{entrée}
{pou}
\vedette{cii.i}
\région{GOs WE PA BO}
\variante{%
\vedette{chi:i}
\région{BO}}
\end{entrée}

\begin{entrée}
{pou de corps ; morpion}
\vedette{ciiza}
\région{GOs}
\variante{%
\vedette{ciilaa}
\région{BO}}
\end{entrée}

\begin{entrée}
{préfixe des noms de chenilles}
\vedette{mèra-}
\région{GOs BO}
\end{entrée}

\begin{entrée}
{puce}
\vedette{cii.i}
\région{GOs WE PA BO}
\variante{%
\vedette{chi:i}
\région{BO}}
\end{entrée}

\begin{entrée}
{punaise (de lit, bois)}
\vedette{pulòn}
\région{PA}
\end{entrée}

\begin{entrée}
{punaise (qui sent mauvais) (Corne)}
\vedette{ci pojo}
\région{BO}
\end{entrée}

\begin{entrée}
{sauterelle}
\vedette{ulò}\homonyme{1}
\région{GOs PA BO}
\end{entrée}

\begin{entrée}
{sauterelle (cette sauterelle à pattes rouges et au corps jaune fait un bruit de crécelle)}
\vedette{pivida}
\région{PA}
\end{entrée}

\begin{entrée}
{sauterelle (de cocotier, grosse, verte)}
\vedette{kawê}
\région{BO PA}
\end{entrée}

\begin{entrée}
{sauterelle (marron, petite)}
\vedette{calaru}
\région{PA}
\end{entrée}

\begin{entrée}
{sauterelle (petite et verte)}
\vedette{kimwado}
\région{GOs WEM WE BO PA}
\end{entrée}

\begin{entrée}
{termite (lit. qui perce le bois)}
\vedette{a-uxi-ce}
\région{GOs}
\end{entrée}

\begin{entrée}
{toile d'araignée ; faire une toile (araignée)}
\vedette{tha-truãrôô}
\région{GOs PA}
\variante{%
\vedette{thruatrôô}
\région{vx (Haudricourt)}}
\variante{%
\vedette{ta-tuarô}
\région{BO [BM]}}
\end{entrée}

\begin{entrée}
{ver de bancoul (au stade juvénile)}
\vedette{phabuno}
\région{GOs}
\end{entrée}

\begin{entrée}
{ver de bancoul (gros, blanc et comestible)}
\vedette{haaxo}
\région{GOs PA BO}
\end{entrée}

\begin{entrée}
{ver de bancoul (long avec des pattes)}
\vedette{guya}\homonyme{2}
\région{GOs}
\end{entrée}

\begin{entrée}
{ver de terre}
\vedette{be}
\région{GOs PA BO}
\end{entrée}

\section{Botanique}

\subsection{Arbre}

\begin{entrée}
{acajou}
\vedette{wòòzi}
\région{GOs}
\variante{%
\vedette{wooli}
\région{PA BO}}
\variante{%
\vedette{woji}
\région{BO}}
\end{entrée}

\begin{entrée}
{arbre}
\vedette{ce}\homonyme{1}
\région{GOs PA BO}
\variante{%
\vedette{ce}
\région{PA}}
\variante{%
\vedette{chee}
\région{BO}}
\end{entrée}

\begin{entrée}
{arbre}
\vedette{pwawalèng}
\région{BO [Corne]}
\end{entrée}

\begin{entrée}
{arbre (à bois dur, bois qui fait des étincelles, n'est donc pas utilisé pour le feu de nuit)}
\vedette{booli}
\région{PA WEM BO}
\variante{%
\vedette{bwooli}
\région{WEM BO}}
\end{entrée}

\begin{entrée}
{arbre à latex (dont la sève est utilisée comme poison pour la pêche stupéfiante)}
\vedette{khööjo}
\région{GOs}
\variante{%
\vedette{koojòng, khoojòng}
\région{PA BO}}
\end{entrée}

\begin{entrée}
{arbre à pain (lit. arbre à bouillir)}
\vedette{ci-phai}
\région{GOs}
\variante{%
\vedette{cin-phai}
\région{PA BO}}
\end{entrée}

\begin{entrée}
{arbre de bord de mer}
\vedette{kou}\homonyme{1}
\région{GOs PA}
\end{entrée}

\begin{entrée}
{arbre (dont les graines sont utilisées comme teinture rouge pour les poils des masques)}
\vedette{pwajã}
\région{PA BO [Corne]}
\end{entrée}

\begin{entrée}
{arbre (et bois qui sent comme le santal)}
\vedette{dèl}
\région{PA}
\end{entrée}

\begin{entrée}
{arbuste de bord de mer à feuilles grises (plante de guerre) ; Gattilier}
\vedette{draadro}
\région{GOs}
\variante{%
\vedette{dadeng}
\région{BO (Corne)}}
\variante{%
\vedette{daadòng}
\région{BO (Dubois, Corne)}}
\end{entrée}

\begin{entrée}
{arbuste (utilisé comme produit anti-puce)}
\vedette{draadro}
\région{GOs}
\variante{%
\vedette{dadeng}
\région{BO (Corne)}}
\variante{%
\vedette{daadòng}
\région{BO (Dubois, Corne)}}
\end{entrée}

\begin{entrée}
{badamier}
\vedette{keeda}
\région{GOs BO PA}
\région{BO}
\variante{%
\vedette{jeeda}}
\end{entrée}

\begin{entrée}
{balassor ; arbre à tapa}
\classe{nom}
\vedette{bumi}
\sens{2}
\région{GOs PA BO}
\variante{%
\vedette{bumîî}
\région{BO}}
\end{entrée}

\begin{entrée}
{bambou qui sert de percussion (lors des danses)}
\vedette{gò-pwãu}
\région{PA BO}
\end{entrée}

\begin{entrée}
{bambou (utilisé pour gratter les 'dimwa' dans l'eau (voir 'bwevòlò')}
\vedette{gò-pwãu}
\région{PA BO}
\end{entrée}

\begin{entrée}
{bancoulier}
\vedette{jò}\homonyme{2}
\région{GOs}
\variante{%
\vedette{jòm}
\région{PA BO}}
\variante{%
\vedette{jem}
\région{BO (Corne)}}
\end{entrée}

\begin{entrée}
{banian}
\classe{nom}
\vedette{bumi}
\sens{1}
\région{GOs PA BO}
\variante{%
\vedette{bumîî}
\région{BO}}
\end{entrée}

\begin{entrée}
{banian (à racines aériennes)}
\vedette{bè}\homonyme{1}
\région{GOs}
\end{entrée}

\begin{entrée}
{banian (avec la sève duquel on fait des balles de cricket)}
\vedette{kausu}
\région{GOs}
\end{entrée}

\begin{entrée}
{banian ; caoutchouc}
\vedette{thò}
\région{GOs}
\end{entrée}

\begin{entrée}
{bois}
\vedette{ce}\homonyme{1}
\région{GOs PA BO}
\variante{%
\vedette{ce}
\région{PA}}
\variante{%
\vedette{chee}
\région{BO}}
\end{entrée}

\begin{entrée}
{"bois de chou"}
\vedette{wabwa}
\région{GOs BO PA}
\variante{%
\vedette{wabwa pwojo}
\région{GO(s)}}
\end{entrée}

\begin{entrée}
{bois de fer (de plaine ou de montagne)}
\vedette{yòò}\homonyme{1}
\région{GOs PA BO}
\variante{%
\vedette{yhòò}
\région{GO(s)}}
\variante{%
\vedette{yòòk}
\région{PA}}
\end{entrée}

\begin{entrée}
{"bois de lait" (on utilise les graines comme perles)}
\vedette{keejò}
\région{GOs}
\end{entrée}

\begin{entrée}
{bois de rose (pousse au bord de mer ; fleurs jaunes semblables à celles du bourao)}
\classe{nom}
\vedette{havwo}
\région{GOs}
\end{entrée}

\begin{entrée}
{bois "pétrole"}
\vedette{buo}
\région{GOs WEM PA BO}
\end{entrée}

\begin{entrée}
{bourao (de bord de mer, à écorce comestible, nourriture de disette)}
\vedette{pòòdra}
\région{GOs}
\variante{%
\vedette{pòòda}
\région{BO}}
\variante{%
\vedette{pòòdaang}
\région{BO}}
\end{entrée}

\begin{entrée}
{bourao (générique)}
\vedette{pòò}
\région{GOsPA BO}
\end{entrée}

\begin{entrée}
{bourao (sorte à écorce comestible)}
\vedette{pòò-hovwo}
\région{GOs}
\end{entrée}

\begin{entrée}
{"cerisier bleu" ; "bois bleu"}
\vedette{thralo}
\région{GOs}
\variante{%
\vedette{thalo}
\région{PA BO}}
\end{entrée}

\begin{entrée}
{cime (arbre)}
\vedette{kumee}
\région{GOs BO}
\end{entrée}

\begin{entrée}
{collet de l'arbre (sa base)}
\vedette{puu-ce}
\région{GOs PA BO}
\end{entrée}

\begin{entrée}
{erythrine à épines}
\vedette{mõõmõ}
\région{GOs PA BO}
\end{entrée}

\begin{entrée}
{erythrine "peuplier"}
\vedette{dra-wawe}
\région{GOs}
\variante{%
\vedette{da-whaawe}
\région{BO (Corne)}}
\variante{%
\vedette{dra-wapwe}
\région{vx}}
\end{entrée}

\begin{entrée}
{faux gaïac}
\vedette{maya}\homonyme{1}
\région{GOs}
\variante{%
\vedette{mãã}
\région{PA}}
\variante{%
\vedette{mhaa}
\région{WEM}}
\variante{%
\vedette{maca}
\région{BO (Corne)}}
\end{entrée}

\begin{entrée}
{faux manguier (dont le fruit contient un noyau très toxique, utilisé comme poison pour la pêche)}
\vedette{khööjo}
\région{GOs}
\variante{%
\vedette{koojòng, khoojòng}
\région{PA BO}}
\end{entrée}

\begin{entrée}
{ficus}
\vedette{camhãã}
\région{PA BO}
\variante{%
\vedette{kyamhãã}
\région{BO (Corne)}}
\end{entrée}

\begin{entrée}
{flamboyant}
\vedette{ce-vada}
\région{GOs PA}
\variante{%
\vedette{ce-pada}
\région{GO(s)}}
\end{entrée}

\begin{entrée}
{gaïac}
\vedette{mîjo}
\région{BO JAWE}
\end{entrée}

\begin{entrée}
{gaïac}
\vedette{mhãã}\homonyme{1}
\région{PA WEM BO}
\variante{%
\vedette{maak, mheek}
\région{BO}}
\end{entrée}

\begin{entrée}
{"gommier"}
\vedette{zaalo}
\région{GOs PA}
\variante{%
\vedette{zhaalo}
\région{GA}}
\variante{%
\vedette{yaalo}
\région{BO}}
\end{entrée}

\begin{entrée}
{goyavier}
\vedette{guya}\homonyme{1}
\région{GOs}
\variante{%
\vedette{gwayal}
\région{PA}}
\end{entrée}

\begin{entrée}
{houp}
\vedette{ce-xou}
\région{GOs PA}
\variante{%
\vedette{hup}
\région{PA}}
\end{entrée}

\begin{entrée}
{jamelonier (sorte de prunier sauvage) ; jamblon}
\vedette{caai}\homonyme{2}
\région{GOs PA BO}
\variante{%
\vedette{caak, caai}
\région{BO}}
\variante{%
\vedette{samelõ}
\région{GO(s)}}
\end{entrée}

\begin{entrée}
{kaori}
\vedette{jeyu}
\région{GOs WE}
\variante{%
\vedette{jeü}
\région{PA BO [BM]}}
\end{entrée}

\begin{entrée}
{mandarinier}
\vedette{medatri}
\région{GOs}
\end{entrée}

\begin{entrée}
{manguier}
\vedette{mãã}\homonyme{2}
\région{GOs}
\variante{%
\vedette{maak}
\région{PA BO}}
\end{entrée}

\begin{entrée}
{morindier ; fromager (petit arbre, arbuste sert de médicament et écorce a des propriétés teinturales, jaune)}
\vedette{hilò}
\région{GO PA}
\variante{%
\vedette{hiloo}}
\end{entrée}

\begin{entrée}
{mûrier}
\vedette{mwaitri}
\région{GO}
\end{entrée}

\begin{entrée}
{niaoulis}
\vedette{zòòni}
\région{GOs PA}
\variante{%
\vedette{zhòòni}
\région{GO(s)}}
\variante{%
\vedette{yooni}
\région{BO (BM, Corne)}}
\variante{%
\vedette{yhoonik}
\région{BO (Corne)}}
\end{entrée}

\begin{entrée}
{orange(r)}
\vedette{orã}
\région{GOs BO}
\end{entrée}

\begin{entrée}
{oranger sauvage ; faux-oranger}
\vedette{buvaa}
\région{GOs BO PA WEM}
\variante{%
\vedette{bupaa}
\région{GO vx}}
\end{entrée}

\begin{entrée}
{palétuvier (à feuilles comestibles)}
\vedette{ûdo}
\région{GOs}
\end{entrée}

\begin{entrée}
{palétuvier (à fruit comestible)}
\vedette{kò}\homonyme{1}
\région{GOs BO}
\end{entrée}

\begin{entrée}
{palétuvier [Corne]}
\vedette{jigo}
\région{BO}
\end{entrée}

\begin{entrée}
{palétuvier gris}
\vedette{kèè}
\région{GOs}
\end{entrée}

\begin{entrée}
{palétuvier gris (court)}
\vedette{hivwi}\homonyme{1}
\région{GOs}
\end{entrée}

\begin{entrée}
{palétuvier rouge (racines aériennes)}
\vedette{kîbö}
\région{GOs PA BO}
\variante{%
\vedette{keebo}
\région{BO}}
\end{entrée}

\begin{entrée}
{palmier calédonien (à croissance rapide)}
\vedette{nuãda}
\région{GOs PA}
\end{entrée}

\begin{entrée}
{peuplier kanak (représente la terre et la femme)}
\vedette{mõõmõ}
\région{GOs PA BO}
\end{entrée}

\begin{entrée}
{pin colonnaire (représente le côté mâle)}
\vedette{waawè}
\région{GOs PA BO}
\variante{%
\vedette{waapwè}
\région{GO(s) vx BO (Corne)}}
\end{entrée}

\begin{entrée}
{pomme-rose [PA]}
\vedette{caai}\homonyme{2}
\région{GOs PA BO}
\variante{%
\vedette{caak, caai}
\région{BO}}
\variante{%
\vedette{samelõ}
\région{GO(s)}}
\end{entrée}

\begin{entrée}
{pommier canaque}
\vedette{caai}\homonyme{2}
\région{GOs PA BO}
\variante{%
\vedette{caak, caai}
\région{BO}}
\variante{%
\vedette{samelõ}
\région{GO(s)}}
\end{entrée}

\begin{entrée}
{souche ; base de l'arbre}
\vedette{bwevwu-cee}
\région{GOs}
\end{entrée}

\begin{entrée}
{tamanou}
\vedette{phiu}
\région{GOs}
\end{entrée}

\begin{entrée}
{tamanou (faux)}
\vedette{mhe}
\région{GOs}
\end{entrée}

\subsection{Description des végétaux}

\subsubsection{Noms des plantes}

\begin{entrée}
{algues de rivière}
\classe{nom}
\vedette{hii}
\sens{2}
\région{GOs BO PA}
\variante{%
\vedette{yi-n}}
\end{entrée}

\begin{entrée}
{aloes}
\vedette{lalue}
\région{GOs PA}
\end{entrée}

\begin{entrée}
{Alphitonia}
\classe{nom}
\vedette{bòdra}
\sens{2}
\région{GOs}
\variante{%
\vedette{bòda}
\région{BO}}
\end{entrée}

\begin{entrée}
{arbuste ; bagayou des vieux (voir le livre des plantes du chemin kanak)}
\vedette{ce-ka, ce-xa}
\région{GOs}
\variante{%
\vedette{ce-kam}
\région{BO (Corne)}}
\end{entrée}

\begin{entrée}
{bambou}
\classe{nom}
\vedette{gò}\homonyme{1}
\sens{1}
\région{GOs BO PA}
\end{entrée}

\begin{entrée}
{bambou (petit, bousse au bord de la rivière)}
\vedette{hãgu}
\région{GOs}
\end{entrée}

\begin{entrée}
{bois de sang ; "sang dragon", Euphorbiacée}
\vedette{ce-kura}
\région{GOs PA}
\variante{%
\vedette{ce-kutra}
\région{GO(s)}}
\end{entrée}

\begin{entrée}
{brède}
\vedette{hê-ka}
\région{GO}
\end{entrée}

\begin{entrée}
{brède à feuilles comestibles ; Morelle noire}
\vedette{dròò-ko}
\région{GOs}
\variante{%
\vedette{dòò-ko}
\région{BO}}
\end{entrée}

\begin{entrée}
{brède (sorte de pissenlit) ; laiteron}
\vedette{phê}
\région{GOs}
\variante{%
\vedette{phêng}
\région{PA}}
\variante{%
\vedette{pang, phang}
\région{BO [BM]}}
\end{entrée}

\begin{entrée}
{cactus sauvage}
\vedette{mwãgi}
\région{GOs}
\end{entrée}

\begin{entrée}
{canne à sucre}
\vedette{ê}\homonyme{2}
\région{GOs}
\variante{%
\vedette{èm}
\région{PA BO}}
\end{entrée}

\begin{entrée}
{"cassis" (arbuste épineux) ; épine}
\vedette{digöö}
\région{GOs}
\variante{%
\vedette{digoony}
\région{BO PA}}
\end{entrée}

\begin{entrée}
{champignon}
\classe{nom}
\vedette{mõ-pwal}
\sens{2}
\région{BO}
\end{entrée}

\begin{entrée}
{champignon (terme générique)}
\vedette{thrao}
\région{GOs}
\variante{%
\vedette{thao}
\région{PABO}}
\end{entrée}

\begin{entrée}
{chou kanak [GOs]}
\vedette{nawêni}
\région{GOs}
\variante{%
\vedette{naõni}
\région{GO(s)}}
\variante{%
\vedette{naõnil}
\région{PA BO}}
\variante{%
\vedette{naõnin}
\région{BO}}
\end{entrée}

\begin{entrée}
{citronnelle}
\vedette{mãebo}
\région{GOs PA BO}
\end{entrée}

\begin{entrée}
{coléus (symbole de vie)}
\vedette{paxawa}
\région{GOs PA}
\end{entrée}

\begin{entrée}
{cordyline (symbole masculin)}
\vedette{di}
\région{GOs PA BO}
\end{entrée}

\begin{entrée}
{cotonnier}
\vedette{kotô}
\région{GOs}
\end{entrée}

\begin{entrée}
{Crinum sp.}
\vedette{uxo}
\région{GO}
\end{entrée}

\begin{entrée}
{croton (lit. fleur-arbre-feuille)}
\vedette{mû-cee-dròò}
\région{GOs}
\end{entrée}

\begin{entrée}
{croton (protège les maisons et les êtres humains)}
\vedette{buu}\homonyme{2}
\région{PA}
\end{entrée}

\begin{entrée}
{cultures}
\vedette{phoê}
\région{GOs PA}
\end{entrée}

\begin{entrée}
{curcuma ; gingembre (comestible)}
\vedette{nye}
\région{GOs}
\variante{%
\vedette{nyèn}
\région{PA BO}}
\variante{%
\vedette{nhyèn}
\région{WEM}}
\end{entrée}

\begin{entrée}
{cycas}
\vedette{mwèèn}
\région{BO [BM]}
\end{entrée}

\begin{entrée}
{cycas (Corne)}
\vedette{juyu}
\région{BO}
\end{entrée}

\begin{entrée}
{'épinard' [PA]}
\vedette{nawêni}
\région{GOs}
\variante{%
\vedette{naõni}
\région{GO(s)}}
\variante{%
\vedette{naõnil}
\région{PA BO}}
\variante{%
\vedette{naõnin}
\région{BO}}
\end{entrée}

\begin{entrée}
{épinard (sorte d') ; feuille d'Aramanthus ; brède pariétaire (herbe à feuilles comestibles)}
\vedette{dòlògò}
\région{GOs}
\variante{%
\vedette{dològòm}
\région{PA BO}}
\end{entrée}

\begin{entrée}
{épineux}
\vedette{pu-döölia}
\région{GOs}
\end{entrée}

\begin{entrée}
{erythrinier à piquants (à fleurs rouges)}
\vedette{thre-mii}
\région{GOs}
\variante{%
\vedette{the-mii}
\région{PA BO}}
\end{entrée}

\begin{entrée}
{figuier sauvage}
\classe{nom}
\vedette{wha}\homonyme{1}
\région{GOs PA BO}
\end{entrée}

\begin{entrée}
{fougère arborescente}
\vedette{kavuxavwu}
\région{GOs}
\variante{%
\vedette{kavu-avu}
\région{PA}}
\variante{%
\vedette{kavu-kavu}
\région{BO (Corne)}}
\end{entrée}

\begin{entrée}
{fougère lianescente [Corne]}
\vedette{thano}
\région{BO}
\end{entrée}

\begin{entrée}
{fougère (petite)}
\vedette{pò}\homonyme{2}
\région{GOs}
\variante{%
\vedette{pòl}
\région{BO}}
\variante{%
\vedette{pul}
\région{PA}}
\end{entrée}

\begin{entrée}
{gingembre (culinaire)}
\classe{nom}
\vedette{katri}
\sens{1}
\région{GOs}
\variante{%
\vedette{kari}
\région{PA BO}}
\end{entrée}

\begin{entrée}
{gingembre mâle non comestible}
\vedette{nyèn êmwen}
\région{PA}
\end{entrée}

\begin{entrée}
{haricot}
\vedette{pû}
\région{GOs PA BO}
\end{entrée}

\begin{entrée}
{herbe}
\vedette{pwêng}
\région{PA BO}
\end{entrée}

\begin{entrée}
{"herbe à éléphant"}
\vedette{giner}
\région{PA}
\end{entrée}

\begin{entrée}
{herbe à paille à tige longue (pousse en touffe, utilisée pour le torchis)}
\classe{nom}
\vedette{phu}\homonyme{2}
\sens{1}
\région{GOs PA BO}
\end{entrée}

\begin{entrée}
{herbe coupante et malodorante}
\vedette{boxola}
\région{PA}
\end{entrée}

\begin{entrée}
{herbe ; paille ; chaume}
\vedette{mãe}
\région{GOs}
\variante{%
\vedette{mae}
\région{BO PA}}
\variante{%
\vedette{mai}
\région{BO}}
\end{entrée}

\begin{entrée}
{herbe ; pelouse}
\vedette{drawalu}
\région{GOs}
\end{entrée}

\begin{entrée}
{herbe ; pelouse}
\vedette{mèròò}
\région{PA BO WEM WE}
\variante{%
\vedette{mããro}
\région{BO [BM]}}
\end{entrée}

\begin{entrée}
{herbe (poussant le long des rivières)}
\vedette{para}\homonyme{1}
\région{BO [Corne]}
\end{entrée}

\begin{entrée}
{herbes vertes (poussant dans les rivières ou terrains marécageux)}
\vedette{paxa}
\région{PA}
\end{entrée}

\begin{entrée}
{hibiscus}
\vedette{driluu}
\région{GOs}
\variante{%
\vedette{dilu, dilo}
\région{PA BO}}
\variante{%
\vedette{diluuc}
\région{BO (Corne)}}
\end{entrée}

\begin{entrée}
{laiteron ; feuille de "pissenlit"}
\vedette{dròò-phê}
\région{GOs}
\variante{%
\vedette{dròò-phoã}
\région{GOs}}
\end{entrée}

\begin{entrée}
{"langue de femme"}
\vedette{kûmèè tòòmwa}
\région{BO}
\end{entrée}

\begin{entrée}
{lantana}
\vedette{kacia}
\région{BO [BM]}
\end{entrée}

\begin{entrée}
{liane}
\vedette{ku}\homonyme{2}
\région{PA BO}
\end{entrée}

\begin{entrée}
{liane (à pétiole subéreux)}
\vedette{wa-cama}
\région{GOs}
\end{entrée}

\begin{entrée}
{liane de bord de mer (utilisée pour la pêche au poison)}
\vedette{kîîgaze}
\région{GOs}
\end{entrée}

\begin{entrée}
{liane de forêt (utilisée pour la construction de cases [Corne])}
\vedette{kâduk}
\région{BO}
\variante{%
\vedette{kâdu}}
\end{entrée}

\begin{entrée}
{liane (des endroits humides, à fleur blanche et à grosses feuilles, parasite des arbres)}
\vedette{throobwa}
\région{GOs}
\variante{%
\vedette{thobwang}
\région{PA}}
\end{entrée}

\begin{entrée}
{liane ; salsepareille}
\vedette{caanô}
\région{PA BO (Corne)}
\end{entrée}

\begin{entrée}
{liane sauvage}
\vedette{kâgao}
\région{GOs BO [Corne]}
\end{entrée}

\begin{entrée}
{liane utilisée pour attacher les gaulettes de la maison}
\vedette{hau}\homonyme{2}
\région{PA}
\end{entrée}

\begin{entrée}
{lys d'eau [BO]}
\classe{nom}
\vedette{mõ-we}
\sens{2}
\région{GOs}
\end{entrée}

\begin{entrée}
{magnania (petit tubercule sauvage)}
\vedette{calii}
\région{GOs PA BO}
\end{entrée}

\begin{entrée}
{maïs}
\vedette{maic}
\région{PA}
\end{entrée}

\begin{entrée}
{maïs (épi de)}
\vedette{pò-pwaale}
\région{GOs}
\variante{%
\vedette{pò-vwale}
\région{GO(s)}}
\end{entrée}

\begin{entrée}
{manioc}
\vedette{manyô}
\région{GOs PA}
\end{entrée}

\begin{entrée}
{mimosa de forêt (fleur à pompon jaune)}
\vedette{mããle}
\région{GOs}
\variante{%
\vedette{maalèm}
\région{BO PA}}
\end{entrée}

\begin{entrée}
{mimosa (faux) [Corne]}
\vedette{amaèk}
\région{BO}
\end{entrée}

\begin{entrée}
{Mimusops parviflora}
\vedette{ce-kui}
\région{GOs}
\variante{%
\vedette{ce-xui}
\région{GO(s)}}
\end{entrée}

\begin{entrée}
{mousse verte de rivière}
\vedette{thre}
\région{GOs}
\variante{%
\vedette{the}
\région{BO}}
\end{entrée}

\begin{entrée}
{nom de la purge à base de Polygonum subsessile (Corne)}
\vedette{phaa-tuuçò}
\région{PA BO}
\variante{%
\vedette{pha-ruyòng, pha-thuyòng}
\région{PA}}
\variante{%
\vedette{phaxayuk}
\région{BO}}
\end{entrée}

\begin{entrée}
{nom des plantes dont les feuilles sont multicolores (comme les crotons)}
\vedette{mû-cee-dròò}
\région{GOs}
\end{entrée}

\begin{entrée}
{pandanus}
\vedette{thra}\homonyme{3}
\région{GOs}
\variante{%
\vedette{thal}
\région{PA BO}}
\end{entrée}

\begin{entrée}
{pandanus sauvage (bord de creek, sert à tresser, au bord des creek, les roussettes mangent les fruits)}
\vedette{phivwâi}
\région{GOs}
\end{entrée}

\begin{entrée}
{patate douce}
\vedette{kumala}
\région{GOs BO PA}
\variante{%
\vedette{kumwala}
\région{BO PA}}
\end{entrée}

\begin{entrée}
{plante}
\vedette{gorolo}
\région{BO [BM, Corne]}
\variante{%
\vedette{gòròlò}
\région{BO [BM]}}
\end{entrée}

\begin{entrée}
{plante}
\vedette{huzooni}
\région{GO}
\end{entrée}

\begin{entrée}
{plante}
\vedette{thraalo}
\région{GO}
\variante{%
\vedette{thaalo}
\région{BO (Corne)}}
\end{entrée}

\begin{entrée}
{plante à fruits rouges (attire les roussettes)}
\vedette{yabo}
\région{GOs}
\end{entrée}

\begin{entrée}
{plante (générique)}
\vedette{phoê}
\région{GOs PA}
\end{entrée}

\begin{entrée}
{plante ; graminée}
\vedette{phaa-tuuçò}
\région{PA BO}
\variante{%
\vedette{pha-ruyòng, pha-thuyòng}
\région{PA}}
\variante{%
\vedette{phaxayuk}
\région{BO}}
\end{entrée}

\begin{entrée}
{plante qui pousse au ras du sol et fait des petites fleurs bleues}
\vedette{panooli}
\région{BO [Corne]}
\end{entrée}

\begin{entrée}
{pois d'angole ; Ambrevade}
\vedette{gana}
\région{GOs BO}
\end{entrée}

\begin{entrée}
{pois d'angole ; Ambrevade}
\vedette{koe}\homonyme{1}
\région{GOs}
\end{entrée}

\begin{entrée}
{pomme cythère}
\vedette{pègalò}
\région{GOs}
\end{entrée}

\begin{entrée}
{roseau}
\vedette{hãgu}
\région{GOs}
\end{entrée}

\begin{entrée}
{roseau}
\vedette{hõgo}
\région{GOs}
\end{entrée}

\begin{entrée}
{roseau}
\vedette{hûda}
\région{GOs PA BO}
\end{entrée}

\begin{entrée}
{salsepareille}
\vedette{jaa}\homonyme{1}
\région{GOs PA BO}
\variante{%
\vedette{jaac}
\région{BO (Corne)}}
\end{entrée}

\begin{entrée}
{sensitive}
\vedette{ce-mããni}
\région{GOs}
\end{entrée}

\begin{entrée}
{Triumfetta rhomboïdea}
\vedette{waat}
\région{BO [Corne]}
\end{entrée}

\begin{entrée}
{Urticacée (les fibres servent à faire des nasses, cordes de fronde)}
\vedette{phulè}
\région{PA BO}
\variante{%
\vedette{phulèng}}
\variante{%
\vedette{phulek}
\région{BO (Corne)}}
\end{entrée}

\begin{entrée}
{vétiver [BO]}
\vedette{mãe}
\région{GOs}
\variante{%
\vedette{mae}
\région{BO PA}}
\variante{%
\vedette{mai}
\région{BO}}
\end{entrée}

\subsubsection{Parties de plantes}

\begin{entrée}
{base}
\classe{nom}
\vedette{puu}\homonyme{1}
\sens{1}
\région{GOs PA BO}
\variante{%
\vedette{pu}
\région{GO(s)}}
\variante{%
\vedette{puu-n}
\région{BO PA}}
\variante{%
\vedette{puxu-n}
\région{BO}}
\end{entrée}

\begin{entrée}
{bourgeon}
\classe{nom}
\vedette{kumè}
\sens{2}
\région{GOs BO PA}
\end{entrée}

\begin{entrée}
{bourgeon}
\vedette{kuvêê}
\région{GOs PA}
\variante{%
\vedette{kuveen}
\région{BO}}
\end{entrée}

\begin{entrée}
{bourgeons ; rejet (de plante) ; germe}
\vedette{kibo}
\région{GOs PA BO}
\variante{%
\vedette{kîbwòn}
\région{PA}}
\end{entrée}

\begin{entrée}
{bout de l'inflorescence de bananier}
\vedette{cò-chaamwa}
\région{GOs}
\end{entrée}

\begin{entrée}
{bouton (en) ; non éclos [Corne]}
\vedette{peera}
\région{BO}
\end{entrée}

\begin{entrée}
{branche}
\classe{nom}
\vedette{hii}
\sens{3}
\région{GOs BO PA}
\variante{%
\vedette{yi-n}}
\end{entrée}

\begin{entrée}
{cime (arbre)}
\vedette{kumee}
\région{GOs BO}
\end{entrée}

\begin{entrée}
{écorce}
\classe{nom}
\vedette{cii}\homonyme{1}
\sens{2}
\région{GOs PA BO}
\end{entrée}

\begin{entrée}
{écorce}
\vedette{cii-ce}
\région{GO PA}
\variante{%
\vedette{ci-cee}
\région{PA}}
\end{entrée}

\begin{entrée}
{écorce de bourao (sert à la confection des jupes anciennes)}
\vedette{cii-pòò}
\région{GOs}
\end{entrée}

\begin{entrée}
{entre-noeuds (bambou, canne à sucre)}
\classe{nom}
\vedette{phajoo}
\sens{1}
\région{GOs PA BO}
\end{entrée}

\begin{entrée}
{épine de}
\vedette{döölia}
\région{GOs PA BO}
\variante{%
\vedette{dolia}
\région{PA BO}}
\end{entrée}

\begin{entrée}
{épines, piquants de la nervure centrale de pandanus}
\vedette{döölia thra}
\région{GOs}
\end{entrée}

\begin{entrée}
{feuille}
\vedette{dròò}
\région{GOs}
\variante{%
\vedette{dòò}
\région{PA BO}}
\end{entrée}

\begin{entrée}
{feuille de canne à sucre}
\vedette{dròò ê}
\région{GOs}
\variante{%
\vedette{dòòèm}
\région{PA}}
\end{entrée}

\begin{entrée}
{feuilles de pandanus de creek}
\vedette{mazii}
\région{GOs}
\end{entrée}

\begin{entrée}
{fibres pour panier}
\vedette{hoxaba}
\région{PA BO}
\variante{%
\vedette{hoyaba}
\région{BO [Corne]}}
\end{entrée}

\begin{entrée}
{fleur}
\vedette{mû-cee}
\région{GOs}
\end{entrée}

\begin{entrée}
{fruit ; graine}
\vedette{pò}\homonyme{1}
\sens{1}
\région{GOs}
\variante{%
\vedette{pò-n}
\région{PA BO}}
\variante{%
\vedette{pwò}
\région{BO}}
\end{entrée}

\begin{entrée}
{graine}
\vedette{êdoa}
\région{GOs}
\variante{%
\vedette{êdo}
\région{GO(s)}}
\end{entrée}

\begin{entrée}
{inflorescence de bananier}
\vedette{mû-chaamwa}
\région{GOs}
\end{entrée}

\begin{entrée}
{jeunes feuilles encore roulées qui sortent du coeur de la plante (taro)}
\vedette{kibo}
\région{GOs PA BO}
\variante{%
\vedette{kîbwòn}
\région{PA}}
\end{entrée}

\begin{entrée}
{jeunes feuilles (par ex. de taro, quand la feuille commence à se déplier)}
\vedette{kuvêê}
\région{GOs PA}
\variante{%
\vedette{kuveen}
\région{BO}}
\end{entrée}

\begin{entrée}
{jonc à corbeille [Corne]}
\vedette{kaaje}
\région{BO}
\variante{%
\vedette{kayè}}
\end{entrée}

\begin{entrée}
{maïs (pied de)}
\vedette{kò-pò-pwaale}
\région{GOs}
\end{entrée}

\begin{entrée}
{noeud de bambou}
\vedette{pudi-go}
\région{GOs}
\variante{%
\vedette{puding-go}
\région{PA BO}}
\end{entrée}

\begin{entrée}
{noyau ; pépin}
\vedette{êdoa}
\région{GOs}
\variante{%
\vedette{êdo}
\région{GO(s)}}
\end{entrée}

\begin{entrée}
{paille ; brindille}
\classe{nom}
\vedette{ja}\homonyme{2}
\sens{1}
\région{BO [Corne]}
\variante{%
\vedette{jan}
\région{BO [BM]}}
\end{entrée}

\begin{entrée}
{pied principal d'une plante (lit. mère des fleurs)}
\vedette{ôã-muu-ce}
\région{GOs}
\variante{%
\vedette{ô-muuc}
\région{PA}}
\end{entrée}

\begin{entrée}
{pied ; tronc}
\classe{nom}
\vedette{puu}\homonyme{1}
\sens{1}
\région{GOs PA BO}
\variante{%
\vedette{pu}
\région{GO(s)}}
\variante{%
\vedette{puu-n}
\région{BO PA}}
\variante{%
\vedette{puxu-n}
\région{BO}}
\end{entrée}

\begin{entrée}
{pousses (toutes plantes)}
\vedette{kuvêê}
\région{GOs PA}
\variante{%
\vedette{kuveen}
\région{BO}}
\end{entrée}

\begin{entrée}
{racine}
\vedette{wa}\homonyme{2}
\région{GO}
\région{PABO}
\variante{%
\vedette{wal}}
\end{entrée}

\begin{entrée}
{racine de (forme en composition ou détermination de wa(l) 'racine')}
\vedette{we}\homonyme{2}
\région{GOs PA}
\région{PA BO}
\variante{%
\vedette{wèè-n}}
\end{entrée}

\begin{entrée}
{rejet d'arbuste ou d'arbre taillé}
\vedette{khoriing}
\région{PA}
\end{entrée}

\begin{entrée}
{rejet (de plante)}
\vedette{kîbwò}
\région{GOs}
\variante{%
\vedette{kîbwòn}
\région{PA}}
\end{entrée}

\begin{entrée}
{résine}
\vedette{oya}
\région{PA}
\end{entrée}

\begin{entrée}
{résine ; sève}
\vedette{dixa-ce}
\région{GOs PA}
\end{entrée}

\begin{entrée}
{sève}
\vedette{we-ce}
\région{GOs BO PA}
\end{entrée}

\begin{entrée}
{souche ; base de l'arbre}
\vedette{bwevwu-cee}
\région{GOs}
\end{entrée}

\begin{entrée}
{tige de maïs}
\vedette{kò-pò-pwaale}
\région{GOs}
\end{entrée}

\begin{entrée}
{tige de taro (tige principale) (lit. mère du taro)}
\vedette{ô-uvia}
\région{GOs}
\end{entrée}

\begin{entrée}
{tronc}
\classe{nom}
\vedette{gòò}
\groupe{A}
\sens{2}
\région{GOs PA BO}
\end{entrée}

\begin{entrée}
{tronc d'arbre}
\vedette{gòò-ce}
\région{GOs}
\end{entrée}

\begin{entrée}
{tronc ; souche}
\classe{nom}
\vedette{bweevwu}
\sens{1}
\région{GOs PA BO}
\end{entrée}

\begin{entrée}
{tronc ; souche ; bout}
\vedette{bwe}\homonyme{1}
\région{BO}
\variante{%
\vedette{bwee}
\région{BO}}
\end{entrée}

\begin{entrée}
{tubercule comestible}
\classe{nom}
\vedette{pai}\homonyme{2}
\sens{1}
\région{GOs}
\variante{%
\vedette{pain}
\région{BO PA}}
\end{entrée}

\begin{entrée}
{tubercule de patate douce}
\vedette{paxa-kumwala}
\région{GOs PA}
\end{entrée}

\begin{entrée}
{tubercule d'igname}
\vedette{paxa-kui}
\région{GOs BO PA}
\end{entrée}

\begin{entrée}
{tubercule du taro d'eau}
\vedette{paxa-uva}
\région{GOs BO PA}
\end{entrée}

\begin{entrée}
{tubercule du taro d'eau (uvha) ; taro d'eau (terme employé dans le contexte coutumier)}
\vedette{kutru}
\région{GOs}
\variante{%
\vedette{kuru}
\région{GO(s)}}
\variante{%
\vedette{kuru}
\région{PA BO}}
\end{entrée}

\subsubsection{Processus liés aux plantes}

\begin{entrée}
{bien formé}
\vedette{zeenô}
\région{GOs PA}
\variante{%
\vedette{zheenô}
\région{GA}}
\variante{%
\vedette{yhèno, zeno}
\région{BO}}
\end{entrée}

\begin{entrée}
{commencer à mûrir (tubercules, fruits)}
\vedette{te}
\région{PA}
\end{entrée}

\begin{entrée}
{croître ; germer}
\vedette{ki}\homonyme{2}
\région{GOs WEM}
\variante{%
\vedette{kim}
\région{PA BO}}
\end{entrée}

\begin{entrée}
{éclore}
\classe{v}
\vedette{gaò}
\sens{1}
\région{GOs PA}
\end{entrée}

\begin{entrée}
{éclore}
\vedette{khibii}
\région{GOs}
\end{entrée}

\begin{entrée}
{fané}
\vedette{mènõ}
\région{GOs}
\variante{%
\vedette{mènõng}
\région{PA BO}}
\end{entrée}

\begin{entrée}
{faner (feuilles d'arbre après un feu)}
\vedette{bò}\homonyme{3}
\région{GOs}
\variante{%
\vedette{bòng}
\région{BO [BM]}}
\end{entrée}

\begin{entrée}
{flétri ; fané ; mort ; feuille fanée}
\vedette{havan}
\région{BO}
\end{entrée}

\begin{entrée}
{fleur ; fleurir}
\vedette{mû}\homonyme{1}
\région{GO}
\variante{%
\vedette{muu}
\région{PA BO}}
\variante{%
\vedette{muuc}
\région{BO}}
\end{entrée}

\begin{entrée}
{fleurir}
\vedette{gaawe}
\région{GOs}
\end{entrée}

\begin{entrée}
{jeune (fruit)}
\classe{v.stat.}
\vedette{aava}
\sens{1}
\région{GOs}
\end{entrée}

\begin{entrée}
{multiplier (se) ; faire des feuilles (arbres)}
\vedette{tili}
\région{GOs}
\end{entrée}

\begin{entrée}
{mûr}
\classe{v.stat.}
\vedette{mii}
\sens{2}
\région{GOs PA BO}
\end{entrée}

\begin{entrée}
{mûr ; arrivé à maturité}
\vedette{zeenô}
\région{GOs PA}
\variante{%
\vedette{zheenô}
\région{GA}}
\variante{%
\vedette{yhèno, zeno}
\région{BO}}
\end{entrée}

\begin{entrée}
{mûr (fruits)}
\vedette{gala-mii}
\région{GOs}
\end{entrée}

\begin{entrée}
{perdre\_ses\_feuilles}
\classe{v}
\vedette{kuli}
\sens{2}
\région{GOs}
\variante{%
\vedette{kule}
\région{PA BO}}
\end{entrée}

\begin{entrée}
{pourri; effriter (s')}
\vedette{bulago}
\région{GOs}
\end{entrée}

\begin{entrée}
{pousser ; grandir (plantes)}
\vedette{ki}\homonyme{2}
\région{GOs WEM}
\variante{%
\vedette{kim}
\région{PA BO}}
\end{entrée}

\begin{entrée}
{presque mûr}
\vedette{zaawane}\homonyme{2}
\région{GOs}
\end{entrée}

\begin{entrée}
{produire ; donner (des fruits, tubercules)}
\vedette{pua}
\région{PA}
\end{entrée}

\begin{entrée}
{repousse des feuilles}
\classe{v}
\vedette{cabo}
\sens{3}
\région{GOs}
\variante{%
\vedette{cabwòl, cabòl}
\région{PA BO WEM}}
\end{entrée}

\begin{entrée}
{reverdir}
\classe{v}
\vedette{cabo}
\sens{3}
\région{GOs}
\variante{%
\vedette{cabwòl, cabòl}
\région{PA BO WEM}}
\end{entrée}

\begin{entrée}
{roussi (par le feu)}
\vedette{bò}\homonyme{3}
\région{GOs}
\variante{%
\vedette{bòng}
\région{BO [BM]}}
\end{entrée}

\begin{entrée}
{séché ; desséché (plantes)}
\vedette{mènõ}
\région{GOs}
\variante{%
\vedette{mènõng}
\région{PA BO}}
\end{entrée}

\begin{entrée}
{tomber (tout seul: fruit, feuilles)}
\classe{v}
\vedette{kuli}
\sens{1}
\région{GOs}
\variante{%
\vedette{kule}
\région{PA BO}}
\end{entrée}

\begin{entrée}
{vert (fruit) ; pas mûr}
\vedette{bu}\homonyme{1}
\région{GOs PA BO}
\end{entrée}

\begin{entrée}
{vert ; pas mûr}
\classe{v.stat.}
\vedette{aava}
\sens{1}
\région{GOs}
\end{entrée}

\begin{entrée}
{vert (tubercules, fruits)}
\vedette{bee}\homonyme{2}
\région{PA}
\end{entrée}

\subsection{Cocotiers}

\begin{entrée}
{bourre de coco}
\vedette{ci-nu}
\région{GOs}
\end{entrée}

\begin{entrée}
{chair de coco (lit. contenu de la noix de coco)}
\vedette{hê-nu}
\région{GOs}
\end{entrée}

\begin{entrée}
{coco germé}
\vedette{kuvêê nu}
\région{PA}
\end{entrée}

\begin{entrée}
{coco germé}
\vedette{nu-ki}
\région{GOs}
\variante{%
\vedette{nu-kim}
\région{PA}}
\end{entrée}

\begin{entrée}
{coco (noix de)}
\vedette{nu}\homonyme{1}
\région{GOs}
\variante{%
\vedette{nu}
\région{BO PA}}
\end{entrée}

\begin{entrée}
{coco sec}
\vedette{nu-mãû}
\région{GOs}
\end{entrée}

\begin{entrée}
{cocotier}
\vedette{nu}\homonyme{1}
\région{GOs}
\variante{%
\vedette{nu}
\région{BO PA}}
\end{entrée}

\begin{entrée}
{coco vert}
\vedette{nu-wee}
\région{GOs}
\end{entrée}

\begin{entrée}
{coeur de cocotier (comestible)}
\vedette{kumèè nu}
\région{GO PA BO}
\end{entrée}

\begin{entrée}
{coquille vide de noix de coco}
\vedette{pii-nu}
\région{GOs}
\variante{%
\vedette{pi-nu}
\région{GO(s)}}
\end{entrée}

\begin{entrée}
{décortiquer le coprah (avec un couteau)}
\vedette{thiò nu}
\région{GOs PA}
\variante{%
\vedette{thixò}
\région{GO(s) PA BO}}
\end{entrée}

\begin{entrée}
{eau de coco ; coco à boire}
\vedette{we-nu}
\région{GOs BO}
\end{entrée}

\begin{entrée}
{étoffe}
\vedette{û}
\région{GOs}
\end{entrée}

\begin{entrée}
{fibre de coco}
\vedette{dii-nu}
\région{GOs BO}
\end{entrée}

\begin{entrée}
{fibre de feuille de cocotier}
\vedette{haza nu}
\région{GOs}
\variante{%
\vedette{hara-nu}
\région{PA}}
\end{entrée}

\begin{entrée}
{fibre prise sur la nervure centrale de la palme de cocotier (sert de lien)}
\vedette{khara-a nu}
\région{PA BO}
\end{entrée}

\begin{entrée}
{gaine de l'inflorescence du cocotier [Corne]}
\vedette{põng}
\région{BO}
\end{entrée}

\begin{entrée}
{gaine de l'inflorescence du cocotier (gaine sèche qui reste de l'inflorescence)}
\vedette{haa-nu}
\région{PA BO}
\end{entrée}

\begin{entrée}
{germe du coco}
\vedette{kîbwoo-nu}
\région{GOs}
\end{entrée}

\begin{entrée}
{lait de coco ; huile de coco}
\vedette{dixa-nu}
\région{GOs PA BO}
\variante{%
\vedette{dika-nu}
\région{GO(s) BO}}
\end{entrée}

\begin{entrée}
{nervure centrale de la palme de cocotier}
\vedette{ko-a dròò-nu}
\région{GOs PA}
\end{entrée}

\begin{entrée}
{nervure centrale des folioles de palmes de cocotier (sert à faire des petits balais)}
\vedette{goo}
\région{GOs}
\variante{%
\vedette{goony}
\région{PA BO}}
\end{entrée}

\begin{entrée}
{partie inférieure de la palme de cocotier (base arrondie de la palme là où elle s'attache au tronc)}
\vedette{haza nu}
\région{GOs}
\variante{%
\vedette{hara-nu}
\région{PA}}
\end{entrée}

\begin{entrée}
{rachis de coco}
\vedette{pô-nu}
\région{GOs}
\variante{%
\vedette{pon}
\région{BO}}
\end{entrée}

\begin{entrée}
{rachis de coco}
\vedette{throli}
\région{GOs}
\end{entrée}

\begin{entrée}
{spathe de cocotier}
\vedette{û}
\région{GOs}
\end{entrée}

\begin{entrée}
{spathe de cocotier (feuille qui enveloppe l'inflorescence ou qui est à la base du pédoncule floral)}
\vedette{buleony}
\région{BO [Corne, BM]}
\variante{%
\vedette{buleon}}
\end{entrée}

\subsection{Ignames}

\begin{entrée}
{billon}
\vedette{nò-khia}
\région{BO}
\end{entrée}

\begin{entrée}
{bouture d'igname (à partir de l'extrémité inférieure de l'igname)}
\vedette{bwe-kui}
\région{GOs PA}
\end{entrée}

\begin{entrée}
{butte d'igname}
\vedette{bu-kui}
\région{GOs PA}
\end{entrée}

\begin{entrée}
{corps de l'igname}
\vedette{gòò-kui}
\région{PA}
\end{entrée}

\begin{entrée}
{côté mâle du massif d'ignames (Dubois)}
\vedette{nò-khia}
\région{BO}
\end{entrée}

\begin{entrée}
{côté mâle du massif d'ignames [Haudricourt]}
\vedette{bwana}
\région{GO}
\end{entrée}

\begin{entrée}
{extrêmité inférieure de l'igname}
\vedette{thrô-kui}
\région{GOs}
\variante{%
\vedette{thô-kui}
\région{PA BO}}
\variante{%
\vedette{pwe-nô kui}
\région{WEM}}
\end{entrée}

\begin{entrée}
{igname}
\vedette{ku-ãgu}
\région{GOs}
\end{entrée}

\begin{entrée}
{igname}
\vedette{kui}
\région{GOs BO PA}
\end{entrée}

\begin{entrée}
{igname}
\vedette{walei}
\région{GOs BO}
\end{entrée}

\begin{entrée}
{igname}
\vedette{zaòl}
\région{PA}
\end{entrée}

\begin{entrée}
{igname (2 sortes : blanche ou jaunâtre)}
\vedette{kû-jaa}\homonyme{1}
\région{GOs}
\end{entrée}

\begin{entrée}
{igname (au goût sucré)}
\vedette{ku-wee}
\région{GOs}
\end{entrée}

\begin{entrée}
{igname blanche}
\vedette{ku-gozi}
\région{GOs}
\end{entrée}

\begin{entrée}
{igname blanche (Dubois)}
\vedette{bea}
\région{BO}
\end{entrée}

\begin{entrée}
{igname blanche, tendre (Dubois)}
\vedette{zara}
\région{BO}
\end{entrée}

\begin{entrée}
{igname (clone)}
\vedette{kãjawa}
\région{GOs}
\variante{%
\vedette{kajaa}
\région{BO (Dubois)}}
\end{entrée}

\begin{entrée}
{igname (clone)}
\vedette{paawa}
\région{GOs}
\end{entrée}

\begin{entrée}
{igname (clone, à chair mauve)}
\vedette{ku-bwii}
\région{GOs}
\end{entrée}

\begin{entrée}
{igname (clone) (Dubois)}
\vedette{evadan}
\région{BO}
\end{entrée}

\begin{entrée}
{igname (clone ; ressemble à une tête de poule)}
\vedette{ku-ko}
\région{GOs PA BO}
\end{entrée}

\begin{entrée}
{igname du chef (comporte deux espèces: variété blanche "pwalamu" et violette "pwang"). Dubois}
\vedette{gu-kui}
\région{GOs BO}
\variante{%
\vedette{gu-kui, gu-xui}
\région{PA}}
\end{entrée}

\begin{entrée}
{igname du chef (variété blanche). Dubois}
\vedette{pwalamu}
\région{BO}
\end{entrée}

\begin{entrée}
{igname du chef (variété violette). (Dubois)}
\vedette{pwang}\homonyme{2}
\région{BO}
\end{entrée}

\begin{entrée}
{igname (grosse et longue)}
\vedette{poxabwa}
\région{GOs}
\variante{%
\vedette{paxâbwa}
\région{PA BO}}
\end{entrée}

\begin{entrée}
{igname (la tige est de la couleur d'un serpent)}
\vedette{ku-bweena}
\région{GOs BO}
\end{entrée}

\begin{entrée}
{igname longue (Charles + Dubois)}
\vedette{hevwe}
\région{GOs}
\variante{%
\vedette{yaave}
\région{PA BO}}
\end{entrée}

\begin{entrée}
{igname longue et dure}
\vedette{papua}
\région{GOs BO}
\end{entrée}

\begin{entrée}
{igname mauve}
\vedette{bwihin}
\région{BO}
\variante{%
\vedette{bwiin}
\région{PA}}
\end{entrée}

\begin{entrée}
{igname mauve}
\vedette{kuwe}
\région{GOs BO}
\end{entrée}

\begin{entrée}
{igname (petite)}
\vedette{kavobe}
\région{GOs BO}
\end{entrée}

\begin{entrée}
{igname (pousse comme une anguille)}
\vedette{ku-peena}
\région{GOs PA BO}
\end{entrée}

\begin{entrée}
{igname (qui ressort de terre)}
\vedette{ku-cabo}
\région{GOs}
\end{entrée}

\begin{entrée}
{igname (ressemble au fruit du figuier)}
\vedette{pò-wha}
\région{GOs}
\end{entrée}

\begin{entrée}
{igname ronde}
\vedette{ku-bwau}
\région{GOs BO PA}
\end{entrée}

\begin{entrée}
{igname ronde (clone) (Dubois + Charles)}
\vedette{mwacoa}
\région{PA}
\end{entrée}

\begin{entrée}
{igname sauvage}
\vedette{ku-tua}
\région{GOs}
\variante{%
\vedette{kutuwa}
\région{GO(s)}}
\end{entrée}

\begin{entrée}
{igname sauvage (variété de);}
\vedette{dimwã}\homonyme{1}
\région{GOs PA BO}
\end{entrée}

\begin{entrée}
{igname (se ramifie comme le manioc)}
\vedette{ku-manyõ}
\région{GOs}
\end{entrée}

\begin{entrée}
{igname sp. (la plus dure)}
\vedette{hu}\homonyme{1}
\région{GOs BO}
\end{entrée}

\begin{entrée}
{igname (variété)}
\vedette{ku-be}
\région{GOs BO}
\end{entrée}

\begin{entrée}
{igname (violette)}
\vedette{kèlèrè}
\région{GOs}
\end{entrée}

\begin{entrée}
{igname (violette)}
\vedette{taru}
\région{GOs}
\end{entrée}

\begin{entrée}
{igname (violette)}
\vedette{zawe}
\région{GOs}
\end{entrée}

\begin{entrée}
{igname violette (Dubois)}
\vedette{cabwau}
\région{BO}
\end{entrée}

\begin{entrée}
{igname violette (Dubois)}
\vedette{ua}
\région{BO}
\end{entrée}

\begin{entrée}
{igname violette et grosse (Dubois)}
\vedette{kotra}
\région{GOs}
\variante{%
\vedette{kora}
\région{BO}}
\end{entrée}

\begin{entrée}
{igname violette, tendre (Dubois)}
\vedette{jinoji}
\région{BO}
\end{entrée}

\begin{entrée}
{partie supérieure du tubercule d'igname}
\vedette{ńho}
\région{GOs}
\end{entrée}

\begin{entrée}
{peau de l'igname}
\vedette{cii-kui}
\région{GOs}
\end{entrée}

\begin{entrée}
{préfixe des ignames}
\vedette{ku-}\homonyme{1}
\région{BO}
\end{entrée}

\begin{entrée}
{prémices (la première igname récoltée de l'année)}
\vedette{paxa-ka}
\région{GOs}
\end{entrée}

\begin{entrée}
{sillon du massif d'ignames}
\vedette{pwaiòng}
\région{BO PA}
\variante{%
\vedette{pwayòng}
\région{BO PA}}
\end{entrée}

\begin{entrée}
{tête de l'igname (qui est replantée)}
\vedette{bwe-kui}
\région{GOs PA}
\end{entrée}

\subsection{Taros}

\begin{entrée}
{bouture de taro (pédoncule de taro muni d'une tige)}
\vedette{uvo-uva}
\région{GOs}
\end{entrée}

\begin{entrée}
{extrémité inférieure du pétiole de taro d'eau}
\vedette{haa-uva}
\région{GOs PA BO}
\end{entrée}

\begin{entrée}
{extrêmité inférieure du taro}
\vedette{thrô-kuru}
\région{GOs}
\end{entrée}

\begin{entrée}
{feuille de taro d'eau}
\vedette{drò-uva}
\end{entrée}

\begin{entrée}
{jeunes feuilles de taro de montagne}
\vedette{kuvêê-uvhia}
\région{GOs PA}
\end{entrée}

\begin{entrée}
{jeunes pousses; repousse (de taro, bananier, déraciné puis transplanté)}
\vedette{zòò-uva}
\région{GOs PA}
\end{entrée}

\begin{entrée}
{pied de taro}
\vedette{uvo-uva}
\région{GOs}
\end{entrée}

\begin{entrée}
{racines du taro d'eau}
\vedette{wèè-uva}
\région{GOs BO}
\end{entrée}

\begin{entrée}
{taro (clone)}
\vedette{pibwena}
\région{PA}
\end{entrée}

\begin{entrée}
{taro (clone de) de terrain sec (Dubois)}
\vedette{canabwe}
\région{BO}
\end{entrée}

\begin{entrée}
{taro (clone) de terrain sec (Dubois)}
\vedette{kaje}
\région{BO}
\end{entrée}

\begin{entrée}
{taro (clone) de terrain sec (Dubois)}
\vedette{mangane}
\région{BO}
\end{entrée}

\begin{entrée}
{taro (clone) de terrain sec (Dubois)}
\vedette{udang}
\région{BO}
\end{entrée}

\begin{entrée}
{taro (clone) de terrain sec (Dubois)}
\vedette{yomaeo}
\région{BO}
\end{entrée}

\begin{entrée}
{taro d'eau (clone ; Dubois)}
\vedette{jali}\homonyme{2}
\région{BO}
\end{entrée}

\begin{entrée}
{taro d'eau (clone) (Dubois)}
\vedette{dobwa}
\région{BO}
\end{entrée}

\begin{entrée}
{taro d'eau (clone) (Dubois)}
\vedette{waga}
\région{BO}
\end{entrée}

\begin{entrée}
{taro (de montagne, nom du tubercule ou du pied de taro)}
\vedette{uvwia}
\end{entrée}

\begin{entrée}
{tarodière en terrasse et irriguée (de taro d'eau)}
\vedette{vana}
\région{GOs}
\end{entrée}

\begin{entrée}
{taro géant (sauvage, à larges feuilles)}
\vedette{pia}\homonyme{2}
\région{GOs PA}
\end{entrée}

\begin{entrée}
{taro (pied de) d'eau (nom générique)}
\vedette{uva}
\région{GOs BO}
\end{entrée}

\subsection{Bananiers et bananes}

\begin{entrée}
{banane "amérique"}
\vedette{ãmatri}
\région{GOs}
\variante{%
\vedette{ãmari}
\région{GO(s)}}
\end{entrée}

\begin{entrée}
{banane (à peau grise, elle a la forme d'une pirogue, sert à la préparation de "wô" 'bateau')}
\vedette{chèèvwe}
\région{GOs}
\variante{%
\vedette{chèèbwe}}
\variante{%
\vedette{cewe}
\région{BO}}
\end{entrée}

\begin{entrée}
{banane ; bananier de la chefferie (on ne peut que la bouillir, il est interdit de la griller)}
\vedette{mugo}
\région{GOs BO}
\end{entrée}

\begin{entrée}
{banane-chef}
\vedette{phweegi}
\région{PA}
\end{entrée}

\begin{entrée}
{'banane de la chefferie' (Charles, PA)}
\vedette{pweza}
\région{GO PA}
\end{entrée}

\begin{entrée}
{banane (espèce de petite taille qui se mange bien mûre)}
\vedette{minyõ}
\région{GOs}
\end{entrée}

\begin{entrée}
{banane (générique) ; bananier}
\vedette{chaamwa}
\région{GOs PA BO WE}
\end{entrée}

\begin{entrée}
{banane (non comestible, dont la sève rouge foncé est utilisée comme peinture lors des danses)}
\vedette{daal}
\région{PA}
\end{entrée}

\begin{entrée}
{banane sucrée (petite)}
\vedette{chaamwa we-ê}
\région{GOs}
\end{entrée}

\begin{entrée}
{bananier ('banane chef')}
\vedette{podi}
\région{GOs PA BO}
\variante{%
\vedette{pwodi}
\région{BO}}
\end{entrée}

\begin{entrée}
{bananier ('banane chef')}
\vedette{puyai}
\région{PA}
\end{entrée}

\begin{entrée}
{bananier (clone ancien)}
\vedette{pweza}
\région{GO PA}
\end{entrée}

\begin{entrée}
{bananier (clone de)}
\vedette{phwããgo}
\région{GOs BO}
\variante{%
\vedette{pwããgo}
\région{PA BO}}
\end{entrée}

\begin{entrée}
{bananier (clone de) ; variété de banane-chef}
\vedette{cabeng}
\région{PA BO}
\end{entrée}

\begin{entrée}
{bananier (clone ; Dubois)}
\vedette{pwedaou}
\région{BO}
\end{entrée}

\begin{entrée}
{bout de l'inflorescence de bananier}
\vedette{cò-chaamwa}
\région{GOs}
\end{entrée}

\begin{entrée}
{bouture de bananier}
\vedette{zò-chaamwa}
\région{GOs}
\end{entrée}

\begin{entrée}
{coeur du bananier}
\vedette{kumèè chaamwa}
\région{GOsPA}
\end{entrée}

\begin{entrée}
{enveloppe de tronc de bananier}
\vedette{ci-chaamwa}
\région{GOs}
\end{entrée}

\begin{entrée}
{feuille sèche de bananier}
\vedette{havade}
\région{GOs}
\end{entrée}

\begin{entrée}
{goût d'une banane (minyô)}
\vedette{balexa}
\région{GOs BO}
\end{entrée}

\begin{entrée}
{main de banane}
\vedette{dee-chaamwa}
\région{GOs BO}
\end{entrée}

\begin{entrée}
{nervure dorsale de la feuille de bananier}
\vedette{du dròò-chaamwa}
\région{GOs}
\end{entrée}

\begin{entrée}
{pousse (ou) rejet de bananier}
\vedette{zò-chaamwa}
\région{GOs}
\end{entrée}

\begin{entrée}
{prélever les rejets d'un bananier 'pweza' pour les replanter}
\vedette{zaa pweza}
\région{PA}
\end{entrée}

\begin{entrée}
{régime de bananes}
\vedette{thò-chaamwa}
\sens{1}
\région{GOs PA BO}
\end{entrée}

\subsection{Cucurbitacées}

\begin{entrée}
{citrouille ; gourde}
\vedette{kaè}
\région{GOs PA BO}
\end{entrée}

\begin{entrée}
{citrouille (lit. citrouille comestible)}
\vedette{kaè cèni}
\région{GOs}
\end{entrée}

\begin{entrée}
{pastèque}
\vedette{kaè thraxilò}
\région{GOs}
\variante{%
\vedette{kaè kuni}
\région{GO(s)}}
\end{entrée}

\subsection{Fruits}

\begin{entrée}
{banane (générique) ; bananier}
\vedette{chaamwa}
\région{GOs PA BO WE}
\end{entrée}

\begin{entrée}
{citron (lit. fruit citron)}
\vedette{pò sitrô}
\région{GOs}
\end{entrée}

\begin{entrée}
{fruit de pomme rose}
\vedette{pò-caai}
\région{GOsPA}
\end{entrée}

\begin{entrée}
{fruit du "gommier"}
\vedette{pò-zaalo}
\région{GOs}
\end{entrée}

\begin{entrée}
{papaye}
\vedette{pò-ci}
\région{GOs}
\variante{%
\vedette{po-cin}
\région{PA WE}}
\end{entrée}

\begin{entrée}
{pomme canaque}
\vedette{pò-caai}
\région{GOsPA}
\end{entrée}

\section{Classificateurs}

\subsection{Préfixes classificateurs sémantiques}

\begin{entrée}
{champ ; emplacement}
\vedette{kê-}
\région{GOs PA}
\variante{%
\vedette{kêê}
\région{GO}}
\end{entrée}

\begin{entrée}
{morceau (allongé) ; partie ; moitié}
\classe{nom}
\vedette{bala-}\homonyme{1}
\sens{1}
\région{GOs BO PA}
\end{entrée}

\begin{entrée}
{préfixe des anguilles}
\vedette{pee-}
\région{BO}
\end{entrée}

\begin{entrée}
{préfixe des boutures de plante (lianes)}
\vedette{kô-}\homonyme{2}
\région{GOs}
\end{entrée}

\begin{entrée}
{préfixe des ignames}
\vedette{ku-}\homonyme{1}
\région{BO}
\end{entrée}

\begin{entrée}
{préfixe des noms de chenilles}
\vedette{mèra-}
\région{GOs BO}
\end{entrée}

\begin{entrée}
{préfixe des paniers}
\vedette{ke-}
\région{PA}
\variante{%
\vedette{kee}
\région{PA}}
\end{entrée}

\begin{entrée}
{préfixe des touffes de bambous ou de bananiers}
\vedette{gapavwu-}
\région{GOs}
\variante{%
\vedette{gavwu-}
\région{PA}}
\end{entrée}

\subsection{Préfixes classificateurs possessifs}

\begin{entrée}
{arme (de jet ou assimilé)}
\vedette{pai}\homonyme{3}
\région{GOs}
\variante{%
\vedette{phai}
\région{PA BO}}
\end{entrée}

\begin{entrée}
{charge ; mission ; fardeau (métaph. se mêler de)}
\classe{nom}
\vedette{phò}\homonyme{1}
\sens{1}
\région{GOs BO}
\variante{%
\vedette{pho-}
\région{PA}}
\end{entrée}

\begin{entrée}
{part}
\classe{nom}
\vedette{phò}\homonyme{1}
\sens{2}
\région{GOs BO}
\variante{%
\vedette{pho-}
\région{PA}}
\end{entrée}

\subsection{Préfixes classificateurs de la nourriture}

\begin{entrée}
{part de canne à sucre}
\vedette{whala-}
\région{PA}
\variante{%
\vedette{wala-}
\région{PA}}
\variante{%
\vedette{waza}
\région{GO(s)}}
\end{entrée}

\begin{entrée}
{part de nourriture donnée dans les coutumes}
\vedette{va}\homonyme{1}
\région{GOs}
\end{entrée}

\begin{entrée}
{part (sa) de féculents}
\vedette{ca-}
\région{BO [Corne, BM]}
\end{entrée}

\subsection{Préfixes classificateurs possessifs de la nourriture}

\begin{entrée}
{canne à sucre (à manger)}
\vedette{waza}
\région{GOs}
\variante{%
\vedette{wala, whala}
\région{PA WE}}
\end{entrée}

\begin{entrée}
{nourriture carnée part (de poisson et viande),}
\vedette{ho-}
\région{GOs BO}
\variante{%
\vedette{hò}
\région{PA}}
\variante{%
\vedette{hu-}
\région{BO}}
\end{entrée}

\begin{entrée}
{part de canne à sucre}
\vedette{whala-}
\région{PA}
\variante{%
\vedette{wala-}
\région{PA}}
\variante{%
\vedette{waza}
\région{GO(s)}}
\end{entrée}

\begin{entrée}
{part de féculents}
\vedette{cè-}
\région{GOs}
\variante{%
\vedette{caa-}
\région{PA BO}}
\end{entrée}

\begin{entrée}
{part de fruits ou de feuilles}
\vedette{kûû-}
\région{GOs PA BO}
\end{entrée}

\begin{entrée}
{part de nourriture donnée dans les coutumes}
\vedette{va}\homonyme{1}
\région{GOs}
\end{entrée}

\begin{entrée}
{part de nourriture qui se mastique}
\vedette{maa-}
\région{PA}
\end{entrée}

\begin{entrée}
{part (sa) de féculents}
\vedette{ca-}
\région{BO [Corne, BM]}
\end{entrée}

\begin{entrée}
{ration ; part de sucreries; médicaments(PA)}
\vedette{ho-}
\région{GOs BO}
\variante{%
\vedette{hò}
\région{PA}}
\variante{%
\vedette{hu-}
\région{BO}}
\end{entrée}

\subsection{Préfixes classificateurs numériques}

\begin{entrée}
{aliments enveloppés dans des feuilles}
\vedette{tou}
\groupe{B}
\région{GOs PA BO}
\end{entrée}

\begin{entrée}
{année, mois CLF}
\vedette{we-}
\région{GOs PA BO}
\end{entrée}

\begin{entrée}
{billon (d'igname), monticule}
\vedette{bu-}
\région{GOs PA}
\end{entrée}

\begin{entrée}
{bottes d'herbes et paquets de feuilles (pandanus, etc.)}
\vedette{bwa-}
\région{GOs PA}
\end{entrée}

\begin{entrée}
{canne à sucre, bois}
\classe{nom}
\vedette{bala-}\homonyme{1}
\sens{2}
\région{GOs BO PA}
\end{entrée}

\begin{entrée}
{champ (d'ignames, etc.)}
\vedette{nõ-}
\région{GOs}
\variante{%
\vedette{nõ-}
\région{PA}}
\end{entrée}

\begin{entrée}
{chants CLF}
\vedette{we-}
\région{GOs PA BO}
\end{entrée}

\begin{entrée}
{charge ; fardeau}
\vedette{phò-}
\région{GOs}
\end{entrée}

\begin{entrée}
{cinq paquets de 3 ignames}
\vedette{mãè-ni}
\région{GOs}
\end{entrée}

\begin{entrée}
{classificateur des animés}
\vedette{a-}\homonyme{2}
\région{GOs PA}
\variante{%
\vedette{aa-}}
\end{entrée}

\begin{entrée}
{coup ; détonation}
\vedette{phwa-}\homonyme{2}
\région{GOs PA}
\end{entrée}

\begin{entrée}
{deux paires (de roussette ou notous dans les dons coutumiers)}
\vedette{wa-tru}
\région{GOs}
\end{entrée}

\begin{entrée}
{deux paires et une demi-paire (de roussette ou notous dans les dons coutumiers)}
\vedette{wa-tru ko ido}
\région{GO}
\end{entrée}

\begin{entrée}
{dix paires (de roussette ou notous dans les dons coutumiers)}
\vedette{wa-truuji}
\région{GO}
\end{entrée}

\begin{entrée}
{files de voitures [PA]}
\vedette{gu-}
\sens{2}
\région{GOs PA}
\end{entrée}

\begin{entrée}
{filoche de poisson [GOs, PA]}
\vedette{gu-}
\sens{1}
\région{GOs PA}
\end{entrée}

\begin{entrée}
{fruit (1, 2, etc.)}
\vedette{pò}\homonyme{1}
\sens{2}
\région{GOs}
\variante{%
\vedette{pò-n}
\région{PA BO}}
\variante{%
\vedette{pwò}
\région{BO}}
\end{entrée}

\begin{entrée}
{grappes de noix de coco}
\vedette{ula}\homonyme{2}
\région{GOs PA}
\end{entrée}

\begin{entrée}
{grappes (par ex. de tomates, orange)}
\vedette{tibö}
\région{PA}
\variante{%
\vedette{tibu}
\région{PA}}
\end{entrée}

\begin{entrée}
{lot de 2 roussettes ou notous (lots cérémoniels)}
\vedette{wa-}
\région{GOs PA BO}
\variante{%
\vedette{wan-}
\région{BO PA}}
\end{entrée}

\begin{entrée}
{lot de 3 ignames}
\vedette{mãè-}
\région{GOs}
\variante{%
\vedette{mãi-}
\région{PA BO}}
\end{entrée}

\begin{entrée}
{lot de 4 taros ou de 4 noix de coco}
\vedette{mãè-}
\région{GOs}
\variante{%
\vedette{mãi-}
\région{PA BO}}
\end{entrée}

\begin{entrée}
{main de bananes}
\vedette{de-}\homonyme{2}
\région{GOs PA}
\end{entrée}

\begin{entrée}
{morceau (objet long)}
\vedette{gòò}
\groupe{B}
\région{GOs PA BO}
\end{entrée}

\begin{entrée}
{morceau ; part}
\vedette{mhãi-}
\groupe{B}
\région{GOs PA BO}
\end{entrée}

\begin{entrée}
{n-fois}
\vedette{õ-}
\région{GOs PA BO}
\variante{%
\vedette{on}
\région{PA BO}}
\end{entrée}

\begin{entrée}
{objets longs CLF (voiture, bateau, arbre couché, poteau)}
\vedette{we-}
\région{GOs PA BO}
\end{entrée}

\begin{entrée}
{objets ronds (fruits, heure, etc.)}
\vedette{po-}
\région{GOs PA}
\end{entrée}

\begin{entrée}
{pieds d'arbre, feuilles, tubercules, racines}
\vedette{pu-}
\région{GOs PA}
\end{entrée}

\begin{entrée}
{quart ou moitié de tortue ou de bœoeuf}
\vedette{alabo-}
\région{PA}
\end{entrée}

\begin{entrée}
{rangée (d'ignames, poteaux, etc.)}
\vedette{îdò-}
\sens{2}
\région{GOs PA BO}
\end{entrée}

\begin{entrée}
{régimede bananes}
\vedette{thò-chaamwa}
\sens{2}
\région{GOs PA BO}
\end{entrée}

\begin{entrée}
{régimes de banane}
\vedette{thò-}
\région{GOs PA BO}
\end{entrée}

\begin{entrée}
{tas de}
\vedette{bwalò}
\groupe{B}
\sens{2}
\région{GOs PA BO}
\variante{%
\vedette{pwalò}
\région{GO(s)}}
\end{entrée}

\begin{entrée}
{tas (distribué dans des cérémonies coutumières)}
\vedette{tou}
\groupe{B}
\région{GOs PA BO}
\end{entrée}

\begin{entrée}
{tissus et étoffes végétales}
\vedette{haa}\homonyme{2}
\région{GOs PA}
\end{entrée}

\begin{entrée}
{trous et tas de prestations cérémonielles}
\vedette{phwa-}\homonyme{1}
\groupe{B}
\région{PA}
\end{entrée}

\begin{entrée}
{un (bâtiment, arbre, qqch qui a une souche et qui est haut)}
\vedette{pu-xè}
\région{PA BO}
\end{entrée}

\begin{entrée}
{une branche (ne s'emploie plus)}
\vedette{hi-xe}\homonyme{3}
\région{GOs PA}
\end{entrée}

\begin{entrée}
{une (feuille)}
\vedette{dròò-xè}
\région{GO}
\variante{%
\vedette{dòò-}
\région{PA}}
\end{entrée}

\begin{entrée}
{une paire (de roussettes ou notous dans les dons coutumiers)}
\vedette{wa-xè}
\région{GOs PA}
\end{entrée}

\begin{entrée}
{un fagot ; un paquet de feuilles, de paille}
\vedette{bwaa-xe}
\région{PA}
\end{entrée}

\begin{entrée}
{un (morceau de bois)}
\vedette{bala-xè}
\région{GOs PA}
\end{entrée}

\begin{entrée}
{un morceau (pastèque, fruit, igname, etc.)}
\vedette{hô-xe}
\région{GOsPA}
\end{entrée}

\begin{entrée}
{un (objet long)}
\vedette{wè-xèè}
\région{GOs PA BO}
\end{entrée}

\begin{entrée}
{un paquet (de trois ignames)}
\vedette{mãè-xè}
\région{GOsBO}
\end{entrée}

\section{Numération}

\subsection{Numéraux cardinaux}

\begin{entrée}
{cent (lit. cinq hommes)}
\vedette{ani êgu}
\région{GOs}
\variante{%
\vedette{anim êgu}
\région{PA}}
\end{entrée}

\begin{entrée}
{cinq}
\vedette{-ni}\homonyme{1}
\région{GOs}
\variante{%
\vedette{-nim}
\région{PA BO}}
\end{entrée}

\begin{entrée}
{cinq}
\vedette{pò-ni}
\région{GO}
\end{entrée}

\begin{entrée}
{cinq (objets longs)}
\vedette{wè-ni}
\région{GOs}
\end{entrée}

\begin{entrée}
{cinquante (lit. deux hommes et 10)}
\vedette{a-tru êgu bwa truuçi}
\région{GOs}
\end{entrée}

\begin{entrée}
{deux}
\vedette{-tru}
\région{GOs}
\variante{%
\vedette{-ru}
\région{PA BO}}
\variante{%
\vedette{-lu}}
\end{entrée}

\begin{entrée}
{deux (générique)}
\vedette{pò-tru}
\région{GOs}
\variante{%
\vedette{pò-ru}
\région{PA}}
\variante{%
\vedette{pò-du, pò-ru}
\région{BO}}
\end{entrée}

\begin{entrée}
{dix}
\vedette{truuçi}
\région{GOs}
\variante{%
\vedette{tuuyi}
\région{PA}}
\variante{%
\vedette{truji}
\région{BO}}
\end{entrée}

\begin{entrée}
{douze (10 et 2)}
\vedette{truuçi bwa pò-tru}
\région{GO}
\variante{%
\vedette{tuuyi bwa pòru}
\région{PA}}
\end{entrée}

\begin{entrée}
{huit}
\vedette{pò-ni-ma-gò}
\région{GOs}
\end{entrée}

\begin{entrée}
{huit (=5 et 3)}
\vedette{-ni-ma-gò}
\région{GOs}
\variante{%
\vedette{ni-ma-kòn}
\région{PA BO}}
\end{entrée}

\begin{entrée}
{il y a deux variétés}
\vedette{pe-pwò-tru meewu}
\région{GOs}
\end{entrée}

\begin{entrée}
{neuf}
\vedette{pò-ni-ma-ba}
\région{GOs}
\end{entrée}

\begin{entrée}
{neuf (=5 et 3)}
\vedette{-ni-ma-ba}
\région{GOs PA BO}
\end{entrée}

\begin{entrée}
{onze (10 et 1)}
\vedette{truuçi bwa pòxe}
\région{GO}
\variante{%
\vedette{tuuyi bwa pòxè}
\région{PA}}
\end{entrée}

\begin{entrée}
{quarante (lit. deux hommes)}
\vedette{a-tru êgu}
\région{GOs}
\variante{%
\vedette{aaru êgu}
\région{BO}}
\end{entrée}

\begin{entrée}
{quatre}
\vedette{-pa}
\région{GOs BO PA}
\end{entrée}

\begin{entrée}
{quatre}
\vedette{pò-pa}
\région{GO}
\end{entrée}

\begin{entrée}
{quatre-vingt dix (lit. quatre hommes et 10)}
\vedette{apa êgu bwa truuçi}
\région{GOs}
\end{entrée}

\begin{entrée}
{quatre-vingt (lit. quatre hommes)}
\vedette{apa êgu}
\région{GOs}
\variante{%
\vedette{aapa}
\région{BO}}
\end{entrée}

\begin{entrée}
{sept}
\vedette{pò-ni-ma-dru}
\région{GOs}
\end{entrée}

\begin{entrée}
{sept (=5 et 2)}
\vedette{-ni-ma-dru}
\région{GOs}
\variante{%
\vedette{-ni ma-du}
\région{BO PA}}
\end{entrée}

\begin{entrée}
{sept (choses longues)}
\vedette{wè-ni ma du}
\région{BO}
\end{entrée}

\begin{entrée}
{six}
\vedette{a-ni-ma-xe}
\région{GOs}
\variante{%
\vedette{a-nim-a-xe}
\région{BO PA}}
\end{entrée}

\begin{entrée}
{six}
\vedette{pò-ni-ma-xe}
\région{GOs}
\end{entrée}

\begin{entrée}
{six (=5 et 1)}
\vedette{-ni-ma-xè}
\région{GOs PA}
\variante{%
\vedette{nim-a-xe}
\région{BO}}
\end{entrée}

\begin{entrée}
{soixante-dix (lit. trois hommes et 10)}
\vedette{a-kò êgu bwa truuçi}
\région{GOs}
\end{entrée}

\begin{entrée}
{soixante (lit. trois hommes)}
\vedette{a-kò êgu}
\région{GOs}
\end{entrée}

\begin{entrée}
{trente (lit. un homme et 10)}
\vedette{axè êgu bwa truuçi}
\région{GO}
\variante{%
\vedette{aaxè ãgu bwa truji}
\région{BO}}
\end{entrée}

\begin{entrée}
{trois}
\vedette{-kò}
\région{GOs}
\variante{%
\vedette{-kòn}
\région{PA BO}}
\end{entrée}

\begin{entrée}
{trois}
\vedette{pò-ko}
\région{GOs}
\end{entrée}

\begin{entrée}
{trois (animés)}
\vedette{a-kò}
\région{GOs}
\variante{%
\vedette{a-kòn}
\région{BO PA}}
\end{entrée}

\begin{entrée}
{un}
\vedette{-xè}
\variante{%
\vedette{po-xe}
\région{GO PA BO}}
\end{entrée}

\begin{entrée}
{un (animé : homme ou animal)}
\vedette{a-xè}\homonyme{1}
\région{GOs PA BO}
\région{GOs}
\variante{%
\vedette{a-kè}}
\end{entrée}

\begin{entrée}
{un autre ; un nouveau}
\vedette{õ-xe}
\région{GOs BO PA}
\variante{%
\vedette{hokè, hogè}
\région{BO}}
\end{entrée}

\begin{entrée}
{une fois}
\vedette{õ-xe}
\région{GOs BO PA}
\variante{%
\vedette{hokè, hogè}
\région{BO}}
\end{entrée}

\begin{entrée}
{une seule fois}
\vedette{õ-xe-nò}
\région{GOs}
\end{entrée}

\begin{entrée}
{une seule fois (Dubois)}
\vedette{õ-xe-on}
\région{BO PA}
\end{entrée}

\begin{entrée}
{un (objet rond) ; un jour}
\vedette{pò-xè}
\région{GO PA}
\end{entrée}

\begin{entrée}
{vingt (lit. un homme)}
\vedette{axè êgu}
\région{GOs PA}
\variante{%
\vedette{aaxe ãgu}
\région{BO}}
\variante{%
\vedette{aaxe êgu}
\région{PA}}
\end{entrée}

\begin{entrée}
{vingt personnes}
\vedette{axè êguu}
\région{GOs PA}
\end{entrée}

\subsection{Numéraux ordinaux}

\begin{entrée}
{dernier (le ) ; fin}
\vedette{ba-ogine}
\région{GOs BO}
\variante{%
\vedette{ba-ogin-en}
\région{PA}}
\end{entrée}

\begin{entrée}
{deuxième}
\vedette{ba-atru}
\région{GOs}
\end{entrée}

\begin{entrée}
{n-ième}
\vedette{ba- ... (le)}
\région{GOs PA BO}
\variante{%
\vedette{na-}
\région{BO vx}}
\end{entrée}

\begin{entrée}
{première fois}
\vedette{ba-õxe}
\région{GOs}
\end{entrée}

\section{Quantificateurs et marques de degré}

\subsection{Quantificateurs}

\begin{entrée}
{assemblage , réunion}
\vedette{bulu}
\groupe{A}
\région{GOs PA BO}
\end{entrée}

\begin{entrée}
{autre (un, d') ; un autre}
\vedette{ãbaa}\homonyme{1}
\région{GOs}
\variante{%
\vedette{ãbaa-n}
\région{BO PA}}
\end{entrée}

\begin{entrée}
{bande (oiseaux, enfants) [PA, BO]}
\vedette{bulu}
\groupe{A}
\région{GOs PA BO}
\end{entrée}

\begin{entrée}
{beaucoup}
\vedette{phavwu}
\région{GOs}
\variante{%
\vedette{phavwuun}
\région{PA}}
\end{entrée}

\begin{entrée}
{beaucoup}
\vedette{phavwuun}
\région{PA BO [BM]}
\end{entrée}

\begin{entrée}
{beaucoup ; nombreux (lit. 3 fois) (Dubois)}
\vedette{pe-gan}
\région{BO}
\end{entrée}

\begin{entrée}
{beaucoup ; nombreux ; trop}
\vedette{haivwö}
\région{GOs BO PA}
\variante{%
\vedette{hai}
\région{BO}}
\variante{%
\vedette{haipo}
\région{vx}}
\end{entrée}

\begin{entrée}
{ce qui s'ajoute à un tas (de dons coutumiers, mais ne peut constituer un tas complet)}
\vedette{mhõdòni}
\région{GOs}
\end{entrée}

\begin{entrée}
{certains ; quelques}
\vedette{ãbaa}\homonyme{1}
\région{GOs}
\variante{%
\vedette{ãbaa-n}
\région{BO PA}}
\end{entrée}

\begin{entrée}
{chacun}
\vedette{phò}\homonyme{2}
\région{GOs}
\end{entrée}

\begin{entrée}
{chacun(e)}
\vedette{kha-axe}
\région{GOs}
\end{entrée}

\begin{entrée}
{complètement ; totalement ; tout ; ensemble}
\vedette{jiu}
\région{GOs}
\variante{%
\vedette{jiu-n}
\région{PA BO [BM]}}
\end{entrée}

\begin{entrée}
{en complément ; en plus}
\vedette{ãbaa-xa}
\région{GOs}
\end{entrée}

\begin{entrée}
{ensemble}
\vedette{bulu}
\groupe{B}
\région{GOs PA BO}
\end{entrée}

\begin{entrée}
{ensemble}
\vedette{cavwe}
\région{GOs}
\end{entrée}

\begin{entrée}
{ensemble}
\classe{COLL ; QNT}
\vedette{cocovwa}
\sens{1}
\région{GOs}
\variante{%
\vedette{cocopa}
\région{GO(s)}}
\variante{%
\vedette{cocova}
\région{BO}}
\end{entrée}

\begin{entrée}
{ensemble}
\vedette{evhe}
\région{PA BO}
\variante{%
\vedette{epe}}
\end{entrée}

\begin{entrée}
{ensemble (dans les interpellations)}
\vedette{baa}\homonyme{1}
\région{GOs PA BO}
\end{entrée}

\begin{entrée}
{entier}
\vedette{kõmwõgi}
\région{GOs}
\variante{%
\vedette{kõmõgin}
\région{BO}}
\end{entrée}

\begin{entrée}
{entier ; rond(BO, Dubois)}
\vedette{kõmwõgi}
\région{GOs}
\variante{%
\vedette{kõmõgin}
\région{BO}}
\end{entrée}

\begin{entrée}
{groupe de personnes (nombreuses) ; foule}
\vedette{phavwu}
\région{GOs}
\variante{%
\vedette{phavwuun}
\région{PA}}
\end{entrée}

\begin{entrée}
{groupe [PA, BO]}
\vedette{bulu}
\groupe{A}
\région{GOs PA BO}
\end{entrée}

\begin{entrée}
{il y a beaucoup ; très (lié à "haivwo")}
\vedette{hai}\homonyme{3}
\région{GOs BO PA}
\end{entrée}

\begin{entrée}
{manque (il) ; rester}
\classe{v.IMPERS}
\vedette{mwêêno}
\sens{1}
\région{GOs BO}
\end{entrée}

\begin{entrée}
{manquer}
\classe{v}
\vedette{paxu}
\sens{2}
\région{GOs PA}
\end{entrée}

\begin{entrée}
{moins}
\vedette{po-hoxè}
\région{GOs}
\end{entrée}

\begin{entrée}
{morceau}
\vedette{mhãi-}
\région{PA}
\end{entrée}

\begin{entrée}
{morceau}
\vedette{mhavwa}
\région{GOs}
\variante{%
\vedette{mhava}
\région{PA BO}}
\end{entrée}

\begin{entrée}
{morceau ; bout de qqch}
\vedette{mhava}
\région{GOs PA}
\variante{%
\vedette{mhava-n}
\région{PA BO}}
\end{entrée}

\begin{entrée}
{morceau (de viande, igname coupée) ; part ; fraction}
\vedette{mhãi-}
\groupe{A}
\région{GOs PA BO}
\end{entrée}

\begin{entrée}
{morceau long et plat (réfère à une surface allongée)}
\vedette{hõõ}
\région{PA}
\end{entrée}

\begin{entrée}
{morceau ; part ; fragment}
\vedette{hõxa}
\région{GOs}
\variante{%
\vedette{hoxa}
\région{PA}}
\end{entrée}

\begin{entrée}
{morceau ; partie}
\classe{nom}
\vedette{gòò}
\groupe{A}
\sens{1}
\région{GOs PA BO}
\end{entrée}

\begin{entrée}
{n-fois}
\vedette{õ-}
\région{GOs PA BO}
\variante{%
\vedette{on}
\région{PA BO}}
\end{entrée}

\begin{entrée}
{paire ; l'autre d'une paire}
\vedette{thilò}
\région{GOs}
\end{entrée}

\begin{entrée}
{parmi ; entre}
\vedette{dõni}
\région{GOs PA BO}
\end{entrée}

\begin{entrée}
{petit ; mince ; un peu}
\vedette{pwònèn}
\région{BO [Corne]}
\variante{%
\vedette{ponèn}
\région{PA}}
\end{entrée}

\begin{entrée}
{peu ; quelques ; quelques}
\vedette{hoxèè}
\région{GOs BO}
\end{entrée}

\begin{entrée}
{peu (un)}
\vedette{pò}\homonyme{3}
\région{GOs PA BO}
\variante{%
\vedette{pwò}
\région{BO}}
\end{entrée}

\begin{entrée}
{peu ; un peu (quantité)}
\vedette{pòńõ}
\région{GOs}
\end{entrée}

\begin{entrée}
{plus ; beaucoup}
\vedette{mhãã}\homonyme{3}
\région{GOs PA BO}
\end{entrée}

\begin{entrée}
{plusieurs (à)}
\vedette{pe-ka-}
\région{GOs}
\end{entrée}

\begin{entrée}
{plusieurs fois}
\vedette{õ-pengan}
\région{PA WE}
\variante{%
\vedette{haivwo}
\région{GO(s)}}
\end{entrée}

\begin{entrée}
{quelques ; plusieurs}
\vedette{jivwa}
\région{GOs PA BO}
\variante{%
\vedette{jipwa}
\région{GO(s)}}
\end{entrée}

\begin{entrée}
{reste (le) ; restant ; surplus}
\vedette{kôgòò}
\région{GOs}
\variante{%
\vedette{kôgò-n}
\région{BO PA}}
\variante{%
\vedette{kugo}
\région{BO PA}}
\end{entrée}

\begin{entrée}
{tas [PA, BO]}
\vedette{bulu}
\groupe{A}
\région{GOs PA BO}
\end{entrée}

\begin{entrée}
{tous}
\vedette{chińõ}
\groupe{C}
\région{GOs PA}
\variante{%
\vedette{cinõ}
\région{BO}}
\end{entrée}

\begin{entrée}
{tous}
\vedette{pevwe}
\région{GOs}
\variante{%
\vedette{pevhe}
\région{PA BO}}
\variante{%
\vedette{pepe}
\région{vx}}
\end{entrée}

\begin{entrée}
{tous}
\vedette{phò}\homonyme{2}
\région{GOs}
\end{entrée}

\begin{entrée}
{tous ; chaque}
\classe{COLL ; QNT}
\vedette{cocovwa}
\sens{1}
\région{GOs}
\variante{%
\vedette{cocopa}
\région{GO(s)}}
\variante{%
\vedette{cocova}
\région{BO}}
\end{entrée}

\begin{entrée}
{tous ; tout le monde ; totalité}
\vedette{jivwa}
\région{GOs PA BO}
\variante{%
\vedette{jipwa}
\région{GO(s)}}
\end{entrée}

\begin{entrée}
{toutes les choses possédées ensemble}
\vedette{pe-pwaixe}
\région{GOs}
\end{entrée}

\begin{entrée}
{toutes sortes de}
\vedette{jivwa meewu}
\région{PA}
\end{entrée}

\begin{entrée}
{tout ; l'ensemble}
\vedette{peve jun}
\région{PA BO}
\end{entrée}

\begin{entrée}
{tout ; tous}
\vedette{õ}\homonyme{3}
\région{GOs}
\variante{%
\vedette{ô}
\région{BO PA}}
\end{entrée}

\begin{entrée}
{tout ; toutes sortes de}
\classe{COLL ; QNT}
\vedette{cocovwa}
\sens{2}
\région{GOs}
\variante{%
\vedette{cocopa}
\région{GO(s)}}
\variante{%
\vedette{cocova}
\région{BO}}
\end{entrée}

\begin{entrée}
{très ; trop}
\vedette{mhãã}\homonyme{3}
\région{GOs PA BO}
\end{entrée}

\begin{entrée}
{trop (en) ; en surplus}
\vedette{kuzaò}
\région{GOs}
\variante{%
\vedette{kuraò}
\région{WEM WEH}}
\end{entrée}

\begin{entrée}
{un bout de}
\vedette{ãbaa}\homonyme{1}
\région{GOs}
\variante{%
\vedette{ãbaa-n}
\région{BO PA}}
\end{entrée}

\begin{entrée}
{un d'une paire ; un seul (d'une paire)}
\vedette{thixèè}
\région{GOs}
\variante{%
\vedette{thaxee}
\région{PA}}
\end{entrée}

\begin{entrée}
{unique ; seul}
\vedette{tee-axe}
\région{PA}
\end{entrée}

\begin{entrée}
{un peu}
\classe{v.stat.}
\vedette{pobe}
\sens{2}
\région{PA BO}
\variante{%
\vedette{pwobe}
\région{BO}}
\end{entrée}

\begin{entrée}
{un peu}
\classe{v.stat.}
\vedette{popobe}
\sens{2}
\région{PA BO [Dubois]}
\variante{%
\vedette{pobe}
\région{WE}}
\end{entrée}

\begin{entrée}
{un peu ; un instant}
\vedette{khõ}
\région{GOs}
\variante{%
\vedette{khò, kò-}
\région{PA}}
\end{entrée}

\begin{entrée}
{un seul(ement)}
\vedette{pò-xè ńõ}
\région{GOs}
\end{entrée}

\subsection{Marques de degré}

\begin{entrée}
{plus encore}
\classe{v ; QNT}
\vedette{biça}
\sens{2}
\région{WEM}
\end{entrée}

\begin{entrée}
{très}
\vedette{para}\homonyme{2}
\région{GOs}
\end{entrée}

\begin{entrée}
{très (+ animés)}
\vedette{pa-}
\région{GOs PA}
\end{entrée}

\begin{entrée}
{très ; vraiment}
\vedette{cii}\homonyme{2}
\région{GOs PA BO}
\variante{%
\vedette{cee}
\région{PA}}
\end{entrée}

\begin{entrée}
{trop}
\classe{v ; QNT}
\vedette{biça}
\sens{2}
\région{WEM}
\end{entrée}

\begin{entrée}
{trop (Corne)}
\vedette{cu}
\région{BO}
\variante{%
\vedette{cuu}}
\end{entrée}

\begin{entrée}
{trop (en) ; en surplus}
\vedette{kuzaò}
\région{GOs}
\variante{%
\vedette{kuraò}
\région{WEM WEH}}
\end{entrée}

\subsection{Distributifs}

\begin{entrée}
{à combien dans chaque ?}
\vedette{pe-ka-poniza ?}
\région{GOs}
\end{entrée}

\begin{entrée}
{chaque ; chacun (+ numéral)}
\vedette{kha-}
\région{GOs PA BO}
\end{entrée}

\begin{entrée}
{séparément ; chacun}
\vedette{vara}
\région{GOs PA}
\end{entrée}

\begin{entrée}
{un par un}
\vedette{ka-poxe}
\région{GOs}
\end{entrée}

\begin{entrée}
{un par un (mettre)}
\vedette{pe-ka-poxe}
\région{GOs}
\end{entrée}

\subsection{Dispersifs}

\begin{entrée}
{à chacun ; à part}
\classe{v ; n}
\vedette{peale}
\sens{1}
\région{PA BO}
\end{entrée}

\begin{entrée}
{différent ; distinct}
\classe{v ; n}
\vedette{peale}
\sens{2}
\région{PA BO}
\end{entrée}

\begin{entrée}
{différent l'un de l'autre}
\vedette{pe-haze}
\région{GOs}
\variante{%
\vedette{pe-aze}
\région{GO(s)}}
\variante{%
\vedette{pe-hale, pe-ale}
\région{BO [BM]}}
\variante{%
\vedette{ve-ale}
\région{BO}}
\end{entrée}

\begin{entrée}
{dispersif ; sans but ; comme ça (ou activité non bornée)}
\vedette{pe-}\homonyme{2}
\sens{3}
\région{GOs PA}
\end{entrée}

\begin{entrée}
{séparément ; chacun de son côté}
\vedette{pe-haze}
\région{GOs}
\variante{%
\vedette{pe-aze}
\région{GO(s)}}
\variante{%
\vedette{pe-hale, pe-ale}
\région{BO [BM]}}
\variante{%
\vedette{ve-ale}
\région{BO}}
\end{entrée}

\section{Eléments grammaticaux}

\subsection{Adverbe}

\begin{entrée}
{adverbe péjoratif}
\vedette{òri}\homonyme{2}
\région{GOs}
\end{entrée}

\begin{entrée}
{soudain}
\vedette{jo}\homonyme{1}
\région{GOs}
\end{entrée}

\subsection{Agent}

\begin{entrée}
{agent}
\vedette{egu}
\région{GO}
\variante{%
\vedette{eku}
\région{GO}}
\end{entrée}

\begin{entrée}
{agent}
\vedette{hu}\homonyme{2}
\région{BO (BM)}
\end{entrée}

\begin{entrée}
{agent}
\vedette{ko}\homonyme{3}
\région{PA GO}
\variante{%
\vedette{xo, o, ku, u}
\région{PA}}
\end{entrée}

\begin{entrée}
{agent}
\vedette{ku}\homonyme{1}
\région{BOGO}
\variante{%
\vedette{ko}
\région{BO}}
\end{entrée}

\begin{entrée}
{agent}
\vedette{u}\homonyme{4}
\région{PA BO}
\variante{%
\vedette{ku, xu, xo}}
\end{entrée}

\begin{entrée}
{sujet (marque de sujet des verbes actifs)}
\vedette{xo}\homonyme{1}
\région{GOs PA}
\variante{%
\vedette{ko, go}
\région{GO(s)}}
\variante{%
\vedette{vwo, o, u}
\région{PA BO}}
\end{entrée}

\subsection{Articles}

\subsection{Aspect}

\begin{entrée}
{accompli}
\vedette{u}\homonyme{1}
\région{GOs PA}
\end{entrée}

\begin{entrée}
{alors ; continuer à}
\vedette{mwã}\homonyme{1}
\région{GOs}
\end{entrée}

\begin{entrée}
{chaque fois que}
\vedette{õn na}
\région{PA}
\end{entrée}

\begin{entrée}
{commencer à ; se mettre à ; être sur le point de}
\classe{INCH}
\vedette{mhaza}
\sens{1}
\région{GOs}
\variante{%
\vedette{maza}}
\variante{%
\vedette{mhara}
\région{PA WEM}}
\end{entrée}

\begin{entrée}
{commencer [Corne]}
\vedette{mèè}
\région{BO}
\end{entrée}

\begin{entrée}
{commencer ; mettre à (se) ; créer}
\vedette{thaavwu}
\région{GOs}
\variante{%
\vedette{thaapu}
\région{GO(s)}}
\variante{%
\vedette{thaavwun, thaapun, taapun}
\région{PA BO}}
\variante{%
\vedette{taavwu(n)}
\région{BO}}
\variante{%
\vedette{teewu}
\région{WEM}}
\end{entrée}

\begin{entrée}
{continuatif ; sans interruption ; duratif}
\vedette{gò}\homonyme{2}
\région{GOs}
\variante{%
\vedette{gòl}
\région{PA}}
\end{entrée}

\begin{entrée}
{continuer ; perpétuer}
\vedette{taagine}
\sens{1}
\région{GOs}
\variante{%
\vedette{taagin}
\région{PA BO}}
\variante{%
\vedette{tagin}
\région{BO}}
\variante{%
\vedette{taagi}
\région{WEM}}
\end{entrée}

\begin{entrée}
{déjà}
\vedette{ògi}
\sens{2}
\région{GOs}
\variante{%
\vedette{ògin}
\région{BO PA WEM}}
\end{entrée}

\begin{entrée}
{écouter un instant, un peu}
\vedette{kò-phaxeen}
\région{PA}
\end{entrée}

\begin{entrée}
{encore ; à nouveau}
\vedette{hoxe}
\région{GOs PA}
\end{entrée}

\begin{entrée}
{encore ; à nouveau [BO]}
\classe{ASP}
\vedette{hõ}\homonyme{2}
\groupe{B}
\sens{1}
\région{GOs}
\variante{%
\vedette{hô}
\région{PA}}
\variante{%
\vedette{hò}
\région{BO}}
\end{entrée}

\begin{entrée}
{encore ; de nouveau}
\vedette{õxè}
\région{GOs}
\variante{%
\vedette{oxa}
\région{BO [BM]}}
\end{entrée}

\begin{entrée}
{encore ; de nouveau}
\vedette{xa}\homonyme{5}
\région{GOs}
\end{entrée}

\begin{entrée}
{encore en train de ; toujours en train de}
\vedette{gaa}\homonyme{1}
\région{GOs PA BO}
\variante{%
\vedette{ga}
\région{BO}}
\end{entrée}

\begin{entrée}
{encore jamais ; jamais}
\vedette{kavwö ... nee ... taagin}
\région{PA BO [Corne]}
\end{entrée}

\begin{entrée}
{encore jamais ; jamais}
\vedette{kavwö ... ne... gò}
\région{GOs}
\end{entrée}

\begin{entrée}
{enfin}
\vedette{mwã}\homonyme{2}
\sens{3}
\région{GO PA}
\end{entrée}

\begin{entrée}
{enfin}
\classe{INCH}
\vedette{mhaza}
\sens{6}
\région{GOs}
\variante{%
\vedette{maza}}
\variante{%
\vedette{mhara}
\région{PA WEM}}
\end{entrée}

\begin{entrée}
{en même temps}
\vedette{thöö}\homonyme{2}
\région{PA}
\variante{%
\vedette{tuu}
\région{BO}}
\end{entrée}

\begin{entrée}
{en train de}
\vedette{ge ... mhenõõ}
\région{GOs PA}
\end{entrée}

\begin{entrée}
{en train de}
\classe{nom}
\vedette{mhenõõ}\homonyme{1}
\sens{2}
\région{GOs BO PA}
\variante{%
\vedette{mènõ}
\région{BO}}
\end{entrée}

\begin{entrée}
{en train de [BO]}
\classe{INCH}
\vedette{mhaza}
\sens{5}
\région{GOs}
\variante{%
\vedette{maza}}
\variante{%
\vedette{mhara}
\région{PA WEM}}
\end{entrée}

\begin{entrée}
{en train de manger}
\vedette{pe-kû}
\région{BO}
\end{entrée}

\begin{entrée}
{en train de (marque la durée)}
\vedette{pe-}\homonyme{1}
\région{BO}
\end{entrée}

\begin{entrée}
{en train de planter}
\vedette{pe-tòe}
\région{BO}
\end{entrée}

\begin{entrée}
{en train de regarder}
\vedette{pe-alo}
\région{BO}
\end{entrée}

\begin{entrée}
{en train de se promener}
\vedette{pe-piina}
\région{GOs BO}
\end{entrée}

\begin{entrée}
{être en train de}
\classe{v.LOC ; progressif}
\vedette{ge}\homonyme{1}
\sens{2}
\région{GOs BO PA}
\end{entrée}

\begin{entrée}
{être la première fois que}
\classe{INCH}
\vedette{mhaza}
\sens{2}
\région{GOs}
\variante{%
\vedette{maza}}
\variante{%
\vedette{mhara}
\région{PA WEM}}
\end{entrée}

\begin{entrée}
{faire en même temps}
\vedette{ne-ra}
\région{WEM}
\end{entrée}

\begin{entrée}
{faire le premier ; faire d'abord}
\vedette{gaa ... gòl}
\région{PA}
\end{entrée}

\begin{entrée}
{finalement ; en guise de fin ; fin}
\vedette{ba-kûûni}
\région{GOs}
\variante{%
\vedette{ba-kuuni}
\région{BO}}
\end{entrée}

\begin{entrée}
{finir ; terminer}
\vedette{kûûni}\homonyme{2}
\région{GOs BO PA}
\end{entrée}

\begin{entrée}
{finir ; terminer ; être prêt}
\vedette{ògi}
\sens{1}
\région{GOs}
\variante{%
\vedette{ògin}
\région{BO PA WEM}}
\end{entrée}

\begin{entrée}
{habitude}
\vedette{ku}\homonyme{6}
\région{GOs}
\end{entrée}

\begin{entrée}
{juste au moment où}
\classe{INCH}
\vedette{mhaza}
\sens{4}
\région{GOs}
\variante{%
\vedette{maza}}
\variante{%
\vedette{mhara}
\région{PA WEM}}
\end{entrée}

\begin{entrée}
{n'avoir jamais assez de}
\vedette{kixa khôôme}
\région{GOs}
\end{entrée}

\begin{entrée}
{ne faire que ; n'avoir de cesse que}
\vedette{kixa khôôme}
\région{GOs}
\end{entrée}

\begin{entrée}
{ne ... plus}
\vedette{kavwö ... mwã}
\région{GOs}
\end{entrée}

\begin{entrée}
{ne plus (avec négation)}
\vedette{haxa}
\région{GOs}
\end{entrée}

\begin{entrée}
{pas encore}
\vedette{kavwö ... gò}
\région{GOs}
\end{entrée}

\begin{entrée}
{pas encore}
\vedette{penõõ}
\région{PA}
\end{entrée}

\begin{entrée}
{pour commencer}
\vedette{ba-thaavwu}
\région{GOs}
\variante{%
\vedette{ba-thaavwun}
\région{PA}}
\end{entrée}

\begin{entrée}
{presque}
\classe{LOC}
\vedette{mõnu}
\sens{2}
\région{GOsPA}
\variante{%
\vedette{mõnu}
\région{PA BO}}
\variante{%
\vedette{mwonu}
\région{BO}}
\end{entrée}

\begin{entrée}
{puis ; et puis ; ensuite}
\vedette{ògi}
\sens{3}
\région{GOs}
\variante{%
\vedette{ògin}
\région{BO PA WEM}}
\end{entrée}

\begin{entrée}
{re-}
\vedette{za xa}
\région{GOs}
\variante{%
\vedette{xa}
\région{GO(s)}}
\end{entrée}

\begin{entrée}
{re- ; à nouveau}
\vedette{mwã}\homonyme{2}
\sens{2}
\région{GO PA}
\end{entrée}

\begin{entrée}
{rester à ; ne faire que}
\vedette{gu}\homonyme{6}
\région{GOs}
\variante{%
\vedette{ku}
\région{PA}}
\end{entrée}

\begin{entrée}
{sans arrêt ; sans cesse ; tout le temps}
\vedette{haa}\homonyme{1}
\région{GOs PA BO}
\end{entrée}

\begin{entrée}
{sans cesse ; constamment ; toujours ; à répétition ; sans arrêt ; tout le temps}
\vedette{chãnã}\homonyme{2}
\région{GOs}
\end{entrée}

\begin{entrée}
{sans cesse ; sans arrêt}
\vedette{ni mhenõ}\homonyme{2}
\région{GO}
\end{entrée}

\begin{entrée}
{souvent (lié à ne2) ; (+ négation : jamais)}
\vedette{ne}\homonyme{3}
\région{GOs PA}
\end{entrée}

\begin{entrée}
{souvent ; toujours}
\vedette{õ-taagi}
\région{GOs}
\variante{%
\vedette{õ-taagin}
\région{PA BO}}
\end{entrée}

\begin{entrée}
{sur le point de ; bientôt}
\classe{LOC}
\vedette{mõnu}
\sens{2}
\région{GOsPA}
\variante{%
\vedette{mõnu}
\région{PA BO}}
\variante{%
\vedette{mwonu}
\région{BO}}
\end{entrée}

\begin{entrée}
{terminer}
\classe{v}
\vedette{kû}\homonyme{2}
\sens{1}
\région{GOs}
\end{entrée}

\begin{entrée}
{toujours en train de}
\vedette{gaa ... gòl}
\région{PA}
\end{entrée}

\begin{entrée}
{toujours ; tout le temps}
\vedette{taagine}
\sens{2}
\région{GOs}
\variante{%
\vedette{taagin}
\région{PA BO}}
\variante{%
\vedette{tagin}
\région{BO}}
\variante{%
\vedette{taagi}
\région{WEM}}
\end{entrée}

\begin{entrée}
{venir de}
\classe{ASP}
\vedette{hõ}\homonyme{2}
\groupe{B}
\sens{2}
\région{GOs}
\variante{%
\vedette{hô}
\région{PA}}
\variante{%
\vedette{hò}
\région{BO}}
\end{entrée}

\begin{entrée}
{venir de ; commencer ; depuis}
\vedette{mada}\homonyme{1}
\région{BO PA}
\variante{%
\vedette{mara}
\région{PA BO}}
\end{entrée}

\begin{entrée}
{venir juste de}
\vedette{gaa ... hô}
\région{PA}
\end{entrée}

\begin{entrée}
{venir tout juste de}
\vedette{gaa mara}
\région{PA}
\end{entrée}

\begin{entrée}
{venir tout juste de}
\classe{INCH}
\vedette{mhaza}
\sens{3}
\région{GOs}
\variante{%
\vedette{maza}}
\variante{%
\vedette{mhara}
\région{PA WEM}}
\end{entrée}

\subsection{Temps}

\begin{entrée}
{autrefois ; il y a longtemps ; avant}
\vedette{ẽgõgò}
\région{GOs}
\variante{%
\vedette{êgòl}
\région{PA BO WEM WE}}
\variante{%
\vedette{êgògòn}
\région{BO}}
\end{entrée}

\begin{entrée}
{autrefois ; il y a longtemps ; avant}
\vedette{êgòl}
\région{PA BO WEM WE}
\variante{%
\vedette{êgõgò}
\région{GO}}
\end{entrée}

\begin{entrée}
{futur}
\vedette{ezoma}
\région{GOs}
\variante{%
\vedette{ruma}
\région{PA}}
\end{entrée}

\begin{entrée}
{futur}
\vedette{ruma}
\région{PA BO}
\variante{%
\vedette{toma}
\région{BO}}
\end{entrée}

\begin{entrée}
{futur}
\vedette{zo}\homonyme{2}
\région{GO}
\variante{%
\vedette{ro}
\région{WEMWE}}
\end{entrée}

\begin{entrée}
{futur}
\vedette{zoma}
\région{GOs}
\end{entrée}

\begin{entrée}
{futur proche}
\vedette{u ru}
\région{WE PA}
\variante{%
\vedette{u ru, ro}}
\end{entrée}

\begin{entrée}
{futur ; prospectif}
\vedette{ru}
\région{PA BO}
\variante{%
\vedette{to, ro}
\région{BO}}
\end{entrée}

\begin{entrée}
{il y a longtemps}
\vedette{pwali mwajin}
\région{PA}
\end{entrée}

\begin{entrée}
{indéfiniment ; un jour}
\vedette{eńiza-mwã}
\région{GO}
\variante{%
\vedette{inira-mwã}
\région{BO}}
\end{entrée}

\begin{entrée}
{toujours (Haudricourt)}
\vedette{ta-ecâna}
\région{GO}
\end{entrée}

\subsection{Modalité, verbes modaux}

\begin{entrée}
{adversatif ; incertain}
\vedette{bala}\homonyme{4}
\sens{1}
\région{GOs}
\end{entrée}

\begin{entrée}
{à la suite, dans la foulée}
\vedette{bala-n}
\région{PA}
\end{entrée}

\begin{entrée}
{au hasard ; de ci de là}
\vedette{hayu}
\sens{1}
\région{GOs BO}
\end{entrée}

\begin{entrée}
{au hasard ; sans but}
\vedette{lòlò}
\région{GOs}
\end{entrée}

\begin{entrée}
{autoriser [BO]}
\classe{v}
\vedette{pa-nuã}
\sens{2}
\région{GOs}
\variante{%
\vedette{pa-nhuã}
\région{PA BO}}
\end{entrée}

\begin{entrée}
{bref}
\vedette{khi}\homonyme{1}
\sens{2}
\région{GOs PA BO WEM}
\end{entrée}

\begin{entrée}
{ça doit être ; ce serait bien}
\vedette{memee}\homonyme{2}
\région{PA}
\variante{%
\vedette{wamee ne}
\région{GO}}
\end{entrée}

\begin{entrée}
{capable}
\classe{n.MODAL}
\vedette{jaxa}
\groupe{B}
\sens{1}
\région{GOs BO}
\end{entrée}

\begin{entrée}
{complètement}
\vedette{bala}\homonyme{4}
\sens{3}
\région{GOs}
\end{entrée}

\begin{entrée}
{contrastif}
\vedette{bala}\homonyme{4}
\sens{2}
\région{GOs}
\end{entrée}

\begin{entrée}
{convenir ; être bien}
\vedette{e}
\région{GOs}
\end{entrée}

\begin{entrée}
{devoir (épistémique)}
\classe{n.MODAL}
\vedette{jaxa}
\groupe{B}
\sens{3}
\région{GOs BO}
\end{entrée}

\begin{entrée}
{difficile ; impossible}
\vedette{pwawa}
\région{WEM WE BO PA}
\end{entrée}

\begin{entrée}
{droit ; autorisation}
\vedette{ku-gòò}
\groupe{B}
\sens{3}
\région{GOs}
\variante{%
\vedette{kô-go}
\région{BO}}
\end{entrée}

\begin{entrée}
{en vain ; sans résultat ; tant pis ! ; ce n'est pas grave}
\vedette{peu}\homonyme{1}
\région{GOs}
\variante{%
\vedette{peul}
\région{PA BO}}
\end{entrée}

\begin{entrée}
{essayer ; à l'essai ; à tout hasard}
\vedette{zaxòe}
\sens{2}
\région{GOsPA}
\variante{%
\vedette{zhaxòe}
\région{GA}}
\variante{%
\vedette{zakòe}
\région{GO(s)}}
\variante{%
\vedette{yaxòe, yagoe}
\région{BO}}
\end{entrée}

\begin{entrée}
{faillir}
\vedette{wa-na}
\région{PA}
\end{entrée}

\begin{entrée}
{faillir ; manquer de}
\classe{n.MODAL}
\vedette{jaxa}
\groupe{B}
\sens{2}
\région{GOs BO}
\end{entrée}

\begin{entrée}
{faillir ; s'en falloir de peu que}
\classe{v.IMPERS}
\vedette{mwêêno}
\sens{2}
\région{GOs BO}
\end{entrée}

\begin{entrée}
{faire spontanément (sans savoir, sans penser au résultat)}
\vedette{draa}\homonyme{2}
\sens{1}
\région{GOs}
\variante{%
\vedette{daa}
\région{BO}}
\end{entrée}

\begin{entrée}
{gaspiller [BO]}
\classe{v}
\vedette{ul}
\sens{2}
\région{PA BO}
\end{entrée}

\begin{entrée}
{impossible (lit. couché mal) ; difficile}
\vedette{kô-raa}
\région{GOs PA BO}
\end{entrée}

\begin{entrée}
{indécis}
\vedette{hayu}
\sens{1}
\région{GOs BO}
\end{entrée}

\begin{entrée}
{inutile}
\classe{v}
\vedette{ul}
\sens{2}
\région{PA BO}
\end{entrée}

\begin{entrée}
{jamais}
\vedette{kô-raa}
\région{GOs PA BO}
\end{entrée}

\begin{entrée}
{laisser ; permettre}
\vedette{kaale}
\région{GOs PA BO}
\end{entrée}

\begin{entrée}
{laisser ; permettre}
\vedette{kee}
\région{GOs}
\end{entrée}

\begin{entrée}
{mauvais ; le mal}
\vedette{thrava}
\région{GOs}
\end{entrée}

\begin{entrée}
{mesure ; assez ; juste ; suffisant}
\classe{n.MODAL}
\vedette{jaxa}
\groupe{A}
\région{GOs BO}
\end{entrée}

\begin{entrée}
{n'avoir jamais assez de}
\vedette{kixa khôôme}
\région{GOs}
\end{entrée}

\begin{entrée}
{ne faire que ; n'avoir de cesse que}
\vedette{kixa khôôme}
\région{GOs}
\end{entrée}

\begin{entrée}
{ne faire que ; n'avoir de cesse que}
\vedette{khôôme}
\sens{2}
\région{GOs PA}
\end{entrée}

\begin{entrée}
{ne jamais faire qqch (= ne pas savoir)}
\vedette{kavwö hine}
\région{GOs}
\end{entrée}

\begin{entrée}
{ne pas falloir}
\vedette{kêbwa}\homonyme{1}
\région{GOs PA BO}
\variante{%
\vedette{kêbwa-n}
\région{BO}}
\end{entrée}

\begin{entrée}
{optatif (à l'initiale de l'énoncé ; exprime: ordre, conseil, souhait)}
\vedette{ne, na}
\région{GOs BO}
\end{entrée}

\begin{entrée}
{ordre ; injonction}
\vedette{gu}\homonyme{5}
\région{GOs}
\variante{%
\vedette{ku}
\région{PA}}
\end{entrée}

\begin{entrée}
{pas du tout}
\vedette{jara}
\région{GOs}
\end{entrée}

\begin{entrée}
{persister à (sens positif)}
\vedette{kò-waayu}
\région{PA}
\end{entrée}

\begin{entrée}
{peut-être ; et si ?}
\vedette{wa-na}
\région{PA}
\end{entrée}

\begin{entrée}
{peut-être que}
\vedette{poxèè na}
\région{GOs PA}
\variante{%
\vedette{poxè}
\région{PA BO}}
\end{entrée}

\begin{entrée}
{peut-être que oui (réponse)}
\vedette{wa-vwo}
\région{GOs}
\end{entrée}

\begin{entrée}
{possible de ; permis de}
\vedette{kô-zo}
\région{GOsWEM}
\variante{%
\vedette{kô-yo}
\région{PA BO}}
\end{entrée}

\begin{entrée}
{pour toujours ; à jamais ; révolu}
\vedette{bala}\homonyme{4}
\sens{4}
\région{GOs}
\end{entrée}

\begin{entrée}
{pouvoir ; falloir ; devoir}
\classe{v.stat.}
\vedette{zo}\homonyme{1}
\sens{3}
\région{GOs PA}
\variante{%
\vedette{zho}
\région{GO(s)}}
\variante{%
\vedette{yo}
\région{BO}}
\end{entrée}

\begin{entrée}
{quand même ; faire à contre-coeur [GOs]}
\vedette{hayu}
\sens{2}
\région{GOs BO}
\end{entrée}

\begin{entrée}
{que !}
\vedette{poi}
\région{GO}
\end{entrée}

\begin{entrée}
{quelconque ; n'importe comment ; en vain}
\vedette{hayu}
\sens{1}
\région{GOs BO}
\end{entrée}

\begin{entrée}
{sans retour ; sans but ; sans limite}
\vedette{hayu}
\sens{1}
\région{GOs BO}
\end{entrée}

\begin{entrée}
{travers (de) ; mal fait}
\classe{v}
\vedette{alaxe}
\sens{2}
\région{GOs PA BO}
\end{entrée}

\begin{entrée}
{un coup}
\vedette{khi}\homonyme{1}
\sens{2}
\région{GOs PA BO WEM}
\end{entrée}

\begin{entrée}
{un peu [GOs]}
\vedette{khi}\homonyme{1}
\sens{1}
\région{GOs PA BO WEM}
\end{entrée}

\begin{entrée}
{utile ; nécessaire ; falloir}
\vedette{hãbaö}
\région{GOs BO}
\end{entrée}

\begin{entrée}
{volontairement ; exprès}
\vedette{draa pune}
\région{GOs}
\end{entrée}

\begin{entrée}
{vouloir}
\vedette{tòòwu}
\région{BO [BM]}
\variante{%
\vedette{thòòwu}}
\end{entrée}

\subsection{Marques assertives}

\begin{entrée}
{assertif}
\vedette{ra}
\région{PA}
\end{entrée}

\begin{entrée}
{c'est}
\vedette{ta}\homonyme{3}
\région{PA}
\variante{%
\vedette{ra}
\région{PA}}
\end{entrée}

\begin{entrée}
{c'est vraiment ... que}
\vedette{za}\homonyme{3}
\région{GOs}
\variante{%
\vedette{ra}
\région{WE WEM}}
\end{entrée}

\begin{entrée}
{non}
\vedette{hai}\homonyme{2}
\région{GOs PA BO}
\variante{%
\vedette{hayai}
\région{GO(s)}}
\end{entrée}

\begin{entrée}
{oui}
\classe{INTJ ; v}
\vedette{èlò}
\sens{1}
\région{GO PA BO}
\end{entrée}

\begin{entrée}
{vraiment ; tout à fait}
\vedette{ra-u}
\région{PA BO}
\end{entrée}

\begin{entrée}
{vraiment ; tout à fait}
\vedette{za}\homonyme{4}
\région{GOs}
\variante{%
\vedette{ra}
\région{PABO}}
\end{entrée}

\subsection{Causatif}

\begin{entrée}
{faire (causatif)}
\vedette{phaza-}
\région{GOs PA}
\variante{%
\vedette{para}
\région{BO}}
\end{entrée}

\begin{entrée}
{faire (faire)}
\vedette{pha-}
\région{GO BO PA}
\variante{%
\vedette{phaa-}}
\end{entrée}

\begin{entrée}
{faire faire qqch. ; laisser}
\vedette{pha- ...-ni}
\région{GOs}
\end{entrée}

\subsection{Comparaison}

\begin{entrée}
{ajuster (pour arriver sur la même ligne)}
\vedette{pe-gaixe}
\région{GOs PA BO [Corne]}
\end{entrée}

\begin{entrée}
{comme (être)}
\vedette{whã}
\région{GOs WEM}
\variante{%
\vedette{wã}
\région{GOs}}
\variante{%
\vedette{wa}
\région{PA BO}}
\variante{%
\vedette{wame}
\région{GO(s)}}
\end{entrée}

\begin{entrée}
{comme ; pareil ; semblable}
\vedette{wa-me}
\région{GOs WEM}
\variante{%
\vedette{wããme}
\région{GO(s)}}
\variante{%
\vedette{wamèèn}
\région{BO}}
\end{entrée}

\begin{entrée}
{comme ; semblable ; pareil}
\vedette{wala-me}
\région{BO}
\end{entrée}

\begin{entrée}
{comme si}
\vedette{wa-mèèn exa}
\région{BO [Corne]}
\end{entrée}

\begin{entrée}
{comparatif}
\classe{v ; n}
\vedette{trûã}\homonyme{2}
\sens{2}
\région{GOs WEM}
\variante{%
\vedette{thûã}
\région{PA}}
\variante{%
\vedette{tûãn}
\région{BO}}
\end{entrée}

\begin{entrée}
{faire comme ; faire ainsi (= dire ainsi)}
\vedette{whã}
\région{GOs WEM}
\variante{%
\vedette{wã}
\région{GOs}}
\variante{%
\vedette{wa}
\région{PA BO}}
\variante{%
\vedette{wame}
\région{GO(s)}}
\end{entrée}

\begin{entrée}
{même alignement ; parallèle ; ex aequo}
\vedette{pe-gaixe}
\région{GOs PA BO [Corne]}
\end{entrée}

\begin{entrée}
{même hauteur}
\vedette{pe-pwali}
\région{GOs}
\variante{%
\vedette{pe-pwawali}
\région{PA}}
\end{entrée}

\begin{entrée}
{même lignée}
\vedette{pe-îdò}
\région{GOs}
\end{entrée}

\begin{entrée}
{même taille ; même mesure}
\vedette{pe-jaxa}
\région{GOs}
\end{entrée}

\begin{entrée}
{nombre égal (être en)}
\vedette{pe-jivwaa}
\région{GOs}
\end{entrée}

\begin{entrée}
{parallèle ; dans le même alignement ; de même niveau ; ex aequo}
\vedette{pe-gaixe}
\région{GOs PA BO [Corne]}
\end{entrée}

\begin{entrée}
{pareil (être) ; semblable}
\vedette{menixe}
\région{GOs PA BO}
\end{entrée}

\begin{entrée}
{pareil (être) ; semblable}
\vedette{pe-menixe}
\région{GOs PA BO}
\end{entrée}

\begin{entrée}
{plus agé ; plus vieux}
\vedette{mhã whama}
\région{GOs}
\end{entrée}

\begin{entrée}
{plus haut que (être)}
\vedette{ge ... bwa}
\région{GOs}
\end{entrée}

\begin{entrée}
{plus jeune}
\vedette{mhã ẽnõ}
\région{GOs}
\end{entrée}

\begin{entrée}
{ressembler (se) ; pareil (être)}
\vedette{pe-balan}
\région{BO}
\end{entrée}

\begin{entrée}
{ressembler (se) ; semblable (être)}
\vedette{pe-wame}
\région{GOs WEM}
\end{entrée}

\begin{entrée}
{semblable ; pareil ; autant}
\vedette{pe-poxe}
\région{GOs}
\end{entrée}

\begin{entrée}
{tel quel (c'est la même apparence)}
\vedette{men-a-me}
\région{GOs}
\end{entrée}

\subsection{Conjonction}

\begin{entrée}
{à cause de ; parce que}
\vedette{wa}\homonyme{3}
\région{BO [Corne]}
\end{entrée}

\begin{entrée}
{alors}
\vedette{ta}\homonyme{1}
\end{entrée}

\begin{entrée}
{alors ; continuer à}
\vedette{mwã}\homonyme{1}
\région{GOs}
\end{entrée}

\begin{entrée}
{alors, puisque}
\vedette{köxa}
\région{GOs}
\end{entrée}

\begin{entrée}
{assertion, focus}
\vedette{xa}\homonyme{2}
\groupe{B}
\région{GOs BO PA}
\variante{%
\vedette{ga}
\région{PA}}
\variante{%
\vedette{ra}
\région{WEM}}
\end{entrée}

\begin{entrée}
{aussi ; et aussi}
\vedette{xo}\homonyme{2}
\région{GOs PA}
\end{entrée}

\begin{entrée}
{avec}
\classe{CNJ}
\vedette{mã}\homonyme{2}
\sens{1}
\région{GOs PA BO}
\end{entrée}

\begin{entrée}
{car ; parce que}
\classe{CNJ}
\vedette{mã}\homonyme{2}
\sens{2}
\région{GOs PA BO}
\end{entrée}

\begin{entrée}
{car ; parce que ; à cause de}
\vedette{pune}
\région{GOs}
\variante{%
\vedette{puni}
\région{PA}}
\end{entrée}

\begin{entrée}
{chaque fois que ; souvent ; tout le temps}
\vedette{õ-waran}
\région{PA}
\end{entrée}

\begin{entrée}
{devant ; avant , d'abord}
\vedette{hêbu}
\groupe{B}
\région{GOs}
\variante{%
\vedette{hêbun}
\région{PA BO}}
\end{entrée}

\begin{entrée}
{du fait que}
\vedette{po nye}
\région{GO}
\end{entrée}

\begin{entrée}
{du fait que ; parce que}
\vedette{pu nye}
\région{GO}
\variante{%
\vedette{po nye}
\région{GO}}
\end{entrée}

\begin{entrée}
{d'une part ... d'autre part}
\classe{CNJ}
\vedette{haxe}
\sens{2}
\région{GOs PA BO}
\variante{%
\vedette{axe}
\région{GO(s) PA BO}}
\variante{%
\vedette{age}
\région{BO}}
\end{entrée}

\begin{entrée}
{en revanche ; pourtant}
\vedette{poxee ma}
\région{GOs}
\end{entrée}

\begin{entrée}
{ensuite ; après}
\vedette{muna-le}
\région{GOs PA}
\end{entrée}

\begin{entrée}
{et}
\vedette{ko}\homonyme{2}
\région{GOs}
\variante{%
\vedette{xo, ka}
\région{GO(s)}}
\end{entrée}

\begin{entrée}
{et}
\classe{CNJ}
\vedette{mani}\homonyme{1}
\sens{2}
\région{GOs BO WEM}
\variante{%
\vedette{meni}
\région{PA}}
\end{entrée}

\begin{entrée}
{et alors}
\vedette{ge}\homonyme{2}
\région{BO}
\end{entrée}

\begin{entrée}
{et alors ; et aussi ; et en même temps}
\vedette{ka}\homonyme{1}
\sens{1}
\région{GOs}
\variante{%
\vedette{ga, xa}
\région{GO(s) PA}}
\variante{%
\vedette{ko}
\région{GO(s)}}
\end{entrée}

\begin{entrée}
{et après ; puis}
\vedette{jo}\homonyme{2}
\région{GOs BO}
\variante{%
\vedette{ço}
\région{GO(s)}}
\end{entrée}

\begin{entrée}
{et ; aussi}
\classe{CNJ}
\vedette{mã}\homonyme{2}
\sens{1}
\région{GOs PA BO}
\end{entrée}

\begin{entrée}
{et, aussi [PA]}
\classe{COORD}
\vedette{xa}\homonyme{2}
\groupe{A}
\sens{1}
\région{GOs BO PA}
\variante{%
\vedette{ga}
\région{PA}}
\variante{%
\vedette{ra}
\région{WEM}}
\end{entrée}

\begin{entrée}
{et ensuite ; et alors [Corne]}
\vedette{kèxi}
\région{BO}
\end{entrée}

\begin{entrée}
{et ; mais}
\vedette{axe}
\région{GO PA}
\variante{%
\vedette{haxe}
\région{GO PA}}
\end{entrée}

\begin{entrée}
{et ; mais ; mais (entre-temps ; pendant ce temps)}
\classe{CNJ}
\vedette{haxe}
\sens{1}
\région{GOs PA BO}
\variante{%
\vedette{axe}
\région{GO(s) PA BO}}
\variante{%
\vedette{age}
\région{BO}}
\end{entrée}

\begin{entrée}
{faire en sorte que}
\vedette{wa-me ne}
\région{GOs}
\end{entrée}

\begin{entrée}
{jusqu'à ce que}
\vedette{hovwa}
\sens{3}
\région{GOs WEM}
\variante{%
\vedette{hova}
\région{PA BO WEM}}
\variante{%
\vedette{hava}
\région{PA BO}}
\end{entrée}

\begin{entrée}
{jusqu'à ce que}
\vedette{hovwa-da xa}
\région{GOs}
\variante{%
\vedette{havha-da}
\région{PA}}
\end{entrée}

\begin{entrée}
{jusqu'à (locatif)}
\vedette{hovwa-da}
\région{GOs}
\variante{%
\vedette{havha-da}
\région{PA}}
\end{entrée}

\begin{entrée}
{jusqu'à (probablement < huli 'suivre')}
\vedette{hu}\homonyme{3}
\région{BO}
\end{entrée}

\begin{entrée}
{mais en revanche}
\vedette{axe poxee}
\région{GOs}
\end{entrée}

\begin{entrée}
{mais [GOs]}
\classe{COORD}
\vedette{xa}\homonyme{2}
\groupe{A}
\sens{2}
\région{GOs BO PA}
\variante{%
\vedette{ga}
\région{PA}}
\variante{%
\vedette{ra}
\région{WEM}}
\end{entrée}

\begin{entrée}
{mais seulement}
\vedette{axe poxe hõ ma}
\région{PA}
\end{entrée}

\begin{entrée}
{origine ; source ; cause}
\classe{nom}
\vedette{puu}\homonyme{1}
\sens{3}
\région{GOs PA BO}
\variante{%
\vedette{pu}
\région{GO(s)}}
\variante{%
\vedette{puu-n}
\région{BO PA}}
\variante{%
\vedette{puxu-n}
\région{BO}}
\end{entrée}

\begin{entrée}
{ou bien}
\vedette{a}\homonyme{2}
\région{GOs}
\variante{%
\vedette{hai, ha, ai}
\région{PA BO}}
\end{entrée}

\begin{entrée}
{ou bien}
\vedette{hai}\homonyme{1}
\région{PA BO}
\variante{%
\vedette{ha, ai}
\région{PA BO}}
\variante{%
\vedette{a}
\région{GO(s)}}
\end{entrée}

\begin{entrée}
{parce que ; à cause de}
\vedette{punyeda}
\région{GOs}
\end{entrée}

\begin{entrée}
{parce que ; du fait que}
\vedette{ma nye}
\région{GOs}
\end{entrée}

\begin{entrée}
{pendant que}
\vedette{kaxe}
\région{GO}
\end{entrée}

\begin{entrée}
{pour ; afin de}
\vedette{pu}\homonyme{3}
\région{PA BO [Corne]}
\variante{%
\vedette{wu}}
\end{entrée}

\begin{entrée}
{pour que}
\classe{CNJ}
\vedette{mã}\homonyme{2}
\sens{3}
\région{GOs PA BO}
\end{entrée}

\begin{entrée}
{pour que}
\vedette{vwö}\homonyme{1}
\région{GOs}
\variante{%
\vedette{vwu}
\région{BO PA}}
\variante{%
\vedette{wi}
\région{BO}}
\end{entrée}

\begin{entrée}
{quand}
\vedette{novwö me}
\région{GO}
\variante{%
\vedette{novwu}
\région{PA}}
\end{entrée}

\begin{entrée}
{quand, lorsque (passé)}
\vedette{novwö exa}
\région{GOs}
\variante{%
\vedette{novw-exa}
\région{PA}}
\end{entrée}

\begin{entrée}
{quand ; lorsque (référence au futur)}
\vedette{novwö na .... ça}\homonyme{1}
\variante{%
\vedette{novwu-na ... ye}
\région{PA}}
\variante{%
\vedette{nou-na ... ye}
\région{PA}}
\variante{%
\vedette{nopu}
\région{vx}}
\end{entrée}

\begin{entrée}
{quand ; lorsque (référence passée)}
\vedette{exa}
\région{PABO}
\variante{%
\vedette{eka}
\région{BO}}
\end{entrée}

\begin{entrée}
{quand (passé)}
\vedette{xa}\homonyme{3}
\région{GOs}
\end{entrée}

\begin{entrée}
{quand ; si}
\vedette{nõ-na}
\région{GO PA}
\variante{%
\vedette{novwö na}}
\variante{%
\vedette{nõ-ne}
\région{GO PA}}
\end{entrée}

\begin{entrée}
{que}
\vedette{a}\homonyme{5}
\région{BO}
\end{entrée}

\begin{entrée}
{que}
\vedette{na}\homonyme{3}
\région{GOs}
\variante{%
\vedette{ne}
\région{GO(s)}}
\variante{%
\vedette{na, ne}
\région{BO PA}}
\end{entrée}

\begin{entrée}
{que ; pour}
\vedette{wi}
\variante{%
\vedette{we}
\région{BO}}
\variante{%
\vedette{vwo}
\région{GO}}
\end{entrée}

\begin{entrée}
{que ; pour que ; afin que ; si}
\vedette{po}\homonyme{3}
\région{GOs}
\variante{%
\vedette{vwo}
\région{GO(s)}}
\variante{%
\vedette{pu}
\région{PA BO}}
\end{entrée}

\begin{entrée}
{que ; qui}
\vedette{xa}\homonyme{4}
\région{GOs PA BO}
\variante{%
\vedette{ka}
\région{vx}}
\end{entrée}

\begin{entrée}
{que ; quotatif ; marque d'incertitude}
\vedette{khõbwe}
\groupe{B}
\région{GOs}
\région{GOs BO}
\variante{%
\vedette{kõbwe}}
\end{entrée}

\begin{entrée}
{qui, que}
\vedette{ka}\homonyme{1}
\sens{2}
\région{GOs}
\variante{%
\vedette{ga, xa}
\région{GO(s) PA}}
\variante{%
\vedette{ko}
\région{GO(s)}}
\end{entrée}

\begin{entrée}
{quoique}
\vedette{kake}
\région{GO}
\end{entrée}

\begin{entrée}
{si (contrefactuel)}
\vedette{haze}\homonyme{1}
\sens{2}
\région{GOs}
\variante{%
\vedette{hale}
\région{PA BO}}
\end{entrée}

\begin{entrée}
{si (contrefactuel)}
\vedette{haze na}
\région{PA}
\variante{%
\vedette{hale na}
\région{PA}}
\end{entrée}

\begin{entrée}
{si (hypothétique)}
\vedette{novwö na .... ça}\homonyme{1}
\variante{%
\vedette{novwu-na ... ye}
\région{PA}}
\variante{%
\vedette{nou-na ... ye}
\région{PA}}
\variante{%
\vedette{nopu}
\région{vx}}
\end{entrée}

\begin{entrée}
{si ; hypothétique}
\vedette{no-me}\homonyme{2}
\région{GOs}
\end{entrée}

\begin{entrée}
{si jamais}
\vedette{novwö na khõbwe}
\région{GOs}
\end{entrée}

\begin{entrée}
{si ; quand ;}
\vedette{na}\homonyme{3}
\région{GOs}
\variante{%
\vedette{ne}
\région{GO(s)}}
\variante{%
\vedette{na, ne}
\région{BO PA}}
\end{entrée}

\begin{entrée}
{tandis que ; pendant}
\vedette{kaxe}
\région{GO}
\end{entrée}

\subsection{Contraste}

\begin{entrée}
{contraste (entre deux agents, deux moments, deux actions dont l'une est antérieure)}
\vedette{tre}
\région{GOs}
\variante{%
\vedette{tee, te}
\région{PA BO}}
\end{entrée}

\begin{entrée}
{contraste (opposition entre des actions, des agents)}
\vedette{draa}\homonyme{3}
\sens{1}
\région{GOs}
\variante{%
\vedette{daa}
\région{PA BO}}
\end{entrée}

\begin{entrée}
{déjà ; état résultant}
\vedette{tre}
\région{GOs}
\variante{%
\vedette{tee, te}
\région{PA BO}}
\end{entrée}

\begin{entrée}
{tour de (au)}
\vedette{draa}\homonyme{3}
\sens{2}
\région{GOs}
\variante{%
\vedette{daa}
\région{PA BO}}
\end{entrée}

\subsection{Démonstratifs}

\begin{entrée}
{agent (préfixe des noms d')}
\vedette{a-}\homonyme{1}
\région{GO PA}
\variante{%
\vedette{aa-}}
\end{entrée}

\begin{entrée}
{ce-ci}
\vedette{-è}\homonyme{2}
\région{GOs}
\end{entrée}

\begin{entrée}
{ce...ci}
\vedette{nhye}
\région{GOs PA BO}
\end{entrée}

\begin{entrée}
{ceci}
\vedette{-ã}
\région{GOs PA}
\end{entrée}

\begin{entrée}
{ceci}
\vedette{hî}
\région{PA BO [BM]}
\end{entrée}

\begin{entrée}
{ce ... -ci; cela}
\vedette{nye, nyã}
\région{GOs PA}
\variante{%
\vedette{nyã ; nye-ã}
\région{GO(s) PA}}
\end{entrée}

\begin{entrée}
{cela}
\vedette{nye-na}
\région{GOs}
\end{entrée}

\begin{entrée}
{cela ;}
\vedette{ji-ni}
\région{GOs}
\variante{%
\vedette{je-nim, ji-nim}
\région{PA}}
\end{entrée}

\begin{entrée}
{ce-là}
\vedette{je}\homonyme{2}
\région{PA BO}
\end{entrée}

\begin{entrée}
{cela (anaphorique) ; celui ; celle}
\vedette{-ò}\homonyme{3}
\région{GOs}
\end{entrée}

\begin{entrée}
{cela (audible, mais invisible)}
\vedette{hili}\homonyme{4}
\région{GOs PA}
\variante{%
\vedette{ili}}
\end{entrée}

\begin{entrée}
{cela (éloigné, invisible mais audible)}
\vedette{ji-li}
\région{GOs PA}
\end{entrée}

\begin{entrée}
{cela en haut}
\vedette{nhye-da}
\région{GOs}
\end{entrée}

\begin{entrée}
{cela (en question)}
\vedette{nyò}
\région{GOs}
\end{entrée}

\begin{entrée}
{cela (inanimé ; distance moyenne, mais visible)}
\vedette{nhye-ba}
\région{GOs WEM WE}
\end{entrée}

\begin{entrée}
{cela là-bas (loin des interlocuteurs)}
\vedette{nyõli}
\région{GOs WEM WE PA BO}
\end{entrée}

\begin{entrée}
{ce ... là (mis à distance, peut être péjoratif)}
\vedette{-ni}\homonyme{2}
\région{GOs}
\variante{%
\vedette{-nim}
\région{PA}}
\end{entrée}

\begin{entrée}
{cela (péjoratif, mis à distance) [PA]}
\vedette{ji-ni}
\région{GOs}
\variante{%
\vedette{je-nim, ji-nim}
\région{PA}}
\end{entrée}

\begin{entrée}
{ce-là ; voilà ; c'est cela !}
\vedette{je-nã}\homonyme{1}
\région{GOs}
\variante{%
\vedette{jene}
\région{PA BO}}
\end{entrée}

\begin{entrée}
{celle-là}
\vedette{je-ò}
\région{GOs PA}
\end{entrée}

\begin{entrée}
{celle-là (sur le côté, latéralement)}
\vedette{je-ba}
\région{GOs PA}
\end{entrée}

\begin{entrée}
{celui-ci (humain)}
\vedette{ã-}
\région{GOs WEM WE}
\end{entrée}

\begin{entrée}
{celui-là}
\vedette{ãnã}
\sens{1}
\région{GOs BO}
\end{entrée}

\begin{entrée}
{celui-là (en question)}
\vedette{ã-ò}
\région{GOs BO}
\end{entrée}

\begin{entrée}
{celui-là (latéralement,visible)}
\vedette{-ba}
\région{GOs}
\end{entrée}

\begin{entrée}
{ces2-ci}
\vedette{èli-ã}
\région{GOs}
\end{entrée}

\begin{entrée}
{ces3-ci}
\vedette{èlò-ã}
\région{GOs}
\end{entrée}

\begin{entrée}
{ces ... ci}
\vedette{lã-ã}
\région{GOs PA BO}
\variante{%
\vedette{lãã, lã}}
\end{entrée}

\begin{entrée}
{ces deux ... ci}
\vedette{li-ã}
\région{GOs PA}
\end{entrée}

\begin{entrée}
{ces deux-ci}
\vedette{liè}
\région{GOs}
\variante{%
\vedette{li-ã}
\région{BO}}
\end{entrée}

\begin{entrée}
{ces deux-là}
\vedette{liè-nã}
\région{GOs PA}
\variante{%
\vedette{li-nã}
\région{GO(s)}}
\variante{%
\vedette{li-ne}
\région{PA}}
\end{entrée}

\begin{entrée}
{ces deux là-bas sur le côté}
\vedette{liè-ba}
\région{GOs PA}
\end{entrée}

\begin{entrée}
{ces deux... là (péjoratif pour les humains, mis à distance)}
\vedette{li-nim}
\région{PA}
\end{entrée}

\begin{entrée}
{ces ... là (péjoratif pour les humains, mis à distance)}
\vedette{la-nim}
\région{PA}
\end{entrée}

\begin{entrée}
{c'est cela ; exactement ; tout à fait}
\vedette{je-nã}\homonyme{1}
\région{GOs}
\variante{%
\vedette{jene}
\région{PA BO}}
\end{entrée}

\begin{entrée}
{cette femme-ci}
\vedette{èjè-ã!}
\sens{1}
\région{GOs}
\end{entrée}

\begin{entrée}
{ceux-ci}
\vedette{èla-ã}
\région{GOs PA}
\end{entrée}

\begin{entrée}
{ceux-là}
\vedette{lã-nã}
\région{PA BO}
\end{entrée}

\begin{entrée}
{ceux-là-bas (DX3)}
\vedette{èla-õli}
\région{GOs PA}
\end{entrée}

\begin{entrée}
{ceux-là en bas}
\vedette{laa-du}
\région{GOs PA}
\end{entrée}

\begin{entrée}
{ceux-là en haut}
\vedette{la-ida}
\région{GOs PA}
\end{entrée}

\begin{entrée}
{ceux-là là-bas}
\vedette{laa-ba}
\région{GOs}
\end{entrée}

\begin{entrée}
{-ci (proche du locuteur et loin de l'interlocuteur) Dubois}
\vedette{-laeo}
\région{GO}
\end{entrée}

\begin{entrée}
{eh ! l'homme là !}
\vedette{ã-na !}
\région{GOs}
\end{entrée}

\begin{entrée}
{eh ! vous! (triel)}
\vedette{èlòè !}
\région{GOs}
\end{entrée}

\begin{entrée}
{elles-là}
\vedette{ila-e}
\région{GOs}
\end{entrée}

\begin{entrée}
{eux-deux là; hé ! les 2 hommes !}
\vedette{ã-mali-na}
\région{GOs BO}
\end{entrée}

\begin{entrée}
{eux là}
\vedette{ã-mala-e}
\région{GOs}
\end{entrée}

\begin{entrée}
{eux là}
\vedette{ã-mala-na}
\région{GOs PA BO}
\end{entrée}

\begin{entrée}
{eux là-bas}
\vedette{ã-mala-ò}
\région{GOs PA BO}
\end{entrée}

\begin{entrée}
{eux là-bas}
\vedette{ã-mala-òòli}
\région{GOs}
\end{entrée}

\begin{entrée}
{instrument à ; sert à}
\vedette{ba-}
\région{GOs PA}
\end{entrée}

\begin{entrée}
{là-bas très loin}
\vedette{nyõli mwa}
\région{PA}
\end{entrée}

\begin{entrée}
{là ; là-bas}
\vedette{-òli}
\région{GOs}
\variante{%
\vedette{hòli}}
\end{entrée}

\begin{entrée}
{là ; là-bas (inanimés ; absent mais connu des interlocuteurs)}
\vedette{ê}\homonyme{1}
\région{GOs}
\end{entrée}

\begin{entrée}
{là ; là-bas (visible)}
\vedette{la-òli}
\région{GOs PA}
\end{entrée}

\begin{entrée}
{là ; là-bas (visible)}
\vedette{li-òli}
\région{GOs PA}
\end{entrée}

\begin{entrée}
{là (visible ou non)}
\vedette{-na}
\région{GOs PA BO}
\end{entrée}

\begin{entrée}
{le, la, les ; ceci}
\vedette{nhya, nhye}\homonyme{2}
\région{GOs PA BO}
\end{entrée}

\begin{entrée}
{les ; ces}
\vedette{ã-mã-}
\région{GOs PA BO}
\end{entrée}

\begin{entrée}
{lui-là}
\vedette{ã-e}
\région{GOs}
\end{entrée}

\begin{entrée}
{lui là-bas}
\vedette{ã-òòli}
\région{GOs}
\end{entrée}

\begin{entrée}
{réfère au passé (plus loin dans le temps que -ò)}
\vedette{hèò}
\région{PA}
\variante{%
\vedette{-èò}
\région{PA}}
\end{entrée}

\begin{entrée}
{voilà ces deux-là}
\classe{DEM.duel}
\vedette{emãli}
\sens{1}
\région{GOs}
\end{entrée}

\begin{entrée}
{voilà ces trois-là}
\vedette{emãlo}
\région{GOs}
\end{entrée}

\begin{entrée}
{voilà (le) ; celui-là (visible)}
\vedette{e-nyoli}
\région{GOs PA}
\end{entrée}

\begin{entrée}
{voilà (les) là-bas (où je montre)}
\vedette{ila-lhãã-ba}
\région{GOs PA BO}
\end{entrée}

\subsection{Dérivation}

\begin{entrée}
{action de ; façon de ; fait de}
\vedette{me-}\homonyme{2}
\région{GOs BO PA}
\end{entrée}

\begin{entrée}
{nominalisateur ; saturateur de transitivité}
\vedette{-vwò}
\région{GOs PA BO}
\end{entrée}

\subsection{Prédicats existentiels}

\begin{entrée}
{faire}
\vedette{thu}
\région{GOs WEM BO PA}
\variante{%
\vedette{tho}
\région{BO}}
\end{entrée}

\begin{entrée}
{il n'y a rien (lit. il n'y a pas chose)}
\vedette{kia po}
\région{BO}
\end{entrée}

\begin{entrée}
{il y a}
\vedette{pu}\homonyme{4}
\région{GOs PA}
\end{entrée}

\begin{entrée}
{il y a ; c'est}
\vedette{thu}
\région{GOs WEM BO PA}
\variante{%
\vedette{tho}
\région{BO}}
\end{entrée}

\begin{entrée}
{il y a (indéfini nonspécifique)}
\vedette{ge-le-xa}
\région{GOs BO}
\end{entrée}

\begin{entrée}
{il y (en) a}
\vedette{ge-le}
\région{GOs PA BO}
\end{entrée}

\subsection{Injonction}

\begin{entrée}
{sors !; dehors !}
\vedette{u-pwa !}
\région{GOs PA}
\variante{%
\vedette{u-vwa}
\région{GO PA}}
\end{entrée}

\begin{entrée}
{vas-y ! (utilisé lors de la couverture du toit et la ligature de la paille, lorsque la personne sur le toit pique l'alène vers le bas)}
\vedette{a-ò !}
\région{GOs PA}
\end{entrée}

\begin{entrée}
{va-t-en !; pars !; vas-y !}
\vedette{a-ò !}
\région{GOs PA}
\end{entrée}

\subsection{Interjection et interpellation}

\subsubsection{Interjection}

\begin{entrée}
{attendez ! (respect) (lit. restez debout)}
\vedette{gaa kòò gò}
\région{GOs}
\variante{%
\vedette{gaa kòòl}
\région{PA BO}}
\end{entrée}

\begin{entrée}
{au revoir !}
\vedette{alawe}
\région{GOs PA}
\variante{%
\vedette{olawe, olaè, olaa}
\région{BO}}
\end{entrée}

\begin{entrée}
{bien fait !}
\vedette{e zo !}
\région{GOs}
\end{entrée}

\begin{entrée}
{dégage ! ; sors de là !}
\vedette{awaze}
\région{GOs}
\end{entrée}

\begin{entrée}
{hélas ; pauvre ! ; cher !}
\vedette{bwanamwa}
\région{GOs}
\variante{%
\vedette{bwaamwa}
\région{GO(s) PA WEM}}
\variante{%
\vedette{bwaa}
\région{PA}}
\end{entrée}

\begin{entrée}
{hmm (rythme le discours de quelqu'un)}
\vedette{ûû}
\région{GO}
\variante{%
\vedette{ôô}}
\end{entrée}

\begin{entrée}
{interpellation}
\vedette{hè}
\région{PA}
\end{entrée}

\begin{entrée}
{on y va ? (2 personnes ou plus)}
\vedette{gase ?}
\région{GOs}
\end{entrée}

\begin{entrée}
{salut ! (se dit quand on est loin)}
\vedette{aeke}
\région{WE}
\end{entrée}

\begin{entrée}
{signaler sa présence par un appel}
\vedette{hoo}\homonyme{1}
\région{PA}
\end{entrée}

\subsubsection{Interpellation}

\begin{entrée}
{eh !}
\vedette{ta-}
\région{BO}
\end{entrée}

\begin{entrée}
{eh ! la femme !}
\vedette{èjè-ã!}
\sens{2}
\région{GOs}
\end{entrée}

\begin{entrée}
{eh ! la femme !}
\vedette{ijèè !}
\région{GOs}
\end{entrée}

\begin{entrée}
{eh l'homme !}
\vedette{ãnã}
\sens{2}
\région{GOs BO}
\end{entrée}

\begin{entrée}
{eh ! l'homme là !}
\vedette{ã-na !}
\région{GOs}
\end{entrée}

\begin{entrée}
{eh toi !}
\vedette{hè-m !}
\région{PA BO}
\end{entrée}

\begin{entrée}
{eh vous deux !}
\classe{DEM.duel}
\vedette{emãli}
\sens{2}
\région{GOs}
\end{entrée}

\begin{entrée}
{eh ! vous! (triel)}
\vedette{èlòè !}
\région{GOs}
\end{entrée}

\begin{entrée}
{ô vous !}
\vedette{bwaa}\homonyme{1}
\région{GOs PA BO}
\variante{%
\vedette{bwaamwa}
\région{GOs PA BO}}
\end{entrée}

\subsection{Interrogatifs}

\begin{entrée}
{à combien dans chaque ?}
\vedette{pe-ka-aniza ?}
\région{GOs}
\variante{%
\vedette{pe-aniza ?}
\région{GO(s)}}
\end{entrée}

\begin{entrée}
{à combien dans chaque ?}
\vedette{pe-ka-poniza ?}
\région{GOs}
\end{entrée}

\begin{entrée}
{à quel endroit ?}
\vedette{ko ia ?}
\région{GOs}
\end{entrée}

\begin{entrée}
{combien?}
\vedette{-niza ?}
\région{GOs}
\variante{%
\vedette{-nira ?}
\région{BO}}
\end{entrée}

\begin{entrée}
{combien?}
\classe{INT}
\vedette{whaya ?}
\sens{2}
\région{GOs BO PA}
\end{entrée}

\begin{entrée}
{combien? (animés)}
\vedette{a-niza ?}
\région{GOs}
\région{PA BO}
\variante{%
\vedette{a-nira ?}}
\end{entrée}

\begin{entrée}
{combien (de choses longues, jours, an) ?}
\vedette{wè-niza ?}
\région{GOs}
\variante{%
\vedette{we-nira ?}
\région{PA BO}}
\end{entrée}

\begin{entrée}
{combien de fois?}
\vedette{õ-niza ?}
\région{GOs}
\variante{%
\vedette{õ-nira ?}
\région{BO}}
\end{entrée}

\begin{entrée}
{combien de morceaux (bois, poisson, anguilles) ?}
\vedette{gò-niza ?}
\région{GO}
\variante{%
\vedette{gò-nira ?}
\région{BO}}
\end{entrée}

\begin{entrée}
{combien de morceaux (pastèque, igname)}
\vedette{hô-niza ?}
\région{GOs}
\end{entrée}

\begin{entrée}
{combien de paquets de 3 (ignames, etc.) ?}
\vedette{mãè-nira ?}
\région{BO}
\end{entrée}

\begin{entrée}
{combien en tout (avons nous en commun)?}
\vedette{pe-po-niza ?}
\région{GOs}
\end{entrée}

\begin{entrée}
{combien ? (inanimés)}
\vedette{pò-niza ?}
\région{GOs}
\variante{%
\vedette{pònita?pònira?}
\région{PA BO}}
\variante{%
\vedette{pwònira?}
\région{BO}}
\end{entrée}

\begin{entrée}
{comment (être, faire) ? (aussi pour les propriétés physiques)}
\classe{INT}
\vedette{whaya ?}
\sens{1}
\région{GOs BO PA}
\end{entrée}

\begin{entrée}
{comment faire ? ; que faire ? (être comment?) ; faire ainsi}
\vedette{kaamwene ?}
\région{GOs}
\variante{%
\vedette{kamwelè ?}
\région{GO(s) BO}}
\variante{%
\vedette{kamweli}
\région{BO}}
\end{entrée}

\begin{entrée}
{comme qui ?}
\vedette{wa-me ti?}
\région{GOs WEM}
\end{entrée}

\begin{entrée}
{comme quoi? ; qui ressemble à quoi?}
\vedette{wa-me da?}
\région{GOs PA}
\variante{%
\vedette{wa da?}
\région{PA}}
\end{entrée}

\begin{entrée}
{faire comment ?}
\vedette{po-za ?}
\région{GOs}
\variante{%
\vedette{po-ra?}
\région{PA BO}}
\end{entrée}

\begin{entrée}
{jusqu'où ?}
\vedette{ça ea ?}
\région{GOs}
\end{entrée}

\begin{entrée}
{le(s)quel(s) ?}
\vedette{ji-a}
\région{GOs BO}
\end{entrée}

\begin{entrée}
{où ?}
\vedette{eva?}
\région{BO PA}
\end{entrée}

\begin{entrée}
{où ?}
\vedette{i ?}
\région{GOs BO}
\variante{%
\vedette{wi?}
\région{GO(s) BO}}
\end{entrée}

\begin{entrée}
{où?}
\vedette{èa?}
\région{GOs}
\variante{%
\vedette{ia?}
\région{GO(s)}}
\variante{%
\vedette{ia, ya}
\région{BO}}
\end{entrée}

\begin{entrée}
{où?}
\vedette{va ?}
\région{WEM BO}
\end{entrée}

\begin{entrée}
{où ? ; quel ?}
\vedette{ia ?}
\région{GOs BO}
\variante{%
\vedette{hia}
\région{PA}}
\end{entrée}

\begin{entrée}
{pour quoi ?}
\vedette{po-da ?}
\région{GO}
\end{entrée}

\begin{entrée}
{pourquoi ?}
\vedette{puneda ?}
\région{GOs BO}
\variante{%
\vedette{punanda?}
\région{PA}}
\end{entrée}

\begin{entrée}
{pourquoi ? ; comment ?}
\vedette{ka ?}
\région{GOs BO}
\end{entrée}

\begin{entrée}
{pourquoi faire ? ; à quoi sert ?}
\vedette{ponga da ?}
\région{GOs BO}
\variante{%
\vedette{puxã da ?}
\région{GO(s)}}
\end{entrée}

\begin{entrée}
{quand? (passé et futur)}
\vedette{èńiza ?}
\région{GOs}
\variante{%
\vedette{ènira ?}
\région{WEM WE PA}}
\variante{%
\vedette{inira ?}
\région{BO}}
\end{entrée}

\begin{entrée}
{que faire ?}
\vedette{kamwe ?}
\région{GOs BO}
\end{entrée}

\begin{entrée}
{que fais-tu ?}
\vedette{po za ?}
\région{GOs}
\variante{%
\vedette{po ra ?}
\région{PA}}
\end{entrée}

\begin{entrée}
{quel est le sens de ? ; quel est le nom de ?}
\vedette{yaaza da?}
\région{GOs}
\variante{%
\vedette{yhaala da ?}
\région{PA}}
\variante{%
\vedette{yaala da ?}
\région{PA}}
\end{entrée}

\begin{entrée}
{quelle longueur ?}
\vedette{pwali ?}\homonyme{1}
\région{PA}
\end{entrée}

\begin{entrée}
{quelle sorte (de) ?}
\vedette{za ?}
\région{GOs}
\variante{%
\vedette{ra?}
\région{PA BO}}
\end{entrée}

\begin{entrée}
{qu'est-ce qu'il y a ? ; qu'est-ce qui se passe?}
\vedette{ka ?}
\région{GOs BO}
\end{entrée}

\begin{entrée}
{qui ?}
\vedette{ti ?}
\région{GO PA BO}
\end{entrée}

\begin{entrée}
{qui (de) ?}
\vedette{ri ?}
\région{PA WE}
\end{entrée}

\begin{entrée}
{qui donc ?}
\vedette{ti-xa ?}
\région{PA}
\end{entrée}

\begin{entrée}
{quoi ?}
\vedette{da?}
\région{GOs PA BO}
\variante{%
\vedette{ta?}
\région{GO(s)}}
\end{entrée}

\begin{entrée}
{quoi?}
\vedette{ra ?}
\région{PA WE}
\variante{%
\vedette{za ?}
\région{GO(s)}}
\end{entrée}

\begin{entrée}
{quoi faire ? (lit. faire quoi ?)}
\vedette{po-ra ?}
\région{BO PA}
\variante{%
\vedette{po-za?}
\région{GO}}
\end{entrée}

\begin{entrée}
{quoi ? ; qu'est ce que ?}
\vedette{za ?}
\région{GOs}
\variante{%
\vedette{ra?}
\région{PA BO}}
\end{entrée}

\subsection{Négation}

\begin{entrée}
{absent ; ne pas/plus être là ; manquer ; sans}
\vedette{kòi}\homonyme{1}
\région{GOs BO}
\variante{%
\vedette{kòe, koi}
\région{PA}}
\end{entrée}

\begin{entrée}
{ne ... pas}
\vedette{ko}\homonyme{4}
\région{GOs}
\variante{%
\vedette{kavwö}}
\end{entrée}

\begin{entrée}
{ne...pas}
\vedette{kavwö}\homonyme{2}
\région{GOs PA BO}
\variante{%
\vedette{kavwa}
\région{GOs BO}}
\variante{%
\vedette{ka, kavwu}
\région{PA}}
\variante{%
\vedette{kapoa}
\région{vx}}
\end{entrée}

\subsection{Négation existentielle}

\begin{entrée}
{il n'y a pas ; rien ; sans}
\vedette{kixa}
\région{GOsPA}
\variante{%
\vedette{kiga}
\région{GO(s)}}
\variante{%
\vedette{kia}
\région{PA BO}}
\variante{%
\vedette{kiaxa; cixa}
\région{PA}}
\end{entrée}

\begin{entrée}
{il n'y a plus}
\vedette{kixa mwa}
\région{GOs}
\end{entrée}

\begin{entrée}
{personne (il n'y a) ; rien (lit. il n'y a pas de x que ...)}
\vedette{kixa na}
\région{GOs}
\variante{%
\vedette{kiaxa ne}
\région{PA}}
\end{entrée}

\subsection{Marque de nombre}

\begin{entrée}
{marque de duel (des déterminants)}
\vedette{mãli}\homonyme{1}
\région{BO [Corne]}
\end{entrée}

\begin{entrée}
{marque de duel (des déterminants)}
\vedette{mãli-}\homonyme{2}
\région{GOs PA}
\end{entrée}

\begin{entrée}
{marque de duel (des déterminants)}
\vedette{meli}
\région{GOs}
\end{entrée}

\begin{entrée}
{marque de non singulier}
\vedette{mã-}\homonyme{3}
\région{GOs PA}
\end{entrée}

\begin{entrée}
{marque de pluriel (des déterminants)}
\vedette{mãla-}
\région{GOs PA}
\end{entrée}

\begin{entrée}
{marque de triel (des déterminants)}
\vedette{mãlò-}
\région{GOs}
\end{entrée}

\subsection{Relateurs et relateurs possessifs}

\begin{entrée}
{de}
\vedette{ne}\homonyme{5}
\end{entrée}

\begin{entrée}
{relateur possessif: de}
\vedette{-a}
\région{GOs BO PA}
\end{entrée}

\begin{entrée}
{voyelle euphonique (parfois réalisée schwa)}
\vedette{a}\homonyme{6}
\région{PA BO}
\end{entrée}

\subsection{Préfixes compositionnels sémantiques}

\subsubsection{Préfixes sémantiques divers}

\begin{entrée}
{action faite en même temps}
\vedette{kha}\homonyme{5}
\région{GOs}
\end{entrée}

\begin{entrée}
{attribue une propriété}
\vedette{kha}\homonyme{1}
\région{GOs PA}
\variante{%
\vedette{ka}
\région{GOs PA}}
\end{entrée}

\begin{entrée}
{contenant ; produit de}
\vedette{paxa-}
\région{GOs}
\end{entrée}

\begin{entrée}
{contenant vide}
\classe{n ; PREF. sémantique (référant à une surface extérieure)}
\vedette{ala-}
\sens{3}
\région{GOs PA BO}
\end{entrée}

\begin{entrée}
{contenu de}
\vedette{hê-}
\sens{1}
\région{GOs}
\variante{%
\vedette{hêê-n}
\région{PA BO}}
\end{entrée}

\subsubsection{Préfixes sémantiques de position}

\begin{entrée}
{assis (faire)}
\vedette{tree}\homonyme{3}
\région{GOs}
\variante{%
\vedette{tee}
\région{PA}}
\end{entrée}

\begin{entrée}
{couché}
\vedette{kô-}\homonyme{1}
\région{GOs PA BO}
\end{entrée}

\begin{entrée}
{debout}
\vedette{ku-}\homonyme{2}
\région{PA BO}
\end{entrée}

\begin{entrée}
{rester}
\vedette{tree-ku}
\région{GO}
\end{entrée}

\subsubsection{Préfixes sémantiques d’action}

\begin{entrée}
{action faite en appuyant avec le pied ou la main}
\classe{PREF}
\vedette{kha}\homonyme{3}
\groupe{B}
\sens{1}
\région{GOs PA BO}
\variante{%
\vedette{khaa}
\région{PA}}
\end{entrée}

\begin{entrée}
{aplatir}
\vedette{khaa-bîni}
\région{GOs}
\variante{%
\vedette{khaa-bîni}
\région{PA}}
\end{entrée}

\begin{entrée}
{boîter ; boiteux}
\vedette{kha-thi}
\région{GOs}
\end{entrée}

\begin{entrée}
{bousculer qqn}
\vedette{khaa-tia}
\région{GOs}
\variante{%
\vedette{khaa-zia}
\région{GOs}}
\end{entrée}

\begin{entrée}
{déchirer en marchant}
\vedette{kha-mudree}
\région{GOs}
\variante{%
\vedette{kha-mude}
\région{PA}}
\end{entrée}

\begin{entrée}
{écraser (avec le pied)}
\vedette{khaa-bîni}
\région{GOs}
\variante{%
\vedette{khaa-bîni}
\région{PA}}
\end{entrée}

\begin{entrée}
{écraser avec le pied (en marchant)}
\vedette{kha-nhyale}
\région{GOs}
\end{entrée}

\begin{entrée}
{emporter}
\vedette{kha-phe}
\groupe{A}
\région{GOs PA BO}
\variante{%
\vedette{kha-vwe}
\région{GO(s)}}
\end{entrée}

\begin{entrée}
{fouler au pied}
\classe{v}
\vedette{khadra}
\sens{2}
\région{GOs}
\variante{%
\vedette{kadae}
\région{BO}}
\end{entrée}

\begin{entrée}
{prendre ; saisir (en partant)}
\vedette{kha-phe}
\groupe{A}
\région{GOs PA BO}
\variante{%
\vedette{kha-vwe}
\région{GO(s)}}
\end{entrée}

\begin{entrée}
{rencontrer par hasard ; rattraper}
\vedette{kha-tròòli}
\région{GOs}
\end{entrée}

\begin{entrée}
{traîner (un cheval) ; emmener (personne)}
\vedette{kha-whili}
\région{GOs BO}
\end{entrée}

\subsubsection{Préfixes sémantiques de déplacement}

\begin{entrée}
{appeler en marchant}
\vedette{kha-tho}
\région{GOs}
\end{entrée}

\begin{entrée}
{attraper (en déplacement)}
\vedette{kha-tree-çimwi}
\région{GOs}
\end{entrée}

\begin{entrée}
{dépasser en se déplaçant}
\vedette{kha-bazae}
\région{GOs}
\end{entrée}

\begin{entrée}
{déplacer (se) à pied}
\classe{PREF}
\vedette{kha}\homonyme{3}
\groupe{B}
\sens{2}
\région{GOs PA BO}
\variante{%
\vedette{khaa}
\région{PA}}
\end{entrée}

\begin{entrée}
{déplacer (se) en portant dans les bras}
\vedette{kha-bwaroe}
\région{PA}
\end{entrée}

\begin{entrée}
{descendre en marchant}
\vedette{kha-thrõbo}
\région{GOs}
\end{entrée}

\begin{entrée}
{faire qqch en se déplaçant à pied ou en mouvement}
\classe{PREF}
\vedette{kha}\homonyme{3}
\groupe{B}
\sens{2}
\région{GOs PA BO}
\variante{%
\vedette{khaa}
\région{PA}}
\end{entrée}

\begin{entrée}
{faire tomber en marchant}
\vedette{kha-ku}
\région{GOs}
\variante{%
\vedette{kha-kule}
\région{PA}}
\end{entrée}

\begin{entrée}
{grimper (en marchant)}
\vedette{kha-da}
\région{GOs PA}
\end{entrée}

\begin{entrée}
{marcher avec une canne}
\vedette{kha-hêgo}
\région{GOs PA}
\end{entrée}

\begin{entrée}
{marcher avec une charge sur le dos}
\vedette{kha-khoońe}
\région{GOs}
\end{entrée}

\begin{entrée}
{marcher sans bruit ; déplacer (se) doucement}
\vedette{kha-çaaxò}
\région{GOs}
\variante{%
\vedette{kacaaò}
\région{BO}}
\end{entrée}

\begin{entrée}
{marcher sur la pointe des pieds [GOs]}
\classe{v}
\vedette{kha-thixò}
\sens{1}
\région{GOs}
\variante{%
\vedette{kha-thixò}
\région{PA}}
\end{entrée}

\begin{entrée}
{monter à pied}
\vedette{kha-da}
\région{GOs PA}
\end{entrée}

\begin{entrée}
{partir en disant au-revoir}
\vedette{kha-alawe}
\région{GOs}
\variante{%
\vedette{kha-olae}
\région{PA}}
\end{entrée}

\begin{entrée}
{prendre un raccourci}
\vedette{kha-kibwaa}
\région{GOs}
\end{entrée}

\begin{entrée}
{rejoindre ; rattraper qqn}
\vedette{kha-tree-çimwi}
\région{GOs}
\end{entrée}

\begin{entrée}
{saisir en se déplaçant (en emportant ou amenant)}
\vedette{kha-cimwî}
\région{PA}
\end{entrée}

\begin{entrée}
{suivre ; longer à pied}
\vedette{kha-hoze}
\région{GOs}
\variante{%
\vedette{a-hoze}
\région{GO(s)}}
\end{entrée}

\begin{entrée}
{tirer en se déplaçant}
\vedette{kha-trivwi}
\région{GO}
\end{entrée}

\begin{entrée}
{traverser à pied (une route)}
\vedette{kha-çöe}
\région{GOs}
\variante{%
\vedette{khaa-jöe}
\région{PA}}
\end{entrée}

\subsection{Prépositions}

\begin{entrée}
{à cause de ; parce que}
\vedette{pu-n ma}
\région{BO}
\end{entrée}

\begin{entrée}
{à (destinataire animé)}
\vedette{cai}
\région{GOs WEM}
\variante{%
\vedette{çai}
\région{GO(s)}}
\variante{%
\vedette{yai}
\région{PA BO}}
\end{entrée}

\begin{entrée}
{à ; pour}
\vedette{hi}\homonyme{3}
\région{PA}
\end{entrée}

\begin{entrée}
{à ; pour}
\vedette{i}\homonyme{1}
\sens{1}
\région{GOs}
\end{entrée}

\begin{entrée}
{à, pour}
\vedette{kòlò}
\sens{3}
\région{GOs PA BO}
\variante{%
\vedette{kòlò-n}
\région{BO}}
\variante{%
\vedette{kòli}
\région{GO(s) PA}}
\end{entrée}

\begin{entrée}
{au sujet de ; à propos de (sens maléfactif)}
\vedette{pexa}
\région{GOs PA}
\end{entrée}

\begin{entrée}
{au sujet de ; envers ; à propos de (+ animés préférentiellement); à cause de}
\vedette{ui}\homonyme{2}
\région{GOs}
\variante{%
\vedette{wi}}
\end{entrée}

\begin{entrée}
{avec (en compagnie)}
\classe{CNJ}
\vedette{mani}\homonyme{1}
\sens{1}
\région{GOs BO WEM}
\variante{%
\vedette{meni}
\région{PA}}
\end{entrée}

\begin{entrée}
{avec ; ensemble}
\vedette{kha-phe}
\groupe{B}
\région{GOs PA BO}
\variante{%
\vedette{kha-vwe}
\région{GO(s)}}
\end{entrée}

\begin{entrée}
{avec (instrumental)}
\vedette{xo}\homonyme{3}
\sens{1}
\région{GOs BO PA}
\variante{%
\vedette{o}
\région{BO}}
\end{entrée}

\begin{entrée}
{avec (lit. derrière + animé)}
\vedette{kai}\homonyme{1}
\région{GOs PA}
\end{entrée}

\begin{entrée}
{avec ; par ; grâce à ; du fait de ; à cause de}
\vedette{u}\homonyme{2}
\région{BO [BM, Corne]}
\end{entrée}

\begin{entrée}
{à ; vers}
\classe{PREP}
\vedette{ca}\homonyme{3}
\sens{1}
\région{GOs}
\variante{%
\vedette{ya, yaa, yai}
\région{PA}}
\end{entrée}

\begin{entrée}
{dans (contenant) ; à ; vers}
\vedette{ni}\homonyme{2}
\région{GOs}
\variante{%
\vedette{ni}
\région{PA BO}}
\end{entrée}

\begin{entrée}
{de (ablatif) ; par rapport à ; envers}
\vedette{nai}
\région{GOs}
\région{BO}
\variante{%
\vedette{nai}}
\end{entrée}

\begin{entrée}
{d'où}
\vedette{na va}
\région{GOs BO PA}
\end{entrée}

\begin{entrée}
{jusqu'à}
\classe{PREP}
\vedette{ca}\homonyme{3}
\sens{2}
\région{GOs}
\variante{%
\vedette{ya, yaa, yai}
\région{PA}}
\end{entrée}

\begin{entrée}
{marque d'objet indirect}
\vedette{i}\homonyme{1}
\sens{2}
\région{GOs}
\end{entrée}

\begin{entrée}
{objet indirect}
\vedette{vwo}\homonyme{1}
\région{BO}
\end{entrée}

\begin{entrée}
{parmi ; entre}
\vedette{pe-dõńi}
\région{GOs BO}
\end{entrée}

\begin{entrée}
{pour}
\vedette{haba}
\région{BO}
\end{entrée}

\begin{entrée}
{pour}
\vedette{ponga}
\région{GOs}
\end{entrée}

\begin{entrée}
{pour}
\vedette{xo}\homonyme{3}
\sens{2}
\région{GOs BO PA}
\variante{%
\vedette{o}
\région{BO}}
\end{entrée}

\begin{entrée}
{proche ; près}
\classe{LOC}
\vedette{mõnu}
\sens{1}
\région{GOsPA}
\variante{%
\vedette{mõnu}
\région{PA BO}}
\variante{%
\vedette{mwonu}
\région{BO}}
\end{entrée}

\begin{entrée}
{vers}
\classe{LOC.DIR}
\vedette{wã}
\région{GOs}
\variante{%
\vedette{wãã}}
\end{entrée}

\subsection{Présentatifs}

\begin{entrée}
{voici (le)}
\vedette{e-nye}
\région{GOs PA BO}
\end{entrée}

\begin{entrée}
{voilà (le) ; celui-là (visible)}
\vedette{e-nyoli}
\région{GOs PA}
\end{entrée}

\begin{entrée}
{voilà (le) en haut}
\vedette{e-nyu-da}
\région{PA}
\end{entrée}

\begin{entrée}
{voilà (le) là-bas (à un endroit plus bas que là où on est)}
\vedette{e-nye-boli}
\région{PA}
\end{entrée}

\begin{entrée}
{voilà (le) là (on le montre et on regarde dans sa direction)}
\vedette{e-nye-ba}
\région{PA}
\end{entrée}

\subsection{Pronoms}

\begin{entrée}
{ces deux (personnes, proche) !}
\vedette{ili-e}
\région{GOs}
\end{entrée}

\begin{entrée}
{c'est à nous trois}
\vedette{pe-iô}
\région{GOs}
\end{entrée}

\begin{entrée}
{chose ; quelque chose}
\vedette{po}\homonyme{1}
\région{GOs BO}
\variante{%
\vedette{poo}
\région{BO}}
\end{entrée}

\begin{entrée}
{elle (au loin)}
\vedette{ijè-òli}
\région{GOs}
\end{entrée}

\begin{entrée}
{elle ; lui}
\vedette{ije}
\région{GOs BO PA}
\région{PA}
\variante{%
\vedette{iye}}
\end{entrée}

\begin{entrée}
{eux}
\vedette{lhaa}
\région{BO}
\end{entrée}

\begin{entrée}
{eux deux}
\vedette{ili}\homonyme{1}
\région{GOs PA}
\end{entrée}

\begin{entrée}
{eux, elles}
\vedette{ilaa}
\région{GOs PA}
\end{entrée}

\begin{entrée}
{eux/elles là-bas}
\vedette{ila-lhãã-òli}
\région{GOs}
\end{entrée}

\begin{entrée}
{eux trois}
\vedette{lòò}
\région{GO}
\variante{%
\vedette{lò}}
\variante{%
\vedette{mhõ}}
\end{entrée}

\begin{entrée}
{eux trois ; eux (paucal)}
\vedette{lhò}
\région{GOs}
\variante{%
\vedette{zò}
\région{WE}}
\end{entrée}

\begin{entrée}
{eux trois; eux (petit groupe)}
\vedette{ilò}
\région{GOs}
\end{entrée}

\begin{entrée}
{eux trois ; leurs (à eux trois)}
\vedette{-lòò}
\région{GO}
\variante{%
\vedette{-lò}}
\end{entrée}

\begin{entrée}
{il, elle}
\vedette{i}\homonyme{2}
\région{PA}
\end{entrée}

\begin{entrée}
{ils}
\vedette{lha}
\région{GOs PA}
\variante{%
\vedette{le}
\région{BO}}
\end{entrée}

\begin{entrée}
{ils}
\vedette{lhi}
\région{GOs PA}
\end{entrée}

\begin{entrée}
{je}
\vedette{nu}\homonyme{2}
\région{GOs}
\variante{%
\vedette{nu}
\région{PA BO}}
\end{entrée}

\begin{entrée}
{le, la}
\vedette{-je}
\région{GO}
\end{entrée}

\begin{entrée}
{le ; la ; son ; sa ; ses}
\vedette{-ye}
\région{PA BO}
\end{entrée}

\begin{entrée}
{les autres}
\vedette{èńoma}
\région{PA BO [BM]}
\end{entrée}

\begin{entrée}
{les ; leur}
\vedette{-li}
\région{GO PA}
\end{entrée}

\begin{entrée}
{les ; leur(s)}
\vedette{-laa}
\région{GOs PA}
\end{entrée}

\begin{entrée}
{me ; mon ; ma ; mes}
\vedette{-nu}
\région{GOs}
\end{entrée}

\begin{entrée}
{moi}
\vedette{inu}
\région{GOs PA}
\end{entrée}

\begin{entrée}
{mon ; ma ; mes}
\vedette{-ny}
\région{BO PA}
\end{entrée}

\begin{entrée}
{nos (deux2incl.)}
\vedette{-î}
\région{GO BO}
\end{entrée}

\begin{entrée}
{notre (plur. incl.)}
\vedette{-iã}
\région{GOsPA BO}
\end{entrée}

\begin{entrée}
{nous 2 ; nos}
\vedette{-êê}
\région{GO}
\end{entrée}

\begin{entrée}
{nous deux}
\vedette{mi}
\région{GOs PA BO}
\end{entrée}

\begin{entrée}
{nous deux (excl.)}
\vedette{ibî}
\région{GOs}
\variante{%
\vedette{ibîn}
\région{PA}}
\variante{%
\vedette{iciibii}
\région{vx}}
\end{entrée}

\begin{entrée}
{nous deux (excl.) ; notre}
\vedette{bi}\homonyme{3}
\région{GOsPA}
\end{entrée}

\begin{entrée}
{nous deux (excl.) ; notre}
\vedette{-bi}
\région{GOs}
\variante{%
\vedette{-bin}
\région{PA BO}}
\end{entrée}

\begin{entrée}
{nous deux (incl.)}
\vedette{-îî}
\région{GOs PA}
\end{entrée}

\begin{entrée}
{nous deux (inclusif)}
\vedette{îî}
\région{GOsPA}
\end{entrée}

\begin{entrée}
{nous (excl)}
\vedette{zava}
\région{GOs}
\variante{%
\vedette{za}
\région{PA}}
\end{entrée}

\begin{entrée}
{nous (excl) (forme ancienne)}
\vedette{izava}
\région{GOs}
\variante{%
\vedette{zava}
\région{GO(s)}}
\variante{%
\vedette{za}
\région{PA}}
\end{entrée}

\begin{entrée}
{nous (incl.)}
\vedette{mwã}\homonyme{3}
\région{GOs}
\variante{%
\vedette{mhwã}
\région{PA}}
\end{entrée}

\begin{entrée}
{nous ; nos}
\vedette{-ãã}
\région{GOs PA BO}
\end{entrée}

\begin{entrée}
{nous ; nos}
\vedette{-zava}
\région{GOs}
\end{entrée}

\begin{entrée}
{nous, notre}
\vedette{-za}
\région{PA BO}
\variante{%
\vedette{-ya}
\région{BO}}
\end{entrée}

\begin{entrée}
{nous (plur. excl.)}
\vedette{iva}
\région{GOs}
\end{entrée}

\begin{entrée}
{nous (plur. excl.)}
\vedette{iza}
\région{PA}
\variante{%
\vedette{iyãã}
\région{BO}}
\end{entrée}

\begin{entrée}
{nous (plur. incl.)}
\vedette{iã}
\région{GOs PA BO}
\end{entrée}

\begin{entrée}
{nous trois}
\vedette{mõ}\homonyme{2}
\région{GOs WEM}
\end{entrée}

\begin{entrée}
{nous-trois}
\vedette{-ôô}
\région{GO}
\end{entrée}

\begin{entrée}
{nous trois ; à nous trois}
\vedette{me}\homonyme{2}
\région{GO}
\end{entrée}

\begin{entrée}
{nous trois excl}
\vedette{ime}
\région{GOs}
\variante{%
\vedette{icòme}
\région{vx}}
\end{entrée}

\begin{entrée}
{nous trois incl. ; nous paucal}
\vedette{iõ}
\région{GOs}
\end{entrée}

\begin{entrée}
{son ; sa ; ses}
\vedette{-n}
\région{BO PA}
\end{entrée}

\begin{entrée}
{toi}
\vedette{içö}
\région{GOs}
\variante{%
\vedette{iyo}
\région{PA BO}}
\variante{%
\vedette{eyo}
\région{BO}}
\end{entrée}

\begin{entrée}
{toi}
\vedette{iyo}
\région{BO PA}
\variante{%
\vedette{eyo}
\région{BO}}
\end{entrée}

\begin{entrée}
{toi, tu}
\vedette{yo}
\région{PA BO}
\variante{%
\vedette{yu}
\région{GO}}
\end{entrée}

\begin{entrée}
{ton ; ta ; tes}
\vedette{-m}
\région{BO PA}
\end{entrée}

\begin{entrée}
{tu ; te}
\vedette{cö, çö, yö}
\région{GOs}
\variante{%
\vedette{co, yo}
\région{PA}}
\variante{%
\vedette{cu, yu}
\région{PABO}}
\end{entrée}

\begin{entrée}
{voilà (le) pas loin}
\vedette{e-jeni}
\région{GOs PA BO}
\end{entrée}

\begin{entrée}
{vous}
\vedette{a}\homonyme{3}
\end{entrée}

\begin{entrée}
{vous 2}
\vedette{cò, yò}\homonyme{3}
\région{GO PA}
\end{entrée}

\begin{entrée}
{vous 2}
\vedette{ijò}
\région{GOs}
\end{entrée}

\begin{entrée}
{vous2; vos}
\vedette{-jò}
\région{GOs}
\end{entrée}

\begin{entrée}
{vous 3}
\vedette{iwe}
\région{GOs}
\variante{%
\vedette{icòòwe}
\région{vx}}
\end{entrée}

\begin{entrée}
{vous (forme ancienne)}
\vedette{izawa}
\région{GOs}
\end{entrée}

\begin{entrée}
{vous (pl.)}
\vedette{zawa}
\région{GOs}
\end{entrée}

\begin{entrée}
{vous (plur.)}
\vedette{hã}\homonyme{3}
\région{BO}
\end{entrée}

\begin{entrée}
{vous (plur.)}
\vedette{izòò}
\région{PA BO}
\end{entrée}

\begin{entrée}
{vous (pluriel)}
\vedette{iwa}
\région{GOs PA}
\end{entrée}

\begin{entrée}
{vous ; vos}
\vedette{-zawa}
\région{GO}
\end{entrée}

\begin{entrée}
{vous, votre (plur)}
\vedette{zò}\homonyme{1}
\région{PA}
\end{entrée}

\subsection{Réciproque}

\begin{entrée}
{battre (se) (avec armes)}
\vedette{pe-kubu}
\région{GOs PA}
\end{entrée}

\begin{entrée}
{chercher querelle (se)}
\vedette{pe-chôã}
\région{GOs}
\end{entrée}

\begin{entrée}
{connaître (se)}
\vedette{pe-hine}
\région{GOs}
\end{entrée}

\begin{entrée}
{debout (être) face à face}
\vedette{pe-ku-alö}
\région{GOs}
\end{entrée}

\begin{entrée}
{détester (se) ; rejeter (se)}
\vedette{pe-kueli}
\région{GOs}
\end{entrée}

\begin{entrée}
{disputer (se)(jeu, compétition) ; garder pour soi}
\vedette{pe-kae}
\région{GOs}
\variante{%
\vedette{pe-kaeny}
\région{PA}}
\end{entrée}

\begin{entrée}
{jouer (se) des tours ; taquiner (se)}
\vedette{pe-chôã}
\région{GOs}
\end{entrée}

\begin{entrée}
{mutuellement}
\vedette{pe-}\homonyme{2}
\sens{1}
\région{GOs PA}
\end{entrée}

\begin{entrée}
{raccorder}
\vedette{pe-ki}
\région{GOs}
\end{entrée}

\begin{entrée}
{serrer (se) la main}
\vedette{pe-cimwi hi}
\région{GOs PA}
\end{entrée}

\subsubsection{Réciproque collectif}

\begin{entrée}
{aller ensemble ; accompagner}
\vedette{pe-mhe}
\région{GOs BO}
\end{entrée}

\begin{entrée}
{assis (être) face à face}
\vedette{pe-tre-alö}
\région{GOs}
\end{entrée}

\begin{entrée}
{à tout à l'heure ; à tout de suite}
\vedette{pe-tròòli èò}
\région{GOs}
\end{entrée}

\begin{entrée}
{à un de ces jours !}
\vedette{pe-tròòli ni tree}
\région{GOs}
\end{entrée}

\begin{entrée}
{bagarer (se) ; battre (se)}
\classe{v}
\vedette{pe-bu}
\sens{1}
\région{GOs WEM}
\end{entrée}

\begin{entrée}
{battre (se)}
\vedette{pe-çabi}
\région{GOs}
\variante{%
\vedette{pe-çabi}
\région{GO(s)}}
\variante{%
\vedette{pe-cabi}
\région{BO [Corne]}}
\end{entrée}

\begin{entrée}
{battre (se) (avec ou sans armes)}
\vedette{pe-wova}
\région{GOs PA}
\variante{%
\vedette{pe-woza}
\région{WEM}}
\end{entrée}

\begin{entrée}
{cogner (se) ; entrer en collision}
\classe{v}
\vedette{pe-bu}
\sens{2}
\région{GOs WEM}
\end{entrée}

\begin{entrée}
{couché (être) face à face}
\vedette{pe-kô-alö}
\région{GOs}
\end{entrée}

\begin{entrée}
{débattre ; discuter ; débat ; discussion}
\vedette{pe-whaguzai}
\région{GOs}
\end{entrée}

\begin{entrée}
{dire au revoir (se) ; à demain !}
\vedette{pe-tròòli mõnõ}
\sens{1}
\région{GOs}
\variante{%
\vedette{pe-ròòli mõnõ}
\région{GO(s)}}
\variante{%
\vedette{pe-tòòli menon}
\région{PA}}
\end{entrée}

\begin{entrée}
{discuter}
\vedette{pe-phweexu}
\région{GOs PA BO}
\end{entrée}

\begin{entrée}
{discuter ; mettre d'accord (se)}
\vedette{pe-khõbwe}
\région{GOs}
\end{entrée}

\begin{entrée}
{disputer (se) ; battre (se) (avec ou sans armes)}
\vedette{pe-wèle}
\région{BO PA}
\end{entrée}

\begin{entrée}
{empiler}
\vedette{pe-na bwa}
\région{GOs}
\end{entrée}

\begin{entrée}
{ensemble}
\vedette{pe-bulu}
\région{GOs BO}
\end{entrée}

\begin{entrée}
{ensemble (devant un nombre: marque un tout)}
\vedette{pe-}\homonyme{2}
\sens{2}
\région{GOs PA}
\end{entrée}

\begin{entrée}
{ensemble (être)}
\vedette{pe-a-vwe}
\région{GOs}
\end{entrée}

\begin{entrée}
{envoyer (s') mutuellement}
\vedette{pe-na}
\région{GOs}
\end{entrée}

\begin{entrée}
{éviter (s')}
\vedette{pe-vii}
\région{GOs}
\end{entrée}

\begin{entrée}
{faire équipe}
\vedette{pe-thu-ba}
\région{GOs}
\end{entrée}

\begin{entrée}
{faire la guerre (se)}
\vedette{pe-thu-paa}
\région{GOs}
\end{entrée}

\begin{entrée}
{faire (se) peur mutuellement}
\vedette{pe-phaza-hããxa}
\région{GOs}
\end{entrée}

\begin{entrée}
{faire une cérémonie coutumière ; faire les échanges coutumiers}
\vedette{pe-navwo}
\région{GOs PA}
\end{entrée}

\begin{entrée}
{joindre (se)}
\vedette{pe-vwii}
\région{GOs}
\variante{%
\vedette{pe-viing}
\région{PA WEM}}
\variante{%
\vedette{phiing}
\région{PA BO}}
\end{entrée}

\begin{entrée}
{marcher en file indienne}
\vedette{pe-gu-xe}
\région{PA}
\end{entrée}

\begin{entrée}
{même équipe}
\vedette{pe-a-bala}
\région{GO PA}
\end{entrée}

\begin{entrée}
{même équipe (être dans la)}
\vedette{pe-bala}
\région{GOs}
\end{entrée}

\begin{entrée}
{mettre bout à bout (et allonger)}
\vedette{pe-kine}
\région{GOs}
\end{entrée}

\begin{entrée}
{monter ensemble}
\vedette{pe-a-da}
\région{GO}
\end{entrée}

\begin{entrée}
{paire ; faire équipe ; être en binôme avec}
\classe{v}
\vedette{pe-thilò}
\sens{1}
\région{GOs}
\variante{%
\vedette{pe-dilo}
\région{BO}}
\end{entrée}

\begin{entrée}
{pardonner (se)}
\vedette{pe-ne-zo-ni}
\région{GOs}
\end{entrée}

\begin{entrée}
{pardonner (se) (lit. attacher le pardon)}
\vedette{pe-nhoi thria}
\région{GOs}
\end{entrée}

\begin{entrée}
{pardonner (se) mutuellement}
\vedette{pe-tò do}
\région{PA BO}
\end{entrée}

\begin{entrée}
{pardonner (se) (pardon coutumier)}
\vedette{pe-tha thria}
\région{GOs}
\variante{%
\vedette{pe-tha thia}
\région{PA}}
\end{entrée}

\begin{entrée}
{provoquer (se)}
\vedette{pe-thi thô}
\région{GOs}
\end{entrée}

\begin{entrée}
{rappeller (se) mutuellement}
\vedette{pe-pha-nõnõmi}
\région{GOs}
\end{entrée}

\begin{entrée}
{rassembler ; assembler}
\vedette{pe-na-bulu-ni}
\région{GOs}
\end{entrée}

\begin{entrée}
{rejoindre (se)}
\vedette{pe-vwii}
\région{GOs}
\variante{%
\vedette{pe-viing}
\région{PA WEM}}
\variante{%
\vedette{phiing}
\région{PA BO}}
\end{entrée}

\begin{entrée}
{rencontrer (se)}
\vedette{pe-trò}
\région{GOs}
\end{entrée}

\begin{entrée}
{rencontrer (se)}
\classe{v ; n}
\vedette{pe-tròòli}
\sens{1}
\région{GOs PA}
\variante{%
\vedette{pe-ròòli}
\région{GO(s)}}
\end{entrée}

\begin{entrée}
{réunion ; réunir (se)}
\classe{v ; n}
\vedette{pe-tròòli}
\sens{1}
\région{GOs PA}
\variante{%
\vedette{pe-ròòli}
\région{GO(s)}}
\end{entrée}

\begin{entrée}
{réunir (se)}
\vedette{pe-thaivwi}
\région{GOs PA}
\end{entrée}

\begin{entrée}
{suivre (se) ; marcher l'un derrière l'autre}
\vedette{pe-whili}
\région{GOs}
\end{entrée}

\begin{entrée}
{superposer (se)}
\vedette{pe-thirãgo}
\région{GOs}
\end{entrée}

\begin{entrée}
{toutes les choses possédées ensemble}
\vedette{pe-pwaixe}
\région{GOs}
\end{entrée}

\subsection{Réfléchi, intensificateur}

\subsubsection{Réfléchi}

\begin{entrée}
{couper (se)}
\vedette{thuvwu zòò}
\région{GOs}
\end{entrée}

\begin{entrée}
{réfléchi (du sujet, agent)}
\vedette{thuu}\homonyme{2}
\région{PA}
\variante{%
\vedette{thuvwu}
\région{GO(s)}}
\end{entrée}

\begin{entrée}
{réfléchi (du sujet, agent)}
\vedette{thuvwu}
\région{GOs}
\variante{%
\vedette{thuu}
\région{PA}}
\variante{%
\vedette{tuu}
\région{BO}}
\end{entrée}

\subsubsection{Intensificateur}

\begin{entrée}
{seul ; de/par soi-même}
\vedette{draa}\homonyme{2}
\sens{2}
\région{GOs}
\variante{%
\vedette{daa}
\région{BO}}
\end{entrée}

\subsection{Marques restrictives}

\begin{entrée}
{seulement}
\vedette{tee}
\région{PA}
\end{entrée}

\begin{entrée}
{seul(ement) (on attend plus)}
\vedette{ńõ}\homonyme{2}
\région{GOs}
\end{entrée}

\begin{entrée}
{seul ; seulement}
\vedette{-on}
\région{BO}
\end{entrée}

\begin{entrée}
{seul ; tout seul ; seulement}
\vedette{hãda}
\région{GOs PA BO}
\variante{%
\vedette{(h)ada}}
\end{entrée}

\subsection{Suffixes transitifs}

\begin{entrée}
{saturateur transitif}
\vedette{xo}\homonyme{4}
\région{GOs}
\variante{%
\vedette{vwo}
\région{GO(s)}}
\end{entrée}

\begin{entrée}
{suffixe transitif}
\vedette{-ni}\homonyme{3}
\région{GOs}
\variante{%
\vedette{-ni}
\région{PA BO}}
\end{entrée}

\subsection{Structure informationnelle}

\begin{entrée}
{focus}
\vedette{vwo}\homonyme{2}
\région{GOs PA}
\variante{%
\vedette{vo}
\région{BO}}
\end{entrée}

\begin{entrée}
{quant à}
\vedette{vwö}\homonyme{2}
\région{GOs}
\end{entrée}

\begin{entrée}
{quant à ... alors}
\vedette{novwö ... ça}
\région{GOs}
\variante{%
\vedette{novwö ... ye}
\région{BO PA}}
\variante{%
\vedette{nopo, novwu}
\région{vx}}
\end{entrée}

\begin{entrée}
{thématisation}
\vedette{ce}\homonyme{2}
\région{GOs}
\variante{%
\vedette{je}
\région{GOs}}
\variante{%
\vedette{ye}
\région{PA}}
\end{entrée}

\begin{entrée}
{thématisation}
\vedette{ça}
\région{GOs PA}
\variante{%
\vedette{ka}
\région{GO(s)}}
\variante{%
\vedette{ce, je, ye}
\région{BO PA}}
\end{entrée}

\begin{entrée}
{thématisation}
\vedette{ye}
\région{PA}
\end{entrée}

\subsection{Vocatifs}

\begin{entrée}
{maman!}
\vedette{nyejo!}
\région{PA}
\end{entrée}

\begin{entrée}
{papa !}
\vedette{caayo!}
\région{BO (Corne)}
\variante{%
\vedette{caayu!}}
\end{entrée}

\begin{entrée}
{vocatif ; exclamatif}
\vedette{-ò}\homonyme{2}
\région{GOs}
\région{PA WEM WE BO}
\variante{%
\vedette{-o}}
\end{entrée}

\end{multicols}
\end{document}