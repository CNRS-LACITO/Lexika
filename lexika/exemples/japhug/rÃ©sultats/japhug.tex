
\documentclass[twoside,11pt]{article}
\title{Essai d'un dictionnaire japhug}
\author{Guillaume Jacques}
\usepackage[paperwidth=185mm,paperheight=260mm,top=16mm,bottom=16mm,left=15mm,right=20mm]{geometry}
\usepackage{multicol}
\setlength{\columnseprule}{1pt}
\setlength{\columnsep}{1.5cm}
\usepackage{changepage}
\setlength\parindent{-0.5em}
\usepackage{color}
\usepackage{fancyhdr}
\pagestyle{fancy}
\fancyheadoffset{3.4em}
\fancyhead[LE,LO]{\rightmark}
\fancyhead[RE,RO]{\leftmark}
\usepackage[bookmarks=true,colorlinks,linkcolor=blue]{hyperref}
\hypersetup{bookmarks=false,bookmarksnumbered,bookmarksopenlevel=5,bookmarksdepth=5,xetex,colorlinks=true,linkcolor=blue,citecolor=blue}
\usepackage[all]{hypcap}
\usepackage{fontspec}
\usepackage{natbib}
\usepackage{booktabs}
\usepackage{polyglossia}
\setdefaultlanguage{french}
\setmainfont{Liberation Serif}
\usepackage{media9}
\usepackage{fontawesome}
\newfontfamily{\prin}[Mapping=tex-text,Ligatures=Common,Scale=MatchUppercase]{Liberation Serif}
\newfontfamily{\fra}[Mapping=tex-text,Ligatures=Common,Scale=MatchUppercase]{EB Garamond}
\newfontfamily{\cmn}[Mapping=tex-text,Ligatures=Common,Scale=MatchUppercase]{SimSun}
\newfontfamily{\jya}[Mapping=tex-text,Ligatures=Common,Scale=MatchUppercase]{EB Garamond}
\newfontfamily{\api}[Mapping=tex-text,Ligatures=Common,Scale=MatchUppercase]{Gentium Plus}
\newcommand{\pfra}[1]{{\fra #1}}
\newcommand{\pcmn}[1]{{\cmn #1}}
\newcommand{\pjya}[1]{{\jya #1}}
\newcommand{\papi}[1]{{\api #1}}
\newcommand{\cerclé}[1]{\raisebox{0pt}{\textcircled{\raisebox{-0.5pt} {\footnotesize{\papi{#1}}}}}}
\newcommand{\caractère}[1]{\begin{center}\textbf{\Large #1}\end{center}}
\newenvironment{entrée}{\par}{}
\newenvironment{sous-entrée}{\begin{adjustwidth}{0.3cm}{}\prin ■ \api}{\end{adjustwidth}}
\newcommand{\vedette}[1]{\textbf{\Large \papi{#1}}}
\newcommand{\homonyme}[1]{\textsubscript{#1}}
\newcommand{\variante}[1]{\textbf{\papi{#1}}}
\newcommand{\classe}[1]{ \textit{#1. }}
\newcommand{\paradigme}[1]{#1 }
\newcommand{\acception}[1]{ \cerclé{#1} }
\newenvironment{définition}{}{\hspace{5pt}}
\newenvironment{déclaration}{}{}
\newenvironment{exemple}{\prin ¶ }{\hspace{5pt}}
\newenvironment{relation-sémantique}{}{}
\newenvironment{forme-mot}{}{}
\newcommand{\synonyme}[1]{\pcmn{ ~【同义词】~#1}}
\newcommand{\antonyme}[1]{\pcmn{ ~【反义词】~#1}}
\newcommand{\confer}[1]{\pcmn{ ~【参考】~#1}}
\newcommand{\étymologie}[1]{\pcmn{ ~【借词】~#1}}
\newcommand{\use}[1]{\pcmn{ ~【用法】~#1}}
\newcommand{\grammar}[1]{\textsc{#1}}
\newcommand{\ComponentA}[1]{\cerclé{I} #1}
\newcommand{\ComponentB}[1]{\cerclé{II} #1}
\newcommand{\stylefv}[1]{\papi{#1}}
\newcommand{\stylefn}[1]{\pcmn{#1}}
\newcommand{\écouter}[1]{\includemedia[activate=onclick,addresource=#1.wav,flashvars={source=#1.wav&autoPlay=true&autoRewind=true&loop=false&hideBar=true&volume=1.0&balance=0.0}]{\faicon{volume-down}}{APlayer.swf}}
\addmediapath{AudioJaphug}
\newenvironment{bottompar}{\par\vspace*{\fill}}{\clearpage}
\newcommand{\ital}[1]{{\normalfont\textit{#1}}}
\newcommand{\caps}[1]{{\normalfont\textsc{#1}}}
\usepackage{totcount}
\newcounter{entrycounter}\setcounter{entrycounter}{0}\regtotcounter{entrycounter}%Compteur du nombre d'entrées
\XeTeXlinebreaklocale "zh"
\XeTeXlinebreakskip = 0pt plus 1pt

\begin{document}
\pagenumbering{roman}
\begin{flushright}
\textit{Pour Archimède}
\end{flushright}
\begin{bottompar}
Jacques, Guillaume 2015-2016. \textit{Dictionnaire Japhug-Chinois-Français, version 1.12} \pcmn{嘉绒-汉-法词典}, Paris: Projet HimalCo. \url{http://himalco.huma-num.fr/}
\end{bottompar}

\newpage

\section*{\pcmn{前言}}
\markboth{\pcmn{前言}}{}

\pcmn{这部《嘉绒-汉-法词典》是茶堡嘉绒语的首部词典,收录\total{entrycounter}个词条。茶堡话分布在马尔康县龙尔甲、沙尔宗、大藏三个乡。虽然按照当地人“茶堡”(\papi{tɕɤpʰɯ})这一地名不包括龙尔甲乡,但是由于龙尔甲乡的方言与沙尔宗乡和大藏乡区别不大,交流对话没有任何障碍,所以仍然用“茶堡话”作为三个乡的共同语言的统称。

本部词典的编辑工作是笔者的语言描写计划的其中一个项目,此外笔者还对茶堡话的语法进行了深入的研究(参考向柏霖 \citeyear{jacques08}以及\citealt{jacques12incorp, jacques13tropative, jacques14antipassive,  jacques15spontaneous}, \citeyear{jacques15causative, jacques16relatives}等)并收集了近80个小时的长篇故事,这些故事的国际音标转写以及语音文件已经发表在PANGLOSS语料库的网站上(\citealt{michailovsky14pangloss})。笔者在语法研究中借鉴了前辈学者林向荣(\citeyear{linxr93jiarong})、孙天心(\citeyear{jackson00sidaba, jackson04zhuangmaoci, jackson06paisheng, jackson14morpho}等)、林幼菁和罗尔武(\citeyear{linluo03})等的研究成果,并从他们的著作和论文中得到了很多启发。


本词典以龙尔甲乡干木鸟村(\papi{kɤmɲɯ})的口音为标准,是笔者与陈珍和柏尔青两位老师自2002年以来至今长期合作的共同研究成果。

词条的一大部分(特别是动词和状貌词的词条)都包含有代表性的例句,很多这些例句是从对话和传统故事中选出来的,并附有语音文件。

每一项词条包含法语和汉语的定义,并注明词类,茶堡话的词类包括下列14种:

\begin{itemize}
\item \ital{adv} 副词
\item \ital{clf} 量词
\item \ital{idph} 状貌词
\item \ital{intj} 感叹词
\item \ital{n} 名词
\item \ital{np} 被领属名词
\item \ital{postp} 介词
\item \ital{pro} 人称代词
\item \ital{vi} 不及物动词
\item \ital{vinh} 不及物动词(无人主语)
\item \ital{vi-t} 半及物动词
\item \ital{vs} 静态动词
\item \ital{vt} 及物动词
\item \ital{pc(x,y)} 复合谓语
\end{itemize}

复合谓语是由两个单词组成的,其中第一个单词 \ital{x} 可以是名词或者动词,第二个单词 \ital{y}必须是动词。缩写符号\ital{pc}后面的两个符号分别表示第一和第二成分的词类,例如 \papi{tɯ-ʑi,loʁ} \ital{pc(np,vs)} “感到恶心”表明,第一个成分 \papi{tɯ-ʑi} 是被领属名词,而第二成分\papi{loʁ} 是静态动词。

附在 \ital{idph} 后的数字表示状貌词形态模式(根据 \citealt{japhug14ideophones}的分类)。

在动词的词条中 \ital{dir} 标注动词所搭配的趋向前缀(参考 \citealt[267-9]{jacques14linking}), 下划线字符 \_ 表示该动词可以和任意趋向连用(用于移位或具体动作动词)。不规则动词(如 \papi{ɕe} “去”或  \papi{ɣɤʑu}  “有”)的词条中附有所有无法预测的形式(特别是词干交替,第二人称/泛称等形式)。

 以下缩写词表示茶堡话动词的派生形态:

\begin{itemize}
\item \caps{acaus} 反使动 
\item \caps{apass} 反被动态
\item \caps{appl} 施用态
\item \caps{autoben} 为己态
\item \caps{caus} 使动态
\item \caps{comp} 复合动词
\item \caps{deexp} 感受者泛化
\item \caps{deidph} 动词化状貌词
\item \caps{denom} 动词化名词
\item \caps{facil} 便利态
\item \caps{incorp} 名词并入
\item \caps{n.orient} 无定方向
\item \caps{pass} 被动态
\item \caps{recip} 互动态
\item \caps{refl} 反身态
\item \caps{trop} 意动态
\item \caps{vert} 回归
\end{itemize}

本词典得到了许多同行业者的修改意见和建议,笔者特别感谢龚勋、韩哲夫、彭国珍、马振国、米可、沙加尔、帅彦辰和章舒娅对初稿的仔细审阅。}
\newpage
\normalfont
\section*{Introduction}
\markboth{Introduction}{}

Ce  dictionnaire décrit le lexique de la langue japhug (\papi{kɯrɯ skɤt}), parlée dans la région de Japhug (\pcmn{茶堡}, \papi{tɕɤpʰɯ}) au district de Mbarkhams (\pcmn{马尔康县}), préfecture de Rngaba (\pcmn{阿坝州}) au Sichuan en Chine, dans les cantons de Gdongbrgyad (\pcmn{龙尔甲乡}, \papi{ʁdɯrɟɤt}), Gsarrdzong (\pcmn{沙尔宗乡},   \papi{sarndzu}) de Datshang (\pcmn{大藏乡}, \papi{tatshi}).  

Seul le dialecte de Kamnyu (\pcmn{干木鸟村}, \papi{kɤmɲɯ}) est représenté dans ce dictionnaire. Cette langue a déjà fait l'objet d'une courte description grammaticale (\citealt{jacques08}) ainsi que d'un recueil d'histoires traditionnelles (\citealt{jacques10gesar}). Un corpus de texte plus important est en cours de publication sur l'archive Pangloss (\citealt{michailovsky14pangloss}). Ces recherches ont bénéficié des travaux précédents sur d'autres langues gyalrong, en particulier la grammaire de \citet{linxr93jiarong} et les articles de Jackson Sun et de Shidanluo (\citealt{jackson00sidaba, jackson04zhuangmaoci, jackson06paisheng, jackson14morpho} etc), ainsi que du travail de Lin Youjing et Norbu sur le dialecte de Datshang (\citealt{linluo03}).

Ce dictionnaire est basé sur les matériaux recueillis à Mbarkhams par l'auteur auprès de Tshendzin (Chenzhen \pcmn{陈珍}) et Dpalcan (Baierqing \pcmn{柏尔青}) depuis juillet 2002. Une grande partie des mots, en particulier les verbes et les idéophones, sont illustrés par des exemples enregistrés représentatifs, dont certains proviennent de conversations ou d'histoires traditionnelles.

Chaque entrée du dictionnaire contient une définition en français et en chinois ainsi que la partie du discours du mot, parmi les suivantes:

\begin{itemize}
\item \textit{adv} adverbe
\item \textit{clf} classificateur
\item \textit{idph} idéophone
\item \textit{intj} interjection
\item \textit{n} nom
\item \textit{np} nom inaliénablement possédé 
\item \textit{postp} postposition
\item \textit{pro} pronom
\item \textit{vi} verbe intransitif
\item \textit{vinh} verbe intransitif sans sujet humain
\item \textit{vi-t} verbe semi-transitif
\item \textit{vs} verbe statif
\item \textit{vt} verbe transitif
\item \textit{pc(x,y)} prédicat complexe 
\end{itemize}

Les parties du discours des premiers et deuxièmes éléments des prédicats complexes sont respectivement \papi{x} et \papi{y}. Par exemple \papi{loʁ,tɯ-ʑi} \textit{pc(vs,np)} `avoir la nausée' signifie que l'élément \papi{loʁ} est morphologiquement un verbe statif, et \papi{tɯ-ʑi} un nom possédé.

Le numéro qui suit \textit{idph} correspond au patron idéophonique (selon la classification décrite dans \citealt{japhug14ideophones}).

Les verbes contiennent après \textit{dir} le ou les préfixes directionnels utilisés pour former les tiroirs verbaux (décrits dans \citealt[267-9]{jacques14linking}). Le symbole \_  est utilisé pour les verbes de mouvement, de manipulation ou d'action concrète compatibles avec les sept séries de préfixes. Pour les verbes irréguliers (tels que \papi{ɕe} `aller' ou \papi{ɣɤʑu}  `exister'), les formes non-prévisibles (thème du passé, seconde personne ou générique) sont toutes indiquées. 

Les dérivations verbales sont indiquées par les abréviations suivantes (voir \citealt{jacques12incorp, jacques13tropative, jacques14antipassive,  jacques15spontaneous}, \citeyear{jacques15causative, jacques16relatives}):

\begin{itemize}
\item \textsc{acaus} anticausatif 
\item \textsc{apass} antipassif
\item \textsc{appl} applicatif
\item \textsc{autoben} autobénéfactif-spontané
\item \textsc{caus} causatif 
\item \textsc{comp} composé
\item \textsc{deexp} dé-expérienceur
\item \textsc{deidph} déidéophonique
\item \textsc{denom} dénominal
\item \textsc{facil} facilitatif
\item \textsc{incorp} incorporation
\item \textsc{n.orient} action non-orientée
\item \textsc{pass} passif
\item \textsc{recip} réciproque
\item \textsc{refl} réfléchi
\item \textsc{trop} tropatif
\item \textsc{vert} vertitif
\end{itemize}

Ce dictionnaire a bénéficié des corrections de nombreux collègues et étudiants, en particulier Anton Antonov, Giorgio Arcodia, Gong Xun, Peng Guozhen, Zev Handel, Alexis Michaud, Laurent Sagart, Shuai Yanchen et Zhang Shuya. Je remercie également Rémy Bonnet, Céline Buret, Séverine Guillaume et Thomas Pellard pour leur aide avec les scripts de conversions du format MDF vers \LaTeX.

Ce travail a été financé par le projet ANR HimalCo  (ANR-12-CORP-0006) et est en relation avec le projet de recherche LR-4.11 ‘‘Automatic Paradigm Generation and Language Description’’ du Labex EFL (fondé par l'ANR/CGI).

\newpage
 \section*{Ordre alphabétique \pcmn{字母顺序}}
 
 \papi{a}  \papi{b}  \papi{β}  \papi{c}  \papi{ɕ}  \papi{d}  \papi{e}  \papi{ɤ}  \papi{f}  \papi{g}  \papi{ɣ}  \papi{ɢ}  \papi{h}  \papi{i}  \papi{j}  \papi{ɟ}  \papi{k}  \papi{l}  \papi{ɬ}  \papi{m}  \papi{n}  \papi{ɲ}  \papi{ŋ}  \papi{ɴ}  \papi{o}  \papi{p}  \papi{q}  \papi{r}  \papi{ʁ}  \papi{s}  \papi{ʂ}  \papi{t}  \papi{u}  \papi{ɯ}  \papi{w}  \papi{x}  \papi{χ}  \papi{y}  \papi{z}  \papi{ʐ}  \papi{ʑ} 		

%\bibliographystyle{unified}
%\bibliography{bibliogj}
%\newpage
\begin{thebibliography}{17}
\markboth{}{}
\providecommand{\natexlab}[1]{#1}
\providecommand{\url}[1]{#1}
\providecommand{\urlprefix}{}
\expandafter\ifx\csname urlstyle\endcsname\relax
  \providecommand{\doi}[1]{doi:\discretionary{}{}{}#1}\else
  \providecommand{\doi}{doi:\discretionary{}{}{}\begingroup
  \urlstyle{rm}\Url}\fi

\bibitem[{Jacques(2008)}]{jacques08}
Jacques, Guillaume. 2008.
\newblock \emph{\pcmn{嘉絨語研究} {J}iāróngy\v{u} yánjiū ({S}tudy on the
  {R}gyalrong language)}.
\newblock Beijing: Minzu chubanshe.

\bibitem[{Jacques(2012)}]{jacques12incorp}
Jacques, Guillaume. 2012.
\newblock {F}rom denominal derivation to incorporation.
\newblock \emph{Lingua} 122(11). 1207--1231.

\bibitem[{Jacques(2013{\natexlab{a}})}]{jacques13tropative}
Jacques, Guillaume. 2013{\natexlab{a}}.
\newblock {A}pplicative and tropative derivations in {J}aphug {R}gyalrong.
\newblock \emph{Linguistics of the Tibeto-Burman Area} 36(2). 1--13.

\bibitem[{Jacques(2013{\natexlab{b}})}]{japhug14ideophones}
Jacques, Guillaume. 2013{\natexlab{b}}.
\newblock {I}deophones in {J}aphug {R}gyalrong.
\newblock \emph{Anthropological Linguistics} 55(3). 256--287.

\bibitem[{Jacques(2014{\natexlab{a}})}]{jacques14linking}
Jacques, Guillaume. 2014{\natexlab{a}}.
\newblock {C}lause linking in {J}aphug {R}gyalrong.
\newblock \emph{Linguistics of the Tibeto-Burman Area} 37(2). 263--327.

\bibitem[{Jacques(2014{\natexlab{b}})}]{jacques14antipassive}
Jacques, Guillaume. 2014{\natexlab{b}}.
\newblock {D}enominal affixes as sources of antipassive markers in {J}aphug
  {R}gyalrong.
\newblock \emph{Lingua} 138. 1--22.

\bibitem[{Jacques(2015{\natexlab{a}})}]{jacques15causative}
Jacques, Guillaume. 2015{\natexlab{a}}.
\newblock {T}he origin of the causative prefix in {R}gyalrong languages and its
  implication for proto-{S}ino-{T}ibetan reconstruction.
\newblock \emph{Folia Linguistica Historica} 36(1). 165–198.

\bibitem[{Jacques(2015{\natexlab{b}})}]{jacques15spontaneous}
Jacques, Guillaume. 2015{\natexlab{b}}.
\newblock {T}he spontaneous-autobenefactive prefix in {J}aphug {R}gyalrong.
\newblock \emph{Linguistics of the Tibeto Burman Area} 38(2). 271–291.

\bibitem[{Jacques(2016)}]{jacques16relatives}
Jacques, Guillaume. 2016.
\newblock {S}ubjects, objects and relativization in {J}aphug.
\newblock \emph{Journal of Chinese Linguistics} 44(1). 1--28.

\bibitem[{Jacques \& Chen(2010)}]{jacques10gesar}
Jacques, Guillaume \& Zhen Chen. 2010.
\newblock \emph{{U}ne version rgyalrong de l'épopée de {G}esar}.
\newblock Osaka: National Museum of Ethnology.

\bibitem[{Lín(1993)}]{linxr93jiarong}
Lín, Xiàngróng. 1993.
\newblock \emph{\pcmn{嘉戎語研究} {J}iāróngy\v{u} yánjiū ({A} study on
  the {R}gyalrong language)}.
\newblock \pcmn{成都:四川民族出版社} Chéngdū: Sìchuān mínzú
  chūb\v{a}nshè.
\newblock (\pcmn{林向榮}).

\bibitem[{Lín \& Luó\v{e}rw\v{u}(2003)}]{linluo03}
Lín, Yòujīng \& Luó\v{e}rw\v{u}. 2003.
\newblock \pcmn{茶堡嘉戎語大藏話的趨向前綴及動詞詞幹變化}
  {C}hábǎo jiāróngyǔ {D}àzànghuà de qūxiàng qiánzhuì
  jí dòngcí cígàn biànhuà ({T}he directional prefixes and verb
  stem alternations in the {D}atshang dialect of {J}aphug).
\newblock \emph{\pcmn{民族語文} Mínzú y\v{u}wén} 4. 19--29.

\bibitem[{Michailovsky et~al.(2014)Michailovsky, Mazaudon, Michaud, Guillaume,
  François \& Adamou}]{michailovsky14pangloss}
Michailovsky, Boyd, Martine Mazaudon, Alexis Michaud, Séverine Guillaume,
  Alexandre François \& Evangelia Adamou. 2014.
\newblock {D}ocumenting and researching endangered languages: the {P}angloss
  {C}ollection.
\newblock \emph{Language Documentation and Conservation} 8. 119–135.

\bibitem[{Sun(2000)}]{jackson00sidaba}
Sun, Jackson T.-S. 2000.
\newblock {P}arallelisms in the {V}erb {M}orphology of {S}idaba r{G}yalrong and
  {L}avrung in r{G}yalrongic.
\newblock \emph{Language and Linguistics} 1(1). 161--190.

\bibitem[{Sun(2006)}]{jackson06paisheng}
Sun, Jackson T.-S. 2006.
\newblock \pcmn{嘉戎語動詞的派生形態} {J}iāróngy\v{ǔ} dòngcí de
  pàishēng xíngtài ({D}erivational morphology in the {R}gyalrong verb).
\newblock \emph{Minzu yuwen \pcmn{民族語文}} 4(3). 3--14.

\bibitem[{Sun(2014)}]{jackson14morpho}
Sun, Jackson T.-S. 2014.
\newblock {S}ino-{T}ibetan: {R}gyalrong.
\newblock In Rochelle Lieber \& Pavol Štekauer (eds.), \emph{{T}he {O}xford
  {H}andbook of {D}erivational {M}orphology}, 630--650. Oxford: Oxford
  University Press.

\bibitem[{Sun \& Shidanluo(2004)}]{jackson04zhuangmaoci}
Sun, Jackson T.-S. \& Shidanluo. 2004.
\newblock \pcmn{草登嘉戎語的狀貌詞} {C}\v{a}odēng {J}iāróngy\v{u} de
  zhuàngmàocí ({T}he ideophones in {T}shobdun {R}gyalrong).
\newblock \emph{Minzu yuwen \pcmn{民族語文}} 5. 1--11.

\end{thebibliography}

\cleardoublepage
\pagenumbering{arabic}
\setmainfont[Mapping=tex-text,Numbers=OldStyle,Ligatures=Common]{Charis SIL}
\begin{multicols}{2}
\lhead{\firstmark}
\rhead{\botmark}
\newpage\caractère{a}

\begin{entrée}
\vedette{\hypertarget{Ⓔaboʁboʁ}{\papi{ aboʁboʁ}}}\markboth{aboʁboʁ}{}\classe{vs}
\paradigme{\textit{dir :} \jya kɤ-}
\begin{définition}\fra se blottir ensemble\end{définition}
\begin{définition}\cmn 集中在一起;缩成一团\end{définition}
\begin{exemple}\jya khɤzɤpɯ ra nɯ-mu ɯ-ɕki ko-k-ɤboʁboʁ-nɯ-ci ma ɲɯ-nɤndʐo-nɯ\cmn 因为冷,狗崽子们集拢在妈妈身边\end{exemple}
\begin{relation-sémantique}\confer{
\hyperlink{Ⓔnɤboʁboʁ}{\textit{ \papi{nɤboʁboʁ}}}
}\end{relation-sémantique}
\begin{relation-sémantique}\confer{
\hyperlink{Ⓔboʁ}{\textit{ \papi{boʁ}}}
}\end{relation-sémantique}\end{entrée}

\begin{entrée}
\vedette{\hypertarget{Ⓔabrɤlbrɤl}{\papi{ abrɤlbrɤl}}}\markboth{abrɤlbrɤl}{} (\variante{abrabrɤl}) \classe{vs}
\begin{définition}\fra espacé, clairsemé\end{définition}
\begin{définition}\cmn 稀疏不密(树)\end{définition}\begin{sous-entrée}
\vedette{\hypertarget{}{\papi{ sɤbrabrɤl}}}\markboth{sɤbrabrɤl}{}\classe{vt}
\paradigme{\textit{dir :} \jya nɯ-}
\begin{définition}\fra rendre espacé\end{définition}
\begin{définition}\cmn 使稀疏\end{définition}
\begin{exemple}\jya tɯ-rɣi kɤ-lɤt tɕe, ɲɯ́-wɣ-sɤbrabrɤl ra\cmn 播种的时候,要播得稀疏一点\end{exemple}
\begin{exemple}\jya laχtɕha kɤ-ta tɕe, ɲɯ́-wɣ-sɤbrabrɤl ra ma nɯ maʁ nɤ andɯndo\cmn 放东西的时候,要放开一点,不然就会粘在一起\end{exemple}
\end{sous-entrée}\end{entrée}

\begin{entrée}
\vedette{\hypertarget{Ⓔaβdɤβde}{\papi{ aβdɤβde}}}\markboth{aβdɤβde}{}\classe{vi}
\paradigme{\textit{dir :} \jya thɯ-}
\begin{définition}\fra être retardé, être reporté à plus tard\end{définition}
\begin{définition}\cmn 拖延\end{définition}
\begin{exemple}\jya kɤ-nɤma nɯ thɯ-aβdɤβde pɯ-ra\cmn 工作拖延了\end{exemple}
\begin{relation-sémantique}\confer{
\hyperlink{Ⓔβde}{\textit{ \papi{βde}}}
}\end{relation-sémantique}
\begin{sous-entrée}
\vedette{\hypertarget{}{\papi{ sɤβdɤβde}}}\markboth{sɤβdɤβde}{}\classe{vt}
\paradigme{\textit{dir :} \jya thɯ-}
\begin{définition}\ 
\begin{déclaration}\grammar{caus}\end{déclaration}\end{définition}
\begin{définition}\fra retarder\end{définition}
\begin{définition}\cmn 拖延\end{définition}
\begin{exemple}\jya kɤ-nɤma thɯ-sɤβdɤβde-t-a\cmn 我拖延了工作\end{exemple}
\begin{exemple}\jya ɯ-ma chɤ-sɤβdɤβde\cmn 他的工作拖延了时间\end{exemple}
\begin{exemple}\jya smɤn kɤ-ndza nɯ ma-thɯ-tɯ-sɤβdɤβde kɯ ɯ-khrɤt ʑo tɤ-ndze ɲɯ-ra ma nɯ mɤɕtʂa mɤ-phɤn\cmn 你要按规定的时间和数量吃药,不然就没有效果。\end{exemple}
\end{sous-entrée}\end{entrée}

\begin{entrée}
\vedette{\hypertarget{Ⓔaβdoʁβdi}{\papi{ aβdoʁβdi}}}\markboth{aβdoʁβdi}{}
\classe{vs}
\begin{définition}\fra aller bien\end{définition}
\begin{définition}\cmn 平安;健康\end{définition}
\begin{exemple}\jya iʑora ɕɤxɕo ku-oβdoʁβdi-j\cmn 我们这段时间都日子过得很平安\end{exemple}
\begin{relation-sémantique}\confer{
\hyperlink{Ⓔβdi}{\textit{ \papi{βdi}}}
}\end{relation-sémantique}
\begin{relation-sémantique}\confer{
\hyperlink{Ⓔɣɤβdi}{\textit{ \papi{ɣɤβdi}}}
}\end{relation-sémantique}\end{entrée}

\begin{entrée}
\vedette{\hypertarget{Ⓔaβdoʁdi}{\papi{ aβdoʁdi}}}\markboth{aβdoʁdi}{}
\begin{relation-sémantique}\confer{
\hyperlink{Ⓔɣɤβdi}{\textit{ \papi{ɣɤβdi}}}
}\end{relation-sémantique}\end{entrée}

\begin{entrée}
\vedette{\hypertarget{Ⓔaβraʁ}{\papi{ aβraʁ}}}\markboth{aβraʁ}{}
\begin{relation-sémantique}\confer{
\hyperlink{Ⓔβraʁ}{\textit{ \papi{βraʁ}}}
}\end{relation-sémantique}\end{entrée}

\begin{entrée}
\vedette{\hypertarget{Ⓔaβrdaβrdoŋ}{\papi{ aβrdaβrdoŋ}}}\markboth{aβrdaβrdoŋ}{}
\classe{vs}
\paradigme{\textit{dir :} \jya thɯ-}
\begin{définition}\fra vigoureux, robuste\end{définition}
\begin{définition}\cmn 粗壮\end{définition}
\begin{exemple}\jya iɕqha tɯrme nɯ kɯ-ɤβrdaβrdoŋ ci ɲɯ-ŋu\cmn 这个人很粗壮\end{exemple}
\begin{relation-sémantique}\synonyme{
\hyperlink{Ⓔjpumqa}{\textit{ \papi{jpumqa}}}
}\end{relation-sémantique}\end{entrée}

\begin{entrée}
\vedette{\hypertarget{Ⓔaβʁum}{\papi{ aβʁum}}}\markboth{aβʁum}{}
\begin{relation-sémantique}\confer{
\hyperlink{Ⓔβʁum}{\textit{ \papi{βʁum}}}
}\end{relation-sémantique}\end{entrée}

\begin{entrée}
\vedette{\hypertarget{Ⓔaβzu}{\papi{ aβzu}}}\markboth{aβzu}{}\classe{vi}
\begin{définition}\ 
\begin{déclaration}\grammar{pass}\end{déclaration}\end{définition}\acception{1}
\paradigme{\textit{dir :} \jya tɤ-}
\paradigme{\textit{dir :} \jya nɯ-}
\begin{définition}\fra devenir\end{définition}
\begin{définition}\cmn 变成\end{définition}
\begin{exemple}\jya @yangyu ɲo-k-ɤβzu-ci\cmn 土豆长大了\end{exemple}
\begin{exemple}\jya tɤɕi kɤ-tɣa to-k-ɤβzu-ci\cmn 青稞可以收割了\end{exemple}\acception{2}
\paradigme{\textit{dir :} \jya thɯ-}
\begin{définition}\fra grandir\end{définition}
\begin{définition}\cmn 长成,成熟\end{définition}
\begin{exemple}\jya kɯki tɤ-pɤtso ʁʑɯnɯ chɤ-k-ɤβzu-ci\cmn 这个孩子已经长成青年了\end{exemple}
\begin{exemple}\jya @yangyu chɤ-k-ɤβzu-ci\cmn 土豆成熟了\end{exemple}
\begin{relation-sémantique}\confer{
\hyperlink{ⒺβzuⒽ1}{\textit{ \papi{βzu1}}}
}\end{relation-sémantique}\end{entrée}

\begin{entrée}
\vedette{\hypertarget{Ⓔaβzdoʁβzdɯ}{\papi{ aβzdoʁβzdɯ}}}\markboth{aβzdoʁβzdɯ}{}
\begin{relation-sémantique}\confer{
\hyperlink{Ⓔβzdɯ}{\textit{ \papi{βzdɯ}}}
}\end{relation-sémantique}\end{entrée}

\begin{entrée}
\vedette{\hypertarget{Ⓔaβzɯrχsɯm}{\papi{ aβzɯrχsɯm}}}\markboth{aβzɯrχsɯm}{}\classe{vs}
\begin{définition}\fra triangulaire\end{définition}
\begin{définition}\cmn 三角形
\begin{déclaration} \étymologie{\papi{bzur.gsum}}\end{déclaration}\end{définition}\end{entrée}

\begin{entrée}
\vedette{\hypertarget{Ⓔaβʑɯrdu}{\papi{ aβʑɯrdu}}}\markboth{aβʑɯrdu}{}\classe{vs}
\paradigme{\textit{dir :} \jya tɤ-}
\begin{définition}\fra carré\end{définition}
\begin{définition}\cmn 四方形
\begin{déclaration} \étymologie{\papi{bʑi.rdo}}\end{déclaration}\end{définition}
\begin{exemple}\jya aβzɯrdu, ɲɯ-ɤβʑɯrdu\cmn 是四方形的\end{exemple}
\begin{exemple}\jya kɯki rdɤstaʁ ki kɯ-ɤβʑɯrdu ci ɲɯ-ŋu\cmn 这块石头是方形的\end{exemple}\end{entrée}

\begin{entrée}
\vedette{\hypertarget{Ⓔacu}{\papi{ acu}}}\markboth{acu}{}
\begin{relation-sémantique}\confer{
\hyperlink{Ⓔcu}{\textit{ \papi{cu}}}
}\end{relation-sémantique}\end{entrée}

\begin{entrée}
\vedette{\hypertarget{Ⓔacɤrlu}{\papi{ acɤrlu}}}\markboth{acɤrlu}{}
\classe{vi}
\paradigme{\textit{dir :} \jya tɤ-}
\begin{définition}\fra être mélangé\end{définition}
\begin{définition}\cmn 混合\end{définition}
\begin{exemple}\jya ɲɯ-ɤcɤrlu, to-k-ɤcɤrlu-ci\cmn 混在一起了\end{exemple}
\begin{exemple}\jya ki tɤjmɤɣ ki ɲɯ-ɤcɤrlu\cmn 这些蘑菇混在一起\end{exemple}
\begin{exemple}\jya kɯrɯ kupa tɯrme ra acɤrlu-j ʑo ɕti\cmn 我们藏族跟汉族混合在一起\end{exemple}\begin{sous-entrée}
\vedette{\hypertarget{}{\papi{ sɤcɤrlu}}}\markboth{sɤcɤrlu}{}
\begin{définition}\fra mélanger\end{définition}
\begin{définition}\cmn 把……混在一起\end{définition}
\begin{exemple}\jya kɯrɯ skɤt cho kupa skɤt tu-sɤcɤrle-a ʑo tɕe tu-ti-a ɕti\cmn 我把藏语和汉语混在一起说\end{exemple}
\begin{relation-sémantique}\synonyme{
\hyperlink{Ⓔatʂoʁloʁ}{\textit{ \papi{atʂoʁloʁ}}}
}\end{relation-sémantique}
\end{sous-entrée}\end{entrée}

\begin{entrée}
\vedette{\hypertarget{Ⓔachala}{\papi{ achala}}}\markboth{achala}{}\classe{vi}
\begin{définition}\fra capable\end{définition}
\begin{définition}\cmn 能干\end{définition}
\begin{exemple}\jya aʑo achala-a tɕe, aj tu-spe-a jɤɣ\cmn 我很能干,我会做\end{exemple}\begin{sous-entrée}
\vedette{\hypertarget{}{\papi{ achɤle}}}\markboth{achɤle}{}
\begin{exemple}\jya ɯʑo achɤle, nɤʑo tɯ-achɤle\cmn 他很能干,你很能干\end{exemple}
\begin{exemple}\jya aʑo achɤle-a tɕe, aj tu-spe-a jɤɣ\cmn 我很能干,我会做\end{exemple}
\end{sous-entrée}\end{entrée}

\begin{entrée}
\vedette{\hypertarget{Ⓔachale}{\papi{ achale}}}\markboth{achale}{}
\begin{relation-sémantique}\confer{
\hyperlink{Ⓔachala}{\textit{ \papi{achala}}}
}\end{relation-sémantique}\end{entrée}

\begin{entrée}
\vedette{\hypertarget{Ⓔachɤt}{\papi{ achɤt}}}\markboth{achɤt}{}
\classe{vs}
\paradigme{\textit{dir :} \jya tɤ-}
\begin{définition}\fra être séparé par\end{définition}
\begin{définition}\cmn 相差;相隔\end{définition}
\begin{exemple}\jya tɕiʑo tɕi-pɤrthɤβ kɯmŋu-pɤrme achɤt\end{exemple}
\begin{exemple}\jya tɕiʑo kɯmŋu-pɤrme achɤt-tɕi\cmn 我们之间相差五岁\end{exemple}
\begin{exemple}\jya jiʑo ji-kha pɤrthɤβ nɯ tɯ-sŋi tʂu jamar achɤt\cmn 我们之间相隔一天的路\end{exemple}\end{entrée}

\begin{entrée}
\vedette{\hypertarget{Ⓔachɯcha}{\papi{ achɯcha}}}\markboth{achɯcha}{}\classe{vs}
\paradigme{\textit{dir :} \ }
\begin{définition}\ 
\end{définition}
\begin{définition}\fra qui a du talent\end{définition}
\begin{définition}\cmn 有才能\end{définition}
\begin{exemple}\jya to-k-ɤchɯcha-ci\cmn 他变得有才能了\end{exemple}\end{entrée}

\begin{entrée}
\vedette{\hypertarget{Ⓔachɯrʁu}{\papi{ achɯrʁu}}}\markboth{achɯrʁu}{}
\classe{vs}
\paradigme{\textit{dir :} \jya thɯ-}
\paradigme{\textit{dir :} \jya tɤ-}
\begin{définition}\fra être froissé\end{définition}
\begin{définition}\cmn 皱着\end{définition}
\begin{exemple}\jya nɤ-ŋga to-k-ɤchɯrʁu-ci\cmn 你的衣服皱了\end{exemple}\begin{sous-entrée}
\vedette{\hypertarget{}{\papi{ sɤchɯrʁu}}}\markboth{sɤchɯrʁu}{}\classe{vt}
\paradigme{\textit{dir :} \jya tɤ-}
\begin{définition}\fra froisser\end{définition}
\begin{définition}\cmn 皱起\end{définition}
\begin{exemple}\jya ɯ-rŋa to-sɤchɯrʁu\cmn 他皱了眉毛(脸)\end{exemple}
\end{sous-entrée}\end{entrée}

\begin{entrée}
\vedette{\hypertarget{Ⓔaci}{\papi{ aci}}}\markboth{aci}{}\classe{vi}
\paradigme{\textit{dir :} \jya nɯ-}
\begin{définition}\fra être mouillé\end{définition}
\begin{définition}\cmn 湿\end{définition}
\begin{exemple}\jya ɲo-k-ɤci-ci\cmn 湿了\end{exemple}
\begin{exemple}\jya ki tɯ-mɯ kɯ pjɤ-χtɕi tɕe, ɲo-k-ɤci-ci\cmn 被雨淋湿了\end{exemple}\begin{sous-entrée}
\vedette{\hypertarget{}{\papi{ sɤci}}}\markboth{sɤci}{}\classe{vt}
\paradigme{\textit{dir :} \jya nɯ-}
\begin{définition}\ 
\begin{déclaration}\grammar{caus}\end{déclaration}\end{définition}
\begin{définition}\fra mouiller\end{définition}
\begin{définition}\cmn 弄湿\end{définition}
\begin{exemple}\jya ɯ-ŋga ɲɤ-sɤci\cmn 他把衣服弄湿了\end{exemple}
\begin{exemple}\jya tɯ-kɤrme mɤ-kɯ-sɤci\cmn 浴帽\end{exemple}
\end{sous-entrée}\end{entrée}

\begin{entrée}
\vedette{\hypertarget{Ⓔacilaj}{\papi{ acilaj}}}\markboth{acilaj}{}
\classe{vi}
\paradigme{\textit{dir :} \jya nɯ-}
\begin{définition}\fra être humide\end{définition}
\begin{définition}\cmn 湿漉漉\end{définition}
\begin{exemple}\jya kɯki kɤ-ɕkho ɲɯ-ra ma ɲɯ-ɤcilaj\cmn 这个东西很湿,要晒一下\end{exemple}
\begin{exemple}\jya kɯki tɯ-ŋga ki ɲɯ-ɤcilaj tɕe, chɯ́-wɣ-ɕkho ɲɯ-ntshi\cmn 这件衣服很湿,只好晒一下\end{exemple}
\begin{exemple}\jya a-βri acilaj\cmn 我身上很湿\end{exemple}
\begin{exemple}\jya a-jaʁ ɲɯ-ɤcilaj\cmn 我的手很湿\end{exemple}\end{entrée}

\begin{entrée}
\vedette{\hypertarget{Ⓔacɯfkri}{\papi{ acɯfkri}}}\markboth{acɯfkri}{}\classe{vs}
\begin{définition}\fra être sale, humide et en désordre\end{définition}
\begin{définition}\cmn 又脏又湿又乱\end{définition}\end{entrée}

\begin{entrée}
\vedette{\hypertarget{Ⓔaɕɤl}{\papi{ aɕɤl}}}\markboth{aɕɤl}{}
\classe{vi}
\paradigme{\textit{dir :} \jya nɯ-}
\begin{définition}\fra souffrir de la cataracte\end{définition}
\begin{définition}\cmn 患有白内障【白眼病】\end{définition}
\begin{exemple}\jya ɯ-mɲaʁ ɲo-k-ɤɕɤl-ci\cmn 他得了白内障\end{exemple}
\begin{exemple}\jya ɲo-k-ɤɕɤl-ci tɕe mɯ́j-mtɤm\cmn 得了白内障,他看不见了\end{exemple}\end{entrée}

\begin{entrée}
\vedette{\hypertarget{Ⓔaɕɤrɣi}{\papi{ aɕɤrɣi}}}\markboth{aɕɤrɣi}{}
\classe{vs}
\begin{définition}\fra être rapide\end{définition}
\begin{définition}\cmn 快,迅速\end{définition}
\begin{exemple}\jya ɯ-rju ɲɯ-ɤɕɤrɣi\cmn 他话说得快\end{exemple}
\begin{exemple}\jya ɯ-mɢla ɲɯ-ɤɕɤrɣi\cmn 他步子走得快\end{exemple}\begin{sous-entrée}
\vedette{\hypertarget{}{\papi{ sɤɕɤrɣi}}}\markboth{sɤɕɤrɣi}{}\classe{vt}
\paradigme{\textit{dir :} \jya tɤ-}
\begin{définition}\fra faire rapidement\end{définition}
\begin{définition}\cmn 做得快\end{définition}
\begin{exemple}\jya ɯʑo kɯ tɯ-rju kɤ-ti ɲɯ-sɤɕɤrɣi\cmn 他说话说得快\end{exemple}
\end{sous-entrée}\end{entrée}

\begin{entrée}
\vedette{\hypertarget{Ⓔaɕɣa}{\papi{ aɕɣa}}}\markboth{aɕɣa}{}
\classe{vi}
\begin{définition}\fra être du même âge\end{définition}
\begin{définition}\cmn 同年\end{définition}
\begin{exemple}\jya tɕiʑo kɯ-ɤɕɣa ŋu-tɕi\cmn 我们俩同岁\end{exemple}
\begin{exemple}\jya ɲɯ-tɯ-ɤɕɣa-ndʑi\cmn 你们俩同岁\end{exemple}
\begin{exemple}\jya nɤj nɤ-mu cho aʑo ni aɕɣa-tɕi\cmn 我跟你母亲同岁\end{exemple}
\begin{relation-sémantique}\confer{
\hyperlink{Ⓔtɯ-ɕɣa}{\textit{ \papi{tɯ-ɕɣa}}}
}\end{relation-sémantique}\end{entrée}

\begin{entrée}
\vedette{\hypertarget{ⒺaɕiⒽ1}{\papi{ aɕi}}}\markboth{aɕi}{}\homonyme{1}\classe{intj}
\begin{définition}\fra interjection qui exprime que l'on revient sur ce que l'on vient de dire\end{définition}
\begin{définition}\cmn 对刚才说的话表示反悔\end{définition}
\begin{exemple}\jya aɕi, nɯ ma-tɤ-tɯ-ste\cmn 算了,你不要那么做\end{exemple}
\begin{exemple}\jya aɕi ma-jɤ-tɯ-ɕe tɕe, kɤ-nɯ-rɤʑi!\cmn 算了,不要走了,留在这里吧\end{exemple}
\begin{exemple}\jya aɕi, a-mɤ-tɤ-tɯ-qhe je\cmn (我收回刚才说的话),你不要放在心上\end{exemple}\end{entrée}

\begin{entrée}
\vedette{\hypertarget{ⒺaɕiⒽ2}{\papi{ aɕi}}}\markboth{aɕi}{}\homonyme{2}
\classe{vs}
\begin{définition}\fra être mélangé dans\end{définition}
\begin{définition}\cmn 掺在一起;和在一起\end{définition}
\begin{exemple}\jya ki smɤn ki ɯ-ŋgɯ pɤjmu aɕi tɕe aɣɯsmɤn\cmn 贝母是这种药的其中一个成分,这样药才有效\end{exemple}
\begin{relation-sémantique}\synonyme{
\hyperlink{Ⓔacu}{\textit{ \papi{acu}}}
}\end{relation-sémantique}\begin{sous-entrée}
\vedette{\hypertarget{}{\papi{ sɤɕi}}}\markboth{sɤɕi}{}\classe{vt}
\paradigme{\textit{dir :} \jya pɯ-}
\begin{définition}\fra ajouter dans\end{définition}
\begin{définition}\cmn 加进
\begin{déclaration}\grammar{caus}\end{déclaration}\end{définition}
\begin{relation-sémantique}\synonyme{
\hyperlink{Ⓔcu}{\textit{ \papi{cu}}}
}\end{relation-sémantique}
\end{sous-entrée}\end{entrée}

\begin{entrée}
\vedette{\hypertarget{Ⓔaɕkala}{\papi{ aɕkala}}}\markboth{aɕkala}{}\classe{vs}
\paradigme{\textit{dir :} \jya tɤ-}
\begin{définition}\ 
\begin{déclaration}\grammar{denom}\end{déclaration}\end{définition}
\begin{définition}\fra être boiteux\end{définition}
\begin{définition}\cmn 跛脚\end{définition}
\begin{exemple}\jya ɯʑo ɲɯ-ɤɕkala\cmn 他跛脚\end{exemple}
\begin{exemple}\jya nɯ ɕɯŋgɯ mɯ-pɯ-aɕkala ri, tham tɕa to-k-ɤɕkala-ci\cmn 他以前不跛脚,现在就跛脚了\end{exemple}
\begin{relation-sémantique}\synonyme{
\hyperlink{Ⓔaʑɤwu}{\textit{ \papi{aʑɤwu}}}
}\end{relation-sémantique}
\begin{relation-sémantique}\confer{
\hyperlink{Ⓔɕkala}{\textit{ \papi{ɕkala}}}
}\end{relation-sémantique}\end{entrée}

\begin{entrée}
\vedette{\hypertarget{Ⓔaɕoʁri}{\papi{ aɕoʁri}}}\markboth{aɕoʁri}{}\classe{vi}
\paradigme{\textit{dir :} \jya tɤ-}
\begin{définition}\fra aller et venir\end{définition}
\begin{définition}\cmn 来回走动穿梭\end{définition}\end{entrée}

\begin{entrée}
\vedette{\hypertarget{Ⓔaɕoχɕi}{\papi{ aɕoχɕi}}}\markboth{aɕoχɕi}{}\classe{vi}
\begin{définition}\fra inspirer et expirer\end{définition}
\begin{définition}\cmn 吸气和呼气\end{définition}\begin{sous-entrée}
\vedette{\hypertarget{}{\papi{ sɤɕoχɕi}}}\markboth{sɤɕoχɕi}{}\classe{vt}
\paradigme{\textit{dir :} \jya tɤ-}
\begin{définition}\fra inspirer et expirer\end{définition}
\begin{définition}\cmn 吸气和呼气\end{définition}
\begin{exemple}\jya ɯ-sŋɯro tɤ-sɤɕoχɕi\cmn 他吸了气又呼了气\end{exemple}
\begin{exemple}\jya ɯ-sŋɯro ɲɯ-ɤsɯ-sɤɕoχɕi\cmn 他在呼吸\end{exemple}
\end{sous-entrée}\end{entrée}

\begin{entrée}
\vedette{\hypertarget{Ⓔaɕpala}{\papi{ aɕpala}}}\markboth{aɕpala}{}\classe{vi}
\begin{définition}\fra ayant des mouvement alertes\end{définition}
\begin{définition}\cmn 动作灵活(小伙子)\end{définition}
\begin{exemple}\jya kɯ-ɤɕpala ci ɲɯ-ŋu\cmn 他动作很灵活\end{exemple}
\begin{exemple}\jya jiɕqha tɯrme nɯ mɤ-kɯ-ɤɕpala ci rʁoʁrʁoʁ ɲɯ-ɕti\cmn 那个人动作不灵活\end{exemple}
\begin{relation-sémantique}\synonyme{
\hyperlink{Ⓔaphala}{\textit{ \papi{aphala}}}
}\end{relation-sémantique}\end{entrée}

\begin{entrée}
\vedette{\hypertarget{Ⓔaɕprɯm}{\papi{ aɕprɯm}}}\markboth{aɕprɯm}{}\classe{vi}
\paradigme{\textit{dir :} \jya kɤ-}
\begin{définition}\fra être mal raccommodé (habits, chaussure)\end{définition}
\begin{définition}\cmn 补得不平整,皱着的,不舒展(衣服、鞋子)\end{définition}
\begin{exemple}\jya ko-k-ɤɕprɯm-ci\cmn 补得不好\end{exemple}
\begin{exemple}\jya ɯ-ɕphɤt mɯ-ko-ɣɤβdi tɕe, ɲɯ-ɤɕprɯm\cmn 补丁没有补好,所以不平整\end{exemple}
\begin{exemple}\jya kɤ-ɣɤβdi mɯ́j-khɯ tɕe, aɕprɯm\cmn (衣服)没有能修补,所以不平整\end{exemple}\begin{sous-entrée}
\vedette{\hypertarget{}{\papi{ aɕprɯmtsɯ}}}\markboth{aɕprɯmtsɯ}{}\classe{vi}
\paradigme{\textit{dir :} \jya kɤ-}
\begin{définition}\fra être mal raccommodé (habits, chaussure)\end{définition}
\begin{définition}\cmn 补得不好;皱着的(衣服、鞋子)\end{définition}
\begin{exemple}\jya ko-k-ɤɕprɯmtsɯ-ci\cmn 补得不好\end{exemple}
\begin{exemple}\jya tɤ-ɕphɤt mɯ-ko-βdi tɕe ɲɯ-ɤɕprɯmtsɯ\cmn 补丁没有补好,所以不平整\end{exemple}
\begin{exemple}\jya pɤjkhu mɯ́j-mna tɕe ɲɯ-ɤɕprɯmtsɯ\cmn (伤口)还没好,没有愈合\end{exemple}
\end{sous-entrée}\begin{sous-entrée}
\vedette{\hypertarget{}{\papi{ sɤɕprɯm}}}\markboth{sɤɕprɯm}{}\classe{vt}
\paradigme{\textit{dir :} \jya kɤ-}
\begin{définition}\fra coudre à l'à peu près\end{définition}
\begin{définition}\cmn 将就缝\end{définition}
\begin{exemple}\jya ɯ-ŋga ka-sɤɕprɯm\cmn 他把衣服将就缝了一下\end{exemple}
\end{sous-entrée}\end{entrée}

\begin{entrée}
\vedette{\hypertarget{Ⓔaɕprɯmtsɯ}{\papi{ aɕprɯmtsɯ}}}\markboth{aɕprɯmtsɯ}{}
\begin{relation-sémantique}\confer{
\hyperlink{Ⓔaɕprɯm}{\textit{ \papi{aɕprɯm}}}
}\end{relation-sémantique}\end{entrée}

\begin{entrée}
\vedette{\hypertarget{Ⓔaɕpɯɕpa}{\papi{ aɕpɯɕpa}}}\markboth{aɕpɯɕpa}{}\classe{vs}
\paradigme{\textit{dir :} \jya nɯ-}
\paradigme{\textit{dir :} \jya kɤ-}
\begin{définition}\fra être plat\end{définition}
\begin{définition}\cmn 瘪,扁\end{définition}
\begin{exemple}\jya ɲo-k-ɤɕpɯɕpa-ci, kɤ-aɕpɯɕpa\cmn 已经压扁了\end{exemple}
\begin{exemple}\jya ɯ-xtu ɲɤ-k-ɤɕpɯɕpa-ci\cmn 他肚子扁了\end{exemple}\begin{sous-entrée}
\vedette{\hypertarget{}{\papi{ sɤɕpɯɕpa}}}\markboth{sɤɕpɯɕpa}{}\classe{vt}
\paradigme{\textit{dir :} \jya nɯ-}
\begin{définition}\ 
\begin{déclaration}\grammar{caus}\end{déclaration}\end{définition}
\begin{définition}\fra aplatir\end{définition}
\begin{définition}\cmn 弄扁\end{définition}
\begin{exemple}\jya nɯ-sɤɕpɯɕpa-t-a\cmn 我把它弄扁了\end{exemple}
\end{sous-entrée}\end{entrée}

\begin{entrée}
\vedette{\hypertarget{Ⓔaɕqhe}{\papi{ aɕqhe}}}\markboth{aɕqhe}{}
\classe{vi}
\paradigme{\textit{dir :} \jya tɤ-}
\begin{définition}\fra tousser\end{définition}
\begin{définition}\cmn 咳嗽\end{définition}
\begin{exemple}\jya ɲɯ-tɯ-ɤɕqhe, ɯʑo ɲɯ-ɤɕqhe\cmn 你咳嗽,他咳嗽\end{exemple}
\begin{exemple}\jya to-k-ɤɕqhe-ci\cmn 他咳嗽了\end{exemple}
\begin{exemple}\jya aʑo ku-oɕqhe-a\cmn 我正在咳嗽\end{exemple}
\begin{exemple}\jya aʑo sɲikuku ʑo tu-oɕqhe-a ŋu\cmn 我天天咳嗽\end{exemple}
\begin{exemple}\jya ɯ-ɲɯ-tɯ-ɤɕqhe ?\cmn 你咳嗽吗?\end{exemple}
\begin{exemple}\jya a-mɤ-tu-sɤ-ɕqhɯ-ɕqhe, smɤn ci tɤ-ndza-t-a\cmn 为了不咳嗽,我吃了药\end{exemple}
\begin{relation-sémantique}\confer{
\hyperlink{Ⓔtɤ-ɕqhe}{\textit{ \papi{tɤ-ɕqhe}}}
}\end{relation-sémantique}\begin{sous-entrée}
\vedette{\hypertarget{}{\papi{ sɤɕqhe}}}\markboth{sɤɕqhe}{}\classe{vt}
\begin{définition}\ 
\begin{déclaration}\grammar{caus}\end{déclaration}\end{définition}
\begin{définition}\fra faire tousser\end{définition}
\begin{définition}\cmn 令人咳嗽\end{définition}
\begin{exemple}\jya aʑo a-rtshɤz ɯ-ŋgɯ thɯ-ari tɕe tɤ́-wɣ-sɤɕqhe-a\cmn (茶)进到肺里了,让我咳了起来\end{exemple}
\begin{exemple}\jya tɤ-khɯ ɯ-di pjɯ-tɯ-mtshɤm tɕe, tú-wɣ-sɤɕqhe-a ŋu\cmn 我一闻到烟味就会咳嗽\end{exemple}
\end{sous-entrée}\end{entrée}

\begin{entrée}
\vedette{\hypertarget{Ⓔaɕqhlu}{\papi{ aɕqhlu}}}\markboth{aɕqhlu}{}\classe{vs}
\begin{définition}\fra concave\end{définition}
\begin{définition}\cmn 凹\end{définition}
\begin{exemple}\jya pjɤ-k-ɤɕqhlu-ci\cmn 凹进去了\end{exemple}
\begin{relation-sémantique}\synonyme{
\hyperlink{Ⓔaχchowolu}{\textit{ \papi{aχchowolu}}}
}\end{relation-sémantique}
\begin{relation-sémantique}\synonyme{
\hyperlink{Ⓔaʁloʁlu}{\textit{ \papi{aʁloʁlu}}}
}\end{relation-sémantique}
\begin{relation-sémantique}\synonyme{
\hyperlink{Ⓔaqhowolu}{\textit{ \papi{aqhowolu}}}
}\end{relation-sémantique}
\begin{relation-sémantique}\synonyme{
\hyperlink{Ⓔasqhlu}{\textit{ \papi{asqhlu}}}
}\end{relation-sémantique}
\begin{relation-sémantique}\synonyme{
\hyperlink{Ⓔarɴɢlɯm}{\textit{ \papi{arɴɢlɯm}}}
}\end{relation-sémantique}\end{entrée}

\begin{entrée}
\vedette{\hypertarget{Ⓔaɕquwa}{\papi{ aɕquwa}}}\markboth{aɕquwa}{}\classe{vs}
\paradigme{\textit{dir :} \jya tɤ-}
\begin{définition}\ 
\begin{déclaration}\grammar{denom}\end{déclaration}\end{définition}
\begin{définition}\fra être aveugle\end{définition}
\begin{définition}\cmn 瞎\end{définition}
\begin{exemple}\jya ɯ-mɲaʁ ntsi aɕquwa (ɯ-mɲaʁ ntsi mɤ-pe)\cmn 他一只眼睛瞎了\end{exemple}
\begin{relation-sémantique}\confer{
\hyperlink{Ⓔɕquwa}{\textit{ \papi{ɕquwa}}}
}\end{relation-sémantique}\end{entrée}

\begin{entrée}
\vedette{\hypertarget{Ⓔaɕtɯɕte}{\papi{ aɕtɯɕte}}}\markboth{aɕtɯɕte}{}
\begin{relation-sémantique}\confer{
\hyperlink{Ⓔɕte}{\textit{ \papi{ɕte}}}
}\end{relation-sémantique}\end{entrée}

\begin{entrée}
\vedette{\hypertarget{Ⓔadaʁlu}{\papi{ adaʁlu}}}\markboth{adaʁlu}{}\classe{vs}
\paradigme{\textit{dir :} \jya thɯ-}
\begin{définition}\fra avec du noir et du blanc (mélangés)\end{définition}
\begin{définition}\cmn 黑色和白色混在一起\end{définition}
\begin{exemple}\jya chɤ-k-ɤdaʁlu-ci\end{exemple}\end{entrée}

\begin{entrée}
\vedette{\hypertarget{Ⓔadrɤt}{\papi{ adrɤt}}}\markboth{adrɤt}{}\classe{vs}
\paradigme{\textit{dir :} \jya pɯ-}
\begin{définition}\fra en désordre\end{définition}
\begin{définition}\cmn 凌乱\end{définition}
\begin{exemple}\jya a-mɤ-pɯ-ɤdrɤt tɕe tɤ-rɤwum\cmn (东西)不要这么乱,收拾一下\end{exemple}
\begin{relation-sémantique}\synonyme{
\hyperlink{Ⓔaphɤlɤjɤt}{\textit{ \papi{aphɤlɤjɤt}}}
}\end{relation-sémantique}\begin{sous-entrée}
\vedette{\hypertarget{}{\papi{ sɤdrɤt}}}\markboth{sɤdrɤt}{}\classe{vt}
\paradigme{\textit{dir :} \jya thɯ-}
\paradigme{\textit{dir :} \jya lɤ-}
\begin{définition}\fra mettre le désordre\end{définition}
\begin{définition}\cmn 乱放\end{définition}
\begin{exemple}\jya tɤ-fkɯm ɯ-ŋgɯ a-pɯ-ɤrku ma ma-lɤ-tɯ-sɤdrɤt\cmn 东西在袋子里放着,不要拿出来到处乱放\end{exemple}
\begin{relation-sémantique}\confer{
\hyperlink{Ⓔnɤqadrɤt}{\textit{ \papi{nɤqadrɤt}}}
}\end{relation-sémantique}
\end{sous-entrée}\end{entrée}

\begin{entrée}
\vedette{\hypertarget{Ⓔadʑɯgli}{\papi{ adʑɯgli}}}\markboth{adʑɯgli}{}
\classe{vi}
\paradigme{\textit{dir :} \jya nɯ-}
\begin{définition}\fra craquer les uns les autres (os)\end{définition}
\begin{définition}\cmn (骨头)互相摩擦发出声音\end{définition}
\begin{exemple}\jya a-mke ɯ-ɕɤrɯ ɲɯ-ɤndʑɯgli ɲɯ-ŋu tɕe glinɤgli ʑo ɲɯ-ti\cmn 我脖子的骨头互相摩擦发出咯咯声\end{exemple}
\begin{relation-sémantique}\confer{
\hyperlink{Ⓔglinɤgli}{\textit{ \papi{glinɤgli}}}
}\end{relation-sémantique}\end{entrée}

\begin{entrée}
\vedette{\hypertarget{Ⓔafɕu}{\papi{ afɕu}}}\markboth{afɕu}{}
\classe{vi}\acception{1}
\paradigme{\textit{dir :} \jya nɯ-}
\begin{définition}\fra se refroidir (objet)\end{définition}
\begin{définition}\cmn 冷却(东西)\end{définition}
\begin{exemple}\jya tʂha nɯ-afɕu kóʁmɯz kú-wɣ-tshi ra\cmn 茶要凉一点喝\end{exemple}
\begin{exemple}\jya ki ɲɯ-sɤɕke tɕe, a-nɯ-ɤfɕu ɲɯ-ntshi\cmn 很烫,要先冷却一下\end{exemple}
\begin{exemple}\jya ki kɤ-tshi ɲɯ-jɤɣ ma ɲɤ-k-ɤfɕu-ci\cmn 这个已经凉了,可以喝\end{exemple}
\begin{exemple}\jya mɯ-ɲɤ-sɤɕke, ɲo-k-ɤfɕu-ci\cmn 已经不烫了,冷却了\end{exemple}
\begin{exemple}\jya a-nɯ-ɤfɕu ku-nɤjam-a\cmn 我在等它冷却一下\end{exemple}
\begin{exemple}\jya nɯkɤcu jiɕqha tɤ-ala pɯ-ŋu ri, tham ɲɤ-k-ɤfɕu-ci\cmn 刚才那个在沸腾,现在冷了\end{exemple}\acception{2}
\paradigme{\textit{dir :} \jya tɤ-}
\begin{définition}\fra se reposer\end{définition}
\begin{définition}\cmn 休息,放松,轻松\end{définition}
\begin{exemple}\jya a-tɤɣɲat tɤ-afɕu\cmn 我轻松了(休息好了)\end{exemple}
\begin{exemple}\jya tɤ-afɕu-a\cmn 我轻松了(休息好了)\end{exemple}
\begin{exemple}\jya nɯ kɯnɤ ʑo ɯ-ro mɯ-pjɤ-k-ɤfɕu-ci\cmn 这样他都不解恨\end{exemple}\begin{sous-entrée}
\vedette{\hypertarget{}{\papi{ sɤfɕu}}}\markboth{sɤfɕu}{}\classe{vt}
\begin{définition}\ 
\begin{déclaration}\grammar{caus}\end{déclaration}\end{définition}\acception{1}
\paradigme{\textit{dir :} \jya nɯ-}
\begin{définition}\fra faire refroidir\end{définition}
\begin{définition}\cmn 使冷却\end{définition}
\begin{exemple}\jya tʂha nɯ-sɤfɕe\cmn 让茶冷却一下\end{exemple}\acception{2}
\paradigme{\textit{dir :} \jya tɤ-}
\begin{définition}\fra se reposer\end{définition}
\begin{définition}\cmn 休息,放松,轻松\end{définition}
\begin{exemple}\jya nɤ-tɤɣɲat tɤ-sɤfɕe\cmn 你放松一下\end{exemple}
\begin{exemple}\jya nɤ-mgɯr ci tɤ-sɤfɕe\cmn 你把背部放松一下(靠一下)\end{exemple}
\end{sous-entrée}\begin{sous-entrée}
\vedette{\hypertarget{}{\papi{ ʑɣɤsɤfɕu}}}\markboth{ʑɣɤsɤfɕu}{}\classe{vi}
\paradigme{\textit{dir :} \jya tɤ-}
\begin{définition}\ 
\begin{déclaration}\grammar{caus}\end{déclaration}
\begin{déclaration}\grammar{refl}\end{déclaration}\end{définition}
\begin{définition}\fra se reposer un peu\end{définition}
\begin{définition}\cmn 歇一会\end{définition}
\begin{exemple}\jya pjɤ-tɯ-ɴqa tɕe, tɤ-ʑɣɤsɤfɕu\cmn 你累了,休息一下\end{exemple}
\end{sous-entrée}\end{entrée}

\begin{entrée}
\vedette{\hypertarget{Ⓔafɕɤra}{\papi{ afɕɤra}}}\markboth{afɕɤra}{}\classe{vs}
\paradigme{\textit{dir :} \jya nɯ-}
\begin{définition}\fra être connu publiquement\end{définition}
\begin{définition}\cmn 公开,传开\end{définition}
\begin{exemple}\jya ki tɯ-tɕha ki ɲɤ-k-ɤfɕɤra\cmn 这个消息已经传开了\end{exemple}
\begin{exemple}\jya mɤ-kɯ-ɤfɕɤra nɯ ɕɯ tso?\cmn 如果不公开的话,谁会知道?\end{exemple}
\begin{relation-sémantique}\confer{
\hyperlink{Ⓔsɤfɕɤra}{\textit{ \papi{sɤfɕɤra}}}
}\end{relation-sémantique}\end{entrée}

\begin{entrée}
\vedette{\hypertarget{Ⓔafɕɤt}{\papi{ afɕɤt}}}\markboth{afɕɤt}{}
\begin{relation-sémantique}\confer{
\hyperlink{ⒺfɕɤtⒽ1}{\textit{ \papi{fɕɤt1}}}
}\end{relation-sémantique}\end{entrée}

\begin{entrée}
\vedette{\hypertarget{Ⓔafkrɤχsɤl}{\papi{ afkrɤχsɤl}}}\markboth{afkrɤχsɤl}{}\classe{vi}
\paradigme{\textit{dir :} \jya tɤ-}
\begin{définition}\fra être clair (voir)\end{définition}
\begin{définition}\cmn 清晰;明显
\begin{déclaration} \étymologie{\papi{bkra.gsal}}\end{déclaration}\end{définition}
\begin{exemple}\jya kɯ-ɲaʁ kɯ-wɣrum ɲɯ-ɤfkrɤχsɤl\cmn 黑色和红色的区别很分明\end{exemple}
\begin{exemple}\jya to-k-ɤfkrɤχsɤl-ci\cmn 变清晰了\end{exemple}\end{entrée}

\begin{entrée}
\vedette{\hypertarget{Ⓔafsuja}{\papi{ afsuja}}}\markboth{afsuja}{}\classe{vs}
\begin{définition}\fra être de même taille\end{définition}
\begin{définition}\cmn 一样大;一样长;平等\end{définition}
\begin{exemple}\jya tɕiʑo afsuja-tɕi / tɕi-tɯ-mbro afsuja\cmn 我们长得一样高\end{exemple}
\begin{relation-sémantique}\confer{
\hyperlink{Ⓔsɤfsuja}{\textit{ \papi{sɤfsuja}}}
}\end{relation-sémantique}\end{entrée}

\begin{entrée}
\vedette{\hypertarget{Ⓔafsujɯja}{\papi{ afsujɯja}}}\markboth{afsujɯja}{}\classe{vi}
\begin{définition}\fra être de même longueur\end{définition}
\begin{définition}\cmn 长短一致\end{définition}
\begin{relation-sémantique}\confer{
\hyperlink{Ⓔsɤfsuja}{\textit{ \papi{sɤfsuja}}}
}\end{relation-sémantique}
\begin{relation-sémantique}\confer{
\hyperlink{Ⓔafsɯfsu}{\textit{ \papi{afsɯfsu}}}
}\end{relation-sémantique}\end{entrée}

\begin{entrée}
\vedette{\hypertarget{Ⓔafsoʁŋgi}{\papi{ afsoʁŋgi}}}\markboth{afsoʁŋgi}{}
\classe{vs}
\paradigme{\textit{dir :} \jya tɤ-}
\begin{définition}\fra être clair\end{définition}
\begin{définition}\cmn 颜色浅;色彩亮丽;亮堂\end{définition}
\begin{exemple}\jya afsoʁŋgi\cmn 很亮\end{exemple}
\begin{exemple}\jya to-k-ɤfsoʁŋgi-ci\cmn 变亮了\end{exemple}
\begin{exemple}\jya kɯki ɯ-mdoʁ kɯ-ɤfsoʁŋgi ci ŋu\cmn 颜色很浅\end{exemple}\end{entrée}

\begin{entrée}
\vedette{\hypertarget{Ⓔafsɯfsu}{\papi{ afsɯfsu}}}\markboth{afsɯfsu}{}
\classe{vs}
\begin{définition}\fra être égal\end{définition}
\begin{définition}\cmn 长短一样\end{définition}
\begin{exemple}\jya nɤ-ndʑu ni ɲɯ-ɤfsɯfsu-ndʑi\cmn 你的筷子的长短一样\end{exemple}
\begin{relation-sémantique}\confer{
\hyperlink{Ⓔɯ-fsu}{\textit{ \papi{ɯ-fsu}}}
}\end{relation-sémantique}
\begin{relation-sémantique}\confer{
\hyperlink{Ⓔafsujɯja}{\textit{ \papi{afsujɯja}}}
}\end{relation-sémantique}\end{entrée}

\begin{entrée}
\vedette{\hypertarget{Ⓔaftɕaka}{\papi{ aftɕaka}}}\markboth{aftɕaka}{}\classe{vs}
\begin{définition}\fra avoir les ornements et vêtements au complet\end{définition}
\begin{définition}\cmn 服饰齐全\end{définition}
\begin{relation-sémantique}\confer{
\hyperlink{Ⓔftɕaka}{\textit{ \papi{ftɕaka}}}
}\end{relation-sémantique}\end{entrée}

\begin{entrée}
\vedette{\hypertarget{Ⓔaɣɤmphɯmphri}{\papi{ aɣɤmphɯmphri}}}\markboth{aɣɤmphɯmphri}{} (\variante{amphɯmphri}) 
\classe{vi}
\paradigme{\textit{dir :} \jya pɯ-}
\begin{définition}\fra être l'un après l'autre\end{définition}
\begin{définition}\cmn 一个接着一个
\end{définition}
\begin{exemple}\jya kɤ-nɤma ɲɯ-dɤn tɕe ɲɯ-ɤmphɯmphri ɕti\cmn 工作很多,连续不断\end{exemple}
\begin{exemple}\jya kɤtsa ra nɯ-kɯ-mŋɤm pjɤ-dɤn tɕe pjɤ-k-ɤɣɤmphɯmphri-ci\cmn 他们这一家人过去很多生病的,一个接着一个\end{exemple}
\begin{exemple}\jya iʑɤɣ ji-ma ra ɲɯ-ɤɣɤmphɯmphri ɕti\cmn 我们的工作连续不断,一个接着一个\end{exemple}\end{entrée}

\begin{entrée}
\vedette{\hypertarget{Ⓔaɣɤŋɯŋoʁ}{\papi{ aɣɤŋɯŋoʁ}}}\markboth{aɣɤŋɯŋoʁ}{}
\begin{relation-sémantique}\confer{
\hyperlink{Ⓔɣɤŋoʁ}{\textit{ \papi{ɣɤŋoʁ}}}
}\end{relation-sémantique}\end{entrée}

\begin{entrée}
\vedette{\hypertarget{Ⓔaɣɤrlɯrlaʁ}{\papi{ aɣɤrlɯrlaʁ}}}\markboth{aɣɤrlɯrlaʁ}{}
\begin{relation-sémantique}\confer{
\hyperlink{Ⓔɣɤrlaʁ}{\textit{ \papi{ɣɤrlaʁ}}}
}\end{relation-sémantique}\end{entrée}

\begin{entrée}
\vedette{\hypertarget{Ⓔaɣɤtɯɣ}{\papi{ aɣɤtɯɣ}}}\markboth{aɣɤtɯɣ}{}
\begin{relation-sémantique}\confer{
\hyperlink{Ⓔɣɤtɯɣ}{\textit{ \papi{ɣɤtɯɣ}}}
}\end{relation-sémantique}\end{entrée}

\begin{entrée}
\vedette{\hypertarget{Ⓔaɣɤzdɯzda}{\papi{ aɣɤzdɯzda}}}\markboth{aɣɤzdɯzda}{}
\begin{relation-sémantique}\confer{
\hyperlink{Ⓔɣɤzda}{\textit{ \papi{ɣɤzda}}}
}\end{relation-sémantique}
\end{entrée}

\begin{entrée}
\vedette{\hypertarget{Ⓔaɣndɯɣnda}{\papi{ aɣndɯɣnda}}}\markboth{aɣndɯɣnda}{}
\classe{vs}
\begin{définition}\fra devenu ferme après avoir été piétinée (terre)\end{définition}
\begin{définition}\cmn 踩紧的\end{définition}
\begin{exemple}\jya tɯ-ji lo-ɕlu-nɯ ri, ɯ-taʁ to-ŋke-nɯ tɕe pjɤ-k-ɤɣndɯɣnda-ci\cmn 他们虽然种了地,但是因为踩在上面把地踩紧了\end{exemple}\end{entrée}

\begin{entrée}
\vedette{\hypertarget{Ⓔaɣrɤɣrum}{\papi{ aɣrɤɣrum}}}\markboth{aɣrɤɣrum}{}\classe{vs}
\begin{définition}\fra être blanchâtre\end{définition}
\begin{définition}\cmn 微白的\end{définition}
\begin{relation-sémantique}\confer{
\hyperlink{Ⓔwɣrum}{\textit{ \papi{wɣrum}}}
}\end{relation-sémantique}\end{entrée}

\begin{entrée}
\vedette{\hypertarget{Ⓔaɣro}{\papi{ aɣro}}}\markboth{aɣro}{}
\classe{vi}
\paradigme{\textit{dir :} \jya nɯ-}
\paradigme{\textit{dir :} \jya tɤ-}
\paradigme{\textit{dir :} \jya pɯ-}
\begin{définition}\fra jouer\end{définition}
\begin{définition}\cmn 玩\end{définition}
\begin{exemple}\jya pɯ-aɣro-tɕi\cmn (过去)我们在玩\end{exemple}
\begin{exemple}\jya a-xtɤɣ nɯ cho tɤ-anɯɣro-tɕi\cmn 我跟哥哥玩了一下\end{exemple}
\begin{exemple}\jya @lanqiu pɯ-asɯ-lɤt-tɕi pɯ-anɯɣro-tɕi\cmn (过去)我们打篮球\end{exemple}\begin{sous-entrée}
\vedette{\hypertarget{}{\papi{ nɤɣro}}}\markboth{nɤɣro}{}\classe{vt}
\paradigme{\textit{dir :} \jya nɯ-}
\paradigme{\textit{dir :} \jya tɤ-}
\paradigme{\textit{dir :} \jya pɯ-}
\begin{définition}\ 
\begin{déclaration}\grammar{appl}\end{déclaration}\end{définition}\acception{1}
\begin{définition}\fra jouer à\end{définition}
\begin{définition}\cmn 玩(某种游戏)\end{définition}
\begin{exemple}\jya @lanqiu nɯ-nɤɣro-t-a\cmn 我打了篮球\end{exemple}
\begin{exemple}\jya ta-ma tɤ-nɤme @lanqiu ma-nɯ-tɯ-nɤɣrɤm\cmn 你工作啦,不要打篮球\end{exemple}
\begin{exemple}\jya aki aʑo @lanqiu pɯ-az-nɤɣro-a\cmn (过去时)我打篮球\end{exemple}\acception{2}
\begin{définition}\fra taquiner\end{définition}
\begin{définition}\cmn 逗弄\end{définition}
\end{sous-entrée}\end{entrée}

\begin{entrée}
\vedette{\hypertarget{Ⓔaɣɯβlu}{\papi{ aɣɯβlu}}}\markboth{aɣɯβlu}{}\classe{vs}
\begin{définition}\fra rusé\end{définition}
\begin{définition}\cmn 足智多谋,狡猾\end{définition}
\begin{relation-sémantique}\confer{
\hyperlink{Ⓔɯ-βlu}{\textit{ \papi{ɯ-βlu}}}
}\end{relation-sémantique}\end{entrée}

\begin{entrée}
\vedette{\hypertarget{Ⓔaɣɯci}{\papi{ aɣɯci}}}\markboth{aɣɯci}{}\classe{vs}
\begin{définition}\ 
\begin{déclaration}\grammar{denom}\end{déclaration}\end{définition}
\begin{définition}\fra qui a du jus\end{définition}
\begin{définition}\cmn 有汁\end{définition}
\begin{relation-sémantique}\confer{
\hyperlink{Ⓔtɯ-ci}{\textit{ \papi{tɯ-ci}}}
}\end{relation-sémantique}\end{entrée}

\begin{entrée}
\vedette{\hypertarget{Ⓔaɣɯɕa}{\papi{ aɣɯɕa}}}\markboth{aɣɯɕa}{}\classe{vs}
\paradigme{\textit{dir :} \jya thɯ-}
\begin{définition}\ 
\begin{déclaration}\grammar{denom}\end{déclaration}\end{définition}
\begin{définition}\fra qui a beaucoup de chair\end{définition}
\begin{définition}\cmn 肉多\end{définition}
\begin{exemple}\jya ki paʁ ki ɲɯ-ɤɣɯɕa\cmn 这只猪肉很多\end{exemple}
\begin{exemple}\jya nɤʑo nɤ-tɯ-ɤɣɯɕa nɯ!\cmn 你长得很胖啊\end{exemple}\end{entrée}

\begin{entrée}
\vedette{\hypertarget{Ⓔaɣɯɕɤrɯ}{\papi{ aɣɯɕɤrɯ}}}\markboth{aɣɯɕɤrɯ}{}\classe{vs}
\begin{définition}\ 
\begin{déclaration}\grammar{denom}\end{déclaration}\end{définition}
\begin{définition}\fra qui a la peau sur les os\end{définition}
\begin{définition}\cmn 瘦\end{définition}
\begin{relation-sémantique}\synonyme{
\hyperlink{Ⓔnɯɲɤmkhe}{\textit{ \papi{nɯɲɤmkhe}}}
}\end{relation-sémantique}
\begin{relation-sémantique}\confer{
\hyperlink{Ⓔɕɤrɯ}{\textit{ \papi{ɕɤrɯ}}}
}\end{relation-sémantique}\end{entrée}

\begin{entrée}
\vedette{\hypertarget{Ⓔaɣɯɕkat}{\papi{ aɣɯɕkat}}}\markboth{aɣɯɕkat}{}
\begin{relation-sémantique}\confer{
\hyperlink{Ⓔɣɯɕkat}{\textit{ \papi{ɣɯɕkat}}}
}\end{relation-sémantique}\end{entrée}

\begin{entrée}
\vedette{\hypertarget{Ⓔaɣɯɕnɯɕna}{\papi{ aɣɯɕnɯɕna}}}\markboth{aɣɯɕnɯɕna}{}\classe{vs}
\begin{définition}\ 
\begin{déclaration}\grammar{denom}\end{déclaration}\end{définition}
\begin{définition}\fra qui a le sens de l'odorat\end{définition}
\begin{définition}\cmn 有嗅觉\end{définition}
\begin{exemple}\jya kɯ-ɤɣɯɕnɯɕna ʁɟa ɕti tɕe, tɤ-di mtshɤm-nɯ\cmn 他们都有嗅觉,也会闻到臭味\end{exemple}
\begin{relation-sémantique}\confer{
\hyperlink{Ⓔtɯ-ɕna}{\textit{ \papi{tɯ-ɕna}}}
}\end{relation-sémantique}\end{entrée}

\begin{entrée}
\vedette{\hypertarget{Ⓔaɣɯɕnɯɕnaβ}{\papi{ aɣɯɕnɯɕnaβ}}}\markboth{aɣɯɕnɯɕnaβ}{}\classe{vs}
\begin{définition}\ 
\begin{déclaration}\grammar{denom}\end{déclaration}\end{définition}
\begin{définition}\fra être visqueux\end{définition}
\begin{définition}\cmn 黏稠,像胶水\end{définition}
\begin{relation-sémantique}\confer{
\hyperlink{Ⓔtɯ-ɕnaβ}{\textit{ \papi{tɯ-ɕnaβ}}}
}\end{relation-sémantique}\end{entrée}

\begin{entrée}
\vedette{\hypertarget{Ⓔaɣɯdɯχɯn}{\papi{ aɣɯdɯχɯn}}}\markboth{aɣɯdɯχɯn}{}
\classe{vi}
\paradigme{\textit{dir :} \jya tɤ-}
\begin{définition}\ 
\begin{déclaration}\grammar{denom}\end{déclaration}\end{définition}
\begin{définition}\fra avoir une bonne odeur\end{définition}
\begin{définition}\cmn 香(气味)\end{définition}
\begin{exemple}\jya jiɕqha tɯsqar nɯ ɲɯ-ɤɣɯdɯχɯn\cmn 这个糌粑气味很香\end{exemple}
\begin{exemple}\jya to-k-ɤɣɯdɯχɯn-ci\cmn 气味比以前香\end{exemple}
\begin{exemple}\jya @pingguo nɯ kɯ-ɤɣɯdɯχɯn ci ɲɯ-ŋu\cmn 那个苹果有香味\end{exemple}
\begin{relation-sémantique}\confer{
\hyperlink{Ⓔɯ-dɯχɯn}{\textit{ \papi{ɯ-dɯχɯn}}}
}\end{relation-sémantique}\end{entrée}

\begin{entrée}
\vedette{\hypertarget{Ⓔaɣɯɣu}{\papi{ aɣɯɣu}}}\markboth{aɣɯɣu}{}
\classe{vi}
\paradigme{\textit{dir :} \jya tɤ-}
\paradigme{\textit{construction :} \jya participe sujet}
\begin{définition}\fra se préparer\end{définition}
\begin{définition}\cmn 准备\end{définition}
\begin{exemple}\jya ku-oɣɯɣu-a\cmn 我在准备\end{exemple}
\begin{exemple}\jya ku-oɣɯɣu\cmn 他在准备\end{exemple}
\begin{exemple}\jya to-k-ɤɣɯɣu-ci\cmn 已经准备好了\end{exemple}
\begin{exemple}\jya rpɣo kɯ-ɕe to-k-ɤɣɯɣu-ci\cmn 他准备到高山上去\end{exemple}
\begin{exemple}\jya cɤmi kɯ-ɕe to-k-ɤɣɯɣu-ci\cmn 他准备去河边\end{exemple}
\begin{exemple}\jya aj @Chengdu kɯ-ɕe tɤ-aɣɯɣu-a\cmn 我准备去成都\end{exemple}
\begin{exemple}\jya rqaco kɯ-ɕe to-k-ɤɣɯɣu-ci\cmn 他准备去呷脚\end{exemple}\end{entrée}

\begin{entrée}
\vedette{\hypertarget{Ⓔaɣɯɣli}{\papi{ aɣɯɣli}}}\markboth{aɣɯɣli}{}\classe{vs}
\begin{définition}\ 
\begin{déclaration}\grammar{denom}\end{déclaration}\end{définition}
\begin{définition}\fra qui produit beaucoup de purin\end{définition}
\begin{définition}\cmn 粪很多的\end{définition}
\begin{relation-sémantique}\confer{
\hyperlink{Ⓔtɯ-ɣli}{\textit{ \papi{tɯ-ɣli}}}
}\end{relation-sémantique}\end{entrée}

\begin{entrée}
\vedette{\hypertarget{ⒺaɣɯjaʁⒽ2}{\papi{ aɣɯjaʁ}}}\markboth{aɣɯjaʁ}{}\homonyme{2} (\variante{aɣɯjɯjaʁ}) 
\classe{vs}\acception{1}
\begin{définition}\fra toucher les objets des autres\end{définition}
\begin{définition}\cmn 乱摸别人的东西(偷东西)\end{définition}
\begin{exemple}\jya jiɕqha tɯrme kɯ-ɤɣɯjaʁ ci ŋu\cmn 这个人会偷东西\end{exemple}\acception{2}
\begin{définition}\fra toucher les objets des autres\end{définition}
\begin{définition}\cmn 乱摸别人的东西(偷东西),做事很快\end{définition}
\begin{exemple}\jya tɤ-pɤtso nɯ kɯ-ɤɣɯjɯjaʁ ci ɲɯ-ŋu\cmn 那个孩子习惯乱摸别人的东西\end{exemple}
\begin{exemple}\jya ɕɯŋgɯ mɯ́j-fse to-k-ɤɣɯjɯjaʁ-ci\cmn 跟以前不一样,现在总是乱碰别人的东西\end{exemple}\acception{3}
\begin{définition}\fra rapide\end{définition}
\begin{définition}\cmn 做事很快\end{définition}
\begin{exemple}\jya jiɕqha nɯ ɲɯ-ɤɣɯjɯjaʁ\cmn 这个人做事很快\end{exemple}
\begin{relation-sémantique}\confer{
\hyperlink{Ⓔtɯ-jaʁ}{\textit{ \papi{tɯ-jaʁ}}}
}\end{relation-sémantique}\end{entrée}

\begin{entrée}
\vedette{\hypertarget{ⒺaɣɯjɯjaʁⒽ1}{\papi{ aɣɯjɯjaʁ}}}\markboth{aɣɯjɯjaʁ}{}\homonyme{1}\classe{vi}
\paradigme{\textit{dir :} \jya tɤ-}
\begin{définition}\fra avoir beaucoup de pattes\end{définition}
\begin{définition}\cmn 有很多只脚(蜈蚣等虫子)\end{définition}
\begin{exemple}\jya qaprɤftsa ɲɯ-ɤɣɯjɯjaʁ\cmn 蜈蚣有很多只脚\end{exemple}
\begin{exemple}\jya ɴɢoɕna ɲɯ-ɤɣɯjɯjaʁ ɲɯ-ɤɣɯmɯmi, ɯ-mɤlɤjaʁ ɲɯ-dɤn\cmn 蜘蛛有很多脚\end{exemple}
\begin{relation-sémantique}\confer{
\hyperlink{Ⓔtɯ-jaʁ}{\textit{ \papi{tɯ-jaʁ}}}
}\end{relation-sémantique}\end{entrée}

\begin{entrée}
\vedette{\hypertarget{Ⓔaɣɯjwaʁ}{\papi{ aɣɯjwaʁ}}}\markboth{aɣɯjwaʁ}{}
\classe{vi}
\paradigme{\textit{dir :} \jya tɤ-}
\begin{définition}\ 
\begin{déclaration}\grammar{denom}\end{déclaration}\end{définition}
\begin{définition}\fra touffu, ayant beaucoup de feuilles\end{définition}
\begin{définition}\cmn 叶子茂盛\end{définition}
\begin{exemple}\jya jiɕqha si nɯ ɲɯ-ɤɣɯjwaʁ\cmn 这棵树长出很多叶子\end{exemple}
\begin{exemple}\jya jiɕqha ɟu nɯ ɲɯ-ɤɣɯjwaʁ\cmn 这棵竹子长出很多叶子\end{exemple}
\begin{exemple}\jya @wosun to-k-ɤɣɯjwaʁ-ci\cmn 莴笋跟以前不一样,现在长出很多叶子\end{exemple}\end{entrée}

\begin{entrée}
\vedette{\hypertarget{Ⓔaɣɯlu}{\papi{ aɣɯlu}}}\markboth{aɣɯlu}{}\classe{vi}
\begin{définition}\ 
\begin{déclaration}\grammar{denom}\end{déclaration}\end{définition}
\begin{définition}\fra qui a beaucoup de lait\end{définition}
\begin{définition}\cmn 产奶多的(母牛)\end{définition}
\begin{exemple}\jya ki nɯŋa ki ɲɯ-ɤɣɯlu\cmn 这个母牛产很多奶\end{exemple}
\begin{relation-sémantique}\confer{
\hyperlink{Ⓔtɤ-lu}{\textit{ \papi{tɤ-lu}}}
}\end{relation-sémantique}\end{entrée}

\begin{entrée}
\vedette{\hypertarget{Ⓔaɣɯli}{\papi{ aɣɯli}}}\markboth{aɣɯli}{}\classe{vs}
\paradigme{\textit{dir :} \jya tɤ-}
\begin{définition}\fra patient\end{définition}
\begin{définition}\cmn 有耐性;不怕失败;不怕吃苦\end{définition}
\begin{exemple}\jya kɤ-rɤβzjoz nɯ, pjɯ-kɯ-ɤɣɯli ʑo ra, nɯ mɤɕtʂa kú-wɣ-spa mɤ-kɯ-cha\cmn 学习要有耐性,不然不是会成功的\end{exemple}
\begin{exemple}\jya nɤʑo ɲɯ-tɯ-ɤɣɯli tɕe ɲɯ-pe\cmn 你很有耐性,这样很好\end{exemple}\end{entrée}

\begin{entrée}
\vedette{\hypertarget{Ⓔaɣɯlɯtshɤt}{\papi{ aɣɯlɯtshɤt}}}\markboth{aɣɯlɯtshɤt}{}\classe{vs}
\begin{définition}\fra à peu près du même âge\end{définition}
\begin{définition}\cmn 跟自己年龄差不多的\end{définition}
\begin{exemple}\jya ndʑiʑo ɲɯ-tɯ-ɤɣɯlɯtshɤt-ndʑi\cmn 你们俩年龄差不多\end{exemple}
\begin{relation-sémantique}\confer{
\hyperlink{Ⓔɯ-lɯtshɤt}{\textit{ \papi{ɯ-lɯtshɤt}}}
}\end{relation-sémantique}\end{entrée}

\begin{entrée}
\vedette{\hypertarget{Ⓔaɣɯmar}{\papi{ aɣɯmar}}}\markboth{aɣɯmar}{}\classe{vi}
\begin{définition}\ 
\begin{déclaration}\grammar{denom}\end{déclaration}\end{définition}
\begin{définition}\fra qui peut produire beaucoup de beurre\end{définition}
\begin{définition}\cmn 可以打出很多酥油\end{définition}
\begin{exemple}\jya ki nɯŋa ɯ-lu ɲɯ-ɤɣɯmar\cmn 这个母牛的奶可以打出很多酥油\end{exemple}
\begin{relation-sémantique}\confer{
\hyperlink{Ⓔta-mar}{\textit{ \papi{ta-mar}}}
}\end{relation-sémantique}\end{entrée}

\begin{entrée}
\vedette{\hypertarget{Ⓔaɣɯmat}{\papi{ aɣɯmat}}}\markboth{aɣɯmat}{}
\classe{vi}
\paradigme{\textit{dir :} \jya thɯ-}
\begin{définition}\ 
\begin{déclaration}\grammar{denom}\end{déclaration}\end{définition}
\begin{définition}\fra faire beaucoup de fruits\end{définition}
\begin{définition}\cmn 结很多果子\end{définition}
\begin{exemple}\jya @pingguo ɲɯ-ɤɣɯmat\cmn 苹果树结很多果子\end{exemple}
\begin{exemple}\jya ʑɴɢɯloʁ ɲɯ-ɣɯmat\cmn 核桃结很多果子\end{exemple}
\begin{exemple}\jya ɣɯjpa ji-tɕɣom chɤ-k-ɤɣɯmat-ci\cmn 今年我们家的花椒结的果子比以前多\end{exemple}\end{entrée}

\begin{entrée}
\vedette{\hypertarget{Ⓔaɣɯmdoʁ}{\papi{ aɣɯmdoʁ}}}\markboth{aɣɯmdoʁ}{}
\classe{vi}
\paradigme{\textit{dir :} \jya tɤ-}
\begin{définition}\ 
\begin{déclaration}\grammar{denom}\end{déclaration}\end{définition}
\begin{définition}\fra avoir la même couleur\end{définition}
\begin{définition}\cmn 颜色相同\end{définition}
\begin{exemple}\jya to-k-ɤɣɯmdoʁ-ci\cmn 颜色以前不一样,现在一样了\end{exemple}
\begin{exemple}\jya tɕiʑo ɣɯ tɕi-ŋga ɲɯ-ɤɣɯmdoʁ\cmn 我们的衣服是同一个颜色的\end{exemple}
\begin{relation-sémantique}\confer{
\hyperlink{Ⓔɯ-mdoʁ}{\textit{ \papi{ɯ-mdoʁ}}}
}\end{relation-sémantique}\end{entrée}

\begin{entrée}
\vedette{\hypertarget{Ⓔaɣɯmɲaʁ}{\papi{ aɣɯmɲaʁ}}}\markboth{aɣɯmɲaʁ}{}\classe{vs}
\begin{définition}\ 
\begin{déclaration}\grammar{denom}\end{déclaration}\end{définition}\acception{1}
\begin{définition}\fra qui a beaucoup de trous\end{définition}
\begin{définition}\cmn 有很多小洞\end{définition}
\begin{exemple}\jya nɤki tshaʁ wuma ʑo ɲɯ-ɤɣɯmɲaʁ\cmn 那个筛子有很多眼儿\end{exemple}\acception{2}
\begin{définition}\fra qui a des yeux\end{définition}
\begin{définition}\cmn 有眼睛\end{définition}
\begin{relation-sémantique}\confer{
\hyperlink{Ⓔtɯ-mɲaʁ}{\textit{ \papi{tɯ-mɲaʁ}}}
}\end{relation-sémantique}\end{entrée}

\begin{entrée}
\vedette{\hypertarget{Ⓔaɣɯmphɯmphru}{\papi{ aɣɯmphɯmphru}}}\markboth{aɣɯmphɯmphru}{}
\classe{vi}
\paradigme{\textit{dir :} \jya tɤ-}
\begin{définition}\fra cohérent, suivi\end{définition}
\begin{définition}\cmn 连贯(话)\end{définition}
\begin{exemple}\jya li to-k-ɤ-ɣɯmphɯmphru-ci\cmn 又变得连贯(连续不断)了\end{exemple}
\begin{exemple}\jya nɯɕɯŋgɯ ɯ-rju pɯ-aɣɯmphɯmphru ri, tham tɕe mɯ-ɲɤ-cha\cmn 他以前讲话比较连贯,现在不行了\end{exemple}\begin{sous-entrée}
\vedette{\hypertarget{}{\papi{ zɣɯmphɯmphru}}}\markboth{zɣɯmphɯmphru}{}\classe{vt}
\begin{définition}\fra faire de façon complète, cohérente\end{définition}
\begin{définition}\cmn 做得连贯\end{définition}
\begin{exemple}\jya χpi kɤ-fɕɤt kɤ-zɣɯmphɯmphru ɲɯ-tɯ-cha\cmn 你能够把故事连贯地讲出来\end{exemple}
\begin{relation-sémantique}\confer{
\hyperlink{Ⓔɯ-mphru}{\textit{ \papi{ɯ-mphru}}}
}\end{relation-sémantique}
\end{sous-entrée}\end{entrée}

\begin{entrée}
\vedette{\hypertarget{Ⓔaɣɯmtɕhi}{\papi{ aɣɯmtɕhi}}}\markboth{aɣɯmtɕhi}{}\classe{vs}
\begin{définition}\ 
\begin{déclaration}\grammar{denom}\end{déclaration}\end{définition}
\begin{définition}\fra bavard\end{définition}
\begin{définition}\cmn 说话很多\end{définition}
\begin{exemple}\jya ki tɯrme ki kɯ-ɤɣɯmtɕhi ci ŋu\cmn 这个人说的话多\end{exemple}
\begin{relation-sémantique}\confer{
\hyperlink{Ⓔtɯ-mtɕhi}{\textit{ \papi{tɯ-mtɕhi}}}
}\end{relation-sémantique}\end{entrée}

\begin{entrée}
\vedette{\hypertarget{Ⓔaɣɯmɯmi}{\papi{ aɣɯmɯmi}}}\markboth{aɣɯmɯmi}{}\classe{vs}
\begin{définition}\ 
\begin{déclaration}\grammar{denom}\end{déclaration}\end{définition}
\begin{définition}\fra qui a beaucoup de pattes\end{définition}
\begin{définition}\cmn 脚很多\end{définition}
\begin{relation-sémantique}\confer{
\hyperlink{ⒺaɣɯjɯjaʁⒽ1}{\textit{ \papi{aɣɯjɯjaʁ}}}
}\end{relation-sémantique}
\begin{relation-sémantique}\confer{
\hyperlink{Ⓔtɯ-mi}{\textit{ \papi{tɯ-mi}}}
}\end{relation-sémantique}\end{entrée}

\begin{entrée}
\vedette{\hypertarget{Ⓔaɣɯndzɣi}{\papi{ aɣɯndzɣi}}}\markboth{aɣɯndzɣi}{}\classe{vs}
\begin{définition}\ 
\begin{déclaration}\grammar{denom}\end{déclaration}\end{définition}
\begin{définition}\fra qui a des crocs\end{définition}
\begin{définition}\cmn 有獠牙\end{définition}
\begin{exemple}\jya srɯnmɯ ɲɯ-ɤɣɯndzɯndzrɯ ɲɯ-ɤɣɯndzɯndzɣi\cmn 妖精有爪子又有獠牙\end{exemple}
\begin{relation-sémantique}\confer{
\hyperlink{Ⓔtɯ-ndzɣi}{\textit{ \papi{tɯ-ndzɣi}}}
}\end{relation-sémantique}\end{entrée}

\begin{entrée}
\vedette{\hypertarget{Ⓔaɣɯndzrɯ}{\papi{ aɣɯndzrɯ}}}\markboth{aɣɯndzrɯ}{}\classe{vs}
\begin{définition}\ 
\begin{déclaration}\grammar{denom}\end{déclaration}\end{définition}
\begin{définition}\fra qui a des griffes\end{définition}
\begin{définition}\cmn 有爪子\end{définition}
\begin{exemple}\jya srɯnmɯ ɲɯ-ɤɣɯndzɯndzrɯ ɲɯ-ɤɣɯndzɯndzɣi\cmn 妖精有爪子又有獠牙\end{exemple}
\begin{relation-sémantique}\confer{
\hyperlink{Ⓔtɯ-ndzrɯ}{\textit{ \papi{tɯ-ndzrɯ}}}
}\end{relation-sémantique}\end{entrée}

\begin{entrée}
\vedette{\hypertarget{Ⓔaɣɯndʑɯɣ}{\papi{ aɣɯndʑɯɣ}}}\markboth{aɣɯndʑɯɣ}{}
\classe{vs}
\paradigme{\textit{dir :} \jya nɯ-}
\begin{définition}\ 
\begin{déclaration}\grammar{denom}\end{déclaration}\end{définition}\acception{1}
\begin{définition}\fra qui a beaucoup de résine\end{définition}
\begin{définition}\cmn 树脂多
\end{définition}
\begin{exemple}\jya tɯrgi ɲɯ-ɤɣɯndʑɯɣ\cmn 杉树有很多树脂\end{exemple}
\begin{exemple}\jya tɤtho ɲɯ-ɤɣɯndʑɯɣ\cmn 松树有很多树脂\end{exemple}\acception{2}
\begin{définition}\fra collant\end{définition}
\begin{définition}\cmn 有粘性\end{définition}
\begin{exemple}\jya ɯ-ɕtʂi pjɤ-ɬoʁ tɕe, ɯ-ɕtʂi ra ɲɯ-ɤɣɯndʑɯɣ ʑo\cmn 他流汗,一身黏糊糊的\end{exemple}
\begin{relation-sémantique}\confer{
\hyperlink{Ⓔtɤ-ndʑɯɣ}{\textit{ \papi{tɤ-ndʑɯɣ}}}
}\end{relation-sémantique}\end{entrée}

\begin{entrée}
\vedette{\hypertarget{Ⓔaɣɯntɤβ}{\papi{ aɣɯntɤβ}}}\markboth{aɣɯntɤβ}{}\classe{vs}
\begin{définition}\fra qui a des bulles, moussant\end{définition}
\begin{définition}\cmn 泡沫多的\end{définition}
\begin{relation-sémantique}\confer{
\hyperlink{Ⓔtɤntɤβ}{\textit{ \papi{tɤntɤβ}}}
}\end{relation-sémantique}
\end{entrée}

\begin{entrée}
\vedette{\hypertarget{Ⓔaɣɯɲchaz}{\papi{ aɣɯɲchaz}}}\markboth{aɣɯɲchaz}{}
\classe{vi}
\begin{définition}\fra identique\end{définition}
\begin{définition}\cmn (话、动作都)一致;整齐\end{définition}
\begin{exemple}\jya ʁmaʁmi ra wuma ʑo ɲɯ-ɤɣɯɲchaz-nɯ, nɯsthɯci ʑo kɯ-dɤn nɤ, tɯrme tɯ-rdoʁ kɯ-ŋu ɲɯ-fse\cmn 士兵操练发出整齐,动作一致好像只有一个人一样\end{exemple}\end{entrée}

\begin{entrée}
\vedette{\hypertarget{Ⓔaɣɯŋgɤr}{\papi{ aɣɯŋgɤr}}}\markboth{aɣɯŋgɤr}{}\classe{vs}
\begin{définition}\ 
\begin{déclaration}\grammar{denom}\end{déclaration}\end{définition}
\begin{définition}\fra qui a beaucoup de lard\end{définition}
\begin{définition}\cmn 猪膘多\end{définition}
\begin{exemple}\jya ki paʁ ki ɲɯ-ɤɣɯŋgɤr\cmn 这只猪膘很多\end{exemple}\end{entrée}

\begin{entrée}
\vedette{\hypertarget{Ⓔaɣɯŋgɯŋgɯ}{\papi{ aɣɯŋgɯŋgɯ}}}\markboth{aɣɯŋgɯŋgɯ}{}\classe{vi}
\paradigme{\textit{dir :} \jya thɯ-}
\begin{définition}\ 
\begin{déclaration}\grammar{denom}\end{déclaration}\end{définition}
\begin{définition}\fra être l'un dans l'autre (sac etc)\end{définition}
\begin{définition}\cmn 一层一层套在一起(袋子)、由很多层组成的\end{définition}
\begin{exemple}\jya cho-k-ɤɣɯŋgɯŋgɯ-ci\cmn 本来不是一层一层套在一起的,现在是了\end{exemple}
\begin{exemple}\jya paʁɕi nɯ kɤ-ɣɯŋgɯŋgɯ ʑo kɤ-ndza sna\cmn 苹果可以连皮子一起吃。\end{exemple}
\begin{exemple}\jya tɯ-rju kɯ-ɤmɯtso tɤ-βze, kɯ-ɤɣɯŋgɯŋgɯ nɯ ma-tɤ-tɯ-ti\cmn 你话讲得清楚一点,不要好像话中有话\end{exemple}\begin{sous-entrée}
\vedette{\hypertarget{}{\papi{ zɣɯŋgɯŋgɯ}}}\markboth{zɣɯŋgɯŋgɯ}{}\classe{vt}
\paradigme{\textit{dir :} \jya thɯ-}
\paradigme{\textit{dir :} \jya pɯ-}
\begin{définition}\fra mettre l'un dans l'autre (sac etc)\end{définition}
\begin{définition}\cmn 连套几层\end{définition}
\begin{exemple}\jya tɤ-fkɯm ʁnɯz chɯ́-wɣ-rku chɯ́-wɣ-zɣɯŋgɯŋgɯ\cmn 把两个袋子套起来\end{exemple}
\begin{exemple}\jya khɯtsa tɯ-rdoʁ ɲɯ́-wɣ-ta, nɯ ɯ-taʁ li ci tɯ-rdoʁ pjɯ́-wɣ-ta pjɯ́-wɣ-zɣɯŋgɯŋgɯ\cmn 在那里放了一个碗,上面又放了另外一个碗,把两个碗套起来放。\end{exemple}
\end{sous-entrée}\end{entrée}

\begin{entrée}
\vedette{\hypertarget{Ⓔaɣɯŋkɯ}{\papi{ aɣɯŋkɯ}}}\markboth{aɣɯŋkɯ}{}\classe{vs}
\begin{définition}\ 
\begin{déclaration}\grammar{denom}\end{déclaration}\end{définition}
\begin{définition}\fra qui a la couenne épaisse\end{définition}
\begin{définition}\cmn 皮子很厚(猪)\end{définition}
\begin{exemple}\jya ki paʁ ki ɲɯ-ɤɣɯŋkɯ\cmn 这只猪的皮子很厚\end{exemple}
\begin{relation-sémantique}\confer{
\hyperlink{Ⓔtɤ-ŋkɯ}{\textit{ \papi{tɤ-ŋkɯ}}}
}\end{relation-sémantique}\end{entrée}

\begin{entrée}
\vedette{\hypertarget{Ⓔaɣɯpɤrtsaβ}{\papi{ aɣɯpɤrtsaβ}}}\markboth{aɣɯpɤrtsaβ}{}
\classe{vi}
\paradigme{\textit{dir :} \jya tɤ-}
\begin{définition}\fra zélé, assidu\end{définition}
\begin{définition}\cmn 勤快\end{définition}
\begin{exemple}\jya iɕqha ta-ma nɯ wuma ɲɯ-ɤɣɯpɤrtsaβ\cmn 他工作很勤快\end{exemple}
\begin{exemple}\jya ki ɯ-ʁɤri staʁ ta-ma wuma ʑo to-k-ɤɣɯpɤrtsaβ-ci\cmn 他工作比以前勤快很多\end{exemple}\end{entrée}

\begin{entrée}
\vedette{\hypertarget{Ⓔaɣɯpharɤβ}{\papi{ aɣɯpharɤβ}}}\markboth{aɣɯpharɤβ}{}
\classe{vi}
\paradigme{\textit{dir :} \jya tɤ-}
\begin{définition}\fra généreux\end{définition}
\begin{définition}\cmn 大方\end{définition}
\begin{exemple}\jya ɕɯŋgɯ staʁ to-k-ɤɣɯpharɤβ-ci\cmn 他比以前大方\end{exemple}
\begin{exemple}\jya ɯ-ndzɤtshi cho wuma kɯ-ɤɣɯpharɤβ ɲɯ-ŋu\cmn 他对吃的东西舍得花钱\end{exemple}\end{entrée}

\begin{entrée}
\vedette{\hypertarget{Ⓔaɣɯpɯpɯ}{\papi{ aɣɯpɯpɯ}}}\markboth{aɣɯpɯpɯ}{}\classe{vs}
\begin{définition}\ 
\begin{déclaration}\grammar{denom}\end{déclaration}\end{définition}
\begin{définition}\fra qui a beaucoup de petits\end{définition}
\begin{définition}\cmn 生很多崽子\end{définition}
\begin{relation-sémantique}\confer{
\hyperlink{Ⓔtɤ-pɯ}{\textit{ \papi{tɤ-pɯ}}}
}\end{relation-sémantique}\end{entrée}

\begin{entrée}
\vedette{\hypertarget{Ⓔaɣɯqe}{\papi{ aɣɯqe}}}\markboth{aɣɯqe}{}\classe{vs}
\begin{définition}\ 
\begin{déclaration}\grammar{denom}\end{déclaration}\end{définition}
\begin{définition}\fra qui produit beaucoup d'excréments\end{définition}
\begin{définition}\cmn 屙很多屎\end{définition}
\begin{exemple}\jya ki paʁ ki ɲɯ-ɤɣɯqe\cmn 这头猪产很多肥料\end{exemple}
\begin{relation-sémantique}\confer{
\hyperlink{Ⓔtɯ-qe}{\textit{ \papi{tɯ-qe}}}
}\end{relation-sémantique}\end{entrée}

\begin{entrée}
\vedette{\hypertarget{Ⓔaɣɯrɟit}{\papi{ aɣɯrɟit}}}\markboth{aɣɯrɟit}{}\classe{vs}
\begin{définition}\ 
\begin{déclaration}\grammar{denom}\end{déclaration}\end{définition}
\begin{définition}\fra qui a beaucoup d'enfants\end{définition}
\begin{définition}\cmn 生很多孩子\end{définition}
\begin{relation-sémantique}\confer{
\hyperlink{Ⓔtɤ-rɟit}{\textit{ \papi{tɤ-rɟit}}}
}\end{relation-sémantique}\end{entrée}

\begin{entrée}
\vedette{\hypertarget{Ⓔaɣɯrkɯrkɯ}{\papi{ aɣɯrkɯrkɯ}}}\markboth{aɣɯrkɯrkɯ}{}
\classe{vi}
\paradigme{\textit{dir :} \jya kɤ-}
\paradigme{\textit{dir :} \jya tɤ-}
\begin{définition}\fra s'enrouler\end{définition}
\begin{définition}\cmn 盘起来\end{définition}
\begin{exemple}\jya qapri ko-k-ɤɣɯrkɯrkɯ-ci\cmn 蛇盘起来了\end{exemple}
\begin{exemple}\jya tɤ-ri ɲɯ-ɤɣɯrkɯrkɯ\cmn 线是盘起来的\end{exemple}\begin{sous-entrée}
\vedette{\hypertarget{}{\papi{ zɣɯrkɯrkɯ}}}\markboth{zɣɯrkɯrkɯ}{}\classe{vt}
\paradigme{\textit{dir :} \jya tɤ-}
\begin{définition}\fra enrouler\end{définition}
\begin{définition}\cmn 卷起来\end{définition}
\begin{exemple}\jya tɤ-ri tɤ-zɣɯrkɯrkɯ-t-a\cmn 我把线卷起来了\end{exemple}
\begin{relation-sémantique}\synonyme{
\hyperlink{Ⓔrɤlkɯɣ}{\textit{ \papi{rɤlkɯɣ}}}
}\end{relation-sémantique}
\end{sous-entrée}\end{entrée}

\begin{entrée}
\vedette{\hypertarget{Ⓔaɣɯrmbi}{\papi{ aɣɯrmbi}}}\markboth{aɣɯrmbi}{}\classe{vs}
\begin{définition}\ 
\begin{déclaration}\grammar{denom}\end{déclaration}\end{définition}
\begin{définition}\fra qui urine souvent\end{définition}
\begin{définition}\cmn 经常撒尿\end{définition}
\begin{exemple}\jya nɤki tɤ-pɤtso nɯ ɲɯ-ɤɣɯrmbi\cmn 那个小孩子经常撒尿\end{exemple}
\begin{relation-sémantique}\confer{
\hyperlink{Ⓔtɯ-rmbi}{\textit{ \papi{tɯ-rmbi}}}
}\end{relation-sémantique}\end{entrée}

\begin{entrée}
\vedette{\hypertarget{Ⓔaɣɯrme}{\papi{ aɣɯrme}}}\markboth{aɣɯrme}{}
\classe{vs}
\paradigme{\textit{dir :} \jya thɯ-}
\begin{définition}\ 
\begin{déclaration}\grammar{denom}\end{déclaration}\end{définition}
\begin{définition}\fra poilu\end{définition}
\begin{définition}\cmn 毛多\end{définition}
\begin{exemple}\jya cho-k-ɤɣɯrme-ci\cmn 他长出了很多毛\end{exemple}
\begin{exemple}\jya ɣzɯ ɯ-rŋa nɯ tɯrme mɤ-fse kɯ aɣɯrmɯrme\cmn 猴子的脸跟人脸不同的地方就是长有很多毛\end{exemple}
\begin{relation-sémantique}\confer{
\hyperlink{Ⓔtɤ-rme}{\textit{ \papi{tɤ-rme}}}
}\end{relation-sémantique}\end{entrée}

\begin{entrée}
\vedette{\hypertarget{Ⓔaɣɯrna}{\papi{ aɣɯrna}}}\markboth{aɣɯrna}{}\classe{vs}
\begin{définition}\ 
\begin{déclaration}\grammar{denom}\end{déclaration}\end{définition}
\begin{définition}\fra qui a des oreilles\end{définition}
\begin{définition}\cmn 长有耳朵\end{définition}
\begin{exemple}\jya nɤki tɯrme ra kɯ-ɤɣɯrna ʁɟa ɕti tɕe mtshɤm-nɯ\cmn 那些人是长了耳朵的,会听见的\end{exemple}
\begin{relation-sémantique}\confer{
\hyperlink{Ⓔtɯ-rna}{\textit{ \papi{tɯ-rna}}}
}\end{relation-sémantique}\end{entrée}

\begin{entrée}
\vedette{\hypertarget{Ⓔaɣɯrnɯɕɯr}{\papi{ aɣɯrnɯɕɯr}}}\markboth{aɣɯrnɯɕɯr}{}
\classe{vi}
\paradigme{\textit{dir :} \jya nɯ-}
\begin{définition}\fra rougeâtre\end{définition}
\begin{définition}\cmn 淡红色\end{définition}
\begin{exemple}\jya ki ɯ-mdoʁ ɲɯ-ɤɣɯrnɯɕɯr\cmn 这个东西变成淡红色\end{exemple}
\begin{exemple}\jya kɯki ɯ-mdoʁ ɲɤ-k-ɤɣɯrnɯɕɯr-ci\cmn 这个东西的颜色变淡红色\end{exemple}
\begin{relation-sémantique}\confer{
\hyperlink{Ⓔɣɯrni}{\textit{ \papi{ɣɯrni}}}
}\end{relation-sémantique}\end{entrée}

\begin{entrée}
\vedette{\hypertarget{Ⓔaɣɯrŋa}{\papi{ aɣɯrŋa}}}\markboth{aɣɯrŋa}{}\classe{vs}
\begin{définition}\ 
\begin{déclaration}\grammar{denom}\end{déclaration}\end{définition}
\begin{définition}\fra se ressembler\end{définition}
\begin{définition}\cmn 脸很像\end{définition}
\begin{exemple}\jya a-pi cho ɯ-ɲɯ-ɤ́ɣɯrŋa-tɕi?\cmn 我跟哥哥像不像?\end{exemple}
\begin{relation-sémantique}\confer{
\hyperlink{Ⓔɣɤrŋa}{\textit{ \papi{ɣɤrŋa}}}
}\end{relation-sémantique}
\begin{relation-sémantique}\confer{
\hyperlink{Ⓔtɯ-rŋa}{\textit{ \papi{tɯ-rŋa}}}
}\end{relation-sémantique}\end{entrée}

\begin{entrée}
\vedette{\hypertarget{Ⓔaɣɯrŋɯl}{\papi{ aɣɯrŋɯl}}}\markboth{aɣɯrŋɯl}{}\classe{vs}
\begin{définition}\ 
\begin{déclaration}\grammar{denom}\end{déclaration}\end{définition}
\begin{définition}\fra qui a beaucoup d'argent\end{définition}
\begin{définition}\cmn 有很多银子,很多钱\end{définition}
\begin{relation-sémantique}\confer{
\hyperlink{Ⓔrŋɯl}{\textit{ \papi{rŋɯl}}}
}\end{relation-sémantique}\end{entrée}

\begin{entrée}
\vedette{\hypertarget{Ⓔaɣɯrpaʁ}{\papi{ aɣɯrpaʁ}}}\markboth{aɣɯrpaʁ}{}
\classe{vi}
\begin{définition}\ 
\begin{déclaration}\grammar{denom}\end{déclaration}\end{définition}
\begin{définition}\fra s'entendre bien\end{définition}
\begin{définition}\cmn 合得来\end{définition}
\begin{exemple}\jya nɯni ɲɯ-ɤɣɯrpaʁ-ndʑi\cmn 那两个人很合得来\end{exemple}
\begin{relation-sémantique}\confer{
\hyperlink{Ⓔanɤrpɯrpaʁ}{\textit{ \papi{anɤrpɯrpaʁ}}}
}\end{relation-sémantique}
\begin{relation-sémantique}\confer{
\hyperlink{Ⓔtɯ-rpaʁ}{\textit{ \papi{tɯ-rpaʁ}}}
}\end{relation-sémantique}\end{entrée}

\begin{entrée}
\vedette{\hypertarget{Ⓔaɣɯrqhu}{\papi{ aɣɯrqhu}}}\markboth{aɣɯrqhu}{}\classe{vs}
\begin{définition}\ 
\begin{déclaration}\grammar{denom}\end{déclaration}\end{définition}
\begin{définition}\fra qui a une écorce épaisse\end{définition}
\begin{définition}\cmn 树皮很厚、很多\end{définition}
\begin{exemple}\jya sɤjku cho mbraj ni aɣɯrqhu-ndʑi\cmn 白桦树和红桦树树皮很多\end{exemple}
\begin{relation-sémantique}\confer{
\hyperlink{Ⓔtɤ-rqhu}{\textit{ \papi{tɤ-rqhu}}}
}\end{relation-sémantique}\end{entrée}

\begin{entrée}
\vedette{\hypertarget{Ⓔaɣɯrtsi}{\papi{ aɣɯrtsi}}}\markboth{aɣɯrtsi}{}\classe{vs}
\paradigme{\textit{dir :} \jya thɯ-}
\begin{définition}\ 
\begin{déclaration}\grammar{denom}\end{déclaration}\end{définition}
\begin{définition}\fra qui produit beaucoup de graisse\end{définition}
\begin{définition}\cmn 产油多\end{définition}
\begin{exemple}\jya ki paʁ ki ɲɯ-ɤɣɯrtsi\cmn 这只猪的油很多\end{exemple}
\begin{relation-sémantique}\confer{
\hyperlink{Ⓔtɤ-rtsi}{\textit{ \papi{tɤ-rtsi}}}
}\end{relation-sémantique}\end{entrée}

\begin{entrée}
\vedette{\hypertarget{Ⓔaɣɯrtsɯrtsɤɣ}{\papi{ aɣɯrtsɯrtsɤɣ}}}\markboth{aɣɯrtsɯrtsɤɣ}{}
\classe{vs}
\paradigme{\textit{dir :} \jya tɤ-}
\begin{définition}\fra composé de sections\end{définition}
\begin{définition}\cmn 一节一节组成的\end{définition}
\begin{exemple}\jya jima ɲɯ-ɤɣɯrtsɯrtsɤɣ\cmn 玉米是一节一节的\end{exemple}
\begin{exemple}\jya ɟu ɲɯ-ɤɣɯrtsɯrtsɤɣ\cmn 竹子是一节一节的\end{exemple}
\begin{relation-sémantique}\confer{
\hyperlink{Ⓔarɤrtsɯrtsɤɣ}{\textit{ \papi{arɤrtsɯrtsɤɣ}}}
}\end{relation-sémantique}
\begin{relation-sémantique}\confer{
\hyperlink{Ⓔtɯ-rtsɤɣ}{\textit{ \papi{tɯ-rtsɤɣ}}}
}\end{relation-sémantique}\end{entrée}

\begin{entrée}
\vedette{\hypertarget{Ⓔaɣɯrtɯrtaʁ}{\papi{ aɣɯrtɯrtaʁ}}}\markboth{aɣɯrtɯrtaʁ}{}\classe{vi}
\paradigme{\textit{dir :} \jya tɤ-}
\begin{définition}\ 
\begin{déclaration}\grammar{denom}\end{déclaration}\end{définition}
\begin{définition}\fra avoir beaucoup de branches\end{définition}
\begin{définition}\cmn 枝桠多\end{définition}
\begin{exemple}\jya kɯki si ɲɯ-ɤɣɯrtɯrtaʁ\cmn 这棵树有很多枝桠\end{exemple}
\begin{exemple}\jya kɯki @huatong ɯ-jɯ ki ɲɯ-ɤɣɯrtɯrtaʁ\cmn 这个话筒的支架有架脚(三脚架)\end{exemple}
\begin{exemple}\jya to-k-ɤɣɯrtɯrtaʁ-ci\cmn 枝桠比以前多\end{exemple}
\begin{relation-sémantique}\confer{
\hyperlink{Ⓔtɤ-rtaʁ}{\textit{ \papi{tɤ-rtaʁ}}}
}\end{relation-sémantique}\end{entrée}

\begin{entrée}
\vedette{\hypertarget{ⒺaɣɯrɯruⒽ1}{\papi{ aɣɯrɯru}}}\markboth{aɣɯrɯru}{}\homonyme{1}
\classe{vs}
\begin{définition}\fra que l'on peut faire en même temps que quelque chose d'autre\end{définition}
\begin{définition}\cmn 可以顺便进行(不需要专门去)\end{définition}
\begin{exemple}\jya aʑo nɯ-anɯri-a tɕe, @cai tu-χti-a ɲɯ-ɤɣɯrɯru tɕe ɲɯ-tsɯm-a ŋu\cmn 我回家的时候,可以顺便买菜带回家\end{exemple}\begin{sous-entrée}
\vedette{\hypertarget{}{\papi{ zɣɯrɯru}}}\markboth{zɣɯrɯru}{}\classe{vi}
\begin{définition}\fra en profiter pour\end{définition}
\begin{définition}\cmn 顺便进行\end{définition}
\begin{exemple}\jya kɤ-nɤma ra tú-wɣ-zɣɯrɯru tɕe mbat\cmn 可以顺道做的事情,不要专门耽误时间就会更方便\end{exemple}
\end{sous-entrée}\end{entrée}

\begin{entrée}
\vedette{\hypertarget{ⒺaɣɯrɯruⒽ2}{\papi{ aɣɯrɯru}}}\markboth{aɣɯrɯru}{}\homonyme{2}
\classe{vs}
\begin{définition}\ 
\begin{déclaration}\grammar{denom}\end{déclaration}\end{définition}
\begin{définition}\fra qui a plusieurs troncs, et peu de feuilles et de branches\end{définition}
\begin{définition}\cmn 树干很多,叶子和树枝不多\end{définition}
\begin{exemple}\jya ki si ki kɯ-ɤɣɯrɯru ci ɲɯ-ŋu ma ɯ-jwaʁ mɯ́j-dɤn\cmn 这棵树树干很多,树叶不多\end{exemple}
\begin{relation-sémantique}\confer{
\hyperlink{Ⓔɯ-ru}{\textit{ \papi{ɯ-ru}}}
}\end{relation-sémantique}\end{entrée}

\begin{entrée}
\vedette{\hypertarget{Ⓔaɣɯrɯz}{\papi{ aɣɯrɯz}}}\markboth{aɣɯrɯz}{}\classe{vs}
\begin{définition}\ 
\begin{déclaration}\grammar{denom}\end{déclaration}\end{définition}
\begin{définition}\fra hériter (d'un trait)\end{définition}
\begin{définition}\cmn 继承\end{définition}
\begin{exemple}\jya ɯʑo ɯ-skɤt kɯ-sna nɯ, ɯ-wa ɲɤ-k-ɤɣɯrɯz-ci\cmn 他的声音很好听,是继承了他父亲\end{exemple}
\begin{exemple}\jya aʑo a-wa fse-a ma ɲɤ-k-ɤɣɯrɯz-a-ci\cmn 我像我父亲因为遗传他\end{exemple}
\begin{exemple}\jya nɤʑo ɲɤ-k-ɤɣɯrɯz-a-ci tɕe a-tɕhaʁa ɣɤʑu\cmn 我遗传你,我有双眼皮\end{exemple}\end{entrée}

\begin{entrée}
\vedette{\hypertarget{Ⓔaɣɯrʑɯrʑɯɣ}{\papi{ aɣɯrʑɯrʑɯɣ}}}\markboth{aɣɯrʑɯrʑɯɣ}{}
\classe{vi}
\paradigme{\textit{dir :} \jya tɤ-}
\begin{définition}\ 
\begin{déclaration}\grammar{denom}\end{déclaration}\end{définition}
\begin{définition}\fra ridé\end{définition}
\begin{définition}\cmn 皱(脸)\end{définition}
\begin{exemple}\jya tɕheme ra nɯ-rti nɯ ɲɯ-ɤɣɯrʑɯrʑɯɣ\cmn 女孩们的裙子是皱着的\end{exemple}
\begin{exemple}\jya to-k-ɤɣɯrʑɯrʑɯɣ-ci\cmn 变得有皱褶了\end{exemple}
\begin{exemple}\jya nɤki nɯ to-rgɤz ma ɯ-rŋa ra to-ɣɯrʑɯrʑɯɣ\cmn 他已经老了,满脸都是皱纹\end{exemple}
\begin{relation-sémantique}\confer{
\hyperlink{Ⓔtɤ-rʑɯɣ}{\textit{ \papi{tɤ-rʑɯɣ}}}
}\end{relation-sémantique}\end{entrée}

\begin{entrée}
\vedette{\hypertarget{Ⓔaɣɯsmɤɣ}{\papi{ aɣɯsmɤɣ}}}\markboth{aɣɯsmɤɣ}{}\classe{vs}
\paradigme{\textit{dir :} \jya thɯ-}
\begin{définition}\ 
\begin{déclaration}\grammar{denom}\end{déclaration}\end{définition}
\begin{définition}\fra qui a beaucoup de laine\end{définition}
\begin{définition}\cmn 毛长得多\end{définition}
\begin{exemple}\jya ki qaʑo ki ɲɯ-ɤɣɯsmɤɣ\cmn 这只羊长了很多毛\end{exemple}
\begin{relation-sémantique}\confer{
\hyperlink{Ⓔsmɤɣ}{\textit{ \papi{smɤɣ}}}
}\end{relation-sémantique}\end{entrée}

\begin{entrée}
\vedette{\hypertarget{Ⓔaɣɯsmɤn}{\papi{ aɣɯsmɤn}}}\markboth{aɣɯsmɤn}{}
\classe{vi}
\paradigme{\textit{dir :} \jya tɤ-}
\begin{définition}\ 
\begin{déclaration}\grammar{denom}\end{déclaration}\end{définition}
\begin{définition}\fra avoir un effet médical\end{définition}
\begin{définition}\cmn 有药性\end{définition}
\begin{exemple}\jya ɕɯŋgɯ staʁnɤ ɯ-jaʁ to-oɣɯsmɤn\cmn (药)医他的手效果比以前好\end{exemple}
\begin{relation-sémantique}\confer{
\hyperlink{Ⓔsmɤn}{\textit{ \papi{smɤn}}}
}\end{relation-sémantique}\end{entrée}

\begin{entrée}
\vedette{\hypertarget{Ⓔaɣɯsɯm}{\papi{ aɣɯsɯm}}}\markboth{aɣɯsɯm}{}
\classe{vi}
\paradigme{\textit{dir :} \jya nɯ-}
\begin{définition}\ 
\begin{déclaration}\grammar{denom}\end{déclaration}\end{définition}
\begin{définition}\fra s’entendre\end{définition}
\begin{définition}\cmn 一条心;想得一致\end{définition}
\begin{exemple}\jya ɲɯ-ɤɣɯsɯm-tɕi\cmn 我们俩想得一致\end{exemple}
\begin{exemple}\jya ʑara ɲɯ-ɤɣɯsɯm-nɯ\cmn 他们想得一致\end{exemple}
\begin{exemple}\jya jisŋi tɕi-tɯkrɤz ɲɯ-ɣi ma ɲɯ-ɤɣɯsɯm-tɕi\cmn 今天我们谈得很融洽\end{exemple}
\begin{exemple}\jya to-k-ɤɣɯsɯm-ndʑi-ci\cmn 他们俩变得齐心了\end{exemple}
\begin{relation-sémantique}\confer{
\hyperlink{Ⓔtɯ-sɯm}{\textit{ \papi{tɯ-sɯm}}}
}\end{relation-sémantique}\end{entrée}

\begin{entrée}
\vedette{\hypertarget{Ⓔaɣɯʂwaŋ}{\papi{ aɣɯʂwaŋ}}}\markboth{aɣɯʂwaŋ}{}
\classe{vs}
\paradigme{\textit{dir :} \jya tɤ-}
\begin{définition}\fra se correspondre\end{définition}
\begin{définition}\cmn 相称\end{définition}
\begin{exemple}\jya nɤ-ŋga cho nɤ-xtsa ra a-pɯ-ɤɣɯmdoʁ tɕe ɲɯ-ɤɣɯʂwaŋ\cmn 你的衣服和鞋子如果颜色一样的话就很搭\end{exemple}
\begin{exemple}\jya tɯ-rju nɯ ɯ-qhu ɯ-ʁɤri kɯ-ɤɣɯʂwaŋ tu-kɯ-ti ra\cmn 话要说得前后一致\end{exemple}\end{entrée}

\begin{entrée}
\vedette{\hypertarget{Ⓔaɣɯtɕha}{\papi{ aɣɯtɕha}}}\markboth{aɣɯtɕha}{}
\classe{vi}
\paradigme{\textit{dir :} \jya tɤ-}
\begin{définition}\ 
\begin{déclaration}\grammar{denom}\end{déclaration}\end{définition}
\begin{définition}\fra être par paire\end{définition}
\begin{définition}\cmn 成双\end{définition}
\begin{exemple}\jya ki tɯ-xtsa ki ɲɯ-ɤɣɯtɕha\cmn 这些鞋子是成双的(码是一样的)\end{exemple}
\begin{exemple}\jya ki tɯ-xtsa mɯ-ɲɯ-ɤɣɯtɕha\cmn 这些鞋子不是成双的(码不一样)\end{exemple}
\begin{exemple}\jya kɯki tɯ-xtsa ɯ-χtɯ to-k-ɤɣɯtɕha-ci\cmn 这些鞋子买了成双的\end{exemple}\begin{sous-entrée}
\vedette{\hypertarget{}{\papi{ aɣɯtɕhɯtɕha}}}\markboth{aɣɯtɕhɯtɕha}{}
\begin{exemple}\jya nɯ-laχtɕha kɯ-ɤɣɯtɕhɯtɕha ʁɟa ɲɯ-ŋu\cmn 他们的东西全是成双的\end{exemple}
\begin{relation-sémantique}\confer{
\hyperlink{Ⓔtɯ-tɕhaⒽ1}{\textit{ \papi{tɯ-tɕha1}}}
}\end{relation-sémantique}
\end{sous-entrée}\end{entrée}

\begin{entrée}
\vedette{\hypertarget{Ⓔaɣɯtʂɤm}{\papi{ aɣɯtʂɤm}}}\markboth{aɣɯtʂɤm}{}\classe{vs}
\paradigme{\textit{dir :} \jya thɯ-}
\begin{définition}\ 
\begin{déclaration}\grammar{denom}\end{déclaration}\end{définition}
\begin{définition}\fra qui produit beaucoup de graisse\end{définition}
\begin{définition}\cmn 产的油脂多\end{définition}
\begin{exemple}\jya ki paʁ ki ɲɯ-ɤɣɯtʂɤm\cmn 这只猪的油很多\end{exemple}\end{entrée}

\begin{entrée}
\vedette{\hypertarget{Ⓔaɣɯtʂɯn}{\papi{ aɣɯtʂɯn}}}\markboth{aɣɯtʂɯn}{}
\classe{vi}
\paradigme{\textit{dir :} \jya pɯ-}
\begin{définition}\ 
\begin{déclaration}\grammar{denom}\end{déclaration}\end{définition}
\begin{définition}\fra être le bienfaiteur\end{définition}
\begin{définition}\cmn 有恩
\begin{déclaration} \étymologie{\papi{drin}}\end{déclaration}\end{définition}
\begin{exemple}\jya jɯfɕɯndʐi wuma pjɤ-k-ɤɣɯtʂɯn-ci\cmn 他前几天(对别人)有恩\end{exemple}
\begin{exemple}\jya a-taʁ tɯ-aɣɯtʂɯn\cmn 你对我有恩\end{exemple}
\begin{relation-sémantique}\confer{
\hyperlink{Ⓔtɯ-tʂɯn}{\textit{ \papi{tɯ-tʂɯn}}}
}\end{relation-sémantique}\end{entrée}

\begin{entrée}
\vedette{\hypertarget{Ⓔaɣɯtɯɣ}{\papi{ aɣɯtɯɣ}}}\markboth{aɣɯtɯɣ}{}
\classe{vs}
\paradigme{\textit{dir :} \jya tɤ-}
\begin{définition}\ 
\begin{déclaration}\grammar{denom}\end{déclaration}
\begin{déclaration}\grammar{denom}\end{déclaration}\end{définition}
\begin{définition}\fra vénéneux\end{définition}
\begin{définition}\cmn 有毒性
\begin{déclaration} \étymologie{\papi{dug}}\end{déclaration}\end{définition}
\begin{exemple}\jya to-k-ɤɣɯtɯɣ-ci\cmn 毒性比以前大\end{exemple}
\begin{exemple}\jya jiɕqha ɣʑo kɯ kó-wɣ-mtsɯɣ-a tɕe, wuma ɲɯ-ɤɣɯtɯɣ\cmn 我被蜜蜂叮了,毒性很大\end{exemple}
\begin{exemple}\jya βɣɤrtshi kɯ kó-wɣ-mtsɯɣ-a tɕe ɲɯ-ɤɣɯtɯɣ\cmn 我被蚊子叮了,有毒\end{exemple}
\begin{exemple}\jya jɯfɕɯndʐi smɤn mɯ-pɯ-aɣɯtɯɣ, jɤxtshi pa-ɣɤjɯ tɕe ɲɯ-ɤɣɯtɯɣ\cmn 前几天农药毒性不够大,这一次他多加了一些,现在就有毒性了\end{exemple}\end{entrée}

\begin{entrée}
\vedette{\hypertarget{Ⓔaɣɯxɕɤt}{\papi{ aɣɯxɕɤt}}}\markboth{aɣɯxɕɤt}{}\classe{vs}
\begin{définition}\ 
\begin{déclaration}\grammar{denom}\end{déclaration}\end{définition}
\begin{définition}\fra efficace\end{définition}
\begin{définition}\cmn 效果好\end{définition}
\begin{exemple}\jya ki smɤn ki ɲɯ-ɤɣɯxɕɤt\cmn 这种药很有效\end{exemple}\end{entrée}

\begin{entrée}
\vedette{\hypertarget{Ⓔaɣɯzɤrŋɤn}{\papi{ aɣɯzɤrŋɤn}}}\markboth{aɣɯzɤrŋɤn}{}
\classe{vi}
\begin{définition}\fra manger tout seul\end{définition}
\begin{définition}\cmn 自己一个人吃\end{définition}
\begin{exemple}\jya nɯ-tɯ-ɤɣɯzɤrŋɤn\cmn 你自己一个人吃了\end{exemple}
\begin{exemple}\jya nɤ-zda mɯ-kɤ-tɯ-nɤjo, nɯ-tɯ-ɤɣɯzɤrŋɤn\cmn 你没有等你的伙伴,你自己一个人吃了\end{exemple}
\begin{relation-sémantique}\synonyme{
\hyperlink{Ⓔrɯndzɤqhɤjɯ}{\textit{ \papi{rɯndzɤqhɤjɯ}}}
}\end{relation-sémantique}\end{entrée}

\begin{entrée}
\vedette{\hypertarget{Ⓔaɣɯzda}{\papi{ aɣɯzda}}}\markboth{aɣɯzda}{}\classe{vi}
\paradigme{\textit{dir :} \jya kɤ-}
\begin{définition}\ 
\begin{déclaration}\grammar{denom}\end{déclaration}\end{définition}
\begin{définition}\fra être ensemble\end{définition}
\begin{définition}\cmn 互相陪伴\end{définition}
\begin{exemple}\jya ko-k-ɤɣɯzda-ndʑi-ci\cmn 他们俩互相陪伴了\end{exemple}\end{entrée}

\begin{entrée}
\vedette{\hypertarget{Ⓔaɣɯzrɯɣ}{\papi{ aɣɯzrɯɣ}}}\markboth{aɣɯzrɯɣ}{}\classe{vs}
\begin{définition}\ 
\begin{déclaration}\grammar{denom}\end{déclaration}\end{définition}
\begin{définition}\fra qui a beaucoup de poux\end{définition}
\begin{définition}\cmn 身上虱子多\end{définition}
\begin{relation-sémantique}\confer{
\hyperlink{Ⓔzrɯɣ}{\textit{ \papi{zrɯɣ}}}
}\end{relation-sémantique}\end{entrée}

\begin{entrée}
\vedette{\hypertarget{Ⓔaɣɯʑɤzdaŋ}{\papi{ aɣɯʑɤzdaŋ}}}\markboth{aɣɯʑɤzdaŋ}{}
\classe{vi}
\paradigme{\textit{dir :} \jya tɤ-}
\begin{définition}\ 
\begin{déclaration}\grammar{denom}\end{déclaration}\end{définition}
\begin{définition}\fra envieux\end{définition}
\begin{définition}\cmn 爱嫉妒别人
\begin{déclaration} \étymologie{\papi{ʑe.sdaŋ}}\end{déclaration}\end{définition}
\begin{exemple}\jya ki ɯ-ʁɤri mɯ-pɯ-aɣɯʑɤzdaŋ ri tham wuma to-k-ɤɣɯʑɤzdaŋ-ci\cmn 他以前不妒忌别人,现在就很妒忌\end{exemple}\end{entrée}

\begin{entrée}
\vedette{\hypertarget{Ⓔaɣɯʑɯʑat}{\papi{ aɣɯʑɯʑat}}}\markboth{aɣɯʑɯʑat}{}\classe{vs}
\paradigme{\textit{dir :} \jya tɤ-}
\begin{définition}\fra espiègle, coquin\end{définition}
\begin{définition}\cmn 调皮\end{définition}
\begin{exemple}\jya a-ʁi aɣɯʑɯʑat\cmn 我的弟弟很调皮\end{exemple}\end{entrée}

\begin{entrée}
\vedette{\hypertarget{Ⓔaj}{\papi{ aj}}}\markboth{aj}{}\classe{pro}
\begin{définition}\fra moi\end{définition}
\begin{définition}\cmn 我\end{définition}
\begin{relation-sémantique}\confer{
\hyperlink{Ⓔaʑo}{\textit{ \papi{aʑo}}}
}\end{relation-sémantique}
\end{entrée}

\begin{entrée}
\vedette{\hypertarget{Ⓔaja}{\papi{ aja}}}\markboth{aja}{}
\begin{relation-sémantique}\confer{
\hyperlink{ⒺjaⒽ1}{\textit{ \papi{ja1}}}
}\end{relation-sémantique}\end{entrée}

\begin{entrée}
\vedette{\hypertarget{Ⓔajɤr}{\papi{ ajɤr}}}\markboth{ajɤr}{}
\classe{vi}
\paradigme{\textit{dir :} \jya nɯ-}
\begin{définition}\fra en biais, de travers\end{définition}
\begin{définition}\cmn 歪,偏\end{définition}
\begin{exemple}\jya ɲɤ-k-ɤjɤr-ci\cmn 以前不歪,现在歪了\end{exemple}
\begin{exemple}\jya @luyinji nɯ ɲɯ-ɤjɤr\cmn 录音机放歪了\end{exemple}\begin{sous-entrée}
\vedette{\hypertarget{}{\papi{ sɤjɤr}}}\markboth{sɤjɤr}{}\classe{vt}
\paradigme{\textit{dir :} \jya nɯ-}
\begin{exemple}\jya nɤkɤcu @luyinji nɯ kɤ-ta ɲɤ-tɯ-sɤjɤr\cmn 你把录音机放歪了\end{exemple}
\end{sous-entrée}\begin{sous-entrée}
\vedette{\hypertarget{}{\papi{ ʑɣɤsɤjɤr}}}\markboth{ʑɣɤsɤjɤr}{}\classe{vt}
\begin{définition}\ 
\begin{déclaration}\grammar{refl}\end{déclaration}
\begin{déclaration}\grammar{caus}\end{déclaration}\end{définition}
\begin{définition}\fra se mettre de travers\end{définition}
\begin{définition}\cmn 斜着身子\end{définition}
\begin{exemple}\jya ɯ-stu kɤ-ɤmdzɯ, ma-nɯ-tɯ-ʑɣɤsɤjɤr ʑo\cmn 你坐直,不要斜着坐\end{exemple}
\begin{exemple}\jya ɯʑo tɤ-ŋke tɕe, ɲɯ-ʑɣɤsɤjɤr ʑo ɲɯ-ŋu\cmn 他走路的时候是斜着身子的\end{exemple}
\end{sous-entrée}\end{entrée}

\begin{entrée}
\vedette{\hypertarget{Ⓔajŋgɯɣ}{\papi{ ajŋgɯɣ}}}\markboth{ajŋgɯɣ}{}\classe{vs}
\begin{définition}\fra courbé\end{définition}
\begin{définition}\cmn 弯着\end{définition}
\begin{exemple}\jya ɯ-phoŋbu ɲɯ-ɤjŋgɯɣ\cmn 他的身子是弯着的\end{exemple}
\begin{relation-sémantique}\confer{
\hyperlink{Ⓔazgrɯ}{\textit{ \papi{azgrɯ}}}
}\end{relation-sémantique}
\begin{relation-sémantique}\confer{
\hyperlink{Ⓔajʁu}{\textit{ \papi{ajʁu}}}
}\end{relation-sémantique}\end{entrée}

\begin{entrée}
\vedette{\hypertarget{Ⓔajpomxtshɯm}{\papi{ ajpomxtshɯm}}}\markboth{ajpomxtshɯm}{}
\classe{vs}
\paradigme{\textit{dir :} \jya thɯ-}
\begin{définition}\fra d'une grosseur inégale\end{définition}
\begin{définition}\cmn 粗细不一\end{définition}
\begin{exemple}\jya tɤ-ri ɲɯ-ɤjpomxtshɯm\cmn 线有一头粗另一头细\end{exemple}
\begin{exemple}\jya thɯ-ajpomxtshɯm\cmn 变得粗细不一了\end{exemple}\begin{sous-entrée}
\vedette{\hypertarget{}{\papi{ sɤjpomxtshɯm}}}\markboth{sɤjpomxtshɯm}{}\classe{vt}
\paradigme{\textit{dir :} \jya thɯ-}
\begin{exemple}\jya chɤ-sɤjpomxtshɯm-a\cmn 我弄得不粗细不一了\end{exemple}
\begin{relation-sémantique}\confer{
\hyperlink{Ⓔjpum}{\textit{ \papi{jpum}}}
}\end{relation-sémantique}
\begin{relation-sémantique}\confer{
\hyperlink{Ⓔxtshɯm}{\textit{ \papi{xtshɯm}}}
}\end{relation-sémantique}
\begin{relation-sémantique}\confer{
\hyperlink{Ⓔjpumxtshɯm}{\textit{ \papi{jpumxtshɯm}}}
}\end{relation-sémantique}
\end{sous-entrée}\end{entrée}

\begin{entrée}
\vedette{\hypertarget{Ⓔajʁu}{\papi{ ajʁu}}}\markboth{ajʁu}{}\classe{vi}
\paradigme{\textit{dir :} \jya tɤ-}
\paradigme{\textit{dir :} \jya nɯ-}
\begin{définition}\fra courbé\end{définition}
\begin{définition}\cmn 弯(路、植物、人的四肢等)\end{définition}
\begin{exemple}\jya ki si ki ɲɯ-ɤjʁu\cmn 这棵树是弯的\end{exemple}
\begin{exemple}\jya kɯki @dianxian ki ɲɯ-ɤjʁu\cmn 这根电线是弯的\end{exemple}
\begin{exemple}\jya kɯki laʁjɯɣ ki kutɕu pɯ-ata tɕe to-k-ɤjʁu-ci\cmn 这根棍子放在这里就变弯了\end{exemple}
\begin{exemple}\jya romɲa kutɕu pɯ-ata ri ɲɤ-k-ɤjʁu-ci\cmn 梁放在这里就变弯了\end{exemple}
\begin{relation-sémantique}\confer{
\hyperlink{Ⓔamɤʁu}{\textit{ \papi{amɤʁu}}}
}\end{relation-sémantique}\begin{sous-entrée}
\vedette{\hypertarget{}{\papi{ sɤjʁu}}}\markboth{sɤjʁu}{}\classe{vt}
\paradigme{\textit{dir :} \jya tɤ-}
\begin{définition}\fra courber\end{définition}
\begin{définition}\cmn 弄弯\end{définition}
\begin{exemple}\jya ɯ-mi ki kɯ-fse to-sɤjʁu\cmn 他这样弓了腿\end{exemple}
\begin{exemple}\jya si ta-sɤjʁu\cmn 他把木头弄弯了\end{exemple}
\begin{exemple}\jya ɕom ta-sɤjʁu\cmn 他把铁弄弯了\end{exemple}
\begin{relation-sémantique}\synonyme{
\hyperlink{Ⓔkɤɣ}{\textit{ \papi{kɤɣ}}}
}\end{relation-sémantique}
\end{sous-entrée}\end{entrée}

\begin{entrée}
\vedette{\hypertarget{Ⓔajtshi}{\papi{ ajtshi}}}\markboth{ajtshi}{}
\begin{relation-sémantique}\confer{
\hyperlink{Ⓔjtshi}{\textit{ \papi{jtshi}}}
}\end{relation-sémantique}\end{entrée}

\begin{entrée}
\vedette{\hypertarget{Ⓔajtɯ}{\papi{ ajtɯ}}}\markboth{ajtɯ}{}
\classe{vi}
\paradigme{\textit{dir :} \jya tɤ-}
\begin{définition}\fra s'accumuler\end{définition}
\begin{définition}\cmn 累积\end{définition}
\begin{exemple}\jya tɯ-ci nɯ tɤ-ajtɯ\cmn 水积起来了\end{exemple}
\begin{exemple}\jya tɤ-scoz nɯ tɤ-rʑaʁ kɯ-rɲɟi tu-kɯ-stu pjɯ́-wɣ-βzjoz tɕe, ku-ojtɯ kɯ-ra ɕti\cmn 文化知识是要长期积累起来的\end{exemple}
\begin{exemple}\jya laχtɕha khro to-k-ɤjtɯ-ci\cmn 积累了很多东西\end{exemple}
\begin{exemple}\jya kɤ-ndza mɯma ajtɯ\cmn 除了食物,什么都可以积累\end{exemple}\begin{sous-entrée}
\vedette{\hypertarget{}{\papi{ ɣɤjtɯ}}}\markboth{ɣɤjtɯ}{}\classe{vi}
\begin{définition}\fra s'accumuler vite, facilement\end{définition}
\begin{définition}\cmn 容易积累
\end{définition}
\begin{exemple}\jya laχtɕha ra ɯ-grɤl kɯ-me ma-tɤ́-wɣ-ntɕhoz tɕe, tɕe ɣɤjtɯ\cmn 东西不要乱用,就积得起来\end{exemple}
\begin{relation-sémantique}\confer{
\hyperlink{ⒺndɯⒽ2}{\textit{ \papi{ndɯ2}}}
}\end{relation-sémantique}
\end{sous-entrée}\begin{sous-entrée}
\vedette{\hypertarget{}{\papi{ sɤjtɯ}}}\markboth{sɤjtɯ}{}\classe{vt}
\paradigme{\textit{dir :} \jya tɤ-}
\begin{définition}\fra accumuler\end{définition}
\begin{définition}\cmn 积攒\end{définition}
\begin{exemple}\jya staχpɯ tɤ-nɯrdoʁ-a, tɤ-sɤjtɯ-t-a\cmn 我把豌豆捡了,积攒起来了\end{exemple}
\begin{exemple}\jya a-zda ra kɯ nɯ-kɤ-mbi ra tɤ-nɯ-ndza-nɯ, aʑo nɯ-nɯ-sɤjtɯ-t-a\cmn 我的伙计们自己吃别人给的东西,我就全部积攒起来了\end{exemple}
\begin{exemple}\jya stoʁ tɤ-sɤjtɯ-t-a\cmn 我攒了胡豆\end{exemple}
\begin{exemple}\jya tɯ-ci tɤ-sɤjtɯ-t-a\cmn 我积累了水\end{exemple}
\begin{exemple}\jya kɤ-ndza tɤ-sɤjtɯ-t-a\cmn 我积累了食物\end{exemple}
\end{sous-entrée}\end{entrée}

\begin{entrée}
\vedette{\hypertarget{Ⓔajχoʁ}{\papi{ ajχoʁ}}}\markboth{ajχoʁ}{}
\classe{vi}
\paradigme{\textit{dir :} \jya kɤ-}
\begin{définition}\fra avoir le ventre plat\end{définition}
\begin{définition}\cmn 肚子瘪\end{définition}
\begin{exemple}\jya ɯ-xtu ko-k-ɤjχoʁ-ci\cmn 他肚子瘪了\end{exemple}
\begin{exemple}\jya tɤ-fkɯm nɯ ɲɯ-ɤjχoʁ\cmn 口袋是瘪的\end{exemple}
\begin{relation-sémantique}\confer{
\hyperlink{Ⓔɲchoʁ}{\textit{ \papi{ɲchoʁ}}}
}\end{relation-sémantique}\end{entrée}

\begin{entrée}
\vedette{\hypertarget{Ⓔakarɯ}{\papi{ akarɯ}}}\markboth{akarɯ}{}\classe{n}
\begin{définition}\fra origan\end{définition}
\begin{définition}\cmn 牛至\end{définition}
\begin{exemple}\jya akarɯ nɯ sɯjno kɯ-xtɕi ci ŋu, ɯ-ru kɯ-xtshɯ-xtshɯm kɯ-ɣɯrni ci ŋu, ʁnɯ-tɣa jamar ma mɤ-mbro, ɯ-jwaʁ kɯ-ɤrtɯm, kɯ-rɲɟi tsa ci ŋu, ɯ-di mnɤm, ɯ-mɯntoʁ kɯ-ɣɯrni ɯ-ŋgɯ kɯ-wɣrum tsa ci ŋu, ɯ-zrɤm kɯ-xtɕɯ-xtɕi ma me, ɯʑo smɤn ɯ-ŋgɯ kɤ-lɤt ɲɯ-sna.\cmn 牛至是一种小型植物,茎很细,呈红色,只有约两拃高,有椭圆形小叶,花红里透白,有香味,根很小。可入药。\end{exemple}\end{entrée}

\begin{entrée}
\vedette{\hypertarget{Ⓔakɤlɤt}{\papi{ akɤlɤt}}}\markboth{akɤlɤt}{}
\classe{vi}
\paradigme{\textit{dir :} \jya thɯ-}
\paradigme{\textit{dir :} \jya nɯ-}
\begin{définition}\fra se détacher\end{définition}
\begin{définition}\cmn 分裂;脱掉\end{définition}
\begin{exemple}\jya ɯ-rpaʁ chɤ-k-ɤkɤlɤt-ci\cmn 他肩膀脱臼了\end{exemple}
\begin{exemple}\jya tɯ-jaʁ ɯ-βzɯr nɯ tɕu ``hu" tu-kɯ-ti qhe tɕe sporɟɤlɯla ɯ-taʁ tú-wɣ-lɤt qhe tɕe a-pɯ-tɯɣ ʑo qhe tɕe ɯʑo ɲɯ-ɤkɤlɤt ɲɯ-ŋu\cmn 在手边“呒呒”地吹一下的时候,呼出的气如果吹到四脚蛇的尾巴上,尾巴就会脱节\end{exemple}
\begin{exemple}\jya qaʁ nɯ ɯ-jɯ cho-χɕoʁ qhe tɕe ɲɯ-ɤkɤlɤt ɕti\cmn 他拿锄头的把子抽了一下就脱落了\end{exemple}
\begin{exemple}\jya nɤ-ŋga kɤ-tʂɯβ ma akɤlɤt ɲɯ-ŋu\cmn 你把衣服补一下,不然就会脱成两半\end{exemple}
\begin{sous-entrée}
\vedette{\hypertarget{}{\papi{ sɤkɤlɤt}}}\markboth{sɤkɤlɤt}{}\classe{vt}
\paradigme{\textit{dir :} \jya nɯ-}
\paradigme{\textit{dir :} \jya pɯ-}
\begin{définition}\fra séparer\end{définition}
\begin{définition}\cmn 分开;断裂(分成两截)\end{définition}
\begin{exemple}\jya nɤki laχtɕha nɯ ma-pɯ-tɯ-sɤkɤlɤt\cmn 你不要把那个东西弄断\end{exemple}
\begin{exemple}\jya rtɤltɕaʁ ci pjɤ-rtɤβ tɕe ɲɤ-sɤkɤlɤt\cmn 他打了马鞭子把他们俩分开了\end{exemple}
\begin{exemple}\jya sporɟɤlɯla nɯ-sɤkɤlat-a\cmn 我把四脚蛇弄脱节了\end{exemple}
\end{sous-entrée}\end{entrée}

\begin{entrée}
\vedette{\hypertarget{Ⓔakɤmtɕoʁ}{\papi{ akɤmtɕoʁ}}}\markboth{akɤmtɕoʁ}{}\classe{vs}
\paradigme{\textit{dir :} \jya lɤ-}
\begin{définition}\ 
\begin{déclaration}\grammar{incorp}\end{déclaration}\end{définition}
\begin{définition}\fra pointu (sur le sommet)\end{définition}
\begin{définition}\cmn (顶部、尖头)很尖\end{définition}
\begin{exemple}\jya ɯ-ku ɲɯ-ɤkɤmtɕoʁ\cmn 尖头很尖\end{exemple}
\begin{exemple}\jya lo-k-ɤkɤmtɕoʁ-ci\cmn (用了很久),顶部就变得很尖\end{exemple}
\begin{exemple}\jya mbrɯtɕɯ ɲɤ-sa tɕe lo-k-ɤkɤmtɕoʁ-ci\cmn 刀子磨损后,顶部变得很尖\end{exemple}
\begin{relation-sémantique}\confer{
\hyperlink{Ⓔtɯ-ku}{\textit{ \papi{tɯ-ku}}}
}\end{relation-sémantique}
\begin{relation-sémantique}\confer{
\hyperlink{Ⓔamtɕoʁ}{\textit{ \papi{amtɕoʁ}}}
}\end{relation-sémantique}\end{entrée}

\begin{entrée}
\vedette{\hypertarget{Ⓔakɤtɕɤβ}{\papi{ akɤtɕɤβ}}}\markboth{akɤtɕɤβ}{}\classe{vs}
\paradigme{\textit{dir :} \jya nɯ-}
\begin{définition}\fra être croisé\end{définition}
\begin{définition}\cmn 交叉\end{définition}\begin{sous-entrée}
\vedette{\hypertarget{}{\papi{ sɤkɤtɕɤβ}}}\markboth{sɤkɤtɕɤβ}{}\classe{vt}
\paradigme{\textit{dir :} \jya nɯ-}
\begin{définition}\fra croiser\end{définition}
\begin{exemple}\jya jiʑo kɯrɯ ɣɯ ji-ŋga nɯ ɯ-naŋma cho ɯ-pɕima ɲɯ́-wɣ-sɤkɤtɕɤβ ra\cmn 我们藏装穿的时候要两边交错着\end{exemple}
\begin{relation-sémantique}\confer{
\hyperlink{Ⓔaqɤtʂha}{\textit{ \papi{aqɤtʂha}}}
}\end{relation-sémantique}
\end{sous-entrée}\end{entrée}

\begin{entrée}
\vedette{\hypertarget{Ⓔakhu}{\papi{ akhu}}}\markboth{akhu}{}\classe{vi}
\paradigme{\textit{dir :} \jya jɤ-}
\begin{définition}\fra appeler\end{définition}
\begin{définition}\cmn 叫(某人)\end{définition}
\begin{exemple}\jya atu tɯrme ci ɲɯ-ɤkhu\cmn 上头有个人在叫(你)\end{exemple}
\begin{exemple}\jya aʑo a-ɕki ɲɯ-ɤkhu\cmn 他在叫我\end{exemple}
\begin{exemple}\jya chɤ-k-ɤkhu-ci\cmn 他从里面(往外面)叫了\end{exemple}
\begin{relation-sémantique}\confer{
\hyperlink{Ⓔakhɤzŋga}{\textit{ \papi{akhɤzŋga}}}
}\end{relation-sémantique}
\begin{relation-sémantique}\confer{
\hyperlink{Ⓔnɤkhɤzŋga}{\textit{ \papi{nɤkhɤzŋga}}}
}\end{relation-sémantique}\end{entrée}

\begin{entrée}
\vedette{\hypertarget{Ⓔakhar}{\papi{ akhar}}}\markboth{akhar}{}
\classe{vi}
\paradigme{\textit{dir :} \jya lɤ-}
\paradigme{\textit{dir :} \jya pɯ-}
\begin{définition}\fra se mettre autour\end{définition}
\begin{définition}\cmn 围着(坐、站)\end{définition}
\begin{exemple}\jya ɲɯ-rɯndzɤtshi-nɯ pjɤ-k-ɤkhar-nɯ-ci\cmn 他们围着吃饭了\end{exemple}
\begin{relation-sémantique}\confer{
\hyperlink{Ⓔsɤkhar}{\textit{ \papi{sɤkhar}}}
}\end{relation-sémantique}
\begin{relation-sémantique}\confer{
\hyperlink{Ⓔnɤkhar}{\textit{ \papi{nɤkhar}}}
}\end{relation-sémantique}\end{entrée}

\begin{entrée}
\vedette{\hypertarget{Ⓔakhɤɟor}{\papi{ akhɤɟor}}}\markboth{akhɤɟor}{}\classe{vs}
\begin{définition}\fra ni rond ni carré\end{définition}
\begin{définition}\cmn 不圆不方\end{définition}
\begin{relation-sémantique}\synonyme{
\hyperlink{Ⓔamkhɤrju}{\textit{ \papi{amkhɤrju}}}
}\end{relation-sémantique}\end{entrée}

\begin{entrée}
\vedette{\hypertarget{Ⓔakhɤzŋga}{\papi{ akhɤzŋga}}}\markboth{akhɤzŋga}{}
\classe{vi}
\paradigme{\textit{dir :} \jya nɯ-}
\paradigme{\textit{dir :} \jya jɤ-}
\begin{définition}\fra crier\end{définition}
\begin{définition}\cmn 喊\end{définition}
\begin{exemple}\jya atu tɯrme ci ɣɤʑu ɲɯ-ɤkhɤzŋga\cmn 上头有个人在喊\end{exemple}
\begin{exemple}\jya staʁthɤr ɲɤ-k-ɤkhɤzŋga-ci\cmn 斯达塔尔喊了\end{exemple}
\begin{relation-sémantique}\confer{
\hyperlink{Ⓔakhu}{\textit{ \papi{akhu}}}
}\end{relation-sémantique}
\begin{relation-sémantique}\confer{
\hyperlink{Ⓔnɤkhɤzŋga}{\textit{ \papi{nɤkhɤzŋga}}}
}\end{relation-sémantique}\end{entrée}

\begin{entrée}
\vedette{\hypertarget{Ⓔakhi}{\papi{ akhi}}}\markboth{akhi}{}\classe{intj}
\begin{définition}\fra exprime que le locuteur estime avoir de la chance\end{définition}
\begin{définition}\cmn 表示自己很幸运\end{définition}
\begin{exemple}\jya akhi ma pɯ-ŋgrɯ!\cmn 很幸运,成功了\end{exemple}\end{entrée}

\begin{entrée}
\vedette{\hypertarget{Ⓔakhra}{\papi{ akhra}}}\markboth{akhra}{}
\classe{vi}
\paradigme{\textit{dir :} \jya tɤ-}
\begin{définition}\ 
\begin{déclaration}\grammar{denom}\end{déclaration}\end{définition}
\begin{définition}\fra bariolé\end{définition}
\begin{définition}\cmn 花的(颜色);有花纹的
\begin{déclaration} \étymologie{\papi{kʰra}}\end{déclaration}\end{définition}
\begin{exemple}\jya jiɕqha tɯ-ŋga nɯ ɲɯ-ɤkhra\cmn 这件衣服是花的\end{exemple}
\begin{exemple}\jya jiɕqha nɯŋa nɯ kɯ-ɤkhra ɲɯ-ŋu\cmn 这头牛是花的\end{exemple}
\begin{exemple}\jya ɯ-rŋa to-k-ɤkhra-ci\cmn 他脸花了\end{exemple}
\begin{exemple}\jya ɯ-skhrɯ mɯ́j-βdi tɕe ɯ-rŋa to-k-ɤkhra-ci\cmn 她怀孕,脸花了(有斑纹)\end{exemple}
\begin{relation-sémantique}\confer{
 \papi{akhrala}
}\end{relation-sémantique}\end{entrée}

\begin{entrée}
\vedette{\hypertarget{Ⓔakhrɤla}{\papi{ akhrɤla}}}\markboth{akhrɤla}{}\classe{vs}
\begin{définition}\fra bariolé\end{définition}
\begin{définition}\cmn 花的(颜色)\end{définition}\end{entrée}

\begin{entrée}
\vedette{\hypertarget{Ⓔakhrɤlɯla}{\papi{ akhrɤlɯla}}}\markboth{akhrɤlɯla}{} (\variante{akhrɤlɯlu}) 
\classe{vs}
\begin{définition}\fra bariolé, aux couleurs bigarrées et voyantes\end{définition}
\begin{définition}\cmn 花花绿绿\end{définition}
\begin{exemple}\jya ɯ-ŋga ɲɯ-ɤkhrɤlɯla ʑo tɕe mɯ́j-mpɕɤr\cmn 他的衣服花花绿绿,不好看\end{exemple}
\begin{relation-sémantique}\confer{
\hyperlink{Ⓔakhra}{\textit{ \papi{akhra}}}
}\end{relation-sémantique}\end{entrée}

\begin{entrée}
\vedette{\hypertarget{Ⓔakundi}{\papi{ akundi}}}\markboth{akundi}{}\classe{vi}
\paradigme{\textit{dir :} \jya kɤ-}
\begin{définition}\fra être aligné de gauche à droite\end{définition}
\begin{définition}\cmn 左右排列的\end{définition}
\begin{exemple}\jya tɕiʑo tɕi-kha ku-okundi ŋu\cmn 我们的家,一个在左边一个在右边\end{exemple}\begin{sous-entrée}
\vedette{\hypertarget{}{\papi{ sɤkundi}}}\markboth{sɤkundi}{}\classe{vt}
\paradigme{\textit{dir :} \jya kɤ-}
\begin{définition}\fra aligner de gauche à droite\end{définition}
\begin{définition}\cmn 排成一左一右\end{définition}
\begin{exemple}\jya laχtɕha kɤ-ta kɤ-sɤkundi-t-a\cmn 我把东西排成一左一右\end{exemple}
\end{sous-entrée}\end{entrée}

\begin{entrée}
\vedette{\hypertarget{Ⓔakɯ}{\papi{ akɯ}}}\markboth{akɯ}{}\classe{adv}
\begin{définition}\fra à l'est\end{définition}
\begin{définition}\cmn 在东边\end{définition}
\begin{relation-sémantique}\confer{
\hyperlink{Ⓔtɕɤkɯ}{\textit{ \papi{tɕɤkɯ}}}
}\end{relation-sémantique}\end{entrée}

\begin{entrée}
\vedette{\hypertarget{Ⓔakɯchoʁle}{\papi{ akɯchoʁle}}}\markboth{akɯchoʁle}{}
\classe{n}
\begin{définition}\fra vent du nord\end{définition}
\begin{définition}\cmn 北风\end{définition}
\begin{relation-sémantique}\confer{
\hyperlink{Ⓔqale}{\textit{ \papi{qale}}}
}\end{relation-sémantique}
\begin{relation-sémantique}\confer{
\hyperlink{Ⓔandichoʁle}{\textit{ \papi{andichoʁle}}}
}\end{relation-sémantique}\end{entrée}

\begin{entrée}
\vedette{\hypertarget{Ⓔakɯzgumba}{\papi{ akɯzgumba}}}\markboth{akɯzgumba}{}\classe{n}
\begin{définition}\fra ver blanc\end{définition}
\begin{définition}\cmn 蛴螬\end{définition}\end{entrée}

\begin{entrée}
\vedette{\hypertarget{Ⓔala}{\papi{ ala}}}\markboth{ala}{}\classe{vi}
\paradigme{\textit{dir :} \jya tɤ-}
\begin{définition}\fra bouillir\end{définition}
\begin{définition}\cmn 开(水)、沸腾\end{définition}
\begin{exemple}\jya tɯ-ci to-k-ɤla-ci\cmn 水开了\end{exemple}\begin{sous-entrée}
\vedette{\hypertarget{}{\papi{ sɤla}}}\markboth{sɤla}{}\classe{vt}
\paradigme{\textit{dir :} \jya nɯ-}
\paradigme{\textit{dir :} \jya kɤ-}
\paradigme{\textit{dir :} \jya tɤ-}
\begin{définition}\ 
\begin{déclaration}\grammar{caus}\end{déclaration}\end{définition}
\begin{définition}\fra faire bouillir\end{définition}
\begin{définition}\cmn 烧开\end{définition}
\begin{exemple}\jya tɯtshi ka-sɤla\cmn 他煲了粥\end{exemple}
\begin{exemple}\jya tʂha ta-sɤla\cmn 他熬了茶\end{exemple}
\end{sous-entrée}\end{entrée}

\begin{entrée}
\vedette{\hypertarget{Ⓔalala}{\papi{ alala}}}\markboth{alala}{}\classe{adv}
\begin{définition}\fra bien sûr\end{définition}
\begin{définition}\cmn 理所当然\end{définition}
\begin{exemple}\jya alala, ʑatsa ɣi-a ɕti\cmn 当然,我可以早点来\end{exemple}\begin{sous-entrée}
\vedette{\hypertarget{}{\papi{ ʁo alala ri}}}\markboth{ʁo alala ri}{}\classe{cnj}
\begin{définition}\fra non seulement ... mais\end{définition}
\begin{définition}\cmn 不光是……而且\end{définition}
\begin{exemple}\jya nɤʑo nɤ-ŋga ʁo alala ri kɯmaʁ kɯ-tu nɯra kɯnɤ tɤ-ndɤm\cmn 不光是你的衣服,其他所有的衣服都要带上\end{exemple}
\end{sous-entrée}\end{entrée}

\begin{entrée}
\vedette{\hypertarget{Ⓔalɤɣɯ}{\papi{ alɤɣɯ}}}\markboth{alɤɣɯ}{}
\classe{vi}
\paradigme{\textit{dir :} \jya nɯ-}
\begin{définition}\fra être connecté\end{définition}
\begin{définition}\cmn 连在一起\end{définition}
\begin{exemple}\jya jiɕqha laχtɕha nɯ ɲɯ-ɤlɤɣɯ-ndʑi\cmn 这两个东西是连在一起的\end{exemple}
\begin{exemple}\jya tɤ-ri nɯ ɲɤ-k-ɤlɤɣɯ-ndʑi-ci tɕe nɯ-rla-t-a\cmn 两根线缠在一起了,我把它们解开了\end{exemple}
\begin{relation-sémantique}\confer{
\hyperlink{Ⓔsɤlɤɣɯ}{\textit{ \papi{sɤlɤɣɯ}}}
}\end{relation-sémantique}\end{entrée}

\begin{entrée}
\vedette{\hypertarget{Ⓔalɤt}{\papi{ alɤt}}}\markboth{alɤt}{}
\begin{relation-sémantique}\confer{
\hyperlink{ⒺlɤtⒽ1}{\textit{ \papi{lɤt1}}}
}\end{relation-sémantique}\end{entrée}

\begin{entrée}
\vedette{\hypertarget{Ⓔaluj}{\papi{ aluj}}}\markboth{aluj}{}
\begin{relation-sémantique}\confer{
\hyperlink{Ⓔluj}{\textit{ \papi{luj}}}
}\end{relation-sémantique}
\end{entrée}

\begin{entrée}
\vedette{\hypertarget{Ⓔalɟɣi}{\papi{ alɟɣi}}}\markboth{alɟɣi}{}\classe{vi}
\begin{définition}\fra bouger (dent)\end{définition}
\begin{définition}\cmn 摇动(牙齿)\end{définition}
\begin{exemple}\jya a-ɕɣa ɲɤ-k-ɤlɟɣi-ci tɕe ɲɯ-mŋɤm\cmn 我的牙齿松动了,很痛\end{exemple}\end{entrée}

\begin{entrée}
\vedette{\hypertarget{Ⓔalo}{\papi{ alo}}}\markboth{alo}{}\classe{adv}
\begin{définition}\fra en amont\end{définition}
\begin{définition}\cmn 上游\end{définition}
\begin{relation-sémantique}\confer{
\hyperlink{Ⓔtɕɤlo}{\textit{ \papi{tɕɤlo}}}
}\end{relation-sémantique}\end{entrée}

\begin{entrée}
\vedette{\hypertarget{Ⓔalothi}{\papi{ alothi}}}\markboth{alothi}{}
\classe{vi}
\paradigme{\textit{dir :} \jya lɤ-}
\begin{définition}\fra être aligné d'amont en aval\end{définition}
\begin{définition}\cmn 一个在上游一个在下游\end{définition}
\begin{exemple}\jya tɕi-kha lu-olothi ŋu\cmn 我们的家一个在上游一个在下游\end{exemple}
\begin{exemple}\jya ndʑiʑo lɤ-ɤlothi-ndʑi tɕe tɤ-ndzur-ndʑi\cmn 你们俩上下站着\end{exemple}\begin{sous-entrée}
\vedette{\hypertarget{}{\papi{ sɤlothi}}}\markboth{sɤlothi}{}\classe{vt}
\paradigme{\textit{dir :} \jya lɤ-}
\begin{définition}\fra aligner de l'amont vers l'aval\end{définition}
\begin{définition}\cmn 排成一个在上游一个在下游\end{définition}
\begin{exemple}\jya laχtɕha kɤ-ta lɤ-sɤlothi-t-a pɯ-ra\cmn 我只好把东西排得一上一下\end{exemple}
\end{sous-entrée}\end{entrée}

\begin{entrée}
\vedette{\hypertarget{Ⓔalpɯm}{\papi{ alpɯm}}}\markboth{alpɯm}{}\classe{vs}
\begin{définition}\fra en commun\end{définition}
\begin{définition}\cmn 共同的\end{définition}
\begin{exemple}\jya kɯki laχtɕha ki alpɯm ɕti\cmn 这个东西是(我们)共同拥有的\end{exemple}
\begin{relation-sémantique}\synonyme{
\hyperlink{Ⓔangɯt}{\textit{ \papi{angɯt}}}
}\end{relation-sémantique}\begin{sous-entrée}
\vedette{\hypertarget{}{\papi{ sɤlpɯm}}}\markboth{sɤlpɯm}{}\classe{vt}
\begin{définition}\fra mettre ensemble\end{définition}
\begin{définition}\cmn 装在一起\end{définition}
\begin{exemple}\jya ji-laχtɕha ra thɯ-sɤlpɯm\cmn 把我们的东西都装在一起\end{exemple}
\end{sous-entrée}\end{entrée}

\begin{entrée}
\vedette{\hypertarget{Ⓔalɯlɤt}{\papi{ alɯlɤt}}}\markboth{alɯlɤt}{}\classe{vi}
\paradigme{\textit{dir :} \jya tɤ-}
\begin{définition}\ 
\begin{déclaration}\grammar{recip}\end{déclaration}\end{définition}
\begin{définition}\fra se battre\end{définition}
\begin{définition}\cmn 打架\end{définition}
\begin{exemple}\jya ʑɤni alɯlɤt-ndʑi ɲɯ-ŋu tɕe tɤ-βri-t-a\cmn 他们俩差一点打架了,我保护了他(把他们俩劝开了)\end{exemple}
\begin{exemple}\jya to-k-ɤnɯmqaj-ndʑi-ci tɕe tu-olɯlɤt-ndʑi\cmn 他们俩吵架了,还打了起来\end{exemple}
\begin{exemple}\jya to-k-ɤlɯlɤt-ndʑi-ci\cmn 他们俩打架了\end{exemple}
\begin{exemple}\jya nɯ-ɕki tɤ-alɯlat-a\cmn 我跟他们打架了\end{exemple}
\begin{relation-sémantique}\confer{
\hyperlink{ⒺlɤtⒽ1}{\textit{ \papi{lɤt1}}}
}\end{relation-sémantique}\end{entrée}

\begin{entrée}
\vedette{\hypertarget{Ⓔalɯlju}{\papi{ alɯlju}}}\markboth{alɯlju}{}
\classe{vs}
\paradigme{\textit{dir :} \jya thɯ-}
\begin{définition}\fra cylindrique\end{définition}
\begin{définition}\cmn 圆柱形\end{définition}
\begin{exemple}\jya kɯ-ɤlɯlju ɲɯ-ŋu\cmn (这只笔)是圆柱形的\end{exemple}\end{entrée}

\begin{entrée}
\vedette{\hypertarget{Ⓔalxaj}{\papi{ alxaj}}}\markboth{alxaj}{}
\classe{vi}
\paradigme{\textit{dir :} \jya nɯ-}
\begin{définition}\fra négligé (habits)\end{définition}
\begin{définition}\cmn 衣冠不整\end{définition}
\begin{exemple}\jya ɯ-ro ɯ-ŋga ɲɯ-ɤlxɤj\cmn 他的上衣是敞开着的,没有扣好\end{exemple}
\begin{exemple}\jya jiɕqha ɯ-ŋga nɯ ɲɤ-k-ɤlxɤj-ci\cmn 他的衣服乱了(他没有发觉)\end{exemple}\end{entrée}

\begin{entrée}
\vedette{\hypertarget{Ⓔaɬɯt}{\papi{ aɬɯt}}}\markboth{aɬɯt}{}\classe{vi}
\paradigme{\textit{dir :} \jya nɯ-}
\begin{définition}\fra dans le désordre (fil, pelote de laine)\end{définition}
\begin{définition}\cmn 乱(线)\end{définition}
\begin{exemple}\jya kɤtɯm pjɤ-ɴɢia tɕe ɲɯ-ɤɬɯt\cmn 线团散了就乱了\end{exemple}
\begin{exemple}\jya kɤtɯm ɲɤ-k-ɤɬɯt-ci\cmn 线团散了\end{exemple}
\begin{relation-sémantique}\synonyme{
\hyperlink{Ⓔatʂoʁloʁ}{\textit{ \papi{atʂoʁloʁ}}}
}\end{relation-sémantique}\begin{sous-entrée}
\vedette{\hypertarget{}{\papi{ sɤɬɯt}}}\markboth{sɤɬɯt}{}\classe{vt}
\paradigme{\textit{dir :} \jya nɯ-}
\paradigme{\textit{dir :} \jya thɯ-}
\begin{définition}\ 
\begin{déclaration}\grammar{caus}\end{déclaration}\end{définition}
\begin{définition}\fra mettre dans le désordre\end{définition}
\begin{définition}\cmn 弄乱(线)\end{définition}
\begin{exemple}\jya tɤ-ri ɲɤ-sɤɬɯt\cmn 他把线弄乱了\end{exemple}
\begin{relation-sémantique}\synonyme{
\hyperlink{Ⓔsɤtʂoʁloʁ}{\textit{ \papi{sɤtʂoʁloʁ}}}
}\end{relation-sémantique}
\end{sous-entrée}\end{entrée}

\begin{entrée}
\vedette{\hypertarget{Ⓔama}{\papi{ ama}}}\markboth{ama}{}\classe{intj}
\begin{définition}\fra exprime la surprise\end{définition}
\begin{définition}\cmn 表示惊奇\end{définition}
\begin{exemple}\jya ama, ɯ-tɯ-mpɕɤr nɯ!\cmn 哎呀,多么漂亮!\end{exemple}\end{entrée}

\begin{entrée}
\vedette{\hypertarget{Ⓔamar}{\papi{ amar}}}\markboth{amar}{}
\begin{relation-sémantique}\confer{
\hyperlink{Ⓔmar}{\textit{ \papi{mar}}}
}\end{relation-sémantique}
\end{entrée}

\begin{entrée}
\vedette{\hypertarget{Ⓔamɤʁu}{\papi{ amɤʁu}}}\markboth{amɤʁu}{}\classe{vs}
\begin{définition}\fra souffrir de rachitisme, avoir les jambe courbée\end{définition}
\begin{définition}\cmn 患佝偻病(弯着脚)
\begin{déclaration}\grammar{incorp}\end{déclaration}\end{définition}
\begin{relation-sémantique}\confer{
\hyperlink{Ⓔtɯ-mi}{\textit{ \papi{tɯ-mi}}}
}\end{relation-sémantique}
\begin{relation-sémantique}\confer{
 \papi{jʁu}
}\end{relation-sémantique}\end{entrée}

\begin{entrée}
\vedette{\hypertarget{Ⓔambɤldʑɤm}{\papi{ ambɤldʑɤm}}}\markboth{ambɤldʑɤm}{}\classe{vs}
\paradigme{\textit{dir :} \jya nɯ-}
\begin{définition}\fra au caractère doux et calme\end{définition}
\begin{définition}\cmn 性格文静\end{définition}
\begin{exemple}\jya ɯʑo ʁo ɲɯ-ɤmbɤldʑɤm ɕti\cmn 他倒是性格平静的人\end{exemple}
\begin{exemple}\jya jiɕqha nɯ ɯ-kɤ-qha ra rkɯn tɕe, kɯ-ɤmbɤldʑɤm ci ɕti\cmn 那个人很少生气,性格文静\end{exemple}\begin{sous-entrée}
\vedette{\hypertarget{}{\papi{ sɤmbɤldʑɤm}}}\markboth{sɤmbɤldʑɤm}{}\classe{vt}
\paradigme{\textit{dir :} \jya nɯ-}
\begin{définition}\fra régler les contentieux entre personnes\end{définition}
\begin{définition}\cmn 解决人之间的矛盾\end{définition}
\begin{exemple}\jya ʑɤni ɲɯ-ɤlɯlɤt-ndʑi ri, nɯ-nɯkhɤda-t-a-ndʑi tɕe, nɯ-sɤmbɤldʑam-a\cmn 他们在打架,我劝了他们,解决了他们的矛盾\end{exemple}
\end{sous-entrée}\end{entrée}

\begin{entrée}
\vedette{\hypertarget{Ⓔambi}{\papi{ ambi}}}\markboth{ambi}{}
\begin{relation-sémantique}\confer{
\hyperlink{Ⓔmbi}{\textit{ \papi{mbi}}}
}\end{relation-sémantique}\end{entrée}

\begin{entrée}
\vedette{\hypertarget{Ⓔamboʁ}{\papi{ amboʁ}}}\markboth{amboʁ}{}
\classe{vi}
\paradigme{\textit{dir :} \jya nɯ-}
\begin{définition}\fra exploser\end{définition}
\begin{définition}\cmn 爆炸;爆裂\end{définition}
\begin{exemple}\jya ɕɤmɯɣdɯ nɯ-amboʁ\cmn 枪爆炸了\end{exemple}
\begin{exemple}\jya qandʑi ɲɯ-ɤmboʁ ŋu\cmn 子弹可能会爆炸\end{exemple}
\begin{exemple}\jya @dahuoji ɲɯ-ɤmboʁ\cmn 打火机爆炸了\end{exemple}
\begin{exemple}\jya mɯzi ɲɤ-k-ɤmboʁ-ci\cmn 火药爆炸了\end{exemple}
\begin{relation-sémantique}\confer{
\hyperlink{Ⓔstoʁmboʁ}{\textit{ \papi{stoʁmboʁ}}}
}\end{relation-sémantique}\end{entrée}

\begin{entrée}
\vedette{\hypertarget{Ⓔambrɤqɤt}{\papi{ ambrɤqɤt}}}\markboth{ambrɤqɤt}{}\classe{vs}
\paradigme{\textit{dir :} \jya nɯ-}
\begin{définition}\fra être différent\end{définition}
\begin{définition}\cmn 有区别\end{définition}
\begin{exemple}\jya kɯ-wɣrum kɯ-ɲaʁ ɲɯ-ɤmbrɤqɤt\cmn 白色和黑色容易分得开\end{exemple}
\begin{exemple}\jya ɲo-k-ɤmbrɤqɤt-ci\cmn 变得有区别\end{exemple}
\begin{exemple}\jya tsɯʁot cho qajdo ndʑi-skɤt ɲɯ-ɤmbrɤqɤt\cmn 野鸡和乌鸦的叫声是有区别的\end{exemple}
\begin{relation-sémantique}\confer{
\hyperlink{Ⓔsɤmbrɤqɤt}{\textit{ \papi{sɤmbrɤqɤt}}}
}\end{relation-sémantique}\end{entrée}

\begin{entrée}
\vedette{\hypertarget{Ⓔambɯmbi}{\papi{ ambɯmbi}}}\markboth{ambɯmbi}{}
\begin{relation-sémantique}\confer{
\hyperlink{Ⓔmbi}{\textit{ \papi{mbi}}}
}\end{relation-sémantique}\end{entrée}

\begin{entrée}
\vedette{\hypertarget{Ⓔamdzɯ}{\papi{ amdzɯ}}}\markboth{amdzɯ}{} (\variante{amdzɯt}) 
\classe{vi}
\paradigme{\textit{dir :} \jya kɤ-}
\paradigme{\textit{dir :} \jya thɯ-}
\begin{définition}\fra s'asseoir\end{définition}
\begin{définition}\cmn 坐\end{définition}
\begin{exemple}\jya kɤ-amdzɯt\cmn 他坐了\end{exemple}
\begin{exemple}\jya ko-k-ɤmdzɯ-ci\cmn 他坐了(我来的时候他已经坐下来了)\end{exemple}
\begin{exemple}\jya ma-tɤ-tɯ-ndzur kɯ thɯ-amdzɯ\cmn 你不要站在那里,坐下(对小孩子,严厉的口气)\end{exemple}\begin{sous-entrée}
\vedette{\hypertarget{}{\papi{ sɤmdzɯ}}}\markboth{sɤmdzɯ}{}\classe{vt}
\paradigme{\textit{dir :} \jya kɤ-}
\begin{définition}\ 
\begin{déclaration}\grammar{caus}\end{déclaration}\end{définition}
\begin{définition}\fra faire asseoir\end{définition}
\begin{définition}\cmn 使坐\end{définition}
\begin{exemple}\jya rŋɯl khri ɯ-taʁ kɤ́-wɣ-sɤmdzɯ ɲɯ-ŋu\cmn 他们让她坐在银座位上\end{exemple}
\begin{relation-sémantique}\confer{
\hyperlink{Ⓔnɤmdzɯ}{\textit{ \papi{nɤmdzɯ}}}
}\end{relation-sémantique}
\end{sous-entrée}\end{entrée}

\begin{entrée}
\vedette{\hypertarget{Ⓔamdʑɯβ}{\papi{ amdʑɯβ}}}\markboth{amdʑɯβ}{}
\classe{vi}
\paradigme{\textit{dir :} \jya pɯ-}
\begin{définition}\fra serré (avec une pince ou avec ses dents)\end{définition}
\begin{définition}\cmn 紧;密封(用钳子、牙齿夹得很紧 )
\begin{déclaration}\use{这个词虽然可以翻译成“紧”,但是意思跟\stylefv{asɯɣ}完全不一样。\stylefv{amdʑɯβ}表示东西夹得很紧,缝隙都看不出来,而\stylefv{asɯɣ}则表示结打得很紧,绳子拉得很紧等意思。}\end{déclaration}\end{définition}
\begin{exemple}\jya ki ɯ-srɯβ ɲɯ-ɤmdʑɯβ (=ɲɯ-mphrɤt)\cmn 这个缝隙很紧\end{exemple}
\begin{exemple}\jya tamɢom nɯnɯ ɲɯ-ɤmdʑɯβ\cmn 夹子夹得很紧\end{exemple}
\begin{exemple}\jya kɯki ɯ-srɯβ pjɤ-tu ri pjɤ-sɯɲcɤr tɕe pjɤ-k-ɤmdʑɯβ-ci\cmn 这个东西原来有缝隙,把它按住了就变得很紧,缝隙都看不出来\end{exemple}\begin{sous-entrée}
\vedette{\hypertarget{}{\papi{ sɤmdʑɯβ}}}\markboth{sɤmdʑɯβ}{}\classe{vt}
\begin{définition}\fra fermer de façon hermétique\end{définition}
\begin{définition}\cmn 夹得很紧;使密封\end{définition}
\end{sous-entrée}\end{entrée}

\begin{entrée}
\vedette{\hypertarget{Ⓔamgri}{\papi{ amgri}}}\markboth{amgri}{}
\classe{vi}
\paradigme{\textit{dir :} \jya nɯ-}
\begin{définition}\fra claire (eau)\end{définition}
\begin{définition}\cmn 清(水)\end{définition}
\begin{exemple}\jya ki tɯ-ci ɲɯ-ɤmgri\cmn 水很清\end{exemple}
\begin{exemple}\jya kɯki tɯ-ci ɲɤ-k-ɤmgri-ci\cmn 水变清了\end{exemple}
\begin{exemple}\jya ki cha mɯm ma ɲɯ-ɤmgri\cmn 这个酒很清,好喝\end{exemple}
\begin{relation-sémantique}\antonyme{
\hyperlink{Ⓔqarndɯm}{\textit{ \papi{qarndɯm}}}
}\end{relation-sémantique}
\begin{relation-sémantique}\confer{
\hyperlink{Ⓔarɤmgrɯndɯr}{\textit{ \papi{arɤmgrɯndɯr}}}
}\end{relation-sémantique}\end{entrée}

\begin{entrée}
\vedette{\hypertarget{Ⓔamɟɤkho}{\papi{ amɟɤkho}}}\markboth{amɟɤkho}{}\classe{vi}
\paradigme{\textit{dir :} \jya nɯ-}
\begin{définition}\ 
\begin{déclaration}\grammar{comp}\end{déclaration}\end{définition}
\begin{définition}\fra remettre, confier\end{définition}
\begin{définition}\cmn 交接\end{définition}
\begin{exemple}\jya tɤ-rɟit nɯ-amɟɤkho-tɕi\cmn 我把孩子交到你手里了;你从我手中把孩子接过来了\end{exemple}
\begin{relation-sémantique}\confer{
 \papi{mɟa2}
}\end{relation-sémantique}
\begin{relation-sémantique}\confer{
\hyperlink{ⒺkhoⒽ2}{\textit{ \papi{kho2}}}
}\end{relation-sémantique}\end{entrée}

\begin{entrée}
\vedette{\hypertarget{Ⓔamkhɤrju}{\papi{ amkhɤrju}}}\markboth{amkhɤrju}{}
\classe{vs}
\paradigme{\textit{dir :} \jya nɯ-}
\begin{définition}\fra ni rond ni carré\end{définition}
\begin{définition}\cmn 不圆不方\end{définition}
\begin{exemple}\jya tɤ-fkɯm ki kɯ-ɤmkhɤrju ɲɯ-ŋu\cmn 这个袋子又不圆又不方\end{exemple}
\begin{exemple}\jya kɯki ɲɯ-ɤmkhɤrju\cmn 这个东西又不圆又不方\end{exemple}
\begin{relation-sémantique}\synonyme{
\hyperlink{Ⓔakhɤɟor}{\textit{ \papi{akhɤɟor}}}
}\end{relation-sémantique}\end{entrée}

\begin{entrée}
\vedette{\hypertarget{Ⓔamnoʁ}{\papi{ amnoʁ}}}\markboth{amnoʁ}{}
\classe{vs}
\paradigme{\textit{dir :} \jya tɤ-}
\begin{définition}\fra très large\end{définition}
\begin{définition}\cmn 容量大\end{définition}
\begin{exemple}\jya ki tɤ-fkɯm ki ɲɯ-ɤmnoʁ tɕe khro ɲɯ-xtɕhɯt\cmn 这个袋子容量很大,可以装很多东西\end{exemple}
\begin{exemple}\jya ki kha ki ɲɯ-ɤmnoʁ\cmn 这个房子容量很大\end{exemple}\end{entrée}

\begin{entrée}
\vedette{\hypertarget{Ⓔamɲaχtshɯm}{\papi{ amɲaχtshɯm}}}\markboth{amɲaχtshɯm}{}\classe{vs}
\paradigme{\textit{dir :} \jya tɤ-}
\begin{définition}\ 
\begin{déclaration}\grammar{incorp}\end{déclaration}\end{définition}
\begin{définition}\fra être mesquin\end{définition}
\begin{définition}\cmn 小心眼;斤斤计较\end{définition}
\begin{exemple}\jya ɲɯ-tɯ-amɲaχtshɯm\cmn 你斤斤计较\end{exemple}
\begin{relation-sémantique}\confer{
\hyperlink{Ⓔtɯ-mɲaʁ}{\textit{ \papi{tɯ-mɲaʁ}}}
}\end{relation-sémantique}
\begin{relation-sémantique}\confer{
\hyperlink{Ⓔxtshɯm}{\textit{ \papi{xtshɯm}}}
}\end{relation-sémantique}\end{entrée}

\begin{entrée}
\vedette{\hypertarget{Ⓔamɲɤm}{\papi{ amɲɤm}}}\markboth{amɲɤm}{}
\classe{vs}
\paradigme{\textit{dir :} \jya thɯ-}
\begin{définition}\fra homogène\end{définition}
\begin{définition}\cmn 均匀
\begin{déclaration}\use{颜色均匀不能用\stylefv{mɲɤm},必须用\stylefv{amɯzɣɯt}}\end{déclaration}\end{définition}
\begin{exemple}\jya jaʁmba ɲɯ-ɤmɲɤm\cmn 厚薄均匀\end{exemple}
\begin{exemple}\jya jpumxtshɯm ɲɯ-ɤmɲɤm\cmn 粗细均匀\end{exemple}
\begin{exemple}\jya stoʁ ndɯβjndʐɤz ɲɯ-ɤmɲɤm\cmn 胡豆大小均匀\end{exemple}
\begin{exemple}\jya ɯ-skɤt ɲɯ-ɤmɲɤm\cmn 他的声音(高低)很均匀\end{exemple}
\begin{exemple}\jya ɯ-mɲɯtɕhɤz ɲɯ-ɤmɲɤm\cmn 他的性格比较稳定\end{exemple}
\begin{exemple}\jya cho-k-ɤmɲɤm-ci\cmn 原来不均匀,现在变得很均匀\end{exemple}
\begin{exemple}\jya kɯ-ɤmɲɤm ɲɯ-ŋke (mɤ-kɯ-nɯna ʑo ɲɯ-ŋke)\cmn (汽车)不停地开\end{exemple}
\begin{relation-sémantique}\synonyme{
\hyperlink{Ⓔamɯzɣɯt}{\textit{ \papi{amɯzɣɯt}}}
}\end{relation-sémantique}\begin{sous-entrée}
\vedette{\hypertarget{}{\papi{ sɤmɲɤm}}}\markboth{sɤmɲɤm}{}
\paradigme{\textit{dir :} \jya thɯ-}
\paradigme{\textit{construction :} \jya infinitive raising}
\begin{définition}\fra rendre homogène\end{définition}
\begin{définition}\cmn 使均匀\end{définition}
\begin{exemple}\jya ɯ-kɤrme kɤ-rɤpjɤz chɤ-sɤmɲɤm\cmn 她把头发编得很均匀\end{exemple}
\begin{exemple}\jya kɤ-ɣndʑɯr chɤ-sɤmɲɤm\cmn 他磨得均匀\end{exemple}
\end{sous-entrée}\end{entrée}

\begin{entrée}
\vedette{\hypertarget{Ⓔamɲɯmɲo}{\papi{ amɲɯmɲo}}}\markboth{amɲɯmɲo}{}
\begin{relation-sémantique}\confer{
\hyperlink{ⒺmɲoⒽ2}{\textit{ \papi{mɲo2}}}
}\end{relation-sémantique}\end{entrée}

\begin{entrée}
\vedette{\hypertarget{Ⓔamŋaʁ}{\papi{ amŋaʁ}}}\markboth{amŋaʁ}{}\classe{vi}
\paradigme{\textit{dir :} \jya nɯ-}
\begin{définition}\fra moelleux (tissus)\end{définition}
\begin{définition}\cmn 松软【泡】(绒布)\end{définition}
\begin{exemple}\jya smɤɣ ɲɤ-saʁ tɕe ɲɯ-ɤmŋaʁ\cmn 羊毛梳了以后就很软\end{exemple}
\begin{exemple}\jya ɲɤ-k-ɤmŋaʁ-ci\cmn 变软了\end{exemple}\begin{sous-entrée}
\vedette{\hypertarget{}{\papi{ sɤmŋaʁ}}}\markboth{sɤmŋaʁ}{}\classe{vt}
\paradigme{\textit{dir :} \jya nɯ-}
\begin{définition}\fra rendre moelleux\end{définition}
\begin{définition}\cmn 弄松软\end{définition}
\begin{exemple}\jya smɤɣ nɯ-sɤmŋaʁ-a\cmn 我把羊毛弄松软了\end{exemple}
\end{sous-entrée}\end{entrée}

\begin{entrée}
\vedette{\hypertarget{Ⓔamqaj}{\papi{ amqaj}}}\markboth{amqaj}{}\classe{vi}
\begin{définition}\fra se disputer\end{définition}
\begin{définition}\cmn 争吵\end{définition}
\begin{exemple}\jya ɯʑo ɲɯ-ɤmqaj\cmn 他在争吵\end{exemple}
\begin{relation-sémantique}\confer{
\hyperlink{Ⓔanɯmqaj}{\textit{ \papi{anɯmqaj}}}
}\end{relation-sémantique}
\begin{relation-sémantique}\confer{
\hyperlink{Ⓔtɯ-mqaj}{\textit{ \papi{tɯ-mqaj}}}
}\end{relation-sémantique}\end{entrée}

\begin{entrée}
\vedette{\hypertarget{Ⓔamtɕhoʁ}{\papi{ amtɕhoʁ}}}\markboth{amtɕhoʁ}{}\classe{vi}
\paradigme{\textit{dir :} \jya thɯ-}
\begin{définition}\fra être en ordre\end{définition}
\begin{définition}\cmn 整齐\end{définition}
\begin{exemple}\jya ndʑu tɯ-spra ɲɯ-ɤmtɕhoʁ\cmn 那把筷子很整齐\end{exemple}
\begin{exemple}\jya ɯ-rɟit ra cho-wxti-nɯ tɕe cho-k-ɤmtɕhoʁ-nɯ-ci\cmn 他的孩子们长大,都能做事了\end{exemple}
\begin{relation-sémantique}\confer{
\hyperlink{Ⓔsɤmtɕhoʁ}{\textit{ \papi{sɤmtɕhoʁ}}}
}\end{relation-sémantique}\end{entrée}

\begin{entrée}
\vedette{\hypertarget{Ⓔamtɕoʁ}{\papi{ amtɕoʁ}}}\markboth{amtɕoʁ}{}\classe{vi}
\paradigme{\textit{dir :} \jya nɯ-}
\begin{définition}\fra pointu\end{définition}
\begin{définition}\cmn 尖\end{définition}
\begin{exemple}\jya jiɕqha mbrɯtɕɯ nɯ ɯ-ku ɲɯ-ɤmtɕoʁ\cmn 这把刀很尖\end{exemple}
\begin{exemple}\jya ndʑu ɯ-ku nɯ ɲɯ-ɤmtɕoʁ\cmn 筷子很尖\end{exemple}
\begin{exemple}\jya qaʁ ɯ-ku ɲɯ-ɤmtɕoʁ\cmn 锄头很尖\end{exemple}
\begin{exemple}\jya ɕɤmtshoʁ ɯ-ku kɯ-ɤmtɕoʁ ŋu\cmn 钉子很尖\end{exemple}
\begin{relation-sémantique}\confer{
\hyperlink{Ⓔakɤmtɕoʁ}{\textit{ \papi{akɤmtɕoʁ}}}
}\end{relation-sémantique}\end{entrée}

\begin{entrée}
\vedette{\hypertarget{Ⓔamthoʁmthɯt}{\papi{ amthoʁmthɯt}}}\markboth{amthoʁmthɯt}{}
\classe{vi}
\paradigme{\textit{dir :} \jya thɯ-}
\begin{définition}\fra suffisant\end{définition}
\begin{définition}\cmn 刚刚够用\end{définition}
\begin{exemple}\jya kɤ-ntɕhoz amthoʁmthɯt\cmn 刚刚够用\end{exemple}
\begin{exemple}\jya japa qhu ʁo tɕe, ji-ŋga ji-ndza ra thɯ-amthoʁmthɯt\cmn 从去年开始,我们吃的、穿的就足够了\end{exemple}\begin{sous-entrée}
\vedette{\hypertarget{}{\papi{ sɤmthoʁmthɯt}}}\markboth{sɤmthoʁmthɯt}{}\classe{vt}
\paradigme{\textit{dir :} \jya thɯ-}
\begin{définition}\ 
\begin{déclaration}\grammar{caus}\end{déclaration}\end{définition}
\begin{définition}\fra rendre suffisant\end{définition}
\begin{définition}\cmn 使够用;补充;填充\end{définition}
\begin{exemple}\jya kɯki laχtɕha ki thɯ-sɤmthoʁmthɯt-a\cmn 我补充了这个东西\end{exemple}
\begin{exemple}\jya tɯmbri kɤ-sɤlɤɣɯ-t-a tɕe thɯ-sɤmthoʁmthɯt-a\cmn 我把绳子接起来了,使它够长\end{exemple}
\begin{relation-sémantique}\confer{
\hyperlink{Ⓔmthɯt}{\textit{ \papi{mthɯt}}}
}\end{relation-sémantique}
\end{sous-entrée}\end{entrée}

\begin{entrée}
\vedette{\hypertarget{Ⓔamthɯn}{\papi{ amthɯn}}}\markboth{amthɯn}{}\classe{vi}
\paradigme{\textit{dir :} \jya nɯ-}
\begin{définition}\fra s'apprécier réciproquement\end{définition}
\begin{définition}\cmn 谈恋爱
\begin{déclaration} \étymologie{\papi{mtʰun}}\end{déclaration}\end{définition}
\begin{exemple}\jya ʑɤni ɲɯ-ɤmthɯn-ndʑi\cmn 他们在谈恋爱\end{exemple}\end{entrée}

\begin{entrée}
\vedette{\hypertarget{Ⓔamɯβde}{\papi{ amɯβde}}}\markboth{amɯβde}{}
\begin{relation-sémantique}\confer{
\hyperlink{Ⓔβde}{\textit{ \papi{βde}}}
}\end{relation-sémantique}\end{entrée}

\begin{entrée}
\vedette{\hypertarget{Ⓔamɯβɟɤt}{\papi{ amɯβɟɤt}}}\markboth{amɯβɟɤt}{}\classe{vi}
\paradigme{\textit{dir :} \jya nɯ-}
\begin{définition}\fra recevoir à parts égales\end{définition}
\begin{définition}\cmn 平均分到\end{définition}
\begin{exemple}\jya jiʑora mɯ-ɲɤ-k-ɤmɯβɟɤt-i ma kɤ-nɯkro ɯ-spa pjɤ-rkɯn\cmn 我们没有分到,因为要分的东西太少了\end{exemple}
\begin{relation-sémantique}\confer{
\hyperlink{Ⓔβɟɤt}{\textit{ \papi{βɟɤt}}}
}\end{relation-sémantique}\end{entrée}

\begin{entrée}
\vedette{\hypertarget{Ⓔamɯfse}{\papi{ amɯfse}}}\markboth{amɯfse}{}
\classe{vi}
\paradigme{\textit{dir :} \jya kɤ-}
\begin{définition}\fra se connaître\end{définition}
\begin{définition}\cmn 互相认识\end{définition}
\begin{exemple}\jya tɕiʑo amɯfse-tɕi\cmn 我们俩互相认识\end{exemple}
\begin{exemple}\jya jiʑo kɤ-amɯfse-j kɯmŋuxpa tɤ-tsu\cmn 我们几个认识已经五年了\end{exemple}
\begin{exemple}\jya tɕiʑo kɤ-amɯfse-tɕi nɯ kɯβdɤxpa tɤ-tsu\cmn 我们俩认识已经四年了\end{exemple}\end{entrée}

\begin{entrée}
\vedette{\hypertarget{ⒺamɯmiⒽ2}{\papi{ amɯmi}}}\markboth{amɯmi}{}\homonyme{2}
\classe{adv}
\begin{définition}\fra certainement\end{définition}
\begin{définition}\cmn 肯定;必然\end{définition}
\begin{exemple}\jya nɯ ɲɯ-ti ri amɯmi ɕti ma nɯ fse ŋgrɤl\cmn 他那样说就肯定是那样的了\end{exemple}\end{entrée}

\begin{entrée}
\vedette{\hypertarget{ⒺamɯmiⒽ1}{\papi{ amɯmi}}}\markboth{amɯmi}{}\homonyme{1}\classe{vi}
\paradigme{\textit{dir :} \jya tɤ-}
\begin{définition}\ 
\begin{déclaration}\grammar{recip}\end{déclaration}\end{définition}
\begin{définition}\fra s’entendre\end{définition}
\begin{définition}\cmn 合得来\end{définition}
\begin{exemple}\jya ʑɤni ɲɯ-ɤmɯmi-ndʑi\cmn 他们俩合得来\end{exemple}
\begin{exemple}\jya ndʑiʑo ni kɯ-ɤmɯmi ɲɯ-tɯ-ŋu-ndʑi\cmn 你们俩合得来\end{exemple}
\begin{relation-sémantique}\synonyme{
\hyperlink{Ⓔɲɟɯɣ}{\textit{ \papi{ɲɟɯɣ}}}
}\end{relation-sémantique}\end{entrée}

\begin{entrée}
\vedette{\hypertarget{Ⓔamɯmto}{\papi{ amɯmto}}}\markboth{amɯmto}{}\classe{vi}
\paradigme{\textit{dir :} \jya nɯ-}
\begin{définition}\ 
\begin{déclaration}\grammar{recip}\end{déclaration}\end{définition}
\begin{définition}\fra se voir\end{définition}
\begin{définition}\cmn 互相看见\end{définition}
\begin{exemple}\jya kutɕu cho @xiaoshuigou zgo nɯra ɲɯ-ɤmɯmto\cmn 这里和小水沟上的山互相看得见\end{exemple}
\begin{exemple}\jya tɕiʑo ni kɤ-ɤmɯmto mɯ-pɯ-rɲo-tɕi\cmn 我们俩从来都没有见过面\end{exemple}
\begin{exemple}\jya ŋotɕu tɯ-rɤʑi ma mɯ-ɲɯ-ɤmɯmto-tɕi\cmn 你在哪里,我们俩见不到对方\end{exemple}\begin{sous-entrée}
\vedette{\hypertarget{}{\papi{ sɤmɯmto}}}\markboth{sɤmɯmto}{}\classe{vt}
\begin{définition}\fra faire se rencontrer\end{définition}
\begin{définition}\cmn 使几个人互相碰见\end{définition}
\begin{relation-sémantique}\confer{
 \papi{mto}
}\end{relation-sémantique}
\end{sous-entrée}\end{entrée}

\begin{entrée}
\vedette{\hypertarget{Ⓔamɯmtshɤm}{\papi{ amɯmtshɤm}}}\markboth{amɯmtshɤm}{}
\classe{vi}
\paradigme{\textit{dir :} \jya tɤ-}
\begin{définition}\ 
\begin{déclaration}\grammar{recip}\end{déclaration}\end{définition}
\begin{définition}\fra s'entendre les uns les autres\end{définition}
\begin{définition}\cmn 互相听得到对方的声音\end{définition}
\begin{exemple}\jya @dianhua kɤ-lɤt mɯ́j-khɯ tɕe mɯ-ɲɯ-ɤmɯmtshɤm-tɕi\cmn 电话没有打成,所以我们俩听不到对方的声音\end{exemple}
\begin{relation-sémantique}\confer{
\hyperlink{Ⓔmtshɤm}{\textit{ \papi{mtshɤm}}}
}\end{relation-sémantique}\begin{sous-entrée}
\vedette{\hypertarget{}{\papi{ asɤmɯmtshɯmtshɤm}}}\markboth{asɤmɯmtshɯmtshɤm}{}\classe{vi}
\paradigme{\textit{dir :} \jya nɯ-}
\begin{définition}\ 
\begin{déclaration}\grammar{refl}\end{déclaration}\end{définition}
\begin{définition}\fra se transmettre les nouvelles les uns aux autres\end{définition}
\begin{définition}\cmn 互相传递消息;互相报信\end{définition}
\begin{exemple}\jya ɲɤ-k-ɤsɤmɯmtshɯmtshɤm-nɯ-ci\cmn 他们互相传递了消息\end{exemple}
\end{sous-entrée}\begin{sous-entrée}
\vedette{\hypertarget{}{\papi{ sɤmɯmtshɤm}}}\markboth{sɤmɯmtshɤm}{}\classe{vt}
\paradigme{\textit{dir :} \jya nɯ-}
\paradigme{\textit{dir :} \jya pɯ-}
\begin{définition}\ 
\begin{déclaration}\grammar{caus}\end{déclaration}\end{définition}
\begin{définition}\fra permettre à ... de s'entendre les uns les autres\end{définition}
\begin{définition}\cmn 让……互相听到对方的声音\end{définition}
\begin{exemple}\jya nɯ-sɤmɯmtsham-a-nɯ\cmn 我让他们互相听到对方的消息\end{exemple}
\end{sous-entrée}\end{entrée}

\begin{entrée}
\vedette{\hypertarget{Ⓔamɯrga}{\papi{ amɯrga}}}\markboth{amɯrga}{}\classe{n}
\begin{définition}\fra albanais\end{définition}
\begin{définition}\cmn 阿尔巴尼亚人\end{définition}\end{entrée}

\begin{entrée}
\vedette{\hypertarget{Ⓔamɯrmbat}{\papi{ amɯrmbat}}}\markboth{amɯrmbat}{}
\classe{vi}
\paradigme{\textit{dir :} \jya tɤ-}
\begin{définition}\ 
\begin{déclaration}\grammar{recip}\end{déclaration}\end{définition}
\begin{définition}\fra être proche\end{définition}
\begin{définition}\cmn 挨近;接近\end{définition}
\begin{exemple}\jya tɕiʑo ni mɯ-ɲɯ-ɤmɯrmbat-tɕi\cmn 我们俩很远\end{exemple}
\begin{exemple}\jya nɤj nɤ-kha cho aʑo a-kha nɯ mɯ-ɲɯ-ɤmɯrmbat-ndʑi\cmn 你的家离我的家很远\end{exemple}
\begin{exemple}\jya jiʑora amɯrmbat-i\cmn 我们很接近\end{exemple}
\begin{relation-sémantique}\confer{
\hyperlink{Ⓔarmbat}{\textit{ \papi{armbat}}}
}\end{relation-sémantique}\end{entrée}

\begin{entrée}
\vedette{\hypertarget{Ⓔamɯrmbɯ}{\papi{ amɯrmbɯ}}}\markboth{amɯrmbɯ}{}\classe{vi}
\paradigme{\textit{dir :} \jya tɤ-}
\begin{définition}\fra être rempli\end{définition}
\begin{définition}\cmn 被堆满;被装满\end{définition}
\begin{exemple}\jya ɯ-tɯ-dɤn kɯ tɤ-k-ɤmɯrmbɯ-ci zjaŋzjaŋ\cmn 装满了\end{exemple}\begin{sous-entrée}
\vedette{\hypertarget{}{\papi{ zmɯrmbɯ}}}\markboth{zmɯrmbɯ}{}\classe{vt}
\paradigme{\textit{dir :} \jya tɤ-}
\begin{définition}\ 
\begin{déclaration}\grammar{caus}\end{déclaration}\end{définition}
\begin{définition}\fra remplir\end{définition}
\begin{définition}\cmn 装满;盛满;堆满\end{définition}
\begin{exemple}\jya tɯsqar tɤ-zmɯrmbɯ-t-a\cmn 我盛满了糌粑\end{exemple}
\begin{exemple}\jya tɤjlu tɤ-zmɯrmbɯ-t-a\cmn 我盛满了面粉\end{exemple}
\begin{exemple}\jya khɯtsa ɯ-ŋgɯ tɯsqar tɤ-zmɯrmbɯ-t-a\cmn 我在碗里盛满了糌粑\end{exemple}
\begin{relation-sémantique}\confer{
\hyperlink{Ⓔrmbɯ}{\textit{ \papi{rmbɯ}}}
}\end{relation-sémantique}
\end{sous-entrée}\end{entrée}

\begin{entrée}
\vedette{\hypertarget{Ⓔamɯrpu}{\papi{ amɯrpu}}}\markboth{amɯrpu}{}
\classe{vi}
\paradigme{\textit{dir :} \jya kɤ-}
\paradigme{\textit{dir :} \jya tɤ-}
\begin{définition}\ 
\begin{déclaration}\grammar{recip}\end{déclaration}\end{définition}
\begin{définition}\fra se heurter\end{définition}
\begin{définition}\cmn 互相碰撞\end{définition}
\begin{exemple}\jya ko-k-ɤmɯrpu-ndʑi-ci\cmn 他们俩相撞了\end{exemple}
\begin{exemple}\jya ɯ-tɯ-ɤŋgɤrŋgɤr kɯ ɲɯ-ɤmɯrpu-tɕi\cmn 很狭窄,所以我们俩就相撞了\end{exemple}\begin{sous-entrée}
\vedette{\hypertarget{}{\papi{ sɤmɯrpu}}}\markboth{sɤmɯrpu}{}\classe{vt}
\paradigme{\textit{dir :} \jya tɤ-}
\begin{définition}\fra faire se heurter\end{définition}
\begin{définition}\cmn 使互相碰撞\end{définition}
\begin{exemple}\jya kɯɕɯŋgɯ tɕe kɯmaʁ smi pjɤ-me tɕe, qapi tú-wɣ-sɤmɯrpu tɕe smi ɲɯ́-wɣ-sɤβzu pjɤ-ra\cmn 过去,没有其它火源,只能用白石头取火。\end{exemple}
\begin{relation-sémantique}\confer{
\hyperlink{Ⓔrpu}{\textit{ \papi{rpu}}}
}\end{relation-sémantique}
\end{sous-entrée}\end{entrée}

\begin{entrée}
\vedette{\hypertarget{Ⓔamɯrqhi}{\papi{ amɯrqhi}}}\markboth{amɯrqhi}{}\classe{vi}
\paradigme{\textit{dir :} \jya tɤ-}
\begin{définition}\ 
\begin{déclaration}\grammar{recip}\end{déclaration}\end{définition}
\begin{définition}\fra loin l'un de l'autre\end{définition}
\begin{définition}\cmn 相距很远\end{définition}
\begin{exemple}\jya a-kha cho a-@bangongshi ɲɯ-ɤmɯrqhi-ndʑi tɕe aʑo ʑa kɤ-zɣɯt mɯ-to-khɯ\cmn 我家离办公室很远,所以就迟到了\end{exemple}
\begin{relation-sémantique}\confer{
\hyperlink{Ⓔarqhi}{\textit{ \papi{arqhi}}}
}\end{relation-sémantique}\end{entrée}

\begin{entrée}
\vedette{\hypertarget{Ⓔamɯsthaβ}{\papi{ amɯsthaβ}}}\markboth{amɯsthaβ}{}
\classe{vi}
\paradigme{\textit{dir :} \jya kɤ-}
\begin{définition}\fra être l'un à côté de l'autre (deux morceaux)\end{définition}
\begin{définition}\cmn 互相接触\end{définition}
\begin{exemple}\jya tɕiʑo kɤ-amɯsthaβ-tɕi\cmn 我们俩挨在一起\end{exemple}
\begin{exemple}\jya tɤjmɤɣ to-ɬoʁ-nɯ tɕe ɲɯ-dɤn tɕe ɲɯ-ɤmɯsthaβ ʑo\cmn 长出了很多菌子,挨在一起\end{exemple}
\begin{exemple}\jya jiʑora kɤ-amɯrmbat-i tɕe kɤ-amɯsthaβ-i ʑo ɕti\cmn 我们挨得很近,挨在一起\end{exemple}\begin{sous-entrée}
\vedette{\hypertarget{}{\papi{ sɤmɯsthaβ}}}\markboth{sɤmɯsthaβ}{}\classe{vt}
\paradigme{\textit{dir :} \jya kɤ-}
\begin{définition}\ 
\begin{déclaration}\grammar{caus}\end{déclaration}\end{définition}
\begin{définition}\fra mettre ensemble\end{définition}
\begin{définition}\cmn 拼凑\end{définition}
\begin{exemple}\jya si ka-sɤmɯsthaβ\cmn 他把木头拼凑在一起了\end{exemple}
\begin{exemple}\jya rdɤstaʁ ka-sɤmɯsthaβ\cmn 他把石头拼凑在一起了\end{exemple}
\end{sous-entrée}\end{entrée}

\begin{entrée}
\vedette{\hypertarget{Ⓔamɯsɯz}{\papi{ amɯsɯz}}}\markboth{amɯsɯz}{}\classe{vi}
\paradigme{\textit{dir :} \jya nɯ-}
\begin{définition}\fra s'ébruiter, être connu de tous\end{définition}
\begin{définition}\cmn 传开\end{définition}
\begin{exemple}\jya nɤʑo jɯfɕɯndʐi kɯre jɤ-tɯ-ɣe tɕe a-rkɯ tɯrme ra nɯ-ɕki ɲɤ-k-ɤmɯsɯz-ci\cmn 你前天到这里的消息已经传开了,我周围的人都知道\end{exemple}
\begin{exemple}\jya kɤ-nɤtsɯ mɤ-ra tɕe a-nɯ-ɤmɯsɯz jɤɣ\cmn 这件事不用保密,可以公开\end{exemple}
\begin{relation-sémantique}\confer{
\hyperlink{Ⓔsɯz}{\textit{ \papi{sɯz}}}
}\end{relation-sémantique}\begin{sous-entrée}
\vedette{\hypertarget{}{\papi{ sɤmɯsɯz}}}\markboth{sɤmɯsɯz}{}\classe{vt}
\paradigme{\textit{dir :} \jya nɯ-}
\begin{définition}\fra transmettre une information\end{définition}
\begin{définition}\cmn 传递消息\end{définition}
\begin{exemple}\jya kɤ-ɕe kɯ-ra nɯ nɯ-sɤmɯsɯz-a-nɯ\cmn 我让他们都知道要去的那个消息\end{exemple}
\end{sous-entrée}\end{entrée}

\begin{entrée}
\vedette{\hypertarget{Ⓔamɯti}{\papi{ amɯti}}}\markboth{amɯti}{}\classe{vi}
\paradigme{\textit{dir :} \jya tɤ-}
\begin{définition}\fra se parler\end{définition}
\begin{définition}\cmn 彼此说\end{définition}
\begin{exemple}\jya tɕhi kɯ-fse nɯ kɤ-ɤmɯti ra ma nɯ mɤɕtʂa mɤ-kɤ-ɤmɯtso\cmn 有什么事情要互相说,才能互相理解\end{exemple}
\begin{exemple}\jya jiʑora tɕhi kɯ-fse nɯra tɤ-amɯti-j\cmn 我们互相说明了情况\end{exemple}\end{entrée}

\begin{entrée}
\vedette{\hypertarget{Ⓔamɯtso}{\papi{ amɯtso}}}\markboth{amɯtso}{}\classe{vi}
\paradigme{\textit{dir :} \jya tɤ-}
\begin{définition}\ 
\begin{déclaration}\grammar{refl}\end{déclaration}\end{définition}
\begin{définition}\fra clair (parole)\end{définition}
\begin{définition}\cmn 清楚(话、事情)\end{définition}
\begin{exemple}\jya tɯ-rju mɯ-ɲɯ-ɤmɯtso\cmn 话不清楚\end{exemple}
\begin{exemple}\jya tɤ-amɯtso-tɕi\cmn 我们俩说通了\end{exemple}
\begin{exemple}\jya kɯɕɯŋgɯ mɯ-pjɤ-k-ɤmɯtso-ndʑi tɕe, tham tɕe tɤ-amɯti-ndʑi tɕe to-k-ɤmɯtso-ndʑi\cmn 他们俩以前说不通,现在互相说话就说通了\end{exemple}\begin{sous-entrée}
\vedette{\hypertarget{}{\papi{ amɯtsɯtso}}}\markboth{amɯtsɯtso}{}\classe{vi}\acception{1}
\begin{définition}\fra se comprendre\end{définition}
\begin{définition}\cmn 互相理解\end{définition}
\begin{exemple}\jya tɕiʑo ni kɯ-ɤmɯtsɯtso ɕti-tɕi\cmn 我们俩很了解对方\end{exemple}
\begin{exemple}\jya kɤ-ɤmɯtsɯtso a-tɤ-tɯ-βze je\cmn 你要讲得很清楚\end{exemple}\acception{2}
\begin{définition}\fra très clair\end{définition}
\begin{définition}\cmn 非常清楚\end{définition}
\begin{exemple}\jya ɯʑo kɯ kɤ-ɤmɯtsɯtso to-βzu\cmn 他讲得很清楚\end{exemple}
\begin{relation-sémantique}\confer{
\hyperlink{Ⓔtso}{\textit{ \papi{tso}}}
}\end{relation-sémantique}
\end{sous-entrée}\begin{sous-entrée}
\vedette{\hypertarget{}{\papi{ sɤmɯtso}}}\markboth{sɤmɯtso}{}\classe{vt}
\paradigme{\textit{dir :} \jya tɤ-}
\begin{définition}\ 
\begin{déclaration}\grammar{caus}\end{déclaration}\end{définition}
\begin{définition}\fra dire clairement\end{définition}
\begin{définition}\cmn 说清楚\end{définition}
\begin{exemple}\jya tɯ-rju ta-sɤmɯtso (tɤ-sɤmɯtso-t-a)\cmn 他(我)把话说清楚了\end{exemple}
\begin{exemple}\jya tɯ-rju mɯ-to-sɤmɯtso\cmn 他没有把话说清楚\end{exemple}
\begin{exemple}\jya mɯ-to-tɯ-sɤmɯtso-t\cmn 你没有说清楚\end{exemple}
\begin{exemple}\jya pɯ-kɯ-fse nɯ kɤ-sɤmɯtso ra\cmn 要把发生的事情说清楚\end{exemple}
\begin{exemple}\jya tɯrme kɯ tɤ-kɤ-tɯt nɯnɯ kɤ-sɤmɯtso\cmn 要把人家讲过的话说清楚\end{exemple}
\end{sous-entrée}\end{entrée}

\begin{entrée}
\vedette{\hypertarget{Ⓔamɯtsɯtso}{\papi{ amɯtsɯtso}}}\markboth{amɯtsɯtso}{}
\begin{relation-sémantique}\confer{
\hyperlink{Ⓔamɯtso}{\textit{ \papi{amɯtso}}}
}\end{relation-sémantique}\end{entrée}

\begin{entrée}
\vedette{\hypertarget{Ⓔamɯtɯɣ}{\papi{ amɯtɯɣ}}}\markboth{amɯtɯɣ}{}
\classe{vi}
\paradigme{\textit{dir :} \jya nɯ-}
\begin{définition}\ 
\begin{déclaration}\grammar{refl}\end{déclaration}\end{définition}
\begin{définition}\fra se rencontrer\end{définition}
\begin{définition}\cmn 相逢\end{définition}
\begin{exemple}\jya ɲo-k-ɤmɯtɯɣ-ndʑi-ci\cmn 他们俩相逢了\end{exemple}\begin{sous-entrée}
\vedette{\hypertarget{}{\papi{ sɤmɯtɯɣ}}}\markboth{sɤmɯtɯɣ}{}\classe{vt}
\paradigme{\textit{dir :} \jya nɯ-}
\begin{définition}\fra faire se rencontrer\end{définition}
\begin{définition}\cmn 使合拢\end{définition}
\begin{exemple}\jya tʂu kɤ-βzu nɯ-sɤmɯtɯɣ-i\cmn (我们把路从两边开始修起)最后在中间合拢\end{exemple}
\begin{relation-sémantique}\confer{
\hyperlink{Ⓔatɯɣ}{\textit{ \papi{atɯɣ}}}
}\end{relation-sémantique}
\end{sous-entrée}\end{entrée}

\begin{entrée}
\vedette{\hypertarget{Ⓔamɯxtɕhɯxtɕhɯt}{\papi{ amɯxtɕhɯxtɕhɯt}}}\markboth{amɯxtɕhɯxtɕhɯt}{}
\classe{vs}
\begin{définition}\fra être serré les uns contre les autres\end{définition}
\begin{définition}\cmn 挤挤挨挨,非常拥挤
\begin{déclaration}\use{只用于否定式}\end{déclaration}\end{définition}
\begin{exemple}\jya ji-tɯ-dɤn kɯ mɯ-ɲɯ-ɤmɯxtɕhɯxtɕhɯt-i ʑo\cmn 我们(人)多到非常拥挤\end{exemple}
\begin{exemple}\jya tʂu ɯ-taʁ @qiche ɯ-tɯ-dɤn kɯ mɯ-ɲɯ-ɤmɯxtɕhɯxtɕhɯt-nɯ ʑo\cmn 路上汽车多到挤也挤不动\end{exemple}\end{entrée}

\begin{entrée}
\vedette{\hypertarget{Ⓔamɯzɣɯt}{\papi{ amɯzɣɯt}}}\markboth{amɯzɣɯt}{}
\classe{vi}
\paradigme{\textit{dir :} \jya nɯ-}
\begin{définition}\fra égal, homogène\end{définition}
\begin{définition}\cmn 均匀\end{définition}
\begin{exemple}\jya ɯ-mdoʁ ɲɯ-ɤmɯzɣɯt\cmn 颜色均匀\end{exemple}
\begin{exemple}\jya mɯ-ɲɯ-ɤmɯzɣɯt\cmn 不均匀\end{exemple}
\begin{exemple}\jya ki tɯ-tɕha ki aʁɤndɯndɤt ɲɤ-k-ɤmɯzɣɯt-ci\cmn 这个消息到处传遍了\end{exemple}\begin{sous-entrée}
\vedette{\hypertarget{}{\papi{ sɤmɯzɣɯt}}}\markboth{sɤmɯzɣɯt}{}\classe{vt}
\paradigme{\textit{dir :} \jya nɯ-}
\begin{définition}\fra rendre homogène, faire de façon homogène\end{définition}
\begin{définition}\cmn 使均匀;做得均匀\end{définition}
\begin{exemple}\jya tɯ-ntʂu ɲɤ-sɤmɯzɣɯt\cmn 他薅均匀了\end{exemple}
\begin{exemple}\jya kɤ-kro nɯ-sɤmɯzɣɯt\cmn 你平均分吧\end{exemple}
\begin{exemple}\jya kɤ-rɤmbi nɯ-sɤmɯzɣɯt\cmn 你平均给吧\end{exemple}
\begin{exemple}\jya nɤ-rŋa kɤ-mar nɯ-sɤmɯzɣɯt\cmn 你要在你脸上涂均匀\end{exemple}
\begin{exemple}\jya kɤ-nɤma nɯ-sɤmɯzɣɯt\cmn 工作要做得全面一点\end{exemple}
\begin{exemple}\jya @Chengdu kɤ-sɤmɯzɣɯt mɯ́j-sɤcha\cmn 不可能走遍整个成都\end{exemple}
\begin{exemple}\jya tɯ-ŋga kɤ-ŋga nɯ-sɤmɯzɣɯt-a\cmn 我所有的衣服都穿过\end{exemple}
\begin{relation-sémantique}\confer{
\hyperlink{Ⓔamɲɤm}{\textit{ \papi{amɲɤm}}}
}\end{relation-sémantique}
\end{sous-entrée}\end{entrée}

\begin{entrée}
\vedette{\hypertarget{Ⓔamɯzwɤr}{\papi{ amɯzwɤr}}}\markboth{amɯzwɤr}{}\classe{vi}
\paradigme{\textit{dir :} \jya nɯ-}\acception{1}
\begin{définition}\fra se répandre (feu)\end{définition}
\begin{définition}\cmn 蔓延(火)\end{définition}
\begin{exemple}\jya taŋi tɤ-ɣe tɕe ɣndʑɤβ a-mɤ-tɤ-lɯɣ ra ma ɲɯ-ɤmɯzwɤr mbat\cmn 天旱的时候不要有火灾,不然活容易蔓延\end{exemple}\acception{2}
\begin{définition}\fra se répandre\end{définition}
\begin{définition}\cmn 传开\end{définition}
\begin{exemple}\jya ki tɯ-tɕha ki ɲɤ-k-ɤmɯzwɤr-ci tɕe ɲɤ-k-ɤmɯsɯz-ci\cmn 这个消息传开了\end{exemple}
\begin{relation-sémantique}\confer{
\hyperlink{ⒺzwɤrⒽ1}{\textit{ \papi{zwɤr1}}}
}\end{relation-sémantique}\end{entrée}

\begin{entrée}
\vedette{\hypertarget{Ⓔanamana}{\papi{ anamana}}}\markboth{anamana}{}
\classe{adv}
\begin{définition}\fra identique\end{définition}
\begin{définition}\cmn 一模一样
\begin{déclaration} \étymologie{\papi{a.na.ma.na}}\end{déclaration}\end{définition}
\begin{exemple}\jya kɯki tɯ-ŋga ki ni anamana ɲɯ-ŋu-ndʑi\cmn 这两件衣服是一模一样的\end{exemple}
\begin{relation-sémantique}\synonyme{
\hyperlink{Ⓔnaχtɕɯɣ}{\textit{ \papi{naχtɕɯɣ}}}
}\end{relation-sémantique}\end{entrée}

\begin{entrée}
\vedette{\hypertarget{Ⓔanaχtɯχto}{\papi{ anaχtɯχto}}}\markboth{anaχtɯχto}{}
\begin{relation-sémantique}\confer{
\hyperlink{Ⓔnaχto}{\textit{ \papi{naχto}}}
}\end{relation-sémantique}
\end{entrée}

\begin{entrée}
\vedette{\hypertarget{Ⓔanɤjɯjo}{\papi{ anɤjɯjo}}}\markboth{anɤjɯjo}{}
\begin{relation-sémantique}\confer{
\hyperlink{Ⓔnɤjo}{\textit{ \papi{nɤjo}}}
}\end{relation-sémantique}\end{entrée}

\begin{entrée}
\vedette{\hypertarget{Ⓔanɤkhɤzŋgɯzŋga}{\papi{ anɤkhɤzŋgɯzŋga}}}\markboth{anɤkhɤzŋgɯzŋga}{}
\begin{relation-sémantique}\confer{
\hyperlink{Ⓔnɤkhɤzŋga}{\textit{ \papi{nɤkhɤzŋga}}}
}\end{relation-sémantique}\end{entrée}

\begin{entrée}
\vedette{\hypertarget{Ⓔanɤmtsɯmtsioʁ}{\papi{ anɤmtsɯmtsioʁ}}}\markboth{anɤmtsɯmtsioʁ}{}
\begin{relation-sémantique}\confer{
\hyperlink{Ⓔnɤmtsioʁ}{\textit{ \papi{nɤmtsioʁ}}}
}\end{relation-sémantique}
\end{entrée}

\begin{entrée}
\vedette{\hypertarget{Ⓔanɤntshɯntshi}{\papi{ anɤntshɯntshi}}}\markboth{anɤntshɯntshi}{}
\begin{relation-sémantique}\confer{
\hyperlink{Ⓔnɤntshi}{\textit{ \papi{nɤntshi}}}
}\end{relation-sémantique}\end{entrée}

\begin{entrée}
\vedette{\hypertarget{Ⓔanɤrkɯrko}{\papi{ anɤrkɯrko}}}\markboth{anɤrkɯrko}{}\classe{vi}
\paradigme{\textit{dir :} \jya tɤ-}
\paradigme{\textit{dir :} \jya pɯ-}
\begin{définition}\fra se forcer les un les autres\end{définition}
\begin{définition}\cmn 互相逼迫\end{définition}
\begin{exemple}\jya jiɕqha tɯrme nɯ kɯ ``tʂha kɤ-tshi ra" ɲɯ-ti ri, aʑo ``mɤ-tshi-a" tɤ-tɯt-a ri, mɯ́j-khɯ tɕe tɤ-anɤrkɯrko-tɕi\cmn 那个人叫我喝茶,我说不喝,我们争了一段时间\end{exemple}
\begin{exemple}\jya ɯʑo kɯ ``nɤʑo kɤ-rɤʑi, aj ju-ɕe-a" ɲɯ-ti ri, aʑo kɯ ``nɤʑo kɤ-rɤʑi tɕe aj ju-ɕe-a" tɤ-tɯt-a tɕe, tɤ-anɤrkɯrko-tɕi\cmn 那个人叫我留下,说他自己会去,我叫他留下,说我自己去,互相争了一下\end{exemple}
\begin{relation-sémantique}\confer{
\hyperlink{ⒺnɤrkoⒽ1}{\textit{ \papi{nɤrko1}}}
}\end{relation-sémantique}\end{entrée}

\begin{entrée}
\vedette{\hypertarget{Ⓔanɤrpɯrpaʁ}{\papi{ anɤrpɯrpaʁ}}}\markboth{anɤrpɯrpaʁ}{}
\classe{vi}
\paradigme{\textit{dir :} \jya nɯ-}
\begin{définition}\fra s'entendre bien ensemble\end{définition}
\begin{définition}\cmn 合得来\end{définition}
\begin{exemple}\jya tɕiʑo tɯtsɣe kɤβzu ɲɯ-ɤnɤrpɯrpaʁ-tɕi\cmn 我们做生意很合得来\end{exemple}
\begin{exemple}\jya tɕiʑo ki tshoŋ βzu-tɕi tɕe, anɤrpɯrpaʁ-tɕi wo\cmn 我们做生意,非常投合\end{exemple}
\begin{exemple}\jya ɯʑo cho ɲɯ-ɤnɤrpɯrpaʁ-tɕi\cmn 我们俩关系融洽\end{exemple}
\begin{relation-sémantique}\confer{
\hyperlink{Ⓔaɣɯrpaʁ}{\textit{ \papi{aɣɯrpaʁ}}}
}\end{relation-sémantique}
\begin{relation-sémantique}\confer{
\hyperlink{Ⓔnɤrpaʁ}{\textit{ \papi{nɤrpaʁ}}}
}\end{relation-sémantique}\end{entrée}

\begin{entrée}
\vedette{\hypertarget{Ⓔanɤrɯre}{\papi{ anɤrɯre}}}\markboth{anɤrɯre}{}
\begin{relation-sémantique}\confer{
 \papi{nɤre}
}\end{relation-sémantique}\end{entrée}

\begin{entrée}
\vedette{\hypertarget{Ⓔanɤsɯso}{\papi{ anɤsɯso}}}\markboth{anɤsɯso}{}
\classe{vi}
\paradigme{\textit{dir :} \jya tɤ-}
\begin{définition}\ 
\begin{déclaration}\grammar{recip}\end{déclaration}\end{définition}
\begin{définition}\fra se manquer l'un à l'autre\end{définition}
\begin{définition}\cmn 互相想念\end{définition}
\begin{exemple}\jya jiɕqha nɯ cho ɲɯ-ɤnɤsɯso-ndʑi\cmn 他跟这个人互相想念着对方\end{exemple}
\begin{exemple}\jya anɤsɯso-ndʑi\cmn 他们俩互相想念着对方\end{exemple}
\begin{relation-sémantique}\confer{
\hyperlink{Ⓔsɯso}{\textit{ \papi{sɯso}}}
}\end{relation-sémantique}\end{entrée}

\begin{entrée}
\vedette{\hypertarget{Ⓔanɤtsɯtsɯ}{\papi{ anɤtsɯtsɯ}}}\markboth{anɤtsɯtsɯ}{}\classe{vs}
\begin{définition}\fra être caché\end{définition}
\begin{définition}\cmn 隐藏着\end{définition}
\begin{relation-sémantique}\confer{
\hyperlink{Ⓔnɤtsɯ}{\textit{ \papi{nɤtsɯ}}}
}\end{relation-sémantique}\end{entrée}

\begin{entrée}
\vedette{\hypertarget{Ⓔanɤtɯtɯɣ}{\papi{ anɤtɯtɯɣ}}}\markboth{anɤtɯtɯɣ}{}
\begin{relation-sémantique}\confer{
\hyperlink{Ⓔatɯɣ}{\textit{ \papi{atɯɣ}}}
}\end{relation-sémantique}\end{entrée}

\begin{entrée}
\vedette{\hypertarget{Ⓔanɤʑɤmŋɯmŋɤn}{\papi{ anɤʑɤmŋɯmŋɤn}}}\markboth{anɤʑɤmŋɯmŋɤn}{}
\begin{relation-sémantique}\confer{
\hyperlink{Ⓔnɤʑɤmŋɤn}{\textit{ \papi{nɤʑɤmŋɤn}}}
}\end{relation-sémantique}\end{entrée}

\begin{entrée}
\vedette{\hypertarget{Ⓔanbaʁ}{\papi{ anbaʁ}}}\markboth{anbaʁ}{}
\classe{vi}
\paradigme{\textit{dir :} \jya kɤ-}
\begin{définition}\fra se cacher\end{définition}
\begin{définition}\cmn 躲藏\end{définition}
\begin{exemple}\jya pjɯ-kɯ-mto mɯ-tɤ-pe tɕe, ku-kɯ-ɤnbaʁ ra\cmn 不适合被人发现的时候,就应该躲起来\end{exemple}
\begin{exemple}\jya nɤʑo cischiz ɕ-kɤ-ɤnbaʁ tɕe a-mɤ-pɯ-tɯ́-wɣ-mto\cmn 你在某个地方躲起来,免得别人看见你\end{exemple}
\begin{exemple}\jya ku-tɯ-ɤnbaʁ mɤ-ra ma tɤɣa jɤ-ɣi\cmn 你别躲起来了,出来\end{exemple}\begin{sous-entrée}
\vedette{\hypertarget{}{\papi{ nɤnbaʁ}}}\markboth{nɤnbaʁ}{}\classe{vt}
\paradigme{\textit{dir :} \jya kɤ-}
\begin{définition}\ 
\begin{déclaration}\grammar{appl}\end{déclaration}\end{définition}
\begin{définition}\fra éviter, se cacher de\end{définition}
\begin{définition}\cmn 躲避\end{définition}
\end{sous-entrée}\begin{sous-entrée}
\vedette{\hypertarget{}{\papi{ nɤnbɯnbaʁ}}}\markboth{nɤnbɯnbaʁ}{}\classe{vi}
\begin{définition}\fra se cacher partout\end{définition}
\begin{définition}\cmn 躲来躲去\end{définition}
\end{sous-entrée}\begin{sous-entrée}
\vedette{\hypertarget{}{\papi{ sɤnbaʁ}}}\markboth{sɤnbaʁ}{}\classe{vt}
\paradigme{\textit{dir :} \jya kɤ-}
\begin{définition}\fra cacher\end{définition}
\begin{définition}\cmn 藏起来\end{définition}
\begin{exemple}\jya nɤki nɯ kɤ-sɤnbaʁ tɕe, a-mɤ-pɯ-mto-nɯ\cmn 你把这个藏起来,免得他们看见\end{exemple}
\end{sous-entrée}\begin{sous-entrée}
\vedette{\hypertarget{}{\papi{ ʑɣɤsɤnbaʁ}}}\markboth{ʑɣɤsɤnbaʁ}{}\classe{vi}
\paradigme{\textit{dir :} \jya kɤ-}
\begin{définition}\fra se cacher\end{définition}
\begin{définition}\cmn 躲起来\end{définition}
\begin{exemple}\jya ko-tɯ-ʑɣɤsɤnbaʁ ɕti tɕe mɯ-pɯ-ta-mto\cmn 你躲起来了,我没有看到你\end{exemple}
\end{sous-entrée}\end{entrée}

\begin{entrée}
\vedette{\hypertarget{Ⓔandu}{\papi{ andu}}}\markboth{andu}{}
\classe{vi}
\paradigme{\textit{dir :} \jya nɯ-}
\begin{définition}\fra s'échanger\end{définition}
\begin{définition}\cmn 兑换\end{définition}
\begin{exemple}\jya ɕku pɯ-rkɯn ɕti ri sqɯ-mpɕar nɯ-andu\cmn 大蒜很少,要十块钱\end{exemple}
\begin{exemple}\jya tɯ-tɯrpa kɯ sqɯ-mpɕar andu\cmn 一斤可以卖十块钱\end{exemple}
\begin{relation-sémantique}\confer{
\hyperlink{Ⓔsɤndu}{\textit{ \papi{sɤndu}}}
}\end{relation-sémantique}\end{entrée}

\begin{entrée}
\vedette{\hypertarget{Ⓔandɤko}{\papi{ andɤko}}}\markboth{andɤko}{}\classe{vi}
\paradigme{\textit{dir :} \jya nɯ-}
\begin{définition}\fra se tendre (allongé, horizontalement)\end{définition}
\begin{définition}\cmn 横着伸直\end{définition}
\begin{exemple}\jya qapri tʂɤχcɤl ʑo ɲɤ-k-ɤndɤko-ci\cmn 蛇在路中间\end{exemple}
\begin{exemple}\jya tɤ-ri nɯ pɯ-ajʁu ɕti ri, ɲɤ-k-ɤndɤko-ci\cmn 线原来是弯的,现在变直了\end{exemple}\begin{sous-entrée}
\vedette{\hypertarget{}{\papi{ sɤndɤko}}}\markboth{sɤndɤko}{}
\begin{exemple}\jya kɯki ɲɯ-ɤjʁu tɕe, ɲɯ-sɤndɤkam-a\cmn 这个东西是弯着的,我把它拉直了\end{exemple}
\begin{relation-sémantique}\synonyme{
\hyperlink{Ⓔastɤko}{\textit{ \papi{astɤko}}}
}\end{relation-sémantique}
\end{sous-entrée}\end{entrée}

\begin{entrée}
\vedette{\hypertarget{Ⓔandɤr}{\papi{ andɤr}}}\markboth{andɤr}{}\classe{vi}
\paradigme{\textit{dir :} \jya nɯ-}
\begin{définition}\fra être touché (blessure)\end{définition}
\begin{définition}\cmn 被碰(疮、伤口)\end{définition}
\begin{exemple}\jya a-ʑmbɤr nɯ-andɤr\cmn 碰了一下我的疮\end{exemple}
\begin{exemple}\jya nɤ-jaʁ pjɤ-ɴɢraʁ ri, ko-tɯ-nɯ-sɯrput tɕe ɲɤ-k-ɤndɤr-ci\cmn 你的手受伤了,你不小心就碰了一下\end{exemple}
\begin{relation-sémantique}\confer{
\hyperlink{ⒺsɤndɤrⒽ1}{\textit{ \papi{sɤndɤr1}}}
}\end{relation-sémantique}\end{entrée}

\begin{entrée}
\vedette{\hypertarget{Ⓔandi}{\papi{ andi}}}\markboth{andi}{}\classe{adv}
\begin{définition}\fra à l'ouest\end{définition}
\begin{définition}\cmn 在西边\end{définition}\end{entrée}

\begin{entrée}
\vedette{\hypertarget{Ⓔandichoʁle}{\papi{ andichoʁle}}}\markboth{andichoʁle}{}\classe{n}
\begin{définition}\fra vent du sud\end{définition}
\begin{définition}\cmn 南风\end{définition}
\begin{relation-sémantique}\confer{
\hyperlink{Ⓔqale}{\textit{ \papi{qale}}}
}\end{relation-sémantique}
\begin{relation-sémantique}\confer{
\hyperlink{Ⓔakɯchoʁle}{\textit{ \papi{akɯchoʁle}}}
}\end{relation-sémantique}\end{entrée}

\begin{entrée}
\vedette{\hypertarget{Ⓔando}{\papi{ ando}}}\markboth{ando}{}\classe{vi}
\paradigme{\textit{dir :} \jya tɤ-}
\begin{définition}\ 
\begin{déclaration}\grammar{pass}\end{déclaration}\end{définition}
\begin{définition}\fra être pris\end{définition}
\begin{définition}\cmn 带上\end{définition}
\begin{exemple}\jya ando ma tɤ-fkɯm ɯ-ŋgɯ arku\cmn 带上了,在口袋里\end{exemple}
\begin{relation-sémantique}\confer{
\hyperlink{Ⓔndo}{\textit{ \papi{ndo}}}
}\end{relation-sémantique}\end{entrée}

\begin{entrée}
\vedette{\hypertarget{Ⓔandɯja}{\papi{ andɯja}}}\markboth{andɯja}{}
\classe{vi}
\paradigme{\textit{dir :} \jya tɤ-}
\begin{définition}\fra se rassembler\end{définition}
\begin{définition}\cmn 聚集\end{définition}
\begin{exemple}\jya to-k-ɤndɯja-nɯ-ci\cmn 他们聚在一起了\end{exemple}
\begin{exemple}\jya jiʑora jisŋi ta-ndɯja-j\cmn 我们今天聚会了\end{exemple}\begin{sous-entrée}
\vedette{\hypertarget{}{\papi{ sɤndɯja}}}\markboth{sɤndɯja}{}\classe{vt}
\paradigme{\textit{dir :} \jya tɤ-}
\begin{définition}\ 
\begin{déclaration}\grammar{caus}\end{déclaration}\end{définition}
\begin{définition}\fra rassembler\end{définition}
\begin{définition}\cmn 召集;聚集\end{définition}
\begin{exemple}\jya tɤ-pɤtso ra tɤ-sɤndɯjat-a-nɯ\cmn 我把孩子们聚集在一起了\end{exemple}
\end{sous-entrée}\end{entrée}

\begin{entrée}
\vedette{\hypertarget{Ⓔandɯndo}{\papi{ andɯndo}}}\markboth{andɯndo}{}
\begin{relation-sémantique}\confer{
\hyperlink{Ⓔndo}{\textit{ \papi{ndo}}}
}\end{relation-sémantique}\end{entrée}

\begin{entrée}
\vedette{\hypertarget{Ⓔandzoʁjoʁ}{\papi{ andzoʁjoʁ}}}\markboth{andzoʁjoʁ}{}\classe{vi}
\paradigme{\textit{dir :} \jya kɤ-}
\begin{définition}\fra coller ensemble\end{définition}
\begin{définition}\cmn 粘在一起\end{définition}
\begin{exemple}\jya laχtɕha ɲɯ-ɤndzoʁjoʁ\cmn 东西粘在一起\end{exemple}\begin{sous-entrée}
\vedette{\hypertarget{}{\papi{ sɤndzoʁjoʁ}}}\markboth{sɤndzoʁjoʁ}{}\classe{vt}
\paradigme{\textit{dir :} \jya kɤ-}
\begin{définition}\fra faire coller ensemble\end{définition}
\begin{définition}\cmn 连接在一起\end{définition}
\end{sous-entrée}\end{entrée}

\begin{entrée}
\vedette{\hypertarget{Ⓔandzɯndza}{\papi{ andzɯndza}}}\markboth{andzɯndza}{}
\classe{vi}
\paradigme{\textit{dir :} \jya tɤ-}
\begin{définition}\ 
\begin{déclaration}\grammar{recip}\end{déclaration}\end{définition}\acception{1}
\begin{définition}\fra se manger les uns les autres\end{définition}
\begin{définition}\cmn 互相吃\end{définition}\acception{2}
\begin{définition}\fra se faire du mal les uns aux autres\end{définition}
\begin{définition}\cmn 互相伤害\end{définition}
\begin{exemple}\jya jiɕqha nɯ to-k-ɤnɯmqaj-ndʑi tɕe tu-ondzɯndza-ndʑi ɲɯ-ŋu\cmn 他们俩吵架了,互相伤害对方\end{exemple}
\begin{relation-sémantique}\confer{
\hyperlink{Ⓔndza}{\textit{ \papi{ndza}}}
}\end{relation-sémantique}\end{entrée}

\begin{entrée}
\vedette{\hypertarget{Ⓔandzɯndzri}{\papi{ andzɯndzri}}}\markboth{andzɯndzri}{}
\begin{relation-sémantique}\confer{
\hyperlink{Ⓔndzri}{\textit{ \papi{ndzri}}}
}\end{relation-sémantique}\end{entrée}

\begin{entrée}
\vedette{\hypertarget{Ⓔandzɯqoʁ}{\papi{ andzɯqoʁ}}}\markboth{andzɯqoʁ}{}
\classe{vi}
\paradigme{\textit{dir :} \jya tɤ-}
\begin{définition}\fra hoqueter\end{définition}
\begin{définition}\cmn 打嗝儿\end{définition}
\begin{exemple}\jya to-k-ɤndzɯqoʁ-ci\cmn 他打嗝了\end{exemple}
\begin{exemple}\jya ma-tɤ-tɯ-ɤndzɯqoʁ ntsɯ ma ɲɯ-sɤɣdɯɣ\cmn 你不要总是打嗝,很讨厌\end{exemple}\end{entrée}

\begin{entrée}
\vedette{\hypertarget{Ⓔandzɯt}{\papi{ andzɯt}}}\markboth{andzɯt}{}
\classe{vi}
\paradigme{\textit{dir :} \jya tɤ-}
\begin{définition}\fra aboyer\end{définition}
\begin{définition}\cmn 狗吠\end{définition}
\begin{exemple}\jya khɯna ɲɯ-ɤndzɯt\cmn 狗在叫\end{exemple}
\begin{exemple}\jya khɯna to-k-ɤndzɯt-ci\cmn 狗叫了一声\end{exemple}
\begin{exemple}\jya qapar ɲɯ-ɤndzɯt\cmn 豺狗在叫\end{exemple}\begin{sous-entrée}
\vedette{\hypertarget{}{\papi{ nɤndzɯt}}}\markboth{nɤndzɯt}{}\classe{vt}
\paradigme{\textit{dir :} \jya tɤ-}
\begin{définition}\ 
\begin{déclaration}\grammar{appl}\end{déclaration}\end{définition}
\begin{définition}\fra aboyer sur qqn\end{définition}
\begin{définition}\cmn 对(某人)吠\end{définition}
\end{sous-entrée}\end{entrée}

\begin{entrée}
\vedette{\hypertarget{Ⓔandʑɤmstu}{\papi{ andʑɤmstu}}}\markboth{andʑɤmstu}{}\classe{vs}
\begin{définition}\fra lisse (tissu)\end{définition}
\begin{définition}\cmn 平整(衣服)
\begin{déclaration}\grammar{comp}\end{déclaration}\end{définition}
\begin{sous-entrée}
\vedette{\hypertarget{}{\papi{ sɤndʑɤmstu}}}\markboth{sɤndʑɤmstu}{}\classe{vt}
\paradigme{\textit{dir :} \jya nɯ-}
\begin{définition}\fra repasser (vêtement)\end{définition}
\begin{définition}\cmn 熨平\end{définition}
\begin{exemple}\jya ma-tɤ-tɯ-mphɯr, nɯ-sɤndʑɤmste\cmn 不要裹起来,把它摆平\end{exemple}
\begin{exemple}\jya tɯ-ŋga nɯ-sɤndʑɤmstu-t-a\cmn 我(用熨斗)把衣服熨平了\end{exemple}
\begin{relation-sémantique}\confer{
\hyperlink{Ⓔndʑɤm}{\textit{ \papi{ndʑɤm}}}
}\end{relation-sémantique}
\begin{relation-sémantique}\confer{
\hyperlink{Ⓔastu}{\textit{ \papi{astu}}}
}\end{relation-sémantique}
\end{sous-entrée}\end{entrée}

\begin{entrée}
\vedette{\hypertarget{Ⓔandʑɯɣndʑɯɣ}{\papi{ andʑɯɣndʑɯɣ}}}\markboth{andʑɯɣndʑɯɣ}{}
\classe{vs}
\begin{définition}\fra l'un à côté de l'autre\end{définition}
\begin{définition}\cmn 一个挨着一个\end{définition}
\begin{exemple}\jya ɯ-ɕɣa wuma ɲɯ-ɤndʑɯɣndʑɯɣ, mɯ-ɲɯ-ɤχa\cmn 他的牙齿很密,没有洞\end{exemple}\end{entrée}

\begin{entrée}
\vedette{\hypertarget{Ⓔandʑɯndʑu}{\papi{ andʑɯndʑu}}}\markboth{andʑɯndʑu}{}\classe{vs}
\paradigme{\textit{dir :} \jya nɯ-}
\begin{définition}\fra raide (cadavre)\end{définition}
\begin{définition}\cmn 僵直\end{définition}
\begin{exemple}\jya pjɤ-si tɕe ɲɤ-k-ɤndʑɯndʑu-ci\cmn 他死了就僵直了\end{exemple}
\begin{relation-sémantique}\confer{
\hyperlink{Ⓔndʑu}{\textit{ \papi{ndʑu}}}
}\end{relation-sémantique}\end{entrée}

\begin{entrée}
\vedette{\hypertarget{Ⓔandʑɯrɣa}{\papi{ andʑɯrɣa}}}\markboth{andʑɯrɣa}{}\classe{vs}
\paradigme{\textit{dir :} \jya tɤ-}
\begin{définition}\fra être voisins\end{définition}
\begin{définition}\cmn 住在一起;挨着的\end{définition}
\begin{relation-sémantique}\confer{
\hyperlink{Ⓔtɤ-rɣa}{\textit{ \papi{tɤ-rɣa}}}
}\end{relation-sémantique}\end{entrée}

\begin{entrée}
\vedette{\hypertarget{Ⓔangɯt}{\papi{ angɯt}}}\markboth{angɯt}{}\classe{vs}
\begin{définition}\fra commun, que l'on possède ensemble\end{définition}
\begin{définition}\cmn 共同的\end{définition}
\begin{exemple}\jya angɯt-i\cmn 我们都有一份\end{exemple}
\begin{relation-sémantique}\synonyme{
\hyperlink{Ⓔalpɯm}{\textit{ \papi{alpɯm}}}
}\end{relation-sémantique}
\begin{relation-sémantique}\confer{
\hyperlink{Ⓔnɤngɯt}{\textit{ \papi{nɤngɯt}}}
}\end{relation-sémantique}\end{entrée}

\begin{entrée}
\vedette{\hypertarget{Ⓔantɤm}{\papi{ antɤm}}}\markboth{antɤm}{}
\classe{vi}
\paradigme{\textit{dir :} \jya nɯ-}
\begin{définition}\fra plat\end{définition}
\begin{définition}\cmn 平(地)\end{définition}
\begin{exemple}\jya stɤmku ɲɯ-ɤntɤm\cmn 草地很平\end{exemple}
\begin{exemple}\jya khɤxtu antɤm\cmn 房背很平\end{exemple}
\begin{exemple}\jya ɯ-thoʁ antɤm\cmn 地很平\end{exemple}
\begin{relation-sémantique}\confer{
\hyperlink{Ⓔapjɤntɤm}{\textit{ \papi{apjɤntɤm}}}
}\end{relation-sémantique}\end{entrée}

\begin{entrée}
\vedette{\hypertarget{Ⓔantɕhoʁjɤr}{\papi{ antɕhoʁjɤr}}}\markboth{antɕhoʁjɤr}{}\classe{vi}
\paradigme{\textit{dir :} \jya nɯ-}
\begin{définition}\fra être incomplet\end{définition}
\begin{définition}\cmn 疏漏\end{définition}
\begin{exemple}\jya nɤ-laχtɕha koŋla tɤ-rɤwum tɕe a-mɤ-nɯ-ɤntɕhoʁjɤr\cmn 你把东西收拾好,不要漏掉\end{exemple}
\begin{exemple}\jya ɯβrɤ-ɲɤ-k-ɤntɕhoʁjɤr-ci kɯ\cmn 应该没有疏漏了吧\end{exemple}
\begin{relation-sémantique}\synonyme{
\hyperlink{Ⓔatɕɯtɕit}{\textit{ \papi{atɕɯtɕit}}}
}\end{relation-sémantique}\begin{sous-entrée}
\vedette{\hypertarget{}{\papi{ sɤntɕhoʁjɤr}}}\markboth{sɤntɕhoʁjɤr}{}\classe{vt}
\end{sous-entrée}\end{entrée}

\begin{entrée}
\vedette{\hypertarget{Ⓔantɕhɯ}{\papi{ antɕhɯ}}}\markboth{antɕhɯ}{}\classe{vs}
\paradigme{\textit{dir :} \jya nɯ-}
\begin{définition}\fra nombreux\end{définition}
\begin{définition}\cmn 数量多\end{définition}
\begin{exemple}\jya jisŋi tɯrme ɲɯ-ɤntɕhɯ-nɯ tɕe ndzɤtshi khro tsa kɤ-βzu ra\cmn 今天人很多,要多做一点饭菜\end{exemple}
\begin{exemple}\jya aʑo ji-rɣa ra nɯ nɯ-kha kɤntɕhɯ-ɣjɤn jɤ-ari-a\cmn 我去了我们邻居的家很多次\end{exemple}
\begin{exemple}\jya aj @chengdu kɤntɕhɯ-ɣjɤn thɯ-ɣe-a\cmn 我来过成都很多次\end{exemple}\begin{sous-entrée}
\vedette{\hypertarget{}{\papi{ sɤntɕhɯ}}}\markboth{sɤntɕhɯ}{}\classe{vt}
\begin{exemple}\jya a-ŋga tɤ-sɤntɕhɯ-t-a / a-ŋga kɯ-ɤntɕhɯ tɤ-ŋga-t-a\cmn 我穿了好几件衣服\end{exemple}
\begin{exemple}\jya rɟɤɣi tɤ-sɤntɕhɯ-t-a\cmn 我吃了好几碗糌粑\end{exemple}
\end{sous-entrée}\end{entrée}

\begin{entrée}
\vedette{\hypertarget{Ⓔanthɣar}{\papi{ anthɣar}}}\markboth{anthɣar}{}
\classe{vi}\acception{1}
\paradigme{\textit{dir :} \jya tɤ-}
\begin{définition}\fra rebondir\end{définition}
\begin{définition}\cmn 弹起来
\begin{déclaration}\use{用于细小的东西;球弹起来必须用\stylefv{mtsaʁ}“跳”}\end{déclaration}\end{définition}
\begin{exemple}\jya tɤ-k-ɤnthɣar-ci\cmn 弹起来了\end{exemple}\acception{2}
\paradigme{\textit{dir :} \jya nɯ-}
\begin{définition}\fra être perdu\end{définition}
\begin{définition}\cmn 丢失了\end{définition}
\begin{exemple}\jya a-taqaβ ɲɤ-k-ɤnthɣar-ci tɕe maŋe\cmn 我的针不见了\end{exemple}\begin{sous-entrée}
\vedette{\hypertarget{}{\papi{ sɤnthɣar}}}\markboth{sɤnthɣar}{}\classe{vt}
\paradigme{\textit{dir :} \jya nɯ-}
\begin{définition}\fra perdre\end{définition}
\begin{définition}\cmn 丢失\end{définition}
\begin{exemple}\jya a-tɤ-fkɯm ɲɤ-nɯ-sɤnthɣar-a tɕe maŋe\cmn 我把袋子丢失了,看不见了\end{exemple}
\end{sous-entrée}\end{entrée}

\begin{entrée}
\vedette{\hypertarget{Ⓔantsɤndu}{\papi{ antsɤndu}}}\markboth{antsɤndu}{}
\classe{vi}
\paradigme{\textit{dir :} \jya nɯ-}
\paradigme{\textit{dir :} \jya tɤ-}
\begin{définition}\fra être échangé par erreur\end{définition}
\begin{définition}\cmn 无意中被交换\end{définition}
\begin{exemple}\jya laχtɕha ɲɯ-ɤntsɤndu\cmn 东西不小心调换了\end{exemple}
\begin{exemple}\jya tɕiʑo tɕi-ŋga nɯ ɲɤ-k-ɤntsɤndu-ci\cmn 我们俩的衣服不小心弄错了\end{exemple}
\begin{exemple}\jya tɯ-rju ɯ-qhu ɯ-ʁɤri ɲɤ-k-ɤntsɤndu-ci\cmn 颠倒说话了\end{exemple}
\begin{relation-sémantique}\confer{
\hyperlink{Ⓔsɤntsɤndu}{\textit{ \papi{sɤntsɤndu}}}
}\end{relation-sémantique}
\begin{relation-sémantique}\confer{
\hyperlink{Ⓔsɤndu}{\textit{ \papi{sɤndu}}}
}\end{relation-sémantique}\end{entrée}

\begin{entrée}
\vedette{\hypertarget{Ⓔantɯ}{\papi{ antɯ}}}\markboth{antɯ}{}\classe{vi}
\begin{définition}\fra avoir l'ouverture tournée vers le haut\end{définition}
\begin{définition}\cmn 放平,口朝上\end{définition}
\begin{exemple}\jya sɤlaŋphɤn antɯ tɕe pe ma nɯ maʁ tɯ-ci jit ɕti\cmn 盆子要放平,不然水流出来\end{exemple}\begin{sous-entrée}
\vedette{\hypertarget{}{\papi{ sɤntɯ}}}\markboth{sɤntɯ}{}\classe{vt}
\paradigme{\textit{dir :} \jya tɤ-}
\begin{exemple}\jya ki tú-wɣ-sɤntɯ ra ma nɯ maʁ nɤ jit\cmn 要把口朝上,不然水会流出来\end{exemple}
\begin{relation-sémantique}\antonyme{
\hyperlink{Ⓔβʁum}{\textit{ \papi{βʁum}}}
}\end{relation-sémantique}
\end{sous-entrée}\end{entrée}

\begin{entrée}
\vedette{\hypertarget{Ⓔanɯɕqhɯɕqhu}{\papi{ anɯɕqhɯɕqhu}}}\markboth{anɯɕqhɯɕqhu}{}
\begin{relation-sémantique}\confer{
\hyperlink{Ⓔnɯɕqhu}{\textit{ \papi{nɯɕqhu}}}
}\end{relation-sémantique}
\end{entrée}

\begin{entrée}
\vedette{\hypertarget{Ⓔanɯɣbɯɣbɯɣ}{\papi{ anɯɣbɯɣbɯɣ}}}\markboth{anɯɣbɯɣbɯɣ}{}
\begin{relation-sémantique}\confer{
\hyperlink{Ⓔnɯɣbɯɣ}{\textit{ \papi{nɯɣbɯɣ}}}
}\end{relation-sémantique}\end{entrée}

\begin{entrée}
\vedette{\hypertarget{Ⓔanɯkɯjŋɤŋgɯ}{\papi{ anɯkɯjŋɤŋgɯ}}}\markboth{anɯkɯjŋɤŋgɯ}{}
\classe{vi}
\paradigme{\textit{dir :} \jya nɯ-}
\begin{définition}\ 
\begin{déclaration}\grammar{recip}\end{déclaration}\end{définition}
\begin{définition}\fra se jurer l'un à l'autre\end{définition}
\begin{définition}\cmn 互相立下誓言\end{définition}
\begin{exemple}\jya ɲɤ-k-ɤnɯkɯjŋɤŋgɯ-ndʑi\cmn 他们俩互相立下了誓言\end{exemple}
\begin{relation-sémantique}\synonyme{
\hyperlink{Ⓔanɯkɯjŋɯjŋu}{\textit{ \papi{anɯkɯjŋɯjŋu}}}
}\end{relation-sémantique}
\begin{relation-sémantique}\synonyme{
\hyperlink{Ⓔnɯkɯjŋu}{\textit{ \papi{nɯkɯjŋu}}}
}\end{relation-sémantique}\end{entrée}

\begin{entrée}
\vedette{\hypertarget{Ⓔanɯkɯjŋɯjŋu}{\papi{ anɯkɯjŋɯjŋu}}}\markboth{anɯkɯjŋɯjŋu}{}
\classe{vi}
\paradigme{\textit{dir :} \jya tɤ-}
\paradigme{\textit{dir :} \jya nɯ-}
\begin{définition}\ 
\begin{déclaration}\grammar{recip}\end{déclaration}\end{définition}
\begin{définition}\fra se jurer l'un à l'autre\end{définition}
\begin{définition}\cmn 互相立下誓言\end{définition}
\begin{exemple}\jya to-k-ɤnɯkɯjŋɯjŋu-ndʑi\cmn 他们俩互相立下了誓言\end{exemple}
\begin{relation-sémantique}\synonyme{
 \papi{anɯkhɯjŋɤŋgɯ}
}\end{relation-sémantique}
\begin{relation-sémantique}\confer{
\hyperlink{Ⓔnɯkɯjŋu}{\textit{ \papi{nɯkɯjŋu}}}
}\end{relation-sémantique}\end{entrée}

\begin{entrée}
\vedette{\hypertarget{Ⓔanɯmbɯmbɣom}{\papi{ anɯmbɯmbɣom}}}\markboth{anɯmbɯmbɣom}{}
\begin{relation-sémantique}\confer{
\hyperlink{Ⓔnɯmbɣom}{\textit{ \papi{nɯmbɣom}}}
}\end{relation-sémantique}\end{entrée}

\begin{entrée}
\vedette{\hypertarget{Ⓔanɯmqaj}{\papi{ anɯmqaj}}}\markboth{anɯmqaj}{}
\classe{vi}
\paradigme{\textit{dir :} \jya tɤ-}
\begin{définition}\fra se disputer\end{définition}
\begin{définition}\cmn 吵架\end{définition}
\begin{exemple}\jya jiɕqha kɤtsa ra ɲɯ-ɤnɯmqaj-nɯ\cmn 他们一家在吵架\end{exemple}
\begin{exemple}\jya to-k-ɤnɯmqaj-ndʑi\cmn 他们吵架了\end{exemple}
\begin{relation-sémantique}\confer{
\hyperlink{Ⓔtɯ-mqaj}{\textit{ \papi{tɯ-mqaj}}}
}\end{relation-sémantique}
\begin{relation-sémantique}\confer{
\hyperlink{Ⓔamqaj}{\textit{ \papi{amqaj}}}
}\end{relation-sémantique}\end{entrée}

\begin{entrée}
\vedette{\hypertarget{Ⓔanɯndzɤqɯqɤr}{\papi{ anɯndzɤqɯqɤr}}}\markboth{anɯndzɤqɯqɤr}{}
\begin{relation-sémantique}\confer{
\hyperlink{Ⓔnɯndzɤqɤr}{\textit{ \papi{nɯndzɤqɤr}}}
}\end{relation-sémantique}\end{entrée}

\begin{entrée}
\vedette{\hypertarget{Ⓔanɯpɕɯpɕoʁ}{\papi{ anɯpɕɯpɕoʁ}}}\markboth{anɯpɕɯpɕoʁ}{}
\classe{vi}
\paradigme{\textit{dir :} \jya tɤ-}
\begin{définition}\ 
\begin{déclaration}\grammar{denom}\end{déclaration}\end{définition}
\begin{définition}\fra tourné dans la même direction\end{définition}
\begin{définition}\cmn 朝同一个方向;方向准确\end{définition}
\begin{exemple}\jya jiɕqha nɯ kɯ-ɤnɯpɕɯpɕoʁ ci ɲɯ-ŋu\cmn 他朝同一个方向\end{exemple}
\begin{exemple}\jya ɲɯ-ɤnɯpɕɯpɕoʁ-ndʑi\cmn 他们俩朝同一个方向\end{exemple}
\begin{exemple}\jya ki tɯ-rju ʁnɯ-ŋka ki mɯ-ɲɯ-ɤnɯpɕɯpɕoʁ kɯ ɲɯ-ɤnɯɕqhɯɕqhu ɕti\cmn 这两句话不相符合,是矛盾的\end{exemple}
\begin{relation-sémantique}\confer{
\hyperlink{Ⓔtɯ-pɕoʁ}{\textit{ \papi{tɯ-pɕoʁ}}}
}\end{relation-sémantique}\end{entrée}

\begin{entrée}
\vedette{\hypertarget{Ⓔanɯphɤrɯri}{\papi{ anɯphɤrɯri}}}\markboth{anɯphɤrɯri}{}
\classe{vi}
\paradigme{\textit{dir :} \jya nɯ-}
\begin{définition}\ 
\begin{déclaration}\grammar{denom}\end{déclaration}\end{définition}
\begin{définition}\fra être l'un en face de l'autre\end{définition}
\begin{définition}\cmn 一个对着一个(隔着河、山对着山)\end{définition}
\begin{exemple}\jya kɯki jiʑora ɲɯ-ɤnɯphɤrɯri-j\cmn 我们隔着(河)相对\end{exemple}
\begin{exemple}\jya tɕɤndi phɤri ra cho ɲɯ-ɤnɯphɤrɯri-j\cmn 我们跟对面的(那一家人)隔着(河)相对\end{exemple}
\begin{exemple}\jya tɯji cho praʁ ɲɯ-ɤnɯphɤrɯri-ndʑi\cmn 这面山是田地,那一面是岩石(相对着)\end{exemple}
\begin{relation-sémantique}\confer{
\hyperlink{Ⓔphɤri}{\textit{ \papi{phɤri}}}
}\end{relation-sémantique}\end{entrée}

\begin{entrée}
\vedette{\hypertarget{Ⓔanɯphotɯtɯɣ}{\papi{ anɯphotɯtɯɣ}}}\markboth{anɯphotɯtɯɣ}{}\classe{vi}
\begin{définition}\fra moyen\end{définition}
\begin{définition}\cmn 恰当的;中等的;不多不少\end{définition}
\begin{exemple}\jya kɯ-ɤnɯphotɯtɯɣ ci ra\cmn 需要一个中等的\end{exemple}
\begin{exemple}\jya tɯ-rju nɯ ɣɯ ɯ-tsa tsa tɤ-βze, kɯ-ɤnɯphotɯtɯɣ tsa tɤ-βze a-mɤ-tɤ-tɕhom\cmn 话要讲得适当一些,不要太多\end{exemple}\end{entrée}

\begin{entrée}
\vedette{\hypertarget{Ⓔanɯrgɯrga}{\papi{ anɯrgɯrga}}}\markboth{anɯrgɯrga}{}
\begin{relation-sémantique}\confer{
\hyperlink{Ⓔnɯrga}{\textit{ \papi{nɯrga}}}
}\end{relation-sémantique}\end{entrée}

\begin{entrée}
\vedette{\hypertarget{Ⓔanɯrŋɤrɯru}{\papi{ anɯrŋɤrɯru}}}\markboth{anɯrŋɤrɯru}{}\classe{vi}
\paradigme{\textit{dir :} \jya kɤ-}
\begin{définition}\ 
\begin{déclaration}\grammar{refl}\end{déclaration}
\begin{déclaration}\grammar{incorp}\end{déclaration}\end{définition}
\begin{définition}\fra se regarder les uns les autres\end{définition}
\begin{définition}\cmn 互相望着\end{définition}
\begin{exemple}\jya tɕiʑo kɤ-anɯrŋɤrɯru-tɕi\cmn 我们俩互相望了一眼\end{exemple}
\begin{relation-sémantique}\confer{
\hyperlink{Ⓔtɯ-rŋa}{\textit{ \papi{tɯ-rŋa}}}
}\end{relation-sémantique}
\begin{relation-sémantique}\confer{
\hyperlink{ⒺruⒽ1}{\textit{ \papi{ru1}}}
}\end{relation-sémantique}
\end{entrée}

\begin{entrée}
\vedette{\hypertarget{Ⓔanɯrqhɯrqhu}{\papi{ anɯrqhɯrqhu}}}\markboth{anɯrqhɯrqhu}{}
\classe{vi}
\paradigme{\textit{dir :} \jya tɤ-}
\begin{définition}\fra l'un dos à l'autre\end{définition}
\begin{définition}\cmn 背对背\end{définition}
\begin{exemple}\jya tɕiʑo ni tɤ-anɯrqhɯrqhu-tɕi\cmn 我们俩背对着背\end{exemple}
\begin{exemple}\jya kɤ-nɯ-rŋgɯ-tɕi tɕe tɤ-anɯrqhɯrqhu-tɕi\cmn 我们俩睡觉的时候背对着背\end{exemple}
\begin{relation-sémantique}\confer{
\hyperlink{Ⓔɯ-qhu}{\textit{ \papi{ɯ-qhu}}}
}\end{relation-sémantique}\end{entrée}

\begin{entrée}
\vedette{\hypertarget{Ⓔanɯrɯɕmɯɕmi}{\papi{ anɯrɯɕmɯɕmi}}}\markboth{anɯrɯɕmɯɕmi}{}
\begin{relation-sémantique}\confer{
\hyperlink{Ⓔrɯɕmi}{\textit{ \papi{rɯɕmi}}}
}\end{relation-sémantique}\end{entrée}

\begin{entrée}
\vedette{\hypertarget{Ⓔanɯrɯtʂɯtʂa}{\papi{ anɯrɯtʂɯtʂa}}}\markboth{anɯrɯtʂɯtʂa}{}
\begin{relation-sémantique}\confer{
\hyperlink{Ⓔnɯrɯtʂa}{\textit{ \papi{nɯrɯtʂa}}}
}\end{relation-sémantique}\end{entrée}

\begin{entrée}
\vedette{\hypertarget{Ⓔanɯʁɤndɯndu}{\papi{ anɯʁɤndɯndu}}}\markboth{anɯʁɤndɯndu}{}\classe{vi}
\paradigme{\textit{dir :} \jya tɤ-}
\begin{définition}\fra s'échanger des travaux les uns les autres\end{définition}
\begin{définition}\cmn 互相换工\end{définition}
\begin{exemple}\jya tɕiʑo anɯʁɤndɯndu-tɕi\cmn 我们俩互相换工\end{exemple}
\begin{relation-sémantique}\confer{
\hyperlink{Ⓔtaʁɤndu}{\textit{ \papi{taʁɤndu}}}
}\end{relation-sémantique}\end{entrée}

\begin{entrée}
\vedette{\hypertarget{Ⓔanɯʁɤrɯri}{\papi{ anɯʁɤrɯri}}}\markboth{anɯʁɤrɯri}{}
\classe{vi}
\paradigme{\textit{dir :} \jya tɤ-}
\begin{définition}\ 
\begin{déclaration}\grammar{denom}\end{déclaration}\end{définition}
\begin{définition}\fra l'un en face de l'autre\end{définition}
\begin{définition}\cmn 面对面\end{définition}
\begin{exemple}\jya jiɕqha nɯni ɲɯ-ɤnɯʁɤrɯri-ndʑi\cmn 他们俩面对面\end{exemple}
\begin{exemple}\jya tɕiʑo ni tu-onɯʁɤrɯri-tɕi ŋu\cmn 我们俩面对面\end{exemple}
\begin{exemple}\jya to-k-ɤnɯʁɤrɯri-ndʑi-ci\cmn 他俩面对面\end{exemple}
\begin{exemple}\jya tu-onɯʁɤrɯri-ndʑi tɕe ku-omdzɯ-ndʑi\cmn 他们俩对着坐\end{exemple}
\begin{relation-sémantique}\confer{
\hyperlink{Ⓔɯ-ʁɤri}{\textit{ \papi{ɯ-ʁɤri}}}
}\end{relation-sémantique}\end{entrée}

\begin{entrée}
\vedette{\hypertarget{Ⓔanɯʁgrɯʁgra}{\papi{ anɯʁgrɯʁgra}}}\markboth{anɯʁgrɯʁgra}{}
\begin{relation-sémantique}\confer{
\hyperlink{Ⓔnɯʁgra}{\textit{ \papi{nɯʁgra}}}
}\end{relation-sémantique}\end{entrée}

\begin{entrée}
\vedette{\hypertarget{Ⓔanɯstɯstu}{\papi{ anɯstɯstu}}}\markboth{anɯstɯstu}{}
\classe{vi}
\paradigme{\textit{dir :} \jya tɤ-}
\begin{définition}\fra être sur une même ligne\end{définition}
\begin{définition}\cmn 呈一条直线\end{définition}
\begin{exemple}\jya khɯtsa cho ɕoʁɕoʁ ɲɯ-ɤnɯstɯstu-ndʑi\cmn 碗和纸在同一水平线上\end{exemple}
\begin{relation-sémantique}\synonyme{
\hyperlink{Ⓔanɯpɕɯpɕoʁ}{\textit{ \papi{anɯpɕɯpɕoʁ}}}
}\end{relation-sémantique}
\begin{relation-sémantique}\confer{
\hyperlink{Ⓔastu}{\textit{ \papi{astu}}}
}\end{relation-sémantique}\end{entrée}

\begin{entrée}
\vedette{\hypertarget{Ⓔanɯsɯkhɯkho}{\papi{ anɯsɯkhɯkho}}}\markboth{anɯsɯkhɯkho}{}
\classe{vi}
\paradigme{\textit{dir :} \jya nɯ-}
\begin{définition}\ 
\begin{déclaration}\grammar{recip}\end{déclaration}\end{définition}
\begin{définition}\fra se battre pour\end{définition}
\begin{définition}\cmn 争夺;互相抢\end{définition}
\begin{exemple}\jya paʁtsa ni ndʑi-ndza ɲɯ-ɤnɯsɯkhɯkho-ndʑi\cmn 两个小猪互相抢食物\end{exemple}
\begin{exemple}\jya tɤ-pɤtso ni ndʑi-kɯmtɕhɯ ɲɯ-ɤnɯsɯkhɯkho-ndʑi\cmn 两个小孩子互相抢玩具\end{exemple}
\begin{relation-sémantique}\confer{
\hyperlink{Ⓔnɯsɯkho}{\textit{ \papi{nɯsɯkho}}}
}\end{relation-sémantique}\end{entrée}

\begin{entrée}
\vedette{\hypertarget{Ⓔanɯsɯslaʁ}{\papi{ anɯsɯslaʁ}}}\markboth{anɯsɯslaʁ}{}\classe{vi}
\paradigme{\textit{dir :} \jya tɤ-}
\begin{définition}\fra assidu, qui travaille rapidement\end{définition}
\begin{définition}\cmn 勤快\end{définition}
\begin{exemple}\jya ta-ma ɲɯ-ɤnɯsɯslaʁ\cmn 他工作做得很快(很勤快)\end{exemple}
\begin{exemple}\jya nɯ ta-ma kɤ-nɤma tu-kɯ-ɤnɯsɯslaʁ ʑo jɤɣ wo\cmn 这个工作可以做得快一些(长辈对晚辈说的话)\end{exemple}
\begin{exemple}\jya kɯ-rɤma jɤ-tɯ-ari tɕe, a-tɤ-tɯ-ɤnɯsɯslaʁ ʑo je!\cmn 你去工作的时候要勤快一些!\end{exemple}
\begin{exemple}\jya ɯʑo ta-ma pe tɕe, tu-onɯsɯslaʁ ʑo ɕti\cmn 他工作得很好,很勤快\end{exemple}
\begin{exemple}\jya to-k-ɤnɯsɯslaʁ-ci tɕe, ʑaʑa ɲɤ-sthɯt\cmn 因为他很勤快所以早就完工了\end{exemple}\end{entrée}

\begin{entrée}
\vedette{\hypertarget{Ⓔaɲaj}{\papi{ aɲaj}}}\markboth{aɲaj}{}
\classe{vi}
\paradigme{\textit{dir :} \jya tɤ-}
\begin{définition}\fra rapide (travail)\end{définition}
\begin{définition}\cmn 迅速;快(工作)\end{définition}
\begin{exemple}\jya jiɕqha nɯ ɯ-ma ɲɯ-ɤɲaj\cmn 他工作得很快\end{exemple}
\begin{exemple}\jya aʑo a-ma aɲaj\cmn 我工作得很快\end{exemple}
\begin{exemple}\jya to-k-ɤɲaj-ci\cmn 比以前快\end{exemple}
\begin{relation-sémantique}\confer{
\hyperlink{Ⓔsɤɲaj}{\textit{ \papi{sɤɲaj}}}
}\end{relation-sémantique}\end{entrée}

\begin{entrée}
\vedette{\hypertarget{Ⓔaɲaʁndzɯm}{\papi{ aɲaʁndzɯm}}}\markboth{aɲaʁndzɯm}{}\classe{vi}
\paradigme{\textit{dir :} \jya tɤ-}
\begin{définition}\fra marron foncé\end{définition}
\begin{définition}\cmn 黑褐色\end{définition}
\begin{exemple}\jya tʂha kɤ-ta-t-a ri, nɯ-ala, ɲɤ-ɬoʁ tɕe ɲɤ-k-ɤɲaʁdzɯm-ci\cmn 我把茶烧开了,把茶叶熬出来了就变成黑褐色\end{exemple}\end{entrée}

\begin{entrée}
\vedette{\hypertarget{Ⓔaɲɟoʁ}{\papi{ aɲɟoʁ}}}\markboth{aɲɟoʁ}{}
\begin{relation-sémantique}\confer{
\hyperlink{Ⓔɲɟoʁ}{\textit{ \papi{ɲɟoʁ}}}
}\end{relation-sémantique}\end{entrée}

\begin{entrée}
\vedette{\hypertarget{Ⓔaŋgɤjom}{\papi{ aŋgɤjom}}}\markboth{aŋgɤjom}{}\classe{vs}
\paradigme{\textit{dir :} \jya nɯ-}
\begin{définition}\fra large, ouvert\end{définition}
\begin{définition}\cmn 宽敞\end{définition}
\begin{exemple}\jya tɯ-ŋga ɲɯ-ɤŋgɤjom\cmn 衣服很宽\end{exemple}
\begin{exemple}\jya kha ɲɯ-ɤŋgɤjom\cmn 房子很宽\end{exemple}
\begin{exemple}\jya ɲɤ-k-ɤŋgɤjom-ci\cmn 变宽了\end{exemple}
\begin{relation-sémantique}\antonyme{
\hyperlink{Ⓔaŋgɤrŋgɤr}{\textit{ \papi{aŋgɤrŋgɤr}}}
}\end{relation-sémantique}
\begin{relation-sémantique}\confer{
\hyperlink{Ⓔjom}{\textit{ \papi{jom}}}
}\end{relation-sémantique}\end{entrée}

\begin{entrée}
\vedette{\hypertarget{Ⓔaŋgɤrŋgɤr}{\papi{ aŋgɤrŋgɤr}}}\markboth{aŋgɤrŋgɤr}{}
\classe{vs}
\paradigme{\textit{dir :} \jya kɤ-}
\begin{définition}\fra à l'étroit\end{définition}
\begin{définition}\cmn 拥挤(房子)\end{définition}
\begin{exemple}\jya ko-k-ɤŋgɤrŋgɤr-ci\cmn 变得很拥挤\end{exemple}
\begin{exemple}\jya tɕi-sta ɲɯ-ɤŋgɤrŋgɤr\cmn 我们俩的房间很窄\end{exemple}
\begin{exemple}\jya ki tɕi-sta ko-k-ɤŋgɤrŋgɤr-ci\cmn 我们俩的房间比以前拥挤了\end{exemple}
\begin{relation-sémantique}\confer{
\hyperlink{Ⓔŋgɤr}{\textit{ \papi{ŋgɤr}}}
}\end{relation-sémantique}
\begin{relation-sémantique}\antonyme{
\hyperlink{Ⓔaŋgɤjom}{\textit{ \papi{aŋgɤjom}}}
}\end{relation-sémantique}\end{entrée}

\begin{entrée}
\vedette{\hypertarget{Ⓔaŋgorji}{\papi{ aŋgorji}}}\markboth{aŋgorji}{}
\classe{vs}
\begin{définition}\fra calme, silencieux\end{définition}
\begin{définition}\cmn 安静\end{définition}
\begin{exemple}\jya ɲɯ-ɤŋgorji\cmn 很安静\end{exemple}
\begin{exemple}\jya kɯ-rɤma ra tɤ-nɯna-nɯ tɕe ɲɯ-ɤŋgorji\cmn 打工的人休息了,现在很安静\end{exemple}\end{entrée}

\begin{entrée}
\vedette{\hypertarget{Ⓔapa}{\papi{ apa}}}\markboth{apa}{}
\classe{vi}
\paradigme{\textit{dir :} \jya nɯ-}\acception{1}
\begin{définition}\fra devenir\end{définition}
\begin{définition}\cmn 变成(自然形成的)\end{définition}\acception{2}
\begin{définition}\fra être fermé\end{définition}
\begin{définition}\cmn 关着\end{définition}
\begin{exemple}\jya kɯm ɲɯ-ɤpa\cmn 门是关着的\end{exemple}\acception{3}
\begin{définition}\fra félicitation pour...\end{définition}
\begin{définition}\cmn 恭喜……\end{définition}
\begin{exemple}\jya ɲɯ-tɯ-mna tɕe pɯ-apa\cmn 你痊愈了,恭喜你了\end{exemple}\acception{4}
\begin{définition}\fra être en tort (au négatif)\end{définition}
\begin{définition}\cmn 有错(否定式)\end{définition}
\begin{exemple}\jya tɯʑo mɯ-pɯ-kɯ-ɤpa tɕe tɯ-zda ɯ-ɕki kɤ-ti mɤ-mbat\cmn 自己有错的时候不好说别人\end{exemple}
\begin{exemple}\jya nɯ mɯ-pɯ-apa-a ko!\cmn 很对不起!(我没有对)\end{exemple}
\begin{relation-sémantique}\confer{
\hyperlink{ⒺpaⒽ1}{\textit{ \papi{pa}}}
}\end{relation-sémantique}
\begin{relation-sémantique}\confer{
\hyperlink{Ⓔsɤpa}{\textit{ \papi{sɤpa}}}
}\end{relation-sémantique}\end{entrée}

\begin{entrée}
\vedette{\hypertarget{Ⓔapɤmbat}{\papi{ apɤmbat}}}\markboth{apɤmbat}{}
\classe{vi}
\paradigme{\textit{dir :} \jya tɤ-}
\begin{définition}\ 
\begin{déclaration}\grammar{comp}\end{déclaration}\end{définition}
\begin{définition}\fra facile à faire\end{définition}
\begin{définition}\cmn 容易做\end{définition}
\begin{exemple}\jya jiɕqha nɯ ʁo kɯ-ɤpɤmbat ci ɲɯ-ŋu\cmn 这倒是一件容易的事情\end{exemple}
\begin{exemple}\jya jiɕqha kɤ-nɤma ʁo ɲɯ-ɤpɤmbat\cmn 这个工作倒很容易做\end{exemple}
\begin{exemple}\jya ɯ-ɲɯ-ɤpɤmbat\cmn 容不容易?\end{exemple}
\begin{exemple}\jya mɯ-ɲɯ-ɤpɤmbat\cmn 不容易\end{exemple}\begin{sous-entrée}
\vedette{\hypertarget{}{\papi{ sɤpɤmbat}}}\markboth{sɤpɤmbat}{}\classe{vt}
\paradigme{\textit{dir :} \jya tɤ-}
\begin{définition}\fra simplifier\end{définition}
\begin{définition}\cmn 简化\end{définition}
\begin{relation-sémantique}\confer{
\hyperlink{Ⓔmbat}{\textit{ \papi{mbat}}}
}\end{relation-sémantique}
\begin{relation-sémantique}\confer{
\hyperlink{ⒺpaⒽ1}{\textit{ \papi{pa}}}
}\end{relation-sémantique}
\end{sous-entrée}\end{entrée}

\begin{entrée}
\vedette{\hypertarget{Ⓔapɤɴqa}{\papi{ apɤɴqa}}}\markboth{apɤɴqa}{}
\classe{vi}
\paradigme{\textit{dir :} \jya tɤ-}
\begin{définition}\fra dur à faire\end{définition}
\begin{définition}\cmn 难做\end{définition}
\begin{exemple}\jya jiɕqha kɤ-nɤma ɲɯ-ɤpɤɴqa\cmn 这个工作很难做\end{exemple}
\begin{relation-sémantique}\antonyme{
\hyperlink{Ⓔapɤmbat}{\textit{ \papi{apɤmbat}}}
}\end{relation-sémantique}
\begin{relation-sémantique}\confer{
\hyperlink{Ⓔɴqa}{\textit{ \papi{ɴqa}}}
}\end{relation-sémantique}
\begin{relation-sémantique}\confer{
\hyperlink{ⒺpaⒽ1}{\textit{ \papi{pa}}}
}\end{relation-sémantique}\end{entrée}

\begin{entrée}
\vedette{\hypertarget{Ⓔapɤpɣi}{\papi{ apɤpɣi}}}\markboth{apɤpɣi}{}\classe{vs}
\begin{définition}\fra gris\end{définition}
\begin{définition}\cmn 带有灰色\end{définition}
\begin{relation-sémantique}\confer{
\hyperlink{Ⓔpɣi}{\textit{ \papi{pɣi}}}
}\end{relation-sémantique}\end{entrée}

\begin{entrée}
\vedette{\hypertarget{Ⓔapɕɯβjɤl}{\papi{ apɕɯβjɤl}}}\markboth{apɕɯβjɤl}{}\classe{vi}
\paradigme{\textit{dir :} \jya tɤ-}
\begin{définition}\fra en biais, penché\end{définition}
\begin{définition}\cmn 斜着\end{définition}
\begin{exemple}\jya ndzom ɲɯ-ɤpɕɯβjɤl tɕe mɯ́j-nɯɣɯŋke\cmn 桥是斜着的,不好走\end{exemple}\begin{sous-entrée}
\vedette{\hypertarget{}{\papi{ sɤpɕɯβjɤl}}}\markboth{sɤpɕɯβjɤl}{}\classe{vt}
\paradigme{\textit{dir :} \jya tɤ-}
\begin{définition}\fra placer en biais, penché\end{définition}
\begin{définition}\cmn 斜着放\end{définition}
\begin{exemple}\jya tɕhi kɤ-thɯ tɕe tú-wɣ-sɤpɕɯβjɤl ra\cmn 搭梯子的时候,要斜着搭\end{exemple}
\end{sous-entrée}\end{entrée}

\begin{entrée}
\vedette{\hypertarget{Ⓔapɣaʁsci}{\papi{ apɣaʁsci}}}\markboth{apɣaʁsci}{}
\classe{vi}
\paradigme{\textit{dir :} \jya tɤ-}
\begin{définition}\ 
\begin{déclaration}\grammar{caus}\end{déclaration}\end{définition}
\begin{définition}\fra se retourner\end{définition}
\begin{définition}\cmn 翻过去\end{définition}\begin{sous-entrée}
\vedette{\hypertarget{}{\papi{ sɤpɣaʁsci}}}\markboth{sɤpɣaʁsci}{}\classe{vt}
\begin{définition}\fra retourner\end{définition}
\begin{définition}\cmn 翻过去\end{définition}
\begin{exemple}\jya qajɣi ta-sɤpɣaʁsci\cmn 他翻了馍馍\end{exemple}
\end{sous-entrée}\begin{sous-entrée}
\vedette{\hypertarget{}{\papi{ ʑɣɤsɤpɣaʁsci}}}\markboth{ʑɣɤsɤpɣaʁsci}{}\classe{vi}
\begin{définition}\ 
\begin{déclaration}\grammar{refl}\end{déclaration}
\begin{déclaration}\grammar{caus}\end{déclaration}\end{définition}
\begin{définition}\fra se retourner\end{définition}
\begin{définition}\cmn 翻身\end{définition}
\begin{exemple}\jya tɤ-ʑɣɤsɤpɣaʁsci-a\cmn 我翻了身(例如,睡觉的时候)\end{exemple}
\begin{exemple}\jya tɤ-ʑɣɤsɤpɣaʁsci\cmn 他翻了身\end{exemple}
\begin{exemple}\jya tu-tɯ-ʑɣɤsɤpɣaʁsci ntsɯ ɲɯ-ŋu, ɲɯ-sɤɣdɯɣ\cmn 你不停地翻身,很烦\end{exemple}
\end{sous-entrée}\end{entrée}

\begin{entrée}
\vedette{\hypertarget{Ⓔapɣɤpɣi}{\papi{ apɣɤpɣi}}}\markboth{apɣɤpɣi}{}
\classe{vi}
\paradigme{\textit{dir :} \jya thɯ-}
\begin{définition}\fra grisâtre\end{définition}
\begin{définition}\cmn 淡灰色\end{définition}
\begin{exemple}\jya tɯ-ŋga ɯ-mdoʁ kɯ-ɤpɣɤpɣi ci ɲɯ-ŋu\cmn 衣服的颜色是淡灰色\end{exemple}
\begin{exemple}\jya rɯdaʁ kɯ-ɤpɣɤpɣi ci ɲɯ-ŋu\cmn 是一头淡灰色的野兽\end{exemple}
\begin{relation-sémantique}\synonyme{
\hyperlink{Ⓔapɣɯlu}{\textit{ \papi{apɣɯlu}}}
}\end{relation-sémantique}
\begin{relation-sémantique}\confer{
\hyperlink{Ⓔpɣi}{\textit{ \papi{pɣi}}}
}\end{relation-sémantique}\end{entrée}

\begin{entrée}
\vedette{\hypertarget{Ⓔapɣɯlu}{\papi{ apɣɯlu}}}\markboth{apɣɯlu}{}
\classe{vi}
\paradigme{\textit{dir :} \jya thɯ-}
\begin{définition}\fra grisâtre\end{définition}
\begin{définition}\cmn 灰扑扑\end{définition}
\begin{exemple}\jya jiɕqha nɯ kɯ-ɤpɣɯlu ci ɲɯ-ŋu\cmn 刚才那个是灰扑扑的\end{exemple}
\begin{relation-sémantique}\synonyme{
\hyperlink{Ⓔapɤpɣi}{\textit{ \papi{apɤpɣi}}}
}\end{relation-sémantique}
\begin{relation-sémantique}\confer{
\hyperlink{Ⓔpɣi}{\textit{ \papi{pɣi}}}
}\end{relation-sémantique}\end{entrée}

\begin{entrée}
\vedette{\hypertarget{Ⓔaphala}{\papi{ aphala}}}\markboth{aphala}{}\classe{vi}
\begin{définition}\fra ayant des mouvement alertes\end{définition}
\begin{définition}\cmn 动作灵活(小伙子)\end{définition}
\begin{exemple}\jya kɯ-ɤphala ci ɲɯ-ŋu\cmn 他动作很灵活\end{exemple}
\begin{relation-sémantique}\synonyme{
\hyperlink{Ⓔaɕpala}{\textit{ \papi{aɕpala}}}
}\end{relation-sémantique}\end{entrée}

\begin{entrée}
\vedette{\hypertarget{Ⓔaphɤlɤjɤt}{\papi{ aphɤlɤjɤt}}}\markboth{aphɤlɤjɤt}{}\classe{vs}
\paradigme{\textit{dir :} \jya pɯ-}
\begin{définition}\fra en désordre\end{définition}
\begin{définition}\cmn 凌乱\end{définition}
\begin{exemple}\jya a-mɤ-pɯ-ɤphɤlɤjɤt kɯ tɤ-rɤwum\cmn (东西)不要这么乱,收拾一下\end{exemple}
\begin{relation-sémantique}\synonyme{
\hyperlink{Ⓔadrɤt}{\textit{ \papi{adrɤt}}}
}\end{relation-sémantique}\begin{sous-entrée}
\vedette{\hypertarget{}{\papi{ sɤphɤlɤjɤt}}}\markboth{sɤphɤlɤjɤt}{}\classe{vt}
\begin{définition}\fra mettre en désordre\end{définition}
\begin{définition}\cmn 乱放\end{définition}
\end{sous-entrée}\end{entrée}

\begin{entrée}
\vedette{\hypertarget{Ⓔapjɤntɤm}{\papi{ apjɤntɤm}}}\markboth{apjɤntɤm}{}\classe{vs}
\paradigme{\textit{dir :} \jya nɯ-}
\begin{définition}\fra plat\end{définition}
\begin{définition}\cmn 平整\end{définition}
\begin{exemple}\jya khɤxtu ɲɯ-ɤpjɤntɤm\cmn 房背是平的\end{exemple}\begin{sous-entrée}
\vedette{\hypertarget{}{\papi{ sɤpjɤntɤm}}}\markboth{sɤpjɤntɤm}{}\classe{vt}
\paradigme{\textit{dir :} \jya nɯ-}
\paradigme{\textit{dir :} \jya thɯ-}
\begin{définition}\ 
\begin{déclaration}\grammar{caus}\end{déclaration}\end{définition}
\begin{définition}\fra aplanir\end{définition}
\begin{définition}\cmn 弄平\end{définition}
\begin{exemple}\jya sɤtɕha nɯ-sɤpjɤntam-a\cmn 我把地弄平了\end{exemple}
\begin{exemple}\jya kha ɯ-sta ɯ-spa nɯ kɤ-sɤpjɤntɤm ra\cmn 要把房基(修房子的地方)弄平\end{exemple}
\begin{relation-sémantique}\confer{
\hyperlink{Ⓔantɤm}{\textit{ \papi{antɤm}}}
}\end{relation-sémantique}
\end{sous-entrée}\end{entrée}

\begin{entrée}
\vedette{\hypertarget{Ⓔapupu}{\papi{ apupu}}}\markboth{apupu}{}\classe{vi}
\paradigme{\textit{dir :} \jya thɯ-}
\paradigme{\textit{dir :} \jya tɤ-}
\begin{définition}\fra prospère\end{définition}
\begin{définition}\cmn 美满\end{définition}
\begin{exemple}\jya nɯnɯ tɯrme nɯ nɯ ɕɯŋgɯ tɕe wuma ʑo pɯ-ŋgɯ ri, tham tɕe chɤ-mɤɕi tɕe cho-k-ɤpupu-ci\cmn 那个人以前很穷,现在变得很富有了\end{exemple}\end{entrée}

\begin{entrée}
\vedette{\hypertarget{Ⓔapɯpa}{\papi{ apɯpa}}}\markboth{apɯpa}{}\classe{vi}
\paradigme{\textit{dir :} \jya tɤ-}\acception{1}
\begin{définition}\fra s'accumuler\end{définition}
\begin{définition}\cmn 积累\end{définition}
\begin{exemple}\jya nɤ-smɤn khro to-k-ɤpɯpa-ci\cmn 你的药积累了很多\end{exemple}
\begin{relation-sémantique}\synonyme{
\hyperlink{Ⓔajtɯ}{\textit{ \papi{ajtɯ}}}
}\end{relation-sémantique}\acception{2}
\begin{définition}\fra être riche\end{définition}
\begin{définition}\cmn 富裕\end{définition}
\begin{relation-sémantique}\synonyme{
\hyperlink{Ⓔɯ-ŋgu,thon}{\textit{ \papi{ɯ-ŋgu,thon}}}
}\end{relation-sémantique}\end{entrée}

\begin{entrée}
\vedette{\hypertarget{Ⓔapɯpri}{\papi{ apɯpri}}}\markboth{apɯpri}{}\classe{vs}
\begin{définition}\fra en continu\end{définition}
\begin{définition}\cmn 连续不断\end{définition}
\begin{exemple}\jya kɯ-ɤpɯpri laʁnɯ-sŋi ʑo pjɤ-rɤʑi\cmn 他连续待了几天\end{exemple}
\begin{exemple}\jya tɯ-xpa ɯ-ŋgɯ kɯ-ɤpɯpri ʑo tɤ-ngo-a\cmn 我在一年里生病了很多次\end{exemple}\begin{sous-entrée}
\vedette{\hypertarget{}{\papi{ sɤpɯpri}}}\markboth{sɤpɯpri}{}\classe{vt}
\begin{définition}\fra faire en continu\end{définition}
\begin{définition}\cmn 连续做……\end{définition}
\end{sous-entrée}\end{entrée}

\begin{entrée}
\vedette{\hypertarget{Ⓔaqandʐɯlu}{\papi{ aqandʐɯlu}}}\markboth{aqandʐɯlu}{}
\classe{vi}
\paradigme{\textit{dir :} \jya nɯ-}
\begin{définition}\fra noirâtre, sombre, violacé\end{définition}
\begin{définition}\cmn 乌黑;紫\end{définition}
\begin{exemple}\jya tɯ-mɯ nɤmkha zdɯm kɯ-ɤqandʐɯlu ʑo ko-ɣi\cmn 天上来了乌云\end{exemple}
\begin{relation-sémantique}\synonyme{
\hyperlink{Ⓔapɣɯlu}{\textit{ \papi{apɣɯlu}}}
}\end{relation-sémantique}\end{entrée}

\begin{entrée}
\vedette{\hypertarget{Ⓔaqarŋɤmbru}{\papi{ aqarŋɤmbru}}}\markboth{aqarŋɤmbru}{}
\classe{vi}
\paradigme{\textit{dir :} \jya thɯ-}
\begin{définition}\fra jaune pâle\end{définition}
\begin{définition}\cmn 淡黄色\end{définition}
\begin{exemple}\jya nɤ-ŋga kɯ-ɤqarŋɤmbru ɲɯ-ŋu\cmn 你的衣服是淡黄色的\end{exemple}
\begin{exemple}\jya ɲɯ-ɤqarŋɤmbru\cmn 是淡黄色的\end{exemple}\end{entrée}

\begin{entrée}
\vedette{\hypertarget{Ⓔaqarŋɯrŋe}{\papi{ aqarŋɯrŋe}}}\markboth{aqarŋɯrŋe}{}
\classe{vs}
\begin{définition}\fra jaune clair\end{définition}
\begin{définition}\cmn 淡黄\end{définition}
\begin{exemple}\jya ɲɯ-ɤqarŋɯrŋe\cmn 是淡黄色的\end{exemple}
\begin{relation-sémantique}\confer{
\hyperlink{Ⓔqarŋe}{\textit{ \papi{qarŋe}}}
}\end{relation-sémantique}
\begin{relation-sémantique}\antonyme{
\hyperlink{Ⓔqarŋɯrŋe}{\textit{ \papi{qarŋɯrŋe}}}
}\end{relation-sémantique}\end{entrée}

\begin{entrée}
\vedette{\hypertarget{Ⓔaqɤrle}{\papi{ aqɤrle}}}\markboth{aqɤrle}{}
\classe{vi}
\paradigme{\textit{dir :} \jya nɯ-}
\begin{définition}\fra être séparé\end{définition}
\begin{définition}\cmn 分开的\end{définition}
\begin{exemple}\jya stoʁ cho staχpɯ aqɤrle\cmn 胡豆和豌豆是分开的\end{exemple}
\begin{relation-sémantique}\confer{
\hyperlink{Ⓔsɤqɤrle}{\textit{ \papi{sɤqɤrle}}}
}\end{relation-sémantique}\end{entrée}

\begin{entrée}
\vedette{\hypertarget{Ⓔaqɤtsa}{\papi{ aqɤtsa}}}\markboth{aqɤtsa}{}\classe{vi}
\paradigme{\textit{dir :} \jya tɤ-}
\begin{définition}\fra préparé\end{définition}
\begin{définition}\cmn 组装起来(机器、器具)\end{définition}\begin{sous-entrée}
\vedette{\hypertarget{}{\papi{ sɤqɤtsa}}}\markboth{sɤqɤtsa}{}\classe{vt}
\begin{définition}\fra préparer\end{définition}
\begin{définition}\cmn 装备好\end{définition}
\begin{exemple}\jya @luyinji tɤ-sɤqɤtse\cmn 你把录音机准备好\end{exemple}
\begin{exemple}\jya mbɣo tɤ-sɤqɤtse\cmn 你把犁组装起来吧\end{exemple}
\begin{exemple}\jya βɣa to-sɤqɤtsa\cmn 他把磨坊准备好了\end{exemple}
\begin{relation-sémantique}\confer{
\hyperlink{Ⓔnɤqɤtsa}{\textit{ \papi{nɤqɤtsa}}}
}\end{relation-sémantique}
\end{sous-entrée}\end{entrée}

\begin{entrée}
\vedette{\hypertarget{Ⓔaqɤtʂha}{\papi{ aqɤtʂha}}}\markboth{aqɤtʂha}{}\classe{vi}
\paradigme{\textit{dir :} \jya nɯ-}
\begin{définition}\fra être croisé\end{définition}
\begin{définition}\cmn 交叉\end{définition}
\begin{exemple}\jya tɤ-ri ɲɯ-ɤqɤtʂha\cmn 线互相交叉\end{exemple}
\begin{exemple}\jya si ra pjɤ-k-ɤqɤtʂha-ci\cmn 树交叉在一起\end{exemple}\begin{sous-entrée}
\vedette{\hypertarget{}{\papi{ sɤqɤtʂha}}}\markboth{sɤqɤtʂha}{}\classe{vt}
\paradigme{\textit{dir :} \jya pɯ-}
\begin{définition}\ 
\begin{déclaration}\grammar{caus}\end{déclaration}\end{définition}
\begin{définition}\fra croiser\end{définition}
\begin{définition}\cmn 使交叉\end{définition}
\begin{exemple}\jya ɯ-mɤlɤjaʁ pjɤ-sɤqɤtʂha\cmn 它把四肢交叉在一起\end{exemple}
\begin{exemple}\jya tɯmbri ra ɲɤ-sɤqɤtʂha-nɯ\cmn 他们把绳子交叉了起来\end{exemple}
\begin{exemple}\jya laʁjɯɣ ɲɤ-sɤqɤtʂha\cmn 他把棍子交叉了起来\end{exemple}
\begin{exemple}\jya ɯ-mi pjɤ-sɤqɤtʂha\cmn 他把脚交叉了起来\end{exemple}
\begin{exemple}\jya aʑo tɤ-ri pɯ-sɤqɤtʂha-t-a\cmn 我把线交叉在一起\end{exemple}
\end{sous-entrée}\end{entrée}

\begin{entrée}
\vedette{\hypertarget{Ⓔaqhe}{\papi{ aqhe}}}\markboth{aqhe}{}\classe{vs}
\paradigme{\textit{dir :} \jya nɯ-}
\begin{définition}\fra guérir\end{définition}
\begin{définition}\cmn 痊愈\end{définition}
\begin{exemple}\jya ɯ-ku ɯ-kɯ-mŋɤm ɲɤ-k-ɤqhe-ci\cmn 他头痛痊愈了\end{exemple}\begin{sous-entrée}
\vedette{\hypertarget{}{\papi{ sɤqhe}}}\markboth{sɤqhe}{}\classe{vt}
\paradigme{\textit{dir :} \jya nɯ-}
\begin{définition}\fra guérir\end{définition}
\begin{définition}\cmn 治疗\end{définition}
\begin{exemple}\jya a-kɯ-mŋɤm na-sɤqhe\cmn 他治好了我的病\end{exemple}
\begin{relation-sémantique}\synonyme{
\hyperlink{Ⓔɣɤmna}{\textit{ \papi{ɣɤmna}}}
}\end{relation-sémantique}
\end{sous-entrée}\end{entrée}

\begin{entrée}
\vedette{\hypertarget{Ⓔaqhoβlu}{\papi{ aqhoβlu}}}\markboth{aqhoβlu}{}
\classe{vs}
\begin{définition}\fra concave\end{définition}
\begin{définition}\cmn 凹进去\end{définition}
\begin{relation-sémantique}\confer{
 \papi{aχchowɤlu}
}\end{relation-sémantique}
\begin{relation-sémantique}\confer{
\hyperlink{Ⓔʁlɯβʁlɯβ}{\textit{ \papi{ʁlɯβʁlɯβ}}}
}\end{relation-sémantique}\end{entrée}

\begin{entrée}
\vedette{\hypertarget{Ⓔaqhowolu}{\papi{ aqhowolu}}}\markboth{aqhowolu}{}
\classe{vs}
\begin{définition}\fra concave\end{définition}
\begin{définition}\cmn 凹\end{définition}
\begin{relation-sémantique}\synonyme{
\hyperlink{Ⓔaχchowolu}{\textit{ \papi{aχchowolu}}}
}\end{relation-sémantique}
\begin{relation-sémantique}\synonyme{
\hyperlink{Ⓔaʁloʁlu}{\textit{ \papi{aʁloʁlu}}}
}\end{relation-sémantique}
\begin{relation-sémantique}\synonyme{
\hyperlink{Ⓔaɕqhlu}{\textit{ \papi{aɕqhlu}}}
}\end{relation-sémantique}
\begin{relation-sémantique}\synonyme{
\hyperlink{Ⓔasqhlu}{\textit{ \papi{asqhlu}}}
}\end{relation-sémantique}
\begin{relation-sémantique}\synonyme{
\hyperlink{Ⓔarɴɢlɯm}{\textit{ \papi{arɴɢlɯm}}}
}\end{relation-sémantique}\end{entrée}

\begin{entrée}
\vedette{\hypertarget{Ⓔaqurle}{\papi{ aqurle}}}\markboth{aqurle}{}
\classe{vi}
\paradigme{\textit{dir :} \jya tɤ-}
\paradigme{\textit{construction :} \jya infinitive}
\begin{définition}\fra collaborer\end{définition}
\begin{définition}\cmn 协作共事;互相帮忙\end{définition}
\begin{exemple}\jya tɤ-scoz kɤ-rɤt pɯ-aqurle-tɕi\cmn 我们一起写了信\end{exemple}
\begin{relation-sémantique}\confer{
\hyperlink{Ⓔqur}{\textit{ \papi{qur}}}
}\end{relation-sémantique}\end{entrée}

\begin{entrée}
\vedette{\hypertarget{Ⓔaraʁ}{\papi{ araʁ}}}\markboth{araʁ}{}\classe{n}
\begin{définition}\fra alcool distillé\end{définition}
\begin{définition}\cmn 白酒
\begin{déclaration} \étymologie{\papi{a.rag}}\end{déclaration}\end{définition}\end{entrée}

\begin{entrée}
\vedette{\hypertarget{Ⓔaraχtɯ}{\papi{ araχtɯ}}}\markboth{araχtɯ}{}\classe{vs}
\begin{définition}\fra endroit bien caché\end{définition}
\begin{définition}\cmn 僻静,不容易被别人发现的地方\end{définition}
\begin{exemple}\jya ki sɤtɕha ɲɯ-ɤraχtɯ tɕe tɯrme kɯ-ɕe rkɯn\cmn 这个地方很僻静,去的人少\end{exemple}\end{entrée}

\begin{entrée}
\vedette{\hypertarget{Ⓔarɤɕɯɕrɤz}{\papi{ arɤɕɯɕrɤz}}}\markboth{arɤɕɯɕrɤz}{}
\classe{vi}
\paradigme{\textit{dir :} \jya nɯ-}
\paradigme{\textit{dir :} \jya thɯ-}
\begin{définition}\fra bariolé en bandes\end{définition}
\begin{définition}\cmn 几种不同颜色的成条形的纹路(一般指某个物体的切面,例如:猪肉的切面一层肥肉、一层瘦肉)\end{définition}
\begin{exemple}\jya kɯki ɯ-mdoʁ kɯ-ɤrɤɕɯɕrɤz ɲɯ-ŋu\cmn 这个东西有不同颜色的纹路\end{exemple}\end{entrée}

\begin{entrée}
\vedette{\hypertarget{Ⓔarɤkhɯmkhɤl}{\papi{ arɤkhɯmkhɤl}}}\markboth{arɤkhɯmkhɤl}{} (\variante{arɤkhɯkhɤl}) \classe{vi}
\paradigme{\textit{dir :} \jya tɤ-}
\begin{définition}\fra hétérogène\end{définition}
\begin{définition}\cmn 不均匀;有些地方有,有些地方没有\end{définition}
\begin{exemple}\jya a-βri kɯ-mŋɤm ɲɯ-ɤrɤkhɯmkhɤl\cmn 身体有些地方疼,一些地方不疼\end{exemple}
\begin{exemple}\jya tɤ-rɤku ɲɯ-ɤrɤkhɯmkhɤl\cmn 有些地方有庄稼,有些地方没有\end{exemple}
\begin{exemple}\jya tɯji ɯ-ŋgɯ tɤ-rɤku wuma ʑo ɲɯ-pe ri, ɲɯ-ɤrɤkhɯmkhɤl\cmn 田里的庄稼长得很好,但是很不均匀\end{exemple}
\begin{relation-sémantique}\confer{
\hyperlink{Ⓔtɯ-khɤl}{\textit{ \papi{tɯ-khɤl}}}
}\end{relation-sémantique}\end{entrée}

\begin{entrée}
\vedette{\hypertarget{Ⓔarɤmboʁɲɟi}{\papi{ arɤmboʁɲɟi}}}\markboth{arɤmboʁɲɟi}{}\classe{vi}
\paradigme{\textit{dir :} \jya nɯ-}
\begin{définition}\ 
\begin{déclaration}\grammar{denom}\end{déclaration}\end{définition}
\begin{définition}\fra tomber en milles morceaux\end{définition}
\begin{définition}\cmn 粉粹\end{définition}
\begin{relation-sémantique}\confer{
\hyperlink{Ⓔmboʁɲɟi}{\textit{ \papi{mboʁɲɟi}}}
}\end{relation-sémantique}\end{entrée}

\begin{entrée}
\vedette{\hypertarget{Ⓔarɤmbɯmbri}{\papi{ arɤmbɯmbri}}}\markboth{arɤmbɯmbri}{}\classe{vi}
\begin{définition}\fra pas rassemblé en un endroit, dispersé sur un chemin\end{définition}
\begin{définition}\cmn 不密集(一些地方多,一些地方少),一路上撒下的\end{définition}
\begin{relation-sémantique}\confer{
\hyperlink{Ⓔrɤmbɯmbri}{\textit{ \papi{rɤmbɯmbri}}}
}\end{relation-sémantique}\end{entrée}

\begin{entrée}
\vedette{\hypertarget{Ⓔarɤmgrɯndɯr}{\papi{ arɤmgrɯndɯr}}}\markboth{arɤmgrɯndɯr}{}
\classe{vi.s}
\paradigme{\textit{dir :} \jya nɯ-}
\begin{définition}\fra dont la lie et la partie liquide ne se mélangent pas\end{définition}
\begin{définition}\cmn 渣滓和液体部分间隔分明
\end{définition}
\begin{relation-sémantique}\confer{
\hyperlink{Ⓔamgri}{\textit{ \papi{amgri}}}
}\end{relation-sémantique}\end{entrée}

\begin{entrée}
\vedette{\hypertarget{Ⓔarɤmgɯmgo}{\papi{ arɤmgɯmgo}}}\markboth{arɤmgɯmgo}{}\classe{vs}
\begin{définition}\fra ayant des grumeaux\end{définition}
\begin{définition}\cmn 不均匀,一坨一坨的(粥)\end{définition}
\begin{exemple}\jya tɯtshi lɤ́-wɣ-sɤla tɕe khro ɲɯ́-wɣ-ɕmi tɕe mɤ-arɤmgɯmgo\cmn 煲粥的时候要搅拌很多次才会均匀\end{exemple}
\begin{relation-sémantique}\synonyme{
\hyperlink{Ⓔarɤtshi}{\textit{ \papi{arɤtshi}}}
}\end{relation-sémantique}\end{entrée}

\begin{entrée}
\vedette{\hypertarget{Ⓔarɤmtɕɯmtɕoʁ}{\papi{ arɤmtɕɯmtɕoʁ}}}\markboth{arɤmtɕɯmtɕoʁ}{}
\classe{vs}
\begin{définition}\fra nombreux et rassemblés\end{définition}
\begin{définition}\cmn 很多;聚集在一起\end{définition}
\begin{exemple}\jya ɯ-zrɯɣ ɯ-tɯ-dɤn kɯ ɲɯ-ɤrɤmtɕɯmtɕoʁ ʑo\cmn 他身上长满了虱子\end{exemple}\end{entrée}

\begin{entrée}
\vedette{\hypertarget{Ⓔarɤmtʂɯmtʂaj}{\papi{ arɤmtʂɯmtʂaj}}}\markboth{arɤmtʂɯmtʂaj}{}
\classe{vi}
\paradigme{\textit{dir :} \jya tɤ-}
\begin{définition}\fra collant\end{définition}
\begin{définition}\cmn 黏糊\end{définition}
\begin{exemple}\jya cha ɲɯ-mɯm ɲɯ-ɤrɤmtʂɯmtʂaj\cmn 酒好喝,很稠厚\end{exemple}
\begin{exemple}\jya mbrɤz ɲɤ-mɲɤt tɕe ɲɯ-ɤrɤmtʂɯmtʂaj\cmn 饭变味了,是黏糊糊的\end{exemple}\end{entrée}

\begin{entrée}
\vedette{\hypertarget{Ⓔarɤmɯzda}{\papi{ arɤmɯzda}}}\markboth{arɤmɯzda}{}\classe{vi}
\paradigme{\textit{dir :} \jya nɯ-}
\begin{définition}\fra se transmettre les informations les uns aux autres\end{définition}
\begin{définition}\cmn 互相传递(信息、消息、话),通知每一个人\end{définition}
\begin{exemple}\jya jiʑora nɯ-arɤmɯzda-j tɕe tɤ-rɤŋgat-i\cmn 我们互相通知了,准备出发\end{exemple}
\begin{relation-sémantique}\synonyme{
\hyperlink{Ⓔarɤzdɯzda}{\textit{ \papi{arɤzdɯzda}}}
}\end{relation-sémantique}
\begin{relation-sémantique}\synonyme{
\hyperlink{Ⓔasɤmɯsɯz}{\textit{ \papi{asɤmɯsɯz}}}
}\end{relation-sémantique}
\begin{relation-sémantique}\synonyme{
\hyperlink{Ⓔasɤmɯmtshɯmtshɤm}{\textit{ \papi{asɤmɯmtshɯmtshɤm}}}
}\end{relation-sémantique}\end{entrée}

\begin{entrée}
\vedette{\hypertarget{Ⓔarɤntɕhɯntɕhɯr}{\papi{ arɤntɕhɯntɕhɯr}}}\markboth{arɤntɕhɯntɕhɯr}{}\classe{vs}
\begin{définition}\fra en mille morceaux\end{définition}
\begin{définition}\cmn 有很多碎片\end{définition}
\begin{relation-sémantique}\confer{
\hyperlink{Ⓔtɯ-ntɕhɯr}{\textit{ \papi{tɯ-ntɕhɯr}}}
}\end{relation-sémantique}\end{entrée}

\begin{entrée}
\vedette{\hypertarget{Ⓔarɤɲɟiɲɟi}{\papi{ arɤɲɟiɲɟi}}}\markboth{arɤɲɟiɲɟi}{}
\classe{vi}
\paradigme{\textit{dir :} \jya nɯ-}
\begin{définition}\fra tomber en lambeaux, tomber en morceau\end{définition}
\begin{définition}\cmn 成碎片\end{définition}
\begin{exemple}\jya @guamian ɲɤ-k-ɤrɤɲɟiɲɟi-ci\cmn 挂面成了碎片\end{exemple}
\begin{exemple}\jya a-@chabei pjɤ-ɴɢrɯ tɕe ɲɤ-k-ɤrɤɲɟiɲɟi-ci\cmn 我的茶杯破了,成了碎片\end{exemple}\begin{sous-entrée}
\vedette{\hypertarget{}{\papi{ zrɤɲɟiɲɟi}}}\markboth{zrɤɲɟiɲɟi}{}\classe{vt}
\paradigme{\textit{dir :} \jya pɯ-}
\paradigme{\textit{dir :} \jya nɯ-}
\begin{définition}\fra briser en mille morceaux\end{définition}
\begin{définition}\cmn 打得粉碎\end{définition}
\begin{exemple}\jya χɕɤl nɯra pɯ-zrɤɲɟiɲɟi-t-a ʑo\cmn 我打碎了玻璃\end{exemple}
\end{sous-entrée}\end{entrée}

\begin{entrée}
\vedette{\hypertarget{Ⓔarɤɲɯɣ}{\papi{ arɤɲɯɣ}}}\markboth{arɤɲɯɣ}{}
\classe{vi}
\paradigme{\textit{dir :} \jya tɤ-}
\begin{définition}\fra en grande quantité\end{définition}
\begin{définition}\cmn 很多\end{définition}
\begin{exemple}\jya ɯʑo ɯ-ŋga arɤɲɯɣ ʑo ɕti\cmn 他有很多衣服\end{exemple}
\begin{exemple}\jya tɤ-scoz a-kɤ-rɤt ɯ-spa arɤɲɯɣ ʑo\cmn 我要写的字很多\end{exemple}\end{entrée}

\begin{entrée}
\vedette{\hypertarget{Ⓔarɤphɤjqa}{\papi{ arɤphɤjqa}}}\markboth{arɤphɤjqa}{}
\classe{vi}
\paradigme{\textit{dir :} \jya nɯ-}
\begin{définition}\fra pousser (plusieurs pousses à partir d'un grain)\end{définition}
\begin{définition}\cmn 一颗种子可以长成很多根苗\end{définition}
\begin{exemple}\jya ki @cai ɲɯ-ɤrɤphɤjqa\cmn 这种菜长很多根苗\end{exemple}
\begin{exemple}\jya tɤɕi ɲɯ-ɤrɤphɤjqa\cmn 青稞长很多根苗\end{exemple}
\begin{exemple}\jya tɤ-rɤku wuma ʑo ɲɯ-ɤrɤphɤjqa\cmn 庄稼长很多根苗\end{exemple}
\begin{exemple}\jya tɤɕi lo-ji tɕe wuma ʑo ɲɤ-k-ɤrɤphɤjqa-ci\cmn 种了青稞以后,长了很多根苗\end{exemple}
\begin{relation-sémantique}\confer{
\hyperlink{Ⓔtɯ-qa}{\textit{ \papi{tɯ-qa}}}
}\end{relation-sémantique}\end{entrée}

\begin{entrée}
\vedette{\hypertarget{Ⓔarɤrɤɣ}{\papi{ arɤrɤɣ}}}\markboth{arɤrɤɣ}{}\classe{vi}
\paradigme{\textit{dir :} \jya nɯ-}
\begin{définition}\fra se produire au moment prévu\end{définition}
\begin{définition}\cmn 在预定的时间发生\end{définition}
\begin{exemple}\jya aʑo a-ʑɯβ ɲɤ-k-ɤrɤɣ-ci\cmn 我在预定的时间就想睡觉了\end{exemple}
\begin{exemple}\jya jisŋi saχsɯ ɯ-qhu tɕe ku-kɯ-rŋgɯ tɕe a-pɯ-kɯ-nɯʑɯβ tɕe ɯ-fso saχsɯ ɯ-mphru, li tɯ-ʑɯβ pjɯ-ɣi ŋu, tɕe nɯ ɯ-sta nɯ-kɤ-βzu nɯ ɲɯ-kɯ-ɤrɤrɤɣ tu-kɯ-ti ŋu\cmn 
今天午饭后睡觉睡着了,第二天午饭后又想瞌睡了,这种习惯性的现象叫做\stylefv{ɲɯ-kɤrɤrɤɣ}
\end{exemple}
\begin{relation-sémantique}\confer{
\hyperlink{Ⓔɯ-rɤɣ}{\textit{ \papi{ɯ-rɤɣ}}}
}\end{relation-sémantique}\end{entrée}

\begin{entrée}
\vedette{\hypertarget{Ⓔarɤrkhɯrkhe}{\papi{ arɤrkhɯrkhe}}}\markboth{arɤrkhɯrkhe}{}
\classe{vs}
\paradigme{\textit{dir :} \jya lɤ-}
\paradigme{\textit{dir :} \jya thɯ-}
\begin{définition}\fra pas cohérent\end{définition}
\begin{définition}\cmn 不平整;不连贯;吞吞吐吐\end{définition}
\begin{exemple}\jya tʂu ɲɯ-ɤrɤrkhɯrkhe\cmn 路不平整\end{exemple}
\begin{exemple}\jya si ɲɯ-ɤrɤrkhɯrkhe\cmn 木料上刻得不均匀\end{exemple}
\begin{exemple}\jya jiɕqha ɯ-rju ɲɯ-ɤrɤrkhɯrkhe\cmn 他说话吞吞吐吐\end{exemple}
\begin{exemple}\jya tɤrɤm ɲɯ-ɤrɤrkhɯrkhe\cmn 木板不平整\end{exemple}
\begin{relation-sémantique}\confer{
\hyperlink{Ⓔrkhe}{\textit{ \papi{rkhe}}}
}\end{relation-sémantique}\end{entrée}

\begin{entrée}
\vedette{\hypertarget{Ⓔarɤrɴɢioʁ}{\papi{ arɤrɴɢioʁ}}}\markboth{arɤrɴɢioʁ}{}\classe{vi}
\begin{définition}\fra avoir une encoche\end{définition}
\begin{définition}\cmn 有一条槽\end{définition}
\begin{relation-sémantique}\confer{
\hyperlink{Ⓔtɤ-rɴɢioʁ}{\textit{ \papi{tɤ-rɴɢioʁ}}}
}\end{relation-sémantique}\end{entrée}

\begin{entrée}
\vedette{\hypertarget{Ⓔarɤrqhɯrqhioʁ}{\papi{ arɤrqhɯrqhioʁ}}}\markboth{arɤrqhɯrqhioʁ}{}\classe{vi}
\begin{définition}\fra qui a des entailles\end{définition}
\begin{définition}\cmn 有一条一条的槽口;纹路\end{définition}
\begin{relation-sémantique}\confer{
\hyperlink{Ⓔtɤ-rqhioʁ}{\textit{ \papi{tɤ-rqhioʁ}}}
}\end{relation-sémantique}\end{entrée}

\begin{entrée}
\vedette{\hypertarget{Ⓔarɤrtsɯrtsɤɣ}{\papi{ arɤrtsɯrtsɤɣ}}}\markboth{arɤrtsɯrtsɤɣ}{}\classe{vs}
\begin{définition}\fra composé de sections\end{définition}
\begin{définition}\cmn 一节一节组成的\end{définition}
\begin{relation-sémantique}\confer{
\hyperlink{Ⓔaɣɯrtsɯrtsɤɣ}{\textit{ \papi{aɣɯrtsɯrtsɤɣ}}}
}\end{relation-sémantique}
\begin{relation-sémantique}\confer{
\hyperlink{Ⓔtɯ-rtsɤɣ}{\textit{ \papi{tɯ-rtsɤɣ}}}
}\end{relation-sémantique}\end{entrée}

\begin{entrée}
\vedette{\hypertarget{Ⓔarɤstoʁsta}{\papi{ arɤstoʁsta}}}\markboth{arɤstoʁsta}{}
\classe{vs}\acception{1}
\begin{définition}\fra fiable\end{définition}
\begin{définition}\cmn 可靠;说话算数\end{définition}
\begin{exemple}\jya jiɕqha nɯ ɯ-rju mɤ-kɯ-ɤrɤstoʁsta ci ŋu\cmn 这是说话不算数的一个人\end{exemple}\acception{2}
\begin{définition}\fra stable\end{définition}
\begin{définition}\cmn 稳定\end{définition}
\begin{relation-sémantique}\antonyme{
\hyperlink{Ⓔɲɟɯrmbloʁ}{\textit{ \papi{ɲɟɯrmbloʁ}}}
}\end{relation-sémantique}\end{entrée}

\begin{entrée}
\vedette{\hypertarget{Ⓔarɤt}{\papi{ arɤt}}}\markboth{arɤt}{}
\begin{relation-sémantique}\confer{
\hyperlink{Ⓔrɤt}{\textit{ \papi{rɤt}}}
}\end{relation-sémantique}\end{entrée}

\begin{entrée}
\vedette{\hypertarget{Ⓔarɤtɕha}{\papi{ arɤtɕha}}}\markboth{arɤtɕha}{}
\begin{relation-sémantique}\confer{
\hyperlink{Ⓔzrɤtɕha}{\textit{ \papi{zrɤtɕha}}}
}\end{relation-sémantique}\end{entrée}

\begin{entrée}
\vedette{\hypertarget{Ⓔarɤtshi}{\papi{ arɤtshi}}}\markboth{arɤtshi}{}
\classe{vi}
\paradigme{\textit{dir :} \jya nɯ-}
\begin{définition}\fra trop cuit\end{définition}
\begin{définition}\cmn 煮得很烂,变得像粥一样\end{définition}
\begin{exemple}\jya kɤ́-wɣ-sqa tɕe a-mɤ-nɯ-ɤrɤtshi ra ma mɤ-mɯm\cmn 煮饭的时候,不要煮得太烂,不然不好吃\end{exemple}\begin{sous-entrée}
\vedette{\hypertarget{}{\papi{ zrɤtshi}}}\markboth{zrɤtshi}{}\classe{vt}
\paradigme{\textit{dir :} \jya nɯ-}
\begin{définition}\fra trop cuire\end{définition}
\begin{définition}\cmn 煮得很烂\end{définition}
\begin{exemple}\jya ko-ɣɤsmi-t-a tɕe ɲɤ-zrɤtshi-t-a\cmn 我煮得太烂了(不小心)\end{exemple}
\end{sous-entrée}\end{entrée}

\begin{entrée}
\vedette{\hypertarget{Ⓔarɤzdɯzda}{\papi{ arɤzdɯzda}}}\markboth{arɤzdɯzda}{}\classe{vi}
\paradigme{\textit{dir :} \jya nɯ-}
\begin{définition}\fra se transmettre les informations les uns aux autres\end{définition}
\begin{définition}\cmn 互相传递(信息、消息、话),通知每一个人\end{définition}
\begin{exemple}\jya ɕe-j tɤ-mda tɕe, kɤ-ɤrɤzdɯzda ra\cmn 我们要走的时候,要互相通知\end{exemple}
\begin{relation-sémantique}\synonyme{
\hyperlink{Ⓔarɤmɯzda}{\textit{ \papi{arɤmɯzda}}}
}\end{relation-sémantique}\end{entrée}

\begin{entrée}
\vedette{\hypertarget{Ⓔarɤʑɯʑrɤz}{\papi{ arɤʑɯʑrɤz}}}\markboth{arɤʑɯʑrɤz}{}
\classe{vi}
\paradigme{\textit{dir :} \jya nɯ-}
\begin{définition}\fra ayant des bandes de couleurs différentes\end{définition}
\begin{définition}\cmn 有几种不同颜色的成条形的纹路(物体表面的颜色)\end{définition}
\begin{exemple}\jya kɯki kɯ-ɤrɤʑɯʑrɤz ci ɲɯ-ŋu\cmn 这个东西有几种不同颜色的纹路\end{exemple}
\begin{relation-sémantique}\confer{
\hyperlink{Ⓔarɤɕɯɕrɤz}{\textit{ \papi{arɤɕɯɕrɤz}}}
}\end{relation-sémantique}
\begin{relation-sémantique}\confer{
\hyperlink{Ⓔtɯ-ʑrɤz}{\textit{ \papi{tɯ-ʑrɤz}}}
}\end{relation-sémantique}\end{entrée}

\begin{entrée}
\vedette{\hypertarget{Ⓔarɕɤt}{\papi{ arɕɤt}}}\markboth{arɕɤt}{}\classe{vi}
\begin{définition}\fra avoir une relation de parenté\end{définition}
\begin{définition}\cmn 有血缘关系
\begin{déclaration}\use{一般用于否定式}\end{déclaration}\end{définition}
\begin{exemple}\jya jiʑo kɯmdza mɤ-arɕɤt-i\cmn 我们没有一点亲戚关系\end{exemple}
\begin{exemple}\jya jiɕqha nɯ nɤ-kɯmdza mɤ-arɕɤt\cmn 那个不是你的亲戚\end{exemple}\end{entrée}

\begin{entrée}
\vedette{\hypertarget{Ⓔarɕo}{\papi{ arɕo}}}\markboth{arɕo}{}
\classe{vi}
\paradigme{\textit{dir :} \jya thɯ-}
\begin{définition}\fra finir\end{définition}
\begin{définition}\cmn 完;用光了\end{définition}
\begin{exemple}\jya tʂha thɯ-arɕo\cmn 没有茶了\end{exemple}
\begin{exemple}\jya kɤndza chɯ-ɤrɕo\cmn 没有吃的\end{exemple}
\begin{exemple}\jya kɤ-ndza a-mɤ-thɯ-ɤrɕo, kɤ-ŋga a-mɤ-thɯ-ɤrɕo\cmn 希望不会缺吃的,也不会缺穿的\end{exemple}
\begin{exemple}\jya a-mɤ-thɯ-ɤrɕo ma ɯ-qhu tɕe me\cmn 不要用完,不然以后就没有了\end{exemple}\begin{sous-entrée}
\vedette{\hypertarget{}{\papi{ ɣɤrɕo}}}\markboth{ɣɤrɕo}{}\classe{vs}
\begin{définition}\fra qui se fini rapidement\end{définition}
\begin{définition}\cmn 很快用完\end{définition}
\begin{relation-sémantique}\synonyme{
\hyperlink{Ⓔɣɤsa}{\textit{ \papi{ɣɤsa}}}
}\end{relation-sémantique}
\end{sous-entrée}\begin{sous-entrée}
\vedette{\hypertarget{}{\papi{ sɤrɕo}}}\markboth{sɤrɕo}{}\classe{vt}
\paradigme{\textit{dir :} \jya thɯ-}
\begin{définition}\fra utiliser complètement\end{définition}
\begin{définition}\cmn 用完\end{définition}
\begin{exemple}\jya ɯ-ro ɯ-ɲɤ-ri nɤ a-pɯ-ɤnɯta jɤɣ ma kɤ-sɤrɕo mɤ-ra\cmn 如果有剩的就留在那里,不一定要用完\end{exemple}
\end{sous-entrée}\end{entrée}

\begin{entrée}
\vedette{\hypertarget{Ⓔarɣi}{\papi{ arɣi}}}\markboth{arɣi}{}\classe{vs}
\paradigme{\textit{dir :} \jya tɤ-}
\begin{définition}\fra s'accumuler\end{définition}
\begin{définition}\cmn 积累\end{définition}
\begin{exemple}\jya tɯ-ci to-k-ɤrɣi-ci\cmn 水积起来了\end{exemple}\begin{sous-entrée}
\vedette{\hypertarget{}{\papi{ sɤrɣi}}}\markboth{sɤrɣi}{}\classe{vt}
\paradigme{\textit{dir :} \jya tɤ-}
\begin{définition}\ 
\begin{déclaration}\grammar{caus}\end{déclaration}\end{définition}
\begin{définition}\fra accumuler\end{définition}
\begin{définition}\cmn 积累\end{définition}
\begin{exemple}\jya tɯftsaʁ ɲɯ-ɣi tɕe, zɯm tɤ-ɕthɯz-a tɕe, nɯ ɯ-ŋgɯ tɤ-sɤrɣi-t-a tɕe, ɯ-thoʁ mɤ-kɯ-ɕe tɤ-sɤβzu-t-a\cmn 屋顶在漏水,我用水桶把水积下来了,免得水流到地板上\end{exemple}
\begin{relation-sémantique}\synonyme{
\hyperlink{Ⓔajtɯ}{\textit{ \papi{ajtɯ}}}
}\end{relation-sémantique}
\end{sous-entrée}\end{entrée}

\begin{entrée}
\vedette{\hypertarget{Ⓔari}{\papi{ ari}}}\markboth{ari}{}\classe{vi}
\paradigme{\textit{dir :} \jya nɯ-}
\paradigme{\textit{dir :} \jya pɯ-}\acception{1}
\begin{définition}\fra couler\end{définition}
\begin{définition}\cmn 漏\end{définition}
\begin{exemple}\jya pjɤ-spoʁ tɕe pjɤ-k-ɤri-ci\cmn 有了洞就漏了水\end{exemple}
\begin{exemple}\jya tɯ-ci pjɤ-k-ɤri-ci\cmn 水漏了\end{exemple}
\begin{exemple}\jya khɯtsa ɲɯ-ɤri\cmn 碗在漏水\end{exemple}\acception{2}
\begin{définition}\fra appartenir à (groupe)\end{définition}
\begin{définition}\cmn 属于\end{définition}
\begin{exemple}\jya aʑo tɤrca tɤ-ari-a\cmn 我加入了\end{exemple}
\begin{exemple}\jya ɯʑo ɯ-zdɤrca mɤ-kɯ-ɤri ci ɲɯ-ŋu\cmn 他是个不合群的人\end{exemple}
\begin{relation-sémantique}\synonyme{
\hyperlink{Ⓔʑɣɤsɤri}{\textit{ \papi{ʑɣɤsɤri}}}
}\end{relation-sémantique}
\begin{relation-sémantique}\confer{
\hyperlink{Ⓔsɤri}{\textit{ \papi{sɤri}}}
}\end{relation-sémantique}\end{entrée}

\begin{entrée}
\vedette{\hypertarget{Ⓔarju}{\papi{ arju}}}\markboth{arju}{}
\classe{vi}
\paradigme{\textit{dir :} \jya tɤ-}
\begin{définition}\fra parler\end{définition}
\begin{définition}\cmn 说话\end{définition}
\begin{exemple}\jya ɲɯ-tɯ-ɤrju tɕe pɯ-kɯ-sɯmtsham-a\cmn 你说话,让我听见了\end{exemple}
\begin{exemple}\jya nɯ ɲɯ-ɤrju\cmn 他在说那个\end{exemple}
\begin{exemple}\jya a-mɤ-tɯ-ɤrju\cmn 不要说话\end{exemple}
\begin{exemple}\jya tɤ-arju-a\cmn 我说了\end{exemple}
\begin{exemple}\jya ma-tɯ-ɤrju\cmn 你别说话\end{exemple}\begin{sous-entrée}
\vedette{\hypertarget{}{\papi{ sɤrju}}}\markboth{sɤrju}{}\classe{vt}
\paradigme{\textit{dir :} \jya tɤ-}
\begin{définition}\ 
\begin{déclaration}\grammar{caus}\end{déclaration}\end{définition}
\begin{définition}\fra laisser, faire parler\end{définition}
\begin{définition}\cmn 让人说话\end{définition}
\begin{exemple}\jya mɯ-tɤ-sɤrju-t-a\cmn 我没有让他说话\end{exemple}
\begin{relation-sémantique}\confer{
\hyperlink{Ⓔtɯ-rju}{\textit{ \papi{tɯ-rju}}}
}\end{relation-sémantique}
\begin{relation-sémantique}\confer{
\hyperlink{Ⓔmasɤrɯrju}{\textit{ \papi{masɤrɯrju}}}
}\end{relation-sémantique}
\end{sous-entrée}\end{entrée}

\begin{entrée}
\vedette{\hypertarget{Ⓔarɟambrɯɣ}{\papi{ arɟambrɯɣ}}}\markboth{arɟambrɯɣ}{}\classe{vs}
\begin{définition}\fra aux yeux entourés de rouge (chien)\end{définition}
\begin{définition}\cmn 四眼狗\end{définition}
\begin{relation-sémantique}\confer{
\hyperlink{Ⓔrɟambrɯɣ}{\textit{ \papi{rɟambrɯɣ}}}
}\end{relation-sémantique}\end{entrée}

\begin{entrée}
\vedette{\hypertarget{Ⓔarɟumtɕɤr}{\papi{ arɟumtɕɤr}}}\markboth{arɟumtɕɤr}{}\classe{vi}
\paradigme{\textit{dir :} \jya tɤ-}
\paradigme{\textit{dir :} \jya rɟum}
\begin{définition}\ 
\begin{déclaration}\grammar{comp}\end{déclaration}\end{définition}
\begin{définition}\fra ayant des largeurs différentes\end{définition}
\begin{définition}\cmn 宽窄不一\end{définition}
\begin{exemple}\jya to-k-ɤrɟumtɕɤr-ci\cmn 宽窄不一了\end{exemple}
\begin{exemple}\jya tɯ-ŋga ɯ-spa ɲɯ-ɤrɟumtɕɤr tɕe, ɣɯ́-nɯβʑit ɲɯ-ra\cmn 衣服的材料宽窄不一,要剪一下\end{exemple}
\begin{relation-sémantique}\confer{
\hyperlink{Ⓔtɕɤr}{\textit{ \papi{tɕɤr}}}
}\end{relation-sémantique}\end{entrée}

\begin{entrée}
\vedette{\hypertarget{Ⓔarku}{\papi{ arku}}}\markboth{arku}{}\classe{vi}
\paradigme{\textit{dir :} \jya tɤ-}
\begin{définition}\fra être dans\end{définition}
\begin{définition}\cmn 装在\end{définition}
\begin{exemple}\jya a-tɯ-ci tu, arku\cmn 杯子里装着水\end{exemple}
\begin{relation-sémantique}\confer{
\hyperlink{Ⓔrku}{\textit{ \papi{rku}}}
}\end{relation-sémantique}\end{entrée}

\begin{entrée}
\vedette{\hypertarget{Ⓔarla}{\papi{ arla}}}\markboth{arla}{}
\begin{relation-sémantique}\confer{
\hyperlink{Ⓔrla}{\textit{ \papi{rla}}}
}\end{relation-sémantique}\end{entrée}

\begin{entrée}
\vedette{\hypertarget{Ⓔarlɯrla}{\papi{ arlɯrla}}}\markboth{arlɯrla}{}\classe{vi}
\paradigme{\textit{dir :} \jya nɯ-}
\begin{définition}\fra s'étendre\end{définition}
\begin{définition}\cmn 舒展\end{définition}
\begin{exemple}\jya nɤ-phoŋbu a-nɯ-ɤrlɯrla!\cmn 舒展一下筋骨!\end{exemple}\begin{sous-entrée}
\vedette{\hypertarget{}{\papi{ sɤrlɯrla}}}\markboth{sɤrlɯrla}{}\classe{vt}
\begin{définition}\fra étendre\end{définition}
\begin{définition}\cmn 舒展,伸展\end{définition}
\begin{exemple}\jya nɤ-phoŋbu nɯ-sɤrlɯrle!\cmn 舒展一下筋骨!\end{exemple}
\begin{exemple}\jya sɯjno tɤ-kɤ-rmbɯ nɯra ɲɯ́-wɣ-sɤrlɯrla tɕe a-nɯ-rom\cmn 把堆在一起的草展开(晒干)就会干\end{exemple}
\begin{exemple}\jya ɯ-rŋa ɲɤ-sɤrlɯrla\cmn 他皱着的眉头舒展了\end{exemple}
\end{sous-entrée}\end{entrée}

\begin{entrée}
\vedette{\hypertarget{Ⓔarlɯt}{\papi{ arlɯt}}}\markboth{arlɯt}{}
\classe{vs}
\begin{définition}\fra nombreux\end{définition}
\begin{définition}\cmn 多\end{définition}
\begin{exemple}\jya tɯrme ɲɯ-ɤrlɯt\cmn 人很多\end{exemple}
\begin{exemple}\jya tɯ-ci ɲɯ-ɤrlɯt\cmn 水很多\end{exemple}
\begin{relation-sémantique}\synonyme{
\hyperlink{Ⓔdɤn}{\textit{ \papi{dɤn}}}
}\end{relation-sémantique}
\begin{relation-sémantique}\synonyme{
\hyperlink{Ⓔxcat}{\textit{ \papi{xcat}}}
}\end{relation-sémantique}\end{entrée}

\begin{entrée}
\vedette{\hypertarget{Ⓔarmɤjɯβ}{\papi{ armɤjɯβ}}}\markboth{armɤjɯβ}{}
\classe{vi}
\begin{définition}\fra être le crépuscule\end{définition}
\begin{définition}\cmn 黄昏;天黑\end{définition}
\begin{exemple}\jya ko-k-ɤrmɤjɯβ-ci\cmn 天黑了\end{exemple}\end{entrée}

\begin{entrée}
\vedette{\hypertarget{Ⓔarmbat}{\papi{ armbat}}}\markboth{armbat}{}\classe{vs}
\paradigme{\textit{dir :} \jya \_}
\begin{définition}\fra proche\end{définition}
\begin{définition}\cmn 近\end{définition}
\begin{exemple}\jya kɤ-ɤrmbat nɯ-ɣi\cmn 你来近一点的地方\end{exemple}
\begin{exemple}\jya ki nɤj nɤ-ɕki ɯ-j-armbat?\cmn 离你那里远不远?\end{exemple}\begin{sous-entrée}
\vedette{\hypertarget{}{\papi{ sɤrmbat}}}\markboth{sɤrmbat}{}\classe{vt}
\paradigme{\textit{dir :} \jya \_}
\begin{définition}\ 
\begin{déclaration}\grammar{caus}\end{déclaration}\end{définition}
\begin{définition}\fra rapprocher\end{définition}
\begin{définition}\cmn 拿过来;使靠近\end{définition}
\begin{exemple}\jya ko-sɤrmbat\cmn 他拿过来了\end{exemple}
\begin{exemple}\jya na-sɤrmbat\cmn 他拿过来了\end{exemple}
\end{sous-entrée}\begin{sous-entrée}
\vedette{\hypertarget{}{\papi{ ʑɣɤsɤrmbat}}}\markboth{ʑɣɤsɤrmbat}{}\classe{vi}
\paradigme{\textit{dir :} \jya \_}
\begin{définition}\ 
\begin{déclaration}\grammar{refl}\end{déclaration}
\begin{déclaration}\grammar{caus}\end{déclaration}\end{définition}
\begin{définition}\fra s'approcher\end{définition}
\begin{définition}\cmn 靠近\end{définition}
\begin{exemple}\jya tɤ-ʑɣɤsɤrmbat-a\cmn 我靠近了\end{exemple}
\begin{relation-sémantique}\antonyme{
\hyperlink{Ⓔarqhi}{\textit{ \papi{arqhi}}}
}\end{relation-sémantique}
\begin{relation-sémantique}\confer{
\hyperlink{Ⓔmbarqhi}{\textit{ \papi{mbarqhi}}}
}\end{relation-sémantique}
\begin{relation-sémantique}\confer{
\hyperlink{Ⓔamɯrmbat}{\textit{ \papi{amɯrmbat}}}
}\end{relation-sémantique}
\end{sous-entrée}\end{entrée}

\begin{entrée}
\vedette{\hypertarget{Ⓔarmbɯrmbɯ}{\papi{ armbɯrmbɯ}}}\markboth{armbɯrmbɯ}{}\classe{vi}
\paradigme{\textit{dir :} \jya tɤ-}
\begin{définition}\fra en tas\end{définition}
\begin{définition}\cmn 堆起来的;积在一起\end{définition}
\begin{relation-sémantique}\confer{
\hyperlink{Ⓔrmbɯ}{\textit{ \papi{rmbɯ}}}
}\end{relation-sémantique}\end{entrée}

\begin{entrée}
\vedette{\hypertarget{Ⓔarndɤtsa}{\papi{ arndɤtsa}}}\markboth{arndɤtsa}{}\classe{vs}
\paradigme{\textit{dir :} \jya nɯ-}
\begin{définition}\fra dressé et imposant\end{définition}
\begin{définition}\cmn 矗立\end{définition}
\begin{exemple}\jya jinde kha ra rcanɯ tɯ-mɯ ɯ-pa kɯ-ɤrndɤtsa ʑo kɯ-mbro tu-βzu-nɯ ɲɯ-ŋu\cmn 现在修的房子非常巨大\end{exemple}\end{entrée}

\begin{entrée}
\vedette{\hypertarget{Ⓔarɲɟɤle}{\papi{ arɲɟɤle}}}\markboth{arɲɟɤle}{}\classe{vi}
\paradigme{\textit{dir :} \jya thɯ-}
\begin{définition}\fra se tendre\end{définition}
\begin{définition}\cmn 伸展\end{définition}
\begin{exemple}\jya cho-k-ɤrɲɟɤle-ci\cmn 他伸出来了\cmn 那条蛇是伸着的\end{exemple}
\begin{exemple}\jya kɯ-ɤrɲɟɤle ci ɲɯ-ŋu kɯ-zri tsa ɲɯ-ŋu ɯ-skɤt ɲɯ-ŋu\cmn “伸着”表示“长一点”的意思\end{exemple}\begin{sous-entrée}
\vedette{\hypertarget{}{\papi{ sɤrɲɟɤle}}}\markboth{sɤrɲɟɤle}{}
\begin{définition}\fra étendre\end{définition}
\begin{définition}\cmn 伸\end{définition}
\begin{exemple}\jya ɯ-mɤlɤjaʁ chɤ-sɤrɲɟɤle\cmn 他伸了四肢、他伸了懒腰\end{exemple}
\begin{exemple}\jya kɤ-nɯmdzɯ, nɤ-mi thɯ-nɯsɤrɲɟɤle\cmn 你坐下,把脚伸一下(请人休息)\end{exemple}
\end{sous-entrée}\end{entrée}

\begin{entrée}
\vedette{\hypertarget{Ⓔarŋɤrtɯm}{\papi{ arŋɤrtɯm}}}\markboth{arŋɤrtɯm}{}\classe{vs}
\begin{définition}\fra qui a le visage rond\end{définition}
\begin{définition}\cmn 脸很圆\end{définition}
\begin{relation-sémantique}\confer{
\hyperlink{Ⓔtɯ-rŋa}{\textit{ \papi{tɯ-rŋa}}}
}\end{relation-sémantique}
\begin{relation-sémantique}\confer{
\hyperlink{Ⓔartɯm}{\textit{ \papi{artɯm}}}
}\end{relation-sémantique}\end{entrée}

\begin{entrée}
\vedette{\hypertarget{Ⓔarŋi}{\papi{ arŋi}}}\markboth{arŋi}{}\classe{vi}
\paradigme{\textit{dir :} \jya nɯ-}
\begin{définition}\fra bleu, vert\end{définition}
\begin{définition}\cmn 蓝色;绿色\end{définition}
\begin{exemple}\jya stomku ɲɤ-k-ɤrŋi-ci\cmn 草坪变绿了\end{exemple}
\begin{exemple}\jya xɕaj ɲɤ-k-ɤrŋi-ci\cmn 草变绿了\end{exemple}
\begin{exemple}\jya tɯ-ŋga kɯ-ɤrŋi ɲɯ-ŋu\cmn 衣服是绿色的\end{exemple}
\begin{exemple}\jya tɯ-mɯ kɯ-ɤrŋi\cmn 青天、天宫\end{exemple}
\begin{exemple}\jya tɯ-ci ɯ-tɯ-rnaʁ kɯ ɲɯ-ɤrŋi ʑo ndɯrndɯr\cmn 水很深,显得很蓝\end{exemple}
\begin{exemple}\jya nɤ-tɯ-ɤrŋi kɯ kupa ʑo ɲɯ-tɯ-fse\cmn 你穿那么蓝的衣服,就像汉人一样\end{exemple}\begin{sous-entrée}
\vedette{\hypertarget{}{\papi{ sɤrŋi}}}\markboth{sɤrŋi}{}\classe{vt}
\begin{définition}\ 
\begin{déclaration}\grammar{caus}\end{déclaration}\end{définition}
\begin{définition}\fra rendre bleu\end{définition}
\begin{définition}\cmn 使变蓝\end{définition}
\begin{exemple}\jya tɯ-ŋga ɲɤ-sɤrŋi-t-a\cmn 我(不小心)把衣服弄蓝了\end{exemple}
\end{sous-entrée}\end{entrée}

\begin{entrée}
\vedette{\hypertarget{Ⓔarŋɯlɯz}{\papi{ arŋɯlɯz}}}\markboth{arŋɯlɯz}{} (\variante{arŋilɯz}) 
\classe{vi}
\paradigme{\textit{dir :} \jya nɯ-}
\begin{définition}\fra bleuâtre\end{définition}
\begin{définition}\cmn 淡蓝色\end{définition}
\begin{exemple}\jya laχtɕha ɯ-mdoʁ ɲɯ-ɤrŋɯlɯz\cmn 那个东西的颜色是淡蓝色的\end{exemple}
\begin{exemple}\jya jiɕqha nɯ ɯ-mdoʁ kɯ-ɤrŋɯlɯz ci ɲɯ-ŋu\cmn 那个东西的颜色是淡蓝色的\end{exemple}
\begin{relation-sémantique}\confer{
\hyperlink{Ⓔarŋi}{\textit{ \papi{arŋi}}}
}\end{relation-sémantique}\end{entrée}

\begin{entrée}
\vedette{\hypertarget{Ⓔarɴɢlɯm}{\papi{ arɴɢlɯm}}}\markboth{arɴɢlɯm}{}
\classe{vi}
\paradigme{\textit{dir :} \jya kɤ-}
\begin{définition}\fra concave\end{définition}
\begin{définition}\cmn 凹(地面)\end{définition}
\begin{exemple}\jya ko-k-ɤrɴɢlɯm-ci\cmn 凹进去了\end{exemple}
\begin{exemple}\jya ɯ-thoʁ ɲɯ-ɤrɴɢlɯm\cmn 地面是凹进去的\end{exemple}
\begin{relation-sémantique}\confer{
\hyperlink{Ⓔsqlɯm}{\textit{ \papi{sqlɯm}}}
}\end{relation-sémantique}\begin{sous-entrée}
\vedette{\hypertarget{}{\papi{ sɤrɴɢlɯm}}}\markboth{sɤrɴɢlɯm}{}\classe{vt}
\begin{définition}\fra rendre concave\end{définition}
\begin{définition}\cmn 让……凹进去\end{définition}
\begin{exemple}\jya rdɤstaʁ pjɤ-ɣi tɕe, tʂu pjɤ-sɤrɴɢlɯm\cmn 大石头从上面滚下来了,让路凹进去了\end{exemple}
\end{sous-entrée}\end{entrée}

\begin{entrée}
\vedette{\hypertarget{Ⓔaro}{\papi{ aro}}}\markboth{aro}{}
\classe{vi-t}
\paradigme{\textit{dir :} \jya tɤ-}
\begin{définition}\fra posséder\end{définition}
\begin{définition}\cmn 拥有\end{définition}
\begin{exemple}\jya ɕɯŋgɯ pɯ-me, tham to-k-ɤro-ci\cmn 以前没有,现在有了\end{exemple}
\begin{exemple}\jya aʑo nɯ aro-a\cmn 我拥有那个东西\end{exemple}
\begin{exemple}\jya nɯ pɯ-aro-nɯ\cmn 他们以前拥有那个东西(现在没有了)\end{exemple}
\begin{exemple}\jya nɯ tɤ-aro-nɯ tu-kɯ-ti tɕe, ɕɯŋgɯ pɯ-me, tham to-tu\cmn “他们拥有了”的意思就是以前没有,现在就有了\end{exemple}
\begin{exemple}\jya nɤʑo nɤ-kɯ-ra nɯ aʑo aro-a\cmn 我有你需要的那个东西\end{exemple}
\begin{exemple}\jya aʑo tɤ-rte aro-a\cmn 我有帽子\end{exemple}
\begin{exemple}\jya tɯ-rɟɯ aro-a\cmn 我有财产\end{exemple}
\begin{exemple}\jya aʑo a-kɯ-ɤro nɯ lonba nɯ-kho-t-a\cmn 我把拥有的东西全部给了(他)\end{exemple}
\begin{exemple}\jya aʑo qaʑo aro-a nɯra kɯki ŋu\cmn 这是我所拥有的绵羊\end{exemple}
\begin{exemple}\jya aʑo tɤ-rɟit χsɯm aro-a (= a-rɟit χsɯm tu)\cmn 我有三个孩子\end{exemple}\end{entrée}

\begin{entrée}
\vedette{\hypertarget{Ⓔarqhi}{\papi{ arqhi}}}\markboth{arqhi}{}
\classe{vs}
\paradigme{\textit{dir :} \jya \_}
\begin{définition}\fra lointain\end{définition}
\begin{définition}\cmn 远\end{définition}
\begin{exemple}\jya jiɕqha sɤtɕha ɲɯ-ɤrqhi\cmn 这个地方很远\end{exemple}
\begin{exemple}\jya wuma arqhi\cmn 非常远\end{exemple}
\begin{relation-sémantique}\antonyme{
\hyperlink{Ⓔarmbat}{\textit{ \papi{armbat}}}
}\end{relation-sémantique}
\begin{relation-sémantique}\confer{
\hyperlink{Ⓔmbarqhi}{\textit{ \papi{mbarqhi}}}
}\end{relation-sémantique}
\begin{relation-sémantique}\confer{
\hyperlink{Ⓔamɯrqhi}{\textit{ \papi{amɯrqhi}}}
}\end{relation-sémantique}\begin{sous-entrée}
\vedette{\hypertarget{}{\papi{ sɤrqhi}}}\markboth{sɤrqhi}{}\classe{vt}
\paradigme{\textit{dir :} \jya \_}
\begin{définition}\fra éloigner\end{définition}
\begin{définition}\cmn 把……离远\end{définition}
\begin{exemple}\jya nɤ-mɲaʁ ɯ-ɕki ma-jɤ-tɯ-sɤrqhi ma mɤ-tɯ-sɯχsɤl\cmn 你的视线不要离他太远了,不然看不清楚\end{exemple}
\end{sous-entrée}\begin{sous-entrée}
\vedette{\hypertarget{}{\papi{ ʑɣɤsɤrqhi}}}\markboth{ʑɣɤsɤrqhi}{}\classe{vi}
\paradigme{\textit{dir :} \jya \_}
\begin{définition}\fra s'éloigner\end{définition}
\begin{définition}\cmn 离得远一点\end{définition}
\end{sous-entrée}\end{entrée}

\begin{entrée}
\vedette{\hypertarget{Ⓔarqɯrqoʁ}{\papi{ arqɯrqoʁ}}}\markboth{arqɯrqoʁ}{}
\classe{vi}
\paradigme{\textit{dir :} \jya kɤ-}
\begin{définition}\ 
\begin{déclaration}\grammar{recip}\end{déclaration}\end{définition}
\begin{définition}\fra se prendre dans les bras\end{définition}
\begin{définition}\cmn 互相拥抱\end{définition}
\begin{exemple}\jya nɤ-rʑaβ cho kɤ-tɯ-ɤrqɯrqoʁ-ndʑi\cmn 你跟你妻子互相拥抱了\end{exemple}
\begin{relation-sémantique}\confer{
\hyperlink{Ⓔrqoʁ}{\textit{ \papi{rqoʁ}}}
}\end{relation-sémantique}\end{entrée}

\begin{entrée}
\vedette{\hypertarget{Ⓔarʁɯrʁu}{\papi{ arʁɯrʁu}}}\markboth{arʁɯrʁu}{}
\classe{vs}
\paradigme{\textit{dir :} \jya tɤ-}
\begin{définition}\fra froissé\end{définition}
\begin{définition}\cmn 皱(衣服)\end{définition}
\begin{exemple}\jya nɤ-ku ɲɯ-ɤrʁɯrʁu\cmn 你的头是皱着的\end{exemple}
\begin{exemple}\jya nɤ-ŋga ɲɯ-ɤrʁɯrʁu\cmn 你的衣服是皱着的\end{exemple}
\begin{exemple}\jya to-k-ɤrʁɯrʁu-ci (=to-k-ɤɣɯrʑɯrʑɯɣ-ci)\cmn 变皱了\end{exemple}
\begin{relation-sémantique}\confer{
\hyperlink{Ⓔachɯrʁu}{\textit{ \papi{achɯrʁu}}}
}\end{relation-sémantique}\begin{sous-entrée}
\vedette{\hypertarget{}{\papi{ sɤrʁɯrʁu}}}\markboth{sɤrʁɯrʁu}{}\classe{vt}
\paradigme{\textit{dir :} \jya lɤ-}
\begin{définition}\ 
\begin{déclaration}\grammar{caus}\end{déclaration}\end{définition}
\begin{définition}\fra ramasser (ses jambes)\end{définition}
\begin{définition}\cmn 收回来(脚)\end{définition}
\begin{exemple}\jya a-mi thɯ-sɤstɤko-t-a, a-mi lɤ-sɤrʁɯrʁu-t-a\cmn 我把脚伸出来了,我把脚收回来了\end{exemple}
\end{sous-entrée}\begin{sous-entrée}
\vedette{\hypertarget{}{\papi{ ʑɣɤsɤrʁɯrʁu}}}\markboth{ʑɣɤsɤrʁɯrʁu}{}\classe{vt}
\paradigme{\textit{dir :} \jya tɤ-}
\paradigme{\textit{dir :} \jya tɤ-}
\begin{définition}\ 
\begin{déclaration}\grammar{caus}\end{déclaration}
\begin{déclaration}\grammar{refl}\end{déclaration}\end{définition}
\begin{définition}\fra se ramasser, se blottir\end{définition}
\begin{définition}\cmn 缩成一团\end{définition}
\end{sous-entrée}\end{entrée}

\begin{entrée}
\vedette{\hypertarget{Ⓔartaʁ}{\papi{ artaʁ}}}\markboth{artaʁ}{}\classe{vi}
\paradigme{\textit{dir :} \jya nɯ-}
\begin{définition}\fra fourchu\end{définition}
\begin{définition}\cmn 分叉;成双\end{définition}
\begin{exemple}\jya ki si nɯ tɯ-ldʑa ma pɯ-me ri ɲɤ-k-ɤrtaʁ-ci\cmn 这棵树原来只有一根,现在分叉了\end{exemple}
\begin{relation-sémantique}\confer{
\hyperlink{Ⓔtɤ-rtaʁ}{\textit{ \papi{tɤ-rtaʁ}}}
}\end{relation-sémantique}
\begin{relation-sémantique}\confer{
\hyperlink{Ⓔjmɤrtaʁ}{\textit{ \papi{jmɤrtaʁ}}}
}\end{relation-sémantique}\end{entrée}

\begin{entrée}
\vedette{\hypertarget{Ⓔartaʁlaʁ}{\papi{ artaʁlaʁ}}}\markboth{artaʁlaʁ}{}
\begin{relation-sémantique}\confer{
\hyperlink{Ⓔrtaʁ}{\textit{ \papi{rtaʁ}}}
}\end{relation-sémantique}\end{entrée}

\begin{entrée}
\vedette{\hypertarget{Ⓔartɕhoʁ}{\papi{ artɕhoʁ}}}\markboth{artɕhoʁ}{}\classe{vs}
\paradigme{\textit{dir :} \jya \_}
\begin{définition}\fra rassemblé à un endroit\end{définition}
\begin{définition}\cmn 堆在一边,集中在一个地方\end{définition}
\begin{exemple}\jya tɯrme ra lo-rɟɯɣ-nɯ tɕe tɕelo lo-k-ɤrtɕhoʁ-nɯ-ci\cmn 那些人跑上去集中在上面了\end{exemple}
\begin{sous-entrée}
\vedette{\hypertarget{}{\papi{ sɤrtɕhoʁ}}}\markboth{sɤrtɕhoʁ}{}\classe{vt}
\paradigme{\textit{dir :} \jya \_}
\begin{définition}\fra rassembler dans un coin\end{définition}
\begin{définition}\cmn 把分散的物体堆在一边
\end{définition}
\end{sous-entrée}\end{entrée}

\begin{entrée}
\vedette{\hypertarget{Ⓔartɕi}{\papi{ artɕi}}}\markboth{artɕi}{}\classe{vs}\acception{1}
\paradigme{\textit{dir :} \jya nɯ-}
\begin{définition}\fra être apaisée (soif)\end{définition}
\begin{définition}\cmn 解(渴)\end{définition}
\begin{exemple}\jya a-ɕpaʁ nɯ-artɕi\cmn 我不渴了\end{exemple}\acception{2}
\paradigme{\textit{dir :} \jya kɤ-}
\begin{définition}\fra avoir assez (dormi)\end{définition}
\begin{définition}\cmn (睡)够\end{définition}
\begin{exemple}\jya a-ʑɯβ kɤ-artɕi\cmn 我睡够了\end{exemple}
\begin{sous-entrée}
\vedette{\hypertarget{}{\papi{ sɤrtɕi}}}\markboth{sɤrtɕi}{}\classe{vt}\acception{1}
\paradigme{\textit{dir :} \jya nɯ-}
\begin{définition}\fra apaiser (la soif)\end{définition}
\begin{définition}\cmn 让……解渴\end{définition}
\begin{exemple}\jya ɯ-ɕpaʁ ɲɯ-sɤrtɕi-a ra ma mɤ-pe\cmn 我要让他解渴\end{exemple}\acception{2}
\paradigme{\textit{dir :} \jya kɤ-}
\begin{définition}\fra permettre de dormir assez\end{définition}
\begin{définition}\cmn 让……睡够\end{définition}
\begin{exemple}\jya ɯ-ʑɯβ ci kú-wɣ-sɤrtɕi ɲɯ-ra\cmn 要让他睡够\end{exemple}
\end{sous-entrée}\end{entrée}

\begin{entrée}
\vedette{\hypertarget{Ⓔartɕɯwa}{\papi{ artɕɯwa}}}\markboth{artɕɯwa}{}\classe{n}
\begin{définition}\fra mendiant\end{définition}
\begin{définition}\cmn 乞丐\end{définition}\end{entrée}

\begin{entrée}
\vedette{\hypertarget{Ⓔartsɯɣdu}{\papi{ artsɯɣdu}}}\markboth{artsɯɣdu}{} (\variante{\_artsɯrdu}) 
\classe{vi}
\paradigme{\textit{dir :} \jya pɯ-}
\begin{définition}\fra occupé\end{définition}
\begin{définition}\cmn 事务繁忙\end{définition}
\begin{exemple}\jya a-ma pɯ-artsɯɣdu\cmn 我工作变得很紧张\end{exemple}
\begin{exemple}\jya stonka pɯ-ɣe tɕe ɯ-kɤtɣa cho kɤ-nɤma ra artsɯɣdu\cmn 秋天来了,收割等农活非常忙碌\end{exemple}
\begin{exemple}\jya ɯ-ŋgɯ jɤznɤ tu-kɯ-znɤʁɤmɟa tɕe, ɯ-ndo tɕe kɤ-nɤma mɤ-kɤ-ɤrtsɯɣdu phɤn\cmn 一开始把时间抓紧的话,到最后就不会那么紧张\end{exemple}
\end{entrée}

\begin{entrée}
\vedette{\hypertarget{Ⓔartsɯrtso}{\papi{ artsɯrtso}}}\markboth{artsɯrtso}{}\classe{vi}
\paradigme{\textit{dir :} \jya tɤ-}
\begin{définition}\fra s'empiler\end{définition}
\begin{définition}\cmn 重叠\end{définition}
\begin{exemple}\jya iʑo ji-khɯtsa dɤn tɕe artsɯrtso ʑo ɕti\cmn 我们有很多碗,都是叠在一起的\end{exemple}\begin{sous-entrée}
\vedette{\hypertarget{}{\papi{ sɤrtsɯrtso}}}\markboth{sɤrtsɯrtso}{}\classe{vt}
\paradigme{\textit{dir :} \jya tɤ-}
\begin{définition}\fra empiler\end{définition}
\begin{définition}\cmn 堆起来;垒起来\end{définition}
\begin{relation-sémantique}\synonyme{
\hyperlink{Ⓔrtsɯɣ}{\textit{ \papi{rtsɯɣ}}}
}\end{relation-sémantique}
\end{sous-entrée}\end{entrée}

\begin{entrée}
\vedette{\hypertarget{Ⓔartɯm}{\papi{ artɯm}}}\markboth{artɯm}{}
\classe{vs}
\paradigme{\textit{dir :} \jya tɤ-}
\begin{définition}\fra rond\end{définition}
\begin{définition}\cmn 圆形\end{définition}
\begin{exemple}\jya kɯki ɯ-mŋu ɲɯ-ɤrtɯm\cmn 它的口是圆形的\end{exemple}
\begin{exemple}\jya qajɣi ɲɯ-ɤrtɯm\cmn 馍馍是圆的\end{exemple}
\begin{exemple}\jya @lanqiu ɲɯ-ɤrtɯm\cmn 篮球是圆的\end{exemple}
\begin{exemple}\jya @baigua ɲɯ-ɤrtɯm\cmn 白瓜是圆的\end{exemple}
\begin{relation-sémantique}\confer{
\hyperlink{Ⓔarŋɤrtɯm}{\textit{ \papi{arŋɤrtɯm}}}
}\end{relation-sémantique}\begin{sous-entrée}
\vedette{\hypertarget{}{\papi{ sɤrtɯm}}}\markboth{sɤrtɯm}{}\classe{vt}
\paradigme{\textit{dir :} \jya tɤ-}
\begin{définition}\ 
\begin{déclaration}\grammar{caus}\end{déclaration}\end{définition}
\begin{définition}\fra rendre rond\end{définition}
\begin{définition}\cmn 弄圆\end{définition}
\begin{exemple}\jya si tɤ-kaɣ-a tɕe tɤ-sɤrtɯm-a\cmn 我把木条弯成了圆形\end{exemple}
\end{sous-entrée}\end{entrée}

\begin{entrée}
\vedette{\hypertarget{Ⓔartɯmloʁ}{\papi{ artɯmloʁ}}}\markboth{artɯmloʁ}{}\classe{vs}
\begin{définition}\fra en forme de boule\end{définition}
\begin{définition}\cmn 球形\end{définition}
\begin{exemple}\jya ki tɯrme ki ɲɯ-tshu ɲɯ-ɤrtɯmloʁ ʑo\cmn 这个人很胖,身体像球形一样\end{exemple}
\begin{exemple}\jya ɲɯ-ɤrtɯmloʁ ʑo tslɯɣtslɯɣ\cmn 又圆又硬\end{exemple}\begin{sous-entrée}
\vedette{\hypertarget{}{\papi{ sɤrtɯmloʁ}}}\markboth{sɤrtɯmloʁ}{}\classe{vt}
\paradigme{\textit{dir :} \jya tɤ-}
\begin{définition}\fra faire une boule (de tsampa)\end{définition}
\begin{définition}\cmn (把糌粑)挼成一坨\end{définition}
\begin{exemple}\jya rɟɤɣi tɤ-sɤrtɯmloʁ-a tɕe tɤ-βzu-t-a\cmn 我把糌粑捏成一团了\end{exemple}
\begin{exemple}\jya tɯ-ŋga to-sɤrtɯmloʁ ʑo tslɯɣtslɯɣ\cmn 他把衣服乱裹成一团\end{exemple}
\end{sous-entrée}\end{entrée}

\begin{entrée}
\vedette{\hypertarget{Ⓔartɯrtɤβ}{\papi{ artɯrtɤβ}}}\markboth{artɯrtɤβ}{}\classe{vi}
\paradigme{\textit{dir :} \jya tɤ-}
\begin{définition}\fra s'emmêler\end{définition}
\begin{définition}\cmn 缠在一起\end{définition}
\begin{exemple}\jya tɤ-ri ɲɯ-ɤrtɯrtɤβ\cmn 线缠在一起\end{exemple}
\begin{exemple}\jya tɯmbri tu-ortɯrtɤβ\cmn 绳子缠在一起\end{exemple}
\begin{exemple}\jya tɤ-ri to-k-ɤrtɯrtɤβ-ci tɕe nɯ-sɤqɤrle-t-a\cmn 线缠在一起了,我就把它分开了\end{exemple}
\begin{relation-sémantique}\confer{
\hyperlink{Ⓔrtɤβ}{\textit{ \papi{rtɤβ}}}
}\end{relation-sémantique}
\begin{relation-sémantique}\synonyme{
 \papi{aɬɤt}
}\end{relation-sémantique}\begin{sous-entrée}
\vedette{\hypertarget{}{\papi{ sɤrtɯrtɤβ}}}\markboth{sɤrtɯrtɤβ}{}\classe{vt}
\begin{définition}\ 
\begin{déclaration}\grammar{caus}\end{déclaration}\end{définition}\acception{1}
\begin{définition}\fra emmêler\end{définition}
\begin{définition}\cmn 缠线\end{définition}\acception{2}
\begin{définition}\fra coller à quelqu'un\end{définition}
\begin{définition}\cmn 缠着别人不放
\begin{déclaration}\use{沙尔宗方言}\end{déclaration}\end{définition}
\begin{exemple}\jya nɯ ma ma-kɯ-sɤrtɯrtaβ-a ma ɲɯ-sɤɣdɯɣ\cmn 你不要再缠着我了,很烦\end{exemple}
\end{sous-entrée}\end{entrée}

\begin{entrée}
\vedette{\hypertarget{Ⓔartɯrtoʁ}{\papi{ artɯrtoʁ}}}\markboth{artɯrtoʁ}{}
\begin{relation-sémantique}\confer{
\hyperlink{Ⓔrtoʁ}{\textit{ \papi{rtoʁ}}}
}\end{relation-sémantique}
\end{entrée}

\begin{entrée}
\vedette{\hypertarget{Ⓔarɯcɤrna}{\papi{ arɯcɤrna}}}\markboth{arɯcɤrna}{}\classe{vs}
\paradigme{\textit{dir :} \jya tɤ-}
\begin{définition}\ 
\begin{déclaration}\grammar{denom}\end{déclaration}\end{définition}\acception{1}
\begin{définition}\fra rond\end{définition}
\begin{définition}\cmn 圆形的\end{définition}\acception{2}
\begin{définition}\fra courbées l'une vers l'autre (cornes)\end{définition}
\begin{définition}\cmn 往里面弯(角)\end{définition}
\begin{exemple}\jya jla ɯ-ʁrɯ ɲɯ-ɤrɯcɤrna\cmn 犏牛的角往里面弯,几乎形成了圆形\end{exemple}
\begin{exemple}\jya ɯ-mŋu ɲɯ-ɤrtɯm tɕe ɲɯ-ɤrɯcɤrna\cmn (碗的)口是圆的\end{exemple}
\begin{relation-sémantique}\synonyme{
\hyperlink{Ⓔartɯm}{\textit{ \papi{artɯm}}}
}\end{relation-sémantique}
\begin{relation-sémantique}\confer{
\hyperlink{Ⓔcɤrna}{\textit{ \papi{cɤrna}}}
}\end{relation-sémantique}\end{entrée}

\begin{entrée}
\vedette{\hypertarget{Ⓔarɯɕoʁɕoʁ}{\papi{ arɯɕoʁɕoʁ}}}\markboth{arɯɕoʁɕoʁ}{}\classe{vs}
\begin{définition}\ 
\begin{déclaration}\grammar{denom}\end{déclaration}\end{définition}
\begin{définition}\fra comme du papier\end{définition}
\begin{définition}\cmn 像纸一样\end{définition}
\begin{exemple}\jya ɯ-tɯ-wɣrum kɯ ɲɯ-ɤrɯɕoʁɕoʁ\cmn 像纸一样白\end{exemple}
\begin{relation-sémantique}\confer{
\hyperlink{Ⓔɕoʁɕoʁ}{\textit{ \papi{ɕoʁɕoʁ}}}
}\end{relation-sémantique}\end{entrée}

\begin{entrée}
\vedette{\hypertarget{Ⓔarɯfsapaʁ}{\papi{ arɯfsapaʁ}}}\markboth{arɯfsapaʁ}{}\classe{vs}
\begin{définition}\ 
\begin{déclaration}\grammar{denom}\end{déclaration}\end{définition}
\begin{définition}\fra comme un animal\end{définition}
\begin{définition}\cmn 像牲畜一样\end{définition}
\begin{exemple}\jya ɯ-tɯ-khe kɯ ɲɯ-ɤrɯfsapaʁ ʑo\cmn 他像牲畜一样笨\end{exemple}
\begin{relation-sémantique}\confer{
\hyperlink{Ⓔfsapaʁ}{\textit{ \papi{fsapaʁ}}}
}\end{relation-sémantique}\end{entrée}

\begin{entrée}
\vedette{\hypertarget{Ⓔarɯfsɤlko}{\papi{ arɯfsɤlko}}}\markboth{arɯfsɤlko}{}\classe{vs}
\begin{définition}\fra reculé, à l'écart (endroit)\end{définition}
\begin{définition}\cmn 偏僻\end{définition}\end{entrée}

\begin{entrée}
\vedette{\hypertarget{Ⓔarɯjɤrɤβ}{\papi{ arɯjɤrɤβ}}}\markboth{arɯjɤrɤβ}{} (\variante{arɯjɤrɯrɤβ}) \classe{vs}
\begin{définition}\fra poli\end{définition}
\begin{définition}\cmn 文明;有礼貌
\begin{déclaration} \étymologie{\papi{ja.rabs}}\end{déclaration}\end{définition}\end{entrée}

\begin{entrée}
\vedette{\hypertarget{Ⓔarɯkhɯtsa}{\papi{ arɯkhɯtsa}}}\markboth{arɯkhɯtsa}{}
\classe{vs}
\begin{définition}\fra comme un bol\end{définition}
\begin{définition}\cmn 像碗一样\end{définition}
\begin{exemple}\jya li sɤlaŋphɤn ɯ-tɯ-xtɕi kɯ ɲɯ-ɤrɯkhɯtsa ɕti\cmn 这个盆子像碗一样小\end{exemple}
\begin{relation-sémantique}\confer{
\hyperlink{Ⓔkhɯtsa}{\textit{ \papi{khɯtsa}}}
}\end{relation-sémantique}\end{entrée}

\begin{entrée}
\vedette{\hypertarget{Ⓔarɯlaba}{\papi{ arɯlaba}}}\markboth{arɯlaba}{}\classe{vi}
\begin{définition}\ 
\begin{déclaration}\grammar{denom}\end{déclaration}\end{définition}
\begin{définition}\fra ressemblant à une trompette\end{définition}
\begin{définition}\cmn 像喇叭的形状\end{définition}
\begin{exemple}\jya kɯ-ɤrɯlaba ci ɲɯ-ŋu\cmn 有喇叭的形状\end{exemple}
\begin{relation-sémantique}\synonyme{
\hyperlink{Ⓔajpomxtshɯm}{\textit{ \papi{ajpomxtshɯm}}}
}\end{relation-sémantique}\end{entrée}

\begin{entrée}
\vedette{\hypertarget{Ⓔarɯldʑaŋkɯ}{\papi{ arɯldʑaŋkɯ}}}\markboth{arɯldʑaŋkɯ}{}
\classe{vs}
\paradigme{\textit{dir :} \jya nɯ-}
\begin{définition}\ 
\begin{déclaration}\grammar{denom}\end{déclaration}\end{définition}
\begin{définition}\fra verdâtre\end{définition}
\begin{définition}\cmn 淡绿色\end{définition}
\begin{exemple}\jya ɲɯ-ɤrɯldʑaŋkɯ\cmn 是淡绿色的\end{exemple}
\begin{relation-sémantique}\confer{
\hyperlink{Ⓔldʑaŋkɯ}{\textit{ \papi{ldʑaŋkɯ}}}
}\end{relation-sémantique}\end{entrée}

\begin{entrée}
\vedette{\hypertarget{Ⓔarɯlɯŋkɤr}{\papi{ arɯlɯŋkɤr}}}\markboth{arɯlɯŋkɤr}{}
\classe{vs}
\paradigme{\textit{dir :} \jya nɯ-}
\begin{définition}\fra bleu ciel\end{définition}
\begin{définition}\cmn 天蓝色\end{définition}
\begin{exemple}\jya kɯ-ɤrɯlɯŋkɤr ci ɲɯ-ŋu\cmn 是天蓝色的东西\end{exemple}\end{entrée}

\begin{entrée}
\vedette{\hypertarget{Ⓔarɯmɯntoʁ}{\papi{ arɯmɯntoʁ}}}\markboth{arɯmɯntoʁ}{}\classe{vs}
\begin{définition}\ 
\begin{déclaration}\grammar{denom}\end{déclaration}\end{définition}
\begin{définition}\fra comme une fleur\end{définition}
\begin{définition}\cmn 像花一样
\end{définition}
\begin{exemple}\jya ɯ-tɯ-ɤrɯmɯntoʁ nɯ!\cmn 他像一朵花一样没有价值\end{exemple}
\begin{exemple}\jya ɯ-tɯ-mpɕɤr kɯ ɲɯ-ɤrɯmɯntoʁ\cmn 漂亮得像花一样\end{exemple}
\begin{relation-sémantique}\confer{
\hyperlink{Ⓔmɯntoʁ}{\textit{ \papi{mɯntoʁ}}}
}\end{relation-sémantique}\end{entrée}

\begin{entrée}
\vedette{\hypertarget{Ⓔarɯqromkemdoʁ}{\papi{ arɯqromkemdoʁ}}}\markboth{arɯqromkemdoʁ}{}\classe{vs}
\begin{définition}\ 
\begin{déclaration}\grammar{denom}\end{déclaration}\end{définition}
\begin{définition}\fra violet\end{définition}
\begin{définition}\cmn 带有紫色\end{définition}
\begin{relation-sémantique}\confer{
\hyperlink{Ⓔqromkemdoʁ}{\textit{ \papi{qromkemdoʁ}}}
}\end{relation-sémantique}\end{entrée}

\begin{entrée}
\vedette{\hypertarget{Ⓔarɯʁmɤrsmɯɣ}{\papi{ arɯʁmɤrsmɯɣ}}}\markboth{arɯʁmɤrsmɯɣ}{}\classe{vs}
\begin{définition}\ 
\begin{déclaration}\grammar{denom}\end{déclaration}\end{définition}
\begin{définition}\fra bordeau\end{définition}
\begin{définition}\cmn 带有紫红色\end{définition}
\begin{relation-sémantique}\confer{
\hyperlink{Ⓔʁmɤrsmɯɣ}{\textit{ \papi{ʁmɤrsmɯɣ}}}
}\end{relation-sémantique}\end{entrée}

\begin{entrée}
\vedette{\hypertarget{Ⓔarɯsrɯn}{\papi{ arɯsrɯn}}}\markboth{arɯsrɯn}{}\classe{vs}
\begin{définition}\fra comme du coton\end{définition}
\begin{définition}\cmn 像棉花一样\end{définition}
\begin{exemple}\jya ɯ-tɯ-mnu kɯ ɲɯ-ɤrɯsrɯn\cmn 像棉花一样柔软\end{exemple}\end{entrée}

\begin{entrée}
\vedette{\hypertarget{Ⓔarɯsɯjno}{\papi{ arɯsɯjno}}}\markboth{arɯsɯjno}{}\classe{vs}
\begin{définition}\ 
\begin{déclaration}\grammar{denom}\end{déclaration}\end{définition}
\begin{définition}\fra comme une herbe\end{définition}
\begin{définition}\cmn 像……草一样……\end{définition}
\begin{exemple}\jya kɤ-phɯt ɯ-tɯ-mbat kɯ ɲɯ-ɤrɯsɯjno ʑo\cmn 像拔一根草一样容易拔\end{exemple}
\begin{relation-sémantique}\confer{
\hyperlink{ⒺsɯjnoⒽ1}{\textit{ \papi{sɯjno1}}}
}\end{relation-sémantique}\end{entrée}

\begin{entrée}
\vedette{\hypertarget{Ⓔarɯtɤjpa}{\papi{ arɯtɤjpa}}}\markboth{arɯtɤjpa}{}\classe{vs}
\begin{définition}\ 
\begin{déclaration}\grammar{denom}\end{déclaration}\end{définition}
\begin{définition}\fra comme de la neige\end{définition}
\begin{définition}\cmn 像雪一样\end{définition}
\begin{exemple}\jya ɯ-tɯ-wɣrum kɯ ɲɯ-ɤrɯtɤjpa ʑo\cmn 像雪一样白\end{exemple}
\begin{relation-sémantique}\confer{
\hyperlink{Ⓔtɤjpa}{\textit{ \papi{tɤjpa}}}
}\end{relation-sémantique}\end{entrée}

\begin{entrée}
\vedette{\hypertarget{Ⓔarɯtɤpɤtso}{\papi{ arɯtɤpɤtso}}}\markboth{arɯtɤpɤtso}{}
\classe{vs}
\begin{définition}\fra comme un enfant\end{définition}
\begin{définition}\cmn 像小孩子一样\end{définition}
\begin{exemple}\jya nɤki tɯrme nɯ ɲɯ-ɤrɯtɤpɤtso\cmn 那个人像小孩子一样\end{exemple}
\begin{relation-sémantique}\confer{
\hyperlink{Ⓔtɤ-pɤtso}{\textit{ \papi{tɤ-pɤtso}}}
}\end{relation-sémantique}\end{entrée}

\begin{entrée}
\vedette{\hypertarget{Ⓔarɯtɤtɕɯ}{\papi{ arɯtɤtɕɯ}}}\markboth{arɯtɤtɕɯ}{}
\classe{vs}
\begin{définition}\fra comme un garçon\end{définition}
\begin{définition}\cmn 像男孩子一样\end{définition}
\begin{exemple}\jya nɤki tɕheme nɯ ɯ-tɯ-rkaŋ kɯ ɲɯ-ɤrɯtɤtɕɯ ʑo ɕti\cmn 那个女子像男生一样壮\end{exemple}
\begin{relation-sémantique}\confer{
\hyperlink{Ⓔtɤ-tɕɯ}{\textit{ \papi{tɤ-tɕɯ}}}
}\end{relation-sémantique}\end{entrée}

\begin{entrée}
\vedette{\hypertarget{Ⓔarɯtɕheme}{\papi{ arɯtɕheme}}}\markboth{arɯtɕheme}{}
\classe{vs}
\begin{définition}\fra comme une fille\end{définition}
\begin{définition}\cmn 像女子一样\end{définition}
\begin{exemple}\jya nɤki tɤ-tɕɯ nɯ ɯ-tɯ-ndʑɤm kɯ ɲɯ-ɤrɯtɕheme ʑo ɕti\cmn 那个男生像女子一样温柔\end{exemple}
\begin{relation-sémantique}\confer{
\hyperlink{Ⓔtɕheme}{\textit{ \papi{tɕheme}}}
}\end{relation-sémantique}\end{entrée}

\begin{entrée}
\vedette{\hypertarget{Ⓔarɯzdɯxthɯɣ}{\papi{ arɯzdɯxthɯɣ}}}\markboth{arɯzdɯxthɯɣ}{}\classe{vs}
\begin{définition}\fra qui arrive à peine à survivre\end{définition}
\begin{définition}\cmn 勉强过日子\end{définition}
\begin{exemple}\jya zdɯxthɯɣ\end{exemple}\end{entrée}

\begin{entrée}
\vedette{\hypertarget{Ⓔaʁɤndɯndɤt}{\papi{ aʁɤndɯndɤt}}}\markboth{aʁɤndɯndɤt}{}\classe{adv}
\begin{définition}\fra partout\end{définition}
\begin{définition}\cmn 到处\end{définition}
\begin{relation-sémantique}\confer{
\hyperlink{Ⓔŋotɕuŋondɤt}{\textit{ \papi{ŋotɕuŋondɤt}}}
}\end{relation-sémantique}
\begin{relation-sémantique}\confer{
\hyperlink{Ⓔnɤndɯndɤt}{\textit{ \papi{nɤndɯndɤt}}}
}\end{relation-sémantique}
\end{entrée}

\begin{entrée}
\vedette{\hypertarget{Ⓔaʁdɤt}{\papi{ aʁdɤt}}}\markboth{aʁdɤt}{}\classe{vi}
\paradigme{\textit{dir :} \jya pɯ-}
\paradigme{\textit{dir :} \jya thɯ-}
\begin{définition}\fra glisser\end{définition}
\begin{définition}\cmn 跌倒\end{définition}
\begin{exemple}\jya a-mi pɯ-aʁdɤt\cmn 我的脚(被)绊了一下\end{exemple}
\begin{exemple}\jya ɲɯ-ɤci tɕe pɯ-aʁdat-a\cmn 地很湿,我就跌倒了\end{exemple}
\begin{exemple}\jya ɲɯ-ɣɤrɤβ tɕe pɯ-aʁdat-a\cmn 坡很陡,我就跌倒了\end{exemple}
\begin{exemple}\jya pjɤ-k-ɤʁdɤt-ci\cmn 他跌倒了\end{exemple}
\begin{exemple}\jya rdɤstaʁ ɯ-taʁ pɯ-aʁdat-a\cmn 我给石头绊倒了\end{exemple}\end{entrée}

\begin{entrée}
\vedette{\hypertarget{Ⓔaʁe}{\papi{ aʁe}}}\markboth{aʁe}{}
\classe{vi-t}
\paradigme{\textit{dir :} \jya pɯ-}
\begin{définition}\fra avoir à boire, avoir à manger\end{définition}
\begin{définition}\cmn 喝到,吃到\end{définition}
\begin{exemple}\jya tɤ-ndza-t-a tɕe pɯ-aʁe-a\cmn 我吃到了\end{exemple}
\begin{exemple}\jya paχɕi to-χtɯ ri mɯ-pɯ-aʁe-a\cmn 他买了苹果,但是我没有吃到\end{exemple}\begin{sous-entrée}
\vedette{\hypertarget{}{\papi{ sɤʁe}}}\markboth{sɤʁe}{}\classe{vt}
\paradigme{\textit{dir :} \jya pɯ-}
\begin{définition}\fra donner à manger\end{définition}
\begin{définition}\cmn 给别人吃\end{définition}
\begin{exemple}\jya cha to-χtɯ tɕe pɯ́-wɣ-sɤʁe-a\cmn 他买了酒就给我喝了\end{exemple}
\begin{exemple}\jya mɯ-pɯ-kɯ-sɤʁe-a\cmn 你没有给我吃\end{exemple}
\end{sous-entrée}\end{entrée}

\begin{entrée}
\vedette{\hypertarget{Ⓔaʁjɤr}{\papi{ aʁjɤr}}}\markboth{aʁjɤr}{}\classe{vi}
\paradigme{\textit{dir :} \jya nɯ-}
\begin{définition}\fra être en retard\end{définition}
\begin{définition}\cmn 耽误时间;迟到\end{définition}
\begin{exemple}\jya toʁde nɯ-aʁjar-a\cmn 我耽误了一下\end{exemple}
\begin{exemple}\jya tɯ-rju ɲɯ-ti tɕe, nɯ-aʁjar-a\cmn 他说话,我就耽误了时间\end{exemple}
\begin{exemple}\jya fso tɕe aʁjar-a ra ɲɯ-ŋu\cmn 我明天要耽误一下时间(没有时间上课)\end{exemple}\begin{sous-entrée}
\vedette{\hypertarget{}{\papi{ saʁjɤr}}}\markboth{saʁjɤr}{}\classe{vt}
\paradigme{\textit{dir :} \jya pɯ-}
\begin{relation-sémantique}\confer{
 \papi{caus}
}\end{relation-sémantique}\acception{1}
\begin{définition}\fra retarder, déranger\end{définition}
\begin{définition}\cmn 耽误;打扰\end{définition}
\begin{relation-sémantique}\synonyme{
\hyperlink{Ⓔβzgɤr}{\textit{ \papi{βzgɤr}}}
}\end{relation-sémantique}\acception{2}
\begin{définition}\fra déranger\end{définition}
\begin{définition}\cmn 打扰\end{définition}
\begin{exemple}\jya ma-pɯ-kɯ-saʁjar-a\cmn 你不要耽误我的时间\end{exemple}
\begin{exemple}\jya pɯ-kɯ-saʁjar-a\cmn 你耽误了我的时间\end{exemple}
\begin{exemple}\jya ɲɯ-ta-saʁjɤr\cmn 我在耽误你的时间\end{exemple}
\begin{exemple}\jya ɯ-zgra kɤ-ʑmbri mɤ-pe ma tɯrme saʁjar-a\cmn 我不能发出声音,不然会打扰别人\end{exemple}
\begin{exemple}\jya tɯ-mɯ ɲɯ-lɤt tɕe, ji-ma pɯ́-wɣ-saʁjɤr-i\cmn 下雨就耽误了我们的工作\end{exemple}
\begin{exemple}\jya khɤjhwi ɲɯ-ra tɕe, kɤ-ɣi mɯ-pɯ-ŋgrɯ, pɯ-ta-saʁjɤr\cmn 因为需要开会没能来到,耽误了你的时间\end{exemple}
\end{sous-entrée}\begin{sous-entrée}
\vedette{\hypertarget{}{\papi{ sɤsaʁjɤr}}}\markboth{sɤsaʁjɤr}{}\classe{vi}
\begin{définition}\ 
\begin{déclaration}\grammar{apass}\end{déclaration}\end{définition}
\end{sous-entrée}\begin{sous-entrée}
\vedette{\hypertarget{}{\papi{ ʑɣɤsaʁjɤr}}}\markboth{ʑɣɤsaʁjɤr}{}\classe{vi}
\begin{définition}\ 
\begin{déclaration}\grammar{refl}\end{déclaration}
\begin{déclaration}\grammar{caus}\end{déclaration}\end{définition}
\begin{définition}\fra prendre du retard\end{définition}
\begin{définition}\cmn 耽误时间\end{définition}
\end{sous-entrée}\end{entrée}

\begin{entrée}
\vedette{\hypertarget{Ⓔaʁɟa}{\papi{ aʁɟa}}}\markboth{aʁɟa}{}
\classe{vi}
\paradigme{\textit{dir :} \jya thɯ-}\acception{1}
\begin{définition}\fra chauve\end{définition}
\begin{définition}\cmn 秃\end{définition}\acception{2}
\begin{exemple}\jya ɯ-kɤrme chɤ-k-ɤʁɟa-ci\cmn 他变成光头了\end{exemple}\acception{2}
\begin{définition}\fra dénudé (terrain)\end{définition}
\begin{définition}\cmn 光秃秃
\begin{déclaration} \étymologie{\papi{gja?}}\end{déclaration}\end{définition}
\begin{exemple}\jya jiɕqha stɤmku ɲɯ-ɤʁɟa\cmn 草坪光秃秃的\end{exemple}
\begin{exemple}\jya @Wenchuan zgo ra ɲɯ-ɤʁɟa-nɯ\cmn 汶川的山是光秃秃的\end{exemple}\begin{sous-entrée}
\vedette{\hypertarget{}{\papi{ saʁɟa}}}\markboth{saʁɟa}{}\classe{vt}
\paradigme{\textit{dir :} \jya thɯ-}
\paradigme{\textit{dir :} \jya nɯ-}
\begin{définition}\ 
\begin{déclaration}\grammar{caus}\end{déclaration}\end{définition}
\begin{définition}\fra raser entièrement\end{définition}
\begin{définition}\cmn 剃光;砍光\end{définition}
\begin{exemple}\jya jiɕqha nɯ ɯ-stu ɲɤ-saʁɟa\cmn 他把那个地方砍光了(拔光了草)\end{exemple}
\end{sous-entrée}\end{entrée}

\begin{entrée}
\vedette{\hypertarget{Ⓔaʁɟɤle}{\papi{ aʁɟɤle}}}\markboth{aʁɟɤle}{}
\classe{vi}
\paradigme{\textit{dir :} \jya nɯ-}
\begin{définition}\fra nu, vide (lieu)\end{définition}
\begin{définition}\cmn 光秃秃(不均匀)\end{définition}
\begin{exemple}\jya tɯ-mɯ nɯ-aʁɟɤle\cmn 天晴了,云散了\end{exemple}
\begin{exemple}\jya ɯ-ku ɲɯ-ɤʁɟɤle\cmn 他头发稀少了\end{exemple}
\begin{relation-sémantique}\confer{
\hyperlink{Ⓔaʁɟa}{\textit{ \papi{aʁɟa}}}
}\end{relation-sémantique}\end{entrée}

\begin{entrée}
\vedette{\hypertarget{Ⓔaʁloʁlu}{\papi{ aʁloʁlu}}}\markboth{aʁloʁlu}{}
\classe{vs}
\begin{définition}\fra concave\end{définition}
\begin{définition}\cmn 凹陷\end{définition}
\begin{exemple}\jya ko-k-ɤʁloʁlu-ci\cmn 变成凹进去了\end{exemple}
\begin{exemple}\jya jiɕqha ɲɯ-ɤʁloʁlu tɕe, nɯ ɯ-stu nɯ to-k-ɤjtɯ-ci tɯ-ci\cmn 那个东西是凹进去的,在那里就积了水\end{exemple}
\begin{exemple}\jya ki sɤtɕha ɲɯ-ɤʁloʁlu tɕe, tɯ-ci tu-orɣi mbat\cmn 这个地方是凹进去的,水会在里面积起来流不走\end{exemple}
\begin{relation-sémantique}\synonyme{
 \papi{aχchowɤlu}
}\end{relation-sémantique}
\begin{relation-sémantique}\confer{
\hyperlink{Ⓔɯ-ʁlu}{\textit{ \papi{ɯ-ʁlu}}}
}\end{relation-sémantique}\end{entrée}

\begin{entrée}
\vedette{\hypertarget{Ⓔaʁmbɯm}{\papi{ aʁmbɯm}}}\markboth{aʁmbɯm}{}
\classe{vi}
\paradigme{\textit{dir :} \jya kɤ-}
\begin{définition}\fra concave\end{définition}
\begin{définition}\cmn 凹(物体)\end{définition}
\begin{exemple}\jya tɯthɯ ko-oʁmbɯm\cmn 锅子凹进去了\end{exemple}
\begin{exemple}\jya ko-k-ɤʁmbɯm-ci\cmn 凹进去了\end{exemple}
\begin{exemple}\jya ɯ-laz ko-nɯrpu tɕe ko-oʁmbɯm\cmn 他不小心撞了额头,额头就凹进去了\end{exemple}
\begin{exemple}\jya ma-kɤ-tɯ-sɯrpe ma tha aʁmbɯm\cmn 你不要撞他,不然就会凹进去\end{exemple}\end{entrée}

\begin{entrée}
\vedette{\hypertarget{Ⓔaʁzrɤwolu}{\papi{ aʁzrɤwolu}}}\markboth{aʁzrɤwolu}{}\classe{vs}
\paradigme{\textit{dir :} \jya nɯ-}
\begin{définition}\fra ébouriffé\end{définition}
\begin{définition}\cmn 乱蓬蓬\end{définition}
\begin{exemple}\jya ɯ-ku ɲɯ-ɤʁzrɤwolu\cmn 他头发乱蓬蓬\end{exemple}
\begin{exemple}\jya nɤ-ku ɲɤ-k-ɤʁzrɤwolu-ci tɕe, kɤ-sɤɕɤt ɲɯ-ra\cmn 你的头发变得乱蓬蓬的,要梳头\end{exemple}
\begin{exemple}\jya nɯ ɕɯŋgɯ kɯ-pɯ-pe pɯ-ɕti ri, tham tɕe ɲɤ-k-ɤʁzrɤwolu-ci\cmn 以前好好的,现在变得乱蓬蓬的\end{exemple}\end{entrée}

\begin{entrée}
\vedette{\hypertarget{Ⓔaʁzɯɣβzaŋ}{\papi{ aʁzɯɣβzaŋ}}}\markboth{aʁzɯɣβzaŋ}{}
\classe{vs}
\paradigme{\textit{dir :} \jya tɤ-}
\begin{définition}\ 
\begin{déclaration}\grammar{denom}\end{déclaration}\end{définition}
\begin{définition}\fra d'allure avenante, impressionnant\end{définition}
\begin{définition}\cmn 相貌堂堂
\begin{déclaration} \étymologie{\papi{gzugs.bzaŋ}}\end{déclaration}\end{définition}
\begin{exemple}\jya ɲɯ-ɤʁzɯɣβzaŋ\cmn 他相貌堂堂\end{exemple}
\begin{exemple}\jya nɤki nɯ kɯ-ɤʁzɯɣβzaŋ ci ŋu\cmn 那是一个相貌堂堂的人\end{exemple}
\begin{exemple}\jya nɯ ɕɯŋgɯ ɯ-ʁzɯɣ mɯ-pɯ-pe ri, tham to-k-ɤʁzɯɣβzaŋ-ci\cmn 以前相貌不好,现在相貌变得很好\end{exemple}
\begin{relation-sémantique}\antonyme{
\hyperlink{Ⓔaʁzɯŋɤn}{\textit{ \papi{aʁzɯŋɤn}}}
}\end{relation-sémantique}\end{entrée}

\begin{entrée}
\vedette{\hypertarget{Ⓔaʁzɯŋɤn}{\papi{ aʁzɯŋɤn}}}\markboth{aʁzɯŋɤn}{}
\classe{vs}
\begin{définition}\ 
\begin{déclaration}\grammar{denom}\end{déclaration}\end{définition}
\begin{définition}\fra d'une apparence peu avenante\end{définition}
\begin{définition}\cmn 相貌不好
\begin{déclaration} \étymologie{\papi{gzugs.ŋan}}\end{déclaration}\end{définition}
\begin{relation-sémantique}\antonyme{
\hyperlink{Ⓔaʁzɯɣβzaŋ}{\textit{ \papi{aʁzɯɣβzaŋ}}}
}\end{relation-sémantique}\end{entrée}

\begin{entrée}
\vedette{\hypertarget{Ⓔasaχpɯχpaʁ}{\papi{ asaχpɯχpaʁ}}}\markboth{asaχpɯχpaʁ}{}
\begin{relation-sémantique}\confer{
\hyperlink{Ⓔsaχpaʁ}{\textit{ \papi{saχpaʁ}}}
}\end{relation-sémantique}\end{entrée}

\begin{entrée}
\vedette{\hypertarget{Ⓔasɤmɯβde}{\papi{ asɤmɯβde}}}\markboth{asɤmɯβde}{}
\begin{relation-sémantique}\confer{
\hyperlink{Ⓔβde}{\textit{ \papi{βde}}}
}\end{relation-sémantique}\end{entrée}

\begin{entrée}
\vedette{\hypertarget{Ⓔasɤmɯmtshɯmtshɤm}{\papi{ asɤmɯmtshɯmtshɤm}}}\markboth{asɤmɯmtshɯmtshɤm}{}
\begin{relation-sémantique}\confer{
\hyperlink{Ⓔamɯmtshɤm}{\textit{ \papi{amɯmtshɤm}}}
}\end{relation-sémantique}\end{entrée}

\begin{entrée}
\vedette{\hypertarget{Ⓔasɤmɯsɯz}{\papi{ asɤmɯsɯz}}}\markboth{asɤmɯsɯz}{}\classe{vi}
\paradigme{\textit{dir :} \jya nɯ-}
\begin{définition}\ 
\begin{déclaration}\grammar{recip}\end{déclaration}\end{définition}
\begin{définition}\fra se transmettre les informations les uns aux autres\end{définition}
\begin{définition}\cmn 互相传递消息\end{définition}
\begin{relation-sémantique}\synonyme{
\hyperlink{Ⓔarɤmɯzda}{\textit{ \papi{arɤmɯzda}}}
}\end{relation-sémantique}
\begin{relation-sémantique}\synonyme{
\hyperlink{Ⓔarɤzdɯzda}{\textit{ \papi{arɤzdɯzda}}}
}\end{relation-sémantique}
\begin{relation-sémantique}\synonyme{
\hyperlink{Ⓔasɤmɯmtshɯmtshɤm}{\textit{ \papi{asɤmɯmtshɯmtshɤm}}}
}\end{relation-sémantique}
\begin{relation-sémantique}\confer{
\hyperlink{Ⓔsɯz}{\textit{ \papi{sɯz}}}
}\end{relation-sémantique}\end{entrée}

\begin{entrée}
\vedette{\hypertarget{Ⓔascaʁa}{\papi{ ascaʁa}}}\markboth{ascaʁa}{}\classe{vs}
\begin{définition}\ 
\begin{déclaration}\grammar{denom}\end{déclaration}\end{définition}
\begin{définition}\fra blanc et noir, comme une pie\end{définition}
\begin{définition}\cmn 黑白相间,像喜鹊的羽毛一样\end{définition}
\begin{relation-sémantique}\confer{
\hyperlink{Ⓔscaʁa}{\textit{ \papi{scaʁa}}}
}\end{relation-sémantique}\end{entrée}

\begin{entrée}
\vedette{\hypertarget{Ⓔascɯsco}{\papi{ ascɯsco}}}\markboth{ascɯsco}{}
\begin{relation-sémantique}\confer{
\hyperlink{Ⓔsco}{\textit{ \papi{sco}}}
}\end{relation-sémantique}\end{entrée}

\begin{entrée}
\vedette{\hypertarget{Ⓔaskɯ}{\papi{ askɯ}}}\markboth{askɯ}{}
\classe{vs}
\begin{définition}\fra qui est blanche sur le dos (vache)\end{définition}
\begin{définition}\cmn 脊梁上有花纹(牛类)\end{définition}
\begin{exemple}\jya nɯ nɯŋa nɯ ɲɯ-ɤskɯ\cmn 这头牛脊梁上有白色\end{exemple}\end{entrée}

\begin{entrée}
\vedette{\hypertarget{Ⓔasqhlu}{\papi{ asqhlu}}}\markboth{asqhlu}{}
\classe{vi}
\paradigme{\textit{dir :} \jya pɯ-}
\begin{définition}\fra concave\end{définition}
\begin{définition}\cmn 凹\end{définition}
\begin{exemple}\jya pjɤ-k-ɤsqhlu-ci\cmn 凹进去了\end{exemple}
\begin{relation-sémantique}\synonyme{
\hyperlink{Ⓔaχchowolu}{\textit{ \papi{aχchowolu}}}
}\end{relation-sémantique}
\begin{relation-sémantique}\synonyme{
\hyperlink{Ⓔaʁloʁlu}{\textit{ \papi{aʁloʁlu}}}
}\end{relation-sémantique}
\begin{relation-sémantique}\synonyme{
\hyperlink{Ⓔaqhowolu}{\textit{ \papi{aqhowolu}}}
}\end{relation-sémantique}
\begin{relation-sémantique}\synonyme{
\hyperlink{Ⓔaɕqhlu}{\textit{ \papi{aɕqhlu}}}
}\end{relation-sémantique}
\begin{relation-sémantique}\synonyme{
\hyperlink{Ⓔarɴɢlɯm}{\textit{ \papi{arɴɢlɯm}}}
}\end{relation-sémantique}\end{entrée}

\begin{entrée}
\vedette{\hypertarget{Ⓔasqɯsqɤr}{\papi{ asqɯsqɤr}}}\markboth{asqɯsqɤr}{}
\begin{relation-sémantique}\confer{
\hyperlink{Ⓔsqɤr}{\textit{ \papi{sqɤr}}}
}\end{relation-sémantique}\end{entrée}

\begin{entrée}
\vedette{\hypertarget{Ⓔastu}{\papi{ astu}}}\markboth{astu}{}
\classe{vs}
\paradigme{\textit{dir :} \jya tɤ-}
\paradigme{\textit{dir :} \jya thɯ-}
\begin{définition}\fra droit\end{définition}
\begin{définition}\cmn 直
\begin{déclaration}\use{\stylefv{astu}和\stylefv{astɤko}的区别在于前者只用于不能扭动的东西上}\end{déclaration}\end{définition}
\begin{exemple}\jya si ɲɯ-ɤstu\cmn 树是直的\end{exemple}
\begin{exemple}\jya tʂu ɲɯ-ɤstu\cmn 路是直的\end{exemple}
\begin{exemple}\jya ɕoŋtɕa ɲɯ-ɤstu\cmn 木材是直的\end{exemple}
\begin{relation-sémantique}\confer{
\hyperlink{Ⓔastɤko}{\textit{ \papi{astɤko}}}
}\end{relation-sémantique}
\begin{relation-sémantique}\confer{
\hyperlink{ⒺsɤstuⒽ2}{\textit{ \papi{sɤstu2}}}
}\end{relation-sémantique}
\begin{relation-sémantique}\confer{
\hyperlink{Ⓔɯ-stuⒽ1}{\textit{ \papi{ɯ-stu1}}}
}\end{relation-sémantique}\begin{sous-entrée}
\vedette{\hypertarget{}{\papi{ ɯ-ʁɤri,astu}}}\markboth{ɯ-ʁɤri,astu}{}
\begin{définition}\fra avoir du succès, réussir\end{définition}
\begin{définition}\cmn 顺利\end{définition}
\begin{exemple}\jya ɯ-ʁɤri ɲɯ-ɤstu\cmn 他很顺利\end{exemple}
\begin{exemple}\jya a-ʁɤri pɯ-astu\cmn 我运气好,很顺利\end{exemple}
\begin{exemple}\jya tɕhi tɤ-nɤmɯma-t-a a-ʁɤri astu\cmn 我无论做什么都很顺利\end{exemple}
\begin{exemple}\jya ɯ-ʁɤri ɯ-mɤ-tɯ-ɤstu kɯ\cmn 他多么地不幸运\end{exemple}
\end{sous-entrée}\end{entrée}

\begin{entrée}
\vedette{\hypertarget{Ⓔastɤko}{\papi{ astɤko}}}\markboth{astɤko}{}\classe{vi}
\paradigme{\textit{dir :} \jya thɯ-}
\paradigme{\textit{dir :} \jya tɤ-}
\begin{définition}\fra droit\end{définition}
\begin{définition}\cmn 直;站直;伸直\end{définition}
\begin{exemple}\jya cho-k-ɤstoko-ci\cmn 变直了\end{exemple}
\begin{exemple}\jya kɯki tɯrme ɲɯ-ɤstɤko\cmn 这个人身材端正,站得很直\end{exemple}
\begin{exemple}\jya ɲɯ-tɯ-ɤstɤko\cmn 你身材端正\end{exemple}
\begin{relation-sémantique}\confer{
\hyperlink{Ⓔastu}{\textit{ \papi{astu}}}
}\end{relation-sémantique}
\begin{relation-sémantique}\confer{
\hyperlink{Ⓔsɤstɤko}{\textit{ \papi{sɤstɤko}}}
}\end{relation-sémantique}\end{entrée}

\begin{entrée}
\vedette{\hypertarget{Ⓔasti}{\papi{ asti}}}\markboth{asti}{}\classe{vi}
\paradigme{\textit{dir :} \jya nɯ-}
\begin{définition}\fra être bouché\end{définition}
\begin{définition}\cmn 堵塞着\end{définition}
\begin{exemple}\jya kɯ-spoʁ ɯ-j-ásti?\cmn 堵了没有\end{exemple}
\begin{exemple}\jya kɯ-spoʁ ɲɤ-k-ɤstɯsti-ci tɕe, kɤ-ɣɤmɯt mɯ́j-khɯ\cmn 洞堵上了,不能吹了\end{exemple}
\begin{relation-sémantique}\confer{
\hyperlink{ⒺstiⒽ1}{\textit{ \papi{sti1}}}
}\end{relation-sémantique}\end{entrée}

\begin{entrée}
\vedette{\hypertarget{Ⓔastɯsti}{\papi{ astɯsti}}}\markboth{astɯsti}{}
\classe{vt}
\paradigme{\textit{dir :} \jya nɯ-}
\paradigme{\textit{dir :} \jya thɯ-}
\begin{définition}\fra être bouché\end{définition}
\begin{définition}\cmn 堵塞\end{définition}
\begin{exemple}\jya ɯ-ɕna ɲɯ-ɤstɯsti\cmn 他鼻子塞了\end{exemple}
\begin{exemple}\jya a-sŋɯro nɯ-astɯsti\cmn 我呼吸有点困难\end{exemple}
\begin{relation-sémantique}\confer{
\hyperlink{ⒺstiⒽ1}{\textit{ \papi{sti1}}}
}\end{relation-sémantique}\end{entrée}

\begin{entrée}
\vedette{\hypertarget{Ⓔasɯɣ}{\papi{ asɯɣ}}}\markboth{asɯɣ}{}\classe{vi}
\paradigme{\textit{dir :} \jya kɤ-}
\begin{définition}\fra serré, tendu\end{définition}
\begin{définition}\cmn 紧\end{définition}
\begin{exemple}\jya ɯ-xtɕɤr ɲɯ-ɤsɯɣ\cmn 绑得很紧\end{exemple}
\begin{exemple}\jya ɯ-ma ɲɯ-ɤsɯɣ\cmn 他工作很紧张\end{exemple}
\begin{relation-sémantique}\confer{
\hyperlink{Ⓔsɤsɯɣ}{\textit{ \papi{sɤsɯɣ}}}
}\end{relation-sémantique}\end{entrée}

\begin{entrée}
\vedette{\hypertarget{Ⓔasɯɣɲɯɣɲaʁ}{\papi{ asɯɣɲɯɣɲaʁ}}}\markboth{asɯɣɲɯɣɲaʁ}{}
\begin{relation-sémantique}\confer{
\hyperlink{Ⓔsɯɣɲaʁ}{\textit{ \papi{sɯɣɲaʁ}}}
}\end{relation-sémantique}\end{entrée}

\begin{entrée}
\vedette{\hypertarget{Ⓔasɯjaʁndzɯʁndzu}{\papi{ asɯjaʁndzɯʁndzu}}}\markboth{asɯjaʁndzɯʁndzu}{}
\begin{relation-sémantique}\confer{
\hyperlink{Ⓔsɯjaʁndzu}{\textit{ \papi{sɯjaʁndzu}}}
}\end{relation-sémantique}\end{entrée}

\begin{entrée}
\vedette{\hypertarget{Ⓔasɯmɲɯmɲo}{\papi{ asɯmɲɯmɲo}}}\markboth{asɯmɲɯmɲo}{}\classe{vi}
\paradigme{\textit{dir :} \jya nɯ-}
\begin{définition}\fra se pardonner\end{définition}
\begin{définition}\cmn 互相谅解\end{définition}
\begin{exemple}\jya ndʑiʑo ʁo tɯ-asɯmɲɯmɲo-ndʑi ɕti nɤ\cmn 你们俩倒挺能互相谅解的\end{exemple}\end{entrée}

\begin{entrée}
\vedette{\hypertarget{Ⓔasɯsu}{\papi{ asɯsu}}}\markboth{asɯsu}{}\classe{vi}
\paradigme{\textit{dir :} \jya tɤ-}
\begin{définition}\fra copuler\end{définition}
\begin{définition}\cmn 交配\end{définition}
\begin{exemple}\jya jiɕqha qajɯ ni ɲɯ-ɤsɯsu-ndʑi\cmn 虫子在交配\end{exemple}\end{entrée}

\begin{entrée}
\vedette{\hypertarget{Ⓔasɯsat}{\papi{ asɯsat}}}\markboth{asɯsat}{}
\begin{relation-sémantique}\confer{
\hyperlink{Ⓔsat}{\textit{ \papi{sat}}}
}\end{relation-sémantique}\end{entrée}

\begin{entrée}
\vedette{\hypertarget{Ⓔasɯxɕɯxɕɤt}{\papi{ asɯxɕɯxɕɤt}}}\markboth{asɯxɕɯxɕɤt}{}
\begin{relation-sémantique}\confer{
\hyperlink{Ⓔsɯxɕɤt}{\textit{ \papi{sɯxɕɤt}}}
}\end{relation-sémantique}
\end{entrée}

\begin{entrée}
\vedette{\hypertarget{Ⓔaʂŋaʁ}{\papi{ aʂŋaʁ}}}\markboth{aʂŋaʁ}{}\classe{vi}
\paradigme{\textit{dir :} \jya nɯ-}
\begin{définition}\fra se faire une entorse, se fouler le pied\end{définition}
\begin{définition}\cmn 扭伤;崴\end{définition}
\begin{exemple}\jya ɯ-mi ɲɯ-k-ɤʂŋaʁ-ci\cmn 他崴了脚\end{exemple}
\begin{exemple}\jya ɯ-mke ɲɯ-k-ɤʂŋaʁ-ci\cmn 他崴了脖子\end{exemple}
\begin{exemple}\jya ɯ-mthɤɣ ɲɯ-k-ɤʂŋaʁ-ci\cmn 他崴了腰\end{exemple}
\begin{exemple}\jya a-mi ``rŋaʁ" ʑo ɲɯ-ti tɕe nɯ-aʂŋaʁ\cmn 我的脚“喀嚓”一声就扭伤了\end{exemple}
\end{entrée}

\begin{entrée}
\vedette{\hypertarget{Ⓔatu}{\papi{ atu}}}\markboth{atu}{}
\begin{relation-sémantique}\confer{
\hyperlink{Ⓔnɤtu}{\textit{ \papi{nɤtu}}}
}\end{relation-sémantique}\end{entrée}

\begin{entrée}
\vedette{\hypertarget{Ⓔata}{\papi{ ata}}}\markboth{ata}{}\classe{vi}
\paradigme{\textit{dir :} \jya \_}
\begin{définition}\ 
\begin{déclaration}\grammar{pass}\end{déclaration}\end{définition}
\begin{définition}\fra être posé\end{définition}
\begin{définition}\cmn 放着\end{définition}
\begin{relation-sémantique}\confer{
\hyperlink{Ⓔta}{\textit{ \papi{ta}}}
}\end{relation-sémantique}
\end{entrée}

\begin{entrée}
\vedette{\hypertarget{Ⓔataʁki}{\papi{ ataʁki}}}\markboth{ataʁki}{}\classe{vi}
\begin{définition}\fra l'un au dessus et l'autre en dessous\end{définition}
\begin{définition}\cmn 一个在上面,一个在下面\end{définition}
\begin{exemple}\jya a-mtɕhi cho a-ɕna tu-otaʁki ŋu\cmn 我的嘴巴和鼻子一个在上面,一个在下面\end{exemple}
\begin{relation-sémantique}\confer{
\hyperlink{ⒺtaʁⒽ3}{\textit{ \papi{taʁ3}}}
}\end{relation-sémantique}\begin{sous-entrée}
\vedette{\hypertarget{}{\papi{ sɤtaʁki}}}\markboth{sɤtaʁki}{}\classe{vt}
\paradigme{\textit{dir :} \jya tɤ-}
\begin{définition}\fra mettre l'un au dessus et l'autre en dessous\end{définition}
\begin{définition}\cmn 一个放上面,一个放下面\end{définition}
\begin{exemple}\jya laχtɕha kɤ-ɕɯɴqoʁ tɤ-sɤtaʁki-t-a\cmn 我把那些东西一个挂在上面,一个挂在下面\end{exemple}
\end{sous-entrée}\end{entrée}

\begin{entrée}
\vedette{\hypertarget{Ⓔatɤr}{\papi{ atɤr}}}\markboth{atɤr}{}
\classe{vi}
\paradigme{\textit{dir :} \jya tɤ-}
\begin{définition}\fra tomber\end{définition}
\begin{définition}\cmn 掉下\end{définition}
\begin{exemple}\jya pjɤ-k-ɤtɤr-ci\cmn 掉下去了\end{exemple}
\begin{exemple}\jya pɯ-atɤr\cmn 掉下去了\end{exemple}
\begin{exemple}\jya nɤ-tʂoŋtʂoŋ rca j-atɤr j-atɤr ʑo ɲɯ-ŋu\cmn 你的杯子很快就要掉下去\end{exemple}\end{entrée}

\begin{entrée}
\vedette{\hypertarget{Ⓔatɕaʁ}{\papi{ atɕaʁ}}}\markboth{atɕaʁ}{}
\classe{vi}
\paradigme{\textit{dir :} \jya nɯ-}
\begin{définition}\fra être sali\end{définition}
\begin{définition}\cmn 沾上(脏东西、水)\end{définition}
\begin{exemple}\jya ɯ-taʁ ɲɤ-k-ɤtɕaʁ-ci\cmn 沾在上面了\end{exemple}
\begin{exemple}\jya a-ŋga ɯ-taʁ tɤndʑɯɣ ɲɤ-k-ɤtɕaʁ-ci\cmn 衣服上沾上了树脂\end{exemple}\begin{sous-entrée}
\vedette{\hypertarget{}{\papi{ atɕaʁlaʁ}}}\markboth{atɕaʁlaʁ}{}\classe{vs}
\paradigme{\textit{dir :} \jya nɯ-}
\end{sous-entrée}\begin{sous-entrée}
\vedette{\hypertarget{}{\papi{ sɤtɕaʁ}}}\markboth{sɤtɕaʁ}{}\classe{vt}
\paradigme{\textit{dir :} \jya nɯ-}
\begin{définition}\fra salir\end{définition}
\begin{définition}\cmn 沾到\end{définition}
\begin{exemple}\jya znde ɯ-taʁ tɤrcoʁ nɯ-sɤtɕaʁ-a\cmn 我把泥巴沾到墙壁上了\end{exemple}
\begin{exemple}\jya snaχtsa nɯ́-wɣ-sɤtɕaʁ-a\cmn 我身上沾了墨\end{exemple}
\end{sous-entrée}\begin{sous-entrée}
\vedette{\hypertarget{}{\papi{ sɤtɕaʁlaʁ}}}\markboth{sɤtɕaʁlaʁ}{}\classe{vt}
\paradigme{\textit{dir :} \jya tɤ-}
\begin{définition}\fra salir partout\end{définition}
\begin{définition}\cmn 到处沾
\begin{déclaration}\grammar{caus}\end{déclaration}\end{définition}
\begin{exemple}\jya tɯtshi kú-wɣ-sɤla tɕe mɯ́j-pe ma tɯthɯ cho scoʁ khɯtsa ɲɯ-sɤtɕaʁlaʁ ʑo ɲɯ-ŋu\cmn 煲粥不好,因为锅子、瓢和碗上都会沾得到处都是\end{exemple}
\end{sous-entrée}\end{entrée}

\begin{entrée}
\vedette{\hypertarget{Ⓔatɕaʁlaʁ}{\papi{ atɕaʁlaʁ}}}\markboth{atɕaʁlaʁ}{}
\begin{relation-sémantique}\confer{
\hyperlink{Ⓔatɕaʁ}{\textit{ \papi{atɕaʁ}}}
}\end{relation-sémantique}\end{entrée}

\begin{entrée}
\vedette{\hypertarget{Ⓔatɕɤβ}{\papi{ atɕɤβ}}}\markboth{atɕɤβ}{}\classe{vi}\acception{1}
\paradigme{\textit{dir :} \jya tɤ-}
\begin{définition}\fra arriver à complétion\end{définition}
\begin{définition}\cmn (时间)到了\end{définition}
\begin{exemple}\jya tɯ-sla tɤ-atɕɤβ ʑo pɯ-rɤʑi-a\cmn 我在这里待了整整一个月\end{exemple}\acception{2}
\paradigme{\textit{dir :} \jya nɯ-}
\begin{définition}\fra croisés (mains, vêtements)\end{définition}
\begin{définition}\cmn 交错着(衣服、手)\end{définition}\begin{sous-entrée}
\vedette{\hypertarget{}{\papi{ sɤtɕɤβ}}}\markboth{sɤtɕɤβ}{}\classe{vt}
\paradigme{\textit{dir :} \jya nɯ-}
\begin{définition}\fra croiser\end{définition}
\begin{définition}\cmn 交错(衣服、手)\end{définition}
\begin{exemple}\jya ɯ-jaʁ ɲɤ-sɤtɕɤβ\cmn 他把手交叉放着\end{exemple}
\end{sous-entrée}\end{entrée}

\begin{entrée}
\vedette{\hypertarget{Ⓔatɕhɯtɕhɯ}{\papi{ atɕhɯtɕhɯ}}}\markboth{atɕhɯtɕhɯ}{}
\begin{relation-sémantique}\confer{
\hyperlink{Ⓔtɕhɯ}{\textit{ \papi{tɕhɯ}}}
}\end{relation-sémantique}\end{entrée}

\begin{entrée}
\vedette{\hypertarget{Ⓔatɕhɯz}{\papi{ atɕhɯz}}}\markboth{atɕhɯz}{}
\classe{vi}
\paradigme{\textit{dir :} \jya tɤ-}
\begin{définition}\fra éternuer\end{définition}
\begin{définition}\cmn 打喷嚏\end{définition}
\begin{exemple}\jya nɯ-tɕhomba-a ɲɯ-ŋu ma ɲɯ-ɤtɕhɯz-a\cmn 我在打喷嚏,感冒了\end{exemple}
\begin{exemple}\jya jiʑora tɕe, tu-kɯ-ɤtɕhɯz tɕe, tɯrme ra kɯ, nɤ-kɯ-nɤkhu tu ɲɯ-ŋu tu-ti-nɯ, nɯ maʁ nɤ, tɯ-ci kɯ a-pɯ-kɯ-sɯɣro tɕe, tɯ-ci nɤ-kɯ-mbi tu ŋu tu-ti-nɯ\cmn 人打喷嚏的时候,大家就说“有人要请你”;人被水呛到的时候,别人就说“有人要给你水”\end{exemple}\begin{sous-entrée}
\vedette{\hypertarget{}{\papi{ sɤtɕhɯz}}}\markboth{sɤtɕhɯz}{}\classe{vt}
\paradigme{\textit{dir :} \jya tɤ-}
\begin{définition}\ 
\begin{déclaration}\grammar{caus}\end{déclaration}\end{définition}
\begin{définition}\fra faire éternuer\end{définition}
\begin{définition}\cmn 令人打喷嚏\end{définition}
\begin{exemple}\jya ɕnɤto tɤ-nɯ-lat-a tó-wɣ-sɤtɕhɯz-a\cmn 我吸了鼻烟,被呛得打喷嚏\end{exemple}
\begin{exemple}\jya hajtsu chɯ́-wɣ-pu tɕe, kɯ-sɤtɕhɯz ŋgrɤl\cmn 烧黑椒的时候会让人打喷嚏\end{exemple}
\end{sous-entrée}\end{entrée}

\begin{entrée}
\vedette{\hypertarget{Ⓔatɕɯmthɯt}{\papi{ atɕɯmthɯt}}}\markboth{atɕɯmthɯt}{}\classe{vi}
\paradigme{\textit{dir :} \jya tɤ-}
\begin{définition}\fra être formé de nombreux morceaux de tissu\end{définition}
\begin{définition}\cmn 由很多布块组成\end{définition}
\begin{exemple}\jya ki raz kɯ ra kɤ-tɕɯmthɯt ɲɯ-khɯ\cmn 可以把这几块布拼在一起\end{exemple}
\begin{exemple}\jya ki raz kɯ ra a-pɯ-tu tɕe, ɲɯ-ɤtɕɯmthɯt ɕti\cmn 有了这些布块就可以拼成有用的东西\end{exemple}\end{entrée}

\begin{entrée}
\vedette{\hypertarget{Ⓔatɕɯtɕit}{\papi{ atɕɯtɕit}}}\markboth{atɕɯtɕit}{}\classe{vi}
\paradigme{\textit{dir :} \jya nɯ-}
\begin{définition}\fra être incomplet\end{définition}
\begin{définition}\cmn 遗漏\end{définition}
\begin{exemple}\jya a-@cai pɯ-atɕɯtɕit\cmn 我的菜掉了一些\end{exemple}
\begin{exemple}\jya laχtɕha koŋla tɤ-rɤwum tɕe a-mɤ-nɯ-ɤtɕɯtɕit\cmn 你要把东西带全,不要带漏\end{exemple}
\begin{relation-sémantique}\synonyme{
\hyperlink{Ⓔantɕhoʁjɤr}{\textit{ \papi{antɕhoʁjɤr}}}
}\end{relation-sémantique}\begin{sous-entrée}
\vedette{\hypertarget{}{\papi{ sɤtɕɯtɕit}}}\markboth{sɤtɕɯtɕit}{}\classe{vt}
\paradigme{\textit{dir :} \jya nɯ-}
\paradigme{\textit{construction :} \jya infinitive}
\begin{définition}\fra perdre\end{définition}
\begin{définition}\cmn 漏掉(一点)\end{définition}
\begin{exemple}\jya ɲɤ-sɤtɕɯtɕit-a\cmn 我漏掉了\end{exemple}
\begin{exemple}\jya kɤ-nɯkɯɕnom ma-nɯ-tɯ-sɤtɕɯtɕit\cmn 你捡青稞穗的时候不要漏掉一些\end{exemple}
\begin{exemple}\jya laχtɕha koŋla tɤ-rɤwum ma tɯ-sɤtɕɯtɕit\cmn 你要把东西带全,不要带漏\end{exemple}
\end{sous-entrée}\end{entrée}

\begin{entrée}
\vedette{\hypertarget{Ⓔatɕɯtʂi}{\papi{ atɕɯtʂi}}}\markboth{atɕɯtʂi}{}
\classe{vi}
\paradigme{\textit{dir :} \jya \_}
\begin{définition}\fra se produire en même temps\end{définition}
\begin{définition}\cmn 几件事同时进行;状况没有改变\end{définition}
\begin{exemple}\jya nɤ-kɯ-mŋɤm kɯ-mna ɯ-ɲɯ-ɤtɕɯtʂi\cmn 你的病继续好转吗?\end{exemple}
\begin{exemple}\jya ɲɯ-ɤtɕɯtʂi ɕti\cmn 没有变化\end{exemple}
\begin{relation-sémantique}\confer{
\hyperlink{Ⓔsɤtɕɯtʂi}{\textit{ \papi{sɤtɕɯtʂi}}}
}\end{relation-sémantique}\end{entrée}

\begin{entrée}
\vedette{\hypertarget{Ⓔatɕɯxtaʁ}{\papi{ atɕɯxtaʁ}}}\markboth{atɕɯxtaʁ}{}
\begin{relation-sémantique}\confer{
\hyperlink{ⒺtaʁⒽ2}{\textit{ \papi{taʁ2}}}
}\end{relation-sémantique}\end{entrée}

\begin{entrée}
\vedette{\hypertarget{Ⓔatɕɯxtʂot}{\papi{ atɕɯxtʂot}}}\markboth{atɕɯxtʂot}{}
\classe{vi}
\paradigme{\textit{dir :} \jya tɤ-}
\begin{définition}\fra prospère\end{définition}
\begin{définition}\cmn 兴旺\end{définition}
\begin{exemple}\jya ɯ-@gongsi ɲɯ-ɤtɕɯxtʂot\cmn 他的公司很兴旺\end{exemple}
\begin{exemple}\jya kɤntɕhaʁ wuma ʑo ɲɯ-ɤtɕɯxtʂot, tɕhi kɯ-ra ʑo ɣɤʑu\cmn 城市非常兴旺,要什么就有什么\end{exemple}\begin{sous-entrée}
\vedette{\hypertarget{}{\papi{ sɤtɕɯxtʂot}}}\markboth{sɤtɕɯxtʂot}{}\classe{vt}
\paradigme{\textit{dir :} \jya nɯ-}
\begin{définition}\ 
\begin{déclaration}\grammar{caus}\end{déclaration}\end{définition}
\begin{définition}\fra rendre prospère, attiser (feu)\end{définition}
\begin{définition}\cmn 使兴旺;拨旺\end{définition}
\begin{exemple}\jya smi nɯ-sɤtɕɯxtʂo-t-a\cmn 我把火拨旺了一些\end{exemple}
\end{sous-entrée}\end{entrée}

\begin{entrée}
\vedette{\hypertarget{Ⓔathɤri}{\papi{ athɤri}}}\markboth{athɤri}{} (\variante{athɤrɯri}) \classe{vi}
\begin{définition}\fra se connecter\end{définition}
\begin{définition}\cmn 自然地连接起来\end{définition}
\begin{relation-sémantique}\confer{
\hyperlink{Ⓔsɤthɤri}{\textit{ \papi{sɤthɤri}}}
}\end{relation-sémantique}\end{entrée}

\begin{entrée}
\vedette{\hypertarget{Ⓔathi}{\papi{ athi}}}\markboth{athi}{}\classe{adv}
\begin{définition}\fra en aval\end{définition}
\begin{définition}\cmn 下游\end{définition}
\begin{relation-sémantique}\confer{
\hyperlink{Ⓔtɕɤthi}{\textit{ \papi{tɕɤthi}}}
}\end{relation-sémantique}\end{entrée}

\begin{entrée}
\vedette{\hypertarget{Ⓔathoʁmphrɤt}{\papi{ athoʁmphrɤt}}}\markboth{athoʁmphrɤt}{}\classe{vt}
\paradigme{\textit{dir :} \jya tɤ-}
\begin{définition}\fra adéquate (pièces)\end{définition}
\begin{définition}\cmn 吻合(零件、盖子)\end{définition}
\begin{exemple}\jya ɕoŋβzu ɯ-kɯ-βzu nɯ ɣɯ ɯ-tɯ-sprɤt nɯra wuma ɲɯ-ɤthoʁmphrɤt\cmn 他做木匠的时候,把接头接得非常好\end{exemple}
\begin{relation-sémantique}\confer{
\hyperlink{Ⓔsɤthoʁmphrɤt}{\textit{ \papi{sɤthoʁmphrɤt}}}
}\end{relation-sémantique}\end{entrée}

\begin{entrée}
\vedette{\hypertarget{Ⓔathoχɕaβ}{\papi{ athoχɕaβ}}}\markboth{athoχɕaβ}{}\classe{vi}
\begin{définition}\fra qui peut être relié\end{définition}
\begin{définition}\cmn (可以)连在一起\end{définition}
\begin{exemple}\jya tɤ-ri ɯ-ɕnɤz ni mɯ-ɲɯ-ɤthoχɕaβ / tɤ-ri ɯ-ɕnɤz mɯ-ɲɯ-ɤthoχɕaβ-ndʑi\cmn 绳子的的两头不能接在一起\end{exemple}
\begin{relation-sémantique}\synonyme{
\hyperlink{Ⓔalɤɣɯ}{\textit{ \papi{alɤɣɯ}}}
}\end{relation-sémantique}\end{entrée}

\begin{entrée}
\vedette{\hypertarget{Ⓔatsa}{\papi{ atsa}}}\markboth{atsa}{}
\classe{vi}
\paradigme{\textit{dir :} \jya pɯ-}
\paradigme{\textit{dir :} \jya kɤ-}
\paradigme{\textit{dir :} \jya tɤ-}
\begin{définition}\fra être planté\end{définition}
\begin{définition}\cmn 插着\end{définition}
\begin{exemple}\jya tɤtshoʁ to-k-ɤtsa-ci\cmn 钉子是插着的\end{exemple}\begin{sous-entrée}
\vedette{\hypertarget{}{\papi{ sɤtsa}}}\markboth{sɤtsa}{}\classe{vt}
\begin{définition}\ 
\begin{déclaration}\grammar{caus}\end{déclaration}\end{définition}\acception{1}
\begin{définition}\fra planter, enfoncer\end{définition}
\begin{définition}\cmn 插\end{définition}\acception{2}
\begin{définition}\fra poignarder\end{définition}
\begin{définition}\cmn 捅(一刀)\end{définition}
\end{sous-entrée}\begin{sous-entrée}
\vedette{\hypertarget{}{\papi{ ʑɣɤsɤtsa}}}\markboth{ʑɣɤsɤtsa}{}\classe{vi}
\paradigme{\textit{dir :} \jya thɯ-}
\begin{définition}\ 
\begin{déclaration}\grammar{refl}\end{déclaration}
\begin{déclaration}\grammar{caus}\end{déclaration}\end{définition}
\begin{définition}\fra se verrouiller automatiquement\end{définition}
\begin{définition}\cmn 自动地锁起来(门)\end{définition}
\begin{exemple}\jya kɯm chɤ-nɯ-ʑɣɤsɤtsa\cmn 门自动地关起来了\end{exemple}
\end{sous-entrée}\end{entrée}

\begin{entrée}
\vedette{\hypertarget{Ⓔatsatsa}{\papi{ atsatsa}}}\markboth{atsatsa}{}\classe{intj}
\begin{définition}\fra exprime la douleur (brûlure)\end{définition}
\begin{définition}\cmn 表示很痛,很烫\end{définition}\end{entrée}

\begin{entrée}
\vedette{\hypertarget{Ⓔatshɤxtshɯ}{\papi{ atshɤxtshɯ}}}\markboth{atshɤxtshɯ}{}\classe{vi}
\paradigme{\textit{dir :} \jya tɤ-}
\paradigme{\textit{construction :} \jya infinitive}
\paradigme{\textit{construction :} \jya degree}
\begin{définition}\fra pressé\end{définition}
\begin{définition}\cmn 急促;紧张\end{définition}
\begin{exemple}\jya ɯ-tɯ-ɕe ɲɯ-ɤtshɤxtshɯ\cmn 他去得很仓促\end{exemple}
\begin{exemple}\jya kɤ-nɤma ɲɯ-ɤtshɤxtshɯ\cmn 他工作很紧张\end{exemple}
\begin{exemple}\jya a-sŋɯro ɲɯ-ɤtshɤxtshɯ\cmn 我呼吸很困难\end{exemple}\end{entrée}

\begin{entrée}
\vedette{\hypertarget{Ⓔatshoʁ}{\papi{ atshoʁ}}}\markboth{atshoʁ}{}
\begin{relation-sémantique}\confer{
\hyperlink{Ⓔtshoʁ}{\textit{ \papi{tshoʁ}}}
}\end{relation-sémantique}\end{entrée}

\begin{entrée}
\vedette{\hypertarget{Ⓔatsɯtsu}{\papi{ atsɯtsu}}}\markboth{atsɯtsu}{}\classe{vi}
\paradigme{\textit{dir :} \jya pɯ-}
\begin{définition}\fra avoir le temps\end{définition}
\begin{définition}\cmn 来得及\end{définition}
\begin{exemple}\jya nɤ-kɯ-mŋɤm a-tɤ-mna tɕe, kɤ-nɤma atsɯtsu ɕti wo\cmn 你的病康复了,工作还来得及做。\cmn kɤ-nɯna mɤ-atsɯtsu\end{exemple}
\begin{exemple}\jya 来不及休息\end{exemple}
\begin{relation-sémantique}\confer{
\hyperlink{Ⓔtsu}{\textit{ \papi{tsu}}}
}\end{relation-sémantique}\end{entrée}

\begin{entrée}
\vedette{\hypertarget{Ⓔatʂoʁloʁ}{\papi{ atʂoʁloʁ}}}\markboth{atʂoʁloʁ}{}
\classe{vi}
\paradigme{\textit{dir :} \jya tɤ-}
\begin{définition}\fra être mélangé\end{définition}
\begin{définition}\cmn 混合\end{définition}
\begin{exemple}\jya tɤɕi qaj atʂoʁloʁ\cmn 青稞和小麦混在一起了\end{exemple}
\begin{exemple}\jya tɯ-rdoʁ tɯ-rdoʁ kɯ fɕɤt-i ma lonba kɯ tɯrca tu-ti-j ri ji-rju atʂoʁloʁ\cmn 我们一个一个地讲故事,要是我们一起讲的话就会乱\end{exemple}\begin{sous-entrée}
\vedette{\hypertarget{}{\papi{ sɤtʂoʁloʁ}}}\markboth{sɤtʂoʁloʁ}{}\classe{vt}
\paradigme{\textit{dir :} \jya tɤ-}
\paradigme{\textit{dir :} \jya pɯ-}
\begin{définition}\fra mélanger\end{définition}
\begin{définition}\cmn 混在一起\end{définition}
\begin{exemple}\jya tɕi-ŋga pɯ-sɤtʂoʁloʁ-a\cmn 我们俩的衣服装在一起\end{exemple}
\end{sous-entrée}\end{entrée}

\begin{entrée}
\vedette{\hypertarget{Ⓔatɯɣ}{\papi{ atɯɣ}}}\markboth{atɯɣ}{}\classe{vi-t}
\paradigme{\textit{dir :} \jya nɯ-}\acception{1}
\begin{définition}\fra rencontrer\end{définition}
\begin{définition}\cmn 遇见\end{définition}\acception{2}
\begin{définition}\fra toucher\end{définition}
\begin{définition}\cmn 触碰\end{définition}
\begin{exemple}\jya nɯ-atɯɣ-ndʑi\cmn 他们俩遇到了(他)\end{exemple}
\begin{exemple}\jya tɕhomba ɲɤ-k-ɤtɯɣ-a-ci\cmn 我感冒了\end{exemple}\begin{sous-entrée}
\vedette{\hypertarget{}{\papi{ anɤtɯtɯɣ}}}\markboth{anɤtɯtɯɣ}{}\classe{vi}
\begin{définition}\ 
\begin{déclaration}\grammar{autoben}\end{déclaration}
\begin{déclaration}\grammar{recip}\end{déclaration}\end{définition}
\begin{définition}\fra se rencontrer par hasard\end{définition}
\begin{définition}\cmn 正巧相遇\end{définition}
\begin{exemple}\jya kɯ-rɤχtɯ jo-ɕe-ndʑi tɕe ɲɤ-k-ɤnɤtɯtɯɣ-ndʑi-ci\cmn 他们俩去卖东西,偶然地碰见了\end{exemple}
\end{sous-entrée}\begin{sous-entrée}
\vedette{\hypertarget{}{\papi{ nɤtɯɣ}}}\markboth{nɤtɯɣ}{}\classe{vi}
\begin{définition}\ 
\begin{déclaration}\grammar{autoben}\end{déclaration}\end{définition}
\begin{définition}\fra se retrouver dans/avec (par hasard)\end{définition}
\begin{définition}\cmn (无意中)遇到\end{définition}
\begin{exemple}\jya jisŋi kɯ-ɣɤndʐo tɕe ɲɤ-nɤtɯɣ-a\cmn 今天我碰上个大冷天\end{exemple}
\begin{exemple}\jya a-ʁɤri pɯ-astu ma kɯ-pe ɲɤ-nɤtɯɣ-a\cmn 我运气好,拿到好东西了\end{exemple}
\end{sous-entrée}\begin{sous-entrée}
\vedette{\hypertarget{}{\papi{ znɤtɯɣ}}}\markboth{znɤtɯɣ}{} (\variante{znɤtɯtɯɣ}) \classe{vt}
\paradigme{\textit{dir :} \jya \_}\acception{1}
\begin{définition}\fra mettre bien en place\end{définition}
\begin{définition}\cmn 对端;对着放\end{définition}\acception{2}
\begin{définition}\fra profiter de l'occasion\end{définition}
\begin{définition}\cmn 趁……的机会、恰恰在那个时候……\end{définition}
\begin{exemple}\jya kɤ-qanɯ ʑo a-jɤ-tɯ-znɤtɯɣ tɕe a-jɤ-tɯ-ɣi nɯ!\cmn 你趁天黑的时候来吧\end{exemple}
\begin{exemple}\jya kɤ-nɯʑɯβ ʑo a-nɯ-tɯ-znɤtɯɣ tɕe mɤ-mtshɤm\cmn 你趁他睡着了的机会,他就听不到\end{exemple}
\begin{exemple}\jya lu-fsoʁ ɕɯŋgɯ ʑo a-nɯ-tɯ-znɤtɯɣ\cmn 你趁天还没有亮\end{exemple}\acception{3}
\begin{définition}\fra laisser à\end{définition}
\begin{définition}\cmn 让给\end{définition}
\begin{exemple}\jya kɯki laχtɕha kɯ-pe tɤ-tu tɕe, nɤʑo ɲɯ-ta-znɤtɯɣ jɤɣ\cmn 有好东西,我可以让给你\end{exemple}
\end{sous-entrée}\begin{sous-entrée}
\vedette{\hypertarget{}{\papi{ ʑɣɤnɤtɯɣ}}}\markboth{ʑɣɤnɤtɯɣ}{}\classe{vi}
\paradigme{\textit{dir :} \jya \_}
\begin{définition}\ 
\begin{déclaration}\grammar{refl}\end{déclaration}\end{définition}
\begin{définition}\fra faire en sorte de rencontrer, d'obtenir\end{définition}
\begin{définition}\cmn 想办法拿到/遇到\end{définition}
\begin{exemple}\jya laχtɕha kɯ-pe tɤ-tu tɕe, ɲɯ-kɯ-ʑɣɤ-nɤtɯɣ ra\cmn 有好东西的时候,要想办法拿到手\end{exemple}
\end{sous-entrée}\begin{sous-entrée}
\vedette{\hypertarget{}{\papi{ ʑɣɤsɤtɯɣ}}}\markboth{ʑɣɤsɤtɯɣ}{}\classe{vi}
\begin{définition}\ 
\begin{déclaration}\grammar{refl}\end{déclaration}
\begin{déclaration}\grammar{caus}\end{déclaration}\end{définition}
\begin{définition}\fra aller à la rencontrer de\end{définition}
\begin{définition}\cmn 去找(某人)\end{définition}
\begin{relation-sémantique}\confer{
\hyperlink{Ⓔamɯtɯɣ}{\textit{ \papi{amɯtɯɣ}}}
}\end{relation-sémantique}
\end{sous-entrée}\end{entrée}

\begin{entrée}
\vedette{\hypertarget{Ⓔatɯta}{\papi{ atɯta}}}\markboth{atɯta}{}\classe{vi}
\paradigme{\textit{dir :} \jya nɯ-}
\begin{définition}\ 
\begin{déclaration}\grammar{recip}\end{déclaration}\end{définition}
\begin{définition}\fra se relâcher les uns les autres\end{définition}
\begin{définition}\cmn 放开对方,放手\end{définition}
\begin{exemple}\jya ɲɤ-k-ɤtɯta-nɯ-ci\cmn 他们放开对方了\end{exemple}\begin{sous-entrée}
\vedette{\hypertarget{}{\papi{ sɤtɯta}}}\markboth{sɤtɯta}{}\classe{vt}
\paradigme{\textit{dir :} \jya nɯ-}
\begin{définition}\fra séparer (des gens qui se battent)\end{définition}
\begin{définition}\cmn 劝开\end{définition}
\begin{relation-sémantique}\confer{
\hyperlink{Ⓔta}{\textit{ \papi{ta}}}
}\end{relation-sémantique}
\end{sous-entrée}\end{entrée}

\begin{entrée}
\vedette{\hypertarget{Ⓔawij}{\papi{ awij}}}\markboth{awij}{}\classe{vi}
\paradigme{\textit{dir :} \jya kɤ-}
\begin{définition}\fra se fermer (yeux)\end{définition}
\begin{définition}\cmn 闭着眼睛\end{définition}
\begin{exemple}\jya tɯ-ʑɯβ pɯ-ɣe tɕe, tɯ-mɲaʁ ɲɯ-ɤwi\cmn 打瞌睡的时候,眼睛是闭着的\end{exemple}\begin{sous-entrée}
\vedette{\hypertarget{}{\papi{ sɤwij}}}\markboth{sɤwij}{}
\paradigme{\textit{dir :} \jya kɤ-}
\begin{définition}\fra fermer (yeux)\end{définition}
\begin{définition}\cmn 闭上(眼睛)\end{définition}
\begin{exemple}\jya ɯ-mɲaʁ ra ko-sɤwij\cmn 他闭上眼睛了\end{exemple}\classe{vt}
\end{sous-entrée}\end{entrée}

\begin{entrée}
\vedette{\hypertarget{Ⓔawɯwum}{\papi{ awɯwum}}}\markboth{awɯwum}{}\classe{vi}
\paradigme{\textit{dir :} \jya tɤ-}
\paradigme{\textit{dir :} \jya thɯ-}
\begin{définition}\fra se rassembler\end{définition}
\begin{définition}\cmn 聚集在一起\end{définition}
\begin{exemple}\jya kɤ-nɤʁaʁ tɤmda tɕe chɯ-ɤwɯwum-i ŋgrɤl\cmn 要玩的时候,我们就聚在一起\end{exemple}
\begin{relation-sémantique}\confer{
\hyperlink{Ⓔwum}{\textit{ \papi{wum}}}
}\end{relation-sémantique}\end{entrée}

\begin{entrée}
\vedette{\hypertarget{Ⓔaxtɕɯxte}{\papi{ axtɕɯxte}}}\markboth{axtɕɯxte}{}
\classe{vs}
\begin{définition}\fra de taille différente\end{définition}
\begin{définition}\cmn 大小不一\end{définition}
\begin{exemple}\jya kɤndʑɯʁi ni wuma ɲɯ-ɤxtɕɯxte-ndʑi\cmn 两兄弟一个大一个小\end{exemple}
\begin{exemple}\jya kɯ-ɤxtɕɯxte ci ɣɤʑu-ndʑi\cmn 有一个大有一个小\end{exemple}
\begin{relation-sémantique}\confer{
\hyperlink{Ⓔxtɕi}{\textit{ \papi{xtɕi}}}
}\end{relation-sémantique}
\begin{relation-sémantique}\confer{
\hyperlink{Ⓔmɯxte}{\textit{ \papi{mɯxte}}}
}\end{relation-sémantique}\end{entrée}

\begin{entrée}
\vedette{\hypertarget{Ⓔaχa}{\papi{ aχa}}}\markboth{aχa}{}
\classe{vi}
\paradigme{\textit{dir :} \jya nɯ-}
\paradigme{\textit{dir :} \jya pɯ-}
\begin{définition}\fra manquer un morceau\end{définition}
\begin{définition}\cmn 缺(口)\end{définition}
\begin{exemple}\jya tʂu ɲɯ-ɤχa\cmn 路面破了坑\end{exemple}
\begin{exemple}\jya ɯ-ɕɣa ɲɯ-ɤχa\cmn 他牙齿缺了个口子\end{exemple}
\begin{exemple}\jya pjɤ-k-ɤχa-ci\cmn 少了个口子\end{exemple}
\begin{exemple}\jya khɯtsa ɲɯ-ɤχa\cmn 碗缺了个口子\end{exemple}
\begin{exemple}\jya ɯ-sta ɲɯ-ɤχa\cmn 他不在(缺席了)\end{exemple}
\begin{relation-sémantique}\confer{
\hyperlink{Ⓔɕɣɤχa}{\textit{ \papi{ɕɣɤχa}}}
}\end{relation-sémantique}\begin{sous-entrée}
\vedette{\hypertarget{}{\papi{ sɤχa}}}\markboth{sɤχa}{}\classe{vt}
\paradigme{\textit{dir :} \jya nɯ-}
\begin{définition}\ 
\begin{déclaration}\grammar{caus}\end{déclaration}\end{définition}
\begin{définition}\fra faire perdre un morceau\end{définition}
\begin{définition}\cmn 使缺口\end{définition}
\begin{exemple}\jya ɲɤ-sɤχa\cmn 他(把这个东西)弄缺了个口子\end{exemple}
\end{sous-entrée}\end{entrée}

\begin{entrée}
\vedette{\hypertarget{Ⓔaχchowolu}{\papi{ aχchowolu}}}\markboth{aχchowolu}{}\classe{vi}
\paradigme{\textit{dir :} \jya kɤ-}
\begin{définition}\fra creux, concave\end{définition}
\begin{définition}\cmn 凹\end{définition}
\begin{exemple}\jya tʂu kɯ-ɤχchowolu ɯ-stu nɯ tɕu, @qiche tu-ɣɤrkhoŋloŋ ʑo ɲɯ-ŋu tɕe ɲɯ-sɤɣdɯɣ\cmn 路凹下去的地方,汽车就会颠簸,坐起来很不舒服\end{exemple}
\begin{relation-sémantique}\synonyme{
\hyperlink{Ⓔaʁloʁlu}{\textit{ \papi{aʁloʁlu}}}
}\end{relation-sémantique}
\begin{relation-sémantique}\confer{
\hyperlink{Ⓔʁlɯβʁlɯβ}{\textit{ \papi{ʁlɯβʁlɯβ}}}
}\end{relation-sémantique}\end{entrée}

\begin{entrée}
\vedette{\hypertarget{Ⓔaχɕɯβ}{\papi{ aχɕɯβ}}}\markboth{aχɕɯβ}{}
\classe{vi}
\paradigme{\textit{dir :} \jya tɤ-}
\begin{définition}\fra être debout ou allongé l'un à côté de l'autre\end{définition}
\begin{définition}\cmn 一起躺着;一起站着
\begin{déclaration}\use{坐着不可以说\stylefv{aχɕɯβ}}\end{déclaration}
\begin{déclaration} \étymologie{\papi{gɕib}}\end{déclaration}\end{définition}
\begin{exemple}\jya stukɤr ɲɯ-ɤχɕɯβ\cmn 梁是双根的\end{exemple}
\begin{exemple}\jya tɯmbri ɲɯ-ɤχɕɯβ\cmn 绳子是两股的\end{exemple}
\begin{exemple}\jya tɕelo ɲɯ-ɤχɕɯβ-ndʑi tɕe ɲɯ-ndzur-ndʑi\cmn 他们俩一起在那里站着\end{exemple}
\begin{relation-sémantique}\confer{
\hyperlink{Ⓔsaχɕɯβ}{\textit{ \papi{saχɕɯβ}}}
}\end{relation-sémantique}\end{entrée}

\begin{entrée}
\vedette{\hypertarget{Ⓔaχɕɯldɤn}{\papi{ aχɕɯldɤn}}}\markboth{aχɕɯldɤn}{}\classe{vs}
\begin{définition}\fra en sécurité\end{définition}
\begin{définition}\cmn 安全;安康\end{définition}
\begin{exemple}\jya ku-oχɕɯldɤn-i\cmn 我们都平安无事\end{exemple}
\begin{relation-sémantique}\confer{
\hyperlink{Ⓔχɕɯldɤn}{\textit{ \papi{χɕɯldɤn}}}
}\end{relation-sémantique}\end{entrée}

\begin{entrée}
\vedette{\hypertarget{Ⓔaχom}{\papi{ aχom}}}\markboth{aχom}{}
\classe{vi}
\paradigme{\textit{dir :} \jya tɤ-}
\begin{définition}\fra bailler\end{définition}
\begin{définition}\cmn 打哈欠\end{définition}
\begin{exemple}\jya a-ʑɯβ ɲɯ-ɣi tɕe, tɤ-aχom-a\cmn 我困了,所以打了哈欠\end{exemple}
\begin{exemple}\jya khɯna ɲɯ-ɤχom\cmn 狗在打哈欠\end{exemple}
\begin{exemple}\jya a, ɲɯ-ɤχom-a\cmn 啊,我在打哈欠\end{exemple}\end{entrée}

\begin{entrée}
\vedette{\hypertarget{Ⓔaχpɯχpjɤt}{\papi{ aχpɯχpjɤt}}}\markboth{aχpɯχpjɤt}{}
\begin{relation-sémantique}\confer{
\hyperlink{Ⓔχpjɤt}{\textit{ \papi{χpjɤt}}}
}\end{relation-sémantique}\end{entrée}

\begin{entrée}
\vedette{\hypertarget{Ⓔaχsi}{\papi{ aχsi}}}\markboth{aχsi}{}
\classe{vs}
\paradigme{\textit{dir :} \jya tɤ-}
\begin{définition}\fra propre\end{définition}
\begin{définition}\cmn 干净;没有剩余\end{définition}
\begin{exemple}\jya tɕhaʁla ra to-raʁrɯz-nɯ tɕe to-k-ɤχsi-ci\cmn 他们扫了院子,现在院子显得很干净了\end{exemple}
\begin{relation-sémantique}\confer{
\hyperlink{Ⓔsaχsi}{\textit{ \papi{saχsi}}}
}\end{relation-sémantique}\end{entrée}

\begin{entrée}
\vedette{\hypertarget{Ⓔaχsom}{\papi{ aχsom}}}\markboth{aχsom}{}
\classe{vi}
\paradigme{\textit{dir :} \jya nɯ-}
\begin{définition}\fra être réveillé\end{définition}
\begin{définition}\cmn 清醒\end{définition}
\begin{exemple}\jya nɯ-aχsom-a\cmn 我醒了\end{exemple}
\begin{exemple}\jya ɲɤ-k-ɤχsom-ci\cmn 他醒了\end{exemple}
\begin{exemple}\jya jiɕqha pɯ-nɯʑɯβa ri, tham tɕe nɯ-aχsom-a\cmn 我刚才睡着了,现在就醒了\end{exemple}
\begin{exemple}\jya nɤki tɤ-pɤtso nɯ kɯ-ɤχsom ci ɲɯ-ŋu\cmn 这个小孩子看起来很聪明\end{exemple}\begin{sous-entrée}
\vedette{\hypertarget{}{\papi{ saχsom}}}\markboth{saχsom}{}\classe{vt}
\paradigme{\textit{dir :} \jya nɯ-}
\begin{définition}\ 
\begin{déclaration}\grammar{caus}\end{déclaration}\end{définition}
\begin{définition}\fra réveiller\end{définition}
\begin{définition}\cmn 弄醒\end{définition}
\begin{exemple}\jya aʑo nɯ-saχsom-a\cmn 我把他弄醒\end{exemple}
\end{sous-entrée}\end{entrée}

\begin{entrée}
\vedette{\hypertarget{Ⓔaχsɯko}{\papi{ aχsɯko}}}\markboth{aχsɯko}{}\classe{vs}
\paradigme{\textit{dir :} \jya tɤ-}
\begin{définition}\fra propre\end{définition}
\begin{définition}\cmn 干净\end{définition}
\begin{exemple}\jya jiɕqha ra nɯ-kha ra wuma ɲɯ-ɤχsɯko\cmn 他们的家很整洁\end{exemple}
\begin{relation-sémantique}\confer{
\hyperlink{Ⓔaχsi}{\textit{ \papi{aχsi}}}
}\end{relation-sémantique}
\begin{sous-entrée}
\vedette{\hypertarget{}{\papi{ saχsɯko}}}\markboth{saχsɯko}{}\classe{vt}
\paradigme{\textit{dir :} \jya tɤ-}
\begin{définition}\fra rendre propre\end{définition}
\begin{définition}\cmn 弄干净\end{définition}
\begin{exemple}\jya kha ra tú-wɣ-raʁrɯz tɕe tú-wɣ-saχsɯko ɲɯ-ra\cmn 要把家里扫干净\end{exemple}
\begin{relation-sémantique}\confer{
\hyperlink{Ⓔsaχsi}{\textit{ \papi{saχsi}}}
}\end{relation-sémantique}
\end{sous-entrée}\end{entrée}

\begin{entrée}
\vedette{\hypertarget{Ⓔaχtɕɤz}{\papi{ aχtɕɤz}}}\markboth{aχtɕɤz}{}\classe{n}
\begin{définition}\fra terme affectueux pour les enfants\end{définition}
\begin{définition}\cmn 对下一代的爱称
\begin{déclaration} \étymologie{\papi{gtɕes}}\end{déclaration}\end{définition}\end{entrée}

\begin{entrée}
\vedette{\hypertarget{Ⓔazbraʁ}{\papi{ azbraʁ}}}\markboth{azbraʁ}{}
\begin{relation-sémantique}\confer{
\hyperlink{Ⓔzbraʁ}{\textit{ \papi{zbraʁ}}}
}\end{relation-sémantique}\end{entrée}

\begin{entrée}
\vedette{\hypertarget{Ⓔazdaʁ}{\papi{ azdaʁ}}}\markboth{azdaʁ}{}
\classe{vi}
\paradigme{\textit{dir :} \jya tɤ-}
\begin{définition}\fra avoir deux couches\end{définition}
\begin{définition}\cmn 有双层\end{définition}
\begin{exemple}\jya jɤlwa ɲɯ-ɤzdaʁ\cmn 有双层窗帘\end{exemple}
\begin{exemple}\jya khɯɣɲɟɯ kɯ-ɤzdaʁ ɣɤʑu\cmn 有双层窗子\end{exemple}\begin{sous-entrée}
\vedette{\hypertarget{}{\papi{ sɤzdaʁ}}}\markboth{sɤzdaʁ}{}\classe{vt}
\paradigme{\textit{dir :} \jya tɤ-}
\begin{définition}\fra mettre deux couches\end{définition}
\begin{définition}\cmn 做成双层的\end{définition}
\begin{exemple}\jya a-ŋga tɤ-sɤzdaʁ-a tɕe mɤ-nɤndʐo-a\cmn 我穿了两层衣服,不觉得冷\end{exemple}
\end{sous-entrée}\end{entrée}

\begin{entrée}
\vedette{\hypertarget{Ⓔazgroʁ}{\papi{ azgroʁ}}}\markboth{azgroʁ}{}
\begin{relation-sémantique}\confer{
\hyperlink{ⒺzgroʁⒽ1}{\textit{ \papi{zgroʁ1}}}
}\end{relation-sémantique}\end{entrée}

\begin{entrée}
\vedette{\hypertarget{Ⓔazgrɯ}{\papi{ azgrɯ}}}\markboth{azgrɯ}{}
\classe{vi}
\paradigme{\textit{dir :} \jya tɤ-}
\begin{définition}\fra courber le dos\end{définition}
\begin{définition}\cmn 弯下腰;鞠躬\end{définition}
\begin{exemple}\jya jiɕqha tɯrme nɯ ɲɯ-ɤzgrɯ\cmn 那个人弯着腰\end{exemple}
\begin{exemple}\jya to-k-ɤzgrɯ-ci\cmn 他弯下了腰\end{exemple}\end{entrée}

\begin{entrée}
\vedette{\hypertarget{Ⓔazgɯr}{\papi{ azgɯr}}}\markboth{azgɯr}{}\classe{vi}
\paradigme{\textit{dir :} \jya tɤ-}
\begin{définition}\fra être recroquevillé\end{définition}
\begin{définition}\cmn 驼背;全身蜷缩\end{définition}
\begin{exemple}\jya ɯʑo to-ngo tɕe, ɲɯ-ɤzgɯr ʑo\cmn 因为他病了,所以全身蜷缩\end{exemple}
\begin{exemple}\jya a-mgɯr tɤ-mŋɤm tɕe tu-ostu-a ŋu, mɯ-tɤ-mŋɤm tɕe pjɯ-ɤzgɯr-a ŋu\cmn 我背痛的时候就坐直,不痛的时候就弯腰\end{exemple}
\begin{exemple}\jya to-k-ɤzgɯr-ci\cmn 他背驼了\end{exemple}
\begin{relation-sémantique}\confer{
\hyperlink{Ⓔazgrɯ}{\textit{ \papi{azgrɯ}}}
}\end{relation-sémantique}
\begin{relation-sémantique}\confer{
\hyperlink{Ⓔazgɯrloʁ}{\textit{ \papi{azgɯrloʁ}}}
}\end{relation-sémantique}
\begin{relation-sémantique}\confer{
\hyperlink{Ⓔɯ-zgɯr}{\textit{ \papi{ɯ-zgɯr}}}
}\end{relation-sémantique}\end{entrée}

\begin{entrée}
\vedette{\hypertarget{Ⓔazgɯrloʁ}{\papi{ azgɯrloʁ}}}\markboth{azgɯrloʁ}{}
\classe{vi}
\paradigme{\textit{dir :} \jya pɯ-}
\begin{définition}\fra se recroqueviller\end{définition}
\begin{définition}\cmn 蜷缩
\begin{déclaration} \étymologie{\papi{sgur}}\end{déclaration}\end{définition}
\begin{exemple}\jya pjɤ-k-ɤzgɯrloʁ-ci\cmn 他缩成了一团\end{exemple}
\begin{relation-sémantique}\confer{
\hyperlink{Ⓔazgɯr}{\textit{ \papi{azgɯr}}}
}\end{relation-sémantique}\end{entrée}

\begin{entrée}
\vedette{\hypertarget{Ⓔazɣɤʁrɯʁre}{\papi{ azɣɤʁrɯʁre}}}\markboth{azɣɤʁrɯʁre}{}
\begin{relation-sémantique}\confer{
\hyperlink{Ⓔɣɤʁre}{\textit{ \papi{ɣɤʁre}}}
}\end{relation-sémantique}\end{entrée}

\begin{entrée}
\vedette{\hypertarget{Ⓔazmɯjqhɯjqha}{\papi{ azmɯjqhɯjqha}}}\markboth{azmɯjqhɯjqha}{}
\classe{vi}
\begin{définition}\fra se faire du mal les uns aux autres\end{définition}
\begin{définition}\cmn 互相伤害\end{définition}
\begin{exemple}\jya ʑɤni to-k-ɤzmɯjqhɯjqha-ndʑi-ci\cmn 他们俩互相伤害了对方\end{exemple}
\begin{relation-sémantique}\confer{
\hyperlink{Ⓔqha}{\textit{ \papi{qha}}}
}\end{relation-sémantique}\end{entrée}

\begin{entrée}
\vedette{\hypertarget{ⒺaʑaʁⒽ1}{\papi{ aʑaʁ}}}\markboth{aʑaʁ}{}\homonyme{1}\classe{vi}
\paradigme{\textit{dir :} \jya pɯ-}
\begin{définition}\fra couler\end{définition}
\begin{définition}\cmn 漏水\end{définition}
\begin{exemple}\jya tɯthɯ ɲɯ-ɤʑaʁ\cmn 锅子在漏水\end{exemple}
\begin{exemple}\jya pjɤ-k-ɤʑaʁ-ci\cmn 漏水了\end{exemple}
\begin{relation-sémantique}\confer{
\hyperlink{Ⓔari}{\textit{ \papi{ari}}}
}\end{relation-sémantique}\end{entrée}

\begin{entrée}
\vedette{\hypertarget{Ⓔaʑɤwu}{\papi{ aʑɤwu}}}\markboth{aʑɤwu}{}\classe{vs}
\paradigme{\textit{dir :} \jya nɯ-}
\begin{définition}\ 
\begin{déclaration}\grammar{denom}\end{déclaration}\end{définition}
\begin{définition}\fra être boiteux\end{définition}
\begin{définition}\cmn 跛脚\end{définition}
\begin{exemple}\jya ɯ-mi ɲɯ-ɤʑɤwu\cmn 他跛脚\end{exemple}
\begin{exemple}\jya kɯ-ɤʑɤwu ʑo jo-nɯɕe\cmn 他跛着脚地回家了\end{exemple}
\begin{relation-sémantique}\synonyme{
\hyperlink{Ⓔaɕkala}{\textit{ \papi{aɕkala}}}
}\end{relation-sémantique}
\begin{relation-sémantique}\confer{
\hyperlink{Ⓔʑɤwu}{\textit{ \papi{ʑɤwu}}}
}\end{relation-sémantique}\end{entrée}

\begin{entrée}
\vedette{\hypertarget{Ⓔaʑɴɢɯʑɴɢoʁ}{\papi{ aʑɴɢɯʑɴɢoʁ}}}\markboth{aʑɴɢɯʑɴɢoʁ}{}
\begin{relation-sémantique}\confer{
\hyperlink{Ⓔʑɴɢoʁ}{\textit{ \papi{ʑɴɢoʁ}}}
}\end{relation-sémantique}\end{entrée}

\begin{entrée}
\vedette{\hypertarget{Ⓔaʑo}{\papi{ aʑo}}}\markboth{aʑo}{}\classe{pro}
\begin{définition}\fra moi\end{définition}
\begin{définition}\cmn 我\end{définition}
\begin{exemple}\jya aʑo maʁ-a!\cmn 不是我!\end{exemple}
\begin{relation-sémantique}\confer{
\hyperlink{Ⓔaj}{\textit{ \papi{aj}}}
}\end{relation-sémantique}
\end{entrée}

\begin{entrée}
\vedette{\hypertarget{Ⓔaʑɯrja}{\papi{ aʑɯrja}}}\markboth{aʑɯrja}{}\classe{vi}
\paradigme{\textit{dir :} \jya nɯ-}
\paradigme{\textit{dir :} \jya thɯ-}
\paradigme{\textit{dir :} \jya \_}
\begin{définition}\fra faire la queue\end{définition}
\begin{définition}\cmn 排队\end{définition}
\begin{exemple}\jya tɯrme ɲɯ-dɤn tɕe ɲɯ-ɤʑɯrja-nɯ ʑo\cmn 人很多,在排队\end{exemple}
\begin{exemple}\jya tɤjmɤɣ ɲɯ-ɤʑɯrja\cmn 菌子长得很整齐\end{exemple}
\begin{exemple}\jya thɯ-aʑɯrja-j\cmn 我们排队了\end{exemple}
\begin{exemple}\jya znde ɯ-taʁ zɯ tɯrme ra ɲɯ-ɤkɤʑɯrja-nɯ-ci\cmn 人们背靠着墙排起了队\end{exemple}\begin{sous-entrée}
\vedette{\hypertarget{}{\papi{ sɤʑɯrja}}}\markboth{sɤʑɯrja}{}\classe{vt}
\paradigme{\textit{dir :} \jya nɯ-}
\paradigme{\textit{dir :} \jya \_}
\begin{définition}\ 
\begin{déclaration}\grammar{caus}\end{déclaration}\end{définition}
\begin{définition}\fra aligner\end{définition}
\begin{définition}\cmn 排列整齐\end{définition}
\begin{exemple}\jya kha ɲɤ-sɤʑɯrja-nɯ ʑo\cmn 他们把房子排列整齐了\end{exemple}
\end{sous-entrée}\end{entrée}

\begin{entrée}
\vedette{\hypertarget{Ⓔaʑɯχtso}{\papi{ aʑɯχtso}}}\markboth{aʑɯχtso}{}\classe{vs}
\paradigme{\textit{dir :} \jya tɤ-}
\begin{définition}\fra propre, bien entretenu\end{définition}
\begin{définition}\cmn 干净,整洁(有人弄干净)\end{définition}
\begin{exemple}\jya tɯ-ci ɲɯ-ɤʑɯχtso\cmn 水很干净\end{exemple}
\begin{exemple}\jya kɤndza ɲɯ-ɤʑɯχtso\cmn 食物很干净\end{exemple}
\begin{exemple}\jya mɯ́j-tɯ-ɤʑɯχtso\cmn 你不干净\end{exemple}
\begin{relation-sémantique}\confer{
\hyperlink{Ⓔχtso}{\textit{ \papi{χtso}}}
}\end{relation-sémantique}\begin{sous-entrée}
\vedette{\hypertarget{}{\papi{ sɤʑɯχtso}}}\markboth{sɤʑɯχtso}{}\classe{vt}
\paradigme{\textit{dir :} \jya tɤ-}
\begin{définition}\ 
\begin{déclaration}\grammar{caus}\end{déclaration}\end{définition}
\begin{définition}\fra rendre propre\end{définition}
\begin{définition}\cmn 弄干净\end{définition}
\begin{exemple}\jya tɤ-sɤʑɯχtso-t-a\cmn 我弄干净了\end{exemple}
\end{sous-entrée}\end{entrée}

\begin{entrée}
\vedette{\hypertarget{Ⓔaʑɯʑu}{\papi{ aʑɯʑu}}}\markboth{aʑɯʑu}{}
\classe{vi}
\paradigme{\textit{dir :} \jya tɤ-}
\begin{définition}\fra faire de la lutte\end{définition}
\begin{définition}\cmn 摔跤;角力\end{définition}
\begin{exemple}\jya tɤ-aʑɯʑu-ndʑi\cmn 他们俩一起角力了\end{exemple}\begin{sous-entrée}
\vedette{\hypertarget{}{\papi{ nɤʑɯʑu}}}\markboth{nɤʑɯʑu}{}\classe{vt}
\paradigme{\textit{dir :} \jya tɤ-}
\begin{définition}\ 
\begin{déclaration}\grammar{appl}\end{déclaration}\end{définition}
\begin{définition}\fra lutter avec\end{définition}
\begin{définition}\cmn 跟……一起角力\end{définition}
\begin{exemple}\jya tɤ́-wɣ-nɤʑɯʑu-a\cmn 他跟我角力了\end{exemple}
\end{sous-entrée}\end{entrée}

\newpage\caractère{b}

\begin{entrée}
\vedette{\hypertarget{Ⓔbabɯ}{\papi{ babɯ}}}\markboth{babɯ}{}
\classe{n}
\begin{définition}\fra cassis\end{définition}
\begin{définition}\cmn 黑茶藨子\end{définition}
\begin{exemple}\jya babɯ nɯ si kɯ-mbɤr tsa ci ŋu. tɯrme ɯ-taʁ kɯ-xtɕɯ-xtɕi ma tu-mbro mɤ-cha. ɯ-ru nɯ kɯ-pɣi tɕe, ɯ-rqhu pjɯ-kɯ-ɴɢaʁ kɯ-fse kɯ-tu ŋu, tɯ-xpa ɲɯ-rɯmɯntoʁ tɕe, nɯɕɯmɯma ɯ-mat chɯ-βze ŋu, ɯ-mat nɯ thɯ-tɯt tɕe ɲaʁ, kɤ-ndza mɯm, chi. ɯ-mat ɯ-ŋgɯ ɯ-rɣi kɯ-ndɯβ kɯ-dɤn tsa tɕe kɯ-mpɯ ŋu. ɯ-jwaʁ ndɯβ cho dɤn. zgoku ɯ-taʁ ɯ-pa ʑo tu-ɬoʁ cha.\cmn 
\stylefv{babɯ} 是矮小的树种,比人高不出多少。树干是灰色的,有好像快要脱落的树皮。当年开花,马上结果,果实成熟后是黑色的,好吃,很甜。果实里有又小又多的种子,是很嫩的,叶子小而多。山上山下都能生长。
\end{exemple}\end{entrée}

\begin{entrée}
\vedette{\hypertarget{Ⓔbɤbɤβ}{\papi{ bɤbɤβ}}}\markboth{bɤbɤβ}{}\classe{idph.2}\acception{1}
\begin{définition}\fra obstiné\end{définition}
\begin{définition}\cmn 形容固执的样子\end{définition}
\begin{exemple}\jya jiɕqha tɯrme nɯ, ɯ-phe ti mɤ-ti maŋe, bɤbɤβ ʑo ku-nɯ-rɤʑi ɕti\cmn 那个人,给他说是没有用的,他什么都听不进去\end{exemple}\acception{2}
\begin{définition}\fra épais, lourd et peu pratique, poussant en touffe (champignons)\end{définition}
\begin{définition}\cmn 形容物体笨重或者菌子长在一块的样子\end{définition}
\begin{exemple}\jya tɤjmɤɣ ɲɯ-xcat ʑo bɤbɤβ ʑo ɲɯ-pa\cmn 菌子很多,都长在一块\end{exemple}
\begin{exemple}\jya sɯmat ɲɯ-ɲaʁ ʑo bɤbɤβ ʑo ɲɯ-pa\cmn 黑果子都长在一块\end{exemple}\begin{sous-entrée}
\vedette{\hypertarget{}{\papi{ bɤβ}}}\markboth{bɤβ}{}\classe{idph.1}
\begin{définition}\fra bruit d'un objet lourd qui tombe de haut\end{définition}
\begin{définition}\cmn 重物从高处掉下来的声音\end{définition}
\end{sous-entrée}\begin{sous-entrée}
\vedette{\hypertarget{}{\papi{ bɤβnɤbɤβ}}}\markboth{bɤβnɤbɤβ}{}\classe{idph.3}
\begin{exemple}\jya tɤjpa bɤβnɤbɤβ ʑo pa-βde\cmn 他把雪(从房背)一块一块扔下来了\end{exemple}
\begin{exemple}\jya bɤβnɤbɤβ ʑo ɲo-nɯjʁo\cmn 他很没有分寸地骂了他\end{exemple}
\end{sous-entrée}\begin{sous-entrée}
\vedette{\hypertarget{}{\papi{ ɣɤbɤbɤβ}}}\markboth{ɣɤbɤbɤβ}{}\classe{vs}
\begin{définition}\fra être bruyant\end{définition}
\begin{définition}\cmn 发出很响的声音\end{définition}
\begin{exemple}\jya qale ɯ-tɯ-wxti kɯ ɲɯ-ɣɤbɤbɤβ ʑo\cmn 风发出很响的声音\end{exemple}
\begin{exemple}\jya tɯrme ʁnɯz ɲɯ-rɯɕmi-ndʑi tɕe, ɲɯ-ɣɤbɤbɤβ-ndʑi\cmn 两个人在说话,很吵,听不清楚他们在讲什么\end{exemple}
\end{sous-entrée}\begin{sous-entrée}
\vedette{\hypertarget{}{\papi{ nɯbɤβ}}}\markboth{nɯbɤβ}{}\classe{vt}
\paradigme{\textit{dir :} \jya pɯ-}
\begin{définition}\ 
\begin{déclaration}\grammar{deidph}\end{déclaration}\end{définition}
\begin{définition}\fra jeter de toutes ses forces\end{définition}
\begin{définition}\cmn 不顾一切地往下扔(重的东西)\end{définition}
\begin{exemple}\jya tɤjpa khoxtu ɕ-tha-βde tɕe, pa-nɯbɤβ ʑo pa-βde\cmn 他把雪从房背上不顾一切地扔下去了\end{exemple}
\begin{exemple}\jya rdɤstaʁ pa-nɯbɤβ ʑo pa-βde\cmn 他把石头不顾一切地扔下去了\end{exemple}
\end{sous-entrée}\end{entrée}

\begin{entrée}
\vedette{\hypertarget{Ⓔbjɯbjɯɣ}{\papi{ bjɯbjɯɣ}}}\markboth{bjɯbjɯɣ}{}\classe{idph.2}
\begin{définition}\fra qui pend en grand nombre, mou\end{définition}
\begin{définition}\cmn 形容多而柔软,向下垂吊的样子\end{définition}
\begin{exemple}\jya jiʑo ji-kha ɯ-ʁɤri ʑmbri tɯ-phɯ tu tɕe, ftɕar tɕe ɯ-jwaʁ ɲɯ-dɤn, ɯ-rtaʁ ɲɯ-mpɯ ɲɯ-dɤn tɕe, bjɯbjɯɣ ʑo pjɯ-ɴqoʁ ɲɯ-ŋu\cmn 他们家前面有一棵柳树,一到春天,树叶茂密,树枝又软又多地吊在那儿。\end{exemple}
\begin{relation-sémantique}\synonyme{
\hyperlink{Ⓔlbjɯlbjɯɣ}{\textit{ \papi{lbjɯlbjɯɣ}}}
}\end{relation-sémantique}\end{entrée}

\begin{entrée}
\vedette{\hypertarget{Ⓔboŋboŋ}{\papi{ boŋboŋ}}}\markboth{boŋboŋ}{}\classe{idph.2}
\begin{définition}\fra qui a la forme d'un œuf\end{définition}
\begin{définition}\cmn 形容鸡蛋的形状;椭圆形\end{définition}\end{entrée}

\begin{entrée}
\vedette{\hypertarget{Ⓔboʁ}{\papi{ boʁ}}}\markboth{boʁ}{}\classe{idph.1}
\begin{définition}\fra d'un seul coup tous ensemble\end{définition}
\begin{définition}\cmn 一下子全部\end{définition}
\begin{exemple}\jya smi ɯ-taʁ tɯ-ci pjɯ́-wɣ-lɤt tɕe boʁ pjɯ-mi ɕti\cmn 火上加了水就会一下子灭掉\end{exemple}\begin{sous-entrée}
\vedette{\hypertarget{}{\papi{ boʁboʁ}}}\markboth{boʁboʁ}{}\classe{idph.2}
\begin{définition}\fra en ordre\end{définition}
\begin{définition}\cmn 形容(收捡得)很整齐的样子\end{définition}
\begin{exemple}\jya nɤ-ŋga boʁboʁ tɤ-ste\cmn 你把衣服收捡得整齐一点\end{exemple}
\end{sous-entrée}\begin{sous-entrée}
\vedette{\hypertarget{}{\papi{ boʁnɤboʁ}}}\markboth{boʁnɤboʁ}{}\classe{idph.3}
\begin{exemple}\jya tɤ-pɤtso ra boʁboʁ nɤ boʁboʁ jɤ-ari-nɯ\cmn 小孩子一下子一起去了\end{exemple}
\begin{relation-sémantique}\confer{
\hyperlink{Ⓔnɤboʁboʁ}{\textit{ \papi{nɤboʁboʁ}}}
}\end{relation-sémantique}
\begin{relation-sémantique}\confer{
\hyperlink{Ⓔaboʁboʁ}{\textit{ \papi{aboʁboʁ}}}
}\end{relation-sémantique}
\end{sous-entrée}\end{entrée}

\begin{entrée}
\vedette{\hypertarget{Ⓔbuqa}{\papi{ buqa}}}\markboth{buqa}{}
\classe{n}
\begin{définition}\fra mycose du pied\end{définition}
\begin{définition}\cmn 脚癣\end{définition}\end{entrée}

\begin{entrée}
\vedette{\hypertarget{Ⓔbrɤβbrɤβ}{\papi{ brɤβbrɤβ}}}\markboth{brɤβbrɤβ}{} (\variante{brɤbrɤβ}) \classe{idph.2}
\begin{définition}\fra rugueux et irrégulier (pierre)\end{définition}
\begin{définition}\cmn 形容石子等物粗糙的样子\end{définition}
\begin{exemple}\jya rdɤstaʁ ɲɯ-dɤn, kɤntɕhaʁ brɤbrɤβ ʑo ɲɯ-pa\cmn 街上小石子多,坎坷不平\end{exemple}
\begin{exemple}\jya stoʁ ɯ-tɯ-jndʐɤz kɯ brɤbrɤβ ʑo ɲɯ-pa\cmn 胡豆的颗粒又大又多\end{exemple}\begin{sous-entrée}
\vedette{\hypertarget{}{\papi{ brɤβnɤbrɤβ}}}\markboth{brɤβnɤbrɤβ}{}\classe{idph.3}
\begin{exemple}\jya ʁmaʁmi ra brɤβnɤbrɤβ ʑo ɲɯ-nɤŋkɯŋke-nɯ\cmn 士兵们身材高大,雄赳赳地路过了这里\end{exemple}
\end{sous-entrée}\begin{sous-entrée}
\vedette{\hypertarget{}{\papi{ brɤβnɤlɤβ}}}\markboth{brɤβnɤlɤβ}{}\classe{idph.4}
\begin{exemple}\jya brɤβnɤlɤβ ʑo ɲɯ-ʑɣɤstu-nɯ\cmn 动作和声音都很大,没有规律地运动着\end{exemple}
\begin{relation-sémantique}\confer{
\hyperlink{Ⓔbrɯzbrɯz}{\textit{ \papi{brɯzbrɯz}}}
}\end{relation-sémantique}
\begin{relation-sémantique}\confer{
\hyperlink{Ⓔbrɯɣbrɯɣ}{\textit{ \papi{brɯɣbrɯɣ}}}
}\end{relation-sémantique}
\end{sous-entrée}\end{entrée}

\begin{entrée}
\vedette{\hypertarget{Ⓔbrɯɣbrɯɣ}{\papi{ brɯɣbrɯɣ}}}\markboth{brɯɣbrɯɣ}{}\classe{idph.2}
\begin{définition}\fra couvert de petits boutons\end{définition}
\begin{définition}\cmn 粗糙,长满了小点点\end{définition}
\end{entrée}

\begin{entrée}
\vedette{\hypertarget{Ⓔbrɯzbrɯz}{\papi{ brɯzbrɯz}}}\markboth{brɯzbrɯz}{}
\classe{idph.2}
\begin{définition}\fra couvert de petits boutons\end{définition}
\begin{définition}\cmn 粗糙,长满了小点点\end{définition}
\begin{exemple}\jya a-βri tɤ-ndɤr brɯzbrɯz ʑo ɲɤ-ɬoʁ\cmn 我身上长满了痘痘\end{exemple}
\begin{relation-sémantique}\confer{
\hyperlink{Ⓔbrɯɣbrɯɣ}{\textit{ \papi{brɯɣbrɯɣ}}}
}\end{relation-sémantique}
\begin{relation-sémantique}\confer{
\hyperlink{Ⓔbrɤβbrɤβ}{\textit{ \papi{brɤβbrɤβ}}}
}\end{relation-sémantique}\end{entrée}

\begin{entrée}
\vedette{\hypertarget{Ⓔbɯɣ}{\papi{ bɯɣ}}}\markboth{bɯɣ}{}\classe{vi}
\paradigme{\textit{dir :} \jya tɤ-}
\paradigme{\textit{dir :} \jya pɯ-}
\begin{définition}\fra avoir le mal du pays\end{définition}
\begin{définition}\cmn 思念家乡\end{définition}
\begin{exemple}\jya ɲɯ-bɯɣ-a\cmn 我想家\end{exemple}
\begin{exemple}\jya pjɤ-bɯɣ\cmn 他以前很想家\end{exemple}
\begin{exemple}\jya tɯrme sɤtɕha jɤ-kɯ-ɤri tɕe, tu-kɯ-bɯɣ ɕti\cmn 去了其他地方,肯定会想家\end{exemple}
\begin{relation-sémantique}\confer{
\hyperlink{Ⓔnɯɣbɯɣ}{\textit{ \papi{nɯɣbɯɣ}}}
}\end{relation-sémantique}\end{entrée}

\begin{entrée}
\vedette{\hypertarget{Ⓔbɯɣbɯɣ}{\papi{ bɯɣbɯɣ}}}\markboth{bɯɣbɯɣ}{}\classe{idph.2}
\begin{définition}\fra concentré\end{définition}
\begin{définition}\cmn 形容草木等集中,茂盛的样子\end{définition}
\begin{exemple}\jya tɯrme kɯ-nɤmɲo jo-dɤn tɕe bɯɣbɯɣ ʑo ɲɯ-pa\end{exemple}
\begin{exemple}\jya bɯɣbɯɣ ʑo jo-ɣi-nɯ tɕe ɲɯ-nɤmɲo-nɯ\cmn 人从四面八方赶来观看\end{exemple}\begin{sous-entrée}
\vedette{\hypertarget{}{\papi{ bɯɣnɤbɯɣ}}}\markboth{bɯɣnɤbɯɣ}{}\classe{idph.3}
\end{sous-entrée}\end{entrée}

\begin{entrée}
\vedette{\hypertarget{Ⓔbɯlɯbali}{\papi{ bɯlɯbali}}}\markboth{bɯlɯbali}{}
\classe{n}
\begin{définition}\fra personne qui n'en fait qu'à sa tête\end{définition}
\begin{définition}\cmn 爱一意孤行的人\end{définition}
\begin{exemple}\jya ɯʑo na-nɯ-ʁjit nɯ tu-nɯ-ste ɲɯ-ɕti, tɯrme bɯlɯbali ci ɲɯ-ŋu\cmn 他想做什么就做什么,不管别人的意见\end{exemple}\end{entrée}

\begin{entrée}
\vedette{\hypertarget{Ⓔbɯwa}{\papi{ bɯwa}}}\markboth{bɯwa}{}
\classe{vt}
\paradigme{\textit{dir :} \jya tɤ-}
\begin{définition}\fra porter un enfant sur le dos\end{définition}
\begin{définition}\cmn 背孩子\end{définition}
\begin{exemple}\jya tɤ-bɯwa-t-a, tɤ-tɯ-bɯwa-t, ta-bɯwa\cmn 我背了他,你背了他,他背了他\end{exemple}
\begin{exemple}\jya kɯki kɤ-ŋke mɯ́j-cha tɕɤn, tɤ-bɯwe\cmn 他不能走,你背他吧\end{exemple}
\begin{exemple}\jya tɤ-bɯwe ɲɯ-ntshi\cmn 只好背了他\end{exemple}
\begin{relation-sémantique}\confer{
\hyperlink{Ⓔzbɯwa}{\textit{ \papi{zbɯwa}}}
}\end{relation-sémantique}\end{entrée}

\newpage\caractère{β}

\begin{entrée}
\vedette{\hypertarget{Ⓔβdaʁ,βzu}{\papi{ βdaʁ,βzu}}}\markboth{βdaʁ,βzu}{}\begin{définition}\fra s'occuper de ses tâches\end{définition}
\begin{définition}\cmn 管(任务、职责),一般前加否定前缀\end{définition}
\begin{exemple}\jya βdaʁ maka tɤ-βzu-t-a me\cmn 我没有管好、没有理他\end{exemple}
\begin{relation-sémantique}\ComponentA{\classe{n}
 \papi{βdaʁ}
}\end{relation-sémantique}
\begin{relation-sémantique}\ComponentB{\classe{vt}
\hyperlink{ⒺβzuⒽ1}{\textit{ \papi{βzu}}}
}\end{relation-sémantique}
\begin{relation-sémantique}\confer{
\hyperlink{Ⓔnɯβdaʁ}{\textit{ \papi{nɯβdaʁ}}}
}\end{relation-sémantique}
\begin{relation-sémantique}\confer{
\hyperlink{ⒺβzuⒽ1}{\textit{ \papi{βzu1}}}
}\end{relation-sémantique}\end{entrée}

\begin{entrée}
\vedette{\hypertarget{ⒺβdaχpuⒽ1}{\papi{ βdaχpu}}}\markboth{βdaχpu}{}\homonyme{1}
\classe{n}
\begin{définition}\fra hôte, maître de maison\end{définition}
\begin{définition}\cmn 主人
\begin{déclaration} \étymologie{\papi{bdag.po}}\end{déclaration}\end{définition}
\begin{relation-sémantique}\confer{
\hyperlink{Ⓔnɯβdaχpu}{\textit{ \papi{nɯβdaχpu}}}
}\end{relation-sémantique}\end{entrée}

\begin{entrée}
\vedette{\hypertarget{ⒺβdaχpuⒽ2}{\papi{ βdaχpu}}}\markboth{βdaχpu}{}\homonyme{2}
\classe{adv}
\begin{définition}\fra en ce qui concerne ...\end{définition}
\begin{définition}\cmn 至于……\end{définition}
\begin{exemple}\jya ɯʑo βdaχpu nɯ, ɯʑo a-tɤ-naχpjɤt\cmn 至于他,(来不来)由他来决定\end{exemple}\end{entrée}

\begin{entrée}
\vedette{\hypertarget{Ⓔβdaχtɕɤl}{\papi{ βdaχtɕɤl}}}\markboth{βdaχtɕɤl}{}\classe{n}
\begin{définition}\fra marche en pierre\end{définition}
\begin{définition}\cmn 石制台阶\end{définition}\end{entrée}

\begin{entrée}
\vedette{\hypertarget{Ⓔβdɤmŋaʁ}{\papi{ βdɤmŋaʁ}}}\markboth{βdɤmŋaʁ}{}\classe{n}
\begin{définition}\fra méthode (pour tenir tête à qqn)\end{définition}
\begin{définition}\cmn (对付别人的)办法、计策
\begin{déclaration} \étymologie{\papi{bdams.mŋags}}\end{déclaration}\end{définition}\end{entrée}

\begin{entrée}
\vedette{\hypertarget{Ⓔβde}{\papi{ βde}}}\markboth{βde}{}\classe{vt}\acception{1}
\paradigme{\textit{dir :} \jya thɯ-}
\paradigme{\textit{dir :} \jya \_}
\begin{définition}\fra jeter\end{définition}
\begin{définition}\cmn 扔\end{définition}
\begin{exemple}\jya cho-βde, tha-βde\cmn 他扔了\end{exemple}
\begin{exemple}\jya laχtɕha mɤ-kɯ-ra nɯra cho-βde\cmn 他扔了不必要的东西\end{exemple}
\begin{exemple}\jya ki laχtɕha ki kɤ-βde ɯ-spa ɕti\cmn 这个东西可以扔\end{exemple}
\begin{exemple}\jya qapri ɯ-χsiu cho-βde\cmn 蛇脱皮了\end{exemple}
\begin{exemple}\jya tɯ-ɴɢar jo-βde\cmn 他吐了痰\end{exemple}\acception{2}
\paradigme{\textit{dir :} \jya nɯ-}
\begin{définition}\fra abandonner\end{définition}
\begin{définition}\cmn 放弃\end{définition}
\begin{exemple}\jya ɲo-βde\cmn 他放弃了\end{exemple}
\begin{exemple}\jya ɯ-ma z-ɲɤ-βde\cmn 他辞职去了\end{exemple}
\begin{exemple}\jya nɤ-kɤ-nɤma nɯ mɤ-kɯ-ftɯɣ a-mɤ-nɯ-tɯ-βde\cmn 你不要放弃没有完成的工作\end{exemple}
\begin{exemple}\jya qartsɯ ɲɤ-βde tɕe χɕitka ko-ndzoʁ\cmn 冬天结束了,春天到来了\end{exemple}\begin{sous-entrée}
\vedette{\hypertarget{}{\papi{ amɯβde}}}\markboth{amɯβde}{}\classe{vi}
\paradigme{\textit{dir :} \jya nɯ-}
\begin{définition}\fra se quitter, se séparer\end{définition}
\begin{définition}\cmn 互相离别,互相分开\end{définition}
\begin{exemple}\jya ɲɤ-k-ɤmɯβde-ndʑi-ci\cmn 他们互相离别了\end{exemple}
\end{sous-entrée}\begin{sous-entrée}
\vedette{\hypertarget{}{\papi{ asɤmɯβde}}}\markboth{asɤmɯβde}{}\classe{vi}
\begin{définition}\fra pouvoir se résigner à se quitter\end{définition}
\begin{définition}\cmn 舍得离别\end{définition}
\begin{exemple}\jya mɯ-pjɤ-k-ɤsɤmɯβde-ndʑi-ci\cmn 他们俩不舍得离别\end{exemple}
\end{sous-entrée}\begin{sous-entrée}
\vedette{\hypertarget{}{\papi{ nɤβdɤle}}}\markboth{nɤβdɤle}{}\classe{vt}\acception{1}
\begin{définition}\fra jeter dans tous les sens\end{définition}
\begin{définition}\cmn 甩来甩去,扔来扔去\end{définition}
\begin{exemple}\jya tɤ-pɤtso kɯ ɯ-kɯmtɕhɯ nɯ ɲɯ-ɤz-nɤβdɤle tɕe ɲɯ-ɤnɯɣro\cmn 小孩子把玩具甩来甩去\end{exemple}\acception{2}
\paradigme{\textit{dir :} \jya nɯ-}
\begin{définition}\fra négliger\end{définition}
\begin{définition}\cmn 忽略,不管\end{définition}
\begin{exemple}\jya tɤ-pɤtso ma-nɯ-tɯ-nɤβdɤle ma mɤ-pe\cmn 你不忽略小孩子\end{exemple}
\end{sous-entrée}\begin{sous-entrée}
\vedette{\hypertarget{}{\papi{ nɯβde}}}\markboth{nɯβde}{}\classe{vt}
\paradigme{\textit{dir :} \jya nɯ-}
\begin{définition}\ 
\begin{déclaration}\grammar{autoben}\end{déclaration}\end{définition}
\begin{définition}\fra perdre\end{définition}
\begin{définition}\cmn 遗失;弄丢\end{définition}
\begin{exemple}\jya laχtɕha ɲɤ-nɯβde-t-a\cmn 我把东西弄丢了\end{exemple}
\begin{exemple}\jya ɲɤ-nɯβde\cmn 他弄丢了\end{exemple}
\begin{exemple}\jya nɤ-rte ma-nɯ-tɯ-nɯβde\cmn 你不要把帽子弄丢\end{exemple}
\end{sous-entrée}\begin{sous-entrée}
\vedette{\hypertarget{}{\papi{ sɤmɯβde}}}\markboth{sɤmɯβde}{}\classe{vt}
\paradigme{\textit{dir :} \jya nɯ-}
\begin{définition}\fra séparer (des gens)\end{définition}
\begin{définition}\cmn 使……分开\end{définition}
\end{sous-entrée}\begin{sous-entrée}
\vedette{\hypertarget{}{\papi{ sɯβde}}}\markboth{sɯβde}{}\classe{vt}
\paradigme{\textit{dir :} \jya pɯ-}
\begin{définition}\ 
\begin{déclaration}\grammar{habil}\end{déclaration}\end{définition}
\begin{définition}\fra pouvoir se résigner à abandonner\end{définition}
\begin{définition}\cmn 舍得离开\end{définition}
\begin{exemple}\jya ɯ-pi ra mɯ-pjɤ-sɯβde\cmn 她舍不得离开她的姐姐们\end{exemple}
\end{sous-entrée}\end{entrée}

\begin{entrée}
\vedette{\hypertarget{Ⓔβdeti}{\papi{ βdeti}}}\markboth{βdeti}{}\classe{n}
\begin{définition}\fra fête du septième mois\end{définition}
\begin{définition}\cmn 七月份的看花节\end{définition}\end{entrée}

\begin{entrée}
\vedette{\hypertarget{Ⓔβdi}{\papi{ βdi}}}\markboth{βdi}{}
\classe{vs}
\paradigme{\textit{dir :} \jya tɤ-}
\paradigme{\textit{dir :} \jya nɯ-}
\paradigme{\textit{dir :} \jya thɯ-}
\paradigme{\textit{construction :} \jya bare infinitive}
\begin{définition}\fra beau\end{définition}
\begin{définition}\cmn 美观
\begin{déclaration} \étymologie{\papi{bde}}\end{déclaration}\end{définition}
\begin{exemple}\jya ɯ-βzu ɲɯ-βdi\cmn 他做得很好\end{exemple}
\begin{exemple}\jya ɯ-sɯm βdi\cmn 他放心了\end{exemple}
\begin{exemple}\jya ɯ-ta mɯ-ɲɤ-βdi\cmn 放得不平\end{exemple}
\begin{relation-sémantique}\confer{
\hyperlink{Ⓔaβdoʁβdi}{\textit{ \papi{aβdoʁβdi}}}
}\end{relation-sémantique}\begin{sous-entrée}
\vedette{\hypertarget{}{\papi{ mɤβdi ma}}}\markboth{mɤβdi ma}{}
\begin{définition}\fra au cas où\end{définition}
\begin{définition}\cmn 万一\end{définition}
\begin{exemple}\jya tɤ-pɤtso pjɯ-ndʐaβ mɤβdi ma mɤ-nɯɣɯfɕɤt\cmn 万一小孩子摔倒了,就不好交代\end{exemple}
\begin{exemple}\jya ɯʑo tu-ngo mɤβdi ma nɤj nɤ-taʁ ŋu\cmn 万一他生病了的话,就是你的责任了\end{exemple}
\end{sous-entrée}\begin{sous-entrée}
\vedette{\hypertarget{}{\papi{ tɯ-skhrɯ mɯ-ɲɤ-βdi}}}\markboth{tɯ-skhrɯ mɯ-ɲɤ-βdi}{}
\begin{définition}\fra tomber enceinte\end{définition}
\begin{définition}\cmn 怀孕\end{définition}
\begin{exemple}\jya ɯ-skhrɯ mɯ-ɲɤ-βdi\cmn 她怀孕了\end{exemple}
\end{sous-entrée}\end{entrée}

\begin{entrée}
\vedette{\hypertarget{Ⓔβdiwa}{\papi{ βdiwa}}}\markboth{βdiwa}{}\classe{n}\acception{1}
\begin{définition}\fra calme\end{définition}
\begin{définition}\cmn 平安\end{définition}\acception{2}
\begin{définition}\fra bonne action\end{définition}
\begin{définition}\cmn 善事
\begin{déclaration} \étymologie{\papi{bde.ba}}\end{déclaration}\end{définition}
\begin{exemple}\jya βdiwa ɲɤ-fkot\cmn 他做了善事\end{exemple}
\end{entrée}

\begin{entrée}
\vedette{\hypertarget{Ⓔβdɯnba}{\papi{ βdɯnba}}}\markboth{βdɯnba}{}\classe{n}
\begin{définition}\fra septième mois\end{définition}
\begin{définition}\cmn 七月
\begin{déclaration} \étymologie{\papi{bdun.pa}}\end{déclaration}\end{définition}
\end{entrée}

\begin{entrée}
\vedette{\hypertarget{Ⓔβdɯscit}{\papi{ βdɯscit}}}\markboth{βdɯscit}{}\classe{n}
\begin{définition}\fra bonheur\end{définition}
\begin{définition}\cmn 幸福
\begin{déclaration} \étymologie{\papi{bde.skʲit}}\end{déclaration}\end{définition}
\begin{exemple}\jya ɯ-βdɯscit ɲɤ-rɤru (=kɤ-scit to-khɯ)\cmn 他开始过幸福的生活了\end{exemple}\end{entrée}

\begin{entrée}
\vedette{\hypertarget{ⒺβdɯtⒽ1}{\papi{ βdɯt}}}\markboth{βdɯt}{}\homonyme{1}
\classe{np}\acception{1}
\begin{définition}\fra gaspillage\end{définition}
\begin{définition}\cmn 浪费\end{définition}
\begin{exemple}\jya khɯna kɤ-χsu ʁo tɯ-χsu βdɯt ɕti\cmn 喂狗简直是白喂\end{exemple}
\begin{exemple}\jya ɯ-βdɯt ma-pɯ-tɯ-sɯβze\cmn 不要浪费\end{exemple}\acception{2}
\begin{définition}\fra dépense\end{définition}
\begin{définition}\cmn 花费\end{définition}
\begin{exemple}\jya nɤ-βdɯt nɯ-tɕat-a\cmn 让你花费了很多(谢你请我吃饭)\end{exemple}
\begin{exemple}\jya nɤ-βdɯt tu ma jɤ-ɣe-j ɕti, dɤn-i ko!\cmn 你花费要多,我们来了很多人\end{exemple}
\begin{relation-sémantique}\confer{
\hyperlink{Ⓔkɤnɯβdɯt}{\textit{ \papi{kɤnɯβdɯt}}}
}\end{relation-sémantique}\end{entrée}

\begin{entrée}
\vedette{\hypertarget{ⒺβdɯtⒽ2}{\papi{ βdɯt}}}\markboth{βdɯt}{}\homonyme{2}
\classe{n}
\begin{définition}\fra une sorte de monstre\end{définition}
\begin{définition}\cmn 魔鬼
\begin{déclaration} \étymologie{\papi{bdud}}\end{déclaration}\end{définition}\end{entrée}

\begin{entrée}
\vedette{\hypertarget{Ⓔβgoz}{\papi{ βgoz}}}\markboth{βgoz}{} (\variante{fkoz}) 
\classe{vt}
\paradigme{\textit{dir :} \jya tɤ-}
\paradigme{\textit{dir :} \jya pɯ-}
\begin{définition}\fra préparer, planifier\end{définition}
\begin{définition}\cmn 计划;设计;准备需要的材料
\begin{déclaration} \étymologie{\papi{bgod}}\end{déclaration}\end{définition}
\begin{exemple}\jya kɯki kha ci tɤ-βgoz-a\cmn 我设计了这个房子\end{exemple}
\begin{exemple}\jya tɯ-ŋga ɯ-spa ci tɤ-βgoz-a\cmn 我准备了缝衣服的材料\end{exemple}
\begin{exemple}\jya tɯ-xtsa kɤ-βzu to-βgoz\cmn 他设计了鞋子的款式\end{exemple}
\begin{exemple}\jya kɯki tɤ-pɤtso ki kɯ-rɤβzjoz tɤ-βgoz-i\cmn 我们准备让这个孩子去读书\end{exemple}\begin{sous-entrée}
\vedette{\hypertarget{}{\papi{ βgoz}}}\markboth{βgoz}{}\classe{vi}
\begin{définition}\fra arranger (marriage)\end{définition}
\begin{définition}\cmn 包办(婚姻)\end{définition}
\begin{exemple}\jya phama pɯ-βgoz pɯ-ŋu ma ʑɤni nɯ-anɯtɯɣ-ndʑi pɯ-maʁ\cmn 他们的婚姻是父母包办的,不是他们自己主张的\end{exemple}
\end{sous-entrée}\end{entrée}

\begin{entrée}
\vedette{\hypertarget{Ⓔβɣu}{\papi{ βɣu}}}\markboth{βɣu}{}
\classe{n}
\begin{définition}\fra louche en bois\end{définition}
\begin{définition}\cmn 木头刻成的小瓢\end{définition}\end{entrée}

\begin{entrée}
\vedette{\hypertarget{Ⓔβɣa}{\papi{ βɣa}}}\markboth{βɣa}{}\classe{n}
\begin{définition}\fra moulin à eau\end{définition}
\begin{définition}\cmn 水磨\end{définition}
\begin{exemple}\jya βɣa ɲɯ-pe tɕe khro tha-ɣndʑɯr\cmn 磨子很好用(水量多),磨了很多粮食\end{exemple}\end{entrée}

\begin{entrée}
\vedette{\hypertarget{Ⓔβɣɤɣur}{\papi{ βɣɤɣur}}}\markboth{βɣɤɣur}{}\classe{n}
\begin{définition}\fra farine qui sort de la meule\end{définition}
\begin{définition}\cmn 从磨撒出来的面粉\end{définition}
\begin{relation-sémantique}\confer{
\hyperlink{Ⓔβɣa}{\textit{ \papi{βɣa}}}
}\end{relation-sémantique}
\begin{relation-sémantique}\confer{
\hyperlink{Ⓔtɤ-ɣur}{\textit{ \papi{tɤ-ɣur}}}
}\end{relation-sémantique}\end{entrée}

\begin{entrée}
\vedette{\hypertarget{Ⓔβɣɤmu}{\papi{ βɣɤmu}}}\markboth{βɣɤmu}{}
\classe{n}
\begin{définition}\fra entonnoir en peau pour verser l'orge dans la meule\end{définition}
\begin{définition}\cmn 用来把青稞灌到磨口中的三角形牛皮,中间有洞\end{définition}
\begin{exemple}\jya βɣɤmu nɯ kɤ-ɣndʑɯr ɯ-spa tɯjpu nɯ ɯ-sɤ-rku tɯ-ndʐi thɯ-kɤ-rɤɣdɯt tɕe βɣa ɯ-ŋgɯ ɯ-pjɯ-kɯ-lɤt ɯ-spa ŋu\cmn 
\stylefv{βɣɤmu}是用来装待磨的粮食的(保持整头牛原形的)牛皮,也是用来把粮食倒入磨坊的工具。
\end{exemple}\end{entrée}

\begin{entrée}
\vedette{\hypertarget{Ⓔβɣɤno}{\papi{ βɣɤno}}}\markboth{βɣɤno}{}
\classe{n}
\begin{définition}\fra partie inférieure de la meule\end{définition}
\begin{définition}\cmn 下磨盘\end{définition}
\begin{exemple}\jya βɣɤno nɯ rdɤstaʁ kɯ-ɤrtɯm nɯ kɤ-βzu ŋu tɕe pjɯ-maŋpa tɕe ku-rɤsta ŋu\cmn 
\stylefv{βɣɤno}是由一块圆形石块凿成的,处于磨子的下面,是固定的。
\end{exemple}\end{entrée}

\begin{entrée}
\vedette{\hypertarget{Ⓔβɣɤŋgɯ}{\papi{ βɣɤŋgɯ}}}\markboth{βɣɤŋgɯ}{}\classe{n}
\begin{définition}\fra partie supérieure du moulin\end{définition}
\begin{définition}\cmn 磨坊的上部\end{définition}
\begin{relation-sémantique}\confer{
\hyperlink{Ⓔβɣa}{\textit{ \papi{βɣa}}}
}\end{relation-sémantique}
\begin{relation-sémantique}\confer{
\hyperlink{Ⓔɯ-ŋgɯ}{\textit{ \papi{ɯ-ŋgɯ}}}
}\end{relation-sémantique}\end{entrée}

\begin{entrée}
\vedette{\hypertarget{Ⓔβɣɤpa}{\papi{ βɣɤpa}}}\markboth{βɣɤpa}{}\classe{n}
\begin{définition}\fra partie inférieure du moulin (dans l'eau)\end{définition}
\begin{définition}\cmn 磨坊的下部(河水流动的地方)\end{définition}
\begin{relation-sémantique}\confer{
\hyperlink{Ⓔβɣa}{\textit{ \papi{βɣa}}}
}\end{relation-sémantique}
\begin{relation-sémantique}\confer{
 \papi{ɯ-pa}
}\end{relation-sémantique}\end{entrée}

\begin{entrée}
\vedette{\hypertarget{Ⓔβɣɤru}{\papi{ βɣɤru}}}\markboth{βɣɤru}{}\classe{n}
\begin{définition}\fra meunier\end{définition}
\begin{définition}\cmn 守磨坊的用人\end{définition}
\begin{définition}\jya \end{définition}
\begin{relation-sémantique}\confer{
\hyperlink{Ⓔrɯru}{\textit{ \papi{rɯru}}}
}\end{relation-sémantique}
\begin{relation-sémantique}\confer{
\hyperlink{Ⓔβɣa}{\textit{ \papi{βɣa}}}
}\end{relation-sémantique}\end{entrée}

\begin{entrée}
\vedette{\hypertarget{Ⓔβɣɤrkhɤrkhɤt}{\papi{ βɣɤrkhɤrkhɤt}}}\markboth{βɣɤrkhɤrkhɤt}{}\classe{n}
\begin{définition}\fra bâton qui sert à frapper l'entonnoir de peau\end{définition}
\begin{définition}\cmn 用来震动牛皮漏斗的木棒\end{définition}
\begin{exemple}\jya βɣɤrkhɤrkhɤt nɯ laʁjɯɣ ci ŋu tɕe tɯ-pɕoʁ chiz nɯ βɣɤmu ɯ-taʁ kú-wɣ-βraʁ tɕe mɤpɕoʁ chiz nɯ βɣɤtu ɯ-taʁ pjɯ-tɯɣ tɕe βɣa kɤ-mtɕɯr tɕe βɣɤrkhɤrkhɤt tu-znɤndɤr tɕe βɣɤmu ɯ-ŋgɯ kɤ-ɣndʑɯr ɯ-spa nɯ pjɯ-nɯɬoʁ ɲɯ-ŋu tɕe βɣa ɯ-ŋgɯ pjɯ-ɕe ŋu\cmn 
\stylefv{βɣɤrkhɤrkhɤt}是一根木棒,一头拴在牛皮漏斗上,另一头靠着石磨。石磨转动时,就使木棒震动,使漏斗里的粮食漏进磨子里去
\end{exemple}
\begin{relation-sémantique}\confer{
\hyperlink{Ⓔβɣa}{\textit{ \papi{βɣa}}}
}\end{relation-sémantique}
\begin{relation-sémantique}\confer{
\hyperlink{Ⓔrkhɤrkhɤt}{\textit{ \papi{rkhɤrkhɤt}}}
}\end{relation-sémantique}\end{entrée}

\begin{entrée}
\vedette{\hypertarget{Ⓔβɣɤrnɤjwaʁ}{\papi{ βɣɤrnɤjwaʁ}}}\markboth{βɣɤrnɤjwaʁ}{}\classe{n}
\begin{définition}\fra pales du moulin\end{définition}
\begin{définition}\cmn 磨坊下面的木板\end{définition}
\begin{exemple}\jya βɣɤrnɤjwaʁ nɯ tɕhɯŋkhɤr ɯ-taʁ tɤrɤm kɤ-kɤ-tshoʁ nɯ ŋu, tɯ-ci βɣɤrnɤjwaʁ ɯ-taʁ chɯ-lɤt tɕe tɕhɯŋkhɤr ku-sɯ-mtɕɯr tɕe tɕhɯŋkhɤr ɯ-taʁ βɣɤsni ndzoʁ tɕe βɣɤtu ku-sɯ-mtɕɯr ŋu\cmn 
\stylefv{βɣɤrnɤjwaʁ}是装在磨房下的水车上的木板,水冲在木板上时,木板使水车转动,上面装有磨心,使上磨盘转动起来。
\end{exemple}
\begin{relation-sémantique}\confer{
\hyperlink{Ⓔβɣa}{\textit{ \papi{βɣa}}}
}\end{relation-sémantique}
\begin{relation-sémantique}\confer{
\hyperlink{Ⓔtɯ-rna}{\textit{ \papi{tɯ-rna}}}
}\end{relation-sémantique}
\begin{relation-sémantique}\confer{
\hyperlink{Ⓔtɤ-jwaʁ}{\textit{ \papi{tɤ-jwaʁ}}}
}\end{relation-sémantique}
\end{entrée}

\begin{entrée}
\vedette{\hypertarget{Ⓔβɣɤrqo}{\papi{ βɣɤrqo}}}\markboth{βɣɤrqo}{}\classe{n}
\begin{définition}\fra joint entre le sac de graines et la meule\end{définition}
\begin{définition}\cmn 粮食口袋和磨斗之间的接头部分\end{définition}
\begin{relation-sémantique}\confer{
\hyperlink{Ⓔβɣa}{\textit{ \papi{βɣa}}}
}\end{relation-sémantique}
\begin{relation-sémantique}\confer{
\hyperlink{Ⓔtɯ-rqo}{\textit{ \papi{tɯ-rqo}}}
}\end{relation-sémantique}\end{entrée}

\begin{entrée}
\vedette{\hypertarget{Ⓔβɣɤrtshi}{\papi{ βɣɤrtshi}}}\markboth{βɣɤrtshi}{}
\classe{n}
\begin{définition}\fra moustique\end{définition}
\begin{définition}\cmn 蚊子\end{définition}\end{entrée}

\begin{entrée}
\vedette{\hypertarget{Ⓔβɣɤsɤprɤt}{\papi{ βɣɤsɤprɤt}}}\markboth{βɣɤsɤprɤt}{}\classe{n}
\begin{définition}\fra écluses du moulin\end{définition}
\begin{définition}\cmn 开关磨坊水道用的闸门\end{définition}
\begin{exemple}\jya βɣɤsɤprɤt nɯ kɤ-ɣndʑɯr pɯ-jɤɣ tɕe tɯ-ci ɯ-sɤ-prɤt tɤrɤm ci tu tɕe nɯnɯ ŋu\cmn 
\stylefv{βɣɤsɤprɤt}是一块木板,磨好面以后,用来把水闸住。
\end{exemple}
\begin{relation-sémantique}\confer{
\hyperlink{Ⓔβɣa}{\textit{ \papi{βɣa}}}
}\end{relation-sémantique}
\begin{relation-sémantique}\confer{
\hyperlink{Ⓔprɤt}{\textit{ \papi{prɤt}}}
}\end{relation-sémantique}\end{entrée}

\begin{entrée}
\vedette{\hypertarget{Ⓔβɣɤsni}{\papi{ βɣɤsni}}}\markboth{βɣɤsni}{}\classe{n}
\begin{définition}\fra axe du moulin\end{définition}
\begin{définition}\cmn 磨心\end{définition}
\begin{exemple}\jya βɣɤsni nɯ βɣɤno ɯ-ku kɯ-sɯ-mtɕɯr ɕom thɯ-kɤ-βzu ŋu, ɯ-tshɯɣa nɯ thɤtɕɯ cho naχtɕɯɣ ri ɯ-jɯ nɯ ɕom ɕti\cmn 磨心用来转动上磨盘。用生铁打成,形状和二锤一样,但是柄也是铁打的。\end{exemple}
\begin{relation-sémantique}\confer{
\hyperlink{Ⓔβɣa}{\textit{ \papi{βɣa}}}
}\end{relation-sémantique}
\begin{relation-sémantique}\confer{
\hyperlink{Ⓔtɯ-sni}{\textit{ \papi{tɯ-sni}}}
}\end{relation-sémantique}
\end{entrée}

\begin{entrée}
\vedette{\hypertarget{Ⓔβɣɤsroʁ}{\papi{ βɣɤsroʁ}}}\markboth{βɣɤsroʁ}{}\classe{n}
\begin{définition}\fra levier utiliser pour contrôler la vitesse de la meule\end{définition}
\begin{définition}\cmn 用来控制磨面粗细的木棒\end{définition}
\begin{exemple}\jya βɣɤsroʁ nɯ si nɯ-kɤ-βzu ci ŋu tɕe kɤ-ɣndʑɯr thɯ́-wɣ-lɤt tɕe pjɯ-jndʐɤz cho βɣa pjɯ-rɟɯɣ tɤ-ra tɕe kɤ-joʁ spa, kɤ-ɣndʑɯr pjɯ-ndɯβ cho pjɯ-ɣɤdal tɤ-ra tɕe kɤ-phɤβ spa ŋu\cmn 
\stylefv{βɣɤsroʁ}是一根木棒。需要磨得粗、转速快的时候,把它抬高;需要磨得细、转速慢的时候,把它放低。
\end{exemple}
\begin{relation-sémantique}\confer{
\hyperlink{Ⓔβɣa}{\textit{ \papi{βɣa}}}
}\end{relation-sémantique}
\begin{relation-sémantique}\confer{
\hyperlink{Ⓔmbɣɤsroʁ}{\textit{ \papi{mbɣɤsroʁ}}}
}\end{relation-sémantique}\end{entrée}

\begin{entrée}
\vedette{\hypertarget{Ⓔβɣɤtu}{\papi{ βɣɤtu}}}\markboth{βɣɤtu}{}
\classe{n}
\begin{définition}\fra partie supérieure de la meule\end{définition}
\begin{définition}\cmn 上磨盘\end{définition}
\begin{exemple}\jya βɣɤtu nɯ rdɤstaʁ kɯ-ɤrtɯm nɯ kɤ-βzu tɕe βɣɤno ɯ-taʁ pjɯ́-wɣ-ta tɕe ɯ-ŋgɯ βɣɤsni tú-wɣ-sthoʁ tɕe ku-mtɕɯr ŋu\cmn 
\stylefv{βɣɤtu}是由一块圆形石块作成的,处于下磨盘的上面,中间装上磨心。可以转动。
\end{exemple}\end{entrée}

\begin{entrée}
\vedette{\hypertarget{Ⓔβɣɤtɤtɤɣ}{\papi{ βɣɤtɤtɤɣ}}}\markboth{βɣɤtɤtɤɣ}{}\classe{n}
\begin{définition}\fra bord de la meule\end{définition}
\begin{définition}\cmn 磨盘边缘\end{définition}
\begin{exemple}\jya βɣɤtɤtɤɣ nɯ kɯ-ɣndʑɯr pɯ́-wɣ-lɤt tɕe tɯ-ɣndʑɤr ɣɯ ɯ-ɣur spa ŋu, rgɤm pɯ-kɤ-sprɤt kɯ-fse ŋu\cmn 
\stylefv{βɣɤtɤtɤɣ}围在磨盘周围,挡着磨出来的面粉,安装得像木箱一样。
\end{exemple}
\begin{relation-sémantique}\confer{
\hyperlink{Ⓔβɣa}{\textit{ \papi{βɣa}}}
}\end{relation-sémantique}
\begin{relation-sémantique}\confer{
\hyperlink{Ⓔtɤtɤɣ}{\textit{ \papi{tɤtɤɣ}}}
}\end{relation-sémantique}\end{entrée}

\begin{entrée}
\vedette{\hypertarget{Ⓔβɣɤχtsiɯ}{\papi{ βɣɤχtsiɯ}}}\markboth{βɣɤχtsiɯ}{}
\classe{n}
\begin{définition}\fra bouche de la meule\end{définition}
\begin{définition}\cmn 磨斗\end{définition}
\begin{exemple}\jya βɣɤχtsiɯ nɯ βɣɤtu ɯ-taʁ ku-ndzoʁ tɕe kɤ-ɣndʑɯr ɯ-spa pjɯ-kɯ-ɣi nɯ ɯ-kɯ-nɤjo ɯ-spa ŋu\cmn 
\stylefv{βɣɤχtsiɯ}是装在上磨盘的中间,用来接住漏斗里出来的粮食。
\end{exemple}\end{entrée}

\begin{entrée}
\vedette{\hypertarget{Ⓔβɣɤza}{\papi{ βɣɤza}}}\markboth{βɣɤza}{}
\classe{n}
\begin{définition}\fra mouche\end{définition}
\begin{définition}\cmn 苍蝇\end{définition}\end{entrée}

\begin{entrée}
\vedette{\hypertarget{Ⓔβɣɯt}{\papi{ βɣɯt}}}\markboth{βɣɯt}{}\classe{vi}
\paradigme{\textit{dir :} \jya kɤ-}
\begin{définition}\fra brûler (poils, plumes)\end{définition}
\begin{définition}\cmn 被烧(毛,羽毛)\end{définition}
\begin{exemple}\jya ɯ-kɤrme ko-βɣɯt\cmn 他头发被烧到\end{exemple}\begin{sous-entrée}
\vedette{\hypertarget{}{\papi{ sɯβɣɯt}}}\markboth{sɯβɣɯt}{}\classe{vt}
\paradigme{\textit{dir :} \jya kɤ-}
\paradigme{\textit{dir :} \jya thɯ-}
\begin{définition}\fra brûler (poils, plumes)\end{définition}
\begin{définition}\cmn 烧(毛,羽毛)\end{définition}
\begin{exemple}\jya tɤ-ŋkɯ thɯ-sɯβɣɯt-a\cmn 我把猪毛烧掉了\end{exemple}
\end{sous-entrée}\end{entrée}

\begin{entrée}
\vedette{\hypertarget{Ⓔβɣɯz}{\papi{ βɣɯz}}}\markboth{βɣɯz}{}\classe{n}
\begin{définition}\fra blaireau\end{définition}
\begin{définition}\cmn 獾【臭猪子】\end{définition}
\begin{exemple}\jya βɣɯz nɯ sɤtɕha kɯ-spoʁ ɯ-ŋgɯ ku-rɤʑi ɲɯ-ŋu, praʁ pa tɕu ku-rɤʑi ɲɯ-ŋu, ftɕar tɕe ma qartsɯ tɕe kɤ-mto me, βɣɯz nɯ ɯ-mtɕhi nɯ ra kɯ-ɤmtɕoʁ, ɯ-ku ɯ-tshɯɣa nɯ ra βʑɯ tsa fse, ɯ-rme nɯ pɣi ri paʁ ɯ-rme ʑo fse tɕe nɤrko. ɯ-xtɤpa ɯ-rme ra aqarŋɯrŋe, qandʐe kɤ-ndza wuma ʑo rga, tɕe ɕɤr tɕe, kha ɯ-rkɯ tɯ-ɣli kɯ-dɤn ɣɯ sɤtɕha ɲɯ-sloʁ tɕe, tɕe tɤ-rɤku ra ɯ-qa chɯ-tɕɤt tɕe chɯ-tʂaβ tɕe pjɯ-sɯ-lni ŋu. ju-phɣo tɕe, kɯ-spoʁ ɯ-ŋgɯ lu-nɯɕe tɕe, tɕe kɤ-sat mɤ-khɯ tɕe, ɯ-kɯm smi pjɯ-βlɯ-nɯ tɕe, lu-sɤkhɯ-nɯ tɕe, tɕe mɯ-tɤ-tɕhaʁ tɕe chɯ-nɯɬoʁ tɕe pjɯ-sat-nɯ ŋgrɤl. βɣɯz nɯ ɯ-di wuma ʑo kɯ-χɕɤβ ci tu, tɕe ɯ-ndʐi nɯ pjɯ́-wɣ-nɤβɟu nɯ maʁ nɤ tɯ-mthɤɣ ɲɯ́-wɣ-rtɤβ tɕe χtɕoŋ kɯ-phɤn ɲɯ-ŋu\cmn 獾住在土洞和岩洞里,只在夏天出现,冬天看不到。獾的嘴很尖,头部的形状像老鼠,毛是灰色的,像猪的毛一样,但比较硬。肚子下面的毛是淡黄色。它很喜欢吃蚯蚓,所以晚上它去拱房子旁边的肥地,会把庄稼拱倒,使庄稼枯萎。逃跑的时候就会钻进洞里,没法打死它,就在洞口烧火让烟子熏它,它受不了从洞里跑出来的时候才能把它打死。獾有很浓的臭味,它的皮子要么当作坐垫,要么围在腰部,可以治风湿病。\end{exemple}
\begin{relation-sémantique}\confer{
\hyperlink{Ⓔnɯβɣɯz}{\textit{ \papi{nɯβɣɯz}}}
}\end{relation-sémantique}\end{entrée}

\begin{entrée}
\vedette{\hypertarget{Ⓔβɟɤt}{\papi{ βɟɤt}}}\markboth{βɟɤt}{}\classe{vi}
\paradigme{\textit{dir :} \jya pɯ-}
\begin{définition}\fra obtenir\end{définition}
\begin{définition}\cmn 得到(某个大家在抢的东西);中奖\end{définition}
\begin{exemple}\jya pɯ-βɟat-a, pɯ-βɟɤt\cmn 我得到了,他得到了\end{exemple}
\begin{exemple}\jya tɤ-mbɣom tɕe a-pɯ-tɯ-βɟɤt\cmn 你快点就会得到\end{exemple}
\begin{exemple}\jya kɤ-βɟɤt khɯ\cmn 可以得到\end{exemple}
\begin{exemple}\jya laχtɕha ɲɯ-rkɯn ri, aʑo pɯ-βɟat-a\cmn 东西虽然很少,但是我还是得到了\end{exemple}
\begin{exemple}\jya laχtɕha kɤ-χtɯ pjɤ-βɟɤt\cmn 买到了东西\end{exemple}
\begin{sous-entrée}
\vedette{\hypertarget{}{\papi{ sɯβɟɤt}}}\markboth{sɯβɟɤt}{}\classe{vt}
\begin{définition}\fra fair eobtenir\end{définition}
\begin{définition}\cmn 令……得到\end{définition}
\begin{exemple}\jya tɤ-ndze ma tha mɤ-ta-sɯβɟɤt\cmn 你吃,不然我会吃完,令你吃不到\end{exemple}
\end{sous-entrée}\end{entrée}

\begin{entrée}
\vedette{\hypertarget{ⒺβɟiⒽ1}{\papi{ βɟi}}}\markboth{βɟi}{}\homonyme{1}
\classe{vt}
\paradigme{\textit{dir :} \jya \_}
\begin{définition}\fra suivre, poursuivre\end{définition}
\begin{définition}\cmn 追赶\end{définition}
\begin{exemple}\jya tɤ-βɟi-t-a\cmn 我追了他\end{exemple}
\begin{exemple}\jya ja-βɟi\end{exemple}
\begin{exemple}\jya tha-βɟi\cmn 他追了他\end{exemple}
\begin{exemple}\jya ɯʑo jɤ-anɯri tɕe, z-ja-βɟi-t-a\cmn 他回去了,我就追了他\end{exemple}
\begin{relation-sémantique}\confer{
\hyperlink{Ⓔnɤβɟɯβɟi}{\textit{ \papi{nɤβɟɯβɟi}}}
}\end{relation-sémantique}
\begin{sous-entrée}
\vedette{\hypertarget{}{\papi{ sɯβɟi}}}\markboth{sɯβɟi}{}\classe{vt}
\paradigme{\textit{dir :} \jya \_}
\begin{définition}\fra faire poursuivre\end{définition}
\begin{définition}\cmn 令……追赶\end{définition}
\end{sous-entrée}\end{entrée}

\begin{entrée}
\vedette{\hypertarget{ⒺβɟiⒽ2}{\papi{ βɟi}}}\markboth{βɟi}{}\homonyme{2}\classe{vs}
\paradigme{\textit{dir :} \jya nɯ-}
\begin{définition}\fra ancien\end{définition}
\begin{définition}\cmn 陈旧\end{définition}
\begin{exemple}\jya tɤ-mthɯm ɲɤ-βɟi\cmn 肉放陈了\end{exemple}
\begin{exemple}\jya laχtɕha ɲɤ-βɟi\cmn 东西旧了\end{exemple}\end{entrée}

\begin{entrée}
\vedette{\hypertarget{Ⓔβlama}{\papi{ βlama}}}\markboth{βlama}{}\classe{n}
\begin{définition}\fra lama\end{définition}
\begin{définition}\cmn 喇嘛
\begin{déclaration} \étymologie{\papi{bla.ma}}\end{déclaration}\end{définition}\end{entrée}

\begin{entrée}
\vedette{\hypertarget{Ⓔβlɤmɤjmɤɣ}{\papi{ βlɤmɤjmɤɣ}}}\markboth{βlɤmɤjmɤɣ}{}\classe{n}
\begin{définition}\fra une espèce de champignon\end{définition}
\begin{définition}\cmn 鹅蛋菌\end{définition}
\begin{exemple}\jya βlɤmɤjmɤɣ nɯ ɕkrɤz ɯ-ŋgɯ tu-ɬoʁ ŋu, ɯ-mgɯrqhu nɯ ʁmɤrsɤr ŋu, ɯ-rʑɯɣ cho ɯ-ru nɯ kɯ-qarŋe ŋu, tɤ-ɬoʁ ɕɯmɯma tɕe kɯ-wɣrum ci kɯ aluj tɕe tɤ-ŋgɯm ʑo fse, tɕe tɤ-wxti tɕe ɯ-luj pjɤ-ɴɢaʁ ŋu.\cmn 鹅蛋菌长在青冈树林里,背面是金黄色,菌褶和主干是黄色的。刚长出来的时候,有一种白色的外层遮盖,像鸡蛋一样,长大了以后外层就会破。\end{exemple}
\end{entrée}

\begin{entrée}
\vedette{\hypertarget{Ⓔβlɤmtɕhɤt}{\papi{ βlɤmtɕhɤt}}}\markboth{βlɤmtɕhɤt}{}\classe{n}
\begin{définition}\fra récitation de soutras\end{définition}
\begin{définition}\cmn (为别人)念经\end{définition}
\begin{relation-sémantique}\confer{
\hyperlink{Ⓔnɯβlɤmtɕhɤt}{\textit{ \papi{nɯβlɤmtɕhɤt}}}
}\end{relation-sémantique}\end{entrée}

\begin{entrée}
\vedette{\hypertarget{Ⓔβluβra}{\papi{ βluβra}}}\markboth{βluβra}{}\classe{n}
\begin{définition}\fra suggestion, idée, conseil\end{définition}
\begin{définition}\cmn 主意;指挥\end{définition}
\begin{exemple}\jya a-βluβra ci tɤ-tɕɤt\cmn 给我出主意吧\end{exemple}
\begin{relation-sémantique}\synonyme{
 \papi{ɯ-ftɕɤfkɤt}
}\end{relation-sémantique}
\begin{relation-sémantique}\synonyme{
\hyperlink{Ⓔɯ-βlaβlu}{\textit{ \papi{ɯ-βlaβlu}}}
}\end{relation-sémantique}
\begin{relation-sémantique}\confer{
\hyperlink{Ⓔrɯβluβra}{\textit{ \papi{rɯβluβra}}}
}\end{relation-sémantique}\end{entrée}

\begin{entrée}
\vedette{\hypertarget{Ⓔβli}{\papi{ βli}}}\markboth{βli}{}\classe{vt}\paradigme{\textit{dir :} \jya pɯ-}
\paradigme{\textit{dir :} \jya lɤ-}
\begin{définition}\fra planter\end{définition}
\begin{définition}\cmn 栽\end{définition}
\begin{exemple}\jya pɯ-βli-t-a, pa-βli\cmn 我栽种了,他栽种了\end{exemple}
\begin{exemple}\jya ɕaŋβli pɯ-βli-t-a\cmn 我栽了树苗\end{exemple}\end{entrée}

\begin{entrée}
\vedette{\hypertarget{Ⓔβlunbu}{\papi{ βlunbu}}}\markboth{βlunbu}{}\classe{n}
\begin{définition}\fra ministre\end{définition}
\begin{définition}\cmn 大臣
\begin{déclaration} \étymologie{\papi{blon.po}}\end{déclaration}\end{définition}
\end{entrée}

\begin{entrée}
\vedette{\hypertarget{Ⓔβlɯ}{\papi{ βlɯ}}}\markboth{βlɯ}{}\classe{vl}\acception{1}
\paradigme{\textit{dir :} \jya tɤ-}
\paradigme{\textit{dir :} \jya thɯ-}
\begin{définition}\fra allumer un feu\end{définition}
\begin{définition}\cmn 烧火\end{définition}
\begin{exemple}\jya smi thɯ-βlɯ-t-a, smi tha-βlɯ\cmn 我烧了火,他烧了火\end{exemple}
\begin{exemple}\jya aʑo tɤ-βlɯ-a tɕe ndzɤtshi βze-a ŋu\cmn 我烧了火就做饭\end{exemple}\acception{2}
\paradigme{\textit{dir :} \jya lɤ-}
\begin{définition}\fra brûler (du bois)\end{définition}
\begin{définition}\cmn 烧(木柴)\end{définition}
\begin{exemple}\jya si lɤ-βlɯ-t-a\cmn 我烧了柴火\end{exemple}
\begin{relation-sémantique}\confer{
\hyperlink{Ⓔnɯrmɤβlɯ}{\textit{ \papi{nɯrmɤβlɯ}}}
}\end{relation-sémantique}\end{entrée}

\begin{entrée}
\vedette{\hypertarget{Ⓔβlɯɣnɤβlɯɣ}{\papi{ βlɯɣnɤβlɯɣ}}}\markboth{βlɯɣnɤβlɯɣ}{}
\classe{idph.3}
\begin{définition}\fra source de lumière irisée qui scintille\end{définition}
\begin{définition}\cmn 形容彩色的光闪光的样子\end{définition}
\begin{exemple}\jya @jingdeng βlɯɣnɤβlɯɣ ɲɯ-ɤsɯ-stu\cmn 警灯在闪光\end{exemple}\end{entrée}

\begin{entrée}
\vedette{\hypertarget{Ⓔβraʁ}{\papi{ βraʁ}}}\markboth{βraʁ}{}\classe{vt}
\paradigme{\textit{dir :} \jya kɤ-}
\begin{définition}\fra attacher\end{définition}
\begin{définition}\cmn 拴\end{définition}
\begin{exemple}\jya kɤ-βraʁ-a, kɤ-tɯ-βraʁ, ka-βraʁ\cmn 我拴了,你拴了,他拴了\end{exemple}
\begin{exemple}\jya khɯna ɲɤ-lɯɣ tɕe kɤ-βraʁ-a\cmn 狗摆脱了绳子,我又把它拴起来了\end{exemple}
\begin{exemple}\jya tɯmbri kɤ-βraʁ-a\cmn 我拴了绳子(在某个地方打了个结)\end{exemple}
\begin{relation-sémantique}\confer{
\hyperlink{Ⓔzbraʁ}{\textit{ \papi{zbraʁ}}}
}\end{relation-sémantique}
\begin{relation-sémantique}\confer{
\hyperlink{Ⓔɯ-βraʁ}{\textit{ \papi{ɯ-βraʁ}}}
}\end{relation-sémantique}\begin{sous-entrée}
\vedette{\hypertarget{}{\papi{ aβraʁ}}}\markboth{aβraʁ}{}\classe{vi}
\begin{définition}\fra attaché\end{définition}
\begin{définition}\cmn 拴着\end{définition}
\begin{exemple}\jya nɯtɕu khɯna pjɤ-k-ɤβraʁ-ci\cmn 那里以前拴过狗\end{exemple}
\end{sous-entrée}\begin{sous-entrée}
\vedette{\hypertarget{}{\papi{ nɤβrɯβraʁ}}}\markboth{nɤβrɯβraʁ}{}\classe{vt}
\begin{définition}\fra attacher dans tous les sens\end{définition}
\begin{définition}\cmn 拴来拴去\end{définition}
\end{sous-entrée}\begin{sous-entrée}
\vedette{\hypertarget{}{\papi{ nɯβraʁ}}}\markboth{nɯβraʁ}{}\classe{vt}
\paradigme{\textit{dir :} \jya tɤ-}
\paradigme{\textit{dir :} \jya pɯ-}
\begin{définition}\fra porter (collier)\end{définition}
\begin{définition}\cmn 戴(项链)\end{définition}
\end{sous-entrée}\end{entrée}

\begin{entrée}
\vedette{\hypertarget{Ⓔβri}{\papi{ βri}}}\markboth{βri}{}
\classe{vt}
\paradigme{\textit{dir :} \jya kɤ-}
\paradigme{\textit{dir :} \jya tɤ-}
\begin{définition}\fra défendre\end{définition}
\begin{définition}\cmn 掩护;维护\end{définition}
\begin{exemple}\jya kɤ-βrit-a, tɤ-βrit-a, ka-βri\cmn 我维护了,他维护了\end{exemple}
\begin{exemple}\jya ɲɯ-ɤlɯlɤt-ndʑi tɕe, tɤ-βri-t-a\cmn 他们打架,我维护了他\end{exemple}
\begin{exemple}\jya khɯna kɯ ɣɯ-mtsɯɣ ɲɯ-ŋu tɕe, kɤ-βri-t-a\cmn 狗快要咬他的时候,我保护了他\end{exemple}\begin{sous-entrée}
\vedette{\hypertarget{}{\papi{ nɯʑɣɤβri}}}\markboth{nɯʑɣɤβri}{}\classe{vi}
\paradigme{\textit{dir :} \jya tɤ-}
\begin{définition}\fra se défendre\end{définition}
\begin{définition}\cmn 保护自己\end{définition}
\end{sous-entrée}\begin{sous-entrée}
\vedette{\hypertarget{}{\papi{ sɤβri}}}\markboth{sɤβri}{}\classe{vi}
\begin{définition}\fra protéger des gens\end{définition}
\begin{définition}\cmn 维护别人\end{définition}
\end{sous-entrée}\end{entrée}

\begin{entrée}
\vedette{\hypertarget{Ⓔβrɟaŋ}{\papi{ βrɟaŋ}}}\markboth{βrɟaŋ}{}\classe{vt}
\paradigme{\textit{dir :} \jya nɯ-}
\begin{définition}\fra tendre (peau)\end{définition}
\begin{définition}\cmn 绷紧
\begin{déclaration} \étymologie{\papi{brgʲaŋ}}\end{déclaration}\end{définition}
\begin{exemple}\jya aʑo tɯ-ndʐi nɯ-βrɟaŋ-a\cmn 我把皮子绷紧了\end{exemple}\end{entrée}

\begin{entrée}
\vedette{\hypertarget{Ⓔβʁa}{\papi{ βʁa}}}\markboth{βʁa}{}\classe{vi}
\paradigme{\textit{dir :} \jya pɯ-}
\begin{définition}\fra gagner\end{définition}
\begin{définition}\cmn 赢
\end{définition}
\begin{exemple}\jya tɤ-alɯlɤt-ndʑi tɕe, ɯʑo pɯ-βʁa\cmn 他们打架了,他赢了\end{exemple}
\begin{exemple}\jya nɤʑo kɤ-ɤnɯɣro tɕe pɯ-tɯ-βʁa\cmn 你赢了游戏\end{exemple}
\begin{relation-sémantique}\confer{
\hyperlink{Ⓔkɯβʁa}{\textit{ \papi{kɯβʁa}}}
}\end{relation-sémantique}
\begin{relation-sémantique}\confer{
\hyperlink{Ⓔtɤβʁa}{\textit{ \papi{tɤβʁa}}}
}\end{relation-sémantique}\begin{sous-entrée}
\vedette{\hypertarget{}{\papi{ sɯβʁa}}}\markboth{sɯβʁa}{}\classe{vt}
\paradigme{\textit{dir :} \jya pɯ-}
\begin{définition}\fra faire gagner\end{définition}
\begin{définition}\cmn 使……赢\end{définition}
\begin{exemple}\jya pɯ́-wɣ-sɯβʁa\cmn 他令我赢了\end{exemple}
\end{sous-entrée}\begin{sous-entrée}
\vedette{\hypertarget{}{\papi{ znɤβʁaβʁa}}}\markboth{znɤβʁaβʁa}{}\classe{vi}
\begin{définition}\fra arrogant\end{définition}
\begin{définition}\cmn 对人粗暴,霸道\end{définition}
\begin{exemple}\jya ma-tɯ-znɤβʁaβʁa\cmn 你不要对人你们粗暴\end{exemple}
\end{sous-entrée}\begin{sous-entrée}
\vedette{\hypertarget{}{\papi{ ʑɣɤsɯβʁa}}}\markboth{ʑɣɤsɯβʁa}{}\classe{vi}
\paradigme{\textit{dir :} \jya pɯ-}
\begin{définition}\fra faire en sorte de gagner\end{définition}
\begin{définition}\cmn 使自己赢\end{définition}
\end{sous-entrée}\end{entrée}

\begin{entrée}
\vedette{\hypertarget{Ⓔβʁuβʁu}{\papi{ βʁuβʁu}}}\markboth{βʁuβʁu}{}\classe{idph.2}\acception{1}
\begin{définition}\fra comme une demi-sphère creuse\end{définition}
\begin{définition}\cmn 形容空心的半球形(老鼠的耳朵)\end{définition}\acception{2}
\begin{définition}\fra tout petit\end{définition}
\begin{définition}\cmn 很小的样子\end{définition}\end{entrée}

\begin{entrée}
\vedette{\hypertarget{Ⓔβʁum}{\papi{ βʁum}}}\markboth{βʁum}{}
\classe{vt}
\paradigme{\textit{dir :} \jya pɯ-}
\paradigme{\textit{dir :} \jya \_}
\begin{définition}\fra renverser\end{définition}
\begin{définition}\cmn 口朝下盖\end{définition}
\begin{exemple}\jya khɯtsa pa-βʁum\cmn 他把碗口朝下\end{exemple}
\begin{exemple}\jya khɯtsa kɤ-βʁum mɤ-pe\cmn 把碗的口朝下是不好的\end{exemple}
\begin{exemple}\jya tɯthɯ kɤ-βʁum mɤ-pe\end{exemple}
\begin{relation-sémantique}\antonyme{
\hyperlink{Ⓔsɤntɯ}{\textit{ \papi{sɤntɯ}}}
}\end{relation-sémantique}\begin{sous-entrée}
\vedette{\hypertarget{}{\papi{ aβʁum}}}\markboth{aβʁum}{}
\begin{définition}\ 
\begin{déclaration}\grammar{pass}\end{déclaration}\end{définition}
\begin{définition}\fra être renversé\end{définition}
\begin{définition}\cmn 口朝下
\end{définition}
\begin{exemple}\jya ki khɯtsa ki aβʁum\cmn 这个碗口朝下\end{exemple}
\end{sous-entrée}\begin{sous-entrée}
\vedette{\hypertarget{}{\papi{ ʑɣɤβʁum}}}\markboth{ʑɣɤβʁum}{}\classe{vi}
\begin{définition}\ 
\begin{déclaration}\grammar{refl}\end{déclaration}\end{définition}
\begin{définition}\fra s'allonger sur le ventre\end{définition}
\begin{définition}\cmn 趴着(躺着)\end{définition}
\begin{exemple}\jya pjɯ-ʑɣɤβʁum ku-nɯ-rŋgɯ ɲɯ-ŋu\cmn 他趴着睡\end{exemple}
\end{sous-entrée}\end{entrée}

\begin{entrée}
\vedette{\hypertarget{Ⓔβʁɯz}{\papi{ βʁɯz}}}\markboth{βʁɯz}{}
\classe{n}
\begin{définition}\fra amadou\end{définition}
\begin{définition}\cmn 火绒\end{définition}\end{entrée}

\begin{entrée}
\vedette{\hypertarget{ⒺβzuⒽ1}{\papi{ βzu}}}\markboth{βzu}{}\homonyme{1}\classe{vt}\acception{1}
\paradigme{\textit{dir :} \jya tɤ-}
\begin{définition}\fra faire\end{définition}
\begin{définition}\cmn 做
\begin{déclaration} \étymologie{\papi{bzo}}\end{déclaration}\end{définition}
\begin{exemple}\jya tɤ-βzu-t-a, ta-βzu, tu-βze-a, tɤ-βze\cmn 我做了,他做了,我做,你做吧\end{exemple}
\begin{exemple}\jya tʂu nɯ-βzu-t-a\cmn 我让了路\end{exemple}
\begin{exemple}\jya pɕaʁ tɤ-βzu-t-a\cmn 我磕了头\end{exemple}
\begin{exemple}\jya rnajɯ pɯ-βzu-t-a\cmn 我做了耳环\end{exemple}
\begin{exemple}\jya qajɣi lɤ-βzu-t-a\cmn 我做了馍馍\end{exemple}
\begin{exemple}\jya sɤfkur thɯ-βzu-t-a\cmn 我捆了柴\end{exemple}
\begin{exemple}\jya ɕkɤbɯ kɤ-βzu-t-a\cmn 我包了韭菜包子\end{exemple}
\begin{exemple}\jya ki @jie a-tɤ-βze ra\cmn 你来接电话!\end{exemple}
\begin{exemple}\jya tɕhi kɯ-fse chɯ-βze-a\cmn 我要做什么样的东西?(例如,铁匠)\end{exemple}\acception{2}
\paradigme{\textit{dir :} \jya kɤ-}
\begin{définition}\fra devenir\end{définition}
\begin{définition}\cmn 当\end{définition}
\begin{exemple}\jya ɯ-slama kɤ-βzu-t-a\cmn 我当了他的徒弟\end{exemple}\acception{3}
\paradigme{\textit{dir :} \jya nɯ-}
\begin{définition}\fra devenir\end{définition}
\begin{définition}\cmn 变成\end{définition}
\begin{exemple}\jya ki kha ki aʑɯɣ ɲɯ-βze ɕti\cmn 这个房子会变成我的了\end{exemple}
\begin{exemple}\jya ki kha ki aʑɯɣ a-nɯ-βze ra\cmn 这个房子要变成我的\end{exemple}
\begin{exemple}\jya tɯjpu a-mɤ-nɯ-βɟi ra ma qajɯ ɲɯ-βze ŋgrɤl (=qajɯ aβzu)\cmn 粮食不要放太久,不然会生虫\end{exemple}\acception{4}
\begin{définition}\fra être possible (impersonnel)\end{définition}
\begin{définition}\cmn 可能(无人称)\end{définition}
\begin{exemple}\jya ʑɯmkhɤm ʑo tu-ndza-nɯ mɤɕtʂa kɤ-mqlaʁ mɯ́j-βze.\cmn (这种草)牛要咀嚼很久才能吞下\end{exemple}
\begin{relation-sémantique}\confer{
 \papi{tɯ-ndzɯ,βzu}
}\end{relation-sémantique}
\begin{relation-sémantique}\confer{
\hyperlink{Ⓔβdaʁ,βzu}{\textit{ \papi{βdaʁ,βzu}}}
}\end{relation-sémantique}
\begin{relation-sémantique}\confer{
 \papi{tɯ-skɤt,βzu}
}\end{relation-sémantique}
\begin{relation-sémantique}\confer{
 \papi{ɯ-qhu,βzu}
}\end{relation-sémantique}
\begin{relation-sémantique}\confer{
 \papi{ɯ-sci,βzu}
}\end{relation-sémantique}
\begin{relation-sémantique}\confer{
\hyperlink{Ⓔaβzu}{\textit{ \papi{aβzu}}}
}\end{relation-sémantique}\begin{sous-entrée}
\vedette{\hypertarget{}{\papi{ nɤβzɯβzu}}}\markboth{nɤβzɯβzu}{}\classe{vt}
\begin{définition}\fra aller faire à des endroits différents\end{définition}
\begin{définition}\cmn 到处去做\end{définition}
\begin{exemple}\jya ɯʑo kɯ tɯtsɣe ɲɯ-ɤz-nɤβzɯβzu\cmn 他到处去做生意\end{exemple}
\end{sous-entrée}\begin{sous-entrée}
\vedette{\hypertarget{}{\papi{ nɯβzu}}}\markboth{nɯβzu}{}\classe{vt}
\begin{définition}\ 
\begin{déclaration}\grammar{autoben}\end{déclaration}\end{définition}
\begin{exemple}\jya nɤʑo tɤ-tɯ-nɯ-βzu-t ɕti\cmn 这是你自己造成的(是你一个人的错)\end{exemple}
\end{sous-entrée}\begin{sous-entrée}
\vedette{\hypertarget{}{\papi{ sɯβzu}}}\markboth{sɯβzu}{}\classe{vt}
\begin{définition}\fra faire faire\end{définition}
\begin{définition}\cmn 使别人做\end{définition}
\begin{exemple}\jya @zuoye tɯ-sɯβze ra ?\cmn 你是不是让他们做作业?\end{exemple}
\end{sous-entrée}\end{entrée}

\begin{entrée}
\vedette{\hypertarget{Ⓔβzaŋɤnŋu}{\papi{ βzaŋɤnŋu}}}\markboth{βzaŋɤnŋu}{}\classe{n}
\begin{définition}\fra bien et mal\end{définition}
\begin{définition}\cmn 好坏
\begin{déclaration} \étymologie{\papi{bzaŋ.ŋan}}\end{déclaration}\end{définition}
\begin{exemple}\jya ɯ-βzaŋɤnŋu kɯ-me\cmn 忘恩负义的人\end{exemple}\end{entrée}

\begin{entrée}
\vedette{\hypertarget{Ⓔβzaŋlɤn}{\papi{ βzaŋlɤn}}}\markboth{βzaŋlɤn}{}\classe{n}
\begin{définition}\fra récompense\end{définition}
\begin{définition}\cmn 奖赏
\begin{déclaration} \étymologie{\papi{bzaŋ.len}}\end{déclaration}\end{définition}
\begin{exemple}\jya nɤ-βzaŋlɤn βze-a ra ma a-tɕɯ ɯ-sroʁ ko-tɯ-ri\cmn 我要报你的恩,因为你救了我儿子的命\end{exemple}\end{entrée}

\begin{entrée}
\vedette{\hypertarget{Ⓔβzaŋsa}{\papi{ βzaŋsa}}}\markboth{βzaŋsa}{}\classe{n}
\begin{définition}\fra ami\end{définition}
\begin{définition}\cmn 朋友
\begin{déclaration} \étymologie{\papi{bzaŋ.sa}}\end{déclaration}\end{définition}
\begin{exemple}\jya ɯʑo kɯ ɯ-βzaŋsa ɯ-tshɤt tú-wɣ-sɯβzu-a ŋu\cmn 他把我当朋友\end{exemple}\end{entrée}

\begin{entrée}
\vedette{\hypertarget{Ⓔβzaʁlu}{\papi{ βzaʁlu}}}\markboth{βzaʁlu}{}\classe{n}
\begin{définition}\fra personne qui agit ou parle sans tenir compte de la situation\end{définition}
\begin{définition}\cmn 说话不严谨的人\end{définition}
\begin{relation-sémantique}\confer{
\hyperlink{Ⓔɣɤβzaʁlaʁ}{\textit{ \papi{ɣɤβzaʁlaʁ}}}
}\end{relation-sémantique}\end{entrée}

\begin{entrée}
\vedette{\hypertarget{Ⓔβzdɤr}{\papi{ βzdɤr}}}\markboth{βzdɤr}{}\classe{vt}
\paradigme{\textit{dir :} \jya nɯ-}
\begin{définition}\fra ajouter de l'huile ou du beurre\end{définition}
\begin{définition}\cmn 加油,加酥油
\begin{déclaration} \étymologie{\papi{sdor}}\end{déclaration}\end{définition}
\begin{exemple}\jya tʂha ci nɯ-βzdar-a\cmn 我在茶里加了酥油\end{exemple}
\begin{exemple}\jya tʂha pɯ-kɤ-βzdɤr\cmn 酥油茶\end{exemple}
\begin{relation-sémantique}\confer{
\hyperlink{Ⓔtɤ-βzdɤr}{\textit{ \papi{tɤ-βzdɤr}}}
}\end{relation-sémantique}\end{entrée}

\begin{entrée}
\vedette{\hypertarget{Ⓔβzdɯ}{\papi{ βzdɯ}}}\markboth{βzdɯ}{}
\classe{vt}
\paradigme{\textit{dir :} \jya tɤ-}
\begin{définition}\fra ramasser\end{définition}
\begin{définition}\cmn 捡起来
\begin{déclaration} \étymologie{\papi{bsdu}}\end{déclaration}\end{définition}
\begin{exemple}\jya tɤ-βzdɯ-t-a\cmn 我捡了\end{exemple}
\begin{exemple}\jya stoʁ tɤ-βzdi\cmn 把胡豆捡了起来\end{exemple}
\begin{exemple}\jya stoʁ pjɤ-ʁndɤr tɕe tɤ-βzdɯ-t-a\cmn 胡豆撒了一地,我就捡起来了\end{exemple}\begin{sous-entrée}
\vedette{\hypertarget{}{\papi{ aβzdoʁβzdɯ}}}\markboth{aβzdoʁβzdɯ}{}\classe{vi}
\paradigme{\textit{dir :} \jya tɤ-}
\paradigme{\textit{dir :} \jya thɯ-}
\begin{définition}\ 
\begin{déclaration}\grammar{pass}\end{déclaration}\end{définition}
\begin{définition}\fra être en ordre\end{définition}
\begin{définition}\cmn 摆得整齐\end{définition}
\begin{exemple}\jya laχtɕha ra aβzdoʁβzdɯ ɕti\cmn 东西放得很整齐\end{exemple}
\end{sous-entrée}\begin{sous-entrée}
\vedette{\hypertarget{}{\papi{ sɤβzdoʁβzdɯ}}}\markboth{sɤβzdoʁβzdɯ}{}
\paradigme{\textit{dir :} \jya tɤ-}
\begin{définition}\fra mettre en ordre\end{définition}
\begin{définition}\cmn 整理,收拾\end{définition}
\begin{exemple}\jya ɯʑo kɯ laχtɕha ra ta-sɤβzdoʁβzdɯ\cmn 他把东西收拾好了\end{exemple}
\begin{relation-sémantique}\synonyme{
\hyperlink{Ⓔrɤwum}{\textit{ \papi{rɤwum}}}
}\end{relation-sémantique}
\end{sous-entrée}\end{entrée}

\begin{entrée}
\vedette{\hypertarget{Ⓔβzgɤr}{\papi{ βzgɤr}}}\markboth{βzgɤr}{}\classe{vt}
\paradigme{\textit{dir :} \jya pɯ-}
\begin{définition}\fra retarder le temps\end{définition}
\begin{définition}\cmn 耽误时间\end{définition}
\begin{exemple}\jya pjɯ-ta-βzgɤr mɯ́j-pe\cmn 我耽误你的时间很不好\end{exemple}
\begin{exemple}\jya tɯrme nɯ ma-pɯ-tɯ-βzgɤr ma ɯ-ʁa me\cmn 你不要耽误他,他没有时间\end{exemple}
\begin{relation-sémantique}\confer{
\hyperlink{Ⓔsaʁjɤr}{\textit{ \papi{saʁjɤr}}}
}\end{relation-sémantique}\end{entrée}

\begin{entrée}
\vedette{\hypertarget{Ⓔβzi}{\papi{ βzi}}}\markboth{βzi}{}\classe{vi}
\paradigme{\textit{dir :} \jya lɤ-}
\begin{définition}\fra devenir saoul\end{définition}
\begin{définition}\cmn 醉
\begin{déclaration} \étymologie{\papi{bzi}}\end{déclaration}\end{définition}
\begin{exemple}\jya ɯʑo lɤ-βzi, ɯʑo lo-βzi\cmn 他喝醉了\end{exemple}
\begin{exemple}\jya aʑo a-ku lɤ-kɯ-βzi ʑo ɲɯ-fse\cmn 我头有点晕\end{exemple}
\begin{exemple}\jya cha ɯ-tshɤt kɤ-tshi ma tɯ-βzi\cmn 酒少喝一些,不然你会喝醉\end{exemple}
\begin{exemple}\jya nɤʑo ɯ-mɤ-lɤ-tɯ-βzi-ci\cmn 你喝醉了吧\end{exemple}\begin{sous-entrée}
\vedette{\hypertarget{}{\papi{ ɣɤβzi}}}\markboth{ɣɤβzi}{}\classe{vs}
\begin{définition}\ 
\begin{déclaration}\grammar{facil}\end{déclaration}\end{définition}
\begin{définition}\fra être facilement saoul\end{définition}
\begin{définition}\cmn 容易醉\end{définition}
\end{sous-entrée}\begin{sous-entrée}
\vedette{\hypertarget{}{\papi{ sɤβzi}}}\markboth{sɤβzi}{} (\variante{sɤsɯβzi}) \classe{vs}
\begin{définition}\ 
\begin{déclaration}\grammar{deexp}\end{déclaration}\end{définition}
\begin{définition}\fra qui monte facilement à la tête\end{définition}
\begin{définition}\cmn 容易上头的\end{définition}
\begin{exemple}\jya cha kɯ-sɤβzi\cmn 容易上头的酒\end{exemple}
\end{sous-entrée}\begin{sous-entrée}
\vedette{\hypertarget{}{\papi{ sɯβzi}}}\markboth{sɯβzi}{}\classe{vt}
\paradigme{\textit{dir :} \jya lɤ-}
\begin{définition}\ 
\begin{déclaration}\grammar{caus}\end{déclaration}\end{définition}
\begin{définition}\fra saouler\end{définition}
\begin{définition}\cmn 灌醉\end{définition}
\end{sous-entrée}\begin{sous-entrée}
\vedette{\hypertarget{}{\papi{ ʑɣɤsɯβzi}}}\markboth{ʑɣɤsɯβzi}{}\classe{vi}
\begin{définition}\ 
\begin{déclaration}\grammar{refl}\end{déclaration}
\begin{déclaration}\grammar{caus}\end{déclaration}\end{définition}
\begin{définition}\fra se rendre saoul\end{définition}
\begin{définition}\cmn 喝醉\end{définition}
\begin{exemple}\jya kɤ-ʑɣɤsɯβzi mɤ-ra\cmn 别喝醉\end{exemple}
\end{sous-entrée}\end{entrée}

\begin{entrée}
\vedette{\hypertarget{Ⓔβzjoz}{\papi{ βzjoz}}}\markboth{βzjoz}{}
\classe{vt}
\paradigme{\textit{dir :} \jya pɯ-}
\paradigme{\textit{dir :} \jya kɤ-}
\begin{définition}\fra étudier\end{définition}
\begin{définition}\cmn 学
\begin{déclaration} \étymologie{\papi{sbʲaŋs}}\end{déclaration}\end{définition}
\begin{exemple}\jya tɤ-scoz kɤ-βzjoz ɴqa\cmn 文字很难学\end{exemple}
\begin{exemple}\jya aʑo nɤ-ɕki a-kɤ-βzjoz dɤn\cmn 我跟你学了很多\end{exemple}
\begin{exemple}\jya kɤ-βzjoz kɤ-sthɯt mɯ́j-khɯ\cmn 学无止境\end{exemple}
\begin{relation-sémantique}\confer{
\hyperlink{Ⓔrɤβzjoz}{\textit{ \papi{rɤβzjoz}}}
}\end{relation-sémantique}\begin{sous-entrée}
\vedette{\hypertarget{}{\papi{ nɯɣɯβzjoz}}}\markboth{nɯɣɯβzjoz}{}\classe{vs}
\begin{définition}\fra facile à apprendre\end{définition}
\begin{définition}\cmn 好学,容易学\end{définition}
\end{sous-entrée}\end{entrée}

\begin{entrée}
\vedette{\hypertarget{Ⓔβzɟɯr}{\papi{ βzɟɯr}}}\markboth{βzɟɯr}{}
\classe{vt}
\paradigme{\textit{dir :} \jya nɯ-}\acception{1}
\begin{définition}\fra transformer\end{définition}
\begin{définition}\cmn 改变
\begin{déclaration} \étymologie{\papi{bsgʲur}}\end{déclaration}\end{définition}
\begin{exemple}\jya nɯ-βzɟɯr-a, na-βzɟɯr\cmn 我改变了,他改变了\end{exemple}
\begin{exemple}\jya mɤ-kɯ-pe tɤ-kɤ-nɤma nɯra ɯ-qhu tɕe ɲɯ́-wɣ-βzɟɯr ra\cmn 没有做好的那些以后要修正\end{exemple}
\begin{exemple}\jya ɯ-ɲɯ-nɯkɯmaʁ-a nɤ ɲɯ-kɯ-sɯ-βzɟɯr-a\cmn 我说错了的话,请帮我纠正一下\end{exemple}
\begin{exemple}\jya ʑara kɯ kɯm nɯ na-βzɟɯr-nɯ\cmn 他们换了门(工人修房子)\end{exemple}\acception{2}
\begin{définition}\fra corriger\end{définition}
\begin{définition}\cmn 纠正\end{définition}
\begin{exemple}\jya kɤ-βzɟɯr mɯ́j-ra, ɲɯ-pe\cmn 不用修改,很好\end{exemple}
\begin{exemple}\jya tɤ-scoz ɲɤ-βzɟɯr\cmn 他把信改了改\end{exemple}\begin{sous-entrée}
\vedette{\hypertarget{}{\papi{ ʑɣɤβzɟɯr}}}\markboth{ʑɣɤβzɟɯr}{}\classe{vi}
\paradigme{\textit{dir :} \jya nɯ-}
\begin{définition}\ 
\begin{déclaration}\grammar{refl}\end{déclaration}\end{définition}
\begin{définition}\fra se transformer\end{définition}
\begin{définition}\cmn 转变\end{définition}
\end{sous-entrée}\end{entrée}

\begin{entrée}
\vedette{\hypertarget{Ⓔβzɯr}{\papi{ βzɯr}}}\markboth{βzɯr}{}\classe{vt}
\paradigme{\textit{dir :} \jya tɤ-}
\begin{définition}\fra déplacer\end{définition}
\begin{définition}\cmn 拿过去;搬走
\begin{déclaration} \étymologie{\papi{bzur}}\end{déclaration}\end{définition}
\begin{exemple}\jya tɤ-βzɯr-a, ta-βzɯr\cmn 我拿走了,他拿走了\end{exemple}
\begin{exemple}\jya ɲɯ-saʁdɯɣ tɕe, tɤ-βzɯr-a\cmn (那个东西)很碍事,所以我把它拿走了\end{exemple}
\begin{exemple}\jya kɯki kutɕu a-mɤ-pɯ-ɤta, tɤ-βzɯr\cmn 这个东西不要放在这里,你把它拿走\end{exemple}
\end{entrée}

\begin{entrée}
\vedette{\hypertarget{Ⓔβzɯrtɕoʁ}{\papi{ βzɯrtɕoʁ}}}\markboth{βzɯrtɕoʁ}{}\classe{n}
\begin{définition}\fra excroissance sur les angles du toit\end{définition}
\begin{définition}\cmn 房背左右两角上的顶端
\begin{déclaration} \étymologie{\papi{bzur.ltɕog}}\end{déclaration}\end{définition}
\begin{exemple}\jya kha ɣɯ znde kɤ-βzu tɤ-jɤɣ tɕe ɯ-ʁɤri pɕoʁ χchoʁe tɯ-βzɯr ɯ-taʁ tu-kɤ-sɤmtɕoʁ nɯ βzɯrtɕoʁ rmi\cmn 
房子的墙壁修完了的时候,在房子正面左右两角顶上修的尖角部分叫\stylefv{βzɯrtɕoʁ}
\end{exemple}\end{entrée}

\begin{entrée}
\vedette{\hypertarget{Ⓔβzuwa}{\papi{ βzuwa}}}\markboth{βzuwa}{}\classe{n}
\begin{définition}\fra artisan\end{définition}
\begin{définition}\cmn 工匠
\begin{déclaration} \étymologie{\papi{bzo.ba}}\end{déclaration}\end{définition}\end{entrée}

\begin{entrée}
\vedette{\hypertarget{Ⓔβʑar}{\papi{ βʑar}}}\markboth{βʑar}{}
\classe{n}
\begin{définition}\fra busard\end{définition}
\begin{définition}\cmn 鵟,鹞\end{définition}\end{entrée}

\begin{entrée}
\vedette{\hypertarget{Ⓔβʑaʁβʑɯɣ}{\papi{ βʑaʁβʑɯɣ}}}\markboth{βʑaʁβʑɯɣ}{}\classe{n}
\begin{définition}\fra résidence\end{définition}
\begin{définition}\cmn 住宿
\begin{déclaration}\use{敬语}\end{déclaration}
\begin{déclaration} \étymologie{\papi{bʑag.bʑugs}}\end{déclaration}\end{définition}
\begin{exemple}\jya βlama ra kɯ kutɕu βʑaʁβʑɯɣ a-kɤ-nɯ-βzu-nɯ jɤɣ\cmn 喇嘛可以在这里就寝\end{exemple}\end{entrée}

\begin{entrée}
\vedette{\hypertarget{Ⓔβʑɤzu}{\papi{ βʑɤzu}}}\markboth{βʑɤzu}{}
\classe{n}
\begin{définition}\fra seau à lait\end{définition}
\begin{définition}\cmn 挤奶桶
\begin{déclaration} \étymologie{\papi{bʑo.zo}}\end{déclaration}\end{définition}\end{entrée}

\begin{entrée}
\vedette{\hypertarget{Ⓔβʑoʁ}{\papi{ βʑoʁ}}}\markboth{βʑoʁ}{}\classe{vt}
\paradigme{\textit{dir :} \jya thɯ-}
\begin{définition}\fra tailler, éplucher\end{définition}
\begin{définition}\cmn 削
\begin{déclaration} \étymologie{\papi{bʑogs}}\end{déclaration}\end{définition}
\begin{exemple}\jya chɯ-βʑoʁ-a, tha-βʑoʁ\cmn 我削,他削了\end{exemple}
\begin{exemple}\jya kɯki laʁdɯn mɯ́j-sna tɕe, chɯ́-wɣ-βʑoʁ ɲɯ-ra\cmn 这个工具不好,要削一下\end{exemple}
\begin{exemple}\jya ɲɯ-jpum tɕe chɯ́-wɣ-βʑoʁ ɲɯ-ra\cmn 太粗,要削一下\end{exemple}
\begin{exemple}\jya sɲɯɣjɯ thɯ-βʑoʁ-a\cmn 我削了笔\end{exemple}\begin{sous-entrée}
\vedette{\hypertarget{}{\papi{ sɯβʑoʁ}}}\markboth{sɯβʑoʁ}{}\acception{1}
\begin{définition}\fra tailler avec\end{définition}
\begin{définition}\cmn 用……削\end{définition}\acception{2}
\begin{définition}\fra pouvoir tailler\end{définition}
\begin{définition}\cmn 削得了\end{définition}
\begin{exemple}\jya mbrɯtɕɯ ɯʑo kɯ ɯʑo ɯ-jɯ mɤ-sɯβʑoʁ\cmn 刀子削不了自己的把子\end{exemple}
\end{sous-entrée}\end{entrée}

\begin{entrée}
\vedette{\hypertarget{Ⓔβʑɯ}{\papi{ βʑɯ}}}\markboth{βʑɯ}{}
\classe{n}
\begin{définition}\fra souris\end{définition}
\begin{définition}\cmn 老鼠\end{définition}\end{entrée}

\begin{entrée}
\vedette{\hypertarget{Ⓔβʑɯlu}{\papi{ βʑɯlu}}}\markboth{βʑɯlu}{}\classe{n}
\begin{définition}\fra année du rat\end{définition}
\begin{définition}\cmn 鼠年\end{définition}
\end{entrée}

\begin{entrée}
\vedette{\hypertarget{Ⓔβʑɯndɤpa}{\papi{ βʑɯndɤpa}}}\markboth{βʑɯndɤpa}{}\classe{n}
\begin{définition}\fra dans cinq ans\end{définition}
\begin{définition}\cmn 五年以后\end{définition}
\begin{relation-sémantique}\confer{
\hyperlink{ⒺβʑɯndiⒽ1}{\textit{ \papi{βʑɯndi1}}}
}\end{relation-sémantique}
\begin{relation-sémantique}\confer{
\hyperlink{Ⓔtɯ-xpa}{\textit{ \papi{tɯ-xpa}}}
}\end{relation-sémantique}
\end{entrée}

\begin{entrée}
\vedette{\hypertarget{ⒺβʑɯndiⒽ1}{\papi{ βʑɯndi}}}\markboth{βʑɯndi}{}\homonyme{1}
\classe{n}
\begin{définition}\fra dans cinq jours\end{définition}
\begin{définition}\cmn 五天以后
\begin{déclaration} \étymologie{\papi{bʑi}}\end{déclaration}\end{définition}\end{entrée}

\begin{entrée}
\vedette{\hypertarget{ⒺβʑɯndiⒽ2}{\papi{ βʑɯndi}}}\markboth{βʑɯndi}{}\homonyme{2}\classe{n}
\begin{définition}\fra bande molletière\end{définition}
\begin{définition}\cmn 裹腿
\begin{déclaration}\use{汉语借词“裹脚”\stylefv{koco}也可以使用}\end{déclaration}\end{définition}
\begin{exemple}\jya βʑɯndi nɯ-nɯrtaβ-a\cmn 我缠了裹腿\end{exemple}\end{entrée}

\begin{entrée}
\vedette{\hypertarget{Ⓔβʑɯpa}{\papi{ βʑɯpa}}}\markboth{βʑɯpa}{}\classe{n}
\begin{définition}\fra quatrième mois\end{définition}
\begin{définition}\cmn 四月
\begin{déclaration} \étymologie{\papi{bʑi.pa}}\end{déclaration}\end{définition}\end{entrée}

\begin{entrée}
\vedette{\hypertarget{Ⓔβʑɯrna}{\papi{ βʑɯrna}}}\markboth{βʑɯrna}{}\classe{n}
\begin{définition}\fra oreilles de souris\end{définition}
\begin{définition}\cmn 老鼠的耳朵\end{définition}
\begin{exemple}\jya βʑɯrna ɲɤ-sprɤt\cmn 植物脱落了子叶,长出了真正的叶子\end{exemple}\end{entrée}

\newpage\caractère{c}

\begin{entrée}
\vedette{\hypertarget{Ⓔcu}{\papi{ cu}}}\markboth{cu}{}\classe{vt}\paradigme{\textit{dir :} \jya kɤ-}
\paradigme{\textit{dir :} \jya nɯ-}
\begin{définition}\fra ajouter des ingrédients\end{définition}
\begin{définition}\cmn 另加配菜\end{définition}
\begin{exemple}\jya ka-cu\cmn 他加了\end{exemple}
\begin{exemple}\jya @cai ɯ-ŋgɯ tɕe ɕku kú-wɣ-cu tɕe mɯm\cmn 在菜里放点葱就好吃\end{exemple}
\begin{exemple}\jya @yangyu ɯ-ŋgɯ tɤjko ci kɤ-ce\cmn 你在土豆里面加一点酸菜\end{exemple}\begin{sous-entrée}
\vedette{\hypertarget{}{\papi{ acu}}}\markboth{acu}{}\classe{vs}
\begin{définition}\ 
\begin{déclaration}\grammar{pass}\end{déclaration}\end{définition}
\begin{définition}\fra être mélangé dans\end{définition}
\begin{définition}\cmn 掺在一起;和在一起\end{définition}
\begin{exemple}\jya rɟɤɣi ɯ-ŋgɯ ta-mar a-pɯ-ɤcu tɕe mɯm\cmn 在糌粑里加一点酥油就好吃\end{exemple}
\begin{relation-sémantique}\synonyme{
\hyperlink{ⒺaɕiⒽ1}{\textit{ \papi{aɕi}}}
}\end{relation-sémantique}
\end{sous-entrée}\begin{sous-entrée}
\vedette{\hypertarget{}{\papi{ nɤcu}}}\markboth{nɤcu}{}\classe{vs}
\begin{définition}\fra bien s'entendre avec\end{définition}
\begin{définition}\cmn 合得来\end{définition}
\begin{exemple}\jya tɯrme ra nɯ-rca ɲɯ-tɯ-nɤcu\cmn 你跟那些人合得来\end{exemple}
\begin{relation-sémantique}\confer{
\hyperlink{Ⓔnɯrɯcu}{\textit{ \papi{nɯrɯcu}}}
}\end{relation-sémantique}
\end{sous-entrée}\end{entrée}

\begin{entrée}
\vedette{\hypertarget{Ⓔca}{\papi{ ca}}}\markboth{ca}{}
\classe{n}
\begin{définition}\fra chevrotin\end{définition}
\begin{définition}\cmn 麝香鹿\end{définition}
\begin{exemple}\jya ca nɯ sɯŋgɯ kɯ-rnaʁ tsa cho stɤmku nɯ ra ku-rɤʑi ɲɯ-ŋu. sɯjno ma mɯ́j-ndze. ɯ-ku nɯ tshɤt ɯ-ku tsa ɲɯ-fse ri ɯ-ʁrɯ maŋe, ɯ-rna nɯ ra tshɤt wuma ɲɯ-fse, ɯ-phoŋbu ɯ-tshɯɣa nɯ ra li tshɤt ɲɯ-fse, ɯ-mɤlɤjaʁ nɯ ra tshɤt ɣɯ sɤznɤ ɲɯ-rɲɟi, ɯ-phoŋbu nɯ ɲɯ-xtshɯm, ɯ-mdoʁ nɯ ɲɯ-pɣi. ɯ-qa nɯ ta-ʁrɯ ɲɯ-ŋu, ɯ-jme kɯ-xtɯ-xtɯt ŋu. ɯ-jme ɯ-pa cho ɯ-xtɤpa lu-kɯ-ɕe nɯ ra ɲɯ-wɣrum. ɲɯ-ɣɤwu tɕe, tshɯtho ɯ-skɤt tu-βze ɲɯ-ŋu. ɯ-ɕa ɯ-ŋgɯ ɯ-tʂɤm me, ɯ-ɕa ʁɟa ŋu, ɯ-ndʐi nɯ ɯ-χtsɤβ a-pɯ-βdi tɕe mba ri wuma ʑo ngɯt, tɕe kɯɕɯŋgɯ tɕe tɯ-xtsa ɯ-ku spa stu kɯ-pe pjɤ-ŋu. ca phu mu tu tɕe, cɤmu nɯ kɯ-rɤpɯ ɯ-spa ɲɯ-ŋu, cɤmtsho nɯ ɯ-mtsho tu tɕe, nɯ wuma ʑo smɤn kɯ-ʑru ɲɯ-ŋu, tɕe ɯ-phɯ wuma ʑo wxti.\cmn 麝香鹿住在大森林和草地上。只吃草。头有点像山羊,但是没有角,耳朵很像山羊的耳朵,四肢比山羊的长一些,身子细,颜色是灰的。有蹄子,尾巴很短。尾巴下面和肚皮都是白色的。叫的时候发出小山羊的叫声。肉里没有脂肪,只有瘦肉。皮子搓揉好了以后,虽然薄但是很结实。在过去,是藏式皮鞋筒最好的材料。麝香鹿有公母,母鹿下崽子,公鹿有麝香,是一种很名贵的药材,价格很高。\end{exemple}\end{entrée}

\begin{entrée}
\vedette{\hypertarget{Ⓔcaŋ}{\papi{ caŋ}}}\markboth{caŋ}{}
\classe{n}
\begin{définition}\fra mur en terre\end{définition}
\begin{définition}\cmn 土墙
\begin{déclaration} \étymologie{\papi{gʲaŋ}}\end{déclaration}\end{définition}\end{entrée}

\begin{entrée}
\vedette{\hypertarget{Ⓔcapɣi}{\papi{ capɣi}}}\markboth{capɣi}{}\classe{n}
\begin{définition}\fra cerf (moschus sifanicus)\end{définition}
\begin{définition}\cmn 马麂\end{définition}
\end{entrée}

\begin{entrée}
\vedette{\hypertarget{Ⓔcaʁ}{\papi{ caʁ}}}\markboth{caʁ}{}
\classe{vs}
\begin{définition}\fra célèbre\end{définition}
\begin{définition}\cmn 出名\end{définition}
\begin{exemple}\jya tɤ-caʁ-a\cmn 我出名了\end{exemple}
\begin{exemple}\jya wuma ʑo pjɤ-cha tɕe, ɯ-rmi to-caʁ\cmn 他很能干,所以就变得很出名\end{exemple}\end{entrée}

\begin{entrée}
\vedette{\hypertarget{Ⓔcaʁɕɣɤz}{\papi{ caʁɕɣɤz}}}\markboth{caʁɕɣɤz}{}\classe{n}
\begin{définition}\fra laine épaisse et fragile\end{définition}
\begin{définition}\cmn 又粗又脆的劣质羊毛\end{définition}\end{entrée}

\begin{entrée}
\vedette{\hypertarget{Ⓔcɤɕna}{\papi{ cɤɕna}}}\markboth{cɤɕna}{}
\classe{n}
\begin{définition}\fra Rumex japonicus\end{définition}
\begin{définition}\cmn 山菠菜;羊蹄\end{définition}
\begin{exemple}\jya cɤɕna nɯ sɯjno ci ŋu, sɯŋgɯ cho stɤmku ra tɯ-ji ɯ-rkɯ ra tu-ɬoʁ ŋgrɤl. ɯ-jwaʁ nɯ aɲaʁndzɯm, ɯ-ru aɣɯrnɯɕɯr, kɯ-ɤrŋi tɕe tu, ɯ-mat tu-βze ɕɯŋgɯ nɤ, kɤ-ndza sna. ɯ-mɯntoʁ mɤ-mpɕɤr, ndɯβ, kɯ-ɣɯrni tsa ŋu. ɯ-mat thɯ-aβzu tɕe, pjɯ-ŋgra tɕe, tu-ɬoʁ mɤ-cha. ɯʑo ɯ-kɯ-sɯ-mphɤl nɯ ɯ-zrɤm ɲɯ-ɕti. tɯ-ɟom jamar ma tɯ-zri mɤ-cha. ɯ-jwaʁ ɯ-qhu nɯ mpɕu, ɯ-βzɯr nɯ ra rʁom.\cmn 山波菜是一种植物,一般生长在森林、草地和地边。叶子深绿色,茎淡红色,也有的是绿色的,结果之前可以吃。花不美,小,有点红。结了果以后,种子掉下而不能生长:使山波菜繁殖的是它的根。只能长一米来高。叶子背面是光滑的,边角有点粗糙。\end{exemple}\end{entrée}

\begin{entrée}
\vedette{\hypertarget{Ⓔcɤjmɤɣ}{\papi{ cɤjmɤɣ}}}\markboth{cɤjmɤɣ}{}\classe{n}
\begin{définition}\fra une espèce de champignon\end{définition}
\begin{définition}\cmn 【獐子菌】\end{définition}
\begin{exemple}\jya cɤjmɤɣ nɯ tɯrgi ɯ-ŋgɯ tu-ɬoʁ ɲɯ-ŋu, ɯ-mdoʁ nɯ kɯ-ɤpɣɯlu ʁɟa ʑo nɯ ŋu, ɯ-rʑɯɣ nɯ ca ɯ-rme ɲɯ-fse, kɯ-wxti tsa ci ɲɯ-ŋu, ɯ-mdoʁ cho ɯ-rʑɯɣ nɯ ra ndʐa cɤjmɤɣ ɲɯ-rmi\cmn 獐子菌生长在杉木林里,通体灰色,菌褶像麝香鹿(獐子)的毛,有点大,因为它的颜色和菌褶像獐子,所以称它为“獐子菌”。\end{exemple}
\begin{relation-sémantique}\confer{
\hyperlink{Ⓔca}{\textit{ \papi{ca}}}
}\end{relation-sémantique}
\begin{relation-sémantique}\confer{
\hyperlink{Ⓔtɤjmɤɣ}{\textit{ \papi{tɤjmɤɣ}}}
}\end{relation-sémantique}\end{entrée}

\begin{entrée}
\vedette{\hypertarget{Ⓔcɤmu}{\papi{ cɤmu}}}\markboth{cɤmu}{}\classe{n}
\begin{définition}\fra chevrotain femelle\end{définition}
\begin{définition}\cmn 母麝香鹿\end{définition}
\begin{relation-sémantique}\confer{
\hyperlink{Ⓔca}{\textit{ \papi{ca}}}
}\end{relation-sémantique}\end{entrée}

\begin{entrée}
\vedette{\hypertarget{Ⓔcɤmi}{\papi{ cɤmi}}}\markboth{cɤmi}{}\classe{n}
\begin{définition}\fra près du fleuve\end{définition}
\begin{définition}\cmn 沿着河流的地方\end{définition}\end{entrée}

\begin{entrée}
\vedette{\hypertarget{Ⓔcɤmirɤku}{\papi{ cɤmirɤku}}}\markboth{cɤmirɤku}{}\classe{n}
\begin{définition}\fra cultures de vallée\end{définition}
\begin{définition}\cmn 河坝农物(玉米)\end{définition}
\begin{relation-sémantique}\confer{
\hyperlink{Ⓔcɤmi}{\textit{ \papi{cɤmi}}}
}\end{relation-sémantique}
\begin{relation-sémantique}\confer{
\hyperlink{Ⓔtɤ-rɤku}{\textit{ \papi{tɤ-rɤku}}}
}\end{relation-sémantique}\end{entrée}

\begin{entrée}
\vedette{\hypertarget{Ⓔcɤmtsaʁ}{\papi{ cɤmtsaʁ}}}\markboth{cɤmtsaʁ}{}\classe{n}
\begin{définition}\fra ruade\end{définition}
\begin{définition}\cmn 尥蹶子\end{définition}
\begin{exemple}\jya mbro kɯ cɤmtsaʁ ja-lɤt tɕe pɯ́-wɣ-βde-a pɯ-ŋgrɤl\cmn 我尥蹶子把我扔下来了\end{exemple}
\begin{relation-sémantique}\synonyme{
\hyperlink{Ⓔnɯsɲɤtqha}{\textit{ \papi{nɯsɲɤtqha}}}
}\end{relation-sémantique}
\begin{relation-sémantique}\confer{
\hyperlink{Ⓔmtsaʁ}{\textit{ \papi{mtsaʁ}}}
}\end{relation-sémantique}\end{entrée}

\begin{entrée}
\vedette{\hypertarget{Ⓔcɤmtsho}{\papi{ cɤmtsho}}}\markboth{cɤmtsho}{}\classe{n}
\begin{définition}\fra chevrotain porte-musc mâle\end{définition}
\begin{définition}\cmn 公麝\end{définition}
\begin{relation-sémantique}\confer{
\hyperlink{Ⓔca}{\textit{ \papi{ca}}}
}\end{relation-sémantique}\end{entrée}

\begin{entrée}
\vedette{\hypertarget{Ⓔcɤndʐi}{\papi{ cɤndʐi}}}\markboth{cɤndʐi}{}\classe{n}
\begin{définition}\fra peau de chevrotain\end{définition}
\begin{définition}\cmn 麝香鹿皮\end{définition}
\begin{relation-sémantique}\confer{
\hyperlink{Ⓔca}{\textit{ \papi{ca}}}
}\end{relation-sémantique}
\begin{relation-sémantique}\confer{
\hyperlink{Ⓔtɯ-ndʐi}{\textit{ \papi{tɯ-ndʐi}}}
}\end{relation-sémantique}\end{entrée}

\begin{entrée}
\vedette{\hypertarget{Ⓔcɤpɤcrɤle}{\papi{ cɤpɤcrɤle}}}\markboth{cɤpɤcrɤle}{}\classe{n}
\begin{définition}\fra nourriture de mauvaise qualité\end{définition}
\begin{définition}\cmn 低等(的食物),素食\end{définition}
\begin{exemple}\jya cɤpɤcrɤle ɲɯ-ta-mbi\cmn 我给你吃得很素\end{exemple}
\begin{relation-sémantique}\confer{
\hyperlink{Ⓔldʐɤpɤldʐɤle}{\textit{ \papi{ldʐɤpɤldʐɤle}}}
}\end{relation-sémantique}\end{entrée}

\begin{entrée}
\vedette{\hypertarget{Ⓔcɤrme}{\papi{ cɤrme}}}\markboth{cɤrme}{}\classe{n}
\begin{définition}\fra poil de chevrotain\end{définition}
\begin{définition}\cmn 麝香鹿毛\end{définition}
\begin{relation-sémantique}\confer{
\hyperlink{Ⓔca}{\textit{ \papi{ca}}}
}\end{relation-sémantique}
\begin{relation-sémantique}\confer{
\hyperlink{Ⓔtɤ-rme}{\textit{ \papi{tɤ-rme}}}
}\end{relation-sémantique}\end{entrée}

\begin{entrée}
\vedette{\hypertarget{Ⓔcɤrna}{\papi{ cɤrna}}}\markboth{cɤrna}{}\classe{n}
\begin{définition}\fra petit pain rond\end{définition}
\begin{définition}\cmn 圆形的小馍馍
\end{définition}
\begin{relation-sémantique}\confer{
\hyperlink{Ⓔarɯcɤrna}{\textit{ \papi{arɯcɤrna}}}
}\end{relation-sémantique}\end{entrée}

\begin{entrée}
\vedette{\hypertarget{Ⓔcɤtʂha}{\papi{ cɤtʂha}}}\markboth{cɤtʂha}{}\classe{n}
\begin{définition}\fra une espèce d'arbrisseau\end{définition}
\begin{définition}\cmn 灌木的一种\end{définition}
\begin{exemple}\jya cɤtʂha nɯ si kɯ-mbɤr ci ŋu, ɯ-ru kɯ-pɣi ci ŋu, ɯ-si wuma ʑo ngɯt, tɕe khɯzi kɤ-βzu sna, ɯ-jwaʁ kɯ-ndɯβ ci ŋu, ɯ-jwaʁ kɯnɤ pɣi, ɯ-mat cho ɯ-mɯntoʁ ra me, cɤmi tsa tu-ɬoʁ ŋu, rpɣo pɕoʁ me.\cmn 
\stylefv{cɤtʂha}是矮小的树种,树干灰色,木质结实,可以用来制造连枷的接头。叶子很小,也是灰色的,既没有花也没有果实。生长在河坝上,高山上不能生长。
\end{exemple}
\end{entrée}

\begin{entrée}
\vedette{\hypertarget{Ⓔchu}{\papi{ chu}}}\markboth{chu}{}\classe{postp}
\begin{définition}\fra direction\end{définition}
\begin{définition}\cmn 方向\end{définition}
\begin{exemple}\jya a-thi kupa chu\cmn 下游的汉区\end{exemple}
\end{entrée}

\begin{entrée}
\vedette{\hypertarget{ⒺchaⒽ2}{\papi{ cha}}}\markboth{cha}{}\homonyme{2}\classe{n}
\begin{définition}\fra alcool fermenté, tchang\end{définition}
\begin{définition}\cmn 酒\end{définition}
\begin{relation-sémantique}\confer{
\hyperlink{Ⓔchɤtshi}{\textit{ \papi{chɤtshi}}}
}\end{relation-sémantique}
\begin{relation-sémantique}\confer{
\hyperlink{Ⓔɣɯchɤtshi}{\textit{ \papi{ɣɯchɤtshi}}}
}\end{relation-sémantique}
\begin{relation-sémantique}\confer{
\hyperlink{Ⓔnɯchɤrga}{\textit{ \papi{nɯchɤrga}}}
}\end{relation-sémantique}\end{entrée}

\begin{entrée}
\vedette{\hypertarget{ⒺchaⒽ1}{\papi{ cha}}}\markboth{cha}{}\homonyme{1}\classe{vi}
\paradigme{\textit{dir :} \jya tɤ-}\acception{1}
\begin{définition}\fra pouvoir\end{définition}
\begin{définition}\cmn 能够\end{définition}
\begin{exemple}\jya mɯ-pɯ-cha tɕe, tɕe tham tɕe ɯ-kɯ-mŋɤm to-mna tɕe, kɤ-rɯndzɤtshi to-cha\cmn 她原来吃饭不行,现在病痊愈了就可以吃饭了\end{exemple}
\begin{exemple}\jya mɯ-ɕɯ-tɯ-cha nɯ-sɯso-t-a\cmn 我怕你不行(忧虑式的例句)\end{exemple}\acception{2}
\begin{définition}\fra être capable\end{définition}
\begin{définition}\cmn 能干\end{définition}
\begin{exemple}\jya aʑo a-kɤ-cha me\cmn 什么都不会\end{exemple}
\begin{exemple}\jya nɤʑo kɯ ɲɯ-tɯ-cha tɕe nɤʑo jɤ-ɕe\cmn 你比较能干,你去吧\end{exemple}
\begin{relation-sémantique}\confer{
\hyperlink{Ⓔsɯxcha}{\textit{ \papi{sɯxcha}}}
}\end{relation-sémantique}
\begin{relation-sémantique}\confer{
\hyperlink{Ⓔsɤcha}{\textit{ \papi{sɤcha}}}
}\end{relation-sémantique}\end{entrée}

\begin{entrée}
\vedette{\hypertarget{Ⓔchaŋskɯ}{\papi{ chaŋskɯ}}}\markboth{chaŋskɯ}{}\classe{n}
\begin{définition}\fra bovidé au dos et au ventre blanc\end{définition}
\begin{définition}\cmn 背部和肚子都是白色的牛
\begin{déclaration} \étymologie{\papi{kʰʲuŋ}}\end{déclaration}\end{définition}\end{entrée}

\begin{entrée}
\vedette{\hypertarget{Ⓔchɤβ}{\papi{ chɤβ}}}\markboth{chɤβ}{}\classe{vt}
\paradigme{\textit{dir :} \jya tɤ-}
\paradigme{\textit{dir :} \jya pɯ-}\acception{1}
\begin{définition}\fra aplatir\end{définition}
\begin{définition}\cmn 弄扁\end{définition}
\begin{exemple}\jya tɤ-chaβ-a, ɯʑo kɯ ta-chɤβ\cmn 我弄扁了,他弄扁了\end{exemple}
\begin{exemple}\jya kɯki laχtɕha ki kɤ-chɤβ mɤ-pe\cmn 把这个东西压扁了不好\end{exemple}\acception{2}
\begin{définition}\fra plier (papier)\end{définition}
\begin{définition}\cmn 折在一起\end{définition}
\begin{exemple}\jya ɕoʁɕoʁ kɤ-chaβ-a\cmn 我把纸折在一起了\end{exemple}\acception{3}
\begin{définition}\fra courber (complètement)\end{définition}
\begin{définition}\cmn (完全)弄弯\end{définition}
\begin{exemple}\jya a-mthɤβ pɯ-chaβ-a\cmn 我弯下腰了\end{exemple}
\begin{exemple}\jya tɯ-sloχpɯn ɯ-ɕki tɯ-mthɤɣ pjɯ́-wɣ-chɤβ ra\cmn (学生)都得在老师面前鞠躬\end{exemple}
\begin{relation-sémantique}\confer{
\hyperlink{Ⓔɲɟɤβ}{\textit{ \papi{ɲɟɤβ}}}
}\end{relation-sémantique}\end{entrée}

\begin{entrée}
\vedette{\hypertarget{Ⓔchɤci}{\papi{ chɤci}}}\markboth{chɤci}{}\classe{n}
\begin{définition}\fra tchang (en jarre)\end{définition}
\begin{définition}\cmn 青稞酒放在坛子里以后挤出来的水\end{définition}
\begin{relation-sémantique}\confer{
\hyperlink{ⒺchaⒽ2}{\textit{ \papi{cha2}}}
}\end{relation-sémantique}
\begin{relation-sémantique}\confer{
\hyperlink{Ⓔtɯ-ci}{\textit{ \papi{tɯ-ci}}}
}\end{relation-sémantique}
\end{entrée}

\begin{entrée}
\vedette{\hypertarget{Ⓔchɤdi}{\papi{ chɤdi}}}\markboth{chɤdi}{}\classe{n}
\begin{définition}\fra odeur d'alcool\end{définition}
\begin{définition}\cmn 酒味\end{définition}
\begin{relation-sémantique}\confer{
\hyperlink{Ⓔtɤ-di}{\textit{ \papi{tɤ-di}}}
}\end{relation-sémantique}
\begin{relation-sémantique}\confer{
\hyperlink{ⒺchaⒽ2}{\textit{ \papi{cha2}}}
}\end{relation-sémantique}\end{entrée}

\begin{entrée}
\vedette{\hypertarget{Ⓔchɤlɤnnɤ}{\papi{ chɤlɤnnɤ}}}\markboth{chɤlɤnnɤ}{}\classe{adv}
\begin{définition}\fra peut-être\end{définition}
\begin{définition}\cmn 也许\end{définition}
\begin{exemple}\jya chɤlɤnnɤ ɕe-a thaŋ, chɤlɤnnɤ mɤ-ɕe-a\cmn 我可能去,可能不去\end{exemple}
\begin{exemple}\jya jisŋi chɤlɤnnɤ tɯ-mɯ lɤt\cmn 今天也许会下雨\end{exemple}
\begin{exemple}\jya a-@dian chɤlɤnnɤ rtaʁ thaŋ nɯ-sɯso-t-a tɕe, nɯ ma @chong mɯ-kɤ-βzu-t-a\cmn 我想可能电足够,所以没有充\end{exemple}
\begin{relation-sémantique}\confer{
\hyperlink{ⒺlɤtⒽ1}{\textit{ \papi{lɤt1}}}
}\end{relation-sémantique}\end{entrée}

\begin{entrée}
\vedette{\hypertarget{Ⓔchɤle}{\papi{ chɤle}}}\markboth{chɤle}{}
\begin{relation-sémantique}\confer{
\hyperlink{Ⓔachala}{\textit{ \papi{achala}}}
}\end{relation-sémantique}\end{entrée}

\begin{entrée}
\vedette{\hypertarget{Ⓔchɤmda}{\papi{ chɤmda}}}\markboth{chɤmda}{}\classe{n}
\begin{définition}\fra tchang en jarre que l'on boit à la paille\end{définition}
\begin{définition}\cmn 坛坛酒\end{définition}
\begin{exemple}\jya chɤmda pɯ-tshi-t-a\cmn 我喝了坛坛酒\end{exemple}
\begin{relation-sémantique}\confer{
\hyperlink{Ⓔnɯchɤmda}{\textit{ \papi{nɯchɤmda}}}
}\end{relation-sémantique}\end{entrée}

\begin{entrée}
\vedette{\hypertarget{Ⓔchɤmdɤru}{\papi{ chɤmdɤru}}}\markboth{chɤmdɤru}{}
\classe{n}
\begin{définition}\fra paille en bambou pour boire le tchang\end{définition}
\begin{définition}\cmn 用竹子制成的管子,用来喝坛坛酒\end{définition}\end{entrée}

\begin{entrée}
\vedette{\hypertarget{Ⓔchɤmthɯm}{\papi{ chɤmthɯm}}}\markboth{chɤmthɯm}{}\classe{n}
\begin{définition}\fra mets et boissons\end{définition}
\begin{définition}\cmn 酒菜\end{définition}
\begin{relation-sémantique}\confer{
\hyperlink{Ⓔtɤ-mthɯm}{\textit{ \papi{tɤ-mthɯm}}}
}\end{relation-sémantique}
\begin{relation-sémantique}\confer{
\hyperlink{ⒺchaⒽ2}{\textit{ \papi{cha2}}}
}\end{relation-sémantique}\end{entrée}

\begin{entrée}
\vedette{\hypertarget{ⒺchɤtⒽ1}{\papi{ chɤt}}}\markboth{chɤt}{}\homonyme{1}
\classe{vs}
\begin{définition}\fra avoir pour différence\end{définition}
\begin{définition}\cmn 区别在于
\begin{déclaration}\use{必须跟两个反义的形容词连用}\end{déclaration}\end{définition}
\begin{exemple}\jya tɕiʑo ni kɯ-mɤku kɯ-maqhu ci chɤt ma, ʁnaʁna jɤ-ɣe-tɕi ɕti\cmn 我们的区别在于一个来得早,另一个来得晚,但是两个都来了\end{exemple}
\begin{relation-sémantique}\confer{
\hyperlink{Ⓔachɤt}{\textit{ \papi{achɤt}}}
}\end{relation-sémantique}\end{entrée}

\begin{entrée}
\vedette{\hypertarget{ⒺchɤtⒽ2}{\papi{ chɤt}}}\markboth{chɤt}{}\homonyme{2}
\classe{vi}
\paradigme{\textit{dir :} \jya pɯ-}
\begin{définition}\fra réussir\end{définition}
\begin{définition}\cmn 成功;得逞
\begin{déclaration}\use{古语}\end{déclaration}\end{définition}
\begin{exemple}\jya tɯ-tɤ-kɤnɤma a-pɯ-chɤt\cmn 希望所做的一切都能成功\end{exemple}
\begin{relation-sémantique}\synonyme{
\hyperlink{Ⓔŋgrɯ}{\textit{ \papi{ŋgrɯ}}}
}\end{relation-sémantique}\end{entrée}

\begin{entrée}
\vedette{\hypertarget{Ⓔchɤtpa}{\papi{ chɤtpa}}}\markboth{chɤtpa}{}
\classe{n}
\begin{définition}\fra différence\end{définition}
\begin{définition}\cmn 区别;差别\end{définition}
\begin{exemple}\jya ɕɤxɕo ndɤre ɯ-tɯ-mɯɕtaʁ kɯ, qartsɯ cho chɤtpa maŋe\cmn 最近很难冷,跟冬天没有什么区别\end{exemple}\end{entrée}

\begin{entrée}
\vedette{\hypertarget{Ⓔchɤtshi}{\papi{ chɤtshi}}}\markboth{chɤtshi}{}\classe{n}
\begin{définition}\fra fait de boire beaucoup\end{définition}
\begin{définition}\cmn 喝很多酒\end{définition}
\begin{relation-sémantique}\confer{
\hyperlink{ⒺchaⒽ2}{\textit{ \papi{cha2}}}
}\end{relation-sémantique}
\begin{relation-sémantique}\confer{
\hyperlink{ⒺtshiⒽ1}{\textit{ \papi{tshi1}}}
}\end{relation-sémantique}
\begin{relation-sémantique}\confer{
\hyperlink{Ⓔɣɯchɤtshi}{\textit{ \papi{ɣɯchɤtshi}}}
}\end{relation-sémantique}\end{entrée}

\begin{entrée}
\vedette{\hypertarget{Ⓔchɤzwa}{\papi{ chɤzwa}}}\markboth{chɤzwa}{}\classe{n}
\begin{définition}\fra lie du vin\end{définition}
\begin{définition}\cmn 酒渣滓\end{définition}
\begin{relation-sémantique}\confer{
\hyperlink{Ⓔtʂhazwa}{\textit{ \papi{tʂhazwa}}}
}\end{relation-sémantique}
\end{entrée}

\begin{entrée}
\vedette{\hypertarget{Ⓔche}{\papi{ che}}}\markboth{che}{}\classe{intj}
\begin{définition}\fra attend un peu\end{définition}
\begin{définition}\cmn 等一下\end{définition}\end{entrée}

\begin{entrée}
\vedette{\hypertarget{Ⓔchɣaʁchɣaʁ}{\papi{ chɣaʁchɣaʁ}}}\markboth{chɣaʁchɣaʁ}{}\classe{idph.2}
\begin{définition}\fra propre et bien rangé\end{définition}
\begin{définition}\cmn 形容又干净又整齐的样子\end{définition}\end{entrée}

\begin{entrée}
\vedette{\hypertarget{Ⓔchi}{\papi{ chi}}}\markboth{chi}{}\classe{vi}
\paradigme{\textit{dir :} \jya nɯ-}
\begin{définition}\fra sucré\end{définition}
\begin{définition}\cmn 甜\end{définition}
\begin{exemple}\jya ɲɯ-chi\cmn 是甜的\end{exemple}
\begin{relation-sémantique}\antonyme{
\hyperlink{Ⓔqiaβ}{\textit{ \papi{qiaβ}}}
}\end{relation-sémantique}\begin{sous-entrée}
\vedette{\hypertarget{}{\papi{ ɣɤchi}}}\markboth{ɣɤchi}{}\classe{vt}
\paradigme{\textit{dir :} \jya pɯ-}
\paradigme{\textit{dir :} \jya nɯ-}
\begin{définition}\ 
\begin{déclaration}\grammar{caus}\end{déclaration}\end{définition}
\begin{définition}\fra sucrer\end{définition}
\begin{définition}\cmn 放糖;使……变得更甜\end{définition}
\begin{exemple}\jya pɯ-ɣɤchi-t-a\cmn 我放了糖\end{exemple}
\end{sous-entrée}\begin{sous-entrée}
\vedette{\hypertarget{}{\papi{ nɤxchi}}}\markboth{nɤxchi}{}\classe{vt}
\paradigme{\textit{dir :} \jya pɯ-}
\begin{définition}\ 
\begin{déclaration}\grammar{trop}\end{déclaration}\end{définition}
\begin{définition}\fra trouver sucré\end{définition}
\begin{définition}\cmn 觉得甜\end{définition}
\end{sous-entrée}\begin{sous-entrée}
\vedette{\hypertarget{}{\papi{ sɯxchi}}}\markboth{sɯxchi}{}\classe{vt}
\begin{définition}\ 
\begin{déclaration}\grammar{caus}\end{déclaration}\end{définition}
\begin{définition}\fra sucrer\end{définition}
\begin{définition}\cmn 使……变得更甜\end{définition}
\end{sous-entrée}\end{entrée}

\begin{entrée}
\vedette{\hypertarget{Ⓔchoŋtɕɯn}{\papi{ choŋtɕɯn}}}\markboth{choŋtɕɯn}{}\classe{n}
\begin{définition}\fra grand oiseau mythologique\end{définition}
\begin{définition}\cmn 大鹏
\begin{déclaration} \étymologie{\papi{kʰʲuŋ.tɕʰen}}\end{déclaration}\end{définition}\end{entrée}

\begin{entrée}
\vedette{\hypertarget{Ⓔchrɤβchrɤβ}{\papi{ chrɤβchrɤβ}}}\markboth{chrɤβchrɤβ}{}\classe{idph.2}
\begin{définition}\fra sale et en désordre\end{définition}
\begin{définition}\cmn 没有打扫干净\end{définition}
\begin{exemple}\jya nɤ-kha ra chrɤβchrɤβ ʑo to-tɯ-stu-t\cmn 你的家没有打扫干净\end{exemple}
\begin{exemple}\jya kha mɯ-thɯ́-wɣ-raʁrɯz tɕe, chrɤβchrɤβ ʑo pa\cmn 家里萌芽扫干净,很脏\end{exemple}\begin{sous-entrée}
\vedette{\hypertarget{}{\papi{ chrɤβnɤchrɤβ}}}\markboth{chrɤβnɤchrɤβ}{}\classe{idph.3}
\begin{exemple}\jya paʁtshi ɲɯ-ɤsɯ-ɕmi chrɤβnɤchrɤβ ɲɯ-ɤsɯ-stu\cmn 他在搅拌猪食,很脏\end{exemple}
\begin{relation-sémantique}\confer{
\hyperlink{Ⓔɣɤchrɤβchrɤβ}{\textit{ \papi{ɣɤchrɤβchrɤβ}}}
}\end{relation-sémantique}
\end{sous-entrée}\end{entrée}

\begin{entrée}
\vedette{\hypertarget{Ⓔchɯ}{\papi{ chɯ}}}\markboth{chɯ}{}\classe{n}
\begin{définition}\ 
\begin{déclaration}\grammar{n.lieu}\end{déclaration}\end{définition}
\begin{définition}\fra l'un des hameaux de Gyutshapa\end{définition}
\begin{définition}\cmn 二茶村的大队之一\end{définition}\end{entrée}

\begin{entrée}
\vedette{\hypertarget{Ⓔchɯβchɯβ}{\papi{ chɯβchɯβ}}}\markboth{chɯβchɯβ}{} (\variante{chɯpchɯp}) \classe{idph.2}
\begin{définition}\fra crasseux\end{définition}
\begin{définition}\cmn 很脏(地)\end{définition}
\begin{exemple}\jya ɯ-thoʁ chɯpchɯp ʑo ɲɯ-pa\cmn 地很脏(到处都有赃物,没有扫干净)\end{exemple}
\begin{exemple}\jya kha nɯ chɯpchɯp ma-kɤ-tɯ-ste\cmn 你不要把家里弄得那么乱\end{exemple}
\begin{exemple}\jya tɕhaχɯ ɲɯ-ɲaʁ chɯβchɯβ ʑo ɲɯ-pa\cmn 茶壶又脏又黑的样子\end{exemple}\begin{sous-entrée}
\vedette{\hypertarget{}{\papi{ chɯβ}}}\markboth{chɯβ}{}\classe{idph.1}
\begin{définition}\fra bruit d'objet qui se casse\end{définition}
\begin{définition}\cmn 东西折断的声音(咔嚓)\end{définition}
\begin{exemple}\jya chɯβ ʑo pa-qlɯt\cmn 他咔嚓一声就弄断了\end{exemple}
\begin{exemple}\jya chɯβ ʑo pɯ-ɴɢlɯt\cmn 咔嚓一声就断了\end{exemple}
\end{sous-entrée}\begin{sous-entrée}
\vedette{\hypertarget{}{\papi{ chɯβnɤchɯβ}}}\markboth{chɯβnɤchɯβ}{}
\begin{définition}\fra manger à grande bouchées\end{définition}
\begin{définition}\cmn 大口大口地吃\end{définition}
\begin{exemple}\jya paʁ chɯβnɤchɯβ ʑo ɲɯ-rɯndzɤtshi\cmn 猪大口大口地吃\end{exemple}
\end{sous-entrée}\end{entrée}

\begin{entrée}
\vedette{\hypertarget{Ⓔchɯchɯβ}{\papi{ chɯchɯβ}}}\markboth{chɯchɯβ}{}\classe{idph.2}
\begin{définition}\fra découpé n'importe comment en petits morceaux\end{définition}
\begin{définition}\cmn 形容(砍)得稀烂的样子\end{définition}
\begin{exemple}\jya si ra chɯchɯβ ʑo to-ta (=to-lɤt)\cmn 他把柴砍得稀烂\end{exemple}
\end{entrée}

\begin{entrée}
\vedette{\hypertarget{Ⓔchɯmu}{\papi{ chɯmu}}}\markboth{chɯmu}{}\classe{n}
\begin{définition}\fra chienne\end{définition}
\begin{définition}\cmn 母狗
\begin{déclaration} \étymologie{\papi{kʰʲi.mo}}\end{déclaration}\end{définition}
\end{entrée}

\begin{entrée}
\vedette{\hypertarget{Ⓔchɯrdom}{\papi{ chɯrdom}}}\markboth{chɯrdom}{}\classe{n}
\begin{définition}\fra chien vagabond\end{définition}
\begin{définition}\cmn 流浪狗\end{définition}\end{entrée}

\begin{entrée}
\vedette{\hypertarget{Ⓔchɯsɲu}{\papi{ chɯsɲu}}}\markboth{chɯsɲu}{}
\classe{n}
\begin{définition}\fra rage\end{définition}
\begin{définition}\cmn 狂犬病
\begin{déclaration} \étymologie{\papi{kʰʲi.smʲo}}\end{déclaration}\end{définition}
\begin{exemple}\jya chɯsɲu ɲɤ-k-ɤtɯɣ-ci\cmn 它得了狂犬病\end{exemple}\end{entrée}

\begin{entrée}
\vedette{\hypertarget{ⒺciⒽ2}{\papi{ ci}}}\markboth{ci}{}\homonyme{2}
\classe{num}\acception{1}
\begin{définition}\fra un\end{définition}
\begin{définition}\cmn 一\end{définition}
\begin{relation-sémantique}\synonyme{
\hyperlink{Ⓔtɤɣ}{\textit{ \papi{tɤɣ}}}
}\end{relation-sémantique}\acception{2}
\begin{définition}\fra indéfini\end{définition}
\begin{définition}\cmn 不定\end{définition}
\begin{exemple}\jya kɯɕɯŋgɯ tɕe ʁzɤmi ci pjɤ-tu-ndʑi\cmn 从前,有一对夫妻\end{exemple}
\begin{exemple}\jya ci ɯ-mphru ci ʑo\cmn 一个接着一个\end{exemple}\acception{3}
\begin{définition}\fra un peu\end{définition}
\begin{définition}\cmn 一点\end{définition}\acception{4}
\begin{définition}\fra une fois\end{définition}
\begin{définition}\cmn 一下\end{définition}
\begin{exemple}\jya li ci tɤ-ti\end{exemple}
\begin{sous-entrée}
\vedette{\hypertarget{}{\papi{ ci ci}}}\markboth{ci ci}{}\acception{1}
\begin{définition}\fra certains\end{définition}
\begin{définition}\cmn 有些\end{définition}\acception{2}
\begin{définition}\fra parfois\end{définition}
\begin{définition}\cmn 有时候\end{définition}
\begin{exemple}\jya tɯ-mɯ ci ci ku-lɤt, ci ci mɯ́j-lɤt\cmn 偶尔下雨,偶尔不下\end{exemple}
\end{sous-entrée}\begin{sous-entrée}
\vedette{\hypertarget{}{\papi{ ci nɤ}}}\markboth{ci nɤ}{}
\begin{définition}\fra pas (même) un\end{définition}
\begin{définition}\cmn 连……也\end{définition}
\begin{exemple}\jya ɯ-kɤrme tɯ-ldʑa ci nɤ a-mɤ-jɤ-tɯ-ɕɣɤz ma mɤ-jɤɣ\cmn 连一根头发都不要带回来\end{exemple}
\end{sous-entrée}\begin{sous-entrée}
\vedette{\hypertarget{}{\papi{ ci ntsɯ}}}\markboth{ci ntsɯ}{}\acception{1}
\begin{définition}\fra chacun (sans exception)\end{définition}
\begin{définition}\cmn 一个也不漏\end{définition}
\begin{exemple}\jya khɯtsa tɯ-rdoʁ tɕe ci ntsɯ pjɤ-ta, maka mɯ-ɲɤ-sɤtɕɯtɕit\cmn 每一碗都放了一块,一个也没有漏掉\end{exemple}\acception{2}
\begin{définition}\fra chaque fois\end{définition}
\begin{définition}\cmn 每一次\end{définition}
\begin{exemple}\jya ci a-mɤ-ɕ-tɤ-ru tɕe, ci ntsɯ tu-dɤn pjɤ-ŋgrɤl\cmn 只要少看一次,就会变得多一点\end{exemple}
\end{sous-entrée}\begin{sous-entrée}
\vedette{\hypertarget{}{\papi{ ci nɯ}}}\markboth{ci nɯ}{}\classe{pro}
\begin{définition}\fra l'autre (dont on a déjà parlé)\end{définition}
\begin{définition}\cmn 另外那个\end{définition}
\begin{exemple}\jya ci nɯ kɯ nɯra to-nɯ-ndo qhe nɯ jo-nɯ-ɕe\cmn 另外那个(人)把那些(东西)拿走了\end{exemple}
\end{sous-entrée}\end{entrée}

\begin{entrée}
\vedette{\hypertarget{ⒺciⒽ1}{\papi{ ci}}}\markboth{ci}{}\homonyme{1}
\classe{vt}
\paradigme{\textit{dir :} \jya pɯ-}
\begin{définition}\fra verser complètement d'un récipient à l'autre\end{définition}
\begin{définition}\cmn 完全倒出来
\begin{déclaration}\use{一般表示把粮食从一个口袋装进另一个口袋}\end{déclaration}\end{définition}
\begin{exemple}\jya ki tɯ-rdoʁ ra pɯ-ci-t-a\cmn 我把这些粮食装进了(那个口袋)\end{exemple}
\begin{exemple}\jya a-tʂha ɲɯ-sɤɕke tɕe, tsuku pɯ-ci\cmn 我的茶很烫,倒一点出来\end{exemple}
\begin{relation-sémantique}\synonyme{
\hyperlink{Ⓔphoʁ}{\textit{ \papi{phoʁ}}}
}\end{relation-sémantique}\end{entrée}

\begin{entrée}
\vedette{\hypertarget{Ⓔcischiz}{\papi{ cischiz}}}\markboth{cischiz}{}\classe{pro}
\begin{définition}\fra n'importe où, à un certain endroit\end{définition}
\begin{définition}\cmn 在某个地方,随便在哪个地方\end{définition}\end{entrée}

\begin{entrée}
\vedette{\hypertarget{Ⓔcit}{\papi{ cit}}}\markboth{cit}{}
\classe{vi}
\paradigme{\textit{dir :} \jya jɤ-}
\begin{définition}\fra bouger\end{définition}
\begin{définition}\cmn 移动;搬迁\end{définition}
\begin{exemple}\jya jɤ-cit-a, ɯʑo jɤ-cit\cmn 我搬了,他搬了\end{exemple}
\begin{exemple}\jya kha ta-βzu-nɯ tɕe, ju-cit-i ra\cmn 他们修了房子,我们要搬\end{exemple}
\begin{relation-sémantique}\synonyme{
\hyperlink{Ⓔspɤr}{\textit{ \papi{spɤr}}}
}\end{relation-sémantique}
\begin{relation-sémantique}\confer{
 \papi{nɤqhɤcit}
}\end{relation-sémantique}\end{entrée}

\begin{entrée}
\vedette{\hypertarget{Ⓔciz tɕe}{\papi{ ciz tɕe}}}\markboth{ciz tɕe}{}\classe{adv}
\begin{définition}\fra à un moment futur\end{définition}
\begin{définition}\cmn 总有一天(未来)\end{définition}
\begin{exemple}\jya ɯʑo ciz tɕe ɣi ɕti\cmn 他总会要来的\end{exemple}
\begin{exemple}\jya aʑo ciz tɕe ɕɯ-rtoʁ-a ra\cmn 我总有一天会去看他\end{exemple}\end{entrée}

\begin{entrée}
\vedette{\hypertarget{Ⓔclaŋclaŋ}{\papi{ claŋclaŋ}}}\markboth{claŋclaŋ}{}
\classe{idph.2}
\begin{définition}\fra rond et lisse\end{définition}
\begin{définition}\cmn 形容又圆又光滑的样子(油光发亮)\end{définition}
\begin{exemple}\jya ɯ-ku pjɤ-nɯ-sɯ-qrɤz tɕe, claŋclaŋ ʑo ɲɯ-pa\cmn 他理了发,头光滑得发亮\end{exemple}\begin{sous-entrée}
\vedette{\hypertarget{}{\papi{ claŋnɤclaŋ}}}\markboth{claŋnɤclaŋ}{}\classe{idph.3}
\begin{exemple}\jya claŋnɤclaŋ kɤ-ari\cmn 他(头光溜得发亮)地去了\end{exemple}
\begin{relation-sémantique}\confer{
\hyperlink{Ⓔɕlaŋɕlaŋ}{\textit{ \papi{ɕlaŋɕlaŋ}}}
}\end{relation-sémantique}
\begin{relation-sémantique}\confer{
\hyperlink{Ⓔslaŋslaŋ}{\textit{ \papi{slaŋslaŋ}}}
}\end{relation-sémantique}
\begin{relation-sémantique}\confer{
\hyperlink{Ⓔɕliɕli}{\textit{ \papi{ɕliɕli}}}
}\end{relation-sémantique}
\end{sous-entrée}\end{entrée}

\begin{entrée}
\vedette{\hypertarget{Ⓔclaʁclaʁ}{\papi{ claʁclaʁ}}}\markboth{claʁclaʁ}{}\classe{idph.2}
\begin{définition}\fra brillant\end{définition}
\begin{définition}\cmn 形容闪亮耀眼的样子\end{définition}
\begin{exemple}\jya rgali wuma ʑo ɯ-ɲɤm ɲɯ-sna tɕe, ɲɯ-nɤmbju ʑo claʁclaʁ\cmn 小奶牛膘满肉肥,皮毛闪闪发亮\end{exemple}\end{entrée}

\begin{entrée}
\vedette{\hypertarget{Ⓔclicli}{\papi{ clicli}}}\markboth{clicli}{}\classe{idph.2}
\begin{définition}\fra noir, lisse et brillant\end{définition}
\begin{définition}\cmn 形容脑袋光溜溜又黑的模样\end{définition}
\begin{relation-sémantique}\confer{
\hyperlink{Ⓔclɯclɯ}{\textit{ \papi{clɯclɯ}}}
}\end{relation-sémantique}\end{entrée}

\begin{entrée}
\vedette{\hypertarget{Ⓔcloʁcloʁ}{\papi{ cloʁcloʁ}}}\markboth{cloʁcloʁ}{}\classe{idph.2}
\begin{définition}\fra noir brillant\end{définition}
\begin{définition}\cmn 形容耀眼的黑色\end{définition}
\begin{exemple}\jya tɯrme ra tɤŋe ɯ-ŋgɯ ntsɯ ɲɯ-rɤma-nɯ tɕe ɲɯ-ɲaʁ-nɯ cloʁcloʁ ʑo ɲɯ-ŋu\cmn 人们长期在太阳底下劳动,皮肤就变得黑油油的\end{exemple}
\begin{exemple}\jya ko-naʁri cloʁcloʁ ʑo\cmn 这个东西上面有油迹,变得黑油油的\end{exemple}\end{entrée}

\begin{entrée}
\vedette{\hypertarget{Ⓔclɯclɯ}{\papi{ clɯclɯ}}}\markboth{clɯclɯ}{}
\classe{idph.2}
\begin{définition}\fra noir et chauve\end{définition}
\begin{définition}\cmn 形容脑袋光溜溜又黑的模样\end{définition}
\begin{exemple}\jya jiɕqha tɤ-rɟit nɯ clɯclɯ ɲɯ-ɲaʁ\cmn 小孩子,脸是黑的,又是光头\end{exemple}\begin{sous-entrée}
\vedette{\hypertarget{}{\papi{ clɯnɤclɯ}}}\markboth{clɯnɤclɯ}{}\classe{idph.3}
\begin{exemple}\jya clɯnɤclɯ ɲɯ-ŋke\cmn 脸黑、光头(的小孩子)在走\end{exemple}
\begin{relation-sémantique}\confer{
\hyperlink{Ⓔclicli}{\textit{ \papi{clicli}}}
}\end{relation-sémantique}
\end{sous-entrée}\end{entrée}

\begin{entrée}
\vedette{\hypertarget{Ⓔco}{\papi{ co}}}\markboth{co}{}
\classe{n}
\begin{définition}\fra vallée\end{définition}
\begin{définition}\cmn 沟\end{définition}
\begin{sous-entrée}
\vedette{\hypertarget{}{\papi{ tɯ-co}}}\markboth{tɯ-co}{}\classe{clf}
\begin{définition}\fra toute la vallée\end{définition}
\begin{définition}\cmn 整条沟\end{définition}
\begin{exemple}\jya mbarkhom rcanɯ, kha tɯ-co ʑo ɣɤʑu\cmn 马尔康整条沟都是房子\end{exemple}
\end{sous-entrée}\end{entrée}

\begin{entrée}
\vedette{\hypertarget{Ⓔcoɕit}{\papi{ coɕit}}}\markboth{coɕit}{}\classe{n}
\begin{définition}\fra vallée\end{définition}
\begin{définition}\cmn 山谷\end{définition}
\end{entrée}

\begin{entrée}
\vedette{\hypertarget{Ⓔcomuco}{\papi{ comuco}}}\markboth{comuco}{}\classe{n}
\begin{définition}\fra Kyomkyo\end{définition}
\begin{définition}\cmn 脚木足\end{définition}\end{entrée}

\begin{entrée}
\vedette{\hypertarget{ⒺcoŋⒽ1}{\papi{ coŋ}}}\markboth{coŋ}{}\homonyme{1}
\classe{n}
\begin{définition}\fra perte, dommage\end{définition}
\begin{définition}\cmn 损失\end{définition}
\begin{exemple}\jya a-coŋ pɯ-ɣe\cmn 我受损了\end{exemple}
\begin{exemple}\jya coŋ wuma ʑo ɲɯ-wxti\cmn 损失很惨重\end{exemple}
\begin{relation-sémantique}\confer{
\hyperlink{ⒺcoŋⒽ2}{\textit{ \papi{coŋ2}}}
}\end{relation-sémantique}\end{entrée}

\begin{entrée}
\vedette{\hypertarget{ⒺcoŋⒽ2}{\papi{ coŋ}}}\markboth{coŋ}{}\homonyme{2}
\classe{vi}
\paradigme{\textit{dir :} \jya pɯ-}
\begin{définition}\fra subir des dommages\end{définition}
\begin{définition}\cmn 受损\end{définition}
\begin{exemple}\jya pɯ-coŋ-a\cmn 我受损了\end{exemple}\end{entrée}

\begin{entrée}
\vedette{\hypertarget{Ⓔcotcot}{\papi{ cotcot}}}\markboth{cotcot}{}
\classe{idph.2}
\begin{définition}\fra petit et mignon\end{définition}
\begin{définition}\cmn 形容又小又可爱的样子\end{définition}
\begin{exemple}\jya wɯɟa cotcot ɲɯ-pa\cmn 调羹很小\end{exemple}
\begin{exemple}\jya tɤmɤru cotcot ʑo ɲɯ-pa\cmn 脚印很小\end{exemple}
\begin{exemple}\jya karkɯm cotcot ʑo to-ɬoʁ\cmn 圆根的子叶很小\end{exemple}
\begin{exemple}\jya tɤ-pɤtso ɯ-jaʁ cotcot ʑo ko-ɕthɯz\cmn 小孩子伸了手,很可爱\end{exemple}\end{entrée}

\begin{entrée}
\vedette{\hypertarget{Ⓔcupa}{\papi{ cupa}}}\markboth{cupa}{}
\classe{n}
\begin{définition}\fra ardoise\end{définition}
\begin{définition}\cmn 石板\end{définition}\end{entrée}

\begin{entrée}
\vedette{\hypertarget{Ⓔcraŋ}{\papi{ craŋ}}}\markboth{craŋ}{}\classe{idph.1}
\begin{définition}\fra bruit du verre qui se brise\end{définition}
\begin{définition}\cmn 玻璃被砸碎的声音\end{définition}\begin{sous-entrée}
\vedette{\hypertarget{}{\papi{ craŋnɤlaŋ}}}\markboth{craŋnɤlaŋ}{}\classe{idph.4}
\begin{exemple}\jya craŋnɤlaŋ ɲɯ-ɤkhɤzŋga\cmn 他大声地喊\end{exemple}
\begin{relation-sémantique}\confer{
\hyperlink{Ⓔɣɤcraŋlaŋ}{\textit{ \papi{ɣɤcraŋlaŋ}}}
}\end{relation-sémantique}
\end{sous-entrée}\end{entrée}

\begin{entrée}
\vedette{\hypertarget{Ⓔcrɯβcrɯβ}{\papi{ crɯβcrɯβ}}}\markboth{crɯβcrɯβ}{}
\classe{idph.2}
\begin{définition}\fra brisé en mille morceaux\end{définition}
\begin{définition}\cmn 玻璃被砸碎\end{définition}
\begin{exemple}\jya χɕɤl crɯβcrɯβ ʑo pjɤ-ɴɢrɯ\cmn 玻璃砸得粉碎\end{exemple}
\begin{relation-sémantique}\confer{
\hyperlink{Ⓔcrɯmcrɯm}{\textit{ \papi{crɯmcrɯm}}}
}\end{relation-sémantique}
\begin{relation-sémantique}\confer{
\hyperlink{Ⓔcɯmcɯm}{\textit{ \papi{cɯmcɯm}}}
}\end{relation-sémantique}\end{entrée}

\begin{entrée}
\vedette{\hypertarget{Ⓔcrɯcrɯ}{\papi{ crɯcrɯ}}}\markboth{crɯcrɯ}{}\classe{idph.2}
\begin{définition}\fra sale, crasseux et plein de morve\end{définition}
\begin{définition}\cmn 形容又脏又湿,吊着鼻涕的样子\end{définition}
\begin{exemple}\jya ki tɤ-pɤtso pjɯ́-wɣ-zraχtɕi ɲɯ-ra ma crɯcrɯ ʑo ɲɯ-pa\cmn 要给这个小孩子洗澡,他很脏的样子\end{exemple}\end{entrée}

\begin{entrée}
\vedette{\hypertarget{Ⓔcrɯɣcrɯɣ}{\papi{ crɯɣcrɯɣ}}}\markboth{crɯɣcrɯɣ}{}
\classe{idph.2}
\begin{définition}\fra en désordre\end{définition}
\begin{définition}\cmn 形容乱七八糟的样子(很多东西)\end{définition}
\begin{exemple}\jya laχtɕha crɯɣcrɯɣ ʑo ko-rmbɯ\cmn 他把东西得乱七八糟\end{exemple}
\begin{relation-sémantique}\confer{
\hyperlink{Ⓔɟrɯɣɟrɯɣ}{\textit{ \papi{ɟrɯɣɟrɯɣ}}}
}\end{relation-sémantique}\begin{sous-entrée}
\vedette{\hypertarget{}{\papi{ ɣɤcrɯɣlɯɣ}}}\markboth{ɣɤcrɯɣlɯɣ}{}\classe{vi}
\begin{définition}\fra beaucoup d'objets en désordre qui s'entrechoquent\end{définition}
\begin{définition}\cmn 很多硬的乱七八糟的东西相碰发出声音\end{définition}
\end{sous-entrée}\end{entrée}

\begin{entrée}
\vedette{\hypertarget{Ⓔcrɯmcrɯm}{\papi{ crɯmcrɯm}}}\markboth{crɯmcrɯm}{}\classe{idph.2}
\begin{définition}\fra brisé en mille morceaux\end{définition}
\begin{définition}\cmn 玻璃被砸碎\end{définition}
\begin{exemple}\jya χɕɤl crɯmcrɯm ʑo pjɤ-ɴɢrɯ\cmn 玻璃砸得粉碎\end{exemple}
\begin{relation-sémantique}\confer{
\hyperlink{Ⓔcrɯβcrɯβ}{\textit{ \papi{crɯβcrɯβ}}}
}\end{relation-sémantique}
\begin{relation-sémantique}\confer{
\hyperlink{Ⓔcɯmcɯm}{\textit{ \papi{cɯmcɯm}}}
}\end{relation-sémantique}\end{entrée}

\begin{entrée}
\vedette{\hypertarget{ⒺcɯⒽ3}{\papi{ cɯ}}}\markboth{cɯ}{}\homonyme{3}
\classe{n}
\begin{définition}\fra pierre\end{définition}
\begin{définition}\cmn 石头\end{définition}
\end{entrée}

\begin{entrée}
\vedette{\hypertarget{ⒺcɯⒽ1}{\papi{ cɯ}}}\markboth{cɯ}{}\homonyme{1}
\classe{vt}
\paradigme{\textit{dir :} \jya tɤ-}
\begin{définition}\fra ouvrir\end{définition}
\begin{définition}\cmn 打开\end{définition}
\begin{exemple}\jya kɯm tɤ-cɯ-t-a\cmn 我开了门\end{exemple}
\begin{exemple}\jya kɯm tɤ-ci\cmn 你开门吧\end{exemple}
\begin{exemple}\jya tʂu tɤ-cɯ-t-a\cmn 我让了路\end{exemple}
\begin{exemple}\jya ɯ-mɲaʁ ɲɤ-cɯ\cmn 他睁开了眼睛\end{exemple}
\begin{exemple}\jya a-rnoʁ ʑo ku-ci ɲɯ-ɕti, tɕe kɤ-sɤŋo mɯ́j-khɯ\cmn 很吵,无法听清楚\end{exemple}
\begin{exemple}\jya ma-tɯ-ɤrju ma a-rnoʁ ma-tɯ-ci ma ɲɯ-sɤɣdɯɣ\cmn 你不要说了,不要吵我\end{exemple}
\begin{relation-sémantique}\confer{
\hyperlink{Ⓔɲɟɯ}{\textit{ \papi{ɲɟɯ}}}
}\end{relation-sémantique}\end{entrée}

\begin{entrée}
\vedette{\hypertarget{ⒺcɯⒽ2}{\papi{ cɯ}}}\markboth{cɯ}{}\homonyme{2}
\classe{vi}\paradigme{\textit{dir :} \jya kɤ-}
\begin{définition}\fra hiberner\end{définition}
\begin{définition}\cmn 冬眠\end{définition}
\begin{exemple}\jya rɯdaʁ ra qartsɯ tɕe ku-cɯ-nɯ ŋu\cmn 动物冬天的时候冬眠\end{exemple}
\begin{exemple}\jya qajɯ kɯnɤ ku-cɯ-nɯ ŋu ŋgrɤl\cmn 虫子也冬眠\end{exemple}
\begin{exemple}\jya χɕitka jɤ-ɣe tɕe chɯ-ɬoʁ-nɯ ŋu, qartsɯ tɕe ku-cɯ-nɯ ŋu\cmn 春天的时候出来,冬天的时候冬眠\end{exemple}\end{entrée}

\begin{entrée}
\vedette{\hypertarget{Ⓔcɯβjiz}{\papi{ cɯβjiz}}}\markboth{cɯβjiz}{}
\classe{n}
\begin{définition}\fra plaque de pierre\end{définition}
\begin{définition}\cmn 石板\end{définition}\end{entrée}

\begin{entrée}
\vedette{\hypertarget{Ⓔcɯβloʁ}{\papi{ cɯβloʁ}}}\markboth{cɯβloʁ}{}
\classe{n}
\begin{définition}\fra flaque d'eau\end{définition}
\begin{définition}\cmn 水洼;水坑\end{définition}
\begin{exemple}\jya cɯβloʁ ɯ-ŋgɯ qaɕpa ɣɤʑu\cmn 在水坑里有个青蛙\end{exemple}\end{entrée}

\begin{entrée}
\vedette{\hypertarget{Ⓔcɯɣɬaj}{\papi{ cɯɣɬaj}}}\markboth{cɯɣɬaj}{}\classe{n}
\begin{définition}\fra symptôme dans lequel la muqueuse buccale devient blanche\end{définition}
\begin{définition}\cmn 口腔内膜变白(症状)\end{définition}
\begin{exemple}\jya ɯ-kɯr cɯɣɬaj ɲɤ-xtsu\cmn 嘴巴里发白了\end{exemple}\end{entrée}

\begin{entrée}
\vedette{\hypertarget{Ⓔcɯm}{\papi{ cɯm}}}\markboth{cɯm}{}\classe{vt}
\paradigme{\textit{dir :} \jya kɤ-}
\begin{définition}\fra retourner vers l'intérieur (les bords d'un tissu découpé avec une paire de ciseaux)\end{définition}
\begin{définition}\cmn 把(用剪刀剪过的)布料的边缘往里面折进去\end{définition}
\begin{exemple}\jya tɯ-ŋga ɯ-tɕhɤz nɯ nɯ-ʁndzar-a tɕe, ɯ-rtsho nɯ kɤ-cɯm-a\cmn 我把衣服的吊边布剪出来,然后把边缘折进去缝起来\end{exemple}\begin{sous-entrée}
\vedette{\hypertarget{}{\papi{ nɯɣɯcɯm}}}\markboth{nɯɣɯcɯm}{}\classe{vs}
\begin{définition}\fra facile à retourner vers l'intérieur (d'un tissu)\end{définition}
\begin{définition}\cmn 容易折进去\end{définition}
\begin{exemple}\jya tɯ-ŋga kɯ-mba nɯra nɯɣɯcɯm, kɯ-jaʁ nɯra mɤ-nɯɣɯcɯm\cmn 布料薄的衣服容易折进去,布料厚的衣服就不容易折进去\end{exemple}
\end{sous-entrée}\end{entrée}

\begin{entrée}
\vedette{\hypertarget{Ⓔcɯmbɤrom}{\papi{ cɯmbɤrom}}}\markboth{cɯmbɤrom}{}\classe{n}
\begin{définition}\fra ampoule\end{définition}
\begin{définition}\cmn 水疱\end{définition}
\begin{exemple}\jya aʑo a-jaʁ pɯ-ɕke tɕe, cɯmbɤrom to-rku\cmn 我烫伤了手就出现了水泡\end{exemple}
\begin{relation-sémantique}\confer{
 \papi{tɕhɯwur}
}\end{relation-sémantique}\end{entrée}

\begin{entrée}
\vedette{\hypertarget{Ⓔcɯmcɯm}{\papi{ cɯmcɯm}}}\markboth{cɯmcɯm}{}
\classe{idph.2}
\begin{définition}\fra brisé en mille morceaux\end{définition}
\begin{définition}\cmn 玻璃被砸碎\end{définition}
\begin{exemple}\jya χɕɤl pjɤ-xtsɯ cɯmcɯm ʑo\cmn 玻璃砸得粉碎\end{exemple}
\begin{relation-sémantique}\confer{
\hyperlink{Ⓔcrɯmcrɯm}{\textit{ \papi{crɯmcrɯm}}}
}\end{relation-sémantique}
\begin{relation-sémantique}\confer{
\hyperlink{Ⓔcrɯβcrɯβ}{\textit{ \papi{crɯβcrɯβ}}}
}\end{relation-sémantique}\end{entrée}

\begin{entrée}
\vedette{\hypertarget{Ⓔcɯnkhɤβ}{\papi{ cɯnkhɤβ}}}\markboth{cɯnkhɤβ}{}\classe{n}
\begin{définition}\fra satin\end{définition}
\begin{définition}\cmn 缎子\end{définition}
\begin{relation-sémantique}\confer{
\hyperlink{Ⓔcɯrzɯn}{\textit{ \papi{cɯrzɯn}}}
}\end{relation-sémantique}\end{entrée}

\begin{entrée}
\vedette{\hypertarget{Ⓔcɯnmu}{\papi{ cɯnmu}}}\markboth{cɯnmu}{}
\classe{n}
\begin{définition}\fra rumeurs, zizanie\end{définition}
\begin{définition}\cmn 谣言(挑拨离间)\end{définition}
\begin{exemple}\jya cɯnmu a-mɤ-tɤ-tɯ-fkri (=a-mɤ-tɤ-tɯ-βze)\cmn 你不要挑拨离间\end{exemple}
\begin{relation-sémantique}\confer{
\hyperlink{Ⓔrɯcɯnmu}{\textit{ \papi{rɯcɯnmu}}}
}\end{relation-sémantique}\end{entrée}

\begin{entrée}
\vedette{\hypertarget{Ⓔcɯŋcɯŋ}{\papi{ cɯŋcɯŋ}}}\markboth{cɯŋcɯŋ}{}\classe{idph.2}
\begin{définition}\fra radieux, ensoleillé\end{définition}
\begin{définition}\cmn 阳光强烈,普照大地\end{définition}
\begin{exemple}\jya tɤŋe cɯŋcɯŋ ʑo pjɤ-zɣɯt tɕe ɲɯ-sɤscit\cmn 阳光很强烈,很舒服\end{exemple}\end{entrée}

\begin{entrée}
\vedette{\hypertarget{Ⓔcɯŋglɯɣ}{\papi{ cɯŋglɯɣ}}}\markboth{cɯŋglɯɣ}{}
\classe{n}
\begin{définition}\fra pilon\end{définition}
\begin{définition}\cmn 杵【捶捶】\end{définition}\end{entrée}

\begin{entrée}
\vedette{\hypertarget{Ⓔcɯpɤspoʁspoʁ}{\papi{ cɯpɤspoʁspoʁ}}}\markboth{cɯpɤspoʁspoʁ}{}\classe{n}
\begin{définition}\fra jouet en forme de meule\end{définition}
\begin{définition}\cmn 水磨模型的玩具\end{définition}\end{entrée}

\begin{entrée}
\vedette{\hypertarget{Ⓔcɯphɯt}{\papi{ cɯphɯt}}}\markboth{cɯphɯt}{}\classe{n}
\begin{définition}\fra ramassage de pierres\end{définition}
\begin{définition}\cmn 捡石头(庄稼地)\end{définition}
\begin{exemple}\jya cɯphɯt nɯ-βzu-t-a (=cɯ nɯ-phɯt-a, nɯ-ɣɯcɯphɯt-a)\cmn 我捡了石头\end{exemple}
\begin{relation-sémantique}\confer{
\hyperlink{ⒺcɯⒽ3}{\textit{ \papi{cɯ3}}}
}\end{relation-sémantique}
\begin{relation-sémantique}\confer{
\hyperlink{Ⓔphɯt}{\textit{ \papi{phɯt}}}
}\end{relation-sémantique}
\begin{relation-sémantique}\confer{
\hyperlink{Ⓔɣɯcɯphɯt}{\textit{ \papi{ɣɯcɯphɯt}}}
}\end{relation-sémantique}\end{entrée}

\begin{entrée}
\vedette{\hypertarget{Ⓔcɯrmbɯ}{\papi{ cɯrmbɯ}}}\markboth{cɯrmbɯ}{}\classe{n}
\begin{définition}\fra tas de pierre\end{définition}
\begin{définition}\cmn 石头堆\end{définition}
\begin{relation-sémantique}\confer{
\hyperlink{ⒺcɯⒽ3}{\textit{ \papi{cɯ3}}}
}\end{relation-sémantique}
\begin{relation-sémantique}\confer{
\hyperlink{Ⓔtɯ-rmbɯ}{\textit{ \papi{tɯ-rmbɯ}}}
}\end{relation-sémantique}\end{entrée}

\begin{entrée}
\vedette{\hypertarget{Ⓔcɯrndʑi}{\papi{ cɯrndʑi}}}\markboth{cɯrndʑi}{}\classe{n}
\begin{définition}\fra sable\end{définition}
\begin{définition}\cmn 沙子\end{définition}\end{entrée}

\begin{entrée}
\vedette{\hypertarget{Ⓔcɯrzɯn}{\papi{ cɯrzɯn}}}\markboth{cɯrzɯn}{}\classe{n}
\begin{définition}\fra satin\end{définition}
\begin{définition}\cmn 缎子的一种\end{définition}
\begin{relation-sémantique}\confer{
\hyperlink{Ⓔcɯnkhɤβ}{\textit{ \papi{cɯnkhɤβ}}}
}\end{relation-sémantique}\end{entrée}

\begin{entrée}
\vedette{\hypertarget{Ⓔcɯʁrɤt}{\papi{ cɯʁrɤt}}}\markboth{cɯʁrɤt}{}
\classe{n}
\begin{définition}\fra charbon; houille\end{définition}
\begin{définition}\cmn 煤\end{définition}\end{entrée}

\begin{entrée}
\vedette{\hypertarget{Ⓔcɯχɕiz}{\papi{ cɯχɕiz}}}\markboth{cɯχɕiz}{}\classe{n}
\begin{définition}\fra terre pierreuse\end{définition}
\begin{définition}\cmn 沙地\end{définition}
\begin{relation-sémantique}\confer{
\hyperlink{ⒺcɯⒽ1}{\textit{ \papi{cɯ}}}
}\end{relation-sémantique}\end{entrée}

\newpage\caractère{ɕ}

\begin{entrée}
\vedette{\hypertarget{Ⓔɕu}{\papi{ ɕu}}}\markboth{ɕu}{}\classe{n}
\begin{définition}\fra carte\end{définition}
\begin{définition}\cmn 牌戏
\begin{déclaration} \étymologie{\papi{ɕo}}\end{déclaration}\end{définition}
\begin{exemple}\jya ɕu pɯ-lɤt-tɕi tɕe pɯ-ta-ɕɯnŋo\cmn 我们俩打牌的时候,我把你打败了\end{exemple}
\begin{relation-sémantique}\confer{
\hyperlink{Ⓔnɯɕu}{\textit{ \papi{nɯɕu}}}
}\end{relation-sémantique}\end{entrée}

\begin{entrée}
\vedette{\hypertarget{Ⓔɕa}{\papi{ ɕa}}}\markboth{ɕa}{}\classe{n}
\begin{définition}\fra viande crue\end{définition}
\begin{définition}\cmn 生肉
\begin{déclaration} \étymologie{\papi{ɕa}}\end{déclaration}\end{définition}
\end{entrée}

\begin{entrée}
\vedette{\hypertarget{ⒺɕaβⒽ1}{\papi{ ɕaβ}}}\markboth{ɕaβ}{}\homonyme{1}
\classe{vt}
\paradigme{\textit{dir :} \jya jɤ-}
\begin{définition}\fra rattraper\end{définition}
\begin{définition}\cmn 赶上;追上\end{définition}
\begin{exemple}\jya jɤ-ɕaβ-a, jɤ-tɯ-ɕaβ, ja-ɕaβ\cmn 我追上了他,你追上了他,他追上了他\end{exemple}
\begin{exemple}\jya a-wɯ a-wi ra mɯ-tɤ-ɕaβ-a\cmn 我们出生的时候我的爷爷奶奶已经去世了\end{exemple}\begin{sous-entrée}
\vedette{\hypertarget{}{\papi{ sɤɕaβ}}}\markboth{sɤɕaβ}{}\classe{vi}
\begin{définition}\ 
\begin{déclaration}\grammar{apass}\end{déclaration}\end{définition}
\begin{exemple}\jya nɯnɯ laχtɕha nɯ kɤ-mɟa mɯ́j-sɤɕaβ mɤ ɲɯ-ɤrqhi, ɲɯ-mbro\cmn 这个东西拿不着,因为太远了(太高了)\end{exemple}
\begin{exemple}\jya mɯ́j-sɤɕaβ\cmn 赶不上\end{exemple}
\end{sous-entrée}\end{entrée}

\begin{entrée}
\vedette{\hypertarget{ⒺɕaβⒽ2}{\papi{ ɕaβ}}}\markboth{ɕaβ}{}\homonyme{2}\classe{vs}
\begin{définition}\fra être assez long\end{définition}
\begin{définition}\cmn 够长(可以连接起来)\end{définition}
\begin{exemple}\jya tɤ-ri ɲɯ-ɕaβ\cmn 线够长\end{exemple}
\begin{exemple}\jya tɯmbri ɲɯ-ɕaβ\cmn 绳子够长\end{exemple}
\begin{exemple}\jya ɯ-tɯ-rɲɟi ɲɯ-ɕaβ\cmn 够长\end{exemple}\end{entrée}

\begin{entrée}
\vedette{\hypertarget{Ⓔɕamɤr}{\papi{ ɕamɤr}}}\markboth{ɕamɤr}{}\classe{n}
\begin{définition}\fra une espèce de cerf\end{définition}
\begin{définition}\cmn 一种鹿\end{définition}
\end{entrée}

\begin{entrée}
\vedette{\hypertarget{Ⓔɕaɲaʁ}{\papi{ ɕaɲaʁ}}}\markboth{ɕaɲaʁ}{}\classe{n}
\begin{définition}\fra une espèce de cerf\end{définition}
\begin{définition}\cmn 一种黑色的鹿\end{définition}
\end{entrée}

\begin{entrée}
\vedette{\hypertarget{Ⓔɕaŋ}{\papi{ ɕaŋ}}}\markboth{ɕaŋ}{}
\classe{vs}
\paradigme{\textit{dir :} \jya tɤ-}
\begin{définition}\fra vieillir (corne d'un cerf)\end{définition}
\begin{définition}\cmn 老(鹿角)\end{définition}
\begin{exemple}\jya ɯ-ʁrɯ to-ɕaŋ\cmn 它的角老了\end{exemple}\end{entrée}

\begin{entrée}
\vedette{\hypertarget{Ⓔɕaŋβli}{\papi{ ɕaŋβli}}}\markboth{ɕaŋβli}{}\classe{n}
\begin{définition}\fra pousse d'arbre\end{définition}
\begin{définition}\cmn 树苗
\begin{déclaration} \étymologie{\papi{ɕiŋ}}\end{déclaration}\end{définition}
\begin{relation-sémantique}\confer{
 \papi{tɯβli}
}\end{relation-sémantique}
\end{entrée}

\begin{entrée}
\vedette{\hypertarget{Ⓔɕaŋdi}{\papi{ ɕaŋdi}}}\markboth{ɕaŋdi}{}\classe{adv}
\begin{définition}\fra d'ici jusqu'à là-bas\end{définition}
\begin{définition}\cmn 从这边直到那边(西边)\end{définition}\end{entrée}

\begin{entrée}
\vedette{\hypertarget{Ⓔɕaŋkɯ}{\papi{ ɕaŋkɯ}}}\markboth{ɕaŋkɯ}{}\classe{adv}
\begin{définition}\fra d'ici jusqu'à là-bas\end{définition}
\begin{définition}\cmn 从这边直到那边(东边)\end{définition}
\begin{exemple}\jya ki ɕaŋdi mbarkhom sɤtɕha ŋu, ki ɕaŋkɯ tɕɯχtsi sɤtɕha ŋu\cmn 从这里到边就是马尔康的地区,这里到那边就是卓克基的地区\end{exemple}\end{entrée}

\begin{entrée}
\vedette{\hypertarget{ⒺɕaŋloⒽ1Ⓗ1}{\papi{ ɕaŋlo}}}\markboth{ɕaŋlo}{}\homonyme{1}
\classe{adv}
\begin{définition}\fra d'ici vers l'amont\end{définition}
\begin{définition}\cmn 从这里一直往上游\end{définition}\begin{sous-entrée}
\vedette{\hypertarget{}{\papi{ ɕaŋlo}}}\markboth{ɕaŋlo}{}\classe{n}
\begin{définition}\fra place des anciens, au sud\end{définition}
\begin{définition}\cmn 老年人坐的地方;往南方\end{définition}
\end{sous-entrée}\end{entrée}

\begin{entrée}
\vedette{\hypertarget{Ⓔɕaŋpa}{\papi{ ɕaŋpa}}}\markboth{ɕaŋpa}{}\classe{adv}
\begin{définition}\fra d'ici jusqu'en bas\end{définition}
\begin{définition}\cmn 从这里一直往下\end{définition}
\end{entrée}

\begin{entrée}
\vedette{\hypertarget{Ⓔɕaŋpɕi}{\papi{ ɕaŋpɕi}}}\markboth{ɕaŋpɕi}{}\classe{postp}
\begin{définition}\fra à partir de ce moment\end{définition}
\begin{définition}\cmn 从那以后
\begin{déclaration} \étymologie{\papi{pʰʲis}}\end{déclaration}\end{définition}\end{entrée}

\begin{entrée}
\vedette{\hypertarget{Ⓔɕaŋpin}{\papi{ ɕaŋpin}}}\markboth{ɕaŋpin}{}\classe{n}
\begin{définition}\fra bord des habits tibétains\end{définition}
\begin{définition}\cmn (女式藏装)缝在外面的吊边\end{définition}\end{entrée}

\begin{entrée}
\vedette{\hypertarget{Ⓔɕaŋtaʁ}{\papi{ ɕaŋtaʁ}}}\markboth{ɕaŋtaʁ}{}\classe{adv}
\begin{définition}\fra au plus\end{définition}
\begin{définition}\cmn 最多\end{définition}
\begin{exemple}\jya tɯ-sŋi tɯ-ɣjɤn ɕaŋtaʁ smɤn kɤ-ndza mɯ́j-ra\cmn 一天不要吃药超过一次\end{exemple}\end{entrée}

\begin{entrée}
\vedette{\hypertarget{Ⓔɕaŋthi}{\papi{ ɕaŋthi}}}\markboth{ɕaŋthi}{}\classe{adv}
\begin{définition}\fra d'ici vers l'aval\end{définition}
\begin{définition}\cmn 从这里一直往上游\end{définition}
\end{entrée}

\begin{entrée}
\vedette{\hypertarget{Ⓔɕaŋtoʁ}{\papi{ ɕaŋtoʁ}}}\markboth{ɕaŋtoʁ}{}\classe{n}\acception{1}
\begin{définition}\fra fruit\end{définition}
\begin{définition}\cmn 果子
\begin{déclaration} \étymologie{\papi{ɕiŋ.tog}}\end{déclaration}\end{définition}
\begin{relation-sémantique}\synonyme{
\hyperlink{Ⓔsɯmat}{\textit{ \papi{sɯmat}}}
}\end{relation-sémantique}\acception{2}
\begin{définition}\fra arbre fruitier\end{définition}
\begin{définition}\cmn 果树\end{définition}\end{entrée}

\begin{entrée}
\vedette{\hypertarget{Ⓔɕaŋɯ}{\papi{ ɕaŋɯ}}}\markboth{ɕaŋɯ}{}
\classe{n}
\begin{définition}\fra période de chaleur des cerfs\end{définition}
\begin{définition}\cmn 鹿的发情期
\begin{déclaration} \étymologie{\papi{ɕʷa.ŋu}}\end{déclaration}\end{définition}\end{entrée}

\begin{entrée}
\vedette{\hypertarget{Ⓔɕaqajɯ}{\papi{ ɕaqajɯ}}}\markboth{ɕaqajɯ}{}\classe{n}
\begin{définition}\fra asticot\end{définition}
\begin{définition}\cmn 蛆\end{définition}
\begin{relation-sémantique}\confer{
\hyperlink{Ⓔɕa}{\textit{ \papi{ɕa}}}
}\end{relation-sémantique}
\begin{relation-sémantique}\confer{
\hyperlink{Ⓔqajɯ}{\textit{ \papi{qajɯ}}}
}\end{relation-sémantique}
\end{entrée}

\begin{entrée}
\vedette{\hypertarget{Ⓔɕar}{\papi{ ɕar}}}\markboth{ɕar}{}
\classe{vt}
\paradigme{\textit{dir :} \jya nɯ-}
\paradigme{\textit{dir :} \jya pɯ-}
\begin{définition}\fra chercher\end{définition}
\begin{définition}\cmn 寻找\end{définition}
\begin{exemple}\jya na-ɕar\cmn 他找了\end{exemple}
\begin{exemple}\jya kɯki kɯ-fse kɯ-pe kɤ-ɕar me\cmn 找不到这么好的\end{exemple}
\begin{exemple}\jya a-laχtɕha nɯ-nɯ-βde-t-a nɯ kɤ-ɕar mtam-a\cmn 我会找到我丢失的东西\end{exemple}
\begin{exemple}\jya nɯ-ɕar-a tɕe pɯ-mto-t-a\cmn 我找到了\end{exemple}\begin{sous-entrée}
\vedette{\hypertarget{}{\papi{ rɤɕar}}}\markboth{rɤɕar}{}\classe{vi}
\begin{définition}\ 
\begin{déclaration}\grammar{apass}\end{déclaration}\end{définition}
\begin{définition}\fra chercher quelque chose\end{définition}
\begin{définition}\cmn 找东西\end{définition}
\end{sous-entrée}\begin{sous-entrée}
\vedette{\hypertarget{}{\papi{ sɤɕar}}}\markboth{sɤɕar}{}\classe{vi}
\begin{définition}\ 
\begin{déclaration}\grammar{apass}\end{déclaration}\end{définition}
\begin{définition}\fra chercher quelqu'un\end{définition}
\begin{définition}\cmn 找人\end{définition}
\begin{relation-sémantique}\confer{
\hyperlink{Ⓔnɤɕɯɕar}{\textit{ \papi{nɤɕɯɕar}}}
}\end{relation-sémantique}
\begin{relation-sémantique}\confer{
\hyperlink{Ⓔnɤɕarlar}{\textit{ \papi{nɤɕarlar}}}
}\end{relation-sémantique}
\end{sous-entrée}\end{entrée}

\begin{entrée}
\vedette{\hypertarget{Ⓔɕaʁja}{\papi{ ɕaʁja}}}\markboth{ɕaʁja}{}\classe{intj}
\begin{définition}\fra bien fait (pour lui)\end{définition}
\begin{définition}\cmn 活该\end{définition}\end{entrée}

\begin{entrée}
\vedette{\hypertarget{Ⓔɕaʁwɯ}{\papi{ ɕaʁwɯ}}}\markboth{ɕaʁwɯ}{}
\classe{n}
\begin{définition}\fra navet séché\end{définition}
\begin{définition}\cmn 干了的芜菁根
\end{définition}\end{entrée}

\begin{entrée}
\vedette{\hypertarget{Ⓔɕaχpu}{\papi{ ɕaχpu}}}\markboth{ɕaχpu}{}
\classe{n}
\begin{définition}\fra ami\end{définition}
\begin{définition}\cmn 朋友
\begin{déclaration} \étymologie{\papi{ɕag.po}}\end{déclaration}\end{définition}
\begin{exemple}\jya a-ɕaχpu ŋu\cmn 是我的朋友\end{exemple}\end{entrée}

\begin{entrée}
\vedette{\hypertarget{Ⓔɕɤci}{\papi{ ɕɤci}}}\markboth{ɕɤci}{}\classe{n}
\begin{définition}\fra soupe\end{définition}
\begin{définition}\cmn 汤\end{définition}
\begin{relation-sémantique}\confer{
\hyperlink{Ⓔɕa}{\textit{ \papi{ɕa}}}
}\end{relation-sémantique}
\begin{relation-sémantique}\confer{
\hyperlink{Ⓔtɯ-ci}{\textit{ \papi{tɯ-ci}}}
}\end{relation-sémantique}
\end{entrée}

\begin{entrée}
\vedette{\hypertarget{Ⓔɕɤfɕo}{\papi{ ɕɤfɕo}}}\markboth{ɕɤfɕo}{} (\variante{ɕɤxɕo}) 
\classe{adv}
\begin{définition}\fra ces derniers jours\end{définition}
\begin{définition}\cmn 这几天\end{définition}\end{entrée}

\begin{entrée}
\vedette{\hypertarget{ⒺɕɤɣⒽ2}{\papi{ ɕɤɣ}}}\markboth{ɕɤɣ}{}\homonyme{2}\classe{n}
\begin{définition}\fra genévrier\end{définition}
\begin{définition}\cmn 柏树
\begin{déclaration} \étymologie{\papi{ɕug}}\end{déclaration}\end{définition}
\begin{exemple}\jya ɕɤɣ nɯ si kɯ-jpum kɯ-mbro ci ŋu, ɯ-ru nɯ tɯrgi sthɯci mɤ-jpum, mɤ-mbro, ɯ-rtaʁ nɯ tɯrgi ɯ-rtaʁ sɤznɤ jpum cho rɲɟi, taʁ tɤ-ari ɯ-jija tu-xtɯt ŋu, ɕɤɣ tɯrgi sɤznɤ ɯ-rtaʁ dɤn, ɯ-jwaʁ nɯ alɯlju, aɣɯrtɯrtaʁ, ɯ-mdoʁ nɯ ldʑaŋsɤr ŋu. ɕɤɣ ɣɯ ɯ-ru nɯ li tɤrɤm wuma ʑo kɯ-ʑru kɤ-sɯ-pa ŋu ma ɯ-rɯmu tu, ɯ-mdoʁ mpɕɤr, ɕɤɣ tɯ-phɯ tɕu ɕoŋtɕa lɤβdɤlɤŋu-rzɯɣ ma kɤ-ʁndzɤr mɤ-rtaʁ. ɯ-jwaʁ nɯ tɤ-fsaŋ spa stu kɯ-ʑru ŋu, pjɯ́-wɣ-tɕɤβ tɕe, ɯ-dɯχɯn wuma mɯm. ɯ-rqhu rɕɯrɕɯβ ʑo pa, aɣɯrnɯɕɯr. qartsɯmɤftɕar ɯ-mdoʁ ɲɯ-nɤsci mɤ-cha.\cmn 柏树是长得又粗又高的树种,树干没有杉树那么粗和高,枝桠比杉树的要粗和长,枝桠越是长在上面,就越短。柏树的枝桠比杉树的要多,叶子是圆柱形的,长很多枝桠,是淡绿色的,柏树也是制造木板的好材料,有条纹,颜色很美观。一棵柏树只能锯成四五节木料。叶子是烧香的最好的材料,烧了味道很香,树皮很粗,带有一点红色。颜色一年四季都不会变。\end{exemple}
\begin{relation-sémantique}\confer{
\hyperlink{Ⓔnɯɕɤɣ}{\textit{ \papi{nɯɕɤɣ}}}
}\end{relation-sémantique}\end{entrée}

\begin{entrée}
\vedette{\hypertarget{ⒺɕɤɣⒽ1}{\papi{ ɕɤɣ}}}\markboth{ɕɤɣ}{}\homonyme{1}\classe{vs}
\paradigme{\textit{dir :} \jya tɤ-}
\begin{définition}\fra nouveau\end{définition}
\begin{définition}\cmn 新\end{définition}
\begin{exemple}\jya jiɕqha laχtɕha ɲɯ-ɕɤɣ\cmn 那个东西是新的\end{exemple}
\begin{relation-sémantique}\antonyme{
\hyperlink{Ⓔmbe}{\textit{ \papi{mbe}}}
}\end{relation-sémantique}\end{entrée}

\begin{entrée}
\vedette{\hypertarget{Ⓔɕɤɣpɣa}{\papi{ ɕɤɣpɣa}}}\markboth{ɕɤɣpɣa}{}\classe{n}
\begin{définition}\fra Turdus sp. (rubrocanus, kessleri)\end{définition}
\begin{définition}\cmn 鸫\end{définition}
\begin{relation-sémantique}\confer{
\hyperlink{ⒺɕɤɣⒽ2}{\textit{ \papi{ɕɤɣ2}}}
}\end{relation-sémantique}
\begin{relation-sémantique}\confer{
\hyperlink{Ⓔpɣa}{\textit{ \papi{pɣa}}}
}\end{relation-sémantique}
\end{entrée}

\begin{entrée}
\vedette{\hypertarget{Ⓔɕɤɣrum}{\papi{ ɕɤɣrum}}}\markboth{ɕɤɣrum}{} (\variante{ɕoʁɣrum}) \classe{n}
\begin{définition}\fra espèce de sarrasin\end{définition}
\begin{définition}\cmn 甜荞\end{définition}
\begin{relation-sémantique}\confer{
\hyperlink{Ⓔɕoʁ}{\textit{ \papi{ɕoʁ}}}
}\end{relation-sémantique}
\begin{relation-sémantique}\confer{
\hyperlink{Ⓔwɣrum}{\textit{ \papi{wɣrum}}}
}\end{relation-sémantique}\end{entrée}

\begin{entrée}
\vedette{\hypertarget{Ⓔɕɤjaʁ}{\papi{ ɕɤjaʁ}}}\markboth{ɕɤjaʁ}{}\classe{n}
\begin{définition}\fra viande des membres antérieurs\end{définition}
\begin{définition}\cmn 牲畜前腿的肉\end{définition}
\begin{relation-sémantique}\confer{
\hyperlink{Ⓔɕa}{\textit{ \papi{ɕa}}}
}\end{relation-sémantique}
\begin{relation-sémantique}\confer{
\hyperlink{Ⓔtɯ-jaʁ}{\textit{ \papi{tɯ-jaʁ}}}
}\end{relation-sémantique}
\end{entrée}

\begin{entrée}
\vedette{\hypertarget{Ⓔɕɤku}{\papi{ ɕɤku}}}\markboth{ɕɤku}{}\classe{n}
\begin{définition}\fra tête coupée\end{définition}
\begin{définition}\cmn 被砍掉的头\end{définition}
\begin{exemple}\jya paʁ ɯ-ɕɤku\cmn (被砍掉的)猪头\end{exemple}
\begin{exemple}\jya ɕɤku ɯ-ɕki ɕɤrna\cmn 对着头说耳朵的坏话(打草惊蛇)\end{exemple}
\begin{relation-sémantique}\confer{
\hyperlink{Ⓔtɯ-ku}{\textit{ \papi{tɯ-ku}}}
}\end{relation-sémantique}
\begin{relation-sémantique}\confer{
\hyperlink{Ⓔɕɤjaʁ}{\textit{ \papi{ɕɤjaʁ}}}
}\end{relation-sémantique}
\begin{relation-sémantique}\confer{
\hyperlink{Ⓔɕɤmi}{\textit{ \papi{ɕɤmi}}}
}\end{relation-sémantique}
\begin{relation-sémantique}\confer{
 \papi{ɕɤrn}
}\end{relation-sémantique}\end{entrée}

\begin{entrée}
\vedette{\hypertarget{Ⓔɕɤkhe}{\papi{ ɕɤkhe}}}\markboth{ɕɤkhe}{}
\classe{n}
\begin{définition}\fra viande maigre\end{définition}
\begin{définition}\cmn 瘦肉\end{définition}\end{entrée}

\begin{entrée}
\vedette{\hypertarget{Ⓔɕɤkhoz}{\papi{ ɕɤkhoz}}}\markboth{ɕɤkhoz}{}\classe{n}
\begin{définition}\fra sac que l'on porte en bandoulière\end{définition}
\begin{définition}\cmn 挎包\end{définition}
\begin{relation-sémantique}\antonyme{
\hyperlink{Ⓔphɯrkhɯɣ}{\textit{ \papi{phɯrkhɯɣ}}}
}\end{relation-sémantique}\end{entrée}

\begin{entrée}
\vedette{\hypertarget{Ⓔɕɤkhrɯ}{\papi{ ɕɤkhrɯ}}}\markboth{ɕɤkhrɯ}{}\classe{n}
\begin{définition}\fra viande séchée\end{définition}
\begin{définition}\cmn 牛肉干\end{définition}
\begin{relation-sémantique}\confer{
\hyperlink{Ⓔɕa}{\textit{ \papi{ɕa}}}
}\end{relation-sémantique}
\begin{relation-sémantique}\confer{
\hyperlink{Ⓔkhrɯ}{\textit{ \papi{khrɯ}}}
}\end{relation-sémantique}
\end{entrée}

\begin{entrée}
\vedette{\hypertarget{Ⓔɕɤldʐa}{\papi{ ɕɤldʐa}}}\markboth{ɕɤldʐa}{}\classe{n}
\begin{définition}\fra viande sanglante\end{définition}
\begin{définition}\cmn 血淋淋的肉块\end{définition}
\begin{exemple}\jya ɯ-mɲaʁ ɯ-ŋgɯ ɕɤldʐa kɯ-fse ko-ɣi\cmn 他眼睛发红(几乎看不到眼珠)\end{exemple}
\end{entrée}

\begin{entrée}
\vedette{\hypertarget{Ⓔɕɤltɕhɯz}{\papi{ ɕɤltɕhɯz}}}\markboth{ɕɤltɕhɯz}{}
\classe{n}
\begin{définition}\fra peler à la base des ongles\end{définition}
\begin{définition}\cmn 指甲的底部脱皮\end{définition}\end{entrée}

\begin{entrée}
\vedette{\hypertarget{Ⓔɕɤltsaʁ}{\papi{ ɕɤltsaʁ}}}\markboth{ɕɤltsaʁ}{}\classe{n}
\begin{définition}\fra habit d'homme en cuir\end{définition}
\begin{définition}\cmn 皮袄(男生穿)\end{définition}
\begin{exemple}\jya tɯ-rcu ɕɤltsaʁ\cmn 皮袄\end{exemple}\end{entrée}

\begin{entrée}
\vedette{\hypertarget{Ⓔɕɤmcɤthoʁ}{\papi{ ɕɤmcɤthoʁ}}}\markboth{ɕɤmcɤthoʁ}{}\classe{n}
\begin{définition}\ 
\begin{déclaration}\grammar{n.lieu}\end{déclaration}\end{définition}
\begin{définition}\fra nom de lieu\end{définition}
\begin{définition}\cmn 龙头滩\end{définition}\end{entrée}

\begin{entrée}
\vedette{\hypertarget{Ⓔɕɤmi}{\papi{ ɕɤmi}}}\markboth{ɕɤmi}{}\classe{n}
\begin{définition}\fra viande des membres postérieurs\end{définition}
\begin{définition}\cmn 牲畜后腿的肉\end{définition}
\begin{relation-sémantique}\confer{
\hyperlink{Ⓔɕa}{\textit{ \papi{ɕa}}}
}\end{relation-sémantique}
\begin{relation-sémantique}\confer{
\hyperlink{Ⓔtɯ-mi}{\textit{ \papi{tɯ-mi}}}
}\end{relation-sémantique}
\end{entrée}

\begin{entrée}
\vedette{\hypertarget{Ⓔɕɤmiɕtʂɤt}{\papi{ ɕɤmiɕtʂɤt}}}\markboth{ɕɤmiɕtʂɤt}{}
\classe{n}
\begin{définition}\fra crochet utilisé pour accrocher la viande cuite\end{définition}
\begin{définition}\cmn 专门用来钩住熟肉的铁钩\end{définition}
\begin{relation-sémantique}\confer{
\hyperlink{ⒺɕomⒽ1}{\textit{ \papi{ɕom1}}}
}\end{relation-sémantique}\end{entrée}

\begin{entrée}
\vedette{\hypertarget{Ⓔɕɤmiŋoʁ}{\papi{ ɕɤmiŋoʁ}}}\markboth{ɕɤmiŋoʁ}{}
\classe{n}
\begin{définition}\fra crochet de cheminée\end{définition}
\begin{définition}\cmn 火钩\end{définition}
\begin{relation-sémantique}\synonyme{
\hyperlink{Ⓔsmɯʁjoʁ}{\textit{ \papi{smɯʁjoʁ}}}
}\end{relation-sémantique}
\begin{relation-sémantique}\confer{
\hyperlink{ⒺɕomⒽ1}{\textit{ \papi{ɕom1}}}
}\end{relation-sémantique}
\begin{relation-sémantique}\confer{
 \papi{tɤjŋoʁ}
}\end{relation-sémantique}\end{entrée}

\begin{entrée}
\vedette{\hypertarget{Ⓔɕɤmloʁ}{\papi{ ɕɤmloʁ}}}\markboth{ɕɤmloʁ}{}\classe{n}
\begin{définition}\fra heurtoir\end{définition}
\begin{définition}\cmn 门环\end{définition}
\begin{exemple}\jya ɕɤmloʁ nɯ ko-ndo tɕe kɯm lo-nɤrkhɯrkhɯβ\cmn 他用门环敲门\end{exemple}
\begin{relation-sémantique}\confer{
\hyperlink{ⒺɕomⒽ1}{\textit{ \papi{ɕom}}}
}\end{relation-sémantique}\end{entrée}

\begin{entrée}
\vedette{\hypertarget{Ⓔɕɤmscoʁ}{\papi{ ɕɤmscoʁ}}}\markboth{ɕɤmscoʁ}{}\classe{n}
\begin{définition}\fra mors\end{définition}
\begin{définition}\cmn 马嚼子\end{définition}
\begin{relation-sémantique}\confer{
\hyperlink{ⒺɕomⒽ1}{\textit{ \papi{ɕom1}}}
}\end{relation-sémantique}
\begin{relation-sémantique}\confer{
\hyperlink{Ⓔscoʁ}{\textit{ \papi{scoʁ}}}
}\end{relation-sémantique}\end{entrée}

\begin{entrée}
\vedette{\hypertarget{Ⓔɕɤmtshoʁ}{\papi{ ɕɤmtshoʁ}}}\markboth{ɕɤmtshoʁ}{}\classe{n}
\begin{définition}\fra clou en fer\end{définition}
\begin{définition}\cmn 铁钉\end{définition}
\end{entrée}

\begin{entrée}
\vedette{\hypertarget{Ⓔɕɤmɯɣdɯ}{\papi{ ɕɤmɯɣdɯ}}}\markboth{ɕɤmɯɣdɯ}{}
\classe{n}
\begin{définition}\fra arme à feu\end{définition}
\begin{définition}\cmn 枪\end{définition}
\begin{relation-sémantique}\confer{
\hyperlink{ⒺɕomⒽ1}{\textit{ \papi{ɕom1}}}
}\end{relation-sémantique}\end{entrée}

\begin{entrée}
\vedette{\hypertarget{Ⓔɕɤnthɤβ}{\papi{ ɕɤnthɤβ}}}\markboth{ɕɤnthɤβ}{}\classe{n}
\begin{définition}\fra robe de moine\end{définition}
\begin{définition}\cmn 和尚服装的一种,穿在腰上\end{définition}\end{entrée}

\begin{entrée}
\vedette{\hypertarget{Ⓔɕɤntsɯt}{\papi{ ɕɤntsɯt}}}\markboth{ɕɤntsɯt}{}\classe{n}
\begin{définition}\fra habit de femme en tissu\end{définition}
\begin{définition}\cmn 布;呢子制成的女装\end{définition}
\end{entrée}

\begin{entrée}
\vedette{\hypertarget{Ⓔɕɤŋi}{\papi{ ɕɤŋi}}}\markboth{ɕɤŋi}{}
\classe{n}
\begin{définition}\fra viande crue\end{définition}
\begin{définition}\cmn 生肉\end{définition}\end{entrée}

\begin{entrée}
\vedette{\hypertarget{Ⓔɕɤr}{\papi{ ɕɤr}}}\markboth{ɕɤr}{}
\classe{n}
\begin{définition}\fra soir\end{définition}
\begin{définition}\cmn 晚上\end{définition}\end{entrée}

\begin{entrée}
\vedette{\hypertarget{Ⓔɕɤrɤɕa}{\papi{ ɕɤrɤɕa}}}\markboth{ɕɤrɤɕa}{}
\classe{n}
\begin{définition}\fra viande du ventre du cochon\end{définition}
\begin{définition}\cmn 猪的肚皮的内层肉\end{définition}\end{entrée}

\begin{entrée}
\vedette{\hypertarget{Ⓔɕɤrkha}{\papi{ ɕɤrkha}}}\markboth{ɕɤrkha}{}\classe{n}
\begin{définition}\fra aube\end{définition}
\begin{définition}\cmn 黎明\end{définition}
\begin{exemple}\jya ɕɤrkha ɲɤ-ɴɢraʁ\cmn 已经破晓了\end{exemple}
\begin{relation-sémantique}\confer{
 \papi{ɕɤrkha,ɴɢraʁ}
}\end{relation-sémantique}
\end{entrée}

\begin{entrée}
\vedette{\hypertarget{Ⓔɕɤrma}{\papi{ ɕɤrma}}}\markboth{ɕɤrma}{}\classe{n}
\begin{définition}\fra brigand\end{définition}
\begin{définition}\cmn 土匪\end{définition}\end{entrée}

\begin{entrée}
\vedette{\hypertarget{Ⓔɕɤrmasŋi}{\papi{ ɕɤrmasŋi}}}\markboth{ɕɤrmasŋi}{}\classe{adv}
\begin{définition}\fra le jour et la nuit\end{définition}
\begin{définition}\cmn 日夜\end{définition}
\begin{relation-sémantique}\confer{
\hyperlink{Ⓔɕɤr}{\textit{ \papi{ɕɤr}}}
}\end{relation-sémantique}
\begin{relation-sémantique}\confer{
\hyperlink{Ⓔtɯ-sŋi}{\textit{ \papi{tɯ-sŋi}}}
}\end{relation-sémantique}\end{entrée}

\begin{entrée}
\vedette{\hypertarget{Ⓔɕɤrna}{\papi{ ɕɤrna}}}\markboth{ɕɤrna}{}\classe{n}
\begin{définition}\fra oreille coupée\end{définition}
\begin{définition}\cmn 被砍掉的耳朵\end{définition}
\begin{exemple}\jya paʁ ɯ-ɕɤrna\cmn 被砍掉的猪耳朵\end{exemple}
\begin{relation-sémantique}\confer{
\hyperlink{Ⓔɕɤku}{\textit{ \papi{ɕɤku}}}
}\end{relation-sémantique}
\begin{relation-sémantique}\confer{
\hyperlink{Ⓔɕɤjaʁ}{\textit{ \papi{ɕɤjaʁ}}}
}\end{relation-sémantique}
\begin{relation-sémantique}\confer{
\hyperlink{Ⓔɕɤmi}{\textit{ \papi{ɕɤmi}}}
}\end{relation-sémantique}
\begin{relation-sémantique}\confer{
\hyperlink{Ⓔtɯ-rna}{\textit{ \papi{tɯ-rna}}}
}\end{relation-sémantique}\end{entrée}

\begin{entrée}
\vedette{\hypertarget{Ⓔɕɤrɯ}{\papi{ ɕɤrɯ}}}\markboth{ɕɤrɯ}{}
\classe{n}
\begin{définition}\fra os\end{définition}
\begin{définition}\cmn 骨头\end{définition}\end{entrée}

\begin{entrée}
\vedette{\hypertarget{Ⓔɕɤrwa}{\papi{ ɕɤrwa}}}\markboth{ɕɤrwa}{}\classe{n}
\begin{définition}\fra musulman\end{définition}
\begin{définition}\cmn 回族
\begin{déclaration} \étymologie{\papi{ɕar.pa}}\end{déclaration}\end{définition}
\end{entrée}

\begin{entrée}
\vedette{\hypertarget{Ⓔɕɤrzaŋ}{\papi{ ɕɤrzaŋ}}}\markboth{ɕɤrzaŋ}{}\classe{n}
\begin{définition}\fra casserole en cuivre\end{définition}
\begin{définition}\cmn 铜锅\end{définition}
\end{entrée}

\begin{entrée}
\vedette{\hypertarget{Ⓔɕɤsca}{\papi{ ɕɤsca}}}\markboth{ɕɤsca}{}\classe{n}
\begin{définition}\fra cerf (mâle), bovidé de couleur noire dont le milieu du corps est blanc\end{définition}
\begin{définition}\cmn 公鹿;全身黑色、腰白色的牛\end{définition}\end{entrée}

\begin{entrée}
\vedette{\hypertarget{Ⓔɕɤt}{\papi{ ɕɤt}}}\markboth{ɕɤt}{}
\classe{vi}
\paradigme{\textit{dir :} \jya nɯ-}
\begin{définition}\fra s'habituer\end{définition}
\begin{définition}\cmn 习惯\end{définition}
\begin{exemple}\jya kɯki aʑo ɕat-a, tɯ-ɕɤt, ɯʑo ɕɤt\cmn 我习惯这个,你习惯,他习惯\end{exemple}
\begin{exemple}\jya ɯʑo ɲɤ-ɕɤt, ɯʑo nɯ-ɕɤt\cmn 他习惯了\end{exemple}
\begin{exemple}\jya tɯ-ŋga kɤ-ŋga nɯ kɤ-ɕɤt ci tu wo\cmn (人)穿衣服是有(自己的)习惯的\end{exemple}
\begin{exemple}\jya ɯʑo tɯ-ŋga kɯ-jaʁ kɤ-ŋga ɲɤ-ɕɤt\cmn 他习惯穿很厚的衣服\end{exemple}
\begin{exemple}\jya aʑo tɯcɯrqɯ kɤ-tshi nɯ-ɕat-a\cmn 我习惯喝冷水\end{exemple}\end{entrée}

\begin{entrée}
\vedette{\hypertarget{Ⓔɕɤwɤr}{\papi{ ɕɤwɤr}}}\markboth{ɕɤwɤr}{}
\classe{n}
\begin{définition}\fra conjonctivite\end{définition}
\begin{définition}\cmn 结膜炎\end{définition}\end{entrée}

\begin{entrée}
\vedette{\hypertarget{Ⓔɕɤχcɤl}{\papi{ ɕɤχcɤl}}}\markboth{ɕɤχcɤl}{}\classe{n}
\begin{définition}\fra minuit\end{définition}
\begin{définition}\cmn 半夜\end{définition}
\begin{relation-sémantique}\confer{
\hyperlink{Ⓔɕɤr}{\textit{ \papi{ɕɤr}}}
}\end{relation-sémantique}
\begin{relation-sémantique}\confer{
\hyperlink{Ⓔɯ-χcɤl}{\textit{ \papi{ɯ-χcɤl}}}
}\end{relation-sémantique}\end{entrée}

\begin{entrée}
\vedette{\hypertarget{Ⓔɕe}{\papi{ ɕe}}}\markboth{ɕe}{}\classe{vi}
\paradigme{\textit{dir :} \jya \_}
\paradigme{\textit{past stem :} \jya ari}
\paradigme{\textit{construction :} \jya participe sujet}\acception{1}
\begin{définition}\fra aller\end{définition}
\begin{définition}\cmn 去\end{définition}
\begin{exemple}\jya jɤ-ari-a, jɤ-ari\cmn 我走了,他走了\end{exemple}
\begin{exemple}\jya kha tu-ɕe-a ŋu\cmn 我在回家的路上\end{exemple}\acception{2}
\paradigme{\textit{dir :} \jya lɤ-}
\begin{définition}\fra entrer\end{définition}
\begin{définition}\cmn 进去\end{définition}
\begin{exemple}\jya kɯm tu-ci-a tɕe kha ɯ-ŋgɯ lɤ-ari-a tɕe a-pɯ-ŋu\cmn 我开门,进了门再说\end{exemple}
\begin{exemple}\jya a-mɲaʁ ɯ-ŋgɯ thɯci ko-ɕe\cmn 我眼睛里进了什么东西\end{exemple}\acception{3}
\paradigme{\textit{dir :} \jya thɯ-}
\begin{définition}\fra sortir\end{définition}
\begin{définition}\cmn 出去\end{définition}
\begin{exemple}\jya tɯ-ɕna ɯ-ŋgɯ ri tɯ-skɤt chɯ-ɕe ɲɯ-ra\cmn 这个音要从鼻子发出来\end{exemple}\acception{4}
\paradigme{\textit{dir :} \jya thɯ-}
\begin{définition}\fra couler\end{définition}
\begin{définition}\cmn 流(水)\end{définition}
\begin{exemple}\jya tɯ-ci chɯ-ɕe ɲɯ-ŋu\cmn 水在流淌\end{exemple}\acception{5}
\paradigme{\textit{dir :} \jya \_}
\begin{définition}\fra être (aligné, étendu) le long de\end{définition}
\begin{définition}\cmn 排过去\end{définition}
\begin{exemple}\jya tɤrɲɟo nɯ jɯɣi ʁɟa ʑo ku-ɕe ɲɯ-ŋu\cmn 架子上排过去的全都是书\end{exemple}\acception{6}
\paradigme{\textit{dir :} \jya nɯ-}
\begin{définition}\fra se passer (temps)\end{définition}
\begin{définition}\cmn 过(时间)\end{définition}
\begin{exemple}\jya kɯmŋu-xpa ɲɤ-ɕe\cmn 过了五年\end{exemple}
\begin{relation-sémantique}\confer{
\hyperlink{Ⓔari}{\textit{ \papi{ari}}}
}\end{relation-sémantique}
\begin{relation-sémantique}\confer{
\hyperlink{Ⓔsɯxɕe}{\textit{ \papi{sɯxɕe}}}
}\end{relation-sémantique}
\begin{relation-sémantique}\confer{
 \papi{tɯ-sɯm,ɕe}
}\end{relation-sémantique}
\begin{relation-sémantique}\confer{
 \papi{tɯ-sroʁ,ɕe}
}\end{relation-sémantique}
\begin{relation-sémantique}\confer{
 \papi{tɯ-βri,ɕe}
}\end{relation-sémantique}\begin{sous-entrée}
\vedette{\hypertarget{}{\papi{ ɯ-taʁ ɕe}}}\markboth{ɯ-taʁ ɕe}{}
\begin{définition}\fra faire comme si\end{définition}
\begin{définition}\cmn 当做是\end{définition}
\begin{exemple}\jya aʑo jisŋi tɤrca pɯ-me-a tɕe, tɤ-kɯ-nɯna ɯ-taʁ nɯ-ari\cmn 我今天没有跟他们一起去,就(大家就)当做我休息(其实不是)\end{exemple}
\end{sous-entrée}\end{entrée}

\begin{entrée}
\vedette{\hypertarget{Ⓔɕɣaʁɕɣaʁ}{\papi{ ɕɣaʁɕɣaʁ}}}\markboth{ɕɣaʁɕɣaʁ}{}
\classe{idph.2}
\begin{définition}\fra pointus et brillants (crocs)\end{définition}
\begin{définition}\cmn 形容獠牙等尖而光滑的样子\end{définition}
\begin{exemple}\jya ɯ-ndzɣi ɕɣaʁɕɣaʁ ʑo pa\cmn 它的獠牙又尖又光滑\end{exemple}\end{entrée}

\begin{entrée}
\vedette{\hypertarget{Ⓔɕɣɤχa}{\papi{ ɕɣɤχa}}}\markboth{ɕɣɤχa}{}\classe{n}
\begin{définition}\fra auquel il manque une dent\end{définition}
\begin{définition}\cmn 少了一颗牙齿的(人)\end{définition}
\begin{relation-sémantique}\confer{
\hyperlink{Ⓔtɯ-ɕɣa}{\textit{ \papi{tɯ-ɕɣa}}}
}\end{relation-sémantique}
\begin{relation-sémantique}\confer{
\hyperlink{Ⓔaχa}{\textit{ \papi{aχa}}}
}\end{relation-sémantique}\end{entrée}

\begin{entrée}
\vedette{\hypertarget{Ⓔɕɣɤz}{\papi{ ɕɣɤz}}}\markboth{ɕɣɤz}{}
\classe{vt}
\paradigme{\textit{dir :} \jya jɤ-}
\begin{définition}\fra rendre\end{définition}
\begin{définition}\cmn 还东西;退回\end{définition}
\begin{exemple}\jya ɯʑo kɯ ja-ɕɣɤz\cmn 他还了\end{exemple}
\begin{exemple}\jya kɯki mɯ́j-ra tɕe, kɤ-ɕɣɤz ɲɯ-ra\cmn 不再需要的话请还(给我)\end{exemple}
\begin{exemple}\jya nɯ-kɯ-ɕɣaz-a\cmn 你还给我了\end{exemple}
\begin{relation-sémantique}\synonyme{
\hyperlink{Ⓔfsɯɣ}{\textit{ \papi{fsɯɣ}}}
}\end{relation-sémantique}\end{entrée}

\begin{entrée}
\vedette{\hypertarget{Ⓔɕico}{\papi{ ɕico}}}\markboth{ɕico}{}\classe{n}
\begin{définition}\fra plastique\end{définition}
\begin{définition}\cmn 塑料\end{définition}
\end{entrée}

\begin{entrée}
\vedette{\hypertarget{Ⓔɕintɕhi}{\papi{ ɕintɕhi}}}\markboth{ɕintɕhi}{}\classe{n}
\begin{définition}\fra jour de repos\end{définition}
\begin{définition}\cmn (放的)假
\begin{déclaration} \étymologie{\papi{\stylefn{星期}}}\end{déclaration}\end{définition}
\begin{exemple}\jya jisŋi a-ɕintɕhi ŋu\cmn 我今天放假\end{exemple}\end{entrée}

\begin{entrée}
\vedette{\hypertarget{Ⓔɕirʁaʁ}{\papi{ ɕirʁaʁ}}}\markboth{ɕirʁaʁ}{}\classe{n}
\begin{définition}\fra une espèce de champignon\end{définition}
\begin{définition}\cmn 青冈菌\end{définition}
\begin{exemple}\jya ɕirʁaʁ nɯ ɕkrɤz kɯ-xtɕi tsa ɯ-ŋgɯ tu-ɬoʁ ŋu, ɯ-mgɯrqhu ɣɯrni ɯ-pa cho ɯ-ru nɯ ra wɣrum, kɤ-ndza mɯm, kɯ-xtɕɯ-xtɕi qiaβ\cmn 青冈菌生长在比较矮小的青冈树林里,背面粉红色,下部和菌柄白色。好吃,略苦。\end{exemple}\end{entrée}

\begin{entrée}
\vedette{\hypertarget{Ⓔɕku}{\papi{ ɕku}}}\markboth{ɕku}{}\classe{n}
\begin{définition}\fra oignon\end{définition}
\begin{définition}\cmn 葱\end{définition}
\end{entrée}

\begin{entrée}
\vedette{\hypertarget{Ⓔɕkala}{\papi{ ɕkala}}}\markboth{ɕkala}{}
\classe{n}
\begin{définition}\fra boiteux\end{définition}
\begin{définition}\cmn 跛子\end{définition}
\begin{relation-sémantique}\synonyme{
\hyperlink{Ⓔʑɤwu}{\textit{ \papi{ʑɤwu}}}
}\end{relation-sémantique}
\begin{relation-sémantique}\confer{
\hyperlink{Ⓔaɕkala}{\textit{ \papi{aɕkala}}}
}\end{relation-sémantique}\end{entrée}

\begin{entrée}
\vedette{\hypertarget{Ⓔɕkatmbri}{\papi{ ɕkatmbri}}}\markboth{ɕkatmbri}{}\classe{n}
\begin{définition}\fra corde pour attacher les marchandises\end{définition}
\begin{définition}\cmn 绑驮子用的绳子\end{définition}
\begin{relation-sémantique}\confer{
\hyperlink{Ⓔtɯ-ɕkat}{\textit{ \papi{tɯ-ɕkat}}}
}\end{relation-sémantique}
\begin{relation-sémantique}\confer{
\hyperlink{Ⓔtɯmbri}{\textit{ \papi{tɯmbri}}}
}\end{relation-sémantique}
\end{entrée}

\begin{entrée}
\vedette{\hypertarget{Ⓔɕkɤbɯ}{\papi{ ɕkɤbɯ}}}\markboth{ɕkɤbɯ}{}
\classe{n}
\begin{définition}\fra pain aux poireaux\end{définition}
\begin{définition}\cmn 韭菜的包子\end{définition}
\begin{relation-sémantique}\confer{
\hyperlink{Ⓔɕku}{\textit{ \papi{ɕku}}}
}\end{relation-sémantique}\end{entrée}

\begin{entrée}
\vedette{\hypertarget{Ⓔɕkɤfkri}{\papi{ ɕkɤfkri}}}\markboth{ɕkɤfkri}{}\classe{n}
\begin{définition}\fra sauce à l'air\end{définition}
\begin{définition}\cmn 大蒜沾水\end{définition}
\begin{relation-sémantique}\confer{
\hyperlink{Ⓔɕku}{\textit{ \papi{ɕku}}}
}\end{relation-sémantique}\end{entrée}

\begin{entrée}
\vedette{\hypertarget{Ⓔɕkɤɣɕkɤɣ}{\papi{ ɕkɤɣɕkɤɣ}}}\markboth{ɕkɤɣɕkɤɣ}{}\classe{idph.2}
\begin{définition}\fra grand, maigre et bossu\end{définition}
\begin{définition}\cmn 形容又高又瘦又驼背的样子\end{définition}
\begin{relation-sémantique}\synonyme{
\hyperlink{Ⓔʑgɤβʑgɤβ}{\textit{ \papi{ʑgɤβʑgɤβ}}}
}\end{relation-sémantique}\end{entrée}

\begin{entrée}
\vedette{\hypertarget{Ⓔɕkɤɣnɤɕkɤɣ}{\papi{ ɕkɤɣnɤɕkɤɣ}}}\markboth{ɕkɤɣnɤɕkɤɣ}{}\classe{idph.3}
\begin{définition}\fra en saccade\end{définition}
\begin{définition}\cmn 一震一震\end{définition}
\begin{exemple}\jya ɕkɤɣnɤɕkɤɣ pa-xtsɯ\cmn 他一震一震地砸了\end{exemple}
\begin{exemple}\jya ɯ-mi ɲɯ-mŋɤm tɕe, ɕkɤɣnɤɕkɤɣ kɤ-anɯri\cmn 他脚很痛,蹒跚地回去了\end{exemple}
\begin{relation-sémantique}\confer{
\hyperlink{Ⓔɣɤɕkɤɣɕkɤɣ}{\textit{ \papi{ɣɤɕkɤɣɕkɤɣ}}}
}\end{relation-sémantique}\end{entrée}

\begin{entrée}
\vedette{\hypertarget{Ⓔɕkɤkhe}{\papi{ ɕkɤkhe}}}\markboth{ɕkɤkhe}{}\classe{n}
\begin{définition}\fra oignon\end{définition}
\begin{définition}\cmn 葱\end{définition}
\begin{exemple}\jya ɕkɤkhe nɯ praʁ kɯ-fse kɯ-tu tsa ɣɯ sɤtɕha tu-ɬoʁ ŋu, ɯ-qa jpum tsa ɲɯ-βze cha, ɯ-rqhu kɤntɕhɯ ʑo tu, ɯ-jwaʁ nɯ kɯ-ɤlɯlju tɕe qhoʁsjɯβ ŋu ri nɤrko, ɯ-tho tu, ɯ-mɯntoʁ kɯ-ɤrŋi ɲɯ-lɤt ŋu. kɤ-ndza mɯm.\cmn 
\stylefv{ɕkɤkhe}生长在岩石一样石头比较多的山上,根粗,有多层皮,叶子也是圆柱形的、空心的,但很结实。有花梗,开蓝色的花。好吃。
\end{exemple}
\begin{relation-sémantique}\confer{
\hyperlink{Ⓔɕku}{\textit{ \papi{ɕku}}}
}\end{relation-sémantique}
\end{entrée}

\begin{entrée}
\vedette{\hypertarget{Ⓔɕkɤkɯm}{\papi{ ɕkɤkɯm}}}\markboth{ɕkɤkɯm}{}\classe{n}
\begin{définition}\fra jardin\end{définition}
\begin{définition}\cmn 菜园\end{définition}
\begin{relation-sémantique}\confer{
\hyperlink{Ⓔɕku}{\textit{ \papi{ɕku}}}
}\end{relation-sémantique}\end{entrée}

\begin{entrée}
\vedette{\hypertarget{Ⓔɕkɤɲcɣa}{\papi{ ɕkɤɲcɣa}}}\markboth{ɕkɤɲcɣa}{}\classe{n}
\begin{définition}\fra poireau sauvage\end{définition}
\begin{définition}\cmn 野韭菜\end{définition}
\begin{exemple}\jya ɕkɤɲcɣa nɯ ɯ-mdoʁ kɯ-pɣi tsa ŋu, ɯ-qa ɯ-zrɤm sɤɣ-ndzoʁ nɯ ɯ-rqhu tu jpum, ɯ-jwaʁ kɯ-jaʁ tɕe, tu-ŋgɤɣ tɕe, tɯ-ɲcɣa ɯ-tshɯɣa fse. ɯ-χcɤl ɯ-spjɯŋ tɤ-ɣe tɕe, ɯ-kɤχcɤl ɲɯ-rɯmɯntoʁ ŋu. ɯ-mɯntoʁ nɯ ɕkɤzoŋzoŋ rmi, li ɕkɤtho rmi. kɤ-ndza sna, ɯ-di mɯm.\cmn 
{ɕkɤɲcɣa}呈灰色,长根的部位有硬皮。根略粗。叶子厚而弯,形似镰刀(\stylefv{tɯɲcɣa})。中间长主心干,顶端开花。花叫做\stylefv{ɕkɤzoŋzoŋ},也叫做\stylefv{ɕkɤtho}。可食用,好吃。
\end{exemple}
\begin{relation-sémantique}\confer{
\hyperlink{Ⓔɕku}{\textit{ \papi{ɕku}}}
}\end{relation-sémantique}
\begin{relation-sémantique}\confer{
\hyperlink{Ⓔtɯɲcɣa}{\textit{ \papi{tɯɲcɣa}}}
}\end{relation-sémantique}\end{entrée}

\begin{entrée}
\vedette{\hypertarget{Ⓔɕkɤphɤr}{\papi{ ɕkɤphɤr}}}\markboth{ɕkɤphɤr}{}\classe{n}
\begin{définition}\fra poireau sauvage\end{définition}
\begin{définition}\cmn 鹿耳韭\end{définition}
\begin{exemple}\jya ɕkɤphɤr nɯ mbraj, sɤjku, tɯrgi kɯ-mbro ɯ-ŋgɯ tu-ɬoʁ ŋu. ɯ-qa ɣɯ ɯ-rqhu kɤntɕhɯ-tɤlɤβ ʑo tu, ɯ-ru me, ɯ-jwaʁ ʁnɯ-mpɕar ma me, ɯ-χcɤl ri ɯ-tho tu-ɬoʁ tɕe, lonba χsɯ-ldʑa ma tu-kɯ-ɬoʁ me. ɯ-jwaʁ nɯ qartshaz ɣɯ ɯ-rna ɯ-tshɯɣa fse, kɯ-ɤrtɯm kɯ-rɲɟi tɕe lu-kɯ-ɤkɤmtɕoʁ ŋu, ɯ-jwaʁ nɯ mpɕu, mpɯ, tsuku kɯ ɕkɤphɤr tu-ti-nɯ ŋu, tsuku kɯ ɕkɤjwaʁ tu-ti-nɯ ŋu. kɤ-ndza mɯm.\cmn 
\stylefv{ɕkɤphɤr}生长在树高较高的红桦、白桦或杉树林里。根长有多层皮,没有茎,只有两片叶子。中间长花梗,总共只有三根。叶子形状像鹿的耳朵,椭圆形,顶部尖。叶子光滑、嫩。有的人叫\stylefv{ɕkɤphɤr},有的人叫\stylefv{ɕkɤjwaʁ}。好吃。
\end{exemple}
\end{entrée}

\begin{entrée}
\vedette{\hypertarget{Ⓔɕkɤpja}{\papi{ ɕkɤpja}}}\markboth{ɕkɤpja}{}\classe{n}
\begin{définition}\fra poireau sauvage\end{définition}
\begin{définition}\cmn 野韭菜\end{définition}
\begin{exemple}\jya ɕkɤpja nɯ sɯŋgɯ arɤkhɯmkhɤl ma sɯŋgɯ thamtɕɤt tu maʁ, ɕkɤpja nɯ kɯ-ndɯβ ci ŋu, ɯ-jwaʁ arŋi, ɯ-ru nɯ ɯ-jwaʁ lu-ɬoʁ ɕɯŋgɯ nɯ kɯ-ɤrɤʑɯʑrɤz ŋu, kɯ-wɣrum tɯ-ʑrɤz, kɯ-ɣɯrni tɯ-ʑrɤz ŋu. ɯ-jwaʁ χsɯm ɕaŋtaʁ me. ɯ-χcɤl ɯ-mɯntoʁ tu-ɬoʁ tɕe, chɯ-do ɕti. mɯntoʁ kɯ-ɣɯrni ɲɯ-lɤt ŋu. tú-wɣ-ndza tɕe wuma ʑo mɤrtsaβ tɕe ɯ-dɯχɯn tu.\cmn 
\stylefv{ɕkɤpja}生长在某些森林中,并不是所有地方都有。\stylefv{ɕkɤpja}长得小,叶呈绿色。叶子长出来之前,茎上有条纹,白一条,红一条。叶子最多只有三片。中间长花梗时,就老了。开红色花。吃起来辣,有香味。
\end{exemple}
\end{entrée}

\begin{entrée}
\vedette{\hypertarget{Ⓔɕkɤrnɤɕkɤr}{\papi{ ɕkɤrnɤɕkɤr}}}\markboth{ɕkɤrnɤɕkɤr}{}\classe{idph.3}
\begin{définition}\fra en boitant\end{définition}
\begin{définition}\cmn 形容人跛脚的样子\end{définition}
\begin{exemple}\jya ɕkɤrnɤɕkɤr ʑo jo-nɯɕe\cmn 他跛着脚地回家了\end{exemple}\end{entrée}

\begin{entrée}
\vedette{\hypertarget{Ⓔɕkɤrɯ}{\papi{ ɕkɤrɯ}}}\markboth{ɕkɤrɯ}{}
\classe{n}
\begin{définition}\fra capricornus sumatraensis\end{définition}
\begin{définition}\cmn 鬣羚\end{définition}
\begin{exemple}\jya ɕkɤrɯ nɯ sɯŋgɯ praʁ ɯ-rchɤβ aʁɤndɯndɤt ku-rɤʑi ɕti, rpɣo, co, sɯŋgɯ ɯ-ŋgɯ kɤsɯfse kɯ-tu ci ɕti, ɯʑo nɯ zɯmi nɯŋa fse, ɯ-qa ta-ʁrɯ nɯŋa ɯ-qa ta-ʁrɯ fse, pjɯ́-wɣ-sat tɕe, ɯ-mɤlɤjaʁ ɯ-ndʐi nɯ sɲɤt sna, ɯ-ʁrɯ tɯ-tɕha kɯ-ɤmtɕɯ-mtɕoʁ ŋu, tɯ-tɣa jamar zri, ɯ-rna nɯnɯ tɤrka ɯ-rna fse, ɯ-mtɕhi nɯ nɯŋa ɯ-mtɕhi fse, kɯ-ɤɲaʁndzɯm kɯ-ɤɣɯrnɯɕɯr tsa ɯ-mdoʁ ŋu, ɯ-rpaʁ nɯ tɕu kɯ-wɣrum tɯ-snaʁ tu.\cmn 鬣羚栖息在森林里,也生活在岩石上,在高山、河坝到处都有,它有点像奶牛,蹄子像奶牛的蹄子。打死了以后,四肢的皮子可以用来作后鞧。有一对很尖的角,有一拃长,耳朵像驴子的耳朵,嘴像牛的嘴,颜色是黑里透红,肩部有一块白点。\end{exemple}\end{entrée}

\begin{entrée}
\vedette{\hypertarget{Ⓔɕkɤtho}{\papi{ ɕkɤtho}}}\markboth{ɕkɤtho}{}\classe{n}
\begin{définition}\fra pédoncule d'ail\end{définition}
\begin{définition}\cmn 蒜梗\end{définition}
\begin{relation-sémantique}\confer{
\hyperlink{Ⓔɕku}{\textit{ \papi{ɕku}}}
}\end{relation-sémantique}
\begin{relation-sémantique}\confer{
\hyperlink{Ⓔɯ-tho}{\textit{ \papi{ɯ-tho}}}
}\end{relation-sémantique}\end{entrée}

\begin{entrée}
\vedette{\hypertarget{Ⓔɕkɤtshoŋ}{\papi{ ɕkɤtshoŋ}}}\markboth{ɕkɤtshoŋ}{}\classe{n}
\begin{définition}\fra oignon\end{définition}
\begin{définition}\cmn 葱
\begin{déclaration} \étymologie{\papi{\stylefn{葱}}}\end{déclaration}\end{définition}
\begin{exemple}\jya ɕkɤtshoŋ nɯ tɯ-ji ɯ-ŋgɯ lu-kɤ-nɯ-ji ci ŋu, ɯ-qa nɯ ɯ-zrɤm kɯ-wɣrum ŋu, ɯ-ru me, ɯ-jwaʁ kɯ-ɤlɯlju tɕe qhoʁsjɯβ ŋu, ɯ-ku tu-omtɕoʁ ŋu, ɯ-dɯχɯn mɯm, tɕeri kɤ-smi tɕe ɯ-di ɲɯ-me ɕti.\cmn 
\stylefv{ɕkɤtshoŋ}是自己种在地里的(农作物),须根白色,没有茎,叶子圆柱形、空心。顶部尖,有香味,但煮熟后就没有香味了。
\end{exemple}
\end{entrée}

\begin{entrée}
\vedette{\hypertarget{Ⓔɕkɤtɯm}{\papi{ ɕkɤtɯm}}}\markboth{ɕkɤtɯm}{}
\classe{n}
\begin{définition}\fra racine de l'ail\end{définition}
\begin{définition}\cmn 大蒜的根\end{définition}
\begin{relation-sémantique}\confer{
\hyperlink{Ⓔɕku}{\textit{ \papi{ɕku}}}
}\end{relation-sémantique}\end{entrée}

\begin{entrée}
\vedette{\hypertarget{Ⓔɕkɤzoŋzoŋ}{\papi{ ɕkɤzoŋzoŋ}}}\markboth{ɕkɤzoŋzoŋ}{}\classe{n}
\begin{définition}\fra espèce d'oignon\end{définition}
\begin{définition}\cmn 葱的一种\end{définition}\end{entrée}

\begin{entrée}
\vedette{\hypertarget{Ⓔɕke}{\papi{ ɕke}}}\markboth{ɕke}{}\classe{vi}
\paradigme{\textit{dir :} \jya pɯ-}\acception{1}
\begin{définition}\fra brûler\end{définition}
\begin{définition}\cmn 烫;被烫到\end{définition}
\begin{exemple}\jya pɯ-ɕke-a, pɯ-ɕke\cmn 我(被)烫到了,他(被)烫到了\end{exemple}
\begin{exemple}\jya tɯcila pjɤ-lwoʁ tɕe pjɤ-ɕke\cmn 把滚烫的水洒了一地,烫到了\end{exemple}
\begin{exemple}\jya a-jaʁ pɯ-ɕke\cmn 我烫伤了手\end{exemple}\acception{2}
\begin{définition}\fra importante (affaire)\end{définition}
\begin{définition}\cmn 重要(事情)\end{définition}
\begin{exemple}\jya jiɕqha nɯ kɯ-ɕke ci ɲɯ-ŋu (=kɯ-ʁzi)\cmn 这件事情很重要\end{exemple}
\begin{relation-sémantique}\confer{
\hyperlink{ⒺsɤɕkeⒽ1}{\textit{ \papi{sɤɕke1}}}
}\end{relation-sémantique}
\begin{relation-sémantique}\confer{
\hyperlink{ⒺsɤɕkeⒽ2}{\textit{ \papi{sɤɕke2}}}
}\end{relation-sémantique}
\begin{relation-sémantique}\confer{
\hyperlink{Ⓔnɤsɤɕke}{\textit{ \papi{nɤsɤɕke}}}
}\end{relation-sémantique}
\begin{relation-sémantique}\confer{
\hyperlink{Ⓔʑɣɤsɤɕke}{\textit{ \papi{ʑɣɤsɤɕke}}}
}\end{relation-sémantique}
\begin{relation-sémantique}\confer{
\hyperlink{Ⓔsɯɕke}{\textit{ \papi{sɯɕke}}}
}\end{relation-sémantique}\end{entrée}

\begin{entrée}
\vedette{\hypertarget{Ⓔɕkho}{\papi{ ɕkho}}}\markboth{ɕkho}{}\classe{vt}
\paradigme{\textit{dir :} \jya tɤ-}
\paradigme{\textit{dir :} \jya lɤ-}
\paradigme{\textit{dir :} \jya nɯ-}
\paradigme{\textit{dir :} \jya thɯ-}
\begin{définition}\fra faire sécher, étendre\end{définition}
\begin{définition}\cmn 晒;铺\end{définition}
\begin{exemple}\jya tɤ-ɕkho-t-a, ta-ɕkho, tɤ-ɕkhɤm\cmn 我晒了,他晒了,你晒吧\end{exemple}
\begin{exemple}\jya tɯ-mɯ tɤ-pe tɕe, tɤŋe nɯ-ɬoʁ tɕe, tɕe tɯ-ŋga ra kɤ-ɕkho ra\cmn 太阳出来了,我们要晒衣服了\end{exemple}
\begin{exemple}\jya a-ŋga tɤ-ɕkho-t-a\cmn 晒衣服\end{exemple}
\begin{exemple}\jya tɤ-βɟu lɤ-ɕkho-t-a\cmn 铺褥子\end{exemple}
\begin{exemple}\jya tɯjpu nɯ-ɕkho-t-a\cmn 我把粮食晒干了\end{exemple}\begin{sous-entrée}
\vedette{\hypertarget{}{\papi{ aɕkho}}}\markboth{aɕkho}{}\classe{vs}
\begin{définition}\fra qui a une ouverture large\end{définition}
\begin{définition}\cmn 底部小,口很大\end{définition}
\begin{exemple}\jya ki tɯthɯ ki ɲɯ-ɤɕkho\cmn 这个锅子底小口大\end{exemple}
\end{sous-entrée}\begin{sous-entrée}
\vedette{\hypertarget{}{\papi{ rɤɕkho}}}\markboth{rɤɕkho}{}\classe{vi}
\paradigme{\textit{dir :} \jya tɤ-}
\paradigme{\textit{dir :} \jya nɯ-}
\begin{définition}\ 
\begin{déclaration}\grammar{apass}\end{déclaration}\end{définition}
\begin{définition}\fra sécher des choses\end{définition}
\begin{définition}\cmn 晒东西\end{définition}
\end{sous-entrée}\end{entrée}

\begin{entrée}
\vedette{\hypertarget{Ⓔɕkliɕkli}{\papi{ ɕkliɕkli}}}\markboth{ɕkliɕkli}{}\classe{idph.2}
\begin{définition}\fra rond et dur\end{définition}
\begin{définition}\cmn 形容长条形的东西圆而硬的样子\end{définition}
\begin{exemple}\jya ʁʑɯnɯ ra nɯ-ʁla kú-wɣ-ndo tɕe ɕkliɕkli ʑo ɲɯ-rko\cmn 小伙子的手臂拿起来圆圆的硬硬的\end{exemple}
\begin{exemple}\jya tɤ-ri pɯ́-wɣ-rɤtɕaʁ tɕe tɯ-mɤpa ɲɯ-rko ɕkliɕkli ʑo\cmn 踩在绳子的上面,感觉脚下很硬,站不稳\end{exemple}
\begin{relation-sémantique}\synonyme{
\hyperlink{Ⓔɕklɯɣɕklɯɣ}{\textit{ \papi{ɕklɯɣɕklɯɣ}}}
}\end{relation-sémantique}\end{entrée}

\begin{entrée}
\vedette{\hypertarget{Ⓔɕklɯɣ}{\papi{ ɕklɯɣ}}}\markboth{ɕklɯɣ}{}
\classe{vt}
\paradigme{\textit{dir :} \jya nɯ-}
\begin{définition}\fra gêner dans le dos (d'un objet dur sur lequel on s'allonge ou que l'on porte sur le dos)\end{définition}
\begin{définition}\cmn 硌着背\end{définition}
\begin{exemple}\jya kɯ-rko tú-wɣ-fkur tɕe tɯ-mgɯr ku-ɕklɯɣ ŋu tɕe ɕɯmŋɤm\cmn 背硬的东西的时候,会硌着背,感觉很痛\end{exemple}\end{entrée}

\begin{entrée}
\vedette{\hypertarget{Ⓔɕklɯɣɕklɯɣ}{\papi{ ɕklɯɣɕklɯɣ}}}\markboth{ɕklɯɣɕklɯɣ}{}\classe{idph.2}
\begin{définition}\fra ferme et rond\end{définition}
\begin{définition}\cmn 形容长条形的东西圆而硬的样子
\end{définition}
\begin{relation-sémantique}\synonyme{
\hyperlink{Ⓔɕkliɕkli}{\textit{ \papi{ɕkliɕkli}}}
}\end{relation-sémantique}\end{entrée}

\begin{entrée}
\vedette{\hypertarget{Ⓔɕkom}{\papi{ ɕkom}}}\markboth{ɕkom}{}
\classe{n}
\begin{définition}\fra muntjac\end{définition}
\begin{définition}\cmn 麂子\end{définition}
\begin{exemple}\jya ɕkom nɯ sɯŋgɯnaχtɕin ɯ-ŋgɯ zɯ ku-rɤʑi ɲɯ-ŋu, zgoku χcɤl tsa zɯ ku-rɤʑi, ɯ-qa taʁrɯ nɯ qaʑo ɯ-qa ta-ʁrɯ fse, ɯʑo kɯ-pɣi ŋu, ɯ-jme tɯ-tɯɣa jamar maŋe, kɯ-wɣrum ŋu, ɯ-ku nɯ qaʑo ɯ-ku tsa fse, ɯ-ʁrɯ me.\cmn 麂子生活在深山老林中的半山地带。蹄子类似绵羊蹄,身灰色,尾巴只有一拃长,白色。头部类似绵羊,没有角。\end{exemple}\end{entrée}

\begin{entrée}
\vedette{\hypertarget{Ⓔɕkrɤɣɕkrɤɣ}{\papi{ ɕkrɤɣɕkrɤɣ}}}\markboth{ɕkrɤɣɕkrɤɣ}{}\classe{idph.2}
\begin{définition}\fra dur et froid (sensation lorsqu'on s'allonge sur le sol)\end{définition}
\begin{définition}\cmn 又硬又冷,没有衣服盖(躺在地上的感觉)\end{définition}
\begin{exemple}\jya ɯ-thoʁ ɕkrɤɣɕkrɤɣ pɯ-nɯ-rŋgɯ-a\cmn 我在地上睡觉,感觉又硬又冷\end{exemple}
\begin{relation-sémantique}\confer{
\hyperlink{Ⓔnɯɕkrɤɣ}{\textit{ \papi{nɯɕkrɤɣ}}}
}\end{relation-sémantique}\end{entrée}

\begin{entrée}
\vedette{\hypertarget{Ⓔɕkrɤz}{\papi{ ɕkrɤz}}}\markboth{ɕkrɤz}{}
\classe{n}
\begin{définition}\fra chêne\end{définition}
\begin{définition}\cmn 青冈树;槲栎\end{définition}
\begin{exemple}\jya ɕkrɤz nɯ zgoku aʁɤndɯndɤt tu-ɬoʁ cha, tu-wxti cha, aɣɯrtɯrtaʁ tɕe, ɯ-ru tɯ-ldʑa kɯ-jpum ɲɯ-βze mɤ-cha, ɯ-rtaʁ jpum, sɯŋgɯ kɯ-wxti nɯ ra ɯ-ru kɯ-zri tu-βze tɕe, ɯ-kɤχcɤl tɕe ɲɯ-ɤɣɯrtɯrtaʁ ŋu. ɯ-jwaʁ nɯ kɯ-ɤrtɯm tsa tɕe ɯ-βzɯr nɯ tɕu ɯ-mdzu kɯ-mtɕoʁ ʑo ku-nɯgrɤl ŋu. ɯ-jwaʁ ɯ-qhu chu kɯ-qarŋe tɯ-ɣndʑɤr kɯ-fse tu, ɯ-jwaʁ ɯ-ʁɤri arŋi mpɕu, ɯ-jwaʁ nɯ qamphoʁ rmi. ɕkrɤz ɯ-mat chɯ-βze ŋu tɕe, ɯ-mat nɯ thɣe rmi, paʁndza wuma ʑo pe, ɯ-ku ri ɯ-mat kɯ-maʁ kɯ-rko ci ku-ndzoʁ ŋu tɕe nɯ tɕamu sna. ɕkrɤz kɯ-do ɣɯ ɯ-ci tɤ-se kɯ-fse pjɯ-kɯ-nɯ-ɬoʁ ci ɣɤʑu tɕe, nɯ ku-jkrɯt tɕe, nɯ tʂha stu kɯ-ʑru ɲɯ-ŋu, smɤn ɲɯ-ŋu. ɕkrɤz ʁnɯ-tɯphu ɣɤʑu tɕe, tɯ-tɯphu nɯ ɯ-ru kɯ-xtshɯm kɯ-zri ɲɯ-ŋu, kɯ-ɤstɤko ɲɯ-ŋu, ɯnɯnɯ qaprɤsi ɲɯ-rmi, li ci tɯ-tɯphu nɯ praʁ kɯ-fse mɤ-kɯ-sɤɣa tu-ɬoʁ ɲɯ-ŋu tɕe, ɯ-ru cho ɯ-rtaʁ ra mɯ-ɲɯ-ɤstɤko, ɲɯ-ɤjʁu tɕe nɯ praʁkɤsi ɲɯ-rmi.\cmn 
青冈树在山上到处都可以生长,长得很大,因为枝桠发达所以没有一棵比较粗的主干,枝桠很粗,在大森林里树干长得高,在顶端上才长枝桠。叶子有点圆,在边缘排列着锋利的刺。叶子的背面有黄色粉状的东西,叶子正面是绿色的,光滑的。叶子叫\stylefv{qamphoʁ}。 青冈树也结果实,这种果实叫\stylefv{thɣe},可以是喂猪的好饲料。在树梢上长出一块不是果实的硬东西,可以熬茶。老青冈树上有一种像血一样的液体流出来,凝结后是最优质的茶,是一种药。有两种青冈树,一种树干细而长,很直,叫\stylefv{qaprɤsi}(蛇树),另一种长在岩石和陡峭的地方,树干和树枝都长得不直,弯曲的,这种叫\stylefv{praʁkɤsi}(岩石上的树)。
\end{exemple}\end{entrée}

\begin{entrée}
\vedette{\hypertarget{Ⓔɕkrɯɣɕkrɯɣ}{\papi{ ɕkrɯɣɕkrɯɣ}}}\markboth{ɕkrɯɣɕkrɯɣ}{}\classe{idph.2}
\begin{définition}\fra dur et rugueux\end{définition}
\begin{définition}\cmn 形容搓紧的线绳硬而粗糙的样子\end{définition}
\begin{exemple}\jya tɤ-fsɤri lú-wɣ-sɤsɯɣ tɕe ɕkrɯɣɕkrɯɣ ʑo pa\cmn 把绳子搓紧的时候,就又硬又粗\end{exemple}\end{entrée}

\begin{entrée}
\vedette{\hypertarget{Ⓔɕkɯβɕkɯβ}{\papi{ ɕkɯβɕkɯβ}}}\markboth{ɕkɯβɕkɯβ}{}
\classe{idph.2}
\begin{définition}\fra dos courbé\end{définition}
\begin{définition}\cmn 形容驼着背的样子(个子高)\end{définition}
\begin{exemple}\jya ɕkɯβɕkɯβ ʑo ɲɯ-ŋu\cmn 他驼着背站在那里\end{exemple}\begin{sous-entrée}
\vedette{\hypertarget{}{\papi{ ɕkɯβnɤɕkɯβ}}}\markboth{ɕkɯβnɤɕkɯβ}{}\classe{idph.3}
\begin{exemple}\jya ɕkɯβnɤɕkɯβ kɤ-ari\cmn 他驼着背去了\end{exemple}
\end{sous-entrée}\end{entrée}

\begin{entrée}
\vedette{\hypertarget{Ⓔɕkɯt}{\papi{ ɕkɯt}}}\markboth{ɕkɯt}{}
\classe{vt}
\paradigme{\textit{dir :} \jya thɯ-}\acception{1}
\begin{définition}\fra finir de manger\end{définition}
\begin{définition}\cmn 吃完\end{définition}
\begin{exemple}\jya tɤ-ɕkɯta, ta-ɕkɯt\cmn 我吃完了,他吃完了\end{exemple}
\begin{exemple}\jya nɤ-khɯtsa ɯ-ŋgɯ nɯ tɤ-ɕkɯt\cmn 你把碗里的吃完\end{exemple}
\begin{exemple}\jya nɤ-@beimu ɯ-thɯ-tɯ-nɯ-ɕkɯt\cmn 你把贝母吃完了吗?\end{exemple}\acception{2}
\begin{définition}\fra finir de boire\end{définition}
\begin{définition}\cmn 喝完\end{définition}
\begin{exemple}\jya ɕkɯt-tɕi\cmn 干杯\end{exemple}\acception{3}
\begin{définition}\fra utiliser complètement\end{définition}
\begin{définition}\cmn 用完\end{définition}
\begin{exemple}\jya qarma pjɤ-ɕkɯt-nɯ\cmn 马鸡没有了\end{exemple}\end{entrée}

\begin{entrée}
\vedette{\hypertarget{Ⓔɕlu}{\papi{ ɕlu}}}\markboth{ɕlu}{}\classe{vl}
\paradigme{\textit{dir :} \jya tɤ-}
\paradigme{\textit{dir :} \jya lɤ-}
\begin{définition}\fra labourer\end{définition}
\begin{définition}\cmn 耕地\end{définition}
\begin{exemple}\jya tɤ-ɕlu-a, tɤ-ɕlu, ku-ɕlu-a\cmn 我耕了地,他耕了地,我正在耕地\end{exemple}
\begin{exemple}\jya ki tɯji ki tɤ-ɕlu-t-a\cmn 我耕了这块地\end{exemple}\end{entrée}

\begin{entrée}
\vedette{\hypertarget{Ⓔɕlaŋɕlaŋ}{\papi{ ɕlaŋɕlaŋ}}}\markboth{ɕlaŋɕlaŋ}{}\classe{idph.2}
\begin{définition}\fra rond et lisse\end{définition}
\begin{définition}\cmn 又圆又光滑(发光)\end{définition}
\begin{exemple}\jya ɯ-laz ɕlaŋɕlaŋ to-tɕɤt\cmn 他把头探出来了,是光头\end{exemple}
\begin{exemple}\jya ɯ-ku ɕlaŋɕlaŋ to-stu\cmn 他的头光滑得发亮\end{exemple}
\begin{exemple}\jya ɯ-ku pjɤ-nɯ-sɯ-qrɤz tɕe, ɕlaŋɕlaŋ ʑo ɲɯ-pa\cmn 他理了发(把头发剃光了就)光滑得发亮\end{exemple}
\begin{relation-sémantique}\confer{
\hyperlink{Ⓔslaŋslaŋ}{\textit{ \papi{slaŋslaŋ}}}
}\end{relation-sémantique}
\begin{relation-sémantique}\confer{
\hyperlink{Ⓔclaŋclaŋ}{\textit{ \papi{claŋclaŋ}}}
}\end{relation-sémantique}
\begin{relation-sémantique}\confer{
\hyperlink{Ⓔrlaŋrlaŋ}{\textit{ \papi{rlaŋrlaŋ}}}
}\end{relation-sémantique}\end{entrée}

\begin{entrée}
\vedette{\hypertarget{Ⓔɕlaʁ}{\papi{ ɕlaʁ}}}\markboth{ɕlaʁ}{}
\classe{idph.1}
\begin{définition}\fra d'un seul coup\end{définition}
\begin{définition}\cmn 一下子\end{définition}\begin{sous-entrée}
\vedette{\hypertarget{}{\papi{ ɕlaʁnɤɕlaʁ}}}\markboth{ɕlaʁnɤɕlaʁ}{}\classe{idph.3}
\begin{définition}\fra en plusieurs coups\end{définition}
\begin{définition}\cmn 几下\end{définition}
\begin{exemple}\jya kɤ-nɤma ɕlaʁnɤɕlaʁ ʑo na-sthɯt\cmn 几下就把工作做完了\end{exemple}
\end{sous-entrée}\begin{sous-entrée}
\vedette{\hypertarget{}{\papi{ phɯɕlaʁ}}}\markboth{phɯɕlaʁ}{}\classe{idph.6}
\begin{relation-sémantique}\confer{
\hyperlink{Ⓔslaʁ}{\textit{ \papi{slaʁ}}}
}\end{relation-sémantique}
\end{sous-entrée}\end{entrée}

\begin{entrée}
\vedette{\hypertarget{Ⓔɕliɕli}{\papi{ ɕliɕli}}}\markboth{ɕliɕli}{}\begin{définition}\fra rond et lisse\end{définition}
\begin{définition}\cmn 又圆又光滑(发光)\end{définition}
\begin{exemple}\jya ɕlinɤɕli kɤ-ari\cmn 他(光头发光地)去了\end{exemple}
\begin{relation-sémantique}\confer{
\hyperlink{Ⓔclaŋclaŋ}{\textit{ \papi{claŋclaŋ}}}
}\end{relation-sémantique}
\begin{relation-sémantique}\confer{
\hyperlink{Ⓔɕlaŋɕlaŋ}{\textit{ \papi{ɕlaŋɕlaŋ}}}
}\end{relation-sémantique}\end{entrée}

\begin{entrée}
\vedette{\hypertarget{Ⓔɕlɯɣ}{\papi{ ɕlɯɣ}}}\markboth{ɕlɯɣ}{}
\classe{vt}
\paradigme{\textit{dir :} \jya pɯ-}
\begin{définition}\fra lâcher sans faire attention\end{définition}
\begin{définition}\cmn 失手使物品掉落\end{définition}
\begin{exemple}\jya nɯ-ɕlɯɣ-a, na-ɕlɯɣ\cmn 我失手了,他失手了\end{exemple}
\begin{exemple}\jya ki kɤ-ɕlɯɣ mɤ-pe (ma ɴɢrɯ ɕti)\cmn 不能让它掉下来,不然会破掉\end{exemple}
\begin{relation-sémantique}\confer{
\hyperlink{Ⓔlɯɣ}{\textit{ \papi{lɯɣ}}}
}\end{relation-sémantique}\begin{sous-entrée}
\vedette{\hypertarget{}{\papi{ sɯɕlɯɣ}}}\markboth{sɯɕlɯɣ}{}\classe{vt}
\paradigme{\textit{dir :} \jya nɯ-}
\begin{définition}\fra détacher\end{définition}
\begin{définition}\cmn 解开\end{définition}
\begin{exemple}\jya tɤ-ri nɯ-kɯ-raʁ nɯ-sɯ-ɕlɯɣ-a tɕe, kɤ-rɯkɤtɯm jɤɣ\cmn 我把卡住了的线解开了,可以牵线了\end{exemple}
\begin{relation-sémantique}\confer{
\hyperlink{Ⓔsɯta}{\textit{ \papi{sɯta}}}
}\end{relation-sémantique}
\end{sous-entrée}\end{entrée}

\begin{entrée}
\vedette{\hypertarget{Ⓔɕmi}{\papi{ ɕmi}}}\markboth{ɕmi}{}
\classe{vt}
\paradigme{\textit{dir :} \jya tɤ-}
\paradigme{\textit{dir :} \jya pɯ-}
\begin{définition}\fra mélanger un liquide\end{définition}
\begin{définition}\cmn 搅拌\end{définition}
\begin{exemple}\jya tɤ-ɕmi-t-a, ta-ɕmi\cmn 我搅拌了,他搅拌了\end{exemple}
\begin{exemple}\jya nɤ@cai tɤ-ɕmi ma tsha ra mɤ-amɯzɣɯt\cmn 拌一下你的菜,盐放得不均匀\end{exemple}\end{entrée}

\begin{entrée}
\vedette{\hypertarget{Ⓔɕmɯɣ}{\papi{ ɕmɯɣ}}}\markboth{ɕmɯɣ}{} (\variante{mɯɣ}) \classe{n}
\begin{définition}\fra léger et soudain (d'un rire)\end{définition}
\begin{définition}\cmn 突然轻轻地笑\end{définition}
\begin{exemple}\jya ɯʑo kɯ pa-mtshɤm tɕe, ɕmɯɣ ɲɤ-nɤre\cmn 听他这样一说,突然笑了一下\end{exemple}\end{entrée}

\begin{entrée}
\vedette{\hypertarget{Ⓔɕnaβndʑɣi}{\papi{ ɕnaβndʑɣi}}}\markboth{ɕnaβndʑɣi}{}
\classe{n}
\begin{définition}\fra morveux\end{définition}
\begin{définition}\cmn 总是流鼻涕的孩子\end{définition}
\begin{exemple}\jya nɤʑo ɕnaβndʑɣi ki\cmn 你这个爱流鼻涕的家伙\end{exemple}\end{entrée}

\begin{entrée}
\vedette{\hypertarget{Ⓔɕnat}{\papi{ ɕnat}}}\markboth{ɕnat}{}
\classe{n}
\begin{définition}\fra élément du métier à tisser (lice)\end{définition}
\begin{définition}\cmn 用来提经线的织具(综)
\begin{déclaration} \étymologie{\papi{snas}}\end{déclaration}\end{définition}
\begin{exemple}\jya ɕnat nɯ ɯ-sqar ɯ-tu-kɯ-sɯ-βzu ɣɯ tɤ-ri ŋu, ɯ-tu-kɯ-rɤɕi ndʑu nɯ ɕnat-ndʑu rmi\cmn 综是用来钩住经线,使经线上下交叉的线,提着这种线的木棒叫提综杆\end{exemple}\end{entrée}

\begin{entrée}
\vedette{\hypertarget{Ⓔɕnɤcat}{\papi{ ɕnɤcat}}}\markboth{ɕnɤcat}{}\classe{num}
\begin{définition}\fra sept ou huit\end{définition}
\begin{définition}\cmn 七八个\end{définition}
\begin{exemple}\jya ɕnɤcɤ-sŋi\cmn 七八天\end{exemple}
\begin{relation-sémantique}\confer{
\hyperlink{Ⓔkɯɕnɯz}{\textit{ \papi{kɯɕnɯz}}}
}\end{relation-sémantique}
\begin{relation-sémantique}\confer{
\hyperlink{Ⓔkɯrcat}{\textit{ \papi{kɯrcat}}}
}\end{relation-sémantique}\end{entrée}

\begin{entrée}
\vedette{\hypertarget{Ⓔɕnɤku}{\papi{ ɕnɤku}}}\markboth{ɕnɤku}{}\classe{n}
\begin{définition}\fra bout du nez\end{définition}
\begin{définition}\cmn 鼻尖\end{définition}
\begin{relation-sémantique}\confer{
\hyperlink{Ⓔtɯ-ɕna}{\textit{ \papi{tɯ-ɕna}}}
}\end{relation-sémantique}
\begin{relation-sémantique}\confer{
\hyperlink{Ⓔtɯ-ku}{\textit{ \papi{tɯ-ku}}}
}\end{relation-sémantique}\end{entrée}

\begin{entrée}
\vedette{\hypertarget{Ⓔɕnɤloʁ}{\papi{ ɕnɤloʁ}}}\markboth{ɕnɤloʁ}{}
\classe{n}
\begin{définition}\fra anneau nasal\end{définition}
\begin{définition}\cmn 牛鼻圈\end{définition}\end{entrée}

\begin{entrée}
\vedette{\hypertarget{Ⓔɕnɤri}{\papi{ ɕnɤri}}}\markboth{ɕnɤri}{}\classe{n}
\begin{définition}\fra corde accrochée à l'anneau nasal\end{définition}
\begin{définition}\cmn 牛鼻绳\end{définition}
\begin{relation-sémantique}\confer{
\hyperlink{Ⓔtɯ-ɕna}{\textit{ \papi{tɯ-ɕna}}}
}\end{relation-sémantique}
\begin{relation-sémantique}\confer{
\hyperlink{Ⓔtɤ-ri}{\textit{ \papi{tɤ-ri}}}
}\end{relation-sémantique}\end{entrée}

\begin{entrée}
\vedette{\hypertarget{Ⓔɕnɤsti}{\papi{ ɕnɤsti}}}\markboth{ɕnɤsti}{}\classe{n}
\begin{définition}\fra personne qui a le nez bouché\end{définition}
\begin{définition}\cmn 鼻子塞了的人\end{définition}
\begin{relation-sémantique}\confer{
\hyperlink{Ⓔtɯ-ɕna}{\textit{ \papi{tɯ-ɕna}}}
}\end{relation-sémantique}
\begin{relation-sémantique}\confer{
\hyperlink{ⒺstiⒽ1}{\textit{ \papi{sti1}}}
}\end{relation-sémantique}\end{entrée}

\begin{entrée}
\vedette{\hypertarget{Ⓔɕnɤto}{\papi{ ɕnɤto}}}\markboth{ɕnɤto}{}
\classe{n}
\begin{définition}\fra tabac à priser\end{définition}
\begin{définition}\cmn 鼻烟
\begin{déclaration} \étymologie{\papi{du.ba}}\end{déclaration}\end{définition}
\begin{exemple}\jya ɕnɤto tɤ-nɯ-lat-a\cmn 我吸了鼻烟。\end{exemple}\end{entrée}

\begin{entrée}
\vedette{\hypertarget{Ⓔɕnɤtoʁrɯ}{\papi{ ɕnɤtoʁrɯ}}}\markboth{ɕnɤtoʁrɯ}{}\classe{n}
\begin{définition}\fra tabatière\end{définition}
\begin{définition}\cmn 鼻烟壶
\begin{déclaration} \étymologie{\papi{sna.du.ba.ru}}\end{déclaration}\end{définition}
\end{entrée}

\begin{entrée}
\vedette{\hypertarget{Ⓔɕnɤxsɯr}{\papi{ ɕnɤxsɯr}}}\markboth{ɕnɤxsɯr}{}\classe{n}
\begin{exemple}\jya ɕnɤxsɯr kɯ-tu ma khɤxsɯr kɯ-me\cmn 只能闻到而吃不到\end{exemple}
\begin{relation-sémantique}\confer{
\hyperlink{Ⓔtɯ-ɕna}{\textit{ \papi{tɯ-ɕna}}}
}\end{relation-sémantique}
\begin{relation-sémantique}\confer{
\hyperlink{Ⓔxsɯr}{\textit{ \papi{xsɯr}}}
}\end{relation-sémantique}\end{entrée}

\begin{entrée}
\vedette{\hypertarget{Ⓔɕnoʁɕnoʁ}{\papi{ ɕnoʁɕnoʁ}}}\markboth{ɕnoʁɕnoʁ}{}
\classe{idph.2}
\begin{définition}\fra pointu\end{définition}
\begin{définition}\cmn 尖\end{définition}
\begin{exemple}\jya ɯ-mtɕhi ɲɯ-ɤmtɕoʁ ʑo ɕnoʁɕnoʁ\cmn 他嘴巴很尖(他在怄气)\end{exemple}
\begin{exemple}\jya pɣɤtɕɯ ɯ-mtsioʁ ɕnoʁɕnoʁ ɲɯ-pa\cmn 鸟的嘴是尖的\end{exemple}\begin{sous-entrée}
\vedette{\hypertarget{}{\papi{ ɕnoʁnɤɕnoʁ}}}\markboth{ɕnoʁnɤɕnoʁ}{}\classe{idph.3}
\begin{exemple}\jya kumpɣa kɯ ɕnoʁnɤɕnoʁ ɲɯ-ɤz-nɯrdoʁ\cmn 母鸡一啄一啄地在啄食\end{exemple}
\begin{exemple}\jya tɤ-pɤtso kɯ tɤ-rɣe ɕnoʁnɤɕnoʁ ʑo ta-nɯrdoʁ\cmn 小孩子用小巧的手很快地把珠子捡起来了\end{exemple}
\end{sous-entrée}\end{entrée}

\begin{entrée}
\vedette{\hypertarget{Ⓔɕnɯɣnɤlɯɣ}{\papi{ ɕnɯɣnɤlɯɣ}}}\markboth{ɕnɯɣnɤlɯɣ}{}\classe{idph.4}
\begin{définition}\fra irrespectueux\end{définition}
\begin{définition}\cmn 形容不稳重,说话没有礼貌,不尊重人的样子\end{définition}
\begin{exemple}\jya jiɕqha nɯ ɯ-zda ra nɯ-phe ɕnɯɣnɤlɯɣ ɲɯ-ʑɣɤstu\cmn 那个人在伙计面前不稳重\end{exemple}\begin{sous-entrée}
\vedette{\hypertarget{}{\papi{ ɣɤɕnɯɣlɯɣ}}}\markboth{ɣɤɕnɯɣlɯɣ}{}\classe{vi}
\begin{exemple}\jya nɯ ɯ-zda ra nɯ-phe ɲɯ-ɣɤɕnɯɣlɯɣ ntsɯ\cmn 他在伙计面前不稳重\end{exemple}
\begin{exemple}\jya ɲɯ-rɯpjɤβlaʁ tɕe ɲɯ-ɣɤɕnɯɣlɯɣ\cmn 他很狡猾,动作也不雅观\end{exemple}
\end{sous-entrée}\end{entrée}

\begin{entrée}
\vedette{\hypertarget{Ⓔɕɲɯɣ}{\papi{ ɕɲɯɣ}}}\markboth{ɕɲɯɣ}{}
\classe{idph.1}
\begin{définition}\fra douleur ressentie lorsqu'on se foule la cheville\end{définition}
\begin{définition}\cmn 崴脚的痛感\end{définition}
\begin{exemple}\jya a-mi ɕɲɯɣ ʑo (ɲɯ-ti) na-ʂŋaʁ\cmn 我崴了脚,一下子就感到剧痛\end{exemple}\end{entrée}

\begin{entrée}
\vedette{\hypertarget{ⒺɕŋaʁɕŋaʁⒽ1}{\papi{ ɕŋaʁɕŋaʁ}}}\markboth{ɕŋaʁɕŋaʁ}{}\homonyme{1}\classe{idph.2}
\begin{définition}\fra jaune vif\end{définition}
\begin{définition}\cmn 深黄色\end{définition}
\begin{exemple}\jya raz kɯ-qarŋe ɕŋaʁɕŋaʁ ʑo ɲɯ-ŋu\cmn 布是深黄色的\end{exemple}\end{entrée}

\begin{entrée}
\vedette{\hypertarget{ⒺɕŋaʁɕŋaʁⒽ2}{\papi{ ɕŋaʁɕŋaʁ}}}\markboth{ɕŋaʁɕŋaʁ}{}\homonyme{2}
\classe{idph.2}
\begin{définition}\fra malin et agaçant (enfant)\end{définition}
\begin{définition}\cmn 形容小孩子又机灵又瘦小又讨厌的样子\end{définition}
\begin{exemple}\jya ki tɤ-rɟit ki ɕŋaʁɕŋaʁ ʑo ɲɯ-pa\cmn 这个孩子摆出一副机灵的样子,真令人讨厌\end{exemple}\begin{sous-entrée}
\vedette{\hypertarget{}{\papi{ ɣɤɕŋaʁɕŋaʁ}}}\markboth{ɣɤɕŋaʁɕŋaʁ}{}\classe{vi}
\paradigme{\textit{dir :} \jya pɯ-}
\begin{définition}\fra parler sans arrêt à tort et à travers\end{définition}
\begin{définition}\cmn 不停地乱说话\end{définition}
\begin{exemple}\jya aʁɤndɯndɤt pɯ-tɯ-ɣɤɕŋaʁɕŋaʁ ntsɯ\cmn 你四处制造谎言\end{exemple}
\end{sous-entrée}\end{entrée}

\begin{entrée}
\vedette{\hypertarget{Ⓔɕŋɤr}{\papi{ ɕŋɤr}}}\markboth{ɕŋɤr}{}
\classe{n}
\begin{définition}\fra givre\end{définition}
\begin{définition}\cmn 霜\end{définition}
\begin{exemple}\jya ɕŋɤr pjɤ-ta\cmn 下了霜\end{exemple}
\begin{exemple}\jya ɕŋɤr pjɤ-ɣi\cmn 下了霜\end{exemple}
\begin{relation-sémantique}\synonyme{
\hyperlink{Ⓔtɯɣur}{\textit{ \papi{tɯɣur}}}
}\end{relation-sémantique}\end{entrée}

\begin{entrée}
\vedette{\hypertarget{Ⓔɕŋiɕŋi}{\papi{ ɕŋiɕŋi}}}\markboth{ɕŋiɕŋi}{}\classe{idph.2}
\begin{définition}\fra qui se retient de rire\end{définition}
\begin{définition}\cmn 形容笑嘻嘻的样子\end{définition}
\begin{exemple}\jya ɕŋiɕŋi ʑo to-ʑɣɤstu\cmn 他做出一副笑嘻嘻的样子\end{exemple}
\end{entrée}

\begin{entrée}
\vedette{\hypertarget{Ⓔɕŋoʁɕŋoʁ}{\papi{ ɕŋoʁɕŋoʁ}}}\markboth{ɕŋoʁɕŋoʁ}{}
\classe{idph.2}
\begin{définition}\fra maigre et laid\end{définition}
\begin{définition}\cmn 形容瘦而不好看的样子\end{définition}
\begin{exemple}\jya kɯ-sɤmbrɯŋgɯ ci ɕŋoʁɕŋoʁ ɲɯ-ŋu\cmn 他又瘦又不好看,让人讨厌\end{exemple}\end{entrée}

\begin{entrée}
\vedette{\hypertarget{Ⓔɕo}{\papi{ ɕo}}}\markboth{ɕo}{}
\classe{vs}
\paradigme{\textit{dir :} \jya nɯ-}
\paradigme{\textit{dir :} \jya tɤ-}
\begin{définition}\fra propre\end{définition}
\begin{définition}\cmn 洗得很干净\end{définition}
\begin{exemple}\jya ɯ-ŋga ɲɯ-ɕo\cmn 他的衣服洗得很干净\end{exemple}\begin{sous-entrée}
\vedette{\hypertarget{}{\papi{ ɣɤɕo}}}\markboth{ɣɤɕo}{}\classe{vt}
\paradigme{\textit{dir :} \jya nɯ-}
\begin{définition}\fra laver\end{définition}
\begin{définition}\cmn 洗干净\end{définition}
\begin{exemple}\jya tɯ-ŋga ɲɤ-ɣɤɕo\cmn 他把衣服洗干净了\end{exemple}
\end{sous-entrée}\end{entrée}

\begin{entrée}
\vedette{\hypertarget{ⒺɕomⒽ1}{\papi{ ɕom}}}\markboth{ɕom}{}\homonyme{1}\classe{n}\acception{1}
\begin{définition}\fra fer\end{définition}
\begin{définition}\cmn 铁\end{définition}\acception{2}
\begin{définition}\fra métal\end{définition}
\begin{définition}\cmn 金属\end{définition}\acception{3}
\begin{définition}\fra clou\end{définition}
\begin{définition}\cmn 铁钉\end{définition}
\begin{relation-sémantique}\synonyme{
\hyperlink{Ⓔtɤtshoʁ}{\textit{ \papi{tɤtshoʁ}}}
}\end{relation-sémantique}
\begin{relation-sémantique}\confer{
\hyperlink{Ⓔɕɤmɯɣdɯ}{\textit{ \papi{ɕɤmɯɣdɯ}}}
}\end{relation-sémantique}
\begin{relation-sémantique}\confer{
\hyperlink{Ⓔɕɤmiŋoʁ}{\textit{ \papi{ɕɤmiŋoʁ}}}
}\end{relation-sémantique}
\begin{relation-sémantique}\confer{
\hyperlink{Ⓔɕɤmloʁ}{\textit{ \papi{ɕɤmloʁ}}}
}\end{relation-sémantique}\end{entrée}

\begin{entrée}
\vedette{\hypertarget{ⒺɕomⒽ2}{\papi{ ɕom}}}\markboth{ɕom}{}\homonyme{2}
\classe{n}
\begin{définition}\fra peau du lait\end{définition}
\begin{définition}\cmn 奶皮\end{définition}
\begin{relation-sémantique}\confer{
\hyperlink{Ⓔrɤɕom}{\textit{ \papi{rɤɕom}}}
}\end{relation-sémantique}
\begin{relation-sémantique}\confer{
\hyperlink{Ⓔtɤlɤɕom}{\textit{ \papi{tɤlɤɕom}}}
}\end{relation-sémantique}\end{entrée}

\begin{entrée}
\vedette{\hypertarget{Ⓔɕommbri}{\papi{ ɕommbri}}}\markboth{ɕommbri}{}\classe{n}
\begin{définition}\fra chaîne\end{définition}
\begin{définition}\cmn 铁链子\end{définition}
\begin{relation-sémantique}\confer{
\hyperlink{ⒺɕomⒽ1}{\textit{ \papi{ɕom}}}
}\end{relation-sémantique}
\begin{relation-sémantique}\confer{
\hyperlink{Ⓔtɯmbri}{\textit{ \papi{tɯmbri}}}
}\end{relation-sémantique}
\end{entrée}

\begin{entrée}
\vedette{\hypertarget{Ⓔɕomnaʁ}{\papi{ ɕomnaʁ}}}\markboth{ɕomnaʁ}{}\classe{n}
\begin{définition}\fra fer noir\end{définition}
\begin{définition}\cmn 黑铁
\begin{déclaration} \étymologie{\papi{nag}}\end{déclaration}\end{définition}
\begin{relation-sémantique}\confer{
\hyperlink{ⒺɕomⒽ1}{\textit{ \papi{ɕom1}}}
}\end{relation-sémantique}\end{entrée}

\begin{entrée}
\vedette{\hypertarget{Ⓔɕomskrɯt}{\papi{ ɕomskrɯt}}}\markboth{ɕomskrɯt}{}\classe{n}
\begin{définition}\fra fil de fer\end{définition}
\begin{définition}\cmn 铁丝\end{définition}
\begin{relation-sémantique}\confer{
\hyperlink{Ⓔskrɯt}{\textit{ \papi{skrɯt}}}
}\end{relation-sémantique}
\begin{relation-sémantique}\confer{
\hyperlink{ⒺɕomⒽ1}{\textit{ \papi{ɕom1}}}
}\end{relation-sémantique}
\end{entrée}

\begin{entrée}
\vedette{\hypertarget{Ⓔɕomthɯ}{\papi{ ɕomthɯ}}}\markboth{ɕomthɯ}{}\classe{n}
\begin{définition}\fra casserole en fer\end{définition}
\begin{définition}\cmn 铁锅\end{définition}
\begin{relation-sémantique}\confer{
\hyperlink{ⒺɕomⒽ1}{\textit{ \papi{ɕom1}}}
}\end{relation-sémantique}
\begin{relation-sémantique}\confer{
 \papi{tɯthɯ}
}\end{relation-sémantique}
\end{entrée}

\begin{entrée}
\vedette{\hypertarget{Ⓔɕoŋβzu}{\papi{ ɕoŋβzu}}}\markboth{ɕoŋβzu}{}\classe{n}
\begin{définition}\fra menuisier\end{définition}
\begin{définition}\cmn 木匠\end{définition}
\begin{exemple}\jya ɕoŋβzu ko-spa\cmn 他学会木工了\end{exemple}
\begin{exemple}\jya a-βɣo nɯ ɕoŋβzu ŋu\cmn 我的伯父是个木匠\end{exemple}
\begin{exemple}\jya nɯ-ɕoŋβzu tu-βze-a\cmn 我给他们做木工\end{exemple}
\begin{relation-sémantique}\confer{
\hyperlink{Ⓔrɯɕoŋβzu}{\textit{ \papi{rɯɕoŋβzu}}}
}\end{relation-sémantique}\end{entrée}

\begin{entrée}
\vedette{\hypertarget{Ⓔɕoŋphu}{\papi{ ɕoŋphu}}}\markboth{ɕoŋphu}{}
\classe{n}
\begin{définition}\fra arbre fruitier\end{définition}
\begin{définition}\cmn 果树\end{définition}
\begin{relation-sémantique}\synonyme{
\hyperlink{Ⓔsɯphɯ}{\textit{ \papi{sɯphɯ}}}
}\end{relation-sémantique}\end{entrée}

\begin{entrée}
\vedette{\hypertarget{Ⓔɕoŋtɕa}{\papi{ ɕoŋtɕa}}}\markboth{ɕoŋtɕa}{}\classe{n}
\begin{définition}\fra bois\end{définition}
\begin{définition}\cmn 木料
\begin{déclaration} \étymologie{\papi{ɕiŋ.tɕʰa}}\end{déclaration}\end{définition}
\end{entrée}

\begin{entrée}
\vedette{\hypertarget{Ⓔɕoʁ}{\papi{ ɕoʁ}}}\markboth{ɕoʁ}{}\classe{n}
\begin{définition}\fra sarrasin\end{définition}
\begin{définition}\cmn 荞麦\end{définition}
\end{entrée}

\begin{entrée}
\vedette{\hypertarget{Ⓔɕoʁɕɣa}{\papi{ ɕoʁɕɣa}}}\markboth{ɕoʁɕɣa}{}\classe{n}
\begin{définition}\fra Artemisia suboligata\end{définition}
\begin{définition}\cmn 茶绒蒿\end{définition}
\end{entrée}

\begin{entrée}
\vedette{\hypertarget{Ⓔɕoʁɕoʁ}{\papi{ ɕoʁɕoʁ}}}\markboth{ɕoʁɕoʁ}{}\classe{n}
\begin{définition}\fra papier\end{définition}
\begin{définition}\cmn 纸
\begin{déclaration} \étymologie{\papi{ɕog}}\end{déclaration}\end{définition}
\begin{exemple}\jya ɕoʁɕoʁ kɯ smi mɤ-mphɯr\cmn 纸包不住火(你的坏事总有一天会暴露出来)\end{exemple}
\begin{relation-sémantique}\confer{
\hyperlink{Ⓔarɯɕoʁɕoʁ}{\textit{ \papi{arɯɕoʁɕoʁ}}}
}\end{relation-sémantique}\end{entrée}

\begin{entrée}
\vedette{\hypertarget{Ⓔɕoʁmboʁ}{\papi{ ɕoʁmboʁ}}}\markboth{ɕoʁmboʁ}{}\classe{n}
\begin{définition}\fra sarrasin grillé\end{définition}
\begin{définition}\cmn 荞麦爆花\end{définition}
\begin{relation-sémantique}\confer{
\hyperlink{Ⓔrŋɤmboʁ}{\textit{ \papi{rŋɤmboʁ}}}
}\end{relation-sémantique}
\begin{relation-sémantique}\confer{
\hyperlink{Ⓔɕoʁ}{\textit{ \papi{ɕoʁ}}}
}\end{relation-sémantique}\end{entrée}

\begin{entrée}
\vedette{\hypertarget{Ⓔɕoʁɲaʁ}{\papi{ ɕoʁɲaʁ}}}\markboth{ɕoʁɲaʁ}{}\classe{n}
\begin{définition}\fra espèce de sarrasin\end{définition}
\begin{définition}\cmn 苦荞\end{définition}
\begin{relation-sémantique}\confer{
\hyperlink{Ⓔɕoʁ}{\textit{ \papi{ɕoʁ}}}
}\end{relation-sémantique}\end{entrée}

\begin{entrée}
\vedette{\hypertarget{Ⓔɕoχpi}{\papi{ ɕoχpi}}}\markboth{ɕoχpi}{}\classe{n}
\begin{définition}\fra pain au sarrasin\end{définition}
\begin{définition}\cmn 荞面馍馍
\begin{déclaration}\use{古语}\end{déclaration}\end{définition}
\begin{relation-sémantique}\confer{
\hyperlink{Ⓔɕoʁ}{\textit{ \papi{ɕoʁ}}}
}\end{relation-sémantique}\end{entrée}

\begin{entrée}
\vedette{\hypertarget{Ⓔɕpaʁ}{\papi{ ɕpaʁ}}}\markboth{ɕpaʁ}{}
\classe{vi}
\paradigme{\textit{dir :} \jya nɯ-}
\begin{définition}\fra avoir soif\end{définition}
\begin{définition}\cmn 渴\end{définition}
\begin{exemple}\jya ɯ-pɯ ɲɤ-ɕpaʁ\cmn 他的儿子口渴了\end{exemple}
\begin{exemple}\jya tɯ-ci mɯ-pjɤ-k-ɤʁe-ci tɕe ɲɯ-ɕpaʁ\cmn 他没有喝到水,所以很渴\end{exemple}
\begin{exemple}\jya nɤ-tɯ-ɕpaʁ thaŋnɤ\cmn 你渴了吧\end{exemple}
\begin{exemple}\jya a-mɤ-ɲɯ-sɤ-ɕpɯ-ɕpaʁ, tʂha ku-tshi-a\cmn 我喝茶,免得渴着自己\end{exemple}\end{entrée}

\begin{entrée}
\vedette{\hypertarget{Ⓔɕpɤr}{\papi{ ɕpɤr}}}\markboth{ɕpɤr}{}\classe{idph.1}
\begin{définition}\fra pleurer soudainement\end{définition}
\begin{définition}\cmn 突然大声哭出来的声音\end{définition}
\begin{exemple}\jya tɤ-pɤtso ɕpɤr ʑo ɲɯ-ɣɤwu\cmn 小孩子哇的一声就哭了\end{exemple}\end{entrée}

\begin{entrée}
\vedette{\hypertarget{Ⓔɕpɤrɕpɤr}{\papi{ ɕpɤrɕpɤr}}}\markboth{ɕpɤrɕpɤr}{} (\variante{ɕpɤɕpɤr}) \classe{idph.2}
\begin{définition}\fra rond et large\end{définition}
\begin{définition}\cmn 形容又圆又大的样子(难看)\end{définition}
\begin{exemple}\jya nɯŋaqe ɕpɤɕpɤr ʑo ɲɯ-pa\cmn 牛屎又圆又大\end{exemple}
\begin{exemple}\jya jlaqe ɕpɤɕpɤr ʑo ɲɯ-pa\cmn 犏牛屎又圆又大\end{exemple}
\begin{exemple}\jya tɯ-ci mbala ɯ-ʁrɯ ɕpɤɕpɤr ʑo ɲɯpa\cmn 水牛的角又圆又宽\end{exemple}
\begin{exemple}\jya ɯ-rŋa ra ɲɯ-wxti ɕpɤrɕpɤr ʑo\cmn 他脸很大,不漂亮\end{exemple}\end{entrée}

\begin{entrée}
\vedette{\hypertarget{Ⓔɕpɤrnɤlɤr}{\papi{ ɕpɤrnɤlɤr}}}\markboth{ɕpɤrnɤlɤr}{}
\classe{idph.4}
\begin{définition}\fra voix très forte\end{définition}
\begin{définition}\cmn 讲话声音很大,不注意场合,也不注意礼貌,说话很放肆\end{définition}
\begin{exemple}\jya ɕpɤrnɤlɤr ɲɯ-ʑɣɤstu\cmn 他讲话声音很大,不注意礼貌\end{exemple}
\begin{relation-sémantique}\confer{
\hyperlink{Ⓔɣɤɕpɤɕpɤr}{\textit{ \papi{ɣɤɕpɤɕpɤr}}}
}\end{relation-sémantique}\end{entrée}

\begin{entrée}
\vedette{\hypertarget{Ⓔɕpɤtnɤɕpɤt}{\papi{ ɕpɤtnɤɕpɤt}}}\markboth{ɕpɤtnɤɕpɤt}{}\classe{idph.3}
\begin{définition}\fra éloquent, sans hésitation\end{définition}
\begin{définition}\cmn 形容说话流利不中断的样子\end{définition}
\begin{exemple}\jya ɕpɤɕpɤt nɤ ɕpɤɕpɤt pɯ-fɕat-a\cmn 我讲故事讲得滔滔不绝\end{exemple}\end{entrée}

\begin{entrée}
\vedette{\hypertarget{Ⓔɕpɣo}{\papi{ ɕpɣo}}}\markboth{ɕpɣo}{}\classe{n}
\begin{définition}\fra dix boisseaux\end{définition}
\begin{définition}\cmn 十升\end{définition}\end{entrée}

\begin{entrée}
\vedette{\hypertarget{Ⓔɕpɣowoʁ}{\papi{ ɕpɣowoʁ}}}\markboth{ɕpɣowoʁ}{}\classe{n}
\begin{définition}\fra grande jarre\end{définition}
\begin{définition}\cmn 大坛子\end{définition}\end{entrée}

\begin{entrée}
\vedette{\hypertarget{Ⓔɕphɤβɕphɤβ}{\papi{ ɕphɤβɕphɤβ}}}\markboth{ɕphɤβɕphɤβ}{}\classe{idph.2}
\begin{définition}\fra couché par terre sans bouger\end{définition}
\begin{définition}\cmn 形容卧在地上不动的样子\end{définition}
\begin{exemple}\jya nɯ tɯrme ɕphɤβɕphɤβ ɲɯ-nɯ-rŋgɯ\cmn 那个人躺在地上不动\end{exemple}
\begin{exemple}\jya jla ɕphɤβɕphɤβ ko-nɯ-rŋgɯ\cmn 犏牛躺在地上不动了\end{exemple}
\begin{exemple}\jya ɯ-thoʁ ɕphɤβɕphɤβ ʑo ɲɯ-rɤʑi, kɤ-nɯqambɯmbjom mɯ́j-cha\cmn (鸟)躺在地上,飞不起来了\end{exemple}\end{entrée}

\begin{entrée}
\vedette{\hypertarget{Ⓔɕphɤɕphɤt}{\papi{ ɕphɤɕphɤt}}}\markboth{ɕphɤɕphɤt}{}\classe{idph.2}
\begin{définition}\fra en lamelle\end{définition}
\begin{définition}\cmn 形容很薄的片状物体\end{définition}
\begin{exemple}\jya tɤrɤm ɕphɤɕphɤt ʑo kɯ-pa ɲɤ-βzu\cmn 他作了很薄的木片\end{exemple}\end{entrée}

\begin{entrée}
\vedette{\hypertarget{Ⓔɕphɤɣnɤɕphɤɣ}{\papi{ ɕphɤɣnɤɕphɤɣ}}}\markboth{ɕphɤɣnɤɕphɤɣ}{}
\classe{idph.3}
\begin{définition}\fra bruit (produit en frappant les habits avec force)\end{définition}
\begin{définition}\cmn 啪啪声(洗衣服时,使劲地拍打衣服)\end{définition}
\begin{exemple}\jya tɯ-ŋga ɲɯ-ɤsɯ-χtɕi tɕe, ɕphɤɣnɤɕphɤɣ ɲɯ-ɤsɯ-stu\cmn 她在洗衣服,发出“啪啪”声\end{exemple}
\begin{exemple}\jya tɯ-ŋga kɯ-jaʁ nɯ ra ɲɯ́-wɣ-χtɕi tɕe ɕphɤɣnɤɕphɤɣ pjɯ́-wɣ-rtɤβ tɕe ɲɯ-ɕo ɕti\cmn 洗厚一点的衣服的时候,要使劲地把衣服拍打才会干净\end{exemple}
\begin{relation-sémantique}\confer{
\hyperlink{Ⓔsɤɕphɤɣɕphɤɣ}{\textit{ \papi{sɤɕphɤɣɕphɤɣ}}}
}\end{relation-sémantique}\end{entrée}

\begin{entrée}
\vedette{\hypertarget{Ⓔɕphɤrɕphɤr}{\papi{ ɕphɤrɕphɤr}}}\markboth{ɕphɤrɕphɤr}{}
\classe{idph.2}
\begin{définition}\fra large et mou\end{définition}
\begin{définition}\cmn 形容宽而柔软,向下垂的样子\end{définition}
\begin{exemple}\jya loŋbutɕhi ɯ-rna nɯ ɲɯ-wxti ɕphɤrɕphɤr ʑo\cmn 大象的耳朵又宽又大,软软地耷拉着\end{exemple}
\begin{exemple}\jya qhɤjmbaʁ ɯ-jwaʁ nɯ ɕphɤrɕphɤr ʑo pa\cmn 酸模的叶子又宽又大\end{exemple}\end{entrée}

\begin{entrée}
\vedette{\hypertarget{Ⓔɕphɤt}{\papi{ ɕphɤt}}}\markboth{ɕphɤt}{}
\classe{vt}
\paradigme{\textit{dir :} \jya kɤ-}
\paradigme{\textit{dir :} \jya pɯ-}
\begin{définition}\fra réparer (un habit)\end{définition}
\begin{définition}\cmn 补\end{définition}
\begin{exemple}\jya kɤ-ɕphat-a, pɯ-ɕphat-a\cmn 我补了\end{exemple}
\begin{exemple}\jya ka-ɕphɤt\cmn 他补了\end{exemple}
\begin{exemple}\jya tɯ-ŋga pjɤ-ɴɢraʁ tɕe, kɤ-ɕphɤt\cmn 衣服破了,你把它补一下\end{exemple}\begin{sous-entrée}
\vedette{\hypertarget{}{\papi{ rɤɕphɤt}}}\markboth{rɤɕphɤt}{}\classe{vi}
\paradigme{\textit{dir :} \jya kɤ-}
\paradigme{\textit{dir :} \jya pɯ-}
\begin{définition}\ 
\begin{déclaration}\grammar{apass}\end{déclaration}\end{définition}
\begin{définition}\fra réparer les habits\end{définition}
\begin{définition}\cmn 补衣服\end{définition}
\begin{relation-sémantique}\confer{
\hyperlink{Ⓔtɤ-ɕphɤt}{\textit{ \papi{tɤ-ɕphɤt}}}
}\end{relation-sémantique}
\end{sous-entrée}\end{entrée}

\begin{entrée}
\vedette{\hypertarget{Ⓔɕphɣo}{\papi{ ɕphɣo}}}\markboth{ɕphɣo}{}\classe{vt}
\paradigme{\textit{dir :} \jya \_}
\begin{définition}\ 
\begin{déclaration}\grammar{caus}\end{déclaration}\end{définition}
\begin{définition}\fra emporter\end{définition}
\begin{définition}\cmn 拿走(不让人家发现)\end{définition}
\begin{exemple}\jya jɤ-ɕphɣo-t-a, ɯʑo kɯ ja-ɕphɣo, jɤ-ɕphɣɤm\cmn 我拿走了,他拿走了,你拿走吧\end{exemple}
\begin{exemple}\jya kɯki laχtɕha ki nɤʑɯɣ ɯ́-ra nɤ jɤ-ɕphɣɤm ma tha mɤ-tɯ-βɟɤt\cmn 这个东西,如果你需要的话,你就拿走吧,不然你就得不到\end{exemple}
\begin{exemple}\jya a-sroʁ ju-ɕphɣam-a\cmn 我要逃命\end{exemple}
\begin{exemple}\jya kɤ-ɕphɣo nɯ, laχtɕha tú-wɣ-ndo tɕe jú-wɣ-tsɯm\cmn 
\stylefv{ɕphɣo}的意思就是拿了东西就带走了
\end{exemple}
\begin{relation-sémantique}\confer{
\hyperlink{Ⓔphɣo}{\textit{ \papi{phɣo}}}
}\end{relation-sémantique}
\begin{sous-entrée}
\vedette{\hypertarget{}{\papi{ ʑɣɤɕphɣo}}}\markboth{ʑɣɤɕphɣo}{}\classe{vi}
\paradigme{\textit{dir :} \jya \_}
\begin{définition}\ 
\begin{déclaration}\grammar{refl}\end{déclaration}\end{définition}
\begin{définition}\fra se sauver\end{définition}
\begin{définition}\cmn 逃命\end{définition}
\begin{exemple}\jya jɤ-ʑɣɤɕphɣo-a\cmn 我逃命了\end{exemple}
\end{sous-entrée}\end{entrée}

\begin{entrée}
\vedette{\hypertarget{Ⓔɕphɯɣɕphɯɣ}{\papi{ ɕphɯɣɕphɯɣ}}}\markboth{ɕphɯɣɕphɯɣ}{}
\classe{idph.2}
\begin{définition}\fra trempé\end{définition}
\begin{définition}\cmn 湿润(衣服)\end{définition}
\begin{exemple}\jya a-ŋga ɕphɯɣɕphɯɣ ɲɤ-k-ɤci-ci\cmn 我的衣服湿了\end{exemple}\begin{sous-entrée}
\vedette{\hypertarget{}{\papi{ ɕphɯɣnɤɕphɯɣ}}}\markboth{ɕphɯɣnɤɕphɯɣ}{}\classe{idph.3}
\begin{exemple}\jya a-taʁ tɯ-ci ɕphɯɣnɤɕphɯɣ ta-lɤt\cmn (互相泼水),他朝我洒水了(一次又一次)\end{exemple}
\end{sous-entrée}\end{entrée}

\begin{entrée}
\vedette{\hypertarget{Ⓔɕplaɕpla}{\papi{ ɕplaɕpla}}}\markboth{ɕplaɕpla}{}\classe{idph.2}\acception{1}
\begin{définition}\fra sans poil, chauve\end{définition}
\begin{définition}\cmn 形容光秃无毛的样子\end{définition}\acception{2}
\begin{définition}\fra obstiné\end{définition}
\begin{définition}\cmn 形容固执的样子
\end{définition}
\begin{exemple}\jya ɕplaɕpla ma-tɯ-ʑɣɤstu\cmn 你不要那么固执,不听指挥\end{exemple}\end{entrée}

\begin{entrée}
\vedette{\hypertarget{Ⓔɕploʁɕploʁ}{\papi{ ɕploʁɕploʁ}}}\markboth{ɕploʁɕploʁ}{}\classe{idph.2}
\begin{définition}\fra rond et lisse\end{définition}
\begin{définition}\cmn 形容圆而光滑的样子\end{définition}
\begin{exemple}\jya nɤ-rte nɯ ɕploʁɕploʁ ʑo ɲɯ-pa\cmn 你的帽子很圆,没有帽边\end{exemple}
\begin{relation-sémantique}\confer{
 \papi{χploχploʁ}
}\end{relation-sémantique}\end{entrée}

\begin{entrée}
\vedette{\hypertarget{Ⓔɕpɯt}{\papi{ ɕpɯt}}}\markboth{ɕpɯt}{}\classe{vt}
\acception{1}
\paradigme{\textit{dir :} \jya thɯ-}
\begin{définition}\fra élever\end{définition}
\begin{définition}\cmn 养育;抚养成大\end{définition}
\begin{exemple}\jya ɯʑo kɯ tɯrme ɯ-rɟit ci tha-ɕpɯt\cmn 他养了别人的孩子\end{exemple}\acception{2}
\paradigme{\textit{dir :} \jya kɤ-}
\begin{définition}\fra adopter\end{définition}
\begin{définition}\cmn 收养\end{définition}
\begin{exemple}\jya thɯ-ɕpɯt-a, kɤ-ɕpɯt, kɤ-tɯ-ɕpɯt, ɯʑo kɯ ka-ɕpɯt\end{exemple}
\begin{exemple}\jya kɤ-ɕpɯt sɤcha\cmn 可以养\end{exemple}
\begin{exemple}\jya χpɤltɕin kɯ @wugui ka-ɕpɯt\cmn 柏尔青养了乌龟\end{exemple}
\begin{exemple}\jya ɯʑo kɯ fsapaʁ ci ka-ɕpɯt\cmn 他养了动物\end{exemple}
\begin{exemple}\jya rɯdaʁ cho fsapaʁ ra kɤ-ɕpɯt ɯ-tɯ-ɴqa mɤ-naχtɕɯɣ, rɯdaʁ kɤ-ɕpɯt ɴqa, fsapaʁ ra kɤ-ɕpɯt mbat\cmn 养家畜跟养野生动物难度不同,养家畜容易,养野生动物难。\end{exemple}\end{entrée}

\begin{entrée}
\vedette{\hypertarget{Ⓔɕqɤjɤr}{\papi{ ɕqɤjɤr}}}\markboth{ɕqɤjɤr}{}\classe{n}
\begin{définition}\fra personne qui louche\end{définition}
\begin{définition}\cmn 斜眼\end{définition}\end{entrée}

\begin{entrée}
\vedette{\hypertarget{Ⓔɕqɤnɕqɤn}{\papi{ ɕqɤnɕqɤn}}}\markboth{ɕqɤnɕqɤn}{}\classe{idph.2}
\begin{définition}\fra rouge\end{définition}
\begin{définition}\cmn 形容红色,没有光的样子\end{définition}
\begin{exemple}\jya prɤɲi nɯ ɕqɤnɕqɤn ʑo ɲɯ-ɣɯrni\cmn 晚霞是红色的\end{exemple}\end{entrée}

\begin{entrée}
\vedette{\hypertarget{Ⓔɕqɤt}{\papi{ ɕqɤt}}}\markboth{ɕqɤt}{}\classe{idph.1}\acception{1}
\begin{définition}\fra bruit d'une pierre jetée contre une surface dure\end{définition}
\begin{définition}\cmn 石头扔在硬的平面上发出的声音
\end{définition}\acception{2}
\begin{définition}\fra douleur soudaine\end{définition}
\begin{définition}\cmn 形容突然痛起来的感觉\end{définition}
\begin{exemple}\jya ɕqɤt ʑo ɲɯ-ti\cmn 突然痛起来\end{exemple}
\begin{exemple}\jya a-ku rdɤstaʁ ɕqɤt ʑo ta-lɤt\cmn 他朝我扔了一块石头,啪的一声打到我头上\end{exemple}\end{entrée}

\begin{entrée}
\vedette{\hypertarget{Ⓔɕquɕqu}{\papi{ ɕquɕqu}}}\markboth{ɕquɕqu}{}\classe{idph.2}
\begin{définition}\fra en fronçant les sourcils\end{définition}
\begin{définition}\cmn 形容皱着眉头的样子\end{définition}
\begin{exemple}\jya a-rŋa ɕquɕqu ʑo ku-ru ɲɯ-ŋu\cmn 他皱着眉头地盯着我\end{exemple}
\begin{relation-sémantique}\confer{
\hyperlink{Ⓔqlɯqlɯ}{\textit{ \papi{qlɯqlɯ}}}
}\end{relation-sémantique}
\begin{relation-sémantique}\confer{
\hyperlink{Ⓔɕqhɯɕqhi}{\textit{ \papi{ɕqhɯɕqhi}}}
}\end{relation-sémantique}\end{entrée}

\begin{entrée}
\vedette{\hypertarget{Ⓔɕqudoŋ}{\papi{ ɕqudoŋ}}}\markboth{ɕqudoŋ}{}\classe{n}
\begin{définition}\fra aveugle (insulte)\end{définition}
\begin{définition}\cmn 瞎子(骂人的话)\end{définition}
\end{entrée}

\begin{entrée}
\vedette{\hypertarget{Ⓔɕqhaloʁ}{\papi{ ɕqhaloʁ}}}\markboth{ɕqhaloʁ}{}\classe{n}
\begin{définition}\fra bâton qui sert à caler la porte\end{définition}
\begin{définition}\cmn 门闩\end{définition}
\begin{exemple}\jya ɕqhaloʁ nɯ-lat-a\cmn 我拴了门\end{exemple}\end{entrée}

\begin{entrée}
\vedette{\hypertarget{Ⓔɕqhlɤt}{\papi{ ɕqhlɤt}}}\markboth{ɕqhlɤt}{}\classe{vi}
\paradigme{\textit{dir :} \jya pɯ-}
\paradigme{\textit{dir :} \jya \_}\acception{1}
\begin{définition}\fra disparaître\end{définition}
\begin{définition}\cmn 消失\end{définition}
\begin{exemple}\jya nɤ-skɤt nɯ ju-ɕqhlɤt kɯ-fse li ju-nɯɬoʁ kɯ-fse ɲɯ-ŋu\cmn 你的声音好像一会儿消失,一会儿又出现了(电话连接有问题)\end{exemple}\acception{2}
\begin{définition}\fra tomber dans\end{définition}
\begin{définition}\cmn 掉进去\end{définition}
\begin{exemple}\jya jɤ-ari tɕe jɤ-ɕqhlɤt\cmn 他去了就掉进去了\end{exemple}
\begin{exemple}\jya nɤki tʂu kɯspoʁ nɯtɕu tɯ-ɕqhlɤt nɤ !\cmn 路中有洞,小心不要掉进去\end{exemple}
\begin{exemple}\jya aʑo kɯspoʁ mɯ-pjɤ-mto-t-a tɕe pjɤ-ɕqhlat-a\cmn 我没有看见有洞,就掉进去了\end{exemple}\acception{3}
\begin{définition}\fra sombrer\end{définition}
\begin{définition}\cmn 沉下去\end{définition}\begin{sous-entrée}
\vedette{\hypertarget{}{\papi{ sɯɕqhlɤt}}}\markboth{sɯɕqhlɤt}{}\classe{vt}
\begin{définition}\fra faire disparaître\end{définition}
\begin{définition}\cmn 使消失\end{définition}
\end{sous-entrée}\begin{sous-entrée}
\vedette{\hypertarget{}{\papi{ ʑɣɤsɯɕqhlɤt}}}\markboth{ʑɣɤsɯɕqhlɤt}{}\classe{vi}
\paradigme{\textit{dir :} \jya \_}
\begin{définition}\ 
\begin{déclaration}\grammar{refl}\end{déclaration}
\begin{déclaration}\grammar{caus}\end{déclaration}\end{définition}
\begin{définition}\fra faire en sorte de disparaître\end{définition}
\begin{définition}\cmn 使自己消失\end{définition}
\begin{exemple}\jya ɯʑo kɤ-ari nɤ kɤ-ari tɕe, kɤ-ʑɣɤsɯɕqhlɤt\cmn 他走下去,走下去,最后不见了踪影\end{exemple}
\end{sous-entrée}\end{entrée}

\begin{entrée}
\vedette{\hypertarget{Ⓔɕqhɯɕqhi}{\papi{ ɕqhɯɕqhi}}}\markboth{ɕqhɯɕqhi}{}\classe{idph.2}
\begin{définition}\fra regardant en travers\end{définition}
\begin{définition}\cmn 形容斜着眼睛看的样子\end{définition}
\begin{exemple}\jya ɕqhɯɕqhi ʑo ku-ru ɲɯ-ŋu\cmn 他斜着眼睛看着他\end{exemple}
\begin{relation-sémantique}\confer{
\hyperlink{Ⓔɕquɕqu}{\textit{ \papi{ɕquɕqu}}}
}\end{relation-sémantique}
\begin{relation-sémantique}\confer{
\hyperlink{Ⓔqlɯqlɯ}{\textit{ \papi{qlɯqlɯ}}}
}\end{relation-sémantique}\end{entrée}

\begin{entrée}
\vedette{\hypertarget{Ⓔɕqlɤβɕqlɤβ}{\papi{ ɕqlɤβɕqlɤβ}}}\markboth{ɕqlɤβɕqlɤβ}{}
\classe{idph.2}
\begin{définition}\fra ayant un air mécontent\end{définition}
\begin{définition}\cmn 形容不高兴的眼神的样子\end{définition}
\begin{exemple}\jya ɯʑo ɯ-sɯm mɯ́j-ɕe tɕe, a-ɕki ɕqlɤβɕqlɤβ ʑo ju-ru ɲɯ-ŋu\cmn 他用不高兴的眼神看着我\end{exemple}\end{entrée}

\begin{entrée}
\vedette{\hypertarget{Ⓔɕqlɯβnɤɕqlɯβ}{\papi{ ɕqlɯβnɤɕqlɯβ}}}\markboth{ɕqlɯβnɤɕqlɯβ}{}\classe{idph.3}
\begin{définition}\fra bruit de personne qui marche dans l'eau, bruit produit lorsque l'on craque des doigts\end{définition}
\begin{définition}\cmn 在水里走的声音,大口地喝水的声音\end{définition}
\begin{exemple}\jya ɕqlɯβnɤɕqlɯβ kɤ-tshi-t-a\cmn 我咕噜咕噜地喝了\end{exemple}
\begin{exemple}\jya ɯ-jaʁndzu ɕqlɯβnɤɕqlɯβ ʑo ta-ʑmbri\cmn 他把手指关节拉得嘎嘣嘎嘣响\end{exemple}
\begin{relation-sémantique}\confer{
\hyperlink{Ⓔqlɯβ}{\textit{ \papi{qlɯβ}}}
}\end{relation-sémantique}\end{entrée}

\begin{entrée}
\vedette{\hypertarget{Ⓔɕqraʁ}{\papi{ ɕqraʁ}}}\markboth{ɕqraʁ}{}
\classe{vs}
\paradigme{\textit{dir :} \jya tɤ-}
\begin{définition}\fra intelligent\end{définition}
\begin{définition}\cmn 聪明\end{définition}
\begin{exemple}\jya ɲɯ-ɕqraʁ\cmn 他很聪明\end{exemple}\begin{sous-entrée}
\vedette{\hypertarget{}{\papi{ nɤɕqraʁ}}}\markboth{nɤɕqraʁ}{}\classe{vt}
\begin{définition}\ 
\begin{déclaration}\grammar{trop}\end{déclaration}\end{définition}
\begin{définition}\fra trouver intelligent\end{définition}
\begin{définition}\cmn 觉得聪明\end{définition}
\begin{exemple}\jya ɯʑo ndɤre, aʑo wuma ɲɯ-nɤɕqraʁ-a\cmn 我觉得他很聪明\end{exemple}
\end{sous-entrée}\begin{sous-entrée}
\vedette{\hypertarget{}{\papi{ sɯɕqraʁ}}}\markboth{sɯɕqraʁ}{}\classe{vt}
\paradigme{\textit{dir :} \jya tɤ-}
\begin{définition}\fra rendre intelligent\end{définition}
\begin{définition}\cmn 令……变聪明\end{définition}
\begin{exemple}\jya tɤkhe kɤ-sɯɕqraʁ ra\cmn 要令笨蛋变聪明\end{exemple}
\end{sous-entrée}\begin{sous-entrée}
\vedette{\hypertarget{}{\papi{ ʑɣɤsɯɕqraʁ}}}\markboth{ʑɣɤsɯɕqraʁ}{}\classe{vi}
\paradigme{\textit{dir :} \jya tɤ-}
\begin{définition}\fra se rendre intelligent\end{définition}
\begin{définition}\cmn 令自己变聪明\end{définition}
\begin{exemple}\jya tɯʑo tu-kɯ-ʑɣɤsɯɕqraʁ ra\cmn 要想办法令自己变聪明\end{exemple}
\begin{relation-sémantique}\confer{
\hyperlink{Ⓔznɤɕqɯɕqraʁ}{\textit{ \papi{znɤɕqɯɕqraʁ}}}
}\end{relation-sémantique}
\end{sous-entrée}\end{entrée}

\begin{entrée}
\vedette{\hypertarget{Ⓔɕquwa}{\papi{ ɕquwa}}}\markboth{ɕquwa}{}
\classe{n}
\begin{définition}\fra aveugle\end{définition}
\begin{définition}\cmn 瞎子\end{définition}
\begin{exemple}\jya ɕquwa nɯ-aβzu-a\cmn 我瞎了\end{exemple}
\begin{relation-sémantique}\confer{
\hyperlink{Ⓔaɕquwa}{\textit{ \papi{aɕquwa}}}
}\end{relation-sémantique}\end{entrée}

\begin{entrée}
\vedette{\hypertarget{Ⓔɕur}{\papi{ ɕur}}}\markboth{ɕur}{}
\classe{vi}
\paradigme{\textit{dir :} \jya pɯ-}
\begin{définition}\fra payer une amende\end{définition}
\begin{définition}\cmn 罚款
\begin{déclaration} \étymologie{\papi{ɕor}}\end{déclaration}\end{définition}
\begin{exemple}\jya tʂu tɕe, ɯ-stu tu-kɯ-ŋke ra ma @fakuan kɯ-fse ɕur\cmn 路上不要违反交通规则,不然就会(被)罚款\end{exemple}
\begin{exemple}\jya ɯʑo kɯ @qiche ko-sɯrpu tɕe pjɤ-ɕur\cmn 他撞了车,所以被罚款\end{exemple}
\begin{exemple}\jya pɕawtʂɯ ɲo-kho pjɤ-ra, pɕawtsɯ khro pjɤ-ɕur\cmn 罚了很多钱\end{exemple}
\begin{exemple}\jya ɕ-to-olɯlɤt tɕe, rŋɯl ɣurʑa pjɤ-ɕur\cmn 因为他跟别人打架了,被罚了一百块\end{exemple}\begin{sous-entrée}
\vedette{\hypertarget{}{\papi{ sɯxɕur}}}\markboth{sɯxɕur}{}\classe{vt}
\paradigme{\textit{dir :} \jya pɯ-}
\begin{définition}\fra faire payer une amende\end{définition}
\begin{définition}\cmn 让人交罚款\end{définition}
\end{sous-entrée}\end{entrée}

\begin{entrée}
\vedette{\hypertarget{ⒺɕriⒽ1}{\papi{ ɕri}}}\markboth{ɕri}{}\homonyme{1}
\classe{vs}
\begin{définition}\fra avoir une fuite\end{définition}
\begin{définition}\cmn (从小的洞或缝里)漏出来\end{définition}
\begin{exemple}\jya tɯ-ci pjɤ-ɕri\cmn 漏了水\end{exemple}
\begin{exemple}\jya qale ɲɤ-ɕri\cmn 漏了空气\end{exemple}
\begin{exemple}\jya tɯthɯ cho tɯŋgu ni ndʑi-pɤrthɤβ nɯtɕu, tɤjlɤβ mɤ-kɯ-ɕri tu-βzu-nɯ\cmn 把锅子和锅子之间的缝隙密封住,不让水蒸气冒出来\end{exemple}
\begin{relation-sémantique}\confer{
\hyperlink{Ⓔaɕoʁri}{\textit{ \papi{aɕoʁri}}}
}\end{relation-sémantique}\end{entrée}

\begin{entrée}
\vedette{\hypertarget{ⒺɕriⒽ2}{\papi{ ɕri}}}\markboth{ɕri}{}\homonyme{2}\classe{vt}
\paradigme{\textit{dir :} \jya pɯ-}
\paradigme{\textit{dir :} \jya nɯ-}
\begin{définition}\fra coudre\end{définition}
\begin{définition}\cmn 交叉着缝\end{définition}
\begin{exemple}\jya aʑo tɤ-sno ɯ-jaʁ pɯ-ɕri-t-a\cmn 我缝了马鞍的垫子\end{exemple}\end{entrée}

\begin{entrée}
\vedette{\hypertarget{Ⓔɕʁɯznɤɕʁɯz}{\papi{ ɕʁɯznɤɕʁɯz}}}\markboth{ɕʁɯznɤɕʁɯz}{}\classe{idph.3}
\begin{définition}\fra croquant\end{définition}
\begin{définition}\cmn 形容吃东西发出声音,干而脆的样子\end{définition}\end{entrée}

\begin{entrée}
\vedette{\hypertarget{Ⓔɕte}{\papi{ ɕte}}}\markboth{ɕte}{}\classe{vt}
\paradigme{\textit{dir :} \jya nɯ-}
\begin{définition}\fra contaminer, infecter\end{définition}
\begin{définition}\cmn 传染\end{définition}
\begin{exemple}\jya ɲɤ-ɕte-t-a, ɲɤ-tɯ-ɕte-t, ɲɤ-ɕte\cmn 我传染给他,你传染给他,他传染给他\end{exemple}
\begin{exemple}\jya tɯ́-wɣ-ɕte\cmn 他会传染给你\end{exemple}
\begin{exemple}\jya ɯʑo ɲɯ-nɯtɕhomba tɕe ɲɤ́-wɣ-ɕte-a\cmn 他把感冒传染给我了\end{exemple}
\begin{exemple}\jya a-tɕhomba ɣɤʑu tɕe, ɲɤ-ɕte-t-a\cmn 我感冒了,传染给他(传染给你了)\end{exemple}\begin{sous-entrée}
\vedette{\hypertarget{}{\papi{ aɕtɯɕte}}}\markboth{aɕtɯɕte}{}\classe{vi}
\begin{définition}\fra se contaminer les uns les autres\end{définition}
\begin{définition}\cmn 互相传染\end{définition}
\begin{exemple}\jya tɯ-ŋgo nɯ ɲɤ-k-ɤɕtɯɕte-nɯ-ci\cmn 他们互相传染了\end{exemple}
\end{sous-entrée}\begin{sous-entrée}
\vedette{\hypertarget{}{\papi{ sɤɕte}}}\markboth{sɤɕte}{}\classe{vs}
\begin{définition}\fra qui se transmet facilement (maladie)\end{définition}
\begin{définition}\cmn 容易传染\end{définition}
\begin{exemple}\jya ki tɯ-ŋgo ki ɲɯ-sɤɕte tɕe kɤ-rɯndzaŋspa ɲɯ-ra\cmn 这种病容易传染,要注意\end{exemple}
\end{sous-entrée}\end{entrée}

\begin{entrée}
\vedette{\hypertarget{Ⓔɕthrɤβɕthrɤβ}{\papi{ ɕthrɤβɕthrɤβ}}}\markboth{ɕthrɤβɕthrɤβ}{}\classe{idph.2}
\begin{définition}\fra objet long et mou\end{définition}
\begin{définition}\cmn 形容又破烂又长的样子\end{définition}
\begin{exemple}\jya tɯ-ŋga ɕthrɤβɕthrɤβ ʑo kɯ-pa to-ŋga\cmn 他穿了又破烂又长的衣服\end{exemple}
\begin{exemple}\jya tɤ-jwaʁ pjɤ-lni ʑo ɕthrɤβɕthrɤβ\cmn 叶子受热后耷拉着,显得很柔软的样子\end{exemple}
\begin{relation-sémantique}\confer{
\hyperlink{Ⓔʑdrɤβʑdrɤβ}{\textit{ \papi{ʑdrɤβʑdrɤβ}}}
}\end{relation-sémantique}\end{entrée}

\begin{entrée}
\vedette{\hypertarget{Ⓔɕthɯz}{\papi{ ɕthɯz}}}\markboth{ɕthɯz}{}
\classe{vt}\acception{1}
\paradigme{\textit{dir :} \jya \_}
\begin{définition}\fra tourner vers\end{définition}
\begin{définition}\cmn 朝向;对着\end{définition}
\begin{exemple}\jya ɯʑo kɯ ko-ɕthɯz, ka-ɕthɯz\cmn 对着他,朝向他\end{exemple}
\begin{exemple}\jya nɤ-ɕki ku-ɕthɯz-a\cmn 我把它朝向你\end{exemple}
\begin{exemple}\jya nɤki kɤ-ɕthɯz tɕe a-pɯ-mtam-a\cmn 给我看一下这个东西\end{exemple}
\begin{exemple}\jya ɯʑo kɯ tɯ-pɤr nɯ ka-ɕthɯz tɕe pɯ-mto-t-a\cmn 他把照片对着(我),我就看了一下\end{exemple}
\begin{exemple}\jya ɕɤɣ ɯ-khɯ tɤ-nɯ-ɕthɯz-a\cmn 我闻了柏树的烟子(治病的方法)\end{exemple}\acception{2}
\begin{définition}\fra prendre (eau)\end{définition}
\begin{définition}\cmn 接住(水)\end{définition}\begin{sous-entrée}
\vedette{\hypertarget{}{\papi{ nɤɕthɯɕthɯz}}}\markboth{nɤɕthɯɕthɯz}{}\classe{vt}
\begin{définition}\fra montrer à tout le monde\end{définition}
\begin{définition}\cmn 给大家看\end{définition}
\begin{relation-sémantique}\confer{
\hyperlink{Ⓔʑɣɤɕthɯz}{\textit{ \papi{ʑɣɤɕthɯz}}}
}\end{relation-sémantique}
\end{sous-entrée}\end{entrée}

\begin{entrée}
\vedette{\hypertarget{Ⓔɕtraŋɕtraŋ}{\papi{ ɕtraŋɕtraŋ}}}\markboth{ɕtraŋɕtraŋ}{}
\classe{idph.2}
\begin{définition}\fra mou et long\end{définition}
\begin{définition}\cmn 形容又软又长的样子\end{définition}
\begin{exemple}\jya razmbe ɕtraŋɕtraŋ ʑo ɲɯ-ɴqoʁ\cmn 烂布条又长又脏地挂在那里\end{exemple}\begin{sous-entrée}
\vedette{\hypertarget{}{\papi{ ɕtraŋnɤɕtraŋ}}}\markboth{ɕtraŋnɤɕtraŋ}{}\classe{idph.3}
\begin{exemple}\jya ɕtraŋnɤɕtraŋ ʑo ɲɯ-ŋke\end{exemple}
\end{sous-entrée}\begin{sous-entrée}
\vedette{\hypertarget{}{\papi{ ɕtrɯŋɯɕtraŋi}}}\markboth{ɕtrɯŋɯɕtraŋi}{}\classe{idph.8}
\begin{exemple}\jya tɯ-mbri ɕtrɯŋɯɕtraŋi ɲɯ-xcat\cmn 绳子很多,很凌乱\end{exemple}
\end{sous-entrée}\end{entrée}

\begin{entrée}
\vedette{\hypertarget{Ⓔɕtrɤβɕtrɤβ}{\papi{ ɕtrɤβɕtrɤβ}}}\markboth{ɕtrɤβɕtrɤβ}{}\classe{idph.2}
\begin{définition}\fra objet long et mou\end{définition}
\begin{définition}\cmn 形容长而软,没有精神的样子\end{définition}
\begin{exemple}\jya ɯʑo lo-βzi tɕe, ɕtrɤβɕtrɤβ ʑo ɲɯ-rŋgɯ\cmn 他喝醉了,无精打采地瘫在那里\end{exemple}\end{entrée}

\begin{entrée}
\vedette{\hypertarget{Ⓔɕtriɕtri}{\papi{ ɕtriɕtri}}}\markboth{ɕtriɕtri}{}\classe{idph.2}
\begin{définition}\fra mou, long et fin\end{définition}
\begin{définition}\cmn 形容柔软,细而长的样子\end{définition}
\begin{exemple}\jya mɯntoʁ ɲɤ-lni ɕtriɕtri ʑo\cmn 花晒蔫了,立不起来\end{exemple}
\begin{exemple}\jya ɕtriɕtri ma-tɯ-ʑɣɤstu\cmn 你不要这么蔫不拉叽\end{exemple}\end{entrée}

\begin{entrée}
\vedette{\hypertarget{Ⓔɕtʂaŋɕtʂaŋ}{\papi{ ɕtʂaŋɕtʂaŋ}}}\markboth{ɕtʂaŋɕtʂaŋ}{}
\classe{idph.2}
\begin{définition}\fra accroché\end{définition}
\begin{définition}\cmn 吊着\end{définition}
\begin{exemple}\jya ɕtʂaŋɕtʂaŋ ɲɯ-ɴqoʁ\cmn 吊着\end{exemple}\begin{sous-entrée}
\vedette{\hypertarget{}{\papi{ ɕtʂaŋnɤɕtʂaŋ}}}\markboth{ɕtʂaŋnɤɕtʂaŋ}{}\classe{idph.3}
\begin{exemple}\jya ɯ-jaʁ ɕtʂaŋnɤɕtʂaŋ ɲɯ-ɤsɯ-stu kɤ-ari\cmn 他甩着就去了\end{exemple}
\end{sous-entrée}\begin{sous-entrée}
\vedette{\hypertarget{}{\papi{ ɕtʂaŋnɤlaŋ}}}\markboth{ɕtʂaŋnɤlaŋ}{}\classe{idph.4}
\begin{exemple}\jya tɯ-nga to-ɕɯɴqoʁ-a tɕe, qale kɯ ɕtʂaŋnɤlaŋ ɲɯ-ɤsɯ-stu\cmn 我挂了衣服,被风吹来吹去\end{exemple}
\end{sous-entrée}\begin{sous-entrée}
\vedette{\hypertarget{}{\papi{ ɣɤɕtʂaŋlaŋ}}}\markboth{ɣɤɕtʂaŋlaŋ}{}\classe{vi}
\paradigme{\textit{dir :} \jya tɤ-}
\begin{définition}\fra se balancer\end{définition}
\begin{définition}\cmn 摇摆;摇晃(吊着的东西)\end{définition}
\begin{exemple}\jya to-ɣɤɕtʂaŋlaŋ\cmn 摇晃了\end{exemple}
\begin{exemple}\jya laχtɕha to-ɕɯɴqoʁ tɕe, ɲɯ-ɣɤɕtʂaŋlaŋ\cmn 挂着的东西在摇晃\end{exemple}
\begin{exemple}\jya nɯ mɯ-pɯ-fse ri to-ɣɤɕtʂaŋlaŋ\cmn 原来不是这样,现在就在那里吊着摇晃\end{exemple}
\end{sous-entrée}\begin{sous-entrée}
\vedette{\hypertarget{}{\papi{ sɤɕtʂaŋlaŋ}}}\markboth{sɤɕtʂaŋlaŋ}{}\classe{vt}
\paradigme{\textit{dir :} \jya tɤ-}
\begin{définition}\fra balancer\end{définition}
\begin{définition}\cmn 甩来甩去\end{définition}
\begin{exemple}\jya ɯ-jaʁ ci ɲɯ-sɤɕtʂaŋlaŋ\cmn 他把手甩来甩去\end{exemple}
\end{sous-entrée}\end{entrée}

\begin{entrée}
\vedette{\hypertarget{Ⓔɕtʂat}{\papi{ ɕtʂat}}}\markboth{ɕtʂat}{}
\classe{vt}
\paradigme{\textit{dir :} \jya nɯ-}
\paradigme{\textit{dir :} \jya thɯ-}
\begin{définition}\fra économiser\end{définition}
\begin{définition}\cmn 节省\end{définition}
\begin{exemple}\jya nɯ-ɕtʂat-a, nɯ-tɯ-ɕtʂat, na-ɕtʂat\cmn 我节约了,你节约了,他节约了\end{exemple}
\begin{exemple}\jya kɯki laχtɕha nɯ kɤ-ɕtʂat ra\cmn 这些东西要省着点用\end{exemple}
\begin{exemple}\jya rŋɯl ra kɤ-ɕtʂat ra ma ʑatsa arɕo\cmn 钱要省着点用,不然很快就用光了\end{exemple}
\begin{exemple}\jya tɯ-ŋga nɯ-ɕtʂat\cmn 你要节约穿衣\end{exemple}\begin{sous-entrée}
\vedette{\hypertarget{}{\papi{ rɤɕtʂat}}}\markboth{rɤɕtʂat}{}\classe{vi}
\paradigme{\textit{dir :} \jya nɯ-}
\begin{définition}\ 
\begin{déclaration}\grammar{apass}\end{déclaration}\end{définition}
\begin{définition}\fra économiser\end{définition}
\begin{définition}\cmn 节约用东西\end{définition}
\begin{exemple}\jya ɯʑo wuma ɲɯ-rɤɕtʂat\cmn 他很节约\end{exemple}
\end{sous-entrée}\begin{sous-entrée}
\vedette{\hypertarget{}{\papi{ ʑɣɤɕtʂat}}}\markboth{ʑɣɤɕtʂat}{}\classe{vi}
\paradigme{\textit{dir :} \jya nɯ-}
\begin{définition}\ 
\begin{déclaration}\grammar{refl}\end{déclaration}\end{définition}
\begin{définition}\fra économiser ses forces\end{définition}
\begin{définition}\cmn 节省自己的体力\end{définition}
\begin{exemple}\jya ɕɤxɕo ɲɯ-ʑɣɤɕtʂat-a ɲɯ-ntshi ma a-phoŋbu mɤ-cha ɲɯ-ŋu\cmn 我这几天要节省自己的体力,因为身体不行\end{exemple}
\end{sous-entrée}\end{entrée}

\begin{entrée}
\vedette{\hypertarget{Ⓔɕtʂo}{\papi{ ɕtʂo}}}\markboth{ɕtʂo}{}
\classe{vt}\acception{1}
\paradigme{\textit{dir :} \jya tɤ-}
\begin{définition}\fra mesurer (avec une louche)\end{définition}
\begin{définition}\cmn 用瓢量(食物)\end{définition}\acception{2}
\paradigme{\textit{dir :} \jya nɯ-}
\begin{définition}\fra mesurer\end{définition}
\begin{définition}\cmn 量\end{définition}
\begin{exemple}\jya tɤ-ɕtʂɤm\cmn 你量一下\end{exemple}
\begin{exemple}\jya tɯ-kɤ-ndza nɯ ra kɤ-ɕtʂo kɯ-ra ɕti\cmn 食物要量过(才能吃)\end{exemple}
\begin{exemple}\jya kɯɕɯŋgɯ tɕe, rɟama pjɤ-me tɕe tú-wɣ-ɕtʂo pjɤ-ŋu\cmn 古时候没有秤,都用瓢来量食物\end{exemple}\begin{sous-entrée}
\vedette{\hypertarget{}{\papi{ rɤɕtʂo}}}\markboth{rɤɕtʂo}{}\classe{vi}
\begin{définition}\fra mesurer des choses\end{définition}
\begin{définition}\cmn 量东西
\begin{déclaration}\grammar{apass}\end{déclaration}\end{définition}
\begin{exemple}\jya ʑara ɣɯ ku-rɤɕtʂo-a\cmn 我给他们量东西\end{exemple}
\end{sous-entrée}\end{entrée}

\begin{entrée}
\vedette{\hypertarget{Ⓔɕtʂɯ}{\papi{ ɕtʂɯ}}}\markboth{ɕtʂɯ}{}\classe{vt}\paradigme{\textit{dir :} \jya kɤ-}
\begin{définition}\fra confier, déposer\end{définition}
\begin{définition}\cmn 寄存
\begin{déclaration}\use{动词的宾语是寄放的人,不是寄放的东西}\end{déclaration}\end{définition}
\begin{exemple}\jya kɤ-ɕtʂɯ-t-a, kɤ-tɯ-ɕtʂɯt, ka-ɕtʂɯ\cmn 我存在他那里了,你存在他那里了,他存在他那里了\end{exemple}
\begin{exemple}\jya kɯki laχtɕha ki a-kɤ-tɯ́-wɣ-ɕtʂɯ tɕe, nɤʑo ɯ-pɯ tɤ-pe\cmn 把这个东西存在你那里吧,你把它保管好\end{exemple}
\begin{exemple}\jya kɯki laχtɕha ki nɤʑo ku-ta-ɕtʂɯ\cmn 我把这个东西存在你那里\end{exemple}
\begin{exemple}\jya ɕ-kú-wɣ-ɕtʂɯ\cmn 有人去寄存\end{exemple}
\begin{exemple}\jya ɕ-kɤ-ɕtʂi\cmn 你存在他那里吧\end{exemple}\begin{sous-entrée}
\vedette{\hypertarget{}{\papi{ rɤɕtʂɯ}}}\markboth{rɤɕtʂɯ}{}
\paradigme{\textit{dir :} \jya kɤ-}
\begin{définition}\ 
\begin{déclaration}\grammar{apass}\end{déclaration}\end{définition}
\begin{définition}\fra confier, déposer\end{définition}
\begin{définition}\cmn 寄存\end{définition}
\begin{exemple}\jya laχtɕha kɤ-rɤɕtʂɯ-a\cmn 我把东西寄存起来了\end{exemple}
\end{sous-entrée}\end{entrée}

\begin{entrée}
\vedette{\hypertarget{Ⓔɕtʂɯɣɕtʂɯɣ}{\papi{ ɕtʂɯɣɕtʂɯɣ}}}\markboth{ɕtʂɯɣɕtʂɯɣ}{}\classe{idph.2}
\begin{définition}\fra en suspension, pendant (objet)\end{définition}
\begin{définition}\cmn 形容吊着的东西\end{définition}\begin{sous-entrée}
\vedette{\hypertarget{}{\papi{ ɕtʂɯɣnɤɕtʂɯɣ}}}\markboth{ɕtʂɯɣnɤɕtʂɯɣ}{}\classe{idph.3}
\begin{définition}\fra balancer\end{définition}
\begin{définition}\cmn 摆动着\end{définition}
\begin{exemple}\jya ɯ-jaʁ ɕtʂɯɣnɤɕtʂɯɣ ɲɯ-ɤsɯ-stu kɤ-ari\cmn 他摆动着手就去了\end{exemple}
\end{sous-entrée}\begin{sous-entrée}
\vedette{\hypertarget{}{\papi{ sɤɕtʂɯlɯɣ}}}\markboth{sɤɕtʂɯlɯɣ}{}\classe{vt}
\begin{définition}\fra balancer\end{définition}
\begin{définition}\cmn 摆动着\end{définition}
\begin{exemple}\jya ɯ-jaʁ laχtɕha ɲɯ-sɤɕtʂɯlɯɣ kɤ-ari\cmn 他手上摆动着东西就去了\end{exemple}
\end{sous-entrée}\end{entrée}

\begin{entrée}
\vedette{\hypertarget{Ⓔɕɯ}{\papi{ ɕɯ}}}\markboth{ɕɯ}{}\classe{pro}
\begin{définition}\fra qui\end{définition}
\begin{définition}\cmn 谁\end{définition}
\begin{exemple}\jya andi tɯrme ci jo-ɣi tɕe, ɕɯ ci ku-nnɯ-ŋu kɯma?\cmn 那边来了人,是谁呢?\end{exemple}
\begin{sous-entrée}
\vedette{\hypertarget{}{\papi{ ɕɯmɤɕɯ}}}\markboth{ɕɯmɤɕɯ}{}
\begin{définition}\fra qui que ce soit\end{définition}
\begin{définition}\cmn 无论谁都……
\end{définition}
\begin{exemple}\jya (ɯ-xtsa) ɲɯ-nɯrge tɕe ɕɯmɤɕɯ ʑo tu-sɯrtoʁ ŋu.\cmn 他很喜欢他的鞋子,到处炫耀给别人看\end{exemple}
\begin{relation-sémantique}\confer{
\hyperlink{Ⓔŋotɕuŋondɤt}{\textit{ \papi{ŋotɕuŋondɤt}}}
}\end{relation-sémantique}
\end{sous-entrée}\end{entrée}

\begin{entrée}
\vedette{\hypertarget{Ⓔɕɯchapaja}{\papi{ ɕɯchapaja}}}\markboth{ɕɯchapaja}{}\classe{adv}
\begin{définition}\fra lutter pour ne pas être le dernier\end{définition}
\begin{définition}\cmn 争先恐后\end{définition}
\begin{exemple}\jya ɕɯchapaja ʑo jo-phɣo-nɯ\cmn 他们争先恐后地逃了\end{exemple}\end{entrée}

\begin{entrée}
\vedette{\hypertarget{Ⓔɕɯfka}{\papi{ ɕɯfka}}}\markboth{ɕɯfka}{}\begin{relation-sémantique}\confer{
\hyperlink{ⒺfkaⒽ1}{\textit{ \papi{fka1}}}
}\end{relation-sémantique}\end{entrée}

\begin{entrée}
\vedette{\hypertarget{Ⓔɕɯfkaβ}{\papi{ ɕɯfkaβ}}}\markboth{ɕɯfkaβ}{}\classe{vt}
\begin{relation-sémantique}\confer{
\hyperlink{Ⓔfkaβ}{\textit{ \papi{fkaβ}}}
}\end{relation-sémantique}\end{entrée}

\begin{entrée}
\vedette{\hypertarget{Ⓔɕɯftaʁ}{\papi{ ɕɯftaʁ}}}\markboth{ɕɯftaʁ}{}\classe{vt}
\paradigme{\textit{dir :} \jya kɤ-}
\begin{définition}\fra se souvenir\end{définition}
\begin{définition}\cmn 记得;记住
\begin{déclaration} \étymologie{\papi{btags}}\end{déclaration}\end{définition}
\begin{exemple}\jya kɤ-ɕɯftaʁ-a, kɤ-tɯ-ɕɯftaʁ, ka-ɕɯftaʁ\cmn 我记住了,你记住了,他记住了\end{exemple}
\begin{exemple}\jya tɤ-scoz nɯ koŋla ʑo kú-wɣ-rtoʁ tɕe, ɲɯ́-wɣ-sɯɣʑaʁ mɤɕtʂa kɤ-ɕɯftaʁ mɤ-sɤcha\cmn 书要反复看,反复复习才能记得住\end{exemple}
\begin{exemple}\jya tɯ-rju kɤ-ɕɯftaʁ ɴqa\cmn 句子很难记住\end{exemple}
\begin{exemple}\jya pɯ-kɯ-sɯxɕat-a nɯ ku-ɕɯftaʁ-a ɲɯ-ra\cmn 我要记住你教给我的(知识)\end{exemple}
\begin{relation-sémantique}\antonyme{
\hyperlink{Ⓔjmɯt}{\textit{ \papi{jmɯt}}}
}\end{relation-sémantique}\begin{sous-entrée}
\vedette{\hypertarget{}{\papi{ ɣɤɕɯftaʁ}}}\markboth{ɣɤɕɯftaʁ}{}\classe{vs}
\begin{définition}\ 
\begin{déclaration}\grammar{facil}\end{déclaration}\end{définition}
\begin{définition}\fra avoir une bonne mémoire\end{définition}
\begin{définition}\cmn 记性好\end{définition}
\begin{exemple}\jya ɲɯ-ɕqraʁ tɕe ɲɯ-ɣɤɕɯftaʁ\cmn 他很聪明,记性很好\end{exemple}
\end{sous-entrée}\begin{sous-entrée}
\vedette{\hypertarget{}{\papi{ nɯɣɯɕɯftaʁ}}}\markboth{nɯɣɯɕɯftaʁ}{}\classe{vs}
\begin{définition}\fra facile à mémoriser\end{définition}
\begin{définition}\cmn 容易记住\end{définition}
\begin{relation-sémantique}\antonyme{
\hyperlink{Ⓔnɯɣɯjmɯt}{\textit{ \papi{nɯɣɯjmɯt}}}
}\end{relation-sémantique}
\end{sous-entrée}\end{entrée}

\begin{entrée}
\vedette{\hypertarget{Ⓔɕɯftɯɣ}{\papi{ ɕɯftɯɣ}}}\markboth{ɕɯftɯɣ}{}
\classe{vt}
\paradigme{\textit{dir :} \jya pɯ-}
\paradigme{\textit{dir :} \jya tɤ-}
\begin{définition}\fra achever\end{définition}
\begin{définition}\cmn 完成\end{définition}
\begin{exemple}\jya pɯ-ɕɯftɯɣ-a, pɯ-tɯ-ɕɯftɯɣ, pa-ɕɯftɯɣ\cmn 我完成了,你完成了,他完成了\end{exemple}
\begin{exemple}\jya kɯki ɯ-ro kɯ-dɤn me tɕe, pɯ-ɕɯftɯɣ\cmn 剩下的不多,你把它完成吧\end{exemple}
\begin{exemple}\jya nɤ-kɤ-nɤma pjɯ-kɤ-ɕɯftɯɣ ci pɯ-ri\cmn 你的工作只剩下快要结束的那一段\end{exemple}
\begin{exemple}\jya @xingqiliu tɕɤn ɕɯftɯɣ-tɕi\cmn 我们俩星期六(把这个工作)做完\end{exemple}
\begin{relation-sémantique}\confer{
 \papi{caus}
}\end{relation-sémantique}
\begin{relation-sémantique}\confer{
\hyperlink{Ⓔftɯɣ}{\textit{ \papi{ftɯɣ}}}
}\end{relation-sémantique}\end{entrée}

\begin{entrée}
\vedette{\hypertarget{Ⓔɕɯɣɕɯɣ}{\papi{ ɕɯɣɕɯɣ}}}\markboth{ɕɯɣɕɯɣ}{}\classe{idph.2}
\begin{définition}\fra en silence\end{définition}
\begin{définition}\cmn 形容安静,毫不作声的样子\end{définition}\end{entrée}

\begin{entrée}
\vedette{\hypertarget{Ⓔɕɯɣmu}{\papi{ ɕɯɣmu}}}\markboth{ɕɯɣmu}{}\classe{vt}
\paradigme{\textit{dir :} \jya nɯ-}
\begin{définition}\ 
\begin{déclaration}\grammar{caus}\end{déclaration}\end{définition}\acception{1}
\begin{définition}\fra effrayer\end{définition}
\begin{définition}\cmn 吓唬\end{définition}
\begin{exemple}\jya nɤ-sɤ-ɕɯɣmu tu-βze-a pɯ-ɕti ma ɯ-stu pɯ-maʁ\cmn 我只是吓唬一下,不是真的\end{exemple}
\begin{exemple}\jya ɯ-sɤ-ɕɯɣmu to-βzu\cmn 对他说了吓人的话,做了吓人的动作\end{exemple}\acception{2}
\begin{définition}\fra menacer\end{définition}
\begin{définition}\cmn 威胁\end{définition}
\begin{exemple}\jya ɯʑo to-mɯrkɯ tɕe, aʑo kɯ ɕɯ́-wɣ-ndʑɯ-a ɲɯ-sɯsɤm tɕe, a-sɤ-ɕɯɣmu ɲɯ-ɤsɯ-βzu (ɲɯ́-wɣ-ɕɯɣmu-a ɲɯ-ŋu)\cmn 他偷了东西,怕我去告状就威胁了我\end{exemple}\begin{sous-entrée}
\vedette{\hypertarget{}{\papi{ ʑɣɤɕɯɣmu}}}\markboth{ʑɣɤɕɯɣmu}{}\classe{vi}
\begin{définition}\ 
\begin{déclaration}\grammar{refl}\end{déclaration}
\begin{déclaration}\grammar{caus}\end{déclaration}\end{définition}
\begin{définition}\fra s'effrayer soi-même\end{définition}
\begin{définition}\cmn 自己吓唬自己\end{définition}
\begin{exemple}\jya nɤʑo tɯ-ʑɣɤɕɯɣmu mɤ-ra ma mɤ-ʁdɯɣ\cmn 你不用害怕,没有事\end{exemple}
\begin{relation-sémantique}\confer{
\hyperlink{ⒺmuⒽ1}{\textit{ \papi{mu1}}}
}\end{relation-sémantique}
\end{sous-entrée}\end{entrée}

\begin{entrée}
\vedette{\hypertarget{Ⓔɕɯɣra}{\papi{ ɕɯɣra}}}\markboth{ɕɯɣra}{}\classe{n}
\begin{définition}\fra crible à gros trous\end{définition}
\begin{définition}\cmn 粗罗筛\end{définition}
\begin{relation-sémantique}\synonyme{
\hyperlink{Ⓔtshaʁ}{\textit{ \papi{tshaʁ}}}
}\end{relation-sémantique}
\begin{relation-sémantique}\confer{
\hyperlink{Ⓔsɯɕɯɣra}{\textit{ \papi{sɯɕɯɣra}}}
}\end{relation-sémantique}\end{entrée}

\begin{entrée}
\vedette{\hypertarget{Ⓔɕɯjaʁ}{\papi{ ɕɯjaʁ}}}\markboth{ɕɯjaʁ}{}\classe{n}
\begin{définition}\fra poutre\end{définition}
\begin{définition}\cmn 走檐上的横梁\end{définition}
\begin{relation-sémantique}\synonyme{
\hyperlink{Ⓔtɤsthoʁsi}{\textit{ \papi{tɤsthoʁsi}}}
}\end{relation-sémantique}\end{entrée}

\begin{entrée}
\vedette{\hypertarget{Ⓔɕɯkhuj}{\papi{ ɕɯkhuj}}}\markboth{ɕɯkhuj}{}\classe{n}
\begin{définition}\fra petite bassine avec un verseur\end{définition}
\begin{définition}\cmn 有嘴的小盆子\end{définition}\end{entrée}

\begin{entrée}
\vedette{\hypertarget{Ⓔɕɯm}{\papi{ ɕɯm}}}\markboth{ɕɯm}{}
\classe{vt}
\paradigme{\textit{dir :} \jya kɤ-}
\paradigme{\textit{dir :} \jya pɯ-}
\begin{définition}\fra couver\end{définition}
\begin{définition}\cmn 孵\end{définition}
\begin{définition}\fra dormir avec (un enfant)\end{définition}
\begin{définition}\cmn (跟孩子)一起睡\end{définition}
\begin{exemple}\jya tɤ-pɤtso pɯ-ɕɯm-a\cmn 我让孩子跟我一起睡\end{exemple}
\begin{exemple}\jya kumpɣa kɯ tɤ-ŋgɯm ko-ɕɯm\cmn 母鸡把蛋孵出来了\end{exemple}\end{entrée}

\begin{entrée}
\vedette{\hypertarget{Ⓔɕɯmɤɕɯ}{\papi{ ɕɯmɤɕɯ}}}\markboth{ɕɯmɤɕɯ}{}
\begin{relation-sémantique}\confer{
\hyperlink{Ⓔɕɯ}{\textit{ \papi{ɕɯ}}}
}\end{relation-sémantique}
\end{entrée}

\begin{entrée}
\vedette{\hypertarget{Ⓔɕɯmbɣom}{\papi{ ɕɯmbɣom}}}\markboth{ɕɯmbɣom}{}
\classe{vt}
\paradigme{\textit{dir :} \jya tɤ-}
\begin{définition}\ 
\begin{déclaration}\grammar{caus}\end{déclaration}\end{définition}
\begin{définition}\fra faire plus vite\end{définition}
\begin{définition}\cmn 加快速度\end{définition}
\begin{exemple}\jya kɯki tɤ-scoz ɲɯ-mbɣom, tɤ-ɕɯmbɣom ʑo pɯ-rɤt\cmn 这封信很急,你要写得快些\end{exemple}
\begin{exemple}\jya ki laχtɕha ki kɤ-ndo ra ŋu ŋu nɤ, tɤ-ɕɯmbɣom ʑo jɤ-ɣɯt ma aʑo ɕe-a ŋu\cmn 这个东西要带的话,你快一点带来,不然我要走了\end{exemple}
\begin{relation-sémantique}\confer{
\hyperlink{Ⓔmbɣom}{\textit{ \papi{mbɣom}}}
}\end{relation-sémantique}\begin{sous-entrée}
\vedette{\hypertarget{}{\papi{ ʑɣɤɕɯmbɣom}}}\markboth{ʑɣɤɕɯmbɣom}{}\classe{vi}
\paradigme{\textit{dir :} \jya tɤ-}
\begin{définition}\ 
\begin{déclaration}\grammar{caus}\end{déclaration}
\begin{déclaration}\grammar{refl}\end{déclaration}\end{définition}
\begin{définition}\fra se presser\end{définition}
\begin{définition}\cmn 赶紧……\end{définition}
\end{sous-entrée}\end{entrée}

\begin{entrée}
\vedette{\hypertarget{Ⓔɕɯmɕɯm}{\papi{ ɕɯmɕɯm}}}\markboth{ɕɯmɕɯm}{}
\classe{idph.2}\acception{1}
\begin{définition}\fra formant une couche fine\end{définition}
\begin{définition}\cmn 构成了薄薄的一层\end{définition}\acception{2}
\begin{définition}\fra calme et agréable\end{définition}
\begin{définition}\cmn 很安静;很舒服\end{définition}
\begin{exemple}\jya tɤjpa ɕɯmɕɯm ci ko-lɤt\cmn 下了薄薄的雪\end{exemple}
\begin{exemple}\jya tɯ-mɯ ɕɯmɕɯm nɤ ɕɯmɕɯm ɲɯ-ɤsɯ-lɤt\cmn 在下毛毛雨\end{exemple}
\begin{exemple}\jya a-βri pɯ-nɯ-χtɕi-t-a tɕe, ɕɯmɕɯm ʑo ɲo-pa\cmn 我洗完澡,很舒服\end{exemple}\begin{sous-entrée}
\vedette{\hypertarget{}{\papi{ ɕɯmɯmi}}}\markboth{ɕɯmɯmi}{}\classe{idph.7}
\begin{exemple}\jya tɯ-mɯ ɕɯmɯmi ʑo ɲɯ-lɤt\cmn 雨下得又细又密(很安静)\end{exemple}
\end{sous-entrée}\begin{sous-entrée}
\vedette{\hypertarget{}{\papi{ sɤɕɯmɕɯm}}}\markboth{sɤɕɯmɕɯm}{}\classe{vt}
\begin{exemple}\jya tɯ-mɯ ɲɯ-sɤɕɯmɕɯm ʑo ɲɯ-ɤsɯ-lɤt\cmn 雨下得又细又密\end{exemple}
\end{sous-entrée}\end{entrée}

\begin{entrée}
\vedette{\hypertarget{Ⓔɕɯmnɤm}{\papi{ ɕɯmnɤm}}}\markboth{ɕɯmnɤm}{}
\classe{vt}
\paradigme{\textit{dir :} \jya nɯ-}
\begin{définition}\ 
\begin{déclaration}\grammar{caus}\end{déclaration}\end{définition}
\begin{définition}\fra causer une odeur\end{définition}
\begin{définition}\cmn 导致有味道\end{définition}
\begin{exemple}\jya thamakha ɯ-di nɯ-tɯ-ɕɯmnɤm\cmn (因为你抽了烟),把家里弄得有烟味\end{exemple}
\begin{exemple}\jya cha ɯ-di na-ɕɯmnɤm\cmn (因为喝了酒),就把家里弄得有酒味\end{exemple}
\begin{exemple}\jya kha nɯ tɕu ɕɤɣ pjɯ́-wɣ-sɤkhɯ tɕe, ɯ-di kɤ-ɕɯmnɤm ɲɯ-ra\cmn 要烧柏树让家里有香味\end{exemple}
\begin{relation-sémantique}\confer{
\hyperlink{Ⓔmnɤm}{\textit{ \papi{mnɤm}}}
}\end{relation-sémantique}
\begin{relation-sémantique}\confer{
\hyperlink{Ⓔnɤmnɤm}{\textit{ \papi{nɤmnɤm}}}
}\end{relation-sémantique}\end{entrée}

\begin{entrée}
\vedette{\hypertarget{Ⓔɕɯmɲatsa}{\papi{ ɕɯmɲatsa}}}\markboth{ɕɯmɲatsa}{}\classe{n}
\begin{définition}\fra violon à deux cordes\end{définition}
\begin{définition}\cmn 胡琴\end{définition}
\end{entrée}

\begin{entrée}
\vedette{\hypertarget{Ⓔɕɯmŋɤm}{\papi{ ɕɯmŋɤm}}}\markboth{ɕɯmŋɤm}{}
\classe{vt}
\paradigme{\textit{dir :} \jya tɤ-}
\begin{définition}\ 
\begin{déclaration}\grammar{caus}\end{déclaration}\end{définition}
\begin{définition}\fra faire mal\end{définition}
\begin{définition}\cmn 弄痛\end{définition}
\begin{exemple}\jya tɤ-ɕɯmŋam-a, tɤ-tɯ-ɕɯmŋɤm, ta-ɕɯmŋɤm\cmn 我弄痛了,你弄痛了,他弄痛了\end{exemple}
\begin{exemple}\jya a-mgɯr tɤŋkhɯt ta-lɤt tɕe, ta-ɕɯmŋɤm\cmn 他在我背上打了一拳,把我打得很痛\end{exemple}
\begin{relation-sémantique}\confer{
\hyperlink{Ⓔmŋɤm}{\textit{ \papi{mŋɤm}}}
}\end{relation-sémantique}\end{entrée}

\begin{entrée}
\vedette{\hypertarget{Ⓔɕɯmthu}{\papi{ ɕɯmthu}}}\markboth{ɕɯmthu}{}\classe{vi}
\paradigme{\textit{dir :} \jya tɤ-}
\paradigme{\textit{dir :} \jya thɯ-}
\begin{définition}\fra poser plein de questions\end{définition}
\begin{définition}\cmn 问很多问题\end{définition}
\begin{exemple}\jya ɕɯmthu-a, tɯ-ɕɯmthu, ɕɯmthu\cmn 我问了很多问题,你问了很多问题,他问了很多问题\end{exemple}
\begin{exemple}\jya tɤ-tɯ-ɕɯmthu, to-ɕɯmthu\cmn 你问了很多问题,他问了很多问题\end{exemple}
\begin{exemple}\jya mɯ-tɤ-tɯ-tso tɕe, a-tɤ-tɯ-ɕɯmthu\cmn 你没有懂的话,可以随便问一下\end{exemple}
\begin{exemple}\jya ɯʑo kha lo-nɯɕe tɕe, thɯ-ɕɯmthu-a\cmn 他回家了,我就问了很多问题\end{exemple}
\begin{exemple}\jya nɤʑo ndɤre nɤ-tɯ-ɕɯmthu nɯ!\cmn 你倒是个爱问问题的人\end{exemple}
\begin{relation-sémantique}\confer{
\hyperlink{ⒺthuⒽ1}{\textit{ \papi{thu1}}}
}\end{relation-sémantique}\end{entrée}

\begin{entrée}
\vedette{\hypertarget{Ⓔɕɯmthuspoʁ}{\papi{ ɕɯmthuspoʁ}}}\markboth{ɕɯmthuspoʁ}{}\classe{n}
\begin{définition}\fra enfant qui aimer poser des questions sans cesse\end{définition}
\begin{définition}\cmn 不停问东问西的孩子\end{définition}
\begin{relation-sémantique}\confer{
\hyperlink{Ⓔɕɯmthu}{\textit{ \papi{ɕɯmthu}}}
}\end{relation-sémantique}
\end{entrée}

\begin{entrée}
\vedette{\hypertarget{Ⓔɕɯmɯma}{\papi{ ɕɯmɯma}}}\markboth{ɕɯmɯma}{}\classe{postp}
\begin{définition}\fra juste au moment où\end{définition}
\begin{définition}\cmn 正当……的时候\end{définition}\begin{sous-entrée}
\vedette{\hypertarget{}{\papi{ nɯɕɯmɯma}}}\markboth{nɯɕɯmɯma}{}
\begin{définition}\fra immédiatement\end{définition}
\begin{définition}\cmn 马上就\end{définition}
\end{sous-entrée}\end{entrée}

\begin{entrée}
\vedette{\hypertarget{Ⓔɕɯngo}{\papi{ ɕɯngo}}}\markboth{ɕɯngo}{}
\classe{vt}
\paradigme{\textit{dir :} \jya tɤ-}
\begin{définition}\ 
\begin{déclaration}\grammar{caus}\end{déclaration}\end{définition}
\begin{définition}\fra rendre malade\end{définition}
\begin{définition}\cmn 使生病\end{définition}
\begin{exemple}\jya ta-ɕɯngo\cmn 他让他生病了\end{exemple}
\begin{exemple}\jya kɤndza nɯ mɯ-nɯ-kɯ-sna ra a-mɤ-tɤ́-wɣ-ndza ra ma kɯ-ɕɯngo\cmn 变了味的食物不能吃,不然会因此生病\end{exemple}
\begin{relation-sémantique}\confer{
\hyperlink{Ⓔngo}{\textit{ \papi{ngo}}}
}\end{relation-sémantique}\end{entrée}

\begin{entrée}
\vedette{\hypertarget{Ⓔɕɯnŋo}{\papi{ ɕɯnŋo}}}\markboth{ɕɯnŋo}{}\classe{vt}
\paradigme{\textit{dir :} \jya pɯ-}
\begin{définition}\ 
\begin{déclaration}\grammar{caus}\end{déclaration}\end{définition}
\begin{définition}\fra gagner, vaincre\end{définition}
\begin{définition}\cmn 赢;打败;战胜\end{définition}
\begin{exemple}\jya ta-ɕɯnŋo\cmn 我赢了你\end{exemple}
\begin{exemple}\jya kɯ-ɕɯnŋo-a\cmn 你赢了我\end{exemple}
\begin{exemple}\jya tɕiʑo ni tɤ-nɯ-saχɕɯβ-tɕi ri, pɯ́-wɣ-ɕɯnŋo-a\cmn 我们俩比赛的时候,他打败了我\end{exemple}
\begin{exemple}\jya ɯʑo kɯ ɯ-zda pa-ɕɯnŋo\cmn 我把对方打败了\end{exemple}
\begin{exemple}\jya nɤʑo pɯ-ta-ɕɯnŋo\cmn 我打败了你\end{exemple}
\begin{exemple}\jya tɕiʑo tɤ-aʑɯʑutɕi pɯ-ta-ɕɯnŋo\cmn 在角力的时候,我把你打败了\end{exemple}
\begin{exemple}\jya tɤ-amɯti-tɕi pɯ-ta-ɕɯnŋo\cmn 我把你讲赢了\end{exemple}
\begin{exemple}\jya ɕu pɯ-lɤt-tɕi tɕe pɯ-ta-ɕɯnŋo\cmn 我们俩打牌的时候,我把你打输了\end{exemple}
\begin{exemple}\jya a-χti pɯ-ɕɯnŋo-t-a\cmn 我把对方打败了\end{exemple}\begin{sous-entrée}
\vedette{\hypertarget{}{\papi{ ʑɣɤɕɯnŋo}}}\markboth{ʑɣɤɕɯnŋo}{}\classe{vi}
\begin{définition}\ 
\begin{déclaration}\grammar{refl}\end{déclaration}
\begin{déclaration}\grammar{caus}\end{déclaration}\end{définition}
\begin{définition}\fra causer soi-même sa propre défaite\end{définition}
\begin{définition}\cmn 使自己失败\end{définition}
\begin{relation-sémantique}\synonyme{
\hyperlink{ⒺkoⒽ1}{\textit{ \papi{ko}}}
}\end{relation-sémantique}
\begin{relation-sémantique}\confer{
\hyperlink{Ⓔnŋo}{\textit{ \papi{nŋo}}}
}\end{relation-sémantique}
\end{sous-entrée}\end{entrée}

\begin{entrée}
\vedette{\hypertarget{Ⓔɕɯntaβ}{\papi{ ɕɯntaβ}}}\markboth{ɕɯntaβ}{}
\begin{relation-sémantique}\confer{
\hyperlink{Ⓔntaβ}{\textit{ \papi{ntaβ}}}
}\end{relation-sémantique}\end{entrée}

\begin{entrée}
\vedette{\hypertarget{Ⓔɕɯŋarɯra}{\papi{ ɕɯŋarɯra}}}\markboth{ɕɯŋarɯra}{}\classe{pro}
\begin{définition}\fra meilleurs les uns que les autres\end{définition}
\begin{définition}\cmn 一个比一个好\end{définition}
\end{entrée}

\begin{entrée}
\vedette{\hypertarget{Ⓔɕɯŋɕɯŋ}{\papi{ ɕɯŋɕɯŋ}}}\markboth{ɕɯŋɕɯŋ}{}
\classe{idph.2}
\begin{définition}\fra bruit de friction métallique\end{définition}
\begin{définition}\cmn 铁皮摩擦的声音(如锯子锯东西的时候)\end{définition}\begin{sous-entrée}
\vedette{\hypertarget{}{\papi{ ɣɤɕɯŋɕɯŋ}}}\markboth{ɣɤɕɯŋɕɯŋ}{}\classe{vi}
\begin{exemple}\jya rɟaŋsoʁ ɯ-zgra ɲɯ-ɣɤɕɯŋɕɯŋ\cmn 锯子发出铁皮摩擦声\end{exemple}
\end{sous-entrée}\end{entrée}

\begin{entrée}
\vedette{\hypertarget{ⒺɕɯŋgɯⒽ1}{\papi{ ɕɯŋgɯ}}}\markboth{ɕɯŋgɯ}{}\homonyme{1}\classe{postp}
\begin{définition}\fra avant\end{définition}
\begin{définition}\cmn 之前
\begin{déclaration}\use{\stylefv{ɕɯŋgɯ ɕɯŋgɯ}表示“最里层”的意思}\end{déclaration}\end{définition}
\begin{exemple}\jya ku-ɣi ɕɯŋgɯ χsɯ-sŋi tɕe a-@dianhua tu-lɤt ɯ-ŋu?\cmn 他来之前三天会给我打电话是吗?\end{exemple}
\begin{relation-sémantique}\confer{
\hyperlink{Ⓔɯ-ŋgɯ}{\textit{ \papi{ɯ-ŋgɯ}}}
}\end{relation-sémantique}\end{entrée}

\begin{entrée}
\vedette{\hypertarget{ⒺɕɯŋgɯⒽ2}{\papi{ ɕɯŋgɯ}}}\markboth{ɕɯŋgɯ}{}\homonyme{2}\begin{relation-sémantique}\confer{
\hyperlink{Ⓔtɤ-mbrɯ,ŋgɯ}{\textit{ \papi{tɤ-mbrɯ,ŋgɯ}}}
}\end{relation-sémantique}
\end{entrée}

\begin{entrée}
\vedette{\hypertarget{Ⓔɕɯŋgɯmɯr}{\papi{ ɕɯŋgɯmɯr}}}\markboth{ɕɯŋgɯmɯr}{}\classe{n}
\begin{définition}\fra la veille\end{définition}
\begin{définition}\cmn 前一天\end{définition}
\begin{relation-sémantique}\confer{
\hyperlink{ⒺɕɯŋgɯⒽ1}{\textit{ \papi{ɕɯŋgɯ}}}
}\end{relation-sémantique}
\begin{relation-sémantique}\confer{
\hyperlink{Ⓔtɯ-ɣmɯr}{\textit{ \papi{tɯ-ɣmɯr}}}
}\end{relation-sémantique}
\end{entrée}

\begin{entrée}
\vedette{\hypertarget{Ⓔɕɯŋke}{\papi{ ɕɯŋke}}}\markboth{ɕɯŋke}{}
\begin{relation-sémantique}\confer{
\hyperlink{Ⓔŋke}{\textit{ \papi{ŋke}}}
}\end{relation-sémantique}\end{entrée}

\begin{entrée}
\vedette{\hypertarget{Ⓔɕɯɴqoʁ}{\papi{ ɕɯɴqoʁ}}}\markboth{ɕɯɴqoʁ}{}\classe{vt}
\paradigme{\textit{dir :} \jya tɤ-}
\paradigme{\textit{dir :} \jya pɯ-}
\begin{définition}\ 
\begin{déclaration}\grammar{caus}\end{déclaration}\end{définition}
\begin{définition}\fra accrocher\end{définition}
\begin{définition}\cmn 挂(在上面)\end{définition}
\begin{exemple}\jya tɤ-ɕɯɴqoʁ-a, ta-ɕɯɴqoʁ\cmn 我挂了,他挂了(这个东西)\end{exemple}
\begin{exemple}\jya kɯki laχtɕha ki tɕɤtu nɯ tɕu tɤ-ɕɯɴqoʁ\cmn 你把这个东西挂在上面\end{exemple}
\begin{exemple}\jya tɯpɤr znde ɯ-taʁ tɤ-ɕɯɴqoʁ-a\cmn 我在墙上挂了一幅画\end{exemple}
\begin{exemple}\jya ɯ-jme pa-ɕɯɴqoʁ nɤ, tɤ́-wɣ-rɤɕi ɲɯ-ŋu.\cmn (狐狸)把尾巴伸进(洞)里,把弟弟拉上来了(狐狸.153)\end{exemple}\begin{sous-entrée}
\vedette{\hypertarget{}{\papi{ ʑɣɤɕɯɴqoʁ}}}\markboth{ʑɣɤɕɯɴqoʁ}{}\classe{vi}
\paradigme{\textit{dir :} \jya tɤ-}
\begin{définition}\ 
\begin{déclaration}\grammar{refl}\end{déclaration}
\begin{déclaration}\grammar{caus}\end{déclaration}\end{définition}
\begin{définition}\fra se laisser pendre\end{définition}
\begin{définition}\cmn 让自己身体悬吊着\end{définition}
\begin{exemple}\jya tɤ-pɤtso to-ʑɣɤɕɯɴqoʁ tɕe, ɲɯ-ɤnɯɣro, ɲɯ-ɣɤɕtʂaŋlaŋ ʑo\cmn 小孩子把自己挂上去了,在那里吊着玩\end{exemple}
\end{sous-entrée}\end{entrée}

\begin{entrée}
\vedette{\hypertarget{Ⓔɕɯphɣo}{\papi{ ɕɯphɣo}}}\markboth{ɕɯphɣo}{}
\begin{relation-sémantique}\confer{
\hyperlink{Ⓔphɣo}{\textit{ \papi{phɣo}}}
}\end{relation-sémantique}\end{entrée}

\begin{entrée}
\vedette{\hypertarget{Ⓔɕɯrɕɯr}{\papi{ ɕɯrɕɯr}}}\markboth{ɕɯrɕɯr}{}\classe{idph.2}
\begin{définition}\fra calme\end{définition}
\begin{définition}\cmn 形容静悄悄的,一点声音也没有\end{définition}
\begin{exemple}\jya ɕɯrɕɯr ʑo lo-pa lo-fsoʁ\cmn 天亮都没有亮\end{exemple}
\begin{exemple}\jya jiɕqha ɕɯrɕɯr ʑo ɲɯ-pa mɯ-ɲɯ-ɤrju-nɯ\cmn 很安静,没有人讲话\end{exemple}
\begin{exemple}\jya ɕɯrɕɯr ʑo lo-pa tɕe tɤ-rɤru-a\cmn 天亮都没有亮我就起床了\end{exemple}
\begin{exemple}\jya tɯrme ra ko-nɯrŋgɯ-nɯ tɕe, ɕɯrɕɯr ʑo ɲɯ-pa\cmn 人们睡着了以后,很安静\end{exemple}
\begin{exemple}\jya ɕɤr tɕe, kha ɯ-ŋgɯ ɕɯrɕɯr ʑo ɲɯ-pa\cmn 晚上家里很安静\end{exemple}\end{entrée}

\begin{entrée}
\vedette{\hypertarget{Ⓔɕɯrdɯm}{\papi{ ɕɯrdɯm}}}\markboth{ɕɯrdɯm}{}\classe{n}
\begin{définition}\fra bois de chauffage non coupé\end{définition}
\begin{définition}\cmn 没有劈开的木柴\end{définition}
\begin{relation-sémantique}\antonyme{
\hyperlink{ⒺsɯpaⒽ1}{\textit{ \papi{sɯpa}}}
}\end{relation-sémantique}\end{entrée}

\begin{entrée}
\vedette{\hypertarget{Ⓔɕɯrga}{\papi{ ɕɯrga}}}\markboth{ɕɯrga}{}
\classe{vt}
\paradigme{\textit{dir :} \jya nɯ-}
\begin{définition}\ 
\begin{déclaration}\grammar{caus}\end{déclaration}\end{définition}
\begin{définition}\fra rendre qqn content\end{définition}
\begin{définition}\cmn 让别人高兴\end{définition}
\begin{exemple}\jya ta-ɕɯrga, ɣɯ́-ɕɯrga-a\cmn 我让你高兴,他让我高兴\end{exemple}
\begin{exemple}\jya nɯ-tɯ́-wɣ-ɕɯrga\cmn 他让你高兴了\end{exemple}
\begin{exemple}\jya aʑo mɤ-pe-a qhe, kɤ-ɕɯrga mɤ-cha-a\cmn 我不好,不能让他高兴\end{exemple}
\begin{exemple}\jya kɯ-pe tú-wɣ-nɤma tɕe, tɯ-zda kɤ-ɕɯrga sɤcha\cmn 把工作做好就可以让自己的朋友高兴\end{exemple}
\begin{exemple}\jya nɯ-ɕɯrga-t-a\cmn 我让他高兴了\end{exemple}
\begin{relation-sémantique}\confer{
 \papi{rga}
}\end{relation-sémantique}\begin{sous-entrée}
\vedette{\hypertarget{}{\papi{ ʑɣɤɕɯrga}}}\markboth{ʑɣɤɕɯrga}{}\classe{vi}
\begin{définition}\ 
\begin{déclaration}\grammar{caus}\end{déclaration}
\begin{déclaration}\grammar{refl}\end{déclaration}\end{définition}
\begin{définition}\fra se faire plaisir\end{définition}
\begin{définition}\cmn 让自己开心\end{définition}
\begin{exemple}\jya laχtɕha ɯ-kɤ-sɯso nɯ to-χtɯ ndɤre ɲɯ-ʑɣɤɕɯrga\cmn 他买到了他想买的东西,很开心\end{exemple}
\end{sous-entrée}\end{entrée}

\begin{entrée}
\vedette{\hypertarget{Ⓔɕɯrŋgɯ}{\papi{ ɕɯrŋgɯ}}}\markboth{ɕɯrŋgɯ}{}
\classe{vt}
\begin{définition}\ 
\begin{déclaration}\grammar{caus}\end{déclaration}\end{définition}\acception{1}
\paradigme{\textit{dir :} \jya kɤ-}
\begin{définition}\fra faire dormir, fermenter le vin\end{définition}
\begin{définition}\cmn 使人睡;使人躺下;发酵(酒)\end{définition}
\begin{exemple}\jya kɤ-ɕɯrŋgɯ-t-a, kɤ-tɯ-ɕɯrŋgɯ-t, kaɕɯrŋgɯ\cmn 我让他睡了,你让他睡了,他让他睡了\end{exemple}
\begin{exemple}\jya tɤ-pɤtso nɯʑɯβ ɲɯ-ŋu tɕe kɤ-ɕɯrŋgɯ-t-a\cmn 小孩子快睡着了,我就让他躺下休息\end{exemple}\acception{2}
\begin{définition}\fra laisser le vin fermenter\end{définition}
\begin{définition}\cmn 让……发酵(酒)\end{définition}
\begin{exemple}\jya cha kɤ-sqa-t-a tɕe kɤ-ɕɯrŋgɯ-t-a\cmn 我让酒发酵了\end{exemple}
\begin{relation-sémantique}\confer{
\hyperlink{ⒺrŋgɯⒽ1}{\textit{ \papi{rŋgɯ1}}}
}\end{relation-sémantique}\end{entrée}

\begin{entrée}
\vedette{\hypertarget{Ⓔɕɯrŋo}{\papi{ ɕɯrŋo}}}\markboth{ɕɯrŋo}{}\classe{vt}
\paradigme{\textit{dir :} \jya nɯ-}
\begin{définition}\ 
\begin{déclaration}\grammar{caus}\end{déclaration}\end{définition}
\begin{définition}\fra prêter (un objet)\end{définition}
\begin{définition}\cmn 借给别人(假定以后能归还原物)\end{définition}
\begin{exemple}\jya ta-ɕɯrŋo, kɯ-ɕɯrŋo-a, ɣɯ́-ɕɯrŋgo-a\cmn 我借给你,你借给我,他借给我\end{exemple}
\begin{relation-sémantique}\confer{
\hyperlink{Ⓔrŋo}{\textit{ \papi{rŋo}}}
}\end{relation-sémantique}
\begin{relation-sémantique}\confer{
\hyperlink{Ⓔznɤŋgɯ}{\textit{ \papi{znɤŋgɯ}}}
}\end{relation-sémantique}\end{entrée}

\begin{entrée}
\vedette{\hypertarget{Ⓔɕɯrɴɢo}{\papi{ ɕɯrɴɢo}}}\markboth{ɕɯrɴɢo}{}
\classe{n}
\begin{définition}\fra Anisodus tanguticus\end{définition}
\begin{définition}\cmn 山茛菪\end{définition}
\begin{exemple}\jya ɕɯrɴɢo nɯ sɯjno kɯ-wxti kɯ-jpumqa ci ŋu. ɯ-mɯntoʁ kɯnɤ kɯ-jaʁjɯ kɯ-wxti tsa ci tshaŋlaŋ taʁ tɤ-kɯ-ɕthɯz kɯ-fse ci ŋu, ɯ-mɯntoʁ wɣrum, ɯ-taʁ kɯ-ɣɯrni kɯ-ɤkhra tu, ɯ-mɯntoʁ ɯ-rqhu nɯ wuma ʑo jaʁ cho rko. ɯ-mɯntoʁ pɯ-rom ɯ-qhu kɯnɤ ɯ-rqhu nɯ tu tɕe, ɯ-rɣi nɯ ku-mphɯr, ku-sɤsɯɣ ʑo ŋu, tɕe mɯ-pɯ́-wɣ-tɕɣaʁ mɤɕtʂa ɯ-rɣi mɤ-nɯɬoʁ. ɯ-mat thɯ-tɯt tɕe, ɯ-rɣi ndɯβ cho dɤn. ɯ-zrɤm wuma ʑo ngɯt, kɤ-phɯt ɴqa. tɤ-rɤku ɯ-ŋgɯ tu-ɬoʁ mɤ-pe ma ɯ-sta ɴqa tɕe tɤ-rɤku mɤ-sɤpe. ɯ-jwaʁ cho ɯ-ru ra aɣɯmdoʁ. nɯŋa kɤ-mbi sna, nɯŋa ɲɯ́-wɣ-mbi tɕe ɯ-lu dɤn tu-ti-nɯ ɲɯ-ŋu.\cmn 莨菪是粗壮的大草,花也很厚实,像朝上的铃铛。花是白色的,上面红色斑点,花萼又厚又硬。花凋谢了之后,花萼还在,包住种子,包得很紧,只有把它挤破种子才能出来。果实成熟后,种子细而多。根长得很结实,难以拔掉。长在庄稼地里不好,因为所占地面积宽,会影响庄稼。叶子和茎颜色一样。可以喂奶牛,据说奶牛吃了奶多。\end{exemple}\end{entrée}

\begin{entrée}
\vedette{\hypertarget{Ⓔɕɯrʁom}{\papi{ ɕɯrʁom}}}\markboth{ɕɯrʁom}{}\classe{n}
\begin{définition}\fra poils de yak épais\end{définition}
\begin{définition}\cmn 牦牛的粗毛\end{définition}
\begin{relation-sémantique}\confer{
\hyperlink{Ⓔrʁom}{\textit{ \papi{rʁom}}}
}\end{relation-sémantique}\end{entrée}

\begin{entrée}
\vedette{\hypertarget{Ⓔɕɯwa}{\papi{ ɕɯwa}}}\markboth{ɕɯwa}{}
\classe{n}
\begin{définition}\fra teigne\end{définition}
\begin{définition}\cmn 癣\end{définition}\end{entrée}

\begin{entrée}
\vedette{\hypertarget{Ⓔɕɯxtɯ}{\papi{ ɕɯxtɯ}}}\markboth{ɕɯxtɯ}{}
\classe{n}
\begin{définition}\fra bord de l'âtre\end{définition}
\begin{définition}\cmn 火塘边(铁或者石头制成的);长方形\end{définition}\end{entrée}

\begin{entrée}
\vedette{\hypertarget{Ⓔɕɯxtɯrɟɤxtsa}{\papi{ ɕɯxtɯrɟɤxtsa}}}\markboth{ɕɯxtɯrɟɤxtsa}{}\classe{n}
\begin{définition}\fra une plante\end{définition}
\begin{définition}\cmn 植物的一种\end{définition}
\begin{exemple}\jya ɕɯxtɯ rɟɤxtsa nɯ ruŋgu, tʂɤrkɯ, stɤmku aʁɤndɯndɤt tu-ɬoʁ ŋu, ɯ-zrɤm wxti, tɯ-boʁ nɤ tɯ-boʁ tu-ɬoʁ tɕe, ɯ-ru tɯ-ldʑa ɯ-taʁ ɯ-jwaʁ ɯ-rme ʑo kɯ-fse kɯ-dɤn ɲɯ-ɬoʁ ŋu, ɯ-ru ngɯt, ɯ-mɯntoʁ kɯ-ndɯ-ndɯβ kɯ-dɯ-dɤn ʑo ɲɯ-lɤt ŋu.\cmn 
\stylefv{ɕɯxtɯ rɟɤxtsa}在草山、草地、路边都可以生长,根大,丛生。茎上的叶子像毛一样密。茎很结实,茎的顶上开小而密的花。
\end{exemple}\end{entrée}

\newpage\caractère{d}

\begin{entrée}
\vedette{\hypertarget{Ⓔdal}{\papi{ dal}}}\markboth{dal}{}\classe{adv}
\begin{définition}\fra plus tard\end{définition}
\begin{définition}\cmn 晚一点\end{définition}
\begin{exemple}\jya nɤʑo jiɕqha dal tsa ri kɤ-tɯ-nɯtʂha ɕti\cmn 你晚一点才吃了早餐\end{exemple}
\begin{exemple}\jya japa dal ri\cmn 几年前\end{exemple}
\begin{exemple}\jya jɯfɕɯndʐi dal ri\cmn 几天前\end{exemple}\end{entrée}

\begin{entrée}
\vedette{\hypertarget{Ⓔdaltsa}{\papi{ daltsa}}}\markboth{daltsa}{}\classe{n}
\begin{définition}\fra lentement\end{définition}
\begin{définition}\cmn 慢
\begin{déclaration} \étymologie{\papi{dal}}\end{déclaration}\end{définition}
\begin{exemple}\jya kɤ-rɯɕmi daltsa tɤ-pe ɲɯ-ra\cmn 请你说话稍微慢一点\end{exemple}
\end{entrée}

\begin{entrée}
\vedette{\hypertarget{Ⓔdaltsɯtsa}{\papi{ daltsɯtsa}}}\markboth{daltsɯtsa}{}\classe{n}
\begin{définition}\fra lentement\end{définition}
\begin{définition}\cmn 慢慢\end{définition}
\begin{relation-sémantique}\confer{
\hyperlink{Ⓔdaltsa}{\textit{ \papi{daltsa}}}
}\end{relation-sémantique}
\end{entrée}

\begin{entrée}
\vedette{\hypertarget{Ⓔdɤlie}{\papi{ dɤlie}}}\markboth{dɤlie}{}\classe{vi}
\begin{définition}\fra bienvenue\end{définition}
\begin{définition}\cmn 欢迎光临,快回家(我们在等着你)\end{définition}
\begin{exemple}\jya dɤlie-ndʑi\cmn 你们俩快回家\end{exemple}\end{entrée}

\begin{entrée}
\vedette{\hypertarget{Ⓔdɤn}{\papi{ dɤn}}}\markboth{dɤn}{}\classe{vs}
\paradigme{\textit{dir :} \jya tɤ-}
\paradigme{\textit{dir :} \jya nɯ-}
\begin{définition}\fra nombreux\end{définition}
\begin{définition}\cmn 多
\begin{déclaration} \étymologie{\papi{ldan}}\end{déclaration}\end{définition}
\begin{exemple}\jya tɯrme ɲɯ-dɤn\cmn 人很多\end{exemple}
\begin{exemple}\jya pɯ-dɤn-i\cmn (当时)我们人很多\end{exemple}
\begin{exemple}\jya kɯ-dɤn me-j\cmn 我们人不多\end{exemple}
\begin{relation-sémantique}\synonyme{
\hyperlink{Ⓔxcat}{\textit{ \papi{xcat}}}
}\end{relation-sémantique}\begin{sous-entrée}
\vedette{\hypertarget{}{\papi{ nɤdɤn}}}\markboth{nɤdɤn}{}\classe{vt}
\begin{définition}\ 
\begin{déclaration}\grammar{trop}\end{déclaration}\end{définition}
\begin{définition}\fra trouver nombreux\end{définition}
\begin{définition}\cmn 觉得很多\end{définition}
\end{sous-entrée}\end{entrée}

\begin{entrée}
\vedette{\hypertarget{Ⓔdɤrʁɯ}{\papi{ dɤrʁɯ}}}\markboth{dɤrʁɯ}{}
\classe{n}
\begin{définition}\fra fougère\end{définition}
\begin{définition}\cmn 蕨苔\end{définition}
\begin{exemple}\jya dɤrʁɯ nɯ sɤtɕha thamtɕɤt ʑo tu-maʁ, sɤtɕha ɴqiaβ tsa sɤjku kɯ-tu ɣɯ pɕoʁ nɯ ra tu-ɬoʁ ŋu. phaʁzla jarma tu-ɬoʁ tɕe tʂɯɣpa ɲɯ-ɤrɕo ɕɯŋgɯ tɕe tu-qalpɕa tɕe tu-rgɤz ɕti. tu-qalpɕa ɕɯŋgɯ nɯ ɯ-ru ɯ-jwaʁ nɯ tɯrme kɯ tɤŋkhɯt tɤ-kɤ-βzu fse. ɯ-ru nɯ mpɯ tɕe kɤ-ndza mɯm.\cmn 蕨苔不是所有的地方都有,只有在山阴能有白桦树的地方才生长。一般五月开始生长到六月底叶子就展开,蕨苔也就老了。叶子展开前,茎上的叶子长得像人握着的拳。茎很柔嫩,好吃。\end{exemple}\end{entrée}

\begin{entrée}
\vedette{\hypertarget{Ⓔdɣɤrdɣɤr}{\papi{ dɣɤrdɣɤr}}}\markboth{dɣɤrdɣɤr}{}
\classe{idph.2}
\begin{définition}\fra bête\end{définition}
\begin{définition}\cmn 形容不聪明,发呆的模样\end{définition}
\begin{exemple}\jya jiɕqha tɯrme nɯ dɣɤrdɣɤr ʑo ɲɯ-rɤʑi ɲɯ-khe ʑo\cmn 那个人在那里发呆\end{exemple}
\begin{relation-sémantique}\confer{
\hyperlink{Ⓔɣɤrɣɤr}{\textit{ \papi{ɣɤrɣɤr}}}
}\end{relation-sémantique}\end{entrée}

\begin{entrée}
\vedette{\hypertarget{Ⓔdioʁdioʁ}{\papi{ dioʁdioʁ}}}\markboth{dioʁdioʁ}{}
\classe{idph.2}
\begin{définition}\fra bien mélangé\end{définition}
\begin{définition}\cmn 形容非常均匀
\end{définition}
\begin{exemple}\jya tɤjlu pɯ́-wɣ-rɤpɣi tɕe, tɯ-ci cho tɤjlu ni tú-wɣ-sɤtʂoʁloʁ dioʁdioʁ ʑo ra ma nɯ mɤɕtʂa mɤ-mɯm\cmn 和面的时候,要把水和面和得很均匀\end{exemple}\end{entrée}

\begin{entrée}
\vedette{\hypertarget{Ⓔdo}{\papi{ do}}}\markboth{do}{}
\classe{vs}
\paradigme{\textit{dir :} \jya thɯ-}
\begin{définition}\fra fibreuse (plante)\end{définition}
\begin{définition}\cmn 老(植物)\end{définition}
\begin{exemple}\jya pɤjka cho-do\cmn 白瓜老了(可以吃了)\end{exemple}\end{entrée}

\begin{entrée}
\vedette{\hypertarget{Ⓔdoŋdoŋ}{\papi{ doŋdoŋ}}}\markboth{doŋdoŋ}{}\classe{idph.2}
\begin{définition}\fra long et épais, cylindrique\end{définition}
\begin{définition}\cmn 形容圆柱形物体粗而长的样子\end{définition}
\begin{exemple}\jya ɕoŋtɕa ɲɯ-jpum doŋdoŋ\cmn 木料非常粗\end{exemple}\end{entrée}

\begin{entrée}
\vedette{\hypertarget{Ⓔdʐoŋdʐoŋ}{\papi{ dʐoŋdʐoŋ}}}\markboth{dʐoŋdʐoŋ}{}\classe{idph.2}
\begin{définition}\fra mou, long et épais\end{définition}
\begin{définition}\cmn 形容软、粗而长的样子\end{définition}
\begin{exemple}\jya tɯ-pu dʐoŋdʐoŋ ʑo ɲɯ-pa\cmn 肠子又软又粗又长\end{exemple}\end{entrée}

\begin{entrée}
\vedette{\hypertarget{Ⓔdrɤβdrɤβ}{\papi{ drɤβdrɤβ}}}\markboth{drɤβdrɤβ}{}\classe{idph.2}
\begin{définition}\fra plein de saleté (eau)\end{définition}
\begin{définition}\cmn 形容浑浊、装满渣滓、没有搅匀的样子\end{définition}
\begin{exemple}\jya tɯtshi drɤβdrɤβ ʑo ɲɯ-pa\cmn 粥很粘稠(没有搅匀)\end{exemple}
\begin{exemple}\jya tɯ-mɯ chɤ-qandʐi drɤβdrɤβ ʑo\cmn 天空布满了乌云\end{exemple}
\begin{exemple}\jya tɯ-ci chɤ-qarndɯm drɤβdrɤβ ʑo\cmn 水变浑浊了\end{exemple}\end{entrée}

\begin{entrée}
\vedette{\hypertarget{Ⓔdroŋdroŋ}{\papi{ droŋdroŋ}}}\markboth{droŋdroŋ}{}\classe{idph.2}
\begin{définition}\fra gros et sale\end{définition}
\begin{définition}\cmn 形容粗、大而脏的样子\end{définition}
\begin{exemple}\jya ɯ-ɕŋaβ droŋdroŋ ʑo chɤ-nɤndzɣi\cmn 他的鼻涕一大根一大根地吊在那里,显得很脏\end{exemple}
\begin{exemple}\jya ɯ-phoŋbu ɲɯ-sɤjloʁ droŋdroŋ\cmn 她身材又胖又高,不好看\end{exemple}\end{entrée}

\begin{entrée}
\vedette{\hypertarget{Ⓔdrɯβ}{\papi{ drɯβ}}}\markboth{drɯβ}{}
\classe{idph.1}
\begin{définition}\fra percer et laisser couler un liquide\end{définition}
\begin{définition}\cmn 刺破貌(脓包)\end{définition}
\begin{exemple}\jya taqaβ drɯβ ʑo tɤ-lat-a tɕe, tɤ-spɯ tɤ-tɕat-a\cmn 我用针刺了一下,把脓排出来了\end{exemple}\begin{sous-entrée}
\vedette{\hypertarget{}{\papi{ drɯβnɤdrɯβ}}}\markboth{drɯβnɤdrɯβ}{}\classe{idph.3}
\begin{exemple}\jya tɤ-spɯ drɯβnɤdrɯβ ta-tɕɣaʁ\cmn 他把脓挤了出来\end{exemple}
\begin{relation-sémantique}\confer{
\hyperlink{Ⓔnɯdrɯβ}{\textit{ \papi{nɯdrɯβ}}}
}\end{relation-sémantique}
\end{sous-entrée}\end{entrée}

\begin{entrée}
\vedette{\hypertarget{Ⓔdʐɯβdʐɯβ}{\papi{ dʐɯβdʐɯβ}}}\markboth{dʐɯβdʐɯβ}{}
\classe{idph.2}
\begin{définition}\fra tendre\end{définition}
\begin{définition}\cmn 嫩\end{définition}
\begin{exemple}\jya ki @bocai ki dʐɯβdʐɯβ ʑo ɲɯ-pa\cmn 这个菠菜长得又粗又嫩\end{exemple}\end{entrée}

\begin{entrée}
\vedette{\hypertarget{Ⓔdʐɯβnɤdʐɯβ}{\papi{ dʐɯβnɤdʐɯβ}}}\markboth{dʐɯβnɤdʐɯβ}{}\classe{idph.3}
\begin{définition}\fra mou sous la dent\end{définition}
\begin{définition}\cmn 形容食物吃起来软的样子\end{définition}\end{entrée}

\begin{entrée}
\vedette{\hypertarget{Ⓔdɯdɯt}{\papi{ dɯdɯt}}}\markboth{dɯdɯt}{}\classe{n}
\begin{définition}\fra tourterelle\end{définition}
\begin{définition}\cmn 斑鸠\end{définition}
\begin{exemple}\jya dɯdɯt nɯ pɣa khro mɤ-kɯ-wxti ci ŋu, qro jamar ma me, ɯ-tshɯɣa ra qro fse, ɯ-mdoʁ nɯ kɯ-ɤrŋi ɯ-ŋgɯz kɯnɤ kɯ-pɣi tsa ŋu, tu-mbri tɕe, `du dɯt cɯ ɯ-ŋgɯ lɤɣ' tu-ti ɲɯ-ŋu, tɤ-rɤku cho qajɯ ndze, tɤ-rɤku kɤ-ndza χɕu, kha tɤ-kɤ-nɯ-sɤro ʑo nɯ ɣɯ-tu-ndze ŋu.\cmn 
斑鸠是一种不太大的鸟,只有鸽子那么大,样子也像鸽子,颜色蓝里带灰,叫声是\stylefv{du dɯt cɯ ɯ-ŋgɯ lɤɣ}。吃粮食和虫子,吃粮食尤其厉害,专门来吃摆放在家里的粮食。
\end{exemple}
\end{entrée}

\begin{entrée}
\vedette{\hypertarget{Ⓔdɯɣ}{\papi{ dɯɣ}}}\markboth{dɯɣ}{}
\classe{vi}
\paradigme{\textit{dir :} \jya tɤ-}
\begin{définition}\fra en avoir assez\end{définition}
\begin{définition}\cmn 厌倦;厌烦;觉得麻烦\end{définition}
\begin{exemple}\jya ɲɯ-tɯ-dɯɣ\cmn 你不想坚持\end{exemple}
\begin{exemple}\jya ɯʑo to-dɯɣ\cmn 他厌倦了\end{exemple}
\begin{exemple}\jya tɯtun pjɯ-ŋgrɯ tɤ-ra tɕe kɯ-ɴqa kɤ-nɤma kɤ-dɯɣ mɤ-βze\cmn 要达到自己的目标的话,不能怕辛苦\end{exemple}
\begin{exemple}\jya kɤ-nɤma tɤ-ɴqa tɕe, tu-dɯɣ ɲɯ-ɕti\cmn 工作进行得很辛苦,他灰心了\end{exemple}
\begin{exemple}\jya kɤ-ɤmdzɯ ɲɯ-dɯɣ-a\cmn 我坐烦了\end{exemple}\begin{sous-entrée}
\vedette{\hypertarget{}{\papi{ sɤɣdɯɣ}}}\markboth{sɤɣdɯɣ}{}\classe{vs}
\begin{définition}\ 
\begin{déclaration}\grammar{deexp}\end{déclaration}\end{définition}
\begin{définition}\fra désagréable, détestable\end{définition}
\begin{définition}\cmn 辛苦;令人心烦\end{définition}
\begin{relation-sémantique}\confer{
\hyperlink{Ⓔnɤsɤɣdɯɣ}{\textit{ \papi{nɤsɤɣdɯɣ}}}
}\end{relation-sémantique}
\end{sous-entrée}\begin{sous-entrée}
\vedette{\hypertarget{}{\papi{ sɯɣdɯɣ}}}\markboth{sɯɣdɯɣ}{}\classe{vt}
\paradigme{\textit{dir :} \jya tɤ-}
\begin{définition}\ 
\begin{déclaration}\grammar{caus}\end{déclaration}\end{définition}
\begin{définition}\fra énerver, fatiguer\end{définition}
\begin{définition}\cmn 令人不耐烦;厌倦\end{définition}
\begin{exemple}\jya nɤ-kɤ-ti ɯ-tɯ-dɤn kɯ ɲɯ-kɯ-sɯɣdɯɣ-a\end{exemple}
\begin{exemple}\jya nɤ-kɤ-rɯrawa ɯ-tɯ-dɤn kɯ ɲɯ-kɯ-sɯɣdɯɣ-a\cmn 你要求得太多,让我不耐烦了\end{exemple}
\end{sous-entrée}\end{entrée}

\begin{entrée}
\vedette{\hypertarget{Ⓔdʐɯɣdʐɯɣ}{\papi{ dʐɯɣdʐɯɣ}}}\markboth{dʐɯɣdʐɯɣ}{}
\classe{idph.2}
\begin{définition}\fra très fort (thé)\end{définition}
\begin{définition}\cmn 酽\end{définition}
\begin{exemple}\jya tʂha tɤ-lu dʐɯɣdʐɯɣ ɲɯ-pa\cmn 茶很酽\end{exemple}
\begin{relation-sémantique}\confer{
\hyperlink{Ⓔldɯɣldɯɣ}{\textit{ \papi{ldɯɣldɯɣ}}}
}\end{relation-sémantique}\end{entrée}

\begin{entrée}
\vedette{\hypertarget{Ⓔdɯrnɤrdɯr}{\papi{ dɯrnɤrdɯr}}}\markboth{dɯrnɤrdɯr}{}\classe{idph.2}
\begin{définition}\fra battement de tambour lointain\end{définition}
\begin{définition}\cmn 形容轻微的敲鼓声\end{définition}
\begin{exemple}\jya tɤrmbɣo dɯnɤrdɯr ʑo ɲɯ-mbri\cmn 听得到轻微的敲鼓声\end{exemple}
\begin{relation-sémantique}\confer{
\hyperlink{Ⓔɣdoŋnɤɣdoŋ}{\textit{ \papi{ɣdoŋnɤɣdoŋ}}}
}\end{relation-sémantique}
\begin{relation-sémantique}\confer{
\hyperlink{Ⓔɣdɯɣnɤɣdɯɣ}{\textit{ \papi{ɣdɯɣnɤɣdɯɣ}}}
}\end{relation-sémantique}\end{entrée}

\begin{entrée}
\vedette{\hypertarget{Ⓔdɯxpa}{\papi{ dɯxpa}}}\markboth{dɯxpa}{}\classe{vs}
\begin{définition}\fra pitoyable, pauvre\end{définition}
\begin{définition}\cmn 可怜\end{définition}
\begin{exemple}\jya dɯxpa-j ɣɤʑu\cmn 我们很可怜\end{exemple}
\begin{exemple}\jya dɯxpa ɣɤʑu-j\cmn 我们很可怜\end{exemple}\end{entrée}

\begin{entrée}
\vedette{\hypertarget{Ⓔdwaŋdwaŋ}{\papi{ dwaŋdwaŋ}}}\markboth{dwaŋdwaŋ}{}
\classe{idph.2}
\begin{définition}\fra ne pas être en possession de ses moyens\end{définition}
\begin{définition}\cmn 神志不清\end{définition}
\begin{exemple}\jya ɲɯ-ngo tɕe, dwaŋdwaŋ ʑo ɲɯ-rɤʑi\cmn 他生病了,神志不清\end{exemple}
\begin{exemple}\jya cha kú-wɣ-tshi tɕe tɯ-ku dwaŋdwaŋ ʑo pa\cmn 喝酒后,头脑就会不清醒\end{exemple}
\begin{relation-sémantique}\synonyme{
\hyperlink{Ⓔjaŋjaŋ}{\textit{ \papi{jaŋjaŋ}}}
}\end{relation-sémantique}\end{entrée}

\begin{entrée}
\vedette{\hypertarget{Ⓔdzambaɬa}{\papi{ dzambaɬa}}}\markboth{dzambaɬa}{}\classe{n}
\begin{définition}\fra type de mammifère\end{définition}
\begin{définition}\cmn 动物的一种(像黄鼠狼)\end{définition}\end{entrée}

\begin{entrée}
\vedette{\hypertarget{Ⓔdzaŋdzaŋ}{\papi{ dzaŋdzaŋ}}}\markboth{dzaŋdzaŋ}{}\classe{idph.2}\acception{1}
\begin{définition}\fra dru\end{définition}
\begin{définition}\cmn 形容又茂盛又粗糙的样子(植物的刺、头发等)\end{définition}
\begin{exemple}\jya ɯ-ku dzaŋdzaŋ ʑo to-stu\cmn 他头发蓬乱\end{exemple}
\begin{exemple}\jya si tɯ-phɯ dzaŋdzaŋ ɣɤʑu\cmn 有一棵枝桠茂盛的大树\end{exemple}
\begin{exemple}\jya ɯ-ku to-rpɯ dzaŋdzaŋ ʑo\cmn 他的头发又长又脏,乱蓬蓬的\end{exemple}\acception{2}
\begin{définition}\fra ne pas être en possession de ses facultés\end{définition}
\begin{définition}\cmn 神志不清(喝醉了,病了)\end{définition}
\begin{exemple}\jya ɲɯ-ŋgo-a tɕe, dzaŋdzaŋ ɲɯ-pa-a\cmn 我喝醉了,神志不清\end{exemple}
\begin{exemple}\jya lo-βzi-a tɕe, dzaŋdzaŋ ʑo ɲɯ-pa-a\cmn 我喝醉了,神志不清\end{exemple}\begin{sous-entrée}
\vedette{\hypertarget{}{\papi{ ɣɤdzaŋdzaŋ}}}\markboth{ɣɤdzaŋdzaŋ}{}\classe{vi}
\begin{définition}\fra avoir les poils longs, en désordre\end{définition}
\begin{définition}\cmn 毛又长又乱\end{définition}
\begin{exemple}\jya mbroχpa khɯna ɲɯ-ɣɤdzaŋdzaŋ thɯ-ɣe\cmn 藏獒竖起毛扑过来了\end{exemple}
\end{sous-entrée}\begin{sous-entrée}
\vedette{\hypertarget{}{\papi{ ɣɤdzaŋlaŋ}}}\markboth{ɣɤdzaŋlaŋ}{}\classe{vi}
\begin{exemple}\jya qambrɯ ɲɯ-ɣɤdzaŋlaŋ ntsɯ ɲɯ-rɤβʁa\cmn 牦牛一身毛乱蓬蓬地大声叫\end{exemple}
\begin{relation-sémantique}\confer{
\hyperlink{Ⓔzoŋzoŋ}{\textit{ \papi{zoŋzoŋ}}}
}\end{relation-sémantique}
\begin{relation-sémantique}\confer{
\hyperlink{Ⓔzaŋzaŋ}{\textit{ \papi{zaŋzaŋ}}}
}\end{relation-sémantique}
\end{sous-entrée}\end{entrée}

\begin{entrée}
\vedette{\hypertarget{Ⓔdzɤjdzɤj}{\papi{ dzɤjdzɤj}}}\markboth{dzɤjdzɤj}{}
\begin{relation-sémantique}\confer{
\hyperlink{Ⓔzɤjzɤj}{\textit{ \papi{zɤjzɤj}}}
}\end{relation-sémantique}\end{entrée}

\begin{entrée}
\vedette{\hypertarget{Ⓔdzoŋdzoŋ}{\papi{ dzoŋdzoŋ}}}\markboth{dzoŋdzoŋ}{}
\classe{idph.2}
\begin{définition}\fra ébouriffé\end{définition}
\begin{définition}\cmn 形容毛发蓬松、竖起来的样子\end{définition}
\begin{exemple}\jya ɯ-rme dzoŋdzoŋ ʑo cho-ɬoʁ\cmn 他的毛发竖起来了\end{exemple}
\begin{exemple}\jya nɤ-kɤrme pɯ-sɤɕɤt ma dzoŋdzoŋ ʑo ɲɯ-pa\cmn 你要梳头,不然你的头发毛松松的\end{exemple}
\begin{exemple}\jya tɯrgi paʁtsa kɯ ɯ-jme dzoŋdzoŋ ʑo tu-tse tɕe tu-nɤŋkɯŋke ŋu\cmn 松鼠竖起尾巴走动,看起来蓬蓬松松的\end{exemple}\begin{sous-entrée}
\vedette{\hypertarget{}{\papi{ dzoŋnɤdzoŋ}}}\markboth{dzoŋnɤdzoŋ}{}\classe{idph.3}
\begin{définition}\fra sensation désagréable ressentie lorsque l'on marche avec une jambe engourdie\end{définition}
\begin{définition}\cmn 形容脚麻木,走得很难受的感觉\end{définition}
\begin{exemple}\jya tu-ŋke-a tɕe a-mi dzoŋnɤdzoŋ ʑo ɲɯ-ti ma chɤ-ndʑɯrpɯt\cmn 我走路走得很难受,因为脚麻木了\end{exemple}
\end{sous-entrée}\end{entrée}

\begin{entrée}
\vedette{\hypertarget{Ⓔdzoʁ}{\papi{ dzoʁ}}}\markboth{dzoʁ}{}\classe{idph.1}
\begin{définition}\fra tout d'un coup (s'agenouiller)\end{définition}
\begin{définition}\cmn 一下子(跪下)\end{définition}
\begin{exemple}\jya ɯ-χpɯm dzoʁ ʑo pjɤ-tshoʁ\cmn 他一下子跪下了(很恭敬的样子)\end{exemple}
\begin{relation-sémantique}\synonyme{
\hyperlink{Ⓔgoʁ}{\textit{ \papi{goʁ}}}
}\end{relation-sémantique}\end{entrée}

\begin{entrée}
\vedette{\hypertarget{Ⓔdzɯɣdzɯɣ}{\papi{ dzɯɣdzɯɣ}}}\markboth{dzɯɣdzɯɣ}{}\classe{idph.2}
\begin{définition}\fra fournis (poils)\end{définition}
\begin{définition}\cmn 形容毛多而密,茂盛的样子\end{définition}
\begin{exemple}\jya ɯ-mtɕhɯrme dzɯɣdzɯɣ ʑo ɲɯ-pa\cmn 他的胡须很茂盛\end{exemple}
\begin{exemple}\jya tshɤrtɯl ɯ-rme dzɯɣdzɯɣ ʑo ɲɯ-pa\cmn 羔羊皮袄的毛很茂盛\end{exemple}\end{entrée}

\begin{entrée}
\vedette{\hypertarget{Ⓔdzɯrdzɯr}{\papi{ dzɯrdzɯr}}}\markboth{dzɯrdzɯr}{}
\classe{idph.2}
\begin{définition}\fra bien droit\end{définition}
\begin{définition}\cmn 形容端正的样子(又小又乖)\end{définition}
\begin{exemple}\jya dzɯrdzɯr ʑo ɲɯ-rɤʑi\cmn 他端端正正地(站)在那里\end{exemple}
\begin{exemple}\jya dzɯrdzɯr ʑo ɲɯ-ɤmdzɯt\cmn 他坐得很端正\end{exemple}\begin{sous-entrée}
\vedette{\hypertarget{}{\papi{ dzɯr}}}\markboth{dzɯr}{}
\begin{définition}\fra prompt, agile\end{définition}
\begin{définition}\cmn 利索(很受规矩的样子)\end{définition}
\begin{exemple}\jya ɯ-χpɯm dzɯr ʑo ta-nɯ-tshoʁ\cmn 他很利索地跪了\end{exemple}
\end{sous-entrée}\begin{sous-entrée}
\vedette{\hypertarget{}{\papi{ dzɯrnɤdzɯr}}}\markboth{dzɯrnɤdzɯr}{}\classe{idph.3}
\begin{exemple}\jya dzɯrnɤdzɯr ɲɯ-ŋke\cmn 他在端端正正地走\end{exemple}
\end{sous-entrée}\end{entrée}

\begin{entrée}
\vedette{\hypertarget{Ⓔdʑaŋdʑaŋ}{\papi{ dʑaŋdʑaŋ}}}\markboth{dʑaŋdʑaŋ}{}\classe{idph.2}
\begin{définition}\fra long et fin\end{définition}
\begin{définition}\cmn 形容细而长的样子\end{définition}\end{entrée}

\begin{entrée}
\vedette{\hypertarget{Ⓔdʑɤrdʑɤr}{\papi{ dʑɤrdʑɤr}}}\markboth{dʑɤrdʑɤr}{}\classe{idph.2}
\begin{définition}\fra debout tout droit\end{définition}
\begin{définition}\cmn 站得很直;身子又细又高;看起来很单薄又孤零零的样子\end{définition}
\begin{exemple}\jya si dʑɤrdʑɤr ta-sɯɣndzur-a\cmn 我把树立起来了\end{exemple}\begin{sous-entrée}
\vedette{\hypertarget{}{\papi{ dʑɤrnɤdʑɤr}}}\markboth{dʑɤrnɤdʑɤr}{}\classe{idph.3}
\begin{exemple}\jya dʑɤrnɤdʑɤr kɤ-ari\cmn 他瘦瘦高高的,走了。\end{exemple}
\begin{exemple}\jya tɤ-pɤtso dʑɤrdʑɤr nɤ dʑɤrdʑɤr ɲɯ-ɤnɯɣro\cmn 小孩子一个人(在草坪上)玩\end{exemple}
\begin{exemple}\jya nɤki nɯ ɯ-tɯ-nɯɲɤmkhe kɯ tu-ŋke tɕe dʑɤrnɤdʑɤr ʑo pa ɕti wo\cmn 那个人很瘦,走路的时候显得又细又高\end{exemple}
\end{sous-entrée}\end{entrée}

\begin{entrée}
\vedette{\hypertarget{Ⓔdʑɯβdʑɯβ}{\papi{ dʑɯβdʑɯβ}}}\markboth{dʑɯβdʑɯβ}{}
\classe{idph.2}
\begin{définition}\fra rugueux et pointu\end{définition}
\begin{définition}\cmn 形容粗糙、尖而密的样子\end{définition}
\begin{exemple}\jya tɯrgi ɣɯ ɯ-jwaʁ dʑɯβdʑɯβ ʑo ɲɯ-pa tɕe, ɲɯ-sɤmtsɯɣ\cmn 杉树的针叶很尖,会刺到人\end{exemple}\end{entrée}

\begin{entrée}
\vedette{\hypertarget{Ⓔdʑɯpdʑɯp}{\papi{ dʑɯpdʑɯp}}}\markboth{dʑɯpdʑɯp}{}\classe{idph.2}
\begin{définition}\fra très épineux\end{définition}
\begin{définition}\cmn 形容植物的刺密集的样子\end{définition}
\begin{exemple}\jya tɤ-mdzu dʑɯpdʑɯp ʑo ɲɯ-pa\cmn 刺很密集,长得很茂盛\end{exemple}\end{entrée}

\newpage\caractère{f}

\begin{entrée}
\vedette{\hypertarget{Ⓔfɕafɕɤt}{\papi{ fɕafɕɤt}}}\markboth{fɕafɕɤt}{}\classe{n}
\begin{définition}\fra discours, parole\end{définition}
\begin{définition}\cmn 演讲;言论\end{définition}
\begin{exemple}\jya ɯ-grɤl kɯ-me fɕafɕɤt ntsɯ ma-tɯ-βze!\cmn 你不要总是胡说八道\end{exemple}
\begin{relation-sémantique}\confer{
\hyperlink{ⒺfɕɤtⒽ1}{\textit{ \papi{fɕɤt1}}}
}\end{relation-sémantique}\end{entrée}

\begin{entrée}
\vedette{\hypertarget{ⒺfɕaʁⒽ1}{\papi{ fɕaʁ}}}\markboth{fɕaʁ}{}\homonyme{1}
\classe{vt}
\paradigme{\textit{dir :} \jya nɯ-}
\begin{définition}\fra se repentir, rembourser\end{définition}
\begin{définition}\cmn 忏悔;赔偿
\begin{déclaration} \étymologie{\papi{bɕags}}\end{déclaration}\end{définition}
\begin{exemple}\jya ɯʑo kɯ kɯmaʁ ta-nɤma ra a-ɕki na-fɕaʁ ɕti\cmn 他向我赔偿了他的过失\end{exemple}
\begin{exemple}\jya pɯ-kɯ-ɣɤtɕa tɕe, mɤ-kɤ-fɕaʁ mɤ-khɯ\cmn 有了过错就不能不赔偿\end{exemple}
\begin{exemple}\jya ɯ-sɤ-fɕaʁ me\cmn 我没有东西来赔偿\end{exemple}\end{entrée}

\begin{entrée}
\vedette{\hypertarget{ⒺfɕaʁⒽ2}{\papi{ fɕaʁ}}}\markboth{fɕaʁ}{}\homonyme{2}
\classe{vs}
\begin{définition}\fra être suffisant, satisfaisant\end{définition}
\begin{définition}\cmn 足够\end{définition}
\begin{exemple}\jya kɯki jamar ɲɯ-fɕaʁ\cmn 这么多大概够了吧\end{exemple}\begin{sous-entrée}
\vedette{\hypertarget{}{\papi{ mɯ́j-fɕaʁ}}}\markboth{mɯ́j-fɕaʁ}{}
\begin{définition}\fra il faut non seulement ..\end{définition}
\begin{définition}\cmn 不但要……\end{définition}
\begin{exemple}\jya tɯ-rɣi pjɯ-tu mɯ́j-fɕaʁ ma tɯ-ɣli kɯnɤ pjɯ-tu ra ma nɯ maʁ nɤ tɤ-rɤku tu-pe mɯ́j-cha\cmn 不但要有种子,也要有肥料,不然庄稼不会长好\end{exemple}
\end{sous-entrée}\end{entrée}

\begin{entrée}
\vedette{\hypertarget{Ⓔfɕɤl}{\papi{ fɕɤl}}}\markboth{fɕɤl}{}\classe{vi}
\paradigme{\textit{dir :} \jya nɯ-}
\begin{définition}\fra avoir la diarrhée\end{définition}
\begin{définition}\cmn 拉肚子
\begin{déclaration} \étymologie{\papi{bɕal}}\end{déclaration}\end{définition}
\begin{exemple}\jya a-xtu ɲɯ-fɕɤl (=ɲɯ-nɯtɯfɕal-a)\cmn 我拉肚子\end{exemple}
\begin{exemple}\jya a-xtu nɯ-fɕɤl, a-xtu pɯ-fɕɤl\cmn 我拉了肚子\end{exemple}\begin{sous-entrée}
\vedette{\hypertarget{}{\papi{ sɯfɕɤl}}}\markboth{sɯfɕɤl}{}\classe{vt}
\paradigme{\textit{dir :} \jya nɯ-}
\begin{définition}\fra causer une diarrhée\end{définition}
\begin{définition}\cmn 令人拉肚子\end{définition}
\begin{exemple}\jya qaɕti tɤ-ndza-t-a tɕe a-xtu na-sɯfɕɤl\cmn 我吃了桃子,令我拉了肚子\end{exemple}
\begin{relation-sémantique}\confer{
\hyperlink{Ⓔnɯtɯfɕɤl}{\textit{ \papi{nɯtɯfɕɤl}}}
}\end{relation-sémantique}
\end{sous-entrée}\end{entrée}

\begin{entrée}
\vedette{\hypertarget{Ⓔfɕɤm}{\papi{ fɕɤm}}}\markboth{fɕɤm}{}\classe{vt}
\paradigme{\textit{dir :} \jya tɤ-}
\paradigme{\textit{dir :} \jya thɯ-}
\begin{définition}\fra étendre\end{définition}
\begin{définition}\cmn 摆出来
\begin{déclaration} \étymologie{\papi{bɕams}}\end{déclaration}\end{définition}
\begin{exemple}\jya ɯʑo kɯ laχtɕha kɤ-ntsɣe ɯ-spa nɯ chɤ-fɕɤm\cmn 我把卖的东西全部摆出来了\end{exemple}
\begin{relation-sémantique}\synonyme{
\hyperlink{Ⓔɕkho}{\textit{ \papi{ɕkho}}}
}\end{relation-sémantique}\end{entrée}

\begin{entrée}
\vedette{\hypertarget{ⒺfɕɤtⒽ1}{\papi{ fɕɤt}}}\markboth{fɕɤt}{}\homonyme{1}\classe{vt}\acception{1}
\paradigme{\textit{dir :} \jya pɯ-}
\begin{définition}\fra raconter\end{définition}
\begin{définition}\cmn 讲故事;讲述
\begin{déclaration} \étymologie{\papi{bɕad}}\end{déclaration}\end{définition}
\begin{exemple}\jya nɤʑo pɯ-tɯ-fɕɤt, ɯʑo kɯ pa-fɕɤt\cmn 你讲述了,他讲述了\end{exemple}
\begin{exemple}\jya a-χpi pa-fɕɤt\cmn 他给我讲了故事\end{exemple}\acception{1}
\paradigme{\textit{dir :} \jya pɯ-}
\begin{définition}\fra discuter\end{définition}
\begin{définition}\cmn 谈;聊天\end{définition}
\begin{exemple}\jya ku-fɕɤt-tɕi\cmn 我们俩在谈\end{exemple}\acception{2}
\paradigme{\textit{dir :} \jya pɯ-}
\paradigme{\textit{dir :} \jya kɤ-}
\begin{définition}\fra danser pour\end{définition}
\begin{définition}\cmn 表演(舞蹈)\end{définition}
\begin{exemple}\jya a-tɯrɟaʁ ci kɤ-fɕɤt\cmn 她为我表演了舞蹈\end{exemple}\begin{sous-entrée}
\vedette{\hypertarget{}{\papi{ afɕɤt}}}\markboth{afɕɤt}{}
\begin{exemple}\jya kɯki mɤ-afɕɤt\cmn 这个(故事)还没有讲\end{exemple}
\end{sous-entrée}\begin{sous-entrée}
\vedette{\hypertarget{}{\papi{ koŋla tú-wɣ-fɕɤt a-pɯ-ŋu tɕe}}}\markboth{koŋla tú-wɣ-fɕɤt a-pɯ-ŋu tɕe}{}
\begin{définition}\fra en fait\end{définition}
\begin{définition}\cmn 实际上;说白了;\end{définition}
\end{sous-entrée}\begin{sous-entrée}
\vedette{\hypertarget{}{\papi{ nɤfɕɯfɕɤt}}}\markboth{nɤfɕɯfɕɤt}{}\classe{vt}
\begin{définition}\fra raconter partout\end{définition}
\begin{définition}\cmn 到处传送\end{définition}
\begin{exemple}\jya ki χpi ki kɤ-nɤfɕɯfɕɤt ci ŋu\cmn 这个故事很出名\end{exemple}
\end{sous-entrée}\begin{sous-entrée}
\vedette{\hypertarget{}{\papi{ nɯɣɯfɕɤt}}}\markboth{nɯɣɯfɕɤt}{}\classe{vs}
\begin{définition}\fra facile à raconter\end{définition}
\begin{définition}\cmn 容易讲述
\begin{déclaration}\use{用于否定式的时候常常表示“很难交代”的意思}\end{déclaration}\end{définition}
\begin{exemple}\jya mɯ́j-nɯɣɯfɕɤt\cmn (这个故事)很难讲述,这件事情很难交代\end{exemple}
\end{sous-entrée}\begin{sous-entrée}
\vedette{\hypertarget{}{\papi{ rɤfɕɤt}}}\markboth{rɤfɕɤt}{}\classe{vi}
\paradigme{\textit{dir :} \jya nɯ-}
\begin{définition}\ 
\begin{déclaration}\grammar{apass}\end{déclaration}\end{définition}
\begin{définition}\fra raconter, rapporter une information\end{définition}
\begin{définition}\cmn 告诉;转告\end{définition}
\end{sous-entrée}\begin{sous-entrée}
\vedette{\hypertarget{}{\papi{ ʑɣɤfɕɤt}}}\markboth{ʑɣɤfɕɤt}{}\classe{vi}
\paradigme{\textit{dir :} \jya tɤ-}
\begin{définition}\fra rapporter sa situation à qqun\end{définition}
\begin{définition}\cmn 把自己的情况告诉别人\end{définition}
\begin{exemple}\jya a-ɕki tɤ-ʑɣɤfɕɤt\cmn 他把他的情况跟我说了\end{exemple}
\end{sous-entrée}\end{entrée}

\begin{entrée}
\vedette{\hypertarget{ⒺfɕɤtⒽ2}{\papi{ fɕɤt}}}\markboth{fɕɤt}{}\homonyme{2}\classe{vi}\acception{1}
\paradigme{\textit{dir :} \jya tɤ-}
\begin{définition}\fra avoir cette chance\end{définition}
\begin{définition}\cmn 享受这个福分\end{définition}
\begin{exemple}\jya ɲɯ-fɕɤt, mɯ́j-fɕɤt\cmn 他有这个福分,没有这个福分\end{exemple}
\begin{exemple}\jya aʑɯɣ mɤ-fɕɤt\cmn 我没有资格享受这个福分\end{exemple}\acception{2}
\paradigme{\textit{dir :} \jya kɤ-}
\begin{exemple}\jya lɯski kɤ-fɕɤt\cmn 当然可以!(满口答应)\end{exemple}\end{entrée}

\begin{entrée}
\vedette{\hypertarget{Ⓔfɕɤtpa}{\papi{ fɕɤtpa}}}\markboth{fɕɤtpa}{}
\classe{n}
\begin{définition}\fra fanfaronnade\end{définition}
\begin{définition}\cmn 大话\end{définition}
\begin{exemple}\jya fɕɤtpa khro ma-pɯ-tɯ-lɤt, ɯ-ma nɯ tɤ-nɤme\end{exemple}
\begin{exemple}\jya fɕɤtpa ɯ-tshɤt pɯ-lɤt tɕe ɯ-ma nɯ tɤ-nɤme\cmn 少说大话,多办实事\end{exemple}\end{entrée}

\begin{entrée}
\vedette{\hypertarget{Ⓔfɕi}{\papi{ fɕi}}}\markboth{fɕi}{}
\classe{vl}
\paradigme{\textit{dir :} \jya thɯ-}
\begin{définition}\fra forger, travailler le métal\end{définition}
\begin{définition}\cmn 打铁;铸造\end{définition}
\begin{exemple}\jya aʑo kɤ-fɕi spe-a\cmn 我会打铁\end{exemple}
\begin{exemple}\jya kɯki aj thɯ-fɕi-t-a ŋu\cmn 这是我铸造的\end{exemple}
\begin{exemple}\jya kɯki ɯʑo kɯ tha-fɕi\cmn 这是他铸造的\end{exemple}
\begin{exemple}\jya aʑo kɯre ku-fɕi-a\cmn 我在这里打铁\end{exemple}
\begin{relation-sémantique}\confer{
\hyperlink{Ⓔnɯɕɣɤthɯt}{\textit{ \papi{nɯɕɣɤthɯt}}}
}\end{relation-sémantique}
\begin{relation-sémantique}\confer{
\hyperlink{Ⓔrɤlɤt}{\textit{ \papi{rɤlɤt}}}
}\end{relation-sémantique}\end{entrée}

\begin{entrée}
\vedette{\hypertarget{Ⓔfɕur}{\papi{ fɕur}}}\markboth{fɕur}{}
\classe{vt}
\paradigme{\textit{dir :} \jya tɤ-}
\paradigme{\textit{dir :} \jya pɯ-}
\begin{définition}\fra filtrer le thé, verser lentement\end{définition}
\begin{définition}\cmn 过滤茶叶,慢慢地倒下去\end{définition}
\begin{exemple}\jya ɯʑo kɯ ta-fɕur\cmn 他过滤了\end{exemple}
\begin{exemple}\jya tʂha tɤ-fɕur-a\cmn 我过滤了茶\end{exemple}
\begin{exemple}\jya tɯ-ci tɤ-fɕur-a\cmn 我过滤了水\end{exemple}
\begin{exemple}\jya nɤʑo ji-tʂha tɤ-fɕur\cmn 你过滤我们的茶吧\end{exemple}\end{entrée}

\begin{entrée}
\vedette{\hypertarget{Ⓔfɕɯɣ}{\papi{ fɕɯɣ}}}\markboth{fɕɯɣ}{}
\classe{vt}\acception{1}
\paradigme{\textit{dir :} \jya thɯ-}
\begin{définition}\fra déchirer (habits)\end{définition}
\begin{définition}\cmn 拆(衣服)\end{définition}
\begin{exemple}\jya (tɯ-ŋga) tha-fɕɯɣ\cmn 他拆了(衣服)\end{exemple}\acception{2}
\paradigme{\textit{dir :} \jya pɯ-}
\paradigme{\textit{dir :} \jya tɤ-}
\begin{définition}\fra déchirer, démolir\end{définition}
\begin{définition}\cmn 拆(房子)\end{définition}
\begin{exemple}\jya pɯ-fɕɯɣ-a\cmn 我拆了\end{exemple}
\begin{exemple}\jya pɯ-tɯ-fɕɯɣ\cmn 你拆了\end{exemple}
\begin{exemple}\jya pa-fɕɯɣ\cmn 他拆了(房子)\end{exemple}
\begin{exemple}\jya kha pjɤ-fɕɯɣ-nɯ\cmn 他们把房子拆了\end{exemple}
\begin{exemple}\jya ɕoŋβzu kɯ tʂɤm ɲɤ-fɕɯɣ\cmn 木匠把板壁拆下来了\end{exemple}
\begin{exemple}\jya kɯfɕi kɯ laχtɕha chɤ-fɕɯɣ\cmn 铁匠把东西拆下来了\end{exemple}
\begin{relation-sémantique}\synonyme{
\hyperlink{Ⓔqia}{\textit{ \papi{qia}}}
}\end{relation-sémantique}\end{entrée}

\begin{entrée}
\vedette{\hypertarget{ⒺfkaⒽ2}{\papi{ fka}}}\markboth{fka}{}\homonyme{2}
\classe{n}
\begin{définition}\fra ordre\end{définition}
\begin{définition}\cmn 命令
\begin{déclaration} \étymologie{\papi{bka}}\end{déclaration}\end{définition}
\end{entrée}

\begin{entrée}
\vedette{\hypertarget{ⒺfkaⒽ1}{\papi{ fka}}}\markboth{fka}{}\homonyme{1}\classe{vi}
\paradigme{\textit{dir :} \jya tɤ-}
\paradigme{\textit{dir :} \jya nɯ-}
\begin{définition}\fra être rassasié\end{définition}
\begin{définition}\cmn 饱\end{définition}
\begin{définition}\fra être gonflé\end{définition}
\begin{définition}\cmn 胀起来;鼓起来\end{définition}
\begin{exemple}\jya tɤ-fka-a\cmn 我饱了\end{exemple}
\begin{exemple}\jya tɤ-tɯ-fka\cmn 你饱了\end{exemple}
\begin{exemple}\jya ɯʑo tɤ-fka\cmn 他饱了\end{exemple}
\begin{exemple}\jya ɯʑo a-tɤ-fka\cmn 让他饱吧!\end{exemple}
\begin{exemple}\jya tɤ-fkɯm ɲo-fka\cmn 口袋鼓起来了\end{exemple}
\begin{relation-sémantique}\confer{
\hyperlink{Ⓔɯ-xtɤfka}{\textit{ \papi{ɯ-xtɤfka}}}
}\end{relation-sémantique}\begin{sous-entrée}
\vedette{\hypertarget{}{\papi{ \_ɕɯfka}}}\markboth{\_ɕɯfka}{}\classe{vt}
\paradigme{\textit{dir :} \jya tɤ-}
\begin{définition}\ 
\begin{déclaration}\grammar{caus}\end{déclaration}\end{définition}
\begin{définition}\fra permettre à qqn de manger à sa faim\end{définition}
\begin{définition}\cmn 喂饱\end{définition}
\begin{exemple}\jya tɤ-ɕɯfka-t-a\cmn 我把他喂饱了\end{exemple}
\begin{exemple}\jya tɤ-tɯ-ɕɯfka-t\cmn 你喂饱了\end{exemple}
\begin{exemple}\jya ɯʑo kɯ ta-ɕɯfka\cmn 他喂饱了\end{exemple}
\begin{exemple}\jya tɤ-pɤtso kɤ-ngu-t-a tɕe tɤ-ɕɯfka-t-a\cmn 我把孩子喂饱了\end{exemple}
\begin{exemple}\jya ɯ-ɣmba qale kɯ ɲɤ-ɕɯfka\cmn 他鼓起了腮帮子\end{exemple}
\end{sous-entrée}\begin{sous-entrée}
\vedette{\hypertarget{}{\papi{ ɣɤfka}}}\markboth{ɣɤfka}{}\classe{vs}
\begin{définition}\ 
\begin{déclaration}\grammar{facil}\end{déclaration}\end{définition}
\begin{définition}\fra être facilement rassasié\end{définition}
\begin{définition}\cmn 容易饱\end{définition}
\end{sous-entrée}\begin{sous-entrée}
\vedette{\hypertarget{}{\papi{ sɤfka}}}\markboth{sɤfka}{}\classe{vs}
\begin{définition}\fra qui rassasie facilement\end{définition}
\begin{définition}\cmn 吃……吃得饱\end{définition}
\begin{exemple}\jya rɟɤɣi tú-wɣ-ndza tɕe sɤfka\cmn 吃糌粑容易吃饱\end{exemple}
\end{sous-entrée}\begin{sous-entrée}
\vedette{\hypertarget{}{\papi{ ʑɣɤɕɯfka}}}\markboth{ʑɣɤɕɯfka}{}\classe{vi}
\paradigme{\textit{dir :} \jya tɤ-}
\begin{définition}\ 
\begin{déclaration}\grammar{caus}\end{déclaration}
\begin{déclaration}\grammar{refl}\end{déclaration}\end{définition}
\begin{définition}\fra manger à sa faim\end{définition}
\begin{définition}\cmn 吃饱\end{définition}
\begin{exemple}\jya aʑo tɤ-ʑɣɤɕɯfka-a\cmn 我吃饱了\end{exemple}
\begin{exemple}\jya tɤ-ʑɣɤɕɯfka je\cmn 你要吃饱啊!不用客气!\end{exemple}
\end{sous-entrée}\end{entrée}

\begin{entrée}
\vedette{\hypertarget{Ⓔfkaβ}{\papi{ fkaβ}}}\markboth{fkaβ}{}
\classe{vt}
\paradigme{\textit{dir :} \jya pɯ-}
\paradigme{\textit{dir :} \jya kɤ-}
\begin{définition}\fra couvrir\end{définition}
\begin{définition}\cmn 盖
\begin{déclaration} \étymologie{\papi{bkab}}\end{déclaration}\end{définition}
\begin{exemple}\jya kɤ-fkaβ-a\cmn 我盖了\end{exemple}
\begin{exemple}\jya tɯ-mɯ pɯ-pa-fkaβ ʑo ɕ-tɤ-khat-a\cmn 我走遍了天下\end{exemple}
\begin{exemple}\jya ka-fkaβ\cmn 他盖了\end{exemple}
\begin{exemple}\jya pɯ-fkaβ-a\cmn 我盖了\end{exemple}
\begin{exemple}\jya tɯthɯ ka-fkaβ\cmn 他把锅子盖上了\end{exemple}
\begin{exemple}\jya tɕoχtsi pɯ-fkaβ\cmn 你盖一下桌子吧!\end{exemple}
\begin{exemple}\jya ɯ-pɤloʁ kɯ ɯ-zgrɯ kɯ-xtɕɯ-xtɕi chɯ-fkaβ jamar ɲɯ-ŋu ma nɯ ma mɯ́j-zri\cmn 他的袖子不完全盖住他的肘,只有那么长\end{exemple}\begin{sous-entrée}
\vedette{\hypertarget{}{\papi{ ɕɯfkaβ}}}\markboth{ɕɯfkaβ}{}\classe{vt}
\paradigme{\textit{dir :} \jya pɯ-}
\paradigme{\textit{dir :} \jya kɤ-}
\begin{définition}\ 
\begin{déclaration}\grammar{caus}\end{déclaration}\end{définition}
\begin{définition}\fra couvrir (avec quelque chose)\end{définition}
\begin{définition}\cmn (用某个东西)盖\end{définition}
\begin{exemple}\jya a-pɯ-ɕɯfkaβ\cmn 把它盖了吧\end{exemple}
\begin{exemple}\jya kɤntɕhoz mɯ-nɯ-ra tɕe, pɯ-ɕɯfkaβ\cmn (这个锅子)不需要用了,(用锅盖)把他盖上吧!\end{exemple}
\begin{exemple}\jya khɤlɤβ kɯ tɯthɯ pɯ-ɕɯfkaβ-a\cmn 我用锅盖盖了锅子\end{exemple}
\end{sous-entrée}\end{entrée}

\begin{entrée}
\vedette{\hypertarget{Ⓔfkɤn}{\papi{ fkɤn}}}\markboth{fkɤn}{}
\classe{vs}\acception{1}
\paradigme{\textit{dir :} \jya tɤ-}
\begin{définition}\fra sur (endroit)\end{définition}
\begin{définition}\cmn 安全(地方)\end{définition}
\begin{exemple}\jya ki sɤtɕha wuma ʑo ɲɯ-fkɤn\cmn 这个地方很安全\end{exemple}\acception{2}
\begin{définition}\fra fiable (personne)\end{définition}
\begin{définition}\cmn 可靠(人)\end{définition}
\begin{relation-sémantique}\confer{
\hyperlink{Ⓔrkɤl}{\textit{ \papi{rkɤl}}}
}\end{relation-sémantique}\end{entrée}

\begin{entrée}
\vedette{\hypertarget{Ⓔfkot}{\papi{ fkot}}}\markboth{fkot}{}
\classe{vt}
\paradigme{\textit{dir :} \jya tɤ-}
\paradigme{\textit{dir :} \jya pɯ-}
\begin{définition}\fra établir\end{définition}
\begin{définition}\cmn 创造;创建;设计
\begin{déclaration} \étymologie{\papi{bkod}}\end{déclaration}\end{définition}
\begin{exemple}\jya pɯ-fkot-a\cmn 我创建了\end{exemple}
\begin{exemple}\jya ɯʑo kɯ ta-fkot\cmn 他创建了\end{exemple}
\begin{exemple}\jya nɤʑo tɤ-tɯ-fkot\cmn 你创建了\end{exemple}
\begin{exemple}\jya ɕoŋβzu kɯ kha kɤ-βzu pjɤ-fkot\cmn 木匠设计了房子\end{exemple}
\begin{exemple}\jya βdiwa ɲɤ-fkot\cmn 他做了善事\end{exemple}\end{entrée}

\begin{entrée}
\vedette{\hypertarget{Ⓔfkoz}{\papi{ fkoz}}}\markboth{fkoz}{}
\begin{relation-sémantique}\confer{
\hyperlink{Ⓔβgoz}{\textit{ \papi{βgoz}}}
}\end{relation-sémantique}
\end{entrée}

\begin{entrée}
\vedette{\hypertarget{Ⓔfkur}{\papi{ fkur}}}\markboth{fkur}{}
\classe{vt}
\paradigme{\textit{dir :} \jya tɤ-}
\begin{définition}\fra porter sur le dos\end{définition}
\begin{définition}\cmn 背\end{définition}
\begin{exemple}\jya tɤ-tɯ-fkur\cmn 你背了\end{exemple}
\begin{exemple}\jya ɯʑo kɯ ta-fkur\cmn 他背了\end{exemple}
\begin{exemple}\jya aʑo mɯ́j-cha-a tɕe nɤj tɤ-fkur\cmn 我不行,你背吧\end{exemple}\begin{sous-entrée}
\vedette{\hypertarget{}{\papi{ nɤfkɯfkur}}}\markboth{nɤfkɯfkur}{}\classe{vt}
\paradigme{\textit{dir :} \jya tɤ-}
\begin{définition}\fra aller et revenir en portant sur le dos\end{définition}
\begin{définition}\cmn 背来背去\end{définition}
\end{sous-entrée}\end{entrée}

\begin{entrée}
\vedette{\hypertarget{Ⓔfkra}{\papi{ fkra}}}\markboth{fkra}{}
\classe{vs}
\paradigme{\textit{dir :} \jya thɯ-}
\begin{définition}\fra magnifique (peau d'animal)\end{définition}
\begin{définition}\cmn 彩色斑斓
\begin{déclaration} \étymologie{\papi{bkra}}\end{déclaration}\end{définition}
\begin{exemple}\jya qachɣɤndʐi ɲɯ-fkra\cmn 狐狸皮子彩色斑斓\end{exemple}
\begin{exemple}\jya kɯrtsɤɣndʐi ɲɯ-fkra\cmn 豹子皮子彩色斑斓\end{exemple}\end{entrée}

\begin{entrée}
\vedette{\hypertarget{Ⓔfkri}{\papi{ fkri}}}\markboth{fkri}{}
\classe{vt}
\paradigme{\textit{dir :} \jya pɯ-}
\begin{définition}\fra ajouter une poudre dans un liquide\end{définition}
\begin{définition}\cmn 在液体里放粉状的物体然后搅拌(例如在汤里放盐)\end{définition}
\begin{exemple}\jya pɯ-fkri-t-a\cmn 我(在汤里)放了(盐、香料等)\end{exemple}
\begin{exemple}\jya pɯ-tɯ-fkri-t\cmn 你放了\end{exemple}
\begin{exemple}\jya pa-fkri\cmn 他放了\end{exemple}
\begin{exemple}\jya tɤjlu pɯ-fkri-t-a\cmn 我放了面粉\end{exemple}
\begin{exemple}\jya @xiangliao pɯ-fkri-t-a\cmn 我放了香料\end{exemple}
\begin{exemple}\jya tsha pɯ-fkri-t-a\cmn 我放了盐\end{exemple}\end{entrée}

\begin{entrée}
\vedette{\hypertarget{Ⓔfkro}{\papi{ fkro}}}\markboth{fkro}{} (\variante{fkrɤm}) \classe{vt}
\paradigme{\textit{dir :} \jya \_}
\begin{définition}\fra mettre en ordre (des objets identiques)\end{définition}
\begin{définition}\cmn 排列整齐(相同的东西)
\begin{déclaration} \étymologie{\papi{bkram}}\end{déclaration}\end{définition}
\begin{exemple}\jya aʑo pɯ-fkro-t-a\cmn 我布置了\end{exemple}
\begin{exemple}\jya aʑo nɯ-βzdɤr pjɯ-fkram-a\cmn 我把酥油分给大家\end{exemple}\end{entrée}

\begin{entrée}
\vedette{\hypertarget{Ⓔfkrɯz}{\papi{ fkrɯz}}}\markboth{fkrɯz}{}
\classe{vs}
\paradigme{\textit{dir :} \jya tɤ-}
\begin{définition}\fra avide, vorace\end{définition}
\begin{définition}\cmn 贪吃;贪得无厌
\begin{déclaration} \étymologie{\papi{bkres}}\end{déclaration}\end{définition}
\begin{exemple}\jya paʁ ɲɯ-fkrɯz\cmn 猪很贪吃\end{exemple}
\begin{exemple}\jya qapar ɲɯ-fkrɯz\cmn 豺狗很贪吃\end{exemple}
\begin{exemple}\jya jiɕqha tɯrme kɯ-fkrɯz ci ɲɯ-ŋu\cmn 他是个贪吃的人\end{exemple}\begin{sous-entrée}
\vedette{\hypertarget{}{\papi{ sɯfkrɯz}}}\markboth{sɯfkrɯz}{}
\begin{exemple}\jya tu-ta-sɯ-fkrɯz\cmn 我引发你的食欲\end{exemple}
\end{sous-entrée}\end{entrée}

\begin{entrée}
\vedette{\hypertarget{Ⓔfkurzʁe}{\papi{ fkurzʁe}}}\markboth{fkurzʁe}{}\classe{n}
\begin{définition}\fra action de transporter sur le dos\end{définition}
\begin{définition}\cmn 背东西\end{définition}
\begin{exemple}\jya fkurzʁe tɤ-βzu-t-a\cmn 我背了很多东西\end{exemple}
\begin{relation-sémantique}\confer{
\hyperlink{Ⓔnɯfkurzʁe}{\textit{ \papi{nɯfkurzʁe}}}
}\end{relation-sémantique}
\begin{relation-sémantique}\confer{
\hyperlink{Ⓔfkur}{\textit{ \papi{fkur}}}
}\end{relation-sémantique}
\begin{relation-sémantique}\confer{
\hyperlink{Ⓔnɯzʁe}{\textit{ \papi{nɯzʁe}}}
}\end{relation-sémantique}\end{entrée}

\begin{entrée}
\vedette{\hypertarget{Ⓔfkɯm}{\papi{ fkɯm}}}\markboth{fkɯm}{}
\classe{vi}
\begin{définition}\fra qui peut contenir (un liquide)\end{définition}
\begin{définition}\cmn 不漏水,能够装水\end{définition}
\begin{exemple}\jya kɯki tɯthɯ ki mɯ́j-spoʁ tɕe tɯ-ci ɲɯ-fkɯm\cmn 这个锅子没有洞,不漏水\end{exemple}
\begin{exemple}\jya ɲɯ-spoʁ tɕe tɯ-ci mɯ́j-fkɯm ma pjɯ-nɯ-ɬoʁ ɲɯ-ɕti\cmn 有个洞,不能装水否则会漏\end{exemple}\end{entrée}

\begin{entrée}
\vedette{\hypertarget{Ⓔfrtɤn}{\papi{ frtɤn}}}\markboth{frtɤn}{}
\classe{vs}
\paradigme{\textit{dir :} \jya tɤ-}
\begin{définition}\fra fiable\end{définition}
\begin{définition}\cmn 可靠
\begin{déclaration} \étymologie{\papi{brtan}}\end{déclaration}\end{définition}
\begin{exemple}\jya tɯrme ɲɯ-frtɤn\cmn 那个人很可靠\end{exemple}
\begin{relation-sémantique}\synonyme{
\hyperlink{Ⓔfkɤn}{\textit{ \papi{fkɤn}}}
}\end{relation-sémantique}\end{entrée}

\begin{entrée}
\vedette{\hypertarget{Ⓔfsaŋ}{\papi{ fsaŋ}}}\markboth{fsaŋ}{}\classe{n}
\begin{définition}\fra fumigation\end{définition}
\begin{définition}\cmn 烟(“求烟”、供神的烟)
\begin{déclaration} \étymologie{\papi{bsaŋ}}\end{déclaration}\end{définition}
\begin{exemple}\jya fsaŋ la-ta\cmn 他求了烟子\end{exemple}
\begin{exemple}\jya fsaŋ tɤ-ta-t-a\cmn 我求了烟子\end{exemple}
\begin{relation-sémantique}\confer{
\hyperlink{Ⓔsɯfsaŋ}{\textit{ \papi{sɯfsaŋ}}}
}\end{relation-sémantique}
\begin{relation-sémantique}\confer{
\hyperlink{Ⓔtɤfsaŋ}{\textit{ \papi{tɤfsaŋ}}}
}\end{relation-sémantique}\end{entrée}

\begin{entrée}
\vedette{\hypertarget{Ⓔfsaŋkaŋ}{\papi{ fsaŋkaŋ}}}\markboth{fsaŋkaŋ}{}\classe{n}
\begin{définition}\fra cheminée\end{définition}
\begin{définition}\cmn 烟囱\end{définition}
\begin{relation-sémantique}\confer{
\hyperlink{Ⓔfsaŋ}{\textit{ \papi{fsaŋ}}}
}\end{relation-sémantique}
\end{entrée}

\begin{entrée}
\vedette{\hypertarget{Ⓔfsaŋkhɯ}{\papi{ fsaŋkhɯ}}}\markboth{fsaŋkhɯ}{}
\classe{n}
\begin{définition}\fra fumigation\end{définition}
\begin{définition}\cmn 拜神的烟子\end{définition}
\begin{exemple}\jya fsaŋkhɯ ta-tɕɤt\cmn 他求了烟子\end{exemple}\end{entrée}

\begin{entrée}
\vedette{\hypertarget{Ⓔfsaŋmtɕhɤt}{\papi{ fsaŋmtɕhɤt}}}\markboth{fsaŋmtɕhɤt}{}\classe{n}
\begin{définition}\fra fumigations\end{définition}
\begin{définition}\cmn 烧香拜佛冒出来的烟
\begin{déclaration} \étymologie{\papi{bsaŋ.mtɕʰod}}\end{déclaration}\end{définition}
\end{entrée}

\begin{entrée}
\vedette{\hypertarget{Ⓔfsapaʁ}{\papi{ fsapaʁ}}}\markboth{fsapaʁ}{}\classe{n}
\begin{définition}\fra bétail\end{définition}
\begin{définition}\cmn 牲畜\end{définition}
\begin{relation-sémantique}\confer{
\hyperlink{Ⓔarɯfsapaʁ}{\textit{ \papi{arɯfsapaʁ}}}
}\end{relation-sémantique}\end{entrée}

\begin{entrée}
\vedette{\hypertarget{Ⓔfsapaʁɣli}{\papi{ fsapaʁɣli}}}\markboth{fsapaʁɣli}{}\classe{n}
\begin{définition}\fra purin\end{définition}
\begin{définition}\cmn 粪\end{définition}
\begin{relation-sémantique}\confer{
\hyperlink{Ⓔfsapaʁ}{\textit{ \papi{fsapaʁ}}}
}\end{relation-sémantique}
\begin{relation-sémantique}\confer{
\hyperlink{Ⓔtɯ-ɣli}{\textit{ \papi{tɯ-ɣli}}}
}\end{relation-sémantique}
\end{entrée}

\begin{entrée}
\vedette{\hypertarget{Ⓔfsaqhe}{\papi{ fsaqhe}}}\markboth{fsaqhe}{}
\classe{n}
\begin{définition}\fra l'année prochaine\end{définition}
\begin{définition}\cmn 明年\end{définition}
\begin{relation-sémantique}\confer{
\hyperlink{Ⓔfso}{\textit{ \papi{fso}}}
}\end{relation-sémantique}\end{entrée}

\begin{entrée}
\vedette{\hypertarget{Ⓔfsɤl}{\papi{ fsɤl}}}\markboth{fsɤl}{}\classe{vt}
\begin{définition}\ 
\begin{déclaration} \étymologie{\papi{bsal}}\end{déclaration}\end{définition}
\begin{exemple}\jya ji-kɯjŋu nɯ ɲɯ-fsal-a ɲɯ-sɯsam-a\cmn 我要兑现我们的誓言\end{exemple}\end{entrée}

\begin{entrée}
\vedette{\hypertarget{Ⓔfsɤndɤpa}{\papi{ fsɤndɤpa}}}\markboth{fsɤndɤpa}{}\classe{n}
\begin{définition}\fra l'année d'après\end{définition}
\begin{définition}\cmn 后年\end{définition}
\begin{relation-sémantique}\confer{
\hyperlink{Ⓔfsɤndi}{\textit{ \papi{fsɤndi}}}
}\end{relation-sémantique}
\end{entrée}

\begin{entrée}
\vedette{\hypertarget{Ⓔfsɤndi}{\papi{ fsɤndi}}}\markboth{fsɤndi}{}\classe{adv}
\begin{définition}\fra après demain\end{définition}
\begin{définition}\cmn 后天\end{définition}
\begin{exemple}\jya fsɤndi tɕe li kɯmaʁ ji-kɤ-nɤma ɣɤʑu\cmn 后天我们这里有另外一件事要做\end{exemple}\end{entrée}

\begin{entrée}
\vedette{\hypertarget{ⒺfseⒽ1}{\papi{ fse}}}\markboth{fse}{}\homonyme{1}
\classe{vs}
\paradigme{\textit{dir :} \jya tɤ-}\acception{1}
\begin{définition}\fra ressembler\end{définition}
\begin{définition}\cmn 像\end{définition}
\begin{exemple}\jya ɯ-mu fse\cmn 他像他母亲\end{exemple}
\begin{exemple}\jya ki kɯ-fse\cmn 这样\end{exemple}
\begin{exemple}\jya nɯ kɯ-fse\cmn 那样\end{exemple}
\begin{exemple}\jya nɯnɯ kɯ-fse kɯ-tu maŋe\cmn 这不算什么\end{exemple}
\begin{exemple}\jya mɤ-kɤ-nɯ-fse ʑo kɯ-me\cmn 各种各样、五花八门\end{exemple}
\begin{exemple}\jya a-tɯ-mbro nɤʑo ɯ-ɲɯ́-fse-a\cmn 我跟你一样高吗?\end{exemple}\acception{2}
\begin{définition}\fra être / se passer de cette manière\end{définition}
\begin{définition}\cmn 这样发生\end{définition}
\begin{exemple}\jya ɯʑo kɯ nɯra fse mɯ-ɲɤ-sɯso\cmn 我没有想到会那样\end{exemple}
\begin{exemple}\jya tɕhi nɯ mɤ-nɯ-fse\cmn 有什么不好?\end{exemple}
\begin{exemple}\jya nɤ-stu tɤ-fse\cmn 你小心一点\end{exemple}
\begin{exemple}\jya rɟɤlpu fka ɕti tɕe, nɯ mɤ-tɯ-fse mɤ-jɤɣ\cmn 这是皇上的旨意,不能不照办\end{exemple}
\begin{exemple}\jya maka nɯ fse mɯ-ɲɤ-sɯso ri\cmn 出乎意料(完全没有想到会那样)\end{exemple}\begin{sous-entrée}
\vedette{\hypertarget{}{\papi{ nɤfse}}}\markboth{nɤfse}{}\classe{vt}
\begin{définition}\fra trouver ressemblant\end{définition}
\begin{définition}\cmn 觉得很像\end{définition}
\begin{relation-sémantique}\confer{
\hyperlink{Ⓔsɤfse}{\textit{ \papi{sɤfse}}}
}\end{relation-sémantique}
\begin{relation-sémantique}\confer{
\hyperlink{Ⓔamɯfse}{\textit{ \papi{amɯfse}}}
}\end{relation-sémantique}
\begin{relation-sémantique}\confer{
\hyperlink{ⒺnɯfseⒽ1}{\textit{ \papi{nɯfse1}}}
}\end{relation-sémantique}
\begin{relation-sémantique}\confer{
\hyperlink{ⒺnɯfseⒽ2}{\textit{ \papi{nɯfse2}}}
}\end{relation-sémantique}
\end{sous-entrée}\end{entrée}

\begin{entrée}
\vedette{\hypertarget{ⒺfseⒽ2}{\papi{ fse}}}\markboth{fse}{}\homonyme{2}
\classe{vt}\acception{1}
\paradigme{\textit{dir :} \jya thɯ-}
\begin{définition}\fra aiguiser\end{définition}
\begin{définition}\cmn 磨(刀)\end{définition}
\begin{exemple}\jya mbrɯtɕɯ thɯ-fse-t-a\cmn 我磨了刀子\end{exemple}
\begin{exemple}\jya thɯ-tɯ-fse-t\cmn 你磨了\end{exemple}
\begin{exemple}\jya tha-fse\cmn 他磨了\end{exemple}
\begin{exemple}\jya ldɯɣɯ tha-fse\cmn 他磨了弯刀\end{exemple}
\begin{exemple}\jya tɯrpa tha-fse\cmn 他磨了斧头\end{exemple}\acception{2}
\paradigme{\textit{dir :} \jya tɤ-}
\begin{définition}\fra frotter\end{définition}
\begin{définition}\cmn 搓\end{définition}
\begin{exemple}\jya a-jaʁ tɤ-fse-t-a\cmn 我搓了一下手\end{exemple}\begin{sous-entrée}
\vedette{\hypertarget{}{\papi{ rɤfse}}}\markboth{rɤfse}{}\classe{vi}
\paradigme{\textit{dir :} \jya thɯ-}
\begin{définition}\ 
\begin{déclaration}\grammar{apass}\end{déclaration}\end{définition}
\begin{définition}\fra aiguiser les couteaux\end{définition}
\begin{définition}\cmn 磨刀\end{définition}
\end{sous-entrée}\begin{sous-entrée}
\vedette{\hypertarget{}{\papi{ sɤfsɯfse}}}\markboth{sɤfsɯfse}{}
\begin{définition}\fra frotter ... l'un contre l'autre\end{définition}
\begin{définition}\cmn 把……互相摩擦\end{définition}
\begin{exemple}\jya ɯ-jaʁ to-sɤfsɯfse\cmn 他搓了搓手\end{exemple}
\end{sous-entrée}\end{entrée}

\begin{entrée}
\vedette{\hypertarget{Ⓔfsjɤnɤfsjɤt}{\papi{ fsjɤnɤfsjɤt}}}\markboth{fsjɤnɤfsjɤt}{}\classe{idph.3}
\begin{définition}\fra qui marche sans efforts\end{définition}
\begin{définition}\cmn 形容七八岁的小孩子走路很轻松的样子\end{définition}
\begin{exemple}\jya tɤ-pɤtso fsjɤnɤfsjɤt ʑo lɤ-ari\cmn 小孩子很轻松地进去了\end{exemple}\end{entrée}

\begin{entrée}
\vedette{\hypertarget{Ⓔfsjitnɤfsjat}{\papi{ fsjitnɤfsjat}}}\markboth{fsjitnɤfsjat}{}\classe{idph.3}
\begin{définition}\fra sifflement\end{définition}
\begin{définition}\cmn 吹口哨的声音\end{définition}
\begin{exemple}\jya fsapaʁ ra nɯ-qhu tɕe fsjitnɤfsjat ʑo ɲɯ-ti tɕe ja-no\cmn 他在牛后面“嗖—嗖”地吹口哨赶牛\end{exemple}\end{entrée}

\begin{entrée}
\vedette{\hypertarget{Ⓔfskɤr}{\papi{ fskɤr}}}\markboth{fskɤr}{}
\classe{vt}
\paradigme{\textit{dir :} \jya tɤ-}
\begin{définition}\fra contourner\end{définition}
\begin{définition}\cmn 绕道;围着走
\begin{déclaration} \étymologie{\papi{skor}}\end{déclaration}\end{définition}
\begin{exemple}\jya ɯʑo kɯ a-tɤ-fskɤr\cmn 他绕过去吧!\end{exemple}
\begin{exemple}\jya aʑo tu-fskar-a\cmn 我绕过去\end{exemple}
\begin{exemple}\jya ki rdɤstaʁ kɯ-wxti ci ɣɤʑu tɕe kɤ-fskɤr ɲɯ-ra\cmn 这块大石头,要绕过去\end{exemple}
\begin{exemple}\jya kɯki tʂu ki ɕɯ-kɤ-fskɤr ra\cmn 要绕路\end{exemple}\begin{sous-entrée}
\vedette{\hypertarget{}{\papi{ nɤfskɯfskɤr}}}\markboth{nɤfskɯfskɤr}{}\classe{vt}
\paradigme{\textit{dir :} \jya tɤ-}
\begin{définition}\fra tourner autour\end{définition}
\begin{définition}\cmn 转来转去\end{définition}
\begin{exemple}\jya si ɣɯ ɯ-rkɯ tú-wɣ-nɤfskɯfskɤr tɕe tɤjmɤɣ ɣɯ-mto\cmn 在树的周围转来转去就会找到菌子\end{exemple}
\begin{relation-sémantique}\confer{
\hyperlink{Ⓔtɯ-tɤfskɤr}{\textit{ \papi{tɯ-tɤfskɤr}}}
}\end{relation-sémantique}
\end{sous-entrée}\end{entrée}

\begin{entrée}
\vedette{\hypertarget{Ⓔfso}{\papi{ fso}}}\markboth{fso}{}
\classe{n}\acception{1}
\begin{définition}\fra demain\end{définition}
\begin{définition}\cmn 明天\end{définition}
\begin{exemple}\jya ɯ-fso\cmn 第二天\end{exemple}\acception{2}
\begin{définition}\fra à l'avenir\end{définition}
\begin{définition}\cmn 将来\end{définition}
\begin{exemple}\jya fso thɯ-tɯ-rgɤz tɕe kɤ-ɣi mɤ-tɯ-cha\cmn 你将来老了就不再能来\end{exemple}\end{entrée}

\begin{entrée}
\vedette{\hypertarget{Ⓔfsomɯr}{\papi{ fsomɯr}}}\markboth{fsomɯr}{}
\classe{n}
\begin{définition}\fra demain soir\end{définition}
\begin{définition}\cmn 明晚\end{définition}
\begin{exemple}\jya ɯ-fsomɯr\cmn 第二天晚上\end{exemple}\end{entrée}

\begin{entrée}
\vedette{\hypertarget{ⒺfsoʁⒽ1}{\papi{ fsoʁ}}}\markboth{fsoʁ}{}\homonyme{1}
\classe{vt}
\paradigme{\textit{dir :} \jya nɯ-}
\begin{définition}\fra gagner (de l'argent)\end{définition}
\begin{définition}\cmn 挣(钱)
\begin{déclaration} \étymologie{\papi{bsogs}}\end{déclaration}\end{définition}
\begin{exemple}\jya (tɯ-rɟɯ) nɯ-fso-ʁa\cmn 我挣了(钱)\end{exemple}
\begin{exemple}\jya nɯ-tɯ-fsoʁ\cmn 你挣了\end{exemple}
\begin{exemple}\jya na-fsoʁ\cmn 他挣了\end{exemple}
\begin{exemple}\jya tɯ-rɟɯ kɤ-fsoʁ ɴqa\cmn 挣钱发财是一件很难的事情\end{exemple}
\begin{exemple}\jya ɯʑo kɯ tɯ-rɟɯ kɤfsoʁ cha\cmn 他很会挣钱\end{exemple}
\begin{exemple}\jya qaʑo kú-wɣ-ftɕa tɕe, pɕawtsɯ sɤ-fsoʁ ŋu\cmn 饲养羊是一种赚钱的方式\end{exemple}\begin{sous-entrée}
\vedette{\hypertarget{}{\papi{ rɤfsoʁ}}}\markboth{rɤfsoʁ}{}\classe{vi}
\begin{définition}\ 
\begin{déclaration}\grammar{apass}\end{déclaration}\end{définition}
\begin{exemple}\jya ɯʑo kɤ-rɤfsoʁ ɲɯ-rkaŋ\cmn 他很会挣钱\end{exemple}
\end{sous-entrée}\end{entrée}

\begin{entrée}
\vedette{\hypertarget{ⒺfsoʁⒽ2}{\papi{ fsoʁ}}}\markboth{fsoʁ}{}\homonyme{2}
\classe{vs}
\paradigme{\textit{dir :} \jya nɯ-}
\paradigme{\textit{dir :} \jya lɤ-}
\begin{définition}\fra clair (ciel)\end{définition}
\begin{définition}\cmn 亮(天色)\end{définition}
\begin{exemple}\jya kɯki ɲɯ-fsoʁ\cmn 这个很亮\end{exemple}
\begin{exemple}\jya tɤtʂu ɲɯ-fsoʁ\cmn 灯很亮\end{exemple}
\begin{exemple}\jya tɯ-mɯ ɲɤ-fsoʁ\cmn 天亮开了(原来是阴天)\end{exemple}
\begin{exemple}\jya kha ɲo-fsoʁ\cmn 房间变亮了\end{exemple}
\begin{exemple}\jya tɯtshot kɯtʂɤɣ ko-zɣɯt tɕe lo-fsoʁ\cmn 到了六点钟,天就亮\end{exemple}
\begin{relation-sémantique}\confer{
\hyperlink{Ⓔtɯfsɤkha}{\textit{ \papi{tɯfsɤkha}}}
}\end{relation-sémantique}\begin{sous-entrée}
\vedette{\hypertarget{}{\papi{ ɣɤfsoʁ}}}\markboth{ɣɤfsoʁ}{}\classe{vt}
\paradigme{\textit{dir :} \jya nɯ-}
\begin{définition}\fra rendre clair\end{définition}
\begin{définition}\cmn 使变亮\end{définition}
\begin{exemple}\jya kɯki kha ɲɯ-qanɯ ri, kɤ-ɣɤfsoʁ khɯ\cmn 这个房间很黑,可以令它变亮\end{exemple}
\end{sous-entrée}\begin{sous-entrée}
\vedette{\hypertarget{}{\papi{ sɯfsoʁ}}}\markboth{sɯfsoʁ}{}\classe{vt}
\paradigme{\textit{dir :} \jya nɯ-}
\begin{définition}\fra rendre clair\end{définition}
\begin{définition}\cmn 使变亮\end{définition}
\begin{relation-sémantique}\antonyme{
\hyperlink{Ⓔqanɯ}{\textit{ \papi{qanɯ}}}
}\end{relation-sémantique}
\end{sous-entrée}\end{entrée}

\begin{entrée}
\vedette{\hypertarget{Ⓔfsosɲɯm}{\papi{ fsosɲɯm}}}\markboth{fsosɲɯm}{}\classe{n}
\begin{définition}\fra aumônes (aux moines)\end{définition}
\begin{définition}\cmn 布施\end{définition}
\begin{exemple}\jya wortɕhi ʑo, a-fsosɲɯm ci pɯ-lɤt\cmn 请你给我布施\end{exemple}
\begin{exemple}\jya fsosɲɯm nɯ-sɤmbi-a\cmn 我讨布施了\end{exemple}\end{entrée}

\begin{entrée}
\vedette{\hypertarget{Ⓔfsraŋ}{\papi{ fsraŋ}}}\markboth{fsraŋ}{} (\variante{fsroŋ}) 
\classe{vt}
\paradigme{\textit{dir :} \jya tɤ-}\acception{1}
\begin{définition}\fra sauver\end{définition}
\begin{définition}\cmn 救……一命
\begin{déclaration} \étymologie{\papi{bsruŋ}}\end{déclaration}\end{définition}
\begin{exemple}\jya tɤ-fsraŋ-a\cmn 我救了他\end{exemple}
\begin{exemple}\jya ta-fsraŋ\cmn 他救了他\end{exemple}
\begin{exemple}\jya tɤ-kɯ-fsraŋ-a\cmn 你救了我\end{exemple}
\begin{exemple}\jya aʑo a-kɯ-mŋɤm pɯ-tu ri, ɯʑo kɯ tɤ́-wɣ-fsraŋ-a ŋu\cmn 在我生病的时候,他救了我的命\end{exemple}
\begin{exemple}\jya a-sroʁ pɯ-sɤɣʑɯr ri tɤ́-wɣ-fsroŋ-a ŋu\cmn 在我生命垂危的时候,他救了我\end{exemple}\acception{2}
\begin{définition}\fra protéger\end{définition}
\begin{définition}\cmn 保护\end{définition}
\begin{relation-sémantique}\synonyme{
\hyperlink{Ⓔtɯ-sroʁ,ri}{\textit{ \papi{tɯ-sroʁ,ri}}}
}\end{relation-sémantique}
\begin{sous-entrée}
\vedette{\hypertarget{}{\papi{ ʑɣɤfsraŋ}}}\markboth{ʑɣɤfsraŋ}{}\classe{vi}
\paradigme{\textit{dir :} \jya tɤ-}
\begin{définition}\ 
\begin{déclaration}\grammar{refl}\end{déclaration}\end{définition}
\begin{définition}\fra se protéger\end{définition}
\begin{définition}\cmn 保护自己\end{définition}
\end{sous-entrée}\end{entrée}

\begin{entrée}
\vedette{\hypertarget{Ⓔfsraŋma}{\papi{ fsraŋma}}}\markboth{fsraŋma}{} (\variante{sraŋma}) 
\classe{n}
\begin{définition}\fra divinité protectrice\end{définition}
\begin{définition}\cmn 神仙
\begin{déclaration} \étymologie{\papi{bsruŋ.ma}}\end{déclaration}\end{définition}\end{entrée}

\begin{entrée}
\vedette{\hypertarget{Ⓔfsusqi}{\papi{ fsusqi}}}\markboth{fsusqi}{}\classe{num}
\begin{définition}\fra trente\end{définition}
\begin{définition}\cmn 三十\end{définition}
\end{entrée}

\begin{entrée}
\vedette{\hypertarget{Ⓔfsusqipa}{\papi{ fsusqipa}}}\markboth{fsusqipa}{}
\begin{relation-sémantique}\confer{
\hyperlink{Ⓔfsusqi}{\textit{ \papi{fsusqi}}}
}\end{relation-sémantique}
\begin{relation-sémantique}\confer{
\hyperlink{Ⓔtɯ-xpa}{\textit{ \papi{tɯ-xpa}}}
}\end{relation-sémantique}\end{entrée}

\begin{entrée}
\vedette{\hypertarget{Ⓔfstɤt}{\papi{ fstɤt}}}\markboth{fstɤt}{}
\classe{vt}
\paradigme{\textit{dir :} \jya tɤ-}
\begin{définition}\fra louer\end{définition}
\begin{définition}\cmn 称赞;抬举
\begin{déclaration} \étymologie{\papi{bstod}}\end{déclaration}\end{définition}
\begin{exemple}\jya tu-ta-fstɤt\cmn 我称赞你\end{exemple}
\begin{exemple}\jya to-fstɤt\cmn 他称赞了他\end{exemple}
\begin{exemple}\jya nɯsthɯci tu-kɯ-fstat-a mɤ-ra wo!\cmn 你过奖了\end{exemple}
\begin{relation-sémantique}\synonyme{
\hyperlink{Ⓔɣɤmɯ}{\textit{ \papi{ɣɤmɯ}}}
}\end{relation-sémantique}\end{entrée}

\begin{entrée}
\vedette{\hypertarget{Ⓔfstɯn}{\papi{ fstɯn}}}\markboth{fstɯn}{}\classe{vt}
\paradigme{\textit{dir :} \jya kɤ-}
\begin{définition}\fra servir, s'occuper de\end{définition}
\begin{définition}\cmn 伺候;照顾
\begin{déclaration} \étymologie{\papi{bstun}}\end{déclaration}\end{définition}
\begin{exemple}\jya kɤ-fstɯn-a\cmn 我照顾了他\end{exemple}
\begin{exemple}\jya kɤ-tɯ-fstɯn\cmn 你照顾了他\end{exemple}
\begin{exemple}\jya ɯʑo kɯ kɤ́-wɣ-fstɯna\cmn 他照顾了我\end{exemple}
\begin{sous-entrée}
\vedette{\hypertarget{}{\papi{ sɤfstɯn}}}\markboth{sɤfstɯn}{}\classe{vi}
\paradigme{\textit{dir :} \jya kɤ-}
\begin{définition}\ 
\begin{déclaration}\grammar{apass}\end{déclaration}\end{définition}
\begin{définition}\fra servir\end{définition}
\begin{définition}\cmn 伺候\end{définition}
\begin{exemple}\jya nɤʑo kɤ-sɤfstɯn tɕhi tu-tɯ-fse ŋu\cmn 你是怎么伺候人家的呢?\end{exemple}
\end{sous-entrée}\begin{sous-entrée}
\vedette{\hypertarget{}{\papi{ ʑɣɤfstɯn}}}\markboth{ʑɣɤfstɯn}{}\classe{vi}
\begin{définition}\ 
\begin{déclaration}\grammar{refl}\end{déclaration}\end{définition}
\begin{définition}\fra s'occuper de soi-même\end{définition}
\begin{définition}\cmn 照顾自己\end{définition}
\end{sous-entrée}\end{entrée}

\begin{entrée}
\vedette{\hypertarget{Ⓔfsɯɣ}{\papi{ fsɯɣ}}}\markboth{fsɯɣ}{}
\classe{vt}
\paradigme{\textit{dir :} \jya nɯ-}
\paradigme{\textit{dir :} \jya thɯ-}
\begin{définition}\fra rendre la monnaie\end{définition}
\begin{définition}\cmn 退还多余的部分,退零钱
\begin{déclaration} \étymologie{\papi{gsug.pa}}\end{déclaration}
\end{définition}
\begin{exemple}\jya aʑo nɯ-fsɯɣ-a\cmn 我还了\end{exemple}
\begin{exemple}\jya nɤʑo nɯ-tɯ-fsɯɣ\cmn 你还了\end{exemple}
\begin{exemple}\jya ɯʑo kɯ na-fsɯɣ\cmn 他还了\end{exemple}
\begin{exemple}\jya ɯ-phɯ ɯ-kho nɯ-kɯ-ro nɯnɯ nɯ́-wɣ-fsɯɣ-a\cmn 他把多给的钱还给我了\end{exemple}
\begin{exemple}\jya ki ɯ-phɯ ki kɤ-kho ɲɤ-ro tɕe ɲɯ-ta-fsɯɣ\cmn 你多给了钱,我找一下零钱给你。\end{exemple}
\begin{exemple}\jya pɕawtsɯ ɲɤ-fsɯɣ\cmn 他找了零钱\end{exemple}
\begin{relation-sémantique}\synonyme{
\hyperlink{Ⓔɕɣɤz}{\textit{ \papi{ɕɣɤz}}}
}\end{relation-sémantique}\end{entrée}

\begin{entrée}
\vedette{\hypertarget{Ⓔfsɯr}{\papi{ fsɯr}}}\markboth{fsɯr}{}\classe{vs}
\paradigme{\textit{dir :} \jya tɤ-}
\begin{définition}\fra avoir besoin de viande\end{définition}
\begin{définition}\cmn 需要吃肉\end{définition}
\begin{exemple}\jya to-ngo tɕe ɲɯ-fsɯr\cmn 他生病了,需要吃肉\end{exemple}
\begin{exemple}\jya nɤʑo tɤ-mthɯm tu-tɯ-ndze ntsɯ ɕti ri, nɯ kɯnɤ ɲɯ-tɯ-fsɯr!\cmn 你总是吃这么多肉,还是那么馋嘴、\end{exemple}\end{entrée}

\begin{entrée}
\vedette{\hypertarget{Ⓔfsɯz}{\papi{ fsɯz}}}\markboth{fsɯz}{}\classe{vi}
\paradigme{\textit{dir :} \jya pɯ-}
\begin{définition}\fra possible\end{définition}
\begin{définition}\cmn 可能\end{définition}
\begin{exemple}\jya fsɯfsɯz ʑo\cmn 千方百计\end{exemple}
\begin{exemple}\jya aʑo a-pɯ-fsɯz tɕe, sɲikuku ʑo ɣi-a ɕti\cmn 只有我可以,我天天来\end{exemple}
\begin{exemple}\jya ɯʑo a-pɯ-fsɯz tɕe, ɯʑo laχtɕha ra nɯ-ndɤm, tɯrme laχtɕha kɯnɤ nɯ-ndɤm\cmn 只要他有机会,他不但会拿自己的东西,也会拿别人的\end{exemple}\end{entrée}

\begin{entrée}
\vedette{\hypertarget{Ⓔftɕa}{\papi{ ftɕa}}}\markboth{ftɕa}{}
\classe{vt}
\paradigme{\textit{dir :} \jya tɤ-}
\begin{définition}\fra posséder\end{définition}
\begin{définition}\cmn 拥有\end{définition}
\begin{exemple}\jya jiʑo @shouji tɤ-ftɕa-j\cmn 我有了手机\end{exemple}
\begin{exemple}\jya qaʑo kú-wɣ-ftɕa tɕe, pɕawtsɯ sɤ-fsoʁ ŋu\cmn 拥有绵羊是赚钱的方式\end{exemple}
\begin{relation-sémantique}\synonyme{
\hyperlink{Ⓔaro}{\textit{ \papi{aro}}}
}\end{relation-sémantique}\end{entrée}

\begin{entrée}
\vedette{\hypertarget{Ⓔftɕaka}{\papi{ ftɕaka}}}\markboth{ftɕaka}{}\classe{n}
\begin{définition}\fra méthode\end{définition}
\begin{définition}\cmn 办法
\begin{déclaration} \étymologie{\papi{btɕa.ka}}\end{déclaration}\end{définition}
\begin{relation-sémantique}\confer{
\hyperlink{Ⓔaftɕaka}{\textit{ \papi{aftɕaka}}}
}\end{relation-sémantique}
\begin{relation-sémantique}\confer{
\hyperlink{Ⓔrɯftɕaka}{\textit{ \papi{rɯftɕaka}}}
}\end{relation-sémantique}
\begin{relation-sémantique}\confer{
\hyperlink{Ⓔsɤftɕaka}{\textit{ \papi{sɤftɕaka}}}
}\end{relation-sémantique}
\begin{sous-entrée}
\vedette{\hypertarget{}{\papi{ ftɕaka,βzu}}}\markboth{ftɕaka,βzu}{}
\begin{définition}\fra essayer par tous les moyens de\end{définition}
\begin{définition}\cmn 想办法……\end{définition}
\begin{exemple}\jya tɤ-rɤku kɤ-mɯrkɯ ftɕaka wuma ʑo βze\cmn 它想尽办法偷吃庄稼\end{exemple}
\begin{relation-sémantique}\ComponentA{\classe{n}
\hyperlink{Ⓔftɕaka}{\textit{ \papi{ftɕaka}}}
}\end{relation-sémantique}
\begin{relation-sémantique}\ComponentB{\classe{vt}
\hyperlink{ⒺβzuⒽ1}{\textit{ \papi{βzu}}}
}\end{relation-sémantique}
\end{sous-entrée}\end{entrée}

\begin{entrée}
\vedette{\hypertarget{Ⓔftɕar}{\papi{ ftɕar}}}\markboth{ftɕar}{}
\classe{n}
\begin{définition}\fra été, période du printemps à l'été\end{définition}
\begin{définition}\cmn 夏天;春夏\end{définition}
\begin{exemple}\jya ftɕar kɤ-ndzoʁ\end{exemple}
\begin{exemple}\jya ftɕar ɯ-qa ka-ta\cmn 春天开始了\end{exemple}
\begin{relation-sémantique}\confer{
\hyperlink{Ⓔftɕɤru}{\textit{ \papi{ftɕɤru}}}
}\end{relation-sémantique}\end{entrée}

\begin{entrée}
\vedette{\hypertarget{Ⓔftɕaʁ}{\papi{ ftɕaʁ}}}\markboth{ftɕaʁ}{}\classe{vt}
\paradigme{\textit{dir :} \jya pɯ-}
\begin{définition}\fra être une brebis gâleuse\end{définition}
\begin{définition}\cmn 一个老鼠屎坏了一锅粥\end{définition}
\begin{exemple}\jya ma-pɯ-kɯ-ftɕaʁ-i\cmn 不要成为我们团队里的老鼠屎\end{exemple}
\begin{relation-sémantique}\confer{
\hyperlink{Ⓔsɯftɕaʁ}{\textit{ \papi{sɯftɕaʁ}}}
}\end{relation-sémantique}
\begin{sous-entrée}
\vedette{\hypertarget{}{\papi{ sɤftɕaʁ}}}\markboth{sɤftɕaʁ}{}
\begin{définition}\ 
\begin{déclaration}\grammar{apass}\end{déclaration}\end{définition}
\begin{exemple}\jya ma-ɕɯ-tɯ-sɤftɕaʁ\cmn 你不要做老鼠屎\end{exemple}
\end{sous-entrée}\end{entrée}

\begin{entrée}
\vedette{\hypertarget{Ⓔftɕɤfkɤt}{\papi{ ftɕɤfkɤt}}}\markboth{ftɕɤfkɤt}{}\classe{n}\acception{1}
\begin{définition}\fra organisation\end{définition}
\begin{définition}\cmn 指挥;安排(事情)
\begin{déclaration} \étymologie{\papi{btɕa.bkod?}}\end{déclaration}\end{définition}
\begin{exemple}\jya jisŋi kɤ-nɤma kɯkɯra thamtɕɤt aʑo ji-ftɕɤfkɤt tu-βze-a\cmn 今天我们要做的工作,我来指挥\end{exemple}\acception{2}
\begin{définition}\fra suggestion, idée, conseil\end{définition}
\begin{définition}\cmn 主意\end{définition}
\begin{exemple}\jya nɯ-ftɕɤfkɤt to-tɕɤt\cmn 他给他们出了主意(给他们做了安排)\end{exemple}
\begin{exemple}\jya a-ftɕɤfkɤt ci tɤ-tɕɤt\cmn 给我出主意吧\end{exemple}
\begin{relation-sémantique}\synonyme{
\hyperlink{Ⓔβluβra}{\textit{ \papi{βluβra}}}
}\end{relation-sémantique}
\begin{relation-sémantique}\confer{
\hyperlink{Ⓔrɯftɕɤfkɤt}{\textit{ \papi{rɯftɕɤfkɤt}}}
}\end{relation-sémantique}\end{entrée}

\begin{entrée}
\vedette{\hypertarget{Ⓔftɕɤl}{\papi{ ftɕɤl}}}\markboth{ftɕɤl}{}
\classe{vt}
\paradigme{\textit{dir :} \jya tɤ-}
\begin{définition}\fra demander à qqn de faire qqch pour soi\end{définition}
\begin{définition}\cmn 请别人做事
\begin{déclaration} \étymologie{\papi{btɕol}}\end{déclaration}\end{définition}
\begin{exemple}\jya tó-wɣ-ftɕala\cmn 他请了我\end{exemple}
\begin{exemple}\jya to-tɯ́-wɣ-ftɕɤl\cmn 他请了你\end{exemple}
\begin{exemple}\jya a-laχtɕha kɤ-χtɯ ci tu-ta-ftɕɤl\cmn 我请你帮我买(某种东西)\end{exemple}\begin{sous-entrée}
\vedette{\hypertarget{}{\papi{ sɤftɕɤl}}}\markboth{sɤftɕɤl}{}\classe{vi}
\begin{définition}\fra demander à des gens de faire qqch pour soi\end{définition}
\begin{définition}\cmn 请人做事\end{définition}
\begin{exemple}\jya ɲɯ-sɤftɕɤl\cmn 他请别人做事\end{exemple}
\begin{relation-sémantique}\synonyme{
\hyperlink{Ⓔsqɤr}{\textit{ \papi{sqɤr}}}
}\end{relation-sémantique}
\end{sous-entrée}\end{entrée}

\begin{entrée}
\vedette{\hypertarget{Ⓔftɕɤru}{\papi{ ftɕɤru}}}\markboth{ftɕɤru}{}\classe{n}
\begin{définition}\fra chemin au milieu des champs (pour éviter d'abîmer les récoltes)\end{définition}
\begin{définition}\cmn 阡陌(夏天为了避免踩坏庄稼而特意在田地里开辟的一条小道)\end{définition}
\begin{relation-sémantique}\confer{
\hyperlink{Ⓔftɕar}{\textit{ \papi{ftɕar}}}
}\end{relation-sémantique}
\begin{relation-sémantique}\confer{
\hyperlink{Ⓔtʂu}{\textit{ \papi{tʂu}}}
}\end{relation-sémantique}
\end{entrée}

\begin{entrée}
\vedette{\hypertarget{Ⓔftɕɤt}{\papi{ ftɕɤt}}}\markboth{ftɕɤt}{}\classe{vt}\acception{1}
\paradigme{\textit{dir :} \jya nɯ-}
\begin{définition}\fra s'abstenir, renoncer à une mauvaise habitude\end{définition}
\begin{définition}\cmn 戒掉
\begin{déclaration} \étymologie{\papi{btɕad}}\end{déclaration}\end{définition}
\begin{exemple}\jya ɯʑo kɯ na-ftɕɤt\cmn 他戒掉了\end{exemple}
\begin{exemple}\jya nɤʑo nɯ-tɯ-ftɕɤt\cmn 你戒掉了\end{exemple}
\begin{exemple}\jya cha kɤ-tshi mɯ́j-sna tɕe nɯ-ftɕat-a\cmn 喝酒是不好的,所以我就戒掉了\end{exemple}
\begin{exemple}\jya thamakha nɯ-ftɕat-a\cmn 我戒了烟\end{exemple}
\begin{exemple}\jya thamakha kɤ-sko na-ftɕɤt\cmn 他戒了烟\end{exemple}\acception{2}
\begin{définition}\fra subjuger, soumettre\end{définition}
\begin{définition}\cmn 征服;使……驯服\end{définition}
\begin{relation-sémantique}\synonyme{
\hyperlink{ⒺftɯlⒽ2}{\textit{ \papi{ftɯl2}}}
}\end{relation-sémantique}\end{entrée}

\begin{entrée}
\vedette{\hypertarget{Ⓔftɕɤz}{\papi{ ftɕɤz}}}\markboth{ftɕɤz}{}\classe{vt}
\paradigme{\textit{dir :} \jya nɯ-}
\begin{définition}\fra castrer\end{définition}
\begin{définition}\cmn 阉割\end{définition}
\begin{exemple}\jya nɯ-ftɕaz-a\cmn 我阉割了\end{exemple}
\begin{exemple}\jya mbala na-ftɕɤz\cmn 他把公牛阉割了\end{exemple}
\begin{exemple}\jya mbala kɤ-ftɕɤz ra\cmn 要把公牛阉割掉\end{exemple}\begin{sous-entrée}
\vedette{\hypertarget{}{\papi{ rɤftɕɤz}}}\markboth{rɤftɕɤz}{}\classe{vi}
\paradigme{\textit{dir :} \jya nɯ-}
\begin{définition}\ 
\begin{déclaration}\grammar{apass}\end{déclaration}\end{définition}
\begin{définition}\fra castrer les animaux\end{définition}
\begin{définition}\cmn 阉割\end{définition}
\begin{relation-sémantique}\synonyme{
\hyperlink{Ⓔɣɤβdi}{\textit{ \papi{ɣɤβdi}}}
}\end{relation-sémantique}
\end{sous-entrée}\end{entrée}

\begin{entrée}
\vedette{\hypertarget{Ⓔftɕhoʁftɕhoʁ}{\papi{ ftɕhoʁftɕhoʁ}}}\markboth{ftɕhoʁftɕhoʁ}{}\classe{idph.2}
\begin{définition}\fra petit et dressé\end{définition}
\begin{définition}\cmn 形容小的物体(如耳朵)立起来的样子
\end{définition}
\begin{exemple}\jya qala ɯ-rna ftɕhoʁftɕhoʁ ʑo ɲɯ-pa\cmn 兔子的耳朵是翘起来的\end{exemple}\end{entrée}

\begin{entrée}
\vedette{\hypertarget{Ⓔftɕhur}{\papi{ ftɕhur}}}\markboth{ftɕhur}{}
\classe{vt}\acception{1}
\paradigme{\textit{dir :} \jya tɤ-}
\paradigme{\textit{dir :} \jya lɤ-}
\begin{définition}\fra relever, mettre verticalement\end{définition}
\begin{définition}\cmn 竖起来;立起来\end{définition}
\begin{exemple}\jya ɯʑo kɯ ta-ftɕhur\cmn 他(把那个东西)立起来了\end{exemple}
\begin{exemple}\jya kɯki laʁdɯn ki pjɤ-ndʐaβ tɕe tu-ftɕhur-a\cmn 这个工具倒了,我把它立起来\end{exemple}
\begin{exemple}\jya ɕoŋtɕa lo-ftɕhur-a\cmn 我把木料竖起来了\end{exemple}
\begin{exemple}\jya khɯna kɯ ɯ-rna to-ftɕhur (=to-znɯndzɯ) ʑo tɕe ɲɯ-sɤŋo\cmn 狗把耳朵竖起来听\end{exemple}
\begin{relation-sémantique}\synonyme{
\hyperlink{Ⓔznɯndzɯ}{\textit{ \papi{znɯndzɯ}}}
}\end{relation-sémantique}\acception{2}
\paradigme{\textit{dir :} \jya pɯ-}
\begin{définition}\fra verser complètement\end{définition}
\begin{définition}\cmn 倒干\end{définition}
\begin{exemple}\jya kɯki ɯ-ro ci ɣɤʑu tɕe, pjɯ-ftɕhur-a tɕe kɤ-tshi\cmn 剩了一点,我把他倒干了,你喝吧\end{exemple}
\begin{exemple}\jya mɯ-to-rɯdzaŋspa-a tɕe, tʂha ɯ-ro nɯ pjɤ-ftɕhur-a\cmn 我不小心把剩下的茶倒干了\end{exemple}
\begin{exemple}\jya laʁjɯɣ to-ftɕhur\cmn 他把棍子立起来了\end{exemple}
\begin{relation-sémantique}\synonyme{
\hyperlink{Ⓔznɯndzɯ}{\textit{ \papi{znɯndzɯ}}}
}\end{relation-sémantique}\end{entrée}

\begin{entrée}
\vedette{\hypertarget{Ⓔftɕit}{\papi{ ftɕit}}}\markboth{ftɕit}{}\classe{vt}
\paradigme{\textit{dir :} \jya pɯ-}\acception{1}
\begin{définition}\fra prendre en charge\end{définition}
\begin{définition}\cmn 掌管\end{définition}
\begin{exemple}\jya ki kha ki nɤʑo pɯ-nɯ-ftɕit tɕe sɤcɯ tɤ-nɯ-ndɤm\cmn 这套房子由你来掌管,钥匙就交给你了\end{exemple}\acception{2}
\begin{définition}\fra prendre le contrôle de\end{définition}
\begin{définition}\cmn 霸占\end{définition}\end{entrée}

\begin{entrée}
\vedette{\hypertarget{Ⓔftɕɯm}{\papi{ ftɕɯm}}}\markboth{ftɕɯm}{}\classe{vt}\acception{1}
\begin{définition}\fra digérer\end{définition}
\begin{définition}\cmn 消化(食物)\end{définition}\acception{2}
\begin{définition}\fra apprivoiser\end{définition}
\begin{définition}\cmn 驯服
\begin{déclaration} \étymologie{\papi{btɕom}}\end{déclaration}\end{définition}
\begin{exemple}\jya jla ki nɯ kóʁmɯz kɤ-sɯxɕɤt ɲɯ-ɕti tɕe kɤ-ftɕɯm ɲɯ-ɴqa\cmn 这头犏牛刚刚开始驯养,很难驯服\end{exemple}\begin{sous-entrée}
\vedette{\hypertarget{}{\papi{ sɯftɕɯm}}}\markboth{sɯftɕɯm}{}\classe{vt}
\begin{définition}\ 
\begin{déclaration}\grammar{habil}\end{déclaration}\end{définition}
\begin{définition}\fra être capable de digérer, d'assimiler (un médicament)\end{définition}
\begin{définition}\cmn 消化得了\end{définition}
\begin{exemple}\jya smɤn ɲɯ-sna tɕe mɯ́j-sɯftɕɯm-a\cmn 药量过多,我吸收不了\end{exemple}
\end{sous-entrée}\end{entrée}

\begin{entrée}
\vedette{\hypertarget{Ⓔftɕɯpa}{\papi{ ftɕɯpa}}}\markboth{ftɕɯpa}{}\classe{n}
\begin{définition}\fra dixième mois\end{définition}
\begin{définition}\cmn 十月
\begin{déclaration} \étymologie{\papi{btɕu.pa}}\end{déclaration}\end{définition}
\end{entrée}

\begin{entrée}
\vedette{\hypertarget{Ⓔftɕɯʁɲiz}{\papi{ ftɕɯʁɲiz}}}\markboth{ftɕɯʁɲiz}{}\classe{n}
\begin{définition}\fra douzième mois\end{définition}
\begin{définition}\cmn 十二月
\begin{déclaration} \étymologie{\papi{btɕu.gɲis.pa}}\end{déclaration}\end{définition}\end{entrée}

\begin{entrée}
\vedette{\hypertarget{Ⓔftɕɯt}{\papi{ ftɕɯt}}}\markboth{ftɕɯt}{}\classe{vt}
\paradigme{\textit{dir :} \jya pɯ-}\acception{1}
\begin{définition}\fra diriger, régner sur\end{définition}
\begin{définition}\cmn 统治\end{définition}
\begin{exemple}\jya rɟɤlpu lo-ndo tɕe sɤtɕha pjɤ-ftɕɯt pjɤ-cha\cmn 他登上王位之后统治了国家\end{exemple}\acception{2}
\begin{définition}\fra être en charge de\end{définition}
\begin{définition}\cmn 掌管\end{définition}
\begin{exemple}\jya ki kɯm ki, nɤʑo sɤcɯ tɤ-ndɤm tɕe pɯ-nɯ-ftɕɯt\cmn 你拿着门的钥匙来掌管(这个房间)\end{exemple}\end{entrée}

\begin{entrée}
\vedette{\hypertarget{Ⓔftɕɯχtɕɯɣ}{\papi{ ftɕɯχtɕɯɣ}}}\markboth{ftɕɯχtɕɯɣ}{}\classe{n}
\begin{définition}\fra onzième mois\end{définition}
\begin{définition}\cmn 十一月
\begin{déclaration} \étymologie{\papi{btɕu.gtɕig.pa}}\end{déclaration}\end{définition}
\end{entrée}

\begin{entrée}
\vedette{\hypertarget{Ⓔfte}{\papi{ fte}}}\markboth{fte}{}\classe{vs}
\paradigme{\textit{dir :} \jya nɯ-}\acception{1}
\begin{définition}\fra émacié et livide\end{définition}
\begin{définition}\cmn 憔悴,没有血色\end{définition}
\begin{exemple}\jya ɯ-kɯ-mŋɤm pjɤ-thɯ tɕe ɯ-rŋa ra ɲɤ-fte ʑo\cmn 他的病严重了,脸上没有血色了\end{exemple}\acception{2}
\begin{définition}\fra se décolorer (habits)\end{définition}
\begin{définition}\cmn 褪色(衣服)\end{définition}
\begin{exemple}\jya ɯ-ŋga khro to-ŋga tɕe ɲɤ-fte ʑo\cmn 他衣服穿了很久,都褪色了\end{exemple}\end{entrée}

\begin{entrée}
\vedette{\hypertarget{Ⓔftsaʁra}{\papi{ ftsaʁra}}}\markboth{ftsaʁra}{}\classe{n}
\begin{définition}\fra plaque de pierre pour empêcher l'eau du toit de couler\end{définition}
\begin{définition}\cmn 用以挡住屋檐漏水的石板或木板\end{définition}\end{entrée}

\begin{entrée}
\vedette{\hypertarget{Ⓔftsɤn}{\papi{ ftsɤn}}}\markboth{ftsɤn}{}
\classe{vs}
\begin{définition}\fra sévère\end{définition}
\begin{définition}\cmn 严格
\begin{déclaration} \étymologie{\papi{btsan}}\end{déclaration}\end{définition}
\begin{exemple}\jya khɯna kɯ-ftsɤn ci ɲɯ-ŋu\cmn 是一条很凶的狗(看家的)\end{exemple}
\begin{exemple}\jya sɤcɯ ɲɯ-ftsɤn\cmn 钥匙很可靠(锁不容易被人家打开)\end{exemple}
\begin{exemple}\jya sɤtɕha ɲɯ-ftsɤn\cmn 这个地方安全\end{exemple}
\begin{exemple}\jya @laoshi kɯ-ftsɤn ci nɯ-atɯɣ-i\cmn 我们遇到了一位严格的老师\end{exemple}
\begin{relation-sémantique}\synonyme{
\hyperlink{Ⓔrkɤl}{\textit{ \papi{rkɤl}}}
}\end{relation-sémantique}\end{entrée}

\begin{entrée}
\vedette{\hypertarget{Ⓔftsɤnbu}{\papi{ ftsɤnbu}}}\markboth{ftsɤnbu}{}\classe{n}
\begin{définition}\fra force\end{définition}
\begin{définition}\cmn 强制性的办法
\begin{déclaration} \étymologie{\papi{btsan.po}}\end{déclaration}\end{définition}
\begin{exemple}\jya kɤ-nɤɲɟɯɲɟu mɯ-mɤ-ɲɯ-khɯ nɤ, ftsɤnbu tú-wɣ-βzu ɲɯ-ɬoʁ\cmn 软的不行,就要来硬的\end{exemple}\end{entrée}

\begin{entrée}
\vedette{\hypertarget{Ⓔftshi}{\papi{ ftshi}}}\markboth{ftshi}{}\classe{vs}
\paradigme{\textit{dir :} \jya tɤ-}\acception{1}
\begin{définition}\fra bon à rien, qui ne vaut rien\end{définition}
\begin{définition}\cmn 不可靠,没有良心(人)\end{définition}
\begin{exemple}\jya jiɕqha nɯ ʁo ɲɯ-ftshi ɕti\cmn 那个倒不怎么好\end{exemple}
\begin{exemple}\jya laχtɕha ɲɯ-ftshi ɕti\cmn 那个东西质量不好\end{exemple}
\begin{exemple}\jya tɯrme ɲɯ-ftshi ɕti\cmn 那个人不可靠\end{exemple}\acception{2}
\begin{définition}\fra aller mieux\end{définition}
\begin{définition}\cmn 变好;减轻\end{définition}
\begin{exemple}\jya smɤnba ɯ-thɯrʑi kɯ, ɯ-kɯ-mŋɤm to-ftshi\cmn 多亏医生,他的病减轻了\end{exemple}
\begin{relation-sémantique}\synonyme{
\hyperlink{ⒺtʂaʁⒽ1}{\textit{ \papi{tʂaʁ}}}
}\end{relation-sémantique}
\begin{relation-sémantique}\synonyme{
\hyperlink{Ⓔmna}{\textit{ \papi{mna}}}
}\end{relation-sémantique}
\begin{sous-entrée}
\vedette{\hypertarget{}{\papi{ ɣɤftshi}}}\markboth{ɣɤftshi}{}\classe{vt}
\paradigme{\textit{dir :} \jya tɤ-}
\begin{définition}\fra faire aller mieux\end{définition}
\begin{définition}\cmn 令……减轻\end{définition}
\begin{exemple}\jya smɤn nɯ kɯ a-kɯ-mŋɤm to-ɣɤftshi pjɤ-cha\cmn 服了药我的病减轻了许多\end{exemple}
\end{sous-entrée}\begin{sous-entrée}
\vedette{\hypertarget{}{\papi{ sɯftshi}}}\markboth{sɯftshi}{}\classe{vt}
\begin{définition}\fra forcer\end{définition}
\begin{définition}\cmn 逼迫
\begin{déclaration}\grammar{只用于否定式}\end{déclaration}\end{définition}
\begin{exemple}\jya ɯʑo kɯ mɤ́-wɣ-sɯftshi-a tɕe, ta-tɯt nɯ kɤ-fse ɬoʁ\cmn 他会逼迫我的,我只好照他说地去办\end{exemple}
\begin{relation-sémantique}\confer{
\hyperlink{Ⓔmɤkɯftshi}{\textit{ \papi{mɤkɯftshi}}}
}\end{relation-sémantique}
\end{sous-entrée}\end{entrée}

\begin{entrée}
\vedette{\hypertarget{Ⓔftsoʁ}{\papi{ ftsoʁ}}}\markboth{ftsoʁ}{}
\classe{n}
\begin{définition}\fra femelle d'hybride de yak et de vache\end{définition}
\begin{définition}\cmn 母犏牛\end{définition}\end{entrée}

\begin{entrée}
\vedette{\hypertarget{Ⓔftsoʁdo}{\papi{ ftsoʁdo}}}\markboth{ftsoʁdo}{}\classe{n}
\begin{définition}\fra vieille femelle de yak hybride\end{définition}
\begin{définition}\cmn 老母犏牛\end{définition}\end{entrée}

\begin{entrée}
\vedette{\hypertarget{Ⓔftsɯɣ}{\papi{ ftsɯɣ}}}\markboth{ftsɯɣ}{}
\classe{vt}\acception{1}
\paradigme{\textit{dir :} \jya kɤ-}
\begin{définition}\fra établir (une organisation)\end{définition}
\begin{définition}\cmn 成立;建立(组织)\end{définition}
\begin{exemple}\jya jiʑo kɤ-ftsɯɣ-i\cmn 我们建立了(某种组织)\end{exemple}
\begin{exemple}\jya ʑara kɯ ka-ftsɯɣ-nɯ\cmn 他们建立了(某种组织)\end{exemple}
\begin{exemple}\jya tɯtɯrca kɤ-ftsɯɣ-i\cmn 我们一起建立了(某种组织)\end{exemple}
\begin{exemple}\jya @nonghui kɤ-ftsɯɣ-i\cmn 我们建立了农会\end{exemple}
\begin{exemple}\jya @renmin @gongshe kɤ-ftsɯɣ-i\cmn 我们建立了人民公社\end{exemple}\acception{2}
\paradigme{\textit{dir :} \jya tɤ-}
\begin{définition}\fra empiler des pierres pour faire une marque\end{définition}
\begin{définition}\cmn 把石头立起来做标记
\begin{déclaration} \étymologie{\papi{btsug}}\end{déclaration}\end{définition}
\begin{exemple}\jya zgoku mɤɕtʂa tɤ-ari-a ri, a-χti kɤ-mto maŋe tɕe, thu χsɯm ta-ftsɯɣ-a tɕe pɯ-nɯɣi-a\cmn 我走到山顶去接丈夫,但见不到他回来,我立了三块石头做标记就回来了\end{exemple}\end{entrée}

\begin{entrée}
\vedette{\hypertarget{Ⓔftsɯr}{\papi{ ftsɯr}}}\markboth{ftsɯr}{}
\classe{vt}
\paradigme{\textit{dir :} \jya pɯ-}\acception{1}
\begin{définition}\fra essorer\end{définition}
\begin{définition}\cmn 拧干\end{définition}
\begin{exemple}\jya pɯ-ftsɯr-a\cmn 我拧干了\end{exemple}
\begin{exemple}\jya ɯʑo kɯ pa-ftsɯr\cmn 他拧干了\end{exemple}
\begin{exemple}\jya kɯki ɲɯ-ɤci tɕe, pɯ-ftsɯr\cmn 这个湿了,你把它拧干吧\end{exemple}
\begin{exemple}\jya tɯ-ŋga ki ɲɤ-k-ɤci-ci, tɕe pɯ-ftsɯr\cmn 这件衣服湿了,你把它拧干吧\end{exemple}\acception{2}
\begin{définition}\fra vider de son eau\end{définition}
\begin{définition}\cmn 倒干;让水流干
\begin{déclaration} \étymologie{\papi{btsir}}\end{déclaration}\end{définition}\end{entrée}

\begin{entrée}
\vedette{\hypertarget{Ⓔftʂi}{\papi{ ftʂi}}}\markboth{ftʂi}{}
\classe{vt}
\paradigme{\textit{dir :} \jya nɯ-}
\paradigme{\textit{dir :} \jya thɯ-}
\begin{définition}\fra faire fondre\end{définition}
\begin{définition}\cmn 使融化\end{définition}
\begin{exemple}\jya ta-mar kɤ-tʂi ɲɯ-ra\cmn 要把酥油融化掉\end{exemple}
\begin{exemple}\jya ta-mar nɯ-ftʂi-t-a\cmn 我把酥油融化了\end{exemple}
\begin{exemple}\jya tɤjpɣom na-ftʂi\cmn 他把冰融化了\end{exemple}
\begin{relation-sémantique}\synonyme{
\hyperlink{Ⓔsɯɣndʐi}{\textit{ \papi{sɯɣndʐi}}}
}\end{relation-sémantique}
\begin{relation-sémantique}\confer{
\hyperlink{Ⓔndʐi}{\textit{ \papi{ndʐi}}}
}\end{relation-sémantique}\end{entrée}

\begin{entrée}
\vedette{\hypertarget{Ⓔftɯɣ}{\papi{ ftɯɣ}}}\markboth{ftɯɣ}{}\classe{vi}
\paradigme{\textit{dir :} \jya pɯ-}
\begin{définition}\fra être accompli\end{définition}
\begin{définition}\cmn 完工\end{définition}
\begin{exemple}\jya kɤ-nɤma mɤ-kɯ-ftɯɣ nɯ kɤ-ɕɯftɯɣ ra\cmn 要完成没有完成的工作\end{exemple}
\begin{relation-sémantique}\confer{
\hyperlink{Ⓔɕɯftɯɣ}{\textit{ \papi{ɕɯftɯɣ}}}
}\end{relation-sémantique}\end{entrée}

\begin{entrée}
\vedette{\hypertarget{ⒺftɯlⒽ1}{\papi{ ftɯl}}}\markboth{ftɯl}{}\homonyme{1}\classe{vt}
\paradigme{\textit{dir :} \jya pɯ-}
\begin{définition}\fra apprivoiser\end{définition}
\begin{définition}\cmn 驯服
\begin{déclaration} \étymologie{\papi{btul}}\end{déclaration}\end{définition}
\begin{exemple}\jya jla pɯ-ftɯl-a\cmn 我驯服了犏牛\end{exemple}
\begin{exemple}\jya ɯʑo kɯ mbro pa-ftɯl\cmn 他驯服了马\end{exemple}
\begin{relation-sémantique}\synonyme{
\hyperlink{Ⓔftɕɤt}{\textit{ \papi{ftɕɤt}}}
}\end{relation-sémantique}\begin{sous-entrée}
\vedette{\hypertarget{}{\papi{ nɯɣɯftɯl}}}\markboth{nɯɣɯftɯl}{}\classe{vi}
\begin{définition}\ 
\begin{déclaration}\grammar{facil}\end{déclaration}\end{définition}
\begin{définition}\fra facile à apprivoiser\end{définition}
\begin{définition}\cmn 容易驯服
\end{définition}
\begin{relation-sémantique}\confer{
\hyperlink{ⒺndɯlⒽ1}{\textit{ \papi{ndɯl1}}}
}\end{relation-sémantique}
\end{sous-entrée}\end{entrée}

\begin{entrée}
\vedette{\hypertarget{ⒺftɯlⒽ2}{\papi{ ftɯl}}}\markboth{ftɯl}{}\homonyme{2}\classe{vi}
\paradigme{\textit{dir :} \jya nɯ-}
\begin{définition}\fra digérer\end{définition}
\begin{définition}\cmn 消化\end{définition}
\begin{exemple}\jya kɤ-ndza kɤ-ftɯl mɯ́j-cha\cmn 不能消化食物\end{exemple}
\begin{exemple}\jya kɯki kɤ-ndza ki a-mɤ-nɯ-tɕhom ra ma kɤ-ftɯl mɯ́j-sɤcha\cmn 这些食物不要吃太多,不然就没有办法消化\end{exemple}\end{entrée}

\newpage\caractère{g}

\begin{entrée}
\vedette{\hypertarget{Ⓔgaŋgaŋ}{\papi{ gaŋgaŋ}}}\markboth{gaŋgaŋ}{}
\classe{idph.2}
\begin{définition}\fra haut et imposant\end{définition}
\begin{définition}\cmn 形容高而大的样子\end{définition}
\begin{exemple}\jya ɯ-phoŋbu wxti gaŋgaŋ ʑo pa\cmn 他的身体又高又大\end{exemple}\end{entrée}

\begin{entrée}
\vedette{\hypertarget{Ⓔgɤgɤɣ}{\papi{ gɤgɤɣ}}}\markboth{gɤgɤɣ}{}
\classe{idph.2}
\begin{définition}\fra courbé\end{définition}
\begin{définition}\cmn 形容不灵活,站得不稳的样子\end{définition}
\begin{exemple}\jya tɯ-ŋga ɕ-tɤ-ɕkho-t-a ri, ɲɯ-ɤqajpɣom tɕe gɤgɤɣ ʑo ɲɤ-pa\cmn 我去晒了衣服,冻到了就变硬了\end{exemple}
\begin{exemple}\jya ki tɤ-wɯ ki chɤ-rgɤz tɕe, gɤgɤɣ ʑo ɲɯ-pa\cmn 这位老年人变老了,站得不稳\end{exemple}
\begin{exemple}\jya ɯʑo lo-βzi tɕe, gɤgɤɣ ʑo ɲɯ-ndzur\cmn 他喝醉了,站不稳\end{exemple}\begin{sous-entrée}
\vedette{\hypertarget{}{\papi{ gɤɣnɤgɤɣ}}}\markboth{gɤɣnɤgɤɣ}{}\classe{idph.3}
\begin{exemple}\jya gɤgɤɣ nɤ gɤgɤɣ kɤ-anɯri\cmn 他很不灵活地去了\end{exemple}
\end{sous-entrée}\begin{sous-entrée}
\vedette{\hypertarget{}{\papi{ gɤɣɯɣi}}}\markboth{gɤɣɯɣi}{}\classe{idph.6}
\begin{exemple}\jya rgɤtpu gɤɣɯɣi ʑo kɤ-ŋke ɲɯ-cha\cmn 老年人只能走一点点\end{exemple}
\end{sous-entrée}\end{entrée}

\begin{entrée}
\vedette{\hypertarget{Ⓔglɤɣglɤɣ}{\papi{ glɤɣglɤɣ}}}\markboth{glɤɣglɤɣ}{}
\classe{idph.2}
\begin{définition}\fra pressé très fort\end{définition}
\begin{définition}\cmn 压得很紧\end{définition}
\begin{exemple}\jya glɤɣglɤɣ ʑo ɲɯ-nɯ-rŋgɯ\cmn 他睡得熟,动也不动\end{exemple}
\begin{exemple}\jya glɤɣglɤɣ ʑo pjɤ-sɯ-ɲcɤr\cmn 压得很紧\end{exemple}
\begin{exemple}\jya mbrɤz tɯfkur tɤ-fkur-a ɯ-tɯ-rʑi glɤɣglɤɣ ʑo ɲɯ-pa\cmn 我背了一袋大米,很重地压着我\end{exemple}
\begin{relation-sémantique}\confer{
\hyperlink{Ⓔɣɤglɤglɤɣ}{\textit{ \papi{ɣɤglɤglɤɣ}}}
}\end{relation-sémantique}\end{entrée}

\begin{entrée}
\vedette{\hypertarget{Ⓔglinɤgli}{\papi{ glinɤgli}}}\markboth{glinɤgli}{}\classe{idph.3}
\begin{définition}\fra craquement d'os\end{définition}
\begin{définition}\cmn 形容(骨头)互相摩擦发出的声音\end{définition}
\begin{exemple}\jya a-mke ɯ-ɕɤrɯ ɲɯ-ɤndʑɯgli ɲɯ-ŋu tɕe glinɤgli ʑo ɲɯ-ti\cmn 我脖子的骨头互相摩擦发出咯咯声\end{exemple}
\begin{relation-sémantique}\confer{
\hyperlink{Ⓔadʑɯgli}{\textit{ \papi{adʑɯgli}}}
}\end{relation-sémantique}\end{entrée}

\begin{entrée}
\vedette{\hypertarget{Ⓔgoʁ}{\papi{ goʁ}}}\markboth{goʁ}{}\classe{idph.1}
\begin{définition}\fra tout d'un coup (s'agenouiller)\end{définition}
\begin{définition}\cmn 一下子(跪下)\end{définition}
\begin{exemple}\jya ɯ-χpɯm goʁ ʑo pjɤ-tshoʁ\cmn 他一下子跪下了(很恭敬的样子)\end{exemple}
\begin{relation-sémantique}\synonyme{
\hyperlink{Ⓔdzoʁ}{\textit{ \papi{dzoʁ}}}
}\end{relation-sémantique}
\begin{relation-sémantique}\synonyme{
\hyperlink{Ⓔzgoʁ}{\textit{ \papi{zgoʁ}}}
}\end{relation-sémantique}\end{entrée}

\begin{entrée}
\vedette{\hypertarget{Ⓔgraŋgraŋ}{\papi{ graŋgraŋ}}}\markboth{graŋgraŋ}{}\classe{idph.2}
\begin{définition}\fra grand et mince\end{définition}
\begin{définition}\cmn 形容又高又瘦的样子(人)\end{définition}
\begin{exemple}\jya aʑo a-slamaχti ci tɤ-tɕɯ tu tɕe, kɯ-mbro ci graŋgraŋ ʑo ŋu\cmn 我有个男同学,个子又高又大\end{exemple}\end{entrée}

\begin{entrée}
\vedette{\hypertarget{Ⓔgrɯβgrɯβ}{\papi{ grɯβgrɯβ}}}\markboth{grɯβgrɯβ}{}
\classe{n}
\begin{définition}\fra matsutake\end{définition}
\begin{définition}\cmn 松茸\end{définition}
\begin{exemple}\jya grɯβgrɯβ nɯ ɕkrɤz wuma ʑo kɯ-wxti kɯ-ʁjɤr ɯ-ŋgɯ tu-ɬoʁ ŋu. grɯβgrɯβ nɯ ɯ-mgɯr ɯ-qhu nɯ kɯ-qandʐi tsa kɯ-wɣrum tsa ŋu. ɯ-rʑɯɣ pɕoʁ cho ɯ-ru nɯ kɯ-wɣrum ŋu, ɯ-dɯχɯn wuma ʑo tu; kɤ-ndza mɯm wuma ʑo aɣɯsmɤn, wuma ʑo ɯ-koŋ kɯ-wxti tɤjmɤɣ ɲɯ-ŋu\cmn 松茸长在茂密的大青冈树林中,它背面带有乌色和白色,菌褶部分和干都是白色的,香味很浓,很好吃,可以入药,是很贵重的菌种。\end{exemple}\end{entrée}

\begin{entrée}
\vedette{\hypertarget{Ⓔgrɯβgrɯβftsa}{\papi{ grɯβgrɯβftsa}}}\markboth{grɯβgrɯβftsa}{}\classe{n}
\begin{définition}\fra une espèce de champignon\end{définition}
\begin{définition}\cmn 一种菌子\end{définition}
\begin{exemple}\jya grɯβgrɯβftsa nɯ ɯ-tshɯɣa cho ɯ-mdoʁ ɯ-sɤɣɬoʁ nɯ ra grɯβgrɯβ cho wuma ʑo naχtɕɯɣ, tɕeri grɯβgrɯβ kɯ-fse ɯ-dɯχɯn me, sɤndɤɣ kɤ-ndza mɤ-sna, kɯ-thɯ nɯ ra pjɯ-si ɲɯ-ŋgrɤl\cmn 
\stylefv{grɯβgrɯβftsa}的形状、颜色和生长的地方都和松茸一样,但是没有松茸那样的香气。有毒,不能吃。中毒严重的话会导致死亡。
\end{exemple}\end{entrée}

\begin{entrée}
\vedette{\hypertarget{Ⓔgrɯɣgrɯɣ}{\papi{ grɯɣgrɯɣ}}}\markboth{grɯɣgrɯɣ}{}\classe{idph.2}
\begin{définition}\fra immobile, très serré\end{définition}
\begin{définition}\cmn 动也不动,绷得很紧\end{définition}
\begin{exemple}\jya tɯmbri grɯɣgrɯɣ ʑo ɲɯ-ɤsɯ-ndo\cmn 他使劲地抓住绳子不放\end{exemple}
\begin{exemple}\jya jla grɯɣgrɯɣ ɯ-ɕna ɲɯ-ɤsɯ-ndo tɕe mɯ́j-ɕlɯɣ\cmn 他使劲地抓住犏牛的鼻绳不放\end{exemple}
\begin{exemple}\jya grɯɣgrɯɣ ʑo ko-βraʁ\cmn 拴得很牢\end{exemple}\begin{sous-entrée}
\vedette{\hypertarget{}{\papi{ grɯɣnɤgrɯɣ}}}\markboth{grɯɣnɤgrɯɣ}{}\classe{idph.3}\acception{1}
\begin{définition}\fra en se pressant\end{définition}
\begin{définition}\cmn 时间抓得很紧\end{définition}
\begin{exemple}\jya ta-ma grɯɣnɤgʁɯɣ ʑo tu-nɤme ɲɯ-ŋu\cmn 他把工作时间抓得很紧\end{exemple}\acception{2}
\begin{définition}\fra se gratter sans arrêt\end{définition}
\begin{définition}\cmn 一次又一次地抠\end{définition}
\begin{exemple}\jya a-mgɯr ʑo grɯɣnɤgrɯɣ nɯ-rɤβraʁ-a\cmn 我一次又一次地抠背\end{exemple}\acception{3}
\begin{définition}\fra bruit émit par la meule\end{définition}
\begin{définition}\cmn 形容推磨的声音\end{définition}
\begin{exemple}\jya mbrɯtɕɯ grɯɣnɤgrɯɣ ʑo chɤ-fse\cmn 他使劲地磨刀\end{exemple}
\end{sous-entrée}\end{entrée}

\begin{entrée}
\vedette{\hypertarget{Ⓔgrɯŋgrɯŋ}{\papi{ grɯŋgrɯŋ}}}\markboth{grɯŋgrɯŋ}{}
\classe{idph.2}\acception{1}
\begin{définition}\fra propre, utilisé jusqu'au bout\end{définition}
\begin{définition}\cmn 干净;用完\end{définition}
\begin{exemple}\jya grɯŋgrɯŋ tu-orɕo\cmn 完全用完了\end{exemple}
\begin{exemple}\jya ji-tsha thɯ-arɕo grɯŋgrɯŋ ʑo\cmn 我们的盐用完了\end{exemple}\acception{2}
\begin{définition}\fra ne pas perdre de temps et bien faire son travail\end{définition}
\begin{définition}\cmn 时间安排得很紧凑;工作进行得很塌实;把线拉得很紧\end{définition}
\begin{exemple}\jya phɯntshoʁ rcanɯ grɯɣgrɯɣ kɯ-pa ku-rɤʑi ɕti, tɤ-rʑaʁ ra kɯmɤlɤxso maka mɤ-sɯxɕe, kɤ-nɤma ra tshɯntshɯn tu-ste ɕti\cmn 朋措把时间抓得很紧,从不浪费,工作也做得很好\end{exemple}
\begin{relation-sémantique}\confer{
\hyperlink{Ⓔkhrɯŋkhrɯŋ}{\textit{ \papi{khrɯŋkhrɯŋ}}}
}\end{relation-sémantique}
\begin{relation-sémantique}\confer{
\hyperlink{Ⓔχɤlχɤl}{\textit{ \papi{χɤlχɤl}}}
}\end{relation-sémantique}\end{entrée}

\begin{entrée}
\vedette{\hypertarget{Ⓔgɯgɯɣ}{\papi{ gɯgɯɣ}}}\markboth{gɯgɯɣ}{}\classe{idph.2}
\begin{définition}\fra ciel très noir\end{définition}
\begin{définition}\cmn 形容天色很黑的样子\end{définition}
\end{entrée}

\begin{entrée}
\vedette{\hypertarget{Ⓔgɯɣnɤgɯɣ}{\papi{ gɯɣnɤgɯɣ}}}\markboth{gɯɣnɤgɯɣ}{}
\classe{idph.3}
\begin{définition}\fra émettant un bruit grave rythmique\end{définition}
\begin{définition}\cmn 发出有节奏的钝音\end{définition}
\begin{exemple}\jya gɯɣnɤgɯɣ pa-xtsɯ\cmn 一阵又一阵地砸了\end{exemple}\end{entrée}

\newpage\caractère{ɣ}

\begin{entrée}
\vedette{\hypertarget{Ⓔɣa}{\papi{ ɣa}}}\markboth{ɣa}{}\classe{adv}
\begin{définition}\fra oui\end{définition}
\begin{définition}\cmn 是的\end{définition}\end{entrée}

\begin{entrée}
\vedette{\hypertarget{Ⓔɣar}{\papi{ ɣar}}}\markboth{ɣar}{}
\classe{vi}
\paradigme{\textit{dir :} \jya nɯ-}\acception{1}
\begin{définition}\fra devenir sauvage\end{définition}
\begin{définition}\cmn 变野\end{définition}
\begin{exemple}\jya ɯʑo ɲo-ɣar\cmn 它(例如猫)变野了\end{exemple}
\begin{exemple}\jya ɯʑo nɯ-ɣar\cmn 他变野了\end{exemple}
\begin{exemple}\jya lɯlu nɯ kha ri pjɯ-χsu-j pɯ-ɕti ri, jɤ-anɯri tɕe ɲo-ɣar\cmn 我们以前家里养猫,但是它走了,变野了\end{exemple}\acception{2}
\begin{définition}\fra devenir anormal, marginal\end{définition}
\begin{définition}\cmn 变得不正常\end{définition}
\begin{exemple}\jya nɯ tɯrme nɯ kɯɕɯŋgɯ kɯ-pɯ-pe pjɤ-ɕti ri, nɯ ɯ-qhu tɕe ɲo-ɣar\cmn 那个人原来好好的,后来就变得不正常。\end{exemple}\acception{3}
\begin{définition}\fra aller vivre tout seul dans un endroit sauvage\end{définition}
\begin{définition}\cmn 到野地去生活\end{définition}\end{entrée}

\begin{entrée}
\vedette{\hypertarget{Ⓔɣɤbɤbɤβ}{\papi{ ɣɤbɤbɤβ}}}\markboth{ɣɤbɤbɤβ}{}
\begin{relation-sémantique}\confer{
\hyperlink{Ⓔbɤbɤβ}{\textit{ \papi{bɤbɤβ}}}
}\end{relation-sémantique}\end{entrée}

\begin{entrée}
\vedette{\hypertarget{Ⓔɣɤbɤβlɤβ}{\papi{ ɣɤbɤβlɤβ}}}\markboth{ɣɤbɤβlɤβ}{}\classe{vi}
\begin{définition}\fra parler de façon grossière et incompréhensible\end{définition}
\begin{définition}\cmn 胡言乱语;语气粗大,说别人听不懂的话\end{définition}\end{entrée}

\begin{entrée}
\vedette{\hypertarget{Ⓔɣɤβdi}{\papi{ ɣɤβdi}}}\markboth{ɣɤβdi}{}
\paradigme{\textit{dir :} \jya tɤ-}
\begin{définition}\ 
\begin{déclaration}\grammar{caus}\end{déclaration}\end{définition}\acception{2}
\paradigme{\textit{dir :} \jya nɯ-}
\begin{définition}\fra réparer, bien organiser\end{définition}
\begin{définition}\cmn 修理;弄好\end{définition}
\begin{exemple}\jya ɯʑo kɯ ta-ɣɤβdi\cmn 他修了\end{exemple}
\begin{exemple}\jya kɯki laχtɕha ki mɯ-to-pe tɕe, tu-ɣɤβdi-a ɲɯ-ntshi\cmn 这个东西坏了,我要把它修理一下\end{exemple}
\begin{exemple}\jya mkhɯrlu to-ɣɤβdi tɕe to-ndʐɯm\cmn 机器修了以后,转得更快\end{exemple}
\begin{exemple}\jya nɤ-sɯm nɯ-nɯ-ɣɤβdi\cmn 你放心吧/你死了这条心吧\end{exemple}
\begin{exemple}\jya kɤ-rɤt mɯ-pjɤ-tɯ-ɣɤβdi-t\cmn 你写得不好\end{exemple}
\begin{exemple}\jya a-kha kɤ-ɣɤβdi tɤ-jɤɣ\cmn 我的房子修完了\end{exemple}\acception{2}
\paradigme{\textit{dir :} \jya nɯ-}
\begin{définition}\fra castrer (verrat)\end{définition}
\begin{définition}\cmn 阉割(公猪)
\end{définition}
\begin{relation-sémantique}\synonyme{
\hyperlink{Ⓔftɕɤz}{\textit{ \papi{ftɕɤz}}}
}\end{relation-sémantique}\classe{vt}\begin{sous-entrée}
\vedette{\hypertarget{}{\papi{ aβdoʁdi}}}\markboth{aβdoʁdi}{}\classe{vs}
\begin{définition}\fra en bonne santé, paisible\end{définition}
\begin{définition}\cmn 安宁(身体状况、生活条件)\end{définition}
\begin{exemple}\jya jiʑora ɕɤfɕo ku-oβdoʁβdi-j\cmn 我们最近(这几年)身体没有异常现象\end{exemple}
\end{sous-entrée}\begin{sous-entrée}
\vedette{\hypertarget{}{\papi{ ɣɤβdoʁdi}}}\markboth{ɣɤβdoʁdi}{}\classe{vt}
\paradigme{\textit{dir :} \jya tɤ-}
\begin{définition}\fra mettre en ordre\end{définition}
\begin{définition}\cmn 整理\end{définition}
\begin{exemple}\jya ta-ɣɤβdoʁβdi\cmn 他整理了\end{exemple}
\begin{exemple}\jya kɤ-ɣɤβdoʁβdi ra\cmn 要整理\end{exemple}
\begin{exemple}\jya nɤ-laχtɕha ra tɤ-rɤwum tɕe ɯ-rtsɯɣ ra tɤ-ɣɤβdoʁβdi\cmn 把你的东西收拾一下,把它们堆整齐\end{exemple}
\end{sous-entrée}\begin{sous-entrée}
\vedette{\hypertarget{}{\papi{ ʑɣɤɣɤβdi}}}\markboth{ʑɣɤɣɤβdi}{}\classe{vi}
\begin{définition}\ 
\begin{déclaration}\grammar{refl}\end{déclaration}
\begin{déclaration}\grammar{caus}\end{déclaration}\end{définition}
\begin{définition}\fra se soulager (en se massant, par exemple)\end{définition}
\begin{définition}\cmn 令自己好受一些(例如,揉自己的身体)\end{définition}
\begin{exemple}\jya tɯʑo kɤ-ʑɣɤɣɤβdi mɤ-sɤcha\cmn 不能令自己好受一些(因为太痛)\end{exemple}
\end{sous-entrée}\end{entrée}

\begin{entrée}
\vedette{\hypertarget{Ⓔɣɤβdoʁβdi}{\papi{ ɣɤβdoʁβdi}}}\markboth{ɣɤβdoʁβdi}{}
\begin{relation-sémantique}\confer{
\hyperlink{Ⓔɣɤβdi}{\textit{ \papi{ɣɤβdi}}}
}\end{relation-sémantique}\end{entrée}

\begin{entrée}
\vedette{\hypertarget{Ⓔɣɤβlo}{\papi{ ɣɤβlo}}}\markboth{ɣɤβlo}{}\classe{vs}
\paradigme{\textit{dir :} \jya nɯ-}
\begin{définition}\fra lent\end{définition}
\begin{définition}\cmn 慢\end{définition}
\begin{exemple}\jya kɤ-rɤma ɲɯ-tɯ-ɣɤβlo\cmn 你工作得很慢!\end{exemple}\begin{sous-entrée}
\vedette{\hypertarget{}{\papi{ zɣɤβlo}}}\markboth{zɣɤβlo}{}\classe{vt}
\paradigme{\textit{dir :} \jya nɯ-}
\begin{définition}\ 
\begin{déclaration}\grammar{caus}\end{déclaration}\end{définition}
\begin{définition}\fra ralentir\end{définition}
\begin{définition}\cmn 减慢\end{définition}
\begin{exemple}\jya kɤ-nɤma wuma ʑo ɲɯ-mbɣom ri, aʑo mɯ́j-cha-a tɕe nɯ-zɣɤβlo-t-a\cmn 工作很急,但是我不会做,所以耽误了时间\end{exemple}
\begin{relation-sémantique}\antonyme{
\hyperlink{Ⓔɣɤji}{\textit{ \papi{ɣɤji}}}
}\end{relation-sémantique}
\end{sous-entrée}\end{entrée}

\begin{entrée}
\vedette{\hypertarget{Ⓔɣɤβloʁβle}{\papi{ ɣɤβloʁβle}}}\markboth{ɣɤβloʁβle}{} (\variante{ɣɤβlɯβle}) 
\classe{vs}
\paradigme{\textit{dir :} \jya thɯ-}
\begin{définition}\fra maladroit\end{définition}
\begin{définition}\cmn 动作慢\end{définition}\end{entrée}

\begin{entrée}
\vedette{\hypertarget{Ⓔɣɤβlɯβlɯɣ}{\papi{ ɣɤβlɯβlɯɣ}}}\markboth{ɣɤβlɯβlɯɣ}{}
\classe{vi}
\paradigme{\textit{dir :} \jya tɤ-}
\begin{définition}\ 
\begin{déclaration}\grammar{deidph}\end{déclaration}\end{définition}
\begin{définition}\fra irisé\end{définition}
\begin{définition}\cmn 发光,显得耀眼\end{définition}
\begin{exemple}\jya tɤtʂu ɲɯ-ɣɤβlɯβlɯɣ\cmn 灯在发光\end{exemple}
\begin{exemple}\jya ɲɯ-nɤmbju ɲɯ-ɣɤβlɯβlɯɣ\cmn 在发光\end{exemple}
\begin{exemple}\jya ɯ-ŋga ɯ-tɕhɤz ɯ-tɯ-dɤn kɯ ɲɯ-ɣɤβlɯβlɯɣ ʑo\cmn 他衣服的彩色布料很多,显得很耀眼\end{exemple}\begin{sous-entrée}
\vedette{\hypertarget{}{\papi{ sɤβlɯβlɯɣ}}}\markboth{sɤβlɯβlɯɣ}{}\classe{vt}
\begin{exemple}\jya ɯ-ŋga ɯ-tɕhɤz ɲɯ-sɤβlɯβlɯɣ ʑo\cmn 他衣服的彩色布料穿着显得很耀眼\end{exemple}
\begin{relation-sémantique}\confer{
\hyperlink{Ⓔβlɯɣnɤβlɯɣ}{\textit{ \papi{βlɯɣnɤβlɯɣ}}}
}\end{relation-sémantique}
\end{sous-entrée}\end{entrée}

\begin{entrée}
\vedette{\hypertarget{Ⓔɣɤβzaʁlaʁ}{\papi{ ɣɤβzaʁlaʁ}}}\markboth{ɣɤβzaʁlaʁ}{}\classe{vi}
\begin{définition}\fra parler /se comporter sans se soucier de la situation\end{définition}
\begin{définition}\cmn 说话、动作不严谨(随便开口,不分场合)\end{définition}
\begin{relation-sémantique}\confer{
\hyperlink{Ⓔβzaʁlu}{\textit{ \papi{βzaʁlu}}}
}\end{relation-sémantique}\end{entrée}

\begin{entrée}
\vedette{\hypertarget{Ⓔɣɤβzi}{\papi{ ɣɤβzi}}}\markboth{ɣɤβzi}{}
\begin{relation-sémantique}\confer{
\hyperlink{Ⓔβzi}{\textit{ \papi{βzi}}}
}\end{relation-sémantique}\end{entrée}

\begin{entrée}
\vedette{\hypertarget{Ⓔɣɤcaʁcaʁ}{\papi{ ɣɤcaʁcaʁ}}}\markboth{ɣɤcaʁcaʁ}{}
\classe{vi}
\paradigme{\textit{dir :} \jya tɤ-}
\paradigme{\textit{dir :} \jya thɯ-}
\begin{définition}\fra bavard, racontant n'importe quoi\end{définition}
\begin{définition}\cmn 多嘴;乱说\end{définition}
\begin{exemple}\jya pɯ-ɣɤcaʁcaʁ-a\cmn 我以前多嘴\end{exemple}
\begin{exemple}\jya pɯ-tɯ-ɣɤcaʁcaʁ\cmn 你以前多嘴\end{exemple}
\begin{exemple}\jya ɯʑo pɯ-ɣɤcaʁcaʁ\cmn 他以前多嘴\end{exemple}
\begin{exemple}\jya jiɕqha nɯ ɯ-mtɕhi dɤn, tu-ɣɤcaʁcaʁ ntsɯ ŋu\cmn 刚才那个(人)爱多嘴乱说\end{exemple}\end{entrée}

\begin{entrée}
\vedette{\hypertarget{Ⓔɣɤcɤtcɤt}{\papi{ ɣɤcɤtcɤt}}}\markboth{ɣɤcɤtcɤt}{}
\classe{vi}
\paradigme{\textit{dir :} \jya tɤ-}\acception{1}
\begin{définition}\fra piailler (oiseau)\end{définition}
\begin{définition}\cmn 叫(鸟)\end{définition}\acception{2}
\begin{définition}\fra prendre la parole sans arrêt (enfant)\end{définition}
\begin{définition}\cmn 不停地插嘴(小孩子)\end{définition}
\begin{exemple}\jya ma-tɯ-ɣɤcɤtcɤt\cmn 你不要不停地插嘴\end{exemple}
\begin{exemple}\jya ɣɤcɤtcat-a\cmn 我不停地插嘴\end{exemple}
\end{entrée}

\begin{entrée}
\vedette{\hypertarget{Ⓔɣɤchi}{\papi{ ɣɤchi}}}\markboth{ɣɤchi}{}
\begin{relation-sémantique}\confer{
\hyperlink{Ⓔchi}{\textit{ \papi{chi}}}
}\end{relation-sémantique}\end{entrée}

\begin{entrée}
\vedette{\hypertarget{Ⓔɣɤchrɤβchrɤβ}{\papi{ ɣɤchrɤβchrɤβ}}}\markboth{ɣɤchrɤβchrɤβ}{}\classe{vi}\acception{1}
\begin{définition}\fra avoir la gorge enrouée\end{définition}
\begin{définition}\cmn 嗓子哑了\end{définition}
\begin{exemple}\jya ɲɯ-nɯtɕhomba tɕe, ɯ-rqo ɲɯ-ɣɤchrɤβchrɤβ\cmn 他感冒了,嗓子哑了\end{exemple}\acception{2}
\begin{définition}\fra émettre du bruit en roulant (petites pierres)\end{définition}
\begin{définition}\cmn 小石头滚下来发出声音\end{définition}
\begin{exemple}\jya praʁ ɲɯ-mbɯt ɲɯ-ɣɤchrɤβchrɤβ\cmn 悬崖塌下来了,小石头滚下来发出声音\end{exemple}\begin{sous-entrée}
\vedette{\hypertarget{}{\papi{ ɣɤchrɤβlɤβ}}}\markboth{ɣɤchrɤβlɤβ}{}
\begin{définition}\fra émettre un bruit de mucus dans la gorge\end{définition}
\begin{définition}\cmn (喉咙里)有痰的声音\end{définition}
\begin{exemple}\jya ɲɯ-tɯ-ɤɕqhe tɕe nɤ-rqo ɲɯ-ɣɤchrɤβlɤβ\cmn 你咳嗽,嗓子发出嘶哑声\end{exemple}
\end{sous-entrée}\begin{sous-entrée}
\vedette{\hypertarget{}{\papi{ sɤschrɤβlɤβ}}}\markboth{sɤschrɤβlɤβ}{}\classe{vt}\acception{1}
\begin{définition}\fra émettre un bruit de mucus dans la gorge\end{définition}
\begin{définition}\cmn 放出喉咙里有痰的声音\end{définition}
\begin{exemple}\jya nɤ-rqo ma-tɯ-sɤchrɤβlɤβ\cmn 你嗓子不要发出嘶哑声\end{exemple}\acception{2}
\begin{définition}\fra émettre du bruit en faisant s'entrechoquer des petits objets\end{définition}
\begin{définition}\cmn 令小东西互相碰撞,发出声音\end{définition}
\begin{exemple}\jya laχtɕha ɲɯ-sɤschrɤβlɤβ\cmn 他把很多小东西装在一起,拿起来的时候发出互相碰撞的声音\end{exemple}
\begin{relation-sémantique}\confer{
\hyperlink{Ⓔchrɤβchrɤβ}{\textit{ \papi{chrɤβchrɤβ}}}
}\end{relation-sémantique}
\end{sous-entrée}\end{entrée}

\begin{entrée}
\vedette{\hypertarget{Ⓔɣɤchrɤβlɤβ}{\papi{ ɣɤchrɤβlɤβ}}}\markboth{ɣɤchrɤβlɤβ}{}
\begin{relation-sémantique}\confer{
\hyperlink{Ⓔɣɤchrɤβchrɤβ}{\textit{ \papi{ɣɤchrɤβchrɤβ}}}
}\end{relation-sémantique}\end{entrée}

\begin{entrée}
\vedette{\hypertarget{Ⓔɣɤchɯchrɯɣ}{\papi{ ɣɤchɯchrɯɣ}}}\markboth{ɣɤchɯchrɯɣ}{} (\variante{ɣɤchrɯɣchrɯɣ}) 
\classe{vs}
\paradigme{\textit{dir :} \jya tɤ-}
\begin{définition}\fra bruit d'objets durs qui s'entrechoquent\end{définition}
\begin{définition}\cmn 很多硬的东西(铁链子的环子,骨头,小石头)互相摩擦和撞击发出的声音\end{définition}
\begin{exemple}\jya ʑɴɢɯloʁ tɤ-fkur-a ɲɯ-ɣɤchrɯchrɯɣ\cmn (口袋里装了核桃),我背起时,(核桃)相碰撞发出的声音\end{exemple}\end{entrée}

\begin{entrée}
\vedette{\hypertarget{Ⓔɣɤcraŋlaŋ}{\papi{ ɣɤcraŋlaŋ}}}\markboth{ɣɤcraŋlaŋ}{}
\classe{vi}
\begin{définition}\fra crier très fort\end{définition}
\begin{définition}\cmn 高声喧哗;大声地叫\end{définition}
\begin{exemple}\jya khɯna ɲɯ-ɣɤcraŋlaŋ\cmn 狗在大声地叫\end{exemple}
\begin{exemple}\jya paʁ ɲɯ-ɣɤcraŋlaŋ\cmn 猪在大声地叫\end{exemple}
\begin{exemple}\jya ɯ-tɯ-ɣɤcraŋlaŋ kɯ kɤ-rɤβzjoz koŋla mɯ́j-khɯ\cmn 他很吵,根本无法念书\end{exemple}\begin{sous-entrée}
\vedette{\hypertarget{}{\papi{ sɤcraŋlaŋ}}}\markboth{sɤcraŋlaŋ}{}\classe{vt}
\paradigme{\textit{dir :} \jya tɤ-}
\begin{définition}\fra faire crier très fort\end{définition}
\begin{définition}\cmn 发出很响的的声音;让……大声地叫\end{définition}
\begin{exemple}\jya paʁ ɲɯ-ɤsɯ-ntɕha tɕe, ta-sɤcraŋlaŋ\cmn 他在宰猪,猪大声地叫\end{exemple}
\begin{exemple}\jya χɕɤl ɲɯ-sɤcraŋlaŋ pa-qrɯ\cmn 他把玻璃砸碎了\end{exemple}
\end{sous-entrée}\end{entrée}

\begin{entrée}
\vedette{\hypertarget{Ⓔɣɤcrɯɣlɯɣ}{\papi{ ɣɤcrɯɣlɯɣ}}}\markboth{ɣɤcrɯɣlɯɣ}{}
\begin{relation-sémantique}\confer{
\hyperlink{Ⓔcrɯɣcrɯɣ}{\textit{ \papi{crɯɣcrɯɣ}}}
}\end{relation-sémantique}
\end{entrée}

\begin{entrée}
\vedette{\hypertarget{Ⓔɣɤcɯqhlɯβ}{\papi{ ɣɤcɯqhlɯβ}}}\markboth{ɣɤcɯqhlɯβ}{}\classe{vs}
\paradigme{\textit{dir :} \jya nɯ-}
\begin{définition}\fra faire du bruit en s'agitant (eau)\end{définition}
\begin{définition}\cmn 发出水晃动声\end{définition}
\begin{exemple}\jya tɯ-ci nɯ zɯm ɯ-ŋgɯ ɲɯ-ɣɤcɯqhlɯβ\cmn 水在桶里摇晃发出声音\end{exemple}\begin{sous-entrée}
\vedette{\hypertarget{}{\papi{ sɤcɯqhlɯβ}}}\markboth{sɤcɯqhlɯβ}{}
\paradigme{\textit{dir :} \jya nɯ-}
\begin{définition}\fra agiter l'eau bruyament\end{définition}
\begin{définition}\cmn 把水摇晃发出声音\end{définition}
\begin{exemple}\jya tɯ-ŋga nɯ́-wɣ-χtɕi tɕe khro ɲɯ́-wɣ-sɤcɯqhlɯβ tɕe ɲɯ-ɕo cha\cmn 洗衣服的时候要多晃几下就洗得干净\end{exemple}
\end{sous-entrée}\end{entrée}

\begin{entrée}
\vedette{\hypertarget{Ⓔɣɤɕu}{\papi{ ɣɤɕu}}}\markboth{ɣɤɕu}{}\classe{vi}\acception{1}
\paradigme{\textit{dir :} \jya thɯ-}
\begin{définition}\fra frais\end{définition}
\begin{définition}\cmn 凉快\end{définition}
\begin{exemple}\jya jiɕqha pɯ-ɣɯtshɤdɯɣ ri, zdɯm jo-ɣi tɕe ko-ɣɤɕu\cmn 刚才很闷热,来了云就凉快一些\end{exemple}\acception{2}
\paradigme{\textit{dir :} \jya kɤ-}
\begin{définition}\fra être couvert (soleil)\end{définition}
\begin{définition}\cmn 天阴\end{définition}
\begin{exemple}\jya tɯ-mɯ chɤ-lɯβ tɕe ɲɯ-ɣɤɕu\cmn 天阴了,现在很凉快\end{exemple}
\begin{exemple}\jya jisŋi wuma ʑo ɲɯ-sɤɕke tɕe, kɤ-ɣɤɕu kóʁmɯz nɤ kɤ-nɤma ɲɯ-ɬoʁ\cmn 今天天气很热,太阳阴了才能劳动\end{exemple}
\begin{relation-sémantique}\confer{
\hyperlink{Ⓔnɤɕu}{\textit{ \papi{nɤɕu}}}
}\end{relation-sémantique}
\begin{relation-sémantique}\confer{
\hyperlink{Ⓔtɤɕu}{\textit{ \papi{tɤɕu}}}
}\end{relation-sémantique}\end{entrée}

\begin{entrée}
\vedette{\hypertarget{Ⓔɣɤɕaʁɕaʁ}{\papi{ ɣɤɕaʁɕaʁ}}}\markboth{ɣɤɕaʁɕaʁ}{}
\classe{vs}
\begin{définition}\fra très amer\end{définition}
\begin{définition}\cmn 很苦\end{définition}
\begin{exemple}\jya tʂha ɲɯ-qiaβ ɲɯ-ɣɤɕaʁɕaʁ\cmn 茶非常苦\end{exemple}
\begin{exemple}\jya smɤn ɲɯ-qiaβ ɲɯ-ɣɤɕaʁɕaʁ\cmn 药非常苦\end{exemple}\end{entrée}

\begin{entrée}
\vedette{\hypertarget{Ⓔɣɤɕe}{\papi{ ɣɤɕe}}}\markboth{ɣɤɕe}{}\classe{vi}
\begin{définition}\fra qui va vite (temps)\end{définition}
\begin{définition}\cmn 过得快(时间);走得早\end{définition}
\begin{exemple}\jya ki a-tɤ-fse tɕe, nɤ-tɤ-rʑaʁ ɯ-ɲɯ-ɣɤɕe\cmn 这样的话,你的时间过得快吗?\end{exemple}
\begin{relation-sémantique}\confer{
\hyperlink{Ⓔɕe}{\textit{ \papi{ɕe}}}
}\end{relation-sémantique}\begin{sous-entrée}
\vedette{\hypertarget{}{\papi{ nɤɣɤɕe}}}\markboth{nɤɣɤɕe}{}\classe{vt}
\begin{définition}\fra trouver que (le temps) va vite\end{définition}
\begin{définition}\cmn 觉得时间过得很快\end{définition}
\begin{exemple}\jya tɤ-rʑaʁ mɯ́j-nɤɣɤɕe-a\cmn 我觉得时间过得很慢\end{exemple}
\end{sous-entrée}\end{entrée}

\begin{entrée}
\vedette{\hypertarget{Ⓔɣɤɕkɤɣɕkɤɣ}{\papi{ ɣɤɕkɤɣɕkɤɣ}}}\markboth{ɣɤɕkɤɣɕkɤɣ}{}\classe{vi}
\paradigme{\textit{dir :} \jya tɤ-}
\begin{définition}\ 
\begin{déclaration}\grammar{deidph}\end{déclaration}\end{définition}
\begin{définition}\fra objet dur faisant du bruit lorsqu'on le frappe, glacé au point d'être dur\end{définition}
\begin{définition}\cmn 硬东西一敲就发出声音;冻得很硬的样子\end{définition}
\begin{exemple}\jya ɲɯ-rko ɲɯ-ɣɤɕkɤɣɕkɤɣ ʑo\cmn 非常硬,摸起来像石头一样\end{exemple}\begin{sous-entrée}
\vedette{\hypertarget{}{\papi{ ɣɤɕkɤɣlɤɣ}}}\markboth{ɣɤɕkɤɣlɤɣ}{}\classe{vs}
\begin{définition}\fra maigre et vieux, mais qui s'active sans arrêt\end{définition}
\begin{définition}\cmn 又瘦又老,还是不停地活动\end{définition}
\begin{exemple}\jya nɯŋa ɲɯ-ɣɤɕkɤɣlɤɣ\cmn 奶牛很不听话地乱蹦乱跳\end{exemple}
\begin{exemple}\jya jiɕqha rgɤtpu nɯ ɲɯ-ɣɤɕkɤɣlɤɣ\cmn 那个(又瘦又高的)老年人走动\end{exemple}
\begin{relation-sémantique}\confer{
\hyperlink{Ⓔɕkɤɣnɤɕkɤɣ}{\textit{ \papi{ɕkɤɣnɤɕkɤɣ}}}
}\end{relation-sémantique}
\end{sous-entrée}\begin{sous-entrée}
\vedette{\hypertarget{}{\papi{ sɤɕkɤɣɕkɤɣ}}}\markboth{sɤɕkɤɣɕkɤɣ}{}\classe{vt}
\paradigme{\textit{dir :} \jya tɤ-}
\begin{définition}\fra faire du bruit en frappant un objet dur\end{définition}
\begin{définition}\cmn 敲打硬的东西\end{définition}
\begin{exemple}\jya nɤ-stu tɤ-fse ma ta-sɤɕkɤɣɕkɤɣ\cmn 你小心,不然我会整你的!\end{exemple}
\begin{exemple}\jya rdɤstaʁ (si) tɤ-sɤɕkɤɣɕkɤɣ-a\cmn 我敲打了石头(木料)\end{exemple}
\end{sous-entrée}\end{entrée}

\begin{entrée}
\vedette{\hypertarget{Ⓔɣɤɕkɤɣlɤɣ}{\papi{ ɣɤɕkɤɣlɤɣ}}}\markboth{ɣɤɕkɤɣlɤɣ}{}
\begin{relation-sémantique}\confer{
\hyperlink{Ⓔɣɤɕkɤɣɕkɤɣ}{\textit{ \papi{ɣɤɕkɤɣɕkɤɣ}}}
}\end{relation-sémantique}\end{entrée}

\begin{entrée}
\vedette{\hypertarget{Ⓔɣɤɕnɯɣlɯɣ}{\papi{ ɣɤɕnɯɣlɯɣ}}}\markboth{ɣɤɕnɯɣlɯɣ}{}
\begin{relation-sémantique}\confer{
\hyperlink{Ⓔɕnɯɣnɤlɯɣ}{\textit{ \papi{ɕnɯɣnɤlɯɣ}}}
}\end{relation-sémantique}\end{entrée}

\begin{entrée}
\vedette{\hypertarget{Ⓔɣɤɕŋaʁɕŋaʁ}{\papi{ ɣɤɕŋaʁɕŋaʁ}}}\markboth{ɣɤɕŋaʁɕŋaʁ}{}
\begin{relation-sémantique}\confer{
\hyperlink{ⒺɕŋaʁɕŋaʁⒽ2}{\textit{ \papi{ɕŋaʁɕŋaʁ2}}}
}\end{relation-sémantique}\end{entrée}

\begin{entrée}
\vedette{\hypertarget{Ⓔɣɤɕo}{\papi{ ɣɤɕo}}}\markboth{ɣɤɕo}{}
\begin{relation-sémantique}\confer{
\hyperlink{Ⓔɕo}{\textit{ \papi{ɕo}}}
}\end{relation-sémantique}\end{entrée}

\begin{entrée}
\vedette{\hypertarget{Ⓔɣɤɕpɤɕpɤr}{\papi{ ɣɤɕpɤɕpɤr}}}\markboth{ɣɤɕpɤɕpɤr}{}
\classe{vi}
\begin{définition}\fra émettre un bruit fort\end{définition}
\begin{définition}\cmn 乱叫\end{définition}
\begin{exemple}\jya @laba ɲɯ-ɣɤɕpɤɕpɤr\cmn 喇叭在乱叫\end{exemple}\begin{sous-entrée}
\vedette{\hypertarget{}{\papi{ ɣɤɕpɤrlɤr}}}\markboth{ɣɤɕpɤrlɤr}{}\classe{vi}
\paradigme{\textit{dir :} \jya tɤ-}
\begin{définition}\fra exprimer son opinion à haute voix sans se soucier de rien\end{définition}
\begin{définition}\cmn 大声说话,不注意场合\end{définition}
\begin{exemple}\jya ɲɯ-ɣɤɕpɤrlɤr\cmn 他在乱吼乱叫\end{exemple}
\begin{exemple}\jya ɲɯ-ɣɤɕpɤrlɤr ɲɯ-rɯɕmi\cmn 他在大声说话\end{exemple}
\begin{exemple}\jya nɤki tɕheme nɯ kɯ-ɣɤɕpɤrlɤr ci ŋu\cmn 那个女的很多嘴\end{exemple}
\begin{exemple}\jya nɤ-mtɕhi kɤ-ndɤm ma-tɯ-ɣɤɕpɤrlɤr\cmn 你闭嘴,不要大声说话\end{exemple}
\begin{relation-sémantique}\confer{
\hyperlink{Ⓔɕpɤrnɤlɤr}{\textit{ \papi{ɕpɤrnɤlɤr}}}
}\end{relation-sémantique}
\end{sous-entrée}\end{entrée}

\begin{entrée}
\vedette{\hypertarget{Ⓔɣɤɕpɤrlɤr}{\papi{ ɣɤɕpɤrlɤr}}}\markboth{ɣɤɕpɤrlɤr}{}
\begin{relation-sémantique}\confer{
\hyperlink{Ⓔɣɤɕpɤɕpɤr}{\textit{ \papi{ɣɤɕpɤɕpɤr}}}
}\end{relation-sémantique}\end{entrée}

\begin{entrée}
\vedette{\hypertarget{Ⓔɣɤɕphɤβlɤβ}{\papi{ ɣɤɕphɤβlɤβ}}}\markboth{ɣɤɕphɤβlɤβ}{}
\classe{vi}
\begin{définition}\fra faire du bruit (battement d'aile)\end{définition}
\begin{définition}\cmn (鸟)拍翅膀发出声音;抖衣服发出声音\end{définition}
\begin{exemple}\jya pɣa ɲɯ-ɣɤɕphɤβlɤβ\cmn 鸟拍翅膀发出声音\end{exemple}\begin{sous-entrée}
\vedette{\hypertarget{}{\papi{ sɤɕphɤβlɤβ}}}\markboth{sɤɕphɤβlɤβ}{}\classe{vt}
\begin{exemple}\jya tɯ-ŋga ɲɯ-sɤɕphɤβlɤβ\cmn 他在抖衣服,发出声音\end{exemple}
\end{sous-entrée}\end{entrée}

\begin{entrée}
\vedette{\hypertarget{Ⓔɣɤɕphɤr}{\papi{ ɣɤɕphɤr}}}\markboth{ɣɤɕphɤr}{}
\classe{vi}
\paradigme{\textit{dir :} \jya nɯ-}
\begin{définition}\fra concilier\end{définition}
\begin{définition}\cmn 劝解\end{définition}
\begin{exemple}\jya ɯʑo nɯ-ɣɤɕphɤr\cmn 他劝解了\end{exemple}
\begin{exemple}\jya ku-ɣɤɕphar-a\cmn 我劝解\end{exemple}
\begin{exemple}\jya ku-tɯ-ɣɤɕphɤr\cmn 你劝解\end{exemple}
\begin{exemple}\jya ɯʑo ku-ɣɤɕphɤr\cmn 他劝解\end{exemple}
\begin{exemple}\jya nɯ-pɤrthɤβ nɯ-ɣɤɕphar-a\cmn 我在他们之间劝解了\end{exemple}
\begin{exemple}\jya ʑɤni ɲɯ-ɤlɯlɤt-ndʑi tɕe, nɯ-ɣɤɕphar-a (=nɯ-nɯkhɤda-t-a-ndʑi)\cmn 他们吵架的时候,我把他们劝开了\end{exemple}
\begin{relation-sémantique}\synonyme{
\hyperlink{Ⓔnɯkhɤda}{\textit{ \papi{nɯkhɤda}}}
}\end{relation-sémantique}\end{entrée}

\begin{entrée}
\vedette{\hypertarget{Ⓔɣɤɕqali}{\papi{ ɣɤɕqali}}}\markboth{ɣɤɕqali}{}\classe{vi}
\paradigme{\textit{dir :} \jya tɤ-}
\begin{définition}\fra crier\end{définition}
\begin{définition}\cmn 嚷;喊\end{définition}
\begin{exemple}\jya aʑo tɤ-ɣɤɕqali-a\cmn 我喊了一下\end{exemple}
\begin{exemple}\jya nɤʑo tɤ-tɯ-ɣɤɕqali\cmn 你喊了一下\end{exemple}
\begin{exemple}\jya ɯʑo tɤ-ɣɤɕqali\cmn 他喊了一下\end{exemple}
\begin{exemple}\jya pɯ-nɤscar-a tɕe, tɤ-ɣɤɕqali-a pɯ-ra\cmn 我吓了一跳,忍不住喊了一声\end{exemple}
\begin{exemple}\jya a-@shouji ɲɯ-ɣɤɕqali\cmn 我的手机在响\end{exemple}
\begin{relation-sémantique}\confer{
\hyperlink{Ⓔtɤɕqali}{\textit{ \papi{tɤɕqali}}}
}\end{relation-sémantique}\begin{sous-entrée}
\vedette{\hypertarget{}{\papi{ sɤɕqali}}}\markboth{sɤɕqali}{}\classe{vt}
\paradigme{\textit{dir :} \jya tɤ-}
\begin{exemple}\jya ɯʑo kɯ ɯ-skɤt ta-sɤɕqali ʑo (=ta-nɤxɕɤt)\cmn 他大声地喊了一下\end{exemple}
\end{sous-entrée}\begin{sous-entrée}
\vedette{\hypertarget{}{\papi{ sɯsɤɕqali}}}\markboth{sɯsɤɕqali}{}\classe{vt}
\begin{définition}\fra faire crier\end{définition}
\begin{définition}\cmn 让……喊\end{définition}
\begin{exemple}\jya ɯʑo kɯ tɤ-pɤtso ta-ʁndɯ tɕe ta-sɯsɤɕqali ʑo\cmn 他打了小孩子,让他大叫了一声\end{exemple}
\end{sous-entrée}\end{entrée}

\begin{entrée}
\vedette{\hypertarget{Ⓔɣɤɕtʂaŋlaŋ}{\papi{ ɣɤɕtʂaŋlaŋ}}}\markboth{ɣɤɕtʂaŋlaŋ}{}
\begin{relation-sémantique}\confer{
\hyperlink{Ⓔɕtʂaŋɕtʂaŋ}{\textit{ \papi{ɕtʂaŋɕtʂaŋ}}}
}\end{relation-sémantique}\end{entrée}

\begin{entrée}
\vedette{\hypertarget{Ⓔɣɤɕɯβɕɯβ}{\papi{ ɣɤɕɯβɕɯβ}}}\markboth{ɣɤɕɯβɕɯβ}{}
\classe{vi}
\paradigme{\textit{dir :} \jya tɤ-}
\begin{définition}\fra murmurer\end{définition}
\begin{définition}\cmn 悄声说话,偷偷说话\end{définition}
\begin{exemple}\jya tɤ-ɣɤɕɯβɕɯβ-a\cmn 我悄声说话了\end{exemple}
\begin{exemple}\jya tɤ-tɯ-ɣɤɕɯβɕɯβ\cmn 你悄声说话了\end{exemple}
\begin{exemple}\jya tɤ-ɣɤɕɯβɕɯβ\cmn 他悄声说话了\end{exemple}
\begin{exemple}\jya ɯʑo ku-ɣɤɕɯβɕɯβ\cmn 他正在悄声说话\end{exemple}
\begin{exemple}\jya tɯrme ra kɯ pjɯ-mtshɤm-nɯ mɯ-tɤ-pe tɕe, tu-kɯ-ɣɤɕɯβɕɯβ tɕe phɤn\cmn 如果不想被别人听见,悄声说话就可以了\end{exemple}\end{entrée}

\begin{entrée}
\vedette{\hypertarget{Ⓔɣɤɕɯftaʁ}{\papi{ ɣɤɕɯftaʁ}}}\markboth{ɣɤɕɯftaʁ}{}
\begin{relation-sémantique}\confer{
\hyperlink{Ⓔɕɯftaʁ}{\textit{ \papi{ɕɯftaʁ}}}
}\end{relation-sémantique}\end{entrée}

\begin{entrée}
\vedette{\hypertarget{Ⓔɣɤɕɯŋɕɯŋ}{\papi{ ɣɤɕɯŋɕɯŋ}}}\markboth{ɣɤɕɯŋɕɯŋ}{}
\begin{relation-sémantique}\confer{
\hyperlink{Ⓔɕɯŋɕɯŋ}{\textit{ \papi{ɕɯŋɕɯŋ}}}
}\end{relation-sémantique}\end{entrée}

\begin{entrée}
\vedette{\hypertarget{Ⓔɣɤdɤn}{\papi{ ɣɤdɤn}}}\markboth{ɣɤdɤn}{}\classe{vt}
\paradigme{\textit{dir :} \jya tɤ-}
\paradigme{\textit{dir :} \jya nɯ-}
\begin{définition}\fra augmenter\end{définition}
\begin{définition}\cmn 增多;增加\end{définition}
\begin{exemple}\jya tɤ-ɣɤdan-a\cmn 我加了\end{exemple}
\begin{exemple}\jya tɤ-tɯ-ɣɤdɤn\cmn 你加了\end{exemple}
\begin{exemple}\jya ɯʑo kɯ ta-ɣɤdɤn\cmn 他加了\end{exemple}
\begin{exemple}\jya tɯ-rju nɯ kɤ-ɣɤdɤn mɤ-ra\cmn 一句都不要加\end{exemple}
\begin{exemple}\jya pɕawtsɯ nɯ ɯ-tsa nɯ ta-rku ma, nɯ ma mɯ-ta-ɣɤdɤn\cmn 他装了必要的钱,没有装更多\end{exemple}
\begin{exemple}\jya mɯ-na-ɣɤdɤn\cmn 话没有多说\end{exemple}\end{entrée}

\begin{entrée}
\vedette{\hypertarget{Ⓔɣɤdi}{\papi{ ɣɤdi}}}\markboth{ɣɤdi}{}
\classe{vs}
\paradigme{\textit{dir :} \jya nɯ-}
\begin{définition}\fra mauvaise (odeur), puer\end{définition}
\begin{définition}\cmn 臭;变味\end{définition}
\begin{exemple}\jya tɤ-mthɯm ɲɯ-ɣɤdi\cmn 肉变味了\end{exemple}
\begin{exemple}\jya ɯβrɤ-ɲɯ-ɣɤdi\cmn 有没有变味?\end{exemple}\begin{sous-entrée}
\vedette{\hypertarget{}{\papi{ zɣɤdi}}}\markboth{zɣɤdi}{}\classe{vt}
\paradigme{\textit{dir :} \jya nɯ-}
\begin{définition}\ 
\begin{déclaration}\grammar{caus}\end{déclaration}\end{définition}
\begin{définition}\fra laisser puer\end{définition}
\begin{définition}\cmn 令东西有臭味\end{définition}
\begin{exemple}\jya tɤ-mthɯm ɲɤ-zɣɤdi-t-a\cmn 我(不小心)把肉(忘在那里了),令它有臭味\end{exemple}
\begin{relation-sémantique}\confer{
\hyperlink{Ⓔtɤ-di}{\textit{ \papi{tɤ-di}}}
}\end{relation-sémantique}
\end{sous-entrée}\end{entrée}

\begin{entrée}
\vedette{\hypertarget{Ⓔɣɤdoŋdoŋ}{\papi{ ɣɤdoŋdoŋ}}}\markboth{ɣɤdoŋdoŋ}{}
\classe{vi}
\paradigme{\textit{dir :} \jya pɯ-}
\begin{définition}\fra couler bruyamment (eau)\end{définition}
\begin{définition}\cmn 水管或某种物体中流出来的水又急又多,发出响声\end{définition}
\begin{exemple}\jya tɯ-ci ɲɯ-ɣɤdoŋdoŋ ʑo\cmn 水流出来很响\end{exemple}\begin{sous-entrée}
\vedette{\hypertarget{}{\papi{ sɤdoŋdoŋ}}}\markboth{sɤdoŋdoŋ}{}
\begin{définition}\fra jeter, verser beaucoup d'eau\end{définition}
\begin{définition}\cmn 泼很多水,倒很多水(发出很多声音)\end{définition}
\begin{exemple}\jya qhajŋgɯ ri tɯ-ci tɤ-sɤdoŋdoŋ-a pɯ-lat-a\cmn 我把水倒进引水槽了(声音大,水多)\end{exemple}
\end{sous-entrée}\end{entrée}

\begin{entrée}
\vedette{\hypertarget{Ⓔɣɤdzaŋdzaŋ}{\papi{ ɣɤdzaŋdzaŋ}}}\markboth{ɣɤdzaŋdzaŋ}{}
\begin{relation-sémantique}\confer{
\hyperlink{Ⓔdzaŋdzaŋ}{\textit{ \papi{dzaŋdzaŋ}}}
}\end{relation-sémantique}\end{entrée}

\begin{entrée}
\vedette{\hypertarget{Ⓔɣɤdzɯlɯt}{\papi{ ɣɤdzɯlɯt}}}\markboth{ɣɤdzɯlɯt}{} (\variante{\_ɣɤdzɯlɯz}) 
\classe{vi}
\paradigme{\textit{dir :} \jya nɯ-}
\paradigme{\textit{dir :} \jya tɤ-}
\begin{définition}\fra s'agiter\end{définition}
\begin{définition}\cmn 动来动去;摇动;蠕动\end{définition}
\begin{exemple}\jya ku-ɣɤdzɯlɯz-a\end{exemple}
\begin{exemple}\jya tu-ɣɤdzɯlɯt-a\cmn 我正在动来动去\end{exemple}
\begin{exemple}\jya ɲɯ-tɯ-ɣɤdzɯlɯz\cmn 你动来动去\end{exemple}
\begin{exemple}\jya kɤ-nɤma tɤra tɕe, ku-ɣɤdzɯlɯt ra\cmn 要工作了,要活跃起来\end{exemple}
\begin{exemple}\jya tɤ-pɤtso ɲɯ-ɣɤdzɯlɯt\cmn 小孩子在动来动去\end{exemple}
\begin{exemple}\jya nɤ-mi ɲɯ-ɣɤdzɯlɯt\cmn 你的脚在动\end{exemple}
\begin{exemple}\jya qajɯ ci ɲɯ-ɣɤdzɯlɯt\cmn 有一条虫在动\end{exemple}
\begin{relation-sémantique}\confer{
\hyperlink{Ⓔsɤdzɯlɯt}{\textit{ \papi{sɤdzɯlɯt}}}
}\end{relation-sémantique}\end{entrée}

\begin{entrée}
\vedette{\hypertarget{Ⓔɣɤdʑɯdʑaŋ}{\papi{ ɣɤdʑɯdʑaŋ}}}\markboth{ɣɤdʑɯdʑaŋ}{}\classe{vi}
\paradigme{\textit{dir :} \jya tɤ-}
\begin{définition}\fra glisser rapidement (d'un gros morceau de bois)\end{définition}
\begin{définition}\cmn 又粗又长的木料很快地滑下来的样子\end{définition}
\begin{sous-entrée}
\vedette{\hypertarget{}{\papi{ sɤdʑɯdʑaŋ}}}\markboth{sɤdʑɯdʑaŋ}{}\classe{vt}
\paradigme{\textit{dir :} \jya tɤ-}
\begin{définition}\fra faire glisser rapidement (un gros morceau de bois)\end{définition}
\begin{définition}\cmn 令又粗又长的木料很快地滑下来\end{définition}
\begin{exemple}\jya si ta-sɤdʑɯdʑaŋ ʑo pa-βde\cmn 把长的树枝从高处任意地摔下来了\end{exemple}
\end{sous-entrée}\end{entrée}

\begin{entrée}
\vedette{\hypertarget{Ⓔɣɤfka}{\papi{ ɣɤfka}}}\markboth{ɣɤfka}{}
\begin{relation-sémantique}\confer{
\hyperlink{ⒺfkaⒽ1}{\textit{ \papi{fka1}}}
}\end{relation-sémantique}\end{entrée}

\begin{entrée}
\vedette{\hypertarget{Ⓔɣɤfsoʁ}{\papi{ ɣɤfsoʁ}}}\markboth{ɣɤfsoʁ}{}
\begin{relation-sémantique}\confer{
\hyperlink{ⒺfsoʁⒽ2}{\textit{ \papi{fsoʁ2}}}
}\end{relation-sémantique}\end{entrée}

\begin{entrée}
\vedette{\hypertarget{Ⓔɣɤftshi}{\papi{ ɣɤftshi}}}\markboth{ɣɤftshi}{}
\begin{relation-sémantique}\confer{
\hyperlink{Ⓔftshi}{\textit{ \papi{ftshi}}}
}\end{relation-sémantique}
\end{entrée}

\begin{entrée}
\vedette{\hypertarget{Ⓔɣɤglɤglɤɣ}{\papi{ ɣɤglɤglɤɣ}}}\markboth{ɣɤglɤglɤɣ}{}
\classe{vi}
\begin{définition}\fra bruyant\end{définition}
\begin{définition}\cmn 很响(一阵一阵敲打声)\end{définition}
\begin{exemple}\jya @qiche ɲɯ-ɣɤglɤglɤɣ\cmn 汽车很响\end{exemple}
\begin{relation-sémantique}\confer{
\hyperlink{Ⓔsɤglɤglɤɣ}{\textit{ \papi{sɤglɤglɤɣ}}}
}\end{relation-sémantique}\end{entrée}

\begin{entrée}
\vedette{\hypertarget{Ⓔɣɤgo}{\papi{ ɣɤgo}}}\markboth{ɣɤgo}{}
\classe{vs}
\begin{définition}\fra naïf, honnête\end{définition}
\begin{définition}\cmn 笨;老实\end{définition}
\begin{exemple}\jya jiɕqha tɯrme kɯ-ɣɤgo ci ɲɯ-ŋu\cmn 他是老实人\end{exemple}
\begin{exemple}\jya ɯʑo kɯ-ɣɤgo ci ɕti tɕe, kɯŋu nɤme ɕti\cmn 他这个人很老实,相信他会把事情做好\end{exemple}\end{entrée}

\begin{entrée}
\vedette{\hypertarget{Ⓔɣɤgɯgɯɣ}{\papi{ ɣɤgɯgɯɣ}}}\markboth{ɣɤgɯgɯɣ}{}
\classe{vi}
\paradigme{\textit{dir :} \jya tɤ-}
\begin{définition}\fra faire du bruit en démarrant (moteur), faire du bruit en soufflant (vent)\end{définition}
\begin{définition}\cmn 发出声音(如汽车发动,吹大风等的声音)\end{définition}
\begin{exemple}\jya mbɣɯrloʁ ɲɯ-ɣɤgɯgɯɣ\cmn 雷很响\end{exemple}
\begin{relation-sémantique}\confer{
\hyperlink{Ⓔgɯɣnɤgɯɣ}{\textit{ \papi{gɯɣnɤgɯɣ}}}
}\end{relation-sémantique}
\begin{relation-sémantique}\synonyme{
\hyperlink{Ⓔɣɤŋgɯrŋgɯr}{\textit{ \papi{ɣɤŋgɯrŋgɯr}}}
}\end{relation-sémantique}\end{entrée}

\begin{entrée}
\vedette{\hypertarget{Ⓔɣɤɣɤjɣɤj}{\papi{ ɣɤɣɤjɣɤj}}}\markboth{ɣɤɣɤjɣɤj}{}\classe{vi}
\begin{définition}\fra avoir la tête qui tourne, ne pas pouvoir tenir sur ses jambe\end{définition}
\begin{définition}\cmn 不停的摇晃的感觉,脚都站不稳\end{définition}\begin{sous-entrée}
\vedette{\hypertarget{}{\papi{ ɣɤɣɤjlɤj}}}\markboth{ɣɤɣɤjlɤj}{}\classe{vi}
\end{sous-entrée}\end{entrée}

\begin{entrée}
\vedette{\hypertarget{Ⓔɣɤɣɤjlɤj}{\papi{ ɣɤɣɤjlɤj}}}\markboth{ɣɤɣɤjlɤj}{}
\begin{relation-sémantique}\confer{
\hyperlink{Ⓔɣɤɣɤjɣɤj}{\textit{ \papi{ɣɤɣɤjɣɤj}}}
}\end{relation-sémantique}\end{entrée}

\begin{entrée}
\vedette{\hypertarget{Ⓔɣɤɣɤmbrɯ}{\papi{ ɣɤɣɤmbrɯ}}}\markboth{ɣɤɣɤmbrɯ}{}\classe{vs}
\begin{définition}\ 
\begin{déclaration}\grammar{facil}\end{déclaration}\end{définition}
\begin{définition}\fra s'énerver facilement\end{définition}
\begin{définition}\cmn 容易生气\end{définition}
\begin{exemple}\jya ɲɯ-ɣɤɣɤmbrɯ\cmn 他容易生气\end{exemple}
\begin{exemple}\jya ɲɯ-tɯ-ɣɤɣɤmbrɯ\cmn 你容易生气\end{exemple}
\begin{relation-sémantique}\confer{
\hyperlink{Ⓔsɤmbrɯ}{\textit{ \papi{sɤmbrɯ}}}
}\end{relation-sémantique}
\begin{relation-sémantique}\confer{
\hyperlink{Ⓔsɤzmbrɯ}{\textit{ \papi{sɤzmbrɯ}}}
}\end{relation-sémantique}\end{entrée}

\begin{entrée}
\vedette{\hypertarget{Ⓔɣɤɣɤtɕɯɣ}{\papi{ ɣɤɣɤtɕɯɣ}}}\markboth{ɣɤɣɤtɕɯɣ}{}
\begin{relation-sémantique}\confer{
\hyperlink{Ⓔɣɤtɕɯɣ}{\textit{ \papi{ɣɤtɕɯɣ}}}
}\end{relation-sémantique}\end{entrée}

\begin{entrée}
\vedette{\hypertarget{Ⓔɣɤɣɤwu}{\papi{ ɣɤɣɤwu}}}\markboth{ɣɤɣɤwu}{}
\begin{relation-sémantique}\confer{
\hyperlink{Ⓔɣɤwu}{\textit{ \papi{ɣɤwu}}}
}\end{relation-sémantique}\end{entrée}

\begin{entrée}
\vedette{\hypertarget{Ⓔɣɤɣi}{\papi{ ɣɤɣi}}}\markboth{ɣɤɣi}{}
\begin{relation-sémantique}\confer{
\hyperlink{Ⓔɣi}{\textit{ \papi{ɣi}}}
}\end{relation-sémantique}\end{entrée}

\begin{entrée}
\vedette{\hypertarget{Ⓔɣɤɣɯrɣɯr}{\papi{ ɣɤɣɯrɣɯr}}}\markboth{ɣɤɣɯrɣɯr}{}\classe{vi}
\paradigme{\textit{dir :} \jya tɤ-}\acception{1}
\begin{définition}\fra ardent (feu)\end{définition}
\begin{définition}\cmn 旺盛(火)\end{définition}
\begin{exemple}\jya smi ɲɯ-ɣɤɣɯrɣɯr ɲɯ-nɯt\cmn 火烧得很旺\end{exemple}\acception{2}
\begin{définition}\fra animé, bruyant\end{définition}
\begin{définition}\cmn 嘈杂(声音);闹哄哄;熙熙攘攘\end{définition}
\begin{exemple}\jya tɯrme ra ɲɯ-ɣɤɣɯrɣɯr-nɯ\cmn 人们很吵\end{exemple}
\begin{exemple}\jya ɲɯ-ɣɤɕqali-nɯ tɕe ɲɯ-ɣɤɣɯrɣɯr-nɯ\cmn 他们在吼叫,很吵\end{exemple}\begin{sous-entrée}
\vedette{\hypertarget{}{\papi{ sɤɣɯrɣɯr}}}\markboth{sɤɣɯrɣɯr}{}\classe{vt}
\end{sous-entrée}\end{entrée}

\begin{entrée}
\vedette{\hypertarget{Ⓔɣɤjaŋri}{\papi{ ɣɤjaŋri}}}\markboth{ɣɤjaŋri}{}\classe{vi}
\paradigme{\textit{dir :} \jya pɯ-}
\begin{définition}\fra aller ça et là\end{définition}
\begin{définition}\cmn 来回走动\end{définition}
\begin{exemple}\jya pɯ-ɣɤjaŋri-a ntsɯ ma tɤ-χtɯ-t-a me\cmn 我在街上随便走动,没有买什么东西\end{exemple}\end{entrée}

\begin{entrée}
\vedette{\hypertarget{Ⓔɣɤjaʁ}{\papi{ ɣɤjaʁ}}}\markboth{ɣɤjaʁ}{}
\begin{relation-sémantique}\confer{
\hyperlink{Ⓔjaʁ}{\textit{ \papi{jaʁ}}}
}\end{relation-sémantique}\end{entrée}

\begin{entrée}
\vedette{\hypertarget{Ⓔɣɤjɤβjɤβ}{\papi{ ɣɤjɤβjɤβ}}}\markboth{ɣɤjɤβjɤβ}{}
\classe{vi}
\paradigme{\textit{dir :} \jya nɯ-}
\begin{définition}\fra toucher à tout, s'amuser avec les petits objets\end{définition}
\begin{définition}\cmn 到处乱摸;东摸西摸(偷东西)\end{définition}
\begin{exemple}\jya βɣɤza ɲɯ-ɣɤjɤβjɤβ\cmn 苍蝇到处乱爬(令人发痒)\end{exemple}
\begin{exemple}\jya ma-tɯ-ɣɤjɤβjɤβ, koŋla nɯ-sɤŋo\cmn 别乱摸东西,认真听\end{exemple}
\begin{exemple}\jya kɯ-mɯrkɯ ɯ-jaʁ ɣɤjɤβjɤβ\cmn 小偷在乱摸\end{exemple}\end{entrée}

\begin{entrée}
\vedette{\hypertarget{Ⓔɣɤjɤrjɤr}{\papi{ ɣɤjɤrjɤr}}}\markboth{ɣɤjɤrjɤr}{}
\begin{relation-sémantique}\confer{
\hyperlink{Ⓔjɤrjɤr}{\textit{ \papi{jɤrjɤr}}}
}\end{relation-sémantique}\end{entrée}

\begin{entrée}
\vedette{\hypertarget{Ⓔɣɤji}{\papi{ ɣɤji}}}\markboth{ɣɤji}{}
\classe{vs}
\paradigme{\textit{dir :} \jya tɤ-}
\begin{définition}\fra rapide\end{définition}
\begin{définition}\cmn 快(动作)
\begin{déclaration}\use{这句话也可以理解成“要多加一点”,\stylefv{tɤɣɤji}也是及物动词\stylefv{ɣɤjɯ}“加”的命令式}\end{déclaration}\end{définition}
\begin{exemple}\jya tɤ-ɣɤji tsa ɲɯ-ra\cmn 要快点\end{exemple}
\begin{exemple}\jya ɯʑo kɤ-nɤma ra ɲɯ-ɣɤji\cmn 他劳动做得很快\end{exemple}
\begin{exemple}\jya aʑo ɣɤji-a\cmn 我很快\end{exemple}
\begin{exemple}\jya kɤ-rɯndzɤtshi ɲɯ-ɣɤji\cmn 他吃饭吃得很快\end{exemple}\begin{sous-entrée}
\vedette{\hypertarget{}{\papi{ zɣɤji}}}\markboth{zɣɤji}{}\classe{vt}
\paradigme{\textit{dir :} \jya tɤ-}
\begin{définition}\ 
\begin{déclaration}\grammar{caus}\end{déclaration}\end{définition}
\begin{définition}\fra accélérer\end{définition}
\begin{définition}\cmn 加快\end{définition}
\begin{exemple}\jya ʑara ɲɯ-rɤma-nɯ tɕe, aʑo kɯ-qur jɤ-ari-a tɕe, tɤ-zɣɤji-t-a-nɯ\cmn 他们在工作,我去帮忙,令他们工作得更快\end{exemple}
\end{sous-entrée}\end{entrée}

\begin{entrée}
\vedette{\hypertarget{Ⓔɣɤjiz}{\papi{ ɣɤjiz}}}\markboth{ɣɤjiz}{}\classe{adv}
\begin{définition}\fra temporairement\end{définition}
\begin{définition}\cmn 暂时\end{définition}
\begin{exemple}\jya tɤ-rʑaʁ kɯ-zri ɯ-sɯso kɤ-lɤt ra ma ɣɤjiz ɯ-βlɯβlu kɤ-lɤt mɤ-pe\cmn 只顾短暂不顾长远是不好的\end{exemple}\end{entrée}

\begin{entrée}
\vedette{\hypertarget{Ⓔɣɤjka}{\papi{ ɣɤjka}}}\markboth{ɣɤjka}{}
\classe{vi}
\begin{définition}\fra bégayer\end{définition}
\begin{définition}\cmn 结巴\end{définition}
\begin{exemple}\jya jiɕqha tɯrme ɲɯ-ɣɤjka\cmn 这个人是结巴\end{exemple}
\begin{exemple}\jya kɤ-rɯɕmi ɲɯ-ɣɤjka\cmn 他说话结巴\end{exemple}\end{entrée}

\begin{entrée}
\vedette{\hypertarget{Ⓔɣɤjlu}{\papi{ ɣɤjlu}}}\markboth{ɣɤjlu}{}
\classe{vs}
\begin{définition}\ 
\begin{déclaration}\grammar{denom}\end{déclaration}\end{définition}
\begin{définition}\fra cru\end{définition}
\begin{définition}\cmn 没有炒熟,吃的时候有生味
\begin{déclaration}\use{生肉不能用\stylefv{ɣɤjlu}}\end{déclaration}\end{définition}
\begin{exemple}\jya tɤ-jlu ɣɤjlu\cmn 面粉没有炒熟\end{exemple}
\begin{exemple}\jya mɯ́j-smi tɕe ɲɯ-ɣɤjlu\cmn 没有炒熟,有生味\end{exemple}
\begin{exemple}\jya stoʁ ɲɯ-ɣɤjlu\cmn 胡豆没有炒熟\end{exemple}
\begin{relation-sémantique}\confer{
\hyperlink{Ⓔtɤjlu}{\textit{ \papi{tɤjlu}}}
}\end{relation-sémantique}\end{entrée}

\begin{entrée}
\vedette{\hypertarget{Ⓔɣɤjmŋo}{\papi{ ɣɤjmŋo}}}\markboth{ɣɤjmŋo}{}
\classe{vt}
\paradigme{\textit{dir :} \jya pɯ-}
\paradigme{\textit{dir :} \jya kɤ-}
\begin{définition}\ 
\begin{déclaration}\grammar{denom}\end{déclaration}\end{définition}
\begin{définition}\fra rêver, rêver de\end{définition}
\begin{définition}\cmn 做梦\end{définition}
\begin{exemple}\jya kɤ-ɣɤjmŋo-t-a, pɯ-ɣɤjmŋo-t-a\cmn 我梦见他了\end{exemple}
\begin{exemple}\jya pɯ-tɯ-ɣɤjmŋo-t\cmn 你梦见他了\end{exemple}
\begin{exemple}\jya pa-ɣɤjmŋo\cmn 他梦见他了\end{exemple}
\begin{exemple}\jya jɯfɕɯr pɯ-ta-ɣɤjmŋo\cmn 我昨天梦见你了\end{exemple}
\begin{exemple}\jya ɯʑo kɯ pjɤ́-wɣ-ɣɤmŋo-a\cmn 他梦见我了\end{exemple}
\begin{exemple}\jya kɯrɯ skɤt pjɯ-kɯ-sɯxɕat-a ɲɯ-ŋu pɯ-ɣɤjmŋo-t-a\cmn 我做梦你在教我藏语\end{exemple}
\begin{relation-sémantique}\confer{
\hyperlink{Ⓔtɯ-jmŋo}{\textit{ \papi{tɯ-jmŋo}}}
}\end{relation-sémantique}\end{entrée}

\begin{entrée}
\vedette{\hypertarget{Ⓔɣɤjmɯt}{\papi{ ɣɤjmɯt}}}\markboth{ɣɤjmɯt}{}
\begin{relation-sémantique}\confer{
\hyperlink{Ⓔjmɯt}{\textit{ \papi{jmɯt}}}
}\end{relation-sémantique}\end{entrée}

\begin{entrée}
\vedette{\hypertarget{Ⓔɣɤjom}{\papi{ ɣɤjom}}}\markboth{ɣɤjom}{}
\classe{vt}
\paradigme{\textit{dir :} \jya nɯ-}
\begin{définition}\fra élargir\end{définition}
\begin{définition}\cmn 扩大;修宽\end{définition}
\begin{exemple}\jya tʂu jiʑora nɯ-ɣɤjom-i\cmn 我们把路扩大了\end{exemple}
\begin{exemple}\jya kɤntɕhaʁ ɣɯ tʂu nɯra ɲɯ-ɣɤjom-nɯ ɲɯ-ŋu tɕe, tɕe nɯtɕu ku-nɤma-nɯ tɕe, tɤ-zgra ɲɯ-wxti wo!\cmn 他们在把路修宽一点,所以很吵\end{exemple}
\begin{relation-sémantique}\confer{
\hyperlink{Ⓔjom}{\textit{ \papi{jom}}}
}\end{relation-sémantique}\end{entrée}

\begin{entrée}
\vedette{\hypertarget{Ⓔɣɤjpum}{\papi{ ɣɤjpum}}}\markboth{ɣɤjpum}{}
\begin{relation-sémantique}\confer{
\hyperlink{Ⓔjpum}{\textit{ \papi{jpum}}}
}\end{relation-sémantique}\end{entrée}

\begin{entrée}
\vedette{\hypertarget{Ⓔɣɤjqaʁ}{\papi{ ɣɤjqaʁ}}}\markboth{ɣɤjqaʁ}{}
\classe{vt}
\paradigme{\textit{dir :} \jya pɯ-}
\begin{définition}\fra se débarrasser\end{définition}
\begin{définition}\cmn 摆脱\end{définition}
\begin{exemple}\jya pɯ-ɣɤjqaʁ-a\cmn 我摆脱了\end{exemple}
\begin{exemple}\jya ɯʑo kɯ pjɤ-ɣɤjqaʁ\cmn 他摆脱了\end{exemple}
\begin{exemple}\jya kɤ-ɣɤjqaʁ me\cmn 无法摆脱\end{exemple}
\begin{exemple}\jya kɯ-ɲɟo tɤ-apa tɕe kɤ-ɣɤjqaʁ me\cmn 一旦灾难来了就无法摆脱\end{exemple}\end{entrée}

\begin{entrée}
\vedette{\hypertarget{Ⓔɣɤjru}{\papi{ ɣɤjru}}}\markboth{ɣɤjru}{}
\classe{vs}
\paradigme{\textit{dir :} \jya tɤ-}
\begin{définition}\fra aux gestes rapides\end{définition}
\begin{définition}\cmn 动作勤快\end{définition}
\begin{exemple}\jya ta-ma ɲɯ-ɣɤjru\cmn 他劳动的时候动作勤快\end{exemple}
\begin{exemple}\jya ta-ma ɲɯ-tɯ-ɣɤjru\cmn 你劳动的时候动作勤快\end{exemple}
\begin{relation-sémantique}\antonyme{
\hyperlink{Ⓔɣɯlaj}{\textit{ \papi{ɣɯlaj}}}
}\end{relation-sémantique}\end{entrée}

\begin{entrée}
\vedette{\hypertarget{Ⓔɣɤjtɯ}{\papi{ ɣɤjtɯ}}}\markboth{ɣɤjtɯ}{}
\begin{relation-sémantique}\confer{
\hyperlink{Ⓔajtɯ}{\textit{ \papi{ajtɯ}}}
}\end{relation-sémantique}\end{entrée}

\begin{entrée}
\vedette{\hypertarget{Ⓔɣɤjɯ}{\papi{ ɣɤjɯ}}}\markboth{ɣɤjɯ}{}\classe{vt}
\paradigme{\textit{dir :} \jya pɯ-}
\paradigme{\textit{dir :} \jya tɤ-}
\begin{définition}\fra ajouter\end{définition}
\begin{définition}\cmn 加,添加\end{définition}
\begin{exemple}\jya pɯ-ɣɤjɯ-t-a\cmn 我加了\end{exemple}
\begin{exemple}\jya pɯ-tɯ-ɣɤjɯ-t\cmn 你加了\end{exemple}
\begin{exemple}\jya pa-ɣɤjɯ\cmn 他加了\end{exemple}
\begin{exemple}\jya aʑo tɤ-ɣɤjɯ-t-a\cmn 我加了\end{exemple}
\begin{exemple}\jya ɯʑo ɣɯ ɯ-rŋɯl ɣurʑa ɣɤʑu tɕe, kɯβdɤ-sqi tɤ-ɣɤjɯ-t-a tɕe, tɕe lonba ɣurʑa kɯβdɤ-sqi tɤ-tu\cmn 他本来有一百元,我添了四十,他现在一共有了一百四十元\end{exemple}
\begin{exemple}\jya qro kɯ tɤ-kɤ-fɕɤt mɤʑɯ pjɯ́-wɣ-ɣɤjɯ ɲɯ-khɯ\cmn (编故事的时候)蚂蚁这个人物所讲的话,可以多加几句\end{exemple}
\begin{exemple}\jya a-@dian thɯ-ɣɤjɯ-t-a tɕe tɤ-amdzɯ-a, ku-ta-nɤjo\cmn 我已经充了电,坐下来了,我在等你\end{exemple}\end{entrée}

\begin{entrée}
\vedette{\hypertarget{Ⓔɣɤjwaʁ}{\papi{ ɣɤjwaʁ}}}\markboth{ɣɤjwaʁ}{}
\classe{vs}
\paradigme{\textit{dir :} \jya nɯ-}
\begin{définition}\fra pousser des feuilles\end{définition}
\begin{définition}\cmn 长出叶子\end{définition}
\begin{exemple}\jya χɕitka jɤ-ɣe tɕe, sɯku ɲɯ-ɣɤjwaʁ ɲɯ-ŋu\cmn 到了春天,树长出叶子\end{exemple}
\begin{relation-sémantique}\confer{
\hyperlink{Ⓔtɤ-jwaʁ}{\textit{ \papi{tɤ-jwaʁ}}}
}\end{relation-sémantique}
\begin{relation-sémantique}\confer{
\hyperlink{Ⓔrɤjwaʁ}{\textit{ \papi{rɤjwaʁ}}}
}\end{relation-sémantique}\end{entrée}

\begin{entrée}
\vedette{\hypertarget{Ⓔɣɤjwɤrlɤr}{\papi{ ɣɤjwɤrlɤr}}}\markboth{ɣɤjwɤrlɤr}{}\classe{vi}
\begin{définition}\fra être secoué\end{définition}
\begin{définition}\cmn 摇晃;摇摆不稳\end{définition}
\begin{exemple}\jya ʑmbrɯ nɯ tɯ-ɣɤjwɤrlɤr to-ʑa\cmn 船开始摇晃了\end{exemple}\begin{sous-entrée}
\vedette{\hypertarget{}{\papi{ sɤjwɤrlɤr}}}\markboth{sɤjwɤrlɤr}{}\classe{vt}
\begin{exemple}\jya khɯtsa ma-tɯ-sɤjwɤrlɤr ma tɯ-lwoʁ\cmn 碗不要摇,会(把水)倒出来\end{exemple}
\end{sous-entrée}\end{entrée}

\begin{entrée}
\vedette{\hypertarget{Ⓔɣɤɟaʁ}{\papi{ ɣɤɟaʁ}}}\markboth{ɣɤɟaʁ}{}
\classe{vt}
\paradigme{\textit{dir :} \jya kɤ-}
\begin{définition}\fra cajoler un enfant\end{définition}
\begin{définition}\cmn 哄\end{définition}
\begin{exemple}\jya tɤ-pɤtso kɤ-ɣɤɟaʁ-a\cmn 我哄了小孩子\end{exemple}
\begin{exemple}\jya kɤ-tɯ-ɣɤɟaʁ\cmn 你哄了他\end{exemple}
\begin{exemple}\jya ka-ɣɤɟaʁ\cmn 他哄了他\end{exemple}
\begin{exemple}\jya tɤ-pɤtso kú-wɣ-ɣɤɟaʁ tɕe rga\cmn 小孩子被哄就高兴\end{exemple}\end{entrée}

\begin{entrée}
\vedette{\hypertarget{Ⓔɣɤɟɯɣɟɯɣ}{\papi{ ɣɤɟɯɣɟɯɣ}}}\markboth{ɣɤɟɯɣɟɯɣ}{}\classe{vi}
\paradigme{\textit{dir :} \jya \_}
\begin{définition}\fra trembler, grouiller\end{définition}
\begin{définition}\cmn 发抖;蠕动\end{définition}
\begin{exemple}\jya ɯ-re ɲɯ-ɬoʁ tɕe ɲɯ-ɣɤɟɯɣɟɯɣ ʑo\cmn 他想笑,一身都在发抖\end{exemple}
\begin{exemple}\jya ɲɯ-ɣɤɟɯɣɟɯɣ ɲɯ-nɤre\cmn 他笑着(不发出声音、全身发抖)\end{exemple}\begin{sous-entrée}
\vedette{\hypertarget{}{\papi{ sɤɟɯɣɟɯɣ}}}\markboth{sɤɟɯɣɟɯɣ}{}\classe{vt}
\begin{définition}\fra se tourner dans tous les sens\end{définition}
\begin{définition}\cmn 扭动\end{définition}
\begin{exemple}\jya jla kɯ ɯ-βri ɲɯ-sɤɟɯɣɟɯɣ\cmn 犏牛扭动它的身体(驱赶苍蝇)\end{exemple}
\begin{exemple}\jya ɯ-tʂɯmpari ɲɯ-sɤɟɯɣɟɯɣ\end{exemple}
\begin{exemple}\jya jɤlwa ɲɯ-sɤɟɯɣɟɯɣ\cmn 他在扭动门帘\end{exemple}
\begin{exemple}\jya tɤɕi tɤ-rku-t-a tɕeri, mɯ́j-xtɕhɯt tɕe nɯ-sɤɟɯɣɟɯɣ-a tɕe mɤʑɯ tɤ-xtɕhɯt\cmn 我把青稞装在口袋里,装不下,抖动了一下就装得下了\end{exemple}
\end{sous-entrée}\end{entrée}

\begin{entrée}
\vedette{\hypertarget{Ⓔɣɤɟɯɣlɯɣ}{\papi{ ɣɤɟɯɣlɯɣ}}}\markboth{ɣɤɟɯɣlɯɣ}{}\classe{vi}
\begin{définition}\fra se tortiller\end{définition}
\begin{définition}\cmn 扭来扭去\end{définition}\end{entrée}

\begin{entrée}
\vedette{\hypertarget{Ⓔɣɤɟɯɟrɯɣ}{\papi{ ɣɤɟɯɟrɯɣ}}}\markboth{ɣɤɟɯɟrɯɣ}{} (\variante{ɣɤɟrɯɣɟrɯɣ}) \classe{vs}
\paradigme{\textit{dir :} \jya tɤ-}
\begin{définition}\fra gargouiller (ventre)\end{définition}
\begin{définition}\cmn 肚子咕噜叫\end{définition}
\begin{exemple}\jya znde ɲɯ-ɣɤɟɯɟrɯɣ pɯ-mbɯt\cmn 墙慢慢地塌下来了\end{exemple}
\begin{exemple}\jya a-xtu ɲɯ-ɣɤɟrɯɣɟrɯɣ\cmn 我肚子咕噜咕噜叫\end{exemple}\begin{sous-entrée}
\vedette{\hypertarget{}{\papi{ sɤɟɯɟrɯɣ}}}\markboth{sɤɟɯɟrɯɣ}{}\classe{vt}
\paradigme{\textit{dir :} \jya tɤ-}
\begin{définition}\fra faire du bruit (en démolissant un mur)\end{définition}
\begin{définition}\cmn 发出声音(拆墙的时候)\end{définition}
\begin{exemple}\jya znde tɤ-sɤɟɯɟrɯɣa pɯ-phɯt-a\cmn 我把墙拆了(发出很响的声音)\end{exemple}
\begin{relation-sémantique}\confer{
\hyperlink{Ⓔɟrɯɣɟrɯɣ}{\textit{ \papi{ɟrɯɣɟrɯɣ}}}
}\end{relation-sémantique}
\end{sous-entrée}\end{entrée}

\begin{entrée}
\vedette{\hypertarget{Ⓔɣɤkɤβjɤβ}{\papi{ ɣɤkɤβjɤβ}}}\markboth{ɣɤkɤβjɤβ}{}\classe{vi}
\paradigme{\textit{dir :} \jya nɯ-}
\begin{définition}\fra bouger dans tous les coins sans savoir quoi faire\end{définition}
\begin{définition}\cmn 急得到处乱动\end{définition}
\begin{exemple}\jya ɲɯ-mbɣom tɕe ɲɯ-ɣɤkɤβjɤβ\cmn 他很急,到处乱动\end{exemple}\begin{sous-entrée}
\vedette{\hypertarget{}{\papi{ sɤkɤβjɤβ}}}\markboth{sɤkɤβjɤβ}{}\classe{vt}
\paradigme{\textit{dir :} \jya tɤ-}
\begin{exemple}\jya tɤ-mbɣom-a ra tɤ-tɯt-a tɕe (tɤ-ɕɯmbɣom-a tɕe) tɤ-sɤkɤβjaβ-a ʑo\cmn 我叫他快点,(令)他(急得)到处乱动\end{exemple}
\begin{relation-sémantique}\confer{
\hyperlink{Ⓔɣɤqhɤβjɤβ}{\textit{ \papi{ɣɤqhɤβjɤβ}}}
}\end{relation-sémantique}
\end{sous-entrée}\end{entrée}

\begin{entrée}
\vedette{\hypertarget{Ⓔɣɤkhe}{\papi{ ɣɤkhe}}}\markboth{ɣɤkhe}{}
\begin{relation-sémantique}\confer{
\hyperlink{Ⓔkhe}{\textit{ \papi{khe}}}
}\end{relation-sémantique}\end{entrée}

\begin{entrée}
\vedette{\hypertarget{Ⓔɣɤkhrɤβjɤβ}{\papi{ ɣɤkhrɤβjɤβ}}}\markboth{ɣɤkhrɤβjɤβ}{}\classe{vi}
\begin{définition}\fra émettre un bruit de grattement incessant\end{définition}
\begin{définition}\cmn 不停地抓东西发出的声音
\end{définition}
\begin{exemple}\jya ɲɯ-ɣɤkhrɤβjɤβ ntsɯ\cmn 他不停地做事\end{exemple}
\begin{exemple}\jya βʑɯ ɲɯ-ɣɤkhrɤβjɤβ tɕe kɤ-ʑɣɤsɯndo\cmn 老鼠不停地抓东西发出声音,最后(被猫)抓到了\end{exemple}
\begin{exemple}\jya kɤ-rɤʑi mɯ́j-cha tɕe pjɯ-ɣɤkhrɤβjɤβ ntsɯ ŋu\cmn 他不能待在那里,不停地做事\end{exemple}\begin{sous-entrée}
\vedette{\hypertarget{}{\papi{ sɤkhrɤβjɤβ}}}\markboth{sɤkhrɤβjɤβ}{}\classe{vt}
\begin{exemple}\jya laχtɕha ra ma-tɯ-sɤkhrɤβjɤβ\cmn 你不要不停地弄那些东西,发出声音\end{exemple}
\end{sous-entrée}\end{entrée}

\begin{entrée}
\vedette{\hypertarget{Ⓔɣɤkhrɤβkhrɤβ}{\papi{ ɣɤkhrɤβkhrɤβ}}}\markboth{ɣɤkhrɤβkhrɤβ}{}\classe{vi}
\begin{définition}\fra émettre du bruit (en secouant un récipient qui contient de petits objets durs)\end{définition}
\begin{définition}\cmn 发出撞击的声音\end{définition}\end{entrée}

\begin{entrée}
\vedette{\hypertarget{Ⓔɣɤkhrɯɣlɯɣ}{\papi{ ɣɤkhrɯɣlɯɣ}}}\markboth{ɣɤkhrɯɣlɯɣ}{}
\begin{relation-sémantique}\confer{
\hyperlink{Ⓔkhrɯɣnɤkhrɯɣ}{\textit{ \papi{khrɯɣnɤkhrɯɣ}}}
}\end{relation-sémantique}\end{entrée}

\begin{entrée}
\vedette{\hypertarget{ⒺɣɤkhɯⒽ1}{\papi{ ɣɤkhɯ}}}\markboth{ɣɤkhɯ}{}\homonyme{1}
\classe{vi}
\paradigme{\textit{dir :} \jya tɤ-}
\begin{définition}\ 
\begin{déclaration}\grammar{denom}\end{déclaration}\end{définition}
\begin{définition}\fra être enfumé\end{définition}
\begin{définition}\cmn 有烟;冒烟\end{définition}
\begin{exemple}\jya kha ɲɯ-ɣɤkhɯ\cmn 满屋子都是烟\end{exemple}
\begin{exemple}\jya smi chɯ́-wɣ-βlɯ tɕe ɣɤkhɯ\cmn 烧火就会冒烟\end{exemple}
\begin{exemple}\jya ɲɯ-tɯ-ɣɤkhɯ\cmn 你家在冒烟(知道你在家里)\end{exemple}
\begin{relation-sémantique}\confer{
\hyperlink{Ⓔnɤkhɯ}{\textit{ \papi{nɤkhɯ}}}
}\end{relation-sémantique}
\begin{relation-sémantique}\confer{
\hyperlink{Ⓔtɤ-khɯ}{\textit{ \papi{tɤ-khɯ}}}
}\end{relation-sémantique}
\begin{relation-sémantique}\confer{
\hyperlink{Ⓔsɤkhɯ}{\textit{ \papi{sɤkhɯ}}}
}\end{relation-sémantique}\end{entrée}

\begin{entrée}
\vedette{\hypertarget{ⒺɣɤkhɯⒽ2}{\papi{ ɣɤkhɯ}}}\markboth{ɣɤkhɯ}{}\homonyme{2}
\classe{vt}
\begin{définition}\ 
\begin{déclaration}\grammar{caus}\end{déclaration}\end{définition}\acception{1}
\paradigme{\textit{dir :} \jya tɤ-}
\begin{définition}\fra forcer quelqu'un à faire quelque chose qu'il n'a pas envie de faire\end{définition}
\begin{définition}\cmn 让别人做 (他不愿意做的事情)\end{définition}
\begin{exemple}\jya tɤ-ɣɤkhɯ-t-a\cmn 我让他同意了\end{exemple}
\begin{exemple}\jya tɤ-ndzɯmbra-t-a tɕe tɤ-ɣɤkhɯ-t-a\cmn 经过我的劝说,他就同意了\end{exemple}\acception{2}
\begin{définition}\fra rendre possible\end{définition}
\begin{définition}\cmn 使……可以……\end{définition}
\begin{exemple}\jya kɯm tɤ-ɣɤβdi-t-a tɕe, kɤ-cɯ tɤ-ɣɤkhɯ-t-a\cmn 我修了门,使它可以打开了\end{exemple}
\begin{relation-sémantique}\confer{
\hyperlink{ⒺkhɯⒽ1}{\textit{ \papi{khɯ1}}}
}\end{relation-sémantique}\end{entrée}

\begin{entrée}
\vedette{\hypertarget{Ⓔɣɤla}{\papi{ ɣɤla}}}\markboth{ɣɤla}{}
\classe{vt}
\paradigme{\textit{dir :} \jya pɯ-}
\begin{définition}\ 
\begin{déclaration}\grammar{caus}\end{déclaration}\end{définition}
\begin{définition}\fra mouiller\end{définition}
\begin{définition}\cmn 泡软;浸泡在水里\end{définition}
\begin{exemple}\jya pɯ-ɣɤla-t-a\cmn 我把它泡软了\end{exemple}
\begin{exemple}\jya pɯ-ɣɤle\cmn 你把它泡一下吧!\end{exemple}
\begin{exemple}\jya tɯ-ci ɯ-ŋgɯ zɯ tɯ-ndʐi pɯ-ɣɤla-t-a\cmn 我把皮子浸泡在水里了\end{exemple}
\begin{relation-sémantique}\confer{
\hyperlink{Ⓔla}{\textit{ \papi{la}}}
}\end{relation-sémantique}\begin{sous-entrée}
\vedette{\hypertarget{}{\papi{ ʑɣɤɣɤla}}}\markboth{ʑɣɤɣɤla}{}\classe{vi}
\paradigme{\textit{dir :} \jya pɯ-}
\begin{définition}\ 
\begin{déclaration}\grammar{refl}\end{déclaration}
\begin{déclaration}\grammar{refl}\end{déclaration}\end{définition}
\begin{définition}\fra se baigner\end{définition}
\begin{définition}\cmn 沐浴\end{définition}
\end{sous-entrée}\end{entrée}

\begin{entrée}
\vedette{\hypertarget{Ⓔɣɤlɤt}{\papi{ ɣɤlɤt}}}\markboth{ɣɤlɤt}{}
\classe{vt}
\paradigme{\textit{dir :} \jya pɯ-}
\begin{définition}\fra fermer à clé\end{définition}
\begin{définition}\cmn 锁门\end{définition}
\begin{exemple}\jya pɯ-ɣɤlat-a\cmn 我锁了\end{exemple}
\begin{exemple}\jya pa-ɣɤlɤt\cmn 他锁了\end{exemple}
\begin{exemple}\jya sɤcɯ pɯ-ɣɤlat-a\cmn 我锁了\end{exemple}
\begin{exemple}\jya kɯm nɯ pɯ-ɣɤ-lat-a\cmn 我锁了门\end{exemple}
\begin{exemple}\jya khɯzgɯr pɯ-ɣɤlat-a\cmn 我锁了\end{exemple}
\begin{relation-sémantique}\confer{
\hyperlink{ⒺlɤtⒽ1}{\textit{ \papi{lɤt1}}}
}\end{relation-sémantique}\end{entrée}

\begin{entrée}
\vedette{\hypertarget{Ⓔɣɤloŋloŋ}{\papi{ ɣɤloŋloŋ}}}\markboth{ɣɤloŋloŋ}{}
\begin{relation-sémantique}\confer{
\hyperlink{Ⓔloŋloŋ}{\textit{ \papi{loŋloŋ}}}
}\end{relation-sémantique}\end{entrée}

\begin{entrée}
\vedette{\hypertarget{Ⓔɣɤltshɤltshɤt}{\papi{ ɣɤltshɤltshɤt}}}\markboth{ɣɤltshɤltshɤt}{}
\begin{relation-sémantique}\confer{
\hyperlink{Ⓔsɤltshɤltshɤt}{\textit{ \papi{sɤltshɤltshɤt}}}
}\end{relation-sémantique}
\end{entrée}

\begin{entrée}
\vedette{\hypertarget{Ⓔɣɤlɯrlɯr}{\papi{ ɣɤlɯrlɯr}}}\markboth{ɣɤlɯrlɯr}{}
\begin{relation-sémantique}\confer{
\hyperlink{Ⓔlɯrlɯr}{\textit{ \papi{lɯrlɯr}}}
}\end{relation-sémantique}\end{entrée}

\begin{entrée}
\vedette{\hypertarget{Ⓔɣɤlɯzlɯz}{\papi{ ɣɤlɯzlɯz}}}\markboth{ɣɤlɯzlɯz}{}\classe{vi}
\begin{définition}\fra se secouer\end{définition}
\begin{définition}\cmn (自动地)摇动\end{définition}\begin{sous-entrée}
\vedette{\hypertarget{}{\papi{ sɤlɯzlɯz}}}\markboth{sɤlɯzlɯz}{}\classe{vt}
\paradigme{\textit{dir :} \jya nɯ-}
\begin{définition}\fra secouer\end{définition}
\begin{définition}\cmn 摇动\end{définition}
\end{sous-entrée}\end{entrée}

\begin{entrée}
\vedette{\hypertarget{Ⓔɣɤlwɤlwɤt}{\papi{ ɣɤlwɤlwɤt}}}\markboth{ɣɤlwɤlwɤt}{}\classe{vi}
\paradigme{\textit{dir :} \jya tɤ-}
\begin{définition}\fra s’agiter\end{définition}
\begin{définition}\cmn 飘动\end{définition}
\begin{relation-sémantique}\confer{
\hyperlink{Ⓔsɤlwɤlwɤt}{\textit{ \papi{sɤlwɤlwɤt}}}
}\end{relation-sémantique}\end{entrée}

\begin{entrée}
\vedette{\hypertarget{Ⓔɣɤɬɤt}{\papi{ ɣɤɬɤt}}}\markboth{ɣɤɬɤt}{}
\classe{vt}
\begin{définition}\fra détendre\end{définition}
\begin{définition}\cmn 放松\end{définition}
\begin{exemple}\jya a-rɕa ɲɯ-ɣɤɬat-a ɲɯ-ra\cmn 我要放松一下\end{exemple}
\begin{exemple}\jya nɤ-rɕa nɯ-ɣɤɬɤt tɕe nɤ-mgɯr mɤ-mŋɤm\cmn 你放松一下,你的背就不会痛了\end{exemple}\end{entrée}

\begin{entrée}
\vedette{\hypertarget{Ⓔɣɤmaʁ}{\papi{ ɣɤmaʁ}}}\markboth{ɣɤmaʁ}{}\classe{vt}
\paradigme{\textit{dir :} \jya nɯ-}\acception{1}
\begin{définition}\fra retirer\end{définition}
\begin{définition}\cmn 取消;让……没有……\end{définition}
\begin{exemple}\jya kɯki nɤj nɤ-kho nɯ-ɣɤmaʁ-a\cmn 我让你没有这个房子了\end{exemple}\acception{2}
\begin{définition}\fra relever de ses fonctions\end{définition}
\begin{définition}\cmn 免职\end{définition}
\begin{exemple}\jya ɯ-khɯrthaŋ nɯ-ɣɤmaʁ-a\cmn 我罢免了他的官职\end{exemple}
\begin{relation-sémantique}\confer{
\hyperlink{ⒺmaʁⒽ1}{\textit{ \papi{maʁ}}}
}\end{relation-sémantique}\end{entrée}

\begin{entrée}
\vedette{\hypertarget{Ⓔɣɤmba}{\papi{ ɣɤmba}}}\markboth{ɣɤmba}{}
\begin{relation-sémantique}\confer{
\hyperlink{Ⓔmba}{\textit{ \papi{mba}}}
}\end{relation-sémantique}\end{entrée}

\begin{entrée}
\vedette{\hypertarget{Ⓔɣɤmbat}{\papi{ ɣɤmbat}}}\markboth{ɣɤmbat}{}
\begin{relation-sémantique}\confer{
\hyperlink{Ⓔmbat}{\textit{ \papi{mbat}}}
}\end{relation-sémantique}\end{entrée}

\begin{entrée}
\vedette{\hypertarget{Ⓔɣɤmbɤr}{\papi{ ɣɤmbɤr}}}\markboth{ɣɤmbɤr}{}\classe{vt}
\paradigme{\textit{dir :} \jya pɯ-}
\begin{définition}\ 
\begin{déclaration}\grammar{caus}\end{déclaration}\end{définition}
\begin{définition}\fra abaisser, rendre moins haut\end{définition}
\begin{définition}\cmn 弄低;弄矮\end{définition}
\begin{exemple}\jya ɯʑo kɯ ɯ-phoŋbu pa-ɣɤmbɤr\cmn 他稍微低下了身子\end{exemple}
\begin{exemple}\jya kha nɯ pjɯ́-ɣw-ɣɤmbɤr tsa jɤɣ\cmn 要把房子弄得矮一点\end{exemple}
\begin{exemple}\jya ɕomskrɯt tɯ-ŋga ɯ-sɤ-ɕkho tɤ-kɤ-βzu nɯ ɲɯ-mbro tɕe, pɯ-ɣɤ-mbara\cmn 因为晒衣服的铁丝弄的太高,我把它放低\end{exemple}
\begin{exemple}\jya ɯ-koŋ pɯ-ɣɤ-mbara\cmn 我减价了\end{exemple}
\begin{exemple}\jya pa-nɯβʑit tɕe pa-ɣɤ-mbɤr\cmn 他锯了一段,把它弄矮了一点\end{exemple}
\begin{relation-sémantique}\confer{
\hyperlink{ⒺmbɤrⒽ1}{\textit{ \papi{mbɤr1}}}
}\end{relation-sémantique}\end{entrée}

\begin{entrée}
\vedette{\hypertarget{Ⓔɣɤmbɤrmbɤr}{\papi{ ɣɤmbɤrmbɤr}}}\markboth{ɣɤmbɤrmbɤr}{}
\begin{relation-sémantique}\confer{
\hyperlink{Ⓔɣɤɲɟɤrɲɟɤr}{\textit{ \papi{ɣɤɲɟɤrɲɟɤr}}}
}\end{relation-sémantique}\end{entrée}

\begin{entrée}
\vedette{\hypertarget{Ⓔɣɤmbɣaʁ}{\papi{ ɣɤmbɣaʁ}}}\markboth{ɣɤmbɣaʁ}{}
\begin{relation-sémantique}\confer{
\hyperlink{Ⓔmbɣaʁ}{\textit{ \papi{mbɣaʁ}}}
}\end{relation-sémantique}\end{entrée}

\begin{entrée}
\vedette{\hypertarget{Ⓔɣɤmbɣo}{\papi{ ɣɤmbɣo}}}\markboth{ɣɤmbɣo}{}
\classe{vs}
\paradigme{\textit{dir :} \jya kɤ-}
\begin{définition}\ 
\begin{déclaration}\grammar{denom}\end{déclaration}\end{définition}
\begin{définition}\fra être sourd\end{définition}
\begin{définition}\cmn 聋\end{définition}
\begin{exemple}\jya ɯ-rna mɯ́j-mtshɤm ɲɯ-ɣɤmbɣo\cmn 他耳朵听不见,他是聋的\end{exemple}
\begin{relation-sémantique}\confer{
\hyperlink{Ⓔtɤmbɣo}{\textit{ \papi{tɤmbɣo}}}
}\end{relation-sémantique}\end{entrée}

\begin{entrée}
\vedette{\hypertarget{Ⓔɣɤmbɣomru}{\papi{ ɣɤmbɣomru}}}\markboth{ɣɤmbɣomru}{}
\classe{vs}
\begin{définition}\fra impatient, pressé\end{définition}
\begin{définition}\cmn 急躁\end{définition}
\begin{exemple}\jya nɤʑo dal tɤ-pe ma-tɯ-ɣɤmbɣomru\cmn 你慢慢做,不要急躁\end{exemple}
\begin{relation-sémantique}\confer{
\hyperlink{Ⓔmbɣom}{\textit{ \papi{mbɣom}}}
}\end{relation-sémantique}\end{entrée}

\begin{entrée}
\vedette{\hypertarget{Ⓔɣɤmbro}{\papi{ ɣɤmbro}}}\markboth{ɣɤmbro}{}
\begin{relation-sémantique}\confer{
\hyperlink{ⒺmbroⒽ1}{\textit{ \papi{mbro}}}
}\end{relation-sémantique}\end{entrée}

\begin{entrée}
\vedette{\hypertarget{Ⓔɣɤmdzu}{\papi{ ɣɤmdzu}}}\markboth{ɣɤmdzu}{}
\classe{vs}
\paradigme{\textit{dir :} \jya tɤ-}
\begin{définition}\ 
\begin{déclaration}\grammar{denom}\end{déclaration}\end{définition}
\begin{définition}\fra avoir beaucoup d'épines\end{définition}
\begin{définition}\cmn 有刺\end{définition}
\begin{exemple}\jya zɲɟa ɲɯ-ɣɤmdzu\cmn 黄刺泡儿有刺\end{exemple}
\begin{relation-sémantique}\confer{
\hyperlink{Ⓔtɤ-mdzu}{\textit{ \papi{tɤ-mdzu}}}
}\end{relation-sémantique}\end{entrée}

\begin{entrée}
\vedette{\hypertarget{Ⓔɣɤme}{\papi{ ɣɤme}}}\markboth{ɣɤme}{}
\classe{vt}
\paradigme{\textit{dir :} \jya nɯ-}
\begin{définition}\ 
\begin{déclaration}\grammar{caus}\end{déclaration}\end{définition}
\begin{définition}\fra perdre\end{définition}
\begin{définition}\cmn 弄丢\end{définition}
\begin{exemple}\jya nɯ-ɣɤme-t-a\cmn 我丢了\end{exemple}
\begin{exemple}\jya nɯ-tɯ-ɣɤme-t\cmn 你丢了\end{exemple}
\begin{exemple}\jya na-ɣɤme\cmn 他丢了\end{exemple}
\begin{exemple}\jya jɯɣi ɯ-tɯ-ɣɤme nɯ ?\cmn 那本书那么快就丢了?\end{exemple}
\begin{exemple}\jya a-laχtɕha ɲɤ-ɣɤme\cmn 我把我的东西弄丢了\end{exemple}\begin{sous-entrée}
\vedette{\hypertarget{}{\papi{ sna,ɣɤme}}}\markboth{sna,ɣɤme}{}
\begin{définition}\fra abîmer\end{définition}
\begin{définition}\cmn 弄烂\end{définition}
\begin{exemple}\jya tɯ-ŋga tɤ-ŋga-t-a tɕe, thɯ-ɴɢraʁ tɕe sna nɯ-ɣɤme-t-a\cmn 我穿衣服的时候就破了,把它弄烂了\end{exemple}
\end{sous-entrée}\begin{sous-entrée}
\vedette{\hypertarget{}{\papi{ ɯ-rca,ɣɤme}}}\markboth{ɯ-rca,ɣɤme}{}
\begin{définition}\fra mettre en désordre\end{définition}
\begin{définition}\cmn 弄得很乱,令人无从做起\end{définition}
\begin{exemple}\jya a-rca ci na-ɣɤme\cmn 他把事情弄得很乱,令我无从做起\end{exemple}
\begin{exemple}\jya kɯki kɤ-nɤma ki tu-sɤpe-a nɯ-sɯso-t-a pɯ-ŋu ri, chɤ-nɯkɯmaʁ-a tɕe ɯ-rca ci ɲɤ-ɣɤme-t-a\cmn 我本来以为会把这个工作做好,但是弄错了,弄得很乱了\end{exemple}
\end{sous-entrée}\begin{sous-entrée}
\vedette{\hypertarget{}{\papi{ ʑɣɤɣɤme}}}\markboth{ʑɣɤɣɤme}{}\classe{vi}
\paradigme{\textit{dir :} \jya nɯ-}
\begin{définition}\ 
\begin{déclaration}\grammar{caus}\end{déclaration}
\begin{déclaration}\grammar{refl}\end{déclaration}\end{définition}
\begin{définition}\fra disparaître\end{définition}
\begin{définition}\cmn 消失(躲起来,不让别人发现自己)\end{définition}
\begin{relation-sémantique}\confer{
\hyperlink{ⒺmeⒽ1}{\textit{ \papi{me}}}
}\end{relation-sémantique}
\end{sous-entrée}\end{entrée}

\begin{entrée}
\vedette{\hypertarget{Ⓔɣɤmi}{\papi{ ɣɤmi}}}\markboth{ɣɤmi}{}
\classe{vt}
\begin{définition}\ 
\begin{déclaration}\grammar{caus}\end{déclaration}\end{définition}\acception{1}
\paradigme{\textit{dir :} \jya nɯ-}
\begin{définition}\fra éteindre la lumière\end{définition}
\begin{définition}\cmn 关(灯)\end{définition}
\begin{exemple}\jya nɯ-ɣɤmi-t-a\cmn 我关了(灯)\end{exemple}
\begin{exemple}\jya nɯ-tɯ-ɣɤmi-t\cmn 你关了(灯)\end{exemple}
\begin{exemple}\jya tɤtʂu a-nɯ-ɣɤmi\cmn 要关灯\end{exemple}\acception{2}
\paradigme{\textit{dir :} \jya pɯ-}
\begin{définition}\fra éteindre un feu\end{définition}
\begin{définition}\cmn 灭(火)\end{définition}
\begin{exemple}\jya smi pɯ-ɣɤmi-t-a\cmn 我灭了火\end{exemple}
\begin{relation-sémantique}\confer{
\hyperlink{ⒺmiⒽ1}{\textit{ \papi{mi1}}}
}\end{relation-sémantique}\end{entrée}

\begin{entrée}
\vedette{\hypertarget{Ⓔɣɤmna}{\papi{ ɣɤmna}}}\markboth{ɣɤmna}{}
\classe{vt}
\paradigme{\textit{dir :} \jya tɤ-}
\begin{définition}\ 
\begin{déclaration}\grammar{caus}\end{déclaration}\end{définition}
\begin{définition}\fra guérir une maladie\end{définition}
\begin{définition}\cmn 治病\end{définition}
\begin{exemple}\jya tɤ-ɣɤmna-t-a\cmn 我治了病\end{exemple}
\begin{exemple}\jya tɤ-tɯ-ɣɤmna-t\cmn 你治了病\end{exemple}
\begin{exemple}\jya ta-ɣɤmna\cmn 他治了病\end{exemple}
\begin{exemple}\jya a-kɯ-mŋɤm ta-ɣɤmna\cmn 他治了我的病\end{exemple}
\begin{exemple}\jya smɤnba kɯ ɯ-kɯ-mŋɤm ta-ɣɤmna\cmn 医生治了他的病\end{exemple}
\begin{exemple}\jya lonba kɤ-ɣɤmna mɯ́j-khɯ\cmn 不能完全治好\end{exemple}\begin{sous-entrée}
\vedette{\hypertarget{}{\papi{ ɣɤmna}}}\markboth{ɣɤmna}{}\classe{vs}
\begin{définition}\fra guérir facilement\end{définition}
\begin{définition}\cmn 容易痊愈\end{définition}
\begin{relation-sémantique}\confer{
\hyperlink{Ⓔmna}{\textit{ \papi{mna}}}
}\end{relation-sémantique}
\end{sous-entrée}\end{entrée}

\begin{entrée}
\vedette{\hypertarget{Ⓔɣɤmɲi}{\papi{ ɣɤmɲi}}}\markboth{ɣɤmɲi}{}
\begin{relation-sémantique}\confer{
\hyperlink{Ⓔmɲi}{\textit{ \papi{mɲi}}}
}\end{relation-sémantique}\end{entrée}

\begin{entrée}
\vedette{\hypertarget{Ⓔɣɤmpɕu}{\papi{ ɣɤmpɕu}}}\markboth{ɣɤmpɕu}{}
\begin{relation-sémantique}\confer{
\hyperlink{Ⓔmpɕu}{\textit{ \papi{mpɕu}}}
}\end{relation-sémantique}\end{entrée}

\begin{entrée}
\vedette{\hypertarget{Ⓔɣɤmpɕɤr}{\papi{ ɣɤmpɕɤr}}}\markboth{ɣɤmpɕɤr}{}
\begin{relation-sémantique}\confer{
\hyperlink{Ⓔmpɕɤr}{\textit{ \papi{mpɕɤr}}}
}\end{relation-sémantique}\end{entrée}

\begin{entrée}
\vedette{\hypertarget{Ⓔɣɤmpja}{\papi{ ɣɤmpja}}}\markboth{ɣɤmpja}{}
\classe{vt}
\paradigme{\textit{dir :} \jya tɤ-}
\paradigme{\textit{dir :} \jya kɤ-}
\begin{définition}\ 
\begin{déclaration}\grammar{caus}\end{déclaration}\end{définition}
\begin{définition}\fra chauffer, réchauffer\end{définition}
\begin{définition}\cmn 加热\end{définition}
\begin{exemple}\jya tɯ-ndza tɤ-ɣɤmpje\cmn 你把饭热了吧\end{exemple}
\begin{exemple}\jya smi ɯ-taʁ kú-wɣ-ɣɤmpja\cmn 在火上加热\end{exemple}\end{entrée}

\begin{entrée}
\vedette{\hypertarget{Ⓔɣɤmpɯ}{\papi{ ɣɤmpɯ}}}\markboth{ɣɤmpɯ}{}
\begin{relation-sémantique}\confer{
\hyperlink{Ⓔmpɯ}{\textit{ \papi{mpɯ}}}
}\end{relation-sémantique}\end{entrée}

\begin{entrée}
\vedette{\hypertarget{Ⓔɣɤmqrɯz}{\papi{ ɣɤmqrɯz}}}\markboth{ɣɤmqrɯz}{}\classe{vi}
\begin{définition}\fra qui fait mal aux pieds lorsque l'on marche dessus\end{définition}
\begin{définition}\cmn 硌脚,地面很粗糙的时候,不穿鞋子走上去感到痛、不舒服的感觉\end{définition}
\begin{exemple}\jya ɯ-thoʁ nɯ ɲɯ-ɣɤmqrɯz\cmn 地面很硌脚\end{exemple}\begin{sous-entrée}
\vedette{\hypertarget{}{\papi{ znɤmqrɯz}}}\markboth{znɤmqrɯz}{}\classe{vt}
\paradigme{\textit{dir :} \jya tɤ-}\acception{1}
\begin{définition}\fra faire mal aux pieds (d'une surface rugueuse, lorsque l'on marche dessus sans chaussures)\end{définition}
\begin{définition}\cmn 硌脚,地面很粗糙的时候,不穿鞋子走上去感到痛,不舒服的感觉\end{définition}
\begin{exemple}\jya ɯ-thoʁ ɲɯ-rʁom tɕe, a-mi ɲɯ-znɤmqrɯz ma a-xtsa maŋe\cmn 地面很粗糙,令我的脚很痛因为我没有穿鞋子\end{exemple}\acception{2}
\begin{définition}\fra causer une sensation acide\end{définition}
\begin{définition}\cmn 酸到\end{définition}
\begin{exemple}\jya paχɕi kɯ-tɕur tɤ-ndza-t-a tɕe a-ɕɣa to-znɤmqrɯz\cmn 我吃了酸的苹果,牙齿被酸到了\end{exemple}
\end{sous-entrée}\end{entrée}

\begin{entrée}
\vedette{\hypertarget{Ⓔɣɤmtɕoʁ}{\papi{ ɣɤmtɕoʁ}}}\markboth{ɣɤmtɕoʁ}{}
\begin{relation-sémantique}\confer{
\hyperlink{Ⓔmtɕoʁ}{\textit{ \papi{mtɕoʁ}}}
}\end{relation-sémantique}\end{entrée}

\begin{entrée}
\vedette{\hypertarget{Ⓔɣɤmthu}{\papi{ ɣɤmthu}}}\markboth{ɣɤmthu}{}\classe{vt}
\paradigme{\textit{dir :} \jya nɯ-}
\begin{définition}\ 
\begin{déclaration}\use{只用于否定式}\end{déclaration}\end{définition}
\begin{définition}\fra rendre faible\end{définition}
\begin{définition}\cmn 令……虚弱\end{définition}
\begin{exemple}\jya a-kɯ-mŋɤm kɯ mɯ-nɯ́-wɣ-ɣɤmthu-a\cmn 我的病令我变得很虚弱\end{exemple}\end{entrée}

\begin{entrée}
\vedette{\hypertarget{Ⓔɣɤmto}{\papi{ ɣɤmto}}}\markboth{ɣɤmto}{}
\begin{relation-sémantique}\confer{
 \papi{mto}
}\end{relation-sémantique}\end{entrée}

\begin{entrée}
\vedette{\hypertarget{Ⓔɣɤmtsɯr}{\papi{ ɣɤmtsɯr}}}\markboth{ɣɤmtsɯr}{}
\begin{relation-sémantique}\confer{
\hyperlink{Ⓔmtsɯr}{\textit{ \papi{mtsɯr}}}
}\end{relation-sémantique}\end{entrée}

\begin{entrée}
\vedette{\hypertarget{Ⓔɣɤmɯ}{\papi{ ɣɤmɯ}}}\markboth{ɣɤmɯ}{}
\classe{vt}
\paradigme{\textit{dir :} \jya pɯ-}
\begin{définition}\fra louer\end{définition}
\begin{définition}\cmn 称赞;表扬\end{définition}
\begin{exemple}\jya aʑo kɯ pjɯ-ɣɤmi-a\cmn 我表扬他\end{exemple}
\begin{exemple}\jya ɲɯ-mkhɤz tɤ-tɯt-a tɕe, pɯ-ɣɤmɯt-a\cmn 我说“他很厉害”,表扬了他\end{exemple}
\begin{exemple}\jya kɯki kɤ-βzjoz ɯʑo ɲɯ-mkhɤz, tɕe pjɤ-ɣɤmɯ\cmn 他学得很好,所以他(另外一个人)表扬了他\end{exemple}
\begin{exemple}\jya ɯ-kɯmdza (ɯ-βzaŋsa) ɲɯ-ɣɤmi\cmn 他称赞他的亲戚(朋友)\end{exemple}\begin{sous-entrée}
\vedette{\hypertarget{}{\papi{ sɤɣɤmɯ}}}\markboth{sɤɣɤmɯ}{} (\variante{sɤzɣɤmɯ}) \classe{vi}
\begin{définition}\ 
\begin{déclaration}\grammar{apass}\end{déclaration}\end{définition}
\begin{définition}\fra louer des gens\end{définition}
\begin{définition}\cmn 表扬人\end{définition}
\end{sous-entrée}\end{entrée}

\begin{entrée}
\vedette{\hypertarget{Ⓔɣɤmɯm}{\papi{ ɣɤmɯm}}}\markboth{ɣɤmɯm}{}
\begin{relation-sémantique}\confer{
\hyperlink{Ⓔmɯm}{\textit{ \papi{mɯm}}}
}\end{relation-sémantique}\end{entrée}

\begin{entrée}
\vedette{\hypertarget{Ⓔɣɤmɯrmɯr}{\papi{ ɣɤmɯrmɯr}}}\markboth{ɣɤmɯrmɯr}{}\classe{vs}
\begin{définition}\fra (surface de l'eau) ayant des rides\end{définition}
\begin{définition}\cmn 有轻微的波纹\end{définition}
\begin{exemple}\jya tɯ-ci ɲɯ-ɣɤmɯrmɯr ʑo\cmn 水面上有轻微的波纹\end{exemple}\end{entrée}

\begin{entrée}
\vedette{\hypertarget{Ⓔɣɤmɯt}{\papi{ ɣɤmɯt}}}\markboth{ɣɤmɯt}{}
\classe{vt}
\paradigme{\textit{dir :} \jya thɯ-}
\begin{définition}\fra souffler\end{définition}
\begin{définition}\cmn 吹(灰)\end{définition}
\begin{exemple}\jya thɯ-ɣɤmɯt-a\cmn 我吹了\end{exemple}
\begin{exemple}\jya ɯʑo kɯ tha-ɣɤmɯt\cmn 他吹了\end{exemple}\end{entrée}

\begin{entrée}
\vedette{\hypertarget{Ⓔɣɤndɣɤndɣɤt}{\papi{ ɣɤndɣɤndɣɤt}}}\markboth{ɣɤndɣɤndɣɤt}{}
\classe{vs}
\paradigme{\textit{dir :} \jya nɯ-}
\begin{définition}\ 
\begin{déclaration}\grammar{deidph}\end{déclaration}\end{définition}
\begin{définition}\fra trembler\end{définition}
\begin{définition}\cmn 颤抖;震动\end{définition}\begin{sous-entrée}
\vedette{\hypertarget{}{\papi{ sɤndɣɤndɣɤt}}}\markboth{sɤndɣɤndɣɤt}{}\classe{vt}
\begin{définition}\fra faire trembler\end{définition}
\begin{définition}\cmn 使震动\end{définition}
\begin{exemple}\jya kha ɲɯ-sɤndɣɤndɣɤt\cmn 他令屋子震动\end{exemple}
\begin{exemple}\jya khɤxtu ɲɯ-nɤmdɯmdar tɕe ɲɯ-sɤndɣɤndɣɤt\cmn 他在房背上东跳西跳,令屋子震动\end{exemple}
\begin{relation-sémantique}\confer{
\hyperlink{Ⓔndɣɤndɣɤt}{\textit{ \papi{ndɣɤndɣɤt}}}
}\end{relation-sémantique}
\end{sous-entrée}\end{entrée}

\begin{entrée}
\vedette{\hypertarget{Ⓔɣɤndʐo}{\papi{ ɣɤndʐo}}}\markboth{ɣɤndʐo}{}
\classe{vs}
\paradigme{\textit{dir :} \jya thɯ-}
\begin{définition}\fra froid (temps)\end{définition}
\begin{définition}\cmn 冷(天气)\end{définition}
\begin{exemple}\jya qartsɯ jo-ɣi ɲɯ-ɣɤndʐo\cmn 到了冬天,天气很冷\end{exemple}
\begin{exemple}\jya qale ɲɤ-sɯβzu, ɲɯ-ɣɤndʐo\cmn 在刮风,很冷\end{exemple}
\begin{relation-sémantique}\confer{
\hyperlink{Ⓔnɤndʐo}{\textit{ \papi{nɤndʐo}}}
}\end{relation-sémantique}
\begin{relation-sémantique}\confer{
\hyperlink{Ⓔtɤndʐo}{\textit{ \papi{tɤndʐo}}}
}\end{relation-sémantique}\end{entrée}

\begin{entrée}
\vedette{\hypertarget{Ⓔɣɤndɯβ}{\papi{ ɣɤndɯβ}}}\markboth{ɣɤndɯβ}{}
\classe{vt}
\paradigme{\textit{dir :} \jya pɯ-}
\paradigme{\textit{dir :} \jya nɯ-}
\begin{définition}\ 
\begin{déclaration}\grammar{caus}\end{déclaration}\end{définition}
\begin{définition}\fra écraser\end{définition}
\begin{définition}\cmn 弄碎
\begin{déclaration}\use{趋向前缀的用法:\stylefv{nɯ}- 用牙齿、手磨弄碎;\stylefv{pɯ}- 用刀弄碎}\end{déclaration}\end{définition}
\begin{exemple}\jya aʑo pɯ-ɣɤndɯβ-a\cmn 我弄碎了\end{exemple}
\begin{exemple}\jya ɯʑo kɯ pa-ɣɤndɯβ\cmn 他弄碎了\end{exemple}
\begin{exemple}\jya paʁndza kɤ-rɤkrɯ pa-ɣɤndɯβ\cmn 他把猪草切得很细\end{exemple}
\begin{relation-sémantique}\confer{
\hyperlink{Ⓔndɯβ}{\textit{ \papi{ndɯβ}}}
}\end{relation-sémantique}\end{entrée}

\begin{entrée}
\vedette{\hypertarget{Ⓔɣɤndɯl}{\papi{ ɣɤndɯl}}}\markboth{ɣɤndɯl}{}
\begin{relation-sémantique}\confer{
\hyperlink{ⒺndɯlⒽ1}{\textit{ \papi{ndɯl1}}}
}\end{relation-sémantique}\end{entrée}

\begin{entrée}
\vedette{\hypertarget{Ⓔɣɤndziaʁ}{\papi{ ɣɤndziaʁ}}}\markboth{ɣɤndziaʁ}{}
\begin{relation-sémantique}\confer{
\hyperlink{Ⓔndziaʁ}{\textit{ \papi{ndziaʁ}}}
}\end{relation-sémantique}\end{entrée}

\begin{entrée}
\vedette{\hypertarget{Ⓔɣɤndzɯrndzɯr}{\papi{ ɣɤndzɯrndzɯr}}}\markboth{ɣɤndzɯrndzɯr}{}\classe{vs}
\paradigme{\textit{dir :} \jya nɯ-}
\begin{définition}\fra trembler\end{définition}
\begin{définition}\cmn 发抖\end{définition}
\begin{exemple}\jya ɯ-βri ɲɯ-ɣɤndzɯrndzɯr\cmn 他的身体在发抖\end{exemple}
\begin{exemple}\jya ɲo-mu tɕe ɲɯ-ɣɤndzɯrndzɯr\cmn 他怕了,全身在发抖\end{exemple}\begin{sous-entrée}
\vedette{\hypertarget{}{\papi{ sɤndzɯrndzɯr}}}\markboth{sɤndzɯrndzɯr}{}\classe{vt}
\begin{définition}\fra faire trembler\end{définition}
\begin{définition}\cmn 使发抖\end{définition}
\begin{relation-sémantique}\synonyme{
\hyperlink{Ⓔɣɤthɣɤthɣɤt}{\textit{ \papi{ɣɤthɣɤthɣɤt}}}
}\end{relation-sémantique}
\end{sous-entrée}\end{entrée}

\begin{entrée}
\vedette{\hypertarget{Ⓔɣɤndʑɤm}{\papi{ ɣɤndʑɤm}}}\markboth{ɣɤndʑɤm}{}
\classe{vt}
\paradigme{\textit{dir :} \jya tɤ-}
\begin{définition}\ 
\begin{déclaration}\grammar{caus}\end{déclaration}\end{définition}
\begin{définition}\fra chauffer la nourriture\end{définition}
\begin{définition}\cmn 热(饭)\end{définition}
\begin{exemple}\jya ɯʑo kɯ ta-ɣɤndʑɤm\cmn 你热了一下\end{exemple}
\begin{exemple}\jya aʑo tu-ɣɤndʑam-a\cmn 我热一下\end{exemple}
\begin{exemple}\jya ki tɤ-lu ki nɤʑo tɤ-ɣɤndʑɤm\cmn 你把牛奶热一下\end{exemple}
\begin{relation-sémantique}\confer{
\hyperlink{Ⓔndʑɤm}{\textit{ \papi{ndʑɤm}}}
}\end{relation-sémantique}\end{entrée}

\begin{entrée}
\vedette{\hypertarget{Ⓔɣɤndʑɣɤrlɤr}{\papi{ ɣɤndʑɣɤrlɤr}}}\markboth{ɣɤndʑɣɤrlɤr}{}\classe{vi}
\begin{définition}\fra piétiner de partout\end{définition}
\begin{définition}\cmn 到处乱踩;动作不雅观\end{définition}
\begin{exemple}\jya a-tɯji ɯ-ŋgɯ ma-tɯ-ɣɤndʑɣɤrlɤr ma a-kɤrkɯm tɯ-rɤtɕaʁ\cmn 你不要在我的田地里乱踩,会把我的菜苗踩死\end{exemple}\end{entrée}

\begin{entrée}
\vedette{\hypertarget{Ⓔɣɤngɯt}{\papi{ ɣɤngɯt}}}\markboth{ɣɤngɯt}{}
\begin{relation-sémantique}\confer{
\hyperlink{Ⓔngɯt}{\textit{ \papi{ngɯt}}}
}\end{relation-sémantique}\end{entrée}

\begin{entrée}
\vedette{\hypertarget{Ⓔɣɤntaβ}{\papi{ ɣɤntaβ}}}\markboth{ɣɤntaβ}{}\classe{vt}
\paradigme{\textit{dir :} \jya \_}
\begin{définition}\ 
\begin{déclaration}\grammar{caus}\end{déclaration}\end{définition}
\begin{définition}\fra poser\end{définition}
\begin{définition}\cmn 放置\end{définition}
\begin{exemple}\jya ka-ɣɤntaβ\cmn 他放了\end{exemple}
\begin{exemple}\jya aʑo kɯki laχtɕha ki tɕɤkɯ zɯ kɤ-ɣɤntaβ-a\cmn 我把这个东西放在那边了\end{exemple}
\begin{exemple}\jya @chabei tɕɤkɯ ka-ɣɤntaβ\cmn 他把茶杯放在那边了\end{exemple}
\begin{exemple}\jya kɤ-nɤma mɯ-mɤ-ɲɯ-tɯ-cha nɤ, nɯ-ɣɤntaβ jɤɣ\cmn 如果你不会做的话,可以放在那里\end{exemple}
\begin{exemple}\jya ɯ-sɯm nɯ-ɣɤntaβ-a\cmn 我令他放心了\end{exemple}
\begin{relation-sémantique}\confer{
\hyperlink{Ⓔɕɯntaβ}{\textit{ \papi{ɕɯntaβ}}}
}\end{relation-sémantique}
\begin{relation-sémantique}\confer{
\hyperlink{Ⓔntaβ}{\textit{ \papi{ntaβ}}}
}\end{relation-sémantique}
\begin{sous-entrée}
\vedette{\hypertarget{}{\papi{ ʑɣɤɣɤntaβ}}}\markboth{ʑɣɤɣɤntaβ}{}\classe{vi}
\begin{définition}\ 
\begin{déclaration}\grammar{refl}\end{déclaration}\end{définition}
\begin{définition}\fra rester sans bouger\end{définition}
\begin{définition}\cmn 坐稳,动都不动\end{définition}
\begin{exemple}\jya tɤ-ndzur, nɯtɕu ma-nɯ-tɯ-ʑɣɤɣɤntaβ ʑo kɯ\cmn 你站起来,不要在那里动都不动\end{exemple}
\end{sous-entrée}\end{entrée}

\begin{entrée}
\vedette{\hypertarget{Ⓔɣɤnɯndzɯlŋɯz}{\papi{ ɣɤnɯndzɯlŋɯz}}}\markboth{ɣɤnɯndzɯlŋɯz}{}
\begin{relation-sémantique}\confer{
\hyperlink{Ⓔnɯndzɯlŋɯz}{\textit{ \papi{nɯndzɯlŋɯz}}}
}\end{relation-sémantique}\end{entrée}

\begin{entrée}
\vedette{\hypertarget{Ⓔɣɤnɯʑɯβ}{\papi{ ɣɤnɯʑɯβ}}}\markboth{ɣɤnɯʑɯβ}{}
\begin{relation-sémantique}\confer{
\hyperlink{Ⓔnɯʑɯβ}{\textit{ \papi{nɯʑɯβ}}}
}\end{relation-sémantique}\end{entrée}

\begin{entrée}
\vedette{\hypertarget{Ⓔɣɤɲɤβɲɤβ}{\papi{ ɣɤɲɤβɲɤβ}}}\markboth{ɣɤɲɤβɲɤβ}{}
\classe{vi}
\paradigme{\textit{dir :} \jya tɤ-}
\begin{définition}\fra parler sans arrêt\end{définition}
\begin{définition}\cmn 不由自主地流出来;不停地唠叨\end{définition}
\begin{exemple}\jya ma-tɯ-ɣɤɲɤβɲɤβ\cmn 你不要啰嗦\end{exemple}
\begin{exemple}\jya nɤ-mtɕhi kɤ-ndɤm, ma-tɯ-ɣɤɲɤβɲɤβ\cmn 你闭嘴,不要啰嗦了\end{exemple}
\begin{exemple}\jya aʑo tɤ-ɣɤɲɤβɲɤβ-a tɕe, ɯʑo kɯ nɯ ta-stu\cmn 我重复讲很多次,他最后还是照做了\end{exemple}\begin{sous-entrée}
\vedette{\hypertarget{}{\papi{ sɤɲɤβɲɤβ}}}\markboth{sɤɲɤβɲɤβ}{}\classe{vt}
\begin{exemple}\jya ɯʑo pɯ-rɯndzɤtshi ri toʁde tɕe kɤ-ndza ra ɲɯ-sɤɲɤβɲɤβ ʑo pa-tɕɤt\cmn 他在吃东西,突然把嘴里的东西不由自主地吐出来了\end{exemple}
\end{sous-entrée}\end{entrée}

\begin{entrée}
\vedette{\hypertarget{Ⓔɣɤɲcɣɤlɤt}{\papi{ ɣɤɲcɣɤlɤt}}}\markboth{ɣɤɲcɣɤlɤt}{}\classe{vs}
\paradigme{\textit{dir :} \jya tɤ-}
\begin{définition}\ 
\begin{déclaration}\grammar{deidph}\end{déclaration}\end{définition}
\begin{définition}\fra bruyant\end{définition}
\begin{définition}\cmn 吵\end{définition}
\begin{exemple}\jya ma-tɯ-ɣɤɲcɣɤlɤt-nɯ, tɕe aʑo pjɯ-nɯʑɯβ-a\cmn 你们不要高声喧哗,让我睡觉吧\end{exemple}
\begin{relation-sémantique}\confer{
\hyperlink{Ⓔɲcɣɤɲcɣɤt}{\textit{ \papi{ɲcɣɤɲcɣɤt}}}
}\end{relation-sémantique}\end{entrée}

\begin{entrée}
\vedette{\hypertarget{Ⓔɣɤɲcɣɤɲcɣɤt}{\papi{ ɣɤɲcɣɤɲcɣɤt}}}\markboth{ɣɤɲcɣɤɲcɣɤt}{}\classe{vs}
\paradigme{\textit{dir :} \jya thɯ-}\acception{1}
\begin{définition}\fra animé, bruyant\end{définition}
\begin{définition}\cmn 嘈杂(声音)、闹哄哄\end{définition}
\begin{exemple}\jya kɤntɕhaʁ ɲɯ-ɣɤɲcɣɤɲcɣɤt\cmn 街上很热闹\end{exemple}\acception{2}
\begin{définition}\fra ardent (feu)\end{définition}
\begin{définition}\cmn 旺盛(火)\end{définition}\begin{sous-entrée}
\vedette{\hypertarget{}{\papi{ sɤɲcɣɤɲcɣɤt}}}\markboth{sɤɲcɣɤɲcɣɤt}{}\classe{vt}
\paradigme{\textit{dir :} \jya thɯ-}
\begin{définition}\fra rendre très ardent\end{définition}
\begin{définition}\cmn 使火烧得更旺盛\end{définition}
\begin{exemple}\jya tɯrma tɯβlɯ chɤ-nɯ-sɤɲcɣɤɲcɣɤt-nɯ kɤ-ti ɲɯ-ŋu\cmn 他们过着幸福的生活(传统故事结尾)\end{exemple}
\begin{relation-sémantique}\confer{
\hyperlink{Ⓔɲcɣɤɲcɣɤt}{\textit{ \papi{ɲcɣɤɲcɣɤt}}}
}\end{relation-sémantique}
\end{sous-entrée}\end{entrée}

\begin{entrée}
\vedette{\hypertarget{Ⓔɣɤɲizɲiz}{\papi{ ɣɤɲizɲiz}}}\markboth{ɣɤɲizɲiz}{}\classe{vs}
\paradigme{\textit{dir :} \jya tɤ-}
\begin{définition}\fra bonne à rien, qui passe son temps à parler pour ne rien dire (fille)\end{définition}
\begin{définition}\cmn 唠叨,不停地讲废话(女孩子)\end{définition}
\begin{exemple}\jya nɤki nɯ ɯ-rju ɲɯ-dɤn tɕe, ɲɯ-ɣɤɲizɲiz ntsɯ\cmn 那个女孩子很爱说话,总是在唠叨\end{exemple}
\begin{exemple}\jya ma-tɯ-ɣɤɲizɲiz\cmn 你不要唠叨\end{exemple}\begin{sous-entrée}
\vedette{\hypertarget{}{\papi{ sɤɲizɲiz}}}\markboth{sɤɲizɲiz}{}\classe{vt}
\begin{définition}\fra tomber sans arrêt, mais en faible quantité (pluie)\end{définition}
\begin{définition}\cmn 雨下个不停但下得不多\end{définition}
\begin{exemple}\jya tɯ-mɯ ɲɯ-sɤɲizɲiz ntsɯ\cmn 毛毛雨下个不停\end{exemple}
\end{sous-entrée}\end{entrée}

\begin{entrée}
\vedette{\hypertarget{Ⓔɣɤɲɟɤrɲɟɤr}{\papi{ ɣɤɲɟɤrɲɟɤr}}}\markboth{ɣɤɲɟɤrɲɟɤr}{}
\classe{vs}
\paradigme{\textit{dir :} \jya tɤ-}
\begin{définition}\fra qui frétille\end{définition}
\begin{définition}\cmn (很庞大的东西)在抖动\end{définition}
\begin{exemple}\jya ɲɯ-ɣɤrcoʁ tɕe ɲɯ-ɣɤɲɟɤrɲɟɤr ʑo\cmn 地面堆了很多泥浆,某个地方一触到,整个地面都在抖动\end{exemple}
\begin{exemple}\jya ɯ-tɯ-tshu ɲɯ-ɣɤɲɟɤrɲɟɤr\cmn 胖得肉都在抖动\end{exemple}
\begin{relation-sémantique}\synonyme{
\hyperlink{Ⓔɣɤmbɤrmbɤr}{\textit{ \papi{ɣɤmbɤrmbɤr}}}
}\end{relation-sémantique}
\begin{relation-sémantique}\confer{
\hyperlink{Ⓔɲɟɤrɲɟɤr}{\textit{ \papi{ɲɟɤrɲɟɤr}}}
}\end{relation-sémantique}\end{entrée}

\begin{entrée}
\vedette{\hypertarget{Ⓔɣɤɲɟɣɤrɲɟɣɤr}{\papi{ ɣɤɲɟɣɤrɲɟɣɤr}}}\markboth{ɣɤɲɟɣɤrɲɟɣɤr}{}\classe{vs}
\begin{définition}\fra long et instable\end{définition}
\begin{définition}\cmn 不稳固\end{définition}
\begin{exemple}\jya romɲa ɲɯ-mpɯ tɕe, ɯ-taʁ tu-kɯ-ŋke tɕe ɲɯ-ɣɤɲɟɣɤrɲɟɣɤr\cmn 小梁不稳固,走在上面一晃一晃的\end{exemple}\end{entrée}

\begin{entrée}
\vedette{\hypertarget{Ⓔɣɤŋɤn}{\papi{ ɣɤŋɤn}}}\markboth{ɣɤŋɤn}{}
\begin{relation-sémantique}\confer{
\hyperlink{Ⓔŋɤn}{\textit{ \papi{ŋɤn}}}
}\end{relation-sémantique}\end{entrée}

\begin{entrée}
\vedette{\hypertarget{Ⓔɣɤŋgi}{\papi{ ɣɤŋgi}}}\markboth{ɣɤŋgi}{}
\classe{vi}
\paradigme{\textit{dir :} \jya pɯ-}
\begin{définition}\fra avoir raison\end{définition}
\begin{définition}\cmn 说得对\end{définition}
\begin{exemple}\jya aʑo pɯ-ɣɤŋgi-a\cmn 我是对的\end{exemple}
\begin{exemple}\jya nɤʑo kɯ nɯ ɲɯ-tɯ-ti, ɲɯ-tɯ-ɣɤŋgi\cmn 你说那些,你是对的\end{exemple}
\begin{exemple}\jya kɯ-rkɯn sɤznɤ kɯ-dɤn nɯ ra pjɯ-ɣɤŋgi-nɯ ra\cmn 比起少数,多数是对的\end{exemple}
\begin{relation-sémantique}\antonyme{
\hyperlink{Ⓔɣɤtɕa}{\textit{ \papi{ɣɤtɕa}}}
}\end{relation-sémantique}\begin{sous-entrée}
\vedette{\hypertarget{}{\papi{ zɣɤŋgi}}}\markboth{zɣɤŋgi}{}\classe{vt}
\paradigme{\textit{dir :} \jya pɯ-}
\begin{définition}\fra être d'accord avec, dire que quelqu'un a raison\end{définition}
\begin{définition}\cmn 同意,说别人是对的\end{définition}
\end{sous-entrée}\begin{sous-entrée}
\vedette{\hypertarget{}{\papi{ ʑɣɤɣɤŋgi}}}\markboth{ʑɣɤɣɤŋgi}{}\classe{vi}
\paradigme{\textit{dir :} \jya pɯ-}
\begin{définition}\ 
\begin{déclaration}\grammar{refl}\end{déclaration}\end{définition}
\begin{définition}\fra dire que l'on a raison\end{définition}
\begin{définition}\cmn 说自己是对的\end{définition}
\end{sous-entrée}\end{entrée}

\begin{entrée}
\vedette{\hypertarget{Ⓔɣɤŋgrɯ}{\papi{ ɣɤŋgrɯ}}}\markboth{ɣɤŋgrɯ}{}
\classe{vt}
\paradigme{\textit{dir :} \jya pɯ-}
\begin{définition}\ 
\begin{déclaration}\grammar{caus}\end{déclaration}\end{définition}
\begin{définition}\fra permettre de réussir\end{définition}
\begin{définition}\cmn 让……成功,得逞\end{définition}
\begin{exemple}\jya nɤʑo nɤ-kɤ-sɯso nɯ aʑo pjɯ-ɣɤŋgri-a jɤɣ.\cmn 我可以实现你的愿望\end{exemple}
\begin{relation-sémantique}\confer{
\hyperlink{Ⓔŋgrɯ}{\textit{ \papi{ŋgrɯ}}}
}\end{relation-sémantique}
\end{entrée}

\begin{entrée}
\vedette{\hypertarget{Ⓔɣɤŋgɯrŋgɯr}{\papi{ ɣɤŋgɯrŋgɯr}}}\markboth{ɣɤŋgɯrŋgɯr}{}
\classe{vs}
\paradigme{\textit{dir :} \jya tɤ-}
\begin{définition}\ 
\begin{déclaration}\grammar{deidph}\end{déclaration}\end{définition}
\begin{définition}\fra fort (bruit)\end{définition}
\begin{définition}\cmn 发出隆隆的响声\end{définition}
\begin{exemple}\jya mbɣɯrloʁ ɲɯ-ɣɤŋgɯrŋgɯr\cmn 雷发出隆隆的声音\end{exemple}
\begin{relation-sémantique}\confer{
\hyperlink{ⒺŋgɯrⒽ2}{\textit{ \papi{ŋgɯr2}}}
}\end{relation-sémantique}\end{entrée}

\begin{entrée}
\vedette{\hypertarget{Ⓔɣɤŋoʁ}{\papi{ ɣɤŋoʁ}}}\markboth{ɣɤŋoʁ}{}
\classe{vt}\acception{1}
\paradigme{\textit{dir :} \jya tɤ-}
\begin{définition}\fra saluer\end{définition}
\begin{définition}\cmn 打招呼\end{définition}
\begin{exemple}\jya tɤ-ɣɤŋoʁ-a\cmn 我给他打了招呼\end{exemple}
\begin{exemple}\jya aʑo @xiangbolin ɯ-phe tɤ-ɣɤŋoʁ-a\cmn 我给向柏霖打了招呼\end{exemple}\acception{2}
\paradigme{\textit{dir :} \jya \_}
\begin{définition}\fra chasser (animaux)\end{définition}
\begin{définition}\cmn 驱逐,赶走(动物)\end{définition}
\begin{exemple}\jya jɤ-ɣɤŋoʁ-a\cmn 我把它赶走了\end{exemple}
\begin{exemple}\jya ja-ɣɤŋoʁ\cmn 他把它赶走了\end{exemple}
\begin{exemple}\jya lɯlu jɤ-ɣɤŋoʁ-a ma tɤ-mthɯm tu-mɯrki ɲɯ-ŋu\cmn 我把猫赶走了因为它在偷肉\end{exemple}\begin{sous-entrée}
\vedette{\hypertarget{}{\papi{ aɣɤŋɯŋoʁ}}}\markboth{aɣɤŋɯŋoʁ}{}\classe{vi}
\paradigme{\textit{dir :} \jya tɤ-}
\begin{définition}\ 
\begin{déclaration}\grammar{recip}\end{déclaration}\end{définition}
\begin{définition}\fra se saluer les uns les autres\end{définition}
\begin{définition}\cmn 互相打招呼\end{définition}
\end{sous-entrée}\end{entrée}

\begin{entrée}
\vedette{\hypertarget{Ⓔɣɤŋoʁle}{\papi{ ɣɤŋoʁle}}}\markboth{ɣɤŋoʁle}{}\classe{vs}
\begin{définition}\fra extraverti\end{définition}
\begin{définition}\cmn 外向,爱说话\end{définition}
\begin{exemple}\jya ɯʑo kɯ-ɣɤŋoʁle ci ŋu\cmn 他是个爱打交道的人\end{exemple}
\begin{relation-sémantique}\synonyme{
\hyperlink{Ⓔɣɤχalala}{\textit{ \papi{ɣɤχalala}}}
}\end{relation-sémantique}\end{entrée}

\begin{entrée}
\vedette{\hypertarget{Ⓔɣɤɴqa}{\papi{ ɣɤɴqa}}}\markboth{ɣɤɴqa}{}
\begin{relation-sémantique}\confer{
\hyperlink{Ⓔɴqa}{\textit{ \papi{ɴqa}}}
}\end{relation-sémantique}\end{entrée}

\begin{entrée}
\vedette{\hypertarget{Ⓔɣɤɴqhi}{\papi{ ɣɤɴqhi}}}\markboth{ɣɤɴqhi}{}
\begin{relation-sémantique}\confer{
\hyperlink{Ⓔɴqhi}{\textit{ \papi{ɴqhi}}}
}\end{relation-sémantique}\end{entrée}

\begin{entrée}
\vedette{\hypertarget{Ⓔɣɤɴɢrɯ}{\papi{ ɣɤɴɢrɯ}}}\markboth{ɣɤɴɢrɯ}{}
\begin{relation-sémantique}\confer{
\hyperlink{Ⓔɴɢrɯ}{\textit{ \papi{ɴɢrɯ}}}
}\end{relation-sémantique}\end{entrée}

\begin{entrée}
\vedette{\hypertarget{Ⓔɣɤpaʁpaʁ}{\papi{ ɣɤpaʁpaʁ}}}\markboth{ɣɤpaʁpaʁ}{}
\classe{vs}
\begin{définition}\fra acide\end{définition}
\begin{définition}\cmn 很酸;很辣\end{définition}
\begin{exemple}\jya ɲɯ-tɕur ɲɯ-ɣɤpaʁpaʁ ʑo\cmn 很酸\end{exemple}
\begin{exemple}\jya ɲɯ-mɤrtsaβ ɲɯ-ɣɤpaʁpaʁ ʑo\cmn 很辣\end{exemple}\end{entrée}

\begin{entrée}
\vedette{\hypertarget{Ⓔɣɤpɣo}{\papi{ ɣɤpɣo}}}\markboth{ɣɤpɣo}{}
\classe{vt}
\paradigme{\textit{dir :} \jya tɤ-}
\begin{définition}\ 
\begin{déclaration}\grammar{denom}\end{déclaration}\end{définition}
\begin{définition}\fra empiler\end{définition}
\begin{définition}\cmn 堆起来
\begin{déclaration}\use{沙尔宗方言}\end{déclaration}\end{définition}
\begin{exemple}\jya ta-ɣɤpɣo\cmn 他堆起来了\end{exemple}
\begin{exemple}\jya ki si kutɕu tɤ-ɣɤpɣo-t-a\cmn 我把这些木头在这里堆起来\end{exemple}
\begin{exemple}\jya tɤ-fkɯm ɯ-ŋgɯ tɯ-rdoʁ chɯ́-wɣ-rku tɕe tú-wɣ-ɣɤpɣo\cmn 把粮食装在口袋里堆起来了\end{exemple}\end{entrée}

\begin{entrée}
\vedette{\hypertarget{Ⓔɣɤphɤn}{\papi{ ɣɤphɤn}}}\markboth{ɣɤphɤn}{}
\begin{relation-sémantique}\confer{
\hyperlink{Ⓔphɤn}{\textit{ \papi{phɤn}}}
}\end{relation-sémantique}\end{entrée}

\begin{entrée}
\vedette{\hypertarget{Ⓔɣɤphrɤβphrɤβ}{\papi{ ɣɤphrɤβphrɤβ}}}\markboth{ɣɤphrɤβphrɤβ}{}
\begin{relation-sémantique}\confer{
\hyperlink{Ⓔphrɤβ}{\textit{ \papi{phrɤβ}}}
}\end{relation-sémantique}\end{entrée}

\begin{entrée}
\vedette{\hypertarget{Ⓔɣɤphɯɕlaʁ}{\papi{ ɣɤphɯɕlaʁ}}}\markboth{ɣɤphɯɕlaʁ}{}
\classe{vi}
\paradigme{\textit{dir :} \jya tɤ-}
\begin{définition}\fra aux mouvements rapides\end{définition}
\begin{définition}\cmn 勤快,动作伶俐\end{définition}
\begin{exemple}\jya tɤ-tɕɯ kɯ-ɣɤphɯɕlaʁ ci ɲɯ-ŋu\cmn 他是一个勤快的孩子\end{exemple}\begin{sous-entrée}
\vedette{\hypertarget{}{\papi{ sɤphɯɕlaʁ}}}\markboth{sɤphɯɕlaʁ}{}\classe{vt}
\paradigme{\textit{dir :} \jya tɤ-}
\begin{définition}\fra agir avec zèle\end{définition}
\begin{définition}\cmn 做得很勤快;马上做完\end{définition}
\begin{exemple}\jya tɤ-sɤphɯɕlaʁ-a tɤ-nɤma-t-a\cmn 我很快就(把工作)做好了\end{exemple}
\begin{exemple}\jya mɯ́j-sɤphɯɕlaʁ\cmn 他做得很慢\end{exemple}
\begin{relation-sémantique}\confer{
\hyperlink{Ⓔɕlaʁ}{\textit{ \papi{ɕlaʁ}}}
}\end{relation-sémantique}
\end{sous-entrée}\end{entrée}

\begin{entrée}
\vedette{\hypertarget{Ⓔɣɤploʁploʁ}{\papi{ ɣɤploʁploʁ}}}\markboth{ɣɤploʁploʁ}{}
\begin{relation-sémantique}\confer{
\hyperlink{Ⓔploʁploʁ}{\textit{ \papi{ploʁploʁ}}}
}\end{relation-sémantique}\end{entrée}

\begin{entrée}
\vedette{\hypertarget{Ⓔɣɤpɯplɯɣ}{\papi{ ɣɤpɯplɯɣ}}}\markboth{ɣɤpɯplɯɣ}{}
\classe{vi}
\paradigme{\textit{dir :} \jya tɤ-}
\begin{définition}\ 
\begin{déclaration}\grammar{deidph}\end{déclaration}\end{définition}
\begin{définition}\fra briller par intermittence, bouillir (eau)\end{définition}
\begin{définition}\cmn 闪光;水沸腾翻滚的样子\end{définition}
\begin{exemple}\jya tʂha ɲɯ-ɤla ɲɯ-ɣɤpɯplɯɣ\cmn 茶在沸腾\end{exemple}
\begin{exemple}\jya tɯ-ci tu-ola tu-ɣɤpɯplɯɣ ɲɯ-ŋu\cmn 水在沸腾\end{exemple}
\begin{exemple}\jya tɤrmbja ɲɯ-ɣɤpɯplɯɣ ʑo\cmn 闪电一闪一闪地发光\end{exemple}\end{entrée}

\begin{entrée}
\vedette{\hypertarget{Ⓔɣɤqhɤβjɤβ}{\papi{ ɣɤqhɤβjɤβ}}}\markboth{ɣɤqhɤβjɤβ}{}
\classe{vi}
\paradigme{\textit{dir :} \jya nɯ-}
\begin{définition}\fra chercher partout\end{définition}
\begin{définition}\cmn 到处乱搜;乱找
\begin{déclaration}\use{一部分人也说\stylefv{ɣɤkɤβjɤβ}}\end{déclaration}\end{définition}
\begin{exemple}\jya aʁɤndɯndɤt ɲɯ-ɣɤqhɤβjɤβ ʑo tu-kɯ-stu ra ma mɯrkɯ\cmn 他到处乱找东西,要注意不然他会偷东西\end{exemple}
\begin{exemple}\jya laχtɕha ma-tɯ-ɣɤqhɤβjɤβ\cmn 你不要乱找东西\end{exemple}
\begin{relation-sémantique}\confer{
\hyperlink{Ⓔɣɤkɤβjɤβ}{\textit{ \papi{ɣɤkɤβjɤβ}}}
}\end{relation-sémantique}\end{entrée}

\begin{entrée}
\vedette{\hypertarget{Ⓔɣɤqlaʁqlaʁ}{\papi{ ɣɤqlaʁqlaʁ}}}\markboth{ɣɤqlaʁqlaʁ}{}\classe{vs}\acception{1}
\begin{définition}\fra clair (ciel)\end{définition}
\begin{définition}\cmn 晴\end{définition}
\begin{exemple}\jya tɯ-mɯ ɲɯ-jɯm ɲɯ-ɣɤqlaʁqlaʁ ʑo\cmn 天很晴\end{exemple}\acception{2}
\begin{définition}\fra très dur\end{définition}
\begin{définition}\cmn 硬邦邦\end{définition}
\begin{exemple}\jya stoʁ ɲɯ-rko ɲɯ-ɣɤqlaʁqlaʁ, kɤ-ndza mɯ́j-sɤsphɯt\cmn 胡豆是硬邦邦的,吃不动\end{exemple}\end{entrée}

\begin{entrée}
\vedette{\hypertarget{Ⓔɣɤqrɤβqrɤβ}{\papi{ ɣɤqrɤβqrɤβ}}}\markboth{ɣɤqrɤβqrɤβ}{}\classe{vi}
\begin{définition}\fra rauque, enroué\end{définition}
\begin{définition}\cmn 嗓子嘶哑的\end{définition}
\begin{exemple}\jya ɯ-rqo ɲɯ-ɣɤqrɤβqrɤβ\cmn 他嗓子嘶哑地讲话\end{exemple}\end{entrée}

\begin{entrée}
\vedette{\hypertarget{Ⓔɣɤqɯβqɯβ}{\papi{ ɣɤqɯβqɯβ}}}\markboth{ɣɤqɯβqɯβ}{}\classe{vs}
\begin{définition}\fra murmurer (eau)\end{définition}
\begin{définition}\cmn 潺潺流水
\begin{déclaration}\use{声音比\stylefv{ɣɤqɯrqɯr}小}\end{déclaration}\end{définition}
\begin{exemple}\jya tɯ-ci ɲɯ-ɣɤqɯβqɯβ ʑo ɲɯ-ɣi\cmn 潺潺流水\end{exemple}
\begin{relation-sémantique}\confer{
\hyperlink{Ⓔɣɤrɕɯrɕɯβ}{\textit{ \papi{ɣɤrɕɯrɕɯβ}}}
}\end{relation-sémantique}
\begin{relation-sémantique}\confer{
\hyperlink{Ⓔɣɤqɯrqɯr}{\textit{ \papi{ɣɤqɯrqɯr}}}
}\end{relation-sémantique}\end{entrée}

\begin{entrée}
\vedette{\hypertarget{Ⓔɣɤqɯrqɯr}{\papi{ ɣɤqɯrqɯr}}}\markboth{ɣɤqɯrqɯr}{}\classe{vs}
\begin{définition}\fra murmurer (eau)\end{définition}
\begin{définition}\cmn 潺潺流水
\begin{déclaration}\use{声音比\stylefv{ɣɤrrɕɯrɕɯβ}小}\end{déclaration}\end{définition}
\begin{exemple}\jya tɯ-ci ɲɯ-ɣɤqɯrqɯr ʑo ɲɯ-ɣi\cmn 潺潺流水\end{exemple}
\begin{relation-sémantique}\synonyme{
\hyperlink{Ⓔɣɤrɕɯrɕɯβ}{\textit{ \papi{ɣɤrɕɯrɕɯβ}}}
}\end{relation-sémantique}
\begin{relation-sémantique}\confer{
\hyperlink{Ⓔɣɤqɯβqɯβ}{\textit{ \papi{ɣɤqɯβqɯβ}}}
}\end{relation-sémantique}\end{entrée}

\begin{entrée}
\vedette{\hypertarget{Ⓔɣɤra}{\papi{ ɣɤra}}}\markboth{ɣɤra}{}
\classe{vt}
\paradigme{\textit{dir :} \jya \_}
\begin{définition}\ 
\begin{déclaration}\grammar{caus}\end{déclaration}\end{définition}
\begin{définition}\fra rendre nécessaire\end{définition}
\begin{définition}\cmn 使……需要\end{définition}
\begin{exemple}\jya a-ku a-kɤrme pɯ-zri ri, pa-qrɤz-nɯ tɕe a-rte ta-ɣɤra\cmn 原来我的头发很长,他们剪了之后,我需要戴帽子\end{exemple}
\begin{relation-sémantique}\confer{
\hyperlink{ⒺraⒽ1}{\textit{ \papi{ra1}}}
}\end{relation-sémantique}\end{entrée}

\begin{entrée}
\vedette{\hypertarget{Ⓔɣɤrɤβ}{\papi{ ɣɤrɤβ}}}\markboth{ɣɤrɤβ}{}
\classe{vs}
\paradigme{\textit{dir :} \jya tɤ-}
\begin{définition}\fra pentu\end{définition}
\begin{définition}\cmn 陡峭
\begin{déclaration} \étymologie{\papi{rab}}\end{déclaration}\end{définition}
\begin{exemple}\jya sɤtɕha ɲɯ-ɣɤrɤβ\cmn (那个)地方很陡峭\end{exemple}
\begin{exemple}\jya tɤ-ɣmbaj ɲɯ-ɣɤrɤβ\cmn 山的那一面很陡峭\end{exemple}
\begin{exemple}\jya zgo ɲɯ-ɣɤrɤβ\cmn 山坡很陡\end{exemple}
\begin{relation-sémantique}\synonyme{
 \papi{ɣɤʑin}
}\end{relation-sémantique}\end{entrée}

\begin{entrée}
\vedette{\hypertarget{Ⓔɣɤrɤru}{\papi{ ɣɤrɤru}}}\markboth{ɣɤrɤru}{}\classe{vi}
\paradigme{\textit{dir :} \jya tɤ-}
\begin{définition}\fra se lever tôt (se lever dès qu'on est réveillé)\end{définition}
\begin{définition}\cmn 起得早(叫了马上就起来)\end{définition}
\begin{exemple}\jya tɤ-rɟit ɲɯ-ɣɤrɤru\cmn 孩子一叫就起床\end{exemple}
\begin{exemple}\jya jiɕqha nɯ, toʁde ku-nɯ-rŋgɯ, nɯɕɯmɯma chɯ-rɤru tɕe kɯ-ɣɤrɤru ci ɲɯ-ŋu\cmn 这个人睡得早,起得早,是个爱早起的人\end{exemple}
\begin{relation-sémantique}\confer{
\hyperlink{Ⓔrɤru}{\textit{ \papi{rɤru}}}
}\end{relation-sémantique}\end{entrée}

\begin{entrée}
\vedette{\hypertarget{Ⓔɣɤrɤt}{\papi{ ɣɤrɤt}}}\markboth{ɣɤrɤt}{}
\classe{vt}
\paradigme{\textit{dir :} \jya thɯ-}
\paradigme{\textit{dir :} \jya \_}
\begin{définition}\fra jeter\end{définition}
\begin{définition}\cmn 扔\end{définition}
\begin{exemple}\jya aʑo thɯ-ɣɤrat-a\cmn 我扔了\end{exemple}
\begin{exemple}\jya jiɕqha laχtɕha nɯ tha-ɣɤrɤt\cmn 他扔了那个东西\end{exemple}
\begin{exemple}\jya qha kɤ-ŋga mɯ́j-sna, tha-ɣɤrɤt\cmn 这件(衣服)不能穿,他就扔了\end{exemple}
\begin{exemple}\jya ki sna maŋe, aj thɯ-ɣɤrat-a\cmn 这个没有用,我就扔了\end{exemple}
\begin{exemple}\jya khɯɣɲɟɯ ɯ-pɕi na-ɣɤrɤt\cmn 他把它扔到窗子外面去了\end{exemple}
\begin{exemple}\jya ɲɤ-tsɣi tɕe thɯ-ɣɤrat-a\cmn (苹果)烂了,所以我把它扔了\end{exemple}\end{entrée}

\begin{entrée}
\vedette{\hypertarget{Ⓔɣɤrchɤrchɤt}{\papi{ ɣɤrchɤrchɤt}}}\markboth{ɣɤrchɤrchɤt}{}
\begin{relation-sémantique}\confer{
\hyperlink{Ⓔrchɤrchɤt}{\textit{ \papi{rchɤrchɤt}}}
}\end{relation-sémantique}\end{entrée}

\begin{entrée}
\vedette{\hypertarget{Ⓔɣɤrchɯɣlɯɣ}{\papi{ ɣɤrchɯɣlɯɣ}}}\markboth{ɣɤrchɯɣlɯɣ}{}
\begin{relation-sémantique}\confer{
\hyperlink{Ⓔrchɯɣrchɯɣ}{\textit{ \papi{rchɯɣrchɯɣ}}}
}\end{relation-sémantique}\end{entrée}

\begin{entrée}
\vedette{\hypertarget{Ⓔɣɤrchɯɣrchɯɣ}{\papi{ ɣɤrchɯɣrchɯɣ}}}\markboth{ɣɤrchɯɣrchɯɣ}{}
\begin{relation-sémantique}\confer{
\hyperlink{Ⓔrchɯɣrchɯɣ}{\textit{ \papi{rchɯɣrchɯɣ}}}
}\end{relation-sémantique}\end{entrée}

\begin{entrée}
\vedette{\hypertarget{Ⓔɣɤrcoʁ}{\papi{ ɣɤrcoʁ}}}\markboth{ɣɤrcoʁ}{}
\classe{vs}
\begin{définition}\ 
\begin{déclaration}\grammar{denom}\end{déclaration}\end{définition}
\begin{définition}\fra boueux\end{définition}
\begin{définition}\cmn 泥泞;满是污泥\end{définition}
\begin{exemple}\jya tʂu ɲɯ-ɣɤrcoʁ\cmn 路很泥泞\end{exemple}
\begin{exemple}\jya ɲɯ-ɣɤrcoʁ tɕe ɲɯ-ɣɤɲɟɤrɲɟɤr\cmn 泥巴很稀\end{exemple}
\begin{relation-sémantique}\confer{
\hyperlink{Ⓔtɤrcoʁ}{\textit{ \papi{tɤrcoʁ}}}
}\end{relation-sémantique}
\begin{relation-sémantique}\confer{
\hyperlink{Ⓔrɤrcoʁ}{\textit{ \papi{rɤrcoʁ}}}
}\end{relation-sémantique}\end{entrée}

\begin{entrée}
\vedette{\hypertarget{Ⓔɣɤrɕo}{\papi{ ɣɤrɕo}}}\markboth{ɣɤrɕo}{}
\begin{relation-sémantique}\confer{
\hyperlink{Ⓔarɕo}{\textit{ \papi{arɕo}}}
}\end{relation-sémantique}\end{entrée}

\begin{entrée}
\vedette{\hypertarget{Ⓔɣɤrɕɯrɕiz}{\papi{ ɣɤrɕɯrɕiz}}}\markboth{ɣɤrɕɯrɕiz}{}\classe{vs}
\begin{définition}\fra murmurer de façon intermittente (eau)\end{définition}
\begin{définition}\cmn 水流动时发出断断续续的声音\end{définition}
\begin{exemple}\jya tɯ-mɯ kɯ-ɣɤrɕɯrɕiz ʑo ɲɯ-ɤsɯ-lɤt\cmn 下雨,发出断断续续的声音\end{exemple}
\begin{relation-sémantique}\confer{
\hyperlink{Ⓔɣɤrɕɯrɕɯβ}{\textit{ \papi{ɣɤrɕɯrɕɯβ}}}
}\end{relation-sémantique}\end{entrée}

\begin{entrée}
\vedette{\hypertarget{Ⓔɣɤrɕɯrɕɯβ}{\papi{ ɣɤrɕɯrɕɯβ}}}\markboth{ɣɤrɕɯrɕɯβ}{} (\variante{ɣɤrɕɯβrɕɯβ}) \classe{vi}\acception{1}
\begin{définition}\fra émettre un bruit de froissement du papier\end{définition}
\begin{définition}\cmn 发出沙沙声\end{définition}\acception{2}
\begin{définition}\fra murmurer (eau)\end{définition}
\begin{définition}\cmn 潺潺流水\end{définition}
\begin{exemple}\jya tɯ-ci ɲɯ-ɣɤrɕɯrɕɯβ ʑo ɲɯ-ɣi\cmn 潺潺流水\end{exemple}
\begin{exemple}\jya tɯ-mɯ ɲɯ-ɣɤrɕɯrɕɯβ ʑo ɲɯ-ɤsɯ-lɤt\cmn 唰唰地下雨\end{exemple}\begin{sous-entrée}
\vedette{\hypertarget{}{\papi{ sɤrɕɯβrɕɯβ}}}\markboth{sɤrɕɯβrɕɯβ}{}\classe{vt}
\begin{exemple}\jya tɯ-mɯ ɲɯ-sɤrɕɯβrɕɯβ ʑo ɲɯ-ɤsɯ-lɤt\cmn 唰唰地下雨\end{exemple}
\begin{relation-sémantique}\confer{
\hyperlink{Ⓔɣɤrsɯβrsɯβ}{\textit{ \papi{ɣɤrsɯβrsɯβ}}}
}\end{relation-sémantique}
\begin{relation-sémantique}\confer{
\hyperlink{Ⓔɣɤrɕɯrɕiz}{\textit{ \papi{ɣɤrɕɯrɕiz}}}
}\end{relation-sémantique}
\begin{relation-sémantique}\confer{
\hyperlink{Ⓔɣɤqɯβqɯβ}{\textit{ \papi{ɣɤqɯβqɯβ}}}
}\end{relation-sémantique}
\end{sous-entrée}\end{entrée}

\begin{entrée}
\vedette{\hypertarget{Ⓔɣɤrdɯl}{\papi{ ɣɤrdɯl}}}\markboth{ɣɤrdɯl}{}\classe{vs}
\begin{définition}\fra plein de poussière\end{définition}
\begin{définition}\cmn 充满灰尘
\end{définition}
\begin{exemple}\jya kɤntɕhaʁ ɯ-tɯ-ɣɤrdɯl ɲɯ-sɤre ʑo\cmn 街上很多灰尘\end{exemple}
\begin{relation-sémantique}\confer{
\hyperlink{Ⓔnɯrdɯl}{\textit{ \papi{nɯrdɯl}}}
}\end{relation-sémantique}
\begin{relation-sémantique}\confer{
\hyperlink{Ⓔrdɯl}{\textit{ \papi{rdɯl}}}
}\end{relation-sémantique}\end{entrée}

\begin{entrée}
\vedette{\hypertarget{Ⓔɣɤrgɤz}{\papi{ ɣɤrgɤz}}}\markboth{ɣɤrgɤz}{}
\begin{relation-sémantique}\confer{
\hyperlink{Ⓔrgɤz}{\textit{ \papi{rgɤz}}}
}\end{relation-sémantique}\end{entrée}

\begin{entrée}
\vedette{\hypertarget{Ⓔɣɤrɣɤβrɣɤβ}{\papi{ ɣɤrɣɤβrɣɤβ}}}\markboth{ɣɤrɣɤβrɣɤβ}{}
\begin{relation-sémantique}\confer{
\hyperlink{Ⓔrɣɤβrɣɤβ}{\textit{ \papi{rɣɤβrɣɤβ}}}
}\end{relation-sémantique}\end{entrée}

\begin{entrée}
\vedette{\hypertarget{Ⓔɣɤrɣɤr}{\papi{ ɣɤrɣɤr}}}\markboth{ɣɤrɣɤr}{}
\classe{idph.2}
\begin{définition}\fra bête\end{définition}
\begin{définition}\cmn 发呆\end{définition}
\begin{exemple}\jya ɣɤrɣɤr ʑo ma-tɤ-tɯ-ʑɣɤstu\cmn 你不要在那里发呆\end{exemple}
\begin{relation-sémantique}\confer{
\hyperlink{Ⓔdɣɤrdɣɤr}{\textit{ \papi{dɣɤrdɣɤr}}}
}\end{relation-sémantique}\end{entrée}

\begin{entrée}
\vedette{\hypertarget{Ⓔɣɤrjɤlɤt}{\papi{ ɣɤrjɤlɤt}}}\markboth{ɣɤrjɤlɤt}{}
\begin{relation-sémantique}\confer{
\hyperlink{Ⓔrjɤrjɤt}{\textit{ \papi{rjɤrjɤt}}}
}\end{relation-sémantique}\end{entrée}

\begin{entrée}
\vedette{\hypertarget{Ⓔɣɤrkhɤrkhɤt}{\papi{ ɣɤrkhɤrkhɤt}}}\markboth{ɣɤrkhɤrkhɤt}{}\classe{vi}
\begin{définition}\fra faire de petits coups légers\end{définition}
\begin{définition}\cmn 发出弹起来的声音,发出轻轻的敲击声\end{définition}\begin{sous-entrée}
\vedette{\hypertarget{}{\papi{ sɤrkhɤrkhɤt}}}\markboth{sɤrkhɤrkhɤt}{}\classe{vt}
\begin{exemple}\jya @yangyu ɲɯ-ɤz-rɤkrɯ tɕe ɲɯ-sɤrkhɤrkhɤt\cmn 他切土豆发出轻轻的敲击声\end{exemple}
\begin{relation-sémantique}\confer{
\hyperlink{Ⓔrkhɤrkhɤt}{\textit{ \papi{rkhɤrkhɤt}}}
}\end{relation-sémantique}
\end{sous-entrée}\end{entrée}

\begin{entrée}
\vedette{\hypertarget{Ⓔɣɤrkɯn}{\papi{ ɣɤrkɯn}}}\markboth{ɣɤrkɯn}{}
\classe{vt}
\paradigme{\textit{dir :} \jya pɯ-}
\paradigme{\textit{dir :} \jya nɯ-}
\begin{définition}\ 
\begin{déclaration}\grammar{caus}\end{déclaration}\end{définition}
\begin{définition}\fra diminuer\end{définition}
\begin{définition}\cmn 减少
\begin{déclaration} \étymologie{\papi{dkon}}\end{déclaration}\end{définition}
\begin{exemple}\jya pɯ-ɣɤrkɯn-a\cmn 我减少了\end{exemple}
\begin{exemple}\jya ɯʑo kɯ pa-ɣɤrkɯn\cmn 他减少了\end{exemple}
\begin{exemple}\jya tɤ-rʑaʁ ɲo-ɣɤrkɯn\cmn 他把时间减少了\end{exemple}
\begin{relation-sémantique}\confer{
\hyperlink{Ⓔrkɯn}{\textit{ \papi{rkɯn}}}
}\end{relation-sémantique}\end{entrée}

\begin{entrée}
\vedette{\hypertarget{Ⓔɣɤrlaʁ}{\papi{ ɣɤrlaʁ}}}\markboth{ɣɤrlaʁ}{}\classe{vt}
\paradigme{\textit{dir :} \jya nɯ-}
\paradigme{\textit{dir :} \jya thɯ-}
\begin{définition}\ 
\begin{déclaration}\grammar{caus}\end{déclaration}\end{définition}
\begin{définition}\fra détruire (une famille)\end{définition}
\begin{définition}\cmn 消灭(家族),弄丢
\begin{déclaration}\use{表达“弄丢”这个意思干木鸟话一般说\stylefv{ɲɤɣɤme}}\end{déclaration}\end{définition}\begin{sous-entrée}
\vedette{\hypertarget{}{\papi{ aɣɤrlɯrlaʁ}}}\markboth{aɣɤrlɯrlaʁ}{}\classe{vi}
\begin{définition}\ 
\begin{déclaration}\grammar{recip}\end{déclaration}
\begin{déclaration}\grammar{caus}\end{déclaration}\end{définition}
\begin{définition}\fra s'évincer mutuellement\end{définition}
\begin{définition}\cmn 互相倾轧\end{définition}
\begin{exemple}\jya kɯɕɯŋgɯ tɕe, tɤru ra tu-onɯsnɯɲɯɲaʁ-nɯ tɕe chɯ-ɤɣɤrlɯrlaʁ-nɯ pjɤ-ŋgrɤl\cmn 古时候,贵族们一直互相倾轧\end{exemple}
\begin{relation-sémantique}\confer{
\hyperlink{Ⓔrlaʁ}{\textit{ \papi{rlaʁ}}}
}\end{relation-sémantique}
\end{sous-entrée}\end{entrée}

\begin{entrée}
\vedette{\hypertarget{Ⓔɣɤrlɤrlɤɣ}{\papi{ ɣɤrlɤrlɤɣ}}}\markboth{ɣɤrlɤrlɤɣ}{}
\classe{vi}
\paradigme{\textit{dir :} \jya tɤ-}
\begin{définition}\fra rouler (objet rond)\end{définition}
\begin{définition}\cmn 滚;扭动(圆东西)\end{définition}
\begin{exemple}\jya jiɕqha laχtɕha nɯ ɲɯ-ɣɤrlɤrlɤɣ\cmn 那个东西在滚\end{exemple}
\begin{exemple}\jya khɯtsa ɯ-ta mɯ-ɲɤ-βdi ɲɯ-ɣɤrlɤrlɤɣ\end{exemple}\begin{sous-entrée}
\vedette{\hypertarget{}{\papi{ sɤrlɤrlɤɣ}}}\markboth{sɤrlɤrlɤɣ}{}\classe{vt}
\paradigme{\textit{dir :} \jya tɤ-}
\begin{exemple}\jya ɯ-ku ɲɯ-sɤrlɤrlɤɣ\cmn 他摇头\end{exemple}
\end{sous-entrée}\end{entrée}

\begin{entrée}
\vedette{\hypertarget{Ⓔɣɤrmɤβrmɤβ}{\papi{ ɣɤrmɤβrmɤβ}}}\markboth{ɣɤrmɤβrmɤβ}{}\classe{vs}
\paradigme{\textit{dir :} \jya tɤ-}
\begin{définition}\fra scintiller\end{définition}
\begin{définition}\cmn 一闪一闪\end{définition}
\begin{exemple}\jya @dian mɯ-tɤ-pe tɕe, @dianshi tu-ɣɤrmɤβrmɤβ ʑo ɲɯ-ŋu\cmn 电不好的时候,电视一闪一闪\end{exemple}\begin{sous-entrée}
\vedette{\hypertarget{}{\papi{ sɤrmɤβrmɤβ}}}\markboth{sɤrmɤβrmɤβ}{}\classe{vt}
\paradigme{\textit{dir :} \jya tɤ-}
\begin{exemple}\jya ɯ-mɲaʁ tu-sɤrmɤβrmɤβ ʑo ɲɯ-ŋu\cmn 他眼睛一眨一眨,眨得很快\end{exemple}
\end{sous-entrée}\end{entrée}

\begin{entrée}
\vedette{\hypertarget{Ⓔɣɤrmɤrmɤβ}{\papi{ ɣɤrmɤrmɤβ}}}\markboth{ɣɤrmɤrmɤβ}{}\classe{vi}
\begin{définition}\fra qui passe rapidement (comme un éclair)\end{définition}
\begin{définition}\cmn 闪得很快\end{définition}
\begin{exemple}\jya nɤ-@diannao ɲɯ-ɣɤrmɤrmɤβ ʑo ri, nɯ kɯnɤ tɤ-scoz ɲɯ-tɯ-mtɤm\cmn 虽然电脑屏幕上的字闪的很快,你还是看得见\end{exemple}\end{entrée}

\begin{entrée}
\vedette{\hypertarget{Ⓔɣɤrndi}{\papi{ ɣɤrndi}}}\markboth{ɣɤrndi}{}\classe{vt}\acception{1}
\paradigme{\textit{dir :} \jya kɤ-}
\begin{définition}\fra soutenir, déployer\end{définition}
\begin{définition}\cmn 搀扶;扶起来;撑住;稳住\end{définition}
\begin{exemple}\jya aʑo ku-ɣɤrndi-a\cmn 我把他扶起来\end{exemple}
\begin{exemple}\jya ki laχtɕha ki kɤ-ɣɤrndi\cmn 你把这个东西扶起来\end{exemple}
\begin{exemple}\jya ki @huatong ki a-mɤ-pɯ-ndʐaβ, kɤ-ɣɤrndi\cmn 不要让这个话筒掉下去,你把它扶起来吧\end{exemple}
\begin{exemple}\jya jiɕqha tɯrme nɯ hanɯni ngo, aj kɤ-ɣɤrndi-t-a\cmn 这个人身体有点不舒服,我把他扶起来了\end{exemple}
\begin{exemple}\jya ndʐaβ ɲɯ-ŋu, kɤ-ɣɤrndi-t-a\cmn 差一点掉下去了,我把它扶起来了\end{exemple}
\begin{exemple}\jya a-wa lo-βzi tɕe, kɤ-ɣɤrndi-t-a\cmn 我父亲醉了,我把他扶起来了\end{exemple}
\begin{exemple}\jya a-βɣe ɯ-ku kɤ-tɯ-ɣɤrndi-t ŋu\cmn 你在我最痛苦的时候收养了我这个孤儿\end{exemple}\acception{2}
\paradigme{\textit{dir :} \jya kɤ-}
\begin{définition}\fra calmer\end{définition}
\begin{définition}\cmn 冷静\end{définition}
\begin{exemple}\jya nɤ-sɯm nɯ kɤ-ɣɤrndi ɲɯ-ra\cmn 你要镇静下来\end{exemple}\begin{sous-entrée}
\vedette{\hypertarget{}{\papi{ ʑɣɤɣɤrndi}}}\markboth{ʑɣɤɣɤrndi}{}\classe{vi}
\paradigme{\textit{dir :} \jya kɤ-}
\begin{définition}\ 
\begin{déclaration}\grammar{refl}\end{déclaration}\end{définition}
\begin{définition}\fra se calmer\end{définition}
\begin{définition}\cmn 镇静\end{définition}
\begin{exemple}\jya nɤʑo kɤ-ʑɣɤɣɤrndi ɲɯ-ra\cmn 你要镇静下来\end{exemple}
\end{sous-entrée}\end{entrée}

\begin{entrée}
\vedette{\hypertarget{ⒺɣɤrɲɟiⒽ2}{\papi{ ɣɤrɲɟi}}}\markboth{ɣɤrɲɟi}{}\homonyme{2}\classe{vt}
\paradigme{\textit{dir :} \jya nɯ-}
\paradigme{\textit{dir :} \jya pɯ-}
\paradigme{\textit{dir :} \jya thɯ-}
\begin{définition}\ 
\begin{déclaration}\grammar{caus}\end{déclaration}\end{définition}
\begin{définition}\fra allonger\end{définition}
\begin{définition}\cmn 弄长\end{définition}
\begin{exemple}\jya jiɕqha tɤ-ri nɯ nɯ-ɣɤrɲɟi-t-a\cmn 我把这根线拉长了\end{exemple}
\begin{exemple}\jya si kɤ-ʁndzɤr pɯ-ɣɤrɲɟi-t-a\cmn 我把木头锯得太长了\end{exemple}
\begin{exemple}\jya tɯ-ŋga nɯ-qrɯ-t-a tɕe nɯ-ɣɤrɲɟi-t-a\cmn 我把衣服剪得太长了\end{exemple}
\begin{exemple}\jya ɯ-mke cho-ɣɤrɲɟi\cmn 他伸了脖子\end{exemple}
\begin{sous-entrée}
\vedette{\hypertarget{}{\papi{ ɣɤrɲɟi}}}\markboth{ɣɤrɲɟi}{}\classe{vs}
\paradigme{\textit{dir :} \jya nɯ-}
\begin{définition}\ 
\begin{déclaration}\grammar{facil}\end{déclaration}\end{définition}
\begin{définition}\fra s'allonger facilement\end{définition}
\begin{définition}\cmn 容易变长\end{définition}
\begin{relation-sémantique}\confer{
\hyperlink{Ⓔrɲɟi}{\textit{ \papi{rɲɟi}}}
}\end{relation-sémantique}
\end{sous-entrée}\end{entrée}

\begin{entrée}
\vedette{\hypertarget{Ⓔɣɤrɲɯɣrɲɯɣ}{\papi{ ɣɤrɲɯɣrɲɯɣ}}}\markboth{ɣɤrɲɯɣrɲɯɣ}{}\classe{vi}
\paradigme{\textit{dir :} \jya tɤ-}\acception{1}
\begin{définition}\fra ramper\end{définition}
\begin{définition}\cmn 蠕动\end{définition}
\begin{exemple}\jya qapri ɲɯ-ɣɤrɲɯɣrɲɯɣ nɯ-ari\cmn 蛇蠕动着过去了\end{exemple}\acception{2}
\begin{définition}\fra percer rapidement\end{définition}
\begin{définition}\cmn 钻得很快\end{définition}
\begin{exemple}\jya mkhɯrlɤmnɯ ɲɯ-ɣɤrɲɯrɲɯɣ\cmn 钻子在钻\end{exemple}
\begin{exemple}\jya ɕomskrɯt nɯ ŋgɤm ɯ-taʁ kɤ-sɤtsa-t-a tɕe, ɲɯ-nɤmpi tɕe, ɲɯ-ɣɤrɲɯɣrɲɯɣ ʑo lɤ-ari\cmn 我把铁丝插在土墙上,因为很软,所以钻得很快\end{exemple}\end{entrée}

\begin{entrée}
\vedette{\hypertarget{Ⓔɣɤrŋa}{\papi{ ɣɤrŋa}}}\markboth{ɣɤrŋa}{}\classe{vs}
\begin{définition}\fra être possible\end{définition}
\begin{définition}\cmn 有……的可能,有……的危险
\begin{déclaration}\grammar{denom}\end{déclaration}\end{définition}
\begin{exemple}\jya khɤxtu ɯ-ndo ʑo ma-thɯ-tɯ-ɕe ma kɯ-ɤtɤr ɲɯ-ɣɤrŋa\cmn 你不要靠近房背的边缘,有掉下去的危险\end{exemple}
\begin{exemple}\jya tɯ-mɯ chɤ-ɲɟɯr tɕe, zdɯm kɯ-ɲaʁ jo-ɣi tɕe, tɯ-mɯ kɤ-lɤt ɲɯ-ɣɤrŋa\cmn 变天了,来了乌云,有可能会下雨\end{exemple}
\begin{exemple}\jya nɤ-jaʁ kɤ-xtsɯɣ ɲɯ-ɣɤrŋa\cmn 你有切到手的危险\end{exemple}
\begin{relation-sémantique}\confer{
\hyperlink{Ⓔtɯ-rŋa}{\textit{ \papi{tɯ-rŋa}}}
}\end{relation-sémantique}\end{entrée}

\begin{entrée}
\vedette{\hypertarget{Ⓔɣɤro}{\papi{ ɣɤro}}}\markboth{ɣɤro}{}
\classe{vt}
\paradigme{\textit{dir :} \jya tɤ-}
\paradigme{\textit{dir :} \jya nɯ-}
\begin{définition}\ 
\begin{déclaration}\grammar{caus}\end{déclaration}\end{définition}
\begin{définition}\fra ajouter, accomplir sa tâche au delà des exigences\end{définition}
\begin{définition}\cmn 增加;完成任务超过标准\end{définition}
\begin{exemple}\jya ɯʑo kɯ ta-ɣɤro\cmn 他增加了\end{exemple}
\begin{exemple}\jya jiɕqha ɯ-khrɤt nɯ staʁ tɤ-ɣɤro-t-a\cmn 我做得比刚才规定的多一些\end{exemple}
\begin{exemple}\jya ɯʑo kɯ a-sci ɲɯ-khɤm ra, ri nɯ staʁ ta-ɣɤro\cmn 他还给我的那些,还多了一点\end{exemple}
\begin{exemple}\jya a-phoʁ ta-ɣɤro-nɯ\cmn 他们给我增加了工资\end{exemple}
\begin{exemple}\jya a-ma nɯ-ɣɤro-t-a\cmn 我完成任务超过标准\end{exemple}\end{entrée}

\begin{entrée}
\vedette{\hypertarget{Ⓔɣɤroʁroʁ}{\papi{ ɣɤroʁroʁ}}}\markboth{ɣɤroʁroʁ}{}\classe{vi}
\begin{définition}\fra couler sans arrêt\end{définition}
\begin{définition}\cmn 不停地流\end{définition}
\begin{exemple}\jya tɤ-se ɲɯ-ɣɤroʁroʁ ʑo\cmn 血流个不停\end{exemple}\end{entrée}

\begin{entrée}
\vedette{\hypertarget{Ⓔɣɤrphɤrphɤβ}{\papi{ ɣɤrphɤrphɤβ}}}\markboth{ɣɤrphɤrphɤβ}{}
\classe{vs}
\paradigme{\textit{dir :} \jya tɤ-}
\begin{définition}\fra battre des ailes\end{définition}
\begin{définition}\cmn 拍打翅膀\end{définition}
\begin{exemple}\jya tsɯʁot ɲɯ-ɣɤrphɤrphɤβ\cmn 野鸡在拍翅膀发出声音\end{exemple}
\begin{exemple}\jya ʁdɯskɤr ɲɯ-ɣɤrphɤrphɤβ\cmn 旗子在飘动,发出声音\end{exemple}
\begin{relation-sémantique}\confer{
\hyperlink{Ⓔʁarphɤβ}{\textit{ \papi{ʁarphɤβ}}}
}\end{relation-sémantique}
\begin{relation-sémantique}\confer{
\hyperlink{Ⓔnɤʁarphɤβ}{\textit{ \papi{nɤʁarphɤβ}}}
}\end{relation-sémantique}\begin{sous-entrée}
\vedette{\hypertarget{}{\papi{ sɤrphɤrphɤβ}}}\markboth{sɤrphɤrphɤβ}{}\classe{vt}
\begin{définition}\fra battre des ailes\end{définition}
\begin{définition}\cmn 拍打翅膀\end{définition}
\begin{exemple}\jya kumpɣa kɯ ɯ-ʁar ɲɯ-sɤrphɤrphɤβ\cmn 鸡在拍打翅膀\end{exemple}
\end{sous-entrée}\end{entrée}

\begin{entrée}
\vedette{\hypertarget{Ⓔɣɤrpi}{\papi{ ɣɤrpi}}}\markboth{ɣɤrpi}{}\classe{vt}
\begin{définition}\ 
\begin{déclaration}\grammar{denom}\end{déclaration}\end{définition}\acception{1}
\paradigme{\textit{dir :} \jya nɯ-}
\begin{définition}\fra exécuter une cérémonie religieuse chez soi\end{définition}
\begin{définition}\cmn 请和尚来家里念经\end{définition}
\begin{exemple}\jya jisŋi jiʑo jikha ɲɯ-ɣɤrpi-j ŋu\cmn 今天我们请了和尚到家里念经\end{exemple}\acception{2}
\paradigme{\textit{dir :} \jya tɤ-}
\begin{définition}\fra frapper\end{définition}
\begin{définition}\cmn 打(比喻和尚念经的时候打鼓)\end{définition}
\begin{exemple}\jya nɤ-stu tɤ-fse ma ta-ɣɤrpi\cmn 你规矩一点,不然我就打你(对小孩子说的)\end{exemple}
\begin{exemple}\jya paχtsa nɯ aʑo nɯ-ɣɤrpi-t-a\cmn 我把小猪打死了\end{exemple}
\begin{relation-sémantique}\confer{
\hyperlink{Ⓔtɤ-rpi}{\textit{ \papi{tɤ-rpi}}}
}\end{relation-sémantique}\end{entrée}

\begin{entrée}
\vedette{\hypertarget{Ⓔɣɤrqhoŋloŋ}{\papi{ ɣɤrqhoŋloŋ}}}\markboth{ɣɤrqhoŋloŋ}{}\classe{vi}
\paradigme{\textit{dir :} \jya tɤ-}
\begin{définition}\fra émettre un bruit (objets durs qui se cognent)\end{définition}
\begin{définition}\cmn 硬的东西相撞的时候发出声音\end{définition}
\begin{exemple}\jya atu ra tshitsuku ɲɯ-ɤz-nɤma-nɯ tɕe tɯrnda ɯ-taʁ ɲɯ-ɣɤrqhoŋloŋ-nɯ ʑo\cmn 上面的那些人不知道在做什么,在地板上很吵\end{exemple}\end{entrée}

\begin{entrée}
\vedette{\hypertarget{Ⓔɣɤrqhoʁrqhoʁ}{\papi{ ɣɤrqhoʁrqhoʁ}}}\markboth{ɣɤrqhoʁrqhoʁ}{}
\classe{vs}
\paradigme{\textit{dir :} \jya tɤ-}
\begin{définition}\ 
\begin{déclaration}\grammar{deidph}\end{déclaration}\end{définition}
\begin{définition}\fra bruit de choc d'un objet dur sur une surface dure\end{définition}
\begin{définition}\cmn 硬的东西敲到木板上时发出声音\end{définition}
\begin{exemple}\jya ʑɴɢɯloʁ pɯ-nɯɕlɯɣ-a tɕe, ɲɯ-ɣɤrqhoʁrqhoʁ ʑo\cmn 我失手,把核桃掉到地上发出声音\end{exemple}
\begin{exemple}\jya tʂha ɲɯ-ɤla ɲɯ-ɣɤrqhoʁrqhoʁ ʑo\cmn 茶开了发出沸腾的声音\end{exemple}\begin{sous-entrée}
\vedette{\hypertarget{}{\papi{ nɯrqhoʁ}}}\markboth{nɯrqhoʁ}{}\classe{vi}
\paradigme{\textit{dir :} \jya pɯ-}
\begin{définition}\fra tirer au fusil sans arrêt\end{définition}
\begin{définition}\cmn 不停地打枪\end{définition}
\begin{exemple}\jya ɕɤmɯɣdɯ ɲɯ-ɤz-nɯrqhoʁ ʑo ɲɯ-ɤsɯ-lɤt\cmn 他啪啪地射枪\end{exemple}
\end{sous-entrée}\begin{sous-entrée}
\vedette{\hypertarget{}{\papi{ sɤrqhoʁrqhoʁ}}}\markboth{sɤrqhoʁrqhoʁ}{}\classe{vt}
\paradigme{\textit{dir :} \jya tɤ-}
\begin{définition}\fra frapper (objet dur)\end{définition}
\begin{définition}\cmn 敲(硬的东西)\end{définition}
\begin{exemple}\jya a-ku ta-sɤrqhoʁrqhoʁ\cmn 他敲了我的头\end{exemple}
\end{sous-entrée}\end{entrée}

\begin{entrée}
\vedette{\hypertarget{Ⓔɣɤrqhɯβrqhɯβ}{\papi{ ɣɤrqhɯβrqhɯβ}}}\markboth{ɣɤrqhɯβrqhɯβ}{} (\variante{ɣɤrqhɯrqhɯβ}) 
\classe{vi}
\begin{définition}\fra bruit d'objets durs et secs qui s'entrechoquent\end{définition}
\begin{définition}\cmn 又干又硬的东西相撞发出声音\end{définition}
\begin{exemple}\jya tɯɲɤt thɯ-ɣe tɕe, rdɤstaʁ ra ɲɯ-ɣɤrqhɯrqhɯβ ɲɯ-ɤmɯrpu-nɯ\cmn 泥石流下来了,里面的大小石头互相撞击\end{exemple}
\begin{relation-sémantique}\confer{
\hyperlink{Ⓔsɤrqhɯrqhɯβ}{\textit{ \papi{sɤrqhɯrqhɯβ}}}
}\end{relation-sémantique}\end{entrée}

\begin{entrée}
\vedette{\hypertarget{Ⓔɣɤrʁaʁ}{\papi{ ɣɤrʁaʁ}}}\markboth{ɣɤrʁaʁ}{}\classe{vi}
\paradigme{\textit{dir :} \jya tɤ-}
\paradigme{\textit{dir :} \jya thɯ-}
\begin{définition}\ 
\begin{déclaration}\grammar{denom}\end{déclaration}\end{définition}
\begin{définition}\fra chasser\end{définition}
\begin{définition}\cmn 打猎\end{définition}
\begin{exemple}\jya kɯ-ɣɤrʁaʁ jɤ-ɕe\cmn 你去打猎吧\end{exemple}
\begin{exemple}\jya ji-lɯlu ɲɯ-ɣɤrʁaʁ\cmn 我们的猫在捉(老鼠)\end{exemple}\begin{sous-entrée}
\vedette{\hypertarget{}{\papi{ kɯɣɤrʁaʁ}}}\markboth{kɯɣɤrʁaʁ}{}\classe{n}
\begin{définition}\fra chasseur\end{définition}
\begin{définition}\cmn 猎人\end{définition}
\begin{relation-sémantique}\confer{
\hyperlink{Ⓔnɤrʁaʁ}{\textit{ \papi{nɤrʁaʁ}}}
}\end{relation-sémantique}
\end{sous-entrée}\end{entrée}

\begin{entrée}
\vedette{\hypertarget{Ⓔɣɤrʁɤβjɤβ}{\papi{ ɣɤrʁɤβjɤβ}}}\markboth{ɣɤrʁɤβjɤβ}{}
\classe{vi}
\paradigme{\textit{dir :} \jya tɤ-}
\begin{définition}\fra se débattre\end{définition}
\begin{définition}\cmn 东抓西抓\end{définition}
\begin{exemple}\jya tɤ-ɣɤrʁɤβjaβ-a\cmn 我抓狂了\end{exemple}
\begin{exemple}\jya jiɕqha tsɯʁot nɯnɯ pɤjkhu mɯ-pjɤ-si, ɲɯ-ɣɤrʁɤβjɤβ\cmn 那只野鸡还没有死,正在挣扎\end{exemple}
\begin{exemple}\jya pjɤ-ndʐaβ tɕe ɲɯ-ɣɤrʁɤβjɤβ\cmn 他摔倒了,正在挣扎\end{exemple}\begin{sous-entrée}
\vedette{\hypertarget{}{\papi{ sɤrʁɤβjɤβ}}}\markboth{sɤrʁɤβjɤβ}{}
\paradigme{\textit{dir :} \jya tɤ-}
\begin{définition}\fra faire se débattre\end{définition}
\begin{définition}\cmn 使人东抓西抓\end{définition}
\begin{exemple}\jya aʑo tɕɣom tɤ-ndzat-a tɕe, pɯ́-wɣ-sɯmkɯt-a tɤ́-wɣ-sɤrʁɤβjaβ-a\cmn 我吃了花椒,把我呛到了,让我东抓西抓\end{exemple}
\end{sous-entrée}\end{entrée}

\begin{entrée}
\vedette{\hypertarget{Ⓔɣɤrsɯβrsɯβ}{\papi{ ɣɤrsɯβrsɯβ}}}\markboth{ɣɤrsɯβrsɯβ}{}
\classe{vs}
\begin{définition}\fra émettre un bruit de froissement\end{définition}
\begin{définition}\cmn 发出沙沙声\end{définition}
\begin{exemple}\jya soʁma ɯ-ŋgɯ ɲɯ-ɣɤrsɯβrsɯβ\cmn 干草里有沙沙声\end{exemple}
\begin{relation-sémantique}\confer{
 \papi{ɣɤrɕɯβrɕɯβ}
}\end{relation-sémantique}\end{entrée}

\begin{entrée}
\vedette{\hypertarget{Ⓔɣɤrtaʁ}{\papi{ ɣɤrtaʁ}}}\markboth{ɣɤrtaʁ}{}
\classe{vt}
\paradigme{\textit{dir :} \jya tɤ-}
\begin{définition}\ 
\begin{déclaration}\grammar{caus}\end{déclaration}\end{définition}
\begin{définition}\fra rendre suffisant\end{définition}
\begin{définition}\cmn 加够;添够;使其具备条件\end{définition}
\begin{exemple}\jya @cai ɯ-spa mɯ́j-rtaʁ tɕe, @yangyu lɤ-tɕat-a tɕe tɤ-ɣɤrtaʁ-a\cmn 菜的材料不够,所以我拿了土豆,这样就够了\end{exemple}
\begin{exemple}\jya stoʁ nɯ-phɯt-a tɕe tɤ-ɣɤrtaʁ-a\cmn 我拿了胡豆,这样就够了\end{exemple}\end{entrée}

\begin{entrée}
\vedette{\hypertarget{Ⓔɣɤrtɕhɣaʁ}{\papi{ ɣɤrtɕhɣaʁ}}}\markboth{ɣɤrtɕhɣaʁ}{}\classe{vi}
\paradigme{\textit{dir :} \jya nɯ-}
\begin{définition}\fra chicaner\end{définition}
\begin{définition}\cmn 计较;找毛病\end{définition}
\begin{relation-sémantique}\synonyme{
\hyperlink{Ⓔɣɤtɕɯqaʁ}{\textit{ \papi{ɣɤtɕɯqaʁ}}}
}\end{relation-sémantique}
\begin{relation-sémantique}\confer{
\hyperlink{Ⓔsɤrtɕhɣaʁ}{\textit{ \papi{sɤrtɕhɣaʁ}}}
}\end{relation-sémantique}
\begin{relation-sémantique}\confer{
\hyperlink{Ⓔtɤ-rtɕhɣaʁ,tɕɤt}{\textit{ \papi{tɤ-rtɕhɣaʁ,tɕɤt}}}
}\end{relation-sémantique}
\end{entrée}

\begin{entrée}
\vedette{\hypertarget{Ⓔɣɤrthɯɣlɯɣ}{\papi{ ɣɤrthɯɣlɯɣ}}}\markboth{ɣɤrthɯɣlɯɣ}{}\classe{vi}
\begin{définition}\fra gêner les gens en bougeant sans arrêt\end{définition}
\begin{définition}\cmn 没有目地走动,影响周边的人\end{définition}
\begin{exemple}\jya tɤ-pɤtso ɯ-rkɯ chiz ɲɯ-ɣɤrthɯɣlɯɣ tɕe jɤ-sɯxɕe-t-a\cmn 那个小孩子在身边影响我,我叫他走开\end{exemple}\end{entrée}

\begin{entrée}
\vedette{\hypertarget{Ⓔɣɤrti}{\papi{ ɣɤrti}}}\markboth{ɣɤrti}{}\classe{n}
\begin{définition}\ 
\begin{déclaration}\grammar{n.lieu}\end{déclaration}\end{définition}
\begin{définition}\fra l'un des hameaux de Kamnyu\end{définition}
\begin{définition}\cmn 干木鸟的大队之一\end{définition}
\end{entrée}

\begin{entrée}
\vedette{\hypertarget{Ⓔɣɤrtshɯɣlɯɣ}{\papi{ ɣɤrtshɯɣlɯɣ}}}\markboth{ɣɤrtshɯɣlɯɣ}{}
\classe{vi}
\begin{définition}\fra être grossier\end{définition}
\begin{définition}\cmn 说粗话\end{définition}
\begin{exemple}\jya ɯ-mbrɯ ɲɯ-ŋgɯ tɕe, ɯ-zda ra nɯ-ɕki ɲɯ-ɣɤrtshɯɣlɯɣ\cmn 他在生气,对别人说粗话\end{exemple}\end{entrée}

\begin{entrée}
\vedette{\hypertarget{Ⓔɣɤrtɯm}{\papi{ ɣɤrtɯm}}}\markboth{ɣɤrtɯm}{}
\classe{vt}
\paradigme{\textit{dir :} \jya tɤ-}
\begin{définition}\ 
\begin{déclaration}\grammar{caus}\end{déclaration}\end{définition}
\begin{définition}\fra enrouler ensemble les fils\end{définition}
\begin{définition}\cmn 缠线\end{définition}
\begin{exemple}\jya tɤ-ri tɤ-ɣɤrtɯm-a\cmn 我缠了线\end{exemple}\end{entrée}

\begin{entrée}
\vedette{\hypertarget{Ⓔɣɤrɯβrɯβ}{\papi{ ɣɤrɯβrɯβ}}}\markboth{ɣɤrɯβrɯβ}{}
\classe{vi}
\begin{définition}\ 
\begin{déclaration}\grammar{deidph}\end{déclaration}\end{définition}\acception{1}
\begin{définition}\fra couler sans arrêt\end{définition}
\begin{définition}\cmn 不停地往下流\end{définition}
\begin{exemple}\jya tɯ-ci ɲɯ-ɣɤrɯβrɯβ ɲɯ-nɯɬoʁ\cmn 一滴一滴地不停地漏水\end{exemple}
\begin{exemple}\jya tɯ-ci ɲɯ-ɣɤrɯβrɯβ\cmn 水不断地往下流\end{exemple}
\begin{exemple}\jya a-tɕɯ ɣɯ ɯ-mci ɲɯ-ɣɤrɯβrɯβ\cmn 我儿子在流口水\end{exemple}\acception{2}
\begin{définition}\fra radoter\end{définition}
\begin{définition}\cmn 不停地唠叨\end{définition}
\begin{exemple}\jya ma-tɯ-ɣɤrɯβrɯβ\cmn 你不要唠叨\end{exemple}
\begin{relation-sémantique}\synonyme{
\hyperlink{Ⓔɣɤtʂɯtʂɯt}{\textit{ \papi{ɣɤtʂɯtʂɯt}}}
}\end{relation-sémantique}\end{entrée}

\begin{entrée}
\vedette{\hypertarget{Ⓔɣɤrɯri}{\papi{ ɣɤrɯri}}}\markboth{ɣɤrɯri}{}\classe{vi}
\begin{définition}\fra souffler sans cesse (vent)\end{définition}
\begin{définition}\cmn (风)吹得很紧,不停地吹\end{définition}
\begin{exemple}\jya qale ɲɯ-ɣɤrɯri ʑo\cmn 风吹得很紧\end{exemple}\end{entrée}

\begin{entrée}
\vedette{\hypertarget{Ⓔɣɤrwɤrwɤt}{\papi{ ɣɤrwɤrwɤt}}}\markboth{ɣɤrwɤrwɤt}{}
\classe{vs}
\begin{définition}\fra bavard\end{définition}
\begin{définition}\cmn 说话滔滔不绝\end{définition}
\begin{exemple}\jya ɯ-tɯ-nɯɕmɯrga kɯ ɲɯ-ɣɤrwɤrwɤt ʑo\cmn 他很爱说话,说得滔滔不绝\end{exemple}\begin{sous-entrée}
\vedette{\hypertarget{}{\papi{ sɤrwɤrwɤt}}}\markboth{sɤrwɤrwɤt}{}\classe{vt}
\begin{exemple}\jya nɤki nɯ jazɣɯt rcanɯ, ɯ-tɯ-rɯɕmi kɯ ɲɯ-sɤrwɤrwɤt ʑo\cmn 那个人到了,讲话讲得滔滔不绝\end{exemple}
\begin{exemple}\jya kɤ-ndɯn ɲɯ-mkhɤz tɕe, ɲɯ-sɤrwɤrwɤt ʑo ɕti\cmn 他读得很熟练\end{exemple}
\end{sous-entrée}\end{entrée}

\begin{entrée}
\vedette{\hypertarget{Ⓔɣɤrzɤβrzɤβ}{\papi{ ɣɤrzɤβrzɤβ}}}\markboth{ɣɤrzɤβrzɤβ}{}
\begin{relation-sémantique}\confer{
\hyperlink{Ⓔrzɤβrzɤβ}{\textit{ \papi{rzɤβrzɤβ}}}
}\end{relation-sémantique}\end{entrée}

\begin{entrée}
\vedette{\hypertarget{Ⓔɣɤrʑo}{\papi{ ɣɤrʑo}}}\markboth{ɣɤrʑo}{}
\classe{vt}
\paradigme{\textit{dir :} \jya pɯ-}
\paradigme{\textit{dir :} \jya nɯ-}
\begin{définition}\fra ne pas exiger que qqn repaie sa dette\end{définition}
\begin{définition}\cmn 免去(别人欠的钱)\end{définition}
\begin{exemple}\jya nɯ-ɣɤrʑo-t-a\cmn 我没有要他还钱\end{exemple}
\begin{exemple}\jya nɤʑo kɯ aʑo pɯ-kɯ-ɣɤrʑo-a\cmn 你没有要我还钱\end{exemple}
\begin{exemple}\jya jiɕqha nɯ kɯ a-sci pɕawtsɯ sqɯ-mpɕar ɲɯ-khɤm pɯ-ra ri, kɯmŋu-mpɕar ma mɯ-nɯ-mɟat-a pɯ-ɣɤrʑo-t-a\cmn 他本来应该给我十块钱,但是我只拿到了五块,我没有要他还钱\end{exemple}
\begin{exemple}\jya jiɕqha nɯ kɯ tɤɕi sqɯ-tɯrpa ɲɯ-khɤm pɯ-ra ri, kɯmŋu-tɯrpa ma mɯ-nɯ-mɟa-t-a tɕe pɯ-ɣɤrʑo-t-a\cmn 他本来应该给我十斤青稞,但是我只拿到了五斤,我没有要他还钱\end{exemple}
\begin{exemple}\jya ɲɯ-kɯ-ɣɤrʑo-a ɯ́-jɤɣ?\cmn 求你免我的债\end{exemple}
\begin{exemple}\jya kɯki laχtɕha ki ɯ-phɯ khro tsa ɲɯ-βze ɕti ri, tsuku ɲɯ-ta-ɣɤrʑo jɤɣ\cmn 这个东西价钱有点高,可以给你降低一点\end{exemple}\end{entrée}

\begin{entrée}
\vedette{\hypertarget{Ⓔɣɤrʑɯɣlɯɣ}{\papi{ ɣɤrʑɯɣlɯɣ}}}\markboth{ɣɤrʑɯɣlɯɣ}{}
\begin{relation-sémantique}\confer{
\hyperlink{Ⓔrʑɯɣrʑɯɣ}{\textit{ \papi{rʑɯɣrʑɯɣ}}}
}\end{relation-sémantique}
\end{entrée}

\begin{entrée}
\vedette{\hypertarget{Ⓔɣɤʁjɤβlɤβ}{\papi{ ɣɤʁjɤβlɤβ}}}\markboth{ɣɤʁjɤβlɤβ}{}
\classe{vi}
\begin{définition}\fra agir en cachette\end{définition}
\begin{définition}\cmn 偷偷摸摸;鬼鬼祟祟\end{définition}
\begin{exemple}\jya jiɕqha nɯ kɯ-ɣɤʁjɤβlɤβ ci ɲɯ-ŋu\cmn 那个(人)是个鬼鬼祟祟的人\end{exemple}
\begin{exemple}\jya ɯʑo ɲɯ-ɣɤʁjɤβlɤβ tɕe, ɯ-stu mɤ-nɤme\cmn 他这样鬼鬼祟祟的样子,不会做什么好事\end{exemple}
\begin{relation-sémantique}\confer{
\hyperlink{Ⓔɣɤqhɤβjɤβ}{\textit{ \papi{ɣɤqhɤβjɤβ}}}
}\end{relation-sémantique}\end{entrée}

\begin{entrée}
\vedette{\hypertarget{Ⓔɣɤʁnɤt}{\papi{ ɣɤʁnɤt}}}\markboth{ɣɤʁnɤt}{}
\begin{relation-sémantique}\confer{
\hyperlink{Ⓔʁnɤt}{\textit{ \papi{ʁnɤt}}}
}\end{relation-sémantique}\end{entrée}

\begin{entrée}
\vedette{\hypertarget{Ⓔɣɤʁre}{\papi{ ɣɤʁre}}}\markboth{ɣɤʁre}{}
\classe{vs}
\begin{définition}\fra être respecté\end{définition}
\begin{définition}\cmn 受人尊重\end{définition}
\begin{exemple}\jya ɲɯ-ɣɤʁre (=ɯ-ʁre)\cmn 他受人尊重\end{exemple}
\begin{exemple}\jya jiɕqha nɯ kɯ-ɣɤʁre ci ŋu\cmn 他是一个受尊重的人\end{exemple}
\begin{relation-sémantique}\synonyme{
\hyperlink{Ⓔŋgro}{\textit{ \papi{ŋgro}}}
}\end{relation-sémantique}
\begin{relation-sémantique}\confer{
\hyperlink{Ⓔsaʁre}{\textit{ \papi{saʁre}}}
}\end{relation-sémantique}
\begin{relation-sémantique}\confer{
\hyperlink{Ⓔnaʁre}{\textit{ \papi{naʁre}}}
}\end{relation-sémantique}
\begin{relation-sémantique}\confer{
\hyperlink{Ⓔɯ-ʁre}{\textit{ \papi{ɯ-ʁre}}}
}\end{relation-sémantique}\begin{sous-entrée}
\vedette{\hypertarget{}{\papi{ azɣɤʁrɯʁre}}}\markboth{azɣɤʁrɯʁre}{}\classe{vi}
\begin{définition}\ 
\begin{déclaration}\grammar{refl}\end{déclaration}\end{définition}
\begin{définition}\fra se respecter les uns les autres\end{définition}
\begin{définition}\cmn 互相尊重\end{définition}
\begin{exemple}\jya tɕiʑo ʑɤŋgɯz azɣɤʁrɯʁre-tɕi\cmn 我们俩互相尊重\end{exemple}
\end{sous-entrée}\begin{sous-entrée}
\vedette{\hypertarget{}{\papi{ zɣɤʁre}}}\markboth{zɣɤʁre}{}\classe{vt}
\paradigme{\textit{dir :} \jya tɤ-}
\begin{définition}\ 
\begin{déclaration}\grammar{caus}\end{déclaration}\end{définition}
\begin{définition}\fra respecter\end{définition}
\begin{définition}\cmn 尊重\end{définition}
\begin{exemple}\jya tu-kɯ-zɣɤʁre-a\cmn 你尊重我\end{exemple}
\end{sous-entrée}\end{entrée}

\begin{entrée}
\vedette{\hypertarget{Ⓔɣɤʁrɯ}{\papi{ ɣɤʁrɯ}}}\markboth{ɣɤʁrɯ}{}
\classe{vi}
\paradigme{\textit{dir :} \jya nɯ-}
\paradigme{\textit{dir :} \jya lɤ-}
\begin{définition}\ 
\begin{déclaration}\grammar{denom}\end{déclaration}\end{définition}
\begin{définition}\fra faire des pousses\end{définition}
\begin{définition}\cmn 发芽(种子)\end{définition}
\begin{exemple}\jya jiɕqha qaj nɯ tɯ-mɯ kɯ pjɤ-χtɕi tɕe ɲɤ-ɣɤʁrɯ\cmn 雨把小麦打湿了就生芽了\end{exemple}
\begin{relation-sémantique}\confer{
\hyperlink{Ⓔta-ʁrɯ}{\textit{ \papi{ta-ʁrɯ}}}
}\end{relation-sémantique}\end{entrée}

\begin{entrée}
\vedette{\hypertarget{Ⓔɣɤʁrɯqa}{\papi{ ɣɤʁrɯqa}}}\markboth{ɣɤʁrɯqa}{}\classe{vi}
\paradigme{\textit{dir :} \jya pɯ-}
\begin{définition}\fra chicaner\end{définition}
\begin{définition}\cmn 计较;反驳\end{définition}
\begin{relation-sémantique}\confer{
\hyperlink{Ⓔsɤtɕɯqaʁ}{\textit{ \papi{sɤtɕɯqaʁ}}}
}\end{relation-sémantique}\end{entrée}

\begin{entrée}
\vedette{\hypertarget{Ⓔɣɤʁzɤβ}{\papi{ ɣɤʁzɤβ}}}\markboth{ɣɤʁzɤβ}{}
\classe{vs}
\begin{définition}\fra grandiose\end{définition}
\begin{définition}\cmn 隆重
\begin{déclaration} \étymologie{\papi{gzab}}\end{déclaration}
\begin{déclaration}\use{沙尔宗方言}\end{déclaration}\end{définition}
\begin{exemple}\jya tɤʁaʁ ɲɯ-ɣɤʁzɤβ\cmn 聚会很隆重\end{exemple}
\begin{exemple}\jya khɤjhwi ɲɯ-ɣɤʁzɤβ\cmn 开会很隆重\end{exemple}\end{entrée}

\begin{entrée}
\vedette{\hypertarget{Ⓔɣɤsa}{\papi{ ɣɤsa}}}\markboth{ɣɤsa}{}
\begin{relation-sémantique}\confer{
\hyperlink{Ⓔsa}{\textit{ \papi{sa}}}
}\end{relation-sémantique}\end{entrée}

\begin{entrée}
\vedette{\hypertarget{Ⓔɣɤsɤmbrɯ}{\papi{ ɣɤsɤmbrɯ}}}\markboth{ɣɤsɤmbrɯ}{}
\classe{vi}
\begin{définition}\fra s'énerver facilement\end{définition}
\begin{définition}\cmn 容易生气\end{définition}
\begin{exemple}\jya jiɕqha nɯ wuma kɯ-ɣɤsɤmbrɯ ci ɲɯ-ŋu\cmn 这个人脾气不好,容易生气\end{exemple}
\begin{relation-sémantique}\confer{
\hyperlink{Ⓔsɤmbrɯ}{\textit{ \papi{sɤmbrɯ}}}
}\end{relation-sémantique}\end{entrée}

\begin{entrée}
\vedette{\hypertarget{Ⓔɣɤscɤscɤt}{\papi{ ɣɤscɤscɤt}}}\markboth{ɣɤscɤscɤt}{}\classe{vi}
\begin{définition}\fra (frapper, crier) à toute vitesse\end{définition}
\begin{définition}\cmn (叫、拍)得很快\end{définition}
\begin{exemple}\jya taχphe ɲɯ-sɤscɤscɤt ʑo ta-lɤt\cmn 他拍手拍得很快\end{exemple}\end{entrée}

\begin{entrée}
\vedette{\hypertarget{Ⓔɣɤscraʁscraʁ}{\papi{ ɣɤscraʁscraʁ}}}\markboth{ɣɤscraʁscraʁ}{}
\begin{relation-sémantique}\confer{
\hyperlink{Ⓔscraʁscraʁ}{\textit{ \papi{scraʁscraʁ}}}
}\end{relation-sémantique}\end{entrée}

\begin{entrée}
\vedette{\hypertarget{ⒺɣɤsmiⒽ1Ⓗ1}{\papi{ ɣɤsmi}}}\markboth{ɣɤsmi}{}\homonyme{1}
\classe{vt}
\paradigme{\textit{dir :} \jya kɤ-}
\begin{définition}\ 
\begin{déclaration}\grammar{caus}\end{déclaration}\end{définition}
\begin{définition}\fra faire cuire\end{définition}
\begin{définition}\cmn 煮熟\end{définition}
\begin{exemple}\jya jaŋjy kɤ-sqa-t-a ri mɯ́j-smi tɕe, mɤʑɯ kɤ-ɣɤsmi-t-a\cmn 我煮了土豆没有煮熟,所以又煮了一次\end{exemple}
\begin{exemple}\jya ndzɤtshi kɤ-βzu kɤ-ɣɤsmi ɲɯ-ra\cmn 做饭的时候要煮熟\end{exemple}
\begin{exemple}\jya kɤ-sqa-t-a tɕe, kɤ-ɣɤsmi-t-a\cmn 我煮熟了\end{exemple}\begin{sous-entrée}
\vedette{\hypertarget{}{\papi{ ɣɤsmi}}}\markboth{ɣɤsmi}{}\classe{vs}
\begin{définition}\fra facile à cuire\end{définition}
\begin{définition}\cmn 容易煮\end{définition}
\begin{relation-sémantique}\confer{
\hyperlink{ⒺsmiⒽ1}{\textit{ \papi{smi1}}}
}\end{relation-sémantique}
\end{sous-entrée}\end{entrée}

\begin{entrée}
\vedette{\hypertarget{Ⓔɣɤsna}{\papi{ ɣɤsna}}}\markboth{ɣɤsna}{}
\classe{vt}
\begin{définition}\ 
\begin{déclaration}\grammar{caus}\end{déclaration}\end{définition}\acception{1}
\paradigme{\textit{dir :} \jya tɤ-}
\begin{définition}\fra réparer\end{définition}
\begin{définition}\cmn 修理\end{définition}
\begin{exemple}\jya qaʁ ɯ-jɯ mɯ́j-sna tɕe nɯ-βzu-t-a tɕe tɤ-ɣɤsna-t-a\cmn 锄头的把子不好,我把它修好了\end{exemple}
\begin{exemple}\jya a-khɯtsa ɲɤ-spoʁ tɕe, na-sti tɕe ta-ɣɤsna\cmn 我的碗破了个洞,他塞住了就可以用了\end{exemple}\acception{2}
\begin{définition}\fra faire du thé trop fort\end{définition}
\begin{définition}\cmn 熬得太浓\end{définition}
\begin{exemple}\jya tʂha kɤ-ta-t-a tɕe kɤ-ɣɤsna-t-a\cmn 我熬了茶熬得很浓\end{exemple}\end{entrée}

\begin{entrée}
\vedette{\hypertarget{Ⓔɣɤsŋaβ}{\papi{ ɣɤsŋaβ}}}\markboth{ɣɤsŋaβ}{}
\classe{vs}
\paradigme{\textit{dir :} \jya tɤ-}
\paradigme{\textit{dir :} \jya thɯ-}
\begin{définition}\fra étouffant\end{définition}
\begin{définition}\cmn 闷热\end{définition}
\begin{exemple}\jya jisŋi ɲɯ-ɣɤsŋaβ\cmn 今天很闷热\end{exemple}
\begin{exemple}\jya kha ɯ-ŋgɯ smi chɯ́-wɣ-βlɯ tɕe ɲɯ-ɣɤsŋaβ\cmn 屋子里烧了火,很闷热\end{exemple}
\begin{relation-sémantique}\synonyme{
\hyperlink{Ⓔɣɯtshɤdɯɣ}{\textit{ \papi{ɣɯtshɤdɯɣ}}}
}\end{relation-sémantique}\end{entrée}

\begin{entrée}
\vedette{\hypertarget{Ⓔɣɤsoŋsoŋ}{\papi{ ɣɤsoŋsoŋ}}}\markboth{ɣɤsoŋsoŋ}{}
\classe{vs}
\paradigme{\textit{dir :} \jya tɤ-}
\begin{définition}\fra émettre un bruit d'ébullition\end{définition}
\begin{définition}\cmn 发出沸腾的声音\end{définition}
\begin{exemple}\jya tɤ-lu tɤ-ala tɕe, tu-ɣɤsoŋsoŋ ʑo tu-mbuz ŋu, wuma ʑo ɣɤji\cmn 牛奶烧开的时候发出沸腾的声音,速度很快地溢出来\end{exemple}
\begin{exemple}\jya tɯ-ci to-ɣɤsoŋsoŋ ʑo to-mbuz\cmn 水发出沸腾声溢出来了\end{exemple}
\begin{relation-sémantique}\confer{
\hyperlink{Ⓔɣɤsthɯsthoŋ}{\textit{ \papi{ɣɤsthɯsthoŋ}}}
}\end{relation-sémantique}\end{entrée}

\begin{entrée}
\vedette{\hypertarget{Ⓔɣɤstaŋlaŋ}{\papi{ ɣɤstaŋlaŋ}}}\markboth{ɣɤstaŋlaŋ}{}\classe{vi}
\begin{définition}\fra sautiller\end{définition}
\begin{définition}\cmn 跳来跳去(大的圆形的东西)\end{définition}
\begin{exemple}\jya rɯdaʁ ra ɲɯ-ɣɤstaŋlaŋ-nɯ ʑo ɲɯ-phɣo-nɯ ɲɯ-ŋu\cmn 野兽一跳一跳地逃跑\end{exemple}
\begin{exemple}\jya @piqiu ɲɯ-ɣɤstaŋlaŋ\cmn 皮球在弹来弹去\end{exemple}
\begin{relation-sémantique}\synonyme{
\hyperlink{Ⓔstɯrstɯr}{\textit{ \papi{stɯrstɯr}}}
}\end{relation-sémantique}
\begin{relation-sémantique}\synonyme{
\hyperlink{Ⓔstɤjstɤj}{\textit{ \papi{stɤjstɤj}}}
}\end{relation-sémantique}\end{entrée}

\begin{entrée}
\vedette{\hypertarget{Ⓔɣɤstɤjlɤj}{\papi{ ɣɤstɤjlɤj}}}\markboth{ɣɤstɤjlɤj}{}
\begin{relation-sémantique}\confer{
\hyperlink{Ⓔstɤjstɤj}{\textit{ \papi{stɤjstɤj}}}
}\end{relation-sémantique}\end{entrée}

\begin{entrée}
\vedette{\hypertarget{Ⓔɣɤsthɯsthoŋ}{\papi{ ɣɤsthɯsthoŋ}}}\markboth{ɣɤsthɯsthoŋ}{}
\classe{vs}
\paradigme{\textit{dir :} \jya tɤ-}
\begin{définition}\fra bruit de l'eau qui bout\end{définition}
\begin{définition}\cmn 发出沸腾的声音\end{définition}
\begin{exemple}\jya tɯ-ci tala tɕe ɲɯɣɤsthɯtshoŋ ʑo ɲɯ-mbuz\cmn 水发出沸腾的声音就溢出来了\end{exemple}
\begin{relation-sémantique}\confer{
\hyperlink{Ⓔɣɤsoŋsoŋ}{\textit{ \papi{ɣɤsoŋsoŋ}}}
}\end{relation-sémantique}\end{entrée}

\begin{entrée}
\vedette{\hypertarget{Ⓔɣɤsthɯsthrɯβ}{\papi{ ɣɤsthɯsthrɯβ}}}\markboth{ɣɤsthɯsthrɯβ}{}\classe{vi}\begin{sous-entrée}
\vedette{\hypertarget{}{\papi{ sɤsthɯsthrɯβ}}}\markboth{sɤsthɯsthrɯβ}{}\classe{vt}
\begin{définition}\fra se moucher le nez bruyamment\end{définition}
\begin{définition}\cmn 擤鼻涕的时候发出声音\end{définition}
\begin{exemple}\jya ɯ-ɕnaβ ta-sɤsthɯsthrɯβ ʑo tha-pɕiz\cmn 他擤了鼻涕,发出了很响的声音\end{exemple}
\begin{relation-sémantique}\confer{
\hyperlink{Ⓔsthrɯβ}{\textit{ \papi{sthrɯβ}}}
}\end{relation-sémantique}
\end{sous-entrée}\end{entrée}

\begin{entrée}
\vedette{\hypertarget{Ⓔɣɤstɯrlɯr}{\papi{ ɣɤstɯrlɯr}}}\markboth{ɣɤstɯrlɯr}{}
\begin{relation-sémantique}\confer{
\hyperlink{Ⓔstɯrstɯr}{\textit{ \papi{stɯrstɯr}}}
}\end{relation-sémantique}\end{entrée}

\begin{entrée}
\vedette{\hypertarget{Ⓔɣɤsɯβsɯβ}{\papi{ ɣɤsɯβsɯβ}}}\markboth{ɣɤsɯβsɯβ}{}
\begin{relation-sémantique}\confer{
\hyperlink{Ⓔsɯβsɯβ}{\textit{ \papi{sɯβsɯβ}}}
}\end{relation-sémantique}\end{entrée}

\begin{entrée}
\vedette{\hypertarget{Ⓔɣɤsɯɣsɯɣ}{\papi{ ɣɤsɯɣsɯɣ}}}\markboth{ɣɤsɯɣsɯɣ}{}\classe{vi}
\paradigme{\textit{dir :} \jya nɯ-}
\begin{définition}\ 
\begin{déclaration}\grammar{deidph}\end{déclaration}\end{définition}
\begin{définition}\fra gigoter\end{définition}
\begin{définition}\cmn 不停地乱动,令人讨厌
\begin{déclaration}\use{特别指小孩子}\end{déclaration}\end{définition}
\begin{exemple}\jya tɤ-pɤtso mɯ́j-khɯ tɕe ɲɯ-ɣɤsɯɣsɯɣ ʑo\cmn 小孩子不听话,不停地乱动\end{exemple}
\begin{exemple}\jya ma-tɯ-ɣɤsɯɣsɯɣ kɯ kɤ-rɤʑi\cmn 不要乱动\end{exemple}\end{entrée}

\begin{entrée}
\vedette{\hypertarget{Ⓔɣɤʂɲɯɣlɯɣ}{\papi{ ɣɤʂɲɯɣlɯɣ}}}\markboth{ɣɤʂɲɯɣlɯɣ}{}\classe{vi}
\begin{définition}\fra gigoter dans tous les sens\end{définition}
\begin{définition}\cmn 动来动去\end{définition}
\begin{exemple}\jya ma-tɯ-ɣɤʂɲɯɣlɯɣ\cmn 你不要动来动去\end{exemple}
\begin{relation-sémantique}\confer{
\hyperlink{Ⓔʂɲɯɣnɤʂɲɯɣ}{\textit{ \papi{ʂɲɯɣnɤʂɲɯɣ}}}
}\end{relation-sémantique}\begin{sous-entrée}
\vedette{\hypertarget{}{\papi{ sɤʂɲɯɣlɯɣ}}}\markboth{sɤʂɲɯɣlɯɣ}{}\classe{vt}
\begin{définition}\fra gigoter dans tous les sens\end{définition}
\begin{définition}\cmn 动来动去\end{définition}
\end{sous-entrée}\end{entrée}

\begin{entrée}
\vedette{\hypertarget{Ⓔɣɤʂχaβʂχaβ}{\papi{ ɣɤʂχaβʂχaβ}}}\markboth{ɣɤʂχaβʂχaβ}{}
\classe{vi}
\begin{définition}\fra émettre un bruit de froissement\end{définition}
\begin{définition}\cmn 发出沙沙声\end{définition}
\begin{exemple}\jya ɲɯ-ɣɤʂχaβʂχaβ\cmn (皮子)发出沙沙声\end{exemple}
\begin{relation-sémantique}\confer{
\hyperlink{Ⓔɣɤʂχɯʂχɯβ}{\textit{ \papi{ɣɤʂχɯʂχɯβ}}}
}\end{relation-sémantique}\end{entrée}

\begin{entrée}
\vedette{\hypertarget{Ⓔɣɤʂχɯʂχɯβ}{\papi{ ɣɤʂχɯʂχɯβ}}}\markboth{ɣɤʂχɯʂχɯβ}{}
\begin{relation-sémantique}\confer{
\hyperlink{Ⓔsɤʂχɯʂχɯβ}{\textit{ \papi{sɤʂχɯʂχɯβ}}}
}\end{relation-sémantique}\end{entrée}

\begin{entrée}
\vedette{\hypertarget{Ⓔɣɤtu}{\papi{ ɣɤtu}}}\markboth{ɣɤtu}{}\classe{vt}
\paradigme{\textit{dir :} \jya tɤ-}
\begin{définition}\fra faire exister\end{définition}
\begin{définition}\cmn 使其存在\end{définition}
\begin{exemple}\jya nɤ-skɤt tɤ-ɣɤte\cmn 你要发出声音\end{exemple}
\begin{exemple}\jya nɤ-χsoŋχsɤz tɤ-ɣɤte\cmn 你要振作精神\end{exemple}
\begin{exemple}\jya nɤ-sŋɯro tɤ-ɣɤte\cmn 你要发出呼吸声\end{exemple}
\begin{exemple}\jya nɤ-kɤ-ntɕhoz kɯ-kɯ-ra nɯ tɤ-ɣɤtu-t-a\cmn 我给准备了你所有要用的东西\end{exemple}
\begin{relation-sémantique}\antonyme{
\hyperlink{Ⓔɣɤme}{\textit{ \papi{ɣɤme}}}
}\end{relation-sémantique}
\begin{relation-sémantique}\confer{
\hyperlink{Ⓔtu}{\textit{ \papi{tu}}}
}\end{relation-sémantique}\end{entrée}

\begin{entrée}
\vedette{\hypertarget{Ⓔɣɤtaʁ}{\papi{ ɣɤtaʁ}}}\markboth{ɣɤtaʁ}{}
\begin{relation-sémantique}\confer{
\hyperlink{ⒺtaʁⒽ2}{\textit{ \papi{taʁ2}}}
}\end{relation-sémantique}\end{entrée}

\begin{entrée}
\vedette{\hypertarget{Ⓔɣɤtɕa}{\papi{ ɣɤtɕa}}}\markboth{ɣɤtɕa}{}\classe{vi}
\paradigme{\textit{dir :} \jya pɯ-}
\begin{définition}\fra avoir tord\end{définition}
\begin{définition}\cmn 错;犯错误\end{définition}
\begin{exemple}\jya pɯ-ɣɤtɕa-a\cmn 我错了\end{exemple}
\begin{exemple}\jya ɯʑo kɯ kɯmaʁ tu-ti ɲɯ-ŋu tɕe ɲɯ-ɣɤtɕa\cmn 他说得不对,他是错的\end{exemple}
\begin{exemple}\jya aʑo kɯmaʁ to-ti-a tɕe pjɤ-ɣɤtɕa-a\cmn 我说得不对,我错了\end{exemple}
\begin{exemple}\jya pɯ-ɣɤtɕa-a tɤ-ti\cmn 你要承认自己做错了\end{exemple}
\begin{relation-sémantique}\antonyme{
\hyperlink{Ⓔɣɤŋgi}{\textit{ \papi{ɣɤŋgi}}}
}\end{relation-sémantique}
\begin{relation-sémantique}\confer{
\hyperlink{Ⓔnɯɣɤtɕa}{\textit{ \papi{nɯɣɤtɕa}}}
}\end{relation-sémantique}\begin{sous-entrée}
\vedette{\hypertarget{}{\papi{ zɣɤtɕa}}}\markboth{zɣɤtɕa}{}\classe{vt}
\paradigme{\textit{dir :} \jya pɯ-}
\begin{définition}\fra ne pas être d'accord avec\end{définition}
\begin{définition}\cmn 不同意,认为别人是错的\end{définition}
\begin{exemple}\jya nɯ tɤ-tɯt-a ri ɯʑo kɯ pɯ́-wɣ-zɣɤtɕa\cmn 他说了那句说,但是他表示不同意\end{exemple}
\end{sous-entrée}\begin{sous-entrée}
\vedette{\hypertarget{}{\papi{ ʑɣɤɣɤtɕa}}}\markboth{ʑɣɤɣɤtɕa}{} (\variante{ʑɣɤzɣɤtɕa}) \classe{vi}
\paradigme{\textit{dir :} \jya pɯ-}
\begin{définition}\ 
\begin{déclaration}\grammar{refl}\end{déclaration}\end{définition}
\begin{définition}\fra admettre sa faute\end{définition}
\begin{définition}\cmn 承认自己的错\end{définition}
\begin{exemple}\jya pjɤ-ʑɣɤɣɤtɕa\cmn 他承认了自己的错误\end{exemple}
\end{sous-entrée}\end{entrée}

\begin{entrée}
\vedette{\hypertarget{Ⓔɣɤtɕɤt}{\papi{ ɣɤtɕɤt}}}\markboth{ɣɤtɕɤt}{}\classe{vt}\acception{1}
\paradigme{\textit{dir :} \jya tɤ-}
\begin{définition}\fra choisir parmi (un homme)\end{définition}
\begin{définition}\cmn 从中挑选(人)\end{définition}
\begin{exemple}\jya nɯ-ŋgɯz nɯ tɕu kɯ-mna ʁnɯz tú-wɣ-ɣɤtɕɤt ɲɯ-ra\cmn 他们当中要选两个当领导\end{exemple}\acception{2}
\paradigme{\textit{dir :} \jya pɯ-}
\paradigme{\textit{dir :} \jya tɤ-}
\begin{définition}\fra fournir\end{définition}
\begin{définition}\cmn 提供\end{définition}
\begin{exemple}\jya nɤʑo kha jɤ-ɣi jɤɣ ma nɤ-kɤndza cho nɤ-kɯ-ra pjɯ-ɣɤtɕat-a jɤɣ\cmn 你可以到我家来,我可以给你提供食物和你所需要的东西\end{exemple}
\begin{sous-entrée}
\vedette{\hypertarget{}{\papi{ ʑɣɤɣɤtɕɤt}}}\markboth{ʑɣɤɣɤtɕɤt}{}\classe{vi}
\paradigme{\textit{dir :} \jya tɤ-}
\begin{définition}\ 
\begin{déclaration}\grammar{refl}\end{déclaration}\end{définition}
\begin{définition}\fra se porter volontaire pour\end{définition}
\begin{définition}\cmn 自告奋勇\end{définition}
\begin{exemple}\jya kɯ-rɤma kɤ-ɕe ɲɯ-ra tɕe, aʑo tɤ-ʑɣɤɣɤtɕat-a\cmn 需要人,我自告奋勇去做了\end{exemple}
\end{sous-entrée}\end{entrée}

\begin{entrée}
\vedette{\hypertarget{Ⓔɣɤtɕɣɤrtɕɣɤr}{\papi{ ɣɤtɕɣɤrtɕɣɤr}}}\markboth{ɣɤtɕɣɤrtɕɣɤr}{}
\begin{relation-sémantique}\confer{
\hyperlink{Ⓔtɕɣɤrtɕɣɤr}{\textit{ \papi{tɕɣɤrtɕɣɤr}}}
}\end{relation-sémantique}\end{entrée}

\begin{entrée}
\vedette{\hypertarget{Ⓔɣɤtɕhaʁ}{\papi{ ɣɤtɕhaʁ}}}\markboth{ɣɤtɕhaʁ}{}\classe{vt}
\begin{définition}\ 
\begin{déclaration}\grammar{caus}\end{déclaration}\end{définition}
\begin{définition}\fra faire diminuer\end{définition}
\begin{définition}\cmn 减\end{définition}
\begin{relation-sémantique}\synonyme{
\hyperlink{Ⓔsɯxtɕhaʁ}{\textit{ \papi{sɯxtɕhaʁ}}}
}\end{relation-sémantique}
\begin{relation-sémantique}\confer{
\hyperlink{Ⓔtɕhaʁ}{\textit{ \papi{tɕhaʁ}}}
}\end{relation-sémantique}
\end{entrée}

\begin{entrée}
\vedette{\hypertarget{Ⓔɣɤtɕhom}{\papi{ ɣɤtɕhom}}}\markboth{ɣɤtɕhom}{}\classe{vt}
\paradigme{\textit{dir :} \jya \_}
\begin{définition}\ 
\begin{déclaration}\grammar{caus}\end{déclaration}\end{définition}
\begin{définition}\fra être excessif, dépasser la mesure\end{définition}
\begin{définition}\cmn 过分;过量;走过头\end{définition}
\begin{exemple}\jya tɤ-\end{exemple}
\begin{exemple}\jya aʑo tɤ-ari-a ri to-ɣɤtɕhom-a\cmn 我走过头了\end{exemple}
\begin{exemple}\jya aʑo jiɕqha rɟɤɣi ʁnɯz tɤ-ndzat-a to-ɣɤtɕhom-a\cmn 我吃了两坨糌粑,吃多了\end{exemple}
\begin{exemple}\jya cha (kɤ-tshi, ɯ-tshi) ko-ɣɤtɕhom\cmn 酒喝多了\end{exemple}
\begin{relation-sémantique}\confer{
\hyperlink{Ⓔtɕhom}{\textit{ \papi{tɕhom}}}
}\end{relation-sémantique}\end{entrée}

\begin{entrée}
\vedette{\hypertarget{Ⓔɣɤtɕhɯβtɕhɯβ}{\papi{ ɣɤtɕhɯβtɕhɯβ}}}\markboth{ɣɤtɕhɯβtɕhɯβ}{}
\begin{relation-sémantique}\confer{
\hyperlink{Ⓔtɕhɯβtɕhɯβ}{\textit{ \papi{tɕhɯβtɕhɯβ}}}
}\end{relation-sémantique}\end{entrée}

\begin{entrée}
\vedette{\hypertarget{Ⓔɣɤtɕur}{\papi{ ɣɤtɕur}}}\markboth{ɣɤtɕur}{}
\begin{relation-sémantique}\confer{
\hyperlink{ⒺtɕurⒽ1}{\textit{ \papi{tɕur1}}}
}\end{relation-sémantique}\end{entrée}

\begin{entrée}
\vedette{\hypertarget{Ⓔɣɤtɕɯɣ}{\papi{ ɣɤtɕɯɣ}}}\markboth{ɣɤtɕɯɣ}{}\classe{vi}
\paradigme{\textit{dir :} \jya nɯ-}
\begin{définition}\fra germer (arbre)\end{définition}
\begin{définition}\cmn 发芽(树)\end{définition}
\begin{exemple}\jya ʑmbri ɲɤ-ɣɤtɕɯɣ\cmn 柳树发芽了\end{exemple}
\begin{exemple}\jya mi ɲɤ-ɣɤtɕɯɣ\cmn 杨树发芽了\end{exemple}\begin{sous-entrée}
\vedette{\hypertarget{}{\papi{ ɣɤɣɤtɕɯɣ}}}\markboth{ɣɤɣɤtɕɯɣ}{}\classe{vs}
\paradigme{\textit{dir :} \jya nɯ-}
\begin{définition}\ 
\begin{déclaration}\grammar{facil}\end{déclaration}\end{définition}
\begin{définition}\fra germer précocement (arbre)\end{définition}
\begin{définition}\cmn 发芽发得早(树)\end{définition}
\begin{exemple}\jya ʑmbri ɲɯ-ɣɤɣɤtɕɯɣ\cmn 柳树发芽发得很早\end{exemple}
\begin{exemple}\jya mi ɲɯ-ɣɤɣɤtɕɯɣ\cmn 杨树发芽发得很早\end{exemple}
\begin{relation-sémantique}\confer{
\hyperlink{Ⓔtɤ-tɕɯɣ}{\textit{ \papi{tɤ-tɕɯɣ}}}
}\end{relation-sémantique}
\end{sous-entrée}\end{entrée}

\begin{entrée}
\vedette{\hypertarget{Ⓔɣɤtɕɯqaʁ}{\papi{ ɣɤtɕɯqaʁ}}}\markboth{ɣɤtɕɯqaʁ}{}
\begin{relation-sémantique}\confer{
\hyperlink{Ⓔsɤtɕɯqaʁ}{\textit{ \papi{sɤtɕɯqaʁ}}}
}\end{relation-sémantique}\end{entrée}

\begin{entrée}
\vedette{\hypertarget{Ⓔɣɤthɤβjɤβ}{\papi{ ɣɤthɤβjɤβ}}}\markboth{ɣɤthɤβjɤβ}{}
\classe{vi}
\begin{définition}\fra chercher des choses n'importe comment\end{définition}
\begin{définition}\cmn 乱找东西(小孩子)\end{définition}
\begin{exemple}\jya ma-tɯ-ɣɤthɤβjɤβ\cmn 你不要乱找东西\end{exemple}\end{entrée}

\begin{entrée}
\vedette{\hypertarget{Ⓔɣɤthɤβthɤβ}{\papi{ ɣɤthɤβthɤβ}}}\markboth{ɣɤthɤβthɤβ}{}
\begin{relation-sémantique}\confer{
\hyperlink{Ⓔsɤthɤβthɤβ}{\textit{ \papi{sɤthɤβthɤβ}}}
}\end{relation-sémantique}\end{entrée}

\begin{entrée}
\vedette{\hypertarget{Ⓔɣɤthɣɤthɣɤt}{\papi{ ɣɤthɣɤthɣɤt}}}\markboth{ɣɤthɣɤthɣɤt}{}\classe{vi}
\begin{définition}\fra trembler\end{définition}
\begin{définition}\cmn 发抖\end{définition}
\begin{exemple}\jya ɲɯ-ngo tɕe ɲɯ-ɣɤthɣɤthɣɤt\cmn 他病了,在发抖\end{exemple}\begin{sous-entrée}
\vedette{\hypertarget{}{\papi{ sɤthɣɤthɣɤt}}}\markboth{sɤthɣɤthɣɤt}{}\classe{vt}
\begin{définition}\fra gigoter\end{définition}
\begin{définition}\cmn 不停地乱动\end{définition}
\begin{exemple}\jya ɯ-mi ɲɯ-sɤthɣɤthɣɤt\cmn 他在抖脚\end{exemple}
\begin{exemple}\jya ma-tɯ-sɤthɣɤthɣɤt\cmn 你不要不停地乱动\end{exemple}
\begin{relation-sémantique}\synonyme{
\hyperlink{Ⓔɣɤndzɯrndzɯr}{\textit{ \papi{ɣɤndzɯrndzɯr}}}
}\end{relation-sémantique}
\end{sous-entrée}\end{entrée}

\begin{entrée}
\vedette{\hypertarget{Ⓔɣɤthɯ}{\papi{ ɣɤthɯ}}}\markboth{ɣɤthɯ}{}
\begin{relation-sémantique}\confer{
\hyperlink{ⒺthɯⒽ2}{\textit{ \papi{thɯ2}}}
}\end{relation-sémantique}
\end{entrée}

\begin{entrée}
\vedette{\hypertarget{Ⓔɣɤtsɤngo}{\papi{ ɣɤtsɤngo}}}\markboth{ɣɤtsɤngo}{}\classe{vs}
\begin{définition}\fra tomber souvent malade\end{définition}
\begin{définition}\cmn 经常生病\end{définition}
\begin{exemple}\jya ɲɯ-ɣɤtsɤngo-a\cmn 我经常生病\end{exemple}
\begin{relation-sémantique}\confer{
\hyperlink{Ⓔngo}{\textit{ \papi{ngo}}}
}\end{relation-sémantique}\end{entrée}

\begin{entrée}
\vedette{\hypertarget{Ⓔɣɤtsɣaʁtsɣaʁ}{\papi{ ɣɤtsɣaʁtsɣaʁ}}}\markboth{ɣɤtsɣaʁtsɣaʁ}{}
\classe{vs}
\paradigme{\textit{dir :} \jya tɤ-}
\begin{définition}\ 
\begin{déclaration}\grammar{deidph}\end{déclaration}\end{définition}
\begin{définition}\fra pénible à supporter (comme la piqûre d'une aiguille )\end{définition}
\begin{définition}\cmn 令人痛、难受(像扎针一样)\end{définition}
\begin{exemple}\jya ɲɯ-mɤrtsaβ ɲɯ-ɣɤtsɣaʁtsɣaʁ\cmn 太辣,很难受\end{exemple}
\begin{exemple}\jya ɯ-tɯ-sɤɕke kɯ ɲɯ-ɣɤtsɣaʁtsɣaʁ\cmn 太烫,很难受\end{exemple}\end{entrée}

\begin{entrée}
\vedette{\hypertarget{Ⓔɣɤtshu}{\papi{ ɣɤtshu}}}\markboth{ɣɤtshu}{}
\classe{vt}
\paradigme{\textit{dir :} \jya thɯ-}
\begin{définition}\ 
\begin{déclaration}\grammar{caus}\end{déclaration}\end{définition}
\begin{définition}\fra faire grossir\end{définition}
\begin{définition}\cmn 使变胖\end{définition}
\begin{exemple}\jya paʁ nɯ (stoʁ) khro pɯ-mbi-j tɕe thɯ-ɣɤtshu-j\cmn 我们给猪喂了很多胡豆,把它养肥了\end{exemple}
\begin{exemple}\jya chó-wɣ-z-nɯɣmbaβ-a ɕti ma chó-wɣ-ɣɤtshu-a maʁ\cmn (蚊子)令我肿了,没有令我变胖(笑话)\end{exemple}
\begin{relation-sémantique}\confer{
\hyperlink{Ⓔtshu}{\textit{ \papi{tshu}}}
}\end{relation-sémantique}\end{entrée}

\begin{entrée}
\vedette{\hypertarget{Ⓔɣɤtsha}{\papi{ ɣɤtsha}}}\markboth{ɣɤtsha}{}
\begin{relation-sémantique}\confer{
 \papi{ɯ-rɕa,ɣɤtsha}
}\end{relation-sémantique}
\end{entrée}

\begin{entrée}
\vedette{\hypertarget{Ⓔɣɤtshoz}{\papi{ ɣɤtshoz}}}\markboth{ɣɤtshoz}{}
\begin{relation-sémantique}\confer{
\hyperlink{Ⓔtshoz}{\textit{ \papi{tshoz}}}
}\end{relation-sémantique}\end{entrée}

\begin{entrée}
\vedette{\hypertarget{Ⓔɣɤtsri}{\papi{ ɣɤtsri}}}\markboth{ɣɤtsri}{}
\classe{vt}
\paradigme{\textit{dir :} \jya pɯ-}
\paradigme{\textit{dir :} \jya lɤ-}
\begin{définition}\fra saler\end{définition}
\begin{définition}\cmn 放盐;令……变得更咸\end{définition}
\begin{exemple}\jya pɯ-ɣɤtsri-t-a\cmn 我放了盐\end{exemple}
\begin{relation-sémantique}\confer{
\hyperlink{Ⓔtsri}{\textit{ \papi{tsri}}}
}\end{relation-sémantique}\end{entrée}

\begin{entrée}
\vedette{\hypertarget{Ⓔɣɤtsrɯ}{\papi{ ɣɤtsrɯ}}}\markboth{ɣɤtsrɯ}{}
\classe{vs}
\paradigme{\textit{dir :} \jya nɯ-}
\paradigme{\textit{dir :} \jya tɤ-}
\begin{définition}\ 
\begin{déclaration}\grammar{denom}\end{déclaration}\end{définition}
\begin{définition}\fra germer\end{définition}
\begin{définition}\cmn 发芽\end{définition}
\begin{exemple}\jya tɤɕi to-ɣɤtsrɯ\cmn 青稞发芽了\end{exemple}
\begin{exemple}\jya stoʁ ɲɯ-ɣɤtsrɯ\cmn 胡豆在发芽\end{exemple}
\begin{relation-sémantique}\confer{
\hyperlink{Ⓔtɤ-tsrɯ}{\textit{ \papi{tɤ-tsrɯ}}}
}\end{relation-sémantique}\end{entrée}

\begin{entrée}
\vedette{\hypertarget{Ⓔɣɤtsɯr}{\papi{ ɣɤtsɯr}}}\markboth{ɣɤtsɯr}{}
\classe{vi}
\paradigme{\textit{dir :} \jya pɯ-}
\begin{définition}\ 
\begin{déclaration}\grammar{denom}\end{déclaration}\end{définition}
\begin{définition}\fra se fêler, avoir des gerçures\end{définition}
\begin{définition}\cmn 裂口;龟裂
\begin{déclaration}\use{pjɤ-ɣɤtsɯr = tɤ-tsɯr pjɤ-ɕe}\end{déclaration}\end{définition}
\begin{exemple}\jya ki znde ki pjɤ-ɣɤtsɯr\cmn 这堵墙裂了\end{exemple}
\begin{exemple}\jya a-jaʁ pjɤ-ɣɤtsɯr\cmn 我的手裂了\end{exemple}
\begin{relation-sémantique}\confer{
\hyperlink{Ⓔtɤ-tsɯr}{\textit{ \papi{tɤ-tsɯr}}}
}\end{relation-sémantique}\end{entrée}

\begin{entrée}
\vedette{\hypertarget{Ⓔɣɤtsɯtsrɯɣ}{\papi{ ɣɤtsɯtsrɯɣ}}}\markboth{ɣɤtsɯtsrɯɣ}{}
\classe{vs}
\paradigme{\textit{dir :} \jya tɤ-}
\begin{définition}\ 
\begin{déclaration}\grammar{deidph}\end{déclaration}\end{définition}
\begin{définition}\fra grincer (bois)\end{définition}
\begin{définition}\cmn 发出尖锐的嘎吱声(木头)\end{définition}
\begin{exemple}\jya tɯ-sta ɲɯ-ɣɤtsɯtsrɯɣ\cmn 床在嘎吱嘎吱响\end{exemple}
\begin{exemple}\jya si ɲɯ-ɣɤtsɯtsrɯɣ\cmn 木头在嘎吱嘎吱响\end{exemple}\end{entrée}

\begin{entrée}
\vedette{\hypertarget{Ⓔɣɤtʂɤjɤt}{\papi{ ɣɤtʂɤjɤt}}}\markboth{ɣɤtʂɤjɤt}{}\classe{vi}
\paradigme{\textit{dir :} \jya tɤ-}
\begin{définition}\fra bouger dans tous les sens (enfant)\end{définition}
\begin{définition}\cmn 乱动(孩子)\end{définition}\end{entrée}

\begin{entrée}
\vedette{\hypertarget{Ⓔɣɤtʂhɯtʂhɯt}{\papi{ ɣɤtʂhɯtʂhɯt}}}\markboth{ɣɤtʂhɯtʂhɯt}{} (\variante{ɣɤtʂhɯtʂhɯz}) 
\classe{vi}
\begin{définition}\ 
\begin{déclaration}\grammar{deidph}\end{déclaration}\end{définition}
\begin{définition}\fra crépiter (feu)\end{définition}
\begin{définition}\cmn 火烧的时候,发出噼啪噼啪声\end{définition}
\begin{exemple}\jya smi ɲɯ-ɣɤtʂhɯtʂhɯz\cmn 火噼啪噼啪作响\end{exemple}\end{entrée}

\begin{entrée}
\vedette{\hypertarget{Ⓔɣɤtʂot}{\papi{ ɣɤtʂot}}}\markboth{ɣɤtʂot}{}
\classe{vt}
\paradigme{\textit{dir :} \jya tɤ-}
\paradigme{\textit{dir :} \jya \_}
\begin{définition}\fra rendre clair\end{définition}
\begin{définition}\cmn 弄清楚
\begin{déclaration}\use{这个动词使用范围有限,不能用来表达“说清楚”等意思,只能用\stylefv{sɤmɯtso}表示这种意思}\end{déclaration}\end{définition}
\begin{exemple}\jya ɯʑo kɯ pa-ɣɤtʂot\cmn 他弄清楚了\end{exemple}
\begin{exemple}\jya jiɕqha tɤ-scoz nɯ pɯ-tɯ-rɤt nɯ mɯ́j-tʂot tɕe aʑo pɯ-ɣɤtʂo-t-a\cmn 你字写得不清楚,我把它写清楚了\end{exemple}
\begin{exemple}\jya kɯki tɯrkɤz to-βzu mɯ́j-tʂot tɕe mɤʑɯ kɤ-ɣɤtʂo-t-a\cmn 花纹刻得不清楚,我又把它刻清楚了\end{exemple}
\begin{relation-sémantique}\confer{
\hyperlink{Ⓔtʂot}{\textit{ \papi{tʂot}}}
}\end{relation-sémantique}\end{entrée}

\begin{entrée}
\vedette{\hypertarget{Ⓔɣɤtʂɯtʂɯt}{\papi{ ɣɤtʂɯtʂɯt}}}\markboth{ɣɤtʂɯtʂɯt}{}
\classe{vi}
\paradigme{\textit{dir :} \jya tɤ-}
\begin{définition}\ 
\begin{déclaration}\grammar{deidph}\end{déclaration}\end{définition}
\begin{définition}\fra radoter sans arrêt\end{définition}
\begin{définition}\cmn 不停地唠叨\end{définition}
\begin{exemple}\jya ma-tɯ-ɣɤtʂɯtʂɯt ma ɲɯ-sɤɣdɯɣ\cmn 你不要不停地唠叨,很讨厌\end{exemple}
\begin{relation-sémantique}\synonyme{
\hyperlink{Ⓔɣɤrɯβrɯβ}{\textit{ \papi{ɣɤrɯβrɯβ}}}
}\end{relation-sémantique}\end{entrée}

\begin{entrée}
\vedette{\hypertarget{Ⓔɣɤtɯɣ}{\papi{ ɣɤtɯɣ}}}\markboth{ɣɤtɯɣ}{}
\classe{vt}\acception{1}
\paradigme{\textit{dir :} \jya thɯ-}
\paradigme{\textit{dir :} \jya lɤ-}
\begin{définition}\fra maintenir fermée en appuyant avec un bâton (porte)\end{définition}
\begin{définition}\cmn (用棍子)把门顶住\end{définition}
\begin{exemple}\jya kɯm thɯ-ɣɤtɯɣ-a (lɤ-ɣɤtɯɣ-a)\cmn 我拴了门\end{exemple}\acception{2}
\paradigme{\textit{dir :} \jya lɤ-}
\begin{définition}\fra soutenir en appuyant avec un bâton\end{définition}
\begin{définition}\cmn 用棍子顶住\end{définition}
\begin{exemple}\jya jiɕqha nɯnɯ lɤ-ɣɤtɯɣ tɕe a-mɤ-thɯ-ndʐaβ\cmn 你把它顶住,不要让它跌倒\end{exemple}
\begin{exemple}\jya jiɕqha laχtɕha nɯ ndʐaβ ɲɯ-ŋu tɕe, aj lɤ-ɣɤtɯɣ-a\cmn 这个东西差一点倒了,我用棍子把它顶住了\end{exemple}\begin{sous-entrée}
\vedette{\hypertarget{}{\papi{ aɣɤtɯɣ}}}\markboth{aɣɤtɯɣ}{}\classe{vi}
\begin{définition}\ 
\begin{déclaration}\grammar{pass}\end{déclaration}\end{définition}
\begin{exemple}\jya kɯm pjɤ-k-ɤɣɤtɯɣ-ci ɕti tɕe, kɤ-cɯ mɯ-pjɤ-khɯ\cmn 门是(被棍子)顶着的,打不开\end{exemple}
\end{sous-entrée}\end{entrée}

\begin{entrée}
\vedette{\hypertarget{Ⓔɣɤtɯt}{\papi{ ɣɤtɯt}}}\markboth{ɣɤtɯt}{}
\begin{relation-sémantique}\confer{
\hyperlink{Ⓔtɯt}{\textit{ \papi{tɯt}}}
}\end{relation-sémantique}\end{entrée}

\begin{entrée}
\vedette{\hypertarget{Ⓔɣɤwu}{\papi{ ɣɤwu}}}\markboth{ɣɤwu}{}
\classe{vi}
\paradigme{\textit{dir :} \jya nɯ-}\acception{1}
\begin{définition}\fra pleurer\end{définition}
\begin{définition}\cmn 哭\end{définition}
\begin{exemple}\jya aʑo nɯ-ɣɤwu-a\cmn 我哭了\end{exemple}
\begin{exemple}\jya ɯʑo nɯ-ɣɤwu\cmn 他哭了\end{exemple}
\begin{exemple}\jya jiɕqha tɤ-pɤtso nɯ ɲɯ-ɣɤwu\cmn 这个小孩子在哭\end{exemple}
\begin{exemple}\jya jiɕqha nɯ ɲɯ-nɯzdɯɣ-a tɕe, nɯ-ɣɤwu-a\cmn 我为他担心所以哭了\end{exemple}\acception{2}
\begin{définition}\fra crier (chat, bœuf, cochon, mouton, loup)\end{définition}
\begin{définition}\cmn 叫(猫、牛、猪、羊、狼)\end{définition}
\begin{exemple}\jya qachɣa ɲɯ-ɣɤwu\cmn 狐狸叫\end{exemple}
\begin{exemple}\jya ɕkɤrɯ ɲɯ-ɣɤwu\cmn 鬣羚叫\end{exemple}
\begin{exemple}\jya xsar ɲɯ-ɣɤwu\cmn 青羊叫\end{exemple}
\begin{exemple}\jya qaʑo ra ɲɯ-mtsɯr-nɯ tɕe ɲɯ-ɣɤwu-nɯ\cmn 绵羊饿了就叫\end{exemple}\begin{sous-entrée}
\vedette{\hypertarget{}{\papi{ ɣɤɣɤwu}}}\markboth{ɣɤɣɤwu}{}\classe{vs}
\begin{définition}\fra qui pleure tout le temps\end{définition}
\begin{définition}\cmn 爱哭;容易哭\end{définition}
\begin{exemple}\jya nɤ-tɯ-ɣɤɣɤwu nɯ!\cmn 你真爱哭呢!\end{exemple}
\begin{relation-sémantique}\confer{
\hyperlink{Ⓔnɤwu}{\textit{ \papi{nɤwu}}}
}\end{relation-sémantique}
\begin{relation-sémantique}\confer{
 \papi{tɤ-wu}
}\end{relation-sémantique}
\end{sous-entrée}\begin{sous-entrée}
\vedette{\hypertarget{}{\papi{ zɣɤwu}}}\markboth{zɣɤwu}{}\classe{vt}
\paradigme{\textit{dir :} \jya nɯ-}
\begin{définition}\fra faire pleurer\end{définition}
\begin{définition}\cmn 让人哭\end{définition}
\begin{exemple}\jya ɯʑo nɯ-zɣɤwu-t-a\cmn 我让他哭了(我把他整哭了)\end{exemple}
\end{sous-entrée}\end{entrée}

\begin{entrée}
\vedette{\hypertarget{Ⓔɣɤwɤt}{\papi{ ɣɤwɤt}}}\markboth{ɣɤwɤt}{}
\classe{vs}
\paradigme{\textit{dir :} \jya nɯ-}
\begin{définition}\ 
\begin{déclaration}\grammar{denom}\end{déclaration}\end{définition}
\begin{définition}\fra s'ouvrir (fleur)\end{définition}
\begin{définition}\cmn 开花\end{définition}
\begin{exemple}\jya khɯjŋga ɲɤ-ɣɤwɤt\cmn 杜鹃花开花了\end{exemple}\end{entrée}

\begin{entrée}
\vedette{\hypertarget{Ⓔɣɤwxti}{\papi{ ɣɤwxti}}}\markboth{ɣɤwxti}{}
\begin{relation-sémantique}\confer{
\hyperlink{Ⓔwxti}{\textit{ \papi{wxti}}}
}\end{relation-sémantique}\end{entrée}

\begin{entrée}
\vedette{\hypertarget{Ⓔɣɤxpra}{\papi{ ɣɤxpra}}}\markboth{ɣɤxpra}{}\classe{vt}
\paradigme{\textit{dir :} \jya tɤ-}
\begin{définition}\fra ordonner\end{définition}
\begin{définition}\cmn 指使
\begin{déclaration}\use{用于比自己小、地位低的人}\end{déclaration}\end{définition}
\begin{exemple}\jya tɤ-ɣɤxprat-a\cmn 我使唤了他\end{exemple}
\begin{exemple}\jya tɤ́-wɣ-ɣɤxpra\cmn 他使唤了我\end{exemple}
\begin{exemple}\jya tɤ-pɤtso tɤ-ɣɤxpra-t-a\cmn 我使唤了小孩子\end{exemple}
\begin{exemple}\jya tɤ-ɣɤxpra-t-a tɕe tɕe tɕɤndi laχtɕha ɯ-kɯ-ru nɯ-sɤɣri-t-a\cmn 我使唤他去那边拿东西\end{exemple}
\begin{relation-sémantique}\confer{
\hyperlink{Ⓔtɤpra}{\textit{ \papi{tɤpra}}}
}\end{relation-sémantique}\begin{sous-entrée}
\vedette{\hypertarget{}{\papi{ sɤɣɤxpra}}}\markboth{sɤɣɤxpra}{}\classe{vi}
\begin{définition}\ 
\begin{déclaration}\grammar{apass}\end{déclaration}\end{définition}
\end{sous-entrée}\end{entrée}

\begin{entrée}
\vedette{\hypertarget{Ⓔɣɤxtɕɤβ}{\papi{ ɣɤxtɕɤβ}}}\markboth{ɣɤxtɕɤβ}{}
\classe{vi}
\paradigme{\textit{dir :} \jya pɯ-}
\begin{définition}\fra couper l'herbe\end{définition}
\begin{définition}\cmn 割饲草(左手抓住草,右手拿镰刀割)\end{définition}
\begin{exemple}\jya pɯ-ɣɤxtɕɤβ\cmn 他割了草\end{exemple}
\begin{exemple}\jya sɯjno kutɕu ɲɯ-dɤn tɕe ɕ-pɯ-ɣɤxtɕaβ-a\cmn 这里草很多,我去割了草\end{exemple}
\begin{exemple}\jya rirɤβ kɤ-ɣɤxtɕɤβ mɤ-sɤcha\cmn 不可能一手就把泰山割掉\end{exemple}\begin{sous-entrée}
\vedette{\hypertarget{}{\papi{ nɤxtɕɤβ}}}\markboth{nɤxtɕɤβ}{}\classe{vt}
\paradigme{\textit{dir :} \jya pɯ-}
\begin{définition}\fra couper l'herbe\end{définition}
\begin{définition}\cmn 割饲草\end{définition}
\begin{relation-sémantique}\confer{
\hyperlink{Ⓔtɤxtɕɤβ}{\textit{ \papi{tɤxtɕɤβ}}}
}\end{relation-sémantique}
\end{sous-entrée}\end{entrée}

\begin{entrée}
\vedette{\hypertarget{Ⓔɣɤxtɕhɯxtɕhɯβ}{\papi{ ɣɤxtɕhɯxtɕhɯβ}}}\markboth{ɣɤxtɕhɯxtɕhɯβ}{}
\begin{relation-sémantique}\confer{
\hyperlink{Ⓔsɤxtɕhɯxtɕhɯβ}{\textit{ \papi{sɤxtɕhɯxtɕhɯβ}}}
}\end{relation-sémantique}\end{entrée}

\begin{entrée}
\vedette{\hypertarget{Ⓔɣɤxtɕi}{\papi{ ɣɤxtɕi}}}\markboth{ɣɤxtɕi}{}
\begin{relation-sémantique}\confer{
\hyperlink{Ⓔxtɕi}{\textit{ \papi{xtɕi}}}
}\end{relation-sémantique}\end{entrée}

\begin{entrée}
\vedette{\hypertarget{Ⓔɣɤxtshɯm}{\papi{ ɣɤxtshɯm}}}\markboth{ɣɤxtshɯm}{}
\begin{relation-sémantique}\confer{
\hyperlink{Ⓔxtshɯm}{\textit{ \papi{xtshɯm}}}
}\end{relation-sémantique}\end{entrée}

\begin{entrée}
\vedette{\hypertarget{Ⓔɣɤxtɯt}{\papi{ ɣɤxtɯt}}}\markboth{ɣɤxtɯt}{}
\begin{relation-sémantique}\confer{
\hyperlink{ⒺxtɯtⒽ1}{\textit{ \papi{xtɯt}}}
}\end{relation-sémantique}\end{entrée}

\begin{entrée}
\vedette{\hypertarget{Ⓔɣɤxɯβxɯβ}{\papi{ ɣɤxɯβxɯβ}}}\markboth{ɣɤxɯβxɯβ}{}
\begin{relation-sémantique}\confer{
\hyperlink{Ⓔxɯβxɯβ}{\textit{ \papi{xɯβxɯβ}}}
}\end{relation-sémantique}
\end{entrée}

\begin{entrée}
\vedette{\hypertarget{Ⓔɣɤxɯrxɯr}{\papi{ ɣɤxɯrxɯr}}}\markboth{ɣɤxɯrxɯr}{}
\begin{relation-sémantique}\confer{
\hyperlink{Ⓔxɯrxɯr}{\textit{ \papi{xɯrxɯr}}}
}\end{relation-sémantique}\end{entrée}

\begin{entrée}
\vedette{\hypertarget{Ⓔɣɤxɯxɯɣ}{\papi{ ɣɤxɯxɯɣ}}}\markboth{ɣɤxɯxɯɣ}{}
\classe{vi}
\begin{définition}\fra faire la sourde oreille\end{définition}
\begin{définition}\cmn 当作耳边风
\begin{déclaration}\use{不接受教育、听不进去}\end{déclaration}\end{définition}
\begin{exemple}\jya ɯ-rna ɯ-ɣmbaj ɲɯ-ɣɤxɯxɯɣ\cmn 他当作耳风\end{exemple}
\begin{relation-sémantique}\confer{
\hyperlink{Ⓔsɤxɯxɯɣ}{\textit{ \papi{sɤxɯxɯɣ}}}
}\end{relation-sémantique}\end{entrée}

\begin{entrée}
\vedette{\hypertarget{Ⓔɣɤχalala}{\papi{ ɣɤχalala}}}\markboth{ɣɤχalala}{}
\classe{vs}
\paradigme{\textit{dir :} \jya tɤ-}
\begin{définition}\fra extraverti\end{définition}
\begin{définition}\cmn 外向\end{définition}
\begin{exemple}\jya jiɕqha tɯrme ɲɯ-ɣɤχalala, kɯ-rɯɕmi ci ɲɯ-ŋu\cmn 那个人很外向,爱说话\end{exemple}
\begin{relation-sémantique}\synonyme{
\hyperlink{Ⓔɣɤŋoʁle}{\textit{ \papi{ɣɤŋoʁle}}}
}\end{relation-sémantique}\end{entrée}

\begin{entrée}
\vedette{\hypertarget{Ⓔɣɤχɤlχɤl}{\papi{ ɣɤχɤlχɤl}}}\markboth{ɣɤχɤlχɤl}{}\classe{vi}
\paradigme{\textit{dir :} \jya tɤ-}
\begin{définition}\fra amical\end{définition}
\begin{définition}\cmn 热情\end{définition}
\begin{exemple}\jya azo jɤ-ɣe-a tɕe, ɯʑo a-ɕki ɲɯ-ɣɤχɤlχɤl ʑo\cmn 我来的时候,他对我很热情\end{exemple}
\begin{relation-sémantique}\confer{
\hyperlink{Ⓔχɤlχɤl}{\textit{ \papi{χɤlχɤl}}}
}\end{relation-sémantique}\end{entrée}

\begin{entrée}
\vedette{\hypertarget{Ⓔɣɤχsrɯ}{\papi{ ɣɤχsrɯ}}}\markboth{ɣɤχsrɯ}{}
\classe{vs}
\begin{définition}\fra beau\end{définition}
\begin{définition}\cmn 好看;英俊\end{définition}
\begin{exemple}\jya kɯ-ɣɤχsrɯ ci tɤ-χtɯ-t-a\cmn 我买了好看的(东西)\end{exemple}
\begin{exemple}\jya tɯrme ɲɯ-ɣɤχsrɯ\cmn 人很英俊\end{exemple}
\begin{exemple}\jya jla ɲɯ-ɣɤχsrɯ\cmn 犏牛很好看\end{exemple}\end{entrée}

\begin{entrée}
\vedette{\hypertarget{ⒺɣɤzbaʁⒽ1Ⓗ1}{\papi{ ɣɤzbaʁ}}}\markboth{ɣɤzbaʁ}{}\homonyme{1}
\classe{vt}
\paradigme{\textit{dir :} \jya tɤ-}
\begin{définition}\ 
\begin{déclaration}\grammar{caus}\end{déclaration}\end{définition}
\begin{définition}\fra sécher\end{définition}
\begin{définition}\cmn 弄干\end{définition}
\begin{exemple}\jya tɯ-ŋga thɯ-ɕkho-t-a tɕe, tɤ-ɣɤzbaʁ-a\cmn 我把衣服晒干了\end{exemple}
\begin{sous-entrée}
\vedette{\hypertarget{}{\papi{ ɣɤzbaʁ}}}\markboth{ɣɤzbaʁ}{}\classe{vs}
\begin{définition}\fra se sécher facilement\end{définition}
\begin{définition}\cmn 容易干\end{définition}
\begin{exemple}\jya qale a-pɯ-tu tɕe, ɲɯ-ɣɤzbaʁ\cmn 有风的话就容易干\end{exemple}
\begin{relation-sémantique}\confer{
\hyperlink{Ⓔzbaʁ}{\textit{ \papi{zbaʁ}}}
}\end{relation-sémantique}
\end{sous-entrée}\end{entrée}

\begin{entrée}
\vedette{\hypertarget{Ⓔɣɤzda}{\papi{ ɣɤzda}}}\markboth{ɣɤzda}{}
\classe{vt}
\paradigme{\textit{dir :} \jya \_}
\begin{définition}\fra saluer (sur le chemin)\end{définition}
\begin{définition}\cmn 打招呼(在路上碰见的时候)\end{définition}
\begin{exemple}\jya ɕ-kɤ-ta-ɣɤzda ri kɤ-mtshɤm mataŋe\cmn 我叫了你一声,但是你没有听到\end{exemple}
\begin{exemple}\jya ɕ-kɤ-ta-ɣɤzda tɕe nɯ-tɯ-ɣe\cmn 我叫了你,你就来了\end{exemple}
\begin{relation-sémantique}\synonyme{
\hyperlink{Ⓔɣɤŋoʁ}{\textit{ \papi{ɣɤŋoʁ}}}
}\end{relation-sémantique}\begin{sous-entrée}
\vedette{\hypertarget{}{\papi{ aɣɤzdɯzda}}}\markboth{aɣɤzdɯzda}{}\classe{vi}
\begin{définition}\ 
\begin{déclaration}\grammar{recip}\end{déclaration}\end{définition}
\begin{définition}\fra se saluer les uns les autres\end{définition}
\begin{définition}\cmn 互相打招呼\end{définition}
\begin{relation-sémantique}\confer{
\hyperlink{Ⓔtɯ-zda}{\textit{ \papi{tɯ-zda}}}
}\end{relation-sémantique}
\begin{relation-sémantique}\confer{
\hyperlink{Ⓔrɤzda}{\textit{ \papi{rɤzda}}}
}\end{relation-sémantique}
\begin{relation-sémantique}\confer{
\hyperlink{Ⓔsɤzda}{\textit{ \papi{sɤzda}}}
}\end{relation-sémantique}
\begin{relation-sémantique}\confer{
\hyperlink{Ⓔnɤzda}{\textit{ \papi{nɤzda}}}
}\end{relation-sémantique}
\end{sous-entrée}\end{entrée}

\begin{entrée}
\vedette{\hypertarget{Ⓔɣɤzdoʁloʁ}{\papi{ ɣɤzdoʁloʁ}}}\markboth{ɣɤzdoʁloʁ}{}\classe{vi}
\paradigme{\textit{dir :} \jya tɤ-}
\begin{définition}\fra être tout petit et vif\end{définition}
\begin{définition}\cmn 做出一副小巧玲珑的样子\end{définition}
\begin{exemple}\jya ma-tɯ-ɣɤzdoʁloʁ\cmn 你不要一副小巧玲珑的样子\end{exemple}
\begin{relation-sémantique}\confer{
\hyperlink{Ⓔzdoʁzdoʁ}{\textit{ \papi{zdoʁzdoʁ}}}
}\end{relation-sémantique}\end{entrée}

\begin{entrée}
\vedette{\hypertarget{Ⓔɣɤzdɯzdɯr}{\papi{ ɣɤzdɯzdɯr}}}\markboth{ɣɤzdɯzdɯr}{} (\variante{ɣɤzdɯrzdɯr}) 
\classe{vi}
\begin{définition}\fra sautiller, rebondir (petits objets ronds)\end{définition}
\begin{définition}\cmn 弹来弹去(豌豆、珠子等)\end{définition}
\begin{exemple}\jya staχpɯ ɲɯ-ɣɤzdɯzdɯr\cmn 豌豆在弹来弹去\end{exemple}
\begin{relation-sémantique}\confer{
\hyperlink{Ⓔzdɯzdɯr}{\textit{ \papi{zdɯzdɯr}}}
}\end{relation-sémantique}
\end{entrée}

\begin{entrée}
\vedette{\hypertarget{Ⓔɣɤzgrɤɣlɤɣ}{\papi{ ɣɤzgrɤɣlɤɣ}}}\markboth{ɣɤzgrɤɣlɤɣ}{}
\classe{vs}
\paradigme{\textit{dir :} \jya tɤ-}
\begin{définition}\ 
\begin{déclaration}\grammar{deidph}\end{déclaration}\end{définition}
\begin{définition}\fra bruit de saut incessant\end{définition}
\begin{définition}\cmn 不停地跳动的声音\end{définition}
\begin{exemple}\jya ɲɯ-ɣɤzgrɤɣlɤɣ\cmn 他在跳动,发出很响的声音\end{exemple}
\begin{exemple}\jya pɯ-ɣɤzgrɤɣlɤɣ-nɯ pɯ-rɟaʁ-nɯ\cmn 他们跳舞,不停地跳动,发出很响的声音\end{exemple}\end{entrée}

\begin{entrée}
\vedette{\hypertarget{Ⓔɣɤzɣɤrlɤr}{\papi{ ɣɤzɣɤrlɤr}}}\markboth{ɣɤzɣɤrlɤr}{}\classe{vi}
\paradigme{\textit{dir :} \jya tɤ-}
\begin{définition}\fra avoir la tête qui tourne\end{définition}
\begin{définition}\cmn 头晕,脚都站不稳的感觉\end{définition}
\begin{exemple}\jya ɲɯ-ɣɤzɣɤrlar-a ʑo\cmn 我脚都站不稳\end{exemple}\begin{sous-entrée}
\vedette{\hypertarget{}{\papi{ sɤzɣɤrlɤr}}}\markboth{sɤzɣɤrlɤr}{}
\paradigme{\textit{dir :} \jya tɤ-}
\begin{définition}\fra qui fait tourner la tête\end{définition}
\begin{définition}\cmn 摇晃,令……头晕\end{définition}\classe{vt}
\end{sous-entrée}\end{entrée}

\begin{entrée}
\vedette{\hypertarget{Ⓔɣɤzɣɯt}{\papi{ ɣɤzɣɯt}}}\markboth{ɣɤzɣɯt}{}
\begin{relation-sémantique}\confer{
\hyperlink{Ⓔzɣɯt}{\textit{ \papi{zɣɯt}}}
}\end{relation-sémantique}\end{entrée}

\begin{entrée}
\vedette{\hypertarget{Ⓔɣɤzjaŋlaŋ}{\papi{ ɣɤzjaŋlaŋ}}}\markboth{ɣɤzjaŋlaŋ}{}
\classe{vi}
\begin{définition}\fra se balancer\end{définition}
\begin{définition}\cmn 摇晃\end{définition}
\begin{exemple}\jya tʂu mɯ́j-pe tɕe, @qiche ɯ-ŋgɯ ku-kɯ-ɤmdzɯ tɕe tu-ɣɤzjaŋlaŋ ɲɯ-ŋu\cmn 路不好,所以坐车的时候摇摇晃晃\end{exemple}\begin{sous-entrée}
\vedette{\hypertarget{}{\papi{ ɣɤzjaŋzjaŋ}}}\markboth{ɣɤzjaŋzjaŋ}{}\classe{vi}
\end{sous-entrée}\begin{sous-entrée}
\vedette{\hypertarget{}{\papi{ nɯzjaŋ}}}\markboth{nɯzjaŋ}{}\classe{vt}
\end{sous-entrée}\begin{sous-entrée}
\vedette{\hypertarget{}{\papi{ sɤzjaŋlaŋ}}}\markboth{sɤzjaŋlaŋ}{}\classe{vt}
\begin{définition}\fra balancer\end{définition}
\begin{définition}\cmn 摇动\end{définition}
\begin{exemple}\jya laʁjɯɣ ɲɯ-sɤzjaŋlaŋ\cmn 他乱动棍子\end{exemple}
\end{sous-entrée}\begin{sous-entrée}
\vedette{\hypertarget{}{\papi{ sɤzjaŋzjaŋ}}}\markboth{sɤzjaŋzjaŋ}{}\classe{vt}
\begin{exemple}\jya mbro ɲɯ-sɤzjaŋzjaŋ ʑo ɲɯ-ɤz-nɯmbrɤpɯ\cmn 他骑着马显得很高\end{exemple}
\end{sous-entrée}\end{entrée}

\begin{entrée}
\vedette{\hypertarget{Ⓔɣɤzjaŋzjaŋ}{\papi{ ɣɤzjaŋzjaŋ}}}\markboth{ɣɤzjaŋzjaŋ}{}
\begin{relation-sémantique}\confer{
\hyperlink{Ⓔɣɤzjaŋlaŋ}{\textit{ \papi{ɣɤzjaŋlaŋ}}}
}\end{relation-sémantique}\end{entrée}

\begin{entrée}
\vedette{\hypertarget{Ⓔɣɤzjɤɣlɤɣ}{\papi{ ɣɤzjɤɣlɤɣ}}}\markboth{ɣɤzjɤɣlɤɣ}{}
\begin{relation-sémantique}\confer{
\hyperlink{Ⓔzjɤɣzjɤɣ}{\textit{ \papi{zjɤɣzjɤɣ}}}
}\end{relation-sémantique}\end{entrée}

\begin{entrée}
\vedette{\hypertarget{Ⓔɣɤzoŋzoŋ}{\papi{ ɣɤzoŋzoŋ}}}\markboth{ɣɤzoŋzoŋ}{}
\classe{vs}
\paradigme{\textit{dir :} \jya thɯ-}
\begin{définition}\ 
\begin{déclaration}\grammar{deidph}\end{déclaration}\end{définition}
\begin{définition}\fra sensation d'engourdissement\end{définition}
\begin{définition}\cmn 觉得麻木\end{définition}
\begin{exemple}\jya nɤ-mi cho-ɣɤzoŋzoŋ\cmn 你的脚麻了\end{exemple}
\begin{exemple}\jya nɤ-mi to-ndʑɯrpɯt tɕe, ɲɯ-ɣɤzoŋzoŋ loβtɕi\cmn 你的脚麻了,是吧\end{exemple}\begin{sous-entrée}
\vedette{\hypertarget{}{\papi{ sɤzoŋzoŋ}}}\markboth{sɤzoŋzoŋ}{}\classe{vt}
\paradigme{\textit{dir :} \jya tɤ-}
\begin{définition}\fra engourdir\end{définition}
\begin{définition}\cmn 令……麻木\end{définition}
\end{sous-entrée}\end{entrée}

\begin{entrée}
\vedette{\hypertarget{ⒺɣɤzriⒽ1}{\papi{ ɣɤzri}}}\markboth{ɣɤzri}{}\homonyme{1}\classe{vt}
\paradigme{\textit{dir :} \jya nɯ-}
\begin{définition}\ 
\begin{déclaration}\grammar{caus}\end{déclaration}\end{définition}
\begin{définition}\fra allonger\end{définition}
\begin{définition}\cmn 弄长\end{définition}
\begin{exemple}\jya tɯ-ŋga nɯ-qrɯ-t-a tɕe nɯ-ɣɤzri-t-a\cmn 我把衣服剪得太长了\end{exemple}
\begin{relation-sémantique}\confer{
\hyperlink{Ⓔzri}{\textit{ \papi{zri}}}
}\end{relation-sémantique}\end{entrée}

\begin{entrée}
\vedette{\hypertarget{ⒺɣɤzriⒽ2}{\papi{ ɣɤzri}}}\markboth{ɣɤzri}{}\homonyme{2}
\classe{vs}
\begin{définition}\fra qui s'allonge vite\end{définition}
\begin{définition}\cmn 长得快;容易变长\end{définition}\begin{sous-entrée}
\vedette{\hypertarget{}{\papi{ nɤɣɤzri}}}\markboth{nɤɣɤzri}{}\classe{vt}
\begin{définition}\fra trouver que ...s'allonge vite\end{définition}
\begin{définition}\cmn 觉得……长得快\end{définition}
\begin{exemple}\jya a-ndzrɯ ɲɯ-nɤɣɤzri-a tɕe tshɯrɟɯn ɲɯ-phɯt-a ŋu\cmn 我觉得指甲长得很快,要经常剪\end{exemple}
\begin{relation-sémantique}\confer{
\hyperlink{Ⓔzri}{\textit{ \papi{zri}}}
}\end{relation-sémantique}
\end{sous-entrée}\end{entrée}

\begin{entrée}
\vedette{\hypertarget{Ⓔɣɤzɯβzɯβ}{\papi{ ɣɤzɯβzɯβ}}}\markboth{ɣɤzɯβzɯβ}{}
\classe{vi}
\paradigme{\textit{dir :} \jya nɯ-}
\paradigme{\textit{dir :} \jya tɤ-}
\begin{définition}\fra astringent\end{définition}
\begin{définition}\cmn 涩;麻\end{définition}\begin{sous-entrée}
\vedette{\hypertarget{}{\papi{ sɤzɯβzɯβ}}}\markboth{sɤzɯβzɯβ}{}
\begin{exemple}\jya tɕɣom tɤ-ndza-t-a tɕe, a-mtɕhi sɤzɯβzɯβ\cmn 我吃了花椒,嘴里发麻\end{exemple}
\end{sous-entrée}\end{entrée}

\begin{entrée}
\vedette{\hypertarget{Ⓔɣɤzɯrzɯr}{\papi{ ɣɤzɯrzɯr}}}\markboth{ɣɤzɯrzɯr}{}\classe{vi}
\begin{définition}\fra ressentir une démangeaison\end{définition}
\begin{définition}\cmn 感觉到很痒\end{définition}
\begin{exemple}\jya a-mtɕhi ɲɯ-ɣɤzɯzɯr\cmn 我的嘴很痒\end{exemple}\end{entrée}

\begin{entrée}
\vedette{\hypertarget{Ⓔɣɤʑu}{\papi{ ɣɤʑu}}}\markboth{ɣɤʑu}{}
\classe{vi}
\begin{définition}\fra y avoir, exister (sensoriel)\end{définition}
\begin{définition}\cmn 有(亲见)\end{définition}
\begin{exemple}\jya nɯ ma mɤ-kɯ-pe ɯ-ɣɤ́ʑu?\cmn 还有没有错误?\end{exemple}
\begin{relation-sémantique}\confer{
\hyperlink{Ⓔmaŋe}{\textit{ \papi{maŋe}}}
}\end{relation-sémantique}\begin{forme-mot}2s : \papi{ɣɤtɤʑu}\end{forme-mot}\end{entrée}

\begin{entrée}
\vedette{\hypertarget{Ⓔɣɤʑɯn}{\papi{ ɣɤʑɯn}}}\markboth{ɣɤʑɯn}{}
\classe{vs}
\paradigme{\textit{dir :} \jya tɤ-}
\begin{définition}\fra pentu\end{définition}
\begin{définition}\cmn 陡峭\end{définition}
\begin{exemple}\jya jiɕqha sɤtɕha ɣɤʑɯn\cmn 那个地方很陡\end{exemple}
\begin{exemple}\jya praʁ ɲɯ-ɣɤʑɯn\cmn 悬崖很陡\end{exemple}
\begin{relation-sémantique}\confer{
\hyperlink{Ⓔnɤʑɯn}{\textit{ \papi{nɤʑɯn}}}
}\end{relation-sémantique}
\begin{relation-sémantique}\synonyme{
\hyperlink{Ⓔɣɤrɤβ}{\textit{ \papi{ɣɤrɤβ}}}
}\end{relation-sémantique}\end{entrée}

\begin{entrée}
\vedette{\hypertarget{Ⓔɣdɤɣdɤt}{\papi{ ɣdɤɣdɤt}}}\markboth{ɣdɤɣdɤt}{}\classe{idph.2}
\begin{définition}\fra court et grassouillet\end{définition}
\begin{définition}\cmn 形容胖而短,看起来很可爱的样子\end{définition}
\begin{exemple}\jya tɤ-pɤtso ɲɯ-tshu, ɯ-mɤlɤjaʁ ra ɣdɤɣdɤt ʑo ɲɯ-pa\cmn 小孩子胖胖的,手脚又粗又短\end{exemple}\end{entrée}

\begin{entrée}
\vedette{\hypertarget{Ⓔɣdɤso}{\papi{ ɣdɤso}}}\markboth{ɣdɤso}{}
\classe{n}
\begin{définition}\fra ver marron\end{définition}
\begin{définition}\cmn 虫的一种\end{définition}\end{entrée}

\begin{entrée}
\vedette{\hypertarget{Ⓔɣdoŋnɤɣdoŋ}{\papi{ ɣdoŋnɤɣdoŋ}}}\markboth{ɣdoŋnɤɣdoŋ}{}\classe{idph.2}
\begin{définition}\fra battement de tambour\end{définition}
\begin{définition}\cmn 形容很响的敲鼓声\end{définition}
\begin{relation-sémantique}\confer{
 \papi{dɯrnɤdɯr}
}\end{relation-sémantique}
\begin{relation-sémantique}\confer{
\hyperlink{Ⓔɣdɯɣnɤɣdɯɣ}{\textit{ \papi{ɣdɯɣnɤɣdɯɣ}}}
}\end{relation-sémantique}\begin{sous-entrée}
\vedette{\hypertarget{}{\papi{ sɤɣdoŋɣdoŋ}}}\markboth{sɤɣdoŋɣdoŋ}{}\classe{vt}
\begin{exemple}\jya tɤrmbɣo ɲɯ-sɤɣdoŋɣdoŋ ʑo\cmn 打鼓打得很响\end{exemple}
\begin{relation-sémantique}\synonyme{
\hyperlink{Ⓔsɤndɤrndɤr}{\textit{ \papi{sɤndɤrndɤr}}}
}\end{relation-sémantique}
\end{sous-entrée}\end{entrée}

\begin{entrée}
\vedette{\hypertarget{Ⓔɣdɯ}{\papi{ ɣdɯ}}}\markboth{ɣdɯ}{}\classe{n}
\begin{définition}\fra jarre de vin\end{définition}
\begin{définition}\cmn 酒坛子\end{définition}
\begin{relation-sémantique}\synonyme{
\hyperlink{Ⓔtɕhɤɣdɯ}{\textit{ \papi{tɕhɤɣdɯ}}}
}\end{relation-sémantique}
\begin{relation-sémantique}\synonyme{
\hyperlink{Ⓔtɕhorzi}{\textit{ \papi{tɕhorzi}}}
}\end{relation-sémantique}\end{entrée}

\begin{entrée}
\vedette{\hypertarget{Ⓔɣdɯβɣdɯβ}{\papi{ ɣdɯβɣdɯβ}}}\markboth{ɣdɯβɣdɯβ}{}
\classe{idph.2}
\begin{définition}\fra court et épais\end{définition}
\begin{définition}\cmn 形容又粗又短的样子\end{définition}
\begin{exemple}\jya ɯ-rtshɯm ci ɣdɯβɣdɯβ ɣɤʑu\cmn 树墩又粗又短\end{exemple}\end{entrée}

\begin{entrée}
\vedette{\hypertarget{Ⓔɣdɯɣnɤɣdɯɣ}{\papi{ ɣdɯɣnɤɣdɯɣ}}}\markboth{ɣdɯɣnɤɣdɯɣ}{}\classe{idph.2}
\begin{définition}\fra battement de tambour\end{définition}
\begin{définition}\cmn 形容轻轻的敲鼓声\end{définition}
\begin{relation-sémantique}\confer{
 \papi{dɯrnɤdɯr}
}\end{relation-sémantique}
\begin{relation-sémantique}\confer{
\hyperlink{Ⓔɣdoŋnɤɣdoŋ}{\textit{ \papi{ɣdoŋnɤɣdoŋ}}}
}\end{relation-sémantique}\end{entrée}

\begin{entrée}
\vedette{\hypertarget{Ⓔɣe}{\papi{ ɣe}}}\markboth{ɣe}{}\classe{part}
\begin{définition}\fra n'est ce pas ?\end{définition}
\begin{définition}\cmn 是不是?
\begin{déclaration}\use{陈述自己的感觉,征求别人的看法}\end{déclaration}\end{définition}
\begin{exemple}\jya nɤki tɕheme nɯ ɲɯ-mpɕɤr ɣe\cmn 那个女孩子挺漂亮的,是不是?\end{exemple}
\begin{exemple}\jya jisŋi tɯ-mɯ ɲɯ-jɯm ɣe\cmn 今天天气挺好的,对不对?\end{exemple}\end{entrée}

\begin{entrée}
\vedette{\hypertarget{Ⓔɣi}{\papi{ ɣi}}}\markboth{ɣi}{}
\classe{vi}
\paradigme{\textit{dir :} \jya jɤ-}
\paradigme{\textit{past stem :} \jya ɣe}
\paradigme{\textit{construction :} \jya subj.part}
\begin{définition}\fra venir\end{définition}
\begin{définition}\cmn 来\end{définition}
\begin{exemple}\jya jiɕqha kɤ-ari-tɕi tɕe, li nɯ-ɣe-tɕi\cmn 我们俩去了然后就回来了\end{exemple}
\begin{exemple}\jya pjɯ-tɯ-ɣi mɤ-ra\cmn 你不用下来(送我们)\end{exemple}
\begin{exemple}\jya a-mu ɯ-lɯz thɯ-ɣe\cmn 我母亲年龄大了\end{exemple}
\begin{exemple}\jya ɯ-ftsa ci chɯ-ɕe ɲɯ-ŋu tɕe, ``thɯ-ɣi ma a-zda me" ɲɯ-ti tɕe thɯ-ari\cmn 他的侄子去成都就说“你来吧,没有人陪我”,他就一同去了\end{exemple}\begin{sous-entrée}
\vedette{\hypertarget{}{\papi{ ɣɤɣi}}}\markboth{ɣɤɣi}{}\classe{vs}
\begin{définition}\fra qui vient facilement\end{définition}
\begin{définition}\cmn 容易来
\begin{déclaration}\use{同\stylefv{ɯ-ʑɯβ}“睡眠”连用}\end{déclaration}\end{définition}
\begin{exemple}\jya ɯ-ʑɯβ ɲɯ-ɣɤɣi\cmn 他容易入眠\end{exemple}
\end{sous-entrée}\begin{sous-entrée}
\vedette{\hypertarget{}{\papi{ tɯ-jaʁ,ɣi}}}\markboth{tɯ-jaʁ,ɣi}{}
\begin{définition}\fra obtenir\end{définition}
\begin{définition}\cmn 拿到;抓到;收到\end{définition}
\begin{exemple}\jya ɯ-jaʁ mɯ-jɤ-ɣe\cmn 没有到手\end{exemple}
\begin{exemple}\jya ɯ-jaʁ tɕhi nɯ-kɯ-ɣe ʑo tu-ndze ɲɯ-ɕti\cmn 它能抓到什么就吃什么\end{exemple}
\begin{relation-sémantique}\ComponentA{\classe{np}
\hyperlink{Ⓔtɯ-jaʁ}{\textit{ \papi{tɯ-jaʁ}}}
}\end{relation-sémantique}
\begin{relation-sémantique}\ComponentB{\classe{vi}
\hyperlink{Ⓔɣi}{\textit{ \papi{ɣi}}}
}\end{relation-sémantique}
\begin{relation-sémantique}\confer{
\hyperlink{Ⓔɣɤnɯʑɯβ}{\textit{ \papi{ɣɤnɯʑɯβ}}}
}\end{relation-sémantique}
\begin{relation-sémantique}\confer{
\hyperlink{Ⓔtɯ-ʑɯβ}{\textit{ \papi{tɯ-ʑɯβ}}}
}\end{relation-sémantique}
\begin{relation-sémantique}\confer{
 \papi{ɯ-χsoŋχsɤz,ɣi}
}\end{relation-sémantique}
\begin{relation-sémantique}\confer{
\hyperlink{Ⓔtɯ-ʁjiz,ɣi}{\textit{ \papi{tɯ-ʁjiz,ɣi}}}
}\end{relation-sémantique}
\begin{relation-sémantique}\confer{
 \papi{tɯ-lɯz,ɣi}
}\end{relation-sémantique}
\end{sous-entrée}\end{entrée}

\begin{entrée}
\vedette{\hypertarget{Ⓔɣɟaβ}{\papi{ ɣɟaβ}}}\markboth{ɣɟaβ}{}
\classe{vt}
\paradigme{\textit{dir :} \jya nɯ-}
\paradigme{\textit{dir :} \jya pɯ-}
\begin{définition}\fra battre le lait\end{définition}
\begin{définition}\cmn 搅牛奶;打酥油\end{définition}
\begin{exemple}\jya pɯ-ɣɟaβ-a\cmn 我搅了(牛奶)\end{exemple}
\begin{exemple}\jya ɯʑo kɯ tɤ-lu na-ɣɟaβ\cmn 他搅了牛奶\end{exemple}
\begin{exemple}\jya tɤ-lu pjɯ́-wɣ-ɣɟaβ kóʁmɯz ta-mar tu-nɯɬoʁ ŋu\cmn 搅了牛奶就有酥油出来\end{exemple}
\begin{relation-sémantique}\confer{
\hyperlink{Ⓔtɯɣɟaβ}{\textit{ \papi{tɯɣɟaβ}}}
}\end{relation-sémantique}\end{entrée}

\begin{entrée}
\vedette{\hypertarget{Ⓔɣɟɯ}{\papi{ ɣɟɯ}}}\markboth{ɣɟɯ}{}
\classe{n}
\begin{définition}\fra tour de garde\end{définition}
\begin{définition}\cmn 碉楼\end{définition}\end{entrée}

\begin{entrée}
\vedette{\hypertarget{Ⓔɣɟɯʁar}{\papi{ ɣɟɯʁar}}}\markboth{ɣɟɯʁar}{}
\classe{n}
\begin{définition}\fra monstre\end{définition}
\begin{définition}\cmn 魔鬼;高大的人,全身长着黑毛\end{définition}\end{entrée}

\begin{entrée}
\vedette{\hypertarget{Ⓔɣɟɯthoʁ}{\papi{ ɣɟɯthoʁ}}}\markboth{ɣɟɯthoʁ}{}\classe{n}
\begin{définition}\ 
\begin{déclaration}\grammar{n.lieu}\end{déclaration}\end{définition}
\begin{définition}\fra hameau de Ercha près du fleuve\end{définition}
\begin{définition}\cmn 二茶村的河坝地区\end{définition}
\end{entrée}

\begin{entrée}
\vedette{\hypertarget{Ⓔɣɟɯtshapa}{\papi{ ɣɟɯtshapa}}}\markboth{ɣɟɯtshapa}{}\classe{n}
\begin{définition}\ 
\begin{déclaration}\grammar{n.lieu}\end{déclaration}\end{définition}
\begin{définition}\fra Ercha (village de Gdongbrgyad)\end{définition}
\begin{définition}\cmn 二茶村\end{définition}
\end{entrée}

\begin{entrée}
\vedette{\hypertarget{Ⓔɣle}{\papi{ ɣle}}}\markboth{ɣle}{}
\classe{vt}
\paradigme{\textit{dir :} \jya pɯ-}
\paradigme{\textit{dir :} \jya tɤ-}
\begin{définition}\fra pétrir, frotter\end{définition}
\begin{définition}\cmn 揉\end{définition}
\begin{exemple}\jya pɯ-ɣle-t-a\cmn 我揉了\end{exemple}
\begin{exemple}\jya pɯ-tɯ-ɣle-t\cmn 你揉了\end{exemple}
\begin{exemple}\jya pa-ɣle\cmn 他揉了\end{exemple}
\begin{exemple}\jya tɤjlu wuma ʑo pa-ɣle tɕe ɲɯ-mɯm\cmn 他揉了面,面就好吃了\end{exemple}\begin{sous-entrée}
\vedette{\hypertarget{}{\papi{ nɤɣlɤɣle}}}\markboth{nɤɣlɤɣle}{}\classe{vt}
\begin{définition}\fra frotter dans tous les sens\end{définition}
\begin{définition}\cmn 揉来揉去\end{définition}
\end{sous-entrée}\end{entrée}

\begin{entrée}
\vedette{\hypertarget{Ⓔɣlɯntɯ}{\papi{ ɣlɯntɯ}}}\markboth{ɣlɯntɯ}{}
\classe{n}
\begin{définition}\fra bouse de vache sèche dans la montagne\end{définition}
\begin{définition}\cmn 山上的干牛屎\end{définition}\end{entrée}

\begin{entrée}
\vedette{\hypertarget{Ⓔɣlɯtɕɤt}{\papi{ ɣlɯtɕɤt}}}\markboth{ɣlɯtɕɤt}{}
\classe{n}
\begin{définition}\fra action de retirer le purin de l'étable\end{définition}
\begin{définition}\cmn 出圈粪\end{définition}
\begin{exemple}\jya ɣlɯtɕɤt wuma ʑo ɴqa\cmn 出圈是很辛苦的\end{exemple}
\begin{relation-sémantique}\confer{
\hyperlink{Ⓔtɯ-ɣli}{\textit{ \papi{tɯ-ɣli}}}
}\end{relation-sémantique}
\begin{relation-sémantique}\confer{
\hyperlink{Ⓔtɕɤt}{\textit{ \papi{tɕɤt}}}
}\end{relation-sémantique}
\begin{relation-sémantique}\confer{
\hyperlink{Ⓔɣɯɣlɯtɕɤt}{\textit{ \papi{ɣɯɣlɯtɕɤt}}}
}\end{relation-sémantique}\end{entrée}

\begin{entrée}
\vedette{\hypertarget{Ⓔɣnɤsqi}{\papi{ ɣnɤsqi}}}\markboth{ɣnɤsqi}{}\classe{num}
\begin{définition}\fra vingt\end{définition}
\begin{définition}\cmn 二十\end{définition}\end{entrée}

\begin{entrée}
\vedette{\hypertarget{Ⓔɣnda}{\papi{ ɣnda}}}\markboth{ɣnda}{}
\classe{vt}
\paradigme{\textit{dir :} \jya pɯ-}
\begin{définition}\fra frapper (avec un marteau), marteler, tasser\end{définition}
\begin{définition}\cmn 捶打;夯结实\end{définition}
\begin{exemple}\jya pɯ-ɣnda-t-a\cmn 我捶打了\end{exemple}
\begin{exemple}\jya pɯ-tɯ-ɣnda-t\cmn 你捶打了\end{exemple}
\begin{exemple}\jya pa-ɣnda\cmn 他捶打了\end{exemple}
\begin{exemple}\jya khɤxtu pjɯ́-wɣ-ɣnda tɕe, tɯftsaʁ mɤ-ɣi\cmn 把房背夯结实以后,下雨的时候不会漏水\end{exemple}
\begin{exemple}\jya tɤtshoʁ pɯ-ɣnda-t-a\cmn 我钉了钉子\end{exemple}\end{entrée}

\begin{entrée}
\vedette{\hypertarget{Ⓔɣndʑɤβ}{\papi{ ɣndʑɤβ}}}\markboth{ɣndʑɤβ}{}\classe{n}
\begin{définition}\fra feu, incendie\end{définition}
\begin{définition}\cmn 损坏性的火\end{définition}
\begin{exemple}\jya ɣndʑɤβ tɤ-ta-t-a\cmn 我放了火\end{exemple}
\begin{exemple}\jya ɣndʑɤβ to-lɯɣ\cmn 失火\end{exemple}
\begin{relation-sémantique}\confer{
\hyperlink{Ⓔɣndʑɤβta}{\textit{ \papi{ɣndʑɤβta}}}
}\end{relation-sémantique}
\begin{relation-sémantique}\confer{
\hyperlink{Ⓔndʑɤβ}{\textit{ \papi{ndʑɤβ}}}
}\end{relation-sémantique}\end{entrée}

\begin{entrée}
\vedette{\hypertarget{Ⓔɣndʑɤβta}{\papi{ ɣndʑɤβta}}}\markboth{ɣndʑɤβta}{}
\classe{n}
\begin{définition}\fra feu\end{définition}
\begin{définition}\cmn 火\end{définition}
\begin{exemple}\jya ɣndʑɤβta pɯ-tu\cmn 烧火了\end{exemple}
\begin{relation-sémantique}\confer{
\hyperlink{Ⓔɣɯɣndʑɤβta}{\textit{ \papi{ɣɯɣndʑɤβta}}}
}\end{relation-sémantique}\end{entrée}

\begin{entrée}
\vedette{\hypertarget{Ⓔɣndʑɯr}{\papi{ ɣndʑɯr}}}\markboth{ɣndʑɯr}{}\classe{vt}
\paradigme{\textit{dir :} \jya thɯ-}
\paradigme{\textit{dir :} \jya nɯ-}
\paradigme{\textit{dir :} \jya pɯ-}
\begin{définition}\fra moudre\end{définition}
\begin{définition}\cmn 磨\end{définition}
\begin{exemple}\jya thɯ-ɣndʑɯr-a\cmn 我磨了\end{exemple}
\begin{exemple}\jya ɯʑo kɯ tha-ɣndʑɯr\cmn 他磨了\end{exemple}
\begin{exemple}\jya tɯjpu chɯ́-wɣ-ndʑɯr mɤɕtʂa kɤ-ndza mɤ-sna\cmn 在没磨之前,不能吃粮食\end{exemple}
\begin{exemple}\jya smɤn nɯ-ɣndʑɯr-a\cmn 我磨了药\end{exemple}
\begin{exemple}\jya tɤjlu thɯ-ɣndʑɯr-a\cmn 我磨了面\end{exemple}
\begin{exemple}\jya pɯ-ɣndʑɯr-a\cmn 我磨了(青稞)\end{exemple}
\begin{relation-sémantique}\confer{
\hyperlink{Ⓔsɤrŋɤɣndʑɯr}{\textit{ \papi{sɤrŋɤɣndʑɯr}}}
}\end{relation-sémantique}
\begin{relation-sémantique}\confer{
 \papi{tɯ-ɣndʑɯr}
}\end{relation-sémantique}\end{entrée}

\begin{entrée}
\vedette{\hypertarget{Ⓔɣni}{\papi{ ɣni}}}\markboth{ɣni}{}
\classe{n}
\begin{définition}\fra renard volant\end{définition}
\begin{définition}\cmn 飞鼠\end{définition}
\begin{exemple}\jya ɣni nɯ stɤmku cho sɯŋgɯ ku-rɤʑi ŋu, ɯ-mdoʁ nɯ βʑɯ ɯ-mdoʁ cho naχtɕɯɣ, ɯ-mgɯr ɯ-χcɤl nɯ tɕu kɯ-ɲaʁ tɯ-ʂɯl tu, ɯ-mi ɯ-ru me, ɯ-mɤpɤl ma me, ɯ-ndzrɯ tu, kú-wɣ-rtoʁ tɕe, ɯ-mɤlɤjaʁ stɤsmɤt ɯ-pɤrthɤβ nɯ tɯ-ndʐi kɯ ɲɯ-ɤlɤɣɯ ɕti, tɕe tɯ-ɕe jɤ-ʑa tɕe, ɯ-ndʐi nɯ ɯ-mɤlɤjaʁ kɯ ɲɯ-sqhiar tɕe sɤtɕha ɯ-taʁ pjɤ-sthaβ ʑo tɕe ɲɯ-nɯqambɯmbjom kɯ-fse ɲɯ-ŋu. kɤ-ŋke nɯ mɯ́j-khɯ rca, ɯ-jme kɯ-xtɕɯ-xtɕi ci ɣɤʑu, ɯ-ku nɯ βʑɯ ɯ-ku ɲɯ-fse tɕe ri ɯ-rna kɯ-saχsɤl maŋe, tu-mbri tɕe, tɤ-pɤtso nɯ-kɤ-ɣɤwu ɯ-skɤt kɯ-fse tu-lɤt ɲɯ-ŋu, mɤʑɯ kɯmaʁ kɤntɕhɯ-tɯphu ɯ-skɤt tu-lɤt ɲɯ-ŋgrɤl\cmn 飞鼠生活在草山和森林里,(皮毛的)颜色和老鼠的一样,背上中间有一条黑色的纹路,(好像)没有腿,只有脚板和爪子。看起来前腿和后腿之间的皮子是连在一起的。它开始走动的时候,皮子就用四肢来展开,贴住地面(离地面很近)飞行。它可能不会走。有小尾巴,头有点像老鼠的头,但看不出耳朵。喊叫的时候,发出和小孩子哭一样的声音,还能发出其它好几种动物的叫声。\end{exemple}\end{entrée}

\begin{entrée}
\vedette{\hypertarget{Ⓔɣot}{\papi{ ɣot}}}\markboth{ɣot}{}
\classe{n}
\begin{définition}\fra lumière et chaleur (du soleil)\end{définition}
\begin{définition}\cmn 光和热量(太阳的)
\begin{déclaration} \étymologie{\papi{ɦod}}\end{déclaration}\end{définition}
\begin{exemple}\jya tɤŋe ɯ-ɣot\cmn 太阳光\end{exemple}
\begin{relation-sémantique}\confer{
\hyperlink{Ⓔsmɯɣot}{\textit{ \papi{smɯɣot}}}
}\end{relation-sémantique}\end{entrée}

\begin{entrée}
\vedette{\hypertarget{Ⓔɣrɤmu}{\papi{ ɣrɤmu}}}\markboth{ɣrɤmu}{} (\variante{rɤmu}) 
\classe{n}
\begin{définition}\fra Thlaspi arvense\end{définition}
\begin{définition}\cmn 菥蓂【苦苦菜】\end{définition}
\begin{exemple}\jya ɣrɤmu nɯ sɯjno kɯ-xtɕɯ-xtɕi ci ŋu, ɯ-jwaʁ nɯ tɕɤr, ɯ-ku tɕe lu-ortɯm tsa ŋu ma mɤ-amtɕoʁ, ɯ-jwaʁ mpɯ. ɯ-jwaʁ cho ɯ-ru ra arŋi. ɯ-χcɤl ɯ-ru tu-ɬoʁ tɕe, ɯ-jwaʁ tu-oʑɯrja ŋu. ɯ-jwaʁ tɤ-arɕo tɕe ɯ-mɯntoʁ tu-oʑɯrja tɕe, ɯ-mat chɯ-βze ŋu. ɯ-mat nɯ kɯ-ɤrtɯm tɕe kɯ-ɤɕpɯɕpa ŋu. ɯ-mɯntoʁ wɣrum. ɯ-mat ɯ-ŋgɯ ɯ-rɣi wuma ʑo dɤn, ndɯβ. ɯ-rɣi ɯ-taʁ tɤ-rʑɯɣ kɯ-fse tu, tu-mbro mɤ-cha, ɯ-ru tu-ɬoʁ ɕɯŋgɯ tɕe, ɯ-jwaʁ kɤ-ndza sna, kɯ-xtɕɯ-xtɕi qiaβ. thɯ-tɯt tɕe, ɯ-ru cho ɯ-mat ra lonba ɲɯ-qarŋe, ɯ-jwaʁ ra lonba pjɯ-ŋgra ŋu. ɯ-ru cho ɯ-mat ma ɲɯ-me ŋu.\cmn 苦苦菜是长得很小的草,叶子细,顶端是圆形的,不尖,很嫩。叶子和茎都是绿色的。中间长茎,叶子排列在茎上。叶子长到一定的高度,然后花排列在上面一段,就结果。果实圆而扁,花是白色的,果实里种子很多,很小。种子表面有皱纹。长不高。茎长出来之前,叶子可以吃,有点苦。成熟以后,茎和果实全变黄,叶子落光,只剩下茎和果实了。\end{exemple}\end{entrée}

\begin{entrée}
\vedette{\hypertarget{Ⓔɣro}{\papi{ ɣro}}}\markboth{ɣro}{}
\classe{vi}
\paradigme{\textit{dir :} \jya pɯ-}
\paradigme{\textit{dir :} \jya thɯ-}
\begin{définition}\fra s'étouffer\end{définition}
\begin{définition}\cmn 呛到\end{définition}
\begin{exemple}\jya tɯsqar (tɯ-ɣndʑɤr) tɤ-moʁ-a tɕe pɯ-ɣro-a\cmn 我吃糌粑的时候呛到了\end{exemple}\begin{sous-entrée}
\vedette{\hypertarget{}{\papi{ sɯɣro}}}\markboth{sɯɣro}{}\classe{vt}
\paradigme{\textit{dir :} \jya pɯ-}
\begin{définition}\ 
\begin{déclaration}\grammar{caus}\end{déclaration}\end{définition}
\begin{définition}\fra étouffer\end{définition}
\begin{définition}\cmn 呛\end{définition}
\begin{exemple}\jya pɯ́-wɣ-sɯɣro-a\cmn 我被呛到了\end{exemple}
\begin{exemple}\jya ɯ-tshɤt kɤ-tshi ma tɯ́-wɣ-sɯɣro\cmn 不要喝得太多,不然会呛着\end{exemple}
\end{sous-entrée}\end{entrée}

\begin{entrée}
\vedette{\hypertarget{Ⓔɣurʑa}{\papi{ ɣurʑa}}}\markboth{ɣurʑa}{}\classe{n}
\begin{définition}\fra cent\end{définition}
\begin{définition}\cmn 100\end{définition}
\begin{exemple}\jya ɣurʑa cho tɯ-rdoʁ\cmn 一百零一\end{exemple}
\begin{exemple}\jya ɯ-ɣurʑa-xpa\cmn 好几百年\end{exemple}
\begin{relation-sémantique}\synonyme{
\hyperlink{Ⓔtɯ-ri}{\textit{ \papi{tɯ-ri}}}
}\end{relation-sémantique}\end{entrée}

\begin{entrée}
\vedette{\hypertarget{Ⓔɣɯβɣɯβ}{\papi{ ɣɯβɣɯβ}}}\markboth{ɣɯβɣɯβ}{}\classe{idph.2}
\begin{définition}\fra gens attroupés autour de quelque chose\end{définition}
\begin{définition}\cmn 围拢起来,人与人之间没有缝隙\end{définition}
\begin{exemple}\jya tɤjmɤɣ ɯ-kɯ-ntsɣe jo-ɣi-nɯ tɕe, kɤntɕhaʁ tɯrme ra ɣɯβɣɯβ ʑo ko-nɤrkhar-nɯ\cmn 卖菌子的人来了,街上很多人围着看\end{exemple}
\begin{exemple}\jya ɲɯ-mpja ɣɯβɣɯβ ʑo\cmn 暖烘烘\end{exemple}
\begin{relation-sémantique}\confer{
\hyperlink{Ⓔnɤɣɯβɣɯβ}{\textit{ \papi{nɤɣɯβɣɯβ}}}
}\end{relation-sémantique}\end{entrée}

\begin{entrée}
\vedette{\hypertarget{Ⓔɣɯcɤno}{\papi{ ɣɯcɤno}}}\markboth{ɣɯcɤno}{}\classe{vi}
\paradigme{\textit{dir :} \jya \_}
\begin{définition}\ 
\begin{déclaration}\grammar{incorp}\end{déclaration}\end{définition}
\begin{définition}\fra faire de la chasse à courre\end{définition}
\begin{définition}\cmn 围猎\end{définition}
\begin{relation-sémantique}\confer{
\hyperlink{Ⓔca}{\textit{ \papi{ca}}}
}\end{relation-sémantique}
\begin{relation-sémantique}\confer{
\hyperlink{Ⓔno}{\textit{ \papi{no}}}
}\end{relation-sémantique}\end{entrée}

\begin{entrée}
\vedette{\hypertarget{Ⓔɣɯchɤtshi}{\papi{ ɣɯchɤtshi}}}\markboth{ɣɯchɤtshi}{}
\classe{vi}
\paradigme{\textit{dir :} \jya kɤ-}
\begin{définition}\ 
\begin{déclaration}\grammar{incorp}\end{déclaration}\end{définition}
\begin{définition}\fra boire du vin (ensemble)\end{définition}
\begin{définition}\cmn 喝酒(几个人一起)\end{définition}
\begin{exemple}\jya pɯ-ɣɯchɤtshi-j\cmn 我们在喝酒(过去)\end{exemple}
\begin{exemple}\jya kɤ-ɣɯchɤtshi-j\cmn 我们喝了酒\end{exemple}
\begin{relation-sémantique}\confer{
\hyperlink{Ⓔchɤtshi}{\textit{ \papi{chɤtshi}}}
}\end{relation-sémantique}\end{entrée}

\begin{entrée}
\vedette{\hypertarget{Ⓔɣɯcɯphɯt}{\papi{ ɣɯcɯphɯt}}}\markboth{ɣɯcɯphɯt}{}
\classe{vi}
\paradigme{\textit{dir :} \jya nɯ-}
\paradigme{\textit{dir :} \jya pɯ-}
\begin{définition}\ 
\begin{déclaration}\grammar{incorp}\end{déclaration}\end{définition}
\begin{définition}\fra ramasser les pierres\end{définition}
\begin{définition}\cmn 拣石头(庄稼地)\end{définition}
\begin{exemple}\jya jisŋi tɯ-ji ɯ-ŋgɯ pɯ-ɣɯcɯphɯt-a\cmn 今天我在田地里捡石头\end{exemple}
\begin{exemple}\jya jiʑo ji-ji ɯ-ŋgɯ pɯ-ɣɯcɯphɯt-a\cmn 我在我们田里捡了石头\end{exemple}
\begin{relation-sémantique}\confer{
\hyperlink{Ⓔcɯphɯt}{\textit{ \papi{cɯphɯt}}}
}\end{relation-sémantique}\end{entrée}

\begin{entrée}
\vedette{\hypertarget{Ⓔɣɯɕkat}{\papi{ ɣɯɕkat}}}\markboth{ɣɯɕkat}{}
\classe{vt}
\paradigme{\textit{dir :} \jya tɤ-}
\begin{définition}\ 
\begin{déclaration}\grammar{denom}\end{déclaration}\end{définition}
\begin{définition}\fra mettre les charges sur les animaux\end{définition}
\begin{définition}\cmn 上驮子\end{définition}
\begin{exemple}\jya mbala tɤ-ɣɯɕkat-i\cmn 我们给牛上了驮子\end{exemple}
\begin{exemple}\jya jla tɤ-ɣɯɕkat-i\cmn 我们给犏牛上了驮子\end{exemple}
\begin{exemple}\jya mbro tɤ-ɣɯɕkat-i\cmn 我们给马上了驮子\end{exemple}
\begin{exemple}\jya jla tɤ-ɣɯɕkat-a\cmn 我给犏牛上了驮子\end{exemple}
\begin{exemple}\jya laχtɕha tɤ-ɣɯɕkat-a\cmn 我驮了东西\end{exemple}
\begin{relation-sémantique}\confer{
\hyperlink{Ⓔtɯ-ɕkat}{\textit{ \papi{tɯ-ɕkat}}}
}\end{relation-sémantique}
\begin{relation-sémantique}\confer{
\hyperlink{Ⓔnɯɕkat}{\textit{ \papi{nɯɕkat}}}
}\end{relation-sémantique}\begin{sous-entrée}
\vedette{\hypertarget{}{\papi{ aɣɯɕkat}}}\markboth{aɣɯɕkat}{}\classe{vi}
\begin{définition}\fra porter comme charge\end{définition}
\begin{définition}\cmn 驮着\end{définition}
\begin{exemple}\jya ki tɤrka ki tɤrɤku aɣɯɕkat\cmn 这头驴子驮的是粮食\end{exemple}
\end{sous-entrée}\end{entrée}

\begin{entrée}
\vedette{\hypertarget{Ⓔɣɯɕoŋtɕa}{\papi{ ɣɯɕoŋtɕa}}}\markboth{ɣɯɕoŋtɕa}{}
\classe{vi}
\paradigme{\textit{dir :} \jya pɯ-}
\begin{définition}\ 
\begin{déclaration}\grammar{denom}\end{déclaration}\end{définition}
\begin{définition}\fra couper du bois\end{définition}
\begin{définition}\cmn 砍木头\end{définition}
\begin{exemple}\jya pɯ-ɣɯɕoŋtɕa-a\cmn 我砍木头了\end{exemple}
\begin{exemple}\jya ɕ-pɯ-ɣɯɕoŋtɕa-a\cmn 我去砍木头了\end{exemple}
\begin{relation-sémantique}\confer{
\hyperlink{Ⓔɕoŋtɕa}{\textit{ \papi{ɕoŋtɕa}}}
}\end{relation-sémantique}\end{entrée}

\begin{entrée}
\vedette{\hypertarget{Ⓔɣɯɕɯ}{\papi{ ɣɯɕɯ}}}\markboth{ɣɯɕɯ}{}
\classe{vs}
\paradigme{\textit{dir :} \jya thɯ-}
\begin{définition}\fra âgé et respecté, calme et avisé\end{définition}
\begin{définition}\cmn 稳重;沉着
\end{définition}
\begin{exemple}\jya iɕqha tɯrme nɯ kɯ-ɣɯɕɯ ci ɲɯ-ŋu, ɲɯ-ɣɯɕɯ\cmn 那个人很稳重\end{exemple}\end{entrée}

\begin{entrée}
\vedette{\hypertarget{Ⓔɣɯfkɯm}{\papi{ ɣɯfkɯm}}}\markboth{ɣɯfkɯm}{}
\classe{vt}
\paradigme{\textit{dir :} \jya tɤ-}
\begin{définition}\ 
\begin{déclaration}\grammar{denom}\end{déclaration}\end{définition}\acception{1}
\begin{définition}\fra mettre dans sa poche\end{définition}
\begin{définition}\cmn 装在口袋里\end{définition}
\begin{exemple}\jya tɯjpu nɯ tɤ-ɣɯfkɯm-a\cmn 我把粮食装在口袋里了\end{exemple}
\begin{exemple}\jya qha laχtɕha nɯ tɤ-ɣɯfkɯm-a\cmn 我把那个东西装在口袋里了\end{exemple}
\begin{exemple}\jya qha laχtɕha nɯra ɣɯfkɯm-i\cmn 我们要把这些东西装在口袋里\end{exemple}\acception{2}
\begin{définition}\fra conserver dans le grenier\end{définition}
\begin{définition}\cmn 储存在仓库里\end{définition}
\begin{relation-sémantique}\confer{
\hyperlink{Ⓔtɤ-fkɯm}{\textit{ \papi{tɤ-fkɯm}}}
}\end{relation-sémantique}\end{entrée}

\begin{entrée}
\vedette{\hypertarget{Ⓔɣɯfsu}{\papi{ ɣɯfsu}}}\markboth{ɣɯfsu}{}
\classe{n}
\begin{définition}\fra ami\end{définition}
\begin{définition}\cmn 朋友\end{définition}
\begin{exemple}\jya ɣɯfsu to-nɯpa-ndʑi\cmn 我们交了朋友\end{exemple}
\begin{relation-sémantique}\confer{
\hyperlink{Ⓔɯ-fsu}{\textit{ \papi{ɯ-fsu}}}
}\end{relation-sémantique}\end{entrée}

\begin{entrée}
\vedette{\hypertarget{Ⓔɣɯɣlɯtɕɤt}{\papi{ ɣɯɣlɯtɕɤt}}}\markboth{ɣɯɣlɯtɕɤt}{}
\classe{vi}
\paradigme{\textit{dir :} \jya thɯ-}
\begin{définition}\ 
\begin{déclaration}\grammar{incorp}\end{déclaration}\end{définition}
\begin{définition}\fra retirer le purin de l'étable pour en faire de l'engrais\end{définition}
\begin{définition}\cmn 出圈粪\end{définition}
\begin{exemple}\jya thɯ-ɣɯɣlɯtɕɤt-i, pɯ-ɣɯɣlɯtɕɤt-i\cmn 我们出圈粪\end{exemple}
\begin{relation-sémantique}\confer{
\hyperlink{Ⓔɣlɯtɕɤt}{\textit{ \papi{ɣlɯtɕɤt}}}
}\end{relation-sémantique}\end{entrée}

\begin{entrée}
\vedette{\hypertarget{Ⓔɣɯɣndʑɤβta}{\papi{ ɣɯɣndʑɤβta}}}\markboth{ɣɯɣndʑɤβta}{}
\classe{vi}
\paradigme{\textit{dir :} \jya tɤ-}
\begin{définition}\ 
\begin{déclaration}\grammar{incorp}\end{déclaration}\end{définition}
\begin{définition}\fra défricher par le feu\end{définition}
\begin{définition}\cmn 烧荒\end{définition}
\begin{exemple}\jya jɯfɕɯr ɕ-pɯ-ɣɯɣndʑɤβta-j\cmn 我们昨天去烧荒了\end{exemple}
\begin{relation-sémantique}\confer{
\hyperlink{Ⓔɣndʑɤβta}{\textit{ \papi{ɣndʑɤβta}}}
}\end{relation-sémantique}
\begin{relation-sémantique}\confer{
\hyperlink{Ⓔɣndʑɤβta}{\textit{ \papi{ɣndʑɤβta}}}
}\end{relation-sémantique}\end{entrée}

\begin{entrée}
\vedette{\hypertarget{Ⓔɣɯjpa}{\papi{ ɣɯjpa}}}\markboth{ɣɯjpa}{}
\classe{adv}
\begin{définition}\fra cette année\end{définition}
\begin{définition}\cmn 今年\end{définition}\end{entrée}

\begin{entrée}
\vedette{\hypertarget{Ⓔɣɯjru}{\papi{ ɣɯjru}}}\markboth{ɣɯjru}{}\classe{vt}
\paradigme{\textit{dir :} \jya kɤ-}
\begin{définition}\fra cuire (poterie)\end{définition}
\begin{définition}\cmn 炙烤(泥制品)\end{définition}
\begin{exemple}\jya tɕhorzi kɤ-ɣɯjru-t-a\cmn 我把坛子炙烤了\end{exemple}
\begin{exemple}\jya tɕhorzi ɯ-ɣɯjru mɯ-ko-rtaʁ\cmn 坛子没有烤熟\end{exemple}\end{entrée}

\begin{entrée}
\vedette{\hypertarget{Ⓔɣɯjtsi}{\papi{ ɣɯjtsi}}}\markboth{ɣɯjtsi}{}
\classe{vt}
\paradigme{\textit{dir :} \jya tɤ-}
\begin{définition}\fra soutenir\end{définition}
\begin{définition}\cmn (用柱子)顶住\end{définition}
\begin{exemple}\jya kɯki tɤ-ɣɯjtsi-t-a\cmn 我把这个顶住了\end{exemple}
\begin{exemple}\jya tɕɤtu nɯ ɯ-pa pjɯ-nɯɣi ɲɯ-ŋu tɕe tɤ-ɣɯjtsi-t-a\cmn 上面的那个东西往下面来,我把往下来的部分顶住了\end{exemple}
\begin{relation-sémantique}\confer{
\hyperlink{Ⓔtɤ-jtsi}{\textit{ \papi{tɤ-jtsi}}}
}\end{relation-sémantique}\end{entrée}

\begin{entrée}
\vedette{\hypertarget{Ⓔɣɯkhɯtshoʁ}{\papi{ ɣɯkhɯtshoʁ}}}\markboth{ɣɯkhɯtshoʁ}{}
\classe{vi}
\paradigme{\textit{dir :} \jya pɯ-}
\paradigme{\textit{dir :} \jya thɯ-}
\begin{définition}\ 
\begin{déclaration}\grammar{incorp}\end{déclaration}\end{définition}
\begin{définition}\fra lâcher les chiens (à la chasse)\end{définition}
\begin{définition}\cmn 放狗打猎\end{définition}
\begin{exemple}\jya jɯfɕo kɯ-ɣɯkhɯtshoʁ jɤ-ari-a\cmn 我今天早上去放狗打猎了\end{exemple}
\begin{exemple}\jya ɕ-pɯ-ɣɯkhɯtshoʁ-a\cmn 我去放狗打猎了\end{exemple}
\begin{exemple}\jya kɯ-ɣɤrʁaʁ ci jo-ɣi tɕe, khɯna to-ndo, ta-ʁrɯ chɤ-ʑmbri tɕe, chɤ-ɣɯkhɯtshoʁ\cmn 来了一个猎人,带着猎狗,吹了螺号,然后把狗放了\end{exemple}
\begin{relation-sémantique}\confer{
\hyperlink{Ⓔkhɯtshoʁ}{\textit{ \papi{khɯtshoʁ}}}
}\end{relation-sémantique}
\begin{relation-sémantique}\confer{
 \papi{khɯna,tshoʁ}
}\end{relation-sémantique}\end{entrée}

\begin{entrée}
\vedette{\hypertarget{Ⓔɣɯlaj}{\papi{ ɣɯlaj}}}\markboth{ɣɯlaj}{}\classe{vs}
\begin{définition}\fra aux gestes\end{définition}
\begin{définition}\cmn 动作慢\end{définition}
\begin{exemple}\jya ta-ma ɲɯ-tɯ-ɣɯlaj\cmn 你劳动的时候动作慢\end{exemple}
\begin{relation-sémantique}\antonyme{
\hyperlink{Ⓔɣɤjru}{\textit{ \papi{ɣɤjru}}}
}\end{relation-sémantique}\end{entrée}

\begin{entrée}
\vedette{\hypertarget{Ⓔɣɯlɤn}{\papi{ ɣɯlɤn}}}\markboth{ɣɯlɤn}{}
\classe{vt}
\paradigme{\textit{dir :} \jya nɯ-}
\paradigme{\textit{dir :} \jya tɤ-}
\begin{définition}\ 
\begin{déclaration}\grammar{denom}\end{déclaration}\end{définition}
\begin{définition}\fra répondre\end{définition}
\begin{définition}\cmn 回答
\begin{déclaration} \étymologie{\papi{len}}\end{déclaration}\end{définition}
\begin{exemple}\jya ɯʑo kɯ na-ɣɯlɤn\cmn 他回答了\end{exemple}
\begin{exemple}\jya jiɕqha ɲɯ-ɤkhu tɕe tɤ-ɣɯlan-a\cmn 他叫了,我就回答了\end{exemple}
\begin{exemple}\jya ɲo-sɯthu tɕe tɤ-ɣɯlan-a\cmn 他问了我就回答了\end{exemple}
\begin{relation-sémantique}\confer{
\hyperlink{Ⓔtɯ-lɤn}{\textit{ \papi{tɯ-lɤn}}}
}\end{relation-sémantique}\end{entrée}

\begin{entrée}
\vedette{\hypertarget{Ⓔɣɯpɕawtsɯfsoʁ}{\papi{ ɣɯpɕawtsɯfsoʁ}}}\markboth{ɣɯpɕawtsɯfsoʁ}{}
\classe{vi}
\paradigme{\textit{dir :} \jya nɯ-}
\begin{définition}\ 
\begin{déclaration}\grammar{incorp}\end{déclaration}\end{définition}
\begin{définition}\fra gagner de l'argent\end{définition}
\begin{définition}\cmn 挣钱\end{définition}
\begin{exemple}\jya ʑ-nɯ-ɣɯpɕawtsɯfsoʁ\cmn 你去挣钱吧\end{exemple}
\begin{exemple}\jya kɯ-nɯpɕawtsɯfsoʁ jɤ-ari\cmn 他去打工挣钱了\end{exemple}
\begin{exemple}\jya kɯ-nɯhɯɲi jɤ-ari tɕe ɲɯ-ɣɯpɕawtsɯfsoʁ\cmn 他去打工挣钱了\end{exemple}\end{entrée}

\begin{entrée}
\vedette{\hypertarget{Ⓔɣɯrɣɯr}{\papi{ ɣɯrɣɯr}}}\markboth{ɣɯrɣɯr}{}
\classe{idph.2}
\begin{définition}\fra beaucoup de gens rassemblés\end{définition}
\begin{définition}\cmn 形容人密集的样子\end{définition}
\begin{exemple}\jya ɣɯrɣɯr ʑo pjɤ-k-ɤkhar-nɯ-ci\cmn 很多人围着他\end{exemple}\begin{sous-entrée}
\vedette{\hypertarget{}{\papi{ ɣɤɣɯrɣɯr}}}\markboth{ɣɤɣɯrɣɯr}{}\classe{vi}
\paradigme{\textit{dir :} \jya tɤ-}\acception{1}
\begin{définition}\fra animé, bruyant\end{définition}
\begin{définition}\cmn 嘈杂(声音);闹哄哄;熙熙攘攘\end{définition}
\begin{exemple}\jya tɯrme ra ɲɯ-ɣɤɣɯrɣɯr-nɯ\cmn 人们很吵\end{exemple}
\begin{exemple}\jya ɲɯ-ɣɤɕqali-nɯ tɕe ɲɯ-ɣɤɣɯrɣɯr-nɯ\cmn 他们在吼叫,很吵\end{exemple}\acception{2}
\begin{définition}\fra ardent (feu)\end{définition}
\begin{définition}\cmn 旺盛(火)\end{définition}
\begin{exemple}\jya smi ɲɯ-ɣɤɣɯrɣɯr ɲɯ-nɯt\cmn 火烧得很旺盛\end{exemple}
\end{sous-entrée}\begin{sous-entrée}
\vedette{\hypertarget{}{\papi{ ɣɯrnɤɣɯr}}}\markboth{ɣɯrnɤɣɯr}{}\classe{idph.3}
\end{sous-entrée}\end{entrée}

\begin{entrée}
\vedette{\hypertarget{Ⓔɣɯri}{\papi{ ɣɯri}}}\markboth{ɣɯri}{}
\classe{vt}
\begin{définition}\ 
\begin{déclaration}\grammar{denom}\end{déclaration}\end{définition}\acception{1}
\paradigme{\textit{dir :} \jya thɯ-}
\begin{définition}\fra insérer des perles sur un fil\end{définition}
\begin{définition}\cmn 穿珠\end{définition}
\begin{exemple}\jya tɤ-rɣe thɯ-ɣɯri\cmn 你穿珠子吧!\end{exemple}
\begin{exemple}\jya rɤjndoʁ (kɯ-fse) cho-ɣɯri\cmn 他穿了芜菁根(用杨柳条子穿,穿了以后晒干)\end{exemple}\acception{2}
\paradigme{\textit{dir :} \jya nɯ-}
\begin{définition}\fra mettre un fil dans le chas d'une aiguille\end{définition}
\begin{définition}\cmn 穿针\end{définition}
\begin{exemple}\jya taqaβrna nɯ-ɣɯri-t-a\cmn 我穿了针\end{exemple}
\begin{relation-sémantique}\confer{
\hyperlink{Ⓔtɤ-ri}{\textit{ \papi{tɤ-ri}}}
}\end{relation-sémantique}\end{entrée}

\begin{entrée}
\vedette{\hypertarget{Ⓔɣɯrɟɤn}{\papi{ ɣɯrɟɤn}}}\markboth{ɣɯrɟɤn}{}
\classe{vt}
\paradigme{\textit{dir :} \jya tɤ-}
\paradigme{\textit{dir :} \jya kɤ-}
\begin{définition}\fra décoré\end{définition}
\begin{définition}\cmn 装饰\end{définition}
\begin{exemple}\jya kha nɯ kɯ-wɣrum to-lɤt-nɯ tɕe to-ɣɯrɟɤn-nɯ\cmn 他们把家涂成白色,这样装饰了\end{exemple}
\begin{exemple}\jya si ɯ-taʁ qarma ɯ-muj ko-tshoʁ-nɯ tɕe ko-ɣɯrɟɤn-nɯ\cmn 他们用马鸡的羽毛装饰了枝桠\end{exemple}\end{entrée}

\begin{entrée}
\vedette{\hypertarget{Ⓔɣɯrɟɯfsoʁ}{\papi{ ɣɯrɟɯfsoʁ}}}\markboth{ɣɯrɟɯfsoʁ}{} (\variante{rɯrɟɯfsoʁ}) 
\classe{vi}
\paradigme{\textit{dir :} \jya nɯ-}
\paradigme{\textit{dir :} \jya tɤ-}
\begin{définition}\ 
\begin{déclaration}\grammar{incorp}\end{déclaration}\end{définition}
\begin{définition}\fra gagner de l'argent\end{définition}
\begin{définition}\cmn 挣钱;财产\end{définition}
\begin{exemple}\jya nɯ-ɣɯrɟɯfsoʁ-a\cmn 我挣了钱\end{exemple}
\begin{relation-sémantique}\confer{
\hyperlink{Ⓔrɟɯfsoʁ}{\textit{ \papi{rɟɯfsoʁ}}}
}\end{relation-sémantique}\end{entrée}

\begin{entrée}
\vedette{\hypertarget{Ⓔɣɯrni}{\papi{ ɣɯrni}}}\markboth{ɣɯrni}{}\classe{vs}
\paradigme{\textit{dir :} \jya nɯ-}
\paradigme{\textit{dir :} \jya thɯ-}
\begin{définition}\fra rouge\end{définition}
\begin{définition}\cmn 红\end{définition}
\begin{exemple}\jya a-rŋa ɯ-ɲó-ɣɯrni\cmn 我的脸变红了吗\end{exemple}
\begin{exemple}\jya tɤ-se pjɤ-ɣɯrni\cmn 出血了\end{exemple}\begin{sous-entrée}
\vedette{\hypertarget{}{\papi{ zɣɯrni}}}\markboth{zɣɯrni}{}\classe{vt}
\begin{définition}\ 
\begin{déclaration}\grammar{caus}\end{déclaration}\end{définition}
\begin{définition}\fra rendre rouge\end{définition}
\begin{définition}\cmn 使其变红\end{définition}
\begin{exemple}\jya nɯ-zɣɯrni-t-a\cmn 他令它变红了\end{exemple}
\end{sous-entrée}\end{entrée}

\begin{entrée}
\vedette{\hypertarget{Ⓔɣɯrŋi}{\papi{ ɣɯrŋi}}}\markboth{ɣɯrŋi}{}
\classe{vs}
\begin{définition}\fra vert (bois)\end{définition}
\begin{définition}\cmn 未干;生(材)\end{définition}
\begin{exemple}\jya si ɲɯ-ɣɯrŋi\cmn 柴没干\end{exemple}
\begin{relation-sémantique}\antonyme{
\hyperlink{Ⓔrom}{\textit{ \papi{rom}}}
}\end{relation-sémantique}\end{entrée}

\begin{entrée}
\vedette{\hypertarget{Ⓔɣɯscur}{\papi{ ɣɯscur}}}\markboth{ɣɯscur}{}
\classe{vt}
\paradigme{\textit{dir :} \jya tɤ-}
\begin{définition}\fra tenir dans les deux mains\end{définition}
\begin{définition}\cmn 捧\end{définition}
\begin{exemple}\jya mbrɤz pjɤ-ʁndɤr tɕe tɤ-ɣɯscur-a\cmn 米撒了,我就捧起来了\end{exemple}
\begin{exemple}\jya tɤ-ŋgɯm a-mɤ-pɯ-ɴɢrɯ ɲɯ-ra tɕe, tɤ-ɣɯscur-a\cmn 为了不要把鸡蛋打烂,我把它捧起了\end{exemple}\end{entrée}

\begin{entrée}
\vedette{\hypertarget{Ⓔɣɯsɯphɯt}{\papi{ ɣɯsɯphɯt}}}\markboth{ɣɯsɯphɯt}{}\classe{vi}
\paradigme{\textit{dir :} \jya pɯ-}
\paradigme{\textit{dir :} \jya lɤ-}
\begin{définition}\ 
\begin{déclaration}\grammar{incorp}\end{déclaration}\end{définition}
\begin{définition}\fra couper du bois pour faire du feu\end{définition}
\begin{définition}\cmn 打柴;砍柴\end{définition}
\begin{exemple}\jya Yingchun kɯ-ɣɯsɯphɯt jo-ɕe\cmn 迎春去砍柴了\end{exemple}
\begin{exemple}\jya jisŋi pɯ-ɣɯsɯphɯt-a\cmn 我今天砍了柴\end{exemple}
\begin{exemple}\jya nɯʑo ɯ-tɯ́-ɣɯsɯphɯt-nɯ\cmn 你们要砍柴吗?\end{exemple}
\begin{exemple}\jya ɕɯ-ɣɯsɯphɯt-a ŋu\cmn 我去砍柴\end{exemple}
\begin{relation-sémantique}\confer{
\hyperlink{Ⓔsɯphɯt}{\textit{ \papi{sɯphɯt}}}
}\end{relation-sémantique}\end{entrée}

\begin{entrée}
\vedette{\hypertarget{Ⓔɣɯt}{\papi{ ɣɯt}}}\markboth{ɣɯt}{}
\classe{vt}
\paradigme{\textit{dir :} \jya jɤ-}
\begin{définition}\fra amener\end{définition}
\begin{définition}\cmn 带来
\begin{déclaration}\use{\stylefv{ɣɯt}是从\stylefv{ɣi}“来”派生出来的,是唯一一个施用后缀\stylefv{-t}的例子}\end{déclaration}\end{définition}
\begin{exemple}\jya tɯji kɯ tɤ-rɤku tɤ-kɤ-ɣɯt nɯ pe\cmn 那块地里长出的庄稼好\end{exemple}
\begin{exemple}\jya nɤki tɯji kɯ tɤ-rɤku kɯ-pɯ-pe ʑo to-ɣɯt\cmn 这块地长出了很好的庄稼\end{exemple}
\begin{exemple}\jya @chezi ɯ-ŋgɯ kɤ-mdzɯt mɤ-cha ma @yunche ɯ-tɯ-βzu saχaʁ ʑo qhe, tɕe kɤ-ɣɯt mɤ-sna wo.\cmn 她不能坐车,因为晕车晕得很厉害,所以不能把她带来\end{exemple}
\begin{exemple}\jya tɤ-scoz ɯ-kɯ-ɣɯt ci pɯ-tu\cmn 有人送信了\end{exemple}
\begin{exemple}\jya ku-ɣɯt-a ɕi? ɲɯ-pe, a-kɤ-tɯ-ɣɯt\cmn 我要不要带来?好的,带来吧\end{exemple}
\begin{relation-sémantique}\confer{
\hyperlink{Ⓔɣi}{\textit{ \papi{ɣi}}}
}\end{relation-sémantique}\end{entrée}

\begin{entrée}
\vedette{\hypertarget{Ⓔɣɯtɕha}{\papi{ ɣɯtɕha}}}\markboth{ɣɯtɕha}{}
\classe{vt}
\paradigme{\textit{dir :} \jya nɯ-}
\begin{définition}\ 
\begin{déclaration}\grammar{denom}\end{déclaration}\end{définition}
\begin{définition}\fra répondre\end{définition}
\begin{définition}\cmn 回答\end{définition}
\begin{exemple}\jya jiɕqha tɤ-tɯ-thu-t nɯ nɯ-ta-ɣɯtɕha\cmn 我已经回答了你刚才问的那些问题\end{exemple}
\begin{relation-sémantique}\confer{
\hyperlink{Ⓔtɯ-tɕhaⒽ1}{\textit{ \papi{tɯ-tɕha1}}}
}\end{relation-sémantique}\end{entrée}

\begin{entrée}
\vedette{\hypertarget{Ⓔɣɯthaʁ}{\papi{ ɣɯthaʁ}}}\markboth{ɣɯthaʁ}{}
\classe{vt}
\paradigme{\textit{dir :} \jya nɯ-}
\begin{définition}\fra séparer deux objets en mettant un autre objet entre eux\end{définition}
\begin{définition}\cmn 在中间放一个东西隔开\end{définition}
\begin{exemple}\jya nɯ-ɣɯthaʁ-a\cmn 我隔开了;我分开了\end{exemple}
\begin{exemple}\jya tɤrɤm ɯ-pɤrthɤβ tɕe ndʑu ci pjɯ́-wɣ-rku tɕe ɲɯ́-wɣ-z-ɣɯthaʁ ŋu\cmn 在两个木板中间插了一根木棒把木板隔开\end{exemple}
\begin{exemple}\jya mbɣɤsroʁ nɯ kɯ mbɣɤru cho mbɣopɤl ɲɯ́-wɣ-z-ɣɯthaʁ ra\cmn 木棒要把犁干和犁头两边顶住(使它们俩稳固)\end{exemple}\end{entrée}

\begin{entrée}
\vedette{\hypertarget{Ⓔɣɯtshɤdɯɣ}{\papi{ ɣɯtshɤdɯɣ}}}\markboth{ɣɯtshɤdɯɣ}{}
\classe{vi}
\paradigme{\textit{dir :} \jya thɯ-}
\begin{définition}\fra chaud (temps)\end{définition}
\begin{définition}\cmn 很热(天气)
\begin{déclaration} \étymologie{\papi{tsʰa.dug}}\end{déclaration}\end{définition}
\begin{exemple}\jya tɯ-mɯ ɲɯ-jɯm, ɲɯ-ɣɯtshɤdɯɣ\cmn 天晴了,很热\end{exemple}\end{entrée}

\begin{entrée}
\vedette{\hypertarget{Ⓔɣɯtʂɤmtshi}{\papi{ ɣɯtʂɤmtshi}}}\markboth{ɣɯtʂɤmtshi}{}
\classe{vi}
\paradigme{\textit{dir :} \jya jɤ-}
\begin{définition}\ 
\begin{déclaration}\grammar{incorp}\end{déclaration}\end{définition}
\begin{définition}\fra conduire le chemin\end{définition}
\begin{définition}\cmn 带路;引路\end{définition}
\begin{exemple}\jya jo-ɣɯtʂɤmtshi\cmn 他带路了\end{exemple}
\begin{exemple}\jya jɤ-ɣɯtʂɤmtshi-a\cmn 我带路了\end{exemple}
\begin{exemple}\jya jiɕqha tʂu mɯ́j-sɯz tɕe, ɯ-kɯ-ɣɯtʂɤmtshi jo-ɕe\cmn 他不认识路,给他带路了\end{exemple}
\begin{relation-sémantique}\confer{
\hyperlink{Ⓔtʂu}{\textit{ \papi{tʂu}}}
}\end{relation-sémantique}
\begin{relation-sémantique}\confer{
\hyperlink{Ⓔmtshi}{\textit{ \papi{mtshi}}}
}\end{relation-sémantique}
\begin{relation-sémantique}\confer{
\hyperlink{Ⓔtʂɤmtshi}{\textit{ \papi{tʂɤmtshi}}}
}\end{relation-sémantique}\end{entrée}

\begin{entrée}
\vedette{\hypertarget{Ⓔɣɯtʂhɤtshi}{\papi{ ɣɯtʂhɤtshi}}}\markboth{ɣɯtʂhɤtshi}{}
\classe{vi}
\paradigme{\textit{dir :} \jya kɤ-}
\paradigme{\textit{dir :} \jya pɯ-}
\begin{définition}\ 
\begin{déclaration}\grammar{incorp}\end{déclaration}\end{définition}
\begin{définition}\fra boire du thé\end{définition}
\begin{définition}\cmn 喝茶\end{définition}
\begin{exemple}\jya jisŋi rcanɯ pɯ-ɣɯtʂhɤtshi-tɕi ko\cmn 我们今天喝了茶\end{exemple}
\begin{exemple}\jya tshi\end{exemple}
\begin{relation-sémantique}\confer{
\hyperlink{Ⓔtʂha}{\textit{ \papi{tʂha}}}
}\end{relation-sémantique}\end{entrée}

\begin{entrée}
\vedette{\hypertarget{Ⓔɣɯχsɤl}{\papi{ ɣɯχsɤl}}}\markboth{ɣɯχsɤl}{}
\classe{vt}
\paradigme{\textit{dir :} \jya tɤ-}
\begin{définition}\fra prouver, réfuter\end{définition}
\begin{définition}\cmn 证实;用证据来反驳错误的说法\end{définition}
\begin{exemple}\jya jɯfɕɯr ji-ŋgumdʑɯɣ kɯ "mɯ-to-tɯ-sɤpe-t" ɲɯ-ti tɕe, jisŋi tɕe aʑo tɤ-nɯ-ɣɯχsal-a\cmn 昨天领导说我做错了事,今天我就证明了我没有错\end{exemple}
\begin{exemple}\jya kɯki tɤ-ndɤm tɕe, ɯ-sɤz-ɣɯχsɤl a-pɯ-ŋu\cmn 你拿着当做证据\end{exemple}
\begin{relation-sémantique}\confer{
\hyperlink{ⒺχsɤlⒽ1}{\textit{ \papi{χsɤl}}}
}\end{relation-sémantique}\end{entrée}

\begin{entrée}
\vedette{\hypertarget{Ⓔɣzɤn}{\papi{ ɣzɤn}}}\markboth{ɣzɤn}{}\classe{n}
\begin{définition}\fra appât, leurre\end{définition}
\begin{définition}\cmn 鱼饵;诱饵
\begin{déclaration} \étymologie{\papi{gzan}}\end{déclaration}\end{définition}
\begin{exemple}\jya ɯ-fsa ɯ-ŋgɯ ɯ-ɣzɤn pɯ-pɯ-me nɤ mɤ-ɕe\cmn 如果在陷阱里没有诱饵,(动物)是不会进去的\end{exemple}\end{entrée}

\begin{entrée}
\vedette{\hypertarget{Ⓔɣzɯ}{\papi{ ɣzɯ}}}\markboth{ɣzɯ}{}
\classe{n}
\begin{définition}\fra singe\end{définition}
\begin{définition}\cmn 猴子\end{définition}\end{entrée}

\begin{entrée}
\vedette{\hypertarget{Ⓔɣzɯlu}{\papi{ ɣzɯlu}}}\markboth{ɣzɯlu}{}\classe{n}
\begin{définition}\fra année du singe\end{définition}
\begin{définition}\cmn 猴年\end{définition}
\begin{relation-sémantique}\confer{
\hyperlink{Ⓔɣzɯ}{\textit{ \papi{ɣzɯ}}}
}\end{relation-sémantique}
\end{entrée}

\begin{entrée}
\vedette{\hypertarget{Ⓔɣzɯɬa}{\papi{ ɣzɯɬa}}}\markboth{ɣzɯɬa}{} (\variante{βzɯɬa}) 
\classe{n}
\begin{définition}\fra pika\end{définition}
\begin{définition}\cmn 崖兔【岩兔儿】\end{définition}
\begin{exemple}\jya ɣzɯɬa nɯ zndɤrchɤβ cho praʁ ɯ-rchɤβ ku-rɤʑi ŋu, ɯ-rme ɯ-mdoʁ nɯ βʑɯ ɯ-mdoʁ asɯndo, kɯ-pɣi ŋu. ɯ-tshɯɣa nɯ ra qala fse, ɯ-rna rɲɟi, ɯ-mtɕhi amtɕoʁ ɯ-jme xtɯt, ɯ-jaʁ xtɯt, ɯ-mi rɲɟi, xɕaj tu-ndze ŋu, tɯ-ji ɯ-rkɯ tɤ-rɤku ra tu-ndze ŋgrɤl, ɯ-zda ra nɯ ɯ-taʁ mɤ-ʁnɤt, ɯʑo ɯ-kɯ-ndza dɤn.\cmn 岩兔住在墙壁缝里和岩洞里,毛的颜色和老鼠的一样,是灰色的。形状像兔子,耳朵长,嘴尖,尾巴短,前腿短,后腿长,吃草,也吃地里的庄稼。不危害其它动物,但是是很多动物的猎物。\end{exemple}\end{entrée}

\begin{entrée}
\vedette{\hypertarget{Ⓔɣzɯthɯz}{\papi{ ɣzɯthɯz}}}\markboth{ɣzɯthɯz}{}\classe{n}
\begin{définition}\fra Selaginella sp.\end{définition}
\begin{définition}\cmn 卷柏\end{définition}
\begin{exemple}\jya ɣzɯthɯz nɯ praʁ ɯ-taʁ ku-ndzoʁ ŋu, cɤmi pɕoʁ sɤtɕha kɯ-mpja nɯ tɕu tu-ɬoʁ ɲɯ-ŋu. praʁ ɯ-taʁ ku-oɲɟoʁ ku-fse tu-ɬoʁ, tɕe ɯ-ru maŋe, ɯ-mɯntoʁ me, ɯ-jwaʁ nɯ ɕɤɣ ɣɯ ɯ-jwaʁ tsa ɲɯ-fse, ɯ-jwaʁ ɯ-χcɤl ɯ-ŋgɯ chu tu-ŋgɤɣ, tɕe ɯ-χcɤl pɕoʁ ku-wum kɯ-fse ɲɯ-ŋu. ɯ-mdoʁ nɯ ftɕar ɲɯ-ɤrŋi, qartsɯ ɲɯ-pɣi ɲɯ-ŋu ri, ɲɯ-rom mɯ́j-cha. kɯɕɯŋgɯ tɕe, ɲɯ-phɯt-nɯ tɕe, tɯ-thɯ ɯ-sɤ-χtɕi tu-βzu-nɯ pɯ-ŋu. tham tɕe ɯ-kɯ-ntɕhoz me.\cmn 卷柏生长在岩石上,在河坝气候温和的地方生长。长的样子好像是贴着岩石,没有茎,没有花,叶子有些类似柏树叶,叶子中间部分往里卷,合在一起。叶子的颜色夏绿冬灰,但不会干枯。以前,人们拆下来用来洗锅子,现在就没有人用了。\end{exemple}
\end{entrée}

\begin{entrée}
\vedette{\hypertarget{Ⓔɣʑɤndza}{\papi{ ɣʑɤndza}}}\markboth{ɣʑɤndza}{}
\classe{n}
\begin{définition}\fra Elshotzia sp.\end{définition}
\begin{définition}\cmn 藿香\end{définition}
\begin{exemple}\jya ɣʑɤndza nɯ sɯjno ŋu, mɤ-mbro, tɤ-rɤku ɯ-rchɤβ kɤ-ɬoʁ rga ɯ-mdoʁ kɯ-ɤrŋi ɯ-ŋgɯz kɯnɤ kɯ-ɤɲaʁndzɯm tsa ŋu, ɯ-rtaʁ dɤn, ɯ-mɯntoʁ nɯ phaʁrzi ɯ-tshɯɣa kɯ-fse tu, kɯ-ɤlɯlju ɯ-tshɯɣa tu, mɯntoʁ dɤn, ɣʑo wuma ʑo rga tɕe ɣʑɤndza rmi, ɯ-di wuma ʑo χɕɤβ\cmn 
藿香是一种植物,长得不高,生长在田地之间,绿色里面带有暗绿色,很多枝桠,花的形状像牙刷,是圆柱形的,花很多,蜜蜂特别喜欢,所以叫\stylefv{ɣʑɤndza}(蜜蜂的食物),味道很浓。
\end{exemple}\end{entrée}

\begin{entrée}
\vedette{\hypertarget{Ⓔɣʑɤzga}{\papi{ ɣʑɤzga}}}\markboth{ɣʑɤzga}{}
\classe{n}
\begin{définition}\fra miel\end{définition}
\begin{définition}\cmn 蜂蜜\end{définition}\begin{sous-entrée}
\vedette{\hypertarget{}{\papi{ ɣʑɤzga ɯ-rqhu}}}\markboth{ɣʑɤzga ɯ-rqhu}{}\classe{n}
\begin{définition}\fra alvéole\end{définition}
\begin{définition}\cmn 蜂房\end{définition}
\begin{exemple}\jya ɣʑɤzga pɯ-phɯt-a / kɤ-tɕat-a\cmn 我找了蜂蜜\end{exemple}
\end{sous-entrée}\end{entrée}

\begin{entrée}
\vedette{\hypertarget{Ⓔɣʑo}{\papi{ ɣʑo}}}\markboth{ɣʑo}{}
\classe{n}
\begin{définition}\fra abeille\end{définition}
\begin{définition}\cmn 蜜蜂\end{définition}
\begin{exemple}\jya ɣʑo nɯ kɤntɕhɯ-tɯphu tu, ɯ-zga kɯ-tu ci tu, sɤtɕha ɣʑo kɤ-ti ci tu, ndzɯrnaʁ kɤ-ti ci tu, li nɯ ɣʑo ɕti, tɕe ɯ-zga kɯ-tu nɯ khro mɤ-wxti tɕe tɯ-khɤl kɯ-dɯdɤn ʑo ku-rɤʑi ɕti, ftɕar mɯntoʁ nɯ-ʁaʁ tɕe, nɯ-tɯ-mbɣom saχaʁ, tɕe ku-rɤzga-nɯ ŋu, tɕe khɤrka ri tu-ndzoʁ-nɯ tɕe, nɯ-zga ku-tshoʁ-nɯ ŋu tɕe tɯ-xpa tɕe zgo kɯ-fse nɯ kɯngɯsqi jamar ku-tshoʁ ɲɯ-cha, tɕe pjɯ́-wɣ-phɯt tɕe, ɕnɤcat tɯ-rpa jamar tu-ŋgrɤl tɕe ɣʑo nɯ ɯ-ku kɯ-ɤmtɕoʁ ci ɯ-mthɤɣ kɯ-xtshɯm ci ɯ-xtu kɯ-wxti tsa ci ŋu, ɯ-ʁar kɯ-nɤmbju kɯ-mbɯ-mba ci ŋu ɯ-rɯmu tu, ɯ-mɤlɤjaʁ kɯtʂɤ-ldʑa tu, tɕeri sɤmtsɯɣ tɕe ɯ-mdzu nɯ ɯ-mphɯz ri ku-ndzoʁ ŋu, tɕe kɤ-kɯ-mtsɯɣ tɕe, ɯʑo ju-nɯɕe ŋu ri, ɯ-mdzu nɯ tɯ-ɕa ɯ-ŋgɯ ku-raʁ tɕe ɯ-mphɯz ɯ-ntɕhɯr ɲɯ-nɯ-phɯt ŋu tɕe ɯʑo pjɯ-si ŋu tu-kɯ-ti ŋu. tɕe kɤ-kɯ-mtsɯɣ tɕe aɣɯtɯɣ tɕe tɯ-ɕa ra ɯ-kho kɯ-jom chɯ-z-nɯɣmbɤβ cha. ɯ-zga nɯ wuma ʑo chi, smɤn kɯ-sna ɲɯ-ŋu. sɤtɕha ɣʑo nɯ sɤtɕha ɯ-ŋgɯ kɯ-spoʁ kɯ-tu nɯ tɕu ku-rɤʑi ɲɯ-ŋu, tɕe ɲɯ-dɤn.\cmn 
蜂有几种,一种叫蜜蜂,另一种住在地洞里,还有一种叫马蜂,也是蜂的一种。蜜蜂不是很大,一个地方住着很多只。春天花开的时候,它们忙着酿蜜。它们在天花板上筑巢,在那里酿蜜。一年可以产九十排蜂蜜,取下来时,至少都有七八斤。蜜蜂头部尖、腰部细、肚子大,翅膀很薄、有光泽、有纹路。有六只脚,蜇人,刺位于尾部。蜇人时,把刺插进人的皮肤后,自己飞走,但刺卡住在皮肤里,尾部的一块会留在刺上,据说蜜蜂会因而丧命。蜇后因为毒性大,皮肉上会起比较大的包。蜂蜜很甜,是一种好药。另一种蜂(\stylefv{sɤtɕha ɣʑo})成群住在地洞里。
\end{exemple}\end{entrée}

\begin{entrée}
\vedette{\hypertarget{Ⓔɣʑokha}{\papi{ ɣʑokha}}}\markboth{ɣʑokha}{}
\classe{n}
\begin{définition}\fra ruche\end{définition}
\begin{définition}\cmn 蜂窝\end{définition}\end{entrée}

\newpage\caractère{h}

\begin{entrée}
\vedette{\hypertarget{Ⓔhu}{\papi{ hu}}}\markboth{hu}{}
\classe{n}
\begin{définition}\fra souffle\end{définition}
\begin{définition}\cmn 吹热气\end{définition}
\begin{exemple}\jya hu nɤ hu ʑo tɤ-tɯt-a tɕe nɯ-ɣɤmpja-t-a\cmn 我吹了几下取暖\end{exemple}\end{entrée}

\begin{entrée}
\vedette{\hypertarget{Ⓔhajtsu}{\papi{ hajtsu}}}\markboth{hajtsu}{}\classe{n}
\begin{définition}\fra piment\end{définition}
\begin{définition}\cmn 辣椒\end{définition}
\end{entrée}

\begin{entrée}
\vedette{\hypertarget{Ⓔhanɯni}{\papi{ hanɯni}}}\markboth{hanɯni}{}\classe{adv}
\begin{définition}\fra un peu\end{définition}
\begin{définition}\cmn 稍微\end{définition}
\end{entrée}

\begin{entrée}
\vedette{\hypertarget{Ⓔhatsɯtsi}{\papi{ hatsɯtsi}}}\markboth{hatsɯtsi}{}\classe{adv}
\begin{définition}\fra un peu\end{définition}
\begin{définition}\cmn 稍微,一点点\end{définition}
\end{entrée}

\begin{entrée}
\vedette{\hypertarget{Ⓔhwɤl}{\papi{ hwɤl}}}\markboth{hwɤl}{}\classe{idph.3}
\begin{définition}\fra qui est découvert d'un seul coup\end{définition}
\begin{définition}\cmn 形容一下子被掀开的样子\end{définition}
\begin{exemple}\jya qale ɲɯ-ɤsɯ-βzu tɕe, a-tʂɯmpa hwɤl ʑo ta-pɣaʁ\cmn 风把我的围腰帕一下子吹开了\end{exemple}
\end{entrée}

\begin{entrée}
\vedette{\hypertarget{Ⓔhwɤrhwɤr}{\papi{ hwɤrhwɤr}}}\markboth{hwɤrhwɤr}{}\classe{idph.2}
\begin{définition}\fra évasé\end{définition}
\begin{définition}\cmn 形容口朝外展开的样子
\end{définition}
\begin{exemple}\jya ki a-rte ki hwɤrhwɤr ʑo ɲɯ-pa\cmn 我的帽子的口朝外展开\end{exemple}
\begin{relation-sémantique}\synonyme{
\hyperlink{Ⓔwɤrwɤr}{\textit{ \papi{wɤrwɤr}}}
}\end{relation-sémantique}\end{entrée}

\newpage\caractère{j}

\begin{entrée}
\vedette{\hypertarget{ⒺjaⒽ1}{\papi{ ja}}}\markboth{ja}{}\homonyme{1}\classe{vt}\acception{1}
\paradigme{\textit{dir :} \jya kɤ-}
\begin{définition}\fra enfermer\end{définition}
\begin{définition}\cmn 关\end{définition}
\begin{exemple}\jya nɯŋa kɤ-je\cmn 你把牛关起来吧\end{exemple}
\begin{exemple}\jya paʁ kɤ-je\cmn 你把猪关起来吧\end{exemple}
\begin{exemple}\jya qaʑo kɤ-ja-t-a\cmn 我把羊关起来了\end{exemple}
\begin{relation-sémantique}\confer{
\hyperlink{Ⓔnɤrkɤja}{\textit{ \papi{nɤrkɤja}}}
}\end{relation-sémantique}\acception{2}
\paradigme{\textit{dir :} \jya lɤ-}
\begin{définition}\fra amasser les grains\end{définition}
\begin{définition}\cmn 把粮食堆在仓库里\end{définition}
\begin{exemple}\jya tɤɕi lɤ-ja-t-a\cmn 我把青稞堆在仓库里了\end{exemple}\begin{sous-entrée}
\vedette{\hypertarget{}{\papi{ aja}}}\markboth{aja}{}\classe{vi}
\begin{définition}\ 
\begin{déclaration}\grammar{pass}\end{déclaration}\end{définition}
\begin{définition}\fra être enfermé\end{définition}
\begin{définition}\cmn 被关(牲畜)\end{définition}
\begin{exemple}\jya nɯtɕu aja\cmn 被关在那里\end{exemple}
\end{sous-entrée}\end{entrée}

\begin{entrée}
\vedette{\hypertarget{ⒺjaⒽ2}{\papi{ ja}}}\markboth{ja}{}\homonyme{2}
\classe{vs}
\paradigme{\textit{dir :} \jya kɤ-}
\begin{définition}\fra se faner\end{définition}
\begin{définition}\cmn 凋谢\end{définition}
\begin{exemple}\jya ɯ-mɯntoʁ ko-ja\cmn 花凋谢了\end{exemple}
\begin{relation-sémantique}\synonyme{
 \papi{rŋil}
}\end{relation-sémantique}\end{entrée}

\begin{entrée}
\vedette{\hypertarget{Ⓔjaftɕɯn}{\papi{ jaftɕɯn}}}\markboth{jaftɕɯn}{}
\classe{n}
\begin{définition}\fra étriers\end{définition}
\begin{définition}\cmn 马镫
\begin{déclaration} \étymologie{\papi{job.tɕan}}\end{déclaration}\end{définition}\end{entrée}

\begin{entrée}
\vedette{\hypertarget{Ⓔjaftɕɯnkha}{\papi{ jaftɕɯnkha}}}\markboth{jaftɕɯnkha}{}\classe{n}
\begin{définition}\fra cheville\end{définition}
\begin{définition}\cmn 踝关节\end{définition}
\end{entrée}

\begin{entrée}
\vedette{\hypertarget{Ⓔjamar}{\papi{ jamar}}}\markboth{jamar}{}\classe{adv}\acception{1}
\begin{définition}\fra environ\end{définition}
\begin{définition}\cmn 左右\end{définition}\acception{2}
\begin{définition}\fra à ce moment\end{définition}
\begin{définition}\cmn 在那个时候
\begin{déclaration} \étymologie{\papi{jar.mar}}\end{déclaration}\end{définition}
\end{entrée}

\begin{entrée}
\vedette{\hypertarget{Ⓔjaŋjaŋ}{\papi{ jaŋjaŋ}}}\markboth{jaŋjaŋ}{}
\classe{idph.2}\acception{1}
\begin{définition}\fra calme\end{définition}
\begin{définition}\cmn 安静\end{définition}\acception{2}
\begin{définition}\fra complètement plat\end{définition}
\begin{définition}\cmn 又平坦又宽阔(田地、草地)\end{définition}
\begin{exemple}\jya jisŋi kɯ-ɣɤɕqali ri maŋe tɕe, jaŋjaŋ ʑo ɲɯ-pa\cmn 今天没有人大声说话,很安静\end{exemple}
\begin{exemple}\jya tɯji jaŋjaŋ ʑo ɲɯ-pa, ɲɯ-jom\cmn 田地很宽阔\end{exemple}\end{entrée}

\begin{entrée}
\vedette{\hypertarget{Ⓔjaŋntsɤrpa}{\papi{ jaŋntsɤrpa}}}\markboth{jaŋntsɤrpa}{}\classe{n}
\begin{définition}\fra hache que l'on peut tenir d'une seule main\end{définition}
\begin{définition}\cmn 单手斧头\end{définition}
\end{entrée}

\begin{entrée}
\vedette{\hypertarget{Ⓔjaŋpho}{\papi{ jaŋpho}}}\markboth{jaŋpho}{}\classe{n}
\begin{définition}\fra fusil\end{définition}
\begin{définition}\cmn 枪
\begin{déclaration} \étymologie{\papi{\stylefn{洋炮}}}\end{déclaration}\end{définition}
\end{entrée}

\begin{entrée}
\vedette{\hypertarget{Ⓔjapa}{\papi{ japa}}}\markboth{japa}{}
\classe{adv}
\begin{définition}\fra l'année dernière\end{définition}
\begin{définition}\cmn 去年\end{définition}
\begin{exemple}\jya japa to-mɯɕtaʁ tsa ma ɣɯjpa mɯ́j-mɯɕtaʁ\end{exemple}\end{entrée}

\begin{entrée}
\vedette{\hypertarget{Ⓔjapandʐi}{\papi{ japandʐi}}}\markboth{japandʐi}{}\classe{n}
\begin{définition}\fra l'année d'avant\end{définition}
\begin{définition}\cmn 前年\end{définition}
\end{entrée}

\begin{entrée}
\vedette{\hypertarget{Ⓔjaqhɤrŋgɤβ,lɤt}{\papi{ jaqhɤrŋgɤβ,lɤt}}}\markboth{jaqhɤrŋgɤβ,lɤt}{}\classe{n}
\paradigme{\textit{dir :} \jya nɯ-}
\begin{définition}\fra attacher les mains derrière le dos\end{définition}
\begin{définition}\cmn 把手捆在背后\end{définition}
\begin{exemple}\jya aʑo jaqhɤrŋgɤβ nɯ́-wɣ-lat-a-nɯ\cmn 他们把我的手捆在背后了\end{exemple}
\begin{exemple}\jya a-jaqhɤrŋgɤβ na-lɤt-nɯ\cmn 他们把我的手捆在背后了\end{exemple}
\begin{relation-sémantique}\ComponentA{\classe{n}
 \papi{jaqhɤrŋgɤβ}
}\end{relation-sémantique}
\begin{relation-sémantique}\ComponentB{\classe{vt}
\hyperlink{ⒺlɤtⒽ1}{\textit{ \papi{lɤt}}}
}\end{relation-sémantique}
\begin{relation-sémantique}\confer{
\hyperlink{ⒺlɤtⒽ1}{\textit{ \papi{lɤt1}}}
}\end{relation-sémantique}\end{entrée}

\begin{entrée}
\vedette{\hypertarget{Ⓔjaramara}{\papi{ jaramara}}}\markboth{jaramara}{}\classe{n}
\begin{définition}\fra chercher à gagner de l'argent par tout les moyens\end{définition}
\begin{définition}\cmn 拉关系,到处想办法赚钱
\begin{déclaration} \étymologie{\papi{jar.mar}}\end{déclaration}\end{définition}
\begin{exemple}\jya jaramara to-βzu-j\cmn 我们拉了关系想办法到处赚钱\end{exemple}\end{entrée}

\begin{entrée}
\vedette{\hypertarget{Ⓔjaʁ}{\papi{ jaʁ}}}\markboth{jaʁ}{}
\classe{vs}
\paradigme{\textit{dir :} \jya tɤ-}
\begin{définition}\fra épais\end{définition}
\begin{définition}\cmn 厚\end{définition}
\begin{exemple}\jya tɤjpa ɲɯ-jaʁ\cmn 雪很厚\end{exemple}\begin{sous-entrée}
\vedette{\hypertarget{}{\papi{ ɣɤjaʁ}}}\markboth{ɣɤjaʁ}{}\classe{vt}
\paradigme{\textit{dir :} \jya tɤ-}
\begin{définition}\ 
\begin{déclaration}\grammar{caus}\end{déclaration}\end{définition}
\begin{exemple}\jya tɯ-ŋga tú-wɣ-ɣɤjaʁ ma ɲɯ-ɣɤndʐo\cmn 衣服要穿厚一些,天气很冷\end{exemple}
\begin{exemple}\jya a-ŋga tɤ-ɣɤjaʁ-a\cmn 我衣服穿厚一些\end{exemple}
\begin{relation-sémantique}\confer{
\hyperlink{ⒺnɤjaʁⒽ2}{\textit{ \papi{nɤjaʁ2}}}
}\end{relation-sémantique}
\begin{relation-sémantique}\antonyme{
\hyperlink{Ⓔmba}{\textit{ \papi{mba}}}
}\end{relation-sémantique}
\end{sous-entrée}\end{entrée}

\begin{entrée}
\vedette{\hypertarget{Ⓔjaʁjɯ}{\papi{ jaʁjɯ}}}\markboth{jaʁjɯ}{}
\classe{vs}
\paradigme{\textit{dir :} \jya tɤ-}
\paradigme{\textit{dir :} \jya thɯ-}
\begin{définition}\fra épais et résistant\end{définition}
\begin{définition}\cmn 厚实\end{définition}
\begin{exemple}\jya ŋgɤjpan kɯ-jaʁjɯ ci ɲɯ-ŋu\cmn 这是厚实的木板\end{exemple}
\begin{relation-sémantique}\confer{
\hyperlink{Ⓔjaʁ}{\textit{ \papi{jaʁ}}}
}\end{relation-sémantique}\end{entrée}

\begin{entrée}
\vedette{\hypertarget{Ⓔjaʁlu}{\papi{ jaʁlu}}}\markboth{jaʁlu}{}
\classe{n}
\begin{définition}\fra manchot\end{définition}
\begin{définition}\cmn 独臂的人\end{définition}\end{entrée}

\begin{entrée}
\vedette{\hypertarget{Ⓔjaʁmɤzdoʁzdoʁ}{\papi{ jaʁmɤzdoʁzdoʁ}}}\markboth{jaʁmɤzdoʁzdoʁ}{}\classe{n}
\begin{définition}\fra espèce d'oiseau\end{définition}
\begin{définition}\cmn 一种鸟\end{définition}
\begin{exemple}\jya jarmɤzdoʁzdoʁ nɯ pɣɤtɕɯ kɯ-xtɕɯ-xtɕi ci ŋu, ɯ-mdoʁ pɣi, ɯ-ʁar ɯ-rkɯ ra hanɯni ɲaʁ, ɲɯ-nɯqambɯmbjom tɕe wuma ʑo ɣɤji, ɯ-mtsioʁ kɯnɤ kɯ-xtɯ-xtɯt ma me, qajɯ tu-ndze ŋu, ɯʑo sti ma kɤ-mto me. ɯ-mi nɯ ra kɯ-pɣi ɯ-ŋgɯz kɯ-qarŋɯrŋe tsa ŋu.\cmn 
\stylefv{jarmɤzdoʁzdoʁ}是很小的鸟,灰色,翅膀边缘略黑。起飞非常快,嘴很短,吃虫子。只能看见它单独飞行。脚也是黑里带黄的。
\end{exemple}\end{entrée}

\begin{entrée}
\vedette{\hypertarget{Ⓔjaʁmba}{\papi{ jaʁmba}}}\markboth{jaʁmba}{}\classe{n}
\begin{définition}\fra épaisseur\end{définition}
\begin{définition}\cmn 厚薄;厚度\end{définition}
\begin{relation-sémantique}\confer{
\hyperlink{Ⓔjaʁ}{\textit{ \papi{jaʁ}}}
}\end{relation-sémantique}
\begin{relation-sémantique}\confer{
\hyperlink{Ⓔmba}{\textit{ \papi{mba}}}
}\end{relation-sémantique}\end{entrée}

\begin{entrée}
\vedette{\hypertarget{Ⓔjasa,ta}{\papi{ jasa,ta}}}\markboth{jasa,ta}{}\classe{vt}
\paradigme{\textit{dir :} \jya tɤ-}
\begin{définition}\fra respect\end{définition}
\begin{définition}\cmn 尊重人\end{définition}
\begin{exemple}\jya tɯrme kɯ-wxti ra jasa tú-wɣ-ta ra\cmn 要尊重长辈\end{exemple}
\begin{relation-sémantique}\synonyme{
\hyperlink{Ⓔzɣɤʁre}{\textit{ \papi{zɣɤʁre}}}
}\end{relation-sémantique}
\begin{relation-sémantique}\ComponentA{\classe{n}
 \papi{jasa}
}\end{relation-sémantique}
\begin{relation-sémantique}\ComponentB{\classe{vt}
\hyperlink{Ⓔta}{\textit{ \papi{ta}}}
}\end{relation-sémantique}\end{entrée}

\begin{entrée}
\vedette{\hypertarget{Ⓔjaχpɤtar}{\papi{ jaχpɤtar}}}\markboth{jaχpɤtar}{}
\classe{n}
\begin{définition}\fra frapper dans mains, applaudir\end{définition}
\begin{définition}\cmn 拍掌,鼓掌\end{définition}
\begin{exemple}\jya jaχpɤtar to-lɤt-nɯ\cmn 他们鼓掌了\end{exemple}
\begin{relation-sémantique}\confer{
\hyperlink{Ⓔtɤtar}{\textit{ \papi{tɤtar}}}
}\end{relation-sémantique}\end{entrée}

\begin{entrée}
\vedette{\hypertarget{Ⓔjɤβjɤβ}{\papi{ jɤβjɤβ}}}\markboth{jɤβjɤβ}{}\classe{idph.2}
\begin{définition}\fra faible lueur (au lever du jour)\end{définition}
\begin{définition}\cmn 蒙蒙亮\end{définition}
\begin{exemple}\jya soz jɤβjɤβ ʑo lɤ-pa tɕe aʑo tɤ-ndzu-a\cmn 今天天蒙蒙亮的时候我就出发了\end{exemple}\end{entrée}

\begin{entrée}
\vedette{\hypertarget{Ⓔjɤɣ}{\papi{ jɤɣ}}}\markboth{jɤɣ}{}
\classe{vs}\acception{1}
\paradigme{\textit{dir :} \jya pɯ-}
\paradigme{\textit{dir :} \jya tɤ-}
\begin{définition}\fra s'accomplir, se finir\end{définition}
\begin{définition}\cmn 完成\end{définition}
\begin{exemple}\jya ji-kha pɤjkhu mɯ-tɤ-jɤɣ\cmn 我们家在装修,还没有完成\end{exemple}
\begin{exemple}\jya aʑɯɣ pɯ-jɤɣ\cmn 我成功了\end{exemple}
\begin{exemple}\jya a-kɤ-nɤma pɯ-jɤɣ\cmn 我把工作做完了\end{exemple}
\begin{exemple}\jya ɯ-kɤ-nɤma jɤɣ mɤ-jɤɣ ʑo tɕe li ɲɤ-khrɤt\cmn 工作还没有做完又给他布置了(另外一个任务)\end{exemple}
\begin{exemple}\jya ɯ-rju jɤɣ mɤ-jɤɣ ʑo tɕe to-nɤla\cmn 话没有说完就答应了\end{exemple}
\begin{exemple}\jya ɯ-pa tɤ-jɤɣ tɕe mɤ-tɯ-mɤrʑaβ mɤ-khɯ\cmn 已经商量好了,你不能不嫁人\end{exemple}\acception{2}
\begin{définition}\fra être possible\end{définition}
\begin{définition}\cmn 可以\end{définition}
\begin{exemple}\jya tɤ-ti jɤɣ ma a-ʁa tu\cmn 你可以说,我有空\end{exemple}
\begin{relation-sémantique}\confer{
\hyperlink{Ⓔsɯɣjɤɣ}{\textit{ \papi{sɯɣjɤɣ}}}
}\end{relation-sémantique}\begin{sous-entrée}
\vedette{\hypertarget{}{\papi{ mɤ-kɯ-jɤɣ kɯ}}}\markboth{mɤ-kɯ-jɤɣ kɯ}{}\classe{adv}
\begin{définition}\fra non seulement\end{définition}
\begin{définition}\cmn 不但\end{définition}
\end{sous-entrée}\end{entrée}

\begin{entrée}
\vedette{\hypertarget{Ⓔjɤɣɤrna}{\papi{ jɤɣɤrna}}}\markboth{jɤɣɤrna}{}\classe{n}
\begin{définition}\fra partie du balcon où l'on fait sécher la nourriture\end{définition}
\begin{définition}\cmn 上走檐(晒粮食用的)\end{définition}\end{entrée}

\begin{entrée}
\vedette{\hypertarget{Ⓔjɤɣɤt}{\papi{ jɤɣɤt}}}\markboth{jɤɣɤt}{}
\classe{n}
\begin{définition}\fra construction suspendue au deuxième étage des maisons tibétaines, servant à faire sécher la nourriture\end{définition}
\begin{définition}\cmn 在藏式房屋二楼的位置上一个向外伸出的悬空部分,外围有木架围着。木架是由几根竖着的木杆和加在上面的许多横杆构成,另外再借着这些垂直穿插着一些细树枝。这个通风的架构可用来晾晒食物或其他物品。【走缘】\end{définition}
\begin{exemple}\jya jɤɣɤt jɤ-ari-a\cmn 我上了厕所\end{exemple}
\begin{exemple}\jya jɤɣɤt nɯ znde ɯ-taʁ tɯ-mɢɯt kú-wɣ-sɤtsa tɕe ɯ-taʁ nɯ tɕu romɲa chɯ́-wɣ-lɤt, romɲa nɯ tɯ-mɢɯt cho pjɯ́-wɣ-sɯ-ɤqɤtʂha tɕe pjɯ́-wɣ-ta ra. romɲa ɯ-taʁ nɯ tɕu χɕaʁ chɯ́-wɣ-ta, tɕe nɯ ɯ-taʁ tɕe tɤ-lmɯz chɯ́-wɣ-ta, nɯ ɯ-taʁ tɕe thɤlwa chɯ́-wɣ-lɤt tɕe tɯ-mɢɯt ɯ-ɕnɤz ɯ-taʁ nɯ tɕu, koʑi komɤl tɯ-ldʑa ka kú-wɣ-lɤt, koʑi komɤl cho romɲa ni pjɯ-ɤnɯpɕɯ-pɕoʁ ɲɯ-ra. nɯ ɯ-taʁ nɯ tɕu jɤɣɤt laχtsɯ pjɯ́-wɣ-tshoʁ tɕe, tɯ-mɢɯt tɯ-ldza ɯ-taʁ nɯ tɕu jɤɣɤt laχtsɯ nɯ tɯ-ldʑa pjɯ-tu ra, nɯ ɯ-taʁ nɯ tɕu li koʑi komɤl pjɯ́-wɣ-tshoʁ, nɯ ɯ-taʁ nɯ tɕu, tɤsthoʁsi chɯ́-wɣ-lɤt tɕe tɤsthoʁsi nɯ tɯ-mɢɯt tú-wɣ-z-nɯjɯn tɕe pjɯ́-wɣ-ta, tɯ-ldʑa pjɯ-tu khɯ, ʁnɯ-ldʑa pɯ-tu kɯnɤ khɯ, tɕe jɤɣɤt laχtsɯ raŋri ɯ-taʁ chɯ́-wɣ-ta ra. nɯ ɯ-taʁ nɯ tɕu li romɲa tɤsthoʁsi cho pjɯ́-wɣ-sɯ-ɤqɤtʂha tɕe pjɯ́-wɣ-ta ra, nɯ ɯ-taʁ nɯ tɕu cupa chɯ́-wɣ-ta, tɕe thɤlwa chɯ́-wɣ-lɤt, tɕe pjɯ́-wɣ-ɣnda, tɕe khɤxtu ɲɯ-βze ŋu. tɕe jɤɣɤt laχtsɯ ɣɯ ɯ-kɯ-spoʁ tu tɕe, nɯ tɕu rorʁe ɲɯ́-wɣ-rʁe tɕe, ftɕar tɕe tɯ-ɣro tú-wɣ-sɤro ŋu, stonka tɕe tɤ-rɤku tú-wɣ-sɤro ŋu tɕe rasti kɯnɤ nɯ tɕu tú-wɣ-sɯɣrom khɯ. tɕe jɤɣɤt nɯ ŋkhorwapa ra ɣɯ nɯ-ɲɤm wuma phɤn.\cmn 
在外墙上斜插小木梁,在上面交叉着放上横梁(作为走缘的地面)。在(横梁)上面铺一层劈好了的木料,再铺上麦草、豌豆草或者枝桠,然后铺上泥土。然后在小木梁的一头放上\stylefv{koʑi}和\stylefv{komɤl}各一条,\stylefv{koʑi} \stylefv{komɤl}和\stylefv{romɲa}朝同一个方向。上面插上柱子,每一根\stylefv{tɯ-mɢɯt}上面要有一根柱子。上面又插上\stylefv{koʑi komɤl},在上面就放上\stylefv{tɤsthoʁsi},\stylefv{tɤ-sthoʁsi}要顺着\stylefv{tɯ-mɢɯt}的方向装上去,可以是一根一根的,也可以是两根两根的,每根柱子上都要放。在那上面还要交叉着放\stylefv{romɲa},上面铺上石板,再铺上泥土,然后把泥土锤紧,也就成了房背。柱子上面有洞,在那里穿上横干,春天可以架饲草,秋天可以架粮食和圆根苗。走缘对农民用处很大。
\end{exemple}
\begin{relation-sémantique}\confer{
\hyperlink{Ⓔrɯjɤɣɤt}{\textit{ \papi{rɯjɤɣɤt}}}
}\end{relation-sémantique}\end{entrée}

\begin{entrée}
\vedette{\hypertarget{Ⓔjɤlwa}{\papi{ jɤlwa}}}\markboth{jɤlwa}{}\classe{n}
\begin{définition}\fra rideau\end{définition}
\begin{définition}\cmn 帘
\begin{déclaration} \étymologie{\papi{jol.ba}}\end{déclaration}\end{définition}
\end{entrée}

\begin{entrée}
\vedette{\hypertarget{Ⓔjɤmtsa}{\papi{ jɤmtsa}}}\markboth{jɤmtsa}{}
\classe{n}
\begin{définition}\fra casserole en fer\end{définition}
\begin{définition}\cmn 炒菜锅(用生铁制成的)\end{définition}
\begin{exemple}\jya jɤmtsa pjɤ-ɲɟɤβ\cmn 炒菜锅凹进去了\end{exemple}\end{entrée}

\begin{entrée}
\vedette{\hypertarget{Ⓔjɤnlaʁ}{\papi{ jɤnlaʁ}}}\markboth{jɤnlaʁ}{}\classe{n}
\begin{définition}\fra nom commun aux balcons et aux terrasses des maisons tibétaines\end{définition}
\begin{définition}\cmn 房背、走缘的通称
\begin{déclaration} \étymologie{\papi{jan.lag}}\end{déclaration}\end{définition}
\begin{exemple}\jya kha ɣɯ jɤɣɤt cho khɤxtu nɯ ra jɤnlaʁ rmi\cmn 
房子的走缘和房背都叫\stylefv{jɤnlaʁ}
\end{exemple}
\end{entrée}

\begin{entrée}
\vedette{\hypertarget{Ⓔjɤntɤn}{\papi{ jɤntɤn}}}\markboth{jɤntɤn}{}
\classe{n}
\begin{définition}\fra savoir\end{définition}
\begin{définition}\cmn 技术;知识
\begin{déclaration} \étymologie{\papi{jon.tan}}\end{déclaration}\end{définition}
\begin{exemple}\jya nɤ-jɤntɤn, @qiche kɤ-lɤt tu\cmn 你的技术是开汽车\end{exemple}
\begin{relation-sémantique}\confer{
\hyperlink{Ⓔnɯjɤntɤn}{\textit{ \papi{nɯjɤntɤn}}}
}\end{relation-sémantique}\end{entrée}

\begin{entrée}
\vedette{\hypertarget{Ⓔjɤŋkhɤphɯt}{\papi{ jɤŋkhɤphɯt}}}\markboth{jɤŋkhɤphɯt}{}\classe{n}
\begin{définition}\fra coup avec le dos de la main\end{définition}
\begin{définition}\cmn 用手背打\end{définition}
\begin{exemple}\jya ɯ-taʁ jɤŋkhɤphɯt ci tɤ-lat-a\cmn 我用手背打了一下他\end{exemple}\end{entrée}

\begin{entrée}
\vedette{\hypertarget{Ⓔjɤrjɤr}{\papi{ jɤrjɤr}}}\markboth{jɤrjɤr}{}\classe{idph.2}
\begin{définition}\fra allure de qqn portant une lourde charge\end{définition}
\begin{définition}\cmn 形容背着很重的东西,很难受的样子;歪歪斜斜的样子\end{définition}
\begin{exemple}\jya jɤrjɤr ʑo ɲɯ-ɤsɯ-ndo\cmn 他拿着很重的东西\end{exemple}
\begin{exemple}\jya jɤrjɤr ʑo ɲɯ-rɤʑi\cmn 他背着很重的东西站在那里\end{exemple}
\begin{exemple}\jya jiɕqha tɕheme ɲɤ-sɤfɕi ɯ-xtu jɤrjɤr ɲɤ-pa\cmn 那个女怀孕了,肚子显得很重\end{exemple}
\begin{exemple}\jya jɤrjɤr ʑo ma-tɯ-ʑɣɤstu\cmn 你别做出这么一副笨拙的样子!\end{exemple}
\begin{exemple}\jya ɯ-fkur ɲɯ-nɤrʑi tɕe, jɤrjɤr ʑo tɤ-ʑɣɤstu\cmn 他觉得背的东西很重,显得很难受\end{exemple}\begin{sous-entrée}
\vedette{\hypertarget{}{\papi{ ɣɤjɤrjɤr}}}\markboth{ɣɤjɤrjɤr}{}\classe{vi}
\begin{définition}\fra chancelant, instable (lorsque l'on porte une lourde charge)\end{définition}
\begin{définition}\cmn 形容摇摇晃晃的样子(因为背着很重的东西)\end{définition}
\end{sous-entrée}\begin{sous-entrée}
\vedette{\hypertarget{}{\papi{ jɤrnɤjɤr}}}\markboth{jɤrnɤjɤr}{}\classe{idph.3}
\begin{exemple}\jya jɤrjɤr nɤ jɤrjɤr ka-tsɯm\cmn 他带走了(很重的东西)\end{exemple}
\end{sous-entrée}\begin{sous-entrée}
\vedette{\hypertarget{}{\papi{ nɤjɤrjɤr}}}\markboth{nɤjɤrjɤr}{}\classe{vt}
\paradigme{\textit{dir :} \jya nɯ-}
\begin{définition}\fra porter une lourde charge (à plusieurs)\end{définition}
\begin{définition}\cmn (几个人)抬很重的东西\end{définition}
\begin{exemple}\jya kɯki lʁa ki ɯ-ŋgɯ ɲɯ-mtshɤt tɕe, nɯ-nɤjɤrjɤr-tɕi tɕe, nɯ-sɤzɣɯt-tɕi\cmn 这个袋子满了,我们俩把它抬回来了\end{exemple}
\end{sous-entrée}\begin{sous-entrée}
\vedette{\hypertarget{}{\papi{ sɤjɤrjɤr}}}\markboth{sɤjɤrjɤr}{}\classe{vt}
\begin{exemple}\jya ɯ-fkur ɲɯ-nɤrʑi tɕe ɲɯ-sɤjɤrjɤr\cmn 背的东西很重,显得路都走不稳\end{exemple}
\end{sous-entrée}\end{entrée}

\begin{entrée}
\vedette{\hypertarget{Ⓔjɤxtshi}{\papi{ jɤxtshi}}}\markboth{jɤxtshi}{}
\classe{adv}
\begin{définition}\fra cette fois-ci\end{définition}
\begin{définition}\cmn 这一次\end{définition}\end{entrée}

\begin{entrée}
\vedette{\hypertarget{Ⓔjɤznɤ}{\papi{ jɤznɤ}}}\markboth{jɤznɤ}{}\classe{post}
\begin{définition}\fra à partir de, pendant\end{définition}
\begin{définition}\cmn 从……一直到;在……的时候\end{définition}
\begin{exemple}\jya andi kɤ-ɣe-a jɤznɤ tɯ-mɯ pɯ-a<nɯ>sɯ-lɤt ɕti\cmn 我从那边来到这里一直都在下雨\end{exemple}
\begin{exemple}\jya jɯfɕɯr jɤznɤ tɯ-mɯ pɯ-asɯ-lɤt tɕe tham to-stat\cmn 昨天开始下雨,现在就停了\end{exemple}
\begin{exemple}\jya aʑo @zhongguo ju-ɣi-a ɕɯŋgɯ jɤznɤ pɯ-a<nɯ>ɕqhe-a ɕti\cmn 我来中国之前已经在咳嗽\end{exemple}
\begin{exemple}\jya tɯ-kɯ-mŋɤm tu-ʑe ɕɯŋgɯ jɤznɤ tú-wɣ-z-nɯsmɤn ra\cmn 在开始生病之前就要治疗\end{exemple}\end{entrée}

\begin{entrée}
\vedette{\hypertarget{Ⓔjɤzɯlu}{\papi{ jɤzɯlu}}}\markboth{jɤzɯlu}{}
\classe{n}
\begin{définition}\fra année du lapin\end{définition}
\begin{définition}\cmn 兔年
\begin{déclaration} \étymologie{\papi{yos.lo}}\end{déclaration}\end{définition}\end{entrée}

\begin{entrée}
\vedette{\hypertarget{Ⓔjɣɤt}{\papi{ jɣɤt}}}\markboth{jɣɤt}{}
\classe{vi}
\paradigme{\textit{dir :} \jya \_}
\begin{définition}\fra revenir\end{définition}
\begin{définition}\cmn 转回
\begin{déclaration}\use{专门去(办事)之后转回去或转回来}\end{déclaration}\end{définition}
\begin{exemple}\jya jiɕqha nɯ-ɣe ri li kɤ-jɣɤt\cmn 他来了,又回去了\end{exemple}
\begin{exemple}\jya jiɕqha nɯ-ɣe-a tɕe li kɤ-jɣat-a\cmn 我来了又回去了\end{exemple}
\begin{exemple}\jya tɕiʑo kɯ-nɯsaχsɯ kɤ-ari-tɕi tɕe, li nɯ-nɯ-jɣɤt-tɕi\cmn 我们去吃中午餐,又回来了\end{exemple}\begin{sous-entrée}
\vedette{\hypertarget{}{\papi{ sɯjɣɤt}}}\markboth{sɯjɣɤt}{}
\paradigme{\textit{dir :} \jya \_}
\begin{définition}\ 
\begin{déclaration}\grammar{caus}\end{déclaration}\end{définition}
\begin{définition}\fra faire revenir\end{définition}
\begin{définition}\cmn 令……回来\end{définition}
\begin{exemple}\jya ɯʑo lɤ-ari ri, thɯ-sɯjɣat-a\cmn 虽然他往上游去了,但是我让他转回来了\end{exemple}\classe{vt}
\end{sous-entrée}\end{entrée}

\begin{entrée}
\vedette{\hypertarget{Ⓔji}{\papi{ ji}}}\markboth{ji}{}
\classe{vt}
\paradigme{\textit{dir :} \jya lɤ-}
\paradigme{\textit{dir :} \jya pɯ-}
\begin{définition}\fra planter\end{définition}
\begin{définition}\cmn 种(菜、植物等)、播种\end{définition}
\begin{exemple}\jya jiɕqha jaŋjy nɯ lɤ-ji-t-a\cmn 我种了土豆\end{exemple}
\begin{exemple}\jya si tɯβli nɯ ɕ-pɯ-ji-t-a\cmn 我种了树苗\end{exemple}
\begin{exemple}\jya tɯrgi tɯβli ɕ-pɯ-ji-t-a\cmn 我种了杉树苗\end{exemple}
\begin{relation-sémantique}\confer{
\hyperlink{Ⓔrɤji}{\textit{ \papi{rɤji}}}
}\end{relation-sémantique}\end{entrée}

\begin{entrée}
\vedette{\hypertarget{Ⓔjiɕqha}{\papi{ jiɕqha}}}\markboth{jiɕqha}{}\classe{adv}
\begin{définition}\fra à l'instant\end{définition}
\begin{définition}\cmn 刚才\end{définition}\end{entrée}

\begin{entrée}
\vedette{\hypertarget{Ⓔjima}{\papi{ jima}}}\markboth{jima}{}\classe{n}
\begin{définition}\fra maïs\end{définition}
\begin{définition}\cmn 玉米
\begin{déclaration} \étymologie{\papi{\stylefn{御麦}}}\end{déclaration}\end{définition}
\end{entrée}

\begin{entrée}
\vedette{\hypertarget{Ⓔjinɤji}{\papi{ jinɤji}}}\markboth{jinɤji}{}\classe{idph.3}
\begin{définition}\fra son de gémissement\end{définition}
\begin{définition}\cmn 形容不停地呻吟的样子\end{définition}
\begin{exemple}\jya ɲɯ-nɤmŋɤm tɕe jinɤji ʑo ɲɯ-ti\cmn 他很痛,不停地呻吟\end{exemple}\end{entrée}

\begin{entrée}
\vedette{\hypertarget{Ⓔjinde}{\papi{ jinde}}}\markboth{jinde}{}\classe{n}
\begin{définition}\fra maintenant\end{définition}
\begin{définition}\cmn 现在\end{définition}
\end{entrée}

\begin{entrée}
\vedette{\hypertarget{Ⓔjisŋi}{\papi{ jisŋi}}}\markboth{jisŋi}{}\classe{n}
\begin{définition}\fra aujourd'hui\end{définition}
\begin{définition}\cmn 今天\end{définition}\end{entrée}

\begin{entrée}
\vedette{\hypertarget{Ⓔjit}{\papi{ jit}}}\markboth{jit}{}
\classe{vi}
\paradigme{\textit{dir :} \jya pɯ-}
\begin{définition}\fra couler\end{définition}
\begin{définition}\cmn 流出来;溢出来;倒出来;洒(水)\end{définition}
\begin{exemple}\jya tɯ-rŋa sɤ-mar nɯ pjɤ-jit\cmn 擦脸的(液体)溢出来了\end{exemple}
\begin{exemple}\jya jiɕqha phoŋ ɯ-ŋgɯ cha nɯ pjɤ-jit\cmn 酒从这个瓶子溢出来了\end{exemple}
\begin{exemple}\jya phoŋ ɯ-ŋgɯ tɤ-lu nɯ pjɤ-jit\cmn 牛奶从瓶子溢出来了\end{exemple}
\begin{exemple}\jya jiɕqha a-tʂha kɯre pɯ-ata ri pjɤ-jit\cmn 茶刚才放在这里,不小心倒了\end{exemple}\end{entrée}

\begin{entrée}
\vedette{\hypertarget{Ⓔjiʑo}{\papi{ jiʑo}}}\markboth{jiʑo}{}\classe{pro}
\begin{définition}\fra nous\end{définition}
\begin{définition}\cmn 我们\end{définition}
\end{entrée}

\begin{entrée}
\vedette{\hypertarget{Ⓔjkrɯt}{\papi{ jkrɯt}}}\markboth{jkrɯt}{}
\classe{vi}
\paradigme{\textit{dir :} \jya kɤ-}
\begin{définition}\fra se solidifier\end{définition}
\begin{définition}\cmn 凝固\end{définition}
\begin{exemple}\jya tɯkri nɯ-sɤla-t-a tɕe ɲo-mɯɕtaʁ tɕe ko-jkrɯt\cmn 我烧了油,冷了就凝固了\end{exemple}\end{entrée}

\begin{entrée}
\vedette{\hypertarget{Ⓔjla}{\papi{ jla}}}\markboth{jla}{}
\classe{n}
\begin{définition}\fra hybride de yak et de vache\end{définition}
\begin{définition}\cmn 犏牛\end{définition}
\begin{exemple}\jya jla nɯ fsapaʁ kɯ-wxti ci ŋu, ɯ-rme jndʐɤz, khro mɤ-rɲɟi, jla ɯ-ɲɤm a-pɯ-pe tɕe ɯ-rme nɤmbju, ɯ-ʁrɯ jpum tsa rɲɟi, ʑɯrɯʑɤri lu-omtɕoʁ ŋu, ɯ-jme ɣɯ ɯ-rme nɯ khro mɤ-dɤn, rɲɟi, ɯ-ʁrɯ ɯ-rca pa tsa ri ɯ-rna ku-ndzoʁ ŋu. nɯ ɯ-pa ri ɯ-mɲaʁ ŋu, jla ɯ-ku wxti, ɯ-mtɕi jpum, ɯ-ɕna ɲɯ́-wɣ-sɯ-spoʁ tɕe, nɯ tɕu ɯ-si tɤ-loʁ ɲɯ́-wɣ-βzu tɕe, tɕe nɯ ɲɯ́-wɣ-rʁe, tɕe nɯ tɕu tɯ-mbri kú-wɣ-tshoʁ tɕe nɯ ɕnɤri ŋu tɕe, ɣɯ-mtshi tɤ-ra tɕe, tɯ-mbri nɯ kú-wɣ-ndo tɕe jla nɯ tɯ-qhu ju-ɣi ŋu, tɕe jla ju-ɕe mɯ-tɤ-jɤɣ tɕe nɯ tɯ-mbri nɯ cischiz kú-wɣ-βraʁ tɕe jla ku-nɯ-rɤʑi ŋu. jla ju-nɯɕɯ-ce tɤ-jɤɣ tɕe, nɯ-ɕnɤri nɯ jla ɯ-ʁrɯ ɯ-taʁ tú-wɣ-rtɤβ tɕe tɤ-mtɯ tú-wɣ-lɤt jla ju-nɯɕe ɕti. jla nɯ kɯ-rɯŋkhorwa ra ɣɯ kɤ-rɤji cho kɤ-ɕlu pjɯ-me mɤ-kɯ-khɯ ʑo ɕti ma tɯ-ɕlu ɯ-kɯ-rɤɕi nɯ ɕti. nɯ a-pɯ-maʁ tɕe, tɤ-rɤku kɤ-ji mɤ-khɯ. tɕe jla sna mɤ-sna cho taŋ mɤ-taŋ nɯ ɯ-jme ɯ-rme ɯ-ʁrɯ ɯ-qɯ-qataʁrɯ ɯ-mtɕhi nɯ ra ku-χpjɤt ŋu. jla nɯ sɯjno ndze, tɯ-jpu ndze, tɯ-kri tshi, tɕeri li ɯ-ŋgo tu tɕe pjɯ-nɯtɕhɯtɯɣ ŋgrɤl, pjɯ-nɯrtsatɯɣ ŋgrɤl. jla kɯmɯxte kɯ-ɲaʁ ŋu tɕeri li kɯ-wɣrum kɯ-ɣɯrni kɯ-pɣi nɯ ra tu, ʑakastaka nɯ-rmi tu : raχtɕoŋ, sɯli, mdzukɤr, mdzumɤr, mdzusŋun, kanaʁ, ɕasca, rɟamar, rɟandzi nɯ ra kɤ-ti tu-ŋgrɤl\cmn 犏牛是高大的牲畜,毛很粗,不长。如果膘好,毛就有光泽。脚又粗又长,逐渐变尖,尾巴上的毛不多、很长,角的下面紧接着就是耳朵,再下面就是眼睛。犏牛的头很大,嘴巴很粗。把鼻子打穿后,就可以穿上鼻圈,再系上一根绳子,那就是鼻索。需要牵走的时候,就拿着那根绳子,犏牛就跟在后面走,不要它走的时候,就把绳子随便拴在什么地方,它就会待在那里。要让它走时,就把绳子缠在角上,打上结,就会走了。犏牛是农民耕田播种不可缺少的,因为只有它能拉犁耕地,没有它,就不能种庄稼。判断犏牛品种的好坏与真假要观察尾巴、毛、角、嘴和蹄子。犏牛吃草,吃粮食,喝油,但是它也有一些病,如水中毒、草中毒等。大部分犏牛是黑色的,但是也有白色、红色和灰色的,各种都有自己的名称。\end{exemple}\end{entrée}

\begin{entrée}
\vedette{\hypertarget{Ⓔjlaʁar}{\papi{ jlaʁar}}}\markboth{jlaʁar}{}
\classe{n}
\begin{définition}\fra veau d'hybride de yak et de vache\end{définition}
\begin{définition}\cmn 犏牛犊\end{définition}\end{entrée}

\begin{entrée}
\vedette{\hypertarget{Ⓔjlɤβndʑu}{\papi{ jlɤβndʑu}}}\markboth{jlɤβndʑu}{}\classe{n}
\begin{définition}\fra bâton duquel on défile la trame\end{définition}
\begin{définition}\cmn 纬线的杆子\end{définition}
\begin{relation-sémantique}\confer{
\hyperlink{Ⓔtɯ-jlɤβ}{\textit{ \papi{tɯ-jlɤβ}}}
}\end{relation-sémantique}
\begin{relation-sémantique}\confer{
\hyperlink{Ⓔndʑu}{\textit{ \papi{ndʑu}}}
}\end{relation-sémantique}\end{entrée}

\begin{entrée}
\vedette{\hypertarget{Ⓔjlɤdo}{\papi{ jlɤdo}}}\markboth{jlɤdo}{}\classe{n}
\begin{définition}\fra vieux yak hybride\end{définition}
\begin{définition}\cmn 老犏牛\end{définition}
\begin{relation-sémantique}\confer{
\hyperlink{Ⓔjla}{\textit{ \papi{jla}}}
}\end{relation-sémantique}
\begin{relation-sémantique}\confer{
\hyperlink{Ⓔɯ-do}{\textit{ \papi{ɯ-do}}}
}\end{relation-sémantique}\end{entrée}

\begin{entrée}
\vedette{\hypertarget{Ⓔjlɤkrɯ}{\papi{ jlɤkrɯ}}}\markboth{jlɤkrɯ}{}
\classe{n}
\begin{définition}\fra hotte\end{définition}
\begin{définition}\cmn 背篼\end{définition}\end{entrée}

\begin{entrée}
\vedette{\hypertarget{Ⓔjlɤmtshi}{\papi{ jlɤmtshi}}}\markboth{jlɤmtshi}{}\classe{n}
\begin{définition}\fra action de mener un yak hybride\end{définition}
\begin{définition}\cmn 牵犏牛\end{définition}
\begin{relation-sémantique}\confer{
\hyperlink{Ⓔnɯjlɤmtshi}{\textit{ \papi{nɯjlɤmtshi}}}
}\end{relation-sémantique}\end{entrée}

\begin{entrée}
\vedette{\hypertarget{Ⓔjlɤndʐi}{\papi{ jlɤndʐi}}}\markboth{jlɤndʐi}{}\classe{n}
\begin{définition}\fra peau de yak hybride\end{définition}
\begin{définition}\cmn 犏牛皮子\end{définition}
\begin{relation-sémantique}\confer{
\hyperlink{Ⓔjla}{\textit{ \papi{jla}}}
}\end{relation-sémantique}
\begin{relation-sémantique}\confer{
\hyperlink{Ⓔtɯ-ndʐi}{\textit{ \papi{tɯ-ndʐi}}}
}\end{relation-sémantique}\end{entrée}

\begin{entrée}
\vedette{\hypertarget{Ⓔjlɤprɤm}{\papi{ jlɤprɤm}}}\markboth{jlɤprɤm}{}\classe{n}
\begin{définition}\fra poudre pour nourrir les yaks hybrides\end{définition}
\begin{définition}\cmn 喂犏牛的粉\end{définition}
\begin{relation-sémantique}\confer{
\hyperlink{Ⓔjla}{\textit{ \papi{jla}}}
}\end{relation-sémantique}
\begin{relation-sémantique}\confer{
\hyperlink{Ⓔtɤ-prɤm}{\textit{ \papi{tɤ-prɤm}}}
}\end{relation-sémantique}
\end{entrée}

\begin{entrée}
\vedette{\hypertarget{Ⓔjlɤtɯndzɯm}{\papi{ jlɤtɯndzɯm}}}\markboth{jlɤtɯndzɯm}{}\classe{n}
\begin{définition}\fra paire de yak hybrides (pour tirer la charrue)\end{définition}
\begin{définition}\cmn 一对犏牛(拖犁头的)\end{définition}
\begin{relation-sémantique}\confer{
\hyperlink{Ⓔjla}{\textit{ \papi{jla}}}
}\end{relation-sémantique}
\begin{relation-sémantique}\confer{
\hyperlink{Ⓔtɯ-tɯndzɯm}{\textit{ \papi{tɯ-tɯndzɯm}}}
}\end{relation-sémantique}\end{entrée}

\begin{entrée}
\vedette{\hypertarget{Ⓔjmɤɣni}{\papi{ jmɤɣni}}}\markboth{jmɤɣni}{}
\classe{n}
\begin{définition}\fra russule rouge\end{définition}
\begin{définition}\cmn 红菇【杉木菌】\end{définition}
\begin{exemple}\jya jmɤɣni nɯ tɯrgi ɯ-ŋgɯ tu-ɬoʁ ŋu, kɤ-ndza mɯm ɯ-mdoʁ nɯ kɯ-ɤɣɯrnɯɕɯr tsa ŋu. ftɕar tɕe tɯrgi kɯ-xtɕi ɯ-ŋgɯ ɯ-ŋgɯ ŋu stonka tɕe tɯrgi kɯ-wxti ɯ-ŋgɯ tu-ɬoʁ ŋu\cmn 杉木菌长在杉木林里,好吃,颜色带有点红色,夏天长在小杉木林里,秋天长在大杉木林里。\end{exemple}
\begin{relation-sémantique}\confer{
\hyperlink{Ⓔɣɯrni}{\textit{ \papi{ɣɯrni}}}
}\end{relation-sémantique}
\begin{relation-sémantique}\confer{
\hyperlink{Ⓔtɤjmɤɣ}{\textit{ \papi{tɤjmɤɣ}}}
}\end{relation-sémantique}\end{entrée}

\begin{entrée}
\vedette{\hypertarget{Ⓔjmɤjmɤr}{\papi{ jmɤjmɤr}}}\markboth{jmɤjmɤr}{}\classe{idph.2}
\begin{définition}\fra doux\end{définition}
\begin{définition}\cmn 形容柔软的样子\end{définition}\end{entrée}

\begin{entrée}
\vedette{\hypertarget{Ⓔjmɤlu}{\papi{ jmɤlu}}}\markboth{jmɤlu}{}
\classe{n}
\begin{définition}\fra sans queue\end{définition}
\begin{définition}\cmn 缺了尾巴的动物\end{définition}
\begin{relation-sémantique}\confer{
\hyperlink{Ⓔtɤ-jme}{\textit{ \papi{tɤ-jme}}}
}\end{relation-sémantique}\end{entrée}

\begin{entrée}
\vedette{\hypertarget{Ⓔjmɤrtaʁ}{\papi{ jmɤrtaʁ}}}\markboth{jmɤrtaʁ}{}
\begin{définition}\fra espèce d'insecte\end{définition}
\begin{définition}\cmn 虫子的一种\end{définition}
\begin{relation-sémantique}\confer{
\hyperlink{Ⓔtɤ-jme}{\textit{ \papi{tɤ-jme}}}
}\end{relation-sémantique}
\begin{relation-sémantique}\confer{
\hyperlink{Ⓔartaʁ}{\textit{ \papi{artaʁ}}}
}\end{relation-sémantique}
\end{entrée}

\begin{entrée}
\vedette{\hypertarget{Ⓔjmɤtɤsti}{\papi{ jmɤtɤsti}}}\markboth{jmɤtɤsti}{}\classe{n}
\begin{définition}\fra espèce de champignon\end{définition}
\begin{définition}\cmn 菌子的一种\end{définition}
\begin{relation-sémantique}\confer{
\hyperlink{Ⓔjmɤɣni}{\textit{ \papi{jmɤɣni}}}
}\end{relation-sémantique}\end{entrée}

\begin{entrée}
\vedette{\hypertarget{Ⓔjmɯt}{\papi{ jmɯt}}}\markboth{jmɯt}{}\classe{vt}
\paradigme{\textit{dir :} \jya nɯ-}
\begin{définition}\fra oublier\end{définition}
\begin{définition}\cmn 忘记\end{définition}
\begin{exemple}\jya ma-nɯ-tɯ-jmɯt\cmn 你不要忘记\end{exemple}
\begin{exemple}\jya ɲɤ-jmɯt\cmn 他忘记了\end{exemple}
\begin{exemple}\jya ɲo-nɯ-jmɯt\cmn 我忘了\end{exemple}
\begin{exemple}\jya ɲo-kɯ-jmɯt-a\cmn 你把我忘了\end{exemple}
\begin{exemple}\jya mɯ́j-tɯ-cha wo ma nɤ-ma ntsɯ, nɤ-koŋtso ntsɯ tu-tɯ-nɤme qhe, kɯmaʁ ra ɲɯ-tɯ-nɯ-jmɯt ɲɯ-ɕti\cmn 你不行吧,因为你只顾着你的工作,会忘记其它事情\end{exemple}
\begin{relation-sémantique}\antonyme{
\hyperlink{Ⓔɕɯftaʁ}{\textit{ \papi{ɕɯftaʁ}}}
}\end{relation-sémantique}\begin{sous-entrée}
\vedette{\hypertarget{}{\papi{ ɣɤjmɯt}}}\markboth{ɣɤjmɯt}{}\classe{vs}
\begin{définition}\ 
\begin{déclaration}\grammar{facil}\end{déclaration}\end{définition}
\begin{définition}\fra qui a une mauvaise mémoire\end{définition}
\begin{définition}\cmn 记性不好;溶剂忘记\end{définition}
\begin{exemple}\jya a-rɯxpa ɲɯ-khe tɕe ɲɯ-ɣɤjmɯt-a\cmn 我记性不好,容易忘记\end{exemple}
\begin{exemple}\jya a-rɯxpa pe tɕe mɤ-ɣɤjmɯt-a\cmn 我记性很好,不容易忘记\end{exemple}
\begin{exemple}\jya mɯ́j-ɣɤjmɯt\cmn 他不容易忘记\end{exemple}
\end{sous-entrée}\begin{sous-entrée}
\vedette{\hypertarget{}{\papi{ nɯɣɯjmɯt}}}\markboth{nɯɣɯjmɯt}{}\classe{vs}
\begin{définition}\fra facile à oublier\end{définition}
\begin{définition}\cmn 容易忘记(事情)\end{définition}
\begin{exemple}\jya ɯʑo kɯ ta-tɯt nɯ kɯ-nɯɣɯjmɯt ci ɲɯ-ŋu\cmn 他讲的话很容易忘记\end{exemple}
\begin{relation-sémantique}\antonyme{
\hyperlink{Ⓔnɯɣɯɕɯftaʁ}{\textit{ \papi{nɯɣɯɕɯftaʁ}}}
}\end{relation-sémantique}
\end{sous-entrée}\end{entrée}

\begin{entrée}
\vedette{\hypertarget{Ⓔjndʐɤz}{\papi{ jndʐɤz}}}\markboth{jndʐɤz}{}
\classe{vs}
\paradigme{\textit{dir :} \jya tɤ-}
\paradigme{\textit{dir :} \jya thɯ-}
\begin{définition}\fra épaisse (poudre)\end{définition}
\begin{définition}\cmn 粗(粉状)\end{définition}
\begin{exemple}\jya ʑɴɢɯloʁ ɲɯ-jndʐɤz\cmn 核桃很大\end{exemple}
\begin{exemple}\jya tɯsqar chɤ-jndʐɤz\cmn 糌粑磨得很粗\end{exemple}\end{entrée}

\begin{entrée}
\vedette{\hypertarget{Ⓔjndʐɯɣ}{\papi{ jndʐɯɣ}}}\markboth{jndʐɯɣ}{}\classe{vi}
\paradigme{\textit{dir :} \jya nɯ-}
\begin{définition}\fra ruminer\end{définition}
\begin{définition}\cmn 反刍\end{définition}
\begin{exemple}\jya nɯŋa ɲɯ-jndʐɯɣ\cmn 牛在反刍\end{exemple}\end{entrée}

\begin{entrée}
\vedette{\hypertarget{Ⓔjnom}{\papi{ jnom}}}\markboth{jnom}{}
\classe{vs}
\begin{définition}\fra flexible\end{définition}
\begin{définition}\cmn 柔软;有弹性\end{définition}
\begin{relation-sémantique}\confer{
\hyperlink{Ⓔldʑɯz}{\textit{ \papi{ldʑɯz}}}
}\end{relation-sémantique}\end{entrée}

\begin{entrée}
\vedette{\hypertarget{Ⓔjŋaʁjŋaʁ}{\papi{ jŋaʁjŋaʁ}}}\markboth{jŋaʁjŋaʁ}{}
\classe{idph.2}
\begin{définition}\fra au corps fin et élancé (femme)\end{définition}
\begin{définition}\cmn 形容身子苗条的样子(女子)\end{définition}
\begin{exemple}\jya iɕqha tɕheme nɯ ɯ-phoŋbu jŋaʁjŋaʁ ʑo ɲɯ-pa\cmn 那个女子身体很苗条\end{exemple}\end{entrée}

\begin{entrée}
\vedette{\hypertarget{Ⓔjom}{\papi{ jom}}}\markboth{jom}{}
\classe{vs}
\paradigme{\textit{dir :} \jya nɯ-}
\begin{définition}\fra large\end{définition}
\begin{définition}\cmn 宽;辽阔\end{définition}
\begin{exemple}\jya sɤtɕha ɲɯ-jom\cmn 地方很宽\end{exemple}
\begin{exemple}\jya ɯ-sɤ-ta ɲɯ-jom\cmn 放的地方很宽\end{exemple}
\begin{exemple}\jya kha ɲɯ-jom\cmn 屋子很宽\end{exemple}
\begin{exemple}\jya ɯ-ro jom (ɯ-sɯmrɟɤz)\cmn 他心胸宽阔\end{exemple}
\begin{exemple}\jya ɯ-mɲaʁsta jom\cmn 他心胸宽阔\end{exemple}
\begin{relation-sémantique}\confer{
\hyperlink{Ⓔɣɤjom}{\textit{ \papi{ɣɤjom}}}
}\end{relation-sémantique}\end{entrée}

\begin{entrée}
\vedette{\hypertarget{Ⓔjoʁ}{\papi{ joʁ}}}\markboth{joʁ}{}
\classe{vt}\acception{1}
\paradigme{\textit{dir :} \jya tɤ-}
\paradigme{\textit{dir :} \jya thɯ-}
\paradigme{\textit{dir :} \jya lɤ-}
\begin{définition}\fra lever\end{définition}
\begin{définition}\cmn 抬起;举起;放在上面\end{définition}
\begin{exemple}\jya kɯki laχtɕha tɤ-joʁ-a\cmn 我把这根东西放上去了\end{exemple}
\begin{exemple}\jya ɯʑo kɯ @luyinji pɯ-asɯ-ndo tɕe la-joʁ\cmn 他把手上拿着的录音机放在上面了\end{exemple}
\begin{exemple}\jya a-jaʁ tɤ-joʁ-a\cmn 我把手举起来了\end{exemple}
\begin{exemple}\jya ɯ-stɤt chɤ-joʁ\cmn 他起了半身\end{exemple}\acception{2}
\paradigme{\textit{dir :} \jya \_}
\begin{définition}\fra ranger, mettre de côté\end{définition}
\begin{définition}\cmn 收起来;收在一边\end{définition}
\begin{exemple}\jya nɤ-ŋga nɯ-tɕɤt tɕe ju-joʁ-a\cmn 你把衣服脱下来,我给你收起来\end{exemple}
\begin{exemple}\jya khɯtsa ra tɤ-wum tɕe lɤ-joʁ!\cmn 你把这些碗收起来放在(柜子里)\end{exemple}
\begin{relation-sémantique}\synonyme{
\hyperlink{Ⓔwum}{\textit{ \papi{wum}}}
}\end{relation-sémantique}\end{entrée}

\begin{entrée}
\vedette{\hypertarget{Ⓔjoʁβzɯr}{\papi{ joʁβzɯr}}}\markboth{joʁβzɯr}{}
\classe{n}
\begin{définition}\fra rangement\end{définition}
\begin{définition}\cmn 收拾\end{définition}
\begin{exemple}\jya joʁβzɯr tɤ-βzu-t-a\cmn 我收拾\end{exemple}
\begin{exemple}\jya kɯra laχtɕha ra ɲɯ-dɤn tɕe, joʁβzu kɤ-βzu ɲɯ-ra\cmn 这里东西很多,需要收拾\end{exemple}
\begin{relation-sémantique}\confer{
\hyperlink{Ⓔrɤjoʁβzɯr}{\textit{ \papi{rɤjoʁβzɯr}}}
}\end{relation-sémantique}\end{entrée}

\begin{entrée}
\vedette{\hypertarget{Ⓔjpɣom}{\papi{ jpɣom}}}\markboth{jpɣom}{}\classe{vi}
\paradigme{\textit{dir :} \jya kɤ-}
\begin{définition}\fra geler\end{définition}
\begin{définition}\cmn 冻\end{définition}
\begin{exemple}\jya ɕɤfɕo ɲɯ-mɯɕtaʁ tɕe tɯ-ci ko-jpɣom\cmn 这几天很冷,水结成冰了\end{exemple}
\begin{exemple}\jya nɤki ɯ-pɕi ma-nɯ-tɯ-te ma ɲɯ-mɯɕtaʁ tɕe jpɣom\cmn 你别把这个东西放在外面,现在很冷就会冻坏\end{exemple}\begin{sous-entrée}
\vedette{\hypertarget{}{\papi{ sɯjpɣom}}}\markboth{sɯjpɣom}{}\classe{vt}
\begin{définition}\fra geler\end{définition}
\begin{définition}\cmn 冻到\end{définition}
\begin{exemple}\jya kɯ-sɯjpɣom ʑo ɲɯ-ŋu\cmn 冷得把人冻僵\end{exemple}
\begin{relation-sémantique}\confer{
\hyperlink{Ⓔtɤjpɣom}{\textit{ \papi{tɤjpɣom}}}
}\end{relation-sémantique}
\begin{relation-sémantique}\confer{
\hyperlink{Ⓔqajpɣom}{\textit{ \papi{qajpɣom}}}
}\end{relation-sémantique}
\end{sous-entrée}\end{entrée}

\begin{entrée}
\vedette{\hypertarget{Ⓔjpum}{\papi{ jpum}}}\markboth{jpum}{}\classe{vs}
\paradigme{\textit{dir :} \jya nɯ-}
\begin{définition}\fra large (diamètre)\end{définition}
\begin{définition}\cmn 粗(直径)\end{définition}
\begin{exemple}\jya si ɲɯ-jpum\cmn 树很粗\end{exemple}\begin{sous-entrée}
\vedette{\hypertarget{}{\papi{ ɣɤjpum}}}\markboth{ɣɤjpum}{}\classe{vt}
\paradigme{\textit{dir :} \jya nɯ-}
\begin{définition}\fra rendre plus large\end{définition}
\begin{définition}\cmn 使变粗\end{définition}
\begin{exemple}\jya tɯ-pu thɯ-ɣɤmɯta tɕe, nɯ-ɣɤjpum-a\cmn 我把(猪)肠子吹胀起来了\end{exemple}
\begin{relation-sémantique}\confer{
\hyperlink{Ⓔjpumqa}{\textit{ \papi{jpumqa}}}
}\end{relation-sémantique}
\begin{relation-sémantique}\antonyme{
\hyperlink{Ⓔxtshɯm}{\textit{ \papi{xtshɯm}}}
}\end{relation-sémantique}
\end{sous-entrée}\end{entrée}

\begin{entrée}
\vedette{\hypertarget{Ⓔjpumqa}{\papi{ jpumqa}}}\markboth{jpumqa}{}
\classe{vs}
\paradigme{\textit{dir :} \jya thɯ-}
\begin{définition}\fra épais\end{définition}
\begin{définition}\cmn 粗壮\end{définition}
\begin{exemple}\jya kɯ-jpumqa ci ɲɯ-ŋu\cmn 是粗壮的\end{exemple}
\begin{relation-sémantique}\synonyme{
\hyperlink{Ⓔaβrdaβrdoŋ}{\textit{ \papi{aβrdaβrdoŋ}}}
}\end{relation-sémantique}
\begin{relation-sémantique}\antonyme{
\hyperlink{Ⓔtɕale}{\textit{ \papi{tɕale}}}
}\end{relation-sémantique}
\begin{relation-sémantique}\confer{
\hyperlink{Ⓔjpum}{\textit{ \papi{jpum}}}
}\end{relation-sémantique}\end{entrée}

\begin{entrée}
\vedette{\hypertarget{Ⓔjpumxtshɯm}{\papi{ jpumxtshɯm}}}\markboth{jpumxtshɯm}{}\classe{n}
\begin{définition}\fra épaisseur\end{définition}
\begin{définition}\cmn 粗细;粗度\end{définition}
\begin{relation-sémantique}\confer{
\hyperlink{Ⓔjpum}{\textit{ \papi{jpum}}}
}\end{relation-sémantique}
\begin{relation-sémantique}\confer{
\hyperlink{Ⓔxtshɯm}{\textit{ \papi{xtshɯm}}}
}\end{relation-sémantique}
\begin{relation-sémantique}\confer{
\hyperlink{Ⓔajpomxtshɯm}{\textit{ \papi{ajpomxtshɯm}}}
}\end{relation-sémantique}\end{entrée}

\begin{entrée}
\vedette{\hypertarget{Ⓔjqu}{\papi{ jqu}}}\markboth{jqu}{}\classe{vt}
\paradigme{\textit{dir :} \jya pɯ-}\acception{1}
\begin{définition}\fra être capable de soulever\end{définition}
\begin{définition}\cmn 背得起\end{définition}
\begin{exemple}\jya ɲɯ-jqe-a\cmn 我背得起\end{exemple}
\begin{exemple}\jya ɯ-ɲɯ́-tɯ-jqe?\cmn 你背得起吗?\end{exemple}
\begin{exemple}\jya mɯ-ɕɯ-tɯ-jqe nɯ-sɯ-so-t-a\cmn 我怕你背不起\end{exemple}
\begin{exemple}\jya jiɕqha laχtɕha nɯ tɤ-fkur-a ri pɯ-jqut-a\cmn 我把那个东西背起来了,我背得动(原来担心背不动)\end{exemple}
\begin{exemple}\jya tɯkhɯr ɲɯ-rʑi ri, ɲɯ-jqe-a\cmn 责任很重但是我背得起\end{exemple}
\begin{exemple}\jya pɤjkhu ɯ-ku mɤ-jqe\cmn (婴儿的脖子)还支撑不了他的头\end{exemple}
\begin{exemple}\jya ɯ-ku to-nɯ-jqu\cmn (婴儿的脖子)支撑得了他的头\end{exemple}\acception{2}
\begin{définition}\fra être capable de supporter\end{définition}
\begin{définition}\cmn 承受得了\end{définition}
\begin{exemple}\jya nɯ sthɯci kɯ-mŋɤm mɤ-jqe\cmn 他承受不了那么痛\end{exemple}\begin{sous-entrée}
\vedette{\hypertarget{}{\papi{ sɤjqu}}}\markboth{sɤjqu}{}\classe{vs}
\begin{définition}\fra qui peut être soulevé\end{définition}
\begin{définition}\cmn 背得动的\end{définition}
\begin{exemple}\jya kɯki tɯjpu ki kɤ-fkur ɲɯ-sɤjqu\cmn 这袋粮食可以背得动\end{exemple}
\end{sous-entrée}\end{entrée}

\begin{entrée}
\vedette{\hypertarget{Ⓔjtshi}{\papi{ jtshi}}}\markboth{jtshi}{}
\classe{vt}
\paradigme{\textit{dir :} \jya nɯ-}
\begin{définition}\fra donner à boire\end{définition}
\begin{définition}\cmn 给别人喝\end{définition}
\begin{exemple}\jya tʂha nɯ-kɯ-jtshi-a\cmn 你给我喝茶了\end{exemple}
\begin{exemple}\jya cha nɯ-kɯ-jtshi-a\cmn 你给我喝酒了\end{exemple}\begin{sous-entrée}
\vedette{\hypertarget{}{\papi{ ajtshi}}}\markboth{ajtshi}{}\classe{vi}
\begin{exemple}\jya nɤʑo ɲɯ-tɯ-jtshi mɤ-ra ma ajtshi\cmn 你不用给他喝了,已经给了\end{exemple}
\begin{exemple}\jya smɤn ɯ-j-ajtshi\cmn 给他喝药了没有\end{exemple}
\begin{exemple}\jya tɤ-pɤtso ra tɤ-lu ajtshi\cmn 牛奶已经给小孩喝了\end{exemple}
\end{sous-entrée}\begin{sous-entrée}
\vedette{\hypertarget{}{\papi{ rɤjtshi}}}\markboth{rɤjtshi}{}\classe{vt}
\begin{définition}\ 
\begin{déclaration}\grammar{apass}\end{déclaration}\end{définition}
\begin{définition}\fra donner à boire à quelqu'un\end{définition}
\begin{définition}\cmn 给别人喝\end{définition}
\begin{exemple}\jya aʑo tɤtɕɯ ra nɯ-ɕki cha ɲɯ-rɤjtshi-a ŋgrɤl\cmn 我给男生们喝酒\end{exemple}
\end{sous-entrée}\begin{sous-entrée}
\vedette{\hypertarget{}{\papi{ sɤjtshi}}}\markboth{sɤjtshi}{}\classe{vt}
\paradigme{\textit{dir :} \jya nɯ-}
\begin{définition}\fra demander à boire\end{définition}
\begin{définition}\cmn 请人给自己喝\end{définition}
\begin{exemple}\jya aʑo tʂha ci ɲɯ-sɤjtshi-a\cmn 我让人给我喝茶\end{exemple}
\begin{exemple}\jya tɕethi thɯ-ɕe tɕe, tʂha ci nɯ-sɤjtshi\cmn 你往下游去,要个茶喝\end{exemple}
\end{sous-entrée}\end{entrée}

\begin{entrée}
\vedette{\hypertarget{Ⓔjtsraβ}{\papi{ jtsraβ}}}\markboth{jtsraβ}{}\classe{vi}
\paradigme{\textit{dir :} \jya nɯ-}
\begin{définition}\fra retarder son départ\end{définition}
\begin{définition}\cmn 推迟出发时间\end{définition}
\begin{exemple}\jya toʁde nɯ-jtsraβ-a\cmn 我推迟了一下时间\end{exemple}
\begin{exemple}\jya ɯʑo toʁde nɯ-jtsraβ\cmn 他推迟了一下时间\end{exemple}
\begin{exemple}\jya ɯʑo ju-ɕe pɯ-ɕti ri, hanɯni ci mɯ́j-ndzu tɕe nɯ-jtsraβ pɯ-ra\cmn 他本来是要走的,但是因为没有准备好,只好推迟了出发的时间\end{exemple}\end{entrée}

\begin{entrée}
\vedette{\hypertarget{Ⓔjtʂhɤβnɤjtʂhɤβ}{\papi{ jtʂhɤβnɤjtʂhɤβ}}}\markboth{jtʂhɤβnɤjtʂhɤβ}{}
\classe{idph.3}
\begin{définition}\fra qui tire et arrache dans tous les sens\end{définition}
\begin{définition}\cmn 形容乱扯的样子\end{définition}
\begin{exemple}\jya ɯ-ŋga jtʂhɤβnɤjtʂhɤβ ʑo jo-rɤɕi\cmn 他乱扯了他的衣服\end{exemple}\end{entrée}

\begin{entrée}
\vedette{\hypertarget{Ⓔjɯβjɯβ}{\papi{ jɯβjɯβ}}}\markboth{jɯβjɯβ}{}
\classe{idph.2}
\begin{définition}\fra beaucoup d'animaux\end{définition}
\begin{définition}\cmn 很多(动物)密密麻麻地站着\end{définition}
\begin{exemple}\jya fsapaʁ jɯβjɯβ ʑo ɲɯ-rɤʑinɯ\cmn 很多牲畜在那里密密麻麻地站着\end{exemple}
\begin{relation-sémantique}\confer{
\hyperlink{Ⓔʑɯβʑɯβ}{\textit{ \papi{ʑɯβʑɯβ}}}
}\end{relation-sémantique}\end{entrée}

\begin{entrée}
\vedette{\hypertarget{Ⓔjɯfɕɯmɯr}{\papi{ jɯfɕɯmɯr}}}\markboth{jɯfɕɯmɯr}{}\classe{n}
\begin{définition}\fra hier soir\end{définition}
\begin{définition}\cmn 昨晚\end{définition}
\begin{relation-sémantique}\confer{
\hyperlink{Ⓔjɯfɕɯr}{\textit{ \papi{jɯfɕɯr}}}
}\end{relation-sémantique}
\begin{relation-sémantique}\confer{
\hyperlink{Ⓔtɯ-ɣmɯr}{\textit{ \papi{tɯ-ɣmɯr}}}
}\end{relation-sémantique}
\end{entrée}

\begin{entrée}
\vedette{\hypertarget{Ⓔjɯfɕɯndʐi}{\papi{ jɯfɕɯndʐi}}}\markboth{jɯfɕɯndʐi}{}
\classe{n}
\begin{définition}\fra ces derniers jours\end{définition}
\begin{définition}\cmn 前几天\end{définition}\end{entrée}

\begin{entrée}
\vedette{\hypertarget{Ⓔjɯfɕɯndʐɯɕɯŋgɯpa}{\papi{ jɯfɕɯndʐɯɕɯŋgɯpa}}}\markboth{jɯfɕɯndʐɯɕɯŋgɯpa}{}\classe{adv}
\begin{définition}\fra il y a trois ans\end{définition}
\begin{définition}\cmn 三年前\end{définition}
\end{entrée}

\begin{entrée}
\vedette{\hypertarget{Ⓔjɯfɕɯndʐɯpa}{\papi{ jɯfɕɯndʐɯpa}}}\markboth{jɯfɕɯndʐɯpa}{}\classe{adv}
\begin{définition}\fra l'année d'avant\end{définition}
\begin{définition}\cmn 前年\end{définition}
\end{entrée}

\begin{entrée}
\vedette{\hypertarget{Ⓔjɯfɕɯndʐɯsŋi}{\papi{ jɯfɕɯndʐɯsŋi}}}\markboth{jɯfɕɯndʐɯsŋi}{}\classe{adv}
\begin{définition}\fra avant-hier\end{définition}
\begin{définition}\cmn 前天\end{définition}
\end{entrée}

\begin{entrée}
\vedette{\hypertarget{Ⓔjɯfɕɯr}{\papi{ jɯfɕɯr}}}\markboth{jɯfɕɯr}{}
\classe{adv}
\begin{définition}\fra hier\end{définition}
\begin{définition}\cmn 昨天\end{définition}\end{entrée}

\begin{entrée}
\vedette{\hypertarget{Ⓔjɯɣi}{\papi{ jɯɣi}}}\markboth{jɯɣi}{}\classe{n}
\begin{définition}\fra lettre\end{définition}
\begin{définition}\cmn 信;书
\begin{déclaration} \étymologie{\papi{ji.ge}}\end{déclaration}\end{définition}
\end{entrée}

\begin{entrée}
\vedette{\hypertarget{Ⓔjɯɣmɯr}{\papi{ jɯɣmɯr}}}\markboth{jɯɣmɯr}{}
\classe{adv}
\begin{définition}\fra ce soir\end{définition}
\begin{définition}\cmn 今晚\end{définition}\end{entrée}

\begin{entrée}
\vedette{\hypertarget{Ⓔjɯl}{\papi{ jɯl}}}\markboth{jɯl}{}\classe{n}
\begin{définition}\fra village\end{définition}
\begin{définition}\cmn 村庄(有地、有房子的地方)
\begin{déclaration} \étymologie{\papi{jul}}\end{déclaration}\end{définition}\end{entrée}

\begin{entrée}
\vedette{\hypertarget{Ⓔjɯlco}{\papi{ jɯlco}}}\markboth{jɯlco}{}\classe{n}
\begin{définition}\fra voisin\end{définition}
\begin{définition}\cmn 邻居\end{définition}
\begin{relation-sémantique}\confer{
\hyperlink{Ⓔtɤ-rɣa}{\textit{ \papi{tɤ-rɣa}}}
}\end{relation-sémantique}\end{entrée}

\begin{entrée}
\vedette{\hypertarget{Ⓔjɯlpa}{\papi{ jɯlpa}}}\markboth{jɯlpa}{}\classe{n}
\begin{définition}\fra villageois\end{définition}
\begin{définition}\cmn 村民
\begin{déclaration} \étymologie{\papi{jul.pa}}\end{déclaration}\end{définition}
\end{entrée}

\begin{entrée}
\vedette{\hypertarget{ⒺjɯmⒽ2}{\papi{ jɯm}}}\markboth{jɯm}{}\homonyme{2}
\classe{n}
\begin{définition}\fra épouse de sprulsku\end{définition}
\begin{définition}\cmn 活佛的妻子
\begin{déclaration} \étymologie{\papi{jum}}\end{déclaration}\end{définition}
\end{entrée}

\begin{entrée}
\vedette{\hypertarget{ⒺjɯmⒽ1}{\papi{ jɯm}}}\markboth{jɯm}{}\homonyme{1}\classe{vs}
\paradigme{\textit{dir :} \jya nɯ-}
\begin{définition}\fra être clair\end{définition}
\begin{définition}\cmn 天晴\end{définition}
\begin{exemple}\jya jisŋi ɲɯ-jɯm\cmn 今天天晴\end{exemple}
\begin{exemple}\jya tɤŋe ɲɯ-jɯm\cmn 太阳很好\end{exemple}\begin{sous-entrée}
\vedette{\hypertarget{}{\papi{ sɯɣjɯm}}}\markboth{sɯɣjɯm}{}\classe{vt}
\paradigme{\textit{dir :} \jya nɯ-}
\begin{définition}\fra apporter le beau temps (par magie)\end{définition}
\begin{définition}\cmn (用法术)令天变晴\end{définition}
\end{sous-entrée}\end{entrée}

\begin{entrée}
\vedette{\hypertarget{Ⓔjɯnbalazɯ}{\papi{ jɯnbalazɯ}}}\markboth{jɯnbalazɯ}{}\classe{cnj}
\begin{définition}\fra non seulement... mais...\end{définition}
\begin{définition}\cmn 既……又……
\begin{déclaration} \étymologie{\papi{jin.pa.la}}\end{déclaration}\end{définition}
\begin{exemple}\jya jisŋi ɲɯ-jɯm jɯnbalazɯ, qale ɣɤʑu\cmn 今天既是晴天又刮风\end{exemple}\end{entrée}

\begin{entrée}
\vedette{\hypertarget{Ⓔjɯxɕo}{\papi{ jɯxɕo}}}\markboth{jɯxɕo}{}
\classe{adv}
\begin{définition}\fra ce matin\end{définition}
\begin{définition}\cmn 今天早上\end{définition}\end{entrée}

\begin{entrée}
\vedette{\hypertarget{Ⓔjwajwa}{\papi{ jwajwa}}}\markboth{jwajwa}{}
\classe{idph.2}
\begin{définition}\fra très fin\end{définition}
\begin{définition}\cmn 形容薄的样子
\end{définition}\end{entrée}

\begin{entrée}
\vedette{\hypertarget{Ⓔjwɤjwɤr}{\papi{ jwɤjwɤr}}}\markboth{jwɤjwɤr}{}\classe{idph.2}
\begin{définition}\fra de travers (chapeau)\end{définition}
\begin{définition}\cmn 歪的,不周正(帽子)\end{définition}
\begin{exemple}\jya nɤ-rte tɤ-sɤste ma jwɤjwɤr ʑo ɲɯ-pa\cmn 你把帽子戴正吧,是歪的\end{exemple}
\begin{relation-sémantique}\confer{
\hyperlink{Ⓔʑwɤʑwɤr}{\textit{ \papi{ʑwɤʑwɤr}}}
}\end{relation-sémantique}\end{entrée}

\begin{entrée}
\vedette{\hypertarget{Ⓔjxɤjxɤt}{\papi{ jxɤjxɤt}}}\markboth{jxɤjxɤt}{}\classe{idph.2}
\begin{définition}\fra au corps fin et élancé, qui est très active (femme)\end{définition}
\begin{définition}\cmn 形容身子苗条、很勤劳的样子(女子)\end{définition}
\begin{exemple}\jya iɕqha tɕheme nɯ ɯ-phoŋbu ra jxɤjxɤt ʑo ɲɯ-pe\cmn 这个女子身体苗条,很勤劳的样子\end{exemple}\end{entrée}

\newpage\caractère{ɟ}

\begin{entrée}
\vedette{\hypertarget{Ⓔɟu}{\papi{ ɟu}}}\markboth{ɟu}{}
\classe{n}
\begin{définition}\fra bambou\end{définition}
\begin{définition}\cmn 竹子\end{définition}\end{entrée}

\begin{entrée}
\vedette{\hypertarget{Ⓔɟar}{\papi{ ɟar}}}\markboth{ɟar}{}\classe{vt}
\paradigme{\textit{dir :} \jya nɯ-}
\begin{définition}\fra séparer le navet et les feuilles du navet\end{définition}
\begin{définition}\cmn 把圆根的根和叶子切开\end{définition}
\begin{sous-entrée}
\vedette{\hypertarget{}{\papi{ rɤɟar}}}\markboth{rɤɟar}{}\classe{vi}
\begin{définition}\ 
\begin{déclaration}\grammar{apass}\end{déclaration}\end{définition}
\begin{exemple}\jya ku-rɤɟar-a\cmn 我在切圆根\end{exemple}
\end{sous-entrée}\end{entrée}

\begin{entrée}
\vedette{\hypertarget{Ⓔɟɤr}{\papi{ ɟɤr}}}\markboth{ɟɤr}{}
\classe{n}
\begin{définition}\fra colle\end{définition}
\begin{définition}\cmn 胶
\begin{déclaration} \étymologie{\papi{ⁿbʲar}}\end{déclaration}\end{définition}\end{entrée}

\begin{entrée}
\vedette{\hypertarget{Ⓔɟɤrɯru}{\papi{ ɟɤrɯru}}}\markboth{ɟɤrɯru}{}
\classe{n}
\begin{définition}\fra tsampa\end{définition}
\begin{définition}\cmn 糌粑的一种吃法\end{définition}
\begin{exemple}\jya khɯtsa ɯ-ŋgɯ tʂha tú-wɣ-rku tɕe, tɕhɯrtsɤm sɤznɤ a-pɯ-dɤn tsa tɕe ɯ-taʁ tɯ-sqar pjɯ́-wɣ-lɤt, tɯ-sqar nɯ tɯ-ci sɤznɤ a-pɯ-dɤn tsa tɕe ɲo-ɕmi tɕe rwoʁrwoʁ ʑo a-nɯ-pa tɕe tú-wɣ-ndza tɕe nɯ ɟɤrɯru rmi.\cmn 
在碗里倒茶,比\stylefv{tɕhɯrtsɤm}多一点,再放上糌粑,那个糌粑比水多一点,搅到出现很多小球就可以吃,这种吃法叫做\stylefv{ɟɤrɯru}
\end{exemple}\end{entrée}

\begin{entrée}
\vedette{\hypertarget{Ⓔɟɤwɤt}{\papi{ ɟɤwɤt}}}\markboth{ɟɤwɤt}{}\classe{n}
\begin{définition}\fra espèce d'herbe\end{définition}
\begin{définition}\cmn 草的一种\end{définition}\end{entrée}

\begin{entrée}
\vedette{\hypertarget{Ⓔɟuli}{\papi{ ɟuli}}}\markboth{ɟuli}{} (\variante{ɟɯlij}) \classe{n}
\begin{définition}\fra flûte\end{définition}
\begin{définition}\cmn 笛子\end{définition}
\begin{exemple}\jya ɟuli tɤ-ʑmbri-t-a\cmn 我吹了笛子\end{exemple}
\begin{relation-sémantique}\confer{
\hyperlink{Ⓔrɯɟuli}{\textit{ \papi{rɯɟuli}}}
}\end{relation-sémantique}\end{entrée}

\begin{entrée}
\vedette{\hypertarget{Ⓔɟrɯɣɟrɯɣ}{\papi{ ɟrɯɣɟrɯɣ}}}\markboth{ɟrɯɣɟrɯɣ}{}\classe{idph.2}\acception{1}
\begin{définition}\fra mal au ventre\end{définition}
\begin{définition}\cmn 形容肚子不舒服的感觉(快要拉肚子的感觉)\end{définition}
\begin{exemple}\jya a-xtu ɟrɯɣɟrɯɣ ʑo ɲɯ-pa ma ɲɯ-sɤɣdɯɣ\cmn 我肚子不舒服\end{exemple}
\begin{relation-sémantique}\confer{
\hyperlink{Ⓔɣɤɟɯɟrɯɣ}{\textit{ \papi{ɣɤɟɯɟrɯɣ}}}
}\end{relation-sémantique}\acception{2}
\begin{définition}\fra en désordre\end{définition}
\begin{définition}\cmn 形容凌乱,不整齐的样子\end{définition}
\begin{exemple}\jya laχtɕha ɟrɯɣɟrɯɣ ʑo ko-rmbɯ\cmn 他把东西得乱七八糟\end{exemple}
\begin{relation-sémantique}\confer{
\hyperlink{Ⓔtɯ-ɟrɯɣ}{\textit{ \papi{tɯ-ɟrɯɣ}}}
}\end{relation-sémantique}
\begin{relation-sémantique}\confer{
\hyperlink{Ⓔcrɯɣcrɯɣ}{\textit{ \papi{crɯɣcrɯɣ}}}
}\end{relation-sémantique}\end{entrée}

\begin{entrée}
\vedette{\hypertarget{Ⓔɟɯ}{\papi{ ɟɯ}}}\markboth{ɟɯ}{}\classe{vi}
\paradigme{\textit{dir :} \jya thɯ-}
\begin{définition}\fra être digéré\end{définition}
\begin{définition}\cmn 消化(缩小)\end{définition}
\begin{exemple}\jya tɤ-ndza-t-a nɯra chɤ-ɟɯ\cmn 我吃的东西已经消化好了\end{exemple}
\begin{relation-sémantique}\confer{
\hyperlink{Ⓔnɯɣɟɯ}{\textit{ \papi{nɯɣɟɯ}}}
}\end{relation-sémantique}\end{entrée}

\begin{entrée}
\vedette{\hypertarget{Ⓔɟɯga}{\papi{ ɟɯga}}}\markboth{ɟɯga}{}
\classe{n}
\begin{définition}\fra chemin tortueux\end{définition}
\begin{définition}\cmn 弯路\end{définition}\end{entrée}

\begin{entrée}
\vedette{\hypertarget{Ⓔɟɯgɯɟɯga}{\papi{ ɟɯgɯɟɯga}}}\markboth{ɟɯgɯɟɯga}{}\classe{n}
\begin{définition}\fra (chemin) tortueux\end{définition}
\begin{définition}\cmn 弯弯曲曲\end{définition}
\begin{exemple}\jya tʂu ɟɯgɯɟɯga ʑo tu-βze ɲɯ-ŋu\cmn 录是弯弯曲曲的\end{exemple}
\begin{relation-sémantique}\confer{
\hyperlink{Ⓔɟɯga}{\textit{ \papi{ɟɯga}}}
}\end{relation-sémantique}\end{entrée}

\begin{entrée}
\vedette{\hypertarget{Ⓔɟɯɣɟɯɣ}{\papi{ ɟɯɣɟɯɣ}}}\markboth{ɟɯɣɟɯɣ}{}\classe{idph.2}\acception{1}
\begin{définition}\fra meuble (terre), déluré\end{définition}
\begin{définition}\cmn 形容土地松散,或衣冠不整的样子\end{définition}\acception{2}
\begin{définition}\fra nombreux (objets longs)\end{définition}
\begin{définition}\cmn 很多(条状的物体)\end{définition}
\begin{exemple}\jya laʑu ɟɯɣɟɯɣ ʑo ɲɯ-ɴqoʁ\cmn 很多条腊肉挂在那里\end{exemple}
\begin{exemple}\jya nɤ-ŋga ɯ-ŋga tɤ-ɣɤβdi ɟɯɣɟɯɣ ʑo ma-tɯ-ʑɣɤ-stu\cmn 你把衣服穿好一点,不要松松垮垮的\end{exemple}\begin{sous-entrée}
\vedette{\hypertarget{}{\papi{ ɟɯɣnɤɟɯɣ}}}\markboth{ɟɯɣnɤɟɯɣ}{}\classe{idph.3}
\begin{définition}\fra grouillant et très nombreux\end{définition}
\begin{définition}\cmn 形容很多(虫子)乱动\end{définition}
\begin{exemple}\jya tɯ-ci ɯ-ŋgɯ zɯ qapri ɟɯɣnɤɟɯɣ ʑo ɲɯ-pa\cmn 水里很多蛇在乱动\end{exemple}
\begin{exemple}\jya tɯ-ci ɯ-ɣmbaj zɯ tɤjpɣom ɟɯɣnɤɟɯɣ ʑo ɲɯ-pa\cmn 水面上有很多冰\end{exemple}
\end{sous-entrée}\begin{sous-entrée}
\vedette{\hypertarget{}{\papi{ nɯɟɯɣɟɯɣ}}}\markboth{nɯɟɯɣɟɯɣ}{}\classe{vi}
\paradigme{\textit{dir :} \jya tɤ-}
\begin{définition}\fra (faire) en grand nombre, en groupe\end{définition}
\begin{définition}\cmn 很多……一起……;成群\end{définition}
\begin{exemple}\jya tɯ-mɯ kɤ-lɤt pa-ʑa tɕe ɲɯ-nɯɟɯɣɟɯɣ-nɯ ʑo kha ɯ-ŋgɯ lɤ-ari-nɯ\cmn 开始下雨的时候,他们成群地回家了\end{exemple}
\end{sous-entrée}\begin{sous-entrée}
\vedette{\hypertarget{}{\papi{ sɤɟɯɣɟɯɣ}}}\markboth{sɤɟɯɣɟɯɣ}{}\classe{vt}
\begin{définition}\fra frétiller\end{définition}
\begin{définition}\cmn 抖动\end{définition}
\begin{exemple}\jya jla kɯ ɯ-βri ɲɯ-sɤɟɯɣɟɯɣ\cmn 犏牛在抖动\end{exemple}
\begin{exemple}\jya ɲɯ-nɤŋkɯŋke tɕe, ɯ-tʂɯmpɤri ra ɲɯ-sɤɟɯɣɟɯɣ ʑo\cmn 她走路的时候围裙腰带的须须跟着抖动\end{exemple}
\end{sous-entrée}\end{entrée}

\begin{entrée}
\vedette{\hypertarget{Ⓔɟɯmɢom}{\papi{ ɟɯmɢom}}}\markboth{ɟɯmɢom}{}\classe{n}
\begin{définition}\fra pincette en bambou\end{définition}
\begin{définition}\cmn 竹子制成的夹子\end{définition}
\begin{relation-sémantique}\confer{
\hyperlink{Ⓔtamɢom}{\textit{ \papi{tamɢom}}}
}\end{relation-sémantique}
\begin{relation-sémantique}\synonyme{
\hyperlink{Ⓔrɟɯmtsɯ}{\textit{ \papi{rɟɯmtsɯ}}}
}\end{relation-sémantique}
\begin{relation-sémantique}\confer{
\hyperlink{Ⓔɟu}{\textit{ \papi{ɟu}}}
}\end{relation-sémantique}\end{entrée}

\begin{entrée}
\vedette{\hypertarget{Ⓔɟɯthoʁ}{\papi{ ɟɯthoʁ}}}\markboth{ɟɯthoʁ}{}\classe{n}
\begin{définition}\ 
\begin{déclaration}\grammar{n.lieu}\end{déclaration}\end{définition}
\begin{définition}\fra l'un des hameaux de Gyutshapa\end{définition}
\begin{définition}\cmn 二茶村的大队之一\end{définition}\end{entrée}

\newpage\caractère{k}

\begin{entrée}
\vedette{\hypertarget{Ⓔka}{\papi{ ka}}}\markboth{ka}{}\classe{adv}
\begin{définition}\fra chaque\end{définition}
\begin{définition}\cmn 各自\end{définition}\end{entrée}

\begin{entrée}
\vedette{\hypertarget{Ⓔka}{\papi{ ka}}}\markboth{ka}{}\classe{vi}
\paradigme{\textit{dir :} \jya tɤ-}
\paradigme{\textit{dir :} \jya lɤ-}
\begin{définition}\fra quitter le sol (nuage)\end{définition}
\begin{définition}\cmn 离开地面(云雾)\end{définition}
\begin{exemple}\jya zdɯm zgo ɯ-taʁ pjɤ-ndzoʁ tɕe mɯ-to-ka\cmn 云贴在山尖上,还没有离开地面\end{exemple}\end{entrée}

\begin{entrée}
\vedette{\hypertarget{Ⓔkaβ}{\papi{ kaβ}}}\markboth{kaβ}{}
\classe{vi}
\paradigme{\textit{dir :} \jya jɤ-}
\begin{définition}\fra porter de l'eau sur son dos\end{définition}
\begin{définition}\cmn 背水\end{définition}
\begin{exemple}\jya aj ɕ-ku-kaβ-a (ʑ-ɲɯ-kaβ-a)\cmn 我去背水\end{exemple}
\begin{relation-sémantique}\confer{
\hyperlink{Ⓔsakaβ}{\textit{ \papi{sakaβ}}}
}\end{relation-sémantique}\end{entrée}

\begin{entrée}
\vedette{\hypertarget{Ⓔkachijmɤɣ}{\papi{ kachijmɤɣ}}}\markboth{kachijmɤɣ}{}\classe{n}
\begin{définition}\fra une espèce de champignon\end{définition}
\begin{définition}\cmn 一种菌子\end{définition}
\begin{exemple}\jya kachijmɤɣ kɤ-ndza sna\cmn “甜菌”可以吃\end{exemple}\end{entrée}

\begin{entrée}
\vedette{\hypertarget{Ⓔkaɣɯ}{\papi{ kaɣɯ}}}\markboth{kaɣɯ}{}
\classe{n}
\begin{définition}\fra collier avec pendentif en argent\end{définition}
\begin{définition}\cmn 胸章\end{définition}\end{entrée}

\begin{entrée}
\vedette{\hypertarget{Ⓔkaldʐa}{\papi{ kaldʐa}}}\markboth{kaldʐa}{}
\classe{n}
\begin{définition}\fra une espèce d'arbrisseau\end{définition}
\begin{définition}\cmn 灌木的一种\end{définition}
\begin{exemple}\jya kaldʐa nɯ si mɤ-jpum ri kɯ-mbro ci ŋu, zgoku kɯ-mbro tɕe, si kɯ-wxti ra nɯ-rca tu-ɬoʁ ŋu. ɯ-ru nɯ ɲaʁ, ɯ-jwaʁ nɯ kɯ-ɤrtɯm kɯ-zri tsa tɕe kɯ-ndɯ-ndɯβ ʑo ŋu. ɯ-mɯntoʁ kɯ-wɣrum ɲɯ-lɤt tɕe, ɯ-mɯntoʁ ɯ-jɯ nɯ zri tsa tɕe, tɯ-ɣnɤsqi jamar tɯtɯrca ku-ndzoʁ ŋu. ɯ-mat thɯ-tɯt tɕe, kɯ-ɣrɯ-wɣrum ŋu, spaχpɯ kɯ-jndʐɤz ʑo fse, ɯ-si nɯ wuma ʑo ngɯt tɕe laʁdɯn ɯ-jɯ ɯ-spa wuma ʑo pe.\cmn 
\stylefv{kaldʐa}是长得不粗但很高的树种,生长在高山的乔木林里。树干是黑色的,叶子是椭圆形的,很小。花是白色的,花梗比较长,一、二十朵长在一起。果实成熟后也是白色的,很像大豌豆,木质结实,是作农具把子的好材料。
\end{exemple}\end{entrée}

\begin{entrée}
\vedette{\hypertarget{Ⓔkamda}{\papi{ kamda}}}\markboth{kamda}{}\classe{n}
\begin{définition}\fra brochette de tranches de navets\end{définition}
\begin{définition}\cmn 穿成一串的芜菁片\end{définition}
\begin{relation-sémantique}\confer{
\hyperlink{Ⓔrɤjndoʁ}{\textit{ \papi{rɤjndoʁ}}}
}\end{relation-sémantique}\end{entrée}

\begin{entrée}
\vedette{\hypertarget{Ⓔkanaʁ}{\papi{ kanaʁ}}}\markboth{kanaʁ}{}\classe{n}
\begin{définition}\fra bovidé de couleur noire dont le ventre et les pattes sont blancs\end{définition}
\begin{définition}\cmn 全身是黑色的,肚皮和脚白色的牛
\begin{déclaration} \étymologie{\papi{dkar.nag}}\end{déclaration}\end{définition}
\end{entrée}

\begin{entrée}
\vedette{\hypertarget{Ⓔkaŋi}{\papi{ kaŋi}}}\markboth{kaŋi}{}\classe{n}
\begin{définition}\fra endroit où l'on place la nourriture préparée\end{définition}
\begin{définition}\cmn 寄放成品食物的地方\end{définition}\end{entrée}

\begin{entrée}
\vedette{\hypertarget{Ⓔkaŋkaŋ}{\papi{ kaŋkaŋ}}}\markboth{kaŋkaŋ}{}\classe{n}
\begin{définition}\fra tasse (avec une poignée)\end{définition}
\begin{définition}\cmn 带把的杯子
\begin{déclaration} \étymologie{\papi{\stylefn{缸缸}}}\end{déclaration}\end{définition}
\end{entrée}

\begin{entrée}
\vedette{\hypertarget{Ⓔkarɣi}{\papi{ karɣi}}}\markboth{karɣi}{}
\classe{n}
\begin{définition}\fra graine de navet\end{définition}
\begin{définition}\cmn 【菜子】\end{définition}
\begin{exemple}\jya karɣi nɯ rasti rɣi kɤ-ti ŋu\cmn 菜子是圆根的种子\end{exemple}\end{entrée}

\begin{entrée}
\vedette{\hypertarget{Ⓔkarɟɯ}{\papi{ karɟɯ}}}\markboth{karɟɯ}{}\classe{n}
\begin{définition}\ 
\begin{déclaration}\grammar{n.lieu}\end{déclaration}\end{définition}
\begin{définition}\fra l'un des hameaux de Gyutshapa\end{définition}
\begin{définition}\cmn 二茶村的大队之一\end{définition}\end{entrée}

\begin{entrée}
\vedette{\hypertarget{Ⓔkarkɯm}{\papi{ karkɯm}}}\markboth{karkɯm}{}\classe{n}
\begin{définition}\fra cotylédon du navet\end{définition}
\begin{définition}\cmn 圆根的子叶\end{définition}
\begin{relation-sémantique}\confer{
\hyperlink{Ⓔɯ-rkɯm}{\textit{ \papi{ɯ-rkɯm}}}
}\end{relation-sémantique}\end{entrée}

\begin{entrée}
\vedette{\hypertarget{Ⓔkatɕa}{\papi{ katɕa}}}\markboth{katɕa}{}\classe{n}
\begin{définition}\ 
\begin{déclaration}\grammar{n.lieu}\end{déclaration}\end{définition}
\begin{définition}\fra Katcha (village de Gdongbrgyad)\end{définition}
\begin{définition}\cmn 尕渣村\end{définition}
\end{entrée}

\begin{entrée}
\vedette{\hypertarget{Ⓔkawa}{\papi{ kawa}}}\markboth{kawa}{}\classe{n}
\begin{définition}\fra bovidé à tête blanche et au corps noir\end{définition}
\begin{définition}\cmn 头色的,而身体黑色的牛\end{définition}\end{entrée}

\begin{entrée}
\vedette{\hypertarget{Ⓔkɤɕŋaʁ}{\papi{ kɤɕŋaʁ}}}\markboth{kɤɕŋaʁ}{}\classe{n}
\begin{définition}\fra avoir l'impression d'entendre\end{définition}
\begin{définition}\cmn 仿佛听见\end{définition}
\begin{exemple}\jya ɯ-pɕi tɕe, kɤkhu ɣɤʑu rcama kɤɕŋaʁ ci ɣɤʑu\cmn 外面仿佛听到喊声\end{exemple}
\begin{exemple}\jya jiɕqha ri kɤɕŋaʁ ci pɯ-tu\cmn 刚才仿佛听见声音\end{exemple}\end{entrée}

\begin{entrée}
\vedette{\hypertarget{Ⓔkɤɕoʁ}{\papi{ kɤɕoʁ}}}\markboth{kɤɕoʁ}{}\classe{n}
\begin{définition}\fra nœud\end{définition}
\begin{définition}\cmn 活结\end{définition}
\begin{exemple}\jya kɯki tɤmtɯɲaʁ mɤ-tú-wɣ-lɤt, kɤɕoʁ tú-wɣ-lɤt ra\cmn 不要打死结,要打活结\end{exemple}\end{entrée}

\begin{entrée}
\vedette{\hypertarget{Ⓔkɤɣ}{\papi{ kɤɣ}}}\markboth{kɤɣ}{}
\classe{vt}
\paradigme{\textit{dir :} \jya pɯ-}
\begin{définition}\fra courber\end{définition}
\begin{définition}\cmn 弄弯\end{définition}
\begin{exemple}\jya jiɕqha si nɯ pɯ-kaɣ-a\cmn 我把那棵树弄弯了\end{exemple}
\begin{exemple}\jya laʁjɯɣ pɯ-kaɣ-a\cmn 我把棍子弄弯了\end{exemple}
\begin{exemple}\jya tɯdi kɤ-kɤɣ ɯ-ɲɯ́-tɯ-wɣ-sɯxcha\cmn 这把弓你拉得动吗?\end{exemple}
\begin{exemple}\jya ɕom kɤ-kɤɣ\cmn 把铁弄弯\end{exemple}
\begin{relation-sémantique}\synonyme{
\hyperlink{Ⓔsɤjʁu}{\textit{ \papi{sɤjʁu}}}
}\end{relation-sémantique}
\begin{relation-sémantique}\confer{
\hyperlink{Ⓔŋgɤɣ}{\textit{ \papi{ŋgɤɣ}}}
}\end{relation-sémantique}\end{entrée}

\begin{entrée}
\vedette{\hypertarget{Ⓔkɤjpɯ}{\papi{ kɤjpɯ}}}\markboth{kɤjpɯ}{}
\classe{n}
\begin{définition}\fra poulain\end{définition}
\begin{définition}\cmn 马驹,小马\end{définition}\end{entrée}

\begin{entrée}
\vedette{\hypertarget{Ⓔkɤlu}{\papi{ kɤlu}}}\markboth{kɤlu}{}\classe{adv}
\begin{définition}\fra sans tête\end{définition}
\begin{définition}\cmn 无头\end{définition}
\begin{relation-sémantique}\confer{
\hyperlink{Ⓔtɯ-ku}{\textit{ \papi{tɯ-ku}}}
}\end{relation-sémantique}\end{entrée}

\begin{entrée}
\vedette{\hypertarget{Ⓔkɤlɤthɤftɕa}{\papi{ kɤlɤthɤftɕa}}}\markboth{kɤlɤthɤftɕa}{}\classe{n}
\begin{définition}\fra ustensiles de cuisine\end{définition}
\begin{définition}\cmn 厨房用品\end{définition}\end{entrée}

\begin{entrée}
\vedette{\hypertarget{Ⓔkɤlɯmlɯm}{\papi{ kɤlɯmlɯm}}}\markboth{kɤlɯmlɯm}{}
\classe{adv}
\begin{définition}\fra ensemble\end{définition}
\begin{définition}\cmn 一起;完整地\end{définition}
\begin{exemple}\jya ɯ-kɤrme cho ɯ-ɕa kɤlɯmlɯm ʑo ka-ndo\cmn 他把头发和皮肉一起抓了起来\end{exemple}\end{entrée}

\begin{entrée}
\vedette{\hypertarget{Ⓔkɤmbrɤβraʁ}{\papi{ kɤmbrɤβraʁ}}}\markboth{kɤmbrɤβraʁ}{}\classe{n}
\begin{définition}\fra relation\end{définition}
\begin{définition}\cmn 联系;瓜葛;牵连\end{définition}
\begin{exemple}\jya nɤj nɤ-kɤ-nɤma cho aʑo a-kɤ-nɤma nɯ ʑaka ɕti ma kɤmbrɤβraʁ me\cmn 你的工作和我的工作没有什么瓜葛\end{exemple}\end{entrée}

\begin{entrée}
\vedette{\hypertarget{Ⓔkɤmda}{\papi{ kɤmda}}}\markboth{kɤmda}{}\classe{n}
\begin{définition}\fra navet séché\end{définition}
\begin{définition}\cmn 用柳条穿起来,晾干了的芜菁根\end{définition}\end{entrée}

\begin{entrée}
\vedette{\hypertarget{Ⓔkɤmɲɯ}{\papi{ kɤmɲɯ}}}\markboth{kɤmɲɯ}{}
\classe{n}
\begin{définition}\ 
\begin{déclaration}\grammar{n.lieu}\end{déclaration}\end{définition}
\begin{définition}\fra Kamnyu\end{définition}
\begin{définition}\cmn 干木鸟村\end{définition}
\begin{exemple}\jya ʑatsa kɤmɲɯ nɯɕe-a\cmn 我很快就要回干木鸟了\end{exemple}\end{entrée}

\begin{entrée}
\vedette{\hypertarget{Ⓔkɤmoʁ}{\papi{ kɤmoʁ}}}\markboth{kɤmoʁ}{}\classe{n}
\begin{définition}\fra tsampa destinée à être consommée sèche\end{définition}
\begin{définition}\cmn 香糌粑\end{définition}
\begin{relation-sémantique}\confer{
\hyperlink{Ⓔmoʁ}{\textit{ \papi{moʁ}}}
}\end{relation-sémantique}\end{entrée}

\begin{entrée}
\vedette{\hypertarget{Ⓔkɤndzɤtshi}{\papi{ kɤndzɤtshi}}}\markboth{kɤndzɤtshi}{}\classe{n}
\begin{définition}\fra repas\end{définition}
\begin{définition}\cmn 餐\end{définition}
\end{entrée}

\begin{entrée}
\vedette{\hypertarget{Ⓔkɤndʑɯβzaŋsa}{\papi{ kɤndʑɯβzaŋsa}}}\markboth{kɤndʑɯβzaŋsa}{}
\classe{n}
\begin{définition}\fra amis\end{définition}
\begin{définition}\cmn 朋友们\end{définition}
\begin{exemple}\jya nɤʑo cho aʑo kɤndʑɯβzaŋsa\cmn 我们是朋友\end{exemple}
\begin{relation-sémantique}\confer{
\hyperlink{Ⓔβzaŋsa}{\textit{ \papi{βzaŋsa}}}
}\end{relation-sémantique}\end{entrée}

\begin{entrée}
\vedette{\hypertarget{Ⓔkɤndʑɯɕaχpu}{\papi{ kɤndʑɯɕaχpu}}}\markboth{kɤndʑɯɕaχpu}{}\classe{n}
\begin{définition}\fra amis\end{définition}
\begin{définition}\cmn 朋友\end{définition}
\begin{relation-sémantique}\synonyme{
\hyperlink{Ⓔkɤndʑɯβzaŋsa}{\textit{ \papi{kɤndʑɯβzaŋsa}}}
}\end{relation-sémantique}
\begin{relation-sémantique}\confer{
\hyperlink{Ⓔɕaχpu}{\textit{ \papi{ɕaχpu}}}
}\end{relation-sémantique}\end{entrée}

\begin{entrée}
\vedette{\hypertarget{Ⓔkɤndʑɯɣe}{\papi{ kɤndʑɯɣe}}}\markboth{kɤndʑɯɣe}{}
\classe{n}
\begin{définition}\fra grand-père et petit-fils\end{définition}
\begin{définition}\cmn 祖父孙子\end{définition}
\begin{exemple}\jya a-ɣe cho tɕiʑo kɤndʑɯɣe\end{exemple}
\begin{relation-sémantique}\confer{
\hyperlink{Ⓔtɤ-ɣe}{\textit{ \papi{tɤ-ɣe}}}
}\end{relation-sémantique}\end{entrée}

\begin{entrée}
\vedette{\hypertarget{Ⓔkɤndʑɯkɯmdza}{\papi{ kɤndʑɯkɯmdza}}}\markboth{kɤndʑɯkɯmdza}{}\classe{n}
\begin{définition}\fra membres d'une même famille\end{définition}
\begin{définition}\cmn 亲戚\end{définition}
\begin{exemple}\jya kɤndʑɯkɯmdza pjɤ-ŋu-ndʑi\cmn 他们俩是亲戚\end{exemple}
\begin{relation-sémantique}\confer{
\hyperlink{Ⓔkɯmdza}{\textit{ \papi{kɯmdza}}}
}\end{relation-sémantique}\end{entrée}

\begin{entrée}
\vedette{\hypertarget{Ⓔkɤndʑɯmu}{\papi{ kɤndʑɯmu}}}\markboth{kɤndʑɯmu}{}
\classe{n}
\begin{définition}\fra mère et fille\end{définition}
\begin{définition}\cmn 母女\end{définition}\end{entrée}

\begin{entrée}
\vedette{\hypertarget{Ⓔkɤndʑɯmɤtsa}{\papi{ kɤndʑɯmɤtsa}}}\markboth{kɤndʑɯmɤtsa}{}
\classe{n}
\begin{définition}\fra cousin (collectif)\end{définition}
\begin{définition}\cmn 堂兄弟姐妹\end{définition}\end{entrée}

\begin{entrée}
\vedette{\hypertarget{Ⓔkɤndʑɯmbro}{\papi{ kɤndʑɯmbro}}}\markboth{kɤndʑɯmbro}{}\classe{n}
\begin{définition}\fra le cavalier et son cheval\end{définition}
\begin{définition}\cmn 马和骑手\end{définition}\end{entrée}

\begin{entrée}
\vedette{\hypertarget{Ⓔkɤndʑɯɲi}{\papi{ kɤndʑɯɲi}}}\markboth{kɤndʑɯɲi}{}
\classe{n}
\begin{définition}\fra tante et neveu\end{définition}
\begin{définition}\cmn 姑母侄子\end{définition}
\begin{relation-sémantique}\confer{
\hyperlink{Ⓔtɤ-ɲi}{\textit{ \papi{tɤ-ɲi}}}
}\end{relation-sémantique}\end{entrée}

\begin{entrée}
\vedette{\hypertarget{Ⓔkɤndʑɯpɤmdɯ}{\papi{ kɤndʑɯpɤmdɯ}}}\markboth{kɤndʑɯpɤmdɯ}{}
\classe{n}
\begin{définition}\fra oncle et neveu\end{définition}
\begin{définition}\cmn 叔叔侄子\end{définition}
\begin{relation-sémantique}\confer{
\hyperlink{Ⓔtɤ-mdɯ}{\textit{ \papi{tɤ-mdɯ}}}
}\end{relation-sémantique}\end{entrée}

\begin{entrée}
\vedette{\hypertarget{Ⓔkɤndʑɯʁi}{\papi{ kɤndʑɯʁi}}}\markboth{kɤndʑɯʁi}{}
\classe{n}
\begin{définition}\fra grand frère et petit frère, grande sœur et petite sœur\end{définition}
\begin{définition}\cmn 哥哥,姐姐,弟弟和妹妹\end{définition}
\begin{relation-sémantique}\confer{
\hyperlink{Ⓔta-ʁi}{\textit{ \papi{ta-ʁi}}}
}\end{relation-sémantique}\end{entrée}

\begin{entrée}
\vedette{\hypertarget{Ⓔkɤndʑɯslamaχti}{\papi{ kɤndʑɯslamaχti}}}\markboth{kɤndʑɯslamaχti}{}\classe{n}
\begin{définition}\fra ami de classe\end{définition}
\begin{définition}\cmn 好同学\end{définition}\end{entrée}

\begin{entrée}
\vedette{\hypertarget{Ⓔkɤndʑɯsqhaj}{\papi{ kɤndʑɯsqhaj}}}\markboth{kɤndʑɯsqhaj}{}
\classe{n}
\begin{définition}\fra sœurs (collectif)\end{définition}
\begin{définition}\cmn 姐妹\end{définition}
\begin{exemple}\jya jiʑo kɤndʑɯsqhaj ŋu-j\cmn 我们是姐妹\end{exemple}
\begin{relation-sémantique}\confer{
\hyperlink{Ⓔtɤ-sqhaj}{\textit{ \papi{tɤ-sqhaj}}}
}\end{relation-sémantique}\end{entrée}

\begin{entrée}
\vedette{\hypertarget{Ⓔkɤndʑɯtɤtɕɯχti}{\papi{ kɤndʑɯtɤtɕɯχti}}}\markboth{kɤndʑɯtɤtɕɯχti}{}\classe{n}
\begin{définition}\fra amis (garçons)\end{définition}
\begin{définition}\cmn 朋友(男孩之间)\end{définition}\end{entrée}

\begin{entrée}
\vedette{\hypertarget{Ⓔkɤndʑɯtɕhemɤχti}{\papi{ kɤndʑɯtɕhemɤχti}}}\markboth{kɤndʑɯtɕhemɤχti}{}\classe{n}
\begin{définition}\fra amies\end{définition}
\begin{définition}\cmn 朋友(女孩之间)\end{définition}\end{entrée}

\begin{entrée}
\vedette{\hypertarget{Ⓔkɤndʑɯwɤɬaʁ}{\papi{ kɤndʑɯwɤɬaʁ}}}\markboth{kɤndʑɯwɤɬaʁ}{}\classe{n}
\begin{définition}\fra tante et neveu\end{définition}
\begin{définition}\cmn 婶母侄子\end{définition}
\begin{relation-sémantique}\confer{
\hyperlink{Ⓔtɤ-ɬaʁ}{\textit{ \papi{tɤ-ɬaʁ}}}
}\end{relation-sémantique}\end{entrée}

\begin{entrée}
\vedette{\hypertarget{Ⓔkɤndʑɯwɤmɯsnom}{\papi{ kɤndʑɯwɤmɯsnom}}}\markboth{kɤndʑɯwɤmɯsnom}{}
\classe{n}
\begin{définition}\fra frères et sœurs\end{définition}
\begin{définition}\cmn 兄弟姐妹\end{définition}
\begin{relation-sémantique}\confer{
\hyperlink{Ⓔtɤ-wɤmɯ}{\textit{ \papi{tɤ-wɤmɯ}}}
}\end{relation-sémantique}
\begin{relation-sémantique}\confer{
\hyperlink{Ⓔtɤ-snom}{\textit{ \papi{tɤ-snom}}}
}\end{relation-sémantique}\end{entrée}

\begin{entrée}
\vedette{\hypertarget{Ⓔkɤndʑɯxtɤɣ}{\papi{ kɤndʑɯxtɤɣ}}}\markboth{kɤndʑɯxtɤɣ}{}
\classe{n}
\begin{définition}\fra frères (collectif)\end{définition}
\begin{définition}\cmn 兄弟\end{définition}
\begin{relation-sémantique}\confer{
\hyperlink{Ⓔtɤ-xtɤɣ}{\textit{ \papi{tɤ-xtɤɣ}}}
}\end{relation-sémantique}\end{entrée}

\begin{entrée}
\vedette{\hypertarget{Ⓔkɤndʑɯχti}{\papi{ kɤndʑɯχti}}}\markboth{kɤndʑɯχti}{}
\classe{n}
\begin{définition}\fra amis\end{définition}
\begin{définition}\cmn 朋友们\end{définition}
\begin{relation-sémantique}\confer{
\hyperlink{Ⓔtɯ-χti}{\textit{ \papi{tɯ-χti}}}
}\end{relation-sémantique}\end{entrée}

\begin{entrée}
\vedette{\hypertarget{Ⓔkɤndʑɯzda}{\papi{ kɤndʑɯzda}}}\markboth{kɤndʑɯzda}{}
\classe{n}
\begin{définition}\fra compagnons\end{définition}
\begin{définition}\cmn 伙伴们\end{définition}
\begin{relation-sémantique}\confer{
\hyperlink{Ⓔtɯ-zda}{\textit{ \papi{tɯ-zda}}}
}\end{relation-sémantique}\end{entrée}

\begin{entrée}
\vedette{\hypertarget{Ⓔkɤntɕhaʁ}{\papi{ kɤntɕhaʁ}}}\markboth{kɤntɕhaʁ}{}
\classe{n}\acception{1}
\begin{définition}\fra rue\end{définition}
\begin{définition}\cmn 街\end{définition}
\begin{exemple}\jya kɤntɕhaʁ rɤʑi-a\cmn 我在街上\end{exemple}
\begin{exemple}\jya kɤntɕhaʁ ci chɤ-ta\cmn 他摆了个摊子\end{exemple}\acception{2}
\begin{définition}\fra village, ville (endroit habité)\end{définition}
\begin{définition}\cmn 村子(人住的地方)\end{définition}
\begin{relation-sémantique}\confer{
\hyperlink{Ⓔnɯkɤntɕhaʁ}{\textit{ \papi{nɯkɤntɕhaʁ}}}
}\end{relation-sémantique}\end{entrée}

\begin{entrée}
\vedette{\hypertarget{Ⓔkɤntɕhɯkɤβdɤm}{\papi{ kɤntɕhɯkɤβdɤm}}}\markboth{kɤntɕhɯkɤβdɤm}{}
\classe{num}
\begin{définition}\fra entre quatre et neuf\end{définition}
\begin{définition}\cmn 几个(八九一下,四五个以上\end{définition}
\begin{exemple}\jya jiɕqha ɯ-rɟit kɤntɕhɯkɤβdɤm tu\cmn 他有好几个孩子\end{exemple}
\begin{relation-sémantique}\confer{
\hyperlink{Ⓔantɕhɯ}{\textit{ \papi{antɕhɯ}}}
}\end{relation-sémantique}\end{entrée}

\begin{entrée}
\vedette{\hypertarget{Ⓔkɤnɯβdɯt}{\papi{ kɤnɯβdɯt}}}\markboth{kɤnɯβdɯt}{} (\variante{kɤsɯβdɯt}) \classe{n}
\begin{définition}\fra dépense\end{définition}
\begin{définition}\cmn 花费\end{définition}
\begin{exemple}\jya nɤ-βdɯt nɯ-tɕat-a!\cmn 让你花费了很多(谢你请我吃饭)\end{exemple}
\begin{exemple}\jya kɤnɯβdɯt pɯ-me, koŋla tɤ-tɯ-ndza-nɯ me\cmn 没有花费很多,你们没有吃多少\end{exemple}
\begin{relation-sémantique}\confer{
\hyperlink{ⒺβdɯtⒽ1}{\textit{ \papi{βdɯt1}}}
}\end{relation-sémantique}\end{entrée}

\begin{entrée}
\vedette{\hypertarget{Ⓔkɤŋgɤxtsa}{\papi{ kɤŋgɤxtsa}}}\markboth{kɤŋgɤxtsa}{}
\classe{n}
\begin{définition}\fra habits et chaussures\end{définition}
\begin{définition}\cmn 衣服和鞋子\end{définition}\end{entrée}

\begin{entrée}
\vedette{\hypertarget{Ⓔkɤpa}{\papi{ kɤpa}}}\markboth{kɤpa}{}\classe{n}
\begin{définition}\fra moyen\end{définition}
\begin{définition}\cmn 办法\end{définition}
\begin{relation-sémantique}\confer{
\hyperlink{ⒺpaⒽ1}{\textit{ \papi{pa}}}
}\end{relation-sémantique}
\begin{relation-sémantique}\synonyme{
\hyperlink{Ⓔkowa}{\textit{ \papi{kowa}}}
}\end{relation-sémantique}\end{entrée}

\begin{entrée}
\vedette{\hypertarget{Ⓔkɤpupu}{\papi{ kɤpupu}}}\markboth{kɤpupu}{}
\classe{n}
\begin{définition}\fra racine de navet cuite\end{définition}
\begin{définition}\cmn 烤熟的芜菁根\end{définition}
\begin{exemple}\jya rɤjndoʁ lú-wɣ-pu tɕe nɯnɯ kɤpupu tu-kɯ-ti ɲɯ-ŋu\end{exemple}\end{entrée}

\begin{entrée}
\vedette{\hypertarget{Ⓔkɤpɯpri}{\papi{ kɤpɯpri}}}\markboth{kɤpɯpri}{}\classe{adv}
\begin{définition}\fra sans arrêt, l'un après l'autre\end{définition}
\begin{définition}\cmn 连续不断地,一个接着一个\end{définition}
\begin{exemple}\jya ɯ-kɤ-nɤma ɲɯ-dɤn tɕe kɤpɯpri ʑo nɤme ɲɯ-ra\cmn 他的工作很多,要连续不断地做\end{exemple}\end{entrée}

\begin{entrée}
\vedette{\hypertarget{Ⓔkɤrɤsla}{\papi{ kɤrɤsla}}}\markboth{kɤrɤsla}{}
\classe{n}
\begin{définition}\fra plusieurs mois\end{définition}
\begin{définition}\cmn 数月\end{définition}
\begin{exemple}\jya kɤrɤsla jɤ-tsu, jɤ-tsu-a\cmn 他已经(走了)几个月,我已经几个月了\end{exemple}
\begin{relation-sémantique}\confer{
\hyperlink{Ⓔtɯ-sla}{\textit{ \papi{tɯ-sla}}}
}\end{relation-sémantique}\end{entrée}

\begin{entrée}
\vedette{\hypertarget{Ⓔkɤrɤxpa}{\papi{ kɤrɤxpa}}}\markboth{kɤrɤxpa}{}
\classe{n}
\begin{définition}\fra plusieurs années\end{définition}
\begin{définition}\cmn 数年\end{définition}
\begin{exemple}\jya kɤrɤxpa jɤ-tsu-a\cmn 我过了很多年\end{exemple}
\begin{relation-sémantique}\confer{
\hyperlink{Ⓔtɯ-xpa}{\textit{ \papi{tɯ-xpa}}}
}\end{relation-sémantique}\end{entrée}

\begin{entrée}
\vedette{\hypertarget{Ⓔkɤrjɤl}{\papi{ kɤrjɤl}}}\markboth{kɤrjɤl}{}
\classe{n}
\begin{définition}\fra porcelaine\end{définition}
\begin{définition}\cmn 瓷碗
\begin{déclaration} \étymologie{\papi{dkar.jol}}\end{déclaration}\end{définition}
\begin{exemple}\jya kɤrjɤl popo, kɤrjɤl tɕhorzi kɤrjɤl khɯtsa\cmn 瓷碗\end{exemple}\end{entrée}

\begin{entrée}
\vedette{\hypertarget{Ⓔkɤrŋu}{\papi{ kɤrŋu}}}\markboth{kɤrŋu}{}\classe{n}
\begin{définition}\fra première période du mois\end{définition}
\begin{définition}\cmn 上半月
\begin{déclaration} \étymologie{\papi{dkar.ŋo}}\end{déclaration}\end{définition}
\end{entrée}

\begin{entrée}
\vedette{\hypertarget{Ⓔkɤrŋijmɤɣ}{\papi{ kɤrŋijmɤɣ}}}\markboth{kɤrŋijmɤɣ}{}
\classe{n}
\begin{définition}\fra une espèce de champignon\end{définition}
\begin{définition}\cmn 一种蘑菇\end{définition}
\begin{exemple}\jya kɤrŋi jmɤɣ to-ɬoʁ\cmn 蓝菌长出来了\end{exemple}
\begin{exemple}\jya kɤrŋijmɤɣ nɯ tɤjmɤɣ ci ŋu stɤmku ri tu-ɬoʁ tɕe arɤkhɯmkhɤl, kɤrŋijmɤɣ ɯ-sɤ-ɬoʁ stɤmku nɯ kɯroz ʑo arŋi, kɯ-ɤrqhi ʑo ju-kɯ-ru tɕe saχsɤl, kɤrŋijmɤɣ ɯʑo kɯnɤ kɯ-ɤrŋi tsa kɯ-ndɯ-ndɯβ ʑo ŋu, kɤ-ndza wuma ʑo mɯm\cmn 蓝菌是长在草地上的蘑菇,不是所有的草地都有,它生长的地方的草特别绿,从很远的地方都可以看到。蓝菌自己带有一点蓝色,长得很小,好吃。\end{exemple}
\begin{relation-sémantique}\synonyme{
\hyperlink{Ⓔtɤrmbjajmɤɣ}{\textit{ \papi{tɤrmbjajmɤɣ}}}
}\end{relation-sémantique}\end{entrée}

\begin{entrée}
\vedette{\hypertarget{Ⓔkɤrpu}{\papi{ kɤrpu}}}\markboth{kɤrpu}{}
\classe{n}
\begin{définition}\fra chaux\end{définition}
\begin{définition}\cmn 石灰
\begin{déclaration} \étymologie{\papi{dkar.po}}\end{déclaration}\end{définition}
\begin{exemple}\jya kha ɯ-taʁ kɤrpu ta-lɤt\cmn 他抹了墙\end{exemple}\end{entrée}

\begin{entrée}
\vedette{\hypertarget{Ⓔkɤrta}{\papi{ kɤrta}}}\markboth{kɤrta}{}\classe{n}
\begin{définition}\fra croix\end{définition}
\begin{définition}\cmn 十字形\end{définition}
\begin{exemple}\jya kɤrta pɯ-ta-t-a\cmn 我打了叉叉(做记号)\end{exemple}\end{entrée}

\begin{entrée}
\vedette{\hypertarget{Ⓔkɤrtsi}{\papi{ kɤrtsi}}}\markboth{kɤrtsi}{}\classe{n}
\begin{définition}\fra plusieurs (jours, mois, année)\end{définition}
\begin{définition}\cmn 好几(天、年)\end{définition}
\begin{exemple}\jya ɯ-sŋi kɤrtsi\cmn 数日,好几天\end{exemple}
\begin{exemple}\jya ɯ-xpa kɤrtsi\cmn 好几年\end{exemple}
\begin{exemple}\jya kɤ-rɤβzjoz nɯ ɯ-sla kɤrtsi ʑo ra ɕti ma laʁnɯ-rʑaʁ mɤ-fɕaʁ\cmn 学习是要好几个月,几天是不够的\end{exemple}
\end{entrée}

\begin{entrée}
\vedette{\hypertarget{Ⓔkɤsɤri}{\papi{ kɤsɤri}}}\markboth{kɤsɤri}{}
\begin{relation-sémantique}\confer{
\hyperlink{Ⓔmtɕhɤnmbrɯ}{\textit{ \papi{mtɕhɤnmbrɯ}}}
}\end{relation-sémantique}\end{entrée}

\begin{entrée}
\vedette{\hypertarget{ⒺkɤstuⒽ1}{\papi{ kɤstu}}}\markboth{kɤstu}{}\homonyme{1}
\classe{n}
\begin{définition}\fra moyen de s'en sortir\end{définition}
\begin{définition}\cmn 办法\end{définition}
\begin{exemple}\jya ɯ-kɤstu maka ɲɤ-me\cmn 他再也没有办法了\end{exemple}
\begin{relation-sémantique}\synonyme{
\hyperlink{Ⓔkɤpa}{\textit{ \papi{kɤpa}}}
}\end{relation-sémantique}\end{entrée}

\begin{entrée}
\vedette{\hypertarget{ⒺkɤstuⒽ2}{\papi{ kɤstu}}}\markboth{kɤstu}{}\homonyme{2}
\classe{n}
\begin{définition}\fra viande séchée conservée dans la peau du pied de cochon\end{définition}
\begin{définition}\cmn 把猪腿的皮剥下来,缝成筒形,塞满瘦肉\end{définition}\end{entrée}

\begin{entrée}
\vedette{\hypertarget{Ⓔkɤsɯfse}{\papi{ kɤsɯfse}}}\markboth{kɤsɯfse}{} (\variante{kɤfsɯfse}) 
\classe{adv}
\begin{définition}\fra entièrement, tout\end{définition}
\begin{définition}\cmn 全部\end{définition}
\begin{exemple}\jya kɤfsɯfse chɤ-k-ɤrɕo-ci\cmn 完全用完了\end{exemple}\end{entrée}

\begin{entrée}
\vedette{\hypertarget{Ⓔkɤtɕhɯ}{\papi{ kɤtɕhɯ}}}\markboth{kɤtɕhɯ}{}
\classe{n}
\begin{définition}\fra coup de tête\end{définition}
\begin{définition}\cmn 用头顶\end{définition}
\begin{exemple}\jya kɤtɕhɯ tɤ-lat-a\cmn 我用头顶了他\end{exemple}
\begin{relation-sémantique}\confer{
\hyperlink{Ⓔtɯ-ku}{\textit{ \papi{tɯ-ku}}}
}\end{relation-sémantique}
\begin{relation-sémantique}\confer{
\hyperlink{Ⓔtɕhɯ}{\textit{ \papi{tɕhɯ}}}
}\end{relation-sémantique}
\begin{relation-sémantique}\confer{
\hyperlink{Ⓔnɤkɤtɕhɯ}{\textit{ \papi{nɤkɤtɕhɯ}}}
}\end{relation-sémantique}\end{entrée}

\begin{entrée}
\vedette{\hypertarget{Ⓔkɤtsa}{\papi{ kɤtsa}}}\markboth{kɤtsa}{}
\classe{n}
\begin{définition}\fra parents et enfants\end{définition}
\begin{définition}\cmn 父母和孩子\end{définition}
\begin{exemple}\jya tɕheme kɤtsa\cmn 母女\end{exemple}
\begin{exemple}\jya tɤ-tɕɯ kɤtsa\cmn 父子\end{exemple}\end{entrée}

\begin{entrée}
\vedette{\hypertarget{Ⓔkɤtɯm}{\papi{ kɤtɯm}}}\markboth{kɤtɯm}{}
\classe{n}
\begin{définition}\fra pelote de laine\end{définition}
\begin{définition}\cmn 线团\end{définition}
\begin{exemple}\jya kɤtɯm tɤ-βze\end{exemple}\end{entrée}

\begin{entrée}
\vedette{\hypertarget{ⒺkɤtɯpaⒽ1}{\papi{ kɤtɯpa}}}\markboth{kɤtɯpa}{}\homonyme{1} (\variante{kɤtipa}) 
\classe{n}
\begin{définition}\fra chicanerie\end{définition}
\begin{définition}\cmn 计较\end{définition}
\begin{exemple}\jya ɯ-kɤtɯpa ɲɯ-dɤn\cmn 他很喜欢找别人的茬,很爱计较\end{exemple}
\begin{exemple}\jya a-kɤtɯpa dɤn\cmn 我爱计较\end{exemple}
\begin{exemple}\jya ɯ-kɤtɯpa me\cmn 他不计较\end{exemple}
\begin{exemple}\jya nɤʑo nɤ-kɤtipa rkɯn\cmn 你计较得少\end{exemple}
\begin{exemple}\jya nɤʑo nɤ-kɤtipa mɤ-dɤn tɕe pe\cmn 你不计较就好\end{exemple}
\begin{relation-sémantique}\confer{
\hyperlink{ⒺkɤtɯpaⒽ2}{\textit{ \papi{kɤtɯpa2}}}
}\end{relation-sémantique}\end{entrée}

\begin{entrée}
\vedette{\hypertarget{ⒺkɤtɯpaⒽ2}{\papi{ kɤtɯpa}}}\markboth{kɤtɯpa}{}\homonyme{2}
\classe{vt}
\begin{définition}\fra dire\end{définition}
\begin{définition}\cmn 说;转告
\begin{déclaration}\use{该动词只用于非过去时第一和第三人称,不能带任何前缀(包括第二人称前缀)}\end{déclaration}\end{définition}
\begin{exemple}\jya nɤʑo tɤ-tɯ-tɯt nɯ, aʑo kɤtɯpe-a ɕti\cmn 你说的话,我会转告(给他)\end{exemple}
\begin{exemple}\jya aʑo tɤ-tɯt-a nɯ, ɯʑo kɯ nɤ-ɕki kɤtɯpe\cmn 我说的话,他会转告给你的\end{exemple}
\begin{exemple}\jya kɤtɯpa-tɕi ɕti\cmn 我们俩会转告\end{exemple}
\begin{relation-sémantique}\confer{
\hyperlink{ⒺkɤtɯpaⒽ1}{\textit{ \papi{kɤtɯpa1}}}
}\end{relation-sémantique}\end{entrée}

\begin{entrée}
\vedette{\hypertarget{Ⓔkɕilu}{\papi{ kɕilu}}}\markboth{kɕilu}{}\classe{n}
\begin{définition}\fra année du chien\end{définition}
\begin{définition}\cmn 狗年
\begin{déclaration} \étymologie{\papi{kʰʲi.lo}}\end{déclaration}\end{définition}
\end{entrée}

\begin{entrée}
\vedette{\hypertarget{Ⓔkuɣrummɤɣ}{\papi{ kuɣrummɤɣ}}}\markboth{kuɣrummɤɣ}{}\classe{n}
\begin{définition}\fra une espèce de champignon\end{définition}
\begin{définition}\cmn 一种蘑菇\end{définition}
\begin{exemple}\jya kuɣrummɤɣ nɯ ɕkrɤz kɯ-xtɕi ɯ-ŋgɯ tu-ɬoʁ ŋu, kɯ-wɣrum ʁɟa ʑo ŋu, kɤ-ndza wuma ʑo mɯm. ftɕar tɕe tu-ɬoʁ ŋu.\cmn 
\stylefv{kuɣrummɤɣ}长在比较矮小的青冈树林里,均呈白色。很好吃,一般在夏天生长。
\end{exemple}\end{entrée}

\begin{entrée}
\vedette{\hypertarget{Ⓔkhu}{\papi{ khu}}}\markboth{khu}{}
\classe{n}
\begin{définition}\fra tigre\end{définition}
\begin{définition}\cmn 老虎\end{définition}\end{entrée}

\begin{entrée}
\vedette{\hypertarget{Ⓔkha}{\papi{ kha}}}\markboth{kha}{}\classe{n}\acception{1}
\begin{définition}\fra maison\end{définition}
\begin{définition}\cmn 房子\end{définition}\acception{2}
\begin{définition}\fra place originelle\end{définition}
\begin{définition}\cmn 原位,原来放过的地方\end{définition}
\begin{relation-sémantique}\confer{
\hyperlink{Ⓔrɤkha}{\textit{ \papi{rɤkha}}}
}\end{relation-sémantique}
\begin{relation-sémantique}\confer{
\hyperlink{Ⓔqhaqhu}{\textit{ \papi{qhaqhu}}}
}\end{relation-sémantique}\end{entrée}

\begin{entrée}
\vedette{\hypertarget{Ⓔkhamu}{\papi{ khamu}}}\markboth{khamu}{}
\classe{n}
\begin{définition}\fra cuisine\end{définition}
\begin{définition}\cmn 炊事\end{définition}
\begin{relation-sémantique}\confer{
 \papi{nɯkhama}
}\end{relation-sémantique}\end{entrée}

\begin{entrée}
\vedette{\hypertarget{Ⓔkhamba}{\papi{ khamba}}}\markboth{khamba}{}\classe{n}
\begin{définition}\fra gris clair\end{définition}
\begin{définition}\cmn 灰白色
\begin{déclaration} \étymologie{\papi{kʰam.pa}}\end{déclaration}\end{définition}
\end{entrée}

\begin{entrée}
\vedette{\hypertarget{Ⓔkhamba}{\papi{ khamba}}}\markboth{khamba}{}
\end{entrée}

\begin{entrée}
\vedette{\hypertarget{Ⓔkhambalawa}{\papi{ khambalawa}}}\markboth{khambalawa}{}\classe{n}
\begin{définition}\fra vêtement en laine grise\end{définition}
\begin{définition}\cmn 灰色的毛裙\end{définition}
\end{entrée}

\begin{entrée}
\vedette{\hypertarget{Ⓔkhaŋfkot}{\papi{ khaŋfkot}}}\markboth{khaŋfkot}{}\classe{n}
\begin{définition}\fra architecte\end{définition}
\begin{définition}\cmn 建筑师,设计房子的人
\begin{déclaration} \étymologie{\papi{kʰaŋ.bkod}}\end{déclaration}\end{définition}
\end{entrée}

\begin{entrée}
\vedette{\hypertarget{Ⓔkhara}{\papi{ khara}}}\markboth{khara}{}\classe{n}
\begin{définition}\fra part du produit de la chasse donnée aux amis\end{définition}
\begin{définition}\cmn 分给朋友的猎物\end{définition}
\begin{exemple}\jya a-khara a-nɯ-tɯ-βze je!\cmn 你给我一份吧\end{exemple}\end{entrée}

\begin{entrée}
\vedette{\hypertarget{Ⓔkhari}{\papi{ khari}}}\markboth{khari}{}\classe{n}
\begin{définition}\fra turban\end{définition}
\begin{définition}\cmn 包头巾
\begin{déclaration} \étymologie{\papi{kʰa.dkris}}\end{déclaration}\end{définition}
\end{entrée}

\begin{entrée}
\vedette{\hypertarget{Ⓔkharwut}{\papi{ kharwut}}}\markboth{kharwut}{}\classe{n}
\begin{définition}\fra fièvre aphteuse\end{définition}
\begin{définition}\cmn 口蹄疫
\begin{déclaration} \étymologie{\papi{*kʰa.rbod}}\end{déclaration}\end{définition}
\end{entrée}

\begin{entrée}
\vedette{\hypertarget{Ⓔkhatoʁ}{\papi{ khatoʁ}}}\markboth{khatoʁ}{}\classe{n}
\begin{définition}\fra de toutes les couleurs\end{définition}
\begin{définition}\cmn 各种颜色
\begin{déclaration} \étymologie{\papi{kʰa.dog}}\end{déclaration}\end{définition}
\begin{exemple}\jya raz ɲɯ-mpɕɤr tɕe khatoʁ ʑo ɣɤʑu\cmn 布很漂亮,是花色的\end{exemple}
\begin{exemple}\jya thaχtsa chɯ́-wɣ-βzu tɕe, tɤ-ri khatoʁ ʑo pjɯ-tu ra\cmn 制作花带的时候要有不同颜色的线\end{exemple}
\begin{exemple}\jya mɯntoʁ ɯ-mdoʁ ɲɯ-mpɕɤr khatoʁ ʑo ɣɤʑu\cmn 花的颜色很漂亮,各种颜色都有\end{exemple}\end{entrée}

\begin{entrée}
\vedette{\hypertarget{Ⓔkhatʂu}{\papi{ khatʂu}}}\markboth{khatʂu}{}\classe{n}
\begin{définition}\fra merci\end{définition}
\begin{définition}\cmn 谢谢
\begin{déclaration} \étymologie{\papi{kʰa.dro}}\end{déclaration}\end{définition}
\end{entrée}

\begin{entrée}
\vedette{\hypertarget{Ⓔkhatʂulɤfsɤm}{\papi{ khatʂulɤfsɤm}}}\markboth{khatʂulɤfsɤm}{}\classe{intj}
\begin{définition}\fra merci infiniment\end{définition}
\begin{définition}\cmn 万分感谢\end{définition}
\end{entrée}

\begin{entrée}
\vedette{\hypertarget{Ⓔkhɤβdɤr}{\papi{ khɤβdɤr}}}\markboth{khɤβdɤr}{}
\classe{n}
\begin{définition}\fra blague\end{définition}
\begin{définition}\cmn 玩笑\end{définition}
\begin{exemple}\jya khɤβdɤr tɤ-βzu-t-a\cmn 我开玩笑了\end{exemple}
\begin{relation-sémantique}\confer{
\hyperlink{Ⓔnɯkhɤβdɤr}{\textit{ \papi{nɯkhɤβdɤr}}}
}\end{relation-sémantique}\end{entrée}

\begin{entrée}
\vedette{\hypertarget{Ⓔkhɤβdi}{\papi{ khɤβdi}}}\markboth{khɤβdi}{}\classe{n}
\begin{définition}\fra belles paroles\end{définition}
\begin{définition}\cmn 好话\end{définition}
\begin{exemple}\jya ɯʑo kɯ khɤβdi tu-βze ɕti\cmn 他说得很好听(其实不知道是不是那回事)\end{exemple}
\begin{relation-sémantique}\synonyme{
\hyperlink{Ⓔkhɤmpɕɤr}{\textit{ \papi{khɤmpɕɤr}}}
}\end{relation-sémantique}\end{entrée}

\begin{entrée}
\vedette{\hypertarget{Ⓔkhɤβɣa}{\papi{ khɤβɣa}}}\markboth{khɤβɣa}{}\classe{n}
\begin{définition}\fra moulin à main\end{définition}
\begin{définition}\cmn 手磨\end{définition}
\begin{relation-sémantique}\confer{
\hyperlink{Ⓔkha}{\textit{ \papi{kha}}}
}\end{relation-sémantique}
\begin{relation-sémantique}\confer{
\hyperlink{Ⓔβɣa}{\textit{ \papi{βɣa}}}
}\end{relation-sémantique}
\end{entrée}

\begin{entrée}
\vedette{\hypertarget{Ⓔkhɤβrda}{\papi{ khɤβrda}}}\markboth{khɤβrda}{}\classe{n}
\begin{définition}\fra parole auspicieuse\end{définition}
\begin{définition}\cmn 吉利的话
\begin{déclaration} \étymologie{\papi{kʰa.brda}}\end{déclaration}\end{définition}
\begin{exemple}\jya khɤβrda kɯ-sna ta-βzu\cmn 他说了吉利的话\end{exemple}
\begin{exemple}\jya nɯnɯ ma-tɯ-ti ma khɤβrda mɤ-kɯ-sna ɲɯ-ŋu\cmn 你别这样说,这句话不吉利\end{exemple}\end{entrée}

\begin{entrée}
\vedette{\hypertarget{Ⓔkhɤβsa}{\papi{ khɤβsa}}}\markboth{khɤβsa}{}\classe{n}
\begin{définition}\fra beignet\end{définition}
\begin{définition}\cmn 油条\end{définition}\end{entrée}

\begin{entrée}
\vedette{\hypertarget{Ⓔkhɤβzaŋ}{\papi{ khɤβzaŋ}}}\markboth{khɤβzaŋ}{}
\classe{intj}
\begin{définition}\fra formule de politesse pour exprimer son arrivée\end{définition}
\begin{définition}\cmn 客人表明自己到来的客套话
\begin{déclaration}\use{主人说\stylefv{nɤ-tʂu},客人说\stylefv{khɤβzaŋ}}\end{déclaration}
\begin{déclaration} \étymologie{\papi{kʰa.bzaŋ}}\end{déclaration}\end{définition}\end{entrée}

\begin{entrée}
\vedette{\hypertarget{Ⓔkhɤcɤl}{\papi{ khɤcɤl}}}\markboth{khɤcɤl}{}\classe{n}
\begin{définition}\fra sujet de discussion\end{définition}
\begin{définition}\cmn 谈话内容\end{définition}
\end{entrée}

\begin{entrée}
\vedette{\hypertarget{Ⓔkhɤɕa}{\papi{ khɤɕa}}}\markboth{khɤɕa}{}\classe{n}
\begin{définition}\fra cerf (cervus elaphus kansuensis)\end{définition}
\begin{définition}\cmn 马鹿\end{définition}
\end{entrée}

\begin{entrée}
\vedette{\hypertarget{Ⓔkhɤɕkhɤr}{\papi{ khɤɕkhɤr}}}\markboth{khɤɕkhɤr}{}
\classe{n}
\begin{définition}\fra place du maître de maison, à l'est\end{définition}
\begin{définition}\cmn 主人坐的地方,往东方\end{définition}\end{entrée}

\begin{entrée}
\vedette{\hypertarget{Ⓔkhɤɕpi}{\papi{ khɤɕpi}}}\markboth{khɤɕpi}{}
\classe{n}
\begin{définition}\fra petit sac en cuir que l'on attache à la taille\end{définition}
\begin{définition}\cmn 拴在腰带上的皮包\end{définition}\end{entrée}

\begin{entrée}
\vedette{\hypertarget{Ⓔkhɤdaʁ}{\papi{ khɤdaʁ}}}\markboth{khɤdaʁ}{}
\classe{n}
\begin{définition}\fra khatag\end{définition}
\begin{définition}\cmn 哈达
\begin{déclaration} \étymologie{\papi{kʰa.btags}}\end{déclaration}\end{définition}
\begin{exemple}\jya khɤdaʁ lɤ-lat-a\cmn 我给了他哈达\end{exemple}
\begin{exemple}\jya sprɯskɯ kɯ a-khɤdaʁ tha-lɤt\cmn 活佛给了我哈达\end{exemple}\end{entrée}

\begin{entrée}
\vedette{\hypertarget{Ⓔkhɤdɤrdɤr}{\papi{ khɤdɤrdɤr}}}\markboth{khɤdɤrdɤr}{}\classe{n}
\begin{définition}\fra neige en grains\end{définition}
\begin{définition}\cmn 雪籽\end{définition}
\begin{exemple}\jya khɤdɤrdɤr pa-lɤt\cmn 下了雪籽\end{exemple}\end{entrée}

\begin{entrée}
\vedette{\hypertarget{Ⓔkhɤdi}{\papi{ khɤdi}}}\markboth{khɤdi}{}
\classe{n}
\begin{définition}\fra place de la maîtresse de maison, au nord\end{définition}
\begin{définition}\cmn 女主人坐的地方,往北方\end{définition}\end{entrée}

\begin{entrée}
\vedette{\hypertarget{Ⓔkhɤfɕɤt}{\papi{ khɤfɕɤt}}}\markboth{khɤfɕɤt}{}\classe{n}
\begin{définition}\fra prière\end{définition}
\begin{définition}\cmn 祈祷
\begin{déclaration} \étymologie{\papi{kʰa.bɕad}}\end{déclaration}\end{définition}
\begin{exemple}\jya ʑɯβdaʁ ɯ-ɕki khɤfɕɤt tɤ-βzu-t-a\cmn 我向山神祈祷了\end{exemple}\end{entrée}

\begin{entrée}
\vedette{\hypertarget{Ⓔkhɤjlɤn}{\papi{ khɤjlɤn}}}\markboth{khɤjlɤn}{}
\classe{n}
\begin{définition}\fra vœux\end{définition}
\begin{définition}\cmn 许愿
\begin{déclaration} \étymologie{\papi{kʰas.len}}\end{déclaration}\end{définition}
\begin{exemple}\jya kɤ-qur khɤjlɤn tɤ-nɯβzu-t-a ɕti\cmn 我答应帮他,我发誓要帮他\end{exemple}
\begin{relation-sémantique}\confer{
\hyperlink{Ⓔnɯkhɤjlɤn}{\textit{ \papi{nɯkhɤjlɤn}}}
}\end{relation-sémantique}\end{entrée}

\begin{entrée}
\vedette{\hypertarget{Ⓔkhɤjmu}{\papi{ khɤjmu}}}\markboth{khɤjmu}{}
\classe{n}
\begin{définition}\fra cuisine, premier étage\end{définition}
\begin{définition}\cmn 藏式房屋的第二楼(厨房)\end{définition}
\begin{exemple}\jya jiʑo kɯrɯ kha ɣɯ khɤjmu ɯ-ŋgɯ sɤ-ɤmdzɯ ʑakastaka tu tɕe khɤɕkhɤr kɤ-ti ci tu tɕe, nɯ tɕu tɤ-tɕɯ ra cho kɯ-ŋgro ra ku-omdzɯ-nɯ ŋu, ɕaŋlo kɤ-ti ci tu tɕe, nɯ tɕu rgɤrgɯn ra cho smi ɯ-kɯ-βlɯ ra ku-omdzɯ-nɯ ŋu, khɤdi nɯ tɕu tɕheme kɯ-nɯkhamu cho kha tɤ-mu nɯ ku-kɯ-ɤmdzɯ ŋu, tɕe nɯ sqhi nɯ pjɯ́-wɣ-nɤkhar ŋu, saŋdi kɤ-ti ci tu tɕe, nɯ ɯ-pɕoʁ nɯ tɕu si kɤ-βlɯ ɯ-spa ɯ-sɤ-ta ŋu, nɯ ɯ-pɕoʁ nɯ tɕu tɯrme kɯ-ɤmdzɯ me. kɤ-βlɯ ɯ-spa ɯ-sɤ-ta ŋu, nɯ ɯ-pɕoʁ nɯ tɕu tɯrme kɯ-ɤmdzɯ me.\cmn 
我们藏族的厨房里有各种座位,一种叫\stylefv{khɤɕkhɤr},是男人和贵宾的座位,一种叫\stylefv{ɕaŋlo},是老年人和烧火的人的座位,一种叫\stylefv{khɤdi},是做饭的女子和家庭主妇的座位,这样围着火上的三脚,另一种是\stylefv{saŋdi},是放烧火柴的地方,没有人在那里坐。
\end{exemple}
\begin{relation-sémantique}\confer{
\hyperlink{Ⓔnɯkhɤrŋgɯ}{\textit{ \papi{nɯkhɤrŋgɯ}}}
}\end{relation-sémantique}\end{entrée}

\begin{entrée}
\vedette{\hypertarget{Ⓔkhɤkɤcu}{\papi{ khɤkɤcu}}}\markboth{khɤkɤcu}{}\classe{n}
\begin{définition}\fra est de la maison\end{définition}
\begin{définition}\cmn 房子的东边\end{définition}
\begin{relation-sémantique}\confer{
\hyperlink{Ⓔkha}{\textit{ \papi{kha}}}
}\end{relation-sémantique}
\begin{relation-sémantique}\confer{
\hyperlink{Ⓔɯ-kɤcu}{\textit{ \papi{ɯ-kɤcu}}}
}\end{relation-sémantique}
\end{entrée}

\begin{entrée}
\vedette{\hypertarget{Ⓔkhɤku raŋri}{\papi{ khɤku raŋri}}}\markboth{khɤku raŋri}{}\classe{n}
\begin{définition}\fra chaque maison\end{définition}
\begin{définition}\cmn 每一户\end{définition}\end{entrée}

\begin{entrée}
\vedette{\hypertarget{Ⓔkhɤkɯm}{\papi{ khɤkɯm}}}\markboth{khɤkɯm}{}\classe{n}
\begin{définition}\fra entrée de la maison\end{définition}
\begin{définition}\cmn 门口\end{définition}
\begin{relation-sémantique}\confer{
\hyperlink{Ⓔkha}{\textit{ \papi{kha}}}
}\end{relation-sémantique}
\begin{relation-sémantique}\confer{
\hyperlink{Ⓔkɯm}{\textit{ \papi{kɯm}}}
}\end{relation-sémantique}\end{entrée}

\begin{entrée}
\vedette{\hypertarget{Ⓔkhɤlɤβ}{\papi{ khɤlɤβ}}}\markboth{khɤlɤβ}{}\classe{n}
\begin{définition}\fra couvercle\end{définition}
\begin{définition}\cmn 锅的盖子
\begin{déclaration} \étymologie{\papi{kʰa.leb}}\end{déclaration}\end{définition}\end{entrée}

\begin{entrée}
\vedette{\hypertarget{Ⓔkhɤli,rgi}{\papi{ khɤli,rgi}}}\markboth{khɤli,rgi}{}
\paradigme{\textit{dir :} \jya tɤ-}
\begin{définition}\fra avoir une bonne renommée\end{définition}
\begin{définition}\cmn 名声好;受人尊重;受人信任
\begin{déclaration}\use{古语}\end{déclaration}\end{définition}
\begin{exemple}\jya ɯ-khɤli ɲɯ-rgi\cmn 他名声很好\end{exemple}
\begin{relation-sémantique}\ComponentA{\classe{n}
 \papi{khɤli}
}\end{relation-sémantique}
\begin{relation-sémantique}\ComponentB{\classe{vs}
 \papi{rgi}
}\end{relation-sémantique}\end{entrée}

\begin{entrée}
\vedette{\hypertarget{Ⓔkhɤmdu}{\papi{ khɤmdu}}}\markboth{khɤmdu}{}
\classe{n}
\begin{définition}\fra rênes\end{définition}
\begin{définition}\cmn 缰绳\end{définition}\end{entrée}

\begin{entrée}
\vedette{\hypertarget{Ⓔkhɤmɬa}{\papi{ khɤmɬa}}}\markboth{khɤmɬa}{}
\classe{n}
\begin{définition}\fra cérémonie\end{définition}
\begin{définition}\cmn 庆祝的仪式
\begin{déclaration} \étymologie{\papi{kʰams.lha?}}\end{déclaration}\end{définition}\end{entrée}

\begin{entrée}
\vedette{\hypertarget{Ⓔkhɤmpɕɤr}{\papi{ khɤmpɕɤr}}}\markboth{khɤmpɕɤr}{}\classe{n}
\begin{définition}\fra belles paroles\end{définition}
\begin{définition}\cmn 好话\end{définition}
\begin{exemple}\jya ɯʑo kɯ khɤmpɕɤr tu-βze ɕti\cmn 他说得很好听(其实不知道是不是那回事)\end{exemple}
\begin{relation-sémantique}\synonyme{
\hyperlink{Ⓔkhɤβdi}{\textit{ \papi{khɤβdi}}}
}\end{relation-sémantique}\end{entrée}

\begin{entrée}
\vedette{\hypertarget{Ⓔkhɤndɤcu}{\papi{ khɤndɤcu}}}\markboth{khɤndɤcu}{}\classe{n}
\begin{définition}\fra ouest de la maison\end{définition}
\begin{définition}\cmn 房子的西边\end{définition}
\begin{relation-sémantique}\confer{
\hyperlink{Ⓔkha}{\textit{ \papi{kha}}}
}\end{relation-sémantique}
\begin{relation-sémantique}\confer{
\hyperlink{Ⓔɯ-ndɤcu}{\textit{ \papi{ɯ-ndɤcu}}}
}\end{relation-sémantique}
\end{entrée}

\begin{entrée}
\vedette{\hypertarget{Ⓔkhɤndɯn}{\papi{ khɤndɯn}}}\markboth{khɤndɯn}{}\classe{n}
\begin{définition}\fra lecture de sutra\end{définition}
\begin{définition}\cmn 念经
\begin{déclaration} \étymologie{\papi{kʰa.ⁿdon}}\end{déclaration}\end{définition}\end{entrée}

\begin{entrée}
\vedette{\hypertarget{Ⓔkhɤndzo}{\papi{ khɤndzo}}}\markboth{khɤndzo}{}
\classe{n}
\begin{définition}\fra étuve\end{définition}
\begin{définition}\cmn 蒸笼\end{définition}\end{entrée}

\begin{entrée}
\vedette{\hypertarget{Ⓔkhɤntshɤm}{\papi{ khɤntshɤm}}}\markboth{khɤntshɤm}{}\classe{n}
\begin{définition}\fra limite\end{définition}
\begin{définition}\cmn 界限;交界地方
\begin{déclaration} \étymologie{\papi{kʰa.mtsʰams}}\end{déclaration}\end{définition}
\begin{exemple}\jya tɯ-kɤrme ɯ-khɤntshɤm\cmn 开始长头发的地方\end{exemple}\end{entrée}

\begin{entrée}
\vedette{\hypertarget{Ⓔkhɤɴqra}{\papi{ khɤɴqra}}}\markboth{khɤɴqra}{}\classe{n}
\begin{définition}\fra maison en ruine\end{définition}
\begin{définition}\cmn 烂房子\end{définition}
\begin{relation-sémantique}\confer{
\hyperlink{Ⓔkha}{\textit{ \papi{kha}}}
}\end{relation-sémantique}
\begin{relation-sémantique}\confer{
\hyperlink{Ⓔɯ-ɴqra}{\textit{ \papi{ɯ-ɴqra}}}
}\end{relation-sémantique}\end{entrée}

\begin{entrée}
\vedette{\hypertarget{Ⓔkhɤpa}{\papi{ khɤpa}}}\markboth{khɤpa}{}
\classe{n}
\begin{définition}\fra rez de chaussée\end{définition}
\begin{définition}\cmn 一楼\end{définition}
\begin{exemple}\jya khɤpa ri rɤʑi-a\cmn 我在院子里\end{exemple}
\begin{relation-sémantique}\confer{
\hyperlink{Ⓔkha}{\textit{ \papi{kha}}}
}\end{relation-sémantique}
\begin{relation-sémantique}\confer{
 \papi{ɯ-pa}
}\end{relation-sémantique}\end{entrée}

\begin{entrée}
\vedette{\hypertarget{Ⓔkhɤphrɯ}{\papi{ khɤphrɯ}}}\markboth{khɤphrɯ}{}\classe{n}
\begin{définition}\fra action d'asperger de l'eau\end{définition}
\begin{définition}\cmn 喷水
\begin{déclaration} \étymologie{\papi{kʰa.pʰru}}\end{déclaration}\end{définition}
\begin{exemple}\jya khɤphrɯ ma-tɤ-tɯ-lɤt\cmn 你不要喷水\end{exemple}
\begin{relation-sémantique}\confer{
\hyperlink{Ⓔnɯkhɤphrɯ}{\textit{ \papi{nɯkhɤphrɯ}}}
}\end{relation-sémantique}\end{entrée}

\begin{entrée}
\vedette{\hypertarget{Ⓔkhɤpɯ}{\papi{ khɤpɯ}}}\markboth{khɤpɯ}{}\classe{n}
\begin{définition}\fra petite cabane\end{définition}
\begin{définition}\cmn 小屋子\end{définition}
\begin{relation-sémantique}\confer{
\hyperlink{Ⓔkha}{\textit{ \papi{kha}}}
}\end{relation-sémantique}
\begin{relation-sémantique}\confer{
\hyperlink{Ⓔtɤ-pɯ}{\textit{ \papi{tɤ-pɯ}}}
}\end{relation-sémantique}\end{entrée}

\begin{entrée}
\vedette{\hypertarget{Ⓔkhɤru}{\papi{ khɤru}}}\markboth{khɤru}{}
\classe{n}
\begin{définition}\fra porte de la cuisine\end{définition}
\begin{définition}\cmn 厨房的门\end{définition}\end{entrée}

\begin{entrée}
\vedette{\hypertarget{Ⓔkhɤrka}{\papi{ khɤrka}}}\markboth{khɤrka}{}
\classe{n}
\begin{définition}\fra plafond\end{définition}
\begin{définition}\cmn 天花板\end{définition}\end{entrée}

\begin{entrée}
\vedette{\hypertarget{Ⓔkhɤrlɤn}{\papi{ khɤrlɤn}}}\markboth{khɤrlɤn}{}\classe{n}
\begin{définition}\fra construction, réparation d'une maison\end{définition}
\begin{définition}\cmn 装修\end{définition}
\begin{exemple}\jya khɤrlɤn tɤ-βzu-j\cmn 我们修房子了\end{exemple}
\begin{relation-sémantique}\confer{
\hyperlink{Ⓔrɯkhɤrlɤn}{\textit{ \papi{rɯkhɤrlɤn}}}
}\end{relation-sémantique}\end{entrée}

\begin{entrée}
\vedette{\hypertarget{Ⓔkhɤrma}{\papi{ khɤrma}}}\markboth{khɤrma}{}\classe{n}
\begin{définition}\fra injure\end{définition}
\begin{définition}\cmn 咒人的话
\begin{déclaration} \étymologie{\papi{*kʰa.rma}}\end{déclaration}\end{définition}
\begin{exemple}\jya a-khɤrma pa-βzu\cmn 他咒了我\end{exemple}
\begin{relation-sémantique}\confer{
\hyperlink{Ⓔsɯkhɤrma}{\textit{ \papi{sɯkhɤrma}}}
}\end{relation-sémantique}
\begin{relation-sémantique}\confer{
\hyperlink{Ⓔɯ-rɟa}{\textit{ \papi{ɯ-rɟa}}}
}\end{relation-sémantique}\end{entrée}

\begin{entrée}
\vedette{\hypertarget{Ⓔkhɤrmi}{\papi{ khɤrmi}}}\markboth{khɤrmi}{}\classe{n}
\begin{définition}\fra nom de la maison\end{définition}
\begin{définition}\cmn 房名\end{définition}
\begin{relation-sémantique}\confer{
\hyperlink{Ⓔkha}{\textit{ \papi{kha}}}
}\end{relation-sémantique}
\begin{relation-sémantique}\confer{
\hyperlink{Ⓔtɤ-rmi}{\textit{ \papi{tɤ-rmi}}}
}\end{relation-sémantique}
\end{entrée}

\begin{entrée}
\vedette{\hypertarget{Ⓔkhɤrtsɤɣ}{\papi{ khɤrtsɤɣ}}}\markboth{khɤrtsɤɣ}{}\classe{n}
\begin{définition}\fra étage\end{définition}
\begin{définition}\cmn 楼房\end{définition}
\begin{relation-sémantique}\confer{
\hyperlink{Ⓔkha}{\textit{ \papi{kha}}}
}\end{relation-sémantique}
\begin{relation-sémantique}\confer{
\hyperlink{Ⓔtɤ-rtsɤɣ}{\textit{ \papi{tɤ-rtsɤɣ}}}
}\end{relation-sémantique}
\end{entrée}

\begin{entrée}
\vedette{\hypertarget{Ⓔkhɤrɯm}{\papi{ khɤrɯm}}}\markboth{khɤrɯm}{}
\classe{n}
\begin{définition}\fra ulcère sur la bouche\end{définition}
\begin{définition}\cmn 嘴上的疮\end{définition}
\begin{exemple}\jya tɯ-mtɕhi ʑmbɤr nɯ khɤrɯm\end{exemple}
\begin{exemple}\jya tɯ-ʑo staʁ kɯ-χtso ra nɯ-khɯ-tsa kú-wɣ-ntɕhoz tɕe, tɯ-mtɕhi maŋtaʁ nɯ tɕu khɤrɯm ɲɯ-ɬoʁ, tɯ-ʑo staʁ kɯ-χtso ra nɯ-khɯtsa kú-wɣ-ntɕhoz tɕe, tɯ-mtɕhi maŋpa nɯ tɕu khɤrɯm ɲɯ-ɬoʁ ŋu, tɯ-ʑo cho kɯ-naχtɕɯɣ ra nɯ-khɯtsa kú-wɣ-ntɕhoz tɕe, tɯ-mtɕhi ɯ-rkɯ nɯ tɕu khɤrɯm ɲɯ-ɬoʁ ŋu tu-kɯ-ti ɲɯ-ŋu\cmn 
人家说,当你用别人的碗时,就会染上\stylefv{khɤrɯm}这种病,如果拥有碗的那个人比你干净,痘痘长在嘴唇的上面;比你脏,痘痘就会长在嘴唇的下面;跟你一样干净,就会长在嘴唇的两边
\end{exemple}
\begin{exemple}\jya kɯmaʁ tɯrme ra nɯ-khɯtsa kú-wɣ-tɕhoz tɕe tɯ-mtɕhi ɯ-taʁ zmbɤr ɲɯ-ɬoʁ ŋgrɤl tɕe, nɯ ʑmbɤr nɯ khɤrɯm rmi\cmn 
用了别人的碗时嘴上会长一种疮,这种疮叫\stylefv{khɤrɯm}
\end{exemple}\end{entrée}

\begin{entrée}
\vedette{\hypertarget{Ⓔkhɤʁɤri}{\papi{ khɤʁɤri}}}\markboth{khɤʁɤri}{}\classe{n}
\begin{définition}\fra devant la maison\end{définition}
\begin{définition}\cmn 房子的前面\end{définition}
\begin{relation-sémantique}\confer{
\hyperlink{Ⓔkha}{\textit{ \papi{kha}}}
}\end{relation-sémantique}
\begin{relation-sémantique}\confer{
\hyperlink{Ⓔɯ-ʁɤri}{\textit{ \papi{ɯ-ʁɤri}}}
}\end{relation-sémantique}
\end{entrée}

\begin{entrée}
\vedette{\hypertarget{Ⓔkhɤsnɯm}{\papi{ khɤsnɯm}}}\markboth{khɤsnɯm}{}\classe{n}
\begin{définition}\fra humidification avec la salive\end{définition}
\begin{définition}\cmn 用口水弄湿
\begin{déclaration} \étymologie{\papi{*kʰa.snum}}\end{déclaration}\end{définition}
\begin{exemple}\jya tɤfsɤri ɯ-taʁ khɤsnɯm thɯ-lat-a\cmn 我用口水把麻绳弄湿\end{exemple}
\begin{relation-sémantique}\confer{
\hyperlink{Ⓔnɯkhɤsnɯm}{\textit{ \papi{nɯkhɤsnɯm}}}
}\end{relation-sémantique}\end{entrée}

\begin{entrée}
\vedette{\hypertarget{Ⓔkhɤsta}{\papi{ khɤsta}}}\markboth{khɤsta}{}\classe{n}
\begin{définition}\fra fondations\end{définition}
\begin{définition}\cmn 房基(准备修房子的地方;把房子拆下来了以后,有过房子的地方)\end{définition}
\begin{relation-sémantique}\confer{
\hyperlink{Ⓔkha}{\textit{ \papi{kha}}}
}\end{relation-sémantique}
\begin{relation-sémantique}\confer{
\hyperlink{Ⓔtɯ-sta}{\textit{ \papi{tɯ-sta}}}
}\end{relation-sémantique}\end{entrée}

\begin{entrée}
\vedette{\hypertarget{Ⓔkhɤt}{\papi{ khɤt}}}\markboth{khɤt}{}
\classe{vt}
\paradigme{\textit{dir :} \jya tɤ-}\acception{1}
\begin{définition}\fra aller partout\end{définition}
\begin{définition}\cmn 到处逛\end{définition}
\begin{exemple}\jya aʁɤndɯndɤt kɤ-nɤmɲo ɕ-to-khɤt\cmn 他到处去观看了\end{exemple}
\begin{exemple}\jya alo ʑɯmkhɤm ɕ-ta-khɤt\cmn 他山上去逛了\end{exemple}\acception{2}
\begin{définition}\fra faire une action pendant longtemps ou à plusieurs reprises\end{définition}
\begin{définition}\cmn 做很多次;发生很多次;持续很久
\begin{déclaration}\use{跟施事格连用}\end{déclaration}\end{définition}
\begin{exemple}\jya a-ɕki tɯjʁo kɯ ta-khɤt ʑo\cmn 他骂我骂了很多次\end{exemple}
\begin{exemple}\jya ta-ma kɯ ta-khɤt ʑo\cmn 他劳动得多\end{exemple}
\begin{exemple}\jya ndzɤtshi kɯ ta-khɤt ʑo\cmn 他吃得多\end{exemple}
\begin{exemple}\jya khɤcɤl kɯ tɤ-khat-a ʑo\cmn 我讲了很久\end{exemple}\end{entrée}

\begin{entrée}
\vedette{\hypertarget{Ⓔkhɤtaʁ}{\papi{ khɤtaʁ}}}\markboth{khɤtaʁ}{}\classe{n}
\begin{définition}\fra deuxième étage, au-dessus de la cuisine\end{définition}
\begin{définition}\cmn 藏式房屋的第二楼\end{définition}
\begin{relation-sémantique}\confer{
\hyperlink{Ⓔkha}{\textit{ \papi{kha}}}
}\end{relation-sémantique}
\begin{relation-sémantique}\confer{
\hyperlink{ⒺtaʁⒽ3}{\textit{ \papi{taʁ3}}}
}\end{relation-sémantique}
\end{entrée}

\begin{entrée}
\vedette{\hypertarget{Ⓔkhɤtɤcu}{\papi{ khɤtɤcu}}}\markboth{khɤtɤcu}{}\classe{n}
\begin{définition}\fra couloir\end{définition}
\begin{définition}\cmn 走廊\end{définition}
\end{entrée}

\begin{entrée}
\vedette{\hypertarget{Ⓔkhɤtɕɯ}{\papi{ khɤtɕɯ}}}\markboth{khɤtɕɯ}{}\classe{n}
\begin{définition}\fra petite chambre\end{définition}
\begin{définition}\cmn 小房间\end{définition}
\end{entrée}

\begin{entrée}
\vedette{\hypertarget{Ⓔkhɤthɤβ}{\papi{ khɤthɤβ}}}\markboth{khɤthɤβ}{}\classe{n}\acception{1}
\begin{définition}\fra dans le village, dans la rue\end{définition}
\begin{définition}\cmn 街上,村子里\end{définition}\acception{2}
\begin{définition}\fra au bas de l'immeuble\end{définition}
\begin{définition}\cmn 楼下\end{définition}
\begin{exemple}\jya aki khɤthɤβ ʑo pɯ-azɣɯt-a\cmn 我已经到楼下了\end{exemple}
\begin{relation-sémantique}\confer{
\hyperlink{Ⓔkha}{\textit{ \papi{kha}}}
}\end{relation-sémantique}
\begin{relation-sémantique}\confer{
\hyperlink{Ⓔthɤβ}{\textit{ \papi{thɤβ}}}
}\end{relation-sémantique}\end{entrée}

\begin{entrée}
\vedette{\hypertarget{Ⓔkhɤtsa}{\papi{ khɤtsa}}}\markboth{khɤtsa}{}\classe{n}
\begin{définition}\fra saleté entre les dents\end{définition}
\begin{définition}\cmn 牙垢
\begin{déclaration} \étymologie{\papi{kʰa.tsa}}\end{déclaration}\end{définition}
\end{entrée}

\begin{entrée}
\vedette{\hypertarget{Ⓔkhɤtshoʁ}{\papi{ khɤtshoʁ}}}\markboth{khɤtshoʁ}{}\classe{n}
\begin{définition}\fra type de pas d'aiguille\end{définition}
\begin{définition}\cmn 缝针的方法\end{définition}\end{entrée}

\begin{entrée}
\vedette{\hypertarget{Ⓔkhɤwɯ}{\papi{ khɤwɯ}}}\markboth{khɤwɯ}{}
\classe{n}
\begin{définition}\fra troisième étage, où l'on dort\end{définition}
\begin{définition}\cmn 四楼,睡觉的地方\end{définition}\end{entrée}

\begin{entrée}
\vedette{\hypertarget{Ⓔkhɤxtu}{\papi{ khɤxtu}}}\markboth{khɤxtu}{}
\classe{n}
\begin{définition}\fra terrasse en haut des maisons tibétaines, toit\end{définition}
\begin{définition}\cmn 屋顶平台【房背】\end{définition}\end{entrée}

\begin{entrée}
\vedette{\hypertarget{Ⓔkhɤxtɤlwɤt}{\papi{ khɤxtɤlwɤt}}}\markboth{khɤxtɤlwɤt}{}\classe{n}
\begin{définition}\fra avant-toit\end{définition}
\begin{définition}\cmn 屋檐\end{définition}
\begin{exemple}\jya khɤxtɤlwɤt ɯ-pa\cmn 屋檐下\end{exemple}\end{entrée}

\begin{entrée}
\vedette{\hypertarget{Ⓔkhɤxtɤmbro}{\papi{ khɤxtɤmbro}}}\markboth{khɤxtɤmbro}{}
\classe{n}
\begin{définition}\fra terrasse la plus haute\end{définition}
\begin{définition}\cmn 最高的房背\end{définition}\end{entrée}

\begin{entrée}
\vedette{\hypertarget{Ⓔkhɤxtɤndo}{\papi{ khɤxtɤndo}}}\markboth{khɤxtɤndo}{}
\classe{n}
\begin{définition}\fra bordure du toit\end{définition}
\begin{définition}\cmn 房背的边缘\end{définition}\end{entrée}

\begin{entrée}
\vedette{\hypertarget{Ⓔkhɤxtɤndorɤm}{\papi{ khɤxtɤndorɤm}}}\markboth{khɤxtɤndorɤm}{}\classe{n}
\begin{définition}\fra parapet du toit\end{définition}
\begin{définition}\cmn 房背上的栏杆\end{définition}
\begin{exemple}\jya khɤxtɤndo laχtsɯ kú-wɣ-lɤt tɕe rorʁe ʁnɯ-ldʑa ntsɯ kú-wɣ-saχɕɯβ tɕe ɯ-rchɤβ nɯtɕu tɤrɤm kú-wɣ-sɤʑɯrja tɕe khɤxtu kú-wɣ-sɤɣur tɕe nɯ tɤrɤm nɯ khɤxtɤndorɤm rmi\cmn 
在房背的边缘排几根小柱头,穿上成双的横杆,在中间插上一排木板,把房背拦住,这种木板叫\stylefv{khɤxtɤndorɤm}
\end{exemple}\end{entrée}

\begin{entrée}
\vedette{\hypertarget{Ⓔkhɤχpi}{\papi{ khɤχpi}}}\markboth{khɤχpi}{}\classe{n}
\begin{définition}\fra proverbe\end{définition}
\begin{définition}\cmn 谚语,俗话\end{définition}
\begin{exemple}\jya khɤχpi kú-wɣ-ta tɕe .... tu-kɯ-ti ɲɯ-ŋgrɤl\cmn 俗话说:\end{exemple}
\begin{relation-sémantique}\confer{
\hyperlink{Ⓔχpi}{\textit{ \papi{χpi}}}
}\end{relation-sémantique}\end{entrée}

\begin{entrée}
\vedette{\hypertarget{Ⓔkhɤzɟi}{\papi{ khɤzɟi}}}\markboth{khɤzɟi}{}\classe{n}
\begin{définition}\fra sac pour nourrir les chevaux, que l'on attache à sa bouche\end{définition}
\begin{définition}\cmn 喂马的食料袋,系在嘴上
\begin{déclaration} \étymologie{\papi{kʰa.sgʲe}}\end{déclaration}\end{définition}
\end{entrée}

\begin{entrée}
\vedette{\hypertarget{Ⓔkhe}{\papi{ khe}}}\markboth{khe}{}\classe{vs}\acception{1}
\paradigme{\textit{dir :} \jya nɯ-}
\begin{définition}\fra stupide\end{définition}
\begin{définition}\cmn 蠢\end{définition}
\begin{exemple}\jya kɯ-khe ci ɲɯ-ŋu\cmn 他是笨蛋\end{exemple}
\begin{exemple}\jya kɯ-khe ci ɲɯ-tɯ-ŋu\cmn 你是笨蛋\end{exemple}
\begin{exemple}\jya tɯrme ɲɯ-khe\cmn 那个人很笨\end{exemple}\acception{2}
\paradigme{\textit{dir :} \jya tɤ-}
\begin{définition}\fra sombre\end{définition}
\begin{définition}\cmn 天阴\end{définition}
\begin{exemple}\jya tɯ-mɯ ɲɯ-khe\cmn 天很阴\end{exemple}
\begin{exemple}\jya tɯ-mɯ to-khe\cmn 天气变得不好\end{exemple}
\begin{exemple}\jya ɯ-ɲɤm ɲɯ-khe\cmn 他很瘦\end{exemple}\begin{sous-entrée}
\vedette{\hypertarget{}{\papi{ ɣɤkhe}}}\markboth{ɣɤkhe}{}\classe{vt}
\paradigme{\textit{dir :} \jya nɯ-}
\begin{définition}\fra considérer comme un imbécile\end{définition}
\begin{définition}\cmn 把……当做傻瓜,诬蔑\end{définition}
\begin{exemple}\jya ma-pɯ-kɯ-ɣɤkhe-a ma ɲɯ-sɤɣdɯɣ\cmn 你不要把我当傻瓜,很讨厌\end{exemple}
\begin{exemple}\jya fsapaʁ ra nɯ-ɲɤm ʑo ɲɯ-ɣɤkhe ŋgrɤl\cmn 蜱令牲畜变瘦\end{exemple}
\end{sous-entrée}\begin{sous-entrée}
\vedette{\hypertarget{}{\papi{ sɤzɣɤkhe}}}\markboth{sɤzɣɤkhe}{}\classe{vi}
\begin{définition}\fra considérer les gens comme des imbéciles\end{définition}
\begin{définition}\cmn 把别人当做傻瓜,诬蔑别人\end{définition}
\begin{relation-sémantique}\confer{
\hyperlink{Ⓔtɤkhe}{\textit{ \papi{tɤkhe}}}
}\end{relation-sémantique}
\begin{relation-sémantique}\confer{
\hyperlink{Ⓔnɯɲɤmkhe}{\textit{ \papi{nɯɲɤmkhe}}}
}\end{relation-sémantique}
\end{sous-entrée}\begin{sous-entrée}
\vedette{\hypertarget{}{\papi{ ʑɣɤɣɤkhe}}}\markboth{ʑɣɤɣɤkhe}{}\classe{vi}
\paradigme{\textit{dir :} \jya pɯ-}
\begin{définition}\fra se moquer de soi-même, faire de l'autodérision\end{définition}
\begin{définition}\cmn 贬低自己\end{définition}
\begin{exemple}\jya ma-pɯ-tɯ-ʑɣɤɣɤkhe\cmn 不要贬低自己\end{exemple}
\end{sous-entrée}\end{entrée}

\begin{entrée}
\vedette{\hypertarget{Ⓔkhi}{\papi{ khi}}}\markboth{khi}{}\classe{part}
\begin{définition}\fra dit-on (ouï-dire)\end{définition}
\begin{définition}\cmn 据说\end{définition}
\end{entrée}

\begin{entrée}
\vedette{\hypertarget{Ⓔkhiɤɣ}{\papi{ khiɤɣ}}}\markboth{khiɤɣ}{}\classe{idph.1}
\begin{définition}\fra bruit de glissement\end{définition}
\begin{définition}\cmn 滑动的声音\end{définition}
\begin{exemple}\jya khiɤɣ ʑo thɯ-ŋgio-a\cmn 我嗞溜一声就滑倒了\end{exemple}\end{entrée}

\begin{entrée}
\vedette{\hypertarget{Ⓔkhiɤt}{\papi{ khiɤt}}}\markboth{khiɤt}{}\classe{idph.1}
\begin{définition}\fra bruit de glissement\end{définition}
\begin{définition}\cmn 滑下来的声音\end{définition}
\begin{exemple}\jya khiɤt ʑo ɲɯ-ti tɕe pɯ-ndʐaβ-a\cmn 我嗞溜一声就摔倒了\end{exemple}\end{entrée}

\begin{entrée}
\vedette{\hypertarget{Ⓔkhikhio}{\papi{ khikhio}}}\markboth{khikhio}{}\classe{n}
\begin{définition}\fra rumeurs\end{définition}
\begin{définition}\cmn 传言;道听途说\end{définition}\end{entrée}

\begin{entrée}
\vedette{\hypertarget{Ⓔkhipatsɯt}{\papi{ khipatsɯt}}}\markboth{khipatsɯt}{}\classe{n}
\begin{définition}\fra une espèce de chien\end{définition}
\begin{définition}\cmn 哈巴狗\end{définition}\end{entrée}

\begin{entrée}
\vedette{\hypertarget{Ⓔkhulu}{\papi{ khulu}}}\markboth{khulu}{}\classe{n}
\begin{définition}\fra année du tigre\end{définition}
\begin{définition}\cmn 虎年\end{définition}
\begin{relation-sémantique}\confer{
\hyperlink{Ⓔkhu}{\textit{ \papi{khu}}}
}\end{relation-sémantique}
\end{entrée}

\begin{entrée}
\vedette{\hypertarget{ⒺkhoⒽ2}{\papi{ kho}}}\markboth{kho}{}\homonyme{2}
\classe{n}
\begin{définition}\fra chambre\end{définition}
\begin{définition}\cmn 房间
\begin{déclaration} \étymologie{\papi{kʰaŋ}}\end{déclaration}\end{définition}\end{entrée}

\begin{entrée}
\vedette{\hypertarget{ⒺkhoⒽ1}{\papi{ kho}}}\markboth{kho}{}\homonyme{1}\classe{vt}
\paradigme{\textit{dir :} \jya nɯ-}
\begin{définition}\fra donner, passer, transmettre\end{définition}
\begin{définition}\cmn 给;递给;传\end{définition}
\begin{exemple}\jya nɯ-khɤm\cmn 递给(我)吧\end{exemple}
\begin{exemple}\jya ɕ-kɤ-khɤm\cmn 你去给(他)\end{exemple}
\begin{exemple}\jya a-tɕha nɯ-khɤm\cmn 把(我需要的消息)告诉我\end{exemple}
\begin{exemple}\jya ɯʑo kɯ a-sci ɲɯ-khɤm ra\cmn 他要还给我\end{exemple}
\begin{relation-sémantique}\confer{
\hyperlink{Ⓔsɯkho}{\textit{ \papi{sɯkho}}}
}\end{relation-sémantique}
\begin{relation-sémantique}\confer{
\hyperlink{Ⓔamɟɤkho}{\textit{ \papi{amɟɤkho}}}
}\end{relation-sémantique}\begin{sous-entrée}
\vedette{\hypertarget{}{\papi{ ʑɣɤkho}}}\markboth{ʑɣɤkho}{}\classe{vi}
\begin{définition}\ 
\begin{déclaration}\grammar{refl}\end{déclaration}\end{définition}
\begin{définition}\fra se donner à\end{définition}
\begin{définition}\cmn 把自己交给\end{définition}
\begin{exemple}\jya nɤʑo ʁgra ɯ-jaʁ nɯtɕu ɲɯ-tɯ-nɯ-ʑɣɤkho ʑo ɯ-mɤ-kɯ-ɕti-ci?\cmn 你是不是把自己交到敌人的手里了?\end{exemple}
\end{sous-entrée}\end{entrée}

\begin{entrée}
\vedette{\hypertarget{Ⓔkhoŋdaʁ}{\papi{ khoŋdaʁ}}}\markboth{khoŋdaʁ}{}\classe{n}
\begin{définition}\fra ancêtre\end{définition}
\begin{définition}\cmn 祖宗
\begin{déclaration} \étymologie{\papi{kʰaŋ.bdag}}\end{déclaration}\end{définition}
\end{entrée}

\begin{entrée}
\vedette{\hypertarget{Ⓔkhoŋrɤl}{\papi{ khoŋrɤl}}}\markboth{khoŋrɤl}{}\classe{n}
\begin{définition}\fra arbre creux\end{définition}
\begin{définition}\cmn 空心树\begin{déclaration} \étymologie{\papi{kʰoŋ.ral}}\end{déclaration}\end{définition}
\begin{exemple}\jya si khoŋrɤl tɤ-kɯ-ɤri\cmn 空心树\end{exemple}\end{entrée}

\begin{entrée}
\vedette{\hypertarget{Ⓔkhorca}{\papi{ khorca}}}\markboth{khorca}{}\classe{n}
\begin{définition}\fra sac à dos fait de peau de veau\end{définition}
\begin{définition}\cmn 小牛皮整体地剥下来,缝成装行李的背包\end{définition}
\end{entrée}

\begin{entrée}
\vedette{\hypertarget{Ⓔkhru}{\papi{ khru}}}\markboth{khru}{}
\classe{n}
\begin{définition}\fra fer blanc\end{définition}
\begin{définition}\cmn 生铁
\begin{déclaration} \étymologie{\papi{kʰro}}\end{déclaration}\end{définition}\end{entrée}

\begin{entrée}
\vedette{\hypertarget{Ⓔkhra}{\papi{ khra}}}\markboth{khra}{}\classe{vt}
\paradigme{\textit{dir :} \jya pɯ-}
\begin{définition}\fra faire une marque\end{définition}
\begin{définition}\cmn 划一刀\end{définition}
\begin{exemple}\jya pɯ-khra-t-a\cmn 我划了一刀\end{exemple}
\begin{exemple}\jya ɯ-ftaʁ tɤ-βzu-t-a tɕe pɯ-khra-t-a\cmn 我划了一刀做记号了\end{exemple}
\begin{relation-sémantique}\confer{
\hyperlink{Ⓔakhra}{\textit{ \papi{akhra}}}
}\end{relation-sémantique}
\begin{relation-sémantique}\confer{
\hyperlink{Ⓔsɤkhra}{\textit{ \papi{sɤkhra}}}
}\end{relation-sémantique}\end{entrée}

\begin{entrée}
\vedette{\hypertarget{Ⓔkhrakhra}{\papi{ khrakhra}}}\markboth{khrakhra}{}\classe{n}
\begin{définition}\fra filet\end{définition}
\begin{définition}\cmn 网\end{définition}\end{entrée}

\begin{entrée}
\vedette{\hypertarget{Ⓔkhrala}{\papi{ khrala}}}\markboth{khrala}{}\classe{n}
\begin{définition}\fra chien au pelage bariolé\end{définition}
\begin{définition}\cmn 身上有花斑的狗
\begin{déclaration} \étymologie{\papi{*kʰra.lʷa}}\end{déclaration}\end{définition}\end{entrée}

\begin{entrée}
\vedette{\hypertarget{Ⓔkhramba}{\papi{ khramba}}}\markboth{khramba}{}\classe{n}
\begin{définition}\fra mensonge\end{définition}
\begin{définition}\cmn 谎言
\begin{déclaration} \étymologie{\papi{kʰram.ba}}\end{déclaration}\end{définition}
\begin{exemple}\jya khramba to-βzu\cmn 他说了谎话\end{exemple}
\begin{exemple}\jya khramba rɟɤlpu\cmn 假的国王\end{exemple}
\begin{relation-sémantique}\confer{
\hyperlink{Ⓔnɯkhramba}{\textit{ \papi{nɯkhramba}}}
}\end{relation-sémantique}
\begin{relation-sémantique}\confer{
\hyperlink{Ⓔrɯkhramba}{\textit{ \papi{rɯkhramba}}}
}\end{relation-sémantique}
\begin{sous-entrée}
\vedette{\hypertarget{}{\papi{ khramba,βzu}}}\markboth{khramba,βzu}{}
\begin{définition}\fra faire semblant\end{définition}
\begin{définition}\cmn 假装\end{définition}
\begin{exemple}\jya khramba-nɤre ɲɤ-βzu\cmn 他假装笑了\end{exemple}
\begin{exemple}\jya khramba-ɣɤwu ma-tɯ-βze\cmn 不要装哭!\end{exemple}
\begin{relation-sémantique}\synonyme{
\hyperlink{ⒺʑɣɤpaⒽ1}{\textit{ \papi{ʑɣɤpa}}}
}\end{relation-sémantique}
\end{sous-entrée}\end{entrée}

\begin{entrée}
\vedette{\hypertarget{Ⓔkhrambakɯm}{\papi{ khrambakɯm}}}\markboth{khrambakɯm}{}\classe{n}
\begin{définition}\fra pommette\end{définition}
\begin{définition}\cmn 酒窝\end{définition}\end{entrée}

\begin{entrée}
\vedette{\hypertarget{Ⓔkhrambaqe}{\papi{ khrambaqe}}}\markboth{khrambaqe}{}
\classe{n}
\begin{définition}\fra menteur, personne malhonnête\end{définition}
\begin{définition}\cmn 骗子
经常说谎的人叫做“\stylefv{khrambaqe}”
\end{définition}
\begin{relation-sémantique}\confer{
\hyperlink{Ⓔtɯ-qe}{\textit{ \papi{tɯ-qe}}}
}\end{relation-sémantique}
\end{entrée}

\begin{entrée}
\vedette{\hypertarget{Ⓔkhrɤlmu}{\papi{ khrɤlmu}}}\markboth{khrɤlmu}{}\classe{n}
\begin{définition}\fra épouse\end{définition}
\begin{définition}\cmn 妻子
\begin{déclaration} \étymologie{\papi{*kʰrel.mo}}\end{déclaration}\end{définition}\end{entrée}

\begin{entrée}
\vedette{\hypertarget{Ⓔkhrɤlpa}{\papi{ khrɤlpa}}}\markboth{khrɤlpa}{}\classe{n}
\begin{définition}\fra époux\end{définition}
\begin{définition}\cmn 丈夫
\begin{déclaration} \étymologie{\papi{*kʰrel.pa}}\end{déclaration}\end{définition}
\end{entrée}

\begin{entrée}
\vedette{\hypertarget{ⒺkhrɤtⒽ1}{\papi{ khrɤt}}}\markboth{khrɤt}{}\homonyme{1}\classe{vt}
\paradigme{\textit{dir :} \jya pɯ-}
\begin{définition}\fra érafler, rayer\end{définition}
\begin{définition}\cmn 划破\end{définition}\begin{sous-entrée}
\vedette{\hypertarget{}{\papi{ rɤkhɯkhrɤt}}}\markboth{rɤkhɯkhrɤt}{} (\variante{rɤkhrɯkhrɤt}) \classe{vt}
\paradigme{\textit{dir :} \jya thɯ-}
\begin{définition}\fra érafler\end{définition}
\begin{définition}\cmn 划来划去\end{définition}
\begin{relation-sémantique}\confer{
\hyperlink{Ⓔtɯ-tɤkhrɤz}{\textit{ \papi{tɯ-tɤkhrɤz}}}
}\end{relation-sémantique}
\end{sous-entrée}\begin{sous-entrée}
\vedette{\hypertarget{}{\papi{ sɯkhrɤt}}}\markboth{sɯkhrɤt}{}\classe{vt}
\begin{définition}\fra rayer avec\end{définition}
\begin{définition}\cmn 用……划破\end{définition}
\begin{exemple}\jya mbrɯtɕɯ kɯ pa-sɯ-khrɤt\cmn 他用刀划破了\end{exemple}
\end{sous-entrée}\end{entrée}

\begin{entrée}
\vedette{\hypertarget{ⒺkhrɤtⒽ2}{\papi{ khrɤt}}}\markboth{khrɤt}{}\homonyme{2}
\classe{vt}
\paradigme{\textit{dir :} \jya nɯ-}
\begin{définition}\fra organiser (un travail), planifier\end{définition}
\begin{définition}\cmn 布置\end{définition}
\begin{exemple}\jya jɯfɕɯr ŋgumdʑɯɣ kɯ ji-ma pa-khrɤt\cmn 昨天领导布置了我们的工作\end{exemple}
\begin{exemple}\jya tɤ-pɤtso ɯ-ma nɯ-khrat-a\cmn 我给小孩子布置了这个任务\end{exemple}\end{entrée}

\begin{entrée}
\vedette{\hypertarget{Ⓔkhri}{\papi{ khri}}}\markboth{khri}{}\classe{n}\acception{1}
\begin{définition}\fra lit\end{définition}
\begin{définition}\cmn 床\end{définition}\acception{2}
\begin{définition}\fra siège\end{définition}
\begin{définition}\cmn 座位
\begin{déclaration} \étymologie{\papi{kʰri}}\end{déclaration}\end{définition}
\end{entrée}

\begin{entrée}
\vedette{\hypertarget{Ⓔkhro}{\papi{ khro}}}\markboth{khro}{}\classe{n}
\begin{définition}\fra beaucoup\end{définition}
\begin{définition}\cmn 很多;很长时间\end{définition}
\end{entrée}

\begin{entrée}
\vedette{\hypertarget{Ⓔkhrɯ}{\papi{ khrɯ}}}\markboth{khrɯ}{}
\classe{vs}
\paradigme{\textit{dir :} \jya nɯ-}
\begin{définition}\fra sec\end{définition}
\begin{définition}\cmn 干(干了之后变硬了)
\begin{déclaration}\use{\stylefv{khrɯ}、\stylefv{zbaʁ}和\stylefv{rom}都可以翻译成“干”,但是在使用上有一些区别。\stylefv{zbaʁ}表示干了以后还是软的、\stylefv{khrɯ}表示干了以后是硬的、而\stylefv{rom}主要用于木料、木头}\end{déclaration}\end{définition}
\begin{exemple}\jya tɯndʐi ɲɤ-khrɯ\cmn 皮子变干了\end{exemple}
\begin{exemple}\jya sɤtɕha ɲɯ-khrɯ\cmn 地很干\end{exemple}
\begin{exemple}\jya tɯthɯ ɯ-ŋgɯ kɤndza pjɤ-khrɯ\cmn 锅子里的饭是干的\end{exemple}
\begin{relation-sémantique}\antonyme{
\hyperlink{Ⓔnɯrlɤn}{\textit{ \papi{nɯrlɤn}}}
}\end{relation-sémantique}
\begin{relation-sémantique}\confer{
\hyperlink{Ⓔɕɤkhrɯ}{\textit{ \papi{ɕɤkhrɯ}}}
}\end{relation-sémantique}
\begin{relation-sémantique}\confer{
\hyperlink{Ⓔɯ-khrakhrɯ}{\textit{ \papi{ɯ-khrakhrɯ}}}
}\end{relation-sémantique}\begin{sous-entrée}
\vedette{\hypertarget{}{\papi{ khrɯ,jɤβ}}}\markboth{khrɯ,jɤβ}{}
\begin{définition}\fra très sec\end{définition}
\begin{définition}\cmn 非常干燥\end{définition}
\begin{exemple}\jya tɯ-mɯ mɯ́j-lɤt tɕe tɯ-ji ra ɲɯ-khrɯ ɲɯ-jɤβ ʑo (=ɲɯ-khrɯ-jɤβ ʑo)\cmn 因为不下雨,这些地方都变得很干燥\end{exemple}
\begin{relation-sémantique}\ComponentA{\classe{vs}
\hyperlink{Ⓔkhrɯ}{\textit{ \papi{khrɯ}}}
}\end{relation-sémantique}
\begin{relation-sémantique}\ComponentB{\classe{vs}
 \papi{jɤβ}
}\end{relation-sémantique}
\end{sous-entrée}\end{entrée}

\begin{entrée}
\vedette{\hypertarget{Ⓔkhrɯɣnɤkhrɯɣ}{\papi{ khrɯɣnɤkhrɯɣ}}}\markboth{khrɯɣnɤkhrɯɣ}{}
\classe{idph.3}
\begin{définition}\fra bruit (balayage, d'un cochon qui se gratte)\end{définition}
\begin{définition}\cmn 扫地、抓痒的声音\end{définition}
\begin{exemple}\jya khrɯɣnɤkhrɯɣ ɲɯ-ŋke\cmn 他在走,发出“咯咯”声\end{exemple}
\begin{exemple}\jya paʁ khrɯɣnɤkhrɯɣ ɲɯ-ʑɣɤrɤβraʁ\cmn 猪在抓痒,发出“咯咯”声\end{exemple}\begin{sous-entrée}
\vedette{\hypertarget{}{\papi{ ɣɤkhrɯɣlɯɣ}}}\markboth{ɣɤkhrɯɣlɯɣ}{}\classe{vi}
\begin{exemple}\jya ɲɯ-ɣɤkhrɯɣlɯɣ ntsɯ\cmn 发出“咯咯”声\end{exemple}
\end{sous-entrée}\begin{sous-entrée}
\vedette{\hypertarget{}{\papi{ khrɯɣnɤlɯɣ}}}\markboth{khrɯɣnɤlɯɣ}{}\classe{idph.4}
\end{sous-entrée}\begin{sous-entrée}
\vedette{\hypertarget{}{\papi{ sɤkhrɯɣkhrɯɣ}}}\markboth{sɤkhrɯɣkhrɯɣ}{} (\variante{sɤkhɯkhrɯɣ}) \classe{vt}
\begin{exemple}\jya βʑɯ kɯ @mianban ɲɯ-ɤsɯ-qhrɯt tɕe ɲɯ-sɤkhrɯɣkhrɯɣ\cmn 老鼠在啃砧板,发出“咯咯”声\end{exemple}
\begin{exemple}\jya (ɕoŋtɕa) ɲɯ-sɤkhɯkhrɯɣ ʑo ɲɯ-ɤz-rɤɕi\cmn 他在拖(木料),发出声音\end{exemple}
\end{sous-entrée}\end{entrée}

\begin{entrée}
\vedette{\hypertarget{Ⓔkhrɯ,jɤβ}{\papi{ khrɯ,jɤβ}}}\markboth{khrɯ,jɤβ}{}
\begin{relation-sémantique}\confer{
 \papi{jɤβ}
}\end{relation-sémantique}\end{entrée}

\begin{entrée}
\vedette{\hypertarget{Ⓔkhrɯm}{\papi{ khrɯm}}}\markboth{khrɯm}{}
\classe{n}
\begin{définition}\fra châtiment\end{définition}
\begin{définition}\cmn 刑
\begin{déclaration} \étymologie{\papi{kʰrims}}\end{déclaration}\end{définition}\end{entrée}

\begin{entrée}
\vedette{\hypertarget{Ⓔkhrɯmbjɤm}{\papi{ khrɯmbjɤm}}}\markboth{khrɯmbjɤm}{}\classe{n}
\begin{définition}\fra sofa tibétain\end{définition}
\begin{définition}\cmn 藏式沙发,坐床\end{définition}\end{entrée}

\begin{entrée}
\vedette{\hypertarget{Ⓔkhrɯŋkhrɯŋ}{\papi{ khrɯŋkhrɯŋ}}}\markboth{khrɯŋkhrɯŋ}{}
\classe{idph.2}
\begin{définition}\fra propre, utilisé jusqu'au bout\end{définition}
\begin{définition}\cmn 干净,用完\end{définition}
\begin{exemple}\jya khrɯŋkhrɯŋ ɯ-tshi pjɤ-ɕkɯt\cmn 它把食物吃光了\end{exemple}
\begin{exemple}\jya khɯtsa ɲo-χtɕi khrɯŋkhrɯŋ ʑo\cmn 碗洗得干干净净\end{exemple}
\begin{relation-sémantique}\confer{
\hyperlink{Ⓔgrɯŋgrɯŋ}{\textit{ \papi{grɯŋgrɯŋ}}}
}\end{relation-sémantique}\end{entrée}

\begin{entrée}
\vedette{\hypertarget{Ⓔkhrɯtɕhɯ}{\papi{ khrɯtɕhɯ}}}\markboth{khrɯtɕhɯ}{}\classe{n}
\begin{définition}\ 
\begin{déclaration}\grammar{n.lieu}\end{déclaration}\end{définition}
\begin{définition}\fra Khrochu\end{définition}
\begin{définition}\cmn 黑水\end{définition}\end{entrée}

\begin{entrée}
\vedette{\hypertarget{Ⓔkhrɯtsu}{\papi{ khrɯtsu}}}\markboth{khrɯtsu}{}\classe{n}
\begin{définition}\fra dix mille\end{définition}
\begin{définition}\cmn 10000
\begin{déclaration} \étymologie{\papi{kʰri.tsho}}\end{déclaration}\end{définition}
\begin{exemple}\jya iɕqha tɯrme nɯ ɯ-khrɯtsu kɯ-ɤro ci ɕti nɤ\cmn 那个人很富有,有十万元\end{exemple}
\end{entrée}

\begin{entrée}
\vedette{\hypertarget{Ⓔkhrɯtsusqi}{\papi{ khrɯtsusqi}}}\markboth{khrɯtsusqi}{}\classe{num}
\begin{définition}\fra cent mille\end{définition}
\begin{définition}\cmn 十万\end{définition}
\begin{relation-sémantique}\confer{
\hyperlink{Ⓔkhrɯtsu}{\textit{ \papi{khrɯtsu}}}
}\end{relation-sémantique}\end{entrée}

\begin{entrée}
\vedette{\hypertarget{Ⓔkhrɯtsɯr}{\papi{ khrɯtsɯr}}}\markboth{khrɯtsɯr}{}\classe{n}
\begin{définition}\fra petit récipient en fer utilisé pour cuire la viande pour les personnes âgées\end{définition}
\begin{définition}\cmn 生铁铸成的小罐子,有盖子,专门给老人炖肉\end{définition}\end{entrée}

\begin{entrée}
\vedette{\hypertarget{Ⓔkhrɯzwa}{\papi{ khrɯzwa}}}\markboth{khrɯzwa}{}\classe{n}
\begin{définition}\fra riz cuit\end{définition}
\begin{définition}\cmn 饭\end{définition}
\end{entrée}

\begin{entrée}
\vedette{\hypertarget{Ⓔkhusri}{\papi{ khusri}}}\markboth{khusri}{}\classe{n}
\begin{définition}\ 
\begin{déclaration}\grammar{n.lieu}\end{déclaration}\end{définition}
\begin{définition}\fra Lixian\end{définition}
\begin{définition}\cmn 理想\end{définition}\end{entrée}

\begin{entrée}
\vedette{\hypertarget{ⒺkhɯⒽ2}{\papi{ khɯ}}}\markboth{khɯ}{}\homonyme{2}
\classe{n}
\begin{définition}\fra poils fins\end{définition}
\begin{définition}\cmn 细毛\end{définition}
\end{entrée}

\begin{entrée}
\vedette{\hypertarget{ⒺkhɯⒽ1}{\papi{ khɯ}}}\markboth{khɯ}{}\homonyme{1}
\classe{vs}
\begin{définition}\fra être possible\end{définition}
\begin{définition}\cmn 可以\end{définition}
\begin{exemple}\jya nɯ kɤti mɤ-kɯ-khɯ me\cmn 那样说没有什么不可以的\end{exemple}\begin{sous-entrée}
\vedette{\hypertarget{}{\papi{ nɯkhɯ}}}\markboth{nɯkhɯ}{}\classe{vi}
\paradigme{\textit{dir :} \jya tɤ-}
\begin{définition}\fra refuser\end{définition}
\begin{définition}\cmn 自己不同意;不听劝(不能怪别人)\end{définition}
\begin{exemple}\jya aʑo kɤ-zrɤma mɯ-tɤ-nɯkhɯ-a ɕti tɕe a-kɤ-nɯmɟa a-pɯ-nɯ-me ɬoʁ\cmn 我自己不同意做事,没有报酬是应该\end{exemple}
\begin{exemple}\jya nɤʑo mɯ-tɤ-tɯ-nɯkhɯ ɕti tɕe, ma-tɤ-kɯ-mpɕa-a je\cmn 是你自己没有同意,你不要责怪我\cmn 我没有听劝\end{exemple}
\begin{relation-sémantique}\confer{
\hyperlink{ⒺɣɤkhɯⒽ2}{\textit{ \papi{ɣɤkhɯ2}}}
}\end{relation-sémantique}
\end{sous-entrée}\end{entrée}

\begin{entrée}
\vedette{\hypertarget{Ⓔkhɯdi}{\papi{ khɯdi}}}\markboth{khɯdi}{}
\classe{n}
\begin{définition}\fra une plante\end{définition}
\begin{définition}\cmn 植物的一种\end{définition}
\begin{exemple}\jya khɯdi nɯ ruŋgu kɯ-mbro tu-kɯ-ɬoʁ sɯjno kɯ-xtɕi ci ŋu, ɯ-jwaʁ nɯ kɯ-zri tɕe kɯ-ɤmtɕoʁ ci ŋu, ɯ-jwaʁ ɯ-qhu nɯ wɣrum, ɯ-jwaʁ mpɕu, ɯ-ru nɯ jima ɯ-ru ʑo fse, ɣɯrni, tɯ-rtsɤɣ tɯ-rtsɤɣ lu-ɬoʁ ŋu. ɯ-mɯntoʁ nɯ ɯ-ru ɯ-kɤχcɤl lu-ɬoʁ ɲɯ-lɤt ŋu. ɯ-mɯntoʁ wɣrum. pɯ-ŋgra tɕe, jima ɯ-mat kɯ-fse ɲɯ-βze ŋu, tɕeri jima kɯ-fse ɯ-rqhu me, kɯ-ɣɯrni ŋu. khɯdi nɯ tɤ-χpɯm ɯ-fsu ɕaŋtaʁ tu-mbro mɤ-cha. ɯ-qa nɯ tɤ-pɤtso ɯ-ŋgo kɯ-phɤn ɲɯ-ŋu khi.\cmn 
\stylefv{khɯdi}是长在很高的草地上的小草。叶子长而尖,叶子背面是白色的,叶子光滑。茎像玉米的茎一样,是红色的,是一节一节长出来的。花开在茎的顶端,是白色的。花凋谢后,果实像玉米的一样,但是没有像玉米棒子一样的皮裹着,是红色的。\stylefv{khɯdi}只能长到人的膝盖那么高。根对孩子的病有疗效。
\end{exemple}\end{entrée}

\begin{entrée}
\vedette{\hypertarget{Ⓔkhɯdo}{\papi{ khɯdo}}}\markboth{khɯdo}{}\classe{n}
\begin{définition}\fra vieux chien\end{définition}
\begin{définition}\cmn 老狗\end{définition}
\begin{relation-sémantique}\confer{
\hyperlink{Ⓔkhɯna}{\textit{ \papi{khɯna}}}
}\end{relation-sémantique}
\begin{relation-sémantique}\confer{
\hyperlink{Ⓔɯ-do}{\textit{ \papi{ɯ-do}}}
}\end{relation-sémantique}\end{entrée}

\begin{entrée}
\vedette{\hypertarget{Ⓔkhɯɣ}{\papi{ khɯɣ}}}\markboth{khɯɣ}{}\classe{n}
\begin{définition}\fra moule pour les balles de fusils traditionnels\end{définition}
\begin{définition}\cmn (子弹)模型\end{définition}\end{entrée}

\begin{entrée}
\vedette{\hypertarget{Ⓔkhɯɣɲɟɯ}{\papi{ khɯɣɲɟɯ}}}\markboth{khɯɣɲɟɯ}{}
\classe{n}
\begin{définition}\fra fenêtre\end{définition}
\begin{définition}\cmn 窗户\end{définition}
\begin{exemple}\jya khɯɣɲɟɯ kɤ-pa-t-a\cmn 我关了窗子\end{exemple}
\begin{exemple}\jya khɯɣɲɟɯ nɯ ɯ-pɕi ɲɯ-sɤɣ-ru ŋu\cmn 窗户是用来看外面的\end{exemple}
\begin{relation-sémantique}\confer{
\hyperlink{Ⓔɯ-ɣɲɟɯ}{\textit{ \papi{ɯ-ɣɲɟɯ}}}
}\end{relation-sémantique}
\begin{relation-sémantique}\confer{
\hyperlink{Ⓔtɤ-khɯ}{\textit{ \papi{tɤ-khɯ}}}
}\end{relation-sémantique}\end{entrée}

\begin{entrée}
\vedette{\hypertarget{Ⓔkhɯjŋga}{\papi{ khɯjŋga}}}\markboth{khɯjŋga}{}
\classe{n}
\begin{définition}\fra rhododendron\end{définition}
\begin{définition}\cmn 羊角花\end{définition}
\begin{exemple}\jya khɯjŋga nɯ ʁnɯ-tɯphu tu, sɤtɕha kɯ-mbɤr tu-kɯ-ɬoʁ ci tu, tɕe nɯ khɯjŋga rmi, khɯjŋga nɯ si mɤ-mbro, ɯ-ru mɤ-astu, ɯ-rtaʁ dɤn, ɯ-jwaʁ nɯ ɯ-ʁɤri nɯ ldʑaŋnaʁ ŋu, ɯ-qhu nɯ kɯ-ɤɣrɤɣrum tsa ŋu, ɯ-jwaʁ nɯ kɯ-ɤrtɯm ɯ-ŋgɯz kɯ-rɲɟi tsa ŋu. ɯ-mɯntoʁ nɯ kɯ-wɣrum ʁɟa tu, kɯ-wɣrum ɯ-ŋgɯz kɯ-ɤɣɯrnɯɕɯr tu, ɯ-mɯntoʁ nɯ kɯngɯsqi jamar tɯtɯrca ɲɯ-lɤt ŋu, wuma ʑo mpɕɤr. phaʁzla jamar ɲɯ-rɯmɯntoʁ ŋu. mɤʑɯ tɯ-tɯphu tu tɕe, zgoku wuma ʑo kɯ-ɣɤndʐo tu-ɬoʁ ŋu, ɯ-ru nɯ khɯjŋga cho ɲɯ-naχtɕɯɣ, ɯ-jwaʁ ɯ-ʁɤri nɯ khɯjŋga ɣɯ cho naχtɕɯɣ, tɕeri ɯ-jwaʁ ɯ-qhu chu nɯ kɯ-qarŋe tɯ-ɣndʑɤr kɯ-fse tu, ɲɯ́-wɣ-nɤmɤle tɕe, pjɯ-kɯ-ŋgra ʑo tu. ɯ-mɯntoʁ khɯjŋga ɣɯ sɤznɤ nɤrko ri ndɯβ. nɯnɯ sŋo rmi. ɯ-ru nɯnɯ mɤ-ngɯt ma ndoʁ, nɯ ma mɤ-sna ri kɤ-nɯ-βlɯ pe.\cmn 
羊角花有两种,一种生长在海拔比较低的地方,叫作\stylefv{khɯjŋga},树干不高,不直,枝桠多,叶子正面是深绿色的,后面带有点白色,叶子是椭圆形的,花有的是纯白的,也有的是粉红色的。花九朵十朵同时开,很美。一般五月份开花。另一种生长在高山比较寒冷的地方,树干和\stylefv{khɯjŋga}的一样,叶子正面也是和\stylefv{khɯjŋga}的一样,但叶子背面有黄色的粉,一碰就会散落。花比和\stylefv{khɯjŋga}的结实,但小一些。这一种叫\stylefv{sŋo} 。树干不结实因为很脆,虽然不能用来制造什么东西,但很好烧。
\end{exemple}\end{entrée}

\begin{entrée}
\vedette{\hypertarget{Ⓔkhɯjŋgɯ}{\papi{ khɯjŋgɯ}}}\markboth{khɯjŋgɯ}{}\classe{np}
\begin{définition}\fra gamelle (du chien)\end{définition}
\begin{définition}\cmn 狗碗\end{définition}
\begin{relation-sémantique}\confer{
\hyperlink{Ⓔɯ-jŋgɯ}{\textit{ \papi{ɯ-jŋgɯ}}}
}\end{relation-sémantique}\end{entrée}

\begin{entrée}
\vedette{\hypertarget{Ⓔkhɯlu}{\papi{ khɯlu}}}\markboth{khɯlu}{}
\classe{n}
\begin{définition}\fra une plante\end{définition}
\begin{définition}\cmn 五朵云\end{définition}
\begin{exemple}\jya khɯlu nɯ sɯjno ci ŋu, ɯ-ru kɯ-ɣɯrni ŋu, ɯ-jwaʁ tɯ-tɯ-rdoʁ ma me, ɯ-ru mpɯ, ɯ-kɤχcɤl tɕe ɯ-mɯntoʁ ɲɯ-lɤt ŋu, ɯ-mɯntoʁ dɤn. ɯ-mɯntoʁ tɯ-rdoʁ tɕe, ɯ-mat ʁnɯ-rdoʁ ntsɯ ɲɯ-βze ŋu. pjɯ́-wɣ-qlɯt tɕe, ɯ-lu tu, sɤndɤɣ. ɯ-lu nɯ tɯ-βri nɯ-ɤtɕaʁ tɕe ʑmbɤr ɲɯ-tɕɤt cha.\cmn 五朵云是一种草,茎是红色的,叶子只有几片,茎很嫩,顶端开花。花比较多。每一朵花都结两个果实。掰开就有乳汁,有毒性。乳汁粘在皮肤上会生疮。\end{exemple}\end{entrée}

\begin{entrée}
\vedette{\hypertarget{Ⓔkhɯmŋu}{\papi{ khɯmŋu}}}\markboth{khɯmŋu}{}\classe{n}
\begin{définition}\fra bord du bol\end{définition}
\begin{définition}\cmn 碗口\end{définition}
\begin{exemple}\jya khɯmŋu mɤ-ɕɤt\cmn (婴儿)还不会用碗口吃东西\end{exemple}
\begin{relation-sémantique}\confer{
\hyperlink{Ⓔkhɯtsa}{\textit{ \papi{khɯtsa}}}
}\end{relation-sémantique}
\begin{relation-sémantique}\confer{
\hyperlink{Ⓔɯ-mŋu}{\textit{ \papi{ɯ-mŋu}}}
}\end{relation-sémantique}\end{entrée}

\begin{entrée}
\vedette{\hypertarget{Ⓔkhɯmtsɯ}{\papi{ khɯmtsɯ}}}\markboth{khɯmtsɯ}{}\classe{n}
\begin{définition}\fra viande du thorax du cochon\end{définition}
\begin{définition}\cmn 猪胸股上的肉\end{définition}
\end{entrée}

\begin{entrée}
\vedette{\hypertarget{Ⓔkhɯna}{\papi{ khɯna}}}\markboth{khɯna}{}\classe{n}
\begin{définition}\fra chien\end{définition}
\begin{définition}\cmn 狗\end{définition}
\end{entrée}

\begin{entrée}
\vedette{\hypertarget{Ⓔkhɯnajme}{\papi{ khɯnajme}}}\markboth{khɯnajme}{}\classe{n}
\begin{définition}\fra une plante\end{définition}
\begin{définition}\cmn 狗尾巴草\end{définition}
\begin{exemple}\jya khɯna-jme nɯ sɯjno mɤ-mbro, ɯ-jwaʁ nɯ xsɤrɯ ɯ-jwaʁ fse ri xtɯt, tɯ-ji ɯ-rkɯ aʁɤndɯndɤt tu-ɬoʁ ɕti. ɯ-mat kɯɕnom fse, ɯ-rme ɯ-tshɯɣa nɯ ra xsɤrɯ ɯ-mat fse, xsɤrɯ ɣɯ ɯ-mat pjɯ-ŋgɤɣ ŋu, khɯna-jme ɯ-mat nɯ tu-ndzur kɯ-fse ɕti ma pjɯ-ŋgɤɣ mɤ-cha, pakuku tu-ɬoʁ cha. ɯ-kɯɕnom khɯna-jme fse.\cmn 
\stylefv{khɯna-jme}是一种小草,叶子像\stylefv{xsɤrɯ}的叶子,但短一些。在地边随处生长。果实像麦穗,穗上的毛像\stylefv{xsɤrɯ}的果实一样。\stylefv{xsɤrɯ}的果实往下弯,而\stylefv{khɯna-jme}的果实是立着的,不弯。年年生长。穗像狗的尾巴一样。
\end{exemple}
\end{entrée}

\begin{entrée}
\vedette{\hypertarget{Ⓔkhɯnalu}{\papi{ khɯnalu}}}\markboth{khɯnalu}{}\classe{n}
\begin{définition}\fra année du chien\end{définition}
\begin{définition}\cmn 狗年\end{définition}
\begin{relation-sémantique}\confer{
\hyperlink{Ⓔkhɯna}{\textit{ \papi{khɯna}}}
}\end{relation-sémantique}
\end{entrée}

\begin{entrée}
\vedette{\hypertarget{Ⓔkhɯndʐi}{\papi{ khɯndʐi}}}\markboth{khɯndʐi}{}\classe{n}
\begin{définition}\fra peau de chien\end{définition}
\begin{définition}\cmn 狗皮子\end{définition}
\begin{relation-sémantique}\confer{
\hyperlink{Ⓔkhɯna}{\textit{ \papi{khɯna}}}
}\end{relation-sémantique}
\begin{relation-sémantique}\confer{
\hyperlink{Ⓔtɯ-ndʐi}{\textit{ \papi{tɯ-ndʐi}}}
}\end{relation-sémantique}\end{entrée}

\begin{entrée}
\vedette{\hypertarget{Ⓔkhɯrɕaŋ}{\papi{ khɯrɕaŋ}}}\markboth{khɯrɕaŋ}{}\classe{n}
\begin{définition}\fra armature en bois pour porter des charges sur le dos\end{définition}
\begin{définition}\cmn 背架
\begin{déclaration} \étymologie{\papi{kʰur.ɕiŋ}}\end{déclaration}\end{définition}
\end{entrée}

\begin{entrée}
\vedette{\hypertarget{Ⓔkhɯrndɯɣ}{\papi{ khɯrndɯɣ}}}\markboth{khɯrndɯɣ}{}\classe{n}
\begin{définition}\fra sanglier solitaire\end{définition}
\begin{définition}\cmn 野猪
\end{définition}\end{entrée}

\begin{entrée}
\vedette{\hypertarget{Ⓔkhɯrtshɤz}{\papi{ khɯrtshɤz}}}\markboth{khɯrtshɤz}{}\classe{n}
\begin{définition}\fra espèce de plante\end{définition}
\begin{définition}\cmn 一种草\end{définition}
\begin{exemple}\jya khɯrtshɤz nɯ sɯjno ci ŋu, tu-mbro mɤ-cha, ɯ-ru kɯ-ɣɯrni ŋu, ɯ-jwaʁ kɯ-ɤrŋi ŋu, ɯ-mɯntoʁ kɯ-ɣɯrni ŋu, tɤ-rɤku rca kɤ-ɬoʁ rga, paʁ kɤ-mbi sna\cmn 
\stylefv{khɯrtshɤz}是一种植物,长得不高,茎红色,叶子绿色,花红色,一般生长在庄稼地里,可以喂猪。
\end{exemple}\end{entrée}

\begin{entrée}
\vedette{\hypertarget{Ⓔkhɯrwum}{\papi{ khɯrwum}}}\markboth{khɯrwum}{}
\classe{n}
\begin{définition}\fra moisissure\end{définition}
\begin{définition}\cmn 霉\end{définition}
\begin{exemple}\jya khɯrwum ɲo-βzu\cmn 生了霉\end{exemple}
\begin{relation-sémantique}\confer{
\hyperlink{Ⓔnɯkhɯrwum}{\textit{ \papi{nɯkhɯrwum}}}
}\end{relation-sémantique}\end{entrée}

\begin{entrée}
\vedette{\hypertarget{Ⓔkhɯtsa}{\papi{ khɯtsa}}}\markboth{khɯtsa}{}\classe{n}
\begin{définition}\fra bol\end{définition}
\begin{définition}\cmn 碗\end{définition}
\begin{exemple}\jya tɕi-khɯtsa ɲɯ-χtɕi ɕti wo\cmn 他在洗(我们俩的)碗\end{exemple}
\begin{relation-sémantique}\confer{
\hyperlink{Ⓔkhɯmŋu}{\textit{ \papi{khɯmŋu}}}
}\end{relation-sémantique}
\begin{relation-sémantique}\confer{
\hyperlink{Ⓔarɯkhɯtsa}{\textit{ \papi{arɯkhɯtsa}}}
}\end{relation-sémantique}
\begin{relation-sémantique}\confer{
\hyperlink{Ⓔsomo khɯtsa}{\textit{ \papi{somo khɯtsa}}}
}\end{relation-sémantique}\end{entrée}

\begin{entrée}
\vedette{\hypertarget{Ⓔkhɯtshoʁ}{\papi{ khɯtshoʁ}}}\markboth{khɯtshoʁ}{}
\classe{n}
\begin{définition}\fra chasse avec des chiens\end{définition}
\begin{définition}\cmn 狩猎(牵着狗)\end{définition}
\begin{exemple}\jya aʑo khɯtshoʁ rga\cmn 我喜欢打猎\end{exemple}
\begin{relation-sémantique}\confer{
\hyperlink{Ⓔɣɯkhɯtshoʁ}{\textit{ \papi{ɣɯkhɯtshoʁ}}}
}\end{relation-sémantique}\end{entrée}

\begin{entrée}
\vedette{\hypertarget{Ⓔkhɯwɯsi}{\papi{ khɯwɯsi}}}\markboth{khɯwɯsi}{}\classe{n}
\begin{définition}\fra tapis (coloré )\end{définition}
\begin{définition}\cmn (彩色的)地毯\end{définition}
\end{entrée}

\begin{entrée}
\vedette{\hypertarget{Ⓔkhɯzɤpɯ}{\papi{ khɯzɤpɯ}}}\markboth{khɯzɤpɯ}{}
\classe{n}
\begin{définition}\fra petit chien\end{définition}
\begin{définition}\cmn 小狗\end{définition}\end{entrée}

\begin{entrée}
\vedette{\hypertarget{Ⓔkhɯzgɯr}{\papi{ khɯzgɯr}}}\markboth{khɯzgɯr}{}\classe{n}
\begin{définition}\fra serrure\end{définition}
\begin{définition}\cmn 锁\end{définition}
\end{entrée}

\begin{entrée}
\vedette{\hypertarget{Ⓔkhɯzi}{\papi{ khɯzi}}}\markboth{khɯzi}{}
\classe{n}
\begin{définition}\fra articulation du fléau\end{définition}
\begin{définition}\cmn 连枷的接头\end{définition}\end{entrée}

\begin{entrée}
\vedette{\hypertarget{Ⓔki}{\papi{ ki}}}\markboth{ki}{}\classe{dem}
\begin{définition}\fra ceci\end{définition}
\begin{définition}\cmn 这个\end{définition}
\end{entrée}

\begin{entrée}
\vedette{\hypertarget{Ⓔkio}{\papi{ kio}}}\markboth{kio}{}
\classe{vt}\acception{1}
\paradigme{\textit{dir :} \jya pɯ-}
\begin{définition}\fra faire glisser\end{définition}
\begin{définition}\cmn 使滑下来\end{définition}
\begin{exemple}\jya jiɕqha ɕoŋtɕa nɯ pa-kio\cmn 他把木料滑下来了\end{exemple}
\begin{exemple}\jya ɕoŋtɕa pɯ-kio-t-a\cmn 我把木料滑下来了\end{exemple}
\begin{exemple}\jya ɯ-thoʁ ɲɯ-sɤŋgio tɕe tɕoχtsi ɲɯ́-wɣ-kio ɲɯ-khɯ\cmn 地面很滑,可以把桌子推过去\end{exemple}\acception{2}
\paradigme{\textit{dir :} \jya \_}
\begin{définition}\fra pousser vers un côté (par la foule)\end{définition}
\begin{définition}\cmn 挤过去(因为人多,很拥挤)\end{définition}\end{entrée}

\begin{entrée}
\vedette{\hypertarget{Ⓔklaŋklaŋ}{\papi{ klaŋklaŋ}}}\markboth{klaŋklaŋ}{}\classe{idph.2}
\begin{définition}\fra complètement emmitouflé\end{définition}
\begin{définition}\cmn 形容裹得又紧又大的样子\end{définition}
\begin{exemple}\jya klaŋklaŋ ʑo to-ʑɣɤmphɯr\cmn 他把自己的全身裹起来了\end{exemple}\end{entrée}

\begin{entrée}
\vedette{\hypertarget{Ⓔklɯɣklɯɣ}{\papi{ klɯɣklɯɣ}}}\markboth{klɯɣklɯɣ}{}\classe{idph.2}
\begin{définition}\fra dur et rond\end{définition}
\begin{définition}\cmn 形容圆而硬的样子\end{définition}
\begin{relation-sémantique}\synonyme{
\hyperlink{Ⓔtslɯɣtslɯɣ}{\textit{ \papi{tslɯɣtslɯɣ}}}
}\end{relation-sémantique}\end{entrée}

\begin{entrée}
\vedette{\hypertarget{Ⓔklɯnklɯn}{\papi{ klɯnklɯn}}}\markboth{klɯnklɯn}{}\classe{idph.2}
\begin{définition}\fra très serré\end{définition}
\begin{définition}\cmn 形容紧紧包裹的样子\end{définition}
\begin{exemple}\jya tɯrtɯthɯ nɯ klɯnklɯn ʑo chɤ-mphɯr\cmn 他把麻布裹得很紧\end{exemple}\end{entrée}

\begin{entrée}
\vedette{\hypertarget{Ⓔkumbrɤl}{\papi{ kumbrɤl}}}\markboth{kumbrɤl}{}\classe{n}
\begin{définition}\fra jeu d'échec\end{définition}
\begin{définition}\cmn 棋
\begin{déclaration} \étymologie{\papi{ⁿbrel}}\end{déclaration}\end{définition}
\begin{relation-sémantique}\confer{
\hyperlink{Ⓔnɯkumbrɤl}{\textit{ \papi{nɯkumbrɤl}}}
}\end{relation-sémantique}\end{entrée}

\begin{entrée}
\vedette{\hypertarget{Ⓔkumkɕi}{\papi{ kumkɕi}}}\markboth{kumkɕi}{} (\variante{kɯmxɕi}) 
\classe{n}
\begin{définition}\fra chien de garde\end{définition}
\begin{définition}\cmn 看门狗
\begin{déclaration} \étymologie{\papi{kʰʲi}}\end{déclaration}\end{définition}\end{entrée}

\begin{entrée}
\vedette{\hypertarget{Ⓔkumpɣa}{\papi{ kumpɣa}}}\markboth{kumpɣa}{}
\classe{n}
\begin{définition}\fra poulet\end{définition}
\begin{définition}\cmn 鸡\end{définition}\end{entrée}

\begin{entrée}
\vedette{\hypertarget{Ⓔkumpɣalu}{\papi{ kumpɣalu}}}\markboth{kumpɣalu}{}\classe{n}
\begin{définition}\fra année du coq\end{définition}
\begin{définition}\cmn 鸡年\end{définition}
\begin{relation-sémantique}\confer{
\hyperlink{Ⓔkumpɣa}{\textit{ \papi{kumpɣa}}}
}\end{relation-sémantique}
\end{entrée}

\begin{entrée}
\vedette{\hypertarget{Ⓔkumpɣamu}{\papi{ kumpɣamu}}}\markboth{kumpɣamu}{}\classe{n}
\begin{définition}\fra poule\end{définition}
\begin{définition}\cmn 母鸡\end{définition}
\begin{relation-sémantique}\confer{
\hyperlink{Ⓔkumpɣa}{\textit{ \papi{kumpɣa}}}
}\end{relation-sémantique}
\end{entrée}

\begin{entrée}
\vedette{\hypertarget{Ⓔkumpɣaphu}{\papi{ kumpɣaphu}}}\markboth{kumpɣaphu}{}\classe{n}
\begin{définition}\fra coq\end{définition}
\begin{définition}\cmn 公鸡\end{définition}
\begin{relation-sémantique}\confer{
\hyperlink{Ⓔkumpɣa}{\textit{ \papi{kumpɣa}}}
}\end{relation-sémantique}
\end{entrée}

\begin{entrée}
\vedette{\hypertarget{Ⓔkumpɣapɯ}{\papi{ kumpɣapɯ}}}\markboth{kumpɣapɯ}{}\classe{n}
\begin{définition}\fra poussin\end{définition}
\begin{définition}\cmn 小鸡\end{définition}
\begin{relation-sémantique}\confer{
\hyperlink{Ⓔkumpɣa}{\textit{ \papi{kumpɣa}}}
}\end{relation-sémantique}
\end{entrée}

\begin{entrée}
\vedette{\hypertarget{Ⓔkumpɣasta}{\papi{ kumpɣasta}}}\markboth{kumpɣasta}{}\classe{n}
\begin{définition}\fra poulailler\end{définition}
\begin{définition}\cmn 鸡圈\end{définition}
\begin{relation-sémantique}\confer{
\hyperlink{Ⓔkumpɣa}{\textit{ \papi{kumpɣa}}}
}\end{relation-sémantique}
\begin{relation-sémantique}\confer{
\hyperlink{Ⓔtɯ-sta}{\textit{ \papi{tɯ-sta}}}
}\end{relation-sémantique}
\end{entrée}

\begin{entrée}
\vedette{\hypertarget{Ⓔkumpɣɤŋgɯm}{\papi{ kumpɣɤŋgɯm}}}\markboth{kumpɣɤŋgɯm}{}\classe{n}
\begin{définition}\fra œuf de poule\end{définition}
\begin{définition}\cmn 鸡蛋\end{définition}
\begin{relation-sémantique}\confer{
\hyperlink{Ⓔtɤ-ŋgɯm}{\textit{ \papi{tɤ-ŋgɯm}}}
}\end{relation-sémantique}\end{entrée}

\begin{entrée}
\vedette{\hypertarget{Ⓔkumpɣɤtɕɯ}{\papi{ kumpɣɤtɕɯ}}}\markboth{kumpɣɤtɕɯ}{}
\classe{n}
\begin{définition}\fra moineau\end{définition}
\begin{définition}\cmn 麻雀\end{définition}\end{entrée}

\begin{entrée}
\vedette{\hypertarget{Ⓔkundi}{\papi{ kundi}}}\markboth{kundi}{}\classe{n}
\begin{définition}\fra de droite à gauche\end{définition}
\begin{définition}\cmn 左右\end{définition}
\end{entrée}

\begin{entrée}
\vedette{\hypertarget{ⒺkoⒽ2}{\papi{ ko}}}\markboth{ko}{}\homonyme{2}
\classe{part}
\begin{définition}\fra assertif\end{définition}
\begin{définition}\cmn 表示确定的语气\end{définition}
\begin{exemple}\jya mɤ-tɯ-cha ko\cmn 你肯定不行\end{exemple}
\begin{exemple}\jya tɯ-maqhu ko\cmn 你肯定迟到\end{exemple}\end{entrée}

\begin{entrée}
\vedette{\hypertarget{ⒺkoⒽ1}{\papi{ ko}}}\markboth{ko}{}\homonyme{1}\classe{vt}
\paradigme{\textit{dir :} \jya pɯ-}
\begin{définition}\fra vaincre\end{définition}
\begin{définition}\cmn 打赢;打败\end{définition}
\begin{exemple}\jya pɯ-ko-t-a\cmn 我打败了他\end{exemple}
\begin{exemple}\jya tɤ-aʑɯʑu-tɕi tɕe, a-χti pɯ-ko-t-a\cmn 在角力时候,我把对方打败了\end{exemple}
\begin{relation-sémantique}\synonyme{
\hyperlink{Ⓔɕɯnŋo}{\textit{ \papi{ɕɯnŋo}}}
}\end{relation-sémantique}\begin{sous-entrée}
\vedette{\hypertarget{}{\papi{ sɯko}}}\markboth{sɯko}{}\classe{vt}
\paradigme{\textit{dir :} \jya pɯ-}
\begin{définition}\fra causer (involontairement) le malheur\end{définition}
\begin{définition}\cmn 让……遭殃(不一定是故意的)
\begin{déclaration}\grammar{caus}\end{déclaration}\end{définition}
\begin{exemple}\jya pɯ-ta-sɯko\cmn 你因为我受了很多苦\end{exemple}
\end{sous-entrée}\end{entrée}

\begin{entrée}
\vedette{\hypertarget{Ⓔkodɤt}{\papi{ kodɤt}}}\markboth{kodɤt}{}\classe{n}
\begin{définition}\fra cave\end{définition}
\begin{définition}\cmn 土窖\end{définition}\end{entrée}

\begin{entrée}
\vedette{\hypertarget{Ⓔkolɤβ}{\papi{ kolɤβ}}}\markboth{kolɤβ}{}\classe{n}
\begin{définition}\fra habit sans manche de moine\end{définition}
\begin{définition}\cmn 披衫(没有袖子的袈裟)\end{définition}\end{entrée}

\begin{entrée}
\vedette{\hypertarget{Ⓔkomɤl}{\papi{ komɤl}}}\markboth{komɤl}{}\classe{n}
\begin{définition}\fra poutre\end{définition}
\begin{définition}\cmn 横梁\end{définition}
\end{entrée}

\begin{entrée}
\vedette{\hypertarget{Ⓔkomɤr}{\papi{ komɤr}}}\markboth{komɤr}{}\classe{n}
\begin{définition}\fra cuir teint en rouge\end{définition}
\begin{définition}\cmn 染成红色的皮子
\begin{déclaration} \étymologie{\papi{ko.ba.dmar}}\end{déclaration}\end{définition}
\end{entrée}

\begin{entrée}
\vedette{\hypertarget{Ⓔkonaʁ}{\papi{ konaʁ}}}\markboth{konaʁ}{}\classe{n}
\begin{définition}\fra cuir teint en noir\end{définition}
\begin{définition}\cmn 染成黑色的皮子
\begin{déclaration} \étymologie{\papi{ko.ba.nag}}\end{déclaration}\end{définition}
\end{entrée}

\begin{entrée}
\vedette{\hypertarget{Ⓔkontsɤɣdɯ}{\papi{ kontsɤɣdɯ}}}\markboth{kontsɤɣdɯ}{}
\classe{n}
\begin{définition}\fra récipient en cuivre ou en fer\end{définition}
\begin{définition}\cmn 红铜;生铁铸成的罐子,有盖子\end{définition}\end{entrée}

\begin{entrée}
\vedette{\hypertarget{Ⓔkontsɤrloŋ}{\papi{ kontsɤrloŋ}}}\markboth{kontsɤrloŋ}{}
\classe{n}
\begin{définition}\fra récipient en cuivre ou en fer\end{définition}
\begin{définition}\cmn 红铜、生铁铸成的罐子,没有盖子\end{définition}\end{entrée}

\begin{entrée}
\vedette{\hypertarget{Ⓔkontsi}{\papi{ kontsi}}}\markboth{kontsi}{}\classe{n}
\begin{définition}\fra récipient\end{définition}
\begin{définition}\cmn 罐子
\begin{déclaration} \étymologie{\papi{\stylefn{罐子}}}\end{déclaration}\end{définition}
\end{entrée}

\begin{entrée}
\vedette{\hypertarget{Ⓔkoŋi}{\papi{ koŋi}}}\markboth{koŋi}{}
\classe{n}
\begin{définition}\fra joug pour un animal\end{définition}
\begin{définition}\cmn 牛轭(单行)\end{définition}
\begin{exemple}\jya jla ɣɯ koŋi\cmn 犏牛轭\end{exemple}\end{entrée}

\begin{entrée}
\vedette{\hypertarget{Ⓔkoŋla}{\papi{ koŋla}}}\markboth{koŋla}{} (\variante{kuŋula}) 
\classe{n}\acception{1}
\begin{définition}\fra chose vraie\end{définition}
\begin{définition}\cmn 真实\end{définition}
\begin{exemple}\jya koŋla tɤ-ti\cmn 你要说真话(不要开玩笑)\end{exemple}\acception{2}
\begin{définition}\fra vraiment\end{définition}
\begin{définition}\cmn 真地\end{définition}\end{entrée}

\begin{entrée}
\vedette{\hypertarget{Ⓔkoŋmarɟɤlpu}{\papi{ koŋmarɟɤlpu}}}\markboth{koŋmarɟɤlpu}{}\classe{n}
\begin{définition}\fra empereur\end{définition}
\begin{définition}\cmn 皇帝
\begin{déclaration} \étymologie{\papi{goŋ.ma rgʲal.po}}\end{déclaration}\end{définition}
\end{entrée}

\begin{entrée}
\vedette{\hypertarget{Ⓔkoŋtaʁ}{\papi{ koŋtaʁ}}}\markboth{koŋtaʁ}{}\classe{n}
\begin{définition}\fra lanière avant de la selle\end{définition}
\begin{définition}\cmn 马鞍的前绳\end{définition}\end{entrée}

\begin{entrée}
\vedette{\hypertarget{Ⓔkóʁmɯz}{\papi{ kóʁmɯz}}}\markboth{kóʁmɯz}{}\classe{adv}\acception{1}
\begin{définition}\fra à l'instant\end{définition}
\begin{définition}\cmn 刚才\end{définition}
\begin{exemple}\jya ɬamu kɯ nɯ kóʁmɯz pɯ-kɯ-fse nɯ ra pjɤ-fɕɤt\cmn 拉莫给他讲了刚才发生的事\end{exemple}
\begin{exemple}\jya aʑo nɯ kóʁmɯz @xiaban pɯ-βzu-t-a tɕe, kha lɤ-nɯɣe-a\cmn 我刚刚下班回家了\end{exemple}
\begin{exemple}\jya nɯ kóʁmɯz nɤ kha jɤ-azɣɯt-a\cmn 我刚刚才到家\end{exemple}\acception{2}
\begin{définition}\fra alors seulement\end{définition}
\begin{définition}\cmn 这才\end{définition}
\begin{relation-sémantique}\confer{
\hyperlink{Ⓔnóʁmɯz}{\textit{ \papi{nóʁmɯz}}}
}\end{relation-sémantique}\end{entrée}

\begin{entrée}
\vedette{\hypertarget{Ⓔkosca}{\papi{ kosca}}}\markboth{kosca}{}
\classe{n}
\begin{définition}\fra cuir non teint\end{définition}
\begin{définition}\cmn 没有染色的皮子
\begin{déclaration} \étymologie{\papi{ko.ba.skʲa}}\end{déclaration}\end{définition}\end{entrée}

\begin{entrée}
\vedette{\hypertarget{Ⓔkota}{\papi{ kota}}}\markboth{kota}{}\classe{n}
\begin{définition}\fra sac en peau que l'on porte à l'épaule\end{définition}
\begin{définition}\cmn 毛皮子缝成的挎包\end{définition}
\end{entrée}

\begin{entrée}
\vedette{\hypertarget{Ⓔkota}{\papi{ kota}}}\markboth{kota}{}\classe{n}
\begin{définition}\fra petit sac de cuir\end{définition}
\begin{définition}\cmn 皮子做的小口袋\end{définition}\end{entrée}

\begin{entrée}
\vedette{\hypertarget{Ⓔkowa}{\papi{ kowa}}}\markboth{kowa}{}\classe{n}
\begin{définition}\fra méthode\end{définition}
\begin{définition}\cmn 办法
\begin{déclaration} \étymologie{\papi{bkol.ba}}\end{déclaration}\end{définition}
\begin{exemple}\jya ɯʑo ɯ-kowa tu\cmn 他有办法\end{exemple}
\begin{relation-sémantique}\confer{
\hyperlink{Ⓔnɯkowa}{\textit{ \papi{nɯkowa}}}
}\end{relation-sémantique}\end{entrée}

\begin{entrée}
\vedette{\hypertarget{Ⓔkoxtɕɯn}{\papi{ koxtɕɯn}}}\markboth{koxtɕɯn}{}
\classe{n}
\begin{définition}\fra soie\end{définition}
\begin{définition}\cmn 丝绸
\begin{déclaration} \étymologie{\papi{gos.tɕʰen}}\end{déclaration}\end{définition}
\begin{exemple}\jya koxtɕɯn kɯ-qarŋe ci ɲɯ-ŋu\end{exemple}\end{entrée}

\begin{entrée}
\vedette{\hypertarget{Ⓔkoxtɕɯnri}{\papi{ koxtɕɯnri}}}\markboth{koxtɕɯnri}{}\classe{n}
\begin{définition}\fra soie\end{définition}
\begin{définition}\cmn 丝绸
\begin{déclaration} \étymologie{\papi{gos.tɕʰen.ras}}\end{déclaration}\end{définition}
\end{entrée}

\begin{entrée}
\vedette{\hypertarget{Ⓔkoʑi}{\papi{ koʑi}}}\markboth{koʑi}{}\classe{n}
\begin{définition}\fra poutre\end{définition}
\begin{définition}\cmn 横梁\end{définition}
\begin{exemple}\jya koʑi komɤl nɯ jɤɣɤt laχtsɯ ɯ-taʁ tɤpjaʁ nɯ ŋu, nɯ jɤɣɤt ɯ-taʁ tɯ-sthoʁsi ɯ-kɯ-sthoʁ nɯ ŋu, rɟɯɣ sɤznɤ xtshɯm\cmn 
\stylefv{koʑi komɤl}是走缘的柱头上的木方条,支撑走缘上面的小梁,比\stylefv{rɟɯɣ}细一些
\end{exemple}\end{entrée}

\begin{entrée}
\vedette{\hypertarget{Ⓔkupa}{\papi{ kupa}}}\markboth{kupa}{}\classe{n}
\begin{définition}\fra chinois\end{définition}
\begin{définition}\cmn 汉人\end{définition}
\begin{exemple}\jya kupa skɤt kɤ-βzu a-pɯ-nɯ-me ɲɯ-sɯsam-a\cmn 我不想说汉语\end{exemple}
\end{entrée}

\begin{entrée}
\vedette{\hypertarget{Ⓔkupajmɤɣ}{\papi{ kupajmɤɣ}}}\markboth{kupajmɤɣ}{}\classe{n}
\begin{définition}\fra matsutake\end{définition}
\begin{définition}\cmn 松茸\end{définition}
\end{entrée}

\begin{entrée}
\vedette{\hypertarget{Ⓔkupaŋga}{\papi{ kupaŋga}}}\markboth{kupaŋga}{}\classe{n}
\begin{définition}\fra habits chinois / occidentaux\end{définition}
\begin{définition}\cmn 汉装\end{définition}
\begin{relation-sémantique}\confer{
\hyperlink{Ⓔtɯ-ŋga}{\textit{ \papi{tɯ-ŋga}}}
}\end{relation-sémantique}
\begin{relation-sémantique}\confer{
\hyperlink{Ⓔkupa}{\textit{ \papi{kupa}}}
}\end{relation-sémantique}
\end{entrée}

\begin{entrée}
\vedette{\hypertarget{Ⓔkuparmbatɕɯβ}{\papi{ kuparmbatɕɯβ}}}\markboth{kuparmbatɕɯβ}{}\classe{n}
\begin{définition}\fra espèce de plante\end{définition}
\begin{définition}\cmn 【冬寒菜】\end{définition}
\end{entrée}

\begin{entrée}
\vedette{\hypertarget{Ⓔkupastaχpɯ}{\papi{ kupastaχpɯ}}}\markboth{kupastaχpɯ}{}\classe{n}
\begin{définition}\fra soja\end{définition}
\begin{définition}\cmn 黄豆\end{définition}
\begin{relation-sémantique}\confer{
\hyperlink{Ⓔstaχpɯ}{\textit{ \papi{staχpɯ}}}
}\end{relation-sémantique}
\end{entrée}

\begin{entrée}
\vedette{\hypertarget{Ⓔkupastoʁ}{\papi{ kupastoʁ}}}\markboth{kupastoʁ}{}\classe{n}
\begin{définition}\fra pois cultivable toute l'année\end{définition}
\begin{définition}\cmn 四季豆\end{définition}
\begin{relation-sémantique}\confer{
\hyperlink{Ⓔstoʁ}{\textit{ \papi{stoʁ}}}
}\end{relation-sémantique}
\end{entrée}

\begin{entrée}
\vedette{\hypertarget{Ⓔkra}{\papi{ kra}}}\markboth{kra}{}
\classe{vt}
\paradigme{\textit{dir :} \jya pɯ-}
\begin{définition}\fra faire tomber\end{définition}
\begin{définition}\cmn 打落\end{définition}
\begin{exemple}\jya jɯfɕo jiʑo ji-ʑɴɢɯloʁ ɕ-pɯ-kra-t-a\cmn 今天早上我去把我们的核桃打下来了\end{exemple}
\begin{exemple}\jya ʑɴɢɯloʁ ɕ-pɯ-kre\cmn 你去把核桃打下来\end{exemple}
\begin{relation-sémantique}\confer{
\hyperlink{Ⓔŋgra}{\textit{ \papi{ŋgra}}}
}\end{relation-sémantique}\end{entrée}

\begin{entrée}
\vedette{\hypertarget{Ⓔkrɤɣ}{\papi{ krɤɣ}}}\markboth{krɤɣ}{}\classe{vt}
\paradigme{\textit{dir :} \jya kɤ-}
\paradigme{\textit{dir :} \jya nɯ-}
\paradigme{\textit{dir :} \jya thɯ-}
\begin{définition}\fra couper\end{définition}
\begin{définition}\cmn 割\end{définition}
\begin{exemple}\jya aʑo nɯ-kraɣ-a\cmn 我割了(草)\end{exemple}
\begin{exemple}\jya sɯjno na-krɤɣ\cmn 他割了草\end{exemple}
\begin{exemple}\jya paʁndza ka-krɤɣ\cmn 他割了猪草\end{exemple}
\begin{exemple}\jya qaʑo thɯ-krɤɣ\cmn 你给绵羊剃毛\end{exemple}
\begin{exemple}\jya a-ndzrɯ ɲɯ-ɣɤzri tɕe ɲɯ-kraɣ-a ntsɯ ɲɯ-ra\cmn 我的指甲长得很快,我必须经常剪\end{exemple}\begin{sous-entrée}
\vedette{\hypertarget{}{\papi{ nɯɣɯkrɤɣ}}}\markboth{nɯɣɯkrɤɣ}{}\classe{vs}
\begin{définition}\fra facile à tondre\end{définition}
\begin{définition}\cmn 容易剪毛(绵羊)\end{définition}
\begin{exemple}\jya ki qaʑo ki ɲɯ-nɯɣɯkrɤɣ\cmn 这只羊容易剪毛\end{exemple}
\end{sous-entrée}\begin{sous-entrée}
\vedette{\hypertarget{}{\papi{ sɯkrɤɣ}}}\markboth{sɯkrɤɣ}{}\classe{vt}
\begin{définition}\fra couper avec\end{définition}
\begin{définition}\cmn 用……割\end{définition}
\begin{exemple}\jya tɯɲcɣa kɯ kú-wɣ-sɯkrɤɣ ra\cmn 要用镰刀割\end{exemple}
\end{sous-entrée}\end{entrée}

\begin{entrée}
\vedette{\hypertarget{Ⓔkrɤlma}{\papi{ krɤlma}}}\markboth{krɤlma}{}\classe{n}
\begin{définition}\fra colique\end{définition}
\begin{définition}\cmn 婴儿的肠病
\begin{déclaration} \étymologie{\papi{grol.ma}}\end{déclaration}\end{définition}
\begin{exemple}\jya krɤlma nɤrŋi ɯ-ŋgo ŋu, ɲɯ-nɯtɯfɕɤl tɕe ɯ-qe kɯ-qarŋe ŋu, karɣi ɯ-mɯntoʁ fse\cmn 
\stylefv{krɤlma}是婴儿的病,他拉肚子时大便是黄色的,像是菜子花。
\end{exemple}
\begin{relation-sémantique}\confer{
\hyperlink{Ⓔnɯkrɤlma}{\textit{ \papi{nɯkrɤlma}}}
}\end{relation-sémantique}\end{entrée}

\begin{entrée}
\vedette{\hypertarget{Ⓔkrul}{\papi{ krul}}}\markboth{krul}{}\classe{vi}
\paradigme{\textit{dir :} \jya pɯ-}
\begin{définition}\fra finir (cérémonie religieuse)\end{définition}
\begin{définition}\cmn 做完(宗教仪式)
\begin{déclaration} \étymologie{\papi{grol}}\end{déclaration}\end{définition}
\begin{exemple}\jya sɲaŋne kɤ-ndo pɯ-krul-a\cmn 我把哑巴经念完了\end{exemple}
\begin{exemple}\jya χpɯn ra kɤ-ɣɤrpi pjɤ-krul-nɯ\cmn 和尚们念完经了\end{exemple}\end{entrée}

\begin{entrée}
\vedette{\hypertarget{Ⓔkro}{\papi{ kro}}}\markboth{kro}{}\classe{vt}
\paradigme{\textit{dir :} \jya pɯ-}
\paradigme{\textit{dir :} \jya thɯ-}
\begin{définition}\fra partager, distribuer\end{définition}
\begin{définition}\cmn 分东西\end{définition}
\begin{exemple}\jya pɯ-kro-t-a\cmn 我分了\end{exemple}
\begin{exemple}\jya kɯki laχtɕha ki pɯ-krɤm\cmn 你把这个东西分(给大家)\end{exemple}\begin{sous-entrée}
\vedette{\hypertarget{}{\papi{ nɤkɯkro}}}\markboth{nɤkɯkro}{}\classe{vt}
\paradigme{\textit{dir :} \jya nɯ-}
\begin{définition}\fra partager avec tout le monde\end{définition}
\begin{définition}\cmn 分来分去\end{définition}
\end{sous-entrée}\begin{sous-entrée}
\vedette{\hypertarget{}{\papi{ nɯkro}}}\markboth{nɯkro}{}\classe{vt}
\paradigme{\textit{dir :} \jya pɯ-}
\begin{définition}\ 
\begin{déclaration}\grammar{autoben}\end{déclaration}\end{définition}
\begin{définition}\fra se partager\end{définition}
\begin{définition}\cmn 自己分东西;彼此分东西\end{définition}
\begin{exemple}\jya tʂha pɯ-nɯkro-tɕi\cmn 我们俩分了茶\end{exemple}
\begin{exemple}\jya laχtɕha pɯ-nɯkro-tɕi\cmn 我们俩分了东西\end{exemple}
\begin{exemple}\jya kɤ-ndza pɯ-nɯkro-tɕi\cmn 我们俩分了东西吃\end{exemple}
\end{sous-entrée}\begin{sous-entrée}
\vedette{\hypertarget{}{\papi{ rɤkro}}}\markboth{rɤkro}{}\classe{vi}
\begin{définition}\ 
\begin{déclaration}\grammar{apass}\end{déclaration}\end{définition}
\begin{définition}\fra partager des choses\end{définition}
\begin{définition}\cmn 分东西\end{définition}
\end{sous-entrée}\begin{sous-entrée}
\vedette{\hypertarget{}{\papi{ znɯkro}}}\markboth{znɯkro}{}\classe{vt}
\paradigme{\textit{dir :} \jya pɯ-}
\begin{définition}\fra partager avec\end{définition}
\begin{définition}\cmn 分给\end{définition}
\end{sous-entrée}\begin{sous-entrée}
\vedette{\hypertarget{}{\papi{ ʑɣɤkro}}}\markboth{ʑɣɤkro}{}\classe{vi}
\paradigme{\textit{dir :} \jya pɯ-}
\begin{définition}\ 
\begin{déclaration}\grammar{refl}\end{déclaration}\end{définition}
\begin{définition}\fra se séparer\end{définition}
\begin{définition}\cmn 分开(一个群体)\end{définition}
\end{sous-entrée}\end{entrée}

\begin{entrée}
\vedette{\hypertarget{Ⓔkroŋwa}{\papi{ kroŋwa}}}\markboth{kroŋwa}{}\classe{n}
\begin{définition}\fra mal de ventre\end{définition}
\begin{définition}\cmn 肚子痛\end{définition}\end{entrée}

\begin{entrée}
\vedette{\hypertarget{Ⓔkropa}{\papi{ kropa}}}\markboth{kropa}{}\classe{n}
\begin{définition}\fra serviteur\end{définition}
\begin{définition}\cmn 仆人\end{définition}
\begin{exemple}\jya βlama mɤ-χsɤl, kropa χsɤl\cmn 喇嘛糊涂,用人清楚\end{exemple}\end{entrée}

\begin{entrée}
\vedette{\hypertarget{Ⓔkrɯβthoβ}{\papi{ krɯβthoβ}}}\markboth{krɯβthoβ}{}\classe{n}
\begin{définition}\fra sprulsku qui peut se marier\end{définition}
\begin{définition}\cmn 可以娶妻的活佛
\begin{déclaration} \étymologie{\papi{grub.thob}}\end{déclaration}\end{définition}\end{entrée}

\begin{entrée}
\vedette{\hypertarget{ⒺkɯⒽ1}{\papi{ kɯ}}}\markboth{kɯ}{}\homonyme{1}
\classe{postp}\acception{1}
\begin{définition}\fra ergatif\end{définition}
\begin{définition}\cmn 施事格\end{définition}\acception{2}
\begin{définition}\fra instrumental\end{définition}
\begin{définition}\cmn 工具格\end{définition}
\begin{exemple}\jya mbrɯtɕɯ kɯ ʑɴɢɯloʁ nɯ-sɯphaʁ-a\cmn 我用刀子把核桃撬开了\end{exemple}\acception{3}
\begin{définition}\fra conjonction\end{définition}
\begin{définition}\cmn 连词\end{définition}\acception{4}
\begin{définition}\fra marque du comparé dans la construction comparative\end{définition}
\begin{définition}\cmn 比较句中表示比较主体\end{définition}
\begin{exemple}\jya ɯ-ʁi sɤz ɯ-pi nɯ kɯ mpɕɤr\cmn 姐姐比妹妹漂亮\end{exemple}\end{entrée}

\begin{entrée}
\vedette{\hypertarget{ⒺkɯⒽ2}{\papi{ kɯ}}}\markboth{kɯ}{}\homonyme{2}
\classe{part}
\begin{définition}\fra marque de question\end{définition}
\begin{définition}\cmn 表示疑问\end{définition}
\begin{exemple}\jya jiɕqha nɯ tɕhi pɯ-rmi kɯ\cmn 刚才那个人叫什么名字呢?\end{exemple}
\end{entrée}

\begin{entrée}
\vedette{\hypertarget{Ⓔkɯβdɤsqi}{\papi{ kɯβdɤsqi}}}\markboth{kɯβdɤsqi}{}\classe{num}
\begin{définition}\fra quarante\end{définition}
\begin{définition}\cmn 四十\end{définition}\end{entrée}

\begin{entrée}
\vedette{\hypertarget{Ⓔkɯβde}{\papi{ kɯβde}}}\markboth{kɯβde}{}\classe{num}
\begin{définition}\fra quatre\end{définition}
\begin{définition}\cmn 四\end{définition}\end{entrée}

\begin{entrée}
\vedette{\hypertarget{Ⓔkɯβʁa}{\papi{ kɯβʁa}}}\markboth{kɯβʁa}{}\classe{n}
\begin{définition}\fra noble\end{définition}
\begin{définition}\cmn 贵族\end{définition}
\end{entrée}

\begin{entrée}
\vedette{\hypertarget{Ⓔkɯchu}{\papi{ kɯchu}}}\markboth{kɯchu}{}
\classe{adv}
\begin{définition}\fra à l'est\end{définition}
\begin{définition}\cmn 在东边\end{définition}
\begin{exemple}\jya kɯchu ɯ-rkɯ ri ku-rɤʑi-a\cmn 我在东边\end{exemple}
\begin{relation-sémantique}\confer{
\hyperlink{Ⓔɯ-kɤcu}{\textit{ \papi{ɯ-kɤcu}}}
}\end{relation-sémantique}\end{entrée}

\begin{entrée}
\vedette{\hypertarget{Ⓔkɯchi}{\papi{ kɯchi}}}\markboth{kɯchi}{}\classe{n}
\begin{définition}\fra sucre\end{définition}
\begin{définition}\cmn 糖\end{définition}\end{entrée}

\begin{entrée}
\vedette{\hypertarget{Ⓔkɯɕmar}{\papi{ kɯɕmar}}}\markboth{kɯɕmar}{}
\classe{n}
\begin{définition}\fra céréales\end{définition}
\begin{définition}\cmn 麦类\end{définition}\end{entrée}

\begin{entrée}
\vedette{\hypertarget{Ⓔkɯɕmɤtʂu}{\papi{ kɯɕmɤtʂu}}}\markboth{kɯɕmɤtʂu}{}
\classe{n}
\begin{définition}\fra allumette\end{définition}
\begin{définition}\cmn 火柴\end{définition}\end{entrée}

\begin{entrée}
\vedette{\hypertarget{Ⓔkɯɕnɤsqi}{\papi{ kɯɕnɤsqi}}}\markboth{kɯɕnɤsqi}{}\classe{num}
\begin{définition}\fra soixante-dix\end{définition}
\begin{définition}\cmn 七十\end{définition}\end{entrée}

\begin{entrée}
\vedette{\hypertarget{Ⓔkɯɕnom}{\papi{ kɯɕnom}}}\markboth{kɯɕnom}{}
\classe{n}
\begin{définition}\fra épi\end{définition}
\begin{définition}\cmn 穗子\end{définition}
\begin{relation-sémantique}\confer{
\hyperlink{Ⓔrɯkɯɕnom}{\textit{ \papi{rɯkɯɕnom}}}
}\end{relation-sémantique}\end{entrée}

\begin{entrée}
\vedette{\hypertarget{Ⓔkɯɕnɯz}{\papi{ kɯɕnɯz}}}\markboth{kɯɕnɯz}{}\classe{num}
\begin{définition}\fra sept\end{définition}
\begin{définition}\cmn 七\end{définition}\end{entrée}

\begin{entrée}
\vedette{\hypertarget{Ⓔkɯɕpaz}{\papi{ kɯɕpaz}}}\markboth{kɯɕpaz}{}\classe{n}
\begin{définition}\fra marmotte\end{définition}
\begin{définition}\cmn 旱獭\end{définition}
\begin{exemple}\jya kɯɕpaz nɯ rɯŋgu tsa ku-rɤʑi ŋu, ɯ-ɣɲɟɯ ɯ-ŋgɯ ku-rɤʑi ɲɯ-ŋu, kɯ-wxti kɯ ɣnɤsqi tɯ-rpa jamar tu, ʁzɯɣ kɯ-ɤɣɯrnɯɕɯr tu, ɯ-scɯʁzɯɣ βɣɯz cho naχtɕɯɣ tsa, qartsɯ kɤ-tsu tɕe, ɯ-ɣɲɟɯ ɯ-ŋgɯ lu-cɯ ɲɯ-ŋgrɤl, wuma ʑo tshu ɲɯ-ŋgrɤl, ftɕar tɕe chɯ-nɯɬoʁ, cɯ tɤkha tɕe pɕaʁ tu-βze tɕe ``ɣɯjpa qartsɯ ɕawu rambɯm a-taʁ a-mɤ-ɣɯ-kɤ-rŋgɯ smɯlɤm" tu-ti ɲɯ-ŋgrɤl, ma ɯʑo qartsɯ tɕe wuma ɲɯ-tshu, ɕawu rambɯm wuma ʑo kɯ-wxti tɕe kɯ-mpja ɲɯ-ŋu, tɕe kɯɕpaz ɯ-taʁ ɯ-stu nɯ tɕu kɤ-rŋgɯ tɕe, ɯ-tɯ-mpja kɯ kɯɕpaz nɯ pjɯ́-wɣ-ftʂi ɲɯ-ŋgrɤl, tɕe núndʐa kɯɕpaz khɤfɕɤt tu-βze ɲɯ-ŋgrɤl. kɯ-ɣɤrʁaʁ ra kɯ ɯ-ɣɲɟɯ ɯ-kɯm zɯ tɤ-rcoʁ rɯlɯ thɤstɯɣ kɯ-tu tu-rtsi-nɯ tɕe, ɯ-ŋgɯ kɯɕpaz thɤstɯɣ tu nɯ pjɯ-sɯχsɤl-nɯ ɲɯ-ŋgrɤl. mbroχpa kɯ fsapaʁ ɯ-βraʁ ŋu tu-ti-nɯ ɲɯ-ŋgrɤl, tɕe tú-wɣ-nɤrʁaʁ qha-nɯ, ɯ-sɤ-dɤn nɯ mbroχpa sɤtɕha ŋu.\cmn 旱獭生活在高山的洞穴里。大的有二十来斤,是红色的,外貌有点像獾。冬天长肥了的时候,就会在洞穴冬眠。据说要冬眠时,它拱着手祈祷:“今年冬天不希望有独角公鹿来睡在我的上面”,因为它在冬天很肥,独角公鹿很大,热量高,会让旱獭融化掉,所以旱獭要祈祷。猎人们数着旱獭洞口贴着的泥球就知道洞穴里住着多少只旱獭。放牧人说它是牲畜的象征,所以讨厌猎捕。旱獭繁殖比较多的地方在牧区。\end{exemple}
\end{entrée}

\begin{entrée}
\vedette{\hypertarget{Ⓔkɯɕte}{\papi{ kɯɕte}}}\markboth{kɯɕte}{}\classe{adv}
\begin{définition}\fra autre\end{définition}
\begin{définition}\cmn 另外;其他\end{définition}
\begin{exemple}\jya kɯɕte nɯ-tɯ-ta-t ŋu ɯ-maʁ\cmn 你是不是放在另外一个地方\end{exemple}
\begin{exemple}\jya kɯɕte tɯ-ŋga tɤ-ŋge\end{exemple}
\begin{exemple}\jya tɯ-ŋga kɯɕte tɤ-ŋge\cmn 你穿另外一件衣服吧\end{exemple}\end{entrée}

\begin{entrée}
\vedette{\hypertarget{Ⓔkɯɕɯŋgɯ}{\papi{ kɯɕɯŋgɯ}}}\markboth{kɯɕɯŋgɯ}{}\classe{n}
\begin{définition}\fra autrefois\end{définition}
\begin{définition}\cmn 古时候\end{définition}
\end{entrée}

\begin{entrée}
\vedette{\hypertarget{Ⓔkɯfɕi}{\papi{ kɯfɕi}}}\markboth{kɯfɕi}{}
\classe{n}
\begin{définition}\fra forgeron\end{définition}
\begin{définition}\cmn 铁匠\end{définition}
\begin{exemple}\jya kɯfɕi ɣɯ ɯ-mbrɯtɕɯ kɤ-ntɕhoz me, ɕoŋβzu ɯ-sɤmdzɯ me\cmn 铁匠没有刀子用,木匠没有凳子坐\end{exemple}\end{entrée}

\begin{entrée}
\vedette{\hypertarget{Ⓔkɯɣe}{\papi{ kɯɣe}}}\markboth{kɯɣe}{}\classe{part}
\begin{définition}\fra question à soi-même\end{définition}
\begin{définition}\cmn 自我反问\end{définition}
\end{entrée}

\begin{entrée}
\vedette{\hypertarget{Ⓔkɯjujmɤlu}{\papi{ kɯjujmɤlu}}}\markboth{kɯjujmɤlu}{}\classe{n}
\begin{définition}\fra animal sans queue (homme)\end{définition}
\begin{définition}\cmn 没有尾巴的动物(人)
\begin{déclaration}\use{古语}\end{déclaration}\end{définition}
\begin{exemple}\jya kɯjujmɤlu ɲɯ-sɲu ŋu tɕe, ma-ɕ-thɯ-tɯ-ʑɣɤ-βde ma ji-kɤ-ndza tu-tu ɕti\cmn (乌鸦说:)你不要去投河自尽,当那个没有尾巴的动物(指人)发疯的时候(指播种子)我们就有吃的了\end{exemple}\end{entrée}

\begin{entrée}
\vedette{\hypertarget{Ⓔkɯjka}{\papi{ kɯjka}}}\markboth{kɯjka}{}\classe{n}
\begin{définition}\fra corbeau à bec rouge (pyrrhocorax pyrrhocorax)\end{définition}
\begin{définition}\cmn 红嘴山鸦【红嘴老鸦】\end{définition}
\begin{exemple}\jya kɯjka nɯ pɣa kɯ-ɲaʁ ci ŋu, qajdo cho ndʑi-tɯ-wxti ndʑi-tshɯɣa ra naχtɕɯɣ, kɯjka nɯ ɯ-βri nɤmbju, ɯ-mtsioʁ cho ɯ-mi nɯ ra kɯ-ɣɯrni ŋu, jɤɣɤt pa znde ɯ-kɯ-spoʁ kɯ-wxti nɯ ra ku-rɤloʁ ŋu, ɯ-loʁ ɯ-spa nɯ si ɯ-rtaʁ kɯ-xtshɯm tsa z-ju-nɯzʁe tɕe ɯ-kɯr ɯ-ŋgɯ ʁɟa ʑo tu-rke tɕe z-ju-nɯ-zʁe ŋu, ʁzɤmi ni tu-oqurle-ndʑi tɕe ku-rɤloʁ-ndʑi ŋu. tu-mbri tɕe, ``ka ca kɤɣ" ntsɯ tu-ti ŋu. tɯ-mɯ cho sɤrwa lɤt tɤkha tɕe wuma ʑo mbri ma sɤrwa ɯ-kɯ-sɯ-lɤt tɤ-rca ɯ-ku a-rku tu-kɯ-ti ɲɯ-ŋu. tɤ-rɤku kɤ-ndza χɕu tɕe pɯ-kɤ-nɯji ɣɯ ɯ-rɣi ra kɯnɤ tu-nɯ-tɕɤt ɕti. srɯnmɯ ŋu tu-kɯ-ti ŋgrɤl.\cmn 
红嘴老鸦是一种黑色的鸟,形状和大小和乌鸦一样,毛有光泽,嘴和脚是红色的。在走缘下面和墙壁上比较大的洞里作窝,垒窝的材料是比较细的树枝,全部是用嘴衔来的,公鸦和母鸦一起垒窝。叫声是\stylefv{ka ca kɤɣ}。下雨或下冰雹时,叫声特别欢快,据说它会参与商议下冰雹的事情。它吃粮食很厉害,连种下去的种子也会挖出来吃。人家说它是妖精。
\end{exemple}
\end{entrée}

\begin{entrée}
\vedette{\hypertarget{Ⓔkɯjkɤkɕi}{\papi{ kɯjkɤkɕi}}}\markboth{kɯjkɤkɕi}{}
\classe{n}
\begin{définition}\fra belette\end{définition}
\begin{définition}\cmn 黄喉貂\end{définition}
\begin{exemple}\jya kɯjkɤkɕi nɯ ʁnɯz nɤ ʁnɯz nɯ tɯtɯrca ku-rɤʑi ɲɯ-ŋgrɤl, ca tu-βɟi ŋgrɤl, ca nɯ sɯku tɤ-ari tɕe ɯ-rcɯrca sɯku tu-ɕe tɕe ɕ-pjɯ-ɣɤrɤt ŋgrɤl. pa-sat tɕe, ca ɯ-rme tu-ndze tu-z-mɤke, ɯ-qhu tɕe nɯ kóʁmɯz ɯ-ɕa tu-ndze chɯ-ɕkɯt-nɯ mɤ-cha. ɯ-mdoʁ nɯ kɯ-ɲaʁ kɯ-ɤɣɯrnɯɕɯr ŋu, kɯ-xtshɯm kɯ-zri kɯ-fse ŋu, ɯ-mtɕhi kɯ-ɤmtɕoʁ, ɯ-rna qachɣa ɯ-rna kɯ-fse ɲɯ-ŋu, ɯ-jme kɯ-zɯ-zri ŋu.\cmn 黄喉貂一对一对地呆在一起,会追麝香鹿,麝香鹿上树了,它也跟着上树,最后把麝香鹿甩下来,把它弄死。吃麝香鹿的时候,先吃完麝香鹿的毛再吃肉,所以吃不完。颜色是黑里带红的,身子细而长,嘴尖,耳朵像狐狸的耳朵,尾巴很长。\end{exemple}\end{entrée}

\begin{entrée}
\vedette{\hypertarget{Ⓔkɯjŋu}{\papi{ kɯjŋu}}}\markboth{kɯjŋu}{}
\classe{n}
\begin{définition}\fra serment\end{définition}
\begin{définition}\cmn 誓言\end{définition}
\begin{exemple}\jya kɯjŋu to-joʁ\cmn 他发誓了\end{exemple}
\begin{exemple}\jya kɯjŋu pjɤ-ta\cmn 他发誓了\end{exemple}
\begin{relation-sémantique}\confer{
\hyperlink{Ⓔnɯkɯjŋu}{\textit{ \papi{nɯkɯjŋu}}}
}\end{relation-sémantique}\end{entrée}

\begin{entrée}
\vedette{\hypertarget{Ⓔkɯjra}{\papi{ kɯjra}}}\markboth{kɯjra}{}\classe{n}
\begin{définition}\fra la plus longue corde d'une tente\end{définition}
\begin{définition}\cmn 帐篷最长的拉线\end{définition}
\end{entrée}

\begin{entrée}
\vedette{\hypertarget{Ⓔkɯjtɯm}{\papi{ kɯjtɯm}}}\markboth{kɯjtɯm}{}\classe{n}
\begin{définition}\ 
\begin{déclaration}\grammar{n.lieu}\end{déclaration}\end{définition}
\begin{définition}\fra l'un des hameaux de Gyutshapa\end{définition}
\begin{définition}\cmn 二茶村的大队之一\end{définition}\end{entrée}

\begin{entrée}
\vedette{\hypertarget{Ⓔkɯki}{\papi{ kɯki}}}\markboth{kɯki}{}\classe{dem}
\begin{définition}\fra ceci\end{définition}
\begin{définition}\cmn 这个\end{définition}
\end{entrée}

\begin{entrée}
\vedette{\hypertarget{Ⓔkɯkrɯ}{\papi{ kɯkrɯ}}}\markboth{kɯkrɯ}{} (\variante{krɯkrɯ}) \classe{idph.2}
\begin{définition}\fra découper en morceaux\end{définition}
\begin{définition}\cmn 割很多刀,切成几块\end{définition}
\begin{exemple}\jya tɯrme ɯ-ŋga nɯ kɯkrɯ ʑo pjɤ-ta\cmn 他在别人的衣服上割了很多刀\end{exemple}
\begin{exemple}\jya paʁ pɯ́-wɣ-ntɕha tɕe kɯkrɯ ʑo ɣɯ-ta ra\cmn 宰猪的时候,要切成很多块\end{exemple}
\begin{relation-sémantique}\confer{
\hyperlink{Ⓔrɤkɯkrɯ}{\textit{ \papi{rɤkɯkrɯ}}}
}\end{relation-sémantique}\end{entrée}

\begin{entrée}
\vedette{\hypertarget{Ⓔkɯlɤɣpopo}{\papi{ kɯlɤɣpopo}}}\markboth{kɯlɤɣpopo}{}\classe{n}
\begin{définition}\fra coccinelle\end{définition}
\begin{définition}\cmn 瓢虫\end{définition}
\begin{exemple}\jya kɯ-lɤɣ popo nɯ qajɯ ci ŋu, tɕe ɯ-βri nɯ ɣɯrni ɯ-taʁ kɯ-ɲaʁ kɯ-ɤkhra tu, rko, ɯ-ʁar ɣɯ ɯ-rqhu ɲɯ-ŋu, ɯ-ʁar nɯ li kɯ-mba tɕe ɯ-rɯmu kɯ-tu ci ŋu, tɕe tɤ-pɤtso ra kɯ nɯ-jaʁ ɯ-taʁ tu-sɯxɕe-nɯ tɕe, ɯ-khɯkha tu-ti-nɯ kɯ ``a-wɯ kɯ-lɤɣ popo, ŋotɕu tɯ-ɕɯ-ɕe nɤ-qhɯ-qhu ɣi-a nɤ-ŋga nɤ-xtsa fkur-a", tɕe nɯ-jaʁndzu ɯ-ku tɤ-nɯɬoʁ tɕe, ju-nɯqambɯmbjom ŋu tɕe, ci ci kɯ-ɤrqhɯ-rqhi ju-ɕe ŋu, ci ci kɯ-ɤrmbɯ-rmbat ku-rɤʑi ŋu tɕe, ŋotɕu jɤ-ari ɯ-pɕoʁ nɯ tɕu thɯ-kɯ-wxti tɕe nɯ-pɕoʁ tɯ-sɤɣ-ɕe ŋu tu-kɯ-ti ɲɯ-ŋgrɤl.\cmn 瓢虫是一种虫,身子是红色的,上面有黑点,很硬,是翅膀的外壳。翅膀很薄,有纹路。小孩子们让它在手指上爬,并说:“瓢虫爷爷,不管你去哪里我都会在后面跟着你,我要背上你的衣服、鞋子”。当瓢虫爬到手指的顶端时,就会飞走,有时飞得远,有时在很近的地方停下来。据说飞到哪个方向,小孩子长大以后就会往哪里去。\end{exemple}\end{entrée}

\begin{entrée}
\vedette{\hypertarget{Ⓔkɯm}{\papi{ kɯm}}}\markboth{kɯm}{}
\classe{n}
\begin{définition}\fra porte\end{définition}
\begin{définition}\cmn 门\end{définition}
\begin{relation-sémantique}\confer{
\hyperlink{Ⓔkumpɣa}{\textit{ \papi{kumpɣa}}}
}\end{relation-sémantique}
\begin{relation-sémantique}\confer{
\hyperlink{Ⓔkumpɣɤtɕɯ}{\textit{ \papi{kumpɣɤtɕɯ}}}
}\end{relation-sémantique}\end{entrée}

\begin{entrée}
\vedette{\hypertarget{Ⓔkɯmu}{\papi{ kɯmu}}}\markboth{kɯmu}{}\classe{n}
\begin{définition}\fra tétras (tetraogallus tibetanus)\end{définition}
\begin{définition}\cmn 藏雪鸡【贝母鸡】\end{définition}
\begin{exemple}\jya kɯmu nɯ pɤjmu ruŋgu zgoku stu kɯ-mbro ku-rɤʑi ŋu, ɯ-mdoʁ kɯ-pɣi ŋu. kɯ-wxti ra kɯ kɯmu kɯ ``kɯ-ɣɤndʐo nɤ kɯ-ɣɤndʐo a-tɤ-nɯχsɤl, kɯ-mpja nɤ kɯ-mpja a-tɤ-nɯχsɤl" tu-ti ŋu tu-ti-nɯ ŋgrɤl ma tɤ-ɣɤndʐo tɕe rdzari ɯ-ku tu-ɕe ŋu, nɯ-mpja tɕe, co zɯ pjɯ-ɣi ŋu. chɯsɲu tu raŋ tɕe, kɯmu ɯ-skɤt a-pɯ-mtshɤm tɕe, phɤn tu-kɯ-ti ɲɯ-ŋgrɤl, kupa kɯ pɤjmu ɯ-kɯ-tɕɤt pɣa ŋu tu-kɯ-ti ɲɯ-ŋgrɤl, tɕe ŋu maʁ mɤ-xsi. tu-mbri tɕe, ``ku ku ku ku ku" tu-ti ŋu. χsɯm kɯβde tɯtɯrca tu-ŋgrɤl.\cmn 
贝母鸡栖息在最高山上,颜色是灰的。老人们有个说法:贝母鸡说:“要冷就冷个够,要热就热个够”,因为它在天气寒冷时到高山上去,天气温暖时就在山沟里出现。当有狂犬病的时候,据说患者听见贝母鸡的叫声就会好起来。汉人说是挖贝母的鸡,不知是不是正确的。叫的时候,叫声是\stylefv{ku ku ku ku ku}。贝母鸡一般三四只一起活动。
\end{exemple}\end{entrée}

\begin{entrée}
\vedette{\hypertarget{Ⓔkɯma}{\papi{ kɯma}}}\markboth{kɯma}{}\classe{part}
\begin{définition}\fra question\end{définition}
\begin{définition}\cmn 会不会\end{définition}
\begin{exemple}\jya kɯtɕu ku-tɯ-rɤʑi tɕe aʑo ju-okhu-a tɯ-mtshɤm ɕi kɯma\cmn 你在这里你会不会听见我叫你呢?\end{exemple}
\end{entrée}

\begin{entrée}
\vedette{\hypertarget{Ⓔkɯmaʁ}{\papi{ kɯmaʁ}}}\markboth{kɯmaʁ}{}
\classe{pro}
\begin{définition}\fra autre\end{définition}
\begin{définition}\cmn 其他\end{définition}
\begin{exemple}\jya li kɯmaʁ nɯ-ari ɕti ma\cmn (我们)又离题了\end{exemple}
\begin{exemple}\jya kɯmaʁ tɯrme\cmn 另一个人\end{exemple}
\begin{relation-sémantique}\confer{
\hyperlink{ⒺmaʁⒽ1}{\textit{ \papi{maʁ1}}}
}\end{relation-sémantique}\end{entrée}

\begin{entrée}
\vedette{\hypertarget{Ⓔkɯmɤɕtʂa}{\papi{ kɯmɤɕtʂa}}}\markboth{kɯmɤɕtʂa}{}\classe{adv}
\begin{définition}\fra jusqu'à maintenant\end{définition}
\begin{définition}\cmn 一直到现在\end{définition}\end{entrée}

\begin{entrée}
\vedette{\hypertarget{Ⓔkɯmɤlɤxso}{\papi{ kɯmɤlɤxso}}}\markboth{kɯmɤlɤxso}{}\classe{adv}
\begin{définition}\fra pour rien\end{définition}
\begin{définition}\cmn 白白,徒劳\end{définition}
\begin{exemple}\jya tɤ-ndze ma kɯmɤlɤxso a-mɤ-thɯ-ɤrɕo\cmn 你吃吧,不要浪费\end{exemple}
\begin{exemple}\jya kɯmɤlɤxso a-mɤ-ɕ-thɯ́-wɣ-βde ma nɤja\cmn 不要浪费,太可惜了\end{exemple}
\begin{relation-sémantique}\confer{
\hyperlink{Ⓔɯ-xso}{\textit{ \papi{ɯ-xso}}}
}\end{relation-sémantique}
\begin{relation-sémantique}\confer{
\hyperlink{Ⓔso}{\textit{ \papi{so}}}
}\end{relation-sémantique}\end{entrée}

\begin{entrée}
\vedette{\hypertarget{Ⓔkɯmɕku}{\papi{ kɯmɕku}}}\markboth{kɯmɕku}{}
\classe{n}
\begin{définition}\fra ail\end{définition}
\begin{définition}\cmn 大蒜\end{définition}
\begin{exemple}\jya kɯmɕku nɯ tɯ-ji ɯ-ŋgɯ lu-kɤ-nɯ-ji ci ŋu, ɯ-qa nɯ ɯ-tɯm rmi, ɯ-tɯm ɣɯ ɯ-ndzoʁ tu. ɯ-jwaʁ ma ɯ-ru me, ɯ-jwaʁ nɯ kɯ-pɣi tsa ŋu, kɯ-tɕɤr kɯ-rɲɟi tsa ŋu, ɯ-tho tu ri ɯ-mɯntoʁ me, ɯ-qa nɯ ɕku ɲɯ-βze ŋu, ɯ-zrɤm dɤn tɕe wɣrum, ɯ-qa nɯ-aβzu tɕe, tɯ-ndzoʁ tɯ-ndzoʁ ɯ-spjɯŋ ɯ-taʁ ku-fskɤr ŋu, ɯ-pɕi ɯ-rqhu kɤntɕhɯ-tɤlɤβ kɯ tu-oluj ŋu, ɯ-ndzoʁ raŋri nɯ li ɯ-rqhu tu, ɣɯrni. thɯ-tɯt ɯ-jija nɯ ɲɯ-mba ŋu. ɯ-ndzoʁ tsuku kɯβde ma me, tsuku ɕnɤcɤt jamar kɯ-tu tu, tɕeri ɯ-ndzoʁ tɤ-wxti tɕe rkɯn, ɯ-ndzoʁ tɤ-xtɕi tɕe dɤn. tsuku tɯ-rdoʁ ma kɯ-me tu, tɕe nɯ ɕku phɤri rmi.\cmn 
大蒜是是自己种在地里的(农作物)。根叫蒜头,蒜头有几个蒜瓣。只有叶子没有茎,叶子带有灰色,又窄又长,有花梗但是不开花,在根部结蒜头,是一瓣一瓣地围着主心干而长的,外面有很多层皮裹着,每个蒜瓣有自己的皮,是红色的。随着大蒜的成熟,外层的皮变得越来越薄。有的只有四个蒜瓣,有的有七八个。蒜瓣越大就越少,越小就越多。有的只有一个,这种叫\stylefv{ɕku phɤri}(对面的葱)。
\end{exemple}
\begin{relation-sémantique}\confer{
\hyperlink{Ⓔɕku}{\textit{ \papi{ɕku}}}
}\end{relation-sémantique}
\begin{relation-sémantique}\confer{
\hyperlink{Ⓔkɯm}{\textit{ \papi{kɯm}}}
}\end{relation-sémantique}\end{entrée}

\begin{entrée}
\vedette{\hypertarget{Ⓔkɯmdza}{\papi{ kɯmdza}}}\markboth{kɯmdza}{}
\classe{n}
\begin{définition}\fra membres de la famille\end{définition}
\begin{définition}\cmn 亲戚
\begin{déclaration} \étymologie{\papi{mdza}}\end{déclaration}
\begin{déclaration} \étymologie{\papi{mdzaɦ}}\end{déclaration}\end{définition}
\begin{exemple}\jya kɯmdza mɤ-arɕɤt-i\cmn 我们没有血缘关系\end{exemple}
\begin{exemple}\jya kɯmdza tu-j\cmn 我们有血缘关系\end{exemple}\end{entrée}

\begin{entrée}
\vedette{\hypertarget{Ⓔkɯmŋu}{\papi{ kɯmŋu}}}\markboth{kɯmŋu}{}\classe{num}
\begin{définition}\fra cinq\end{définition}
\begin{définition}\cmn 五\end{définition}\end{entrée}

\begin{entrée}
\vedette{\hypertarget{Ⓔkɯmŋɤsqi}{\papi{ kɯmŋɤsqi}}}\markboth{kɯmŋɤsqi}{}\classe{num}
\begin{définition}\fra cinquante\end{définition}
\begin{définition}\cmn 五十\end{définition}
\begin{relation-sémantique}\confer{
\hyperlink{Ⓔsqi}{\textit{ \papi{sqi}}}
}\end{relation-sémantique}\end{entrée}

\begin{entrée}
\vedette{\hypertarget{Ⓔkɯmrka}{\papi{ kɯmrka}}}\markboth{kɯmrka}{}\classe{n}
\begin{définition}\fra poutre au-dessus de la porte\end{définition}
\begin{définition}\cmn 门框的上梁\end{définition}\end{entrée}

\begin{entrée}
\vedette{\hypertarget{Ⓔkɯmʁla}{\papi{ kɯmʁla}}}\markboth{kɯmʁla}{}
\classe{n}
\begin{définition}\fra cadre de la porte\end{définition}
\begin{définition}\cmn 门框\end{définition}\end{entrée}

\begin{entrée}
\vedette{\hypertarget{Ⓔkɯmtɕhoχsɯm}{\papi{ kɯmtɕhoχsɯm}}}\markboth{kɯmtɕhoχsɯm}{}\classe{n}
\begin{définition}\fra Triratna\end{définition}
\begin{définition}\cmn 三宝
\begin{déclaration} \étymologie{\papi{kun.mtɕʰog.gsum}}\end{déclaration}\end{définition}\end{entrée}

\begin{entrée}
\vedette{\hypertarget{Ⓔkɯmtɕhɯ}{\papi{ kɯmtɕhɯ}}}\markboth{kɯmtɕhɯ}{}\classe{n}
\begin{définition}\fra jouet\end{définition}
\begin{définition}\cmn 玩具\end{définition}
\begin{relation-sémantique}\confer{
\hyperlink{Ⓔnɯkɯmtɕhɯ}{\textit{ \papi{nɯkɯmtɕhɯ}}}
}\end{relation-sémantique}\end{entrée}

\begin{entrée}
\vedette{\hypertarget{Ⓔkɯmthoʁdɤn}{\papi{ kɯmthoʁdɤn}}}\markboth{kɯmthoʁdɤn}{}\classe{n}
\begin{définition}\fra seuil\end{définition}
\begin{définition}\cmn 门槛
\begin{déclaration} \étymologie{\papi{gdan}}\end{déclaration}\end{définition}
\begin{relation-sémantique}\confer{
\hyperlink{Ⓔtɤ-ʁdɤn}{\textit{ \papi{tɤ-ʁdɤn}}}
}\end{relation-sémantique}
\begin{relation-sémantique}\confer{
\hyperlink{Ⓔkɯm}{\textit{ \papi{kɯm}}}
}\end{relation-sémantique}
\end{entrée}

\begin{entrée}
\vedette{\hypertarget{Ⓔkɯnɤ}{\papi{ kɯnɤ}}}\markboth{kɯnɤ}{}\classe{adv}
\begin{définition}\fra aussi\end{définition}
\begin{définition}\cmn 也是\end{définition}
\begin{exemple}\jya ɯʑo ku-nɯkho ɕti ri, nɯ tɕu kɯnɤ ɯ-kɯ-ra ɲɯ-dɤn\cmn 他是借住的,居然还有那么多要求\end{exemple}
\end{entrée}

\begin{entrée}
\vedette{\hypertarget{Ⓔkɯndzarmɯ}{\papi{ kɯndzarmɯ}}}\markboth{kɯndzarmɯ}{}
\classe{n}
\begin{définition}\fra ondée\end{définition}
\begin{définition}\cmn 阵雨(以后不再下雨的预兆)\end{définition}
\begin{exemple}\jya kɯndzarmɯ ɲɯ-ɤsɯ-βzu tɕe, ki ɯ-qhu tɕe mɤ-lɤt ɲɯ-ŋu\cmn 现在在下阵雨,以后不会再下\end{exemple}
\begin{relation-sémantique}\confer{
\hyperlink{Ⓔtɯ-mɯ}{\textit{ \papi{tɯ-mɯ}}}
}\end{relation-sémantique}
\begin{relation-sémantique}\confer{
\hyperlink{Ⓔndzar}{\textit{ \papi{ndzar}}}
}\end{relation-sémantique}
\end{entrée}

\begin{entrée}
\vedette{\hypertarget{Ⓔkɯngɯsqi}{\papi{ kɯngɯsqi}}}\markboth{kɯngɯsqi}{}\classe{num}
\begin{définition}\fra quatre-vingt dix\end{définition}
\begin{définition}\cmn 九十\end{définition}
\begin{relation-sémantique}\confer{
\hyperlink{Ⓔsqi}{\textit{ \papi{sqi}}}
}\end{relation-sémantique}\end{entrée}

\begin{entrée}
\vedette{\hypertarget{Ⓔkɯngɯt}{\papi{ kɯngɯt}}}\markboth{kɯngɯt}{}\classe{num}
\begin{définition}\fra neuf\end{définition}
\begin{définition}\cmn 九\end{définition}\end{entrée}

\begin{entrée}
\vedette{\hypertarget{Ⓔkɯngɯt tɤrqhɤɴɢaʁ}{\papi{ kɯngɯt tɤrqhɤɴɢaʁ}}}\markboth{kɯngɯt tɤrqhɤɴɢaʁ}{}\classe{n}
\begin{définition}\fra une espèce d'arbrisseau\end{définition}
\begin{définition}\cmn 灌木的一种\end{définition}
\begin{exemple}\jya kɯngɯt tɤrqhɤɴɢaʁ nɯ si kɯ-mbro tsa ci ŋu, ɯ-mdoʁ kɯ-pɣi tsa ci ŋu, ɯ-ru ɣɯ ɯ-rqhu pjɤ-ɴɢaʁ nɤ pjɤ-ɴɢaʁ ŋu, wuma ʑo mɤ-mbro tɕe kɤ-ntɕhoz kɯ-sna tu-βze mɤ-cha. ɲɯ́-wɣ-phɯt tɕe, kɤ-nɯ-βlɯ sna. ɯ-rqhu kɯ-dɤn núndʐa, kɯngɯt tɤrqhɤɴɢaʁ ɲɯ-rmi.\cmn 
\stylefv{kɯngɯt tɤrqhɤɴɢaʁ}是一种长的不高的树,颜色是土灰色的,树干的树皮不断的脱落。因为树长得不高,不能用来做什么。砍下后可以用来烧火。因为树皮多,所以叫\stylefv{kɯngɯt tɤrqhɤɴɢaʁ}(脱了九层皮)
\end{exemple}\end{entrée}

\begin{entrée}
\vedette{\hypertarget{Ⓔkɯngɯt tɤrtsɤɣ}{\papi{ kɯngɯt tɤrtsɤɣ}}}\markboth{kɯngɯt tɤrtsɤɣ}{}\classe{n}
\begin{définition}\fra Leonurus sp.\end{définition}
\begin{définition}\cmn 益母草\end{définition}
\begin{exemple}\jya kɯngɯt tɤrtsɤɣ nɯ sɯjno ɯ-rtaʁ mɤ-kɯ-dɤn, mɤ-kɯ-mbro tsa ci ŋu. ɯ-ru nɯ kɯ-ɤβʑɯrdu tu-ɬoʁ ŋu, ɯ-rtsɤɣ tu-oʑɯrja ŋu, ɯ-rtsɤɣ raŋri nɯ tɕu ɯ-jwaʁ kɯ-xtɕɯ-xtɕi ɲɯ-βze tɕe, ɯ-rca nɯ tɕu ɯ-mɯntoʁ kɯ-dɯ-dɤn ʑo ko-fskɤr, tɕe ɯ-mat chɯ-βze ŋu. kɯ-ɕɯŋgɯ tɕe, tɤ-mu ra kɯ kɯngɯt tɤrtsɤɣ ɲɯ-phɯt-nɯ tɕe, tɤ-lu kɤ-kɤ-z-rɤɕom ɯ-taʁ nɯ tɕu ɲɯ-ta-nɯ tɕe, tɤlɤɕom kɯ-jaʁ ku-te cha tu-ti-nɯ pɯ-ŋu. kɯngɯt tɤrtsɤɣ nɯ kha kɯngɯ-rtsɤɣ kɤ-ti ɲɯ-ŋu.\cmn 
益母草是没有很多枝桠,长得不高的草。茎是四方形的,是由几节组成的,在每一节上,茎的一周长出小叶子,然后长出小花,最后结果。以前,大娘们把它扯一根,放在凉着结奶皮的牛奶上,据说这样可以使奶皮结得好一些。 \stylefv{kɯngɯt tɤrtsɤɣ} 是“九层楼”的意思。
\end{exemple}
\end{entrée}

\begin{entrée}
\vedette{\hypertarget{Ⓔkɯni}{\papi{ kɯni}}}\markboth{kɯni}{}\classe{dem}
\begin{définition}\fra ces deux choses\end{définition}
\begin{définition}\cmn 这两个\end{définition}
\end{entrée}

\begin{entrée}
\vedette{\hypertarget{Ⓔkɯnɯkhamu}{\papi{ kɯnɯkhamu}}}\markboth{kɯnɯkhamu}{}\classe{n}
\begin{définition}\fra cuisinier\end{définition}
\begin{définition}\cmn 炊事员\end{définition}
\begin{relation-sémantique}\confer{
\hyperlink{Ⓔnɯkhamu}{\textit{ \papi{nɯkhamu}}}
}\end{relation-sémantique}
\end{entrée}

\begin{entrée}
\vedette{\hypertarget{Ⓔkɯnɯmar}{\papi{ kɯnɯmar}}}\markboth{kɯnɯmar}{}\classe{n}
\begin{définition}\fra une espèce d'arbrisseau\end{définition}
\begin{définition}\cmn 灌木的一种\end{définition}
\begin{exemple}\jya kɯnɯmar nɯ si ci ŋu wuma ʑo tu-mbro cho ɲɯ-jpɯm mɤ-cha. ɯ-jwaʁ nɯ kɯ-tɕɤr kɯ-rɲɟi tsa ŋu, ɯ-ru nɯ kɯ-ɲaʁ ɯ-ŋgɯz kɯ-ɣɯrni tsa ŋu, ɯ-ru ɯ-ŋgɯ nɯ kɯ-wɣrum kɯ-fse xtsoŋxtsoŋ ɕti tɕe, si mɤ-ngɯt. ɯ-mɯntoʁ kɯ-wɣrum kɯ-dɤn ʑo ɲɯ-ɬoʁ ŋu. si ɯ-kɤχcɤl ri ɲɯ-lɤt ŋu.\cmn 
\stylefv{kɯnɯmar}是一种树,不能长高,也能长粗。叶子细长,树干黑里透红。树干里有白色泡沫,所以木质不结实。开很多白花,集中在树梢上。
\end{exemple}
\end{entrée}

\begin{entrée}
\vedette{\hypertarget{Ⓔkɯnɯwulaʁ}{\papi{ kɯnɯwulaʁ}}}\markboth{kɯnɯwulaʁ}{}\classe{n}
\begin{définition}\fra personne assujettie à la corvée\end{définition}
\begin{définition}\cmn 专为土司背东西的人(乌拉)
\begin{déclaration} \étymologie{\papi{ɦu.lag}}\end{déclaration}\end{définition}
\end{entrée}

\begin{entrée}
\vedette{\hypertarget{Ⓔkɯɲidi}{\papi{ kɯɲidi}}}\markboth{kɯɲidi}{}\classe{n}
\begin{définition}\fra odeur d'homme\end{définition}
\begin{définition}\cmn 人的气味
\begin{déclaration}\use{传统故事中,魔鬼说的话}\end{déclaration}\end{définition}
\end{entrée}

\begin{entrée}
\vedette{\hypertarget{Ⓔkɯpɤz}{\papi{ kɯpɤz}}}\markboth{kɯpɤz}{}\classe{n}
\begin{définition}\fra un type de ver\end{définition}
\begin{définition}\cmn 虫的一种\end{définition}
\end{entrée}

\begin{entrée}
\vedette{\hypertarget{Ⓔkɯqurʑŋgri}{\papi{ kɯqurʑŋgri}}}\markboth{kɯqurʑŋgri}{}\classe{n}
\begin{définition}\fra vénus, étoile du matin\end{définition}
\begin{définition}\cmn 金星\end{définition}
\begin{relation-sémantique}\confer{
\hyperlink{Ⓔqur}{\textit{ \papi{qur}}}
}\end{relation-sémantique}
\begin{relation-sémantique}\confer{
\hyperlink{Ⓔʑŋgri}{\textit{ \papi{ʑŋgri}}}
}\end{relation-sémantique}
\end{entrée}

\begin{entrée}
\vedette{\hypertarget{Ⓔkɯra}{\papi{ kɯra}}}\markboth{kɯra}{} (\variante{kɯkɯra}) \classe{dem}
\begin{définition}\fra ces choses\end{définition}
\begin{définition}\cmn 这些\end{définition}
\end{entrée}

\begin{entrée}
\vedette{\hypertarget{Ⓔkɯrcat}{\papi{ kɯrcat}}}\markboth{kɯrcat}{}\classe{num}
\begin{définition}\fra huit\end{définition}
\begin{définition}\cmn 八\end{définition}\end{entrée}

\begin{entrée}
\vedette{\hypertarget{Ⓔkɯrcɤsqi}{\papi{ kɯrcɤsqi}}}\markboth{kɯrcɤsqi}{}\classe{num}
\begin{définition}\fra quatre-vingt\end{définition}
\begin{définition}\cmn 八十\end{définition}
\begin{relation-sémantique}\confer{
\hyperlink{Ⓔsqi}{\textit{ \papi{sqi}}}
}\end{relation-sémantique}\end{entrée}

\begin{entrée}
\vedette{\hypertarget{Ⓔkɯre}{\papi{ kɯre}}}\markboth{kɯre}{}\classe{adv}
\begin{définition}\fra ici\end{définition}
\begin{définition}\cmn 在这里\end{définition}
\end{entrée}

\begin{entrée}
\vedette{\hypertarget{Ⓔkɯrnɯ}{\papi{ kɯrnɯ}}}\markboth{kɯrnɯ}{}\classe{n}
\begin{définition}\fra mite\end{définition}
\begin{définition}\cmn 蛀虫\end{définition}
\begin{exemple}\jya kɯrnɯ nɯ qajɯ kɯ-ɣɯrni ci ŋu, ɯ-jme ri ɯ-rme kɯ-tu ci ŋu, zndɤrchɤβ ɯ-thoʁ aʁɤndɯndɤt ʑo tu, ɯ-βri ra lo-rɤrkhɯrkhe kɯ-fse ŋu, ɯʑo xtɕi ri wuma ʑo ŋɤn, ma tɯ-ŋga smɤɣ thɯ-kɤ-βzu kɯ-fse nɯ ra tu-ndze ŋu, tɯ-jpu tu-ndze ŋu, tu-ndze mɤ-kɯ-jɤɣ kɯ ku-sɯ-ɤɴqhi ŋu. tɯ-jpu ra ɯ-ntɕha-ntɕhɯr ɲɯ-sɤβze ŋu, tɕe kɤ-ɣɤme ftɕaka nɯ tɯ-ŋga ɯ-taʁ tɕe qajɯ sɤ-sat ɣɯ smɤn tú-wɣ-lɤt, tɤ-rɤku nɯ tɯ-ci ɯ-ŋgɯ pjɯ́-wɣ-ɣɤla tɕe, tɕe qajɯ cho tɤ-rɤku ɯ-ntɕha-ntɕhɯr nɯ ra tɯ-ci kɯ ɯ-taʁ tu-ɣɯt ŋu ma ʑo tɕe, tɕe tú-wɣ-tɕɤt tɕe ɲɯ́-wɣ-ɣɤme khɯ.\cmn 蛀虫是一种红色的虫,尾巴上有一撮毛,墙壁缝里地上到处都有。身子像割了几刀,有类似刀刻的花纹。蛀虫虽然小但是有害,吃毛织品的衣服,吃粮食,不但吃还要把它弄脏,把粮食弄成一块一块的渣滓一样。消灭的办法是在衣服上喷杀虫剂,把粮食泡在水里,虫和粮食渣滓就会浮上水面,这样拿出来就可以消灭它。\end{exemple}
\end{entrée}

\begin{entrée}
\vedette{\hypertarget{Ⓔkɯrŋi}{\papi{ kɯrŋi}}}\markboth{kɯrŋi}{}\classe{n}
\begin{définition}\fra fauve\end{définition}
\begin{définition}\cmn 猛兽\end{définition}\end{entrée}

\begin{entrée}
\vedette{\hypertarget{Ⓔkɯrŋukɯɣndʑɯr}{\papi{ kɯrŋukɯɣndʑɯr}}}\markboth{kɯrŋukɯɣndʑɯr}{}\classe{n}
\begin{définition}\fra faucheur\end{définition}
\begin{définition}\cmn 盲蛛\end{définition}
\begin{exemple}\jya kɯrŋu kɯɣndʑɯr ɯ-mi kɯ-rɲɟɯ-rɲɟi ʑo kɯrcɤ-ldʑa tu, ɯ-phoŋbu nɯ kɯ-ɤrtɯ-rtɯm rloʁrloʁ kɯ-xtɕi tsa ŋu, ɯ-taʁ ri ɯ-mɲaʁ tɯ-tɕha ɣɤʑu, ɯ-phoŋbu cho ɯ-mi ra lonba ɲɯ-ɲaʁ, ɯ-mi nɯ ɯ-phoŋbu ku-fskɤr ʑo ku-ndzoʁ ɲɯ-ŋu. ɯ-mi nɯ-mbɯt tɕe tɤ-rʑaʁ kɯ-rɲɟi ʑo tu-mɯnmu ɲɯ-cha. kɯrŋu kɯɣndʑɯr ɯʑo ɯ-kɯ-ndza ɣɤʑu ma ɯʑo kɯ ɯ-zda kɤ-ndza mɯ́j-cha.\cmn 盲蛛有八只很长的脚,身子是球形的,很小,上面有一对眼睛,脚和身体全是黑色的,脚处于身体的周围。拔掉了脚后还能动一段时间。有动物吃盲蛛,但它不吃其它动物。\end{exemple}
\begin{relation-sémantique}\confer{
\hyperlink{Ⓔɣndʑɯr}{\textit{ \papi{ɣndʑɯr}}}
}\end{relation-sémantique}\end{entrée}

\begin{entrée}
\vedette{\hypertarget{Ⓔkɯrowɤro}{\papi{ kɯrowɤro}}}\markboth{kɯrowɤro}{}\classe{n}
\begin{définition}\fra en trop\end{définition}
\begin{définition}\cmn 多余的\end{définition}
\begin{exemple}\jya kɯrowɤro ʑo ma-tɯ-nɤme\cmn 你不要做多余的工作\end{exemple}
\begin{exemple}\jya kɯrowɤro ʑo ma-tɯ-nɯskɤt\cmn 你不要讲多余的话\end{exemple}
\begin{exemple}\jya tɯ-rju kɯ-rowɤro nɯ ra mɤ-ra\cmn 不要有多余的话\end{exemple}\end{entrée}

\begin{entrée}
\vedette{\hypertarget{Ⓔkɯroz}{\papi{ kɯroz}}}\markboth{kɯroz}{}\classe{adv}
\begin{définition}\fra spécialement, particulièrement\end{définition}
\begin{définition}\cmn 特别\end{définition}
\begin{exemple}\jya ɯʑo kɯroz ʑo mbro\cmn 他长得特别高\end{exemple}
\end{entrée}

\begin{entrée}
\vedette{\hypertarget{Ⓔkɯrtsɤɣ}{\papi{ kɯrtsɤɣ}}}\markboth{kɯrtsɤɣ}{}
\classe{n}
\begin{définition}\fra panthère\end{définition}
\begin{définition}\cmn 豹子
\begin{déclaration} \étymologie{\papi{gzigs}}\end{déclaration}\end{définition}
\begin{exemple}\jya kɯrtsɤɣ nɯ aʁɤndɯndɤt tu ŋgrɤl, ɯ-mdoʁ kɤ-qarŋɯrŋe ɯ-taʁ kɯ-ɲaʁ kɯ-ɤkhra ŋu, ɯ-mtɕhirme kɯ-wɣrum ŋu, ɯ-mɤlɤjaʁ ɯ-ndzrɯ cho lɯlu ɯ-mɤndzrɯ fse, ɯ-jme kɯnɤ akhra, ɯ-jme ɯ-ku tɯ-snaʁ kɯ-wɣrum ŋu. tshɤt qaʑo wuma kɤ-ndza rga, nɯŋa kɯ-fse li pjɯ-sat cha, jla mbro kɯ-fse nɯ ra kɯ-dɤn mɤ-kɤm.\cmn 豹子到处都有,颜色是淡黄色,上有黑色的斑点,胡须是白色的,爪子和猫的一样,尾巴也是花的,尾巴的尖端有白点。很喜欢吃羊,也能杀死奶牛,但没有能力打败犏牛和马。\end{exemple}\end{entrée}

\begin{entrée}
\vedette{\hypertarget{Ⓔkɯrɯ}{\papi{ kɯrɯ}}}\markboth{kɯrɯ}{}\classe{n}
\begin{définition}\fra tibétain\end{définition}
\begin{définition}\cmn 藏族\end{définition}
\begin{exemple}\jya kɯrɯ skɤt nɯ sɤʁʑi ɯ-ku zɯ tɯskɤt stu kɯ-mpɕɤr ɕti.\cmn 藏语是世界上最好听的语言\end{exemple}
\end{entrée}

\begin{entrée}
\vedette{\hypertarget{Ⓔkɯrɯɕɤmɯɣdɯ}{\papi{ kɯrɯɕɤmɯɣdɯ}}}\markboth{kɯrɯɕɤmɯɣdɯ}{}\classe{n}
\begin{définition}\fra arquebuse\end{définition}
\begin{définition}\cmn 民火枪\end{définition}
\begin{relation-sémantique}\confer{
\hyperlink{Ⓔɕɤmɯɣdɯ}{\textit{ \papi{ɕɤmɯɣdɯ}}}
}\end{relation-sémantique}
\end{entrée}

\begin{entrée}
\vedette{\hypertarget{Ⓔkɯrɯɕoŋβzu}{\papi{ kɯrɯɕoŋβzu}}}\markboth{kɯrɯɕoŋβzu}{}\classe{n}
\begin{définition}\fra menuisier\end{définition}
\begin{définition}\cmn 木匠
\begin{déclaration} \étymologie{\papi{ɕiŋ.bzo}}\end{déclaration}\end{définition}
\end{entrée}

\begin{entrée}
\vedette{\hypertarget{Ⓔkɯrɯŋga}{\papi{ kɯrɯŋga}}}\markboth{kɯrɯŋga}{}\classe{n}
\begin{définition}\fra habits tibétains\end{définition}
\begin{définition}\cmn 藏装\end{définition}
\begin{relation-sémantique}\confer{
\hyperlink{Ⓔtɯ-ŋga}{\textit{ \papi{tɯ-ŋga}}}
}\end{relation-sémantique}
\begin{relation-sémantique}\confer{
\hyperlink{Ⓔkɯrɯ}{\textit{ \papi{kɯrɯ}}}
}\end{relation-sémantique}
\end{entrée}

\begin{entrée}
\vedette{\hypertarget{Ⓔkɯsɤɣru}{\papi{ kɯsɤɣru}}}\markboth{kɯsɤɣru}{}
\classe{n}
\begin{définition}\fra miroir\end{définition}
\begin{définition}\cmn 镜子\end{définition}
\begin{exemple}\jya aʑo kɯsɤɣru ɯ-ŋgɯ kɤ-ru-a\cmn 我照了镜子\end{exemple}
\begin{relation-sémantique}\synonyme{
\hyperlink{Ⓔχɕɤlzgoŋ}{\textit{ \papi{χɕɤlzgoŋ}}}
}\end{relation-sémantique}
\begin{relation-sémantique}\confer{
\hyperlink{ⒺruⒽ1}{\textit{ \papi{ru1}}}
}\end{relation-sémantique}
\end{entrée}

\begin{entrée}
\vedette{\hypertarget{Ⓔkɯspoʁ}{\papi{ kɯspoʁ}}}\markboth{kɯspoʁ}{}\classe{n}
\begin{définition}\fra trou\end{définition}
\begin{définition}\cmn 洞\end{définition}
\begin{relation-sémantique}\confer{
\hyperlink{Ⓔspoʁ}{\textit{ \papi{spoʁ}}}
}\end{relation-sémantique}
\end{entrée}

\begin{entrée}
\vedette{\hypertarget{Ⓔkɯsthi}{\papi{ kɯsthi}}}\markboth{kɯsthi}{}\classe{n}
\begin{définition}\fra autant\end{définition}
\begin{définition}\cmn 这么多\end{définition}
\end{entrée}

\begin{entrée}
\vedette{\hypertarget{Ⓔkɯtaʁ}{\papi{ kɯtaʁ}}}\markboth{kɯtaʁ}{}\classe{n}
\begin{définition}\fra geste\end{définition}
\begin{définition}\cmn 动作\end{définition}
\end{entrée}

\begin{entrée}
\vedette{\hypertarget{Ⓔkɯtɕapɯ}{\papi{ kɯtɕapɯ}}}\markboth{kɯtɕapɯ}{}\classe{n}
\begin{définition}\fra roturier\end{définition}
\begin{définition}\cmn 平民\end{définition}
\end{entrée}

\begin{entrée}
\vedette{\hypertarget{Ⓔkɯtʂɤɣ}{\papi{ kɯtʂɤɣ}}}\markboth{kɯtʂɤɣ}{}\classe{num}
\begin{définition}\fra six\end{définition}
\begin{définition}\cmn 六\end{définition}\end{entrée}

\begin{entrée}
\vedette{\hypertarget{Ⓔkɯtʂɤsqi}{\papi{ kɯtʂɤsqi}}}\markboth{kɯtʂɤsqi}{}\classe{num}
\begin{définition}\fra soixante\end{définition}
\begin{définition}\cmn 六十\end{définition}
\begin{relation-sémantique}\confer{
\hyperlink{Ⓔsqi}{\textit{ \papi{sqi}}}
}\end{relation-sémantique}\end{entrée}

\begin{entrée}
\vedette{\hypertarget{Ⓔkɯz}{\papi{ kɯz}}}\markboth{kɯz}{}\classe{intj}
\begin{définition}\fra vas-y d'abord!\end{définition}
\begin{définition}\cmn 你在前面走!\end{définition}
\begin{exemple}\jya kɯz, tɤ-mbɣom!\cmn 走吧,快点!\end{exemple}\end{entrée}

\begin{entrée}
\vedette{\hypertarget{Ⓔkɯzɣa}{\papi{ kɯzɣa}}}\markboth{kɯzɣa}{}\classe{adv}
\begin{définition}\fra très longtemps, de nombreuses fois, vraiment, décidément\end{définition}
\begin{définition}\cmn 很久;很多次;确实\end{définition}
\begin{exemple}\jya kɯzɣa ʑo ɲɤ-ɕar\cmn 找了很久,费了很大的工夫\end{exemple}
\begin{exemple}\jya kɯzɣa ʑo ɲɯ-tɯ-mtsɯr !\cmn 你确实很饿\end{exemple}
\end{entrée}

\begin{entrée}
\vedette{\hypertarget{Ⓔkɯʑŋgu}{\papi{ kɯʑŋgu}}}\markboth{kɯʑŋgu}{}\classe{n}
\begin{définition}\ 
\begin{déclaration}\grammar{n.lieu}\end{déclaration}\end{définition}
\begin{définition}\fra l'un des hameaux de Gyutshapa\end{définition}
\begin{définition}\cmn 二茶村的大队之一\end{définition}\end{entrée}

\begin{entrée}
\vedette{\hypertarget{Ⓔkuwu}{\papi{ kuwu}}}\markboth{kuwu}{}\classe{n}
\begin{définition}\fra gypaète barbu\end{définition}
\begin{définition}\cmn 胡兀鹫
\begin{déclaration} \étymologie{\papi{ko.bo}}\end{déclaration}\end{définition}
\end{entrée}

\begin{entrée}
\vedette{\hypertarget{Ⓔkwitsɯt}{\papi{ kwitsɯt}}}\markboth{kwitsɯt}{}\classe{n}
\begin{définition}\fra armoire\end{définition}
\begin{définition}\cmn 柜子
\begin{déclaration} \étymologie{\papi{\stylefn{柜子}}}\end{déclaration}\end{définition}
\end{entrée}

\begin{entrée}
\vedette{\hypertarget{Ⓔkuxtɕo}{\papi{ kuxtɕo}}}\markboth{kuxtɕo}{}\classe{n}
\begin{définition}\fra hotte\end{définition}
\begin{définition}\cmn 背篼\end{définition}
\begin{exemple}\jya ɯ-kɯxtɕɯxtɕo ʑo\cmn 一背篼一背篼的\end{exemple}\end{entrée}

\newpage\caractère{l}

\begin{entrée}
\vedette{\hypertarget{Ⓔlu}{\papi{ lu}}}\markboth{lu}{}\classe{n}
\begin{définition}\fra signe astrologique\end{définition}
\begin{définition}\cmn 生肖
\begin{déclaration} \étymologie{\papi{lo}}\end{déclaration}\end{définition}
\begin{exemple}\jya nɤʑo tɕhi lu tɯ-ŋu\cmn 你属什么?\end{exemple}
\begin{exemple}\jya a-lu na-rtsi\cmn 他给我算了命\end{exemple}
\begin{relation-sémantique}\confer{
 \papi{ɯ-lu,rtsɯβ}
}\end{relation-sémantique}\end{entrée}

\begin{entrée}
\vedette{\hypertarget{Ⓔla}{\papi{ la}}}\markboth{la}{}\classe{vt}\acception{1}
\paradigme{\textit{dir :} \jya nɯ-}
\begin{définition}\fra s'imbiber d'eau\end{définition}
\begin{définition}\cmn 泡胀\end{définition}
\begin{exemple}\jya stoʁ nɯ-la ri tɕe ɲɯ-mpɯ ŋu\cmn 胡豆浸泡胀了就会变软\end{exemple}\acception{2}
\paradigme{\textit{dir :} \jya pɯ-}
\begin{définition}\fra tomber dans l'eau\end{définition}
\begin{définition}\cmn 掉进水里\end{définition}
\begin{exemple}\jya tɯ-ci ɯ-ŋgɯ pjɤ-la (=pjɤ-ɕe)\cmn 他掉进水里了\end{exemple}
\begin{relation-sémantique}\confer{
\hyperlink{Ⓔɣɤla}{\textit{ \papi{ɣɤla}}}
}\end{relation-sémantique}\end{entrée}

\begin{entrée}
\vedette{\hypertarget{Ⓔlaβde}{\papi{ laβde}}}\markboth{laβde}{}\classe{num}
\begin{définition}\fra à peu près quatre\end{définition}
\begin{définition}\cmn 大概四个\end{définition}
\begin{relation-sémantique}\confer{
\hyperlink{Ⓔkɯβde}{\textit{ \papi{kɯβde}}}
}\end{relation-sémantique}
\end{entrée}

\begin{entrée}
\vedette{\hypertarget{Ⓔlaβdelɤŋu}{\papi{ laβdelɤŋu}}}\markboth{laβdelɤŋu}{}\classe{num}
\begin{définition}\fra quatre ou cinq\end{définition}
\begin{définition}\cmn 四五个\end{définition}
\begin{exemple}\jya laβdelɤŋu-sŋi\cmn 四五天\end{exemple}
\begin{relation-sémantique}\confer{
\hyperlink{Ⓔkɯβde}{\textit{ \papi{kɯβde}}}
}\end{relation-sémantique}
\begin{relation-sémantique}\confer{
\hyperlink{Ⓔkɯmŋu}{\textit{ \papi{kɯmŋu}}}
}\end{relation-sémantique}\end{entrée}

\begin{entrée}
\vedette{\hypertarget{Ⓔlaβzɣi}{\papi{ laβzɣi}}}\markboth{laβzɣi}{}\classe{n}
\begin{définition}\fra navet cuit à la vapeur\end{définition}
\begin{définition}\cmn 蒸了的芜菁根\end{définition}
\begin{relation-sémantique}\confer{
\hyperlink{Ⓔrɤjndoʁ}{\textit{ \papi{rɤjndoʁ}}}
}\end{relation-sémantique}\end{entrée}

\begin{entrée}
\vedette{\hypertarget{Ⓔlaftaʁ}{\papi{ laftaʁ}}}\markboth{laftaʁ}{}\classe{n}
\begin{définition}\fra caution\end{définition}
\begin{définition}\cmn 押金;纪念;信号\end{définition}\end{entrée}

\begin{entrée}
\vedette{\hypertarget{Ⓔlaftɕɤn}{\papi{ laftɕɤn}}}\markboth{laftɕɤn}{}\classe{n}
\begin{définition}\fra chance\end{définition}
\begin{définition}\cmn 运气\end{définition}
\begin{exemple}\jya ɯʑo laftɕɤn ci ŋu\cmn 他很幸运\end{exemple}\end{entrée}

\begin{entrée}
\vedette{\hypertarget{Ⓔlahaŋ}{\papi{ lahaŋ}}}\markboth{lahaŋ}{}\classe{n}
\begin{définition}\fra soufre\end{définition}
\begin{définition}\cmn 硫磺
\begin{déclaration} \étymologie{\papi{\stylefn{硫磺}}}\end{déclaration}\end{définition}
\end{entrée}

\begin{entrée}
\vedette{\hypertarget{Ⓔlajɯ}{\papi{ lajɯ}}}\markboth{lajɯ}{}\classe{n}
\begin{définition}\fra chant de montagne\end{définition}
\begin{définition}\cmn 山歌\end{définition}
\begin{définition}\cmn 他唱了山歌\end{définition}
\begin{exemple}\jya ɯʑo kɯ lajɯ pjɤ-βzu\end{exemple}
\begin{relation-sémantique}\confer{
\hyperlink{Ⓔrɯlajɯ}{\textit{ \papi{rɯlajɯ}}}
}\end{relation-sémantique}\end{entrée}

\begin{entrée}
\vedette{\hypertarget{Ⓔlaloŋ}{\papi{ laloŋ}}}\markboth{laloŋ}{}
\classe{adv}
\begin{définition}\fra partout\end{définition}
\begin{définition}\cmn 统统;到处\end{définition}
\begin{exemple}\jya sɤtɕha laloŋ ʑo ɕ-to-khɤt\cmn 他把所有地方都统统走了一遍\end{exemple}
\begin{relation-sémantique}\confer{
\hyperlink{Ⓔsaloŋ}{\textit{ \papi{saloŋ}}}
}\end{relation-sémantique}\end{entrée}

\begin{entrée}
\vedette{\hypertarget{Ⓔlankatsɯt}{\papi{ lankatsɯt}}}\markboth{lankatsɯt}{}\classe{n}
\begin{définition}\fra type d'habit en poil de yak\end{définition}
\begin{définition}\cmn 氆氇做成的一种衣服\end{définition}\end{entrée}

\begin{entrée}
\vedette{\hypertarget{Ⓔlaŋ}{\papi{ laŋ}}}\markboth{laŋ}{}\classe{vi}
\paradigme{\textit{dir :} \jya tɤ-}
\begin{définition}\fra se soulever\end{définition}
\begin{définition}\cmn 起义
\begin{déclaration} \étymologie{\papi{laŋ}}\end{déclaration}\end{définition}
\begin{exemple}\jya mkhɤrmaŋ to-laŋ-nɯ\cmn 老百姓起义了\end{exemple}
\end{entrée}

\begin{entrée}
\vedette{\hypertarget{Ⓔlaŋgɯ}{\papi{ laŋgɯ}}}\markboth{laŋgɯ}{}
\begin{définition}\ 
\begin{déclaration}\grammar{n.lieu}\end{déclaration}\end{définition}
\begin{définition}\fra l'un des hameaux de Gyutshapa\end{définition}
\begin{définition}\cmn 二茶村的大队之一\end{définition}\end{entrée}

\begin{entrée}
\vedette{\hypertarget{Ⓔlaŋlaŋ}{\papi{ laŋlaŋ}}}\markboth{laŋlaŋ}{}\classe{n}
\begin{définition}\fra une espèce de cerisier\end{définition}
\begin{définition}\cmn 野樱桃的一种\end{définition}
\begin{exemple}\jya laŋlaŋ nɯ si khro mɤ-mbro, zgoku kɯ-mbɤr tsa tu-ɬoʁ ŋu, ɯ-ru ɣɯrni, ɯ-jwaʁ mɤ-jndʐɤz, kɯ-ɤrtɯm tsa ŋu. χɕitka tɕe, ɯ-jwaʁ ɲɯ-lɤt ɕɯŋgɯ ɯ-mɯntoʁ ɲɯ-lɤt ŋu. ɯ-jwaʁ na-lɤt ɯ-rca tɕe, ɯ-mɯntoʁ pjɯ-ŋgra ŋu, tɕe ɯ-mat tɯ-βzu ɲɯ-ʑe ŋu. ɯ-mat thɯ-tɯt tɕe, ɣɯrni, artɯm rloʁrloʁ ʑo, tú-wɣ-ndza tɕe, kɯ-chi tu, kɯ-tɕur tu, ɯ-ŋgɯ ɯ-rɣi nɯ wxti tsa.\cmn 野樱桃长得不高,生长在半山上,树干是红色的,叶子不大,有点圆。春天,在长叶之前就开花。在长叶的同时,花就凋谢了,就开始结果。果实成熟后,是红色的,球形的。吃起来有的很甜,有的酸,里面的种子比较大。\end{exemple}
\end{entrée}

\begin{entrée}
\vedette{\hypertarget{Ⓔlargi}{\papi{ largi}}}\markboth{largi}{}\classe{n}
\begin{définition}\fra vieux moine\end{définition}
\begin{définition}\cmn 老和尚\end{définition}\end{entrée}

\begin{entrée}
\vedette{\hypertarget{Ⓔlaʁdɯn}{\papi{ laʁdɯn}}}\markboth{laʁdɯn}{}
\classe{n}
\begin{définition}\fra outil\end{définition}
\begin{définition}\cmn 工具
\begin{déclaration} \étymologie{\papi{lag.ldan}}\end{déclaration}\end{définition}\end{entrée}

\begin{entrée}
\vedette{\hypertarget{Ⓔlaʁjɤt}{\papi{ laʁjɤt}}}\markboth{laʁjɤt}{}\classe{n}
\begin{définition}\fra travail manuel\end{définition}
\begin{définition}\cmn 手工\end{définition}
\begin{relation-sémantique}\confer{
\hyperlink{Ⓔrɯlaʁjɤt}{\textit{ \papi{rɯlaʁjɤt}}}
}\end{relation-sémantique}\end{entrée}

\begin{entrée}
\vedette{\hypertarget{Ⓔlaʁjoʁ}{\papi{ laʁjoʁ}}}\markboth{laʁjoʁ}{}\classe{n}
\begin{définition}\fra assistant\end{définition}
\begin{définition}\cmn 帮手
\begin{déclaration} \étymologie{\papi{lag.gjog}}\end{déclaration}\end{définition}
\begin{relation-sémantique}\confer{
\hyperlink{Ⓔnɯlaʁjoʁ}{\textit{ \papi{nɯlaʁjoʁ}}}
}\end{relation-sémantique}\end{entrée}

\begin{entrée}
\vedette{\hypertarget{Ⓔlaʁjɯɣ}{\papi{ laʁjɯɣ}}}\markboth{laʁjɯɣ}{}\classe{n}
\begin{définition}\fra bâton\end{définition}
\begin{définition}\cmn 棍子
\begin{déclaration} \étymologie{\papi{lag.dbʲug}}\end{déclaration}\end{définition}\end{entrée}

\begin{entrée}
\vedette{\hypertarget{Ⓔlaʁma}{\papi{ laʁma}}}\markboth{laʁma}{}\classe{cnj}
\begin{définition}\fra bien que, à part\end{définition}
\begin{définition}\cmn 虽然;除此以外\end{définition}
\end{entrée}

\begin{entrée}
\vedette{\hypertarget{Ⓔlaʁnɤ}{\papi{ laʁnɤ}}}\markboth{laʁnɤ}{}\classe{conj}
\begin{définition}\fra au moins\end{définition}
\begin{définition}\cmn ……倒\end{définition}
\begin{exemple}\jya aʑo laʁnɤ tɤ-nɯmgo-a ma nɤʑo nɤ-ndzɤtshi pɯ-me tɕe tɯ-mtsɯr\cmn 我倒是吃过饭了,你没有吃饭就肯定饿了\end{exemple}
\end{entrée}

\begin{entrée}
\vedette{\hypertarget{Ⓔlaʁnɯχsɯm}{\papi{ laʁnɯχsɯm}}}\markboth{laʁnɯχsɯm}{}\classe{num}
\begin{définition}\fra deux ou trois\end{définition}
\begin{définition}\cmn 两三个\end{définition}\end{entrée}

\begin{entrée}
\vedette{\hypertarget{Ⓔlaʁnɯz}{\papi{ laʁnɯz}}}\markboth{laʁnɯz}{}\classe{num}
\begin{définition}\fra un ou deux, quelques\end{définition}
\begin{définition}\cmn 一两个\end{définition}
\begin{exemple}\jya laʁnɯ-sŋi\cmn 一两天、几天\end{exemple}\end{entrée}

\begin{entrée}
\vedette{\hypertarget{Ⓔlaʁŋkhɤr}{\papi{ laʁŋkhɤr}}}\markboth{laʁŋkhɤr}{}\classe{n}
\begin{définition}\fra moulin à prière\end{définition}
\begin{définition}\cmn 手摇转经筒
\begin{déclaration} \étymologie{\papi{lag.ⁿkʰor}}\end{déclaration}\end{définition}\end{entrée}

\begin{entrée}
\vedette{\hypertarget{Ⓔlaʁrda}{\papi{ laʁrda}}}\markboth{laʁrda}{}\classe{n}
\begin{définition}\fra geste\end{définition}
\begin{définition}\cmn 手势
\begin{déclaration} \étymologie{\papi{lag.brda}}\end{déclaration}\end{définition}\end{entrée}

\begin{entrée}
\vedette{\hypertarget{Ⓔlaʁrdɤβ}{\papi{ laʁrdɤβ}}}\markboth{laʁrdɤβ}{}
\classe{n}
\begin{définition}\fra coup de patte avant\end{définition}
\begin{définition}\cmn 用前腿打\end{définition}
\begin{exemple}\jya ɯʑo jla ci tu tɕe laʁrdɤβ kɤ-lɤt rga\cmn 我们有一头犏牛,特别喜欢用前腿踢人\end{exemple}\end{entrée}

\begin{entrée}
\vedette{\hypertarget{Ⓔlaʁrŋa}{\papi{ laʁrŋa}}}\markboth{laʁrŋa}{}\classe{n}
\begin{définition}\fra tambour à main\end{définition}
\begin{définition}\cmn 长柄鼓
\begin{déclaration} \étymologie{\papi{lag.rŋa}}\end{déclaration}\end{définition}
\end{entrée}

\begin{entrée}
\vedette{\hypertarget{Ⓔlaʁsta}{\papi{ laʁsta}}}\markboth{laʁsta}{}\classe{n}
\begin{définition}\fra marteau\end{définition}
\begin{définition}\cmn 锤子\end{définition}\end{entrée}

\begin{entrée}
\vedette{\hypertarget{Ⓔlaʁsɯɣma}{\papi{ laʁsɯɣma}}}\markboth{laʁsɯɣma}{}\classe{cnj}
\begin{définition}\fra à part ça\end{définition}
\begin{définition}\cmn 除此以外\end{définition}
\begin{exemple}\jya tɯ-mɯ ci kɤ-lɤt laʁsɯɣma, jisŋi pɯ-sɤscit\cmn 今天很开心,只不过下了一点雨\end{exemple}
\end{entrée}

\begin{entrée}
\vedette{\hypertarget{Ⓔlaʁsɯɣnɤma}{\papi{ laʁsɯɣnɤma}}}\markboth{laʁsɯɣnɤma}{}\classe{cnj}
\begin{définition}\fra à part ça\end{définition}
\begin{définition}\cmn 除此以外;只是\end{définition}
\begin{exemple}\jya nɯ kɤ-kho nɯ ɲɤ-nɯmɟa laʁsɯɣnɤma, nɯ ma mɯ-to-nɯβdaʁ\cmn 他只是把送的东西接过去了,其它理也没有理。\end{exemple}
\end{entrée}

\begin{entrée}
\vedette{\hypertarget{Ⓔlaʁzu}{\papi{ laʁzu}}}\markboth{laʁzu}{}\classe{n}
\begin{définition}\fra type d'offrande au dieux, un des éléments utilisé pour faire les gtorma\end{définition}
\begin{définition}\cmn 供奉鬼神的一种物品
\begin{déclaration} \étymologie{\papi{lag.bzo}}\end{déclaration}\end{définition}
\end{entrée}

\begin{entrée}
\vedette{\hypertarget{Ⓔlawa}{\papi{ lawa}}}\markboth{lawa}{}\classe{n}
\begin{définition}\fra laine\end{définition}
\begin{définition}\cmn 羊毛\end{définition}\end{entrée}

\begin{entrée}
\vedette{\hypertarget{Ⓔlaχɕi}{\papi{ laχɕi}}}\markboth{laχɕi}{}\classe{n}
\begin{définition}\fra métier (manuel)\end{définition}
\begin{définition}\cmn 手艺\end{définition}
\begin{exemple}\jya ɯ-laχɕi tu\cmn 他有手艺\end{exemple}\end{entrée}

\begin{entrée}
\vedette{\hypertarget{Ⓔlaχsɯm}{\papi{ laχsɯm}}}\markboth{laχsɯm}{}\classe{num}
\begin{définition}\fra deux ou trois\end{définition}
\begin{définition}\cmn 两三个\end{définition}
\begin{exemple}\jya laχsɯ-sŋi\cmn 两三天\end{exemple}\end{entrée}

\begin{entrée}
\vedette{\hypertarget{Ⓔlaχtɕha}{\papi{ laχtɕha}}}\markboth{laχtɕha}{}\classe{n}
\begin{définition}\fra objet\end{définition}
\begin{définition}\cmn 东西
\begin{déclaration} \étymologie{\papi{lag.tɕʰa}}\end{déclaration}\end{définition}\end{entrée}

\begin{entrée}
\vedette{\hypertarget{Ⓔlaχthɤβ}{\papi{ laχthɤβ}}}\markboth{laχthɤβ}{}\classe{n}
\begin{définition}\fra médecin traditionnel qui répare les fractures\end{définition}
\begin{définition}\cmn 专门治疗骨折、脱臼的土医生\end{définition}\end{entrée}

\begin{entrée}
\vedette{\hypertarget{Ⓔlaχtsɯ}{\papi{ laχtsɯ}}}\markboth{laχtsɯ}{}\classe{n}
\begin{définition}\fra poutre du balcon\end{définition}
\begin{définition}\cmn 走缘的柱子
\end{définition}\end{entrée}

\begin{entrée}
\vedette{\hypertarget{Ⓔlaʑu}{\papi{ laʑu}}}\markboth{laʑu}{}\classe{n}
\begin{définition}\fra viande fumée\end{définition}
\begin{définition}\cmn 腊肉
\begin{déclaration} \étymologie{\papi{\stylefn{腊肉}}}\end{déclaration}\end{définition}
\end{entrée}

\begin{entrée}
\vedette{\hypertarget{Ⓔlɤchu}{\papi{ lɤchu}}}\markboth{lɤchu}{}\classe{adv}
\begin{définition}\fra en amont\end{définition}
\begin{définition}\cmn 在上游\end{définition}
\begin{relation-sémantique}\confer{
\hyperlink{Ⓔɯ-locu}{\textit{ \papi{ɯ-locu}}}
}\end{relation-sémantique}\end{entrée}

\begin{entrée}
\vedette{\hypertarget{Ⓔlɤftsɤz}{\papi{ lɤftsɤz}}}\markboth{lɤftsɤz}{} (\variante{lɤftɕɤz}) 
\classe{n}
\begin{définition}\fra endroit sur le toit où l'on plante un rlung-rta et où l'on élève un tas de silex\end{définition}
\begin{définition}\cmn 屋顶上插有经幡,用白燧石堆积而成的石堆(嘉绒式敖包)
\begin{déclaration} \étymologie{\papi{la.btsas}}\end{déclaration}\end{définition}
\begin{exemple}\jya lɤftsɤz nɯ khɤxtɤmbro ɣɯ ɯ-qhu znde tu-kɯ-ɣi ɣɯ akɯ andi ɯ-χcɤl li znde kɯ-xtɕi ci kɯ-ɤβʑɯrdu, kha ɯ-znde sɤznɤ tɤ-ʁar jamar tu-ro ra, znde cho ɯ-tɯ-jaʁ kɯ-naχtɕɯɣ thɯ-kɤ-βzu ci ŋu, ɯ-χcɤl kɯ-spoʁ ŋu, ɯ-kɤχcɤl zɯ qapi tú-wɣ-rmbɯ ŋu, ɯ-χcɤl nɯ tɕu tshɤχɕaŋ, rloŋrta, nɯ ra pjɯ́-wɣ-sɤtsa ŋu, kɯrɯ kha ɯ-lɤftsɤz kɯ-me me.\cmn 
\stylefv{lɤftsɤz}是房背墙顶上左右两边的中间再修一堵和外墙一样厚的四方形小墙,中间有小洞,顶上要堆上白石头,在中间插上\stylefv{tshɤχɕaŋ}、经幡等。所有藏房都有\stylefv{lɤftsɤz}。
\end{exemple}\end{entrée}

\begin{entrée}
\vedette{\hypertarget{Ⓔlɤftɯɣ}{\papi{ lɤftɯɣ}}}\markboth{lɤftɯɣ}{}
\classe{n}
\begin{définition}\fra humus\end{définition}
\begin{définition}\cmn 腐殖土\end{définition}\end{entrée}

\begin{entrée}
\vedette{\hypertarget{Ⓔlɤɣ}{\papi{ lɤɣ}}}\markboth{lɤɣ}{}\classe{vl}
\paradigme{\textit{dir :} \jya nɯ-}
\begin{définition}\fra faire paître les animaux\end{définition}
\begin{définition}\cmn 放牧\end{définition}
\begin{exemple}\jya a-mu kɯ ji-fsapaʁ ra na-lɤɣ\cmn 我母亲把牲畜带去放了\end{exemple}
\begin{exemple}\jya fsapaʁ ra ɕ-pɯ-laɣ-a\cmn 我把牲畜带去放了\end{exemple}\end{entrée}

\begin{entrée}
\vedette{\hypertarget{Ⓔlɤjmu}{\papi{ lɤjmu}}}\markboth{lɤjmu}{}\classe{n}
\begin{définition}\fra servante\end{définition}
\begin{définition}\cmn 女仆人\end{définition}
\end{entrée}

\begin{entrée}
\vedette{\hypertarget{Ⓔlɤn}{\papi{ lɤn}}}\markboth{lɤn}{}
\classe{vi}
\paradigme{\textit{dir :} \jya pɯ-}
\begin{définition}\fra avoir la responsabilité de, c'est la faute de...\end{définition}
\begin{définition}\cmn 要为某事负责任;这件事怪……\end{définition}
\begin{exemple}\jya nɯ ma-tɤ-tɯ-ste nɯ-sɯso-t-a ri mɯ́j-tɯ-khɯ tɕe nɤʑo tɯ-lɤn\cmn 我想了“你不要那样做”,你没有听,都怪你了\end{exemple}
\begin{exemple}\jya tɤ-rɟit mɤ-kɯ-pe nɯ phama lɤn\cmn 孩子不成材怪父母\end{exemple}
\begin{exemple}\jya laʁtɕha ɲo-me tɕe, nɤʑo pɯ-tɯ-lɤn\cmn 东西丢了,你有责任\end{exemple}
\begin{exemple}\jya tɤ-pɤtso mɤ-kɯ-khɯ nɯ chɯ-kɤ-sɯɣli ɲɯ-lɤn\cmn 小孩子不听话,就是因为平时惯了他\end{exemple}\begin{sous-entrée}
\vedette{\hypertarget{}{\papi{ nɯlɤn}}}\markboth{nɯlɤn}{}\classe{vi}
\paradigme{\textit{dir :} \jya pɯ-}
\begin{définition}\ 
\begin{déclaration}\grammar{autoben}\end{déclaration}\end{définition}
\begin{définition}\fra n'avoir à s'en prendre qu'à soi-même\end{définition}
\begin{définition}\cmn 自食其果\end{définition}
\begin{exemple}\jya ɯʑo pjɤ-nɯlɤn ɕti\cmn 他自食其果\end{exemple}
\begin{exemple}\jya nɤʑo mɯ́j-tɯ-nɯlɤn\cmn 不是你的错\end{exemple}
\end{sous-entrée}\end{entrée}

\begin{entrée}
\vedette{\hypertarget{Ⓔlɤntsa}{\papi{ lɤntsa}}}\markboth{lɤntsa}{}\classe{n}
\begin{définition}\fra un motif bouddhique\end{définition}
\begin{définition}\cmn 佛教的图纹\end{définition}
\end{entrée}

\begin{entrée}
\vedette{\hypertarget{Ⓔlɤŋu}{\papi{ lɤŋu}}}\markboth{lɤŋu}{}\classe{num}
\begin{définition}\fra à peu près cinq\end{définition}
\begin{définition}\cmn 大概五个\end{définition}
\begin{relation-sémantique}\confer{
\hyperlink{Ⓔkɯmŋu}{\textit{ \papi{kɯmŋu}}}
}\end{relation-sémantique}
\end{entrée}

\begin{entrée}
\vedette{\hypertarget{Ⓔlɤŋɤtʂɤɣ}{\papi{ lɤŋɤtʂɤɣ}}}\markboth{lɤŋɤtʂɤɣ}{}\classe{num}
\begin{définition}\fra cinq ou six\end{définition}
\begin{définition}\cmn 五六个\end{définition}
\begin{exemple}\jya lɤŋɤtʂɤ-sŋi\cmn 五六天\end{exemple}
\begin{relation-sémantique}\confer{
\hyperlink{Ⓔkɯmŋu}{\textit{ \papi{kɯmŋu}}}
}\end{relation-sémantique}
\begin{relation-sémantique}\confer{
\hyperlink{Ⓔkɯtʂɤɣ}{\textit{ \papi{kɯtʂɤɣ}}}
}\end{relation-sémantique}\end{entrée}

\begin{entrée}
\vedette{\hypertarget{Ⓔlɤpɯɣ}{\papi{ lɤpɯɣ}}}\markboth{lɤpɯɣ}{}\classe{n}
\begin{définition}\fra radis\end{définition}
\begin{définition}\cmn 萝卜
\begin{déclaration} \étymologie{\papi{la.pʰug}}\end{déclaration}\end{définition}\end{entrée}

\begin{entrée}
\vedette{\hypertarget{Ⓔlɤsɤr}{\papi{ lɤsɤr}}}\markboth{lɤsɤr}{}
\classe{n}
\begin{définition}\fra nouvel an\end{définition}
\begin{définition}\cmn 新年
\begin{déclaration} \étymologie{\papi{lo.gsar}}\end{déclaration}\end{définition}
\begin{exemple}\jya lɤsɤr a-pɯ-tɯ-scit-nɯ\cmn 新年快樂\end{exemple}
\begin{exemple}\jya @eryue @sanhao tɕe ji-lɤsɤr ɲɯ-ŋu, kɯrɯlɤsɤr\cmn 我们的新年是二月三号\end{exemple}\end{entrée}

\begin{entrée}
\vedette{\hypertarget{Ⓔlɤsɤr cito}{\papi{ lɤsɤr cito}}}\markboth{lɤsɤr cito}{}\classe{n}
\begin{définition}\fra premier jour de l'année\end{définition}
\begin{définition}\cmn 年初一\end{définition}\end{entrée}

\begin{entrée}
\vedette{\hypertarget{Ⓔlɤskɤr tɕhɯʁɲiz}{\papi{ lɤskɤr tɕhɯʁɲiz}}}\markboth{lɤskɤr tɕhɯʁɲiz}{}\classe{n}
\begin{définition}\fra les douze signes astrologiques\end{définition}
\begin{définition}\cmn 十二生肖\end{définition}\end{entrée}

\begin{entrée}
\vedette{\hypertarget{ⒺlɤtⒽ1}{\papi{ lɤt}}}\markboth{lɤt}{}\homonyme{1}
\classe{vt}\acception{1}
\paradigme{\textit{dir :} \jya \_}
\begin{définition}\fra jeter\end{définition}
\begin{définition}\cmn 扔\end{définition}
\begin{exemple}\jya ɯ-mci to-lɤt\cmn 他吐了口水\end{exemple}\acception{2}
\paradigme{\textit{dir :} \jya tɤ-}
\begin{définition}\fra tirer\end{définition}
\begin{définition}\cmn 射\end{définition}
\begin{exemple}\jya ɕɤmɯɣdɯ to-lɤt\cmn 他开枪了\end{exemple}
\begin{exemple}\jya tɯdi to-lɤt\cmn 他射了箭\end{exemple}\acception{3}
\paradigme{\textit{dir :} \jya tɤ-}
\begin{définition}\fra frapper\end{définition}
\begin{définition}\cmn 打\end{définition}
\begin{exemple}\jya ɯ-ku zɯ tɤŋkhɯt to-lɤt\cmn 在他头上打了一拳\end{exemple}
\begin{exemple}\jya tɯ-mɯrtsɯɣ ci to-lɤt\cmn 捏了一下\end{exemple}
\begin{exemple}\jya tɯ-qartsɯ ta-lɤt\cmn (马)踢了一脚\end{exemple}\acception{4}
\paradigme{\textit{dir :} \jya pɯ-}
\paradigme{\textit{dir :} \jya kɤ-}
\begin{définition}\fra mettre, ajouter\end{définition}
\begin{définition}\cmn 放(进),加;倒(茶)\end{définition}
\begin{exemple}\jya tsha pjɤ-lɤt, ko-lɤt\cmn 放了盐\end{exemple}
\begin{exemple}\jya tɯ-ci pjɤ-lɤt\cmn 洒了水\end{exemple}
\begin{exemple}\jya tɯ-ci ko-lɤt\cmn (在菜汤里)加了水\end{exemple}\acception{5}
\paradigme{\textit{dir :} \jya tɤ-}
\begin{définition}\fra appliquer\end{définition}
\begin{définition}\cmn 涂\end{définition}
\begin{exemple}\jya tɤ-mthɯm ɯ-taʁ tsha to-lɤt (=to-mar)\cmn 把盐擦在肉上(腌制腊肉的方法)\end{exemple}
\begin{relation-sémantique}\synonyme{
\hyperlink{Ⓔmar}{\textit{ \papi{mar}}}
}\end{relation-sémantique}\acception{6}
\begin{définition}\fra utiliser\end{définition}
\begin{définition}\cmn 用(工具)\end{définition}
\begin{exemple}\jya mkhɯrlu jo-lɤt\cmn 他开了车\end{exemple}
\begin{exemple}\jya mkhɯrlu ɯ-kɯ-lɤt\cmn 驾驶员\end{exemple}
\begin{exemple}\jya tɤ-mtsɯ to-lɤt\cmn 他扣了扣子\end{exemple}
\begin{exemple}\jya tɤ-mtɯ ko-lɤt, cho-lɤt\cmn 他打了(个)结\end{exemple}
\begin{exemple}\jya mbro ɯ-jme nɯ tɤ-mtɯ χsɯm tha-lɤt\cmn 在马尾巴上打了三个结\end{exemple}
\begin{exemple}\jya taqaβ to-lɤt\cmn 他用针扎了\end{exemple}
\begin{exemple}\jya a-@dianhua ɯ-kɯ-lɤt ci ɣɤʑu\cmn 有人打电话给我\end{exemple}
\begin{exemple}\jya nɤ-@dianhua ɲɯ-lat-a\cmn 我会给你打电话\end{exemple}
\begin{exemple}\jya pɤjkhu ja-lɤt me\cmn 他还没有打(电话)\end{exemple}
\begin{exemple}\jya qraʁ thɯ-lat-a\cmn 我铸造了犁铧\end{exemple}
\begin{exemple}\jya sɤcɯ pjɤ-lɤt-ndʑi\cmn 他们俩锁了门\end{exemple}
\begin{exemple}\jya tɕhaʁla zɯ tɯmbri ci kɤ-lat-a (kɤ-mtshi-t-a, kɤ-rɤɕi-t-a)\cmn 我在院子里拉了一根绳子(晒衣服)\end{exemple}\acception{7}
\paradigme{\textit{dir :} \jya kɤ-}
\begin{définition}\fra tomber (pluie, neige etc)\end{définition}
\begin{définition}\cmn 下(雨、雪等)\end{définition}
\begin{exemple}\jya tɯ-mɯ ko-lɤt\cmn 下雨了\end{exemple}
\begin{exemple}\jya tɯ-mɯ ɲɯ-ɤsɯ-lɤt\cmn 正在下雨\end{exemple}
\begin{exemple}\jya tɤjpa ko-lɤt\cmn 下雪了\end{exemple}
\begin{exemple}\jya sɤrwa cho-lɤt, tha-lɤt\cmn 下了冰雹\end{exemple}\acception{8}
\paradigme{\textit{dir :} \jya thɯ-}
\begin{définition}\fra mettre bas\end{définition}
\begin{définition}\cmn 生崽;开(花)\end{définition}
\begin{exemple}\jya paχtsa chɤ-lɤt\cmn 下猪崽\end{exemple}
\begin{exemple}\jya mɯntoʁ ɲɤ-lɤt (=ɲɤ-rɯmɯntoʁ)\cmn 开花\end{exemple}
\begin{relation-sémantique}\synonyme{
\hyperlink{Ⓔrɯmɯntoʁ}{\textit{ \papi{rɯmɯntoʁ}}}
}\end{relation-sémantique}\acception{9}
\paradigme{\textit{dir :} \jya nɯ-}
\begin{définition}\fra relâcher\end{définition}
\begin{définition}\cmn 释放\end{définition}
\begin{exemple}\jya kɯ-nɯkhrɯm nɯ ra ɲo-lɤt\cmn 他把囚犯释放了\end{exemple}
\begin{exemple}\jya a-@fangjia lɤt-nɯ\cmn 他们要让我放假\end{exemple}
\begin{exemple}\jya ɕɯntɕhi ʁnɯ-sŋi ɲɤ-lɤt-nɯ\cmn 他们放了两天假\end{exemple}\acception{10}
\begin{définition}\fra ramener\end{définition}
\begin{définition}\cmn 送回\end{définition}
\begin{exemple}\jya kha mɤɕtʂa ɣɯ-jɤ́-wɣ-lat-a\cmn 他把我送回家了\end{exemple}
\begin{relation-sémantique}\confer{
\hyperlink{Ⓔnɤscɤlɤt}{\textit{ \papi{nɤscɤlɤt}}}
}\end{relation-sémantique}\acception{11}
\begin{définition}\fra verbe léger\end{définition}
\begin{définition}\cmn 助动词\end{définition}
\begin{exemple}\jya aʑo rɤɣo ci pɯ-lat-a\cmn 我弹奏了音乐\end{exemple}
\begin{relation-sémantique}\synonyme{
\hyperlink{Ⓔɣɤjɯ}{\textit{ \papi{ɣɤjɯ}}}
}\end{relation-sémantique}
\begin{relation-sémantique}\confer{
\hyperlink{Ⓔchɤlɤnnɤ}{\textit{ \papi{chɤlɤnnɤ}}}
}\end{relation-sémantique}
\begin{relation-sémantique}\confer{
\hyperlink{Ⓔɣɤlɤt}{\textit{ \papi{ɣɤlɤt}}}
}\end{relation-sémantique}
\begin{relation-sémantique}\confer{
\hyperlink{Ⓔrɤlɤt}{\textit{ \papi{rɤlɤt}}}
}\end{relation-sémantique}
\begin{relation-sémantique}\confer{
\hyperlink{Ⓔjaqhɤrŋgɤβ,lɤt}{\textit{ \papi{jaqhɤrŋgɤβ,lɤt}}}
}\end{relation-sémantique}
\begin{relation-sémantique}\confer{
\hyperlink{Ⓔtaʁmbra,lɤt}{\textit{ \papi{taʁmbra,lɤt}}}
}\end{relation-sémantique}
\begin{relation-sémantique}\confer{
\hyperlink{Ⓔtɤlɟɣo,lɤt}{\textit{ \papi{tɤlɟɣo,lɤt}}}
}\end{relation-sémantique}
\begin{relation-sémantique}\confer{
\hyperlink{Ⓔtɤqɤt,lɤt}{\textit{ \papi{tɤqɤt,lɤt}}}
}\end{relation-sémantique}
\begin{relation-sémantique}\confer{
 \papi{tɯ-sɯso,lɤt}
}\end{relation-sémantique}
\begin{relation-sémantique}\confer{
 \papi{ɯ-rpu,lɤt}
}\end{relation-sémantique}
\begin{relation-sémantique}\confer{
 \papi{ɯ-tɕhɯ,lɤt}
}\end{relation-sémantique}\begin{sous-entrée}
\vedette{\hypertarget{}{\papi{ alɤt}}}\markboth{alɤt}{}\classe{vi}
\begin{définition}\ 
\begin{déclaration}\grammar{pass}\end{déclaration}\end{définition}
\begin{exemple}\jya kɯm sɤcɯ mɤ-alɤt\cmn 门没有锁上\end{exemple}
\begin{relation-sémantique}\confer{
\hyperlink{Ⓔsɤlɤt}{\textit{ \papi{sɤlɤt}}}
}\end{relation-sémantique}
\begin{relation-sémantique}\confer{
\hyperlink{Ⓔakɤlɤt}{\textit{ \papi{akɤlɤt}}}
}\end{relation-sémantique}
\end{sous-entrée}\begin{sous-entrée}
\vedette{\hypertarget{}{\papi{ sɯlɤt}}}\markboth{sɯlɤt}{}\classe{vt}
\begin{définition}\ 
\begin{déclaration}\grammar{caus}\end{déclaration}\end{définition}
\begin{exemple}\jya cha tɯ-@beibei to-sɯlɤt (=to-sɯrku)\cmn 他请(服务员)给自己倒一杯酒\end{exemple}
\end{sous-entrée}\end{entrée}

\begin{entrée}
\vedette{\hypertarget{Ⓔlɤtaŋ}{\papi{ lɤtaŋ}}}\markboth{lɤtaŋ}{}\classe{n}
\begin{définition}\fra conscience\end{définition}
\begin{définition}\cmn 良心\end{définition}
\begin{exemple}\jya nɤ-lɤtaŋ ɯ-tɯ-me nɯ!\cmn 你真没有良心!\end{exemple}\end{entrée}

\begin{entrée}
\vedette{\hypertarget{Ⓔlɤtɕhom}{\papi{ lɤtɕhom}}}\markboth{lɤtɕhom}{}
\classe{n}
\begin{définition}\fra baratte\end{définition}
\begin{définition}\cmn 打酥油茶的木桶\end{définition}
\begin{relation-sémantique}\synonyme{
\hyperlink{Ⓔtɤlɤndʑu}{\textit{ \papi{tɤlɤndʑu}}}
}\end{relation-sémantique}\end{entrée}

\begin{entrée}
\vedette{\hypertarget{Ⓔlɤzŋɤn}{\papi{ lɤzŋɤn}}}\markboth{lɤzŋɤn}{}\classe{n}
\begin{définition}\fra mauvaise chance\end{définition}
\begin{définition}\cmn 霉运;晦气
\begin{déclaration} \étymologie{\papi{las.ŋan}}\end{déclaration}\end{définition}\end{entrée}

\begin{entrée}
\vedette{\hypertarget{Ⓔlɤzŋɤntɕɤn}{\papi{ lɤzŋɤntɕɤn}}}\markboth{lɤzŋɤntɕɤn}{}\classe{n}
\begin{définition}\fra malchanceux\end{définition}
\begin{définition}\cmn 最倒霉的人
\begin{déclaration} \étymologie{\papi{las.ŋan.tɕan}}\end{déclaration}\end{définition}\end{entrée}

\begin{entrée}
\vedette{\hypertarget{Ⓔlbjɯlbjɯɣ}{\papi{ lbjɯlbjɯɣ}}}\markboth{lbjɯlbjɯɣ}{}\classe{idph.2}
\begin{définition}\fra qui pend en grand nombre, mou\end{définition}
\begin{définition}\cmn 形容又多又柔软的样子,往下垂吊
\end{définition}
\begin{exemple}\jya ʑmbri ɣɯ ɯ-rtaʁ ɲɯ-mpɯ tɕe ɯ-jwaʁ ɲɯ-dɤn pjɯ-ɴqoʁ kɯ-fse tɕe, lbjɯlbjɯɣ ʑo ɲɯ-pa\cmn 柳树的枝桠很软,叶子很多,往下垂的样子\end{exemple}
\begin{relation-sémantique}\synonyme{
\hyperlink{Ⓔbjɯbjɯɣ}{\textit{ \papi{bjɯbjɯɣ}}}
}\end{relation-sémantique}\end{entrée}

\begin{entrée}
\vedette{\hypertarget{Ⓔlchɤlchɤt}{\papi{ lchɤlchɤt}}}\markboth{lchɤlchɤt}{}\classe{idph.2}\acception{1}
\begin{définition}\fra petit (homme)\end{définition}
\begin{définition}\cmn 形容矮墩墩的样子(人)\end{définition}
\begin{exemple}\jya tɯrme ɯ-phoŋbu kɯ-mbɤr ci lchɤlchɤt ɲɯ-ŋu\cmn 那个人矮墩墩的\end{exemple}
\begin{exemple}\jya ɯ-phoŋbu mɤ-kɯ-mbro lchɤlchɤt ci ɲɯ-ŋu\cmn 他矮墩墩的\end{exemple}\acception{2}
\begin{définition}\fra non rempli\end{définition}
\begin{définition}\cmn 形容没有满的样子(口袋)\end{définition}
\begin{exemple}\jya tɤ-fkɯm lchɤlchɤt ci ɲɯ-ŋu\cmn 口袋不满\end{exemple}
\begin{exemple}\jya lchɤlchɤt ci ɲɤ-rku\cmn 没有装满\end{exemple}
\begin{relation-sémantique}\confer{
\hyperlink{Ⓔlchɯɣlchɯɣ}{\textit{ \papi{lchɯɣlchɯɣ}}}
}\end{relation-sémantique}\end{entrée}

\begin{entrée}
\vedette{\hypertarget{Ⓔlchɣaʁlchɣaʁ}{\papi{ lchɣaʁlchɣaʁ}}}\markboth{lchɣaʁlchɣaʁ}{}
\classe{idph.2}
\begin{définition}\fra souple, agréable à porter\end{définition}
\begin{définition}\cmn 形容柔软(衣服、皮子)的触感\end{définition}
\begin{exemple}\jya tɯ-ndʐi lchɣaʁlchɣaʁ ci ɲɯ-ŋu\cmn 是很柔软的皮子\end{exemple}
\begin{exemple}\jya ɯ-ŋga lchɣaʁlchɣaʁ ci ɲɯ-ŋu\cmn 他的衣服很柔软\end{exemple}
\begin{exemple}\jya lchɣaʁlchɣaʁ ɲɯ-nɯɣɯŋga\cmn 很柔软,很好穿\end{exemple}\end{entrée}

\begin{entrée}
\vedette{\hypertarget{Ⓔlchɯɣlchɯɣ}{\papi{ lchɯɣlchɯɣ}}}\markboth{lchɯɣlchɯɣ}{}\classe{idph.2}
\begin{définition}\fra pas rempli\end{définition}
\begin{définition}\cmn 没有装满\end{définition}
\begin{exemple}\jya tɤ-fkɯm ɯ-ŋgɯ zɯ stoʁ tɯɣnɤsqɯ-tɯrpa ci ma lchɯɣlchɯɣ maŋe\cmn 口袋里只有二十斤胡豆,没有装满\end{exemple}\end{entrée}

\begin{entrée}
\vedette{\hypertarget{Ⓔlchɯmlchɯm}{\papi{ lchɯmlchɯm}}}\markboth{lchɯmlchɯm}{}\classe{idph.2}
\begin{définition}\fra le niveau de l'eau qui baisse lentement\end{définition}
\begin{définition}\cmn 形容水位慢慢地降下去的样子\end{définition}
\begin{exemple}\jya mtshu lchɯmlchɯm ʑo pjɤ-schɤt\cmn 湖面的水慢慢地降下去\end{exemple}\end{entrée}

\begin{entrée}
\vedette{\hypertarget{Ⓔlcɯɣlcɯɣ}{\papi{ lcɯɣlcɯɣ}}}\markboth{lcɯɣlcɯɣ}{}\classe{idph.2}
\begin{définition}\fra trempé\end{définition}
\begin{définition}\cmn 湿透\end{définition}
\begin{exemple}\jya kó-wɣ-sphjaʁ lcɯɣlcɯɣ\cmn 湿透了\end{exemple}
\begin{exemple}\jya a-ŋga ra nɯ-aci lcɯɣlcɯɣ ʑo\cmn 我的衣服变湿了\end{exemple}
\begin{exemple}\jya a-ŋga ra lcɯɣlcɯɣ ʑo nɯ-pa\cmn 我的衣服变湿了\end{exemple}\end{entrée}

\begin{entrée}
\vedette{\hypertarget{Ⓔldʐaŋldʐaŋ}{\papi{ ldʐaŋldʐaŋ}}}\markboth{ldʐaŋldʐaŋ}{}\classe{idph.2}
\begin{définition}\fra pendu\end{définition}
\begin{définition}\cmn 吊着(很大的东西)\end{définition}
\begin{exemple}\jya fsapaʁ to-ʑɣɤtshi tɕe, ldʐaŋldʐaŋ ɲɯ-ɴqoʁ\cmn 牲畜(不小心)把自己勒死了,在那里吊着\end{exemple}
\begin{relation-sémantique}\confer{
\hyperlink{Ⓔɕtʂaŋɕtʂaŋ}{\textit{ \papi{ɕtʂaŋɕtʂaŋ}}}
}\end{relation-sémantique}\begin{sous-entrée}
\vedette{\hypertarget{}{\papi{ ɣɤldʐaŋlaŋ}}}\markboth{ɣɤldʐaŋlaŋ}{}\classe{vi}
\begin{exemple}\jya ɲɯ-ɣɤldʐaŋlaŋ\cmn 在甩来甩去\end{exemple}
\end{sous-entrée}\begin{sous-entrée}
\vedette{\hypertarget{}{\papi{ ldʐaŋnɤlaŋ}}}\markboth{ldʐaŋnɤlaŋ}{}\classe{idph.4}
\begin{exemple}\jya ldʐaŋnɤlaŋ ɲɯ-ʑɣɤstu\cmn 在甩来甩去\end{exemple}
\end{sous-entrée}\end{entrée}

\begin{entrée}
\vedette{\hypertarget{Ⓔldʐɤβldʐɤβ}{\papi{ ldʐɤβldʐɤβ}}}\markboth{ldʐɤβldʐɤβ}{}\classe{idph.2}
\begin{définition}\fra en désordre (fils pendus)\end{définition}
\begin{définition}\cmn 形容挂着的线、布条等凌乱的样子\end{définition}
\begin{exemple}\jya ɯ-ŋga chɤ-ɴɢraʁ ldʐɤβldʐɤβ ʑo ri, ɲɯ-ɤ<nɯ>sɯ-ŋga\cmn 他的衣服破破烂烂的,他还是穿着\end{exemple}\end{entrée}

\begin{entrée}
\vedette{\hypertarget{Ⓔldʐɤpɤldʐɤle}{\papi{ ldʐɤpɤldʐɤle}}}\markboth{ldʐɤpɤldʐɤle}{}\classe{n}
\begin{définition}\fra habits de mauvaise qualité, abîmés\end{définition}
\begin{définition}\cmn 简陋的衣服\end{définition}
\begin{exemple}\jya tɯ-ŋga ldʐɤpɤldʐɤle ʑo tu-nɯ-ŋge ɕti\cmn 他不讲究穿着\end{exemple}
\begin{relation-sémantique}\confer{
\hyperlink{Ⓔcɤpɤcrɤle}{\textit{ \papi{cɤpɤcrɤle}}}
}\end{relation-sémantique}\end{entrée}

\begin{entrée}
\vedette{\hypertarget{Ⓔldɯɣi}{\papi{ ldɯɣi}}}\markboth{ldɯɣi}{}
\classe{n}
\begin{définition}\fra bharal (ovis ammon)\end{définition}
\begin{définition}\cmn 盘羊\end{définition}
\begin{exemple}\jya ldɯɣi nɯ zgoku stu kɯ-mbro rɯtɕhɤβ ɯ-ŋgɯ zɯ ku-rɤʑi ɲɯ-ŋu, kɯ-dɤn kɯ ɣurʑa tu ɲɯ-ŋgrɤl, kɯ-rkɯn kɯ ʁnɯz χsɯm ku-rɤʑi tu ɲɯ-ŋgrɤl, jɤ-ari-nɯ tɕe stu kɯ-mɤku nɯ ju-ɕe ɯ-qhɯ-qhu zɯ kɯmdi ju-ɕe-nɯ, ɯʑo qaʑo cho kɯ-naχtɕɯɣ ŋu, zgoku-rɤʑi ma co ɯ-ŋgɯ zɯ ku-rɤʑi mɯ́j-ŋgrɤl, ɯ-rme nɯ caʁɕɣɤz kɯ-tu kɯ-wɣrum ɲɯ-ŋu.\cmn 盘羊生活在高山上,没有草木的岩石上。多的有一百只左右,少的只有两三只一起生活,它们走动时所有的盘羊跟在一个领头的后面,和绵羊一样。只出现在高山上,不会下河坝来。有很多白色的粗毛。\end{exemple}\end{entrée}

\begin{entrée}
\vedette{\hypertarget{Ⓔldɯɣldɯɣ}{\papi{ ldɯɣldɯɣ}}}\markboth{ldɯɣldɯɣ}{}
\classe{idph.2}
\begin{définition}\fra ciel sombre, rempli de nuages\end{définition}
\begin{définition}\cmn 形容浓云密布的样子\end{définition}
\begin{exemple}\jya tɯ-mɯ ldɯɣldɯɣ ʑo ɲɯ-pa\cmn 天上的云密密麻麻的\end{exemple}
\begin{exemple}\jya tɤ-lu tʂha ldɯɣldɯɣ ɲɯ-pa\cmn 奶茶好喝\end{exemple}
\begin{exemple}\jya jisŋi ɲɯ-lɯβ ldɯɣldɯɣ ʑo\cmn 今天天色很阴的样子\end{exemple}
\begin{relation-sémantique}\confer{
\hyperlink{Ⓔdʐɯɣdʐɯɣ}{\textit{ \papi{dʐɯɣdʐɯɣ}}}
}\end{relation-sémantique}\end{entrée}

\begin{entrée}
\vedette{\hypertarget{Ⓔldɯɣɯ}{\papi{ ldɯɣɯ}}}\markboth{ldɯɣɯ}{}
\classe{n}
\begin{définition}\fra couteau courbé\end{définition}
\begin{définition}\cmn 弯刀\end{définition}\end{entrée}

\begin{entrée}
\vedette{\hypertarget{Ⓔldɯm}{\papi{ ldɯm}}}\markboth{ldɯm}{}
\classe{vs}
\paradigme{\textit{dir :} \jya tɤ-}
\begin{définition}\fra sérieux\end{définition}
\begin{définition}\cmn 稳重;谨慎\end{définition}
\begin{exemple}\jya ki tɯrme ki kɯ-ldɯm ci ɲɯ-ŋu\cmn 这个人很稳重\end{exemple}
\begin{exemple}\jya tɤ-pɤtso ɲɯ-ldɯm\cmn 孩子很听话\end{exemple}\end{entrée}

\begin{entrée}
\vedette{\hypertarget{Ⓔldzɣɤβldzɣɤβ}{\papi{ ldzɣɤβldzɣɤβ}}}\markboth{ldzɣɤβldzɣɤβ}{}
\classe{idph.2}
\begin{définition}\fra en désordre\end{définition}
\begin{définition}\cmn 凌乱(衣服、线)\end{définition}
\begin{exemple}\jya tɤ-ri ɲɤ-k-ɤɬɯt-ci ldzɣɤβldzɣɤβ ʑo\cmn 线凌乱了\end{exemple}
\begin{exemple}\jya ɯ-ŋga ɲɯ-ɤdrɤt ldzɣɤβldzɣɤβ ʑo\cmn 他衣服放得很凌乱\end{exemple}\begin{sous-entrée}
\vedette{\hypertarget{}{\papi{ ldzɣɤβnɤldzɣɤβ}}}\markboth{ldzɣɤβnɤldzɣɤβ}{}\classe{idph.3}
\begin{exemple}\jya ɯ-ŋga ɯ-mɤ-tɯ-pe ldzɣɤβnɤldzɣɤβ ʑo kɤ-ɕqhlɤt\cmn 他衣衫褴褛地过去了\end{exemple}
\end{sous-entrée}\end{entrée}

\begin{entrée}
\vedette{\hypertarget{Ⓔldʑaŋkɯ}{\papi{ ldʑaŋkɯ}}}\markboth{ldʑaŋkɯ}{}\classe{n}
\begin{définition}\fra vert\end{définition}
\begin{définition}\cmn 绿(布料、线)
\begin{déclaration} \étymologie{\papi{ldʑaŋ.kʰu}}\end{déclaration}\end{définition}
\begin{relation-sémantique}\confer{
\hyperlink{Ⓔarɯldʑaŋkɯ}{\textit{ \papi{arɯldʑaŋkɯ}}}
}\end{relation-sémantique}\end{entrée}

\begin{entrée}
\vedette{\hypertarget{Ⓔldʑaŋnaʁ}{\papi{ ldʑaŋnaʁ}}}\markboth{ldʑaŋnaʁ}{}\classe{n}
\begin{définition}\fra vert foncé\end{définition}
\begin{définition}\cmn 深绿色
\begin{déclaration} \étymologie{\papi{ldʑaŋ.nag}}\end{déclaration}\end{définition}\end{entrée}

\begin{entrée}
\vedette{\hypertarget{Ⓔldʑoʁ}{\papi{ ldʑoʁ}}}\markboth{ldʑoʁ}{}\classe{vi}
\paradigme{\textit{dir :} \jya tɤ-}
\paradigme{\textit{dir :} \jya nɯ-}
\begin{définition}\fra parvenir à complétion\end{définition}
\begin{définition}\cmn 完成\end{définition}
\begin{exemple}\jya kɤ-nɤma ra tɤ-ldʑoʁ\cmn 事情完成了\end{exemple}
\begin{exemple}\jya kha ta-ma kɯ-ldʑoʁ me\cmn 家务是做不完的\end{exemple}\begin{sous-entrée}
\vedette{\hypertarget{}{\papi{ sɯldʑoʁ}}}\markboth{sɯldʑoʁ}{}\classe{vt}
\begin{définition}\fra mener à complétion\end{définition}
\begin{définition}\cmn 完成\end{définition}
\begin{exemple}\jya ta-ma kɤ-sɯldʑoʁ mɯ́j-khɯ ma ɲɯ-dɤn\cmn 事情不能一下完成,因为头绪多\end{exemple}
\end{sous-entrée}\end{entrée}

\begin{entrée}
\vedette{\hypertarget{Ⓔldʑɯβ}{\papi{ ldʑɯβ}}}\markboth{ldʑɯβ}{}\classe{vt}
\begin{définition}\fra pouvoir aider\end{définition}
\begin{définition}\cmn 帮得了\end{définition}
\begin{exemple}\jya aʑo kɯ nɤʑo mɤ-ta-ldʑɯβ\cmn (我是弱者),没有能力帮你\end{exemple}
\begin{exemple}\jya xɕɤndʑu kɯ zdoŋbu mɤ-ldʑɯβ\cmn 弱者帮不了强者\end{exemple}\end{entrée}

\begin{entrée}
\vedette{\hypertarget{Ⓔldʑɯŋldʑɯŋ}{\papi{ ldʑɯŋldʑɯŋ}}}\markboth{ldʑɯŋldʑɯŋ}{}\classe{idph.2}
\begin{définition}\fra bleu (ciel)\end{définition}
\begin{définition}\cmn 形容天的蓝色
\begin{déclaration} \étymologie{\papi{ldʑaŋ}}\end{déclaration}\end{définition}
\begin{exemple}\jya tɯ-mɯ ɲɯ-ɤrŋi ldʑɯŋldʑɯŋ ʑo\cmn 天空蓝蓝的样子\end{exemple}
\end{entrée}

\begin{entrée}
\vedette{\hypertarget{Ⓔldʑɯz}{\papi{ ldʑɯz}}}\markboth{ldʑɯz}{}
\classe{vs}
\paradigme{\textit{dir :} \jya nɯ-}
\paradigme{\textit{dir :} \jya tɤ-}
\begin{définition}\fra flexible (branche)\end{définition}
\begin{définition}\cmn 有韧性‘柔韧\end{définition}
\begin{exemple}\jya ki ʑmbri ɲɯ-ldʑɯz\cmn 这棵杨树有韧性\end{exemple}\end{entrée}

\begin{entrée}
\vedette{\hypertarget{Ⓔlɣa}{\papi{ lɣa}}}\markboth{lɣa}{}
\classe{vt}
\paradigme{\textit{dir :} \jya lɤ-}
\paradigme{\textit{dir :} \jya tɤ-}
\begin{définition}\fra creuser\end{définition}
\begin{définition}\cmn 挖\end{définition}
\begin{exemple}\jya ŋgɤm nɯ lɤ-lɣa-t-a\cmn 我挖了土坡\end{exemple}
\begin{exemple}\jya sɤtɕha tɤ-lɣa-t-a\cmn 我挖了地\end{exemple}\end{entrée}

\begin{entrée}
\vedette{\hypertarget{Ⓔlɣɤβlɣɤβ}{\papi{ lɣɤβlɣɤβ}}}\markboth{lɣɤβlɣɤβ}{}\classe{idph.2}
\begin{définition}\fra épais (vêtements)\end{définition}
\begin{définition}\cmn 沉重;厚实(衣服)\end{définition}
\begin{exemple}\jya a@pugai lɣɤβlɣɤβ ʑo ɲɯ-pa\cmn 我的被子很沉重\end{exemple}
\begin{exemple}\jya lɣɤβlɣɤβ ʑo pɯ́-wɣ-ɲcar-a\cmn 他重重地压着我\end{exemple}
\begin{exemple}\jya ɲɯ-rʑi lɣɤβlɣɤβ ʑo\cmn 比较重\end{exemple}\begin{sous-entrée}
\vedette{\hypertarget{}{\papi{ lɣɤβnɤlɣɤβ}}}\markboth{lɣɤβnɤlɣɤβ}{}\classe{idph.3}
\begin{exemple}\jya lɣɤβnɤlɣɤβ pɯ-ŋke-a\cmn 我穿着笨重的衣服走路\end{exemple}
\begin{relation-sémantique}\confer{
\hyperlink{Ⓔlxɤβlxɤβ}{\textit{ \papi{lxɤβlxɤβ}}}
}\end{relation-sémantique}
\begin{relation-sémantique}\confer{
\hyperlink{Ⓔlxɯlxi}{\textit{ \papi{lxɯlxi}}}
}\end{relation-sémantique}
\end{sous-entrée}\end{entrée}

\begin{entrée}
\vedette{\hypertarget{ⒺliⒽ1}{\papi{ li}}}\markboth{li}{}\homonyme{1}
\classe{adv}
\begin{définition}\fra encore\end{définition}
\begin{définition}\cmn 再\end{définition}
\begin{exemple}\jya li ci tɤ-ti\cmn 你再说一遍\end{exemple}
\begin{exemple}\jya li ci tshupa nɯ tɕhi rmi\cmn 下一个村子叫什么名字?\end{exemple}\end{entrée}

\begin{entrée}
\vedette{\hypertarget{ⒺliⒽ2}{\papi{ li}}}\markboth{li}{}\homonyme{2}
\classe{n}
\begin{définition}\fra cuivre\end{définition}
\begin{définition}\cmn 铜
\begin{déclaration} \étymologie{\papi{li}}\end{déclaration}\end{définition}\end{entrée}

\begin{entrée}
\vedette{\hypertarget{ⒺliⒽ3}{\papi{ li}}}\markboth{li}{}\homonyme{3}
\classe{vs}
\paradigme{\textit{dir :} \jya thɯ-}
\begin{définition}\fra mal élevé, gâté\end{définition}
\begin{définition}\cmn 娇生惯养,被宠坏\end{définition}
\begin{exemple}\jya cho-li\cmn 他变得很任性\end{exemple}
\begin{exemple}\jya jiɕqha ɯ-rɟit nɯ wuma ɲɯ-li\cmn 他的孩子被宠坏了\end{exemple}\begin{sous-entrée}
\vedette{\hypertarget{}{\papi{ sɤsɯɣli}}}\markboth{sɤsɯɣli}{}\classe{vi}
\begin{définition}\ 
\begin{déclaration}\grammar{apass}\end{déclaration}\end{définition}
\begin{définition}\fra gâter les enfants\end{définition}
\begin{définition}\cmn 惯养小孩\end{définition}
\end{sous-entrée}\begin{sous-entrée}
\vedette{\hypertarget{}{\papi{ sɯɣli}}}\markboth{sɯɣli}{}\classe{vt}
\paradigme{\textit{dir :} \jya thɯ-}
\begin{définition}\ 
\begin{déclaration}\grammar{caus}\end{déclaration}\end{définition}
\begin{définition}\fra trop gâter\end{définition}
\begin{définition}\cmn 宠坏\end{définition}
\begin{exemple}\jya ɯ-pɤtso ɲɯ-sɯɣli\cmn 他在惯着孩子\end{exemple}
\begin{exemple}\jya thɯ-sɯɣli-t-a\cmn 我惯养了他\end{exemple}
\begin{exemple}\jya thɯ́-wɣ-sɯɣli-a\cmn 他惯养了我\end{exemple}
\end{sous-entrée}\end{entrée}

\begin{entrée}
\vedette{\hypertarget{Ⓔluj}{\papi{ luj}}}\markboth{luj}{}
\classe{vt}
\paradigme{\textit{dir :} \jya tɤ-}
\begin{définition}\fra recouvrir complètement la surface pour cacher\end{définition}
\begin{définition}\cmn 包起来;遮蔽;裹住\end{définition}
\begin{exemple}\jya kɯki @huatong ki tú-wɣ-luj\cmn 把这个话筒遮住\end{exemple}\begin{sous-entrée}
\vedette{\hypertarget{}{\papi{ aluj}}}\markboth{aluj}{}\classe{vi}
\paradigme{\textit{dir :} \jya tɤ-}
\begin{définition}\fra être recouvert\end{définition}
\begin{définition}\cmn 被遮住,被裹起来\end{définition}
\end{sous-entrée}\begin{sous-entrée}
\vedette{\hypertarget{}{\papi{ sɯluj}}}\markboth{sɯluj}{}\classe{vt}
\paradigme{\textit{dir :} \jya tɤ-}
\begin{définition}\fra couvrir (avec qqch)\end{définition}
\begin{définition}\cmn 遮住(用东西)\end{définition}
\begin{exemple}\jya kɯki ɯ-fkɯm tú-wɣ-βzu tú-wɣ-sɯluj\cmn 要做个套子把这个东西掩盖起来\end{exemple}
\begin{exemple}\jya tɯ-ŋga tɤ-nga-t-a tɕe tɤ́-wɣ-sɯluj-a\cmn 我穿了衣服,把我遮住了\end{exemple}
\end{sous-entrée}\end{entrée}

\begin{entrée}
\vedette{\hypertarget{Ⓔljɤɣljɤɣ}{\papi{ ljɤɣljɤɣ}}}\markboth{ljɤɣljɤɣ}{}
\classe{idph.2}
\begin{définition}\fra épais et long\end{définition}
\begin{définition}\cmn 形容粗而长,看上去蓬松,但摸起来有些硬的样子\end{définition}
\begin{exemple}\jya paʁ chɤ-tshu ljɤɣljɤɣ ʑo\cmn 猪变得又胖又长\end{exemple}\end{entrée}

\begin{entrée}
\vedette{\hypertarget{Ⓔljulju}{\papi{ ljulju}}}\markboth{ljulju}{}\classe{idph.2}
\begin{définition}\fra cylindrique\end{définition}
\begin{définition}\cmn 圆柱形\end{définition}
\begin{relation-sémantique}\confer{
\hyperlink{Ⓔalɯlju}{\textit{ \papi{alɯlju}}}
}\end{relation-sémantique}\end{entrée}

\begin{entrée}
\vedette{\hypertarget{Ⓔlɟɣaʁ}{\papi{ lɟɣaʁ}}}\markboth{lɟɣaʁ}{}
\classe{vt}
\paradigme{\textit{dir :} \jya pɯ-}
\paradigme{\textit{dir :} \jya \_}
\begin{définition}\fra étendre (un habit, une corde) sur un autre objet\end{définition}
\begin{définition}\cmn 搭上去(衣服、毛巾、绳子等)\end{définition}
\begin{exemple}\jya nɯki tɯ-ŋga tɤ-lɟɣaʁ\cmn 你把衣服搭上去\end{exemple}
\begin{exemple}\jya pa-lɟɣaʁ\cmn 他(把衣服)搭了\end{exemple}
\begin{exemple}\jya nɯki tɯ-mbri nɯ kɤcu kɤ-lɟɣaʁ\cmn 你把绳子搭在那里\end{exemple}\end{entrée}

\begin{entrée}
\vedette{\hypertarget{Ⓔlkɯɣ}{\papi{ lkɯɣ}}}\markboth{lkɯɣ}{}\classe{vi}
\paradigme{\textit{dir :} \jya kɤ-}
\begin{définition}\fra être ankylosé après avoir manqué d'exercice\end{définition}
\begin{définition}\cmn 因为缺乏锻炼,关节僵硬的感觉\end{définition}
\begin{exemple}\jya ko-lkɯɣ-a\cmn 我的关节都不灵\end{exemple}
\begin{exemple}\jya nɤ-βri ra a-ʑ-nɯ-tɯ-zmɯnme tɕe a-mɤ-kɤ-tɯ-lkɯɣ\cmn 你要锻炼一下身体,这样就不会有关节不灵的感觉\end{exemple}\end{entrée}

\begin{entrée}
\vedette{\hypertarget{Ⓔlmɤlmɤr}{\papi{ lmɤlmɤr}}}\markboth{lmɤlmɤr}{}\classe{idph.2}
\begin{définition}\fra mou\end{définition}
\begin{définition}\cmn 形容很软,好像没有骨头的样子\end{définition}
\begin{exemple}\jya mbro ɯ-mtɕhi lmɤlmɤr ʑo ɲɯ-pa\cmn 马的嘴很软\end{exemple}\end{entrée}

\begin{entrée}
\vedette{\hypertarget{Ⓔlni}{\papi{ lni}}}\markboth{lni}{}
\classe{vi}
\paradigme{\textit{dir :} \jya nɯ-}
\begin{définition}\fra flétrir à cause de la chaleur\end{définition}
\begin{définition}\cmn 蔫\end{définition}
\begin{exemple}\jya sɯjno tɤ-phɯt-a tɕe ɲo-lni\cmn 我拔了草就蔫了\end{exemple}
\begin{exemple}\jya razti ɲo-lni\cmn 圆根蔫了\end{exemple}\end{entrée}

\begin{entrée}
\vedette{\hypertarget{Ⓔlɲɯɣlɲɯɣ}{\papi{ lɲɯɣlɲɯɣ}}}\markboth{lɲɯɣlɲɯɣ}{}\classe{idph.2}
\begin{définition}\fra habillé de façon négligé\end{définition}
\begin{définition}\cmn 形容衣冠不整,(衣服)松,拖得很长的样子\end{définition}
\begin{exemple}\jya kɤ́-ŋgɯ-ŋga lɲɯɣlɲɯɣ kɯ kɯ-rɤma nɤ-scɯʁzɯɣ maʁ\cmn 你衣服穿得又松有长,看起来不像是干活的人\end{exemple}\end{entrée}

\begin{entrée}
\vedette{\hypertarget{Ⓔlŋaʁlŋaʁ}{\papi{ lŋaʁlŋaʁ}}}\markboth{lŋaʁlŋaʁ}{}\classe{idph.2}
\begin{définition}\fra petit et mince\end{définition}
\begin{définition}\cmn 形容瘦而小,令人讨厌的样子\end{définition}
\begin{exemple}\jya paʁtsa nɯ ɲɯ-ndɯβ-nɯ ʑo lŋaʁlŋaʁ ɕti\cmn 猪崽子又小又瘦,令人讨厌\end{exemple}\begin{sous-entrée}
\vedette{\hypertarget{}{\papi{ lŋaʁnɤlŋaʁ}}}\markboth{lŋaʁnɤlŋaʁ}{}\classe{idph.3}
\begin{exemple}\jya ɯ-rɟit nɯ mɯ́j-scit tɕe, lŋaʁnɤlŋaʁ ʑo chɯ-ɣɤwu ɲɯ-ŋu\cmn 他的孩子不乖,一直在哭,令人讨厌\end{exemple}
\end{sous-entrée}\end{entrée}

\begin{entrée}
\vedette{\hypertarget{Ⓔlŋɤβnɤlŋɤβ}{\papi{ lŋɤβnɤlŋɤβ}}}\markboth{lŋɤβnɤlŋɤβ}{}\classe{idph.3}
\begin{définition}\fra qui a la bouche grande ouverte\end{définition}
\begin{définition}\cmn 形容嘴巴开得很大,耷拉着耳朵,不美观的样子\end{définition}
\begin{exemple}\jya khɯna kɯ tɤ-mthɯm lŋɤβnɤlŋɤβ ʑo ɲɯ-ɤsɯ-ndza\cmn 狗在大口大口地吃肉\end{exemple}
\begin{relation-sémantique}\confer{
\hyperlink{Ⓔnɯlŋɤβ}{\textit{ \papi{nɯlŋɤβ}}}
}\end{relation-sémantique}\end{entrée}

\begin{entrée}
\vedette{\hypertarget{Ⓔlŋɤlŋɤt}{\papi{ lŋɤlŋɤt}}}\markboth{lŋɤlŋɤt}{}\classe{idph.2}
\begin{définition}\fra beaucoup de (fruits) accrochés\end{définition}
\begin{définition}\cmn 很多,很大的东西(挂着)\end{définition}
\begin{exemple}\jya ɯ-mat lŋɤlŋɤt ʑo ko-tshoʁ\cmn 结了很多的果子\end{exemple}
\begin{exemple}\jya ʁmɯrtsɯ kɯ ɯ-mat lŋɤlŋɤt ko-tshoʁ\cmn 黑泡儿结了很多果子\end{exemple}
\begin{exemple}\jya ɯ-taʁ lŋɤlŋɤt ʑo pjɤ-lɟɣaʁ\cmn 搭在上面,显得很大\end{exemple}
\begin{exemple}\jya tɯ-ŋga lŋɤlŋɤt ʑo to-ɕɯɴqoʁ\cmn 他挂了很多衣服\end{exemple}\begin{sous-entrée}
\vedette{\hypertarget{}{\papi{ lŋɤnɤlŋɤt}}}\markboth{lŋɤnɤlŋɤt}{}\classe{idph.3}
\begin{exemple}\jya lŋɤnɤlŋɤt ɲɯ-nɯndzɯlŋɯz\cmn 他在打瞌睡,头一点一点的\end{exemple}
\end{sous-entrée}\end{entrée}

\begin{entrée}
\vedette{\hypertarget{ⒺloⒽ2}{\papi{ lo}}}\markboth{lo}{}\homonyme{2} (\variante{loβ}) 
\classe{part}
\begin{définition}\fra d'accord\end{définition}
\begin{définition}\cmn 吧\end{définition}
\begin{exemple}\jya nɤ-ŋga tɤ-ŋge ɲɯ-mna lo\cmn 你把衣服穿上好吗?\end{exemple}
\begin{relation-sémantique}\confer{
\hyperlink{Ⓔlotɕi}{\textit{ \papi{lotɕi}}}
}\end{relation-sémantique}\end{entrée}

\begin{entrée}
\vedette{\hypertarget{ⒺloⒽ1}{\papi{ lo}}}\markboth{lo}{}\homonyme{1}
\classe{vi}
\paradigme{\textit{dir :} \jya tɤ-}
\begin{définition}\fra avoir l'immunité contre\end{définition}
\begin{définition}\cmn 有免疫能力\end{définition}
\begin{exemple}\jya aʑo tɤ-mbrɯm tɤ-lo-a\end{exemple}
\begin{exemple}\jya aʑo tɤ-mbrɯm tɤ-kɯ-lo ŋu-a\cmn 我对麻子有免疫能力\end{exemple}\end{entrée}

\begin{entrée}
\vedette{\hypertarget{Ⓔlochu}{\papi{ lochu}}}\markboth{lochu}{}\classe{adv}
\begin{définition}\fra en amont\end{définition}
\begin{définition}\cmn 在上游\end{définition}\end{entrée}

\begin{entrée}
\vedette{\hypertarget{Ⓔlonba}{\papi{ lonba}}}\markboth{lonba}{}\classe{adv}
\begin{définition}\fra tout\end{définition}
\begin{définition}\cmn 一切
\begin{déclaration} \étymologie{\papi{lon.pa}}\end{déclaration}\end{définition}
\begin{exemple}\jya lonbɯnba\cmn 所有一切\end{exemple}\end{entrée}

\begin{entrée}
\vedette{\hypertarget{Ⓔloŋbutɕhi}{\papi{ loŋbutɕhi}}}\markboth{loŋbutɕhi}{}\classe{n}
\begin{définition}\fra éléphant\end{définition}
\begin{définition}\cmn 大象
\begin{déclaration} \étymologie{\papi{glaŋ.po.tɕʰe}}\end{déclaration}\end{définition}\end{entrée}

\begin{entrée}
\vedette{\hypertarget{Ⓔloŋloŋ}{\papi{ loŋloŋ}}}\markboth{loŋloŋ}{}
\classe{idph.2}\acception{1}
\begin{définition}\fra imposant\end{définition}
\begin{définition}\cmn 高大;颜色比较黑\end{définition}
\begin{exemple}\jya jla loŋloŋ ʑo ɲɯ-rɤʑi\cmn 犏牛在那里又黑又高大\end{exemple}
\begin{exemple}\jya si loŋloŋ ʑo ɲɯ-pa\cmn 树很高大\end{exemple}
\begin{exemple}\jya loŋloŋ ʑo ɲɯ-rŋgɯ\cmn 他在那里睡觉(很高大的样子)\end{exemple}\begin{sous-entrée}
\vedette{\hypertarget{}{\papi{ ɣɤloŋloŋ}}}\markboth{ɣɤloŋloŋ}{}\classe{vi}
\begin{définition}\fra s'élever (fumée, nuage)\end{définition}
\begin{définition}\cmn 缭缭升起,缭绕上升(烟子、云)\end{définition}
\begin{exemple}\jya zgo ɯ-taʁ zdɯm ɲɯ-ɣɤloŋloŋ\cmn 山被云雾笼罩\end{exemple}
\begin{exemple}\jya tɤ-khɯ ɲɯ-ɣɤloŋloŋ\cmn 烟子寥寥升起\end{exemple}
\end{sous-entrée}\begin{sous-entrée}
\vedette{\hypertarget{}{\papi{ loŋnɤloŋ}}}\markboth{loŋnɤloŋ}{}\classe{idph.3}
\begin{exemple}\jya zdɯm loŋnɤloŋ ʑo jo-ɣi\cmn 乌云从四方聚拢而来\end{exemple}
\end{sous-entrée}\begin{sous-entrée}
\vedette{\hypertarget{}{\papi{ loŋɯŋi}}}\markboth{loŋɯŋi}{}
\begin{définition}\fra s'élever lentement\end{définition}
\begin{définition}\cmn 慢慢地升起\end{définition}
\begin{exemple}\jya tɤ-khɯ loŋɯŋi ʑo to-ɬoʁ\cmn 烟子慢慢地冒出来了\end{exemple}
\end{sous-entrée}\end{entrée}

\begin{entrée}
\vedette{\hypertarget{Ⓔloʁ}{\papi{ loʁ}}}\markboth{loʁ}{}
\classe{vs}
\paradigme{\textit{dir :} \jya nɯ-}
\begin{définition}\fra travailler, se déformer à cause de l'humidité\end{définition}
\begin{définition}\cmn 木板受潮而变形\end{définition}
\begin{exemple}\jya tɤrɤmɕkho ɲo-loʁ\cmn 地板变形了\end{exemple}
\begin{exemple}\jya tʂɤm ɲɤ-loʁ\cmn 板壁变形了\end{exemple}
\begin{exemple}\jya kɯm ɲɤ-loʁ tɕe kɤ-nɤcɯpa ɲɯ-ɴqa\cmn 门变形了,很难开关\end{exemple}\end{entrée}

\begin{entrée}
\vedette{\hypertarget{Ⓔloʁskɤr}{\papi{ loʁskɤr}}}\markboth{loʁskɤr}{}\classe{n}
\begin{définition}\fra perles insérées dans les tresses\end{définition}
\begin{définition}\cmn 辫子上穿的一圈一圈的珠子\end{définition}\end{entrée}

\begin{entrée}
\vedette{\hypertarget{Ⓔlotɕi}{\papi{ lotɕi}}}\markboth{lotɕi}{}\classe{part}
\begin{définition}\fra non\end{définition}
\begin{définition}\cmn 呢\end{définition}
\begin{exemple}\jya nɤ-smɤn ko-tɯ-tshi-t lotɕi\cmn 你不是喝了药的吗?\end{exemple}
\begin{exemple}\jya ʑatsa tɯ-nɯɕe lotɕi\cmn 你不是快要回去了吗?\end{exemple}
\begin{relation-sémantique}\confer{
\hyperlink{ⒺloⒽ2}{\textit{ \papi{lo2}}}
}\end{relation-sémantique}\end{entrée}

\begin{entrée}
\vedette{\hypertarget{Ⓔlqɤnɤlqɤt}{\papi{ lqɤnɤlqɤt}}}\markboth{lqɤnɤlqɤt}{}\classe{idph.2}
\begin{définition}\fra lentement, en titubant\end{définition}
\begin{définition}\cmn 形容走路慢而摇晃的样子\end{définition}
\begin{exemple}\jya tɤ-pɤtso nɯ lqɤnɤlqɤt tu-ŋke ɲɯ-cha\cmn 小孩子能摇摇晃晃地走路\end{exemple}
\begin{relation-sémantique}\confer{
\hyperlink{Ⓔsɤlqɤlqɤt}{\textit{ \papi{sɤlqɤlqɤt}}}
}\end{relation-sémantique}\end{entrée}

\begin{entrée}
\vedette{\hypertarget{Ⓔlʁa}{\papi{ lʁa}}}\markboth{lʁa}{}
\classe{n}
\begin{définition}\fra sac en toile\end{définition}
\begin{définition}\cmn 麻袋\end{définition}
\begin{exemple}\jya lʁa nɯ tasa thɯ-kɤ-βzu tɤ-fkɯm ŋu, tasa kɤ́-wɣ-phɯt tɕe khɤxtu zɯ pjɯ́-wɣ-z-nɯɣur tɕe ɲɯ́-wɣ-ta, tɤ-rom tɕe ɯ-rɣi nɯ pjɯ́-wɣ-kra tɕe ɯ-pɯ tú-wɣ-pa tɕe ɯ-ru tú-wɣ-xtɕɤr tɕe ɲɯ́-wɣ-ta, ftɕar tɕe tɯ-mɯ kɤ-lɤt tɕe, ɯ-thoʁ pjɯ́-wɣ-sɤʑɯrja tɕe kɤ-la ʑo tɕe chɯ́-wɣ-ʑɴɢu tɕe tɯ-zboʁ tɯ-zboʁ tú-wɣ-xtɕɤr, tɕe tɤ-zbaʁ tɕe lú-wɣ-pɣo, lú-wɣ-rɯm, tɕe tɤ-βri tú-wɣ-βzu, tɤ-βri nɯ kú-wɣ-sqa, ɯ-ŋgɯ thɤfkɤlɤɣi lú-wɣ-lɤt tɕe ɲɯ́-wɣ-sɯ-ɤla. thɯ́-wɣ-tɕɤt nɯ-afɕu tɕe ɲɯ́-wɣ-χtɕi, ɲɯ́-wɣ-sɯ-wɣrum ʑo tɕe ɲɯ́-wɣ-ɕkho, nɯ a-tɤ-zbaʁ tɕe pjɯ́-wɣ-xtsɯ ra, ɲɯ́-wɣ-ɣɤmpɯ ʑo tɕe kú-wɣ-sɤrɤt tɕe kɤ-taʁ kú-wɣ-thɯ. kɤ-taʁ nɯ ŋgɤrom tu, tɯ-taχte tu, tɯ-taχte nɯ kɯ kɯ-mna kɤ-sɯpa ŋu, ɯ-spa ɴqa, kɤ-taʁ ɴqa, tɕeri ngɯt, ŋgɤrom nɯ ɯ-spa mbat, kɤ-taʁ mbat, tɕeri kɤ-ntɕhoz tɕe mɤ-ngɯt, kɤ-taʁ thɯ-jɤɣ tɕe, li nɯ ɣɯ-χtɕi ra tɤ-zbaʁ tɕe ɣɯ-xtsɯ ra, pɯ́-wɣ-xtsɯ kóʁmɯz nɤ ɲɯ́-wɣ-qrɯ tɕe, tɕe chɯ́-wɣ-tʂɯβ, tɕe kɯ-ɤβʑɯrdu kɯ-rɲɟi tsa ɲɯ́-wɣ-βzu, ɯ-qa χchoʁe nɯ tɕu ɯ-jndɯz kú-wɣ-tshoʁ ra, ɯ-mŋu zɯ tɤ-fsɤri maʁ nɤ qase ɯ-sɤ-xtɕɤr kú-wɣ-tshoʁ ra, tɕe nɯ kóʁmɯz nɤ lʁa kɤ-ntɕhoz tu-βze ŋu. nɯnɯ kɯrɯ ra ji-tɯjpu ɯ-sɤ-rku ŋu, tɤ-fkɯm lʁa kɯ-maʁ ʁnɯ-tɯ-phɯ tu, tɯ-ndʐi kɯ thɯ-kɤ-βzu ci tu tɕe, nɯ zgrawa rmi, smɤɣ kɯ thɯ-kɤ-βzu ci tu tɕe, nɯnɯ zɟi rmi.\cmn 
麻袋是用大麻做成的口袋。大麻割了以后放在屋顶上,让它受霜。干了以后,把种子抖掉后收藏好,再把麻杆捆起来放好。到春天下雨的时候,把麻杆摆在地面上,让它浸泡,然后剥下麻皮,一把一把地捆在一起。晾干后先用手搓成细线,用吊干搓紧,再反搓成一绞一绞,然后再煮。里面要放比较多的灶灰,让它煮开一段时间才取出来。冷却后再洗白凉干,然后捶打到变柔和为止,才把线卸下来,再牵在牵杆上(准备织)。织布的方式有两种,单巴子和斜纹子。斜纹子是比较优质的,用的材料多一些,织起来难一些,但是比较结实。单巴子材料用得不多,织起来容易一些,但用起来不结实。织完后,又要洗,干了以后还要捶打,然后再裁下来,缝好,做成长方形的。在底部的左右两边要装饰上流苏,在口部要扎上麻绳或者皮绳用来封口。这样才能使用麻袋。那是我们藏民装粮食的口袋。除了麻袋,还有两种口袋,一种是用皮做成的,叫\stylefv{zgrawa},另一种是用羊毛做成的,叫\stylefv{zɟi}。
\end{exemple}\end{entrée}

\begin{entrée}
\vedette{\hypertarget{Ⓔlʁɤtɕɯ}{\papi{ lʁɤtɕɯ}}}\markboth{lʁɤtɕɯ}{}\classe{n}
\begin{définition}\fra petit sac\end{définition}
\begin{définition}\cmn 小口袋\end{définition}\end{entrée}

\begin{entrée}
\vedette{\hypertarget{Ⓔlʁɯba}{\papi{ lʁɯba}}}\markboth{lʁɯba}{}\classe{n}
\begin{définition}\fra muet\end{définition}
\begin{définition}\cmn 哑巴
\begin{déclaration} \étymologie{\papi{lkug.pa?glen.pa?}}\end{déclaration}\end{définition}
\end{entrée}

\begin{entrée}
\vedette{\hypertarget{Ⓔltɤβ}{\papi{ ltɤβ}}}\markboth{ltɤβ}{}
\classe{vt}
\paradigme{\textit{dir :} \jya kɤ-}
\begin{définition}\fra plier\end{définition}
\begin{définition}\cmn 折起来;折叠
\begin{déclaration} \étymologie{\papi{lteb}}\end{déclaration}\end{définition}
\begin{exemple}\jya kɤ-ltaβ-a\cmn 我折了\end{exemple}
\begin{exemple}\jya nɤki tɯ-ŋga nɯ kɤ-ltɤβ\cmn 你把那件衣服折一下\end{exemple}
\begin{relation-sémantique}\synonyme{
\hyperlink{Ⓔzdɤβ}{\textit{ \papi{zdɤβ}}}
}\end{relation-sémantique}\end{entrée}

\begin{entrée}
\vedette{\hypertarget{Ⓔltɕaʁ}{\papi{ ltɕaʁ}}}\markboth{ltɕaʁ}{}\classe{vt}
\paradigme{\textit{dir :} \jya tɤ-}
\begin{définition}\fra frapper le beurre\end{définition}
\begin{définition}\cmn 用手拍打酥油(挤出奶水)
\end{définition}
\begin{exemple}\jya ta-mar tɤ-ltɕaʁ\cmn 你拍打一下酥油吧\end{exemple}\end{entrée}

\begin{entrée}
\vedette{\hypertarget{Ⓔltɕhaŋltɕhaŋ}{\papi{ ltɕhaŋltɕhaŋ}}}\markboth{ltɕhaŋltɕhaŋ}{}\classe{idph.2}
\begin{définition}\fra long et pendant\end{définition}
\begin{définition}\cmn 形容长的物体悬吊着的样子\end{définition}
\begin{exemple}\jya mbro ɯ-jme ltɕhaŋltɕhaŋ ʑo pa\cmn 马的尾巴吊着很长\end{exemple}\end{entrée}

\begin{entrée}
\vedette{\hypertarget{Ⓔltɕhɤltɕhɤt}{\papi{ ltɕhɤltɕhɤt}}}\markboth{ltɕhɤltɕhɤt}{}
\classe{idph.2}
\begin{définition}\fra être suspendu (épis de céréales, touffe de fils etc...)\end{définition}
\begin{définition}\cmn 形容一簇絮状的东西,穗子吊着的样子\end{définition}\begin{sous-entrée}
\vedette{\hypertarget{}{\papi{ sɤltɕhɤltɕhɤt}}}\markboth{sɤltɕhɤltɕhɤt}{}\classe{vt}
\begin{définition}\fra secouer légèrement (un objet long et fin)\end{définition}
\begin{définition}\cmn 轻轻滴抖动(软、细长的东西);轻轻地撒(水)\end{définition}
\begin{exemple}\jya tɯ-ci ɲɯ-sɤltɕhɤltɕhɤt\cmn 他在轻轻地撒水\end{exemple}
\begin{exemple}\jya sɯjwaʁ ɲɯ-sɤltɕhɤltɕhɤt\cmn 他在轻轻摇动树叶\end{exemple}
\begin{relation-sémantique}\synonyme{
\hyperlink{Ⓔsɤʑdraŋlaŋ}{\textit{ \papi{sɤʑdraŋlaŋ}}}
}\end{relation-sémantique}
\begin{relation-sémantique}\synonyme{
\hyperlink{Ⓔsɤɕtʂɯlɯɣ}{\textit{ \papi{sɤɕtʂɯlɯɣ}}}
}\end{relation-sémantique}
\end{sous-entrée}\end{entrée}

\begin{entrée}
\vedette{\hypertarget{Ⓔltɕhɯɣltɕhɯɣ}{\papi{ ltɕhɯɣltɕhɯɣ}}}\markboth{ltɕhɯɣltɕhɯɣ}{}\classe{idph.2}
\begin{définition}\fra long et fin, suspendu\end{définition}
\begin{définition}\cmn 形容细长,吊着的样子\end{définition}
\begin{exemple}\jya ʑmbri ɯ-jwaʁ ltɕhɯɣltɕhɯɣ ɲɯ-pa\cmn 杨树叶子在吊着\end{exemple}\begin{sous-entrée}
\vedette{\hypertarget{}{\papi{ ltɕhɯɣnɤlɯɣ}}}\markboth{ltɕhɯɣnɤlɯɣ}{}\classe{idph.4}
\begin{exemple}\jya tɤ-pɤtso kɯ ɯ-jaʁ tɤtar ltɕhɯɣnɤlɯɣ ɲɯ-ɤsɯ-stu\cmn 小孩子在摇动棍子\end{exemple}
\end{sous-entrée}\begin{sous-entrée}
\vedette{\hypertarget{}{\papi{ sɤltɕhɯɣlɯɣ}}}\markboth{sɤltɕhɯɣlɯɣ}{}\classe{vt}
\begin{exemple}\jya jla kɯ ɯ-jme ɲɯ-sɤltɕhɯɣlɯɣ\cmn 犏牛在摆动尾巴\end{exemple}
\end{sous-entrée}\end{entrée}

\begin{entrée}
\vedette{\hypertarget{Ⓔlthjɤlthjɤt}{\papi{ lthjɤlthjɤt}}}\markboth{lthjɤlthjɤt}{}\classe{idph.2}
\begin{définition}\fra propre et bien repassé\end{définition}
\begin{définition}\cmn 形容又干净又柔软又平整样子\end{définition}
\begin{exemple}\jya naŋʁɯ lthjɤlthjɤt ci ɲɯ-ŋu\cmn 衬衣很柔软\end{exemple}
\begin{exemple}\jya ɯ-ŋga ɯ-tɯ-ɕo kɯ lthjɤlthjɤt ʑo ɲɯ-pa\cmn 他的衣服又干净又柔软\end{exemple}
\begin{exemple}\jya tɕhemɤpɯ lthjɤlthjɤt to-nɯ-rɤmpɕɤr\cmn 女孩子打扮得很漂亮\end{exemple}\end{entrée}

\begin{entrée}
\vedette{\hypertarget{Ⓔlthɯlthɯɣ}{\papi{ lthɯlthɯɣ}}}\markboth{lthɯlthɯɣ}{}
\classe{idph.2}
\begin{définition}\fra moelleux, mou\end{définition}
\begin{définition}\cmn 形容又平整又柔软的感觉\end{définition}
\begin{exemple}\jya tshɤrtɯl lthɯlthɯɣ ɲɯ-ŋu\cmn 羔羊皮衣里面是毛茸茸的\end{exemple}
\begin{exemple}\jya tɯji pɯ-kɯ-jɤɣ ɣɯ sɤtɕha lthɯlthɯɣ ʑo pa\cmn 下完种的地很松软\end{exemple}\end{entrée}

\begin{entrée}
\vedette{\hypertarget{Ⓔlthɯmlthɯm}{\papi{ lthɯmlthɯm}}}\markboth{lthɯmlthɯm}{}\classe{idph.2}
\begin{définition}\fra faible\end{définition}
\begin{définition}\cmn 形容没有精神,软弱的样子\end{définition}
\begin{exemple}\jya ɲɯ-ngo rca ma, lthɯmlthɯm ʑo ɲɯ-rɤʑi\cmn 他病了,在那里很软弱,没有精神的样子\end{exemple}\begin{sous-entrée}
\vedette{\hypertarget{}{\papi{ lthɯmɯmi}}}\markboth{lthɯmɯmi}{}\classe{idph.7}
\begin{définition}\fra qui vient sans qu'on s'en rende compte\end{définition}
\begin{définition}\cmn 不知不觉地,产生了睡意(很舒服的感觉)\end{définition}
\begin{exemple}\jya a-ʑɯβ lthɯmɯmi ʑo pɯ-ɣe\cmn 我不知不觉地睡着了\end{exemple}
\end{sous-entrée}\end{entrée}

\begin{entrée}
\vedette{\hypertarget{Ⓔltshɤltshɤt}{\papi{ ltshɤltshɤt}}}\markboth{ltshɤltshɤt}{}
\classe{idph.2}
\begin{définition}\fra petit et faible\end{définition}
\begin{définition}\cmn 形容瘦而小的东西竖立着的样子\end{définition}\begin{sous-entrée}
\vedette{\hypertarget{}{\papi{ ltshɤnɤltshɤt}}}\markboth{ltshɤnɤltshɤt}{}
\begin{définition}\ 
\end{définition}
\begin{exemple}\jya tɤ-pɤtso ltshɤnɤltshɤt ʑo ɲɯ-ŋke\cmn 小孩子走路显得又小又弱\end{exemple}
\begin{relation-sémantique}\confer{
\hyperlink{Ⓔɣɤltshɤltshɤt}{\textit{ \papi{ɣɤltshɤltshɤt}}}
}\end{relation-sémantique}
\end{sous-entrée}\end{entrée}

\begin{entrée}
\vedette{\hypertarget{Ⓔlɯβ}{\papi{ lɯβ}}}\markboth{lɯβ}{}
\classe{vs}
\paradigme{\textit{dir :} \jya thɯ-}
\paradigme{\textit{dir :} \jya tɤ-}
\begin{définition}\fra être sombre\end{définition}
\begin{définition}\cmn 天阴\end{définition}
\begin{exemple}\jya tɯ-mɯ chɤ-lɯβ\cmn 天变阴了\end{exemple}
\begin{exemple}\jya tɯ-mɯ ɲɯ-lɯβ\cmn 天很阴\end{exemple}
\begin{relation-sémantique}\synonyme{
\hyperlink{Ⓔqanɯ}{\textit{ \papi{qanɯ}}}
}\end{relation-sémantique}\end{entrée}

\begin{entrée}
\vedette{\hypertarget{Ⓔlɯɣ}{\papi{ lɯɣ}}}\markboth{lɯɣ}{}
\classe{vi}\acception{1}
\paradigme{\textit{dir :} \jya \_}
\begin{définition}\fra se détacher\end{définition}
\begin{définition}\cmn 解脱;滑\end{définition}
\begin{exemple}\jya kutɕu kɤ-βraʁ-a ri, kɤ-lɯɣ\cmn 我在这里拴了一下,但是滑了\end{exemple}
\begin{exemple}\jya a@caidao pɯ-lɯɣ tɕe, a-jaʁ ta-xtsɯɣ\cmn 我的菜刀滑了一下,切到了我的手\end{exemple}
\begin{exemple}\jya khɯna nɯ-lɯɣ\cmn 狗脱链了\end{exemple}\acception{2}
\paradigme{\textit{dir :} \jya \_}
\begin{définition}\fra traverser\end{définition}
\begin{définition}\cmn 穿过\end{définition}
\begin{exemple}\jya ɯʑo sɯŋgɯ kɤ-lɯɣ\cmn 他穿过了森林\end{exemple}
\begin{relation-sémantique}\synonyme{
\hyperlink{Ⓔpjɤl}{\textit{ \papi{pjɤl}}}
}\end{relation-sémantique}\acception{3}
\paradigme{\textit{dir :} \jya tɤ-}
\begin{définition}\fra se produire (incendie)\end{définition}
\begin{définition}\cmn 着(火)\end{définition}
\begin{exemple}\jya ɣndʑɤβ to-lɯɣ\cmn 着火了\end{exemple}
\begin{relation-sémantique}\confer{
\hyperlink{Ⓔɕlɯɣ}{\textit{ \papi{ɕlɯɣ}}}
}\end{relation-sémantique}\begin{sous-entrée}
\vedette{\hypertarget{}{\papi{ sɯɣlɯɣ}}}\markboth{sɯɣlɯɣ}{}\classe{vt}
\paradigme{\textit{dir :} \jya nɯ-}
\begin{définition}\fra détacher\end{définition}
\begin{définition}\cmn 解开\end{définition}
\begin{exemple}\jya na-sɯɣlɯɣ\cmn 他(把线)解开了\end{exemple}
\begin{relation-sémantique}\synonyme{
\hyperlink{Ⓔsɯɕlɯɣ}{\textit{ \papi{sɯɕlɯɣ}}}
}\end{relation-sémantique}
\end{sous-entrée}\begin{sous-entrée}
\vedette{\hypertarget{}{\papi{ ʑɣɤsɯɣlɯɣ}}}\markboth{ʑɣɤsɯɣlɯɣ}{}\classe{vi}
\paradigme{\textit{dir :} \jya nɯ-}
\begin{définition}\ 
\begin{déclaration}\grammar{refl}\end{déclaration}
\begin{déclaration}\grammar{caus}\end{déclaration}\end{définition}
\begin{définition}\fra se détacher\end{définition}
\begin{définition}\cmn 解脱;逃脱\end{définition}
\begin{exemple}\jya khɯna ɲɤ-ʑɣɤsɯɣlɯɣ\cmn 狗脱链了\end{exemple}
\begin{exemple}\jya tɯrme kú-wɣ-ja ri ɲɤ-ʑɣɤsɯɣlɯɣ\cmn 他被关(进监狱),但是逃脱了\end{exemple}
\end{sous-entrée}\end{entrée}

\begin{entrée}
\vedette{\hypertarget{Ⓔlɯɣlu}{\papi{ lɯɣlu}}}\markboth{lɯɣlu}{}\classe{n}
\begin{définition}\fra année du mouton\end{définition}
\begin{définition}\cmn 羊年
\begin{déclaration} \étymologie{\papi{lug.lo}}\end{déclaration}\end{définition}
\end{entrée}

\begin{entrée}
\vedette{\hypertarget{Ⓔlɯɣmbrɯm}{\papi{ lɯɣmbrɯm}}}\markboth{lɯɣmbrɯm}{}\classe{n}
\begin{définition}\fra maladie du ventre\end{définition}
\begin{définition}\cmn 风丹
\begin{déclaration} \étymologie{\papi{lug.ⁿbrum}}\end{déclaration}\end{définition}
\begin{exemple}\jya ɲɯ-rɤʑa pha tɯ-phoŋbu ʑo kɯ-ɤmɯrmɯrmbat ʑo tɤ-ndɤr ɲɯ-ɬoʁ tɕe nɯ lɯɣmbrɯm rmi. tɯ-phoŋbu nɯ-aci cho nɯ-mɯɕtaʁ tɕe ɲɯ-ɬoʁ ŋu\cmn 全身痒,痘痘长得很密,叫作风丹。身体淋湿,发冷的情况下就会出现。\end{exemple}\end{entrée}

\begin{entrée}
\vedette{\hypertarget{Ⓔlɯlu}{\papi{ lɯlu}}}\markboth{lɯlu}{}\classe{n}
\begin{définition}\fra chat\end{définition}
\begin{définition}\cmn 猫\end{définition}
\begin{relation-sémantique}\confer{
\hyperlink{Ⓔlɯlɤmu}{\textit{ \papi{lɯlɤmu}}}
}\end{relation-sémantique}
\begin{relation-sémantique}\confer{
\hyperlink{Ⓔlɯlɤpɯ}{\textit{ \papi{lɯlɤpɯ}}}
}\end{relation-sémantique}\end{entrée}

\begin{entrée}
\vedette{\hypertarget{Ⓔlɯlɤmu}{\papi{ lɯlɤmu}}}\markboth{lɯlɤmu}{}\classe{n}
\begin{définition}\fra chatte\end{définition}
\begin{définition}\cmn 母猫\end{définition}\end{entrée}

\begin{entrée}
\vedette{\hypertarget{Ⓔlɯlɤpɯ}{\papi{ lɯlɤpɯ}}}\markboth{lɯlɤpɯ}{}\classe{n}
\begin{définition}\fra chaton\end{définition}
\begin{définition}\cmn 猫崽子\end{définition}
\begin{relation-sémantique}\confer{
\hyperlink{Ⓔlɯlu}{\textit{ \papi{lɯlu}}}
}\end{relation-sémantique}\end{entrée}

\begin{entrée}
\vedette{\hypertarget{Ⓔlɯlɤrgɤn}{\papi{ lɯlɤrgɤn}}}\markboth{lɯlɤrgɤn}{}\classe{n}
\begin{définition}\fra vieux chat\end{définition}
\begin{définition}\cmn 老猫\end{définition}
\begin{relation-sémantique}\confer{
\hyperlink{Ⓔlɯlu}{\textit{ \papi{lɯlu}}}
}\end{relation-sémantique}\end{entrée}

\begin{entrée}
\vedette{\hypertarget{Ⓔlɯlukɯra}{\papi{ lɯlukɯra}}}\markboth{lɯlukɯra}{}\classe{adv}
\begin{définition}\fra avidité insatiable\end{définition}
\begin{définition}\cmn 贪得无厌\end{définition}
\begin{exemple}\jya lɯlukɯra ma-tɤ-tɯ-βze\cmn 你不要贪得无厌\end{exemple}\end{entrée}

\begin{entrée}
\vedette{\hypertarget{Ⓔlɯrlɯr}{\papi{ lɯrlɯr}}}\markboth{lɯrlɯr}{}
\classe{idph.2}
\begin{définition}\fra petit objet rond (qui roule)\end{définition}
\begin{définition}\cmn 形容又小又圆的东西在滚(如皮球、洋芋、乒乓球等)的样子\end{définition}\begin{sous-entrée}
\vedette{\hypertarget{}{\papi{ ɣɤlɯrlɯr}}}\markboth{ɣɤlɯrlɯr}{}\classe{vi}
\begin{exemple}\jya staχpɯ ɲɯ-ɣɤlɯrlɯr chɯ-ndʐaβ ɲɯ-ŋu\cmn 绿豆滚下去\end{exemple}
\end{sous-entrée}\begin{sous-entrée}
\vedette{\hypertarget{}{\papi{ lɯrɯri}}}\markboth{lɯrɯri}{}
\begin{définition}\fra s'élever lentement (fumée, vapeur)\end{définition}
\begin{définition}\cmn 形容烟,蒸汽慢慢冒上来的样子\end{définition}
\begin{exemple}\jya tɤ-khɯ lɯrɯri tu-ɬoʁ ɲɯ-ŋu\cmn 烟慢慢地冒上来\end{exemple}
\begin{exemple}\jya smi ɲɯ-ɤsɯ-βlɯ tɕe, tɤ-khɯ lɯrɯri ta-tɕɤt\cmn 他烧火,令烟慢慢地冒上来\end{exemple}
\end{sous-entrée}\begin{sous-entrée}
\vedette{\hypertarget{}{\papi{ sɤlɯrlɯr}}}\markboth{sɤlɯrlɯr}{}\classe{vt}
\begin{exemple}\jya tɤ-sɤlɯrlɯr-a thɯ-tʂaβ-a\cmn 我令(球)滚下去了\end{exemple}
\begin{exemple}\jya tɤ-pɤtso kɯ ɯ-kɯmtɕhɯ ɲɯ-sɤlɯrlɯr tha-tʂaβ\cmn 小孩子把玩具滚了下去\end{exemple}
\end{sous-entrée}\end{entrée}

\begin{entrée}
\vedette{\hypertarget{Ⓔlɯski}{\papi{ lɯski}}}\markboth{lɯski}{}\classe{cnj}
\begin{définition}\fra bien sûr\end{définition}
\begin{définition}\cmn 当然\end{définition}
\end{entrée}

\begin{entrée}
\vedette{\hypertarget{Ⓔlɯtoʁ}{\papi{ lɯtoʁ}}}\markboth{lɯtoʁ}{}\classe{n}
\begin{définition}\fra récolte des plantes que l'on a semées et des plantes sauvage\end{définition}
\begin{définition}\cmn 庄稼和野草
\begin{déclaration} \étymologie{\papi{lo.tog}}\end{déclaration}\end{définition}
\end{entrée}

\begin{entrée}
\vedette{\hypertarget{Ⓔlɯxpa}{\papi{ lɯxpa}}}\markboth{lɯxpa}{}\classe{n}
\begin{définition}\fra berger (qui élève des moutons)\end{définition}
\begin{définition}\cmn 放羊的人
\begin{déclaration} \étymologie{\papi{lug.pa}}\end{déclaration}\end{définition}\end{entrée}

\begin{entrée}
\vedette{\hypertarget{Ⓔlɯz}{\papi{ lɯz}}}\markboth{lɯz}{}\classe{vi}
\begin{définition}\fra rester\end{définition}
\begin{définition}\cmn 留下\end{définition}
\begin{exemple}\jya nɤ-zda jɤ-anɯri-nɯ tɕe nɤʑo nɯ-lɯz\cmn 你的伙伴回去了,你留下吧\end{exemple}
\begin{exemple}\jya nɯ-lɯz-a\cmn 我留下了\end{exemple}
\begin{relation-sémantique}\confer{
\hyperlink{Ⓔsɯɣlɯz}{\textit{ \papi{sɯɣlɯz}}}
}\end{relation-sémantique}\end{entrée}

\begin{entrée}
\vedette{\hypertarget{Ⓔlɯzlɯz}{\papi{ lɯzlɯz}}}\markboth{lɯzlɯz}{}\classe{idph.2}
\begin{définition}\fra petit et immobile\end{définition}
\begin{définition}\cmn 形容很小的物体动也不动的样子\end{définition}
\begin{exemple}\jya tɤ-pɤtso tɯ-sta zɯ lɯzlɯz ʑo ɲɯ-nɯ-rŋgɯ\cmn 小孩子在床上动也不动地睡觉\end{exemple}\begin{sous-entrée}
\vedette{\hypertarget{}{\papi{ lɯznɤlɯz}}}\markboth{lɯznɤlɯz}{}
\begin{définition}\fra sans se presser\end{définition}
\begin{définition}\cmn 形容不慌不忙的的样子\end{définition}
\begin{exemple}\jya lɯznɤlɯz ʑo ɲɯ-rɤma\cmn 他不慌不忙地做事\end{exemple}
\end{sous-entrée}\end{entrée}

\begin{entrée}
\vedette{\hypertarget{Ⓔlwɤrlwɤr}{\papi{ lwɤrlwɤr}}}\markboth{lwɤrlwɤr}{}
\classe{idph.2}
\begin{définition}\fra énorme\end{définition}
\begin{définition}\cmn 形容大块的样子\end{définition}
\begin{exemple}\jya a-mthɯm kɯ-wxti lwɤrlwɤr ʑo na-βzu\cmn 肉给我切了一大块(他很款待我)\end{exemple}\begin{sous-entrée}
\vedette{\hypertarget{}{\papi{ lwɤrnɤlwɤr}}}\markboth{lwɤrnɤlwɤr}{}\classe{idph.3}
\end{sous-entrée}\end{entrée}

\begin{entrée}
\vedette{\hypertarget{Ⓔlwɤz}{\papi{ lwɤz}}}\markboth{lwɤz}{}
\classe{vi}
\paradigme{\textit{dir :} \jya tɤ-}
\begin{définition}\fra retomber malade\end{définition}
\begin{définition}\cmn 犯病\end{définition}
\begin{exemple}\jya a-kɯ-mŋɤm to-lwɤz\cmn 我犯病了\end{exemple}\end{entrée}

\begin{entrée}
\vedette{\hypertarget{Ⓔlwoʁ}{\papi{ lwoʁ}}}\markboth{lwoʁ}{}
\classe{vt}
\paradigme{\textit{dir :} \jya thɯ-}
\paradigme{\textit{dir :} \jya pɯ-}
\begin{définition}\fra asperger\end{définition}
\begin{définition}\cmn 泼水\end{définition}
\begin{exemple}\jya tɯ-ci pɯ-lwoʁ\cmn 把水倒掉吧\end{exemple}
\begin{exemple}\jya nɤki ɯ-ro nɯ thɯ-lwoʁ\cmn 你把剩下的那个倒掉吧\end{exemple}
\begin{exemple}\jya sɤlaŋphɤn ɯ-ŋgɯ tɯ-ci nɯ ra thɯ-lwoʁ-a\cmn 我把盆子里的水倒掉了\end{exemple}
\begin{exemple}\jya tɯ-mɯ ɲɯ-ɤsɯ-lwoʁ ʑo\cmn 下了倾盆大雨\end{exemple}
\begin{relation-sémantique}\confer{
\hyperlink{ⒺciⒽ1}{\textit{ \papi{ci}}}
}\end{relation-sémantique}\end{entrée}

\begin{entrée}
\vedette{\hypertarget{Ⓔlwɯlwɯɣ}{\papi{ lwɯlwɯɣ}}}\markboth{lwɯlwɯɣ}{}\classe{idph.2}
\begin{définition}\fra ébouriffé\end{définition}
\begin{définition}\cmn 形容凌乱而蓬松的样子\end{définition}
\begin{exemple}\jya tɤtɕɯ nɯ ɯ-kɤrme lwɯlwɯɣ ʑo ɲɯ-pa tɕe pjɯ́-wɣ-qrɤz ɲɯ-ra\cmn 那个男孩子的头发又长又乱、乱蓬蓬的,非得把它剃掉不可\end{exemple}\end{entrée}

\begin{entrée}
\vedette{\hypertarget{Ⓔlxɤβlxɤβ}{\papi{ lxɤβlxɤβ}}}\markboth{lxɤβlxɤβ}{}
\classe{idph.2}
\begin{définition}\fra épais (vêtements)\end{définition}
\begin{définition}\cmn 沉重;厚实(衣服)\end{définition}
\begin{exemple}\jya tɯ-ŋga kɯ-jɯ-jaʁ ʑo lxɤβlxɤβ tɤ-ŋga-t-a\cmn 我穿了很厚的衣服\end{exemple}\begin{sous-entrée}
\vedette{\hypertarget{}{\papi{ lxɤβnɤlxɤβ}}}\markboth{lxɤβnɤlxɤβ}{}\classe{idph.3}
\begin{exemple}\jya lxɤβnɤlxɤβ ɲɯ-ŋke\cmn 他穿了很重的衣服在走路\end{exemple}
\begin{relation-sémantique}\confer{
\hyperlink{Ⓔlɣɤβlɣɤβ}{\textit{ \papi{lɣɤβlɣɤβ}}}
}\end{relation-sémantique}
\end{sous-entrée}\end{entrée}

\begin{entrée}
\vedette{\hypertarget{Ⓔlxɯlxi}{\papi{ lxɯlxi}}}\markboth{lxɯlxi}{}
\classe{idph.2}
\begin{définition}\fra épais, lourd\end{définition}
\begin{définition}\cmn 形容厚实,笨重的样子\end{définition}
\begin{exemple}\jya kɯ-rʑi tsa ci lxɯlxi ɲɯ-ŋu\cmn 很笨重\end{exemple}
\begin{exemple}\jya tɯ-ŋga kɯ-jaʁ tsa ci lxɯlxi ɲɯ-ŋu\cmn 衣服很厚实\end{exemple}
\begin{exemple}\jya kɯ-khe ci lxɯlxi ɲɯ-tɯ-ŋu\cmn 你有点笨\end{exemple}
\begin{relation-sémantique}\confer{
\hyperlink{Ⓔlɣɤβlɣɤβ}{\textit{ \papi{lɣɤβlɣɤβ}}}
}\end{relation-sémantique}\end{entrée}

\newpage\caractère{ɬ}

\begin{entrée}
\vedette{\hypertarget{Ⓔɬa}{\papi{ ɬa}}}\markboth{ɬa}{}\classe{n}
\begin{définition}\fra bouddha\end{définition}
\begin{définition}\cmn 佛
\begin{déclaration} \étymologie{\papi{lha}}\end{déclaration}\end{définition}
\end{entrée}

\begin{entrée}
\vedette{\hypertarget{Ⓔɬarɯɣ}{\papi{ ɬarɯɣ}}}\markboth{ɬarɯɣ}{}\classe{n}
\begin{définition}\fra réincarnation d'un dieu\end{définition}
\begin{définition}\cmn 神的化身
\begin{déclaration} \étymologie{\papi{lha.rigs}}\end{déclaration}\end{définition}
\end{entrée}

\begin{entrée}
\vedette{\hypertarget{Ⓔɬasaŋga}{\papi{ ɬasaŋga}}}\markboth{ɬasaŋga}{}\classe{n}
\begin{définition}\fra habits du Tibet central\end{définition}
\begin{définition}\cmn 西藏服装\end{définition}
\begin{relation-sémantique}\confer{
\hyperlink{Ⓔtɯ-ŋga}{\textit{ \papi{tɯ-ŋga}}}
}\end{relation-sémantique}
\end{entrée}

\begin{entrée}
\vedette{\hypertarget{Ⓔɬaχpo}{\papi{ ɬaχpo}}}\markboth{ɬaχpo}{}
\classe{adv}
\begin{définition}\fra exprime une impulsion soudaine que l'on essaie de réprimer\end{définition}
\begin{définition}\cmn 干脆……算了(一时冲动)\end{définition}
\begin{exemple}\jya nɤʑo taʁndo maka mɯ́j-tɯ-tso tɕe, ɬaχpo ʑo mɯ-tu-ta-ʁndɯ\cmn 你不听话,干脆打你一顿算了\end{exemple}\end{entrée}

\begin{entrée}
\vedette{\hypertarget{Ⓔɬɤliaʁ}{\papi{ ɬɤliaʁ}}}\markboth{ɬɤliaʁ}{}\classe{n}
\begin{définition}\fra nom d'un village de Sarndzu\end{définition}
\begin{définition}\cmn 沙尔宗的一个村\end{définition}\end{entrée}

\begin{entrée}
\vedette{\hypertarget{Ⓔɬɤɬɤt}{\papi{ ɬɤɬɤt}}}\markboth{ɬɤɬɤt}{}
\classe{idph.2}
\begin{définition}\fra de bonne humeur\end{définition}
\begin{définition}\cmn 形容心情很舒畅的样子\end{définition}
\begin{exemple}\jya aʑo a-sɯm ɬɤɬɤt ʑo ɲɯ-pa\cmn 我心情很好\end{exemple}
\begin{exemple}\jya aʑo a-sɯm ɬɤɬɤt ɲɯ-nɯ-ste-a ɕti\cmn 我把心情放舒畅些\end{exemple}
\end{entrée}

\begin{entrée}
\vedette{\hypertarget{Ⓔɬɤndʐi}{\papi{ ɬɤndʐi}}}\markboth{ɬɤndʐi}{}
\classe{n}
\begin{définition}\fra démon\end{définition}
\begin{définition}\cmn 鬼
\begin{déclaration} \étymologie{\papi{lha.ⁿdre}}\end{déclaration}\end{définition}\end{entrée}

\begin{entrée}
\vedette{\hypertarget{Ⓔɬɤndʐismi}{\papi{ ɬɤndʐismi}}}\markboth{ɬɤndʐismi}{}\classe{n}
\begin{définition}\fra feu follet\end{définition}
\begin{définition}\cmn 磷火【鬼火】\end{définition}\end{entrée}

\begin{entrée}
\vedette{\hypertarget{Ⓔɬɤndʐitɤlɤtshaʁ}{\papi{ ɬɤndʐitɤlɤtshaʁ}}}\markboth{ɬɤndʐitɤlɤtshaʁ}{}\classe{n}
\begin{définition}\fra Delphinium sp.\end{définition}
\begin{définition}\cmn 翠雀花\end{définition}
\begin{exemple}\jya ɬɤndʐitɤlɤtshaʁ nɯ si kɯ-xtɕi, rɯŋgu, tɯ-ji rkɯ, xɕaj kɯ-dɤn ɯ-rchɤβ tu-ɬoʁ ŋu, ɯ-ru cho ɯ-jwaʁ pɣi tsa arɯlɯŋkɤr, ɯ-mɯntoʁ nɯ kɯ-ɤrŋi tɕe kɯ-ɤɲaʁndzɯm ʑo ŋu. ɯ-ru xtshɯm, xɕaj sɤz kɯ-dɤn mɤ-mbro, ɯ-mɯntoʁ ɯ-tshɯɣa nɯ tɤlɤtshaʁ fse tɕe núndʐa nɯ ɬɤndʐitɤlɤtshaʁ rmi\cmn 
翠雀花生长在小树丛、草坪、地边、茂盛的草丛中,茎和叶子是淡蓝色的,带有一点灰色。花是深蓝色的。茎很细,比一般的草长不出多少,花的形状像滤牛奶的漏斗,所以叫做\stylefv{ɬɤndʐitɤlɤtshaʁ}(鬼的漏斗)
\end{exemple}\end{entrée}

\begin{entrée}
\vedette{\hypertarget{Ⓔɬɤndʐitɤtsoʁ}{\papi{ ɬɤndʐitɤtsoʁ}}}\markboth{ɬɤndʐitɤtsoʁ}{}\classe{n}
\begin{définition}\fra une plante\end{définition}
\begin{définition}\cmn 植物的一种\end{définition}
\begin{exemple}\jya ɬɤndʐi tɤtsoʁ nɯ sɯjno ci ŋu. ɯ-ku nɯ tɤtsoʁ cho naχtɕɯɣ, ɬɤndʐi tɤtsoʁ ɯ-ku wxti. ɯ-ru me, ɯ-jwaʁ nɯ pɣɤ-muj ɯ-tshɯɣa kɯ-fse ŋu, ɯ-ri kɯ-fse kɯ-ɣɯrni ju-kɯ-ɕe ci tu. ɯ-jwaʁ nɯ pjɤ-ɕkho kɯ-fse ŋu tɕe, ɯ-χcɤl ri ɯ-mɯntoʁ ɲɯ-βze ŋu. ɯ-mɯntoʁ tɯ-rdoʁ ma me, kɯ-qarne ŋu. ɯ-qa nɯ tɤtsoʁ cho naχtɕɯɣ, tɕeri ɬɤndʐi tɤtsoʁ ɣɯ ɯ-qa nɯ ɯ-rme ci tu. ɯ-rme nɯ ɲaʁ, mɤ-dɤn ri rɲɟi. ɬɤndʐi tɤtsoʁ nɯ tú-wɣ-ndza tɕe, tɯ-mdʑu ɲɯ-sɤzɯβzɯβ ŋu, chɯ́-wɣ-mqlaʁ tɕe, tɯ-rqo kɤ-sɯɣ kɯ-fse ɲɯ-sɤβze ŋu tɕe, kɤ-ndza mɤ-sna. tɤtsoʁ nɯ tú-wɣ-ndza tɕe, chi, tɯ-mdʑu ɯ-kɯ-sɤzɯβzɯβ me, tɕe nɯ mɯm.\cmn 
\stylefv{ɬɤndʐi tɤtsoʁ}是一种草,苗和人参果的一样,但大一些。没有茎,叶子像鸟的羽毛的形状,叶柄上好像有红线。叶子向四面展开,中间开花。只有一朵花,是黄色的。根也和人参果一样,但上面有毛,不多但是很长。吃了使舌头发麻,吞下去,使喉咙有很紧的感觉,所以不能吃。人参果吃起来是甜的,舌头没有麻的感觉,好吃。
\end{exemple}
\end{entrée}

\begin{entrée}
\vedette{\hypertarget{Ⓔɬɤndʐithamaka}{\papi{ ɬɤndʐithamaka}}}\markboth{ɬɤndʐithamaka}{}\classe{n}
\begin{définition}\fra vesse-de-loup\end{définition}
\begin{définition}\cmn 马勃\end{définition}
\begin{relation-sémantique}\synonyme{
\hyperlink{Ⓔsalaboŋboŋ}{\textit{ \papi{salaboŋboŋ}}}
}\end{relation-sémantique}\end{entrée}

\begin{entrée}
\vedette{\hypertarget{Ⓔɬɤntshɤm}{\papi{ ɬɤntshɤm}}}\markboth{ɬɤntshɤm}{}\classe{n}
\begin{définition}\fra nakṣatra anurādhās\end{définition}
\begin{définition}\cmn 房宿
\begin{déclaration} \étymologie{\papi{lha.mtsʰams}}\end{déclaration}\end{définition}
\begin{exemple}\jya ɬɤntshɤm cho sla ni nɯ-atɯɣ-ndʑi tɕe, sla nɯ ɬɤntshɤm ɯ-rqoʁ zɯ a-pɯ-ɕe tɕe ɣɯjpa taχpa pe kɤ-ti ŋu, sla nɯ ɬɤntshɤm ɣɯ ɯ-jaʁmɤχa a-pɯ-ɕe tɕe, taχpa nɤkɤro kɤ-ti ŋu, ɯ-thɤcu a-pɯ-ɕe tɕe, taχpa mɤ-pe kɤ-ti ŋu\cmn 房宿和月亮相逢时,假如月亮到房宿的“手腕”以下,收成很好,假如月亮到房宿的“大拇指”和“食指”之间,收成中等,假如月亮再往下,收成不好。\end{exemple}\end{entrée}

\begin{entrée}
\vedette{\hypertarget{Ⓔɬɤt}{\papi{ ɬɤt}}}\markboth{ɬɤt}{}
\classe{vs}
\paradigme{\textit{dir :} \jya pɯ-}
\begin{définition}\fra vieillir, se dégrader\end{définition}
\begin{définition}\cmn 衰老\end{définition}
\begin{exemple}\jya tɯrnda pjɤ-ɬɤt\cmn 房子老化破损了\end{exemple}\end{entrée}

\begin{entrée}
\vedette{\hypertarget{ⒺɬoʁⒽ1}{\papi{ ɬoʁ}}}\markboth{ɬoʁ}{}\homonyme{1}\classe{vs}
\begin{définition}\fra devoir\end{définition}
\begin{définition}\cmn 必须\end{définition}
\begin{exemple}\jya nɯ tu-ste-a ɲɯ-ɬoʁ\cmn 我必须这样做\end{exemple}\end{entrée}

\begin{entrée}
\vedette{\hypertarget{ⒺɬoʁⒽ2}{\papi{ ɬoʁ}}}\markboth{ɬoʁ}{}\homonyme{2}\classe{vi}\acception{1}
\begin{définition}\fra sortir, partir\end{définition}
\begin{définition}\cmn 出来;发生\end{définition}
\begin{exemple}\jya tɤŋe tɤ-ɬoʁ\cmn 太阳升起了\end{exemple}
\begin{exemple}\jya @cai lɤ-ji-tɕi tɕe to-ɬoʁ\cmn 我们俩种了菜就长出来了\end{exemple}
\begin{exemple}\jya mbro ɯ-taʁ pɯ-ɬoʁ-a\cmn 我下了马\end{exemple}
\begin{exemple}\jya ɯ-kɯ-ɬoʁ mɯ́j-pe\cmn 产量不高\end{exemple}
\begin{exemple}\jya ɯ-re pjɤ-ɬoʁ\cmn 他笑了一声\end{exemple}\acception{2}
\begin{définition}\fra mettre bas (bovidé)\end{définition}
\begin{définition}\cmn 生崽子(牛类)\end{définition}
\begin{exemple}\jya nɯŋa ɲo-ɬoʁ (=chɤ-rɤpɯ)\cmn 奶牛生了崽子\end{exemple}\end{entrée}

\begin{entrée}
\vedette{\hypertarget{Ⓔɬɯɣnɤɬɯɣ}{\papi{ ɬɯɣnɤɬɯɣ}}}\markboth{ɬɯɣnɤɬɯɣ}{}
\classe{idph.3}
\begin{définition}\fra qui bouge\end{définition}
\begin{définition}\cmn 形容动物因为呼吸而发生的运动的样子\end{définition}
\begin{exemple}\jya paʁ mɯ-pjɤ-si ma ɯ-xtu ɬɯɣnɤɬɯɣ ʑo ɲɯ-pa\cmn 猪没有死,它肚子还在动(表示它在呼吸)\end{exemple}\end{entrée}

\newpage\caractère{m}

\begin{entrée}
\vedette{\hypertarget{ⒺmuⒽ2}{\papi{ mu}}}\markboth{mu}{}\homonyme{2}
\classe{adv}
\begin{définition}\fra pas du tout\end{définition}
\begin{définition}\cmn 根本没有\end{définition}
\end{entrée}

\begin{entrée}
\vedette{\hypertarget{ⒺmuⒽ1}{\papi{ mu}}}\markboth{mu}{}\homonyme{1}
\classe{vi}
\paradigme{\textit{dir :} \jya nɯ-}
\begin{définition}\fra avoir peur\end{définition}
\begin{définition}\cmn 害怕\end{définition}
\begin{exemple}\jya aʑo pɯ-mu-a ma ɲɯ-sɤɣmu\cmn 因为很可怕,我害怕了\end{exemple}
\begin{relation-sémantique}\confer{
\hyperlink{Ⓔnɯɣmu}{\textit{ \papi{nɯɣmu}}}
}\end{relation-sémantique}
\begin{relation-sémantique}\confer{
\hyperlink{Ⓔɕɯɣmu}{\textit{ \papi{ɕɯɣmu}}}
}\end{relation-sémantique}
\begin{relation-sémantique}\confer{
\hyperlink{Ⓔsɤɣmu}{\textit{ \papi{sɤɣmu}}}
}\end{relation-sémantique}\end{entrée}

\begin{entrée}
\vedette{\hypertarget{Ⓔma}{\papi{ ma}}}\markboth{ma}{} (\variante{mɯma}) \classe{postp}
\begin{définition}\fra à part\end{définition}
\begin{définition}\cmn 除了\end{définition}
\begin{exemple}\jya nɯ ma kɯ-tu me\cmn 没有其它的了\end{exemple}\end{entrée}

\begin{entrée}
\vedette{\hypertarget{Ⓔmacatsɯt}{\papi{ macatsɯt}}}\markboth{macatsɯt}{}\classe{n}
\begin{définition}\fra chaise en bambou\end{définition}
\begin{définition}\cmn 竹子编成的椅子\end{définition}
\end{entrée}

\begin{entrée}
\vedette{\hypertarget{Ⓔmahi}{\papi{ mahi}}}\markboth{mahi}{}\classe{n}
\begin{définition}\fra buffle\end{définition}
\begin{définition}\cmn 水牛\end{définition}\end{entrée}

\begin{entrée}
\vedette{\hypertarget{Ⓔmaka}{\papi{ maka}}}\markboth{maka}{}
\classe{adv}
\begin{définition}\fra pas du tout\end{définition}
\begin{définition}\cmn 根本
\begin{déclaration}\use{多用于否定句}\end{déclaration}\end{définition}
\begin{exemple}\jya maka ʑo pɯ-mto-t-a me / pɯ-mto-t-a maka me\cmn 我什么也没有听到\end{exemple}
\begin{exemple}\jya ta-tɯt maka kɯ-tu me\cmn 他什么也没有说\end{exemple}
\begin{exemple}\jya ɯ-kɤ-nɯfse maka ʑo me\cmn 他谁也不认识\end{exemple}\end{entrée}

\begin{entrée}
\vedette{\hypertarget{Ⓔmaldo}{\papi{ maldo}}}\markboth{maldo}{}
\classe{n}
\begin{définition}\fra érable\end{définition}
\begin{définition}\cmn 枫树\end{définition}
\begin{exemple}\jya maldo nɯ si wuma ʑo kɯ-mbro kɯ-wxti kɯ-jpum ci ŋu, ɯ-ru nɯ kɯ-pɣi tsa ci ŋu, ɯ-ru ɯ-taʁ nɯ ra ɯ-zbɤβ kɯ-fse tu, ɯ-rtaʁ dɤn, ɯ-jwaʁ wuma ʑo wxti, ɯ-βzɯr kɯmŋu tu, ɯ-si wuma ʑo ngɯt, mpɕɤr ma ɯ-rɯmu dɤn. ɯ-zrɤm nɯ rtazga ɯ-spa kɯ-pe ɲɯ-ŋu, ɯ-zbɤβ nɯ ra khɯtsa ɯ-spa tu-sɯ-βzu-nɯ ɲɯ-ŋgrɤl, ɯ-ru nɯ ŋgɤjpɤn chɯ-lɤt-nɯ tɕe, wuma ʑo kɯ-pe ɲɯ-ŋu, kha laχtɕha tɕhi kɯ-ra kɤ-βzu ɲɯ-sna.\cmn 枫树是长得又高、又大、又粗的树种,树干带有一点灰色,树干上有树瘤,枝桠多,叶子很大,有五个角,木质很结实,因为有很多花纹所以很美。根是作马鞍的好材料,树瘤用来作木碗。树干可以锯成板子,质量很好,是制作各种家具的好材料。\end{exemple}\end{entrée}

\begin{entrée}
\vedette{\hypertarget{Ⓔmaldzɯ}{\papi{ maldzɯ}}}\markboth{maldzɯ}{}
\classe{n}
\begin{définition}\fra une plante\end{définition}
\begin{définition}\cmn 植物的一种\end{définition}
\begin{exemple}\jya maldzɯ nɯ ruŋgu kɯ-mbro kɯ-ɣɤndʐo tsa tu-kɯ-ɬoʁ sɯjno ci ŋu, ɯ-jwaʁ ɯ-tshɯɣa nɯ ra qarɣɤpɤt cho naχtɕɯɣ, ɯ-jwaʁ nɯ ra ɯ-rme tu, ɯ-ru kɯ-xtshɯ-xtshɯm ŋu, kɯ-qandʐi tɕe ɯ-taʁ ɯ-rme kɯ-tu ŋu, ɯ-mɯntoʁ li ɯ-fkɯm nɯ li ɯ-rme tu, tɕe pɯ-ɴɢaʁ tɕe ɯ-ŋgɯ ɯ-mɯntoʁ ɲɤ-nɯɬoʁ ŋu. ɯ-mɯntoʁ nɯ-nɯɬoʁ ɕɯmɯma tɕe, aʁrɯrʁu tɕe ʑɯrɯʑɤri pjɯ-ɤstɤko tɕe kɯ-ɣɯrni ŋu, ɯ-mɯntoʁ χsɯ-mpɕar ma me, . kɯ-rɲɟi tɕe kɯ-ɤmtɕoʁ ŋu. ɯ-rɣi me.\cmn 
\stylefv{maldzɯ} 是生长在气候比较寒冷的高山上的一种草。叶子形状和鹿茸花的一样,叶子上有毛,茎很细,是乌色的,上面也有一点毛。花萼上也有毛,掉下了以后,里面就露出花来。花刚露出来时,是皱着的,逐渐伸展。花是红色的,只有三片花瓣,是长而尖的。没有种子。
\end{exemple}\end{entrée}

\begin{entrée}
\vedette{\hypertarget{Ⓔmaŋ}{\papi{ maŋ}}}\markboth{maŋ}{}\classe{vi}
\begin{définition}\fra beaucoup\end{définition}
\begin{définition}\cmn 很多
\begin{déclaration} \étymologie{\papi{maŋ}}\end{déclaration}\end{définition}
\begin{exemple}\jya ɯ-pjɤβlaʁ ɲɯ-maŋ\cmn 他想得多\end{exemple}\end{entrée}

\begin{entrée}
\vedette{\hypertarget{Ⓔmaŋdi}{\papi{ maŋdi}}}\markboth{maŋdi}{}\classe{vs}
\paradigme{\textit{dir :} \jya nɯ-}
\begin{définition}\fra être à l'ouest\end{définition}
\begin{définition}\cmn 在西方\end{définition}
\end{entrée}

\begin{entrée}
\vedette{\hypertarget{Ⓔmaŋe}{\papi{ maŋe}}}\markboth{maŋe}{}\classe{vi}
\begin{définition}\fra ne pas avoir (sensoriel)\end{définition}
\begin{définition}\cmn 没有(亲验)\end{définition}
\begin{exemple}\jya @shangge @xingqi @dianhua ɯ-kɯ-lɤt mataŋe\cmn 你上个星期没有打电话来\end{exemple}
\begin{exemple}\jya ɕɤxɕo kɤ-mtshɤm mataŋe\cmn 最近没有你的消息\end{exemple}
\begin{exemple}\jya pɯ-nɯ-tu pɯ-nɯ-me maŋe\cmn 可有可无\end{exemple}
\begin{exemple}\jya nɯ ma kɤ-pa maŋe\cmn 没有其它办法\end{exemple}
\begin{relation-sémantique}\confer{
\hyperlink{Ⓔɣɤʑu}{\textit{ \papi{ɣɤʑu}}}
}\end{relation-sémantique}\begin{forme-mot}2s : \papi{mataŋe}\end{forme-mot}\end{entrée}

\begin{entrée}
\vedette{\hypertarget{Ⓔmaŋkɯ}{\papi{ maŋkɯ}}}\markboth{maŋkɯ}{}\classe{vs}
\paradigme{\textit{dir :} \jya kɤ-}
\begin{définition}\fra être à l'est\end{définition}
\begin{définition}\cmn 在东方\end{définition}\end{entrée}

\begin{entrée}
\vedette{\hypertarget{Ⓔmaŋlo}{\papi{ maŋlo}}}\markboth{maŋlo}{}\classe{vs}
\paradigme{\textit{dir :} \jya lɤ-}
\begin{définition}\fra être en amont\end{définition}
\begin{définition}\cmn 在上游\end{définition}\begin{sous-entrée}
\vedette{\hypertarget{}{\papi{ ʑɣɤmaŋlo}}}\markboth{ʑɣɤmaŋlo}{}\classe{vi}
\paradigme{\textit{dir :} \jya lɤ-}
\begin{définition}\ 
\begin{déclaration}\grammar{refl}\end{déclaration}\end{définition}
\begin{définition}\fra se mettre en amont\end{définition}
\begin{définition}\cmn 到上游的地方\end{définition}
\end{sous-entrée}\end{entrée}

\begin{entrée}
\vedette{\hypertarget{Ⓔmaŋpa}{\papi{ maŋpa}}}\markboth{maŋpa}{}\classe{vs}
\paradigme{\textit{dir :} \jya pɯ-}
\begin{définition}\fra être en bas\end{définition}
\begin{définition}\cmn 在下面\end{définition}\end{entrée}

\begin{entrée}
\vedette{\hypertarget{Ⓔmaŋtaʁ}{\papi{ maŋtaʁ}}}\markboth{maŋtaʁ}{}\classe{vs}
\paradigme{\textit{dir :} \jya tɤ-}
\begin{définition}\fra être en haut\end{définition}
\begin{définition}\cmn 在上面\end{définition}
\end{entrée}

\begin{entrée}
\vedette{\hypertarget{Ⓔmaŋthi}{\papi{ maŋthi}}}\markboth{maŋthi}{}\classe{vs}
\paradigme{\textit{dir :} \jya thɯ-}
\begin{définition}\fra être en aval\end{définition}
\begin{définition}\cmn 在下游\end{définition}
\end{entrée}

\begin{entrée}
\vedette{\hypertarget{Ⓔmaqhu}{\papi{ maqhu}}}\markboth{maqhu}{}\classe{vs}
\paradigme{\textit{dir :} \jya nɯ-}\acception{1}
\begin{définition}\fra être tard\end{définition}
\begin{définition}\cmn 迟到\end{définition}\acception{2}
\begin{définition}\fra être après\end{définition}
\begin{définition}\cmn 以后\end{définition}\end{entrée}

\begin{entrée}
\vedette{\hypertarget{Ⓔmar}{\papi{ mar}}}\markboth{mar}{}\classe{vt}
\paradigme{\textit{dir :} \jya pɯ-}
\begin{définition}\fra enduire\end{définition}
\begin{définition}\cmn 涂;擦
\begin{déclaration}\use{擦脸,涂(酥油)}\end{déclaration}\end{définition}
\begin{exemple}\jya ɯʑo kɯ tɯ-ndʐi na-mar\cmn 他给皮子上油了\end{exemple}
\begin{exemple}\jya nɤ-ɕnaβ aʁɤndɯndɤt ma-nɯ-tɯ-mar ma ɲɯ-sɤjloʁ\cmn 你别到处擦鼻涕,很恶心\end{exemple}
\begin{exemple}\jya khɤndzo ɣɯ-ta tɤ-mda tɕe, ɲɯ́-wɣ-mar ra\cmn 用蒸笼的时候,要先擦一点油\end{exemple}
\begin{relation-sémantique}\confer{
 \papi{ta-ʁɟaz,mar}
}\end{relation-sémantique}\begin{sous-entrée}
\vedette{\hypertarget{}{\papi{ \_amar}}}\markboth{\_amar}{}\classe{vi}
\begin{définition}\ 
\begin{déclaration}\grammar{pass}\end{déclaration}\end{définition}
\begin{définition}\fra être enduit\end{définition}
\begin{définition}\cmn 被涂在……\end{définition}
\end{sous-entrée}\begin{sous-entrée}
\vedette{\hypertarget{}{\papi{ ʑɣɤmar}}}\markboth{ʑɣɤmar}{}\classe{vi}
\begin{définition}\ 
\begin{déclaration}\grammar{refl}\end{déclaration}\end{définition}
\begin{définition}\fra s'enduire\end{définition}
\begin{définition}\cmn 给自己涂\end{définition}
\end{sous-entrée}\end{entrée}

\begin{entrée}
\vedette{\hypertarget{Ⓔmarɲaŋ}{\papi{ marɲaŋ}}}\markboth{marɲaŋ}{}\classe{n}
\begin{définition}\fra beurre rance\end{définition}
\begin{définition}\cmn 陈酥油\end{définition}\end{entrée}

\begin{entrée}
\vedette{\hypertarget{Ⓔmarsɤr}{\papi{ marsɤr}}}\markboth{marsɤr}{}\classe{n}
\begin{définition}\fra beurre frais\end{définition}
\begin{définition}\cmn 新鲜酥油\end{définition}\end{entrée}

\begin{entrée}
\vedette{\hypertarget{Ⓔmarwɤr}{\papi{ marwɤr}}}\markboth{marwɤr}{}\classe{n}
\begin{définition}\fra boîte où l'on met le beurre\end{définition}
\begin{définition}\cmn 酥油盒\end{définition}
\end{entrée}

\begin{entrée}
\vedette{\hypertarget{ⒺmaʁⒽ2}{\papi{ maʁ}}}\markboth{maʁ}{}\homonyme{2}
\classe{n}
\begin{définition}\fra taille (chaussures)\end{définition}
\begin{définition}\cmn 码(鞋子)
\begin{déclaration} \étymologie{\papi{\stylefn{码}}}\end{déclaration}\end{définition}
\begin{exemple}\jya nɤ-xtsa nɯ thɤstɯ-maʁ tu-tɯ-ŋge ŋu?\cmn 你穿多少码的鞋子?\end{exemple}\end{entrée}

\begin{entrée}
\vedette{\hypertarget{ⒺmaʁⒽ1}{\papi{ maʁ}}}\markboth{maʁ}{}\homonyme{1}
\classe{vs}
\begin{définition}\fra ne pas être\end{définition}
\begin{définition}\cmn 不是\end{définition}
\begin{relation-sémantique}\antonyme{
\hyperlink{Ⓔŋu}{\textit{ \papi{ŋu}}}
}\end{relation-sémantique}
\begin{relation-sémantique}\confer{
\hyperlink{Ⓔrɯkɯmaʁ}{\textit{ \papi{rɯkɯmaʁ}}}
}\end{relation-sémantique}
\begin{relation-sémantique}\confer{
\hyperlink{Ⓔnɯkɯmaʁ}{\textit{ \papi{nɯkɯmaʁ}}}
}\end{relation-sémantique}
\begin{relation-sémantique}\confer{
\hyperlink{Ⓔnɤɣmaʁ}{\textit{ \papi{nɤɣmaʁ}}}
}\end{relation-sémantique}
\begin{relation-sémantique}\confer{
\hyperlink{Ⓔkɯmaʁ}{\textit{ \papi{kɯmaʁ}}}
}\end{relation-sémantique}
\begin{relation-sémantique}\confer{
\hyperlink{Ⓔznɤmaʁmaʁ}{\textit{ \papi{znɤmaʁmaʁ}}}
}\end{relation-sémantique}
\begin{relation-sémantique}\confer{
\hyperlink{Ⓔɣɤmaʁ}{\textit{ \papi{ɣɤmaʁ}}}
}\end{relation-sémantique}
\begin{sous-entrée}
\vedette{\hypertarget{}{\papi{ maʁ kɯ}}}\markboth{maʁ kɯ}{}\classe{cnj}
\begin{définition}\fra non seulement\end{définition}
\begin{définition}\cmn 不但……也……\end{définition}
\end{sous-entrée}\begin{sous-entrée}
\vedette{\hypertarget{}{\papi{ nɯ maʁ nɤ}}}\markboth{nɯ maʁ nɤ}{}\classe{cnj}
\begin{définition}\fra ou bien, sinon\end{définition}
\begin{définition}\cmn 不然;要么……要么\end{définition}
\end{sous-entrée}\begin{sous-entrée}
\vedette{\hypertarget{}{\papi{ pjɯsɤɣmaʁ me}}}\markboth{pjɯsɤɣmaʁ me}{}
\begin{définition}\fra il n'y a aucun doute\end{définition}
\begin{définition}\cmn 毫无疑问\end{définition}
\begin{exemple}\jya kɯ-mɯrkɯ ɯʑo ɕti ma pjɯ-sɤɣ-maʁ me\cmn 毫无疑问,他就是小偷\end{exemple}
\end{sous-entrée}\begin{sous-entrée}
\vedette{\hypertarget{}{\papi{ tɕhi kɯ-fse ci kɯnɤ}}}\markboth{tɕhi kɯ-fse ci kɯnɤ}{}\classe{cnj}
\begin{définition}\fra au moins\end{définition}
\begin{définition}\cmn 至少;起码\end{définition}
\end{sous-entrée}\begin{sous-entrée}
\vedette{\hypertarget{}{\papi{ tɕhi maʁ nɤ}}}\markboth{tɕhi maʁ nɤ}{}\classe{cnj}
\begin{définition}\fra au moins\end{définition}
\begin{définition}\cmn 至少\end{définition}
\end{sous-entrée}\end{entrée}

\begin{entrée}
\vedette{\hypertarget{Ⓔmasɤmdɤla}{\papi{ masɤmdɤla}}}\markboth{masɤmdɤla}{}\classe{adv}
\begin{définition}\fra en avance\end{définition}
\begin{définition}\cmn 提前\end{définition}
\begin{exemple}\jya nɤki tɤ-rɟit nɯ masɤmdɤla ʑo to-ŋke\cmn 那个小孩子提前会走路了\end{exemple}
\begin{exemple}\jya jisŋi masɤmdɤla ʑo tɤ-nɯsaχsɯ-j\cmn 我们今天提前吃了午餐\end{exemple}
\begin{relation-sémantique}\confer{
\hyperlink{Ⓔmda}{\textit{ \papi{mda}}}
}\end{relation-sémantique}\end{entrée}

\begin{entrée}
\vedette{\hypertarget{Ⓔmasɤrɯrju}{\papi{ masɤrɯrju}}}\markboth{masɤrɯrju}{}\classe{adv}
\begin{définition}\fra en cachette\end{définition}
\begin{définition}\cmn 悄悄的\end{définition}
\begin{relation-sémantique}\confer{
\hyperlink{Ⓔarju}{\textit{ \papi{arju}}}
}\end{relation-sémantique}\end{entrée}

\begin{entrée}
\vedette{\hypertarget{Ⓔmatɕi}{\papi{ matɕi}}}\markboth{matɕi}{}\classe{cnj}
\begin{définition}\fra sinon, parce que\end{définition}
\begin{définition}\cmn 不然,因为\end{définition}
\begin{exemple}\jya nɤ-ŋga nɯ-nɯ-tɕɤt matɕi ɲɯ-ɣɯtshɤdɯɣ nɤ\cmn 你脱下衣服,不然很热\end{exemple}
\begin{exemple}\jya mɤʑɯ nɤ-kɯ tsa kɤ-cit matɕi nɤ-ndi nɯ mɯ́j-xtɕhɯt nɤ\cmn 你往左边站一点,不然你右边的那个人站不下(坐不下)\end{exemple}\end{entrée}

\begin{entrée}
\vedette{\hypertarget{Ⓔmaχpɯn}{\papi{ maχpɯn}}}\markboth{maχpɯn}{}\classe{n}
\begin{définition}\fra stratège, général,\end{définition}
\begin{définition}\cmn 军师;将军
\begin{déclaration} \étymologie{\papi{dmag.dpon}}\end{déclaration}\end{définition}
\begin{exemple}\jya maχpɯn lo-ndo\cmn 他当了军师\end{exemple}\end{entrée}

\begin{entrée}
\vedette{\hypertarget{Ⓔmaχtɕɯ}{\papi{ maχtɕɯ}}}\markboth{maχtɕɯ}{}
\classe{intj}
\begin{définition}\fra je te l'avais bien dit\end{définition}
\begin{définition}\cmn 本来应该这样
\begin{déclaration}\use{责备别人的语气}\end{déclaration}\end{définition}
\begin{exemple}\jya maχtɕɯ tɤ-tɯt-a nɯ mɤ-tɯ-ste kɯ\cmn 你怎么没有照我说地去做呢?\end{exemple}
\begin{exemple}\jya maχtɕɯ ma-jɤ-tɯ-ɕe tɤ-tɯt-a ri mɯ́j-tɯ-khɯ tɕe\cmn 我本来叫你不要去,但是(你)没有听(果然出了问题)\end{exemple}\end{entrée}

\begin{entrée}
\vedette{\hypertarget{Ⓔmɤɕi}{\papi{ mɤɕi}}}\markboth{mɤɕi}{}\classe{vs}
\paradigme{\textit{dir :} \jya thɯ-}
\begin{définition}\fra riche\end{définition}
\begin{définition}\cmn 富有\end{définition}
\begin{exemple}\jya jiɕqha nɯ kɯ-fse kɯ-mɤɕi me\cmn 没有人比他有钱\end{exemple}
\begin{exemple}\jya mbroχpa thɯ-mɤɕi rɯʁgiwa, roŋwa thɯ-mɤɕi rɯkhɤrlɤn, kupa thɯ-mɤɕi rɯstɯnmɯ\cmn 牧民富有了就请人念经,农民富有了就修房子,汉族富有了就结婚\end{exemple}\end{entrée}

\begin{entrée}
\vedette{\hypertarget{Ⓔmɤɕtʂa}{\papi{ mɤɕtʂa}}}\markboth{mɤɕtʂa}{}
\classe{adv}\acception{1}
\begin{définition}\fra jusqu'à\end{définition}
\begin{définition}\cmn 一直到\end{définition}
\begin{exemple}\jya lɤsɤr ɯ-qhu mɤɕtʂa a-ʁa me ɲɯ-ŋu\cmn 一直到新年以后我都没有空\end{exemple}
\begin{exemple}\jya nɯ mɤɕtʂa nɯ kɤ-ti mɯ-pɯ-mtsha-ma\cmn 我从来没有听过(别人这样)说\end{exemple}\acception{2}
\begin{définition}\fra sinon\end{définition}
\begin{définition}\cmn 不然\end{définition}
\begin{exemple}\jya nɯ mɤɕtʂa aʑo mɤ-ɣi-a\cmn 不然我是不会来的\end{exemple}\end{entrée}

\begin{entrée}
\vedette{\hypertarget{Ⓔmɤdɤmɲɤm}{\papi{ mɤdɤmɲɤm}}}\markboth{mɤdɤmɲɤm}{}\classe{n}
\begin{définition}\fra une espèce d'arbrisseau\end{définition}
\begin{définition}\cmn 灌木的一种\end{définition}
\begin{exemple}\jya mɤdɤmɲɤm nɯ si kɯ-mbro tsa ci ŋu, ɯ-ru nɯ ra kɯ-pɣi ci ŋu, ɯ-jwaʁ ndɯβ ri dɤn, arŋi, ɯ-si nɯ mɤ-jpum, kɤ-ntɕhoz mɤ-sna ma ndoʁ tɕe mɤ-ngɯt, kɤ-nɯ-βlɯ ma mɤ-sna, ɯ-mɯntoʁ kɯ-ɤɣɯrnɯɕɯr ɲɯ-lɤt tɕe, ɯ-rtaʁ ɯ-kɤχcɤl zɯ kɯ-ndɯ-ndɯβ kɯ-dɯ-dɤn tɯtɯrca ku-ndzoʁ ŋu.\cmn 
\stylefv{mɤdɤmɲɤm}是一种长的比较高的树种,树干是灰色的,叶子小而多,是绿色的,树干不粗,不能使用因为脆,不结实。只能用来烧火。开淡红色的花,在树枝的顶端一朵朵地开在一起。
\end{exemple}\end{entrée}

\begin{entrée}
\vedette{\hypertarget{Ⓔmɤ́ɣrɤz}{\papi{ mɤ́ɣrɤz}}}\markboth{mɤ́ɣrɤz}{}\classe{cnj}
\begin{définition}\fra de toute manière, en revanche\end{définition}
\begin{définition}\cmn 反而\end{définition}\end{entrée}

\begin{entrée}
\vedette{\hypertarget{Ⓔmɤku}{\papi{ mɤku}}}\markboth{mɤku}{}\classe{vs}
\paradigme{\textit{dir :} \jya \_}
\begin{définition}\fra être avant\end{définition}
\begin{définition}\cmn 以前;在前面\end{définition}
\begin{exemple}\jya lɤ-mɤku-a\cmn 我在前面\end{exemple}
\begin{exemple}\jya kɯ-nɯsaχsɯ ju-ɣi-j pɯ-ŋu tɕe aʑo ɲɯ-mɤku-a pɯ-ŋu\cmn 我们来吃中午餐的时候,我走在前面\end{exemple}\begin{sous-entrée}
\vedette{\hypertarget{}{\papi{ zmɤku}}}\markboth{zmɤku}{}\classe{vt}
\paradigme{\textit{dir :} \jya \_}
\begin{définition}\fra faire avant\end{définition}
\begin{définition}\cmn 先做\end{définition}
\begin{exemple}\jya a-tʂha ci pɯ-zmɤke pɯ-rke\cmn 你先给我倒茶\end{exemple}
\begin{exemple}\jya nɤʑo ɯ-ɲɯ́-tɯ-mbɣom nɤ nɤ-@chepiao kɤ-χtɯ tu-ta-zmɤku jɤɣ\cmn 你如果急的话,我可以让你先买票\end{exemple}
\begin{relation-sémantique}\confer{
\hyperlink{Ⓔtɯ-ku}{\textit{ \papi{tɯ-ku}}}
}\end{relation-sémantique}
\end{sous-entrée}\end{entrée}

\begin{entrée}
\vedette{\hypertarget{Ⓔmɤkɯftshi}{\papi{ mɤkɯftshi}}}\markboth{mɤkɯftshi}{}\classe{adv}
\begin{définition}\fra forcer\end{définition}
\begin{définition}\cmn 逼迫\end{définition}
\begin{exemple}\jya mɤkɯftshi tú-wɣ-sɯ-ndza-a pɯ-ɕti\cmn 他逼我吃\end{exemple}
\begin{exemple}\jya mɯ-pɯ-kɯ-nɯ-cha kɯnɤ, mɤkɯftshi tú-wɣ-znɤma kɯ-ra ɕti\cmn 虽然不能做,但是还是被迫做\end{exemple}
\begin{exemple}\jya tɤ-fka-a ɕti ri, mɤkɯftshi ʑo tɤ-ndza-t-a pɯ-ra\cmn 我饱了,但还是被迫吃了\end{exemple}
\begin{relation-sémantique}\confer{
\hyperlink{Ⓔsɯftshi}{\textit{ \papi{sɯftshi}}}
}\end{relation-sémantique}\end{entrée}

\begin{entrée}
\vedette{\hypertarget{Ⓔmɤlɤn}{\papi{ mɤlɤn}}}\markboth{mɤlɤn}{}\classe{n}
\begin{définition}\fra absolument\end{définition}
\begin{définition}\cmn 一定;必须\end{définition}
\end{entrée}

\begin{entrée}
\vedette{\hypertarget{Ⓔmɤlmɤl}{\papi{ mɤlmɤl}}}\markboth{mɤlmɤl}{}\classe{idph.2}
\begin{définition}\fra très meuble (terre)\end{définition}
\begin{définition}\cmn 形容土地松软的样子\end{définition}
\begin{exemple}\jya tɯji mɤlmɤl ʑo cho-sthɯt-nɯ\cmn 他们下种完了,田地很松软\end{exemple}
\begin{exemple}\jya tɯji pɯ-jɤɣ tɕe, tɯji ra mɤlmɤl ʑo ɲɯ-pa\cmn 下种完了的时候,整个地面又松软又平整\end{exemple}\end{entrée}

\begin{entrée}
\vedette{\hypertarget{Ⓔmɤlɯm}{\papi{ mɤlɯm}}}\markboth{mɤlɯm}{}\classe{vs}
\paradigme{\textit{dir :} \jya tɤ-}
\paradigme{\textit{dir :} \jya thɯ-}
\begin{définition}\fra aux dimensions importantes\end{définition}
\begin{définition}\cmn 体积大\end{définition}
\begin{exemple}\jya ɯʑo ʁnɯ-pɤrme chɤ-zɣɯt tɕe chɤ-mɯlɯm\cmn 他到两岁,变大了\end{exemple}
\begin{relation-sémantique}\confer{
\hyperlink{Ⓔtɯ-lɯm}{\textit{ \papi{tɯ-lɯm}}}
}\end{relation-sémantique}\end{entrée}

\begin{entrée}
\vedette{\hypertarget{Ⓔmɤmu}{\papi{ mɤmu}}}\markboth{mɤmu}{}\classe{vi}
\paradigme{\textit{dir :} \jya lɤ-}
\begin{définition}\fra être la part la plus importante\end{définition}
\begin{définition}\cmn 是主要的;占多数\end{définition}
\begin{exemple}\jya kɯre kɤ-rɤma nɯ jiʑora kɤ-mɤmu ɬoʁ\cmn 这个工作的重点部分要我们做\end{exemple}
\begin{exemple}\jya jiʑo ji-tɯrme nɯra kɤ-mɤmu ɬoʁ\cmn 要以我们的人为重点\end{exemple}
\begin{exemple}\jya a-tɤ-rʑaʁ kɯ-mɤmu nɯ nɯre pɯ-ari ɕti\cmn 我的时间主要花在那一方面\end{exemple}
\begin{exemple}\jya tɯ-ci kɤ-tshi a-kɤ-mɤmu ra\cmn 你主要还是喝水(不只要吃药)\end{exemple}\begin{sous-entrée}
\vedette{\hypertarget{}{\papi{ zmɤmu}}}\markboth{zmɤmu}{}\classe{vt}
\paradigme{\textit{dir :} \jya tɤ-}
\begin{définition}\fra considérer comme le plus important\end{définition}
\begin{définition}\cmn 认为是最重要的\end{définition}
\begin{exemple}\jya kutɕu jɤ-tɯ-ɣe tɕe, nɤ-kɤnɤma nɯ kɤ-zmɤmu ra\cmn 你既然来到这里,要把工作看成是最重要\end{exemple}
\begin{exemple}\jya ɯʑo kɯ ɯʑo ɯ-ma ntsɯ tu-znɤme ɲɯ-ɕti, tɯʑo tɯma ra kɤ-nɤma mɯ́j-ŋgrɯ\cmn 我总是把自己的工作当做是最重要的事情,我们这边的事情就做不成\end{exemple}
\end{sous-entrée}\begin{sous-entrée}
\vedette{\hypertarget{}{\papi{ ʑɣɤmɤmu}}}\markboth{ʑɣɤmɤmu}{}
\paradigme{\textit{dir :} \jya tɤ-}
\begin{définition}\ 
\begin{déclaration}\grammar{refl}\end{déclaration}\end{définition}
\begin{définition}\fra se proposer spontanément pour prendre en charge\end{définition}
\begin{définition}\cmn 自己占主要地位;自我推荐\end{définition}
\begin{exemple}\jya jiɕqha nɯ-phe tɯ-rju kɤ-βzu ɲɯ-ra ri, aʑo tɤ-ʑɣɤmɤmu-a\cmn 要跟他们说,主要出面的是我(主要的话是我说的)\end{exemple}
\begin{exemple}\jya jiɕqha nɯ-phe tɯ-rju kɤ-βzu ɲɯ-ra ri, aʑo mɤ-ʑɣɤmɤmu-a\cmn 要跟他们说,但是我不会主要出面(婉转的意思:那句话不好说)\end{exemple}
\end{sous-entrée}\end{entrée}

\begin{entrée}
\vedette{\hypertarget{Ⓔmɤmbrɯmɤmbrɤt}{\papi{ mɤmbrɯmɤmbrɤt}}}\markboth{mɤmbrɯmɤmbrɤt}{}
\classe{adv}
\begin{définition}\fra par intermittence\end{définition}
\begin{définition}\cmn 断断续续\end{définition}
\begin{relation-sémantique}\confer{
\hyperlink{Ⓔmbrɤt}{\textit{ \papi{mbrɤt}}}
}\end{relation-sémantique}\end{entrée}

\begin{entrée}
\vedette{\hypertarget{Ⓔmɤmbɯr}{\papi{ mɤmbɯr}}}\markboth{mɤmbɯr}{}
\classe{vs}
\paradigme{\textit{dir :} \jya \_}
\begin{définition}\fra saillant\end{définition}
\begin{définition}\cmn 凸
\begin{déclaration} \étymologie{\papi{ⁿbur}}\end{déclaration}\end{définition}
\begin{exemple}\jya znde ɲɯ-mɤmbɯr tɕe, mbɯt ɲɯ-ŋu\cmn 墙有个凸出的地方,快要塌下来了\end{exemple}
\begin{exemple}\jya pjɤ-mɤmbɯr\cmn 地上凸出来了\end{exemple}\end{entrée}

\begin{entrée}
\vedette{\hypertarget{Ⓔmɤnɯɕaŋ}{\papi{ mɤnɯɕaŋ}}}\markboth{mɤnɯɕaŋ}{}\classe{adv}
\begin{définition}\fra je ne sais pas (expression toute faite, emprunt au situ)\end{définition}
\begin{définition}\cmn 我不知道(四土话借词)\end{définition}
\end{entrée}

\begin{entrée}
\vedette{\hypertarget{Ⓔmɤŋgɯ}{\papi{ mɤŋgɯ}}}\markboth{mɤŋgɯ}{}
\classe{vi}
\begin{définition}\fra être à l'intérieur\end{définition}
\begin{définition}\cmn 在里面\end{définition}
\begin{exemple}\jya ɯʑo tɯrme kɯ-mɤŋgɯ ɕti tɕe ɯ-rŋa tu.\cmn 他是重要的人物,他面子很大\end{exemple}
\begin{exemple}\jya stu kɯ-mɤŋgɯ\cmn 最里层\end{exemple}
\begin{relation-sémantique}\antonyme{
\hyperlink{Ⓔmɤpɕi}{\textit{ \papi{mɤpɕi}}}
}\end{relation-sémantique}
\begin{relation-sémantique}\confer{
\hyperlink{Ⓔɯ-ŋgɯ}{\textit{ \papi{ɯ-ŋgɯ}}}
}\end{relation-sémantique}\begin{sous-entrée}
\vedette{\hypertarget{}{\papi{ zmɤŋgɯ}}}\markboth{zmɤŋgɯ}{}\classe{vt}
\paradigme{\textit{dir :} \jya thɯ-}
\begin{définition}\fra porter à l'intérieur\end{définition}
\begin{définition}\cmn 穿在里面\end{définition}
\begin{exemple}\jya kɯki tɯ-ŋga ki chɯ́-wɣ-z-mɤŋgɯ ɲɯ-ra\cmn 这件衣服要穿在里面\end{exemple}
\begin{exemple}\jya ki tɯ-ŋga ki chɯ́-wɣ-z-mɤŋgɯ tɕe sɤscit\cmn 这件衣服穿在里面舒服\end{exemple}
\end{sous-entrée}\end{entrée}

\begin{entrée}
\vedette{\hypertarget{Ⓔmɤŋi}{\papi{ mɤŋi}}}\markboth{mɤŋi}{}\classe{n}
\begin{définition}\ 
\begin{déclaration}\grammar{n.lieu}\end{déclaration}\end{définition}
\begin{définition}\fra Mangi (village de Gdongbrgyad)\end{définition}
\begin{définition}\cmn 蒙岩村\end{définition}
\end{entrée}

\begin{entrée}
\vedette{\hypertarget{Ⓔmɤpaχcɤl}{\papi{ mɤpaχcɤl}}}\markboth{mɤpaχcɤl}{}\classe{vs}
\paradigme{\textit{dir :} \jya tɤ-}
\begin{définition}\fra être au centre\end{définition}
\begin{définition}\cmn 在中间\end{définition}
\begin{exemple}\jya ɯʑo to-mɤpaχcɤl\cmn 他站在中间了\end{exemple}
\begin{relation-sémantique}\confer{
\hyperlink{Ⓔmɤχcɤl}{\textit{ \papi{mɤχcɤl}}}
}\end{relation-sémantique}\end{entrée}

\begin{entrée}
\vedette{\hypertarget{Ⓔmɤpɤrthɤβ}{\papi{ mɤpɤrthɤβ}}}\markboth{mɤpɤrthɤβ}{}\classe{vi}
\paradigme{\textit{dir :} \jya tɤ-}
\begin{définition}\fra être entre deux\end{définition}
\begin{définition}\cmn 在两个的中间\end{définition}
\begin{exemple}\jya tɯrme χsɯm a-pɯ-tu-j, ɯ-χcɤl nɯ kɯ-mɤpɤrthɤβ\cmn 我们有三个人,中间的那个在(我们俩)中间\end{exemple}
\begin{exemple}\jya kɯ-mɤku nɯ tshɯraŋ ɲɯ-ŋu, kɯ-maqhu nɯ lɤβzaŋ ɲɯ-ŋu, waŋtɕin ɲɯ-mɤpɤrthɤβ\cmn 前面的是次让,后面的是罗桑,王金在中间\end{exemple}
\begin{relation-sémantique}\confer{
\hyperlink{Ⓔɯ-pɤrthɤβ}{\textit{ \papi{ɯ-pɤrthɤβ}}}
}\end{relation-sémantique}\begin{sous-entrée}
\vedette{\hypertarget{}{\papi{ ʑɣɤmɤpɤrthɤβ}}}\markboth{ʑɣɤmɤpɤrthɤβ}{}\classe{vi}
\paradigme{\textit{dir :} \jya tɤ-}
\begin{définition}\ 
\begin{déclaration}\grammar{refl}\end{déclaration}\end{définition}
\begin{définition}\fra se mettre au milieu\end{définition}
\begin{définition}\cmn 走到中间\end{définition}
\begin{exemple}\jya to-ʑɣɤmɤpɤrthɤβ\cmn 他走到中间了\end{exemple}
\end{sous-entrée}\end{entrée}

\begin{entrée}
\vedette{\hypertarget{Ⓔmɤpɕi}{\papi{ mɤpɕi}}}\markboth{mɤpɕi}{}
\classe{vi}
\begin{définition}\ 
\begin{déclaration}\grammar{denom}\end{déclaration}\end{définition}
\begin{définition}\fra se trouver à l'extérieur\end{définition}
\begin{définition}\cmn 在外面\end{définition}
\begin{exemple}\jya stu kɯ-mɤpɕi\cmn 最外层的\end{exemple}
\begin{relation-sémantique}\confer{
\hyperlink{Ⓔɯ-pɕi}{\textit{ \papi{ɯ-pɕi}}}
}\end{relation-sémantique}\begin{sous-entrée}
\vedette{\hypertarget{}{\papi{ zmɤpɕi}}}\markboth{zmɤpɕi}{}\classe{vt}\acception{1}
\paradigme{\textit{dir :} \jya thɯ-}
\begin{définition}\fra porter à l'extérieur\end{définition}
\begin{définition}\cmn 穿在外面\end{définition}
\begin{exemple}\jya ki tɯ-ŋga ki chɯ́-wɣ-z-mɤpɕi tɕe mpɕɤr\cmn 这件衣服穿在外面就美观\end{exemple}\acception{2}
\begin{définition}\fra considérer comme un étranger\end{définition}
\begin{définition}\cmn 当外人
\end{définition}
\begin{exemple}\jya nɯ́-wɣ-zmɤpɕi-a-nɯ\cmn 他们把我当外人\end{exemple}
\begin{relation-sémantique}\antonyme{
\hyperlink{Ⓔmɤŋgɯ}{\textit{ \papi{mɤŋgɯ}}}
}\end{relation-sémantique}
\begin{relation-sémantique}\confer{
\hyperlink{Ⓔɯ-pɕi}{\textit{ \papi{ɯ-pɕi}}}
}\end{relation-sémantique}
\end{sous-entrée}\end{entrée}

\begin{entrée}
\vedette{\hypertarget{Ⓔmɤpɕoʁ}{\papi{ mɤpɕoʁ}}}\markboth{mɤpɕoʁ}{}\classe{n}
\begin{définition}\fra l'envers, l'autre côté\end{définition}
\begin{définition}\cmn 反面
\begin{déclaration} \étymologie{\papi{ma.pʰʲogs}}\end{déclaration}\end{définition}
\end{entrée}

\begin{entrée}
\vedette{\hypertarget{Ⓔmɤrdɤli}{\papi{ mɤrdɤli}}}\markboth{mɤrdɤli}{}\classe{n}
\begin{définition}\fra personne sans foi ni loi\end{définition}
\begin{définition}\cmn 无法无天的人\end{définition}
\begin{exemple}\jya mɤrdɤli ɲɤ-ɕe qhe, nɯ ɯ-qhu kɤ-ndzɯmbra mɤ-nɤjtshɯ\cmn 他变得无法无天,从此以后,再教育也没有用\end{exemple}\end{entrée}

\begin{entrée}
\vedette{\hypertarget{Ⓔmɤrdom}{\papi{ mɤrdom}}}\markboth{mɤrdom}{}
\classe{n}
\begin{définition}\fra fléau\end{définition}
\begin{définition}\cmn 连枷\end{définition}
\begin{exemple}\jya mɤrdom ɯ-mu\cmn 连枷的把手\end{exemple}
\begin{exemple}\jya mɤrdom-mɲa\cmn 连枷打粮食的部分\end{exemple}\end{entrée}

\begin{entrée}
\vedette{\hypertarget{Ⓔmɤrnɤsɤŋo}{\papi{ mɤrnɤsɤŋo}}}\markboth{mɤrnɤsɤŋo}{}
\classe{n}
\begin{définition}\fra mal comprendre une parole\end{définition}
\begin{définition}\cmn 听错\end{définition}
\begin{exemple}\jya ɯʑo kɯ ɲɯ-ti ri, mɤrnɤsɤŋo ɲɤ-βzu-t-a\cmn 我听错了他讲的话\end{exemple}
\begin{relation-sémantique}\confer{
\hyperlink{Ⓔtɯ-rna}{\textit{ \papi{tɯ-rna}}}
}\end{relation-sémantique}
\begin{relation-sémantique}\confer{
 \papi{sɤŋo1}
}\end{relation-sémantique}\end{entrée}

\begin{entrée}
\vedette{\hypertarget{Ⓔmɤro}{\papi{ mɤro}}}\markboth{mɤro}{}
\classe{n}
\begin{définition}\fra endroit sur lequel on fait sécher la nourriture\end{définition}
\begin{définition}\cmn 粮架\end{définition}
\begin{exemple}\jya mɤro nɯ ɕoŋtɕa mɤ-kɯ-jpum tsa pjɯ́-wɣ-phɯt tɕe, choʁe chɯ́-wɣ-βʑoʁ tɕe ɲɯ́-wɣ-sɤɕpɯɕpa tɕe kɯ-spoʁ χsɯm tú-wɣ-sɤʑɯrja, tɕe nɯ kɯ-spoʁ ɯ-ŋgɯ rorʁe nɯ mɤro ɯ-kɯ-spoʁ ɯ-ŋgɯ ɲɯ-ɕe kɯ-tɕhɯt ɯ-tshɤt ma mɤ-kɯ-jpum pjɯ-ŋu ra. tɕe mɤ-ro ɯ-thoʁ pjɯ́-wɣ-lɣa tɕe ɲɯ́-wɣ-sɤʑɯrja tɕe pjɯ́-wɣ-sɤtsa tɕe, ɯ-kɯ-spoʁ mɤro raŋri ɣɯ tú-wɣ-z-nɯstɯ-stu tɕe rorʁe ɲɯ́-wɣ-rʁe tɕe tɤ-rɤku kú-wɣ-sɤro ŋu. tɕe nɯ tɤ-rɤku tú-wɣ-rɤwum cho tú-wɣ-sɯɣ-rom kɤ-nɯmga ŋu.\cmn 粮架就是把不太粗的木料砍下来,左右两边削下来,使它变扁,然后顺着木料的长度打三个洞,然后在洞里穿木棒,木棒的粗度要刚好配合洞的大小。在地面挖洞把粮架插在里面,这样排成一行,使粮架的每一个洞对端,然后穿木棒就把粮食架上去。粮架的作用是收拾粮食让它干。\end{exemple}\end{entrée}

\begin{entrée}
\vedette{\hypertarget{Ⓔmɤrom}{\papi{ mɤrom}}}\markboth{mɤrom}{}
\classe{vs}
\paradigme{\textit{dir :} \jya thɯ-}
\paradigme{\textit{dir :} \jya tɤ-}
\begin{définition}\fra enfler\end{définition}
\begin{définition}\cmn 肿\end{définition}
\begin{exemple}\jya to-mɤrom\cmn 肿了\end{exemple}
\begin{exemple}\jya a-mi thɯ-mɤrom\cmn 我的脚肿了\end{exemple}\end{entrée}

\begin{entrée}
\vedette{\hypertarget{Ⓔmɤrpaʁ}{\papi{ mɤrpaʁ}}}\markboth{mɤrpaʁ}{}\classe{vt}
\paradigme{\textit{dir :} \jya tɤ-}
\begin{définition}\ 
\begin{déclaration}\grammar{denom}\end{déclaration}\end{définition}
\begin{définition}\fra porter à l’épaule\end{définition}
\begin{définition}\cmn 扛在肩上\end{définition}
\begin{exemple}\jya tɤ-mɤrpaʁ-a\cmn 我扛了\end{exemple}
\begin{exemple}\jya ki laχtɕha ki tɤ-mɤrpaʁ\cmn 你把这根东西扛在肩上\end{exemple}
\begin{relation-sémantique}\synonyme{
\hyperlink{Ⓔnɤrpaʁku}{\textit{ \papi{nɤrpaʁku}}}
}\end{relation-sémantique}
\begin{relation-sémantique}\confer{
\hyperlink{Ⓔtɯ-rpaʁ}{\textit{ \papi{tɯ-rpaʁ}}}
}\end{relation-sémantique}\end{entrée}

\begin{entrée}
\vedette{\hypertarget{Ⓔmɤrtsaβ}{\papi{ mɤrtsaβ}}}\markboth{mɤrtsaβ}{}\classe{vs}
\paradigme{\textit{dir :} \jya nɯ-}
\begin{définition}\fra piquant\end{définition}
\begin{définition}\cmn 辣\end{définition}
\begin{exemple}\jya ɲɯ-tɯ-mɤrtsaβ\cmn 你脾气很泼辣\end{exemple}\begin{sous-entrée}
\vedette{\hypertarget{}{\papi{ nɤmɤrtsaβ}}}\markboth{nɤmɤrtsaβ}{}\classe{vt}
\paradigme{\textit{dir :} \jya pɯ-}
\begin{définition}\ 
\begin{déclaration}\grammar{trop}\end{déclaration}\end{définition}
\begin{définition}\fra trouver trop piquant\end{définition}
\begin{définition}\cmn 觉得辣\end{définition}
\end{sous-entrée}\begin{sous-entrée}
\vedette{\hypertarget{}{\papi{ zmɤrtsaβ}}}\markboth{zmɤrtsaβ}{}\classe{vt}
\paradigme{\textit{dir :} \jya pɯ-}
\begin{définition}\fra rendre piquant\end{définition}
\begin{définition}\cmn 把辣椒加多\end{définition}
\begin{exemple}\jya pɯ-zmɤrtsaβ-a\cmn 我把辣椒加多了\end{exemple}
\begin{exemple}\jya a-jaʁ na-qhrɯt tɕe na-zmɤrtsaβ\cmn 我被刮到手了,感觉辣乎乎的\end{exemple}
\begin{exemple}\jya a-jaʁ mtshalu kɯ ka-mtsɯɣ tɕe na-zmɤrtsaβ\cmn 我的手碰到荨麻就很痛\end{exemple}
\end{sous-entrée}\end{entrée}

\begin{entrée}
\vedette{\hypertarget{Ⓔmɤrʑaβ}{\papi{ mɤrʑaβ}}}\markboth{mɤrʑaβ}{}\classe{vi}
\paradigme{\textit{dir :} \jya nɯ-}
\begin{définition}\ 
\begin{déclaration}\grammar{denom}\end{déclaration}\end{définition}
\begin{définition}\fra se marier (fille)\end{définition}
\begin{définition}\cmn 嫁人\end{définition}
\begin{exemple}\jya ɯ-me nɯ-mɤrʑaβ\cmn 他女儿结了婚\end{exemple}
\begin{relation-sémantique}\confer{
\hyperlink{Ⓔtɤ-rʑaβ}{\textit{ \papi{tɤ-rʑaβ}}}
}\end{relation-sémantique}\end{entrée}

\begin{entrée}
\vedette{\hypertarget{Ⓔmɤsɲɯm}{\papi{ mɤsɲɯm}}}\markboth{mɤsɲɯm}{}
\classe{vs}
\paradigme{\textit{dir :} \jya tɤ-}
\begin{définition}\fra aimer manger les tissus, les cordes (bovidé)\end{définition}
\begin{définition}\cmn 喜欢吃麻织品,毛织品,皮革(牛)\end{définition}
\begin{exemple}\jya fsapaʁ ɲɯ-mɤsɲɯm\cmn 牲畜爱吃纺织品\end{exemple}\end{entrée}

\begin{entrée}
\vedette{\hypertarget{Ⓔmɤstɤkɤmi}{\papi{ mɤstɤkɤmi}}}\markboth{mɤstɤkɤmi}{}\classe{adv}
\begin{définition}\fra de façon incompréhensible\end{définition}
\begin{définition}\cmn 莫名其妙\end{définition}
\begin{exemple}\jya tʂu mɤstɤkɤmi ʑo pjɤ-mbɯt\cmn 路莫名其妙地塌下来了\end{exemple}\end{entrée}

\begin{entrée}
\vedette{\hypertarget{Ⓔmɤtɕɯ}{\papi{ mɤtɕɯ}}}\markboth{mɤtɕɯ}{}
\classe{vi}
\paradigme{\textit{dir :} \jya jɤ-}
\begin{définition}\fra aller dans la maison de son épouse après le mariage\end{définition}
\begin{définition}\cmn 入赘(在别人家当女婿)\end{définition}
\begin{exemple}\jya nɤʑo z-jɤ-tɯ-mɤtɕɯ ɕti\cmn 你是去上门的\end{exemple}
\begin{exemple}\jya nɤʑo ɣɯ-jɤ-tɯ-mɤtɕɯ ɕti\cmn 你是来上门的\end{exemple}
\begin{relation-sémantique}\confer{
\hyperlink{Ⓔtɤ-tɕɯ}{\textit{ \papi{tɤ-tɕɯ}}}
}\end{relation-sémantique}\end{entrée}

\begin{entrée}
\vedette{\hypertarget{Ⓔmɤtsamɤmu}{\papi{ mɤtsamɤmu}}}\markboth{mɤtsamɤmu}{}\classe{n}
\begin{définition}\fra nouvelle famille\end{définition}
\begin{définition}\cmn 新建立的家庭\end{définition}\end{entrée}

\begin{entrée}
\vedette{\hypertarget{Ⓔmɤtsomɤmu}{\papi{ mɤtsomɤmu}}}\markboth{mɤtsomɤmu}{}\classe{n}
\begin{définition}\fra conflit avec la belle-famille\end{définition}
\begin{définition}\cmn 婆媳关系不和(的家庭)\end{définition}
\begin{exemple}\jya ʑara nɯ-mɤtsomɤmu ɲɯ-thɯ\cmn 他们家的婆媳关系非常严重\end{exemple}\end{entrée}

\begin{entrée}
\vedette{\hypertarget{Ⓔmɤtɯmaʁri}{\papi{ mɤtɯmaʁri}}}\markboth{mɤtɯmaʁri}{}\classe{cnj}
\begin{définition}\fra comme ... le dit\end{définition}
\begin{définition}\cmn 正如……所说的\end{définition}
\begin{exemple}\jya nɤj mɤtɯmaʁri = nɤj mɤ-tɯ-ti maʁ ri\cmn 正如你说的\end{exemple}
\begin{exemple}\jya ɯʑo mɤtɯmaʁri = ɯʑo mɤ-ti maʁ ri\cmn 正如他说的\end{exemple}
\begin{relation-sémantique}\confer{
\hyperlink{Ⓔti}{\textit{ \papi{ti}}}
}\end{relation-sémantique}
\begin{relation-sémantique}\confer{
\hyperlink{ⒺmaʁⒽ1}{\textit{ \papi{maʁ1}}}
}\end{relation-sémantique}\end{entrée}

\begin{entrée}
\vedette{\hypertarget{Ⓔmɤχcɤl}{\papi{ mɤχcɤl}}}\markboth{mɤχcɤl}{}
\classe{vi}
\paradigme{\textit{dir :} \jya tɤ-}
\begin{définition}\ 
\begin{déclaration}\grammar{denom}\end{déclaration}\end{définition}
\begin{définition}\fra être au milieu\end{définition}
\begin{définition}\cmn 在中间\end{définition}
\begin{exemple}\jya nɤʑo jɤ-mɤku tɕe aj tu-mɤχcal-a (tu-mɯpɤχcal-a)\cmn 你走前面吧,我走中间\end{exemple}
\begin{relation-sémantique}\confer{
 \papi{mɤpɤχcɤl}
}\end{relation-sémantique}
\begin{relation-sémantique}\confer{
\hyperlink{Ⓔɯ-χcɤl}{\textit{ \papi{ɯ-χcɤl}}}
}\end{relation-sémantique}\end{entrée}

\begin{entrée}
\vedette{\hypertarget{Ⓔmɤzɯr}{\papi{ mɤzɯr}}}\markboth{mɤzɯr}{}
\classe{vi}
\paradigme{\textit{dir :} \jya lɤ-}
\paradigme{\textit{dir :} \jya thɯ-}
\paradigme{\textit{dir :} \jya kɤ-}
\paradigme{\textit{dir :} \jya nɯ-}
\begin{définition}\ 
\begin{déclaration}\grammar{denom}\end{déclaration}\end{définition}\acception{1}
\begin{définition}\fra être sur le côté\end{définition}
\begin{définition}\cmn 在旁边\end{définition}
\begin{exemple}\jya pɯ-ŋke-j pɯ-ŋu tɕe, aʑo ɲɯ-mɤzɯr-a pɯ-ŋu (chɯ-mɤzɯr-a pɯ-ŋu)\cmn 我们走的时候,我在边缘\end{exemple}\acception{2}
\begin{définition}\fra reculé (endroit)\end{définition}
\begin{définition}\cmn 偏远;偏僻
\begin{déclaration} \étymologie{\papi{zur}}\end{déclaration}\end{définition}\end{entrée}

\begin{entrée}
\vedette{\hypertarget{Ⓔmɤʑɯ}{\papi{ mɤʑɯ}}}\markboth{mɤʑɯ}{}\classe{adv}
\begin{définition}\fra encore\end{définition}
\begin{définition}\cmn 还有\end{définition}
\begin{exemple}\jya jisŋi mɤʑɯ nɤ-kɤ-thu ɯ́-tu\cmn 你今天还有问题吗?\end{exemple}
\begin{exemple}\jya mɤʑɯ tu-ti-a ŋu nɤ?\cmn 我再说一次吗?\end{exemple}
\begin{exemple}\jya mɤʑɯ pɯ-pɯ-ŋu nɤ\cmn 还有就是\end{exemple}
\begin{relation-sémantique}\confer{
\hyperlink{ⒺʑɯⒽ1}{\textit{ \papi{ʑɯ}}}
}\end{relation-sémantique}\end{entrée}

\begin{entrée}
\vedette{\hypertarget{Ⓔmba}{\papi{ mba}}}\markboth{mba}{}
\classe{vs}
\paradigme{\textit{dir :} \jya nɯ-}\acception{1}
\begin{définition}\fra mince\end{définition}
\begin{définition}\cmn 薄\end{définition}
\begin{exemple}\jya ɕoʁɕoʁ ɲɯ-mba\cmn 纸很薄\end{exemple}
\begin{exemple}\jya nɤ-ŋga ɲɯ-mba\cmn 你的衣服很薄\end{exemple}
\begin{exemple}\jya ɯ-sɯm ɲɯ-mba\cmn 他心软\end{exemple}\acception{2}
\begin{définition}\fra peu profond\end{définition}
\begin{définition}\cmn 浅\end{définition}\begin{sous-entrée}
\vedette{\hypertarget{}{\papi{ ɣɤmba}}}\markboth{ɣɤmba}{}\classe{vt}
\begin{définition}\fra rendre fin\end{définition}
\begin{définition}\cmn 弄薄\end{définition}
\begin{relation-sémantique}\antonyme{
\hyperlink{Ⓔjaʁ}{\textit{ \papi{jaʁ}}}
}\end{relation-sémantique}
\end{sous-entrée}\end{entrée}

\begin{entrée}
\vedette{\hypertarget{Ⓔmbala}{\papi{ mbala}}}\markboth{mbala}{}\classe{n}
\begin{définition}\fra bœuf\end{définition}
\begin{définition}\cmn 黄牛\end{définition}
\end{entrée}

\begin{entrée}
\vedette{\hypertarget{Ⓔmbalalu}{\papi{ mbalalu}}}\markboth{mbalalu}{}\classe{n}
\begin{définition}\fra année du bœuf\end{définition}
\begin{définition}\cmn 牛年\end{définition}
\end{entrée}

\begin{entrée}
\vedette{\hypertarget{Ⓔmbalapɯ}{\papi{ mbalapɯ}}}\markboth{mbalapɯ}{}\classe{n}
\begin{définition}\fra petit de vache\end{définition}
\begin{définition}\cmn 小黄牛\end{définition}
\end{entrée}

\begin{entrée}
\vedette{\hypertarget{Ⓔmbanaʁxtsa}{\papi{ mbanaʁxtsa}}}\markboth{mbanaʁxtsa}{}\classe{n}
\begin{définition}\fra botte en cuir noir\end{définition}
\begin{définition}\cmn 黑皮鞋\end{définition}
\end{entrée}

\begin{entrée}
\vedette{\hypertarget{Ⓔmbaqhu}{\papi{ mbaqhu}}}\markboth{mbaqhu}{}\classe{n}
\begin{définition}\ 
\begin{déclaration}\grammar{n.lieu}\end{déclaration}\end{définition}
\begin{définition}\fra l'un des hameaux de Gyutshapa\end{définition}
\begin{définition}\cmn 二茶村的大队之一\end{définition}\end{entrée}

\begin{entrée}
\vedette{\hypertarget{Ⓔmbarkhom}{\papi{ mbarkhom}}}\markboth{mbarkhom}{}\classe{n}
\begin{définition}\ 
\begin{déclaration}\grammar{n.lieu}\end{déclaration}\end{définition}
\begin{définition}\fra Mbarkham\end{définition}
\begin{définition}\cmn 马尔康\end{définition}\end{entrée}

\begin{entrée}
\vedette{\hypertarget{Ⓔmbarqhi}{\papi{ mbarqhi}}}\markboth{mbarqhi}{}\classe{n}
\begin{définition}\fra distance\end{définition}
\begin{définition}\cmn 距离\end{définition}
\begin{exemple}\jya mbarqhi ɯ-ɲɯ́-βdi kɯ tú-wɣ-tshɤt ɲɯ-ra\cmn 要试一下(话筒离嘴的)距离合不合适\end{exemple}
\begin{relation-sémantique}\confer{
\hyperlink{Ⓔarqhi}{\textit{ \papi{arqhi}}}
}\end{relation-sémantique}
\begin{relation-sémantique}\confer{
\hyperlink{Ⓔarmbat}{\textit{ \papi{armbat}}}
}\end{relation-sémantique}\end{entrée}

\begin{entrée}
\vedette{\hypertarget{Ⓔmbaʁ}{\papi{ mbaʁ}}}\markboth{mbaʁ}{}
\classe{vi}
\paradigme{\textit{dir :} \jya nɯ-}
\paradigme{\textit{dir :} \jya thɯ-}\acception{1}
\begin{définition}\fra se casser\end{définition}
\begin{définition}\cmn 破裂,裂开
(桌子等东西)有缝隙的话就可以说\stylefv{ɲɤ-mbaʁ}
\end{définition}
\begin{exemple}\jya rkɤsnom ɯ-srɯβ ɲɤ-mbaʁ\cmn 裤子(的针脚)裂缝了\end{exemple}
\begin{exemple}\jya tɯ-ŋga ɯ-srɯβ ɲɤ-mbaʁ\cmn 衣服裂缝了\end{exemple}
\begin{exemple}\jya ɟu chɤ-mbaʁ\cmn 竹子裂开了\end{exemple}
\begin{exemple}\jya ɕoŋtɕa ɲɤ-mbaʁ\cmn 木料裂开了\end{exemple}
\begin{exemple}\jya znde ɲɤ-mbaʁ\cmn 墙裂开了\end{exemple}\acception{2}
\begin{définition}\fra avoir une ouverture (habit)\end{définition}
\begin{définition}\cmn 叉开(衣服)\end{définition}
\begin{exemple}\jya ɕɤntsɯt nɯ χchoʁe ʑo ɯ-ndo tu-kɯ-mbaʁ tu\cmn 女式长衫的下面左右两边都是叉开的\end{exemple}\end{entrée}

\begin{entrée}
\vedette{\hypertarget{Ⓔmbaʁŋgu}{\papi{ mbaʁŋgu}}}\markboth{mbaʁŋgu}{}\classe{n}
\begin{définition}\fra masque de danse\end{définition}
\begin{définition}\cmn 跳神时戴的面具
\begin{déclaration} \étymologie{\papi{ⁿbag.mgo}}\end{déclaration}\end{définition}\end{entrée}

\begin{entrée}
\vedette{\hypertarget{Ⓔmbat}{\papi{ mbat}}}\markboth{mbat}{}\classe{vs}
\paradigme{\textit{dir :} \jya tɤ-}\acception{1}
\begin{définition}\fra léger (travail)\end{définition}
\begin{définition}\cmn 轻松\end{définition}
\begin{exemple}\jya (kɤ-nɤtsoʁ) mɯ́j-mbat ma ɣɯ-lɣa ra\cmn 找人参果不容易,因为要挖地\end{exemple}
\begin{exemple}\jya kɤ-ɕe ɲɯ-mbat\cmn 去很轻松\end{exemple}\acception{2}
\begin{définition}\fra être presque fini\end{définition}
\begin{définition}\cmn 快要没有了\end{définition}
\begin{exemple}\jya tɤŋe tɤ-mbat / ɲɤ-mbat\cmn 太阳快要落山了\end{exemple}
\begin{exemple}\jya nɤ-tɤ-rʑaʁ tɤ-mbat\cmn 你快没有时间了\end{exemple}
\begin{exemple}\jya kɤ-ndza to-mbat\cmn 快没有食物了\end{exemple}\acception{3}
\begin{définition}\fra bon marché\end{définition}
\begin{définition}\cmn 便宜\end{définition}
\begin{exemple}\jya ɯ-phɯ ɲɯ-mbat\cmn 很便宜\end{exemple}\begin{sous-entrée}
\vedette{\hypertarget{}{\papi{ ɣɤmbat}}}\markboth{ɣɤmbat}{}\classe{vt}
\paradigme{\textit{dir :} \jya tɤ-}
\begin{définition}\ 
\begin{déclaration}\grammar{caus}\end{déclaration}\end{définition}
\begin{définition}\fra baisser le prix\end{définition}
\begin{définition}\cmn 降价\end{définition}
\begin{exemple}\jya (ɯ-phɯ) tɤ-ɣɤmba-t-a\cmn 我减价了\end{exemple}
\begin{relation-sémantique}\confer{
\hyperlink{ⒺstuⒽ1}{\textit{ \papi{stu1}}}
}\end{relation-sémantique}
\end{sous-entrée}\end{entrée}

\begin{entrée}
\vedette{\hypertarget{ⒺmbɤβⒽ1}{\papi{ mbɤβ}}}\markboth{mbɤβ}{}\homonyme{1}\classe{vi}
\paradigme{\textit{dir :} \jya kɤ-}
\begin{définition}\fra avoir fini de fermenter,avoir été distillé (alcool fort)\end{définition}
\begin{définition}\cmn 酿出来(酒、奶渣等)\end{définition}
\begin{exemple}\jya cha ko-mbɤβ\cmn 酒酿出来了\end{exemple}
\begin{exemple}\jya tɕhɯrwa wuma ɲɯ-mbɤβ\cmn 奶渣酿出来\end{exemple}
\begin{relation-sémantique}\synonyme{
\hyperlink{Ⓔaβzu}{\textit{ \papi{aβzu}}}
}\end{relation-sémantique}\begin{sous-entrée}
\vedette{\hypertarget{}{\papi{ sɯɣmbɤβ}}}\markboth{sɯɣmbɤβ}{}\classe{vt}
\paradigme{\textit{dir :} \jya kɤ-}
\begin{définition}\fra faire fermenter\end{définition}
\begin{définition}\cmn 把……酿出来\end{définition}
\begin{exemple}\jya cha kɤ-sɯɣmbɤβ mɯ́j-khɯ\cmn 没有办法把酒酿出来\end{exemple}
\end{sous-entrée}\end{entrée}

\begin{entrée}
\vedette{\hypertarget{ⒺmbɤβⒽ2}{\papi{ mbɤβ}}}\markboth{mbɤβ}{}\homonyme{2}
\classe{vi}
\paradigme{\textit{dir :} \jya pɯ-}\acception{1}
\begin{définition}\fra se calmer, se taire(d'une foule)\end{définition}
\begin{définition}\cmn 平静下来(喧哗的人)\end{définition}
\begin{exemple}\jya mkhɤrmaŋ ra pjɤ-mbɤβ-nɯ\cmn 老百姓平静下来了\end{exemple}\acception{2}
\begin{définition}\fra se refroidir (eau bouillante)\end{définition}
\begin{définition}\cmn 冷却(开水)\end{définition}
\begin{exemple}\jya tɯthɯ ɯ-ŋgɯ tɯ-ci ɲɯ-ɤla tɕe mbuz ɲɯ-ŋu, tɕe tɯ-ci kɯ-mɯɕtaʁ kɤ-lat-a tɕe pɯ-mbɤβ\cmn 锅里的水开了,快要溢出来,所以我倒了一点冷水冷却一下\end{exemple}\acception{3}
\begin{définition}\fra être bien peignée (chevelure, vers le bas)\end{définition}
\begin{définition}\cmn (头发)整齐,梳理好了的\end{définition}
\begin{exemple}\jya nɤ-ku pɯ-sɤɕɤt ma a-pɯ-mbɤβ tɕe ʁzraŋʁzraŋ a-mɤ-pɯ-pa\cmn 梳一下头,头发要整齐,不要乱蓬蓬的\end{exemple}\end{entrée}

\begin{entrée}
\vedette{\hypertarget{ⒺmbɤβⒽ3}{\papi{ mbɤβ}}}\markboth{mbɤβ}{}\homonyme{3}
\classe{vi}
\paradigme{\textit{dir :} \jya pɯ-}
\begin{définition}\fra camper\end{définition}
\begin{définition}\cmn 宿营\end{définition}
\begin{exemple}\jya mbroχpa ra nɯre ri pjɤ-mbɤβ-nɯ\cmn 草地人在那里宿营了\end{exemple}\end{entrée}

\begin{entrée}
\vedette{\hypertarget{ⒺmbɤrⒽ1}{\papi{ mbɤr}}}\markboth{mbɤr}{}\homonyme{1}
\classe{vs}
\paradigme{\textit{dir :} \jya pɯ-}
\begin{définition}\fra bas, petit\end{définition}
\begin{définition}\cmn 低;矮\end{définition}
\begin{exemple}\jya nɤʑo ɲɯ-tɯ-mbro, aʑo ɲɯ-mbar-a\cmn 你很高,我很矮\end{exemple}
\begin{exemple}\jya tɤ-pɤtso kɯ-mbro ɣɤʑu, kɯ-mbɤr ɣɤʑu\cmn 有的孩子长得高,有的长得矮\end{exemple}
\begin{exemple}\jya ɯ-phoŋbu ɲɯ-mbɤr\cmn 他身材很矮\end{exemple}
\begin{relation-sémantique}\confer{
\hyperlink{Ⓔɣɤmbɤr}{\textit{ \papi{ɣɤmbɤr}}}
}\end{relation-sémantique}
\begin{relation-sémantique}\antonyme{
\hyperlink{ⒺmbroⒽ1}{\textit{ \papi{mbro1}}}
}\end{relation-sémantique}\end{entrée}

\begin{entrée}
\vedette{\hypertarget{ⒺmbɤrⒽ2}{\papi{ mbɤr}}}\markboth{mbɤr}{}\homonyme{2}
\classe{vi}
\paradigme{\textit{dir :} \jya pɯ-}
\begin{définition}\ 
\begin{déclaration}\grammar{acaus}\end{déclaration}\end{définition}
\begin{définition}\fra bien épousseté\end{définition}
\begin{définition}\cmn 拍得干净\end{définition}
\begin{relation-sémantique}\confer{
\hyperlink{Ⓔsɤphɤr}{\textit{ \papi{sɤphɤr}}}
}\end{relation-sémantique}\end{entrée}

\begin{entrée}
\vedette{\hypertarget{Ⓔmbɤt}{\papi{ mbɤt}}}\markboth{mbɤt}{}\classe{interj}
\begin{définition}\fra vite!\end{définition}
\begin{définition}\cmn 快点!\end{définition}
\begin{exemple}\jya mbɤt, tɤ-mda!\cmn 快点,时间到了\end{exemple}\end{entrée}

\begin{entrée}
\vedette{\hypertarget{Ⓔmbɤxɕɯβ}{\papi{ mbɤxɕɯβ}}}\markboth{mbɤxɕɯβ}{}\classe{n}
\begin{définition}\fra une plante\end{définition}
\begin{définition}\cmn 植物的一种\end{définition}
\begin{exemple}\jya mbɤxɕɯβ nɯ sɯjno kɯ-xtɕi ci ŋu, ɯ-ru ɣɯrni, ɯ-jwaʁ rʁom, rɕɯβrɕɯβ ʑo pa, ɯ-mɯntoʁ qarŋe, pjɯ́-wɣ-qlɯt tɕe ɯ-lu tu, paʁndza sna, tɯrme kɤ-ndza mɤ-sna.\cmn 
\stylefv{mbɤxɕɯβ}是一种矮小的植物,茎是红色的,叶子像砂纸一样粗糙,花是黄色的,把茎折断时有乳汁,可以喂猪,人不能吃
\end{exemple}\end{entrée}

\begin{entrée}
\vedette{\hypertarget{Ⓔmbe}{\papi{ mbe}}}\markboth{mbe}{}
\classe{vs}
\paradigme{\textit{dir :} \jya nɯ-}
\begin{définition}\fra ancien\end{définition}
\begin{définition}\cmn 旧\end{définition}
\begin{exemple}\jya tɯ-ŋga ɲɯ-mbe\cmn 衣服是旧的\end{exemple}
\begin{relation-sémantique}\antonyme{
\hyperlink{ⒺɕɤɣⒽ1}{\textit{ \papi{ɕɤɣ}}}
}\end{relation-sémantique}\end{entrée}

\begin{entrée}
\vedette{\hypertarget{Ⓔmbɣaʁ}{\papi{ mbɣaʁ}}}\markboth{mbɣaʁ}{}
\classe{vi}
\paradigme{\textit{dir :} \jya \_}
\begin{définition}\ 
\begin{déclaration}\grammar{acaus}\end{déclaration}\end{définition}
\begin{définition}\fra se retourner\end{définition}
\begin{définition}\cmn 翻身\end{définition}
\begin{exemple}\jya @qiche cho-mbɣaʁ\cmn 汽车翻车了\end{exemple}\begin{sous-entrée}
\vedette{\hypertarget{}{\papi{ ɣɤmbɣaʁ}}}\markboth{ɣɤmbɣaʁ}{}\classe{vs}
\begin{définition}\ 
\begin{déclaration}\grammar{facil}\end{déclaration}\end{définition}
\begin{définition}\fra se retourner facilement (faire facilement un tonneau, d'une voiture)\end{définition}
\begin{définition}\cmn 容易翻车\end{définition}
\begin{exemple}\jya ɲchɣaʁʑɤr kɯ-tu tʂu tɕe, @qiche kɤ-lɤt ɲɯ-sɤɣʑɯr ma ɲchɣaʁʑɤr nɯ mɯ́j-saχsɤl tɕe ɲɯ-ɣɤmbɣaʁ (chɯ-mbɣaʁ ɲɯ-mbat)\cmn 路面结冰的时候,开车非常危险,因为结冰的道路不明显,容易翻车\end{exemple}
\begin{relation-sémantique}\confer{
\hyperlink{Ⓔnɤmbɣaʁlaʁ}{\textit{ \papi{nɤmbɣaʁlaʁ}}}
}\end{relation-sémantique}
\begin{relation-sémantique}\confer{
\hyperlink{Ⓔpɣaʁ}{\textit{ \papi{pɣaʁ}}}
}\end{relation-sémantique}
\begin{relation-sémantique}\confer{
\hyperlink{Ⓔapɣaʁsci}{\textit{ \papi{apɣaʁsci}}}
}\end{relation-sémantique}
\end{sous-entrée}\end{entrée}

\begin{entrée}
\vedette{\hypertarget{Ⓔmbɣɤjroʁ}{\papi{ mbɣɤjroʁ}}}\markboth{mbɣɤjroʁ}{}
\classe{n}
\begin{définition}\fra sillon\end{définition}
\begin{définition}\cmn 犁沟\end{définition}\end{entrée}

\begin{entrée}
\vedette{\hypertarget{Ⓔmbɣɤru}{\papi{ mbɣɤru}}}\markboth{mbɣɤru}{}
\classe{n}
\begin{définition}\fra partie de la charrue\end{définition}
\begin{définition}\cmn 犁杆\end{définition}
\begin{exemple}\jya mbɣopɤl ɯ-taʁ chɯ́-wɣ-tshoʁ tɕe ɯ-ɕnɤz nɯ stuxsi ɯ-taʁ lú-wɣ-βraʁ tɕe nɯ mbɣɤru rmi\cmn 固定铧头的部分上面再装一根木杆,木杆的另一头拴在牛轭上,这根木杆叫犁干。\end{exemple}\end{entrée}

\begin{entrée}
\vedette{\hypertarget{Ⓔmbɣɤsroʁ}{\papi{ mbɣɤsroʁ}}}\markboth{mbɣɤsroʁ}{}
\classe{n}
\begin{définition}\fra partie de la charrue\end{définition}
\begin{définition}\cmn 犁的组成部分之一\end{définition}
\begin{exemple}\jya mbɣɤsroʁ nɯ mbɣɤru cho mbɣopɤl ni ndʑi-kɯ-ɣɯthaʁ ŋu\cmn 
\stylefv{mbɣɤsroʁ}是连接犁头和主干和的部分。
\end{exemple}\end{entrée}

\begin{entrée}
\vedette{\hypertarget{Ⓔmbɣɤtɕɯkala}{\papi{ mbɣɤtɕɯkala}}}\markboth{mbɣɤtɕɯkala}{}\classe{n}
\begin{définition}\ 
\begin{déclaration}\grammar{n.lieu}\end{déclaration}\end{définition}
\begin{définition}\fra montagne du village de Mangi\end{définition}
\begin{définition}\cmn 蒙岩村的一座山之一\end{définition}
\end{entrée}

\begin{entrée}
\vedette{\hypertarget{Ⓔmbɣo}{\papi{ mbɣo}}}\markboth{mbɣo}{}
\classe{n}
\begin{définition}\fra charrue\end{définition}
\begin{définition}\cmn 犁头\end{définition}\end{entrée}

\begin{entrée}
\vedette{\hypertarget{Ⓔmbɣom}{\papi{ mbɣom}}}\markboth{mbɣom}{}
\classe{vs}
\paradigme{\textit{dir :} \jya tɤ-}
\begin{définition}\fra occupé, pressé\end{définition}
\begin{définition}\cmn 忙\end{définition}
\begin{exemple}\jya tɤ-mbɣom-a tɕe kɤ-ari-a\cmn 我很急地去了\end{exemple}
\begin{exemple}\jya tɤ-mbɣom ma tɯ-maqhu\cmn 你快一点,不然你会迟到\end{exemple}
\begin{exemple}\jya ma-tɤ-tɯ-mbɣom\cmn 不用那么着急\end{exemple}
\begin{exemple}\jya ɯ-ɲɯ́-tɯ-mbɣom?\cmn 你很急吗?\end{exemple}
\begin{exemple}\jya kɤ-mbɣom mɤ-ra / tɯ-mbɣom mɤ-ra\cmn 不用着急\end{exemple}
\begin{relation-sémantique}\confer{
\hyperlink{Ⓔɕɯmbɣom}{\textit{ \papi{ɕɯmbɣom}}}
}\end{relation-sémantique}
\begin{relation-sémantique}\confer{
\hyperlink{Ⓔɣɤmbɣomru}{\textit{ \papi{ɣɤmbɣomru}}}
}\end{relation-sémantique}\end{entrée}

\begin{entrée}
\vedette{\hypertarget{Ⓔmbɣopɤl}{\papi{ mbɣopɤl}}}\markboth{mbɣopɤl}{}
\classe{n}
\begin{définition}\fra partie de la charrue\end{définition}
\begin{définition}\cmn 用来固定铧头的部分\end{définition}
\begin{exemple}\jya mbɣopɤl nɯ qraʁ ɯ-sɤ-tshoʁ nɯ ŋu, mbɣopɤl ɯ-pa qraʁ tú-wɣ-tshoʁ, mbɣɤru chɯ́-wɣ-tshoʁ tɕe kɤ-ntɕhoz tɯ-sna ɕti\cmn 
\stylefv{mbɣopɤl}是用来固定铧头的部分,\stylefv{mbɣopɤl}下端装上铧头,上端连上犁杆,就可以用了。
\end{exemple}\end{entrée}

\begin{entrée}
\vedette{\hypertarget{Ⓔmbɣorna}{\papi{ mbɣorna}}}\markboth{mbɣorna}{}\classe{n}
\begin{définition}\fra partie de la charrue\end{définition}
\begin{définition}\cmn 犁把\end{définition}
\begin{exemple}\jya mbɣorna nɯ mbɣopɤl ɯ-ku ɯ-taʁ ku-ndzoʁ tɕe kɯ-ɕlu ɣɯ ɯ-jaʁ sɯ-ndo spa ŋu\cmn 犁把是装在犁头上,供耕地人掌握方向的把手\end{exemple}
\end{entrée}

\begin{entrée}
\vedette{\hypertarget{Ⓔmbɣɯrloʁ}{\papi{ mbɣɯrloʁ}}}\markboth{mbɣɯrloʁ}{}\classe{n}
\begin{définition}\fra tonnerre\end{définition}
\begin{définition}\cmn 雷
\begin{déclaration} \étymologie{\papi{mbrug.glog}}\end{déclaration}\end{définition}
\begin{exemple}\jya mbɣɯrloʁ to-βzu\cmn 打雷了\end{exemple}\end{entrée}

\begin{entrée}
\vedette{\hypertarget{Ⓔmbi}{\papi{ mbi}}}\markboth{mbi}{}\classe{vt}
\paradigme{\textit{dir :} \jya nɯ-}
\paradigme{\textit{dir :} \jya pɯ-}
\paradigme{\textit{dir :} \jya nɯ-}
\begin{définition}\fra donner\end{définition}
\begin{définition}\cmn 给
\begin{déclaration}\grammar{secondatif}\end{déclaration}\end{définition}
\begin{exemple}\jya ɯʑo kɯ nɯ́-wɣ-mbi-a\cmn 他给我了\end{exemple}\begin{sous-entrée}
\vedette{\hypertarget{}{\papi{ ambi}}}\markboth{ambi}{}\classe{vi}
\begin{définition}\ 
\begin{déclaration}\grammar{pass}\end{déclaration}\end{définition}
\begin{définition}\fra être donné\end{définition}
\begin{définition}\cmn 给了\end{définition}
\begin{exemple}\jya ɯʑo ambi ma jɯfɕɯr nɯ-mbi-t-a\cmn 已经给了他,我昨天就给了\end{exemple}
\begin{exemple}\jya tɤ-pɤtso ɯ-smɤn ambi\cmn 小孩子的药已经给了\end{exemple}
\begin{exemple}\jya ki tɤ-pɤtso saχsɯ ɣɯ ɯ-smɤn ambi, tɯrmɯ ɣɯ nɯ mɤ-ambi\cmn 给了中午的药,没有给下午的(药)\end{exemple}
\end{sous-entrée}\begin{sous-entrée}
\vedette{\hypertarget{}{\papi{ ambɯmbi}}}\markboth{ambɯmbi}{}\classe{vi}
\begin{définition}\ 
\begin{déclaration}\grammar{refl}\end{déclaration}\end{définition}
\begin{définition}\fra se donner les uns les autres\end{définition}
\begin{définition}\cmn 互相送\end{définition}
\begin{exemple}\jya ɕnɤto ɲɯ-ɤmbɯmbi-nɯ ŋgrɤl\cmn 他们互相送鼻烟\end{exemple}
\end{sous-entrée}\end{entrée}

\begin{entrée}
\vedette{\hypertarget{Ⓔmbijtshi}{\papi{ mbijtshi}}}\markboth{mbijtshi}{}
\classe{vt}
\paradigme{\textit{dir :} \jya nɯ-}
\begin{définition}\ 
\begin{déclaration}\grammar{comp}\end{déclaration}\end{définition}
\begin{définition}\fra donner à boire et à manger\end{définition}
\begin{définition}\cmn 喂东西吃;喂东西喝\end{définition}
\begin{exemple}\jya a-mu kɯ nɯ́-wɣ-mbijtshi-a\cmn 我母亲给了我吃喝\end{exemple}
\begin{relation-sémantique}\confer{
\hyperlink{Ⓔngɤjtshi}{\textit{ \papi{ngɤjtshi}}}
}\end{relation-sémantique}\end{entrée}

\begin{entrée}
\vedette{\hypertarget{Ⓔmbjiz}{\papi{ mbjiz}}}\markboth{mbjiz}{}
\classe{vs}
\paradigme{\textit{dir :} \jya nɯ-}
\begin{définition}\fra s'effacer (couleur)\end{définition}
\begin{définition}\cmn 褪色\end{définition}
\begin{exemple}\jya tɯ-nga ɲo-mbjiz\cmn 衣服褪色了\end{exemple}
\begin{exemple}\jya tɤ-scoz ɲo-mbjiz\cmn 字褪色了\end{exemple}\begin{sous-entrée}
\vedette{\hypertarget{}{\papi{ sɯmbjiz}}}\markboth{sɯmbjiz}{}\classe{vt}
\begin{définition}\ 
\begin{déclaration}\grammar{caus}\end{déclaration}\end{définition}
\end{sous-entrée}\end{entrée}

\begin{entrée}
\vedette{\hypertarget{Ⓔmbjom}{\papi{ mbjom}}}\markboth{mbjom}{}\classe{vs}
\paradigme{\textit{dir :} \jya tɤ-}
\paradigme{\textit{dir :} \jya thɯ-}
\begin{définition}\fra rapide\end{définition}
\begin{définition}\cmn 快(跑的速度)\end{définition}
\begin{exemple}\jya kɯ-mbjom ci ɲɯ-ŋu\cmn 他是个跑得快的人\end{exemple}
\begin{exemple}\jya @qiche ɲɯ-mbjom\cmn 汽车开得很快\end{exemple}
\begin{exemple}\jya wo, nɤ-tɯ-mbjom !\cmn 啊,你这么快就(回来了)!\end{exemple}\begin{sous-entrée}
\vedette{\hypertarget{}{\papi{ ɣɤmbjom}}}\markboth{ɣɤmbjom}{}\classe{vt}
\begin{définition}\ 
\begin{déclaration}\grammar{caus}\end{déclaration}\end{définition}
\begin{définition}\fra faire accélérer\end{définition}
\begin{définition}\cmn 加快\end{définition}
\begin{exemple}\jya rdɤstaʁ kɤ-lɤt jú-wɣ-nɤxɕɤt tɕe kɤ-ɣɤmbjom khɯ\cmn 用力扔石头可以令它飞出去得更快\end{exemple}
\end{sous-entrée}\begin{sous-entrée}
\vedette{\hypertarget{}{\papi{ ʑɣɤɣɤmbjom}}}\markboth{ʑɣɤɣɤmbjom}{}\classe{vi}
\paradigme{\textit{dir :} \jya nɯ-}
\begin{définition}\ 
\begin{déclaration}\grammar{refl}\end{déclaration}
\begin{déclaration}\grammar{caus}\end{déclaration}\end{définition}
\begin{définition}\fra s'efforcer à aller le plus vite possible\end{définition}
\begin{définition}\cmn 使自己跑得更快\end{définition}
\begin{exemple}\jya nɤ-mbro qale sthɯci a-nɯ-ʑɣɤɣɤmbjom\cmn 你的马要令自己跑得像风一样快\end{exemple}
\end{sous-entrée}\end{entrée}

\begin{entrée}
\vedette{\hypertarget{Ⓔmblɯt}{\papi{ mblɯt}}}\markboth{mblɯt}{}\classe{vi.nh}
\paradigme{\textit{dir :} \jya nɯ-}
\paradigme{\textit{dir :} \jya thɯ-}
\begin{définition}\ 
\begin{déclaration}\grammar{acaus}\end{déclaration}\end{définition}
\begin{définition}\fra être détruit, disparaître\end{définition}
\begin{définition}\cmn 绝种;绝后\end{définition}
\begin{exemple}\jya ji-paʁrɟit nɯ-mblɯt\cmn 我们的猪绝种了\end{exemple}
\begin{relation-sémantique}\confer{
\hyperlink{Ⓔplɯt}{\textit{ \papi{plɯt}}}
}\end{relation-sémantique}\end{entrée}

\begin{entrée}
\vedette{\hypertarget{Ⓔmboʁ}{\papi{ mboʁ}}}\markboth{mboʁ}{}
\classe{n}
\begin{définition}\fra tissu de lin rectangulaire\end{définition}
\begin{définition}\cmn 正方形的麻布\end{définition}
\begin{exemple}\jya mboʁ nɯ ɯ-spa tasa ŋu, tasa lú-wɣ-pɣo tɕe lú-wɣ-rɯm tɕe tú-wɣ-rɯtɤβri tɕe kú-wɣ-sqa, kɤ́-wɣ-sqa tɕe ɯ-ŋgɯ thɤfkɤlɤɣi kú-wɣ-lɤt ra, nɯ mɤɕtʂa ku-smi mɤ-cha. kɤ-smi tɕe, ɲɯ́-wɣ-z-mɯɕtaʁ tɕe ɲɯ́-wɣ-χtɕi tɕe tɤ-zbaʁ tɕe kú-wɣ-sɤrɤt tɕe kɤ-taʁ kú-wɣ-thɯ ŋu. tɕe kɤ-taʁ thɯ-jɤɣ tɕe tɯ-rtɯthɯ ŋu tɕe ɣɯ-xtsɯ ra ɯ-xtsɯ pɯ-rtaʁ tɕe ɲɯ-mba ŋu ɲɯ-mpɕu ŋu tɕe chɯ́-wɣ-tʂɯβ tɕe, kɯ-ɤβʑɯrdɯ-rdu ʑo ɲɯ́-wɣ-βzu tɕe ɯ-βzɯr tɯ-ka nɯ tɕu ɯ-jndɯz cho ɯ-ltɕi kú-wɣ-tshoʁ tɕe kɤ-ntɕhoz tu-βze ŋu. tɕe ɯ-mboʁ nɯ kɯ-tɣa tɕe tɯ-fcaʁ ŋu, tɯ-mbri cho tɯ-ɲcɣa ɣɯ ɯ-fkɯm ŋu, tɯ-mthɤɣ ɲɯ́-wɣ-rtɤβ ŋu tɕe mɤ-saʁdɯɣ.\cmn 
\stylefv{mboʁ} 的材料是大麻。要把麻捻成细线,搓紧,再反搓成一绞一绞的,然后煮了。煮的时候,要放大量的草木灰,不然煮不熟。煮熟了以后晾凉,然后洗干净。干了以后,把线卸下来堆在一旁,然后在牵杆上牵起来。织完了以后就成麻布,还要捶打。捶打好了以后,麻布就变薄了,变得很光滑,然后就把麻布缝成四方形的,在四个角做上流苏和彩色布条,这样就可以用了。\stylefv{mboʁ}在收割的时候可以垫背或者裹绳子和镰刀,可以围在腰间,不碍事。
\end{exemple}\end{entrée}

\begin{entrée}
\vedette{\hypertarget{Ⓔmboʁkhɯr}{\papi{ mboʁkhɯr}}}\markboth{mboʁkhɯr}{}
\classe{n}
\begin{définition}\fra paquet\end{définition}
\begin{définition}\cmn 包裹(用正方形的布)\end{définition}
\begin{relation-sémantique}\confer{
\hyperlink{Ⓔrɯmboʁkhɯr}{\textit{ \papi{rɯmboʁkhɯr}}}
}\end{relation-sémantique}\end{entrée}

\begin{entrée}
\vedette{\hypertarget{Ⓔmboʁɲɟi}{\papi{ mboʁɲɟi}}}\markboth{mboʁɲɟi}{}\classe{adv}
\begin{définition}\fra en milles morceaux\end{définition}
\begin{définition}\cmn 粉粹\end{définition}
\begin{exemple}\jya mboʁɲɟi ʑo ɲɤ-ɕe\cmn 粉粹了\end{exemple}
\begin{relation-sémantique}\confer{
\hyperlink{Ⓔarɤmboʁɲɟi}{\textit{ \papi{arɤmboʁɲɟi}}}
}\end{relation-sémantique}\end{entrée}

\begin{entrée}
\vedette{\hypertarget{Ⓔmbraj}{\papi{ mbraj}}}\markboth{mbraj}{}
\classe{n}
\begin{définition}\fra bouleau rouge\end{définition}
\begin{définition}\cmn 红桦树\end{définition}
\begin{exemple}\jya mbraj nɯ sɤjku cho kɯ-naχtɕɯɣ ŋu ri, mbraj ɣɯ ɯ-rqhu nɯ kɯ-ɣɯrni ŋu, sɤjku ɣɯ ɯ-rqhu kɯ-wɣrum ŋu, ndʑi-jwaʁ ɲɯ-naχtɕɯɣ\cmn 红桦树和白桦树相同,但红桦树的树皮是红色的,而白桦树是白色的,它们俩的叶子一样。\end{exemple}\end{entrée}

\begin{entrée}
\vedette{\hypertarget{Ⓔmbraʑɯm}{\papi{ mbraʑɯm}}}\markboth{mbraʑɯm}{}
\classe{n}
\begin{définition}\fra une espèce de champignon\end{définition}
\begin{définition}\cmn 一种蘑菇\end{définition}
\begin{exemple}\jya mbraʑɯm nɯ stɤmku kɯ-ɤmɯrmbɯrmbat ʑo tu-ɬoʁ, tɤjmɤɣ kɯ-ndɯ-ndɯβ ʑo ŋu, thɯ-kɤ-ɣɯri ʑo fse, kɯ-wɣrum ŋu, kɤ-ndza wuma mɯm, ftɕar tɕe tu-ɬoʁ ŋu\cmn 
\stylefv{mbraʑɯm}是长在草地上的一丛丛的小菌子,像是用线串起来的,呈白色,好吃,夏天生长。
\end{exemple}\end{entrée}

\begin{entrée}
\vedette{\hypertarget{Ⓔmbrɤjqhɤt}{\papi{ mbrɤjqhɤt}}}\markboth{mbrɤjqhɤt}{}
\classe{n}
\begin{définition}\fra gentiane\end{définition}
\begin{définition}\cmn 秦艽\end{définition}
\begin{exemple}\jya mbrɤjqhɤt nɯ sɯŋgɯ tu-kɯ-ɬoʁ sɯjno ci ŋu. smɤn kɤ-βzu ɲɯ-sna, ɯ-jwaʁ rɲɟi, jaʁ, mpɕu, ɯ-ru kɯ-nɤrko tsa ci ŋu, ɯ-ru ɯ-taʁ ɯ-jwaʁ nɯ tɯ-rtsɤɣ tɯ-rtsɤɣ lu-oʑɯrja tɕe, ɯ-ru cho ɯ-jwaʁ ni ndʑi-pɤrthɤβ ɯ-mɯntoʁ ɲɯ-lɤt ŋu, ɯ-mɯntoʁ wɣrum. ɯ-zrɤm nɯ wɣrum, ɯ-ru cho ɯ-zrɤm ɯ-khɤntshɤm ri ɯ-rme kɯ-rɲɟi kɯ-ngɯt ʑo tu. qartsɯ tɕe pjɯ-rom ɯ-fsaqhe tɕe pjɯ-ɬoʁ ŋu.\cmn 秦艽是生长在森林里的草,可以入药。叶子长、厚、光滑。茎比较结实。叶子一节一节地长在茎上,叶子和茎的中间开花,花是白色的。根也是白色的,在根和茎交界处有又长又结实的毛。冬天枯萎,第二年又生长。\end{exemple}\end{entrée}

\begin{entrée}
\vedette{\hypertarget{Ⓔmbrɤmbrɯ}{\papi{ mbrɤmbrɯ}}}\markboth{mbrɤmbrɯ}{}\classe{n}
\begin{définition}\fra légumineuse\end{définition}
\begin{définition}\cmn 豆类\end{définition}\end{entrée}

\begin{entrée}
\vedette{\hypertarget{Ⓔmbrɤndzoʁloʁ}{\papi{ mbrɤndzoʁloʁ}}}\markboth{mbrɤndzoʁloʁ}{}\classe{n}
\begin{définition}\fra auge\end{définition}
\begin{définition}\cmn 马槽\end{définition}
\end{entrée}

\begin{entrée}
\vedette{\hypertarget{Ⓔmbrɤrɟɯɣ}{\papi{ mbrɤrɟɯɣ}}}\markboth{mbrɤrɟɯɣ}{}
\classe{n}
\begin{définition}\fra course de cheval\end{définition}
\begin{définition}\cmn 赛马\end{définition}
\begin{exemple}\jya mbrɤrɟɯɣ βzu-j\cmn 我们赛马\end{exemple}
\begin{relation-sémantique}\confer{
\hyperlink{Ⓔnɯmbrɤrɟɯɣ}{\textit{ \papi{nɯmbrɤrɟɯɣ}}}
}\end{relation-sémantique}\end{entrée}

\begin{entrée}
\vedette{\hypertarget{Ⓔmbrɤsɤm}{\papi{ mbrɤsɤm}}}\markboth{mbrɤsɤm}{}\classe{n}
\begin{définition}\fra vannerie en forme de cuve utilisée pour faire sécher les grains\end{définition}
\begin{définition}\cmn 放在房顶用来晒粮食,竹子编成的盆型的簸箕\end{définition}
\end{entrée}

\begin{entrée}
\vedette{\hypertarget{Ⓔmbrɤsno}{\papi{ mbrɤsno}}}\markboth{mbrɤsno}{}
\classe{n}
\begin{définition}\fra selle\end{définition}
\begin{définition}\cmn 马鞍\end{définition}\end{entrée}

\begin{entrée}
\vedette{\hypertarget{Ⓔmbrɤstshi}{\papi{ mbrɤstshi}}}\markboth{mbrɤstshi}{}\classe{n}
\begin{définition}\fra gruau de riz\end{définition}
\begin{définition}\cmn 粥;稀饭\end{définition}
\end{entrée}

\begin{entrée}
\vedette{\hypertarget{Ⓔmbrɤt}{\papi{ mbrɤt}}}\markboth{mbrɤt}{}\classe{vi}
\paradigme{\textit{dir :} \jya nɯ-}
\begin{définition}\ 
\begin{déclaration}\grammar{acaus}\end{déclaration}\end{définition}
\begin{définition}\fra se casser, se couper (corde, fil)\end{définition}
\begin{définition}\cmn 断(线)\end{définition}
\begin{exemple}\jya tɤ-ri ɲo-mbrɤt\cmn 线断了\end{exemple}
\begin{exemple}\jya tɯmbri ɲɤ-mbrɤt\cmn 绳子断了\end{exemple}
\begin{exemple}\jya jiɕqha nɯ-mbrɤt loβ !\cmn 刚才电话断了!\end{exemple}
\begin{relation-sémantique}\confer{
\hyperlink{Ⓔprɤt}{\textit{ \papi{prɤt}}}
}\end{relation-sémantique}\end{entrée}

\begin{entrée}
\vedette{\hypertarget{Ⓔmbrɤtaʁ}{\papi{ mbrɤtaʁ}}}\markboth{mbrɤtaʁ}{}\classe{adv}
\begin{définition}\fra à cheval\end{définition}
\begin{définition}\cmn 马背上\end{définition}
\begin{exemple}\jya mbrɤtaʁ cha-a\cmn 我会骑马\end{exemple}
\begin{relation-sémantique}\confer{
\hyperlink{ⒺmbroⒽ1}{\textit{ \papi{mbro}}}
}\end{relation-sémantique}
\begin{relation-sémantique}\confer{
\hyperlink{ⒺtaʁⒽ3}{\textit{ \papi{taʁ3}}}
}\end{relation-sémantique}\end{entrée}

\begin{entrée}
\vedette{\hypertarget{Ⓔmbrɤz}{\papi{ mbrɤz}}}\markboth{mbrɤz}{}
\classe{n}
\begin{définition}\fra riz\end{définition}
\begin{définition}\cmn 米
\begin{déclaration} \étymologie{\papi{ⁿbras}}\end{déclaration}\end{définition}\end{entrée}

\begin{entrée}
\vedette{\hypertarget{Ⓔmbre}{\papi{ mbre}}}\markboth{mbre}{}\classe{vi}
\begin{définition}\fra auspicieux (prédiction)\end{définition}
\begin{définition}\cmn 吉祥(预兆)\end{définition}
\begin{exemple}\jya a-mphrɯmɯ ɲɯ-mbre\cmn 算的卦很吉祥\end{exemple}\end{entrée}

\begin{entrée}
\vedette{\hypertarget{ⒺmbriⒽ1}{\papi{ mbri}}}\markboth{mbri}{}\homonyme{1}
\classe{vi}
\paradigme{\textit{dir :} \jya tɤ-}
\begin{définition}\fra fort (bruit), crier\end{définition}
\begin{définition}\cmn 响;叫\end{définition}
\begin{exemple}\jya mbɣɯrloʁ ɲɯ-mbri\cmn 打雷了\end{exemple}
\begin{exemple}\jya pɣɤtɕɯ to-mbri\cmn 鸟叫了\end{exemple}
\begin{exemple}\jya ɯ-rmi ɲɤ-mbri (=to-caʁ)\cmn 他出名了\end{exemple}
\begin{relation-sémantique}\confer{
\hyperlink{ⒺʑmbriⒽ1}{\textit{ \papi{ʑmbri1}}}
}\end{relation-sémantique}\end{entrée}

\begin{entrée}
\vedette{\hypertarget{ⒺmbriⒽ2}{\papi{ mbri}}}\markboth{mbri}{}\homonyme{2}\classe{vi}
\paradigme{\textit{dir :} \jya thɯ-}
\paradigme{\textit{dir :} \jya nɯ-}
\begin{définition}\ 
\begin{déclaration}\grammar{acaus}\end{déclaration}\end{définition}
\begin{définition}\fra se déchirer soudainement (habit)\end{définition}
\begin{définition}\cmn 烂掉(衣服)
\begin{déclaration}\use{衣服穿了很久就突然撕破}\end{déclaration}\end{définition}
\begin{exemple}\jya tɯ-ŋga cho-mbri\end{exemple}
\begin{exemple}\jya tɯ-ŋga ɲɤ-mbri\cmn 衣服破了\end{exemple}
\begin{relation-sémantique}\confer{
\hyperlink{ⒺpriⒽ1}{\textit{ \papi{pri1}}}
}\end{relation-sémantique}\end{entrée}

\begin{entrée}
\vedette{\hypertarget{ⒺmbroⒽ2}{\papi{ mbro}}}\markboth{mbro}{}\homonyme{2}
\classe{n}
\begin{définition}\fra cheval\end{définition}
\begin{définition}\cmn 马\end{définition}
\begin{relation-sémantique}\confer{
\hyperlink{Ⓔmbrɤtaʁ}{\textit{ \papi{mbrɤtaʁ}}}
}\end{relation-sémantique}
\begin{relation-sémantique}\confer{
\hyperlink{Ⓔnɯmbrɤpɯ}{\textit{ \papi{nɯmbrɤpɯ}}}
}\end{relation-sémantique}\end{entrée}

\begin{entrée}
\vedette{\hypertarget{ⒺmbroⒽ1}{\papi{ mbro}}}\markboth{mbro}{}\homonyme{1}\classe{vs}
\paradigme{\textit{dir :} \jya tɤ-}
\begin{définition}\fra haut\end{définition}
\begin{définition}\cmn 高\end{définition}
\begin{exemple}\jya ɯ-phoŋbu ɲɯ-mbro\cmn 他身材很高\end{exemple}
\begin{exemple}\jya aʑo staʁnɤ tɤ-mbro\cmn 他比我高了\end{exemple}
\begin{exemple}\jya nɤʑo jamar tɤ-mbro\cmn 他长得跟你一样高了\end{exemple}
\begin{relation-sémantique}\antonyme{
\hyperlink{ⒺmbɤrⒽ1}{\textit{ \papi{mbɤr1}}}
}\end{relation-sémantique}\begin{sous-entrée}
\vedette{\hypertarget{}{\papi{ ɣɤmbro}}}\markboth{ɣɤmbro}{}\classe{vt}
\paradigme{\textit{dir :} \jya tɤ-}
\begin{définition}\fra augmenter\end{définition}
\begin{définition}\cmn 提高\end{définition}
\begin{exemple}\jya nɤ-@xuetang tu-ɣɤmbrɤm ɯ́-cha\cmn 会不会使你的血糖升高?\end{exemple}
\end{sous-entrée}\end{entrée}

\begin{entrée}
\vedette{\hypertarget{Ⓔmbrolu}{\papi{ mbrolu}}}\markboth{mbrolu}{}\classe{n}
\begin{définition}\fra année du cheval\end{définition}
\begin{définition}\cmn 马年\end{définition}
\end{entrée}

\begin{entrée}
\vedette{\hypertarget{Ⓔmbrondza}{\papi{ mbrondza}}}\markboth{mbrondza}{}\classe{n}
\begin{définition}\fra nourriture pour cheval\end{définition}
\begin{définition}\cmn 马料\end{définition}
\end{entrée}

\begin{entrée}
\vedette{\hypertarget{Ⓔmbroqa}{\papi{ mbroqa}}}\markboth{mbroqa}{}\classe{n}
\begin{définition}\fra sabot\end{définition}
\begin{définition}\cmn 马蹄,马的脚\end{définition}
\end{entrée}

\begin{entrée}
\vedette{\hypertarget{Ⓔmbrosta}{\papi{ mbrosta}}}\markboth{mbrosta}{}\classe{n}
\begin{définition}\fra écurie\end{définition}
\begin{définition}\cmn 马厩\end{définition}
\begin{relation-sémantique}\synonyme{
\hyperlink{Ⓔrtakhaŋ}{\textit{ \papi{rtakhaŋ}}}
}\end{relation-sémantique}\end{entrée}

\begin{entrée}
\vedette{\hypertarget{Ⓔmbroχkɕi}{\papi{ mbroχkɕi}}}\markboth{mbroχkɕi}{}\classe{n}
\begin{définition}\fra chien tibétain\end{définition}
\begin{définition}\cmn 藏獒\end{définition}\end{entrée}

\begin{entrée}
\vedette{\hypertarget{Ⓔmbroχpa}{\papi{ mbroχpa}}}\markboth{mbroχpa}{}\classe{n}
\begin{définition}\fra nomades\end{définition}
\begin{définition}\cmn 牧民
\begin{déclaration} \étymologie{\papi{ⁿbrog.pa}}\end{déclaration}\end{définition}
\end{entrée}

\begin{entrée}
\vedette{\hypertarget{Ⓔmbrozga}{\papi{ mbrozga}}}\markboth{mbrozga}{}\classe{n}
\begin{définition}\fra selle\end{définition}
\begin{définition}\cmn 马鞍\end{définition}
\end{entrée}

\begin{entrée}
\vedette{\hypertarget{Ⓔmbrɯɣlu}{\papi{ mbrɯɣlu}}}\markboth{mbrɯɣlu}{}\classe{n}
\begin{définition}\fra année du dragon\end{définition}
\begin{définition}\cmn 龙年
\begin{déclaration} \étymologie{\papi{ⁿbrug.lo}}\end{déclaration}\end{définition}
\end{entrée}

\begin{entrée}
\vedette{\hypertarget{Ⓔmbrɯtɕɯ}{\papi{ mbrɯtɕɯ}}}\markboth{mbrɯtɕɯ}{}
\classe{n}
\begin{définition}\fra couteau\end{définition}
\begin{définition}\cmn 刀\end{définition}
\begin{exemple}\jya mbrɯtɕɯ ɯ-ɕɣa\cmn 刀刃\end{exemple}
\begin{exemple}\jya mbrɯtɕɯ ɯ-pɯ\cmn 小刀\end{exemple}\end{entrée}

\begin{entrée}
\vedette{\hypertarget{Ⓔmbɯlwa}{\papi{ mbɯlwa}}}\markboth{mbɯlwa}{}\classe{n}
\begin{définition}\fra salaire d'un lama\end{définition}
\begin{définition}\cmn 和尚的工资
\begin{déclaration} \étymologie{\papi{ⁿbul.ba}}\end{déclaration}\end{définition}\end{entrée}

\begin{entrée}
\vedette{\hypertarget{Ⓔmbɯmχtɤr}{\papi{ mbɯmχtɤr}}}\markboth{mbɯmχtɤr}{}
\classe{n}
\begin{définition}\fra cent mille\end{définition}
\begin{définition}\cmn 100000
\begin{déclaration} \étymologie{\papi{ⁿbum.tʰer}}\end{déclaration}\end{définition}\end{entrée}

\begin{entrée}
\vedette{\hypertarget{Ⓔmbɯrlɤn}{\papi{ mbɯrlɤn}}}\markboth{mbɯrlɤn}{}
\classe{n}
\begin{définition}\fra rabot\end{définition}
\begin{définition}\cmn 刨
\begin{déclaration} \étymologie{\papi{ⁿbur.len}}\end{déclaration}\end{définition}
\begin{relation-sémantique}\confer{
\hyperlink{Ⓔnɯmbɯrlɤn}{\textit{ \papi{nɯmbɯrlɤn}}}
}\end{relation-sémantique}\end{entrée}

\begin{entrée}
\vedette{\hypertarget{Ⓔmbɯrlɤnndoʁ}{\papi{ mbɯrlɤnndoʁ}}}\markboth{mbɯrlɤnndoʁ}{}\classe{n}
\begin{définition}\fra copeaux\end{définition}
\begin{définition}\cmn 刨花\end{définition}
\end{entrée}

\begin{entrée}
\vedette{\hypertarget{Ⓔmbɯsɯt}{\papi{ mbɯsɯt}}}\markboth{mbɯsɯt}{}\classe{n}
\begin{définition}\fra râpeuse\end{définition}
\begin{définition}\cmn 丝丝(擦成)\end{définition}
\begin{exemple}\jya lɤpɯɣ mbɯsɯt thɯ-lat-a\cmn 我把萝卜擦成丝丝\end{exemple}\end{entrée}

\begin{entrée}
\vedette{\hypertarget{Ⓔmbɯt}{\papi{ mbɯt}}}\markboth{mbɯt}{}
\classe{vi}
\paradigme{\textit{dir :} \jya thɯ-}
\paradigme{\textit{dir :} \jya pɯ-}
\begin{définition}\fra s’écrouler\end{définition}
\begin{définition}\cmn 塌毁下来;垮下来\end{définition}
\begin{exemple}\jya ŋgɤm ki mbɯt ɲɯ-ŋu\cmn 这个土坡快要塌下来\end{exemple}
\begin{exemple}\jya tʂu ɲɯ-mbɯt pɯ-mto-t-a\cmn 我看到路在塌方\end{exemple}
\begin{exemple}\jya kha ɲɯ-mbɯt\cmn 房子在垮\end{exemple}
\begin{exemple}\jya tʂu cho-mbɯt\cmn 路塌下来了\end{exemple}\end{entrée}

\begin{entrée}
\vedette{\hypertarget{Ⓔmbuz}{\papi{ mbuz}}}\markboth{mbuz}{}
\classe{vi.nh}
\paradigme{\textit{dir :} \jya tɤ-}
\paradigme{\textit{dir :} \jya thɯ-}
\begin{définition}\fra déborder\end{définition}
\begin{définition}\cmn 溢出来\end{définition}
\begin{exemple}\jya tʂha pjɤ-mbuz\cmn 茶溢出来了\end{exemple}
\begin{exemple}\jya tɯ-ci cho-mbuz\cmn 水溢出来了\end{exemple}\begin{sous-entrée}
\vedette{\hypertarget{}{\papi{ sɯɣmbɯz}}}\markboth{sɯɣmbɯz}{}\classe{vt}
\paradigme{\textit{dir :} \jya tɤ-}
\paradigme{\textit{dir :} \jya thɯ-}
\begin{définition}\fra laisser ... déborder\end{définition}
\begin{définition}\cmn 令……溢出来\end{définition}
\begin{exemple}\jya tɤ-lu ɲɤ-nɯ-jmɯt-a tɕe tɤ-sɯɣmbuz-a\cmn 我把(在煮)的牛奶忘了,就溢出来了\end{exemple}
\end{sous-entrée}\end{entrée}

\begin{entrée}
\vedette{\hypertarget{Ⓔmchɯn}{\papi{ mchɯn}}}\markboth{mchɯn}{}
\classe{vt}
\paradigme{\textit{dir :} \jya tɤ-}
\begin{définition}\fra percevoir la vraie nature de quelqu'un (pouvoir de sprulsku)\end{définition}
\begin{définition}\cmn 活佛看穿别人的本性
\begin{déclaration} \étymologie{\papi{mkʰʲen}}\end{déclaration}\end{définition}
\begin{exemple}\jya sprɯskɯ kɯ kɯ-mchɯn ɕti\cmn 
活佛能看穿人的本性(\stylefv{kɯ}-是宾语泛指标记)
\end{exemple}\end{entrée}

\begin{entrée}
\vedette{\hypertarget{Ⓔmchɯnba}{\papi{ mchɯnba}}}\markboth{mchɯnba}{}\classe{n}
\begin{définition}\fra don de prédiction\end{définition}
\begin{définition}\cmn 预知力
\begin{déclaration} \étymologie{\papi{mkʰʲen.pa}}\end{déclaration}\end{définition}
\begin{exemple}\jya ɯ-mchɯnba ɣɤʑu\cmn 他有预知力\end{exemple}\end{entrée}

\begin{entrée}
\vedette{\hypertarget{Ⓔmúcin}{\papi{ múcin}}}\markboth{múcin}{}\classe{adv}
\begin{définition}\fra pas du tout\end{définition}
\begin{définition}\cmn 根本没有\end{définition}
\end{entrée}

\begin{entrée}
\vedette{\hypertarget{Ⓔmu,cɯɣ}{\papi{ mu,cɯɣ}}}\markboth{mu,cɯɣ}{}\classe{vi}
\begin{définition}\fra vivre constament dans la peur\end{définition}
\begin{définition}\cmn 提心吊胆,恐慌不安\end{définition}
\begin{exemple}\jya kɤ-mu kɤ-cɯɣ kɤ-rɤʑi kɯ-ra ɲɯ-sɤdɯɣ\cmn 整天提心吊胆的感觉不好受\end{exemple}
\begin{exemple}\jya ɲɯ-mu ɲɯ-cɯɣ ʑo\cmn 他提心吊胆\end{exemple}
\begin{exemple}\jya ma-tɯ-mu ma-tɯ-cɯɣ ma mɤ-ʁdɯɣ nɤ\cmn 你不要这样提心吊胆,不会有是的\end{exemple}
\begin{relation-sémantique}\ComponentA{\classe{vi}
\hyperlink{ⒺmuⒽ1}{\textit{ \papi{mu}}}
}\end{relation-sémantique}
\begin{relation-sémantique}\ComponentB{\classe{vi}
 \papi{cɯɣ}
}\end{relation-sémantique}\end{entrée}

\begin{entrée}
\vedette{\hypertarget{Ⓔmcɯphɯt}{\papi{ mcɯphɯt}}}\markboth{mcɯphɯt}{}\classe{n}
\begin{définition}\fra crachat\end{définition}
\begin{définition}\cmn (吐的)口水\end{définition}
\begin{exemple}\jya mcɯphɯt thɯ-lat-a\cmn 我吐了口水\end{exemple}
\begin{relation-sémantique}\confer{
 \papi{sɯmcɯphɯt}
}\end{relation-sémantique}\end{entrée}

\begin{entrée}
\vedette{\hypertarget{Ⓔmcɯrɯβrɯβ}{\papi{ mcɯrɯβrɯβ}}}\markboth{mcɯrɯβrɯβ}{}
\classe{n}
\begin{définition}\fra personne qui bave tout le temps\end{définition}
\begin{définition}\cmn 总是流口水的人\end{définition}
\begin{exemple}\jya nɤʑo mcɯrɯβrɯβ ki\cmn 你这个爱流口水的家伙\end{exemple}
\begin{relation-sémantique}\confer{
\hyperlink{Ⓔtɯ-mci}{\textit{ \papi{tɯ-mci}}}
}\end{relation-sémantique}
\begin{relation-sémantique}\confer{
 \papi{rɯβrɯβ}
}\end{relation-sémantique}\end{entrée}

\begin{entrée}
\vedette{\hypertarget{Ⓔmda}{\papi{ mda}}}\markboth{mda}{}\classe{vi.nh}
\paradigme{\textit{dir :} \jya tɤ-}
\begin{définition}\fra arriver au moment de\end{définition}
\begin{définition}\cmn 到时间\end{définition}
\begin{exemple}\jya pɤjkhu mɯ́j-mda\cmn 还没有到时间\end{exemple}
\begin{exemple}\jya saχsɯ pɤjkhu mɯ́j-mda\cmn 还没有到中午餐\end{exemple}
\begin{exemple}\jya ʑa qanɯ ɲɯ-ŋu tɕe, kɤ-nɯɕe mda\cmn 快天黑了,该回去了\end{exemple}
\begin{exemple}\jya tɤŋe tɤ-anɯri tɕe kɤ-nɯɕe mda\cmn 太阳落山了,该回去\end{exemple}
\begin{exemple}\jya nɤj tɯ-nɯɕe mda\cmn 你该回去了\end{exemple}
\begin{exemple}\jya a-βra tɤ-mda\cmn 轮到我了\end{exemple}
\begin{exemple}\jya tɤ-rɤku kɯ-mda\cmn 成熟的庄稼\end{exemple}
\begin{exemple}\jya tɤ-rɤku kɤ-phɯt tɤ-mda\cmn 收割的时间到了\end{exemple}
\begin{exemple}\jya kɤ-nɯftɕaka tɤ-mda\cmn 到了准备(晚餐)的时间\end{exemple}
\begin{exemple}\jya nɤ-tɯ-ci (kɤ-rku) ɯ-ɲɯ́-mda?\cmn 你杯子还有没有水?\end{exemple}
\begin{relation-sémantique}\confer{
\hyperlink{Ⓔmasɤmdɤla}{\textit{ \papi{masɤmdɤla}}}
}\end{relation-sémantique}\end{entrée}

\begin{entrée}
\vedette{\hypertarget{Ⓔmdandzɯn}{\papi{ mdandzɯn}}}\markboth{mdandzɯn}{}\classe{n}
\begin{définition}\fra grosse perle du chapelet\end{définition}
\begin{définition}\cmn 玛尼珠最大的珠子
\end{définition}
\begin{exemple}\jya mphruwa nɯ tɯ-mke pjɯ́-wɣ-nɯ-rʁe tɕe, ɯ-mdandzɯn nɯ tɯ-ʁɤri ɯ-stu ɲɯ́-wɣ-znɤtɯɣ ŋu\cmn 脖子上戴玛尼珠的时候,大珠子要放在前面\end{exemple}\end{entrée}

\begin{entrée}
\vedette{\hypertarget{Ⓔmdarɯ}{\papi{ mdarɯ}}}\markboth{mdarɯ}{}\classe{n}
\begin{définition}\fra damaru\end{définition}
\begin{définition}\cmn 鼗
\begin{déclaration} \étymologie{\papi{ɖa.ma.ru}}\end{déclaration}\end{définition}\end{entrée}

\begin{entrée}
\vedette{\hypertarget{Ⓔmdaʁʑɯɣ}{\papi{ mdaʁʑɯɣ}}}\markboth{mdaʁʑɯɣ}{}
\classe{n}
\begin{définition}\fra arc et flèches\end{définition}
\begin{définition}\cmn 弓箭
\begin{déclaration} \étymologie{\papi{mda.gʑu}}\end{déclaration}\end{définition}\end{entrée}

\begin{entrée}
\vedette{\hypertarget{Ⓔmdi}{\papi{ mdi}}}\markboth{mdi}{}
\classe{vs}
\begin{définition}\fra être au complet, être entièrement là\end{définition}
\begin{définition}\cmn 全部;齐全\end{définition}
\begin{exemple}\jya laχtɕha ɲɯ-mdi\cmn 东西齐全\end{exemple}
\begin{exemple}\jya nɤ-laχtɕha kɯ-mdi ʑo tɤ-wum\cmn 你收拾你所有的东西吧!\end{exemple}
\begin{relation-sémantique}\confer{
\hyperlink{Ⓔmdoʁmdi}{\textit{ \papi{mdoʁmdi}}}
}\end{relation-sémantique}\end{entrée}

\begin{entrée}
\vedette{\hypertarget{Ⓔmdoʁmdi}{\papi{ mdoʁmdi}}}\markboth{mdoʁmdi}{}
\classe{vs}
\paradigme{\textit{dir :} \jya tɤ-}
\begin{définition}\ 
\begin{déclaration}\grammar{rdpl}\end{déclaration}\end{définition}
\begin{définition}\fra entier, complet (un objet)\end{définition}
\begin{définition}\cmn 完整(指一个物体,不能表示零散的东西聚齐)\end{définition}
\begin{exemple}\jya tɯ-ndʐi ɲɯ-mdoʁmdi\cmn 皮子是完整的\end{exemple}
\begin{relation-sémantique}\confer{
\hyperlink{Ⓔmdi}{\textit{ \papi{mdi}}}
}\end{relation-sémantique}\end{entrée}

\begin{entrée}
\vedette{\hypertarget{Ⓔmdɯ}{\papi{ mdɯ}}}\markboth{mdɯ}{}\classe{vi}
\paradigme{\textit{dir :} \jya thɯ-}\acception{1}
\begin{définition}\fra vivre jusqu'à\end{définition}
\begin{définition}\cmn 到达(年龄)\end{définition}
\begin{exemple}\jya ɯ-lɯz nɯ cho-mdɯ\cmn 他年龄大了\end{exemple}
\begin{exemple}\jya kɯrcɤsqi (pɤrme) cho-mdɯ\cmn 他活到80岁\end{exemple}\acception{2}
\begin{définition}\fra être assez long (fil)\end{définition}
\begin{définition}\cmn 够长(线;绳子)\end{définition}
\begin{exemple}\jya tɤ-ri nɯ kú-wɣ-lɤt qhe ɲɯ-zri tɕe kuchɯ-rkɯ mɤɕtʂa ku-mdɯ ɲɯ-cha\cmn 这根线很长,牵过去可以牵到那边的墙角\end{exemple}
\begin{relation-sémantique}\synonyme{
\hyperlink{ⒺɕaβⒽ2}{\textit{ \papi{ɕaβ2}}}
}\end{relation-sémantique}\end{entrée}

\begin{entrée}
\vedette{\hypertarget{Ⓔmdʐɯɕɯɣ}{\papi{ mdʐɯɕɯɣ}}}\markboth{mdʐɯɕɯɣ}{}\classe{n}
\begin{définition}\fra punaise\end{définition}
\begin{définition}\cmn 臭虫
\begin{déclaration} \étymologie{\papi{ⁿdre.ɕig}}\end{déclaration}\end{définition}
\end{entrée}

\begin{entrée}
\vedette{\hypertarget{Ⓔmdɯnri}{\papi{ mdɯnri}}}\markboth{mdɯnri}{}\classe{n}
\begin{définition}\fra cérémonie effectuée lorsqu'un membre de la famille part en voyage\end{définition}
\begin{définition}\cmn 家里有人出行的时候,为了保佑他安全顺利而念的经\end{définition}
\begin{relation-sémantique}\synonyme{
\hyperlink{Ⓔndʐɯnɬa}{\textit{ \papi{ndʐɯnɬa}}}
}\end{relation-sémantique}
\end{entrée}

\begin{entrée}
\vedette{\hypertarget{Ⓔmdɯt}{\papi{ mdɯt}}}\markboth{mdɯt}{}
\classe{vt}
\paradigme{\textit{dir :} \jya thɯ-}
\begin{définition}\fra vouloir de tout son cœur\end{définition}
\begin{définition}\cmn 一心想着(有把握会成功)\end{définition}
\begin{exemple}\jya a-ɣe kɤ-zrɤβzjoz kɯ-ra nɯ chɯ-mdɯt-a ʑo ɕti\cmn 我一心想着让孙子继续读书\end{exemple}
\begin{exemple}\jya aʑo a-phoŋbu kɤ-sɯɣndʐɯm nɯ chɯ-mdɯt-a ʑo ɕti\cmn 我一心想锻炼身体\end{exemple}
\begin{exemple}\jya aʑo kɯrɯ-skɤt kɤ-βzjoz nɯ chɯ-mdɯt-a ʑo ɕti\cmn 我一心想着学藏语\end{exemple}
\end{entrée}

\begin{entrée}
\vedette{\hypertarget{Ⓔmdɯtpa}{\papi{ mdɯtpa}}}\markboth{mdɯtpa}{}\classe{n}
\begin{définition}\fra nœud que fait un lama en attachant un bsrungs\end{définition}
\begin{définition}\cmn 喇嘛打护身线的结
\begin{déclaration} \étymologie{\papi{mdud.pa}}\end{déclaration}\end{définition}
\end{entrée}

\begin{entrée}
\vedette{\hypertarget{Ⓔmdzadi}{\papi{ mdzadi}}}\markboth{mdzadi}{}\classe{n}
\begin{définition}\fra puce\end{définition}
\begin{définition}\cmn 跳蚤\end{définition}
\begin{exemple}\jya mdzadi wuma kɯ-sɤndza ɲɯ-ŋu sɤɣdɯɣ ŋotɕu χtɕɯrɯpa ku-kɯ-rŋgɯ kɯnɤ tɯ-phe ju-zɣɯt ɕti\cmn 跳蚤咬人很不舒服,你在哪里睡觉,它都会跟着你的\end{exemple}\end{entrée}

\begin{entrée}
\vedette{\hypertarget{Ⓔmdzadikɤdɤɣ}{\papi{ mdzadikɤdɤɣ}}}\markboth{mdzadikɤdɤɣ}{}
\classe{n}
\begin{définition}\fra espèce d'arbrisseau\end{définition}
\begin{définition}\cmn 灌木的一种\end{définition}
\begin{exemple}\jya mdzadikɤdɤɣ nɯ si kɯ-mbɯ-mbɤr ci ŋu, ɯ-di wuma ʑo mnɤm, ɯ-ru kɯ-pɣi tsa ŋu, ɯ-jwaʁ kɯ-tɕɤr kɯ-rɲɟi tsa ŋu, ɯ-mɯntoʁ ɯ-tshɯɣa phaʁrzi kɯ-fse tɕe tɯ-pɕoʁ ci ma kɯ-me kɯ-ndɯ-ndɯβ kɯ-dɯ-dɤn kɯ-wɣrum ŋu\cmn 
\stylefv{mdzadikɤdɤɣ}是矮小的树,味道很浓,树干是灰色的,叶子细长,花的形状像牙刷一样,只长在一面,长得小而密,是白色的。
\end{exemple}\end{entrée}

\begin{entrée}
\vedette{\hypertarget{Ⓔmdzar}{\papi{ mdzar}}}\markboth{mdzar}{}
\classe{vt}
\paradigme{\textit{dir :} \jya pɯ-}
\begin{définition}\ 
\begin{déclaration}\grammar{caus}\end{déclaration}\end{définition}
\begin{définition}\fra égoutter, essorer\end{définition}
\begin{définition}\cmn 滗干
\begin{déclaration}\grammar{caus}\end{déclaration}
\end{définition}
\begin{relation-sémantique}\confer{
\hyperlink{Ⓔndzar}{\textit{ \papi{ndzar}}}
}\end{relation-sémantique}
\begin{relation-sémantique}\synonyme{
\hyperlink{Ⓔsɯɣndzar}{\textit{ \papi{sɯɣndzar}}}
}\end{relation-sémantique}\end{entrée}

\begin{entrée}
\vedette{\hypertarget{Ⓔmdzɤz}{\papi{ mdzɤz}}}\markboth{mdzɤz}{}
\classe{vs}
\paradigme{\textit{dir :} \jya tɤ-}
\begin{définition}\fra honorable\end{définition}
\begin{définition}\cmn 体面;好听
\begin{déclaration} \étymologie{\papi{mdzes}}\end{déclaration}\end{définition}
\begin{exemple}\jya ɯ-sɯm pjɤ-sɤzdɯɣ ri, kɯ-mdzɤz to-βzu tɕe ɯ-rŋa ɲɤ-nɤre\cmn 他虽然心里难受,他为了体面做出笑容\end{exemple}
\begin{exemple}\jya nɯ tu-kɯ-ti tɕe ɲɯ-mdzɤz\cmn 这样说好听一点\end{exemple}
\begin{exemple}\jya nɤ-sɯm mɯ-pɯ-anɯri kɯnɤ, kɯ-mdzɤz tɤ-βze ma ma-tɤ-tɯ-nɯjʁo\cmn 你即使心里不高兴,你要做得好看一些,不要骂人\end{exemple}\begin{sous-entrée}
\vedette{\hypertarget{}{\papi{ znɤmdzɤz}}}\markboth{znɤmdzɤz}{}\classe{vt}
\begin{définition}\fra ne pas oser faire un reproche\end{définition}
\begin{définition}\cmn 打不开情面(否定式)\end{définition}
\begin{exemple}\jya mɯ́j-znɤmdzɤz\cmn 他不好意思说\end{exemple}
\begin{exemple}\jya to-nɤma nɯ mɯ́j-nɤpe ri, mɯ́j-pe kɤ-ti nɯ mɯ́j-znɤmdzɤz\cmn 他觉得他工作得不好,但是不好意思说\end{exemple}
\end{sous-entrée}\end{entrée}

\begin{entrée}
\vedette{\hypertarget{Ⓔmdzukɤr}{\papi{ mdzukɤr}}}\markboth{mdzukɤr}{}\classe{n}
\begin{définition}\fra mdzo de couleur blanche\end{définition}
\begin{définition}\cmn 白色的犏牛
\begin{déclaration} \étymologie{\papi{mdzo.dkar}}\end{déclaration}\end{définition}\end{entrée}

\begin{entrée}
\vedette{\hypertarget{Ⓔmdzumɤr}{\papi{ mdzumɤr}}}\markboth{mdzumɤr}{}\classe{n}
\begin{définition}\fra mdzo de couleur marron claire\end{définition}
\begin{définition}\cmn 褐色的犏牛
\begin{déclaration} \étymologie{\papi{mdzo.dmar}}\end{déclaration}\end{définition}\end{entrée}

\begin{entrée}
\vedette{\hypertarget{Ⓔmdzoz}{\papi{ mdzoz}}}\markboth{mdzoz}{}
\classe{vt}
\paradigme{\textit{dir :} \jya nɯ-}
\paradigme{\textit{dir :} \jya tɤ-}
\begin{définition}\fra interdire, éviter\end{définition}
\begin{définition}\cmn 忌讳;禁止;回避
\begin{déclaration} \étymologie{\papi{mdzaŋs}}\end{déclaration}\end{définition}
\begin{exemple}\jya nɤki @chabei nɯ tɤ-mdzoz, a-mɤ-pɯ-ɴɢrɯ\cmn 小心茶杯,不要打破了\end{exemple}
\begin{exemple}\jya ɯʑo kɯ na-mdzoz\cmn 他避开了\end{exemple}
\begin{exemple}\jya kɯki to-rɯkɯmaʁ ri, aʑo nɯ-mdzoz-a tɕe mɯ-jɤ-ari-a\cmn 他们家里出了事故(死了人),我回避了他的葬礼,没有去\end{exemple}
\begin{exemple}\jya jiɕqha nɯ pjɤ-si, aʑo nɯ-mdzoz-a\cmn 那个人死了,我回避了他的葬礼\end{exemple}
\begin{exemple}\jya a-ɕqhe ɣɤʑu tɕe, cha ɲɯ-mdzoz-a ɲɯ-ntshi\cmn 我咳嗽,忌酒\end{exemple}
\begin{relation-sémantique}\confer{
\hyperlink{Ⓔtɯmdzoz}{\textit{ \papi{tɯmdzoz}}}
}\end{relation-sémantique}\end{entrée}

\begin{entrée}
\vedette{\hypertarget{Ⓔmdzurgi}{\papi{ mdzurgi}}}\markboth{mdzurgi}{}\classe{n}
\begin{définition}\ 
\begin{déclaration}\grammar{n.lieu}\end{déclaration}\end{définition}
\begin{définition}\fra Mdzorge\end{définition}
\begin{définition}\cmn 若尔盖\end{définition}\end{entrée}

\begin{entrée}
\vedette{\hypertarget{Ⓔmdzusŋun}{\papi{ mdzusŋun}}}\markboth{mdzusŋun}{}\classe{n}
\begin{définition}\fra yak hybride au pelage bariolé\end{définition}
\begin{définition}\cmn 杂色的犏牛(看起来是青色的)
\begin{déclaration} \étymologie{\papi{mdzo.sŋon}}\end{déclaration}\end{définition}\end{entrée}

\begin{entrée}
\vedette{\hypertarget{Ⓔmdzɯt}{\papi{ mdzɯt}}}\markboth{mdzɯt}{}
\classe{vi}\acception{1}
\begin{définition}\fra décider\end{définition}
\begin{définition}\cmn 决定,做主\end{définition}
\begin{exemple}\jya nɤʑo tɯ-mdzɯt\cmn 你说了算\end{exemple}
\begin{exemple}\jya aʑo mɤ-nɯ-mdzɯt-a\cmn 我做不了主\end{exemple}
\begin{exemple}\jya ji-kɯ-mdzɯt\cmn 我们的领导\end{exemple}\acception{2}
\begin{définition}\fra (pas) forcément\end{définition}
\begin{définition}\cmn (不)一定\end{définition}
\begin{exemple}\jya nɯ ʁo mɤ-mdzɯt\cmn 那个倒不一定\end{exemple}
\begin{relation-sémantique}\synonyme{
\hyperlink{Ⓔrɤmdzɯt}{\textit{ \papi{rɤmdzɯt}}}
}\end{relation-sémantique}\end{entrée}

\begin{entrée}
\vedette{\hypertarget{Ⓔmdzuzga}{\papi{ mdzuzga}}}\markboth{mdzuzga}{}
\classe{n}
\begin{définition}\fra attelage\end{définition}
\begin{définition}\cmn 驮架
\begin{déclaration} \étymologie{\papi{mdzo.sga}}\end{déclaration}
\end{définition}\end{entrée}

\begin{entrée}
\vedette{\hypertarget{Ⓔmdʑɤl}{\papi{ mdʑɤl}}}\markboth{mdʑɤl}{}\classe{vi}
\paradigme{\textit{dir :} \jya tɤ-}
\begin{définition}\fra faire un pèlerinage\end{définition}
\begin{définition}\cmn 朝圣
\begin{déclaration} \étymologie{\papi{mdʑal}}\end{déclaration}\end{définition}
\begin{exemple}\jya ɯʑo ɬasa ɕ-to-mdʑɤl\cmn 他去拉萨朝圣了\end{exemple}\end{entrée}

\begin{entrée}
\vedette{\hypertarget{ⒺmeⒽ2}{\papi{ me}}}\markboth{me}{}\homonyme{2}
\classe{postp}
\begin{définition}\fra que ce soit ... ou bien ...\end{définition}
\begin{définition}\cmn 不管……都\end{définition}
\begin{exemple}\jya aʑo me, nɤʑo me, ɯʑo me, kɤsɯfse ɕe-j ra\cmn 不管是我、你还是他,我们都要去\end{exemple}
\end{entrée}

\begin{entrée}
\vedette{\hypertarget{ⒺmeⒽ1}{\papi{ me}}}\markboth{me}{}\homonyme{1}\classe{vi}
\paradigme{\textit{dir :} \jya nɯ-}
\begin{définition}\fra not exist\end{définition}
\begin{définition}\cmn 不存在;没有\end{définition}
\begin{exemple}\jya aʑo mɤ-kɯ-pe ku-me-a\end{exemple}
\begin{exemple}\jya aʑɯɣ mɤ-kɯ-pe ku-me\cmn 我没有什么不好的\end{exemple}
\begin{exemple}\jya ɕkrɤz tɤ-me tɕe nɯnɯ xɕaj ɲɯ-βzu-nɯ sna\cmn 
没有青冈树的时候可以用\stylefv{xɕaj}来代替
\end{exemple}
\begin{exemple}\jya ɯʑo nɯ-me\cmn 他没有了(过世了)\end{exemple}
\begin{relation-sémantique}\antonyme{
\hyperlink{Ⓔtu}{\textit{ \papi{tu}}}
}\end{relation-sémantique}
\begin{relation-sémantique}\synonyme{
\hyperlink{Ⓔmaŋe}{\textit{ \papi{maŋe}}}
}\end{relation-sémantique}
\begin{relation-sémantique}\confer{
 \papi{tɤ-rca,me}
}\end{relation-sémantique}\end{entrée}

\begin{entrée}
\vedette{\hypertarget{Ⓔmgrɯn}{\papi{ mgrɯn}}}\markboth{mgrɯn}{}
\classe{vt}
\paradigme{\textit{dir :} \jya nɯ-}
\begin{définition}\ 
\begin{déclaration}\grammar{secondatif}\end{déclaration}\end{définition}
\begin{définition}\fra recevoir (un hôte), régaler (un hôte) avec\end{définition}
\begin{définition}\cmn 接待;款待
\begin{déclaration} \étymologie{\papi{mgron}}\end{déclaration}\end{définition}
\begin{exemple}\jya ɯʑo kɯ ɯ-zda na-mgrɯn tɕe, tʂha na-jtshi\cmn 他接待了客人,给他茶喝\end{exemple}
\begin{exemple}\jya aʑo ci ɲɯ-ta-mgrɯn\cmn 我请你吧\end{exemple}\begin{sous-entrée}
\vedette{\hypertarget{}{\papi{ sɤmgrɯn}}}\markboth{sɤmgrɯn}{} (\variante{rɤmgrɯn}) \classe{vi}
\paradigme{\textit{dir :} \jya nɯ-}
\begin{définition}\fra recevoir des hôtes\end{définition}
\begin{définition}\cmn 接待客人\end{définition}
\end{sous-entrée}\end{entrée}

\begin{entrée}
\vedette{\hypertarget{Ⓔmgɯ}{\papi{ mgɯ}}}\markboth{mgɯ}{}\classe{vt}
\paradigme{\textit{dir :} \jya nɯ-}
\begin{définition}\fra avoir confiance en, admirer\end{définition}
\begin{définition}\cmn 佩服;信任\end{définition}
\begin{exemple}\jya ɯʑo kɯ ɲɯ-tɯ́-wɣ-mgɯ\cmn 他很信任你\end{exemple}
\begin{exemple}\jya nɤʑo nɤ-ma tɤ-tɯ-nɤma-t nɯ ra ɲɯ-ta-mgɯ\cmn 我很相信你把工作做好了\end{exemple}\begin{sous-entrée}
\vedette{\hypertarget{}{\papi{ ʑɣɤmgɯ}}}\markboth{ʑɣɤmgɯ}{}\classe{vi}
\begin{définition}\fra avoir confiance en soi\end{définition}
\begin{définition}\cmn 自信\end{définition}
\begin{exemple}\jya nɯsthɯci nɤ-kɤ-cha kɯ-tu ci tɯ-ŋu ɕi kɯma, nɤ-tɯ-ʑɣɤmgɯ nɯ!\cmn 你是不是有那么大的本事,你这么自信\end{exemple}
\end{sous-entrée}\end{entrée}

\begin{entrée}
\vedette{\hypertarget{ⒺmiⒽ2}{\papi{ mi}}}\markboth{mi}{}\homonyme{2}\classe{n}
\begin{définition}\fra peuplier\end{définition}
\begin{définition}\cmn 杨树\end{définition}
\begin{exemple}\jya mi nɯ si kɯ-mbro kɯ-wxti ci ŋu, tɯ-ci kɯ-tu zɯ tu-ɬoʁ ŋu, wuma ʑo ɣɤ-wxti, ɯ-jwaʁ tɯ-sni ɯ-tshɯɣa kɯ-fse ŋu, ɯ-jwaʁ mba, mpɕu, kɯ-wxti tsa ɲɯ-βze cha, ɯ-mnɯ ɣɯ jwaʁ jndʐɤz, thɯ-do tɕe ɯ-jwaʁ ɲɯ-ndɯβ ŋu. mi ɣɯ ɯ-βri nɯ kɯ-ɤrŋi ŋu, ɯ-rtaʁ dɤn, ɯ-si nɯ kɤ-ntɕhoz sna, tɕe mpɕɤr ri khro mɤ-ngɯt.\cmn 白杨是高大的树种,生长在有水的地方,长得很快,叶子薄、光滑、长得比较大,树苗的叶子很大,树老叶子也就变小。树身是绿色的,枝桠多,木料可以使用,很美但不是很结实。\end{exemple}\end{entrée}

\begin{entrée}
\vedette{\hypertarget{ⒺmiⒽ1}{\papi{ mi}}}\markboth{mi}{}\homonyme{1}\classe{vs}
\paradigme{\textit{dir :} \jya pɯ-}
\paradigme{\textit{dir :} \jya nɯ-}
\paradigme{\textit{dir :} \jya thɯ-}
\begin{définition}\fra s’éteindre\end{définition}
\begin{définition}\cmn 灭\end{définition}
\begin{exemple}\jya smi ɲɤ-mi\cmn 火灭了\end{exemple}
\begin{exemple}\jya tɤtʂu ɲɤ-mi\cmn 灯灭了\end{exemple}
\begin{exemple}\jya ɣndʑɤβ pjɤ-mi\cmn 火灾灭了\end{exemple}
\begin{exemple}\jya pɯlthi kɤ-nɯt pɯ-tsu tɕe ʑaʑa mɯ́j-mi\cmn 引火线点燃以后久久不熄\end{exemple}
\begin{relation-sémantique}\confer{
\hyperlink{Ⓔɣɤmi}{\textit{ \papi{ɣɤmi}}}
}\end{relation-sémantique}\end{entrée}

\begin{entrée}
\vedette{\hypertarget{Ⓔmɟa}{\papi{ mɟa}}}\markboth{mɟa}{}
\classe{vt}\acception{1}
\paradigme{\textit{dir :} \jya nɯ-}
\paradigme{\textit{dir :} \jya kɤ-}
\paradigme{\textit{dir :} \jya \_}
\begin{définition}\fra prendre\end{définition}
\begin{définition}\cmn 拿;捡\end{définition}
\begin{exemple}\jya nɯ-mɟe\cmn 你拿一下\end{exemple}
\begin{exemple}\jya nɤkɤcu tɕhaχɯ kɤ-mɟe\cmn 你把那边的茶壶拿过来\end{exemple}\acception{2}
\begin{définition}\fra absorber\end{définition}
\begin{définition}\cmn 吸收\end{définition}
\begin{exemple}\jya a-phoŋbu kɯ smɤn mɯ́j-mɟe tɕe kɤ-nɯsmɤn mɯ́j-khɯ\cmn 我吸收不了药,没有办法把病治好\end{exemple}\acception{3}
\begin{définition}\fra recevoir (des mains de quelqu'un)\end{définition}
\begin{définition}\cmn 接到(从别人的手中)\end{définition}
\begin{exemple}\jya tɤ-scoz ci nɯ-mɟa-t-a\cmn 我接到了一封信\end{exemple}\acception{4}
\begin{définition}\fra ouvrir (couvercle)\end{définition}
\begin{définition}\cmn 揭开\end{définition}
\begin{exemple}\jya tɤ-fkaβ tɤ-mɟa-t-a (=tɤ-pɣaʁ-a)\cmn 我揭开了盖子\end{exemple}\begin{sous-entrée}
\vedette{\hypertarget{}{\papi{ sɯmɟa}}}\markboth{sɯmɟa}{}\classe{vt}
\begin{définition}\fra allumer\end{définition}
\begin{définition}\cmn 点燃\end{définition}
\begin{exemple}\jya smi tɤ-sɯmɟa-t-a\cmn 我点燃了火\end{exemple}
\begin{relation-sémantique}\confer{
\hyperlink{Ⓔnɯmɟa}{\textit{ \papi{nɯmɟa}}}
}\end{relation-sémantique}
\begin{relation-sémantique}\confer{
\hyperlink{Ⓔamɟɤkho}{\textit{ \papi{amɟɤkho}}}
}\end{relation-sémantique}
\begin{relation-sémantique}\confer{
\hyperlink{Ⓔtɯ-mɟaⒽ2}{\textit{ \papi{tɯ-mɟa2}}}
}\end{relation-sémantique}
\end{sous-entrée}\end{entrée}

\begin{entrée}
\vedette{\hypertarget{Ⓔmɟoʁra}{\papi{ mɟoʁra}}}\markboth{mɟoʁra}{}\classe{n}
\begin{définition}\fra corne où l'on met la poudre à fusil\end{définition}
\begin{définition}\cmn 火药角
\begin{déclaration} \étymologie{\papi{mgʲogs.rwa}}\end{déclaration}\end{définition}
\end{entrée}

\begin{entrée}
\vedette{\hypertarget{Ⓔmkɤɣɯr}{\papi{ mkɤɣɯr}}}\markboth{mkɤɣɯr}{}
\classe{n}
\begin{définition}\fra collier\end{définition}
\begin{définition}\cmn 项链\end{définition}\end{entrée}

\begin{entrée}
\vedette{\hypertarget{ⒺmkhuⒽ1Ⓗ1}{\papi{ mkhu}}}\markboth{mkhu}{}\homonyme{1}\classe{vs}
\begin{définition}\fra être nécessiteux\end{définition}
\begin{définition}\cmn 缺吃少穿
\begin{déclaration} \étymologie{\papi{mkʰo}}\end{déclaration}\end{définition}
\begin{exemple}\jya jiɕqha nɯ kɯ-mkhu ci ɕti\cmn 那个人缺吃少穿\end{exemple}
\begin{exemple}\jya tɤ-mkhu\cmn 他要了(别人给他东西)\end{exemple}
\begin{exemple}\jya ɯ-pɕawtsɯ maŋe, rŋɯl ɲɯ-mkhu\cmn 他没有钱,很缺钱\end{exemple}
\begin{exemple}\jya tɯ-ŋga ɲɯ-mkhu\cmn 他缺衣服\end{exemple}\begin{sous-entrée}
\vedette{\hypertarget{}{\papi{ mkhu}}}\markboth{mkhu}{}\classe{vi}
\paradigme{\textit{dir :} \jya tɤ-}
\begin{définition}\fra réclamer\end{définition}
\begin{définition}\cmn 向别人要\end{définition}
\begin{exemple}\jya laχtɕha ci pjɤ-tu tɕe ɲɤ-nɯsŋom tɕe to-mkhu\cmn 因为想得到那个东西,他向别人要了\end{exemple}
\end{sous-entrée}\begin{sous-entrée}
\vedette{\hypertarget{}{\papi{ znɤmkhɯmkhu}}}\markboth{znɤmkhɯmkhu}{}\classe{vs}
\begin{définition}\fra réclamer des choses\end{définition}
\begin{définition}\cmn 向别人要东西\end{définition}
\begin{exemple}\jya a-ɕki ɲɯ-znɤmkhɯmkhu\cmn 他向我要了东西\end{exemple}
\end{sous-entrée}\end{entrée}

\begin{entrée}
\vedette{\hypertarget{Ⓔmkhɤrmaŋ}{\papi{ mkhɤrmaŋ}}}\markboth{mkhɤrmaŋ}{}
\classe{n}
\begin{définition}\fra peuple\end{définition}
\begin{définition}\cmn 百姓
\begin{déclaration} \étymologie{\papi{ⁿkʰor.dmaŋs}}\end{déclaration}\end{définition}\end{entrée}

\begin{entrée}
\vedette{\hypertarget{Ⓔmkhɤrŋa}{\papi{ mkhɤrŋa}}}\markboth{mkhɤrŋa}{}\classe{n}
\begin{définition}\fra gong\end{définition}
\begin{définition}\cmn 锣
\begin{déclaration} \étymologie{\papi{ⁿkʰar.rŋa}}\end{déclaration}\end{définition}
\end{entrée}

\begin{entrée}
\vedette{\hypertarget{Ⓔmkhɤscoʁ}{\papi{ mkhɤscoʁ}}}\markboth{mkhɤscoʁ}{}\classe{n}
\begin{définition}\fra masque\end{définition}
\begin{définition}\cmn 口罩\end{définition}\end{entrée}

\begin{entrée}
\vedette{\hypertarget{Ⓔmkhɤz}{\papi{ mkhɤz}}}\markboth{mkhɤz}{}
\classe{vs}
\paradigme{\textit{dir :} \jya tɤ-}
\paradigme{\textit{dir :} \jya thɯ-}
\begin{définition}\fra être bon à\end{définition}
\begin{définition}\cmn 擅长
\begin{déclaration} \étymologie{\papi{mkʰas}}\end{déclaration}\end{définition}
\begin{exemple}\jya ɕoŋβzu ɲɯ-mkhɤz\cmn 木匠很厉害\end{exemple}
\begin{exemple}\jya kɤ-rɤrɤt ɲɯ-mkhɤz\cmn 他很擅长写字\end{exemple}\end{entrée}

\begin{entrée}
\vedette{\hypertarget{Ⓔmkhoŋ}{\papi{ mkhoŋ}}}\markboth{mkhoŋ}{}
\classe{n}
\begin{définition}\fra abri à bestiaux (dans les pâturages)\end{définition}
\begin{définition}\cmn (牧场上的)牛棚
\end{définition}\end{entrée}

\begin{entrée}
\vedette{\hypertarget{Ⓔmkhroŋ}{\papi{ mkhroŋ}}}\markboth{mkhroŋ}{}\classe{vi}
\paradigme{\textit{dir :} \jya tɤ-}
\begin{définition}\fra se réincarner\end{définition}
\begin{définition}\cmn 投胎;转世
\begin{déclaration} \étymologie{\papi{ⁿkʰruŋ}}\end{déclaration}\end{définition}
\begin{exemple}\jya sprɯskɯ to-mkhroŋ\cmn 活佛转世了\end{exemple}\end{entrée}

\begin{entrée}
\vedette{\hypertarget{Ⓔmkhrɯmkhaŋ}{\papi{ mkhrɯmkhaŋ}}}\markboth{mkhrɯmkhaŋ}{}\classe{n}
\begin{définition}\fra prison\end{définition}
\begin{définition}\cmn 监狱
\begin{déclaration} \étymologie{\papi{kʰrims.kʰaŋ}}\end{déclaration}\end{définition}
\end{entrée}

\begin{entrée}
\vedette{\hypertarget{Ⓔmkhrɯn}{\papi{ mkhrɯn}}}\markboth{mkhrɯn}{}
\classe{vs}
\paradigme{\textit{dir :} \jya tɤ-}
\begin{définition}\fra avare\end{définition}
\begin{définition}\cmn 吝啬\end{définition}
\begin{exemple}\jya ɲɯ-znɤje tɕe ɲɯ-mkhrɯn\cmn 他很不舍得,很吝啬\end{exemple}\end{entrée}

\begin{entrée}
\vedette{\hypertarget{Ⓔmkhɯrlu}{\papi{ mkhɯrlu}}}\markboth{mkhɯrlu}{}
\classe{n}
\begin{définition}\fra roue\end{définition}
\begin{définition}\cmn 轮子
\begin{déclaration} \étymologie{\papi{ⁿkʰor.lo}}\end{déclaration}\end{définition}\end{entrée}

\begin{entrée}
\vedette{\hypertarget{Ⓔmkhɯrlɤmnɯ}{\papi{ mkhɯrlɤmnɯ}}}\markboth{mkhɯrlɤmnɯ}{}\classe{n}
\begin{définition}\fra perceuse\end{définition}
\begin{définition}\cmn 钻子
\begin{déclaration} \étymologie{\papi{ⁿkʰor.lo}}\end{déclaration}\end{définition}
\end{entrée}

\begin{entrée}
\vedette{\hypertarget{Ⓔmkɯm}{\papi{ mkɯm}}}\markboth{mkɯm}{}\classe{vi}
\paradigme{\textit{dir :} \jya thɯ-}
\paradigme{\textit{dir :} \jya lɤ-}
\begin{définition}\fra avoir la tête tournée dans une certaine direction dans son lit\end{définition}
\begin{définition}\cmn 头朝着(哪个方向)睡
\begin{déclaration}\use{这个动词干木鸟话很少用,主要是大藏话,干木鸟话一般说\stylefv{alo lɤ-ru-a}、\stylefv{athi thɯ-ru-a}表达这种意思}\end{déclaration}\end{définition}
\begin{exemple}\jya soz tɤ-tɕɯ nɯ rɤru tɤkha tɕe, tɕheme nɯ ɣɯ ɯ-jme pɕoʁ nɯ tɕu ntsɯ chɯ-mkɯm pjɤ-ŋu\cmn 早上,那个男孩子起来的时候,发现自己的头总是朝着他妻子的脚(独角鬼13)\end{exemple}
\begin{relation-sémantique}\confer{
\hyperlink{Ⓔtɤ-mkɯm}{\textit{ \papi{tɤ-mkɯm}}}
}\end{relation-sémantique}
\begin{relation-sémantique}\confer{
\hyperlink{Ⓔnɤmkɯm}{\textit{ \papi{nɤmkɯm}}}
}\end{relation-sémantique}\end{entrée}

\begin{entrée}
\vedette{\hypertarget{Ⓔmkɯt}{\papi{ mkɯt}}}\markboth{mkɯt}{}\classe{vi}
\paradigme{\textit{dir :} \jya tɤ-}
\begin{définition}\fra avaler de travers (avec du xanthoxyle)\end{définition}
\begin{définition}\cmn 呛到(吃花椒的时候)\end{définition}
\begin{exemple}\jya to-mkɯt-a\cmn 我呛到了\end{exemple}
\begin{sous-entrée}
\vedette{\hypertarget{}{\papi{ sɯmkɯt}}}\markboth{sɯmkɯt}{}\classe{vt}
\paradigme{\textit{dir :} \jya pɯ-}
\begin{définition}\ 
\begin{déclaration}\grammar{caus}\end{déclaration}\end{définition}\acception{1}
\begin{définition}\fra faire que quelqu'un s'étrangle, avale de travers\end{définition}
\begin{définition}\cmn 呛到\end{définition}
\begin{exemple}\jya @cai ɯ-ŋgɯ tɕɣom ɣɤʑu tɕe pɯ́-wɣ-sɯmkɯt-a\cmn 菜里面有花椒,把我呛到了\end{exemple}\acception{2}
\begin{définition}\fra faire en sorte que quelqu'un n'arrive pas à répondre\end{définition}
\begin{définition}\cmn 使人答不上来(开玩笑的时候)\end{définition}
\begin{exemple}\jya jiɕqha nɯ kɯ khɤβdɤr ɲɯ-ɤsɯ-βzu tɕe, ɯ-sci tɤ-βzu-t-a (=ɯ-ntsi tɤ-βzu-t-a) tɕe, pɯ-sɯmkɯ-t-a\cmn 这个人在讲笑话,我也说了一句,让他答不上来了\end{exemple}
\end{sous-entrée}\end{entrée}

\begin{entrée}
\vedette{\hypertarget{Ⓔmnu}{\papi{ mnu}}}\markboth{mnu}{}
\classe{vs}
\paradigme{\textit{dir :} \jya nɯ-}\acception{1}
\begin{définition}\fra lisse\end{définition}
\begin{définition}\cmn 光滑\end{définition}\acception{2}
\begin{définition}\fra doux\end{définition}
\begin{définition}\cmn 柔软\end{définition}
\begin{exemple}\jya tɯ-ŋga ɲɯ-mnu\cmn 衣服很柔软\end{exemple}
\begin{exemple}\jya cha ɲɯ-mnu\cmn 酒的浓度低\end{exemple}
\begin{relation-sémantique}\confer{
\hyperlink{Ⓔmnumne}{\textit{ \papi{mnumne}}}
}\end{relation-sémantique}
\begin{relation-sémantique}\confer{
\hyperlink{Ⓔmnule}{\textit{ \papi{mnule}}}
}\end{relation-sémantique}
\begin{relation-sémantique}\antonyme{
\hyperlink{Ⓔrʁom}{\textit{ \papi{rʁom}}}
}\end{relation-sémantique}
\begin{relation-sémantique}\antonyme{
\hyperlink{Ⓔtsaβ}{\textit{ \papi{tsaβ}}}
}\end{relation-sémantique}\end{entrée}

\begin{entrée}
\vedette{\hypertarget{Ⓔmna}{\papi{ mna}}}\markboth{mna}{}
\classe{vi}
\paradigme{\textit{dir :} \jya tɤ-}\acception{1}
\begin{définition}\fra meilleur\end{définition}
\begin{définition}\cmn 优秀\end{définition}\acception{2}
\begin{définition}\fra guérir\end{définition}
\begin{définition}\cmn 痊愈;康复\end{définition}
\begin{exemple}\jya a-ɕqhe to-mna\cmn 我的咳嗽好了\end{exemple}
\begin{exemple}\jya nɤ-ɕqhe ɯ-ɲɯ́-mna?\cmn 你的咳嗽好了没有?\end{exemple}
\begin{exemple}\jya smɤn tu-ndze-a tɕe, a-ɕqhe a-tɤ-mna\cmn 我吃药,希望我的咳嗽会好\end{exemple}
\begin{exemple}\jya (smɤnba kɯ tɤ́-wɣ-nɯsman-a) mna ɕi mɤ-mna mɤxsi\cmn 医生给我治了病,不知道有没有效\end{exemple}
\begin{exemple}\jya a-tɤ-mna tsa tɕe, tɕe nɯɣi\cmn (他母亲)好一些,他就会回来\end{exemple}
\begin{relation-sémantique}\confer{
\hyperlink{Ⓔɣɤmna}{\textit{ \papi{ɣɤmna}}}
}\end{relation-sémantique}
\begin{relation-sémantique}\antonyme{
\hyperlink{Ⓔʑɤn}{\textit{ \papi{ʑɤn}}}
}\end{relation-sémantique}\end{entrée}

\begin{entrée}
\vedette{\hypertarget{Ⓔmnɤm}{\papi{ mnɤm}}}\markboth{mnɤm}{}
\classe{vi}
\paradigme{\textit{dir :} \jya nɯ-}
\begin{définition}\fra avoir une odeur\end{définition}
\begin{définition}\cmn 发出气味\end{définition}
\begin{exemple}\jya cha ɯ-di ɲɯ-mnɤm\cmn 酒有味道\end{exemple}
\begin{exemple}\jya ɯ-dɯχɯn ɲɯ-mnɤm\cmn 香味很浓\end{exemple}
\begin{exemple}\jya nɤki kɤ-ndza nɯ ɲɯ-mɲɤt tɕe ɯ-di ɲɯ-mnɤm\cmn 那个食物坏了,有臭味了\end{exemple}
\begin{relation-sémantique}\confer{
\hyperlink{Ⓔnɤmnɤm}{\textit{ \papi{nɤmnɤm}}}
}\end{relation-sémantique}
\begin{relation-sémantique}\confer{
\hyperlink{Ⓔɕɯmnɤm}{\textit{ \papi{ɕɯmnɤm}}}
}\end{relation-sémantique}\end{entrée}

\begin{entrée}
\vedette{\hypertarget{Ⓔmnɤt}{\papi{ mnɤt}}}\markboth{mnɤt}{}
\classe{vt}
\paradigme{\textit{dir :} \jya \_}
\begin{définition}\fra refaire\end{définition}
\begin{définition}\cmn 重新做
\begin{déclaration}\use{\stylefv{mnɤt}没有固有的趋向前缀,它的趋向前缀和前面的辅助动词一致。例如,\stylefv{rɤrɤt}“写”的固有趋向前缀是\stylefv{pɯ-},所以当\stylefv{mnɤt}和\stylefv{rɤrɤt}连用时(表示“重新写”)),\stylefv{mnɤt}必须前加\stylefv{pɯ-}这个前缀}\end{déclaration}\end{définition}
\begin{exemple}\jya kɤ-nɤma tú-wɣ-sɤpe tɕe kɤ-mnɤt mɤra\cmn 如果工作做好了的话,不需要重新做\end{exemple}
\begin{exemple}\jya kɤ-nɤma tɤ-mnɤt ma mɯ́jpe\cmn 你把工作重新做一遍,因为没有做好\end{exemple}
\begin{exemple}\jya kɤ-rɤrɤt pɯ-mnɤt\cmn 你重新写一遍\end{exemple}
\begin{exemple}\jya kɤ-ti tɤ-mnɤt\cmn 你重新讲一遍\end{exemple}
\begin{exemple}\jya kɤ-ɕe ci jɤ-mnɤt ɲɯ-ntshi\cmn 需要重新去一趟\end{exemple}
\begin{exemple}\jya kɤ-nɯrɤɣo ci thɯ-mnɤt ɲɯ-ntshi\cmn 要重新唱一遍\end{exemple}\end{entrée}

\begin{entrée}
\vedette{\hypertarget{Ⓔmnule}{\papi{ mnule}}}\markboth{mnule}{}\classe{vs}
\begin{définition}\fra doux et chaux\end{définition}
\begin{définition}\cmn 又柔软又光滑又暖和\end{définition}
\begin{relation-sémantique}\confer{
\hyperlink{Ⓔmnumne}{\textit{ \papi{mnumne}}}
}\end{relation-sémantique}
\begin{relation-sémantique}\confer{
\hyperlink{Ⓔmnu}{\textit{ \papi{mnu}}}
}\end{relation-sémantique}\end{entrée}

\begin{entrée}
\vedette{\hypertarget{Ⓔmnumne}{\papi{ mnumne}}}\markboth{mnumne}{}
\classe{vs}
\begin{définition}\ 
\begin{déclaration}\grammar{rdpl}\end{déclaration}\end{définition}
\begin{définition}\fra doux et chaux\end{définition}
\begin{définition}\cmn 又柔软又光滑又暖和\end{définition}
\begin{exemple}\jya kɯ-mnumne ci ɲɯ-ŋu\cmn 它又柔软又光滑\end{exemple}
\begin{relation-sémantique}\confer{
\hyperlink{Ⓔmnu}{\textit{ \papi{mnu}}}
}\end{relation-sémantique}
\begin{relation-sémantique}\confer{
\hyperlink{Ⓔmnule}{\textit{ \papi{mnule}}}
}\end{relation-sémantique}\end{entrée}

\begin{entrée}
\vedette{\hypertarget{Ⓔmɲaqrɯ}{\papi{ mɲaqrɯ}}}\markboth{mɲaqrɯ}{}\classe{n}
\begin{définition}\fra regard\end{définition}
\begin{définition}\cmn 瞪眼\end{définition}
\begin{exemple}\jya ɯʑo kɯ a-ɕki mɲaqrɯ ci to-lɤt\cmn 他瞪了我一眼\end{exemple}
\begin{relation-sémantique}\confer{
\hyperlink{Ⓔnɯmɲaqrɯ}{\textit{ \papi{nɯmɲaqrɯ}}}
}\end{relation-sémantique}\end{entrée}

\begin{entrée}
\vedette{\hypertarget{Ⓔmɲaʁlaχtɕhɯ}{\papi{ mɲaʁlaχtɕhɯ}}}\markboth{mɲaʁlaχtɕhɯ}{}
\classe{n}
\begin{définition}\fra gaspillage\end{définition}
\begin{définition}\cmn 浪费\end{définition}
\begin{exemple}\jya mɲaʁlaχtɕhɯ ma-nɯ-tɯ-sɯxɕe\cmn 你不要浪费\end{exemple}
\begin{exemple}\jya kɯki tɯ-mgo tɤ-βzu-t-a ri, koŋla mɯ-ta-ndza-nɯ tɕe mɲaʁlaχtɕhɯ nɯ-ari\cmn 我做的饭他们没有吃完,被浪费了\end{exemple}\end{entrée}

\begin{entrée}
\vedette{\hypertarget{Ⓔmɲaʁmtsaʁ}{\papi{ mɲaʁmtsaʁ}}}\markboth{mɲaʁmtsaʁ}{}\classe{n}
\begin{définition}\fra criquet\end{définition}
\begin{définition}\cmn 蝗虫\end{définition}
\end{entrée}

\begin{entrée}
\vedette{\hypertarget{Ⓔmɲaʁtɕhɯβ}{\papi{ mɲaʁtɕhɯβ}}}\markboth{mɲaʁtɕhɯβ}{}\classe{n}
\begin{définition}\fra clin d'œil\end{définition}
\begin{définition}\cmn 眨眼\end{définition}
\begin{exemple}\jya mɲaʁtɕhɯβ ci pɯ-lat-a\cmn 我眨了眼\end{exemple}\end{entrée}

\begin{entrée}
\vedette{\hypertarget{Ⓔmɲɤm}{\papi{ mɲɤm}}}\markboth{mɲɤm}{}\classe{n}
\begin{définition}\fra espèce d'arbre\end{définition}
\begin{définition}\cmn 【野白杨】\end{définition}
\begin{exemple}\jya mɲɤm nɯ sɤtɕha kɯ-mbɤr tsa zgoku tu-ɬoʁ ŋu, ɯ-jwaʁ nɯ mi ɣɯ cho naχtɕɯɣ ri xtɕi cho mba, ɯ-ru kɯ-pɣi ŋu, wuma ʑo ɣɤ-wxti, ɯ-si nɯ mɯ-tɤ-nɤkhɯ mɤɕtʂa mɤ-ngɯt\cmn 野白杨生长在半山上,叶子和白杨一样,但比较小和薄,树干是灰色的,长得很快,木质要经过烟熏才结实。\end{exemple}
\end{entrée}

\begin{entrée}
\vedette{\hypertarget{Ⓔmɲɤt}{\papi{ mɲɤt}}}\markboth{mɲɤt}{}\classe{vs}
\paradigme{\textit{dir :} \jya nɯ-}
\begin{définition}\fra se détériorer (nourriture), faible, maigre (personne)\end{définition}
\begin{définition}\cmn 食物变味;人虚弱、瘦\end{définition}
\begin{exemple}\jya tɤ-mthɯm ɲɤ-mɲɤt\cmn 肉变味了\end{exemple}
\begin{exemple}\jya nɤʑo ɲɤ-tɯ-mɲɤt\cmn 你瘦了\end{exemple}\end{entrée}

\begin{entrée}
\vedette{\hypertarget{Ⓔmɲi}{\papi{ mɲi}}}\markboth{mɲi}{}\classe{vs}
\begin{définition}\fra peu\end{définition}
\begin{définition}\cmn 少\end{définition}
\begin{sous-entrée}
\vedette{\hypertarget{}{\papi{ ɣɤmɲi}}}\markboth{ɣɤmɲi}{}\classe{vt}
\paradigme{\textit{dir :} \jya pɯ-}
\begin{définition}\fra donner une part trop petite\end{définition}
\begin{définition}\cmn 分得少\end{définition}
\begin{exemple}\jya tɯkro pa-βzu tɕe, tsuku ɣɯ pa-ɣɤdɤn, tsuku ɣɯ pa-ɣɤmɲi\cmn 他分东西的时候,有些人分得多,有些人分得少\end{exemple}
\begin{relation-sémantique}\synonyme{
\hyperlink{Ⓔrkɯn}{\textit{ \papi{rkɯn}}}
}\end{relation-sémantique}
\end{sous-entrée}\begin{sous-entrée}
\vedette{\hypertarget{}{\papi{ sɤdɤmɲi}}}\markboth{sɤdɤmɲi}{}
\paradigme{\textit{dir :} \jya pɯ-}
\begin{définition}\fra partager de façon injuste\end{définition}
\begin{définition}\cmn 分得不均匀\end{définition}
\begin{exemple}\jya kɯ-rɤkro tɤ-kɯ-ŋu tɕe kɤ-sɤdɤmɲi mɤ-pe\cmn 分东西的时候,不要分得不均匀\end{exemple}\classe{vt}
\end{sous-entrée}\end{entrée}

\begin{entrée}
\vedette{\hypertarget{ⒺmɲoⒽ1}{\papi{ mɲo}}}\markboth{mɲo}{}\homonyme{1}
\classe{vt}
\paradigme{\textit{dir :} \jya tɤ-}
\begin{définition}\fra préparer\end{définition}
\begin{définition}\cmn 准备\end{définition}
\begin{exemple}\jya laχtɕha tɤ-mɲɤm\cmn 你准备东西吧\end{exemple}
\begin{exemple}\jya ɕoŋtɕa tɤ-mɲɤm\cmn 你准备木料吧\end{exemple}
\begin{exemple}\jya kɤ-ndza tɤ-mɲɤm\cmn 你准备食物吧\end{exemple}
\begin{exemple}\jya nɤ-ŋga tɤ-mɲo-t-a\cmn 我给你准备了衣服\end{exemple}
\begin{exemple}\jya nɤ-tʂha tɤ-mɲo-t-a\cmn 我给你准备早饭\end{exemple}
\begin{exemple}\jya nɤ-kɤ-nɤma tɤ-mɲo-t-a\cmn 我准备了你的工作\end{exemple}
\begin{relation-sémantique}\confer{
\hyperlink{Ⓔsɯmɲo}{\textit{ \papi{sɯmɲo}}}
}\end{relation-sémantique}
\begin{relation-sémantique}\synonyme{
\hyperlink{Ⓔnɯftɕaka}{\textit{ \papi{nɯftɕaka}}}
}\end{relation-sémantique}\begin{sous-entrée}
\vedette{\hypertarget{}{\papi{ rɤmɲo}}}\markboth{rɤmɲo}{}\classe{vi}
\paradigme{\textit{dir :} \jya tɤ-}
\begin{définition}\ 
\begin{déclaration}\grammar{apass}\end{déclaration}\end{définition}
\begin{définition}\fra préparer\end{définition}
\begin{définition}\cmn 准备\end{définition}
\end{sous-entrée}\begin{sous-entrée}
\vedette{\hypertarget{}{\papi{ ɯ-mɲoz}}}\markboth{ɯ-mɲoz}{}
\begin{définition}\ 
\begin{déclaration}\use{不规则的不定式}\end{déclaration}\end{définition}
\begin{exemple}\jya pjɯ-ɲɟo ɕɯŋgɯ tɕe ɯ-mɲoz tú-wɣ-βzu ra\cmn 在损失发生之前我做好准备了\end{exemple}
\begin{exemple}\jya mɤ-mbɯt ɯ-jtsi, mɤ-ɲɟo ɯ-mɲoz\cmn 倒塌之前顶柱子,损失到来以前要预防\end{exemple}
\end{sous-entrée}\begin{sous-entrée}
\vedette{\hypertarget{}{\papi{ ʑɣɤmɲo}}}\markboth{ʑɣɤmɲo}{}\classe{vi}
\paradigme{\textit{dir :} \jya tɤ-}
\begin{définition}\ 
\begin{déclaration}\grammar{refl}\end{déclaration}\end{définition}
\begin{définition}\fra se préparer\end{définition}
\begin{définition}\cmn 准备\end{définition}
\begin{exemple}\jya kɯ-ɕe tɤ-ʑɣɤmɲo-a\cmn 我准备去\end{exemple}
\begin{relation-sémantique}\synonyme{
\hyperlink{Ⓔrɤŋgat}{\textit{ \papi{rɤŋgat}}}
}\end{relation-sémantique}
\end{sous-entrée}\end{entrée}

\begin{entrée}
\vedette{\hypertarget{ⒺmɲoⒽ2}{\papi{ mɲo}}}\markboth{mɲo}{}\homonyme{2}\classe{vs}
\begin{définition}\fra être prévisible\end{définition}
\begin{définition}\cmn 在预料之中\end{définition}
\begin{exemple}\jya nɤ-laχtɕha tɯ-nɯβde ɲɯ-mɲo ɕti ma aʁɤndɯndɤt ʑo ɕɯ-tɯ-khɤt ɲɯ-ra\cmn 你总是到处走,肯定会丢东西\end{exemple}
\begin{exemple}\jya ɯʑo ndʐaβ ɲɯ-mɲo ɕti ma tɤ-ŋke tɕe tʂu pjɯ-ru ɲɯ-maʁ\cmn 他走路的时候不看路,肯定会摔跤\end{exemple}\begin{sous-entrée}
\vedette{\hypertarget{}{\papi{ amɲɯmɲo}}}\markboth{amɲɯmɲo}{}\classe{vs}
\begin{définition}\fra comme prévu\end{définition}
\begin{définition}\cmn 预料之中\end{définition}
\begin{exemple}\jya ɯʑo kɯ jɯfɕɯr "ɣi-a" ɲɯ-ti tɕe, jɯxɕo tɕe amɲɯmɲo ʑo jo-ɣi\cmn 他昨天说要来,今天早上果然就来了\end{exemple}
\begin{relation-sémantique}\confer{
\hyperlink{ⒺmɲoⒽ1}{\textit{ \papi{mɲo1}}}
}\end{relation-sémantique}
\end{sous-entrée}\end{entrée}

\begin{entrée}
\vedette{\hypertarget{Ⓔmɲoχpɣa}{\papi{ mɲoχpɣa}}}\markboth{mɲoχpɣa}{}\classe{n}
\begin{définition}\fra décorations (sur les pains à la vapeur)\end{définition}
\begin{définition}\cmn 包子上的花纹\end{définition}\end{entrée}

\begin{entrée}
\vedette{\hypertarget{Ⓔmɲɯka}{\papi{ mɲɯka}}}\markboth{mɲɯka}{}\classe{n}
\begin{définition}\fra humiliation\end{définition}
\begin{définition}\cmn 耻辱\end{définition}
\begin{exemple}\jya ɯ-mɲɯka pjɤ-ɬoʁ\cmn 他受耻辱了\end{exemple}
\begin{relation-sémantique}\confer{
\hyperlink{Ⓔnɯmɲɯka}{\textit{ \papi{nɯmɲɯka}}}
}\end{relation-sémantique}\end{entrée}

\begin{entrée}
\vedette{\hypertarget{Ⓔmɲɯmaŋ}{\papi{ mɲɯmaŋ}}}\markboth{mɲɯmaŋ}{}\classe{n}
\begin{définition}\fra tout le monde\end{définition}
\begin{définition}\cmn 群众
\begin{déclaration} \étymologie{\papi{mi.dmaŋs}}\end{déclaration}\end{définition}\end{entrée}

\begin{entrée}
\vedette{\hypertarget{Ⓔmɲɯrgɤt}{\papi{ mɲɯrgɤt}}}\markboth{mɲɯrgɤt}{}
\classe{n}
\begin{définition}\fra yéti\end{définition}
\begin{définition}\cmn 野人
\begin{déclaration} \étymologie{\papi{mi.rgod}}\end{déclaration}\end{définition}\end{entrée}

\begin{entrée}
\vedette{\hypertarget{Ⓔmɲɯrɟɤβ}{\papi{ mɲɯrɟɤβ}}}\markboth{mɲɯrɟɤβ}{}\classe{n}
\begin{définition}\fra laic\end{définition}
\begin{définition}\cmn 凡人,非宗教徒\end{définition}\end{entrée}

\begin{entrée}
\vedette{\hypertarget{Ⓔmɲɯrɟit}{\papi{ mɲɯrɟit}}}\markboth{mɲɯrɟit}{}\classe{n}
\begin{définition}\fra progéniture\end{définition}
\begin{définition}\cmn 孩子
\begin{déclaration} \étymologie{\papi{mi}}\end{déclaration}\end{définition}
\end{entrée}

\begin{entrée}
\vedette{\hypertarget{Ⓔmɲɯrɯri}{\papi{ mɲɯrɯri}}}\markboth{mɲɯrɯri}{}\classe{n}
\begin{définition}\fra chaque personne\end{définition}
\begin{définition}\cmn 每个人
\begin{déclaration} \étymologie{\papi{mi.re.re}}\end{déclaration}\end{définition}
\begin{exemple}\jya mɲɯrɯri nɯ kɯ nɯ-kɤ-sɯso nɯ to-nɯ-ti-nɯ\cmn 每个人讲了自己的想法\end{exemple}
\begin{exemple}\jya mɲɯrɯri nɯ ɯ-tɯ-rju ɯ-tshɯɣa mɯ́j-naχtɕɯɣ\cmn 每个人讲话的方式都不一样\end{exemple}\end{entrée}

\begin{entrée}
\vedette{\hypertarget{Ⓔmɲɯʁʑi}{\papi{ mɲɯʁʑi}}}\markboth{mɲɯʁʑi}{}
\classe{n}
\begin{définition}\fra humeur\end{définition}
\begin{définition}\cmn 脾气,态度
\begin{déclaration} \étymologie{\papi{mi.gʑi}}\end{déclaration}\end{définition}
\begin{exemple}\jya ɯ-mɲɯʁʑi βdi\cmn 他脾气好\end{exemple}
\begin{relation-sémantique}\confer{
\hyperlink{Ⓔnɯmɲɯʁʑi}{\textit{ \papi{nɯmɲɯʁʑi}}}
}\end{relation-sémantique}\end{entrée}

\begin{entrée}
\vedette{\hypertarget{Ⓔmɲɯtɕhɤz}{\papi{ mɲɯtɕhɤz}}}\markboth{mɲɯtɕhɤz}{}\classe{n}
\begin{définition}\fra humeur\end{définition}
\begin{définition}\cmn 脾气\end{définition}
\begin{exemple}\jya ɯ-mɲɯtɕhɤz tu\cmn 他有他的脾气\end{exemple}
\begin{exemple}\jya nɤ-mɲɯtɕhɤz ɲɯ-βdi (ɲɯ-tɯ-nɯmɲɯtɕhɤz)\cmn 你脾气很好\end{exemple}
\begin{relation-sémantique}\confer{
\hyperlink{Ⓔmɲɯʁʑi}{\textit{ \papi{mɲɯʁʑi}}}
}\end{relation-sémantique}\end{entrée}

\begin{entrée}
\vedette{\hypertarget{Ⓔmŋaʁnɤmŋaʁ}{\papi{ mŋaʁnɤmŋaʁ}}}\markboth{mŋaʁnɤmŋaʁ}{}\classe{idph.3}
\begin{définition}\fra ouvrant la bouche avec difficulté, à l'article de la mort\end{définition}
\begin{définition}\cmn 形容嘴巴一开一张,有气无力的样子\end{définition}
\begin{exemple}\jya ɯ-kɯ-mŋɤm ɯ-tɯ-thɯ kɯ ɯ-ɣmɤr mŋaʁnɤmŋaʁ tu-ste ma mɯ́j-cha\cmn 他病得很严重,嘴巴只能慢慢地一开一张(没有力气把话说出来)\end{exemple}\end{entrée}

\begin{entrée}
\vedette{\hypertarget{Ⓔmŋɤm}{\papi{ mŋɤm}}}\markboth{mŋɤm}{}
\classe{vs}
\paradigme{\textit{dir :} \jya tɤ-}
\begin{définition}\fra avoir mal\end{définition}
\begin{définition}\cmn 痛\end{définition}
\begin{exemple}\jya a-xtu ɲɯ-mŋɤm\cmn 我肚子很痛\end{exemple}
\begin{exemple}\jya ndʑi-kɯ-mŋɤm ɯβrɤ-pɯ-tu?\cmn 你们俩没有生病吧?\end{exemple}\begin{sous-entrée}
\vedette{\hypertarget{}{\papi{ nɤmŋɤm}}}\markboth{nɤmŋɤm}{}\classe{vt}
\begin{définition}\ 
\begin{déclaration}\grammar{trop}\end{déclaration}\end{définition}
\begin{définition}\fra avoir mal à\end{définition}
\begin{définition}\cmn 感觉到痛\end{définition}
\begin{exemple}\jya ɯ-xtu ɲɯ-nɤmŋɤm\cmn 他感到肚子痛\end{exemple}
\begin{relation-sémantique}\confer{
\hyperlink{Ⓔɕɯmŋɤm}{\textit{ \papi{ɕɯmŋɤm}}}
}\end{relation-sémantique}
\end{sous-entrée}\end{entrée}

\begin{entrée}
\vedette{\hypertarget{Ⓔmŋɤrmŋɤr}{\papi{ mŋɤrmŋɤr}}}\markboth{mŋɤrmŋɤr}{}
\classe{idph.2}
\begin{définition}\fra regardant en allongeant le cou\end{définition}
\begin{définition}\cmn 形容伸着脖子到处张望的样子\end{définition}
\begin{exemple}\jya khɤxtu ri ɲɯ-ndzur tɕe mŋɤrmŋɤr ʑo pjɯ-nɤrɯra ɲɯ-ŋu\cmn 他在房背上伸着脖子到处张望\end{exemple}\end{entrée}

\begin{entrée}
\vedette{\hypertarget{Ⓔmŋulɤn}{\papi{ mŋulɤn}}}\markboth{mŋulɤn}{}
\classe{n}
\begin{définition}\fra semelle\end{définition}
\begin{définition}\cmn 鞋底\end{définition}\end{entrée}

\begin{entrée}
\vedette{\hypertarget{Ⓔmŋurɯm}{\papi{ mŋurɯm}}}\markboth{mŋurɯm}{}\classe{n}
\begin{définition}\fra ouverture qui peut se refermer en tirant sur un fil (pantalon, sac)\end{définition}
\begin{définition}\cmn (可以收拢的)口(口袋、裤子)\end{définition}
\begin{exemple}\jya lʁɤtɕɯ mŋurɯm kɤ-βzu-t-a\cmn 我把口袋的口收拢了。\end{exemple}
\end{entrée}

\begin{entrée}
\vedette{\hypertarget{Ⓔmŋɯn}{\papi{ mŋɯn}}}\markboth{mŋɯn}{}\classe{vs}
\paradigme{\textit{dir :} \jya pɯ-}\acception{1}
\begin{définition}\fra content (d'avoir obtenu quelque chose)\end{définition}
\begin{définition}\cmn (得到了某个东西)很满意\end{définition}
\begin{exemple}\jya laχtɕha nɯ-kɯ-mbi-a nɯ pɯ-mŋɯn\cmn 你给了我的东西,我很满意(我很需要这个东西)\end{exemple}
\begin{exemple}\jya kɯ-mŋɯn maŋe\cmn 不需要(这个东西)\end{exemple}
\begin{exemple}\jya wo ki pɯ-mŋɯn rcanɯ wuma pɯ-pe\cmn 很需要(这个东西),很好\end{exemple}
\begin{exemple}\jya nɯ́-wɣ-mbi-a tɕe pɯ-mŋɯn\cmn 他给了我,我很满意\end{exemple}
\begin{exemple}\jya kɯ-mŋɯn ʑo ta-ndza\cmn 他很痛快地吃了\end{exemple}\acception{2}
\begin{définition}\fra rassuré\end{définition}
\begin{définition}\cmn 放心;心里踏实\end{définition}
\begin{exemple}\jya kɤ-nɯna mɯ́j-mŋɯn\cmn (工作没有做好就)无法安心地休息\end{exemple}
\begin{relation-sémantique}\synonyme{
 \papi{tʂɯn}
}\end{relation-sémantique}
\begin{relation-sémantique}\confer{
\hyperlink{Ⓔnɤmŋɯn}{\textit{ \papi{nɤmŋɯn}}}
}\end{relation-sémantique}\end{entrée}

\begin{entrée}
\vedette{\hypertarget{Ⓔmochi}{\papi{ mochi}}}\markboth{mochi}{}\classe{n}
\begin{définition}\fra chien\end{définition}
\begin{définition}\cmn 狗
\begin{déclaration} \étymologie{\papi{kʰʲi}}\end{déclaration}\end{définition}
\end{entrée}

\begin{entrée}
\vedette{\hypertarget{Ⓔmoɣɤz}{\papi{ moɣɤz}}}\markboth{moɣɤz}{}
\classe{n}
\begin{définition}\fra habit en laine féminin\end{définition}
\begin{définition}\cmn 女子的毛织品衣服\end{définition}\end{entrée}

\begin{entrée}
\vedette{\hypertarget{Ⓔmoli}{\papi{ moli}}}\markboth{moli}{}\classe{n}
\begin{définition}\fra chatte\end{définition}
\begin{définition}\cmn 母猫\end{définition}
\end{entrée}

\begin{entrée}
\vedette{\hypertarget{Ⓔmɢom}{\papi{ mɢom}}}\markboth{mɢom}{}\classe{vt}
\paradigme{\textit{dir :} \jya tɤ-}
\begin{définition}\fra se mordre les lèvres (de rage)\end{définition}
\begin{définition}\cmn 咬牙切齿(用上唇咬着下唇的表情)\end{définition}
\begin{exemple}\jya to-sɤmbrɯ tɕe ɯ-mtɕhi to-mɢom\cmn 他生气了就做出咬牙切齿的表情\end{exemple}
\begin{exemple}\jya nɤ-mtɕhi ma-tɤ-tɯ-mɢom\cmn 你不要咬牙切齿\end{exemple}
\begin{relation-sémantique}\confer{
\hyperlink{Ⓔtamɢom}{\textit{ \papi{tamɢom}}}
}\end{relation-sémantique}\end{entrée}

\begin{entrée}
\vedette{\hypertarget{Ⓔmoʁ}{\papi{ moʁ}}}\markboth{moʁ}{}\classe{vt}
\paradigme{\textit{dir :} \jya tɤ-}
\begin{définition}\fra manger de la tsampa\end{définition}
\begin{définition}\cmn 吃粉状的食物;吃干糌粑\end{définition}
\begin{exemple}\jya tɯ-ɣndʑɤr to-moʁ\cmn 他吃了干糌粑\end{exemple}
\begin{exemple}\jya aʑo tɯ-ɣndʑɤr tɤ-moʁ-a\cmn 我吃了干糌粑\end{exemple}\begin{sous-entrée}
\vedette{\hypertarget{}{\papi{ sɯɣmoʁ}}}\markboth{sɯɣmoʁ}{} (\variante{sɯmoʁ}) \classe{vt}
\paradigme{\textit{dir :} \jya tɤ-}
\begin{définition}\fra donner de la tsampa à manger\end{définition}
\begin{définition}\cmn 给……吃干糌粑\end{définition}
\begin{exemple}\jya nɤ-ɣndʑɤr ci tu-kɯ-sɯɣmoʁ-a\cmn 给我吃一点糌粑\end{exemple}
\end{sous-entrée}\end{entrée}

\begin{entrée}
\vedette{\hypertarget{Ⓔmoʁmoʁ}{\papi{ moʁmoʁ}}}\markboth{moʁmoʁ}{}
\classe{idph.2}
\begin{définition}\fra très fine (poudre)\end{définition}
\begin{définition}\cmn 形容粉末极细,或者土壤软而看起来肥沃的样子\end{définition}
\begin{exemple}\jya sɯβɣi ɲɯ-ndɯβ moʁmoʁ ʑo\cmn 木屑是细的\end{exemple}
\begin{exemple}\jya βɣa kɯ tɯ-ɣndʑɤr kɯ-ndɯβ ʑo moʁmoʁ chɤ-tɕɤt (chɤ-ɣɤndɯβ ʑo moʁmoʁ)\cmn 水磨把糌粑磨得很细\end{exemple}\end{entrée}

\begin{entrée}
\vedette{\hypertarget{Ⓔmpɤβmpɤβ}{\papi{ mpɤβmpɤβ}}}\markboth{mpɤβmpɤβ}{}\classe{idph.2}
\begin{définition}\fra épais et mou\end{définition}
\begin{définition}\cmn 形容泡沫或蘑菇厚而软的样子\end{définition}
\begin{relation-sémantique}\synonyme{
\hyperlink{Ⓔmphɤβmphɤβ}{\textit{ \papi{mphɤβmphɤβ}}}
}\end{relation-sémantique}\end{entrée}

\begin{entrée}
\vedette{\hypertarget{Ⓔmpɕu}{\papi{ mpɕu}}}\markboth{mpɕu}{}
\classe{vs}
\paradigme{\textit{dir :} \jya nɯ-}
\begin{définition}\fra lisse\end{définition}
\begin{définition}\cmn 光滑\end{définition}
\begin{exemple}\jya laʁdɯn ɯ-jɯ ɲɯ-mpɕu\cmn 工具的把子是光滑的\end{exemple}
\begin{exemple}\jya tɤ-jtsi ɲɯ-mpɕu\cmn 柱子是光滑的\end{exemple}\begin{sous-entrée}
\vedette{\hypertarget{}{\papi{ ɣɤmpɕu}}}\markboth{ɣɤmpɕu}{}\classe{vt}
\paradigme{\textit{dir :} \jya nɯ-}
\begin{définition}\fra rendre lisse\end{définition}
\begin{définition}\cmn 弄光滑\end{définition}
\begin{exemple}\jya tɯ-ci kɯ tɕhɯrdu ɲo-ɣɤmpɕu ʑo\cmn 水把卵石磨光滑了\end{exemple}
\begin{exemple}\jya ɲɯ́-wɣ-ɣɤmpɕu tɕe kɤ-ŋga sɤscit\cmn 把衣服弄软了穿起来舒服\end{exemple}
\end{sous-entrée}\end{entrée}

\begin{entrée}
\vedette{\hypertarget{Ⓔmpɕa}{\papi{ mpɕa}}}\markboth{mpɕa}{}\classe{vt}
\paradigme{\textit{dir :} \jya tɤ-}
\begin{définition}\fra reprocher, ne pas pouvoir pardonner\end{définition}
\begin{définition}\cmn 责怪
\begin{déclaration} \étymologie{\papi{ⁿpʰʲa}}\end{déclaration}\end{définition}
\begin{exemple}\jya nɤʑɯɣ nɯ mɯ-tɤ-nɤma-t-a tɕe ma-tɤ-kɯ-mpɕa-a a ma\cmn 你的那个我没有做,不要责怪我\end{exemple}\end{entrée}

\begin{entrée}
\vedette{\hypertarget{Ⓔmpɕɤr}{\papi{ mpɕɤr}}}\markboth{mpɕɤr}{}\classe{vi}
\paradigme{\textit{dir :} \jya thɯ-}
\begin{définition}\fra beau\end{définition}
\begin{définition}\cmn 美,好看,好听
\begin{déclaration} \étymologie{\papi{mtɕʰor}}\end{déclaration}\end{définition}
\begin{relation-sémantique}\confer{
\hyperlink{Ⓔɯ-kɯmpɕɤr}{\textit{ \papi{ɯ-kɯmpɕɤr}}}
}\end{relation-sémantique}\begin{sous-entrée}
\vedette{\hypertarget{}{\papi{ ɣɤmpɕɤr}}}\markboth{ɣɤmpɕɤr}{}\classe{vt}
\begin{définition}\ 
\begin{déclaration}\grammar{caus}\end{déclaration}\end{définition}
\begin{définition}\fra rendre beau\end{définition}
\begin{définition}\cmn 令……变漂亮\end{définition}
\begin{exemple}\jya tɯpɤr pɯ-lat-a tɕe, mɯ-nɯ-ta-ɣɤmpɕɤr ɯ́-ŋu?\cmn 我拍照片的时候,把你拍得不好看吗?\end{exemple}
\begin{relation-sémantique}\confer{
\hyperlink{Ⓔnɤmpɕɤr}{\textit{ \papi{nɤmpɕɤr}}}
}\end{relation-sémantique}
\begin{relation-sémantique}\confer{
\hyperlink{Ⓔrɤmpɕɤr}{\textit{ \papi{rɤmpɕɤr}}}
}\end{relation-sémantique}
\end{sous-entrée}\end{entrée}

\begin{entrée}
\vedette{\hypertarget{Ⓔmpɕɯmɤr}{\papi{ mpɕɯmɤr}}}\markboth{mpɕɯmɤr}{}\classe{n}
\begin{définition}\fra célébration\end{définition}
\begin{définition}\cmn 庆祝\end{définition}
\begin{exemple}\jya ɯ-mpɕɯmɤr tɤ-βzu-t-a\cmn 我祝贺了他\end{exemple}
\end{entrée}

\begin{entrée}
\vedette{\hypertarget{Ⓔmphɤβmphɤβ}{\papi{ mphɤβmphɤβ}}}\markboth{mphɤβmphɤβ}{}\classe{idph.2}
\begin{définition}\fra épais et mou\end{définition}
\begin{définition}\cmn 形容泡沫或蘑菇厚而软的样子\end{définition}
\begin{relation-sémantique}\synonyme{
\hyperlink{Ⓔmpɤβmpɤβ}{\textit{ \papi{mpɤβmpɤβ}}}
}\end{relation-sémantique}\end{entrée}

\begin{entrée}
\vedette{\hypertarget{Ⓔmphɣaʁmphɣaʁ}{\papi{ mphɣaʁmphɣaʁ}}}\markboth{mphɣaʁmphɣaʁ}{}
\classe{idph.2}
\begin{définition}\fra très serré\end{définition}
\begin{définition}\cmn 形容紧的样子\end{définition}
\begin{exemple}\jya ɯ-mthɤɣ mphɣaʁmphɣaʁ ʑo ko-xtɕɤr\cmn 他把腰带系得很紧\end{exemple}\end{entrée}

\begin{entrée}
\vedette{\hypertarget{Ⓔmphrɤt}{\papi{ mphrɤt}}}\markboth{mphrɤt}{}
\classe{vs}
\paradigme{\textit{dir :} \jya tɤ-}
\begin{définition}\fra bien fermé, sans interstice\end{définition}
\begin{définition}\cmn 关紧,密封,没有缝隙
\begin{déclaration} \étymologie{\papi{ⁿpʰrod}}\end{déclaration}\end{définition}
\begin{exemple}\jya rgɤm ɲɯ-mphrɤt\cmn 盒子关得很密封\end{exemple}
\begin{exemple}\jya khɯɣɲɟɯ mɯ́j-mphrɤt tɕe a-lɤ-mphrɤt ɲɯ-ra\cmn 窗子没有关紧,要关紧\end{exemple}\acception{1}\begin{sous-entrée}
\vedette{\hypertarget{}{\papi{ sɯmphrɤt}}}\markboth{sɯmphrɤt}{}\classe{vt}
\paradigme{\textit{dir :} \jya tɤ-}
\begin{définition}\ 
\begin{déclaration}\grammar{caus}\end{déclaration}\end{définition}
\begin{définition}\fra mettre ensemble (des pièces) de façon parfaitement bien agencée\end{définition}
\begin{définition}\cmn 使……变得严密\end{définition}
\begin{exemple}\jya kɤ-sprɤt to-sɯ-mphrɤt\cmn 他(把零件)组合得很严密\end{exemple}\acception{2}
\paradigme{\textit{dir :} \jya \_}
\begin{définition}\fra bien fermer\end{définition}
\begin{définition}\cmn 关紧\end{définition}
\begin{exemple}\jya aʑo khɯɣɲɟɯ kɤ-sɯmphrat-a\cmn 我关了窗子\end{exemple}
\begin{exemple}\jya kɯm mɯ-chɤ-sɯmphrat-a tɕe ɯ-pɕi tɤ-zgra nɯ ɲɯ-sɤmtshɤm\cmn 我没有吧门关紧听到外面的声音\end{exemple}
\end{sous-entrée}\end{entrée}

\begin{entrée}
\vedette{\hypertarget{Ⓔmphrɯɣ}{\papi{ mphrɯɣ}}}\markboth{mphrɯɣ}{}
\classe{n}
\begin{définition}\fra habit tibétain en laine\end{définition}
\begin{définition}\cmn 氆氇
\begin{déclaration} \étymologie{\papi{pʰrug}}\end{déclaration}\end{définition}\end{entrée}

\begin{entrée}
\vedette{\hypertarget{Ⓔmphrɯmdɯt}{\papi{ mphrɯmdɯt}}}\markboth{mphrɯmdɯt}{}\classe{n}
\begin{définition}\fra groupe de neuf nœuds (sur un khatag ou avec un fil normal)\end{définition}
\begin{définition}\cmn 在哈达上打九个结
\begin{déclaration} \étymologie{\papi{ⁿpʰreŋ.mdud}}\end{déclaration}\end{définition}
\end{entrée}

\begin{entrée}
\vedette{\hypertarget{Ⓔmphrɯmɯ}{\papi{ mphrɯmɯ}}}\markboth{mphrɯmɯ}{}\classe{n}
\begin{définition}\fra prédiction\end{définition}
\begin{définition}\cmn 看相;算命\end{définition}
\begin{exemple}\jya βlama kɯ a-mphrɯmɯ pa-ru\cmn 喇嘛给我算了命\end{exemple}
\begin{relation-sémantique}\confer{
\hyperlink{Ⓔrɯmphrɯmɯ}{\textit{ \papi{rɯmphrɯmɯ}}}
}\end{relation-sémantique}\end{entrée}

\begin{entrée}
\vedette{\hypertarget{Ⓔmphruwa}{\papi{ mphruwa}}}\markboth{mphruwa}{}\classe{n}
\begin{définition}\fra chapelet\end{définition}
\begin{définition}\cmn 玛尼珠子\end{définition}\end{entrée}

\begin{entrée}
\vedette{\hypertarget{Ⓔmphɯl}{\papi{ mphɯl}}}\markboth{mphɯl}{}\classe{vi}
\paradigme{\textit{dir :} \jya nɯ-}
\begin{définition}\ 
\begin{déclaration} \étymologie{\papi{ⁿpʰel}}\end{déclaration}\end{définition}
\begin{définition}\fra se multiplier (animaux)\end{définition}
\begin{définition}\cmn 繁殖(动物)\end{définition}
\begin{exemple}\jya paʁ ɲɤ-mphɯl\cmn 猪繁殖了\end{exemple}\end{entrée}

\begin{entrée}
\vedette{\hypertarget{Ⓔmphɯli}{\papi{ mphɯli}}}\markboth{mphɯli}{}\classe{n}
\begin{définition}\fra fibres de sésame\end{définition}
\begin{définition}\cmn 芝麻皮(用来制火绒、麻布的纬线)\end{définition}\end{entrée}

\begin{entrée}
\vedette{\hypertarget{Ⓔmphɯr}{\papi{ mphɯr}}}\markboth{mphɯr}{}
\classe{vt}
\paradigme{\textit{dir :} \jya kɤ-}
\paradigme{\textit{dir :} \jya tɤ-}
\begin{définition}\fra envelopper\end{définition}
\begin{définition}\cmn 包\end{définition}
\begin{exemple}\jya kɯ-chi ɕoʁɕoʁ ɯ-ŋgɯ tɤ-mphɯr-a\cmn 我把这颗糖包在纸里了\end{exemple}
\begin{exemple}\jya ki tɯ-ŋga ki tɤ-mphɯr\cmn 你把衣服折起来\end{exemple}
\begin{exemple}\jya smɤn tɤ-mphɯr-a\cmn 我包了药\end{exemple}
\begin{exemple}\jya ɕkɤbɯ kɤ-mphɯr-a\cmn 我包了韭菜包子\end{exemple}\begin{sous-entrée}
\vedette{\hypertarget{}{\papi{ amphɯmphɯr}}}\markboth{amphɯmphɯr}{}\classe{vs}
\begin{définition}\fra envelopper de plusieurs couches\end{définition}
\begin{définition}\cmn 一层一层地裹着\end{définition}
\end{sous-entrée}\begin{sous-entrée}
\vedette{\hypertarget{}{\papi{ nɤmphoʁmphɯr}}}\markboth{nɤmphoʁmphɯr}{}\classe{vt}
\paradigme{\textit{dir :} \jya kɤ-}
\begin{définition}\fra conserver qqch en l'enveloppant dans toutes sortes de choses\end{définition}
\begin{définition}\cmn 为了保存某个东西,把它一直包来包去\end{définition}
\begin{exemple}\jya ki laχtɕha ki a-mɤ-nɯ-me nɯ-sɯso-t-a tɕe, pɯ-nɤmphoʁmphɯr-a ntsɯ ɕti\cmn 我怕把这个东西丢失了,所以一直把它包在一个地方\end{exemple}
\end{sous-entrée}\begin{sous-entrée}
\vedette{\hypertarget{}{\papi{ rɤmphɯr}}}\markboth{rɤmphɯr}{}\classe{vi}
\paradigme{\textit{dir :} \jya tɤ-}
\begin{définition}\fra préparer des pains ou des raviolis\end{définition}
\begin{définition}\cmn 包包子;包饺子\end{définition}
\begin{exemple}\jya kɤ-rɤmphɯr-a\cmn 我包了包子\end{exemple}
\end{sous-entrée}\begin{sous-entrée}
\vedette{\hypertarget{}{\papi{ ʑɣɤmphɯr}}}\markboth{ʑɣɤmphɯr}{}\classe{vi}
\paradigme{\textit{dir :} \jya kɤ-}
\begin{définition}\ 
\begin{déclaration}\grammar{refl}\end{déclaration}\end{définition}
\begin{définition}\fra s'envelopper dans\end{définition}
\begin{définition}\cmn 把自己裹在…里\end{définition}
\end{sous-entrée}\begin{sous-entrée}
\vedette{\hypertarget{}{\papi{ ʑɣɤsɯmphɯr}}}\markboth{ʑɣɤsɯmphɯr}{}
\paradigme{\textit{dir :} \jya tɤ-}
\begin{définition}\ 
\begin{déclaration}\grammar{refl}\end{déclaration}
\begin{déclaration}\grammar{caus}\end{déclaration}\end{définition}
\begin{définition}\fra s'envelopper dans\end{définition}
\begin{définition}\cmn 把自己裹在…里\end{définition}
\end{sous-entrée}\end{entrée}

\begin{entrée}
\vedette{\hypertarget{Ⓔmphɯrpa}{\papi{ mphɯrpa}}}\markboth{mphɯrpa}{}\classe{n}
\begin{définition}\fra bâton pour frapper les chiens\end{définition}
\begin{définition}\cmn 打狗棒\end{définition}
\end{entrée}

\begin{entrée}
\vedette{\hypertarget{Ⓔmpja}{\papi{ mpja}}}\markboth{mpja}{}\classe{vs}
\paradigme{\textit{dir :} \jya thɯ-}
\begin{définition}\fra chaud\end{définition}
\begin{définition}\cmn 热\end{définition}
\begin{exemple}\jya tɯ-ŋga ɲɯ-mpja\cmn 衣服很热\end{exemple}
\begin{exemple}\jya kɯ-mpja tɤ-ndze\cmn 你吃热的吧\end{exemple}
\begin{exemple}\jya nɤ-ŋga ra kɯ-mpja tɤ-ŋge\cmn 衣服穿暖一些\end{exemple}
\begin{relation-sémantique}\confer{
\hyperlink{Ⓔɣɤmpja}{\textit{ \papi{ɣɤmpja}}}
}\end{relation-sémantique}\end{entrée}

\begin{entrée}
\vedette{\hypertarget{Ⓔmpɯ}{\papi{ mpɯ}}}\markboth{mpɯ}{}\classe{vs}
\paradigme{\textit{dir :} \jya kɤ-}\acception{1}
\begin{définition}\fra être mou\end{définition}
\begin{définition}\cmn 柔软\end{définition}\acception{2}
\begin{définition}\fra être tendre\end{définition}
\begin{définition}\cmn 嫩\end{définition}
\begin{exemple}\jya tɯ-pa ɲɯ-mpɯ\cmn 坐的地方很软\end{exemple}
\begin{exemple}\jya sɤtɕha ko-mpɯ\cmn 地变软了\end{exemple}
\begin{exemple}\jya @yangyu nɯnɯ chɯ́-wɣ-pu ɕɯŋgɯ tɕe rko ri, thɯ-smi tɕe ɲɯ-mpɯ ŋu loβ\cmn 洋芋煨熟之前很硬,熟了之后就是软的\end{exemple}\begin{sous-entrée}
\vedette{\hypertarget{}{\papi{ ɣɤmpɯ}}}\markboth{ɣɤmpɯ}{}\classe{vt}
\paradigme{\textit{dir :} \jya kɤ-}
\paradigme{\textit{dir :} \jya nɯ-}
\begin{définition}\ 
\begin{déclaration}\grammar{caus}\end{déclaration}\end{définition}
\end{sous-entrée}\begin{sous-entrée}
\vedette{\hypertarget{}{\papi{ nɤmpɯ}}}\markboth{nɤmpɯ}{}\classe{vt}
\begin{définition}\ 
\begin{déclaration}\grammar{trop}\end{déclaration}\end{définition}
\begin{définition}\fra trouver mou\end{définition}
\begin{définition}\cmn 觉得软\end{définition}
\end{sous-entrée}\end{entrée}

\begin{entrée}
\vedette{\hypertarget{Ⓔmpɯmnu}{\papi{ mpɯmnu}}}\markboth{mpɯmnu}{}\classe{vs}
\paradigme{\textit{dir :} \jya nɯ-}
\begin{définition}\ 
\begin{déclaration}\grammar{comp}\end{déclaration}\end{définition}
\begin{définition}\fra mou\end{définition}
\begin{définition}\cmn 柔软\end{définition}
\begin{exemple}\jya kɯki tɯ-nga ki ɲɯ-mpɯmnu tɕe, kɤ-ŋga ɲɯ-sɤscit\cmn 这件衣服很软,穿起来很舒服\end{exemple}\end{entrée}

\begin{entrée}
\vedette{\hypertarget{Ⓔmqlaʁ}{\papi{ mqlaʁ}}}\markboth{mqlaʁ}{}\classe{vt}
\paradigme{\textit{dir :} \jya thɯ-}
\begin{définition}\fra avaler\end{définition}
\begin{définition}\cmn 吞\end{définition}
\begin{exemple}\jya tɯ-mgo thɯ-mqlaʁ-a\cmn 我把饭吞了\end{exemple}
\begin{exemple}\jya tɯ-ci thɯ-mqlaʁ-a\cmn 我咽了水\end{exemple}
\begin{exemple}\jya a-mci thɯ-mqlaʁ-a\cmn 我吞了口水\end{exemple}
\begin{exemple}\jya smɤn thɯ-mqlaʁ-a\cmn 我吞了药\end{exemple}\end{entrée}

\begin{entrée}
\vedette{\hypertarget{Ⓔmtɕhaʁnɤmtɕhaʁ}{\papi{ mtɕhaʁnɤmtɕhaʁ}}}\markboth{mtɕhaʁnɤmtɕhaʁ}{}
\classe{idph.3}
\begin{définition}\fra bruit que l'on fait lorsqu'on évalue le goût d'un aliment après l'avoir avalé\end{définition}
\begin{définition}\cmn 细嚼慢咽的声音\end{définition}
\begin{exemple}\jya kɤ-rɯndzɤtshi ɯ-ʁjiz mtɕhaʁnɤmtɕhaʁ mɯ́j-ɣi\cmn 他细嚼慢咽发出声音,不想吃饭\end{exemple}\begin{sous-entrée}
\vedette{\hypertarget{}{\papi{ sɤmtɕhaʁmtɕhaʁ}}}\markboth{sɤmtɕhaʁmtɕhaʁ}{}\classe{vt}
\begin{exemple}\jya ɲɯ-sɤmtɕhaʁmtɕhaʁ ndzɤtshi ɯ-ʁjiz mɯ́j-ɣi\cmn 他细嚼慢咽发出声音,不想吃饭\end{exemple}
\end{sous-entrée}\end{entrée}

\begin{entrée}
\vedette{\hypertarget{Ⓔmtɕhɤnmbrɯ}{\papi{ mtɕhɤnmbrɯ}}}\markboth{mtɕhɤnmbrɯ}{}\classe{n}
\begin{définition}\fra grains destinés aux monastères\end{définition}
\begin{définition}\cmn 供神的粮食
\begin{déclaration} \étymologie{\papi{mtɕʰod.ⁿbru}}\end{déclaration}\end{définition}
\begin{exemple}\jya rgɯnba smi pɯ́-wɣ-βlɯ tɕe, (fsaŋ lɤ́-wɣ-ta tɕe) tɤ-khɯ tu-tɕɤt nɯ ɣɯ ɯ-ŋgɯ kú-wɣ-lɤt cho rgɯnba lú-wɣ-sɤri ŋu\cmn 在寺庙里求烟仔,就把(粮食)用柏树枝熏一下供神\end{exemple}
\begin{relation-sémantique}\synonyme{
\hyperlink{Ⓔkɤsɤri}{\textit{ \papi{kɤsɤri}}}
}\end{relation-sémantique}\end{entrée}

\begin{entrée}
\vedette{\hypertarget{Ⓔmtɕhɤnmi}{\papi{ mtɕhɤnmi}}}\markboth{mtɕhɤnmi}{}\classe{n}
\begin{définition}\fra lampe à beurre allumée\end{définition}
\begin{définition}\cmn 点过的酥油灯
\begin{déclaration} \étymologie{\papi{mtɕʰod.me}}\end{déclaration}\end{définition}
\end{entrée}

\begin{entrée}
\vedette{\hypertarget{Ⓔmtɕhɤtkho}{\papi{ mtɕhɤtkho}}}\markboth{mtɕhɤtkho}{}
\classe{n}
\begin{définition}\fra endroit où l'on fait des fumigations rituelles\end{définition}
\begin{définition}\cmn 烧香的地方;经堂
\begin{déclaration} \étymologie{\papi{mtɕʰod.kʰaŋ}}\end{déclaration}\end{définition}\end{entrée}

\begin{entrée}
\vedette{\hypertarget{Ⓔmtɕhi}{\papi{ mtɕhi}}}\markboth{mtɕhi}{}
\classe{n}
\begin{définition}\fra argousier\end{définition}
\begin{définition}\cmn 沙棘\end{définition}
\begin{exemple}\jya mtɕhi nɯ zgoku aʁɤndɯndɤt tu-ɬoʁ cha, ɯ-ru mɤ-astu, kɤ-jʁɯ-jʁu ŋu, ɯ-rtaʁ dɤn, ɯ-mdzu wuma ʑo mtɕoʁ, jpum. ɯ-ru nɯ kɯ-ɲaʁ ŋu. ɯ-rqhu jaʁ, ɯ-si nɯ kɤ-ɣɯŋgɯŋgɯ kɯ-fse ŋu, ɯ-rtaʁ ɯ-rqhu nɯ mpɕu, ɯ-si nɯ kɤ-βlɯ kɯnɤ mɤ-sna, ɯ-jwaʁ nɯ kɯ-pɣi tsa tɕe kɯ-ndɯ-ndɯβ, pɣɤmuj ɯ-tshɯɣa fse. ɯ-mɯntoʁ kɤ-mto me, ɯ-mat kɤ-tshoʁ ɕɯmɯma tɕe, ldʑaŋsɤr ŋu, thɯ-tɯt wuma ʑo qarŋe, ɯ-tɯ-tɕur saχaʁ ri mɤ-sɤndɤɣ.\cmn 沙棘在高山上到处生长,树干不直,弯弯曲曲的,枝桠多,刺又锋利又粗。树干是黑色,树皮很厚。树木是由一圈一圈的年轮组成的,枝桠的树皮是光滑的。木质连烧火都不好。叶子是灰色的,很小,形状像羽毛。看不见花,果实才开始结时是浅绿色,成熟后是黄色。吃起来很酸,但没有毒性。\end{exemple}\end{entrée}

\begin{entrée}
\vedette{\hypertarget{Ⓔmtɕhinaʁ}{\papi{ mtɕhinaʁ}}}\markboth{mtɕhinaʁ}{}\classe{n}
\begin{définition}\fra espèce de chien dont la bouche est noire et le corps rouge\end{définition}
\begin{définition}\cmn 黑嘴巴的狗\end{définition}\end{entrée}

\begin{entrée}
\vedette{\hypertarget{Ⓔmtɕho}{\papi{ mtɕho}}}\markboth{mtɕho}{}
\classe{vt}
\paradigme{\textit{dir :} \jya pɯ-}
\begin{définition}\fra fixer\end{définition}
\begin{définition}\cmn 固定\end{définition}
\begin{exemple}\jya pjɯ-mtɕham-a\cmn 我固定\end{exemple}
\begin{exemple}\jya rɟɤɕi thɯ-rku-t-a tɕe, pɯ-mtɕho-t-a tɕe nɯ-sɤsɯɣ-a.\cmn 我套了楦头,把它固定了并弄紧了\end{exemple}
\begin{exemple}\jya qaʁ ɯ-jɯ thɯ-tshoʁ-a tɕe thɯ-mtɕho-t-a\cmn 我把锄头装上了把子并把它固定了\end{exemple}
\begin{relation-sémantique}\synonyme{
\hyperlink{Ⓔrɤsta}{\textit{ \papi{rɤsta}}}
}\end{relation-sémantique}\end{entrée}

\begin{entrée}
\vedette{\hypertarget{Ⓔmtɕhortɯn}{\papi{ mtɕhortɯn}}}\markboth{mtɕhortɯn}{}
\classe{n}
\begin{définition}\fra stûpa\end{définition}
\begin{définition}\cmn 塔
\begin{déclaration} \étymologie{\papi{mtɕʰod.rten}}\end{déclaration}\end{définition}\end{entrée}

\begin{entrée}
\vedette{\hypertarget{Ⓔmtɕhostɤt}{\papi{ mtɕhostɤt}}}\markboth{mtɕhostɤt}{}\classe{vt}
\begin{définition}\fra louer\end{définition}
\begin{définition}\cmn 赞美;奉承
\begin{déclaration}\use{古语}\end{déclaration}\end{définition}
\begin{exemple}\jya jisŋi skɤrma kɯ-sna tu-ta-mtɕhostɤt-nɯ ŋu tɕe\cmn 在今天这个吉利的日子,我赞颂你们(山神)\end{exemple}\end{entrée}

\begin{entrée}
\vedette{\hypertarget{Ⓔmtɕhot}{\papi{ mtɕhot}}}\markboth{mtɕhot}{}\classe{interj}
\begin{définition}\fra mot de prière\end{définition}
\begin{définition}\cmn 求神仙保佑\end{définition}\begin{sous-entrée}
\vedette{\hypertarget{}{\papi{ ɯ-mtɕhot}}}\markboth{ɯ-mtɕhot}{}\classe{np}
\begin{définition}\fra morceau de pain jeté lors d'un souhait\end{définition}
\begin{définition}\cmn 请求神仙保佑的时候扔上去的一块馍馍\end{définition}
\begin{définition}\jya 
\begin{déclaration} \étymologie{\papi{mtɕʰod}}\end{déclaration}\end{définition}
\end{sous-entrée}\end{entrée}

\begin{entrée}
\vedette{\hypertarget{Ⓔmtɕhɯβ}{\papi{ mtɕhɯβ}}}\markboth{mtɕhɯβ}{}
\classe{vi}\acception{1}
\paradigme{\textit{dir :} \jya nɯ-}
\paradigme{\textit{dir :} \jya kɤ-}
\begin{définition}\fra mouiller, s’infiltrer\end{définition}
\begin{définition}\cmn 浸入;渗入\end{définition}
\begin{exemple}\jya tɯ-ŋga ɯ-ŋgɯ tɯ-ci ɲɤ-ɕe tɕe ɲɤ-mtɕhɯβ\cmn 水渗入到衣服里,浸湿了衣服\end{exemple}
\begin{exemple}\jya a-ŋga ɲɤ-k-ɤci-ci tɕe tɯ-ci kɯ ɲɤ-mtɕhɯβ\cmn 我的衣服湿了,浸泡了水\end{exemple}
\begin{exemple}\jya si (ɕoŋtɕa) ɯ-ŋgɯ tɯ-ci ɲɤ-mtɕhɯβ\cmn 水渗透到木头里\end{exemple}
\begin{exemple}\jya stoʁ ko-mtɕhɯβ\cmn 胡豆(被水)泡了\end{exemple}\acception{2}
\paradigme{\textit{dir :} \jya kɤ-}
\begin{définition}\fra ajouter\end{définition}
\begin{définition}\cmn 添加\end{définition}
\begin{exemple}\jya kɤ-ndza mɯ́j-nɤrtaʁ tɕe kɤ-mtɕhɯβ ɲɯ-ɬoʁ\cmn 他觉得食物不够,要加一点\end{exemple}
\begin{exemple}\jya aʑo kɤ-ndza mɯ́j-nɤrtaʁ-a tɕe ku-kɯ-mtɕhɯβ-a ɲɯ-ntshi\cmn 我觉得食物不够,你要给我添一点\end{exemple}\begin{sous-entrée}
\vedette{\hypertarget{}{\papi{ sɤmtɕhɯβ}}}\markboth{sɤmtɕhɯβ}{}\classe{vi}
\begin{définition}\ 
\begin{déclaration}\grammar{apass}\end{déclaration}\end{définition}
\begin{définition}\fra ajouter\end{définition}
\begin{définition}\cmn 给别人添加一点\end{définition}
\end{sous-entrée}\end{entrée}

\begin{entrée}
\vedette{\hypertarget{Ⓔmtɕhɯtsaʁ}{\papi{ mtɕhɯtsaʁ}}}\markboth{mtɕhɯtsaʁ}{}\classe{n}
\begin{définition}\fra une maladie (boutons purulents sur la bouche, fièvre)\end{définition}
\begin{définition}\cmn 嘴唇上生疮,化脓,发烧\end{définition}
\begin{exemple}\jya ɯ-mtɕhɯtsaʁ ɲɤ-ɬoʁ\cmn 他嘴上生了疮\end{exemple}
\begin{relation-sémantique}\confer{
\hyperlink{Ⓔnɯmtɕhɯtsaʁ}{\textit{ \papi{nɯmtɕhɯtsaʁ}}}
}\end{relation-sémantique}\end{entrée}

\begin{entrée}
\vedette{\hypertarget{Ⓔmtɕoʁ}{\papi{ mtɕoʁ}}}\markboth{mtɕoʁ}{}\classe{vs}
\paradigme{\textit{dir :} \jya tɤ-}
\begin{définition}\fra aiguisé\end{définition}
\begin{définition}\cmn 锋利\end{définition}
\begin{exemple}\jya mbrɯtɕɯ ɲɯ-mtɕoʁ\cmn 刀很锋利\end{exemple}
\begin{exemple}\jya tɯmnɯ ɲɯ-mtɕoʁ\cmn 锥子很锋利\end{exemple}
\begin{exemple}\jya rɟaŋsoʁ ɲɯ-mtɕoʁ\cmn 锯子很锋利\end{exemple}
\begin{exemple}\jya mɤ-kɯ-mtɕoʁ\cmn 钝\end{exemple}\begin{sous-entrée}
\vedette{\hypertarget{}{\papi{ ɣɤmtɕoʁ}}}\markboth{ɣɤmtɕoʁ}{}\classe{vt}
\paradigme{\textit{dir :} \jya tɤ-}
\begin{définition}\ 
\begin{déclaration}\grammar{caus}\end{déclaration}\end{définition}
\begin{définition}\fra aiguiser\end{définition}
\begin{définition}\cmn 使锋利\end{définition}
\begin{exemple}\jya mbrɯtɕɯ cho-fse tɕe to-ɣɤmtɕoʁ\cmn 把刀磨得很锋利了\end{exemple}
\begin{relation-sémantique}\confer{
\hyperlink{Ⓔamtɕoʁ}{\textit{ \papi{amtɕoʁ}}}
}\end{relation-sémantique}
\end{sous-entrée}\end{entrée}

\begin{entrée}
\vedette{\hypertarget{Ⓔmtɕɯr}{\papi{ mtɕɯr}}}\markboth{mtɕɯr}{}\classe{vi}
\paradigme{\textit{dir :} \jya \_}
\begin{définition}\fra tourner\end{définition}
\begin{définition}\cmn 转动\end{définition}
\begin{exemple}\jya aʑo kɤ-mtɕɯr-a\cmn 我转过了(转身)\end{exemple}
\begin{exemple}\jya a-kɤrnoʁ ɲɯ-mtɕɯr\cmn 我头晕\end{exemple}
\begin{exemple}\jya mkhɯrlu ɲɯ-mtɕɯr\cmn 车在转动\end{exemple}
\begin{exemple}\jya zdɯm ɲɯ-mtɕɯr\cmn 云在转动\end{exemple}
\begin{exemple}\jya tɯ-mɯ mɤ-mtɕɯr zdɯm mtɕɯr\cmn 人世间的事变化无常\end{exemple}
\begin{relation-sémantique}\confer{
\hyperlink{Ⓔnɤmtɕɯrlu}{\textit{ \papi{nɤmtɕɯrlu}}}
}\end{relation-sémantique}
\begin{relation-sémantique}\confer{
\hyperlink{Ⓔsɯmtɕɯr}{\textit{ \papi{sɯmtɕɯr}}}
}\end{relation-sémantique}\end{entrée}

\begin{entrée}
\vedette{\hypertarget{Ⓔmthu}{\papi{ mthu}}}\markboth{mthu}{}\classe{vs}
\paradigme{\textit{dir :} \jya thɯ-}\acception{1}
\begin{définition}\fra qui a confiance en lui\end{définition}
\begin{définition}\cmn 有自信(性情)\end{définition}
\begin{exemple}\jya nɤ-sɯm ɯ-tɯ-mthu nɯ\cmn 你自以为是\end{exemple}
\begin{exemple}\jya ɯ-snoŋwa ɲɯ-mthu\cmn 他很有自信,觉得自高自大\end{exemple}\acception{2}
\begin{définition}\fra qui a de la graisse\end{définition}
\begin{définition}\cmn 膘情好;强壮\end{définition}
\begin{exemple}\jya ɯ-ɲɤm ɲɯ-pe tɕe ɲɯ-mthu\cmn (牛)膘情很好,很强壮\end{exemple}
\begin{exemple}\jya ɯ-ɲɤm mɯ́j-pe tɕe mɯ́j-mthu, ndʐaβ ɲɯ-ŋu\cmn 牛膘情不好,不强壮,快要倒了\end{exemple}
\begin{exemple}\jya nɯŋɤdo nɯ ɲɯ-mthu, mɯ́j-mthu\cmn 老母牛膘情很好,膘情不好\end{exemple}\acception{4}
\begin{définition}\fra trop haut (coup de fusil)\end{définition}
\begin{définition}\cmn 打枪瞄高
\begin{déclaration} \étymologie{\papi{mtʰo}}\end{déclaration}\end{définition}
\begin{relation-sémantique}\antonyme{
\hyperlink{Ⓔʁma}{\textit{ \papi{ʁma}}}
}\end{relation-sémantique}
\begin{relation-sémantique}\confer{
\hyperlink{Ⓔnɤmthu}{\textit{ \papi{nɤmthu}}}
}\end{relation-sémantique}\end{entrée}

\begin{entrée}
\vedette{\hypertarget{Ⓔmthama}{\papi{ mthama}}}\markboth{mthama}{}\classe{n}\acception{1}
\begin{définition}\fra le dernier\end{définition}
\begin{définition}\cmn 最后;最终\end{définition}
\begin{exemple}\jya tɕetha ɯ-qhu mthama tɕe kɯ-pe ci βze\cmn 最终会有好结果\end{exemple}\acception{2}
\begin{définition}\fra le plus mauvais\end{définition}
\begin{définition}\cmn 最低级
\begin{déclaration} \étymologie{\papi{mtʰa.ma}}\end{déclaration}\end{définition}
\begin{exemple}\jya nɤki laχtɕha nɯ ʁo stu mɤ-kɯ-pe mthama ʑo nɯ ɲɯ-ɕti\cmn 这个东西是最差的\end{exemple}\end{entrée}

\begin{entrée}
\vedette{\hypertarget{Ⓔmthɤrpɯ}{\papi{ mthɤrpɯ}}}\markboth{mthɤrpɯ}{}\classe{n}
\begin{définition}\fra décoration en argent qui pend de la ceinture\end{définition}
\begin{définition}\cmn 垂吊在腰带上的银装饰品\end{définition}\end{entrée}

\begin{entrée}
\vedette{\hypertarget{Ⓔmthuri}{\papi{ mthuri}}}\markboth{mthuri}{}\classe{n}
\begin{définition}\fra feu mon père\end{définition}
\begin{définition}\cmn 已故的父亲
\begin{déclaration} \étymologie{\papi{mtʰo.ris}}\end{déclaration}\end{définition}
\begin{exemple}\jya a-wa mthuri\cmn 我已故的父亲\end{exemple}\end{entrée}

\begin{entrée}
\vedette{\hypertarget{Ⓔmthɯ}{\papi{ mthɯ}}}\markboth{mthɯ}{}
\classe{n}
\begin{définition}\fra mantra\end{définition}
\begin{définition}\cmn 咒经
\begin{déclaration} \étymologie{\papi{mtʰu}}\end{déclaration}\end{définition}
\begin{exemple}\jya mthɯ cho-lɤt\cmn 他念了咒经\end{exemple}
\begin{exemple}\jya ɯ-mthɯ thɯ-lat-a\cmn 我对他念了咒语\end{exemple}
\begin{relation-sémantique}\confer{
\hyperlink{ⒺnɯmthɯⒽ2}{\textit{ \papi{nɯmthɯ2}}}
}\end{relation-sémantique}\end{entrée}

\begin{entrée}
\vedette{\hypertarget{Ⓔmthɯmɤr}{\papi{ mthɯmɤr}}}\markboth{mthɯmɤr}{}\classe{n}
\begin{définition}\fra sceau\end{définition}
\begin{définition}\cmn 印章\end{définition}
\begin{exemple}\jya mthɯmɤr pɯ-ta-t-a\cmn 我盖了个章\end{exemple}
\begin{relation-sémantique}\synonyme{
\hyperlink{Ⓔthotsi}{\textit{ \papi{thotsi}}}
}\end{relation-sémantique}\end{entrée}

\begin{entrée}
\vedette{\hypertarget{Ⓔmthɯrnda}{\papi{ mthɯrnda}}}\markboth{mthɯrnda}{}
\classe{n}
\begin{définition}\fra rênes\end{définition}
\begin{définition}\cmn 缰绳
\begin{déclaration} \étymologie{\papi{mtʰur.mda}}\end{déclaration}\end{définition}\end{entrée}

\begin{entrée}
\vedette{\hypertarget{Ⓔmthɯt}{\papi{ mthɯt}}}\markboth{mthɯt}{}\classe{vt}
\paradigme{\textit{dir :} \jya pɯ-}
\paradigme{\textit{dir :} \jya thɯ-}
\begin{définition}\fra relier, rallonger\end{définition}
\begin{définition}\cmn 连接起来(补充一部分)
\begin{déclaration} \étymologie{\papi{mtʰud}}\end{déclaration}\end{définition}
\begin{exemple}\jya ki tɯmbri ki mɯ́j-rtaʁ tɕe kɤ-mthɯt ɲɯ-ɬoʁ\cmn 这条绳子不够,需要把它连接起来\end{exemple}
\begin{exemple}\jya tɤ-ri nɯ kɯ mɯ́j-ɕaβ tɕe kɤ-mthɯt-a\cmn 绳子不够长,我就接了一段\end{exemple}
\begin{exemple}\jya ɯ-ŋga ɯ-ndo pɯ-mthɯt-a\cmn 我补了他的衣角\end{exemple}
\begin{exemple}\jya tɤ-pɤloʁ ɯ-ku thɯ-mthɯt-a\cmn 我补了袖子\end{exemple}
\begin{relation-sémantique}\synonyme{
\hyperlink{Ⓔsɤlɤɣɯ}{\textit{ \papi{sɤlɤɣɯ}}}
}\end{relation-sémantique}\end{entrée}

\begin{entrée}
\vedette{\hypertarget{Ⓔmthɯxtɕɤr}{\papi{ mthɯxtɕɤr}}}\markboth{mthɯxtɕɤr}{}
\classe{n}
\begin{définition}\fra ceinture (large, tissée à la main)\end{définition}
\begin{définition}\cmn 腰带\end{définition}
\begin{relation-sémantique}\confer{
\hyperlink{Ⓔtɯ-mthɤɣ}{\textit{ \papi{tɯ-mthɤɣ}}}
}\end{relation-sémantique}
\begin{relation-sémantique}\confer{
\hyperlink{Ⓔxtɕɤr}{\textit{ \papi{xtɕɤr}}}
}\end{relation-sémantique}\end{entrée}

\begin{entrée}
\vedette{\hypertarget{Ⓔmti}{\papi{ mti}}}\markboth{mti}{} (\variante{mtɯj}) 
\classe{n}
\begin{définition}\fra turquoise\end{définition}
\begin{définition}\cmn 碧玉;绿松石\end{définition}\end{entrée}

\begin{entrée}
\vedette{\hypertarget{ⒺmtoⒽ2}{\papi{ mto}}}\markboth{mto}{}\homonyme{2}\classe{vt}
\paradigme{\textit{dir :} \jya pɯ-}
\begin{définition}\fra voir\end{définition}
\begin{définition}\cmn 看见\end{définition}
\begin{exemple}\jya lɯlu kɯ aʑo ʁnɯz ʑo ka-ndo pɯ-mto-t-a\cmn 我看见过猫抓了两只(小鸡)\end{exemple}
\begin{sous-entrée}
\vedette{\hypertarget{}{\papi{ amto}}}\markboth{amto}{}\classe{vi}
\begin{définition}\fra être visible\end{définition}
\begin{définition}\cmn 看得见\end{définition}
\begin{exemple}\jya khɯɣɲɟɯ ju-kɯ-ru tɕe, qhaqhu nɯra tɯrme nɯra amto\cmn 往窗子外面看的话,看得见房子后面的人\end{exemple}
\end{sous-entrée}\begin{sous-entrée}
\vedette{\hypertarget{}{\papi{ ɣɤmto}}}\markboth{ɣɤmto}{}\classe{vt}
\paradigme{\textit{dir :} \jya tɤ-}
\begin{définition}\fra rendre la vue (à un aveugle)\end{définition}
\begin{définition}\cmn 令(盲人)复明\end{définition}
\begin{relation-sémantique}\confer{
\hyperlink{Ⓔamɯmto}{\textit{ \papi{amɯmto}}}
}\end{relation-sémantique}
\begin{relation-sémantique}\confer{
\hyperlink{Ⓔsɤmto}{\textit{ \papi{sɤmto}}}
}\end{relation-sémantique}
\end{sous-entrée}\begin{sous-entrée}
\vedette{\hypertarget{}{\papi{ mto}}}\markboth{mto}{}\classe{vs}
\paradigme{\textit{dir :} \jya tɤ-}\acception{1}
\begin{définition}\fra être capable de voir\end{définition}
\begin{définition}\cmn 看得见\end{définition}
\begin{exemple}\jya ɯ-mɲaʁ ɲɯ-mto ʂɯŋʂɯŋ ʑo\cmn 他眼睛看得很清楚(视力很好)\end{exemple}
\begin{exemple}\jya ɯ-mɲaʁ χchoʁe ni to-mto\cmn 他的双眼复明了\end{exemple}\acception{2}
\begin{définition}\fra être fiable (prédiction)\end{définition}
\begin{définition}\cmn 灵(算卦)\end{définition}
\begin{exemple}\jya ɯ-mphrɯmɯ wuma ʑo mto\cmn 他算的卦很灵\end{exemple}
\end{sous-entrée}\begin{sous-entrée}
\vedette{\hypertarget{}{\papi{ sɯmto}}}\markboth{sɯmto}{}\classe{vt}
\paradigme{\textit{dir :} \jya pɯ-}
\begin{définition}\ 
\begin{déclaration}\grammar{caus}\end{déclaration}\end{définition}
\begin{définition}\fra montrer, laisser voir\end{définition}
\begin{définition}\cmn 让人看见;给人看\end{définition}
\end{sous-entrée}\begin{sous-entrée}
\vedette{\hypertarget{}{\papi{ ʑɣɤmto}}}\markboth{ʑɣɤmto}{}\classe{vi}
\paradigme{\textit{dir :} \jya pɯ-}
\begin{définition}\ 
\begin{déclaration}\grammar{refl}\end{déclaration}\end{définition}
\begin{définition}\fra se voir\end{définition}
\begin{définition}\cmn 看到自己\end{définition}
\begin{exemple}\jya tɯ-ci ɯ-ŋgɯ ɕ-pɯ-ru-a ri, pjɯ-ntɕhar-a ɲɯ-ŋu tɕe pɯ-ʑɣɤmto-a\cmn 我往水里看了一下,里面有我的倒影,我看到我自己了\end{exemple}
\end{sous-entrée}\begin{sous-entrée}
\vedette{\hypertarget{}{\papi{ ʑɣɤsɯmto}}}\markboth{ʑɣɤsɯmto}{}\classe{vi}
\paradigme{\textit{dir :} \jya pɯ-}
\begin{définition}\ 
\begin{déclaration}\grammar{refl}\end{déclaration}
\begin{déclaration}\grammar{caus}\end{déclaration}\end{définition}
\begin{définition}\fra se faire voir\end{définition}
\begin{définition}\cmn 让别人看到自己\end{définition}
\begin{exemple}\jya ma-pɯ-tɯ-ʑɣɤsɯmto\cmn 你不要让人看见你\end{exemple}
\begin{exemple}\jya ɯʑo mɯ-to-rɯndzaŋspa tɕe pjɤ-ʑɣɤsɯmto\cmn 他不小心让人看见了\end{exemple}
\end{sous-entrée}\end{entrée}

\begin{entrée}
\vedette{\hypertarget{Ⓔmtsaʁ}{\papi{ mtsaʁ}}}\markboth{mtsaʁ}{}
\classe{vi}
\paradigme{\textit{dir :} \jya \_}
\begin{définition}\fra sauter\end{définition}
\begin{définition}\cmn 跳\end{définition}
\begin{exemple}\jya @lanqiu mɯ́j-mtsaʁ, ɯ-@qi mɯ́j-rtaʁ\cmn 篮球跳不了,气不够\end{exemple}
\begin{exemple}\jya aʑo pjɯ-kɯ-ɣɤrat-a-nɯ mɤ-ra ma aʑo pjɯ-nɯ-mtsaʁ-a jɤɣ\cmn 你们不需要把我扔下去,我自己跳\end{exemple}
\begin{exemple}\jya ɯ-rtsa ɲɯ-mtsaʁ, ɯ-sni ɲɯ-mtsaʁ\cmn 他的脉搏在跳动\end{exemple}
\begin{relation-sémantique}\synonyme{
\hyperlink{Ⓔnɯmdar}{\textit{ \papi{nɯmdar}}}
}\end{relation-sémantique}
\begin{relation-sémantique}\confer{
\hyperlink{Ⓔnɯmbrɯmtsaʁ}{\textit{ \papi{nɯmbrɯmtsaʁ}}}
}\end{relation-sémantique}\end{entrée}

\begin{entrée}
\vedette{\hypertarget{Ⓔmtshalu}{\papi{ mtshalu}}}\markboth{mtshalu}{}\classe{n}
\begin{définition}\fra ortie\end{définition}
\begin{définition}\cmn 荨麻【和麻】\end{définition}
\begin{exemple}\jya mtshalu kɯ kɤ́-wɣ-mtsɯɣ-a\cmn 被荨麻蛰到了\end{exemple}
\begin{exemple}\jya mtshalu nɯ sɯjno ci ŋu, kha ɯ-rkɯ kɯ-ɤrmbat tsa zndɤrchɤβ ɯ-ŋgɯ nɯ ra dɤn, mtshalu ɯ-ru ɯ-taʁ ɯ-jwaʁ ɯ-taʁ nɯ ra ɯ-rme dɤn tɕeri ɯ-rme sɤmtsɯɣ, ɯ-rme tɯ-ɕa ɲɯ-ɤtɯɣ tɕe ɕɯmŋɤm tɤ-ndɤr ʑo tu-tɕɤt ɕti, ɯ-mat tu-βze ɕɯŋgɯ kɤ-ndza sna. tɯ-xpa tɯ-xpa tu-ɬoʁ ŋu.\cmn 和麻是一种植物,生长在房子周边的墙缝里,和麻的茎和叶子上都长有细毛但这些细毛会蛰人,细毛碰到皮肤时,导致疼痛,皮肤上生痘痘,结果之前可以吃,每年都会发芽。\end{exemple}
\begin{relation-sémantique}\confer{
\hyperlink{Ⓔmtshalɤɲaʁ}{\textit{ \papi{mtshalɤɲaʁ}}}
}\end{relation-sémantique}
\begin{relation-sémantique}\confer{
\hyperlink{Ⓔmtshalɤɣrum}{\textit{ \papi{mtshalɤɣrum}}}
}\end{relation-sémantique}
\begin{relation-sémantique}\confer{
\hyperlink{Ⓔqarmamtshalu}{\textit{ \papi{qarmamtshalu}}}
}\end{relation-sémantique}
\begin{relation-sémantique}\confer{
\hyperlink{Ⓔnɯmtshalu}{\textit{ \papi{nɯmtshalu}}}
}\end{relation-sémantique}\end{entrée}

\begin{entrée}
\vedette{\hypertarget{Ⓔmtshalɤɣrum}{\papi{ mtshalɤɣrum}}}\markboth{mtshalɤɣrum}{}\classe{n}
\begin{définition}\fra espèce d'ortie\end{définition}
\begin{définition}\cmn 荨麻的一种\end{définition}\end{entrée}

\begin{entrée}
\vedette{\hypertarget{Ⓔmtshalɤɲaʁ}{\papi{ mtshalɤɲaʁ}}}\markboth{mtshalɤɲaʁ}{}\classe{n}
\begin{définition}\fra ortie\end{définition}
\begin{définition}\cmn 荨麻的一种\end{définition}\end{entrée}

\begin{entrée}
\vedette{\hypertarget{Ⓔmtshɤm}{\papi{ mtshɤm}}}\markboth{mtshɤm}{}\classe{vt}
\paradigme{\textit{dir :} \jya pɯ-}\acception{1}
\begin{définition}\fra entendre\end{définition}
\begin{définition}\cmn 听见\end{définition}
\begin{exemple}\jya ɯ-pɯ́-tɯ-mtshɤm\cmn 你听见了吗?\end{exemple}\acception{2}
\begin{définition}\fra sentir\end{définition}
\begin{définition}\cmn 闻到\end{définition}
\begin{exemple}\jya tɤ-khɯ ɯ-di pjɯ-tɯ-mtshɤm tɕe tú-wɣ-sɤɕqhe-a ŋu\cmn 我一闻到烟味都会咳嗽\end{exemple}
\begin{exemple}\jya ɯ-di ɲɯ-mtsham-a\cmn 我闻到它的气味了\end{exemple}\acception{3}
\begin{définition}\fra ressentir\end{définition}
\begin{définition}\cmn 感觉到\end{définition}
\begin{exemple}\jya tɤ-mpja ɲɯ-mtsham-a\cmn 我感觉到热\end{exemple}
\begin{exemple}\jya tɤ-ŋɤm ɲɯ-mtsham-a\cmn 我感觉到痛\end{exemple}
\begin{exemple}\jya tɤkhe tshɤfkri mɤ-mtshɤm\cmn 傻子感觉不到汤里加了很多盐\end{exemple}
\begin{exemple}\jya tɤkhe tshɤdɯɣ mɤ-mtshɤm\cmn 傻子感觉不到热\end{exemple}
\begin{relation-sémantique}\confer{
\hyperlink{Ⓔsɤmtshɤm}{\textit{ \papi{sɤmtshɤm}}}
}\end{relation-sémantique}
\begin{relation-sémantique}\confer{
\hyperlink{Ⓔamɯmtshɤm}{\textit{ \papi{amɯmtshɤm}}}
}\end{relation-sémantique}\end{entrée}

\begin{entrée}
\vedette{\hypertarget{Ⓔmtshɤmdi}{\papi{ mtshɤmdi}}}\markboth{mtshɤmdi}{}\classe{n}
\begin{définition}\ 
\begin{déclaration}\grammar{n.lieu}\end{déclaration}\end{définition}
\begin{définition}\fra nom commun à plusieurs champs à Kamnyu\end{définition}
\begin{définition}\cmn 干木鸟村几块田地的统称\end{définition}\end{entrée}

\begin{entrée}
\vedette{\hypertarget{Ⓔmtshɤmŋu}{\papi{ mtshɤmŋu}}}\markboth{mtshɤmŋu}{}\classe{n}
\begin{définition}\fra plage\end{définition}
\begin{définition}\cmn 海边\end{définition}\end{entrée}

\begin{entrée}
\vedette{\hypertarget{Ⓔmtshɤri}{\papi{ mtshɤri}}}\markboth{mtshɤri}{}\classe{adv}
\begin{définition}\fra étrange\end{définition}
\begin{définition}\cmn 奇怪
\begin{déclaration} \étymologie{\papi{mtshar}}\end{déclaration}\end{définition}
\begin{exemple}\jya mtshɤri ɲɯ-sɯsam-a ma a-zda ra mɯ́j-nɤndʐo-nɯ ri aʑo ɲɯ-nɤndʐo-a\cmn 我觉得奇怪,其他人不冷,我却觉得很冷\end{exemple}
\begin{exemple}\jya nɯ tɕhi pɯ-ŋu kɯma mɯ́j-sɯʁjit-a tɕe mtshɤri ɲɯ-sɯsam-a\cmn 很奇怪,我根本就想不起到底是什么回事\end{exemple}
\begin{relation-sémantique}\confer{
\hyperlink{Ⓔsɤmtshɤr}{\textit{ \papi{sɤmtshɤr}}}
}\end{relation-sémantique}\end{entrée}

\begin{entrée}
\vedette{\hypertarget{Ⓔmtshɤt}{\papi{ mtshɤt}}}\markboth{mtshɤt}{}
\classe{vs}
\paradigme{\textit{dir :} \jya tɤ-}
\begin{définition}\fra rempli\end{définition}
\begin{définition}\cmn 满(再也不能装了)\end{définition}
\begin{exemple}\jya tɯ-ci to-mtshɤt\cmn 满了水\end{exemple}
\begin{exemple}\jya lʁa to-mtshɤt\cmn 口袋满了\end{exemple}
\begin{exemple}\jya khɯtsa to-mtshɤt\cmn 碗满了\end{exemple}
\begin{exemple}\jya zɯm to-mtshɤt\cmn 桶满了\end{exemple}
\begin{relation-sémantique}\synonyme{
\hyperlink{ⒺfkaⒽ1}{\textit{ \papi{fka1}}}
}\end{relation-sémantique}
\begin{relation-sémantique}\confer{
\hyperlink{Ⓔsɯmtshɤt}{\textit{ \papi{sɯmtshɤt}}}
}\end{relation-sémantique}\end{entrée}

\begin{entrée}
\vedette{\hypertarget{Ⓔmtshi}{\papi{ mtshi}}}\markboth{mtshi}{}
\classe{vt}
\paradigme{\textit{dir :} \jya tɤ-}
\paradigme{\textit{dir :} \jya \_}
\begin{définition}\fra conduire\end{définition}
\begin{définition}\cmn 带路;牵\end{définition}
\begin{exemple}\jya jla tɤ-mtshi-t-a\cmn 我牵了犏牛\end{exemple}
\begin{exemple}\jya mbro tɤ-mtshi-t-a\cmn 我牵了马\end{exemple}
\begin{exemple}\jya tɕhaʁla zɯ tɯmbri ci kɤ-mtshi-t-a (kɤ-lat-a)\cmn 我在院子里拉了一根绳子(晒衣服)\end{exemple}\begin{sous-entrée}
\vedette{\hypertarget{}{\papi{ sɤmtshi}}}\markboth{sɤmtshi}{}\classe{vi}
\paradigme{\textit{dir :} \jya tɤ-}
\begin{définition}\fra mener la marche\end{définition}
\begin{définition}\cmn 领头\end{définition}
\begin{exemple}\jya aʑo ju-sɤmtshi-a jɤɣ\cmn 我可以带路\end{exemple}
\end{sous-entrée}\end{entrée}

\begin{entrée}
\vedette{\hypertarget{Ⓔmtshukha}{\papi{ mtshukha}}}\markboth{mtshukha}{}\classe{n}
\begin{définition}\fra bord du lac\end{définition}
\begin{définition}\cmn 海岸;湖边\end{définition}\end{entrée}

\begin{entrée}
\vedette{\hypertarget{Ⓔmtshoŋ}{\papi{ mtshoŋ}}}\markboth{mtshoŋ}{}\classe{vs}
\paradigme{\textit{dir :} \jya tɤ-}
\begin{définition}\fra être complet\end{définition}
\begin{définition}\cmn 齐备;齐全
\begin{déclaration} \étymologie{\papi{mtsʰuŋs}}\end{déclaration}\end{définition}
\begin{exemple}\jya laχtɕha ɯ-kɯ-χtɯ jɤ-ari-a ri, kɯ-mtshoŋ ʑo ɣɤʑu\cmn 我去买东西,需要的东西全部都有\end{exemple}
\begin{exemple}\jya @xinhua @shudian ɯ-ŋgɯ zɯ laχtɕha kɤ-χtɯ kɯ-mtshoŋ ʑo ɣɤʑu\cmn 在新华书店里,需要的东西全部都有\end{exemple}
\begin{exemple}\jya ɯ-ftɕaka ɲɯ-mtshoŋ\cmn 都准备齐全\end{exemple}
\begin{exemple}\jya a-ftɕaka tɤ-mtshoŋ\cmn 我什么都准备齐全了\end{exemple}
\begin{exemple}\jya ɯʑo ɯ-tɕha ɲɯ-mtshoŋ, mɤ-kɯ-rtaʁ maŋe\cmn 他条件齐全,没有什么欠缺的\end{exemple}\begin{sous-entrée}
\vedette{\hypertarget{}{\papi{ sɯmtshoŋ}}}\markboth{sɯmtshoŋ}{}\classe{vt}
\paradigme{\textit{dir :} \jya tɤ-}
\begin{définition}\fra préparer complètement\end{définition}
\begin{définition}\cmn 准备齐全\end{définition}
\begin{exemple}\jya a-laχtɕha tɤ-sɯmtshoŋ-a\cmn 我把需要的东西都准备好了\end{exemple}
\end{sous-entrée}\end{entrée}

\begin{entrée}
\vedette{\hypertarget{Ⓔmtshoʁlaŋ}{\papi{ mtshoʁlaŋ}}}\markboth{mtshoʁlaŋ}{}\classe{n}
\begin{définition}\fra animal mythique vivant dans la mer (hippopotame)\end{définition}
\begin{définition}\cmn 海象(河马)
\begin{déclaration} \étymologie{\papi{mtsʰo.glaŋ}}\end{déclaration}\end{définition}
\end{entrée}

\begin{entrée}
\vedette{\hypertarget{Ⓔmtshoʁzaŋ}{\papi{ mtshoʁzaŋ}}}\markboth{mtshoʁzaŋ}{}\classe{n}
\begin{définition}\fra la plus grosses des casseroles en cuivre\end{définition}
\begin{définition}\cmn 最大的红铜锅子
\begin{déclaration} \étymologie{\papi{mtsʰog.zaŋs}}\end{déclaration}\end{définition}
\begin{exemple}\jya mtshoʁzaŋ nɯ tɯthɯ ɯ-ŋgɯz stu ʑo kɯ-wxti ŋu, tɯ-ci kɯɕnɯz zɯm tɕhɯt tu-kɯ-ti pjɤ-ŋu tɕe, tɯ-zɯm nɯ kɯtʂɤsqi kɯɕnɤsqi tɯ-rpa kɯ-tɕhɯt tu. ɯ-spa nɯ zaŋ ŋu, ɯ-mŋu kɯ-jɯ-jom ŋu, ɯ-mŋu tɕhi kɯ-jom nɯ ɯ-phoŋbu kɯnɤ jpum, ɯ-mthɤɣ ri ɯ-xtɕɤr kɯ-fse ci ku-ɕe ŋu tɕe nɯ ɯ-mthɯxtɕɤr tu-ti-nɯ ŋgrɤl. ɯ-mthɯxtɕɤr cho ɯ-mŋu ɯ-pɤrthɤβ nɯ ra thɯci tsuku ʑo ɯ-χpi ku-oʑɯrja ŋu, qaɟy ɯ-χpi, pɣa ɯ-χpi, lɤntsa, ʁjaŋtʂoŋ, rɟanaʁ tɕaʁri, kɯɕnom nɯ ra tu. mtshoʁzaŋ nɯ kɯɕɯŋgɯ rgɯnba kɯ-wxti tɯrme kɯ-dɤn ɣɯ nɯ-tʂha sɤ-ta, tɯ-tshi ɯ-sɤ-βzu nɯ ra pjɤ-ŋu.\cmn 
\stylefv{mtshoʁzaŋ}是锅里面最大的一种,据说能装七桶水,每一桶有六七十斤重。是用红铜铸成的。口很宽,口有多宽锅身就有多粗,在锅身中间箍着一根薄铜条,人们说是锅子的腰带。腰带和口之间排列着各种各样的图案,有鱼形花纹、鸟形花纹以及各种佛教图纹。过去,在大寺庙和人多的时候用来熬茶,煮粥。
\end{exemple}\end{entrée}

\begin{entrée}
\vedette{\hypertarget{Ⓔmtshɯβ}{\papi{ mtshɯβ}}}\markboth{mtshɯβ}{}
\classe{vi}
\paradigme{\textit{dir :} \jya nɯ-}
\begin{définition}\fra se noyer\end{définition}
\begin{définition}\cmn 溺死\end{définition}
\begin{exemple}\jya ɲɤ-mtshɯβ\cmn 他溺死了\end{exemple}\begin{sous-entrée}
\vedette{\hypertarget{}{\papi{ sɯmtshɯβ}}}\markboth{sɯmtshɯβ}{}\classe{vt}
\paradigme{\textit{dir :} \jya nɯ-}
\begin{définition}\fra noyer\end{définition}
\begin{définition}\cmn 令…溺死\end{définition}
\begin{exemple}\jya tɯ-ci kɯ ɲɤ́-wɣ-sɯmtshɯβ tɕe pjɤ-si\cmn 他被水溺死了\end{exemple}
\end{sous-entrée}\end{entrée}

\begin{entrée}
\vedette{\hypertarget{Ⓔmtshɯntɕha}{\papi{ mtshɯntɕha}}}\markboth{mtshɯntɕha}{}
\classe{n}
\begin{définition}\fra arme\end{définition}
\begin{définition}\cmn 武器
\begin{déclaration} \étymologie{\papi{mtsʰon.tɕʰa}}\end{déclaration}\end{définition}\end{entrée}

\begin{entrée}
\vedette{\hypertarget{Ⓔmtshɯzwɤr}{\papi{ mtshɯzwɤr}}}\markboth{mtshɯzwɤr}{}\classe{n}
\begin{définition}\fra lac\end{définition}
\begin{définition}\cmn 湖\end{définition}
\end{entrée}

\begin{entrée}
\vedette{\hypertarget{Ⓔmtsɯɣ}{\papi{ mtsɯɣ}}}\markboth{mtsɯɣ}{}
\classe{vt}
\paradigme{\textit{dir :} \jya kɤ-}
\begin{définition}\fra mordre\end{définition}
\begin{définition}\cmn 咬\end{définition}
\begin{exemple}\jya khɯna kɯ ndʐuwa ka-mtsɯɣ\cmn 狗咬了客人\end{exemple}
\begin{exemple}\jya lɯlu kɯ βʑɯ ka-mtsɯɣ\cmn 猫咬了老鼠\end{exemple}
\begin{exemple}\jya ɣʑo kɯ kɤ́-wɣ-mtsɯɣ-a\cmn 蜜蜂蛰了我\end{exemple}\begin{sous-entrée}
\vedette{\hypertarget{}{\papi{ sɤmtsɯɣ}}}\markboth{sɤmtsɯɣ}{}\classe{vs}
\paradigme{\textit{dir :} \jya tɤ-}
\begin{définition}\ 
\begin{déclaration}\grammar{apass}\end{déclaration}\end{définition}
\begin{définition}\fra mordre les gens\end{définition}
\begin{définition}\cmn 咬人;蜇人\end{définition}
\begin{exemple}\jya qapri ɲɯ-sɤmtsɯɣ\cmn 蛇咬人\end{exemple}
\begin{exemple}\jya βɣɤrtshi ɲɯ-sɤmtsɯɣ\cmn 蚊子咬人\end{exemple}
\begin{exemple}\jya khɯna ɲɯ-sɤmtsɯɣ\cmn 狗咬人\end{exemple}
\end{sous-entrée}\end{entrée}

\begin{entrée}
\vedette{\hypertarget{Ⓔmtsɯr}{\papi{ mtsɯr}}}\markboth{mtsɯr}{}
\classe{vi}\acception{1}
\paradigme{\textit{dir :} \jya nɯ-}
\begin{définition}\fra avoir faim\end{définition}
\begin{définition}\cmn 饿\end{définition}
\begin{exemple}\jya saχsɯ tɤ-ndze ma tɯrmɯ tɯ-mtsɯr\cmn 你吃中午饭,不然下午就会饿\end{exemple}
\begin{exemple}\jya tɤ-nɯsaχsɯ ma tɤ-pɤri ɕɯŋgɯ tɯ-mtsɯr\cmn 你吃中午饭,不然在晚餐之前就会饿)\end{exemple}
\begin{exemple}\jya aʑo ɲɯ-mtsɯr-a\cmn 我很饿\end{exemple}
\begin{exemple}\jya ɯʑo ɲɯ-mtsɯr\cmn 他很饿\end{exemple}
\begin{exemple}\jya nɤʑo ɲɯ-tɯ-mtsɯr ɯ-ŋu\cmn 你饿不饿\end{exemple}
\begin{exemple}\jya kɤ-mtsɯr ɲɯ-sɤɣdɯɣ\cmn 挨饿很辛苦\end{exemple}\acception{2}
\paradigme{\textit{dir :} \jya thɯ-}
\begin{définition}\fra avoir très faim\end{définition}
\begin{définition}\cmn 饿得厉害\end{définition}\begin{sous-entrée}
\vedette{\hypertarget{}{\papi{ ɣɤmtsɯr}}}\markboth{ɣɤmtsɯr}{}
\begin{définition}\ 
\begin{déclaration}\grammar{facil}\end{déclaration}\end{définition}
\begin{définition}\fra avoir faim facilement\end{définition}
\begin{définition}\cmn 容易饿\end{définition}
\begin{relation-sémantique}\confer{
\hyperlink{Ⓔfsɯr}{\textit{ \papi{fsɯr}}}
}\end{relation-sémantique}
\begin{relation-sémantique}\confer{\classe{vs}
\hyperlink{Ⓔsɤmtsɯr}{\textit{ \papi{sɤmtsɯr}}}
}\end{relation-sémantique}
\end{sous-entrée}\begin{sous-entrée}
\vedette{\hypertarget{}{\papi{ sɯmtsɯr}}}\markboth{sɯmtsɯr}{}\classe{vt}
\paradigme{\textit{dir :} \jya nɯ-}
\begin{définition}\fra avoir faim\end{définition}
\begin{définition}\cmn 令……挨饿\end{définition}
\begin{exemple}\jya nɯ-ta-sɯmtsɯr\cmn 我让你饿了(没有及时给你饭吃)\end{exemple}
\end{sous-entrée}\end{entrée}

\begin{entrée}
\vedette{\hypertarget{Ⓔmtsɯrɕpaʁ}{\papi{ mtsɯrɕpaʁ}}}\markboth{mtsɯrɕpaʁ}{}\classe{vi}
\paradigme{\textit{dir :} \jya nɯ-}
\begin{définition}\ 
\begin{déclaration}\grammar{comp}\end{déclaration}\end{définition}
\begin{définition}\fra avoir soif et faim\end{définition}
\begin{définition}\cmn 又饿又渴\end{définition}
\begin{exemple}\jya ɯʑo ʁo mtsɯrɕpaʁ ntsɯ ɕti kɯ\cmn 他一直又饿又渴的样子\end{exemple}\end{entrée}

\begin{entrée}
\vedette{\hypertarget{Ⓔmtʂɤkhoz}{\papi{ mtʂɤkhoz}}}\markboth{mtʂɤkhoz}{}\classe{n}
\begin{définition}\fra bavette\end{définition}
\begin{définition}\cmn 口水巾\end{définition}\end{entrée}

\begin{entrée}
\vedette{\hypertarget{Ⓔmɯβʑi}{\papi{ mɯβʑi}}}\markboth{mɯβʑi}{}\classe{n}
\begin{définition}\fra nakṣatra hasta\end{définition}
\begin{définition}\cmn 轸宿
\begin{déclaration} \étymologie{\papi{me.bʑi}}\end{déclaration}\end{définition}
\end{entrée}

\begin{entrée}
\vedette{\hypertarget{Ⓔmɯɕtaʁ}{\papi{ mɯɕtaʁ}}}\markboth{mɯɕtaʁ}{}
\classe{vs}
\paradigme{\textit{dir :} \jya thɯ-}
\begin{définition}\fra froid\end{définition}
\begin{définition}\cmn 冷\end{définition}
\begin{exemple}\jya tɤjpa ɲɯ-mɯɕtaʁ\cmn 雪很冷\end{exemple}
\begin{exemple}\jya qale ɲɯ-mɯɕtaʁ\cmn 风很冷\end{exemple}
\begin{exemple}\jya rɯŋgu ɲɯ-mɯɕtaʁ\cmn 牧场很冷\end{exemple}
\begin{exemple}\jya ɯ-ku ra pjɤ-mɯɕtaʁ ʑo\cmn 他被吓到了\end{exemple}\begin{sous-entrée}
\vedette{\hypertarget{}{\papi{ zmɯɕtaʁ}}}\markboth{zmɯɕtaʁ}{}\classe{vt}
\begin{définition}\fra rendre froid\end{définition}
\begin{définition}\cmn 使变冷\end{définition}
\end{sous-entrée}\end{entrée}

\begin{entrée}
\vedette{\hypertarget{Ⓔmɯɣ}{\papi{ mɯɣ}}}\markboth{mɯɣ}{}
\begin{relation-sémantique}\confer{
\hyperlink{Ⓔɕmɯɣ}{\textit{ \papi{ɕmɯɣ}}}
}\end{relation-sémantique}\end{entrée}

\begin{entrée}
\vedette{\hypertarget{Ⓔmɯɣmɯɣ}{\papi{ mɯɣmɯɣ}}}\markboth{mɯɣmɯɣ}{}\classe{idph.2}\begin{définition}\fra souriant\end{définition}
\begin{définition}\cmn 形容笑嘻嘻的模样\end{définition}
\begin{exemple}\jya mɯɣmɯɣ ɲɯ-nɤre\cmn 他笑嘻嘻的\end{exemple}\end{entrée}

\begin{entrée}
\vedette{\hypertarget{Ⓔmɯjphɤt}{\papi{ mɯjphɤt}}}\markboth{mɯjphɤt}{}\classe{vi}
\paradigme{\textit{dir :} \jya lɤ-}
\paradigme{\textit{dir :} \jya pɯ-}
\begin{définition}\fra vomir\end{définition}
\begin{définition}\cmn 呕吐\end{définition}
\begin{exemple}\jya @yunche ɲo-βzu tɕe lo-mɯjphɤt\cmn 他晕车就吐了\end{exemple}
\begin{relation-sémantique}\synonyme{
\hyperlink{Ⓔqioʁ}{\textit{ \papi{qioʁ}}}
}\end{relation-sémantique}\end{entrée}

\begin{entrée}
\vedette{\hypertarget{Ⓔmɯjrɯ}{\papi{ mɯjrɯ}}}\markboth{mɯjrɯ}{}\classe{vi}
\begin{définition}\fra bien élevé\end{définition}
\begin{définition}\cmn 受到良好的教育,孝顺\end{définition}
\begin{exemple}\jya ɯ-rɟit ra ɲɯ-mɯjrɯ-nɯ\cmn 他的孩子接受过很好的教育\end{exemple}\begin{sous-entrée}
\vedette{\hypertarget{}{\papi{ zmɯjrɯ}}}\markboth{zmɯjrɯ}{}\classe{vt}
\paradigme{\textit{dir :} \jya nɯ-}
\begin{définition}\fra bien élever\end{définition}
\begin{définition}\cmn 教育好\end{définition}
\begin{exemple}\jya tɤ-pɤtso ɲɯ-tɯ-zmɯjri\cmn 你把孩子教育得很好\end{exemple}
\end{sous-entrée}\end{entrée}

\begin{entrée}
\vedette{\hypertarget{Ⓔmɯm}{\papi{ mɯm}}}\markboth{mɯm}{}\classe{vs}
\paradigme{\textit{dir :} \jya tɤ-}
\begin{définition}\fra bon (goût)\end{définition}
\begin{définition}\cmn 香(味道)、好吃\end{définition}
\begin{exemple}\jya tɤ-mthɯm ɲɯ-mɯm\cmn 肉好吃\end{exemple}
\begin{exemple}\jya kɤ-ndzɤtshi ɲɯ-mɯm\cmn 食物好吃\end{exemple}\begin{sous-entrée}
\vedette{\hypertarget{}{\papi{ ɣɤmɯm}}}\markboth{ɣɤmɯm}{}\classe{vt}
\paradigme{\textit{dir :} \jya tɤ-}
\begin{définition}\ 
\begin{déclaration}\grammar{caus}\end{déclaration}\end{définition}
\begin{définition}\fra rendre bon à manger\end{définition}
\begin{définition}\cmn 使好吃\end{définition}
\begin{exemple}\jya tɯmgo zmɤrɤβ to-ɣɤmɯm\cmn 他把菜弄得很好吃\end{exemple}
\end{sous-entrée}\begin{sous-entrée}
\vedette{\hypertarget{}{\papi{ nɤmɯm}}}\markboth{nɤmɯm}{}\classe{vt}
\paradigme{\textit{dir :} \jya tɤ-}
\begin{définition}\ 
\begin{déclaration}\grammar{trop}\end{déclaration}\end{définition}
\begin{définition}\fra trouver bon à manger\end{définition}
\begin{définition}\cmn 觉得好吃\end{définition}
\begin{exemple}\jya to-nɤmɯm\cmn 他觉得好吃了(原来觉得不好吃)\end{exemple}
\end{sous-entrée}\end{entrée}

\begin{entrée}
\vedette{\hypertarget{Ⓔmɯma}{\papi{ mɯma}}}\markboth{mɯma}{}\classe{postp}
\begin{définition}\fra à part\end{définition}
\begin{définition}\cmn 除了\end{définition}
\begin{exemple}\jya kɤ-ndza mɯma a-jtɯ\cmn 除了食物,什么都能积攒\end{exemple}
\end{entrée}

\begin{entrée}
\vedette{\hypertarget{Ⓔmɯmta}{\papi{ mɯmta}}}\markboth{mɯmta}{}
\classe{vs}
\paradigme{\textit{dir :} \jya thɯ-}
\begin{définition}\fra parler dans son sommeil, noctambule\end{définition}
\begin{définition}\cmn 说梦话,患梦游症\end{définition}
\begin{exemple}\jya kɯ-mɯmta ci ɲɯ-ŋu\cmn 他是一个说梦话的人\end{exemple}
\begin{exemple}\jya cho-mɯmta\cmn 他说梦话了\end{exemple}
\begin{exemple}\jya ɲɯ-tɯ-mɯmta\cmn 你在说梦话\end{exemple}\end{entrée}

\begin{entrée}
\vedette{\hypertarget{Ⓔmɯmtsrɯɣ}{\papi{ mɯmtsrɯɣ}}}\markboth{mɯmtsrɯɣ}{}\classe{vt}
\paradigme{\textit{dir :} \jya tɤ-}
\paradigme{\textit{dir :} \jya pɯ-}
\begin{définition}\fra aspirer à la paille\end{définition}
\begin{définition}\cmn 吸吮(用吸管)\end{définition}
\begin{exemple}\jya ɯʑo kɯ chɤmdɤru pa-mɯmtsrɯɣ\cmn 他吸了杂酒\end{exemple}\end{entrée}

\begin{entrée}
\vedette{\hypertarget{Ⓔmɯndʐamɯχtɕɯɣ}{\papi{ mɯndʐamɯχtɕɯɣ}}}\markboth{mɯndʐamɯχtɕɯɣ}{}\classe{n}
\begin{définition}\fra toutes sortes\end{définition}
\begin{définition}\cmn 各种各样
\begin{déclaration}\use{古语}\end{déclaration}
\begin{déclaration} \étymologie{\papi{mi.ⁿdra.mi.gtɕig}}\end{déclaration}\end{définition}\end{entrée}

\begin{entrée}
\vedette{\hypertarget{Ⓔmɯnmu}{\papi{ mɯnmu}}}\markboth{mɯnmu}{}\classe{vi}
\paradigme{\textit{dir :} \jya tɤ-}
\paradigme{\textit{dir :} \jya nɯ-}
\begin{définition}\fra bouger\end{définition}
\begin{définition}\cmn 动\end{définition}
\begin{exemple}\jya ɯʑo ɲɯ-mɯnmu\cmn 他在动\end{exemple}
\begin{exemple}\jya fsapaʁ ɲɯ-mɯnmu\cmn 牲畜在动\end{exemple}
\begin{relation-sémantique}\confer{
\hyperlink{Ⓔnmu}{\textit{ \papi{nmu}}}
}\end{relation-sémantique}\begin{sous-entrée}
\vedette{\hypertarget{}{\papi{ zmɯnmu}}}\markboth{zmɯnmu}{}\classe{vt}
\paradigme{\textit{dir :} \jya tɤ-}
\begin{définition}\ 
\begin{déclaration}\grammar{caus}\end{déclaration}\end{définition}
\begin{définition}\fra faire bouger\end{définition}
\begin{définition}\cmn 使移动\end{définition}
\begin{exemple}\jya rŋgɯ tɤ-zmɯnmu-t-a\cmn 我把大石包移动了\end{exemple}
\begin{exemple}\jya ɯ-jaʁ kɤ-zmɯnmu ɯ-ɲɯ́-khɯ?\cmn 他能不能移动他的手?(受了伤的人)\end{exemple}
\end{sous-entrée}\end{entrée}

\begin{entrée}
\vedette{\hypertarget{Ⓔmɯntoʁ}{\papi{ mɯntoʁ}}}\markboth{mɯntoʁ}{}\classe{n}
\begin{définition}\fra fleur\end{définition}
\begin{définition}\cmn 花
\begin{déclaration} \étymologie{\papi{me.tog}}\end{déclaration}\end{définition}
\begin{exemple}\jya mɯntoʁ ɯ-spa\cmn 蓓蕾\end{exemple}
\begin{exemple}\jya stɤmku mɯntoʁ ɲɤ-lɤt\cmn 平原上的花开了\end{exemple}
\begin{exemple}\jya thaχtsa ɯ-χcɤl nɯ ɯ-mɯntoʁ rmi\cmn 
花带中间的(图像)叫\stylefv{mɯntoʁ}
\end{exemple}
\begin{relation-sémantique}\confer{
\hyperlink{Ⓔrɯmɯntoʁ}{\textit{ \papi{rɯmɯntoʁ}}}
}\end{relation-sémantique}
\begin{relation-sémantique}\confer{
\hyperlink{Ⓔarɯmɯntoʁ}{\textit{ \papi{arɯmɯntoʁ}}}
}\end{relation-sémantique}\end{entrée}

\begin{entrée}
\vedette{\hypertarget{Ⓔmɯntoʁ sɤrtɕɯn}{\papi{ mɯntoʁ sɤrtɕɯn}}}\markboth{mɯntoʁ sɤrtɕɯn}{}\classe{n}
\begin{définition}\fra fleur jaune\end{définition}
\begin{définition}\cmn 金色的黄花
\begin{déclaration} \étymologie{\papi{me.tog.gser.tɕan}}\end{déclaration}\end{définition}\end{entrée}

\begin{entrée}
\vedette{\hypertarget{Ⓔmɯɴɢɯ}{\papi{ mɯɴɢɯ}}}\markboth{mɯɴɢɯ}{}\classe{n}
\begin{définition}\fra Ligularia fischeria\end{définition}
\begin{définition}\cmn 山紫菀\end{définition}
\begin{exemple}\jya mɯɴɢɯ nɯ sɯjno ci ŋu. tɯ-ci ɯ-rkɯ tu-ɬoʁ rga. ɯ-jwaʁ ɯ-ru tu, ɯ-jwaʁ kɯ-ɤrtɯ-rtɯm kɯ-wxtɯ-wxti ŋu, ɯ-jwaʁ ɯ-qhuʁɤri kɯ-mpɕɯ-mpɕu ŋu, ɯ-χcɤl ɯ-spjɯŋ tu-ɬoʁ tɕe, nɯ ɯ-kɤχcɤl ri ɲɯ-rɯmɯntoʁ. ɯ-mɯntoʁ kɯ-qarŋɯ-rŋe kɯ-mpɕɯ-mpɕɤr ŋu, wuma ʑo nɤmbju, dɤn. ɯ-ru jpum. ɯ-ŋgɯ kɯ-so ŋu, tɤ-jko ɯ-cu kú-wɣ-nɯ-lɤt sna. fsapaʁ ndza pe. tɯ-ci kɯ-me ra tɕe tu-ɬoʁ mɤ-cha.\cmn 山紫菀是一种植物,一般生长在水边。叶子有茎,又大又圆。叶子前后两面都是光滑的。中间的茎生长时,在顶开花。花是大黄色的,很美,有光泽,长得很多。茎很粗,是空心的。可以放进酸菜里吃,也可以喂牲畜。没有水的地方长不出来。\end{exemple}
\end{entrée}

\begin{entrée}
\vedette{\hypertarget{Ⓔmɯrkuj}{\papi{ mɯrkuj}}}\markboth{mɯrkuj}{}\classe{n}
\begin{définition}\fra espèce d'herbe\end{définition}
\begin{définition}\cmn 草的一种\end{définition}
\begin{relation-sémantique}\synonyme{
\hyperlink{Ⓔzgri}{\textit{ \papi{zgri}}}
}\end{relation-sémantique}\end{entrée}

\begin{entrée}
\vedette{\hypertarget{Ⓔmɯrkɯ}{\papi{ mɯrkɯ}}}\markboth{mɯrkɯ}{}\classe{vl}
\paradigme{\textit{dir :} \jya tɤ-}
\begin{définition}\fra voler\end{définition}
\begin{définition}\cmn 偷\end{définition}
\begin{exemple}\jya ɯʑo kɯ ta-mɯrkɯ ŋu\cmn 是他偷的\end{exemple}\begin{sous-entrée}
\vedette{\hypertarget{}{\papi{ kɯmɯrkɯ}}}\markboth{kɯmɯrkɯ}{}\classe{n}
\begin{définition}\fra voleur\end{définition}
\begin{définition}\cmn 小偷\end{définition}
\end{sous-entrée}\end{entrée}

\begin{entrée}
\vedette{\hypertarget{Ⓔmɯrkɯrku}{\papi{ mɯrkɯrku}}}\markboth{mɯrkɯrku}{}\classe{n}
\begin{définition}\fra tous les soirs\end{définition}
\begin{définition}\cmn 每天晚上\end{définition}
\end{entrée}

\begin{entrée}
\vedette{\hypertarget{Ⓔmɯrmɯmbju}{\papi{ mɯrmɯmbju}}}\markboth{mɯrmɯmbju}{}
\classe{n}
\begin{définition}\fra hirondelle\end{définition}
\begin{définition}\cmn 燕子\end{définition}
\begin{exemple}\jya mɯrmɯmbju nɯ pɣa kɯ-xtɕi tsa ci ŋu, ɯ-βri ɲaʁ, ɯ-xtɤpa wɣrum, ɯ-jme artaʁ, tsɯntu fse, ɯ-ku xtɕi, jɤɣɤt ɯ-pa ri tɤ-rcoʁ kɯ kha tu-nɯ-βze ŋu, kɯ-dɯ-dɤn tɯtɯrca ku-rɤʑi-nɯ ŋu. tɯ-mɯ lɤt tɤ-kha tɕɤkɯ-ki ʑo ɲɯ-nɯqambɯmbjom ŋu tɕe, nɯ tɕu tɕe kɤ-mto dɤn ma nɯ mɤɕtʂa kɤ-mto rkɯn.\cmn 燕子是一种比较小的鸟,身子黑,腹部白,尾巴是分叉的,像剪刀一样。在走缘下用稀泥打窝,成群地生活在一起。快要下雨的时候,飞得很低,所以就见的多一些,在其它时候见的少一些。\end{exemple}\end{entrée}

\begin{entrée}
\vedette{\hypertarget{Ⓔmɯrmɯr}{\papi{ mɯrmɯr}}}\markboth{mɯrmɯr}{}\classe{idph.2}
\begin{définition}\fra très fin (poudre)\end{définition}
\begin{définition}\cmn 形容磨得很细的样子\end{définition}
\begin{exemple}\jya tɯsqar ɲɯ-ndɯβ mɯrmɯr ʑo\cmn 糌粑磨得很细\end{exemple}\end{entrée}

\begin{entrée}
\vedette{\hypertarget{Ⓔmɯrnɤmɯr}{\papi{ mɯrnɤmɯr}}}\markboth{mɯrnɤmɯr}{}\classe{idph.2}
\begin{définition}\fra bouchée après bouchée\end{définition}
\begin{définition}\cmn 形容吃东西一口接着一口的样子\end{définition}
\begin{exemple}\jya rtɕhɯʁjɯ nɯ kɯ tɯrtɕhi ɯ-jwaʁ mɯrnɤmɯr tu-ndze ŋu\cmn 毛虫把酸酸草的叶子一口接着一口地吃得很快\end{exemple}\end{entrée}

\begin{entrée}
\vedette{\hypertarget{Ⓔmɯrʁɯz}{\papi{ mɯrʁɯz}}}\markboth{mɯrʁɯz}{}\classe{vt}
\paradigme{\textit{dir :} \jya pɯ-}
\begin{définition}\fra griffer\end{définition}
\begin{définition}\cmn 抓\end{définition}
\begin{exemple}\jya lɯlu kɯ pɯ́-wɣ-mɯrʁɯz-a tɕe a-jaʁ pjɤ-qraʁ\cmn 猫把我抓了一下,抓破了我的手\end{exemple}
\begin{exemple}\jya ma-pɯ-tɯ-mɯrʁɯz ma tɯ-ɕɯmŋɤm\cmn 你别抓他,你令他很痛\end{exemple}\begin{sous-entrée}
\vedette{\hypertarget{}{\papi{ sɤmɯrʁɯz}}}\markboth{sɤmɯrʁɯz}{}\classe{vi}
\begin{définition}\fra griffer les gens\end{définition}
\begin{définition}\cmn 抓人\end{définition}
\begin{exemple}\jya ma-tɯ-sɤmɯrʁɯz ma lɯlu ʑo ɲɯ-tɯ-fse\cmn 你别抓人,你像一只猫!(教育小孩子时)\end{exemple}
\begin{relation-sémantique}\confer{
\hyperlink{Ⓔtɯ-mɯrʁɯz}{\textit{ \papi{tɯ-mɯrʁɯz}}}
}\end{relation-sémantique}
\end{sous-entrée}\end{entrée}

\begin{entrée}
\vedette{\hypertarget{Ⓔmɯrtsɯɣ}{\papi{ mɯrtsɯɣ}}}\markboth{mɯrtsɯɣ}{}
\classe{vt}
\paradigme{\textit{dir :} \jya nɯ-}
\begin{définition}\fra pincer (pour faire mal)\end{définition}
\begin{définition}\cmn 捏(惩罚人的方式)\end{définition}
\begin{exemple}\jya nɯ́-wɣ-mɯrtsɯɣ-a\cmn 他捏了我\end{exemple}
\begin{exemple}\jya tɤ-pɤtso taʁndo mɯ-tɤ-tso tɕe, ɲɯ́-wɣ-mɯrtsɯɣ tɕe phɤn\cmn 小孩子不听话的时候捏一下就会好\end{exemple}\begin{sous-entrée}
\vedette{\hypertarget{}{\papi{ sɤmɯrtsɯɣ}}}\markboth{sɤmɯrtsɯɣ}{}\classe{vi}
\paradigme{\textit{dir :} \jya nɯ-}
\begin{définition}\fra pincer les gens\end{définition}
\begin{définition}\cmn 捏别人\end{définition}
\begin{exemple}\jya ki tɯrme kɤ-sɤmɯrtsɯɣ rga\cmn 这个人喜欢捏别人\end{exemple}
\begin{relation-sémantique}\confer{
\hyperlink{Ⓔtɯmɯrtsɯɣ}{\textit{ \papi{tɯmɯrtsɯɣ}}}
}\end{relation-sémantique}
\end{sous-entrée}\end{entrée}

\begin{entrée}
\vedette{\hypertarget{Ⓔmɯrʑa}{\papi{ mɯrʑa}}}\markboth{mɯrʑa}{}\classe{n}
\begin{définition}\ 
\begin{déclaration}\grammar{n.lieu}\end{déclaration}\end{définition}
\begin{définition}\fra Merja (village de Gdongbrgyad)\end{définition}
\begin{définition}\cmn 木尔渣村\end{définition}
\end{entrée}

\begin{entrée}
\vedette{\hypertarget{Ⓔmɯsta}{\papi{ mɯsta}}}\markboth{mɯsta}{}
\classe{vs}
\begin{définition}\fra ancien\end{définition}
\begin{définition}\cmn 古老\end{définition}
\begin{exemple}\jya wuma kɯ-mɯsta ci ɲɯ-ŋu\cmn 非常古老\end{exemple}\end{entrée}

\begin{entrée}
\vedette{\hypertarget{Ⓔmɯsti}{\papi{ mɯsti}}}\markboth{mɯsti}{}
\classe{vs}
\paradigme{\textit{dir :} \jya tɤ-}
\begin{définition}\fra seul\end{définition}
\begin{définition}\cmn 孤单\end{définition}
\begin{exemple}\jya jiɕqha nɯ ɯʑosti ɲɯ-ŋu, kɯ-mɯsti ci ɲɯ-ŋu\cmn 他一个人,是个孤单的人\end{exemple}
\begin{relation-sémantique}\confer{
\hyperlink{Ⓔɯ-sti}{\textit{ \papi{ɯ-sti}}}
}\end{relation-sémantique}\end{entrée}

\begin{entrée}
\vedette{\hypertarget{Ⓔmɯtɕhɯmɯrɯz}{\papi{ mɯtɕhɯmɯrɯz}}}\markboth{mɯtɕhɯmɯrɯz}{}\classe{adv}
\begin{définition}\fra toute sortes de\end{définition}
\begin{définition}\cmn 各种各样
\begin{déclaration} \étymologie{\papi{mi.tɕʰi.mi.rigs}}\end{déclaration}\end{définition}
\begin{exemple}\jya tɯ-ŋga mɯtɕhɯmɯrɯz ɲɯ-xcat ʑo\cmn 有各种各样的衣服\end{exemple}\end{entrée}

\begin{entrée}
\vedette{\hypertarget{Ⓔmɯxte}{\papi{ mɯxte}}}\markboth{mɯxte}{}
\classe{vs}
\begin{définition}\fra être la majorité\end{définition}
\begin{définition}\cmn 占多数\end{définition}
\begin{exemple}\jya kɯ-mɯxte ɯʑo kɯ ja-nɯtsɯm ɕti\cmn 他把大部分东西带回家了\end{exemple}
\begin{exemple}\jya kɯ-mɯxte aʑo tɤ-nɤma-t-a\cmn 大多数都是我做的\end{exemple}
\begin{exemple}\jya a-tɤ-rʑaʁ kɯ-mɯxte thɯ-arɕo ɕti\cmn 我的时间大部分都过完了\end{exemple}
\begin{exemple}\jya jiʑo kutɕu kɯrɯ tɯrme mɯxte-j\cmn 我们藏族在这里占多数的人口\end{exemple}\end{entrée}

\begin{entrée}
\vedette{\hypertarget{Ⓔmɯχtɤn}{\papi{ mɯχtɤn}}}\markboth{mɯχtɤn}{}\classe{vs}
\paradigme{\textit{dir :} \jya tɤ-}
\begin{définition}\fra dans une situation stable\end{définition}
\begin{définition}\cmn 踏实,稳定
\begin{déclaration} \étymologie{\papi{gtan}}\end{déclaration}\end{définition}
\begin{exemple}\jya aʑɯɣ tɤ-mɯχtɤn\cmn 我的情况稳定了(不可能再有变动)\end{exemple}
\begin{exemple}\jya a-z-rɤʑi tɤ-mɯχtɤn\cmn 我住的地方固定了\end{exemple}\end{entrée}

\begin{entrée}
\vedette{\hypertarget{Ⓔmɯzi}{\papi{ mɯzi}}}\markboth{mɯzi}{}
\classe{n}
\begin{définition}\fra poudre\end{définition}
\begin{définition}\cmn 火药
\begin{déclaration} \étymologie{\papi{mu.zi}}\end{déclaration}\end{définition}\end{entrée}

\newpage\caractère{n}

\begin{entrée}
\vedette{\hypertarget{Ⓔnaŋɕa}{\papi{ naŋɕa}}}\markboth{naŋɕa}{}
\classe{n}
\begin{définition}\fra doublure\end{définition}
\begin{définition}\cmn 夹层衣服的内层部分\end{définition}
\begin{relation-sémantique}\antonyme{
\hyperlink{Ⓔɯ-ʁjoʁ}{\textit{ \papi{ɯ-ʁjoʁ}}}
}\end{relation-sémantique}\end{entrée}

\begin{entrée}
\vedette{\hypertarget{Ⓔnaŋma}{\papi{ naŋma}}}\markboth{naŋma}{}\classe{n}
\begin{définition}\fra partie intérieure des habits tibétains\end{définition}
\begin{définition}\cmn 衣服里子(藏装)
\begin{déclaration} \étymologie{\papi{naŋ.ma}}\end{déclaration}\end{définition}
\end{entrée}

\begin{entrée}
\vedette{\hypertarget{Ⓔnaŋrzoŋ}{\papi{ naŋrzoŋ}}}\markboth{naŋrzoŋ}{}
\classe{n}
\begin{définition}\fra rénovation (habitation)\end{définition}
\begin{définition}\cmn 装修\end{définition}
\begin{exemple}\jya jiʑo naŋrzoŋ ku-osɯ-βzu-j\cmn 我在装修房子\end{exemple}
\begin{relation-sémantique}\confer{
\hyperlink{Ⓔkhɤrlɤn}{\textit{ \papi{khɤrlɤn}}}
}\end{relation-sémantique}\end{entrée}

\begin{entrée}
\vedette{\hypertarget{Ⓔnaŋʁɯ}{\papi{ naŋʁɯ}}}\markboth{naŋʁɯ}{}\classe{n}
\begin{définition}\fra chemise\end{définition}
\begin{définition}\cmn 衬衣\end{définition}\end{entrée}

\begin{entrée}
\vedette{\hypertarget{Ⓔnaŋtɕɯ}{\papi{ naŋtɕɯ}}}\markboth{naŋtɕɯ}{}
\classe{n}
\begin{définition}\fra organes internes\end{définition}
\begin{définition}\cmn 内脏\end{définition}\end{entrée}

\begin{entrée}
\vedette{\hypertarget{Ⓔnaʁa}{\papi{ naʁa}}}\markboth{naʁa}{}
\paradigme{\textit{dir :} \jya pɯ-}
\begin{définition}\fra travailler pour un salaire\end{définition}
\begin{définition}\cmn 打工\end{définition}
\begin{exemple}\jya kɯ-naʁa jo-ɕe\cmn 他去打工了\end{exemple}
\begin{relation-sémantique}\synonyme{
\hyperlink{Ⓔnɯŋgra}{\textit{ \papi{nɯŋgra}}}
}\end{relation-sémantique}
\begin{relation-sémantique}\confer{
\hyperlink{Ⓔta-ʁa}{\textit{ \papi{ta-ʁa}}}
}\end{relation-sémantique}\classe{vi}\end{entrée}

\begin{entrée}
\vedette{\hypertarget{Ⓔnaʁdɤz}{\papi{ naʁdɤz}}}\markboth{naʁdɤz}{}
\classe{vt}
\paradigme{\textit{dir :} \jya tɤ-}
\begin{définition}\fra détester\end{définition}
\begin{définition}\cmn 讨厌;排挤;排斥\end{définition}
\begin{exemple}\jya ɲɯ́-wɣ-naʁdaz-a\cmn 他讨厌我、想排挤我\end{exemple}
\begin{exemple}\jya kɤnɤma mɤ-kɯ-cha nɯra ɲɯ-naʁdɤz\cmn 他排斥不会做工的那些人\end{exemple}
\begin{relation-sémantique}\confer{
\hyperlink{Ⓔɯ-ʁdɤz}{\textit{ \papi{ɯ-ʁdɤz}}}
}\end{relation-sémantique}\begin{sous-entrée}
\vedette{\hypertarget{}{\papi{ sɤnaʁdɤz}}}\markboth{sɤnaʁdɤz}{}\classe{vi}
\paradigme{\textit{dir :} \jya tɤ-}
\begin{définition}\ 
\begin{déclaration}\grammar{apass}\end{déclaration}\end{définition}
\begin{définition}\fra détester les gens\end{définition}
\begin{définition}\cmn 排斥别人\end{définition}
\begin{exemple}\jya ɯʑo ɲɯ-sɤnaʁdɤz, ci nɯ jo-nɯɕe pjɤ-ra\cmn 他排斥别人,有人(因为他)必须离开\end{exemple}
\end{sous-entrée}\end{entrée}

\begin{entrée}
\vedette{\hypertarget{Ⓔnaʁdɯɣ}{\papi{ naʁdɯɣ}}}\markboth{naʁdɯɣ}{}
\classe{vt}
\paradigme{\textit{dir :} \jya tɤ-}
\begin{définition}\ 
\begin{déclaration}\grammar{trop}\end{déclaration}\end{définition}
\begin{définition}\fra chicaner\end{définition}
\begin{définition}\cmn 计较,嫌弃\end{définition}
\begin{exemple}\jya ta-ma tú-wɣ-znɤma jɤɣ ma mɤ-naʁdɯɣ\cmn 可以让他做事,他不会介意的\end{exemple}
\begin{exemple}\jya tɯ́-wɣ-naʁdɯɣ\cmn 他会跟你计较\end{exemple}
\begin{exemple}\jya ɯʑo ɯ-ŋga tú-wɣ-ŋga jɤɣ ma mɤ-naʁdɯɣ\cmn 可以穿他的衣服,他不会介意的\end{exemple}
\begin{exemple}\jya ndzɤtshi mɤ-kɯ-mɯm jarma kɯnɤ, naʁdɯɣ\cmn 菜的味道他都要计较\end{exemple}
\begin{exemple}\jya ki kɯ-xtɕi jamar ʑo mɤ-naʁdɯɣ\cmn 这么小的事情,他不会计较\end{exemple}
\begin{exemple}\jya mɤ-naʁdɯɣ-a\cmn 我不在乎\end{exemple}
\begin{relation-sémantique}\confer{
\hyperlink{ⒺʁdɯɣⒽ1}{\textit{ \papi{ʁdɯɣ1}}}
}\end{relation-sémantique}\end{entrée}

\begin{entrée}
\vedette{\hypertarget{Ⓔnaʁju}{\papi{ naʁju}}}\markboth{naʁju}{}
\classe{vt}
\paradigme{\textit{dir :} \jya nɯ-}
\paradigme{\textit{dir :} \jya thɯ-}
\paradigme{\textit{dir :} \jya tɤ-}
\paradigme{\textit{dir :} \jya kɤ-}
\paradigme{\textit{dir :} \jya lɤ-}
\begin{définition}\fra curer (les dents, un trou etc)\end{définition}
\begin{définition}\cmn 剔(牙齿)、掏(洞等)\end{définition}
\begin{exemple}\jya a-ɕɣa nɯ-naʁju-t-a\cmn 我剔了牙齿\end{exemple}
\begin{exemple}\jya zndɤrchɤβ thɯ-naʁju-t-a\cmn 我掏了缝隙\end{exemple}
\begin{exemple}\jya kɯspoʁ pjɤ-sti tɕe tɤ-naʁju-t-a\cmn 洞塞了,我把它捅了一下\end{exemple}
\begin{exemple}\jya a-rna kɤ-naʁju-t-a\cmn 我掏了耳朵\end{exemple}
\begin{exemple}\jya a-ɕna lɤ-naʁju-t-a\cmn 我抠了鼻子\end{exemple}\end{entrée}

\begin{entrée}
\vedette{\hypertarget{Ⓔnaʁjɯβ}{\papi{ naʁjɯβ}}}\markboth{naʁjɯβ}{}\classe{vt}
\paradigme{\textit{dir :} \jya tɤ-}
\begin{définition}\fra se cacher derrière\end{définition}
\begin{définition}\cmn 躲在……后\end{définition}
\begin{exemple}\jya tu-kɯ-naʁjɯβ-a tɕe a-pɯ-tɯ́-wɣ-mto\cmn 你躲在我身后,不让他们看见你\end{exemple}
\begin{relation-sémantique}\synonyme{
\hyperlink{Ⓔnaʁrɯm}{\textit{ \papi{naʁrɯm}}}
}\end{relation-sémantique}
\begin{relation-sémantique}\confer{
\hyperlink{Ⓔta-ʁjɯβ}{\textit{ \papi{ta-ʁjɯβ}}}
}\end{relation-sémantique}\begin{sous-entrée}
\vedette{\hypertarget{}{\papi{ znaʁjɯβ}}}\markboth{znaʁjɯβ}{}\classe{vt}
\begin{définition}\ 
\begin{déclaration}\grammar{caus}\end{déclaration}\end{définition}
\begin{relation-sémantique}\synonyme{
 \papi{sɤnbaʁ}
}\end{relation-sémantique}
\begin{relation-sémantique}\confer{
\hyperlink{Ⓔsaʁjɯβ}{\textit{ \papi{saʁjɯβ}}}
}\end{relation-sémantique}
\end{sous-entrée}\end{entrée}

\begin{entrée}
\vedette{\hypertarget{Ⓔnaʁlo}{\papi{ naʁlo}}}\markboth{naʁlo}{}\classe{n}
\begin{définition}\fra casserole en fer\end{définition}
\begin{définition}\cmn 生铁锅\end{définition}
\end{entrée}

\begin{entrée}
\vedette{\hypertarget{Ⓔnaʁnɤt}{\papi{ naʁnɤt}}}\markboth{naʁnɤt}{}
\begin{relation-sémantique}\confer{
\hyperlink{Ⓔʁnɤt}{\textit{ \papi{ʁnɤt}}}
}\end{relation-sémantique}\end{entrée}

\begin{entrée}
\vedette{\hypertarget{Ⓔnaʁŋu}{\papi{ naʁŋu}}}\markboth{naʁŋu}{}\classe{n}
\begin{définition}\fra seconde période du mois\end{définition}
\begin{définition}\cmn 下半月
\begin{déclaration} \étymologie{\papi{nag.ŋo}}\end{déclaration}\end{définition}
\end{entrée}

\begin{entrée}
\vedette{\hypertarget{Ⓔnaʁre}{\papi{ naʁre}}}\markboth{naʁre}{}\classe{vt}
\begin{définition}\fra respecter et craindre\end{définition}
\begin{définition}\cmn 敬重,敬畏\end{définition}
\begin{exemple}\jya ɯ-sloχpɯn ɲɯ-naʁre\cmn 他敬重他的老师\end{exemple}
\begin{relation-sémantique}\confer{
\hyperlink{Ⓔsaʁre}{\textit{ \papi{saʁre}}}
}\end{relation-sémantique}
\begin{relation-sémantique}\confer{
 \papi{ɯ-rʁe}
}\end{relation-sémantique}\end{entrée}

\begin{entrée}
\vedette{\hypertarget{Ⓔnaʁri}{\papi{ naʁri}}}\markboth{naʁri}{}\classe{vs}
\paradigme{\textit{dir :} \jya kɤ-}
\begin{définition}\ 
\begin{déclaration}\grammar{denom}\end{déclaration}\end{définition}
\begin{définition}\fra hourdé de graisse\end{définition}
\begin{définition}\cmn 沾满油渍\end{définition}
\begin{exemple}\jya tɯ-ŋga ɲɯ-naʁri tɕe, kɤ-ŋga mɯ́j-sɯsam-a\cmn 衣服上沾满油渍,我不想穿\end{exemple}
\begin{relation-sémantique}\confer{
\hyperlink{Ⓔta-ʁri}{\textit{ \papi{ta-ʁri}}}
}\end{relation-sémantique}\end{entrée}

\begin{entrée}
\vedette{\hypertarget{Ⓔnaʁrɯm}{\papi{ naʁrɯm}}}\markboth{naʁrɯm}{}\classe{vt}
\begin{définition}\fra se cacher en se plaçant derrière\end{définition}
\begin{définition}\cmn 躲在……后面\end{définition}
\begin{exemple}\jya tu-ta-naʁrɯm tɕe a-mɤ-pɯ́-wɣ-mto-a\cmn 我要躲在你后面,不让别人看见我\end{exemple}
\begin{relation-sémantique}\synonyme{
\hyperlink{Ⓔnaʁjɯβ}{\textit{ \papi{naʁjɯβ}}}
}\end{relation-sémantique}
\begin{relation-sémantique}\confer{
\hyperlink{Ⓔsaʁrɯm}{\textit{ \papi{saʁrɯm}}}
}\end{relation-sémantique}\end{entrée}

\begin{entrée}
\vedette{\hypertarget{Ⓔnaʁzi}{\papi{ naʁzi}}}\markboth{naʁzi}{}\classe{vt}
\paradigme{\textit{dir :} \jya tɤ-}
\begin{définition}\ 
\begin{déclaration}\grammar{trop}\end{déclaration}\end{définition}
\begin{définition}\fra avoir besoin de\end{définition}
\begin{définition}\cmn 需要用\end{définition}
\begin{exemple}\jya ɲɯ-naʁzi\cmn 他需要\end{exemple}
\begin{exemple}\jya tɯrme kɯ-dɤn tsa naʁzi-a\cmn 我需要多一点人\end{exemple}
\begin{exemple}\jya a-tɤ́-wɣ-qur-a naʁzi-a\cmn 我需要他帮我\end{exemple}
\begin{exemple}\jya @cai ɕɯ-kɤ-χtɯ naʁzia\cmn 我需要去买菜\end{exemple}
\begin{exemple}\jya to-naʁzi\cmn 他以前不需要,现在需要了\end{exemple}
\begin{exemple}\jya nɤj nɤ-kɤ-naʁzi ɯ-tu ?\cmn 有没有什么需要的?\end{exemple}
\begin{exemple}\jya khɯtsa ɯ-tɯ-naʁzi ?\cmn 你需不需要碗?\end{exemple}
\begin{exemple}\jya kɤ-ʁndɯ ɯ-tɯ-naʁzi ?\cmn 你想挨打是吧?(教育小孩子)\end{exemple}
\begin{exemple}\jya mɤ-ta-naʁzi\cmn 我不需要你\end{exemple}
\begin{exemple}\jya nɤʑo nɤ-kɤ-nɤʁzi ɯ-ɣɤʑu nɤ, aʑo a-ɕki a-ɣɯ-jɤ-tɯ-re\cmn 如果你有需要的话,你就在我这里来拿\end{exemple}
\begin{relation-sémantique}\confer{
\hyperlink{Ⓔʁzi}{\textit{ \papi{ʁzi}}}
}\end{relation-sémantique}\end{entrée}

\begin{entrée}
\vedette{\hypertarget{Ⓔnatɕhɯ}{\papi{ natɕhɯ}}}\markboth{natɕhɯ}{}
\classe{n}
\begin{définition}\fra marais\end{définition}
\begin{définition}\cmn 沼泽(草坪上)
\begin{déclaration} \étymologie{\papi{na.tɕʰu}}\end{déclaration}\end{définition}\end{entrée}

\begin{entrée}
\vedette{\hypertarget{Ⓔnaχaʁ}{\papi{ naχaʁ}}}\markboth{naχaʁ}{}
\classe{vt}
\begin{définition}\fra être surpris par\end{définition}
\begin{définition}\cmn 对……感到惊奇\end{définition}
\begin{exemple}\jya aʑo ɲɯ-naχaʁ-a ɕti\cmn 我对这件事感到惊奇\end{exemple}
\begin{relation-sémantique}\synonyme{
\hyperlink{Ⓔnɤmtshɤr}{\textit{ \papi{nɤmtshɤr}}}
}\end{relation-sémantique}
\begin{relation-sémantique}\confer{
\hyperlink{Ⓔsaχaʁ}{\textit{ \papi{saχaʁ}}}
}\end{relation-sémantique}
\end{entrée}

\begin{entrée}
\vedette{\hypertarget{Ⓔnaχɕɯn}{\papi{ naχɕɯn}}}\markboth{naχɕɯn}{}\classe{vt}
\begin{définition}\fra trouver propre\end{définition}
\begin{définition}\cmn 觉得干净\end{définition}
\begin{relation-sémantique}\synonyme{
\hyperlink{Ⓔnaχtso}{\textit{ \papi{naχtso}}}
}\end{relation-sémantique}
\begin{relation-sémantique}\confer{
\hyperlink{Ⓔsaχɕɯn}{\textit{ \papi{saχɕɯn}}}
}\end{relation-sémantique}\end{entrée}

\begin{entrée}
\vedette{\hypertarget{Ⓔnaχkɯ}{\papi{ naχkɯ}}}\markboth{naχkɯ}{}\classe{n}
\begin{définition}\fra thé sans lait\end{définition}
\begin{définition}\cmn 黑茶(不加牛奶)\end{définition}\end{entrée}

\begin{entrée}
\vedette{\hypertarget{Ⓔnaχpjɤt}{\papi{ naχpjɤt}}}\markboth{naχpjɤt}{}
\begin{relation-sémantique}\confer{
\hyperlink{Ⓔχpjɤt}{\textit{ \papi{χpjɤt}}}
}\end{relation-sémantique}\end{entrée}

\begin{entrée}
\vedette{\hypertarget{Ⓔnaχsoz}{\papi{ naχsoz}}}\markboth{naχsoz}{}\classe{vs}
\paradigme{\textit{dir :} \jya tɤ-}
\begin{définition}\fra frais\end{définition}
\begin{définition}\cmn 新鲜
\begin{déclaration} \étymologie{\papi{gsos}}\end{déclaration}\end{définition}
\begin{exemple}\jya tɯβli ɲɯ-naχsoz\cmn 苗子很新鲜(很有活力)\end{exemple}
\begin{exemple}\jya jiɕqha nɯ kɯ-naχsoz ci ɲɯ-ŋu\cmn 他是一个(面貌)很精神的人\end{exemple}\begin{sous-entrée}
\vedette{\hypertarget{}{\papi{ sɤnaχsoz}}}\markboth{sɤnaχsoz}{}\classe{vs}
\begin{définition}\fra vivifiant\end{définition}
\begin{définition}\cmn 使人清醒\end{définition}
\begin{exemple}\jya qale a-pɯ-mɯɕtaʁ tɕe, wuma ʑo ɲɯ-sɤnaχsoz\cmn 风冷的时候就叫人清醒\end{exemple}
\end{sous-entrée}\begin{sous-entrée}
\vedette{\hypertarget{}{\papi{ znaχsoz}}}\markboth{znaχsoz}{}\classe{vt}
\begin{définition}\fra vivifier, remettre d'alpomb, réveiller\end{définition}
\begin{définition}\cmn 使……清醒\end{définition}
\begin{exemple}\jya tʂha kú-wɣ-tshi tɕe, ɲɯ-kɯ-znaχsoz\cmn 喝了茶就觉得清醒\end{exemple}
\end{sous-entrée}\begin{sous-entrée}
\vedette{\hypertarget{}{\papi{ ʑɣɤnaχsoz}}}\markboth{ʑɣɤnaχsoz}{} (\variante{ʑɣɤznaχsoz}) \classe{vi}
\paradigme{\textit{dir :} \jya tɤ-}
\begin{définition}\fra se remettre d'aplomb\end{définition}
\begin{définition}\cmn 使自己提起精神\end{définition}
\begin{exemple}\jya tʂha kɤ-tshi tɕe tɤ-ʑɣɤnaχsoz\cmn 喝点茶,提起精神\end{exemple}
\end{sous-entrée}\end{entrée}

\begin{entrée}
\vedette{\hypertarget{Ⓔnaχtɕɯɣ}{\papi{ naχtɕɯɣ}}}\markboth{naχtɕɯɣ}{}
\classe{vs}
\paradigme{\textit{dir :} \jya tɤ-}
\begin{définition}\fra semblable\end{définition}
\begin{définition}\cmn 一样
\begin{déclaration} \étymologie{\papi{gtɕig}}\end{déclaration}\end{définition}
\begin{exemple}\jya naχtɕɯɣ ɕti\cmn 无所谓,都一样\end{exemple}
\begin{exemple}\jya mɤ-kɯ-naχtɕɯɣ tu thaŋ nɯ-sɯso-t-a\cmn 我想(两种说法的意思)可能不一样\end{exemple}
\begin{exemple}\jya ɯʑo cho tɕi-tshɯɣa naχtɕɯɣ-tɕi\cmn 我跟他长得很像\end{exemple}\begin{sous-entrée}
\vedette{\hypertarget{}{\papi{ znaχtɕɯɣ}}}\markboth{znaχtɕɯɣ}{}\classe{vt}
\paradigme{\textit{dir :} \jya tɤ-}
\begin{définition}\ 
\begin{déclaration}\grammar{caus}\end{déclaration}\end{définition}\acception{1}
\begin{définition}\fra faire pareil, rendre semblable\end{définition}
\begin{définition}\cmn 使一样\end{définition}
\begin{exemple}\jya ndʑi-mbɯlwa ta-znaχtɕɯɣ\cmn 他给了他们俩一样的工资\end{exemple}
\begin{exemple}\jya ɕkat ta-znaχtɕɯɣ\cmn 他把驮子做成一样\end{exemple}\acception{2}
\begin{définition}\fra trouver semblable\end{définition}
\begin{définition}\cmn 觉得一样\end{définition}
\end{sous-entrée}\end{entrée}

\begin{entrée}
\vedette{\hypertarget{Ⓔnaχthɤβ}{\papi{ naχthɤβ}}}\markboth{naχthɤβ}{}\classe{vt}
\paradigme{\textit{dir :} \jya kɤ-}
\begin{définition}\fra en profiter pour\end{définition}
\begin{définition}\cmn 趁机会\end{définition}
\begin{exemple}\jya aʁa tu ʑo kú-wɣ-naχthɤβ ra\cmn 要趁我有空的时候\end{exemple}
\begin{exemple}\jya aʑo mɤ-rɤʑia ʑo ko-naχthɤβ\cmn 他趁我不在的时候\end{exemple}
\begin{exemple}\jya ko-naχthɤβ tɕe kɯ-chi to-ndza\cmn 他趁了这个机会吃了糖\end{exemple}
\begin{exemple}\jya ɯʑo mɯ́j-rɤʑi ʑo kɤ-naχthaβ-a tɕe nɯ-nɯɣe-a ma nɯ maʁ nɤ kɤ-nɯɣi mɯ́j-nɤle\cmn 我趁了他不在的时候回来,不然的话他不让我回来\end{exemple}\end{entrée}

\begin{entrée}
\vedette{\hypertarget{Ⓔnaχti}{\papi{ naχti}}}\markboth{naχti}{}\classe{vt}
\paradigme{\textit{dir :} \jya kɤ-}
\begin{définition}\fra devenir ami\end{définition}
\begin{définition}\cmn 结为伴侣\end{définition}
\begin{exemple}\jya kɯ-mɤku ɯ-rʑaβ nɯ ɲɤ-βde tɕe kɯ-maqhu kɯmaʁ ci ko-naχti\cmn 他跟原来的妻子离了婚,跟另外一个结为伴侣了\end{exemple}
\begin{relation-sémantique}\confer{
\hyperlink{Ⓔtɯ-χti}{\textit{ \papi{tɯ-χti}}}
}\end{relation-sémantique}
\begin{relation-sémantique}\confer{
\hyperlink{Ⓔsaχti}{\textit{ \papi{saχti}}}
}\end{relation-sémantique}\end{entrée}

\begin{entrée}
\vedette{\hypertarget{Ⓔnaχto}{\papi{ naχto}}}\markboth{naχto}{}
\classe{vt}
\paradigme{\textit{dir :} \jya \_}
\begin{définition}\fra regarder fixement\end{définition}
\begin{définition}\cmn 盯\end{définition}
\begin{exemple}\jya kɤ-naχto-t-a\cmn 我盯着他看了\end{exemple}
\begin{sous-entrée}
\vedette{\hypertarget{}{\papi{ anaχtɯχto}}}\markboth{anaχtɯχto}{}\classe{vi}
\begin{définition}\ 
\begin{déclaration}\grammar{recip}\end{déclaration}\end{définition}
\begin{définition}\fra se regarder fixement les uns uns les autres\end{définition}
\begin{définition}\cmn 互相盯着\end{définition}
\end{sous-entrée}\end{entrée}

\begin{entrée}
\vedette{\hypertarget{Ⓔnaχtso}{\papi{ naχtso}}}\markboth{naχtso}{}
\begin{relation-sémantique}\confer{
\hyperlink{Ⓔχtso}{\textit{ \papi{χtso}}}
}\end{relation-sémantique}\end{entrée}

\begin{entrée}
\vedette{\hypertarget{Ⓔnɤboʁboʁ}{\papi{ nɤboʁboʁ}}}\markboth{nɤboʁboʁ}{}\classe{vt}
\paradigme{\textit{dir :} \jya kɤ-}
\begin{définition}\fra s'attrouper autour de\end{définition}
\begin{définition}\cmn 簇拥\end{définition}
\begin{exemple}\jya tɯrme ci a-pɯ-ndʐaβ tɕe kɯmaʁ tɯrme ra ku-nɤboʁboʁ-nɯ tɕe tu-qur-nɯ ŋgrɤl\cmn 当有人摔跤的时候,其他人会拥上来帮他\end{exemple}
\begin{relation-sémantique}\synonyme{
\hyperlink{Ⓔnɤɣɯβɣɯβ}{\textit{ \papi{nɤɣɯβɣɯβ}}}
}\end{relation-sémantique}\end{entrée}

\begin{entrée}
\vedette{\hypertarget{Ⓔnɤβdɤle}{\papi{ nɤβdɤle}}}\markboth{nɤβdɤle}{}
\begin{relation-sémantique}\confer{
\hyperlink{Ⓔβde}{\textit{ \papi{βde}}}
}\end{relation-sémantique}\end{entrée}

\begin{entrée}
\vedette{\hypertarget{Ⓔnɤβdi}{\papi{ nɤβdi}}}\markboth{nɤβdi}{}
\classe{vi}
\paradigme{\textit{dir :} \jya kɤ-}
\begin{définition}\fra sois en bonne santé\end{définition}
\begin{définition}\cmn 祝你平安
\begin{déclaration}\use{只用于命令式}\end{déclaration}\end{définition}
\begin{exemple}\jya aj nɯɕe-a ŋu, kɤ-nɤβdi\cmn 我回去了,祝你平安\end{exemple}
\begin{exemple}\jya aʑo nɯ mɤɕtʂa pɯ-rɤʑi-a, kɤ-nɤβdi\cmn 我坐了这么多,现在要走了,祝你平安\end{exemple}
\begin{exemple}\jya kɤ-nɤβdi je !\cmn 祝你平安(离别的人出发时对留住的人说的)\end{exemple}
\begin{exemple}\jya kɤ-nɤβdi-ndʑi\cmn 祝你们俩平安\end{exemple}\end{entrée}

\begin{entrée}
\vedette{\hypertarget{Ⓔnɤβɟu}{\papi{ nɤβɟu}}}\markboth{nɤβɟu}{}
\classe{vt}
\paradigme{\textit{dir :} \jya pɯ-}
\begin{définition}\fra se servir de ... comme d'un matelas\end{définition}
\begin{définition}\cmn 垫着坐\end{définition}
\begin{exemple}\jya @bandeng pɯ-nɤβɟe\cmn 坐在板凳上吧\end{exemple}
\begin{exemple}\jya tɤ-βɟu pɯ-nɤβɟe\cmn 用垫子垫着坐吧\end{exemple}
\begin{exemple}\jya ɯ-thoʁ ɲɯ-ɤci tɕe, a-ŋga pɯ-nɯ-nɤβɟu-t-a\cmn 因为地上很湿,所以我垫了衣服坐\end{exemple}
\begin{relation-sémantique}\confer{
\hyperlink{Ⓔtɤ-βɟu}{\textit{ \papi{tɤ-βɟu}}}
}\end{relation-sémantique}\end{entrée}

\begin{entrée}
\vedette{\hypertarget{Ⓔnɤβɟɯβɟi}{\papi{ nɤβɟɯβɟi}}}\markboth{nɤβɟɯβɟi}{}
\classe{vt}
\paradigme{\textit{dir :} \jya tɤ-}
\begin{définition}\ 
\begin{déclaration}\grammar{n.orient}\end{déclaration}\end{définition}
\begin{définition}\fra poursuivre dans tous les sens\end{définition}
\begin{définition}\cmn 追来追去\end{définition}
\begin{exemple}\jya aj nɤj tu-ta-nɯβɟɯβji\cmn 我把你追来追去\end{exemple}
\begin{exemple}\jya nɤʑo kɯ aʑo tu-kɯnɤβɟɯβɟi-a\cmn 你把我追来追去\end{exemple}
\begin{exemple}\jya khɯna kɯ tshɤt ta-nɤβɟɯβɟi\cmn 狗把山羊追得到处跑了\end{exemple}
\begin{relation-sémantique}\confer{
\hyperlink{ⒺβɟiⒽ1}{\textit{ \papi{βɟi1}}}
}\end{relation-sémantique}\end{entrée}

\begin{entrée}
\vedette{\hypertarget{Ⓔnɤβrɯβraʁ}{\papi{ nɤβrɯβraʁ}}}\markboth{nɤβrɯβraʁ}{}
\begin{relation-sémantique}\confer{
\hyperlink{Ⓔβraʁ}{\textit{ \papi{βraʁ}}}
}\end{relation-sémantique}\end{entrée}

\begin{entrée}
\vedette{\hypertarget{Ⓔnɤβzɯβzu}{\papi{ nɤβzɯβzu}}}\markboth{nɤβzɯβzu}{}
\begin{relation-sémantique}\confer{
\hyperlink{ⒺβzuⒽ1}{\textit{ \papi{βzu1}}}
}\end{relation-sémantique}\end{entrée}

\begin{entrée}
\vedette{\hypertarget{Ⓔnɤcu}{\papi{ nɤcu}}}\markboth{nɤcu}{}
\begin{relation-sémantique}\confer{
\hyperlink{Ⓔacu}{\textit{ \papi{acu}}}
}\end{relation-sémantique}\end{entrée}

\begin{entrée}
\vedette{\hypertarget{Ⓔnɤcɯpa}{\papi{ nɤcɯpa}}}\markboth{nɤcɯpa}{}\classe{vt}
\paradigme{\textit{dir :} \jya tɤ-}
\begin{définition}\fra fermer et ouvrir\end{définition}
\begin{définition}\cmn 开和关\end{définition}
\begin{relation-sémantique}\confer{
\hyperlink{ⒺcɯⒽ1}{\textit{ \papi{cɯ}}}
}\end{relation-sémantique}
\begin{relation-sémantique}\confer{
\hyperlink{ⒺpaⒽ1}{\textit{ \papi{pa1}}}
}\end{relation-sémantique}\end{entrée}

\begin{entrée}
\vedette{\hypertarget{Ⓔnɤɕu}{\papi{ nɤɕu}}}\markboth{nɤɕu}{}
\classe{vi}
\paradigme{\textit{dir :} \jya kɤ-}
\begin{définition}\fra se protéger du soleil\end{définition}
\begin{définition}\cmn 避暑\end{définition}
\begin{exemple}\jya si ɯ-pa kɤ-nɤɕu-a\cmn 我在树下面避暑了\end{exemple}
\begin{relation-sémantique}\confer{
\hyperlink{Ⓔɣɤɕu}{\textit{ \papi{ɣɤɕu}}}
}\end{relation-sémantique}\end{entrée}

\begin{entrée}
\vedette{\hypertarget{Ⓔnɤɕarlar}{\papi{ nɤɕarlar}}}\markboth{nɤɕarlar}{}\classe{vt}
\paradigme{\textit{dir :} \jya nɯ-}
\begin{définition}\ 
\begin{déclaration}\grammar{n.orient}\end{déclaration}\end{définition}
\begin{définition}\fra chercher partout\end{définition}
\begin{définition}\cmn 到处找\end{définition}
\begin{relation-sémantique}\confer{
\hyperlink{Ⓔɕar}{\textit{ \papi{ɕar}}}
}\end{relation-sémantique}
\begin{relation-sémantique}\synonyme{
\hyperlink{Ⓔnɤɕɯɕar}{\textit{ \papi{nɤɕɯɕar}}}
}\end{relation-sémantique}
\begin{relation-sémantique}\confer{
\hyperlink{Ⓔɕar}{\textit{ \papi{ɕar}}}
}\end{relation-sémantique}\end{entrée}

\begin{entrée}
\vedette{\hypertarget{Ⓔnɤɕejɣi}{\papi{ nɤɕejɣi}}}\markboth{nɤɕejɣi}{} (\variante{nɤɕejɣɯjɣi}) \classe{vi}
\paradigme{\textit{dir :} \jya \_}
\begin{définition}\fra aller et venir\end{définition}
\begin{définition}\cmn 来回;来来往往(次数多)\end{définition}
\begin{exemple}\jya kɯm a-pɯ-ɲɟɯ, khɯɣɲɟɯ a-pɯ-ɲɟɯ tɕe, kha ɯ-ŋgɯ qale nɯ ju-nɤɕejɣi ɲɯ-cha, tɕe qale ɯ-mbe ju-ɕe, qale kɯ-ɕɤɣ ju-ɣi ɲɯ-cha tɕe ɲɯ-sɤscit\cmn 如果门和窗子是开着的,风可以在屋里流动,旧的空气流出,新鲜空气流进来,这样就显得舒服一些\end{exemple}
\begin{relation-sémantique}\confer{
\hyperlink{Ⓔɕe}{\textit{ \papi{ɕe}}}
}\end{relation-sémantique}
\begin{relation-sémantique}\confer{
\hyperlink{Ⓔɣi}{\textit{ \papi{ɣi}}}
}\end{relation-sémantique}\end{entrée}

\begin{entrée}
\vedette{\hypertarget{Ⓔnɤɕkhɯɕkho}{\papi{ nɤɕkhɯɕkho}}}\markboth{nɤɕkhɯɕkho}{}
\classe{vi}
\paradigme{\textit{dir :} \jya nɯ-}
\begin{définition}\ 
\begin{déclaration}\grammar{n.orient}\end{déclaration}\end{définition}
\begin{définition}\fra faire sécher pendant plusieurs jours en retournant régulièrement\end{définition}
\begin{définition}\cmn 晒几天,翻来覆去地晒\end{définition}
\begin{exemple}\jya stoʁ staχpɯ ɲɤ-k-ɤci tɕe, jisŋi na-nɤɕkhɯɕkho\cmn 胡豆和豌豆湿了,所以他今天把它们晒干了\end{exemple}
\begin{exemple}\jya tɯ-ŋga mɯ-tɤ-kɯ-zbaʁ ri, jisŋi na-nɤɕkhɯɕkho\cmn 衣服没有干,所以他今天晒干了\end{exemple}\end{entrée}

\begin{entrée}
\vedette{\hypertarget{Ⓔnɤɕqa}{\papi{ nɤɕqa}}}\markboth{nɤɕqa}{}\classe{vt}
\paradigme{\textit{dir :} \jya nɯ-}
\begin{définition}\fra supporter\end{définition}
\begin{définition}\cmn 忍耐\end{définition}
\begin{exemple}\jya jɯfɕɯr a-xtu tɤ-mŋɤm tɕe nɯ-nɤɕqa-t-a\cmn 昨天我肚子疼起来了,但是我还是忍了\end{exemple}
\begin{exemple}\jya ɯ-sŋɯro kɤ-lɤt ɲɤ-nɤɕqa\cmn 他屏住呼吸了\end{exemple}
\begin{exemple}\jya tɯ-rju kɯ-ŋɤn ɲɯ-ti tɕe, na-nɤɕqa\cmn 他说了很难听的话,但是他还是忍了\end{exemple}\begin{sous-entrée}
\vedette{\hypertarget{}{\papi{ znɤɕqa}}}\markboth{znɤɕqa}{}\classe{vt}
\paradigme{\textit{dir :} \jya nɯ-}
\begin{définition}\fra être capable de supporter\end{définition}
\begin{définition}\cmn 忍得了\end{définition}
\begin{exemple}\jya ɯ-mi ka-ɣle ri, ɲɯ-znɤɕqe\cmn 他崴了脚,但是还是忍住了\end{exemple}
\begin{exemple}\jya mɯ́j-znɤɕqe-a\cmn 我忍不住(我受不了)\end{exemple}
\begin{relation-sémantique}\confer{
\hyperlink{Ⓔsɤɕqa}{\textit{ \papi{sɤɕqa}}}
}\end{relation-sémantique}
\end{sous-entrée}\end{entrée}

\begin{entrée}
\vedette{\hypertarget{Ⓔnɤɕqraʁ}{\papi{ nɤɕqraʁ}}}\markboth{nɤɕqraʁ}{}
\begin{relation-sémantique}\confer{
\hyperlink{Ⓔɕqraʁ}{\textit{ \papi{ɕqraʁ}}}
}\end{relation-sémantique}\end{entrée}

\begin{entrée}
\vedette{\hypertarget{Ⓔnɤɕthɯɕthɯz}{\papi{ nɤɕthɯɕthɯz}}}\markboth{nɤɕthɯɕthɯz}{}
\begin{relation-sémantique}\confer{
\hyperlink{Ⓔɕthɯz}{\textit{ \papi{ɕthɯz}}}
}\end{relation-sémantique}\end{entrée}

\begin{entrée}
\vedette{\hypertarget{Ⓔnɤɕtʂaʁli}{\papi{ nɤɕtʂaʁli}}}\markboth{nɤɕtʂaʁli}{}\classe{vt}
\paradigme{\textit{dir :} \jya nɯ-}
\begin{définition}\fra torturer\end{définition}
\begin{définition}\cmn 折磨\end{définition}\end{entrée}

\begin{entrée}
\vedette{\hypertarget{Ⓔnɤɕɯɕar}{\papi{ nɤɕɯɕar}}}\markboth{nɤɕɯɕar}{}
\classe{vt}
\paradigme{\textit{dir :} \jya nɯ-}
\begin{définition}\ 
\begin{déclaration}\grammar{n.orient}\end{déclaration}\end{définition}
\begin{définition}\fra chercher partout\end{définition}
\begin{définition}\cmn 到处找\end{définition}
\begin{exemple}\jya nɯ-nɤɕɯɕar-a\cmn 我到处找了\end{exemple}
\begin{relation-sémantique}\synonyme{
\hyperlink{Ⓔnɤɕarlar}{\textit{ \papi{nɤɕarlar}}}
}\end{relation-sémantique}
\begin{relation-sémantique}\confer{
\hyperlink{Ⓔɕar}{\textit{ \papi{ɕar}}}
}\end{relation-sémantique}\end{entrée}

\begin{entrée}
\vedette{\hypertarget{Ⓔnɤɕɯɕe}{\papi{ nɤɕɯɕe}}}\markboth{nɤɕɯɕe}{}
\classe{vi}
\paradigme{\textit{dir :} \jya \_}
\begin{définition}\ 
\begin{déclaration}\grammar{n.orient}\end{déclaration}\end{définition}
\begin{définition}\fra aller partout\end{définition}
\begin{définition}\cmn 到处走,出远门\end{définition}
\begin{exemple}\jya a-mɤ-jɤ-nɤɕɯɕe\cmn 不要让它到处走(一般说牲畜)\end{exemple}
\begin{relation-sémantique}\confer{
\hyperlink{Ⓔɕe}{\textit{ \papi{ɕe}}}
}\end{relation-sémantique}\end{entrée}

\begin{entrée}
\vedette{\hypertarget{Ⓔnɤɕɯɕi}{\papi{ nɤɕɯɕi}}}\markboth{nɤɕɯɕi}{}
\classe{vt}
\paradigme{\textit{dir :} \jya \_}
\begin{définition}\fra traîner par terre\end{définition}
\begin{définition}\cmn 在地上拖\end{définition}
\begin{exemple}\jya ɕoŋtɕa kɤ-fkur mɯ́j-sɤcha tɕe thɯ-nɤɕɯɕi-t-a ɕti\cmn 木料背不动,所以我拖了\end{exemple}
\begin{relation-sémantique}\synonyme{
\hyperlink{Ⓔnɤkhɯkhrɯt}{\textit{ \papi{nɤkhɯkhrɯt}}}
}\end{relation-sémantique}
\end{entrée}

\begin{entrée}
\vedette{\hypertarget{Ⓔnɤdɤn}{\papi{ nɤdɤn}}}\markboth{nɤdɤn}{}
\begin{relation-sémantique}\confer{
\hyperlink{Ⓔdɤn}{\textit{ \papi{dɤn}}}
}\end{relation-sémantique}\end{entrée}

\begin{entrée}
\vedette{\hypertarget{Ⓔnɤfcaʁ}{\papi{ nɤfcaʁ}}}\markboth{nɤfcaʁ}{}
\classe{vt}
\paradigme{\textit{dir :} \jya tɤ-}
\begin{définition}\fra se servir (d'un tissu= pour protéger son dos lorsque l'on porte des charges sur le dos\end{définition}
\begin{définition}\cmn 当作背垫\end{définition}
\begin{exemple}\jya mboʁ tɤ-nɤfcaʁ-a\cmn 我把正方形布料当作背垫了\end{exemple}
\begin{relation-sémantique}\confer{
\hyperlink{Ⓔtɯfcaʁ}{\textit{ \papi{tɯfcaʁ}}}
}\end{relation-sémantique}\end{entrée}

\begin{entrée}
\vedette{\hypertarget{Ⓔnɤfɕɯfɕɤt}{\papi{ nɤfɕɯfɕɤt}}}\markboth{nɤfɕɯfɕɤt}{}
\begin{relation-sémantique}\confer{
\hyperlink{ⒺfɕɤtⒽ1}{\textit{ \papi{fɕɤt1}}}
}\end{relation-sémantique}\end{entrée}

\begin{entrée}
\vedette{\hypertarget{Ⓔnɤfkɯfkur}{\papi{ nɤfkɯfkur}}}\markboth{nɤfkɯfkur}{}
\begin{relation-sémantique}\confer{
\hyperlink{Ⓔfkur}{\textit{ \papi{fkur}}}
}\end{relation-sémantique}\end{entrée}

\begin{entrée}
\vedette{\hypertarget{Ⓔnɤfse}{\papi{ nɤfse}}}\markboth{nɤfse}{}
\classe{vt}
\paradigme{\textit{dir :} \jya pɯ-}
\begin{définition}\ 
\begin{déclaration}\grammar{trop}\end{déclaration}\end{définition}
\begin{définition}\fra trouver semblable\end{définition}
\begin{définition}\cmn 觉得相似;觉得好像\end{définition}
\begin{exemple}\jya jiɕqha nɯ ɲɯ-nɤfse-a\cmn 我觉得这个人很像他\end{exemple}
\begin{exemple}\jya jiɕqha nɯnɯ xiangbolin ŋu thaŋ ri, ɲɯ-nɤfse-a ri, ŋu maʁ mɤxsi\cmn 我觉得这个人很像向柏霖,但不清楚是不是他\end{exemple}
\begin{exemple}\jya nɤʑo ndɤre a-mu ɲɯ-ta-nɤfse\cmn 我觉得你很像我的母亲\end{exemple}
\begin{exemple}\jya jɯfɕɯr pɯ-kɯ-fse nɯ ra a-jmŋo ʑo ɲɯ-nɤfse-a\cmn 昨天发生的事情我觉得像一场梦一样\end{exemple}
\begin{relation-sémantique}\confer{
\hyperlink{ⒺfseⒽ1}{\textit{ \papi{fse1}}}
}\end{relation-sémantique}\end{entrée}

\begin{entrée}
\vedette{\hypertarget{Ⓔnɤfsɯfse}{\papi{ nɤfsɯfse}}}\markboth{nɤfsɯfse}{}
\classe{vi}
\paradigme{\textit{dir :} \jya tɤ-}\acception{1}
\begin{définition}\fra être présomptueux\end{définition}
\begin{définition}\cmn 自以为是;装模作样\end{définition}
\begin{exemple}\jya nɤ-kɤ-nɤfsɯfse ra mɤ-ra, nɤ-stu tɤ-fse\cmn 你不要装模作样,要规矩一点\end{exemple}
\begin{exemple}\jya nɯ ɕɯŋgɯ kɯ-fse mɯ-to-nɤfsɯfse tɕe tham tɕe ɲɯ-pe\cmn 他没有以前那样自以为是了,现在好了\end{exemple}\acception{2}
\begin{définition}\fra faire n'importe quoi\end{définition}
\begin{définition}\cmn 乱来\end{définition}
\begin{exemple}\jya ma-tɯ-nɤfsɯfse\cmn 你不要乱来\end{exemple}\end{entrée}

\begin{entrée}
\vedette{\hypertarget{Ⓔnɤfsɯr}{\papi{ nɤfsɯr}}}\markboth{nɤfsɯr}{}
\classe{vt}
\paradigme{\textit{dir :} \jya tɤ-}
\begin{définition}\ 
\begin{déclaration}\grammar{denom}\end{déclaration}\end{définition}
\begin{définition}\fra se servir comme cible\end{définition}
\begin{définition}\cmn 当靶子
\begin{déclaration}\use{必须是固定的东西}\end{déclaration}\end{définition}
\begin{exemple}\jya tɤ-nɤfsɯr-a\cmn 我把它当靶子了\end{exemple}
\begin{exemple}\jya rdɤstaʁ tɤ-lat-a tɕe tɤ-nɤfsɯr-a\cmn 我扔了一块石头,当做靶子\end{exemple}
\begin{exemple}\jya tɤfsɯr ʑ-lɤ-ta-t-a tɕe tɤ-nɤfsɯr-a\cmn 我把靶子放在那里,当做靶子\end{exemple}
\begin{exemple}\jya ʁmaʁmi ra kɯ χɕɤlphoŋ ɲɯ-ɤz-nɤfsɯr-nɯ\cmn 士兵们把瓶子当靶子(练习打枪)\end{exemple}
\begin{relation-sémantique}\confer{
\hyperlink{Ⓔtɤfsɯr}{\textit{ \papi{tɤfsɯr}}}
}\end{relation-sémantique}\end{entrée}

\begin{entrée}
\vedette{\hypertarget{Ⓔnɤgɯr}{\papi{ nɤgɯr}}}\markboth{nɤgɯr}{}\classe{vt}\acception{1}
\begin{définition}\fra faire la plus grande partie de\end{définition}
\begin{définition}\cmn 大多数都是……\end{définition}
\begin{exemple}\jya tɯ-ɣli kɤ-nɯzʁe nɤj pɯ-tɯ-nɤgɯr\cmn 肥料大多数都是你背的\end{exemple}\acception{2}
\begin{définition}\fra accepter de son plein gré (des critiques)\end{définition}
\begin{définition}\cmn 心服口服地接受批评\end{définition}
\begin{exemple}\jya mɯ́j-nɤgɯr-a\cmn 我不服气\end{exemple}
\end{entrée}

\begin{entrée}
\vedette{\hypertarget{Ⓔnɤɣa}{\papi{ nɤɣa}}}\markboth{nɤɣa}{}\classe{vi}\acception{1}
\begin{définition}\fra ne pas avoir de vertige\end{définition}
\begin{définition}\cmn 不畏高\end{définition}
\begin{exemple}\jya praʁ ɯ-taʁ ɲɯ-nɤɣa\cmn 他在悬崖上不畏高\end{exemple}
\begin{exemple}\jya kɯ-mbro kɤ-ɕe ɲɯ-cha, ɲɯ-nɤɣa\cmn 他可以去很高的地方,他不畏高\end{exemple}
\begin{exemple}\jya ndzom ɯ-taʁ ɲɯ-nɤɣa\cmn 他在桥上不畏高\end{exemple}
\begin{exemple}\jya mɯ́j-nɤɣa tɕe, kɯ-mbro kɤ-ɕe mɯ́j-nɤz\cmn 他畏高,不敢去到高地\end{exemple}
\begin{relation-sémantique}\antonyme{
\hyperlink{Ⓔnɤjmbɣom}{\textit{ \papi{nɤjmbɣom}}}
}\end{relation-sémantique}\acception{2}
\begin{définition}\fra être complètement visible\end{définition}
\begin{définition}\cmn 完全看得到;明显\end{définition}
\begin{exemple}\jya tɤɣal ɲɯ-rɤʑi tɕe ɲɯ-nɤɣa\cmn 完全看得到他\end{exemple}
\begin{relation-sémantique}\synonyme{
\hyperlink{Ⓔsɤmto}{\textit{ \papi{sɤmto}}}
}\end{relation-sémantique}\end{entrée}

\begin{entrée}
\vedette{\hypertarget{Ⓔnɤɣɤɕe}{\papi{ nɤɣɤɕe}}}\markboth{nɤɣɤɕe}{}
\begin{relation-sémantique}\confer{
\hyperlink{Ⓔɣɤɕe}{\textit{ \papi{ɣɤɕe}}}
}\end{relation-sémantique}\end{entrée}

\begin{entrée}
\vedette{\hypertarget{Ⓔnɤɣɤzri}{\papi{ nɤɣɤzri}}}\markboth{nɤɣɤzri}{}
\begin{relation-sémantique}\confer{
\hyperlink{Ⓔzri}{\textit{ \papi{zri}}}
}\end{relation-sémantique}\end{entrée}

\begin{entrée}
\vedette{\hypertarget{Ⓔnɤɣɟaj}{\papi{ nɤɣɟaj}}}\markboth{nɤɣɟaj}{}\classe{vt}
\paradigme{\textit{dir :} \jya tɤ-}
\begin{définition}\fra forcer, soulever avec un levier\end{définition}
\begin{définition}\cmn 撬开\end{définition}
\begin{exemple}\jya rŋgɯ tɤ-nɤɣɟaj-a\cmn 我把石包撬开了\end{exemple}
\begin{exemple}\jya kɯm kɤ-cɯ mɯ́j-khɯ tɕe tɤ-nɤɣɟaj-i\cmn 门打不开了,我们就把它撬开了\end{exemple}
\begin{relation-sémantique}\confer{
\hyperlink{ⒺtɤɣɟajⒽ1}{\textit{ \papi{tɤɣɟaj}}}
}\end{relation-sémantique}\end{entrée}

\begin{entrée}
\vedette{\hypertarget{Ⓔnɤɣlɤɣle}{\papi{ nɤɣlɤɣle}}}\markboth{nɤɣlɤɣle}{}
\begin{relation-sémantique}\confer{
\hyperlink{Ⓔɣle}{\textit{ \papi{ɣle}}}
}\end{relation-sémantique}
\end{entrée}

\begin{entrée}
\vedette{\hypertarget{Ⓔnɤɣmaʁ}{\papi{ nɤɣmaʁ}}}\markboth{nɤɣmaʁ}{}
\classe{vt}
\paradigme{\textit{dir :} \jya nɯ-}
\begin{définition}\ 
\begin{déclaration}\grammar{trop}\end{déclaration}\end{définition}
\begin{définition}\fra considérer comme injuste, regretter\end{définition}
\begin{définition}\cmn 觉得不对;后悔\end{définition}
\begin{exemple}\jya jiɕqha tɤ-kɯ-nɤmqe-a tɕe nɯ-nɤɣmaʁ-a\cmn 你刚才骂我,我觉得是不对的\end{exemple}
\begin{exemple}\jya tɤ-ta-nɤmqe tɕe nɯ-nɤɣmaʁ-a\cmn 我骂了你,现在后悔了\end{exemple}
\begin{relation-sémantique}\confer{
\hyperlink{ⒺmaʁⒽ1}{\textit{ \papi{maʁ1}}}
}\end{relation-sémantique}\end{entrée}

\begin{entrée}
\vedette{\hypertarget{Ⓔnɤɣmɤr}{\papi{ nɤɣmɤr}}}\markboth{nɤɣmɤr}{}\classe{vt}
\paradigme{\textit{dir :} \jya tɤ-}
\begin{définition}\ 
\begin{déclaration}\grammar{denom}\end{déclaration}\end{définition}
\begin{définition}\fra tenir dans la bouche\end{définition}
\begin{définition}\cmn 含在嘴里\end{définition}
\begin{exemple}\jya aʑo kɯ-chi tɤ-nɤɣmar-a\cmn 我把糖含在嘴里了\end{exemple}
\begin{relation-sémantique}\confer{
\hyperlink{Ⓔtɯ-ɣmɤr}{\textit{ \papi{tɯ-ɣmɤr}}}
}\end{relation-sémantique}\end{entrée}

\begin{entrée}
\vedette{\hypertarget{Ⓔnɤɣmbat}{\papi{ nɤɣmbat}}}\markboth{nɤɣmbat}{}
\classe{vi}\acception{1}
\paradigme{\textit{dir :} \jya pɯ-}
\begin{définition}\ 
\begin{déclaration}\grammar{trop}\end{déclaration}\end{définition}
\begin{définition}\fra finir facilement\end{définition}
\begin{définition}\cmn 很轻松地做完\end{définition}
\begin{exemple}\jya laχtɕha kɤ-nɯzʁe-t-a, pɯ-nɤɣmbat-a\cmn 我搬东西,很轻松地搬完了\end{exemple}
\begin{exemple}\jya jisŋi @kaoshi tɤ-βzu-t-a, pɯ-nɤɣmbat-a\cmn 今天的考试很轻松就做完了\end{exemple}\acception{2}
\paradigme{\textit{dir :} \jya tɤ-}
\begin{définition}\fra être presque fini\end{définition}
\begin{définition}\cmn 快没有了,只剩下一点\end{définition}
\begin{exemple}\jya tɤ-fkɯm ɯ-ŋgɯ kɤ-rku nɯ to-nɤɣmbat\cmn 装在口袋里的东西快没有了\end{exemple}
\begin{relation-sémantique}\confer{
\hyperlink{Ⓔmbat}{\textit{ \papi{mbat}}}
}\end{relation-sémantique}\end{entrée}

\begin{entrée}
\vedette{\hypertarget{Ⓔnɤɣmbɤβ}{\papi{ nɤɣmbɤβ}}}\markboth{nɤɣmbɤβ}{}\classe{vt}
\paradigme{\textit{dir :} \jya pɯ-}
\paradigme{\textit{dir :} \jya tɤ-}
\begin{définition}\fra être disposé à écouter\end{définition}
\begin{définition}\cmn 服从;愿意听\end{définition}
\begin{exemple}\jya aj tu-ti-a nɯ maka mɯ́j-kɯ-nɤɣmbaβ-a\cmn 我说的话你怎么都愿意听\end{exemple}
\begin{exemple}\jya tɤ-ndzɯmbra-t-a tɕe, ɲɯ-nɤɣmbɤβ\cmn 我教育了他,他就听了\end{exemple}\end{entrée}

\begin{entrée}
\vedette{\hypertarget{Ⓔnɤɣro}{\papi{ nɤɣro}}}\markboth{nɤɣro}{}
\begin{relation-sémantique}\confer{
\hyperlink{Ⓔaɣro}{\textit{ \papi{aɣro}}}
}\end{relation-sémantique}\end{entrée}

\begin{entrée}
\vedette{\hypertarget{Ⓔnɤɣrɯ}{\papi{ nɤɣrɯ}}}\markboth{nɤɣrɯ}{}
\classe{vt}
\paradigme{\textit{dir :} \jya tɤ-}
\begin{définition}\fra avoir besoin de\end{définition}
\begin{définition}\cmn 需要\end{définition}
\begin{exemple}\jya tɕhi kɯ-fse tɯ-nɤɣri\cmn 你需要怎么样的东西?\end{exemple}
\begin{exemple}\jya nɤʑo ɲɯ-tɯ-nɤɣri ndʐa aj chɯ-fɕi-a ŋu\cmn 我会根据你需要的东西打铁的\end{exemple}
\begin{exemple}\jya ki ɲɯ-nɤɣri-a, ki mɯ́j-nɤɣri-a\cmn 我需要这个,不需要那个\end{exemple}
\begin{exemple}\jya mɤ-kɤ-nɤɣrɯ nɯ aʑɯɣ mɯ́j-ra\cmn 不需要的东西我不要\end{exemple}\end{entrée}

\begin{entrée}
\vedette{\hypertarget{Ⓔnɤɣɯβɣɯβ}{\papi{ nɤɣɯβɣɯβ}}}\markboth{nɤɣɯβɣɯβ}{}\classe{vt}
\paradigme{\textit{dir :} \jya kɤ-}
\begin{définition}\fra s'attrouper\end{définition}
\begin{définition}\cmn 簇拥,围拢\end{définition}
\begin{exemple}\jya tɤ-mthɯm ci a-pɯ-tu tɕe, khɯna ra kɯ ku-nɤɣɯβɣɯβ-nɯ ʑo ŋu\cmn 当有一块肉在那里的时候,狗就蜂拥而来\end{exemple}
\begin{relation-sémantique}\synonyme{
\hyperlink{Ⓔnɤboʁboʁ}{\textit{ \papi{nɤboʁboʁ}}}
}\end{relation-sémantique}\end{entrée}

\begin{entrée}
\vedette{\hypertarget{Ⓔnɤɣʑa}{\papi{ nɤɣʑa}}}\markboth{nɤɣʑa}{}\classe{vt}
\paradigme{\textit{dir :} \jya tɤ-}
\begin{définition}\fra récolter\end{définition}
\begin{définition}\cmn 拔出来;收割(大麻)\end{définition}
\begin{exemple}\jya tasa tɤ-nɤɣʑa-t-a\cmn 我把大麻收割了(选出不能结种子的大麻)\end{exemple}
\begin{relation-sémantique}\confer{
 \papi{tasɤɣʑa}
}\end{relation-sémantique}\end{entrée}

\begin{entrée}
\vedette{\hypertarget{Ⓔnɤj}{\papi{ nɤj}}}\markboth{nɤj}{}\classe{pro}
\begin{définition}\fra toi\end{définition}
\begin{définition}\cmn 你\end{définition}
\begin{relation-sémantique}\confer{
\hyperlink{Ⓔnɤʑo}{\textit{ \papi{nɤʑo}}}
}\end{relation-sémantique}
\end{entrée}

\begin{entrée}
\vedette{\hypertarget{Ⓔnɤja}{\papi{ nɤja}}}\markboth{nɤja}{}
\classe{vi}
\paradigme{\textit{dir :} \jya pɯ-}
\begin{définition}\fra être dommage\end{définition}
\begin{définition}\cmn 可惜\end{définition}
\begin{exemple}\jya jiɕqha laχtɕha pjɤ-ɴɢrɯ, pɯ-nɤja\cmn 这个东西破了,很可惜\end{exemple}
\begin{relation-sémantique}\confer{
\hyperlink{Ⓔznɤja}{\textit{ \papi{znɤja}}}
}\end{relation-sémantique}\end{entrée}

\begin{entrée}
\vedette{\hypertarget{ⒺnɤjaʁⒽ1}{\papi{ nɤjaʁ}}}\markboth{nɤjaʁ}{}\homonyme{1}
\classe{vt}
\paradigme{\textit{dir :} \jya kɤ-}
\begin{définition}\ 
\begin{déclaration}\grammar{denom}\end{déclaration}\end{définition}
\begin{définition}\fra toucher\end{définition}
\begin{définition}\cmn 抚摸\end{définition}
\begin{exemple}\jya ɯʑo kɯ kɤ́-wɣ-nɤjaʁ-a\cmn 他抚摸了我\end{exemple}
\begin{relation-sémantique}\synonyme{
\hyperlink{Ⓔnɤmɤle}{\textit{ \papi{nɤmɤle}}}
}\end{relation-sémantique}
\begin{relation-sémantique}\synonyme{
\hyperlink{Ⓔnɤmɯma}{\textit{ \papi{nɤmɯma}}}
}\end{relation-sémantique}
\begin{relation-sémantique}\confer{
\hyperlink{Ⓔtɯ-jaʁ}{\textit{ \papi{tɯ-jaʁ}}}
}\end{relation-sémantique}\end{entrée}

\begin{entrée}
\vedette{\hypertarget{ⒺnɤjaʁⒽ2}{\papi{ nɤjaʁ}}}\markboth{nɤjaʁ}{}\homonyme{2}
\classe{vt}
\begin{définition}\ 
\begin{déclaration}\grammar{trop}\end{déclaration}\end{définition}
\begin{définition}\fra trouver trop épais\end{définition}
\begin{définition}\cmn 觉得太厚\end{définition}
\begin{exemple}\jya a-ŋga ɲɯ-nɤjaʁ-a\cmn 我觉得衣服太厚\end{exemple}
\begin{relation-sémantique}\confer{
\hyperlink{Ⓔjaʁ}{\textit{ \papi{jaʁ}}}
}\end{relation-sémantique}\end{entrée}

\begin{entrée}
\vedette{\hypertarget{Ⓔnɤjɤrjɤr}{\papi{ nɤjɤrjɤr}}}\markboth{nɤjɤrjɤr}{}
\begin{relation-sémantique}\confer{
\hyperlink{Ⓔjɤrjɤr}{\textit{ \papi{jɤrjɤr}}}
}\end{relation-sémantique}\end{entrée}

\begin{entrée}
\vedette{\hypertarget{Ⓔnɤjɣɯjɣɤt}{\papi{ nɤjɣɯjɣɤt}}}\markboth{nɤjɣɯjɣɤt}{}
\classe{vi}
\paradigme{\textit{dir :} \jya \_}
\begin{définition}\ 
\begin{déclaration}\grammar{n.orient}\end{déclaration}\end{définition}
\begin{définition}\fra aller et revenir\end{définition}
\begin{définition}\cmn 走了又转回来\end{définition}
\begin{exemple}\jya aki jiɕqha pɯ-ari-a li tɤ-jɣat-a tɕe, pɯ-nɤjɣɯjɣat-a ntsɯ\cmn 我往下去了,又往上转来,我反复去了几次又回来了\end{exemple}
\begin{relation-sémantique}\confer{
\hyperlink{Ⓔjɣɤt}{\textit{ \papi{jɣɤt}}}
}\end{relation-sémantique}\end{entrée}

\begin{entrée}
\vedette{\hypertarget{Ⓔnɤjkɯz}{\papi{ nɤjkɯz}}}\markboth{nɤjkɯz}{}\classe{vt}
\paradigme{\textit{dir :} \jya tɤ-}
\begin{définition}\ 
\begin{déclaration}\grammar{denom}\end{déclaration}\end{définition}
\begin{définition}\fra faire quelque chose en cachette\end{définition}
\begin{définition}\cmn 瞒着\end{définition}
\begin{exemple}\jya laχtɕha tɤ-nɤjkɯz-a tɕe kɤ-ɣɯt-a\cmn 我偷偷地把东西带过来了\end{exemple}
\begin{exemple}\jya aʑo tɤ-kɯ-nɤjkɯz-a tɕe a-laχtɕha jɤ-tɯ-tsɯm\cmn 你瞒着我把我的东西带来了\end{exemple}
\begin{exemple}\jya laχtɕha to-mɯrkɯ tɤ́-wɣ-nɤjkɯz-a\cmn 他偷了东西,没有让我发现\end{exemple}
\begin{exemple}\jya tɯrme tɤ-nɤjkɯz-a\cmn 我瞒着人家做了\end{exemple}\begin{sous-entrée}
\vedette{\hypertarget{}{\papi{ sɤnɤjkɯz}}}\markboth{sɤnɤjkɯz}{}\classe{vi}
\paradigme{\textit{dir :} \jya tɤ-}
\begin{définition}\ 
\begin{déclaration}\grammar{apass}\end{déclaration}\end{définition}
\begin{définition}\fra faire quelque chose en cachette des autres\end{définition}
\begin{définition}\cmn 瞒着别人\end{définition}
\begin{relation-sémantique}\confer{
\hyperlink{Ⓔtɤjkɯz}{\textit{ \papi{tɤjkɯz}}}
}\end{relation-sémantique}
\end{sous-entrée}\end{entrée}

\begin{entrée}
\vedette{\hypertarget{Ⓔnɤjlɤβsqa}{\papi{ nɤjlɤβsqa}}}\markboth{nɤjlɤβsqa}{}\classe{vt}
\paradigme{\textit{dir :} \jya kɤ-}
\begin{définition}\ 
\begin{déclaration}\grammar{incorp}\end{déclaration}\end{définition}
\begin{définition}\fra cuire\end{définition}
\begin{définition}\cmn 炖\end{définition}
\begin{exemple}\jya tɯ-pu kɤ-nɤjlɤβsqa-t-a\cmn 我炖血肠\end{exemple}
\begin{relation-sémantique}\confer{
\hyperlink{Ⓔtɤjlɤβ}{\textit{ \papi{tɤjlɤβ}}}
}\end{relation-sémantique}
\begin{relation-sémantique}\confer{
\hyperlink{Ⓔsqa}{\textit{ \papi{sqa}}}
}\end{relation-sémantique}\end{entrée}

\begin{entrée}
\vedette{\hypertarget{Ⓔnɤjmɤɣ}{\papi{ nɤjmɤɣ}}}\markboth{nɤjmɤɣ}{}
\classe{vi}
\paradigme{\textit{dir :} \jya pɯ-}
\begin{définition}\ 
\begin{déclaration}\grammar{denom}\end{déclaration}\end{définition}
\begin{définition}\fra chercher des champignons\end{définition}
\begin{définition}\cmn 找菌子\end{définition}
\begin{exemple}\jya jisŋi kɯ-nɤjmɤɣ jɤ-ari-a\end{exemple}
\begin{exemple}\jya jisŋi aj pɯ-nɤjmaɣ-a\cmn 我今天去找菌子了\end{exemple}
\begin{exemple}\jya aʑo ɕɯ-nɤjmaɣ-a ŋu, nɤʑo ɯ-tɯ́-ɣi?\cmn 我要去找菌子,你来不来?\end{exemple}
\begin{exemple}\jya ɯʑo pɯ-nɤjmɤɣ, aʑo kɤ-nɤjo-t-a\cmn 他找了菌子,我等了他\end{exemple}
\begin{relation-sémantique}\confer{
\hyperlink{Ⓔtɤjmɤɣ}{\textit{ \papi{tɤjmɤɣ}}}
}\end{relation-sémantique}\end{entrée}

\begin{entrée}
\vedette{\hypertarget{Ⓔnɤjmbɣom}{\papi{ nɤjmbɣom}}}\markboth{nɤjmbɣom}{}
\classe{vs}
\paradigme{\textit{dir :} \jya tɤ-}
\begin{définition}\fra avoir le vertige\end{définition}
\begin{définition}\cmn 畏高\end{définition}
\begin{exemple}\jya praʁku tɕe ɲɯ-nɤjmbɣom\cmn 他在悬崖上畏高\end{exemple}
\begin{relation-sémantique}\antonyme{
\hyperlink{Ⓔnɤɣa}{\textit{ \papi{nɤɣa}}}
}\end{relation-sémantique}\end{entrée}

\begin{entrée}
\vedette{\hypertarget{Ⓔnɤjmbrɯma}{\papi{ nɤjmbrɯma}}}\markboth{nɤjmbrɯma}{}
\classe{n}
\begin{définition}\fra bol\end{définition}
\begin{définition}\cmn 瓷碗,上面有青稞穗子的印
\begin{déclaration} \étymologie{\papi{nas.ⁿbru.ma}}\end{déclaration}\end{définition}\end{entrée}

\begin{entrée}
\vedette{\hypertarget{Ⓔnɤjmŋozdɯɣ}{\papi{ nɤjmŋozdɯɣ}}}\markboth{nɤjmŋozdɯɣ}{}\classe{vi}
\paradigme{\textit{dir :} \jya pɯ-}
\begin{définition}\fra beaucoup rêver, avoir un cauchemard\end{définition}
\begin{définition}\cmn 做很多梦,做噩梦\end{définition}
\begin{exemple}\jya jɯfɕɯɕɤr ɯβrɤ-pɯ-tɯ-nɤjmŋozdɯɣ?\cmn 你昨天没有做噩梦吧\end{exemple}
\begin{exemple}\jya pɯ-nɤjmŋozdɯɣ-a ʑo ti thɯ́-wɣ-sɯsta-a\cmn 他把我从梦中叫醒了\end{exemple}
\begin{relation-sémantique}\confer{
\hyperlink{Ⓔtɤjmŋozdɯɣ}{\textit{ \papi{tɤjmŋozdɯɣ}}}
}\end{relation-sémantique}\end{entrée}

\begin{entrée}
\vedette{\hypertarget{Ⓔnɤjndɤt}{\papi{ nɤjndɤt}}}\markboth{nɤjndɤt}{}\classe{vt}\acception{1}
\begin{définition}\fra trouver mignon, trouver adorable\end{définition}
\begin{définition}\cmn 觉得可爱\end{définition}\acception{2}
\paradigme{\textit{dire :} \jya kɤ-}
\begin{définition}\fra taquiner\end{définition}
\begin{définition}\cmn 逗着玩\end{définition}
\begin{relation-sémantique}\confer{
\hyperlink{Ⓔsɤjndɤt}{\textit{ \papi{sɤjndɤt}}}
}\end{relation-sémantique}
\begin{relation-sémantique}\synonyme{
 \papi{nɤsɤjndɤt}
}\end{relation-sémantique}
\end{entrée}

\begin{entrée}
\vedette{\hypertarget{Ⓔnɤjno}{\papi{ nɤjno}}}\markboth{nɤjno}{}\classe{vi}
\paradigme{\textit{dir :} \jya \_}
\begin{définition}\fra galoper et ruer (cheval)\end{définition}
\begin{définition}\cmn 一边奔跑一边跳(马)\end{définition}
\begin{exemple}\jya mbro jɤ-nɤjno ʑo jɤ-anɯri\cmn 马奔跑了\end{exemple}
\begin{relation-sémantique}\confer{
\hyperlink{Ⓔno}{\textit{ \papi{no}}}
}\end{relation-sémantique}\end{entrée}

\begin{entrée}
\vedette{\hypertarget{Ⓔnɤjo}{\papi{ nɤjo}}}\markboth{nɤjo}{}\classe{vt}
\paradigme{\textit{dir :} \jya pɯ-}
\paradigme{\textit{dir :} \jya kɤ-}\acception{1}
\begin{définition}\fra attendre\end{définition}
\begin{définition}\cmn 等候\end{définition}
\begin{exemple}\jya pɯ-nɤjo-t-a, kɤ-nɤjo-t-a\cmn 我等了他\end{exemple}
\begin{exemple}\jya pɯ-kɯ-nɤjo-a ɯ́-ŋu\cmn 你有没有等我?\end{exemple}
\begin{exemple}\jya tɕe nɯ ɣɯ-nɤjo ɲɯ-ra\cmn 那件事情,需要等\end{exemple}
\begin{exemple}\jya li tɯrme @dianhua kɤ-nɤjo-t-a pɯ-ra\cmn 我又接了人家的电话\end{exemple}
\begin{exemple}\jya pɯ-kɯ-nɤjo-a je!\cmn 等一下我!\end{exemple}\acception{2}
\begin{définition}\fra prendre (de l'eau, des grains etc)\end{définition}
\begin{définition}\cmn 接(液体、颗粒)\end{définition}\acception{3}
\begin{définition}\fra passer par, faire l'expérience de\end{définition}
\begin{définition}\cmn 经历\end{définition}
\begin{exemple}\jya ɯʑo kɯ kɯ-sɤscit ko-nɤjo ri, li kɯ-sɤɣdɯɣ kɯnɤ pjɤ-rɲo\cmn 他经历过快乐的日子,也体验过困难的日子\end{exemple}
\begin{relation-sémantique}\synonyme{
\hyperlink{Ⓔrɲo}{\textit{ \papi{rɲo}}}
}\end{relation-sémantique}\begin{sous-entrée}
\vedette{\hypertarget{}{\papi{ anɤjɯjo}}}\markboth{anɤjɯjo}{}\classe{vi}
\begin{définition}\fra s'attendre l'un l'autre\end{définition}
\begin{définition}\cmn 互相等待\end{définition}
\begin{exemple}\jya kɯ-mbɣom kɯ-dal anɤjɯjo\cmn 急躁的和缓慢的到最后会一起到达\end{exemple}
\end{sous-entrée}\begin{sous-entrée}
\vedette{\hypertarget{}{\papi{ sɤznɤjo}}}\markboth{sɤznɤjo}{} (\variante{sɤnɤjo}) \classe{vi}
\paradigme{\textit{dir :} \jya kɤ-}
\begin{définition}\fra attendre des gens\end{définition}
\begin{définition}\cmn 等别人\end{définition}
\begin{exemple}\jya ma-kɤ-tɯ-sɤznɤjo\cmn 你不要等别人了\end{exemple}
\end{sous-entrée}\begin{sous-entrée}
\vedette{\hypertarget{}{\papi{ znɤjo}}}\markboth{znɤjo}{}
\paradigme{\textit{dir :} \jya tɤ-}\acception{1}
\begin{définition}\fra prendre (de l'eau) avec\end{définition}
\begin{définition}\cmn 用……来接(液体)\end{définition}
\begin{exemple}\jya tɤ-lu pɯ-tɕat-a tɕe, βʑɤzu kɯ tɤ-znɤjo-t-a\cmn 我挤奶的时候用奶桶接住\end{exemple}
\begin{exemple}\jya ki tɯthɯ kɯ tú-wɣ-znɤjo\cmn 用这个锅子接水\end{exemple}\acception{2}
\begin{définition}\fra ne pas pouvoir attendre\end{définition}
\begin{définition}\cmn 等不及(用否定式)\end{définition}
\begin{exemple}\jya aʑo ɲɯ-ɕpaʁ-a, ju-zɣɯt mɯ́j-znɤjam-a tɕe tɯcɯrqɯ kɤ-tshi-t-a\cmn 我很渴,等不及他的到来就喝了冷水(有人会送热水来)\end{exemple}\classe{vt}
\end{sous-entrée}\begin{sous-entrée}
\vedette{\hypertarget{}{\papi{ ʑɣɤznɤjo}}}\markboth{ʑɣɤznɤjo}{}\classe{vi}
\paradigme{\textit{dir :} \jya kɤ-}
\begin{définition}\ 
\begin{déclaration}\grammar{refl}\end{déclaration}
\begin{déclaration}\grammar{caus}\end{déclaration}\end{définition}
\begin{définition}\fra faire attendre les autres\end{définition}
\begin{définition}\cmn 令别人等\end{définition}
\begin{exemple}\jya ma-kɤ-tɯ-ʑɣɤ-z-nɤjo\cmn 你让别人等你\end{exemple}
\end{sous-entrée}\end{entrée}

\begin{entrée}
\vedette{\hypertarget{Ⓔnɤjoʁjoʁ}{\papi{ nɤjoʁjoʁ}}}\markboth{nɤjoʁjoʁ}{}\classe{vt}
\paradigme{\textit{dir :} \jya tɤ-}
\begin{définition}\fra flatter\end{définition}
\begin{définition}\cmn 吹捧;奉承\end{définition}
\begin{relation-sémantique}\confer{
\hyperlink{Ⓔjoʁ}{\textit{ \papi{joʁ}}}
}\end{relation-sémantique}\end{entrée}

\begin{entrée}
\vedette{\hypertarget{Ⓔnɤjtshɯ}{\papi{ nɤjtshɯ}}}\markboth{nɤjtshɯ}{}\classe{vs}\acception{1}
\paradigme{\textit{dir :} \jya kɤ-}
\begin{définition}\fra pouvoir être utilisé\end{définition}
\begin{définition}\cmn 可以用……(不用再找其它的)\end{définition}
\begin{exemple}\jya a-ŋga jɤ-nɯ-ɣɯt-a tɕe, nɯ ma kɤ-χtɯ mɯ́j-ra ma ɲɯ-nɤjtshɯ\cmn 我带了自己的衣服来,不用再买了,穿这件就行了\end{exemple}
\begin{exemple}\jya nɤ-z-rɤrɤt ɯ-thɯ-arɕo nɤ, ki ɯ-taʁ pɯ-rɤt tɕe a-kɤ-nɤjtshɯ tɕe nɯ ma kɤ-χtɯ a-mɤ-pɯ-ra\cmn 如果你纸用完了,写在这个上面来代替,不用再买了\end{exemple}\acception{2}
\paradigme{\textit{dir :} \jya pɯ-}
\begin{définition}\fra être utile\end{définition}
\begin{définition}\cmn 有用\end{définition}
\begin{exemple}\jya nɤj ʑa jɤ-tɯ-ari mɯ-pjɤ-nɤjtshɯ\cmn 我早去没有用\end{exemple}\acception{3}
\begin{définition}\fra être capable de se charger de\end{définition}
\begin{définition}\cmn 可以胜任\end{définition}
\begin{exemple}\jya nɤʑo kha kɤ-nɤma tɯ-nɤjtshɯ\cmn 家里的事情,你可以胜任(就不用我做了)\end{exemple}\end{entrée}

\begin{entrée}
\vedette{\hypertarget{Ⓔnɤjwaʁ}{\papi{ nɤjwaʁ}}}\markboth{nɤjwaʁ}{}\classe{vt}
\paradigme{\textit{dir :} \jya thɯ-}
\begin{définition}\ 
\begin{déclaration}\grammar{denom}\end{déclaration}\end{définition}
\begin{définition}\fra arracher les feuilles\end{définition}
\begin{définition}\cmn 扯叶子\end{définition}
\begin{exemple}\jya ki si ki ɯ-rtaʁ thɯ-nɤjwaʁ-a\cmn 我把这棵树上的树枝扯下了叶子\end{exemple}
\begin{exemple}\jya @wosun thɯ-nɤjwaʁ\cmn 你拔莴笋的叶子吧\end{exemple}
\begin{exemple}\jya ɯʑo kɯ @wosun tha-nɤjwaʁ\cmn 他把莴笋的叶子拔了\end{exemple}
\begin{relation-sémantique}\confer{
\hyperlink{Ⓔtɤ-jwaʁ}{\textit{ \papi{tɤ-jwaʁ}}}
}\end{relation-sémantique}\end{entrée}

\begin{entrée}
\vedette{\hypertarget{Ⓔnɤkɤphɤr}{\papi{ nɤkɤphɤr}}}\markboth{nɤkɤphɤr}{}
\classe{vt}
\paradigme{\textit{dir :} \jya pɯ-}
\paradigme{\textit{dir :} \jya kɤ-}
\begin{définition}\ 
\begin{déclaration}\grammar{incorp}\end{déclaration}\end{définition}
\begin{définition}\fra frapper avec le fléau sur les épis\end{définition}
\begin{définition}\cmn 对着青稞穗打连枷\end{définition}
\begin{exemple}\jya ɯʑo kɯ tɤɕi pa-nɤkɤphɤr\cmn 他用连枷打了青稞穗\end{exemple}
\begin{relation-sémantique}\confer{
\hyperlink{Ⓔsɤphɤr}{\textit{ \papi{sɤphɤr}}}
}\end{relation-sémantique}
\begin{relation-sémantique}\confer{
\hyperlink{Ⓔtɯ-ku}{\textit{ \papi{tɯ-ku}}}
}\end{relation-sémantique}\end{entrée}

\begin{entrée}
\vedette{\hypertarget{Ⓔnɤkɤro}{\papi{ nɤkɤro}}}\markboth{nɤkɤro}{}
\classe{vs}
\paradigme{\textit{dir :} \jya tɤ-}
\begin{définition}\fra acceptable\end{définition}
\begin{définition}\cmn 还可以\end{définition}
\begin{exemple}\jya ɲɯ-nɤkɤro, zdɯxthɯɣ ɲɯ-ŋu\cmn 还可以,勉强可以\end{exemple}
\begin{exemple}\jya mɯ́j-mŋɤm, ɲɯ-nɤkɤro ɲɯ-khɯ\cmn 不痛,还可以\end{exemple}\begin{sous-entrée}
\vedette{\hypertarget{}{\papi{ kɯ-nɤkɤro}}}\markboth{kɯ-nɤkɤro}{}
\begin{définition}\fra un certain temps\end{définition}
\begin{définition}\cmn 好一段时间\end{définition}
\begin{exemple}\jya kɯ-nɤkɤro ʑo pɯ-ta-nɤjo\cmn 我等了你好一阵子\end{exemple}
\end{sous-entrée}\begin{sous-entrée}
\vedette{\hypertarget{}{\papi{ znɤkɤro}}}\markboth{znɤkɤro}{}\classe{vt}
\paradigme{\textit{dir :} \jya tɤ-}
\begin{définition}\ 
\begin{déclaration}\grammar{caus}\end{déclaration}\end{définition}
\begin{définition}\fra trouver acceptable\end{définition}
\begin{définition}\cmn 觉得还可以\end{définition}
\begin{exemple}\jya ɯ-tɯ-rɤt nɯ to-znɤkɤro ɕti\cmn 他觉得他写得还可以\end{exemple}
\end{sous-entrée}\end{entrée}

\begin{entrée}
\vedette{\hypertarget{Ⓔnɤkɤtɕhɯ}{\papi{ nɤkɤtɕhɯ}}}\markboth{nɤkɤtɕhɯ}{}
\classe{vt}
\paradigme{\textit{dir :} \jya kɤ-}
\paradigme{\textit{dir :} \jya tɤ-}
\begin{définition}\ 
\begin{déclaration}\grammar{incorp}\end{déclaration}\end{définition}
\begin{définition}\fra donner un coup de tête\end{définition}
\begin{définition}\cmn 用头顶
\begin{déclaration}\use{主语必须是人,动物顶就用\stylefv{tɕhɯ}}\end{déclaration}\end{définition}
\begin{exemple}\jya kɤ́-wɣ-nɤkɤtɕhɯ-a\cmn 他用头顶了我\end{exemple}
\begin{relation-sémantique}\confer{
\hyperlink{Ⓔtɯ-ku}{\textit{ \papi{tɯ-ku}}}
}\end{relation-sémantique}
\begin{relation-sémantique}\confer{
\hyperlink{Ⓔtɕhɯ}{\textit{ \papi{tɕhɯ}}}
}\end{relation-sémantique}
\begin{relation-sémantique}\confer{
\hyperlink{Ⓔkɤtɕhɯ}{\textit{ \papi{kɤtɕhɯ}}}
}\end{relation-sémantique}\end{entrée}

\begin{entrée}
\vedette{\hypertarget{Ⓔnɤkhu}{\papi{ nɤkhu}}}\markboth{nɤkhu}{}
\classe{vt}
\paradigme{\textit{dir :} \jya nɯ-}
\paradigme{\textit{dir :} \jya kɤ-}
\begin{définition}\ 
\begin{déclaration}\grammar{appl}\end{déclaration}\end{définition}
\begin{définition}\fra inviter\end{définition}
\begin{définition}\cmn 请客
\begin{déclaration}\use{\stylefv{tɯlu}“德鲁”是干木鸟村的一个地名}\end{déclaration}\end{définition}
\begin{exemple}\jya tɯlu ra kɯ aʑo nɯ́-wɣ-nɤkhu-a-nɯ\cmn 德鲁那家人请了我\end{exemple}
\begin{exemple}\jya aʑo tɯlu ra kɤ-nɤkhu-t-a-nɯ\cmn 我请了德鲁那家人\end{exemple}
\begin{exemple}\jya aʑo nɯ-kha ɯ-kɯ-nɤkhu jɤ-ari-a\cmn 我去请他们了\end{exemple}
\begin{exemple}\jya nɯ-kha nɯtɕu kɤ-nɤkhu jɤ-ari-a\cmn 我去他们家坐客\end{exemple}\begin{sous-entrée}
\vedette{\hypertarget{}{\papi{ sɤnɤkhu}}}\markboth{sɤnɤkhu}{}\classe{vi}
\begin{définition}\ 
\begin{déclaration}\grammar{apass}\end{déclaration}\end{définition}
\begin{définition}\fra inviter des gens\end{définition}
\begin{définition}\cmn 请客\end{définition}
\begin{relation-sémantique}\confer{
\hyperlink{Ⓔakhu}{\textit{ \papi{akhu}}}
}\end{relation-sémantique}
\end{sous-entrée}\end{entrée}

\begin{entrée}
\vedette{\hypertarget{Ⓔnɤkhar}{\papi{ nɤkhar}}}\markboth{nɤkhar}{}\classe{vt}
\paradigme{\textit{dir :} \jya pɯ-}
\paradigme{\textit{dir :} \jya tɤ-}
\begin{définition}\fra entourer\end{définition}
\begin{définition}\cmn 包围;围着坐\end{définition}
\begin{exemple}\jya @zhuozi pɯ-nɤkhar-i\cmn 我们把桌子围起来了\end{exemple}
\begin{exemple}\jya @zhuozi pɯ-nɤkhar-i tɕe tɤ-rɯndzɤtshi-j\cmn 我们围着桌子坐了就吃饭了\end{exemple}
\begin{exemple}\jya kɤndza pɯ-nɤkhar-i\cmn 我们围着食物坐了\end{exemple}
\begin{exemple}\jya paʁ kɤ-nɤkhar-i\cmn 我们把猪围住了\end{exemple}\end{entrée}

\begin{entrée}
\vedette{\hypertarget{Ⓔnɤkhɤzŋga}{\papi{ nɤkhɤzŋga}}}\markboth{nɤkhɤzŋga}{}
\classe{vt}
\paradigme{\textit{dir :} \jya nɯ-}
\begin{définition}\ 
\begin{déclaration}\grammar{appl}\end{déclaration}\end{définition}
\begin{définition}\fra appeler\end{définition}
\begin{définition}\cmn 喊;叫\end{définition}
\begin{exemple}\jya nɯ-nɤkhɤzŋga-t-a\cmn 我喊了他一声\end{exemple}
\begin{exemple}\jya aʑo kɯ tɯrme ɲɯ-nɤkhɤzŋge-a\cmn 我在叫人\end{exemple}\begin{sous-entrée}
\vedette{\hypertarget{}{\papi{ anɤkhɤzŋgɯzŋga}}}\markboth{anɤkhɤzŋgɯzŋga}{}\classe{vi}
\begin{définition}\fra s'appeler les uns les autres\end{définition}
\begin{définition}\cmn 互相喊叫\end{définition}
\begin{relation-sémantique}\confer{
\hyperlink{Ⓔakhu}{\textit{ \papi{akhu}}}
}\end{relation-sémantique}
\end{sous-entrée}\end{entrée}

\begin{entrée}
\vedette{\hypertarget{Ⓔnɤkhe}{\papi{ nɤkhe}}}\markboth{nɤkhe}{}\classe{vt}
\paradigme{\textit{dir :} \jya tɤ-}
\paradigme{\textit{dir :} \jya kɤ-}
\begin{définition}\fra maltraiter\end{définition}
\begin{définition}\cmn 欺负\end{définition}
\begin{exemple}\jya jiɕqha kɯ nɯ ɲɯ-ti tú-wɣ-nɤkhe-a\cmn 这个人说了那一句话,把我欺负了\end{exemple}
\begin{exemple}\jya ɯʑo kɯ tú-wɣ-nɤkhe-a ɲɯ-ŋu\cmn 他欺负我\end{exemple}
\begin{exemple}\jya ma-tɤ-kɯ-nɤkhe-a\cmn 你不要欺负我\end{exemple}
\begin{exemple}\jya ko-nɤkhe\cmn 他强奸了她\end{exemple}\begin{sous-entrée}
\vedette{\hypertarget{}{\papi{ sɤnɤkhe}}}\markboth{sɤnɤkhe}{}\classe{vi}
\begin{définition}\fra maltraiter les autres\end{définition}
\begin{définition}\cmn 欺负人\end{définition}
\begin{exemple}\jya jiɕqha tɯrme kɯ-sɤnɤkhe ci ɲɯ-ŋu\cmn 那个人总是欺负别人的\end{exemple}
\end{sous-entrée}\end{entrée}

\begin{entrée}
\vedette{\hypertarget{Ⓔnɤkhɯ}{\papi{ nɤkhɯ}}}\markboth{nɤkhɯ}{}
\classe{vi}
\paradigme{\textit{dir :} \jya tɤ-}
\begin{définition}\ 
\begin{déclaration}\grammar{denom}\end{déclaration}\end{définition}
\begin{définition}\fra être enfumé\end{définition}
\begin{définition}\cmn 被熏到(人)\end{définition}
\begin{exemple}\jya ɲɯ-nɤkhɯ-a\cmn 我被熏到\end{exemple}\begin{sous-entrée}
\vedette{\hypertarget{}{\papi{ znɤkhɯ}}}\markboth{znɤkhɯ}{}\classe{vt}
\paradigme{\textit{dir :} \jya tɤ-}
\begin{définition}\ 
\begin{déclaration}\grammar{caus}\end{déclaration}\end{définition}
\begin{définition}\fra enfumer\end{définition}
\begin{définition}\cmn 熏到\end{définition}
\begin{exemple}\jya smi pjɯ́-wɣ-βlɯ tɕe, tu-kɯ-znɤkhɯ ɲɯ-ŋu\cmn 火烧了,(烟)把人熏到\end{exemple}
\begin{relation-sémantique}\confer{
\hyperlink{Ⓔtɤ-khɯ}{\textit{ \papi{tɤ-khɯ}}}
}\end{relation-sémantique}
\begin{relation-sémantique}\confer{
\hyperlink{ⒺɣɤkhɯⒽ1}{\textit{ \papi{ɣɤkhɯ1}}}
}\end{relation-sémantique}
\begin{relation-sémantique}\confer{
\hyperlink{Ⓔsɤkhɯ}{\textit{ \papi{sɤkhɯ}}}
}\end{relation-sémantique}
\end{sous-entrée}\end{entrée}

\begin{entrée}
\vedette{\hypertarget{Ⓔnɤkhɯkhrɯt}{\papi{ nɤkhɯkhrɯt}}}\markboth{nɤkhɯkhrɯt}{}
\classe{vt}
\paradigme{\textit{dir :} \jya \_}
\begin{définition}\fra traîner\end{définition}
\begin{définition}\cmn 拖
\begin{déclaration}\use{(背不动的东西)}\end{déclaration}
\begin{déclaration}\use{沙尔宗话可以单独说\stylefv{khrɯt},但干木鸟话不能说}\end{déclaration}\end{définition}
\begin{exemple}\jya thɯ-nɤkhɯkhrɯt-a\cmn 我拖了\end{exemple}
\begin{exemple}\jya ɕoŋtɕa tha-nɤkhɯkhrɯt\cmn 他拖了木料\end{exemple}
\begin{exemple}\jya laχtɕha tha-nɤkhɯkhrɯt\cmn 他拖了东西\end{exemple}
\begin{exemple}\jya nɤki nɯ kɤ-fkɯr mɯ-mɤ-ɲɯ-tɯ-cha nɤ, nɯ-nɤkhɯkhrɯt\cmn 如果你不能背,你就拖吧\end{exemple}
\begin{relation-sémantique}\synonyme{
\hyperlink{Ⓔnɤɕɯɕi}{\textit{ \papi{nɤɕɯɕi}}}
}\end{relation-sémantique}\end{entrée}

\begin{entrée}
\vedette{\hypertarget{Ⓔnɤkɯka}{\papi{ nɤkɯka}}}\markboth{nɤkɯka}{}\classe{vt}
\paradigme{\textit{dir :} \jya nɯ-}
\begin{définition}\fra mâcher\end{définition}
\begin{définition}\cmn 咀嚼\end{définition}
\begin{exemple}\jya nɯ-nɤkɯka-t-a\cmn 我咀嚼了\end{exemple}
\begin{exemple}\jya stoʁ nɯ mɤ-kɤ-nɤkɯka kɤ-mqlaʁ mɤ-khɯ\cmn 胡豆要先咀嚼才可以吞\end{exemple}\end{entrée}

\begin{entrée}
\vedette{\hypertarget{Ⓔnɤkɯkro}{\papi{ nɤkɯkro}}}\markboth{nɤkɯkro}{}
\begin{relation-sémantique}\confer{
\hyperlink{Ⓔkro}{\textit{ \papi{kro}}}
}\end{relation-sémantique}
\end{entrée}

\begin{entrée}
\vedette{\hypertarget{Ⓔnɤkɯt}{\papi{ nɤkɯt}}}\markboth{nɤkɯt}{}\classe{vi}
\paradigme{\textit{dir :} \jya \_}
\begin{définition}\fra être acculé\end{définition}
\begin{définition}\cmn 没有退路\end{définition}\begin{sous-entrée}
\vedette{\hypertarget{}{\papi{ znɤkɯt}}}\markboth{znɤkɯt}{}\classe{vt}
\begin{définition}\fra acculer\end{définition}
\begin{définition}\cmn 逼上绝路\end{définition}
\begin{exemple}\jya lɤ-kɯ-znɤkɯt-a\cmn 你把我逼上绝路了\end{exemple}
\end{sous-entrée}\end{entrée}

\begin{entrée}
\vedette{\hypertarget{Ⓔnɤlu}{\papi{ nɤlu}}}\markboth{nɤlu}{}
\classe{vt}
\paradigme{\textit{dir :} \jya pɯ-}
\begin{définition}\ 
\begin{déclaration}\grammar{denom}\end{déclaration}\end{définition}
\begin{définition}\fra traire\end{définition}
\begin{définition}\cmn 挤奶\end{définition}
\begin{exemple}\jya nɯŋa pɯ-tɯ-nɤlu-t\cmn 你挤了(奶牛的)奶\end{exemple}
\begin{exemple}\jya pɯ-nɤle (= tɤ-lu pɯ-tɕɤt)\cmn 你挤奶吧\end{exemple}
\begin{relation-sémantique}\confer{
\hyperlink{Ⓔtɤ-lu}{\textit{ \papi{tɤ-lu}}}
}\end{relation-sémantique}\end{entrée}

\begin{entrée}
\vedette{\hypertarget{Ⓔnɤla}{\papi{ nɤla}}}\markboth{nɤla}{}\classe{vt}
\paradigme{\textit{dir :} \jya tɤ-}
\begin{définition}\fra acquiescer, être d’accord\end{définition}
\begin{définition}\cmn 同意;允许;答应\end{définition}
\begin{exemple}\jya tɤ-nɤla-t-a\cmn 我答应了\end{exemple}
\begin{exemple}\jya aʑo a-kɤ-ti nɯ ta-nɤla\cmn 他答应了我所说的话\end{exemple}
\begin{exemple}\jya aʑo a-kɤ-ti nɯ tɕhindʐa mɤ-tɯ-nɤle\cmn 你怎么不答应我说的话?\end{exemple}
\begin{exemple}\jya mɯ́j-nɤle\cmn 他不答应,他拒绝\end{exemple}
\begin{exemple}\jya mɯ-ɕɯ-tɯ-nɤle nɯ-sɯso-t-a\cmn 我怕你不答应\end{exemple}\end{entrée}

\begin{entrée}
\vedette{\hypertarget{Ⓔnɤldaʁldaʁ}{\papi{ nɤldaʁldaʁ}}}\markboth{nɤldaʁldaʁ}{}\classe{vt}
\paradigme{\textit{dir :} \jya nɯ-}
\begin{définition}\fra accueillir chaleureusement\end{définition}
\begin{définition}\cmn 热情款待\end{définition}\end{entrée}

\begin{entrée}
\vedette{\hypertarget{Ⓔnɤliɤliɤt}{\papi{ nɤliɤliɤt}}}\markboth{nɤliɤliɤt}{}\classe{vt}
\paradigme{\textit{dir :} \jya tɤ-}
\begin{définition}\fra faire la fête à qqn (chien)\end{définition}
\begin{définition}\cmn 摆尾巴、热烈地扑上(狗)\end{définition}
\begin{exemple}\jya khɯna kɯ tɤ́-wɣ-nɤliɤliɤt-a\cmn 狗朝我摇尾巴\end{exemple}\end{entrée}

\begin{entrée}
\vedette{\hypertarget{Ⓔnɤma}{\papi{ nɤma}}}\markboth{nɤma}{}\classe{vt}
\paradigme{\textit{dir :} \jya tɤ-}
\begin{définition}\ 
\begin{déclaration}\grammar{denom}\end{déclaration}\end{définition}
\begin{définition}\fra travailler\end{définition}
\begin{définition}\cmn 干活\end{définition}
\begin{exemple}\jya tɤ-nɤma-t-a\cmn 我干活了\end{exemple}
\begin{exemple}\jya pɤjkhu mɯ-tɤ-nɤma-t-a\cmn 我还没有干活\end{exemple}
\begin{exemple}\jya a-sɯm mɤ-kɯ-ɕe khro ʑo nɤme-a ra\cmn 我要做许多不想做的工作\end{exemple}
\begin{exemple}\jya tɕhi ku-tɯ-nɤme?\cmn 你在做什么?\end{exemple}
\begin{exemple}\jya spɯ-spe-a nɯ tɤ-nɤma-t-a\cmn 我会做的都做了\end{exemple}\begin{sous-entrée}
\vedette{\hypertarget{}{\papi{ anɤma}}}\markboth{anɤma}{}\classe{vi}
\begin{définition}\fra être fait\end{définition}
\begin{définition}\cmn 做好(事情)\end{définition}
\begin{exemple}\jya pɤjkhu mɤ-anɤma\cmn 事情还没有做好\end{exemple}
\end{sous-entrée}\begin{sous-entrée}
\vedette{\hypertarget{}{\papi{ nɯɣɯnɤma}}}\markboth{nɯɣɯnɤma}{}\classe{vs}
\begin{définition}\fra facile à faire\end{définition}
\begin{définition}\cmn 容易办\end{définition}
\begin{relation-sémantique}\confer{
\hyperlink{Ⓔta-ma}{\textit{ \papi{ta-ma}}}
}\end{relation-sémantique}
\begin{relation-sémantique}\confer{
\hyperlink{Ⓔrɤma}{\textit{ \papi{rɤma}}}
}\end{relation-sémantique}
\end{sous-entrée}\begin{sous-entrée}
\vedette{\hypertarget{}{\papi{ znɤma}}}\markboth{znɤma}{}
\begin{définition}\ 
\begin{déclaration}\grammar{habil}\end{déclaration}\end{définition}
\begin{définition}\fra pouvoir faire\end{définition}
\begin{définition}\cmn 能做;做得了\end{définition}
\begin{exemple}\jya aʑo nɯ thamtɕɤt mɯ́j-znɤme-a\cmn 我做不了那么多事情\end{exemple}
\end{sous-entrée}\end{entrée}

\begin{entrée}
\vedette{\hypertarget{Ⓔnɤmar}{\papi{ nɤmar}}}\markboth{nɤmar}{}\classe{vs}
\paradigme{\textit{dir :} \jya tɤ-}
\begin{définition}\fra gras (surface)\end{définition}
\begin{définition}\cmn 油油的\end{définition}
\begin{exemple}\jya ɯ-jaʁ ɲɯ-nɤmar\cmn 他的手很油\end{exemple}
\begin{relation-sémantique}\confer{
\hyperlink{Ⓔta-mar}{\textit{ \papi{ta-mar}}}
}\end{relation-sémantique}\end{entrée}

\begin{entrée}
\vedette{\hypertarget{Ⓔnɤmɤla}{\papi{ nɤmɤla}}}\markboth{nɤmɤla}{}
\begin{relation-sémantique}\confer{
\hyperlink{Ⓔnɤmɤle}{\textit{ \papi{nɤmɤle}}}
}\end{relation-sémantique}\end{entrée}

\begin{entrée}
\vedette{\hypertarget{Ⓔnɤmɤle}{\papi{ nɤmɤle}}}\markboth{nɤmɤle}{} (\variante{nɤmɤla}) 
\classe{vt}\acception{1}
\paradigme{\textit{dir :} \jya pɯ-}
\begin{définition}\fra toucher\end{définition}
\begin{définition}\cmn 摸;弄\end{définition}
\begin{exemple}\jya pɯ-nɤmɤle-t-a\cmn 我摸了\end{exemple}
\begin{exemple}\jya kɯki laχtɕha tɯ-nɤmɤle mɤ-ra\cmn 你不要摸这个东西\end{exemple}
\begin{exemple}\jya ki ɯ-phɯ ɲɯ-wxti tɕe, ma-pɯ-tɯ-ɣɤrɤt ma a-mɤ-pɯ-ɴɢrɯ ma ma-tɯ-nɤmɤle\cmn 这个东西很贵重,你不要让它摔下来,不要让它破了,不要乱摸\end{exemple}\acception{2}
\paradigme{\textit{dir :} \jya tɤ-}
\begin{définition}\fra faire une tâche\end{définition}
\begin{définition}\cmn 做一个工作(完成整个制作过程)\end{définition}
\begin{exemple}\jya ɯʑo kɯ tɤ-lu tu-nɤmɤle ŋgrɤl\cmn 是他平时负责牛奶的工作\end{exemple}
\begin{exemple}\jya ɯʑo kɯ kha tu-nɤmɤle ŋgrɤl\cmn 是她平时做家务\end{exemple}\end{entrée}

\begin{entrée}
\vedette{\hypertarget{Ⓔnɤmbɤβ}{\papi{ nɤmbɤβ}}}\markboth{nɤmbɤβ}{}
\classe{vs}
\paradigme{\textit{dir :} \jya tɤ-}
\begin{définition}\fra se gonfler (corps)\end{définition}
\begin{définition}\cmn 胀(身体)\end{définition}
\begin{exemple}\jya ɯ-xtu nɤmbɤβ\cmn (他喝了这么多水),肚子(可能)会胀的\end{exemple}
\begin{exemple}\jya pjɤ-si tɕe to-nɤmbɤβ xtaŋxtaŋ ʑo\cmn 死了(身体)就浮肿了\end{exemple}
\begin{exemple}\jya nɯɣmbɤβ\end{exemple}
\begin{relation-sémantique}\confer{
\hyperlink{Ⓔtɯ-ɣmbɤβ}{\textit{ \papi{tɯ-ɣmbɤβ}}}
}\end{relation-sémantique}\end{entrée}

\begin{entrée}
\vedette{\hypertarget{Ⓔnɤmbɤndɤr}{\papi{ nɤmbɤndɤr}}}\markboth{nɤmbɤndɤr}{}\classe{vt}
\begin{définition}\fra aider à marcher, s'occuper de (malade, persone saoûle)\end{définition}
\begin{définition}\cmn 搀扶和照顾(病人,喝醉的人)\end{définition}
\begin{exemple}\jya ɯʑo lo-βzi tɕe tɤ-nɤmbɤndɤr-i pɯ-ra\cmn 他醉了,我们只好搀扶他\end{exemple}\end{entrée}

\begin{entrée}
\vedette{\hypertarget{Ⓔnɤmbɣaʁlaʁ}{\papi{ nɤmbɣaʁlaʁ}}}\markboth{nɤmbɣaʁlaʁ}{}
\classe{vi}
\paradigme{\textit{dir :} \jya \_}
\begin{définition}\ 
\begin{déclaration}\grammar{n.orient}\end{déclaration}\end{définition}
\begin{définition}\fra se tourner et se retourner (à droite puis à gauche)\end{définition}
\begin{définition}\cmn (左右)打滚,辗转\end{définition}
\begin{exemple}\jya mbro ɲɯ-nɤmbɣaʁlaʁ\cmn 马在打滚\end{exemple}
\begin{exemple}\jya laʁtɕha ɯ-thoʁ nɯ fse ɲɯ-nɤmbɣaʁlaʁ\cmn 东西在地上乱七八糟\end{exemple}
\begin{exemple}\jya ma-nɯ-tɯ-nɤmbɣaʁlaʁ\cmn 你不要打滚\end{exemple}
\begin{relation-sémantique}\confer{
\hyperlink{Ⓔmbɣaʁ}{\textit{ \papi{mbɣaʁ}}}
}\end{relation-sémantique}\end{entrée}

\begin{entrée}
\vedette{\hypertarget{Ⓔnɤmbju}{\papi{ nɤmbju}}}\markboth{nɤmbju}{}
\classe{vs}
\paradigme{\textit{dir :} \jya tɤ-}\acception{1}
\begin{définition}\fra éblouissant\end{définition}
\begin{définition}\cmn 耀眼\end{définition}
\begin{exemple}\jya tɤŋe ɲɯ-nɤmbju\cmn 太阳很耀眼\end{exemple}\acception{2}
\begin{définition}\fra brillant\end{définition}
\begin{définition}\cmn 闪光,发光\end{définition}
\begin{exemple}\jya tɤtʂu nɤmbju\cmn 灯发光\end{exemple}
\begin{exemple}\jya ɕɤr tɕe sɯtɤpɯz nɤmbju\cmn 晚上朽木会发光\end{exemple}\begin{sous-entrée}
\vedette{\hypertarget{}{\papi{ znɤmbju}}}\markboth{znɤmbju}{}\classe{vt}
\begin{définition}\fra éblouir\end{définition}
\begin{définition}\cmn 令……眼花\end{définition}
\begin{exemple}\jya tɤŋe kɯ a-mɲaʁ ɲɯ-znɤmbju\cmn 太阳令我眼花\end{exemple}
\end{sous-entrée}\end{entrée}

\begin{entrée}
\vedette{\hypertarget{Ⓔnɤmbrɯ}{\papi{ nɤmbrɯ}}}\markboth{nɤmbrɯ}{}\classe{vt}
\paradigme{\textit{dir :} \jya tɤ-}
\begin{définition}\fra s'énerver contre\end{définition}
\begin{définition}\cmn 生……的气\end{définition}
\begin{exemple}\jya ma-tɤ-kɯ-nɤmbrɯ-a\cmn 你不要生我的气\end{exemple}
\begin{relation-sémantique}\confer{
\hyperlink{Ⓔsɤmbrɯ}{\textit{ \papi{sɤmbrɯ}}}
}\end{relation-sémantique}\end{entrée}

\begin{entrée}
\vedette{\hypertarget{Ⓔnɤmbrɯm}{\papi{ nɤmbrɯm}}}\markboth{nɤmbrɯm}{}\classe{vi}
\paradigme{\textit{dir :} \jya tɤ-}
\begin{définition}\fra attraper la variole\end{définition}
\begin{définition}\cmn 得麻子\end{définition}
\begin{exemple}\jya tɯ-mɲɯtsi tɯ-ɣjɤn a-tɤ-kɯ-nɤmbrɯm qhe, tɕe ɯ-qhu mɯ́j-kɯ-nɤmbrɯm tɕe tu-kɯ-lo ɲɯ-ŋu.\cmn 一辈子得麻子就终身不再得这种病,有免疫能力\end{exemple}\end{entrée}

\begin{entrée}
\vedette{\hypertarget{Ⓔnɤmbrɯmtɕɤz}{\papi{ nɤmbrɯmtɕɤz}}}\markboth{nɤmbrɯmtɕɤz}{}
\classe{vs}
\begin{définition}\fra grêlé\end{définition}
\begin{définition}\cmn 麻子
\begin{déclaration} \étymologie{\papi{ⁿbrum}}\end{déclaration}\end{définition}
\begin{exemple}\jya jiɕqha nɯ ɯ-rŋa ɲɯ-nɤmbrɯmtɕɤz\cmn 那个人的满脸是麻子\end{exemple}\end{entrée}

\begin{entrée}
\vedette{\hypertarget{Ⓔnɤmda}{\papi{ nɤmda}}}\markboth{nɤmda}{}
\classe{vt}
\paradigme{\textit{dir :} \jya tɤ-}
\paradigme{\textit{dir :} \jya pɯ-}
\begin{définition}\ 
\begin{déclaration}\grammar{trop}\end{déclaration}\end{définition}
\begin{définition}\fra sentir que le moment est arrivé\end{définition}
\begin{définition}\cmn 觉得时间到了\end{définition}
\begin{exemple}\jya saχsɯ tɤ-nɤmda-t-a\cmn 我觉得中午餐的时间到了(肚子很饿)\end{exemple}
\begin{exemple}\jya saχsɯ jisŋi mɯ-pɯ-nɤmda-t-a\cmn 今天我没有发觉中午餐的时间到了\end{exemple}
\begin{exemple}\jya kɤ-nɯɕe tɤ-nɤmda-t-a\cmn 我觉得回去的时间到了\end{exemple}
\begin{exemple}\jya kɤ-nɯɕe ɲɯ-nɤmde-a\cmn 我现在觉得回去的时间到了\end{exemple}
\begin{exemple}\jya aʑo ɲɯ-ɕe-a ra ri pɤjkhu mɯ́j-nɤmde-a\cmn 我要回去,但是时间还没有到\end{exemple}
\begin{relation-sémantique}\confer{
\hyperlink{Ⓔmda}{\textit{ \papi{mda}}}
}\end{relation-sémantique}\end{entrée}

\begin{entrée}
\vedette{\hypertarget{Ⓔnɤmdɯmdar}{\papi{ nɤmdɯmdar}}}\markboth{nɤmdɯmdar}{}
\begin{relation-sémantique}\confer{
\hyperlink{Ⓔnɯmdar}{\textit{ \papi{nɯmdar}}}
}\end{relation-sémantique}\end{entrée}

\begin{entrée}
\vedette{\hypertarget{Ⓔnɤmdzɯ}{\papi{ nɤmdzɯ}}}\markboth{nɤmdzɯ}{}
\classe{vt}
\paradigme{\textit{dir :} \jya kɤ-}
\begin{définition}\ 
\begin{déclaration}\grammar{appl}\end{déclaration}\end{définition}
\begin{définition}\fra garder\end{définition}
\begin{définition}\cmn 看管;看守\end{définition}
\begin{exemple}\jya ki smi ki kɤ-nɤmdzi\cmn 请你注意这个火\end{exemple}
\begin{exemple}\jya ki tɤ-pɤtso kɤ-nɤmdzi a-mɤ-pɯ-ndʐaβ\cmn 你看着这个孩子,不要让他摔跤\end{exemple}
\begin{relation-sémantique}\confer{
\hyperlink{Ⓔamdzɯ}{\textit{ \papi{amdzɯ}}}
}\end{relation-sémantique}
\begin{relation-sémantique}\synonyme{
\hyperlink{Ⓔrɯru}{\textit{ \papi{rɯru}}}
}\end{relation-sémantique}\end{entrée}

\begin{entrée}
\vedette{\hypertarget{Ⓔnɤme}{\papi{ nɤme}}}\markboth{nɤme}{}\classe{vt}
\paradigme{\textit{dir :} \jya tɤ-}
\begin{définition}\fra adopter (une fille)\end{définition}
\begin{définition}\cmn 领养(女儿)\end{définition}
\begin{exemple}\jya tɯrme ɯ-me to-nɤme\cmn 他们领养别人的女儿\end{exemple}
\begin{relation-sémantique}\synonyme{
\hyperlink{Ⓔnɤtɕɯ}{\textit{ \papi{nɤtɕɯ}}}
}\end{relation-sémantique}
\begin{relation-sémantique}\confer{
\hyperlink{Ⓔtɯ-me}{\textit{ \papi{tɯ-me}}}
}\end{relation-sémantique}\end{entrée}

\begin{entrée}
\vedette{\hypertarget{Ⓔnɤmuj}{\papi{ nɤmuj}}}\markboth{nɤmuj}{}\classe{vt}
\paradigme{\textit{dir :} \jya thɯ-}
\begin{définition}\fra enlever les plumes\end{définition}
\begin{définition}\cmn 拔羽毛\end{définition}
\begin{exemple}\jya pɣa thɯ-nɤmuj-a\cmn 我拔了鸡的羽毛\end{exemple}
\begin{relation-sémantique}\confer{
\hyperlink{Ⓔtɤ-muj}{\textit{ \papi{tɤ-muj}}}
}\end{relation-sémantique}\end{entrée}

\begin{entrée}
\vedette{\hypertarget{Ⓔnɤmujmaj}{\papi{ nɤmujmaj}}}\markboth{nɤmujmaj}{}\classe{vi}
\paradigme{\textit{dir :} \jya thɯ-}
\begin{définition}\ 
\begin{déclaration}\grammar{denom}\end{déclaration}\end{définition}
\begin{définition}\fra élaguer\end{définition}
\begin{définition}\cmn 修剪(树枝)\end{définition}
\begin{relation-sémantique}\confer{
\hyperlink{Ⓔɯ-mujmaj}{\textit{ \papi{ɯ-mujmaj}}}
}\end{relation-sémantique}\end{entrée}

\begin{entrée}
\vedette{\hypertarget{Ⓔnɤmkha}{\papi{ nɤmkha}}}\markboth{nɤmkha}{}\classe{n}
\begin{définition}\fra ciel\end{définition}
\begin{définition}\cmn 天
\begin{déclaration} \étymologie{\papi{nam.mkʰa}}\end{déclaration}\end{définition}
\begin{relation-sémantique}\synonyme{
\hyperlink{Ⓔtɯ-mɯ}{\textit{ \papi{tɯ-mɯ}}}
}\end{relation-sémantique}\end{entrée}

\begin{entrée}
\vedette{\hypertarget{Ⓔnɤmkɯm}{\papi{ nɤmkɯm}}}\markboth{nɤmkɯm}{}
\classe{vt}
\paradigme{\textit{dir :} \jya thɯ-}
\paradigme{\textit{dir :} \jya kɤ-}
\begin{définition}\ 
\begin{déclaration}\grammar{denom}\end{déclaration}\end{définition}
\begin{définition}\fra utiliser ... comme un oreiller\end{définition}
\begin{définition}\cmn 枕(躺着的时候把头放在东西上)\end{définition}
\begin{exemple}\jya tɯ-ŋga nɯ ka-nɤmkɯm\cmn 他枕了衣服睡\end{exemple}\end{entrée}

\begin{entrée}
\vedette{\hypertarget{Ⓔnɤmnɤm}{\papi{ nɤmnɤm}}}\markboth{nɤmnɤm}{}
\classe{vt}
\paradigme{\textit{dir :} \jya tɤ-}
\begin{définition}\ 
\begin{déclaration}\grammar{trop}\end{déclaration}\end{définition}
\begin{définition}\fra sentir\end{définition}
\begin{définition}\cmn 闻嗅(故意)
\begin{déclaration} \étymologie{\papi{mnam}}\end{déclaration}\end{définition}
\begin{exemple}\jya khɯna kɯ ta-nɤmnɤm\cmn 狗闻了一下\end{exemple}
\begin{exemple}\jya ki tɤ-nɤmnɤm\cmn 你闻一下这个\end{exemple}
\begin{exemple}\jya tɤ-mthɯm aj tɤ-nɤmnam-a\cmn 我闻了一下肉\end{exemple}
\begin{exemple}\jya tɤ-mthɯm ɯ-ɲɤ́-ɣɤdi kɯ tɤ-nɤmnam-a\cmn 我闻了一下肉有没有变味\end{exemple}
\begin{exemple}\jya ɯβrɤ-ɲɯ-ɣɤdi tɤ-nɤmnam-a\cmn 我闻了一下有没有变味\end{exemple}
\begin{relation-sémantique}\confer{
\hyperlink{Ⓔmnɤm}{\textit{ \papi{mnɤm}}}
}\end{relation-sémantique}
\begin{relation-sémantique}\confer{
\hyperlink{Ⓔɕɯmnɤm}{\textit{ \papi{ɕɯmnɤm}}}
}\end{relation-sémantique}\end{entrée}

\begin{entrée}
\vedette{\hypertarget{Ⓔnɤmɲo}{\papi{ nɤmɲo}}}\markboth{nɤmɲo}{}
\classe{vl}
\paradigme{\textit{dir :} \jya kɤ-}
\begin{définition}\fra regarder\end{définition}
\begin{définition}\cmn 观看\end{définition}
\begin{exemple}\jya @dianshi ma-kɤ-tɯ-nɤmɲɤm\cmn 你不要看电视\end{exemple}
\begin{relation-sémantique}\confer{
\hyperlink{Ⓔnɤmɲole}{\textit{ \papi{nɤmɲole}}}
}\end{relation-sémantique}
\begin{relation-sémantique}\confer{
\hyperlink{Ⓔsɤmɲo}{\textit{ \papi{sɤmɲo}}}
}\end{relation-sémantique}\end{entrée}

\begin{entrée}
\vedette{\hypertarget{Ⓔnɤmɲole}{\papi{ nɤmɲole}}}\markboth{nɤmɲole}{}
\classe{vi}
\paradigme{\textit{dir :} \jya nɯ-}
\begin{définition}\fra regarder le paysage\end{définition}
\begin{définition}\cmn 观看风景;到处观看\end{définition}
\begin{exemple}\jya @Chengdu kɯra nɯ-nɤmɲole\cmn 你观看成都(的风景)吧\end{exemple}
\begin{exemple}\jya kɯra aʁɤndɯndɤt nɯ-nɤmɲole\cmn 到处观看风景吧!\end{exemple}
\begin{exemple}\jya nɯ-nɤmɲole-a\cmn 我观看了\end{exemple}\begin{sous-entrée}
\vedette{\hypertarget{}{\papi{ znɤmɲole}}}\markboth{znɤmɲole}{}\classe{vt}
\begin{définition}\fra faire regarder le paysage\end{définition}
\begin{définition}\cmn 带……到处观看\end{définition}
\begin{relation-sémantique}\confer{
\hyperlink{Ⓔnɤmɲo}{\textit{ \papi{nɤmɲo}}}
}\end{relation-sémantique}
\end{sous-entrée}\end{entrée}

\begin{entrée}
\vedette{\hypertarget{Ⓔnɤmŋɤm}{\papi{ nɤmŋɤm}}}\markboth{nɤmŋɤm}{}
\begin{relation-sémantique}\confer{
\hyperlink{Ⓔmŋɤm}{\textit{ \papi{mŋɤm}}}
}\end{relation-sémantique}\end{entrée}

\begin{entrée}
\vedette{\hypertarget{Ⓔnɤmŋɯn}{\papi{ nɤmŋɯn}}}\markboth{nɤmŋɯn}{}
\classe{vt}
\paradigme{\textit{dir :} \jya pɯ-}
\begin{définition}\ 
\begin{déclaration}\grammar{trop}\end{déclaration}
\begin{déclaration}\grammar{trop}\end{déclaration}\end{définition}
\begin{définition}\fra être reconnaissant\end{définition}
\begin{définition}\cmn 感激;满意\end{définition}
\begin{exemple}\jya pɯ-ta-nɤmŋɯn\cmn 我令你满意了\end{exemple}
\begin{exemple}\jya jiɕqha nɯ kɯ laχtɕha ci nɯ́-wɣ-mbi-a rcanɯ pɯ-nɤmŋɯn-a\cmn 我很感激这个人给我这个东西\end{exemple}
\begin{exemple}\jya jiɕqha nɯ kɯ cha ci nɯ́-wɣ-jtshi-a rcanɯ ci pɯ-nɤmŋɯn-a\cmn 我很感激这个人给我喝酒\end{exemple}
\begin{relation-sémantique}\confer{
\hyperlink{Ⓔmŋɯn}{\textit{ \papi{mŋɯn}}}
}\end{relation-sémantique}
\begin{relation-sémantique}\synonyme{
\hyperlink{Ⓔnɤxpe}{\textit{ \papi{nɤxpe}}}
}\end{relation-sémantique}\end{entrée}

\begin{entrée}
\vedette{\hypertarget{Ⓔnɤmpɕɤr}{\papi{ nɤmpɕɤr}}}\markboth{nɤmpɕɤr}{}\classe{vt}
\paradigme{\textit{dir :} \jya nɯ-}
\begin{définition}\ 
\begin{déclaration}\grammar{trop}\end{déclaration}\end{définition}
\begin{définition}\fra trouver beau\end{définition}
\begin{définition}\cmn 觉得漂亮\end{définition}
\begin{exemple}\jya nɤʑo ɲɯ-ta-nɤmpɕɤr\cmn 我觉得你很漂亮\end{exemple}
\begin{exemple}\jya jiɕqha tɕheme nɯ ɲɯ-nɤmpɕar-a\cmn 我觉得这个姑娘长得很漂亮\end{exemple}
\begin{exemple}\jya jiɕqha tɯ-ŋga nɯ ɲɯ-nɤmpɕar-a\cmn 我觉得这件衣服很漂亮\end{exemple}\begin{sous-entrée}
\vedette{\hypertarget{}{\papi{ anɤmpɕɯpɕɤr}}}\markboth{anɤmpɕɯpɕɤr}{}\classe{vi}
\begin{définition}\fra se trouver beau les uns les autres\end{définition}
\begin{définition}\cmn 互相觉得漂亮\end{définition}
\begin{exemple}\jya ɲɯ-ɤnɤmpɕɯmpɕɤr-ndʑi\cmn 他们俩互相觉得漂亮\end{exemple}
\begin{relation-sémantique}\confer{
\hyperlink{Ⓔmpɕɤr}{\textit{ \papi{mpɕɤr}}}
}\end{relation-sémantique}
\end{sous-entrée}\begin{sous-entrée}
\vedette{\hypertarget{}{\papi{ sɤnɤmpɕɤr}}}\markboth{sɤnɤmpɕɤr}{}\classe{vi}
\begin{définition}\ 
\begin{déclaration}\grammar{apass}\end{déclaration}\end{définition}
\begin{définition}\fra trouver beau (les gens)\end{définition}
\begin{définition}\cmn 觉得别人漂亮\end{définition}
\begin{exemple}\jya nɤki tɤtɕɯpɯ nɯ kɯ-sɤ-nɤmpɕɤr ci ŋu\cmn 那个男孩子觉得(所有的姑娘都)很漂亮\end{exemple}
\end{sous-entrée}\begin{sous-entrée}
\vedette{\hypertarget{}{\papi{ ʑɣɤnɤmpɕɤr}}}\markboth{ʑɣɤnɤmpɕɤr}{}\classe{vi}
\begin{définition}\ 
\begin{déclaration}\grammar{refl}\end{déclaration}
\begin{déclaration}\grammar{trop}\end{déclaration}\end{définition}
\begin{définition}\fra se trouver beau\end{définition}
\begin{définition}\cmn 觉得自己漂亮\end{définition}
\begin{exemple}\jya tɕhemɤpɯ ra kɯ ``aʑo mpɕar-a" ntsɯ sɯso-nɯ ɕti nɤ, ɲɯ-ʑɣɤ-nɤmpɕɤrnɯ ɕti\cmn 姑娘们总觉得自己长得很漂亮\end{exemple}
\end{sous-entrée}\end{entrée}

\begin{entrée}
\vedette{\hypertarget{Ⓔnɤmphoʁmphɯr}{\papi{ nɤmphoʁmphɯr}}}\markboth{nɤmphoʁmphɯr}{}
\begin{relation-sémantique}\confer{
\hyperlink{Ⓔmphɯr}{\textit{ \papi{mphɯr}}}
}\end{relation-sémantique}\end{entrée}

\begin{entrée}
\vedette{\hypertarget{Ⓔnɤmphruʑa}{\papi{ nɤmphruʑa}}}\markboth{nɤmphruʑa}{}\classe{vt}
\paradigme{\textit{dir :} \jya tɤ-}
\paradigme{\textit{dir :} \jya \_}
\begin{définition}\ 
\begin{déclaration}\grammar{incorp}\end{déclaration}\end{définition}
\begin{définition}\fra faire l'un après l'autre\end{définition}
\begin{définition}\cmn 一个挨着一个地做
\begin{déclaration}\use{根据语境,\stylefv{nɤmphruʑa}可以附加不同的趋向前缀,如例句里面\stylefv{kú-wɣ-nɤmphruʑa}有\stylefv{kɤ-}前缀,和动词\stylefv{rtoʁ}“看”一样}\end{déclaration}\end{définition}
\begin{exemple}\jya kɯki ta-ma ki rcanɯ tɤ-nɤmphruʑa-t-a ʑo\cmn 我把任务一个接着一个的做了\end{exemple}
\begin{exemple}\jya @guazi tɯ-tɤ-fkɯm nɯ tɤ-nɤmphruʑa-t-a tɕe tɤ-ndza-t-a\cmn 我把那一袋的瓜子一个个地吃了\end{exemple}
\begin{relation-sémantique}\confer{
\hyperlink{ⒺʑaⒽ1}{\textit{ \papi{ʑa1}}}
}\end{relation-sémantique}
\begin{relation-sémantique}\confer{
\hyperlink{Ⓔɯ-mphru}{\textit{ \papi{ɯ-mphru}}}
}\end{relation-sémantique}
\begin{relation-sémantique}\confer{
\hyperlink{Ⓔsɤʑa}{\textit{ \papi{sɤʑa}}}
}\end{relation-sémantique}\end{entrée}

\begin{entrée}
\vedette{\hypertarget{Ⓔnɤmpɯ}{\papi{ nɤmpɯ}}}\markboth{nɤmpɯ}{}
\begin{relation-sémantique}\confer{
\hyperlink{Ⓔmpɯ}{\textit{ \papi{mpɯ}}}
}\end{relation-sémantique}\end{entrée}

\begin{entrée}
\vedette{\hypertarget{Ⓔnɤmqe}{\papi{ nɤmqe}}}\markboth{nɤmqe}{}\classe{vt}
\paradigme{\textit{dir :} \jya tɤ-}
\begin{définition}\fra insulter, gronder\end{définition}
\begin{définition}\cmn 骂\end{définition}
\begin{exemple}\jya tɤ-nɤmqe-t-a\cmn 我骂了他\end{exemple}
\begin{exemple}\jya ta-nɤmqe\cmn 他骂了他\end{exemple}
\begin{exemple}\jya jiɕqha tɤ-mu nɯ kɯ ɯ-rɟit ta-nɤmqe\cmn 那个母亲骂了他的儿子\end{exemple}
\begin{exemple}\jya ma-tɤ-kɯ-nɤmqe-a\cmn 你不要骂我\end{exemple}
\begin{exemple}\jya tɤ-ta-nɤmqe\cmn 我骂了你\end{exemple}
\begin{relation-sémantique}\synonyme{
\hyperlink{Ⓔnɯjʁo}{\textit{ \papi{nɯjʁo}}}
}\end{relation-sémantique}\begin{sous-entrée}
\vedette{\hypertarget{}{\papi{ ʑɣɤnɤmqe}}}\markboth{ʑɣɤnɤmqe}{}\classe{vi}
\begin{définition}\ 
\begin{déclaration}\grammar{refl}\end{déclaration}\end{définition}
\begin{définition}\fra se faire insulter\end{définition}
\begin{définition}\cmn 招人骂\end{définition}
\end{sous-entrée}\end{entrée}

\begin{entrée}
\vedette{\hypertarget{Ⓔnɤmtɕɯrlu}{\papi{ nɤmtɕɯrlu}}}\markboth{nɤmtɕɯrlu}{} (\variante{nɤmtɕɯrlɯr}) 
\paradigme{\textit{dir :} \jya thɯ-}
\begin{définition}\ 
\begin{déclaration}\grammar{n.orient}\end{déclaration}\end{définition}
\begin{définition}\fra tourner en rond\end{définition}
\begin{définition}\cmn 转来转去\end{définition}
\begin{exemple}\jya nɯtɕu ma-tɯ-nɤmtɕɯrlɯr ntsɯ\cmn 你别总是在那里转来转去\end{exemple}
\begin{exemple}\jya pɯ-nɤmtɕɯrlɯr-a ntsɯ\cmn 我总是(在那里)转来转去(过去时)\end{exemple}
\begin{exemple}\jya nɤʑo sɲikuku kha ɯ-ŋgɯ ntsɯ chɯ-tɯ-nɤmtɕhɯrlu\cmn 你总是每天在家里转来转去(乌鸦之言)\end{exemple}
\begin{relation-sémantique}\confer{
\hyperlink{Ⓔmtɕɯr}{\textit{ \papi{mtɕɯr}}}
}\end{relation-sémantique}\classe{vi}\end{entrée}

\begin{entrée}
\vedette{\hypertarget{Ⓔnɤmthu}{\papi{ nɤmthu}}}\markboth{nɤmthu}{}
\classe{vt}
\paradigme{\textit{dir :} \jya nɯ-}
\begin{définition}\fra envier\end{définition}
\begin{définition}\cmn 羡慕\end{définition}
\begin{exemple}\jya ɯ-scɯʁzɯɣ ɲɯ-βdi, nɯ-nɤmthu-t-a\cmn 她相貌美观,我很羡慕\end{exemple}
\begin{exemple}\jya ɲɯ-ta-nɤmthu\cmn 我很羡慕你\end{exemple}
\begin{relation-sémantique}\synonyme{
\hyperlink{Ⓔnɤsma}{\textit{ \papi{nɤsma}}}
}\end{relation-sémantique}
\begin{relation-sémantique}\confer{
\hyperlink{Ⓔmthu}{\textit{ \papi{mthu}}}
}\end{relation-sémantique}\end{entrée}

\begin{entrée}
\vedette{\hypertarget{Ⓔnɤmthɯn}{\papi{ nɤmthɯn}}}\markboth{nɤmthɯn}{}
\classe{vt}
\paradigme{\textit{dir :} \jya nɯ-}
\begin{définition}\ 
\begin{déclaration}\grammar{trop}\end{déclaration}\end{définition}
\begin{définition}\fra aimer\end{définition}
\begin{définition}\cmn 爱;喜欢(异性)\end{définition}
\begin{exemple}\jya jiɕqha tɕheme nɯ ɲɤ-nɤmthɯn ɲɯ-ŋu\cmn 他喜欢了这个女人\end{exemple}
\begin{exemple}\jya jiɕqha nɯ ɯ-nmaʁ ɣɤʑu ri, li ɲɤ-nɤmthɯn\cmn 那个(女人)虽然有丈夫,但是爱上了另外一个\end{exemple}
\begin{exemple}\jya aj ɲɯ-ta-nɤmthɯn! hehe, nɯ ma-tɯ-ti!\cmn 我喜欢你!(女孩子的答复)你别那样说!\end{exemple}
\begin{relation-sémantique}\synonyme{
\hyperlink{Ⓔnɤntshi}{\textit{ \papi{nɤntshi}}}
}\end{relation-sémantique}
\begin{relation-sémantique}\confer{
\hyperlink{Ⓔamthɯn}{\textit{ \papi{amthɯn}}}
}\end{relation-sémantique}\end{entrée}

\begin{entrée}
\vedette{\hypertarget{Ⓔnɤmtshɤr}{\papi{ nɤmtshɤr}}}\markboth{nɤmtshɤr}{}\classe{vt}
\paradigme{\textit{dir :} \jya nɯ-}
\begin{définition}\ 
\begin{déclaration}\grammar{trop}\end{déclaration}\end{définition}
\begin{définition}\fra trouver étrange\end{définition}
\begin{définition}\cmn 觉得奇怪;感到惊讶
\begin{déclaration} \étymologie{\papi{mtsʰar}}\end{déclaration}\end{définition}
\begin{exemple}\jya jiɕqha nɯ tɤ-aʑɯʑu-ndʑi ri, ``mɤ-cha" nɯ-sɯso-t-a ri, pɯ-cha rcanɯ nɯ-nɤmtshar-a\cmn 在角力的时候,我以为他不行,但他居然赢了,我感到很惊讶\end{exemple}
\begin{exemple}\jya ``mɤ-tɯ-cha" nɯ-sɯso-t-a ri, pɯ-tɯ-cha rcanɯ, nɯ-ta-nɤmtshɤr\cmn 我以为你不行,但是你居然成功了,(你令)我感到很惊讶\end{exemple}
\begin{exemple}\jya jiʑora ji-skɤt ɯ-ɲɯ́-nɤmtshɤr-nɯ?\cmn 他们觉得我们的语言奇怪吗?\end{exemple}
\begin{relation-sémantique}\synonyme{
\hyperlink{Ⓔnaχaʁ}{\textit{ \papi{naχaʁ}}}
}\end{relation-sémantique}\begin{sous-entrée}
\vedette{\hypertarget{}{\papi{ ʑɣɤnɤmtshɤr}}}\markboth{ʑɣɤnɤmtshɤr}{}\classe{vi}
\paradigme{\textit{dir :} \jya nɯ-}
\begin{définition}\ 
\begin{déclaration}\grammar{refl}\end{déclaration}
\begin{déclaration}\grammar{trop}\end{déclaration}\end{définition}
\begin{définition}\fra trouver bizarre quelque chose à propos de soi\end{définition}
\begin{définition}\cmn 觉得自己的事情奇怪\end{définition}
\begin{exemple}\jya ɲɯ-ʑɣɤnɤmtshar-a ɕti ma pɯ-nɯʑɯβ-a ɕti ri, nɯ kɯnɤ mɤ-kɯ-mbrɤt ʑo ɲɯ-ɤχom-a\cmn 我觉得很奇怪,虽然我睡了,还是不停地打哈欠\end{exemple}
\begin{relation-sémantique}\confer{
\hyperlink{Ⓔsɤmtshɤr}{\textit{ \papi{sɤmtshɤr}}}
}\end{relation-sémantique}
\end{sous-entrée}\end{entrée}

\begin{entrée}
\vedette{\hypertarget{Ⓔnɤmtsioʁ}{\papi{ nɤmtsioʁ}}}\markboth{nɤmtsioʁ}{}
\classe{vt}
\paradigme{\textit{dir :} \jya tɤ-}
\begin{définition}\ 
\begin{déclaration}\grammar{denom}\end{déclaration}\end{définition}
\begin{définition}\fra donner un coup de bec\end{définition}
\begin{définition}\cmn 啄
\begin{déclaration}\use{表示“鸡啄食物”的话不能用\stylefv{nɤmtsioʁ},必须用\stylefv{nɯrdoʁ}}\end{déclaration}\end{définition}
\begin{exemple}\jya kumpɣa kɯ tɤ́-wɣ-nɤmtsioʁ-a\cmn 我被鸡啄了一口\end{exemple}
\begin{exemple}\jya ta-nɤmtsioʁ\cmn 鸡啄了一口\end{exemple}
\begin{sous-entrée}
\vedette{\hypertarget{}{\papi{ anɤmtsɯmtsioʁ}}}\markboth{anɤmtsɯmtsioʁ}{}\classe{vi}
\paradigme{\textit{dir :} \jya tɤ-}
\begin{définition}\ 
\begin{déclaration}\grammar{recip}\end{déclaration}\end{définition}
\begin{définition}\fra se donner des coups de bec les uns aux autres
\begin{déclaration}\grammar{互相啄}\end{déclaration}\end{définition}
\begin{exemple}\jya to-k-ɤnɤmtsɯmtsioʁ-nɯ-ci\cmn (公鸡)互相啄了起来\end{exemple}
\end{sous-entrée}\begin{sous-entrée}
\vedette{\hypertarget{}{\papi{ sɤnɤmtsioʁ}}}\markboth{sɤnɤmtsioʁ}{}\classe{vi}
\paradigme{\textit{dir :} \jya tɤ-}
\begin{définition}\ 
\begin{déclaration}\grammar{apass}\end{déclaration}\end{définition}
\begin{définition}\fra donner des coups de bec aux gens\end{définition}
\begin{définition}\cmn 啄人\end{définition}
\begin{exemple}\jya kumpɣa ɲɯ-sɤnɤmtsioʁ\cmn 鸡老爱啄人\end{exemple}
\begin{relation-sémantique}\confer{
\hyperlink{Ⓔɯ-mtsioʁ}{\textit{ \papi{ɯ-mtsioʁ}}}
}\end{relation-sémantique}
\end{sous-entrée}\end{entrée}

\begin{entrée}
\vedette{\hypertarget{Ⓔnɤmɯm}{\papi{ nɤmɯm}}}\markboth{nɤmɯm}{}
\begin{relation-sémantique}\confer{
\hyperlink{Ⓔmɯm}{\textit{ \papi{mɯm}}}
}\end{relation-sémantique}\end{entrée}

\begin{entrée}
\vedette{\hypertarget{Ⓔnɤmɯma}{\papi{ nɤmɯma}}}\markboth{nɤmɯma}{}
\classe{vt}
\paradigme{\textit{dir :} \jya nɯ-}\acception{1}
\begin{définition}\fra caresser\end{définition}
\begin{définition}\cmn 抚摸\end{définition}
\begin{exemple}\jya nɯ-nɤmɯma-t-a, kɤ-nɤmɯma-t-a\cmn 我摸了\end{exemple}
\begin{exemple}\jya @zhuozi nɯ-nɤmɯma-t-a ri ɲɯ-mpɕu\cmn 我摸了一下桌子,很光滑\end{exemple}\acception{2}
\begin{définition}\fra tâtonner, chercher à tâtons\end{définition}
\begin{définition}\cmn (暗中)摸索、探索\end{définition}
\begin{relation-sémantique}\confer{
\hyperlink{Ⓔznɤmɯma}{\textit{ \papi{znɤmɯma}}}
}\end{relation-sémantique}\end{entrée}

\begin{entrée}
\vedette{\hypertarget{Ⓔnɤnɤm}{\papi{ nɤnɤm}}}\markboth{nɤnɤm}{}\classe{vs}
\begin{définition}\fra être amplement suffisant\end{définition}
\begin{définition}\cmn 绰绰有余\end{définition}
\begin{exemple}\jya a-kɤ-ndza kɤ-ŋga pɯ-nɤnɤm\cmn 我以前吃穿都充裕\end{exemple}\end{entrée}

\begin{entrée}
\vedette{\hypertarget{Ⓔnɤnbaʁ}{\papi{ nɤnbaʁ}}}\markboth{nɤnbaʁ}{}
\begin{relation-sémantique}\confer{
\hyperlink{Ⓔanbaʁ}{\textit{ \papi{anbaʁ}}}
}\end{relation-sémantique}\end{entrée}

\begin{entrée}
\vedette{\hypertarget{Ⓔnɤnbɯnbaʁ}{\papi{ nɤnbɯnbaʁ}}}\markboth{nɤnbɯnbaʁ}{}
\begin{relation-sémantique}\confer{
\hyperlink{Ⓔanbaʁ}{\textit{ \papi{anbaʁ}}}
}\end{relation-sémantique}\end{entrée}

\begin{entrée}
\vedette{\hypertarget{Ⓔnɤndʐaβlaβ}{\papi{ nɤndʐaβlaβ}}}\markboth{nɤndʐaβlaβ}{}
\classe{vi}
\paradigme{\textit{dir :} \jya \_}
\begin{définition}\ 
\begin{déclaration}\grammar{n.orient}\end{déclaration}\end{définition}
\begin{définition}\fra rouler\end{définition}
\begin{définition}\cmn 滚来滚去
\begin{déclaration}\use{\stylefv{nɤmbɣaʁlaʁ}表示故意打滚,而\stylefv{nɤndʐaβlaβ}则表示不小心滚下}\end{déclaration}\end{définition}
\begin{exemple}\jya ɲɯ-saʁdɤt tɕe pɯ-nɤndʐaβlaβ-a\cmn 地很滑,我摔跤了很多次(爬起来又摔跤了几次)\end{exemple}
\begin{exemple}\jya mkhɯrlu aʁɤndɯndɤt ɲɯ-nɤndʐaβlaβ\cmn 轮子到处滚动\end{exemple}
\begin{relation-sémantique}\confer{
\hyperlink{Ⓔndʐaβ}{\textit{ \papi{ndʐaβ}}}
}\end{relation-sémantique}\end{entrée}

\begin{entrée}
\vedette{\hypertarget{Ⓔnɤndɤɣ}{\papi{ nɤndɤɣ}}}\markboth{nɤndɤɣ}{}\classe{vi}
\paradigme{\textit{dir :} \jya pɯ-}
\begin{définition}\fra être empoisonné\end{définition}
\begin{définition}\cmn 中毒\end{définition}\begin{sous-entrée}
\vedette{\hypertarget{}{\papi{ znɤndɤɣ}}}\markboth{znɤndɤɣ}{}\classe{vt}
\paradigme{\textit{dir :} \jya pɯ-}
\begin{définition}\fra empoisonner\end{définition}
\begin{définition}\cmn 使人中毒\end{définition}
\begin{exemple}\jya ɯ-tshɯɣa a-mɤ-tɤ-βdi tɕe pjɯ-kɯ-znɤndɤɣ ŋu\cmn 如果弄得不好就会中毒(例如,蘑菇加工的方法)\end{exemple}
\begin{exemple}\jya ma-tɤ-tɯ-ndze ma tɯ́-wɣ-znɤndɤɣ\cmn 不要吃,会中毒\end{exemple}
\begin{relation-sémantique}\confer{
\hyperlink{Ⓔsɤndɤɣ}{\textit{ \papi{sɤndɤɣ}}}
}\end{relation-sémantique}
\end{sous-entrée}\end{entrée}

\begin{entrée}
\vedette{\hypertarget{Ⓔnɤndɤɣri}{\papi{ nɤndɤɣri}}}\markboth{nɤndɤɣri}{}
\classe{vi}
\paradigme{\textit{dir :} \jya pɯ-}
\begin{définition}\ 
\begin{déclaration}\grammar{denom}\end{déclaration}\end{définition}
\begin{définition}\fra avoir un enfant illégitime (femme)\end{définition}
\begin{définition}\cmn 生私生子\end{définition}\begin{sous-entrée}
\vedette{\hypertarget{}{\papi{ znɤndɤɣri}}}\markboth{znɤndɤɣri}{}\classe{vt}
\paradigme{\textit{dir :} \jya pɯ-}
\begin{définition}\ 
\begin{déclaration}\grammar{caus}\end{déclaration}\end{définition}
\begin{définition}\fra avoir un enfant illégitime (homme)\end{définition}
\begin{définition}\cmn 令(一个女子)生私生子\end{définition}
\begin{exemple}\jya kɯmaʁ tɤ-tɕɯ nɯ kɯ pa-znɤndɤɣri\cmn 她跟别的男人有了私生子\end{exemple}
\begin{relation-sémantique}\confer{
\hyperlink{Ⓔtɤndɤɣri}{\textit{ \papi{tɤndɤɣri}}}
}\end{relation-sémantique}
\end{sous-entrée}\end{entrée}

\begin{entrée}
\vedette{\hypertarget{Ⓔnɤndɤr}{\papi{ nɤndɤr}}}\markboth{nɤndɤr}{}\classe{vi}
\paradigme{\textit{dir :} \jya tɤ-}
\begin{définition}\fra vibrer\end{définition}
\begin{définition}\cmn 振动
\begin{déclaration} \étymologie{\papi{ⁿdar}}\end{déclaration}\end{définition}
\begin{exemple}\jya nɯ pɯ-atɤr tɕe tɤ-nɤndɤr\cmn 掉下去就震动了\end{exemple}\begin{sous-entrée}
\vedette{\hypertarget{}{\papi{ znɤndɤr}}}\markboth{znɤndɤr}{}\classe{vt}
\paradigme{\textit{dir :} \jya tɤ-}
\begin{définition}\ 
\begin{déclaration}\grammar{caus}\end{déclaration}\end{définition}
\begin{définition}\fra faire vibrer\end{définition}
\begin{définition}\cmn (使)振动\end{définition}
\begin{exemple}\jya ta-znɤndɤr\cmn 他(让那个东西)震动了一下\end{exemple}
\begin{exemple}\jya tɤ-znɤndar-a\cmn 我震动了一下\end{exemple}
\begin{exemple}\jya kutɕu tɤŋkhɯt tɤ-lat-a tɕe, @luyinji tɤ-znɤndar-a\cmn 我在这里拍了一拳,把录音机震动了一下\end{exemple}
\end{sous-entrée}\end{entrée}

\begin{entrée}
\vedette{\hypertarget{Ⓔnɤndʐɤrqɯ}{\papi{ nɤndʐɤrqɯ}}}\markboth{nɤndʐɤrqɯ}{}\classe{vs}
\paradigme{\textit{dir :} \jya tɤ-}
\begin{définition}\fra avoir froid, être frileux\end{définition}
\begin{définition}\cmn 觉得冷,怕冷\end{définition}
\begin{exemple}\jya kɯre ku-nɤndʐɤrqɯ-a\cmn 我在这里觉得冷\end{exemple}
\begin{exemple}\jya qartsɯ tɕe kɯ-nɤndʐɤrqɯ ntsɯ ŋgrɤl\cmn 冬天的时候人总是觉得冷\end{exemple}
\begin{exemple}\jya ɣɯjpa qartsɯ pɯ-ɣɤndʐo tɕe aʑo pɯ-nɤndʐɤrqɯ-a ntsɯ\cmn 今年冬天我一直觉得很冷\end{exemple}
\begin{relation-sémantique}\confer{
\hyperlink{Ⓔtɤndʐo}{\textit{ \papi{tɤndʐo}}}
}\end{relation-sémantique}\end{entrée}

\begin{entrée}
\vedette{\hypertarget{Ⓔnɤndʐi}{\papi{ nɤndʐi}}}\markboth{nɤndʐi}{}\classe{vt}
\paradigme{\textit{dir :} \jya pɯ-}
\begin{définition}\fra enlever la peau\end{définition}
\begin{définition}\cmn 剥皮\end{définition}
\begin{exemple}\jya paʁ-ku pɯ-nɤndʐi-t-a\cmn 我剥了猪头的皮\end{exemple}
\begin{relation-sémantique}\confer{
\hyperlink{Ⓔtɯ-ndʐi}{\textit{ \papi{tɯ-ndʐi}}}
}\end{relation-sémantique}\end{entrée}

\begin{entrée}
\vedette{\hypertarget{Ⓔnɤndʐo}{\papi{ nɤndʐo}}}\markboth{nɤndʐo}{}\classe{vi}
\paradigme{\textit{dir :} \jya nɯ-}
\begin{définition}\fra prendre froid\end{définition}
\begin{définition}\cmn 着凉\end{définition}
\begin{exemple}\jya nɯ-nɤndʐo-a\cmn 我着凉了\end{exemple}
\begin{exemple}\jya a-ŋga ɲɯ-mba tɕe nɯ-nɤndʐo-a\cmn 我的衣服很薄,我着凉了\end{exemple}\begin{sous-entrée}
\vedette{\hypertarget{}{\papi{ ʑɣɤnɤndʐo}}}\markboth{ʑɣɤnɤndʐo}{}\classe{vi}
\paradigme{\textit{dir :} \jya nɯ-}
\begin{définition}\ 
\begin{déclaration}\grammar{refl}\end{déclaration}\end{définition}
\begin{définition}\fra attraper froid\end{définition}
\begin{définition}\cmn 令自己受凉\end{définition}
\begin{exemple}\jya nɯ-ʑɣɤnɤndʐo-a\cmn 我令自己受凉了\end{exemple}
\begin{exemple}\jya ma-nɯ-tɯ-ʑɣɤ-nɤndʐo\cmn 你不要令自己受凉\end{exemple}
\begin{relation-sémantique}\confer{
\hyperlink{Ⓔtɤndʐo}{\textit{ \papi{tɤndʐo}}}
}\end{relation-sémantique}
\end{sous-entrée}\end{entrée}

\begin{entrée}
\vedette{\hypertarget{Ⓔnɤndɯndɤt}{\papi{ nɤndɯndɤt}}}\markboth{nɤndɯndɤt}{}
\classe{vi}
\paradigme{\textit{dir :} \jya \_}
\begin{définition}\fra aller n'importe où\end{définition}
\begin{définition}\cmn 到处走\end{définition}
\begin{exemple}\jya ma-ɕɯ-tɯ-nɤndɯndɤt\cmn 你不要到处乱走\end{exemple}
\begin{relation-sémantique}\confer{
\hyperlink{Ⓔŋotɕuŋondɤt}{\textit{ \papi{ŋotɕuŋondɤt}}}
}\end{relation-sémantique}
\begin{relation-sémantique}\confer{
\hyperlink{Ⓔaʁɤndɯndɤt}{\textit{ \papi{aʁɤndɯndɤt}}}
}\end{relation-sémantique}
\begin{sous-entrée}
\vedette{\hypertarget{}{\papi{ znɤndɯndɤt}}}\markboth{znɤndɯndɤt}{}\classe{vt}
\begin{définition}\ 
\begin{déclaration}\grammar{caus}\end{déclaration}\end{définition}
\begin{définition}\fra laisser aller n'importe oæ\end{définition}
\begin{exemple}\cmn 让……到处乱走\end{exemple}
\end{sous-entrée}\end{entrée}

\begin{entrée}
\vedette{\hypertarget{Ⓔnɤndɯndo}{\papi{ nɤndɯndo}}}\markboth{nɤndɯndo}{}
\classe{vt}
\paradigme{\textit{dir :} \jya tɤ-}
\begin{définition}\ 
\begin{déclaration}\grammar{n.orient}\end{déclaration}\end{définition}
\begin{définition}\fra emmener partout\end{définition}
\begin{définition}\cmn 随身带着;带来带去\end{définition}
\begin{exemple}\jya laχtɕha nɯ kɤ-nɤndɯndo ntsɯ ɲɯ-ra\cmn 这个东西要随身带上\end{exemple}
\begin{exemple}\jya @zixingche ɲɯ-ɤz-nɤndɯndo\cmn 他去哪里都带自行车\end{exemple}
\begin{relation-sémantique}\confer{
\hyperlink{Ⓔndo}{\textit{ \papi{ndo}}}
}\end{relation-sémantique}\end{entrée}

\begin{entrée}
\vedette{\hypertarget{Ⓔnɤndɯt}{\papi{ nɤndɯt}}}\markboth{nɤndɯt}{}
\classe{vt}
\paradigme{\textit{dir :} \jya tɤ-}
\begin{définition}\fra se disputer\end{définition}
\begin{définition}\cmn 争论;争吵\end{définition}
\begin{exemple}\jya jiɕqha nɯ cho tɤ-anɯmqaj-tɕi tɕe tɤ-nɤndɯt-tɕi\cmn 我跟这个人吵架了,争吵了\end{exemple}
\begin{exemple}\jya laχtɕha tɤ-nɤndɯt-tɕi\cmn 我们俩争了那个东西\end{exemple}
\begin{exemple}\jya ma-tɤ-tɯ-nɤndɯt-nɯ\cmn 你们不要再争了\end{exemple}
\begin{exemple}\jya mɤ-nɤndɯt-tɕi\cmn 我们俩不再争了\end{exemple}\end{entrée}

\begin{entrée}
\vedette{\hypertarget{Ⓔnɤndza}{\papi{ nɤndza}}}\markboth{nɤndza}{}\classe{vi}
\paradigme{\textit{dir :} \jya kɤ-}
\begin{définition}\fra attraper la lèpre\end{définition}
\begin{définition}\cmn 患上麻疯病\end{définition}
\begin{exemple}\jya kɤ-kɯ-nɤndza\cmn 麻风病患者(骂人的话)\end{exemple}
\begin{exemple}\jya kɤ-kɯ-nɤndza nɯ tɯŋgo stu kɯ-ŋɤn ŋu\cmn 麻风病是最严重的病\end{exemple}\end{entrée}

\begin{entrée}
\vedette{\hypertarget{Ⓔnɤndzɤβ}{\papi{ nɤndzɤβ}}}\markboth{nɤndzɤβ}{}
\begin{relation-sémantique}\confer{
\hyperlink{Ⓔndzɤβ}{\textit{ \papi{ndzɤβ}}}
}\end{relation-sémantique}\end{entrée}

\begin{entrée}
\vedette{\hypertarget{Ⓔnɤndzɯt}{\papi{ nɤndzɯt}}}\markboth{nɤndzɯt}{}
\begin{relation-sémantique}\confer{
\hyperlink{Ⓔandzɯt}{\textit{ \papi{andzɯt}}}
}\end{relation-sémantique}\end{entrée}

\begin{entrée}
\vedette{\hypertarget{Ⓔnɤndʑaʁlaʁ}{\papi{ nɤndʑaʁlaʁ}}}\markboth{nɤndʑaʁlaʁ}{}\classe{vi}
\paradigme{\textit{dir :} \jya pɯ-}
\begin{définition}\ 
\begin{déclaration}\grammar{n.orient}\end{déclaration}\end{définition}
\begin{définition}\fra nager\end{définition}
\begin{définition}\cmn 游泳;漂浮;游来游去\end{définition}
\begin{exemple}\jya aj pɯ-nɤndʑaʁlaʁ-a\cmn 我游来游去了\end{exemple}
\begin{relation-sémantique}\confer{
\hyperlink{Ⓔndʑaʁ}{\textit{ \papi{ndʑaʁ}}}
}\end{relation-sémantique}\end{entrée}

\begin{entrée}
\vedette{\hypertarget{Ⓔnɤndʑe}{\papi{ nɤndʑe}}}\markboth{nɤndʑe}{}\classe{vi}
\paradigme{\textit{dir :} \jya pɯ-}
\begin{définition}\fra profiter de quelque chose\end{définition}
\begin{définition}\cmn 占便宜\end{définition}
\begin{exemple}\jya pɯ-tɯ-nɤndʑe\cmn 你占了便宜\end{exemple}
\begin{exemple}\jya tɯtsɣe tɤ-βzu-tɕi, aʑo pɯ-nɯzɟɯ-a, nɤʑo pɯ-tɯ-nɤndʑe\cmn 我们俩做了生意,我吃亏了,你占了便宜\end{exemple}
\begin{exemple}\jya ʑɴɢɯloʁ pɯ-nɯkro-tɕi, nɤʑo pɯ-tɯ-nɤndʑe, aʑo pɯ-nɯzɟɯ-a\cmn 我们俩分核桃,你占了便宜,我吃亏了\end{exemple}
\begin{relation-sémantique}\antonyme{
\hyperlink{Ⓔnɯzɟɯ}{\textit{ \papi{nɯzɟɯ}}}
}\end{relation-sémantique}\end{entrée}

\begin{entrée}
\vedette{\hypertarget{Ⓔnɤndʑɣi}{\papi{ nɤndʑɣi}}}\markboth{nɤndʑɣi}{}\classe{vt}
\paradigme{\textit{dir :} \jya thɯ-}
\paradigme{\textit{dir :} \jya pɯ-}
\begin{définition}\fra avoir de la morve au nez\end{définition}
\begin{définition}\cmn 吊着鼻涕\end{définition}
\begin{exemple}\jya ɯʑo kɯ ɯ-ɕnaβ pjɯ-nɤndʑɣi ŋgrɤl\cmn 他平时吊着鼻涕\end{exemple}\end{entrée}

\begin{entrée}
\vedette{\hypertarget{Ⓔnɤngɯt}{\papi{ nɤngɯt}}}\markboth{nɤngɯt}{}
\classe{vt}
\paradigme{\textit{dir :} \jya tɤ-}
\begin{définition}\fra posséder en commun, partager\end{définition}
\begin{définition}\cmn 共同拥有\end{définition}
\begin{exemple}\jya @cai tɤ-nɤngɯt-tɕi\cmn 我们俩一起吃了一碗菜\end{exemple}
\begin{exemple}\jya ki @cidian ki nɤngɯt-tɕi\cmn 我们一起拥有这本词典\end{exemple}
\begin{exemple}\jya tɤ-nɤngɯt-a\cmn 我也有了一份\end{exemple}\begin{sous-entrée}
\vedette{\hypertarget{}{\papi{ znɤngɯt}}}\markboth{znɤngɯt}{}\classe{vt}
\paradigme{\textit{dir :} \jya tɤ-}
\begin{définition}\fra partager avec\end{définition}
\begin{définition}\cmn 跟……分、一起用\end{définition}
\begin{relation-sémantique}\confer{
\hyperlink{Ⓔtɤngɯt}{\textit{ \papi{tɤngɯt}}}
}\end{relation-sémantique}
\begin{relation-sémantique}\confer{
\hyperlink{Ⓔangɯt}{\textit{ \papi{angɯt}}}
}\end{relation-sémantique}
\end{sous-entrée}\end{entrée}

\begin{entrée}
\vedette{\hypertarget{Ⓔnɤntshɣɤz}{\papi{ nɤntshɣɤz}}}\markboth{nɤntshɣɤz}{}
\classe{vt}
\paradigme{\textit{dir :} \jya nɯ-}
\begin{définition}\fra heurter, se cogner contre\end{définition}
\begin{définition}\cmn 撞到(无意)\end{définition}
\begin{exemple}\jya nɯ́-wɣ-nɤntshɣaz-a\cmn 他撞到我了\end{exemple}
\begin{exemple}\jya @qiche ɲɯ-dɤn tɕe ɲɯ-kɯ-nɤntshɣɤz ɲɯ-ŋu\cmn 汽车很多,会撞到人\end{exemple}\begin{sous-entrée}
\vedette{\hypertarget{}{\papi{ sɤnɤntshɣɤz}}}\markboth{sɤnɤntshɣɤz}{}\classe{vi}
\begin{définition}\ 
\begin{déclaration}\grammar{apass}\end{déclaration}\end{définition}
\begin{définition}\fra heurter les gens\end{définition}
\begin{définition}\cmn 撞到人\end{définition}
\end{sous-entrée}\end{entrée}

\begin{entrée}
\vedette{\hypertarget{Ⓔnɤntshi}{\papi{ nɤntshi}}}\markboth{nɤntshi}{}
\classe{vt}
\paradigme{\textit{dir :} \jya nɯ-}
\begin{définition}\ 
\begin{déclaration}\grammar{trop}\end{déclaration}\end{définition}
\begin{définition}\fra aimer\end{définition}
\begin{définition}\cmn 爱;喜欢(异性)
\begin{déclaration}\use{没有发生男女之间的关系才用这个词,不然就用\stylefv{nɤmthɯn}}\end{déclaration}\end{définition}
\begin{exemple}\jya ɲɯ-nɤntshi-a tɕe aʑo ɕɯ-the-a ra\cmn 我喜欢她,我要去向她求婚\end{exemple}
\begin{exemple}\jya jiɕqha tɕheme nɯ ɲɯ-nɤntshi-a\cmn 我喜欢这个女子\end{exemple}
\begin{relation-sémantique}\synonyme{
\hyperlink{Ⓔnɤmthɯn}{\textit{ \papi{nɤmthɯn}}}
}\end{relation-sémantique}
\begin{relation-sémantique}\confer{
\hyperlink{ⒺntshiⒽ2}{\textit{ \papi{ntshi2}}}
}\end{relation-sémantique}\begin{sous-entrée}
\vedette{\hypertarget{}{\papi{ anɤntshɯntshi}}}\markboth{anɤntshɯntshi}{}\classe{vi}
\begin{définition}\ 
\begin{déclaration}\grammar{appl}\end{déclaration}
\begin{déclaration}\grammar{recip}\end{déclaration}\end{définition}
\begin{définition}\fra s'aimer l'un l'autre\end{définition}
\begin{définition}\cmn 相爱\end{définition}
\begin{exemple}\jya ɲɯ-ɤnɤntshɯntshi-ndʑi\cmn 他们俩相爱\end{exemple}
\begin{relation-sémantique}\synonyme{
\hyperlink{Ⓔanɯrgɯrga}{\textit{ \papi{anɯrgɯrga}}}
}\end{relation-sémantique}
\end{sous-entrée}\end{entrée}

\begin{entrée}
\vedette{\hypertarget{Ⓔnɤɲchɯɲchɤm}{\papi{ nɤɲchɯɲchɤm}}}\markboth{nɤɲchɯɲchɤm}{}
\classe{vi}
\paradigme{\textit{dir :} \jya thɯ-}
\begin{définition}\fra rôder\end{définition}
\begin{définition}\cmn 无聊地到处流浪;到处走动\end{définition}
\begin{exemple}\jya kɯ-nɤɲchɯɲchɤm jɤ-ari-a\cmn 我去到处流浪\end{exemple}\end{entrée}

\begin{entrée}
\vedette{\hypertarget{Ⓔnɤɲi}{\papi{ nɤɲi}}}\markboth{nɤɲi}{}\classe{vt}
\paradigme{\textit{dir :} \jya tɤ-}
\begin{définition}\fra se tenir avec\end{définition}
\begin{définition}\cmn 拄着……,仗着……\end{définition}
\begin{exemple}\jya laʁjɯɣ ci tɤ-nɤɲi tɕe mɤ-tɯ-ndʐaβ\cmn 你拄着拐杖就不会摔倒\end{exemple}
\begin{relation-sémantique}\confer{
\hyperlink{Ⓔtɤɲi}{\textit{ \papi{tɤɲi}}}
}\end{relation-sémantique}\end{entrée}

\begin{entrée}
\vedette{\hypertarget{Ⓔnɤɲɟɯɲɟu}{\papi{ nɤɲɟɯɲɟu}}}\markboth{nɤɲɟɯɲɟu}{}
\classe{vt}
\paradigme{\textit{dir :} \jya tɤ-}
\begin{définition}\fra appâter\end{définition}
\begin{définition}\cmn 引过来;吸引\end{définition}
\begin{exemple}\jya jla tɤ-nɤɲɟɯɲɟu-t-a\cmn 我把犏牛引过来了\end{exemple}
\begin{exemple}\jya daltsɯtsa tɤ-tɯt-a tɕe tɤ-nɤɲɟɯɲɟu-t-a\cmn 我慢慢地说,把他骗过来了\end{exemple}\begin{sous-entrée}
\vedette{\hypertarget{}{\papi{ znɤɲɟɯɲɟu}}}\markboth{znɤɲɟɯɲɟu}{}\classe{vt}
\paradigme{\textit{dir :} \jya tɤ-}
\begin{définition}\ 
\begin{déclaration}\grammar{caus}\end{déclaration}\end{définition}
\begin{définition}\fra attirer\end{définition}
\begin{définition}\cmn 用东西引过来\end{définition}
\begin{exemple}\jya jla tsha kɯ tɤ-znɤɲɟɯɲɟu-t-a\cmn 我用盐巴把犏牛引过来了\end{exemple}
\end{sous-entrée}\end{entrée}

\begin{entrée}
\vedette{\hypertarget{Ⓔnɤŋɤβ}{\papi{ nɤŋɤβ}}}\markboth{nɤŋɤβ}{}
\begin{relation-sémantique}\confer{
\hyperlink{Ⓔsɤŋɤβ}{\textit{ \papi{sɤŋɤβ}}}
}\end{relation-sémantique}\end{entrée}

\begin{entrée}
\vedette{\hypertarget{Ⓔnɤŋgɤr}{\papi{ nɤŋgɤr}}}\markboth{nɤŋgɤr}{}
\begin{relation-sémantique}\confer{
\hyperlink{Ⓔŋgɤr}{\textit{ \papi{ŋgɤr}}}
}\end{relation-sémantique}
\end{entrée}

\begin{entrée}
\vedette{\hypertarget{Ⓔnɤŋgiolo}{\papi{ nɤŋgiolo}}}\markboth{nɤŋgiolo}{}
\begin{relation-sémantique}\confer{
\hyperlink{Ⓔŋgio}{\textit{ \papi{ŋgio}}}
}\end{relation-sémantique}\end{entrée}

\begin{entrée}
\vedette{\hypertarget{Ⓔnɤŋgɯ}{\papi{ nɤŋgɯ}}}\markboth{nɤŋgɯ}{}\classe{vt}
\paradigme{\textit{dir :} \jya nɯ-}
\begin{définition}\fra emprunter\end{définition}
\begin{définition}\cmn 向别人借(不能归还原物)\end{définition}
\begin{exemple}\jya aʑo mɯ-ɲɯ-ɤro-a tɕe, nɤ-phe nɯ-nɤŋgɯ-t-a\end{exemple}
\begin{exemple}\jya aʑɯɣ maŋe tɕe, nɤj nɤ-phe nɯ-nɤŋgɯ-t-a\cmn 我没有,所以向你借了\end{exemple}
\begin{exemple}\jya a-pɯ-tɯ-ɤro tɕe nɤ-phe kɤ-nɤŋgɯ khɯ, a-mɤ-pɯ-tɯ-ɤro tɕe, nɤ-phe kɤ-nɤŋgɯ mɤ-khɯ\cmn 你有的话可以向你借,没有的话就不能借\end{exemple}
\begin{relation-sémantique}\confer{
\hyperlink{Ⓔrŋo}{\textit{ \papi{rŋo}}}
}\end{relation-sémantique}
\begin{relation-sémantique}\confer{
\hyperlink{Ⓔtɤŋgɯ}{\textit{ \papi{tɤŋgɯ}}}
}\end{relation-sémantique}\begin{sous-entrée}
\vedette{\hypertarget{}{\papi{ znɤŋgɯ}}}\markboth{znɤŋgɯ}{}\classe{vt}
\paradigme{\textit{dir :} \jya nɯ-}
\begin{définition}\fra prêter\end{définition}
\begin{définition}\cmn 借给别人(不能归还原物)\end{définition}
\begin{exemple}\jya (kɤndza, rŋɯl) nɯ-znɤŋgɯ-t-a\cmn 我借给他了(食物、钱)\end{exemple}
\begin{relation-sémantique}\confer{
\hyperlink{Ⓔɕɯrŋo}{\textit{ \papi{ɕɯrŋo}}}
}\end{relation-sémantique}
\end{sous-entrée}\end{entrée}

\begin{entrée}
\vedette{\hypertarget{Ⓔnɤŋgɯŋga}{\papi{ nɤŋgɯŋga}}}\markboth{nɤŋgɯŋga}{}
\begin{relation-sémantique}\confer{
\hyperlink{Ⓔŋga}{\textit{ \papi{ŋga}}}
}\end{relation-sémantique}\end{entrée}

\begin{entrée}
\vedette{\hypertarget{Ⓔnɤŋka}{\papi{ nɤŋka}}}\markboth{nɤŋka}{}
\classe{vt}
\paradigme{\textit{dir :} \jya nɯ-}
\begin{définition}\ 
\begin{déclaration}\grammar{denom}\end{déclaration}\end{définition}
\begin{définition}\fra ronger\end{définition}
\begin{définition}\cmn 嚼
\begin{déclaration}\use{表示“啃烂、咬破”等意思,必须用\stylefv{ndza}“吃”}\end{déclaration}\end{définition}
\begin{exemple}\jya kɯchi nɯ-nɤŋka-t-a\cmn 我嚼了糖\end{exemple}
\begin{exemple}\jya kɯchi na-nɤŋka\cmn 他嚼了糖\end{exemple}
\begin{exemple}\jya smɤn nɯ-nɤŋka-t-a\cmn 我嚼了药\end{exemple}
\begin{exemple}\jya ki ɯ-mdʑu ɲɤ-nɤŋka tɕe si ɲɯ-ŋu\cmn (这头牛)已经把舌头露出来了,快要死了\end{exemple}
\begin{relation-sémantique}\confer{
\hyperlink{Ⓔtɯ-ŋka}{\textit{ \papi{tɯ-ŋka}}}
}\end{relation-sémantique}\end{entrée}

\begin{entrée}
\vedette{\hypertarget{Ⓔnɤŋkhɯt}{\papi{ nɤŋkhɯt}}}\markboth{nɤŋkhɯt}{}\classe{vt}
\paradigme{\textit{dir :} \jya tɤ-}
\begin{définition}\ 
\begin{déclaration}\grammar{denom}\end{déclaration}\end{définition}
\begin{définition}\fra donner un coup de poing\end{définition}
\begin{définition}\cmn 用拳头捶打\end{définition}
\begin{relation-sémantique}\confer{
\hyperlink{Ⓔtɤŋkhɯt}{\textit{ \papi{tɤŋkhɯt}}}
}\end{relation-sémantique}\end{entrée}

\begin{entrée}
\vedette{\hypertarget{Ⓔnɤŋkɯŋke}{\papi{ nɤŋkɯŋke}}}\markboth{nɤŋkɯŋke}{}\classe{vi}
\paradigme{\textit{dir :} \jya \_}
\begin{définition}\ 
\begin{déclaration}\grammar{n.orient}\end{déclaration}\end{définition}
\begin{définition}\fra passer\end{définition}
\begin{définition}\cmn 走来走去\end{définition}
\begin{exemple}\jya aki @bazi ɯ-ŋgɯ ɲɯ-nɤŋkɯŋke\cmn 他在下面的坝子里走来走去\end{exemple}
\begin{exemple}\jya tshɯrɟɯn ɕ-tu-nɤŋkɯŋke-a ŋu ri, ɕɤxɕo kɯkɯra mɤʑɯ tʂu, kɤntɕhaʁ ra ku-oz-ɣɤβdi-nɯ tɕe, sɤŋke khro maŋe wo\cmn 我虽然经常去散步,这几天他们又在修路,路不好走\end{exemple}
\begin{relation-sémantique}\confer{
\hyperlink{Ⓔŋke}{\textit{ \papi{ŋke}}}
}\end{relation-sémantique}\end{entrée}

\begin{entrée}
\vedette{\hypertarget{Ⓔnɤŋɯr}{\papi{ nɤŋɯr}}}\markboth{nɤŋɯr}{}\classe{vt}
\paradigme{\textit{dir :} \jya nɯ-}
\begin{définition}\fra respecter\end{définition}
\begin{définition}\cmn 尊重\end{définition}
\begin{exemple}\jya ɲɯ́-wɣ-nɤŋɯr-a\cmn 他尊重我\end{exemple}
\begin{relation-sémantique}\synonyme{
\hyperlink{Ⓔnaʁre}{\textit{ \papi{naʁre}}}
}\end{relation-sémantique}\begin{sous-entrée}
\vedette{\hypertarget{}{\papi{ sɤŋɯr}}}\markboth{sɤŋɯr}{}\classe{vs}
\begin{définition}\fra être respecté\end{définition}
\begin{définition}\cmn 得到尊重\end{définition}
\begin{relation-sémantique}\synonyme{
\hyperlink{Ⓔsaʁre}{\textit{ \papi{saʁre}}}
}\end{relation-sémantique}
\end{sous-entrée}\end{entrée}

\begin{entrée}
\vedette{\hypertarget{Ⓔnɤɴɢiɤt}{\papi{ nɤɴɢiɤt}}}\markboth{nɤɴɢiɤt}{}
\begin{relation-sémantique}\confer{
\hyperlink{Ⓔɴɢiɤt}{\textit{ \papi{ɴɢiɤt}}}
}\end{relation-sémantique}\end{entrée}

\begin{entrée}
\vedette{\hypertarget{Ⓔnɤɴqa}{\papi{ nɤɴqa}}}\markboth{nɤɴqa}{}
\classe{vt}
\paradigme{\textit{dir :} \jya nɯ-}
\begin{définition}\ 
\begin{déclaration}\grammar{trop}\end{déclaration}\end{définition}
\begin{définition}\fra trouver le travail dur\end{définition}
\begin{définition}\cmn 觉得辛苦\end{définition}
\begin{exemple}\jya jiɕqha tɤton ɲɯ-nɤɴqe\cmn 他在上坡。觉得很辛苦\end{exemple}
\begin{exemple}\jya tɤton na-nɤɴqa\cmn 他在上坡。觉得很辛苦了\end{exemple}
\begin{exemple}\jya ɯ-phɯ ɲɯ-nɤɴqe\cmn 他觉得贵\end{exemple}
\begin{exemple}\jya @yingyu kɤ-βzjoz ɲɯ-nɤɴqe\cmn 他觉得学英语很难\end{exemple}
\begin{relation-sémantique}\confer{
\hyperlink{Ⓔɴqa}{\textit{ \papi{ɴqa}}}
}\end{relation-sémantique}\end{entrée}

\begin{entrée}
\vedette{\hypertarget{Ⓔnɤɴqhi}{\papi{ nɤɴqhi}}}\markboth{nɤɴqhi}{}
\begin{relation-sémantique}\confer{
\hyperlink{Ⓔɴqhi}{\textit{ \papi{ɴqhi}}}
}\end{relation-sémantique}\end{entrée}

\begin{entrée}
\vedette{\hypertarget{Ⓔnɤɴqi}{\papi{ nɤɴqi}}}\markboth{nɤɴqi}{}
\classe{vt}
\paradigme{\textit{dir :} \jya nɯ-}
\begin{définition}\fra avoir la flemme de, ne pas avoir envie de faire...\end{définition}
\begin{définition}\cmn 懒得……;怕麻烦\end{définition}
\begin{exemple}\jya ɯʑo kɯ ɲɯ-nɤɴqi tɕe ɯ-zda jo-sɯxɕe\cmn 他因为怕麻烦叫别人去了\end{exemple}
\begin{exemple}\jya ta-ma kɤ-nɤma nɯ-nɤɴqi-t-a\cmn 我变得很懒,不想劳动\end{exemple}
\begin{relation-sémantique}\confer{
\hyperlink{Ⓔnɯpɤɴqi}{\textit{ \papi{nɯpɤɴqi}}}
}\end{relation-sémantique}\end{entrée}

\begin{entrée}
\vedette{\hypertarget{Ⓔnɤpɤβdaʁ}{\papi{ nɤpɤβdaʁ}}}\markboth{nɤpɤβdaʁ}{} (\variante{nɤpɤdaʁ}) 
\classe{vt}
\paradigme{\textit{dir :} \jya tɤ-}
\begin{définition}\fra s'occuper de (enfant)\end{définition}
\begin{définition}\cmn 抚养;照顾(孩子)\end{définition}
\begin{exemple}\jya tɤ-pɤtso tɤ-nɤpɤβdaʁ-a\cmn 我抚养了孩子\end{exemple}
\begin{relation-sémantique}\synonyme{
\hyperlink{Ⓔsɤβlo}{\textit{ \papi{sɤβlo}}}
}\end{relation-sémantique}
\end{entrée}

\begin{entrée}
\vedette{\hypertarget{Ⓔnɤpɤmbat}{\papi{ nɤpɤmbat}}}\markboth{nɤpɤmbat}{}
\classe{vi}
\paradigme{\textit{dir :} \jya tɤ-}
\begin{définition}\ 
\begin{déclaration}\grammar{comp}\end{déclaration}\end{définition}
\begin{définition}\fra faire n'importe comment\end{définition}
\begin{définition}\cmn 做得很粗糙;敷衍了事\end{définition}
\begin{exemple}\jya tɤ-nɤpɤmba-t-a\cmn 我将就做了\end{exemple}
\begin{exemple}\jya to-nɤpɤmbat\cmn 他将就做了\end{exemple}
\begin{exemple}\jya aj jisŋi tɯ-ŋga ci thɯ-tʂɯβ-a ri, tɤ-nɤpɤmba-t-a\cmn 我今天缝了一件衣服,缝得很粗糙\end{exemple}
\begin{exemple}\jya ta-ma kɤ-nɤpɤmbat mɤ-ra\cmn 工作不要敷衍了事\end{exemple}
\begin{relation-sémantique}\confer{
\hyperlink{ⒺpaⒽ1}{\textit{ \papi{pa1}}}
}\end{relation-sémantique}
\begin{relation-sémantique}\confer{
\hyperlink{Ⓔmbat}{\textit{ \papi{mbat}}}
}\end{relation-sémantique}\end{entrée}

\begin{entrée}
\vedette{\hypertarget{Ⓔnɤpɤru}{\papi{ nɤpɤru}}}\markboth{nɤpɤru}{}\classe{vt}
\paradigme{\textit{dir :} \jya tɤ-}
\begin{définition}\fra garder\end{définition}
\begin{définition}\cmn 保管;管理\end{définition}
\begin{exemple}\jya aʑo jiɕqha laχtɕha nɯ tɤ-nɤpɤru-t-a\cmn 我把这个东西保管(好)了\end{exemple}
\begin{exemple}\jya tɤ-rɤku lɤ́-wɣ-ji tɕe ɣɯ-nɤpɤru ra\cmn 种庄稼的时候一定要有人管理\end{exemple}
\begin{relation-sémantique}\confer{
\hyperlink{ⒺruⒽ1}{\textit{ \papi{ru1}}}
}\end{relation-sémantique}\end{entrée}

\begin{entrée}
\vedette{\hypertarget{Ⓔnɤpɤri}{\papi{ nɤpɤri}}}\markboth{nɤpɤri}{}
\classe{vi}
\paradigme{\textit{dir :} \jya tɤ-}
\begin{définition}\ 
\begin{déclaration}\grammar{denom}\end{déclaration}\end{définition}
\begin{définition}\fra diner\end{définition}
\begin{définition}\cmn 吃晚饭\end{définition}
\begin{exemple}\jya tɤ-nɤpɤri-a\cmn 我已经吃了晚餐\end{exemple}
\begin{exemple}\jya jɯɣmɯr kɤ-nɯ-rma tɕe, tɯ-rca nɤpɤri-j\cmn 你今天晚上留宿,我们一起吃晚餐\end{exemple}
\begin{relation-sémantique}\confer{
\hyperlink{Ⓔtɤ-pɤri}{\textit{ \papi{tɤ-pɤri}}}
}\end{relation-sémantique}\begin{sous-entrée}
\vedette{\hypertarget{}{\papi{ znɤpɤri}}}\markboth{znɤpɤri}{}\classe{vt}
\paradigme{\textit{dir :} \jya tɤ-}
\begin{définition}\fra donner un diner\end{définition}
\begin{définition}\cmn 给……吃晚餐\end{définition}
\end{sous-entrée}\end{entrée}

\begin{entrée}
\vedette{\hypertarget{Ⓔnɤpe}{\papi{ nɤpe}}}\markboth{nɤpe}{}
\begin{relation-sémantique}\confer{
\hyperlink{Ⓔpe}{\textit{ \papi{pe}}}
}\end{relation-sémantique}\end{entrée}

\begin{entrée}
\vedette{\hypertarget{Ⓔnɤphɤtphɤt}{\papi{ nɤphɤtphɤt}}}\markboth{nɤphɤtphɤt}{}\classe{vt}
\paradigme{\textit{dir :} \jya kɤ-}
\begin{définition}\fra tapoter\end{définition}
\begin{définition}\cmn (轻轻地)拍打\end{définition}
\begin{exemple}\jya tɤ-pɤtso kɤ-nɤphɤtphat-a kɤ-znɯʑɯβ-a\cmn 我拍打小孩子让他睡着了\end{exemple}\begin{sous-entrée}
\vedette{\hypertarget{}{\papi{ znɤphɤtphɤt}}}\markboth{znɤphɤtphɤt}{}\classe{vt}
\paradigme{\textit{dir :} \jya kɤ-}
\begin{définition}\fra tapoter avec\end{définition}
\begin{définition}\cmn 用……(轻轻地)拍打\end{définition}
\end{sous-entrée}\end{entrée}

\begin{entrée}
\vedette{\hypertarget{Ⓔnɤphɯphɣo}{\papi{ nɤphɯphɣo}}}\markboth{nɤphɯphɣo}{}
\classe{vi}
\paradigme{\textit{dir :} \jya \_}
\begin{définition}\ 
\begin{déclaration}\grammar{n.orient}\end{déclaration}\end{définition}
\begin{définition}\fra fuir dans tous les sens\end{définition}
\begin{définition}\cmn 逃来逃去;到处乱跑\end{définition}
\begin{exemple}\jya aʑo pɯ-nɤphɯphɣo-a ntsɯ\cmn 我到处乱跑了\end{exemple}
\begin{relation-sémantique}\confer{
\hyperlink{Ⓔphɣo}{\textit{ \papi{phɣo}}}
}\end{relation-sémantique}\end{entrée}

\begin{entrée}
\vedette{\hypertarget{Ⓔnɤphɯphɯ}{\papi{ nɤphɯphɯ}}}\markboth{nɤphɯphɯ}{}\classe{vi}
\paradigme{\textit{dir :} \jya thɯ-}
\begin{définition}\fra mendier\end{définition}
\begin{définition}\cmn 乞讨【讨口】\end{définition}
\begin{exemple}\jya ɯʑo thɯ-nɤphɯphɯ\cmn 他乞讨了\end{exemple}\begin{sous-entrée}
\vedette{\hypertarget{}{\papi{ kɯ-nɤphɯphɯ}}}\markboth{kɯ-nɤphɯphɯ}{}
\begin{définition}\fra mendiant\end{définition}
\begin{définition}\cmn 乞丐\end{définition}
\end{sous-entrée}\end{entrée}

\begin{entrée}
\vedette{\hypertarget{Ⓔnɤphɯxtsɯ}{\papi{ nɤphɯxtsɯ}}}\markboth{nɤphɯxtsɯ}{}
\classe{vi}
\paradigme{\textit{dir :} \jya thɯ-}
\paradigme{\textit{dir :} \jya pɯ-}
\begin{définition}\ 
\begin{déclaration}\grammar{incorp}\end{déclaration}\end{définition}
\begin{définition}\fra écraser les mottes de terre\end{définition}
\begin{définition}\cmn 打土巴\end{définition}
\begin{exemple}\jya aʑo pɯ-nɤphɯxtsɯ-a\cmn 我打了土巴\end{exemple}
\begin{relation-sémantique}\confer{
\hyperlink{Ⓔtɤphɯxtsɯ}{\textit{ \papi{tɤphɯxtsɯ}}}
}\end{relation-sémantique}\end{entrée}

\begin{entrée}
\vedette{\hypertarget{Ⓔnɤprɯ}{\papi{ nɤprɯ}}}\markboth{nɤprɯ}{}\classe{vi}
\paradigme{\textit{dir :} \jya kɤ-}
\begin{définition}\fra se protéger de la pluie\end{définition}
\begin{définition}\cmn 遮雨
\begin{déclaration}\grammar{denom}\end{déclaration}\end{définition}
\begin{exemple}\jya @san ɯ-pa nɤprɯ-tɕi\cmn 我们俩用伞遮雨\end{exemple}
\begin{exemple}\jya praχpa nɤprɯ-tɕi\cmn 我们俩在悬崖下遮雨\end{exemple}
\begin{exemple}\jya jɤɣɤt ɯ-pa kɤ-nɤprɯ-a\cmn 我在走缘下遮雨了\end{exemple}
\begin{exemple}\jya tɯ-mɯ ɲɯ-ɤsɯ-lɤt tɕe, kɤ-nɤprɯ-a\cmn 下雨的时候,我躲雨了\end{exemple}
\begin{relation-sémantique}\confer{
\hyperlink{Ⓔtɤprɯ}{\textit{ \papi{tɤprɯ}}}
}\end{relation-sémantique}\end{entrée}

\begin{entrée}
\vedette{\hypertarget{Ⓔnɤpɯpa}{\papi{ nɤpɯpa}}}\markboth{nɤpɯpa}{}\classe{vt}
\paradigme{\textit{dir :} \jya tɤ-}
\begin{définition}\fra s'occuper de\end{définition}
\begin{définition}\cmn 照顾\end{définition}
\begin{relation-sémantique}\synonyme{
\hyperlink{Ⓔnɯβdaʁ}{\textit{ \papi{nɯβdaʁ}}}
}\end{relation-sémantique}\end{entrée}

\begin{entrée}
\vedette{\hypertarget{Ⓔnɤpɯprɤt}{\papi{ nɤpɯprɤt}}}\markboth{nɤpɯprɤt}{}
\begin{relation-sémantique}\confer{
\hyperlink{Ⓔprɤt}{\textit{ \papi{prɤt}}}
}\end{relation-sémantique}\end{entrée}

\begin{entrée}
\vedette{\hypertarget{Ⓔnɤqa}{\papi{ nɤqa}}}\markboth{nɤqa}{}
\classe{vt}
\paradigme{\textit{dir :} \jya thɯ-}
\begin{définition}\ 
\begin{déclaration}\grammar{denom}\end{déclaration}\end{définition}
\begin{définition}\fra arracher à la racine\end{définition}
\begin{définition}\cmn 拔根\end{définition}
\begin{exemple}\jya thɯ-nɤqa-t-a\cmn 我拔了根\end{exemple}
\begin{exemple}\jya ki sɯjno thɯ-nɤqa-t-a\cmn 我拔了这个草\end{exemple}
\begin{relation-sémantique}\confer{
\hyperlink{Ⓔtɯ-qa}{\textit{ \papi{tɯ-qa}}}
}\end{relation-sémantique}\end{entrée}

\begin{entrée}
\vedette{\hypertarget{Ⓔnɤqadrɤt}{\papi{ nɤqadrɤt}}}\markboth{nɤqadrɤt}{}\classe{vt}
\paradigme{\textit{dir :} \ }
\begin{définition}\ \begin{déclaration}\grammar{incorp}\end{déclaration}\end{définition}
\begin{définition}\fra griffer partout\end{définition}
\begin{définition}\cmn 到处乱抓(鸟)\end{définition}
\begin{exemple}\jya kumpɣa kɯ ɯ-ndza cho qajɯ ɲɯ-ɕar tɕe, sɤtɕha aʁɤndɯndɤt ʑo chɯ-nɤqadrɤt ŋu\cmn 鸡在找食物和虫子的时候,会到到处乱抓地面\end{exemple}
\begin{relation-sémantique}\confer{
\hyperlink{Ⓔadrɤt}{\textit{ \papi{adrɤt}}}
}\end{relation-sémantique}
\begin{relation-sémantique}\confer{
\hyperlink{Ⓔtɯ-qa}{\textit{ \papi{tɯ-qa}}}
}\end{relation-sémantique}\end{entrée}

\begin{entrée}
\vedette{\hypertarget{Ⓔnɤqɤrqɤr}{\papi{ nɤqɤrqɤr}}}\markboth{nɤqɤrqɤr}{}
\begin{relation-sémantique}\confer{
\hyperlink{Ⓔqɤr}{\textit{ \papi{qɤr}}}
}\end{relation-sémantique}\end{entrée}

\begin{entrée}
\vedette{\hypertarget{Ⓔnɤqɤtsa}{\papi{ nɤqɤtsa}}}\markboth{nɤqɤtsa}{}\classe{vt}
\paradigme{\textit{dir :} \jya tɤ-}
\begin{définition}\ 
\begin{déclaration}\grammar{trop}\end{déclaration}\end{définition}
\begin{définition}\fra adéquat\end{définition}
\begin{définition}\cmn 合适\end{définition}
\begin{exemple}\jya kɯki rɟɯma ki nɯtɕu tú-wɣ-rku ɲɯ-jɤɣ ma ɲɯ-nɤqɤtse\cmn 这个螺丝可以装在那里,因为刚合适\end{exemple}\begin{sous-entrée}
\vedette{\hypertarget{}{\papi{ anɤqɤtsɯtsa}}}\markboth{anɤqɤtsɯtsa}{}\classe{vi}
\begin{définition}\ 
\begin{déclaration}\grammar{recip}\end{déclaration}\end{définition}
\begin{définition}\fra adéquat\end{définition}
\begin{définition}\cmn 合适;相配\end{définition}
\begin{exemple}\jya ɯ-fkɯm cho ɯ-ŋgɯ nɯ ɲɯ-ɤnɤqɤtsɯtsa-ndʑi\cmn 口袋和内容刚刚合适\end{exemple}
\begin{relation-sémantique}\confer{
\hyperlink{Ⓔaqɤtsa}{\textit{ \papi{aqɤtsa}}}
}\end{relation-sémantique}
\end{sous-entrée}\end{entrée}

\begin{entrée}
\vedette{\hypertarget{Ⓔnɤqɤʑa}{\papi{ nɤqɤʑa}}}\markboth{nɤqɤʑa}{}
\classe{vt}
\paradigme{\textit{dir :} \jya \_}
\begin{définition}\ 
\begin{déclaration}\grammar{incorp}\end{déclaration}\end{définition}
\begin{définition}\fra faire complètement du début à la fin\end{définition}
\begin{définition}\cmn 从头到尾地做\end{définition}
\begin{exemple}\jya kha tɤ-nɤqɤʑa-t-a ʑo tɤ-βzu-t-a ɕti\end{exemple}
\begin{exemple}\jya aj kha tɤ-nɤqɤʑa-t-a ɕti\cmn 这个房子是我一手修建的\end{exemple}
\begin{exemple}\jya tɤ-pɤtso kɤ-sɯxɕɤt pɯ-nɤqɤʑa-t-a pɯ-ra\cmn 我只好从头到尾教了这个小孩子\end{exemple}
\begin{relation-sémantique}\confer{
\hyperlink{Ⓔtɯ-qa}{\textit{ \papi{tɯ-qa}}}
}\end{relation-sémantique}
\begin{relation-sémantique}\confer{
\hyperlink{ⒺʑaⒽ1}{\textit{ \papi{ʑa1}}}
}\end{relation-sémantique}\end{entrée}

\begin{entrée}
\vedette{\hypertarget{Ⓔnɤqharu}{\papi{ nɤqharu}}}\markboth{nɤqharu}{}
\classe{vi}
\paradigme{\textit{dir :} \jya \_}
\begin{définition}\ 
\begin{déclaration}\grammar{incorp}\end{déclaration}\end{définition}
\begin{définition}\fra se retourner\end{définition}
\begin{définition}\cmn 转身;回头\end{définition}
\begin{exemple}\jya a-qhu chu lɤ-tɯ-ɣe tɕe thɯ-nɤqharu-a\cmn 你从我后面来了,我回头看了一下\end{exemple}
\begin{relation-sémantique}\confer{
\hyperlink{Ⓔɯ-qhu}{\textit{ \papi{ɯ-qhu}}}
}\end{relation-sémantique}
\begin{relation-sémantique}\confer{
\hyperlink{ⒺruⒽ1}{\textit{ \papi{ru1}}}
}\end{relation-sémantique}
\begin{relation-sémantique}\confer{
\hyperlink{Ⓔqharu}{\textit{ \papi{qharu}}}
}\end{relation-sémantique}\end{entrée}

\begin{entrée}
\vedette{\hypertarget{Ⓔnɤqhɤβde}{\papi{ nɤqhɤβde}}}\markboth{nɤqhɤβde}{}
\classe{vt}
\paradigme{\textit{dir :} \jya nɯ-}
\begin{définition}\ 
\begin{déclaration}\grammar{incorp}\end{déclaration}\end{définition}\acception{1}
\begin{définition}\fra reporter à plus tard\end{définition}
\begin{définition}\cmn 拖到以后\end{définition}
\begin{exemple}\jya nɤ-kɤnɤma ra ɲɯ-tɯ-nɤqhɤβde ntsɯ ɲɯ-ŋu\cmn 你总是把事情拖到以后干\end{exemple}\acception{2}
\begin{définition}\fra négliger\end{définition}
\begin{définition}\cmn 不管\end{définition}
\begin{exemple}\jya tɤ-pɤtso kɤ-nɤqhɤβde mɤ-khɯ\cmn 不能忽视小孩子\end{exemple}
\begin{relation-sémantique}\confer{
\hyperlink{Ⓔɯ-qhu}{\textit{ \papi{ɯ-qhu}}}
}\end{relation-sémantique}
\begin{relation-sémantique}\confer{
\hyperlink{Ⓔβde}{\textit{ \papi{βde}}}
}\end{relation-sémantique}\end{entrée}

\begin{entrée}
\vedette{\hypertarget{Ⓔnɤqhɤŋga}{\papi{ nɤqhɤŋga}}}\markboth{nɤqhɤŋga}{}
\classe{vt}
\paradigme{\textit{dir :} \jya thɯ-}
\paradigme{\textit{dir :} \jya kɤ-}
\begin{définition}\ 
\begin{déclaration}\grammar{incorp}\end{déclaration}\end{définition}
\begin{définition}\fra se mettre un habit sur les épaules\end{définition}
\begin{définition}\cmn 披衣\end{définition}
\begin{exemple}\jya a-ŋga thɯ-nɤqhɤŋga-t-a\cmn 我披上了衣服\end{exemple}
\begin{relation-sémantique}\confer{
\hyperlink{Ⓔɯ-qhu}{\textit{ \papi{ɯ-qhu}}}
}\end{relation-sémantique}
\begin{relation-sémantique}\confer{
\hyperlink{Ⓔŋga}{\textit{ \papi{ŋga}}}
}\end{relation-sémantique}\end{entrée}

\begin{entrée}
\vedette{\hypertarget{Ⓔnɤqhɤwɯr}{\papi{ nɤqhɤwɯr}}}\markboth{nɤqhɤwɯr}{}
\classe{vt}
\paradigme{\textit{dir :} \jya thɯ-}
\paradigme{\textit{dir :} \jya tɤ-}
\begin{définition}\ 
\begin{déclaration}\grammar{denom}\end{déclaration}\end{définition}
\begin{définition}\fra se mettre un habit sur les épaules pour se protéger de la pluie\end{définition}
\begin{définition}\cmn 披衣挡雨\end{définition}
\begin{exemple}\jya @pugai thɯ-nɤqhɤwɯr-a\cmn 我披上了铺盖(被子)挡雨\end{exemple}
\begin{exemple}\jya tɯ-ŋga thɯ-nɤqhɤwɯr-a\cmn 我披上了衣服\end{exemple}
\begin{exemple}\jya ɯʑo kɯ tɯ-ŋga tha-nɤqhɤwɯr\cmn 他披上了衣服\end{exemple}
\begin{relation-sémantique}\confer{
\hyperlink{Ⓔtɯwɯr}{\textit{ \papi{tɯwɯr}}}
}\end{relation-sémantique}\end{entrée}

\begin{entrée}
\vedette{\hypertarget{Ⓔnɤqhrɯmbɤβ}{\papi{ nɤqhrɯmbɤβ}}}\markboth{nɤqhrɯmbɤβ}{}
\classe{vi}
\paradigme{\textit{dir :} \jya tɤ-}
\begin{définition}\fra roter\end{définition}
\begin{définition}\cmn 打饱嗝\end{définition}
\begin{exemple}\jya tɤ-nɤqhrɯmbɤβ\cmn 他打嗝了\end{exemple}
\begin{relation-sémantique}\confer{
\hyperlink{Ⓔtɯ-qhrɯmbɤβ}{\textit{ \papi{tɯ-qhrɯmbɤβ}}}
}\end{relation-sémantique}\end{entrée}

\begin{entrée}
\vedette{\hypertarget{Ⓔnɤru}{\papi{ nɤru}}}\markboth{nɤru}{}
\classe{vi}
\paradigme{\textit{dir :} \jya kɤ-}
\begin{définition}\fra voler de la nourriture (animaux)\end{définition}
\begin{définition}\cmn 偷吃粮食(牲畜)\end{définition}
\begin{exemple}\jya tshɤt ɲɯ-nɤru\cmn 山羊在偷吃\end{exemple}
\begin{exemple}\jya nɯŋa ɲɯ-nɤru\cmn 牛在偷吃\end{exemple}
\begin{exemple}\jya paʁ ɲɯ-nɤru\cmn 猪在偷吃\end{exemple}
\begin{exemple}\jya tshɤt ko-nɤru\cmn 山羊偷吃了\end{exemple}\end{entrée}

\begin{entrée}
\vedette{\hypertarget{Ⓔnɤrɤɟaʁ}{\papi{ nɤrɤɟaʁ}}}\markboth{nɤrɤɟaʁ}{}
\classe{vi}
\paradigme{\textit{dir :} \jya thɯ-}
\begin{définition}\fra échanger des plaisanteries\end{définition}
\begin{définition}\cmn 说说笑笑\end{définition}
\begin{relation-sémantique}\confer{
\hyperlink{Ⓔtɤre tɤɟaʁ}{\textit{ \papi{tɤre tɤɟaʁ}}}
}\end{relation-sémantique}\end{entrée}

\begin{entrée}
\vedette{\hypertarget{Ⓔnɤrɕu}{\papi{ nɤrɕu}}}\markboth{nɤrɕu}{}
\classe{vi}
\paradigme{\textit{dir :} \jya nɯ-}
\begin{définition}\fra égratigner\end{définition}
\begin{définition}\cmn 皮肤擦破\end{définition}
\begin{exemple}\jya a-jaʁ ɲo-nɤrɕu\cmn 我的手擦破了皮\end{exemple}\begin{sous-entrée}
\vedette{\hypertarget{}{\papi{ znɤrɕu}}}\markboth{znɤrɕu}{}\classe{vt}
\paradigme{\textit{dir :} \jya nɯ-}
\begin{définition}\ 
\begin{déclaration}\grammar{caus}\end{déclaration}\end{définition}
\begin{exemple}\jya a-jaʁ nɯ-tɯ-znɤrɕu-t\cmn 你擦破了我的手\end{exemple}
\end{sous-entrée}\end{entrée}

\begin{entrée}
\vedette{\hypertarget{Ⓔnɤrɕɤmŋɤm}{\papi{ nɤrɕɤmŋɤm}}}\markboth{nɤrɕɤmŋɤm}{}
\classe{vt}
\paradigme{\textit{dir :} \jya nɯ-}
\begin{définition}\ 
\begin{déclaration}\grammar{incorp}\end{déclaration}\end{définition}
\begin{définition}\fra chérir\end{définition}
\begin{définition}\cmn 疼爱\end{définition}
\begin{exemple}\jya a-ʁi ɲɯ-nɤrɕɤmŋam-a\cmn 我很疼爱我的弟弟\end{exemple}
\begin{exemple}\jya ɯ-rɕa,mŋɤm\end{exemple}\end{entrée}

\begin{entrée}
\vedette{\hypertarget{ⒺnɤreⒽ1Ⓗ1}{\papi{ nɤre}}}\markboth{nɤre}{}\homonyme{1}\classe{vi}
\paradigme{\textit{dir :} \jya nɯ-}\acception{1}
\begin{définition}\fra rire\end{définition}
\begin{définition}\cmn 笑\end{définition}
\begin{exemple}\jya nɯ-kɯ-nɤre-a\cmn 你笑了我\end{exemple}
\begin{exemple}\jya nɯ-nɤre-a\cmn 我笑了\end{exemple}\acception{2}
\begin{définition}\fra se réouvrir (blessure)\end{définition}
\begin{définition}\cmn 重复开裂(伤口)\end{définition}
\begin{exemple}\jya ɯ-tɯ-ɣmaz ɲɤ-nɤre\cmn 他的伤口复裂了\end{exemple}\begin{sous-entrée}
\vedette{\hypertarget{}{\papi{ anɤrɯre}}}\markboth{anɤrɯre}{}\classe{vi}
\begin{définition}\fra se moquer les uns des autres\end{définition}
\begin{définition}\cmn 互相取笑\end{définition}
\begin{exemple}\jya ɲɯ-ɤnɤrɯre-ndʑi\cmn 他们俩互相取笑\end{exemple}
\end{sous-entrée}\begin{sous-entrée}
\vedette{\hypertarget{}{\papi{ nɤre}}}\markboth{nɤre}{}\classe{vt}
\paradigme{\textit{dir :} \jya nɯ-}
\begin{définition}\fra se moquer de\end{définition}
\begin{définition}\cmn 笑\end{définition}
\begin{exemple}\jya nɯ-tɯ-nɤre-t\cmn 你笑了他\end{exemple}
\begin{exemple}\jya nɯ-nɤre-t-a\cmn 我笑了他\end{exemple}
\end{sous-entrée}\begin{sous-entrée}
\vedette{\hypertarget{}{\papi{ sɤnɤre}}}\markboth{sɤnɤre}{}\classe{vi}
\begin{définition}\ 
\begin{déclaration}\grammar{apass}\end{déclaration}\end{définition}
\begin{définition}\fra se moquer des gens\end{définition}
\begin{définition}\cmn 取笑人\end{définition}
\end{sous-entrée}\begin{sous-entrée}
\vedette{\hypertarget{}{\papi{ znɤre}}}\markboth{znɤre}{}\classe{vt}
\begin{définition}\fra faire rire\end{définition}
\begin{définition}\cmn 笑令人笑\end{définition}
\end{sous-entrée}\end{entrée}

\begin{entrée}
\vedette{\hypertarget{Ⓔnɤrgɤŋɯ}{\papi{ nɤrgɤŋɯ}}}\markboth{nɤrgɤŋɯ}{}\classe{vi}
\paradigme{\textit{dir :} \jya nɯ-}
\begin{définition}\fra pleurer de bonheur\end{définition}
\begin{définition}\cmn 高兴得哭
\begin{déclaration} \étymologie{\papi{rga.ŋu}}\end{déclaration}\end{définition}
\begin{exemple}\jya aʑo jɤ-azɣɯt-a tɕe, a-mu ɲɯ-nɤrgɤŋɯ\cmn 我到了的时候,我目前高兴得哭了出来\end{exemple}\end{entrée}

\begin{entrée}
\vedette{\hypertarget{Ⓔnɤrɣɤma}{\papi{ nɤrɣɤma}}}\markboth{nɤrɣɤma}{}\classe{vi}
\paradigme{\textit{dir :} \jya tɤ-}
\begin{définition}\fra implorer la pluie\end{définition}
\begin{définition}\cmn 求雨\end{définition}
\begin{exemple}\jya aʑo ɕ-tɤ-nɤrɣɤma-a\cmn 我去求雨了\end{exemple}\end{entrée}

\begin{entrée}
\vedette{\hypertarget{Ⓔnɤrɟɯrɟɯɣ}{\papi{ nɤrɟɯrɟɯɣ}}}\markboth{nɤrɟɯrɟɯɣ}{}
\classe{vi}
\paradigme{\textit{dir :} \jya \_}
\begin{définition}\ 
\begin{déclaration}\grammar{n.orient}\end{déclaration}\end{définition}
\begin{définition}\fra courir dans tous les sens\end{définition}
\begin{définition}\cmn 跑来跑去\end{définition}
\begin{exemple}\jya aj tɤ-nɤrɟɯrɟɯɣ-a\cmn 我跑来跑去了\end{exemple}
\begin{exemple}\jya ɯ-kɯ-rtoʁ tɤ-nɤrɟɯrɟɯɣ-a\cmn 我到处跑去看他了\end{exemple}
\begin{relation-sémantique}\confer{
\hyperlink{ⒺrɟɯɣⒽ1}{\textit{ \papi{rɟɯɣ1}}}
}\end{relation-sémantique}\end{entrée}

\begin{entrée}
\vedette{\hypertarget{Ⓔnɤrkɤja}{\papi{ nɤrkɤja}}}\markboth{nɤrkɤja}{}\classe{vt}
\paradigme{\textit{dir :} \jya \_}
\begin{définition}\fra s'occuper des animaux\end{définition}
\begin{définition}\cmn 管理牲畜(赶、放牧)\end{définition}
\begin{exemple}\jya fsapaʁ ku-nɤrkɤje-a\cmn 我在管理牲畜\end{exemple}
\end{entrée}

\begin{entrée}
\vedette{\hypertarget{Ⓔnɤrkhɯrkhɯβ}{\papi{ nɤrkhɯrkhɯβ}}}\markboth{nɤrkhɯrkhɯβ}{} (\variante{\_nɤrkhɯβrkhɯβ}) 
\classe{vt}
\begin{définition}\ 
\begin{déclaration}\grammar{deidph}\end{déclaration}\end{définition}\acception{1}
\paradigme{\textit{dir :} \jya kɤ-}
\paradigme{\textit{dir :} \jya nɯ-}
\begin{définition}\fra frapper à la porte\end{définition}
\begin{définition}\cmn 敲门\end{définition}
\begin{exemple}\jya kɯm kɤ-nɤrkhɯrkhɯβ-a\cmn 我敲了门\end{exemple}
\begin{exemple}\jya na-nɤrkhɯrkhɯβ\cmn 他敲了门\end{exemple}
\begin{exemple}\jya kɯm ɲɤ-nɤrkhɯrkhɯβ\cmn 他敲了门\end{exemple}\acception{2}
\paradigme{\textit{dir :} \jya lɤ-}
\begin{définition}\fra frapper en faisant du bruit\end{définition}
\begin{définition}\cmn 敲响\end{définition}
\begin{exemple}\jya ɯʑo kɯ kɯm la-nɤrkhɯβrkhɯβ tɕe ɯ-zgra ta-ʑmbri tɕe pɯ-mtsham-a\cmn 他敲了门,发出了声音我就听到了\end{exemple}\begin{sous-entrée}
\vedette{\hypertarget{}{\papi{ sɤrkhɯβrkhɯβ}}}\markboth{sɤrkhɯβrkhɯβ}{}\classe{vt}
\paradigme{\textit{dir :} \jya tɤ-}
\begin{relation-sémantique}\confer{
\hyperlink{Ⓔrkhɯβrkhɯβ}{\textit{ \papi{rkhɯβrkhɯβ}}}
}\end{relation-sémantique}
\end{sous-entrée}\end{entrée}

\begin{entrée}
\vedette{\hypertarget{ⒺnɤrkoⒽ1}{\papi{ nɤrko}}}\markboth{nɤrko}{}\homonyme{1}\classe{vt}
\paradigme{\textit{dir :} \jya tɤ-}
\begin{définition}\ 
\begin{déclaration}\grammar{trop}\end{déclaration}\end{définition}
\begin{définition}\fra tenir, supporter\end{définition}
\begin{définition}\cmn 坚持\end{définition}
\begin{exemple}\jya tɤ-nɤrko-t-a\cmn 我坚持了\end{exemple}
\begin{exemple}\jya aj ɲɯ-ngo-a ri tɤ-nɤrko-t-a\cmn 我病了但是坚持了\end{exemple}
\begin{exemple}\jya nɤ-ndzɤtshi tɤ-nɤrkɤm\cmn 你坚持吃吧\end{exemple}
\begin{exemple}\jya nɤ-ndzɤtshi tɤ-nɤrkɤm je tɕe ʑa a-tɤ-tɯ-mna\cmn 你坚持吃,希望早日康复\end{exemple}
\begin{exemple}\jya mɤʑɯ laʁnɤ-rʑaʁ pɯ-ri tɕe tɤ-nɤrkɤm je\cmn 只剩下几天,一定要坚持到底!\end{exemple}
\begin{exemple}\jya nɤʑo nɯ sthɯci kɯ-ɤrqhi ju-tɯ-ɣi ɕti tɕe, aʑo ku-rɤʑe-a ma mɯ́j-ra tɕe, tu-nɤrkam-a\cmn 你从那么远的地方来,我这里也没有事做,我会坚持的\end{exemple}
\begin{exemple}\jya kɤ-nɤrko me\cmn 只有那么点能力,已经尽力了\end{exemple}
\begin{exemple}\jya tɤ-nɤrko-t-a tɕe tɤ-ndza-t-a pɯ-ra (= tɤrkoz ʑo tɤ-ndza-t-a pɯ-ra)\cmn 我被迫吃了\end{exemple}\begin{sous-entrée}
\vedette{\hypertarget{}{\papi{ znɤrko}}}\markboth{znɤrko}{}\classe{vt}
\paradigme{\textit{dir :} \jya tɤ-}\acception{1}
\begin{définition}\ 
\begin{déclaration}\grammar{caus}\end{déclaration}\end{définition}
\begin{définition}\fra forcer\end{définition}
\begin{définition}\cmn 强迫
\begin{déclaration}\grammar{habil}\end{déclaration}\end{définition}
\begin{exemple}\jya kɤ-ndza ta-znɤrko\cmn 他强迫他吃了\end{exemple}\acception{2}
\begin{définition}\fra pouvoir tenir, supporter\end{définition}
\begin{définition}\cmn 能坚持\end{définition}
\begin{exemple}\jya ɯ-kɯ-mŋɤm ɲɯ-thɯ ri, wuma ʑo ɲɯ-znɤrkɤm\cmn 他虽然病得很严重,但是还是坚持得住\end{exemple}
\begin{exemple}\jya kɤ-rɤma wuma ɲɯ-znɤrkɤm\cmn 他坚持工作\end{exemple}
\end{sous-entrée}\begin{sous-entrée}
\vedette{\hypertarget{}{\papi{ ʑɣɤnɤrko}}}\markboth{ʑɣɤnɤrko}{}\classe{vi}
\paradigme{\textit{dir :} \jya tɤ-}
\begin{définition}\ 
\begin{déclaration}\grammar{refl}\end{déclaration}\end{définition}
\begin{définition}\fra se forcer\end{définition}
\begin{définition}\cmn 强迫自己\end{définition}
\begin{exemple}\jya kɤ-ndza a-tɤ-tɯ-ʑɣɤnɤrko\cmn 你强迫自己吃\end{exemple}
\begin{relation-sémantique}\confer{
\hyperlink{Ⓔrko}{\textit{ \papi{rko}}}
}\end{relation-sémantique}
\begin{relation-sémantique}\confer{
\hyperlink{Ⓔanɤrkɯrko}{\textit{ \papi{anɤrkɯrko}}}
}\end{relation-sémantique}
\end{sous-entrée}\end{entrée}

\begin{entrée}
\vedette{\hypertarget{ⒺnɤrkoⒽ2}{\papi{ nɤrko}}}\markboth{nɤrko}{}\homonyme{2} (\variante{nɤrkɯrko}) \classe{vs}
\begin{définition}\fra résistant\end{définition}
\begin{définition}\cmn 结实;硬;不容易变形
\end{définition}
\begin{relation-sémantique}\confer{
\hyperlink{Ⓔrko}{\textit{ \papi{rko}}}
}\end{relation-sémantique}\end{entrée}

\begin{entrée}
\vedette{\hypertarget{Ⓔnɤrkɯn}{\papi{ nɤrkɯn}}}\markboth{nɤrkɯn}{}
\begin{relation-sémantique}\confer{
\hyperlink{Ⓔrkɯn}{\textit{ \papi{rkɯn}}}
}\end{relation-sémantique}\end{entrée}

\begin{entrée}
\vedette{\hypertarget{Ⓔnɤrkɯrku}{\papi{ nɤrkɯrku}}}\markboth{nɤrkɯrku}{}
\classe{vt}
\paradigme{\textit{dir :} \jya pɯ-}
\begin{définition}\ 
\begin{déclaration}\grammar{n.orient}\end{déclaration}\end{définition}
\begin{définition}\fra verser à tout le monde\end{définition}
\begin{définition}\cmn 给所有人倒茶;酒等\end{définition}
\begin{exemple}\jya tʂha pɯ-nɤrkɯrku-t-a\cmn 我给大家倒了茶\end{exemple}
\begin{exemple}\jya tʂha pɯ-nɤrkɯrke\cmn 你给大家倒茶吧\end{exemple}
\begin{relation-sémantique}\confer{
\hyperlink{Ⓔrku}{\textit{ \papi{rku}}}
}\end{relation-sémantique}\end{entrée}

\begin{entrée}
\vedette{\hypertarget{Ⓔnɤrme}{\papi{ nɤrme}}}\markboth{nɤrme}{}\classe{vt}
\paradigme{\textit{dir :} \jya thɯ-}\acception{1}
\begin{définition}\fra enlever les poils\end{définition}
\begin{définition}\cmn 拔毛
\end{définition}
\begin{exemple}\jya qaʑo thɯ-nɤrme-t-a\cmn 我剪了羊毛\end{exemple}
\begin{exemple}\jya pɣa thɯ-nɤrme-t-a\cmn 我拔了鸡的毛\end{exemple}\acception{2}
\begin{définition}\fra enlever les mauvaises herbes du sol\end{définition}
\begin{définition}\cmn 把地面长出来的杂草铲掉\end{définition}
\begin{relation-sémantique}\confer{
\hyperlink{Ⓔtɤ-rme}{\textit{ \papi{tɤ-rme}}}
}\end{relation-sémantique}\end{entrée}

\begin{entrée}
\vedette{\hypertarget{Ⓔnɤrmi}{\papi{ nɤrmi}}}\markboth{nɤrmi}{}\classe{vt}
\paradigme{\textit{dir :} \jya tɤ-}\acception{1}
\begin{définition}\fra dire le nom de\end{définition}
\begin{définition}\cmn 叫……的名字\end{définition}
\begin{exemple}\jya nɤ-rca jɤ-kɯ-ɣe nɯ ɕɯ ŋu nɯ tɤ-nɤrmi ɲɯ-ra. (ɯ-rmi tɤ-βze ɲɯ-ra)\cmn 跟你一起来的那个人,你说一下他叫什么名字\end{exemple}
\begin{exemple}\jya tɯrme nɯ-kɯ-si nɯ ma-tɤ-tɯ-nɤrmi.\cmn 已经过世了的人,不要叫出他的名字\end{exemple}\acception{2}
\begin{définition}\fra accepter un nom\end{définition}
\begin{définition}\cmn 承认自己的名字\end{définition}
\begin{exemple}\jya nɯŋa nɯ kɯ ɯ-rmi mɯ́j-nɤrmi\cmn 那头牛不接受它的名字\end{exemple}
\begin{relation-sémantique}\confer{
\hyperlink{Ⓔtɤ-rmi}{\textit{ \papi{tɤ-rmi}}}
}\end{relation-sémantique}\end{entrée}

\begin{entrée}
\vedette{\hypertarget{Ⓔnɤrnoʁ}{\papi{ nɤrnoʁ}}}\markboth{nɤrnoʁ}{}\classe{vt}
\paradigme{\textit{dir :} \jya nɯ-}
\begin{définition}\fra réfléchir\end{définition}
\begin{définition}\cmn 动脑筋\end{définition}
\begin{exemple}\jya koŋla ɲɯ́-wɣ-nɤrnoʁ ɲɯ-ra\cmn 要真的动脑筋\end{exemple}
\begin{exemple}\jya tɯ-rju nɯ ɲɯ́-wɣ-nɤrnoʁ ɯ-jɯja ɲɯ-ɲɯ-ɤpɤɴqa ɲɯ-ɕti\cmn 我们的语言分析得越深入就越复杂\end{exemple}
\begin{relation-sémantique}\confer{
\hyperlink{Ⓔtɯ-rnoʁ}{\textit{ \papi{tɯ-rnoʁ}}}
}\end{relation-sémantique}\end{entrée}

\begin{entrée}
\vedette{\hypertarget{Ⓔnɤrŋi}{\papi{ nɤrŋi}}}\markboth{nɤrŋi}{}
\classe{n}
\begin{définition}\fra bébé\end{définition}
\begin{définition}\cmn 婴儿\end{définition}\end{entrée}

\begin{entrée}
\vedette{\hypertarget{Ⓔnɤro}{\papi{ nɤro}}}\markboth{nɤro}{}
\classe{vt}
\paradigme{\textit{dir :} \jya kɤ-}
\begin{définition}\fra utiliser pour la première fois\end{définition}
\begin{définition}\cmn 第一次用;开始吃\end{définition}
\begin{exemple}\jya kɯki kɤ-ndza ki aʑo ku-nɤram-a ŋu\cmn 我开始吃这顿餐\end{exemple}
\begin{exemple}\jya ki qajɣi ki aj kɤ-nɤro-t-a\cmn 我馍馍开始吃馍馍了\end{exemple}\end{entrée}

\begin{entrée}
\vedette{\hypertarget{Ⓔnɤrpaʁ}{\papi{ nɤrpaʁ}}}\markboth{nɤrpaʁ}{}
\classe{vt}
\paradigme{\textit{dir :} \jya tɤ-}
\begin{définition}\ 
\begin{déclaration}\grammar{denom}\end{déclaration}\end{définition}\acception{1}
\begin{définition}\fra porter à l’épaule\end{définition}
\begin{définition}\cmn 扛\end{définition}
\begin{exemple}\jya tɤ-nɤrpaʁ-a\cmn 我扛了\end{exemple}\acception{2}
\begin{définition}\fra s'entendre bien avec\end{définition}
\begin{définition}\cmn 和别人投合\end{définition}
\begin{exemple}\jya nɯnɯ smɤnba nɯ kɯ tɤ́-wɣ-nɯsman-a tɕe ɲɯ-nɤrpaʁ-a\cmn 那个医生给我治病治得非常好\end{exemple}
\begin{exemple}\jya kɤ-nɤma ɲɯ-nɤrpaʁ ma kɤ-nɯ-rɤʑi mɯ́j-nɤrpaʁ\cmn 他适合工作,不是适合闲着\end{exemple}
\begin{relation-sémantique}\confer{
\hyperlink{Ⓔtɯ-rpaʁ}{\textit{ \papi{tɯ-rpaʁ}}}
}\end{relation-sémantique}
\begin{relation-sémantique}\confer{
\hyperlink{Ⓔmɤrpaʁ}{\textit{ \papi{mɤrpaʁ}}}
}\end{relation-sémantique}
\begin{relation-sémantique}\confer{
\hyperlink{Ⓔnɤrpaʁku}{\textit{ \papi{nɤrpaʁku}}}
}\end{relation-sémantique}
\begin{relation-sémantique}\confer{
\hyperlink{Ⓔanɤrpɯrpaʁ}{\textit{ \papi{anɤrpɯrpaʁ}}}
}\end{relation-sémantique}\end{entrée}

\begin{entrée}
\vedette{\hypertarget{Ⓔnɤrpaʁku}{\papi{ nɤrpaʁku}}}\markboth{nɤrpaʁku}{}
\classe{vt}
\paradigme{\textit{dir :} \jya tɤ-}
\begin{définition}\ 
\begin{déclaration}\grammar{denom}\end{déclaration}\end{définition}
\begin{définition}\fra porter à l'épaule\end{définition}
\begin{définition}\cmn 扛\end{définition}
\begin{exemple}\jya tɤrɤm tɤ-nɤrpaʁku-t-a\cmn 我扛了木板\end{exemple}
\begin{relation-sémantique}\synonyme{
\hyperlink{Ⓔmɤrpaʁ}{\textit{ \papi{mɤrpaʁ}}}
}\end{relation-sémantique}
\begin{relation-sémantique}\confer{
\hyperlink{Ⓔtɯ-rpaʁ}{\textit{ \papi{tɯ-rpaʁ}}}
}\end{relation-sémantique}\end{entrée}

\begin{entrée}
\vedette{\hypertarget{Ⓔnɤrpɯrpu}{\papi{ nɤrpɯrpu}}}\markboth{nɤrpɯrpu}{}
\classe{vt}
\paradigme{\textit{dir :} \jya \_}
\begin{définition}\ 
\begin{déclaration}\grammar{n.orient}\end{déclaration}\end{définition}
\begin{définition}\fra se cogner partout\end{définition}
\begin{définition}\cmn 撞来撞去;到处乱撞\end{définition}
\begin{exemple}\jya laχtɕha ma-nɯ-tɯ-nɤrpɯrpe\cmn 你不要乱撞东西\end{exemple}
\begin{relation-sémantique}\confer{
\hyperlink{Ⓔrpu}{\textit{ \papi{rpu}}}
}\end{relation-sémantique}\end{entrée}

\begin{entrée}
\vedette{\hypertarget{Ⓔnɤrqaʁ}{\papi{ nɤrqaʁ}}}\markboth{nɤrqaʁ}{}
\classe{vt}
\paradigme{\textit{dir :} \jya nɯ-}
\begin{définition}\fra enlever la peau (navet)\end{définition}
\begin{définition}\cmn 剥皮(圆根)\end{définition}
\begin{exemple}\jya na-nɤrqaʁ\cmn 他剥了皮\end{exemple}
\begin{exemple}\jya rasti nɯ-nɤrqaʁ\cmn 你剥圆根的皮吧\end{exemple}
\begin{exemple}\jya rasti nɯ-nɤrqaʁ-a\cmn 我剥了圆根的皮\end{exemple}\end{entrée}

\begin{entrée}
\vedette{\hypertarget{Ⓔnɤrqhu}{\papi{ nɤrqhu}}}\markboth{nɤrqhu}{}
\classe{vt}
\paradigme{\textit{dir :} \jya pɯ-}
\paradigme{\textit{dir :} \jya thɯ-}
\begin{définition}\ 
\begin{déclaration}\grammar{denom}\end{déclaration}\end{définition}
\begin{définition}\fra éplucher, décortiquer\end{définition}
\begin{définition}\cmn 用手剥\end{définition}
\begin{exemple}\jya pejka pa-nɤrqhu\cmn 他把白瓜剥了\end{exemple}
\begin{exemple}\jya si pa-nɤrqhu\cmn 他把树皮剥了\end{exemple}
\begin{exemple}\jya ʑɴɢɯloʁ pa-nɤrqhu\cmn 他把核桃剥了壳\end{exemple}
\begin{exemple}\jya ʑɴɢɯloʁ pɯ-nɤrqhe\cmn 你把核桃壳剥一下\end{exemple}
\begin{relation-sémantique}\confer{
\hyperlink{Ⓔtɤ-rqhu}{\textit{ \papi{tɤ-rqhu}}}
}\end{relation-sémantique}\end{entrée}

\begin{entrée}
\vedette{\hypertarget{Ⓔnɤrʁaʁ}{\papi{ nɤrʁaʁ}}}\markboth{nɤrʁaʁ}{}\classe{vt}
\paradigme{\textit{dir :} \jya tɤ-}
\begin{définition}\fra chasser\end{définition}
\begin{définition}\cmn 打猎\end{définition}
\begin{exemple}\jya pɣɤtɕɯ tɤ-nɤrʁaʁ\cmn 你打鸟吧\end{exemple}
\begin{exemple}\jya qarma ɕ-tɤ-nɤrʁaʁ\cmn 你去打马鸡吧\end{exemple}
\begin{exemple}\jya pɣɤtɕɯ ɲɯ-ɤz-nɤrʁaʁ\cmn 他在猎鸟\end{exemple}
\begin{exemple}\jya tɕɤkɯ tsɯʁot ci ɣɤʑu tɕe ta-nɤrʁaʁ\cmn 那边有一只野鸡,他去打了\end{exemple}
\begin{exemple}\jya βʑɯ wuma ɲɯ-nɤrʁaʁ\cmn (猫)捉很多老鼠\end{exemple}
\begin{relation-sémantique}\confer{
\hyperlink{Ⓔɣɤrʁaʁ}{\textit{ \papi{ɣɤrʁaʁ}}}
}\end{relation-sémantique}\end{entrée}

\begin{entrée}
\vedette{\hypertarget{ⒺnɤrtaʁⒽ1}{\papi{ nɤrtaʁ}}}\markboth{nɤrtaʁ}{}\homonyme{1}\classe{vt}
\paradigme{\textit{dir :} \jya thɯ-}
\paradigme{\textit{dir :} \jya pɯ-}
\begin{définition}\ 
\begin{déclaration}\grammar{denom}\end{déclaration}\end{définition}
\begin{définition}\fra élaguer\end{définition}
\begin{définition}\cmn 砍树的枝桠\end{définition}
\begin{exemple}\jya tɯrgi thɯ-nɤrtaʁ-a\cmn 我砍了杉树的树叶\end{exemple}
\begin{exemple}\jya si pɯ-nɤrtaʁ\cmn 你砍树枝吧\end{exemple}
\begin{relation-sémantique}\confer{
\hyperlink{Ⓔtɤ-rtaʁ}{\textit{ \papi{tɤ-rtaʁ}}}
}\end{relation-sémantique}\end{entrée}

\begin{entrée}
\vedette{\hypertarget{ⒺnɤrtaʁⒽ2}{\papi{ nɤrtaʁ}}}\markboth{nɤrtaʁ}{}\homonyme{2}\classe{vt}
\paradigme{\textit{dir :} \jya pɯ-}
\paradigme{\textit{dir :} \jya tɤ-}
\begin{définition}\ 
\begin{déclaration}\grammar{trop}\end{déclaration}\end{définition}
\begin{définition}\fra trouver suffisant\end{définition}
\begin{définition}\cmn 觉得够\end{définition}
\begin{exemple}\jya pɯ-nɤrtaʁ-a\cmn 我觉得足够\end{exemple}
\begin{exemple}\jya ɲɯ-nɤrtaʁ\cmn 他觉得足够\end{exemple}
\begin{exemple}\jya laχtɕha nɯ-kɯ-mbi-a pɯ-nɤrtaʁ-a\cmn 你给的东西已经够了\end{exemple}
\begin{exemple}\jya kɤ-nɤrtaʁ-nɯ pjɤ-me\cmn 他们贪得无厌\end{exemple}
\begin{relation-sémantique}\confer{
\hyperlink{Ⓔrtaʁ}{\textit{ \papi{rtaʁ}}}
}\end{relation-sémantique}\end{entrée}

\begin{entrée}
\vedette{\hypertarget{Ⓔnɤrte}{\papi{ nɤrte}}}\markboth{nɤrte}{}
\classe{vt}
\paradigme{\textit{dir :} \jya tɤ-}
\begin{définition}\fra porter un chapeau\end{définition}
\begin{définition}\cmn 戴帽子\end{définition}
\begin{exemple}\jya tɤ-rte tu-nɤrte-a ŋu\cmn 我戴帽子\end{exemple}
\begin{relation-sémantique}\confer{
\hyperlink{Ⓔtɤ-rte}{\textit{ \papi{tɤ-rte}}}
}\end{relation-sémantique}\end{entrée}

\begin{entrée}
\vedette{\hypertarget{Ⓔnɤrtoχpjɤt}{\papi{ nɤrtoχpjɤt}}}\markboth{nɤrtoχpjɤt}{} (\variante{\_nɤrtɤχpjɤt}) \classe{vt}
\paradigme{\textit{dir :} \jya kɤ-}
\begin{définition}\ 
\begin{déclaration}\grammar{comp}\end{déclaration}\end{définition}
\begin{définition}\fra observer\end{définition}
\begin{définition}\cmn 仔细观察;打量\end{définition}
\begin{exemple}\jya ʑara kɯ tɕiʑo tu-rɤma-tɕi nɯ tú-wɣ-nɤrtoχpjɤt-tɕi ɲɯ-ŋu\cmn 他们在观察我们做事\end{exemple}
\begin{exemple}\jya jiɕqha nɯ kɯ kú-wɣ-nɤrtoχpjat-a ɲɯ-ŋu\cmn 那个人在观察我\end{exemple}
\begin{exemple}\jya nɤʑo kɯ kú-wɣ-nɤrtɤχpjat-a ɲɯ-ŋu\cmn 你在观察我\end{exemple}
\begin{exemple}\jya aj kɤ-ta-nɤrtɤχpjɤt\cmn 我观察了你\end{exemple}
\begin{relation-sémantique}\confer{
\hyperlink{Ⓔrtoʁ}{\textit{ \papi{rtoʁ}}}
}\end{relation-sémantique}
\begin{relation-sémantique}\confer{
\hyperlink{Ⓔχpjɤt}{\textit{ \papi{χpjɤt}}}
}\end{relation-sémantique}\end{entrée}

\begin{entrée}
\vedette{\hypertarget{Ⓔnɤrtɯrtoʁ}{\papi{ nɤrtɯrtoʁ}}}\markboth{nɤrtɯrtoʁ}{}
\begin{relation-sémantique}\confer{
\hyperlink{Ⓔrtoʁ}{\textit{ \papi{rtoʁ}}}
}\end{relation-sémantique}\end{entrée}

\begin{entrée}
\vedette{\hypertarget{Ⓔnɤrɯra}{\papi{ nɤrɯra}}}\markboth{nɤrɯra}{}
\classe{vi}
\paradigme{\textit{dir :} \jya \_}\acception{1}
\begin{définition}\fra regarder dans tous les sens, observer\end{définition}
\begin{définition}\cmn 四处张望;观察\end{définition}
\begin{exemple}\jya tɯrme ɯ-ɣɤʑu nɯ-nɤrɯra\cmn 你看一下有没有人\end{exemple}\acception{2}
\begin{définition}\fra faire attention à\end{définition}
\begin{définition}\cmn 注意\end{définition}
\begin{exemple}\jya tshɤt ra pɯ-nɤrɯra\cmn 你看着山羊\end{exemple}
\begin{exemple}\jya tɤ-pɤtso nɯ-nɤrɯra je\cmn 你看着小孩子\end{exemple}
\begin{exemple}\jya nɯ-nɤrɯra je ma a-mɤ-tɤ-ta-xtsɯɣ\cmn 你小心不要被我打到\end{exemple}
\end{entrée}

\begin{entrée}
\vedette{\hypertarget{Ⓔnɤrwa}{\papi{ nɤrwa}}}\markboth{nɤrwa}{}
\classe{n}
\begin{définition}\fra pâturage\end{définition}
\begin{définition}\cmn 牧场
\begin{déclaration} \étymologie{\papi{nor.ba}}\end{déclaration}\end{définition}\end{entrée}

\begin{entrée}
\vedette{\hypertarget{Ⓔnɤrwɯ}{\papi{ nɤrwɯ}}}\markboth{nɤrwɯ}{}
\classe{n}
\begin{définition}\fra trésor\end{définition}
\begin{définition}\cmn 宝贝
\begin{déclaration} \étymologie{\papi{nor.bu}}\end{déclaration}\end{définition}\end{entrée}

\begin{entrée}
\vedette{\hypertarget{Ⓔnɤrʑaβ}{\papi{ nɤrʑaβ}}}\markboth{nɤrʑaβ}{}
\classe{vt}
\paradigme{\textit{dir :} \jya kɤ-}
\begin{définition}\ 
\begin{déclaration}\grammar{denom}\end{déclaration}\end{définition}
\begin{définition}\fra se marier (garçon)\end{définition}
\begin{définition}\cmn 娶妻子\end{définition}
\begin{exemple}\jya tɯrme ɯ-rʑaβ ka-nɤrʑaβ\cmn 他娶了人家的妻子\end{exemple}
\begin{relation-sémantique}\confer{
\hyperlink{Ⓔtɤ-rʑaβ}{\textit{ \papi{tɤ-rʑaβ}}}
}\end{relation-sémantique}
\begin{relation-sémantique}\confer{
\hyperlink{Ⓔmɤrʑaβ}{\textit{ \papi{mɤrʑaβ}}}
}\end{relation-sémantique}\end{entrée}

\begin{entrée}
\vedette{\hypertarget{Ⓔnɤrʑaʁ}{\papi{ nɤrʑaʁ}}}\markboth{nɤrʑaʁ}{}\classe{vi}
\paradigme{\textit{dir :} \jya tɤ-}
\paradigme{\textit{dir :} \jya \_}
\begin{définition}\ 
\begin{déclaration}\grammar{denom}\end{déclaration}\end{définition}
\begin{définition}\fra rester longtemps\end{définition}
\begin{définition}\cmn 待很久\end{définition}
\begin{exemple}\jya alo thɯ-ɣe-a tɕe, jɤxtshi aj mbarkhom thɯ-nɤrʑaʁ-a\cmn 这一次,我在马尔康待了很久\end{exemple}
\begin{exemple}\jya japa pɯ-nɤrʑaʁ-a, ɣɯjpa ʑatsa ju-ɕe-a ra\cmn 我去年待了很长时间,今年要早点回去\end{exemple}
\begin{exemple}\jya fsapaʁ thɯ-sci mɤ-kɯ-nɤrʑaʁ tɕe, tu-ŋke ɕti\cmn 牲畜出生不久就会走路\end{exemple}
\begin{relation-sémantique}\confer{
 \papi{tɤ-rʑaʁ1}
}\end{relation-sémantique}\end{entrée}

\begin{entrée}
\vedette{\hypertarget{Ⓔnɤʁarphɤβ}{\papi{ nɤʁarphɤβ}}}\markboth{nɤʁarphɤβ}{}\classe{vt}
\paradigme{\textit{dir :} \jya pɯ-}
\begin{définition}\ 
\begin{déclaration}\grammar{incorp}\end{déclaration}\end{définition}
\begin{définition}\fra frapper avec ses ailes\end{définition}
\begin{définition}\cmn 用翅膀拍打\end{définition}
\begin{exemple}\jya qaliaʁ kɯ paʁtsa pjɤ-nɤʁarphɤβ\cmn 老鹰用翅膀拍打了小猪\end{exemple}
\begin{relation-sémantique}\confer{
\hyperlink{Ⓔʁarphɤβ}{\textit{ \papi{ʁarphɤβ}}}
}\end{relation-sémantique}\end{entrée}

\begin{entrée}
\vedette{\hypertarget{Ⓔnɤʁaʁ}{\papi{ nɤʁaʁ}}}\markboth{nɤʁaʁ}{}
\classe{vi}\acception{1}
\paradigme{\textit{dir :} \jya nɯ-}
\begin{définition}\fra faire la fête, s'amuser\end{définition}
\begin{définition}\cmn 休息\end{définition}\acception{2}
\paradigme{\textit{dir :} \jya pɯ-}
\begin{définition}\fra se prélasser au soleil\end{définition}
\begin{définition}\cmn 晒太阳\end{définition}
\begin{exemple}\jya ɯʑo pɯ-nɤʁaʁ\cmn 他晒了太阳\end{exemple}
\begin{exemple}\jya jisŋi tɤŋe ɲɯ-wxti, pɯ-nɤʁaʁ-a\cmn 今天太阳很大,我晒了太阳\end{exemple}
\begin{relation-sémantique}\confer{
\hyperlink{Ⓔtɤʁaʁ}{\textit{ \papi{tɤʁaʁ}}}
}\end{relation-sémantique}
\begin{relation-sémantique}\confer{
\hyperlink{Ⓔnɤʁɯmʁaʁ}{\textit{ \papi{nɤʁɯmʁaʁ}}}
}\end{relation-sémantique}\end{entrée}

\begin{entrée}
\vedette{\hypertarget{Ⓔnɤʁdɤn}{\papi{ nɤʁdɤn}}}\markboth{nɤʁdɤn}{}
\classe{vt}
\begin{définition}\ 
\begin{déclaration}\grammar{denom}\end{déclaration}\end{définition}\acception{1}
\paradigme{\textit{dir :} \jya pɯ-}
\paradigme{\textit{dir :} \jya tɤ-}
\begin{définition}\fra placer sous\end{définition}
\begin{définition}\cmn 垫\end{définition}\acception{2}
\paradigme{\textit{dir :} \jya kɤ-}
\paradigme{\textit{dir :} \jya nɯ-}
\begin{définition}\fra inviter\end{définition}
\begin{définition}\cmn 邀请(敬语)
\begin{déclaration}\use{“邀请”这种意思是古语\stylefv{kɯɕɯŋgɯ ɯ-skɤt},只出现在传统故事中}\end{déclaration}
\begin{déclaration} \étymologie{\papi{gdan}}\end{déclaration}\end{définition}
\begin{exemple}\jya smɤnmi mitoʁ kuɕana ɕ-ku-nɤʁdɤn tɕe, tɕetha aʑo a-kɯ-mŋɤm phɤn ɕti\cmn 如果他请到古夏纳的话,我的病马上就会好\end{exemple}\begin{sous-entrée}
\vedette{\hypertarget{}{\papi{ znɤʁdɤn}}}\markboth{znɤʁdɤn}{}
\paradigme{\textit{dir :} \jya pɯ-}
\paradigme{\textit{dir :} \jya tɤ-}
\begin{définition}\fra placer quelque chose sous\end{définition}
\begin{définition}\cmn 垫\end{définition}
\begin{exemple}\jya @bandeng ɯ-pa tɤ-znɤʁdɤn\cmn 你在板凳的下面垫一块东西(保护地板)\end{exemple}
\begin{exemple}\jya si kɯ pɯ-znɤʁdɤn\cmn 你用木块垫一下\end{exemple}
\begin{exemple}\jya @dianshi ɯ-ʁdɤn ci ɲɯ-ra, tɕe si pɯ-znɤʁdan-a\cmn 电视需要东西垫着,我用木头垫了一些\end{exemple}
\begin{exemple}\jya tɤ-mkɯm kɯ tɤ-znɤʁdɤn\cmn 你用枕头垫一下吧\end{exemple}
\begin{exemple}\jya jɯfɕɯr tɤ-mkɯm kutɕu pɯ-tɯ-ta-t tɕe, kɯki tɤ-tɯ-znɤʁdɤn\cmn 你昨天放了枕头,用来垫这个(话筒)\end{exemple}
\begin{exemple}\jya khɯtsa ɯ-pa nɯ tɕu ɕico kɤ-znɤʁdɤn mɤ-sna\cmn 在碗下面不能用塑料垫着\end{exemple}
\begin{relation-sémantique}\confer{
\hyperlink{Ⓔtɤ-ʁdɤn}{\textit{ \papi{tɤ-ʁdɤn}}}
}\end{relation-sémantique}
\end{sous-entrée}\end{entrée}

\begin{entrée}
\vedette{\hypertarget{Ⓔnɤʁndɯʁndɯ}{\papi{ nɤʁndɯʁndɯ}}}\markboth{nɤʁndɯʁndɯ}{}
\classe{vt}
\paradigme{\textit{dir :} \jya tɤ-}
\begin{définition}\ 
\begin{déclaration}\grammar{n.orient}\end{déclaration}\end{définition}
\begin{définition}\fra battre à n'importe quelle occasion\end{définition}
\begin{définition}\cmn 打来打去;随便乱打\end{définition}
\begin{exemple}\jya tshɤt qaʑo ra tɤ-nɤʁndɯʁndɯ-t-a\cmn 我打了山羊和绵羊\end{exemple}\end{entrée}

\begin{entrée}
\vedette{\hypertarget{Ⓔnɤʁnoŋ}{\papi{ nɤʁnoŋ}}}\markboth{nɤʁnoŋ}{}
\classe{vi}
\paradigme{\textit{dir :} \jya nɯ-}
\begin{définition}\fra trouver dommage, regretter\end{définition}
\begin{définition}\cmn 觉得可惜;后悔
\begin{déclaration} \étymologie{\papi{gnoŋ}}\end{déclaration}\end{définition}
\begin{exemple}\jya @beibei pɯ-qrɯ-t-a tɕe ɲɯ-nɤʁnoŋ-a\cmn 我打破了杯子,觉得很可惜\end{exemple}
\begin{exemple}\jya @beibei pa-qrɯ tɕe ɲɯ-nɤʁnoŋ\cmn 他打破了杯子,觉得很可惜\end{exemple}
\begin{exemple}\jya khɯtsa pɯ-kɤ-qrɯ nɯ kɤ-nɤʁnoŋ me\cmn 这个碗打破了,没有关系\end{exemple}\end{entrée}

\begin{entrée}
\vedette{\hypertarget{Ⓔnɤʁombi}{\papi{ nɤʁombi}}}\markboth{nɤʁombi}{}
\classe{vt}
\paradigme{\textit{dir :} \jya nɯ-}
\begin{définition}\fra perdre l'espoir\end{définition}
\begin{définition}\cmn 灰心;对……没有希望\end{définition}
\begin{exemple}\jya jiɕqha kɤ-nɤma nɯ wuma ʑo ɲɯ-ɴqa tɕe, nɯ-nɤʁombi-t-a ɕti\cmn 这件事情做起来很难,我对它没有希望了\end{exemple}
\begin{relation-sémantique}\confer{
\hyperlink{Ⓔtɯ-ʁo,mbi}{\textit{ \papi{tɯ-ʁo,mbi}}}
}\end{relation-sémantique}
\begin{relation-sémantique}\confer{
\hyperlink{Ⓔsɤʁombi}{\textit{ \papi{sɤʁombi}}}
}\end{relation-sémantique}\end{entrée}

\begin{entrée}
\vedette{\hypertarget{Ⓔnɤʁɯmʁaʁ}{\papi{ nɤʁɯmʁaʁ}}}\markboth{nɤʁɯmʁaʁ}{}\classe{vi}
\begin{définition}\fra s'amuser à droite et à gauche\end{définition}
\begin{définition}\cmn 到处玩耍\end{définition}
\begin{relation-sémantique}\confer{
\hyperlink{Ⓔnɤʁaʁ}{\textit{ \papi{nɤʁaʁ}}}
}\end{relation-sémantique}\end{entrée}

\begin{entrée}
\vedette{\hypertarget{Ⓔnɤsaʁdɯɣ}{\papi{ nɤsaʁdɯɣ}}}\markboth{nɤsaʁdɯɣ}{}
\classe{vt}
\begin{définition}\ 
\begin{déclaration}\grammar{trop}\end{déclaration}\end{définition}
\begin{définition}\fra trouver désagréable\end{définition}
\begin{définition}\cmn 觉得讨厌\end{définition}
\begin{exemple}\jya nɤ-kɤcu si nɯ ɲɯ-nɤsaʁdɯɣ-a\cmn 你那边的树,我觉得很麻烦\end{exemple}
\begin{exemple}\jya nɤ-kɤcu laχtɕha nɯ ɲɯ-nɤsaʁdɯɣ-a\cmn 你那边的东西,我觉得很麻烦\end{exemple}
\begin{exemple}\jya aʑo kutɕu ku-rɤʑit-a ɣɯ-nɤsaʁdɯɣ-a ɲɯ-sɯsam-a\cmn 我想他觉得我在这里很碍事\end{exemple}
\begin{relation-sémantique}\confer{
\hyperlink{ⒺʁdɯɣⒽ1}{\textit{ \papi{ʁdɯɣ1}}}
}\end{relation-sémantique}\end{entrée}

\begin{entrée}
\vedette{\hypertarget{Ⓔnɤsɤɕke}{\papi{ nɤsɤɕke}}}\markboth{nɤsɤɕke}{}
\begin{relation-sémantique}\confer{
\hyperlink{ⒺsɤɕkeⒽ1}{\textit{ \papi{sɤɕke1}}}
}\end{relation-sémantique}
\end{entrée}

\begin{entrée}
\vedette{\hypertarget{Ⓔnɤsɤɣ}{\papi{ nɤsɤɣ}}}\markboth{nɤsɤɣ}{}
\classe{vt}
\paradigme{\textit{dir :} \jya tɤ-}
\begin{définition}\fra être jaloux\end{définition}
\begin{définition}\cmn 吃醋\end{définition}
\begin{exemple}\jya jiɕqha kɯ tú-wɣ-nɤsaɣ-a ɲɯ-ŋu\cmn 他吃我的醋\end{exemple}
\begin{exemple}\jya jiɕqha nɯ tɤ-nɤsaɣ-a\cmn 我吃他的醋\end{exemple}\begin{sous-entrée}
\vedette{\hypertarget{}{\papi{ sɤnɤsɤɣ}}}\markboth{sɤnɤsɤɣ}{}\classe{vi}
\paradigme{\textit{dir :} \jya tɤ-}
\begin{définition}\ 
\begin{déclaration}\grammar{apass}\end{déclaration}\end{définition}
\begin{définition}\fra être jaloux\end{définition}
\begin{définition}\cmn 吃醋\end{définition}
\begin{exemple}\jya ɲɯ-sɤnɤsɤɣ\cmn 他在吃醋\end{exemple}
\begin{exemple}\jya tu-kɯ-nɤsɤɣ a-pɯ-ŋu tɕe, tɕe ɯ-rʑaβ ɲɯ́-wɣ-nɤmthɯn ndʐa ɲɯ-ŋu\cmn 别人吃自己的醋,是因为自己喜欢上别人的妻子\end{exemple}
\end{sous-entrée}\end{entrée}

\begin{entrée}
\vedette{\hypertarget{Ⓔnɤsɤɣdɯɣ}{\papi{ nɤsɤɣdɯɣ}}}\markboth{nɤsɤɣdɯɣ}{}\classe{vt}
\paradigme{\textit{dir :} \jya nɯ-}
\begin{définition}\ 
\begin{déclaration}\grammar{trop}\end{déclaration}\end{définition}
\begin{définition}\fra se sentir mal, s'ennuyer\end{définition}
\begin{définition}\cmn 觉得不舒服;觉得无聊\end{définition}
\begin{exemple}\jya na-nɤsɤɣdɯɣ\cmn 他觉得不舒服\end{exemple}
\begin{exemple}\jya tɯrme kɯ tɯ-rju mɤ-kɯ-mpɕɤr ta-tɯt tɕe, ɯʑo kɯ na-nɤsɤɣdɯɣ\cmn 有人说了不好听的话,他就觉得不舒服\end{exemple}
\begin{exemple}\jya aʑo nɤ-phe ku-rɤʑit-a ɲɯ-tɯ-nɤsɤɣdɯɣ\cmn 我住在你这里,你觉得很麻烦\end{exemple}
\begin{relation-sémantique}\confer{
\hyperlink{Ⓔsɤɣdɯɣ}{\textit{ \papi{sɤɣdɯɣ}}}
}\end{relation-sémantique}
\begin{relation-sémantique}\confer{
\hyperlink{Ⓔdɯɣ}{\textit{ \papi{dɯɣ}}}
}\end{relation-sémantique}\end{entrée}

\begin{entrée}
\vedette{\hypertarget{Ⓔnɤsɤjloʁ}{\papi{ nɤsɤjloʁ}}}\markboth{nɤsɤjloʁ}{}
\begin{relation-sémantique}\confer{
\hyperlink{Ⓔsɤjloʁ}{\textit{ \papi{sɤjloʁ}}}
}\end{relation-sémantique}
\end{entrée}

\begin{entrée}
\vedette{\hypertarget{Ⓔnɤsɤre}{\papi{ nɤsɤre}}}\markboth{nɤsɤre}{}
\begin{relation-sémantique}\confer{
\hyperlink{Ⓔsɤre}{\textit{ \papi{sɤre}}}
}\end{relation-sémantique}\end{entrée}

\begin{entrée}
\vedette{\hypertarget{Ⓔnɤsɤscit}{\papi{ nɤsɤscit}}}\markboth{nɤsɤscit}{}
\classe{vt}
\paradigme{\textit{dir :} \jya pɯ-}
\begin{définition}\ 
\begin{déclaration}\grammar{trop}\end{déclaration}\end{définition}
\begin{définition}\fra trouver agréable\end{définition}
\begin{définition}\cmn 觉得舒服\end{définition}
\begin{exemple}\jya wo, ki sɤtɕha ki ɲɯ-nɤsɤscit-a rcanɯ\cmn 我觉得这个地方很舒服\end{exemple}
\begin{exemple}\jya ki tɯ-ŋga ki ɲɯ-nɤsɤscit-a\cmn 我觉得这件衣服穿起来很舒服\end{exemple}
\begin{exemple}\jya jɯfɕɯr pɯ-nɤʁaʁ-i, pɯ-nɤsɤscit-a\cmn 我们昨天晒太阳,我觉得很开心\end{exemple}
\begin{relation-sémantique}\confer{
\hyperlink{Ⓔscit}{\textit{ \papi{scit}}}
}\end{relation-sémantique}
\begin{relation-sémantique}\confer{
\hyperlink{Ⓔsɤscit}{\textit{ \papi{sɤscit}}}
}\end{relation-sémantique}\end{entrée}

\begin{entrée}
\vedette{\hypertarget{Ⓔnɤscɤlɤt}{\papi{ nɤscɤlɤt}}}\markboth{nɤscɤlɤt}{}\classe{vt}
\begin{définition}\ 
\begin{déclaration}\grammar{comp}\end{déclaration}\end{définition}
\begin{définition}\fra aller chercher et ramener\end{définition}
\begin{définition}\cmn 接送\end{définition}
\begin{exemple}\jya nɤʑo chɤ-tɯ-wxti tɕe nɯ ma nɤ-kɯ-nɤscɤlɤt mɤ-ra\cmn 你长大了,不再需要人接送\end{exemple}
\begin{relation-sémantique}\confer{
\hyperlink{Ⓔsco}{\textit{ \papi{sco}}}
}\end{relation-sémantique}
\begin{relation-sémantique}\confer{
\hyperlink{ⒺlɤtⒽ1}{\textit{ \papi{lɤt1}}}
}\end{relation-sémantique}\end{entrée}

\begin{entrée}
\vedette{\hypertarget{Ⓔnɤscɤr}{\papi{ nɤscɤr}}}\markboth{nɤscɤr}{}
\classe{vi}
\paradigme{\textit{dir :} \jya pɯ-}
\begin{définition}\fra être saisi de frayeur\end{définition}
\begin{définition}\cmn 受惊\end{définition}
\begin{exemple}\jya pɯ-nɤscar-a\cmn 我吓了一跳\end{exemple}\begin{sous-entrée}
\vedette{\hypertarget{}{\papi{ znɤscɤr}}}\markboth{znɤscɤr}{}\classe{vt}
\paradigme{\textit{dir :} \jya pɯ-}
\begin{définition}\ 
\begin{déclaration}\grammar{caus}\end{déclaration}\end{définition}
\begin{définition}\fra effrayer\end{définition}
\begin{définition}\cmn 惊吓\end{définition}
\begin{exemple}\jya @laba tha-ʑmbri tɕe pɯ́-wɣ-znɤscar-a\cmn 他吹了喇叭,把我吓了一跳\end{exemple}
\begin{exemple}\jya a-mgɯr zɯ taχphe ta-lɤt tɕe pɯ́-wɣ-znɤscar-a\cmn 他用手掌拍了我的背部,把我吓了一跳\end{exemple}
\begin{relation-sémantique}\confer{
 \papi{tɤ-scɤr}
}\end{relation-sémantique}
\end{sous-entrée}\end{entrée}

\begin{entrée}
\vedette{\hypertarget{Ⓔnɤsci}{\papi{ nɤsci}}}\markboth{nɤsci}{}\classe{vt}
\paradigme{\textit{dir :} \jya tɤ-}
\paradigme{\textit{dir :} \jya nɯ-}
\paradigme{\textit{dir :} \jya thɯ-}
\begin{définition}\fra changer\end{définition}
\begin{définition}\cmn 换\end{définition}
\begin{exemple}\jya tɯthɯ ɯ-ŋgɯ tɯ-ci ka-nɤsci\cmn 他换了锅子里的水\end{exemple}
\begin{exemple}\jya a-ŋga tɤ-nɤsci-t-a\cmn 我换了衣服\end{exemple}
\begin{exemple}\jya ɯ-ŋga ta-nɤsci\cmn 他换了衣服\end{exemple}
\begin{exemple}\jya a-ŋga ɲɤ-ci tɕe, tɤ-nɤsci-t-a\cmn 我的衣服湿了,我来换\end{exemple}
\begin{exemple}\jya a-ma tɤ-nɤsci-t-a\cmn 我换了工作\end{exemple}
\begin{exemple}\jya a-sɤtɕha nɯ-nɤsci-t-a\cmn 我换了住的地方\end{exemple}
\begin{exemple}\jya laʁdɯn ɯ-jɯ thɯ-nɤsci-t-a\cmn 我换了工具的把子\end{exemple}
\begin{relation-sémantique}\synonyme{
\hyperlink{Ⓔsɤndu}{\textit{ \papi{sɤndu}}}
}\end{relation-sémantique}\end{entrée}

\begin{entrée}
\vedette{\hypertarget{Ⓔnɤscɯɕa}{\papi{ nɤscɯɕa}}}\markboth{nɤscɯɕa}{}\classe{vt}
\paradigme{\textit{dir :} \jya pɯ-}
\begin{définition}\fra tanner\end{définition}
\begin{définition}\cmn 把皮子刮干净(鞣制)\end{définition}
\begin{exemple}\jya pɯ-nɤscɯɕa-t-a\cmn 我刮干净了\end{exemple}
\begin{exemple}\jya kɯ-mɤku, tɯ-ndʐi nɯ pjɯ́-wɣ-nɤscɯɕa tɕe nɯ kóʁmɯz nɤ ta-mar kú-wɣ-mar tɕe z-ɲɯ́-wɣ-χtsɤβ ra\cmn 首先要把皮子刮干净,然后涂一层油,然后揉\end{exemple}\end{entrée}

\begin{entrée}
\vedette{\hypertarget{Ⓔnɤscɯscɤt}{\papi{ nɤscɯscɤt}}}\markboth{nɤscɯscɤt}{}
\begin{relation-sémantique}\confer{
\hyperlink{Ⓔscɤt}{\textit{ \papi{scɤt}}}
}\end{relation-sémantique}\end{entrée}

\begin{entrée}
\vedette{\hypertarget{Ⓔnɤsma}{\papi{ nɤsma}}}\markboth{nɤsma}{}\classe{vt}
\paradigme{\textit{dir :} \jya pɯ-}
\paradigme{\textit{dir :} \jya nɯ-}
\begin{définition}\fra admirer\end{définition}
\begin{définition}\cmn 羡慕\end{définition}
\begin{exemple}\jya a-zda ra nɯ ɲɯ-ɤro-nɯ tɕe ɲɯ-nɤsme-a-nɯ\cmn 他们拥有那个东西,我很羡慕他们\end{exemple}
\begin{exemple}\jya jɯlco ra kɯ wuma ʑo pjɤ́-wɣ-nɤsma-ndʑi\cmn 村民们很羡慕他们\end{exemple}
\begin{exemple}\jya kɯ-mɤɕi cho kɯ-ɤchɯcha nɯra mɤ-nɤsme-a\cmn 我不羡慕有钱和才能的人\end{exemple}\begin{sous-entrée}
\vedette{\hypertarget{}{\papi{ sɤnɤsma}}}\markboth{sɤnɤsma}{}\classe{vi}
\begin{définition}\ 
\begin{déclaration}\grammar{apass}\end{déclaration}\end{définition}
\begin{définition}\fra admirer les autres\end{définition}
\begin{définition}\cmn 羡慕别人\end{définition}
\end{sous-entrée}\begin{sous-entrée}
\vedette{\hypertarget{}{\papi{ sɤsma}}}\markboth{sɤsma}{}\classe{vs}
\begin{définition}\ 
\begin{déclaration}\grammar{deexp}\end{déclaration}\end{définition}
\begin{définition}\fra admirable\end{définition}
\begin{définition}\cmn 值得羡慕\end{définition}
\begin{exemple}\jya ɯʑo kɯ-sɤsma ci ŋu\cmn 他是个值得羡慕的人\end{exemple}
\end{sous-entrée}\end{entrée}

\begin{entrée}
\vedette{\hypertarget{Ⓔnɤsna}{\papi{ nɤsna}}}\markboth{nɤsna}{}
\classe{vt}
\paradigme{\textit{dir :} \jya tɤ-}
\begin{définition}\ 
\begin{déclaration}\grammar{trop}\end{déclaration}\end{définition}
\begin{définition}\fra vouloir avoir\end{définition}
\begin{définition}\cmn 想要\end{définition}
\begin{exemple}\jya ɲɯ-nɤsne-a\cmn 我想要\end{exemple}
\begin{exemple}\jya ɯ-phɯ ɲɯ-wxti tɕe, ma-tɤ-tɯ-nɤsne\cmn 这个东西很贵,你不能要(我给不起)\end{exemple}
\begin{exemple}\jya tɤ-mthɯm mɯ-nɯ-kɯ-ɣɤdi ra nɤsna-j ma nɯ-kɯ-ɣɤdi ra mɤ-nɤsna-j\cmn 我想要没变味的肉,不想要变味了的肉\end{exemple}
\begin{relation-sémantique}\confer{
\hyperlink{Ⓔsna}{\textit{ \papi{sna}}}
}\end{relation-sémantique}\end{entrée}

\begin{entrée}
\vedette{\hypertarget{Ⓔnɤsnɯndo}{\papi{ nɤsnɯndo}}}\markboth{nɤsnɯndo}{}
\classe{vt}
\paradigme{\textit{dir :} \jya kɤ-}
\begin{définition}\ 
\begin{déclaration}\grammar{incorp}\end{déclaration}\end{définition}
\begin{définition}\fra en vouloir à, garder rancune envers\end{définition}
\begin{définition}\cmn 对……怀恨在心\end{définition}
\begin{exemple}\jya kɤ-nɤsnɯndo-t-a\cmn 我对你怀恨在心\end{exemple}
\begin{exemple}\jya kɤ-kɯ-nɤsnɯndo-a\cmn 你对我怀恨在心\end{exemple}
\begin{relation-sémantique}\confer{
\hyperlink{Ⓔtɯ-sni}{\textit{ \papi{tɯ-sni}}}
}\end{relation-sémantique}
\begin{relation-sémantique}\confer{
\hyperlink{Ⓔndo}{\textit{ \papi{ndo}}}
}\end{relation-sémantique}
\end{entrée}

\begin{entrée}
\vedette{\hypertarget{Ⓔnɤsɲɯsɲu}{\papi{ nɤsɲɯsɲu}}}\markboth{nɤsɲɯsɲu}{}
\begin{relation-sémantique}\confer{
\hyperlink{Ⓔsɲu}{\textit{ \papi{sɲu}}}
}\end{relation-sémantique}
\end{entrée}

\begin{entrée}
\vedette{\hypertarget{Ⓔnɤsŋɯt}{\papi{ nɤsŋɯt}}}\markboth{nɤsŋɯt}{}\classe{vt}
\paradigme{\textit{dir :} \jya tɤ-}
\paradigme{\textit{dir :} \jya lɤ-}
\begin{définition}\fra ronger, tenir avec ses dent\end{définition}
\begin{définition}\cmn 啃;用牙齿咬住\end{définition}
\begin{exemple}\jya @gangbi ma-tɤ-tɯ-nɤsŋɯt\cmn 你不要啃你的钢笔\end{exemple}
\begin{exemple}\jya βʑɯ kɯ @dianxian to-nɤsŋɯt\cmn 老鼠把电线啃了\end{exemple}
\begin{exemple}\jya tɯ-ŋga to-nɤsŋɯt\cmn 啃了衣服\end{exemple}
\begin{exemple}\jya tɤ-ri to-nɤsŋɯt\cmn 啃了线\end{exemple}
\begin{exemple}\jya @zhuozi to-nɤsŋɯt\cmn 啃了桌子\end{exemple}
\begin{exemple}\jya nɯ-jaʁmu lu-nɤsŋɯt-nɯ ŋgrɤl\cmn (婴儿)爱嘬大拇指\end{exemple}
\begin{relation-sémantique}\confer{
\hyperlink{Ⓔtɤ-sŋɯt}{\textit{ \papi{tɤ-sŋɯt}}}
}\end{relation-sémantique}\end{entrée}

\begin{entrée}
\vedette{\hypertarget{Ⓔnɤso}{\papi{ nɤso}}}\markboth{nɤso}{}\classe{vt}
\paradigme{\textit{dir :} \jya nɯ-}
\begin{définition}\fra manquer\end{définition}
\begin{définition}\cmn 想念很久\end{définition}
\begin{exemple}\jya tɤjko nɯ-nɤso-t-a\cmn 我想念酸菜很久了\end{exemple}
\begin{relation-sémantique}\synonyme{
\hyperlink{Ⓔnɯɣbɯɣ}{\textit{ \papi{nɯɣbɯɣ}}}
}\end{relation-sémantique}\end{entrée}

\begin{entrée}
\vedette{\hypertarget{Ⓔnɤsphjarlar}{\papi{ nɤsphjarlar}}}\markboth{nɤsphjarlar}{}
\classe{vt}
\paradigme{\textit{dir :} \jya tɤ-}
\begin{définition}\fra étendre\end{définition}
\begin{définition}\cmn 展开(衣服、布料)\end{définition}
\begin{exemple}\jya tɯ-ŋga tɤ-nɤsphjarlar-a ma mɯ́j-zbaʁ\cmn 我把衣服晾开了,因为还没有干\end{exemple}
\begin{relation-sémantique}\confer{
\hyperlink{Ⓔsphjar}{\textit{ \papi{sphjar}}}
}\end{relation-sémantique}\end{entrée}

\begin{entrée}
\vedette{\hypertarget{Ⓔnɤstu}{\papi{ nɤstu}}}\markboth{nɤstu}{}
\classe{vt}
\paradigme{\textit{dir :} \jya nɯ-}
\begin{définition}\fra croire (quelqu'un)\end{définition}
\begin{définition}\cmn 相信(人)\end{définition}
\begin{exemple}\jya aʑo ɲɯ-ta-nɤstu\cmn 我相信你\end{exemple}
\begin{exemple}\jya jiɕqha nɯ rɯkhramba tɕe, ma-nɯ-tɯ-nɤste\cmn 这个人在说谎,你不要相信他\end{exemple}
\begin{exemple}\jya ɯʑo kɯ ta-tɯt nɯ ma-nɯ-tɯ-nɤste\cmn 你不要相信他刚才讲的话\end{exemple}\begin{sous-entrée}
\vedette{\hypertarget{}{\papi{ sɤnɤstu}}}\markboth{sɤnɤstu}{}\classe{vi}
\begin{définition}\ 
\begin{déclaration}\grammar{apass}\end{déclaration}\end{définition}
\begin{définition}\fra avoir confiance en les gens\end{définition}
\begin{définition}\cmn 相信别人\end{définition}
\begin{relation-sémantique}\confer{
\hyperlink{ⒺstuⒽ2}{\textit{ \papi{stu2}}}
}\end{relation-sémantique}
\end{sous-entrée}\begin{sous-entrée}
\vedette{\hypertarget{}{\papi{ ʑɣɤnɤstu}}}\markboth{ʑɣɤnɤstu}{}\classe{vi}
\begin{définition}\ 
\begin{déclaration}\grammar{refl}\end{déclaration}\end{définition}
\begin{exemple}\jya ɯ-zda mɯ́j-nɤste ma ɯʑo ɲɯ-ʑɣɤ-nɤstu\cmn 他不相信别人,只相信自己\end{exemple}
\begin{exemple}\jya mɯ́j-ʑɣɤnɤstu\cmn 他没有自信\end{exemple}
\end{sous-entrée}\end{entrée}

\begin{entrée}
\vedette{\hypertarget{Ⓔnɤsta}{\papi{ nɤsta}}}\markboth{nɤsta}{}
\classe{vt}
\paradigme{\textit{dir :} \jya nɯ-}
\paradigme{\textit{dir :} \jya kɤ-}
\begin{définition}\fra s'habituer\end{définition}
\begin{définition}\cmn 习惯(环境)\end{définition}
\begin{exemple}\jya ki sɤtɕha kɤ-nɤsta-t-a\cmn 我习惯了这个地方\end{exemple}
\begin{exemple}\jya nɯʑo nɯ-kha aʑo nɯ-nɤsta-t-a\cmn 我习惯了你们的家\end{exemple}\end{entrée}

\begin{entrée}
\vedette{\hypertarget{Ⓔnɤstɤr}{\papi{ nɤstɤr}}}\markboth{nɤstɤr}{}
\classe{vt}
\paradigme{\textit{dir :} \jya \_}
\begin{définition}\fra tirer d'un seul coup\end{définition}
\begin{définition}\cmn 突然一拉\end{définition}
\begin{exemple}\jya nɤ-kɤcu nɯ nɯ-nɤstɤr\cmn 你突然拉了你那边的那个东西\end{exemple}
\begin{exemple}\jya akɤcu laχtɕha nɯ-nɤstar-a\cmn 我突然拉了我这边的东西\end{exemple}
\begin{exemple}\jya kɤ-nɤstar-a ʑo kɤ-ɣɤrat-a\cmn 我拿了然后扔了(手机)\end{exemple}
\begin{exemple}\jya qandʑɣi kɯ a-mthɯm tha-nɤstɤr ʑo tha-nɯtsɯm\cmn 鹰把我的肉突然地抢走了\end{exemple}\end{entrée}

\begin{entrée}
\vedette{\hypertarget{Ⓔnɤstumbat}{\papi{ nɤstumbat}}}\markboth{nɤstumbat}{}
\begin{relation-sémantique}\confer{
\hyperlink{ⒺstuⒽ1}{\textit{ \papi{stu1}}}
}\end{relation-sémantique}\end{entrée}

\begin{entrée}
\vedette{\hypertarget{Ⓔnɤstɯstu}{\papi{ nɤstɯstu}}}\markboth{nɤstɯstu}{}\classe{vt}
\begin{définition}\fra causer des ennuis à\end{définition}
\begin{définition}\cmn 找……麻烦\end{définition}\end{entrée}

\begin{entrée}
\vedette{\hypertarget{Ⓔnɤtu}{\papi{ nɤtu}}}\markboth{nɤtu}{} (\variante{atu}) \classe{adv}
\begin{définition}\fra sur les trois pierres du foyer\end{définition}
\begin{définition}\cmn 在三脚架上\end{définition}
\begin{exemple}\jya nɤtu tɯthɯ ɯ-ŋgɯ lɤ-lɤt\cmn 倒在锅子上(在三脚架)\end{exemple}\end{entrée}

\begin{entrée}
\vedette{\hypertarget{Ⓔnɤtar}{\papi{ nɤtar}}}\markboth{nɤtar}{}
\classe{vt}
\paradigme{\textit{dir :} \jya kɤ-}
\paradigme{\textit{dir :} \jya nɯ-}
\begin{définition}\fra battre avec un bâton\end{définition}
\begin{définition}\cmn 用木棍打\end{définition}
\begin{exemple}\jya nɯŋa kɤ-nɤtar\cmn 你用木棍打一下牛吧\end{exemple}
\begin{exemple}\jya nɤ-stu tɤ-fse ma tha ci ta-nɤtar\cmn 你要守规矩,不然我就会打你一下。\end{exemple}
\begin{relation-sémantique}\confer{
\hyperlink{Ⓔtɤtar}{\textit{ \papi{tɤtar}}}
}\end{relation-sémantique}\end{entrée}

\begin{entrée}
\vedette{\hypertarget{Ⓔnɤtɕɯ}{\papi{ nɤtɕɯ}}}\markboth{nɤtɕɯ}{}\classe{vt}
\paradigme{\textit{dir :} \jya tɤ-}
\begin{définition}\fra adopter (un garçon)\end{définition}
\begin{définition}\cmn 领养(男孩子)\end{définition}
\begin{relation-sémantique}\synonyme{
\hyperlink{Ⓔnɤme}{\textit{ \papi{nɤme}}}
}\end{relation-sémantique}
\begin{relation-sémantique}\confer{
\hyperlink{Ⓔtɤ-tɕɯ}{\textit{ \papi{tɤ-tɕɯ}}}
}\end{relation-sémantique}\end{entrée}

\begin{entrée}
\vedette{\hypertarget{Ⓔnɤthɤβ}{\papi{ nɤthɤβ}}}\markboth{nɤthɤβ}{}\classe{vi}
\paradigme{\textit{dir :} \jya kɤ-}
\begin{définition}\fra entourer des deux côtés\end{définition}
\begin{définition}\cmn 从两面围起来、挡着\end{définition}
\begin{exemple}\jya kɤ-nɤthɤβ-tɕi\cmn 我们俩围起来了\end{exemple}
\begin{exemple}\jya nɤthɤβ-tɕi tɕe scɤt-tɕi\cmn 我们俩先站在两边,再搬过去(这个东西很重,需要两个人一起抬)\end{exemple}\end{entrée}

\begin{entrée}
\vedette{\hypertarget{Ⓔnɤthɯthu}{\papi{ nɤthɯthu}}}\markboth{nɤthɯthu}{}
\classe{vt}
\paradigme{\textit{dir :} \jya nɯ-}
\begin{définition}\ 
\begin{déclaration}\grammar{n.orient}\end{déclaration}\end{définition}
\begin{définition}\fra demander partout\end{définition}
\begin{définition}\cmn 到处去问\end{définition}
\begin{exemple}\jya aʑo nɯ-nɤthɯthu-t-a\cmn 我到处问了\end{exemple}
\begin{exemple}\jya tʂu ɲɯ-ɤz-nɤthɯthu\cmn 他到处问路\end{exemple}
\begin{relation-sémantique}\confer{
\hyperlink{ⒺthuⒽ1}{\textit{ \papi{thu1}}}
}\end{relation-sémantique}\end{entrée}

\begin{entrée}
\vedette{\hypertarget{Ⓔnɤtsa}{\papi{ nɤtsa}}}\markboth{nɤtsa}{}
\classe{n}
\begin{définition}\fra maladie\end{définition}
\begin{définition}\cmn 病痛\end{définition}
\begin{exemple}\jya nɤ-nɤtsa ɲɯ-ɣɤrŋa\cmn 你有病痛的可能\end{exemple}\end{entrée}

\begin{entrée}
\vedette{\hypertarget{Ⓔnɤtsa}{\papi{ nɤtsa}}}\markboth{nɤtsa}{}
\classe{vs}
\paradigme{\textit{dir :} \jya tɤ-}
\begin{définition}\fra adapté, convenable\end{définition}
\begin{définition}\cmn 合适\end{définition}
\begin{exemple}\jya ``a-pi" tu-kɯ-ti kɯ ɲɯ-nɤtsa lo\cmn 说“哥哥”比较合适(编故事的时候,豹子对马说话时用“哥哥”比较合适,因为马比较高大)\end{exemple}
\begin{exemple}\jya ki kowa ki mɯ́j-nɤtsa\cmn 这个办法不合适\end{exemple}
\begin{exemple}\jya nɤki tɤ-rte nɯ tu-tɯ-nɤrte ɲɯ-tɯ-nɤtsa\cmn 你适合戴这顶帽子\end{exemple}
\begin{relation-sémantique}\confer{
\hyperlink{Ⓔnɯtsa}{\textit{ \papi{nɯtsa}}}
}\end{relation-sémantique}\end{entrée}

\begin{entrée}
\vedette{\hypertarget{Ⓔnɤtsɤtkhɯ}{\papi{ nɤtsɤtkhɯ}}}\markboth{nɤtsɤtkhɯ}{}\classe{vs}
\begin{définition}\fra être obéissant\end{définition}
\begin{définition}\cmn 听指挥\end{définition}
\begin{exemple}\jya ki tɯrme ki mɤ-kɯ-nɤtsɤtkhɯ ci ɲɯ-ŋu\cmn 这个人不听话\end{exemple}
\begin{exemple}\jya a-mdʑu ki mɯ́j-nɤtsɤtkhɯ\cmn 我这个音发不出来\end{exemple}\end{entrée}

\begin{entrée}
\vedette{\hypertarget{Ⓔnɤtshɤxtshɯ}{\papi{ nɤtshɤxtshɯ}}}\markboth{nɤtshɤxtshɯ}{}
\classe{vt}
\paradigme{\textit{dir :} \jya tɤ-}
\begin{définition}\fra inciter, pousser\end{définition}
\begin{définition}\cmn 催\end{définition}
\begin{exemple}\jya mɯ́j-mbɣom tɕe tɤ-nɤtshɤxtshɯ-t-a\cmn 他很不积极,所以我就催了他一下\end{exemple}
\begin{exemple}\jya ma-tɤ-kɯ-nɤtshɤxtshɯ-a ma nɯ ma mɯ́j-cha-a\cmn 你不要催我\end{exemple}\end{entrée}

\begin{entrée}
\vedette{\hypertarget{Ⓔnɤtshɯtshɤt}{\papi{ nɤtshɯtshɤt}}}\markboth{nɤtshɯtshɤt}{}
\classe{vt}
\paradigme{\textit{dir :} \jya tɤ-}
\paradigme{\textit{dir :} \jya kɤ-}
\begin{définition}\ 
\begin{déclaration}\grammar{n.orient}\end{déclaration}\end{définition}
\begin{définition}\fra tenter de déterminer\end{définition}
\begin{définition}\cmn 试探\end{définition}
\begin{exemple}\jya ŋu ɕi maʁ, khramba tu-βze ŋu maʁ nɯ tɤ-nɤtshɯtshat-a\cmn 我试探了一下他是不是在说谎\end{exemple}
\begin{relation-sémantique}\confer{
\hyperlink{ⒺtshɤtⒽ1}{\textit{ \papi{tshɤt1}}}
}\end{relation-sémantique}\end{entrée}

\begin{entrée}
\vedette{\hypertarget{Ⓔnɤtsoʁ}{\papi{ nɤtsoʁ}}}\markboth{nɤtsoʁ}{}
\classe{vi}
\paradigme{\textit{dir :} \jya lɤ-}
\begin{définition}\ 
\begin{déclaration}\grammar{denom}\end{déclaration}\end{définition}
\begin{définition}\fra ramasser des gromas\end{définition}
\begin{définition}\cmn 挖人参果\end{définition}
\begin{exemple}\jya lɤ-nɤtsoʁ\cmn 他挖了人参果\end{exemple}
\begin{exemple}\jya pɯ-nɤtsoʁ-tɕi\cmn 我们俩挖了人参果\end{exemple}
\begin{relation-sémantique}\confer{
\hyperlink{Ⓔtɤtsoʁ}{\textit{ \papi{tɤtsoʁ}}}
}\end{relation-sémantique}\end{entrée}

\begin{entrée}
\vedette{\hypertarget{Ⓔnɤtsɯ}{\papi{ nɤtsɯ}}}\markboth{nɤtsɯ}{}
\classe{vt}
\paradigme{\textit{dir :} \jya kɤ-}
\begin{définition}\fra cacher, garder le secret\end{définition}
\begin{définition}\cmn 保密;藏起来\end{définition}
\begin{exemple}\jya aʑo kɯnɤ kɤ-nɤtsɯ-t-a\cmn 我也保密了\end{exemple}
\begin{exemple}\jya kɤ-tɯ-nɤtsɯ-t\cmn 你保密了\end{exemple}
\begin{exemple}\jya jiɕqha khɯtsa pɯ-tɯ-qrɯ-t nɯ kɤ-nɤtsi\cmn 你把碗打破了,你不要跟人家说\end{exemple}
\begin{relation-sémantique}\confer{
\hyperlink{Ⓔanɤtsɯtsɯ}{\textit{ \papi{anɤtsɯtsɯ}}}
}\end{relation-sémantique}
\begin{relation-sémantique}\confer{
 \papi{ɯ-tsɯ}
}\end{relation-sémantique}\begin{sous-entrée}
\vedette{\hypertarget{}{\papi{ ʑɣɤnɤtsɯ}}}\markboth{ʑɣɤnɤtsɯ}{}\classe{vi}
\paradigme{\textit{dir :} \jya kɤ-}
\begin{définition}\fra dissimuler (ses actions)\end{définition}
\begin{définition}\cmn 隐瞒(自己的行为)\end{définition}
\begin{exemple}\jya kɯmaʁ tɤ́-wɣ-nɤma qhe kɤ-ʑɣɤnɤtsɯ mɤ-phɤn, tɯrme ra kɯ ciz qhe sɯz-nɯ ɕti.\cmn 自己做了坏事的时候隐瞒是没有用的,因为人们始终是会知道的\end{exemple}
\end{sous-entrée}\end{entrée}

\begin{entrée}
\vedette{\hypertarget{Ⓔnɤtsɯmɣɯt}{\papi{ nɤtsɯmɣɯt}}}\markboth{nɤtsɯmɣɯt}{}
\classe{vt}
\paradigme{\textit{dir :} \jya tɤ-}
\begin{définition}\ 
\begin{déclaration}\grammar{comp}\end{déclaration}\end{définition}
\begin{définition}\fra déplacer les objets dans tous les sens\end{définition}
\begin{définition}\cmn 把东西拿过来拿过去\end{définition}
\begin{exemple}\jya aʑo tɤ-nɤtsɯmɣɯt-a\cmn 我拿过去拿过来了\end{exemple}
\begin{exemple}\jya tɤ-scoz kɯ-nɤtsɯmɣɯt\cmn 邮递员\end{exemple}
\begin{exemple}\jya kɤ-nɤtsɯmɣɯt mɤ-cha-a\cmn (我又不是邮递员),我不能把东西拿过去拿过来\end{exemple}
\begin{exemple}\jya tɯ-rju ɲɯ-ɤz-nɤtsɯmɣɯt\cmn 他在传播谣言\end{exemple}
\begin{relation-sémantique}\confer{
\hyperlink{Ⓔtsɯm}{\textit{ \papi{tsɯm}}}
}\end{relation-sémantique}
\begin{relation-sémantique}\confer{
\hyperlink{Ⓔɣɯt}{\textit{ \papi{ɣɯt}}}
}\end{relation-sémantique}\end{entrée}

\begin{entrée}
\vedette{\hypertarget{Ⓔnɤtsɯtsɯm}{\papi{ nɤtsɯtsɯm}}}\markboth{nɤtsɯtsɯm}{}
\begin{relation-sémantique}\confer{
\hyperlink{Ⓔtsɯm}{\textit{ \papi{tsɯm}}}
}\end{relation-sémantique}\end{entrée}

\begin{entrée}
\vedette{\hypertarget{Ⓔnɤtʂa}{\papi{ nɤtʂa}}}\markboth{nɤtʂa}{}
\classe{vs}
\paradigme{\textit{dir :} \jya tɤ-}
\paradigme{\textit{dir :} \jya \_}
\begin{définition}\fra serré à fond (le couvercle d'une boîte)\end{définition}
\begin{définition}\cmn 盖子盖得很紧\end{définition}
\begin{exemple}\jya ɯ-sti ɲɯ-nɤtʂa\cmn 塞得很紧\end{exemple}\begin{sous-entrée}
\vedette{\hypertarget{}{\papi{ znɤtʂa}}}\markboth{znɤtʂa}{}\classe{vt}
\paradigme{\textit{dir :} \jya \_}
\begin{définition}\ 
\begin{déclaration}\grammar{caus}\end{déclaration}\end{définition}
\begin{exemple}\jya ɯ-ŋgɯ nɯ-znɤtʂa-t-a\cmn 我在里面装得很紧\end{exemple}
\end{sous-entrée}\end{entrée}

\begin{entrée}
\vedette{\hypertarget{Ⓔnɤtʂaβlaβ}{\papi{ nɤtʂaβlaβ}}}\markboth{nɤtʂaβlaβ}{}
\classe{vt}
\paradigme{\textit{dir :} \jya \_}
\begin{définition}\ 
\begin{déclaration}\grammar{n.orient}\end{déclaration}\end{définition}
\begin{définition}\fra faire rouler dans tous les sens\end{définition}
\begin{définition}\cmn 使滚来滚去\end{définition}
\begin{exemple}\jya @lanqiu nɯ-nɤtʂaβlaβ-a\cmn 我让篮球滚来滚去了\end{exemple}
\begin{relation-sémantique}\confer{
\hyperlink{Ⓔtʂaβ}{\textit{ \papi{tʂaβ}}}
}\end{relation-sémantique}\end{entrée}

\begin{entrée}
\vedette{\hypertarget{Ⓔnɤtʂaŋ}{\papi{ nɤtʂaŋ}}}\markboth{nɤtʂaŋ}{}
\begin{relation-sémantique}\confer{
\hyperlink{Ⓔtʂaŋ}{\textit{ \papi{tʂaŋ}}}
}\end{relation-sémantique}\end{entrée}

\begin{entrée}
\vedette{\hypertarget{Ⓔnɤtʂɤtshi}{\papi{ nɤtʂɤtshi}}}\markboth{nɤtʂɤtshi}{}\classe{vt}
\paradigme{\textit{dir :} \jya \_}
\begin{définition}\ 
\begin{déclaration}\grammar{incorp}\end{déclaration}\end{définition}
\begin{définition}\fra bloquer le chemin\end{définition}
\begin{définition}\cmn 挡路\end{définition}
\begin{exemple}\jya ɯʑo kɯ tɤ́-wɣ-nɤtʂɤtshi-a\cmn 他挡了我的路\end{exemple}
\begin{relation-sémantique}\confer{
\hyperlink{ⒺtshiⒽ3}{\textit{ \papi{tshi3}}}
}\end{relation-sémantique}
\begin{relation-sémantique}\confer{
\hyperlink{Ⓔtʂu}{\textit{ \papi{tʂu}}}
}\end{relation-sémantique}
\end{entrée}

\begin{entrée}
\vedette{\hypertarget{Ⓔnɤtɯɣ}{\papi{ nɤtɯɣ}}}\markboth{nɤtɯɣ}{}
\begin{relation-sémantique}\confer{
\hyperlink{Ⓔatɯɣ}{\textit{ \papi{atɯɣ}}}
}\end{relation-sémantique}\end{entrée}

\begin{entrée}
\vedette{\hypertarget{Ⓔnɤtɯta}{\papi{ nɤtɯta}}}\markboth{nɤtɯta}{}
\begin{relation-sémantique}\confer{
\hyperlink{Ⓔta}{\textit{ \papi{ta}}}
}\end{relation-sémantique}\end{entrée}

\begin{entrée}
\vedette{\hypertarget{Ⓔnɤtɯti}{\papi{ nɤtɯti}}}\markboth{nɤtɯti}{}
\classe{vt}
\paradigme{\textit{dir :} \jya tɤ-}
\paradigme{\textit{dir :} \jya thɯ-}
\paradigme{\textit{past stem :} \jya nɤtɯtɯt}
\begin{définition}\ 
\begin{déclaration}\grammar{n.orient}\end{déclaration}\end{définition}
\begin{définition}\fra dire à tout le monde\end{définition}
\begin{définition}\cmn 到处说\end{définition}
\begin{exemple}\jya tɤ-nɤtɯtɯt-a\cmn 我到处说了\end{exemple}
\begin{exemple}\jya @kaihui ɲɯ-ra tɤ-nɤtɯti\cmn 你告诉大家需要开会\end{exemple}
\begin{exemple}\jya nɤʑo tɤ-tɯ-nɤtɯtɯt ɯ́-tu?\cmn 你到处说了没有\end{exemple}
\begin{relation-sémantique}\confer{
\hyperlink{Ⓔti}{\textit{ \papi{ti}}}
}\end{relation-sémantique}\end{entrée}

\begin{entrée}
\vedette{\hypertarget{Ⓔnɤwu}{\papi{ nɤwu}}}\markboth{nɤwu}{}\classe{vt}
\paradigme{\textit{dir :} \jya nɯ-}
\begin{définition}\fra pleurer pour\end{définition}
\begin{définition}\cmn 为……而哭\end{définition}
\begin{exemple}\jya ɲɯ-kɯ-nɤwu-a mɤ-ra\cmn 你不用为我哭\end{exemple}
\begin{exemple}\jya nɯ kɤ-nɤwu ci me nɤ\cmn 那没有什么好哭的\end{exemple}
\begin{relation-sémantique}\confer{
 \papi{tɤ-wu}
}\end{relation-sémantique}
\begin{relation-sémantique}\confer{
\hyperlink{Ⓔɣɤwu}{\textit{ \papi{ɣɤwu}}}
}\end{relation-sémantique}\end{entrée}

\begin{entrée}
\vedette{\hypertarget{Ⓔnɤwɤt}{\papi{ nɤwɤt}}}\markboth{nɤwɤt}{}\classe{vs}
\paradigme{\textit{dir :} \jya nɯ-}
\begin{définition}\fra qui s'ouvre vers l'extérieur en forme de cloche inversée\end{définition}
\begin{définition}\cmn 向外开着的形状(形成圆锥形)\end{définition}
\begin{exemple}\jya @beibei ɯ-mŋu ɲɯ-nɤwɤt ɲɯ-ŋu\cmn 杯子的口子向外开着\end{exemple}
\begin{relation-sémantique}\synonyme{
\hyperlink{Ⓔɣɤrɣɤr}{\textit{ \papi{ɣɤrɣɤr}}}
}\end{relation-sémantique}\end{entrée}

\begin{entrée}
\vedette{\hypertarget{Ⓔnɤwxti}{\papi{ nɤwxti}}}\markboth{nɤwxti}{}
\classe{vt}
\paradigme{\textit{dir :} \jya tɤ-}
\begin{définition}\ 
\begin{déclaration}\grammar{trop}\end{déclaration}\end{définition}
\begin{définition}\fra trouver trop grand\end{définition}
\begin{définition}\cmn 觉得太大\end{définition}
\begin{exemple}\jya ɯ-xtsa ɲɯ-nɤwxti\cmn 他觉得鞋子太大了\end{exemple}
\begin{exemple}\jya tɯ-xtsa tɤ-χtɯ-t-a ɲɯ-nɤwxti-a\cmn 我买了鞋子,但是觉得太大了\end{exemple}
\begin{exemple}\jya tɯ-ŋga tɤ-χtɯ-t-a nɯ ɲɯ-nɤwxti-a\cmn 我觉得衣服买太大了\end{exemple}
\begin{relation-sémantique}\confer{
\hyperlink{Ⓔwxti}{\textit{ \papi{wxti}}}
}\end{relation-sémantique}\end{entrée}

\begin{entrée}
\vedette{\hypertarget{Ⓔnɤxchi}{\papi{ nɤxchi}}}\markboth{nɤxchi}{}
\begin{relation-sémantique}\confer{
\hyperlink{Ⓔchi}{\textit{ \papi{chi}}}
}\end{relation-sémantique}\end{entrée}

\begin{entrée}
\vedette{\hypertarget{Ⓔnɤxɕɤt}{\papi{ nɤxɕɤt}}}\markboth{nɤxɕɤt}{}
\classe{vt}
\paradigme{\textit{dir :} \jya tɤ-}
\paradigme{\textit{dir :} \jya thɯ-}
\paradigme{\textit{dir :} \jya pɯ-}
\begin{définition}\ 
\begin{déclaration}\grammar{denom}\end{déclaration}\end{définition}\acception{1}
\begin{définition}\fra faire un gros effort\end{définition}
\begin{définition}\cmn 用力\end{définition}\acception{2}
\begin{définition}\fra faire qqch avec force\end{définition}
\begin{définition}\cmn 使劲\end{définition}
\begin{exemple}\jya ki rdɤstaʁ ɲɯ-rʑi ri, tɤ-nɤxɕat-a tɕe tɤ-mɟa-t-a\cmn 这块石头很重,我很用力捡起来了\end{exemple}
\begin{exemple}\jya pɯ-nɤxɕɤt tɕe pɯ-ɣɤrɤt\cmn 你用力扔吧\end{exemple}
\begin{exemple}\jya rɤɣo tha-nɤxɕɤt\cmn 他唱歌唱得很大声\end{exemple}
\begin{relation-sémantique}\confer{
\hyperlink{Ⓔtɯ-xɕɤt}{\textit{ \papi{tɯ-xɕɤt}}}
}\end{relation-sémantique}\end{entrée}

\begin{entrée}
\vedette{\hypertarget{Ⓔnɤxpe}{\papi{ nɤxpe}}}\markboth{nɤxpe}{}
\classe{vt}
\paradigme{\textit{dir :} \jya pɯ-}
\begin{définition}\ 
\begin{déclaration}\grammar{trop}\end{déclaration}\end{définition}
\begin{définition}\fra trouver bien\end{définition}
\begin{définition}\cmn 觉得很好\end{définition}
\begin{exemple}\jya pɯ-nɤxpe-t-a\cmn 我觉得很好了\end{exemple}
\begin{relation-sémantique}\confer{
\hyperlink{Ⓔpe}{\textit{ \papi{pe}}}
}\end{relation-sémantique}\end{entrée}

\begin{entrée}
\vedette{\hypertarget{Ⓔnɤxtɕɤβ}{\papi{ nɤxtɕɤβ}}}\markboth{nɤxtɕɤβ}{}
\begin{relation-sémantique}\confer{
\hyperlink{Ⓔɣɤxtɕɤβ}{\textit{ \papi{ɣɤxtɕɤβ}}}
}\end{relation-sémantique}\end{entrée}

\begin{entrée}
\vedette{\hypertarget{Ⓔnɤxtɕhɯβ}{\papi{ nɤxtɕhɯβ}}}\markboth{nɤxtɕhɯβ}{}\classe{vt}
\paradigme{\textit{dir :} \jya nɯ-}\acception{1}
\begin{définition}\fra s'appuyer sur, dépendre de\end{définition}
\begin{définition}\cmn 依靠\end{définition}
\begin{exemple}\jya kɤ-nɤma nɯ, tɯʑo tu-kɯ-nɯ-stu ɲɯ́-wɣ-nɤma ra ma tɯ-zda kɤ-nɤxtɕhɯβ mɤ-pe\cmn 在工作方面要自己努力,不要依靠别人\end{exemple}\acception{2}
\begin{définition}\fra profiter de\end{définition}
\begin{définition}\cmn 趁着\end{définition}
\begin{exemple}\jya ɯʑo ju-kɯ-ɕe nɯ nɯ-nɤxtɕhɯβ-a tɕe ɯ-rca jɤ-ari-a pɯ-ŋu\cmn 我趁着他去那边跟他一起去\end{exemple}
\begin{relation-sémantique}\confer{
\hyperlink{Ⓔɯ-tɕhɯβ}{\textit{ \papi{ɯ-tɕhɯβ}}}
}\end{relation-sémantique}\end{entrée}

\begin{entrée}
\vedette{\hypertarget{Ⓔnɤxtɕi}{\papi{ nɤxtɕi}}}\markboth{nɤxtɕi}{}
\classe{vt}
\paradigme{\textit{dir :} \jya tɤ-}
\begin{définition}\ 
\begin{déclaration}\grammar{trop}\end{déclaration}\end{définition}
\begin{définition}\fra trouver trop petit\end{définition}
\begin{définition}\cmn 觉得太小\end{définition}
\begin{exemple}\jya aʑo ɲɯ-nɤxtɕi-a\cmn 我觉得太小\end{exemple}
\begin{exemple}\jya nɤ-mi ɲɯ-wxti, nɤ-xtsa ɲɯ-tɯ-nɤxtɕi\cmn 你脚很大,鞋子太小了\end{exemple}
\begin{exemple}\jya kɯki ɯ-spa mɯ́j-rtaʁ, ɲɯ-nɤxtɕi-a\cmn 材料不够,我觉得太小\end{exemple}
\begin{relation-sémantique}\confer{
\hyperlink{Ⓔxtɕi}{\textit{ \papi{xtɕi}}}
}\end{relation-sémantique}\end{entrée}

\begin{entrée}
\vedette{\hypertarget{Ⓔnɤxtɕur}{\papi{ nɤxtɕur}}}\markboth{nɤxtɕur}{}
\begin{relation-sémantique}\confer{
\hyperlink{ⒺtɕurⒽ1}{\textit{ \papi{tɕur1}}}
}\end{relation-sémantique}\end{entrée}

\begin{entrée}
\vedette{\hypertarget{Ⓔnɤxthɯ}{\papi{ nɤxthɯ}}}\markboth{nɤxthɯ}{}
\classe{vt}
\paradigme{\textit{dir :} \jya nɯ-}
\begin{définition}\ 
\begin{déclaration}\grammar{trop}\end{déclaration}\end{définition}
\begin{définition}\fra trouver grave\end{définition}
\begin{définition}\cmn 觉得严重\end{définition}
\begin{exemple}\jya jɤxtshi tɕhomba aʑo pɯ-nɤxthɯ-t-a\cmn 这一次感冒,我觉得很严重了\end{exemple}
\begin{exemple}\jya jɤxtshi tɕhomba, ɲɯ-tɯ-nɤxthi\cmn 这一次感冒,你觉得很严重\end{exemple}
\begin{exemple}\jya ɯ-kɯ-mŋɤm ɲɯ-nɤxthi\cmn 他觉得很严重\end{exemple}
\begin{relation-sémantique}\confer{
\hyperlink{ⒺthɯⒽ2}{\textit{ \papi{thɯ2}}}
}\end{relation-sémantique}\end{entrée}

\begin{entrée}
\vedette{\hypertarget{Ⓔnɤxtsa}{\papi{ nɤxtsa}}}\markboth{nɤxtsa}{}
\classe{vt}
\paradigme{\textit{dir :} \jya tɤ-}
\begin{définition}\fra forcer\end{définition}
\begin{définition}\cmn 撬\end{définition}
\begin{exemple}\jya sɤcɯ tɤ-nɤxtsa-t-a\cmn 我把锁撬开了\end{exemple}
\begin{exemple}\jya kɯm tɤ-nɤxtsa-t-a\cmn 我把门撬开了\end{exemple}\end{entrée}

\begin{entrée}
\vedette{\hypertarget{Ⓔnɤxtʂɯ}{\papi{ nɤxtʂɯ}}}\markboth{nɤxtʂɯ}{}\classe{vt}
\paradigme{\textit{dir :} \jya kɤ-}
\paradigme{\textit{dir :} \jya nɯ-}
\paradigme{\textit{dir :} \jya lɤ-}
\paradigme{\textit{dir :} \jya thɯ-}
\begin{définition}\fra apporter en passant\end{définition}
\begin{définition}\cmn 顺便带\end{définition}
\begin{exemple}\jya aʑo ɲɯ-nɤxtʂi-a\cmn 我顺便带过去\end{exemple}
\begin{exemple}\jya nɤʑo nɯ-nɤxtʂi\cmn 你顺便带过来吧\end{exemple}
\begin{exemple}\jya tɕɤndi jiʑo ji-laχtɕha ata tɕe kɤ-nɤxtʂi\cmn 我们的东西在那边,你顺便带过来吧\end{exemple}
\begin{exemple}\jya nɤʑɯɣ kɤ-nɤxtʂɯ-t-a\cmn 我顺便带给你了\end{exemple}
\begin{relation-sémantique}\synonyme{
\hyperlink{Ⓔnɯpjaχpa}{\textit{ \papi{nɯpjaχpa}}}
}\end{relation-sémantique}\end{entrée}

\begin{entrée}
\vedette{\hypertarget{Ⓔnɤxtʂɯn}{\papi{ nɤxtʂɯn}}}\markboth{nɤxtʂɯn}{}
\classe{vt}
\paradigme{\textit{dir :} \jya nɯ-}
\begin{définition}\ 
\begin{déclaration}\grammar{trop}\end{déclaration}\end{définition}
\begin{définition}\fra remercier, avoir de la reconnaissance pour\end{définition}
\begin{définition}\cmn 感激
\begin{déclaration} \étymologie{\papi{drin}}\end{déclaration}\end{définition}
\begin{exemple}\jya nɯ-nɤxtʂɯn-a\cmn 我很感激了\end{exemple}
\begin{exemple}\jya ɲɯ-nɤxtʂɯn\cmn 他很感激\end{exemple}
\begin{exemple}\jya laχtɕha nɯ-kɯ-mbi-a pɯ-nɤxtʂɯn-a\cmn 我很感激你把东西给我了\end{exemple}
\begin{exemple}\jya aʑo nɯ-mbi-t-a tɕe ɲɯ-nɤxtʂɯn\cmn 我给了他,他很感激\end{exemple}
\begin{relation-sémantique}\confer{
\hyperlink{Ⓔtɯ-tʂɯn}{\textit{ \papi{tɯ-tʂɯn}}}
}\end{relation-sémantique}\end{entrée}

\begin{entrée}
\vedette{\hypertarget{Ⓔnɤxtɯt}{\papi{ nɤxtɯt}}}\markboth{nɤxtɯt}{}
\classe{vt}
\paradigme{\textit{dir :} \jya tɤ-}
\begin{définition}\ 
\begin{déclaration}\grammar{trop}\end{déclaration}\end{définition}
\begin{définition}\fra trouver trop court\end{définition}
\begin{définition}\cmn 觉得太短\end{définition}
\begin{exemple}\jya a-ŋga ɲɯ-nɤxtɯt-a\cmn 我觉得衣服太小\end{exemple}
\begin{relation-sémantique}\confer{
\hyperlink{ⒺxtɯtⒽ1}{\textit{ \papi{xtɯt}}}
}\end{relation-sémantique}\end{entrée}

\begin{entrée}
\vedette{\hypertarget{Ⓔnɤχɤmthi}{\papi{ nɤχɤmthi}}}\markboth{nɤχɤmthi}{}\classe{vi}
\paradigme{\textit{dir :} \jya thɯ-}
\begin{définition}\fra être bouche bée\end{définition}
\begin{définition}\cmn 目瞪口呆\end{définition}
\begin{exemple}\jya thɯ-nɤχɤmthi-a\cmn 我目瞪口呆了\end{exemple}
\begin{relation-sémantique}\synonyme{
\hyperlink{Ⓔnɯcaχto}{\textit{ \papi{nɯcaχto}}}
}\end{relation-sémantique}
\begin{sous-entrée}
\vedette{\hypertarget{}{\papi{ znɤχɤmthi}}}\markboth{znɤχɤmthi}{}\classe{vt}
\paradigme{\textit{dir :} \jya thɯ-}
\begin{définition}\fra rendre bouche bée\end{définition}
\begin{définition}\cmn 令人目瞪口呆\end{définition}
\begin{exemple}\jya chɤ́-wɣ-znɤχɤmthi ʑo\cmn 令他目瞪口呆\end{exemple}
\begin{relation-sémantique}\synonyme{
\hyperlink{Ⓔznɯcaχto}{\textit{ \papi{znɯcaχto}}}
}\end{relation-sémantique}
\end{sous-entrée}\end{entrée}

\begin{entrée}
\vedette{\hypertarget{Ⓔnɤχcɤl}{\papi{ nɤχcɤl}}}\markboth{nɤχcɤl}{}
\classe{vi}
\paradigme{\textit{dir :} \jya \_}
\begin{définition}\ 
\begin{déclaration}\grammar{denom}\end{déclaration}\end{définition}
\begin{définition}\fra être au milieu\end{définition}
\begin{définition}\cmn 到中间
\begin{déclaration} \étymologie{\papi{dkʲil}}\end{déclaration}\end{définition}
\begin{exemple}\jya tɯ-ci kɤ-nɤχcɤl (tɯ-ci ɯ-ŋgɯ kɤ-ndʑaʁ tɕe kɤ-nɤχcɤl)\cmn 他到水的中间了\end{exemple}
\begin{exemple}\jya ndzom ɯ-taʁ kɤ-nɤχcɤl\cmn 他到桥的中间\end{exemple}
\begin{exemple}\jya ki kɤ-nɤχcal-a\cmn 我到中间了(例如,我看书看到一半)\end{exemple}
\begin{exemple}\jya tɯ-ci kɤ-nɤχcal-a\cmn 我到了河的中间(游过去)\end{exemple}
\begin{exemple}\jya tʂu jɤ-nɤχcal-a ʑo\cmn 我走了一半的路\end{exemple}
\begin{relation-sémantique}\confer{
\hyperlink{Ⓔɯ-χcɤl}{\textit{ \papi{ɯ-χcɤl}}}
}\end{relation-sémantique}\end{entrée}

\begin{entrée}
\vedette{\hypertarget{Ⓔnɤχphe}{\papi{ nɤχphe}}}\markboth{nɤχphe}{}
\classe{vt}
\paradigme{\textit{dir :} \jya tɤ-}
\paradigme{\textit{dir :} \jya pɯ-}
\begin{définition}\ 
\begin{déclaration}\grammar{denom}\end{déclaration}\end{définition}
\begin{définition}\fra frapper\end{définition}
\begin{définition}\cmn 拍(用手掌)\end{définition}
\begin{exemple}\jya aʑo a-xtu tɤ-nɤχphe-t-a\cmn 我拍了我的肚子\end{exemple}
\begin{exemple}\jya ɯʑo kɯ ɯ-xtu ta-nɤχphe\cmn 他拍了自己的肚子\end{exemple}
\begin{exemple}\jya tɕoχtsi pɯ-nɤχphe-t-a (=tɕoχtsi ɯ-taʁ taχphe pɯ-lat-a)\cmn 我拍了桌子\end{exemple}
\begin{relation-sémantique}\confer{
\hyperlink{Ⓔtaχphe}{\textit{ \papi{taχphe}}}
}\end{relation-sémantique}\end{entrée}

\begin{entrée}
\vedette{\hypertarget{Ⓔnɤz}{\papi{ nɤz}}}\markboth{nɤz}{}\classe{vi}
\paradigme{\textit{dir :} \jya tɤ-}
\begin{définition}\fra oser\end{définition}
\begin{définition}\cmn 敢\end{définition}
\begin{exemple}\jya ɲɯ-nɤz\cmn 他敢\end{exemple}
\begin{exemple}\jya aʑo nɯ tu-ti-a naz-a\cmn 我敢说这个\end{exemple}
\begin{exemple}\jya aj ku-ɕe-a naz-a\cmn 我敢去\end{exemple}
\begin{exemple}\jya ki ɯ-ʁɤri mɯ-pɯ-naz-a, tham to-naz-a\cmn 我以前不敢,现在敢了\end{exemple}
\begin{exemple}\jya a-sɯm kɯ-kɯ-fse nɯ kɤ-ti mɤ-naz-a\cmn 我不敢说我的心里话\end{exemple}\begin{sous-entrée}
\vedette{\hypertarget{}{\papi{ sɤnɤz}}}\markboth{sɤnɤz}{}\classe{vs}
\begin{définition}\ 
\begin{déclaration}\grammar{deexp}\end{déclaration}\end{définition}
\begin{exemple}\jya ki khɯna kɯ-ɤtɯɣ mɯ́j-sɤnɤz ma ku-kɯ-mtsɯɣ ɲɯ-ɕti\cmn 不敢碰到这条狗,因为会咬人\end{exemple}
\begin{exemple}\jya tɤ-pɤtso kɤ-sɯsɤβlo mɯ́j-sɤnɤz\cmn 不敢让他看小孩子\end{exemple}
\end{sous-entrée}\begin{sous-entrée}
\vedette{\hypertarget{}{\papi{ sɯɣnɤz}}}\markboth{sɯɣnɤz}{}\classe{vt}
\paradigme{\textit{dir :} \jya nɯ-}
\begin{définition}\fra faire oser\end{définition}
\begin{définition}\cmn 令……敢\end{définition}
\begin{exemple}\jya ɯʑo kɯ nɯ ɲɯ-ti tɕe kɤ-ɕe mɯ-nɯ́-wɣ-sɯɣnaz-a\cmn 他讲的话令我不敢去了\end{exemple}
\end{sous-entrée}\end{entrée}

\begin{entrée}
\vedette{\hypertarget{Ⓔnɤzda}{\papi{ nɤzda}}}\markboth{nɤzda}{}
\classe{vt}
\paradigme{\textit{dir :} \jya tɤ-}
\begin{définition}\ 
\begin{déclaration}\grammar{denom}\end{déclaration}\end{définition}
\begin{définition}\fra inviter quelqu'un à se joindre à son groupe\end{définition}
\begin{définition}\cmn 请别人加入自己的队伍\end{définition}
\begin{exemple}\jya aʑo tɤ-nɤzda-t-a\cmn 我请他加入了\end{exemple}
\begin{exemple}\jya ɯʑo kɯ tɤ́-wɣ-nɤzda-a\cmn 他请我加入了\end{exemple}
\begin{exemple}\jya jisŋi tɤjmɤɣ ɯ-kɯ-ɕar aj ɕe-a ŋu, tu-ta-nɤzda ɯ-tɯ-ɣi?\cmn 今天我去找蘑菇,请你一起去好吗?\end{exemple}
\begin{exemple}\jya kɯ-ɣɯɕoŋtɕa aj ɕe-a ŋu, tu-ta-nɤzda ɯ-tɯ-ɣi?\cmn 我去砍木头,请你一起去好吗?\end{exemple}
\begin{exemple}\jya nɤʑo si ɕɯ-tɯ-phɯt ɲɯ-ŋu, tu-kɯ-nɤzda-a tɕe aj kɯnɤ ɣi-a\cmn 你要去砍树,可以结伴一起去\end{exemple}
\begin{exemple}\jya tu-kɯ-nɤzda-a ɯ́-jɤɣ\cmn 我能不能跟你一起去\end{exemple}
\begin{exemple}\jya tu-ta-nɤzda ɕe-tɕi\cmn 我们俩一起去\end{exemple}
\begin{relation-sémantique}\confer{
\hyperlink{Ⓔtɯ-zda}{\textit{ \papi{tɯ-zda}}}
}\end{relation-sémantique}
\begin{relation-sémantique}\confer{
\hyperlink{Ⓔɣɤzda}{\textit{ \papi{ɣɤzda}}}
}\end{relation-sémantique}
\begin{relation-sémantique}\confer{
\hyperlink{Ⓔrɤzda}{\textit{ \papi{rɤzda}}}
}\end{relation-sémantique}
\begin{relation-sémantique}\confer{
\hyperlink{Ⓔsɤzda}{\textit{ \papi{sɤzda}}}
}\end{relation-sémantique}\end{entrée}

\begin{entrée}
\vedette{\hypertarget{Ⓔnɤzɲɟoʁ}{\papi{ nɤzɲɟoʁ}}}\markboth{nɤzɲɟoʁ}{}\classe{vt}
\paradigme{\textit{dir :} \jya tɤ-}
\begin{définition}\fra frapper, fouetter\end{définition}
\begin{définition}\cmn 抽打(用鞭子或者木棒)\end{définition}
\begin{exemple}\jya nɤ-stu tɤ-fse ma ta-nɤzɲɟoʁ\cmn 你小心一点,不然我会抽打你的\end{exemple}
\begin{relation-sémantique}\confer{
\hyperlink{Ⓔtɤzɲɟoʁ}{\textit{ \papi{tɤzɲɟoʁ}}}
}\end{relation-sémantique}\end{entrée}

\begin{entrée}
\vedette{\hypertarget{Ⓔnɤzraʁ}{\papi{ nɤzraʁ}}}\markboth{nɤzraʁ}{}
\classe{vi}
\paradigme{\textit{dir :} \jya nɯ-}
\begin{définition}\ 
\begin{déclaration}\grammar{denom}\end{déclaration}\end{définition}
\begin{définition}\fra avoir honte, être gêné, se sentir embarrassé\end{définition}
\begin{définition}\cmn 害羞\end{définition}
\begin{exemple}\jya tɕheme kɤ-rqoʁ mɯ́j-nɤzraʁ\cmn 他抱女孩子不感到害羞\end{exemple}
\begin{exemple}\jya khramba tu-βze ɲɯ-ŋu, mɯ́j-nɤzraʁ\cmn 他说谎不感到害羞\end{exemple}
\begin{exemple}\jya jiɕqha nɯ kɯ thɯci ɲɯ-ti, aʑo nɯ-nɤzraʁ-a\cmn 这个人说了一些什么话,我感到很害羞\end{exemple}
\begin{relation-sémantique}\confer{
\hyperlink{Ⓔtɤzraʁ}{\textit{ \papi{tɤzraʁ}}}
}\end{relation-sémantique}\begin{sous-entrée}
\vedette{\hypertarget{}{\papi{ sɤzraʁ}}}\markboth{sɤzraʁ}{}\classe{vs}
\begin{définition}\fra honteux\end{définition}
\begin{définition}\cmn 可耻\end{définition}
\begin{exemple}\jya kɤ-mɯrkɯ ndɤre ɲɯ-sɤzraʁ\cmn 偷东西是可耻的行为\end{exemple}
\end{sous-entrée}\begin{sous-entrée}
\vedette{\hypertarget{}{\papi{ znɤzraʁ}}}\markboth{znɤzraʁ}{}\classe{vt}
\paradigme{\textit{dir :} \jya nɯ-}
\begin{définition}\fra embarrasser\end{définition}
\begin{définition}\cmn 令……不好意思\end{définition}
\begin{exemple}\jya nɯ ma-tɤ-tɯ-ti ma tɯ-znɤzraʁ\cmn 你不要说这些话,让他不好意思(吃)\end{exemple}
\end{sous-entrée}\end{entrée}

\begin{entrée}
\vedette{\hypertarget{Ⓔnɤzri}{\papi{ nɤzri}}}\markboth{nɤzri}{}
\classe{vt}
\paradigme{\textit{dir :} \jya tɤ-}
\begin{définition}\ 
\begin{déclaration}\grammar{trop}\end{déclaration}\end{définition}
\begin{définition}\fra trouver trop long\end{définition}
\begin{définition}\cmn 觉得太长\end{définition}
\begin{exemple}\jya a-ŋga tɤ-χtɯ-t-a nɯ ɲɯ-nɤzri-a\cmn 我觉得衣服买太长了\end{exemple}
\begin{relation-sémantique}\confer{
\hyperlink{Ⓔzri}{\textit{ \papi{zri}}}
}\end{relation-sémantique}\end{entrée}

\begin{entrée}
\vedette{\hypertarget{Ⓔnɤʑɤmŋɤn}{\papi{ nɤʑɤmŋɤn}}}\markboth{nɤʑɤmŋɤn}{} (\variante{nɯʑɤmŋɤn}) \classe{vt}
\paradigme{\textit{dir :} \jya tɤ-}
\begin{définition}\fra envier\end{définition}
\begin{définition}\cmn 妒忌
\begin{déclaration} \étymologie{\papi{ʑe.ŋan}}\end{déclaration}\end{définition}
\begin{exemple}\jya aʑo laχtɕha tɤ-nɯ-χtɯ-t-a tɕe, ɲɯ-kɯ-nɯʑɤmŋan-a\cmn 我给自己买了东西,你在妒忌我\end{exemple}\begin{sous-entrée}
\vedette{\hypertarget{}{\papi{ anɤʑɤmŋɯmŋɤn}}}\markboth{anɤʑɤmŋɯmŋɤn}{}\classe{vi}
\begin{définition}\fra s'envier les uns les autres\end{définition}
\begin{définition}\cmn 互相妒忌\end{définition}
\end{sous-entrée}\begin{sous-entrée}
\vedette{\hypertarget{}{\papi{ sɤnɯʑɤmŋɤn}}}\markboth{sɤnɯʑɤmŋɤn}{}\classe{vi}
\paradigme{\textit{dir :} \jya tɤ-}
\begin{définition}\fra envier les gens\end{définition}
\begin{définition}\cmn 妒忌别人\end{définition}
\begin{exemple}\jya ma-tɤ-tɯ-sɤnɤʑɤmŋɤn\cmn 你不要妒忌别人\end{exemple}
\end{sous-entrée}\end{entrée}

\begin{entrée}
\vedette{\hypertarget{Ⓔnɤʑo}{\papi{ nɤʑo}}}\markboth{nɤʑo}{}\classe{pro}
\begin{définition}\fra toi\end{définition}
\begin{définition}\cmn 你\end{définition}
\begin{relation-sémantique}\confer{
\hyperlink{Ⓔnɤj}{\textit{ \papi{nɤj}}}
}\end{relation-sémantique}
\end{entrée}

\begin{entrée}
\vedette{\hypertarget{Ⓔnɤʑru}{\papi{ nɤʑru}}}\markboth{nɤʑru}{}
\begin{relation-sémantique}\confer{
\hyperlink{Ⓔʑru}{\textit{ \papi{ʑru}}}
}\end{relation-sémantique}\end{entrée}

\begin{entrée}
\vedette{\hypertarget{Ⓔnɤʑri}{\papi{ nɤʑri}}}\markboth{nɤʑri}{}
\classe{vi}
\paradigme{\textit{dir :} \jya nɯ-}
\begin{définition}\ 
\begin{déclaration}\grammar{denom}\end{déclaration}\end{définition}
\begin{définition}\fra être mouillé par la rosée\end{définition}
\begin{définition}\cmn 沾上露水\end{définition}
\begin{exemple}\jya tɯ-mɯ pjɤ-lɤt tɕe aj nɯ-nɤʑri-a\cmn 下雨了,我沾上了露水\end{exemple}
\begin{relation-sémantique}\confer{
\hyperlink{Ⓔtɤʑri}{\textit{ \papi{tɤʑri}}}
}\end{relation-sémantique}\end{entrée}

\begin{entrée}
\vedette{\hypertarget{Ⓔnɤʑɯloʁ}{\papi{ nɤʑɯloʁ}}}\markboth{nɤʑɯloʁ}{}
\classe{vi}
\paradigme{\textit{dir :} \jya nɯ-}
\paradigme{\textit{dir :} \jya pɯ-}
\begin{définition}\ 
\begin{déclaration}\grammar{incorp}\end{déclaration}\end{définition}
\begin{définition}\fra avoir la nausée\end{définition}
\begin{définition}\cmn 感到恶心;觉得恶心\end{définition}
\begin{exemple}\jya pɯ-nɤʑɯloʁ-a\cmn 我感到恶心\end{exemple}
\begin{exemple}\jya tɤ-pɤtso kɯ ɯ-qe na-lɤt tɕe pɯ-nɤʑɯloʁ-a\cmn 小孩子屙了屎,我觉得很恶心\end{exemple}
\begin{exemple}\jya khɯna lo-qioʁ tɕe pɯ-nɤʑɯloʁ-a\cmn 狗在那里吐了,我觉得很恶心\end{exemple}
\begin{relation-sémantique}\confer{
\hyperlink{Ⓔsɤʑɯloʁ}{\textit{ \papi{sɤʑɯloʁ}}}
}\end{relation-sémantique}
\begin{relation-sémantique}\confer{
\hyperlink{Ⓔtɯ-ʑi,loʁ}{\textit{ \papi{tɯ-ʑi,loʁ}}}
}\end{relation-sémantique}\end{entrée}

\begin{entrée}
\vedette{\hypertarget{Ⓔnɤʑɯn}{\papi{ nɤʑɯn}}}\markboth{nɤʑɯn}{}
\classe{vs}
\begin{définition}\fra en pente\end{définition}
\begin{définition}\cmn 陡;斜度高\end{définition}
\begin{exemple}\jya tʂu ki pjɯ-nɤʑɯn ɲɯ-ŋu, tɕe mɯ́j-nɯɣɯŋke ri, mɯ́j-ʁdɯɣ ma ɲɯ-nɯtɕhɯrɟɯɣ tɕe ɲɯ-ɣɤzbaʁ\cmn 这条路很陡,虽然不好走,但是水流得很快(下雨的时候不会积水,路面就很快干)\end{exemple}
\begin{relation-sémantique}\confer{
\hyperlink{Ⓔɣɤʑɯn}{\textit{ \papi{ɣɤʑɯn}}}
}\end{relation-sémantique}
\begin{relation-sémantique}\confer{
\hyperlink{Ⓔɣɤʑɯn}{\textit{ \papi{ɣɤʑɯn}}}
}\end{relation-sémantique}\end{entrée}

\begin{entrée}
\vedette{\hypertarget{Ⓔnɤʑɯʑu}{\papi{ nɤʑɯʑu}}}\markboth{nɤʑɯʑu}{}
\begin{relation-sémantique}\confer{
\hyperlink{Ⓔaʑɯʑu}{\textit{ \papi{aʑɯʑu}}}
}\end{relation-sémantique}\end{entrée}

\begin{entrée}
\vedette{\hypertarget{Ⓔnbraʁ}{\papi{ nbraʁ}}}\markboth{nbraʁ}{}
\classe{vt}
\paradigme{\textit{dir :} \jya lɤ-}
\begin{définition}\fra rendre la terre plus meuble\end{définition}
\begin{définition}\cmn 锄(麦,青稞)\end{définition}
\begin{exemple}\jya tɤɕi tɤ-nbraʁ-a\cmn 我锄了地种青稞\end{exemple}
\begin{exemple}\jya tɯ-ji ɯ-qhu kɯ-ɤrmbat tɕe, kɤ-nbraʁ tu-mda ŋu\cmn 收割之后不久,就是松土的时候\end{exemple}\end{entrée}

\begin{entrée}
\vedette{\hypertarget{Ⓔndʐa}{\papi{ ndʐa}}}\markboth{ndʐa}{}
\classe{n}
\begin{définition}\fra cause\end{définition}
\begin{définition}\cmn 原因\end{définition}
\begin{exemple}\jya aʑɯɣ ndʐa ŋu ɕi, nɤʑɯɣ ndʐa ŋu, aj mɯ́j-tso-a\cmn 不知道是你的原因,还是我的原因\end{exemple}\end{entrée}

\begin{entrée}
\vedette{\hypertarget{Ⓔndʐaβ}{\papi{ ndʐaβ}}}\markboth{ndʐaβ}{}
\classe{vi}
\paradigme{\textit{dir :} \jya pɯ-}
\paradigme{\textit{dir :} \jya thɯ-}
\paradigme{\textit{dir :} \jya \_}
\begin{définition}\ 
\begin{déclaration}\grammar{acaus}\end{déclaration}\end{définition}\acception{1}
\begin{définition}\fra tomber à la renverse\end{définition}
\begin{définition}\cmn 摔倒\end{définition}
\begin{exemple}\jya ɯʑo pjɤ-ndʐaβ\cmn 他摔倒了\end{exemple}
\begin{exemple}\jya pɯ-ndʐaβ-a\cmn 我摔倒了\end{exemple}\acception{2}
\begin{définition}\fra dégringoler, rouler\end{définition}
\begin{définition}\cmn 滚下\end{définition}
\begin{exemple}\jya tʂu mɯ́j-sɤɣa tɕe pjɤ-ndʐaβ\cmn 路不安全,他滚下去了\end{exemple}
\begin{exemple}\jya nɯŋa alo pjɤ-ndʐaβ\cmn 奶牛在那里滚下去了\end{exemple}\acception{3}
\begin{définition}\fra s'effondrer (arbre, mur)\end{définition}
\begin{définition}\cmn 倒塌\end{définition}
\begin{exemple}\jya znde cho-ndʐaβ\cmn 墙倒下了\end{exemple}
\begin{relation-sémantique}\confer{
\hyperlink{Ⓔtʂaβ}{\textit{ \papi{tʂaβ}}}
}\end{relation-sémantique}
\begin{relation-sémantique}\confer{
\hyperlink{Ⓔsɤndʐaβ}{\textit{ \papi{sɤndʐaβ}}}
}\end{relation-sémantique}\end{entrée}

\begin{entrée}
\vedette{\hypertarget{Ⓔndɤɣ}{\papi{ ndɤɣ}}}\markboth{ndɤɣ}{}\classe{vs}
\begin{définition}\fra avoir trempé (assez)\end{définition}
\begin{définition}\cmn 泡好\end{définition}
\begin{exemple}\jya tɤjlu ɲɤ-ndɤɣ\cmn 面粉泡好了\end{exemple}
\begin{exemple}\jya nɤ-ŋga pɯ-ɣɤle a-nɯ-ndɤɣ tɕe kɤ-χtɕi mbat\cmn 把衣服泡好,这样就容易洗干净\end{exemple}\begin{sous-entrée}
\vedette{\hypertarget{}{\papi{ sɯɣndɤɣ}}}\markboth{sɯɣndɤɣ}{}\classe{vt}
\paradigme{\textit{dir :} \jya pɯ-}
\paradigme{\textit{dir :} \jya thɯ-}
\paradigme{\textit{dir :} \jya nɯ-}
\begin{définition}\ 
\begin{déclaration}\grammar{caus}\end{déclaration}\end{définition}
\begin{définition}\fra tremper\end{définition}
\begin{définition}\cmn 浸泡\end{définition}
\begin{exemple}\jya paʁtshi chɤ-sɯɣndɤɣ\cmn 他把猪食浸泡了\end{exemple}
\begin{exemple}\jya tɯ-ŋga ko-ɴqhi tɕe pjɯ́-wɣ-sɯɣndɤɣ ɲɯ-ɬoʁ\cmn 衣服脏了就要浸泡\end{exemple}
\end{sous-entrée}\end{entrée}

\begin{entrée}
\vedette{\hypertarget{Ⓔndɤre}{\papi{ ndɤre}}}\markboth{ndɤre}{}\classe{cnj}
\begin{définition}\fra par contre\end{définition}
\begin{définition}\cmn 反而\end{définition}\end{entrée}

\begin{entrée}
\vedette{\hypertarget{Ⓔndɤrmbjom}{\papi{ ndɤrmbjom}}}\markboth{ndɤrmbjom}{}\classe{vs}
\begin{définition}\fra aux mouvements rapides\end{définition}
\begin{définition}\cmn 勤快,动作伶俐
\end{définition}
\begin{relation-sémantique}\synonyme{
\hyperlink{Ⓔɣɤphɯɕlaʁ}{\textit{ \papi{ɣɤphɯɕlaʁ}}}
}\end{relation-sémantique}
\begin{relation-sémantique}\confer{
\hyperlink{Ⓔmbjom}{\textit{ \papi{mbjom}}}
}\end{relation-sémantique}\begin{sous-entrée}
\vedette{\hypertarget{}{\papi{ sɯndɤrmbjom}}}\markboth{sɯndɤrmbjom}{}\classe{vs}
\begin{définition}\fra rendre rapide\end{définition}
\begin{définition}\cmn 令……动作伶俐\end{définition}
\begin{exemple}\jya tɤ-pɤtso ɲɯ́-wɣ-sɯxtɕɤt tɕe kɤ-sɯndɤrmbjom tu\cmn 对孩子教导好了的话,就可以令他变得勤快\end{exemple}
\end{sous-entrée}\end{entrée}

\begin{entrée}
\vedette{\hypertarget{Ⓔndɤrndɤr}{\papi{ ndɤrndɤr}}}\markboth{ndɤrndɤr}{}\classe{idph.2}
\begin{définition}\fra grand, imposant\end{définition}
\begin{définition}\cmn 强壮\end{définition}
\begin{exemple}\jya jla ci rcanɯ kɯ-wxti ndɤrndɤr ci ɲɯ-ŋu\cmn 犏牛很强壮\end{exemple}
\begin{exemple}\jya ɯʑo ɯ-phoŋbu ndɤrndɤr ʑo ɲɯ-pa\cmn 他身材很高大\end{exemple}\begin{sous-entrée}
\vedette{\hypertarget{}{\papi{ mɤlɤndɤr}}}\markboth{mɤlɤndɤr}{}
\begin{définition}\fra imposant\end{définition}
\begin{définition}\cmn 高大\end{définition}
\begin{exemple}\jya kha rcanɯ mɤlɤndɤr ci ɲɯ-ŋu\cmn 房子很高大\end{exemple}
\end{sous-entrée}\end{entrée}

\begin{entrée}
\vedette{\hypertarget{Ⓔndɣɤndɣɤt}{\papi{ ndɣɤndɣɤt}}}\markboth{ndɣɤndɣɤt}{}
\classe{idph.2}
\begin{définition}\fra (bois) entassé, très haut\end{définition}
\begin{définition}\cmn 形容柴堆得很高的样子\end{définition}
\begin{exemple}\jya sɯpɣo ndɣɤndɣɤt ʑo pjɤ-ta\cmn 柴垛子堆得很高\end{exemple}
\begin{relation-sémantique}\confer{
\hyperlink{Ⓔɣɤndɣɤndɣɤt}{\textit{ \papi{ɣɤndɣɤndɣɤt}}}
}\end{relation-sémantique}\end{entrée}

\begin{entrée}
\vedette{\hypertarget{Ⓔndɣɤt}{\papi{ ndɣɤt}}}\markboth{ndɣɤt}{}\classe{idph.1}
\begin{définition}\fra sursaut de frayeur\end{définition}
\begin{définition}\cmn (吓了)一跳\end{définition}
\begin{exemple}\jya ndɣɤt ʑo pjɤ́-wɣ-znɤscɤr\cmn 把他吓了一跳\end{exemple}\end{entrée}

\begin{entrée}
\vedette{\hypertarget{Ⓔndʐi}{\papi{ ndʐi}}}\markboth{ndʐi}{}
\classe{vs}
\paradigme{\textit{dir :} \jya pɯ-}
\begin{définition}\ 
\begin{déclaration}\grammar{acaus}\end{déclaration}\end{définition}
\begin{définition}\fra fondre\end{définition}
\begin{définition}\cmn 融化\end{définition}
\begin{exemple}\jya tɯ-ci ɯ-rkɯ tɤjpɣom ra pjɤ-ndʐi\cmn 河边的冰融掉了\end{exemple}
\begin{exemple}\jya zgoku tɤjpa ra pjɤ-ndʐi\cmn 山上的雪融掉了\end{exemple}
\begin{exemple}\jya ta-mar pɯ-ndʐi\cmn 酥油融了\end{exemple}\begin{sous-entrée}
\vedette{\hypertarget{}{\papi{ sɯɣndʐi}}}\markboth{sɯɣndʐi}{}\classe{vt}
\begin{définition}\fra faire fondre\end{définition}
\begin{définition}\cmn 使…融化\end{définition}
\begin{relation-sémantique}\confer{
\hyperlink{Ⓔftʂi}{\textit{ \papi{ftʂi}}}
}\end{relation-sémantique}
\end{sous-entrée}\end{entrée}

\begin{entrée}
\vedette{\hypertarget{Ⓔndjɤndjɤt}{\papi{ ndjɤndjɤt}}}\markboth{ndjɤndjɤt}{}
\classe{idph.2}
\begin{définition}\fra imposant et gracieux\end{définition}
\begin{définition}\cmn 亭亭玉立\end{définition}
\begin{exemple}\jya tɕhemɤpɯ ci ndjɤndjɤt ɲɯ-ŋu\cmn 女孩子亭亭玉立\end{exemple}
\begin{exemple}\jya tɕheme nɯ ɯ-phoŋbu ɯ-tɯ-mpɕɤr kɯ ndjɤndjɤt ʑo ɲɯ-pa\cmn 这个女子(身材)亭亭玉立\end{exemple}\end{entrée}

\begin{entrée}
\vedette{\hypertarget{Ⓔndo}{\papi{ ndo}}}\markboth{ndo}{}\classe{vt}\acception{1}
\paradigme{\textit{dir :} \jya \_}
\begin{définition}\fra tenir, prendre\end{définition}
\begin{définition}\cmn 拿\end{définition}
\begin{exemple}\jya mbrɯtɕɯ ɲɯ-tɯ-ɤsɯ-ndo\cmn 你在拿刀\end{exemple}
\begin{exemple}\jya nɤ-mtɕhi kɤ-ndɤm\cmn 你闭嘴\end{exemple}
\begin{exemple}\jya nɤ-jaʁ kɤ-ndɤm\cmn 不要插手\end{exemple}\acception{2}
\paradigme{\textit{dir :} \jya kɤ-}
\begin{définition}\fra attraper\end{définition}
\begin{définition}\cmn 抓到;捕到\end{définition}
\begin{exemple}\jya ɣzɯ ɣɯ ɯ-pɯ ci ko-ndo tɕe jo-ɣɯt\cmn (我们的邻居)抓到了猴崽子并把它带来了\end{exemple}\acception{3}
\paradigme{\textit{dir :} \jya tɤ-}
\begin{définition}\fra devenir, prendre une charge\end{définition}
\begin{définition}\cmn 当上\end{définition}
\begin{exemple}\jya rɟɤlpu to-ndo\cmn 他当上国王\end{exemple}\acception{4}
\paradigme{\textit{dir :} \jya tɤ-}
\begin{définition}\fra tomber enceinte (animal femelle)\end{définition}
\begin{définition}\cmn 怀胎(动物)\end{définition}
\begin{exemple}\jya paʁ ɣɯ ɯ-pɯ to-ndo\cmn 母猪怀了胎了\end{exemple}\acception{5}
\paradigme{\textit{dir :} \jya kɤ-}
\begin{définition}\fra attraper (une maladie grave)\end{définition}
\begin{définition}\cmn 得病(致命的病)\end{définition}
\begin{exemple}\jya ɯʑo kɯ @aizheng ko-ndo\cmn 他得了癌症\end{exemple}\acception{6}
\paradigme{\textit{dir :} \jya kɤ-}
\begin{définition}\fra garder (chemin)\end{définition}
\begin{définition}\cmn 守(路)\end{définition}
\begin{exemple}\jya tʂu kɤ-ndo-t-a\cmn 我守了路\end{exemple}
\begin{exemple}\jya aʑo kɯre tʂu ku-osɯ-ndo-a\cmn 我正在守路\end{exemple}\acception{7}
\paradigme{\textit{dir :} \jya kɤ-}
\begin{définition}\fra entrer dans ses ... ans\end{définition}
\begin{définition}\cmn 进入……岁\end{définition}
\begin{exemple}\jya kɯβdesqamnɯz ko-ndo\cmn 他开始进入42岁了\end{exemple}
\begin{relation-sémantique}\confer{
\hyperlink{Ⓔnɤndɯndo}{\textit{ \papi{nɤndɯndo}}}
}\end{relation-sémantique}
\begin{relation-sémantique}\confer{
\hyperlink{Ⓔando}{\textit{ \papi{ando}}}
}\end{relation-sémantique}
\begin{relation-sémantique}\confer{
\hyperlink{Ⓔnɤsnɯndo}{\textit{ \papi{nɤsnɯndo}}}
}\end{relation-sémantique}
\begin{relation-sémantique}\confer{
 \papi{ɯ-rtsawa,ndo}
}\end{relation-sémantique}\begin{sous-entrée}
\vedette{\hypertarget{}{\papi{ andɯndo}}}\markboth{andɯndo}{}\classe{vi}
\paradigme{\textit{dir :} \jya nɯ-}
\begin{définition}\ 
\begin{déclaration}\grammar{recip}\end{déclaration}\end{définition}
\begin{définition}\fra s'attacher\end{définition}
\begin{définition}\cmn 粘在一起\end{définition}
\end{sous-entrée}\begin{sous-entrée}
\vedette{\hypertarget{}{\papi{ sɤndɯndo}}}\markboth{sɤndɯndo}{}\classe{vt}
\paradigme{\textit{dir :} \jya nɯ-}
\begin{définition}\fra clouer, coller ensemble\end{définition}
\begin{définition}\cmn 钉在一起;粘在一起\end{définition}
\begin{exemple}\jya qraʁ ɯ-kɯ-spoʁ ci tu tɕe, ɕɤmtshoʁ pjɯ́-wɣ-no tɕe mbɣo cho qraʁ ni ɲɯ́-wɣ-sɤndɯndo ra\cmn 铧头中间有个洞,在那里钉个钉子,把铧头钉在犁头上\end{exemple}
\end{sous-entrée}\begin{sous-entrée}
\vedette{\hypertarget{}{\papi{ sɯndo}}}\markboth{sɯndo}{}\classe{vt}
\paradigme{\textit{dir :} \jya \_}
\begin{définition}\ 
\begin{déclaration}\grammar{caus}\end{déclaration}\end{définition}
\begin{définition}\fra prendre avec\end{définition}
\begin{définition}\cmn 用……拿、用……固定、使带走、配种\end{définition}
\begin{exemple}\jya tɤtshoʁ kɯ kɤ-sɯ-ndo-t-a\cmn 我用钉子固定了\end{exemple}
\begin{exemple}\jya tɤ-ri kɯ (tɯ-xtsa, tɯ-ŋga) kɤ-sɯ-ndo-t-a\cmn 我用线固定了\end{exemple}
\end{sous-entrée}\begin{sous-entrée}
\vedette{\hypertarget{}{\papi{ tɯ-phoŋbu,ndo}}}\markboth{tɯ-phoŋbu,ndo}{}
\begin{définition}\fra trembler de froid\end{définition}
\begin{définition}\cmn (冷得)发抖(无法控制)\end{définition}
\begin{exemple}\jya a-phoŋbu kɤ-ndo ʑo mɯ́j-khɯ\cmn 我冷得发抖\end{exemple}
\begin{relation-sémantique}\ComponentA{\classe{np}
\hyperlink{Ⓔtɯ-phoŋbu}{\textit{ \papi{tɯ-phoŋbu}}}
}\end{relation-sémantique}
\begin{relation-sémantique}\ComponentB{\classe{vt}
\hyperlink{Ⓔndo}{\textit{ \papi{ndo}}}
}\end{relation-sémantique}
\end{sous-entrée}\begin{sous-entrée}
\vedette{\hypertarget{}{\papi{ ɯ-mdoʁ,ndo}}}\markboth{ɯ-mdoʁ,ndo}{}
\begin{définition}\fra avoir (couleur)\end{définition}
\begin{définition}\cmn 有(颜色)\end{définition}
\begin{relation-sémantique}\ComponentA{\classe{np}
\hyperlink{Ⓔɯ-mdoʁ}{\textit{ \papi{ɯ-mdoʁ}}}
}\end{relation-sémantique}
\begin{relation-sémantique}\ComponentB{\classe{vt}
\hyperlink{Ⓔndo}{\textit{ \papi{ndo}}}
}\end{relation-sémantique}
\end{sous-entrée}\begin{sous-entrée}
\vedette{\hypertarget{}{\papi{ ʑɣɤsɯndo}}}\markboth{ʑɣɤsɯndo}{}\classe{vi}
\paradigme{\textit{dir :} \jya kɤ-}
\begin{définition}\ 
\begin{déclaration}\grammar{caus}\end{déclaration}
\begin{déclaration}\grammar{refl}\end{déclaration}\end{définition}\acception{1}
\begin{définition}\fra se faire attraper\end{définition}
\begin{définition}\cmn 被人抓\end{définition}\acception{2}
\begin{définition}\fra se contrôler\end{définition}
\begin{définition}\cmn 自我控制\end{définition}
\begin{exemple}\jya ɯʑo kɤ-ʑɣɤsɯndo mɤ-kɯ-cha ci ɲɯ-ŋu\cmn 他是不能控制自己的人\end{exemple}
\end{sous-entrée}\end{entrée}

\begin{entrée}
\vedette{\hypertarget{Ⓔndom}{\papi{ ndom}}}\markboth{ndom}{}
\classe{vi.nh}
\paradigme{\textit{dir :} \jya nɯ-}
\begin{définition}\ 
\begin{déclaration}\grammar{acaus}\end{déclaration}\end{définition}
\begin{définition}\fra horizontal\end{définition}
\begin{définition}\cmn 横\end{définition}
\begin{exemple}\jya tʂu ri ɕoŋtɕa ci pjɤ-ndom\cmn 路中间横放着一条木柴\end{exemple}
\begin{exemple}\jya pjɤ-ndʐaβ tɕe pjɤ-ndom\cmn 他跌倒了,躺在地上了\end{exemple}
\begin{relation-sémantique}\confer{
\hyperlink{Ⓔxthom}{\textit{ \papi{xthom}}}
}\end{relation-sémantique}\end{entrée}

\begin{entrée}
\vedette{\hypertarget{Ⓔndoʁ}{\papi{ ndoʁ}}}\markboth{ndoʁ}{}
\classe{vs}
\paradigme{\textit{dir :} \jya nɯ-}
\paradigme{\textit{dir :} \jya tɤ-}
\begin{définition}\fra croquant, craquant (fruit, branche d'arbre)\end{définition}
\begin{définition}\cmn 脆(水果、树枝)\end{définition}
\begin{exemple}\jya si ɲɯ-ndoʁ\cmn 木头是脆的\end{exemple}
\begin{exemple}\jya ʑmbri ɯ-rtaʁ a-nɯ-lni tɕe, tɕe mɤ-ndoʁ tɕe kɤ-qlɯt mɤ-khɯ\cmn 杨柳枝晒干了就不脆,不容易折\end{exemple}\end{entrée}

\begin{entrée}
\vedette{\hypertarget{Ⓔndʐoʁ}{\papi{ ndʐoʁ}}}\markboth{ndʐoʁ}{}\classe{vi}
\paradigme{\textit{dir :} \jya nɯ-}
\begin{définition}\fra être effrayé (animal, surtout cheval)\end{définition}
\begin{définition}\cmn 受惊(动物,特别是马)到处乱跑乱跳
\begin{déclaration} \étymologie{\papi{ⁿdrog}}\end{déclaration}\end{définition}
\begin{exemple}\jya mbro ɲɤ-ndʐoʁ\cmn 马受惊了\end{exemple}\begin{sous-entrée}
\vedette{\hypertarget{}{\papi{ sɯɣndʐoʁ}}}\markboth{sɯɣndʐoʁ}{}\classe{vt}
\paradigme{\textit{dir :} \jya nɯ-}
\begin{définition}\fra effrayer\end{définition}
\begin{définition}\cmn 令(动物)受惊\end{définition}
\begin{exemple}\jya khɯna kɯ tshɤt na-sɯɣndʐoʁ\cmn 狗令山羊受惊了\end{exemple}
\end{sous-entrée}\end{entrée}

\begin{entrée}
\vedette{\hypertarget{ⒺndɯⒽ1}{\papi{ ndɯ}}}\markboth{ndɯ}{}\homonyme{1}\classe{vi}
\paradigme{\textit{dir :} \jya \_}
\begin{définition}\ 
\begin{déclaration}\grammar{acaus}\end{déclaration}\end{définition}\acception{1}
\begin{définition}\fra être construit, être praticable (chemin, pont)\end{définition}
\begin{définition}\cmn 开通(桥、路)\end{définition}
\begin{exemple}\jya tʂu ko-ndɯ (lo-ndɯ, ɲɤ-ndɯ, pjɤ-ndɯ)\cmn 路通了\end{exemple}
\begin{exemple}\jya ndzom ko-ndɯ\cmn 桥通了\end{exemple}
\begin{exemple}\jya kɯki ɯ-stu ki tʂu pɯ-me ri, pjɤ-ndɯ\cmn 以前没有路,现在就通了\end{exemple}
\begin{exemple}\jya kukɯtɕu tʂu pjɤ-nɯ-ndɯndɯ ɕti\cmn 以前这里路是通的\end{exemple}
\begin{exemple}\jya tɕhi to-ndɯ\cmn 有梯子\end{exemple}
\begin{exemple}\jya ɯ-jroʁ ko-ndɯ\cmn 他留了痕迹\end{exemple}\acception{2}
\begin{définition}\fra apparaître (arc-en-ciel)\end{définition}
\begin{définition}\cmn 出现(彩虹)\end{définition}
\begin{exemple}\jya ndʑa pjɤ-ndɯ\cmn 出现了彩虹\end{exemple}
\begin{relation-sémantique}\confer{
\hyperlink{ⒺthɯⒽ1}{\textit{ \papi{thɯ1}}}
}\end{relation-sémantique}\end{entrée}

\begin{entrée}
\vedette{\hypertarget{ⒺndɯⒽ2}{\papi{ ndɯ}}}\markboth{ndɯ}{}\homonyme{2}\classe{vi}
\paradigme{\textit{dir :} \jya tɤ-}
\begin{définition}\fra s'accumuler\end{définition}
\begin{définition}\cmn 积累\end{définition}
\begin{exemple}\jya ji-kɯ-rtoʁ pɯ-dɤn tɕɤn, laχtɕha khro to-ndɯ\cmn 来看我们的人很多,所以收了很多东西\end{exemple}
\begin{exemple}\jya jisŋi tɯtsɣe nɯ khro to-ndɯ\cmn 我们今天生意(很好),赚到很多\end{exemple}
\begin{exemple}\jya tɯmpɕar nɤ tɯmpɕar ntsɯ na-kho-nɯ tɕe, tham tɕe sqamŋu-mpɕar jamar to-ndɯ\cmn (虽然)他们一块一块地给,(但是)现在筹到十五块了\end{exemple}
\begin{exemple}\jya wo nɯ sthɯci kɯ-rkɯn nɤ, ndɯ ci me loβ\cmn (他们给的)那么少,积攒不到很多\end{exemple}
\begin{exemple}\jya tɕhaʁra kɯ-rɯru nɯ kɯ kɯmŋu toŋtsi ku-wum ɲɯ-ɕti ri, tɯrmɯkha tɕe ʁnɯ-ri χsɯ-ri tu-ndɯ ɲɯ-ɕti / ku-ojtɯ ɕti\cmn 看厕所的人虽然每个人只拿五毛钱,到了下午就能积攒到两三百块钱\end{exemple}
\begin{relation-sémantique}\confer{
\hyperlink{Ⓔajtɯ}{\textit{ \papi{ajtɯ}}}
}\end{relation-sémantique}\end{entrée}

\begin{entrée}
\vedette{\hypertarget{Ⓔndɯβ}{\papi{ ndɯβ}}}\markboth{ndɯβ}{}
\classe{vs}
\paradigme{\textit{dir :} \jya nɯ-}
\begin{définition}\fra fine (poudre)\end{définition}
\begin{définition}\cmn 细(粉状)\end{définition}
\begin{exemple}\jya mɯ́j-ndʐɤz kɯ ɲɯ-ndɯβ\cmn 不粗,很细\end{exemple}
\begin{relation-sémantique}\antonyme{
\hyperlink{Ⓔjndʐɤz}{\textit{ \papi{jndʐɤz}}}
}\end{relation-sémantique}
\begin{relation-sémantique}\confer{
\hyperlink{Ⓔɣɤndɯβ}{\textit{ \papi{ɣɤndɯβ}}}
}\end{relation-sémantique}\end{entrée}

\begin{entrée}
\vedette{\hypertarget{Ⓔndɯchu}{\papi{ ndɯchu}}}\markboth{ndɯchu}{}\classe{adv}
\begin{définition}\fra à l'ouest\end{définition}
\begin{définition}\cmn 在西部\end{définition}
\begin{relation-sémantique}\confer{
\hyperlink{Ⓔɯ-ndɤcu}{\textit{ \papi{ɯ-ndɤcu}}}
}\end{relation-sémantique}\end{entrée}

\begin{entrée}
\vedette{\hypertarget{Ⓔndʐɯɣlɤm}{\papi{ ndʐɯɣlɤm}}}\markboth{ndʐɯɣlɤm}{}\classe{n}
\begin{définition}\fra règle\end{définition}
\begin{définition}\cmn 规矩,政策\end{définition}\end{entrée}

\begin{entrée}
\vedette{\hypertarget{Ⓔndɯɣsa}{\papi{ ndɯɣsa}}}\markboth{ndɯɣsa}{}\classe{n}
\begin{définition}\fra lieu de résidence\end{définition}
\begin{définition}\cmn 住处(敬语)
\begin{déclaration} \étymologie{\papi{ɴdug.sa}}\end{déclaration}\end{définition}\end{entrée}

\begin{entrée}
\vedette{\hypertarget{ⒺndɯlⒽ1}{\papi{ ndɯl}}}\markboth{ndɯl}{}\homonyme{1}\classe{vi}
\paradigme{\textit{dir :} \jya pɯ-}
\begin{définition}\fra être apprivoisé\end{définition}
\begin{définition}\cmn 被驯服\end{définition}
\begin{exemple}\jya kɯki jla ki kɤ-ndɯl wuma ʑo pɯ-mbat\cmn 这头犏牛很好驯服\end{exemple}\begin{sous-entrée}
\vedette{\hypertarget{}{\papi{ ɣɤndɯl}}}\markboth{ɣɤndɯl}{}\classe{vi}
\begin{définition}\ 
\begin{déclaration}\grammar{facil}\end{déclaration}\end{définition}
\begin{définition}\fra facile à apprivoiser\end{définition}
\begin{définition}\cmn 容易驯服\end{définition}
\begin{relation-sémantique}\synonyme{
\hyperlink{Ⓔnɯɣɯftɯl}{\textit{ \papi{nɯɣɯftɯl}}}
}\end{relation-sémantique}
\end{sous-entrée}\begin{sous-entrée}
\vedette{\hypertarget{}{\papi{ sɯɣndɯl}}}\markboth{sɯɣndɯl}{}\classe{vt}
\paradigme{\textit{dir :} \jya pɯ-}
\begin{définition}\fra apprivoiser\end{définition}
\begin{définition}\cmn 驯服\end{définition}
\begin{exemple}\jya mbro nɯ pɯ-sɯɣndɯl-a\cmn 我驯服了这匹马\end{exemple}
\end{sous-entrée}\end{entrée}

\begin{entrée}
\vedette{\hypertarget{ⒺndɯlⒽ2}{\papi{ ndɯl}}}\markboth{ndɯl}{}\homonyme{2}
\classe{vt}
\paradigme{\textit{dir :} \jya thɯ-}
\paradigme{\textit{dir :} \jya nɯ-}
\begin{définition}\fra pulvériser, mettre en poudre\end{définition}
\begin{définition}\cmn 磨细\end{définition}
\begin{exemple}\jya ɕnɤto thɯ-ndɯl-a\cmn 我把鼻烟磨细了\end{exemple}\end{entrée}

\begin{entrée}
\vedette{\hypertarget{Ⓔndʐɯm}{\papi{ ndʐɯm}}}\markboth{ndʐɯm}{}
\classe{vs}
\paradigme{\textit{dir :} \jya tɤ-}\acception{1}
\begin{définition}\fra rapide\end{définition}
\begin{définition}\cmn 快\end{définition}\acception{2}
\begin{définition}\fra courant (parole)\end{définition}
\begin{définition}\cmn 流利(语言)\end{définition}
\begin{exemple}\jya @piqiu ɲɯ-ndʐɯm\cmn 皮球转动得很快\end{exemple}
\begin{exemple}\jya mkhɯrlu ɲɯ-ndʐɯm\cmn 轮子转动得很快\end{exemple}
\begin{exemple}\jya qale ɲɯ-ndʐɯm\cmn 风不停地吹\end{exemple}
\begin{exemple}\jya ɯ-tɯ-ɤre ɲɯ-ndʐɯm\cmn 他笑个不停\end{exemple}
\begin{exemple}\jya ɯ-ɕmi ɲɯ-ndʐɯm\cmn 他滔滔不绝地讲\end{exemple}
\begin{exemple}\jya kɤ-ti a-kɤ-ndʐɯm ɲɯ-sɯsam-a\cmn 我希望能讲得流利一点\end{exemple}\begin{sous-entrée}
\vedette{\hypertarget{}{\papi{ sɯɣndʐɯm}}}\markboth{sɯɣndʐɯm}{}\classe{vt}
\begin{définition}\ 
\begin{déclaration}\grammar{caus}\end{déclaration}\end{définition}
\begin{définition}\fra faire des exercices\end{définition}
\begin{définition}\cmn 锻炼\end{définition}
\begin{exemple}\jya a-phoŋbu ɲɯ-sɯɣndʐɯm-a ra ɲɯ-sɯsam-a\cmn 我觉得自己要锻炼身体\end{exemple}
\begin{exemple}\jya kɯrɯ skɤt kɤ-ti tu-sɯɣndʐɯm-a ra ɲɯ-sɯsam-a\cmn 我想讲藏语讲的流利一点\end{exemple}
\end{sous-entrée}\end{entrée}

\begin{entrée}
\vedette{\hypertarget{Ⓔndɯn}{\papi{ ndɯn}}}\markboth{ndɯn}{}
\classe{vt}
\paradigme{\textit{dir :} \jya pɯ-}
\paradigme{\textit{dir :} \jya kɤ-}
\begin{définition}\fra lire à haute voix\end{définition}
\begin{définition}\cmn 读
\begin{déclaration} \étymologie{\papi{ⁿdon}}\end{déclaration}\end{définition}
\begin{exemple}\jya tɕe nɤ-χpi pɯ-ndɯn jɤɣ\cmn 好吧,你念你的故事吧\end{exemple}
\begin{exemple}\jya ɯ-mphru pɯ-ʑe tɕe pɯ-ndɯn\cmn 从头开始念吧\end{exemple}\begin{sous-entrée}
\vedette{\hypertarget{}{\papi{ rɤndɯn}}}\markboth{rɤndɯn}{}\classe{vi}
\paradigme{\textit{dir :} \jya pɯ-}
\begin{définition}\ 
\begin{déclaration}\grammar{apass}\end{déclaration}\end{définition}
\begin{définition}\fra lire à haute voix\end{définition}
\begin{définition}\cmn 诵经;背书\end{définition}
\begin{exemple}\jya χpɯn ra ɲɯ-rɤndɯn-nɯ\cmn 和尚们在诵经\end{exemple}
\end{sous-entrée}\end{entrée}

\begin{entrée}
\vedette{\hypertarget{Ⓔndʐɯnbu}{\papi{ ndʐɯnbu}}}\markboth{ndʐɯnbu}{}\classe{n}
\begin{définition}\fra hôte\end{définition}
\begin{définition}\cmn 远方的客人
\begin{déclaration} \étymologie{\papi{mgron.po}}\end{déclaration}\end{définition}
\begin{exemple}\jya ndʐɯnbu ɕe-a\cmn 我要出差\end{exemple}
\begin{relation-sémantique}\confer{
\hyperlink{Ⓔnɯndʐɯnbu}{\textit{ \papi{nɯndʐɯnbu}}}
}\end{relation-sémantique}\end{entrée}

\begin{entrée}
\vedette{\hypertarget{Ⓔndʐɯnɬa}{\papi{ ndʐɯnɬa}}}\markboth{ndʐɯnɬa}{}\classe{n}
\begin{définition}\fra cérémonie effectuée lorsqu'un membre de la famille part en voyage\end{définition}
\begin{définition}\cmn 家里有人出行的时候,为了保佑他安全顺利而念的经\end{définition}
\begin{relation-sémantique}\synonyme{
\hyperlink{Ⓔmdɯnri}{\textit{ \papi{mdɯnri}}}
}\end{relation-sémantique}
\end{entrée}

\begin{entrée}
\vedette{\hypertarget{Ⓔndʐɯnphɤrscoʁ}{\papi{ ndʐɯnphɤrscoʁ}}}\markboth{ndʐɯnphɤrscoʁ}{}\classe{n}
\begin{définition}\fra louche en cuivre\end{définition}
\begin{définition}\cmn 红铜勺子\end{définition}
\begin{exemple}\jya ndʐɯnphɤrscoʁ nɯ scoʁ ɯ-jɯ kɯ-zri tsa ci ŋu, ɯ-jɯ nɯ ɕom ŋu, ɯ-pɤl nɯ ɯ-pa nɯ ra zan ŋu, ɯ-taʁ tɕe raʁ ku-ɕe ŋu, ɯ-mŋu nɯ li zaŋ ŋu, tɕe kɯɕɯŋgɯ tʂha ɯ-z-nɤrkɯku pjɤ-ŋu, tɯ-jno sɤ-rku mɤ-sna ma zaŋ cho raʁ ni ʁnaʁna ʑo kɯ-nɤmar nɯ-atɕaʁ tɕe ʁja ɲɯ-ɬoʁ ŋu tɕe, ʁja nɯ kɯ tɯrme tu-kɯ-ɕɯngo ŋgrɤl.\cmn 红铜勺子是把比较长的勺子,把是铁作成的,勺子头外层部分是红铜,里层是黄铜,勺沿也是红铜。过去是专门用来舀茶的器具,不能用来舀菜,因为黄铜和红铜沾上油会生锈,锈会导致人生病。\end{exemple}\end{entrée}

\begin{entrée}
\vedette{\hypertarget{Ⓔndɯχu}{\papi{ ndɯχu}}}\markboth{ndɯχu}{} (\variante{dɯχu}) 
\classe{n}
\begin{définition}\fra lis\end{définition}
\begin{définition}\cmn 百合\end{définition}
\begin{exemple}\jya ndɯχu nɯ sɯjno ci ŋu, sɯŋgɯ cho praʁ ɯ-rchɤβ ra kɤ-ɬoʁ rga, ɯ-qa nɯ ɕkɤtɯm kɯ-fse ɲɯ-βze ŋu, ɯ-ru nɯ kɯ-rɲɟi tsa ʑo tu-fse cha, ɯ-rtaʁ ɲɯ-ɬoʁ mɤ-cha. ɯ-jwaʁ nɯ kɯ-xtɕi tsa kɯ-ɤmtɕoʁ tsa ŋu. ɯ-mɯntoʁ nɯ wuma ʑo mpɕɤr, ʁmɤrsɤr ŋu tɕe, ɲɯ-ʁaʁ tɕe, ɯ-mɯntoʁ nɯ pjɯ-nɯqhɤɴɢaʁ ɯ-qhu chu pɕoʁ tu-ŋgɤɣ ŋu, tɕe ɯ-taʁ kɯ-ɲaʁ kɯ-ɤkhra tu.\cmn 百合是一种植物,一般生长在灌木丛中和岩石上,根部像大蒜的根一样,茎细长,不分叉。叶子小而尖。花很美,金黄色。开花时,花瓣向后面卷起来,上面有黑点。\end{exemple}\end{entrée}

\begin{entrée}
\vedette{\hypertarget{Ⓔndʐɯz}{\papi{ ndʐɯz}}}\markboth{ndʐɯz}{}\classe{vs}\acception{1}
\begin{définition}\fra bien s'entendre\end{définition}
\begin{définition}\cmn 谈得来\end{définition}
\begin{exemple}\jya tɕiʑo tɕi-khɤcɤl ndʐɯz ŋgrɤl\cmn 我们俩很谈得来\end{exemple}\acception{2}
\begin{définition}\fra bien suivre le rythme (danse)\end{définition}
\begin{définition}\cmn 跟着节奏跳(舞)\end{définition}
\begin{exemple}\jya ɯʑo kɤ-sɤmtshi ɲɯ-mkhɤz tɕe tɯrɟaʁ ɲɯ-ndʐɯz\cmn 他很会领舞,所以舞跳得很整齐\end{exemple}\end{entrée}

\begin{entrée}
\vedette{\hypertarget{Ⓔndʐuwa}{\papi{ ndʐuwa}}}\markboth{ndʐuwa}{}\classe{n}
\begin{définition}\fra hôte\end{définition}
\begin{définition}\cmn 客人\end{définition}
\begin{exemple}\jya jisŋi a-ndʐuwa ɣɤʑu wo\cmn 我今天有我客人\end{exemple}\end{entrée}

\begin{entrée}
\vedette{\hypertarget{Ⓔndzu}{\papi{ ndzu}}}\markboth{ndzu}{}
\classe{vi}
\paradigme{\textit{dir :} \jya tɤ-}
\begin{définition}\fra être prêt\end{définition}
\begin{définition}\cmn 准备好了,正要出发\end{définition}
\begin{exemple}\jya kɤ-ɕe pɤjkhu mɤ-ndzu-a\cmn 我还没有准备出发\end{exemple}
\begin{exemple}\jya kɤ-ɕe tɤ-ndzu-a\cmn 我准备出发了\end{exemple}
\begin{exemple}\jya ʑa a-tɤ-ndzu ɲɯ-ra tɕe tɤ-ɕɯmbɣom\cmn 你要早一点启程的话,就让他快点\end{exemple}\begin{sous-entrée}
\vedette{\hypertarget{}{\papi{ sɯɣndzu}}}\markboth{sɯɣndzu}{}\classe{vt}
\paradigme{\textit{dir :} \jya tɤ-}
\begin{définition}\ 
\begin{déclaration}\grammar{caus}\end{déclaration}\end{définition}
\begin{définition}\fra préparer au départ\end{définition}
\begin{définition}\cmn 让……启程\end{définition}
\end{sous-entrée}\end{entrée}

\begin{entrée}
\vedette{\hypertarget{Ⓔndza}{\papi{ ndza}}}\markboth{ndza}{}
\classe{vt}
\paradigme{\textit{dir :} \jya tɤ-}
\paradigme{\textit{dir :} \jya thɯ-}\acception{1}
\begin{définition}\fra manger\end{définition}
\begin{définition}\cmn 吃\end{définition}
\begin{définition}\fra tɕhi tɤ-tɯ-ndza-t?\end{définition}
\begin{définition}\cmn 你吃了什么?\end{définition}\acception{2}
\begin{définition}\fra mâcher\end{définition}
\begin{définition}\cmn 咀嚼\end{définition}
\begin{exemple}\jya βʑɯ kɯ tɯ-ŋga to-ndza\cmn 老鼠把衣服咬破了\end{exemple}\begin{sous-entrée}
\vedette{\hypertarget{}{\papi{ nɯɣɯndza}}}\markboth{nɯɣɯndza}{}\classe{vs}
\begin{définition}\fra agréable à manger\end{définition}
\begin{définition}\cmn 吃着顺口\end{définition}
\end{sous-entrée}\begin{sous-entrée}
\vedette{\hypertarget{}{\papi{ sɤndza}}}\markboth{sɤndza}{}\classe{vs}
\begin{définition}\ 
\begin{déclaration}\grammar{apass}\end{déclaration}\end{définition}
\begin{définition}\fra piquer\end{définition}
\begin{définition}\cmn 刺人\end{définition}
\begin{exemple}\jya mɤ-sɤndza ma ɯ-mdzu me\cmn 没有刺所以刺不到人\end{exemple}
\begin{relation-sémantique}\confer{
\hyperlink{Ⓔandzɯndza}{\textit{ \papi{andzɯndza}}}
}\end{relation-sémantique}
\begin{relation-sémantique}\confer{
\hyperlink{Ⓔsɯndza}{\textit{ \papi{sɯndza}}}
}\end{relation-sémantique}
\end{sous-entrée}\begin{sous-entrée}
\vedette{\hypertarget{}{\papi{ sɯndza}}}\markboth{sɯndza}{}\classe{vt}
\paradigme{\textit{dir :} \jya tɤ-}\acception{1}
\begin{définition}\fra manger avec, faire manger\end{définition}
\begin{définition}\cmn 用……吃\end{définition}
\begin{exemple}\jya ɯ-ʁe nɯ kɯ tu-sɯndze ŋu\cmn 她用左(手)吃饭(因为右手受伤了)\end{exemple}\acception{2}
\begin{définition}\fra faire manger\end{définition}
\begin{définition}\cmn 使……吃
\begin{déclaration}\grammar{habil}\end{déclaration}\end{définition}\acception{3}
\begin{définition}\fra pouvoir manger\end{définition}
\begin{définition}\cmn 吃得下\end{définition}
\begin{exemple}\jya nɯ thamtɕɤt mɯ́j-sɯndze-a (mɯ́j-sɯ-ɕkɯt-a)\cmn 我吃不下那么多\end{exemple}
\end{sous-entrée}\end{entrée}

\begin{entrée}
\vedette{\hypertarget{Ⓔndzamthaŋ}{\papi{ ndzamthaŋ}}}\markboth{ndzamthaŋ}{}\classe{n}
\begin{définition}\ 
\begin{déclaration}\grammar{n.lieu}\end{déclaration}\end{définition}
\begin{définition}\fra Ndzamthang\end{définition}
\begin{définition}\cmn 壤塘\end{définition}\end{entrée}

\begin{entrée}
\vedette{\hypertarget{Ⓔndzar}{\papi{ ndzar}}}\markboth{ndzar}{}
\classe{vi}
\paradigme{\textit{dir :} \jya pɯ-}
\begin{définition}\fra s'égoutter complètement\end{définition}
\begin{définition}\cmn 滤干;滗干\end{définition}
\begin{exemple}\jya tɯ-mɯ pjɤ-ndzar\cmn 雨下得不会再下\end{exemple}
\begin{exemple}\jya a-ŋga ɯ-taʁ tɯ-ci mɯ-pɯ-ndzar mɤɕtʂa pɯ-ndzur-a\cmn 一直站着,等到衣服上的水滴干了\end{exemple}
\begin{exemple}\jya tɯ-ŋga nɯ-χtɕi-t-a tɕe a-pɯ-ndzar tɕe chɯ́-wɣ-ɕkho jɤɣ\cmn 我洗了衣服,现在要滴干了才能晒\end{exemple}
\begin{exemple}\jya tɯ-ŋga nɯ-χtɕi-t-a tɕe a-pɯ-ndzar tɕe tɕetha ʑatsa zbaʁ\cmn 我洗了衣服,现在要滴干了就会很快变干\end{exemple}
\begin{exemple}\jya @cai tɤ́-wɣ-χtɕi tɕe @shaoji ɯ-ŋgɯ kɤ-rku ra ma kɤ-ndzar mɤ-cha\cmn 洗了菜以后要放在簸箕里,不然就不能变干\end{exemple}
\begin{exemple}\jya nɤ-ŋga nɯ-tɯ-χtɕi tɕe a-tɤ-tɯ-ɕɯɴqoʁ tɕe, tɕe kɤ-ndzar cha\cmn 洗了衣服以后一定要把它挂起来才能变干\end{exemple}
\begin{relation-sémantique}\confer{
\hyperlink{Ⓔkɯndzarmɯ}{\textit{ \papi{kɯndzarmɯ}}}
}\end{relation-sémantique}
\begin{relation-sémantique}\confer{
\hyperlink{Ⓔmdzar}{\textit{ \papi{mdzar}}}
}\end{relation-sémantique}\begin{sous-entrée}
\vedette{\hypertarget{}{\papi{ sɯɣndzar}}}\markboth{sɯɣndzar}{}\classe{vt}
\begin{définition}\ 
\begin{déclaration}\grammar{caus}\end{déclaration}\end{définition}
\begin{définition}\fra laisser décanter, s'égoutter\end{définition}
\begin{définition}\cmn 把……滤干\end{définition}
\begin{exemple}\jya mbrɤz tɤ́-wɣ-χtɕi tɕe, kɤ-sɯɣndzar ra\cmn 淘了米就要把它滤干\end{exemple}
\end{sous-entrée}\end{entrée}

\begin{entrée}
\vedette{\hypertarget{Ⓔndzaʁlaŋ}{\papi{ ndzaʁlaŋ}}}\markboth{ndzaʁlaŋ}{}\classe{n}
\begin{définition}\fra le monde\end{définition}
\begin{définition}\cmn 世界
\begin{déclaration} \étymologie{\papi{ⁿdzam.gliŋ}}\end{déclaration}\end{définition}\end{entrée}

\begin{entrée}
\vedette{\hypertarget{Ⓔndzɤβ}{\papi{ ndzɤβ}}}\markboth{ndzɤβ}{}
\classe{vs}
\paradigme{\textit{dir :} \jya nɯ-}
\begin{définition}\fra collant, épais (gruau)\end{définition}
\begin{définition}\cmn 稠(粥)、黏(比较干)\end{définition}
\begin{exemple}\jya tɯtshi ɲɯ-ndzɤβ\cmn 粥很稠\end{exemple}
\begin{exemple}\jya tɕhɯβroʁ kɯ-ndzɤβ\cmn 稠糌粑\end{exemple}
\begin{exemple}\jya smar chɤ-ndzɤβ\cmn 河里的泥沙变多了\end{exemple}\begin{sous-entrée}
\vedette{\hypertarget{}{\papi{ nɤndzɤβ}}}\markboth{nɤndzɤβ}{}\classe{vt}
\begin{définition}\ 
\begin{déclaration}\grammar{trop}\end{déclaration}\end{définition}
\begin{définition}\fra trouver épais\end{définition}
\begin{définition}\cmn 觉得很稠(水分少)\end{définition}
\begin{relation-sémantique}\antonyme{
\hyperlink{Ⓔŋgri}{\textit{ \papi{ŋgri}}}
}\end{relation-sémantique}
\end{sous-entrée}\end{entrée}

\begin{entrée}
\vedette{\hypertarget{Ⓔndzɤβrta}{\papi{ ndzɤβrta}}}\markboth{ndzɤβrta}{}\classe{n}
\begin{définition}\fra grosse perle du rosaire, utilisée pour compter les tours\end{définition}
\begin{définition}\cmn 计数用的珠子(玛尼珠)\end{définition}\end{entrée}

\begin{entrée}
\vedette{\hypertarget{Ⓔndzɤpri}{\papi{ ndzɤpri}}}\markboth{ndzɤpri}{}
\classe{n}
\begin{définition}\fra ours brun\end{définition}
\begin{définition}\cmn 马熊\end{définition}
\begin{relation-sémantique}\confer{
\hyperlink{ⒺpriⒽ2}{\textit{ \papi{pri2}}}
}\end{relation-sémantique}\end{entrée}

\begin{entrée}
\vedette{\hypertarget{Ⓔndzɤqhɤjɯ}{\papi{ ndzɤqhɤjɯ}}}\markboth{ndzɤqhɤjɯ}{}\classe{n}
\begin{définition}\fra (fait) de manger dans son coin\end{définition}
\begin{définition}\cmn 偷吃\end{définition}
\begin{exemple}\jya ndzɤqhɤjɯ to-βzu\cmn 他偷吃了很多\end{exemple}
\begin{relation-sémantique}\confer{
\hyperlink{Ⓔrɯndzɤqhɤjɯ}{\textit{ \papi{rɯndzɤqhɤjɯ}}}
}\end{relation-sémantique}\end{entrée}

\begin{entrée}
\vedette{\hypertarget{Ⓔndzɤrndzɤr}{\papi{ ndzɤrndzɤr}}}\markboth{ndzɤrndzɤr}{}\classe{idph.2}
\begin{définition}\fra seul\end{définition}
\begin{définition}\cmn 形容独自一人的样子\end{définition}
\begin{exemple}\jya tɯrme tɯ-rdoʁ ndzɤrndzɤr\cmn 单独一个人\end{exemple}\end{entrée}

\begin{entrée}
\vedette{\hypertarget{Ⓔndzɤt}{\papi{ ndzɤt}}}\markboth{ndzɤt}{}\classe{vi}
\paradigme{\textit{dir :} \jya thɯ-}
\paradigme{\textit{dir :} \jya tɤ-}
\begin{définition}\fra grandir\end{définition}
\begin{définition}\cmn 长大\end{définition}
\begin{exemple}\jya stoʁ ɕɤxɕo tɕe to-ndzɤt\cmn 胡豆这几天长大\end{exemple}
\begin{exemple}\jya tɤ-pɤtso cho-ndzɤt\cmn 孩子长大了\end{exemple}\end{entrée}

\begin{entrée}
\vedette{\hypertarget{ⒺndzɤtshiⒽ1}{\papi{ ndzɤtshi}}}\markboth{ndzɤtshi}{}\homonyme{1}
\classe{n}
\begin{définition}\fra plat\end{définition}
\begin{définition}\cmn 饭菜\end{définition}
\begin{relation-sémantique}\confer{
\hyperlink{Ⓔrɯndzɤtshi}{\textit{ \papi{rɯndzɤtshi}}}
}\end{relation-sémantique}\end{entrée}

\begin{entrée}
\vedette{\hypertarget{ⒺndzɤtshiⒽ2}{\papi{ ndzɤtshi}}}\markboth{ndzɤtshi}{}\homonyme{2}
\classe{vt}
\paradigme{\textit{dir :} \jya tɤ-}
\begin{définition}\ 
\begin{déclaration}\grammar{comp}\end{déclaration}\end{définition}
\begin{définition}\fra manger et boire\end{définition}
\begin{définition}\cmn 吃喝\end{définition}
\begin{exemple}\jya ɯʑo kɯ tɯ-mgo ra to-ndzɤtshi\cmn 他把饭吃了\end{exemple}\end{entrée}

\begin{entrée}
\vedette{\hypertarget{Ⓔndzi}{\papi{ ndzi}}}\markboth{ndzi}{}\classe{vs}
\begin{définition}\fra enroué\end{définition}
\begin{définition}\cmn 嗓子哑了\end{définition}
\begin{exemple}\jya a-rqo ko-ndzi\cmn 我嗓子哑了\end{exemple}
\begin{sous-entrée}
\vedette{\hypertarget{}{\papi{ sɯɣndzi}}}\markboth{sɯɣndzi}{}\classe{vt}
\begin{définition}\ 
\begin{déclaration}\grammar{caus}\end{déclaration}\end{définition}
\begin{définition}\fra rendre enroué\end{définition}
\begin{définition}\cmn 令……嗓子哑\end{définition}
\end{sous-entrée}\end{entrée}

\begin{entrée}
\vedette{\hypertarget{Ⓔndziaʁ}{\papi{ ndziaʁ}}}\markboth{ndziaʁ}{}\classe{vs}\acception{1}
\paradigme{\textit{dir :} \jya thɯ-}
\begin{définition}\fra solide (nœud)\end{définition}
\begin{définition}\cmn 紧(结)\end{définition}
\begin{définition}\cmn 这个结打得很紧,很难解开\end{définition}
\begin{exemple}\jya kɯki tɤ-mtɯ chɤ-ndziaʁ tɕe kɤ-rla ɲɯ-ɴqa\end{exemple}\acception{2}
\begin{définition}\fra foncé\end{définition}
\begin{définition}\cmn 深(颜色)\end{définition}
\begin{exemple}\jya ɯ-mdoʁ ɲɯ-ndziaʁ\cmn 颜色很深\end{exemple}\acception{3}
\begin{définition}\fra complet (temps)\end{définition}
\begin{définition}\cmn 满、到期(时间)\end{définition}
\begin{exemple}\jya tɯ-xpa ɲɤ-ndziaʁ\cmn 满了一周年\end{exemple}\begin{sous-entrée}
\vedette{\hypertarget{}{\papi{ ɣɤndziaʁ}}}\markboth{ɣɤndziaʁ}{}\classe{vt}
\paradigme{\textit{dir :} \jya thɯ-}
\begin{exemple}\jya tɤ-mtɯ thɯ-ɣɤndziaʁ tɕe a-mɤ-nɯ-nɯ-ɬoʁ\cmn 你把结打得紧些,不要让它散开\end{exemple}
\end{sous-entrée}\end{entrée}

\begin{entrée}
\vedette{\hypertarget{ⒺndzomⒽ1}{\papi{ ndzom}}}\markboth{ndzom}{}\homonyme{1}\classe{n}
\begin{définition}\fra pont\end{définition}
\begin{définition}\cmn 桥\end{définition}
\begin{relation-sémantique}\confer{
\hyperlink{ⒺndzomⒽ2}{\textit{ \papi{ndzom2}}}
}\end{relation-sémantique}\end{entrée}

\begin{entrée}
\vedette{\hypertarget{ⒺndzomⒽ2}{\papi{ ndzom}}}\markboth{ndzom}{}\homonyme{2}\classe{vi}
\paradigme{\textit{dir :} \jya kɤ-}
\begin{définition}\ 
\begin{déclaration}\grammar{denom}\end{déclaration}\end{définition}
\begin{définition}\fra couvrir le fleuve (glace)\end{définition}
\begin{définition}\cmn 覆盖河流(冰层)
\end{définition}
\begin{exemple}\jya tɯ-ci ɯ-taʁ tɤjpɣom ko-ndzom\cmn 水上结成了一层冰\end{exemple}
\begin{relation-sémantique}\confer{
\hyperlink{ⒺndzomⒽ1}{\textit{ \papi{ndzom1}}}
}\end{relation-sémantique}\end{entrée}

\begin{entrée}
\vedette{\hypertarget{Ⓔndzoʁ}{\papi{ ndzoʁ}}}\markboth{ndzoʁ}{}\classe{vi}
\paradigme{\textit{dir :} \jya kɤ-}
\begin{définition}\ 
\begin{déclaration}\grammar{acaus}\end{déclaration}\end{définition}
\begin{définition}\fra porter (fruits; partie d'un objet)\end{définition}
\begin{définition}\cmn 结果子,带有\end{définition}
\begin{exemple}\jya ɯ-mat ko-ndzoʁ\cmn 结果了\end{exemple}
\begin{exemple}\jya ftɕar ko-ndzoʁ\cmn 春天到了\end{exemple}
\begin{exemple}\jya qartsɯ ko-ndzoʁ\cmn 冬天到了\end{exemple}
\begin{exemple}\jya jisŋi @qihao ko-ndzoʁ\cmn 今天已经是七号了\end{exemple}
\begin{exemple}\jya zdɯm zgo ɯ-taʁ pjɤ-ndzoʁ tɕe mɯ-to-ka\cmn 云贴在山尖上,还没有离开地面\end{exemple}
\begin{relation-sémantique}\confer{
\hyperlink{Ⓔtshoʁ}{\textit{ \papi{tshoʁ}}}
}\end{relation-sémantique}\end{entrée}

\begin{entrée}
\vedette{\hypertarget{Ⓔndzoʁtɣa}{\papi{ ndzoʁtɣa}}}\markboth{ndzoʁtɣa}{}
\classe{n}
\begin{définition}\fra empan (pouce et index)\end{définition}
\begin{définition}\cmn 一拃(大拇指和食指之间的距离)\end{définition}\end{entrée}

\begin{entrée}
\vedette{\hypertarget{Ⓔndzur}{\papi{ ndzur}}}\markboth{ndzur}{}
\classe{vi}
\paradigme{\textit{dir :} \jya tɤ-}
\begin{définition}\fra être debout\end{définition}
\begin{définition}\cmn 站\end{définition}
\begin{exemple}\jya aʑo tɤ-ndzur-a\cmn 我站起来了\end{exemple}
\begin{exemple}\jya nɤʑo to-tɯ-ndzur\cmn 你站起来了\end{exemple}
\begin{relation-sémantique}\confer{
\hyperlink{Ⓔsɯɣndzur}{\textit{ \papi{sɯɣndzur}}}
}\end{relation-sémantique}\end{entrée}

\begin{entrée}
\vedette{\hypertarget{Ⓔndzri}{\papi{ ndzri}}}\markboth{ndzri}{}
\classe{vt}
\paradigme{\textit{dir :} \jya tɤ-}
\paradigme{\textit{dir :} \jya pɯ-}
\paradigme{\textit{dir :} \jya \_}
\begin{définition}\fra tordre\end{définition}
\begin{définition}\cmn 拧\end{définition}
\begin{exemple}\jya aʑo tɯ-ŋga nɯ-χtɕi-t-a tɕe pɯ-ndzri-t-a tɕe tɯ-ci pɯ-tɕat-a\cmn 我把衣服洗了,然后把水拧出来了\end{exemple}
\begin{exemple}\jya tɯ-ŋga kɤ-χtɕi nɯ-jɤɣ tɕe tú-wɣ-ndzri ra, tɕe ɯ-ci a-pɯ-nɯ-ɬoʁ ra\cmn 洗完了衣服就要拧,这样水就会出来\end{exemple}
\begin{exemple}\jya a-ɕa ma-tɤ-tɯ-ndzri ma ɲɯ-mŋɤm\cmn 你不要拧我的肉,很痛!\end{exemple}
\begin{exemple}\jya rɟɯma kɤ-ndzri tɕe a-tɤ-ɤsɯɣ\cmn 你拧一下螺丝就会紧\end{exemple}\begin{sous-entrée}
\vedette{\hypertarget{}{\papi{ andzɯndzri}}}\markboth{andzɯndzri}{}\classe{vs}
\paradigme{\textit{dir :} \jya lɤ-}
\begin{définition}\fra être tordu, enroulé (fil, corde)\end{définition}
\begin{définition}\cmn 揪着(绳子)\end{définition}
\end{sous-entrée}\end{entrée}

\begin{entrée}
\vedette{\hypertarget{Ⓔndzrɯ}{\papi{ ndzrɯ}}}\markboth{ndzrɯ}{}\classe{n}
\begin{définition}\fra poinçon\end{définition}
\begin{définition}\cmn 凿子\end{définition}
\begin{exemple}\jya ndzrɯ pɯ-lat-a\cmn 我用了凿子\end{exemple}\end{entrée}

\begin{entrée}
\vedette{\hypertarget{Ⓔndzɯ}{\papi{ ndzɯ}}}\markboth{ndzɯ}{}
\classe{vt}
\paradigme{\textit{dir :} \jya tɤ-}
\begin{définition}\fra éduquer\end{définition}
\begin{définition}\cmn 教育\end{définition}
\begin{exemple}\jya tɤ-pɤtso kɤ-ndzɯ ra\cmn 一定要教育小孩子\end{exemple}
\begin{exemple}\jya a-mu kɯ tɤ́-wɣ-ndzɯ-a\cmn 我母亲教育了我\end{exemple}
\begin{relation-sémantique}\confer{
\hyperlink{Ⓔndzɯmbra}{\textit{ \papi{ndzɯmbra}}}
}\end{relation-sémantique}
\begin{relation-sémantique}\confer{
\hyperlink{Ⓔsindzɯ}{\textit{ \papi{sindzɯ}}}
}\end{relation-sémantique}\begin{sous-entrée}
\vedette{\hypertarget{}{\papi{ tɯ-sɯm,sɯndzɯ}}}\markboth{tɯ-sɯm,sɯndzɯ}{}\classe{vt}
\begin{définition}\fra réconforter\end{définition}
\begin{définition}\cmn 调解心态\end{définition}
\begin{exemple}\jya ɯʑo kɯ ɯ-sɯm ɲɯ-nɯ-sɯndzi\cmn 他在调解自己的心态\end{exemple}
\end{sous-entrée}\begin{sous-entrée}
\vedette{\hypertarget{}{\papi{ ʑɣɤndzɯ}}}\markboth{ʑɣɤndzɯ}{}\classe{vi}
\paradigme{\textit{dir :} \jya tɤ-}
\begin{définition}\fra se réconforter soi-même\end{définition}
\begin{définition}\cmn 宽慰自己\end{définition}
\begin{exemple}\jya tɯʑo kɤ-nɯ-ʑɣɤndzɯ a-pɯ-kɯ-cha ra\cmn 自己要会宽慰自己\end{exemple}
\end{sous-entrée}\end{entrée}

\begin{entrée}
\vedette{\hypertarget{Ⓔndzɯɣ}{\papi{ ndzɯɣ}}}\markboth{ndzɯɣ}{}\classe{vs}
\begin{définition}\fra soigneux\end{définition}
\begin{définition}\cmn 谨慎,做事很有条理\end{définition}
\begin{exemple}\jya nɤki tɯrme nɯ tɕhi tɤ-nɯ-mɯ-ma wuma ʑo kɯ-ndzɯɣ ci ŋu\cmn 这个人无论做什么事都非常谨慎\end{exemple}\begin{sous-entrée}
\vedette{\hypertarget{}{\papi{ kɯndzɯɣ}}}\markboth{kɯndzɯɣ}{}
\begin{définition}\fra on dirait que\end{définition}
\begin{définition}\cmn 看起来,好像是
\begin{déclaration}\use{沙尔宗和大藏方言词,干木鸟话说\stylefv{kɤti}}\end{déclaration}\end{définition}
\begin{exemple}\jya ɯʑo kɯ ɕoŋβzu ɲɯ-βze kɯndzɯɣ ŋu (=kɤti ŋu)\cmn 他看起来是在做木工的\end{exemple}
\end{sous-entrée}\end{entrée}

\begin{entrée}
\vedette{\hypertarget{Ⓔndzɯmbra}{\papi{ ndzɯmbra}}}\markboth{ndzɯmbra}{}\classe{vt}
\paradigme{\textit{dir :} \jya tɤ-}
\begin{définition}\fra éduquer\end{définition}
\begin{définition}\cmn 教育\end{définition}
\begin{exemple}\jya a-mu kɯ tɤ́-wɣ-ndzɯmbra-a\cmn 我母亲教育了我\end{exemple}
\begin{relation-sémantique}\confer{
\hyperlink{Ⓔndzɯ}{\textit{ \papi{ndzɯ}}}
}\end{relation-sémantique}\end{entrée}

\begin{entrée}
\vedette{\hypertarget{Ⓔndzɯŋɯ}{\papi{ ndzɯŋɯ}}}\markboth{ndzɯŋɯ}{}\classe{n}
\begin{définition}\fra récipient en terre\end{définition}
\begin{définition}\cmn 泥巴捏成的罐子\end{définition}
\end{entrée}

\begin{entrée}
\vedette{\hypertarget{Ⓔndzɯpe}{\papi{ ndzɯpe}}}\markboth{ndzɯpe}{}
\classe{n}
\begin{définition}\fra s'asseoir par terre avec les deux jambes l'une sur l'autre en travers (la manière dont les femmes doivent s'asseoir lorsqu'elles n'ont pas de travail à faire)\end{définition}
\begin{définition}\cmn 双腿斜在一边坐着(妇女坐的姿势)\end{définition}
\begin{exemple}\jya ndzɯpe nɯ-βzu-t-a\cmn 我坐了\end{exemple}
\begin{relation-sémantique}\confer{
\hyperlink{Ⓔsɯndzɯpe}{\textit{ \papi{sɯndzɯpe}}}
}\end{relation-sémantique}\end{entrée}

\begin{entrée}
\vedette{\hypertarget{Ⓔndzɯr}{\papi{ ndzɯr}}}\markboth{ndzɯr}{}\classe{vs}
\begin{définition}\fra être au complet\end{définition}
\begin{définition}\cmn 齐全\end{définition}
\begin{exemple}\jya tɯrme jo-ndzɯr-nɯ\cmn 人们齐全了\end{exemple}
\begin{relation-sémantique}\synonyme{
\hyperlink{Ⓔtshoz}{\textit{ \papi{tshoz}}}
}\end{relation-sémantique}\begin{sous-entrée}
\vedette{\hypertarget{}{\papi{ sɯɣndzɯr}}}\markboth{sɯɣndzɯr}{}\classe{vt}
\begin{définition}\ 
\begin{déclaration}\grammar{caus}\end{déclaration}\end{définition}
\begin{définition}\fra préparer au complet\end{définition}
\begin{définition}\cmn 准备齐全\end{définition}
\begin{relation-sémantique}\synonyme{
\hyperlink{Ⓔsɯxtshoz}{\textit{ \papi{sɯxtshoz}}}
}\end{relation-sémantique}
\end{sous-entrée}\end{entrée}

\begin{entrée}
\vedette{\hypertarget{Ⓔndzɯrnaʁ}{\papi{ ndzɯrnaʁ}}}\markboth{ndzɯrnaʁ}{}
\classe{n}
\begin{définition}\fra guêpe\end{définition}
\begin{définition}\cmn 马蜂\end{définition}
\begin{exemple}\jya ndzɯrnaʁ nɯ ɯʑo mɯ́j-wxti ri, ku-kɯ-mtsɯɣ tɕe wuma ɲɯ-ɤɣɯtɯɣ tɕe ʁʑɯnɯ kɯ tɤŋkhɯt tɤ-kɤ-lɤt fse tu-kɯ-ti ɲɯ-ŋu ma kɤ-kɯ-mtsɯɣ tɕe pjɯ-kɯ-tʂaβ ɲɯ-ŋgrɤl, laʁnɯ-sŋi ku-kɯ-ɕɯ-rŋgɯ ɲɯ-ngrɤl, tɕe ɯʑo sɯku ri ku-ndzoʁ tɕe ɯ-kho kɯ-wxtɯ-wxti tu-nɯ-βze ɲɯ-cha. ɯ-kho nɯ kɯ-ɤrtɯ-rtɯm ɲɯ-ŋu. ɯ-spa nɯ tɕhi ŋu ma ku-nɤmɯma tɕe kɯ-mnɯ-mnu kɯ-mpɯ-mpɯ ci ɲɯ-ŋu, ɯ-ŋgɯ tɯ-mɯ cho qale ku-ɕe mɯ́j-cha, ɯ-kɯ-spoʁ pa pjɯ-ru ɲɯ-ŋu tɕe ɯʑo ɯ-kɯm ɲɯ-ŋu. kú-wɣ-nɯrca tɕe ɲɯ-kɯ-mtsɯɣ ma nɯ maʁ nɤ mɯ́j-kɯ-mtsɯɣ.\cmn 马蜂虽小,蜇人时毒性特别大,据说同被年轻人打了一拳一样难受,会晕倒,一两天不能起床。它栖息在树上,可以做很大的窝,不知是什么材料作成的,摸起来很柔和、很软,风雨透不进。洞口朝下,是蜂窝的门。有人骚扰时就会蜇,不然不会轻易蜇人。\end{exemple}\end{entrée}

\begin{entrée}
\vedette{\hypertarget{Ⓔndʑu}{\papi{ ndʑu}}}\markboth{ndʑu}{}
\classe{n}
\begin{définition}\fra baguettes\end{définition}
\begin{définition}\cmn 筷子\end{définition}
\begin{exemple}\jya ndʑu kɯ tɤ-sɯ-mɟa-t-a\cmn 我用筷子夹了\end{exemple}
\begin{relation-sémantique}\confer{
\hyperlink{Ⓔandʑɯndʑu}{\textit{ \papi{andʑɯndʑu}}}
}\end{relation-sémantique}\end{entrée}

\begin{entrée}
\vedette{\hypertarget{Ⓔndʑa}{\papi{ ndʑa}}}\markboth{ndʑa}{}
\classe{n}
\begin{définition}\fra arc-en-ciel\end{définition}
\begin{définition}\cmn 彩虹
\begin{déclaration} \étymologie{\papi{ndʑa}}\end{déclaration}\end{définition}
\begin{exemple}\jya thaχtsa ɯ-rkɯ khatoʁ lu-kɯ-ɕe nɯ ɯ-ndʑa rmi\cmn 
花带边的彩色竖条叫\stylefv{ɯ-ndʑa}
\end{exemple}\end{entrée}

\begin{entrée}
\vedette{\hypertarget{Ⓔndʑaʁ}{\papi{ ndʑaʁ}}}\markboth{ndʑaʁ}{}\classe{vi}
\paradigme{\textit{dir :} \jya \_}\acception{1}
\begin{définition}\fra flotter, nager\end{définition}
\begin{définition}\cmn 漂浮;游泳\end{définition}
\begin{exemple}\jya kɤ-ndʑaʁ ndɤre sɤɣmu tɕe aj mɤ-cha-a\cmn 游泳很恐怖,我不敢\end{exemple}\acception{2}
\begin{définition}\fra traverser la rivière à gué\end{définition}
\begin{définition}\cmn 涉过去\end{définition}\end{entrée}

\begin{entrée}
\vedette{\hypertarget{Ⓔndʑɤβ}{\papi{ ndʑɤβ}}}\markboth{ndʑɤβ}{}\classe{vi}
\paradigme{\textit{dir :} \jya tɤ-}
\paradigme{\textit{dir :} \jya pɯ-}
\begin{définition}\ 
\begin{déclaration}\grammar{acaus}\end{déclaration}\end{définition}
\begin{définition}\fra brûler\end{définition}
\begin{définition}\cmn 燃烧\end{définition}
\begin{exemple}\jya ɕoʁɕoʁ to-ndʑɤβ\cmn 纸燃起来了\end{exemple}
\begin{exemple}\jya xɕaj a-tɤ-ndʑɤβ tɕe pe ma ɯ-fsɤqhe tɕe tɕe nɯ ɯ-stu nɯ kɯ-tshu tu-ɬoʁ ŋu, tɕe fsapaʁ ra rga-nɯ\cmn 草烧了是一件好事,因为第二年那个地方就会长出茂盛的草,牲畜们喜欢\end{exemple}
\begin{exemple}\jya tɯ-ŋga pjɤ-ndʑɤβ\cmn 衣服烧了\end{exemple}
\begin{exemple}\jya kha pjɤ-ndʑɤβ\cmn 房子烧了\end{exemple}
\begin{relation-sémantique}\confer{
\hyperlink{Ⓔtɕɤβ}{\textit{ \papi{tɕɤβ}}}
}\end{relation-sémantique}
\begin{relation-sémantique}\confer{
\hyperlink{Ⓔɣndʑɤβ}{\textit{ \papi{ɣndʑɤβ}}}
}\end{relation-sémantique}\end{entrée}

\begin{entrée}
\vedette{\hypertarget{Ⓔndʑɤm}{\papi{ ndʑɤm}}}\markboth{ndʑɤm}{}\classe{vs}
\paradigme{\textit{dir :} \jya thɯ-}\acception{1}
\begin{définition}\fra chaud, tiède\end{définition}
\begin{définition}\cmn 温暖
\begin{déclaration} \étymologie{\papi{ⁿdʑam}}\end{déclaration}\end{définition}
\begin{exemple}\jya kɯ-ndʑɤm tɤ-ndze\cmn 趁热吃!\end{exemple}\acception{2}
\begin{définition}\fra tendre (personne)\end{définition}
\begin{définition}\cmn 温柔\end{définition}
\begin{relation-sémantique}\confer{
\hyperlink{Ⓔɣɤndʑɤm}{\textit{ \papi{ɣɤndʑɤm}}}
}\end{relation-sémantique}\end{entrée}

\begin{entrée}
\vedette{\hypertarget{Ⓔndʑɤrndʑɤr}{\papi{ ndʑɤrndʑɤr}}}\markboth{ndʑɤrndʑɤr}{}
\classe{idph.2}
\begin{définition}\fra très fin\end{définition}
\begin{définition}\cmn 很细\end{définition}
\begin{exemple}\jya kɯ-xtshɯ-xtshɯm ndʑɤrndʑɤr ɲɯ-ŋu\cmn 非常细\end{exemple}
\begin{exemple}\jya tɤ-ri kɯ-fse ci ndʑɤrndʑɤr pjɤ-ri\end{exemple}\end{entrée}

\begin{entrée}
\vedette{\hypertarget{Ⓔndʑɤrtaʁ}{\papi{ ndʑɤrtaʁ}}}\markboth{ndʑɤrtaʁ}{}\classe{n}
\begin{définition}\fra fourchette de bois\end{définition}
\begin{définition}\cmn 木叉子\end{définition}
\begin{relation-sémantique}\confer{
\hyperlink{Ⓔtɤ-rtaʁ}{\textit{ \papi{tɤ-rtaʁ}}}
}\end{relation-sémantique}
\begin{relation-sémantique}\confer{
\hyperlink{Ⓔndʑu}{\textit{ \papi{ndʑu}}}
}\end{relation-sémantique}\end{entrée}

\begin{entrée}
\vedette{\hypertarget{Ⓔndʑɣaʁ}{\papi{ ndʑɣaʁ}}}\markboth{ndʑɣaʁ}{}
\classe{vi}
\paradigme{\textit{dir :} \jya tɤ-}
\paradigme{\textit{dir :} \jya thɯ-}
\paradigme{\textit{dir :} \jya pɯ-}
\begin{définition}\ 
\begin{déclaration}\grammar{acaus}\end{déclaration}\end{définition}
\begin{définition}\fra s'extraire (spontanément)\end{définition}
\begin{définition}\cmn (自动)挤出来\end{définition}
\begin{exemple}\jya tɤ-ndʑɣaʁ\cmn (脓包)挤出来了\end{exemple}
\begin{exemple}\jya ɯ-tɯcɯste chɤ-ndʑɣaʁ\cmn 她羊水破了\end{exemple}
\begin{relation-sémantique}\confer{
\hyperlink{Ⓔtɕɣaʁ}{\textit{ \papi{tɕɣaʁ}}}
}\end{relation-sémantique}\end{entrée}

\begin{entrée}
\vedette{\hypertarget{Ⓔndʑɣɤrndʑɣɤr}{\papi{ ndʑɣɤrndʑɣɤr}}}\markboth{ndʑɣɤrndʑɣɤr}{}\classe{idph.2}
\begin{définition}\fra long (à propos des dents)\end{définition}
\begin{définition}\cmn 形容牙齿长\end{définition}
\begin{exemple}\jya ɯ-ɕɣa ɯ-tɯ-zri kɯ ndʑɣɤrndʑɣɤr ʑo ɲɯ-pa\cmn 他的牙齿很长\end{exemple}\end{entrée}

\begin{entrée}
\vedette{\hypertarget{Ⓔndʑiŋgri}{\papi{ ndʑiŋgri}}}\markboth{ndʑiŋgri}{}
\classe{n}
\begin{définition}\fra Sambucus sp.\end{définition}
\begin{définition}\cmn 血满草【臭草】\end{définition}
\begin{exemple}\jya ndʑiŋgri ɯ-di wuma ʑo sɤjloʁ, sɯjno kɯ-wxti tsa ci ŋu, ɯ-jwaʁ kɯ-tɕɤr tɕe kɯ-rɲɟi tsa ŋu, ɯ-βzɯr nɯ ra ɯ-ɕɣa kɯ-fse kɯ-xtɕɯ-xtɕi tu, ɯ-mɯntoʁ ɯ-tshɯɣa nɯ san tsa fse, ɯ-mdoʁ kɯ-wɣrum ŋu, ɯ-mat thɯ-tɯt tɕe, kɯ-ɣɯrni ŋu, wuma ʑo qiaβ.\cmn 臭草是一种发出臭味的大植物,叶子窄而长,叶边有小齿,花的形状像伞,花是白色的,果实成熟后是红色的,很苦。\end{exemple}\end{entrée}

\begin{entrée}
\vedette{\hypertarget{Ⓔndʑiʑo}{\papi{ ndʑiʑo}}}\markboth{ndʑiʑo}{}\classe{pro}
\begin{définition}\fra vous deux\end{définition}
\begin{définition}\cmn 你们俩\end{définition}
\end{entrée}

\begin{entrée}
\vedette{\hypertarget{Ⓔndʑrɯ}{\papi{ ndʑrɯ}}}\markboth{ndʑrɯ}{}
\classe{n}
\begin{définition}\fra lente\end{définition}
\begin{définition}\cmn 虮子\end{définition}
\begin{relation-sémantique}\confer{
\hyperlink{Ⓔzrɯɣ}{\textit{ \papi{zrɯɣ}}}
}\end{relation-sémantique}\end{entrée}

\begin{entrée}
\vedette{\hypertarget{Ⓔndʑrɯɕɤt}{\papi{ ndʑrɯɕɤt}}}\markboth{ndʑrɯɕɤt}{}\classe{n}
\begin{définition}\fra peigne à lentes\end{définition}
\begin{définition}\cmn 捉虱子用的梳子\end{définition}
\end{entrée}

\begin{entrée}
\vedette{\hypertarget{Ⓔndʑɯ}{\papi{ ndʑɯ}}}\markboth{ndʑɯ}{}
\classe{vt}
\paradigme{\textit{dir :} \jya kɤ-}
\begin{définition}\fra accuser\end{définition}
\begin{définition}\cmn 告状;投诉
\begin{déclaration} \étymologie{\papi{ʑu}}\end{déclaration}\end{définition}
\begin{exemple}\jya khɯtsa pɯ-tɯ-qrɯt, aʑo mɤ-ta-ndʑɯ\cmn 你把碗打破了,我不会告你的状\end{exemple}\begin{sous-entrée}
\vedette{\hypertarget{}{\papi{ sɤndʑɯ}}}\markboth{sɤndʑɯ}{}\classe{vi}
\begin{définition}\ 
\begin{déclaration}\grammar{apass}\end{déclaration}\end{définition}
\begin{définition}\fra accuser (des gens)\end{définition}
\begin{définition}\cmn 告状\end{définition}
\end{sous-entrée}\end{entrée}

\begin{entrée}
\vedette{\hypertarget{Ⓔndʑɯɣ}{\papi{ ndʑɯɣ}}}\markboth{ndʑɯɣ}{}
\classe{vi}
\paradigme{\textit{dir :} \jya nɯ-}
\begin{définition}\fra être détruit\end{définition}
\begin{définition}\cmn 灭亡
\begin{déclaration} \étymologie{\papi{ⁿdʑig}}\end{déclaration}\end{définition}
\begin{exemple}\jya skalpa ɲɤ-ndʑɯɣ\cmn 世界灭亡了;时代变了\end{exemple}\end{entrée}

\begin{entrée}
\vedette{\hypertarget{Ⓔndʑɯnɯ}{\papi{ ndʑɯnɯ}}}\markboth{ndʑɯnɯ}{}
\classe{n}
\begin{définition}\fra Angelica sp.\end{définition}
\begin{définition}\cmn 当归\end{définition}
\begin{exemple}\jya ndʑɯnɯ nɯ sɯjno ci ŋu. sɯŋgɯ tu-kɯ-ɬoʁ ci tu, lu-kɤ-nɯ-ji ci tu tɕe lonba ʑo kɯ-naχtɕɯɣ ɕti, ɯ-jwaʁ dɤn, ndɯβ, ɯ-χcɤl ɯ-ru tu-ɬoʁ tɕe, ɯ-jwaʁ cho aɣɯmdoʁ, ɯ-ru ɯ-kɤχcɤl tɕe, ɯ-mɯntoʁ ɲɯ-lɤt ŋu, ɯ-mɯntoʁ wɣrum, ɯ-mat dɤn. ɯ-di χɕɤβ, ɯ-qa nɯ kú-wɣ-sqa tɕe tú-wɣ-ndza tɕe smɤn kɯ-ʑru ci ɲɯ-ŋu. ɯ-jwaʁ kɤ-ndza sna.\cmn 当归是一种草,有的长在森林了,也有自己种的,都完全一样。叶子多、细小,中间长茎,茎和叶子颜色相同。茎的顶端开花,花是白色的,果实多,香味浓。根煮来吃是一种名贵的药。叶子也可以吃。\end{exemple}\end{entrée}

\begin{entrée}
\vedette{\hypertarget{Ⓔndʑɯrpɯt}{\papi{ ndʑɯrpɯt}}}\markboth{ndʑɯrpɯt}{}
\classe{vs}
\paradigme{\textit{dir :} \jya thɯ-}
\paradigme{\textit{dir :} \jya tɤ-}
\begin{définition}\fra être engourdi\end{définition}
\begin{définition}\cmn 麻木\end{définition}
\begin{exemple}\jya a-mi ɲɯ-ndʑɯrpɯt\cmn 我的脚是麻木的\end{exemple}
\begin{relation-sémantique}\confer{
\hyperlink{Ⓔɣɤzoŋzoŋ}{\textit{ \papi{ɣɤzoŋzoŋ}}}
}\end{relation-sémantique}\end{entrée}

\begin{entrée}
\vedette{\hypertarget{Ⓔndʑɯrwɯz}{\papi{ ndʑɯrwɯz}}}\markboth{ndʑɯrwɯz}{}\classe{n}
\begin{définition}\fra Sonchus brachyotus\end{définition}
\begin{définition}\cmn 苣荬菜【苦苦草】\end{définition}
\begin{exemple}\jya ndʑɯrwɯz nɯ sɯjno ci ŋu, ɯ-jwaʁ nɯ kɯ-tɕɤr kɯ-rɲɟi tsa ŋu, kɯ-ɤɣrɤɣrum tsa ŋu, ɯ-spjɯŋ tu-ɬoʁ tɕe, kɯ-zri tsa tɯ-ɟom jarma tu-βze cha. ɯ-ru ɯ-taʁ ɯ-jwaʁ tu-oʑɯrja ŋu. ɯ-kɤχcɤl tɕe ɲɯ-rɯmɯntoʁ ŋu. ɯ-mɯntoʁ kɯ-qarŋe ŋu. ɯ-jwaʁ ɯ-βzɯr nɯ ra ɯ-mdzu kɯ-fse kɯ-xtɕɯ-xtɕi tu. pɯ́-wɣ-qlɯt tɕe, ɯ-lu tu, qiaβ. fsapaʁ ndza ma mɤ-sna.\cmn 苦苦草是一种植物,叶子窄而长,带有白色,茎长出来时,可以长到一米来高。叶子排列在茎上。茎的顶上开花。花是黄色的。叶子边缘有小刺。折断时有乳汁,很苦,只能喂牲畜。\end{exemple}\end{entrée}

\begin{entrée}
\vedette{\hypertarget{Ⓔnétɕi}{\papi{ nétɕi}}}\markboth{nétɕi}{}
\classe{part}
\begin{définition}\fra non ?\end{définition}
\begin{définition}\cmn 吧\end{définition}
\begin{exemple}\jya nɤ-smɤn mɯ-ko-tɯ-tshi-t nétɕi\cmn 你没有喝药吧?\end{exemple}
\begin{exemple}\jya tɤ-mbɣom tu-ti-a ŋu nétɕi\cmn 我不是叫你快点吗?\end{exemple}
\begin{exemple}\jya japa kɯnɤ nɤ-ɴɢar ɯ-ŋgɯ tɤ-se pɯ-tu nétɕi\cmn 去年你的痰也有痰对吧\end{exemple}\end{entrée}

\begin{entrée}
\vedette{\hypertarget{Ⓔngu}{\papi{ ngu}}}\markboth{ngu}{}
\classe{vt}\acception{1}
\paradigme{\textit{dir :} \jya pɯ-}
\begin{définition}\fra donner à manger (aux animaux)\end{définition}
\begin{définition}\cmn 喂(动物)\end{définition}
\begin{exemple}\jya paʁ pɯ-ngu-t-a\cmn 我喂了猪\end{exemple}
\begin{exemple}\jya fsapaʁ ra pɯ-ngu-t-a\cmn 我喂了牲畜\end{exemple}\acception{2}
\paradigme{\textit{dir :} \jya nɯ-}
\begin{définition}\fra donner la bouchée\end{définition}
\begin{définition}\cmn 亲口喂\end{définition}
\begin{exemple}\jya pɣɤmu kɯ ɯ-pɯ na-ngu\cmn 母鸡喂了小鸡\end{exemple}
\begin{exemple}\jya ɯ-pɯ ɲɯ-nge ɲɯ-ŋu\cmn (母鸡)在喂小鸡\end{exemple}
\begin{relation-sémantique}\synonyme{
\hyperlink{Ⓔχsu}{\textit{ \papi{χsu}}}
}\end{relation-sémantique}
\begin{relation-sémantique}\synonyme{
\hyperlink{Ⓔɕpɯt}{\textit{ \papi{ɕpɯt}}}
}\end{relation-sémantique}\end{entrée}

\begin{entrée}
\vedette{\hypertarget{Ⓔngɤjtshi}{\papi{ ngɤjtshi}}}\markboth{ngɤjtshi}{}\classe{vt}
\paradigme{\textit{dir :} \jya nɯ-}
\begin{définition}\ 
\begin{déclaration}\grammar{comp}\end{déclaration}\end{définition}
\begin{définition}\fra nourrir\end{définition}
\begin{définition}\cmn 喂(给别人吃喝)\end{définition}
\begin{exemple}\jya tɤwɯ nɯ kɯ ɲó-wɣ-ngɤjtshi\cmn 老年人喂了他们\end{exemple}
\begin{relation-sémantique}\confer{
\hyperlink{Ⓔngu}{\textit{ \papi{ngu}}}
}\end{relation-sémantique}
\begin{relation-sémantique}\confer{
\hyperlink{Ⓔjtshi}{\textit{ \papi{jtshi}}}
}\end{relation-sémantique}
\begin{relation-sémantique}\confer{
\hyperlink{Ⓔmbijtshi}{\textit{ \papi{mbijtshi}}}
}\end{relation-sémantique}\end{entrée}

\begin{entrée}
\vedette{\hypertarget{Ⓔnge}{\papi{ nge}}}\markboth{nge}{}\classe{vs}
\begin{définition}\fra très solide\end{définition}
\begin{définition}\cmn 非常结实\end{définition}
\begin{exemple}\jya laχtɕha tɤ́-wɣ-χtɯ tɕe, kɯ-ngɯ-nge tsa tú-wɣ-nɯ-qɤr ra, tɕe tɤ́-wɣ-ntɕhoz khro mdɯ\cmn 买东西的的时候,要挑选结实的,这样就可以用得长久\end{exemple}
\begin{relation-sémantique}\confer{
\hyperlink{Ⓔngɯnge}{\textit{ \papi{ngɯnge}}}
}\end{relation-sémantique}\end{entrée}

\begin{entrée}
\vedette{\hypertarget{Ⓔngo}{\papi{ ngo}}}\markboth{ngo}{}
\classe{vi}
\paradigme{\textit{dir :} \jya tɤ-}
\begin{définition}\fra tomber malade\end{définition}
\begin{définition}\cmn 生病\end{définition}
\begin{exemple}\jya ɲɯ-nɯtɕhomba-a tɕe tɤ-ngo-a\cmn 我感冒了,生病了\end{exemple}
\begin{exemple}\jya kɤ-ngo ɲɯ-nɯɣme-a\cmn 我很怕生病\end{exemple}
\begin{relation-sémantique}\confer{
\hyperlink{Ⓔɣɤtsɤngo}{\textit{ \papi{ɣɤtsɤngo}}}
}\end{relation-sémantique}
\begin{relation-sémantique}\confer{
\hyperlink{Ⓔɕɯngo}{\textit{ \papi{ɕɯngo}}}
}\end{relation-sémantique}
\begin{relation-sémantique}\confer{
\hyperlink{Ⓔtɯ-ŋgo}{\textit{ \papi{tɯ-ŋgo}}}
}\end{relation-sémantique}\end{entrée}

\begin{entrée}
\vedette{\hypertarget{Ⓔngrɯβ}{\papi{ ngrɯβ}}}\markboth{ngrɯβ}{}
\paradigme{\textit{dir :} \jya pɯ-}
\begin{définition}\fra accomplir\end{définition}
\begin{définition}\cmn 成功;完成
\begin{déclaration} \étymologie{\papi{ⁿgrub}}\end{déclaration}\end{définition}
\begin{exemple}\jya nɤ-kɤ-nɯsmɯlɤm nɯ a-pɯ-ngrɯβ\cmn 希望你祈求的事情会成功!\end{exemple}\begin{sous-entrée}
\vedette{\hypertarget{}{\papi{ sɯngrɯβ}}}\markboth{sɯngrɯβ}{}\classe{vt}
\begin{définition}\fra réaliser\end{définition}
\begin{définition}\cmn 使……成功\end{définition}
\end{sous-entrée}\end{entrée}

\begin{entrée}
\vedette{\hypertarget{Ⓔngɯnge}{\papi{ ngɯnge}}}\markboth{ngɯnge}{}\classe{vs}
\begin{définition}\fra résistant\end{définition}
\begin{définition}\cmn 结实\end{définition}
\begin{relation-sémantique}\confer{
\hyperlink{Ⓔngɯt}{\textit{ \papi{ngɯt}}}
}\end{relation-sémantique}
\begin{relation-sémantique}\confer{
\hyperlink{Ⓔnge}{\textit{ \papi{nge}}}
}\end{relation-sémantique}\end{entrée}

\begin{entrée}
\vedette{\hypertarget{Ⓔngɯt}{\papi{ ngɯt}}}\markboth{ngɯt}{}\classe{vs}
\paradigme{\textit{dir :} \jya tɤ-}
\begin{définition}\fra solide\end{définition}
\begin{définition}\cmn 结实\end{définition}
\begin{exemple}\jya kɯki ɲɯ-ngɯt\cmn 这个东西很结实\end{exemple}
\begin{exemple}\jya tɯmbri ɲɯ-ngɯt\cmn 绳子很结实\end{exemple}
\begin{exemple}\jya tɯ-ŋga ɲɯ-ngɯt\cmn 衣服很结实\end{exemple}
\begin{relation-sémantique}\confer{
\hyperlink{Ⓔngɯnge}{\textit{ \papi{ngɯnge}}}
}\end{relation-sémantique}
\begin{relation-sémantique}\confer{
\hyperlink{Ⓔnge}{\textit{ \papi{nge}}}
}\end{relation-sémantique}\begin{sous-entrée}
\vedette{\hypertarget{}{\papi{ ɣɤngɯt}}}\markboth{ɣɤngɯt}{}\classe{vt}
\paradigme{\textit{dir :} \jya tɤ-}
\begin{définition}\fra rendre solide\end{définition}
\begin{définition}\cmn 加固\end{définition}
\end{sous-entrée}\begin{sous-entrée}
\vedette{\hypertarget{}{\papi{ zɣɤngɯt}}}\markboth{zɣɤngɯt}{}\classe{vt}
\begin{définition}\fra rendre solide avec ...\end{définition}
\begin{définition}\cmn 用……加固\end{définition}
\begin{exemple}\jya tɯ-ŋga ɯ-tɯ-tʂɯβ kɤntɕhɯ kɤ-lat-a tɕe kɤ-zɣɤngɯt-a\cmn 我把衣服多缝了几道,使它更结实了\end{exemple}
\end{sous-entrée}\end{entrée}

\begin{entrée}
\vedette{\hypertarget{Ⓔni}{\papi{ ni}}}\markboth{ni}{}\classe{det}
\begin{définition}\fra duel\end{définition}
\begin{définition}\cmn 双数\end{définition}
\end{entrée}

\begin{entrée}
\vedette{\hypertarget{Ⓔnmu}{\papi{ nmu}}}\markboth{nmu}{}
\classe{vi}
\paradigme{\textit{dir :} \jya nɯ-}
\begin{définition}\fra trembler (tremblement de terre)\end{définition}
\begin{définition}\cmn 震动(地震)\end{définition}
\begin{exemple}\jya waɟɯ ɲɤ-nmu\cmn 发生了地震\end{exemple}
\begin{relation-sémantique}\confer{
\hyperlink{Ⓔmɯnmu}{\textit{ \papi{mɯnmu}}}
}\end{relation-sémantique}\end{entrée}

\begin{entrée}
\vedette{\hypertarget{Ⓔnúndʐa}{\papi{ núndʐa}}}\markboth{núndʐa}{}\classe{adv}
\begin{définition}\fra pour cette raison\end{définition}
\begin{définition}\cmn 因此\end{définition}\end{entrée}

\begin{entrée}
\vedette{\hypertarget{Ⓔnŋo}{\papi{ nŋo}}}\markboth{nŋo}{}
\classe{vi}
\paradigme{\textit{dir :} \jya pɯ-}
\begin{définition}\fra échouer, perdre\end{définition}
\begin{définition}\cmn 败;输\end{définition}
\begin{exemple}\jya nɤʑo pɯ-tɯ-nŋo\cmn 你输了\end{exemple}
\begin{exemple}\jya nɤ-ɕki pɯ-nŋo-a\cmn 我输给你了\end{exemple}
\begin{relation-sémantique}\confer{
\hyperlink{Ⓔɕɯnŋo}{\textit{ \papi{ɕɯnŋo}}}
}\end{relation-sémantique}\end{entrée}

\begin{entrée}
\vedette{\hypertarget{Ⓔno}{\papi{ no}}}\markboth{no}{}
\classe{vt}\acception{1}
\paradigme{\textit{dir :} \jya \_}
\begin{définition}\fra mener (animaux), chasser\end{définition}
\begin{définition}\cmn 赶;驱逐\end{définition}
\begin{définition}\fra clouer\end{définition}
\begin{définition}\cmn 钉(钉子 )\end{définition}
\begin{exemple}\jya fsapaʁ kɤ-nɤm\cmn 你赶牲畜吧\end{exemple}\acception{2}
\begin{exemple}\jya ɕɤmtshoʁ kɤ-no-t-a\cmn 我钉了钉子\end{exemple}
\begin{exemple}\jya ɕɤmtshoʁ kɤ-nɤm\cmn 你钉钉子吧\end{exemple}
\begin{exemple}\jya tɤtshoʁ pɯ-nɤm\cmn 你钉木钉吧\end{exemple}
\begin{relation-sémantique}\confer{
\hyperlink{Ⓔnɯno}{\textit{ \papi{nɯno}}}
}\end{relation-sémantique}\end{entrée}

\begin{entrée}
\vedette{\hypertarget{Ⓔnoŋstɤn}{\papi{ noŋstɤn}}}\markboth{noŋstɤn}{}\classe{n}
\begin{définition}\fra tapis de selle\end{définition}
\begin{définition}\cmn 鞍垫
\begin{déclaration} \étymologie{\papi{naŋ.stan}}\end{déclaration}\end{définition}\end{entrée}

\begin{entrée}
\vedette{\hypertarget{Ⓔnor}{\papi{ nor}}}\markboth{nor}{}\classe{vi}
\paradigme{\textit{dir :} \jya nɯ-}
\begin{définition}\fra se tromper\end{définition}
\begin{définition}\cmn 弄错(指不严重的错误)
\begin{déclaration} \étymologie{\papi{nor}}\end{déclaration}\end{définition}
\begin{exemple}\jya ɲɤ-nor\cmn 他弄错了\end{exemple}
\begin{relation-sémantique}\synonyme{
\hyperlink{Ⓔnɯkɯmaʁ}{\textit{ \papi{nɯkɯmaʁ}}}
}\end{relation-sémantique}\end{entrée}

\begin{entrée}
\vedette{\hypertarget{Ⓔnóʁmɯz}{\papi{ nóʁmɯz}}}\markboth{nóʁmɯz}{}\classe{adv}
\begin{définition}\fra alors seulement\end{définition}
\begin{définition}\cmn 那才\end{définition}
\begin{relation-sémantique}\confer{
\hyperlink{Ⓔkóʁmɯz}{\textit{ \papi{kóʁmɯz}}}
}\end{relation-sémantique}\end{entrée}

\begin{entrée}
\vedette{\hypertarget{Ⓔntaβ}{\papi{ ntaβ}}}\markboth{ntaβ}{}\classe{vs}
\paradigme{\textit{dir :} \jya kɤ-}
\begin{définition}\fra stable\end{définition}
\begin{définition}\cmn 稳当\end{définition}
\begin{exemple}\jya ɯ-mdzɯ ko-ntaβ ma kɯmaʁ rɯsɯso mɯ-ɲɤ-ra\cmn 他安心了,不需要再考虑其它事情\end{exemple}
\begin{exemple}\jya ɯ-sɯm ko-ntaβ\cmn 他放心了\end{exemple}
\begin{exemple}\jya ɯ-ʑɯβ ko-ntaβ\cmn 他睡得很熟\end{exemple}\begin{sous-entrée}
\vedette{\hypertarget{}{\papi{ ɕɯntaβ}}}\markboth{ɕɯntaβ}{}\classe{vt}
\paradigme{\textit{dir :} \jya nɯ-}
\begin{définition}\ 
\begin{déclaration}\grammar{caus}\end{déclaration}\end{définition}
\begin{définition}\fra laisser là\end{définition}
\begin{définition}\cmn 放在那里\end{définition}
\begin{exemple}\jya kɤ-ndza mɯ-mɤ-ɲɯ-tɯ-ɕkɯt nɤ, nɯ-ɕɯ-ntaβ jɤɣ\cmn 你如果吃不完的话,可以放在那里\end{exemple}
\end{sous-entrée}\begin{sous-entrée}
\vedette{\hypertarget{}{\papi{ ʑɣɤɕɯntaβ}}}\markboth{ʑɣɤɕɯntaβ}{}\classe{vi}
\begin{définition}\ 
\begin{déclaration}\grammar{refl}\end{déclaration}
\begin{déclaration}\grammar{caus}\end{déclaration}\end{définition}
\begin{définition}\fra rester sans rien faire\end{définition}
\begin{définition}\cmn 该做的不想做;不想动\end{définition}
\begin{exemple}\jya ɲɯ-ʑɣɤ-ɕɯntaβ\cmn 他安静下来了\end{exemple}
\begin{relation-sémantique}\confer{
\hyperlink{Ⓔɣɤntaβ}{\textit{ \papi{ɣɤntaβ}}}
}\end{relation-sémantique}
\end{sous-entrée}\end{entrée}

\begin{entrée}
\vedette{\hypertarget{Ⓔntɕha}{\papi{ ntɕha}}}\markboth{ntɕha}{}
\classe{vt}\acception{1}
\paradigme{\textit{dir :} \jya pɯ-}
\begin{définition}\fra tuer (animal)\end{définition}
\begin{définition}\cmn 宰(动物)\end{définition}\acception{2}
\paradigme{\textit{dir :} \jya nɯ-}
\begin{définition}\fra découper en morceaux (animal)\end{définition}
\begin{définition}\cmn 剥皮;切分(动物)\end{définition}
\begin{exemple}\jya tshɤt pjɤ-si tɕe nɯ-ntɕhe\cmn 山羊死了,切分了它\end{exemple}
\begin{exemple}\jya ji-tshɤt pjɤ-si tɕe na-ntɕha\cmn 我们的山羊死了,他就把它切分了\end{exemple}\begin{sous-entrée}
\vedette{\hypertarget{}{\papi{ rɤntɕha}}}\markboth{rɤntɕha}{}\classe{vi}
\paradigme{\textit{dir :} \jya pɯ-}
\begin{définition}\fra tuer des animaux\end{définition}
\begin{définition}\cmn 屠宰\end{définition}
\end{sous-entrée}\end{entrée}

\begin{entrée}
\vedette{\hypertarget{Ⓔntɕhɤr}{\papi{ ntɕhɤr}}}\markboth{ntɕhɤr}{}
\classe{vi}
\paradigme{\textit{dir :} \jya kɤ-}
\begin{définition}\fra éclairer\end{définition}
\begin{définition}\cmn 映照
\begin{déclaration} \étymologie{\papi{ⁿtɕʰar}}\end{déclaration}\end{définition}
\begin{exemple}\jya @dianying ko-ntɕhɤr\cmn 电影上映了\end{exemple}
\begin{exemple}\jya tɤŋe ɯ-ɣot ko-ntɕhɤr\cmn 太阳光照下来了\end{exemple}
\begin{exemple}\jya χɕɤlzgoŋ ɯ-ŋgɯ ko-ntɕhar-a\cmn 我在镜子里\end{exemple}
\begin{exemple}\jya ɯ-jmŋo ɯ-ŋgɯ sɯŋgi ko-ntɕhɤr\cmn 他梦见了狮子\end{exemple}
\begin{exemple}\jya χɕɤlzgoŋ ɯ-ŋgɯ kɤ-ntɕhɤr-tɕi nɯ ra pɯ-mto-t-a\cmn 我在镜子里看见了我们俩的照影\end{exemple}
\begin{exemple}\jya jɯfɕɯɕɤr a-ʑɯβ ɯ-mɤ-tɯ-ɣi kɯ ʑa ɯ-mɤ-fsoʁ ma kɤ-sɯso ʑo ɲɯ-ntɕhɤr\cmn 我昨天晚上睡不着,在我想象中,多么希望早点天亮\end{exemple}\end{entrée}

\begin{entrée}
\vedette{\hypertarget{Ⓔntɕhɣaʁ}{\papi{ ntɕhɣaʁ}}}\markboth{ntɕhɣaʁ}{}
\classe{vi}
\paradigme{\textit{dir :} \jya kɤ-}
\paradigme{\textit{dir :} \jya tɤ-}
\begin{définition}\fra éclabousser\end{définition}
\begin{définition}\cmn 溅起来\end{définition}
\begin{exemple}\jya tɯ-ci kɤ-ntɕhɣaʁ\cmn 水溅起来了\end{exemple}
\begin{exemple}\jya tɯ-ci a-taʁ tɤ-ntɕhɣaʁ\cmn 水溅到我身上\end{exemple}
\begin{exemple}\jya tɤrcoʁ kɤ-ntɕhɣaʁ\cmn 稀泥溅起来了\end{exemple}\begin{sous-entrée}
\vedette{\hypertarget{}{\papi{ sɯntɕhɣaʁ}}}\markboth{sɯntɕhɣaʁ}{}\classe{vt}
\paradigme{\textit{dir :} \jya kɤ-}
\begin{définition}\fra éclabousser\end{définition}
\begin{définition}\cmn 使溅起来\end{définition}
\begin{exemple}\jya tɤ-rɯndzaŋspa tsa, tɯ-ci nɯ ma-kɤ-tɯ-sɯntɕhɣaʁ\cmn 小心一点,不要让水溅起来\end{exemple}
\end{sous-entrée}\end{entrée}

\begin{entrée}
\vedette{\hypertarget{Ⓔntɕhomŋga}{\papi{ ntɕhomŋga}}}\markboth{ntɕhomŋga}{}\classe{n}
\begin{définition}\fra habits de danse\end{définition}
\begin{définition}\cmn 跳神时穿的服装\end{définition}
\end{entrée}

\begin{entrée}
\vedette{\hypertarget{Ⓔntɕhoz}{\papi{ ntɕhoz}}}\markboth{ntɕhoz}{}
\classe{vt}
\paradigme{\textit{dir :} \jya tɤ-}
\paradigme{\textit{dir :} \jya kɤ-}
\begin{définition}\fra utiliser\end{définition}
\begin{définition}\cmn 使用\end{définition}
\begin{exemple}\jya khɯtsa kɤ-ntɕhoz-a\cmn 我用了碗\end{exemple}
\begin{exemple}\jya nɤʑo kɤ-ntɕhoz ɯ-tɯ-spe?\cmn 你会不会用?\end{exemple}
\begin{exemple}\jya ``ɯɟɤm" nɯ tɯ-rju ɯ-ŋgɯ tɕe kɤ-ntɕhoz tɕhi tú-wɣ-stu ŋu"\cmn 
怎么在句子中用\stylefv{ɯɟɤm}这个词
\end{exemple}
\begin{exemple}\jya nɯnɯ ŋotɕu tɤ-tɯ-nɯ-tɕhɯ-ntɕhoz khɯ\cmn (这一句话)你在什么环境都可以用\end{exemple}
\begin{exemple}\jya kɤ-ntɕhoz tɕhi ɕɯ-ste ɲɯ-ŋu?\cmn 这有什么用呢?\end{exemple}\begin{sous-entrée}
\vedette{\hypertarget{}{\papi{ nɯɣɯntɕhoz}}}\markboth{nɯɣɯntɕhoz}{}\classe{vs}
\begin{définition}\ 
\begin{déclaration}\grammar{facil}\end{déclaration}\end{définition}
\begin{définition}\fra facile à utiliser\end{définition}
\begin{définition}\cmn 容易用\end{définition}
\begin{exemple}\jya ki qaʁ ki wuma ɲɯ-nɯɣɯntɕhoz\cmn 这个锄头很好用\end{exemple}
\end{sous-entrée}\end{entrée}

\begin{entrée}
\vedette{\hypertarget{Ⓔntɕhɯɣ}{\papi{ ntɕhɯɣ}}}\markboth{ntɕhɯɣ}{}
\classe{vi}
\paradigme{\textit{dir :} \jya tɤ-}
\begin{définition}\fra s'abîmer\end{définition}
\begin{définition}\cmn 损坏\end{définition}
\begin{exemple}\jya a-ɕɣa kɤ-ɴɢrɯ nɯ mɯ-pɯ-ɴɢrɯ ri, ɯ-qa to-ntɕhɯɣ tɕe ɲɯ-mŋɤm\cmn 我的牙齿没有裂,但是牙龈损坏了,很痛\end{exemple}
\begin{exemple}\jya mkhɯrlu to-ntɕhɯɣ tɕe kɤ-lɤt mɯ́jsna\cmn 车损坏了,不能再开了\end{exemple}
\begin{exemple}\jya nɤ-wa nɯ mɯ́j-tɯ-ntɕhɯɣ\cmn 你不亏是你父亲的儿子\end{exemple}\end{entrée}

\begin{entrée}
\vedette{\hypertarget{Ⓔnthar}{\papi{ nthar}}}\markboth{nthar}{}\classe{vt}
\paradigme{\textit{dir :} \jya nɯ-}
\begin{définition}\fra rouler la pâte\end{définition}
\begin{définition}\cmn 擀面\end{définition}
\begin{exemple}\jya ɯʑo kɯ pɤjpe na-nthar\cmn 他擀了面\end{exemple}\end{entrée}

\begin{entrée}
\vedette{\hypertarget{Ⓔnthɤβ}{\papi{ nthɤβ}}}\markboth{nthɤβ}{}\classe{vt}
\paradigme{\textit{dir :} \jya nɯ-}
\paradigme{\textit{dir :} \jya kɤ-}
\begin{définition}\fra serrer, coincer\end{définition}
\begin{définition}\cmn 夹住;夹到\end{définition}
\begin{exemple}\jya ki tɤ-ri ɲɤ-nthɤβ\cmn 这根线夹到了\end{exemple}
\begin{exemple}\jya a-jaʁ na-nthɤβ\cmn 他夹到了我的手\end{exemple}
\begin{exemple}\jya kɯm kɤ-pa-t-a, a-jaʁ na-nthɤβ\cmn 我关了门,夹到了我的手\end{exemple}\begin{sous-entrée}
\vedette{\hypertarget{}{\papi{ sɯnthɤβ}}}\markboth{sɯnthɤβ}{}\classe{vt}
\paradigme{\textit{dir :} \jya nɯ-}
\paradigme{\textit{dir :} \jya kɤ-}
\begin{définition}\ 
\begin{déclaration}\grammar{caus}\end{déclaration}\end{définition}
\begin{définition}\fra faire se coincer\end{définition}
\begin{définition}\cmn 使夹住\end{définition}
\begin{exemple}\jya a-jaʁ kɤ-nɯ-sɯnthaβ-a\cmn 我自己夹到了自己的手\end{exemple}
\end{sous-entrée}\end{entrée}

\begin{entrée}
\vedette{\hypertarget{Ⓔnthor}{\papi{ nthor}}}\markboth{nthor}{}
\classe{vi}
\paradigme{\textit{dir :} \jya \_}
\begin{définition}\fra rôder\end{définition}
\begin{définition}\cmn 流浪
\begin{déclaration} \étymologie{\papi{ⁿtʰor}}\end{déclaration}\end{définition}
\begin{exemple}\jya mbro ɯ-zda maŋe tɕe ɲɤ-nthor\cmn 马离了群就流浪\end{exemple}\end{entrée}

\begin{entrée}
\vedette{\hypertarget{Ⓔntoʁntoʁ}{\papi{ ntoʁntoʁ}}}\markboth{ntoʁntoʁ}{}
\classe{idph.2}
\begin{définition}\fra petit, rond et dur\end{définition}
\begin{définition}\cmn 形容小、圆而硬的样子\end{définition}
\begin{exemple}\jya mɯntoʁ ɯ-tɯ-mpɕɤr kɯ ntoʁntoʁ ʑo ɲɯ-pa\cmn 花又小又圆又漂亮\end{exemple}\end{entrée}

\begin{entrée}
\vedette{\hypertarget{Ⓔntsɣe}{\papi{ ntsɣe}}}\markboth{ntsɣe}{}\classe{vt}
\paradigme{\textit{dir :} \jya nɯ-}
\begin{définition}\fra vendre\end{définition}
\begin{définition}\cmn 卖\end{définition}
\begin{exemple}\jya @luyinji kɤ-ntsɣe ɯ-spa ɯ-ɲɯ́-ŋu\cmn 有没有录音机卖\end{exemple}
\begin{exemple}\jya ʑɴɢɯloʁ nɯ-ntsɣe-t-a\cmn 我卖了核桃\end{exemple}\begin{sous-entrée}
\vedette{\hypertarget{}{\papi{ ʑɣɤntsɣe}}}\markboth{ʑɣɤntsɣe}{}\classe{vi}
\begin{définition}\ 
\begin{déclaration}\grammar{refl}\end{déclaration}\end{définition}
\begin{définition}\fra se trahir\end{définition}
\begin{définition}\cmn 出卖自己\end{définition}
\begin{exemple}\jya nɤʑo ɲɯ-tɯ-nɯ-ʑɣɤntsɣe ʑo ɲɯ-ɕti\cmn 你自己出卖自己\end{exemple}
\begin{relation-sémantique}\confer{
\hyperlink{Ⓔtɯtsɣe}{\textit{ \papi{tɯtsɣe}}}
}\end{relation-sémantique}
\begin{relation-sémantique}\confer{
\hyperlink{Ⓔrɤtsɣe}{\textit{ \papi{rɤtsɣe}}}
}\end{relation-sémantique}
\begin{relation-sémantique}\confer{
\hyperlink{Ⓔraχtɯtsɣe}{\textit{ \papi{raχtɯtsɣe}}}
}\end{relation-sémantique}
\end{sous-entrée}\end{entrée}

\begin{entrée}
\vedette{\hypertarget{Ⓔntshɤβ}{\papi{ ntshɤβ}}}\markboth{ntshɤβ}{}
\paradigme{\textit{dir :} \jya tɤ-}
\begin{définition}\fra être affolé\end{définition}
\begin{définition}\cmn 慌张
\begin{déclaration} \étymologie{\papi{ⁿtsʰab}}\end{déclaration}\end{définition}
\begin{exemple}\jya ɯʑo ɲɤ-mu tɕe to-ntshɤβ\cmn 他受到惊吓就慌张起来了\end{exemple}\classe{vs}\begin{sous-entrée}
\vedette{\hypertarget{}{\papi{ ntshɤβ,rlu}}}\markboth{ntshɤβ,rlu}{}
\paradigme{\textit{dir :} \jya tɤ-}
\begin{définition}\fra être affolé\end{définition}
\begin{définition}\cmn 慌张\end{définition}
\begin{exemple}\jya ma-tɯ-ntshɤβ-rlu-ndʑi\cmn 你们俩不要慌张\end{exemple}
\begin{exemple}\jya tɤ-ntshɤβ-tɤ-rlu-a ʑo tɤ-rɤŋgat-a ɕti tɕe a-sɤcɯ kɤ-ndo ɲɤ-nɯjmɯt-a\cmn 我出发的时候慌张到忘了带钥匙\end{exemple}
\begin{relation-sémantique}\ComponentA{\classe{vs}
\hyperlink{Ⓔntshɤβ}{\textit{ \papi{ntshɤβ}}}
}\end{relation-sémantique}
\begin{relation-sémantique}\ComponentB{\classe{vs}
 \papi{rlu}
}\end{relation-sémantique}
\end{sous-entrée}\begin{sous-entrée}
\vedette{\hypertarget{}{\papi{ sɯntshɤβ}}}\markboth{sɯntshɤβ}{}\classe{vt}
\begin{définition}\fra affolé, rendre\end{définition}
\begin{définition}\cmn 令……紧张、慌张\end{définition}
\begin{exemple}\jya ma-kɯ-sɯntshaβ-a ma tha a-laχtɕha kɯ-sɯjmɯt-a\cmn 你不要令我紧张,你会令我忘记带东西\end{exemple}
\begin{relation-sémantique}\synonyme{
\hyperlink{Ⓔɕɯmbɣom}{\textit{ \papi{ɕɯmbɣom}}}
}\end{relation-sémantique}
\end{sous-entrée}\begin{sous-entrée}
\vedette{\hypertarget{}{\papi{ ʑɣɤsɯntshɤβ}}}\markboth{ʑɣɤsɯntshɤβ}{}\classe{vi}
\paradigme{\textit{dir :} \jya tɤ-}
\begin{définition}\fra paniquer, se mettre dans tous ses états\end{définition}
\begin{définition}\cmn 慌张\end{définition}
\end{sous-entrée}\end{entrée}

\begin{entrée}
\vedette{\hypertarget{Ⓔntshɤβ,rlu}{\papi{ ntshɤβ,rlu}}}\markboth{ntshɤβ,rlu}{}
\begin{relation-sémantique}\confer{
\hyperlink{Ⓔntshɤβ}{\textit{ \papi{ntshɤβ}}}
}\end{relation-sémantique}\end{entrée}

\begin{entrée}
\vedette{\hypertarget{Ⓔntshɤr}{\papi{ ntshɤr}}}\markboth{ntshɤr}{}
\classe{vi}
\paradigme{\textit{dir :} \jya nɯ-}
\begin{définition}\fra hennir\end{définition}
\begin{définition}\cmn 叫(马叫)\end{définition}
\begin{exemple}\jya mbro ɲɯ-ntshɤr\cmn 马在嘶叫\end{exemple}
\begin{exemple}\jya mbro nɯ-ntshɤr\cmn 马嘶叫了\end{exemple}\end{entrée}

\begin{entrée}
\vedette{\hypertarget{ⒺntshiⒽ1}{\papi{ ntshi}}}\markboth{ntshi}{}\homonyme{1}\classe{vt}
\paradigme{\textit{dir :} \jya nɯ-}
\begin{définition}\fra sélectionner\end{définition}
\begin{définition}\cmn 挑选;拣\end{définition}
\begin{exemple}\jya rasti nɯ-ntshi\cmn 你选一下圆根\end{exemple}
\begin{exemple}\jya aʑo nɯ-ntshi-t-a\cmn 我选了\end{exemple}\end{entrée}

\begin{entrée}
\vedette{\hypertarget{ⒺntshiⒽ2}{\papi{ ntshi}}}\markboth{ntshi}{}\homonyme{2}
\classe{vs}
\begin{définition}\fra mieux valoir que, devoir\end{définition}
\begin{définition}\cmn 只好\end{définition}\end{entrée}

\begin{entrée}
\vedette{\hypertarget{Ⓔntshoʁ}{\papi{ ntshoʁ}}}\markboth{ntshoʁ}{}
\classe{vi}
\paradigme{\textit{dir :} \jya pɯ-}
\begin{définition}\fra réciter des soutras en groupe\end{définition}
\begin{définition}\cmn 念经
\begin{déclaration} \étymologie{\papi{ⁿtsʰog}}\end{déclaration}\end{définition}
\begin{exemple}\jya χpɯn ra ɲɯ-ntshoʁ-nɯ\cmn 和尚们在念经\end{exemple}\end{entrée}

\begin{entrée}
\vedette{\hypertarget{Ⓔntsɯ}{\papi{ ntsɯ}}}\markboth{ntsɯ}{}\classe{adv}\acception{1}
\begin{définition}\fra tout le temps\end{définition}
\begin{définition}\cmn 总是\end{définition}\acception{2}
\begin{définition}\fra à chaque ...\end{définition}
\begin{définition}\cmn 每一...\end{définition}
\end{entrée}

\begin{entrée}
\vedette{\hypertarget{Ⓔntʂu}{\papi{ ntʂu}}}\markboth{ntʂu}{}
\classe{vl}
\paradigme{\textit{dir :} \jya lɤ-}
\begin{définition}\fra sarcler\end{définition}
\begin{définition}\cmn 薅锄;锄草\end{définition}
\begin{exemple}\jya la-ntʂu\cmn 他锄了草\end{exemple}
\begin{exemple}\jya lɤ-ntʂu-t-a\cmn 我锄了草\end{exemple}
\begin{exemple}\jya lɤ-ntʂu-j\cmn 我们锄了草\end{exemple}
\begin{exemple}\jya tɤɕi ɯ-ŋgɯ lɤ-ntʂu-a\cmn 我锄了青稞\end{exemple}
\begin{exemple}\jya tɤɕi lɤ-ntʂu-t-a\cmn 我锄了青稞\end{exemple}
\begin{relation-sémantique}\confer{
\hyperlink{Ⓔtʂu}{\textit{ \papi{tʂu}}}
}\end{relation-sémantique}\end{entrée}

\begin{entrée}
\vedette{\hypertarget{Ⓔnɯ}{\papi{ nɯ}}}\markboth{nɯ}{}\classe{dem}
\begin{définition}\fra celà\end{définition}
\begin{définition}\cmn 那个\end{définition}
\end{entrée}

\begin{entrée}
\vedette{\hypertarget{Ⓔnɯbabɯ}{\papi{ nɯbabɯ}}}\markboth{nɯbabɯ}{}\classe{vi}
\paradigme{\textit{dir :} \jya \_}
\begin{définition}\fra ramasser du cassis\end{définition}
\begin{définition}\cmn 捡黑茶藨子\end{définition}
\begin{relation-sémantique}\confer{
\hyperlink{Ⓔbabɯ}{\textit{ \papi{babɯ}}}
}\end{relation-sémantique}\end{entrée}

\begin{entrée}
\vedette{\hypertarget{Ⓔnɯbɤβ}{\papi{ nɯbɤβ}}}\markboth{nɯbɤβ}{}
\begin{relation-sémantique}\confer{
\hyperlink{Ⓔbɤbɤβ}{\textit{ \papi{bɤbɤβ}}}
}\end{relation-sémantique}\end{entrée}

\begin{entrée}
\vedette{\hypertarget{Ⓔnɯβdaʁ}{\papi{ nɯβdaʁ}}}\markboth{nɯβdaʁ}{}
\classe{vt}
\paradigme{\textit{dir :} \jya tɤ-}
\begin{définition}\ 
\begin{déclaration}\grammar{denom}\end{déclaration}\end{définition}\acception{1}
\begin{définition}\fra surveiller\end{définition}
\begin{définition}\cmn 看管(孩子、东西等)\end{définition}
\begin{exemple}\jya laχtɕha tɤ-nɯβdaʁ\cmn 你把东西看好\end{exemple}\acception{2}
\begin{définition}\fra contrôler\end{définition}
\begin{définition}\cmn 控制;管理\end{définition}\acception{3}
\paradigme{\textit{dir :} \jya nɯ-}
\begin{définition}\fra prendre la responsabilité\end{définition}
\begin{définition}\cmn 承担;被冤枉
\begin{déclaration} \étymologie{\papi{bdag}}\end{déclaration}\end{définition}
\begin{exemple}\jya ɯʑo kɯ ta-nɤma pɯ-ɕti ri, aʑo kɯ nɯ-nɯβdaʁ-a pɯ-ra\cmn 本来是他做的事情,最后让我承担了\end{exemple}\begin{sous-entrée}
\vedette{\hypertarget{}{\papi{ znɯβdaʁ}}}\markboth{znɯβdaʁ}{}\classe{vt}
\paradigme{\textit{dir :} \jya nɯ-}
\begin{définition}\fra faire porter la responsabilité à\end{définition}
\begin{définition}\cmn 让……承担\end{définition}
\begin{exemple}\jya mɤ-kɯ-tʂaŋ ɲɯ́-wɣ-znɯβdaʁ-a-nɯ ɲɯ-ŋu\cmn 他们冤枉我(要我承担不公平的事情)\end{exemple}
\begin{exemple}\jya ki laχtɕha kɯra nɤʑo tɤ-kɤ-znɯβdaʁ tɕe ɯ-pɯ tɤ-kɤ-pa ɲɯ-ŋu\cmn 这些东西是为你的名义贮存的\end{exemple}
\begin{relation-sémantique}\synonyme{
\hyperlink{Ⓔnɤpɯpa}{\textit{ \papi{nɤpɯpa}}}
}\end{relation-sémantique}
\end{sous-entrée}\end{entrée}

\begin{entrée}
\vedette{\hypertarget{Ⓔnɯβdaχpu}{\papi{ nɯβdaχpu}}}\markboth{nɯβdaχpu}{}\classe{vi}
\paradigme{\textit{dir :} \jya lɤ-}\acception{1}
\begin{définition}\fra s'accaparer\end{définition}
\begin{définition}\cmn 归为己有\end{définition}
\begin{exemple}\jya kɯki nɤʑɯɣ ɕti tɕe, aʑo lu-nɯβdaχpu-a mɤ-pe\cmn 这是你的东西,我不应该归为己有\end{exemple}\acception{2}
\begin{définition}\fra se prendre pour le maître de maison\end{définition}
\begin{définition}\cmn 喧宾夺主\end{définition}
\begin{exemple}\jya ki tɯrme ɯ-kha ɲɯ-ɕti tɕe, ma-lɤ-tɯ-nɯβdaχpu ma βdaχpu ri ɲɯ-tɯ-maʁ\cmn 这是别人的家,你不要喧宾夺主,你又不是主人\end{exemple}
\begin{relation-sémantique}\confer{
\hyperlink{ⒺβdaχpuⒽ1}{\textit{ \papi{βdaχpu}}}
}\end{relation-sémantique}\end{entrée}

\begin{entrée}
\vedette{\hypertarget{Ⓔnɯβde}{\papi{ nɯβde}}}\markboth{nɯβde}{}
\begin{relation-sémantique}\confer{
\hyperlink{Ⓔβde}{\textit{ \papi{βde}}}
}\end{relation-sémantique}\end{entrée}

\begin{entrée}
\vedette{\hypertarget{Ⓔnɯβɣɤmu}{\papi{ nɯβɣɤmu}}}\markboth{nɯβɣɤmu}{}
\classe{vi}
\paradigme{\textit{dir :} \jya pɯ-}
\begin{définition}\ 
\begin{déclaration}\grammar{denom}\end{déclaration}\end{définition}
\begin{définition}\fra surveiller la meule\end{définition}
\begin{définition}\cmn 看守水磨\end{définition}
\begin{exemple}\jya pɯ-nɯβɣɤmu-a\cmn 我守过水磨\end{exemple}
\begin{relation-sémantique}\confer{
\hyperlink{Ⓔβɣɤmu}{\textit{ \papi{βɣɤmu}}}
}\end{relation-sémantique}\end{entrée}

\begin{entrée}
\vedette{\hypertarget{Ⓔnɯβɣe}{\papi{ nɯβɣe}}}\markboth{nɯβɣe}{}\paradigme{\textit{dir :} \jya pɯ-}
\begin{définition}\fra perdre un membre de sa famille\end{définition}
\begin{définition}\cmn 失去亲人\end{définition}
\begin{exemple}\jya tɤ-rɟit pɯ-kɯ-nɯβɣe\cmn 孤儿\end{exemple}
\begin{exemple}\jya tɤ-pɤtso pjɤ-nɯβɣe\cmn 小孩子变成了孤儿\end{exemple}\classe{vi}\end{entrée}

\begin{entrée}
\vedette{\hypertarget{Ⓔnɯβɣɯz}{\papi{ nɯβɣɯz}}}\markboth{nɯβɣɯz}{}\classe{vi}
\paradigme{\textit{dir :} \jya tɤ-}
\begin{définition}\fra chasser les blaireaux\end{définition}
\begin{définition}\cmn 抓獾\end{définition}
\begin{exemple}\jya ɕ-tu-nɯβɣɯz-nɯ tɕe, βɣɯz nɯ pjɯ-sat-nɯ ŋgrɤl\cmn 他们抓獾并杀獾\end{exemple}
\begin{relation-sémantique}\confer{
\hyperlink{Ⓔβɣɯz}{\textit{ \papi{βɣɯz}}}
}\end{relation-sémantique}\end{entrée}

\begin{entrée}
\vedette{\hypertarget{Ⓔnɯβlu}{\papi{ nɯβlu}}}\markboth{nɯβlu}{}
\classe{vt}
\paradigme{\textit{dir :} \jya pɯ-}
\begin{définition}\ 
\begin{déclaration}\grammar{denom}\end{déclaration}\end{définition}
\begin{définition}\fra tromper\end{définition}
\begin{définition}\cmn 欺骗\end{définition}
\begin{exemple}\jya tɤ-pɤtso pɯ-nɯβlu-t-a\cmn 我骗了小孩子\end{exemple}
\begin{exemple}\jya jiɕqha nɯ kɯ pɯ́-wɣ-nɯβlu-a\cmn 这个人把我骗了\end{exemple}\begin{sous-entrée}
\vedette{\hypertarget{}{\papi{ sɤnɯβlu}}}\markboth{sɤnɯβlu}{}\classe{vi}
\begin{définition}\ 
\begin{déclaration}\grammar{apass}\end{déclaration}\end{définition}
\begin{définition}\fra tromper les gens\end{définition}
\begin{définition}\cmn 骗人\end{définition}
\end{sous-entrée}\begin{sous-entrée}
\vedette{\hypertarget{}{\papi{ ʑɣɤnɯβlu}}}\markboth{ʑɣɤnɯβlu}{}
\paradigme{\textit{dir :} \jya pɯ-}
\begin{définition}\fra se laisser tromper\end{définition}
\begin{définition}\cmn 被骗\end{définition}
\begin{relation-sémantique}\confer{
\hyperlink{Ⓔɯ-βlu}{\textit{ \papi{ɯ-βlu}}}
}\end{relation-sémantique}\classe{vi}
\end{sous-entrée}\end{entrée}

\begin{entrée}
\vedette{\hypertarget{Ⓔnɯβlɤmtɕhɤt}{\papi{ nɯβlɤmtɕhɤt}}}\markboth{nɯβlɤmtɕhɤt}{}\classe{vt}
\paradigme{\textit{dir :} \jya tɤ-}
\begin{définition}\fra réciter des soutras\end{définition}
\begin{définition}\cmn 念经\end{définition}
\begin{exemple}\jya nɤʑo tu-ta-nɯβlɤmtɕhɤt ɯ́-jɤɣ?\cmn 请你为我们念经行吗?\end{exemple}
\begin{relation-sémantique}\confer{
\hyperlink{Ⓔβlɤmtɕhɤt}{\textit{ \papi{βlɤmtɕhɤt}}}
}\end{relation-sémantique}\end{entrée}

\begin{entrée}
\vedette{\hypertarget{Ⓔnɯβlɯz}{\papi{ nɯβlɯz}}}\markboth{nɯβlɯz}{}\classe{vt}
\paradigme{\textit{dir :} \jya pɯ-}
\begin{définition}\fra réciter par cœur, faire sans modèle\end{définition}
\begin{définition}\cmn 背诵;不用模型地做\end{définition}
\begin{exemple}\jya kɤ-rɤt pɯ-nɯβlɯz-a ɕti\cmn 我没有样板也画出来了\end{exemple}
\begin{relation-sémantique}\confer{
\hyperlink{Ⓔtɯ-βlɯz}{\textit{ \papi{tɯ-βlɯz}}}
}\end{relation-sémantique}\end{entrée}

\begin{entrée}
\vedette{\hypertarget{Ⓔnɯβra}{\papi{ nɯβra}}}\markboth{nɯβra}{}\classe{vt}
\paradigme{\textit{dir :} \jya tɤ-}\acception{1}
\begin{définition}\fra fournir\end{définition}
\begin{définition}\cmn 提供\end{définition}
\begin{exemple}\jya nɤ-ŋga tɤ-nɯβra-t-a\cmn 我给你提供了衣服\end{exemple}\acception{2}
\begin{définition}\fra qui a pour usage de ...\end{définition}
\begin{définition}\cmn 有……的功能\end{définition}\end{entrée}

\begin{entrée}
\vedette{\hypertarget{Ⓔnɯβraʁ}{\papi{ nɯβraʁ}}}\markboth{nɯβraʁ}{}
\begin{relation-sémantique}\confer{
\hyperlink{Ⓔβraʁ}{\textit{ \papi{βraʁ}}}
}\end{relation-sémantique}
\end{entrée}

\begin{entrée}
\vedette{\hypertarget{Ⓔnɯβzaŋsa}{\papi{ nɯβzaŋsa}}}\markboth{nɯβzaŋsa}{}
\classe{vt}
\paradigme{\textit{dir :} \jya tɤ-}
\begin{définition}\ 
\begin{déclaration}\grammar{denom}\end{déclaration}\end{définition}
\begin{définition}\fra devenir ami\end{définition}
\begin{définition}\cmn 交朋友\end{définition}
\begin{exemple}\jya tɤ-nɯβzaŋsa-t-a\cmn 我跟他交了朋友\end{exemple}
\begin{exemple}\jya tɤ́-wɣ-nɯβzaŋsa-a\cmn 他跟我交了朋友\end{exemple}
\begin{exemple}\jya tɯrme mɤ-kɯ-frtɤn nɯ a-mɤ-tɤ-tɯ-nɯβzaŋse\cmn 你不要跟不可靠的人交朋友\end{exemple}
\begin{relation-sémantique}\synonyme{
\hyperlink{Ⓔnɯɣɯfsu}{\textit{ \papi{nɯɣɯfsu}}}
}\end{relation-sémantique}
\begin{relation-sémantique}\confer{
\hyperlink{Ⓔβzaŋsa}{\textit{ \papi{βzaŋsa}}}
}\end{relation-sémantique}\end{entrée}

\begin{entrée}
\vedette{\hypertarget{Ⓔnɯβʑit}{\papi{ nɯβʑit}}}\markboth{nɯβʑit}{}
\classe{vt}\acception{1}
\paradigme{\textit{dir :} \jya nɯ-}
\begin{définition}\fra faire diminuer\end{définition}
\begin{définition}\cmn 减一部分;减一段\end{définition}
\begin{exemple}\jya laχtɕha ɲɤ-nɯβʑit\cmn 他减了一些东西\end{exemple}
\begin{exemple}\jya ji-kɤndza ɲɤ-nɯβʑit\cmn 他减了我们的食物\end{exemple}
\begin{exemple}\jya pɕawtsɯ ɲɤ-nɯβʑit\cmn 我减了钱(贪污了)\end{exemple}
\begin{exemple}\jya nɤ-fkur pjɤ-nɯβʑit\cmn 他减少了你的负担\end{exemple}\acception{2}
\paradigme{\textit{dir :} \jya pɯ-}
\begin{définition}\fra raccourcir\end{définition}
\begin{définition}\cmn 弄短一点\end{définition}
\begin{exemple}\jya ɕoŋtɕa ɲɯ-zri tɕe pjɤ-nɯβʑit\cmn 木料太长,他弄短了一些\end{exemple}\end{entrée}

\begin{entrée}
\vedette{\hypertarget{Ⓔnɯcaχto}{\papi{ nɯcaχto}}}\markboth{nɯcaχto}{}\classe{vi}
\paradigme{\textit{dir :} \jya thɯ-}
\begin{définition}\fra être bouche bée\end{définition}
\begin{définition}\cmn 目瞪口呆\end{définition}
\begin{exemple}\jya thɯ-nɯcaχto-a\cmn 我目瞪口呆了\end{exemple}
\begin{relation-sémantique}\synonyme{
\hyperlink{Ⓔnɤχɤmthi}{\textit{ \papi{nɤχɤmthi}}}
}\end{relation-sémantique}
\begin{sous-entrée}
\vedette{\hypertarget{}{\papi{ znɯcaχto}}}\markboth{znɯcaχto}{}\classe{vt}
\paradigme{\textit{dir :} \jya thɯ-}
\begin{définition}\fra rendre bouche bée\end{définition}
\begin{définition}\cmn 令人目瞪口呆\end{définition}
\begin{exemple}\jya chɤ́-wɣ-znɯcaχto ʑo\cmn 令他目瞪口呆\end{exemple}
\begin{relation-sémantique}\synonyme{
\hyperlink{Ⓔznɤχɤmthi}{\textit{ \papi{znɤχɤmthi}}}
}\end{relation-sémantique}
\end{sous-entrée}\end{entrée}

\begin{entrée}
\vedette{\hypertarget{Ⓔnɯcɤɕna}{\papi{ nɯcɤɕna}}}\markboth{nɯcɤɕna}{}
\classe{vi}
\paradigme{\textit{dir :} \jya pɯ-}
\paradigme{\textit{dir :} \jya \_}
\begin{définition}\fra ramasser du rumex japonicus\end{définition}
\begin{définition}\cmn 采集山菠菜\end{définition}
\begin{exemple}\jya ɕ-pɯ-nɯcɤɕna\cmn 你去采集山菠菜吧\end{exemple}
\begin{relation-sémantique}\confer{
\hyperlink{Ⓔcɤɕna}{\textit{ \papi{cɤɕna}}}
}\end{relation-sémantique}\end{entrée}

\begin{entrée}
\vedette{\hypertarget{Ⓔnɯcha}{\papi{ nɯcha}}}\markboth{nɯcha}{}\classe{vs}
\paradigme{\textit{dir :} \jya lɤ-}
\begin{définition}\ 
\begin{déclaration}\grammar{denom}\end{déclaration}\end{définition}
\begin{définition}\fra être saoul\end{définition}
\begin{définition}\cmn 喝醉\end{définition}
\begin{relation-sémantique}\synonyme{
\hyperlink{Ⓔβzi}{\textit{ \papi{βzi}}}
}\end{relation-sémantique}
\begin{relation-sémantique}\confer{
\hyperlink{ⒺchaⒽ2}{\textit{ \papi{cha2}}}
}\end{relation-sémantique}\end{entrée}

\begin{entrée}
\vedette{\hypertarget{Ⓔnɯchɤmda}{\papi{ nɯchɤmda}}}\markboth{nɯchɤmda}{}\classe{vt}
\paradigme{\textit{dir :} \ }
\begin{définition}\ 
\begin{déclaration}\grammar{denom}\end{déclaration}\end{définition}
\begin{définition}\fra boire de l'alcool à la paille\end{définition}
\begin{définition}\cmn 喝干干酒\end{définition}
\begin{exemple}\jya tɯ́-wɣ-nɯchɤmda ɕti\cmn (妖精)会在你背后插吸管喝你的血\end{exemple}
\begin{relation-sémantique}\confer{
\hyperlink{Ⓔchɤmda}{\textit{ \papi{chɤmda}}}
}\end{relation-sémantique}\end{entrée}

\begin{entrée}
\vedette{\hypertarget{Ⓔnɯchɤrga}{\papi{ nɯchɤrga}}}\markboth{nɯchɤrga}{}
\classe{vs}
\begin{définition}\ 
\begin{déclaration}\grammar{incorp}\end{déclaration}\end{définition}
\begin{définition}\fra aimer boire de l'alcool\end{définition}
\begin{définition}\cmn 喜欢喝酒\end{définition}
\begin{exemple}\jya ki kɯ-nɯchɤrga ci ŋu\cmn 他是酒鬼\end{exemple}
\begin{relation-sémantique}\confer{
\hyperlink{ⒺchaⒽ2}{\textit{ \papi{cha2}}}
}\end{relation-sémantique}
\begin{relation-sémantique}\confer{
 \papi{rga2}
}\end{relation-sémantique}\end{entrée}

\begin{entrée}
\vedette{\hypertarget{Ⓔnɯchɯβ}{\papi{ nɯchɯβ}}}\markboth{nɯchɯβ}{}\classe{vt}
\paradigme{\textit{dir :} \jya tɤ-}
\begin{définition}\fra s'empiffrer\end{définition}
\begin{définition}\cmn 大口大口地吃\end{définition}
\begin{exemple}\jya tɤ-nɯchɯβ-a ʑo tɤ-ndza-t-a\cmn 我大口大口地吃了\end{exemple}
\begin{relation-sémantique}\synonyme{
\hyperlink{Ⓔnɯkhɯɣ}{\textit{ \papi{nɯkhɯɣ}}}
}\end{relation-sémantique}
\begin{relation-sémantique}\synonyme{
\hyperlink{Ⓔnɯlŋɤβ}{\textit{ \papi{nɯlŋɤβ}}}
}\end{relation-sémantique}\end{entrée}

\begin{entrée}
\vedette{\hypertarget{Ⓔnɯchɯra}{\papi{ nɯchɯra}}}\markboth{nɯchɯra}{}\classe{vi}
\paradigme{\textit{dir :} \jya pɯ-}
\begin{définition}\fra monter la garde\end{définition}
\begin{définition}\cmn 站岗;守卫\end{définition}
\begin{exemple}\jya ɯʑo ku-nɯchɯra\cmn 他在站岗\end{exemple}
\begin{relation-sémantique}\synonyme{
\hyperlink{Ⓔnɯsuwa}{\textit{ \papi{nɯsuwa}}}
}\end{relation-sémantique}\end{entrée}

\begin{entrée}
\vedette{\hypertarget{Ⓔnɯci}{\papi{ nɯci}}}\markboth{nɯci}{}
\classe{vi}
\paradigme{\textit{dir :} \jya pɯ-}
\begin{définition}\ 
\begin{déclaration}\grammar{denom}\end{déclaration}\end{définition}
\begin{définition}\fra boire sans se servir de ses mains\end{définition}
\begin{définition}\cmn 直接用嘴巴喝地下的水(不用手)
\begin{déclaration}\use{一般指动物}\end{déclaration}\end{définition}
\begin{exemple}\jya tɯ-ci ɯ-ŋgɯ pɯ-nɯci\cmn 它喝了河流的水\end{exemple}
\begin{relation-sémantique}\confer{
\hyperlink{Ⓔtɯ-ci}{\textit{ \papi{tɯ-ci}}}
}\end{relation-sémantique}\end{entrée}

\begin{entrée}
\vedette{\hypertarget{Ⓔnɯco}{\papi{ nɯco}}}\markboth{nɯco}{}\classe{vt}
\paradigme{\textit{dir :} \jya \_}
\begin{définition}\fra suivre\end{définition}
\begin{définition}\cmn 跟踪;顺着走\end{définition}
\begin{exemple}\jya zgo nɯnɯ lɤ-nɯco-t-a\cmn 我沿着这个山顶走了\end{exemple}
\begin{exemple}\jya kɯki khri ki nɯ-nɯco-t-a ma nɯ ma mɯ́j-cha-a\cmn 我只好沿着床边走,其它还不行(病人说的话)\end{exemple}
\begin{exemple}\jya tʂu maŋe tɕe, tɯ-ci lɤ-nɯco-t-a tɕe lɤ-ari-a\cmn 因为没有路,我顺着水流去了\end{exemple}
\begin{exemple}\jya a-tɕɯ kɯ znde ku-nɯcɤm ŋu (ɲɯ-ɤz-nɯco)\cmn 我儿子扶着墙慢慢走\end{exemple}\begin{sous-entrée}
\vedette{\hypertarget{}{\papi{ znɯɲco}}}\markboth{znɯɲco}{}\classe{vt}
\paradigme{\textit{dir :} \jya nɯ-}
\paradigme{\textit{dir :} \jya pɯ-}
\begin{définition}\fra suivre les instructions de\end{définition}
\begin{définition}\cmn 依照……的说法
\end{définition}
\begin{exemple}\jya ɯʑo kɯ kɤ-ɕe ɲɯ-sɯsɤm qhe, tɕe nɯ-znɯɲco-t-a\cmn 他想去那里,我就依他了\end{exemple}
\begin{relation-sémantique}\confer{
\hyperlink{Ⓔnɯɴqhu}{\textit{ \papi{nɯɴqhu}}}
}\end{relation-sémantique}
\end{sous-entrée}\end{entrée}

\begin{entrée}
\vedette{\hypertarget{Ⓔnɯcɯnthaʁ}{\papi{ nɯcɯnthaʁ}}}\markboth{nɯcɯnthaʁ}{}
\classe{vt}\acception{1}
\paradigme{\textit{dir :} \jya pɯ-}
\begin{définition}\fra hacher de la viande\end{définition}
\begin{définition}\cmn 剁肉\end{définition}
\begin{exemple}\jya jiɕqha tɤ-mthɯm pɯ-nɯcɯnthaʁ\cmn 你剁一下肉吧\end{exemple}\acception{2}
\paradigme{\textit{dir :} \jya thɯ-}
\begin{définition}\fra hacher de l'ail\end{définition}
\begin{définition}\cmn 剁大蒜\end{définition}\end{entrée}

\begin{entrée}
\vedette{\hypertarget{Ⓔnɯɕu}{\papi{ nɯɕu}}}\markboth{nɯɕu}{}\classe{vi}
\paradigme{\textit{dir :} \jya pɯ-}
\begin{définition}\ 
\begin{déclaration}\grammar{denom}\end{déclaration}\end{définition}
\begin{définition}\fra jouer aux cartes\end{définition}
\begin{définition}\cmn 打牌\end{définition}
\begin{exemple}\jya jiʑo pɯ-nɯɕu-j\cmn 我们打牌了\end{exemple}
\begin{relation-sémantique}\confer{
\hyperlink{Ⓔɕu}{\textit{ \papi{ɕu}}}
}\end{relation-sémantique}\end{entrée}

\begin{entrée}
\vedette{\hypertarget{Ⓔnɯɕɤɣ}{\papi{ nɯɕɤɣ}}}\markboth{nɯɕɤɣ}{}\classe{vi}
\paradigme{\textit{dir :} \jya pɯ-}
\begin{définition}\fra couper des genévriers (pour faire des fumigations)\end{définition}
\begin{définition}\cmn 砍柏树枝桠\end{définition}
\begin{exemple}\jya ɕ-pɯ-nɯɕaɣ-a\cmn 我砍了柏树枝桠\end{exemple}\end{entrée}

\begin{entrée}
\vedette{\hypertarget{Ⓔnɯɕɤlɤmbɯmbjom}{\papi{ nɯɕɤlɤmbɯmbjom}}}\markboth{nɯɕɤlɤmbɯmbjom}{}
\classe{vi}
\paradigme{\textit{dir :} \jya \_}
\begin{définition}\fra faire une course\end{définition}
\begin{définition}\cmn 赛跑\end{définition}
\begin{exemple}\jya ɲɯ-nɯɕɤlɤmbɯmbjom-tɕi a-pɯ-ŋu tɕe, nɤʑo mɤ-tɯ-cha\cmn 如果我们赛跑的话,你肯定不行\end{exemple}
\begin{relation-sémantique}\synonyme{
\hyperlink{Ⓔnɯsaχɕɯβ}{\textit{ \papi{nɯsaχɕɯβ}}}
}\end{relation-sémantique}\end{entrée}

\begin{entrée}
\vedette{\hypertarget{Ⓔnɯɕɤmɯɣdɯ}{\papi{ nɯɕɤmɯɣdɯ}}}\markboth{nɯɕɤmɯɣdɯ}{}
\classe{vt}
\paradigme{\textit{dir :} \jya tɤ-}
\begin{définition}\ 
\begin{déclaration}\grammar{denom}\end{déclaration}\end{définition}
\begin{définition}\fra tirer au fusil\end{définition}
\begin{définition}\cmn 射枪\end{définition}
\begin{exemple}\jya tɤfsɯr ra tɤ-nɯɕɤmɯɣdɯ-t-a\cmn 我用枪射了靶子\end{exemple}
\begin{relation-sémantique}\confer{
\hyperlink{Ⓔɕɤmɯɣdɯ}{\textit{ \papi{ɕɤmɯɣdɯ}}}
}\end{relation-sémantique}\end{entrée}

\begin{entrée}
\vedette{\hypertarget{Ⓔnɯɕɤrɤz}{\papi{ nɯɕɤrɤz}}}\markboth{nɯɕɤrɤz}{}
\classe{vi}
\begin{définition}\fra tenir de, ressembler (à ses parents)\end{définition}
\begin{définition}\cmn 遗传;有点像(父母)
\begin{déclaration}\use{古语}\end{déclaration}\end{définition}
\begin{exemple}\jya ɯ-wa ɲɯ-nɯɕɤrɤz\cmn 他有点像父亲\end{exemple}
\begin{exemple}\jya ɯ-phoŋbu ɯ-wa ɲɯ-nɯɕɤrɤz\cmn 他的身体有点像他父亲的\end{exemple}\end{entrée}

\begin{entrée}
\vedette{\hypertarget{Ⓔnɯɕe}{\papi{ nɯɕe}}}\markboth{nɯɕe}{}
\classe{vi}
\paradigme{\textit{dir :} \jya \_}
\paradigme{\textit{past stem :} \jya anɯri}
\begin{définition}\ 
\begin{déclaration}\grammar{vert}\end{déclaration}\end{définition}
\begin{définition}\fra rentrer chez soi\end{définition}
\begin{définition}\cmn 回家\end{définition}
\begin{exemple}\jya nɤʑo tɯ-nɯɕe ɕi, nɤ-mu ɯ-ɕki ?\cmn 你回不回家,你母亲那边?\end{exemple}
\begin{relation-sémantique}\confer{
\hyperlink{Ⓔɕe}{\textit{ \papi{ɕe}}}
}\end{relation-sémantique}\end{entrée}

\begin{entrée}
\vedette{\hypertarget{Ⓔnɯɕɣɤthɯt}{\papi{ nɯɕɣɤthɯt}}}\markboth{nɯɕɣɤthɯt}{}\classe{vt}
\paradigme{\textit{dir :} \jya thɯ-}
\begin{définition}\fra réparer (une lame, un soc de charrue)\end{définition}
\begin{définition}\cmn 补(铧头、斧头等)\end{définition}
\begin{relation-sémantique}\confer{
\hyperlink{Ⓔtɯ-ɕɣa}{\textit{ \papi{tɯ-ɕɣa}}}
}\end{relation-sémantique}
\begin{relation-sémantique}\confer{
\hyperlink{Ⓔmthɯt}{\textit{ \papi{mthɯt}}}
}\end{relation-sémantique}\end{entrée}

\begin{entrée}
\vedette{\hypertarget{Ⓔnɯɕkat}{\papi{ nɯɕkat}}}\markboth{nɯɕkat}{}
\classe{vi}
\begin{définition}\ 
\begin{déclaration}\grammar{denom}\end{déclaration}\end{définition}
\begin{définition}\fra transporter à dos d'animal\end{définition}
\begin{définition}\cmn 驮东西\end{définition}
\begin{exemple}\jya aʑo kɯ-nɯɕkat ŋu-a\cmn 我是驮东西的人\end{exemple}
\begin{relation-sémantique}\confer{
\hyperlink{Ⓔɣɯɕkat}{\textit{ \papi{ɣɯɕkat}}}
}\end{relation-sémantique}\end{entrée}

\begin{entrée}
\vedette{\hypertarget{Ⓔnɯɕkrɤɣ}{\papi{ nɯɕkrɤɣ}}}\markboth{nɯɕkrɤɣ}{}
\classe{vt}
\paradigme{\textit{dir :} \jya \_}
\begin{définition}\ 
\begin{déclaration}\grammar{deidph}\end{déclaration}\end{définition}
\begin{définition}\fra renverser avec force son adversaire\end{définition}
\begin{définition}\cmn 力气很大,很轻松地把对方摔下去\end{définition}
\begin{exemple}\jya tɤ-aʑɯʑu-ndʑi tɕe, ɯ-zda pa-nɯɕkrɤɣ ʑo pa-tʂaβ\cmn 在角力的时候,他很轻松地把对方摔下去了\end{exemple}
\begin{relation-sémantique}\confer{
\hyperlink{Ⓔɕkrɤɣɕkrɤɣ}{\textit{ \papi{ɕkrɤɣɕkrɤɣ}}}
}\end{relation-sémantique}
\begin{relation-sémantique}\confer{
\hyperlink{Ⓔnɯʑgrɤɣ}{\textit{ \papi{nɯʑgrɤɣ}}}
}\end{relation-sémantique}\end{entrée}

\begin{entrée}
\vedette{\hypertarget{Ⓔnɯɕmɯrga}{\papi{ nɯɕmɯrga}}}\markboth{nɯɕmɯrga}{}
\classe{vs}
\begin{définition}\ 
\begin{déclaration}\grammar{incorp}\end{déclaration}\end{définition}
\begin{définition}\fra bavard\end{définition}
\begin{définition}\cmn 爱说话\end{définition}
\begin{exemple}\jya ɯʑo wuma kɯ-nɯɕmɯrga ci ɲɯ-ŋu\end{exemple}
\begin{relation-sémantique}\confer{
\hyperlink{Ⓔrɯɕmi}{\textit{ \papi{rɯɕmi}}}
}\end{relation-sémantique}
\begin{relation-sémantique}\confer{
\hyperlink{Ⓔrɯɕmɯχtɤm}{\textit{ \papi{rɯɕmɯχtɤm}}}
}\end{relation-sémantique}
\begin{relation-sémantique}\confer{
 \papi{rga}
}\end{relation-sémantique}\end{entrée}

\begin{entrée}
\vedette{\hypertarget{Ⓔnɯɕpɯz}{\papi{ nɯɕpɯz}}}\markboth{nɯɕpɯz}{}
\classe{vt}
\paradigme{\textit{dir :} \jya tɤ-}
\begin{définition}\fra se déguiser, imiter\end{définition}
\begin{définition}\cmn 打扮;模仿\end{définition}
\begin{exemple}\jya staʁthɤr kɯ ɯ-pi ta-nɯɕpɯz\cmn 斯达塔尔学了他的哥哥\end{exemple}
\begin{exemple}\jya tɤ-pɤtso ra kɯ ʁmaʁmi ɲɯ-ɤz-nɯɕpɯz-nɯ\cmn 孩子们在打扮成士兵\end{exemple}
\begin{exemple}\jya tɤ-pɤtso ra kɯ ɕɤmɯɣdɯ ɯ-kɯ-lɤt ɲɯ-ɤz-nɯɕpɯz-nɯ\cmn 孩子们装作在打枪\end{exemple}
\begin{exemple}\jya tɤ́-wɣ-nɯɕpɯz-a\cmn 他模仿了我的模样\end{exemple}
\begin{relation-sémantique}\confer{
\hyperlink{Ⓔɯ-ɕpɯz}{\textit{ \papi{ɯ-ɕpɯz}}}
}\end{relation-sémantique}\end{entrée}

\begin{entrée}
\vedette{\hypertarget{Ⓔnɯɕqhu}{\papi{ nɯɕqhu}}}\markboth{nɯɕqhu}{}\classe{vt}\acception{1}
\paradigme{\textit{dir :} \jya \_}
\begin{définition}\fra tourner le dos à\end{définition}
\begin{définition}\cmn 转身背向别人\end{définition}
\begin{exemple}\jya khɯɣɲɟɯ ku-nɯɕqhe-a ŋu\cmn 我背向窗子\end{exemple}\acception{2}
\paradigme{\textit{dir :} \jya tɤ-}
\begin{définition}\fra trahir, revenir sur sa parole\end{définition}
\begin{définition}\cmn 背叛;违背约定;违反约定\end{définition}
\begin{exemple}\jya nɤʑo tu-kɯ-nɯɕqhu-a ɲɯ-ŋu\cmn 你背叛我\end{exemple}
\begin{relation-sémantique}\synonyme{
\hyperlink{Ⓔnɯɣɤtɕa}{\textit{ \papi{nɯɣɤtɕa}}}
}\end{relation-sémantique}
\begin{relation-sémantique}\antonyme{
\hyperlink{Ⓔnɯʁɤri}{\textit{ \papi{nɯʁɤri}}}
}\end{relation-sémantique}
\begin{relation-sémantique}\confer{
\hyperlink{Ⓔɯ-qhu}{\textit{ \papi{ɯ-qhu}}}
}\end{relation-sémantique}
\begin{relation-sémantique}\confer{
\hyperlink{Ⓔznɯɕqhɯɕqhu}{\textit{ \papi{znɯɕqhɯɕqhu}}}
}\end{relation-sémantique}\begin{sous-entrée}
\vedette{\hypertarget{}{\papi{ anɯɕqhɯɕqhu}}}\markboth{anɯɕqhɯɕqhu}{}\classe{vi}
\begin{définition}\ 
\begin{déclaration}\grammar{recip}\end{déclaration}\end{définition}\acception{1}
\begin{définition}\fra être opposé, être le contraire (parole)\end{définition}
\begin{définition}\cmn 相反(话)\end{définition}
\begin{exemple}\jya ndʑiʑo ndʑi-kɤ-ti ɲɯ-ɤnɯɕqhɯɕqhu\cmn 他说的话跟你说的话是相反的\end{exemple}\acception{2}
\begin{définition}\fra revenir chacun sur sa parole\end{définition}
\begin{définition}\cmn 互相违背(约定的事情)\end{définition}
\begin{exemple}\jya tɯkrɤz tɤ-βzu-tɕi ŋu tɕe, mɤ-anɯɕqhɯɕqhu-tɕi ra nɤ!\cmn 我们商量好了,不要违背约定\end{exemple}
\end{sous-entrée}\end{entrée}

\begin{entrée}
\vedette{\hypertarget{Ⓔnɯɕtar}{\papi{ nɯɕtar}}}\markboth{nɯɕtar}{}\classe{vi}
\paradigme{\textit{dir :} \jya pɯ-}
\begin{définition}\fra avoir une leçon\end{définition}
\begin{définition}\cmn 受教训\end{définition}
\begin{exemple}\jya aʑo pɯ-nɯɕtar-a\cmn 我受过那个教训\end{exemple}
\begin{exemple}\jya a-pɯ-tɯ-nɯɕtar ɲɯ-ra wo!\cmn 你应该吸取教训!\end{exemple}
\begin{exemple}\jya mɯ-ɲɤ-stu-nɯ ma pjɤ-nɯɕtar-nɯ\cmn 他们吸取了教训,再也不相信他了\end{exemple}
\begin{exemple}\jya aʑo kutɕu kɤ-ɣi pɯ-nɯɕtar-a ma ɲɯ-ɤrqhi\cmn 我来到这里很辛苦,因为很远\end{exemple}\begin{sous-entrée}
\vedette{\hypertarget{}{\papi{ sɤnɯɕtar}}}\markboth{sɤnɯɕtar}{} (\variante{sɤɕtar}) \classe{vs}
\begin{définition}\fra qui donne une leçon\end{définition}
\begin{définition}\cmn 令人受教训\end{définition}
\end{sous-entrée}\begin{sous-entrée}
\vedette{\hypertarget{}{\papi{ znɯɕtar}}}\markboth{znɯɕtar}{}\classe{vt}
\paradigme{\textit{dir :} \jya pɯ-}
\begin{définition}\fra donner une leçon\end{définition}
\begin{définition}\cmn 教训\end{définition}
\end{sous-entrée}\end{entrée}

\begin{entrée}
\vedette{\hypertarget{Ⓔnɯɕɯβɟɤlɯlu}{\papi{ nɯɕɯβɟɤlɯlu}}}\markboth{nɯɕɯβɟɤlɯlu}{}
\classe{vi}
\begin{définition}\fra à qui mieux mieux\end{définition}
\begin{définition}\cmn 争先恐后\end{définition}
\begin{exemple}\jya laχtɕha kɤ-χtɯ ɲɯ-pe tɕe, to-nɯɕɯβɟɤlɯlu-nɯ ʑo to-χtɯ-nɯ\cmn 买的东西很好,所以他们争着买\end{exemple}
\begin{exemple}\jya nɤki nɯ kɯ-fse ɯ-qhu tɕe kɤ-χtɯ tu ɕti tɕe, kɤ-nɯɕɯβɟɤlɯlu mɤ-ra wo\cmn 那个东西以后还有的买,不必争\end{exemple}\end{entrée}

\begin{entrée}
\vedette{\hypertarget{Ⓔnɯɕɯlu}{\papi{ nɯɕɯlu}}}\markboth{nɯɕɯlu}{}\classe{vs}
\begin{définition}\fra qui peut être traite pendant longtemps (vache)\end{définition}
\begin{définition}\cmn 挤奶期长(的奶牛)\end{définition}
\begin{exemple}\jya ki nɯŋa ki kɯ-nɯɕɯlu ci ŋu\cmn 这个奶牛的挤奶期比较长\end{exemple}
\end{entrée}

\begin{entrée}
\vedette{\hypertarget{Ⓔnɯɕɯrɲɟo}{\papi{ nɯɕɯrɲɟo}}}\markboth{nɯɕɯrɲɟo}{}
\classe{vi}
\paradigme{\textit{dir :} \jya pɯ-}
\begin{définition}\fra rougir (feuilles d'arbre en automne)\end{définition}
\begin{définition}\cmn 秋天叶子变色\end{définition}
\begin{exemple}\jya si pjɤ-nɯɕɯrɲɟo\cmn 树的叶子变红了\end{exemple}\end{entrée}

\begin{entrée}
\vedette{\hypertarget{Ⓔnɯɕɯʁjɯ}{\papi{ nɯɕɯʁjɯ}}}\markboth{nɯɕɯʁjɯ}{}\classe{vi}
\paradigme{\textit{dir :} \jya nɯ-}
\begin{définition}\fra faire le mort\end{définition}
\begin{définition}\cmn 装死\end{définition}
\begin{exemple}\jya ma-nɯ-tɯ-nɯɕɯʁjɯ kɯ nɯ-rɤma\cmn 你不要装自己不会做,要劳动!\end{exemple}
\begin{relation-sémantique}\synonyme{
\hyperlink{Ⓔraʁjɯ}{\textit{ \papi{raʁjɯ}}}
}\end{relation-sémantique}\end{entrée}

\begin{entrée}
\vedette{\hypertarget{Ⓔnɯdu}{\papi{ nɯdu}}}\markboth{nɯdu}{}\classe{vi}
\paradigme{\textit{dir :} \jya pɯ-}
\begin{définition}\fra tirer à la courte paille\end{définition}
\begin{définition}\cmn 抽签\end{définition}
\begin{exemple}\jya ɕɯ-nɯdu-j\cmn 我们来抽签\end{exemple}
\begin{relation-sémantique}\confer{
\hyperlink{Ⓔɯ-du}{\textit{ \papi{ɯ-du}}}
}\end{relation-sémantique}\end{entrée}

\begin{entrée}
\vedette{\hypertarget{Ⓔnɯdrɯβ}{\papi{ nɯdrɯβ}}}\markboth{nɯdrɯβ}{}
\classe{vt}
\begin{définition}\fra encorner à de nombreuses reprises\end{définition}
\begin{définition}\cmn 一次又一次地顶\end{définition}
\begin{exemple}\jya srɯnmɯ nɯ to-nɯdrɯβ ʑo to-tɕhɯ\cmn 水牛把妖精一次又一次地顶了\end{exemple}
\begin{relation-sémantique}\confer{
\hyperlink{Ⓔdrɯβ}{\textit{ \papi{drɯβ}}}
}\end{relation-sémantique}\end{entrée}

\begin{entrée}
\vedette{\hypertarget{Ⓔnɯfɕi}{\papi{ nɯfɕi}}}\markboth{nɯfɕi}{}
\classe{vt}
\paradigme{\textit{dir :} \jya nɯ-}
\begin{définition}\fra s'inquiéter, ne pas vouloir faire (un travail)\end{définition}
\begin{définition}\cmn 怕麻烦;担心\end{définition}
\begin{exemple}\jya nɯ kɤ-nɤma aʑo ɲɯ-nɯfɕi-a ɕti\cmn 我不想做这个这个工作,很怕麻烦\end{exemple}
\begin{exemple}\jya kɯki ɯ-tɯ-dɤn ɲɯ-tɕhom tɕe aj ɲɯ-nɯfɕi-a\cmn 太多工作,我怕麻烦\end{exemple}
\begin{exemple}\jya zgoku ɲɯ-mbro tɕe kɤ-ɕe ɲɯ-nɯfɕi\cmn 山很高,他想去,怕麻烦\end{exemple}
\begin{exemple}\jya khro tɤ-nɤma-t-a tɕe, mɤʑɯ kɤ-nɤma ɲɯ-nɯfɕi-a ɕti\cmn 我已经工作很多了,不想再做了\end{exemple}\end{entrée}

\begin{entrée}
\vedette{\hypertarget{Ⓔnɯfkurzʁe}{\papi{ nɯfkurzʁe}}}\markboth{nɯfkurzʁe}{}\classe{vi}
\paradigme{\textit{dir :} \jya \_}
\begin{définition}\fra transporter des charges sur le dos\end{définition}
\begin{définition}\cmn 背东西\end{définition}
\begin{exemple}\jya jisŋi ɕ-pɯ-nɯfkurzʁe-j tɕe wuma ʑo pɯ-ɴqa\cmn 我们今天背了很多东西,很辛苦\end{exemple}
\begin{relation-sémantique}\confer{
\hyperlink{Ⓔnɯzʁe}{\textit{ \papi{nɯzʁe}}}
}\end{relation-sémantique}
\begin{relation-sémantique}\confer{
\hyperlink{Ⓔfkur}{\textit{ \papi{fkur}}}
}\end{relation-sémantique}
\begin{relation-sémantique}\confer{
\hyperlink{Ⓔfkurzʁe}{\textit{ \papi{fkurzʁe}}}
}\end{relation-sémantique}\end{entrée}

\begin{entrée}
\vedette{\hypertarget{ⒺnɯfseⒽ2}{\papi{ nɯfse}}}\markboth{nɯfse}{}\homonyme{2}
\classe{adv}\acception{1}
\begin{définition}\fra comme cela, sans but particulier\end{définition}
\begin{définition}\cmn 就这样;笼统地;没有目标地;随便\end{définition}
\begin{exemple}\jya nɯfse ɕ-tu-nɤŋkɯŋke-a ŋu\cmn 我(没有目标地)逛街\end{exemple}\acception{2}
\begin{définition}\fra malgré tout\end{définition}
\begin{définition}\cmn 无论怎么样都……,不顾一切
\end{définition}
\begin{exemple}\jya ɲɯ-mɯtɕaʁ ri, nɯfse ʑo pɯ-rɤʑit-a pɯ-ta-nɤjo.\cmn 虽然很冷,我还是在那里等了你\end{exemple}
\begin{relation-sémantique}\confer{
\hyperlink{ⒺfseⒽ1}{\textit{ \papi{fse1}}}
}\end{relation-sémantique}\end{entrée}

\begin{entrée}
\vedette{\hypertarget{ⒺnɯfseⒽ1}{\papi{ nɯfse}}}\markboth{nɯfse}{}\homonyme{1}\classe{vt}
\paradigme{\textit{dir :} \jya kɤ-}
\begin{définition}\fra reconnaître, être familier\end{définition}
\begin{définition}\cmn 认得;熟悉(人物)\end{définition}
\begin{exemple}\jya ɯ-ɲɯ́-kɯ-nɯfse-a?\cmn 你认得我吗?\end{exemple}
\begin{exemple}\jya ɯʑo kɤ-nɯfse-t-a\cmn 我认得他了\end{exemple}
\begin{exemple}\jya ki tɯrme ɲɯ-nɯfse-a\cmn 我认识这个人\end{exemple}
\begin{exemple}\jya nɤ-kɤ-nɯfse kɯ-me nɯ sɤzdɯɣ\cmn 没有你认识的人,很难受\end{exemple}\end{entrée}

\begin{entrée}
\vedette{\hypertarget{Ⓔnɯfsosɲɯm}{\papi{ nɯfsosɲɯm}}}\markboth{nɯfsosɲɯm}{}\classe{vi}
\begin{définition}\ 
\begin{déclaration}\grammar{denom}\end{déclaration}\end{définition}
\begin{définition}\fra demander l'aumône (moines)\end{définition}
\begin{définition}\cmn 讨布施\end{définition}
\begin{exemple}\jya kɯ-nɯfsosɲɯm jɤ-ari-a\cmn 我去讨布施了f\end{exemple}
\begin{relation-sémantique}\confer{
\hyperlink{Ⓔfsosɲɯm}{\textit{ \papi{fsosɲɯm}}}
}\end{relation-sémantique}\end{entrée}

\begin{entrée}
\vedette{\hypertarget{Ⓔnɯftɕaka}{\papi{ nɯftɕaka}}}\markboth{nɯftɕaka}{}\classe{vt}
\paradigme{\textit{dir :} \jya tɤ-}
\begin{définition}\ 
\begin{déclaration}\grammar{denom}\end{déclaration}\end{définition}
\begin{définition}\fra préparer\end{définition}
\begin{définition}\cmn 准备;收拾\end{définition}
\begin{relation-sémantique}\synonyme{
\hyperlink{ⒺmɲoⒽ1}{\textit{ \papi{mɲo}}}
}\end{relation-sémantique}
\begin{relation-sémantique}\confer{
\hyperlink{Ⓔrɯftɕaka}{\textit{ \papi{rɯftɕaka}}}
}\end{relation-sémantique}
\begin{relation-sémantique}\confer{
\hyperlink{Ⓔftɕaka}{\textit{ \papi{ftɕaka}}}
}\end{relation-sémantique}\end{entrée}

\begin{entrée}
\vedette{\hypertarget{Ⓔnɯftsaʁ}{\papi{ nɯftsaʁ}}}\markboth{nɯftsaʁ}{}\classe{vi}
\paradigme{\textit{dir :} \jya pɯ-}
\begin{définition}\ 
\begin{déclaration}\grammar{denom}\end{déclaration}\end{définition}
\begin{définition}\fra couler goutte à goutte\end{définition}
\begin{définition}\cmn 滴水\end{définition}
\begin{exemple}\jya @kongtiao ɯ-ŋgɯ tɯ-ci ɲɯ-nɯftsaʁ\cmn 空调在滴水\end{exemple}
\begin{exemple}\jya khɤxtu ɲɯ-nɯftsaʁ\cmn 房背在滴水\end{exemple}
\begin{exemple}\jya @ganggang ɲɯ-spoʁ rcáma, tɯ-ci ɲɯ-nɯftsaʁ\cmn 也许是杯子漏了,因为在滴水\end{exemple}
\begin{exemple}\jya tɯ-ftsaʁ tʂhɤtnɤtʂhɤt ʑo ɲɯ-nɯftsaʁ\cmn 一滴一滴地漏水\end{exemple}
\begin{relation-sémantique}\confer{
\hyperlink{Ⓔtɯftsaʁ}{\textit{ \papi{tɯftsaʁ}}}
}\end{relation-sémantique}\end{entrée}

\begin{entrée}
\vedette{\hypertarget{Ⓔnɯgrɤl}{\papi{ nɯgrɤl}}}\markboth{nɯgrɤl}{}\classe{vi}
\paradigme{\textit{dir :} \jya kɤ-}
\begin{définition}\ 
\begin{déclaration}\grammar{denom}\end{déclaration}\end{définition}
\begin{définition}\fra être en rang\end{définition}
\begin{définition}\cmn 平排\end{définition}
\begin{exemple}\jya ʁmaʁmi ra ko-nɯgrɤl-nɯ\cmn 士兵排成队伍了\end{exemple}
\begin{relation-sémantique}\confer{
\hyperlink{Ⓔɯ-grɤl}{\textit{ \papi{ɯ-grɤl}}}
}\end{relation-sémantique}\end{entrée}

\begin{entrée}
\vedette{\hypertarget{Ⓔnɯɣɤja}{\papi{ nɯɣɤja}}}\markboth{nɯɣɤja}{}
\classe{vl}
\paradigme{\textit{dir :} \jya tɤ-}
\begin{définition}\fra résister à\end{définition}
\begin{définition}\cmn 反抗;对抗;顶嘴\end{définition}
\begin{exemple}\jya nɤ-mu nɤ-wa ma-tɤ-tɯ-nɯɣɤje\cmn 你不要跟你父母顶嘴\end{exemple}
\begin{relation-sémantique}\synonyme{
\hyperlink{Ⓔnɯkhɤja}{\textit{ \papi{nɯkhɤja}}}
}\end{relation-sémantique}\end{entrée}

\begin{entrée}
\vedette{\hypertarget{Ⓔnɯɣɤtɕa}{\papi{ nɯɣɤtɕa}}}\markboth{nɯɣɤtɕa}{}
\classe{vt}
\paradigme{\textit{dir :} \jya pɯ-}
\begin{définition}\fra revenir sur sa parole\end{définition}
\begin{définition}\cmn 违反约定\end{définition}
\begin{exemple}\jya jɯfɕɯr tɯkrɤz kɯ-βdɯ-βdi tɤ-nɯ-βzu-tɕi ɕti ri, nɤʑo pjɤ-kɯ-nɯɣɤtɕa-a.\cmn 我们昨天商量得好好的,但是你违反了约定\end{exemple}
\begin{relation-sémantique}\synonyme{
\hyperlink{Ⓔnɯɕqhu}{\textit{ \papi{nɯɕqhu}}}
}\end{relation-sémantique}
\begin{relation-sémantique}\confer{
\hyperlink{Ⓔɣɤtɕa}{\textit{ \papi{ɣɤtɕa}}}
}\end{relation-sémantique}
\end{entrée}

\begin{entrée}
\vedette{\hypertarget{Ⓔnɯɣbɯɣ}{\papi{ nɯɣbɯɣ}}}\markboth{nɯɣbɯɣ}{}\classe{vt}
\paradigme{\textit{dir :} \jya nɯ-}
\begin{définition}\ 
\begin{déclaration}\grammar{appl}\end{déclaration}\end{définition}
\begin{définition}\fra penser à\end{définition}
\begin{définition}\cmn 想念\end{définition}
\begin{exemple}\jya mbarkhom ɲɯ-nɯɣbɯɣ-a\cmn 我想念马尔康\end{exemple}
\begin{exemple}\jya @faguo ɲɯ-nɯɣbɯɣ-a\cmn 我想念法国\end{exemple}
\begin{exemple}\jya ɲɯ-ta-nɯɣbɯɣ-nɯ\cmn 我想念你们\end{exemple}\begin{sous-entrée}
\vedette{\hypertarget{}{\papi{ anɯɣbɯɣbɯɣ}}}\markboth{anɯɣbɯɣbɯɣ}{}\classe{vi}
\begin{définition}\fra se manquer les uns aux autres\end{définition}
\begin{définition}\cmn 互相思念\end{définition}
\begin{exemple}\jya ɲɯ-ɤnɯɣbɯɣbɯɣ-ndʑi\cmn 他们俩互相思念\end{exemple}
\begin{relation-sémantique}\confer{
\hyperlink{Ⓔbɯɣ}{\textit{ \papi{bɯɣ}}}
}\end{relation-sémantique}
\end{sous-entrée}\end{entrée}

\begin{entrée}
\vedette{\hypertarget{Ⓔnɯɣe}{\papi{ nɯɣe}}}\markboth{nɯɣe}{}\classe{part}
\begin{définition}\fra n'est ce pas ?\end{définition}
\begin{définition}\cmn 陈述自己的感觉,征求别人的看法“是吗”(只能和程度动词化名词合用)\end{définition}
\begin{exemple}\jya ɯ-tɯ-pe nɯɣe\cmn 很好是吗\end{exemple}
\begin{exemple}\jya jisŋi tɯ-mɯ ɯ-tɯ-jɯm nɯɣe\cmn 今天天气很好是吗\end{exemple}
\end{entrée}

\begin{entrée}
\vedette{\hypertarget{Ⓔnɯɣi}{\papi{ nɯɣi}}}\markboth{nɯɣi}{}
\classe{vi}
\paradigme{\textit{dir :} \jya \_}
\paradigme{\textit{past stem :} \jya nɯɣe}
\begin{définition}\ 
\begin{déclaration}\grammar{vert}\end{déclaration}\end{définition}
\begin{définition}\fra revenir\end{définition}
\begin{définition}\cmn 回来\end{définition}
\begin{exemple}\jya @shiwuhao ri lɤ-ari tɕe, pɤjkhu mɯ-thɯ-nɯɣe\cmn 他十五号就去了,还没有回来\end{exemple}
\begin{exemple}\jya fso tɕe chɯ-nɯɣi ɲɯ-khɯ khi\cmn 他说明天就可以回来\end{exemple}
\begin{exemple}\jya ʑatsa lu-nɯɣi ɕti\cmn 他很快回家\end{exemple}
\begin{relation-sémantique}\confer{
\hyperlink{Ⓔɣi}{\textit{ \papi{ɣi}}}
}\end{relation-sémantique}\end{entrée}

\begin{entrée}
\vedette{\hypertarget{Ⓔnɯɣɟɯ}{\papi{ nɯɣɟɯ}}}\markboth{nɯɣɟɯ}{}
\classe{vi}
\paradigme{\textit{dir :} \jya nɯ-}
\begin{définition}\fra mourir de faim\end{définition}
\begin{définition}\cmn 饿死\end{définition}
\begin{exemple}\jya ɲɤ-nɯɣɟɯ\cmn 他饿死了\end{exemple}
\begin{exemple}\jya fsapaʁ ɲɤ-nɯɣɟɯ\cmn 牲畜饿死了\end{exemple}
\begin{relation-sémantique}\confer{
 \papi{ɣɯ}
}\end{relation-sémantique}\begin{sous-entrée}
\vedette{\hypertarget{}{\papi{ znɯɣɟɯ}}}\markboth{znɯɣɟɯ}{}\classe{vt}
\paradigme{\textit{dir :} \jya nɯ-}
\begin{définition}\fra faire mourir de faim\end{définition}
\begin{définition}\cmn 令……饿死\end{définition}
\end{sous-entrée}\end{entrée}

\begin{entrée}
\vedette{\hypertarget{Ⓔnɯɣmu}{\papi{ nɯɣmu}}}\markboth{nɯɣmu}{}
\classe{vt}
\paradigme{\textit{dir :} \jya nɯ-}
\begin{définition}\fra avoir peur de\end{définition}
\begin{définition}\cmn 害怕\end{définition}
\begin{exemple}\jya khu ɲɯ-nɯɣme-a\cmn 我怕老虎\end{exemple}
\begin{exemple}\jya qapri ɲɯ-nɯɣme-a\cmn 我怕蛇\end{exemple}
\begin{exemple}\jya ɬɤndʐi ɲɯ-nɯɣme-a\cmn 我怕鬼\end{exemple}
\begin{exemple}\jya nɯ-nɯɣmu-t-a\cmn 我怕了\end{exemple}
\begin{relation-sémantique}\confer{
\hyperlink{ⒺmuⒽ1}{\textit{ \papi{mu1}}}
}\end{relation-sémantique}
\begin{relation-sémantique}\confer{
\hyperlink{Ⓔɕɯɣmu}{\textit{ \papi{ɕɯɣmu}}}
}\end{relation-sémantique}
\begin{relation-sémantique}\confer{
\hyperlink{Ⓔsɤɣmu}{\textit{ \papi{sɤɣmu}}}
}\end{relation-sémantique}\end{entrée}

\begin{entrée}
\vedette{\hypertarget{Ⓔnɯɣmaz}{\papi{ nɯɣmaz}}}\markboth{nɯɣmaz}{}
\classe{vi}
\paradigme{\textit{dir :} \jya tɤ-}
\begin{définition}\ 
\begin{déclaration}\grammar{denom}\end{déclaration}\end{définition}
\begin{définition}\fra se blesser\end{définition}
\begin{définition}\cmn 受伤\end{définition}
\begin{exemple}\jya to-nɯɣmaz\cmn 他受了伤\end{exemple}
\begin{exemple}\jya aʑo tɤ-nɯɣmaz-a\cmn 我受了伤\end{exemple}
\begin{exemple}\jya a-jaʁ tɤ-nɯɣmaz\cmn 我的手受伤了\end{exemple}
\begin{relation-sémantique}\confer{
\hyperlink{Ⓔtɯ-ɣmaz}{\textit{ \papi{tɯ-ɣmaz}}}
}\end{relation-sémantique}\begin{sous-entrée}
\vedette{\hypertarget{}{\papi{ znɯɣmaz}}}\markboth{znɯɣmaz}{}\classe{vt}
\paradigme{\textit{dir :} \jya tɤ-}
\begin{définition}\fra blesser\end{définition}
\begin{définition}\cmn 弄伤\end{définition}
\begin{exemple}\jya ɯ-mi to-nɯ-znɯɣmaz\cmn 他不小心把脚弄伤了\end{exemple}
\begin{exemple}\jya qartshaz ɕɤmɯɣdɯ kɯ tó-wɣ-znɯɣmaz\cmn 鹿被枪打伤了\end{exemple}
\end{sous-entrée}\end{entrée}

\begin{entrée}
\vedette{\hypertarget{Ⓔnɯɣmbɤβ}{\papi{ nɯɣmbɤβ}}}\markboth{nɯɣmbɤβ}{}\classe{vs}
\paradigme{\textit{dir :} \jya thɯ-}
\begin{définition}\ 
\begin{déclaration}\grammar{denom}\end{déclaration}\end{définition}
\begin{définition}\fra enfler\end{définition}
\begin{définition}\cmn 肿起来\end{définition}
\begin{exemple}\jya βɣɤza (βɣɤrtshi) kɯ kó-wɣ-mtsɯɣ-a tɕe chɤ-nɯɣmbɤβ\cmn 苍蝇(蚊子)叮了我就肿了\end{exemple}
\begin{exemple}\jya a-jaʁ, a-mi chɤ-nɯɣmbɤβ\cmn 我的手,我的脚肿了\end{exemple}
\begin{exemple}\jya pɯ́-wɣ-ʁndɯ tɕe chɤ-nɯɣmbɤβ\cmn 被打了就肿了\end{exemple}\begin{sous-entrée}
\vedette{\hypertarget{}{\papi{ znɯɣmbɤβ}}}\markboth{znɯɣmbɤβ}{}\classe{vt}
\paradigme{\textit{dir :} \jya thɯ-}
\begin{relation-sémantique}\confer{
\hyperlink{Ⓔtɯ-ɣmbɤβ}{\textit{ \papi{tɯ-ɣmbɤβ}}}
}\end{relation-sémantique}
\end{sous-entrée}\end{entrée}

\begin{entrée}
\vedette{\hypertarget{Ⓔnɯɣmɯm}{\papi{ nɯɣmɯm}}}\markboth{nɯɣmɯm}{}
\classe{vs}
\paradigme{\textit{dir :} \jya tɤ-}
\begin{définition}\fra avoir des exigences sur la nourriture\end{définition}
\begin{définition}\cmn 讲究食物\end{définition}
\begin{exemple}\jya tɯrme tsuku wuma ʑo kɯ-nɯɣmɯm tu\cmn 有的人很讲究食物\end{exemple}
\begin{relation-sémantique}\confer{
\hyperlink{Ⓔmɯm}{\textit{ \papi{mɯm}}}
}\end{relation-sémantique}\end{entrée}

\begin{entrée}
\vedette{\hypertarget{Ⓔnɯɣɲo}{\papi{ nɯɣɲo}}}\markboth{nɯɣɲo}{}
\begin{relation-sémantique}\confer{
\hyperlink{Ⓔɲo}{\textit{ \papi{ɲo}}}
}\end{relation-sémantique}\end{entrée}

\begin{entrée}
\vedette{\hypertarget{Ⓔnɯɣur}{\papi{ nɯɣur}}}\markboth{nɯɣur}{}
\classe{vi}
\paradigme{\textit{dir :} \jya pɯ-}
\begin{définition}\ 
\begin{déclaration}\grammar{denom}\end{déclaration}\end{définition}
\begin{définition}\fra subir le gel (plante)\end{définition}
\begin{définition}\cmn 遭霜了\end{définition}
\begin{exemple}\jya @cai pjɤ-nɯɣur\cmn 菜遭霜了\end{exemple}
\begin{exemple}\jya @yangyu pjɤ-nɯɣur\cmn 土豆遭霜了\end{exemple}
\begin{exemple}\jya tɯqe pjɯ-nɯɣur ra\cmn 要经历各种艰难才懂得某些道理\end{exemple}
\begin{relation-sémantique}\confer{
\hyperlink{Ⓔtɯɣur}{\textit{ \papi{tɯɣur}}}
}\end{relation-sémantique}\end{entrée}

\begin{entrée}
\vedette{\hypertarget{Ⓔnɯɣrɤβ}{\papi{ nɯɣrɤβ}}}\markboth{nɯɣrɤβ}{}\classe{vt}
\paradigme{\textit{dir :} \jya \_}
\begin{définition}\fra tendre la main pour attraper\end{définition}
\begin{définition}\cmn 伸手去抓\end{définition}
\begin{exemple}\jya lɤ-nɯɣraβ-a ri mɯ́j-ɕaβ-a\cmn 我把手伸过去,但是够不着\end{exemple}\end{entrée}

\begin{entrée}
\vedette{\hypertarget{Ⓔnɯɣɯβzjoz}{\papi{ nɯɣɯβzjoz}}}\markboth{nɯɣɯβzjoz}{}
\begin{relation-sémantique}\confer{
\hyperlink{Ⓔβzjoz}{\textit{ \papi{βzjoz}}}
}\end{relation-sémantique}\end{entrée}

\begin{entrée}
\vedette{\hypertarget{Ⓔnɯɣɯcɯm}{\papi{ nɯɣɯcɯm}}}\markboth{nɯɣɯcɯm}{}
\begin{relation-sémantique}\confer{
\hyperlink{Ⓔcɯm}{\textit{ \papi{cɯm}}}
}\end{relation-sémantique}\end{entrée}

\begin{entrée}
\vedette{\hypertarget{Ⓔnɯɣɯɕe}{\papi{ nɯɣɯɕe}}}\markboth{nɯɣɯɕe}{}\classe{vs}
\begin{définition}\fra praticable\end{définition}
\begin{définition}\cmn 方便去,好走\end{définition}
\begin{relation-sémantique}\confer{
\hyperlink{Ⓔɕe}{\textit{ \papi{ɕe}}}
}\end{relation-sémantique}\end{entrée}

\begin{entrée}
\vedette{\hypertarget{Ⓔnɯɣɯɕɯftaʁ}{\papi{ nɯɣɯɕɯftaʁ}}}\markboth{nɯɣɯɕɯftaʁ}{}
\begin{relation-sémantique}\confer{
\hyperlink{Ⓔɕɯftaʁ}{\textit{ \papi{ɕɯftaʁ}}}
}\end{relation-sémantique}\end{entrée}

\begin{entrée}
\vedette{\hypertarget{Ⓔnɯɣɯfɕɤt}{\papi{ nɯɣɯfɕɤt}}}\markboth{nɯɣɯfɕɤt}{}
\begin{relation-sémantique}\confer{
\hyperlink{ⒺfɕɤtⒽ1}{\textit{ \papi{fɕɤt1}}}
}\end{relation-sémantique}\end{entrée}

\begin{entrée}
\vedette{\hypertarget{Ⓔnɯɣɯfsu}{\papi{ nɯɣɯfsu}}}\markboth{nɯɣɯfsu}{}
\classe{vt}
\paradigme{\textit{dir :} \jya tɤ-}
\begin{définition}\ 
\begin{déclaration}\grammar{denom}\end{déclaration}\end{définition}
\begin{définition}\fra devenir ami\end{définition}
\begin{définition}\cmn 交朋友\end{définition}
\begin{exemple}\jya tɤ-nɯɣɯfsu-t-a\cmn 我跟他交了朋友\end{exemple}
\begin{exemple}\jya jiɕqha nɯ tɯrme kɯ-pe ci ɲɯ-ŋu, tɤ-nɯɣɯfsu-t-a\cmn 这个人很好,我跟他交了朋友\end{exemple}
\begin{relation-sémantique}\synonyme{
\hyperlink{Ⓔnɯβzaŋsa}{\textit{ \papi{nɯβzaŋsa}}}
}\end{relation-sémantique}
\begin{relation-sémantique}\confer{
\hyperlink{Ⓔɣɯfsu}{\textit{ \papi{ɣɯfsu}}}
}\end{relation-sémantique}\end{entrée}

\begin{entrée}
\vedette{\hypertarget{Ⓔnɯɣɯftɯl}{\papi{ nɯɣɯftɯl}}}\markboth{nɯɣɯftɯl}{}
\classe{vi}
\paradigme{\textit{dir :} \jya pɯ-}
\begin{définition}\ 
\begin{déclaration}\grammar{facil}\end{déclaration}\end{définition}
\begin{définition}\fra être facile à apprivoiser\end{définition}
\begin{définition}\cmn 容易驯服\end{définition}
\begin{exemple}\jya mbro nɯ ɲɯ-nɯɣɯftɯl\cmn 那匹马容易驯服\end{exemple}
\begin{exemple}\jya jla ɲɯ-nɯɣɯftɯl\cmn 犏牛容易驯服\end{exemple}
\begin{relation-sémantique}\synonyme{
\hyperlink{Ⓔɣɤndɯl}{\textit{ \papi{ɣɤndɯl}}}
}\end{relation-sémantique}
\begin{relation-sémantique}\confer{
\hyperlink{ⒺftɯlⒽ1}{\textit{ \papi{ftɯl1}}}
}\end{relation-sémantique}\end{entrée}

\begin{entrée}
\vedette{\hypertarget{Ⓔnɯɣɯjmɯt}{\papi{ nɯɣɯjmɯt}}}\markboth{nɯɣɯjmɯt}{}
\begin{relation-sémantique}\confer{
\hyperlink{Ⓔjmɯt}{\textit{ \papi{jmɯt}}}
}\end{relation-sémantique}\end{entrée}

\begin{entrée}
\vedette{\hypertarget{Ⓔnɯɣɯjpa}{\papi{ nɯɣɯjpa}}}\markboth{nɯɣɯjpa}{}\classe{vs}
\paradigme{\textit{dir :} \jya tɤ-}
\begin{définition}\ 
\begin{déclaration}\grammar{facil}\end{déclaration}\end{définition}
\begin{définition}\fra facile, pratique\end{définition}
\begin{définition}\cmn 方便;好办\end{définition}
\begin{exemple}\jya ʑɤŋgɯz kɯ-nɯɣɯjpa nɯ-βzu-tɕi\cmn 我们互相提供方便\end{exemple}
\begin{exemple}\jya tɤ-tɯt-a nɯ a-tɤ-tɯ-ste tɕe, ʑɤŋgɯz a-pɯ-nɯɣɯjpa\cmn 你如果照我说的去办,你我都方便\end{exemple}\end{entrée}

\begin{entrée}
\vedette{\hypertarget{Ⓔnɯɣɯmto}{\papi{ nɯɣɯmto}}}\markboth{nɯɣɯmto}{}
\classe{vs}
\paradigme{\textit{dir :} \jya tɤ-}
\begin{définition}\fra très visible\end{définition}
\begin{définition}\cmn 容易发现\end{définition}
\begin{exemple}\jya tɤɣal ɲɯ-rɤʑi, ɲɯ-nɯɣɯmto\cmn 很明显,容易发现\end{exemple}
\begin{relation-sémantique}\confer{
 \papi{mto}
}\end{relation-sémantique}\end{entrée}

\begin{entrée}
\vedette{\hypertarget{Ⓔnɯɣɯnɤma}{\papi{ nɯɣɯnɤma}}}\markboth{nɯɣɯnɤma}{}
\begin{relation-sémantique}\confer{
\hyperlink{Ⓔnɤma}{\textit{ \papi{nɤma}}}
}\end{relation-sémantique}\end{entrée}

\begin{entrée}
\vedette{\hypertarget{Ⓔnɯɣɯndza}{\papi{ nɯɣɯndza}}}\markboth{nɯɣɯndza}{}
\classe{vs}
\paradigme{\textit{dir :} \jya tɤ-}
\begin{définition}\ 
\begin{déclaration}\grammar{facil}\end{déclaration}\end{définition}
\begin{définition}\fra bon à manger\end{définition}
\begin{définition}\cmn 好吃\end{définition}
\begin{exemple}\jya tɤ-mthɯm ɲɯ-nɯɣɯndza, mɯ́j-nɯɣɯndza\cmn 肉好吃,不好吃\end{exemple}
\begin{exemple}\jya @cai ɲɯ-nɯɣɯndza, mɯ́j-nɯɣɯndza\cmn 菜好吃,不好吃\end{exemple}
\begin{relation-sémantique}\synonyme{
\hyperlink{Ⓔmɯm}{\textit{ \papi{mɯm}}}
}\end{relation-sémantique}
\begin{relation-sémantique}\confer{
\hyperlink{Ⓔndza}{\textit{ \papi{ndza}}}
}\end{relation-sémantique}\end{entrée}

\begin{entrée}
\vedette{\hypertarget{Ⓔnɯɣɯntɕhoz}{\papi{ nɯɣɯntɕhoz}}}\markboth{nɯɣɯntɕhoz}{}
\classe{vs}
\paradigme{\textit{dir :} \jya tɤ-}
\begin{définition}\ 
\begin{déclaration}\grammar{facil}\end{déclaration}\end{définition}\acception{1}
\begin{définition}\fra facile à utiliser\end{définition}
\begin{définition}\cmn 好用;用起来很方便\end{définition}
\begin{exemple}\jya laʁdɯn ɲɯ-nɯɣɯntɕhoz, mɯ́j-nɯɣɯntɕhoz\cmn 工具好用,不好用\end{exemple}\acception{2}
\begin{définition}\fra qui sait tout faire\end{définition}
\begin{définition}\cmn 勤快,什么都会做\end{définition}
\begin{exemple}\jya jiɕqha tɯrme nɯ kɯ-nɯɣɯntɕhoz ci ɲɯ-ŋu\cmn 这个人是个很勤快,什么都会做的人\end{exemple}
\begin{relation-sémantique}\confer{
\hyperlink{Ⓔntɕhoz}{\textit{ \papi{ntɕhoz}}}
}\end{relation-sémantique}\end{entrée}

\begin{entrée}
\vedette{\hypertarget{Ⓔnɯɣɯŋga}{\papi{ nɯɣɯŋga}}}\markboth{nɯɣɯŋga}{}
\begin{relation-sémantique}\confer{
\hyperlink{Ⓔŋga}{\textit{ \papi{ŋga}}}
}\end{relation-sémantique}
\end{entrée}

\begin{entrée}
\vedette{\hypertarget{Ⓔnɯɣɯŋke}{\papi{ nɯɣɯŋke}}}\markboth{nɯɣɯŋke}{}
\classe{vs}
\paradigme{\textit{dir :} \jya tɤ-}
\begin{définition}\fra praticable (chemin)\end{définition}
\begin{définition}\cmn 好走(路)\end{définition}
\begin{exemple}\jya kɯki tʂu ki mɯ-to-pe tɕe mɯ́j-nɯɣɯŋke\cmn 这条路不好走\end{exemple}
\begin{exemple}\jya tʂu to-ɣɤβdi-nɯ tɕe to-nɯɣɯŋke tɕe to-pe\cmn 他们修了路以后就好走\end{exemple}
\begin{relation-sémantique}\confer{
\hyperlink{Ⓔŋke}{\textit{ \papi{ŋke}}}
}\end{relation-sémantique}\end{entrée}

\begin{entrée}
\vedette{\hypertarget{Ⓔnɯɣɯphɯt}{\papi{ nɯɣɯphɯt}}}\markboth{nɯɣɯphɯt}{}
\begin{relation-sémantique}\confer{
\hyperlink{Ⓔphɯt}{\textit{ \papi{phɯt}}}
}\end{relation-sémantique}\end{entrée}

\begin{entrée}
\vedette{\hypertarget{Ⓔnɯɣɯqaʁ}{\papi{ nɯɣɯqaʁ}}}\markboth{nɯɣɯqaʁ}{}
\begin{relation-sémantique}\confer{
\hyperlink{ⒺqaʁⒽ1}{\textit{ \papi{qaʁ1}}}
}\end{relation-sémantique}\end{entrée}

\begin{entrée}
\vedette{\hypertarget{Ⓔnɯɣɯsɤlɤɣɯ}{\papi{ nɯɣɯsɤlɤɣɯ}}}\markboth{nɯɣɯsɤlɤɣɯ}{}
\begin{relation-sémantique}\confer{
\hyperlink{Ⓔsɤlɤɣɯ}{\textit{ \papi{sɤlɤɣɯ}}}
}\end{relation-sémantique}\end{entrée}

\begin{entrée}
\vedette{\hypertarget{Ⓔnɯɣɯt}{\papi{ nɯɣɯt}}}\markboth{nɯɣɯt}{}
\classe{vt}
\paradigme{\textit{dir :} \jya \_}
\begin{définition}\ 
\begin{déclaration}\grammar{vert}\end{déclaration}\end{définition}
\begin{définition}\fra ramener\end{définition}
\begin{définition}\cmn 拿回来\end{définition}
\begin{exemple}\jya sɯɴɢoʁ tɤ-nɯɣɯt-a\cmn 我把干柴拿回来了\end{exemple}
\begin{exemple}\jya @yangyu tɤ-nɯɣɯt-a\cmn 我把土豆拿回来了\end{exemple}
\begin{relation-sémantique}\confer{
\hyperlink{Ⓔɣɯt}{\textit{ \papi{ɣɯt}}}
}\end{relation-sémantique}\end{entrée}

\begin{entrée}
\vedette{\hypertarget{Ⓔnɯɣɯti}{\papi{ nɯɣɯti}}}\markboth{nɯɣɯti}{}
\begin{relation-sémantique}\confer{
\hyperlink{Ⓔti}{\textit{ \papi{ti}}}
}\end{relation-sémantique}\end{entrée}

\begin{entrée}
\vedette{\hypertarget{Ⓔnɯɣɯtshi}{\papi{ nɯɣɯtshi}}}\markboth{nɯɣɯtshi}{}
\begin{relation-sémantique}\confer{
\hyperlink{ⒺtshiⒽ1}{\textit{ \papi{tshi1}}}
}\end{relation-sémantique}\end{entrée}

\begin{entrée}
\vedette{\hypertarget{Ⓔnɯɣɯʑɴɢu}{\papi{ nɯɣɯʑɴɢu}}}\markboth{nɯɣɯʑɴɢu}{}
\begin{relation-sémantique}\confer{
\hyperlink{Ⓔʑɴɢu}{\textit{ \papi{ʑɴɢu}}}
}\end{relation-sémantique}\end{entrée}

\begin{entrée}
\vedette{\hypertarget{Ⓔnɯɣʑɯr}{\papi{ nɯɣʑɯr}}}\markboth{nɯɣʑɯr}{}
\classe{vi}
\paradigme{\textit{dir :} \jya tɤ-}\acception{1}
\begin{définition}\fra être en état d'alerte\end{définition}
\begin{définition}\cmn 警惕;害怕出事\end{définition}
\begin{exemple}\jya ɲɯ-nɯɣʑɯr-a ɕti ma kɯmaʁ ɯβrɤ-fse ma ɲɯ-sɯsam-a\cmn 我怕会出事\end{exemple}\acception{2}
\begin{définition}\fra ne pas oser (manger)\end{définition}
\begin{définition}\cmn 不好意思(吃)\end{définition}
\begin{exemple}\jya nɯnɯ ndʐuwa kɤ-rɯndzɤtshi mɯ́j-cha ma ɲɯ-nɯɣʑɯr\cmn 客人不敢吃,不好意思吃\end{exemple}
\begin{relation-sémantique}\synonyme{
\hyperlink{Ⓔraʁle}{\textit{ \papi{raʁle}}}
}\end{relation-sémantique}\begin{sous-entrée}
\vedette{\hypertarget{}{\papi{ znɯɣʑɯr}}}\markboth{znɯɣʑɯr}{}\classe{vt}
\paradigme{\textit{dir :} \jya tɤ-}\acception{1}
\begin{définition}\fra embarrasser\end{définition}
\begin{définition}\cmn 令人不好意思\end{définition}
\begin{exemple}\jya nɯ kɯ-fse ma-tɤ-tɯ-ti ma tɯ-znɯɣʑɯr\cmn 你不要说那些话,令他不好意思\end{exemple}\acception{2}
\begin{définition}\fra faire peur\end{définition}
\begin{définition}\cmn 令人觉得危险;令人担心会出事\end{définition}
\begin{exemple}\jya nɯ kɯ-fse paχɕi chɯ-tɯ-βʑoʁ tɕe ɲɯ-kɯ-znɯɣʑɯr-a\cmn 你这样削苹果,令我害怕会出事\end{exemple}
\end{sous-entrée}\end{entrée}

\begin{entrée}
\vedette{\hypertarget{Ⓔnɯhɯɲi}{\papi{ nɯhɯɲi}}}\markboth{nɯhɯɲi}{}\classe{vt}
\begin{définition}\fra travailler\end{définition}
\begin{définition}\cmn 打工
\begin{déclaration} \étymologie{\papi{\stylefn{副业}}}\end{déclaration}\end{définition}
\end{entrée}

\begin{entrée}
\vedette{\hypertarget{Ⓔnɯjaŋsa}{\papi{ nɯjaŋsa}}}\markboth{nɯjaŋsa}{}\classe{vi}
\begin{définition}\fra oisif\end{définition}
\begin{définition}\cmn 闲着\end{définition}
\begin{exemple}\jya aʑo ɕ-pɯ-nɯjaŋsa-a ɕti wo\cmn 我去那边闲逛\end{exemple}\end{entrée}

\begin{entrée}
\vedette{\hypertarget{Ⓔnɯjɤntɤn}{\papi{ nɯjɤntɤn}}}\markboth{nɯjɤntɤn}{}\classe{vt}
\paradigme{\textit{dir :} \jya pɯ-}
\begin{définition}\ 
\begin{déclaration}\grammar{denom}\end{déclaration}\end{définition}
\begin{définition}\fra avoir comme passion\end{définition}
\begin{définition}\cmn 有这个爱好
\begin{déclaration} \étymologie{\papi{jon.tan}}\end{déclaration}\end{définition}
\begin{exemple}\jya kɤ-rɤβzjoz ntsɯ pjɯ-tɯ-nɯjɤntɤn ŋu\cmn 你很专心地学习\end{exemple}
\begin{relation-sémantique}\confer{
\hyperlink{Ⓔjɤntɤn}{\textit{ \papi{jɤntɤn}}}
}\end{relation-sémantique}\end{entrée}

\begin{entrée}
\vedette{\hypertarget{Ⓔnɯjɣa}{\papi{ nɯjɣa}}}\markboth{nɯjɣa}{}
\classe{vi}
\paradigme{\textit{dir :} \jya tɤ-}
\begin{définition}\fra dire toujours oui\end{définition}
\begin{définition}\cmn 总是答应别人\end{définition}
\begin{exemple}\jya kɯ-nɯjɣa ci ɲɯ-ŋu\cmn 他是一个总是答应别人的人\end{exemple}
\begin{relation-sémantique}\confer{
\hyperlink{Ⓔɣa}{\textit{ \papi{ɣa}}}
}\end{relation-sémantique}\end{entrée}

\begin{entrée}
\vedette{\hypertarget{Ⓔnɯjlɤlɤɣ}{\papi{ nɯjlɤlɤɣ}}}\markboth{nɯjlɤlɤɣ}{}\classe{vi}
\paradigme{\textit{dir :} \jya nɯ-}
\paradigme{\textit{dir :} \jya pɯ-}
\begin{définition}\ 
\begin{déclaration}\grammar{incorp}\end{déclaration}\end{définition}
\begin{définition}\fra faire paître un yak hybride\end{définition}
\begin{définition}\cmn 放犏牛\end{définition}
\begin{exemple}\jya nɯ-nɯjlɤlaɣ-a\cmn 我放了犏牛\end{exemple}
\begin{exemple}\jya kɯ-nɯjlɤlɤɣ lɤ-ari-a\cmn 我去放犏牛了\end{exemple}
\begin{exemple}\jya kɯ-nɯjlɤlɤɣ lo-ɕe\end{exemple}
\begin{relation-sémantique}\confer{
\hyperlink{Ⓔjla}{\textit{ \papi{jla}}}
}\end{relation-sémantique}
\begin{relation-sémantique}\confer{
\hyperlink{Ⓔlɤɣ}{\textit{ \papi{lɤɣ}}}
}\end{relation-sémantique}\end{entrée}

\begin{entrée}
\vedette{\hypertarget{Ⓔnɯjlɤmtshi}{\papi{ nɯjlɤmtshi}}}\markboth{nɯjlɤmtshi}{}
\classe{vi}
\begin{définition}\ 
\begin{déclaration}\grammar{incorp}\end{déclaration}\end{définition}
\begin{définition}\fra mener un yak hybride (pendant le labour)\end{définition}
\begin{définition}\cmn 牵犏牛(耕地的时候)\end{définition}
\begin{exemple}\jya tɤ-nɯjlɤmtshi-a\cmn 我牵了犏牛\end{exemple}
\begin{exemple}\jya tɤ-pɤtso pɯ-ŋu-a tɕe, kɤ-nɯjlɤmtshi pɯ-rɲo-t-a\cmn 我小的时候,我曾经牵过犏牛\end{exemple}
\begin{relation-sémantique}\confer{
\hyperlink{Ⓔjlɤmtshi}{\textit{ \papi{jlɤmtshi}}}
}\end{relation-sémantique}\end{entrée}

\begin{entrée}
\vedette{\hypertarget{ⒺnɯjlɤndzaⒽ1}{\papi{ nɯjlɤndza}}}\markboth{nɯjlɤndza}{}\homonyme{1} (\variante{nɤjlɤndza}) \classe{vt}
\paradigme{\textit{dir :} \jya thɯ-}
\paradigme{\textit{dir :} \jya tɤ-}
\begin{définition}\fra aimer manger des petites collations\end{définition}
\begin{définition}\cmn 爱吃零食\end{définition}
\begin{exemple}\jya ɲɯ-ɤz-nɯjlɤndza\cmn 他在吃零食\end{exemple}
\begin{exemple}\jya nɤʑo kɯchi ɲɯ-tɯ-ɤz-nɯjlɤndza\cmn 你在吃糖\end{exemple}
\begin{exemple}\jya @guazi thɯ-nɯjlɤndza-t-a\cmn 我吃了瓜子\end{exemple}
\begin{exemple}\jya kɯchi tɤ-nɯjlɤndza-t-a\cmn 我吃了糖\end{exemple}
\begin{relation-sémantique}\confer{
\hyperlink{Ⓔndza}{\textit{ \papi{ndza}}}
}\end{relation-sémantique}\end{entrée}

\begin{entrée}
\vedette{\hypertarget{ⒺnɯjlɤndzaⒽ2}{\papi{ nɯjlɤndza}}}\markboth{nɯjlɤndza}{}\homonyme{2}\classe{vi}
\begin{définition}\fra couper de l'herbe pour les yaks hybrides\end{définition}
\begin{définition}\cmn 割牛草\end{définition}
\begin{exemple}\jya kɯ-nɯjlɤndza jɤ-ari-a\cmn 我去割牛草了\end{exemple}
\begin{relation-sémantique}\synonyme{
\hyperlink{Ⓔɣɤxtɕɤβ}{\textit{ \papi{ɣɤxtɕɤβ}}}
}\end{relation-sémantique}\end{entrée}

\begin{entrée}
\vedette{\hypertarget{Ⓔnɯjmɤzdɤβ}{\papi{ nɯjmɤzdɤβ}}}\markboth{nɯjmɤzdɤβ}{}
\classe{vi}
\paradigme{\textit{dir :} \jya tɤ-}
\begin{définition}\ 
\begin{déclaration}\grammar{incorp}\end{déclaration}\end{définition}
\begin{définition}\fra dormir dans une direction inverse\end{définition}
\begin{définition}\cmn 打脚蹬(朝相反的方向睡,交叉着脚)\end{définition}
\begin{exemple}\jya ʑɤni to-nɯjmɤzdɤβ-ndʑi\cmn 他们俩朝相反的方向睡了\end{exemple}
\begin{relation-sémantique}\confer{
\hyperlink{Ⓔzdɤβ}{\textit{ \papi{zdɤβ}}}
}\end{relation-sémantique}\end{entrée}

\begin{entrée}
\vedette{\hypertarget{Ⓔnɯjmŋo}{\papi{ nɯjmŋo}}}\markboth{nɯjmŋo}{}
\classe{vi}
\paradigme{\textit{dir :} \jya pɯ-}
\begin{définition}\ 
\begin{déclaration}\grammar{denom}\end{déclaration}\end{définition}
\begin{définition}\fra être l'objet du rêve de quelqu'un\end{définition}
\begin{définition}\cmn 出现在别人的梦中
\begin{déclaration}\use{表示好兆头}\end{déclaration}\end{définition}
\begin{exemple}\jya jɯfɕɯɕɤr pɯ-ta-ɣɤjmŋo tɕe ɲɯ-tɯ-nɯjmŋo\cmn 昨天我梦见你了(是好兆头)\end{exemple}
\begin{exemple}\jya mɯ́j-nɯjmŋo-nɯ\cmn 梦见他们是不好的兆头\end{exemple}
\begin{relation-sémantique}\confer{
\hyperlink{Ⓔtɯ-jmŋo}{\textit{ \papi{tɯ-jmŋo}}}
}\end{relation-sémantique}
\begin{relation-sémantique}\confer{
\hyperlink{Ⓔɣɤjmŋo}{\textit{ \papi{ɣɤjmŋo}}}
}\end{relation-sémantique}\end{entrée}

\begin{entrée}
\vedette{\hypertarget{Ⓔnɯjroʁ}{\papi{ nɯjroʁ}}}\markboth{nɯjroʁ}{}\classe{vt}
\paradigme{\textit{dir :} \jya tɤ-}
\begin{définition}\ 
\begin{déclaration}\grammar{denom}\end{déclaration}\end{définition}
\begin{définition}\fra suivre à la trace\end{définition}
\begin{définition}\cmn 追踪\end{définition}
\begin{exemple}\jya jɯfɕɯr, pri ci tɤ-nɯɕɤmɯɣdɯ-t-a tɕe jɤ-anɯri tɕe tɤ-nɯjroʁ-a\cmn 昨天我射了一头熊,它逃走了,我追踪了它\end{exemple}
\begin{relation-sémantique}\confer{
\hyperlink{Ⓔtɤ-jroʁ}{\textit{ \papi{tɤ-jroʁ}}}
}\end{relation-sémantique}
\begin{relation-sémantique}\confer{
\hyperlink{Ⓔrɤjroʁ}{\textit{ \papi{rɤjroʁ}}}
}\end{relation-sémantique}\end{entrée}

\begin{entrée}
\vedette{\hypertarget{Ⓔnɯjʁo}{\papi{ nɯjʁo}}}\markboth{nɯjʁo}{}
\classe{vi}
\paradigme{\textit{dir :} \jya tɤ-}
\begin{définition}\fra insulter, gronder\end{définition}
\begin{définition}\cmn 骂\end{définition}
\begin{exemple}\jya ɲɯ-tɯ-nɯjʁo\cmn 你在骂人\end{exemple}
\begin{exemple}\jya tɤ-nɯjʁo-a\cmn 我骂了他\end{exemple}
\begin{exemple}\jya ma-tɤ-tɯ-nɯjʁo\cmn 你不要骂(我)\end{exemple}\begin{sous-entrée}
\vedette{\hypertarget{}{\papi{ nɯjʁojʁe}}}\markboth{nɯjʁojʁe}{}\classe{vi}
\paradigme{\textit{dir :} \jya tɤ-}
\begin{définition}\fra insulter\end{définition}
\begin{définition}\cmn 乱骂\end{définition}
\begin{relation-sémantique}\synonyme{
\hyperlink{Ⓔnɤmqe}{\textit{ \papi{nɤmqe}}}
}\end{relation-sémantique}
\end{sous-entrée}\end{entrée}

\begin{entrée}
\vedette{\hypertarget{Ⓔnɯjʁojʁe}{\papi{ nɯjʁojʁe}}}\markboth{nɯjʁojʁe}{}
\begin{relation-sémantique}\confer{
\hyperlink{Ⓔnɯjʁo}{\textit{ \papi{nɯjʁo}}}
}\end{relation-sémantique}\end{entrée}

\begin{entrée}
\vedette{\hypertarget{Ⓔnɯjtʂhɤβ}{\papi{ nɯjtʂhɤβ}}}\markboth{nɯjtʂhɤβ}{}
\classe{vt}
\paradigme{\textit{dir :} \jya pɯ-}\acception{1}
\begin{définition}\fra érafler avec force\end{définition}
\begin{définition}\cmn 使劲地刮\end{définition}
\begin{exemple}\jya ɯ-ŋga ra pjɯ-nɯjtʂhɤβ ʑo ju-rɤɕi pjɤ-ŋu\cmn 把它的衣服使劲地刮破了\end{exemple}\acception{2}
\begin{définition}\fra abîmer en griffant\end{définition}
\begin{définition}\cmn 抓烂\end{définition}
\begin{exemple}\jya lɯlu tɯ-mɯrʁɯz kɯ pjɯ-kɯ-nɯjtʂhɤβ ʑo ŋgrɤl\cmn 猫会在身上乱抓(把衣服和皮肤抓烂)\end{exemple}\end{entrée}

\begin{entrée}
\vedette{\hypertarget{Ⓔnɯɟɯɣɟɯɣ}{\papi{ nɯɟɯɣɟɯɣ}}}\markboth{nɯɟɯɣɟɯɣ}{}
\begin{relation-sémantique}\confer{
\hyperlink{Ⓔɟɯɣɟɯɣ}{\textit{ \papi{ɟɯɣɟɯɣ}}}
}\end{relation-sémantique}\end{entrée}

\begin{entrée}
\vedette{\hypertarget{Ⓔnɯkɤntɕhaʁ}{\papi{ nɯkɤntɕhaʁ}}}\markboth{nɯkɤntɕhaʁ}{}
\classe{vi}
\paradigme{\textit{dir :} \jya pɯ-}
\begin{définition}\ 
\begin{déclaration}\grammar{denom}\end{déclaration}\end{définition}
\begin{définition}\fra aller dans la rue\end{définition}
\begin{définition}\cmn 上街\end{définition}
\begin{exemple}\jya jiɕqha jiʑora @jieshang ɕ-pɯ-nɯkɤntɕhaʁ-i\cmn 我们刚才上街了\end{exemple}
\begin{exemple}\jya @chenlaoshi cho jiʑora pɯ-nɯkɤntɕhaʁ-i\cmn 我们跟陈老师上街了\end{exemple}
\begin{relation-sémantique}\confer{
\hyperlink{Ⓔkɤntɕhaʁ}{\textit{ \papi{kɤntɕhaʁ}}}
}\end{relation-sémantique}\end{entrée}

\begin{entrée}
\vedette{\hypertarget{Ⓔnɯkɤrŋi}{\papi{ nɯkɤrŋi}}}\markboth{nɯkɤrŋi}{}\classe{vi}
\paradigme{\textit{dir :} \jya nɯ-}
\paradigme{\textit{dir :} \jya pɯ-}
\begin{définition}\ 
\begin{déclaration}\grammar{denom}\end{déclaration}\end{définition}
\begin{définition}\fra aller chercher des herbes sauvages\end{définition}
\begin{définition}\cmn 去采集野菜\end{définition}
\begin{exemple}\jya sɯŋgɯ ʑ-nɯ-nɯkɤrŋi-a\cmn 我到森林去采集野菜了\end{exemple}
\begin{relation-sémantique}\confer{
\hyperlink{Ⓔarŋi}{\textit{ \papi{arŋi}}}
}\end{relation-sémantique}\end{entrée}

\begin{entrée}
\vedette{\hypertarget{Ⓔnɯkhamu}{\papi{ nɯkhamu}}}\markboth{nɯkhamu}{}\classe{vi}
\paradigme{\textit{dir :} \jya tɤ-}
\begin{définition}\ 
\begin{déclaration}\grammar{denom}\end{déclaration}\end{définition}
\begin{définition}\fra faire à manger\end{définition}
\begin{définition}\cmn 做饭\end{définition}
\begin{exemple}\jya jɯfɕɯr pɯ-nɯkhamu-a\cmn 我昨天做了饭\end{exemple}
\begin{exemple}\jya jisŋi nɤʑo tɤ-nɯkhamu\cmn 你今天做饭吧\end{exemple}
\begin{exemple}\jya aʑo nɯkhamu-a ra\cmn 我要做饭了\end{exemple}
\begin{relation-sémantique}\confer{
\hyperlink{Ⓔkhamu}{\textit{ \papi{khamu}}}
}\end{relation-sémantique}\end{entrée}

\begin{entrée}
\vedette{\hypertarget{Ⓔnɯkhaŋrcɤl}{\papi{ nɯkhaŋrcɤl}}}\markboth{nɯkhaŋrcɤl}{}\classe{vi}
\paradigme{\textit{dir :} \jya thɯ-}
\begin{définition}\fra les quatre fers en l'air\end{définition}
\begin{définition}\cmn 四脚朝天\end{définition}\begin{sous-entrée}
\vedette{\hypertarget{}{\papi{ znɯkhaŋrcɤl}}}\markboth{znɯkhaŋrcɤl}{}\classe{vt}
\paradigme{\textit{dir :} \jya thɯ-}
\begin{définition}\fra mettre les quatre fers en l'air\end{définition}
\begin{définition}\cmn 让……四脚朝天\end{définition}
\begin{exemple}\jya ɯʑo pɯ-znɤjpɯjpe ri, ɯʑo sɤz kɯ-cha ra jo-ɣi-nɯ tɕe chɤ́-wɣ-znɯkhaŋrcɤl ɕti\cmn 他以前很傲慢,但是他遇见了比自己厉害的人,挫了他的傲气\end{exemple}
\end{sous-entrée}\end{entrée}

\begin{entrée}
\vedette{\hypertarget{Ⓔnɯkhaŋχɯ}{\papi{ nɯkhaŋχɯ}}}\markboth{nɯkhaŋχɯ}{}
\classe{vi}
\paradigme{\textit{dir :} \jya thɯ-}
\begin{définition}\fra être accroupi\end{définition}
\begin{définition}\cmn 蹲\end{définition}
\begin{exemple}\jya ma-thɯ-tɯ-nɯkhaŋχɯ\cmn 你不要蹲下(没有礼貌)\end{exemple}\end{entrée}

\begin{entrée}
\vedette{\hypertarget{Ⓔnɯkharwut}{\papi{ nɯkharwut}}}\markboth{nɯkharwut}{}
\classe{vi}
\paradigme{\textit{dir :} \jya tɤ-}
\begin{définition}\ 
\begin{déclaration}\grammar{denom}\end{déclaration}\end{définition}
\begin{définition}\fra avoir la fièvre aphteuse\end{définition}
\begin{définition}\cmn 得口蹄疫\end{définition}
\begin{exemple}\jya jla ɲɯ-nɯkharwut\cmn 犏牛有口蹄疫\end{exemple}
\begin{relation-sémantique}\confer{
\hyperlink{Ⓔkharwut}{\textit{ \papi{kharwut}}}
}\end{relation-sémantique}\end{entrée}

\begin{entrée}
\vedette{\hypertarget{Ⓔnɯkhɤβdɤr}{\papi{ nɯkhɤβdɤr}}}\markboth{nɯkhɤβdɤr}{}
\classe{vt}
\paradigme{\textit{dir :} \jya tɤ-}
\begin{définition}\ 
\begin{déclaration}\grammar{denom}\end{déclaration}\end{définition}
\begin{définition}\fra dire des blagues\end{définition}
\begin{définition}\cmn 开玩笑;讲笑话\end{définition}
\begin{exemple}\jya jɯfɕɯr tɤ-kɯ-nɯkhɤβdar-a\cmn 你昨天跟我开了玩笑\end{exemple}
\begin{exemple}\jya tɤ-ta-nɯkhɤβdɤr ɕti ma a-stu maʁ\cmn 我只跟你开了玩笑\end{exemple}
\begin{relation-sémantique}\confer{
\hyperlink{Ⓔkhɤβdɤr}{\textit{ \papi{khɤβdɤr}}}
}\end{relation-sémantique}\end{entrée}

\begin{entrée}
\vedette{\hypertarget{Ⓔnɯkhɤβɣa}{\papi{ nɯkhɤβɣa}}}\markboth{nɯkhɤβɣa}{}\classe{vi}
\begin{définition}\fra ne pas rester en place, aller à droite et à gauche (oiseau)\end{définition}
\begin{définition}\cmn 在地面来回转动(鸟)\end{définition}
\begin{exemple}\jya qro ɲɯ-nɯkhɤβɣa\cmn 鸽子在来回转动\end{exemple}
\begin{exemple}\jya nɤ-stu ku-kɤ-rɤʑi maŋe tɕe qro kɯ-nɯkhɤβɣa ʑo ɲɯ-tɯ-fse\cmn 你坐好,不要坐立不安\end{exemple}\end{entrée}

\begin{entrée}
\vedette{\hypertarget{Ⓔnɯkhɤda}{\papi{ nɯkhɤda}}}\markboth{nɯkhɤda}{}
\classe{vt}
\paradigme{\textit{dir :} \jya nɯ-}
\begin{définition}\fra convaincre, calmer, raisonner qqn\end{définition}
\begin{définition}\cmn 劝说\end{définition}
\begin{exemple}\jya ɲɯ-tɯ-anɯmqaj-ndʑi tɕe nɯ-ta-nɯkhɤda\cmn 你们俩吵架了,我劝了你一下了\end{exemple}
\begin{relation-sémantique}\antonyme{
\hyperlink{Ⓔɣɤɕphɤr}{\textit{ \papi{ɣɤɕphɤr}}}
}\end{relation-sémantique}\end{entrée}

\begin{entrée}
\vedette{\hypertarget{Ⓔnɯkhɤja}{\papi{ nɯkhɤja}}}\markboth{nɯkhɤja}{}
\classe{vl}
\paradigme{\textit{dir :} \jya tɤ-}
\begin{définition}\fra résister\end{définition}
\begin{définition}\cmn 顶嘴\end{définition}
\begin{exemple}\jya a-mu tɤ-nɯkhɤja-t-a\cmn 我跟我母亲顶嘴了\end{exemple}
\begin{exemple}\jya a-wa tɤ-nɯkhɤja-t-a\cmn 我跟我父亲顶嘴了\end{exemple}
\begin{exemple}\jya nɤʑo ɲɯ-tɯ-nɯkhɤja\cmn 你在顶嘴\end{exemple}
\begin{exemple}\jya nɤʑo ma-tɤ-kɯ-nɯkhɤja-a\cmn 你不要跟我顶嘴\end{exemple}
\begin{relation-sémantique}\synonyme{
\hyperlink{Ⓔnɯɣɤja}{\textit{ \papi{nɯɣɤja}}}
}\end{relation-sémantique}\end{entrée}

\begin{entrée}
\vedette{\hypertarget{Ⓔnɯkhɤjlɤn}{\papi{ nɯkhɤjlɤn}}}\markboth{nɯkhɤjlɤn}{}
\classe{vt}
\paradigme{\textit{dir :} \jya tɤ-}
\begin{définition}\ 
\begin{déclaration}\grammar{denom}\end{déclaration}\end{définition}
\begin{définition}\fra faire un souhait\end{définition}
\begin{définition}\cmn 许愿\end{définition}
\begin{exemple}\jya tɤ-nɯkhɤjlan-a\cmn 我许愿了\end{exemple}
\begin{exemple}\jya ɬasa tu-ɕe-a nɯ-sɯso-t-a ri, mɯ-pɯ-ŋgrɯ tɕe tɤ-nɯkhɤjlan-a\cmn 我想过要去拉萨,没有去成,但是我许了愿将来一定会去\end{exemple}
\begin{relation-sémantique}\confer{
\hyperlink{Ⓔkhɤjlɤn}{\textit{ \papi{khɤjlɤn}}}
}\end{relation-sémantique}\end{entrée}

\begin{entrée}
\vedette{\hypertarget{Ⓔnɯkhɤlɤmdzɯmdzɯ}{\papi{ nɯkhɤlɤmdzɯmdzɯ}}}\markboth{nɯkhɤlɤmdzɯmdzɯ}{}
\classe{vi}
\paradigme{\textit{dir :} \jya pɯ-}
\paradigme{\textit{dir :} \jya thɯ-}
\begin{définition}\fra s'accroupir\end{définition}
\begin{définition}\cmn 蹲\end{définition}
\begin{exemple}\jya ɲɯ-tɯ-nɯkhɤlɤmdzɯmdzɯ\cmn 你是蹲着的\end{exemple}
\begin{relation-sémantique}\synonyme{
\hyperlink{Ⓔnɯkhaŋχɯ}{\textit{ \papi{nɯkhaŋχɯ}}}
}\end{relation-sémantique}
\begin{relation-sémantique}\confer{
\hyperlink{Ⓔamdzɯ}{\textit{ \papi{amdzɯ}}}
}\end{relation-sémantique}\end{entrée}

\begin{entrée}
\vedette{\hypertarget{Ⓔnɯkhɤphrɯ}{\papi{ nɯkhɤphrɯ}}}\markboth{nɯkhɤphrɯ}{}
\classe{vt}
\paradigme{\textit{dir :} \jya tɤ-}
\begin{définition}\fra asperger (avec la bouche)\end{définition}
\begin{définition}\cmn 喷(水)
\end{définition}
\begin{exemple}\jya tɯ-ndʐi to-khrɯ tɕe, tɤ-nɯkhɤphrɯ-t-a tɕe nɯ-ɣɤla-t-a\cmn 皮子干了,我喷了一下口水就令它湿润\end{exemple}
\begin{relation-sémantique}\confer{
\hyperlink{Ⓔkhɤphrɯ}{\textit{ \papi{khɤphrɯ}}}
}\end{relation-sémantique}\end{entrée}

\begin{entrée}
\vedette{\hypertarget{Ⓔnɯkhɤrŋgɯ}{\papi{ nɯkhɤrŋgɯ}}}\markboth{nɯkhɤrŋgɯ}{}\classe{vi}
\paradigme{\textit{dir :} \jya lɤ-}
\begin{définition}\ 
\begin{déclaration}\grammar{incorp}\end{déclaration}\end{définition}
\begin{définition}\fra s'allonger n'importe où pour se reposer\end{définition}
\begin{définition}\cmn 随便躺在某个地方休息\end{définition}
\begin{relation-sémantique}\confer{
\hyperlink{Ⓔkhɤjmu}{\textit{ \papi{khɤjmu}}}
}\end{relation-sémantique}
\begin{relation-sémantique}\confer{
\hyperlink{ⒺrŋgɯⒽ1}{\textit{ \papi{rŋgɯ}}}
}\end{relation-sémantique}\end{entrée}

\begin{entrée}
\vedette{\hypertarget{Ⓔnɯkhɤsnɯm}{\papi{ nɯkhɤsnɯm}}}\markboth{nɯkhɤsnɯm}{}\classe{vt}
\begin{définition}\fra mouiller avec de la salive\end{définition}
\begin{définition}\cmn 用口水弄湿\end{définition}
\begin{relation-sémantique}\confer{
\hyperlink{Ⓔkhɤsnɯm}{\textit{ \papi{khɤsnɯm}}}
}\end{relation-sémantique}\end{entrée}

\begin{entrée}
\vedette{\hypertarget{Ⓔnɯkho}{\papi{ nɯkho}}}\markboth{nɯkho}{}
\classe{vi}
\paradigme{\textit{dir :} \jya tɤ-}
\begin{définition}\ 
\begin{déclaration}\grammar{denom}\end{déclaration}\end{définition}
\begin{définition}\fra passer la nuit chez quelqu'un\end{définition}
\begin{définition}\cmn 借宿\end{définition}
\begin{exemple}\jya jɯfɕɯr ji-kɯ-nɯkho ʁnɯz pɯ-tu, χsɯm pɯ-tu\cmn 昨天我们家有两三个客人\end{exemple}
\begin{exemple}\jya tɤ-nɯkho-a\cmn 我在他家借宿了\end{exemple}
\begin{relation-sémantique}\confer{
\hyperlink{ⒺkhoⒽ1}{\textit{ \papi{kho}}}
}\end{relation-sémantique}\begin{sous-entrée}
\vedette{\hypertarget{}{\papi{ znɯkho}}}\markboth{znɯkho}{}\classe{vt}
\paradigme{\textit{dir :} \jya tɤ-}
\begin{définition}\ 
\begin{déclaration}\grammar{caus}\end{déclaration}\end{définition}
\begin{définition}\fra inviter chez soi pour la nuit\end{définition}
\begin{définition}\cmn 请人留宿\end{définition}
\begin{exemple}\jya tɤ-znɯkho-t-a\cmn 我请他留宿了\end{exemple}
\begin{exemple}\jya aʑo ci tɯ-rʑaʁ tu-kɯ-znɯkho-a-nɯ ɯ́-jɤɣ?\cmn 请问能否让我借宿一晚?\end{exemple}
\end{sous-entrée}\end{entrée}

\begin{entrée}
\vedette{\hypertarget{Ⓔnɯkhramba}{\papi{ nɯkhramba}}}\markboth{nɯkhramba}{}
\classe{vt}
\paradigme{\textit{dir :} \jya tɤ-}
\begin{définition}\ 
\begin{déclaration}\grammar{denom}\end{déclaration}\end{définition}
\begin{définition}\fra tromper, mentir à quelqu'un\end{définition}
\begin{définition}\cmn 欺骗;撒谎\end{définition}
\begin{exemple}\jya jɯfɕɯr tɤ-ta-nɯkhramba\cmn 我昨天骗了你\end{exemple}
\begin{exemple}\jya nɤʑo ma-tɤ-kɯ-nɯkhramba-a\cmn 你不要骗我\end{exemple}\begin{sous-entrée}
\vedette{\hypertarget{}{\papi{ sɤnɯkhramba}}}\markboth{sɤnɯkhramba}{}\classe{vi}
\paradigme{\textit{dir :} \jya tɤ-}
\begin{définition}\ 
\begin{déclaration}\grammar{apass}\end{déclaration}\end{définition}
\begin{définition}\fra tromper les gens\end{définition}
\begin{définition}\cmn 骗别人\end{définition}
\end{sous-entrée}\begin{sous-entrée}
\vedette{\hypertarget{}{\papi{ ʑɣɤnɯkhramba}}}\markboth{ʑɣɤnɯkhramba}{}\classe{vi}
\paradigme{\textit{dir :} \jya tɤ-}
\begin{définition}\ 
\begin{déclaration}\grammar{refl}\end{déclaration}\end{définition}
\begin{définition}\fra être trompé\end{définition}
\begin{définition}\cmn 被骗\end{définition}
\begin{relation-sémantique}\confer{
\hyperlink{Ⓔkhramba}{\textit{ \papi{khramba}}}
}\end{relation-sémantique}
\begin{relation-sémantique}\confer{
\hyperlink{Ⓔrɯkhramba}{\textit{ \papi{rɯkhramba}}}
}\end{relation-sémantique}
\end{sous-entrée}\end{entrée}

\begin{entrée}
\vedette{\hypertarget{Ⓔnɯkhrɯɣ}{\papi{ nɯkhrɯɣ}}}\markboth{nɯkhrɯɣ}{}
\classe{vt}
\paradigme{\textit{dir :} \jya nɯ-}
\begin{définition}\fra accrocher et déchirer\end{définition}
\begin{définition}\cmn 钩住;钩破\end{définition}
\begin{exemple}\jya kɤ-tɯ-ari nɤ ɲɯ-kɯ-nɯkhrɯɣ-a, nɯ-tɯ-ɣe nɤ ɲɯ-kɯ-nɯkhrɯɣ-a\cmn 你过去就把我的衣服钩破,过来也把我的衣服钩破\end{exemple}
\begin{exemple}\jya si kɯ a-ŋga na-nɯkhrɯɣ\cmn 我的衣服被树钩破了\end{exemple}\end{entrée}

\begin{entrée}
\vedette{\hypertarget{Ⓔnɯkhrɯm}{\papi{ nɯkhrɯm}}}\markboth{nɯkhrɯm}{}
\classe{vi}
\paradigme{\textit{dir :} \jya kɤ-}
\begin{définition}\ 
\begin{déclaration}\grammar{denom}\end{déclaration}\end{définition}\acception{1}
\begin{définition}\fra être puni, recevoir un châtiment\end{définition}
\begin{définition}\cmn 受罚\end{définition}\acception{2}
\begin{définition}\fra aller en prison\end{définition}
\begin{définition}\cmn 坐牢
\begin{déclaration} \étymologie{\papi{kʰrims}}\end{déclaration}\end{définition}
\begin{exemple}\jya ɲɯ-nɯkhrɯm-nɯ\cmn 他们在坐牢\end{exemple}\begin{sous-entrée}
\vedette{\hypertarget{}{\papi{ sɤznɯkhrɯm}}}\markboth{sɤznɯkhrɯm}{}\classe{vi}
\begin{définition}\fra infliger des punitions\end{définition}
\begin{définition}\cmn 惩罚人\end{définition}
\begin{exemple}\jya kɯ-znɯkhrɯm\cmn 刽子手\end{exemple}
\end{sous-entrée}\begin{sous-entrée}
\vedette{\hypertarget{}{\papi{ znɯkhrɯm}}}\markboth{znɯkhrɯm}{}\classe{vt}
\paradigme{\textit{dir :} \jya kɤ-}
\begin{définition}\ 
\begin{déclaration}\grammar{caus}\end{déclaration}\end{définition}
\begin{définition}\fra infliger une punition, châtier\end{définition}
\begin{définition}\cmn 惩罚,用刑\end{définition}
\end{sous-entrée}\end{entrée}

\begin{entrée}
\vedette{\hypertarget{Ⓔnɯkhɯ}{\papi{ nɯkhɯ}}}\markboth{nɯkhɯ}{}
\begin{relation-sémantique}\confer{
\hyperlink{ⒺkhɯⒽ1}{\textit{ \papi{khɯ1}}}
}\end{relation-sémantique}\end{entrée}

\begin{entrée}
\vedette{\hypertarget{Ⓔnɯkhɯɣ}{\papi{ nɯkhɯɣ}}}\markboth{nɯkhɯɣ}{}\classe{vt}
\paradigme{\textit{div :} \jya kɤ-}
\begin{définition}\fra boire à longs traits\end{définition}
\begin{définition}\cmn 大口大口地喝(很急的样子)\end{définition}
\begin{exemple}\jya tɯ-ci ko-nɯkhɯɣ ʑo ko-tshi\cmn 他大口大口地喝了水\end{exemple}
\begin{relation-sémantique}\synonyme{
\hyperlink{Ⓔnɯchɯβ}{\textit{ \papi{nɯchɯβ}}}
}\end{relation-sémantique}\end{entrée}

\begin{entrée}
\vedette{\hypertarget{Ⓔnɯkhɯr}{\papi{ nɯkhɯr}}}\markboth{nɯkhɯr}{}
\classe{vt}
\paradigme{\textit{dir :} \jya pɯ-}
\begin{définition}\ 
\begin{déclaration}\grammar{denom}\end{déclaration}\end{définition}
\begin{définition}\fra commander, gérer\end{définition}
\begin{définition}\cmn 担当;管理
\begin{déclaration} \étymologie{\papi{kʰur}}\end{déclaration}\end{définition}
\begin{exemple}\jya aʑo @daduizhang pɯ-az-nɯkhɯr-a\cmn 我以前当大队长\end{exemple}\end{entrée}

\begin{entrée}
\vedette{\hypertarget{Ⓔnɯkhɯrthaŋ}{\papi{ nɯkhɯrthaŋ}}}\markboth{nɯkhɯrthaŋ}{}
\classe{vi}
\paradigme{\textit{dir :} \jya tɤ-}
\begin{définition}\ 
\begin{déclaration}\grammar{denom}\end{déclaration}\end{définition}
\begin{définition}\fra occuper un poste\end{définition}
\begin{définition}\cmn 当官\end{définition}
\begin{exemple}\jya χsɯ-xpa pɯ-nɯkhɯrthaŋ-a\cmn 我当了三年官\end{exemple}\end{entrée}

\begin{entrée}
\vedette{\hypertarget{Ⓔnɯkhɯrwum}{\papi{ nɯkhɯrwum}}}\markboth{nɯkhɯrwum}{}\classe{vi}
\paradigme{\textit{dir :} \jya nɯ-}
\begin{définition}\ 
\begin{déclaration}\grammar{denom}\end{déclaration}\end{définition}
\begin{définition}\fra moisir\end{définition}
\begin{définition}\cmn 发霉\end{définition}
\begin{exemple}\jya kɤ-ndza ɲɤ-nɯkhɯrwum\cmn 食物发霉了\end{exemple}
\begin{exemple}\jya nɯ-kɯ-nɯkhɯrwum kɤ-ndza mɤ-sna\cmn 发霉的食物不能吃\end{exemple}
\begin{relation-sémantique}\confer{
\hyperlink{Ⓔkhɯrwum}{\textit{ \papi{khɯrwum}}}
}\end{relation-sémantique}\end{entrée}

\begin{entrée}
\vedette{\hypertarget{Ⓔnɯkumbrɤl}{\papi{ nɯkumbrɤl}}}\markboth{nɯkumbrɤl}{}
\classe{vi}
\paradigme{\textit{dir :} \jya pɯ-}
\begin{définition}\fra jouer aux échecs\end{définition}
\begin{définition}\cmn 下棋\end{définition}
\begin{exemple}\jya pjɤ-nɯkumbrɤl\cmn 他下棋了\end{exemple}
\begin{relation-sémantique}\confer{
\hyperlink{Ⓔkumbrɤl}{\textit{ \papi{kumbrɤl}}}
}\end{relation-sémantique}\end{entrée}

\begin{entrée}
\vedette{\hypertarget{Ⓔnɯkon}{\papi{ nɯkon}}}\markboth{nɯkon}{}
\classe{vt}
\paradigme{\textit{dir :} \jya tɤ-}
\begin{définition}\ 
\begin{déclaration}\grammar{denom}\end{déclaration}\end{définition}
\begin{définition}\fra gérer, s’occuper de\end{définition}
\begin{définition}\cmn 管理
\begin{déclaration} \étymologie{\papi{\stylefn{管}}}\end{déclaration}\end{définition}
\begin{exemple}\jya thamtham aʑo @linyegongzuo ku-oz-nɯkon-a\cmn 我现在管理林业工作\end{exemple}\begin{sous-entrée}
\vedette{\hypertarget{}{\papi{ znɯkon}}}\markboth{znɯkon}{}\classe{vt}
\begin{définition}\fra pouvoir contrôler\end{définition}
\begin{définition}\cmn 能管住\end{définition}
\begin{exemple}\jya iɕqha tɤ-pɤtso nɯ ɯ-tɯ-li kɯ ɯ-kɯ-znɯkon me\cmn 这个小孩子很调皮,没有管得住他\end{exemple}
\end{sous-entrée}\begin{sous-entrée}
\vedette{\hypertarget{}{\papi{ ʑɣɤnɯkon}}}\markboth{ʑɣɤnɯkon}{}\classe{vi}
\paradigme{\textit{dir :} \jya tɤ-}
\begin{définition}\ 
\begin{déclaration}\grammar{refl}\end{déclaration}\end{définition}
\begin{définition}\fra ne se soucier que de soi\end{définition}
\begin{définition}\cmn 只顾自己\end{définition}
\end{sous-entrée}\end{entrée}

\begin{entrée}
\vedette{\hypertarget{Ⓔnɯkoŋ}{\papi{ nɯkoŋ}}}\markboth{nɯkoŋ}{}
\classe{vs}
\paradigme{\textit{dir :} \jya tɤ-}
\begin{définition}\ 
\begin{déclaration}\grammar{denom}\end{déclaration}
\begin{déclaration}\grammar{denom}\end{déclaration}\end{définition}
\begin{définition}\fra cher\end{définition}
\begin{définition}\cmn 贵
\begin{déclaration} \étymologie{\papi{goŋ}}\end{déclaration}\end{définition}
\begin{exemple}\jya laχtɕha ɲɯ-nɯkoŋ, mɯ́j-nɯkoŋ\cmn 东西很便宜,不便宜\end{exemple}\end{entrée}

\begin{entrée}
\vedette{\hypertarget{Ⓔnɯkowa}{\papi{ nɯkowa}}}\markboth{nɯkowa}{}
\classe{vt}
\paradigme{\textit{dir :} \jya tɤ-}
\begin{définition}\ 
\begin{déclaration}\grammar{denom}\end{déclaration}\end{définition}
\begin{définition}\fra préparer\end{définition}
\begin{définition}\cmn 准备;想办法\end{définition}
\begin{exemple}\jya tɤ-nɯkowa-t-a\cmn 我准备了\end{exemple}
\begin{exemple}\jya ku-oz-nɯkowa-a\cmn 我正在准备\end{exemple}
\begin{exemple}\jya kha kɤ-βzu tɤ-nɯkowa-t-a\cmn 我准备修房子了\end{exemple}
\begin{exemple}\jya kɤ-rɤβzjoz tɤ-nɯkowa-ta\cmn 我做了读书的准备\end{exemple}
\begin{exemple}\jya ki tɤ-nɯkowa-t-a\cmn 我想了这个办法\end{exemple}
\begin{exemple}\jya tshi tsuku ʑo to-nɯkowa ri mɯ-pjɤ-cha\cmn 他想尽办法,但是没有成功\end{exemple}
\begin{relation-sémantique}\confer{
\hyperlink{Ⓔkowa}{\textit{ \papi{kowa}}}
}\end{relation-sémantique}
\begin{relation-sémantique}\synonyme{
\hyperlink{Ⓔnɯftɕaka}{\textit{ \papi{nɯftɕaka}}}
}\end{relation-sémantique}
\begin{relation-sémantique}\synonyme{
\hyperlink{ⒺmɲoⒽ1}{\textit{ \papi{mɲo}}}
}\end{relation-sémantique}
\begin{relation-sémantique}\confer{
\hyperlink{Ⓔkowa}{\textit{ \papi{kowa}}}
}\end{relation-sémantique}\end{entrée}

\begin{entrée}
\vedette{\hypertarget{Ⓔnɯkrɤlma}{\papi{ nɯkrɤlma}}}\markboth{nɯkrɤlma}{}
\classe{vi}
\paradigme{\textit{dir :} \jya tɤ-}
\begin{définition}\fra attraper une maladie de l'intestin\end{définition}
\begin{définition}\cmn 拉肚子(孩子)\end{définition}
\begin{exemple}\jya tɤ-pɤtso ɲɯ-nɯkrɤlma\cmn 小孩子在拉肚子\end{exemple}\end{entrée}

\begin{entrée}
\vedette{\hypertarget{Ⓔnɯkrɤz}{\papi{ nɯkrɤz}}}\markboth{nɯkrɤz}{}\classe{vt}
\paradigme{\textit{dir :} \jya tɤ-}
\begin{définition}\ 
\begin{déclaration}\grammar{denom}\end{déclaration}\end{définition}
\begin{définition}\fra discuter\end{définition}
\begin{définition}\cmn 商量\end{définition}
\begin{exemple}\jya kɤ-nɯkhɤjhwi pɯ-az-nɯkrɤz-tɕi\cmn 我们俩在商量开会的事\end{exemple}
\begin{exemple}\jya ɯ-koŋ ta-nɯkrɤz-nɯ\cmn 他们商量了价钱\end{exemple}
\begin{exemple}\jya tɤ-nɯkrɤz-tɕi\cmn 我们俩商量了\end{exemple}
\begin{exemple}\jya ju-kɤ-ɕe to-nɯkrɤz-nɯ\cmn 他们商量了要不要去\end{exemple}
\begin{relation-sémantique}\confer{
\hyperlink{Ⓔtɯkrɤz}{\textit{ \papi{tɯkrɤz}}}
}\end{relation-sémantique}
\begin{relation-sémantique}\confer{
\hyperlink{Ⓔrɤkrɤz}{\textit{ \papi{rɤkrɤz}}}
}\end{relation-sémantique}\end{entrée}

\begin{entrée}
\vedette{\hypertarget{Ⓔnɯkro}{\papi{ nɯkro}}}\markboth{nɯkro}{}
\begin{relation-sémantique}\confer{
\hyperlink{Ⓔkro}{\textit{ \papi{kro}}}
}\end{relation-sémantique}\end{entrée}

\begin{entrée}
\vedette{\hypertarget{Ⓔnɯkrɯβ}{\papi{ nɯkrɯβ}}}\markboth{nɯkrɯβ}{}
\classe{vi}
\paradigme{\textit{dir :} \jya pɯ-}
\begin{définition}\fra tomber malade à cause de nourriture avariée\end{définition}
\begin{définition}\cmn 食物中毒\end{définition}\begin{sous-entrée}
\vedette{\hypertarget{}{\papi{ znɯkrɯβ}}}\markboth{znɯkrɯβ}{}\classe{vt}
\paradigme{\textit{dir :} \jya pɯ-}
\begin{définition}\fra rendre malade (nourriture avariée)\end{définition}
\begin{définition}\cmn 使中毒\end{définition}
\begin{exemple}\jya tɤ-mthɯm ɲɤ-ɣɤdi tɕe pɯ́-wɣ-znɯkrɯβ-a\cmn 肉坏了,令我生病了\end{exemple}
\begin{exemple}\jya kɤ-ndza kɯ pjɤ́-wɣ-znɯkrɯβ-a\cmn 食物把我吃生病\end{exemple}
\end{sous-entrée}\end{entrée}

\begin{entrée}
\vedette{\hypertarget{Ⓔnɯkɯɕnom}{\papi{ nɯkɯɕnom}}}\markboth{nɯkɯɕnom}{}
\classe{vi}
\paradigme{\textit{dir :} \jya pɯ-}
\begin{définition}\ 
\begin{déclaration}\grammar{denom}\end{déclaration}\end{définition}
\begin{définition}\fra ramasser les épis tombés sur le sol après la récolte\end{définition}
\begin{définition}\cmn 收割后捡地上的青稞穗\end{définition}
\begin{exemple}\jya tɤɕi kɤ-phɯt-i tɕe, pɯ-nɯkɯɕnom-a\cmn 我们收割的时候,我捡了青稞穗\end{exemple}
\begin{relation-sémantique}\confer{
\hyperlink{Ⓔkɯɕnom}{\textit{ \papi{kɯɕnom}}}
}\end{relation-sémantique}\end{entrée}

\begin{entrée}
\vedette{\hypertarget{Ⓔnɯkɯjŋu}{\papi{ nɯkɯjŋu}}}\markboth{nɯkɯjŋu}{}
\classe{vt}
\paradigme{\textit{dir :} \jya tɤ-}
\paradigme{\textit{dir :} \jya nɯ-}
\begin{définition}\fra jurer\end{définition}
\begin{définition}\cmn 发誓\end{définition}
\begin{exemple}\jya ɲɤ-nɯkɯjŋu\cmn 他立下誓言\end{exemple}
\begin{exemple}\jya tɤ-nɯkɯjŋu-t-a (= kɯjŋu tɤ-joʁ-a)\cmn 我立下誓言\end{exemple}
\begin{relation-sémantique}\confer{
\hyperlink{Ⓔkɯjŋu}{\textit{ \papi{kɯjŋu}}}
}\end{relation-sémantique}
\begin{relation-sémantique}\confer{
\hyperlink{Ⓔanɯkɯjŋɯjŋu}{\textit{ \papi{anɯkɯjŋɯjŋu}}}
}\end{relation-sémantique}
\begin{relation-sémantique}\confer{
\hyperlink{Ⓔanɯkɯjŋɤŋgɯ}{\textit{ \papi{anɯkɯjŋɤŋgɯ}}}
}\end{relation-sémantique}\end{entrée}

\begin{entrée}
\vedette{\hypertarget{Ⓔnɯkɯlu}{\papi{ nɯkɯlu}}}\markboth{nɯkɯlu}{}
\classe{vi}
\paradigme{\textit{dir :} \jya pɯ-}
\begin{définition}\fra se perdre\end{définition}
\begin{définition}\cmn 迷路\end{définition}
\begin{exemple}\jya tʂu pjɤ-nɯkɯlu-a\cmn 我迷路了\end{exemple}
\begin{exemple}\jya nɯ zgoku nɯ kɤ-ɕe mɯ-pɯ-rɲo-t-a tɕe, pɯ-nɯɣi-a ri pɯ-nɯkɯlu-a\cmn 因为我从来没有去过这座山,我从上面回来的时候就迷路了\end{exemple}\begin{sous-entrée}
\vedette{\hypertarget{}{\papi{ znɯkɯlu}}}\markboth{znɯkɯlu}{}\classe{vt}
\paradigme{\textit{dir :} \jya pɯ-}
\begin{définition}\ 
\begin{déclaration}\grammar{caus}\end{déclaration}\end{définition}
\begin{définition}\fra faire se perdre\end{définition}
\begin{définition}\cmn 令……迷路\end{définition}
\begin{exemple}\jya pɯ́-wɣ-znɯkɯlu-a\cmn 他令我迷路了\end{exemple}
\end{sous-entrée}\end{entrée}

\begin{entrée}
\vedette{\hypertarget{Ⓔnɯkɯmaʁ}{\papi{ nɯkɯmaʁ}}}\markboth{nɯkɯmaʁ}{}
\classe{vi}
\paradigme{\textit{dir :} \jya thɯ-}
\paradigme{\textit{dir :} \jya nɯ-}
\begin{définition}\ 
\begin{déclaration}\grammar{denom}\end{déclaration}\end{définition}
\begin{définition}\fra se tromper\end{définition}
\begin{définition}\cmn 无意中犯错误\end{définition}
\begin{exemple}\jya tɯ-rju ɲɯ-nɯkɯmaʁ\cmn 他说错了\end{exemple}
\begin{exemple}\jya kɤ-rɤt ɲɤ-tɯ-nɯkɯmaʁ ri, mɯ́j-tɯ-nɯsɯrtoʁ\cmn 你写错了,但是你没有发现\end{exemple}
\begin{exemple}\jya jiɕqha kɤ-sthoʁ ɲɤ-nɯkɯmaʁ-a\cmn 我刚才按错了(手机)\end{exemple}
\begin{relation-sémantique}\confer{
\hyperlink{Ⓔkɯmaʁ}{\textit{ \papi{kɯmaʁ}}}
}\end{relation-sémantique}
\begin{relation-sémantique}\synonyme{
\hyperlink{Ⓔnor}{\textit{ \papi{nor}}}
}\end{relation-sémantique}\end{entrée}

\begin{entrée}
\vedette{\hypertarget{Ⓔnɯkɯmpɕɤr}{\papi{ nɯkɯmpɕɤr}}}\markboth{nɯkɯmpɕɤr}{}
\classe{vt}
\paradigme{\textit{dir :} \jya thɯ-}
\begin{définition}\ 
\begin{déclaration}\grammar{denom}\end{déclaration}\end{définition}
\begin{définition}\fra porter aux grandes occasions\end{définition}
\begin{définition}\cmn 打扮(在特定的情况才穿的衣服)\end{définition}
\begin{exemple}\jya ki tɤrɣe ki nɯ chɯ-nɯkɯmpɕɤr ɕti ma ɯ-xso tɕe mɤ-ntɕhoz\cmn 珍珠只在特殊的情况才戴,平时不戴\end{exemple}
\begin{relation-sémantique}\confer{
\hyperlink{Ⓔmpɕɤr}{\textit{ \papi{mpɕɤr}}}
}\end{relation-sémantique}\end{entrée}

\begin{entrée}
\vedette{\hypertarget{Ⓔnɯkɯmtɕhɯ}{\papi{ nɯkɯmtɕhɯ}}}\markboth{nɯkɯmtɕhɯ}{}\classe{vt}
\paradigme{\textit{dir :} \jya nɯ-}
\begin{définition}\fra jouer avec\end{définition}
\begin{définition}\cmn 把……当做玩具\end{définition}
\begin{exemple}\jya a-tɕɯ kɯ qartshaz ɯ-χpi ɲɯ-nɯkɯmtɕhi ŋu\cmn 我儿子在玩鹿形状的玩具\end{exemple}
\begin{relation-sémantique}\confer{
\hyperlink{Ⓔkɯmtɕhɯ}{\textit{ \papi{kɯmtɕhɯ}}}
}\end{relation-sémantique}\end{entrée}

\begin{entrée}
\vedette{\hypertarget{Ⓔnɯlaʁjoʁ}{\papi{ nɯlaʁjoʁ}}}\markboth{nɯlaʁjoʁ}{}\classe{vi}
\begin{définition}\ 
\begin{déclaration}\grammar{incorp}\end{déclaration}\end{définition}
\begin{définition}\fra servir d'assistant\end{définition}
\begin{définition}\cmn 当帮手\end{définition}
\begin{relation-sémantique}\confer{
\hyperlink{Ⓔlaʁjoʁ}{\textit{ \papi{laʁjoʁ}}}
}\end{relation-sémantique}\end{entrée}

\begin{entrée}
\vedette{\hypertarget{Ⓔnɯlɤmba}{\papi{ nɯlɤmba}}}\markboth{nɯlɤmba}{}
\classe{vt}
\paradigme{\textit{dir :} \jya tɤ-}
\begin{définition}\fra soutenir, s'occuper de\end{définition}
\begin{définition}\cmn 扶持;照顾\end{définition}
\begin{exemple}\jya jiɕqha lo-βzi tɕe tɤ-nɯlɤmba-t-a\cmn 他醉了,我就把他扶起来了\end{exemple}
\begin{relation-sémantique}\synonyme{
\hyperlink{Ⓔɣɤrndi}{\textit{ \papi{ɣɤrndi}}}
}\end{relation-sémantique}\end{entrée}

\begin{entrée}
\vedette{\hypertarget{Ⓔnɯlɤn}{\papi{ nɯlɤn}}}\markboth{nɯlɤn}{}
\begin{relation-sémantique}\confer{
\hyperlink{Ⓔlɤn}{\textit{ \papi{lɤn}}}
}\end{relation-sémantique}\end{entrée}

\begin{entrée}
\vedette{\hypertarget{Ⓔnɯlɤsɤr}{\papi{ nɯlɤsɤr}}}\markboth{nɯlɤsɤr}{}
\classe{vi}
\paradigme{\textit{dir :} \jya tɤ-}
\begin{définition}\ 
\begin{déclaration}\grammar{denom}\end{déclaration}\end{définition}
\begin{définition}\fra fêter le nouvel an\end{définition}
\begin{définition}\cmn 过年
\begin{déclaration} \étymologie{\papi{lo.gsar}}\end{déclaration}\end{définition}
\begin{exemple}\jya kɤ-nɯlɤsɤr to-mda\cmn 到了过年的时候了\end{exemple}
\begin{relation-sémantique}\confer{
\hyperlink{Ⓔlɤsɤr}{\textit{ \papi{lɤsɤr}}}
}\end{relation-sémantique}\end{entrée}

\begin{entrée}
\vedette{\hypertarget{Ⓔnɯlŋɤβ}{\papi{ nɯlŋɤβ}}}\markboth{nɯlŋɤβ}{}\classe{vt}
\paradigme{\textit{dir :} \jya tɤ-}
\begin{définition}\fra s'empiffrer\end{définition}
\begin{définition}\cmn 大口大口地吃\end{définition}
\begin{exemple}\jya tɤ-nɯlŋaβ-a ʑo tɤ-ndza-t-a\cmn 我大口大口地吃了\end{exemple}
\begin{relation-sémantique}\synonyme{
\hyperlink{Ⓔnɯchɯβ}{\textit{ \papi{nɯchɯβ}}}
}\end{relation-sémantique}\end{entrée}

\begin{entrée}
\vedette{\hypertarget{Ⓔnɯlɯka}{\papi{ nɯlɯka}}}\markboth{nɯlɯka}{}
\classe{vs}
\paradigme{\textit{dir :} \jya tɤ-}
\begin{définition}\fra être séparé et ne pas être dérangé par les autres, ne pas avoir besoin de s'occuper de toutes sortes de choses\end{définition}
\begin{définition}\cmn 被隔开(不受别人的干扰,不需要管多种事情)\end{définition}
\begin{exemple}\jya nɯ ɕɯŋgɯ tɕe, thɯci tsuku ʑo tɯtɯrca ɣɯ-nɤma pɯ-ra tɕe pɯ-sɤɣdɯɣ ma tham tɕe tɯ-tɯphu ma ɣɯ-nɤma mɯ́j-ra tɕe, ɲɯ-sɤscit ma ɲɯ-kɯ-nɯlɯka tɕe\cmn 以前要同时做好几种事情,现在只需要做一种事,不再需要管那么多,很轻松\end{exemple}\end{entrée}

\begin{entrée}
\vedette{\hypertarget{Ⓔnɯɬoʁ}{\papi{ nɯɬoʁ}}}\markboth{nɯɬoʁ}{}\classe{vi}
\paradigme{\textit{dir :} \jya \_}
\begin{définition}\ 
\begin{déclaration}\grammar{autoben}\end{déclaration}\end{définition}
\begin{définition}\fra se détacher\end{définition}
\begin{définition}\cmn 散开;自动的出来\end{définition}
\begin{exemple}\jya laχtɕha nɯɬoʁ ɲɯ-ŋu\cmn 这个东西快要掉出来了\end{exemple}
\begin{exemple}\jya tɯ-xtsɤ-ri ɲɤ-nɯɬoʁ\cmn 鞋带散了\end{exemple}
\begin{exemple}\jya a-mi ɲɤ-nɯɬoʁ\cmn 我脚脱臼了\end{exemple}
\begin{exemple}\jya paʁ ɯ-naŋtɕɯ chɤ-nɯɬoʁ\cmn 猪的内脏出来了\end{exemple}
\begin{relation-sémantique}\confer{
\hyperlink{ⒺɬoʁⒽ2}{\textit{ \papi{ɬoʁ2}}}
}\end{relation-sémantique}\end{entrée}

\begin{entrée}
\vedette{\hypertarget{Ⓔnɯmɤɕɯŋgɯ}{\papi{ nɯmɤɕɯŋgɯ}}}\markboth{nɯmɤɕɯŋgɯ}{}\classe{adv}
\begin{définition}\fra autrefois\end{définition}
\begin{définition}\cmn 以前\end{définition}\end{entrée}

\begin{entrée}
\vedette{\hypertarget{Ⓔnɯmbe}{\papi{ nɯmbe}}}\markboth{nɯmbe}{}
\classe{vt}
\paradigme{\textit{dir :} \jya tɤ-}
\begin{définition}\fra dédommager\end{définition}
\begin{définition}\cmn 赔偿\end{définition}
\begin{exemple}\jya a-ŋga ɲɤ-tɯ-βde tɕe tɤ-nɯmbe\cmn 你把我的衣服弄丢了,你要给我赔\end{exemple}
\begin{exemple}\jya ɯ-laχtɕha ɲɤ-nɯβde-t-a tɕe tɤ-nɯmbe-t-a\cmn 我把他的东西弄丢了,就给他赔了\end{exemple}
\begin{relation-sémantique}\synonyme{
\hyperlink{Ⓔrɤli}{\textit{ \papi{rɤli}}}
}\end{relation-sémantique}\end{entrée}

\begin{entrée}
\vedette{\hypertarget{Ⓔnɯmbɣom}{\papi{ nɯmbɣom}}}\markboth{nɯmbɣom}{}\classe{vt}
\paradigme{\textit{dir :} \jya tɤ-}
\begin{définition}\ 
\begin{déclaration}\grammar{appl}\end{déclaration}\end{définition}
\begin{définition}\fra avoir hâte de\end{définition}
\begin{définition}\cmn 盼望\end{définition}
\begin{exemple}\jya tɤ-ta-nɯmbɣom\cmn 我很想你了\end{exemple}
\begin{exemple}\jya ɯ-mu to-nɯmbɣom\cmn 他很想妈妈了\end{exemple}
\begin{exemple}\jya ɯʑo ju-nɯɣi ɲɯ-nɯmbɣom-a\cmn 我盼望他早日回来\end{exemple}
\begin{exemple}\jya lɤsɤr ju-zɣɯt ɲɯ-nɯmbɣom-a\cmn 我盼望新年\end{exemple}
\begin{exemple}\jya jɯfɕɯr a-ʑɯβ mɯ-pɯ-ɣe tɕe, lu-fsoʁ tɤ-nɯmbɣom-a\cmn 昨天睡不着,盼望天亮\end{exemple}
\begin{relation-sémantique}\synonyme{
\hyperlink{Ⓔnɯɣbɯɣ}{\textit{ \papi{nɯɣbɯɣ}}}
}\end{relation-sémantique}
\begin{relation-sémantique}\confer{
\hyperlink{Ⓔmbɣom}{\textit{ \papi{mbɣom}}}
}\end{relation-sémantique}\begin{sous-entrée}
\vedette{\hypertarget{}{\papi{ anɯmbɯmbɣom}}}\markboth{anɯmbɯmbɣom}{}\classe{vi}
\begin{définition}\fra se manquer les uns aux autres\end{définition}
\begin{définition}\cmn 互相思念\end{définition}
\end{sous-entrée}\end{entrée}

\begin{entrée}
\vedette{\hypertarget{Ⓔnɯmbjɯm}{\papi{ nɯmbjɯm}}}\markboth{nɯmbjɯm}{}
\classe{vi}
\paradigme{\textit{dir :} \jya thɯ-}
\begin{définition}\fra se chauffer au feu\end{définition}
\begin{définition}\cmn 烤火取暖\end{définition}
\begin{exemple}\jya smi ɯ-phe thɯ-nɯmbjɯm\cmn 你烤火取暖吧\end{exemple}
\begin{exemple}\jya smi ɯ-phe thɯ-nɯmbjɯm-a\cmn 我烤火取暖了\end{exemple}
\begin{relation-sémantique}\synonyme{
\hyperlink{Ⓔnɯsmɯɣjɯm}{\textit{ \papi{nɯsmɯɣjɯm}}}
}\end{relation-sémantique}\end{entrée}

\begin{entrée}
\vedette{\hypertarget{Ⓔnɯmbrɤpɯ}{\papi{ nɯmbrɤpɯ}}}\markboth{nɯmbrɤpɯ}{}
\classe{v}
\paradigme{\textit{dir :} \jya tɤ-}
\begin{définition}\ 
\begin{déclaration}\grammar{incorp}\end{déclaration}\end{définition}
\begin{définition}\fra monter (à cheval )\end{définition}
\begin{définition}\cmn 骑\end{définition}
\begin{exemple}\jya mbro tɤ-nɯmbrɤpɯ-t-a\cmn 我骑了马\end{exemple}
\begin{exemple}\jya qambrɯ tɤ-nɯmbrɤpɯ-t-a\cmn 我骑了牦牛\end{exemple}\begin{sous-entrée}
\vedette{\hypertarget{}{\papi{ ʑɣɤnɯmbrɤpɯ}}}\markboth{ʑɣɤnɯmbrɤpɯ}{}\classe{vi}
\paradigme{\textit{dir :} \jya tɤ-}
\begin{définition}\fra se laisser monter\end{définition}
\begin{définition}\cmn 让……骑在自己背上\end{définition}
\begin{exemple}\jya mbro tɯrme nɯ kɯ to-ʑɣɤnɯmbrɤpɯ\cmn 马让人骑在它背上了\end{exemple}
\begin{relation-sémantique}\confer{
\hyperlink{ⒺmbroⒽ2}{\textit{ \papi{mbro2}}}
}\end{relation-sémantique}
\end{sous-entrée}\end{entrée}

\begin{entrée}
\vedette{\hypertarget{Ⓔnɯmbrɤrɟɯɣ}{\papi{ nɯmbrɤrɟɯɣ}}}\markboth{nɯmbrɤrɟɯɣ}{}
\classe{vi}
\paradigme{\textit{dir :} \jya tɤ-}
\begin{définition}\ 
\begin{déclaration}\grammar{incorp}\end{déclaration}\end{définition}
\begin{définition}\fra faire une course de cheval\end{définition}
\begin{définition}\cmn 赛马\end{définition}
\begin{exemple}\jya lɤsɤr ɯ-raŋ pɯ-nɯmbrɤrɟɯɣ-nɯ\cmn 过年的时候,他们在赛马\end{exemple}
\begin{relation-sémantique}\confer{
\hyperlink{Ⓔmbrɤrɟɯɣ}{\textit{ \papi{mbrɤrɟɯɣ}}}
}\end{relation-sémantique}\end{entrée}

\begin{entrée}
\vedette{\hypertarget{Ⓔnɯmbrɤzɯ}{\papi{ nɯmbrɤzɯ}}}\markboth{nɯmbrɤzɯ}{}
\classe{vt}
\paradigme{\textit{dir :} \jya pɯ-}
\begin{définition}\ 
\begin{déclaration}\grammar{denom}\end{déclaration}\end{définition}
\begin{définition}\fra obtenir le fruit de son travail\end{définition}
\begin{définition}\cmn 得到自己的劳动成果\end{définition}
\begin{exemple}\jya pɯ-nɯmbrɤzɯ-j\cmn 我们得到自己的劳动成果\end{exemple}
\begin{exemple}\jya japa ndɤre sɤrwa pɯ-tu tɕe kɤ-nɯmbrɤzɯ pɯ-rkɯn\cmn 去年下了冰雹,(农民们的)收获不多\end{exemple}
\begin{relation-sémantique}\confer{
\hyperlink{Ⓔɯ-mbrɤzɯ}{\textit{ \papi{ɯ-mbrɤzɯ}}}
}\end{relation-sémantique}\end{entrée}

\begin{entrée}
\vedette{\hypertarget{Ⓔnɯmbrɯmtsaʁ}{\papi{ nɯmbrɯmtsaʁ}}}\markboth{nɯmbrɯmtsaʁ}{}\classe{vi}
\paradigme{\textit{dir :} \jya tɤ-}
\begin{définition}\ 
\begin{déclaration}\grammar{incorp}\end{déclaration}\end{définition}
\begin{définition}\fra sauter à la corde\end{définition}
\begin{définition}\cmn 跳绳\end{définition}
\begin{relation-sémantique}\confer{
\hyperlink{Ⓔtɯmbri}{\textit{ \papi{tɯmbri}}}
}\end{relation-sémantique}
\begin{relation-sémantique}\confer{
\hyperlink{Ⓔmtsaʁ}{\textit{ \papi{mtsaʁ}}}
}\end{relation-sémantique}\end{entrée}

\begin{entrée}
\vedette{\hypertarget{Ⓔnɯmbɯrlɤn}{\papi{ nɯmbɯrlɤn}}}\markboth{nɯmbɯrlɤn}{}
\classe{vt}
\paradigme{\textit{dir :} \jya thɯ-}
\begin{définition}\ 
\begin{déclaration}\grammar{denom}\end{déclaration}\end{définition}
\begin{définition}\fra raboter\end{définition}
\begin{définition}\cmn 刨
\begin{déclaration} \étymologie{\papi{ⁿbur.len}}\end{déclaration}\end{définition}
\begin{exemple}\jya si thɯ-nɯmbɯrlɤn\cmn 你刨一下树\end{exemple}
\begin{exemple}\jya tɤrɤm thɯ-nɯmbɯrlan-a\cmn 我刨了木板\end{exemple}
\begin{relation-sémantique}\confer{
\hyperlink{Ⓔmbɯrlɤn}{\textit{ \papi{mbɯrlɤn}}}
}\end{relation-sémantique}\end{entrée}

\begin{entrée}
\vedette{\hypertarget{Ⓔnɯmbɯsɯt}{\papi{ nɯmbɯsɯt}}}\markboth{nɯmbɯsɯt}{}\classe{vt}
\paradigme{\textit{dir :} \jya thɯ-}
\begin{définition}\ 
\begin{déclaration}\grammar{denom}\end{déclaration}\end{définition}
\begin{définition}\fra râper\end{définition}
\begin{définition}\cmn 擦成丝丝\end{définition}
\begin{relation-sémantique}\confer{
\hyperlink{Ⓔmbɯsɯt}{\textit{ \papi{mbɯsɯt}}}
}\end{relation-sémantique}\end{entrée}

\begin{entrée}
\vedette{\hypertarget{Ⓔnɯmdar}{\papi{ nɯmdar}}}\markboth{nɯmdar}{}\classe{vi}
\paradigme{\textit{dir :} \jya \_}
\begin{définition}\fra sauter\end{définition}
\begin{définition}\cmn 跳\end{définition}
\begin{exemple}\jya kɤ-nɯmdar-a\cmn 我跳了\end{exemple}
\begin{exemple}\jya tɕɤki tɯ-ci ɣɤʑu tɕe, kɤ-ŋke mɯ́j-khɯ tɕe thɯ-nɯmdar-a\cmn 下面有水,我不能走就跳过去了\end{exemple}
\begin{relation-sémantique}\synonyme{
\hyperlink{Ⓔmtsaʁ}{\textit{ \papi{mtsaʁ}}}
}\end{relation-sémantique}\begin{sous-entrée}
\vedette{\hypertarget{}{\papi{ nɤmdɯmdar}}}\markboth{nɤmdɯmdar}{}\classe{vi}
\begin{définition}\ 
\begin{déclaration}\grammar{n.orient}\end{déclaration}\end{définition}
\begin{définition}\fra sauter dans tous les sens\end{définition}
\begin{définition}\cmn 跳来跳去\end{définition}
\end{sous-entrée}\end{entrée}

\begin{entrée}
\vedette{\hypertarget{Ⓔnɯmdaʁʑɯɣ}{\papi{ nɯmdaʁʑɯɣ}}}\markboth{nɯmdaʁʑɯɣ}{}
\classe{vt}
\paradigme{\textit{dir :} \jya tɤ-}
\begin{définition}\ 
\begin{déclaration}\grammar{denom}\end{déclaration}\end{définition}
\begin{définition}\fra tirer à l'arc\end{définition}
\begin{définition}\cmn 射箭\end{définition}
\begin{exemple}\jya tɤ-fsɯr tɤ-nɯmdaʁzɯɣ-a\cmn 我对着靶子射箭了\end{exemple}
\begin{relation-sémantique}\confer{
\hyperlink{Ⓔmdaʁʑɯɣ}{\textit{ \papi{mdaʁʑɯɣ}}}
}\end{relation-sémantique}\end{entrée}

\begin{entrée}
\vedette{\hypertarget{Ⓔnɯmdoʁ}{\papi{ nɯmdoʁ}}}\markboth{nɯmdoʁ}{}\classe{vi}
\paradigme{\textit{dir :} \jya tɤ-}
\begin{définition}\ 
\begin{déclaration}\grammar{denom}\end{déclaration}\end{définition}
\begin{définition}\fra avoir l'air de\end{définition}
\begin{définition}\cmn 好像\end{définition}
\begin{exemple}\jya jisŋi tɯ-mɯ kɯ-lɤt ɲɯ-nɯmdoʁ\cmn 今天好像要下雨\end{exemple}
\begin{exemple}\jya ɲɯ-tɯ-nɯmdoʁ\cmn 你又强壮又高大\end{exemple}
\begin{relation-sémantique}\confer{
\hyperlink{Ⓔɯ-mdoʁ}{\textit{ \papi{ɯ-mdoʁ}}}
}\end{relation-sémantique}\begin{sous-entrée}
\vedette{\hypertarget{}{\papi{ znɯmdoʁ}}}\markboth{znɯmdoʁ}{}\classe{vt}
\begin{définition}\fra bien faire (un certain type de travail)\end{définition}
\begin{définition}\cmn 【好样】(那一方面的工作)做得好\end{définition}
\begin{exemple}\jya nɤki nɯ kɤ-sɤsɯxɕɤt kɯ-znɯmdoʁ ci pɯ-ŋu\cmn 那个人以前是个很好的老师\end{exemple}
\begin{exemple}\jya ɯʑo kɯ tɯ-ta-nɤma ʑo nɯ tu-znɯmdoʁɕti\cmn 他做的每一样工作都做得非常好\end{exemple}
\end{sous-entrée}\end{entrée}

\begin{entrée}
\vedette{\hypertarget{Ⓔnɯmdɯm}{\papi{ nɯmdɯm}}}\markboth{nɯmdɯm}{}
\classe{vt}
\paradigme{\textit{dir :} \jya \_}
\begin{définition}\fra manger en marchant\end{définition}
\begin{définition}\cmn 一边走一边吃\end{définition}
\begin{exemple}\jya kɯ-chi ɲɯ-tɯ-ɤz-nɯmdɯm\cmn 你一边走一边吃糖\end{exemple}
\begin{exemple}\jya @paopaotang ɲɯ-tɯ-ɤz-nɯmdɯm\cmn 你一边走一边吃泡泡糖\end{exemple}
\begin{relation-sémantique}\confer{
\hyperlink{Ⓔnɯndzɤmdɯm}{\textit{ \papi{nɯndzɤmdɯm}}}
}\end{relation-sémantique}\end{entrée}

\begin{entrée}
\vedette{\hypertarget{Ⓔnɯmga}{\papi{ nɯmga}}}\markboth{nɯmga}{}
\classe{vt}
\paradigme{\textit{dir :} \jya kɤ-}\acception{1}
\begin{définition}\fra avoir l'intention\end{définition}
\begin{définition}\cmn 有意\end{définition}\acception{2}
\begin{définition}\fra être bien fait pour\end{définition}
\begin{définition}\cmn 活该\end{définition}
\begin{exemple}\jya nɯ kɤ-tɯ-nɯmga-t\cmn 你活该!\end{exemple}
\begin{exemple}\jya pɯ-tɯ-sɤnɯrtɕa ntsɯ tɕe taχphe ci nɯ-tɯ́-wɣ-sɤʁe tɕe, nɯ kɤ-tɯ-nɯmga-t\cmn 你总是惹人家,你被打耳光是你活该\end{exemple}\begin{sous-entrée}
\vedette{\hypertarget{}{\papi{ kɤ-nɯmga}}}\markboth{kɤ-nɯmga}{}
\begin{définition}\fra afin de\end{définition}
\begin{définition}\cmn 为了\end{définition}
\begin{exemple}\jya ftsoʁ nɯ tɕe tɕe ɯ-lu kɤ-nɯmga ɲɯ-ŋu\cmn 母犏牛是为了牛奶(而养的)\end{exemple}
\end{sous-entrée}\end{entrée}

\begin{entrée}
\vedette{\hypertarget{Ⓔnɯmgo}{\papi{ nɯmgo}}}\markboth{nɯmgo}{}\classe{vi}
\paradigme{\textit{dir :} \jya tɤ-}
\begin{définition}\ 
\begin{déclaration}\grammar{denom}\end{déclaration}\end{définition}
\begin{définition}\fra déjeuner\end{définition}
\begin{définition}\cmn 吃中午饭\end{définition}
\begin{relation-sémantique}\synonyme{
\hyperlink{Ⓔrɯndzɤtshi}{\textit{ \papi{rɯndzɤtshi}}}
}\end{relation-sémantique}
\begin{relation-sémantique}\confer{
\hyperlink{Ⓔtɯ-mgo}{\textit{ \papi{tɯ-mgo}}}
}\end{relation-sémantique}\end{entrée}

\begin{entrée}
\vedette{\hypertarget{Ⓔnɯmgro}{\papi{ nɯmgro}}}\markboth{nɯmgro}{}
\classe{vt}
\paradigme{\textit{dir :} \jya thɯ-}
\paradigme{\textit{dir :} \jya pɯ-}\acception{1}
\begin{définition}\fra attendre\end{définition}
\begin{définition}\cmn 盼望\end{définition}
\begin{exemple}\jya pɯ-ta-nɯmgro\cmn 我盼望你\end{exemple}
\begin{exemple}\jya pjɯ-tɯ-ɣi pɯ-ta-nɯmgro\cmn 我盼望你来\end{exemple}\acception{2}
\begin{définition}\fra espérer\end{définition}
\begin{définition}\cmn 希望\end{définition}
\begin{exemple}\jya nɤʑo kɤ-si ma nɤ-kɤ-nɯmgro ɲɤ-me\cmn 你只有死路一条\end{exemple}
\begin{relation-sémantique}\confer{
\hyperlink{Ⓔsɤmgro}{\textit{ \papi{sɤmgro}}}
}\end{relation-sémantique}\begin{sous-entrée}
\vedette{\hypertarget{}{\papi{ anɯmgɯmgro}}}\markboth{anɯmgɯmgro}{}\classe{vi}
\begin{définition}\fra s'attendre les uns les autres\end{définition}
\begin{définition}\cmn 互相盼望\end{définition}
\end{sous-entrée}\end{entrée}

\begin{entrée}
\vedette{\hypertarget{Ⓔnɯmɟa}{\papi{ nɯmɟa}}}\markboth{nɯmɟa}{}
\classe{vt}
\paradigme{\textit{dir :} \jya tɤ-}
\paradigme{\textit{dir :} \jya pɯ-}
\begin{définition}\fra obtenir\end{définition}
\begin{définition}\cmn 得到,捡\end{définition}
\begin{exemple}\jya tɤ-nɯmɟa-t-a\cmn 我捡了\end{exemple}
\begin{exemple}\jya ɯ-thoʁ nɯ ɲɯ-ɤta tɕe, tɤ-nɯmɟa-t-a\cmn 地上有那个东西,我就捡了\end{exemple}
\begin{relation-sémantique}\confer{
\hyperlink{Ⓔmɟa}{\textit{ \papi{mɟa}}}
}\end{relation-sémantique}\end{entrée}

\begin{entrée}
\vedette{\hypertarget{Ⓔnɯmkɤɣɯr}{\papi{ nɯmkɤɣɯr}}}\markboth{nɯmkɤɣɯr}{}
\classe{vt}
\paradigme{\textit{dir :} \jya thɯ-}
\begin{définition}\ 
\begin{déclaration}\grammar{denom}\end{déclaration}\end{définition}
\begin{définition}\fra porter sur le cou comme un collier\end{définition}
\begin{définition}\cmn 把某物体当项链戴在脖子上\end{définition}
\begin{exemple}\jya laχtɕha thɯ-nɯmkɤɣɯr-a\cmn 我把东西戴在脖子上了\end{exemple}
\begin{exemple}\jya ɯʑo kɯ laχtɕha tha-nɯmkɤɣɯr\cmn 他把东西戴在脖子上\end{exemple}
\begin{relation-sémantique}\confer{
\hyperlink{Ⓔmkɤɣɯr}{\textit{ \papi{mkɤɣɯr}}}
}\end{relation-sémantique}\end{entrée}

\begin{entrée}
\vedette{\hypertarget{Ⓔnɯmkɤqloʁ}{\papi{ nɯmkɤqloʁ}}}\markboth{nɯmkɤqloʁ}{}\classe{vi}
\paradigme{\textit{dir :} \jya thɯ-}
\begin{définition}\fra se prendre les pieds dans quelque chose et tomber\end{définition}
\begin{définition}\cmn 绊倒
\begin{déclaration}\use{\stylefv{nɯmkɤqloʁ}比\stylefv{aʁdɤt}稍微严重一些,前者表示全身摔倒在地,后缀表示也许可以用手扶起来}\end{déclaration}\end{définition}
\begin{exemple}\jya thɯ-nɯkɤqloʁ-a\cmn 我被绊倒了\end{exemple}\begin{sous-entrée}
\vedette{\hypertarget{}{\papi{ znɯmkɤqloʁ}}}\markboth{znɯmkɤqloʁ}{}\classe{vt}
\paradigme{\textit{dir :} \jya thɯ-}
\begin{définition}\ 
\begin{déclaration}\grammar{caus}\end{déclaration}\end{définition}
\begin{définition}\fra faire un croc-en-jambe (faire tomber vers l'avant)\end{définition}
\begin{définition}\cmn 绊拽\end{définition}
\begin{exemple}\jya thɯ-kɯ-z-nɯmkɤqloʁ-a (=pɯ-kɯ-tʂaβ-a)\cmn 你把我绊拽住了\end{exemple}
\begin{exemple}\jya ma-tɯ-ste ma tɯ-z-nɯmkɤqloʁ\cmn 你别这样,不然你会把他绊拽住\end{exemple}
\end{sous-entrée}\end{entrée}

\begin{entrée}
\vedette{\hypertarget{Ⓔnɯmɢla}{\papi{ nɯmɢla}}}\markboth{nɯmɢla}{}
\classe{vt}
\paradigme{\textit{dir :} \jya \_}
\begin{définition}\ 
\begin{déclaration}\grammar{denom}\end{déclaration}\end{définition}
\begin{définition}\fra passer par dessus\end{définition}
\begin{définition}\cmn 跨过\end{définition}
\begin{exemple}\jya si kɤ-nɯmɢla-t-a\cmn 我跨过了(倒下的)树干\end{exemple}
\begin{exemple}\jya rdɤstaʁ kɤ-nɯmɢla-t-a\cmn 我从石头上跨过去了\end{exemple}
\begin{relation-sémantique}\confer{
\hyperlink{Ⓔtɯ-mɢla}{\textit{ \papi{tɯ-mɢla}}}
}\end{relation-sémantique}\end{entrée}

\begin{entrée}
\vedette{\hypertarget{Ⓔnɯmnɤl}{\papi{ nɯmnɤl}}}\markboth{nɯmnɤl}{}
\classe{vi}
\paradigme{\textit{dir :} \jya pɯ-}
\begin{définition}\fra être sali, être rendu impur\end{définition}
\begin{définition}\cmn 被玷污,受晦气(迷信的说法)\end{définition}\begin{sous-entrée}
\vedette{\hypertarget{}{\papi{ znɯmnɤl}}}\markboth{znɯmnɤl}{}\classe{vt}
\paradigme{\textit{dir :} \jya pɯ-}
\begin{définition}\ 
\begin{déclaration}\grammar{caus}\end{déclaration}\end{définition}
\begin{définition}\fra salir, rendre impur\end{définition}
\begin{définition}\cmn 令……沾上玷污气,沾上晦气\end{définition}
\begin{exemple}\jya tɯ-ŋga nɯnɯ ma-thɯ-tɯ-ŋge ma mɤ-χtso tɕe tú-wɣ-z-nɯmnɤl\cmn 你别穿这件衣服,不干净,你会沾上晦气的\end{exemple}
\begin{exemple}\jya tɕhi tɤ-tɯ-ari tɕe, tɯrme ɯ-pa a-mɤ-tɯ-ɕe ma tú-wɣ-z-nɯmnɤl\cmn 你上楼梯的时候,不要在别人下面不然你会沾上晦气的\end{exemple}
\begin{exemple}\jya tɯ-ŋga kɯ pjɤ́-wɣ-z-nɯmnal-a\cmn 这件令我受了晦气\end{exemple}
\end{sous-entrée}\end{entrée}

\begin{entrée}
\vedette{\hypertarget{Ⓔnɯmɲaqrɯ}{\papi{ nɯmɲaqrɯ}}}\markboth{nɯmɲaqrɯ}{}\classe{vt}
\paradigme{\textit{dir :} \jya tɤ-}
\begin{définition}\fra épier\end{définition}
\begin{définition}\cmn 瞪眼\end{définition}
\begin{exemple}\jya ma-tɤ-kɯ-nɯmɲaqrɯ-a\cmn 你不要瞪我\end{exemple}
\begin{relation-sémantique}\confer{
\hyperlink{Ⓔmɲaqrɯ}{\textit{ \papi{mɲaqrɯ}}}
}\end{relation-sémantique}\end{entrée}

\begin{entrée}
\vedette{\hypertarget{Ⓔnɯmɲɯɣ}{\papi{ nɯmɲɯɣ}}}\markboth{nɯmɲɯɣ}{}
\classe{vi}
\paradigme{\textit{dir :} \jya kɤ-}
\begin{définition}\ 
\begin{déclaration}\grammar{denom}\end{déclaration}\end{définition}
\begin{définition}\fra attraper le cancer de l'estomac\end{définition}
\begin{définition}\cmn 得胃癌\end{définition}
\begin{exemple}\jya ko-nɯmɲɯɣ\cmn 他得了胃癌\end{exemple}
\begin{relation-sémantique}\confer{
\hyperlink{Ⓔtɯ-mɲɯɣ}{\textit{ \papi{tɯ-mɲɯɣ}}}
}\end{relation-sémantique}\end{entrée}

\begin{entrée}
\vedette{\hypertarget{Ⓔnɯmɲɯka}{\papi{ nɯmɲɯka}}}\markboth{nɯmɲɯka}{}
\classe{vi}
\paradigme{\textit{dir :} \jya pɯ-}
\begin{définition}\ 
\begin{déclaration}\grammar{denom}\end{déclaration}\end{définition}
\begin{définition}\fra être humilié\end{définition}
\begin{définition}\cmn 被羞辱\end{définition}
\begin{exemple}\jya jiɕqha ndɤre tɯrme ɯ-ʁɤri pɯ-nɯmɲɯka\cmn 他在别人面前被羞辱了\end{exemple}
\begin{relation-sémantique}\confer{
\hyperlink{Ⓔmɲɯka}{\textit{ \papi{mɲɯka}}}
}\end{relation-sémantique}\end{entrée}

\begin{entrée}
\vedette{\hypertarget{Ⓔnɯmɲɯʁʑi}{\papi{ nɯmɲɯʁʑi}}}\markboth{nɯmɲɯʁʑi}{}\classe{vs}
\begin{définition}\fra avoir bon caractère\end{définition}
\begin{définition}\cmn 脾气好\end{définition}
\begin{relation-sémantique}\synonyme{
\hyperlink{Ⓔnɯmɲɯtɕhɤz}{\textit{ \papi{nɯmɲɯtɕhɤz}}}
}\end{relation-sémantique}
\begin{relation-sémantique}\confer{
\hyperlink{Ⓔmɲɯʁʑi}{\textit{ \papi{mɲɯʁʑi}}}
}\end{relation-sémantique}\end{entrée}

\begin{entrée}
\vedette{\hypertarget{Ⓔnɯmɲɯtɕhɤz}{\papi{ nɯmɲɯtɕhɤz}}}\markboth{nɯmɲɯtɕhɤz}{}\classe{vs}
\paradigme{\textit{dir :} \jya tɤ-}
\begin{définition}\fra avoir bon caractère\end{définition}
\begin{définition}\cmn 脾气很好\end{définition}
\begin{relation-sémantique}\synonyme{
\hyperlink{Ⓔnɯmɲɯʁʑi}{\textit{ \papi{nɯmɲɯʁʑi}}}
}\end{relation-sémantique}
\begin{relation-sémantique}\confer{
\hyperlink{Ⓔmɲɯtɕhɤz}{\textit{ \papi{mɲɯtɕhɤz}}}
}\end{relation-sémantique}\end{entrée}

\begin{entrée}
\vedette{\hypertarget{Ⓔnɯmŋu}{\papi{ nɯmŋu}}}\markboth{nɯmŋu}{}\classe{vt}
\paradigme{\textit{dir :} \jya kɤ-}
\begin{définition}\fra boire sans main, en mettant directement sa bouche sur...\end{définition}
\begin{définition}\cmn (不用手)直接用嘴对着……的口喝\end{définition}
\begin{exemple}\jya khɯtsa kɤ-nɯmŋu-t-a\cmn 我直接用嘴对着碗口喝了(水)\end{exemple}
\begin{relation-sémantique}\confer{
\hyperlink{Ⓔɯ-mŋu}{\textit{ \papi{ɯ-mŋu}}}
}\end{relation-sémantique}\end{entrée}

\begin{entrée}
\vedette{\hypertarget{Ⓔnɯmŋa}{\papi{ nɯmŋa}}}\markboth{nɯmŋa}{}
\classe{vi}
\paradigme{\textit{dir :} \jya tɤ-}
\begin{définition}\fra être impressionnant\end{définition}
\begin{définition}\cmn 醒目耀眼
\begin{déclaration} \étymologie{\papi{mŋa}}\end{déclaration}\end{définition}
\begin{exemple}\jya ɯ-ŋga ɲɯ-nɯmŋa\cmn 他的衣服醒目耀眼\end{exemple}\end{entrée}

\begin{entrée}
\vedette{\hypertarget{Ⓔnɯmpa}{\papi{ nɯmpa}}}\markboth{nɯmpa}{}\classe{vt}
\paradigme{\textit{dir :} \jya nɯ-}
\begin{définition}\fra s'occuper de\end{définition}
\begin{définition}\cmn 照顾\end{définition}
\begin{exemple}\jya tɤ-pɤtso ra kɤ-nɯmpa ɲɯ-ɴqa\cmn 照顾小孩子很难\end{exemple}
\begin{sous-entrée}
\vedette{\hypertarget{}{\papi{ sɤnɯmpa}}}\markboth{sɤnɯmpa}{}\classe{vi}
\begin{définition}\ 
\begin{déclaration}\grammar{apass}\end{déclaration}\end{définition}
\begin{définition}\fra s'occuper des gens\end{définition}
\begin{définition}\cmn 照顾别人\end{définition}
\end{sous-entrée}\begin{sous-entrée}
\vedette{\hypertarget{}{\papi{ ʑɣɤnɯmpa}}}\markboth{ʑɣɤnɯmpa}{}\classe{vi}
\begin{définition}\ 
\begin{déclaration}\grammar{refl}\end{déclaration}\end{définition}
\begin{définition}\fra s'occuper de soi-même\end{définition}
\begin{définition}\cmn 照顾自己\end{définition}
\begin{exemple}\jya kɤ-ʑɣɤnɯmpa pjɯ-kɯ-cha ra\cmn 一定要会照顾自己\end{exemple}
\end{sous-entrée}\end{entrée}

\begin{entrée}
\vedette{\hypertarget{Ⓔnɯmtɕhu}{\papi{ nɯmtɕhu}}}\markboth{nɯmtɕhu}{}
\classe{vt}
\paradigme{\textit{dir :} \jya tɤ-}
\begin{définition}\fra dire du mal\end{définition}
\begin{définition}\cmn 说别人的坏话\end{définition}
\begin{exemple}\jya jiɕqha nɯ kɯ a-qhu ɲɯ́-wɣ-nɯmtɕhu-a\cmn 那个人在我背后说我的坏话\end{exemple}\begin{sous-entrée}
\vedette{\hypertarget{}{\papi{ sɤnɯmtɕhu}}}\markboth{sɤnɯmtɕhu}{}\classe{vi}
\paradigme{\textit{dir :} \jya tɤ-}
\begin{définition}\fra dire du mal des gens\end{définition}
\begin{définition}\cmn 说别人的坏话\end{définition}
\begin{exemple}\jya jiɕqha nɯ ɲɯ-sɤnɯmtɕhu\cmn 那个人说别人的坏话\end{exemple}
\end{sous-entrée}\end{entrée}

\begin{entrée}
\vedette{\hypertarget{Ⓔnɯmtɕhɯtsaʁ}{\papi{ nɯmtɕhɯtsaʁ}}}\markboth{nɯmtɕhɯtsaʁ}{}\classe{vi}
\begin{définition}\fra avoir des ulcères sur la bouche\end{définition}
\begin{définition}\cmn 嘴上生疮\end{définition}
\begin{relation-sémantique}\confer{
\hyperlink{Ⓔmtɕhɯtsaʁ}{\textit{ \papi{mtɕhɯtsaʁ}}}
}\end{relation-sémantique}\end{entrée}

\begin{entrée}
\vedette{\hypertarget{Ⓔnɯmtɕi}{\papi{ nɯmtɕi}}}\markboth{nɯmtɕi}{}\classe{vi}
\paradigme{\textit{dir :} \jya thɯ-}
\begin{définition}\ 
\begin{déclaration}\grammar{denom}\end{déclaration}\end{définition}
\begin{définition}\fra tôt\end{définition}
\begin{définition}\cmn 起得早;来得早\end{définition}
\begin{exemple}\jya aʑo ɲɯ-nɯmtɕi-a, nɤʑo mɯ́j-tɯ-nɯmtɕi\cmn 我起得早,你起得晚\end{exemple}
\begin{exemple}\jya ɯʑo sɤskɯsku ʑo chɯ-nɯmtɕi ɲɯ-ŋu\cmn 他每天早上早起\end{exemple}
\begin{exemple}\jya fso tɕe a-tu-sɤ-nɯmtɕɯmtɕi, ʑa ku-nɯ-rŋgɯ-a ra\cmn 我为了明天早起,就要早点睡觉\end{exemple}
\begin{exemple}\jya pɣɤtɕɯ mɤ-kɯ-nɯtɕi qajɯ mɤ-aʁe\cmn 晚起的鸟吃不到虫子\end{exemple}
\begin{relation-sémantique}\confer{
\hyperlink{Ⓔtɯmtɕi}{\textit{ \papi{tɯmtɕi}}}
}\end{relation-sémantique}\end{entrée}

\begin{entrée}
\vedette{\hypertarget{ⒺnɯmthɯⒽ1}{\papi{ nɯmthɯ}}}\markboth{nɯmthɯ}{}\homonyme{1}
\classe{vt}
\paradigme{\textit{dir :} \jya tɤ-}
\begin{définition}\fra faire un bénéfice au dépend de\end{définition}
\begin{définition}\cmn 赚……的钱
\begin{déclaration} \étymologie{\papi{mtʰo}}\end{déclaration}\end{définition}
\begin{exemple}\jya tɤ-ta-nɯmthɯ\cmn 我赚了你的钱\end{exemple}
\begin{exemple}\jya tɤ́-wɣ-nɯmthɯ-a\cmn 他赚了我的钱\end{exemple}\begin{sous-entrée}
\vedette{\hypertarget{}{\papi{ sɤnɯmthɯ}}}\markboth{sɤnɯmthɯ}{}\classe{vi}
\begin{définition}\fra faire un bénéfice\end{définition}
\begin{définition}\cmn 赚别人的钱\end{définition}
\end{sous-entrée}\end{entrée}

\begin{entrée}
\vedette{\hypertarget{ⒺnɯmthɯⒽ2}{\papi{ nɯmthɯ}}}\markboth{nɯmthɯ}{}\homonyme{2}\classe{vt}
\paradigme{\textit{dir :} \jya thɯ-}
\begin{définition}\ 
\begin{déclaration}\grammar{denom}\end{déclaration}\end{définition}
\begin{définition}\fra maudire\end{définition}
\begin{définition}\cmn 诅咒(念咒经)\end{définition}
\begin{exemple}\jya cho-nɯmthɯ (=mthɯ cho-lɤt)\cmn (喇嘛)诅咒了他\end{exemple}
\begin{relation-sémantique}\confer{
\hyperlink{Ⓔmthɯ}{\textit{ \papi{mthɯ}}}
}\end{relation-sémantique}\end{entrée}

\begin{entrée}
\vedette{\hypertarget{Ⓔnɯmto}{\papi{ nɯmto}}}\markboth{nɯmto}{}\classe{vt}\acception{1}
\paradigme{\textit{dir :} \jya pɯ-}
\begin{définition}\fra trouver qqch par terre\end{définition}
\begin{définition}\cmn 捡东西\end{définition}
\begin{exemple}\jya laχtɕha pa-nɯmto\cmn 他捡到东西了\end{exemple}
\begin{exemple}\jya @gangbi pa-nɯmto\cmn 他捡到钢笔了\end{exemple}
\begin{exemple}\jya pɕawtsɯ pa-nɯmto\cmn 他捡到钱了\end{exemple}\acception{2}
\paradigme{\textit{dir :} \jya tɤ-}
\begin{définition}\fra viser\end{définition}
\begin{définition}\cmn 瞄准\end{définition}
\begin{exemple}\jya tɤfsɯr tɤ-nɯmto-t-a\cmn 我瞄准了靶子\end{exemple}
\begin{exemple}\jya tɤfsɯr ɲɯ-ɤz-nɯmto-nɯ\cmn 他们在瞄准靶子\end{exemple}
\begin{relation-sémantique}\confer{
 \papi{mto}
}\end{relation-sémantique}\end{entrée}

\begin{entrée}
\vedette{\hypertarget{Ⓔnɯmtshalu}{\papi{ nɯmtshalu}}}\markboth{nɯmtshalu}{}\classe{vi}
\begin{définition}\ 
\begin{déclaration}\grammar{denom}\end{déclaration}\end{définition}
\begin{définition}\fra ramasser des orties\end{définition}
\begin{définition}\cmn 找荨麻\end{définition}
\begin{relation-sémantique}\confer{
\hyperlink{Ⓔmtshalu}{\textit{ \papi{mtshalu}}}
}\end{relation-sémantique}
\end{entrée}

\begin{entrée}
\vedette{\hypertarget{Ⓔnɯna}{\papi{ nɯna}}}\markboth{nɯna}{}\classe{vi}
\paradigme{\textit{dir :} \jya tɤ-}
\begin{définition}\fra se reposer\end{définition}
\begin{définition}\cmn 休息\end{définition}
\begin{exemple}\jya tʂu tɤ-nɯna-a\cmn 我在路上休息了\end{exemple}
\begin{exemple}\jya ɲɯ-ɴqa tɕe tɤ-nɯna-a\cmn 很辛苦,我就休息了\end{exemple}
\begin{exemple}\jya jisŋi toʁde tɤ-nɯna tɕe, jɤ-anɯri\cmn 她今天(在我家)休息了一会就回去了\end{exemple}
\begin{exemple}\jya kɤ-nɯβlu mɤ-tɯ-cha, tɤ-nɯna\cmn 你休想骗我\end{exemple}
\begin{relation-sémantique}\confer{
\hyperlink{Ⓔznɯna}{\textit{ \papi{znɯna}}}
}\end{relation-sémantique}\end{entrée}

\begin{entrée}
\vedette{\hypertarget{Ⓔnɯndʐɯnbu}{\papi{ nɯndʐɯnbu}}}\markboth{nɯndʐɯnbu}{}\classe{vi}
\paradigme{\textit{dir :} \jya nɯ-}
\begin{définition}\fra partir de chez soi\end{définition}
\begin{définition}\cmn 出门;出差\end{définition}
\begin{exemple}\jya nɤʑo kɯsthɯci ʑo kɯ-ɤrqhi kɯ-nɯndʐɯnbu jo-tɯ-ɣi\cmn 你从这么远的地方出差来这里\end{exemple}
\begin{relation-sémantique}\confer{
\hyperlink{Ⓔndʐɯnbu}{\textit{ \papi{ndʐɯnbu}}}
}\end{relation-sémantique}\begin{sous-entrée}
\vedette{\hypertarget{}{\papi{ znɯndʐɯnbu}}}\markboth{znɯndʐɯnbu}{}\classe{vt}
\begin{définition}\ 
\begin{déclaration}\grammar{caus}\end{déclaration}\end{définition}
\begin{définition}\fra faire voyager\end{définition}
\begin{définition}\cmn 让……出差\end{définition}
\end{sous-entrée}\end{entrée}

\begin{entrée}
\vedette{\hypertarget{Ⓔnɯndzɤmbɣom}{\papi{ nɯndzɤmbɣom}}}\markboth{nɯndzɤmbɣom}{}\classe{vi}
\paradigme{\textit{dir :} \jya tɤ-}
\begin{définition}\ 
\begin{déclaration}\grammar{comp}\end{déclaration}\end{définition}
\begin{définition}\fra être pressé de manger\end{définition}
\begin{définition}\cmn 急着要吃;馋嘴\end{définition}
\begin{exemple}\jya tɤ-nɯndzɤmbɣom-a\cmn 我急着要吃了\end{exemple}
\begin{exemple}\jya nɤʑo nɤ-tɯ-nɯndzɤmbɣom nɯ\cmn 你很馋嘴\end{exemple}
\begin{relation-sémantique}\synonyme{
\hyperlink{Ⓔfkrɯz}{\textit{ \papi{fkrɯz}}}
}\end{relation-sémantique}
\begin{relation-sémantique}\confer{
\hyperlink{Ⓔndza}{\textit{ \papi{ndza}}}
}\end{relation-sémantique}
\begin{relation-sémantique}\confer{
\hyperlink{Ⓔmbɣom}{\textit{ \papi{mbɣom}}}
}\end{relation-sémantique}\end{entrée}

\begin{entrée}
\vedette{\hypertarget{Ⓔnɯndzɤmdɯm}{\papi{ nɯndzɤmdɯm}}}\markboth{nɯndzɤmdɯm}{}
\classe{vi}
\paradigme{\textit{dir :} \jya tɤ-}
\begin{définition}\ 
\begin{déclaration}\grammar{comp}\end{déclaration}\end{définition}
\begin{définition}\fra aimer manger des petites collations\end{définition}
\begin{définition}\cmn 爱吃零食\end{définition}
\begin{exemple}\jya ɲɯ-tɯ-nɯndzɤmdɯm\cmn 你爱吃零食\end{exemple}
\begin{exemple}\jya aj tɤ-nɯndzɤmdɯm-a\cmn 我爱吃零食\end{exemple}
\begin{exemple}\jya kɤ-nɯndzɤmdɯm χɕu\cmn 他最喜欢吃零食\end{exemple}
\begin{relation-sémantique}\confer{
\hyperlink{Ⓔnɯmdɯm}{\textit{ \papi{nɯmdɯm}}}
}\end{relation-sémantique}
\begin{relation-sémantique}\confer{
\hyperlink{Ⓔndza}{\textit{ \papi{ndza}}}
}\end{relation-sémantique}\end{entrée}

\begin{entrée}
\vedette{\hypertarget{Ⓔnɯndzɤqɤr}{\papi{ nɯndzɤqɤr}}}\markboth{nɯndzɤqɤr}{}
\classe{vt}
\paradigme{\textit{dir :} \jya pɯ-}
\begin{définition}\ 
\begin{déclaration}\grammar{comp}\end{déclaration}\end{définition}
\begin{définition}\fra ne pas laisser quelqu'un manger avec soi\end{définition}
\begin{définition}\cmn 不让别人吃
这里有食物,不让我吃,这种情况叫做\stylefv{kɤnɯndzɤqɤr}
\end{définition}
\begin{exemple}\jya pɯ-kɯ-nɯndzɤqar-a\cmn 你没有叫我吃\end{exemple}\begin{sous-entrée}
\vedette{\hypertarget{}{\papi{ anɯndzɤqɯqɤr}}}\markboth{anɯndzɤqɯqɤr}{}\classe{vi}
\begin{définition}\fra manger chacun dans son coin\end{définition}
\begin{définition}\cmn 各自吃各的\end{définition}
\begin{exemple}\jya tɯrme ʁnɯ-rdoʁ ma maŋe-tɕi tɕe, kɤ-ɤnɯndzɤqɯqɤr mɤ-nɯ-cha-tɕi\cmn 只有我们俩,不能各自吃的各的\end{exemple}
\begin{relation-sémantique}\confer{
\hyperlink{Ⓔndza}{\textit{ \papi{ndza}}}
}\end{relation-sémantique}
\begin{relation-sémantique}\confer{
\hyperlink{Ⓔqɤr}{\textit{ \papi{qɤr}}}
}\end{relation-sémantique}
\end{sous-entrée}\end{entrée}

\begin{entrée}
\vedette{\hypertarget{Ⓔnɯndzɤqhɤjɯ}{\papi{ nɯndzɤqhɤjɯ}}}\markboth{nɯndzɤqhɤjɯ}{}
\begin{relation-sémantique}\confer{
\hyperlink{Ⓔndzɤqhɤjɯ}{\textit{ \papi{ndzɤqhɤjɯ}}}
}\end{relation-sémantique}\end{entrée}

\begin{entrée}
\vedette{\hypertarget{Ⓔnɯndzɤsma}{\papi{ nɯndzɤsma}}}\markboth{nɯndzɤsma}{}
\classe{vi}
\begin{définition}\fra vouloir manger\end{définition}
\begin{définition}\cmn 想吃东西
\begin{déclaration}\use{原来生病很不想吃饭,病好了之后就想吃}\end{déclaration}\end{définition}
\begin{exemple}\jya tɤ-ngo-a tɕe ɲɯ-nɯndzɤsma-a\cmn 我病了,现在就想吃东西\end{exemple}
\begin{relation-sémantique}\confer{
\hyperlink{Ⓔndza}{\textit{ \papi{ndza}}}
}\end{relation-sémantique}
\begin{relation-sémantique}\confer{
\hyperlink{Ⓔnɤsma}{\textit{ \papi{nɤsma}}}
}\end{relation-sémantique}\end{entrée}

\begin{entrée}
\vedette{\hypertarget{Ⓔnɯndzom}{\papi{ nɯndzom}}}\markboth{nɯndzom}{}
\classe{vi}
\begin{définition}\fra couler le long\end{définition}
\begin{définition}\cmn 顺着某个东西流下来\end{définition}
\begin{exemple}\jya tɯ-ci nɯ sɯku ɯ-taʁ pjɤ-nɯndzom\cmn 水顺着树梢流下来了\end{exemple}
\begin{exemple}\jya tɯ-ci nɯ si ɯ-rtaʁ ɯ-taʁ pjɤ-nɯndzom\cmn 水顺着树枝流下来了\end{exemple}
\begin{exemple}\jya ɯ-kɤrme ɯ-taʁ tɯ-ci pjɤ-nɯndzom\cmn 水顺着他的头发流下来了\end{exemple}\end{entrée}

\begin{entrée}
\vedette{\hypertarget{Ⓔnɯndzɯ}{\papi{ nɯndzɯ}}}\markboth{nɯndzɯ}{}
\classe{vi}
\paradigme{\textit{dir :} \jya tɤ-}
\begin{définition}\fra vertical\end{définition}
\begin{définition}\cmn 竖\end{définition}\begin{sous-entrée}
\vedette{\hypertarget{}{\papi{ znɯndzɯ}}}\markboth{znɯndzɯ}{}\classe{vt}
\paradigme{\textit{dir :} \jya lɤ-}
\begin{définition}\ 
\begin{déclaration}\grammar{caus}\end{déclaration}\end{définition}
\begin{définition}\fra mettre à la verticale\end{définition}
\begin{définition}\cmn 竖起来\end{définition}
\begin{exemple}\jya laχtɕha lɤ-znɯndzɯ-t-a\cmn 我把东西竖起来了\end{exemple}
\begin{exemple}\jya ɕoŋtɕa lɤ-znɯndzɯ-t-a\cmn 我把木料竖起来了\end{exemple}
\begin{relation-sémantique}\synonyme{
\hyperlink{Ⓔftɕhur}{\textit{ \papi{ftɕhur}}}
}\end{relation-sémantique}
\end{sous-entrée}\end{entrée}

\begin{entrée}
\vedette{\hypertarget{Ⓔnɯndzɯlŋɯz}{\papi{ nɯndzɯlŋɯz}}}\markboth{nɯndzɯlŋɯz}{}\classe{vi}
\paradigme{\textit{dir :} \jya pɯ-}
\paradigme{\textit{dir :} \jya thɯ-}
\begin{définition}\fra somnoler\end{définition}
\begin{définition}\cmn 打瞌睡\end{définition}
\begin{exemple}\jya a-ʑɯβ ɲɯ-ɣi, pɯ-nɯndzɯlŋɯz-a\cmn 我想睡了,我在打瞌睡\end{exemple}
\begin{exemple}\jya ma-thɯ-tɯ-nɯndzɯlŋɯz\cmn 你不要打瞌睡\end{exemple}\begin{sous-entrée}
\vedette{\hypertarget{}{\papi{ ɣɤnɯndzɯlŋɯz}}}\markboth{ɣɤnɯndzɯlŋɯz}{}\classe{vs}
\begin{définition}\ 
\begin{déclaration}\grammar{facil}\end{déclaration}\end{définition}
\begin{définition}\fra somnoler facilement\end{définition}
\begin{définition}\cmn 容易打瞌睡\end{définition}
\begin{exemple}\jya ɲɯ-ɣɤnɯndzɯlŋɯz\cmn 他容易打瞌睡\end{exemple}
\end{sous-entrée}\end{entrée}

\begin{entrée}
\vedette{\hypertarget{Ⓔnɯni}{\papi{ nɯni}}}\markboth{nɯni}{}\classe{dem}
\begin{définition}\fra ces deux choses\end{définition}
\begin{définition}\cmn 那两个\end{définition}
\end{entrée}

\begin{entrée}
\vedette{\hypertarget{Ⓔnɯnŋɤtʂo}{\papi{ nɯnŋɤtʂo}}}\markboth{nɯnŋɤtʂo}{}
\classe{vi}
\paradigme{\textit{dir :} \jya nɯ-}
\paradigme{\textit{dir :} \jya pɯ-}
\begin{définition}\ 
\begin{déclaration}\grammar{incorp}\end{déclaration}\end{définition}
\begin{définition}\fra rembourser sa dette\end{définition}
\begin{définition}\cmn 还债\end{définition}
\begin{exemple}\jya nɯ-nɯnŋɤtʂo-a\cmn 我还了债\end{exemple}
\begin{relation-sémantique}\confer{
\hyperlink{Ⓔtɯ-nŋa}{\textit{ \papi{tɯ-nŋa}}}
}\end{relation-sémantique}
\begin{relation-sémantique}\confer{
\hyperlink{Ⓔtʂo}{\textit{ \papi{tʂo}}}
}\end{relation-sémantique}\end{entrée}

\begin{entrée}
\vedette{\hypertarget{Ⓔnɯno}{\papi{ nɯno}}}\markboth{nɯno}{}
\classe{vt}
\paradigme{\textit{dir :} \jya pɯ-}
\paradigme{\textit{dir :} \jya \_}
\begin{définition}\ 
\begin{déclaration}\grammar{vert}\end{déclaration}\end{définition}
\begin{définition}\fra ramener (le bétail) à la maison\end{définition}
\begin{définition}\cmn 把牲畜赶回家\end{définition}
\begin{exemple}\jya pɯ-nɯno-t-a\cmn 我(把牲畜)赶回家了\end{exemple}
\begin{relation-sémantique}\confer{
\hyperlink{Ⓔno}{\textit{ \papi{no}}}
}\end{relation-sémantique}\end{entrée}

\begin{entrée}
\vedette{\hypertarget{Ⓔnɯnthoʁnthɯɣ}{\papi{ nɯnthoʁnthɯɣ}}}\markboth{nɯnthoʁnthɯɣ}{}
\classe{vi}
\paradigme{\textit{dir :} \jya tɤ-}
\paradigme{\textit{dir :} \jya thɯ-}
\begin{définition}\fra ramasser les détritus\end{définition}
\begin{définition}\cmn 捡废物\end{définition}
\begin{exemple}\jya ɯ-thoʁ ra tɤ-nɯnthoʁnthɯɣ-a\cmn 我捡了地上的垃圾\end{exemple}
\begin{exemple}\jya hanɯni ɕ-tu-nɯnthoʁnthɯɣ-a nɤ\cmn 我要去捡一下!\end{exemple}
\begin{exemple}\jya kɤ-nɯthoʁnthɯɣ mɤ-ra\cmn 不要到处捡垃圾\end{exemple}\end{entrée}

\begin{entrée}
\vedette{\hypertarget{Ⓔnɯntsho}{\papi{ nɯntsho}}}\markboth{nɯntsho}{}
\classe{vt}
\paradigme{\textit{dir :} \jya thɯ-}
\paradigme{\textit{dir :} \jya nɯ-}
\begin{définition}\fra manger la viande sur les os\end{définition}
\begin{définition}\cmn 吃骨头上面剩下的肉\end{définition}
\begin{exemple}\jya ɕɤrɯ thɯ-nɯntshɤm\cmn 你把骨头上的肉吃了\end{exemple}
\begin{exemple}\jya ɕɤrɯ na-nɯntsho\cmn 他吃骨头上的肉\end{exemple}\begin{sous-entrée}
\vedette{\hypertarget{}{\papi{ znɯntsho}}}\markboth{znɯntsho}{}\classe{vt}
\paradigme{\textit{dir :} \jya nɯ-}
\begin{définition}\ 
\begin{déclaration}\grammar{caus}\end{déclaration}\end{définition}
\begin{définition}\fra faire manger la viande sur les os\end{définition}
\begin{définition}\cmn 令人吃骨头上面剩下的肉\end{définition}
\end{sous-entrée}\end{entrée}

\begin{entrée}
\vedette{\hypertarget{Ⓔnɯntsɯɣ}{\papi{ nɯntsɯɣ}}}\markboth{nɯntsɯɣ}{}
\classe{vt}
\paradigme{\textit{dir :} \jya tɤ-}
\begin{définition}\fra lécher\end{définition}
\begin{définition}\cmn 舔\end{définition}
\begin{exemple}\jya khɯtsa tɤ-nɯntsɯɣ-a\cmn 我舔了碗\end{exemple}
\begin{exemple}\jya khɯna kɯ ɯ-jŋgɯ to-nɯntsɯɣ\cmn 狗舔了它的碗\end{exemple}\begin{sous-entrée}
\vedette{\hypertarget{}{\papi{ znɯntsɯɣ}}}\markboth{znɯntsɯɣ}{}\classe{vt}
\begin{définition}\fra faire lécher\end{définition}
\begin{définition}\cmn 让……舔\end{définition}
\begin{exemple}\jya tɕhɯrkɯ ta-znɯntsɯɣ\cmn 让你舔狗碗(骂人的话)\end{exemple}
\end{sous-entrée}\end{entrée}

\begin{entrée}
\vedette{\hypertarget{ⒺnɯnɯⒽ2}{\papi{ nɯnɯ}}}\markboth{nɯnɯ}{}\homonyme{2}
\classe{dem}
\begin{définition}\fra celà\end{définition}
\begin{définition}\cmn 那个\end{définition}
\end{entrée}

\begin{entrée}
\vedette{\hypertarget{ⒺnɯnɯⒽ1}{\papi{ nɯnɯ}}}\markboth{nɯnɯ}{}\homonyme{1}\classe{vt}
\paradigme{\textit{dir :} \jya kɤ-}
\begin{définition}\ 
\begin{déclaration}\grammar{denom}\end{déclaration}\end{définition}
\begin{définition}\fra sucer, aspirer\end{définition}
\begin{définition}\cmn 吸; 吸吮\end{définition}
\begin{exemple}\jya chɤmdɤru kɤ-nɯnɯ-t-a\cmn 我吸了坛吸管(喝坛坛酒)\end{exemple}
\begin{relation-sémantique}\confer{
\hyperlink{Ⓔtɯ-nɯ}{\textit{ \papi{tɯ-nɯ}}}
}\end{relation-sémantique}\end{entrée}

\begin{entrée}
\vedette{\hypertarget{Ⓔnɯɲɤmkhe}{\papi{ nɯɲɤmkhe}}}\markboth{nɯɲɤmkhe}{}
\classe{vs}
\paradigme{\textit{dir :} \jya nɯ-}
\begin{définition}\ 
\begin{déclaration}\grammar{incorp}\end{déclaration}\end{définition}
\begin{définition}\fra maigre\end{définition}
\begin{définition}\cmn 瘦\end{définition}
\begin{exemple}\jya jiɕqha nɯ ɲɯ-nɯɲɤmkhe\cmn 那个很瘦\end{exemple}
\begin{exemple}\jya nɯ-fsapaʁ ɲɯ-nɯɲɤmkhe\cmn 他们的牲畜很瘦\end{exemple}\begin{sous-entrée}
\vedette{\hypertarget{}{\papi{ ʑɣɤznɯɲɤmkhe}}}\markboth{ʑɣɤznɯɲɤmkhe}{}\classe{vi}
\paradigme{\textit{dir :} \jya nɯ-}
\begin{définition}\ 
\begin{déclaration}\grammar{refl}\end{déclaration}
\begin{déclaration}\grammar{caus}\end{déclaration}\end{définition}
\begin{définition}\fra se faire maigrir\end{définition}
\begin{définition}\cmn 令自己变瘦\end{définition}
\begin{relation-sémantique}\antonyme{
\hyperlink{Ⓔnɯɲɤmsɯ}{\textit{ \papi{nɯɲɤmsɯ}}}
}\end{relation-sémantique}
\begin{relation-sémantique}\confer{
\hyperlink{Ⓔtɯ-ɲɤm}{\textit{ \papi{tɯ-ɲɤm}}}
}\end{relation-sémantique}
\begin{relation-sémantique}\confer{
\hyperlink{Ⓔkhe}{\textit{ \papi{khe}}}
}\end{relation-sémantique}
\end{sous-entrée}\end{entrée}

\begin{entrée}
\vedette{\hypertarget{Ⓔnɯɲɤmsɯ}{\papi{ nɯɲɤmsɯ}}}\markboth{nɯɲɤmsɯ}{}
\classe{vi}
\paradigme{\textit{dir :} \jya thɯ-}
\begin{définition}\ 
\begin{déclaration}\grammar{incorp}\end{déclaration}\end{définition}
\begin{définition}\fra gros, gras\end{définition}
\begin{définition}\cmn 肥;胖\end{définition}
\begin{exemple}\jya mbala ɲɯ-nɯɲɤmsɯ\cmn 牛很肥壮\end{exemple}
\begin{exemple}\jya mbala cho-nɯɲɤmsɯ\cmn 牛变胖了\end{exemple}
\begin{relation-sémantique}\antonyme{
\hyperlink{Ⓔnɯɲɤmkhe}{\textit{ \papi{nɯɲɤmkhe}}}
}\end{relation-sémantique}
\begin{relation-sémantique}\confer{
\hyperlink{Ⓔtɯ-ɲɤm}{\textit{ \papi{tɯ-ɲɤm}}}
}\end{relation-sémantique}\end{entrée}

\begin{entrée}
\vedette{\hypertarget{ⒺnɯŋaⒽ1}{\papi{ nɯŋa}}}\markboth{nɯŋa}{}\homonyme{1}
\classe{n}
\begin{définition}\fra vache\end{définition}
\begin{définition}\cmn 母牛\end{définition}
\end{entrée}

\begin{entrée}
\vedette{\hypertarget{ⒺnɯŋaⒽ2}{\papi{ nɯŋa}}}\markboth{nɯŋa}{}\homonyme{2}\classe{vt}
\paradigme{\textit{dir :} \jya tɤ-}
\begin{définition}\fra accepter de faire qqch pour qqn\end{définition}
\begin{définition}\cmn 答应为别人做事\end{définition}
\begin{exemple}\jya tɤ-nɯŋa-t-a\cmn 我答应了\end{exemple}
\begin{exemple}\jya nɤ-kɯ-qur tɤ-nɯŋa-t-a\cmn 我答应要帮你了\end{exemple}\end{entrée}

\begin{entrée}
\vedette{\hypertarget{Ⓔnɯŋɤdo}{\papi{ nɯŋɤdo}}}\markboth{nɯŋɤdo}{}\classe{n}
\begin{définition}\fra vieille vache\end{définition}
\begin{définition}\cmn 老奶牛\end{définition}\end{entrée}

\begin{entrée}
\vedette{\hypertarget{Ⓔnɯŋgu}{\papi{ nɯŋgu}}}\markboth{nɯŋgu}{}
\classe{vs}
\paradigme{\textit{dir :} \jya \_}
\paradigme{\textit{dir :} \jya \_}
\begin{définition}\fra prématuré\end{définition}
\begin{définition}\cmn 过早\end{définition}
\begin{exemple}\jya saχsɯ to-nɯŋgu\cmn 午餐吃得太早\end{exemple}
\begin{exemple}\jya kɤ-ji lo-nɯŋgu\cmn 他种得太早\end{exemple}
\begin{exemple}\jya (tɤ-rɤku) kɤ-phɯt ko-nɯŋgu\cmn (庄稼)割得太早\end{exemple}
\begin{exemple}\jya jɯxɕo aʑo kɤ-ɣi ko-nɯŋgu-a\cmn 我今天早上来早了\end{exemple}\begin{sous-entrée}
\vedette{\hypertarget{}{\papi{ znɯŋgu}}}\markboth{znɯŋgu}{}\classe{vt}
\paradigme{\textit{dir :} \jya tɤ-}
\begin{définition}\fra faire de façon prématurée\end{définition}
\begin{définition}\cmn 做得太早\end{définition}
\begin{exemple}\jya tɤ-znɯŋgu-t-a\cmn 我做得太早了\end{exemple}
\end{sous-entrée}\end{entrée}

\begin{entrée}
\vedette{\hypertarget{Ⓔnɯŋgɤkhe}{\papi{ nɯŋgɤkhe}}}\markboth{nɯŋgɤkhe}{}\classe{vi}
\begin{définition}\fra porter habituellement de vieux habits\end{définition}
\begin{définition}\cmn (习惯)穿破旧的衣服\end{définition}
\begin{exemple}\jya aʑo nɯŋgɤkhe-a\cmn 我习惯穿破烂的衣服\end{exemple}
\begin{relation-sémantique}\confer{
\hyperlink{Ⓔtɯ-ŋga}{\textit{ \papi{tɯ-ŋga}}}
}\end{relation-sémantique}
\begin{relation-sémantique}\confer{
\hyperlink{Ⓔkhe}{\textit{ \papi{khe}}}
}\end{relation-sémantique}\end{entrée}

\begin{entrée}
\vedette{\hypertarget{Ⓔnɯŋgɤxtsa}{\papi{ nɯŋgɤxtsa}}}\markboth{nɯŋgɤxtsa}{}
\classe{vt}
\paradigme{\textit{dir :} \jya tɤ-}
\begin{définition}\fra s'habiller richement, être prêt à faire des dépenses dans les habits\end{définition}
\begin{définition}\cmn 穿得很豪华\end{définition}
\begin{exemple}\jya tɤ-nɯŋgɤxtsa-t-a\cmn 我舍得穿了\end{exemple}
\begin{exemple}\jya sɯŋgi kɤ-nɯŋgɤxtsa cha, mgɯnbu mɤ-nɯŋgɤxtse\cmn 僧吉舍得穿,袞布不舍得\end{exemple}
\begin{exemple}\jya tɯ-ŋga\end{exemple}
\begin{exemple}\jya tɯ-xtsa\end{exemple}\end{entrée}

\begin{entrée}
\vedette{\hypertarget{Ⓔnɯŋgumdʑɯɣ}{\papi{ nɯŋgumdʑɯɣ}}}\markboth{nɯŋgumdʑɯɣ}{}\classe{vi}
\paradigme{\textit{dir :} \jya lɤ-}
\begin{définition}\fra devenir chef\end{définition}
\begin{définition}\cmn 当领导\end{définition}
\begin{exemple}\jya lo-nɯŋgumdʑɯɣ\cmn 他当了领导\end{exemple}
\begin{relation-sémantique}\confer{
 \papi{ŋgumdʑɯɣ}
}\end{relation-sémantique}\end{entrée}

\begin{entrée}
\vedette{\hypertarget{Ⓔnɯŋgumtha}{\papi{ nɯŋgumtha}}}\markboth{nɯŋgumtha}{}
\classe{vt}
\paradigme{\textit{dir :} \jya tɤ-}
\begin{définition}\fra s'occuper de\end{définition}
\begin{définition}\cmn 照顾\end{définition}
\begin{exemple}\jya tɤ-pɤtso tɤ-nɯŋgumtha-t-a\cmn 我照顾了孩子\end{exemple}
\begin{exemple}\jya rgargɯn ɲɯ-ŋu tɕe tɤ-nɯŋgumtha-t-a\cmn 我照顾了老人家\end{exemple}
\begin{exemple}\jya ɲɯ-ɲɯ-βzi nɤ kɯpɯpe tɤ-nɯŋgumthe\cmn 如果他醉了的话,请你好好照顾他!\end{exemple}\end{entrée}

\begin{entrée}
\vedette{\hypertarget{Ⓔnɯŋgra}{\papi{ nɯŋgra}}}\markboth{nɯŋgra}{}\classe{vi}
\paradigme{\textit{dir :} \jya tɤ-}
\begin{définition}\ 
\begin{déclaration}\grammar{denom}\end{déclaration}\end{définition}
\begin{définition}\fra être payé pour un travail\end{définition}
\begin{définition}\cmn 拿到工钱\end{définition}
\begin{exemple}\jya laχtɕha kɤ-tsɯm tɤ-nɯŋgra-a\cmn 我把东西拿去了,得到了工钱\end{exemple}\begin{sous-entrée}
\vedette{\hypertarget{}{\papi{ znɯŋgra}}}\markboth{znɯŋgra}{}\classe{vt}
\paradigme{\textit{dir :} \jya tɤ-}
\begin{définition}\ 
\begin{déclaration}\grammar{caus}\end{déclaration}\end{définition}\acception{1}
\begin{définition}\fra engager\end{définition}
\begin{définition}\cmn 雇佣\end{définition}
\begin{exemple}\jya ɯʑo kɯ tɯrme ci a-tɤ-znɯŋgre ɲɯ-ntshi\cmn 他只好雇佣人\end{exemple}\acception{2}
\begin{définition}\fra louer\end{définition}
\begin{définition}\cmn 租(房子)\end{définition}
\begin{exemple}\jya kha ci to-znɯŋgra\cmn 他租了房子\end{exemple}
\begin{relation-sémantique}\confer{
\hyperlink{Ⓔtɯ-ŋgra}{\textit{ \papi{tɯ-ŋgra}}}
}\end{relation-sémantique}
\end{sous-entrée}\end{entrée}

\begin{entrée}
\vedette{\hypertarget{Ⓔnɯŋgurtɕaʁ}{\papi{ nɯŋgurtɕaʁ}}}\markboth{nɯŋgurtɕaʁ}{}\classe{vt}
\paradigme{\textit{dir :} \jya pɯ-}
\begin{définition}\fra coudre selon un type de pas d'aiguille\end{définition}
\begin{définition}\cmn 缝针的方法\end{définition}
\begin{relation-sémantique}\confer{
\hyperlink{Ⓔŋgurtɕaʁ}{\textit{ \papi{ŋgurtɕaʁ}}}
}\end{relation-sémantique}\end{entrée}

\begin{entrée}
\vedette{\hypertarget{Ⓔnɯŋke}{\papi{ nɯŋke}}}\markboth{nɯŋke}{}
\classe{vt}
\paradigme{\textit{dir :} \jya pɯ-}
\paradigme{\textit{dir :} \jya \_}
\begin{définition}\ 
\begin{déclaration}\grammar{appl}\end{déclaration}\end{définition}
\begin{définition}\fra aller pour faire quelque chose\end{définition}
\begin{définition}\cmn 到处走做某件事情
\begin{déclaration}\use{不能把\stylefv{nɯŋke}“走路去做”跟\stylefv{ŋke}的为己式\stylefv{nɯŋke}相混淆,前者是及物动词,后缀是不及物动词}\end{déclaration}\end{définition}
\begin{exemple}\jya @yangyu kɤ-χtɯ ɕ-pɯ-nɯŋke-t-a\cmn 我为了去买土豆走了一趟\end{exemple}
\begin{exemple}\jya tɯ-ŋga kɤ-χtɯ ɕ-pɯ-nɯŋke-t-a\cmn 我为了去买衣服走了一趟\end{exemple}
\begin{exemple}\jya smɤnba ɕ-pɯ-nɯŋke-t-a\cmn 我为了找医生走了一趟\end{exemple}
\begin{relation-sémantique}\confer{
\hyperlink{Ⓔŋke}{\textit{ \papi{ŋke}}}
}\end{relation-sémantique}\end{entrée}

\begin{entrée}
\vedette{\hypertarget{Ⓔnɯŋumit}{\papi{ nɯŋumit}}}\markboth{nɯŋumit}{}
\classe{vt}
\paradigme{\textit{dir :} \jya tɤ-}
\begin{définition}\fra humilier\end{définition}
\begin{définition}\cmn 侮辱;欺负
\begin{déclaration}\use{不能用于动物}\end{déclaration}
\begin{déclaration} \étymologie{\papi{ŋo.med}}\end{déclaration}\end{définition}
\begin{exemple}\jya jiɕqha tɤ-pɤtso nɯ ɲɯ-ɤz-nɯŋumit-nɯ\cmn 他们在欺负那个小孩子\end{exemple}
\begin{exemple}\jya jiɕqha tɯrme ɲɯ-ɤz-nɯŋumit-nɯ\cmn 他们在欺负那个人\end{exemple}
\begin{exemple}\jya mɤ-ta-nɤkhe, mɤ-ta-nɯŋumit\cmn 我不会欺负你的\end{exemple}\begin{sous-entrée}
\vedette{\hypertarget{}{\papi{ sɤnɯŋumit}}}\markboth{sɤnɯŋumit}{}\classe{vi}
\begin{définition}\ 
\begin{déclaration}\grammar{apass}\end{déclaration}\end{définition}
\begin{définition}\fra humilier les gens\end{définition}
\begin{définition}\cmn 侮辱人\end{définition}
\end{sous-entrée}\end{entrée}

\begin{entrée}
\vedette{\hypertarget{Ⓔnɯŋundʑu}{\papi{ nɯŋundʑu}}}\markboth{nɯŋundʑu}{}
\classe{vt}
\paradigme{\textit{dir :} \jya tɤ-}\acception{1}
\begin{définition}\fra attirer (animal)\end{définition}
\begin{définition}\cmn 引过来(动物)\end{définition}
\begin{exemple}\jya jla tɤ-nɯŋundʑu-t-a\cmn 我把犏牛引过来了(用盐)\end{exemple}\acception{2}
\begin{définition}\fra calmer, apaiser (quelqu'un qui est fâché)\end{définition}
\begin{définition}\cmn 说几句好话,令别人没有那么生气\end{définition}
\begin{exemple}\jya ɯ-mbrɯ ɲɯ-ŋgɯ tɕe, tɤ-nɯŋundʑu-t-a tɕe nɯ-ʑi\cmn 我说了几句好话,他就平静下来了\end{exemple}
\begin{exemple}\jya tɤ́-wɣ-nɯŋundʑu-a\cmn 他跟我说了几句好话\end{exemple}
\begin{relation-sémantique}\confer{
\hyperlink{Ⓔrɯŋundʑu}{\textit{ \papi{rɯŋundʑu}}}
}\end{relation-sémantique}\end{entrée}

\begin{entrée}
\vedette{\hypertarget{Ⓔnɯɴqhu}{\papi{ nɯɴqhu}}}\markboth{nɯɴqhu}{}\classe{vt}
\paradigme{\textit{dir :} \jya \_}
\begin{définition}\ 
\begin{déclaration}\grammar{denom}\end{déclaration}\end{définition}
\begin{définition}\fra suivre\end{définition}
\begin{définition}\cmn 跟踪(偷偷地)\end{définition}
\begin{exemple}\jya kɤ-anɯri tɕe ɯ-qhu kɤ-nɯɴqhu-t-a\cmn 我回去了,我就跟踪了他\end{exemple}\begin{sous-entrée}
\vedette{\hypertarget{}{\papi{ znɯɴqhu}}}\markboth{znɯɴqhu}{}\classe{vt}
\paradigme{\textit{dir :} \jya pɯ-}
\paradigme{\textit{dir :} \jya nɯ-}
\begin{définition}\fra suivre, se conformer à\end{définition}
\begin{définition}\cmn 照办\end{définition}
\begin{exemple}\jya nɤʑo nɤ-kɤti nɯ pjɯ-znɯɴqhe-a ŋu\cmn 我依着你的说法去做\end{exemple}
\begin{relation-sémantique}\confer{
\hyperlink{Ⓔɯ-qhu}{\textit{ \papi{ɯ-qhu}}}
}\end{relation-sémantique}
\begin{relation-sémantique}\synonyme{
\hyperlink{Ⓔznɯjɯn}{\textit{ \papi{znɯjɯn}}}
}\end{relation-sémantique}
\end{sous-entrée}\end{entrée}

\begin{entrée}
\vedette{\hypertarget{Ⓔnɯɴɢɯlɯjɤt}{\papi{ nɯɴɢɯlɯjɤt}}}\markboth{nɯɴɢɯlɯjɤt}{}\classe{vi}
\paradigme{\textit{dir :} \jya nɯ-}
\begin{définition}\fra se séparer\end{définition}
\begin{définition}\cmn 分散;走散\end{définition}
\begin{exemple}\jya nɯ-nɯɴɢɯlɯjɤt-i\cmn 我们走散了\end{exemple}
\begin{relation-sémantique}\synonyme{
\hyperlink{Ⓔɴɢɤt}{\textit{ \papi{ɴɢɤt}}}
}\end{relation-sémantique}\end{entrée}

\begin{entrée}
\vedette{\hypertarget{Ⓔnɯpa}{\papi{ nɯpa}}}\markboth{nɯpa}{}
\begin{relation-sémantique}\confer{
\hyperlink{ⒺpaⒽ1}{\textit{ \papi{pa}}}
}\end{relation-sémantique}\end{entrée}

\begin{entrée}
\vedette{\hypertarget{Ⓔnɯpaʁlɤɣ}{\papi{ nɯpaʁlɤɣ}}}\markboth{nɯpaʁlɤɣ}{}
\classe{vi}
\paradigme{\textit{dir :} \jya nɯ-}
\begin{définition}\ 
\begin{déclaration}\grammar{incorp}\end{déclaration}\end{définition}
\begin{définition}\fra laisser sortir un cochon\end{définition}
\begin{définition}\cmn 放猪
\begin{déclaration}\use{尕脚方言}\end{déclaration}\end{définition}
\begin{exemple}\jya nɯ-nɯpaʁlɤɣ\cmn 他放了猪\end{exemple}
\begin{exemple}\jya nɯ-nɯpaʁlaɣ-a\cmn 我放了猪\end{exemple}
\begin{relation-sémantique}\confer{
\hyperlink{Ⓔpaʁ}{\textit{ \papi{paʁ}}}
}\end{relation-sémantique}
\begin{relation-sémantique}\confer{
\hyperlink{Ⓔlɤɣ}{\textit{ \papi{lɤɣ}}}
}\end{relation-sémantique}\end{entrée}

\begin{entrée}
\vedette{\hypertarget{Ⓔnɯpaχɕi}{\papi{ nɯpaχɕi}}}\markboth{nɯpaχɕi}{}\classe{vi}
\begin{définition}\fra aller cueillir des pommes\end{définition}
\begin{définition}\cmn 采苹果\end{définition}
\begin{relation-sémantique}\confer{
 \papi{paχɕi}
}\end{relation-sémantique}\end{entrée}

\begin{entrée}
\vedette{\hypertarget{Ⓔnɯpɤŋgɯŋgru}{\papi{ nɯpɤŋgɯŋgru}}}\markboth{nɯpɤŋgɯŋgru}{}\classe{vs}
\paradigme{\textit{dir :} \jya nɯ-}
\begin{définition}\fra avoir une crampe\end{définition}
\begin{définition}\cmn 抽筋\end{définition}
\begin{exemple}\jya a-mi ɲɯ-nɯpɤŋgɯŋgru ɲɯ-ŋu\cmn 我的脚经常抽筋\end{exemple}
\begin{exemple}\jya nɤ-mi nɯ-nɯpɤŋgɯŋgru tɕe, ɯ-thoʁ pɯ-te tɕe pɯ-sthoʁ ʑo tɕe phɤn\cmn 脚抽筋的时候,把脚板着地,使劲地蹬就会好\end{exemple}
\begin{relation-sémantique}\confer{
\hyperlink{Ⓔtɯ-ŋgru}{\textit{ \papi{tɯ-ŋgru}}}
}\end{relation-sémantique}\end{entrée}

\begin{entrée}
\vedette{\hypertarget{Ⓔnɯpɤɴqi}{\papi{ nɯpɤɴqi}}}\markboth{nɯpɤɴqi}{}\classe{vs}
\paradigme{\textit{dir :} \jya nɯ-}
\begin{définition}\fra paresseux\end{définition}
\begin{définition}\cmn 懒\end{définition}
\begin{relation-sémantique}\confer{
\hyperlink{Ⓔnɤɴqi}{\textit{ \papi{nɤɴqi}}}
}\end{relation-sémantique}\end{entrée}

\begin{entrée}
\vedette{\hypertarget{Ⓔnɯpɕɯru}{\papi{ nɯpɕɯru}}}\markboth{nɯpɕɯru}{}\classe{vs}
\begin{définition}\fra agréable à regarder\end{définition}
\begin{définition}\cmn 外表好看\end{définition}
\begin{exemple}\jya ɯ-ʁzɯɣ ndɤre ɲɯ-nɯpɕɯru\cmn 她的外表倒是美观\end{exemple}
\begin{relation-sémantique}\confer{
\hyperlink{ⒺruⒽ1}{\textit{ \papi{ru1}}}
}\end{relation-sémantique}\end{entrée}

\begin{entrée}
\vedette{\hypertarget{Ⓔnɯpɣa}{\papi{ nɯpɣa}}}\markboth{nɯpɣa}{}\classe{vi}
\paradigme{\textit{dir :} \jya pɯ-}
\begin{définition}\fra chasser des oiseaux\end{définition}
\begin{définition}\cmn 打鸟\end{définition}
\begin{exemple}\jya ɯʑo ɕ-pɯ-nɯpɣa\cmn 他去打鸟了\end{exemple}
\begin{relation-sémantique}\confer{
\hyperlink{Ⓔpɣa}{\textit{ \papi{pɣa}}}
}\end{relation-sémantique}\end{entrée}

\begin{entrée}
\vedette{\hypertarget{Ⓔnɯpɣɤɲaʁ}{\papi{ nɯpɣɤɲaʁ}}}\markboth{nɯpɣɤɲaʁ}{}\classe{vi}
\paradigme{\textit{dir :} \jya pɯ-}
\begin{définition}\fra chasser le faisan\end{définition}
\begin{définition}\cmn 打勺鸡\end{définition}
\begin{exemple}\jya ɕ-pɯ-nɯpɣɤɲaʁ-a\cmn 我去打勺鸡\end{exemple}
\begin{relation-sémantique}\confer{
\hyperlink{Ⓔpɣɤɲaʁ}{\textit{ \papi{pɣɤɲaʁ}}}
}\end{relation-sémantique}\end{entrée}

\begin{entrée}
\vedette{\hypertarget{Ⓔnɯphu}{\papi{ nɯphu}}}\markboth{nɯphu}{}\classe{vi}
\paradigme{\textit{dir :} \jya tɤ-}
\begin{définition}\fra s'accoupler\end{définition}
\begin{définition}\cmn 交配\end{définition}\end{entrée}

\begin{entrée}
\vedette{\hypertarget{Ⓔnɯphaʁɲɤl}{\papi{ nɯphaʁɲɤl}}}\markboth{nɯphaʁɲɤl}{}
\classe{vi}
\paradigme{\textit{dir :} \jya nɯ-}
\paradigme{\textit{dir :} \jya \_}
\begin{définition}\ 
\begin{déclaration}\grammar{incorp}\end{déclaration}\end{définition}
\begin{définition}\fra s’allonger\end{définition}
\begin{définition}\cmn 躺\end{définition}
\begin{exemple}\jya ma-nɯ-tɯ-nɯphaʁɲɤl\cmn 你不要躺下\end{exemple}
\begin{exemple}\jya ɲɯ-tɯ-nɤɴqi tɕe ɲɯ-tɯ-nɯphaʁɲɤl\cmn 你很懒,你在那里躺着\end{exemple}
\begin{exemple}\jya pɯ-tɯ-nɯphaʁɲɤl ntsɯ\cmn 你总是在那里躺着\end{exemple}
\begin{exemple}\jya tɤ-tɕɯ nɯ ɯ-rkɯ nɯ tɕu pjɤ-nɯphaʁɲɤl\cmn 那个男子在旁边躺着\end{exemple}
\begin{relation-sémantique}\confer{
\hyperlink{Ⓔphaʁɲɤl}{\textit{ \papi{phaʁɲɤl}}}
}\end{relation-sémantique}\end{entrée}

\begin{entrée}
\vedette{\hypertarget{Ⓔnɯphawu}{\papi{ nɯphawu}}}\markboth{nɯphawu}{}
\classe{vt}
\paradigme{\textit{dir :} \jya nɯ-}
\begin{définition}\fra dépendre de, profiter (du l'influence d'autres gens)\end{définition}
\begin{définition}\cmn 依靠别人;借别人的势力\end{définition}
\begin{exemple}\jya ɯ-βɣo ɲɯ-ɤz-nɯphawu\cmn 他在依赖他的伯父\end{exemple}
\begin{exemple}\jya ɯ-wa ɲɯ-ɤz-nɯphawu\cmn 他在依赖他的父亲\end{exemple}
\begin{exemple}\jya tɤru ɲɯ-ɤz-nɯphawu\cmn 他在依赖头人\end{exemple}
\begin{exemple}\jya nɤ-wa ma-nɯ-tɯ-nɯphawe\cmn 你不要依赖你的父亲\end{exemple}\end{entrée}

\begin{entrée}
\vedette{\hypertarget{Ⓔnɯphɯ}{\papi{ nɯphɯ}}}\markboth{nɯphɯ}{}
\classe{vs}
\paradigme{\textit{dir :} \jya tɤ-}
\begin{définition}\fra d'un prix convenable\end{définition}
\begin{définition}\cmn 价格合适\end{définition}
\begin{exemple}\jya ɲɯ-nɯphɯ\cmn 价格合适\end{exemple}
\begin{exemple}\jya ki ɯ-phɯ ɯ-tɯ-wxti nɯ nɯ sthɯci ndɤre mɯ́j-nɯphɯ\cmn 这个东西很贵,价格太高了\end{exemple}
\begin{relation-sémantique}\confer{
\hyperlink{Ⓔɯ-phɯ}{\textit{ \papi{ɯ-phɯ}}}
}\end{relation-sémantique}\end{entrée}

\begin{entrée}
\vedette{\hypertarget{Ⓔnɯphɯrɤm}{\papi{ nɯphɯrɤm}}}\markboth{nɯphɯrɤm}{}\classe{vt}
\paradigme{\textit{dir :} \jya thɯ-}
\begin{définition}\fra herser\end{définition}
\begin{définition}\cmn 耙地\end{définition}
\begin{exemple}\jya ki tɯji ki thɯ-nɯphɯram-a\cmn 我耙了这块地\end{exemple}\begin{sous-entrée}
\vedette{\hypertarget{}{\papi{ rɯphɯrɤm}}}\markboth{rɯphɯrɤm}{}\classe{vi}
\paradigme{\textit{dir :} \jya thɯ-}
\begin{définition}\fra herser\end{définition}
\begin{définition}\cmn 耙地\end{définition}
\begin{relation-sémantique}\confer{
\hyperlink{Ⓔphɯrɤm}{\textit{ \papi{phɯrɤm}}}
}\end{relation-sémantique}
\end{sous-entrée}\end{entrée}

\begin{entrée}
\vedette{\hypertarget{Ⓔnɯpjaχpa}{\papi{ nɯpjaχpa}}}\markboth{nɯpjaχpa}{}\classe{vt}
\paradigme{\textit{dir :} \jya tɤ-}\acception{1}
\begin{définition}\fra tenir sous l'aisselle\end{définition}
\begin{définition}\cmn 夹在腋下\end{définition}
\begin{exemple}\jya ɯʑo kɯ jɯɣi to-nɯpjaχpa\cmn 他把书夹在腋下\end{exemple}\acception{2}
\begin{définition}\fra en profiter pour prendre\end{définition}
\begin{définition}\cmn 顺便带\end{définition}
\begin{définition}\jya \end{définition}
\begin{relation-sémantique}\synonyme{
\hyperlink{Ⓔnɤxtʂɯ}{\textit{ \papi{nɤxtʂɯ}}}
}\end{relation-sémantique}\end{entrée}

\begin{entrée}
\vedette{\hypertarget{Ⓔnɯpodɯdi}{\papi{ nɯpodɯdi}}}\markboth{nɯpodɯdi}{}
\classe{vs}
\begin{définition}\fra chatouiller\end{définition}
\begin{définition}\cmn 腰、胳肢窝发痒【入劲】\end{définition}\begin{sous-entrée}
\vedette{\hypertarget{}{\papi{ znɯpodɯdi}}}\markboth{znɯpodɯdi}{}\classe{vt}
\paradigme{\textit{dir :} \jya nɯ-}
\begin{définition}\ 
\begin{déclaration}\grammar{caus}\end{déclaration}\end{définition}
\begin{définition}\fra chatouiller\end{définition}
\begin{définition}\cmn 挠(别人)痒\end{définition}
\begin{exemple}\jya nɯ́-wɣ-znɯpodɯdi-a\cmn 他挠我痒痒了\end{exemple}
\begin{exemple}\jya nɯ-znɯpodɯdi-t-a\cmn 我挠他痒痒了\end{exemple}
\end{sous-entrée}\end{entrée}

\begin{entrée}
\vedette{\hypertarget{Ⓔnɯpolɯli}{\papi{ nɯpolɯli}}}\markboth{nɯpolɯli}{}
\classe{vi}
\paradigme{\textit{dir :} \jya thɯ-}
\begin{définition}\fra s'allonger sur le ventre\end{définition}
\begin{définition}\cmn 俯卧,趴\end{définition}
\begin{exemple}\jya khɤxtu zɯ chɤ-nɯpolɯli\cmn 他趴在房背上了\end{exemple}
\begin{exemple}\jya stɤmku ri chɤ-nɯpolɯli\cmn 他趴在草地上了\end{exemple}\end{entrée}

\begin{entrée}
\vedette{\hypertarget{Ⓔnɯpoʁ}{\papi{ nɯpoʁ}}}\markboth{nɯpoʁ}{}\classe{vt}
\paradigme{\textit{dir :} \jya kɤ-}
\begin{définition}\fra embrasser\end{définition}
\begin{définition}\cmn 亲吻\end{définition}
\begin{exemple}\jya ɯ-rʑaβ ka-nɯpoʁ\cmn 他把妻子亲了一下了\end{exemple}
\begin{exemple}\jya tɤ-pɤtso ka-nɯpoʁ\cmn 他把孩子亲了一下\end{exemple}
\begin{exemple}\jya @waiguoren ra kɤ-nɯpoʁ rga-nɯ\cmn 外国人喜欢亲嘴\end{exemple}
\begin{exemple}\jya kɤ́-wɣ-nɯpoʁ-a\cmn 他亲了我一下\end{exemple}\begin{sous-entrée}
\vedette{\hypertarget{}{\papi{ znɯpoʁ}}}\markboth{znɯpoʁ}{}\classe{vt}
\paradigme{\textit{dir :} \jya kɤ-}
\begin{définition}\fra laisser embrasser\end{définition}
\begin{définition}\cmn 让……亲吻\end{définition}
\end{sous-entrée}\end{entrée}

\begin{entrée}
\vedette{\hypertarget{Ⓔnɯproʁmba}{\papi{ nɯproʁmba}}}\markboth{nɯproʁmba}{}\classe{vl}
\paradigme{\textit{dir :} \jya tɤ-}
\begin{définition}\fra imiter les gestes\end{définition}
\begin{définition}\cmn 学别人的动作\end{définition}
\begin{exemple}\jya ma-tɤ-kɯ-nɯproʁmba-a\cmn 你不要学我!\end{exemple}\end{entrée}

\begin{entrée}
\vedette{\hypertarget{Ⓔnɯpɯmɲɯɣ}{\papi{ nɯpɯmɲɯɣ}}}\markboth{nɯpɯmɲɯɣ}{}\classe{vi}
\paradigme{\textit{dir :} \jya tɤ-}
\begin{définition}\fra viser\end{définition}
\begin{définition}\cmn 瞄准\end{définition}
\begin{exemple}\jya tɤ-nɯpɯmɲɯɣ-a\cmn 我瞄准了\end{exemple}
\end{entrée}

\begin{entrée}
\vedette{\hypertarget{Ⓔnɯqaɕti}{\papi{ nɯqaɕti}}}\markboth{nɯqaɕti}{}\classe{vi}
\paradigme{\textit{dir :} \jya \_}
\begin{définition}\fra aller chercher des pêches\end{définition}
\begin{définition}\cmn 捡桃子\end{définition}
\begin{relation-sémantique}\confer{
\hyperlink{Ⓔqaɕti}{\textit{ \papi{qaɕti}}}
}\end{relation-sémantique}\end{entrée}

\begin{entrée}
\vedette{\hypertarget{Ⓔnɯqajɯ}{\papi{ nɯqajɯ}}}\markboth{nɯqajɯ}{}\classe{vi}
\paradigme{\textit{dir :} \jya pɯ-}
\paradigme{\textit{dir :} \jya lɤ-}
\paradigme{\textit{dir :} \jya nɯ-}
\begin{définition}\fra chercher des vers\end{définition}
\begin{définition}\cmn 找虫子\end{définition}
\begin{relation-sémantique}\synonyme{
\hyperlink{Ⓔnɯqandʐe}{\textit{ \papi{nɯqandʐe}}}
}\end{relation-sémantique}
\begin{relation-sémantique}\confer{
\hyperlink{Ⓔrɯqajɯ}{\textit{ \papi{rɯqajɯ}}}
}\end{relation-sémantique}
\begin{relation-sémantique}\confer{
\hyperlink{Ⓔqajɯ}{\textit{ \papi{qajɯ}}}
}\end{relation-sémantique}\end{entrée}

\begin{entrée}
\vedette{\hypertarget{Ⓔnɯqaɟy}{\papi{ nɯqaɟy}}}\markboth{nɯqaɟy}{}
\classe{vi}
\paradigme{\textit{dir :} \jya lɤ-}
\paradigme{\textit{dir :} \jya pɯ-}
\begin{définition}\ 
\begin{déclaration}\grammar{denom}\end{déclaration}\end{définition}
\begin{définition}\fra pêcher du poisson\end{définition}
\begin{définition}\cmn 钓鱼\end{définition}
\begin{exemple}\jya pɯ-nɯqaɟya\cmn 我钓了鱼\end{exemple}
\begin{exemple}\jya ɲɯ-nɯqaɟy\cmn 他在钓鱼\end{exemple}
\begin{exemple}\jya ma-tɯ-nɯqaɟy-nɯ ma mɤj-ɤɣ\cmn 不准钓鱼\end{exemple}
\begin{relation-sémantique}\confer{
\hyperlink{Ⓔqaɟy}{\textit{ \papi{qaɟy}}}
}\end{relation-sémantique}\end{entrée}

\begin{entrée}
\vedette{\hypertarget{Ⓔnɯqambɯmbjom}{\papi{ nɯqambɯmbjom}}}\markboth{nɯqambɯmbjom}{}
\classe{vi}
\paradigme{\textit{dir :} \jya \_}
\begin{définition}\fra voler\end{définition}
\begin{définition}\cmn 飞\end{définition}
\begin{exemple}\jya tsɯʁot thɯ-nɯqambɯmbjom\cmn 野鸡飞走了\end{exemple}
\begin{exemple}\jya pɣa thɯ-nɯqambɯmbjom\cmn 鸟飞走了\end{exemple}\end{entrée}

\begin{entrée}
\vedette{\hypertarget{Ⓔnɯqandʐe}{\papi{ nɯqandʐe}}}\markboth{nɯqandʐe}{}\classe{vi}
\paradigme{\textit{dir :} \jya tɤ-}
\paradigme{\textit{dir :} \jya pɯ-}
\paradigme{\textit{dir :} \jya nɯ-}
\begin{définition}\fra chercher des vers de terre\end{définition}
\begin{définition}\cmn 找蚯蚓\end{définition}
\begin{relation-sémantique}\confer{
\hyperlink{Ⓔqandʐe}{\textit{ \papi{qandʐe}}}
}\end{relation-sémantique}\end{entrée}

\begin{entrée}
\vedette{\hypertarget{Ⓔnɯqarma}{\papi{ nɯqarma}}}\markboth{nɯqarma}{}\classe{vi}
\paradigme{\textit{dir :} \jya pɯ-}
\begin{définition}\fra chasser des crossoptilons\end{définition}
\begin{définition}\cmn 打马鸡\end{définition}
\begin{exemple}\jya ɯʑo ɕ-pɯ-nɯqarma\cmn 他去打马鸡了\end{exemple}
\begin{relation-sémantique}\confer{
\hyperlink{Ⓔqarma}{\textit{ \papi{qarma}}}
}\end{relation-sémantique}\end{entrée}

\begin{entrée}
\vedette{\hypertarget{Ⓔnɯqhaɴɢaʁ}{\papi{ nɯqhaɴɢaʁ}}}\markboth{nɯqhaɴɢaʁ}{}
\classe{vi}
\begin{définition}\ 
\begin{déclaration}\grammar{incorp}\end{déclaration}\end{définition}\acception{1}
\paradigme{\textit{dir :} \jya pɯ-}
\begin{définition}\fra tomber en arrière, se pencher vers l'arrière\end{définition}
\begin{définition}\cmn 往后仰;往后摔下来(人)\end{définition}
\begin{exemple}\jya pɯ-nɯqhaɴɢaʁ-a tɕe pɯ-ndʐaβ-a\cmn 我往后一仰就摔倒了\end{exemple}
\begin{exemple}\jya rɟɤlpu tɯ-ɲɟɤt kɯ ɯ-qhu pjɤ-nɯqhaɴɢaʁ tɕe pjɤ-si\cmn 国王后悔不已,往后仰摔下来了就死了\end{exemple}\acception{2}
\paradigme{\textit{dir :} \jya \_}
\begin{définition}\fra s'écrouler vers l'arrière\end{définition}
\begin{définition}\cmn 往后倒(房子)\end{définition}
\begin{exemple}\jya kha nɯ-nɯqhaɴɢaʁ (tɕe pjɯ-ndʐaβ ɕti)\cmn 房子倒塌了\end{exemple}
\begin{relation-sémantique}\confer{
\hyperlink{Ⓔɯ-qhu}{\textit{ \papi{ɯ-qhu}}}
}\end{relation-sémantique}\end{entrée}

\begin{entrée}
\vedette{\hypertarget{Ⓔnɯqhapa}{\papi{ nɯqhapa}}}\markboth{nɯqhapa}{}
\classe{vi}
\begin{définition}\fra surveiller la maison (de quelqu'un d'autre)\end{définition}
\begin{définition}\cmn 看家(别人的)\end{définition}
\begin{exemple}\jya aʑo ɲɯ-nɯqhapa-a\cmn 我在看家\end{exemple}\end{entrée}

\begin{entrée}
\vedette{\hypertarget{Ⓔnɯqhaχɕu}{\papi{ nɯqhaχɕu}}}\markboth{nɯqhaχɕu}{}
\begin{relation-sémantique}\confer{
\hyperlink{Ⓔrɯqhaχɕu}{\textit{ \papi{rɯqhaχɕu}}}
}\end{relation-sémantique}\end{entrée}

\begin{entrée}
\vedette{\hypertarget{Ⓔnɯqhɤcit}{\papi{ nɯqhɤcit}}}\markboth{nɯqhɤcit}{}\classe{vi}
\paradigme{\textit{dir :} \jya nɯ-}
\begin{définition}\ 
\begin{déclaration}\grammar{incorp}\end{déclaration}\end{définition}
\begin{définition}\fra reculer\end{définition}
\begin{définition}\cmn 后退\end{définition}
\begin{exemple}\jya kɤ-nɤma thamtɕɤt nɯ pɯ-nnɯ-sɤɣdɯɣ kɯnɤ tu-kɯ-stu tu-kɯ-mbat ra ma ɲɯ-kɯ-nɯqhɤcit tɕe mɤ-pe\cmn 工作无论再艰苦都要坚持下去,不要退却\end{exemple}
\begin{relation-sémantique}\confer{
\hyperlink{Ⓔcit}{\textit{ \papi{cit}}}
}\end{relation-sémantique}
\begin{relation-sémantique}\confer{
\hyperlink{Ⓔɯ-qhu}{\textit{ \papi{ɯ-qhu}}}
}\end{relation-sémantique}\end{entrée}

\begin{entrée}
\vedette{\hypertarget{Ⓔnɯqhɤjŋɯjŋa}{\papi{ nɯqhɤjŋɯjŋa}}}\markboth{nɯqhɤjŋɯjŋa}{}
\classe{vi}
\paradigme{\textit{dir :} \jya nɯ-}
\begin{définition}\ 
\begin{déclaration}\grammar{incorp}\end{déclaration}\end{définition}
\begin{définition}\fra se tenir très droit, avoir la tête presque courbée vers l'arrière\end{définition}
\begin{définition}\cmn 挺着身子仰着头;头往后仰;趾高气昂\end{définition}
\begin{exemple}\jya jiɕqha tɯrme ɲɯ-nɯqhɤjŋɯjŋa\cmn 那个人把身体挺得很直\end{exemple}
\begin{exemple}\jya tɕheme nɯ ɲɯ-znɤjpɯjpe ɲɯ-nɯqhɤjŋɯjŋa\cmn 那个女孩子趾高气昂\end{exemple}
\begin{exemple}\jya tɕheme nɯ ɲɯ-rɯɕaŋchi tɕe ɲɯ-nɯqhɤjŋɯjŋa\cmn 那个女孩子摇摆弄姿,趾高气昂\end{exemple}
\begin{relation-sémantique}\confer{
\hyperlink{Ⓔɯ-qhu}{\textit{ \papi{ɯ-qhu}}}
}\end{relation-sémantique}\end{entrée}

\begin{entrée}
\vedette{\hypertarget{Ⓔnɯqhɤstɯstu}{\papi{ nɯqhɤstɯstu}}}\markboth{nɯqhɤstɯstu}{}
\classe{vi}
\paradigme{\textit{dir :} \jya \_}
\begin{définition}\ 
\begin{déclaration}\grammar{incorp}\end{déclaration}\end{définition}
\begin{définition}\fra reculer\end{définition}
\begin{définition}\cmn 退后\end{définition}
\begin{exemple}\jya kɤ-nɯqhɤstɯstu-a\cmn 我后退了\end{exemple}
\begin{exemple}\jya qapri pjɤ-mto tɕe kɤ-nɯqhɤstɯstu\cmn 他看到蛇就后退了\end{exemple}
\begin{relation-sémantique}\confer{
\hyperlink{Ⓔɯ-qhu}{\textit{ \papi{ɯ-qhu}}}
}\end{relation-sémantique}
\begin{relation-sémantique}\confer{
\hyperlink{Ⓔastu}{\textit{ \papi{astu}}}
}\end{relation-sémantique}\begin{sous-entrée}
\vedette{\hypertarget{}{\papi{ znɯqhɤstɯstu}}}\markboth{znɯqhɤstɯstu}{}\classe{vt}
\paradigme{\textit{dir :} \jya \_}
\begin{définition}\ 
\begin{déclaration}\grammar{caus}\end{déclaration}\end{définition}
\begin{définition}\fra faire reculer\end{définition}
\begin{définition}\cmn 使……退后\end{définition}
\end{sous-entrée}\end{entrée}

\begin{entrée}
\vedette{\hypertarget{Ⓔnɯqhoχɕɤr}{\papi{ nɯqhoχɕɤr}}}\markboth{nɯqhoχɕɤr}{}\classe{vi}
\paradigme{\textit{dir :} \jya pɯ-}
\begin{définition}\fra avoir une diarrhée\end{définition}
\begin{définition}\cmn 拉肚子\end{définition}
\begin{relation-sémantique}\synonyme{
\hyperlink{Ⓔnɯtɯfɕɤl}{\textit{ \papi{nɯtɯfɕɤl}}}
}\end{relation-sémantique}\end{entrée}

\begin{entrée}
\vedette{\hypertarget{Ⓔnɯqru}{\papi{ nɯqru}}}\markboth{nɯqru}{}
\classe{vt}
\begin{définition}\ 
\begin{déclaration}\grammar{vert}\end{déclaration}\end{définition}
\begin{définition}\fra ramener à la maison\end{définition}
\begin{définition}\cmn 接回来\end{définition}
\begin{exemple}\jya kɯmaʁ sɤtɕha pjɤ-rɤʑi tɕeri, wuma ʑo pjɤ-nɯmbɣom tɕe z-jo-nɯqru\cmn (他女儿)在另外一个地方,很想念她,就把她接回来了\end{exemple}
\begin{relation-sémantique}\confer{
\hyperlink{Ⓔqru}{\textit{ \papi{qru}}}
}\end{relation-sémantique}
\end{entrée}

\begin{entrée}
\vedette{\hypertarget{Ⓔnɯqro}{\papi{ nɯqro}}}\markboth{nɯqro}{}\classe{vi}
\paradigme{\textit{dir :} \jya lɤ-}
\paradigme{\textit{dir :} \jya tu-}
\begin{définition}\fra chercher des fourmis\end{définition}
\begin{définition}\cmn 找蚂蚁(熊)\end{définition}
\begin{exemple}\jya pri lu-nɯqro ŋu\cmn 熊在找蚂蚁\end{exemple}
\begin{relation-sémantique}\confer{
\hyperlink{ⒺqroⒽ2}{\textit{ \papi{qro2}}}
}\end{relation-sémantique}\end{entrée}

\begin{entrée}
\vedette{\hypertarget{Ⓔnɯqɯqoʁ}{\papi{ nɯqɯqoʁ}}}\markboth{nɯqɯqoʁ}{}\classe{vi}
\paradigme{\textit{dir :} \jya nɯ-}
\begin{définition}\fra utiliser toutes ses forces\end{définition}
\begin{définition}\cmn 用尽全身的力气\end{définition}
\begin{exemple}\jya ta-ma ɯ-tshɤt ɯ-tsa ra ma kɤ-nɯqɯqoʁ mɤ-βdi\cmn 劳动不要太过卖力\end{exemple}\end{entrée}

\begin{entrée}
\vedette{\hypertarget{Ⓔnɯra}{\papi{ nɯra}}}\markboth{nɯra}{} (\variante{nɯnɯra}) \classe{dem}
\begin{définition}\fra ces choses\end{définition}
\begin{définition}\cmn 那些\end{définition}
\end{entrée}

\begin{entrée}
\vedette{\hypertarget{Ⓔnɯrɤɣo}{\papi{ nɯrɤɣo}}}\markboth{nɯrɤɣo}{}
\classe{vi}
\paradigme{\textit{dir :} \jya thɯ-}
\begin{définition}\fra chanter\end{définition}
\begin{définition}\cmn 唱\end{définition}
\begin{exemple}\jya jiɕqha nɯ kɤ-nɯrɤɣo rga\cmn 那个人喜欢唱歌\end{exemple}
\begin{exemple}\jya thɯ-nɯrɤɣo-a (=rɤɣo thɯ-βzu-t-a)\cmn 他唱了歌\end{exemple}
\begin{relation-sémantique}\confer{
\hyperlink{Ⓔrɤɣo}{\textit{ \papi{rɤɣo}}}
}\end{relation-sémantique}\end{entrée}

\begin{entrée}
\vedette{\hypertarget{Ⓔnɯrɤŋom}{\papi{ nɯrɤŋom}}}\markboth{nɯrɤŋom}{}
\classe{vt}
\paradigme{\textit{dir :} \jya nɯ-}
\begin{définition}\fra subir un outrage\end{définition}
\begin{définition}\cmn 受气\end{définition}
\begin{exemple}\jya na-nɯrɤŋom\cmn 他受了气\end{exemple}
\begin{exemple}\jya tɤ-nɤmqe-t-a tɕe ɲɯ-nɯrɤŋom\cmn 我骂了他,他就受气\end{exemple}
\begin{relation-sémantique}\confer{
\hyperlink{Ⓔrɤŋom}{\textit{ \papi{rɤŋom}}}
}\end{relation-sémantique}\end{entrée}

\begin{entrée}
\vedette{\hypertarget{Ⓔnɯrɤscoz}{\papi{ nɯrɤscoz}}}\markboth{nɯrɤscoz}{}
\begin{relation-sémantique}\confer{
\hyperlink{Ⓔrɤscoz}{\textit{ \papi{rɤscoz}}}
}\end{relation-sémantique}\end{entrée}

\begin{entrée}
\vedette{\hypertarget{Ⓔnɯrɤtʂha}{\papi{ nɯrɤtʂha}}}\markboth{nɯrɤtʂha}{}\classe{vi}
\paradigme{\textit{dir :} \jya kɤ-}
\begin{définition}\ 
\begin{déclaration}\grammar{denom}\end{déclaration}\end{définition}
\begin{définition}\fra manger à l'extérieur\end{définition}
\begin{définition}\cmn 在野外吃便饭(到外面劳动时)\end{définition}
\begin{exemple}\jya kɯ-lɤɣ tɤ-ari-tɕi tɕe rɯŋgu kɤ-nɯrɤtʂha-tɕi\cmn 我们俩去放牧时在野外吃了午餐\end{exemple}
\begin{relation-sémantique}\confer{
\hyperlink{Ⓔtʂha}{\textit{ \papi{tʂha}}}
}\end{relation-sémantique}\end{entrée}

\begin{entrée}
\vedette{\hypertarget{Ⓔnɯrɤʑi}{\papi{ nɯrɤʑi}}}\markboth{nɯrɤʑi}{}
\begin{relation-sémantique}\confer{
\hyperlink{Ⓔrɤʑi}{\textit{ \papi{rɤʑi}}}
}\end{relation-sémantique}\end{entrée}

\begin{entrée}
\vedette{\hypertarget{Ⓔnɯrchɯrchɤβ}{\papi{ nɯrchɯrchɤβ}}}\markboth{nɯrchɯrchɤβ}{}
\classe{vi}
\paradigme{\textit{dir :} \jya pɯ-}
\begin{définition}\ 
\begin{déclaration}\grammar{denom}\end{déclaration}\end{définition}
\begin{définition}\fra aller dans des endroits où il y a peu d'espace\end{définition}
\begin{définition}\cmn 在密集的地方走来走去;钻来钻去(森林里,人群里)\end{définition}
\begin{exemple}\jya ɯʑo pɯ-nɯrchɯrchɤβ\cmn 他钻来钻去了\end{exemple}
\begin{exemple}\jya sɯŋgɯ pɯ-nɯrchɯrchaβ-a\cmn 我在森林里钻来钻去\end{exemple}
\begin{exemple}\jya tɯrme nɯra nɯ-rchɤβ pɯ-nɯrchɯrchaβ-a\cmn 我在人群中窜来窜去\end{exemple}
\begin{relation-sémantique}\confer{
\hyperlink{Ⓔɯ-rchɤβ}{\textit{ \papi{ɯ-rchɤβ}}}
}\end{relation-sémantique}\end{entrée}

\begin{entrée}
\vedette{\hypertarget{Ⓔnɯrchɯrchɯɣ}{\papi{ nɯrchɯrchɯɣ}}}\markboth{nɯrchɯrchɯɣ}{}
\begin{relation-sémantique}\confer{
 \papi{rchɯrchɯɣ}
}\end{relation-sémantique}\end{entrée}

\begin{entrée}
\vedette{\hypertarget{Ⓔnɯrɕɤt}{\papi{ nɯrɕɤt}}}\markboth{nɯrɕɤt}{}
\classe{vi}\acception{1}
\paradigme{\textit{dir :} \jya pɯ-}
\begin{définition}\fra faire une crise d'épilepsie, tomber inconscient\end{définition}
\begin{définition}\cmn 癫痫发作;昏迷\end{définition}\acception{2}
\paradigme{\textit{dir :} \jya \_}
\begin{définition}\fra toucher légèrement, frotter en passant\end{définition}
\begin{définition}\cmn 轻轻地擦过去;碰一下\end{définition}
\begin{exemple}\jya jiɕqha nɯ ɯ-taʁ nɯ-nɯrɕɤt\cmn 他轻轻地碰了一下\end{exemple}
\begin{exemple}\jya nɯ-nɯrɕat-a\cmn 我轻轻地碰了一下\end{exemple}
\begin{exemple}\jya ɯ-mɲaʁ ɲɯ-nɯrɕɤt cinɤ mɤ-kɯ-ɤtsɯtsu qhe ɲɤ-ɕqhlɤt\cmn 还没有来得及看一眼就消失了\end{exemple}\begin{sous-entrée}
\vedette{\hypertarget{}{\papi{ znɯrɕɤt}}}\markboth{znɯrɕɤt}{}\classe{vt}
\paradigme{\textit{dir :} \jya \_}
\begin{définition}\ 
\begin{déclaration}\grammar{caus}\end{déclaration}\end{définition}
\begin{définition}\fra toucher légèrement, frotter en passant (avec un objet)\end{définition}
\begin{définition}\cmn (用某个东西)轻轻地碰一下\end{définition}
\begin{exemple}\jya nɯ-ari-ndʑi tɕe, lo rŋgɯ nɯ ɯ-ɕki nɯ tɕu phuɲi ɲɤ-znɯrɕɤt-ndʑi\cmn 他们俩到了之后,在大石包上用树枝轻轻地擦过去了\end{exemple}
\end{sous-entrée}\end{entrée}

\begin{entrée}
\vedette{\hypertarget{Ⓔnɯrdɤstaʁ}{\papi{ nɯrdɤstaʁ}}}\markboth{nɯrdɤstaʁ}{}\classe{vt}
\paradigme{\textit{dir :} \jya tɤ-}
\begin{définition}\ 
\begin{déclaration}\grammar{denom}\end{déclaration}\end{définition}
\begin{définition}\fra jeter une pierre à\end{définition}
\begin{définition}\cmn 扔石头\end{définition}
\begin{exemple}\jya tɤ́-wɣ-nɯrdɤstaʁ-a\cmn 他向我扔了石头\end{exemple}
\begin{exemple}\jya khɯna tɤ-nɯrdɤstaʁ-a (khɯna ɯ-ɕki rdɤstaʁ tɤ-lat-a)\cmn 我向狗扔石头\end{exemple}
\begin{exemple}\jya ndzɯrnaʁ ɯ-kha nɯ to-nɯrdɤstaʁ-nɯ tɕe, ndzɯrnaʁ pjɤ-nɯɬoʁ-nɯ tɕe kó-wɣ-mtsɯɣ tɕe pjɤ́-wɣ-tʂaβ\cmn 他们向马蜂窝扔了石头,马蜂出来了,把他们蛰了,他们痛得摔倒了\end{exemple}
\begin{relation-sémantique}\confer{
\hyperlink{Ⓔrdɤstaʁ}{\textit{ \papi{rdɤstaʁ}}}
}\end{relation-sémantique}\end{entrée}

\begin{entrée}
\vedette{\hypertarget{Ⓔnɯrdoʁ}{\papi{ nɯrdoʁ}}}\markboth{nɯrdoʁ}{}\classe{vt}
\paradigme{\textit{dir :} \jya tɤ-}
\begin{définition}\ 
\begin{déclaration}\grammar{denom}\end{déclaration}\end{définition}
\begin{définition}\fra ramasser un à un des petits morceaux\end{définition}
\begin{définition}\cmn 一个一个地捡起来\end{définition}
\begin{exemple}\jya ta-nɯrdoʁ\cmn 他捡了\end{exemple}
\begin{exemple}\jya kumpɣa kɯ ɯ-kɤ-ndza ɲɯ-ɤz-nɯrdoʁ\cmn 鸡在啄食物\end{exemple}
\begin{exemple}\jya staχpɯ ta-nɯrdoʁ\cmn 他捡了豌豆\end{exemple}
\begin{exemple}\jya stoʁ ta-nɯrdoʁ\cmn 他捡了胡豆\end{exemple}
\begin{exemple}\jya ʑɴɢɯloʁ ta-nɯrdoʁ\cmn 他捡了核桃\end{exemple}
\begin{relation-sémantique}\confer{
\hyperlink{Ⓔtɯ-rdoʁ}{\textit{ \papi{tɯ-rdoʁ}}}
}\end{relation-sémantique}\end{entrée}

\begin{entrée}
\vedette{\hypertarget{Ⓔnɯrdɯl}{\papi{ nɯrdɯl}}}\markboth{nɯrdɯl}{}
\classe{vi}
\paradigme{\textit{dir :} \jya kɤ-}
\begin{définition}\fra avoir plein de poussière\end{définition}
\begin{définition}\cmn 沾满了灰尘\end{définition}
\begin{exemple}\jya ki tɯ-ŋga ki ko-nɯrdɯl\cmn 这件衣服沾上了灰尘\end{exemple}\begin{sous-entrée}
\vedette{\hypertarget{}{\papi{ znɯrdɯl}}}\markboth{znɯrdɯl}{}\classe{vt}
\paradigme{\textit{dir :} \jya kɤ-}
\begin{définition}\fra rendre plein de poussière\end{définition}
\begin{définition}\cmn 使沾满灰尘\end{définition}
\begin{exemple}\jya kɤntɕhaʁ ɯ-tɯ-ɣɤrdɯl ɲɯ-saχaʁ tɕe, tɤ-kɯ-ŋke tɕe, ku-kɯ-z-nɯrdɯl ɲɯ-ŋu\cmn 街上很多灰尘,走路的时候身上会沾到灰尘\end{exemple}
\begin{relation-sémantique}\confer{
\hyperlink{Ⓔɣɤrdɯl}{\textit{ \papi{ɣɤrdɯl}}}
}\end{relation-sémantique}
\begin{relation-sémantique}\confer{
\hyperlink{Ⓔrdɯl}{\textit{ \papi{rdɯl}}}
}\end{relation-sémantique}
\end{sous-entrée}\end{entrée}

\begin{entrée}
\vedette{\hypertarget{Ⓔnɯre}{\papi{ nɯre}}}\markboth{nɯre}{}\classe{adv}
\begin{définition}\fra là\end{définition}
\begin{définition}\cmn 在(你)那里\end{définition}
\end{entrée}

\begin{entrée}
\vedette{\hypertarget{Ⓔnɯrga}{\papi{ nɯrga}}}\markboth{nɯrga}{}
\classe{vt}
\paradigme{\textit{dir :} \jya nɯ-}
\begin{définition}\ 
\begin{déclaration}\grammar{appl}\end{déclaration}\end{définition}
\begin{définition}\fra aimer\end{définition}
\begin{définition}\cmn 喜欢
\begin{déclaration} \étymologie{\papi{dga}}\end{déclaration}\end{définition}
\begin{exemple}\jya jiɕqha tɕheme nɯ ɲɯ-nɯrge-a\cmn 我喜欢那个女子\end{exemple}
\begin{exemple}\jya tɕheme nɯ nɯ-nɯrga-t-a\cmn 我喜欢上那个女子\end{exemple}
\begin{exemple}\jya pɯ-nɯrga-t-a\cmn 我以前喜欢她\end{exemple}
\begin{exemple}\jya ɲɯ-ta-nɯrga\cmn 我喜欢你\end{exemple}
\begin{relation-sémantique}\confer{
 \papi{rga}
}\end{relation-sémantique}
\begin{relation-sémantique}\antonyme{
\hyperlink{Ⓔqha}{\textit{ \papi{qha}}}
}\end{relation-sémantique}\begin{sous-entrée}
\vedette{\hypertarget{}{\papi{ anɯrgɯrga}}}\markboth{anɯrgɯrga}{}\classe{vi}
\begin{définition}\ 
\begin{déclaration}\grammar{appl}\end{déclaration}
\begin{déclaration}\grammar{recip}\end{déclaration}\end{définition}
\begin{définition}\fra s'aimer l'un l'autre\end{définition}
\begin{définition}\cmn 相爱\end{définition}
\end{sous-entrée}\begin{sous-entrée}
\vedette{\hypertarget{}{\papi{ sɤnɯrga}}}\markboth{sɤnɯrga}{}\classe{vi}
\begin{définition}\ 
\begin{déclaration}\grammar{appl}\end{déclaration}
\begin{déclaration}\grammar{apass}\end{déclaration}\end{définition}
\begin{définition}\fra qui aime les gens\end{définition}
\begin{définition}\cmn 喜欢别人的\end{définition}
\begin{exemple}\jya nɤʑo ɲɯ-tɯ-sɤnɯrga\cmn 你喜欢别人\end{exemple}
\begin{exemple}\jya nɤki tɤ-pɤtso nɯ kɯ-sɤnɯrga ci ŋu\cmn 那个小孩子喜欢人\end{exemple}
\end{sous-entrée}\end{entrée}

\begin{entrée}
\vedette{\hypertarget{Ⓔnɯrɟaŋ}{\papi{ nɯrɟaŋ}}}\markboth{nɯrɟaŋ}{}\classe{vs}\acception{1}
\begin{définition}\fra qui se transmet loin\end{définition}
\begin{définition}\cmn 传得远\end{définition}
\begin{exemple}\jya ɯ-skɤt ɲɯ-nɯrɟaŋ\cmn 他的声音传得很远\end{exemple}\acception{2}
\begin{définition}\fra répandue (langue)\end{définition}
\begin{définition}\cmn 使用范围最广(语言)\end{définition}
\begin{exemple}\jya stu kɯ-nɯrɟaŋ nɯ ɲɯ-tɯ-spe\cmn 你学会了最实用的(那个语言)(反话,指的是茶堡话)\end{exemple}
\begin{relation-sémantique}\confer{
\hyperlink{Ⓔrɟaŋ}{\textit{ \papi{rɟaŋ}}}
}\end{relation-sémantique}\end{entrée}

\begin{entrée}
\vedette{\hypertarget{Ⓔnɯrɟɤntɕa}{\papi{ nɯrɟɤntɕa}}}\markboth{nɯrɟɤntɕa}{}
\classe{vt}
\paradigme{\textit{dir :} \jya tɤ-}
\begin{définition}\ 
\begin{déclaration}\grammar{denom}\end{déclaration}\end{définition}
\begin{définition}\fra se parer (de bijoux)\end{définition}
\begin{définition}\cmn 打扮(一身带满装饰)
\begin{déclaration} \étymologie{\papi{rgʲan.tɕʰa}}\end{déclaration}\end{définition}
\begin{exemple}\jya tɤ-nɯrɟɤntɕa-t-a\cmn 我打扮了\end{exemple}\end{entrée}

\begin{entrée}
\vedette{\hypertarget{Ⓔnɯrɟɯrŋom}{\papi{ nɯrɟɯrŋom}}}\markboth{nɯrɟɯrŋom}{}
\classe{vi}
\paradigme{\textit{dir :} \jya nɯ-}
\begin{définition}\ 
\begin{déclaration}\grammar{incorp}\end{déclaration}\end{définition}
\begin{définition}\fra convoiter des richesses\end{définition}
\begin{définition}\cmn 贪财\end{définition}
\begin{exemple}\jya ɲɯ-nɯrɟɯrŋom\cmn 他贪财\end{exemple}
\begin{exemple}\jya kɤ-nɯrɟɯrŋom sɤzraʁ\cmn 贪财是一件可耻的事情\end{exemple}
\begin{relation-sémantique}\confer{
\hyperlink{Ⓔrɟɯrŋom}{\textit{ \papi{rɟɯrŋom}}}
}\end{relation-sémantique}
\begin{relation-sémantique}\confer{
\hyperlink{Ⓔsŋom}{\textit{ \papi{sŋom}}}
}\end{relation-sémantique}
\begin{relation-sémantique}\confer{
\hyperlink{Ⓔtɯ-rɟɯ}{\textit{ \papi{tɯ-rɟɯ}}}
}\end{relation-sémantique}\end{entrée}

\begin{entrée}
\vedette{\hypertarget{Ⓔnɯrkorlɯt}{\papi{ nɯrkorlɯt}}}\markboth{nɯrkorlɯt}{}
\classe{vs}
\paradigme{\textit{dir :} \jya tɤ-}
\begin{définition}\ 
\begin{déclaration}\grammar{comp}\end{déclaration}\end{définition}
\begin{définition}\fra être entêté\end{définition}
\begin{définition}\cmn 顽固;固执\end{définition}
\begin{exemple}\jya ɲɯ-nɯrkorlɯt\cmn 他很顽固\end{exemple}
\begin{exemple}\jya nɯki tɯrme nɯ ɲɯ-nɯrkorlɯt tɕe mɯ́j-khɯ\cmn 这个人很固执,没有同意\end{exemple}
\begin{exemple}\jya to-nɯrkorlɯt\cmn 他变得很固执;他当时很固执(没有答应别人)\end{exemple}
\begin{exemple}\jya ma-tɤ-tɯ-nɯrkorlɯt\cmn 你不要这么固执\end{exemple}
\begin{relation-sémantique}\confer{
\hyperlink{Ⓔrko}{\textit{ \papi{rko}}}
}\end{relation-sémantique}
\begin{relation-sémantique}\confer{
\hyperlink{Ⓔarlɯt}{\textit{ \papi{arlɯt}}}
}\end{relation-sémantique}\end{entrée}

\begin{entrée}
\vedette{\hypertarget{Ⓔnɯrlɤn}{\papi{ nɯrlɤn}}}\markboth{nɯrlɤn}{}
\classe{vs}
\paradigme{\textit{dir :} \jya nɯ-}
\begin{définition}\fra vert (bois), humide\end{définition}
\begin{définition}\cmn 湿;生(木头)
\begin{déclaration} \étymologie{\papi{rlan}}\end{déclaration}\end{définition}
\begin{exemple}\jya pjɤ-nɯrlɤn tɕe sɯtɕhaʁ ko-ɕe\cmn 木头是湿的,所以缩了\end{exemple}
\begin{exemple}\jya kɯ-nɯrlɤn\cmn 生木头\end{exemple}\begin{sous-entrée}
\vedette{\hypertarget{}{\papi{ znɯrlɤn}}}\markboth{znɯrlɤn}{}\classe{vt}
\begin{définition}\fra rendre humide\end{définition}
\begin{définition}\cmn 弄湿\end{définition}
\end{sous-entrée}\end{entrée}

\begin{entrée}
\vedette{\hypertarget{Ⓔnɯrmɤβlɯ}{\papi{ nɯrmɤβlɯ}}}\markboth{nɯrmɤβlɯ}{}\classe{vt}
\paradigme{\textit{dir :} \jya tɤ-}
\begin{définition}\fra dépendre de\end{définition}
\begin{définition}\cmn (家庭)靠……\end{définition}
\begin{exemple}\jya nɤʑo ku-ta-nɯrmɤβlɯ ɕti\cmn 我们家全靠你\end{exemple}
\begin{exemple}\jya nɤʑo ji-kɤ-nɯrmɤβlɯ tɯ-ɕti\cmn 我们家全靠你\end{exemple}
\begin{relation-sémantique}\confer{
\hyperlink{Ⓔrma}{\textit{ \papi{rma}}}
}\end{relation-sémantique}
\begin{relation-sémantique}\confer{
\hyperlink{Ⓔβlɯ}{\textit{ \papi{βlɯ}}}
}\end{relation-sémantique}\end{entrée}

\begin{entrée}
\vedette{\hypertarget{Ⓔnɯrmɤkro}{\papi{ nɯrmɤkro}}}\markboth{nɯrmɤkro}{}
\classe{vi}
\paradigme{\textit{dir :} \jya pɯ-}
\begin{définition}\ 
\begin{déclaration}\grammar{incorp}\end{déclaration}\end{définition}
\begin{définition}\fra partager le patrimoine\end{définition}
\begin{définition}\cmn 分家;分财产\end{définition}
\begin{exemple}\jya pɯ-nɯrmɤkro-tɕi\cmn 我们俩分家了\end{exemple}
\begin{relation-sémantique}\confer{
\hyperlink{Ⓔtɯrma}{\textit{ \papi{tɯrma}}}
}\end{relation-sémantique}
\begin{relation-sémantique}\confer{
\hyperlink{Ⓔkro}{\textit{ \papi{kro}}}
}\end{relation-sémantique}\end{entrée}

\begin{entrée}
\vedette{\hypertarget{Ⓔnɯrmɤmbe}{\papi{ nɯrmɤmbe}}}\markboth{nɯrmɤmbe}{}
\classe{vi}
\paradigme{\textit{dir :} \jya thɯ-}
\begin{définition}\ 
\begin{déclaration}\grammar{incorp}\end{déclaration}\end{définition}
\begin{définition}\fra muer (mammifère)\end{définition}
\begin{définition}\cmn 脱毛\end{définition}
\begin{exemple}\jya ji-fsapaʁ chɤ-pe, thɯ-nɯrmɤmbe\cmn 我们家的牲畜身体很壮就换了毛\end{exemple}
\begin{exemple}\jya nɯŋa thɯ-nɯrmɤmbe\cmn 奶牛换了毛\end{exemple}
\begin{exemple}\jya jla thɯ-nɯrmɤmbe\cmn 犏牛换了毛\end{exemple}
\begin{relation-sémantique}\confer{
\hyperlink{Ⓔtɤ-rme}{\textit{ \papi{tɤ-rme}}}
}\end{relation-sémantique}
\begin{relation-sémantique}\confer{
\hyperlink{Ⓔmbe}{\textit{ \papi{mbe}}}
}\end{relation-sémantique}
\begin{relation-sémantique}\confer{
\hyperlink{Ⓔrmɤmbe}{\textit{ \papi{rmɤmbe}}}
}\end{relation-sémantique}\end{entrée}

\begin{entrée}
\vedette{\hypertarget{Ⓔnɯrmɤʑu}{\papi{ nɯrmɤʑu}}}\markboth{nɯrmɤʑu}{}\classe{vi}
\paradigme{\textit{dir :} \jya tɤ-}
\paradigme{\textit{dir :} \jya thɯ-}
\begin{définition}\fra faire son intéressant\end{définition}
\begin{définition}\cmn 爱在人多的场合中表现自己【耍人来疯】\end{définition}
\begin{exemple}\jya jiɕqha nɯ wuma nɯrmɤʑu\cmn 那个人很喜欢表现自己\end{exemple}
\begin{exemple}\jya ma-tɤ-tɯ-nɯrmɤʑu\cmn 你不要耍人来疯\end{exemple}
\begin{exemple}\jya tɤ-pɤtso ɲɯ-nɯrmɤʑu\cmn 小孩子很喜欢表现自己\end{exemple}\end{entrée}

\begin{entrée}
\vedette{\hypertarget{Ⓔnɯrmbɯχtɕi}{\papi{ nɯrmbɯχtɕi}}}\markboth{nɯrmbɯχtɕi}{}
\classe{vt}
\paradigme{\textit{dir :} \jya tɤ-}
\begin{définition}\ 
\begin{déclaration}\grammar{incorp}\end{déclaration}\end{définition}
\begin{définition}\fra asperger de liquide\end{définition}
\begin{définition}\cmn 朝人喷尿
\end{définition}
\begin{exemple}\jya tɤ-kɯ-nɯrmbɯχtɕi-a\cmn 你朝我喷了液体\end{exemple}
\begin{relation-sémantique}\confer{
\hyperlink{Ⓔχtɕi}{\textit{ \papi{χtɕi}}}
}\end{relation-sémantique}
\begin{relation-sémantique}\confer{
\hyperlink{Ⓔtɯ-rmbi}{\textit{ \papi{tɯ-rmbi}}}
}\end{relation-sémantique}\end{entrée}

\begin{entrée}
\vedette{\hypertarget{Ⓔnɯrmɯ}{\papi{ nɯrmɯ}}}\markboth{nɯrmɯ}{}\classe{vi}
\paradigme{\textit{dir :} \jya nɯ-}
\begin{définition}\fra se coucher tard\end{définition}
\begin{définition}\cmn 晚睡\end{définition}
\begin{relation-sémantique}\confer{
\hyperlink{Ⓔtɯrmɯ}{\textit{ \papi{tɯrmɯ}}}
}\end{relation-sémantique}
\begin{relation-sémantique}\confer{
\hyperlink{Ⓔnɯrmɯsoz}{\textit{ \papi{nɯrmɯsoz}}}
}\end{relation-sémantique}
\end{entrée}

\begin{entrée}
\vedette{\hypertarget{Ⓔnɯrmɯsoz}{\papi{ nɯrmɯsoz}}}\markboth{nɯrmɯsoz}{}\classe{vi}
\paradigme{\textit{dir :} \jya tɤ-}
\begin{définition}\fra se lever tôt et se coucher tard\end{définition}
\begin{définition}\cmn 早起晚睡\end{définition}
\begin{relation-sémantique}\confer{
\hyperlink{Ⓔsoz}{\textit{ \papi{soz}}}
}\end{relation-sémantique}
\begin{relation-sémantique}\confer{
\hyperlink{Ⓔtɯrmɯ}{\textit{ \papi{tɯrmɯ}}}
}\end{relation-sémantique}\end{entrée}

\begin{entrée}
\vedette{\hypertarget{Ⓔnɯrŋu}{\papi{ nɯrŋu}}}\markboth{nɯrŋu}{}
\classe{vi}
\paradigme{\textit{dir :} \jya kɤ-}
\begin{définition}\fra attraper une maladie de la peau\end{définition}
\begin{définition}\cmn 患上一种皮肤病\end{définition}
\begin{exemple}\jya paʁ ko-nɯrŋu\cmn 猪得了皮肤病\end{exemple}
\begin{exemple}\jya mbro ko-nɯrŋu\cmn 马得了皮肤病\end{exemple}
\begin{exemple}\jya kɤ-kɯ-nɯrŋu\cmn 皮肤病患者\end{exemple}
\begin{exemple}\jya kɤ-kɯ-nɯrŋu nɯ tɯ-βri thamtɕɤt ʑo ʑmbɤr kɯ-fse ɲɯ-mtshɤt ʑo ɲɯ-ŋu, tɤ-rme ra chɯ-ɤʁɟa tɕe ɲɯ-pɣi ʑo ɲɯ-rʁom ʑo ɲɯ-ŋu.\cmn 
\stylefv{kɤ-kɯ-nɯrŋu} 是一种皮肤病,全身长满疮一样的东西,毛全脱光,皮肤变灰色,很粗糙。
\end{exemple}\end{entrée}

\begin{entrée}
\vedette{\hypertarget{Ⓔnɯrŋgɯ}{\papi{ nɯrŋgɯ}}}\markboth{nɯrŋgɯ}{}
\begin{relation-sémantique}\confer{
\hyperlink{ⒺrŋgɯⒽ1}{\textit{ \papi{rŋgɯ1}}}
}\end{relation-sémantique}\end{entrée}

\begin{entrée}
\vedette{\hypertarget{Ⓔnɯrŋgɯmbri}{\papi{ nɯrŋgɯmbri}}}\markboth{nɯrŋgɯmbri}{}\classe{vi}
\begin{définition}\ 
\begin{déclaration}\grammar{comp}\end{déclaration}\end{définition}
\begin{définition}\fra gémir\end{définition}
\begin{définition}\cmn 呻吟\end{définition}
\begin{relation-sémantique}\confer{
\hyperlink{ⒺrŋgɯⒽ1}{\textit{ \papi{rŋgɯ1}}}
}\end{relation-sémantique}
\begin{relation-sémantique}\confer{
\hyperlink{ⒺmbriⒽ1}{\textit{ \papi{mbri1}}}
}\end{relation-sémantique}\end{entrée}

\begin{entrée}
\vedette{\hypertarget{Ⓔnɯrpu}{\papi{ nɯrpu}}}\markboth{nɯrpu}{}
\begin{relation-sémantique}\confer{
\hyperlink{Ⓔrpu}{\textit{ \papi{rpu}}}
}\end{relation-sémantique}
\end{entrée}

\begin{entrée}
\vedette{\hypertarget{Ⓔnɯrqhoʁ}{\papi{ nɯrqhoʁ}}}\markboth{nɯrqhoʁ}{}
\begin{relation-sémantique}\confer{
\hyperlink{Ⓔɣɤrqhoʁrqhoʁ}{\textit{ \papi{ɣɤrqhoʁrqhoʁ}}}
}\end{relation-sémantique}\end{entrée}

\begin{entrée}
\vedette{\hypertarget{Ⓔnɯrʁe}{\papi{ nɯrʁe}}}\markboth{nɯrʁe}{}
\classe{vt}
\paradigme{\textit{dir :} \jya lɤ-}
\begin{définition}\ 
\begin{déclaration}\grammar{autoben}\end{déclaration}\end{définition}
\begin{définition}\fra porter (un bracelet)\end{définition}
\begin{définition}\cmn 戴(手镯、耳环)\end{définition}
\begin{exemple}\jya zgroʁ la-nɯrʁe\cmn 他戴了手镯\end{exemple}
\begin{exemple}\jya zgroʁ lɤ-nɯrʁe-t-a\cmn 我戴了手镯\end{exemple}
\begin{exemple}\jya rnɤjɯ la-nɯrʁe\cmn 他戴了耳环\end{exemple}
\begin{exemple}\jya srɯnloʁ lɤ-nɯrʁe-t-a\cmn 我戴了戒指\end{exemple}
\begin{relation-sémantique}\confer{
\hyperlink{Ⓔrʁe}{\textit{ \papi{rʁe}}}
}\end{relation-sémantique}\end{entrée}

\begin{entrée}
\vedette{\hypertarget{Ⓔnɯrʁɯrpu}{\papi{ nɯrʁɯrpu}}}\markboth{nɯrʁɯrpu}{}\classe{vt}
\paradigme{\textit{dir :} \jya tɤ-}
\begin{définition}\ 
\begin{déclaration}\grammar{incorp}\end{déclaration}\end{définition}
\begin{définition}\fra donner des coups de corne\end{définition}
\begin{définition}\cmn 用角打\end{définition}
\begin{exemple}\jya jla kɯ tɤ́-wɣ-nɯʁrɯrpu-a\cmn 犏牛用角打了我\end{exemple}\begin{sous-entrée}
\vedette{\hypertarget{}{\papi{ sɤnɯʁrɯrpu}}}\markboth{sɤnɯʁrɯrpu}{}\classe{vi}
\begin{définition}\fra donner des coups de corne aux gens\end{définition}
\begin{définition}\cmn 用角打人\end{définition}
\end{sous-entrée}\end{entrée}

\begin{entrée}
\vedette{\hypertarget{Ⓔnɯrʁɯrʁa}{\papi{ nɯrʁɯrʁa}}}\markboth{nɯrʁɯrʁa}{}
\classe{vi}
\paradigme{\textit{dir :} \jya tɤ-}
\begin{définition}\fra grimper\end{définition}
\begin{définition}\cmn 爬\end{définition}
\begin{exemple}\jya znde ɯ-taʁ tɤ-nɯrʁɯrʁa\cmn 他爬了墙\end{exemple}
\begin{exemple}\jya znde ɯ-taʁ ma-tɤ-tɯ-nɯrʁɯrʁa ma tɯ-atɤr\cmn 你不要爬墙,小心摔下来\end{exemple}
\begin{exemple}\jya si ɯ-taʁ tɤ-nɯrʁɯrʁa\cmn 他爬了树\end{exemple}\end{entrée}

\begin{entrée}
\vedette{\hypertarget{Ⓔnɯrtɤβ}{\papi{ nɯrtɤβ}}}\markboth{nɯrtɤβ}{}\classe{vt}
\paradigme{\textit{dir :} \jya nɯ-}
\paradigme{\textit{dir :} \jya thɯ-}
\begin{définition}\fra porter\end{définition}
\begin{définition}\cmn 戴;穿;系\end{définition}
\begin{exemple}\jya tɕhoma nɯ-nɯrtaβ-a\cmn 我穿了皮带\end{exemple}
\begin{exemple}\jya βʑɯndi nɯ-nɯrtaβ-a\cmn 我穿了裹腿\end{exemple}
\begin{exemple}\jya mthɯxtɕɤr nɯ nɯ-nɯrtaβ-a\cmn 我穿了腰带\end{exemple}
\begin{exemple}\jya @weijin thɯ-nɯrtaβ-a\cmn 我戴了围巾\end{exemple}
\begin{relation-sémantique}\confer{
\hyperlink{Ⓔrtɤβ}{\textit{ \papi{rtɤβ}}}
}\end{relation-sémantique}\end{entrée}

\begin{entrée}
\vedette{\hypertarget{Ⓔnɯrtɕa}{\papi{ nɯrtɕa}}}\markboth{nɯrtɕa}{}\classe{vt}
\paradigme{\textit{dir :} \jya kɤ-}
\begin{définition}\fra taquiner\end{définition}
\begin{définition}\cmn 逗弄(话)、骚扰\end{définition}
\begin{exemple}\jya kɤ-nɯrtɕa-t-a\cmn 我逗弄他了\end{exemple}
\begin{exemple}\jya kɤ́-wɣ-nɯrtɕa-a\cmn 他逗弄我了\end{exemple}
\begin{exemple}\jya kɤ-ta-nɯrtɕa\cmn 我逗弄你了\end{exemple}
\begin{relation-sémantique}\synonyme{
\hyperlink{Ⓔnɤjndɤt}{\textit{ \papi{nɤjndɤt}}}
}\end{relation-sémantique}\begin{sous-entrée}
\vedette{\hypertarget{}{\papi{ sɤnɯrtɕa}}}\markboth{sɤnɯrtɕa}{}\classe{vi}
\paradigme{\textit{dir :} \jya kɤ-}
\begin{définition}\ 
\begin{déclaration}\grammar{apass}\end{déclaration}\end{définition}
\begin{définition}\fra taquiner les gens\end{définition}
\begin{définition}\cmn 惹人家\end{définition}
\end{sous-entrée}\end{entrée}

\begin{entrée}
\vedette{\hypertarget{Ⓔnɯrtɕhɯɴɢɯɴɢaʁ}{\papi{ nɯrtɕhɯɴɢɯɴɢaʁ}}}\markboth{nɯrtɕhɯɴɢɯɴɢaʁ}{}
\classe{vi}
\paradigme{\textit{dir :} \jya lɤ-}
\begin{définition}\fra s'écailler\end{définition}
\begin{définition}\cmn 一片一片地裂开\end{définition}
\begin{exemple}\jya tɤtho ɯ-mat nɯ ɯ-rɣi nɯ-ɬoʁ tɤkha tɕe ɯ-rqhu nɯ lu-nɯrtɕhɯɴɢ̣ɴɢaʁ ŋu, tɕe ɯ-rqhu nɯ tɤ-ɴɢaʁ tɕe, nɯ ɯ-ŋgɯ ɯ-rɣi nɯ pjɯ-nɯɬoʁ ɲɯ-ŋu\cmn 松树的果子里面的种子快要脱落的时候,它的果皮一片一片地裂开,然后脱落\end{exemple}
\begin{exemple}\jya pɯ-kɯ-ndʐaβ tɕe, tɯ-ndʐi lu-nɯrtɕhɯɴɢɯɴɢaʁ ŋgrɤl\cmn 摔倒的时候,经常会把皮肤擦伤\end{exemple}
\begin{relation-sémantique}\confer{
\hyperlink{Ⓔɴɢaʁ}{\textit{ \papi{ɴɢaʁ}}}
}\end{relation-sémantique}
\end{entrée}

\begin{entrée}
\vedette{\hypertarget{Ⓔnɯrtsa}{\papi{ nɯrtsa}}}\markboth{nɯrtsa}{}\classe{vt}
\paradigme{\textit{dir :} \jya nɯ-}
\begin{définition}\fra rechercher la cause de\end{définition}
\begin{définition}\cmn 追究\end{définition}
\begin{exemple}\jya nɯɕɯŋgɯ pɯ-kɯ-fse mɯ-ɲɤ-nɯrtsa\cmn 他没有追究以前发生的事情\end{exemple}
\begin{exemple}\jya tɕhi pjɤ-fse ɲɯ-ŋu kɯ a-nɯ-tɯ-nɯrtse\cmn 你要追究到底发生了什么事情\end{exemple}
\begin{relation-sémantique}\confer{
\hyperlink{Ⓔɯ-rtsa,tɕɤt}{\textit{ \papi{ɯ-rtsa,tɕɤt}}}
}\end{relation-sémantique}\end{entrée}

\begin{entrée}
\vedette{\hypertarget{Ⓔnɯrtsɤl}{\papi{ nɯrtsɤl}}}\markboth{nɯrtsɤl}{}
\classe{vs}
\paradigme{\textit{dir :} \jya tɤ-}
\begin{définition}\fra être bon en équitation\end{définition}
\begin{définition}\cmn 骑马的技术好
\begin{déclaration} \étymologie{\papi{rtsal}}\end{déclaration}\end{définition}
\begin{exemple}\jya mbro kɤ-nɯmbrɤpɯ wuma ɲɯ-nɯrtsɤl\cmn 他骑马的技术非常好\end{exemple}\end{entrée}

\begin{entrée}
\vedette{\hypertarget{Ⓔnɯrtsɤtɯɣ}{\papi{ nɯrtsɤtɯɣ}}}\markboth{nɯrtsɤtɯɣ}{}
\classe{vi}
\paradigme{\textit{dir :} \jya pɯ-}
\begin{définition}\ 
\begin{déclaration}\grammar{denom}\end{déclaration}\end{définition}
\begin{définition}\fra s'empoisonner en mangeant une herbe (bovidé)\end{définition}
\begin{définition}\cmn 草中毒(牛)
\begin{déclaration} \étymologie{\papi{rtsʷa.dug}}\end{déclaration}\end{définition}
\begin{exemple}\jya jla pjɤ-nɯrtsɤtɯɣ\cmn 犏牛吃草中毒了\end{exemple}
\begin{exemple}\jya qambrɯ pjɤ-nɯrtsɤtɯɣ\cmn 犏牦牛吃草中毒了\end{exemple}
\begin{relation-sémantique}\confer{
\hyperlink{Ⓔnɯtɕhɯtɯɣ}{\textit{ \papi{nɯtɕhɯtɯɣ}}}
}\end{relation-sémantique}\end{entrée}

\begin{entrée}
\vedette{\hypertarget{Ⓔnɯrtsɯ}{\papi{ nɯrtsɯ}}}\markboth{nɯrtsɯ}{}
\classe{vi}
\paradigme{\textit{dir :} \jya \_}
\begin{définition}\fra ramper\end{définition}
\begin{définition}\cmn 爬行
\begin{déclaration}\use{只用于人,爬行动物要用\stylefv{ŋke}“走”,如\stylefv{tɕhɯχpri ɲɯ-ŋke} “四脚蛇在爬行”}\end{déclaration}\end{définition}
\begin{exemple}\jya tɤ-pɤtso ɲɯ-nɯrtsɯ\cmn 小孩子在爬行\end{exemple}\end{entrée}

\begin{entrée}
\vedette{\hypertarget{Ⓔnɯrtsɯpɣaʁ}{\papi{ nɯrtsɯpɣaʁ}}}\markboth{nɯrtsɯpɣaʁ}{}
\classe{vt}
\paradigme{\textit{dir :} \jya lɤ-}
\begin{définition}\fra retourner la terre après la récolte\end{définition}
\begin{définition}\cmn 庄稼收割了以后重新翻地\end{définition}
\begin{exemple}\jya tɯji lɤ-nɯrtsɯpɣaʁ-a\cmn 我翻了地\end{exemple}
\begin{exemple}\jya tɯtɣa pɯ-jɤɣ tɕe, kɤ-nɯrtsɯpɣaʁ mda\cmn 收割结束了之后就是翻地的时候了\end{exemple}
\begin{exemple}\jya tɯji mɤ-kɤ-nɯrtsɯpɣaʁ mɤ-khɯ\cmn 不翻田地是不行的\end{exemple}
\begin{exemple}\jya pɯ-nɯrtsɯpɣaʁ-a ri mɯ́j-phɤn\cmn 我耕了这块地,但还是不行\end{exemple}
\begin{exemple}\jya mɤ-kɤ-tɣa kɤ-nɯrtsɯpɣaʁ mɤ-ŋgrɤl\cmn 在没有收割之前不能翻地\end{exemple}
\begin{relation-sémantique}\confer{
\hyperlink{Ⓔpɣaʁ}{\textit{ \papi{pɣaʁ}}}
}\end{relation-sémantique}
\begin{relation-sémantique}\confer{
\hyperlink{Ⓔrtsɯpɣaʁ}{\textit{ \papi{rtsɯpɣaʁ}}}
}\end{relation-sémantique}
\begin{relation-sémantique}\confer{
\hyperlink{Ⓔrɯrtsɯpɣaʁ}{\textit{ \papi{rɯrtsɯpɣaʁ}}}
}\end{relation-sémantique}\end{entrée}

\begin{entrée}
\vedette{\hypertarget{Ⓔnɯrɯ}{\papi{ nɯrɯ}}}\markboth{nɯrɯ}{}
\classe{vi}
\paradigme{\textit{dir :} \jya nɯ-}
\begin{définition}\fra brouter l'herbe\end{définition}
\begin{définition}\cmn 吃草\end{définition}
\begin{exemple}\jya fsapaʁ ɲɯ-nɯrɯ\cmn 牲畜在吃草\end{exemple}
\begin{exemple}\jya fsapaʁ ra ɲɯ-nɯrɯ-nɯ\cmn 牲畜在吃草\end{exemple}
\begin{relation-sémantique}\synonyme{
\hyperlink{Ⓔnɯsɤlɤɣ}{\textit{ \papi{nɯsɤlɤɣ}}}
}\end{relation-sémantique}\end{entrée}

\begin{entrée}
\vedette{\hypertarget{Ⓔnɯrɯcu}{\papi{ nɯrɯcu}}}\markboth{nɯrɯcu}{}\classe{vs}
\begin{définition}\fra bien s'entendre avec\end{définition}
\begin{définition}\cmn 合得来\end{définition}
\begin{exemple}\jya ndʐɯɣlɤm kɯ-nɯrɯcu\cmn 合法的\end{exemple}
\begin{exemple}\jya tɯrme ra nɯ-rca ɲɯ-tɯ-nɯrɯcu\cmn 你跟这些人合得来\end{exemple}
\begin{relation-sémantique}\confer{
\hyperlink{Ⓔnɤcu}{\textit{ \papi{nɤcu}}}
}\end{relation-sémantique}\end{entrée}

\begin{entrée}
\vedette{\hypertarget{Ⓔnɯrɯtʂa}{\papi{ nɯrɯtʂa}}}\markboth{nɯrɯtʂa}{}
\classe{vt}
\paradigme{\textit{dir :} \jya tɤ-}
\begin{définition}\ 
\begin{déclaration}\grammar{denom}\end{déclaration}\end{définition}
\begin{définition}\fra envier\end{définition}
\begin{définition}\cmn 妒忌\end{définition}
\begin{exemple}\jya jiɕqha nɯ kɯ ɲɯ́-wɣ-nɯrɯtʂa-a\cmn 那个人妒忌我\end{exemple}
\begin{exemple}\jya ma-tɤ-kɯ-nɯrɯtʂa-a\cmn 你不要妒忌我\end{exemple}\begin{sous-entrée}
\vedette{\hypertarget{}{\papi{ anɯrɯtʂɯtʂa}}}\markboth{anɯrɯtʂɯtʂa}{}
\begin{définition}\fra s'envier les uns les autres\end{définition}
\begin{définition}\cmn 互相妒忌\end{définition}
\end{sous-entrée}\begin{sous-entrée}
\vedette{\hypertarget{}{\papi{ sɤnɯrɯtʂa}}}\markboth{sɤnɯrɯtʂa}{}\classe{vi}
\begin{définition}\ 
\begin{déclaration}\grammar{apass}\end{déclaration}\end{définition}
\begin{définition}\fra envier les gens\end{définition}
\begin{définition}\cmn 妒忌人家\end{définition}
\begin{relation-sémantique}\confer{
\hyperlink{Ⓔrɯtʂa}{\textit{ \papi{rɯtʂa}}}
}\end{relation-sémantique}
\end{sous-entrée}\end{entrée}

\begin{entrée}
\vedette{\hypertarget{Ⓔnɯrɯz}{\papi{ nɯrɯz}}}\markboth{nɯrɯz}{}\classe{vi}
\paradigme{\textit{dir :} \jya thɯ-}
\begin{définition}\fra faire l'un après l'autre\end{définition}
\begin{définition}\cmn 轮流\end{définition}
\begin{exemple}\jya @zhiban chɯ-nɯrɯz-nɯ ɲɯ-ra\cmn 他们要轮流值班\end{exemple}
\begin{exemple}\jya tɕiʑo ni kɤ-rɤma tu-nɯrɯz-tɕi ŋu\cmn 我们轮流劳动\end{exemple}\begin{sous-entrée}
\vedette{\hypertarget{}{\papi{ znɯrɯrɯz}}}\markboth{znɯrɯrɯz}{}\classe{vt}
\paradigme{\textit{dir :} \jya tɤ-}
\begin{définition}\fra utiliser l'un après l'autre\end{définition}
\begin{définition}\cmn 轮流着用\end{définition}
\begin{exemple}\jya kɯki a-ŋga ʁnɯz ki tu-znɯrɯrɯz-a ŋu, jisŋi ki tɤ-ŋga-t-a tɕe fso tɕe ci nɯ ɯ-βra tu-ŋge-a ŋu\cmn 我把这两件衣服轮流穿,今天穿了这件,明天就会穿那件\end{exemple}
\end{sous-entrée}\end{entrée}

\begin{entrée}
\vedette{\hypertarget{Ⓔnɯrzandɤɣ}{\papi{ nɯrzandɤɣ}}}\markboth{nɯrzandɤɣ}{}
\classe{vi}
\paradigme{\textit{dir :} \jya pɯ-}
\begin{définition}\fra attraper le mal des hauteurs\end{définition}
\begin{définition}\cmn 发生高山反应
\begin{déclaration} \étymologie{\papi{rdza.dug}}\end{déclaration}\end{définition}
\begin{exemple}\jya pjɤ-nɯrzandaɣ-a\cmn 我有了高山反应\end{exemple}
\begin{exemple}\jya aj mucin ʑo mɯ́j-nɯrzandaɣ-a\cmn 我根本就不会有高山反应\end{exemple}\end{entrée}

\begin{entrée}
\vedette{\hypertarget{Ⓔnɯrʑɯɣ}{\papi{ nɯrʑɯɣ}}}\markboth{nɯrʑɯɣ}{}
\classe{vt}
\paradigme{\textit{dir :} \jya pɯ-}
\begin{définition}\fra couper très vite grâce à un couteau bien aiguisé\end{définition}
\begin{définition}\cmn 切得很快
\begin{déclaration}\use{表示刀很锋利}\end{déclaration}\end{définition}
\begin{exemple}\jya paʁndza pa-nɯrʑɯɣ ʑo pa-rɤkrɯ\cmn 他把猪草切得很快(刀很锋利)\end{exemple}
\begin{exemple}\jya pɯ-nɯrʑɯɣ-a pɯ-ʁndzar-a\cmn 我锯得很快(锯子很锋利)\end{exemple}
\begin{exemple}\jya ftɕɤfkɤt pa-nɯrʑɯɣ ʑo\cmn 他很果断的指挥了别人\end{exemple}\end{entrée}

\begin{entrée}
\vedette{\hypertarget{Ⓔnɯʁɤri}{\papi{ nɯʁɤri}}}\markboth{nɯʁɤri}{}
\classe{vt}
\paradigme{\textit{dir :} \jya \_}\acception{1}
\begin{définition}\fra se mettre en face de\end{définition}
\begin{définition}\cmn 转身向\end{définition}
\begin{exemple}\jya aʑo tɕoχtsi lu-nɯʁɤri-a ŋu\cmn 我转身面向桌子\end{exemple}
\begin{exemple}\jya aʑo khɯɣɲɟɯ nɯ-nɯʁɤri-t-a\cmn 我向窗子转身了\end{exemple}\acception{2}
\begin{définition}\fra faire face à, tenir tête à\end{définition}
\begin{définition}\cmn 对付;阻挡\end{définition}
\begin{exemple}\jya aʑo nɤʑo tu-ta-nɯʁɤri jɤɣ\cmn 我可以对付你\end{exemple}
\begin{relation-sémantique}\antonyme{
\hyperlink{Ⓔnɯɕqhu}{\textit{ \papi{nɯɕqhu}}}
}\end{relation-sémantique}
\begin{relation-sémantique}\confer{
\hyperlink{Ⓔɯ-ʁɤri}{\textit{ \papi{ɯ-ʁɤri}}}
}\end{relation-sémantique}\end{entrée}

\begin{entrée}
\vedette{\hypertarget{Ⓔnɯʁgra}{\papi{ nɯʁgra}}}\markboth{nɯʁgra}{}
\classe{vt}
\paradigme{\textit{dir :} \jya tɤ-}
\begin{définition}\ 
\begin{déclaration}\grammar{denom}\end{déclaration}\end{définition}
\begin{définition}\fra considérer comme un ennemi\end{définition}
\begin{définition}\cmn 敌视
\begin{déclaration} \étymologie{\papi{dgra}}\end{déclaration}\end{définition}
\begin{exemple}\jya jiɕqha kɯ tú-wɣ-nɯʁgra-a ɲɯ-ŋu\cmn 那个人跟我有仇\end{exemple}\begin{sous-entrée}
\vedette{\hypertarget{}{\papi{ anɯʁgrɯʁgra}}}\markboth{anɯʁgrɯʁgra}{}\classe{vi}
\begin{définition}\fra se considérer les uns les autres comme des ennemis\end{définition}
\begin{définition}\cmn 互相敌视\end{définition}
\begin{relation-sémantique}\confer{
\hyperlink{Ⓔʁgra}{\textit{ \papi{ʁgra}}}
}\end{relation-sémantique}
\end{sous-entrée}\end{entrée}

\begin{entrée}
\vedette{\hypertarget{Ⓔnɯʁjoʁ}{\papi{ nɯʁjoʁ}}}\markboth{nɯʁjoʁ}{}\classe{vt}
\paradigme{\textit{dir :} \jya nɯ-}
\begin{définition}\ 
\begin{déclaration}\grammar{denom}\end{déclaration}\end{définition}\acception{1}
\begin{définition}\fra commander\end{définition}
\begin{définition}\cmn 使唤\end{définition}
\begin{exemple}\jya nɤʑo kɯ aʑo ɲɯ-kɯ-nɯʁjoʁ-a ɲɯ-ŋu\cmn 你在使唤我\end{exemple}
\begin{exemple}\jya aj ɲɯ-ta-nɯʁjoʁ, a-tʂha ci pɯ-rke\cmn 请你给我倒一点茶\end{exemple}
\begin{exemple}\jya nɯ-nɯʁjoʁ-a tɕe laχtɕha ka-nɯxtʂɯ\cmn 我使唤了他,他就把东西顺便带来了\end{exemple}
\begin{exemple}\jya nɯ-nɯʁjoʁ-a tɕe kɤ-z-nɯxtʂɯ-t-a\cmn 我使唤了他,令他把东西顺便带来了\end{exemple}\acception{2}
\begin{définition}\fra travailler pour quelqu'un\end{définition}
\begin{définition}\cmn 当别人的帮工\end{définition}
\begin{exemple}\jya a-kɯ-nɯʁjoʁ jɤ-sɯɣe-t-a\cmn 我雇佣了他\end{exemple}
\begin{relation-sémantique}\confer{
\hyperlink{Ⓔʁjoʁ}{\textit{ \papi{ʁjoʁ}}}
}\end{relation-sémantique}\end{entrée}

\begin{entrée}
\vedette{\hypertarget{Ⓔnɯʁjɯβtshɤt}{\papi{ nɯʁjɯβtshɤt}}}\markboth{nɯʁjɯβtshɤt}{}
\classe{vt}
\paradigme{\textit{dir :} \jya tɤ-}
\begin{définition}\ 
\begin{déclaration}\grammar{denom}\end{déclaration}\end{définition}
\begin{définition}\fra estimer\end{définition}
\begin{définition}\cmn 估计\end{définition}
\begin{exemple}\jya fsusqi-tɯrpa tɤ-nɯʁjɯβtshat-a\cmn 我估计有三十斤\end{exemple}
\begin{exemple}\jya tɤ-nɯʁjɯβtshat-a tɕe ɕoŋtɕa pɯ-ʁndzar-a\cmn 我估计了一下就锯了木料\end{exemple}
\begin{relation-sémantique}\confer{
\hyperlink{Ⓔʁjɯβtshɤt}{\textit{ \papi{ʁjɯβtshɤt}}}
}\end{relation-sémantique}\end{entrée}

\begin{entrée}
\vedette{\hypertarget{Ⓔnɯʁlɤwɯr}{\papi{ nɯʁlɤwɯr}}}\markboth{nɯʁlɤwɯr}{}
\classe{vi}
\paradigme{\textit{dir :} \jya tɤ-}
\begin{définition}\ 
\begin{déclaration}\grammar{denom}\end{déclaration}\end{définition}
\begin{définition}\fra faire soudainement\end{définition}
\begin{définition}\cmn 突然做\end{définition}
\begin{exemple}\jya tɤ-nɯʁlɤwɯr ʑo tɕe jɤ-ari\cmn 他突然就走了\end{exemple}
\begin{relation-sémantique}\confer{
\hyperlink{Ⓔʁlɤwɯr}{\textit{ \papi{ʁlɤwɯr}}}
}\end{relation-sémantique}\end{entrée}

\begin{entrée}
\vedette{\hypertarget{Ⓔnɯʁlɯmbɯɣ}{\papi{ nɯʁlɯmbɯɣ}}}\markboth{nɯʁlɯmbɯɣ}{}
\classe{vt}
\paradigme{\textit{dir :} \jya tɤ-}
\begin{définition}\fra estimer\end{définition}
\begin{définition}\cmn 估计\end{définition}
\begin{exemple}\jya tɤ-nɯʁlɯmbɯɣa\cmn 我估计了一下\end{exemple}
\begin{exemple}\jya @liangmi ɲɯ-ra ri tɤ-nɯʁlɯmbɯɣ-a tɕe pɯ-ʁndzar-a\cmn 需要两米的木头,我估计了一下就锯了\end{exemple}
\begin{exemple}\jya fso tɕe tɯ-mɯ lɤt mɤ-lɤt mɤxsi ma aj tɤ-nɯʁlɯmbɯɣ-a ɕti\cmn 明天不知道下不下雨,我只是估计一下\end{exemple}
\begin{exemple}\jya kɯki ŋu maʁ mɤxsi ma tɤ-nɯʁlɯmbɯɣ-a ɕti\cmn 不知道是不是正确的,这是我猜想的\end{exemple}
\begin{exemple}\jya tu-nɯʁlɯmbɯɣ-a ɕti wo\cmn 我猜的(一句话的意思)\end{exemple}
\begin{exemple}\jya nɯ tu-nɯʁlɯmbɯɣ-a ɕti ri, ɯʑo kɯ kɤ-nɤma nɯ sɤpe\cmn 我估计他会把工作做好\end{exemple}
\begin{relation-sémantique}\synonyme{
\hyperlink{Ⓔnɯʁjɯβtshɤt}{\textit{ \papi{nɯʁjɯβtshɤt}}}
}\end{relation-sémantique}\end{entrée}

\begin{entrée}
\vedette{\hypertarget{Ⓔnɯʁmaʁmi}{\papi{ nɯʁmaʁmi}}}\markboth{nɯʁmaʁmi}{}\classe{vi}
\paradigme{\textit{dir :} \jya tɤ-}
\begin{définition}\ 
\begin{déclaration}\grammar{denom}\end{déclaration}\end{définition}
\begin{définition}\fra être soldat, faire son service militaire\end{définition}
\begin{définition}\cmn 当兵\end{définition}
\begin{relation-sémantique}\confer{
\hyperlink{Ⓔʁmaʁmi}{\textit{ \papi{ʁmaʁmi}}}
}\end{relation-sémantique}\end{entrée}

\begin{entrée}
\vedette{\hypertarget{Ⓔnɯʁndomnɤt}{\papi{ nɯʁndomnɤt}}}\markboth{nɯʁndomnɤt}{}
\classe{vi}
\paradigme{\textit{dir :} \jya pɯ-}
\paradigme{\textit{dir :} \jya tɤ-}
\begin{définition}\ 
\begin{déclaration}\grammar{incorp}\end{déclaration}\end{définition}
\begin{définition}\fra répéter sans arrêt, radoter\end{définition}
\begin{définition}\cmn 重复讲说过的话;啰唆\end{définition}
\begin{exemple}\jya ma-pɯ-tɯ-nɯʁndomnɤt ntsɯ\cmn 你不要不停地重复讲同一句话\end{exemple}
\begin{exemple}\jya a-mu cho-rgɤz tɕe, wuma ʑo nɯʁndomnɤt\cmn 我母亲老了,不停地重复讲说过的话\end{exemple}
\begin{exemple}\jya ɯʑo cha ku-tshi tɕe, tu-nɯʁndomnɤt ɲɯ-ŋu\cmn 他喝了酒就会重复讲说过的话\end{exemple}
\begin{relation-sémantique}\confer{
\hyperlink{Ⓔtaʁndo}{\textit{ \papi{taʁndo}}}
}\end{relation-sémantique}\end{entrée}

\begin{entrée}
\vedette{\hypertarget{Ⓔnɯʁnoŋ}{\papi{ nɯʁnoŋ}}}\markboth{nɯʁnoŋ}{}\classe{vi}
\begin{définition}\fra avoir des remors\end{définition}
\begin{définition}\cmn 内疚;有愧\end{définition}
\begin{relation-sémantique}\synonyme{
\hyperlink{Ⓔɲɟɤt}{\textit{ \papi{ɲɟɤt}}}
}\end{relation-sémantique}
\begin{relation-sémantique}\confer{
\hyperlink{Ⓔtɯ-ʁnoŋ}{\textit{ \papi{tɯ-ʁnoŋ}}}
}\end{relation-sémantique}\end{entrée}

\begin{entrée}
\vedette{\hypertarget{Ⓔnɯʁɲɤlwa}{\papi{ nɯʁɲɤlwa}}}\markboth{nɯʁɲɤlwa}{}\classe{vi}
\paradigme{\textit{dir :} \jya pɯ-}
\begin{définition}\fra souffrir le martyre\end{définition}
\begin{définition}\cmn 受苦难\end{définition}
\begin{exemple}\jya pɯ-nɯʁɲɤlwa-a\cmn 我受了很多苦\end{exemple}
\begin{relation-sémantique}\confer{
\hyperlink{Ⓔʁɲɤlwa}{\textit{ \papi{ʁɲɤlwa}}}
}\end{relation-sémantique}\end{entrée}

\begin{entrée}
\vedette{\hypertarget{Ⓔnɯʁzɯɣ}{\papi{ nɯʁzɯɣ}}}\markboth{nɯʁzɯɣ}{}
\classe{vs}
\paradigme{\textit{dir :} \jya thɯ-}
\begin{définition}\fra beau\end{définition}
\begin{définition}\cmn 美观
\begin{déclaration} \étymologie{\papi{gzigs}}\end{déclaration}\end{définition}
\begin{exemple}\jya jiɕqha tɤ-tɕɯ nɯ ɲɯ-nɯʁzɯɣ\cmn 那个男子很英俊\end{exemple}
\begin{exemple}\jya jiɕqha laχtɕha ɲɯ-nɯʁzɯɣ\cmn 那个东西好看\end{exemple}\end{entrée}

\begin{entrée}
\vedette{\hypertarget{Ⓔnɯʁʑɯnɯ}{\papi{ nɯʁʑɯnɯ}}}\markboth{nɯʁʑɯnɯ}{}\classe{vs}
\paradigme{\textit{dir :} \jya thɯ-}
\paradigme{\textit{dir :} \jya tɤ-}
\begin{définition}\ 
\begin{déclaration}\grammar{denom}\end{déclaration}\end{définition}
\begin{définition}\fra grandir et devenir un jeune homme\end{définition}
\begin{définition}\cmn 长成青年\end{définition}
\begin{relation-sémantique}\confer{
\hyperlink{Ⓔnɯʁʑɯnɯ}{\textit{ \papi{nɯʁʑɯnɯ}}}
}\end{relation-sémantique}\end{entrée}

\begin{entrée}
\vedette{\hypertarget{Ⓔnɯsarsi}{\papi{ nɯsarsi}}}\markboth{nɯsarsi}{}\classe{vi}
\paradigme{\textit{dir :} \jya \_}
\begin{définition}\ 
\begin{déclaration}\grammar{denom}\end{déclaration}\end{définition}
\begin{définition}\fra ramasser des abricots\end{définition}
\begin{définition}\cmn 摘杏\end{définition}
\begin{relation-sémantique}\confer{
\hyperlink{Ⓔsarsi}{\textit{ \papi{sarsi}}}
}\end{relation-sémantique}\end{entrée}

\begin{entrée}
\vedette{\hypertarget{Ⓔnɯsaχɕɯβ}{\papi{ nɯsaχɕɯβ}}}\markboth{nɯsaχɕɯβ}{}\classe{vi}
\begin{définition}\fra faire une compétition\end{définition}
\begin{définition}\cmn 比试\end{définition}
\begin{exemple}\jya tɕiʑo ni tɤ-nɯ-saχɕɯβ-tɕi ri, ɯʑo kɯ pɯ́-wɣ-ɕɯnŋo-a\cmn 我们俩比试一下了,他把我打败了\end{exemple}\end{entrée}

\begin{entrée}
\vedette{\hypertarget{Ⓔnɯsaχsɯ}{\papi{ nɯsaχsɯ}}}\markboth{nɯsaχsɯ}{}\classe{vi}
\paradigme{\textit{dir :} \jya tɤ-}
\begin{définition}\ 
\begin{déclaration}\grammar{denom}\end{déclaration}\end{définition}
\begin{définition}\fra prendre le repas de midi\end{définition}
\begin{définition}\cmn 吃中午饭\end{définition}
\begin{exemple}\jya nɯsaχsɯ-j\cmn 我们吃中午饭\end{exemple}
\begin{exemple}\jya jɯfɕɯr sloχpɯn cho tɯɣrɤz tɤ-nɯsaχsɯ-j\cmn 我们昨天跟老师一起吃了中午饭\end{exemple}
\begin{exemple}\jya ɯ-tɤ-tɯ-nɯsaχsɯ?\cmn 你吃饭了吗?\end{exemple}
\begin{relation-sémantique}\confer{
\hyperlink{ⒺsaχsɯⒽ1}{\textit{ \papi{saχsɯ}}}
}\end{relation-sémantique}\end{entrée}

\begin{entrée}
\vedette{\hypertarget{Ⓔnɯsɤlɤɣ}{\papi{ nɯsɤlɤɣ}}}\markboth{nɯsɤlɤɣ}{}\classe{vi}
\paradigme{\textit{dir :} \jya nɯ}
\begin{définition}\fra manger de l'herbe\end{définition}
\begin{définition}\cmn 吃草\end{définition}
\begin{exemple}\jya nɯŋa ɲɯ-nɯsɤlɤɣ\cmn 牛在吃草\end{exemple}
\begin{exemple}\jya nɤʑo nɯ-nɯsɤlɤɣ\cmn 你吃草!(人对动物说)\end{exemple}
\begin{relation-sémantique}\synonyme{
\hyperlink{Ⓔnɯrɯ}{\textit{ \papi{nɯrɯ}}}
}\end{relation-sémantique}\begin{sous-entrée}
\vedette{\hypertarget{}{\papi{ znɯsɤlɤɣ}}}\markboth{znɯsɤlɤɣ}{}\classe{vt}
\paradigme{\textit{dir :} \jya nɯ-}
\begin{définition}\ 
\begin{déclaration}\grammar{caus}\end{déclaration}\end{définition}
\begin{définition}\fra faire manger de l'herbe\end{définition}
\begin{définition}\cmn 给动物喂草\end{définition}
\begin{définition}\cmn 我给你吃草(人对动物说)\end{définition}
\begin{exemple}\jya ɲɯ-ta-znɯsɤlɤɣ\end{exemple}
\end{sous-entrée}\end{entrée}

\begin{entrée}
\vedette{\hypertarget{Ⓔnɯsɤra}{\papi{ nɯsɤra}}}\markboth{nɯsɤra}{}\classe{vi}
\paradigme{\textit{dir :} \jya tɤ-}
\begin{définition}\fra se gratter (contre des arbres)\end{définition}
\begin{définition}\cmn 抓痒(奶牛;牦牛)\end{définition}\end{entrée}

\begin{entrée}
\vedette{\hypertarget{Ⓔnɯsɤraʁ}{\papi{ nɯsɤraʁ}}}\markboth{nɯsɤraʁ}{}
\classe{vt}
\paradigme{\textit{dir :} \jya tɤ-}
\begin{définition}\fra parier\end{définition}
\begin{définition}\cmn 打赌\end{définition}
\begin{exemple}\jya nɯ-sɤraʁ-tɕi\cmn 我们俩打赌\end{exemple}
\begin{exemple}\jya tɤ-nɯsɤraʁ-tɕi\cmn 我们俩打赌了\end{exemple}
\begin{exemple}\jya tɤ-sɯxtsɯɣ-a tɕe tɕhi a-tɤ-fse, mɯ-tɤ-sɯxtsɯɣ-a tɕe tɕhi fse nɯ-sɤraʁ-tɕi\cmn 我们俩打赌,如果我打中了的话就怎么样,打不中的话又怎么样\end{exemple}
\begin{relation-sémantique}\synonyme{
\hyperlink{Ⓔnɯtɤraʁ}{\textit{ \papi{nɯtɤraʁ}}}
}\end{relation-sémantique}\end{entrée}

\begin{entrée}
\vedette{\hypertarget{Ⓔnɯscɯʁzɯɣ}{\papi{ nɯscɯʁzɯɣ}}}\markboth{nɯscɯʁzɯɣ}{}
\classe{vs}
\begin{définition}\fra beau\end{définition}
\begin{définition}\cmn 美丽;漂亮
\begin{déclaration} \étymologie{\papi{skʲe.gzugs}}\end{déclaration}\end{définition}
\begin{exemple}\jya jiɕqha nɯ mɯ́j-nɯscɯʁzɯɣ\cmn 那个人不漂亮\end{exemple}
\begin{exemple}\jya jiɕqha nɯ ɲɯ-nɯscɯʁzɯɣ\cmn 那个人很漂亮\end{exemple}\end{entrée}

\begin{entrée}
\vedette{\hypertarget{Ⓔnɯsɣa}{\papi{ nɯsɣa}}}\markboth{nɯsɣa}{}
\classe{vs}
\paradigme{\textit{dir :} \jya kɤ-}
\begin{définition}\ 
\begin{déclaration}\grammar{denom}\end{déclaration}\end{définition}
\begin{définition}\fra rouiller\end{définition}
\begin{définition}\cmn 生锈\end{définition}
\begin{exemple}\jya ɕom ko-nɯsɣa\cmn 铁生锈了\end{exemple}
\begin{exemple}\jya laʁdɯn ɲɯ-nɯsɣa\cmn 工具生锈\end{exemple}
\begin{relation-sémantique}\confer{
\hyperlink{Ⓔsɣa}{\textit{ \papi{sɣa}}}
}\end{relation-sémantique}\end{entrée}

\begin{entrée}
\vedette{\hypertarget{Ⓔnɯskɤt}{\papi{ nɯskɤt}}}\markboth{nɯskɤt}{}
\classe{vt}
\begin{définition}\fra parler\end{définition}
\begin{définition}\cmn 说话
\begin{déclaration}\use{主要用于否定式}\end{déclaration}\end{définition}
\begin{exemple}\jya nɤʑo mɤ-tɯ-nɯskɤt me\cmn 你无话不说\end{exemple}
\begin{exemple}\jya ɯʑo kɯ mɯ-ta-nɯskɤt ʑo me\cmn 他以前说很多话\end{exemple}
\begin{exemple}\jya nɯ kɯ-fse kɤ-nɯskɤt mɤ-ra\cmn 不要这样说\end{exemple}
\begin{exemple}\jya mɤ-nɯskɤt-ndʑi ʑo maŋe\cmn 他们俩没有什么不说的\end{exemple}
\begin{relation-sémantique}\confer{
\hyperlink{Ⓔtɯ-skɤt}{\textit{ \papi{tɯ-skɤt}}}
}\end{relation-sémantique}\end{entrée}

\begin{entrée}
\vedette{\hypertarget{Ⓔnɯskhrɯ}{\papi{ nɯskhrɯ}}}\markboth{nɯskhrɯ}{}\classe{vt}\paradigme{\textit{dir :} \jya kɤ-}
\begin{définition}\fra être enceinte d'un enfant\end{définition}
\begin{définition}\cmn 怀上(孩子)\end{définition}
\begin{exemple}\jya ɯ-rɟit ko-nɯskhrɯ\cmn 她怀上了孩子\end{exemple}
\begin{relation-sémantique}\confer{
\hyperlink{Ⓔtɯ-skhrɯ}{\textit{ \papi{tɯ-skhrɯ}}}
}\end{relation-sémantique}\end{entrée}

\begin{entrée}
\vedette{\hypertarget{Ⓔnɯslɯɣ}{\papi{ nɯslɯɣ}}}\markboth{nɯslɯɣ}{}\classe{vi}
\paradigme{\textit{dir :} \jya tɤ-}
\begin{définition}\fra être hourdé de, être sali\end{définition}
\begin{définition}\cmn 沾上\end{définition}
\begin{exemple}\jya a-ŋga ɯ-thoʁ pjɤ-k-ɤtɤr-ci tɕe to-nɯslɯɣ\cmn 我衣服掉到地上,沾上了灰尘\end{exemple}
\begin{exemple}\jya soʁma ɯ-ŋgɯ ju-kɯ-ɕe tɕe, tɯ-ŋga tu-nɯslɯɣ ŋu\cmn 走进麦草里,衣服会沾上麦草\end{exemple}\begin{sous-entrée}
\vedette{\hypertarget{}{\papi{ znɯslɯɣ}}}\markboth{znɯslɯɣ}{}\classe{vt}
\paradigme{\textit{dir :} \jya tɤ-}
\begin{définition}\fra salir avec\end{définition}
\begin{définition}\cmn 使沾上\end{définition}
\begin{exemple}\jya ɯʑo kɯ ɯ-ŋga tɤrcoʁ to-znɯslɯɣ\cmn 他把自己衣服沾上了泥巴\end{exemple}
\end{sous-entrée}\end{entrée}

\begin{entrée}
\vedette{\hypertarget{Ⓔnɯslɯt}{\papi{ nɯslɯt}}}\markboth{nɯslɯt}{}\classe{vi}
\paradigme{\textit{dir :} \jya tɤ-}
\begin{définition}\fra être plein de poussières et de saletés (après avoir été laissé dans un endroit sale)\end{définition}
\begin{définition}\cmn (不管地下脏躺在哪里)满身都是灰尘,沾满灰尘和脏东西\end{définition}
\begin{exemple}\jya a-ŋga to-nɯslɯt\cmn 我衣服上沾满了灰尘\end{exemple}\begin{sous-entrée}
\vedette{\hypertarget{}{\papi{ znɯslɯt}}}\markboth{znɯslɯt}{}\classe{vt}
\paradigme{\textit{dir :} \jya tɤ-}
\begin{exemple}\jya nɤ-tɕɯ kɯ ɯ-ŋga ɯ-tó-znɯslɯt?\cmn 你儿子把衣服放在那里沾上灰尘了吗?\end{exemple}
\begin{exemple}\jya ɯ-thoʁ ɲɯ-ɴqhi tɕe ɲɯ-kɯ-znɯslɯt\cmn 地面很脏,令人沾上灰尘\end{exemple}
\end{sous-entrée}\end{entrée}

\begin{entrée}
\vedette{\hypertarget{Ⓔnɯsmɤn}{\papi{ nɯsmɤn}}}\markboth{nɯsmɤn}{}\classe{vt}
\paradigme{\textit{dir :} \jya tɤ-}
\begin{définition}\ 
\begin{déclaration}\grammar{denom}\end{déclaration}\end{définition}
\begin{définition}\fra guérir\end{définition}
\begin{définition}\cmn 治病\end{définition}
\begin{exemple}\jya nɯnɯ ɣɯ ɯ-jaʁ pjɤ-ɴɢraʁ tɕe to-nɯsmɤn\cmn 他的手破了,(医生)就把它治好了\end{exemple}
\begin{exemple}\jya smɤnba kɯ to-nɯsmɤn\cmn 医生把他的病治好了\end{exemple}
\begin{exemple}\jya smɤnba kɯ smɤn to-βzu tɕe to-nɯsmɤn\cmn 医生用药把他治好了\end{exemple}
\begin{exemple}\jya ɲɯ-ngo tɕe ta-nɯsmɤn\cmn 他病了,(医生)把病治好了\end{exemple}
\begin{exemple}\jya smɤnba kɯ tɤ́-wɣ-nɯsman-a\cmn 医生给我治了病\end{exemple}\begin{sous-entrée}
\vedette{\hypertarget{}{\papi{ znɯsmɤn}}}\markboth{znɯsmɤn}{}\classe{vt}
\paradigme{\textit{dir :} \jya tɤ-}
\begin{définition}\ 
\begin{déclaration}\grammar{caus}\end{déclaration}\end{définition}
\begin{définition}\fra guérir avec\end{définition}
\begin{définition}\cmn 用……治病\end{définition}
\begin{exemple}\jya ɯ-smɤn ra la-tsɯm tɕe ɕ-to-z-nɯsmɤn tɕe to-pe\cmn 他带了药,就把她治好了\end{exemple}
\begin{relation-sémantique}\confer{
\hyperlink{Ⓔaɣɯsmɤn}{\textit{ \papi{aɣɯsmɤn}}}
}\end{relation-sémantique}
\begin{relation-sémantique}\confer{
\hyperlink{Ⓔsmɤn}{\textit{ \papi{smɤn}}}
}\end{relation-sémantique}
\end{sous-entrée}\begin{sous-entrée}
\vedette{\hypertarget{}{\papi{ ʑɣɤnɯsmɤn}}}\markboth{ʑɣɤnɯsmɤn}{}\classe{vi}
\begin{définition}\ 
\begin{déclaration}\grammar{refl}\end{déclaration}\end{définition}
\begin{définition}\fra se traiter\end{définition}
\begin{définition}\cmn 看病\end{définition}
\begin{exemple}\jya nɤʑo ɕɯ-tɯ-ʑɣɤnɯsmɤn ɯ-ŋu?\cmn 你去看病\end{exemple}
\end{sous-entrée}\end{entrée}

\begin{entrée}
\vedette{\hypertarget{Ⓔnɯsmɤphɤβ}{\papi{ nɯsmɤphɤβ}}}\markboth{nɯsmɤphɤβ}{}
\classe{vt}
\paradigme{\textit{dir :} \jya pɯ-}
\begin{définition}\fra humilier\end{définition}
\begin{définition}\cmn 侮辱;污蔑\end{définition}
\begin{exemple}\jya pjɤ-nɯsmɤphɤβ\cmn 他侮辱了他\end{exemple}
\begin{exemple}\jya pjɯ-kɯ-nɯsmɤphaβ-a ɲɯ-ŋu\cmn 你在侮辱我\end{exemple}\end{entrée}

\begin{entrée}
\vedette{\hypertarget{Ⓔnɯsmɯɣjɯm}{\papi{ nɯsmɯɣjɯm}}}\markboth{nɯsmɯɣjɯm}{}
\classe{vi}
\begin{définition}\fra se chauffer au feu\end{définition}
\begin{définition}\cmn 烤火取暖\end{définition}
\begin{relation-sémantique}\synonyme{
\hyperlink{Ⓔnɯmbjɯm}{\textit{ \papi{nɯmbjɯm}}}
}\end{relation-sémantique}\end{entrée}

\begin{entrée}
\vedette{\hypertarget{Ⓔnɯsmɯlɤm}{\papi{ nɯsmɯlɤm}}}\markboth{nɯsmɯlɤm}{}
\classe{vt}
\paradigme{\textit{dir :} \jya nɯ-}
\begin{définition}\fra espérer\end{définition}
\begin{définition}\cmn 愿望;盼望\end{définition}
\begin{exemple}\jya ɬasa kɤ-ɕe ɲɯ-nɯsmɯlɤm\cmn 他很希望去拉萨\end{exemple}
\begin{exemple}\jya nɤ-kɤ-nɯsmɯlɤm nɯ a-pɯ-fse\cmn 祝你愿望实现\end{exemple}
\begin{exemple}\jya pjɯ-kɤ-cha nɯ ɲɯ-nɯsmɯlam-a\cmn 我希望(这件事情)成功\end{exemple}
\begin{relation-sémantique}\confer{
\hyperlink{Ⓔsmɯlɤm}{\textit{ \papi{smɯlɤm}}}
}\end{relation-sémantique}\end{entrée}

\begin{entrée}
\vedette{\hypertarget{Ⓔnɯsmɯrjɯɣ}{\papi{ nɯsmɯrjɯɣ}}}\markboth{nɯsmɯrjɯɣ}{}
\classe{vt}
\paradigme{\textit{dir :} \jya thɯ-}
\begin{définition}\fra courber à la chaleur\end{définition}
\begin{définition}\cmn 用高温令木条变形;弄弯\end{définition}
\begin{exemple}\jya ʑmbrɯɟoʁ ci ɣɤʑu tɕe nɯ-nɯsmɯrjɯɣ-a\cmn 我把杨柳枝条弄弯了\end{exemple}
\begin{exemple}\jya kɯki si ki thɯ-nɯsmɯrjɯɣ-a\cmn 我把这块木条弄弯了\end{exemple}
\begin{exemple}\jya kɯki ɲɯ-jpum tɕe, kɤ-nɯsmɯrjɯɣ mɯ́j-khɯ\cmn 这个东西很粗,不能弄弯\end{exemple}\end{entrée}

\begin{entrée}
\vedette{\hypertarget{Ⓔnɯsnɯɲaʁ}{\papi{ nɯsnɯɲaʁ}}}\markboth{nɯsnɯɲaʁ}{}\classe{vt}
\paradigme{\textit{dir :} \jya tɤ-}
\begin{définition}\ 
\begin{déclaration}\grammar{incorp}\end{déclaration}\end{définition}
\begin{définition}\fra causer du mal\end{définition}
\begin{définition}\cmn 伤害,陷害\end{définition}
\begin{exemple}\jya tɤ-nɯsnɯɲaʁ-a\cmn 我陷害了他\end{exemple}
\begin{exemple}\jya jiɕqha nɯ kɯ tú-wɣ-nɯsnɯsɲaʁ-a ɲɯ-ŋu\cmn 那个人在陷害我\end{exemple}
\begin{exemple}\jya tɤ́-wɣ-nɯsnɯɲaʁ-a\cmn 他陷害了我\end{exemple}
\begin{relation-sémantique}\confer{
\hyperlink{Ⓔsnɯɲaʁ}{\textit{ \papi{snɯɲaʁ}}}
}\end{relation-sémantique}\end{entrée}

\begin{entrée}
\vedette{\hypertarget{Ⓔnɯsɲaŋne}{\papi{ nɯsɲaŋne}}}\markboth{nɯsɲaŋne}{}
\classe{vi}
\paradigme{\textit{dir :} \jya pɯ-}
\begin{définition}\fra jeûner\end{définition}
\begin{définition}\cmn 念哑巴经(禁食斋)\end{définition}
\begin{exemple}\jya pɯ-nɯsɲaŋne-a\cmn 我念了哑巴经\end{exemple}
\begin{relation-sémantique}\confer{
\hyperlink{Ⓔsɲaŋne}{\textit{ \papi{sɲaŋne}}}
}\end{relation-sémantique}
\begin{relation-sémantique}\confer{
\hyperlink{Ⓔrɯsɲaŋne}{\textit{ \papi{rɯsɲaŋne}}}
}\end{relation-sémantique}\end{entrée}

\begin{entrée}
\vedette{\hypertarget{Ⓔnɯsɲɤtqha}{\papi{ nɯsɲɤtqha}}}\markboth{nɯsɲɤtqha}{}\classe{vi}
\paradigme{\textit{dir :} \jya tɤ-}
\begin{définition}\ 
\begin{déclaration}\grammar{incorp}\end{déclaration}\end{définition}
\begin{définition}\fra faire une ruade (lorsque le cheval ne supporte plus sa selle)\end{définition}
\begin{définition}\cmn (因为受不了鞍子)尥蹶子\end{définition}
\begin{exemple}\jya mbro to-nɯsɲɤtqha\cmn 马尥蹶子了\end{exemple}
\begin{relation-sémantique}\synonyme{
\hyperlink{Ⓔcɤmtsaʁ}{\textit{ \papi{cɤmtsaʁ}}}
}\end{relation-sémantique}
\begin{relation-sémantique}\confer{
\hyperlink{Ⓔsɲɤt}{\textit{ \papi{sɲɤt}}}
}\end{relation-sémantique}
\begin{relation-sémantique}\confer{
\hyperlink{Ⓔqha}{\textit{ \papi{qha}}}
}\end{relation-sémantique}\end{entrée}

\begin{entrée}
\vedette{\hypertarget{Ⓔnɯsɲɯβri}{\papi{ nɯsɲɯβri}}}\markboth{nɯsɲɯβri}{}\classe{vt}
\begin{définition}\fra chérir\end{définition}
\begin{définition}\cmn 心疼\end{définition}
\begin{relation-sémantique}\synonyme{
\hyperlink{Ⓔnɤrɕɤmŋɤm}{\textit{ \papi{nɤrɕɤmŋɤm}}}
}\end{relation-sémantique}\end{entrée}

\begin{entrée}
\vedette{\hypertarget{Ⓔnɯsŋom}{\papi{ nɯsŋom}}}\markboth{nɯsŋom}{}
\classe{vt}
\paradigme{\textit{dir :} \jya nɯ-}
\begin{définition}\ 
\begin{déclaration}\grammar{appl}\end{déclaration}\end{définition}
\begin{définition}\fra désirer, convoiter\end{définition}
\begin{définition}\cmn 贪\end{définition}
\begin{exemple}\jya nɯ-nɯsŋom-a\cmn 我贪(这个东西)了\end{exemple}
\begin{exemple}\jya aʑo jiɕqha tɯ-xtsa nɯ ɲɯ-nɯsŋom-a\cmn 我很想得到那双鞋子\end{exemple}
\begin{exemple}\jya tɯ-ŋga nɯ ɲɯ-nɯsŋom-a\cmn 我很想得到那件衣服\end{exemple}
\begin{exemple}\jya nɯ kɤ-χtɯ ma-nɯ-tɯ-nɯsŋom\cmn 你不要有买那个的欲望(买不完)\end{exemple}
\begin{exemple}\jya tɯ-rɟɯ nɯ-nɯsŋom-a\cmn 我贪了财\end{exemple}
\begin{relation-sémantique}\confer{
\hyperlink{Ⓔsŋom}{\textit{ \papi{sŋom}}}
}\end{relation-sémantique}\end{entrée}

\begin{entrée}
\vedette{\hypertarget{Ⓔnɯspjɤtɕha}{\papi{ nɯspjɤtɕha}}}\markboth{nɯspjɤtɕha}{}\classe{vt}
\paradigme{\textit{dir :} \jya tɤ-}
\begin{définition}\fra faire une mauvaise action\end{définition}
\begin{définition}\cmn 做坏事;搞鬼\end{définition}
\begin{exemple}\jya ki nɤʑo tɤ-tɯ-nɯspjɤtɕha-t ŋu!\cmn 这是你搞的鬼!\end{exemple}
\begin{relation-sémantique}\confer{
\hyperlink{Ⓔspjɤtɕha}{\textit{ \papi{spjɤtɕha}}}
}\end{relation-sémantique}\end{entrée}

\begin{entrée}
\vedette{\hypertarget{Ⓔnɯsqar}{\papi{ nɯsqar}}}\markboth{nɯsqar}{}\classe{vs}
\paradigme{\textit{dir :} \jya tɤ-}
\begin{définition}\fra facile à séparer (fils)\end{définition}
\begin{définition}\cmn 容易分开\end{définition}
\begin{exemple}\jya ki kɤ-taʁ ɲɯ-mbat ma ɲɯ-nɯsqar\cmn 因为(线)容易分开,所以很好织\end{exemple}
\begin{relation-sémantique}\confer{
\hyperlink{Ⓔɯ-sqar}{\textit{ \papi{ɯ-sqar}}}
}\end{relation-sémantique}\end{entrée}

\begin{entrée}
\vedette{\hypertarget{Ⓔnɯsroʁmbrɤt}{\papi{ nɯsroʁmbrɤt}}}\markboth{nɯsroʁmbrɤt}{}\classe{vi}
\paradigme{\textit{dir :} \jya thɯ-}
\begin{définition}\ 
\begin{déclaration}\grammar{incorp}\end{déclaration}\end{définition}
\begin{définition}\fra se débattre en agonisant\end{définition}
\begin{définition}\cmn 临死挣扎\end{définition}
\begin{relation-sémantique}\confer{
\hyperlink{Ⓔmbrɤt}{\textit{ \papi{mbrɤt}}}
}\end{relation-sémantique}
\begin{relation-sémantique}\confer{
\hyperlink{Ⓔtɯ-sroʁ}{\textit{ \papi{tɯ-sroʁ}}}
}\end{relation-sémantique}\end{entrée}

\begin{entrée}
\vedette{\hypertarget{Ⓔnɯsrɯɣndɤr}{\papi{ nɯsrɯɣndɤr}}}\markboth{nɯsrɯɣndɤr}{}\classe{vi}
\paradigme{\textit{dir :} \jya nɯ-}
\begin{définition}\fra avoir de l'acné\end{définition}
\begin{définition}\cmn 长青春痘\end{définition}
\begin{exemple}\jya ɯʑo ɲɯ-nɯsrɯɣndɤr\cmn 他长青春痘\end{exemple}
\begin{relation-sémantique}\confer{
\hyperlink{Ⓔsrɯndɤr}{\textit{ \papi{srɯndɤr}}}
}\end{relation-sémantique}\end{entrée}

\begin{entrée}
\vedette{\hypertarget{Ⓔnɯstu}{\papi{ nɯstu}}}\markboth{nɯstu}{}\classe{vs}
\paradigme{\textit{dir :} \jya tɤ-}
\begin{définition}\fra être correct\end{définition}
\begin{définition}\cmn 准\end{définition}
\begin{exemple}\jya ɯ-jaʁ ɲɯ-nɯstu\cmn 他的枪法很准\end{exemple}
\begin{relation-sémantique}\confer{
\hyperlink{Ⓔɯ-stuⒽ2}{\textit{ \papi{ɯ-stu2}}}
}\end{relation-sémantique}\begin{sous-entrée}
\vedette{\hypertarget{}{\papi{ znɯstu}}}\markboth{znɯstu}{}\classe{vt}
\paradigme{\textit{dir :} \jya \_}
\begin{définition}\ 
\begin{déclaration}\grammar{caus}\end{déclaration}\end{définition}\acception{1}
\begin{définition}\fra faire correctement\end{définition}
\begin{définition}\cmn 做得准\end{définition}
\begin{exemple}\jya kɤ-ti ta-z-nɯstu\cmn 他说话说准了\end{exemple}
\begin{exemple}\jya kɤ-ɕe ka-z-nɯstu\cmn 他走准了\end{exemple}\acception{2}
\begin{définition}\fra atteindre la cible\end{définition}
\begin{définition}\cmn 射中\end{définition}
\end{sous-entrée}\end{entrée}

\begin{entrée}
\vedette{\hypertarget{Ⓔnɯstɤβtshɤt}{\papi{ nɯstɤβtshɤt}}}\markboth{nɯstɤβtshɤt}{}\classe{vi}
\paradigme{\textit{dir :} \jya tɤ-}
\begin{définition}\fra faire un concours de force\end{définition}
\begin{définition}\cmn 比力气\end{définition}
\begin{relation-sémantique}\confer{
\hyperlink{Ⓔstɤβtshɤt}{\textit{ \papi{stɤβtshɤt}}}
}\end{relation-sémantique}\end{entrée}

\begin{entrée}
\vedette{\hypertarget{Ⓔnɯstɤraʁndo}{\papi{ nɯstɤraʁndo}}}\markboth{nɯstɤraʁndo}{}
\classe{vi}
\paradigme{\textit{dir :} \jya tɤ-}
\begin{définition}\fra se parler à soi-même\end{définition}
\begin{définition}\cmn 喃喃自语;自己跟自己说话
\end{définition}
\begin{exemple}\jya tɤ-nɯstɤraʁndo-a\cmn 我自己跟自己说话了\end{exemple}
\begin{relation-sémantique}\confer{
\hyperlink{Ⓔtaʁndo}{\textit{ \papi{taʁndo}}}
}\end{relation-sémantique}
\begin{relation-sémantique}\confer{
\hyperlink{Ⓔɯ-sti}{\textit{ \papi{ɯ-sti}}}
}\end{relation-sémantique}\end{entrée}

\begin{entrée}
\vedette{\hypertarget{Ⓔnɯstɤrɟɯɣ}{\papi{ nɯstɤrɟɯɣ}}}\markboth{nɯstɤrɟɯɣ}{}
\classe{vt}
\paradigme{\textit{dir :} \jya \_}
\begin{définition}\ 
\begin{déclaration}\grammar{deidph}\end{déclaration}\end{définition}
\begin{définition}\fra courir\end{définition}
\begin{définition}\cmn 跑\end{définition}
\begin{exemple}\jya ɲɯ-nɯstɤrɟɯɣ\cmn 他在跑\end{exemple}
\begin{exemple}\jya ɲɯ-nɯstɤrɟɯɣ tɕe kɤ-ari\cmn 他跑去了\end{exemple}
\begin{relation-sémantique}\confer{
\hyperlink{Ⓔstɤrɟɯɣ}{\textit{ \papi{stɤrɟɯɣ}}}
}\end{relation-sémantique}
\begin{relation-sémantique}\confer{
\hyperlink{ⒺrɟɯɣⒽ1}{\textit{ \papi{rɟɯɣ1}}}
}\end{relation-sémantique}\end{entrée}

\begin{entrée}
\vedette{\hypertarget{Ⓔnɯsthamtɕɤt}{\papi{ nɯsthamtɕɤt}}}\markboth{nɯsthamtɕɤt}{}\classe{adv}
\begin{définition}\fra autant\end{définition}
\begin{définition}\cmn 那么多
\begin{déclaration} \étymologie{\papi{tʰams.tɕad}}\end{déclaration}\end{définition}
\end{entrée}

\begin{entrée}
\vedette{\hypertarget{Ⓔnɯsthoʁ}{\papi{ nɯsthoʁ}}}\markboth{nɯsthoʁ}{}\classe{vt}
\paradigme{\textit{dir :} \jya pɯ-}
\begin{définition}\fra avoir des relations sexuelles\end{définition}
\begin{définition}\cmn 性交\end{définition}
\begin{exemple}\jya tɕheme pɯ-nɯsthoʁ-a, pɯ-tɯ-nɯsthoʁ\end{exemple}
\begin{relation-sémantique}\confer{
\hyperlink{Ⓔsthoʁ}{\textit{ \papi{sthoʁ}}}
}\end{relation-sémantique}\end{entrée}

\begin{entrée}
\vedette{\hypertarget{Ⓔnɯsthɯt}{\papi{ nɯsthɯt}}}\markboth{nɯsthɯt}{}
\begin{relation-sémantique}\confer{
\hyperlink{Ⓔsthɯt}{\textit{ \papi{sthɯt}}}
}\end{relation-sémantique}\end{entrée}

\begin{entrée}
\vedette{\hypertarget{Ⓔnɯsɯku}{\papi{ nɯsɯku}}}\markboth{nɯsɯku}{}
\classe{vi}
\paradigme{\textit{dir :} \jya tɤ-}
\begin{définition}\ 
\begin{déclaration}\grammar{denom}\end{déclaration}\end{définition}
\begin{définition}\fra grimper aux arbres\end{définition}
\begin{définition}\cmn 爬树\end{définition}
\begin{exemple}\jya ɣzɯ kɤ-nɯsɯku ɲɯ-cha\cmn 猴子会爬树\end{exemple}
\begin{exemple}\jya tɯrme kɤ-nɯsɯku kɯ-cha tu, nɯnɯ kɯ-cha mɤ-kɯ-cha tu\cmn 有的人会爬树,这件事,有的会有的不会\end{exemple}
\begin{exemple}\jya ma-tɯ-nɯsɯku ma tɯ-atɤr\cmn 你不要爬树,你会摔下来的\end{exemple}
\begin{relation-sémantique}\confer{
\hyperlink{Ⓔsɯku}{\textit{ \papi{sɯku}}}
}\end{relation-sémantique}\end{entrée}

\begin{entrée}
\vedette{\hypertarget{Ⓔnɯsɯkho}{\papi{ nɯsɯkho}}}\markboth{nɯsɯkho}{}
\classe{vt}
\paradigme{\textit{dir :} \jya nɯ-}
\begin{définition}\fra extorquer, dévaliser\end{définition}
\begin{définition}\cmn 抢\end{définition}
\begin{exemple}\jya nɯ ma-nɯ-tɯ-nɯsɯkhɤm\cmn 你不要抢这个东西\end{exemple}
\begin{exemple}\jya paʁmu kɯ ɯ-rɟit kɤ-ndza ɲɯ-nɯsɯkhɤm ɲɯ-ŋu\cmn 母猪总是抢它的崽子的食物\end{exemple}
\begin{exemple}\jya tɤ-pɤtso kɯ ɯ-zda ɯ-kɯmtɕhɯ na-nɯsɯkho\cmn 小孩子抢了他的同学的玩具\end{exemple}
\begin{exemple}\jya ma-nɯ-kɯ-nɯsɯkho-a\cmn 你别抢我的东西\end{exemple}
\begin{relation-sémantique}\confer{
\hyperlink{ⒺkhoⒽ1}{\textit{ \papi{kho1}}}
}\end{relation-sémantique}
\begin{relation-sémantique}\confer{
\hyperlink{Ⓔanɯsɯkhɯkho}{\textit{ \papi{anɯsɯkhɯkho}}}
}\end{relation-sémantique}\begin{sous-entrée}
\vedette{\hypertarget{}{\papi{ sɤnɯsɯkho}}}\markboth{sɤnɯsɯkho}{} (\variante{nɯsɤsɯkho}) \classe{vi}
\begin{définition}\fra extorquer, dévaliser les gens\end{définition}
\begin{définition}\cmn 抢人家的东西\end{définition}
\begin{exemple}\jya ɯʑo sɤnɯsɯkho ŋgrɤl\cmn 他抢别人的东西\end{exemple}
\begin{relation-sémantique}\synonyme{
\hyperlink{Ⓔnɯtɕaχpa}{\textit{ \papi{nɯtɕaχpa}}}
}\end{relation-sémantique}
\end{sous-entrée}\begin{sous-entrée}
\vedette{\hypertarget{}{\papi{ znɯsɤsɯkho}}}\markboth{znɯsɤsɯkho}{}\classe{vt}
\begin{définition}\fra inciter / forcer à dévaliser les gens\end{définition}
\begin{définition}\cmn 致使……抢东西\end{définition}
\end{sous-entrée}\end{entrée}

\begin{entrée}
\vedette{\hypertarget{Ⓔnɯsɯmŋɤn}{\papi{ nɯsɯmŋɤn}}}\markboth{nɯsɯmŋɤn}{}
\classe{vt}
\paradigme{\textit{dir :} \jya tɤ-}
\begin{définition}\fra se méfier\end{définition}
\begin{définition}\cmn 怀疑
\begin{déclaration} \étymologie{\papi{sems.ŋan}}\end{déclaration}\end{définition}
\begin{exemple}\jya jiɕqha nɯ ŋu maʁ mɤ-xsi ri tɤ-nɯsɯmŋan-a\cmn 不知道是不是他(偷了东西),但是我怀疑他了\end{exemple}
\begin{exemple}\jya ma-tɤ-kɯ-nɯsɯmŋan-a ma aʑo maʁ-a\cmn 你不要怀疑我,不是我做的\end{exemple}
\begin{exemple}\jya ma-tɤ-tɯ-nɯsɯmŋɤn\cmn 你不要怀疑他\end{exemple}
\begin{exemple}\jya sɯmŋɤn\end{exemple}\end{entrée}

\begin{entrée}
\vedette{\hypertarget{Ⓔnɯsɯmʁɲiz}{\papi{ nɯsɯmʁɲiz}}}\markboth{nɯsɯmʁɲiz}{}\classe{vi}
\paradigme{\textit{dir :} \jya tɤ-}
\begin{définition}\fra hésiter\end{définition}
\begin{définition}\cmn 犹豫
\begin{déclaration} \étymologie{\papi{sems.gɲis}}\end{déclaration}\end{définition}
\begin{exemple}\jya nɤʑo ɲɯ-tɯ-nɯsɯmʁɲiz netɕi\cmn 你还在犹豫啊\end{exemple}
\end{entrée}

\begin{entrée}
\vedette{\hypertarget{Ⓔnɯsɯmɯzdɯɣ}{\papi{ nɯsɯmɯzdɯɣ}}}\markboth{nɯsɯmɯzdɯɣ}{}
\classe{vi}
\paradigme{\textit{dir :} \jya thɯ-}
\begin{définition}\ 
\begin{déclaration}\grammar{denom}\end{déclaration}\end{définition}\acception{1}
\begin{définition}\fra être inquiet\end{définition}
\begin{définition}\cmn 担心\end{définition}
\begin{exemple}\jya nɤ-rɟit ɲɯ-ngo, ma-tɯ-nɯsɯmɯzdɯɣ\cmn 你儿子生病了,但是不要担心\end{exemple}\acception{2}
\begin{définition}\fra être malheureux, être triste\end{définition}
\begin{définition}\cmn 不高兴;伤心
\begin{déclaration} \étymologie{\papi{sems.sdug}}\end{déclaration}\end{définition}
\begin{exemple}\jya thɯ-nɯsɯmɯzdɯɣ mɤ-ra\cmn 你不要伤心\end{exemple}
\begin{exemple}\jya ma-tɯ-nɯsɯmɯzdɯɣ\cmn 你不要伤心\end{exemple}
\begin{exemple}\jya tɕhindʐa ku-tɯ-nɯsɯmɯzdɯɣ?\cmn 你为什么不高兴?\end{exemple}
\begin{relation-sémantique}\confer{
\hyperlink{Ⓔsɯmɯzdɯɣ}{\textit{ \papi{sɯmɯzdɯɣ}}}
}\end{relation-sémantique}\end{entrée}

\begin{entrée}
\vedette{\hypertarget{Ⓔnɯsɯŋgɯ}{\papi{ nɯsɯŋgɯ}}}\markboth{nɯsɯŋgɯ}{}
\classe{vi}
\begin{définition}\fra aller dans la forêt\end{définition}
\begin{définition}\cmn 在森林里逛;打猎\end{définition}
\begin{exemple}\jya kɯ-nɯsɯŋgɯ jɤ-ari-a\cmn 我去了森林\end{exemple}
\begin{exemple}\jya ɕɯ-nɯsɯŋgɯ-a\cmn 我要去森林\end{exemple}
\begin{exemple}\jya nɤʑo ɕɯ-tɯ-nɯsɯŋgɯ ɯ-ŋu?\cmn 你去森林吗?\end{exemple}
\begin{relation-sémantique}\confer{
\hyperlink{Ⓔsɯŋgɯ}{\textit{ \papi{sɯŋgɯ}}}
}\end{relation-sémantique}\end{entrée}

\begin{entrée}
\vedette{\hypertarget{Ⓔnɯsɯɴɢoʁ}{\papi{ nɯsɯɴɢoʁ}}}\markboth{nɯsɯɴɢoʁ}{}
\classe{vi}
\paradigme{\textit{dir :} \jya nɯ-}
\paradigme{\textit{dir :} \jya \_}
\begin{définition}\ 
\begin{déclaration}\grammar{denom}\end{déclaration}\end{définition}
\begin{définition}\fra ramasser du bois mort\end{définition}
\begin{définition}\cmn 捡干柴\end{définition}
\begin{exemple}\jya ji-si chɤ-k-ɤrɕo-ci tɕe, ʑ-nɯ-nɯsɯɴɢoʁ-a pɯ-ra\cmn 我们的柴用完了,所以我就捡柴去了\end{exemple}
\begin{exemple}\jya sɯŋgɯ pɯ-nɯsɯɴɢoʁ-a\cmn 我在森林里捡柴了\end{exemple}
\begin{relation-sémantique}\confer{
\hyperlink{ⒺsiⒽ1}{\textit{ \papi{si1}}}
}\end{relation-sémantique}
\begin{relation-sémantique}\confer{
\hyperlink{Ⓔtaɴɢoʁ}{\textit{ \papi{taɴɢoʁ}}}
}\end{relation-sémantique}\end{entrée}

\begin{entrée}
\vedette{\hypertarget{Ⓔnɯsɯrtoʁ}{\papi{ nɯsɯrtoʁ}}}\markboth{nɯsɯrtoʁ}{}
\classe{vt}
\begin{définition}\fra s'apercevoir\end{définition}
\begin{définition}\cmn 发觉,看得出\end{définition}
\begin{exemple}\jya tɯʑo mɤ-kɯ-pe nɯ a-pɯ́-wɣ-nɯsɯrtoʁ tɕe pe\cmn 发现自己的缺点是一件好事\end{exemple}
\begin{exemple}\jya ɲɯ-nɯsɯrtoʁ-a\cmn 我看得出\end{exemple}
\begin{exemple}\jya nɤki tɤ-scoz nɯ kɤ-rɤt ɲɤ-tɯ-nɯkɯmaʁ ri, mɯ́j-tɯ-nɯsɯrtoʁ\cmn 你写错了字,但是你没有看出来\end{exemple}
\begin{exemple}\jya nɤki tʂu kɯmaʁ jo-tɯ-nɯɕe ri mɯ́j-tɯ-nɯsɯrtoʁ\cmn 你走错了路,但是你没有看出来\end{exemple}
\begin{relation-sémantique}\synonyme{
\hyperlink{Ⓔsɯχsɤl}{\textit{ \papi{sɯχsɤl}}}
}\end{relation-sémantique}
\begin{relation-sémantique}\confer{
\hyperlink{Ⓔrtoʁ}{\textit{ \papi{rtoʁ}}}
}\end{relation-sémantique}\end{entrée}

\begin{entrée}
\vedette{\hypertarget{Ⓔnɯsɯzʁe}{\papi{ nɯsɯzʁe}}}\markboth{nɯsɯzʁe}{}\classe{vi}
\paradigme{\textit{dir :} \jya \_}
\begin{définition}\ 
\begin{déclaration}\grammar{incorp}\end{déclaration}\end{définition}
\begin{définition}\fra porter du bois (sur le dos)\end{définition}
\begin{définition}\cmn (来回)背柴\end{définition}
\begin{relation-sémantique}\confer{
\hyperlink{ⒺsiⒽ1}{\textit{ \papi{si}}}
}\end{relation-sémantique}
\begin{relation-sémantique}\confer{
\hyperlink{Ⓔnɯzʁe}{\textit{ \papi{nɯzʁe}}}
}\end{relation-sémantique}
\begin{relation-sémantique}\confer{
\hyperlink{Ⓔsɯzʁe}{\textit{ \papi{sɯzʁe}}}
}\end{relation-sémantique}\end{entrée}

\begin{entrée}
\vedette{\hypertarget{Ⓔnɯsuwa}{\papi{ nɯsuwa}}}\markboth{nɯsuwa}{}
\classe{vi}
\paradigme{\textit{dir :} \jya pɯ-}
\begin{définition}\fra monter la garde\end{définition}
\begin{définition}\cmn 站岗;放哨\end{définition}
\begin{exemple}\jya ɯʑo ku-nɯsuwa\cmn 他在放哨\end{exemple}
\begin{relation-sémantique}\synonyme{
\hyperlink{Ⓔnɯchɯra}{\textit{ \papi{nɯchɯra}}}
}\end{relation-sémantique}
\begin{relation-sémantique}\confer{
\hyperlink{Ⓔsuwa}{\textit{ \papi{suwa}}}
}\end{relation-sémantique}\end{entrée}

\begin{entrée}
\vedette{\hypertarget{Ⓔnɯt}{\papi{ nɯt}}}\markboth{nɯt}{}\classe{vi}
\paradigme{\textit{dir :} \jya tɤ-}
\begin{définition}\fra brûler\end{définition}
\begin{définition}\cmn 燃起\end{définition}
\begin{exemple}\jya smi to-nɯt\cmn 火燃起来了\end{exemple}
\begin{exemple}\jya si to-nɯt\cmn 柴火燃起来了\end{exemple}\end{entrée}

\begin{entrée}
\vedette{\hypertarget{Ⓔnɯta}{\papi{ nɯta}}}\markboth{nɯta}{}
\begin{relation-sémantique}\confer{
\hyperlink{Ⓔta}{\textit{ \papi{ta}}}
}\end{relation-sémantique}\end{entrée}

\begin{entrée}
\vedette{\hypertarget{Ⓔnɯtal}{\papi{ nɯtal}}}\markboth{nɯtal}{}\classe{vi}
\paradigme{\textit{dir :} \jya tɤ-}
\begin{définition}\fra jouer au jianzi\end{définition}
\begin{définition}\cmn 踢毽子\end{définition}
\begin{relation-sémantique}\confer{
\hyperlink{Ⓔtal}{\textit{ \papi{tal}}}
}\end{relation-sémantique}\end{entrée}

\begin{entrée}
\vedette{\hypertarget{Ⓔnɯtɤpɤtso}{\papi{ nɯtɤpɤtso}}}\markboth{nɯtɤpɤtso}{}\classe{vt}
\paradigme{\textit{dir :} \jya tɤ-}
\begin{définition}\fra considérer comme un enfant\end{définition}
\begin{définition}\cmn 把……当成小孩子\end{définition}
\begin{exemple}\jya tú-wɣ-nɯtɤpɤtso-a ŋgrɤl\cmn 他把我当成小孩子\end{exemple}
\begin{relation-sémantique}\confer{
\hyperlink{Ⓔtɤ-pɤtso}{\textit{ \papi{tɤ-pɤtso}}}
}\end{relation-sémantique}\end{entrée}

\begin{entrée}
\vedette{\hypertarget{Ⓔnɯtɤraʁ}{\papi{ nɯtɤraʁ}}}\markboth{nɯtɤraʁ}{}\paradigme{\textit{dir :} \jya tɤ-}
\begin{définition}\ 
\begin{déclaration}\grammar{denom}\end{déclaration}\end{définition}
\begin{définition}\fra parier\end{définition}
\begin{définition}\cmn 打赌\end{définition}
\begin{exemple}\jya tɤ-nɯtɤraʁ-ndʑi\cmn 他们俩打赌了\end{exemple}
\begin{relation-sémantique}\synonyme{
\hyperlink{Ⓔnɯsɤraʁ}{\textit{ \papi{nɯsɤraʁ}}}
}\end{relation-sémantique}\classe{vt}\end{entrée}

\begin{entrée}
\vedette{\hypertarget{Ⓔnɯtɕu}{\papi{ nɯtɕu}}}\markboth{nɯtɕu}{}\classe{adv}\acception{1}
\begin{définition}\fra là-bas\end{définition}
\begin{définition}\cmn 在那里\end{définition}\acception{2}
\begin{définition}\fra à ce moment-là\end{définition}
\begin{définition}\cmn 在那个时候\end{définition}
\end{entrée}

\begin{entrée}
\vedette{\hypertarget{Ⓔnɯtɕarloŋ}{\papi{ nɯtɕarloŋ}}}\markboth{nɯtɕarloŋ}{}\classe{vi}
\paradigme{\textit{dir :} \jya pɯ-}
\begin{définition}\fra avoir une sensation désagréable (après avoir bu un thé trop fort)\end{définition}
\begin{définition}\cmn 醉茶\end{définition}\begin{sous-entrée}
\vedette{\hypertarget{}{\papi{ znɯtɕarloŋ}}}\markboth{znɯtɕarloŋ}{}\classe{vi}
\paradigme{\textit{dir :} \jya pɯ-}
\begin{définition}\fra causer une réaction désagréable (thé)\end{définition}
\begin{définition}\cmn 使……醉茶\end{définition}
\begin{exemple}\jya tʂha kɯ-sna kɯ-tɕhom kɤ-tshi-t-a pɯ́-wɣ-znɯtɕarloŋ-a\cmn 我喝了过浓的茶就醉茶了\end{exemple}
\end{sous-entrée}\end{entrée}

\begin{entrée}
\vedette{\hypertarget{Ⓔnɯtɕaχpa}{\papi{ nɯtɕaχpa}}}\markboth{nɯtɕaχpa}{}
\classe{vt}
\paradigme{\textit{dir :} \jya pɯ-}
\begin{définition}\ 
\begin{déclaration}\grammar{denom}\end{déclaration}\end{définition}
\begin{définition}\fra extorquer\end{définition}
\begin{définition}\cmn 抢
\begin{déclaration} \étymologie{\papi{dʑag.pa}}\end{déclaration}\end{définition}
\begin{exemple}\jya tʂu tɕe pjɤ́-wɣ-nɯtɕaχpa\cmn 他在路上被抢了\end{exemple}
\begin{relation-sémantique}\synonyme{
\hyperlink{Ⓔnɯsɯkho}{\textit{ \papi{nɯsɯkho}}}
}\end{relation-sémantique}
\begin{relation-sémantique}\confer{
\hyperlink{Ⓔtɕaχpa}{\textit{ \papi{tɕaχpa}}}
}\end{relation-sémantique}\end{entrée}

\begin{entrée}
\vedette{\hypertarget{Ⓔnɯtɕɤmɯ}{\papi{ nɯtɕɤmɯ}}}\markboth{nɯtɕɤmɯ}{}\classe{vi}
\paradigme{\textit{dir :} \jya lɤ-}
\begin{définition}\ 
\begin{déclaration}\grammar{denom}\end{déclaration}\end{définition}
\begin{définition}\fra devenir none\end{définition}
\begin{définition}\cmn 当尼姑\end{définition}
\begin{relation-sémantique}\confer{
\hyperlink{Ⓔtɕɤmɯ}{\textit{ \papi{tɕɤmɯ}}}
}\end{relation-sémantique}
\begin{relation-sémantique}\confer{
\hyperlink{Ⓔrɯtɕɤmɯ}{\textit{ \papi{rɯtɕɤmɯ}}}
}\end{relation-sémantique}\end{entrée}

\begin{entrée}
\vedette{\hypertarget{Ⓔnɯtɕetha}{\papi{ nɯtɕetha}}}\markboth{nɯtɕetha}{}\classe{vt}
\paradigme{\textit{dir :} \jya kɤ-}
\begin{définition}\fra sonder\end{définition}
\begin{définition}\cmn 试探\end{définition}
\begin{exemple}\jya ɯʑo kɯ kɤ́-wɣ-nɯtɕetha\cmn 他试探我了\end{exemple}\begin{sous-entrée}
\vedette{\hypertarget{}{\papi{ sɤnɯtɕetha}}}\markboth{sɤnɯtɕetha}{}\classe{vi}
\begin{définition}\ 
\begin{déclaration}\grammar{apass}\end{déclaration}\end{définition}
\begin{définition}\fra sonder les gens\end{définition}
\begin{définition}\cmn 试探人\end{définition}
\begin{exemple}\jya kɤ-sɤnɯtɕetha mɯ́j-tɯ-spe wo!\cmn 你不会试探别人\end{exemple}
\begin{relation-sémantique}\confer{
\hyperlink{Ⓔtɕetha}{\textit{ \papi{tɕetha}}}
}\end{relation-sémantique}
\end{sous-entrée}\end{entrée}

\begin{entrée}
\vedette{\hypertarget{Ⓔnɯtɕɣom}{\papi{ nɯtɕɣom}}}\markboth{nɯtɕɣom}{}\classe{vi}
\paradigme{\textit{dir :} \jya pɯ-}
\begin{définition}\ 
\begin{déclaration}\grammar{denom}\end{déclaration}\end{définition}
\begin{définition}\fra ramasser du xanthoxyle\end{définition}
\begin{définition}\cmn 摘花椒\end{définition}
\begin{exemple}\jya ɕ-pɯ-tɕɣom\cmn 我去摘花椒了\end{exemple}
\begin{relation-sémantique}\confer{
\hyperlink{Ⓔtɕɣom}{\textit{ \papi{tɕɣom}}}
}\end{relation-sémantique}\end{entrée}

\begin{entrée}
\vedette{\hypertarget{Ⓔnɯtɕhaʁ}{\papi{ nɯtɕhaʁ}}}\markboth{nɯtɕhaʁ}{}
\classe{vi}
\paradigme{\textit{dir :} \jya tɤ-}
\begin{définition}\fra manger du fourrage (cheval)\end{définition}
\begin{définition}\cmn 吃饲料(马)\end{définition}
\begin{exemple}\jya mbro ɲɯ-nɯtɕhaʁ\cmn 马在吃饲料\end{exemple}
\begin{relation-sémantique}\confer{
\hyperlink{Ⓔɯ-tɕhaʁⒽ1}{\textit{ \papi{ɯ-tɕhaʁ1}}}
}\end{relation-sémantique}\end{entrée}

\begin{entrée}
\vedette{\hypertarget{Ⓔnɯtɕhɤjlɯz}{\papi{ nɯtɕhɤjlɯz}}}\markboth{nɯtɕhɤjlɯz}{}\classe{vi}
\begin{définition}\fra observer une coutume\end{définition}
\begin{définition}\cmn 遵循某种风俗\end{définition}
\begin{exemple}\jya kɯki kɯ-fse aʑo kɤ-nɯtɕhɤjlɯz mɤ-cha-a\cmn 我不能接受这种风俗\end{exemple}
\begin{relation-sémantique}\confer{
\hyperlink{Ⓔtɕhɤjlɯz}{\textit{ \papi{tɕhɤjlɯz}}}
}\end{relation-sémantique}\end{entrée}

\begin{entrée}
\vedette{\hypertarget{Ⓔnɯtɕhɤjʁo}{\papi{ nɯtɕhɤjʁo}}}\markboth{nɯtɕhɤjʁo}{}\classe{vs}
\begin{définition}\fra être en état d'ébriété\end{définition}
\begin{définition}\cmn 发酒疯\end{définition}
\begin{exemple}\jya jɯfɕɯr lɤ-βzi tɕe tɤ-nɯtɕhɤjʁo\cmn 他昨天喝醉了就发了酒疯\end{exemple}
\begin{relation-sémantique}\confer{
\hyperlink{Ⓔtɕhɤjʁo}{\textit{ \papi{tɕhɤjʁo}}}
}\end{relation-sémantique}\end{entrée}

\begin{entrée}
\vedette{\hypertarget{Ⓔnɯtɕhɤl}{\papi{ nɯtɕhɤl}}}\markboth{nɯtɕhɤl}{}\classe{vi}
\paradigme{\textit{dir :} \jya pɯ-}
\begin{définition}\fra être puni\end{définition}
\begin{définition}\cmn 受到惩罚
\begin{déclaration} \étymologie{\papi{tɕʰad}}\end{déclaration}\end{définition}
\begin{sous-entrée}
\vedette{\hypertarget{}{\papi{ znɯtɕhɤl}}}\markboth{znɯtɕhɤl}{}\classe{vt}
\paradigme{\textit{dir :} \jya pɯ-}
\begin{définition}\fra punir\end{définition}
\begin{définition}\cmn 惩罚\end{définition}
\begin{exemple}\jya pɯ́-wɣ-znɯtɕhal-a tɕe nɯ-kho-t-a\cmn 我被罚就交了罚款\end{exemple}
\begin{exemple}\jya ɯʑo ɕ-to-mɯrkɯ tɕe, pɯ-znɯtɕhal-a\cmn 他偷了东西我就惩罚了他\end{exemple}
\begin{relation-sémantique}\synonyme{
\hyperlink{Ⓔznɯtɕhɤtpa}{\textit{ \papi{znɯtɕhɤtpa}}}
}\end{relation-sémantique}
\begin{relation-sémantique}\confer{
\hyperlink{Ⓔɯ-tɕhɤl}{\textit{ \papi{ɯ-tɕhɤl}}}
}\end{relation-sémantique}
\end{sous-entrée}\end{entrée}

\begin{entrée}
\vedette{\hypertarget{Ⓔnɯtɕhomba}{\papi{ nɯtɕhomba}}}\markboth{nɯtɕhomba}{}
\paradigme{\textit{dir :} \jya tɤ-}
\begin{définition}\fra attraper un rhume\end{définition}
\begin{définition}\cmn 感冒\end{définition}
\begin{exemple}\jya adianhua jɤ-tɯ-lɤt ri, pɯ-nɯtɕhomba-a\cmn 你给我打电话的时候,我在感冒\end{exemple}
\begin{relation-sémantique}\confer{
\hyperlink{Ⓔtɕhomba}{\textit{ \papi{tɕhomba}}}
}\end{relation-sémantique}\classe{vi}\end{entrée}

\begin{entrée}
\vedette{\hypertarget{Ⓔnɯtɕhʁɯβ}{\papi{ nɯtɕhʁɯβ}}}\markboth{nɯtɕhʁɯβ}{}
\begin{relation-sémantique}\confer{
\hyperlink{Ⓔtɕhʁɯβnɤtɕhʁɯβ}{\textit{ \papi{tɕhʁɯβnɤtɕhʁɯβ}}}
}\end{relation-sémantique}\end{entrée}

\begin{entrée}
\vedette{\hypertarget{Ⓔnɯtɕhɯrɟɯɣ}{\papi{ nɯtɕhɯrɟɯɣ}}}\markboth{nɯtɕhɯrɟɯɣ}{}\classe{vi}
\begin{définition}\fra où l'eau coule vite\end{définition}
\begin{définition}\cmn 水流得很急(地方)\end{définition}
\begin{exemple}\jya ɯ-sta a-pɯ-ɣɤʑɯn tɕe nɯtɕhɯrɟɯɣ\cmn 如果有斜坡的话水流得很急\end{exemple}\end{entrée}

\begin{entrée}
\vedette{\hypertarget{Ⓔnɯtɕhɯtɕɯnpaχɕi}{\papi{ nɯtɕhɯtɕɯnpaχɕi}}}\markboth{nɯtɕhɯtɕɯnpaχɕi}{}\classe{vi}
\paradigme{\textit{dir :} \jya nɯ-}
\begin{définition}\fra aller cueillir des poires\end{définition}
\begin{définition}\cmn 采梨子\end{définition}\end{entrée}

\begin{entrée}
\vedette{\hypertarget{Ⓔnɯtɕhɯtɯɣ}{\papi{ nɯtɕhɯtɯɣ}}}\markboth{nɯtɕhɯtɯɣ}{}
\classe{vi}
\paradigme{\textit{dir :} \jya pɯ-}
\begin{définition}\ 
\begin{déclaration}\grammar{denom}\end{déclaration}\end{définition}
\begin{définition}\fra s'empoisonner en buvant de l'eau (bovidé)\end{définition}
\begin{définition}\cmn 水中毒(牛)
\begin{déclaration} \étymologie{\papi{tɕʰu.dug}}\end{déclaration}\end{définition}
\begin{exemple}\jya jla pjɤ-nɯtɕhɯtɯɣ\cmn 犏牛喝水中毒了\end{exemple}
\begin{exemple}\jya qambrɯ pjɤ-nɯtɕhɯtɯɣ\cmn 牦牛喝水中毒了\end{exemple}
\begin{relation-sémantique}\confer{
\hyperlink{Ⓔnɯrtsɤtɯɣ}{\textit{ \papi{nɯrtsɤtɯɣ}}}
}\end{relation-sémantique}\end{entrée}

\begin{entrée}
\vedette{\hypertarget{Ⓔnɯtɕhɯwɯt}{\papi{ nɯtɕhɯwɯt}}}\markboth{nɯtɕhɯwɯt}{}
\classe{vt}
\paradigme{\textit{dir :} \jya nɯ-}
\begin{définition}\ 
\begin{déclaration}\grammar{denom}\end{déclaration}\end{définition}
\begin{définition}\fra faire bouillir la peau afin de pouvoir enlever les poils\end{définition}
\begin{définition}\cmn 烫;用开水将毛褪掉\end{définition}
\begin{exemple}\jya paʁ ɲɤ-nɯtɕhɯwɯt\cmn 他烫了猪皮\end{exemple}\end{entrée}

\begin{entrée}
\vedette{\hypertarget{Ⓔnɯtɕʁɯβ}{\papi{ nɯtɕʁɯβ}}}\markboth{nɯtɕʁɯβ}{}
\begin{relation-sémantique}\confer{
\hyperlink{Ⓔtɕʁɯβnɤtɕʁɯβ}{\textit{ \papi{tɕʁɯβnɤtɕʁɯβ}}}
}\end{relation-sémantique}\end{entrée}

\begin{entrée}
\vedette{\hypertarget{Ⓔnɯthaj}{\papi{ nɯthaj}}}\markboth{nɯthaj}{}
\classe{vt}
\paradigme{\textit{dir :} \jya tɤ-}
\begin{définition}\fra soulever (à plusieurs)\end{définition}
\begin{définition}\cmn 抬(几个人一起)
\begin{déclaration} \étymologie{\papi{\stylefn{抬}}}\end{déclaration}\end{définition}
\begin{exemple}\jya ɕoŋtɕa tɤ-nɯthaj-tɕi\cmn 我们俩把木料抬起来了\end{exemple}\end{entrée}

\begin{entrée}
\vedette{\hypertarget{Ⓔnɯthɤstɯɣ}{\papi{ nɯthɤstɯɣ}}}\markboth{nɯthɤstɯɣ}{}
\classe{vi}
\paradigme{\textit{dir :} \jya tɤ-}
\begin{définition}\ 
\begin{déclaration}\grammar{denom}\end{déclaration}\end{définition}
\begin{définition}\fra jouer à un jeu de hasard (où il faut deviner combien d'objets son adversaire tient dans la main)\end{définition}
\begin{définition}\cmn 赌胡豆的游戏
\begin{déclaration}\use{其中一个玩家抓一把胡豆,要对方猜拿了多少,猜对了,第一玩家就把胡豆给第二玩家,猜不对的话,第二玩家就要赔}\end{déclaration}\end{définition}
\begin{exemple}\jya stoʁ tɤ-nɯthɤstɯɣ-tɕi\cmn 我们俩赌了胡豆\end{exemple}
\begin{exemple}\jya khɯtsa ɯ-ŋgɯ stoʁ tɤ-rku-tɕi tɕe tɤ-nɯthɤstɯɣ-tɕi, a-pɯ-tɕhaʁ tɕe nɤʑo ɲɯ-kɯ-ɣɤjɯ-a, ɯ-tsa ɲɯ-βze nɤ nɤʑo jɤ-nɯtsɯm\cmn 我们俩在碗里装了胡豆就赌了有多少,要是猜少了的话就给我赔,要是猜对了的话你就把它带走\end{exemple}
\begin{relation-sémantique}\confer{
\hyperlink{Ⓔthɤstɯɣ}{\textit{ \papi{thɤstɯɣ}}}
}\end{relation-sémantique}\end{entrée}

\begin{entrée}
\vedette{\hypertarget{Ⓔnɯthɣe}{\papi{ nɯthɣe}}}\markboth{nɯthɣe}{}\classe{vi}
\paradigme{\textit{dir :} \jya \_}
\begin{définition}\ 
\begin{déclaration}\grammar{denom}\end{déclaration}\end{définition}
\begin{définition}\fra ramasser des glands\end{définition}
\begin{définition}\cmn 捡青冈籽\end{définition}
\begin{relation-sémantique}\confer{
\hyperlink{Ⓔthɣe}{\textit{ \papi{thɣe}}}
}\end{relation-sémantique}\end{entrée}

\begin{entrée}
\vedette{\hypertarget{Ⓔnɯthɯ}{\papi{ nɯthɯ}}}\markboth{nɯthɯ}{}
\classe{vt}
\paradigme{\textit{dir :} \jya kɤ-}
\begin{définition}\fra utiliser comme une casserole\end{définition}
\begin{définition}\cmn 用锅子;当锅子用\end{définition}
\begin{exemple}\jya nɤʑo tʂha kɤ-tɯ-ta-t tɕe ko-tɯ-nɯthɯ-t\cmn 你煮茶的时候用了锅子\end{exemple}
\begin{relation-sémantique}\confer{
 \papi{tɯthɯ}
}\end{relation-sémantique}\end{entrée}

\begin{entrée}
\vedette{\hypertarget{Ⓔnɯtsa}{\papi{ nɯtsa}}}\markboth{nɯtsa}{}\classe{vs}
\paradigme{\textit{dir :} \jya tɤ-}
\begin{définition}\fra convenir\end{définition}
\begin{définition}\cmn 适合\end{définition}
\begin{exemple}\jya ki tɤ-rte ki nɤʑɯɣ ɲɯ-nɯtsa\cmn 你适合戴这种帽子\end{exemple}
\begin{exemple}\jya ki tɤ-rte ki ɲɯ-tɯ-nɯtsa\cmn 你适合戴这种帽子\end{exemple}
\begin{exemple}\jya tu-tɯ-nɯtsa ʑo tɕe phɣo-a ɕti\cmn 一到合适的时间,我就会逃跑\end{exemple}
\begin{relation-sémantique}\confer{
\hyperlink{Ⓔɯ-tsa}{\textit{ \papi{ɯ-tsa}}}
}\end{relation-sémantique}
\begin{relation-sémantique}\confer{
\hyperlink{Ⓔnɤtsa}{\textit{ \papi{nɤtsa}}}
}\end{relation-sémantique}\end{entrée}

\begin{entrée}
\vedette{\hypertarget{Ⓔnɯtshɤβ}{\papi{ nɯtshɤβ}}}\markboth{nɯtshɤβ}{}\classe{vt}\acception{1}
\paradigme{\textit{dir :} \jya kɤ-}
\begin{définition}\fra confronter ensemble\end{définition}
\begin{définition}\cmn 一起对付一个人\end{définition}
\begin{exemple}\jya ʑara kɯ ɯʑo kɤ-ʁndɯ ko-nɯtshɤβ-nɯ\cmn 他们一起打了他\end{exemple}
\begin{exemple}\jya ʑara kɯ ɯʑo ko-nɯtshɤβ-nɯ ʑo to-nɤmqe-nɯ\cmn 他们一起骂了他\end{exemple}\acception{2}
\paradigme{\textit{dir :} \jya nɯ-}
\begin{définition}\fra faire ensemble\end{définition}
\begin{définition}\cmn 共同做一件事情\end{définition}
\begin{exemple}\jya tɕiʑo kɯki laχtɕha kɤ-χtɯ tɤ-nɯtshɤβ-tɕi ŋu\cmn 我们一起买了这个东西\end{exemple}
\begin{exemple}\jya ki tɯ-khɯtsa kɤndza tɤ-nɯtshɤβ-tɕi\cmn 我们一起吃了这一碗\end{exemple}\end{entrée}

\begin{entrée}
\vedette{\hypertarget{Ⓔnɯtshɤdɯɣ}{\papi{ nɯtshɤdɯɣ}}}\markboth{nɯtshɤdɯɣ}{}\classe{vi}
\paradigme{\textit{dir :} \jya nɯ-}
\begin{définition}\fra souffrir de la chaleur\end{définition}
\begin{définition}\cmn 受热,中暑\end{définition}
\begin{exemple}\jya jisŋi tɤŋe ɲɯ-sɤɕke tɕe ɲɯ-nɯtshɤdɯɣ-a\cmn 今天太阳很大,我觉得很难受\end{exemple}
\begin{relation-sémantique}\confer{
\hyperlink{Ⓔtshɤdɯɣ}{\textit{ \papi{tshɤdɯɣ}}}
}\end{relation-sémantique}
\begin{relation-sémantique}\confer{
\hyperlink{Ⓔɣɯtshɤdɯɣ}{\textit{ \papi{ɣɯtshɤdɯɣ}}}
}\end{relation-sémantique}\end{entrée}

\begin{entrée}
\vedette{\hypertarget{Ⓔnɯtshɤtʂot}{\papi{ nɯtshɤtʂot}}}\markboth{nɯtshɤtʂot}{}
\classe{vi}
\paradigme{\textit{dir :} \jya nɯ-}
\paradigme{\textit{dir :} \jya tɤ-}
\begin{définition}\ 
\begin{déclaration}\grammar{denom}\end{déclaration}\end{définition}
\begin{définition}\fra avoir la fièvre\end{définition}
\begin{définition}\cmn 发烧
\begin{déclaration} \étymologie{\papi{tsʰa.drod}}\end{déclaration}\end{définition}
\begin{exemple}\jya ɲɯ-nɯtɕhomba tɕe ɲɯ-nɯtshɤtʂot\cmn 他感冒就发烧\end{exemple}
\begin{exemple}\jya nɯ-nɯtshɤtʂot-a\cmn 我发烧了\end{exemple}
\begin{relation-sémantique}\confer{
\hyperlink{Ⓔtshɤtʂot}{\textit{ \papi{tshɤtʂot}}}
}\end{relation-sémantique}\end{entrée}

\begin{entrée}
\vedette{\hypertarget{Ⓔnɯtsɯm}{\papi{ nɯtsɯm}}}\markboth{nɯtsɯm}{}
\begin{relation-sémantique}\confer{
\hyperlink{Ⓔtsɯm}{\textit{ \papi{tsɯm}}}
}\end{relation-sémantique}
\end{entrée}

\begin{entrée}
\vedette{\hypertarget{Ⓔnɯtsɯʁot}{\papi{ nɯtsɯʁot}}}\markboth{nɯtsɯʁot}{}\classe{vi}
\paradigme{\textit{dir :} \jya pɯ-}
\begin{définition}\fra chasser le faisan\end{définition}
\begin{définition}\cmn 打野鸡\end{définition}
\begin{exemple}\jya ɯʑo ɕ-pɯ-nɯtsɯʁot ri, mɯ-pjɤ-cha\cmn 他去打野鸡,但是没有成功\end{exemple}
\begin{relation-sémantique}\confer{
\hyperlink{Ⓔtsɯʁot}{\textit{ \papi{tsɯʁot}}}
}\end{relation-sémantique}\end{entrée}

\begin{entrée}
\vedette{\hypertarget{ⒺnɯtʂuⒽ1}{\papi{ nɯtʂu}}}\markboth{nɯtʂu}{}\homonyme{1}\classe{vi}
\paradigme{\textit{dir :} \jya tɤ-}
\begin{définition}\fra bien se passer\end{définition}
\begin{définition}\cmn 顺利\end{définition}
\begin{exemple}\jya jɤxtshi tɕi-tɯtsɣe pɯ-nɯtʂu\cmn 最近我们俩的生意很好\end{exemple}
\begin{relation-sémantique}\confer{
\hyperlink{Ⓔtʂu}{\textit{ \papi{tʂu}}}
}\end{relation-sémantique}\end{entrée}

\begin{entrée}
\vedette{\hypertarget{ⒺnɯtʂuⒽ2}{\papi{ nɯtʂu}}}\markboth{nɯtʂu}{}\homonyme{2}\classe{vt}
\paradigme{\textit{dir :} \jya kɤ-}
\begin{définition}\fra prendre en passant\end{définition}
\begin{définition}\cmn (一路上看见什么东西就)顺便拿走\end{définition}
\begin{exemple}\jya sɯmat ɣɤʑu tɕe kɤ-nɯtʂu-t-a\cmn (路上)有水果,我顺便拿了\end{exemple}
\begin{exemple}\jya paʁndza tʂɯtʂu kɤ-nɯtʂu-t-a tɕe nɯ-phɯt-a\cmn 我在路上看到猪草,顺便割了一把\end{exemple}
\begin{relation-sémantique}\confer{
\hyperlink{Ⓔtʂu}{\textit{ \papi{tʂu}}}
}\end{relation-sémantique}\end{entrée}

\begin{entrée}
\vedette{\hypertarget{Ⓔnɯtʂawku}{\papi{ nɯtʂawku}}}\markboth{nɯtʂawku}{}
\classe{vt}
\paradigme{\textit{dir :} \jya tɤ-}
\begin{définition}\fra prendre soin\end{définition}
\begin{définition}\cmn 照顾
\begin{déclaration} \étymologie{\papi{\stylefn{照顾}}}\end{déclaration}\end{définition}\begin{sous-entrée}
\vedette{\hypertarget{}{\papi{ ʑɣɤnɯtʂawku}}}\markboth{ʑɣɤnɯtʂawku}{}
\begin{définition}\ 
\begin{déclaration}\grammar{refl}\end{déclaration}\end{définition}
\begin{définition}\fra prendre soin de soi\end{définition}
\begin{définition}\cmn 照顾自己\end{définition}
\begin{exemple}\jya nɤʑo tu-tɯ-ʑɣɤnɯtʂawku mɯ́j-tɯ-cha ɕti\cmn 你都不会照顾自己\end{exemple}
\end{sous-entrée}\end{entrée}

\begin{entrée}
\vedette{\hypertarget{Ⓔnɯtʂɤqɤsti}{\papi{ nɯtʂɤqɤsti}}}\markboth{nɯtʂɤqɤsti}{}
\paradigme{\textit{dir :} \jya tɤ-}
\begin{définition}\ 
\begin{déclaration}\grammar{incorp}\end{déclaration}\end{définition}
\begin{définition}\fra bloquer le chemin\end{définition}
\begin{définition}\cmn 挡路(走在前面的人,挡后面行人的去路)\end{définition}
\begin{exemple}\jya ma-tɤ-kɯ-nɯtʂɤqɤsti-a\cmn 你不要挡我的路\end{exemple}
\begin{relation-sémantique}\confer{
\hyperlink{ⒺstiⒽ1}{\textit{ \papi{sti1}}}
}\end{relation-sémantique}
\begin{relation-sémantique}\confer{
\hyperlink{Ⓔtʂu}{\textit{ \papi{tʂu}}}
}\end{relation-sémantique}\classe{vt}\end{entrée}

\begin{entrée}
\vedette{\hypertarget{Ⓔnɯtʂha}{\papi{ nɯtʂha}}}\markboth{nɯtʂha}{}\paradigme{\textit{dir :} \jya kɤ-}
\begin{définition}\ 
\begin{déclaration}\grammar{denom}\end{déclaration}\end{définition}
\begin{définition}\fra prendre le petit déjeuner\end{définition}
\begin{définition}\cmn 吃早饭\end{définition}
\begin{exemple}\jya kɤ-nɯtʂha-ndʑi\cmn 他们俩吃了早餐\end{exemple}
\begin{relation-sémantique}\confer{
\hyperlink{Ⓔtʂha}{\textit{ \papi{tʂha}}}
}\end{relation-sémantique}\classe{vi}\end{entrée}

\begin{entrée}
\vedette{\hypertarget{Ⓔnɯtʂhɤɣndʑɤr}{\papi{ nɯtʂhɤɣndʑɤr}}}\markboth{nɯtʂhɤɣndʑɤr}{}\paradigme{\textit{dir :} \jya tɤ-}
\begin{définition}\ 
\begin{déclaration}\grammar{denom}\end{déclaration}\end{définition}
\begin{définition}\fra manger (serviteur)\end{définition}
\begin{définition}\cmn 吃糌粑
\begin{déclaration}\use{吃糌粑、喝点茶,没有其它食物(旧社会仆人们吃的早餐)}\end{déclaration}\end{définition}
\begin{exemple}\jya soz to-nɯtʂhɤɣndʑɤr-ndʑi\cmn 他们俩早上吃了\end{exemple}
\begin{relation-sémantique}\confer{
\hyperlink{Ⓔtɯ-ɣndʑɤr}{\textit{ \papi{tɯ-ɣndʑɤr}}}
}\end{relation-sémantique}\classe{vi}\end{entrée}

\begin{entrée}
\vedette{\hypertarget{Ⓔnɯtʂhɤlu}{\papi{ nɯtʂhɤlu}}}\markboth{nɯtʂhɤlu}{}
\classe{vt}
\begin{définition}\ 
\begin{déclaration}\grammar{denom}\end{déclaration}\end{définition}
\begin{définition}\fra verser du lait dans le thé\end{définition}
\begin{définition}\cmn 把牛奶倒到马茶里\end{définition}
\begin{exemple}\jya alo ji-tɤ-lu tha-ɣɯt-nɯ tɤ-nɯtʂhɤlu-j / (tʂhɤlu tɤ-nɯlɤt-i)\cmn 他们把牛奶带下来了,我们就倒在茶里喝了\end{exemple}\end{entrée}

\begin{entrée}
\vedette{\hypertarget{Ⓔnɯtɯcizʁe}{\papi{ nɯtɯcizʁe}}}\markboth{nɯtɯcizʁe}{}\classe{vi}
\begin{définition}\ 
\begin{déclaration}\grammar{incorp}\end{déclaration}\end{définition}
\begin{définition}\fra transporter de l'eau\end{définition}
\begin{définition}\cmn 背水\end{définition}
\begin{exemple}\jya ɕ-pɯ-nɯtɯcizʁe-a\cmn 我去背水了\end{exemple}
\begin{relation-sémantique}\confer{
\hyperlink{Ⓔtɯ-ci}{\textit{ \papi{tɯ-ci}}}
}\end{relation-sémantique}
\begin{relation-sémantique}\confer{
\hyperlink{Ⓔtɯcizʁe}{\textit{ \papi{tɯcizʁe}}}
}\end{relation-sémantique}
\begin{relation-sémantique}\confer{
\hyperlink{Ⓔnɯzʁe}{\textit{ \papi{nɯzʁe}}}
}\end{relation-sémantique}\end{entrée}

\begin{entrée}
\vedette{\hypertarget{Ⓔnɯtɯfɕɤl}{\papi{ nɯtɯfɕɤl}}}\markboth{nɯtɯfɕɤl}{}
\classe{vi}
\paradigme{\textit{dir :} \jya nɯ-}
\begin{définition}\ 
\begin{déclaration}\grammar{denom}\end{déclaration}\end{définition}
\begin{définition}\fra avoir la diarrhée\end{définition}
\begin{définition}\cmn 拉肚子\end{définition}
\begin{exemple}\jya nɯ-nɯtɯfɕal-a\cmn 我拉了肚子\end{exemple}
\begin{exemple}\jya a-xtu ɲɯ-mŋɤm tɕe nɯ-nɯtɯfɕal-a\cmn 我肚子疼,拉了肚子\end{exemple}\begin{sous-entrée}
\vedette{\hypertarget{}{\papi{ znɯtɯfɕɤl}}}\markboth{znɯtɯfɕɤl}{}\classe{vt}
\paradigme{\textit{dir :} \jya nɯ-}
\begin{définition}\fra causer la diarrhée\end{définition}
\begin{définition}\cmn 令人拉肚子\end{définition}
\begin{relation-sémantique}\confer{
\hyperlink{Ⓔfɕɤl}{\textit{ \papi{fɕɤl}}}
}\end{relation-sémantique}
\begin{relation-sémantique}\synonyme{
\hyperlink{Ⓔnɯqhoχɕɤr}{\textit{ \papi{nɯqhoχɕɤr}}}
}\end{relation-sémantique}
\end{sous-entrée}\end{entrée}

\begin{entrée}
\vedette{\hypertarget{Ⓔnɯtɯrgi}{\papi{ nɯtɯrgi}}}\markboth{nɯtɯrgi}{}\classe{vi}
\paradigme{\textit{dir :} \jya pɯ-}
\begin{définition}\fra couper et ramasser des branches de sapin pour faire des fumigations\end{définition}
\begin{définition}\cmn 把杉树枝桠砍回来\end{définition}\end{entrée}

\begin{entrée}
\vedette{\hypertarget{Ⓔnɯtɯrgilaŋlaŋ}{\papi{ nɯtɯrgilaŋlaŋ}}}\markboth{nɯtɯrgilaŋlaŋ}{}\classe{vi}
\paradigme{\textit{dir :} \jya \_}
\begin{définition}\ 
\begin{déclaration}\grammar{denom}\end{déclaration}\end{définition}
\begin{définition}\fra ramasser des cônes de pin\end{définition}
\begin{définition}\cmn 捡杉木果\end{définition}
\begin{exemple}\jya ɕ-pɯ-nɯtɯrgilaŋlaŋ-a\cmn 我去捡杉木果了\end{exemple}
\begin{relation-sémantique}\confer{
\hyperlink{Ⓔtɯrgilaŋlaŋ}{\textit{ \papi{tɯrgilaŋlaŋ}}}
}\end{relation-sémantique}\end{entrée}

\begin{entrée}
\vedette{\hypertarget{Ⓔnɯtɯtɕhɯ}{\papi{ nɯtɯtɕhɯ}}}\markboth{nɯtɯtɕhɯ}{}\classe{vt}
\paradigme{\textit{dir :} \jya tɤ-}
\begin{définition}\fra poignarder\end{définition}
\begin{définition}\cmn 刺杀\end{définition}
\begin{exemple}\jya tshjencɯ kɯ to-znɯtɯtɕhɯ tɕe pjɤ-sat\cmn 用短刀把他刺杀了\end{exemple}\begin{sous-entrée}
\vedette{\hypertarget{}{\papi{ sɤnɯtɯtɕhɯ}}}\markboth{sɤnɯtɯtɕhɯ}{}\classe{vi}
\begin{définition}\ 
\begin{déclaration}\grammar{apass}\end{déclaration}\end{définition}
\begin{définition}\fra assassiner les gens\end{définition}
\begin{définition}\cmn 刺杀人\end{définition}
\end{sous-entrée}\end{entrée}

\begin{entrée}
\vedette{\hypertarget{Ⓔnɯtɯtɕhɯ}{\papi{ nɯtɯtɕhɯ}}}\markboth{nɯtɯtɕhɯ}{}
\paradigme{\textit{dir :} \jya tɤ-}
\begin{définition}\ 
\begin{déclaration}\grammar{denom}\end{déclaration}\end{définition}
\begin{définition}\fra assassiner\end{définition}
\begin{définition}\cmn 刺杀\end{définition}
\begin{relation-sémantique}\confer{
\hyperlink{Ⓔtɯtɕhɯ}{\textit{ \papi{tɯtɕhɯ}}}
}\end{relation-sémantique}\classe{vt}\begin{sous-entrée}
\vedette{\hypertarget{}{\papi{ sɤnɯtɯtɕhɯ}}}\markboth{sɤnɯtɯtɕhɯ}{}\classe{vi}
\begin{définition}\ 
\begin{déclaration}\grammar{apass}\end{déclaration}\end{définition}
\begin{définition}\fra assassiner les gens\end{définition}
\begin{définition}\cmn 刺杀人\end{définition}
\end{sous-entrée}\end{entrée}

\begin{entrée}
\vedette{\hypertarget{Ⓔnɯtɯtso}{\papi{ nɯtɯtso}}}\markboth{nɯtɯtso}{}\classe{vt}
\paradigme{\textit{dir :} \jya pɯ-}
\begin{définition}\fra avoir de l'expérience\end{définition}
\begin{définition}\cmn 懂事,有经验,得到教训\end{définition}
\begin{exemple}\jya pjɯ-kɯ-nɯtɯtso ra ma nɯ ɯ-qhu tɕe kɯmaʁ kɤ-nɤma tɤ-ra tɕe kɤ-sɤpe kɯ-cha (=tɯ-kɯ-tso pjɯ-tu ra)\cmn 要经过经验和教训才能把以后的事情做好\end{exemple}
\begin{relation-sémantique}\confer{
\hyperlink{Ⓔtso}{\textit{ \papi{tso}}}
}\end{relation-sémantique}\end{entrée}

\begin{entrée}
\vedette{\hypertarget{Ⓔnɯtɯtʂaŋ}{\papi{ nɯtɯtʂaŋ}}}\markboth{nɯtɯtʂaŋ}{}
\begin{relation-sémantique}\confer{
\hyperlink{Ⓔtʂaŋ}{\textit{ \papi{tʂaŋ}}}
}\end{relation-sémantique}\end{entrée}

\begin{entrée}
\vedette{\hypertarget{Ⓔnɯxso}{\papi{ nɯxso}}}\markboth{nɯxso}{}\classe{vs}
\paradigme{\textit{dir :} \jya tɤ-}
\begin{définition}\fra être vide\end{définition}
\begin{définition}\cmn 空的\end{définition}
\begin{exemple}\jya khɯtsa to-nɯxso tɕe ɯ-ŋgɯ kɤ-rku ɲɤ-me\cmn 碗空了\end{exemple}
\begin{relation-sémantique}\confer{
\hyperlink{Ⓔso}{\textit{ \papi{so}}}
}\end{relation-sémantique}\end{entrée}

\begin{entrée}
\vedette{\hypertarget{Ⓔnɯxsɯ}{\papi{ nɯxsɯ}}}\markboth{nɯxsɯ}{}\classe{vi}\acception{1}
\paradigme{\textit{dir :} \jya lɤ-}
\paradigme{\textit{dir :} \jya thɯ-}
\begin{définition}\fra regarder en cachette\end{définition}
\begin{définition}\cmn 偷看\end{définition}
\begin{exemple}\jya ku-nɯxsɯ ɲɯ-ŋu\cmn 他在偷看\end{exemple}
\begin{exemple}\jya tɤ-pɤtso ɲɯ-nɯxsɯ\cmn 小孩子在偷看\end{exemple}
\begin{exemple}\jya nɯ tɯrme ɲɯ-nɯxsɯ\cmn 那个人在偷看\end{exemple}
\begin{exemple}\jya ma-lɤ-tɯ-nɯxsɯ tɕe lɤ-ɣi\cmn 你不要偷看,你进来吧\end{exemple}\acception{2}
\paradigme{\textit{dir :} \jya lɤ-}
\begin{définition}\fra laisser apparaître un petit bout\end{définition}
\begin{définition}\cmn 露出了一部分\end{définition}
\begin{exemple}\jya nɤ-laχtɕha ɲɯ-nɯxsɯ tɕe tɯ-nɯ-βde ma\cmn 你的东西露出来,消息不要丢掉\end{exemple}\end{entrée}

\begin{entrée}
\vedette{\hypertarget{Ⓔnɯxtɕhɤz}{\papi{ nɯxtɕhɤz}}}\markboth{nɯxtɕhɤz}{}\classe{vi}
\paradigme{\textit{dir :} \jya tɤ-}
\begin{définition}\fra avoir un tel tempérament, une telle habitude, une telle propension\end{définition}
\begin{définition}\cmn 有这样的性格,有这样的习惯\end{définition}
\begin{exemple}\jya ɯʑo ɲɯ-nɯxtɕhɤz\cmn 他有那个习惯\end{exemple}
\begin{exemple}\jya nɤʑo kɯnɤ tɯ-nɯxtɕhɤz\cmn 你也有那个习惯\end{exemple}
\begin{exemple}\jya aʑo ɲɯ-nɯxtɕhaz-a\cmn 我有这个习惯\end{exemple}
\begin{exemple}\jya nɤki nɤ-kɤ-fse nɯnɯ nɤʑo tɯ-nɯxtɕhɤz ɕti\cmn 你有这样的习惯\end{exemple}
\begin{exemple}\jya nɯ kɯ-fse kɤ-ti tɯ-nɯxtɕhɤz ɕti\cmn 你习惯这样说\end{exemple}
\begin{exemple}\jya kɤ-ŋɤn tɯ-nɯxtɕhɤz ɕti\cmn 你习惯做坏事\end{exemple}
\begin{exemple}\jya nɯ kɯ-fse kɤ-βzu nɯxtɕhɤz\cmn 他习惯那样做\end{exemple}
\begin{exemple}\jya tɤresɤpɯpa kɤ-βzu nɯxtɕhɤz\cmn 他习惯取笑别人\end{exemple}
\begin{exemple}\jya khramba kɤ-βzu mɤ-nɯxtɕhɤz\cmn 他不习惯说谎\end{exemple}\end{entrée}

\begin{entrée}
\vedette{\hypertarget{Ⓔnɯxtshi}{\papi{ nɯxtshi}}}\markboth{nɯxtshi}{}\classe{adv}
\begin{définition}\fra cette fois-là\end{définition}
\begin{définition}\cmn 那一次\end{définition}
\begin{exemple}\jya rgɯnba tɤ-ari tɕe, nɯ́xtshi nɯ tɯrme wuma pɯ-dɤn\cmn 我们去寺庙大那一次,人很多\end{exemple}
\end{entrée}

\begin{entrée}
\vedette{\hypertarget{Ⓔnɯχɤnloʁ}{\papi{ nɯχɤnloʁ}}}\markboth{nɯχɤnloʁ}{}\classe{vi}
\paradigme{\textit{dir :} \jya pɯ-}
\begin{définition}\fra être peu réactif (à cause de l'âge)\end{définition}
\begin{définition}\cmn 思维迟钝(因为年老)\end{définition}
\begin{exemple}\jya kɤ-nɯχɤnloʁnɯ nɯ mɯ́j-pe\cmn 上了年纪思维迟钝很不好\end{exemple}\end{entrée}

\begin{entrée}
\vedette{\hypertarget{Ⓔnɯχpɯn}{\papi{ nɯχpɯn}}}\markboth{nɯχpɯn}{}\classe{vi}
\paradigme{\textit{dir :} \jya lɤ-}
\begin{définition}\ 
\begin{déclaration}\grammar{denom}\end{déclaration}\end{définition}
\begin{définition}\fra devenir moine\end{définition}
\begin{définition}\cmn 当和尚\end{définition}
\begin{relation-sémantique}\confer{
\hyperlink{Ⓔχpɯn}{\textit{ \papi{χpɯn}}}
}\end{relation-sémantique}
\begin{relation-sémantique}\confer{
\hyperlink{Ⓔrɤχpɯn}{\textit{ \papi{rɤχpɯn}}}
}\end{relation-sémantique}\end{entrée}

\begin{entrée}
\vedette{\hypertarget{Ⓔnɯχpɯnbu}{\papi{ nɯχpɯnbu}}}\markboth{nɯχpɯnbu}{}\classe{vi}
\paradigme{\textit{dir :} \jya lɤ-}
\begin{définition}\fra avoir le pouvoir\end{définition}
\begin{définition}\cmn 掌权\end{définition}
\begin{exemple}\jya lɤ-nɯχpɯnbu (=χpɯnbu la-ndo)\cmn 他掌权了\end{exemple}
\begin{relation-sémantique}\confer{
\hyperlink{Ⓔχpɯnbu}{\textit{ \papi{χpɯnbu}}}
}\end{relation-sémantique}\end{entrée}

\begin{entrée}
\vedette{\hypertarget{Ⓔnɯχsɯmtoʁ}{\papi{ nɯχsɯmtoʁ}}}\markboth{nɯχsɯmtoʁ}{}
\classe{vi}
\begin{définition}\fra vivre\end{définition}
\begin{définition}\cmn 活;生存
\begin{déclaration}\use{针对人或者动物}\end{déclaration}\end{définition}
\begin{exemple}\jya jiɕqha nɯ ɲɯ-nɯχsɯmtoʁ\cmn 那个还活着\end{exemple}
\begin{exemple}\jya kɯtɕu kɤ-nɯχsɯmtoʁ ɲɯ-ɴqa ɕti\cmn 这里生存很辛苦\end{exemple}
\begin{exemple}\jya ɯ-ɲɯ-nɯχsɯmtoʁ\cmn 他还活着吗\end{exemple}\end{entrée}

\begin{entrée}
\vedette{\hypertarget{Ⓔnɯχtɕɤn}{\papi{ nɯχtɕɤn}}}\markboth{nɯχtɕɤn}{}\classe{vs}
\paradigme{\textit{dir :} \jya kɤ-}
\begin{définition}\fra terrible\end{définition}
\begin{définition}\cmn 恐怖;凶猛
\begin{déclaration}\use{仅用于传统故事中,形容施了魔法的物体(水、巨石、森林等)}\end{déclaration}\end{définition}
\begin{exemple}\jya ndzaʁlaŋ tɯrme jo-ɣi tɕe, tɯ-ci kɤ-kɯ-nɯχtɕɤn ɯ-ŋgɯ ɕ-pjɯ́-wɣ-βde ɲɯ-ra\cmn (妖界)来了凡人,我们要把他扔进恐怖的水里!\end{exemple}
\begin{exemple}\jya tɯ-ci kɤ-kɯ-nɯχtɕɤn nɯnɯ tɕe tɕe tu-ola ʑo kɯ-fse pjɤ-ɕti tɕe, nɯ pjɯ-tɯ-βde nɯ tɯrme pjɯ-kɯ-si pjɤ-ŋgrɤl\cmn “恐怖的水”在沸腾一样,(妖)把人一扔进去就必死无疑\end{exemple}
\end{entrée}

\begin{entrée}
\vedette{\hypertarget{Ⓔnɯχtɕɯrɯ}{\papi{ nɯχtɕɯrɯ}}}\markboth{nɯχtɕɯrɯ}{}\classe{vi}
\paradigme{\textit{dir :} \jya nɯ-}
\begin{définition}\fra se déshabiller complètement\end{définition}
\begin{définition}\cmn 把衣服脱光\end{définition}\begin{sous-entrée}
\vedette{\hypertarget{}{\papi{ znɯχtɕɯrɯ}}}\markboth{znɯχtɕɯrɯ}{}\classe{vt}
\begin{définition}\fra déshabiller complètement\end{définition}
\begin{définition}\cmn 把……的衣服脱光\end{définition}
\begin{exemple}\jya tɤ-pɤtso nɯ-znɯχtɕɯrɯ-t-a\cmn 我把小孩子的衣服脱光了\end{exemple}
\begin{relation-sémantique}\confer{
\hyperlink{Ⓔχtɕɯrɯ}{\textit{ \papi{χtɕɯrɯ}}}
}\end{relation-sémantique}
\end{sous-entrée}\end{entrée}

\begin{entrée}
\vedette{\hypertarget{Ⓔnɯχtɯntʂu}{\papi{ nɯχtɯntʂu}}}\markboth{nɯχtɯntʂu}{}
\classe{vi}
\paradigme{\textit{dir :} \jya tɤ-}
\begin{définition}\fra convivial, sociable\end{définition}
\begin{définition}\cmn 合得来;合群\end{définition}
\begin{exemple}\jya jiɕqha nɯ ɲɯ-nɯχtɯntʂu\cmn 他跟别人合得来\end{exemple}
\begin{exemple}\jya ɯ-zda ra nɯ-rca ɲɯ-nɯχtɯntʂu\cmn 他跟他的伙伴很合得来\end{exemple}
\begin{exemple}\jya jiɕqha nɯ mɯ́j-nɯχtɯntʂu\cmn 他很孤僻\end{exemple}
\begin{exemple}\jya tu-nɯχtɯntʂu-a\cmn 我跟别人合得来\end{exemple}\end{entrée}

\begin{entrée}
\vedette{\hypertarget{Ⓔnɯzarzɯr}{\papi{ nɯzarzɯr}}}\markboth{nɯzarzɯr}{}
\classe{vi}
\paradigme{\textit{dir :} \jya pɯ-}
\begin{définition}\fra avoir la tête qui tourne (ovins, tremblante du mouton?)\end{définition}
\begin{définition}\cmn 头晕倒下(羊)\end{définition}
\begin{exemple}\jya qaʑo pjɤ-nɯzarzɯr\cmn 绵羊头晕倒下了\end{exemple}
\begin{exemple}\jya tshɤt kɤ-nɯzarzɯr nɯxtɕhɤz\cmn 山羊经常头晕倒下\end{exemple}
\begin{exemple}\jya qaʑo cho tshɤt ɯ-kɤrnoʁ ɲɯ-mtɕɯr tɕe pjɯ-ndʐaβ tɕe nɯ pjɯ-nɯzarzɯr ŋu tɕe ɯ-rna ɯ-taʁ pjɯ́-wɣ-qraʁ tɕe tɤ-se a-pɯ-ɬoʁ tɕe tu-mna ŋgrɤl\cmn 绵羊和山羊头晕突然倒下,在羊的耳朵上割一刀,让血流出来就会好。\end{exemple}\end{entrée}

\begin{entrée}
\vedette{\hypertarget{Ⓔnɯzaχtɤt}{\papi{ nɯzaχtɤt}}}\markboth{nɯzaχtɤt}{}\classe{vi}
\paradigme{\textit{dir :} \jya thɯ-}
\begin{définition}\fra manger de la nourriture pour les morts\end{définition}
\begin{définition}\cmn 吃死人的食物(骂人的话)\end{définition}
\begin{relation-sémantique}\confer{
\hyperlink{Ⓔzaχtɤt}{\textit{ \papi{zaχtɤt}}}
}\end{relation-sémantique}
\begin{relation-sémantique}\synonyme{
\hyperlink{Ⓔnɯzɤmpo}{\textit{ \papi{nɯzɤmpo}}}
}\end{relation-sémantique}\end{entrée}

\begin{entrée}
\vedette{\hypertarget{Ⓔnɯzɤmpo}{\papi{ nɯzɤmpo}}}\markboth{nɯzɤmpo}{}\classe{vi}
\paradigme{\textit{dir :} \jya thɯ-}
\begin{définition}\ 
\begin{déclaration}\grammar{denom}\end{déclaration}\end{définition}
\begin{définition}\fra manger de la nourriture pour les morts\end{définition}
\begin{définition}\cmn 吃死人的食物(骂人的话)\end{définition}
\begin{relation-sémantique}\confer{
\hyperlink{Ⓔnɯzaχtɤt}{\textit{ \papi{nɯzaχtɤt}}}
}\end{relation-sémantique}
\begin{relation-sémantique}\confer{
\hyperlink{Ⓔzɤmpo}{\textit{ \papi{zɤmpo}}}
}\end{relation-sémantique}\end{entrée}

\begin{entrée}
\vedette{\hypertarget{Ⓔnɯzɤsna}{\papi{ nɯzɤsna}}}\markboth{nɯzɤsna}{}
\classe{vl}
\paradigme{\textit{dir :} \jya thɯ-}
\begin{définition}\ 
\begin{déclaration}\grammar{denom}\end{déclaration}\end{définition}
\begin{définition}\fra manger la nourriture pour les morts\end{définition}
\begin{définition}\cmn 吃死人的食物\end{définition}
\begin{exemple}\jya thɯ-nɯzɤsne\cmn 你去死吧!\end{exemple}
\begin{relation-sémantique}\confer{
\hyperlink{Ⓔzɤsna}{\textit{ \papi{zɤsna}}}
}\end{relation-sémantique}\end{entrée}

\begin{entrée}
\vedette{\hypertarget{Ⓔnɯzdɯɣ}{\papi{ nɯzdɯɣ}}}\markboth{nɯzdɯɣ}{}
\classe{vt}
\paradigme{\textit{dir :} \jya thɯ-}
\begin{définition}\fra s'inquiéter pour quelqu'un\end{définition}
\begin{définition}\cmn 为别人担心
\begin{déclaration} \étymologie{\papi{sdug}}\end{déclaration}\end{définition}
\begin{exemple}\jya a-mu ɲɯ-nɯzdɯɣ-a\cmn 我担心我的母亲\end{exemple}
\begin{exemple}\jya a-rɟit ɲɯ-nɯzdɯɣ-a\cmn 我担心我的儿子\end{exemple}
\begin{exemple}\jya nɤj nɤ-rʑaβ ɲɯ-tɯ-nɯzdɯɣ\cmn 你担心你的妻子\end{exemple}
\begin{exemple}\jya ɲɯ-ta-nɯzdɯɣ\cmn 我为你担心\end{exemple}
\begin{sous-entrée}
\vedette{\hypertarget{}{\papi{ sɤnɯzdɯɣ}}}\markboth{sɤnɯzdɯɣ}{}\classe{vs}
\begin{définition}\ 
\begin{déclaration}\grammar{deexp}\end{déclaration}\end{définition}
\begin{définition}\fra causer de l'inquiétude\end{définition}
\begin{définition}\cmn 令人担心\end{définition}
\end{sous-entrée}\begin{sous-entrée}
\vedette{\hypertarget{}{\papi{ znɯzdɯɣ}}}\markboth{znɯzdɯɣ}{}\classe{vt}
\paradigme{\textit{dir :} \jya nɯ-}
\begin{définition}\ 
\begin{déclaration}\grammar{caus}\end{déclaration}\end{définition}
\begin{définition}\fra causer de l'inquiétude à qqun\end{définition}
\begin{définition}\cmn 令……担心\end{définition}
\begin{exemple}\jya nɯ-znɯzdɯɣ-a\cmn 我让他担心了\end{exemple}
\end{sous-entrée}\end{entrée}

\begin{entrée}
\vedette{\hypertarget{Ⓔnɯzdɯsŋɤl}{\papi{ nɯzdɯsŋɤl}}}\markboth{nɯzdɯsŋɤl}{}\classe{vi}
\paradigme{\textit{dir :} \jya pɯ-}
\begin{définition}\fra supporter toutes sortes de difficultés\end{définition}
\begin{définition}\cmn 受尽苦难
\begin{déclaration} \étymologie{\papi{sdug.bsŋal}}\end{déclaration}\end{définition}
\begin{exemple}\jya kɯ-rtaʁ ʑo pɯ-nɯzdɯsŋal-a\cmn 我受够了苦难\end{exemple}
\begin{relation-sémantique}\confer{
\hyperlink{Ⓔtɤzdɯɣ}{\textit{ \papi{tɤzdɯɣ}}}
}\end{relation-sémantique}\end{entrée}

\begin{entrée}
\vedette{\hypertarget{Ⓔnɯzdɯxpa}{\papi{ nɯzdɯxpa}}}\markboth{nɯzdɯxpa}{} (\variante{znɯzdɯxpa}) 
\classe{vt}
\paradigme{\textit{dir :} \jya nɯ-}
\begin{définition}\fra avoir pitié de\end{définition}
\begin{définition}\cmn 可怜别人\end{définition}
\begin{exemple}\jya nɯ-nɯzdɯxpa-t-a-nɯ\cmn 我可怜他们\end{exemple}
\begin{exemple}\jya jiɕqha nɯ ɯ-ŋgu mɯ́j-thon, ɲɯ-nɯzdɯxpe-a\cmn 他很穷,我觉得很可怜\end{exemple}
\begin{relation-sémantique}\confer{
\hyperlink{Ⓔsɤzdɯxpa}{\textit{ \papi{sɤzdɯxpa}}}
}\end{relation-sémantique}\end{entrée}

\begin{entrée}
\vedette{\hypertarget{Ⓔnɯzgomdʑo}{\papi{ nɯzgomdʑo}}}\markboth{nɯzgomdʑo}{}
\classe{vi}
\paradigme{\textit{dir :} \jya \_}
\begin{définition}\fra se promener dans la montagne et admirer le paysage\end{définition}
\begin{définition}\cmn 在山上观光\end{définition}
\begin{exemple}\jya jiɕqha nɯ kɯ-nɯzgomdʑo jɤ-ari\cmn 他去山上观光了\end{exemple}
\begin{relation-sémantique}\synonyme{
\hyperlink{Ⓔnɤmɲole}{\textit{ \papi{nɤmɲole}}}
}\end{relation-sémantique}\end{entrée}

\begin{entrée}
\vedette{\hypertarget{Ⓔnɯzgrɯtɕhɯ}{\papi{ nɯzgrɯtɕhɯ}}}\markboth{nɯzgrɯtɕhɯ}{}
\classe{vt}
\paradigme{\textit{dir :} \jya tɤ-}
\begin{définition}\ 
\begin{déclaration}\grammar{incorp}\end{déclaration}\end{définition}
\begin{définition}\fra donner un coup de coude\end{définition}
\begin{définition}\cmn 用肘碰\end{définition}
\begin{exemple}\jya tɤ́-wɣ-nɯzgrɯtɕhɯ-a\cmn 他用肘打了我\end{exemple}
\begin{relation-sémantique}\confer{
\hyperlink{Ⓔzgrɯtɕhɯ}{\textit{ \papi{zgrɯtɕhɯ}}}
}\end{relation-sémantique}\end{entrée}

\begin{entrée}
\vedette{\hypertarget{Ⓔnɯzɣɯt}{\papi{ nɯzɣɯt}}}\markboth{nɯzɣɯt}{}
\begin{relation-sémantique}\confer{
\hyperlink{Ⓔzɣɯt}{\textit{ \papi{zɣɯt}}}
}\end{relation-sémantique}\end{entrée}

\begin{entrée}
\vedette{\hypertarget{Ⓔnɯzjaŋ}{\papi{ nɯzjaŋ}}}\markboth{nɯzjaŋ}{}
\begin{relation-sémantique}\confer{
\hyperlink{Ⓔɣɤzjaŋlaŋ}{\textit{ \papi{ɣɤzjaŋlaŋ}}}
}\end{relation-sémantique}\end{entrée}

\begin{entrée}
\vedette{\hypertarget{Ⓔnɯzɟɯ}{\papi{ nɯzɟɯ}}}\markboth{nɯzɟɯ}{}\classe{vi}
\paradigme{\textit{dir :} \jya pɯ-}
\begin{définition}\fra pâtir de quelque chose\end{définition}
\begin{définition}\cmn 吃亏\end{définition}
\begin{exemple}\jya pɯ-nɯzɟɯ\cmn 他吃亏了\end{exemple}
\begin{exemple}\jya tɯtsɣe tɤ-βzu-tɕi, aʑo pɯ-nɯzɟɯ-a, nɤʑo pɯ-tɯ-nɤndʑe\cmn 我们俩做了生意,我吃亏了,你占了便宜\end{exemple}
\begin{exemple}\jya ʑɴɢɯloʁ pɯ-nɯkro-tɕi, nɤʑo pɯ-tɯ-nɤndʑe, aʑo pɯ-nɯzɟɯ-a\cmn 我们俩分核桃,你占了便宜,我吃亏了\end{exemple}
\begin{relation-sémantique}\antonyme{
\hyperlink{Ⓔnɤndʑe}{\textit{ \papi{nɤndʑe}}}
}\end{relation-sémantique}\begin{sous-entrée}
\vedette{\hypertarget{}{\papi{ znɯzɟɯ}}}\markboth{znɯzɟɯ}{}\classe{vt}
\paradigme{\textit{dir :} \jya pɯ-}
\begin{définition}\ 
\begin{déclaration}\grammar{caus}\end{déclaration}\end{définition}
\begin{définition}\fra faire pâtir quelqu'un de quelque chose\end{définition}
\begin{définition}\cmn 使吃亏\end{définition}
\begin{exemple}\jya pɯ-kɯ-znɯzɟɯ-a\cmn 你让我吃亏了\end{exemple}
\end{sous-entrée}\end{entrée}

\begin{entrée}
\vedette{\hypertarget{Ⓔnɯzrɯɣru}{\papi{ nɯzrɯɣru}}}\markboth{nɯzrɯɣru}{}
\classe{vi}
\paradigme{\textit{dir :} \jya pɯ-}
\begin{définition}\ 
\begin{déclaration}\grammar{incorp}\end{déclaration}\end{définition}
\begin{définition}\fra chercher les poux\end{définition}
\begin{définition}\cmn 找虱子\end{définition}
\begin{exemple}\jya jiɕqha mbroχpa ɲɯ-nɯzrɯɣru\cmn 那个牧民在找虱子\end{exemple}
\begin{exemple}\jya pɯ-nɯzrɯɣru-a\cmn 我在找虱子\end{exemple}
\begin{relation-sémantique}\confer{
\hyperlink{Ⓔzrɯɣru}{\textit{ \papi{zrɯɣru}}}
}\end{relation-sémantique}\end{entrée}

\begin{entrée}
\vedette{\hypertarget{Ⓔnɯzʁe}{\papi{ nɯzʁe}}}\markboth{nɯzʁe}{}
\classe{vt}
\paradigme{\textit{dir :} \jya \_}
\begin{définition}\fra transporter un par un\end{définition}
\begin{définition}\cmn 一个一个地搬运\end{définition}
\begin{exemple}\jya rdɤstaʁ lɤ-nɯzʁe-t-a\cmn 我把石头搬过去了\end{exemple}
\begin{relation-sémantique}\confer{
\hyperlink{Ⓔnɯsɯzʁe}{\textit{ \papi{nɯsɯzʁe}}}
}\end{relation-sémantique}\end{entrée}

\begin{entrée}
\vedette{\hypertarget{Ⓔnɯʑɤla}{\papi{ nɯʑɤla}}}\markboth{nɯʑɤla}{}\classe{vt}
\paradigme{\textit{dir :} \jya kɤ-}
\begin{définition}\fra transmettre (poux)\end{définition}
\begin{définition}\cmn 传染(虱子)\end{définition}
\begin{exemple}\jya a-tɕɯ ɯ-zda ra kɯ zrɯɣ kó-wɣ-nɯʑɤla\cmn 我儿子的同学给他传染了虱子\end{exemple}
\begin{relation-sémantique}\synonyme{
\hyperlink{Ⓔɕte}{\textit{ \papi{ɕte}}}
}\end{relation-sémantique}
\end{entrée}

\begin{entrée}
\vedette{\hypertarget{Ⓔnɯʑɤŋɤn}{\papi{ nɯʑɤŋɤn}}}\markboth{nɯʑɤŋɤn}{}\classe{vt}
\paradigme{\textit{dir :} \jya tɤ-}
\begin{définition}\fra taquiner\end{définition}
\begin{définition}\cmn 逗着……玩\end{définition}
\begin{exemple}\jya tu-ta-nɯʑɤŋɤn ɕti tɕe ma-tɤ-tɯ-qhe je\cmn 我只是逗你的,你不要生气\end{exemple}
\begin{relation-sémantique}\synonyme{
\hyperlink{Ⓔnɯrtɕa}{\textit{ \papi{nɯrtɕa}}}
}\end{relation-sémantique}\end{entrée}

\begin{entrée}
\vedette{\hypertarget{Ⓔnɯʑɤzdaŋ}{\papi{ nɯʑɤzdaŋ}}}\markboth{nɯʑɤzdaŋ}{}\classe{vt}
\paradigme{\textit{dir :} \jya tɤ-}
\begin{définition}\fra envier, vouloir imiter\end{définition}
\begin{définition}\cmn 妒忌
\begin{déclaration} \étymologie{\papi{ʑe.sdaŋ}}\end{déclaration}\end{définition}
\begin{sous-entrée}
\vedette{\hypertarget{}{\papi{ sɤnɯʑɤzdaŋ}}}\markboth{sɤnɯʑɤzdaŋ}{}\classe{vi}
\begin{définition}\fra envier les gens\end{définition}
\begin{définition}\cmn 妒忌别人\end{définition}
\begin{exemple}\jya ma-tɯ-sɤnɯʑɤzdaŋ\cmn 你不要妒忌别人\end{exemple}
\begin{relation-sémantique}\synonyme{
\hyperlink{Ⓔnɤʑɤmŋɤn}{\textit{ \papi{nɤʑɤmŋɤn}}}
}\end{relation-sémantique}
\end{sous-entrée}\end{entrée}

\begin{entrée}
\vedette{\hypertarget{Ⓔnɯʑgrɤɣ}{\papi{ nɯʑgrɤɣ}}}\markboth{nɯʑgrɤɣ}{}\classe{vt}
\paradigme{\textit{dir :} \jya pɯ-}
\begin{définition}\fra renverser avec force\end{définition}
\begin{définition}\cmn 很轻松地把对方摔下去\end{définition}
\begin{exemple}\jya pa-nɯʑgrɤɣ ʑo pa-tʂaβ\cmn 他很轻松地把他摔下去了\end{exemple}
\begin{relation-sémantique}\synonyme{
\hyperlink{Ⓔnɯɕkrɤɣ}{\textit{ \papi{nɯɕkrɤɣ}}}
}\end{relation-sémantique}
\begin{relation-sémantique}\confer{
\hyperlink{Ⓔʑgrɤɣʑgrɤɣ}{\textit{ \papi{ʑgrɤɣʑgrɤɣ}}}
}\end{relation-sémantique}\end{entrée}

\begin{entrée}
\vedette{\hypertarget{Ⓔnɯʑɣɤβri}{\papi{ nɯʑɣɤβri}}}\markboth{nɯʑɣɤβri}{}
\begin{relation-sémantique}\confer{
\hyperlink{Ⓔβri}{\textit{ \papi{βri}}}
}\end{relation-sémantique}\end{entrée}

\begin{entrée}
\vedette{\hypertarget{Ⓔnɯʑmbɤr}{\papi{ nɯʑmbɤr}}}\markboth{nɯʑmbɤr}{}\classe{vi}
\paradigme{\textit{dir :} \jya nɯ-}
\begin{définition}\ 
\begin{déclaration}\grammar{denom}\end{déclaration}\end{définition}
\begin{définition}\fra avoir une pustule\end{définition}
\begin{définition}\cmn 生疮\end{définition}
\begin{exemple}\jya a-rŋa ɯ-ʑmbɤr ɲɤ-ɬoʁ, ɲɯ-nɯʑmbar-a\cmn 我脸上生了疮\end{exemple}
\begin{exemple}\jya pɯ-nɯʑmbɤr tɕe ɯ-sta ʁmazgrɯβ tu\cmn 他生过疮,有伤疤\end{exemple}
\begin{relation-sémantique}\confer{
\hyperlink{Ⓔʑmbɤr}{\textit{ \papi{ʑmbɤr}}}
}\end{relation-sémantique}\end{entrée}

\begin{entrée}
\vedette{\hypertarget{Ⓔnɯʑo}{\papi{ nɯʑo}}}\markboth{nɯʑo}{}\classe{pro}
\begin{définition}\fra vous\end{définition}
\begin{définition}\cmn 你们\end{définition}
\end{entrée}

\begin{entrée}
\vedette{\hypertarget{Ⓔnɯʑɯβ}{\papi{ nɯʑɯβ}}}\markboth{nɯʑɯβ}{}\classe{vi}
\paradigme{\textit{dir :} \jya pɯ-}
\paradigme{\textit{dir :} \jya kɤ-}
\begin{définition}\fra s’endormir\end{définition}
\begin{définition}\cmn 睡着\end{définition}
\begin{exemple}\jya pɯ-nɯʑɯβ-a\cmn 我睡着了\end{exemple}
\begin{exemple}\jya ma-tɯ-ɤrju-nɯ tɕe, aj pjɯ-nɯʑɯβ-a\cmn 你们不要说话,我在睡觉\end{exemple}\begin{sous-entrée}
\vedette{\hypertarget{}{\papi{ ɣɤnɯʑɯβ}}}\markboth{ɣɤnɯʑɯβ}{}\classe{vs}
\begin{définition}\fra qui arrive facilement à s'endormir\end{définition}
\begin{définition}\cmn 容易入眠;容易睡着\end{définition}
\begin{exemple}\jya ɲɯ-ɣɤnɯʑɯβ\cmn (小孩子)容易入眠\end{exemple}
\begin{relation-sémantique}\confer{
 \papi{ɯ-ʑɯβ}
}\end{relation-sémantique}
\end{sous-entrée}\end{entrée}

\begin{entrée}
\vedette{\hypertarget{Ⓔnɯʑɯβri}{\papi{ nɯʑɯβri}}}\markboth{nɯʑɯβri}{}
\classe{vi}
\paradigme{\textit{dir :} \jya pɯ-}
\begin{définition}\fra veiller; avoir une insomnie\end{définition}
\begin{définition}\cmn 失眠;熬夜‘没有睡够\end{définition}
\begin{exemple}\jya pjɯ-nɯʑɯβri-a\cmn 我熬夜\end{exemple}
\begin{exemple}\jya a-ndʐuwa pɯ-tu tɕe kɤ-nɯʑɯβ mɯ-pɯ-ŋgrɯ tɕe pjɤ-nɯʑɯβri-a\cmn 因为有客人没有睡成\end{exemple}
\begin{exemple}\jya jɯfɕɯr pɯ-nɯʑɯβri-a, jisŋi a-ʑɯβ ɲɯ-ɣi\cmn 我昨天熬夜了,今天就打瞌睡\end{exemple}
\begin{relation-sémantique}\confer{
\hyperlink{Ⓔtɯ-ʑɯβ}{\textit{ \papi{tɯ-ʑɯβ}}}
}\end{relation-sémantique}
\begin{relation-sémantique}\confer{
\hyperlink{ⒺriⒽ1}{\textit{ \papi{ri}}}
}\end{relation-sémantique}\end{entrée}

\newpage\caractère{ɲ}

\begin{entrée}
\vedette{\hypertarget{Ⓔɲakhri}{\papi{ ɲakhri}}}\markboth{ɲakhri}{}\classe{n}
\begin{définition}\fra lit\end{définition}
\begin{définition}\cmn 床
\begin{déclaration} \étymologie{\papi{ɲal.kʰri}}\end{déclaration}\end{définition}\end{entrée}

\begin{entrée}
\vedette{\hypertarget{Ⓔɲaɲa}{\papi{ ɲaɲa}}}\markboth{ɲaɲa}{}
\classe{n}
\begin{définition}\fra agneau\end{définition}
\begin{définition}\cmn 绵羊羔\end{définition}
\begin{exemple}\jya ɲaɲa ɣɯ ɯ-rme\cmn 羊羔的毛\end{exemple}\end{entrée}

\begin{entrée}
\vedette{\hypertarget{Ⓔɲaʁ}{\papi{ ɲaʁ}}}\markboth{ɲaʁ}{}\classe{vs}
\paradigme{\textit{dir :} \jya nɯ-}
\begin{définition}\fra noir\end{définition}
\begin{définition}\cmn 黑(颜色)\end{définition}
\begin{relation-sémantique}\confer{
\hyperlink{Ⓔsɯɣɲaʁ}{\textit{ \papi{sɯɣɲaʁ}}}
}\end{relation-sémantique}\end{entrée}

\begin{entrée}
\vedette{\hypertarget{Ⓔɲaʁtɣi}{\papi{ ɲaʁtɣi}}}\markboth{ɲaʁtɣi}{}
\classe{n}
\begin{définition}\fra empan (pouce et majeur)\end{définition}
\begin{définition}\cmn 一拃(大拇指和中指之间的距离)\end{définition}
\begin{exemple}\jya ɲaʁtɣi tɯ-tɣa\end{exemple}\end{entrée}

\begin{entrée}
\vedette{\hypertarget{Ⓔɲat}{\papi{ ɲat}}}\markboth{ɲat}{}\classe{vi}
\paradigme{\textit{dir :} \jya nɯ-}
\begin{définition}\fra être fatigué\end{définition}
\begin{définition}\cmn 累\end{définition}
\begin{exemple}\jya aʑo pɯ-rɤma-a tɕe ɲɤ-ɲat-a\cmn 我劳动了就累了\end{exemple}\begin{sous-entrée}
\vedette{\hypertarget{}{\papi{ sɯɣɲat}}}\markboth{sɯɣɲat}{}\classe{vt}
\paradigme{\textit{dir :} \jya nɯ-}
\begin{définition}\fra fatiguer\end{définition}
\begin{définition}\cmn 令人累\end{définition}
\begin{exemple}\jya ki kɤ-nɤma ɲɯ-ɴqa tɕe ɲɯ-kɯ-sɯɣɲat\cmn 这个工作很辛苦,令人很累\end{exemple}
\begin{relation-sémantique}\confer{
\hyperlink{Ⓔsɤɣɲat}{\textit{ \papi{sɤɣɲat}}}
}\end{relation-sémantique}
\end{sous-entrée}\end{entrée}

\begin{entrée}
\vedette{\hypertarget{Ⓔɲawa}{\papi{ ɲawa}}}\markboth{ɲawa}{}\classe{n}
\begin{définition}\fra accouplement (animaux)\end{définition}
\begin{définition}\cmn 交配(动物)\end{définition}\end{entrée}

\begin{entrée}
\vedette{\hypertarget{Ⓔɲɤβɲɤβ}{\papi{ ɲɤβɲɤβ}}}\markboth{ɲɤβɲɤβ}{}
\classe{idph.2}
\begin{définition}\fra très mou\end{définition}
\begin{définition}\cmn 软绵绵\end{définition}
\begin{exemple}\jya ɲɯ-ngo tɕe, ɲɤβɲɤβ ʑo ɲɯ-rɤʑi\cmn 他病了就没有精神\end{exemple}
\begin{exemple}\jya ko-smi ɲɤβɲɤβ ʑo\cmn 熟得很软\end{exemple}\end{entrée}

\begin{entrée}
\vedette{\hypertarget{Ⓔɲɤβrɯɣ}{\papi{ ɲɤβrɯɣ}}}\markboth{ɲɤβrɯɣ}{}\classe{n}
\begin{définition}\fra espèce d'arbre\end{définition}
\begin{définition}\cmn 树的一种\end{définition}\end{entrée}

\begin{entrée}
\vedette{\hypertarget{Ⓔɲɤndɤpa}{\papi{ ɲɤndɤpa}}}\markboth{ɲɤndɤpa}{}\classe{adv}
\begin{définition}\fra dans quatre ans\end{définition}
\begin{définition}\cmn 四年以后\end{définition}\end{entrée}

\begin{entrée}
\vedette{\hypertarget{Ⓔɲɤndi}{\papi{ ɲɤndi}}}\markboth{ɲɤndi}{}\classe{adv}
\begin{définition}\fra dans quatre jours\end{définition}
\begin{définition}\cmn 四天以后\end{définition}\end{entrée}

\begin{entrée}
\vedette{\hypertarget{Ⓔɲɤntsho}{\papi{ ɲɤntsho}}}\markboth{ɲɤntsho}{}
\classe{n}
\begin{définition}\fra pistolet\end{définition}
\begin{définition}\cmn 手枪\end{définition}\end{entrée}

\begin{entrée}
\vedette{\hypertarget{Ⓔɲɤsma}{\papi{ ɲɤsma}}}\markboth{ɲɤsma}{}\classe{n}
\begin{définition}\fra feue (décédée)\end{définition}
\begin{définition}\cmn 去世了的女子\end{définition}
\begin{relation-sémantique}\confer{
\hyperlink{Ⓔɲɤspa}{\textit{ \papi{ɲɤspa}}}
}\end{relation-sémantique}\end{entrée}

\begin{entrée}
\vedette{\hypertarget{Ⓔɲɤspa}{\papi{ ɲɤspa}}}\markboth{ɲɤspa}{}\classe{n}
\begin{définition}\fra feu (homme décédé)\end{définition}
\begin{définition}\cmn 去世了的男人
\begin{déclaration} \étymologie{\papi{ɲes.pa}}\end{déclaration}\end{définition}
\begin{exemple}\jya ɲɤspa a-nɯ-tɯ-βze\cmn 你去死吧!\end{exemple}\end{entrée}

\begin{entrée}
\vedette{\hypertarget{Ⓔɲɤzma}{\papi{ ɲɤzma}}}\markboth{ɲɤzma}{}\classe{n}
\begin{définition}\fra feue (femme décédée)\end{définition}
\begin{définition}\cmn 去世了的女子
\begin{déclaration} \étymologie{\papi{ɲes.ma}}\end{déclaration}\end{définition}\end{entrée}

\begin{entrée}
\vedette{\hypertarget{Ⓔɲcɤr}{\papi{ ɲcɤr}}}\markboth{ɲcɤr}{}
\classe{vt}
\paradigme{\textit{dir :} \jya pɯ-}
\begin{définition}\fra appuyer\end{définition}
\begin{définition}\cmn (用全身的力量) 按;压住\end{définition}
\begin{exemple}\jya rdɤstaʁ kɯ si nɯ pɯ-sɯ-ɲcɤr\cmn 你用石头把柴压着(防止柴被人拿)\end{exemple}
\begin{exemple}\jya ki ɕoʁɕoʁ ki thɯci kɯ pɯ-sɯ-ɲcɤr ma qale kɯ nɯtsɯm\cmn 你用什么东西把纸压住,不然会被风吹走\end{exemple}
\begin{exemple}\jya paʁ pɯ-ɲcar-a\cmn 我把猪压住了\end{exemple}
\begin{exemple}\jya tɯrme ɯ-stu mɤ-kɯ-fse nɯ pjɯ́-wɣ-ɲcɤr tɕe ɯ-stu kú-wɣ-z-rɤʑi ŋu\cmn 人不听话要压一下,使他温顺\end{exemple}\end{entrée}

\begin{entrée}
\vedette{\hypertarget{Ⓔɲcɣɤɲcɣɤt}{\papi{ ɲcɣɤɲcɣɤt}}}\markboth{ɲcɣɤɲcɣɤt}{}\classe{idph.2}
\begin{définition}\fra très nombreux\end{définition}
\begin{définition}\cmn 很多(朝着同一个方向);嘈杂(声音);旺盛\end{définition}
\begin{exemple}\jya ɲcɣɤɲcɣɤt ʑo chɤ-k-ɤkhar-nɯ-ci\cmn 很多人围着坐\end{exemple}
\begin{exemple}\jya ɲcɣɤɲcɣɤt ʑo pjɤ-ɕkho\cmn 铺了很多(东西)在地上\end{exemple}
\begin{exemple}\jya tɤtʂu ɲcɣɤɲcɣɤt ʑo to-zwɤr-nɯ\cmn 他们把灯开得亮堂堂的\end{exemple}
\begin{exemple}\jya laχtɕha ɲcɣɤɲcɣɤt ʑo to-fɕɤm-nɯ\cmn 他们把(很多)东西摆出来了\end{exemple}
\begin{exemple}\jya tɯrme ɲcɣɤɲcɣɤt ʑo ɲɯ-ɤmdzɯt-nɯ\cmn 很多人在那里坐着,面向一个方向\end{exemple}\begin{sous-entrée}
\vedette{\hypertarget{}{\papi{ ɲcɣɤnɤɲcɣɤt}}}\markboth{ɲcɣɤnɤɲcɣɤt}{}\classe{idph.3}
\begin{exemple}\jya ɲcɣɤnɤɲcɣɤt ɲɯ-rɟaʁ-nɯ\cmn 很多人在跳舞\end{exemple}
\begin{relation-sémantique}\confer{
\hyperlink{Ⓔɣɤɲcɣɤɲcɣɤt}{\textit{ \papi{ɣɤɲcɣɤɲcɣɤt}}}
}\end{relation-sémantique}
\end{sous-entrée}\end{entrée}

\begin{entrée}
\vedette{\hypertarget{Ⓔɲchaʁɲchaʁ}{\papi{ ɲchaʁɲchaʁ}}}\markboth{ɲchaʁɲchaʁ}{}\classe{idph.2}
\begin{définition}\fra un peu froid\end{définition}
\begin{définition}\cmn 形容些微的寒冷\end{définition}
\begin{exemple}\jya jɯxɕo tɯ-mɯ ko-lɤt tɕe, ɯ-pɕi ɲɯ-mɯɕtaʁ ɲchaʁɲchaʁ\cmn 今天早上下了雨,外面有点冷\end{exemple}\end{entrée}

\begin{entrée}
\vedette{\hypertarget{Ⓔɲchɤɲchɤr}{\papi{ ɲchɤɲchɤr}}}\markboth{ɲchɤɲchɤr}{}
\classe{idph.2}
\begin{définition}\fra dilué\end{définition}
\begin{définition}\cmn 形容流体稀\end{définition}
\begin{exemple}\jya a-rɟɤɣi ɯ-ci pjɤ-dɤn tɕe ɲɯ-ŋgri ʑo ɲchɤɲchɤr, tɕe kɤ-rɤlaj mɯ́j-khɯ\cmn 糌粑里的水分太多,太稀了,没有办法挼\end{exemple}\end{entrée}

\begin{entrée}
\vedette{\hypertarget{Ⓔɲchɣaʁ}{\papi{ ɲchɣaʁ}}}\markboth{ɲchɣaʁ}{}
\classe{n}
\begin{définition}\fra écorce de bouleau\end{définition}
\begin{définition}\cmn 白桦树皮\end{définition}\end{entrée}

\begin{entrée}
\vedette{\hypertarget{Ⓔɲchɣaʁʑɤr}{\papi{ ɲchɣaʁʑɤr}}}\markboth{ɲchɣaʁʑɤr}{}
\classe{n}
\begin{définition}\fra verglas\end{définition}
\begin{définition}\cmn 冰雪路
\end{définition}\end{entrée}

\begin{entrée}
\vedette{\hypertarget{Ⓔɲchoʁ}{\papi{ ɲchoʁ}}}\markboth{ɲchoʁ}{}
\classe{vi}
\paradigme{\textit{dir :} \jya pɯ-}
\paradigme{\textit{dir :} \jya nɯ-}
\begin{définition}\fra se dégonfler\end{définition}
\begin{définition}\cmn 瘪下去
\end{définition}
\begin{exemple}\jya @piqiu pjɤ-ɲchoʁ\cmn 皮球瘪下去了\end{exemple}
\begin{exemple}\jya kɤrŋi kú-wɣ-sqa tɕe, lú-wɣ-sqa ɕɯmɯma dɤn ri, kɤ-smi tɕe tɕe ɲɯ-rkɯn ŋu tɕe tɕe ɲɤ-ɲchoʁ tu-kɯ-ti ŋu\cmn 蔬菜熟了以后就萎缩了\end{exemple}
\begin{exemple}\jya ɯ-lɯm ɲɯ-kɯ-xtɕi nɯ tɕe ɲɯ-kɯ-ɲchoʁ tu-kɯ-ti\cmn 体积变小就做“瘪下去”\end{exemple}
\begin{exemple}\jya ɲɯ-mtsɯr-a tɕe a-xtu ɲɤ-ɲchoʁ\cmn 我很饿,肚子瘪了\end{exemple}
\begin{relation-sémantique}\synonyme{
\hyperlink{Ⓔxɕɯβ}{\textit{ \papi{xɕɯβ}}}
}\end{relation-sémantique}\end{entrée}

\begin{entrée}
\vedette{\hypertarget{Ⓔɲcriɲcri}{\papi{ ɲcriɲcri}}}\markboth{ɲcriɲcri}{} (\variante{ɲcɯɲcri}) 
\classe{idph.2}
\begin{définition}\fra liquide, dilué (boue)\end{définition}
\begin{définition}\cmn 稀(泥巴)\end{définition}
\begin{relation-sémantique}\confer{
\hyperlink{Ⓔɲcrɯɣɲcrɯɣ}{\textit{ \papi{ɲcrɯɣɲcrɯɣ}}}
}\end{relation-sémantique}
\begin{relation-sémantique}\confer{
\hyperlink{Ⓔscrɯscri}{\textit{ \papi{scrɯscri}}}
}\end{relation-sémantique}
\begin{relation-sémantique}\confer{
\hyperlink{Ⓔχcrɯχcri}{\textit{ \papi{χcrɯχcri}}}
}\end{relation-sémantique}
\begin{relation-sémantique}\confer{
\hyperlink{Ⓔscrɯscri}{\textit{ \papi{scrɯscri}}}
}\end{relation-sémantique}\end{entrée}

\begin{entrée}
\vedette{\hypertarget{Ⓔɲcrɯɣɲcrɯɣ}{\papi{ ɲcrɯɣɲcrɯɣ}}}\markboth{ɲcrɯɣɲcrɯɣ}{}
\classe{idph.2}
\begin{définition}\fra mou et peu épais (boue, crotte)\end{définition}
\begin{définition}\cmn 形容(泥巴、牛粪) 又稀又软的样子\end{définition}
\begin{exemple}\jya tɤrcoʁ ɲcrɯɣɲcrɯɣ ɲɯ-ŋu\cmn 泥巴很稀\end{exemple}
\begin{relation-sémantique}\confer{
\hyperlink{Ⓔɲcriɲcri}{\textit{ \papi{ɲcriɲcri}}}
}\end{relation-sémantique}
\begin{relation-sémantique}\confer{
\hyperlink{Ⓔscrɯscri}{\textit{ \papi{scrɯscri}}}
}\end{relation-sémantique}\end{entrée}

\begin{entrée}
\vedette{\hypertarget{Ⓔɲɟa}{\papi{ ɲɟa}}}\markboth{ɲɟa}{}
\classe{vs}
\paradigme{\textit{dir :} \jya kɤ-}
\begin{définition}\fra trop vieux (animal)\end{définition}
\begin{définition}\cmn 老得不行(动物)\end{définition}
\begin{exemple}\jya jiɕqha nɯ ko-ɲɟa\cmn 那个老得不行了\end{exemple}
\begin{exemple}\jya fsapaʁ ko-ɲɟa\cmn 牲畜老得不行了\end{exemple}\end{entrée}

\begin{entrée}
\vedette{\hypertarget{Ⓔɲɟɤβ}{\papi{ ɲɟɤβ}}}\markboth{ɲɟɤβ}{}
\classe{vi}
\paradigme{\textit{dir :} \jya pɯ-}
\paradigme{\textit{dir :} \jya kɤ-}
\begin{définition}\ 
\begin{déclaration}\grammar{acaus}\end{déclaration}\end{définition}\acception{1}
\begin{définition}\fra s’aplatir\end{définition}
\begin{définition}\cmn 压扁;凹进去\end{définition}
\begin{exemple}\jya tɯthɯ pjɤ-ɲɟɤβ\cmn 锅子凹进去了\end{exemple}
\begin{exemple}\jya jɤmtsa pjɤ-ɲɟɤβ\cmn 炒菜锅凹进去了\end{exemple}
\begin{exemple}\jya khɯtsa pjɤ-ɲɟɤβ\cmn 碗凹进去了\end{exemple}\acception{2}
\begin{définition}\fra avoir une fracture\end{définition}
\begin{définition}\cmn 骨折\end{définition}
\begin{exemple}\jya a-rnom ko-ɲɟɤβ\cmn 我肋骨折了\end{exemple}
\begin{exemple}\jya a-ɣrɯmke pjɤ-ɲɟɤβ\cmn 我的手腕骨折了\end{exemple}
\begin{relation-sémantique}\confer{
\hyperlink{Ⓔchɤβ}{\textit{ \papi{chɤβ}}}
}\end{relation-sémantique}\end{entrée}

\begin{entrée}
\vedette{\hypertarget{Ⓔɲɟɤlɤsnɯrsnɯr}{\papi{ ɲɟɤlɤsnɯrsnɯr}}}\markboth{ɲɟɤlɤsnɯrsnɯr}{} (\variante{ɲɟɤlɤsnɯsnɯr}) 
\classe{n}
\begin{définition}\fra viande séchée conservée dans les intestins\end{définition}
\begin{définition}\cmn 装在肠子里的瘦肉\end{définition}\end{entrée}

\begin{entrée}
\vedette{\hypertarget{Ⓔɲɟɤrɲɟɤr}{\papi{ ɲɟɤrɲɟɤr}}}\markboth{ɲɟɤrɲɟɤr}{}
\classe{idph.2}
\begin{définition}\fra dodu\end{définition}
\begin{définition}\cmn 形容体貌肥嘟嘟,庞大的样子\end{définition}
\begin{exemple}\jya mtshu ɲɟɤrɲɟɤr ʑo ɲɯ-pa\cmn 湖又大又满\end{exemple}
\begin{exemple}\jya ɯʑo ɯ-tɯ-tshu kɯ ɲɟɤrɲɟɤr ʑo ɲɯ-pa\cmn 他胖得满身都是肉\end{exemple}
\begin{exemple}\jya jla ɲɟɤrɲɟɤr ɲɯ-rɤʑi\cmn 犏牛身躯庞大地站在那里\end{exemple}
\begin{exemple}\jya paʁ ɲɯ-tshu ɲɟɤrɲɟɤr ʑo\cmn 猪肥嘟嘟的\end{exemple}
\begin{exemple}\jya tɤɕi tɯ-fkur ʑo ɲɟɤrɲɟɤr to-rku\cmn 青稞袋子装得很满(大)\end{exemple}\begin{sous-entrée}
\vedette{\hypertarget{}{\papi{ mɤlɤɲɟɤr}}}\markboth{mɤlɤɲɟɤr}{}\classe{idph.6}
\begin{définition}\fra énorme\end{définition}
\begin{définition}\cmn 形容高大(搬都搬不动)\end{définition}
\begin{exemple}\jya rŋgɯ mɤlɤɲɟɤr ci ɲɯ-ŋu tɕe kɤ-ɣɤrɤt mɯ́j-sɤcha\cmn 石包很高大,搬也搬不动\end{exemple}
\begin{relation-sémantique}\confer{
\hyperlink{Ⓔɣɤɲɟɤrɲɟɤr}{\textit{ \papi{ɣɤɲɟɤrɲɟɤr}}}
}\end{relation-sémantique}
\end{sous-entrée}\begin{sous-entrée}
\vedette{\hypertarget{}{\papi{ ɲɟɤrnɤɲɟɤr}}}\markboth{ɲɟɤrnɤɲɟɤr}{}\classe{idph.3}
\begin{exemple}\jya kɯ-tshu ci ɲɟɤrnɤɲɟɤr ɲɯ-ŋu ɲɯ-nɤŋkɯŋke\cmn 有个胖子在走来走去\end{exemple}
\begin{exemple}\jya paʁ ɲɟɤrnɤɲɟɤr ʑo ɲɯ-ŋke\cmn 肥嘟嘟的猪在走\end{exemple}
\end{sous-entrée}\end{entrée}

\begin{entrée}
\vedette{\hypertarget{Ⓔɲɟɤt}{\papi{ ɲɟɤt}}}\markboth{ɲɟɤt}{}
\classe{vi}
\paradigme{\textit{dir :} \jya tɤ-}
\begin{définition}\fra regretter\end{définition}
\begin{définition}\cmn 后悔
\begin{déclaration} \étymologie{\papi{ⁿgʲod}}\end{déclaration}\end{définition}
\begin{exemple}\jya ɲɯ-ɲɟat-a\cmn 我后悔\end{exemple}
\begin{exemple}\jya tɤ-ɲɟat-a\cmn 我后悔了\end{exemple}
\begin{exemple}\jya to-ɲɟɤt\cmn 他后悔了\end{exemple}
\begin{exemple}\jya jiɕqha tɤ-tɯt-a nɯ ɲɯ-ɲɟat-a\cmn 我后悔刚才讲的话\end{exemple}\end{entrée}

\begin{entrée}
\vedette{\hypertarget{Ⓔɲɟo}{\papi{ ɲɟo}}}\markboth{ɲɟo}{}\classe{vi}
\paradigme{\textit{dir :} \jya pɯ-}
\begin{définition}\fra subir des dommages\end{définition}
\begin{définition}\cmn 受损;受伤;受灾\end{définition}
\begin{exemple}\jya ɯʑo pɯ-ɲɟo\cmn 他遇到灾难了\end{exemple}
\begin{exemple}\jya nɤʑo pɯ-tɯ-ɲɟo\cmn 你遇到灾难了\end{exemple}
\begin{exemple}\jya laχtɕha pjɤ-ɲɟo\cmn 东西损坏了\end{exemple}
\begin{exemple}\jya ɯ-mi pjɤ-ɲɟo\cmn 他的脚坏了\end{exemple}
\begin{exemple}\jya pɯ-ndʐaβ-a tɕe pɯ-ɲɟo-a\cmn 他摔跤了就受伤了\end{exemple}
\begin{exemple}\jya kɯ-ɲɟo a-pɯme tɕe mɤ-kɯ-pe me\cmn 只要没有损失就没有什么不好的\end{exemple}
\begin{exemple}\jya ɣɯjpa tɯ-xpa kɯ-ɲɟo me\cmn 今年一年都没有损失\end{exemple}
\begin{exemple}\jya pɯ-kɯ-ɲɟo me-a\cmn 我没有受到损伤\end{exemple}
\begin{exemple}\jya ɲɟɯ-ɲɟo ʑo pɯ-ɲɟo\cmn 他遭到很多灾难\end{exemple}\begin{sous-entrée}
\vedette{\hypertarget{}{\papi{ sɯɣɲɟo}}}\markboth{sɯɣɲɟo}{}\classe{vt}
\paradigme{\textit{dir :} \jya pɯ-}
\begin{définition}\ 
\begin{déclaration}\grammar{caus}\end{déclaration}\end{définition}
\begin{définition}\fra abîmer\end{définition}
\begin{définition}\cmn 弄坏\end{définition}
\begin{exemple}\jya @tuolaji pjɤ-sɯɣɲɟo\cmn 他把拖拉机弄坏了\end{exemple}
\begin{exemple}\jya pɯ-sɯɣɲɟo-t-a\cmn 我弄坏了\end{exemple}
\end{sous-entrée}\end{entrée}

\begin{entrée}
\vedette{\hypertarget{Ⓔɲɟoʁ}{\papi{ ɲɟoʁ}}}\markboth{ɲɟoʁ}{}\classe{vt}
\paradigme{\textit{dir :} \jya kɤ-}
\paradigme{\textit{dir :} \jya nɯ-}
\paradigme{\textit{dir :} \jya pɯ-}
\begin{définition}\fra coller\end{définition}
\begin{définition}\cmn 贴\end{définition}
\begin{exemple}\jya nɯ-ɲɟoʁ-a, pɯ-ɲɟoʁ-a\cmn 我贴了\end{exemple}
\begin{exemple}\jya tɕetu ki kɤ-ɲɟoʁ-a\cmn 我把这个东西贴在上面了\end{exemple}
\begin{exemple}\jya znde ɯ-taʁ ko-ɲɟoʁ\cmn 他贴在墙上了\end{exemple}\begin{sous-entrée}
\vedette{\hypertarget{}{\papi{ aɲɟoʁ}}}\markboth{aɲɟoʁ}{}\classe{vi}
\begin{définition}\ 
\begin{déclaration}\grammar{pass}\end{déclaration}\end{définition}
\begin{définition}\fra être collé\end{définition}
\begin{définition}\cmn 贴着\end{définition}
\end{sous-entrée}\begin{sous-entrée}
\vedette{\hypertarget{}{\papi{ ʑɣɤsɤɲɟoʁ}}}\markboth{ʑɣɤsɤɲɟoʁ}{}\classe{vi}
\paradigme{\textit{dir :} \jya kɤ-}
\paradigme{\textit{dir :} \jya pɯ-}
\paradigme{\textit{dir :} \ }
\begin{définition}\ 
\begin{déclaration}\grammar{refl}\end{déclaration}
\begin{déclaration}\grammar{caus}\end{déclaration}\end{définition}
\begin{définition}\fra se coller sur\end{définition}
\begin{définition}\cmn 把(自己的)身体贴在\end{définition}
\begin{exemple}\jya kɤ-anbaʁ-a tɕe znde ɯ-taʁ kɤ-ʑɣɤsɤɲɟoʁ-a\cmn 我躲起来的时候把身体贴在墙上了\end{exemple}
\end{sous-entrée}\end{entrée}

\begin{entrée}
\vedette{\hypertarget{Ⓔɲɟɯ}{\papi{ ɲɟɯ}}}\markboth{ɲɟɯ}{}\classe{vi}
\paradigme{\textit{dir :} \jya kɤ-}
\paradigme{\textit{dir :} \jya tɤ-}
\begin{définition}\ 
\begin{déclaration}\grammar{acaus}\end{déclaration}\end{définition}
\begin{définition}\fra être ouvert, s'ouvrir\end{définition}
\begin{définition}\cmn 开着;自动打开\end{définition}
\begin{exemple}\jya qale ta-βzu tɕe, kɯm kɤ-ɲɟɯ\cmn 风一吹,门就开了\end{exemple}
\begin{exemple}\jya tʂu to-ɲɟɯ\cmn 路开了\end{exemple}
\begin{exemple}\jya rgɤm tɤ-ɲɟɯ\cmn 盒子开了\end{exemple}
\begin{exemple}\jya kɯm a-pɯ-nɯ-ɲɟɯ je!\cmn 让门开着\end{exemple}
\begin{relation-sémantique}\confer{
\hyperlink{ⒺcɯⒽ1}{\textit{ \papi{cɯ1}}}
}\end{relation-sémantique}
\begin{relation-sémantique}\confer{
\hyperlink{Ⓔɯ-ɣɲɟɯ}{\textit{ \papi{ɯ-ɣɲɟɯ}}}
}\end{relation-sémantique}\end{entrée}

\begin{entrée}
\vedette{\hypertarget{Ⓔɲɟɯɣ}{\papi{ ɲɟɯɣ}}}\markboth{ɲɟɯɣ}{}\classe{vi}
\paradigme{\textit{dir :} \jya tɤ-}
\begin{définition}\fra s'entendre\end{définition}
\begin{définition}\cmn 合得来\end{définition}
\begin{exemple}\jya ɲɯ-ɲɟɯɣ-ndʑi\cmn 他们合得来\end{exemple}
\begin{exemple}\jya ki ɯ-ʁɤri mɯ-pɯ-amɯmi-ndʑi ri, tham to-ɲɟɯɣ-ndʑi\cmn 他们以前关系不好,现在合得来了\end{exemple}
\begin{relation-sémantique}\synonyme{
\hyperlink{ⒺamɯmiⒽ1}{\textit{ \papi{amɯmi}}}
}\end{relation-sémantique}\end{entrée}

\begin{entrée}
\vedette{\hypertarget{Ⓔɲɟɯr}{\papi{ ɲɟɯr}}}\markboth{ɲɟɯr}{}\classe{vi}
\paradigme{\textit{dir :} \jya nɯ-}
\paradigme{\textit{dir :} \jya tɤ-}
\begin{définition}\fra changer\end{définition}
\begin{définition}\cmn 变化
\begin{déclaration} \étymologie{\papi{ⁿgʲur}}\end{déclaration}\end{définition}
\begin{exemple}\jya kutɕu kɤ-tɯ-nɤrʑaʁ tɕe nɤ-skɤt ɲɤ-ɲɟɯr\cmn 你在这里待久了,你的语言就变了\end{exemple}\end{entrée}

\begin{entrée}
\vedette{\hypertarget{Ⓔɲɟɯrmbloʁ}{\papi{ ɲɟɯrmbloʁ}}}\markboth{ɲɟɯrmbloʁ}{} (\variante{nɤɲɟɯrmbloʁ}) 
\paradigme{\textit{dir :} \jya tɤ-}
\begin{définition}\fra être instable\end{définition}
\begin{définition}\cmn 动摇不定;不稳定\end{définition}
\begin{exemple}\jya ɯʑo kɯ-nɤɲɟɯrmbloʁ ci ɲɯ-ɕti tɕe, tɤ-mda tɕe tɕhi ɲɯ-fse mɤxsi\cmn 他是个动摇不定的人,不知道最好会怎么样\end{exemple}
\begin{exemple}\jya a-mɤ-tɯ-nɤɲɟɯrmbloʁ je!\cmn 你不要动摇不定\end{exemple}
\begin{relation-sémantique}\antonyme{
\hyperlink{Ⓔarɤstoʁsta}{\textit{ \papi{arɤstoʁsta}}}
}\end{relation-sémantique}
\begin{relation-sémantique}\confer{
\hyperlink{Ⓔɲɟɯr}{\textit{ \papi{ɲɟɯr}}}
}\end{relation-sémantique}\classe{vi}\end{entrée}

\begin{entrée}
\vedette{\hypertarget{Ⓔɲɟɯrnor}{\papi{ ɲɟɯrnor}}}\markboth{ɲɟɯrnor}{}\classe{n}
\begin{définition}\fra erreur\end{définition}
\begin{définition}\cmn 错觉\end{définition}
\begin{relation-sémantique}\confer{
\hyperlink{Ⓔsɤɲɟɯrnor}{\textit{ \papi{sɤɲɟɯrnor}}}
}\end{relation-sémantique}\end{entrée}

\begin{entrée}
\vedette{\hypertarget{Ⓔɲo}{\papi{ ɲo}}}\markboth{ɲo}{}
\classe{vs}
\paradigme{\textit{dir :} \jya tɤ-}
\begin{définition}\fra déjà préparé\end{définition}
\begin{définition}\cmn 现成\end{définition}
\begin{exemple}\jya ji-saχsɯ to-ɲo\cmn 我们的午餐已经准备好了\end{exemple}
\begin{exemple}\jya tɤ-pɤri to-ɲo\cmn 晚餐已经做好了\end{exemple}
\begin{exemple}\jya kɯ-ɲo-ndza\cmn 现成的食物\end{exemple}
\begin{exemple}\jya kɯ-ɲo-ŋga\cmn 现成的衣服\end{exemple}
\begin{relation-sémantique}\confer{
\hyperlink{Ⓔsɯɣɲo}{\textit{ \papi{sɯɣɲo}}}
}\end{relation-sémantique}
\begin{relation-sémantique}\confer{
\hyperlink{ⒺmɲoⒽ1}{\textit{ \papi{mɲo1}}}
}\end{relation-sémantique}
\begin{relation-sémantique}\synonyme{
\hyperlink{Ⓔndzu}{\textit{ \papi{ndzu}}}
}\end{relation-sémantique}\begin{sous-entrée}
\vedette{\hypertarget{}{\papi{ nɯɣɲo}}}\markboth{nɯɣɲo}{}\classe{vt}
\begin{définition}\fra être certain que\end{définition}
\begin{définition}\cmn 觉得……一定会……
\begin{déclaration}\grammar{appl}\end{déclaration}\end{définition}
\begin{exemple}\jya aʑo kɤ-βʁa nɯ kɤ-nɯɣɲo ɕti\cmn 我注定会赢\end{exemple}
\begin{exemple}\jya aʑo kɤ-βʁa nɯ kú-wɣ-nɯɣɲo-a-nɯ ɕti\cmn 他们觉得我一定会赢\end{exemple}
\end{sous-entrée}\end{entrée}

\begin{entrée}
\vedette{\hypertarget{Ⓔɲur}{\papi{ ɲur}}}\markboth{ɲur}{}\classe{vs}
\paradigme{\textit{dir :} \jya nɯ-}
\begin{définition}\fra être fatigué\end{définition}
\begin{définition}\cmn 困倦\end{définition}\end{entrée}

\begin{entrée}
\vedette{\hypertarget{Ⓔɲɯɣɲɯɣ}{\papi{ ɲɯɣɲɯɣ}}}\markboth{ɲɯɣɲɯɣ}{}
\classe{idph.2}
\begin{définition}\fra objets granulaires entassés\end{définition}
\begin{définition}\cmn 形容粉状的东西堆得很高、很软的样子\end{définition}
\begin{exemple}\jya qajβɣi ɲɯɣɲɯɣ ʑo ɲɯ-ɤrmbɯ\cmn 糠秕堆得很高\end{exemple}
\begin{exemple}\jya ɲɯɣɲɯɣ ʑo ɲɯ-ɤta\cmn 堆得很高\end{exemple}
\begin{exemple}\jya qambɯt (tɤ-rɤku, tɯ-ɣli) ɲɯɣɲɯɣ ʑo ɲɯ-ɤrmbɯ\cmn 沙子(庄稼、肥料)堆得很高\end{exemple}\begin{sous-entrée}
\vedette{\hypertarget{}{\papi{ ɲɯɣnɤɲɯɣ}}}\markboth{ɲɯɣnɤɲɯɣ}{}\classe{idph.3}
\begin{définition}\fra (se déplacer sur) un sol poudreux et mou\end{définition}
\begin{définition}\cmn 形容在泥沙里走动时很不方便(地很软)的样子\end{définition}
\begin{exemple}\jya nɯ ɯ-taʁ tu-kɯ-ŋke tɕe, ɲɯɣnɤɲɯɣ ʑo ɲɯ-ti\cmn 在那上面走动很不方便\end{exemple}
\begin{relation-sémantique}\confer{
\hyperlink{Ⓔlɲɯɣlɲɯɣ}{\textit{ \papi{lɲɯɣlɲɯɣ}}}
}\end{relation-sémantique}
\end{sous-entrée}\end{entrée}

\newpage\caractère{ŋ}

\begin{entrée}
\vedette{\hypertarget{Ⓔŋu}{\papi{ ŋu}}}\markboth{ŋu}{}\classe{vs}
\paradigme{\textit{dir :} \jya tɤ-}
\begin{définition}\fra être\end{définition}
\begin{définition}\cmn 是\end{définition}
\begin{sous-entrée}
\vedette{\hypertarget{}{\papi{ ŋu nɤ}}}\markboth{ŋu nɤ}{}
\begin{définition}\fra en ce qui concerne, à propos\end{définition}
\begin{définition}\cmn 至于,关于\end{définition}
\begin{exemple}\jya aʑo ŋu nɤ, mɯ-tu-kɯ-nɯkon-a-nɯ ɲɯ-ŋu\cmn 我呢,你们都不关心我\end{exemple}
\end{sous-entrée}\begin{sous-entrée}
\vedette{\hypertarget{}{\papi{ pɯpɯŋu nɤ}}}\markboth{pɯpɯŋu nɤ}{}
\begin{définition}\fra en ce qui concerne, à propos\end{définition}
\begin{définition}\cmn 至于,关于\end{définition}
\end{sous-entrée}\begin{sous-entrée}
\vedette{\hypertarget{}{\papi{ wuma tɤ-ŋu tɕe}}}\markboth{wuma tɤ-ŋu tɕe}{}
\begin{définition}\fra en réalité\end{définition}
\begin{définition}\cmn 实际上\end{définition}
\end{sous-entrée}\end{entrée}

\begin{entrée}
\vedette{\hypertarget{Ⓔŋa}{\papi{ ŋa}}}\markboth{ŋa}{}
\classe{vt}
\paradigme{\textit{dir :} \jya kɤ-}
\begin{définition}\fra devoir de l'argent, faire un crédit\end{définition}
\begin{définition}\cmn 赊帐\end{définition}
\begin{exemple}\jya kɤ-ŋa-t-a\cmn 我欠了钱\end{exemple}
\begin{exemple}\jya nɤ-rŋɯl kɤ-ŋa-t-a, ɯ-qhu tɕe ɲɯ-kham-a\cmn 我欠了你钱,以后会还\end{exemple}
\begin{exemple}\jya ɣnɤsqi ɲɯ-ra, sqɯ-mpɕar nɯ-kho-t-a, nɯma mɯ́j-rtaʁ tɕe sqɯ-mpɕar kɤ-ŋa-t-a\cmn 原来需要二十块钱,我给了十块然后钱不够,我就赊了十块钱\end{exemple}
\begin{exemple}\jya kɯ-rɤχtɯ jɤ-ari-a ri, a-rŋɯl mɯ́j-rtaʁ tɕe kɤ-ŋa-t-a, tɕe a-nŋa nɯ-ɬoʁ\cmn 我去买东西钱不够就赊了账\end{exemple}
\begin{relation-sémantique}\confer{
\hyperlink{Ⓔrɤnŋa}{\textit{ \papi{rɤnŋa}}}
}\end{relation-sémantique}
\begin{relation-sémantique}\confer{
\hyperlink{Ⓔtɯ-nŋa}{\textit{ \papi{tɯ-nŋa}}}
}\end{relation-sémantique}\end{entrée}

\begin{entrée}
\vedette{\hypertarget{Ⓔŋɤɣlitɕaʁmbɯm}{\papi{ ŋɤɣlitɕaʁmbɯm}}}\markboth{ŋɤɣlitɕaʁmbɯm}{}
\classe{n}
\begin{définition}\fra bousier\end{définition}
\begin{définition}\cmn 蜣螂【牛屎虫】\end{définition}\end{entrée}

\begin{entrée}
\vedette{\hypertarget{Ⓔŋɤjɕtsa}{\papi{ ŋɤjɕtsa}}}\markboth{ŋɤjɕtsa}{}\classe{n}
\begin{définition}\fra chef de village\end{définition}
\begin{définition}\cmn 村长\end{définition}\end{entrée}

\begin{entrée}
\vedette{\hypertarget{Ⓔŋɤn}{\papi{ ŋɤn}}}\markboth{ŋɤn}{}\classe{vi}
\paradigme{\textit{dir :} \jya nɯ-}\acception{1}
\begin{définition}\fra mauvais\end{définition}
\begin{définition}\cmn 坏\end{définition}\acception{2}
\begin{définition}\fra féroce\end{définition}
\begin{définition}\cmn 凶
\begin{déclaration} \étymologie{\papi{ŋan}}\end{déclaration}\end{définition}
\begin{sous-entrée}
\vedette{\hypertarget{}{\papi{ ɣɤŋɤn}}}\markboth{ɣɤŋɤn}{}\classe{vt}
\paradigme{\textit{dir :} \jya tɤ-}
\begin{définition}\ 
\end{définition}
\begin{exemple}\jya ɯ-sɯm to-ɣɤŋɤn tɕe ɯʑo ɯ-mɤ-kɯ-pe to-nɯ-βzu\cmn 他起了坏心,反而自食其果\end{exemple}
\end{sous-entrée}\end{entrée}

\begin{entrée}
\vedette{\hypertarget{Ⓔŋɤnŋɤt}{\papi{ ŋɤnŋɤt}}}\markboth{ŋɤnŋɤt}{}\classe{idph.2}
\begin{définition}\fra la tête vers le bas\end{définition}
\begin{définition}\cmn 形容低头的样子\end{définition}
\begin{exemple}\jya pjɤ-nɯʑɯβ tɕe ɯ-ku ŋɤnŋɤt ʑo pjɤ-phɤβ\cmn 他低着头睡着了\end{exemple}
\begin{relation-sémantique}\antonyme{
\hyperlink{Ⓔŋɤrŋɤr}{\textit{ \papi{ŋɤrŋɤr}}}
}\end{relation-sémantique}\end{entrée}

\begin{entrée}
\vedette{\hypertarget{Ⓔŋɤnɯ}{\papi{ ŋɤnɯ}}}\markboth{ŋɤnɯ}{}\classe{n}
\begin{définition}\fra pis de la vache\end{définition}
\begin{définition}\cmn 奶牛的乳房\end{définition}
\end{entrée}

\begin{entrée}
\vedette{\hypertarget{Ⓔŋɤnɯkɯmtsɯɣ}{\papi{ ŋɤnɯkɯmtsɯɣ}}}\markboth{ŋɤnɯkɯmtsɯɣ}{}
\classe{n}
\begin{définition}\fra une espèce d'arbrisseau\end{définition}
\begin{définition}\cmn 【红青椒】\end{définition}
\begin{exemple}\jya ŋɤnɯ kɯmtsɯɣ nɯ ruŋgu sɯŋgɯ arɤndɯndɤt ʑo tu-ɬoʁ ɕti, ɯ-zrɤm wuma ʑo wxti, kɯ-ɣɯrni ŋu. smɤnrɯɣ ci ɲɯ-ŋu. ɯ-ru nɯ kɯ-ɤβʑɯrdu ŋu, ɯ-jwaʁ kɯ-ɤβzɯrχsɯm ŋu, kɯ-pɣi tsa ŋu, ɯ-qhu nɯ ɯ-rme kɯ-fse tu, ɯ-ru kɯ-zri tsa tu-ɬoʁ tɕe, ɯ-rtaʁ ɲɯ-ɬoʁ ŋu, ɯ-mɯntoʁ nɯ ɯ-ru ɯ-taʁ lu-oʑɯrja ŋu, kɯ-ɣɯrni tsa ŋu. ɣʑo wuma ʑo rga.\cmn 红青椒到处都可以生长,包括草坪里和森林里。根长得很大,是红色的,是一种药材。茎是四方形的,叶子是三角形的,呈灰色,背面有细毛。茎长了以后就会分杈,花排列在茎上,是淡红色的。蜜蜂喜欢这种植物。\end{exemple}\end{entrée}

\begin{entrée}
\vedette{\hypertarget{Ⓔŋɤqa}{\papi{ ŋɤqa}}}\markboth{ŋɤqa}{}
\classe{n}
\begin{définition}\fra une espèce de champignon\end{définition}
\begin{définition}\cmn 一种蘑菇\end{définition}
\begin{exemple}\jya ŋɤqa nɯ stɤmku ri tu-ɬoʁ ŋu tɕe, ɯ-mgɯr ɯ-qhu nɯ kɯ-wɣrum ŋu, ɯ-rʑɯɣ nɯ kɯ-ɤɣɯrnɯɕɯr ŋu, ɯ-ru nɯ li kɯ-wɣrum ŋu, kɤ-ndza mɯm, phaʁzla ɯ-ŋgɯ tu-ɬoʁ ŋu tɕe phazla ŋɤqa kɤ-ti tu.\cmn 
\stylefv{ŋɤqa} 长在草地上,背面和干是白色的,菌褶带有红色,好吃。长在五月份,所以叫“五月\stylefv{ŋɤqa}”。
\end{exemple}\end{entrée}

\begin{entrée}
\vedette{\hypertarget{Ⓔŋɤqe}{\papi{ ŋɤqe}}}\markboth{ŋɤqe}{}\classe{n}
\begin{définition}\fra bouse de vache\end{définition}
\begin{définition}\cmn 牛粪\end{définition}
\end{entrée}

\begin{entrée}
\vedette{\hypertarget{Ⓔŋɤrŋɤr}{\papi{ ŋɤrŋɤr}}}\markboth{ŋɤrŋɤr}{}\classe{idph.2}
\begin{définition}\fra qui tend le cou\end{définition}
\begin{définition}\cmn 形容伸脖子的样子\end{définition}
\begin{exemple}\jya ɯ-ku ŋɤrŋɤr ʑo to-joʁ\cmn 他伸了脖子(看)\end{exemple}
\begin{exemple}\jya ɯ-ku ŋɤrŋɤr ʑo chɤ-tɕɤt\cmn 他把头(从窗户)探出来伸着脖子看\end{exemple}\begin{sous-entrée}
\vedette{\hypertarget{}{\papi{ ŋɤrnɤŋɤr}}}\markboth{ŋɤrnɤŋɤr}{}\classe{idph.2}
\begin{relation-sémantique}\antonyme{
\hyperlink{Ⓔŋɤnŋɤt}{\textit{ \papi{ŋɤnŋɤt}}}
}\end{relation-sémantique}
\end{sous-entrée}\end{entrée}

\begin{entrée}
\vedette{\hypertarget{Ⓔŋɤtɕɯkɤti,khɯ}{\papi{ ŋɤtɕɯkɤti,khɯ}}}\markboth{ŋɤtɕɯkɤti,khɯ}{}
\paradigme{\textit{dir :} \jya tɤ-}
\begin{définition}\fra obéir en tous points\end{définition}
\begin{définition}\cmn 完全顺从\end{définition}
\begin{exemple}\jya ŋɤtɕɯkɤti a-tɤ-tɯ-khɯ ra\cmn 你要完全顺从吩咐\end{exemple}
\begin{exemple}\jya ŋɤtɕɯkɤti tu-sɯkhi-a ra\cmn 我要令他完全顺从\end{exemple}
\begin{relation-sémantique}\ComponentA{\classe{n}
 \papi{ŋɤtɕɯkɤti}
}\end{relation-sémantique}
\begin{relation-sémantique}\ComponentB{\classe{vs}
\hyperlink{ⒺkhɯⒽ1}{\textit{ \papi{khɯ}}}
}\end{relation-sémantique}\end{entrée}

\begin{entrée}
\vedette{\hypertarget{Ⓔŋga}{\papi{ ŋga}}}\markboth{ŋga}{}\classe{vt}
\paradigme{\textit{dir :} \jya tɤ-}
\begin{définition}\fra mettre (un vêtement)\end{définition}
\begin{définition}\cmn 穿衣服
\begin{déclaration}\use{穿衣服、穿鞋子、戴帽子都可以用\stylefv{ŋga},戴帽子可以说\stylefv{nɤrte}。戴眼镜要用\stylefv{ta},戴手表用\stylefv{rku}}\end{déclaration}\end{définition}
\begin{exemple}\jya tɤ-rte tɤ-ŋge\cmn 你戴上帽子\end{exemple}
\begin{exemple}\jya tɯ-ŋga pɯ-ŋga-t-a\cmn (睡觉的时候)我把衣服盖在身上了\end{exemple}\begin{sous-entrée}
\vedette{\hypertarget{}{\papi{ nɤŋgɯŋga}}}\markboth{nɤŋgɯŋga}{}\classe{vt}
\paradigme{\textit{dir :} \jya tɤ-}
\begin{définition}\fra mettre n'importe quel habit\end{définition}
\begin{définition}\cmn 随便穿\end{définition}
\begin{exemple}\jya tɯrme ɯ-ŋga ra ma-tɯ-nɤŋgɯŋge ma nɤʑɯɣ tɤ-nɯ-ŋge\cmn 不要随便穿别人的衣服,穿自己的\end{exemple}
\begin{relation-sémantique}\confer{
\hyperlink{Ⓔʑŋga}{\textit{ \papi{ʑŋga}}}
}\end{relation-sémantique}
\end{sous-entrée}\begin{sous-entrée}
\vedette{\hypertarget{}{\papi{ nɯɣɯŋga}}}\markboth{nɯɣɯŋga}{}\classe{vs}
\begin{définition}\ 
\begin{déclaration}\grammar{facil}\end{déclaration}\end{définition}
\begin{définition}\fra facile, agréable à mettre (habit)\end{définition}
\begin{définition}\cmn 容易穿;穿着舒服\end{définition}
\begin{exemple}\jya kɯki a-ŋga ki wuma ʑo ɲɯ-nɯɣɯŋga\cmn 我这件衣服穿着很舒服\end{exemple}
\begin{exemple}\jya nɤki nɯ ɯ-ɲɯ́-nɯɣɯŋga?\cmn 你那件好穿吗?\end{exemple}
\end{sous-entrée}\end{entrée}

\begin{entrée}
\vedette{\hypertarget{Ⓔŋgɤɣ}{\papi{ ŋgɤɣ}}}\markboth{ŋgɤɣ}{}
\classe{vi.nh}
\paradigme{\textit{dir :} \jya \_}
\begin{définition}\ 
\begin{déclaration}\grammar{acaus}\end{déclaration}\end{définition}
\begin{définition}\fra courbé\end{définition}
\begin{définition}\cmn 弯(树)\end{définition}
\begin{exemple}\jya tɯɲcɣa chɯ-ŋgɤɣ ɲɯ-ŋu\cmn 镰刀是弯的\end{exemple}
\begin{exemple}\jya ɕɤmiŋoʁ chɯ-ŋgɤɣ ɲɯ-ŋu\cmn 铁钩是弯的\end{exemple}
\begin{exemple}\jya paχɕi ɲɤ-ɣɤmat tɕe, ɯ-rtaʁ ra pjɤ-ŋgɤɣ ʑo\cmn 因为苹果熟了,导致树枝弯下来了\end{exemple}
\begin{relation-sémantique}\confer{
\hyperlink{Ⓔkɤɣ}{\textit{ \papi{kɤɣ}}}
}\end{relation-sémantique}\end{entrée}

\begin{entrée}
\vedette{\hypertarget{Ⓔŋgɤjpɤn}{\papi{ ŋgɤjpɤn}}}\markboth{ŋgɤjpɤn}{}\classe{n}
\begin{définition}\fra planche de bois\end{définition}
\begin{définition}\cmn 木板\end{définition}
\begin{exemple}\jya ŋgɤjpɤn nɯ rɟaŋsoʁ kɯ thɯ-kɤ-sɯ-phaʁ ŋu. tɤrɤm nɯ tɯ-rpa kɯ ɯ-rɯmu thɯ-kɤ-z-nɯɴqhu tɕe thɯ-kɤ-phaʁ ŋu.\cmn 
\stylefv{ŋgɤjpɤn}是用锯子锯成的板子,\stylefv{tɤrɤm}是用斧头顺着纹路劈成的。
\end{exemple}
\end{entrée}

\begin{entrée}
\vedette{\hypertarget{Ⓔŋgɤlɤʁɟa}{\papi{ ŋgɤlɤʁɟa}}}\markboth{ŋgɤlɤʁɟa}{}
\classe{n}
\begin{définition}\fra chauve\end{définition}
\begin{définition}\cmn 秃子\end{définition}\end{entrée}

\begin{entrée}
\vedette{\hypertarget{Ⓔŋgɤm}{\papi{ ŋgɤm}}}\markboth{ŋgɤm}{}
\classe{n}
\begin{définition}\fra pente de terre (à 90 degrés)\end{définition}
\begin{définition}\cmn (垂直)的土坡【土崖】\end{définition}\end{entrée}

\begin{entrée}
\vedette{\hypertarget{Ⓔŋgɤr}{\papi{ ŋgɤr}}}\markboth{ŋgɤr}{}
\classe{vs}
\paradigme{\textit{dir :} \jya kɤ-}
\begin{définition}\fra étroit\end{définition}
\begin{définition}\cmn 狭窄
\begin{déclaration}\use{ɯ-ro ŋgɤr}\end{déclaration}
\begin{déclaration}\use{小气}\end{déclaration}\end{définition}
\begin{exemple}\jya kɯspoʁ ɲɯ-ŋgɤr\cmn 洞很狭窄\end{exemple}
\begin{exemple}\jya sɤxɕe ɲɯ-ŋgɤr\cmn 去的地方很狭窄\end{exemple}
\begin{exemple}\jya nɤ-ro ɯ-tɯ-ŋgɤr\cmn 你很小气\end{exemple}
\begin{relation-sémantique}\antonyme{
\hyperlink{Ⓔjom}{\textit{ \papi{jom}}}
}\end{relation-sémantique}
\begin{relation-sémantique}\confer{
\hyperlink{Ⓔaŋgɤrŋgɤr}{\textit{ \papi{aŋgɤrŋgɤr}}}
}\end{relation-sémantique}
\begin{relation-sémantique}\confer{
 \papi{znɤngɤr}
}\end{relation-sémantique}\begin{sous-entrée}
\vedette{\hypertarget{}{\papi{ nɤŋgɤr}}}\markboth{nɤŋgɤr}{}
\begin{définition}\ 
\begin{déclaration}\grammar{trop}\end{déclaration}\end{définition}
\begin{définition}\fra trouver trop étroit\end{définition}
\begin{définition}\cmn 觉得狭窄\end{définition}
\begin{relation-sémantique}\confer{
 \papi{vt}
}\end{relation-sémantique}
\end{sous-entrée}\end{entrée}

\begin{entrée}
\vedette{\hypertarget{Ⓔŋgɤrom}{\papi{ ŋgɤrom}}}\markboth{ŋgɤrom}{}
\classe{n}
\begin{définition}\fra méthode de tissage\end{définition}
\begin{définition}\cmn 织布的方法,两根线交错着【单巴子】\end{définition}\end{entrée}

\begin{entrée}
\vedette{\hypertarget{Ⓔŋgio}{\papi{ ŋgio}}}\markboth{ŋgio}{}
\classe{vi}
\paradigme{\textit{dir :} \jya pɯ-}
\begin{définition}\ 
\begin{déclaration}\grammar{acaus}\end{déclaration}\end{définition}
\begin{définition}\fra glisser\end{définition}
\begin{définition}\cmn (从高处一直)滑下来\end{définition}
\begin{exemple}\jya pɯ-ŋgio\cmn 他滑下来了\end{exemple}
\begin{exemple}\jya pɯ-ŋgio-a\cmn 我滑下来了\end{exemple}
\begin{exemple}\jya ɕoŋtɕa pɯ-ŋgio\cmn 木料滑下来了\end{exemple}
\begin{exemple}\jya rdɤstaʁ pjɤ-ŋgio\cmn 石头滑下来了\end{exemple}\begin{sous-entrée}
\vedette{\hypertarget{}{\papi{ nɤŋgiolo}}}\markboth{nɤŋgiolo}{}\classe{vi}
\begin{définition}\ 
\begin{déclaration}\grammar{n.orient}\end{déclaration}\end{définition}
\begin{définition}\fra glisser dans tous les sens\end{définition}
\begin{définition}\cmn 滑来滑去\end{définition}
\begin{relation-sémantique}\confer{
\hyperlink{Ⓔkio}{\textit{ \papi{kio}}}
}\end{relation-sémantique}
\begin{relation-sémantique}\synonyme{
\hyperlink{Ⓔaʁdɤt}{\textit{ \papi{aʁdɤt}}}
}\end{relation-sémantique}
\end{sous-entrée}\end{entrée}

\begin{entrée}
\vedette{\hypertarget{Ⓔŋgumdʑɯɣpa}{\papi{ ŋgumdʑɯɣpa}}}\markboth{ŋgumdʑɯɣpa}{}
\begin{relation-sémantique}\confer{
\hyperlink{Ⓔŋgumdʑɯɣpa}{\textit{ \papi{ŋgumdʑɯɣpa}}}
}\end{relation-sémantique}\end{entrée}

\begin{entrée}
\vedette{\hypertarget{Ⓔŋgo}{\papi{ ŋgo}}}\markboth{ŋgo}{}\classe{vi}
\paradigme{\textit{dir :} \jya tɤ-}
\begin{définition}\fra cuire (les momos)\end{définition}
\begin{définition}\cmn 烤熟(馍馍)\end{définition}
\begin{exemple}\jya qajɣi to-ŋgo\cmn 馍馍烤熟了\end{exemple}\begin{sous-entrée}
\vedette{\hypertarget{}{\papi{ sɯŋgo}}}\markboth{sɯŋgo}{}\classe{vt}
\begin{exemple}\jya qajɣi tɤ-sɯŋgo-t-a\cmn 我把馍馍烤熟了\end{exemple}
\end{sous-entrée}\end{entrée}

\begin{entrée}
\vedette{\hypertarget{Ⓔŋgoŋpu}{\papi{ ŋgoŋpu}}}\markboth{ŋgoŋpu}{}\classe{n}
\begin{définition}\fra désastre\end{définition}
\begin{définition}\cmn 祸事
\begin{déclaration} \étymologie{\papi{ⁿgoŋ.po}}\end{déclaration}\end{définition}\end{entrée}

\begin{entrée}
\vedette{\hypertarget{Ⓔŋgorli}{\papi{ ŋgorli}}}\markboth{ŋgorli}{}
\classe{n}
\begin{définition}\fra bovidé sans corne\end{définition}
\begin{définition}\cmn 无角牛\end{définition}\end{entrée}

\begin{entrée}
\vedette{\hypertarget{Ⓔŋgra}{\papi{ ŋgra}}}\markboth{ŋgra}{}
\classe{vi}
\paradigme{\textit{dir :} \jya pɯ-}
\begin{définition}\ 
\begin{déclaration}\grammar{acaus}\end{déclaration}\end{définition}
\begin{définition}\fra tomber\end{définition}
\begin{définition}\cmn 掉下来
\begin{déclaration}\use{指植物的果实、石头等}\end{déclaration}
\begin{déclaration}\use{滑坡只能说\stylefv{tɯ-ɲɤt pjɤ-ɣi},不能用\stylefv{ŋgra}}\end{déclaration}\end{définition}
\begin{exemple}\jya ʑɴɢɯloʁ pjɤ-ŋgra\cmn 核桃掉下来了\end{exemple}
\begin{exemple}\jya sɯmat pjɤ-ŋgra\cmn 水果掉下来了\end{exemple}
\begin{exemple}\jya rdɤstaʁ pjɤ-ŋgra\cmn 石头掉下来了\end{exemple}
\begin{exemple}\jya tɤɕi ɲɯ-ŋgra\cmn 青稞颗粒掉落\end{exemple}
\begin{exemple}\jya qaj ɯ-tɯ-mda thɯ-tɕhom tɕe kɤ́-wɣ-tɣa tɕe pjɯ-ŋgra ɕti\cmn 小麦太熟了,收割的时候掉到地面上\end{exemple}
\begin{relation-sémantique}\confer{
\hyperlink{Ⓔkra}{\textit{ \papi{kra}}}
}\end{relation-sémantique}\end{entrée}

\begin{entrée}
\vedette{\hypertarget{Ⓔŋgrɤl}{\papi{ ŋgrɤl}}}\markboth{ŋgrɤl}{}\classe{vs}
\paradigme{\textit{dir :} \jya tɤ-}
\begin{définition}\fra être habituellement ainsi\end{définition}
\begin{définition}\cmn (平时)是这样\end{définition}
\begin{sous-entrée}
\vedette{\hypertarget{}{\papi{ sɯŋgrɤl}}}\markboth{sɯŋgrɤl}{}\classe{vt}
\begin{définition}\ 
\begin{déclaration}\grammar{caus}\end{déclaration}\end{définition}
\end{sous-entrée}\begin{sous-entrée}
\vedette{\hypertarget{}{\papi{ zrɯŋgrɤl}}}\markboth{zrɯŋgrɤl}{}\classe{vt}
\begin{définition}\ 
\begin{déclaration}\grammar{caus}\end{déclaration}\end{définition}
\begin{définition}\fra changer de\end{définition}
\begin{définition}\cmn 改用\end{définition}
\begin{exemple}\jya kɯɕɯŋgɯ tɕe tɯ-ci kɯ βɣa chɯ-sɯmtɕɯr-i tɕe kɤ-ɣndʑɯr chɯ-sɯ-lɤt-i pɯ-ŋu ri, tham tɕe mkhɯrlu kɯ kɤ-sɯɣndʑɯr nɯ-zrɯŋgrɤl-i\cmn 我以前用水力推磨,现在改用机器磨面\end{exemple}
\end{sous-entrée}\end{entrée}

\begin{entrée}
\vedette{\hypertarget{Ⓔŋgri}{\papi{ ŋgri}}}\markboth{ŋgri}{}
\classe{vs}
\paradigme{\textit{dir :} \jya nɯ-}
\begin{définition}\fra fin (gruau)\end{définition}
\begin{définition}\cmn 稀(粥)\end{définition}
\begin{exemple}\jya tɯtshi ɲɯ-ŋgri\cmn 粥很稀\end{exemple}
\begin{relation-sémantique}\antonyme{
\hyperlink{Ⓔndzɤβ}{\textit{ \papi{ndzɤβ}}}
}\end{relation-sémantique}\end{entrée}

\begin{entrée}
\vedette{\hypertarget{Ⓔŋgro}{\papi{ ŋgro}}}\markboth{ŋgro}{}
\classe{vs}
\paradigme{\textit{dir :} \jya thɯ-}
\begin{définition}\fra puissant, important, honorable\end{définition}
\begin{définition}\cmn 有权有势;有地位;值得尊重的人\end{définition}
\begin{exemple}\jya βlama ɲɯ-ŋgro\cmn 喇嘛很受尊重\end{exemple}
\begin{relation-sémantique}\synonyme{
\hyperlink{Ⓔɣɤʁre}{\textit{ \papi{ɣɤʁre}}}
}\end{relation-sémantique}\end{entrée}

\begin{entrée}
\vedette{\hypertarget{Ⓔŋgurtɕaʁ}{\papi{ ŋgurtɕaʁ}}}\markboth{ŋgurtɕaʁ}{}\classe{n}
\begin{définition}\fra type de pas d'aiguille\end{définition}
\begin{définition}\cmn 缝针的方法\end{définition}
\begin{relation-sémantique}\confer{
\hyperlink{Ⓔnɯŋgurtɕaʁ}{\textit{ \papi{nɯŋgurtɕaʁ}}}
}\end{relation-sémantique}\end{entrée}

\begin{entrée}
\vedette{\hypertarget{Ⓔŋgrɯ}{\papi{ ŋgrɯ}}}\markboth{ŋgrɯ}{}
\classe{vi}
\paradigme{\textit{dir :} \jya pɯ-}
\begin{définition}\fra accomplir\end{définition}
\begin{définition}\cmn 成功;完成
\begin{déclaration} \étymologie{\papi{ⁿgrub}}\end{déclaration}\end{définition}
\begin{exemple}\jya jisŋi ji-sɯphɯt pɯ-ŋgrɯ\cmn 今天我们砍柴的任务完成了\end{exemple}
\begin{exemple}\jya jisŋi ji-ma pɯ-ŋgrɯ\cmn 今天我们的工作完成了\end{exemple}
\begin{exemple}\jya qartsɤβ pɯ-ŋgrɯ\cmn 收割完成了\end{exemple}
\begin{exemple}\jya aʑɯɣ pɯ-ŋgrɯ\cmn 我成功了\end{exemple}
\begin{exemple}\jya aʑo a-kɤ-sɯso nɯ pɯ-ŋgrɯ\cmn 我的愿望实现了\end{exemple}
\begin{exemple}\jya aʑo a-kɤ-nɤma nɯ pɯ-ŋgrɯ\cmn 我的工作完成了\end{exemple}
\begin{relation-sémantique}\synonyme{
\hyperlink{Ⓔngrɯβ}{\textit{ \papi{ngrɯβ}}}
}\end{relation-sémantique}
\begin{relation-sémantique}\confer{
 \papi{sɯŋgrɯ}
}\end{relation-sémantique}\end{entrée}

\begin{entrée}
\vedette{\hypertarget{Ⓔŋguskor}{\papi{ ŋguskor}}}\markboth{ŋguskor}{}
\classe{n}
\begin{définition}\fra fouet\end{définition}
\begin{définition}\cmn 皮鞭(打猪用的)\end{définition}\end{entrée}

\begin{entrée}
\vedette{\hypertarget{Ⓔŋgute}{\papi{ ŋgute}}}\markboth{ŋgute}{}\classe{np}
\begin{définition}\fra qui a une grande tête\end{définition}
\begin{définition}\cmn 大头\end{définition}\end{entrée}

\begin{entrée}
\vedette{\hypertarget{Ⓔŋgɯ}{\papi{ ŋgɯ}}}\markboth{ŋgɯ}{}
\classe{vs}
\paradigme{\textit{dir :} \jya nɯ-}
\begin{définition}\fra pauvre\end{définition}
\begin{définition}\cmn 穷\end{définition}
\begin{exemple}\jya jiɕqha nɯ kɯ-ŋgɯ ci ɲɯ-ŋu\cmn 他是个穷人\end{exemple}\end{entrée}

\begin{entrée}
\vedette{\hypertarget{Ⓔŋgɯŋgri}{\papi{ ŋgɯŋgri}}}\markboth{ŋgɯŋgri}{} (\variante{ŋgringri}) \classe{idph.2}
\begin{définition}\fra long et dur, qui ne casse pas facilement\end{définition}
\begin{définition}\cmn 形容硬而细,不易折断的样子\end{définition}\end{entrée}

\begin{entrée}
\vedette{\hypertarget{ⒺŋgɯrⒽ1}{\papi{ ŋgɯr}}}\markboth{ŋgɯr}{}\homonyme{1}
\classe{n}
\begin{définition}\fra chant mystique\end{définition}
\begin{définition}\cmn 道情
\begin{déclaration} \étymologie{\papi{mgur}}\end{déclaration}\end{définition}
\end{entrée}

\begin{entrée}
\vedette{\hypertarget{ⒺŋgɯrⒽ2}{\papi{ ŋgɯr}}}\markboth{ŋgɯr}{}\homonyme{2}
\classe{idph.2}
\begin{définition}\fra bruit du canon\end{définition}
\begin{définition}\cmn 放炮的声音\end{définition}
\begin{relation-sémantique}\confer{
\hyperlink{Ⓔɣɤŋgɯrŋgɯr}{\textit{ \papi{ɣɤŋgɯrŋgɯr}}}
}\end{relation-sémantique}\end{entrée}

\begin{entrée}
\vedette{\hypertarget{Ⓔŋgɯrŋgɯr}{\papi{ ŋgɯrŋgɯr}}}\markboth{ŋgɯrŋgɯr}{}
\classe{idph.2}
\begin{définition}\fra large et profonde (étendue d'eau)\end{définition}
\begin{définition}\cmn 形容水面很宽,水很深的样子\end{définition}
\begin{exemple}\jya tɯ-ci ŋgɯrŋgɯr ʑo ɲɯ-pa\cmn 水很深\end{exemple}
\begin{exemple}\jya cha tɯ-khɯtsa ʑo ŋgɯrŋgɯr kɤ-tshi-t-a\cmn 我喝了满满的一碗酒\end{exemple}\end{entrée}

\begin{entrée}
\vedette{\hypertarget{Ⓔŋgɯsɯ}{\papi{ ŋgɯsɯ}}}\markboth{ŋgɯsɯ}{}
\classe{n}
\begin{définition}\fra Adenophora sp.\end{définition}
\begin{définition}\cmn 沙参\end{définition}
\begin{exemple}\jya ŋgɯsɯ tʂu ɯ-rkɯ tɯ-ji ɯ-rkɯ zgoku pɕoʁ ra tu-ɬoʁ ŋu, ɯ-qa nɯ smɤn ɲɯ-ŋu khi, ɯ-ŋgɯ nɯ kɯ-wɣrum ŋu, ɯ-pɕi nɯ kɯ-pɣi tsa ŋu, ɯ-zrɤm nɯ mɤ-ndoʁ, ɯ-ru nɯ kɯ-xtshɯm kɯ-zri tsa ŋu, kɯ-qandʐi tsa ŋu, ɯ-jwaʁ nɯ kɯ-ɤrtɯm tɕe kɯ-ndɯβ tsa ŋu, mɤʑɯ tɯ-tɯphu nɯ ɯ-jwaʁ kɯ-tɕɤr tɕe kɯ-ɤmtɕoʁ tsa ŋu, ɯ-mɯntoʁ nɯ tshaŋlaŋ kɯ-fse tɕe ɯ-mdoʁ lɯŋkɤr ŋu tɕɤn ɯ-ru ɯ-taʁ lu-oʑɯrja ŋu.\cmn 
\stylefv{ŋgɯsɯ}生长在地边,路边和山上,听说它的根是药材,根里面是白色的,外面是土灰色的,不脆。茎细长,乌色。一种\stylefv{ŋgɯsɯ}有小而圆的叶子,而另一种有窄而尖的叶子。两种\stylefv{ŋgɯsɯ}的花像铃铛,天蓝色,排列在茎的顶端上。
\end{exemple}\end{entrée}

\begin{entrée}
\vedette{\hypertarget{Ⓔŋkɤɲɟo}{\papi{ ŋkɤɲɟo}}}\markboth{ŋkɤɲɟo}{}\classe{vi}
\paradigme{\textit{dir :} \jya pɯ-}
\begin{définition}\fra passer\end{définition}
\begin{définition}\cmn 路过;来往\end{définition}
\begin{exemple}\jya ɯʑo pɯ-ŋkɤɲɟo\cmn 他路过(这里)\end{exemple}
\end{entrée}

\begin{entrée}
\vedette{\hypertarget{Ⓔŋke}{\papi{ ŋke}}}\markboth{ŋke}{}
\classe{vi}
\paradigme{\textit{dir :} \jya \_}
\begin{définition}\fra marcher\end{définition}
\begin{définition}\cmn 走路\end{définition}
\begin{exemple}\jya kɤ-ŋke kɤ-ari-a\cmn 我走路去了\end{exemple}
\begin{exemple}\jya pɯ-ŋke-a\cmn 我走了一下\end{exemple}
\begin{exemple}\jya ɯʑo kɤ-ŋke kɤ-anɯri\cmn 他走路回去了\end{exemple}
\begin{exemple}\jya aʑo nɯ-nɯ-ŋke-a, mkhɯrlu pɯ-me\cmn 我走路去了,没有车\end{exemple}
\begin{exemple}\jya @yangma tɤ-me tɕe, kɤ-nɯ-ŋke ɬoʁ\cmn 没有自行车的时候只好自己走\end{exemple}
\begin{exemple}\jya nɯɕɯŋgɯ tɕiʑo lɤ-ari-tɕi nɯ tɤ-ŋke-tɕi pɯ-ra ma tham tɕe @qiche ɯ-ŋgu tu-kɯ-ɕe khɯ\cmn 以前我们去的时候必须走路,现在可以坐车\end{exemple}
\begin{exemple}\jya sɤŋke mɯ́j-pe\cmn 不好走\end{exemple}
\begin{exemple}\jya @dian ko-znɯna tɕe @dianti kɯnɤ mɯ́j-ŋke\cmn 停电了,电梯也不走\end{exemple}\begin{sous-entrée}
\vedette{\hypertarget{}{\papi{ ɕɯŋke}}}\markboth{ɕɯŋke}{}\classe{vt}
\paradigme{\textit{dir :} \jya caus}\acception{1}
\begin{définition}\fra faire marcher\end{définition}
\begin{définition}\cmn 使走动\end{définition}\acception{2}
\begin{définition}\fra emporter\end{définition}
\begin{définition}\cmn 带走\end{définition}
\begin{exemple}\jya aʑo tɤ-pɤtso tu-fkur-a tɕe tu-ɕɯŋke-a ŋu\cmn 我背着小孩子走动\end{exemple}
\begin{relation-sémantique}\confer{
\hyperlink{Ⓔnɯɣɯŋke}{\textit{ \papi{nɯɣɯŋke}}}
}\end{relation-sémantique}
\begin{relation-sémantique}\confer{
\hyperlink{Ⓔnɯŋke}{\textit{ \papi{nɯŋke}}}
}\end{relation-sémantique}
\end{sous-entrée}\end{entrée}

\begin{entrée}
\vedette{\hypertarget{Ⓔŋkhor}{\papi{ ŋkhor}}}\markboth{ŋkhor}{}
\classe{vi}
\paradigme{\textit{dir :} \jya \_}
\begin{définition}\fra se rapprocher (animal)\end{définition}
\begin{définition}\cmn 慢慢地接近(动物)
\begin{déclaration} \étymologie{\papi{ⁿkʰor}}\end{déclaration}\end{définition}
\begin{exemple}\jya kɯki khɯna ki ɕɤfɕo aʑo a-phe ku-ŋkhor ɲɯ-ŋu\cmn 这几天这条狗开始接近我了\end{exemple}\end{entrée}

\begin{entrée}
\vedette{\hypertarget{Ⓔŋkhorwapa}{\papi{ ŋkhorwapa}}}\markboth{ŋkhorwapa}{}\classe{n}
\begin{définition}\fra paysan\end{définition}
\begin{définition}\cmn 农民
\begin{déclaration} \étymologie{\papi{ⁿkʰor.ba.pa}}\end{déclaration}\end{définition}\end{entrée}

\begin{entrée}
\vedette{\hypertarget{Ⓔŋkhrɯl}{\papi{ ŋkhrɯl}}}\markboth{ŋkhrɯl}{}\classe{vi}
\paradigme{\textit{dir :} \jya tɤ-}\acception{1}
\begin{définition}\fra se desserrer\end{définition}
\begin{définition}\cmn 变松(螺丝;门)\end{définition}
\begin{exemple}\jya ɯ-ŋkhrɯli to-ŋkhrɯl\cmn 螺丝松了\end{exemple}
\begin{exemple}\jya ɯ-kɯm to-ŋkhrɯl kɤ-nɤcɯpa mɯ́j-khɯ\cmn 门变松,开关都不方便了\end{exemple}\acception{2}
\begin{définition}\fra fléchir (résolution)\end{définition}
\begin{définition}\cmn 动摇(决心)\end{définition}
\begin{exemple}\jya ɯ-sɯm to-ŋkhrɯl (=ɲɤ-ɲɟɯr)\cmn 他开始动摇了\end{exemple}\end{entrée}

\begin{entrée}
\vedette{\hypertarget{Ⓔŋkhrɯli}{\papi{ ŋkhrɯli}}}\markboth{ŋkhrɯli}{}\classe{n}
\begin{définition}\fra vis\end{définition}
\begin{définition}\cmn 螺丝\end{définition}
\begin{exemple}\jya ŋkhrɯli tɤ-spra-t-a\cmn 我拧了螺丝\end{exemple}\end{entrée}

\begin{entrée}
\vedette{\hypertarget{Ⓔŋoj}{\papi{ ŋoj}}}\markboth{ŋoj}{}
\classe{pro}
\begin{définition}\fra où\end{définition}
\begin{définition}\cmn 哪里\end{définition}
\begin{exemple}\jya nɤʑo ŋoj ku-tɯ-rɤʑi?\cmn 你在哪里?\end{exemple}
\begin{exemple}\jya ŋoj nɯ-ari ma mɯ-ɲɤ-mto-t-a\cmn (书包)丢到哪里了,我看不到\end{exemple}
\begin{exemple}\jya ŋoj tɯ-ɕe\cmn 你去哪里?\end{exemple}
\begin{relation-sémantique}\confer{
\hyperlink{Ⓔŋotɕu}{\textit{ \papi{ŋotɕu}}}
}\end{relation-sémantique}\end{entrée}

\begin{entrée}
\vedette{\hypertarget{Ⓔŋoʁ}{\papi{ ŋoʁ}}}\markboth{ŋoʁ}{}\classe{n}
\begin{définition}\fra crochet (pour attacher les vêtements)\end{définition}
\begin{définition}\cmn 用来勾住披衫(雨衣)的铁钩\end{définition}\end{entrée}

\begin{entrée}
\vedette{\hypertarget{Ⓔŋotɕu}{\papi{ ŋotɕu}}}\markboth{ŋotɕu}{}\classe{pro}
\begin{définition}\fra où\end{définition}
\begin{définition}\cmn 哪里\end{définition}
\begin{exemple}\jya ŋotɕu ku-tɯ-rɤʑi?\cmn 你在哪里?\end{exemple}
\begin{exemple}\jya ŋotɕu tɯ-ɕe?\cmn 你去哪儿?\end{exemple}
\begin{exemple}\jya kɯki ŋotɕu ɲɯ-ŋgrɤl?\cmn 怎么可以这样?\end{exemple}
\begin{exemple}\jya ŋotɕu chiz ku-tɯ-nɯ-rɤʑi kɯ?\cmn 不知道你住在哪个地方?\end{exemple}
\begin{exemple}\jya ŋotɕu sɤtɕha ɣɯ ɯ-tɯrme nɯ pɯ-nnɯ-ŋɯ-ŋu ʑo khɯ\cmn 什么地方的人都可以\end{exemple}
\begin{relation-sémantique}\confer{
\hyperlink{Ⓔŋoj}{\textit{ \papi{ŋoj}}}
}\end{relation-sémantique}\end{entrée}

\begin{entrée}
\vedette{\hypertarget{Ⓔŋotɕuŋondɤt}{\papi{ ŋotɕuŋondɤt}}}\markboth{ŋotɕuŋondɤt}{}
\classe{adv}
\begin{définition}\fra partout\end{définition}
\begin{définition}\cmn 到处\end{définition}
\begin{exemple}\jya ŋotɕuŋondɤt ʑo tu\cmn 到处都有\end{exemple}
\begin{relation-sémantique}\synonyme{
\hyperlink{Ⓔaʁɤndɯndɤt}{\textit{ \papi{aʁɤndɯndɤt}}}
}\end{relation-sémantique}
\begin{relation-sémantique}\confer{
\hyperlink{Ⓔŋotɕu}{\textit{ \papi{ŋotɕu}}}
}\end{relation-sémantique}
\begin{relation-sémantique}\confer{
\hyperlink{Ⓔɕɯmɤɕɯ}{\textit{ \papi{ɕɯmɤɕɯ}}}
}\end{relation-sémantique}
\begin{relation-sémantique}\confer{
\hyperlink{Ⓔnɤndɯndɤt}{\textit{ \papi{nɤndɯndɤt}}}
}\end{relation-sémantique}\end{entrée}

\newpage\caractère{ɴ}

\begin{entrée}
\vedette{\hypertarget{Ⓔɴɢu}{\papi{ ɴɢu}}}\markboth{ɴɢu}{}
\classe{vs}
\paradigme{\textit{dir :} \jya nɯ-}
\begin{définition}\fra relâché\end{définition}
\begin{définition}\cmn 松\end{définition}
\begin{exemple}\jya ta-ma ɲɯ-ɴɢu\cmn 工作很轻松\end{exemple}
\begin{exemple}\jya ki ɯ-xtɕɤr ki ɲɯ-ɴɢu\cmn 系得很松\end{exemple}
\begin{exemple}\jya ɯ-sɤ-xtɕɤr ɲɯ-ɴɢu\cmn 系得很松\end{exemple}
\begin{exemple}\jya tɤ-mtɯ ɲɯ-ɴɢu\cmn 结很松\end{exemple}
\begin{relation-sémantique}\antonyme{
\hyperlink{Ⓔasɯɣ}{\textit{ \papi{asɯɣ}}}
}\end{relation-sémantique}
\begin{relation-sémantique}\confer{
\hyperlink{Ⓔɴɢule}{\textit{ \papi{ɴɢule}}}
}\end{relation-sémantique}\begin{sous-entrée}
\vedette{\hypertarget{}{\papi{ ɣɤɴɢu}}}\markboth{ɣɤɴɢu}{}
\paradigme{\textit{dir :} \jya nɯ-}
\begin{définition}\fra desserrer\end{définition}
\begin{définition}\cmn 松开,放开\end{définition}
\begin{exemple}\jya nɤki tɤ-mtɯ ɯ-tɯ-ɤsɯɣ ɲɯ-tɕhom tɕe, ɲo-ɣɤɴɢu\cmn 那个结太紧了,他把它松了一下\end{exemple}
\end{sous-entrée}\end{entrée}

\begin{entrée}
\vedette{\hypertarget{Ⓔɴɢarmɯ}{\papi{ ɴɢarmɯ}}}\markboth{ɴɢarmɯ}{}
\classe{n}
\begin{définition}\fra vache bâtarde\end{définition}
\begin{définition}\cmn 母杂种牛\end{définition}\end{entrée}

\begin{entrée}
\vedette{\hypertarget{Ⓔɴɢarpa}{\papi{ ɴɢarpa}}}\markboth{ɴɢarpa}{}
\classe{n}
\begin{définition}\fra bœuf bâtard\end{définition}
\begin{définition}\cmn 杂种牛\end{définition}\end{entrée}

\begin{entrée}
\vedette{\hypertarget{Ⓔɴɢartɯm,ɣɯt}{\papi{ ɴɢartɯm,ɣɯt}}}\markboth{ɴɢartɯm,ɣɯt}{} (\variante{ɴɢaftɯm}) 
\begin{définition}\cmn 俯冲\end{définition}
\begin{exemple}\jya qaliaʁ kɯ ɴɢartɯm pjɤ-ɣɯt\cmn 老鹰俯冲(下去)了\end{exemple}
\begin{exemple}\jya kɯjka kɯ ɴɢartɯm pa-ɣɯt\cmn 红嘴乌鸦俯冲(下去)了\end{exemple}
\begin{relation-sémantique}\ComponentA{\classe{n}
 \papi{ɴɢartɯm}
}\end{relation-sémantique}
\begin{relation-sémantique}\ComponentB{\classe{vt}
\hyperlink{Ⓔɣɯt}{\textit{ \papi{ɣɯt}}}
}\end{relation-sémantique}\end{entrée}

\begin{entrée}
\vedette{\hypertarget{Ⓔɴɢaʁ}{\papi{ ɴɢaʁ}}}\markboth{ɴɢaʁ}{}
\classe{vi}
\paradigme{\textit{dir :} \jya pɯ-}
\begin{définition}\ 
\begin{déclaration}\grammar{acaus}\end{déclaration}\end{définition}
\begin{définition}\fra perdre sa peau\end{définition}
\begin{définition}\cmn 自动脱皮
\begin{déclaration}\use{蛇皮不能说\stylefv{ɴɢaʁ},必须说\stylefv{βde}}\end{déclaration}\end{définition}
\begin{exemple}\jya sɯrqhu pjɤ-ɴɢaʁ\cmn 树皮脱了\end{exemple}
\begin{exemple}\jya a-mi ɯ-rqhu pjɤ-ɴɢaʁ\cmn 我的脚脱皮了\end{exemple}
\begin{exemple}\jya ʑmbɤr ɯ-rqhu pjɤ-ɴɢaʁ\cmn 疮脱皮了\end{exemple}
\begin{relation-sémantique}\confer{
\hyperlink{ⒺqaʁⒽ1}{\textit{ \papi{qaʁ1}}}
}\end{relation-sémantique}\end{entrée}

\begin{entrée}
\vedette{\hypertarget{Ⓔɴɢaʁrɯm}{\papi{ ɴɢaʁrɯm}}}\markboth{ɴɢaʁrɯm}{}
\classe{n}
\begin{définition}\fra ombre (bâtiments, montagne)\end{définition}
\begin{définition}\cmn 阴影\end{définition}\end{entrée}

\begin{entrée}
\vedette{\hypertarget{Ⓔɴɢɤjom}{\papi{ ɴɢɤjom}}}\markboth{ɴɢɤjom}{}
\classe{n}
\begin{définition}\fra Rumex crispus\end{définition}
\begin{définition}\cmn 皱叶酸模\end{définition}
\begin{exemple}\jya ɴɢɤjom nɯ ruŋgu kɯ-mbro tsa tu-ɬoʁ ŋu, ɯ-ru nɯ qhɤjmbaʁ ɯ-ru cho naχtɕɯɣ, ɯ-jwaʁ nɯ qhɤjmbaʁ ɯ-jwaʁ sɤznɤ khro ʑo ndɯβ, ɯ-rme kɯ-fse kɯ-xtɕɯ-xtɕi tu, ɯ-mɯntoʁ kɯ-wɣrum ŋu, kɯ-ndɯ-ndɯβ ʑo ŋu, ɯ-ru nɯ tú-wɣ-ndza tɕe wuma ʑo tɕur. zgoku pa pɕoʁ ra maka mɤ-ɬoʁ.\cmn 
酸模生长在比较高的草山上。茎和\stylefv{qhɤjmbaʁ}的一样,叶子比\stylefv{qhɤjmbaʁ}的叶子小得多,有点细毛。花是白色的,细小。茎吃起来很酸。山下根本不能生长。
\end{exemple}\end{entrée}

\begin{entrée}
\vedette{\hypertarget{Ⓔɴɢɤt}{\papi{ ɴɢɤt}}}\markboth{ɴɢɤt}{}
\classe{vi}
\paradigme{\textit{dir :} \jya nɯ-}
\begin{définition}\fra se séparer\end{définition}
\begin{définition}\cmn 分散;分手\end{définition}
\begin{exemple}\jya tɯtɯrca pɯ-rɤʑi-tɕi, tɕe nɯ-nɯ-ɴɢɤt-tɕi\cmn 我们俩原来在一起,然后就分手了\end{exemple}
\begin{exemple}\jya ʁzɤmi ni pɯ-nɯ-ɴɢɤt-tɕi\cmn 我们离婚了\end{exemple}
\begin{exemple}\jya tɤ-pi nɯ tɤ-rɯstɯnmɯ tɕe, kɤndʑɯxtɤɣ ni nɯ-nɯ-ɴɢɤt-ndʑi\cmn 哥哥结婚了,兄弟俩就分开了\end{exemple}
\begin{relation-sémantique}\confer{
\hyperlink{Ⓔqɤt}{\textit{ \papi{qɤt}}}
}\end{relation-sémantique}
\begin{relation-sémantique}\confer{
\hyperlink{Ⓔnɯɴɢɯlɯjɤt}{\textit{ \papi{nɯɴɢɯlɯjɤt}}}
}\end{relation-sémantique}
\begin{relation-sémantique}\confer{
\hyperlink{Ⓔznɯɴɢɤt}{\textit{ \papi{znɯɴɢɤt}}}
}\end{relation-sémantique}\end{entrée}

\begin{entrée}
\vedette{\hypertarget{Ⓔɴɢia}{\papi{ ɴɢia}}}\markboth{ɴɢia}{}\classe{vi}
\begin{définition}\ 
\begin{déclaration}\grammar{acaus}\end{déclaration}\end{définition}
\begin{définition}\fra se détacher, se dérouler (fil)\end{définition}
\begin{définition}\cmn 散(线)\end{définition}
\begin{exemple}\jya kɤtɯm pjɤ-ɴɢia tɕe ɲɤ-ɬɯt\cmn 线团散了就乱了\end{exemple}
\begin{exemple}\jya tɤ-mtsɯ ɲɤ-ɴɢia\cmn 结散了\end{exemple}
\begin{exemple}\jya tɤ-fkɯm ɯ-mŋu ɲɤ-nɯ-ɴɢia (=tɤ-fkɯm ɯ-xtɕɤr ɲɤ-nɯ-ɬoʁ)\cmn 口袋的口自动解开了\end{exemple}
\begin{relation-sémantique}\confer{
\hyperlink{Ⓔqia}{\textit{ \papi{qia}}}
}\end{relation-sémantique}\end{entrée}

\begin{entrée}
\vedette{\hypertarget{Ⓔɴɢiɤβɴɢiɤβ}{\papi{ ɴɢiɤβɴɢiɤβ}}}\markboth{ɴɢiɤβɴɢiɤβ}{}
\classe{idph.2}
\begin{définition}\fra nonchalant\end{définition}
\begin{définition}\cmn 形容不慌不忙的样子\end{définition}\begin{sous-entrée}
\vedette{\hypertarget{}{\papi{ ɴɢiɤβnɤɴɢiɤβ}}}\markboth{ɴɢiɤβnɤɴɢiɤβ}{}\classe{idph.3}
\begin{définition}\fra nonchalant\end{définition}
\begin{définition}\cmn 不慌不忙地(做事)\end{définition}
\begin{exemple}\jya ɴɢiɤβnɤɴɢiɤβ kɤ-ari\cmn 他不慌不忙地去了\end{exemple}
\begin{exemple}\jya ɴɢiɤβnɤɴɢiɤβ ɲɯ-rɤma\cmn 他不慌不忙地劳动\end{exemple}
\end{sous-entrée}\end{entrée}

\begin{entrée}
\vedette{\hypertarget{Ⓔɴɢiɤt}{\papi{ ɴɢiɤt}}}\markboth{ɴɢiɤt}{}
\classe{vs}
\paradigme{\textit{dir :} \jya tɤ-}
\begin{définition}\fra désordonné\end{définition}
\begin{définition}\cmn 不爱整理的\end{définition}
\begin{exemple}\jya ma-tɤ-tɯ-ɴɢiɤt tɕe nɤ-ŋga ra tɤ-rɤwum\cmn 你不要真么乱,收拾一下你的衣服\end{exemple}\begin{sous-entrée}
\vedette{\hypertarget{}{\papi{ nɤɴɢiɤt}}}\markboth{nɤɴɢiɤt}{}\classe{vt}
\paradigme{\textit{dir :} \jya tɤ-}
\begin{définition}\fra ne pas faire attention à\end{définition}
\begin{définition}\cmn 不重视\end{définition}
\begin{exemple}\jya nɤ-xtu ɲɯ-tɯ-nɤɴɢiɤt\cmn 你亏待你的肚子(吃得太少)\end{exemple}
\begin{exemple}\jya nɤ-kɤ-nɤma ra ma-tɤ-tɯ-nɤɴɢiɤt ma kɤ-nɤɴɢiɤt kɯ kɯ-nɤɴɢiɤt kɤ-ti\cmn 你要专心地做这个工作,如果你不重视它的话,它也会不重视你(工作就做不出来)\end{exemple}
\end{sous-entrée}\end{entrée}

\begin{entrée}
\vedette{\hypertarget{Ⓔɴɢule}{\papi{ ɴɢule}}}\markboth{ɴɢule}{}
\classe{vs}
\paradigme{\textit{dir :} \jya nɯ-}
\begin{définition}\fra oisif\end{définition}
\begin{définition}\cmn 松懈\end{définition}
\begin{exemple}\jya ta-ma kɯ-ɴɢule ci ɲɯ-ŋu\cmn 他是工作不勤快的人\end{exemple}
\begin{exemple}\jya kɤ-nɤma ɲɯ-ɴɢule\cmn 他工作得不勤快\end{exemple}
\begin{relation-sémantique}\confer{
\hyperlink{Ⓔɴɢu}{\textit{ \papi{ɴɢu}}}
}\end{relation-sémantique}\end{entrée}

\begin{entrée}
\vedette{\hypertarget{Ⓔɴɢlɯt}{\papi{ ɴɢlɯt}}}\markboth{ɴɢlɯt}{}\classe{vi}
\paradigme{\textit{dir :} \jya pɯ-}
\begin{définition}\ 
\begin{déclaration}\grammar{acaus}\end{déclaration}\end{définition}
\begin{définition}\fra se casser\end{définition}
\begin{définition}\cmn 折;断(自动)
\begin{déclaration}\use{这个动词的主语必须是条状的东西,如棍子、骨头等}\end{déclaration}\end{définition}
\begin{exemple}\jya pɯ-ɴɢlɯt\cmn 折了\end{exemple}
\begin{exemple}\jya ɕoŋtɕa pɯ-ɴɢlɯt\cmn 木料折了\end{exemple}
\begin{exemple}\jya laʁdɯn pjɤ-ɴɢlɯt\cmn 工具断了\end{exemple}
\begin{exemple}\jya a-jaʁ pjɤ-ɴɢlɯt\cmn 我的手折了\end{exemple}
\begin{exemple}\jya ɯ-mi pjɤ-ɴɢlɯt\cmn 他的脚折了\end{exemple}
\begin{exemple}\jya ɯ-rnom ko-ɴɢlɯt\cmn 他的肋骨折了\end{exemple}
\begin{exemple}\jya mbɣo pjɤ-ɴɢlɯt\cmn 犁断了\end{exemple}
\begin{exemple}\jya ɯ-jɯ pjɤ-ɴɢlɯt\cmn 把子断了\end{exemple}\begin{sous-entrée}
\vedette{\hypertarget{}{\papi{ ɣɤɴɢlɯt}}}\markboth{ɣɤɴɢlɯt}{}\classe{vs}
\begin{définition}\ 
\begin{déclaration}\grammar{facil}\end{déclaration}\end{définition}
\begin{définition}\fra qui se casse facilement\end{définition}
\begin{définition}\cmn 容易断\end{définition}
\begin{relation-sémantique}\confer{
\hyperlink{Ⓔqlɯt}{\textit{ \papi{qlɯt}}}
}\end{relation-sémantique}
\end{sous-entrée}\end{entrée}

\begin{entrée}
\vedette{\hypertarget{Ⓔɴɢoɕna}{\papi{ ɴɢoɕna}}}\markboth{ɴɢoɕna}{}
\classe{n}
\begin{définition}\fra grosse araignée\end{définition}
\begin{définition}\cmn 大蜘蛛\end{définition}
\begin{exemple}\jya ɴɢoɕna cho porɤt ni ɲɯ-naχtɕɯɣ-ndʑi, ɴɢoɕna kɯ-wxti, ɯ-mi ra jpum, porɤt nɯ xtɕi, ɯ-mi ra xtshɯm, kɯ-pɣi ŋu-ndʑi, ɴɢoɕna kɯ aɲaʁndzɯm. ɴɢoɕna cho porɤt ni ndʑi-xtu ɯ-ŋgɯ ndʑi-ri kɯ-fse chɯ-nɯ-tɕɤt-ndʑi tɕe, kha tu-nɯ-βzu-nɯ ɲɯ-ŋgrɤl. nɯ kha nɯ ɴɢoɕnamɤjɯ rmi. ndʑi-kha nɯ tɕu kɯmaʁ qajɯ ra nɯ-fsa tu-sɯpa-nɯ tɕe, ka-ndo tɕe tu-ndza-nɯ ɲɯ-ŋu. nɯ-pɯ wuma ɲɯ-dɤn, koŋla mɯ-thɯ-wxti-nɯ mɤɕtʂa ɯ-fkɯm ci ɣɤʑu tɕe tu-nɤfkɯ-fkur ɲɯ-ra.\cmn 大蜘蛛和小蜘蛛很相似,大蜘蛛比较大,脚粗一点,小蜘蛛比较小,脚细一点。两种都是灰色的,大蜘蛛是暗灰色的。它们从肚子里抽出丝来制造“房子”,这种房子叫蜘蛛网,用这个网作抓其它虫子的圈套,一旦网缠住了(小虫),它们就会把它吃掉。蜘蛛有很多幼虫,在幼虫尚未长大之前,一直把它们放在一种袋子里背来背去。\end{exemple}\end{entrée}

\begin{entrée}
\vedette{\hypertarget{Ⓔɴɢoɕnamɤjɯ}{\papi{ ɴɢoɕnamɤjɯ}}}\markboth{ɴɢoɕnamɤjɯ}{}\classe{n}
\begin{définition}\fra toile d'araignée\end{définition}
\begin{définition}\cmn 蜘蛛网\end{définition}\end{entrée}

\begin{entrée}
\vedette{\hypertarget{Ⓔɴɢolo}{\papi{ ɴɢolo}}}\markboth{ɴɢolo}{}
\classe{n}
\begin{définition}\fra une espèce d'arbrisseau\end{définition}
\begin{définition}\cmn 灌木的一种\end{définition}
\begin{exemple}\jya ɴɢolo nɯ si kɯ-mbɤr tsa ci ŋu. ɯ-ru ɯ-pɕi nɯ ra kɯ-pɣi ci ŋu. ɯ-mdzu kɯ-rɲɟi kɯ-mtɕoʁ tɯ-khɤl ʑo χsɯ-ldʑa ntsɯ ku-ndzoʁ ŋu. ɯ-rtaʁ ɯ-taʁ ra kɯnɤ ɯ-mdzu tu. ɯ-jwaʁ ɯ-tshɯɣa nɯ babɯ ɯ-jwaʁ cho naχtɕɯɣ. ɯ-mɯntoʁ tshaŋlaŋ kɯ-fse ɲɯ-βze ŋu, kɯ-wɣrum ɯ-rkɯ zɯ kɯ-ɣɯrni tu-fskɤr ŋu. kɯ-ɤrqhi jɯ-kɯ-ru mɤ-saχsɤl jɯ-kɯ-ɤrmbat tɕe, ɯ-mɯntoʁ tú-wɣ-rtoʁ tɕe mpɕɤr. ɯ-jɯ tu tɕe, pjɯ-ɴqoʁ tu-fse ŋu. ɯ-mat thɯ-aβzu tɕe, arŋi tɕe ʂɣɤlʂɣɤl ʑo pa, ɯ-ŋgɯ ɯ-rdoʁ ra kɯnɤ saχsɤl. ɯ-mat ɯ-βri ɯ-rme kɯ-xtɯt tsa tu. tú-wɣ-ndza tɕe, ɯ-tɯ-tɕur saχaʁ. tɤ-pɤtso ra kɤ-nɤɣro kɯ-fse ma kɤ-ndza mɤ-sna. ɯ-mnɯ kɯ-ɕɤɣ tɤ-kɯ-ɬoʁ nɯ ɣɯrni ɯ-mdzu tu ri ɯ-rqhu pjɯ́-wɣ-qaʁ tɕe tú-wɣ-ndza tɕe mpɯ, kɯ-xtɕɯ-xtɕi chi. tɤmdzɤqaqa rmi.\cmn 
\stylefv{ɴɢolo}是一种矮小的树,树皮是灰色的,刺又长又锋利,三根长在一起。枝桠上也有刺。叶子的形状和\stylefv{babɯ}的叶子一样。花像铃铛一样,花瓣是白色的,边上镶有红色。远处看不出,走近了看就觉得花很美。花有花梗,是吊着的。果实结了以后,看起来很透明,连里面的种子也看得见。果实外面也长有短毛。吃起来很酸。除了小孩子吃着玩以外就不能吃。发出的新苗是红色的,虽然有刺,但剥了皮吃起来很嫩,有点甜。这种新苗叫\stylefv{tɤmdzɤqaqa}。
\end{exemple}\end{entrée}

\begin{entrée}
\vedette{\hypertarget{Ⓔɴɢolophɯcɯ}{\papi{ ɴɢolophɯcɯ}}}\markboth{ɴɢolophɯcɯ}{}
\classe{n}
\begin{définition}\fra grès\end{définition}
\begin{définition}\cmn 砂岩\end{définition}
\begin{exemple}\jya ɴɢolophɯcɯ, kɤ́-wɣ-rtoʁ tɕe, tɤ-phɯ fse ri rdɤstaʁ jamar rko\cmn 砂岩看起来像土块,但是像石头那么硬\end{exemple}\end{entrée}

\begin{entrée}
\vedette{\hypertarget{Ⓔɴɢorna}{\papi{ ɴɢorna}}}\markboth{ɴɢorna}{}
\classe{n}
\begin{définition}\fra une plante\end{définition}
\begin{définition}\cmn 植物的一种\end{définition}
\begin{exemple}\jya ɴɢorna nɯ sɯjno kɯ-mpɯ-mpɯ ci ŋu. ɯ-ru kɯ-zɯ-zri ŋu tɕe, ɯ-zda sɯjno ɯ-taʁ tu-nɯrʁɯrʁa ra ma ɯʑo tu-nɯ-ndzur mɤ-cha ma mpɯ. ɯ-jwaʁ cho ɯ-ru ra kɯ-rʁɯ-rʁom ŋu. tɯ-ɕa ra pjɯ-qraʁ cha. tɯ-ŋga ɯ-taʁ ku-ndzoʁ ŋu. ɯ-mat nɯ kɯ-ɤrtɯ-rtɯm tɕe, li ɯ-taʁ ɯ-rme kɯ-tu ŋu, ɯ-jwaʁ ɯ-rchɤβ ri ku-ndzoʁ ŋu. thɯ-tɯt tɕe ɲaʁ. ɯ-jwaʁ kɯ-ndɯ-ndɯβ ŋu, ɯ-ru ɯ-taʁ kɤ-fskɤr tɕe, tɯ-rtsɤɣ tɯ-rtsɤɣ ku-ndzoʁ ŋu. ɴɢorna nɯ fsapaʁndza sna.\cmn 
\stylefv{ɴɢorna}是一种很软的草,茎长得很长,必须爬在其它草上因为它身子软,立不起来。叶子和茎很粗糙,可以把皮肉刮破,还粘在衣服上。果实是球形的,上面也有毛,生长在叶子的中间。成熟了以后是黑色的。叶子很小,是绕着茎一节一节地长着。\stylefv{ɴɢorna}是牲畜的饲料。
\end{exemple}\end{entrée}

\begin{entrée}
\vedette{\hypertarget{Ⓔɴqa}{\papi{ ɴqa}}}\markboth{ɴqa}{}
\classe{vs}
\paradigme{\textit{dir :} \jya thɯ-}
\begin{définition}\fra dur (travail)\end{définition}
\begin{définition}\cmn 辛苦;难做\end{définition}
\begin{exemple}\jya ta-ma ɲɯ-ɴqa\cmn 工作很辛苦\end{exemple}
\begin{exemple}\jya ɯ-pɯ́-ɴqa\cmn 辛苦了吗\end{exemple}
\begin{exemple}\jya nɤ-tʂha ɲɯ-ɴqa\cmn 你放了很多茶叶,茶很浓\end{exemple}\begin{sous-entrée}
\vedette{\hypertarget{}{\papi{ ɣɤɴqa}}}\markboth{ɣɤɴqa}{}\classe{vt}
\paradigme{\textit{dir :} \jya tɤ-}
\begin{définition}\fra rendre difficile\end{définition}
\begin{définition}\cmn 使困难\end{définition}
\begin{exemple}\jya ɯ-phɯ to-ɣɤɴqa\cmn 他加价了\end{exemple}
\begin{relation-sémantique}\antonyme{
\hyperlink{Ⓔmbat}{\textit{ \papi{mbat}}}
}\end{relation-sémantique}
\begin{relation-sémantique}\confer{
\hyperlink{Ⓔnɤɴqa}{\textit{ \papi{nɤɴqa}}}
}\end{relation-sémantique}
\end{sous-entrée}\end{entrée}

\begin{entrée}
\vedette{\hypertarget{Ⓔɴqhɤβɴqhɤβ}{\papi{ ɴqhɤβɴqhɤβ}}}\markboth{ɴqhɤβɴqhɤβ}{}\classe{idph.2}
\begin{définition}\fra épais et dur\end{définition}
\begin{définition}\cmn 形容厚实的样子\end{définition}
\begin{exemple}\jya qhɤjmbaʁ ɣɯ ɯ-jwaʁ ɴqhɤβɴqhɤβ ʑo pa\cmn 红青椒的叶子很厚实的样子\end{exemple}
\begin{relation-sémantique}\synonyme{
\hyperlink{Ⓔɴqhɯɴqhi}{\textit{ \papi{ɴqhɯɴqhi}}}
}\end{relation-sémantique}\end{entrée}

\begin{entrée}
\vedette{\hypertarget{Ⓔɴqhi}{\papi{ ɴqhi}}}\markboth{ɴqhi}{}\classe{vs}
\paradigme{\textit{dir :} \jya kɤ-}
\begin{définition}\fra sale\end{définition}
\begin{définition}\cmn 脏\end{définition}
\begin{exemple}\jya ki khɯtsa ɲɯ-ɴqhi\cmn 这个碗很脏\end{exemple}
\begin{exemple}\jya βɣɤza nɯ kɯ-ɴqhi ɲɯ-rga\cmn 苍蝇爱脏的东西\end{exemple}
\begin{exemple}\jya ko-ɴqhi\cmn 变脏了\end{exemple}
\begin{relation-sémantique}\confer{
\hyperlink{Ⓔtɤlɤɴqhi}{\textit{ \papi{tɤlɤɴqhi}}}
}\end{relation-sémantique}\begin{sous-entrée}
\vedette{\hypertarget{}{\papi{ ɣɤɴqhi}}}\markboth{ɣɤɴqhi}{}\classe{vs}
\begin{définition}\fra qui se salit vite\end{définition}
\begin{définition}\cmn 脏得快\end{définition}
\end{sous-entrée}\begin{sous-entrée}
\vedette{\hypertarget{}{\papi{ nɤɴqhi}}}\markboth{nɤɴqhi}{}\classe{vt}
\begin{définition}\ 
\begin{déclaration}\grammar{trop}\end{déclaration}\end{définition}
\begin{exemple}\jya khɯtsa ɲɯ-nɤɴqhi tɕe, mɯ́j-nɤpe\cmn 他觉得碗很脏,觉得不好\end{exemple}
\end{sous-entrée}\begin{sous-entrée}
\vedette{\hypertarget{}{\papi{ sɯɴqhi}}}\markboth{sɯɴqhi}{}\classe{vt}
\paradigme{\textit{dir :} \jya kɤ-}
\begin{définition}\ 
\begin{déclaration}\grammar{caus}\end{déclaration}\end{définition}
\begin{définition}\fra salir\end{définition}
\begin{définition}\cmn 弄脏\end{définition}
\begin{exemple}\jya tɤ-pɤtso kɯ ɯ-ŋga ko-sɯɴqhi\cmn 小孩子把自己衣服弄脏了\end{exemple}
\begin{exemple}\jya tɤ-lu kɯ ɯ-sɤ-rku thamtɕɤt sɯ-ɴqhi ɕti\cmn 牛奶会弄脏容器\end{exemple}
\end{sous-entrée}\end{entrée}

\begin{entrée}
\vedette{\hypertarget{Ⓔɴqhɯɴqhi}{\papi{ ɴqhɯɴqhi}}}\markboth{ɴqhɯɴqhi}{}\classe{idph.2}
\begin{définition}\fra épais\end{définition}
\begin{définition}\cmn 形容叶子、菌子、脸等厚实的样子\end{définition}
\begin{relation-sémantique}\synonyme{
\hyperlink{Ⓔɴqhɤβɴqhɤβ}{\textit{ \papi{ɴqhɤβɴqhɤβ}}}
}\end{relation-sémantique}\end{entrée}

\begin{entrée}
\vedette{\hypertarget{Ⓔɴqiaβ}{\papi{ ɴqiaβ}}}\markboth{ɴqiaβ}{}\classe{n}
\begin{définition}\fra ubac\end{définition}
\begin{définition}\cmn 山阴,背阴的山坡\end{définition}
\begin{définition}\jya \end{définition}\end{entrée}

\begin{entrée}
\vedette{\hypertarget{Ⓔɴqiazwɤr}{\papi{ ɴqiazwɤr}}}\markboth{ɴqiazwɤr}{}\classe{n}
\begin{définition}\fra espèce d'armoise\end{définition}
\begin{définition}\cmn 阳山的艾蒿\end{définition}
\begin{relation-sémantique}\confer{
\hyperlink{ⒺzwɤrⒽ2}{\textit{ \papi{zwɤr2}}}
}\end{relation-sémantique}
\begin{relation-sémantique}\confer{
\hyperlink{Ⓔɴqiaβ}{\textit{ \papi{ɴqiaβ}}}
}\end{relation-sémantique}\end{entrée}

\begin{entrée}
\vedette{\hypertarget{Ⓔɴqoʁ}{\papi{ ɴqoʁ}}}\markboth{ɴqoʁ}{}
\classe{vi}
\paradigme{\textit{dir :} \jya kɤ-}
\paradigme{\textit{dir :} \jya tɤ-}
\begin{définition}\fra être accroché, se tenir\end{définition}
\begin{définition}\cmn 悬挂;扶住\end{définition}
\begin{exemple}\jya kɤ-ɴqoʁ ma tɯ-atɤr\cmn 你抓稳,小心不要掉下来\end{exemple}
\begin{exemple}\jya aʑo si ɯ-taʁ zɯ kɤ-ɴqoʁ-a\cmn 我抓住了树\end{exemple}
\begin{exemple}\jya ndʐaβ-a ɲɯ-ŋu tɕe, @langan ɯ-taʁ kɤ-ɴqoʁ-a\cmn 我差一点跌倒了,但抓住了栏杆\end{exemple}
\begin{exemple}\jya ɕomskrɯt ɯ-taʁ tɯ-ŋga χsɯm ɲɯ-ɴqoʁ\cmn 三件衣服挂在铁丝上\end{exemple}
\begin{relation-sémantique}\confer{
\hyperlink{Ⓔɕɯɴqoʁ}{\textit{ \papi{ɕɯɴqoʁ}}}
}\end{relation-sémantique}
\begin{relation-sémantique}\confer{
\hyperlink{Ⓔʑɴɢoʁ}{\textit{ \papi{ʑɴɢoʁ}}}
}\end{relation-sémantique}\end{entrée}

\begin{entrée}
\vedette{\hypertarget{Ⓔɴɢraʁ}{\papi{ ɴɢraʁ}}}\markboth{ɴɢraʁ}{}
\classe{vi}
\paradigme{\textit{dir :} \jya pɯ-}
\paradigme{\textit{dir :} \jya thɯ-}
\begin{définition}\ 
\begin{déclaration}\grammar{acaus}\end{déclaration}\end{définition}
\begin{définition}\fra se déchirer\end{définition}
\begin{définition}\cmn 破烂(衣服、皮)\end{définition}
\begin{exemple}\jya a-ŋga pjɤ-ɴɢraʁ\cmn 我的衣服破了\end{exemple}
\begin{exemple}\jya nɤ-ŋga nɤ-xtsa pjɤ-ɴɢraʁ\cmn 你的衣服鞋子都破了\end{exemple}
\begin{exemple}\jya a-jaʁ pjɤ-ɴɢraʁ\cmn 我的手破了(皮肤破了)\end{exemple}
\begin{exemple}\jya nɤ-ŋga cho-ɴɢraʁ\cmn 你的衣服破了\end{exemple}
\begin{exemple}\jya ɕoʁɕoʁ chɤ-ɴɢraʁ\cmn 纸撕破了\end{exemple}
\begin{relation-sémantique}\confer{
\hyperlink{ⒺqraʁⒽ1}{\textit{ \papi{qraʁ1}}}
}\end{relation-sémantique}\begin{sous-entrée}
\vedette{\hypertarget{}{\papi{ ɕɤrkha,ɴɢraʁ}}}\markboth{ɕɤrkha,ɴɢraʁ}{}
\paradigme{\textit{dir :} \jya nɯ-}
\begin{définition}\fra percer, se lever (aube)\end{définition}
\begin{définition}\cmn 破晓\end{définition}
\begin{exemple}\jya ɕɤrkha ɲɤ-ɴɢraʁ\cmn 破晓了\end{exemple}
\begin{relation-sémantique}\ComponentA{\classe{n}
\hyperlink{Ⓔɕɤrkha}{\textit{ \papi{ɕɤrkha}}}
}\end{relation-sémantique}
\begin{relation-sémantique}\ComponentB{\classe{vi}
\hyperlink{Ⓔɴɢraʁ}{\textit{ \papi{ɴɢraʁ}}}
}\end{relation-sémantique}
\end{sous-entrée}\end{entrée}

\begin{entrée}
\vedette{\hypertarget{Ⓔɴɢrɤz}{\papi{ ɴɢrɤz}}}\markboth{ɴɢrɤz}{}\classe{vi}
\paradigme{\textit{dir :} \jya nɯ-}
\begin{définition}\ 
\begin{déclaration}\grammar{acaus}\end{déclaration}\end{définition}
\begin{définition}\fra se réduire en poussière au moindre toucher (objets secs)\end{définition}
\begin{définition}\cmn 一摸就烂(干的东西)\end{définition}
\begin{exemple}\jya xɕaj pjɤ-rom tɕe ɲɤ-ɴɢrɤz ɲɯ-ɕti\cmn 草干了以后一摸就烂了\end{exemple}
\begin{exemple}\jya tɤ-jwaʁ ɲɤ-ɴɢrɤz\cmn 叶子一摸就烂了\end{exemple}
\begin{relation-sémantique}\confer{
\hyperlink{Ⓔqrɤz}{\textit{ \papi{qrɤz}}}
}\end{relation-sémantique}\end{entrée}

\begin{entrée}
\vedette{\hypertarget{Ⓔɴɢrɯ}{\papi{ ɴɢrɯ}}}\markboth{ɴɢrɯ}{}\classe{vi}
\paradigme{\textit{dir :} \jya pɯ-}
\begin{définition}\ 
\begin{déclaration}\grammar{acaus}\end{déclaration}\end{définition}
\begin{définition}\fra se casser\end{définition}
\begin{définition}\cmn 破;碎(自动)
\begin{déclaration}\use{用于陶瓷、玻璃等易碎品,用法和\stylefv{ɴɢlɯt}“折”、\stylefv{ɴɢraʁ}“破烂”等动词不一样}\end{déclaration}\end{définition}
\begin{exemple}\jya χɕɤlzgoŋ ki pjɤ-ɴɢrɯ\cmn 镜子破了\end{exemple}
\begin{exemple}\jya khɯtsa pɯ-ɴɢrɯ\cmn 碗破了\end{exemple}
\begin{exemple}\jya popo pjɤ-ɴɢrɯ\cmn 砂锅破了\end{exemple}
\begin{exemple}\jya ɕɯ-ɴɢrɯ nɯ-sɯsota\cmn 我怕会破\end{exemple}\begin{sous-entrée}
\vedette{\hypertarget{}{\papi{ ɣɤɴɢrɯ}}}\markboth{ɣɤɴɢrɯ}{}\classe{vs}
\begin{définition}\fra qui se casse facilement\end{définition}
\begin{définition}\cmn 容易破\end{définition}
\begin{exemple}\jya ki khɯtsa ki mɯ́j-ngɯt tɕe ɲɯ-ɣɤɴɢrɯ\cmn 这个碗不结实,容易破\end{exemple}
\begin{relation-sémantique}\confer{
\hyperlink{Ⓔqrɯ}{\textit{ \papi{qrɯ}}}
}\end{relation-sémantique}
\end{sous-entrée}\end{entrée}

\begin{entrée}
\vedette{\hypertarget{Ⓔɴɢuʁɤr}{\papi{ ɴɢuʁɤr}}}\markboth{ɴɢuʁɤr}{} (\variante{ŋguʁɤr}) 
\classe{n}
\begin{définition}\fra tissu de laine\end{définition}
\begin{définition}\cmn 呢子
\begin{déclaration} \étymologie{\papi{bal}}\end{déclaration}\end{définition}\end{entrée}

\begin{entrée}
\vedette{\hypertarget{Ⓔɴɢɯɴɢli}{\papi{ ɴɢɯɴɢli}}}\markboth{ɴɢɯɴɢli}{}\classe{idph.2}
\begin{définition}\fra écarquillant les yeux\end{définition}
\begin{définition}\cmn 眼睛睁得很大的样子\end{définition}
\begin{exemple}\jya qala kɯ ɯ-mɲaʁ ɴɢɯɴɢli ʑo to-stu\cmn 兔子把眼睛睁得很大\end{exemple}
\begin{relation-sémantique}\confer{
\hyperlink{Ⓔqɯqli}{\textit{ \papi{qɯqli}}}
}\end{relation-sémantique}\end{entrée}

\newpage\caractère{o}

\begin{entrée}
\vedette{\hypertarget{Ⓔommanipɤnmehoŋʂi}{\papi{ ommanipɤnmehoŋʂi}}}\markboth{ommanipɤnmehoŋʂi}{}\classe{n}
\begin{définition}\fra un mantra\end{définition}
\begin{définition}\cmn 六字真言
\begin{déclaration} \étymologie{\papi{om.ma.ni.padme.hum.hri}}\end{déclaration}\end{définition}\end{entrée}

\newpage\caractère{p}

\begin{entrée}
\vedette{\hypertarget{Ⓔpu}{\papi{ pu}}}\markboth{pu}{}\classe{vt}
\paradigme{\textit{dir :} \jya thɯ-}
\paradigme{\textit{dir :} \jya lɤ-}
\begin{définition}\fra cuire dans les braises\end{définition}
\begin{définition}\cmn 煨\end{définition}
\begin{exemple}\jya jaŋjy lɤ-pe\cmn 你把土豆烤一下\end{exemple}
\begin{exemple}\jya qajɣi lɤ-pe\cmn 你把馍馍烤一下\end{exemple}
\begin{exemple}\jya tɤ-mthɯm thɯ-pu-t-a\cmn 我烤了肉\end{exemple}
\begin{exemple}\jya pɤjka lɤ-pu-t-a\cmn 我烤了白瓜\end{exemple}
\begin{relation-sémantique}\confer{
\hyperlink{Ⓔtɯpu}{\textit{ \papi{tɯpu}}}
}\end{relation-sémantique}\end{entrée}

\begin{entrée}
\vedette{\hypertarget{ⒺpaⒽ3}{\papi{ pa}}}\markboth{pa}{}\homonyme{3}
\classe{adv}
\begin{définition}\fra en bas\end{définition}
\begin{définition}\cmn 下面\end{définition}
\begin{sous-entrée}
\vedette{\hypertarget{}{\papi{ ɯ-pa}}}\markboth{ɯ-pa}{}\classe{np}
\begin{définition}\fra le bas\end{définition}
\begin{définition}\cmn 下面\end{définition}
\end{sous-entrée}\end{entrée}

\begin{entrée}
\vedette{\hypertarget{ⒺpaⒽ1}{\papi{ pa}}}\markboth{pa}{}\homonyme{1}
\classe{vt}\acception{1}
\paradigme{\textit{dir :} \jya tɤ-}
\begin{définition}\fra faire\end{définition}
\begin{définition}\cmn 办\end{définition}
\begin{exemple}\jya tɕhi tu-pe-a?\cmn 我怎么办?\end{exemple}
\begin{exemple}\jya aʑo akɯ ɕe-a, nɤʑo tɕhi tɯ-pe\cmn 我往东边去,你呢?\end{exemple}
\begin{exemple}\jya tɕhi tú-wɣ-pa ?\cmn 怎么办?\end{exemple}
\begin{exemple}\jya kutɕu tɕhi ɯ-kɯ-pa jɤ-tɯ-ɣe?\cmn 你来这里做什么?\end{exemple}
\begin{exemple}\jya dal tsa tɤ-pe\cmn 慢慢做\end{exemple}
\begin{relation-sémantique}\confer{
\hyperlink{Ⓔkɤpa}{\textit{ \papi{kɤpa}}}
}\end{relation-sémantique}\acception{2}
\paradigme{\textit{dir :} \jya kɤ-}
\begin{définition}\fra fermer\end{définition}
\begin{définition}\cmn 关\end{définition}
\begin{exemple}\jya kɯm kɤ-pe\cmn 关门吧!\end{exemple}
\begin{exemple}\jya @diandeng kɤ-pe\cmn 关电灯吧!\end{exemple}
\begin{exemple}\jya khɯɣɲɟɯ kɤ-pe\cmn 关窗户吧!\end{exemple}
\begin{exemple}\jya ɯʑo kɯ a-@dianhua pjɤ-pa\cmn 他挂了我电话\end{exemple}
\begin{exemple}\jya nɤ-@shouji kɤ-nɯ-pe ɯ́-jɤɣ?\cmn 麻烦你把手机关一下\end{exemple}
\begin{relation-sémantique}\confer{
\hyperlink{Ⓔapa}{\textit{ \papi{apa}}}
}\end{relation-sémantique}
\begin{relation-sémantique}\confer{\classe{vt}
\hyperlink{Ⓔsɤpa}{\textit{ \papi{sɤpa}}}
}\end{relation-sémantique}\begin{sous-entrée}
\vedette{\hypertarget{}{\papi{ nɯpa}}}\markboth{nɯpa}{}\classe{vt}
\paradigme{\textit{dir :} \jya tɤ-}
\begin{définition}\fra se mettre d'accord\end{définition}
\begin{définition}\cmn 商量好
\begin{déclaration}\grammar{autoben}\end{déclaration}\end{définition}
\begin{exemple}\jya ju-kɤ-ɕe tɤ-nɯpa-tɕi\cmn 我们商量好要(一起)出发\end{exemple}
\begin{exemple}\jya ɣufsu (βzaŋsa) tɤ-nɯ-pa-tɕi\cmn 我们俩交了朋友\end{exemple}
\end{sous-entrée}\begin{sous-entrée}
\vedette{\hypertarget{}{\papi{ nɯʑɣɤpa}}}\markboth{nɯʑɣɤpa}{}\classe{vi}
\paradigme{\textit{dir :} \jya kɤ-}
\paradigme{\textit{dir :} \jya thɯ-}
\begin{définition}\ 
\begin{déclaration}\grammar{refl}\end{déclaration}
\begin{déclaration}\grammar{autoben}\end{déclaration}\end{définition}
\begin{définition}\fra se fermer par soi-même\end{définition}
\begin{définition}\cmn 自动关\end{définition}
\begin{exemple}\jya kɯm ko-nɯʑɣɤpa\cmn 门自动关了\end{exemple}
\end{sous-entrée}\begin{sous-entrée}
\vedette{\hypertarget{}{\papi{ ɯ-pɯ,pa}}}\markboth{ɯ-pɯ,pa}{}
\paradigme{\textit{dir :} \jya tɤ-}
\begin{définition}\fra conserver\end{définition}
\begin{définition}\cmn 保管\end{définition}
\begin{exemple}\jya laχtɕha ɯ-pɯ tɤ-pe\cmn 你把东西保管好!\end{exemple}
\begin{exemple}\jya tɤ-lu tɤ-rʑaʁ kɯ-rɲɟi ɯ-pɯ kɤ-pa mɯ́j-khɯ tu-rpjɯ ɲɯ-ɕti\cmn 牛奶不能长时间保存,不然就会腥(变质)\end{exemple}
\begin{exemple}\jya nɤ-jaʁ ɯ-pɯ ma-tɤ-tɯ-nɯ-pe\cmn 你不要不管,要帮一下忙\end{exemple}
\begin{relation-sémantique}\ComponentA{\classe{np}
 \papi{ɯ-pɯ}
}\end{relation-sémantique}
\begin{relation-sémantique}\ComponentB{
\hyperlink{ⒺpaⒽ1}{\textit{ \papi{pa}}}
}\end{relation-sémantique}
\end{sous-entrée}\end{entrée}

\begin{entrée}
\vedette{\hypertarget{ⒺpaⒽ2}{\papi{ pa}}}\markboth{pa}{}\homonyme{2}
\classe{vs}
\paradigme{\textit{dir :} \jya tɤ-}
\begin{définition}\fra auxiliaire employé avec les idéophones\end{définition}
\begin{définition}\cmn 助动词(和状貌词连用)\end{définition}
\begin{exemple}\jya ko-nɯ-rŋgɯ tɕe, ɯ-sɤ-rŋgɯ-ŋga loŋloŋ ʑo ɲɯ-pa\cmn 他睡着了以后,床上是拱起来的\end{exemple}\end{entrée}

\begin{entrée}
\vedette{\hypertarget{Ⓔpakuku}{\papi{ pakuku}}}\markboth{pakuku}{} (\variante{xpakuku}) 
\classe{adv}
\begin{définition}\fra tous les ans\end{définition}
\begin{définition}\cmn 每年\end{définition}\end{entrée}

\begin{entrée}
\vedette{\hypertarget{Ⓔpalaxtsa}{\papi{ palaxtsa}}}\markboth{palaxtsa}{}\classe{n}
\begin{définition}\fra botte en peau de chevrotain à la couture grossière\end{définition}
\begin{définition}\cmn 靴筒全是獐皮子的一种靴子,针脚缝得比较粗糙\end{définition}
\begin{exemple}\jya palaxtsa nɯ ɯ-rkɯ kosca ŋu, tɕeri ɯ-ɕna ɣɯ ɯ-komɤr me, ɯ-xtsɤrkɯ ni ci kɤ-kɤ-sɯpa tɕe kɤ-kɤ-tʂɯβ ŋu. ɯ-ɕna kɯ-ɤmtɕoʁ tɕe tu-kɯ-ŋgɤɣ kɯ-fse nɯ tu, ɯ-xsɤrkɯ cho ɯ-xtsɤku ɣɯ kɤ-tʂɯβ nɯ konaʁ xtsa ɣɯ ɯ-tʂɯβ cho mɤ-naχtɕɯɣ, nɯ ɯ-ro nɯ ra konaʁxtsa cho lonba naχtɕɯɣ. palaxtsa nɯ mɤ-mpɕɤr ri ngɯt, kɯ-rɤma kɤ-ŋga nɤtsa.\cmn 
\stylefv{palaxtsa}的鞋边是用没有染过的皮子做成的,脚尖部分没有红皮子。鞋边直接合在一起缝成的,脚尖部分很尖,有钩起来的部分。鞋边和鞋筒连接的缝法和黑皮鞋不一样,其它完全一样。\stylefv{palaxtsa}不美观但很结实,适合干活的人穿。
\end{exemple}
\end{entrée}

\begin{entrée}
\vedette{\hypertarget{Ⓔpaltsaʁ}{\papi{ paltsaʁ}}}\markboth{paltsaʁ}{} (\variante{pɤltsaʁ}) 
\classe{n}
\begin{définition}\fra glaise qui est appliquée sur les plaques de pierres\end{définition}
\begin{définition}\cmn 涂在石板上的稀泥\end{définition}\end{entrée}

\begin{entrée}
\vedette{\hypertarget{Ⓔpandɤɕku}{\papi{ pandɤɕku}}}\markboth{pandɤɕku}{}\classe{n}
\begin{définition}\fra renoncule\end{définition}
\begin{définition}\cmn 毛茛\end{définition}
\begin{exemple}\jya pandɤɕku nɯ sɯjno kɯ-mbɯ-mbɤr ci ŋu, ɯ-ru cho ɯ-jwaʁ ra kɯ-ɤrŋi ŋu, ɯ-zrɤm xtɕi, stɤmku tʂu ɯ-rkɯ tu-ɬoʁ ŋu, ɯ-mɯntoʁ kɯ-qarŋe ŋu, ɯ-ru ɣɯ ɯ-βzɯr lu-ɕe ŋu, tú-wɣ-ndza tɕe mɤrtsaβ tɕe núndʐa pandɤɕku rmi.\cmn 
毛茛是矮小的植物,茎和叶子全是绿色的,根细小。生长在山上的路边上,开黄色花,茎上长出很多菱角。吃起来很辣,所以叫\stylefv{pandɤɕku}(班地的葱)
\end{exemple}
\end{entrée}

\begin{entrée}
\vedette{\hypertarget{Ⓔpandi}{\papi{ pandi}}}\markboth{pandi}{}\classe{n}
\begin{définition}\fra gelugspa\end{définition}
\begin{définition}\cmn 黄教
\begin{déclaration} \étymologie{\papi{panɖita}}\end{déclaration}\end{définition}
\end{entrée}

\begin{entrée}
\vedette{\hypertarget{Ⓔpanja}{\papi{ panja}}}\markboth{panja}{}\classe{n}
\begin{définition}\fra feu (moine)\end{définition}
\begin{définition}\cmn 已故(指和尚)\end{définition}
\end{entrée}

\begin{entrée}
\vedette{\hypertarget{Ⓔpaqe}{\papi{ paqe}}}\markboth{paqe}{}\classe{n}
\begin{définition}\fra purin\end{définition}
\begin{définition}\cmn 猪屎\end{définition}
\begin{relation-sémantique}\confer{
\hyperlink{Ⓔpaʁ}{\textit{ \papi{paʁ}}}
}\end{relation-sémantique}
\begin{relation-sémantique}\confer{
\hyperlink{Ⓔtɯ-qe}{\textit{ \papi{tɯ-qe}}}
}\end{relation-sémantique}\end{entrée}

\begin{entrée}
\vedette{\hypertarget{Ⓔparɕaŋ}{\papi{ parɕaŋ}}}\markboth{parɕaŋ}{}
\classe{n}
\begin{définition}\fra xylographe\end{définition}
\begin{définition}\cmn 印版
\begin{déclaration} \étymologie{\papi{par.ɕiŋ}}\end{déclaration}\end{définition}
\begin{exemple}\jya parɕaŋ pɯ-rkaz-a\cmn 我刻了印版\end{exemple}\end{entrée}

\begin{entrée}
\vedette{\hypertarget{Ⓔpaʁ}{\papi{ paʁ}}}\markboth{paʁ}{}\classe{n}
\begin{définition}\fra porc\end{définition}
\begin{définition}\cmn 猪
\begin{déclaration} \étymologie{\papi{pʰag}}\end{déclaration}\end{définition}
\begin{exemple}\jya paʁ nɯ kha pjɯ-kɤ-nɯ-χsu ŋu tɕe, ɯ-ɕa kɤ-ndza ɯ-spa ŋu, paʁ nɯ ɯ-rme kɯ-ɲaʁ tu, kɯ-ɤkhra tu, kɯ-ɣɯrni tu, kɯ-wɣrum tu, tsuku ɯ-xtu kɯ-wɣrum ɯ-ro ra kɯ-ɲaʁ tu, ɯ-mi kɯ-wɣrum tu, ɯ-jme ɯ-ku kɯ-wɣrum tu, ɯ-laz tɯ-snaʁ kɯ-wɣrum tu, paʁ kɯ mɤ-ndze ʑo me, tɯ-xpa ʁnɯ-xpa jamar cho-χsu tɕe kɤ-ntɕha tu-βze ŋu, paʁ ɯ-pɯ ta-ndo tɕe, kɯβdɤ-sla tɕe chɯ-rɤpɯ ŋu, ɯ-pɯ nɯ tɯ-ɣjɤn na-lɤt tɕe, kɯ-dɤn sqi sqaptɯɣ jamar kɯ-tu tu, ʁnɯz jamar ma kɯ-me tu, paʁɕa nɯ ɯ-ɕɤmi ɕa cho ɯ-rnom ɯ-ɕa nɯ stu kɯ-mɯm ŋu. paʁɕa laʑu chɯ́-wɣ-tɕɤt khɯ, tɤ-ŋgɤr pjɯ́-wɣ-qaʁ khɯ, tɕhi kɯ-ra tú-wɣ-stu khɯ.\cmn 猪是自己在家里喂的,可以用来吃肉。猪有黑毛的,有花毛的,有红毛的,有白毛的。有些腹部是白色的,其它的腹部是黑色的。有的脚是白的,有的尾端白,有的额头上有白点。猪什么都吃,喂了一两年就可以宰了。猪怀小猪要四个月才下,每胎多的有十一头,少的只有两头。猪肉是大腿上和排骨最香。可以弄成腊肉,也可以剥成猪膘,根据需要怎么弄都行。\end{exemple}
\begin{relation-sémantique}\confer{
\hyperlink{Ⓔpaqe}{\textit{ \papi{paqe}}}
}\end{relation-sémantique}
\begin{relation-sémantique}\confer{
\hyperlink{Ⓔpaskɤɣ}{\textit{ \papi{paskɤɣ}}}
}\end{relation-sémantique}
\begin{relation-sémantique}\confer{
\hyperlink{Ⓔpɤjŋgɯ}{\textit{ \papi{pɤjŋgɯ}}}
}\end{relation-sémantique}\end{entrée}

\begin{entrée}
\vedette{\hypertarget{Ⓔpaʁɕɤɣ}{\papi{ paʁɕɤɣ}}}\markboth{paʁɕɤɣ}{}\classe{n}
\begin{définition}\fra Ephedra sinica\end{définition}
\begin{définition}\cmn 草麻黄\end{définition}
\begin{exemple}\jya paʁɕɤɣ nɯ kɤntɕhɯ-tɯphu tu, si ci tu, sɯjno ci tu. si nɯ wuma sthɯci mɤ-mbro, ɕɤɣ ɯ-ŋgɯ tɯ-tɯphu ŋu. ɯ-jwaʁ nɯ kɯ-ɤrqhi ju-kɯ-ru tɕe, kɯ-ɤndɯndo kɯ-fse ŋu, kɯ-ɤrmbat tɕe, tɕe kɯ-ɤndɯndo maʁ, ɯ-ru ɕɤɣ cho naχtɕɯɣ, ɯ-rtaʁ wuma ʑo xtɯt, lu-rɲɟi mɤ-cha, ɯ-jwaʁ nɯ qartsɯmɤftɕar ʑo ɯ-tɯ-ɤrŋi kɯ nɤmar ʑo, pjɯ́-wɣ-sɤkhɯ tɕe, ɯ-di ɕɤɣ ɣɯ sthɯci mɤ-mɯm. paʁɕɤɣ li sɯjno ci tu tɕe, tɯ-ji ɯ-ŋgɯ tɤ-rɤku ɯ-rca tu-ɬoʁ ŋu, ɯ-zrɤm xtɕi, kɤ-phɯt mbat, ɯ-ru me, ɯ-jwaʁ nɯ taqaβ kɯ-fse tɕe, kɯ-rɲɟi ŋu, ɯ-rtsɤɣ tu. ɯ-rtsɤɣ ɯ-taʁ ɯ-mɯntoʁ ɲɯ-βze cha, ɯ-di ɯʑoz ci tu.\cmn 
\stylefv{paʁɕɤɣ}有几种,有一种是树,另一种是草。树的那种长得不太高,是柏树的一种。叶子从远方看好像是连在一起的,近看却不是连在一起的。树干和柏树的一样,枝桠很短,长不长,叶子一年四季的都是油绿色的。熏的时候,味道不如柏树香。另一种\stylefv{paʁɕɤɣ} 是草,和庄稼长在一起,根小,容易拔掉,没有茎,叶子像针一样,很长,有节。节上开花。有异样的味道。
\end{exemple}
\end{entrée}

\begin{entrée}
\vedette{\hypertarget{Ⓔpaʁɕi}{\papi{ paʁɕi}}}\markboth{paʁɕi}{}\classe{n}
\begin{définition}\fra pomme\end{définition}
\begin{définition}\cmn 苹果\end{définition}
\end{entrée}

\begin{entrée}
\vedette{\hypertarget{Ⓔpaʁɕi}{\papi{ paʁɕi}}}\markboth{paʁɕi}{}\classe{n}
\begin{définition}\fra pomme\end{définition}
\begin{définition}\cmn 苹果\end{définition}
\end{entrée}

\begin{entrée}
\vedette{\hypertarget{Ⓔpaʁɟu}{\papi{ paʁɟu}}}\markboth{paʁɟu}{}
\classe{n}
\begin{définition}\fra verrat\end{définition}
\begin{définition}\cmn 种公猪\end{définition}\end{entrée}

\begin{entrée}
\vedette{\hypertarget{Ⓔpaʁkɤjlɤβ}{\papi{ paʁkɤjlɤβ}}}\markboth{paʁkɤjlɤβ}{}\classe{n}
\begin{définition}\fra tête de cochon\end{définition}
\begin{définition}\cmn 猪头(把骨头取出而剩下的皮肉缝成筒形,两只耳朵各一头,鼻子在中间)\end{définition}\end{entrée}

\begin{entrée}
\vedette{\hypertarget{Ⓔpaʁkhar}{\papi{ paʁkhar}}}\markboth{paʁkhar}{}\classe{n}
\begin{définition}\fra enclos à cochons\end{définition}
\begin{définition}\cmn 猪圈
\begin{déclaration} \étymologie{\papi{ⁿkʰor}}\end{déclaration}\end{définition}\end{entrée}

\begin{entrée}
\vedette{\hypertarget{Ⓔpaʁkho}{\papi{ paʁkho}}}\markboth{paʁkho}{}\classe{n}
\begin{définition}\fra porcherie\end{définition}
\begin{définition}\cmn 猪的房间\end{définition}\end{entrée}

\begin{entrée}
\vedette{\hypertarget{Ⓔpaʁlu}{\papi{ paʁlu}}}\markboth{paʁlu}{}\classe{n}
\begin{définition}\fra année du porc\end{définition}
\begin{définition}\cmn 猪年\end{définition}
\end{entrée}

\begin{entrée}
\vedette{\hypertarget{Ⓔpaʁmu}{\papi{ paʁmu}}}\markboth{paʁmu}{}\classe{n}
\begin{définition}\fra truie\end{définition}
\begin{définition}\cmn 母猪\end{définition}
\end{entrée}

\begin{entrée}
\vedette{\hypertarget{Ⓔpaʁmurɯ}{\papi{ paʁmurɯ}}}\markboth{paʁmurɯ}{}\classe{n}
\begin{définition}\fra os de cochon conservés dans l'intestin\end{définition}
\begin{définition}\cmn 猪骨头(装在大肠里保存)\end{définition}
\end{entrée}

\begin{entrée}
\vedette{\hypertarget{Ⓔpaʁndza}{\papi{ paʁndza}}}\markboth{paʁndza}{}\classe{n}
\begin{définition}\fra nourriture du cochons\end{définition}
\begin{définition}\cmn 猪食\end{définition}
\end{entrée}

\begin{entrée}
\vedette{\hypertarget{Ⓔpaʁtsa}{\papi{ paʁtsa}}}\markboth{paʁtsa}{}\classe{n}
\begin{définition}\fra porcelet\end{définition}
\begin{définition}\cmn 猪崽\end{définition}
\end{entrée}

\begin{entrée}
\vedette{\hypertarget{Ⓔpaʁtsarna}{\papi{ paʁtsarna}}}\markboth{paʁtsarna}{}\classe{n}
\begin{définition}\fra une plante\end{définition}
\begin{définition}\cmn 植物的一种\end{définition}
\begin{exemple}\jya paʁtsarna nɯ sɯjno kɯ-tshu tsa ci ŋu, ɯ-ru cho ɯ-jwaʁ ɯ-ru nɯ ra wuma ʑo mpɯ, wuma ʑo arŋi. ɯ-ru ɯ-ŋgɯ kɯ-so ŋu. ɯ-jwaʁ nɯ kɤ-βzɯr χsɯm ŋu tɕe, paʁrna tsa fse, tɕe núndʐa paʁtsarna ɲɯ-rmi. ɯ-ru ɯ-kɤχcɤl zɯ tɤ-fkɯm kɯ-fse ci ɯ-ku kɯ-ɤmtɕoʁ ci tu-ɬoʁ. ʑɯrɯʑɤri tɕe ɲɯ-mbaʁ tɕe ɯ-mɯntoʁ ɯ-spa kɯɕnom kɯ-fse tu-βze tɕe ɯ-mɯntoʁ kɯ-qarŋe tu-oʑɯrja ɲɯ-ŋu. ki sɯjno ki ɲɯ-tshu ɲɯ-mpɯ tɕe paʁ cho nɯŋa jla ra kɤsɯfse ɲɯ-rga-nɯ, tɯrme kɤ-ndza mɤ-sna.\cmn 
\stylefv{paʁtsarna} 是一种又肥又嫩的草。茎和叶子的很嫩,很绿。茎是空心的。叶子是三角形的,有点像猪耳朵,所以称它为\stylefv{paʁtsarna}。茎的顶端上长出像口袋一样,上面尖的东西,逐渐破开了以后,里面漏出穗状的花,花是黄色的,成行的排列在茎上。
\end{exemple}
\end{entrée}

\begin{entrée}
\vedette{\hypertarget{Ⓔpaʁtshi}{\papi{ paʁtshi}}}\markboth{paʁtshi}{}\classe{n}
\begin{définition}\fra nourriture pour cochon\end{définition}
\begin{définition}\cmn 喂猪的\end{définition}
\end{entrée}

\begin{entrée}
\vedette{\hypertarget{Ⓔpaʁzwamɯntoʁ}{\papi{ paʁzwamɯntoʁ}}}\markboth{paʁzwamɯntoʁ}{}
\classe{n}
\begin{définition}\fra pissenlit\end{définition}
\begin{définition}\cmn 蒲公英\end{définition}
\begin{exemple}\jya paʁzwa mɯntoʁ nɯ ɯ-jwaʁ lu-orɤrkhɯrkhe, ɯ-ku lu-omtɕoʁ ŋu, ɯ-spjɯŋ nɯ qhoʁsjɯβ ŋu. ɯ-spjɯŋ tu-ɬoʁ tɕe, ɯ-kɤχcɤl ri ɲɯ-rɯmɯntoʁ ŋu. ɯ-mɯntoʁ nɯ-rom tɕe, tu-ɣɤmɯt tɕe qale rca ju-nɯɕe ŋu. ki sɯjno ki tɤ-ɕqhe smɤn, tɯ-ɕɣa kɯ-mŋɤm smɤn, tshɤtʂot smɤn ŋu\cmn 蒲公英叶子不连贯而尖,茎是空心的。茎长出来时,顶上开花。当花凋谢的时候,一吹花絮就顺着风飞走。这种草有治咳、治牙痛,退烧等作用。\end{exemple}\end{entrée}

\begin{entrée}
\vedette{\hypertarget{Ⓔpaskɤɣ}{\papi{ paskɤɣ}}}\markboth{paskɤɣ}{}
\classe{n}
\begin{définition}\fra porc à engraisser\end{définition}
\begin{définition}\cmn 催肥的猪;肥猪\end{définition}
\begin{relation-sémantique}\confer{
\hyperlink{Ⓔpaʁ}{\textit{ \papi{paʁ}}}
}\end{relation-sémantique}
\begin{relation-sémantique}\confer{
\hyperlink{Ⓔskɤɣ}{\textit{ \papi{skɤɣ}}}
}\end{relation-sémantique}\end{entrée}

\begin{entrée}
\vedette{\hypertarget{Ⓔpasrɯ}{\papi{ pasrɯ}}}\markboth{pasrɯ}{}
\classe{n}
\begin{définition}\fra ivoire\end{définition}
\begin{définition}\cmn 象牙
\begin{déclaration} \étymologie{\papi{ba.so}}\end{déclaration}\end{définition}\end{entrée}

\begin{entrée}
\vedette{\hypertarget{Ⓔpaχpatʂɯβ}{\papi{ paχpatʂɯβ}}}\markboth{paχpatʂɯβ}{}\classe{n}
\begin{définition}\fra type de pas d'aiguille\end{définition}
\begin{définition}\cmn 缝针的方法\end{définition}\end{entrée}

\begin{entrée}
\vedette{\hypertarget{Ⓔpɤβ}{\papi{ pɤβ}}}\markboth{pɤβ}{}\classe{n}
\begin{définition}\fra natte\end{définition}
\begin{définition}\cmn 席子\end{définition}
\end{entrée}

\begin{entrée}
\vedette{\hypertarget{Ⓔpɤɕɤt}{\papi{ pɤɕɤt}}}\markboth{pɤɕɤt}{}
\classe{n}
\begin{définition}\fra pouce\end{définition}
\begin{définition}\cmn 一寸\end{définition}\end{entrée}

\begin{entrée}
\vedette{\hypertarget{Ⓔpɤɕthɤɣ}{\papi{ pɤɕthɤɣ}}}\markboth{pɤɕthɤɣ}{}
\classe{n}
\begin{définition}\fra sangle ventrale\end{définition}
\begin{définition}\cmn 马肚带\end{définition}\end{entrée}

\begin{entrée}
\vedette{\hypertarget{Ⓔpɤjka}{\papi{ pɤjka}}}\markboth{pɤjka}{}
\classe{n}
\begin{définition}\fra courge\end{définition}
\begin{définition}\cmn 瓜子的一种【南瓜】
\begin{déclaration} \étymologie{\papi{\stylefn{北瓜}}}\end{déclaration}\end{définition}
\begin{exemple}\jya pɤjka nɯ lu-kɤ-ji ci ŋu, tɤ-rɤku rca pjɯ́-wɣ-ji ŋu. sɤtɕha kɯ-mpja tsa ɬoʁ, rpɣo pɕoʁ kɯ-mɯɕtaʁ tu-ɬoʁ mɤ-cha. ɯ-jwaʁ ɯ-qhu pɕoʁ ɯ-mdzu dɤn tɕe ɯ-rʁom ʑo tɤjmɤɣ ɯ-sɤ-χtɕi pe, ɯ-jwaʁ wxti, ɯ-βzɯr kɯmŋu tu, ɯ-ru kɯ-zri tsa jɯ-ɕe ŋu, ɯ-taʁ tu-ɬoʁ mɤ-cha, ɯ-thoʁ pjɯ-ɤɲɟoʁ tɕe jɯ-ɕe ŋu, si cho sɯjno ru na-tɯɣ tɕe, li tu-ortɯ-rtɤβ tɕe ɯ-taʁ tu-ɕe cha. ɯ-ru ɯ-taʁ kɯnɤ ɯ-mdzu tu. tɕe ɯ-ru ɯ-taʁ tɕe, ɯ-jwaʁ ku-ndzoʁ tɕe, ɯ-jwaʁ χsɯm jamar nɯ-ɬoʁ tɕe, ɯ-mɯntoʁ ku-ndzoʁ ŋu. ɯ-mɯntoʁ ʁnɯ-tɯphu tɕe, tɯ-tɯphu nɯ ɯ-mat chɯ-βze mɤ-cha tɕe, nɯ phu ŋu tu-ti-nɯ ŋgrɤl, li ci tɯ-tɯphu nɯ, ɯ-mat chɯ-βze cha tɕe, nɯ mu ŋu tu-ti-nɯ ŋu, ɯ-mɯntoʁ ɯ-mdoʁ kɯ-qarŋɯ-rŋe ʑo ŋu, kɯ-ɤrɯlaba ŋu, ɯ-βzɯr kɯmŋu tu. pɤjka ɯ-mat kɯ-wxtɯ-wxti chɯ-βze cha, kɯ-ɤrtɯm tɕe kɯ-zri ŋu. sɤtɕha ɯ-taʁ pjɯ-kɯ-ru nɯ aqarŋɯrŋe, tɯ-mɯ ɯ-pɕoʁ tu-kɯ-ru nɯ, ldʑaŋnaʁ ŋu. pɤjka stu kɯ-wxti nɯ ɣnɤsqi tɯ-rpa jarma chɯ-βze cha. thɯ-do tɕe ɯ-rqhu nɯ rko tɕe ɯ-ɕa ɣɯ ɯ-ŋgɯ ɯ-rɣi chɯ-fka ŋu. pɤjka nɯ kɤ-ndza sna, tɯ-mgo zmɤrɤβ ŋu. thɯ-do tɕe, nɯ fse kú-wɣ-nɯ-sqa tɕe, ɯ-lpɤɣ tú-wɣ-ndza tɕe mɯm.\cmn 南瓜是自己种的一种植物,是种在庄稼地里,种在气候比较温和的地方,山上气候冷的地方的能生长。叶子的背面有很多小刺,很粗糙,可以用来刷洗菌子。叶子大,有个棱角,茎可以长得比较长,不是往上长,是爬在地上的。遇到了树干或草茎就会缠着往上长。茎上也有刺。茎上长出三四片叶子的时候就开花。有两种花,一种是不能结果的,一种是不能结果的,有人说它是公的,另一种是可以结果的,有人说它是母的。花是大黄色的,形状像喇叭,有五个棱角。南瓜能结很大的果实,是椭圆形的。朝地的那面是淡黄色是,朝天的那面是深绿色的。最大的南瓜有20多斤左右。老了皮子变硬,种子也就饱满了。南瓜可以吃,是一种菜。老南瓜煮以后,成块的吃了很好吃。\end{exemple}
\begin{exemple}\jya pɤjka tɤ-kɯ-ɬoʁ nɯ, ɯ-mɯntoʁ ku-ndzoʁ ŋu, ɯ-mɯntoʁ kɤ-ndzoʁ tɕe, tɤ-pɤtso (ɯ-ɕki) ``ma-ɕ-kɤ-tɯ-sɯjaʁndze ma mɯntoʁ mbɯt" tu-kɯ-ti ŋu\cmn 白瓜开花的时候,给孩子说:“你不要指,不然它的花会掉下来”\end{exemple}\end{entrée}

\begin{entrée}
\vedette{\hypertarget{Ⓔpɤjkhu}{\papi{ pɤjkhu}}}\markboth{pɤjkhu}{}\classe{adv}
\begin{définition}\fra encore\end{définition}
\begin{définition}\cmn 还;先;暂时\end{définition}
\begin{exemple}\jya pɤjkhu a-ŋga tu-ŋge-a, pɤjkhu a-rte tu-nɤrte-a ra\cmn 我还要穿衣服,还要戴帽子\end{exemple}
\begin{exemple}\jya @shiwuhao ri lɤ-ari tɕe, pɤjkhu mɯ-thɯ-nɯɣe\cmn 他十五号就去了,还没有回来\end{exemple}
\begin{exemple}\jya pɤjkhu nɤʑo thɯ-ɣi ra\cmn 暂时先要你一个人下来\end{exemple}
\begin{exemple}\jya pɤjkhu aʑo mɤ-ndzu-a\cmn 我还没有准备好\end{exemple}\end{entrée}

\begin{entrée}
\vedette{\hypertarget{Ⓔpɤjmu}{\papi{ pɤjmu}}}\markboth{pɤjmu}{}\classe{n}
\begin{définition}\fra beimu\end{définition}
\begin{définition}\cmn 贝母
\begin{déclaration} \étymologie{\papi{\stylefn{贝母}}}\end{déclaration}\end{définition}
\begin{exemple}\jya pɤjmu ruŋgu\cmn 贝母生长的草山(最寒冷的地方)\end{exemple}
\end{entrée}

\begin{entrée}
\vedette{\hypertarget{Ⓔpɤjmɤtsɯŋgɤr}{\papi{ pɤjmɤtsɯŋgɤr}}}\markboth{pɤjmɤtsɯŋgɤr}{}
\classe{n}
\begin{définition}\fra lard au dessus de la queue\end{définition}
\begin{définition}\cmn 猪尾部划出来的膘\end{définition}\end{entrée}

\begin{entrée}
\vedette{\hypertarget{Ⓔpɤjŋgɯ}{\papi{ pɤjŋgɯ}}}\markboth{pɤjŋgɯ}{}\classe{n}
\begin{définition}\fra auge\end{définition}
\begin{définition}\cmn 猪槽\end{définition}
\begin{définition}\jya \end{définition}
\begin{relation-sémantique}\confer{
\hyperlink{Ⓔpaʁ}{\textit{ \papi{paʁ}}}
}\end{relation-sémantique}\end{entrée}

\begin{entrée}
\vedette{\hypertarget{Ⓔpɤjpe}{\papi{ pɤjpe}}}\markboth{pɤjpe}{}\classe{n}
\begin{définition}\fra pâte aplatie\end{définition}
\begin{définition}\cmn 面块\end{définition}
\begin{exemple}\jya kupa pɤjpe\cmn 面条\end{exemple}\end{entrée}

\begin{entrée}
\vedette{\hypertarget{Ⓔpɤkhije}{\papi{ pɤkhije}}}\markboth{pɤkhije}{}\classe{intj}
\begin{définition}\fra attend un peu\end{définition}
\begin{définition}\cmn 等一下\end{définition}
\end{entrée}

\begin{entrée}
\vedette{\hypertarget{Ⓔpɤkɯ}{\papi{ pɤkɯ}}}\markboth{pɤkɯ}{}\classe{n}
\begin{définition}\fra viande du tronc du cochon\end{définition}
\begin{définition}\cmn 猪的排骨\end{définition}\end{entrée}

\begin{entrée}
\vedette{\hypertarget{Ⓔpɤlɤjlɯz}{\papi{ pɤlɤjlɯz}}}\markboth{pɤlɤjlɯz}{}
\classe{n}
\begin{définition}\fra méthode\end{définition}
\begin{définition}\cmn 办法\end{définition}\end{entrée}

\begin{entrée}
\vedette{\hypertarget{Ⓔpɤlɤtɕɯ}{\papi{ pɤlɤtɕɯ}}}\markboth{pɤlɤtɕɯ}{}
\classe{n}
\begin{définition}\fra momo au beurre\end{définition}
\begin{définition}\cmn 酥油馍馍\end{définition}\end{entrée}

\begin{entrée}
\vedette{\hypertarget{Ⓔpɤliaʁ}{\papi{ pɤliaʁ}}}\markboth{pɤliaʁ}{}\classe{n}
\begin{définition}\fra rouleau à pâtisserie\end{définition}
\begin{définition}\cmn 擀面棍\end{définition}\end{entrée}

\begin{entrée}
\vedette{\hypertarget{Ⓔpɤŋɤxɕaj}{\papi{ pɤŋɤxɕaj}}}\markboth{pɤŋɤxɕaj}{}\classe{n}
\begin{définition}\fra espèce d'herbe\end{définition}
\begin{définition}\cmn 草的一种(猪可以吃的)\end{définition}
\begin{relation-sémantique}\confer{
\hyperlink{ⒺxɕajⒽ1}{\textit{ \papi{xɕaj1}}}
}\end{relation-sémantique}\end{entrée}

\begin{entrée}
\vedette{\hypertarget{Ⓔpɤqa}{\papi{ pɤqa}}}\markboth{pɤqa}{}\classe{n}
\begin{définition}\fra pattes de cochon farcies\end{définition}
\begin{définition}\cmn 有肉馅的猪脚\end{définition}
\begin{relation-sémantique}\confer{
\hyperlink{Ⓔpaʁ}{\textit{ \papi{paʁ}}}
}\end{relation-sémantique}
\begin{relation-sémantique}\confer{
\hyperlink{Ⓔtɯ-qa}{\textit{ \papi{tɯ-qa}}}
}\end{relation-sémantique}\end{entrée}

\begin{entrée}
\vedette{\hypertarget{Ⓔpɤrmɤloŋ}{\papi{ pɤrmɤloŋ}}}\markboth{pɤrmɤloŋ}{}
\classe{n}
\begin{définition}\fra gaspillage\end{définition}
\begin{définition}\cmn 浪费\end{définition}
\begin{exemple}\jya pɤrmɤloŋ mɯ-nɯ-ari\cmn 没有浪费\end{exemple}
\begin{exemple}\jya pɤrmɤloŋ ma-nɯ-tɯ-sɯxɕe\cmn 你别浪费\end{exemple}\end{entrée}

\begin{entrée}
\vedette{\hypertarget{Ⓔpɕaʁ}{\papi{ pɕaʁ}}}\markboth{pɕaʁ}{}
\classe{n}
\begin{définition}\fra prosternation\end{définition}
\begin{définition}\cmn 礼拜
\begin{déclaration}\use{一般跟\stylefv{βzu}连用,表示“磕头”的意思}\end{déclaration}\end{définition}
\begin{exemple}\jya sroŋma ɯ-ʁɤri pɕaʁ lɤ-βzu-t-a\cmn 我在护神前做了礼拜\end{exemple}
\begin{relation-sémantique}\confer{
\hyperlink{Ⓔrɤpɕaʁ}{\textit{ \papi{rɤpɕaʁ}}}
}\end{relation-sémantique}\end{entrée}

\begin{entrée}
\vedette{\hypertarget{Ⓔpɕawtsɯ}{\papi{ pɕawtsɯ}}}\markboth{pɕawtsɯ}{}\classe{n}
\begin{définition}\fra billet de banque\end{définition}
\begin{définition}\cmn 钞票
\begin{déclaration} \étymologie{\papi{\stylefn{票子}}}\end{déclaration}\end{définition}\end{entrée}

\begin{entrée}
\vedette{\hypertarget{Ⓔpɕintɕɤt}{\papi{ pɕintɕɤt}}}\markboth{pɕintɕɤt}{}\classe{adv}
\begin{définition}\fra à partir de ce moment là\end{définition}
\begin{définition}\cmn 从此以后
\begin{déclaration} \étymologie{\papi{pʰʲin.tɕʰad}}\end{déclaration}\end{définition}
\begin{exemple}\jya nɯ ɯ-qhu pɕintɕɤt tɕe\cmn 从此以后\end{exemple}
\end{entrée}

\begin{entrée}
\vedette{\hypertarget{Ⓔpɕirɯ}{\papi{ pɕirɯ}}}\markboth{pɕirɯ}{}\classe{n}
\begin{définition}\fra arrière de la selle\end{définition}
\begin{définition}\cmn 后鞍桥
\begin{déclaration} \étymologie{\papi{pʰʲi.ru}}\end{déclaration}\end{définition}\end{entrée}

\begin{entrée}
\vedette{\hypertarget{Ⓔpɕiz}{\papi{ pɕiz}}}\markboth{pɕiz}{}\classe{vt}
\paradigme{\textit{dir :} \jya nɯ-}
\paradigme{\textit{dir :} \jya \_}
\begin{définition}\fra essuyer\end{définition}
\begin{définition}\cmn 擦
\begin{déclaration} \étymologie{\papi{pʰʲis}}\end{déclaration}\end{définition}
\begin{exemple}\jya @zhuozi nɯ-pɕiz\cmn 擦一下桌子\end{exemple}
\begin{exemple}\jya nɤ-rŋa pɯ-pɕiz\cmn 擦一下脸\end{exemple}
\begin{exemple}\jya nɤ-ŋga nɯ-pɕiz\cmn 擦一下衣服\end{exemple}
\begin{exemple}\jya nɤ-ɕnaβ nɯ-pɕiz\cmn 擦一下鼻涕\end{exemple}
\begin{exemple}\jya kɯki kɯ a-jaʁ mɯ́j-pɕiz\cmn 我的手用这个擦不到\end{exemple}\end{entrée}

\begin{entrée}
\vedette{\hypertarget{Ⓔpɕizgɤkɯm}{\papi{ pɕizgɤkɯm}}}\markboth{pɕizgɤkɯm}{}
\classe{n}
\begin{définition}\fra grande porte\end{définition}
\begin{définition}\cmn 大门
\begin{déclaration} \étymologie{\papi{pʰʲi.sgo}}\end{déclaration}\end{définition}\end{entrée}

\begin{entrée}
\vedette{\hypertarget{Ⓔpɕoʁβʑi}{\papi{ pɕoʁβʑi}}}\markboth{pɕoʁβʑi}{}\classe{n}
\begin{définition}\fra dans toutes les directions\end{définition}
\begin{définition}\cmn 四周
\begin{déclaration} \étymologie{\papi{pʰʲogs.bʑi}}\end{déclaration}\end{définition}\end{entrée}

\begin{entrée}
\vedette{\hypertarget{Ⓔpɕɯɣ}{\papi{ pɕɯɣ}}}\markboth{pɕɯɣ}{}\classe{idph.1}
\begin{définition}\fra bruit de l'eau qu'on jette, bruit d'une vague qui se brise sur le bord du fleuve\end{définition}
\begin{définition}\cmn 泼水的声音、波浪撞水边的声音\end{définition}
\begin{exemple}\jya tɯ-ci pɕɯɣ ʑo ta-lɤt\cmn 他扑一声把水泼了\end{exemple}\begin{sous-entrée}
\vedette{\hypertarget{}{\papi{ ɣɤpɕɯlɯɣ}}}\markboth{ɣɤpɕɯlɯɣ}{}
\begin{définition}\fra asperger dans tous les sens\end{définition}
\begin{définition}\cmn 泼过去泼过来\end{définition}
\begin{exemple}\jya khɯtsa ɯ-ŋgɯ tɯ-ci tɯ-asɯ-ndo tɕe, dal tsa tɤ-ŋke ma ɲɯ-ɣɤpɕɯlɯɣ ʑo tɕe jit\cmn 你在碗里端水,走路慢一点,不然就会到处流出来\end{exemple}
\end{sous-entrée}\begin{sous-entrée}
\vedette{\hypertarget{}{\papi{ sɤpɕɯlɯɣ}}}\markboth{sɤpɕɯlɯɣ}{}\classe{vt}
\begin{définition}\fra faire tomber de l'eau dans tous les sens (en transportant un bol)\end{définition}
\begin{définition}\cmn 端水的时候走路不稳当,把水泼过去泼过来\end{définition}
\begin{exemple}\jya tɯ-ci ɲɯ-sɤpɕɯɣlɯɣ\cmn 他把水泼过去泼过来\end{exemple}
\end{sous-entrée}\end{entrée}

\begin{entrée}
\vedette{\hypertarget{Ⓔpe}{\papi{ pe}}}\markboth{pe}{}
\classe{vs}
\paradigme{\textit{dir :} \jya tɤ-}\acception{1}
\begin{définition}\fra bien\end{définition}
\begin{définition}\cmn 好\end{définition}\acception{2}
\begin{définition}\fra bon\end{définition}
\begin{définition}\cmn 善良
\begin{déclaration}\use{\stylefv{mɤ-kɯ-pe kɯ-me ʑo}表示“没有什么原因就……”的意思}\end{déclaration}\end{définition}
\begin{exemple}\jya nɯ ɲɯ-pe mɤ-kɯ-pe maŋe\cmn 那很好,没有错误\end{exemple}
\begin{exemple}\jya nɤʑɯɣ nɤ-kha ra ɯ-ɲɯ-pe-nɯ?\cmn 你家的人好不好?\end{exemple}
\begin{exemple}\jya nɯ kɯ-fse kɯ-pe me\cmn 那是最好的了,没有比这个好的\end{exemple}
\begin{exemple}\jya tɤ-pɤtso pɯ-ŋke ri mɤ-kɯ-pe kɯ-me ʑo pɯ-ndʐaβ\cmn 小孩子走路的时候,没有什么原因就摔倒了\end{exemple}
\begin{exemple}\jya mɯ-tu-tɯ-pe mɤ-βdi ma tɯ-kɤ-stu pjɯ-tu ra\cmn 如果你哪一天遇到不幸的事情,就要有解决的办法\end{exemple}
\begin{exemple}\jya nɯ ɕɯŋgɯ aʑo mɯ-tɤ-pe-a zɯ, ɯʑo wuma ʑo pɯ-stu\cmn 以前我遇到问题的时候,他帮了我很多\end{exemple}
\begin{relation-sémantique}\antonyme{
\hyperlink{Ⓔŋɤn}{\textit{ \papi{ŋɤn}}}
}\end{relation-sémantique}\begin{sous-entrée}
\vedette{\hypertarget{}{\papi{ nɤpe}}}\markboth{nɤpe}{}\classe{vt}
\paradigme{\textit{dir :} \jya tɤ-}\acception{1}
\begin{définition}\fra aimer\end{définition}
\begin{définition}\cmn 喜欢\end{définition}\acception{2}
\begin{définition}\fra être content de\end{définition}
\begin{définition}\cmn 因为……觉得高兴\end{définition}
\end{sous-entrée}\begin{sous-entrée}
\vedette{\hypertarget{}{\papi{ sɤpe}}}\markboth{sɤpe}{}\classe{vt}
\begin{définition}\fra améliorer\end{définition}
\begin{définition}\cmn 做得更好\end{définition}
\end{sous-entrée}\end{entrée}

\begin{entrée}
\vedette{\hypertarget{Ⓔpɣa}{\papi{ pɣa}}}\markboth{pɣa}{}\classe{n}
\begin{définition}\fra oiseau\end{définition}
\begin{définition}\cmn 鸟\end{définition}\end{entrée}

\begin{entrée}
\vedette{\hypertarget{Ⓔpɣaʁ}{\papi{ pɣaʁ}}}\markboth{pɣaʁ}{}\classe{vt}\acception{1}
\paradigme{\textit{dir :} \jya \_}
\begin{définition}\fra retourner\end{définition}
\begin{définition}\cmn 打翻\end{définition}
\begin{exemple}\jya nɤ-ŋga ɲɤ-tɯ-pɣaʁ\cmn 你把衣服穿反了\end{exemple}
\begin{exemple}\jya sɯpɣo tha-pɣaʁ\cmn 他把柴垛弄翻了\end{exemple}
\begin{exemple}\jya a-ŋga kɤ-ŋga ɲɤ-nɯ-pɣaʁ-a ri, mɯ-pjɤ-sɯχsal-a\cmn 我把衣服穿反了但是没有注意\end{exemple}\acception{2}
\begin{définition}\fra labourer\end{définition}
\begin{définition}\cmn 耕(地)\end{définition}
\begin{exemple}\jya tɯji la-pɣaʁ, lɤ-pɣaʁ-a\cmn 他耕了地,我耕了地\end{exemple}
\begin{exemple}\jya stɤmku la-pɣaʁ\cmn 他耕了草坪\end{exemple}\acception{3}
\begin{définition}\fra faire s'effondrer (mur)\end{définition}
\begin{définition}\cmn 推倒(墙)\end{définition}
\begin{exemple}\jya znde tha-pɣaʁ\cmn 他把墙推倒了\end{exemple}\acception{4}
\begin{définition}\fra ouvrir (couvercle)\end{définition}
\begin{définition}\cmn 揭开\end{définition}
\begin{exemple}\jya ɯ-fkaβ tɤ-pɣaʁ-a (=tɤ-mɟa-t-a)\cmn 我揭开了盖子\end{exemple}\begin{sous-entrée}
\vedette{\hypertarget{}{\papi{ rɤpɣaʁ}}}\markboth{rɤpɣaʁ}{}\classe{vi}
\begin{définition}\ 
\begin{déclaration}\grammar{apass}\end{déclaration}\end{définition}
\begin{définition}\fra défricher\end{définition}
\begin{définition}\cmn 开荒\end{définition}
\end{sous-entrée}\begin{sous-entrée}
\vedette{\hypertarget{}{\papi{ ʑɣɤpɣaʁ}}}\markboth{ʑɣɤpɣaʁ}{}\classe{vi}
\paradigme{\textit{dir :} \jya \_}
\begin{définition}\ 
\begin{déclaration}\grammar{refl}\end{déclaration}\end{définition}
\begin{définition}\fra se retourner\end{définition}
\begin{définition}\cmn 翻身\end{définition}
\begin{exemple}\jya aʑo stɤmku ri pɯ-rŋgɯ-a tɕe thɯ-ʑɣɤpɣaʁ-a ma a-pa qajɯ ɣɤʑu\cmn 我躺在草坪的时候翻了身,因为我的下面有虫子\end{exemple}
\begin{exemple}\jya thɯ-ʑɣɤpɣaʁ nɤ thɯ-ʑɣɤpɣaʁ\cmn 他(在地上)滚动\end{exemple}
\begin{relation-sémantique}\confer{
\hyperlink{Ⓔmbɣaʁ}{\textit{ \papi{mbɣaʁ}}}
}\end{relation-sémantique}
\begin{relation-sémantique}\confer{
\hyperlink{Ⓔapɣaʁsci}{\textit{ \papi{apɣaʁsci}}}
}\end{relation-sémantique}
\end{sous-entrée}\end{entrée}

\begin{entrée}
\vedette{\hypertarget{Ⓔpɣatɤste}{\papi{ pɣatɤste}}}\markboth{pɣatɤste}{}\classe{n}
\begin{définition}\fra espèce de rapace\end{définition}
\begin{définition}\cmn 猛禽的一种\end{définition}
\begin{exemple}\jya pɣa tɤ-ste nɯ qale kɯ-tshi cho naχtɕɯɣ, qajdo sɤznɤ xtɕi, ɯ-βri kɯ-pɣi ŋu, rɯdaʁ kɯ-xtɕi ra tu-ndze ŋgrɤl, tɤ-rɤku tu-ndze mɤ-ŋgrɤl.\cmn 
\stylefv{pɣatɤste}和\stylefv{qalekɯtshi}相似,比乌鸦小,身子是灰色的,吃小动物,不吃粮食。
\end{exemple}
\end{entrée}

\begin{entrée}
\vedette{\hypertarget{Ⓔpɣɤjmɤt}{\papi{ pɣɤjmɤt}}}\markboth{pɣɤjmɤt}{}\classe{n}
\begin{définition}\fra type d'herbe\end{définition}
\begin{définition}\cmn 草的一种\end{définition}\end{entrée}

\begin{entrée}
\vedette{\hypertarget{Ⓔpɣɤkhɯ}{\papi{ pɣɤkhɯ}}}\markboth{pɣɤkhɯ}{}
\classe{n}
\begin{définition}\fra hibou\end{définition}
\begin{définition}\cmn 猫头鹰\end{définition}
\begin{exemple}\jya pɣɤkhɯ nɯ pɣa ci ŋu, wxti tsa qandʑɣi jamar tu, lɤŋɤtʂɤ-tɯrpa jamar zɣɯt, ɯ-phoŋbu nɯ kɯ-pɣi tɕe kɯ-qandʐi tsa ŋu, ɯ-ku nɯ lɯlu ku tsa fse ri pɣa ŋu tɕe ɯ-mtsioʁ tu, ɯ-mtsioʁ moŋtaʁ ɣɯ ɯ-ku nɯ lu-ŋgɤɣ tsa ɲɯ-ŋu, ɯ-mɲaʁ nɯ ɕɤr tɕe ɲɯ-mto ma sŋi tɕe mɯ́j-mto tɕe ɯʑo ɯ-kɤ-ndza kɯ-ɕar nɯ ɕɤr ʁɟa ju-ɬoʁ ɲɯ-ŋu, tɕe βʑɯ tu-ndze, qaɲi tu-ndze, qala kɯ-fse nɯ ra tu-ndze ɲɯ-ŋu. ɕɤr tɕe tu-mbri ŋgrɤl, tɕe `uhu' tu-ti ɲɯ-ŋu, tɕe tsuku tɯrme wuma ʑo kɯ-nɯrmɯ kɯ-nɯmtɕi tu-nɯ tɕe ɕɤr ɲɯ-kɯ-rɤma-nɯ pɣɤkhɯ tu-sɤrmi-nɯ ŋgrɤl.\cmn 猫头鹰是一种鸟,和隼一样大,可以达到五六斤重,身子灰里带黑。头有点像猫,但有鸟喙,因为是鸟。上鸟喙的顶端是钩着的。眼睛只有夜里才看见东西,白天看不见。它只有晚上才出来觅食,吃老鼠、鼹鼠、兔子等动物。它会在夜里叫,叫声是‘uhu’。所以人们把喜欢晚收工,早出工,晚上工作的人称作“猫头鹰”。\end{exemple}\end{entrée}

\begin{entrée}
\vedette{\hypertarget{Ⓔpɣɤloʁ}{\papi{ pɣɤloʁ}}}\markboth{pɣɤloʁ}{}
\classe{n}
\begin{définition}\fra nid\end{définition}
\begin{définition}\cmn 鸟巢\end{définition}\end{entrée}

\begin{entrée}
\vedette{\hypertarget{Ⓔpɣɤlpɣɤl}{\papi{ pɣɤlpɣɤl}}}\markboth{pɣɤlpɣɤl}{}\classe{idph.2}
\begin{définition}\fra brillant\end{définition}
\begin{définition}\cmn 形容光芒耀眼,金光闪闪的样子\end{définition}
\begin{exemple}\jya χɕɤl ɲɯ-nɤmbju pɣɤlpɣɤl ʑo\cmn 玻璃金光闪闪\end{exemple}
\begin{exemple}\jya ʁmbɣi pɣɤlpɣɤl ʑo ɲɤ-ɬoʁ\cmn 太阳出来了,光芒四射\end{exemple}\begin{sous-entrée}
\vedette{\hypertarget{}{\papi{ pɣɤlnɤpɣɤl}}}\markboth{pɣɤlnɤpɣɤl}{}\classe{idph.3}\acception{1}
\begin{définition}\fra scintillant\end{définition}
\begin{définition}\cmn 形容一闪一闪的样子\end{définition}\acception{2}
\begin{définition}\fra en marchant\end{définition}
\begin{définition}\cmn 形容一步一步走动的样子\end{définition}
\end{sous-entrée}\begin{sous-entrée}
\vedette{\hypertarget{}{\papi{ pɣɤlpɣɤl nɤ pɣɤlpɣɤl}}}\markboth{pɣɤlpɣɤl nɤ pɣɤlpɣɤl}{}\classe{idph.3}
\begin{définition}\fra en courant\end{définition}
\begin{définition}\cmn 形容步子迈得快的样子\end{définition}
\begin{exemple}\jya pɣɤlpɣɤl nɤ pɣɤlpɣɤl ʑo pjɤ-ɣi\cmn 她跑回来了\end{exemple}
\end{sous-entrée}\end{entrée}

\begin{entrée}
\vedette{\hypertarget{Ⓔpɣɤmbri}{\papi{ pɣɤmbri}}}\markboth{pɣɤmbri}{}
\classe{n}
\begin{définition}\fra chant d'oiseau\end{définition}
\begin{définition}\cmn 鸟的叫声\end{définition}
\begin{relation-sémantique}\confer{
\hyperlink{ⒺmbriⒽ1}{\textit{ \papi{mbri1}}}
}\end{relation-sémantique}
\begin{relation-sémantique}\confer{
\hyperlink{Ⓔpɣa}{\textit{ \papi{pɣa}}}
}\end{relation-sémantique}\end{entrée}

\begin{entrée}
\vedette{\hypertarget{Ⓔpɣɤmuj}{\papi{ pɣɤmuj}}}\markboth{pɣɤmuj}{}\classe{n}
\begin{définition}\fra plumes d'oiseau\end{définition}
\begin{définition}\cmn 羽毛\end{définition}
\begin{relation-sémantique}\confer{
\hyperlink{Ⓔtɤ-muj}{\textit{ \papi{tɤ-muj}}}
}\end{relation-sémantique}
\begin{relation-sémantique}\confer{
\hyperlink{Ⓔpɣa}{\textit{ \papi{pɣa}}}
}\end{relation-sémantique}
\end{entrée}

\begin{entrée}
\vedette{\hypertarget{Ⓔpɣɤɲaʁ}{\papi{ pɣɤɲaʁ}}}\markboth{pɣɤɲaʁ}{}\classe{n}
\begin{définition}\fra faisan (pucrasia macrolopha)\end{définition}
\begin{définition}\cmn 勺鸡\end{définition}
\begin{relation-sémantique}\confer{
\hyperlink{Ⓔɲaʁ}{\textit{ \papi{ɲaʁ}}}
}\end{relation-sémantique}
\end{entrée}

\begin{entrée}
\vedette{\hypertarget{Ⓔpɣɤrnoʁ}{\papi{ pɣɤrnoʁ}}}\markboth{pɣɤrnoʁ}{}\classe{n}
\begin{définition}\fra une espèce de champignon\end{définition}
\begin{définition}\cmn 【鸡油菌】\end{définition}
\begin{exemple}\jya pɣɤrnoʁ nɯ si kɯ-xtɕi tsa ɯ-taʁ ku-ndzoʁ tɕe kɯ-qarŋe tsa ŋu, kɤ-ndza sna\cmn 鸡油菌是长在小树上的一种菌子,呈黄色,可以吃。\end{exemple}
\begin{relation-sémantique}\confer{
\hyperlink{Ⓔtɯ-rnoʁ}{\textit{ \papi{tɯ-rnoʁ}}}
}\end{relation-sémantique}
\end{entrée}

\begin{entrée}
\vedette{\hypertarget{Ⓔpɣɤrtsɤɣ}{\papi{ pɣɤrtsɤɣ}}}\markboth{pɣɤrtsɤɣ}{}
\classe{n}
\begin{définition}\fra maladie de la peau de la main\end{définition}
\begin{définition}\cmn 疣\end{définition}\end{entrée}

\begin{entrée}
\vedette{\hypertarget{Ⓔpɣɤtɕɯtɤŋgɤr}{\papi{ pɣɤtɕɯtɤŋgɤr}}}\markboth{pɣɤtɕɯtɤŋgɤr}{}\classe{n}
\begin{définition}\fra une plante\end{définition}
\begin{définition}\cmn 植物的一种\end{définition}
\begin{exemple}\jya pɣɤtɕɯ tɤŋgɤr nɯ stɤmku tu-ɬoʁ ŋu. ɯ-jwaʁ nɯ kɯ-ɤrtɯm ŋu, tɕe ɯ-thoʁ pjɯ-ɤɲɟoʁ ʑo ɲɯ-ŋu. ɯ-jwaʁ ʁnɯz ma maŋe, kɯ-jɯ-jaʁ ɲɯ-ŋu, tɕe ɲɯ-ndoʁ. pjɯ́-wɣ-qrɯt tɕe ɯ-ci kɯ-ɤrɤmtʂɯmtʂaj kɯ-fse ɲɯ-ɬoʁ ɲɯ-ŋu. ɯ-ru tu-ɬoʁ tɕe, ɯ-ru ʁnɯ-tɣa jamar tɤ-zri tɕe, ɯ-mɯntoʁ ɲɯ-ɬoʁ ɲɯ-ŋu. ɯ-mɯntoʁ pa zɯ tɤ-fkɯm kɯ-fse kɯ-xtɕɯ-xtɕi ɣɤʑɯ tɕe, ɯ-mɯntoʁ ɯ-pa pjɤ-ɴqoʁ, ɯ-mŋu ɯ-taʁ tu-ru ɲɯ-ŋu. ɯ-ŋgɯ tɯ-ci ɲɯ-mtshɤt tɕe, ɲɯ-chi. ɯ-mɯntoʁ nɯ kɯ-qarŋe tɕe, staχpɯ ɯ-mɯntoʁ tsa ɲɯ-fse. ɯ-mɯntoʁ kɯ-dɤn tu-oʑɯrja ɲɯ-ŋu.\cmn 
\stylefv{pɣɤtɕɯ tɤ-ŋgɤr}生长在草地上。叶子是圆形的,贴在地面上。只有两片叶子,又厚又脆,掰开时里面流出有粘性的液体。茎长出两拃的时候就开花。花的下面有一个像口袋一样的小东西,吊在花的下面,口朝上。里面装满水,很甜。花是黄色的,像豌豆的花一样,花很多,成行地长在茎上。
\end{exemple}\end{entrée}

\begin{entrée}
\vedette{\hypertarget{Ⓔpɣɤzraʁ}{\papi{ pɣɤzraʁ}}}\markboth{pɣɤzraʁ}{}
\classe{n}
\begin{définition}\fra espèce d'oiseau\end{définition}
\begin{définition}\cmn 一种鸟\end{définition}
\begin{exemple}\jya pɣɤzraʁ nɯ pɣa kɯ-xtɕi tsa ci ŋu, ɯ-phoŋbu nɯ tɤŋkhɯt jamar ma me, ɯ-βri kɯ-pɣi ɯ-taʁ kɯ-ɲaʁ kɯ-ɤkhra ŋu, zgo ɯ-χcɤl ɕaŋtaʁ sɤtɕha kɯ-mbro ku-rɤʑi ŋu, qajɯ kɯ-fse, sɯmat kɯ-fse tu-ndze ŋgrɤl. kɯmŋu kɯtʂɤɣ jamar tɯtɯrca ku-rɤʑi-nɯ ŋu. qartsɯmɤftɕar kɤ-mto tu ɕti. ɯ-mtɕhi nɯ kɯ-ɲaʁ ŋu, ɯ-mi nɯ ra kɯ-wɣrum tɕe kɯ-pɣi kɯ-fse ŋu. tɤ-rɤku ɯ-taʁ rɯŋɯŋɤn mɤ-ŋgrɤl\cmn 
\stylefv{pɣɤzraʁ}是一种比较小的鸟,整个身子只有拳头那么大,身子是灰色上面有黑条纹。栖息在半山以上,海拔高的地方,吃虫子和野果等,五六只成群一起生活。一年四季都可以看到它。嘴是黑色的,脚是白里带灰色。一般不会破坏庄稼。
\end{exemple}\end{entrée}

\begin{entrée}
\vedette{\hypertarget{Ⓔpɣi}{\papi{ pɣi}}}\markboth{pɣi}{}\classe{vs}
\paradigme{\textit{dir :} \jya nɯ-}
\begin{définition}\fra gris\end{définition}
\begin{définition}\cmn 灰\end{définition}
\end{entrée}

\begin{entrée}
\vedette{\hypertarget{Ⓔpɣo}{\papi{ pɣo}}}\markboth{pɣo}{}\classe{vt}
\paradigme{\textit{dir :} \jya lɤ-}
\begin{définition}\fra faire de la ficelle en roulant dans les mains (dans le sens des aiguilles d'une montre)\end{définition}
\begin{définition}\cmn 捻线(顺时针)\end{définition}
\begin{exemple}\jya smɤɣ lɤ-pɣo-t-a\cmn 我捻了羊毛\end{exemple}
\begin{exemple}\jya srɯn lɤ-pɣo-t-a\cmn 我捻了棉花\end{exemple}
\begin{exemple}\jya tɤ-rme lɤpɣo-t-a\cmn 我捻了毛\end{exemple}\begin{sous-entrée}
\vedette{\hypertarget{}{\papi{ nɯɣɯpɣo}}}\markboth{nɯɣɯpɣo}{}\classe{vs}
\begin{définition}\fra facile à enfiler\end{définition}
\begin{définition}\cmn 容易捻\end{définition}
\end{sous-entrée}\end{entrée}

\begin{entrée}
\vedette{\hypertarget{Ⓔpɣotaʁ}{\papi{ pɣotaʁ}}}\markboth{pɣotaʁ}{}\classe{n}
\begin{définition}\fra tissage\end{définition}
\begin{définition}\cmn 吊线和织布\end{définition}
\begin{relation-sémantique}\confer{
\hyperlink{Ⓔpɣo}{\textit{ \papi{pɣo}}}
}\end{relation-sémantique}
\begin{relation-sémantique}\confer{
\hyperlink{ⒺtaʁⒽ1}{\textit{ \papi{taʁ1}}}
}\end{relation-sémantique}\end{entrée}

\begin{entrée}
\vedette{\hypertarget{Ⓔphu}{\papi{ phu}}}\markboth{phu}{}\classe{vs}
\begin{définition}\fra fier\end{définition}
\begin{définition}\cmn 有自信\end{définition}
\begin{exemple}\jya ɯ-sɯm ɲɯ-phu\cmn 他很有自信\end{exemple}\end{entrée}

\begin{entrée}
\vedette{\hypertarget{Ⓔpha}{\papi{ pha}}}\markboth{pha}{}\classe{n}
\begin{définition}\fra entier, complet\end{définition}
\begin{définition}\cmn 整个
\begin{déclaration} \étymologie{\papi{pʰal(tɕʰer)}}\end{déclaration}\end{définition}
\begin{exemple}\jya pha ɯ-ku ʑo cho-wɣrum\cmn 整个头都白了\end{exemple}
\begin{exemple}\jya pha ɯ-phoŋbu ʑo ɲɯ-mŋɤm\cmn 他全身痛\end{exemple}
\begin{exemple}\jya pha kɤntɕhaʁ ʑo ɕ-tɤ-khat-a\cmn 我走了遍了整个城市\end{exemple}
\end{entrée}

\begin{entrée}
\vedette{\hypertarget{Ⓔphaloŋri}{\papi{ phaloŋri}}}\markboth{phaloŋri}{}\classe{adv}
\begin{définition}\fra pleins d'endroits (dans la montagne)\end{définition}
\begin{définition}\cmn 很多地方(山川)\end{définition}
\begin{exemple}\jya phaloŋri ʑo ɕ-to-khɤt/ɕ-to-ŋke\cmn 他走遍了所有的山川\end{exemple}\end{entrée}

\begin{entrée}
\vedette{\hypertarget{Ⓔphama}{\papi{ phama}}}\markboth{phama}{}\classe{n}
\begin{définition}\fra parents\end{définition}
\begin{définition}\cmn 父母
\begin{déclaration} \étymologie{\papi{pʰa.ma}}\end{déclaration}\end{définition}
\end{entrée}

\begin{entrée}
\vedette{\hypertarget{Ⓔphantsɯt}{\papi{ phantsɯt}}}\markboth{phantsɯt}{}\classe{n}
\begin{définition}\fra assiette\end{définition}
\begin{définition}\cmn 盘子
\begin{déclaration} \étymologie{\papi{\stylefn{盘子}}}\end{déclaration}\end{définition}\end{entrée}

\begin{entrée}
\vedette{\hypertarget{Ⓔphaʁ}{\papi{ phaʁ}}}\markboth{phaʁ}{}
\classe{vt}
\paradigme{\textit{dir :} \jya nɯ-}
\paradigme{\textit{dir :} \jya thɯ-}
\paradigme{\textit{dir :} \jya tɤ-}
\begin{définition}\fra couper\end{définition}
\begin{définition}\cmn 劈;剖\end{définition}
\begin{exemple}\jya si nɯ-phaʁ-a\cmn 我劈了木头\end{exemple}
\begin{exemple}\jya ɕoŋtɕa nɯ-phaʁ-a\cmn 我劈了柴\end{exemple}
\begin{exemple}\jya tɤ-fkɯm nɯ-phaʁ-a\cmn 我把口袋弄坏了(装的东西太多就破了)\end{exemple}
\begin{exemple}\jya tɯ-ŋga thɯ-phaʁ-a\cmn 我撕了衣服\end{exemple}
\begin{exemple}\jya tɤ-ri thɯ-phaʁ-a\cmn 我撕了线\end{exemple}
\begin{exemple}\jya paχɕi ɯ-sɤ-phaʁ maŋe, mbrɯtɕɯ a-pɯ-tu tɕe pe ri\cmn 没有东西切苹果,有刀子就好了\end{exemple}
\begin{relation-sémantique}\confer{
\hyperlink{Ⓔmbaʁ}{\textit{ \papi{mbaʁ}}}
}\end{relation-sémantique}\end{entrée}

\begin{entrée}
\vedette{\hypertarget{Ⓔphaʁlu}{\papi{ phaʁlu}}}\markboth{phaʁlu}{}\classe{n}
\begin{définition}\fra année du porc\end{définition}
\begin{définition}\cmn 猪年
\begin{déclaration} \étymologie{\papi{pʰag.lo}}\end{déclaration}\end{définition}
\end{entrée}

\begin{entrée}
\vedette{\hypertarget{Ⓔphaʁɲɤl}{\papi{ phaʁɲɤl}}}\markboth{phaʁɲɤl}{}\classe{n}
\begin{définition}\fra fait de s'allonger sur le côté\end{définition}
\begin{définition}\cmn 半身躺着\end{définition}
\begin{relation-sémantique}\confer{
\hyperlink{Ⓔtɯ-phaʁ}{\textit{ \papi{tɯ-phaʁ}}}
}\end{relation-sémantique}
\begin{relation-sémantique}\confer{
\hyperlink{Ⓔnɯphaʁɲɤl}{\textit{ \papi{nɯphaʁɲɤl}}}
}\end{relation-sémantique}\end{entrée}

\begin{entrée}
\vedette{\hypertarget{Ⓔphaʁrgot}{\papi{ phaʁrgot}}}\markboth{phaʁrgot}{}\classe{n}
\begin{définition}\fra sanglier\end{définition}
\begin{définition}\cmn 野猪
\begin{déclaration} \étymologie{\papi{pʰag.rgod}}\end{déclaration}\end{définition}\end{entrée}

\begin{entrée}
\vedette{\hypertarget{Ⓔphaʁrzi}{\papi{ phaʁrzi}}}\markboth{phaʁrzi}{}\classe{n}
\begin{définition}\fra brosse à dent traditionnelle en poil de cochon\end{définition}
\begin{définition}\cmn 用猪鬃毛作成的牙刷
\begin{déclaration} \étymologie{\papi{pʰag.ze}}\end{déclaration}\end{définition}\end{entrée}

\begin{entrée}
\vedette{\hypertarget{Ⓔphaʁzlasqamŋu}{\papi{ phaʁzlasqamŋu}}}\markboth{phaʁzlasqamŋu}{}\classe{n}
\begin{définition}\fra deux semaines\end{définition}
\begin{définition}\cmn 半个月
\begin{déclaration} \étymologie{\papi{zla}}\end{déclaration}\end{définition}\end{entrée}

\begin{entrée}
\vedette{\hypertarget{Ⓔphaʁzoŋ}{\papi{ phaʁzoŋ}}}\markboth{phaʁzoŋ}{}\classe{n}
\begin{définition}\fra sûtra lu pour les cochons lorsqu'ils sont malades\end{définition}
\begin{définition}\cmn 为猪念的经
\begin{déclaration} \étymologie{\papi{pʰag.??}}\end{déclaration}\end{définition}
\end{entrée}

\begin{entrée}
\vedette{\hypertarget{ⒺphɤβⒽ2}{\papi{ phɤβ}}}\markboth{phɤβ}{}\homonyme{2}
\classe{n}
\begin{définition}\fra ferment de vin\end{définition}
\begin{définition}\cmn 酿酒用的曲子\end{définition}\end{entrée}

\begin{entrée}
\vedette{\hypertarget{ⒺphɤβⒽ1}{\papi{ phɤβ}}}\markboth{phɤβ}{}\homonyme{1}
\classe{vt}\acception{1}
\paradigme{\textit{dir :} \jya pɯ-}
\paradigme{\textit{dir :} \jya thɯ-}
\begin{définition}\fra abaisser\end{définition}
\begin{définition}\cmn 弄低,降下来\end{définition}
\begin{exemple}\jya a-ku pɯ-phaβ-a ma nɯ-rpe-a ɲɯ-ŋu\cmn 我低了头,因为差一点撞了\end{exemple}
\begin{exemple}\jya nɤ-ku ko-tɯ-nɯ-rpu-t tɕe pjɤ-tɯ-phɤβ pjɤ-ra\cmn 你撞了头,你本来应该低头\end{exemple}
\begin{exemple}\jya nɤ-ku pɯ-phɤβ ma tɯ-nɯ-rpe\cmn 你低头,小心撞了\end{exemple}
\begin{exemple}\jya ki laχtɕha ɯ-phɯ ɲɯ-wxti, ɯ-koŋ pɯ-phɤβ\cmn 这个东西很贵,价格卖便宜一点吧\end{exemple}
\begin{relation-sémantique}\confer{
\hyperlink{ⒺmbɤβⒽ2}{\textit{ \papi{mbɤβ2}}}
}\end{relation-sémantique}\acception{2}
\paradigme{\textit{dir :} \jya pɯ-}
\begin{définition}\fra peigner\end{définition}
\begin{définition}\cmn 梳理(头发)\end{définition}
\begin{exemple}\jya nɤ-kɤrme pɯ-phɤβ\cmn 梳理一下头发\end{exemple}\acception{3}
\paradigme{\textit{dir :} \jya nɯ-}
\begin{définition}\fra appliquer une couche de graisse (sur les poteries qui viennent de sortir du four)\end{définition}
\begin{définition}\cmn 上油(给出窑的坛子)\end{définition}
\begin{exemple}\jya tɯfcɤr nɯ-phɤβ-i\cmn 我们给刚出窑的坛子上了一层油\end{exemple}\end{entrée}

\begin{entrée}
\vedette{\hypertarget{Ⓔphɤn}{\papi{ phɤn}}}\markboth{phɤn}{}\classe{vs}
\paradigme{\textit{dir :} \jya tɤ-}
\begin{définition}\fra avoir un (bon) effet\end{définition}
\begin{définition}\cmn 有效果,起作用
\begin{déclaration} \étymologie{\papi{pʰan}}\end{déclaration}\end{définition}
\begin{exemple}\jya pɯ-me mɤ-phɤn ma ɣɯ-ndza ra, pɯ-tu mɤ-phɤn ma ɣɯ-ɕtʂat ra\cmn 粮食再少也不能不吃,粮食再多也不能不节约\end{exemple}\begin{sous-entrée}
\vedette{\hypertarget{}{\papi{ ɣɤphɤn}}}\markboth{ɣɤphɤn}{}\classe{vt}
\begin{définition}\fra donner un effet\end{définition}
\begin{définition}\cmn 让……起作用\end{définition}
\begin{exemple}\jya kɤ-rɤβzjoz tu-kɯ-stu tɕe kɤ-spa ɣɯ ɯ-kɯ-ɣɤphɤn ŋu\cmn 只有努力才有可能学会\end{exemple}
\end{sous-entrée}\end{entrée}

\begin{entrée}
\vedette{\hypertarget{Ⓔphɤnba}{\papi{ phɤnba}}}\markboth{phɤnba}{}\classe{n}
\begin{définition}\fra bienfait\end{définition}
\begin{définition}\cmn 好处\end{définition}
\begin{exemple}\jya kɯrɯ skɤt pjɯ-βzjoz-a tɕe a-phɤnba tu\cmn 学藏语对我是有用的\end{exemple}\end{entrée}

\begin{entrée}
\vedette{\hypertarget{Ⓔphɤnthoʁ}{\papi{ phɤnthoʁ}}}\markboth{phɤnthoʁ}{}\classe{n}
\begin{définition}\fra avantage\end{définition}
\begin{définition}\cmn 好处
\begin{déclaration} \étymologie{\papi{pʰan.tʰog}}\end{déclaration}\end{définition}
\begin{relation-sémantique}\confer{
\hyperlink{Ⓔphɤn}{\textit{ \papi{phɤn}}}
}\end{relation-sémantique}\end{entrée}

\begin{entrée}
\vedette{\hypertarget{Ⓔphɤr}{\papi{ phɤr}}}\markboth{phɤr}{}
\classe{vt}
\paradigme{\textit{dir :} \jya nɯ-}
\paradigme{\textit{dir :} \jya kɤ-}
\begin{définition}\fra secouer, agiter pour faire tout tomber d'un récipient\end{définition}
\begin{définition}\cmn 抖掉;倒完\end{définition}
\begin{exemple}\jya nɤki tɤ-fkɯm ɯ-ŋgɯ kɤ-phɤr\cmn 你(把青稞)抖进口袋里吧\end{exemple}
\begin{exemple}\jya ɯ-kɯr ɲɤ-phɤr\cmn 我张开了嘴巴(目瞪口呆)\end{exemple}
\begin{exemple}\jya pjɤ-nɤscɤr tɕe, ɯ-kɯr ɲɤ-phɤr\cmn 我被吓到了,就张开了嘴巴\end{exemple}
\begin{relation-sémantique}\confer{
\hyperlink{Ⓔsɤphɤr}{\textit{ \papi{sɤphɤr}}}
}\end{relation-sémantique}
\begin{relation-sémantique}\confer{
\hyperlink{ⒺmbɤrⒽ2}{\textit{ \papi{mbɤr2}}}
}\end{relation-sémantique}\end{entrée}

\begin{entrée}
\vedette{\hypertarget{Ⓔphɤri}{\papi{ phɤri}}}\markboth{phɤri}{}\classe{adv}
\begin{définition}\fra de l'autre côté\end{définition}
\begin{définition}\cmn 对面;对岸
\begin{déclaration} \étymologie{\papi{pʰa.rol}}\end{déclaration}\end{définition}
\begin{exemple}\jya kɯchu phɤri, ndɯchu phɤri\end{exemple}
\end{entrée}

\begin{entrée}
\vedette{\hypertarget{Ⓔphɤrikɯnɤwu}{\papi{ phɤrikɯnɤwu}}}\markboth{phɤrikɯnɤwu}{}
\classe{n}
\begin{définition}\fra une plante\end{définition}
\begin{définition}\cmn 植物的一种\end{définition}
\begin{exemple}\jya phɤri kɯnɤwu nɯ kha ɯ-rkɯ sɯjno kɯ-dɤn kɯ-fse fsapaʁ ɯ-ɣli kɯ-dɤn kɯ-fse ra tu-ɬoʁ ŋu. tɯrme tɯ-fsu jamar tu-βze cha. ɯ-jwaʁ nɯ ɕɤɣ ɯ-jwaʁ ɯ-tshɯɣa fse ri nɯ sɤz rɟum, ɯ-ru nɯ khro mɤ-jpum, ɯ-ru kɯ-zɯ-zri tɤ-ɬoʁ tɕe, ɯ-taʁ ɯ-mɯntoʁ ku-ndzoʁ ŋu, tɕe ɯ-mɯntoʁ ɯ-tshɯɣa nɯ stoʁ ɯ-mɯntoʁ fse, ɯ-mdoʁ nɯ nɤmkha ɯ-mdoʁ kɯ-ɤɲaʁndzɯm tsa ŋu. ɯʑo sɯjno nɯ tɯ-ɕɣa ɯ-smɤn kɤ-βʑu ɲɯ-khɯ khi, kɯɕɯŋgɯ kɯ-wxti ra kɯ phɤri kɯnɤwu tu-sɤrmi-nɯ pɯ-ŋu. kɤ-ndza mɤ-sna. fsapaʁ ndza kɯnɤ mɤ-sna\cmn 
\stylefv{phɤri kɯnɤwu} 是长在房子周围,草多、牲畜粪多的地方。可以长到人的高度。叶子形状像柏树的叶子,但宽一些,茎不粗,茎长得长了,就在上面开花。花的形状像胡豆的花,是深蓝色的。据说这种草可以作成牙痛的药。以前老人们称它是\stylefv{phɤri kɯnɤwu}(为对面哭的意思)。不能吃,连牲畜都不能吃。
\end{exemple}\end{entrée}

\begin{entrée}
\vedette{\hypertarget{Ⓔphɤrtɕaʁ}{\papi{ phɤrtɕaʁ}}}\markboth{phɤrtɕaʁ}{}
\classe{n}
\begin{définition}\fra support sur lequel on met le bout inférieur du fuseau\end{définition}
\begin{définition}\cmn 用来放纺锤的下一端的托子(一般是碗的底部,或者小石板)\end{définition}\end{entrée}

\begin{entrée}
\vedette{\hypertarget{Ⓔphɤtɕhɯχtɤr}{\papi{ phɤtɕhɯχtɤr}}}\markboth{phɤtɕhɯχtɤr}{}
\classe{n}
\begin{définition}\fra éparpiller partout\end{définition}
\begin{définition}\cmn 撒得一地都是\end{définition}
\begin{exemple}\jya phɤtɕhɯχtɤr ʑo pɯ-ta-t-a (pɯ-lat-a)\cmn 我撒得一地都是了\end{exemple}
\begin{relation-sémantique}\confer{
\hyperlink{Ⓔrɯtɕhɯχtɤr}{\textit{ \papi{rɯtɕhɯχtɤr}}}
}\end{relation-sémantique}\end{entrée}

\begin{entrée}
\vedette{\hypertarget{Ⓔphɣo}{\papi{ phɣo}}}\markboth{phɣo}{}\classe{vi}
\paradigme{\textit{dir :} \jya nɯ-}
\begin{définition}\fra fuir\end{définition}
\begin{définition}\cmn 逃跑\end{définition}
\begin{exemple}\jya khɯna ɣɤʑu tɕe nɯ-phɣo-a\cmn 我看到狗就跑了\end{exemple}\begin{sous-entrée}
\vedette{\hypertarget{}{\papi{ sɯphɣo}}}\markboth{sɯphɣo}{} (\variante{ɕɯphɣo}) \classe{vt}
\paradigme{\textit{dir :} \jya nɯ-}
\begin{définition}\ 
\begin{déclaration}\grammar{caus}\end{déclaration}\end{définition}
\begin{définition}\fra laisser fuir\end{définition}
\begin{définition}\cmn 使逃跑\end{définition}
\begin{exemple}\jya ɯʑo kɤ-phɣo ɲɯ-sɯsɤm ri, mɯ-nɯ-sɯphɣo-t-a\cmn 他想逃跑,但是我没有让他(得逞)\end{exemple}
\begin{relation-sémantique}\confer{
\hyperlink{Ⓔɕphɣo}{\textit{ \papi{ɕphɣo}}}
}\end{relation-sémantique}
\end{sous-entrée}\end{entrée}

\begin{entrée}
\vedette{\hypertarget{Ⓔphima}{\papi{ phima}}}\markboth{phima}{}\classe{n}
\begin{définition}\fra partie extérieur des habits tibétains\end{définition}
\begin{définition}\cmn 衣服面子(藏装)
\begin{déclaration} \étymologie{\papi{pʰʲi.ma}}\end{déclaration}\end{définition}\end{entrée}

\begin{entrée}
\vedette{\hypertarget{Ⓔphuɲi}{\papi{ phuɲi}}}\markboth{phuɲi}{}
\classe{n}
\begin{définition}\fra une espèce d'arbrisseau\end{définition}
\begin{définition}\cmn 灌木的一种\end{définition}
\begin{exemple}\jya phuɲi nɯ si kɯ-mbɯ-mbɤr ci ŋu, ɯ-ru ɣɯ ɯ-mdoʁ nɯ aɣɯrnɯɕɯr, ɯ-rqhu rɕɯβrɕɯβ ʑo pa, ɯ-jwaʁ xtɕi ri jaʁ tsa ɯ-rme kɯ-fse tu. ɯ-mɯntoʁ kɯ-qarŋɯ-rŋe ŋu. ɯ-mnɯ nɯ ɲɯ́-wɣ-phɯt tɕe, kɯ-ndɯβ nɯ ra saχsɯ ɲɯ́-wɣ-βzu sna, kɯ-jndʐɤz nɯ ra zɣɤmbu ɲɯ́-wɣ-βzu sna.\cmn 
\stylefv{phuɲi}是矮小的树种,树干是淡红色,树皮很粗糙(到处都裂开、快要脱落的样子),叶子比较小,但是有点厚,上面有毛。花是淡黄色的。把它的苗拔下后,小的可以作耍把,大的可以作扫把。
\end{exemple}\end{entrée}

\begin{entrée}
\vedette{\hypertarget{Ⓔphoβraŋ}{\papi{ phoβraŋ}}}\markboth{phoβraŋ}{}\classe{n}
\begin{définition}\fra palais\end{définition}
\begin{définition}\cmn 宫殿
\begin{déclaration}\use{古语}\end{déclaration}
\begin{déclaration} \étymologie{\papi{pʰo.braŋ}}\end{déclaration}\end{définition}
\end{entrée}

\begin{entrée}
\vedette{\hypertarget{Ⓔphochi}{\papi{ phochi}}}\markboth{phochi}{}\classe{n}
\begin{définition}\fra chien\end{définition}
\begin{définition}\cmn 公狗
\begin{déclaration} \étymologie{\papi{pho.kʰʲi}}\end{déclaration}\end{définition}
\end{entrée}

\begin{entrée}
\vedette{\hypertarget{Ⓔpholi}{\papi{ pholi}}}\markboth{pholi}{}\classe{n}
\begin{définition}\fra chat\end{définition}
\begin{définition}\cmn 公猫\end{définition}
\end{entrée}

\begin{entrée}
\vedette{\hypertarget{Ⓔphoŋ}{\papi{ phoŋ}}}\markboth{phoŋ}{}\classe{n}
\begin{définition}\fra bouteille\end{définition}
\begin{définition}\cmn 瓶子
\begin{déclaration} \étymologie{\papi{bum.bu}}\end{déclaration}\end{définition}\end{entrée}

\begin{entrée}
\vedette{\hypertarget{Ⓔphoŋnɤphoŋ}{\papi{ phoŋnɤphoŋ}}}\markboth{phoŋnɤphoŋ}{}\classe{idph.3}
\begin{définition}\fra bruit de morceaux de bois qui s'entrechoquent\end{définition}
\begin{définition}\cmn 形容木头撞击的声音\end{définition}
\begin{exemple}\jya kɯm ɯ-taʁ laχtɕha a-kɤ-rpu tɕe phoŋnɤphoŋ ti\cmn 东西撞击在木门上就发出砰砰声\end{exemple}
\begin{relation-sémantique}\synonyme{
 \papi{ɕkhoŋnɤɕkhoŋ}
}\end{relation-sémantique}\end{entrée}

\begin{entrée}
\vedette{\hypertarget{Ⓔphoŋsti}{\papi{ phoŋsti}}}\markboth{phoŋsti}{}
\classe{n}
\begin{définition}\fra bouchon\end{définition}
\begin{définition}\cmn 瓶盖\end{définition}
\begin{relation-sémantique}\confer{
\hyperlink{ⒺstiⒽ1}{\textit{ \papi{sti1}}}
}\end{relation-sémantique}\end{entrée}

\begin{entrée}
\vedette{\hypertarget{Ⓔphoroʁ}{\papi{ phoroʁ}}}\markboth{phoroʁ}{} (\variante{phɤroʁ}) 
\classe{n}
\begin{définition}\fra corbeau (corvus corax)\end{définition}
\begin{définition}\cmn 渡鸦
\begin{déclaration} \étymologie{\papi{pʰo.rog}}\end{déclaration}\end{définition}\end{entrée}

\begin{entrée}
\vedette{\hypertarget{Ⓔphoʁ}{\papi{ phoʁ}}}\markboth{phoʁ}{}
\classe{vt}
\paradigme{\textit{dir :} \jya \_}
\begin{définition}\fra verser une partie\end{définition}
\begin{définition}\cmn (从大袋子)舀出来,分装在(小袋子里)\end{définition}
\begin{exemple}\jya tɯ-ci thɯ-phoʁ, lɤ-phoʁ\cmn 把水舀出来\end{exemple}
\begin{exemple}\jya tɤɕi kɤ-phoʁ\cmn 把青稞分装在小袋子里吧\end{exemple}
\begin{exemple}\jya nɤ-tʂha thɯ-phoʁ-a, kɤ-tshi ma\cmn 我分了一点茶给你,你喝吧\end{exemple}
\begin{relation-sémantique}\confer{
\hyperlink{ⒺciⒽ1}{\textit{ \papi{ci}}}
}\end{relation-sémantique}\end{entrée}

\begin{entrée}
\vedette{\hypertarget{Ⓔphoʁphoʁ}{\papi{ phoʁphoʁ}}}\markboth{phoʁphoʁ}{}
\classe{idph.2}
\begin{définition}\fra solide\end{définition}
\begin{définition}\cmn 结实,安然无恙,衣冠端正\end{définition}
\begin{exemple}\jya laχtɕha ra phoʁphoʁ ʑo ɯ-pɯ to-nɯ-panɯ\cmn 他们很小心地保管了东西\end{exemple}
\begin{exemple}\jya kɯm phoʁphoʁ ʑo ko-nɯ-pa-nɯ\cmn 他们把门关得很紧\end{exemple}
\begin{exemple}\jya tɯ-ŋga phoʁphoʁ ʑo ɲɤ-nɯ-ta (ɯ-kɯ-nɯ-ndo pjɤ-me)\cmn 衣服放在那里很安全(没有人拿)\end{exemple}\begin{sous-entrée}
\vedette{\hypertarget{}{\papi{ phoʁnɤphoʁ}}}\markboth{phoʁnɤphoʁ}{}\classe{idph.3}
\begin{exemple}\jya phoʁnɤphoʁ ko-nɯɕe\cmn 他一点也没有耽搁就回去了\end{exemple}
\end{sous-entrée}\end{entrée}

\begin{entrée}
\vedette{\hypertarget{Ⓔphosɤr}{\papi{ phosɤr}}}\markboth{phosɤr}{}\classe{n}
\begin{définition}\fra jeune garçon\end{définition}
\begin{définition}\cmn 青年\end{définition}
\begin{relation-sémantique}\confer{
 \papi{pho.gsar}
}\end{relation-sémantique}\end{entrée}

\begin{entrée}
\vedette{\hypertarget{Ⓔphot}{\papi{ phot}}}\markboth{phot}{}
\classe{vi}
\paradigme{\textit{dir :} \jya tɤ-}
\begin{définition}\fra oser\end{définition}
\begin{définition}\cmn 敢
\begin{déclaration} \étymologie{\papi{pʰod}}\end{déclaration}\end{définition}
\begin{exemple}\jya jiɕqha nɯ kɤ-saʁndɯ ɲɯ-phot\cmn 他敢打人\end{exemple}
\begin{exemple}\jya ku-ɕe ɲɯ-phot\cmn 他敢去\end{exemple}
\begin{exemple}\jya praʁ ɯ-ku ku-ɕe ɲɯ-phot\cmn 他敢到悬崖去\end{exemple}
\begin{exemple}\jya sɯku tu-ɕe ɲɯ-phot\cmn 他敢到树上\end{exemple}
\begin{exemple}\jya tɕhi pɯ-nɯ-ŋɯ-ŋu phot\cmn 他什么都敢做\end{exemple}
\begin{exemple}\jya ɯ-mɤ-kɤ-phot me\cmn 他什么都敢做\end{exemple}\end{entrée}

\begin{entrée}
\vedette{\hypertarget{Ⓔphozgra}{\papi{ phozgra}}}\markboth{phozgra}{}\classe{n}
\begin{définition}\fra cri (lama)\end{définition}
\begin{définition}\cmn 吼声(喇嘛的)\end{définition}\end{entrée}

\begin{entrée}
\vedette{\hypertarget{Ⓔphrɤβ}{\papi{ phrɤβ}}}\markboth{phrɤβ}{}\classe{idph.1}
\begin{définition}\fra bruit de grains ou de billes qui s'éparpillent sur le sol\end{définition}
\begin{définition}\cmn 很多颗粒一下子撒在地上的声音\end{définition}
\begin{exemple}\jya ɯʑo kɯ stoʁ tɯ-spra to-ndo tɕe phrɤβ ʑo pa-βde\cmn 他抓了一把胡豆一下子扔下去了\end{exemple}\begin{sous-entrée}
\vedette{\hypertarget{}{\papi{ ɣɤphrɤβphrɤβ}}}\markboth{ɣɤphrɤβphrɤβ}{}\classe{vs}
\begin{définition}\ 
\end{définition}
\begin{exemple}\jya sɤrwa chɯ-lɤt tɕe ɲɯ-ɣɤphrɤβphrɤβ ʑo\cmn 冰雹发出噼噼啪啪的声音(连续不停地,很快)\end{exemple}
\end{sous-entrée}\begin{sous-entrée}
\vedette{\hypertarget{}{\papi{ phrɤβnɤphrɤβ}}}\markboth{phrɤβnɤphrɤβ}{}\classe{idph.3}
\begin{exemple}\jya sɤrwa pa-lɤt tɕe, ɯ-zgra phrɤβphrɤβ nɤ phrɤβphrɤβ ʑo ɲɯ-ti\cmn 打了冰雹,发出噼噼啪啪的声音\end{exemple}
\end{sous-entrée}\end{entrée}

\begin{entrée}
\vedette{\hypertarget{Ⓔphrɤl}{\papi{ phrɤl}}}\markboth{phrɤl}{}
\classe{vt}
\paradigme{\textit{dir :} \jya pɯ-}
\begin{définition}\fra expliquer\end{définition}
\begin{définition}\cmn 解释\end{définition}
\begin{exemple}\jya tɯ-rju pɯ-phrɤl\cmn 你解释一下\end{exemple}
\begin{exemple}\jya tɯ-rju pɯ-phral-a tɕe nɤʑo tɯ-sɯz\cmn 你已经解释了,你现在知道\end{exemple}
\begin{exemple}\jya tsuku tɤ-mu kɯ ɯ-rɟit ɯ-ɕki ``ɲɯ-tɯ-nɯkhɤja" to-ti ri, ɯ-rɟit nɯ kɯ ``tu-nɯkhɤja-a maʁ, pjɯ-phral-a ɕti" to-ti\cmn 有些母亲对孩子说“你顶嘴”,而孩子回答:“我不是顶嘴,是在解释”\end{exemple}\end{entrée}

\begin{entrée}
\vedette{\hypertarget{Ⓔphɯ}{\papi{ phɯ}}}\markboth{phɯ}{}\classe{n}
\begin{définition}\fra souffle\end{définition}
\begin{définition}\cmn 吹冷气\end{définition}
\begin{exemple}\jya tɯ-ci ɲɯ-sɤɕke tɕe phɯ tɤ-ti tɕe kɤ-tshi\cmn 水很烫,你吹一下才喝\end{exemple}
\end{entrée}

\begin{entrée}
\vedette{\hypertarget{Ⓔphɯɣ}{\papi{ phɯɣ}}}\markboth{phɯɣ}{}
\classe{vt}
\paradigme{\textit{dir :} \jya pɯ-}
\begin{définition}\fra déployer, ouvrir (parapluie, tente)\end{définition}
\begin{définition}\cmn 撑开\end{définition}
\begin{exemple}\jya @san pɯ-phɯɣ\cmn 你把伞撑开吧\end{exemple}
\begin{exemple}\jya zgɤr pɯ-phɯɣ\cmn 你把帐篷撑开吧\end{exemple}\end{entrée}

\begin{entrée}
\vedette{\hypertarget{Ⓔphɯɣphɯɣ}{\papi{ phɯɣphɯɣ}}}\markboth{phɯɣphɯɣ}{}\classe{n}
\begin{définition}\fra son\end{définition}
\begin{définition}\cmn 麦类的粗糠秕\end{définition}
\end{entrée}

\begin{entrée}
\vedette{\hypertarget{Ⓔphɯl}{\papi{ phɯl}}}\markboth{phɯl}{}
\classe{vt}
\paradigme{\textit{dir :} \jya lɤ-}
\begin{définition}\fra offrir\end{définition}
\begin{définition}\cmn 献给;上供
\begin{déclaration} \étymologie{\papi{pʰul}}\end{déclaration}\end{définition}
\begin{exemple}\jya βlama ɯ-ʁɤri lɤ-phɯl-a\cmn 我献给喇嘛了\end{exemple}
\begin{exemple}\jya tɯjpu lɤ-phɯl-a\cmn 我给他献了粮食\end{exemple}\end{entrée}

\begin{entrée}
\vedette{\hypertarget{Ⓔphɯqha}{\papi{ phɯqha}}}\markboth{phɯqha}{}\classe{n}
\begin{définition}\fra grosse racine\end{définition}
\begin{définition}\cmn 干树根\end{définition}\end{entrée}

\begin{entrée}
\vedette{\hypertarget{Ⓔphɯrɤm}{\papi{ phɯrɤm}}}\markboth{phɯrɤm}{}\classe{n}
\begin{définition}\fra herse\end{définition}
\begin{définition}\cmn 耙\end{définition}
\begin{exemple}\jya phɯrɤm thɯ-rɤɕi-t-a\cmn 我耙了地\end{exemple}
\begin{relation-sémantique}\confer{
\hyperlink{Ⓔrɯphɯrɤm}{\textit{ \papi{rɯphɯrɤm}}}
}\end{relation-sémantique}
\begin{relation-sémantique}\confer{
\hyperlink{Ⓔnɯphɯrɤm}{\textit{ \papi{nɯphɯrɤm}}}
}\end{relation-sémantique}\end{entrée}

\begin{entrée}
\vedette{\hypertarget{Ⓔphɯrkhɯɣ}{\papi{ phɯrkhɯɣ}}}\markboth{phɯrkhɯɣ}{}
\classe{n}
\begin{définition}\fra sac que l'on porte en bandoulière\end{définition}
\begin{définition}\cmn 挎包
\begin{déclaration} \étymologie{\papi{kʰug}}\end{déclaration}\end{définition}
\begin{relation-sémantique}\antonyme{
\hyperlink{Ⓔɕɤkhoz}{\textit{ \papi{ɕɤkhoz}}}
}\end{relation-sémantique}\end{entrée}

\begin{entrée}
\vedette{\hypertarget{Ⓔphɯsɤti}{\papi{ phɯsɤti}}}\markboth{phɯsɤti}{}
\classe{n}
\begin{définition}\fra soufflet\end{définition}
\begin{définition}\cmn 吹火筒\end{définition}
\begin{relation-sémantique}\confer{
\hyperlink{Ⓔti}{\textit{ \papi{ti}}}
}\end{relation-sémantique}\end{entrée}

\begin{entrée}
\vedette{\hypertarget{Ⓔphɯt}{\papi{ phɯt}}}\markboth{phɯt}{}
\classe{vt}\acception{1}
\paradigme{\textit{dir :} \jya lɤ-}
\paradigme{\textit{dir :} \jya pɯ-}
\begin{définition}\fra couper\end{définition}
\begin{définition}\cmn 割;砍\end{définition}
\begin{exemple}\jya si lɤ-phɯt-a\cmn 我砍了树\end{exemple}
\begin{exemple}\jya a-ndzrɯ nɯ-phɯt-a (=nɯ-kraɣ-a, nɯ-ʁndzar-a)\cmn 我剪了指甲\end{exemple}\acception{2}
\paradigme{\textit{dir :} \jya nɯ-}
\begin{définition}\fra arracher, cueillir\end{définition}
\begin{définition}\cmn 拔;扯;摘\end{définition}
\begin{exemple}\jya mɯntoʁ nɯ-phɯt-a\cmn 我摘了花\end{exemple}
\begin{exemple}\jya sɯjno nɯ-phɯt-a, lɤ-phɯt-a\cmn 我拔了草\end{exemple}\acception{3}
\paradigme{\textit{dir :} \jya pɯ-}
\begin{définition}\fra enlever, diminuer\end{définition}
\begin{définition}\cmn 减\end{définition}
\begin{exemple}\jya ɣnɤsqi ɯ-ŋgɯ χsɯm pjɯ́-wɣ-phɯt tɕe, sqaɕnɯz ma ɲɯ-me ŋu\cmn 二十减三等于十七\end{exemple}\acception{4}
\paradigme{\textit{dir :} \jya pɯ-}
\begin{définition}\fra démolir\end{définition}
\begin{définition}\cmn 拆\end{définition}
\begin{exemple}\jya znde pɯ-phɯt-a\cmn 我拆了墙\end{exemple}
\begin{relation-sémantique}\confer{
\hyperlink{Ⓔɣɯsɯphɯt}{\textit{ \papi{ɣɯsɯphɯt}}}
}\end{relation-sémantique}\begin{sous-entrée}
\vedette{\hypertarget{}{\papi{ nɯɣɯphɯt}}}\markboth{nɯɣɯphɯt}{}\classe{vs}
\begin{définition}\ 
\begin{déclaration}\grammar{facil}\end{déclaration}\end{définition}
\begin{définition}\fra être facile à cueillir\end{définition}
\begin{définition}\cmn 容易摘\end{définition}
\end{sous-entrée}\begin{sous-entrée}
\vedette{\hypertarget{}{\papi{ sɤsphɯt}}}\markboth{sɤsphɯt}{}\classe{vs}
\begin{définition}\fra que l'on peut casser (avec les dents)\end{définition}
\begin{définition}\cmn 吃得动\end{définition}
\end{sous-entrée}\begin{sous-entrée}
\vedette{\hypertarget{}{\papi{ sphɯt}}}\markboth{sphɯt}{}\classe{vt}
\begin{définition}\ 
\begin{déclaration}\grammar{habil}\end{déclaration}\end{définition}
\begin{définition}\fra pouvoir couper\end{définition}
\begin{définition}\cmn 切得动;咬得动
\begin{déclaration}\use{一般用于否定式}\end{déclaration}\end{définition}
\begin{exemple}\jya a-ɕɣa kɯ kɤ-ndza ra mɯ́j-sphɯt / a-ɕɣa tɤ-lat-a ri mɯ́j-sphɯt\cmn 我的牙齿吃不动那些食物\end{exemple}
\begin{exemple}\jya mbrɯtɕɯ ki mɯ́j-mtɕoʁ tɕe mɯ́j-sphɯt\cmn 这把刀不锋利,切不动\end{exemple}
\end{sous-entrée}\end{entrée}

\begin{entrée}
\vedette{\hypertarget{Ⓔpijma}{\papi{ pijma}}}\markboth{pijma}{}\classe{n}
\begin{définition}\fra terre jaune (de mauvaise qualité)\end{définition}
\begin{définition}\cmn 黄泥巴\end{définition}\end{entrée}

\begin{entrée}
\vedette{\hypertarget{Ⓔpiwalu}{\papi{ piwalu}}}\markboth{piwalu}{}\classe{n}
\begin{définition}\fra année du rat\end{définition}
\begin{définition}\cmn 鼠年
\begin{déclaration} \étymologie{\papi{bʲi.ba.lo}}\end{déclaration}\end{définition}
\end{entrée}

\begin{entrée}
\vedette{\hypertarget{Ⓔpuj}{\papi{ puj}}}\markboth{puj}{}\classe{n}
\begin{définition}\fra type de sapin\end{définition}
\begin{définition}\cmn 杉树的一种\end{définition}\end{entrée}

\begin{entrée}
\vedette{\hypertarget{Ⓔpja}{\papi{ pja}}}\markboth{pja}{}\classe{n}
\begin{définition}\fra oiseau\end{définition}
\begin{définition}\cmn 鸟
\begin{déclaration}\use{古语}\end{déclaration}\end{définition}
\end{entrée}

\begin{entrée}
\vedette{\hypertarget{Ⓔpjalu}{\papi{ pjalu}}}\markboth{pjalu}{}\classe{n}
\begin{définition}\fra année du coq\end{définition}
\begin{définition}\cmn 鸡年
\begin{déclaration} \étymologie{\papi{bʲa.lo}}\end{déclaration}\end{définition}
\end{entrée}

\begin{entrée}
\vedette{\hypertarget{Ⓔpjɤβlaʁ}{\papi{ pjɤβlaʁ}}}\markboth{pjɤβlaʁ}{}\classe{n}
\begin{définition}\fra tromperie\end{définition}
\begin{définition}\cmn 阴谋;骗局\end{définition}
\begin{exemple}\jya pjɤβlaʁ to-βzu\cmn 他用了阴谋\end{exemple}
\begin{relation-sémantique}\confer{
\hyperlink{Ⓔrɯpjɤβlaʁ}{\textit{ \papi{rɯpjɤβlaʁ}}}
}\end{relation-sémantique}\end{entrée}

\begin{entrée}
\vedette{\hypertarget{Ⓔpjɤl}{\papi{ pjɤl}}}\markboth{pjɤl}{}
\classe{vt}\acception{1}
\paradigme{\textit{dir :} \jya tɤ-}
\begin{définition}\fra contourner\end{définition}
\begin{définition}\cmn 绕过\end{définition}
\begin{exemple}\jya nɯtɕu khɯna ci ɣɤʑu, tɤ-pjal-a\cmn 那边看到有狗,我就绕过去了\end{exemple}
\begin{exemple}\jya mɯ-mɤ-ɲɯ-tɯ-mbɣom nɤ tu-kɯ-nɯ-pjal-a ma kɯsthɯci a-zrɯɣ a-ndʑrɯ ɲɯ-dɤn\cmn 你不急的话你绕过我吧,我身上有这么多的虱子和虮子\end{exemple}\acception{2}
\paradigme{\textit{dir :} \jya kɤ-}
\begin{définition}\fra traverser\end{définition}
\begin{définition}\cmn 穿过
\begin{déclaration} \étymologie{\papi{bʲol}}\end{déclaration}\end{définition}
\begin{exemple}\jya sɯŋgɯ kɤ-pjal-a\cmn 我穿过了森林\end{exemple}\end{entrée}

\begin{entrée}
\vedette{\hypertarget{Ⓔpjɤrgɤt}{\papi{ pjɤrgɤt}}}\markboth{pjɤrgɤt}{}
\classe{n}
\begin{définition}\fra vautour (gyps himalayensis)\end{définition}
\begin{définition}\cmn 高山兀鹫
\begin{déclaration} \étymologie{\papi{bʲa.rgod}}\end{déclaration}\end{définition}
\begin{exemple}\jya pjɤrgɤt nɯ praʁ kɯ-mbro ɯ-ku zɯ ɕ-ku-rma ɲɯ-ŋu, ɯ-ro cho ɯ-ku nɯ kɯ-wɣrum ɲɯ-ŋu, wuma ʑo ɲɯ-wxti. fsusqɯ-tɯrpa jamar tu, kɯ-ɤɲaʁndzɯm tu ɲɯ-ŋgrɤl, pjɤrgɤt ɯ-ŋgɯz kɯ-wxti ɲɯ-ŋu, pjɤrgɤt βlama ŋu nɯ tu-kɯ-ti ɲɯ-ŋgrɤl, fsapaʁ cho tɯrme ɕa tu-ndze ɲɯ-ŋgrɤl, co ra mɤ-ɣi, nɤmkha zɯ tu-nɯpjɤŋkhɤr rga, ɯ-mɤlɤjaʁ ɯ-ndzrɯ ɕɤmiɕtʂɤt fse. ɯ-rme kɯ-wɣrum kɯ-zri ɲɯ-ŋu, ɯ-mtsioʁ nɯ tu-ŋgɤɣ ŋu.\cmn 兀鹫一般栖息在悬崖峭壁上,胸部和头部是白色的,很大,有三十来斤。有的是黑色的,是在兀鹫之中比较大的,据说是它们的喇嘛。它吃牲畜和人肉。不会飞到山谷里,喜欢在空中旋转。爪子像铁钩一样,羽毛白而长,嘴是勾起来的。\end{exemple}\end{entrée}

\begin{entrée}
\vedette{\hypertarget{Ⓔpjɤt}{\papi{ pjɤt}}}\markboth{pjɤt}{}
\classe{vt}
\paradigme{\textit{dir :} \jya nɯ-}
\begin{définition}\fra bourrer (saucisson)\end{définition}
\begin{définition}\cmn 装满(肠子、肺)\end{définition}
\begin{exemple}\jya tɯ-pu nɯ-pjɤt\cmn 把肠子装满\end{exemple}
\begin{exemple}\jya paʁ ɯ-rtshɤz nɯ-pjɤt\cmn 把猪肺装满\end{exemple}\begin{sous-entrée}
\vedette{\hypertarget{}{\papi{ rɤpjɤt}}}\markboth{rɤpjɤt}{}\classe{vi}
\paradigme{\textit{dir :} \jya nɯ-}
\begin{définition}\ 
\begin{déclaration}\grammar{apass}\end{déclaration}\end{définition}
\begin{définition}\fra bourrer\end{définition}
\begin{définition}\cmn 装满(肠子、肺)\end{définition}
\end{sous-entrée}\end{entrée}

\begin{entrée}
\vedette{\hypertarget{Ⓔpjɤʑŋgur}{\papi{ pjɤʑŋgur}}}\markboth{pjɤʑŋgur}{}
\classe{n}
\begin{définition}\fra saucisson\end{définition}
\begin{définition}\cmn 猪肉香肠\end{définition}
\begin{exemple}\jya pjɤʑŋgur ɲɯ́-wɣ-rku\cmn 把肉泥灌入香肠\end{exemple}
\begin{relation-sémantique}\confer{
\hyperlink{Ⓔpjɤt}{\textit{ \papi{pjɤt}}}
}\end{relation-sémantique}\end{entrée}

\begin{entrée}
\vedette{\hypertarget{Ⓔpjɤʑrɤz}{\papi{ pjɤʑrɤz}}}\markboth{pjɤʑrɤz}{}
\classe{n}
\begin{définition}\fra méthode de tissage\end{définition}
\begin{définition}\cmn 织布的方法,四根线交错着,纹路的方向变来变去\end{définition}\end{entrée}

\begin{entrée}
\vedette{\hypertarget{Ⓔpjɯrɯ}{\papi{ pjɯrɯ}}}\markboth{pjɯrɯ}{}
\classe{n}
\begin{définition}\fra corail\end{définition}
\begin{définition}\cmn 珊瑚
\begin{déclaration} \étymologie{\papi{bʲu.ru}}\end{déclaration}\end{définition}\end{entrée}

\begin{entrée}
\vedette{\hypertarget{Ⓔpuɟy}{\papi{ puɟy}}}\markboth{puɟy}{}\classe{n}
\begin{définition}\fra instrument pour bourrer les saucisses et les boudins\end{définition}
\begin{définition}\cmn 装血肠的工具\end{définition}\end{entrée}

\begin{entrée}
\vedette{\hypertarget{Ⓔplaŋplaŋ}{\papi{ plaŋplaŋ}}}\markboth{plaŋplaŋ}{}\classe{idph.2}
\begin{définition}\fra étendu et lisse\end{définition}
\begin{définition}\cmn 形容又光滑又宽的样子\end{définition}
\begin{exemple}\jya tɤjpɣom plaŋplaŋ kɯ-pa ʑo nɯ-atɯɣ-ndʑi\cmn 他们俩遇到一块又光滑又宽的冰片\end{exemple}\end{entrée}

\begin{entrée}
\vedette{\hypertarget{Ⓔplaʁplaʁ}{\papi{ plaʁplaʁ}}}\markboth{plaʁplaʁ}{}
\classe{idph.2}\acception{1}
\begin{définition}\fra doux au toucher\end{définition}
\begin{définition}\cmn 形容摸起来很光滑的样子\end{définition}\acception{2}
\begin{définition}\fra tout blanc\end{définition}
\begin{définition}\cmn 形容一片纯白的样子\end{définition}
\begin{exemple}\jya tɤjpa plaʁplaʁ ʑo ko-sɯwɣrum\cmn 雪把大地染成了白茫茫的一片\end{exemple}
\begin{exemple}\jya ɕɤfɕo ndɤre, thɯ-nɯrmɤmbe-nɯ, koŋla plaʁplaʁ ʑo thɯ-pa-nɯ\cmn 这几天它们脱了毛,变得光溜溜的\end{exemple}\begin{sous-entrée}
\vedette{\hypertarget{}{\papi{ ɣɤplaʁplaʁ}}}\markboth{ɣɤplaʁplaʁ}{}\classe{vi}
\begin{exemple}\jya qapri ɣɯ ɯ-mdʑu ɲɯ-ɣɤplaʁplaʁ ʑo\cmn 蛇的舌头一伸一缩\end{exemple}
\end{sous-entrée}\begin{sous-entrée}
\vedette{\hypertarget{}{\papi{ plaʁnɤplaʁ}}}\markboth{plaʁnɤplaʁ}{}\classe{idph.3}
\begin{exemple}\jya qapri kɯ ɯ-mdʑu plaʁnɤplaʁ ʑo ɲɯ-ɤsɯ-stu\cmn 蛇(慢慢地)把舌头一伸一缩的\end{exemple}
\end{sous-entrée}\begin{sous-entrée}
\vedette{\hypertarget{}{\papi{ sɤplaʁplaʁ}}}\markboth{sɤplaʁplaʁ}{}\classe{vt}
\begin{exemple}\jya qapri kɯ ɯ-mdʑɯ ɲɯ-sɤplaʁplaʁ\cmn 蛇(快速地)把舌头一伸一缩\end{exemple}
\end{sous-entrée}\end{entrée}

\begin{entrée}
\vedette{\hypertarget{Ⓔploʁploʁ}{\papi{ ploʁploʁ}}}\markboth{ploʁploʁ}{}\classe{idph.2}
\begin{définition}\fra en boule\end{définition}
\begin{définition}\cmn 形容圆形的\end{définition}
\begin{relation-sémantique}\confer{
 \papi{χploχploʁ}
}\end{relation-sémantique}
\begin{relation-sémantique}\confer{
\hyperlink{Ⓔɕploʁɕploʁ}{\textit{ \papi{ɕploʁɕploʁ}}}
}\end{relation-sémantique}\begin{sous-entrée}
\vedette{\hypertarget{}{\papi{ ɣɤploʁploʁ}}}\markboth{ɣɤploʁploʁ}{}\classe{vi}
\begin{définition}\fra bouillir à gros bouillon\end{définition}
\begin{définition}\cmn 滚烫,冒出水泡\end{définition}
\begin{exemple}\jya tɯ-ci ɲɯ-ɤla ɲɯ-ɣɤploʁploʁ ʑo\cmn 水沸腾了,冒出水泡\end{exemple}
\end{sous-entrée}\end{entrée}

\begin{entrée}
\vedette{\hypertarget{Ⓔplɯt}{\papi{ plɯt}}}\markboth{plɯt}{}
\classe{vt}
\paradigme{\textit{dir :} \jya nɯ-}
\paradigme{\textit{dir :} \jya thɯ-}\acception{1}
\begin{définition}\fra détruire\end{définition}
\begin{définition}\cmn 灭亡\end{définition}
\begin{exemple}\jya jima ɯ-rɣi nɯ-plɯt-i\cmn 我们把玉米的种子用得一点都不剩了\end{exemple}\acception{2}
\begin{définition}\fra anéantir la descendance\end{définition}
\begin{définition}\cmn 断根\end{définition}
\begin{relation-sémantique}\confer{
\hyperlink{Ⓔmblɯt}{\textit{ \papi{mblɯt}}}
}\end{relation-sémantique}\end{entrée}

\begin{entrée}
\vedette{\hypertarget{Ⓔpopo}{\papi{ popo}}}\markboth{popo}{}\classe{n}
\begin{définition}\fra récipient en terre\end{définition}
\begin{définition}\cmn 沙锅\end{définition}
\begin{exemple}\jya popo nɯ tɤ-rcoʁ kɯ tɤ-kɤ-sɯ-βzu ŋu, ɯ-mŋu artɯm, ɯ-xtu kɯnɤ artɯm, ɯ-xtu cho ɯ-mŋu ɯ-pɤrthɤβ ɯ-mke ci tu, ɯ-jɯ kɯ-rɟɯ-rɟum tu, tɤ-mthɯm sɤ-sqa pe, tɤ-ala pɯ-tsu tɕe ʑaʑa mɤ-fɕu tɕe tɤ-mthɯm pɯ́-wɣ-sqa tɕe wuma ʑo ku-smi cha tɕe rgargɯn ra nɯ-kɤ-ndza wuma ʑo nɤtsa.\cmn 
\stylefv{popo}是用泥做成的,口和肚子都是圆形的,口和肚子之间有颈,把很粗,是煮肉的好器具。一旦煮开了,不容易凉,煮肉煮得很熟,很适合老年人吃。
\end{exemple}
\end{entrée}

\begin{entrée}
\vedette{\hypertarget{Ⓔporɤt}{\papi{ porɤt}}}\markboth{porɤt}{}
\classe{n}
\begin{définition}\fra petite araignée\end{définition}
\begin{définition}\cmn 小蜘蛛\end{définition}\end{entrée}

\begin{entrée}
\vedette{\hypertarget{Ⓔpoʁlɯ}{\papi{ poʁlɯ}}}\markboth{poʁlɯ}{}
\classe{n}
\begin{définition}\fra type d'avoine\end{définition}
\begin{définition}\cmn 莜麦\end{définition}\end{entrée}

\begin{entrée}
\vedette{\hypertarget{Ⓔpoʁlɯrmbjɤβ}{\papi{ poʁlɯrmbjɤβ}}}\markboth{poʁlɯrmbjɤβ}{}\classe{n}
\begin{définition}\fra avoine en bottes\end{définition}
\begin{définition}\cmn 捆成一把的莜麦杆\end{définition}
\end{entrée}

\begin{entrée}
\vedette{\hypertarget{Ⓔposti}{\papi{ posti}}}\markboth{posti}{}\classe{n}
\begin{définition}\fra planche en bois recouvrant le mur dans la cuisine\end{définition}
\begin{définition}\cmn 【墙裙】贴在墙壁上的木板,大概只有一米高\end{définition}\end{entrée}

\begin{entrée}
\vedette{\hypertarget{Ⓔpoxco}{\papi{ poxco}}}\markboth{poxco}{}
\classe{n}
\begin{définition}\fra récipient en cuivre\end{définition}
\begin{définition}\cmn 铜罐子,口小腹大,有斜着的倒水用的嘴口,没有盖子\end{définition}
\begin{exemple}\jya poxco nɯ kontsi nɯ ɯ-ŋgɯz kɯ-wxti tsa ci ŋu, tɯ-ci sqamnɯz tɯ-rpa jamar tɕhɯt, nɯ sɤz kɯ-xtɕi tɕi tu, poxco ɯ-mtɕhi ɯ-phaʁ ntsi ri ɲɯ-ru ŋu, ɯ-xtu kɯ-wxtɯ-wxti ŋu, ɯ-spa nɯ zaŋ ŋu, ɯ-jɯ wuma ʑo ngɯt. kɯɕɯŋgɯ tɕe tʂha ɯ-sɤ-ta pjɤ-ŋu, tɯrme kɯ-ɤro pjɤ-rkɯn. tɯrme kɯ-tshu nɯ-xtu kɯ-wxti nɯ ra poxco tu-sɤrmi-nɯ ŋgrɤl.\cmn 
\stylefv{poxco}是罐子里面比较大的一种,可以容下十二斤水,比较小的也有。罐子嘴是斜的,有很大的肚子,是用红铜铸成的,把很结实。过去,是用来熬茶的罐子,拥有这种罐子的人不多。肚子很大,比较胖的人有时候说他们是\stylefv{poxco}。
\end{exemple}\end{entrée}

\begin{entrée}
\vedette{\hypertarget{Ⓔpru}{\papi{ pru}}}\markboth{pru}{}\classe{n}
\begin{définition}\ 
\begin{déclaration}\grammar{n.lieu}\end{déclaration}\end{définition}
\begin{définition}\fra un nom de hameau\end{définition}
\begin{définition}\cmn 房名\end{définition}\end{entrée}

\begin{entrée}
\vedette{\hypertarget{Ⓔpraʁ}{\papi{ praʁ}}}\markboth{praʁ}{}
\classe{n}
\begin{définition}\fra falaise\end{définition}
\begin{définition}\cmn 崖山
\begin{déclaration} \étymologie{\papi{brag}}\end{déclaration}\end{définition}\end{entrée}

\begin{entrée}
\vedette{\hypertarget{Ⓔpraʁɕku}{\papi{ praʁɕku}}}\markboth{praʁɕku}{}\classe{n}
\begin{définition}\fra oignon\end{définition}
\begin{définition}\cmn 葱\end{définition}
\begin{exemple}\jya praʁɕku nɯ tɯ-ji ɯ-ŋgɯ lu-kɤ-nɯ-ji ci ŋu, ɯ-jwaʁ nɯ kɤ-ɕpɯ-ɕpa kɯ-tɕɤr tɕe kɯ-rɲɟi tsa ŋu, ɯ-ru me, ɯ-tho tu, kɤ-ndza mɯm. ɯ-mdoʁ ldʑaŋnaʁ ŋu.\cmn 
\stylefv{praʁɕku}是自己种在地里的(农作物),叶子又扁又窄又长,没有茎,有花梗,好吃。颜色是深绿色。
\end{exemple}
\end{entrée}

\begin{entrée}
\vedette{\hypertarget{Ⓔpraʁkɤsi}{\papi{ praʁkɤsi}}}\markboth{praʁkɤsi}{}
\classe{n}
\begin{définition}\fra une espèce de chêne\end{définition}
\begin{définition}\cmn 槲栎的一种\end{définition}\end{entrée}

\begin{entrée}
\vedette{\hypertarget{Ⓔpraʁkhaŋ}{\papi{ praʁkhaŋ}}}\markboth{praʁkhaŋ}{}\classe{n}
\begin{définition}\fra grotte\end{définition}
\begin{définition}\cmn 山洞\end{définition}
\end{entrée}

\begin{entrée}
\vedette{\hypertarget{Ⓔpraʁɬɤrɲaŋ}{\papi{ praʁɬɤrɲaŋ}}}\markboth{praʁɬɤrɲaŋ}{}\classe{n}
\begin{définition}\fra falaise\end{définition}
\begin{définition}\cmn 悬崖
\begin{déclaration} \étymologie{\papi{brag lha.rɲiŋ}}\end{déclaration}\end{définition}
\end{entrée}

\begin{entrée}
\vedette{\hypertarget{Ⓔpraʁɲɤl}{\papi{ praʁɲɤl}}}\markboth{praʁɲɤl}{}\classe{n}
\begin{définition}\fra grotte\end{définition}
\begin{définition}\cmn 岩洞\end{définition}\end{entrée}

\begin{entrée}
\vedette{\hypertarget{Ⓔpraʁsrɯm}{\papi{ praʁsrɯm}}}\markboth{praʁsrɯm}{} (\variante{praʁʂɯm}) \classe{n}
\begin{définition}\fra un mammifère\end{définition}
\begin{définition}\cmn 哺乳动物的一种
\begin{déclaration} \étymologie{\papi{brag.sram}}\end{déclaration}\end{définition}
\begin{exemple}\jya praʁsrɯm nɯ praʁ ɯ-ŋgɯ ku-kɯ-rɤʑi tɕe ɯ-rme wuma ɲɯ-mpɕɤr tɕe spoŋsrɤm cho tɕhɯɕrɤm nɯra kɯ-naχtɕɯɣ ɲɯ-ŋu-nɯ ɯʑo praʁ ɯ-ŋgɯ ku-rɤʑi ɲɯ-ŋu tɕe núndʐa praʁsrɯm ɲɯ-rmi\cmn 
\stylefv{praʁsrɯm}是生活在岩石里的一种动物,毛长得很美,同水獭和\stylefv{spoŋsrɤm} 一样,因为生活在岩石里所以叫 \stylefv{praʁsrɯm}
\end{exemple}\end{entrée}

\begin{entrée}
\vedette{\hypertarget{Ⓔpraʁsrɯn}{\papi{ praʁsrɯn}}}\markboth{praʁsrɯn}{}
\classe{n}
\begin{définition}\fra nain, une sorte de démon\end{définition}
\begin{définition}\cmn 鬼的一种(矮人,投小石头)
\begin{déclaration} \étymologie{\papi{brag.srin}}\end{déclaration}\end{définition}\end{entrée}

\begin{entrée}
\vedette{\hypertarget{Ⓔpraʁʑɯn}{\papi{ praʁʑɯn}}}\markboth{praʁʑɯn}{}
\classe{n}
\begin{définition}\fra grotte\end{définition}
\begin{définition}\cmn 山洞\end{définition}\end{entrée}

\begin{entrée}
\vedette{\hypertarget{Ⓔpraχpa}{\papi{ praχpa}}}\markboth{praχpa}{}\classe{n}
\begin{définition}\fra caverne sous la falaise\end{définition}
\begin{définition}\cmn 岩洞\end{définition}
\end{entrée}

\begin{entrée}
\vedette{\hypertarget{Ⓔprɤdɤja,ta}{\papi{ prɤdɤja,ta}}}\markboth{prɤdɤja,ta}{}\acception{1}
\begin{définition}\fra griffer partout\end{définition}
\begin{définition}\cmn 到处乱抓\end{définition}\acception{2}
\begin{définition}\fra mettre en désordre (oiseau)\end{définition}
\begin{définition}\cmn 乱撒东西(鸟)\end{définition}
\begin{exemple}\jya kumpɣa kɯ ɯ-thoʁ ra chɯ-rɤβraʁ tɕe, prɤdɤja ʑo pjɯ-te ŋu\cmn 鸡把地面抓得到处都是抓过的痕迹\end{exemple}
\begin{relation-sémantique}\ComponentA{\classe{n}
 \papi{prɤdɤja}
}\end{relation-sémantique}
\begin{relation-sémantique}\ComponentB{\classe{vt}
\hyperlink{Ⓔta}{\textit{ \papi{ta}}}
}\end{relation-sémantique}\end{entrée}

\begin{entrée}
\vedette{\hypertarget{Ⓔprɤftsa}{\papi{ prɤftsa}}}\markboth{prɤftsa}{}
\classe{n}
\begin{définition}\fra ours noir\end{définition}
\begin{définition}\cmn 狗熊\end{définition}
\begin{relation-sémantique}\confer{
\hyperlink{ⒺpriⒽ2}{\textit{ \papi{pri2}}}
}\end{relation-sémantique}
\begin{relation-sémantique}\confer{
\hyperlink{Ⓔtɤ-ftsa}{\textit{ \papi{tɤ-ftsa}}}
}\end{relation-sémantique}\end{entrée}

\begin{entrée}
\vedette{\hypertarget{Ⓔprɤku}{\papi{ prɤku}}}\markboth{prɤku}{}\classe{n}
\begin{définition}\ 
\begin{déclaration}\grammar{n.lieu}\end{déclaration}\end{définition}
\begin{définition}\fra l'un des hameaux de Kamnyu\end{définition}
\begin{définition}\cmn 干木鸟的大队之一\end{définition}
\end{entrée}

\begin{entrée}
\vedette{\hypertarget{Ⓔprɤm}{\papi{ prɤm}}}\markboth{prɤm}{}\classe{vt}
\paradigme{\textit{dir :} \jya pɯ-}
\begin{définition}\fra ajouter de la farine\end{définition}
\begin{définition}\cmn 加面粉(水、汤里)\end{définition}
\begin{exemple}\jya paʁtshi pɯ-prɤm\cmn 在猪食里加一点面粉吧\end{exemple}
\begin{exemple}\jya a-tʂha ci pɯ-prɤm\cmn 给我的茶加点面粉吧\end{exemple}
\begin{relation-sémantique}\confer{
\hyperlink{Ⓔtɤ-prɤm}{\textit{ \papi{tɤ-prɤm}}}
}\end{relation-sémantique}\end{entrée}

\begin{entrée}
\vedette{\hypertarget{Ⓔprɤmɤl}{\papi{ prɤmɤl}}}\markboth{prɤmɤl}{} (\variante{n-prAmAl}) \classe{n}
\begin{définition}\fra partie couverte sous les sapins où l'on peut se protéger de la pluie\end{définition}
\begin{définition}\cmn 高大的杉树能遮雨的地方\end{définition}\end{entrée}

\begin{entrée}
\vedette{\hypertarget{Ⓔprɤmchi}{\papi{ prɤmchi}}}\markboth{prɤmchi}{}
\classe{n}
\begin{définition}\fra bile d'ours\end{définition}
\begin{définition}\cmn 熊胆\end{définition}
\begin{relation-sémantique}\confer{
\hyperlink{ⒺpriⒽ2}{\textit{ \papi{pri2}}}
}\end{relation-sémantique}
\begin{relation-sémantique}\confer{
\hyperlink{Ⓔtɯ-mchi}{\textit{ \papi{tɯ-mchi}}}
}\end{relation-sémantique}\end{entrée}

\begin{entrée}
\vedette{\hypertarget{Ⓔprɤndʐi}{\papi{ prɤndʐi}}}\markboth{prɤndʐi}{}\classe{n}
\begin{définition}\fra peau d'ours\end{définition}
\begin{définition}\cmn 熊皮子\end{définition}
\begin{relation-sémantique}\confer{
\hyperlink{ⒺpriⒽ2}{\textit{ \papi{pri2}}}
}\end{relation-sémantique}
\begin{relation-sémantique}\confer{
\hyperlink{Ⓔtɯ-ndʐi}{\textit{ \papi{tɯ-ndʐi}}}
}\end{relation-sémantique}\end{entrée}

\begin{entrée}
\vedette{\hypertarget{Ⓔprɤɲaʁ}{\papi{ prɤɲaʁ}}}\markboth{prɤɲaʁ}{}
\classe{n}
\begin{définition}\fra ours noir\end{définition}
\begin{définition}\cmn 狗熊\end{définition}
\begin{relation-sémantique}\confer{
\hyperlink{ⒺpriⒽ2}{\textit{ \papi{pri2}}}
}\end{relation-sémantique}\end{entrée}

\begin{entrée}
\vedette{\hypertarget{Ⓔprɤɲi}{\papi{ prɤɲi}}}\markboth{prɤɲi}{}\classe{n}
\begin{définition}\fra reflets empourprés (au crépuscule ou à l'aube)\end{définition}
\begin{définition}\cmn 早霞;晚霞\end{définition}\end{entrée}

\begin{entrée}
\vedette{\hypertarget{Ⓔprɤschɯ}{\papi{ prɤschɯ}}}\markboth{prɤschɯ}{}\classe{n}
\begin{définition}\ 
\begin{déclaration}\grammar{n.lieu}\end{déclaration}\end{définition}
\begin{définition}\fra l'un des hameaux de Kamnyu\end{définition}
\begin{définition}\cmn 干木鸟的大队之一\end{définition}
\end{entrée}

\begin{entrée}
\vedette{\hypertarget{Ⓔprɤt}{\papi{ prɤt}}}\markboth{prɤt}{}
\classe{vt}
\paradigme{\textit{dir :} \jya nɯ-}
\paradigme{\textit{dir :} \jya pɯ-}
\begin{définition}\fra casser (fil, corde), couper\end{définition}
\begin{définition}\cmn 弄断(线);迸断
\begin{déclaration}\use{用刀弄断不能说\stylefv{prɤt},只能说\stylefv{ʁndzɤr}}\end{déclaration}\end{définition}
\begin{exemple}\jya tɯmbri pɯ-prɤt\cmn 你把绳子弄断吧\end{exemple}
\begin{exemple}\jya ɕomskrɯt nɯ-prɤt\cmn 你把铁丝弄断吧\end{exemple}
\begin{exemple}\jya tɤ-ri nɯ-prɤt\cmn 你把线弄断吧\end{exemple}
\begin{relation-sémantique}\confer{
\hyperlink{Ⓔmbrɤt}{\textit{ \papi{mbrɤt}}}
}\end{relation-sémantique}
\begin{relation-sémantique}\confer{
\hyperlink{Ⓔrɤmprɤt}{\textit{ \papi{rɤmprɤt}}}
}\end{relation-sémantique}\begin{sous-entrée}
\vedette{\hypertarget{}{\papi{ nɤpɯprɤt}}}\markboth{nɤpɯprɤt}{}\classe{vt}
\begin{définition}\fra couper dans tous les sens\end{définition}
\begin{définition}\cmn 断来断去\end{définition}
\end{sous-entrée}\begin{sous-entrée}
\vedette{\hypertarget{}{\papi{ sɯprɤt}}}\markboth{sɯprɤt}{}\classe{vt}
\paradigme{\textit{dir :} \jya pɯ-}
\begin{définition}\fra faire abandonner (une mauvaise habitude) à qqn\end{définition}
\begin{définition}\cmn 让……断绝(坏习惯)\end{définition}
\begin{exemple}\jya a-wa ɯ-thamakha-sko kɤ-sɯprɤt mɯ-pɯ-cha-a\cmn 我没能控制住我父亲抽烟的习惯\end{exemple}
\end{sous-entrée}\end{entrée}

\begin{entrée}
\vedette{\hypertarget{ⒺpriⒽ2}{\papi{ pri}}}\markboth{pri}{}\homonyme{2}\classe{n}
\begin{définition}\fra ours\end{définition}
\begin{définition}\cmn 熊\end{définition}
\begin{exemple}\jya pri nɯ χsɯ-tɯphu tu, ndzɤpri kɯ-rmi ci tu, prɤɲaʁ kɯ-rmi ci tu, prɤftsa kɯ-rmi ci tu, ndzɤpri nɯ pɤjmu ruŋgɯ nɯ tɕu tu ɲɯ-ngrɤl, sɤ-ndza ɲɯ-ŋgrɤl, ɯ-mdoʁ nɯ kɯ-pɣi kɯ-ɤɣɯrnɯɕɯr ɲɯ-ŋu, ɯ-mɤlɤjaʁ nɯ prɤɲaʁ cho kɯ-naχtɕɯɣ ɕti, qrorni cho sɯmat tu-ndze ɲɯ-ŋgrɤl. prɤɲaʁ nɯ kɯ-wxti ŋu, tɯ-ji ɯ-ŋgɯ kɯ-nɤru ju-ɣi ŋgrɤl, jima stoʁ ndze ŋgrɤl, tú-wɣ-nɤrʁaʁ tɕe tɯ-phe jɤ-armbat tɕe χsɯ-mɢla jamar kɯ-tu tɕe tu-ndzur ɲɯ-ŋgrɤl, tɕe tu-kɯ-mtsɯɣ tɤ-ɣɯɣu ŋgrɤl. kɤ-nɯsɯku wuma rkaŋ, ɕkrɤz ɯ-mat tu-ndze tɕe pjɯ-χtsɤβ ŋgrɤl ɕti. ɯʑo nɯ kɯ-ɲaʁ ŋu, ɯ-ro χcɤl zɯ kɯ-wɣrum tɯ-snaʁ tu, ɯ-mtɕhi amtɕoʁ tsa. ɯ-mɤlɤjaʁ tɯrme ɣɯ tsa fse. prɤftsa kɯ-xtɕi tsa ŋu, prɤɲaʁ cho naχtɕɯɣ tsa, sɤ-ndza tu-kɯ-ti ɲɯ-ŋu.\cmn 
熊有三种,一种叫马熊、一种叫狗熊、另一种叫\stylefv{prɤftsa}。马熊一般生活在贝母山上,会吃人,颜色是灰里带红的,四肢和狗熊的一样,吃红蚂蚁和树果。狗熊很大,经常到田地里来偷吃粮食,吃玉米和胡豆。捕猎它的时候,相隔距离只有三步的时候它就会站立起来,准备咬人。善于爬树,吃青冈树的果实的时候还要把(树的枝桠)搓揉起来。身子是黑色的,在胸部中间有白毛,嘴巴有点尖。四只脚和人的一样。\stylefv{prɤftsa}长得比较小,有点像狗熊,据说吃人。
\end{exemple}\end{entrée}

\begin{entrée}
\vedette{\hypertarget{ⒺpriⒽ1}{\papi{ pri}}}\markboth{pri}{}\homonyme{1}\classe{vt}
\paradigme{\textit{dir :} \jya thɯ-}
\begin{définition}\fra déchirer\end{définition}
\begin{définition}\cmn 撕
\begin{déclaration}\use{\stylefv{pri}在用法上和\stylefv{qraʁ}“撕烂”不完全一样,前者用于容易撕的物体(如布、纸),而后者用于难撕的物体(如皮子)}\end{déclaration}\end{définition}
\begin{exemple}\jya tɯ-ŋga thɯ-pri-t-a\cmn 我撕了衣服\end{exemple}
\begin{exemple}\jya jɯɣi thɯ-pri-t-a\cmn 我撕了书\end{exemple}
\begin{relation-sémantique}\confer{
\hyperlink{ⒺmbriⒽ2}{\textit{ \papi{mbri2}}}
}\end{relation-sémantique}\end{entrée}

\begin{entrée}
\vedette{\hypertarget{Ⓔprɯ}{\papi{ prɯ}}}\markboth{prɯ}{}\classe{vs}
\begin{définition}\fra imperméable\end{définition}
\begin{définition}\cmn 不进水\end{définition}
\begin{exemple}\jya tɯwɯr nɯ tú-wɣ-nɤqhɤŋga jɤɣ ma wuma ʑo prɯ\cmn 雨衣可以披在肩上,不容易进水\end{exemple}\end{entrée}

\begin{entrée}
\vedette{\hypertarget{Ⓔprɯɣprɯɣ}{\papi{ prɯɣprɯɣ}}}\markboth{prɯɣprɯɣ}{}
\classe{idph.2}\acception{1}
\begin{définition}\fra très serré\end{définition}
\begin{définition}\cmn 形容很紧,不容易解开\end{définition}
\begin{exemple}\jya rgɤm ɯ-ŋgɯ prɯɣprɯɣ ʑo pjɯ-ɤrku ɕti\cmn 箱子里装得很满\end{exemple}
\begin{exemple}\jya tɤ-fkɯm ɯ-ŋgɯ prɯɣprɯɣ ʑo chɯ-ɤrku\cmn 口袋里装得很满\end{exemple}
\begin{exemple}\jya tɤ-mtɯ prɯɣprɯɣ chɤ-lɤt\cmn 把扣子扣得很紧\end{exemple}\acception{2}
\begin{définition}\fra bien rassasié\end{définition}
\begin{définition}\cmn 形容吃饱的样子\end{définition}
\begin{exemple}\jya a-tɯ-fka kɯ prɯɣprɯɣ ʑo ɲɯ-pa\cmn 我吃得很饱\end{exemple}
\begin{exemple}\jya tɤ-fka prɯɣprɯɣ ʑo\cmn 他很饱\end{exemple}\end{entrée}

\begin{entrée}
\vedette{\hypertarget{Ⓔprɯŋprɯŋ}{\papi{ prɯŋprɯŋ}}}\markboth{prɯŋprɯŋ}{}
\classe{idph.2}
\begin{définition}\fra solide\end{définition}
\begin{définition}\cmn 形容(绑得)很紧、很扎实\end{définition}
\begin{exemple}\jya prɯŋprɯŋ ʑo ko-tʂɯβ\cmn 他缝得很扎实\end{exemple}\end{entrée}

\begin{entrée}
\vedette{\hypertarget{Ⓔpɯɕɯɣ}{\papi{ pɯɕɯɣ}}}\markboth{pɯɕɯɣ}{}
\classe{n}
\begin{définition}\fra sac à poudre\end{définition}
\begin{définition}\cmn 装火药的皮袋\end{définition}
\begin{exemple}\jya pɯɕɯɣ nɯ ɕɤmɯɣdɯ ɣɯ ɯ-fkɯm mɯ-ɲɯ-kɯ-sɤci ɯ-spa ŋu\cmn 枪袋是用来防止枪杆受潮的。\end{exemple}\end{entrée}

\begin{entrée}
\vedette{\hypertarget{Ⓔpɯɣ}{\papi{ pɯɣ}}}\markboth{pɯɣ}{}\classe{vs}
\paradigme{\textit{dir :} \jya tɤ-}
\begin{définition}\fra se gonfler\end{définition}
\begin{définition}\cmn 膨胀\end{définition}
\begin{exemple}\jya tɤɕi kɤ́-wɣ-sqa tɕe to-pɯɣ tɕe tɯthɯ ɯ-ŋgɯ mɯ-ɲɤ-xtɕhɯt\cmn 煮青稞的时候,煮熟了就发胀,锅里就装不下了\end{exemple}\end{entrée}

\begin{entrée}
\vedette{\hypertarget{Ⓔpɯlthi}{\papi{ pɯlthi}}}\markboth{pɯlthi}{}
\classe{n}
\begin{définition}\fra mèche\end{définition}
\begin{définition}\cmn 火绳\end{définition}
\begin{exemple}\jya pɯlthi nɯ smi sɤ-zwɤr ŋu\cmn 火绳是用来点火的部件。\end{exemple}\end{entrée}

\begin{entrée}
\vedette{\hypertarget{Ⓔpɯnbu}{\papi{ pɯnbu}}}\markboth{pɯnbu}{}\classe{n}
\begin{définition}\fra bonpo\end{définition}
\begin{définition}\cmn 黑教
\begin{déclaration} \étymologie{\papi{bon.po}}\end{déclaration}\end{définition}
\end{entrée}

\begin{entrée}
\vedette{\hypertarget{Ⓔpɯthaŋ}{\papi{ pɯthaŋ}}}\markboth{pɯthaŋ}{}\classe{n}
\begin{définition}\fra type de métal\end{définition}
\begin{définition}\cmn 白铜\end{définition}\end{entrée}

\begin{entrée}
\vedette{\hypertarget{Ⓔpɯtshɯŋ}{\papi{ pɯtshɯŋ}}}\markboth{pɯtshɯŋ}{}
\classe{n}
\begin{définition}\fra petit récipient en cuivre\end{définition}
\begin{définition}\cmn 红铜小罐子\end{définition}
\begin{exemple}\jya pɯtshɯŋ nɯ kontsi kɯ-xtɕɯ-xtɕi ci ŋu, tɯ-ci rɟɤpɕɤt jamar ma mɤ-tɕhɯt, ɯ-spa nɯ li tu-kɯ-ti ɲɯ-ŋu, ɯ-mdoʁ aɣrɤɣrum tsa, ɯ-mŋu zɯ ɯ-mtɕhi kɯ-zri tsa ʑo tu, ɯ-jɯ wuma ʑo ngɯt, ɯ-xtu mɤ-wxti, ɯ-qa nɯ kɯ-ntɯ-ntaβ ŋu. kɯɕɯŋgɯ tɕe, tɯrme kɯ-ŋgro ra ɣɯ nɯ-cha ɯ-z-nɤrkɯrku pjɤ-ŋu.\cmn 
\stylefv{pɯtshɯŋ}是比较小的罐子,只容得下半斤水,据说是用铜做成,颜色有点白,有比较长的嘴,把很结实,肚子不大,底部放起来很稳。过去,是用来给贵宾倒酒的罐子。
\end{exemple}\end{entrée}

\begin{entrée}
\vedette{\hypertarget{Ⓔpɯwɯ}{\papi{ pɯwɯ}}}\markboth{pɯwɯ}{}
\classe{n}
\begin{définition}\fra âne\end{définition}
\begin{définition}\cmn 驴子\end{définition}\end{entrée}

\begin{entrée}
\vedette{\hypertarget{Ⓔpɯz}{\papi{ pɯz}}}\markboth{pɯz}{}\classe{vs}
\paradigme{\textit{dir :} \jya tɤ-}
\begin{définition}\fra pourri (bois), usé jusqu'à la corde (habits)\end{définition}
\begin{définition}\cmn 朽烂(木头),破烂(衣服)\end{définition}
\end{entrée}

\newpage\caractère{q}

\begin{entrée}
\vedette{\hypertarget{Ⓔqachɣa}{\papi{ qachɣa}}}\markboth{qachɣa}{}
\classe{n}
\begin{définition}\fra renard\end{définition}
\begin{définition}\cmn 狐狸\end{définition}\end{entrée}

\begin{entrée}
\vedette{\hypertarget{Ⓔqachɣamɯntoʁ}{\papi{ qachɣamɯntoʁ}}}\markboth{qachɣamɯntoʁ}{}\classe{n}
\begin{définition}\fra Eugeron breviscapus\end{définition}
\begin{définition}\cmn 短葶飞蓬\end{définition}
\begin{relation-sémantique}\antonyme{
\hyperlink{Ⓔqachɣarte}{\textit{ \papi{qachɣarte}}}
}\end{relation-sémantique}\end{entrée}

\begin{entrée}
\vedette{\hypertarget{Ⓔqachɣarte}{\papi{ qachɣarte}}}\markboth{qachɣarte}{}\classe{n}
\begin{définition}\fra Eugeron breviscapus\end{définition}
\begin{définition}\cmn 短葶飞蓬\end{définition}
\begin{relation-sémantique}\confer{
\hyperlink{Ⓔqachɣamɯntoʁ}{\textit{ \papi{qachɣamɯntoʁ}}}
}\end{relation-sémantique}\end{entrée}

\begin{entrée}
\vedette{\hypertarget{Ⓔqachɣɤndʐi}{\papi{ qachɣɤndʐi}}}\markboth{qachɣɤndʐi}{}
\classe{n}
\begin{définition}\fra peau de renard\end{définition}
\begin{définition}\cmn 狐狸皮子\end{définition}
\begin{exemple}\jya qachɣɤndʐi ɲɯ-fkra\cmn 狐狸皮子彩色斑斓\end{exemple}
\begin{relation-sémantique}\confer{
\hyperlink{Ⓔqachɣa}{\textit{ \papi{qachɣa}}}
}\end{relation-sémantique}\end{entrée}

\begin{entrée}
\vedette{\hypertarget{Ⓔqaɕɣi}{\papi{ qaɕɣi}}}\markboth{qaɕɣi}{}
\classe{n}
\begin{définition}\fra grosse mouche\end{définition}
\begin{définition}\cmn 大苍蝇\end{définition}
\begin{exemple}\jya qaɕɣi nɯ βɣɤza sɤznɤ wxti, βɣɤza kɯβde jamar tu, ɯ-tshɯɣa nɯ βɣɤza cho naχtɕɯɣ, ɯ-ku artɯm, ɯ-mɲaʁ kɯ-wxtɯ-wxti tu, ɯ-ʁrɯ tu, ɯ-mɤlɤjaʁ kɯtʂɤ-ldʑa tu, ɯ-ʁar mba ɯ-rɯmu tu, mɤ-sɤmtsɯɣ, tɕeri wuma ʑo ŋɤn ma ftɕar tɕe ɕɤkhe ɯ-taʁ ʑo ɯ-qe ku-lɤt ŋu, nɯ ɯ-qe nɯ laʁnɤ-rʑaʁ ʑo tɕe tɯ-mɯnmɯ tu-ʑe, tɕe chɯ-wxti tɕe, ɕa tu-ndze ɲɯ-ŋu, kɯ-mɯm ɯ-stu ʑo tu-ndze ɲɯ-ŋu, tɕe tu-ndze mɤ-kɯ-jɤɣ kɯ ɯ-di ɲɯ-ɕɯmnɤm tɕe, mɯ-ta-ndza nɯ ra kɯnɤ ɯ-di ɲɯ-ɕɯmnɤm tɕe kɤ-ndza mɯ-ɲɯ-ɣɤsne ŋu. thu-wxti tɕe, pjɯ-nɯɬoʁ tɕe, kɯm ɯ-qhu zndɤrchɤβ nɯ ra ju-nɯrtsɯ-nɯ tɕe ju-ormbɯrmbɯ-nɯ ŋu, ɯntɕe nɯ-rqhu ɲɯ-βze tɕe, qartsɯ nɯ tɕu ku-rɤʑi-nɯ, χɕitka tɕe, ɯ-rqhu ɯ-ŋgɯ ri qaɕɣi ɲɯ-βze tɕe, li ju-nɯɬoʁ-nɯ ŋu, tɕe tɯrme ra kɯ qaɕɣi pa-mto-nɯ tɕe pjɯ-sat-nɯ ɕti ma ɯ-kɯ-qha ʁɟa ŋu.\cmn 大苍蝇比小苍蝇大四倍,形状完全和小苍蝇一样,头部是圆形的,有一对大眼睛,有触角,有六只脚。翅膀很薄,有纹路。不蜇人,但是很坏。夏天在瘦肉上下卵子(屙屎),过一两就开始动,吃肉,专门吃最好吃的部位,不但吃,还使肉变臭,连没有吃到的地方都会发臭,使得肉不能再吃了。长成了以后就出来掉到地上,爬到门后面墙壁缝里聚集在一起,然后变成蛹,在那里过冬天。到了春天蛹就变成苍蝇,从里面飞出来。人们看到苍蝇时就会把它弄死,因为都很讨厌它。\end{exemple}\end{entrée}

\begin{entrée}
\vedette{\hypertarget{Ⓔqaɕpa}{\papi{ qaɕpa}}}\markboth{qaɕpa}{}
\classe{n}
\begin{définition}\fra grenouille\end{définition}
\begin{définition}\cmn 青蛙
\begin{déclaration} \étymologie{\papi{sbal}}\end{déclaration}\end{définition}\end{entrée}

\begin{entrée}
\vedette{\hypertarget{Ⓔqaɕparaz}{\papi{ qaɕparaz}}}\markboth{qaɕparaz}{}\classe{n}
\begin{définition}\fra espèce d'herbe\end{définition}
\begin{définition}\cmn 草的一种\end{définition}
\end{entrée}

\begin{entrée}
\vedette{\hypertarget{Ⓔqaɕpɤrnoʁ}{\papi{ qaɕpɤrnoʁ}}}\markboth{qaɕpɤrnoʁ}{}
\classe{n}
\begin{définition}\fra fraise sauvage\end{définition}
\begin{définition}\cmn 野草莓\end{définition}
\begin{relation-sémantique}\confer{
\hyperlink{Ⓔqaɕpa}{\textit{ \papi{qaɕpa}}}
}\end{relation-sémantique}
\begin{relation-sémantique}\confer{
\hyperlink{Ⓔtɯ-rnoʁ}{\textit{ \papi{tɯ-rnoʁ}}}
}\end{relation-sémantique}\end{entrée}

\begin{entrée}
\vedette{\hypertarget{Ⓔqaɕti}{\papi{ qaɕti}}}\markboth{qaɕti}{}
\classe{n}
\begin{définition}\fra pêche\end{définition}
\begin{définition}\cmn 桃子\end{définition}
\begin{exemple}\jya qaɕti nɯ si kɯ-wxti tsa ci ŋu, ɯ-ru nɯ kɯ-pɣi ŋu, ɯ-rtaʁ dɤn, wuma ʑo ɲɯ-jpum mɤ-cha, ɯ-jwaʁ nɯ kɯ-tɕɤr tɕe kɯ-ɤmtɕoʁ tsa ci ŋu. ɯ-mɯntoʁ nɯ ɯ-jwaʁ ɲɤ-lɤt ɕɯŋgɯ ɲɯ-lɤt ŋu, ɯ-mɯntoʁ kɯ-wɣrum tu, kɯ-ɣɯrni tu, ɯ-mat nɯ ɯ-rme kɯ-fse sɯβ-sɯβ tu, arŋi tsa thɯ-tɯt tɕe qarŋe. tú-wɣ-ndza tɕe kɯ-chi tu, kɯ-tɕur tu. ɯ-si nɯ ngɯt tɕe laʁdɯn ɯ-jɯ kɤ-nɯ-βzu sna.\cmn 桃树是比较高大的树种,树干是灰色的,枝桠多,都长得不是很粗,叶子细而尖,花开在叶子长出来之前,有的是白色的,有的是红色的。果实上有绒毛,先是绿色的,成熟后变黄。吃起来有的是甜的,有的是酸的。木质结实,可以用来作农具的把子。\end{exemple}
\begin{relation-sémantique}\confer{
\hyperlink{Ⓔnɯqaɕti}{\textit{ \papi{nɯqaɕti}}}
}\end{relation-sémantique}\end{entrée}

\begin{entrée}
\vedette{\hypertarget{Ⓔqafsa}{\papi{ qafsa}}}\markboth{qafsa}{}\classe{n}
\begin{définition}\fra type de plante rampante\end{définition}
\begin{définition}\cmn 藤树的一种\end{définition}\end{entrée}

\begin{entrée}
\vedette{\hypertarget{Ⓔqaj}{\papi{ qaj}}}\markboth{qaj}{}
\classe{n}
\begin{définition}\fra blé\end{définition}
\begin{définition}\cmn 麦\end{définition}\end{entrée}

\begin{entrée}
\vedette{\hypertarget{Ⓔqajdu}{\papi{ qajdu}}}\markboth{qajdu}{}
\classe{n}
\begin{définition}\fra une espèce d'arbre\end{définition}
\begin{définition}\cmn 乔木的一种\end{définition}
\begin{exemple}\jya qajdu nɯ si khro mɤ-kɯ-mbro tsa ci ŋu, zgoku aʁɤndɯndɤt tu-ɬoʁ cha, si ɯ-βri nɯ kɯ-ɲaʁ ʁɟa ʑo ŋu, ɯ-jwaʁ kɯnɤ ldʑaŋnaʁ ɕti, ɯ-mdzu wuma ʑo mtɕoʁ, tɯ-ɕa ɯ-taʁ kɤ-tsa tɕe tɤ-spɯ ʑo tu-sɤβze cha. ɯ-mat kɯnɤ thɯ-tɯt tɕe kɯ-ɲaʁ chɯ-βze ŋu\cmn 
\stylefv{qajdu}是长得不是高的树种,山上山下到处都可以生长,树皮(树身)全是黑色的,连树叶都是深绿色的。刺很锋利,刺到皮肉上会长脓包。果实成熟后也是黑色的
\end{exemple}
\begin{exemple}\jya nɤ-sŋi ɯ-tɯ-ɲaʁ kɯ qajdu ʑo ɲɯ-tɯ-fse\cmn 
你的心黑得像\stylefv{qajdu}一样(你真狠心)。
\end{exemple}\end{entrée}

\begin{entrée}
\vedette{\hypertarget{Ⓔqajdo}{\papi{ qajdo}}}\markboth{qajdo}{}
\classe{n}
\begin{définition}\fra corbeau (corvus corone)\end{définition}
\begin{définition}\cmn 小嘴乌鸦\end{définition}\end{entrée}

\begin{entrée}
\vedette{\hypertarget{Ⓔqajdoɕku}{\papi{ qajdoɕku}}}\markboth{qajdoɕku}{}
\classe{n}
\begin{définition}\fra poireau sauvage\end{définition}
\begin{définition}\cmn 高山上的野韭菜\end{définition}
\begin{exemple}\jya qajdoɕku nɯ aʁɤndɯndɤt sɯŋgɯ tu-ɬoʁ cha, ɯ-jwaʁ ɯ-qhuchu nɯ wɣrum, nɯ maʁ nɯ, ɕkɤpja tsa fse, ri ɯ-ru ɯ-taʁ kɯ-ɤrɤʑɯʑrɤz nɯ me. ɯ-mɯntoʁ nɤmkha mdoʁ ɲɯ-lɤt ŋu. tú-wɣ-ndza tɕe, ɯ-di mɤ-mɯm.\cmn 
\stylefv{qajdoɕku}能在森林里到处生长,叶子背面是白色的,其他部位跟 \stylefv{ɕkɤpja}一样,但茎上没有纹路,开天蓝色的花。吃起来不香。
\end{exemple}\end{entrée}

\begin{entrée}
\vedette{\hypertarget{Ⓔqajɣi}{\papi{ qajɣi}}}\markboth{qajɣi}{}
\classe{n}
\begin{définition}\fra pain\end{définition}
\begin{définition}\cmn 馍馍\end{définition}\end{entrée}

\begin{entrée}
\vedette{\hypertarget{Ⓔqajo}{\papi{ qajo}}}\markboth{qajo}{}
\classe{n}
\begin{définition}\fra récipient en terre\end{définition}
\begin{définition}\cmn 土罐子\end{définition}\end{entrée}

\begin{entrée}
\vedette{\hypertarget{Ⓔqajpɣom}{\papi{ qajpɣom}}}\markboth{qajpɣom}{}
\classe{vs}
\paradigme{\textit{dir :} \jya nɯ-}
\begin{définition}\fra geler\end{définition}
\begin{définition}\cmn 冻到\end{définition}
\begin{exemple}\jya @yangyu ɲɤ-qajpɣom\cmn 洋芋冻到了\end{exemple}
\begin{exemple}\jya lɤpɯɣ ɲɤ-qajpɣom\cmn 萝卜冻到了\end{exemple}
\begin{relation-sémantique}\confer{
\hyperlink{Ⓔjpɣom}{\textit{ \papi{jpɣom}}}
}\end{relation-sémantique}\end{entrée}

\begin{entrée}
\vedette{\hypertarget{Ⓔqajru}{\papi{ qajru}}}\markboth{qajru}{}
\classe{n}
\begin{définition}\fra tige de blé\end{définition}
\begin{définition}\cmn 麦秸\end{définition}\end{entrée}

\begin{entrée}
\vedette{\hypertarget{Ⓔqajrqhu}{\papi{ qajrqhu}}}\markboth{qajrqhu}{}
\classe{n}
\begin{définition}\fra son\end{définition}
\begin{définition}\cmn 麦麸\end{définition}\end{entrée}

\begin{entrée}
\vedette{\hypertarget{Ⓔqajrutʂu}{\papi{ qajrutʂu}}}\markboth{qajrutʂu}{}\classe{n}
\begin{définition}\fra torche\end{définition}
\begin{définition}\cmn 火把\end{définition}
\end{entrée}

\begin{entrée}
\vedette{\hypertarget{Ⓔqajtsrɯ}{\papi{ qajtsrɯ}}}\markboth{qajtsrɯ}{}
\classe{n}
\begin{définition}\fra pousses de blé\end{définition}
\begin{définition}\cmn 麦苗\end{définition}\end{entrée}

\begin{entrée}
\vedette{\hypertarget{Ⓔqajtʂha}{\papi{ qajtʂha}}}\markboth{qajtʂha}{}
\classe{n}
\begin{définition}\fra vautour (aegyptius monachus)\end{définition}
\begin{définition}\cmn 秃鹫\end{définition}
\begin{exemple}\jya qajtʂha nɯ kuwu cho naχtɕɯɣ ri sɤznɤ xtɕi tsa, ɯ-ku kɯ-ɲaʁ ŋu, ɯ-mtɕhi-rme tu tɕe nɯ kɯnɤ kɯ-ɲaʁ ŋu, ɯ-phoŋbu kɯ-pɣi kɯ-ɤɲaʁndzɯm tsa ŋu, tɕe praʁ ɯ-ŋgɯ ku-rɤʑi ŋu, kɯ-rgɤz ra kɯ qajtʂha tɤ-ŋke tɕe tɯ-mɯ khe tu-ti-nɯ ŋgrɤl. qajɯ kɯ-fse ra tu-ndze ma kɯmaʁ rɯdaʁ kɯ-fse ra mɤ-ndze, tɤ-rɤku ri mɤ-ndze.\cmn 秃鹫和胡兀鹫长得差不多,但小一点,头是黑色的,有胡须,也是黑色的,身子是灰里带黑的,栖息在岩洞里。老年人们说,当秃鹫出现时,天会变阴。它吃虫子,不吃其它动物,也不吃粮食。\end{exemple}\end{entrée}

\begin{entrée}
\vedette{\hypertarget{Ⓔqajɯ}{\papi{ qajɯ}}}\markboth{qajɯ}{}
\classe{n}
\begin{définition}\fra insecte, vers\end{définition}
\begin{définition}\cmn 虫\end{définition}
\begin{relation-sémantique}\confer{
\hyperlink{Ⓔrɯqajɯ}{\textit{ \papi{rɯqajɯ}}}
}\end{relation-sémantique}\end{entrée}

\begin{entrée}
\vedette{\hypertarget{Ⓔqajɯβlama}{\papi{ qajɯβlama}}}\markboth{qajɯβlama}{}\classe{n}
\begin{définition}\fra petite sangsue\end{définition}
\begin{définition}\cmn 小水蛭\end{définition}\end{entrée}

\begin{entrée}
\vedette{\hypertarget{Ⓔqajɯkɯrɤtɣa}{\papi{ qajɯkɯrɤtɣa}}}\markboth{qajɯkɯrɤtɣa}{}
\classe{n}
\begin{définition}\fra chenille arpenteuse\end{définition}
\begin{définition}\cmn 尺蠖\end{définition}
\begin{exemple}\jya qajɯ kɯ-rɤtɣa nɯ qajɯ kɯ-mpɯ-mpɯ kɯ-xtɕɯ-xtɕi ci ŋu, tɕe ɯ-zda ra kɯ-fse tu-nɯrtsɯ ɲɯ-maʁ, ju-rɤtɣe tɕe, ju-ɕe ɲɯ-ŋu, ɯ-ku nɯ kɯ-ɤrqhi tsa ju-tsrɤt ju-te tɕe, jme nɯ ju-mɟe tɕe ɯ-ku ɯ-ɕki ʑo ju-sɤzɣɯt tɕe, li ɯ-ku nɯ ju-tsrɤt, tɕe nɯ kɯ-fse ju-ɕe ɲɯ-ŋu tɕe, núndʐa qajɯ kɯ-rɤtɣa ɲɯ-rmi.\cmn 尺蠖是一种又软又小的虫子,不像其它虫子一样爬着走,好像是人在用手指量尺寸那样走动,先把头部往前伸出去,然后把尾部收到头部那里,然后又把头部伸出去,就这样行走,所以叫作尺蠖。\end{exemple}\end{entrée}

\begin{entrée}
\vedette{\hypertarget{Ⓔqajɯkɯsɤtʂot}{\papi{ qajɯkɯsɤtʂot}}}\markboth{qajɯkɯsɤtʂot}{}
\classe{n}
\begin{définition}\fra luciole\end{définition}
\begin{définition}\cmn 萤火虫\end{définition}\end{entrée}

\begin{entrée}
\vedette{\hypertarget{Ⓔqajɯsmɤnba}{\papi{ qajɯsmɤnba}}}\markboth{qajɯsmɤnba}{}
\classe{n}
\begin{définition}\fra sangsue\end{définition}
\begin{définition}\cmn 水蛭\end{définition}\end{entrée}

\begin{entrée}
\vedette{\hypertarget{Ⓔqajɯstoʁ}{\papi{ qajɯstoʁ}}}\markboth{qajɯstoʁ}{}\classe{n}
\begin{définition}\fra un insecte\end{définition}
\begin{définition}\cmn 昆虫是一种\end{définition}
\begin{exemple}\jya qajɯ stoʁ nɯ qajɯ kɯ-ɲaʁ ci ŋu, ɯ-ʁar ɯ-rqhu nɯ kɯ-rkɯ-rko ci ŋu, tɕe nɤmbju, ɯ-xtu fka, ɯ-smɤt tɕe chɯ-ɤmtɕoʁ ŋu, ɯ-ku kɯ-xtɕɯ-xtɕi ŋu, ɯ-mtɕhi amtɕoʁ, ɯ-mɤlɤjaʁ kɯtʂɤ-ldʑa tu, zndɤrchɤβ, soʁma ɯ-rchɤβ, aʁɤndɯndɤt kɯ-ɴqhi nɯ ra ku-rɤʑi ŋu.\cmn 
\stylefv{qajɯ stoʁ}是一种黑色的虫,翅膀的壳很硬,有光泽,肚子很胀,尾部是尖的,头部很小,嘴很尖,有六只脚,生活所有脏的地方,比如在墙壁缝和干草堆
\end{exemple}\end{entrée}

\begin{entrée}
\vedette{\hypertarget{Ⓔqajʑmbraʁ}{\papi{ qajʑmbraʁ}}}\markboth{qajʑmbraʁ}{}
\classe{n}
\begin{définition}\fra barbe de blé\end{définition}
\begin{définition}\cmn 麦芒\end{définition}\end{entrée}

\begin{entrée}
\vedette{\hypertarget{Ⓔqaɟy}{\papi{ qaɟy}}}\markboth{qaɟy}{} (\variante{qaɟwi}) 
\classe{n}
\begin{définition}\fra poisson\end{définition}
\begin{définition}\cmn 鱼\end{définition}\end{entrée}

\begin{entrée}
\vedette{\hypertarget{Ⓔqaɟɤɣi}{\papi{ qaɟɤɣi}}}\markboth{qaɟɤɣi}{}
\classe{n}
\begin{définition}\fra avoine\end{définition}
\begin{définition}\cmn 燕麦\end{définition}
\begin{exemple}\jya qaɟɤɣi nɯ sɯjno ci ŋu, ɯ-jwaʁ ɯ-ru nɯ ra tɤɕi cho naχtɕɯɣ, ɯ-jwaʁ ɯ-βzɯr nɯ ɯ-rme kɯ-xtɕɯ-xtɕi tu, ɯ-mat nɯ tɤɕi cho mɤ-naχtɕɯɣ, mbrɤz kɯɕnom cho naχtɕɯɣ, ɯ-mat nɯ ɯ-rqhu wuma ʑo jaʁ tɕe, pjɯ-tsɣi mɤ-cha, tɕe tɕhi kɯ-fse tɤ-nɯ-nɤrʑaʁ kɯnɤ pjɯ-tsɣi mɤ-cha tɕe tu-ɬoʁ ɕti, tɕe ɯ-ʑmbraʁ nɯ chɤ-ndzɯ-ndzri ŋu tɕe, nɯ-aci tɕe lu-orlɯ-rla ŋu, tɕe ɯ-mat tu-sɯ-mtɕɯr cha. qaɟɤɣi tɤ-rɤku ɯ-rchɤβ tu-ɬoʁ tɕe, tɤ-rɤku ɯ-taʁ ʁnɤt tɕe, sɯjno wuma ʑo kɯ-ŋɤn ŋu. tɕe kɯ-rɤma ra kɯ wuma ʑo qha-nɯ tɕe, tɕhi kɤ-cha ɲɯ-kɤ-ɣɤme ftɕaka tu-βzu-nɯ ŋu.\cmn 燕麦是一种草,叶子和茎和青稞一样,但是叶子的边缘有小毛,果实和青稞的不一样,和大米的穗一样。果实的皮很厚,不容易腐烂,再长的时间也不会腐烂,能生长。它的芒是拧着长的,受潮时就会自然松开,使果实转动。燕麦生长在庄稼里,对庄稼造成危害,是很坏的草,所以劳动人民很讨厌它,用一切办法来消灭它。\end{exemple}\end{entrée}

\begin{entrée}
\vedette{\hypertarget{Ⓔqaɟyri}{\papi{ qaɟyri}}}\markboth{qaɟyri}{}\classe{n}
\begin{définition}\fra type de pas d'aiguille\end{définition}
\begin{définition}\cmn 缝针的方法\end{définition}\end{entrée}

\begin{entrée}
\vedette{\hypertarget{Ⓔqala}{\papi{ qala}}}\markboth{qala}{}
\classe{n}
\begin{définition}\fra lapin\end{définition}
\begin{définition}\cmn 兔子\end{définition}\end{entrée}

\begin{entrée}
\vedette{\hypertarget{Ⓔqalalu}{\papi{ qalalu}}}\markboth{qalalu}{}\classe{n}
\begin{définition}\fra année du lapin\end{définition}
\begin{définition}\cmn 兔年\end{définition}
\end{entrée}

\begin{entrée}
\vedette{\hypertarget{Ⓔqalamɯjpɤt}{\papi{ qalamɯjpɤt}}}\markboth{qalamɯjpɤt}{}
\classe{n}
\begin{définition}\fra coton sauvage\end{définition}
\begin{définition}\cmn 野棉花\end{définition}
\begin{exemple}\jya qalamɯjpɤt nɯ tɯ-ji ɯ-rkɯ si ɯ-rchɤβ ra tu-ɬoʁ ŋu, ɯ-jwaʁ nɯ aɣrɤɣrum tɕe ɯ-rme kɯ-fse tu, ɯ-ru nɯ kɯ-ngɯ-ngɯt ʑo ŋu, ɯ-mɯntoʁ nɯ ɯ-ʁɤri nɯ kɯ-wɣrum, ɯ-qhuchu nɯ ʁmɤrsmɯɣ ŋu, ɯ-mat nɯ thɯ-tɯt tɕe, ɯ-ŋgɯ srɯn kɯ-fse ɲɯ-nɯɬoʁ ŋu, ɯ-rɣi nɯ kɯ-ndɯ-ndɯβ ʑo ŋu.\cmn 野棉花生长在田地边缘和树木之间,叶子淡白,有毛,茎很结实,花正面白色,背面紫色,果实成熟后,里面就漏出像棉花一样的东西,种子非常小。\end{exemple}\end{entrée}

\begin{entrée}
\vedette{\hypertarget{Ⓔqalarnaftɕɯχa}{\papi{ qalarnaftɕɯχa}}}\markboth{qalarnaftɕɯχa}{}\classe{n}
\begin{définition}\fra lapin ayant dix trous dans les oreilles\end{définition}
\begin{définition}\cmn 耳朵上有十个缺口的兔子(故事里)\end{définition}
\end{entrée}

\begin{entrée}
\vedette{\hypertarget{Ⓔqale}{\papi{ qale}}}\markboth{qale}{}
\classe{n}
\begin{définition}\fra vent\end{définition}
\begin{définition}\cmn 风\end{définition}
\begin{exemple}\jya qale to-βzu (jo-ɣɯt)\cmn 风刮起来了\end{exemple}
\begin{exemple}\jya iʑo pɯ-nɤʁaʁ-i tɕe pɯ-scit-i ri, tɯ-mɯ qale ja-ɣɯt tɕe tɤ-nɯɣe-j pɯ-ra\cmn 我们在外面晒太阳的时候,风刮起来了,下雨了,我们只好回家了\end{exemple}
\begin{relation-sémantique}\confer{
\hyperlink{Ⓔakɯchoʁle}{\textit{ \papi{akɯchoʁle}}}
}\end{relation-sémantique}\end{entrée}

\begin{entrée}
\vedette{\hypertarget{Ⓔqalekɯtshi}{\papi{ qalekɯtshi}}}\markboth{qalekɯtshi}{}
\classe{n}
\begin{définition}\fra rapace\end{définition}
\begin{définition}\cmn 鹰科,无法定到种\end{définition}
\begin{exemple}\jya qalekɯtshi nɯ pɣa kɯ-xtɕi tsa ci ŋu, ɲɯ-nɯqambɯmbjom ɯ-raŋ zɯ, nɤmkha zɯ tɯtshot χsɯ-skɤrma jamar ɯ-stu ku-rɤʑi ɲɯ-ŋgrɤl, tɕe tsuku kɯ tɕe qale ɯ-kɯ-tshi, tsɯku kɯ tɕe qale ɯ-kɯ-ndza tu-kɯ-ti ŋu, tɕhi tu-ste ŋgrɤl nɯ mɤ-xsi. ɯ-βri nɯ kɯ-pɣi tsa ci ŋu.\cmn 
\stylefv{qalekɯtshi}是一种比较小的鸟,飞的时候,可以在空中停住三分钟左右,有的人说它在挡风,有的说在吃风,不知道哪一种说法是对的。身子是灰色的。
\end{exemple}\end{entrée}

\begin{entrée}
\vedette{\hypertarget{Ⓔqaliaʁ}{\papi{ qaliaʁ}}}\markboth{qaliaʁ}{}
\classe{n}
\begin{définition}\fra aigle\end{définition}
\begin{définition}\cmn 雕
\begin{déclaration} \étymologie{\papi{glag}}\end{déclaration}\end{définition}
\begin{exemple}\jya qaliaʁ nɯ praʁ ɯ-ŋgɯ zɯ ku-rɤʑi ŋu, ʁnɯ-tɯphu tu, tɯ-tɯphu nɯ thaŋkɤr rmi, ɯʑo ɯ-mdoʁ nɯ kɯ-ɲaʁ ŋu, ɯ-ʁar nɯ kɯ-wɣrum tɯ-tɯ-snaʁ kɯ-tu tu, qala, βʑɯ, paʁtsa cho ɲaɲa nɯ ra tu-ndze ŋgrɤl. li ci thaŋnaʁ kɯ-rmi ci tu tɕe, kɯ-ɲaʁ ʁɟa ʑo ŋu, nɯ kɯ-cha nɯ ŋu, paʁtsa cho ɲaɲa nɯ phɤri ɲɯ-nɯ-tsɯm cha, ca pjɤ-sat ŋgrɤl. ɯ-mi nɯ ɕɤmiɕtʂɤt fse. mphrɯmɯ pjɯ-re ŋgrɤl tu-kɯ-ti ɲɯ-ŋu ma nɤmkha ɯ-stu ʑo tu-ɕe tɕe ɯ-stu li pjɯ-jɣɤt tɕe ɴɢartɯm pjɯ-ɣɯt ŋu. ɯ-mɲaʁ wuma ʑo mto tu-kɯ-ti ɲɯ-ŋgrɤl, phɤri ku-ru tɕe qaʑo kɯ ɯ-qe thɤstɯ-rdoʁ pa-lɤt mtɤm tu-kɯ-ti ɲɯ-ŋgrɤl. ɕa a-tɤ-ndze tɤ-fka ɯ-qhu tɕe, co a-pɯ-ŋu tɕe chɯ-nɯqambɯmbjom mɤ-cha.\cmn 
雕栖息在岩洞里,有两种,一种叫\stylefv{thaŋkɤr},是黑色的,翅膀下面有白点,吃兔子、老鼠、小猪、小羊等。另一种叫\stylefv{thaŋnaʁ},全身都是黑色的,比较凶,不但可以把小猪和小羊带走,而且能杀死麝香鹿。爪子像铁钩一样。人家说它会算卦,因为它在空中往上直飞,又转回往下直飞。据说它视力很强,能看清对面山上的羊屙的屎有多少颗。如果在山沟里的话,肉吃饱了以后就飞不起来。
\end{exemple}\end{entrée}

\begin{entrée}
\vedette{\hypertarget{Ⓔqalpɕa}{\papi{ qalpɕa}}}\markboth{qalpɕa}{}
\classe{vi}
\paradigme{\textit{dir :} \jya tɤ-}
\begin{définition}\fra s'ouvrir (feuille de fougère)\end{définition}
\begin{définition}\cmn 展开(蕨苔的叶子)【开反】\end{définition}
\begin{exemple}\jya dɤrʁɯ to-qalpɕa\cmn 蕨苔展开了\end{exemple}\end{entrée}

\begin{entrée}
\vedette{\hypertarget{Ⓔqambalɯla}{\papi{ qambalɯla}}}\markboth{qambalɯla}{}
\classe{n}
\begin{définition}\fra papillon\end{définition}
\begin{définition}\cmn 蝴蝶\end{définition}
\begin{exemple}\jya qambalɯla kɯ rŋgɯ mɤ-fkaβ\cmn 蝴蝶盖不住大石包\end{exemple}\end{entrée}

\begin{entrée}
\vedette{\hypertarget{Ⓔqambɣo}{\papi{ qambɣo}}}\markboth{qambɣo}{}
\classe{n}
\begin{définition}\fra cérumen\end{définition}
\begin{définition}\cmn 耳垢\end{définition}\end{entrée}

\begin{entrée}
\vedette{\hypertarget{Ⓔqambrɯ}{\papi{ qambrɯ}}}\markboth{qambrɯ}{}
\classe{n}
\begin{définition}\fra yak\end{définition}
\begin{définition}\cmn 公牦牛\end{définition}\end{entrée}

\begin{entrée}
\vedette{\hypertarget{Ⓔqambɯt}{\papi{ qambɯt}}}\markboth{qambɯt}{}
\classe{n}
\begin{définition}\fra sable\end{définition}
\begin{définition}\cmn 沙子\end{définition}\end{entrée}

\begin{entrée}
\vedette{\hypertarget{Ⓔqamdɯxtsa}{\papi{ qamdɯxtsa}}}\markboth{qamdɯxtsa}{}
\classe{n}
\begin{définition}\fra botte dont le haut est est peau de chevrotain, et le milieu en cuir teint en rouge\end{définition}
\begin{définition}\cmn 靴筒上部是獐皮子,下部是染成红色的牛皮的一种靴子\end{définition}\end{entrée}

\begin{entrée}
\vedette{\hypertarget{Ⓔqamdzi}{\papi{ qamdzi}}}\markboth{qamdzi}{}\classe{n}
\begin{définition}\ 
\begin{déclaration}\grammar{n.lieu}\end{déclaration}\end{définition}
\begin{définition}\fra Dkarmdzes\end{définition}
\begin{définition}\cmn 甘孜州\end{définition}\end{entrée}

\begin{entrée}
\vedette{\hypertarget{Ⓔqame}{\papi{ qame}}}\markboth{qame}{}
\classe{n}
\begin{définition}\fra grain de beauté\end{définition}
\begin{définition}\cmn 黑痔\end{définition}\end{entrée}

\begin{entrée}
\vedette{\hypertarget{Ⓔqamphoʁ}{\papi{ qamphoʁ}}}\markboth{qamphoʁ}{}\classe{n}
\begin{définition}\fra feuille de chêne\end{définition}
\begin{définition}\cmn 青冈树的叶子\end{définition}\end{entrée}

\begin{entrée}
\vedette{\hypertarget{Ⓔqamphoʁɕɯrʁaʁ}{\papi{ qamphoʁɕɯrʁaʁ}}}\markboth{qamphoʁɕɯrʁaʁ}{}\classe{n}
\begin{définition}\fra espèce de champignon\end{définition}
\begin{définition}\cmn 菌子的一种\end{définition}\end{entrée}

\begin{entrée}
\vedette{\hypertarget{Ⓔqamtɕɯr}{\papi{ qamtɕɯr}}}\markboth{qamtɕɯr}{}
\classe{n}
\begin{définition}\fra musaraigne\end{définition}
\begin{définition}\cmn 尖鼠;鼩鼱\end{définition}
\begin{exemple}\jya qamtɕɯr nɯ βʑɯ cho ndʑi-rme ra naχtɕɯɣ, tɕeri qamtɕɯr nɯ ɯ-mtɕhi mɤʑɯ ʑo amtɕoʁ, ɯ-jme xtɯt cho xtshɯm, qamtɕɯr nɯ βʑɯ sɤznɤ khro xtɕi, βʑɯ kɯ tɤ-mthɯm, tɯ-jpu, tɯ-ŋga tɕhi kɯ-tu tu-ndze ŋu ma qamtɕɯr kɯ tɤ-mthɯm kɯnɤ kɯ-tshu ma mɤ-ndze tɕeri tɤ-mthɯm tu-ndze ɯ-qhu ɯ-sta rmbi ɲɯ-lɤt ŋu tɕe ɯ-di wuma ʑo sɤjloʁ tɤ-mthɯm kɤ-ndza mɤ-kɯ-sna ɲɯ-sɤβze cha, tɕe wuma ʑo ŋɤn. tu-mbri tɕe ɯ-skɤt wuma ʑo amtɕoʁ χɕɤβ.\cmn 鼩鼱和老鼠的毛一样,但鼩鼱的嘴比较尖,尾巴细而短。鼩鼱比老鼠小得多。老鼠吃肉、衣服、粮食,有什么就吃什么,而鼩鼱只吃肥肉,吃肉之后还在那里撒尿,味道很臭,使得肉不能吃,是很坏的动物。叫起来声音又尖又大。\end{exemple}\end{entrée}

\begin{entrée}
\vedette{\hypertarget{Ⓔqamtsɯrmdzu}{\papi{ qamtsɯrmdzu}}}\markboth{qamtsɯrmdzu}{}\classe{n}
\begin{définition}\fra espèce de plante\end{définition}
\begin{définition}\cmn 植物的一种\end{définition}\end{entrée}

\begin{entrée}
\vedette{\hypertarget{Ⓔqamtsɯrpɣɤtɕɯ}{\papi{ qamtsɯrpɣɤtɕɯ}}}\markboth{qamtsɯrpɣɤtɕɯ}{}
\classe{n}
\begin{définition}\fra espèce indéterminée\end{définition}
\begin{définition}\cmn 鸟的一种\end{définition}\end{entrée}

\begin{entrée}
\vedette{\hypertarget{Ⓔqamɯrwa}{\papi{ qamɯrwa}}}\markboth{qamɯrwa}{}
\classe{n}
\begin{définition}\fra chauve-souris\end{définition}
\begin{définition}\cmn 蝙蝠\end{définition}\end{entrée}

\begin{entrée}
\vedette{\hypertarget{Ⓔqandʐe}{\papi{ qandʐe}}}\markboth{qandʐe}{}
\classe{n}\acception{1}
\begin{définition}\fra ver de terre\end{définition}
\begin{définition}\cmn 蚯蚓\end{définition}\acception{2}
\begin{définition}\fra nom d'une constellation\end{définition}
\begin{définition}\cmn 星宿的名字,东边升起来落到西边去\end{définition}
\begin{exemple}\jya qandʐe ɯ-jme\cmn 蚯蚓星宿\end{exemple}\end{entrée}

\begin{entrée}
\vedette{\hypertarget{Ⓔqandʐethɤlwɤɕtʂat}{\papi{ qandʐethɤlwɤɕtʂat}}}\markboth{qandʐethɤlwɤɕtʂat}{}\classe{n}
\begin{définition}\fra riche mais économe\end{définition}
\begin{définition}\cmn 虽然很富有但是非常节约的人\end{définition}\end{entrée}

\begin{entrée}
\vedette{\hypertarget{ⒺqandʐiⒽ2}{\papi{ qandʐi}}}\markboth{qandʐi}{}\homonyme{2}\classe{n}
\begin{définition}\fra salmonidé\end{définition}
\begin{définition}\cmn 鲑【猫儿鱼】\end{définition}
\begin{exemple}\jya qandʐi nɯ qaɟy ɯ-ngɯz stu kɯ-wxti ŋu, ɯ-ɕɣa tu, ɯ-mtɕhirme tu, kú-wɣ-sqa tɕe ɯ-ɕa ɲɯ-ndʐi mɤ-cha, ɯ-punaŋtɕa tɕhɯtɯɣ ɯ-smɤn ŋu, ʁnɯz ʁnɯz tɯtɯrca tu-ŋke ɲɯ-ŋgrɤl, qaɟy kɯ-fse kɯ-dɤn maŋe, zlawa χsɯmba raŋ tɕe tɯ-ci ɯ-ŋgɯ tɤton lu-ɣi ɲɯ-ŋgrɤl. nɯ lɤ-ɣe ɯ-raŋ mdaʁʑɯɣ pjɯ́-wɣ-sɤtsa tɕe, pjɯ́-wɣ-sat pjɤ-ŋgrɤl. kɯ-wxti kɯ tɯrme ɯ-fsu tu ɲɯ-ŋgrɤl, kɯ-xtɕi nɯ tɯ-tɯ-ɣa jamar ma kɯ-me tu ɲɯ-ŋgrɤl. nɯ kɯnɤ qaɟy kɯ-wxti chɯ-ndze ɲɯ-ŋu.\cmn 鲑在鱼当中是最大的一种,有牙齿,有胡须。肉煮不化,其内脏是一种药材,如果牛因喝水而中毒,可以用此解毒。一般是一对一对地游动,没有像其它鱼那么多。到了三月,它们往河流的上游游去,正当这个时候,插下长矛把它杀死。大的和人一样大,小的只有一拃长。鲑可以吃其它比较大的鱼。\end{exemple}\end{entrée}

\begin{entrée}
\vedette{\hypertarget{ⒺqandʐiⒽ1}{\papi{ qandʐi}}}\markboth{qandʐi}{}\homonyme{1}\classe{vi}
\paradigme{\textit{dir :} \jya kɤ-}
\paradigme{\textit{dir :} \jya tɤ-}
\begin{définition}\fra noirâtre, sombre, violacé\end{définition}
\begin{définition}\cmn 乌;紫\end{définition}
\begin{exemple}\jya tɯ-mɯ ko-qandʐi\cmn 天很阴(要下雨了)\end{exemple}\begin{sous-entrée}
\vedette{\hypertarget{}{\papi{ sqandʐi}}}\markboth{sqandʐi}{}
\paradigme{\textit{dir :} \jya tɤ-}
\begin{définition}\ 
\begin{déclaration}\grammar{caus}\end{déclaration}\end{définition}
\begin{définition}\fra faire un bleu\end{définition}
\begin{définition}\cmn 弄紫\end{définition}
\begin{exemple}\jya tɤ́-wɣ-xtsɯɣ-a tɕe, to-sqandʐi\cmn 他打中我了,就弄紫了\end{exemple}
\begin{exemple}\jya nɤ-rŋa kɤ-tɯ-nɯ-rpu-t tɕe to-sqandʐi\cmn 你撞到脸,就弄紫了\end{exemple}\classe{vt}
\end{sous-entrée}\end{entrée}

\begin{entrée}
\vedette{\hypertarget{Ⓔqandzɤjo}{\papi{ qandzɤjo}}}\markboth{qandzɤjo}{}
\classe{n}
\begin{définition}\fra espèce d'arbrisseau\end{définition}
\begin{définition}\cmn 灌木的一种\end{définition}
\begin{exemple}\jya qandzɤjo nɯ si kɯ-mbɤr tsa ci ŋu, ɯ-jwaʁ ndɯβ, ɯʑo tu-mbro cho ɲɯ-jpum mɤ-cha, ɯ-si ngɯt tɕe kɯɕɯŋgɯ tɕe ndʑu ɯ-spa pɯ-ŋu. ɯ-mɯntoʁ kɯ-ndɯβ tsa ɲɯ-lɤt ŋu tɕe, wɣrum.\cmn 
\stylefv{qandzɤjo}是一种矮小的树,叶子细小,长不高也长不粗,木质结实,是过去作筷子的材料。开细小的白花。
\end{exemple}\end{entrée}

\begin{entrée}
\vedette{\hypertarget{Ⓔqandzɤjoɕku}{\papi{ qandzɤjoɕku}}}\markboth{qandzɤjoɕku}{}\classe{n}
\begin{définition}\fra poireau sauvage\end{définition}
\begin{définition}\cmn 高山上的野韭菜\end{définition}
\begin{exemple}\jya qandzɤjo ɕku nɯ zgoku kɯ-mbro ʑo tu-ɬoʁ ŋu, sɤtɕha kɯ-mɯɕtaʁ tsa tu-ɬoʁ cha. ɯ-mdoʁ pɣi, si kɯ-ndɯβ ɯ-rchɤβ ra tu-ɬoʁ ŋu. ɯ-qa ri ɯ-zrɤm sɤɣ-ndzoʁ ɯ-stu kɯ-xtɕɯ-xtɕi jpum. ɯ-zrɤm nɯ kɯ-xtshɯm, kɯ-wɣrum ŋu. ɯ-ru tu-tɯ-ɬoʁ ɯ-stu nɯ arɤʑɯʑrɤz, ɯ-ŋgɯ tɕe ɯ-spjɯŋ tu-ɬoʁ tɕe, ɯ-taʁ tɕe li ɯ-jwaʁ ɲɯ-βze ŋu. ɯ-jwaʁ nɯ tɕɤr tɕe rɲɟi, ɯ-jwaʁ χsɯm jamar tɤ-ɬoʁ ɯ-qhu tɕe, tɕe ɯ-spjɯŋ ɯ-ŋgɯ ɯ-ku nɯ tɕu ɲɯ-rɯmɯntoʁ ŋu. pha ɯ-phoŋbu nɯ mɤrtsaβ, ɯ-dɯχɯn χɕɤβ. kɤ-ndza sna.\cmn 
\stylefv{qandzɤjoɕku}生长在高山上,能生长在比较寒冷的地方,是灰色的,一般生长在灌木丛里。长根的部位有点粗。根又细又白。茎长出来的部位有条纹,里面长主心干,在上面又长叶子。叶子细而长。叶子长到两三片时,在顶部开花。全身都是辣的,香味很浓。可以吃。
\end{exemple}
\end{entrée}

\begin{entrée}
\vedette{\hypertarget{Ⓔqandzi}{\papi{ qandzi}}}\markboth{qandzi}{}\classe{n}
\begin{définition}\fra type de sapin\end{définition}
\begin{définition}\cmn 杉树的一种\end{définition}\end{entrée}

\begin{entrée}
\vedette{\hypertarget{Ⓔqandʑɣi}{\papi{ qandʑɣi}}}\markboth{qandʑɣi}{}
\classe{n}
\begin{définition}\fra faucon (falco cherrug)\end{définition}
\begin{définition}\cmn 猎隼\end{définition}
\begin{exemple}\jya qandʑɣi nɯ kɯ-pɣi ci ŋu, ɯ-ʁar ra zɯmi kɯ-wɣrum tɯ-snaʁ ka tu, ɕa rga, stoʁ ji ɯ-raŋ tɕe ju-ɣi ŋu, qartsɯ tɕe ju-nɯɕe ŋu. kɯrɯ ra kɯ tu-ti-nɯ stu kɯ-mɤku pɯ́-wɣ-mto tɕe, ɯ-mgɯr ɯ-qhu a-pɯ-mto tɕe, tɯ-xpa kɯ-fka tu-kɯ-ti ŋgrɤl, ɯ-xtu kɤ-mto a-pɯ́-wɣ-z-mɤku tɕe kɯ-mtsɯr tu-kɯ-ti ŋɯ-ŋgrɤl. tɯ-jaʁ tɤ-mthɯm tú-wɣ-ndo tɕe pjɯ-ɣi tɕe tɯ-jaʁ pjɯ-qraʁ tɕe tɤ-mthɯm ɲɯ-kɯ-nɯsɯkho ɲɯ-ŋgrɤl.\cmn 猎隼是灰色的,翅膀上有白色斑点,爱吃肉,在种胡豆的季节出现,到了冬天就回去。藏民有一种说法,当它第一次出现时,如果先看见它的背部,这一年吃得饱,如果先看见它的腹部就会挨饿。如果你手上拿着肉,它会飞下来,抓破你的手,然后把肉抢走了。\end{exemple}\end{entrée}

\begin{entrée}
\vedette{\hypertarget{Ⓔqandʑi}{\papi{ qandʑi}}}\markboth{qandʑi}{}\classe{n}
\begin{définition}\fra étain\end{définition}
\begin{définition}\cmn 锡\end{définition}
\end{entrée}

\begin{entrée}
\vedette{\hypertarget{Ⓔqanɯ}{\papi{ qanɯ}}}\markboth{qanɯ}{}
\classe{vs}
\paradigme{\textit{dir :} \jya kɤ-}
\begin{définition}\fra sombre\end{définition}
\begin{définition}\cmn 暗,黑(天色)\end{définition}
\begin{exemple}\jya ʑa qanɯ ɲɯ-ŋu\cmn 快要黑了\end{exemple}
\begin{exemple}\jya ko-qanɯ\cmn 天黑了\end{exemple}
\begin{relation-sémantique}\confer{
\hyperlink{Ⓔsqanɯ}{\textit{ \papi{sqanɯ}}}
}\end{relation-sémantique}\end{entrée}

\begin{entrée}
\vedette{\hypertarget{Ⓔqaɲi}{\papi{ qaɲi}}}\markboth{qaɲi}{}\classe{n}
\begin{définition}\fra taupe\end{définition}
\begin{définition}\cmn 鼹鼠【田鼠】\end{définition}
\begin{exemple}\jya qaɲi nɯ ɯ-mdoʁ cho ɯ-rme nɯ ra βʑɯ kɯ-fse ŋu, ɯʑo ɯ-tshɯɣa nɯ ɯ-mɤlɤjaʁ ra ɲɯ-xtɯt, ɯ-ndzrɯ wuma ɲɯ-mtɕoʁ, ɯ-phoŋbu ɲɯ-tshu, ɯ-jme kɯ-xtɯ-xtɯt kɯ-xtshɯm ci ɣɤʑu, ɯ-ku nɯ βʑɯ kɯ-fse mɯ-ɲɯ-ɤmtɕoʁ, ɯ-rna nɯ ɯ-rme ɯ-ŋgɯ ku-kɯ-raʁ kɯ-fse ci ma maŋe, ɯ-mɲaʁ kɯ-xtɕɯ-xtɕi ɲɯ-ŋu, ɯ-ɕna paʁ ɯ-ɕna tsa ɲɯ-fse tɕe ɲɯ-rko ma sɤtɕha ɯ-pa ɯ-ɕna kɯ ju-sɯ-sloʁ ɲɯ-ra. ɯ-mɤlɤjaʁ kɯ ju-z-rɤβraʁ ɲɯ-ra tɕe ɯ-jroʁ ju-tɕɤt tɕe ɯʑo ju-ɕe pjɯ-tɕhɯt ɲɯ-ra. tɕe ɯ-jroʁ ja-tɕɤt tɕe, ɯʑo tshɯrɟɯn tu-ŋke kɯ-ra nɯ ra ɲɯ-jom. tɕe stonka tɕe sɯjno ɯ-qa, tɤtsoʁ ɯ-qa, tɤ-rɤku ɯ-mat nɯ ra ɲɯ-sɤjti tɕe ɯ-jroʁ ɯ-ŋgɯ nɯ ju-sɯ-mtshɤt, tɕe qartsɯ ɯ-ndza spa ɲɯ-ŋu. qartsɯ tɕe mɯ-tha-ɕkɯt nɯ, ftɕar tɕe tu-ɬoʁ tɕe ɲɯ-saχsɤl. kɯ-sɤmtshɤr nɯ tɤ-rɤku ɯ-mat pa-nɯ-phɯt tɕe, ɯ-ru maka mɤ-kɯ-ɴɢlɯt, mɤ-kɯ-ɤjʁu tu-βze ɲɯ-cha. kɯɕnom nɯ kɤ́-ʑmbɯ-ʑmbraʁ ju-tsɯm ɲɯ-ŋu, stoʁ staχpɯ nɯ ɯ-qiɯ nɯ ɯ-rdoʁ ju-tsɯm ɲɯ-ŋu, ɯ-qiɯ nɯ kɤ́-rqhɯ-rqhu ju-tsɯm ɲɯ-ŋu.\cmn 鼹鼠毛色和老鼠一样,它的形状是:四肢短、爪子很锐利,身子很肥,有一条又细又短的尾巴,头不像老鼠的那么尖,耳朵陷在毛里,眼睛很小。鼻子有点像猪的鼻子但比较硬,因为它在地下用鼻子拱土。它用四肢挖土,要挖出一条容得下它身子的通道。通道挖了以后,它经常经过的地方宽一些。到了秋天,它就把草根、人参果的根和粮食储存起来,塞满它的通道,作冬天的食物。冬天没有吃完的,到春天会长出来就可以发现。奇怪的是它摘了庄稼的果实,能做到秆不折断也不弯,穗子连同芒一起带去。胡豆和豌豆有一半只拿颗粒,另一半则连壳一起拿去。\end{exemple}
\begin{relation-sémantique}\confer{
\hyperlink{Ⓔqaɲɯɣɲɟɯ}{\textit{ \papi{qaɲɯɣɲɟɯ}}}
}\end{relation-sémantique}\end{entrée}

\begin{entrée}
\vedette{\hypertarget{Ⓔqaɲɯɣɲɟɯ}{\papi{ qaɲɯɣɲɟɯ}}}\markboth{qaɲɯɣɲɟɯ}{}\classe{n}
\begin{définition}\fra taupière\end{définition}
\begin{définition}\cmn 鼹鼠洞\end{définition}
\begin{relation-sémantique}\confer{
\hyperlink{Ⓔqaɲi}{\textit{ \papi{qaɲi}}}
}\end{relation-sémantique}
\begin{relation-sémantique}\confer{
\hyperlink{Ⓔɯ-ɣɲɟɯ}{\textit{ \papi{ɯ-ɣɲɟɯ}}}
}\end{relation-sémantique}\end{entrée}

\begin{entrée}
\vedette{\hypertarget{Ⓔqapar}{\papi{ qapar}}}\markboth{qapar}{}
\classe{n}
\begin{définition}\fra cuon alpinus\end{définition}
\begin{définition}\cmn 豺\end{définition}
\begin{exemple}\jya qapar nɯ khɯna kɯ-fse ŋu, kɯ-pɣi kɯ-ɤɣɯrnɯɕɯr ŋu, ɕnɤcat tɯtɯrca tu ɲɯ-ngrɤl, fsapaʁ tu-ndze tɕe, tu-βɟi ɯ-ʑɤrʑɯr tɤ-sŋɯt tu-lɤt tɕe fsapaʁ tu-ndze ɲɯ-ŋgrɤl, fsapaʁ pjɯ-si ɕɯŋgɯ tɕe ɯ-xtu ɯ-ŋgɯ chɯ-ɕe tɕe ɯ-punaŋtɕa chɯ-ɕkɯt ɲɯ-ŋgrɤl. ɯ-mɤlɤjaʁ khɯna ɯ-mɤlɤjaʁ kɯ-fse ɲɯ-ŋu, ɯ-jme khɯna jme staʁ jpum, zoŋzoŋ pa.\cmn 豺狗像狗一样,灰里带红色,八九只一起行动,吃牲畜,一边追一边咬住就吃,在牲畜死之前,它钻进牲畜肚子里把内脏吃完。四肢和狗的一样,尾巴比狗的粗,毛茸茸的。\end{exemple}\end{entrée}

\begin{entrée}
\vedette{\hypertarget{Ⓔqapɤtɯm}{\papi{ qapɤtɯm}}}\markboth{qapɤtɯm}{}
\classe{n}
\begin{définition}\fra silex arrondi\end{définition}
\begin{définition}\cmn 圆形的燧石\end{définition}\end{entrée}

\begin{entrée}
\vedette{\hypertarget{Ⓔqapɣɤmtɯmtɯ}{\papi{ qapɣɤmtɯmtɯ}}}\markboth{qapɣɤmtɯmtɯ}{}
\classe{n}
\begin{définition}\fra houppe\end{définition}
\begin{définition}\cmn 戴胜\end{définition}
\begin{exemple}\jya qapɣɤmtɯmtɯ nɯ tɯ-ci ɯ-rkɯ ntsɯ ku-rɤʑi ŋu, tɯ-ci ɯ-ŋgɯ qajɯ ra tu-ndze ŋu, ɯʑo kɯ-xtɕi tsa ci ŋu, ɯ-mtsioʁ mɤ-rɲɟi, ɯ-kɤχcɤl ri ɯ-rme tɯ-mtɕoʁ tu tɕe nɯ ɯ-mtɯ ŋu tu-kɯ-ti ɲɯ-ŋu tɕe, tɕe tɤ-mbri tɕe ɯ-mtɯ nɯ ɲɤ-χtɤr kɯ-fse, ɯ-phoŋbu ɯ-muj ra mpɕɤr nɤmbju, ɯ-jme khro mɤ-rɲɟi, ɯ-mi qarŋe tsa ŋu.\cmn 戴胜生活在大河边,吃河里的虫子,它身子小,嘴不长,在头顶有一撮毛(羽冠),人家说是它的髻。它每叫一声,就会把羽冠散开一下。身上的羽毛很美丽,有光泽,尾巴不长,脚是淡黄色的。\end{exemple}\end{entrée}

\begin{entrée}
\vedette{\hypertarget{Ⓔqapɣo}{\papi{ qapɣo}}}\markboth{qapɣo}{}\classe{n}
\begin{définition}\ 
\begin{déclaration}\grammar{n.lieu}\end{déclaration}\end{définition}
\begin{définition}\fra un village de Sarndzu\end{définition}
\begin{définition}\cmn 沙尔宗的一个村\end{définition}\end{entrée}

\begin{entrée}
\vedette{\hypertarget{ⒺqapiⒽ1}{\papi{ qapi}}}\markboth{qapi}{}\homonyme{1}\classe{n}
\begin{définition}\fra silex\end{définition}
\begin{définition}\cmn 燧石\end{définition}
\begin{exemple}\jya qapi nɯ rdɤstaʁ kɯ-wɣrum ŋu, kha znde ɣɯ ɯ-kɤχcɤl kɯ-ɤmtɕoʁ zɯ ɲɯ́-wɣ-ta ŋgrɤl tɕe βzɯrtɕoʁ ɲɯ-rmi. tɯ-ji kɯ-wxti ɣɯ ɯ-χcɤl ɲɯ́-wɣ-ta ɲɯ-ŋgrɤl tɕe, nɯ tɕu tɕe ʑaŋɬa ɲɯ-rmi. tɕe qapi nɯ ɯ-taʁ tɕaʁmɤr pjɯ́-wɣ-lɤt tɕe smɯtɕɣom tu-ɬoʁ ŋgrɤl tɕe kɯɕɯŋgɯ βʁɯz smi ɯ-sɤ-sɯ-mɟa nɯ qapi cho tɕaʁmɤr ni kɯ tú-wɣ-sɯ-βzu-nɯ pjɤ-ŋu. smi kɯ-me tɕe tɯrme tɯ-ndzɤtshi kɤ-βzu mɤ-khɯ tɕe kɯɕɯŋgɯ qapi nɯ wuma ʑo kɯ-ʁzi pjɤ-ɕti.\cmn 
\stylefv{qapi}是一种白石头,这种石头放在屋顶的四角上,这时候叫\stylefv{βzɯrtɕoʁ}。有人会把它放在最大的田地中间,这时候叫\stylefv{ʑaŋɬa}。在白石头上打火镰就会迸出火星。过去,引燃火绒的就是白石头和火镰。没有火,人们就不能做饭,所以过去白石头是非常重要的一个东西。
\end{exemple}\end{entrée}

\begin{entrée}
\vedette{\hypertarget{ⒺqapiⒽ2}{\papi{ qapi}}}\markboth{qapi}{}\homonyme{2}
\classe{n}
\begin{définition}\fra une espèce de cerisier\end{définition}
\begin{définition}\cmn 野樱桃的一种\end{définition}
\begin{exemple}\jya qapi nɯ si kɯ-mbɯ-mbro ci ŋu, ɯ-ru wuma ɲɯ-jpum mɤ-cha, ɯ-ru ɯ-rtaʁ nɯ ra ɯ-mdzu kɯ-χɕu kɯ-jpum ʑo tu, ɯ-jwaʁ mɤ-jndʐɤz, kɯ-ɤrtɯm tsa ŋu. ɯ-mɯntoʁ kɯ-wɣrum ɲɯ-lɤt ŋu, ɯ-mat tha-βzu tɕe, kɯβde kɯmŋu jamar tɯtɯrca ku-ndzoʁ, ʑakastaka nɯ-jɯ kɯ-xtshɯm tɯ-ka tu. qapi zgo kɯ-mbro tsa kɯ-mbɤr tsa aʁɤndɯndɤt tu-ɬoʁ cha.\cmn 野樱桃长得很高,但树干不是很粗,树干和树枝上都长有又尖又粗的刺,叶子不大,呈圆形。开白花。结果的时候,四五个结在一起,每一个都有细小的茎。山上山下都可以生长。\end{exemple}
\end{entrée}

\begin{entrée}
\vedette{\hypertarget{Ⓔqaprɤftsa}{\papi{ qaprɤftsa}}}\markboth{qaprɤftsa}{}
\classe{n}
\begin{définition}\fra mille patte\end{définition}
\begin{définition}\cmn 蜈蚣\end{définition}\end{entrée}

\begin{entrée}
\vedette{\hypertarget{Ⓔqaprɤkhɯsloŋ}{\papi{ qaprɤkhɯsloŋ}}}\markboth{qaprɤkhɯsloŋ}{}
\classe{n}
\begin{définition}\fra Arisaema consanguineum\end{définition}
\begin{définition}\cmn 天南星\end{définition}
\begin{exemple}\jya qaprɤkhɯsloŋ nɯ sɯjno ci ŋu, kɯ-mɤku ɯ-ru kɯ-jpum ʑo tu-ɬoʁ tɕe, ɯ-taʁ tsa ri tɕe, ɯ-jwaʁ ɲɯ-ɬoʁ, ɯ-jwaʁ ɯ-sɤɣɬoʁ tɯ-ldʑa ma me, ɯ-mat nɯ jima ɯ-mat cho naχtɕɯɣ, thɯ-tɯt tɕe chɯ-ɣɯrni ŋu. ɯ-qa nɯ kɯ-ɤrtɯm ɲɯ-βze ŋu, sɤndɤɣ.\cmn 天南星是一种植物。首先,长出一根粗壮的茎,长了一段后,才开始长出叶子,长叶的枝只有一根,果实像玉米果实一样,成熟了就变红。根是圆形的。有毒性。\end{exemple}\end{entrée}

\begin{entrée}
\vedette{\hypertarget{Ⓔqaprɤsi}{\papi{ qaprɤsi}}}\markboth{qaprɤsi}{}
\classe{n}
\begin{définition}\fra une espèce de chêne\end{définition}
\begin{définition}\cmn 槲栎的一种\end{définition}\end{entrée}

\begin{entrée}
\vedette{\hypertarget{Ⓔqapri}{\papi{ qapri}}}\markboth{qapri}{}\classe{n}
\begin{définition}\fra serpent\end{définition}
\begin{définition}\cmn 蛇
\begin{déclaration} \étymologie{\papi{sbrul}}\end{déclaration}\end{définition}
\begin{exemple}\jya qapri ɯ-χsjɯβ chɤ-βde\cmn 蛇脱皮了\end{exemple}
\begin{relation-sémantique}\confer{
\hyperlink{Ⓔtɕhɯχpri}{\textit{ \papi{tɕhɯχpri}}}
}\end{relation-sémantique}\end{entrée}

\begin{entrée}
\vedette{\hypertarget{Ⓔqapribɯxsi}{\papi{ qapribɯxsi}}}\markboth{qapribɯxsi}{}\classe{n}
\begin{définition}\fra serpent géant\end{définition}
\begin{définition}\cmn 巨蛇\end{définition}\end{entrée}

\begin{entrée}
\vedette{\hypertarget{Ⓔqaprilu}{\papi{ qaprilu}}}\markboth{qaprilu}{}\classe{n}
\begin{définition}\fra année du serpent\end{définition}
\begin{définition}\cmn 蛇年\end{définition}
\end{entrée}

\begin{entrée}
\vedette{\hypertarget{Ⓔqaprimdʑu}{\papi{ qaprimdʑu}}}\markboth{qaprimdʑu}{}
\classe{n}
\begin{définition}\fra une composée\end{définition}
\begin{définition}\cmn 【剪刀菜】\end{définition}
\begin{exemple}\jya qapri mdʑu nɯ sɯjno ɯ-zrɤm ra kɯ-xtɕi ci ŋu, ɯ-ru cho ɯ-jwaʁ nɯ ra kɯ-ɤrŋi ɯ-ŋgɯz kɯ-pɣi ŋu, ɯ-jwaʁ nɯ qapri mdʑu ɯ-tshɯɣa ɲɯ-fse, ɯ-mɯntoʁ nɤmkha mdoʁ ŋu, pjɯ́-wɣ-qlɯt tɕe ɯ-lu tu. fsapaʁ ra kɤ-ndza rga-nɯ, tɯrme kɤ-ndza mɤ-sna.\cmn 剪刀菜是根部不发达的植物,茎和叶子绿里带有灰色,叶子的形状像蛇的舌头,花是天蓝色的。折断的时候有乳汁。牲畜喜欢吃,人不能吃。\end{exemple}\end{entrée}

\begin{entrée}
\vedette{\hypertarget{Ⓔqapɯ}{\papi{ qapɯ}}}\markboth{qapɯ}{}\classe{vi}
\paradigme{\textit{dir :} \jya nɯ-}
\begin{définition}\fra tomber en friche\end{définition}
\begin{définition}\cmn 变成荒地\end{définition}
\begin{exemple}\jya tɯ-ji ra ɲɤ-qapɯ-nɯ\cmn 田地变成荒地\end{exemple}\begin{sous-entrée}
\vedette{\hypertarget{}{\papi{ sqapɯ}}}\markboth{sqapɯ}{}\classe{vt}
\paradigme{\textit{dir :} \jya nɯ-}
\begin{définition}\fra laisser en friche\end{définition}
\begin{définition}\cmn 让……变成荒地\end{définition}
\end{sous-entrée}\end{entrée}

\begin{entrée}
\vedette{\hypertarget{Ⓔqar}{\papi{ qar}}}\markboth{qar}{}
\classe{vs}
\begin{définition}\fra tabou\end{définition}
\begin{définition}\cmn 禁忌的地方\end{définition}
\begin{exemple}\jya ki sɤtɕha ɲɯ-qar\cmn 这个地方是禁忌\end{exemple}
\begin{exemple}\jya si kɤ-phɯt mɤ-βdi, sɤtɕha kɤ-lɣa mɤ-βdi. tɯ-ŋgo ɲɯ-sɤβze ɲɯ-ŋgrɤl, tu-kɯ-ɕɯngo ɲɯ-ŋu.\cmn (在那种地方)砍树是不好的,挖地是不好的。这样会令人得病\end{exemple}\end{entrée}

\begin{entrée}
\vedette{\hypertarget{Ⓔqarɤt}{\papi{ qarɤt}}}\markboth{qarɤt}{}
\classe{n}
\begin{définition}\fra râteau\end{définition}
\begin{définition}\cmn 耙子\end{définition}\end{entrée}

\begin{entrée}
\vedette{\hypertarget{Ⓔqarcɯm}{\papi{ qarcɯm}}}\markboth{qarcɯm}{}\classe{vs}
\paradigme{\textit{dir :} \jya thɯ-}
\begin{définition}\fra s'assombrir (ciel)\end{définition}
\begin{définition}\cmn 天阴\end{définition}
\begin{exemple}\jya tɯ-mɯ chɤ-qarcɯm\cmn 天阴了\end{exemple}
\begin{relation-sémantique}\synonyme{
\hyperlink{Ⓔqanɯ}{\textit{ \papi{qanɯ}}}
}\end{relation-sémantique}
\begin{relation-sémantique}\confer{
\hyperlink{Ⓔsqarcɯm}{\textit{ \papi{sqarcɯm}}}
}\end{relation-sémantique}\end{entrée}

\begin{entrée}
\vedette{\hypertarget{Ⓔqarɣɤpɤt}{\papi{ qarɣɤpɤt}}}\markboth{qarɣɤpɤt}{}
\classe{n}
\begin{définition}\fra une plante\end{définition}
\begin{définition}\cmn 【鹿茸花】\end{définition}
\begin{exemple}\jya qarɣɤpɤt nɯ, zgo wuma ʑo kɯ-mbro, sɤtɕha kɯ-ɣɤndʐo ɯ-pɕoʁ tu-ɬoʁ ŋu. ɯ-zrɤm mɯ́j-wxti, ɯ-ru ɯ-ŋgɯ qhoʁsjɯβ ɲɯ-ŋu, ɯ-ru ɯ-taʁ ɯ-jwaʁ ɲɯ-ɬoʁ tɕe, nɯ ɯ-rchɤβ ri ɯ-mɯntoʁ tu-ɬoʁ ɲɯ-ŋu. ɯ-ru cho ɯ-jwaʁ nɯ ra ɯ-rme rsɯβrsɯβ ʑo ɲɯ-pa, ɯ-jwaʁ cho ɯ-ru kɯ-ɤɣrɤɣrum ɲɯ-ŋu, ɯ-rme nɯ kɯ-pɣi ɲɯ-ŋu, ɯ-mɯntoʁ tɯ-rdoʁ tɯ-rdoʁ ɲɯ-ŋu, ɯ-mɯntoʁ ɯ-rqhu nɯ li ɯ-rme rsɯβrsɯβ ɲɯ-pa, ɯ-rqhu nɯ pɯ-ɴɢaʁ tɕe, ɯ-mɯntoʁ wuma ʑo kɯ-qarŋe ɲɤ-ɬoʁ ɲɯ-ŋu, ɯ-mɯntoʁ ɲɯ-wxti, kɯ-ɤrtɯ-rtɯm ɲɯ-ŋu, wuma ʑo ɲɯ-mpɕɤr.\cmn 鹿茸花生长在高山上,比较寒冷的地方里。根不大,茎是空心的,茎上长叶子,叶子和茎的中间长花。茎和叶子看起来毛茸茸的,茎和叶子带有一点白色,毛是灰色的,叶子是一朵一朵地长出来,花的外层也是毛茸茸的,外层剖开时,就开黄色的花,花很大,圆形,非常美。\end{exemple}\end{entrée}

\begin{entrée}
\vedette{\hypertarget{Ⓔqarɣe}{\papi{ qarɣe}}}\markboth{qarɣe}{}
\classe{n}
\begin{définition}\fra espèce d'arbrisseau\end{définition}
\begin{définition}\cmn 灌木的一种\end{définition}
\begin{exemple}\jya qarɣe nɯ si ci ŋu, khro mɤ-mbro, ɯ-ru nɯ kɯ-ɣɯrni ɯ-ŋgɯ ri kɯ-qarŋe tsa ŋu, ɯ-ru nɯ ndoʁ, χɕitka tɕe kɯmaʁ si ra ɕɯŋgɯ ɯ-jwaʁ ɲɯ-lɤt ɕɯŋgɯ ɲɯ-rɯmɯntoʁ ŋu. ɯ-ru cho ɯ-mɯntoʁ nɯ ɯ-dɯχɯn mnɤm, ndzɤtshi ɯ-cu kɤ-nɯlɤt sna\cmn 
\stylefv{qarɣe}是一种树,长的不高,树干红里带黄,木质很脆。到了春天,开花时间比其他树早,甚至在自己出叶之前开花,树干和花都有香味,可以用来作香料。
\end{exemple}\end{entrée}

\begin{entrée}
\vedette{\hypertarget{Ⓔqarɣɯ}{\papi{ qarɣɯ}}}\markboth{qarɣɯ}{}
\classe{n}
\begin{définition}\fra rosacée\end{définition}
\begin{définition}\cmn 蔷薇科的一种\end{définition}
\begin{exemple}\jya qarɣɯ nɯ si ci ŋu wuma sthɯci mɤ-mbro, ɯ-mdzu wuma ʑo jaʁ mtɕoʁ, ɯ-mdzu ɯ-kɤχcɤl ŋgɤɣ ʑo ŋu, ɯ-jwaʁ kɯ-ndɯβ tsa kɯ-ɤrtɯm ŋu, ɯ-jwaʁ βzɯr nɯ ra ɯ-mdzu kɯ-fse tu, ɯ-mɯntoʁ kɯ-ɣɯrni tu kɯ-wɣrum tu, ɯ-mat nɯ pɣa ra kɯ tu-ndza-nɯ ŋu ma nɯ ma ɯ-kɯ-ndza me\cmn 
\stylefv{qarɣɯ}是一种树,长得不怎么高,刺长得又厚又尖锐,刺的顶端还有钩,叶子小而圆,边缘还有小刺,花有的是红色的,有的是白色的。果实只有鸟吃,其他动物都不吃。
\end{exemple}\end{entrée}

\begin{entrée}
\vedette{\hypertarget{Ⓔqarma}{\papi{ qarma}}}\markboth{qarma}{}
\classe{n}
\begin{définition}\fra crossoptilon\end{définition}
\begin{définition}\cmn 马鸡\end{définition}
\begin{exemple}\jya qarma nɯ ʁnɯ-tɯphu tu, kɯ-ɲaʁ ci tu, qarma ɲaʁ rmi, tɕe kɯ-wɣrum ci tu, qarmaɣrum rmi. qarma nɯ ɯ-ku ɲaʁ, ɯ-mɲaʁ ɯ-rkɯ nɯ ɣɯrni, ɯ-rna nɯ ɯ-rme ɯ-ŋgɯ ku-raʁ ʑo ŋu, kɯ-ɤrqhi ju-kɯ-ru mɤ-saχsɤl, ɯ-jme rɲɟi, qarmaɲaʁ nɯ ɯ-ʁar ɯ-ku ɲaʁ, ɯ-jme ɯ-kɯ-ɲaʁ dɤn, ɯ-mi qarŋe. qarmaɣrum nɯ kɯmaʁ ra naχtɕɯɣ, ɯ-ʁar lonba wɣrum, ɯ-jme ɯ-kɯ-ɲaʁ rkɯn. ɯ-mi li kɯ-qarne ŋu. sɯŋgɯ wuma ʑo ku-rɤʑi-ndʑi ɕti. nɯ-ɕpaʁ tɕe, qarma kɯβdɤsqi kɯmŋɤsqi jamar tɯtɯrca co zɯ kɯ-nɯci pjɯ-ɣi-nɯ ŋgrɤl. qarma nɯ pɣa kɯ-wxti ci ɕti tɕe, tɯ-rdoʁ kɯ sqamŋu tɯ-rpa jamar kɯ-zɣɯt tu, ɯ-ŋgɯm kɯnɤ kɯmpɣa ŋgɯm sɤz wxti, qajɯ cho sɯmat ra tu-ndze ɲɯ-ŋu, tɤ-rɤku ra li ɣɯ-tu-ndze ngrɤl. tu-mbri tɕe, ɯ-mtɕhi tu-mbri ɯ-rca ɯ-mphɯz kɯnɤ tu-mbri ŋu, xɕiri kɯ tu-ti tɕe, `a-pi qarma ɯ-ɕa mɯm ri, tɯ-sɤtɕha ɲɯ-kɯ-sɯ-spɤr ŋu' tu-ti ɲɯ-ŋgrɤl ma qarma lɤ-zo tɕe xɕiri kɯ ɯ-mke ku-ndɤm tɕe mɯ-ɲɯ-te tɕe, qarma ɲɯ-nɯqambɯmbjom tɕe phɤri ʑo ku-nɯɕe tɕe xɕiri tɤrcɯrca ku-tsɯm ɲɯ-ŋgrɤl, xɕiri kɯ nɯ kɯnɤ mɯ-ɲɯ-te tɕe núndʐa nɯ tu-ti ɲɯ-ŋu.\cmn 马鸡有两种,一种是黑色的,叫黑马鸡(蓝马鸡),另一种是白色的,叫白马鸡。马鸡的头是黑色的,眼睛周围有一圈是红色的,耳朵陷在羽毛里,从远处看不清楚,尾巴长。黑马鸡的翅膀顶端是黑色的,尾巴非常黑,脚是黄色的。白马鸡其它部位和黑马鸡一样,只是翅膀全白,尾巴黑羽毛少一些,脚也是黄色的。两种都生活在森林深处。口渴的时候,就四五十只一起下到山沟边来喝水。马鸡是一种比较大的鸟,一只都有十五斤左右,蛋也比鸡蛋大。它喜欢吃虫子和野果,也会来偷吃粮食。叫起来的时候,嘴和尾巴都发出声音。据说黄鼠狼说:“虽然马鸡哥哥的肉好吃,但是它会使你离开自己的家乡”。因为马鸡着地时,黄鼠狼把它的脖子抓住不放,马鸡会飞到对面的山上,也会把黄鼠狼带到那边去,黄鼠狼也还是不放,所以就有这样的说法。\end{exemple}\end{entrée}

\begin{entrée}
\vedette{\hypertarget{Ⓔqarmaɣrum}{\papi{ qarmaɣrum}}}\markboth{qarmaɣrum}{}\classe{n}
\begin{définition}\fra faisan (crossoptilon crossoptilon)\end{définition}
\begin{définition}\cmn 白马鸡\end{définition}
\end{entrée}

\begin{entrée}
\vedette{\hypertarget{Ⓔqarmami}{\papi{ qarmami}}}\markboth{qarmami}{}\classe{n}
\begin{définition}\fra plumes rouges de crossoptilon\end{définition}
\begin{définition}\cmn 马鸡的红羽毛\end{définition}\end{entrée}

\begin{entrée}
\vedette{\hypertarget{Ⓔqarmamtshalu}{\papi{ qarmamtshalu}}}\markboth{qarmamtshalu}{}
\classe{n}
\begin{définition}\fra une plante\end{définition}
\begin{définition}\cmn 植物的一种\end{définition}
\begin{exemple}\jya qarma mtshalu nɯ sɯjno ci ŋu, mtshalu cho naχtɕɯɣ, tɕeri ɯ-mdoʁ nɯ afsoʁŋgi, ɯ-rme kɯ-xtɕɯ-xtɕi tu ri mɤ-sɤmtsɯɣ, kɤ-ndza sna. qarma mtshalu kɤntɕhɯ-tɯphu tu tɕe, tɯtɯphu nɯ ɯ-mɯntoʁ kɯ-wɣrum ɲɯ-lɤt ŋu, ɯ-mat tha-βzu tɕe, kɯ-rɲɟi tsa chɯ-βze tɕe, thɯ-fka tsaʁ tɕe, kú-wɣ-ndo tɕe ɲɤ-mboʁ ŋu. ɲɤ-mboʁ tɕe ɯ-rdoʁ pjɯ-nɯɬoʁ, ɯ-fkɯm rʁoʁ ʑo tu-ti tɕe lo-rʁɯrʁu ŋu. tɯrme nɯɕɯmɯma ʑo tu-kɯ-sɤmbrɯ nɤ qarma mtshalu tu-sɤrmi-nɯ ŋgrɤl\cmn 
\stylefv{qarma mtshalu}是一种植物,和荨麻一样,但颜色浅一些,虽然有一些毛,但不蜇人,有几种\stylefv{qarma mtshalu}可以吃。其中一个开白花,结果实的时候,果实有点长,结得饱满时,用手一摸就会突然破掉,滑漏出种子,种子的壳就会卷起来。突然生气的人,人们说他是\stylefv{qarma mtshalu}。
\end{exemple}\end{entrée}

\begin{entrée}
\vedette{\hypertarget{Ⓔqarmaɲaʁ}{\papi{ qarmaɲaʁ}}}\markboth{qarmaɲaʁ}{}\classe{n}
\begin{définition}\fra faisan (crossoptilon auritum)\end{définition}
\begin{définition}\cmn 蓝马鸡\end{définition}
\end{entrée}

\begin{entrée}
\vedette{\hypertarget{Ⓔqarndɯm}{\papi{ qarndɯm}}}\markboth{qarndɯm}{}
\classe{vi}
\paradigme{\textit{dir :} \jya nɯ-}
\begin{définition}\fra trouble (eau)\end{définition}
\begin{définition}\cmn 混浊\end{définition}
\begin{exemple}\jya tɯ-ci ɲɯ-qarndɯm\cmn 水是浑浊的\end{exemple}\begin{sous-entrée}
\vedette{\hypertarget{}{\papi{ sqarndɯm}}}\markboth{sqarndɯm}{}\classe{vt}
\begin{définition}\fra troubler (eau)\end{définition}
\begin{définition}\cmn 弄浊\end{définition}
\begin{relation-sémantique}\antonyme{
\hyperlink{Ⓔamgri}{\textit{ \papi{amgri}}}
}\end{relation-sémantique}
\end{sous-entrée}\end{entrée}

\begin{entrée}
\vedette{\hypertarget{Ⓔqarŋe}{\papi{ qarŋe}}}\markboth{qarŋe}{}
\classe{vi}
\paradigme{\textit{dir :} \jya thɯ-}
\paradigme{\textit{dir :} \jya nɯ-}
\begin{définition}\fra jaune\end{définition}
\begin{définition}\cmn 黄\end{définition}
\begin{exemple}\jya βlama ɯ-ŋga nɯ kɯ-qarŋe ɲɯ-ŋu\cmn 喇嘛的衣服是黄色的\end{exemple}
\begin{exemple}\jya koxtɕɯn kɯ-qarŋe\cmn 黄色的丝缎\end{exemple}\end{entrée}

\begin{entrée}
\vedette{\hypertarget{Ⓔqarŋɯrŋe}{\papi{ qarŋɯrŋe}}}\markboth{qarŋɯrŋe}{}
\classe{vs}
\begin{définition}\fra jaune foncé\end{définition}
\begin{définition}\cmn 深黄色\end{définition}
\begin{relation-sémantique}\confer{
\hyperlink{Ⓔqarŋe}{\textit{ \papi{qarŋe}}}
}\end{relation-sémantique}
\begin{relation-sémantique}\antonyme{
\hyperlink{Ⓔaqarŋɯrŋe}{\textit{ \papi{aqarŋɯrŋe}}}
}\end{relation-sémantique}\end{entrée}

\begin{entrée}
\vedette{\hypertarget{Ⓔqartsɤβ}{\papi{ qartsɤβ}}}\markboth{qartsɤβ}{}
\classe{n}
\begin{définition}\fra récolte\end{définition}
\begin{définition}\cmn 秋收\end{définition}\end{entrée}

\begin{entrée}
\vedette{\hypertarget{Ⓔqartshaʁrɯ}{\papi{ qartshaʁrɯ}}}\markboth{qartshaʁrɯ}{}
\classe{n}
\begin{définition}\fra ramure de cerf\end{définition}
\begin{définition}\cmn 鹿角\end{définition}\end{entrée}

\begin{entrée}
\vedette{\hypertarget{Ⓔqartshaz}{\papi{ qartshaz}}}\markboth{qartshaz}{}
\classe{n}
\begin{définition}\fra cerf\end{définition}
\begin{définition}\cmn 鹿\end{définition}
\begin{exemple}\jya qartshaz nɯ ruŋgu ku-rɤʑi ŋu, sŋo ɯ-ŋgɯ zɯ ku-rɤʑi, mbro kɯ-fse kɯ-wxti ŋu, ɯ-rme nɯ kɯ-ɤɣɯrnɯɕɯr kɯ-qarŋe kɯ-fse ŋu, kɯ-wxti ɕasca nɯ ɯ-ʁrɯ sqarcɤt kɯ-tu tu ɲɯ-ŋgrɤl, mɤʑɯ ɯ-ʁrɯ tɯ-ldʑa ma kɯ-me tu ɲɯ-ŋgrɤl, nɤrwɯ ŋu tu-kɯ-ti ɲɯ-ŋgrɤl, nɯnɯ qartshaz ɕawu rambɯm ŋu tu-kɯ-ti ɲɯ-ŋgrɤl, ma kɤ-mto rkɯn. kɯ-xtɕi kɯ tʂɯɣlaʁ nɯ dɤn, zgo zɯ ʁɟa ku-rɤʑi, kɯ-rɯndzɤtshi cho tɯ-rmɯkha rndzɤkɤŋe kɤ-ɣe tɕe ku-ɬoʁ ɲɯ-ŋgrɤl, wuma ʑo ku-rɯndzaŋspa, sɯjno tɯ-tɯkɯr ɲɯ-nɯ-phɯt tɕe, ci ɲɯ-nɤrɯra ku-ŋgrɤl, ɯ-qataʁrɯ nɯ nɯŋa ɯ-qataʁrɯ fse. ɯ-mɤlɤjaʁ nɯ mbro ɯ-mɤlɤjaʁ fse, ɯ-jme kɯ-xtɕɯ-xtɕi ma me, ɯ-rna nɯ nɯŋa ɯ-rna fse, ɯ-mtɕhi nɯ nɯŋa ɯ-mtɕhi tsa fse. ɕɤŋɯ raŋ tɕe qartshaz phu nɯ zgo kɯ-mbro zɯ tu-ɕe tɕe nɯ tɕu chɯ-ɣɤwu tɕe, kɯ-kɯ-ɤmɯmtshɤm ɣɯ qartshaz mu nɯ ju-ɣi-nɯ ra ɲɯ-ŋgrɤl, tɕe nɯ thɯ-awɤwum-nɯ tɕe qartshaz ɕaphu nɯ kɯ pjɯ-fskɤr tɕe nɯ tɕu ku-je tɕe tɯ-ci tɯ-mɯm mɯ-pjɤ-sɯ-tshi ɲɯ-ŋgrɤl, nɯ tɕu tɕe ɯ-sŋi kɤrtsi ʑo ku-z-rɤʑi, tɕe qartshaz mu nɯ ra ɲɯ-ɕpaʁ-nɯ, tɕe tɯ-ci tshi ɯ-raŋ nɯ ku-naχthɤβ tɕe tu-nɯphu ŋu tu-kɯ-ti ɲɯ-ŋgrɤl. ɯ-ʁrɯ nɯ smɤn ŋu, tu-ɕaŋ ɕɯŋgɯ tɕe, ɯ-rme sɯβ-sɯβ tu raŋ tɕe mpɯ, tɤ-ɕaŋ tɕe kɯ-rkɯ-rko ŋu, staχpɯ kɤ-phɯt ɯ-raŋ tɕe ʁrɯ tu-ɕaŋ ŋu, tɕe lu-nɯ-βde ŋu tu-kɯ-ti ɲɯ-ŋgrɤl, ɯ-ʁrɯ na-nɯ-βde tɕe, tɯ-rdoʁ ɲɯ-nɯ-βde ma tɯ-tɕha tɯtɯrca ɲɯ-nɯ-βde mɯ́j-ŋgrɤl. ɯ-sni ɯ-se nɯ smɤn stu kɯ-ʑru ŋu tu-kɯ-ti ɲɯ-ŋgrɤl.\cmn 
鹿生活在高山上,在高山羊角花的树林里,有马那么大,毛是红里带黄的。大公鹿有的长十八支叉形角,还有的只有一只角的。据说(十八支角)是宝贝,叫作\stylefv{ɕawu rambɯm},很罕见。一般小鹿有十二支角,这种比较多。鹿一直生活在高山上,到了黄昏,太阳落山(阴影出现)的时候,它就出来觅食和进行其它活动。它非常谨慎,每吃一口草就要左右张望一下。蹄子像牛蹄子一样,四肢像马的一样,只有很小的尾巴,耳朵和嘴巴像牛的一样。发情的时候,公鹿到山的最高处,在那边叫喊,所能听到叫声的母鹿就会往它的方向来。在它们聚集以后,公鹿就会在它们周围绕一圈,不让它们离开,而且一口水都不让它们喝,让它们在那里待数天。母鹿很渴,喝水时公鹿就乘机交配。鹿角是药材,老了之前,毛茸茸的时候,是软的,老了就变硬。据说在割豌豆时鹿角就会老掉,鹿角掉落时,每次只掉一只,不会同时掉一对。据说鹿心脏的血是一种很有疗效的药材。
\end{exemple}\end{entrée}

\begin{entrée}
\vedette{\hypertarget{Ⓔqartshɤɕku}{\papi{ qartshɤɕku}}}\markboth{qartshɤɕku}{}
\classe{n}
\begin{définition}\fra oignon\end{définition}
\begin{définition}\cmn 葱的一种\end{définition}
\begin{exemple}\jya qartshaɕku nɯ ruŋgu kɯ-mbro ɣɯ tɕhɯtoʁ ku tsa tu-ɬoʁ ɲɯ-ŋu, ɯ-tshɯɣa nɯ ɕkɤtshoŋ cho ɲɯ-naχtɕɯɣ, ɯ-ru maŋe, ɯ-jwaʁ nɯ kɯ-ɤlɯlju tɕe qhoʁsjɯβ ɲɯ-ŋu, ɯ-tho ɣɤʑu, ɯ-qa li ɯ-rqhu kɤntɕhɯ-tɤlɤβ ɣɤʑu. ɯ-dɯχɯn wuma ɲɯ-mɯm.\cmn 
\stylefv{qartshɤɕku}生长在高山的水草地里,形状和葱子(即通常说的“野葱”)一样,没有茎,叶子是空心的、圆柱形的。有花梗。根也有很多层皮。味道很香。
\end{exemple}
\begin{relation-sémantique}\confer{
\hyperlink{Ⓔqartshaz}{\textit{ \papi{qartshaz}}}
}\end{relation-sémantique}\end{entrée}

\begin{entrée}
\vedette{\hypertarget{Ⓔqartshɤndʐi}{\papi{ qartshɤndʐi}}}\markboth{qartshɤndʐi}{}
\classe{n}
\begin{définition}\fra peau de cerf\end{définition}
\begin{définition}\cmn 鹿皮\end{définition}
\begin{relation-sémantique}\confer{
\hyperlink{Ⓔqartshaz}{\textit{ \papi{qartshaz}}}
}\end{relation-sémantique}
\begin{relation-sémantique}\confer{
\hyperlink{Ⓔtɯ-ndʐi}{\textit{ \papi{tɯ-ndʐi}}}
}\end{relation-sémantique}\end{entrée}

\begin{entrée}
\vedette{\hypertarget{Ⓔqartshi}{\papi{ qartshi}}}\markboth{qartshi}{}
\classe{n}
\begin{définition}\fra cigale; criquet\end{définition}
\begin{définition}\cmn 蝉;蟋蟀\end{définition}
\begin{exemple}\jya qartshi nɯ sɯjno ŋgɯ, tɤ-rɤku ɯ-ŋgɯ ku-rɤʑi ŋu, tɤ-rɤku tu-ndze ŋgrɤl, qartshi nɯ ɯ-ʁrɯ tu, ɯ-mɲaʁ kɯ-wxtɯ-wxti ŋu, χɕɤlmɯɣ tɤ-kɯ-nɯ-ta ʑo fse, ɯ-ku xtɕi tsa ɯ-phoŋbu wxti, ɯ-mi kɯ-wxti ʑo tu, ɯ-mɤpɤl ɲɯ-zɲɟa, ɯ-ʁar mba, ɯ-rʑɯɣ tu, tɤ-ŋke tɕe kɯ-ɤrqhi ʑo ju-mtsaʁ cha, ɯ-ʁar kɯ-dɤn kɤ-ntɕhoz mɤ-ra.\cmn 蟋蟀生活在草丛和庄稼里,吃庄稼。蟋蟀有触角,眼睛很大,好像戴了眼镜一样。头小,身子大。有很大的脚,爪子抓东西很稳,翅膀很薄,有褶皱。走动时,可以跳到较远的距离,翅膀起不了很大的作用。\end{exemple}\end{entrée}

\begin{entrée}
\vedette{\hypertarget{Ⓔqartsɯ}{\papi{ qartsɯ}}}\markboth{qartsɯ}{}
\classe{n}
\begin{définition}\fra hiver\end{définition}
\begin{définition}\cmn 冬天\end{définition}
\begin{exemple}\jya qartsɯ kɤ-ndzoʁ\cmn 冬天开始了\end{exemple}\end{entrée}

\begin{entrée}
\vedette{\hypertarget{Ⓔqartsɯmɤftɕar}{\papi{ qartsɯmɤftɕar}}}\markboth{qartsɯmɤftɕar}{}
\classe{adv}
\begin{définition}\fra toute l'année\end{définition}
\begin{définition}\cmn 一年四季\end{définition}
\begin{relation-sémantique}\confer{
\hyperlink{Ⓔqartsɯ}{\textit{ \papi{qartsɯ}}}
}\end{relation-sémantique}
\begin{relation-sémantique}\confer{
\hyperlink{Ⓔftɕar}{\textit{ \papi{ftɕar}}}
}\end{relation-sémantique}\end{entrée}

\begin{entrée}
\vedette{\hypertarget{Ⓔqartsɯɲaʁ}{\papi{ qartsɯɲaʁ}}}\markboth{qartsɯɲaʁ}{}
\classe{n}
\begin{définition}\fra hiver très froid\end{définition}
\begin{définition}\cmn 寒冬腊月\end{définition}\end{entrée}

\begin{entrée}
\vedette{\hypertarget{Ⓔqartsɯχtu}{\papi{ qartsɯχtu}}}\markboth{qartsɯχtu}{}\classe{n}
\begin{définition}\fra cérémonie bouddhique de l'hiver\end{définition}
\begin{définition}\cmn 在冬天举行的一种佛教活动\end{définition}\end{entrée}

\begin{entrée}
\vedette{\hypertarget{Ⓔqarwoʁ}{\papi{ qarwoʁ}}}\markboth{qarwoʁ}{}
\classe{n}
\begin{définition}\fra récipient en terre\end{définition}
\begin{définition}\cmn 沙锅\end{définition}
\begin{exemple}\jya qarwoʁ nɯ tɤ-rcoʁ kɯ tɤ-kɤ-sɯ-βzu ŋu, kɯɕɯŋgɯ tɕe rgargɯn ra ɣɯ nɯ-tʂha ɯ-sɤ-ta pjɤ-ŋu, ɯ-mŋu ri ɯ-mtɕhi tu tɕe tʂha pɯ́-wɣ-rku tɕe mɤ-jit. qarwoʁ ɯ-ŋgɯ tʂha ŋotɕu tɤ-ala tɕe thɯ́-wɣ-nɯ-mɟa kɯnɤ pjɯ-mbuz pjɤ-ŋu, ʑaʑa ʑo ɯ-tɯ-ɤla mɯ-pjɯ-ʑi tɕe tɯrme ɯ-ro kɯ-ŋgɤr ɯ-mbrɯ ʑa mɤ-kɯ-ʑi nɯ ra qarwoʁ tu-sɤrmi-nɯ tu-ŋgrɤl.\cmn 
\stylefv{qarwoʁ}是用泥捏成的,过去是专门给老年人熬茶的,开口呈嘴形,倒茶时不会洒。\stylefv{qarwoʁ}里的茶一旦烧开了,即使把它从火边移开仍然要溢出来,开个不停,所以人们把脾气暴躁的人说成是\stylefv{qarwoʁ}。
\end{exemple}\end{entrée}

\begin{entrée}
\vedette{\hypertarget{ⒺqaʁⒽ2}{\papi{ qaʁ}}}\markboth{qaʁ}{}\homonyme{2}\classe{n}
\begin{définition}\fra bêche\end{définition}
\begin{définition}\cmn 锄头\end{définition}\end{entrée}

\begin{entrée}
\vedette{\hypertarget{ⒺqaʁⒽ1}{\papi{ qaʁ}}}\markboth{qaʁ}{}\homonyme{1}
\classe{vt}\acception{1}
\paradigme{\textit{dir :} \jya nɯ-}
\paradigme{\textit{dir :} \jya pɯ-}
\paradigme{\textit{dir :} \jya \_}
\begin{définition}\fra enlever (peau, écorce), écorcher\end{définition}
\begin{définition}\cmn 剥皮\end{définition}
\begin{exemple}\jya nɤ-ŋga nɯ-qaʁ\cmn 你脱衣服吧\end{exemple}
\begin{exemple}\jya sɯrqhu pɯ-qaʁ\cmn 你剥树皮吧\end{exemple}
\begin{exemple}\jya ɯ-ndʐi nɯ-qaʁ\cmn 你剥它的皮子吧\end{exemple}
\begin{exemple}\jya skɤmndʐi pɯ-qaʁ-a\cmn 我剥了牛皮\end{exemple}
\begin{exemple}\jya ɲchɣaʁ tɤ-qaʁ-a\cmn 我剥了白桦树皮(往上剥)\end{exemple}\acception{2}
\paradigme{\textit{dir :} \jya nɯ-}
\begin{définition}\fra rejeter\end{définition}
\begin{définition}\cmn 排挤;当外人\end{définition}
\begin{exemple}\jya aʑo ɲɯ-kɯ-qaʁ-a ɲɯ-ŋu\cmn 你把我当外人\end{exemple}\begin{sous-entrée}
\vedette{\hypertarget{}{\papi{ nɯɣɯqaʁ}}}\markboth{nɯɣɯqaʁ}{}\classe{vs}
\begin{définition}\fra facile à enlever\end{définition}
\begin{définition}\cmn 容易剥
\begin{déclaration}\grammar{facil}\end{déclaration}\end{définition}
\begin{relation-sémantique}\confer{
\hyperlink{Ⓔɴɢaʁ}{\textit{ \papi{ɴɢaʁ}}}
}\end{relation-sémantique}
\end{sous-entrée}\end{entrée}

\begin{entrée}
\vedette{\hypertarget{Ⓔqase}{\papi{ qase}}}\markboth{qase}{}
\classe{n}
\begin{définition}\fra corde en peau\end{définition}
\begin{définition}\cmn 皮条\end{définition}\end{entrée}

\begin{entrée}
\vedette{\hypertarget{Ⓔqatɯkɯr}{\papi{ qatɯkɯr}}}\markboth{qatɯkɯr}{}
\classe{n}
\begin{définition}\fra mauvais conseils\end{définition}
\begin{définition}\cmn 反面教育\end{définition}
\begin{relation-sémantique}\confer{
\hyperlink{Ⓔznɯqatɯkɯr}{\textit{ \papi{znɯqatɯkɯr}}}
}\end{relation-sémantique}\end{entrée}

\begin{entrée}
\vedette{\hypertarget{Ⓔqawɤr}{\papi{ qawɤr}}}\markboth{qawɤr}{} (\variante{qawɯwɤr}) 
\classe{vi}
\paradigme{\textit{dir :} \jya tɤ-}
\begin{définition}\fra s'ouvrir (champignon)\end{définition}
\begin{définition}\cmn 展开(菌子)【开反】\end{définition}
\begin{exemple}\jya tɤjmɤɣ to-qawɤr\cmn 蘑菇展开了\end{exemple}\end{entrée}

\begin{entrée}
\vedette{\hypertarget{Ⓔqawoʁ}{\papi{ qawoʁ}}}\markboth{qawoʁ}{}
\classe{n}
\begin{définition}\fra grotte\end{définition}
\begin{définition}\cmn 洞\end{définition}\end{entrée}

\begin{entrée}
\vedette{\hypertarget{Ⓔqawɯz}{\papi{ qawɯz}}}\markboth{qawɯz}{}
\classe{n}
\begin{définition}\fra edelweiss\end{définition}
\begin{définition}\cmn 火绒草\end{définition}
\begin{exemple}\jya qawɯz nɯ sɯjno kɯ-mbɯ-mbɤr ci ŋu, sɯjno ɯ-jwaʁ nɯ ra ɯ-rme kɯ-tu, kɯ-jaʁ kɯ-fse ci ŋu. ɯ-mɯntoʁ ɯ-jwaʁ ɯ-ru lonba ʑo kɯ-wɣrum ŋu, tɕe ɲɯ́-wɣ-sɯɣ-rom tɕe, tú-wɣ-rŋu, tɕe tɯ-ɕke ta-ʑa tɕe, pjɯ́-wɣ-nɤtar tɕe pɯ-ndɯβ tɕe srɯn kɯ-fse ɲɯ-ɤndɯndo ŋu tɕe, ɯnɯ qawɯz ŋu tɕe, tɕaʁmɤr tɤ́-wɣ-lɤt tɕe, ɯ-smi sɤ-mɟa ŋu. tɕaʁmɤr kɤ-lɤt tɕe, qawɯz tɯ-snaʁ cho qapi ni tɯ-tɯ-ndo kú-wɣ-βzu tɕe tɯ-jaʁ ntsi kɯ tɕaʁmɤr pjɯ́-wɣ-sɯ-lɤt tɕe, qapi cho qawɯz ni ndʑi-ɕnɤz nɯtɕu tɕaʁmɤr pjɯ́-wɣ-lɤt, tɕe qapi ɯ-taʁ smɯtɕɣom tu-ɬoʁ tɕe, qawɯz ɯ-taʁ smi nɯ tu-nɯt ŋu. kɯɕɯŋgɯ tɕe smi kɤ-sɯ-βzu nɯ tu-stu-nɯ pjɤ-ŋu, tham tɕe nɯ kɯ-ntɕhoz me.\cmn 火绒草是一种矮小的植物,叶子长有毛,有点厚。花、叶子和茎都是白色的。晒干了以后,干炒到开始焦黄时就用细条抽打,打细了以后就像棉花一样连在一起,这也是火绒草。打火镰时,是用来引火的东西。打火镰时,把燧石和一小块火绒草一起拿在手上另一只手拿着火镰,打在燧石和火绒草的一角,燧石上迸出火星,火绒草就会着火。过去取火就是这样的,现在没有人用这种方法了。\end{exemple}\end{entrée}

\begin{entrée}
\vedette{\hypertarget{Ⓔqaʑmbri}{\papi{ qaʑmbri}}}\markboth{qaʑmbri}{}
\classe{n}
\begin{définition}\fra plante grimpante\end{définition}
\begin{définition}\cmn 藤子\end{définition}
\begin{exemple}\jya qaʑmbri nɯ tɯ-mbri kɯ-fse kɯ-rɲɟɯ-rɲɟi tɕe ɯ-rtsɤɣ raŋri ɯ-mɯntoʁ cho ɯ-jwaʁ ku-ndzoʁ ŋu. ɯ-jwaʁ ɯ-mdoʁ ldʑaŋnaʁ ŋu. ɯ-mɯntoʁ tshanlaŋ kɯ-fse kɯ-qarŋe ŋu\cmn 藤子长得像绳子一样,很长,分节长叶开花。叶子呈深绿色,花像铃铛一样,呈黄色\end{exemple}\end{entrée}

\begin{entrée}
\vedette{\hypertarget{Ⓔqaʑo}{\papi{ qaʑo}}}\markboth{qaʑo}{}
\classe{n}
\begin{définition}\fra mouton\end{définition}
\begin{définition}\cmn 绵羊
\begin{déclaration} \étymologie{\papi{gjaŋ}}\end{déclaration}\end{définition}\end{entrée}

\begin{entrée}
\vedette{\hypertarget{Ⓔqaʑolu}{\papi{ qaʑolu}}}\markboth{qaʑolu}{}\classe{n}
\begin{définition}\fra année du mouton\end{définition}
\begin{définition}\cmn 羊年\end{définition}
\end{entrée}

\begin{entrée}
\vedette{\hypertarget{Ⓔqɤr}{\papi{ qɤr}}}\markboth{qɤr}{}\classe{vt}
\paradigme{\textit{dir :} \jya thɯ-}
\paradigme{\textit{dir :} \jya nɯ-}
\paradigme{\textit{dir :} \jya tɤ-}
\begin{définition}\fra choisir\end{définition}
\begin{définition}\cmn 挑选\end{définition}
\begin{exemple}\jya stoʁrɣi thɯ-qar-a\cmn 我选了胡豆种子\end{exemple}
\begin{exemple}\jya jaŋjy nɯ-qar-a\cmn 我选了土豆\end{exemple}
\begin{exemple}\jya pɯ-kɯ-tsɣi cho mɤ-kɯ-tsɣi nɯ-qar-a\cmn 我选出了没有烂的\end{exemple}
\begin{relation-sémantique}\confer{
\hyperlink{Ⓔsɤqɤrle}{\textit{ \papi{sɤqɤrle}}}
}\end{relation-sémantique}
\begin{relation-sémantique}\confer{
\hyperlink{Ⓔnɯndzɤqɤr}{\textit{ \papi{nɯndzɤqɤr}}}
}\end{relation-sémantique}
\begin{relation-sémantique}\confer{
\hyperlink{Ⓔʑɣɤqɤr}{\textit{ \papi{ʑɣɤqɤr}}}
}\end{relation-sémantique}\begin{sous-entrée}
\vedette{\hypertarget{}{\papi{ nɤqɤrqɤr}}}\markboth{nɤqɤrqɤr}{}\classe{vt}
\begin{définition}\fra être indécis et ne pas savoir quoi choisir\end{définition}
\begin{définition}\cmn 反复挑选\end{définition}
\begin{exemple}\jya kɯ-tu nɯ tɤ-ndze ma kɤ-nɤqɤrqɤr ntsɯ me\cmn 有的那个你就吃吧,不要东选西选了\end{exemple}
\end{sous-entrée}\begin{sous-entrée}
\vedette{\hypertarget{}{\papi{ nɯqɤr}}}\markboth{nɯqɤr}{}
\paradigme{\textit{dir :} \jya tɤ-}
\begin{définition}\ 
\begin{déclaration}\grammar{autoben}\end{déclaration}\end{définition}
\begin{exemple}\jya tɤ-nɯ-qɤr\cmn 你自己选吧\end{exemple}
\begin{exemple}\jya tɤ-nɯqar-a\cmn 我选了\end{exemple}
\begin{exemple}\jya nɤ-@gongzuo ma tɤ-nɯqɤr\cmn 你自己选你的工作\end{exemple}
\begin{exemple}\jya a-sɯm kɯ-ɕe tu-nɯqar-a ɕti\cmn 我选择我愿意做的工作\end{exemple}
\end{sous-entrée}\end{entrée}

\begin{entrée}
\vedette{\hypertarget{Ⓔqɤt}{\papi{ qɤt}}}\markboth{qɤt}{}
\classe{vt}
\paradigme{\textit{dir :} \jya nɯ-}
\paradigme{\textit{dir :} \jya pɯ-}
\begin{définition}\fra séparer\end{définition}
\begin{définition}\cmn 分开\end{définition}
\begin{exemple}\jya mbro jla nɯ-qɤt\cmn 你把马和犏牛分开\end{exemple}
\begin{exemple}\jya a-mi nɯ-qat-a\cmn 我把脚分开了\end{exemple}
\begin{exemple}\jya a-kɤrme pɯ-qat-a\cmn 我分了头发\end{exemple}
\begin{exemple}\jya a-ŋga nɯ-qat-a\cmn 我把衣服解开了(由两边分)\end{exemple}
\begin{relation-sémantique}\synonyme{
\hyperlink{Ⓔsɤqɤrle}{\textit{ \papi{sɤqɤrle}}}
}\end{relation-sémantique}
\begin{relation-sémantique}\confer{
\hyperlink{Ⓔɴɢɤt}{\textit{ \papi{ɴɢɤt}}}
}\end{relation-sémantique}
\begin{relation-sémantique}\confer{
 \papi{tɤqɤt}
}\end{relation-sémantique}\end{entrée}

\begin{entrée}
\vedette{\hypertarget{Ⓔqha}{\papi{ qha}}}\markboth{qha}{}
\classe{vt}
\paradigme{\textit{dir :} \jya tɤ-}
\paradigme{\textit{dir :} \jya nɯ-}
\begin{définition}\fra s'énerver, détester\end{définition}
\begin{définition}\cmn 生气;讨厌\end{définition}
\begin{exemple}\jya hajtsu qhe-a\cmn 我讨厌黑椒\end{exemple}
\begin{exemple}\jya jiɕqha nɯ kɯ nɯ ɲɯ-ti ɲɯ-qhe-a\cmn 那个人说了这个,我很讨厌\end{exemple}
\begin{exemple}\jya ɲɯ-ta-qha\cmn 我讨厌你\end{exemple}
\begin{exemple}\jya ma-tɤ-tɯ-qhe\cmn 你不要讨厌他\end{exemple}
\begin{exemple}\jya tɤrkoz pɯ-maʁ, ma-tɤ-tɯ-qhe\cmn 他不是故意的,你不要讨厌他\end{exemple}
\begin{exemple}\jya smi kɯ tɯ-ci ɲɯ-qhe\cmn 火讨厌水\end{exemple}
\begin{exemple}\jya khɯna kɯ lɯlu ɲɯ-qhe\cmn 狗讨厌猫\end{exemple}
\begin{exemple}\jya tɯ-ci kɤ-lɤt ɲɯ-qhe\cmn 他讨厌灌水\end{exemple}
\begin{exemple}\jya tɯ-ci ɕɯ-kɤ-ru ɲɯ-qhe\cmn 他讨厌去背水\end{exemple}
\begin{relation-sémantique}\antonyme{
\hyperlink{Ⓔnɯrga}{\textit{ \papi{nɯrga}}}
}\end{relation-sémantique}\end{entrée}

\begin{entrée}
\vedette{\hypertarget{Ⓔqhajŋgɯ}{\papi{ qhajŋgɯ}}}\markboth{qhajŋgɯ}{}
\classe{n}
\begin{définition}\fra voie d'eau du moulin\end{définition}
\begin{définition}\cmn 磨坊引水槽\end{définition}
\begin{exemple}\jya qhajŋgɯ nɯ ɕoŋtɕa kɯ-wxti ʑo chɯ́-wɣ-phɯt tɕe chɯ́-wɣ-nɯrtaʁ chɯ́-wɣ-nɤrqhu tɕe ɯ-rqhioʁ chɯ́-wɣ-tɕɤt tɕe βɣa ɯ-lɤcu chɯ́-wɣ-tshoʁ tɕe ɯ-ŋgɯ tɯ-ci chɯ́-wɣ-sɯ-ɣe tɕe tɕhɯŋkhɤr ɯ-taʁ chɯ́-wɣ-sɯ-lɤt ɯ-spa ŋu\cmn 
\stylefv{qhajŋgɯ}(是一种引水槽)。把一棵大树砍下,砍掉枝桠,刨去树皮然后挖一道槽,然后装在磨坊上,引水通过槽冲转水车。
\end{exemple}\end{entrée}

\begin{entrée}
\vedette{\hypertarget{Ⓔqhaqhu}{\papi{ qhaqhu}}}\markboth{qhaqhu}{}
\classe{n}
\begin{définition}\fra derrière la maison\end{définition}
\begin{définition}\cmn 房子的后面\end{définition}
\begin{relation-sémantique}\confer{
\hyperlink{Ⓔkha}{\textit{ \papi{kha}}}
}\end{relation-sémantique}\end{entrée}

\begin{entrée}
\vedette{\hypertarget{Ⓔqharu}{\papi{ qharu}}}\markboth{qharu}{}\classe{n}
\begin{définition}\fra regard en arrière\end{définition}
\begin{définition}\cmn 回头\end{définition}
\begin{relation-sémantique}\confer{
\hyperlink{Ⓔnɤqharu}{\textit{ \papi{nɤqharu}}}
}\end{relation-sémantique}
\begin{relation-sémantique}\confer{
\hyperlink{Ⓔɯ-qhu}{\textit{ \papi{ɯ-qhu}}}
}\end{relation-sémantique}
\begin{relation-sémantique}\confer{
\hyperlink{ⒺruⒽ1}{\textit{ \papi{ru1}}}
}\end{relation-sémantique}\end{entrée}

\begin{entrée}
\vedette{\hypertarget{Ⓔqhaχɕu}{\papi{ qhaχɕu}}}\markboth{qhaχɕu}{}\classe{n}
\begin{définition}\fra vantardise\end{définition}
\begin{définition}\cmn 炫耀\end{définition}
\begin{exemple}\jya qhaχɕu ma-tɤ-tɯ-βze\cmn 你不要炫耀\end{exemple}
\begin{relation-sémantique}\confer{
\hyperlink{Ⓔrɯqhaχɕu}{\textit{ \papi{rɯqhaχɕu}}}
}\end{relation-sémantique}
\begin{relation-sémantique}\confer{
\hyperlink{Ⓔnɯqhaχɕu}{\textit{ \papi{nɯqhaχɕu}}}
}\end{relation-sémantique}\end{entrée}

\begin{entrée}
\vedette{\hypertarget{Ⓔqhɤjmbaʁ}{\papi{ qhɤjmbaʁ}}}\markboth{qhɤjmbaʁ}{}
\classe{n}
\begin{définition}\fra Rumex sp.\end{définition}
\begin{définition}\cmn 酸模\end{définition}
\begin{exemple}\jya qhɤjmbaʁ nɯ kha ɯ-rkɯ ra tɯ-ɣli kɯ-dɤn ɣɯ sɤtɕha pɕoʁ tu-ɬoʁ ŋu, ɯ-jwaʁ ɯ-tshɯɣa nɯ ŋɤnɯkɯmtsɯɣ cho naχtɕɯɣ, tɕeri qhɤjmbaʁ ɯ-jwaʁ nɯ mba ɯ-rme me, arŋi, qhɤjmbaʁ ɯ-ru tu-ɬoʁ tɕe, tɯ-rtsɤɣ tɕe ci ntsɯ ɯ-jwaʁ ku-ndzoʁ ŋu. ɯ-ru tɤ-ari ɯ-jija ɯ-jwaʁ tu-xtɕi ŋu. ɯ-ru ɯ-stɤt tɕe, ɯ-mɯntoʁ ɲɯ-lɤt. ɯ-mɯntoʁ aɣɯrnɯɕɯr. ɯʑo ɯ-ru nɯ kɯ-wxti nɯ tɯrme fsu jamar tu, nɯ maʁ nɤ tɤ-mthɤɣ fsu jamar ma me, paʁndza sna ma tɯrme kɤ-ndza mɤ-sna. ɯ-di mɤ-χɕɤβ. qhɤjmbaʁ li ci tɯ-tɯphu tu tɕe, kɯmaʁ ra naχtɕɯɣ tɕeri ɯ-jwaʁ rɲɟi cho jaʁ, mpɯ, tɯrme kɤ-ndza sna. ɯ-jwaʁ ɯ-sɤɣ-ndzoʁ nɯ tɕu ɯ-ru cho ɯ-pɤrthɤβ nɯ tɕu, tɯ-ci kɯ-fse kɯ-ɤrɤmtʂɯmtʂaj tu, ɯ-ru jpum tsa tɕe mpɯ.\cmn 
酸模生长在房子旁边,肥料比较多的地方,叶子的形状和红青椒的相同,但是酸模的叶子比较薄,没有毛,是绿色的。\stylefv{qhɤjmbaʁ}的茎上每一节都长叶子。随着茎的长高,叶子就变小。茎的顶端开花。花是淡红色的。茎可以长到一个人那么高,但一般的只长到人的腰部那么高。可以喂猪,人不能吃。味道不浓。还有一种\stylefv{qhɤjmbaʁ},其他(部分)和(前一种)一样,就是叶子比较长,比较厚,嫩,人可以吃。在茎上长叶子的部位之间,有一种粘液。茎粗而嫩。
\end{exemple}\end{entrée}

\begin{entrée}
\vedette{\hypertarget{Ⓔqhɤndɤpa}{\papi{ qhɤndɤpa}}}\markboth{qhɤndɤpa}{}
\classe{num}
\begin{définition}\fra dans trois ans\end{définition}
\begin{définition}\cmn 三年以后\end{définition}\end{entrée}

\begin{entrée}
\vedette{\hypertarget{Ⓔqhɤndi}{\papi{ qhɤndi}}}\markboth{qhɤndi}{}
\classe{num}
\begin{définition}\fra dans trois jours\end{définition}
\begin{définition}\cmn 大后天\end{définition}\end{entrée}

\begin{entrée}
\vedette{\hypertarget{Ⓔqhɤtɯɣ}{\papi{ qhɤtɯɣ}}}\markboth{qhɤtɯɣ}{}
\classe{n}
\begin{définition}\fra bâton qui sert à caler la porte\end{définition}
\begin{définition}\cmn 关上门后,在地上斜顶住门的木棒\end{définition}\end{entrée}

\begin{entrée}
\vedette{\hypertarget{Ⓔqhi}{\papi{ qhi}}}\markboth{qhi}{}\classe{vs}
\paradigme{\textit{dir :} \jya tɤ-}
\paradigme{\textit{dir :} \jya thɯ-}\acception{1}
\begin{définition}\fra difficile à dompter (cheval)\end{définition}
\begin{définition}\cmn 难以驯服的马\end{définition}\acception{2}
\begin{définition}\fra insolent\end{définition}
\begin{définition}\cmn 放肆\end{définition}
\end{entrée}

\begin{entrée}
\vedette{\hypertarget{Ⓔqhinɤqhi}{\papi{ qhinɤqhi}}}\markboth{qhinɤqhi}{}\classe{idph.3}
\begin{définition}\fra bruit d'essoufflement\end{définition}
\begin{définition}\cmn 形容气喘吁吁的样子\end{définition}
\begin{exemple}\jya a-fkur ɲɯ-rʑi tɕe, qhjinɤqhji ʑo tɤ-tɯt-a pɯ-ra\cmn 因为我背的东西太重,所以有点气喘吁吁\end{exemple}
\begin{relation-sémantique}\synonyme{
\hyperlink{Ⓔχinɤχi}{\textit{ \papi{χinɤχi}}}
}\end{relation-sémantique}\end{entrée}

\begin{entrée}
\vedette{\hypertarget{Ⓔqhuj}{\papi{ qhuj}}}\markboth{qhuj}{}
\classe{n}
\begin{définition}\fra ce soir\end{définition}
\begin{définition}\cmn 今天晚上(到了晚上就不能用这个词了)\end{définition}\end{entrée}

\begin{entrée}
\vedette{\hypertarget{Ⓔqhjɤβqhjɤβ}{\papi{ qhjɤβqhjɤβ}}}\markboth{qhjɤβqhjɤβ}{}\classe{idph.2}
\begin{définition}\fra de couleur terne\end{définition}
\begin{définition}\cmn 颜色不鲜艳\end{définition}
\begin{exemple}\jya tɤŋe qhjɤβqhjɤβ ci ɣɤʑu\cmn (云多)太阳出不来\end{exemple}
\begin{relation-sémantique}\synonyme{
\hyperlink{Ⓔqhjiqhji}{\textit{ \papi{qhjiqhji}}}
}\end{relation-sémantique}\end{entrée}

\begin{entrée}
\vedette{\hypertarget{Ⓔqhjiqhji}{\papi{ qhjiqhji}}}\markboth{qhjiqhji}{}\classe{idph.2}
\begin{définition}\fra de couleur terne\end{définition}
\begin{définition}\cmn 颜色不鲜艳\end{définition}
\begin{exemple}\jya tɕoχtsi ɯ-mdoʁ nɯ kɯ-qarŋe qhjiqhji ɲɯ-ŋu\cmn 桌子是淡黄色的\end{exemple}
\begin{exemple}\jya nɤ-@diannao ɯ-mdoʁ kɯ-ɤrŋi qhjiqhji ɲɯ-ŋu\cmn 你电脑的颜色是淡蓝色的\end{exemple}
\begin{exemple}\jya a-ʑi ɲɯ-loʁ tɕe, a-rqo qhjiqhji ʑo ɲɯ-pa\cmn 我感到恶心,喉咙里很不舒服\end{exemple}
\begin{relation-sémantique}\synonyme{
\hyperlink{Ⓔqhjɤβqhjɤβ}{\textit{ \papi{qhjɤβqhjɤβ}}}
}\end{relation-sémantique}\end{entrée}

\begin{entrée}
\vedette{\hypertarget{Ⓔqhlaʁ}{\papi{ qhlaʁ}}}\markboth{qhlaʁ}{}
\classe{idph.1}
\begin{définition}\fra disparaître d'un coup\end{définition}
\begin{définition}\cmn 突然消失\end{définition}
\begin{exemple}\jya qhlaʁ ʑo to-ɕqhlɤt\cmn 突然间消失了\end{exemple}\end{entrée}

\begin{entrée}
\vedette{\hypertarget{Ⓔqhlɤβqhlɤβ}{\papi{ qhlɤβqhlɤβ}}}\markboth{qhlɤβqhlɤβ}{}\classe{idph.2}\acception{1}
\begin{définition}\fra moyen, pas extrême (température, lumière etc)\end{définition}
\begin{définition}\cmn 形容温度、光等中等,不极端\end{définition}
\begin{exemple}\jya tɯ-mɯ kɯ-jɯm tɕi ɲɯ-maʁ, kɯ-lɯβ tɕi ɲɯ-maʁ, qhlɤβqhlɤβ ʑo ɲɯ-pa\cmn 天不晴也不阴,灰扑扑的\end{exemple}\acception{2}
\begin{définition}\fra grisâtre\end{définition}
\begin{définition}\cmn 灰扑扑的\end{définition}\end{entrée}

\begin{entrée}
\vedette{\hypertarget{Ⓔqhloŋ}{\papi{ qhloŋ}}}\markboth{qhloŋ}{}\classe{idph.1}
\begin{définition}\fra plouf\end{définition}
\begin{définition}\cmn 噗通(突然掉进水里的声音)\end{définition}
\begin{exemple}\jya qhloŋ ʑo pjɤ-ɕqhlɤt\cmn 噗通一声就掉进水里了\end{exemple}\begin{sous-entrée}
\vedette{\hypertarget{}{\papi{ phɯqhloŋ}}}\markboth{phɯqhloŋ}{}\classe{idph.7}
\end{sous-entrée}\begin{sous-entrée}
\vedette{\hypertarget{}{\papi{ qhloŋnɤloŋ}}}\markboth{qhloŋnɤloŋ}{}\classe{idph.4}
\end{sous-entrée}\begin{sous-entrée}
\vedette{\hypertarget{}{\papi{ sɤqhloŋloŋ}}}\markboth{sɤqhloŋloŋ}{}\classe{vt}
\begin{définition}\fra faire du bruit en se débattant dans l'eau\end{définition}
\begin{définition}\cmn 在水里不停地发出“噗通”的声音(动物在水里挣扎的声音)\end{définition}
\begin{exemple}\jya tɯ-ci ɯ-ŋgɯ ɲɯ-nɯsroʁmbrɤt tɕe ɲɯ-sɤqhloŋloŋ ʑo\cmn 他在水里挣扎,发出噗通噗通的声音\end{exemple}
\end{sous-entrée}\end{entrée}

\begin{entrée}
\vedette{\hypertarget{Ⓔqhloŋqhloŋ}{\papi{ qhloŋqhloŋ}}}\markboth{qhloŋqhloŋ}{}\classe{idph.2}
\begin{définition}\fra trouble (eau)\end{définition}
\begin{définition}\cmn 形容浑浊的样子\end{définition}
\begin{exemple}\jya tɯ-ci qhloŋqhloŋ ʑo ɲɯ-qarndɯm\cmn 水很浑浊的样子\end{exemple}\end{entrée}

\begin{entrée}
\vedette{\hypertarget{Ⓔqhlɯ}{\papi{ qhlɯ}}}\markboth{qhlɯ}{}
\classe{n}
\begin{définition}\fra naga\end{définition}
\begin{définition}\cmn 龙;水神
\begin{déclaration} \étymologie{\papi{klu}}\end{déclaration}\end{définition}
\begin{exemple}\jya nɤʑo qhlɯ to-tɯ-sɤzmbrɯ-t\cmn 你惹了水神\end{exemple}\end{entrée}

\begin{entrée}
\vedette{\hypertarget{Ⓔqhlɯqhlu}{\papi{ qhlɯqhlu}}}\markboth{qhlɯqhlu}{} (\variante{qhluqhlu}) \classe{idph.2}
\begin{définition}\fra un peu laiteux\end{définition}
\begin{définition}\cmn 形容带有一点乳白色的(液体)
\end{définition}
\begin{exemple}\jya tɤ-lu ci qhluqhlu tɤ-lat-a\cmn 我倒了一点点牛奶(茶水里只有一点乳白色)\end{exemple}\end{entrée}

\begin{entrée}
\vedette{\hypertarget{Ⓔqhoʁqhoʁ}{\papi{ qhoʁqhoʁ}}}\markboth{qhoʁqhoʁ}{}
\classe{n}
\begin{définition}\fra lingot\end{définition}
\begin{définition}\cmn 一块银子\end{définition}\end{entrée}

\begin{entrée}
\vedette{\hypertarget{Ⓔqhoʁsjɯβ}{\papi{ qhoʁsjɯβ}}}\markboth{qhoʁsjɯβ}{}
\classe{n}
\begin{définition}\fra creux\end{définition}
\begin{définition}\cmn 空心\end{définition}\end{entrée}

\begin{entrée}
\vedette{\hypertarget{Ⓔqhrɯt}{\papi{ qhrɯt}}}\markboth{qhrɯt}{}\classe{vt}
\paradigme{\textit{dir :} \jya tɤ-}
\paradigme{\textit{dir :} \jya nɯ-}
\paradigme{\textit{dir :} \jya \_}
\begin{définition}\fra gratter complètement\end{définition}
\begin{définition}\cmn 刮干净\end{définition}
\begin{exemple}\jya tɯthɯ ta-qhrɯt\cmn 他刮了锅子\end{exemple}
\begin{exemple}\jya ɲɤ-k-ɤtɕaʁ-ci ɲɤ-qhrɯt\cmn 有东西粘上去了他就刮了\end{exemple}
\begin{exemple}\jya nɯ-qhrɯt-a\cmn 我刮了\end{exemple}
\begin{exemple}\jya nɯ-tɯ-qhrɯt\cmn 你刮了\end{exemple}
\begin{exemple}\jya tɤrcoʁ ra nɯ-qhrɯt\cmn 你把那些泥刮一下\end{exemple}
\begin{exemple}\jya tɤndɤr nɯ-qhrɯt-a\cmn 我把粉刺刮了一下\end{exemple}
\begin{exemple}\jya βʑɯ kɯ tɕoχtsi ɯ-taʁ kɯ-ɴqhi nɯ ɲɯ-ɤz-nɯ-qhrɯt\cmn 老鼠在啃桌子上的脏东西\end{exemple}\acception{2}
\begin{définition}\fra faire complètement\end{définition}
\begin{définition}\cmn 做得彻底\end{définition}\end{entrée}

\begin{entrée}
\vedette{\hypertarget{Ⓔqia}{\papi{ qia}}}\markboth{qia}{}\classe{vt}
\paradigme{\textit{dir :} \jya pɯ-}
\paradigme{\textit{dir :} \jya thɯ-}
\begin{définition}\fra déchirer, démolir\end{définition}
\begin{définition}\cmn 拆(线)\end{définition}
\begin{exemple}\jya nɤki tɤ-tɯ-mphɯr nɯ pɯ-qie\cmn 你把包了的东西拆开\end{exemple}
\begin{exemple}\jya nɤj tɤ-tɯ-mphɯr aj pjɯ-qie-a\cmn 我拆你包了的东西\end{exemple}
\begin{exemple}\jya tɤ-mphɯr-a nɯ pɯ-qia-t-a\cmn 我把包了的东西拆开\end{exemple}
\begin{exemple}\jya nɤ-ŋga thɯ-qie\cmn 你把(缝了的)衣服拆开\end{exemple}
\begin{exemple}\jya tɤ-tɯ-βzu-t nɯ pɯ-qie\cmn 你把做了的东西拆开\end{exemple}
\begin{exemple}\jya ɯ-kɤpjɤz tha-qia\cmn 他把辫子散开了\end{exemple}
\begin{relation-sémantique}\synonyme{
\hyperlink{Ⓔfɕɯɣ}{\textit{ \papi{fɕɯɣ}}}
}\end{relation-sémantique}
\begin{relation-sémantique}\confer{
\hyperlink{Ⓔɴɢia}{\textit{ \papi{ɴɢia}}}
}\end{relation-sémantique}\end{entrée}

\begin{entrée}
\vedette{\hypertarget{Ⓔqiaβ}{\papi{ qiaβ}}}\markboth{qiaβ}{}\classe{vi}
\paradigme{\textit{dir :} \jya nɯ-}
\begin{définition}\fra amer\end{définition}
\begin{définition}\cmn 苦\end{définition}
\begin{exemple}\jya ɲɯ-qiaβ\end{exemple}\begin{sous-entrée}
\vedette{\hypertarget{}{\papi{ nɤqiaβ}}}\markboth{nɤqiaβ}{}\classe{vt}
\paradigme{\textit{dir :} \jya pɯ-}
\begin{définition}\ 
\begin{déclaration}\grammar{trop}\end{déclaration}\end{définition}
\begin{exemple}\jya pɯ-nɤqiaβ-a\cmn 我觉得太苦了\end{exemple}
\begin{relation-sémantique}\antonyme{
\hyperlink{Ⓔchi}{\textit{ \papi{chi}}}
}\end{relation-sémantique}
\end{sous-entrée}\end{entrée}

\begin{entrée}
\vedette{\hypertarget{Ⓔqioʁ}{\papi{ qioʁ}}}\markboth{qioʁ}{}\classe{vi}
\paradigme{\textit{dir :} \jya lɤ-}
\begin{définition}\fra vomir\end{définition}
\begin{définition}\cmn 呕吐\end{définition}
\begin{exemple}\jya lɤ-qioʁ-a\cmn 我吐了\end{exemple}
\begin{exemple}\jya ɲɯ-ngo-a tɕe lɤ-qioʁ-a\cmn 我病了就吐了\end{exemple}
\begin{exemple}\jya lo-tɯ-qioʁ\cmn 你吐了\end{exemple}
\begin{exemple}\jya tɤ-kɯ-nɯtɕhomba tɕe ɲɯ-kɯ-qioʁ\cmn 感冒的时候就会吐\end{exemple}
\begin{relation-sémantique}\synonyme{
\hyperlink{Ⓔmɯjphɤt}{\textit{ \papi{mɯjphɤt}}}
}\end{relation-sémantique}
\begin{relation-sémantique}\confer{
\hyperlink{Ⓔtɯqioʁ}{\textit{ \papi{tɯqioʁ}}}
}\end{relation-sémantique}\end{entrée}

\begin{entrée}
\vedette{\hypertarget{Ⓔqlaqla}{\papi{ qlaqla}}}\markboth{qlaqla}{}
\classe{idph.2}
\begin{définition}\fra regard étonné, les yeux écarquillés\end{définition}
\begin{définition}\cmn 形容惊奇的眼光,眼睛睁得很大的样子
\begin{déclaration}\use{多指小孩子}\end{déclaration}\end{définition}
\begin{exemple}\jya a-rŋa qlaqla ku-ru ɲɯ-ŋu\cmn 他把眼睛睁得很大,看着我的脸\end{exemple}
\begin{exemple}\jya ɯ-mɲaʁ qlaqla ʑo to-stu\cmn 他把眼睛睁得很大\end{exemple}\end{entrée}

\begin{entrée}
\vedette{\hypertarget{Ⓔqlɤβqlɤβ}{\papi{ qlɤβqlɤβ}}}\markboth{qlɤβqlɤβ}{}\classe{idph.2}\acception{1}
\begin{définition}\fra les yeux ouverts\end{définition}
\begin{définition}\cmn 形容眼睛睁开着的样子\end{définition}
\begin{exemple}\jya ɯ-mɲaʁ ɲɯ-ɲɟɯ qlɤβqlɤβ ʑo ɲɯ-pa\cmn 他眼睛睁得大大的\end{exemple}\acception{2}
\begin{définition}\fra trouble\end{définition}
\begin{définition}\cmn 形容水浑浊的样子\end{définition}
\begin{exemple}\jya cɯβloʁ nɯ kɯ-xtɕɯ-xtɕi ɲɯ-qarndɯm qlɤβqlɤβ\cmn 池塘的水有点浑浊\end{exemple}\begin{sous-entrée}
\vedette{\hypertarget{}{\papi{ qlɤβnɤqlɤβ}}}\markboth{qlɤβnɤqlɤβ}{}\classe{idph.3}
\begin{exemple}\jya ɯ-mɲaʁ qlɤβnɤqlɤβ ɲɯ-ɤsɯ-stu\cmn 他眼睛一眨一眨的\end{exemple}
\end{sous-entrée}\end{entrée}

\begin{entrée}
\vedette{\hypertarget{Ⓔqloŋ}{\papi{ qloŋ}}}\markboth{qloŋ}{}
\begin{relation-sémantique}\confer{
\hyperlink{Ⓔqhloŋ}{\textit{ \papi{qhloŋ}}}
}\end{relation-sémantique}\end{entrée}

\begin{entrée}
\vedette{\hypertarget{Ⓔqlɯβ}{\papi{ qlɯβ}}}\markboth{qlɯβ}{}
\classe{idph.1}
\begin{définition}\fra bruit d'un objet jeté dans l'eau\end{définition}
\begin{définition}\cmn 形容东西跳到水里的声音\end{définition}
\begin{exemple}\jya tɯ-ci ɯ-ŋgɯ qlɯβ ʑo pa-βde\cmn 他啪哒一声扔到水里了\end{exemple}
\begin{exemple}\jya qlɯβ ʑo ka-tshi\cmn 咕噜一声就喝了\end{exemple}\begin{sous-entrée}
\vedette{\hypertarget{}{\papi{ qlɯβnɤqlɯβ}}}\markboth{qlɯβnɤqlɯβ}{}\classe{idph.3}
\begin{exemple}\jya qlɯβnɤqlɯβ ɲɯ-nɤŋkɯŋke\cmn 啪哒啪哒地走\end{exemple}
\begin{relation-sémantique}\confer{
\hyperlink{Ⓔɕqlɯβnɤɕqlɯβ}{\textit{ \papi{ɕqlɯβnɤɕqlɯβ}}}
}\end{relation-sémantique}
\end{sous-entrée}\end{entrée}

\begin{entrée}
\vedette{\hypertarget{Ⓔqlɯqlɯ}{\papi{ qlɯqlɯ}}}\markboth{qlɯqlɯ}{}
\classe{idph.2}
\begin{définition}\fra les yeux écarquillés\end{définition}
\begin{définition}\cmn 形容瞪着眼睛的样子\end{définition}
\begin{exemple}\jya qlɯqlɯ ʑo ku-ru ɲɯ-ŋu\cmn 他瞪着眼睛盯着他\end{exemple}
\begin{relation-sémantique}\confer{
\hyperlink{Ⓔɕquɕqu}{\textit{ \papi{ɕquɕqu}}}
}\end{relation-sémantique}
\begin{relation-sémantique}\confer{
\hyperlink{Ⓔɕqhɯɕqhi}{\textit{ \papi{ɕqhɯɕqhi}}}
}\end{relation-sémantique}\end{entrée}

\begin{entrée}
\vedette{\hypertarget{Ⓔqlɯt}{\papi{ qlɯt}}}\markboth{qlɯt}{}\classe{vt}
\paradigme{\textit{dir :} \jya pɯ-}
\begin{définition}\fra rompre, casser\end{définition}
\begin{définition}\cmn 折断(棍子)\end{définition}
\begin{exemple}\jya si ɯ-rtaʁ pɯ-qlɯt-a\cmn 我把树折断了\end{exemple}
\begin{exemple}\jya jiɕqha @qiche ɯ-taʁ chɤ-mbɣaʁ tɕe, ɯ-mi pjɤ-nɯqlɯt\cmn 他出了车祸,把脚弄断了\end{exemple}
\begin{exemple}\jya nɤ-mu nɯ ndʑu tɯ-ldʑa cinɤ ʑo a-mɤ-nɯ-tɯ-sɯ-qlɯt ra\cmn 你不要让你母亲做任何家务(连一个木棒都不要让她折断)\end{exemple}
\begin{relation-sémantique}\confer{
\hyperlink{Ⓔɴɢlɯt}{\textit{ \papi{ɴɢlɯt}}}
}\end{relation-sémantique}\end{entrée}

\begin{entrée}
\vedette{\hypertarget{Ⓔqoʁmɢlɤcit}{\papi{ qoʁmɢlɤcit}}}\markboth{qoʁmɢlɤcit}{}
\classe{n}
\begin{définition}\fra faire son premier pas (enfant)\end{définition}
\begin{définition}\cmn 会走路(婴儿)\end{définition}
\begin{exemple}\jya a-tɕɯ ɯ-qoʁmɢlɤcit pɯ-cha\cmn 我儿子会走路了\end{exemple}
\begin{relation-sémantique}\confer{
\hyperlink{Ⓔɯ-qoʁ}{\textit{ \papi{ɯ-qoʁ}}}
}\end{relation-sémantique}
\begin{relation-sémantique}\confer{
\hyperlink{Ⓔtɯ-mɢla}{\textit{ \papi{tɯ-mɢla}}}
}\end{relation-sémantique}
\begin{relation-sémantique}\confer{
\hyperlink{Ⓔcit}{\textit{ \papi{cit}}}
}\end{relation-sémantique}\end{entrée}

\begin{entrée}
\vedette{\hypertarget{Ⓔqru}{\papi{ qru}}}\markboth{qru}{}
\classe{vt}
\paradigme{\textit{dir :} \jya tɤ-}
\paradigme{\textit{dir :} \jya \_}
\begin{définition}\fra accueillir, aller chercher\end{définition}
\begin{définition}\cmn 迎接\end{définition}
\begin{exemple}\jya tɤ-kɯ-qru-a\cmn 你接了我\end{exemple}
\begin{exemple}\jya a-rʑaβ ɕ-tɤ-qru-t-a\cmn 我娶了妻子\end{exemple}
\begin{exemple}\jya tɤ-pɤtso ɕ-tɤ-qru-t-a\cmn 我去接了孩子\end{exemple}
\begin{exemple}\jya ndʐuwa ɕ-tɤ-qru-t-a\cmn 我去接了客人\end{exemple}
\begin{exemple}\jya ɯ-rʑaβ ja-qru\cmn 他娶了妻子\end{exemple}
\begin{exemple}\jya ɯʑo kɯ ``a-ɣɯ-jɤ-kɯ-qru-a ra" ɲɯ-ti\cmn 他要求我接他(直译:他说“你要来接我”)\end{exemple}
\begin{relation-sémantique}\confer{
\hyperlink{Ⓔnɯqru}{\textit{ \papi{nɯqru}}}
}\end{relation-sémantique}\begin{sous-entrée}
\vedette{\hypertarget{}{\papi{ saqru}}}\markboth{saqru}{}\classe{vi}
\begin{définition}\ 
\begin{déclaration}\grammar{apass}\end{déclaration}\end{définition}
\end{sous-entrée}\end{entrée}

\begin{entrée}
\vedette{\hypertarget{Ⓔqur}{\papi{ qur}}}\markboth{qur}{}\classe{vt}
\paradigme{\textit{dir :} \jya tɤ-}
\begin{définition}\fra aider\end{définition}
\begin{définition}\cmn 帮助\end{définition}
\begin{exemple}\jya tɤ-qur-a\cmn 我帮了他\end{exemple}
\begin{exemple}\jya tɤ́-wɣ-qur-a\cmn 他帮了我\end{exemple}
\begin{exemple}\jya ɣɯ-tɤ́-wɣ-qur-a\cmn 他来帮我了\end{exemple}
\begin{exemple}\jya ɕ-pɯ-qur-tɕi\cmn 我们去帮他了\end{exemple}
\begin{exemple}\jya tɤ-kɯ-qur-a\cmn 你帮了我\end{exemple}
\begin{exemple}\jya tɤ́-wɣ-qur-a tɕe, jɤ-scɤt-tɕi\cmn 他帮了我,我们俩一起搬了\end{exemple}
\begin{exemple}\jya tɤ-scoz kɤ-rɤt tɤ́-wɣ-qur-a\cmn 他帮我写信了\end{exemple}
\begin{exemple}\jya ɯʑo kɯ ``a-ɣɯ-jɤ-kɯ-qur-a ra" ɲɯ-ti\cmn 他要求我帮他(直译:他说“你要帮我”)\end{exemple}\begin{sous-entrée}
\vedette{\hypertarget{}{\papi{ sɤqur}}}\markboth{sɤqur}{}\classe{vi}
\begin{définition}\ 
\begin{déclaration}\grammar{apass}\end{déclaration}\end{définition}
\begin{définition}\fra aider les gens\end{définition}
\begin{définition}\cmn 帮人家\end{définition}
\begin{exemple}\jya tɕheme kɯ-sɤqur\cmn 女帮手\end{exemple}
\begin{relation-sémantique}\confer{
\hyperlink{Ⓔqurɣa}{\textit{ \papi{qurɣa}}}
}\end{relation-sémantique}
\begin{relation-sémantique}\confer{
\hyperlink{Ⓔaqurle}{\textit{ \papi{aqurle}}}
}\end{relation-sémantique}
\end{sous-entrée}\end{entrée}

\begin{entrée}
\vedette{\hypertarget{Ⓔqra}{\papi{ qra}}}\markboth{qra}{}
\classe{n}
\begin{définition}\fra femelle de yak\end{définition}
\begin{définition}\cmn 母牦牛\end{définition}\end{entrée}

\begin{entrée}
\vedette{\hypertarget{ⒺqraʁⒽ2}{\papi{ qraʁ}}}\markboth{qraʁ}{}\homonyme{2}\classe{n}
\begin{définition}\fra soc\end{définition}
\begin{définition}\cmn 铧头\end{définition}
\begin{exemple}\jya kɤ-ɕlu tɤ-mda tɕe mbɣo ɯ-pa qraʁ kɤ-tshoʁ ra\cmn 当要耕地的时候,要在犁头下面装好铧头\end{exemple}\end{entrée}

\begin{entrée}
\vedette{\hypertarget{ⒺqraʁⒽ1}{\papi{ qraʁ}}}\markboth{qraʁ}{}\homonyme{1}\classe{vt}
\paradigme{\textit{dir :} \jya thɯ-}
\paradigme{\textit{dir :} \jya pɯ-}
\begin{définition}\fra déchirer\end{définition}
\begin{définition}\cmn 撕\end{définition}
\begin{exemple}\jya nɤ-ŋga chɤ-tɯ-qraʁ\cmn 你把衣服撕破了\end{exemple}
\begin{exemple}\jya lɯlu kɯ pɯ́-wɣ-mɯrʁɯz-a tɕe a-jaʁ pjɤ-qraʁ\cmn 猫把我抓了一下,抓破了我的手\end{exemple}
\begin{relation-sémantique}\confer{
\hyperlink{Ⓔɴɢraʁ}{\textit{ \papi{ɴɢraʁ}}}
}\end{relation-sémantique}\end{entrée}

\begin{entrée}
\vedette{\hypertarget{Ⓔqrɤβqrɤβ}{\papi{ qrɤβqrɤβ}}}\markboth{qrɤβqrɤβ}{}\classe{idph.2}
\begin{définition}\fra bariolé et disharmonieux\end{définition}
\begin{définition}\cmn 形容驳杂,不好看的样子\end{définition}
\begin{exemple}\jya ɯ-muj ɯ-mdoʁ ɲɯ-ɤkhra ʑo qrɤβqrɤβ tɕe mɯ́j-mpɕɤr\cmn 它的羽毛颜色东一块西一块的,不好看\end{exemple}\end{entrée}

\begin{entrée}
\vedette{\hypertarget{Ⓔqrɤmɟoʁ}{\papi{ qrɤmɟoʁ}}}\markboth{qrɤmɟoʁ}{}
\classe{n}
\begin{définition}\fra une espèce d'arbre\end{définition}
\begin{définition}\cmn 乔木的一种\end{définition}
\begin{exemple}\jya qrɤmɟoʁ nɯ si kɯ-rkɯn tsa ci ŋu, mbro tsa ɲɯ-jpum mɤ-cha, ɯ-ru nɯ kɯ-ɣɯrni ŋu, ɯ-jwaʁ nɯ kɯ-tɕɤr tɕe kɯ-ɤmtɕoʁ tsa ci ŋu, ɯ-si nɯ kɯ-ndoʁ tsa ci mɤ-kɯ-ngɯt, ɯ-mɯntoʁ cho ɯ-mat ra me, sɤtɕha kɯ-mpja tu-ɬoʁ, sɤtɕha kɯ-ɣɤndʐo tu-ɬoʁ mɤ-cha.\cmn 
\stylefv{qrɤmɟoʁ}是罕见的树种,比较高但是长得不粗,树干是红色的,叶子细而尖,木质脆,不结实。既没有花也没有果实,生长在温暖的地方,寒冷的地方不能生长。
\end{exemple}\end{entrée}

\begin{entrée}
\vedette{\hypertarget{Ⓔqrɤntshom}{\papi{ qrɤntshom}}}\markboth{qrɤntshom}{}\classe{n}
\begin{définition}\fra espèce d'arbrisseau\end{définition}
\begin{définition}\cmn 灌木的一种\end{définition}
\begin{exemple}\jya qrɤntshom nɯ si ci ŋu, ɯ-jwaʁ ɯ-qhuchu nɯ ra tɯ-ɣndʑɤr ʑo kɯ-fse tu, ɯ-jwaʁ kɯ-zri tsa kɯ-ɤmtɕoʁ tsa ŋu, si mɤ-mbro cɯrmbɯ ɯ-ŋgɯ tsa tu-ɬoʁ ŋu, ɯ-mɯntoʁ nɯ kɯ-ndɯ-ndɯβ kɯ-dɯ-dɤn kɯɕnom ʑo kɯ-fse ŋu. ɯ-mdoʁ nɯ kɯ-ɣɯrni tsa ŋu.\cmn 
\stylefv{qrɤntshom}是一种树,叶子背面有像面粉一样的东西,叶子有点长和有点尖,树不高,生长在石头较多的地方,花小而密,穗状,浅红色。
\end{exemple}\end{entrée}

\begin{entrée}
\vedette{\hypertarget{Ⓔqrɤz}{\papi{ qrɤz}}}\markboth{qrɤz}{}\classe{vt}
\paradigme{\textit{dir :} \jya thɯ-}
\paradigme{\textit{dir :} \jya pɯ-}
\begin{définition}\fra raser\end{définition}
\begin{définition}\cmn 剃
\begin{déclaration}\use{给绵羊剃毛应该用\stylefv{krɤɣ},不能用\stylefv{qrɤz}}\end{déclaration}\end{définition}
\begin{exemple}\jya nɤ-ku pɯ-qrɤz\cmn 你剃一下头\end{exemple}
\begin{exemple}\jya tɤ-rme pɯ-qrɤz\cmn 你剃一下毛\end{exemple}
\begin{exemple}\jya ɯ-ku thɯ-qraz-a\cmn 我给他剃了头\end{exemple}
\begin{exemple}\jya a-mtɕhirme pɯ-nɯ-qraz-a\cmn 我剃了胡子\end{exemple}
\begin{relation-sémantique}\confer{
\hyperlink{Ⓔɴɢrɤz}{\textit{ \papi{ɴɢrɤz}}}
}\end{relation-sémantique}\end{entrée}

\begin{entrée}
\vedette{\hypertarget{Ⓔqrubu}{\papi{ qrubu}}}\markboth{qrubu}{}
\classe{n}
\begin{définition}\fra coquillage\end{définition}
\begin{définition}\cmn 贝壳\end{définition}\end{entrée}

\begin{entrée}
\vedette{\hypertarget{Ⓔqurɣa}{\papi{ qurɣa}}}\markboth{qurɣa}{}
\classe{vt}
\paradigme{\textit{dir :} \jya tɤ-}
\begin{définition}\fra aider\end{définition}
\begin{définition}\cmn 帮助\end{définition}
\begin{exemple}\jya tɤ-qurɣa-t-a\cmn 我帮了他\end{exemple}
\begin{relation-sémantique}\confer{
\hyperlink{Ⓔqur}{\textit{ \papi{qur}}}
}\end{relation-sémantique}\end{entrée}

\begin{entrée}
\vedette{\hypertarget{ⒺqroⒽ1}{\papi{ qro}}}\markboth{qro}{}\homonyme{1}\classe{n}
\begin{définition}\fra pigeon\end{définition}
\begin{définition}\cmn 鸽子\end{définition}
\begin{exemple}\jya qro nɯ pɣa ci ŋu, khro mɤ-wxti, pha ɯ-phoŋbu ʑo wɣrum ri ɯ-jme ɯ-ku ri kɯ-ɲaʁ tu, ɯ-mi qarŋe, tɤ-rɤku cho qajɯ tu-ndze ŋu, ftɕar tɕe tɯ-tɕha tɯ-tɕha ku-rɤʑi ɲɯ-ŋu, qartsɯ tɕe ɯ-ɣurʑa ʑo tɯtɯrca ku-rɤʑi ɲɯ-ŋu. kha ku-kɤ-nɤχsu ci ɣɤʑu, ɯʑo ku-kɯ-nɯ-rɤʑi nɯ ɣɤʑu tɕe praʁ pa wuma ʑo kɯ-mbro zɯ ku-rma ɲɯ-ŋu. ɯ-ɕa nɯ wuma ʑo smɤn ɲɯ-ŋu, ɯ-ku nɯ tɯ-ku kɯ-mŋɤm smɤn ɲɯ-ŋu, ɯ-se nɯ li tɯ-ku tɯ-kɤrnoʁ kɯ-mtɕɯr kɯ-phɤn ɲɯ-ŋu. qro tɤ-rɤku ndze ri wuma mɤ-ʁnɤt tɕe ɯ-kɯ-qha rkɯn. qro kɯ-nɯkhɤβɣa kɤ-ti tu tɕe, khɤxtu zɯ ku-zo tɕe, qro nɯ χcha tɯ-tɤxɯr ku-mtɕɯr, ʁe tɯ-tɤxɯr ku-mtɕɯr, tɕe ɯ-khɯkha, `kuku kuku' tu-ti ŋu. nɯ kɤntɕhɯ-ɣjɤn tu-fse ŋu.\cmn 
鸽子是一种鸟,不是很大,身子是白的,但是尾巴末端有黑点,脚是黄色的,吃粮食和虫子。夏天,它们成对地生活,冬天数百只在一起。有的是人在家里喂的,有的是(在野外)自己生活的,栖息在很高的岩洞里。鸽子肉是一种药材,它的头可以治头疼病,血可以治头晕。虽然鸽子吃粮食,但是不是很厉害,讨厌它的人不多。有人也说鸽子会用手磨,因为着落在房背上的时候,就会向左转一圈,向右转一圈,同时又叫\stylefv{kuku kuku},这要重复很多次。
\end{exemple}\end{entrée}

\begin{entrée}
\vedette{\hypertarget{ⒺqroⒽ2}{\papi{ qro}}}\markboth{qro}{}\homonyme{2}
\classe{n}
\begin{définition}\fra fourmi\end{définition}
\begin{définition}\cmn 蚂蚁\end{définition}
\begin{relation-sémantique}\confer{
\hyperlink{Ⓔnɯqro}{\textit{ \papi{nɯqro}}}
}\end{relation-sémantique}\end{entrée}

\begin{entrée}
\vedette{\hypertarget{Ⓔqromkemdoʁ}{\papi{ qromkemdoʁ}}}\markboth{qromkemdoʁ}{}\classe{n}
\begin{définition}\fra violet\end{définition}
\begin{définition}\cmn 紫色
\begin{déclaration} \étymologie{\papi{mdog}}\end{déclaration}\end{définition}
\end{entrée}

\begin{entrée}
\vedette{\hypertarget{Ⓔqroɲaʁ}{\papi{ qroɲaʁ}}}\markboth{qroɲaʁ}{}\classe{n}
\begin{définition}\fra petite fourmi\end{définition}
\begin{définition}\cmn 小蚂蚁\end{définition}
\end{entrée}

\begin{entrée}
\vedette{\hypertarget{Ⓔqrormbɯ}{\papi{ qrormbɯ}}}\markboth{qrormbɯ}{}\classe{n}
\begin{définition}\fra fourmilière\end{définition}
\begin{définition}\cmn 蚂蚁巢\end{définition}
\end{entrée}

\begin{entrée}
\vedette{\hypertarget{Ⓔqrorni}{\papi{ qrorni}}}\markboth{qrorni}{}\classe{n}
\begin{définition}\fra fourmi\end{définition}
\begin{définition}\cmn 蚂蚁\end{définition}
\end{entrée}

\begin{entrée}
\vedette{\hypertarget{Ⓔqrose}{\papi{ qrose}}}\markboth{qrose}{}
\classe{n}
\begin{définition}\fra une espèce d'arbrisseau\end{définition}
\begin{définition}\cmn 灌木的一种【酸酸泡儿】\end{définition}
\begin{exemple}\jya qrose nɯ si wuma mɤ-kɯ-mbro ci ŋu. ɯ-ru ɣɯrni, aɣɯrtɯrtaʁ, mɤ-jpum. ɯ-jwaʁ ɯ-βzɯr kɯmŋu kɯ-tu nɯ ŋu, ɯ-jwaʁ ɯ-taʁ ɯ-rme kɯ-fse kɯ-xtɕi tu ri, mɤ-rʁom. babɯ ɯ-jwaʁ sɤznɤ jndʐɤz. ɯ-mat nɯ thɯ-kɤ-ɣɯri kɯ-fse tɯ-tɤri tɯ-tɤri pjɤ-ɴqoʁ ŋu, tɕe ɯ-mat thɯ-tɯt kɯ-mɤku ɣɯrni, konla thɯ-tɯt tɕe ɲaʁ. ɲɯ́-wɣ-tɕɣaʁ tɕe, ɯ-ŋgɯ tɤ-se kɯ-fse ɲɯ-nɯɬoʁ ŋu. tú-wɣ-ndza tɕe wuma ʑo tɕur, tɤ-pɤtso ra rga-nɯ ma tɯrme kɯ-wxti ra kɤ-ndza mɤ-kɯ-cha.\cmn 
酸酸泡儿是长得不高的一种树。树干是红色的,长很多枝条,不粗。叶子有五个角,叶上有毛,但不粗糙。比\stylefv{babɯ}的叶子大。果实像穿起来的,一串一串地挂在枝上。果实开始成熟时,是红色,完全成熟时,是黑色的,把它捏破时,里面会流出血一样的液体。吃起来很酸,小孩子喜欢吃,但大人不敢吃。
\end{exemple}\end{entrée}

\begin{entrée}
\vedette{\hypertarget{Ⓔqrotsoʁ}{\papi{ qrotsoʁ}}}\markboth{qrotsoʁ}{}
\classe{n}
\begin{définition}\fra une plante\end{définition}
\begin{définition}\cmn 植物的一种\end{définition}
\begin{exemple}\jya qrotsoʁ nɯ sɯjno kɯ-xtɕɯ-xtɕi ci ŋu, ɯ-ku nɯ kɯ-ɤrŋi ŋgɯz kɯ-ɲaʁ ci ŋu, ɯ-mɯntoʁ wɣrum, ɯ-mɯntoʁ nɯ ɯ-jwaʁ rca ri ku-ndzoʁ ŋu, ɯ-ru xtshɯm, ɯ-jwaʁ cho aɣɯmdoʁ, ɯ-qa nɯ kɯ-ɤɣrɤɣrum ci ɲɯ-ŋu, ɯ-tshɯɣa nɯ @ou tsa fse, wuma ʑo ndoʁ. mɤ-sɤndɤɣ ri ɯ-kɯ-ndza me.\cmn 
\stylefv{qrotsoʁ}是一种小草。苗是深绿色的,花是白色的,花和叶子长在一起,茎很细,颜色和叶子一样,根很白,形状和藕一样,很脆。没有毒性但是没有人吃。
\end{exemple}\end{entrée}

\begin{entrée}
\vedette{\hypertarget{Ⓔqrɯ}{\papi{ qrɯ}}}\markboth{qrɯ}{}
\classe{vt}\acception{1}
\paradigme{\textit{dir :} \jya pɯ-}
\begin{définition}\fra casser\end{définition}
\begin{définition}\cmn 捣碎\end{définition}
\begin{exemple}\jya khɯtsa pa-qrɯ\cmn 他把碗打破了\end{exemple}
\begin{exemple}\jya phoŋ pa-qrɯ\cmn 他把瓶子打破了\end{exemple}
\begin{exemple}\jya cupa pa-qrɯ\cmn 他把石板打破了\end{exemple}\acception{2}
\paradigme{\textit{dir :} \jya nɯ-}
\begin{définition}\fra tailler (habit)\end{définition}
\begin{définition}\cmn 裁(衣服)\end{définition}
\begin{exemple}\jya tɯ-ŋga na-qrɯ\cmn 他裁了衣服\end{exemple}
\begin{exemple}\jya aʑo tɯ-ŋga ɲɯ-qri-a\cmn 我裁衣服\end{exemple}
\begin{relation-sémantique}\confer{
\hyperlink{Ⓔɴɢrɯ}{\textit{ \papi{ɴɢrɯ}}}
}\end{relation-sémantique}\end{entrée}

\begin{entrée}
\vedette{\hypertarget{Ⓔqrɯnqrɯn}{\papi{ qrɯnqrɯn}}}\markboth{qrɯnqrɯn}{}\classe{idph.2}
\begin{définition}\fra dont les bandes sont très claires\end{définition}
\begin{définition}\cmn 形容花纹、纹路清晰\end{définition}
\begin{exemple}\jya kɯrtsɤɣ ɲɯ-ɤkhra qrɯnqrɯn\cmn 豹子的花纹很清晰\end{exemple}\end{entrée}

\begin{entrée}
\vedette{\hypertarget{Ⓔqrɯt}{\papi{ qrɯt}}}\markboth{qrɯt}{}\classe{idph.1}
\begin{définition}\fra regard furtif\end{définition}
\begin{définition}\cmn 瞟一眼\end{définition}
\begin{exemple}\jya pjɯ-rɤrɤt ɯ-khɯkha, qrɯt ntsɯ ku-ru ɲɯ-ŋu\cmn 他一边写一边朝这里瞟\end{exemple}
\end{entrée}

\begin{entrée}
\vedette{\hypertarget{Ⓔqusput}{\papi{ qusput}}}\markboth{qusput}{}
\classe{n}
\begin{définition}\fra coucou\end{définition}
\begin{définition}\cmn 杜鹃【布谷鸟】\end{définition}
\begin{exemple}\jya qusput nɯ pɣa ci ŋu, khro mɤ-wxti, ɯ-phoŋbu nɯ ra kɯ-pɣi ci ŋu, ɯ-mtsioʁ ɲaʁ, ɯ-jme rɲɟi tsa, phaʁzla sqaptɯɣ tɕe ju-ɣi ŋu, tɕe ``phaʁzla sqaptɯɣ tɕe mɯ-mɤ-jɤ-azɣɯt-a nɤ pɯ-si-a ra ŋu" tu-ti ɲɯ-ŋgrɤl, qusput tɤ-mbri tɕe ɯ-jme kundi ju-sɯxɕe pɯ-pɯ-ŋu nɤ, ɣɯjpa taχpa nɤkɤro kɤ-ti ɲɯ-ŋu, qusput tɤ-mbri tɕe ɯ-jme taʁki tu-sɯ-khɤt pɯ-pɯ-ŋu nɤ taχpa wuma ʑo pe tu-kɯ-ti ɲɯ-ŋgrɤl ma kundi ku-ɕe tɕe tɤtɤɣ ɯ-mŋu fsu ma me tu-kɯ-ti ɲɯ-ŋu, taʁki tu-βze pɯ-pɯ-ŋu nɤ tɤtɤɣ pjɯ-ɣnde ŋu tu-kɯ-ti ɲɯ-ŋu.\cmn 杜鹃是一种鸟,长得不大,身子是灰色的,嘴是黑色的,尾巴有点长。它五月十一号就到(开始叫),据说“如果我五月十一号还没有到的话,那就是我死了的兆头”。杜鹃叫的时候,如果把尾巴左右摆动,今年的庄稼收成基本可以,如果是上下摆动的话今年的收成特别好,因为如果尾巴左右摆动的话就说明粮食只能装到柜子的口边,如果是上下摆动的话,就说明它在把柜子里的粮食捶得很紧(粮食柜很满)。\end{exemple}\end{entrée}

\begin{entrée}
\vedette{\hypertarget{Ⓔqusputmbro}{\papi{ qusputmbro}}}\markboth{qusputmbro}{}\classe{n}
\begin{définition}\fra libellule\end{définition}
\begin{définition}\cmn 蜻蜓\end{définition}
\end{entrée}

\begin{entrée}
\vedette{\hypertarget{Ⓔqɯmdroŋ}{\papi{ qɯmdroŋ}}}\markboth{qɯmdroŋ}{}\classe{n}
\begin{définition}\fra oie sauvage\end{définition}
\begin{définition}\cmn 大雁\end{définition}
\begin{exemple}\jya qɯmdroŋ nɯ pɣa wuma ʑo kɯ-wxti, ɯ-mke kɯ-zɯ-zri ci ŋu, ɯ-mdoʁ kɯ-pɣi tsa ŋu, ɯ-ro ra kɯ-wɣrum ɣɤʑu. qartsɯ tɕe athi pɕoʁ chɯ-ɕe-nɯ, ftɕar tɕe alo mbroχpa pɕoʁ lu-ɕe-nɯ ɲɯ-ŋu, tɕe tɯtɯrca ʁɟa ʑo ɣnɤsqi ro jamar tu-ŋke-nɯ ɲɯ-ŋu. tɯ-tɯ-rdoʁ a-nɯ-nɯ-βde tɕe, nɯ pjɯ-nɯkɯlu tɕe kɤ-nɯɕe mɯ-ɲɯ-cha tɕe ɯ-zda nɯ kɯ kɤ-ɕar ɯ-xɕɤt kɯ ɯ-ʁar ndzom tu-rɤspɯ ʑo ɲɯ-ŋgrɤl.\cmn 大雁是一种比较大的鸟,颈很长,带有灰色,胸部是白色的。冬天,它们飞往南方,夏天就飞往北方的牧区,二十多只成群飞行。如果有只掉队了,迷失了方向,就回不去了,它的伴侣会用尽全力地找它,一直到翅膀化脓了还在找。\end{exemple}\end{entrée}

\begin{entrée}
\vedette{\hypertarget{Ⓔqɯqli}{\papi{ qɯqli}}}\markboth{qɯqli}{}\classe{idph.2}
\begin{définition}\fra qui a les yeux grand ouverts\end{définition}
\begin{définition}\cmn 形容眼睛睁得很大的样子\end{définition}
\begin{exemple}\jya qala kɯ ɯ-mɲaʁ qɯqli ʑo to-stu\cmn 兔子把眼睛睁得很大\end{exemple}
\begin{exemple}\jya jiɕqha nɯ kɯ ɯ-mɲaʁ qɯqli ʑo to-stu\cmn 他把眼睛睁得很大\end{exemple}
\begin{relation-sémantique}\confer{
\hyperlink{Ⓔɴɢɯɴɢli}{\textit{ \papi{ɴɢɯɴɢli}}}
}\end{relation-sémantique}\end{entrée}

\newpage\caractère{r}

\begin{entrée}
\vedette{\hypertarget{ⒺruⒽ1}{\papi{ ru}}}\markboth{ru}{}\homonyme{1}\classe{vi}
\paradigme{\textit{dir :} \jya \_}
\paradigme{\textit{case :} \jya ɯ-ɕki}
\begin{définition}\fra regarder\end{définition}
\begin{définition}\cmn 看
\begin{déclaration}\use{其它茶堡话方言用\stylefv{mkɯm}表示这个意思}\end{déclaration}\end{définition}
\begin{exemple}\jya a-tɤ-lu ɯ-jo-ɣɯt kɯ ɕ-pjɯ-ru-a\cmn 我要去看牛奶送来了没有\end{exemple}
\begin{exemple}\jya a-ɕki jɤ-ru\cmn 你往我这边看\end{exemple}
\begin{exemple}\jya tɯrme nɯra z-jɤ-ru\cmn 你看一下这些人\end{exemple}
\begin{exemple}\jya tɯrme ra nɯ-ɕki z-jɤ-ru\cmn 你看一下这些人\end{exemple}
\begin{exemple}\jya ɯ-jme ɯ-pɕoʁ jɤ-ru\cmn 他颠倒着睡(头朝着脚的方向)\end{exemple}
\begin{exemple}\jya ci a-mɤ-ɕ-tɤ-ru tɕe, ci ntsɯ tu-dɤn pjɤ-ŋgrɤl\cmn 只要少看一次,就会变得多一点\end{exemple}
\begin{relation-sémantique}\confer{
\hyperlink{Ⓔnɯpɕɯru}{\textit{ \papi{nɯpɕɯru}}}
}\end{relation-sémantique}
\begin{relation-sémantique}\confer{
\hyperlink{Ⓔanɯrŋɤrɯru}{\textit{ \papi{anɯrŋɤrɯru}}}
}\end{relation-sémantique}\begin{sous-entrée}
\vedette{\hypertarget{}{\papi{ sɯɣru}}}\markboth{sɯɣru}{}\classe{vt}
\paradigme{\textit{dir :} \jya \_}
\begin{définition}\ 
\begin{déclaration}\grammar{caus}\end{déclaration}\end{définition}
\begin{définition}\fra laisser regarder, faire voir, montrer\end{définition}
\begin{définition}\cmn 让人看\end{définition}
\begin{exemple}\jya mɯ-jɤ-sɯɣru-t-a\cmn 我没有让他看\end{exemple}
\end{sous-entrée}\end{entrée}

\begin{entrée}
\vedette{\hypertarget{ⒺruⒽ2}{\papi{ ru}}}\markboth{ru}{}\homonyme{2}\classe{vt}
\paradigme{\textit{dir :} \jya \_}
\begin{définition}\fra amener\end{définition}
\begin{définition}\cmn 带
\begin{déclaration}\use{一般带有离己或朝己前缀}\end{déclaration}\end{définition}
\begin{exemple}\jya si ɕ-pɯ-asɯ-ru-a\cmn (前一段时间)我背柴了回来\end{exemple}
\begin{exemple}\jya a-zrɯɣ pɯ-re ra\cmn 你帮我找虱子吧\end{exemple}
\begin{exemple}\jya nɤ-ɕki jɯɣi ki ɣɯ-tɤ-ru-t-a\cmn 我从你那里把书拿上来了\end{exemple}
\begin{exemple}\jya a-tɤ-lu jo-ɣɯt tɕe ɕ-tɤ-ru-t-a\cmn 牛奶带来了,我把它拿上来了\end{exemple}\begin{sous-entrée}
\vedette{\hypertarget{}{\papi{ nɯru}}}\markboth{nɯru}{}
\begin{exemple}\jya ʑ-lɤ-nɯ-ru-t-a\cmn 我自己去拿来\end{exemple}
\end{sous-entrée}\end{entrée}

\begin{entrée}
\vedette{\hypertarget{ⒺruⒽ3}{\papi{ ru}}}\markboth{ru}{}\homonyme{3}\classe{vt}
\paradigme{\textit{dir :} \jya pɯ-}
\begin{définition}\fra prédire l'avenir\end{définition}
\begin{définition}\cmn 算命\end{définition}
\begin{exemple}\jya mphrɯmɯ pɯ-re\cmn 你算命吧\end{exemple}
\begin{exemple}\jya mphrɯmɯ pɯ-ru-t-a\cmn 我算了命\end{exemple}
\begin{exemple}\jya βlama nɯ-sqar-a tɕe mphrɯmɯ pa-ru\cmn 我请了喇嘛算命\end{exemple}\begin{sous-entrée}
\vedette{\hypertarget{}{\papi{ sɯru}}}\markboth{sɯru}{}\classe{vt}
\paradigme{\textit{dir :} \jya pɯ-}
\begin{définition}\fra demander de regarder l'avenir\end{définition}
\begin{définition}\cmn 请人算命\end{définition}
\begin{exemple}\jya mphrɯmɯ ɕ-pɯ-sɯre\cmn 你去请人算命吧\end{exemple}
\begin{exemple}\jya aʑɯɣ kɯnɤ a-mphrɯmɯ a-pɯ-tɯ-sɯre ɯ-tɯ́-cha\cmn 你能不能叫别人帮我算命\end{exemple}
\end{sous-entrée}\end{entrée}

\begin{entrée}
\vedette{\hypertarget{ⒺraⒽ2}{\papi{ ra}}}\markboth{ra}{}\homonyme{2}
\classe{det}
\begin{définition}\fra pluriel\end{définition}
\begin{définition}\cmn 复数\end{définition}
\end{entrée}

\begin{entrée}
\vedette{\hypertarget{ⒺraⒽ1}{\papi{ ra}}}\markboth{ra}{}\homonyme{1}
\classe{vs}\acception{1}
\begin{définition}\fra devoir\end{définition}
\begin{définition}\cmn 应该\end{définition}
\begin{exemple}\jya kɯ-ɣɯsɯphɯt ɕe-a ra\cmn 我要去砍柴\end{exemple}\acception{2}
\begin{définition}\fra avoir besoin de\end{définition}
\begin{définition}\cmn 需要\end{définition}
\begin{exemple}\jya nɤʑɯɣ tɕhi ra?\cmn 你需要什么?\end{exemple}
\begin{exemple}\jya aʑɯɣ kɯ-ra me\cmn 我什么也不需要\end{exemple}
\begin{relation-sémantique}\confer{
\hyperlink{Ⓔɣɤra}{\textit{ \papi{ɣɤra}}}
}\end{relation-sémantique}\begin{sous-entrée}
\vedette{\hypertarget{}{\papi{ mɤra ma}}}\markboth{mɤra ma}{}\classe{cnj}
\begin{définition}\fra non seulement\end{définition}
\begin{définition}\cmn 不但\end{définition}
\begin{exemple}\jya jisŋi tɤ-rɯndzɤtshi-a tɕe tɯmgo nɯ mɤra ma tɤjko kɯnɤ tɤ-ndza-t-a\cmn 我今天不但吃了饭,还吃了酸菜\end{exemple}
\end{sous-entrée}\end{entrée}

\begin{entrée}
\vedette{\hypertarget{Ⓔraka}{\papi{ raka}}}\markboth{raka}{}\classe{n}
\begin{définition}\fra façon dont poussent les cornes (animaux)\end{définition}
\begin{définition}\cmn 长势(角)\end{définition}
\begin{exemple}\jya kɯki jla ki ɯ-raka wuma ʑo ɲɯ-βdi\cmn 这头犏牛的角长得很好\end{exemple}\end{entrée}

\begin{entrée}
\vedette{\hypertarget{Ⓔralarala}{\papi{ ralarala}}}\markboth{ralarala}{}\classe{adv}
\begin{définition}\fra très rapide\end{définition}
\begin{définition}\cmn 飞快\end{définition}\end{entrée}

\begin{entrée}
\vedette{\hypertarget{Ⓔrambɯm}{\papi{ rambɯm}}}\markboth{rambɯm}{}\classe{n}
\begin{définition}\fra ramure de cerf\end{définition}
\begin{définition}\cmn 雄鹿的角
\begin{déclaration} \étymologie{\papi{rwa.ⁿbum}}\end{déclaration}\end{définition}\end{entrée}

\begin{entrée}
\vedette{\hypertarget{ⒺraŋⒽ2}{\papi{ raŋ}}}\markboth{raŋ}{}\homonyme{2}
\classe{n}
\begin{définition}\fra soi-même\end{définition}
\begin{définition}\cmn 自己
\begin{déclaration} \étymologie{\papi{raŋ}}\end{déclaration}\end{définition}
\begin{exemple}\jya nɤʑo tu-tɯ-ti mɤ-ra ma aʑo raŋ tu-nɯti-a jɤɣ\cmn 不用你说,我自己说\end{exemple}
\begin{exemple}\jya kɯki laχtɕha ki aʑo raŋ ɣɯ pɯ-kɯ-nɯ-tɯ-tu ɕti\cmn 这个东西是我自己曾经有过的\end{exemple}\end{entrée}

\begin{entrée}
\vedette{\hypertarget{ⒺraŋⒽ1}{\papi{ raŋ}}}\markboth{raŋ}{}\homonyme{1}\classe{vs}
\paradigme{\textit{dir :} \jya tɤ-}
\begin{définition}\fra long (temps)\end{définition}
\begin{définition}\cmn 长(时间,路程)
\begin{déclaration} \étymologie{\papi{riŋ}}\end{déclaration}\end{définition}
\begin{exemple}\jya ki ɲɯ-fse nɯ ɲɯ-raŋ\cmn 这样、那样\end{exemple}
\begin{exemple}\jya mɤ-kɯ-fse mɤ-kɯ-raŋ ʑo maŋe\cmn 有很多种现象,预料不到的现象都有\end{exemple}\end{entrée}

\begin{entrée}
\vedette{\hypertarget{Ⓔraŋri}{\papi{ raŋri}}}\markboth{raŋri}{}\classe{adv}
\begin{définition}\fra chacun\end{définition}
\begin{définition}\cmn 每个
\begin{déclaration} \étymologie{\papi{raŋ.re}}\end{déclaration}\end{définition}
\begin{exemple}\jya kɤndza thɯ́-wɣ-kro tɕe, tɯrme raŋri ɣɯ ɲɯ́-wɣ-kho ra\cmn 分食物的时候,要分给每一个人\end{exemple}
\begin{exemple}\jya tɯrme raŋri ɣɯ nɯ-mɲaʁ tu ɕti\cmn 每个人有眼睛\end{exemple}\begin{sous-entrée}
\vedette{\hypertarget{}{\papi{ rɯri}}}\markboth{rɯri}{}\classe{adv}
\begin{définition}\fra chacun\end{définition}
\begin{définition}\cmn 每个
\begin{déclaration} \étymologie{\papi{re.re}}\end{déclaration}\end{définition}
\end{sous-entrée}\end{entrée}

\begin{entrée}
\vedette{\hypertarget{Ⓔraŋɯŋi}{\papi{ raŋɯŋi}}}\markboth{raŋɯŋi}{}
\classe{idph.7}
\begin{définition}\fra qui pend (substance visqueuse)\end{définition}
\begin{définition}\cmn 形容(黏稠的物质,如鼻涕)吊着的样子
\begin{déclaration} \étymologie{\papi{raŋ}}\end{déclaration}\end{définition}
\begin{exemple}\jya nɤɕnaβ raŋɯŋi ʑo pɯ-nɯɬoʁ\cmn 你的鼻涕流出来了\end{exemple}
\begin{exemple}\jya ɲɯ-nɯtɕhomba tɕe, ɯ-ɕnɤmtsrɯɣ raŋɯŋi ɲɯ-ɤsɯ-stu\cmn 他感冒了就吊着鼻涕\end{exemple}
\begin{exemple}\jya tɤjko pjɤ-tɕur tɕe, raŋɯŋi ɲɯ-pa\cmn 酸菜很酸,吊着有黏性的液体\end{exemple}\end{entrée}

\begin{entrée}
\vedette{\hypertarget{ⒺraʁⒽ2}{\papi{ raʁ}}}\markboth{raʁ}{}\homonyme{2}\classe{n}
\begin{définition}\fra laiton\end{définition}
\begin{définition}\cmn 黄铜
\begin{déclaration} \étymologie{\papi{rag}}\end{déclaration}\end{définition}\end{entrée}

\begin{entrée}
\vedette{\hypertarget{ⒺraʁⒽ1}{\papi{ raʁ}}}\markboth{raʁ}{}\homonyme{1}\classe{vi}\acception{1}
\paradigme{\textit{dir :} \jya kɤ-}
\paradigme{\textit{dir :} \jya thɯ-}
\begin{définition}\fra être bloqué\end{définition}
\begin{définition}\cmn 卡住\end{définition}
\begin{exemple}\jya tɤtshoʁ ko-raʁ\cmn 钉子卡住了(取不出来)\end{exemple}
\begin{exemple}\jya tʂhazwa a-rqo thɯ-raʁ\cmn 茶叶卡在我的喉咙里了\end{exemple}
\begin{exemple}\jya kɯ-spoʁ ɯ-ŋgɯ cho-raʁ\cmn 在洞里卡住了\end{exemple}\acception{2}
\paradigme{\textit{dir :} \jya nɯ-}
\begin{définition}\fra s’accrocher\end{définition}
\begin{définition}\cmn 钩住\end{définition}
\begin{exemple}\jya a-ŋga nɯ-raʁ\cmn 我的衣服被钩住了\end{exemple}\begin{sous-entrée}
\vedette{\hypertarget{}{\papi{ sɯɣraʁ}}}\markboth{sɯɣraʁ}{}\classe{vt}
\begin{définition}\fra accrocher\end{définition}
\begin{définition}\cmn 卡住;钩住\end{définition}
\end{sous-entrée}\end{entrée}

\begin{entrée}
\vedette{\hypertarget{Ⓔraʁdoŋ}{\papi{ raʁdoŋ}}}\markboth{raʁdoŋ}{}
\classe{n}
\begin{définition}\fra trompette\end{définition}
\begin{définition}\cmn 长号角
\begin{déclaration} \étymologie{\papi{rag.duŋ}}\end{déclaration}\end{définition}\end{entrée}

\begin{entrée}
\vedette{\hypertarget{Ⓔraʁjɯ}{\papi{ raʁjɯ}}}\markboth{raʁjɯ}{}\classe{vi}
\paradigme{\textit{dir :} \jya nɯ-}
\begin{définition}\fra faire semblant d'être incapable (pour ne pas avoir à travailler)\end{définition}
\begin{définition}\cmn 装作自己不会做\end{définition}
\begin{exemple}\jya ma-nɯ-tɯ-raʁjɯ kɯ nɯ-rɤma\cmn 你不要装作自己不会,要劳动!\end{exemple}
\begin{relation-sémantique}\synonyme{
\hyperlink{Ⓔnɯɕɯʁjɯ}{\textit{ \papi{nɯɕɯʁjɯ}}}
}\end{relation-sémantique}\end{entrée}

\begin{entrée}
\vedette{\hypertarget{Ⓔraʁle}{\papi{ raʁle}}}\markboth{raʁle}{}\classe{vi}
\paradigme{\textit{dir :} \jya tɤ-}
\begin{définition}\fra être poli\end{définition}
\begin{définition}\cmn 客气\end{définition}
\begin{exemple}\jya tɤ-raʁle-a tɕe kɯ-dɤn mɯ-tɤ-ndza-t-a\cmn 我为了表现客气,没有吃很多\end{exemple}
\begin{exemple}\jya ma-tɤ-tɯ-raʁle je\cmn 不用客气\end{exemple}
\begin{relation-sémantique}\confer{
\hyperlink{Ⓔɯ-ʁle}{\textit{ \papi{ɯ-ʁle}}}
}\end{relation-sémantique}\end{entrée}

\begin{entrée}
\vedette{\hypertarget{Ⓔraʁmaʁ}{\papi{ raʁmaʁ}}}\markboth{raʁmaʁ}{}\classe{part}
\begin{définition}\fra d'accord ?\end{définition}
\begin{définition}\cmn 好吗?\end{définition}
\begin{exemple}\jya nɤ-ŋga tɤ-ŋge raʁmaʁ ma ɲɯ-ɣɤndʐo\cmn 你多穿点衣服好吗,天气很冷。\end{exemple}\end{entrée}

\begin{entrée}
\vedette{\hypertarget{Ⓔraʁraʁ}{\papi{ raʁraʁ}}}\markboth{raʁraʁ}{}
\classe{idph.2}
\begin{définition}\fra noir comme du charbon\end{définition}
\begin{définition}\cmn 形容极黑\end{définition}
\begin{exemple}\jya qaʑo kɯ ɯ-pɯ kɯ-ɲaʁ raʁraʁ ʑo ɲɤ-lɤt\cmn 绵羊下了黑黢黢的小羊羔\end{exemple}\end{entrée}

\begin{entrée}
\vedette{\hypertarget{Ⓔraʁrɤt}{\papi{ raʁrɤt}}}\markboth{raʁrɤt}{}\classe{vi}
\paradigme{\textit{dir :} \jya pɯ-}
\begin{définition}\fra faire du charbon\end{définition}
\begin{définition}\cmn 烧木炭
\begin{déclaration}\grammar{denom}\end{déclaration}\end{définition}
\begin{relation-sémantique}\confer{
\hyperlink{Ⓔta-ʁrɤt}{\textit{ \papi{ta-ʁrɤt}}}
}\end{relation-sémantique}
\end{entrée}

\begin{entrée}
\vedette{\hypertarget{Ⓔraʁrɯz}{\papi{ raʁrɯz}}}\markboth{raʁrɯz}{}\classe{vt}
\paradigme{\textit{dir :} \jya thɯ-}
\begin{définition}\fra balayer\end{définition}
\begin{définition}\cmn 扫\end{définition}
\begin{exemple}\jya ɯ-thoʁ ko-ɴqhi tɕe thɯ-raʁrɯz\cmn 地变脏了,你扫一下吧\end{exemple}\begin{sous-entrée}
\vedette{\hypertarget{}{\papi{ rɤroʁrɯz}}}\markboth{rɤroʁrɯz}{}\classe{vi}
\paradigme{\textit{dir :} \jya tɤ-}
\begin{définition}\ 
\begin{déclaration}\grammar{apass}\end{déclaration}
\begin{déclaration}\grammar{denom}\end{déclaration}\end{définition}
\begin{définition}\fra changer les affaire et balayer le sol\end{définition}
\begin{définition}\cmn 收拾和扫地\end{définition}
\begin{exemple}\jya kha ra ɲɯ-ɤdrɤt tɕe tɤ-rɤroʁrɯz-a\cmn 家里很乱,我又收拾了,又扫了地\end{exemple}
\begin{relation-sémantique}\confer{
\hyperlink{Ⓔroʁrɯz}{\textit{ \papi{roʁrɯz}}}
}\end{relation-sémantique}
\end{sous-entrée}\begin{sous-entrée}
\vedette{\hypertarget{}{\papi{ zraʁrɯz}}}\markboth{zraʁrɯz}{}\classe{vt}
\paradigme{\textit{dir :} \jya thɯ-}
\begin{définition}\ 
\begin{déclaration}\grammar{caus}\end{déclaration}\end{définition}
\begin{définition}\fra balayer avec\end{définition}
\begin{définition}\cmn 用……扫\end{définition}
\begin{exemple}\jya zɣɤmbu kɯ thɯ-z-raʁrɯz-a\cmn 我用扫把扫了\end{exemple}
\end{sous-entrée}\end{entrée}

\begin{entrée}
\vedette{\hypertarget{Ⓔrasqaβ}{\papi{ rasqaβ}}}\markboth{rasqaβ}{}\classe{n}
\begin{définition}\fra aiguille à coudre\end{définition}
\begin{définition}\cmn 缝衣针\end{définition}
\begin{relation-sémantique}\confer{
\hyperlink{Ⓔtaqaβ}{\textit{ \papi{taqaβ}}}
}\end{relation-sémantique}\end{entrée}

\begin{entrée}
\vedette{\hypertarget{Ⓔrasti}{\papi{ rasti}}}\markboth{rasti}{}
\classe{n}
\begin{définition}\fra navet (Brassica sp.)\end{définition}
\begin{définition}\cmn 芜菁【圆根】\end{définition}
\begin{exemple}\jya rasti nɯ zgoku tsa tu-ɬoʁ cha. ɯ-mdoʁ nɯ kɯ-pɣi tsa ŋu. ɯ-qa nɯ kɯ-ɤrtɯm tɕe, kɯ-wxti ʑo ɲɯ-βze cha. stu kɯ-wxti nɯ sqamŋu spaprɤɣ ɯ-tɯrpa jamar ɲɯ-βze cha. ɯ-taʁ tɕe ɯ-sku nɯ kɯ-dɯ-dɤn tu-ɬoʁ cha, ɯ-pɕi nɯ pjɯ́-wɣ-nɯ-phɯt tɕe, ɯ-ŋgɯ mɤʑɯ ɲɯ-βze ɕti, tɕe nɯ pjɯ-kɯ-phɯt nɤ, ɲɯ-kɤ-nɤrqaʁ rmi. nɯ ɯ-ku nɯ rasti rmi, ɯ-qa nɯ rɤjndoʁ rmi. rasti nɯ kú-wɣ-sqa tɕe, kɤ-smi tɕe, pjɯ́-wɣ-sɯɣ-tɕur tɕe, tɤ-jko rmi, tɕe tɯ-mgo zmɤrɤβ wuma ʑo pe. tɯrme kɤ-ndza kɯmɯxte ʑo rga-nɯ. ɯ-jndoʁ nɯ paʁndza wuma ʑɤ pe. kú-wɣ-sqa tɕe, kɤ-smi tɕe, tɯrme kɯnɤ tú-wɣ-ndza ŋgrɤl. ɯ-jndoʁ kɤ-kɤ-sqa nɯ rɤsqa rmi.\cmn 
圆根生长在比较高的山地里,有点灰色,它的根是圆形的,长得很大。最大可以长到15、16斤。上面长很多苗,外层的苗拔下后,里面的苗会继续长。这种拔苗的方法叫\stylefv{kɤ-nɤrqaʁ}。苗叫\stylefv{rasti},根叫\stylefv{rɤjndoʁ}。圆根苗煮熟了以后弄酸了,叫酸菜,是很好的菜。大多数人喜欢吃。圆根是喂猪的好饲料。煮熟后,人也可以吃。煮熟的圆根叫\stylefv{rɤsqa}。
\end{exemple}
\begin{relation-sémantique}\synonyme{
\hyperlink{Ⓔtɤjkɤspa}{\textit{ \papi{tɤjkɤspa}}}
}\end{relation-sémantique}\end{entrée}

\begin{entrée}
\vedette{\hypertarget{Ⓔraχpi}{\papi{ raχpi}}}\markboth{raχpi}{}\classe{vi}
\paradigme{\textit{dir :} \jya tɤ-}
\begin{définition}\ 
\begin{déclaration}\grammar{denom}\end{déclaration}\end{définition}
\begin{définition}\fra raconter\end{définition}
\begin{définition}\cmn 陈述\end{définition}
\begin{relation-sémantique}\confer{
\hyperlink{Ⓔχpi}{\textit{ \papi{χpi}}}
}\end{relation-sémantique}\end{entrée}

\begin{entrée}
\vedette{\hypertarget{Ⓔraχtɕɤz}{\papi{ raχtɕɤz}}}\markboth{raχtɕɤz}{}
\classe{vt}
\paradigme{\textit{dir :} \jya tɤ-}
\begin{définition}\fra bien traiter, accueillir chaleureusement\end{définition}
\begin{définition}\cmn 款待\end{définition}
\begin{exemple}\jya jɯfɕɯr kɯ-mɯm to-sɤftɕaka tɤ́-wɣ-raχtɕaz-a\cmn 他昨天做了好吃的,款待了我\end{exemple}
\begin{exemple}\jya ndʐuwa ta-raχtɕɤz\cmn 他款待了客人\end{exemple}\begin{sous-entrée}
\vedette{\hypertarget{}{\papi{ sɤraχtɕɤz}}}\markboth{sɤraχtɕɤz}{}\classe{vi}
\begin{définition}\ 
\begin{déclaration}\grammar{apass}\end{déclaration}\end{définition}
\begin{définition}\fra bien traiter les gens\end{définition}
\begin{définition}\cmn 关心别人\end{définition}
\begin{exemple}\jya nɤki smɤnba nɯ ɲɯ-sɤraχtɕɤz\cmn 那个医生关心别人\end{exemple}
\begin{exemple}\jya nɤki tɤ-mu nɯ kɯ-sɤraχtɕɤz ci ŋu\cmn 那位大娘是关心别人的\end{exemple}
\end{sous-entrée}\begin{sous-entrée}
\vedette{\hypertarget{}{\papi{ ʑɣɤraχtɕɤz}}}\markboth{ʑɣɤraχtɕɤz}{}\classe{vi}
\begin{définition}\ 
\begin{déclaration}\grammar{refl}\end{déclaration}\end{définition}
\begin{définition}\fra bien se traiter soi-même\end{définition}
\begin{définition}\cmn 优待自己\end{définition}
\end{sous-entrée}\end{entrée}

\begin{entrée}
\vedette{\hypertarget{Ⓔraχtɕi}{\papi{ raχtɕi}}}\markboth{raχtɕi}{}
\begin{relation-sémantique}\confer{
\hyperlink{Ⓔχtɕi}{\textit{ \papi{χtɕi}}}
}\end{relation-sémantique}\end{entrée}

\begin{entrée}
\vedette{\hypertarget{Ⓔraχtɕoŋ}{\papi{ raχtɕoŋ}}}\markboth{raχtɕoŋ}{}\classe{n}
\begin{définition}\fra bovidé de couleur noire\end{définition}
\begin{définition}\cmn 黑色的牛
\begin{déclaration} \étymologie{\papi{rwa.tɕʰuŋ?}}\end{déclaration}\end{définition}
\end{entrée}

\begin{entrée}
\vedette{\hypertarget{Ⓔraχtɕɯmχtɕɤz}{\papi{ raχtɕɯmχtɕɤz}}}\markboth{raχtɕɯmχtɕɤz}{}
\classe{vt}
\begin{définition}\fra bien traiter\end{définition}
\begin{définition}\cmn 优待\end{définition}
\begin{exemple}\jya nɤʑo ɲɯ-tɯ-raχtɕɯmχtɕɤz\cmn 你优待他\end{exemple}
\begin{exemple}\jya ɲɯ-tɯ́-wɣ-raχtɕɯmχtɕɤz\cmn 他优待你\end{exemple}\end{entrée}

\begin{entrée}
\vedette{\hypertarget{Ⓔraχtɕɯʁɟo}{\papi{ raχtɕɯʁɟo}}}\markboth{raχtɕɯʁɟo}{} (\variante{rɤχtɕɯʁɟo}) \classe{vi}
\paradigme{\textit{dir :} \jya pɯ-}
\begin{définition}\fra laver\end{définition}
\begin{définition}\cmn 洗澡\end{définition}
\begin{exemple}\jya paʁ ra cɯβloʁ ɯ-ŋgɯ pɯ-rɤχtɕɯʁɟo-nɯ\cmn 猪在池塘里洗澡了\end{exemple}
\begin{relation-sémantique}\confer{
\hyperlink{Ⓔχtɕi}{\textit{ \papi{χtɕi}}}
}\end{relation-sémantique}\end{entrée}

\begin{entrée}
\vedette{\hypertarget{Ⓔraχtsur}{\papi{ raχtsur}}}\markboth{raχtsur}{}
\begin{relation-sémantique}\confer{
\hyperlink{Ⓔχtsur}{\textit{ \papi{χtsur}}}
}\end{relation-sémantique}\end{entrée}

\begin{entrée}
\vedette{\hypertarget{Ⓔraχtɯ}{\papi{ raχtɯ}}}\markboth{raχtɯ}{}
\begin{relation-sémantique}\confer{
\hyperlink{Ⓔχtɯ}{\textit{ \papi{χtɯ}}}
}\end{relation-sémantique}\end{entrée}

\begin{entrée}
\vedette{\hypertarget{Ⓔraχtɯtsɣe}{\papi{ raχtɯtsɣe}}}\markboth{raχtɯtsɣe}{}\classe{vi}
\paradigme{\textit{dir :} \jya thɯ-}
\begin{définition}\ 
\begin{déclaration}\grammar{comp}\end{déclaration}\end{définition}
\begin{définition}\fra faire du commerce\end{définition}
\begin{définition}\cmn 做生意\end{définition}
\begin{relation-sémantique}\confer{
\hyperlink{Ⓔχtɯ}{\textit{ \papi{χtɯ}}}
}\end{relation-sémantique}
\begin{relation-sémantique}\confer{
\hyperlink{Ⓔntsɣe}{\textit{ \papi{ntsɣe}}}
}\end{relation-sémantique}
\end{entrée}

\begin{entrée}
\vedette{\hypertarget{Ⓔraz}{\papi{ raz}}}\markboth{raz}{}
\classe{n}
\begin{définition}\fra tissu\end{définition}
\begin{définition}\cmn 布
\begin{déclaration} \étymologie{\papi{ras}}\end{déclaration}\end{définition}\end{entrée}

\begin{entrée}
\vedette{\hypertarget{Ⓔrazmbe}{\papi{ razmbe}}}\markboth{razmbe}{}\classe{n}
\begin{définition}\fra vieux tissu\end{définition}
\begin{définition}\cmn 旧的布料\end{définition}\end{entrée}

\begin{entrée}
\vedette{\hypertarget{Ⓔrazɴɢu}{\papi{ razɴɢu}}}\markboth{razɴɢu}{}
\classe{n}
\begin{définition}\fra espèce d'arbrisseau\end{définition}
\begin{définition}\cmn 灌木的一种\end{définition}
\begin{exemple}\jya razɴɢu nɯ si kɯ-mbɯ-mbɤr ci ŋu, rdɤstaʁ cho praʁ ɯ-taʁ jɯ-ʑɣɤɲɟoʁ tɕe, tu-ɬoʁ ŋu. ɯ-jwaʁ cho ɯ-ru nɯ ra ɯ-mat nɯ ra tɤ-ru cho aɣɯmdoʁ, ɯ-mat cho ɯ-jwaʁ ra fsapaʁ ra kɯ tu-ndza-nɯ ŋgrɤl\cmn 
\stylefv{razɴɢu} 是一种非常矮小的树,爬在石头和岩石上。叶子、干和果实颜色和火棘一样,牲畜都吃它的果实和叶子。
\end{exemple}\end{entrée}

\begin{entrée}
\vedette{\hypertarget{Ⓔrazri}{\papi{ razri}}}\markboth{razri}{}
\classe{n}
\begin{définition}\fra fil en coton\end{définition}
\begin{définition}\cmn 棉线\end{définition}
\begin{relation-sémantique}\confer{
\hyperlink{Ⓔtɤ-ri}{\textit{ \papi{tɤ-ri}}}
}\end{relation-sémantique}
\begin{relation-sémantique}\confer{
\hyperlink{Ⓔraz}{\textit{ \papi{raz}}}
}\end{relation-sémantique}\end{entrée}

\begin{entrée}
\vedette{\hypertarget{Ⓔrɤβraʁ}{\papi{ rɤβraʁ}}}\markboth{rɤβraʁ}{}\classe{vt}\acception{1}
\paradigme{\textit{dir :} \jya nɯ-}
\begin{exemple}\jya a-mgɯr nɯ-rɤβraʁ\cmn 你抠我的背部吧\end{exemple}
\begin{exemple}\jya nɯ-nɯ-rɤβraʁ-a\cmn 我抠了一下痒\end{exemple}
\begin{exemple}\jya ma-nɯ-tɯ-rɤβraʁ\cmn 你不要抓痒!\end{exemple}
\begin{exemple}\jya paʁ na-rɤβraʁ\cmn 他给猪抓痒了\end{exemple}
\begin{exemple}\jya paʁ ɲɯ́-wɣ-rɤβraʁ tɕe rga\cmn 猪很喜欢抓痒\end{exemple}\acception{2}
\paradigme{\textit{dir :} \jya pɯ-}
\begin{définition}\fra ratisser\end{définition}
\begin{définition}\cmn 耙\end{définition}
\begin{exemple}\jya sɤtɕha ra pa-rɤβraʁ\cmn 他耙了地\end{exemple}\begin{sous-entrée}
\vedette{\hypertarget{}{\papi{ zrɤβraʁ}}}\markboth{zrɤβraʁ}{}\classe{vt}
\begin{définition}\ 
\begin{déclaration}\grammar{caus}\end{déclaration}\end{définition}\acception{1}
\paradigme{\textit{dir :} \jya nɯ-}
\begin{définition}\fra gratter avec\end{définition}
\begin{définition}\cmn 用……抠\end{définition}
\begin{exemple}\jya tɤ-lu ɯ-sta kɯ-khrɯ ɲɯ-ŋu tɕe, kɤ-χtɕi ɲɯ-ɴqa, tɕe a-ndzrɯ kɯ nɯ-zrɤβraʁ-a ʑo pɯ-ra\cmn 装过牛奶的瓶子里面干硬了,很难洗,我只好用指甲抠掉\end{exemple}\acception{2}
\paradigme{\textit{dir :} \jya pɯ-}
\begin{définition}\fra ratisser avec\end{définition}
\begin{définition}\cmn 用……耙\end{définition}
\begin{exemple}\jya qarɤt kɯ pɯ-zrɤβraʁ-a\cmn 我用耙子耙了\end{exemple}
\end{sous-entrée}\begin{sous-entrée}
\vedette{\hypertarget{}{\papi{ ʑɣɤrɤβraʁ}}}\markboth{ʑɣɤrɤβraʁ}{}\classe{vi}
\begin{définition}\ 
\begin{déclaration}\grammar{refl}\end{déclaration}\end{définition}
\begin{définition}\fra se gratter\end{définition}
\begin{définition}\cmn 给自己抓痒\end{définition}
\begin{exemple}\jya paʁ khrɯɣnɤkhrɯɣ ɲɯ-ʑɣɤrɤβraʁ\cmn 猪在抓痒\end{exemple}
\end{sous-entrée}\end{entrée}

\begin{entrée}
\vedette{\hypertarget{Ⓔrɤβʁa}{\papi{ rɤβʁa}}}\markboth{rɤβʁa}{} (\variante{rɤβʁɯʁa}) 
\classe{vi}
\paradigme{\textit{dir :} \jya tɤ-}
\begin{définition}\fra rugir\end{définition}
\begin{définition}\cmn (猫、豹子)吼叫\end{définition}
\begin{exemple}\jya lɯlu ɲɯ-rɤβʁa\cmn 猫在吼叫\end{exemple}
\begin{exemple}\jya kɯrtsɤɣ ɲɯ-rɤβʁa\cmn 豹子在吼叫\end{exemple}
\begin{exemple}\jya mbala ɲɯ-rɤβʁa\cmn 公牛在吼叫\end{exemple}\end{entrée}

\begin{entrée}
\vedette{\hypertarget{Ⓔrɤβzjoz}{\papi{ rɤβzjoz}}}\markboth{rɤβzjoz}{}
\classe{vi}
\paradigme{\textit{dir :} \jya pɯ-}
\begin{définition}\ 
\begin{déclaration}\grammar{apass}\end{déclaration}\end{définition}
\begin{définition}\fra étudier\end{définition}
\begin{définition}\cmn 学习\end{définition}
\begin{exemple}\jya pɯ-rɤβzjoz-a\cmn 我读书了\end{exemple}
\begin{relation-sémantique}\confer{
\hyperlink{Ⓔβzjoz}{\textit{ \papi{βzjoz}}}
}\end{relation-sémantique}\end{entrée}

\begin{entrée}
\vedette{\hypertarget{Ⓔrɤcɤβ}{\papi{ rɤcɤβ}}}\markboth{rɤcɤβ}{}\classe{vi}
\paradigme{\textit{dir :} \jya thɯ-}
\begin{définition}\fra pousser des cosses\end{définition}
\begin{définition}\cmn 结荚果\end{définition}
\begin{relation-sémantique}\confer{
\hyperlink{Ⓔɯ-cɤβ}{\textit{ \papi{ɯ-cɤβ}}}
}\end{relation-sémantique}\end{entrée}

\begin{entrée}
\vedette{\hypertarget{Ⓔrɤɕar}{\papi{ rɤɕar}}}\markboth{rɤɕar}{}
\begin{relation-sémantique}\confer{
\hyperlink{Ⓔɕar}{\textit{ \papi{ɕar}}}
}\end{relation-sémantique}\end{entrée}

\begin{entrée}
\vedette{\hypertarget{Ⓔrɤɕi}{\papi{ rɤɕi}}}\markboth{rɤɕi}{} (\variante{rɤɕit}) 
\classe{vt}
\paradigme{\textit{dir :} \jya \_}
\begin{définition}\fra tirer\end{définition}
\begin{définition}\cmn 拉\end{définition}
\begin{exemple}\jya tɯmbri kɤ-rɤɕi\cmn 你拉一下绳子吧\end{exemple}
\begin{exemple}\jya a-jaʁ kɤ-rɤɕi\cmn 你拉我的手吧\end{exemple}
\begin{exemple}\jya nɯ-rɤɕi-t-a\cmn 我拉了\end{exemple}
\begin{exemple}\jya tɯrɟaʁ kɯ-rɲɟɯ-rɲɟi ʑo ko-rɤɕi-nɯ (ko-mtshi-nɯ)\cmn 他们跳舞的队伍拉得很长\end{exemple}\begin{sous-entrée}
\vedette{\hypertarget{}{\papi{ ʑɣɤrɤɕi}}}\markboth{ʑɣɤrɤɕi}{}\classe{vi}
\begin{définition}\ 
\begin{déclaration}\grammar{refl}\end{déclaration}\end{définition}
\begin{définition}\fra se défaire de\end{définition}
\begin{définition}\cmn 摆脱,挣脱\end{définition}
\begin{exemple}\jya a-jaʁ ka-ndo ri, kɤ-ʑɣɤrɤɕi-a tɕe kɤ-sɯɕlɯɣ-a\cmn 他抓住了我的手,我就挣脱了,让他松手了\end{exemple}
\end{sous-entrée}\end{entrée}

\begin{entrée}
\vedette{\hypertarget{Ⓔrɤɕkho}{\papi{ rɤɕkho}}}\markboth{rɤɕkho}{}
\begin{relation-sémantique}\confer{
\hyperlink{Ⓔɕkho}{\textit{ \papi{ɕkho}}}
}\end{relation-sémantique}\end{entrée}

\begin{entrée}
\vedette{\hypertarget{Ⓔrɤɕom}{\papi{ rɤɕom}}}\markboth{rɤɕom}{}
\classe{vi}
\paradigme{\textit{dir :} \jya kɤ-}
\begin{définition}\ 
\begin{déclaration}\grammar{denom}\end{déclaration}\end{définition}
\begin{définition}\fra apparaître (peau du lait)\end{définition}
\begin{définition}\cmn 结奶皮\end{définition}
\begin{exemple}\jya tɤ-lu ko-rɤɕom\cmn 牛奶结了奶皮\end{exemple}
\begin{relation-sémantique}\confer{
\hyperlink{ⒺɕomⒽ2}{\textit{ \papi{ɕom2}}}
}\end{relation-sémantique}\end{entrée}

\begin{entrée}
\vedette{\hypertarget{Ⓔrɤɕon}{\papi{ rɤɕon}}}\markboth{rɤɕon}{}\classe{vi}
\paradigme{\textit{dir :} \jya tɤ-}
\begin{définition}\fra porter témoigner\end{définition}
\begin{définition}\cmn 作证\end{définition}
\begin{exemple}\jya tu-nɯzdɯɣ mɤ-ra ma aʑo tu-rɤɕon-a jɤɣ\cmn 你不用担心(他冤枉你),我可以给你作证\end{exemple}\end{entrée}

\begin{entrée}
\vedette{\hypertarget{Ⓔrɤɕphɤt}{\papi{ rɤɕphɤt}}}\markboth{rɤɕphɤt}{}
\begin{relation-sémantique}\confer{
\hyperlink{Ⓔɕphɤt}{\textit{ \papi{ɕphɤt}}}
}\end{relation-sémantique}\end{entrée}

\begin{entrée}
\vedette{\hypertarget{Ⓔrɤɕtʂat}{\papi{ rɤɕtʂat}}}\markboth{rɤɕtʂat}{}
\begin{relation-sémantique}\confer{
\hyperlink{Ⓔɕtʂat}{\textit{ \papi{ɕtʂat}}}
}\end{relation-sémantique}\end{entrée}

\begin{entrée}
\vedette{\hypertarget{Ⓔrɤɕtʂo}{\papi{ rɤɕtʂo}}}\markboth{rɤɕtʂo}{}
\begin{relation-sémantique}\confer{
\hyperlink{Ⓔɕtʂo}{\textit{ \papi{ɕtʂo}}}
}\end{relation-sémantique}\end{entrée}

\begin{entrée}
\vedette{\hypertarget{Ⓔrɤɕtʂɯ}{\papi{ rɤɕtʂɯ}}}\markboth{rɤɕtʂɯ}{}
\begin{relation-sémantique}\confer{
\hyperlink{Ⓔɕtʂɯ}{\textit{ \papi{ɕtʂɯ}}}
}\end{relation-sémantique}\end{entrée}

\begin{entrée}
\vedette{\hypertarget{Ⓔrɤfcɤr}{\papi{ rɤfcɤr}}}\markboth{rɤfcɤr}{}\classe{vi}
\paradigme{\textit{dir :} \jya tɤ-}
\begin{définition}\ 
\begin{déclaration}\grammar{denom}\end{déclaration}\end{définition}
\begin{définition}\fra faire de la poterie\end{définition}
\begin{définition}\cmn 做泥工\end{définition}
\begin{exemple}\jya aʑo tu-rɤfcar-a ɲɯ-sɯsam-a ma jinde kɯ-rɤfcaʁ maŋe\cmn 我想做泥工因为现在没有泥匠\end{exemple}
\begin{relation-sémantique}\confer{
\hyperlink{Ⓔtɯfcɤr}{\textit{ \papi{tɯfcɤr}}}
}\end{relation-sémantique}\end{entrée}

\begin{entrée}
\vedette{\hypertarget{Ⓔrɤfɕɤt}{\papi{ rɤfɕɤt}}}\markboth{rɤfɕɤt}{}
\begin{relation-sémantique}\confer{
\hyperlink{ⒺfɕɤtⒽ1}{\textit{ \papi{fɕɤt1}}}
}\end{relation-sémantique}\end{entrée}

\begin{entrée}
\vedette{\hypertarget{Ⓔrɤfɕɯfɕɤt}{\papi{ rɤfɕɯfɕɤt}}}\markboth{rɤfɕɯfɕɤt}{}
\classe{vt}
\paradigme{\textit{dir :} \jya thɯ-}
\begin{définition}\fra déchirer dans tous les sens\end{définition}
\begin{définition}\cmn 乱撕\end{définition}
\begin{exemple}\jya ɯ-ŋga chɤ-rɤfɕɯfɕɤt\cmn 他乱撕了他的衣服\end{exemple}
\begin{exemple}\jya ɯ-@benzi chɤ-rɤfɕɯfɕɤt\cmn 他乱撕了他的本子\end{exemple}
\begin{exemple}\jya tha-rɤfɕɯfɕɤt\cmn 他乱撕了\end{exemple}\end{entrée}

\begin{entrée}
\vedette{\hypertarget{Ⓔrɤfse}{\papi{ rɤfse}}}\markboth{rɤfse}{}
\begin{relation-sémantique}\confer{
\hyperlink{ⒺfseⒽ2}{\textit{ \papi{fse2}}}
}\end{relation-sémantique}\end{entrée}

\begin{entrée}
\vedette{\hypertarget{Ⓔrɤfsjit}{\papi{ rɤfsjit}}}\markboth{rɤfsjit}{}\classe{vi}
\paradigme{\textit{dir :} \jya thɯ-}
\begin{définition}\fra siffler\end{définition}
\begin{définition}\cmn 吹口哨\end{définition}
\begin{exemple}\jya ɯʑo ci thɯ-rɤfsjit\cmn 他吹了口哨\end{exemple}
\begin{relation-sémantique}\confer{
\hyperlink{Ⓔtɤfsjit}{\textit{ \papi{tɤfsjit}}}
}\end{relation-sémantique}\end{entrée}

\begin{entrée}
\vedette{\hypertarget{Ⓔrɤfsoʁ}{\papi{ rɤfsoʁ}}}\markboth{rɤfsoʁ}{}
\begin{relation-sémantique}\confer{
\hyperlink{ⒺfsoʁⒽ1}{\textit{ \papi{fsoʁ1}}}
}\end{relation-sémantique}\end{entrée}

\begin{entrée}
\vedette{\hypertarget{Ⓔrɤftɕɤz}{\papi{ rɤftɕɤz}}}\markboth{rɤftɕɤz}{}
\begin{relation-sémantique}\confer{
\hyperlink{Ⓔftɕɤz}{\textit{ \papi{ftɕɤz}}}
}\end{relation-sémantique}\end{entrée}

\begin{entrée}
\vedette{\hypertarget{Ⓔrɤɣdɤt}{\papi{ rɤɣdɤt}}}\markboth{rɤɣdɤt}{}
\classe{vt}
\paradigme{\textit{dir :} \jya pɯ-}
\begin{définition}\fra découper en sections\end{définition}
\begin{définition}\cmn 砍;锯成一段一段\end{définition}
\begin{exemple}\jya si pa-rɤɣdɤt\cmn 他把木头锯成了几段\end{exemple}
\begin{exemple}\jya ɕoŋtɕa pa-rɤɣdɤt\cmn 他把木料锯成了几段\end{exemple}
\begin{exemple}\jya ɕom pɯ-rɤɣdat-a\cmn 我把铁锯了几段\end{exemple}
\begin{exemple}\jya tɯmbri pɯ-rɤɣdat-a\cmn 我把绳子剪了几段\end{exemple}
\begin{relation-sémantique}\confer{
\hyperlink{Ⓔrɤrzɯɣ}{\textit{ \papi{rɤrzɯɣ}}}
}\end{relation-sémantique}\end{entrée}

\begin{entrée}
\vedette{\hypertarget{Ⓔrɤɣdɯt}{\papi{ rɤɣdɯt}}}\markboth{rɤɣdɯt}{}\classe{vt}
\paradigme{\textit{dir :} \jya thɯ-}
\paradigme{\textit{dir :} \jya nɯ-}
\begin{définition}\fra écorcher\end{définition}
\begin{définition}\cmn 剥皮(保持完整的皮子)
\begin{déclaration}\use{可以用干草填塞动物驱壳,保存动物原来的形状}\end{déclaration}\end{définition}
\begin{exemple}\jya thɯ-rɤɣdɯt-a, tha-rɤɣdɯt\cmn 我剥了皮、他剥了皮\end{exemple}
\begin{exemple}\jya qala tha-rɤɣdɯt\cmn 他剥了兔子的皮\end{exemple}
\begin{exemple}\jya tshɤt qaʑo tha-rɤɣdɯt\cmn 他剥了羊的皮\end{exemple}
\begin{exemple}\jya xɕiri kɯ kumpɣa ɲɤ-rɤɣdɯt\cmn 黄鼠狼喝光了鸡的血(还没有吃到肉)\end{exemple}
\begin{exemple}\jya tɯrme kɯ tu-kɤ-ntɕha nɯnɯ, ɯ-rqhu mɯ-tu-kɤ-phaʁ nɯ, ɯ-ndʐi mɯ-tu-kɤ-phaʁ nɯ tɕe, chɤ-rɤɣdɯt tu-kɯ-ti ɲɯ-ŋu. xɕiri kɯ kumpɣa ta-ndza tɕe, ɯ-se ku-tshi ma, ɯ-ɕa mɯ-tu-ndze ɲɯ-βde nɯnɯ, li ɲɤ-rɤɣdɯt tu-kɯ-ti khɯ.\cmn 
宰动物的时候,不破皮子地剥皮叫做\stylefv{chɤrɤɣdɯt}。黄鼠狼喝鸡的血,不吃它的肉也叫做\stylefv{ɲɤrɤɣdɯt}
\end{exemple}\end{entrée}

\begin{entrée}
\vedette{\hypertarget{Ⓔrɤɣlɤn}{\papi{ rɤɣlɤn}}}\markboth{rɤɣlɤn}{}
\classe{vt}
\paradigme{\textit{dir :} \jya tɤ-}
\begin{définition}\ 
\begin{déclaration}\grammar{denom}\end{déclaration}\end{définition}
\begin{définition}\fra prendre pour prétexte\end{définition}
\begin{définition}\cmn 拿为借口
\begin{déclaration} \étymologie{\papi{len}}\end{déclaration}\end{définition}
\begin{exemple}\jya nɯ aʑo tu-kɯ-rɤɣlan-a ɲɯ-ŋu\cmn 你在怪我\end{exemple}
\begin{exemple}\jya nɤʑo tu-tɯ́-wɣ-ɲɯ-ŋu\cmn 他在怪你\end{exemple}
\begin{exemple}\jya nɤʑo ɲɤ-tɯ-nɯβde-t, aj tu-kɯ-rɤɣlan-a ɲɯ-ŋu\cmn 你弄丢了还怪我\end{exemple}\end{entrée}

\begin{entrée}
\vedette{\hypertarget{Ⓔrɤɣndi}{\papi{ rɤɣndi}}}\markboth{rɤɣndi}{}
\classe{vt}
\paradigme{\textit{dir :} \jya thɯ-}
\paradigme{\textit{dir :} \jya pɯ-}
\paradigme{\textit{dir :} \jya \_}
\begin{définition}\fra bourrer\end{définition}
\begin{définition}\cmn 硬塞\end{définition}
\begin{exemple}\jya tɤ-fkɯm ɯ-ŋgɯ tha-rɤɣndi\cmn 他硬塞进口袋里\end{exemple}
\begin{exemple}\jya kɯ-spoʁ ɯ-ŋgɯ pa-rɤɣndi\cmn 他硬塞进洞里\end{exemple}
\begin{exemple}\jya tɤ-fkɯm ɯ-ŋgɯ tɯ-ŋga thɯ-rɤɣndi-t-a\cmn 我把衣服硬塞进口袋里\end{exemple}\end{entrée}

\begin{entrée}
\vedette{\hypertarget{Ⓔrɤɣo}{\papi{ rɤɣo}}}\markboth{rɤɣo}{}\classe{n}
\begin{définition}\fra chanson\end{définition}
\begin{définition}\cmn 歌\end{définition}
\begin{exemple}\jya rɤɣo thɯ-βzu-t-a\cmn 我唱了歌\end{exemple}
\begin{exemple}\jya rɤɣo pɯ-lat-a\cmn 我演奏了音乐\end{exemple}
\begin{exemple}\jya rɤɣo ci thɯ-tɯt-a\cmn 我唱了一首歌\end{exemple}
\begin{exemple}\jya a-rɤɣo ci thɯ-βze\cmn 给我唱一首歌吧!\end{exemple}
\begin{relation-sémantique}\confer{
\hyperlink{Ⓔnɯrɤɣo}{\textit{ \papi{nɯrɤɣo}}}
}\end{relation-sémantique}\end{entrée}

\begin{entrée}
\vedette{\hypertarget{Ⓔrɤɣrɯ}{\papi{ rɤɣrɯ}}}\markboth{rɤɣrɯ}{}
\classe{vi}
\paradigme{\textit{dir :} \jya pɯ-}
\begin{définition}\fra traiter une douleur en appliquant un objet chaud\end{définition}
\begin{définition}\cmn 用热的东西治关节炎\end{définition}
\begin{exemple}\jya χtɕoŋ kɯ-tu nɯnɯ, kɯɕpaz ɯ-tʂɤm tɯ-χpɯm ɲɯ́-wɣ-mar tɕe, chɯ́-wɣ-ɣɤmpja tɕe kɤ-rɤɣrɯ kɤ-ti ɲɯ-ŋu\cmn 关节炎患者在膝盖上涂旱獭油发热,这是治疗关节炎的方法\end{exemple}\begin{sous-entrée}
\vedette{\hypertarget{}{\papi{ zrɤɣrɯ}}}\markboth{zrɤɣrɯ}{}\classe{vt}
\begin{définition}\fra appliquer un objet chaud sur une articulation\end{définition}
\begin{définition}\cmn 用热的东西敷身体的某个关节治关节炎\end{définition}
\begin{exemple}\jya a-χpɯm pɯ-z-rɤɣrɯ-t-a\cmn 我用热的东西敷了膝盖\end{exemple}
\end{sous-entrée}\end{entrée}

\begin{entrée}
\vedette{\hypertarget{Ⓔrɤji}{\papi{ rɤji}}}\markboth{rɤji}{}\classe{vi}
\paradigme{\textit{dir :} \jya lɤ-}
\paradigme{\textit{dir :} \jya pɯ-}
\begin{définition}\ 
\begin{déclaration}\grammar{apass}\end{déclaration}\end{définition}
\begin{définition}\fra planter, semer\end{définition}
\begin{définition}\cmn 播种\end{définition}
\begin{exemple}\jya aʑo pɯ-rɤji-a\cmn 我播种了\end{exemple}
\begin{relation-sémantique}\confer{
\hyperlink{Ⓔji}{\textit{ \papi{ji}}}
}\end{relation-sémantique}\end{entrée}

\begin{entrée}
\vedette{\hypertarget{Ⓔrɤjla}{\papi{ rɤjla}}}\markboth{rɤjla}{}
\classe{n}
\begin{définition}\fra habit d'homme en lin\end{définition}
\begin{définition}\cmn 布制成的男装
\begin{déclaration} \étymologie{\papi{ras + lwa.ba}}\end{déclaration}\end{définition}\end{entrée}

\begin{entrée}
\vedette{\hypertarget{Ⓔrɤjndoʁ}{\papi{ rɤjndoʁ}}}\markboth{rɤjndoʁ}{}
\classe{n}
\begin{définition}\fra navet\end{définition}
\begin{définition}\cmn 芜菁根\end{définition}
\begin{relation-sémantique}\confer{
\hyperlink{Ⓔɕaʁwɯ}{\textit{ \papi{ɕaʁwɯ}}}
}\end{relation-sémantique}
\begin{relation-sémantique}\confer{
\hyperlink{Ⓔlaβzɣi}{\textit{ \papi{laβzɣi}}}
}\end{relation-sémantique}
\begin{relation-sémantique}\confer{
\hyperlink{Ⓔkamda}{\textit{ \papi{kamda}}}
}\end{relation-sémantique}
\begin{relation-sémantique}\confer{
\hyperlink{Ⓔrgawɯ}{\textit{ \papi{rgawɯ}}}
}\end{relation-sémantique}
\begin{relation-sémantique}\confer{
\hyperlink{Ⓔrasti}{\textit{ \papi{rasti}}}
}\end{relation-sémantique}\end{entrée}

\begin{entrée}
\vedette{\hypertarget{Ⓔrɤjoʁβzɯr}{\papi{ rɤjoʁβzɯr}}}\markboth{rɤjoʁβzɯr}{}
\classe{vt}
\paradigme{\textit{dir :} \jya tɤ-}
\begin{définition}\ 
\begin{déclaration}\grammar{comp}\end{déclaration}\end{définition}
\begin{définition}\fra débarrasser, ranger une pièce\end{définition}
\begin{définition}\cmn 收拾;弄整齐\end{définition}
\begin{exemple}\jya laχtɕha tɤ-rɤjoʁβzɯr\cmn 你收拾一下东西\end{exemple}
\begin{exemple}\jya nɯ fse a-mɤ-pɯ-ɤnɯta, tɤ-rɤjoʁβzɯr\cmn 东西不能这样放着,你收拾一下\end{exemple}
\begin{relation-sémantique}\confer{
\hyperlink{Ⓔjoʁ}{\textit{ \papi{joʁ}}}
}\end{relation-sémantique}
\begin{relation-sémantique}\confer{
\hyperlink{Ⓔβzɯr}{\textit{ \papi{βzɯr}}}
}\end{relation-sémantique}
\begin{relation-sémantique}\confer{
\hyperlink{Ⓔjoʁβzɯr}{\textit{ \papi{joʁβzɯr}}}
}\end{relation-sémantique}\end{entrée}

\begin{entrée}
\vedette{\hypertarget{Ⓔrɤjroʁ}{\papi{ rɤjroʁ}}}\markboth{rɤjroʁ}{}
\classe{vi}
\paradigme{\textit{dir :} \jya \_}\acception{1}
\begin{définition}\fra laissant de longues traces\end{définition}
\begin{définition}\cmn 留下长条的痕迹\end{définition}\acception{2}
\begin{définition}\fra ayant des rayures\end{définition}
\begin{définition}\cmn 有纹路\end{définition}\acception{3}
\paradigme{\textit{dir :} \jya pɯ-}
\begin{définition}\fra (avoir de la morve) qui pend au nez\end{définition}
\begin{définition}\cmn 挂着(鼻涕)\end{définition}
\begin{exemple}\jya ɯ-ɕnaβ ra pjɯ-rɤjroʁ ʑo\cmn 他挂着鼻涕\end{exemple}\begin{sous-entrée}
\vedette{\hypertarget{}{\papi{ zrɤjroʁ}}}\markboth{zrɤjroʁ}{}\classe{vt}
\begin{définition}\fra laisser une trace\end{définition}
\begin{définition}\cmn 留下纹路\end{définition}
\begin{relation-sémantique}\confer{
\hyperlink{Ⓔtɯ-jroʁ}{\textit{ \papi{tɯ-jroʁ}}}
}\end{relation-sémantique}
\end{sous-entrée}\end{entrée}

\begin{entrée}
\vedette{\hypertarget{Ⓔrɤjtshi}{\papi{ rɤjtshi}}}\markboth{rɤjtshi}{}
\begin{relation-sémantique}\confer{
\hyperlink{Ⓔjtshi}{\textit{ \papi{jtshi}}}
}\end{relation-sémantique}\end{entrée}

\begin{entrée}
\vedette{\hypertarget{Ⓔrɤjɯɣ}{\papi{ rɤjɯɣ}}}\markboth{rɤjɯɣ}{}
\classe{vt}
\paradigme{\textit{dir :} \jya lɤ-}
\begin{définition}\fra faire de la ficelle en roulant dans les mains\end{définition}
\begin{définition}\cmn 搓线(用手)
\end{définition}
\begin{exemple}\jya tɯ-ŋgru lu-kɤ-rɤjɯɣ\cmn 搓成线的牛筋\end{exemple}
\begin{relation-sémantique}\synonyme{
\hyperlink{Ⓔpɣo}{\textit{ \papi{pɣo}}}
}\end{relation-sémantique}
\begin{relation-sémantique}\synonyme{
\hyperlink{Ⓔrɯm}{\textit{ \papi{rɯm}}}
}\end{relation-sémantique}\end{entrée}

\begin{entrée}
\vedette{\hypertarget{Ⓔrɤjwaʁ}{\papi{ rɤjwaʁ}}}\markboth{rɤjwaʁ}{}\classe{vs}
\paradigme{\textit{dir :} \jya nɯ-}
\paradigme{\textit{dir :} \jya tɤ-}
\begin{définition}\fra pousser des feuilles\end{définition}
\begin{définition}\cmn 长出叶子\end{définition}
\begin{exemple}\jya χɕitka jɤ-ɣe tɕe, sɯku ɲɯ-rɤjwaʁ ɲɯ-ŋu\cmn 到了春天,树长出叶子\end{exemple}
\begin{relation-sémantique}\confer{
\hyperlink{Ⓔtɤ-jwaʁ}{\textit{ \papi{tɤ-jwaʁ}}}
}\end{relation-sémantique}
\begin{relation-sémantique}\confer{
\hyperlink{Ⓔɣɤjwaʁ}{\textit{ \papi{ɣɤjwaʁ}}}
}\end{relation-sémantique}\end{entrée}

\begin{entrée}
\vedette{\hypertarget{Ⓔrɤɟar}{\papi{ rɤɟar}}}\markboth{rɤɟar}{}
\begin{relation-sémantique}\confer{
\hyperlink{Ⓔɟar}{\textit{ \papi{ɟar}}}
}\end{relation-sémantique}
\end{entrée}

\begin{entrée}
\vedette{\hypertarget{Ⓔrɤɟom}{\papi{ rɤɟom}}}\markboth{rɤɟom}{}\classe{vt}
\paradigme{\textit{dir :} \jya \_}
\begin{définition}\ 
\begin{déclaration}\grammar{denom}\end{déclaration}\end{définition}
\begin{définition}\fra mesurer\end{définition}
\begin{définition}\cmn 一排一排地量\end{définition}
\begin{exemple}\jya tɯmbri nɯ-rɤɟom-a\cmn 我把绳子一排一排地量了一下\end{exemple}
\begin{exemple}\jya si nɯ-rɤɟom-a\cmn 我把木头一排一排地量了一下\end{exemple}
\begin{relation-sémantique}\confer{
\hyperlink{Ⓔtɯ-ɟom}{\textit{ \papi{tɯ-ɟom}}}
}\end{relation-sémantique}\end{entrée}

\begin{entrée}
\vedette{\hypertarget{Ⓔrɤkha}{\papi{ rɤkha}}}\markboth{rɤkha}{}
\classe{vi}
\paradigme{\textit{dir :} \jya tɤ-}
\begin{définition}\fra construire une maison\end{définition}
\begin{définition}\cmn 修房子\end{définition}
\begin{exemple}\jya aʑo tu-nɯ-rɤkha-a ŋu\cmn 我修(自己的)房子\end{exemple}
\begin{relation-sémantique}\confer{
\hyperlink{Ⓔkha}{\textit{ \papi{kha}}}
}\end{relation-sémantique}\end{entrée}

\begin{entrée}
\vedette{\hypertarget{Ⓔrɤkhɯkhrɤt}{\papi{ rɤkhɯkhrɤt}}}\markboth{rɤkhɯkhrɤt}{}
\begin{relation-sémantique}\confer{
\hyperlink{ⒺkhrɤtⒽ1}{\textit{ \papi{khrɤt1}}}
}\end{relation-sémantique}\end{entrée}

\begin{entrée}
\vedette{\hypertarget{Ⓔrɤkrɤz}{\papi{ rɤkrɤz}}}\markboth{rɤkrɤz}{}
\classe{vi}
\paradigme{\textit{dir :} \jya tɤ-}
\begin{définition}\ 
\begin{déclaration}\grammar{denom}\end{déclaration}\end{définition}
\begin{définition}\fra discuter\end{définition}
\begin{définition}\cmn 商量
\begin{déclaration} \étymologie{\papi{gros}}\end{déclaration}\end{définition}
\begin{exemple}\jya tɤ-rɤkrɤz-tɕi\cmn 我们俩商量过了\end{exemple}
\begin{exemple}\jya khro tɤ-rɤkrɤz-nɯ\cmn 他们商量了很久\end{exemple}
\begin{relation-sémantique}\confer{
\hyperlink{Ⓔtɯkrɤz}{\textit{ \papi{tɯkrɤz}}}
}\end{relation-sémantique}
\begin{relation-sémantique}\confer{
\hyperlink{Ⓔnɯkrɤz}{\textit{ \papi{nɯkrɤz}}}
}\end{relation-sémantique}\end{entrée}

\begin{entrée}
\vedette{\hypertarget{Ⓔrɤkro}{\papi{ rɤkro}}}\markboth{rɤkro}{}
\begin{relation-sémantique}\confer{
\hyperlink{Ⓔkro}{\textit{ \papi{kro}}}
}\end{relation-sémantique}
\end{entrée}

\begin{entrée}
\vedette{\hypertarget{Ⓔrɤkrɯ}{\papi{ rɤkrɯ}}}\markboth{rɤkrɯ}{}\classe{vt}
\paradigme{\textit{dir :} \jya pɯ-}
\paradigme{\textit{dir :} \jya thɯ-}
\begin{définition}\fra couper\end{définition}
\begin{définition}\cmn 切\end{définition}
\begin{exemple}\jya si thɯ-rɤkrɯ-t-a\cmn 我切了木头\end{exemple}
\begin{exemple}\jya yangyu pɯ-rɤkrɯ-t-a\cmn 我切了洋芋\end{exemple}
\begin{exemple}\jya tɤ-mthɯm pɯ-rɤkrɯ-t-a\cmn 我切了肉\end{exemple}
\begin{relation-sémantique}\confer{
\hyperlink{Ⓔkɯkrɯ}{\textit{ \papi{kɯkrɯ}}}
}\end{relation-sémantique}\begin{sous-entrée}
\vedette{\hypertarget{}{\papi{ zrɤkrɯ}}}\markboth{zrɤkrɯ}{}\classe{vt}
\begin{définition}\ 
\begin{déclaration}\grammar{caus}\end{déclaration}\end{définition}
\begin{définition}\fra couper avec\end{définition}
\begin{définition}\cmn 用……切\end{définition}
\begin{exemple}\jya ɯ-ɕɣa kɯ-tu nɯ kɯ pjɯ́-wɣ-z-rɤkrɯ kɤ-ti ɕti\cmn 有刀刃的东西都可以用来切\end{exemple}
\end{sous-entrée}\end{entrée}

\begin{entrée}
\vedette{\hypertarget{Ⓔrɤkɯkrɯ}{\papi{ rɤkɯkrɯ}}}\markboth{rɤkɯkrɯ}{}\classe{vt}
\paradigme{\textit{dir :} \jya pɯ-}
\begin{définition}\fra découper en morceaux\end{définition}
\begin{définition}\cmn 切成几块\end{définition}
\begin{relation-sémantique}\confer{
\hyperlink{Ⓔkɯkrɯ}{\textit{ \papi{kɯkrɯ}}}
}\end{relation-sémantique}
\begin{relation-sémantique}\confer{
\hyperlink{Ⓔrɤkrɯ}{\textit{ \papi{rɤkrɯ}}}
}\end{relation-sémantique}\end{entrée}

\begin{entrée}
\vedette{\hypertarget{Ⓔrɤlaj}{\papi{ rɤlaj}}}\markboth{rɤlaj}{}
\classe{vt}
\paradigme{\textit{dir :} \jya pɯ-}
\begin{définition}\fra pétrir la pâte\end{définition}
\begin{définition}\cmn 揉面;挼\end{définition}
\begin{exemple}\jya tɤjlu pɯ-rɤlaj-a\cmn 我揉了面\end{exemple}
\begin{exemple}\jya tɤrcoʁ pɯ-rɤlaj-a\cmn 我揉了泥土\end{exemple}
\begin{exemple}\jya tɤjlu pjɯ́-wɣ-rɤlaj tɕe mɯm\cmn 面要揉才好吃\end{exemple}\end{entrée}

\begin{entrée}
\vedette{\hypertarget{Ⓔrɤlɤt}{\papi{ rɤlɤt}}}\markboth{rɤlɤt}{}\classe{vi}
\begin{définition}\fra mouler, fondre (le métal)\end{définition}
\begin{définition}\cmn 铸造\end{définition}
\begin{relation-sémantique}\confer{
\hyperlink{ⒺlɤtⒽ1}{\textit{ \papi{lɤt}}}
}\end{relation-sémantique}\end{entrée}

\begin{entrée}
\vedette{\hypertarget{Ⓔrɤli}{\papi{ rɤli}}}\markboth{rɤli}{}
\classe{vt}
\paradigme{\textit{dir :} \jya tɤ-}
\begin{définition}\fra dédommager\end{définition}
\begin{définition}\cmn 赔偿\end{définition}
\begin{exemple}\jya a-laχtɕha ɲɤ-tɯ-βde-t, tɤ-rɤli\cmn 你把我的东西弄丢了,你要赔!\end{exemple}
\begin{exemple}\jya ɯ-laχtɕha ɲɤ-nɯβde-t-a, tɤ-rɤli-t-a\cmn 我把他的东西弄丢了,我给他赔了\end{exemple}
\begin{relation-sémantique}\synonyme{
\hyperlink{Ⓔnɯmbe}{\textit{ \papi{nɯmbe}}}
}\end{relation-sémantique}\end{entrée}

\begin{entrée}
\vedette{\hypertarget{Ⓔrɤluj}{\papi{ rɤluj}}}\markboth{rɤluj}{}\classe{vs}
\paradigme{\textit{dir :} \jya tɤ-}
\begin{définition}\fra pleurer sans cesse (enfant)\end{définition}
\begin{définition}\cmn 小孩子不停的哭喊;撒娇\end{définition}
\end{entrée}

\begin{entrée}
\vedette{\hypertarget{Ⓔrɤlkɯɣ}{\papi{ rɤlkɯɣ}}}\markboth{rɤlkɯɣ}{}\classe{vt}
\paradigme{\textit{dir :} \jya tɤ-}
\begin{définition}\fra enrouler en cercle\end{définition}
\begin{définition}\cmn 卷成一圈\end{définition}
\begin{exemple}\jya tɯmbri tɤ-rɤlkɯɣ-a\cmn 我把绳子卷成一圈一圈\end{exemple}
\begin{relation-sémantique}\confer{
\hyperlink{Ⓔtɯ-tɤlkɯɣ}{\textit{ \papi{tɯ-tɤlkɯɣ}}}
}\end{relation-sémantique}
\begin{relation-sémantique}\synonyme{
\hyperlink{Ⓔzɣɯrkɯrkɯ}{\textit{ \papi{zɣɯrkɯrkɯ}}}
}\end{relation-sémantique}
\begin{relation-sémantique}\synonyme{
\hyperlink{Ⓔrɤstɯm}{\textit{ \papi{rɤstɯm}}}
}\end{relation-sémantique}\end{entrée}

\begin{entrée}
\vedette{\hypertarget{Ⓔrɤloʁ}{\papi{ rɤloʁ}}}\markboth{rɤloʁ}{}
\classe{vi}
\paradigme{\textit{dir :} \jya kɤ-}
\begin{définition}\ 
\begin{déclaration}\grammar{denom}\end{déclaration}\end{définition}
\begin{définition}\fra faire un nid\end{définition}
\begin{définition}\cmn 打窝\end{définition}
\begin{exemple}\jya qajdo ɲɯ-rɤloʁ\cmn 乌鸦在做巢\end{exemple}
\begin{exemple}\jya paʁ ɲɯ-rɤloʁ\cmn 猪在做窝\end{exemple}
\begin{relation-sémantique}\confer{
\hyperlink{Ⓔtɤ-loʁⒽ1}{\textit{ \papi{tɤ-loʁ1}}}
}\end{relation-sémantique}\end{entrée}

\begin{entrée}
\vedette{\hypertarget{Ⓔrɤma}{\papi{ rɤma}}}\markboth{rɤma}{}
\classe{vi}
\paradigme{\textit{dir :} \jya tɤ-}
\paradigme{\textit{dir :} \jya pɯ-}
\begin{définition}\fra travailler\end{définition}
\begin{définition}\cmn 劳动\end{définition}
\begin{exemple}\jya jisŋi ɕ-pɯ-rɤma-a\cmn 我今天去劳动了\end{exemple}
\begin{relation-sémantique}\confer{
\hyperlink{Ⓔta-ma}{\textit{ \papi{ta-ma}}}
}\end{relation-sémantique}
\begin{relation-sémantique}\confer{
\hyperlink{Ⓔnɤma}{\textit{ \papi{nɤma}}}
}\end{relation-sémantique}\begin{sous-entrée}
\vedette{\hypertarget{}{\papi{ zrɤma}}}\markboth{zrɤma}{}\classe{vt}
\begin{définition}\ 
\begin{déclaration}\grammar{caus}\end{déclaration}\end{définition}
\begin{définition}\fra faire travailler\end{définition}
\begin{définition}\cmn 让……工作\end{définition}
\end{sous-entrée}\end{entrée}

\begin{entrée}
\vedette{\hypertarget{Ⓔrɤmat}{\papi{ rɤmat}}}\markboth{rɤmat}{}
\classe{vi}
\paradigme{\textit{dir :} \jya thɯ-}
\begin{définition}\ 
\begin{déclaration}\grammar{denom}\end{déclaration}\end{définition}
\begin{définition}\fra faire des fruits\end{définition}
\begin{définition}\cmn 结果子\end{définition}
\begin{exemple}\jya jiɕqha nɯ cho-rɤmat\cmn 那个结了果\end{exemple}
\begin{relation-sémantique}\confer{
\hyperlink{Ⓔɯ-mat}{\textit{ \papi{ɯ-mat}}}
}\end{relation-sémantique}\end{entrée}

\begin{entrée}
\vedette{\hypertarget{Ⓔrɤmbi}{\papi{ rɤmbi}}}\markboth{rɤmbi}{}
\classe{vi}
\paradigme{\textit{dir :} \jya nɯ-}
\begin{définition}\ 
\begin{déclaration}\grammar{apass}\end{déclaration}\end{définition}
\begin{définition}\fra donner à quelqu'un\end{définition}
\begin{définition}\cmn 给别人\end{définition}
\begin{exemple}\jya @yangyu nɯ-rɤmbi-a\cmn 我给了洋芋\end{exemple}
\begin{exemple}\jya stoʁ nɯ-rɤmbi-a\cmn 我给了胡豆\end{exemple}
\begin{exemple}\jya nɯ-rɤmbi-j ma khɯnɤmu ɲɯ-ɕti tɕe a-mɤ-thɯ-rɤpɯ ma mɯ́j-saχɕɯn\cmn 我们把它拿去送人了。因为是母狗,生崽的话很不卫生。\end{exemple}
\begin{relation-sémantique}\confer{
\hyperlink{Ⓔmbi}{\textit{ \papi{mbi}}}
}\end{relation-sémantique}\end{entrée}

\begin{entrée}
\vedette{\hypertarget{Ⓔrɤmboʁɲɟi}{\papi{ rɤmboʁɲɟi}}}\markboth{rɤmboʁɲɟi}{}\classe{vt}
\paradigme{\textit{dir :} \jya nɯ-}
\begin{définition}\fra écraser en petits morceaux\end{définition}
\begin{définition}\cmn 弄碎(搓成碎片)\end{définition}
\begin{relation-sémantique}\confer{
\hyperlink{Ⓔrɤɲɟiɲɟi}{\textit{ \papi{rɤɲɟiɲɟi}}}
}\end{relation-sémantique}
\begin{relation-sémantique}\confer{
\hyperlink{Ⓔrɤɲɟoʁɲɟi}{\textit{ \papi{rɤɲɟoʁɲɟi}}}
}\end{relation-sémantique}\end{entrée}

\begin{entrée}
\vedette{\hypertarget{Ⓔrɤmbɯmbri}{\papi{ rɤmbɯmbri}}}\markboth{rɤmbɯmbri}{}\classe{vt}
\paradigme{\textit{dir :} \jya tɤ-}
\begin{définition}\fra perdre au fil du chemin\end{définition}
\begin{définition}\cmn 一边走路一边撒下\end{définition}
\begin{exemple}\jya tɤ-fkɯm pjɤ-spoʁ tɕe, ɯ-ŋgɯ kɯ-tu nɯra to-tɯ-rɤmbɯmbri-t\cmn 袋子有洞,所以把里面的东西撒得一路都是\end{exemple}
\begin{relation-sémantique}\confer{
\hyperlink{Ⓔarɤmbɯmbri}{\textit{ \papi{arɤmbɯmbri}}}
}\end{relation-sémantique}\end{entrée}

\begin{entrée}
\vedette{\hypertarget{Ⓔrɤmdzɯt}{\papi{ rɤmdzɯt}}}\markboth{rɤmdzɯt}{}
\classe{vi}
\paradigme{\textit{dir :} \jya tɤ-}\acception{1}
\begin{définition}\fra décider\end{définition}
\begin{définition}\cmn 决定,……说了算\end{définition}
\begin{exemple}\jya nɤʑo tɤ-rɤmdzɯt\cmn 你说了算\end{exemple}
\begin{exemple}\jya znde pjɯ́-wɣ-phɯt tɕe ma-pɯ́-wɣ-phɯt, nɤʑo tɯ-rɤmdzɯt\cmn 你决定要不要拆石墙\end{exemple}\acception{2}
\begin{définition}\fra dépendre de\end{définition}
\begin{définition}\cmn 取决于\end{définition}
\begin{exemple}\jya nɯ kɤ-nɤma pe mɤ-pe rɤmdzɯt\cmn 看工作做得好不好\end{exemple}
\begin{relation-sémantique}\synonyme{
\hyperlink{Ⓔzrɤtɕha}{\textit{ \papi{zrɤtɕha}}}
}\end{relation-sémantique}
\begin{relation-sémantique}\confer{
\hyperlink{Ⓔmdzɯt}{\textit{ \papi{mdzɯt}}}
}\end{relation-sémantique}\end{entrée}

\begin{entrée}
\vedette{\hypertarget{Ⓔrɤmgrɯn}{\papi{ rɤmgrɯn}}}\markboth{rɤmgrɯn}{}
\begin{relation-sémantique}\confer{
\hyperlink{Ⓔmgrɯn}{\textit{ \papi{mgrɯn}}}
}\end{relation-sémantique}\end{entrée}

\begin{entrée}
\vedette{\hypertarget{Ⓔrɤmnɯ}{\papi{ rɤmnɯ}}}\markboth{rɤmnɯ}{}\classe{vi}
\paradigme{\textit{dir :} \jya nɯ-}
\begin{définition}\fra germer (arbre)\end{définition}
\begin{définition}\cmn 发芽(树)\end{définition}
\end{entrée}

\begin{entrée}
\vedette{\hypertarget{Ⓔrɤmɲo}{\papi{ rɤmɲo}}}\markboth{rɤmɲo}{}
\begin{relation-sémantique}\confer{
\hyperlink{ⒺmɲoⒽ1}{\textit{ \papi{mɲo1}}}
}\end{relation-sémantique}\end{entrée}

\begin{entrée}
\vedette{\hypertarget{Ⓔrɤmpɕɤr}{\papi{ rɤmpɕɤr}}}\markboth{rɤmpɕɤr}{}
\classe{vi}
\paradigme{\textit{dir :} \jya tɤ-}
\begin{définition}\fra se maquiller\end{définition}
\begin{définition}\cmn 打扮(装饰)\end{définition}
\begin{exemple}\jya jisŋi to-tɯ-rɤmpɕɤr\cmn 你今天打扮了\end{exemple}
\begin{exemple}\jya tɤ-rɤmpɕar-a\cmn 我打扮了\end{exemple}
\begin{relation-sémantique}\confer{
\hyperlink{Ⓔmpɕɤr}{\textit{ \papi{mpɕɤr}}}
}\end{relation-sémantique}\begin{sous-entrée}
\vedette{\hypertarget{}{\papi{ zrɤmpɕɤr}}}\markboth{zrɤmpɕɤr}{}\classe{vt}
\paradigme{\textit{dir :} \jya tɤ-}
\begin{définition}\fra maquiller, décorer\end{définition}
\begin{définition}\cmn 装饰;装扮\end{définition}
\begin{exemple}\jya stɯnmɯ tɤ-mda tɕe, kɯ-rɯstɯnmɯ tɤ-tɕɯ tɕheme nɯni tú-wɣ-zrɤmpɕɤr ra\cmn 办婚礼的时候要打扮新郎和新娘\end{exemple}
\end{sous-entrée}\end{entrée}

\begin{entrée}
\vedette{\hypertarget{Ⓔrɤmphrɯm}{\papi{ rɤmphrɯm}}}\markboth{rɤmphrɯm}{}
\classe{vt}
\paradigme{\textit{dir :} \jya \_}
\begin{définition}\fra aligner en rangées\end{définition}
\begin{définition}\cmn 排整齐\end{définition}
\begin{exemple}\jya kɯmɕku kɤ-ji tɕe, lothi chɯ́-wɣ-rɤmphrɯm tɕe, tɯ-rdoʁ ɯ-thɤcu nɯ tɕu ntsɯ tɯ-rdoʁ kɯ-fse chɯ́-wɣ-ji tɕe tɕe chɯ́-wɣ-rɤmphrɯm ɲɯ-ŋu\cmn 种大蒜的时候,要一个接着一个地排下来\end{exemple}
\begin{relation-sémantique}\synonyme{
\hyperlink{Ⓔsɤʑɯrja}{\textit{ \papi{sɤʑɯrja}}}
}\end{relation-sémantique}
\begin{relation-sémantique}\confer{
\hyperlink{Ⓔtɯ-tɤmphrɯm}{\textit{ \papi{tɯ-tɤmphrɯm}}}
}\end{relation-sémantique}\end{entrée}

\begin{entrée}
\vedette{\hypertarget{Ⓔrɤmphɯr}{\papi{ rɤmphɯr}}}\markboth{rɤmphɯr}{}
\begin{relation-sémantique}\confer{
\hyperlink{Ⓔmphɯr}{\textit{ \papi{mphɯr}}}
}\end{relation-sémantique}\end{entrée}

\begin{entrée}
\vedette{\hypertarget{Ⓔrɤmprɤt}{\papi{ rɤmprɤt}}}\markboth{rɤmprɤt}{}\classe{vt}
\paradigme{\textit{dir :} \jya nɯ-}
\begin{définition}\fra interrompre\end{définition}
\begin{définition}\cmn 中断\end{définition}
\begin{exemple}\jya ɯʑo kɯ ɯ-ma ɲɤ-rɤmprɤt\cmn 他中断了工作\end{exemple}
\begin{exemple}\jya tɯ-mɯ ka-lɤt tɕe kha kɤ-βzu nɯ-rɤmprɤt-i pɯ-ra\cmn 因为下雨,我们中断了修房子的工程\end{exemple}
\begin{relation-sémantique}\confer{
\hyperlink{Ⓔprɤt}{\textit{ \papi{prɤt}}}
}\end{relation-sémantique}\end{entrée}

\begin{entrée}
\vedette{\hypertarget{Ⓔrɤmrɯmrɤmrɯm}{\papi{ rɤmrɯmrɤmrɯm}}}\markboth{rɤmrɯmrɤmrɯm}{}\classe{idph.2}
\begin{définition}\fra (dire) les uns après les autres\end{définition}
\begin{définition}\cmn 接二连三(说)\end{définition}
\begin{exemple}\jya rɤmrɯmrɤmrɯm ʑo to-nɯ-ti-nɯ\cmn 他们接二连三地说出了自己的想法\end{exemple}\end{entrée}

\begin{entrée}
\vedette{\hypertarget{Ⓔrɤmɯthu}{\papi{ rɤmɯthu}}}\markboth{rɤmɯthu}{}
\classe{vi}
\paradigme{\textit{dir :} \jya tɤ-}
\paradigme{\textit{dir :} \jya nɯ-}
\begin{définition}\fra demander partout\end{définition}
\begin{définition}\cmn 到处问\end{définition}
\begin{exemple}\jya aʑo tɤ-rɤmɯthu-a\cmn 我到处问了\end{exemple}
\begin{relation-sémantique}\synonyme{
\hyperlink{Ⓔnɤthɯthu}{\textit{ \papi{nɤthɯthu}}}
}\end{relation-sémantique}
\begin{relation-sémantique}\confer{
\hyperlink{ⒺthuⒽ1}{\textit{ \papi{thu1}}}
}\end{relation-sémantique}
\end{entrée}

\begin{entrée}
\vedette{\hypertarget{Ⓔrɤndaŋ}{\papi{ rɤndaŋ}}}\markboth{rɤndaŋ}{}
\classe{vi}
\paradigme{\textit{dir :} \jya pɯ-}
\paradigme{\textit{dir :} \jya thɯ-}
\begin{définition}\fra réfléchir\end{définition}
\begin{définition}\cmn 思考,考虑\end{définition}
\begin{exemple}\jya ta-ma kɤ-nɤma tɕe, koŋla pjɯ-kɯ-rɤndaŋ ra\cmn 工作的时候需要思考\end{exemple}
\begin{relation-sémantique}\confer{
\hyperlink{Ⓔɯ-ndaŋ,lɤt}{\textit{ \papi{ɯ-ndaŋ,lɤt}}}
}\end{relation-sémantique}\end{entrée}

\begin{entrée}
\vedette{\hypertarget{Ⓔrɤndɯn}{\papi{ rɤndɯn}}}\markboth{rɤndɯn}{}
\begin{relation-sémantique}\confer{
\hyperlink{Ⓔndɯn}{\textit{ \papi{ndɯn}}}
}\end{relation-sémantique}\end{entrée}

\begin{entrée}
\vedette{\hypertarget{Ⓔrɤndzraʁ}{\papi{ rɤndzraʁ}}}\markboth{rɤndzraʁ}{}\classe{vt}
\paradigme{\textit{dir :} \jya tɤ-}
\paradigme{\textit{dir :} \jya nɯ-}
\begin{définition}\fra pétrir\end{définition}
\begin{définition}\cmn 用手捏\end{définition}
\begin{exemple}\jya rɟɤɣi tɤ-rɤndzraʁ-a\cmn 我把糌粑捏成一坨了\end{exemple}
\begin{exemple}\jya tɤrcoʁ nɯ-rɤndzraʁ-a\cmn 我把泥土捏成一坨了\end{exemple}
\begin{exemple}\jya tɯ-pu nɯ-rɤndzraʁ\cmn 我把血肠的馅儿捏下去了\end{exemple}
\begin{exemple}\jya tɯrme ɲɤ-rɤndzraʁ\cmn 他把人捏成一坨了(打得很惨了)\end{exemple}\end{entrée}

\begin{entrée}
\vedette{\hypertarget{Ⓔrɤndzri}{\papi{ rɤndzri}}}\markboth{rɤndzri}{}\classe{vt}
\paradigme{\textit{dir :} \jya kɤ-}
\begin{définition}\fra tordre la paille pour la faire sécher\end{définition}
\begin{définition}\cmn 拧(干草)\end{définition}
\begin{exemple}\jya tɯɣro kɤ-rɤndzri-t-a\cmn 我把干草拧成一绞了\end{exemple}
\begin{relation-sémantique}\confer{
\hyperlink{Ⓔndzri}{\textit{ \papi{ndzri}}}
}\end{relation-sémantique}
\begin{relation-sémantique}\confer{
\hyperlink{Ⓔtɯ-tɤndzri}{\textit{ \papi{tɯ-tɤndzri}}}
}\end{relation-sémantique}\end{entrée}

\begin{entrée}
\vedette{\hypertarget{Ⓔrɤnŋa}{\papi{ rɤnŋa}}}\markboth{rɤnŋa}{}\classe{vi}
\paradigme{\textit{dir :} \jya pɯ-}
\begin{définition}\ 
\begin{déclaration}\grammar{apass}\end{déclaration}\end{définition}
\begin{définition}\fra devoir de l'argent\end{définition}
\begin{définition}\cmn 欠账\end{définition}
\begin{exemple}\jya ɯʑɤɣ, ɯ-phe pɯ-rɤnŋa-a\cmn 我以前欠他的钱\end{exemple}
\begin{exemple}\jya aʑo nɤ-ɕki rɤnŋa-a\cmn 我欠你的钱\end{exemple}
\begin{relation-sémantique}\confer{
\hyperlink{Ⓔŋa}{\textit{ \papi{ŋa}}}
}\end{relation-sémantique}
\begin{relation-sémantique}\confer{
\hyperlink{Ⓔtɯ-nŋa}{\textit{ \papi{tɯ-nŋa}}}
}\end{relation-sémantique}\end{entrée}

\begin{entrée}
\vedette{\hypertarget{Ⓔrɤntɕha}{\papi{ rɤntɕha}}}\markboth{rɤntɕha}{}
\begin{relation-sémantique}\confer{
\hyperlink{Ⓔntɕha}{\textit{ \papi{ntɕha}}}
}\end{relation-sémantique}\end{entrée}

\begin{entrée}
\vedette{\hypertarget{Ⓔrɤntɕhom}{\papi{ rɤntɕhom}}}\markboth{rɤntɕhom}{}
\classe{vi}
\paradigme{\textit{dir :} \jya tɤ-}
\begin{définition}\fra effectuer une danse rituelle\end{définition}
\begin{définition}\cmn 跳神
\begin{déclaration} \étymologie{\papi{ⁿtɕʰams}}\end{déclaration}\end{définition}
\begin{exemple}\jya rgɯnba ɲɯ-rɤntɕhom-nɯ\cmn 庙里在跳神\end{exemple}\end{entrée}

\begin{entrée}
\vedette{\hypertarget{Ⓔrɤntshom}{\papi{ rɤntshom}}}\markboth{rɤntshom}{}
\classe{vi}
\paradigme{\textit{dir :} \jya kɤ-}
\begin{définition}\fra faire une retraite\end{définition}
\begin{définition}\cmn 闭关修行
\begin{déclaration} \étymologie{\papi{bsɲen.mtsʰams}}\end{déclaration}\end{définition}
\begin{exemple}\jya βlama ɲɯ-rɤntshom\cmn 喇嘛在修行\end{exemple}
\begin{exemple}\jya βlama ko-rɤntshom\cmn 喇嘛修行了\end{exemple}\end{entrée}

\begin{entrée}
\vedette{\hypertarget{Ⓔrɤɲɟiɲɟi}{\papi{ rɤɲɟiɲɟi}}}\markboth{rɤɲɟiɲɟi}{}\classe{vt}
\paradigme{\textit{dir :} \jya nɯ-}
\begin{définition}\fra écraser en petits morceaux\end{définition}
\begin{définition}\cmn 弄碎(搓成碎片)\end{définition}
\begin{relation-sémantique}\confer{
\hyperlink{Ⓔrɤmboʁɲɟi}{\textit{ \papi{rɤmboʁɲɟi}}}
}\end{relation-sémantique}
\begin{relation-sémantique}\confer{
\hyperlink{Ⓔrɤɲɟoʁɲɟi}{\textit{ \papi{rɤɲɟoʁɲɟi}}}
}\end{relation-sémantique}\end{entrée}

\begin{entrée}
\vedette{\hypertarget{Ⓔrɤɲɟoʁɲɟi}{\papi{ rɤɲɟoʁɲɟi}}}\markboth{rɤɲɟoʁɲɟi}{}\classe{vt}
\paradigme{\textit{dir :} \jya pɯ-}
\begin{définition}\fra écraser en petits morceaux\end{définition}
\begin{définition}\cmn 弄碎(搓成碎片)\end{définition}
\begin{exemple}\jya @yangyu pɯ-rɤɲɟoʁɲɟi-t-a\cmn 我把土豆弄碎了\end{exemple}
\begin{relation-sémantique}\confer{
\hyperlink{Ⓔrɤɲɟiɲɟi}{\textit{ \papi{rɤɲɟiɲɟi}}}
}\end{relation-sémantique}
\begin{relation-sémantique}\confer{
\hyperlink{Ⓔrɤmboʁɲɟi}{\textit{ \papi{rɤmboʁɲɟi}}}
}\end{relation-sémantique}\end{entrée}

\begin{entrée}
\vedette{\hypertarget{Ⓔrɤŋgat}{\papi{ rɤŋgat}}}\markboth{rɤŋgat}{}
\classe{vi}
\paradigme{\textit{dir :} \jya tɤ-}
\begin{définition}\fra se préparer à, être sur le point de\end{définition}
\begin{définition}\cmn 准备
\begin{déclaration}\use{补语动词必须带\stylefv{kɯ-}名物化前缀}\end{déclaration}\end{définition}
\begin{exemple}\jya kɯ-ɕe tɤ-rɤŋgat-a\cmn 我准备出发\end{exemple}
\begin{exemple}\jya tɯ-mɯ kɯ-lɤt ɲɯ-rɤŋgat\cmn 快要下雨了\end{exemple}
\begin{exemple}\jya kɯ-mbɯt ɲɯ-rɤŋgat\cmn 快要垮了\end{exemple}
\begin{relation-sémantique}\synonyme{
\hyperlink{Ⓔʑɣɤmɲo}{\textit{ \papi{ʑɣɤmɲo}}}
}\end{relation-sémantique}\end{entrée}

\begin{entrée}
\vedette{\hypertarget{Ⓔrɤŋgɯm}{\papi{ rɤŋgɯm}}}\markboth{rɤŋgɯm}{}
\classe{vi}
\paradigme{\textit{dir :} \jya thɯ-}
\begin{définition}\ 
\begin{déclaration}\grammar{denom}\end{déclaration}\end{définition}
\begin{définition}\fra pondre\end{définition}
\begin{définition}\cmn 下蛋\end{définition}
\begin{exemple}\jya pɣa chɤ-rɤŋgɯm\cmn 鸟下了蛋\end{exemple}
\begin{exemple}\jya kumpɣa ɲɯ-rɤŋgɯm\cmn 鸟在下蛋\end{exemple}
\begin{relation-sémantique}\confer{
\hyperlink{Ⓔtɤ-ŋgɯm}{\textit{ \papi{tɤ-ŋgɯm}}}
}\end{relation-sémantique}\end{entrée}

\begin{entrée}
\vedette{\hypertarget{Ⓔrɤŋom}{\papi{ rɤŋom}}}\markboth{rɤŋom}{}
\classe{n}
\begin{définition}\fra énervement\end{définition}
\begin{définition}\cmn 气愤\end{définition}
\begin{exemple}\jya rɤŋom kɯ pɯwɯ pjɤ-sat\cmn 气得把驴子弄死了\end{exemple}
\begin{relation-sémantique}\confer{
\hyperlink{Ⓔnɯrɤŋom}{\textit{ \papi{nɯrɤŋom}}}
}\end{relation-sémantique}\end{entrée}

\begin{entrée}
\vedette{\hypertarget{Ⓔrɤɴqra}{\papi{ rɤɴqra}}}\markboth{rɤɴqra}{}
\classe{vt}
\paradigme{\textit{dir :} \jya nɯ-}
\begin{définition}\ 
\begin{déclaration}\grammar{denom}\end{déclaration}\end{définition}
\begin{définition}\fra (faire de façon) incomplète\end{définition}
\begin{définition}\cmn (做得)不完整\end{définition}
\begin{exemple}\jya jiɕqha pɯ-mdoʁmdi ri, pɯ-rɤɴqra-t-a\cmn 刚才是完整的,我弄了个缺口(例如馍馍咬了一口)\end{exemple}
\begin{exemple}\jya a-qajɣi pɯ-rɤɴqra-t-a\cmn 我没有把馍馍吃完\end{exemple}
\begin{exemple}\jya kɯrtsɤɣ kɯ paʁ to-ndza tɕe, chɤ-rɤɴqra\cmn 豹子吃了猪,吃得不完整\end{exemple}
\begin{exemple}\jya jɯfɕɯr χpi kɤ-fɕɤt mɯ-pɯ-sthɯt-a tɕe, nɯ-rɤɴqra-t-a\cmn 昨天我没有把故事讲完,讲得不完整\end{exemple}
\begin{relation-sémantique}\confer{
\hyperlink{Ⓔɯ-ɴqra}{\textit{ \papi{ɯ-ɴqra}}}
}\end{relation-sémantique}\end{entrée}

\begin{entrée}
\vedette{\hypertarget{Ⓔrɤpɕaʁ}{\papi{ rɤpɕaʁ}}}\markboth{rɤpɕaʁ}{}
\classe{vi}
\paradigme{\textit{dir :} \jya lɤ-}
\begin{définition}\ 
\begin{déclaration}\grammar{denom}\end{déclaration}\end{définition}
\begin{définition}\fra se prosterner\end{définition}
\begin{définition}\cmn 跪下磕头\end{définition}
\begin{exemple}\jya lɤ-rɤpɕaʁa-a (=pɕaʁ lɤ-βzu-t-a)\cmn 我磕了头\end{exemple}
\begin{relation-sémantique}\confer{
\hyperlink{Ⓔpɕaʁ}{\textit{ \papi{pɕaʁ}}}
}\end{relation-sémantique}\end{entrée}

\begin{entrée}
\vedette{\hypertarget{Ⓔrɤpɣaʁ}{\papi{ rɤpɣaʁ}}}\markboth{rɤpɣaʁ}{}
\begin{relation-sémantique}\confer{
\hyperlink{Ⓔpɣaʁ}{\textit{ \papi{pɣaʁ}}}
}\end{relation-sémantique}\end{entrée}

\begin{entrée}
\vedette{\hypertarget{Ⓔrɤpɣi}{\papi{ rɤpɣi}}}\markboth{rɤpɣi}{}
\classe{vt}
\paradigme{\textit{dir :} \jya pɯ-}
\paradigme{\textit{dir :} \jya lɤ-}
\begin{définition}\fra mélanger la farine et l'eau\end{définition}
\begin{définition}\cmn 和面\end{définition}
\begin{exemple}\jya qajɣi βze-a pɯ-ŋu tɕe, tɤjlu pɯ-rɤpɣi-t-a\cmn 我准备做馍馍,就和了面\end{exemple}\end{entrée}

\begin{entrée}
\vedette{\hypertarget{Ⓔrɤphɯ}{\papi{ rɤphɯ}}}\markboth{rɤphɯ}{}\classe{vt}
\begin{définition}\fra donner un prix\end{définition}
\begin{définition}\cmn 定价格\end{définition}
\begin{exemple}\jya ki laχtɕha ki kɤ-rɤphɯ-t-a\cmn 我定了这个东西的价格\end{exemple}
\begin{relation-sémantique}\confer{
\hyperlink{Ⓔʑɣɤrɤphɯ}{\textit{ \papi{ʑɣɤrɤphɯ}}}
}\end{relation-sémantique}
\begin{relation-sémantique}\confer{
\hyperlink{Ⓔɯ-phɯ}{\textit{ \papi{ɯ-phɯ}}}
}\end{relation-sémantique}
\end{entrée}

\begin{entrée}
\vedette{\hypertarget{Ⓔrɤpjɤt}{\papi{ rɤpjɤt}}}\markboth{rɤpjɤt}{}
\begin{relation-sémantique}\confer{
\hyperlink{Ⓔpjɤt}{\textit{ \papi{pjɤt}}}
}\end{relation-sémantique}\end{entrée}

\begin{entrée}
\vedette{\hypertarget{Ⓔrɤpjɤz}{\papi{ rɤpjɤz}}}\markboth{rɤpjɤz}{}
\classe{vt}
\paradigme{\textit{dir :} \jya thɯ-}
\begin{définition}\ 
\begin{déclaration}\grammar{denom}\end{déclaration}\end{définition}
\begin{définition}\fra tresser (les cheveux, fils)\end{définition}
\begin{définition}\cmn 编(头发,线)\end{définition}
\begin{exemple}\jya nɤ-ku thɯ-rɤpjɤz\cmn 你编辫子吧\end{exemple}
\begin{relation-sémantique}\confer{
\hyperlink{Ⓔtɤpjɤz}{\textit{ \papi{tɤpjɤz}}}
}\end{relation-sémantique}\end{entrée}

\begin{entrée}
\vedette{\hypertarget{Ⓔrɤpɯ}{\papi{ rɤpɯ}}}\markboth{rɤpɯ}{}
\classe{vi}
\paradigme{\textit{dir :} \jya thɯ-}
\begin{définition}\fra mettre bas (animaux)\end{définition}
\begin{définition}\cmn 生崽子(动物)\end{définition}
\begin{exemple}\jya nɯŋa thɯ-rɤpɯ\cmn 奶牛生了崽子\end{exemple}
\begin{exemple}\jya paʁ thɯ-rɤpɯ\cmn 猪生了崽子\end{exemple}
\begin{relation-sémantique}\confer{
\hyperlink{Ⓔtɤ-pɯ}{\textit{ \papi{tɤ-pɯ}}}
}\end{relation-sémantique}
\begin{sous-entrée}
\vedette{\hypertarget{}{\papi{ zrɤpɯ}}}\markboth{zrɤpɯ}{}\classe{vt}
\begin{définition}\ 
\begin{déclaration}\grammar{caus}\end{déclaration}\end{définition}
\begin{définition}\fra faire avoir des petits (animaux)\end{définition}
\begin{définition}\cmn 让……生崽\end{définition}
\end{sous-entrée}\end{entrée}

\begin{entrée}
\vedette{\hypertarget{Ⓔrɤqur}{\papi{ rɤqur}}}\markboth{rɤqur}{}\classe{vi}
\paradigme{\textit{dir :} \jya tɤ-}
\begin{définition}\fra ramasser\end{définition}
\begin{définition}\cmn 收藏\end{définition}
\begin{exemple}\jya laχtɕha tɤ-rɤqur-a\cmn 我把东西收起来了\end{exemple}
\begin{exemple}\jya laχtɕha kɤ-ntɕhoz mɤ-kɯ-ra nɯ ra tɤ-rɤqur-a\cmn 我把不用的东西收藏起来了\end{exemple}
\begin{exemple}\jya laχtɕha kɯ-ra nɯ ra a-pɯ-ɤta, mɤ-kɯ-ra nɯ ra tɤ-rɤqur\cmn 需要的东西方在那里,不需要的东西收藏好\end{exemple}
\begin{exemple}\jya a-ŋga nɯ thamtɕɤt mɯ́j-ra tɕe tɤ-rɤqur-a\cmn 我不需要那么多衣服,所以就把它收藏起来了\end{exemple}\end{entrée}

\begin{entrée}
\vedette{\hypertarget{Ⓔrɤru}{\papi{ rɤru}}}\markboth{rɤru}{}\classe{vi}
\paradigme{\textit{dir :} \jya tɤ-}\acception{1}
\begin{définition}\fra se lever\end{définition}
\begin{définition}\cmn 起床;起来\end{définition}
\begin{exemple}\jya jɯfɕo tɤ-rɤru-a\cmn 我今天早上起来了\end{exemple}
\begin{exemple}\jya tʂu (tʂɤχcɤl) tɤ-rɤru ma ɲɯ-tɯ-saʁdɯɣ\cmn 你起来,你挡到(我的路)\end{exemple}
\begin{exemple}\jya jiɕqha nɯ nɯ-nɤkhɤzŋga-t-a ri, maka mɯ́j-rɤru\cmn 我叫了,但是他根本不起床\end{exemple}\acception{2}
\begin{définition}\fra fermenter (vin)\end{définition}
\begin{définition}\cmn 发酵(酒)\end{définition}
\begin{exemple}\jya cha to-rɤru (=ɲɤ-xtsu)\cmn 酒发酵了\end{exemple}\begin{sous-entrée}
\vedette{\hypertarget{}{\papi{ zrɤru}}}\markboth{zrɤru}{}\classe{vt}
\paradigme{\textit{dir :} \jya tɤ-}
\begin{définition}\fra faire lever\end{définition}
\begin{définition}\cmn 让……起床\end{définition}
\begin{exemple}\jya tɤ-zrɤru-t-a\cmn 我让他起床了\end{exemple}
\begin{exemple}\jya tɤndʐo kɯ a-ŋgo to-zrɤru\cmn 冷的天气令我生病了\end{exemple}
\end{sous-entrée}\end{entrée}

\begin{entrée}
\vedette{\hypertarget{Ⓔrɤraʁzɯz}{\papi{ rɤraʁzɯz}}}\markboth{rɤraʁzɯz}{}
\begin{relation-sémantique}\confer{
\hyperlink{Ⓔraʁrɯz}{\textit{ \papi{raʁrɯz}}}
}\end{relation-sémantique}\end{entrée}

\begin{entrée}
\vedette{\hypertarget{Ⓔrɤrɤt}{\papi{ rɤrɤt}}}\markboth{rɤrɤt}{}
\begin{relation-sémantique}\confer{
\hyperlink{Ⓔrɤt}{\textit{ \papi{rɤt}}}
}\end{relation-sémantique}\end{entrée}

\begin{entrée}
\vedette{\hypertarget{Ⓔrɤrcoʁ}{\papi{ rɤrcoʁ}}}\markboth{rɤrcoʁ}{}
\classe{vi}
\paradigme{\textit{dir :} \jya pɯ-}
\begin{définition}\fra mélanger de l’eau et de la terre\end{définition}
\begin{définition}\cmn 和泥\end{définition}
\begin{exemple}\jya jisŋi pɯ-rɤrcoʁa (=tɤrcoʁ pɯ-βzu-t-a)\cmn 我今天和了泥\end{exemple}
\begin{exemple}\jya kha ɲɯ-ɤsɯ-βzu-nɯ tɕe, pɯ-rɤrcoʁ-a\cmn 他们在修房子,所以我就和了泥\end{exemple}
\begin{exemple}\jya ʑala ɲɯ-ɤsɯ-lɤt tɕe, pɯ-rɤrcoʁ-a\cmn 他在糊墙,所以我就和了泥\end{exemple}
\begin{relation-sémantique}\confer{
\hyperlink{Ⓔtɤrcoʁ}{\textit{ \papi{tɤrcoʁ}}}
}\end{relation-sémantique}
\begin{relation-sémantique}\confer{
\hyperlink{Ⓔɣɤrcoʁ}{\textit{ \papi{ɣɤrcoʁ}}}
}\end{relation-sémantique}\end{entrée}

\begin{entrée}
\vedette{\hypertarget{Ⓔrɤrɟit}{\papi{ rɤrɟit}}}\markboth{rɤrɟit}{}\classe{vi}
\paradigme{\textit{dir :} \jya thɯ-}
\begin{définition}\ 
\begin{déclaration}\grammar{denom}\end{déclaration}\end{définition}
\begin{définition}\fra donner naissance à un enfant\end{définition}
\begin{définition}\cmn 生孩子\end{définition}
\begin{exemple}\jya chɤ-rɤrɟit (=ɯ-rɟit to-tu)\cmn 她生了孩子\end{exemple}
\begin{relation-sémantique}\confer{
\hyperlink{Ⓔtɤ-rɟit}{\textit{ \papi{tɤ-rɟit}}}
}\end{relation-sémantique}
\begin{sous-entrée}
\vedette{\hypertarget{}{\papi{ zrɤrɟit}}}\markboth{zrɤrɟit}{}\classe{vt}
\paradigme{\textit{dir :} \jya thɯ-}
\begin{définition}\ 
\begin{déclaration}\grammar{caus}\end{déclaration}\end{définition}
\begin{définition}\fra faire donner naissance à un enfant\end{définition}
\begin{définition}\cmn 使……生孩子\end{définition}
\end{sous-entrée}\end{entrée}

\begin{entrée}
\vedette{\hypertarget{Ⓔrɤrka}{\papi{ rɤrka}}}\markboth{rɤrka}{}
\classe{vi}
\paradigme{\textit{dir :} \jya thɯ-}
\begin{définition}\fra faire des jumeaux\end{définition}
\begin{définition}\cmn 生双胞胎\end{définition}
\begin{exemple}\jya tshɤt chɤ-rɤrka\cmn 山羊生了双胞胎\end{exemple}
\begin{relation-sémantique}\confer{
\hyperlink{ⒺtɤrkaⒽ2}{\textit{ \papi{tɤrka2}}}
}\end{relation-sémantique}\end{entrée}

\begin{entrée}
\vedette{\hypertarget{Ⓔrɤrma}{\papi{ rɤrma}}}\markboth{rɤrma}{}
\classe{vi}
\paradigme{\textit{dir :} \jya kɤ-}
\begin{définition}\fra habiter\end{définition}
\begin{définition}\cmn 居住\end{définition}
\begin{exemple}\jya jiʑo chengdu kɤ-rɤrma-j\cmn 我们住在成都了\end{exemple}
\begin{relation-sémantique}\confer{
\hyperlink{Ⓔrma}{\textit{ \papi{rma}}}
}\end{relation-sémantique}
\begin{relation-sémantique}\confer{
\hyperlink{Ⓔtɯrma}{\textit{ \papi{tɯrma}}}
}\end{relation-sémantique}\end{entrée}

\begin{entrée}
\vedette{\hypertarget{Ⓔrɤrmbɣo}{\papi{ rɤrmbɣo}}}\markboth{rɤrmbɣo}{}\classe{vi}
\paradigme{\textit{dir :} \jya lɤ-}
\begin{définition}\fra s'accumuler (eau)\end{définition}
\begin{définition}\cmn 积水\end{définition}\begin{sous-entrée}
\vedette{\hypertarget{}{\papi{ zrɤrmbɣo}}}\markboth{zrɤrmbɣo}{}\classe{vt}
\paradigme{\textit{dir :} \jya lɤ-}
\begin{définition}\fra retenir (accumuler) de l'eau\end{définition}
\begin{définition}\cmn 积水(大范围)\end{définition}
\begin{exemple}\jya tɯ-ci lú-wɣ-zrɤrmbɣo ɲɯ-ra\cmn 要积水\end{exemple}
\end{sous-entrée}\end{entrée}

\begin{entrée}
\vedette{\hypertarget{Ⓔrɤroʁ}{\papi{ rɤroʁ}}}\markboth{rɤroʁ}{}
\begin{relation-sémantique}\confer{
\hyperlink{Ⓔroʁ}{\textit{ \papi{roʁ}}}
}\end{relation-sémantique}\end{entrée}

\begin{entrée}
\vedette{\hypertarget{Ⓔrɤroʁrɯz}{\papi{ rɤroʁrɯz}}}\markboth{rɤroʁrɯz}{}
\begin{relation-sémantique}\confer{
\hyperlink{Ⓔraʁrɯz}{\textit{ \papi{raʁrɯz}}}
}\end{relation-sémantique}\end{entrée}

\begin{entrée}
\vedette{\hypertarget{Ⓔrɤrti}{\papi{ rɤrti}}}\markboth{rɤrti}{}
\classe{vi}
\paradigme{\textit{dir :} \jya thɯ-}
\begin{définition}\fra mettre bas (cheval)\end{définition}
\begin{définition}\cmn 生崽(马)\end{définition}
\begin{exemple}\jya mbro chɤ-rɤrti\cmn 马生崽了\end{exemple}
\begin{relation-sémantique}\synonyme{
\hyperlink{Ⓔrɤpɯ}{\textit{ \papi{rɤpɯ}}}
}\end{relation-sémantique}
\begin{relation-sémantique}\synonyme{
\hyperlink{ⒺɬoʁⒽ1}{\textit{ \papi{ɬoʁ}}}
}\end{relation-sémantique}
\begin{relation-sémantique}\synonyme{
\hyperlink{Ⓔrɤrɟit}{\textit{ \papi{rɤrɟit}}}
}\end{relation-sémantique}
\begin{relation-sémantique}\confer{
\hyperlink{Ⓔɯ-rti}{\textit{ \papi{ɯ-rti}}}
}\end{relation-sémantique}\end{entrée}

\begin{entrée}
\vedette{\hypertarget{Ⓔrɤrtsɤɣ}{\papi{ rɤrtsɤɣ}}}\markboth{rɤrtsɤɣ}{}
\classe{vs}
\paradigme{\textit{dir :} \jya tɤ-}
\begin{définition}\fra avoir la tige qui pousse\end{définition}
\begin{définition}\cmn 拔节\end{définition}
\begin{exemple}\jya tɤɕi to-rɤrtsɤɣ\cmn 青稞拔节了\end{exemple}
\begin{relation-sémantique}\confer{
 \papi{tɤrtsɤɣ}
}\end{relation-sémantique}\end{entrée}

\begin{entrée}
\vedette{\hypertarget{Ⓔrɤrtshɯm}{\papi{ rɤrtshɯm}}}\markboth{rɤrtshɯm}{}\classe{vi}
\paradigme{\textit{dir :} \jya nɯ-}
\begin{définition}\fra ne pas finir\end{définition}
\begin{définition}\cmn 没有做完\end{définition}
\begin{exemple}\jya a-ma na-rɤrtshɯm\cmn 我的工作没有做完\end{exemple}
\begin{exemple}\jya ɯ-rju nɯ ɲɤ-rɤrtshɯm\cmn 他话没有说完\end{exemple}
\begin{exemple}\jya ta-ma ɲɤ-rɤrtshɯm\cmn 工作没有做完\end{exemple}
\begin{exemple}\jya kɤ-nɤma nɯ kɤ-tshɯt mɯ-pɯ-ŋgrɯ tɕe, nɯ-rɤrtshɯm-i pɯ-ra\cmn 工作没有能完成,所以我们只好半途而废了\end{exemple}\begin{sous-entrée}
\vedette{\hypertarget{}{\papi{ arɤrtshɯm}}}\markboth{arɤrtshɯm}{}\classe{vi}
\paradigme{\textit{dir :} \jya nɯ-}
\begin{définition}\ 
\begin{déclaration}\grammar{pass}\end{déclaration}\end{définition}
\begin{définition}\fra ne pas être fini\end{définition}
\begin{définition}\cmn 没有完成\end{définition}
\begin{exemple}\jya ɲɤ-k-ɤrɤrtshɯm-ci\cmn 没有做完\end{exemple}
\begin{exemple}\jya kɤ-nɤma kɤ-tshɯt mɯ-pjɤ-khɯ tɕe ɲɤ-k-ɤrɤrtshɯm-ci\cmn 工作没有能做完,所以没有完成\end{exemple}
\begin{relation-sémantique}\confer{
\hyperlink{Ⓔɯ-rtshɯm}{\textit{ \papi{ɯ-rtshɯm}}}
}\end{relation-sémantique}
\end{sous-entrée}\end{entrée}

\begin{entrée}
\vedette{\hypertarget{Ⓔrɤrzɯɣ}{\papi{ rɤrzɯɣ}}}\markboth{rɤrzɯɣ}{} (\variante{rɤrzɯrzɯɣ}) 
\classe{vt}
\paradigme{\textit{dir :} \jya pɯ-}
\begin{définition}\fra couper en sections\end{définition}
\begin{définition}\cmn 锯成一段一段、一节一节\end{définition}
\begin{exemple}\jya si pɯ-rɤrzɯɣ-a\cmn 我把木头锯成几段了\end{exemple}
\begin{exemple}\jya tɯmbri pɯ-rɤrzɯɣ-a\cmn 我把绳子剪成几段了\end{exemple}
\begin{relation-sémantique}\synonyme{
\hyperlink{Ⓔrɤɣdɤt}{\textit{ \papi{rɤɣdɤt}}}
}\end{relation-sémantique}
\begin{relation-sémantique}\confer{
\hyperlink{Ⓔtɯ-rzɯɣ}{\textit{ \papi{tɯ-rzɯɣ}}}
}\end{relation-sémantique}\end{entrée}

\begin{entrée}
\vedette{\hypertarget{Ⓔrɤscɤt}{\papi{ rɤscɤt}}}\markboth{rɤscɤt}{}
\begin{relation-sémantique}\confer{
\hyperlink{Ⓔscɤt}{\textit{ \papi{scɤt}}}
}\end{relation-sémantique}\end{entrée}

\begin{entrée}
\vedette{\hypertarget{Ⓔrɤscoz}{\papi{ rɤscoz}}}\markboth{rɤscoz}{}\classe{vi}
\paradigme{\textit{dir :} \jya tɤ-}
\begin{définition}\ 
\begin{déclaration}\grammar{denom}\end{déclaration}\end{définition}
\begin{définition}\fra écrire\end{définition}
\begin{définition}\cmn 写出来
\begin{déclaration}\use{主要用于“自己造成”这个说法,带有为己前缀\stylefv{nɯ-}}\end{déclaration}\end{définition}
\begin{exemple}\jya ɲɯ-rɤscoz (tɤ-scoz ɲɯ-ɤsɯ-rɤt)\cmn 他在写信\end{exemple}
\begin{exemple}\jya jaχpa rɯmu tɤ-tɯ-nɯrɤscoz ɕti\cmn 这是你签的字(这是你自己一手造成的)\end{exemple}
\begin{relation-sémantique}\confer{
\hyperlink{Ⓔtɤ-scoz}{\textit{ \papi{tɤ-scoz}}}
}\end{relation-sémantique}\end{entrée}

\begin{entrée}
\vedette{\hypertarget{Ⓔrɤskɤr}{\papi{ rɤskɤr}}}\markboth{rɤskɤr}{}
\begin{relation-sémantique}\confer{
\hyperlink{Ⓔskɤr}{\textit{ \papi{skɤr}}}
}\end{relation-sémantique}\end{entrée}

\begin{entrée}
\vedette{\hypertarget{Ⓔrɤsloʁ}{\papi{ rɤsloʁ}}}\markboth{rɤsloʁ}{}
\begin{relation-sémantique}\confer{
\hyperlink{Ⓔsloʁ}{\textit{ \papi{sloʁ}}}
}\end{relation-sémantique}\end{entrée}

\begin{entrée}
\vedette{\hypertarget{Ⓔrɤspɤr}{\papi{ rɤspɤr}}}\markboth{rɤspɤr}{}
\begin{relation-sémantique}\confer{
\hyperlink{Ⓔspɤr}{\textit{ \papi{spɤr}}}
}\end{relation-sémantique}\end{entrée}

\begin{entrée}
\vedette{\hypertarget{Ⓔrɤspra}{\papi{ rɤspra}}}\markboth{rɤspra}{}
\classe{vt}
\paradigme{\textit{dir :} \jya kɤ-}
\paradigme{\textit{dir :} \jya nɯ-}
\begin{définition}\ 
\begin{déclaration}\grammar{denom}\end{déclaration}\end{définition}
\begin{définition}\fra prendre par poignées\end{définition}
\begin{définition}\cmn 一把一把地拿\end{définition}
\begin{exemple}\jya sɯjno nɯ-rɤspra-t-a\cmn 我一把一把地拿了草\end{exemple}
\begin{relation-sémantique}\confer{
\hyperlink{Ⓔtɯ-spra}{\textit{ \papi{tɯ-spra}}}
}\end{relation-sémantique}\end{entrée}

\begin{entrée}
\vedette{\hypertarget{Ⓔrɤspɯ}{\papi{ rɤspɯ}}}\markboth{rɤspɯ}{}
\classe{vi}
\paradigme{\textit{dir :} \jya tɤ-}
\begin{définition}\ 
\begin{déclaration}\grammar{denom}\end{déclaration}\end{définition}
\begin{définition}\fra laisser couler du pus\end{définition}
\begin{définition}\cmn 化脓\end{définition}
\begin{exemple}\jya a-jaʁ to-rɤspɯ\cmn 我的手化脓了\end{exemple}
\begin{relation-sémantique}\confer{
\hyperlink{Ⓔtɤ-spɯ}{\textit{ \papi{tɤ-spɯ}}}
}\end{relation-sémantique}\end{entrée}

\begin{entrée}
\vedette{\hypertarget{Ⓔrɤsqa}{\papi{ rɤsqa}}}\markboth{rɤsqa}{}
\classe{n}
\begin{définition}\fra navet cuit\end{définition}
\begin{définition}\cmn 煮熟了的圆根\end{définition}
\begin{relation-sémantique}\confer{
\hyperlink{Ⓔrasti}{\textit{ \papi{rasti}}}
}\end{relation-sémantique}\end{entrée}

\begin{entrée}
\vedette{\hypertarget{Ⓔrɤstu}{\papi{ rɤstu}}}\markboth{rɤstu}{}
\classe{vs}
\paradigme{\textit{dir :} \jya tɤ-}
\begin{définition}\fra honnête\end{définition}
\begin{définition}\cmn 老实\end{définition}
\begin{exemple}\jya kɯ-rɤstu ci ɲɯ-ŋu, ɯ-stu tu-ti ɲɯ-ɕti\cmn 他是个老实人,他说真话\end{exemple}
\begin{relation-sémantique}\confer{
\hyperlink{Ⓔɯ-stuⒽ1}{\textit{ \papi{ɯ-stu1}}}
}\end{relation-sémantique}\end{entrée}

\begin{entrée}
\vedette{\hypertarget{Ⓔrɤsta}{\papi{ rɤsta}}}\markboth{rɤsta}{}\classe{vi}
\paradigme{\textit{dir :} \jya kɤ-}
\begin{définition}\fra immobile, fixé, rester à un endroit\end{définition}
\begin{définition}\cmn 固定;长期;待在一个地方不动;定型\end{définition}
\begin{exemple}\jya ɯ-kɯ-rɤma kɯ-rɤsta ʑo ŋu\cmn 他长期给他做事\end{exemple}
\begin{exemple}\jya nɤʑo nɯtɕu kɤ-rɤsta\cmn 你待在那里不要动\end{exemple}
\begin{sous-entrée}
\vedette{\hypertarget{}{\papi{ zrɤsta}}}\markboth{zrɤsta}{}\classe{vt}
\begin{définition}\fra fixer\end{définition}
\begin{définition}\cmn 固定\end{définition}
\end{sous-entrée}\end{entrée}

\begin{entrée}
\vedette{\hypertarget{Ⓔrɤstɯm}{\papi{ rɤstɯm}}}\markboth{rɤstɯm}{}\classe{vt}
\paradigme{\textit{dir :} \jya tɤ-}
\begin{définition}\fra enrouler une corde autour de son bras pour la ranger\end{définition}
\begin{définition}\cmn 把凌乱的绳子收拢整齐\end{définition}
\begin{exemple}\jya tɯmbri tɤ-rɤstɯm-a\cmn 我把绳子收拢整齐\end{exemple}
\begin{relation-sémantique}\synonyme{
\hyperlink{Ⓔrɤlkɯɣ}{\textit{ \papi{rɤlkɯɣ}}}
}\end{relation-sémantique}
\begin{relation-sémantique}\confer{
\hyperlink{Ⓔstɯm}{\textit{ \papi{stɯm}}}
}\end{relation-sémantique}\end{entrée}

\begin{entrée}
\vedette{\hypertarget{Ⓔrɤt}{\papi{ rɤt}}}\markboth{rɤt}{}\classe{vt}
\paradigme{\textit{dir :} \jya pɯ-}\acception{1}
\begin{définition}\fra écrire\end{définition}
\begin{définition}\cmn 写\end{définition}
\begin{exemple}\jya tɤ-scoz ci pɯ-rat-a\cmn 我写了一封信\end{exemple}\acception{2}
\begin{définition}\fra dessiner\end{définition}
\begin{définition}\cmn 画\end{définition}\begin{sous-entrée}
\vedette{\hypertarget{}{\papi{ arɤt}}}\markboth{arɤt}{}\classe{vi}
\begin{définition}\ 
\begin{déclaration}\grammar{pass}\end{déclaration}\end{définition}
\begin{définition}\fra être écrit\end{définition}
\begin{définition}\cmn 写着\end{définition}
\begin{exemple}\jya ɯ-taʁ a-rmi arɤt\cmn 上面写着我的名字\end{exemple}
\end{sous-entrée}\begin{sous-entrée}
\vedette{\hypertarget{}{\papi{ rɤrɤt}}}\markboth{rɤrɤt}{}\classe{vi}
\begin{définition}\ 
\begin{déclaration}\grammar{apass}\end{déclaration}\end{définition}
\begin{définition}\fra écrire\end{définition}
\begin{définition}\cmn 写字\end{définition}
\end{sous-entrée}\begin{sous-entrée}
\vedette{\hypertarget{}{\papi{ sɯrɤt}}}\markboth{sɯrɤt}{}\classe{vt}\acception{1}
\begin{définition}\fra écrire/dessiner avec\end{définition}
\begin{définition}\cmn 用……写\end{définition}\acception{2}
\begin{définition}\fra faire écrire/dessiner\end{définition}
\begin{définition}\cmn 请……写……\end{définition}
\end{sous-entrée}\begin{sous-entrée}
\vedette{\hypertarget{}{\papi{ zrɤrɤt}}}\markboth{zrɤrɤt}{}\classe{vt}
\begin{définition}\ 
\begin{déclaration}\grammar{apass}\end{déclaration}
\begin{déclaration}\grammar{caus}\end{déclaration}\end{définition}
\begin{définition}\fra faire écrire, dessiner\end{définition}
\begin{définition}\cmn 请……写字\end{définition}
\end{sous-entrée}\end{entrée}

\begin{entrée}
\vedette{\hypertarget{Ⓔrɤtɕaʁ}{\papi{ rɤtɕaʁ}}}\markboth{rɤtɕaʁ}{}
\classe{vt}
\paradigme{\textit{dir :} \jya pɯ-}
\begin{définition}\fra fouler du pied\end{définition}
\begin{définition}\cmn 踩\end{définition}
\begin{exemple}\jya @cai ma-pɯ-tɯ-rɤtɕaʁ\cmn 你别踩菜\end{exemple}
\begin{exemple}\jya a-mi ma-pɯ-tɯ-rɤtɕaʁ\cmn 你别踩我的脚\end{exemple}
\begin{exemple}\jya mbro pɯ́-wɣ-rɤtɕaʁ-a\cmn 马踩到我了\end{exemple}\begin{sous-entrée}
\vedette{\hypertarget{}{\papi{ zrɤtɕaʁ}}}\markboth{zrɤtɕaʁ}{}
\begin{exemple}\jya wuma ʑo tɤ-mbɣom tɕe ɯ-thoʁ mɯ-pjɤ-zrɤtɕaʁ ʑo jo-rɟɯɣ\cmn 他很急,脚不着地地跑了过去\end{exemple}
\end{sous-entrée}\end{entrée}

\begin{entrée}
\vedette{\hypertarget{Ⓔrɤtɕɤβ}{\papi{ rɤtɕɤβ}}}\markboth{rɤtɕɤβ}{}
\begin{relation-sémantique}\confer{
\hyperlink{Ⓔtɕɤβ}{\textit{ \papi{tɕɤβ}}}
}\end{relation-sémantique}\end{entrée}

\begin{entrée}
\vedette{\hypertarget{Ⓔrɤtɕhaʁ}{\papi{ rɤtɕhaʁ}}}\markboth{rɤtɕhaʁ}{}\classe{vt}
\paradigme{\textit{dir :} \jya tɤ-}
\begin{définition}\fra attacher en enroulant\end{définition}
\begin{définition}\cmn 一把一把地捆起来\end{définition}\end{entrée}

\begin{entrée}
\vedette{\hypertarget{Ⓔrɤtɕɯmtɕaʁ}{\papi{ rɤtɕɯmtɕaʁ}}}\markboth{rɤtɕɯmtɕaʁ}{}
\classe{vt}
\paradigme{\textit{dir :} \jya pɯ-}
\begin{définition}\fra piétiner et écraser\end{définition}
\begin{définition}\cmn 乱踩(植物)\end{définition}
\begin{exemple}\jya tɯji ɯ-ŋgɯ ma-jɤ-tɯ-ɕe-nɯ ma, tɤ-rɤku tɯ-rɤtɕɯmtɕaʁ-nɯ, tɕe tu-wxti mɤ-cha\cmn 你们不要到地里去,会踩到庄稼,以后就长不大\end{exemple}
\begin{relation-sémantique}\confer{
\hyperlink{Ⓔrɤtɕaʁ}{\textit{ \papi{rɤtɕaʁ}}}
}\end{relation-sémantique}\end{entrée}

\begin{entrée}
\vedette{\hypertarget{Ⓔrɤtɣa}{\papi{ rɤtɣa}}}\markboth{rɤtɣa}{}
\classe{vt}
\paradigme{\textit{dir :} \jya nɯ-}
\paradigme{\textit{dir :} \jya thɯ-}
\begin{définition}\ 
\begin{déclaration}\grammar{denom}\end{déclaration}\end{définition}
\begin{définition}\fra mesurer avec deux doigts\end{définition}
\begin{définition}\cmn 用大拇指和食指量尺寸\end{définition}
\begin{exemple}\jya thɯ-rɤtɣa-t-a\cmn 我量了木头\end{exemple}
\begin{exemple}\jya tɯmbri nɯ-rɤtɣa-t-a\cmn 我量了绳子\end{exemple}
\begin{relation-sémantique}\synonyme{
\hyperlink{Ⓔrɤɟom}{\textit{ \papi{rɤɟom}}}
}\end{relation-sémantique}
\begin{relation-sémantique}\confer{
\hyperlink{Ⓔtɯ-tɣaⒽ1}{\textit{ \papi{tɯ-tɣa1}}}
}\end{relation-sémantique}\end{entrée}

\begin{entrée}
\vedette{\hypertarget{Ⓔrɤthu}{\papi{ rɤthu}}}\markboth{rɤthu}{}
\begin{relation-sémantique}\confer{
\hyperlink{ⒺthuⒽ1}{\textit{ \papi{thu1}}}
}\end{relation-sémantique}\end{entrée}

\begin{entrée}
\vedette{\hypertarget{Ⓔrɤtho}{\papi{ rɤtho}}}\markboth{rɤtho}{}\classe{vi}
\paradigme{\textit{dir :} \jya lɤ-}
\paradigme{\textit{dir :} \jya tɤ-}
\begin{définition}\fra pousser des bourgeons\end{définition}
\begin{définition}\cmn 长蓓蕾\end{définition}
\begin{relation-sémantique}\confer{
\hyperlink{Ⓔɯ-tho}{\textit{ \papi{ɯ-tho}}}
}\end{relation-sémantique}\end{entrée}

\begin{entrée}
\vedette{\hypertarget{Ⓔrɤthuthe}{\papi{ rɤthuthe}}}\markboth{rɤthuthe}{}\classe{vi}
\paradigme{\textit{dir :} \jya nɯ-}
\begin{définition}\fra demander la permission\end{définition}
\begin{définition}\cmn 征求……的同意\end{définition}
\begin{exemple}\jya a-ɕki ɣɯ-nɯ-rɤthuthe\cmn 我征求了我的同意\end{exemple}
\begin{exemple}\jya mɤ-kɤ-rɤthuthe kɯ jo-ɣi\cmn 他没有征求同意就来了\end{exemple}
\begin{relation-sémantique}\confer{
\hyperlink{ⒺthuⒽ1}{\textit{ \papi{thu}}}
}\end{relation-sémantique}\end{entrée}

\begin{entrée}
\vedette{\hypertarget{Ⓔrɤtsɣe}{\papi{ rɤtsɣe}}}\markboth{rɤtsɣe}{}
\classe{vi}
\paradigme{\textit{dir :} \jya tɤ-}
\paradigme{\textit{dir :} \jya nɯ-}
\begin{définition}\ 
\begin{déclaration}\grammar{apass}\end{déclaration}\end{définition}
\begin{définition}\fra vendre\end{définition}
\begin{définition}\cmn 卖\end{définition}
\begin{exemple}\jya tɤ-rɤtsɣe-tɕi\cmn 我们俩卖了东西\end{exemple}
\begin{exemple}\jya jiɕqha nɯ ɲɯ-rɤtsɣe\cmn 那个人在卖东西\end{exemple}
\begin{relation-sémantique}\confer{
\hyperlink{Ⓔntsɣe}{\textit{ \papi{ntsɣe}}}
}\end{relation-sémantique}\end{entrée}

\begin{entrée}
\vedette{\hypertarget{Ⓔrɤtshɤt}{\papi{ rɤtshɤt}}}\markboth{rɤtshɤt}{}\classe{vt}
\paradigme{\textit{dir :} \jya tɤ-}
\begin{définition}\fra essayer, comparer\end{définition}
\begin{définition}\cmn 试一下,比一下\end{définition}
\begin{exemple}\jya tɤ-rɤtshat-a\cmn 我试了一下\end{exemple}
\begin{exemple}\jya ɯʑo kɯ tɯ-ŋga ta-rɤtshɤt\cmn 他试了衣服\end{exemple}
\begin{relation-sémantique}\confer{
\hyperlink{ⒺtshɤtⒽ1}{\textit{ \papi{tshɤt1}}}
}\end{relation-sémantique}\end{entrée}

\begin{entrée}
\vedette{\hypertarget{Ⓔrɤtshɯɣ}{\papi{ rɤtshɯɣ}}}\markboth{rɤtshɯɣ}{}
\classe{vi}
\paradigme{\textit{dir :} \jya tɤ-}
\begin{définition}\fra faire avec modération\end{définition}
\begin{définition}\cmn 小心做,适当地做
\end{définition}
\begin{exemple}\jya cha kɤ-tshi tu-rɤtshɯɣ-a ɕti\cmn 我小心不要喝太多酒\end{exemple}\end{entrée}

\begin{entrée}
\vedette{\hypertarget{Ⓔrɤtʂɯβ}{\papi{ rɤtʂɯβ}}}\markboth{rɤtʂɯβ}{}
\begin{relation-sémantique}\confer{
\hyperlink{Ⓔtʂɯβ}{\textit{ \papi{tʂɯβ}}}
}\end{relation-sémantique}\end{entrée}

\begin{entrée}
\vedette{\hypertarget{Ⓔrɤwaŋ}{\papi{ rɤwaŋ}}}\markboth{rɤwaŋ}{}
\classe{n}
\begin{définition}\fra responsabilité\end{définition}
\begin{définition}\cmn 责任
\begin{déclaration} \étymologie{\papi{raŋ.dbaŋ}}\end{déclaration}\end{définition}
\begin{exemple}\jya ɯʑo to-ngo ri, aʑo a-rɤwaŋ pɯ-me\cmn 虽然他生病了,但是我没有责任\end{exemple}\end{entrée}

\begin{entrée}
\vedette{\hypertarget{Ⓔrɤwum}{\papi{ rɤwum}}}\markboth{rɤwum}{}
\classe{vt}
\paradigme{\textit{dir :} \jya tɤ-}
\begin{définition}\fra ranger, rassembler (des objets dispersés)\end{définition}
\begin{définition}\cmn 收拾;收拢\end{définition}
\begin{exemple}\jya laχtɕha tɤ-rɤwum\cmn 你收拾一下东西\end{exemple}
\begin{exemple}\jya tɯ-ŋga tɤ-rɤwum\cmn 你收拾一下衣服\end{exemple}
\begin{exemple}\jya fsapaʁ tɤ-rɤwum\cmn 你把牲畜赶回来\end{exemple}
\begin{exemple}\jya ɯ-ndo tɕe kɤ-rɤwum mɤ-kɯ-sɤcha ʑo ɲɯ-βze ŋu\cmn 到最后就不可收拾\end{exemple}
\begin{relation-sémantique}\confer{
\hyperlink{Ⓔwum}{\textit{ \papi{wum}}}
}\end{relation-sémantique}
\begin{relation-sémantique}\confer{
\hyperlink{Ⓔstɯm}{\textit{ \papi{stɯm}}}
}\end{relation-sémantique}\end{entrée}

\begin{entrée}
\vedette{\hypertarget{Ⓔrɤχpɯn}{\papi{ rɤχpɯn}}}\markboth{rɤχpɯn}{}\classe{vi}
\paradigme{\textit{dir :} \jya lɤ-}
\begin{définition}\ 
\begin{déclaration}\grammar{denom}\end{déclaration}\end{définition}
\begin{définition}\fra devenir moine\end{définition}
\begin{définition}\cmn 当和尚\end{définition}
\begin{exemple}\jya kɤ-rɤχpɯn mɯ́j-nɤlɤn\cmn 他不答应当和尚\end{exemple}
\begin{relation-sémantique}\confer{
\hyperlink{Ⓔχpɯn}{\textit{ \papi{χpɯn}}}
}\end{relation-sémantique}
\begin{relation-sémantique}\confer{
\hyperlink{Ⓔnɯχpɯn}{\textit{ \papi{nɯχpɯn}}}
}\end{relation-sémantique}\end{entrée}

\begin{entrée}
\vedette{\hypertarget{Ⓔrɤzboʁ}{\papi{ rɤzboʁ}}}\markboth{rɤzboʁ}{}
\classe{vt}
\paradigme{\textit{dir :} \jya tɤ-}
\begin{définition}\ 
\begin{déclaration}\grammar{denom}\end{déclaration}\end{définition}
\begin{définition}\fra prendre par poignées\end{définition}
\begin{définition}\cmn 一把一把地拿\end{définition}
\begin{exemple}\jya sɯjno tɤ-rɤzboʁ-a\cmn 我一把一把地拿了草\end{exemple}
\begin{relation-sémantique}\synonyme{
\hyperlink{Ⓔrɤspra}{\textit{ \papi{rɤspra}}}
}\end{relation-sémantique}
\begin{relation-sémantique}\confer{
\hyperlink{Ⓔtɯ-zboʁ}{\textit{ \papi{tɯ-zboʁ}}}
}\end{relation-sémantique}\end{entrée}

\begin{entrée}
\vedette{\hypertarget{Ⓔrɤzda}{\papi{ rɤzda}}}\markboth{rɤzda}{}
\classe{vt}
\paradigme{\textit{dir :} \jya tɤ-}
\begin{définition}\fra saluer (avant le départ)\end{définition}
\begin{définition}\cmn 打招呼\end{définition}
\begin{exemple}\jya ɕ-ta-rɤzda\cmn 他向他打了招呼\end{exemple}
\begin{exemple}\jya tɤ-tɯ-rɤŋgat tɕe, ɯʑo ɕ-tɤ-rɤzde je\cmn 你要出发的时候,给他打个招呼\end{exemple}
\begin{exemple}\jya tɯ-ɕe tɤ-mda tɕe, a-ɣɯ-tɤ-kɯ-rɤzda-a je\cmn 当你要走的时候,来跟我打一声招呼\end{exemple}
\begin{relation-sémantique}\confer{
\hyperlink{Ⓔnɤzda}{\textit{ \papi{nɤzda}}}
}\end{relation-sémantique}
\begin{relation-sémantique}\confer{
\hyperlink{Ⓔɣɤzda}{\textit{ \papi{ɣɤzda}}}
}\end{relation-sémantique}
\begin{relation-sémantique}\confer{
\hyperlink{Ⓔtɯ-zda}{\textit{ \papi{tɯ-zda}}}
}\end{relation-sémantique}\end{entrée}

\begin{entrée}
\vedette{\hypertarget{Ⓔrɤzga}{\papi{ rɤzga}}}\markboth{rɤzga}{}
\classe{vi}
\paradigme{\textit{dir :} \jya kɤ-}
\begin{définition}\ 
\begin{déclaration}\grammar{denom}\end{déclaration}\end{définition}
\begin{définition}\fra faire du miel\end{définition}
\begin{définition}\cmn 酿蜜\end{définition}
\begin{exemple}\jya ɣʑo ɲɯ-rɤzga\cmn 蜜蜂在酿蜜\end{exemple}
\begin{exemple}\jya jiɕqha tɯrme nɯ kɯ-rɯkɯŋu kɯ ɣʑo kɯ-rɤzga ʑo ɲɯ-fse\cmn 这个人把所有的财物都带回家,跟采蜜的蜜蜂一样\end{exemple}
\begin{relation-sémantique}\confer{
\hyperlink{ⒺzgaⒽ2}{\textit{ \papi{zga2}}}
}\end{relation-sémantique}\end{entrée}

\begin{entrée}
\vedette{\hypertarget{Ⓔrɤzgɤr}{\papi{ rɤzgɤr}}}\markboth{rɤzgɤr}{}
\classe{vi}
\paradigme{\textit{dir :} \jya thɯ-}
\begin{définition}\ 
\begin{déclaration}\grammar{denom}\end{déclaration}\end{définition}\acception{1}
\begin{définition}\fra planter une tente\end{définition}
\begin{définition}\cmn 搭帐篷\end{définition}
\begin{exemple}\jya staχpɯ cho-rɤzgɤr\cmn 豌豆到处蔓延(在其他地方“搭帐篷”)\end{exemple}\acception{2}
\begin{définition}\fra réparer une tente\end{définition}
\begin{définition}\cmn 缝帐篷\end{définition}
\begin{exemple}\jya iʑora kɯre ku-rɤzgɤr-i\cmn 我们在缝帐篷\end{exemple}
\begin{relation-sémantique}\confer{
\hyperlink{Ⓔzgɤr}{\textit{ \papi{zgɤr}}}
}\end{relation-sémantique}\end{entrée}

\begin{entrée}
\vedette{\hypertarget{Ⓔrɤznde}{\papi{ rɤznde}}}\markboth{rɤznde}{}
\classe{vi}
\paradigme{\textit{dir :} \jya tɤ-}
\begin{définition}\ 
\begin{déclaration}\grammar{apass}\end{déclaration}\end{définition}
\begin{définition}\fra construire en empilant\end{définition}
\begin{définition}\cmn 垒起来\end{définition}
\begin{exemple}\jya tɤ-rɤznde-a (=znde tɤ-βzu-t-a)\cmn 我修了墙\end{exemple}
\begin{relation-sémantique}\confer{
\hyperlink{ⒺzndeⒽ1}{\textit{ \papi{znde1}}}
}\end{relation-sémantique}
\begin{relation-sémantique}\confer{
\hyperlink{ⒺzndeⒽ2}{\textit{ \papi{znde2}}}
}\end{relation-sémantique}\end{entrée}

\begin{entrée}
\vedette{\hypertarget{Ⓔrɤʑa}{\papi{ rɤʑa}}}\markboth{rɤʑa}{}
\classe{vs}
\paradigme{\textit{dir :} \jya tɤ-}
\begin{définition}\fra gratter\end{définition}
\begin{définition}\cmn 痒\end{définition}
\begin{exemple}\jya ɯ-mgɯr ɲɯ-rɤʑa\cmn 他背部很痒\end{exemple}
\begin{exemple}\jya βɣɤrtshi kɯ tu-kɯ-ndza tɕe ɯ-sta ɲɯ-rɤʑa\cmn 被蚊子咬就很痒\end{exemple}\end{entrée}

\begin{entrée}
\vedette{\hypertarget{Ⓔrɤʑi}{\papi{ rɤʑi}}}\markboth{rɤʑi}{} (\variante{rɤʑit}) 
\classe{vi}
\paradigme{\textit{dir :} \jya kɤ-}\acception{1}
\begin{définition}\fra rester\end{définition}
\begin{définition}\cmn 留下\end{définition}
\begin{exemple}\jya nɯtɕu ko-rɤʑi\cmn 他留在那里了\end{exemple}
\begin{exemple}\jya kɤ-rɤʑit-a\cmn 我呆了\end{exemple}
\begin{exemple}\jya ʁnɯ-sla pɯ-rɤʑi-nɯ\cmn 他们呆了两个月\end{exemple}\acception{2}
\begin{définition}\fra se trouver à un certain endroit\end{définition}
\begin{définition}\cmn 在某个地方\end{définition}
\begin{exemple}\jya ŋotɕu ku-tɯ-rɤʑi?\cmn 你在哪里?\end{exemple}\begin{sous-entrée}
\vedette{\hypertarget{}{\papi{ nɯrɤʑi}}}\markboth{nɯrɤʑi}{}
\begin{définition}\ 
\begin{déclaration}\grammar{autoben}\end{déclaration}\end{définition}
\begin{définition}\fra se reposer chez soi\end{définition}
\begin{définition}\cmn 在家休息\end{définition}
\begin{exemple}\jya jisŋi kɤ-nɯrɤʑi\cmn 今天呆在家里吧\end{exemple}
\begin{exemple}\jya kɤ-nɯrɤʑi-a\cmn 我呆在家里了\end{exemple}
\begin{exemple}\jya ʁnɯ-sŋi, χsɯ-sŋi jamar ma mɤ-rɤʑi\cmn 他只待两三天(就回来)\end{exemple}
\begin{relation-sémantique}\confer{
\hyperlink{Ⓔzrɤʑi}{\textit{ \papi{zrɤʑi}}}
}\end{relation-sémantique}
\end{sous-entrée}\begin{sous-entrée}
\vedette{\hypertarget{}{\papi{ ɯ-taʁ,rɤʑi}}}\markboth{ɯ-taʁ,rɤʑi}{}
\begin{définition}\fra dépendre de ... pour vivre\end{définition}
\begin{définition}\cmn 靠……维持生活\end{définition}
\begin{exemple}\jya lu-rɤji, paʁ pjɯ-χse, nɯnɯ ɯ-taʁ ku-rɤʑi ɕti kɯmaʁ ɯ-phoʁ kɯ-tu me\cmn 他靠种地和喂猪维持生活,没有其它收入来源\end{exemple}
\end{sous-entrée}\end{entrée}

\begin{entrée}
\vedette{\hypertarget{Ⓔrbɤrbɤt}{\papi{ rbɤrbɤt}}}\markboth{rbɤrbɤt}{}\classe{idph.2}
\begin{définition}\fra ordonné\end{définition}
\begin{définition}\cmn 形容排列整齐的样子\end{définition}
\begin{exemple}\jya tɯrme ra rbɤrbɤt ʑo ɲɯ-ndzur-nɯ\cmn 人站得很整齐\end{exemple}
\begin{exemple}\jya khɯtsa ɲɯ-ɤʑɯrja rbɤrbɤt ʑo\cmn 碗摆得很整齐\end{exemple}\end{entrée}

\begin{entrée}
\vedette{\hypertarget{Ⓔrboʁrboʁ}{\papi{ rboʁrboʁ}}}\markboth{rboʁrboʁ}{}\classe{idph.2}
\begin{définition}\fra en grappe\end{définition}
\begin{définition}\cmn 形容果子等集中、不分散的样子\end{définition}
\begin{exemple}\jya @putao ɣɯ ɯ-mat nɯ rboʁrboʁ kɯ-pa ʑo ko-tshoʁ\cmn 葡萄结得又多又密\end{exemple}\end{entrée}

\begin{entrée}
\vedette{\hypertarget{Ⓔrbɯɣrbɯɣ}{\papi{ rbɯɣrbɯɣ}}}\markboth{rbɯɣrbɯɣ}{}\classe{idph.2}
\begin{définition}\fra l'un a la suite de l'autre, très nombreux (graines dans une cosse)\end{définition}
\begin{définition}\cmn 形容果子很多,一个接着一个\end{définition}
\begin{exemple}\jya staχpɯcɤβ rbɯɣrbɯɣ ʑo chɤ-bzu\cmn 豌豆长了很多果子\end{exemple}\end{entrée}

\begin{entrée}
\vedette{\hypertarget{Ⓔrca}{\papi{ rca}}}\markboth{rca}{}
\classe{interj}
\begin{définition}\fra expression d'une surprise, (oui, bien sûr)\end{définition}
\begin{définition}\cmn 表示惊叹语气\end{définition}
\begin{exemple}\jya wo, ɲɯ-ŋu rca!\cmn 对,是的!\end{exemple}\end{entrée}

\begin{entrée}
\vedette{\hypertarget{Ⓔrcánɯ}{\papi{ rcánɯ}}}\markboth{rcánɯ}{}\classe{adv}
\begin{définition}\fra exprime la surprise; utilisé avec l'irréel, exprime qu'il n'y a pas d'espoir pour que l'action se réalise\end{définition}
\begin{définition}\cmn 表示惊叹语气的助词(和非实然式合用时表示“没有希望实现”)\end{définition}
\begin{exemple}\jya nɯ rcánɯ ɲɯ-pe ko\cmn 这倒是很好\end{exemple}
\begin{exemple}\jya ɯʑo rcánɯ tɯ-tʂɯβ ri chɯ-βze, ɕoŋβzu ri ɲɯ-βze, smɤnba ri tu-βze, mɤ-spe ʑo me\cmn 他又搞裁缝又做木工还要行医,什么都会\end{exemple}
\begin{exemple}\jya nɤʑo rcánɯ ci lɤ-tɯ-nɤrʑaʁ\cmn 你倒是呆了很久\end{exemple}\end{entrée}

\begin{entrée}
\vedette{\hypertarget{Ⓔrcaŋ}{\papi{ rcaŋ}}}\markboth{rcaŋ}{}
\classe{idph.1}
\begin{définition}\fra tomber sur le dos\end{définition}
\begin{définition}\cmn 仰着跌倒的样子\end{définition}
\begin{exemple}\jya rcaŋ ʑo kɤ-rŋgɯ\cmn 他仰着躺下了\end{exemple}
\begin{exemple}\jya rcaŋ ʑo pɯ-ndʐaβa\cmn 我仰着跌倒了\end{exemple}\begin{sous-entrée}
\vedette{\hypertarget{}{\papi{ ɣɤrcaŋlaŋ}}}\markboth{ɣɤrcaŋlaŋ}{}\classe{vi}
\begin{exemple}\jya ɲɯ-ɣɤrcaŋlaŋ ntsɯ\cmn 他脚不停地在动,坐不住\end{exemple}
\end{sous-entrée}\end{entrée}

\begin{entrée}
\vedette{\hypertarget{Ⓔrcaŋpɕaʁ}{\papi{ rcaŋpɕaʁ}}}\markboth{rcaŋpɕaʁ}{}
\classe{n}
\begin{définition}\fra prosternations le long d'une route jusqu'à un lieu saint\end{définition}
\begin{définition}\cmn 朝圣;磕长头
\begin{déclaration} \étymologie{\papi{skʲaŋ.pʰʲag}}\end{déclaration}\end{définition}
\begin{exemple}\jya ɬasa rcaŋpɕaʁ to-tsɯm\cmn 他去拉萨朝圣\end{exemple}
\begin{relation-sémantique}\confer{
\hyperlink{Ⓔrɯrcaŋpɕaʁ}{\textit{ \papi{rɯrcaŋpɕaʁ}}}
}\end{relation-sémantique}\end{entrée}

\begin{entrée}
\vedette{\hypertarget{Ⓔrcaχtoŋ}{\papi{ rcaχtoŋ}}}\markboth{rcaχtoŋ}{}\classe{intj}
\begin{définition}\fra mange de la merde !\end{définition}
\begin{définition}\cmn 一句脏话
\begin{déclaration} \étymologie{\papi{skʲag.gtoŋ}}\end{déclaration}\end{définition}\end{entrée}

\begin{entrée}
\vedette{\hypertarget{Ⓔrcɤlpa}{\papi{ rcɤlpa}}}\markboth{rcɤlpa}{}\classe{n}
\begin{définition}\fra estomac de bovidé\end{définition}
\begin{définition}\cmn 牛胃
\begin{déclaration} \étymologie{\papi{rkʲal.pa}}\end{déclaration}\end{définition}
\end{entrée}

\begin{entrée}
\vedette{\hypertarget{Ⓔrcɤmbeŋga}{\papi{ rcɤmbeŋga}}}\markboth{rcɤmbeŋga}{}
\classe{n}
\begin{définition}\fra personne qui porte une veste usée\end{définition}
\begin{définition}\cmn 穿破旧皮袄的人
\end{définition}\end{entrée}

\begin{entrée}
\vedette{\hypertarget{Ⓔrcharcha}{\papi{ rcharcha}}}\markboth{rcharcha}{} (\variante{rcarca}) 
\classe{idph.2}
\begin{définition}\fra souriant\end{définition}
\begin{définition}\cmn 笑嘻嘻
\begin{déclaration}\use{多指小孩子}\end{déclaration}\end{définition}
\begin{exemple}\jya tɤ-pɤtso rcharcha ɲɯ-nɤre\cmn 小孩子是笑嘻嘻的\end{exemple}\end{entrée}

\begin{entrée}
\vedette{\hypertarget{Ⓔrchɤrchɤt}{\papi{ rchɤrchɤt}}}\markboth{rchɤrchɤt}{}\classe{idph.2}
\begin{définition}\fra cliquetis\end{définition}
\begin{définition}\cmn 敲玻璃、铁皮的叮当声\end{définition}\begin{sous-entrée}
\vedette{\hypertarget{}{\papi{ ɣɤrchɤrchɤt}}}\markboth{ɣɤrchɤrchɤt}{}\classe{vi}
\end{sous-entrée}\begin{sous-entrée}
\vedette{\hypertarget{}{\papi{ sɤrchɤrchɤt}}}\markboth{sɤrchɤrchɤt}{}\classe{vt}
\begin{exemple}\jya tɤ-pɤtso kɯ ɯ-kɯmtɕhɯ ɲɯ-ɤz-nɤkhɯkhrɯt ɲɯ-sɤrchɤrchɤt\cmn 小孩子把玩具拖着,发出叮当声\end{exemple}
\begin{exemple}\jya tɤmɯmɯm ɲɯ-sɤrchɤrchɤt\cmn 铃铛发出叮当声\end{exemple}
\begin{relation-sémantique}\confer{
\hyperlink{Ⓔrkhɯβrkhɯβ}{\textit{ \papi{rkhɯβrkhɯβ}}}
}\end{relation-sémantique}
\end{sous-entrée}\end{entrée}

\begin{entrée}
\vedette{\hypertarget{Ⓔrchɯɣnɤlɯɣ}{\papi{ rchɯɣnɤlɯɣ}}}\markboth{rchɯɣnɤlɯɣ}{}
\begin{relation-sémantique}\confer{
\hyperlink{Ⓔrchɯɣrchɯɣ}{\textit{ \papi{rchɯɣrchɯɣ}}}
}\end{relation-sémantique}\end{entrée}

\begin{entrée}
\vedette{\hypertarget{Ⓔrchɯɣrchɯɣ}{\papi{ rchɯɣrchɯɣ}}}\markboth{rchɯɣrchɯɣ}{}\classe{idph.2}\acception{1}
\begin{définition}\fra ridé\end{définition}
\begin{définition}\cmn 形容皱纹很多的样子\end{définition}
\begin{exemple}\jya ɯ-rŋa tɤrʑɯɣ rchɯɣrchɯɣ ʑo to-βzu\cmn 他脸上长了很多皱纹\end{exemple}\acception{2}
\begin{définition}\fra beaucoup de ... ensemble\end{définition}
\begin{définition}\cmn 形容很多(人;动物)在一起\end{définition}
\begin{exemple}\jya rchɯɣrchɯɣ ʑo ku-tɯ-rɤʑi-nɯ ma mɯ́j-tɯ-rɤma-nɯ\cmn 你们在这里这么多人,什么都不做\end{exemple}
\begin{sous-entrée}
\vedette{\hypertarget{}{\papi{ ɣɤrchɯɣlɯɣ}}}\markboth{ɣɤrchɯɣlɯɣ}{}\classe{vi}
\begin{définition}\fra rigoler ensemble\end{définition}
\begin{définition}\cmn 嬉皮笑脸;说说笑笑\end{définition}
\begin{exemple}\jya jiɕqha tɯrme ni ɲɯ-ɣɤrchɯɣlɯɣ-ndʑi\cmn 这两个人在说说笑笑\end{exemple}
\end{sous-entrée}\begin{sous-entrée}
\vedette{\hypertarget{}{\papi{ ɣɤrchɯɣrchɯɣ}}}\markboth{ɣɤrchɯɣrchɯɣ}{}\classe{vt}
\paradigme{\textit{dir :} \jya tɤ-}
\begin{définition}\fra s'entrechoquer en faisant du bruit (objets durs)\end{définition}
\begin{définition}\cmn 硬的东西相碰发出声音\end{définition}
\begin{exemple}\jya jiɕqha mkhɯrlu nɯ ɲɯ-ɣɤrchɯɣrchɯɣ\cmn 很多转经筒同时转动发出声音\end{exemple}
\begin{relation-sémantique}\confer{
\hyperlink{Ⓔɣɤrqhɯβrqhɯβ}{\textit{ \papi{ɣɤrqhɯβrqhɯβ}}}
}\end{relation-sémantique}
\end{sous-entrée}\begin{sous-entrée}
\vedette{\hypertarget{}{\papi{ nɯrchɯrchɯɣ}}}\markboth{nɯrchɯrchɯɣ}{}\classe{vt}
\paradigme{\textit{dir :} \jya nɯ-}
\begin{définition}\fra dire d'une même voix\end{définition}
\begin{définition}\cmn 异口同声\end{définition}
\begin{définition}\fra bruit de deux objets durs qui s'entrechoquent\end{définition}
\begin{définition}\cmn 形容硬的东西相撞时发出的声音\end{définition}
\begin{exemple}\jya ɲɤ-nɯrchɯrchɯɣ-ndʑi ʑo tɕe "aʑo ɲɯ-ɣɤŋgi" ɲɯ-ti-ndʑi\cmn 她们俩异口同声地说我是对的\end{exemple}
\end{sous-entrée}\begin{sous-entrée}
\vedette{\hypertarget{}{\papi{ sɤrchɯɣlɯɣ}}}\markboth{sɤrchɯɣlɯɣ}{}\classe{vt}
\paradigme{\textit{dir :} \jya tɤ-}
\begin{exemple}\jya ɕɤrɯ ra ɲɯ-sɤrchɯɣlɯɣ-nɯ ntsɯ\cmn 不停地令骨头相撞发出声音ɣ\end{exemple}
\end{sous-entrée}\end{entrée}

\begin{entrée}
\vedette{\hypertarget{Ⓔrcirci}{\papi{ rcirci}}}\markboth{rcirci}{} (\variante{rchirchi}) 
\classe{idph.2}
\begin{définition}\fra être souriant\end{définition}
\begin{définition}\cmn 形容笑得很高兴的模样\end{définition}
\begin{exemple}\jya rchirchi ɲɯ-nɤre\cmn 他笑得很高兴\end{exemple}
\begin{exemple}\jya ɯʑo ɲɯ-rga tɕe rcirci ʑo ɲɯ-ʑɣɤstu\cmn 因为他很高兴,笑得牙齿都露出来了\end{exemple}
\begin{relation-sémantique}\confer{
\hyperlink{Ⓔrcharcha}{\textit{ \papi{rcharcha}}}
}\end{relation-sémantique}
\begin{relation-sémantique}\confer{
\hyperlink{Ⓔmɯɣmɯɣ}{\textit{ \papi{mɯɣmɯɣ}}}
}\end{relation-sémantique}\end{entrée}

\begin{entrée}
\vedette{\hypertarget{Ⓔrcɯɣrcɯɣ}{\papi{ rcɯɣrcɯɣ}}}\markboth{rcɯɣrcɯɣ}{}\classe{idph.2}
\begin{définition}\fra rire sans s'arrêter\end{définition}
\begin{définition}\cmn 形容笑个不停的样子\end{définition}
\begin{exemple}\jya mɤ-tɯ-ɣɤrcɯɣrcɯɣ\cmn 你不要笑个不停(女孩子一起说说笑笑时这么说)\end{exemple}\end{entrée}

\begin{entrée}
\vedette{\hypertarget{Ⓔrɕɯβrɕɯβ}{\papi{ rɕɯβrɕɯβ}}}\markboth{rɕɯβrɕɯβ}{}
\classe{idph.2}
\begin{définition}\fra rugueux\end{définition}
\begin{définition}\cmn 形容粗糙(像砂纸一样),快要脱皮的样子\end{définition}
\begin{exemple}\jya tɯrgi ɯ-rqhu rɕɯβrɕɯβ ʑo ɲɯ-pa\cmn 杉树的树皮很粗糙,快要脱皮的样子\end{exemple}
\begin{relation-sémantique}\confer{
\hyperlink{Ⓔsɤrɕɯβrɕɯβ}{\textit{ \papi{sɤrɕɯβrɕɯβ}}}
}\end{relation-sémantique}\end{entrée}

\begin{entrée}
\vedette{\hypertarget{Ⓔrda}{\papi{ rda}}}\markboth{rda}{}
\classe{n}
\begin{définition}\fra signe\end{définition}
\begin{définition}\cmn 信号;兆头
\begin{déclaration} \étymologie{\papi{brda}}\end{déclaration}\end{définition}
\begin{exemple}\jya rda to-lɤt (to-ʑmbri)\cmn 他敲钟了\end{exemple}
\begin{exemple}\jya ʁmaʁmi ra kɤ-rɤru tɤ-mda tɕe ɯ-rda tu-lɤt-nɯ ŋgrɤl\cmn 军人起床的时间到了就吹起床号\end{exemple}\end{entrée}

\begin{entrée}
\vedette{\hypertarget{Ⓔrdardɯl}{\papi{ rdardɯl}}}\markboth{rdardɯl}{}
\classe{n}
\begin{définition}\ 
\begin{déclaration}\grammar{n.rdpl}\end{déclaration}\end{définition}
\begin{définition}\fra poussière\end{définition}
\begin{définition}\cmn 灰尘\end{définition}
\begin{relation-sémantique}\confer{
\hyperlink{Ⓔrdɯl}{\textit{ \papi{rdɯl}}}
}\end{relation-sémantique}\end{entrée}

\begin{entrée}
\vedette{\hypertarget{Ⓔrdɤβ}{\papi{ rdɤβ}}}\markboth{rdɤβ}{}
\classe{vi}
\paradigme{\textit{dir :} \jya thɯ-}
\begin{définition}\fra échouer en affaire\end{définition}
\begin{définition}\cmn 做亏本生意
\begin{déclaration} \étymologie{\papi{rdab.tɕʰal}}\end{déclaration}\end{définition}
\begin{exemple}\jya thɯ-rdaβ-a, thɯ-rdɤβ\cmn 我亏本,他亏本\end{exemple}
\begin{exemple}\jya jɤxtshi tɯtsɣe thɯ-rdaβ-a\cmn 我这一次做了亏本生意\end{exemple}\end{entrée}

\begin{entrée}
\vedette{\hypertarget{Ⓔrdɤβzu}{\papi{ rdɤβzu}}}\markboth{rdɤβzu}{}\classe{n}
\begin{définition}\fra maçon\end{définition}
\begin{définition}\cmn 石匠
\begin{déclaration} \étymologie{\papi{rdo.bzu}}\end{déclaration}\end{définition}
\begin{exemple}\jya rdɤβzu ŋu-a\cmn 我是石匠\end{exemple}
\begin{relation-sémantique}\confer{
\hyperlink{Ⓔrɯrdɤβzu}{\textit{ \papi{rɯrdɤβzu}}}
}\end{relation-sémantique}\end{entrée}

\begin{entrée}
\vedette{\hypertarget{Ⓔrdɤl}{\papi{ rdɤl}}}\markboth{rdɤl}{}
\classe{vi}
\paradigme{\textit{dir :} \jya kɤ-}
\begin{définition}\fra aller trop loin\end{définition}
\begin{définition}\cmn 走过头\end{définition}
\begin{exemple}\jya nɯtɕu ku-rɤʑi pɯ-ra ri, kɤ-rdɤl\cmn 他本应该在这里停,但是他走过头了\end{exemple}
\begin{exemple}\jya ɯʑo ɕɯŋgɯ ko-rdal-a\cmn 我不小心走过他了\end{exemple}
\begin{exemple}\jya aʑo kɤ-nɯʑɯβ ko-rdal-a\cmn 我睡过头了\end{exemple}
\begin{exemple}\jya tɯtshot ko-rdɤl\cmn 时间已经超了\end{exemple}\begin{sous-entrée}
\vedette{\hypertarget{}{\papi{ sɯrdɤl}}}\markboth{sɯrdɤl}{}
\paradigme{\textit{dir :} \jya kɤ-}
\begin{exemple}\jya tɯtshot ko-sɯrdɤl-tɕi\cmn 我们已经超了时间了\end{exemple}\classe{vt}
\end{sous-entrée}\end{entrée}

\begin{entrée}
\vedette{\hypertarget{Ⓔrdɤmbɯm}{\papi{ rdɤmbɯm}}}\markboth{rdɤmbɯm}{}
\classe{n}
\begin{définition}\fra tas de pierre, symbole bouddhique\end{définition}
\begin{définition}\cmn 马尼堆;敖包
\begin{déclaration} \étymologie{\papi{rdo.ⁿbum}}\end{déclaration}\end{définition}
\begin{exemple}\jya rdɤmbɯm nɯ rdɤstaʁ ʁɟa ʑo kɤ-kɤ-rmbɯ ŋu tɕeri kú-wɣ-sɯ-ɤrtɯm tɕe ɯ-pɕi nɯ znde tú-wɣ-βzu ra. nɯ nɯ tɯ-ɟom ro ro jamar kɯ-mbro tɕe nɯ ɯ-taʁ nɯ tɕu nɯ sɤz kɯ-xtɕi tsa znde kú-wɣ-sɯ-ɤrtɯm ra tɕe nɯ ɯ-taʁ tɕe rdɤstaʁ kú-wɣ-rmbɯ tɕe rdɤstaʁ kɯ-wɣrum ɯ-qapi a-pɯ-dɤn tɕe pe. tɕe nɯ tɯrme tsuku kɯ rdɤmbɯm tu-ti-nɯ tsuku kɯ tshoko tu-ti-nɯ ŋu.\cmn 
敖包全部是用石头堆成的,但是要把它弄成圆形,在外面砌墙,要砌得一庹多高,在(石堆)上面再砌小一点的圆墙。在上面堆上石头,白石头多一点就好。有的人把敖包叫作\stylefv{rdɤmbɯm},有的叫\stylefv{tshoko}。
\end{exemple}\end{entrée}

\begin{entrée}
\vedette{\hypertarget{Ⓔrdɤrkɤz}{\papi{ rdɤrkɤz}}}\markboth{rdɤrkɤz}{}\classe{n}
\begin{définition}\fra gravure sur pierre\end{définition}
\begin{définition}\cmn 石刻
\begin{déclaration} \étymologie{\papi{rdo.brkos}}\end{déclaration}\end{définition}\end{entrée}

\begin{entrée}
\vedette{\hypertarget{Ⓔrdɤstaʁ}{\papi{ rdɤstaʁ}}}\markboth{rdɤstaʁ}{}
\classe{n}
\begin{définition}\fra pierre\end{définition}
\begin{définition}\cmn 石头
\begin{déclaration} \étymologie{\papi{rdo}}\end{déclaration}\end{définition}
\begin{relation-sémantique}\confer{
\hyperlink{Ⓔnɯrdɤstaʁ}{\textit{ \papi{nɯrdɤstaʁ}}}
}\end{relation-sémantique}\end{entrée}

\begin{entrée}
\vedette{\hypertarget{Ⓔrdom}{\papi{ rdom}}}\markboth{rdom}{}
\classe{vi}
\paradigme{\textit{dir :} \jya nɯ-}
\begin{définition}\fra vagabonder\end{définition}
\begin{définition}\cmn 流浪(带贬义)\end{définition}
\begin{exemple}\jya ɲɯ-rdom\cmn 他在流浪\end{exemple}
\begin{exemple}\jya kɯ-nɤphɯphɯ aʁɤndɯndɤt ɲɯ-rdom\cmn 乞丐到处流浪\end{exemple}\end{entrée}

\begin{entrée}
\vedette{\hypertarget{Ⓔrdoŋ}{\papi{ rdoŋ}}}\markboth{rdoŋ}{}
\classe{vt}
\paradigme{\textit{dir :} \jya pɯ-}
\begin{définition}\fra battre\end{définition}
\begin{définition}\cmn 打; 夯土墙
\begin{déclaration} \étymologie{\papi{rduŋ}}\end{déclaration}\end{définition}
\begin{exemple}\jya nɤ-stu tɤ-fse ma ta-rdoŋ\cmn 你表现好一点,不然就打你\end{exemple}
\begin{exemple}\jya caŋ pɯ-rdoŋ-a\cmn 我捶了墙\end{exemple}
\begin{exemple}\jya caŋ pa-rdoŋ\cmn 他捶了墙\end{exemple}
\begin{relation-sémantique}\synonyme{
\hyperlink{Ⓔʁndɯ}{\textit{ \papi{ʁndɯ}}}
}\end{relation-sémantique}\end{entrée}

\begin{entrée}
\vedette{\hypertarget{Ⓔrdɯl}{\papi{ rdɯl}}}\markboth{rdɯl}{}
\classe{n}
\begin{définition}\fra poussière\end{définition}
\begin{définition}\cmn 尘土
\begin{déclaration} \étymologie{\papi{rdul}}\end{déclaration}\end{définition}\end{entrée}

\begin{entrée}
\vedette{\hypertarget{Ⓔrdzardza}{\papi{ rdzardza}}}\markboth{rdzardza}{}\classe{idph.2}\acception{1}
\begin{définition}\fra qui n'accepte pas les critiques\end{définition}
\begin{définition}\cmn 形容爱顶嘴的样子\end{définition}
\begin{exemple}\jya a-phe kɯ-nɯkhɤja ci rdzardza ɲɯ-tɯ-ŋu\cmn 你跟我顶嘴\end{exemple}
\begin{exemple}\jya nɤki tɤ-pɤtso nɯ taʁndo mɤ-tso tɕe rdzardza ʑo tu-ʑɣɤstu ɕti\cmn 那个孩子不听话,总是爱顶嘴\end{exemple}\acception{2}
\begin{définition}\fra plein de vie (petite plante)\end{définition}
\begin{définition}\cmn 形容神气勃勃的样子(小植物)\end{définition}\begin{sous-entrée}
\vedette{\hypertarget{}{\papi{ rdzanɤrdza}}}\markboth{rdzanɤrdza}{}\classe{idph.3}
\begin{définition}\fra qui se battent (animaux)\end{définition}
\begin{définition}\cmn 形容互相打架的样子\end{définition}
\begin{exemple}\jya paχtsa ni ɲɯ-ɤlɯlɤt-ndʑi rdzanɤrdza ɲɯ-ʑɣɤstu-ndʑi\cmn 两个猪仔在打架\end{exemple}
\begin{exemple}\jya ki khɯna ki rdzanɤrdza kɯ-ŋɤn (kɯ-ftsɤn) ci ɲɯ-ŋu\cmn 这条狗很凶\end{exemple}
\end{sous-entrée}\end{entrée}

\begin{entrée}
\vedette{\hypertarget{Ⓔrdzari}{\papi{ rdzari}}}\markboth{rdzari}{}\classe{n}
\begin{définition}\fra sommet de la montagne\end{définition}
\begin{définition}\cmn 最高山上
\begin{déclaration} \étymologie{\papi{rdza.ri}}\end{déclaration}\end{définition}
\end{entrée}

\begin{entrée}
\vedette{\hypertarget{Ⓔrdzɯrdzi}{\papi{ rdzɯrdzi}}}\markboth{rdzɯrdzi}{}
\classe{idph.2}
\begin{définition}\fra pleine d'épines (plante)\end{définition}
\begin{définition}\cmn 有刺(植物)\end{définition}
\begin{exemple}\jya zɲɟa rdzɯrdzi ɲɯ-pa\cmn 黄刺泡儿有刺\end{exemple}
\begin{exemple}\jya rtɕhɯʁju rdzɯrdzi ɲɯ-pa\cmn 毛虫有毛\end{exemple}
\begin{relation-sémantique}\confer{
\hyperlink{Ⓔrzɯrzi}{\textit{ \papi{rzɯrzi}}}
}\end{relation-sémantique}\end{entrée}

\begin{entrée}
\vedette{\hypertarget{ⒺrgaⒽ1Ⓗ1}{\papi{ rga}}}\markboth{rga}{}\homonyme{1}\classe{vs}
\paradigme{\textit{dir :} \jya nɯ-}
\begin{définition}\fra être content\end{définition}
\begin{définition}\cmn 高兴\end{définition}
\begin{exemple}\jya aʑo ɲɯ-rga-a\cmn (听到这个消息)我很高兴\end{exemple}
\begin{exemple}\jya nɯ ma kɤ-nɯrga mɯ́j-khɯ ʑo ɲɯ-rga-a\cmn 我无比得高兴\end{exemple}\begin{sous-entrée}
\vedette{\hypertarget{}{\papi{ rga}}}\markboth{rga}{}\classe{vi-t}
\paradigme{\textit{dir :} \jya nɯ-}
\begin{définition}\fra aimer\end{définition}
\begin{définition}\cmn 喜欢
\begin{déclaration} \étymologie{\papi{dga}}\end{déclaration}\end{définition}
\begin{exemple}\jya kɤ-rɯɕmi mɯ́j-rga\cmn 他不爱说话\end{exemple}
\begin{relation-sémantique}\confer{
\hyperlink{Ⓔnɯrga}{\textit{ \papi{nɯrga}}}
}\end{relation-sémantique}
\begin{relation-sémantique}\confer{
\hyperlink{Ⓔɕɯrga}{\textit{ \papi{ɕɯrga}}}
}\end{relation-sémantique}
\begin{relation-sémantique}\confer{
\hyperlink{Ⓔnɯchɤrga}{\textit{ \papi{nɯchɤrga}}}
}\end{relation-sémantique}
\begin{relation-sémantique}\confer{
\hyperlink{Ⓔnɯɕmɯrga}{\textit{ \papi{nɯɕmɯrga}}}
}\end{relation-sémantique}
\end{sous-entrée}\end{entrée}

\begin{entrée}
\vedette{\hypertarget{Ⓔrgali}{\papi{ rgali}}}\markboth{rgali}{}
\classe{n}
\begin{définition}\fra génisse\end{définition}
\begin{définition}\cmn 小奶牛\end{définition}\end{entrée}

\begin{entrée}
\vedette{\hypertarget{Ⓔrgargɯn}{\papi{ rgargɯn}}}\markboth{rgargɯn}{}\classe{n}
\begin{définition}\ 
\begin{déclaration}\grammar{n.rdpl}\end{déclaration}\end{définition}
\begin{définition}\fra personne âgée\end{définition}
\begin{définition}\cmn 老年人
\begin{déclaration} \étymologie{\papi{dgan}}\end{déclaration}\end{définition}\end{entrée}

\begin{entrée}
\vedette{\hypertarget{Ⓔrgawa}{\papi{ rgawa}}}\markboth{rgawa}{}\classe{n}
\begin{définition}\fra cérémonie pour les morts\end{définition}
\begin{définition}\cmn 为死人办的仪式
\begin{déclaration} \étymologie{\papi{dga.ba}}\end{déclaration}\end{définition}
\end{entrée}

\begin{entrée}
\vedette{\hypertarget{Ⓔrgawɯ}{\papi{ rgawɯ}}}\markboth{rgawɯ}{}\classe{n}
\begin{définition}\fra navet séché\end{définition}
\begin{définition}\cmn 干了的芜菁根\end{définition}
\begin{relation-sémantique}\confer{
\hyperlink{Ⓔrɤjndoʁ}{\textit{ \papi{rɤjndoʁ}}}
}\end{relation-sémantique}\end{entrée}

\begin{entrée}
\vedette{\hypertarget{Ⓔrgɤl}{\papi{ rgɤl}}}\markboth{rgɤl}{}\classe{idph.1}
\begin{définition}\fra tou d'un coup\end{définition}
\begin{définition}\cmn 突然\end{définition}
\begin{exemple}\jya rgɤl ʑo pɯ-ndʐaβ\cmn 他突然间摔下去了\end{exemple}
\end{entrée}

\begin{entrée}
\vedette{\hypertarget{Ⓔrgɤm}{\papi{ rgɤm}}}\markboth{rgɤm}{}
\classe{n}
\begin{définition}\fra boîte\end{définition}
\begin{définition}\cmn 箱子
\begin{déclaration} \étymologie{\papi{sgam}}\end{déclaration}\end{définition}\end{entrée}

\begin{entrée}
\vedette{\hypertarget{Ⓔrgɤmpɯ}{\papi{ rgɤmpɯ}}}\markboth{rgɤmpɯ}{}\classe{n}
\begin{définition}\fra petite boîte\end{définition}
\begin{définition}\cmn 小盒子\end{définition}
\begin{relation-sémantique}\confer{
\hyperlink{Ⓔtɤ-pɯ}{\textit{ \papi{tɤ-pɯ}}}
}\end{relation-sémantique}\end{entrée}

\begin{entrée}
\vedette{\hypertarget{Ⓔrgɤnmɯ}{\papi{ rgɤnmɯ}}}\markboth{rgɤnmɯ}{}
\classe{n}
\begin{définition}\fra vieillarde\end{définition}
\begin{définition}\cmn 老太太
\begin{déclaration} \étymologie{\papi{rgan.mo}}\end{déclaration}\end{définition}\end{entrée}

\begin{entrée}
\vedette{\hypertarget{Ⓔrgɤtpu}{\papi{ rgɤtpu}}}\markboth{rgɤtpu}{}
\classe{n}
\begin{définition}\fra vieillard\end{définition}
\begin{définition}\cmn 老人
\begin{déclaration} \étymologie{\papi{rgad.po}}\end{déclaration}\end{définition}\end{entrée}

\begin{entrée}
\vedette{\hypertarget{Ⓔrgɤz}{\papi{ rgɤz}}}\markboth{rgɤz}{}
\classe{vs}
\paradigme{\textit{dir :} \jya thɯ-}
\begin{définition}\fra être vieux, vieillir\end{définition}
\begin{définition}\cmn 年老
\begin{déclaration} \étymologie{\papi{rgas}}\end{déclaration}\end{définition}
\begin{exemple}\jya jiɕqha rgɤtpu nɯ cho-rgɤz\cmn 那位老年人变老了\end{exemple}
\begin{exemple}\jya nɯŋa do chɤ-rgɤz\cmn 老奶牛变老了\end{exemple}
\begin{exemple}\jya kha to-rgɤz\cmn 房子老了\end{exemple}\begin{sous-entrée}
\vedette{\hypertarget{}{\papi{ ɣɤrgɤz}}}\markboth{ɣɤrgɤz}{}\classe{vs}
\begin{définition}\fra qui vieillit facilement\end{définition}
\begin{définition}\cmn 容易衰老\end{définition}
\begin{exemple}\jya tɤtho nɯ mɤ-ɣɤrgɤz ma anɯrŋi nɤ anɯrŋi ɕti, pjɯ-nɯɕɯrɲɟo mɤ-ŋgrɤl, ɯ-mdoʁ ɲɯ-nɤsci mɤ-ŋgrɤl\cmn 松树不容易衰老,一直都是绿的,不变色\end{exemple}
\end{sous-entrée}\end{entrée}

\begin{entrée}
\vedette{\hypertarget{Ⓔrgonma}{\papi{ rgonma}}}\markboth{rgonma}{}
\classe{n}
\begin{définition}\fra jument\end{définition}
\begin{définition}\cmn 母马
\begin{déclaration} \étymologie{\papi{rgod.ma}}\end{déclaration}\end{définition}\end{entrée}

\begin{entrée}
\vedette{\hypertarget{Ⓔrgoʁphɤr}{\papi{ rgoʁphɤr}}}\markboth{rgoʁphɤr}{}\classe{n}
\begin{définition}\fra bol en bois ayant un couvercle\end{définition}
\begin{définition}\cmn 有盖子的木碗\end{définition}\end{entrée}

\begin{entrée}
\vedette{\hypertarget{Ⓔrgot}{\papi{ rgot}}}\markboth{rgot}{}\classe{vs}
\begin{définition}\fra robuste\end{définition}
\begin{définition}\cmn 身强力壮;不好对付
\begin{déclaration} \étymologie{\papi{rgod}}\end{déclaration}\end{définition}\end{entrée}

\begin{entrée}
\vedette{\hypertarget{Ⓔrgɯdɯ}{\papi{ rgɯdɯ}}}\markboth{rgɯdɯ}{}\classe{n}
\begin{définition}\fra léproserie\end{définition}
\begin{définition}\cmn 麻风病医院\end{définition}
\end{entrée}

\begin{entrée}
\vedette{\hypertarget{Ⓔrgɯkhra}{\papi{ rgɯkhra}}}\markboth{rgɯkhra}{}
\classe{n}
\begin{définition}\fra vache à pois\end{définition}
\begin{définition}\cmn 黑白相间;红白相间的牛
\begin{déclaration} \étymologie{\papi{kʰra}}\end{déclaration}\end{définition}\end{entrée}

\begin{entrée}
\vedette{\hypertarget{Ⓔrgɯnba}{\papi{ rgɯnba}}}\markboth{rgɯnba}{}
\classe{n}
\begin{définition}\fra temple\end{définition}
\begin{définition}\cmn 庙
\begin{déclaration} \étymologie{\papi{dgon.pa}}\end{déclaration}\end{définition}\end{entrée}

\begin{entrée}
\vedette{\hypertarget{Ⓔrgɯni}{\papi{ rgɯni}}}\markboth{rgɯni}{}\classe{n}
\begin{définition}\fra vache rousse\end{définition}
\begin{définition}\cmn 红毛牛\end{définition}
\end{entrée}

\begin{entrée}
\vedette{\hypertarget{Ⓔrgɯnmdɯt}{\papi{ rgɯnmdɯt}}}\markboth{rgɯnmdɯt}{}\classe{n}
\begin{définition}\fra groupe de neuf nœuds (sur un khatag ou avec un fil normal)\end{définition}
\begin{définition}\cmn 在哈达上打九个结\end{définition}
\end{entrée}

\begin{entrée}
\vedette{\hypertarget{Ⓔrgɯnsa}{\papi{ rgɯnsa}}}\markboth{rgɯnsa}{}
\classe{n}
\begin{définition}\fra pâturage d'hiver\end{définition}
\begin{définition}\cmn 冬天的牧场
\begin{déclaration} \étymologie{\papi{dgun.sa}}\end{déclaration}\end{définition}\end{entrée}

\begin{entrée}
\vedette{\hypertarget{Ⓔrgɯntɕɯn}{\papi{ rgɯntɕɯn}}}\markboth{rgɯntɕɯn}{}\classe{n}
\begin{définition}\fra grand monastère\end{définition}
\begin{définition}\cmn 大庙\end{définition}\end{entrée}

\begin{entrée}
\vedette{\hypertarget{Ⓔrgɯpa}{\papi{ rgɯpa}}}\markboth{rgɯpa}{}\classe{n}
\begin{définition}\fra neuvième mois\end{définition}
\begin{définition}\cmn 九月
\begin{déclaration} \étymologie{\papi{dgu.pa}}\end{déclaration}\end{définition}
\end{entrée}

\begin{entrée}
\vedette{\hypertarget{Ⓔrgɯskɯ}{\papi{ rgɯskɯ}}}\markboth{rgɯskɯ}{}\classe{n}
\begin{définition}\fra vache dont la tête, le ventre et le haut du dos sont blancs, les membres et les flancs noirs\end{définition}
\begin{définition}\cmn 头、肚子和背梁白色,四肢和两侧黑色的牛\end{définition}
\end{entrée}

\begin{entrée}
\vedette{\hypertarget{Ⓔrɣɤβrɣɤβ}{\papi{ rɣɤβrɣɤβ}}}\markboth{rɣɤβrɣɤβ}{}
\classe{idph.2}
\begin{définition}\fra transparent\end{définition}
\begin{définition}\cmn 形容透明而光亮的样子\end{définition}
\begin{exemple}\jya tɤŋgɤr rɣɤβrɣɤβ ʑo ɲɯ-pa\cmn 猪膘显得很透明\end{exemple}
\begin{exemple}\jya ɯ-qom rɣɤβrɣɤβ ʑo to-stu\cmn (小孩子)快要眼泪哗哗的样子\end{exemple}\begin{sous-entrée}
\vedette{\hypertarget{}{\papi{ ɣɤrɣɤβrɣɤβ}}}\markboth{ɣɤrɣɤβrɣɤβ}{}\classe{vi}
\begin{exemple}\jya tɤŋgɤr ɲɯ-ɣɤrɣɤβrɣɤβ\cmn 猪膘显得很透明\end{exemple}
\end{sous-entrée}\begin{sous-entrée}
\vedette{\hypertarget{}{\papi{ sɤrɣɤβrɣɤβ}}}\markboth{sɤrɣɤβrɣɤβ}{}\classe{vt}
\begin{exemple}\jya tɤ-pɤtso kɯ ɯ-qom ɲɯ-sɤrɣɤβrɣɤβ ʑo\cmn 小孩子快要眼泪哗哗的样子\end{exemple}
\end{sous-entrée}\end{entrée}

\begin{entrée}
\vedette{\hypertarget{Ⓔrɣurɣu}{\papi{ rɣurɣu}}}\markboth{rɣurɣu}{}
\classe{idph.2}
\begin{définition}\fra semblable à une ampoule\end{définition}
\begin{définition}\cmn 形容物体像水泡的样子\end{définition}
\begin{exemple}\jya qaɕpa ɯ-ŋgɯm rɣurɣu ʑo ɲɯ-pa\cmn 青蛙卵又圆又透明\end{exemple}
\begin{exemple}\jya a-jaʁ cɯmbɤrom rɣurɣu ʑo to-rku\cmn 我手上长了水泡\end{exemple}\end{entrée}

\begin{entrée}
\vedette{\hypertarget{Ⓔri}{\papi{ ri}}}\markboth{ri}{}\classe{sfp}
\begin{définition}\fra et ... alors?\end{définition}
\begin{définition}\cmn 呢\end{définition}
\begin{exemple}\jya a-wa ri?\cmn 我们爸爸呢?\end{exemple}\end{entrée}

\begin{entrée}
\vedette{\hypertarget{ⒺriⒽ1}{\papi{ ri}}}\markboth{ri}{}\homonyme{1}
\classe{postp}
\begin{définition}\fra locatif\end{définition}
\begin{définition}\cmn 在\end{définition}\end{entrée}

\begin{entrée}
\vedette{\hypertarget{ⒺriⒽ2}{\papi{ ri}}}\markboth{ri}{}\homonyme{2}
\classe{cnj}
\begin{définition}\fra mais\end{définition}
\begin{définition}\cmn 但是\end{définition}\end{entrée}

\begin{entrée}
\vedette{\hypertarget{ⒺriⒽ3}{\papi{ ri}}}\markboth{ri}{}\homonyme{3}\classe{vi}
\paradigme{\textit{dir :} \jya nɯ-}
\begin{définition}\fra rester\end{définition}
\begin{définition}\cmn 剩下\end{définition}
\begin{exemple}\jya ɯʑo nɯ-ri\cmn 剩下了他\end{exemple}
\begin{exemple}\jya ɯ-ro ɲɤ-ri\cmn 有剩余的\end{exemple}
\begin{exemple}\jya a-ʑɯβ ko-ri (=pjɤ-rtaʁ)\cmn 我睡够了\end{exemple}
\begin{relation-sémantique}\confer{
\hyperlink{Ⓔβri}{\textit{ \papi{βri}}}
}\end{relation-sémantique}
\begin{relation-sémantique}\confer{
\hyperlink{Ⓔtɯ-sroʁ,ri}{\textit{ \papi{tɯ-sroʁ,ri}}}
}\end{relation-sémantique}
\begin{relation-sémantique}\confer{
\hyperlink{Ⓔnɯʑɯβri}{\textit{ \papi{nɯʑɯβri}}}
}\end{relation-sémantique}\end{entrée}

\begin{entrée}
\vedette{\hypertarget{Ⓔriβdaʁ}{\papi{ riβdaʁ}}}\markboth{riβdaʁ}{}\classe{n}
\begin{définition}\fra divinité des montagnes\end{définition}
\begin{définition}\cmn 山神
\begin{déclaration} \étymologie{\papi{ri.bdag}}\end{déclaration}\end{définition}
\end{entrée}

\begin{entrée}
\vedette{\hypertarget{Ⓔrirɤβ}{\papi{ rirɤβ}}}\markboth{rirɤβ}{}
\classe{n}
\begin{définition}\fra haute montagne\end{définition}
\begin{définition}\cmn 高山
\begin{déclaration}\use{古语}\end{déclaration}
\begin{déclaration} \étymologie{\papi{ri.rab}}\end{déclaration}\end{définition}\end{entrée}

\begin{entrée}
\vedette{\hypertarget{Ⓔrjaŋrjaŋ}{\papi{ rjaŋrjaŋ}}}\markboth{rjaŋrjaŋ}{}
\classe{idph.2}
\begin{définition}\fra long et cylindrique\end{définition}
\begin{définition}\cmn 长,圆柱形的\end{définition}\begin{sous-entrée}
\vedette{\hypertarget{}{\papi{ mɤlɤrjaŋ}}}\markboth{mɤlɤrjaŋ}{}
\begin{exemple}\jya stukɤr mɤlɤrjaŋ pjɤ-phɯt\cmn 他砍了很长(的一棵树做)檩子\end{exemple}
\begin{relation-sémantique}\confer{
\hyperlink{Ⓔrjoʁrjoʁ}{\textit{ \papi{rjoʁrjoʁ}}}
}\end{relation-sémantique}
\end{sous-entrée}\end{entrée}

\begin{entrée}
\vedette{\hypertarget{Ⓔrjɤrjɤt}{\papi{ rjɤrjɤt}}}\markboth{rjɤrjɤt}{}\classe{idph.2}\acception{1}
\begin{définition}\fra long et fin\end{définition}
\begin{définition}\cmn 细长\end{définition}
\begin{exemple}\jya rjɤrjɤt ci ɲɯ-ŋu\cmn 是细长细长的\end{exemple}
\begin{exemple}\jya romɲa tsuku rjɤrjɤt pjɤ-phɯt-nɯ\cmn 他们砍了许多(树,做成)小梁\end{exemple}\acception{2}
\begin{définition}\fra rester peu de temps\end{définition}
\begin{définition}\cmn 耽误很短时间\end{définition}
\begin{exemple}\jya rjɤrjɤt ci ɕ-to-ti\cmn 快快当当地去说了\end{exemple}
\begin{exemple}\jya rjɤrjɤt ci tɯpri ʑ-ɲɤ-nɯmnɤt\cmn 快快当当地传达了口信\end{exemple}
\begin{exemple}\jya rjɤrjɤt ci ɕe-a ŋu\cmn 我只去耽误一会就回来\end{exemple}
\begin{exemple}\jya tɯpri ɯ-kɯ-nɯmnɤt ci rjɤrjɤt jɤ-ari-a\cmn 我快快当当地传达了口信\end{exemple}\begin{sous-entrée}
\vedette{\hypertarget{}{\papi{ ɣɤrjɤlɤt}}}\markboth{ɣɤrjɤlɤt}{}\classe{vi}
\begin{définition}\fra frétiller\end{définition}
\begin{définition}\cmn 颤动\end{définition}
\begin{exemple}\jya qaɟɯ ɲɯ-ɣɤrjɤlɤt\cmn 鱼在颤动\end{exemple}
\end{sous-entrée}\begin{sous-entrée}
\vedette{\hypertarget{}{\papi{ rjɤnɤlɤt}}}\markboth{rjɤnɤlɤt}{}
\begin{définition}\fra long et fin, aux mouvements agiles\end{définition}
\begin{définition}\cmn 又细又长,动作灵活\end{définition}
\begin{exemple}\jya βʑɯ nɯ kha zɯ rjɤnɤlɤt ʑo tu-ŋke ŋu\cmn 老鼠,身子和尾巴又细又长,在房间里走动\end{exemple}
\end{sous-entrée}\begin{sous-entrée}
\vedette{\hypertarget{}{\papi{ rjɤnɤrjɤt}}}\markboth{rjɤnɤrjɤt}{}\classe{idph.3}
\begin{exemple}\jya qapri ci rjɤrjɤt nɤ rjɤrjɤt thɯ-ari\cmn 蛇细长细长地过去了\end{exemple}
\end{sous-entrée}\begin{sous-entrée}
\vedette{\hypertarget{}{\papi{ sɤrjɤrjɤt}}}\markboth{sɤrjɤrjɤt}{}\classe{vt}
\begin{définition}\fra faire frétiller, faire vibrer\end{définition}
\begin{définition}\cmn 使……颤动\end{définition}
\end{sous-entrée}\begin{sous-entrée}
\vedette{\hypertarget{}{\papi{ stɤrjɤt}}}\markboth{stɤrjɤt}{}
\begin{définition}\fra long et fin, agile\end{définition}
\begin{définition}\cmn 又细又长;很灵活\end{définition}
\begin{exemple}\jya βʑɯ stɤrjɤt ʑo nɯ-ɕqhlɤt\cmn 老鼠一下子就消失了\end{exemple}
\end{sous-entrée}\end{entrée}

\begin{entrée}
\vedette{\hypertarget{Ⓔrjoʁrjoʁ}{\papi{ rjoʁrjoʁ}}}\markboth{rjoʁrjoʁ}{}\classe{idph.2}
\begin{définition}\fra cylindrique\end{définition}
\begin{définition}\cmn 形容圆柱形\end{définition}
\end{entrée}

\begin{entrée}
\vedette{\hypertarget{Ⓔrɟa}{\papi{ rɟa}}}\markboth{rɟa}{}
\classe{n}
\begin{définition}\fra pantholops hodgsoni\end{définition}
\begin{définition}\cmn 藏羚
\begin{déclaration} \étymologie{\papi{rgʲa}}\end{déclaration}\end{définition}\end{entrée}

\begin{entrée}
\vedette{\hypertarget{Ⓔrɟama}{\papi{ rɟama}}}\markboth{rɟama}{}
\classe{n}
\begin{définition}\fra balance\end{définition}
\begin{définition}\cmn 称
\begin{déclaration} \étymologie{\papi{rgʲa.ma}}\end{déclaration}\end{définition}\end{entrée}

\begin{entrée}
\vedette{\hypertarget{Ⓔrɟamar}{\papi{ rɟamar}}}\markboth{rɟamar}{}\classe{n}
\begin{définition}\fra bovidé de couleur noire dont le haut du dos et le bout des oreilles sont marrons clairs\end{définition}
\begin{définition}\cmn 全身黑色,耳尖和背梁棕色的牛\end{définition}
\end{entrée}

\begin{entrée}
\vedette{\hypertarget{Ⓔrɟambrɯɣ}{\papi{ rɟambrɯɣ}}}\markboth{rɟambrɯɣ}{}
\classe{n}
\begin{définition}\fra espèce de chien dont le corps est noir et les yeux sont entourés de rouge\end{définition}
\begin{définition}\cmn 四眼狗
\begin{déclaration} \étymologie{\papi{rgʲa.ⁿbrug}}\end{déclaration}\end{définition}\end{entrée}

\begin{entrée}
\vedette{\hypertarget{Ⓔrɟamtshu}{\papi{ rɟamtshu}}}\markboth{rɟamtshu}{}\classe{n}
\begin{définition}\fra mer\end{définition}
\begin{définition}\cmn 海
\begin{déclaration} \étymologie{\papi{rgʲa.mtsʰo}}\end{déclaration}\end{définition}
\end{entrée}

\begin{entrée}
\vedette{\hypertarget{Ⓔrɟanatɕaʁri}{\papi{ rɟanatɕaʁri}}}\markboth{rɟanatɕaʁri}{}
\classe{n}
\begin{définition}\fra la grande muraille de chine\end{définition}
\begin{définition}\cmn 万里长城;一种佛教图纹
\begin{déclaration} \étymologie{\papi{rgʲa.nag ltɕags.ri}}\end{déclaration}\end{définition}\end{entrée}

\begin{entrée}
\vedette{\hypertarget{Ⓔrɟandzi}{\papi{ rɟandzi}}}\markboth{rɟandzi}{}\classe{n}
\begin{définition}\fra bovidé de couleur noire avec une tache blanche sur le front\end{définition}
\begin{définition}\cmn 全身黑色的牛,在额头上带有白色\end{définition}
\end{entrée}

\begin{entrée}
\vedette{\hypertarget{Ⓔrɟaŋ}{\papi{ rɟaŋ}}}\markboth{rɟaŋ}{}\classe{n}
\begin{définition}\fra lointain\end{définition}
\begin{définition}\cmn 长远
\begin{déclaration} \étymologie{\papi{rgʲaŋ}}\end{déclaration}\end{définition}
\begin{exemple}\jya rɟaŋ ɣɯ ɯ-sɯso kɤ-lɤt ra\cmn 要有长远打算\end{exemple}
\begin{exemple}\jya rɟaŋ ʑo z-jo-fskɤr\cmn 他绕了个大圈\end{exemple}
\begin{relation-sémantique}\confer{
\hyperlink{Ⓔrɯrɟaŋrɟɤz}{\textit{ \papi{rɯrɟaŋrɟɤz}}}
}\end{relation-sémantique}
\begin{relation-sémantique}\confer{
\hyperlink{Ⓔnɯrɟaŋ}{\textit{ \papi{nɯrɟaŋ}}}
}\end{relation-sémantique}\end{entrée}

\begin{entrée}
\vedette{\hypertarget{Ⓔrɟaŋsoʁ}{\papi{ rɟaŋsoʁ}}}\markboth{rɟaŋsoʁ}{}\classe{n}
\begin{définition}\fra scie\end{définition}
\begin{définition}\cmn 锯
\begin{déclaration} \étymologie{\papi{sog.le}}\end{déclaration}\end{définition}
\end{entrée}

\begin{entrée}
\vedette{\hypertarget{Ⓔrɟara}{\papi{ rɟara}}}\markboth{rɟara}{}\classe{n}
\begin{définition}\fra cour\end{définition}
\begin{définition}\cmn 院子\end{définition}
\begin{relation-sémantique}\synonyme{
\hyperlink{Ⓔtɕhaʁla}{\textit{ \papi{tɕhaʁla}}}
}\end{relation-sémantique}\end{entrée}

\begin{entrée}
\vedette{\hypertarget{Ⓔrɟaʁ}{\papi{ rɟaʁ}}}\markboth{rɟaʁ}{}
\classe{vi}
\paradigme{\textit{dir :} \jya pɯ-}
\paradigme{\textit{dir :} \jya kɤ-}
\begin{définition}\fra danser\end{définition}
\begin{définition}\cmn 跳舞\end{définition}
\begin{exemple}\jya jiɕqha nɯ ɲɯ-rɟaʁ-nɯ\cmn 他们在跳舞\end{exemple}
\begin{exemple}\jya pjɯ-rɟaʁ-a ŋgrɤl, aj tɯrɟaʁ rga-a\cmn 我喜欢跳舞\end{exemple}
\begin{relation-sémantique}\confer{
\hyperlink{Ⓔtɯrɟaʁ}{\textit{ \papi{tɯrɟaʁ}}}
}\end{relation-sémantique}\end{entrée}

\begin{entrée}
\vedette{\hypertarget{Ⓔrɟaspa}{\papi{ rɟaspa}}}\markboth{rɟaspa}{}
\classe{adv}
\begin{définition}\fra à peu près, plutôt\end{définition}
\begin{définition}\cmn 差不多;比较\end{définition}
\begin{exemple}\jya nɤ-tɕhomba rɟaspa to-mna tɕe ɲɯ-pe\cmn 你的感冒差不多好了\end{exemple}
\begin{relation-sémantique}\synonyme{
\hyperlink{Ⓔnɤkɤro}{\textit{ \papi{nɤkɤro}}}
}\end{relation-sémantique}\end{entrée}

\begin{entrée}
\vedette{\hypertarget{Ⓔrɟawu}{\papi{ rɟawu}}}\markboth{rɟawu}{}
\classe{n}
\begin{définition}\fra barbe\end{définition}
\begin{définition}\cmn 连鬓胡
\begin{déclaration} \étymologie{\papi{rgʲa.bo}}\end{déclaration}\end{définition}\end{entrée}

\begin{entrée}
\vedette{\hypertarget{Ⓔrɟɤβlun}{\papi{ rɟɤβlun}}}\markboth{rɟɤβlun}{}
\classe{n}
\begin{définition}\fra ministre\end{définition}
\begin{définition}\cmn 宰相;大臣
\begin{déclaration} \étymologie{\papi{rgʲa.blon}}\end{déclaration}\end{définition}\end{entrée}

\begin{entrée}
\vedette{\hypertarget{Ⓔrɟɤɕi}{\papi{ rɟɤɕi}}}\markboth{rɟɤɕi}{}
\classe{n}
\begin{définition}\fra embauchoir\end{définition}
\begin{définition}\cmn 楦头
\end{définition}\end{entrée}

\begin{entrée}
\vedette{\hypertarget{Ⓔrɟɤdoʁ}{\papi{ rɟɤdoʁ}}}\markboth{rɟɤdoʁ}{}\classe{n}
\begin{définition}\fra fibres de chanvre\end{définition}
\begin{définition}\cmn 大麻皮(用来织麻布的纬线和经线,编绳子)\end{définition}\end{entrée}

\begin{entrée}
\vedette{\hypertarget{Ⓔrɟɤdɯm}{\papi{ rɟɤdɯm}}}\markboth{rɟɤdɯm}{}\classe{n}
\begin{définition}\fra grumeaux (dans la pâte de farine)\end{définition}
\begin{définition}\cmn 没有搅均匀的面坨坨\end{définition}
\begin{exemple}\jya tɤjlu pɯ-tɯ-lɤt tɕe koŋla nɯ-ɕmi tɕe, rɟɤdɯm a-mɤ-nɯ-βze\cmn 你和面的时候,要好好搅拌不要有坨坨\end{exemple}\end{entrée}

\begin{entrée}
\vedette{\hypertarget{Ⓔrɟɤɣi}{\papi{ rɟɤɣi}}}\markboth{rɟɤɣi}{}
\classe{n}
\begin{définition}\fra tsampa\end{définition}
\begin{définition}\cmn 糌粑的一种吃法\end{définition}
\begin{exemple}\jya rɟɤɣi pɯ-nɯ-lat-a\cmn 我倒了糌粑\end{exemple}
\begin{exemple}\jya rɟɤɣi tɤ-nɯ-βzu-t-a\cmn 我挼了糌粑\end{exemple}
\begin{exemple}\jya rɟɤɣi nɯ khɯtsa ɯ-ŋgɯ rtsɤmtɕhɯ pjɯ́-wɣ-lɤt tɕe ɯ-taʁ ta-mar tɯ-snaʁ pjɯ́-wɣ-lɤt tɕe ɯ-taʁ tɯ-ɣndʑɤr khɯtsa tú-wɣ-sɯ-mtshɤt tɕe tú-wɣ-ɕmi tɕe tú-wɣ-rɤlaj tɕe tú-wɣ-ndza ŋu.\cmn 
\stylefv{rɟɤɣi}就是在碗里倒上糌粑水,然后放上一小块酥油,再放上满碗的糌粑然后搅匀,挼好就可以吃了。
\end{exemple}\end{entrée}

\begin{entrée}
\vedette{\hypertarget{Ⓔrɟɤkɤr}{\papi{ rɟɤkɤr}}}\markboth{rɟɤkɤr}{} (\variante{rɟɤkɤrji}) \classe{n}
\begin{définition}\fra Inde\end{définition}
\begin{définition}\cmn 印度
\begin{déclaration} \étymologie{\papi{rgʲa dkar}}\end{déclaration}\end{définition}
\begin{exemple}\jya rɟɤkɤr zɯ βlama ra kɯ srɯnmɯ ra tu-nɯkon-nɯ ɲɯ-ŋu\cmn 在印度,喇嘛们把全世界的妖精管制在那里。\end{exemple}\end{entrée}

\begin{entrée}
\vedette{\hypertarget{Ⓔrɟɤlkhɤβ}{\papi{ rɟɤlkhɤβ}}}\markboth{rɟɤlkhɤβ}{}\classe{n}
\begin{définition}\fra pays\end{définition}
\begin{définition}\cmn 国家
\begin{déclaration} \étymologie{\papi{rgʲal.kʰab}}\end{déclaration}\end{définition}\end{entrée}

\begin{entrée}
\vedette{\hypertarget{Ⓔrɟɤlsa}{\papi{ rɟɤlsa}}}\markboth{rɟɤlsa}{}
\classe{n}
\begin{définition}\fra palais\end{définition}
\begin{définition}\cmn 宫殿
\begin{déclaration} \étymologie{\papi{rgʲal.sa}}\end{déclaration}\end{définition}
\begin{exemple}\jya tɕɯχtsi rɟɤlsa\cmn 卓克基官寨\end{exemple}\end{entrée}

\begin{entrée}
\vedette{\hypertarget{Ⓔrɟɤntɕa}{\papi{ rɟɤntɕa}}}\markboth{rɟɤntɕa}{}
\classe{n}
\begin{définition}\fra bijoux, décoration\end{définition}
\begin{définition}\cmn 装饰品
\begin{déclaration} \étymologie{\papi{rgʲan.tɕʰa}}\end{déclaration}\end{définition}\end{entrée}

\begin{entrée}
\vedette{\hypertarget{Ⓔrɟɤŋgɤɣ}{\papi{ rɟɤŋgɤɣ}}}\markboth{rɟɤŋgɤɣ}{}
\classe{n}
\begin{définition}\fra poutre horizontale\end{définition}
\begin{définition}\cmn 横梁\end{définition}
\begin{exemple}\jya rɟɤŋgɤɣ nɯ rɟɯɣ cho nɯ kɯ-naχtɕɯɣ nɯ\cmn 
\stylefv{rɟɤŋgɤɣ}跟\stylefv{rɟɯɣ}一样
\end{exemple}\end{entrée}

\begin{entrée}
\vedette{\hypertarget{Ⓔrɟɤpɕɤt}{\papi{ rɟɤpɕɤt}}}\markboth{rɟɤpɕɤt}{}
\classe{n}
\begin{définition}\fra demi livre\end{définition}
\begin{définition}\cmn 半斤
\begin{déclaration} \étymologie{\papi{*rgʲa.pʰʲed}}\end{déclaration}\end{définition}\end{entrée}

\begin{entrée}
\vedette{\hypertarget{Ⓔrɟɤskɤt}{\papi{ rɟɤskɤt}}}\markboth{rɟɤskɤt}{}
\classe{n}
\begin{définition}\fra escalier en bois\end{définition}
\begin{définition}\cmn 板梯
\begin{déclaration} \étymologie{\papi{rgʲa.skas}}\end{déclaration}\end{définition}\end{entrée}

\begin{entrée}
\vedette{\hypertarget{Ⓔrɟɤskhi}{\papi{ rɟɤskhi}}}\markboth{rɟɤskhi}{}
\classe{n}
\begin{définition}\fra vannerie\end{définition}
\begin{définition}\cmn 簸箕\end{définition}\end{entrée}

\begin{entrée}
\vedette{\hypertarget{Ⓔrɟɤthaʁ}{\papi{ rɟɤthaʁ}}}\markboth{rɟɤthaʁ}{}\classe{n}
\begin{définition}\fra verrou\end{définition}
\begin{définition}\cmn 插销\end{définition}\end{entrée}

\begin{entrée}
\vedette{\hypertarget{Ⓔrɟɤthɤβ}{\papi{ rɟɤthɤβ}}}\markboth{rɟɤthɤβ}{}
\classe{n}
\begin{définition}\fra four chinois\end{définition}
\begin{définition}\cmn 炉子
\begin{déclaration} \étymologie{\papi{rgʲa.tʰab}}\end{déclaration}\end{définition}\end{entrée}

\begin{entrée}
\vedette{\hypertarget{Ⓔrɟɤtpa}{\papi{ rɟɤtpa}}}\markboth{rɟɤtpa}{}\classe{n}
\begin{définition}\fra huitième mois\end{définition}
\begin{définition}\cmn 八月
\begin{déclaration} \étymologie{\papi{brgʲad.pa}}\end{déclaration}\end{définition}\end{entrée}

\begin{entrée}
\vedette{\hypertarget{Ⓔrɟɤtsha}{\papi{ rɟɤtsha}}}\markboth{rɟɤtsha}{}
\classe{n}
\begin{définition}\fra plaque de sel\end{définition}
\begin{définition}\cmn 盐的大块
\begin{déclaration} \étymologie{\papi{rgʲa.tsʰʷa}}\end{déclaration}\end{définition}
\begin{exemple}\jya jiʑo pɯ-xtɕi-j tɕe, rɟɤtsha ɯ-ntɕhɯr ntsɯ tu-nɯntsɯɣ-i pɯ-ŋu\cmn 我们小时候一直舔盐块\end{exemple}\end{entrée}

\begin{entrée}
\vedette{\hypertarget{Ⓔrɟɤxtsa}{\papi{ rɟɤxtsa}}}\markboth{rɟɤxtsa}{}
\classe{n}
\begin{définition}\fra botte à semelle épaisse\end{définition}
\begin{définition}\cmn 鞋底又厚又硬的靴子\end{définition}\end{entrée}

\begin{entrée}
\vedette{\hypertarget{Ⓔrɟɤz}{\papi{ rɟɤz}}}\markboth{rɟɤz}{}
\classe{vs}
\begin{définition}\fra être développé\end{définition}
\begin{définition}\cmn 发达\end{définition}
\begin{exemple}\jya nɤki nɯ ɯ-ɕa wuma ʑo kɯ-rɟɤz ci ɲɯ-ŋu\cmn 那个人肌肉很发达\end{exemple}
\begin{exemple}\jya ɯ-sɯm rɟɤz (=ɯ-ro jom; ɯ-mɲaʁsta jom)\cmn 他心胸宽阔\end{exemple}\end{entrée}

\begin{entrée}
\vedette{\hypertarget{Ⓔrɟitpa}{\papi{ rɟitpa}}}\markboth{rɟitpa}{}\classe{n}
\begin{définition}\fra lignée, famille\end{définition}
\begin{définition}\cmn 家族
\begin{déclaration} \étymologie{\papi{rgʲud.pa}}\end{déclaration}\end{définition}
\begin{relation-sémantique}\confer{
\hyperlink{Ⓔtɤ-rɟit}{\textit{ \papi{tɤ-rɟit}}}
}\end{relation-sémantique}
\end{entrée}

\begin{entrée}
\vedette{\hypertarget{Ⓔrɟum}{\papi{ rɟum}}}\markboth{rɟum}{}
\classe{vs}
\paradigme{\textit{dir :} \jya nɯ-}
\begin{définition}\fra large\end{définition}
\begin{définition}\cmn 宽(布,纸)\end{définition}
\begin{exemple}\jya ki ɯ-spa ɲɯ-rɟum\cmn 很宽\end{exemple}
\begin{relation-sémantique}\antonyme{
\hyperlink{Ⓔtɕɤr}{\textit{ \papi{tɕɤr}}}
}\end{relation-sémantique}
\begin{relation-sémantique}\confer{
\hyperlink{Ⓔarɟumtɕɤr}{\textit{ \papi{arɟumtɕɤr}}}
}\end{relation-sémantique}\end{entrée}

\begin{entrée}
\vedette{\hypertarget{Ⓔrɟɯfsoʁ}{\papi{ rɟɯfsoʁ}}}\markboth{rɟɯfsoʁ}{}\classe{n}
\begin{définition}\fra fait de gagner de l'argent\end{définition}
\begin{définition}\cmn 挣钱\end{définition}
\begin{relation-sémantique}\confer{
\hyperlink{Ⓔtɯ-rɟɯ}{\textit{ \papi{tɯ-rɟɯ}}}
}\end{relation-sémantique}
\begin{relation-sémantique}\confer{
\hyperlink{ⒺfsoʁⒽ1}{\textit{ \papi{fsoʁ1}}}
}\end{relation-sémantique}
\begin{relation-sémantique}\confer{
\hyperlink{Ⓔɣɯrɟɯfsoʁ}{\textit{ \papi{ɣɯrɟɯfsoʁ}}}
}\end{relation-sémantique}\end{entrée}

\begin{entrée}
\vedette{\hypertarget{ⒺrɟɯɣⒽ2}{\papi{ rɟɯɣ}}}\markboth{rɟɯɣ}{}\homonyme{2}
\classe{n}
\begin{définition}\fra poutre horizontale\end{définition}
\begin{définition}\cmn 横梁\end{définition}
\begin{exemple}\jya tɤ-jtsi ɯ-χto ɯ-taʁ kɯ-rɤsta ɣɯ tɤpjaʁ nɯ rɟɯɣ rmi\cmn 
固定在柱子的杈子上面的横梁叫\stylefv{rɟɯɣ}
\end{exemple}\end{entrée}

\begin{entrée}
\vedette{\hypertarget{ⒺrɟɯɣⒽ1}{\papi{ rɟɯɣ}}}\markboth{rɟɯɣ}{}\homonyme{1}\classe{vi}
\paradigme{\textit{dir :} \jya \_}
\begin{définition}\fra courir\end{définition}
\begin{définition}\cmn 跑
\begin{déclaration} \étymologie{\papi{rgʲug}}\end{déclaration}\end{définition}
\begin{exemple}\jya kɤ-rɟɯɣ-a, nɯ-rɟɯɣ-a\cmn 我跑了\end{exemple}
\begin{relation-sémantique}\confer{
\hyperlink{Ⓔnɤrɟɯrɟɯɣ}{\textit{ \papi{nɤrɟɯrɟɯɣ}}}
}\end{relation-sémantique}\begin{sous-entrée}
\vedette{\hypertarget{}{\papi{ sɯrɟɯɣ}}}\markboth{sɯrɟɯɣ}{}\classe{vt}\acception{1}
\begin{définition}\fra faire courir\end{définition}
\begin{définition}\cmn 使……跑\end{définition}\acception{2}
\begin{définition}\fra courir avec, apporter en courant\end{définition}
\begin{définition}\cmn 带着……跑;跑步拿来\end{définition}
\begin{exemple}\jya andi ɲɯ-mbɣom tɕe, ki sɤcɯ ki ɲɯ-sɯrɟɯɣ-a ɲɯ-ntshi\cmn 他们在那边有急事,我只好跑着把钥匙送过去\end{exemple}\acception{3}
\begin{définition}\fra courir au moyen de\end{définition}
\begin{définition}\cmn 用……跑\end{définition}
\begin{exemple}\jya rɯdaʁ kɯ ɯ-mi kɯβde-ldʑa ju-sɯrɟɯɣ ɲɯ-ɕti\cmn 动物用四只脚跑\end{exemple}
\end{sous-entrée}\end{entrée}

\begin{entrée}
\vedette{\hypertarget{Ⓔrɟɯma}{\papi{ rɟɯma}}}\markboth{rɟɯma}{}
\classe{n}
\begin{définition}\fra vis\end{définition}
\begin{définition}\cmn 螺丝\end{définition}
\begin{exemple}\jya rɟɯma tɤ-sprat-a\cmn 我拧了螺丝\end{exemple}\end{entrée}

\begin{entrée}
\vedette{\hypertarget{Ⓔrɟɯmtsɯ}{\papi{ rɟɯmtsɯ}}}\markboth{rɟɯmtsɯ}{}\classe{n}
\begin{définition}\fra pincette en bambou\end{définition}
\begin{définition}\cmn 竹子制成的夹子\end{définition}
\begin{relation-sémantique}\synonyme{
\hyperlink{Ⓔɟɯmɢom}{\textit{ \papi{ɟɯmɢom}}}
}\end{relation-sémantique}
\end{entrée}

\begin{entrée}
\vedette{\hypertarget{Ⓔrɟɯnaŋlaŋspjɤt}{\papi{ rɟɯnaŋlaŋspjɤt}}}\markboth{rɟɯnaŋlaŋspjɤt}{}\classe{n}
\begin{définition}\fra richesses\end{définition}
\begin{définition}\cmn 财富
\begin{déclaration} \étymologie{\papi{rgʲu.naŋ.loŋ.spʲod}}\end{déclaration}\end{définition}
\begin{exemple}\jya rɟɯnaŋlaŋspjɤt pjɯ-kɤ-khɯ\cmn 保佑我们能顺利地创造财富\end{exemple}\end{entrée}

\begin{entrée}
\vedette{\hypertarget{Ⓔrɟɯrŋom}{\papi{ rɟɯrŋom}}}\markboth{rɟɯrŋom}{}
\classe{n}
\begin{définition}\fra convoitise des richesses\end{définition}
\begin{définition}\cmn 贪财\end{définition}
\begin{exemple}\jya rɟɯrŋom ma-tɯ-βze\cmn 不要贪财\end{exemple}
\begin{relation-sémantique}\confer{
\hyperlink{Ⓔtɯ-rɟɯ}{\textit{ \papi{tɯ-rɟɯ}}}
}\end{relation-sémantique}
\begin{relation-sémantique}\confer{
\hyperlink{Ⓔsŋom}{\textit{ \papi{sŋom}}}
}\end{relation-sémantique}
\begin{relation-sémantique}\confer{
\hyperlink{Ⓔnɯrɟɯrŋom}{\textit{ \papi{nɯrɟɯrŋom}}}
}\end{relation-sémantique}\end{entrée}

\begin{entrée}
\vedette{\hypertarget{Ⓔrɟɯstɤβ}{\papi{ rɟɯstɤβ}}}\markboth{rɟɯstɤβ}{}\classe{n}
\begin{définition}\fra capacité à gagner de l'argent\end{définition}
\begin{définition}\cmn 赚钱的本事,财力
\begin{déclaration} \étymologie{\papi{rgʲu.stobs}}\end{déclaration}\end{définition}
\end{entrée}

\begin{entrée}
\vedette{\hypertarget{Ⓔrɟɯtɕɯn}{\papi{ rɟɯtɕɯn}}}\markboth{rɟɯtɕɯn}{}\classe{n}
\begin{définition}\fra résor\end{définition}
\begin{définition}\cmn 宝物,贵重物品
\begin{déclaration} \étymologie{\papi{rgʲu.tɕʰen}}\end{déclaration}\end{définition}\end{entrée}

\begin{entrée}
\vedette{\hypertarget{Ⓔrku}{\papi{ rku}}}\markboth{rku}{}\classe{vt}
\paradigme{\textit{dir :} \jya tɤ-}
\paradigme{\textit{dir :} \jya \_}\acception{1}
\begin{définition}\fra mettre dans\end{définition}
\begin{définition}\cmn 装进
\begin{déclaration}\use{tɤɕi, stoʁ, staχpɯ, tɯ-ci, nɯ `tɤ-rku-t-a' tu-kɯ-ti ŋu, laχtɕha kɯ-fse nɯnɯ, nɯ `pɯ-rku-t-a' tu-kɯ-ti ŋu}\end{déclaration}
\begin{déclaration}\use{装青稞、胡豆、豌豆、水等应该说\stylefv{tɤrkuta},装东西的时候应该说\stylefv{pɯrkuta}}\end{déclaration}\end{définition}
\begin{exemple}\jya laχtɕha khɯɣɲɟɯ ɯ-ŋgɯ nɯ-rku-t-a\cmn 我把东西装进窗子里了\end{exemple}
\begin{exemple}\jya ɯ-ŋgɯ ɲɤ-rku\cmn 他放在里面了\end{exemple}
\begin{exemple}\jya kɤ-rku xtɕhɯt\cmn 装得下\end{exemple}
\begin{exemple}\jya mɤʑɯ tú-wɣ-rku tɕhɯt\cmn 还装得下\end{exemple}
\begin{exemple}\jya tɤɕi lʁa ɯ-ŋgɯ tɤ-rku-t-a\cmn 我把青稞装到口袋里\end{exemple}
\begin{exemple}\jya tɤ-fkɯm ɯ-ŋgɯ thɯ-rku-t-a\cmn 我装进口袋里了\end{exemple}
\begin{exemple}\jya kɯɕnom ɯ-rdoʁ chɤ-rku\cmn 青稞穗的颗粒结满了\end{exemple}\acception{2}
\paradigme{\textit{dir :} \jya pɯ-}
\begin{définition}\fra verser\end{définition}
\begin{définition}\cmn 倒进\end{définition}
\begin{exemple}\jya tʂha pɯ-rku-t-a\cmn 我倒了茶\end{exemple}\acception{3}
\paradigme{\textit{dir :} \jya kɤ-}
\begin{définition}\fra traiter une luxation\end{définition}
\begin{définition}\cmn 治(扭伤了的关节)\end{définition}
\begin{exemple}\jya ɯ-jaʁ ɲɤ-nɯ-ɬoʁ tɕe, kɤ-rku-t-a\cmn 他的手扭伤了,我给他治了\end{exemple}\begin{sous-entrée}
\vedette{\hypertarget{}{\papi{ nɤrkɯrku}}}\markboth{nɤrkɯrku}{}\classe{vt}
\begin{définition}\fra mettre n'importe où\end{définition}
\begin{définition}\cmn 装来装去,到处塞\end{définition}
\begin{relation-sémantique}\confer{
\hyperlink{Ⓔarku}{\textit{ \papi{arku}}}
}\end{relation-sémantique}
\end{sous-entrée}\begin{sous-entrée}
\vedette{\hypertarget{}{\papi{ nɯrku}}}\markboth{nɯrku}{}
\begin{définition}\ 
\begin{déclaration}\grammar{autoben}\end{déclaration}\end{définition}
\begin{définition}\fra porter\end{définition}
\begin{définition}\cmn 戴\end{définition}
\begin{exemple}\jya tɯtshot lu-nɯrke-a\cmn 我戴手表\end{exemple}
\end{sous-entrée}\begin{sous-entrée}
\vedette{\hypertarget{}{\papi{ sɤrkɯrku}}}\markboth{sɤrkɯrku}{}\classe{vt}
\paradigme{\textit{dir :} \jya pɯ-}
\begin{définition}\fra ranger\end{définition}
\begin{définition}\cmn 收拾好;放好\end{définition}
\begin{exemple}\jya laχtɕha pɯ-sɤrkɯrku-t-a\cmn 我把东西收拾好了\end{exemple}
\end{sous-entrée}\begin{sous-entrée}
\vedette{\hypertarget{}{\papi{ tɯ-ku,rku}}}\markboth{tɯ-ku,rku}{}
\begin{définition}\fra s'occuper de, participer à\end{définition}
\begin{définition}\cmn 多管闲事;参与\end{définition}
\begin{exemple}\jya nɤ-ku sɤ-rku kɯ-me, nɯra a-mɤ-thɯ-tɯ-nɯkon\cmn 没有你的事,你不要管\end{exemple}
\begin{exemple}\jya nɤ-ku ma-kɤ-tɯ-nɯ-rke\cmn 你不要管闲事\end{exemple}
\begin{exemple}\jya a-ku mɯ-kɤ-nɯrku-t-a\cmn 我没有参与\end{exemple}
\begin{relation-sémantique}\ComponentA{\classe{np}
\hyperlink{Ⓔtɯ-ku}{\textit{ \papi{tɯ-ku}}}
}\end{relation-sémantique}
\begin{relation-sémantique}\ComponentB{\classe{vt}
\hyperlink{Ⓔrku}{\textit{ \papi{rku}}}
}\end{relation-sémantique}
\end{sous-entrée}\begin{sous-entrée}
\vedette{\hypertarget{}{\papi{ ɯ-pa,nɯrku}}}\markboth{ɯ-pa,nɯrku}{}
\paradigme{\textit{dir :} \jya pɯ-}
\begin{définition}\fra oppresser\end{définition}
\begin{définition}\cmn 征服;压在自己下面\end{définition}
\begin{exemple}\jya a-pa pjɯ-nɯrke-a ra\cmn 我一定要征服他们\end{exemple}
\end{sous-entrée}\begin{sous-entrée}
\vedette{\hypertarget{}{\papi{ ʑɣɤrku}}}\markboth{ʑɣɤrku}{}\classe{vi}
\begin{définition}\ 
\begin{déclaration}\grammar{refl}\end{déclaration}
\begin{déclaration}\grammar{caus}\end{déclaration}\end{définition}
\begin{définition}\fra se mettre dans\end{définition}
\begin{définition}\cmn 把自己装进\end{définition}
\begin{exemple}\jya aʑo lʁa ɯ-ŋgɯ thɯ-ʑɣɤrku-a\cmn 我把自己装进口袋里了\end{exemple}
\begin{relation-sémantique}\synonyme{
\hyperlink{Ⓔʑɣɤsɤri}{\textit{ \papi{ʑɣɤsɤri}}}
}\end{relation-sémantique}
\end{sous-entrée}\begin{sous-entrée}
\vedette{\hypertarget{}{\papi{ ʑɣɤsɯrku}}}\markboth{ʑɣɤsɯrku}{}\classe{vi}
\begin{définition}\ 
\begin{déclaration}\grammar{refl}\end{déclaration}
\begin{déclaration}\grammar{caus}\end{déclaration}\end{définition}
\begin{définition}\fra se mettre dans\end{définition}
\begin{définition}\cmn 让自己加入\end{définition}
\end{sous-entrée}\end{entrée}

\begin{entrée}
\vedette{\hypertarget{Ⓔrkaŋ}{\papi{ rkaŋ}}}\markboth{rkaŋ}{}\classe{vs}
\begin{définition}\fra vigoureux\end{définition}
\begin{définition}\cmn 硬朗
\begin{déclaration}\use{用于否定形式时,表示“怀孕”的意思}\end{déclaration}\end{définition}
\begin{exemple}\jya a-mu kɯɕnɯsqaprɤɣ ʑo thɯ-azɣɯt ri, wuma ʑo rkaŋ\cmn 虽然我母亲已经76岁,但是她很能干\end{exemple}
\begin{exemple}\jya ki tɕheme ki mɯ-ɲɤ-rkaŋ\cmn 这个女子怀孕了\end{exemple}\end{entrée}

\begin{entrée}
\vedette{\hypertarget{Ⓔrkɤdɯt}{\papi{ rkɤdɯt}}}\markboth{rkɤdɯt}{}
\classe{n}
\begin{définition}\fra clarinette\end{définition}
\begin{définition}\cmn 唢呐
\begin{déclaration} \étymologie{\papi{rkaŋ.??}}\end{déclaration}\end{définition}\end{entrée}

\begin{entrée}
\vedette{\hypertarget{Ⓔrkɤl}{\papi{ rkɤl}}}\markboth{rkɤl}{}
\classe{vs}
\paradigme{\textit{dir :} \jya tɤ-}
\begin{définition}\fra en sécurité (endroit)\end{définition}
\begin{définition}\cmn 安全;防危险(地方)、不容易受到攻击\end{définition}
\begin{exemple}\jya kɯki sɤtɕha ki, ɯ-rkɯ thamtɕɤt praʁ kɯ ku-fskɤr ɲɯ-ɕti tɕe, ɲɯ-rkɤl\cmn 这个地方周围都是悬崖,很安全\end{exemple}
\begin{exemple}\jya kha pjɯ́-wɣ-sɤtsa tɕe kɯ-rkɤl kɤ-nɯmga ŋu\cmn 锁了房子的门是为了安全\end{exemple}\end{entrée}

\begin{entrée}
\vedette{\hypertarget{Ⓔrkɤsnom}{\papi{ rkɤsnom}}}\markboth{rkɤsnom}{}
\classe{n}
\begin{définition}\fra pantalon\end{définition}
\begin{définition}\cmn 裤子
\begin{déclaration} \étymologie{\papi{rkaŋ.snam}}\end{déclaration}\end{définition}\end{entrée}

\begin{entrée}
\vedette{\hypertarget{Ⓔrkɤtu}{\papi{ rkɤtu}}}\markboth{rkɤtu}{}\classe{n}
\begin{définition}\fra marteau de bois\end{définition}
\begin{définition}\cmn 木槌(用来敲打麻织品)\end{définition}\end{entrée}

\begin{entrée}
\vedette{\hypertarget{Ⓔrkɤz}{\papi{ rkɤz}}}\markboth{rkɤz}{}
\classe{vt}
\paradigme{\textit{dir :} \jya pɯ-}
\begin{définition}\fra graver, sculpter\end{définition}
\begin{définition}\cmn 雕刻
\begin{déclaration} \étymologie{\papi{rkos}}\end{déclaration}\end{définition}
\begin{exemple}\jya parɕaŋ pɯ-rkaz-a\cmn 我刻了印版\end{exemple}
\begin{exemple}\jya parɕaŋ pa-rkɤz\cmn 他刻了印版\end{exemple}\begin{sous-entrée}
\vedette{\hypertarget{}{\papi{ rɤrkɤz}}}\markboth{rɤrkɤz}{}
\begin{définition}\ 
\begin{déclaration}\grammar{apass}\end{déclaration}\end{définition}
\begin{définition}\fra graver un xylographe\end{définition}
\begin{définition}\cmn 刻印版\end{définition}
\begin{exemple}\jya kɯ-rɤrkɤz\cmn 刻印版的人\end{exemple}
\end{sous-entrée}\end{entrée}

\begin{entrée}
\vedette{\hypertarget{Ⓔrkhɤrkhɤt}{\papi{ rkhɤrkhɤt}}}\markboth{rkhɤrkhɤt}{}\classe{idph.2}
\begin{définition}\fra bruit de frappement léger\end{définition}
\begin{définition}\cmn 形容轻轻的敲击声\end{définition}
\begin{exemple}\jya tɕheme nɯ ɯ-xtsa ɯ-qa kɯ-ɤmtɕoʁ kɯ-mbro to-ŋga tɕe tu-ŋke tɕe rkhɤnɤrkhɤt ɲɯ-ti\cmn 那位女孩子穿着高跟鞋,走路的时候就有咚咚咚的声音\end{exemple}\end{entrée}

\begin{entrée}
\vedette{\hypertarget{Ⓔrkhe}{\papi{ rkhe}}}\markboth{rkhe}{}\classe{vt}
\paradigme{\textit{dir :} \jya pɯ-}
\begin{définition}\fra graver\end{définition}
\begin{définition}\cmn (一节一节地)刻\end{définition}
\begin{exemple}\jya tɕoχtsi pɯ-rkhe-t-a\cmn 我刻了桌子\end{exemple}
\begin{exemple}\jya ɲɯ-rʑi tɕe a-jaʁ pjɤ-rkhe\cmn 东西很重,我手上留了个印\end{exemple}
\begin{exemple}\jya pjɤ-tɯ-rkhe-t\cmn 你刻了\end{exemple}\end{entrée}

\begin{entrée}
\vedette{\hypertarget{Ⓔrkhoŋnɤrkhoŋ}{\papi{ rkhoŋnɤrkhoŋ}}}\markboth{rkhoŋnɤrkhoŋ}{}\classe{idph.3}
\begin{définition}\fra bruit d'un pierre qui se cogne contre du bois\end{définition}
\begin{définition}\cmn 形容石头撞击木头的声音\end{définition}
\begin{exemple}\jya tɤrɤm ɯ-taʁ rdɤstaʁ tú-wɣ-lɤt tɕe, rkhoŋnɤrkhoŋ tu-ti ŋu\cmn 在木板上扔石头的时候就会发出砰砰声\end{exemple}
\begin{relation-sémantique}\synonyme{
\hyperlink{Ⓔphoŋnɤphoŋ}{\textit{ \papi{phoŋnɤphoŋ}}}
}\end{relation-sémantique}\end{entrée}

\begin{entrée}
\vedette{\hypertarget{Ⓔrkhɯβrkhɯβ}{\papi{ rkhɯβrkhɯβ}}}\markboth{rkhɯβrkhɯβ}{} (\variante{rkhɯrkhɯp}) \classe{idph.2}
\begin{définition}\fra bruit de coup sur une planche de bois\end{définition}
\begin{définition}\cmn 敲木板的声音\end{définition}
\begin{relation-sémantique}\synonyme{
\hyperlink{Ⓔrchɤrchɤt}{\textit{ \papi{rchɤrchɤt}}}
}\end{relation-sémantique}
\begin{relation-sémantique}\confer{
\hyperlink{Ⓔnɤrkhɯrkhɯβ}{\textit{ \papi{nɤrkhɯrkhɯβ}}}
}\end{relation-sémantique}\end{entrée}

\begin{entrée}
\vedette{\hypertarget{Ⓔrko}{\papi{ rko}}}\markboth{rko}{}\classe{vs}
\paradigme{\textit{dir :} \jya thɯ-}
\paradigme{\textit{dir :} \jya nɯ-}
\paradigme{\textit{dir :} \jya tɤ-}\acception{1}
\begin{définition}\fra dur\end{définition}
\begin{définition}\cmn 硬\end{définition}
\begin{exemple}\jya nɤrŋi ɯ-kɤcɯɣ chɤ-rko\cmn 婴儿的囟门闭合了\end{exemple}\acception{2}
\begin{définition}\fra obstiné\end{définition}
\begin{définition}\cmn 顽固\end{définition}
\begin{exemple}\jya ɲɯ-rko\cmn 很硬/他很顽固\end{exemple}
\begin{relation-sémantique}\confer{
\hyperlink{Ⓔnɯrkorlɯt}{\textit{ \papi{nɯrkorlɯt}}}
}\end{relation-sémantique}\end{entrée}

\begin{entrée}
\vedette{\hypertarget{Ⓔrkoŋɟɤl}{\papi{ rkoŋɟɤl}}}\markboth{rkoŋɟɤl}{}
\classe{n}
\begin{définition}\fra démon à un pied\end{définition}
\begin{définition}\cmn 独脚鬼
\begin{déclaration} \étymologie{\papi{rkaŋ.rgʲal}}\end{déclaration}\end{définition}\end{entrée}

\begin{entrée}
\vedette{\hypertarget{Ⓔrkoŋtoŋ}{\papi{ rkoŋtoŋ}}}\markboth{rkoŋtoŋ}{}
\classe{n}\acception{1}
\begin{définition}\fra fémur\end{définition}
\begin{définition}\cmn 胫骨\end{définition}\acception{2}
\begin{définition}\fra trompette en fémur humain\end{définition}
\begin{définition}\cmn 胫骨号筒
\begin{déclaration} \étymologie{\papi{rkaŋ.duŋ}}\end{déclaration}\end{définition}\end{entrée}

\begin{entrée}
\vedette{\hypertarget{Ⓔrkorsa}{\papi{ rkorsa}}}\markboth{rkorsa}{}
\classe{n}
\begin{définition}\fra toilette\end{définition}
\begin{définition}\cmn 厕所\end{définition}\end{entrée}

\begin{entrée}
\vedette{\hypertarget{Ⓔrkɯn}{\papi{ rkɯn}}}\markboth{rkɯn}{}\classe{vs}
\paradigme{\textit{dir :} \jya nɯ-}
\begin{définition}\fra peu\end{définition}
\begin{définition}\cmn 少
\begin{déclaration} \étymologie{\papi{dkon}}\end{déclaration}\end{définition}
\begin{exemple}\jya nɤ-mɤ-kɤ-tso to-rkɯn\cmn 你不懂的东西变少了\end{exemple}\begin{sous-entrée}
\vedette{\hypertarget{}{\papi{ nɤrkɯn}}}\markboth{nɤrkɯn}{}\classe{vt}
\begin{définition}\fra trouver trop peu, manquer de\end{définition}
\begin{définition}\cmn 觉得少,缺\end{définition}
\begin{exemple}\jya a-kɤ-ndza kɤ-tshi ra pɯ-nɤrkɯn-a pɯ-ra\cmn 我以前吃喝都觉得欠缺\end{exemple}
\begin{relation-sémantique}\confer{
\hyperlink{Ⓔɣɤrkɯn}{\textit{ \papi{ɣɤrkɯn}}}
}\end{relation-sémantique}
\end{sous-entrée}\end{entrée}

\begin{entrée}
\vedette{\hypertarget{Ⓔrkɯwɯ}{\papi{ rkɯwɯ}}}\markboth{rkɯwɯ}{}
\classe{n}
\begin{définition}\fra lampe à beurre\end{définition}
\begin{définition}\cmn 酥油灯\end{définition}\end{entrée}

\begin{entrée}
\vedette{\hypertarget{Ⓔrla}{\papi{ rla}}}\markboth{rla}{}
\classe{vt}
\paradigme{\textit{dir :} \jya nɯ-}
\begin{définition}\fra détacher\end{définition}
\begin{définition}\cmn 解开\end{définition}
\begin{exemple}\jya tɤ-mtɯ nɯ-rla-t-a\cmn 我解开了结\end{exemple}
\begin{exemple}\jya tɤ-mtɯ na-rla\cmn 他解开了结\end{exemple}
\begin{exemple}\jya nɤ-xtsa nɯ-rle\cmn 你解开鞋子吧\end{exemple}\begin{sous-entrée}
\vedette{\hypertarget{}{\papi{ arla}}}\markboth{arla}{}\classe{vi}
\begin{définition}\ 
\begin{déclaration}\grammar{pass}\end{déclaration}\end{définition}
\begin{exemple}\jya tɤ-mtsɯ arla\cmn 结是解开的\end{exemple}
\end{sous-entrée}\end{entrée}

\begin{entrée}
\vedette{\hypertarget{Ⓔrlaŋrlaŋ}{\papi{ rlaŋrlaŋ}}}\markboth{rlaŋrlaŋ}{}\classe{idph.2}
\begin{définition}\fra rond\end{définition}
\begin{définition}\cmn 圆形(又圆又大)\end{définition}
\begin{exemple}\jya tɤphɯ ɯ-tɯ-wxti kɯ rlaŋrlaŋ ʑo ɲɯ-pa\cmn 土巴又圆又大\end{exemple}
\begin{exemple}\jya @yangyu kɯwxtɯwxti rlaŋrlaŋ ʑo thɯ-nɯɬoʁ\cmn (人家挖出来的时候)洋芋又大又圆\end{exemple}
\begin{exemple}\jya sla to-kɯ-ɤrtɯm ci rlaŋrlaŋ\cmn 圆圆的月亮\end{exemple}\begin{sous-entrée}
\vedette{\hypertarget{}{\papi{ rlaŋnɤrlaŋ}}}\markboth{rlaŋnɤrlaŋ}{}\classe{idph.3}
\begin{exemple}\jya rŋgɯ rlaŋnɤlaŋ pɯ-ndʐaβ\cmn 圆形的大石包滚下去了\end{exemple}
\begin{relation-sémantique}\confer{
\hyperlink{Ⓔslaŋslaŋ}{\textit{ \papi{slaŋslaŋ}}}
}\end{relation-sémantique}
\begin{relation-sémantique}\confer{
\hyperlink{Ⓔɕlaŋɕlaŋ}{\textit{ \papi{ɕlaŋɕlaŋ}}}
}\end{relation-sémantique}
\begin{relation-sémantique}\confer{
\hyperlink{Ⓔclaŋclaŋ}{\textit{ \papi{claŋclaŋ}}}
}\end{relation-sémantique}
\begin{relation-sémantique}\confer{
\hyperlink{Ⓔrloŋrloŋ}{\textit{ \papi{rloŋrloŋ}}}
}\end{relation-sémantique}
\begin{relation-sémantique}\confer{
\hyperlink{Ⓔrloʁrloʁ}{\textit{ \papi{rloʁrloʁ}}}
}\end{relation-sémantique}
\end{sous-entrée}\end{entrée}

\begin{entrée}
\vedette{\hypertarget{Ⓔrlaʁ}{\papi{ rlaʁ}}}\markboth{rlaʁ}{}
\classe{vi}
\paradigme{\textit{dir :} \jya nɯ-}
\begin{définition}\fra disparaître\end{définition}
\begin{définition}\cmn 失踪
\begin{déclaration} \étymologie{\papi{brlag}}\end{déclaration}\end{définition}
\begin{exemple}\jya a-taqaβ ɲɤ-rlaʁ\cmn 我的针不见了\end{exemple}
\begin{exemple}\jya a-mbrɯtɕɯ ɲɤ-rlaʁ\cmn 我的刀不见了\end{exemple}
\begin{exemple}\jya a-laχtɕha ɲɤ-rlaʁ\cmn 我的东西不见了\end{exemple}
\begin{exemple}\jya kɯ-rlaʁ kha\cmn 被遗弃的房子\end{exemple}
\begin{relation-sémantique}\confer{
\hyperlink{Ⓔɣɤrlaʁ}{\textit{ \papi{ɣɤrlaʁ}}}
}\end{relation-sémantique}\end{entrée}

\begin{entrée}
\vedette{\hypertarget{Ⓔrlaʁrlaʁ}{\papi{ rlaʁrlaʁ}}}\markboth{rlaʁrlaʁ}{}\classe{idph.2}
\begin{définition}\fra rond et dur\end{définition}
\begin{définition}\cmn 形容圆又硬的样子\end{définition}
\begin{exemple}\jya tɤ-fkɯm ɯ-ŋgɯ tɯsqar chɤ-sɯmtshɤt rlaʁrlaʁ ʑo ɲɯ-pa\cmn 袋子里的糌粑装得很满,涨鼓鼓的\end{exemple}\end{entrée}

\begin{entrée}
\vedette{\hypertarget{Ⓔrloŋrloŋ}{\papi{ rloŋrloŋ}}}\markboth{rloŋrloŋ}{}
\classe{idph.2}
\begin{définition}\fra sphérique\end{définition}
\begin{définition}\cmn 形容球形(比较大)的样子\end{définition}
\begin{exemple}\jya jla ŋgorli rloŋrloŋ ɲɯ-pa\cmn 无角犏牛显得圆圆的\end{exemple}
\begin{relation-sémantique}\confer{
\hyperlink{Ⓔrloʁrloʁ}{\textit{ \papi{rloʁrloʁ}}}
}\end{relation-sémantique}
\begin{relation-sémantique}\confer{
\hyperlink{Ⓔrlaŋrlaŋ}{\textit{ \papi{rlaŋrlaŋ}}}
}\end{relation-sémantique}
\begin{relation-sémantique}\confer{
\hyperlink{Ⓔrwoʁrwoʁ}{\textit{ \papi{rwoʁrwoʁ}}}
}\end{relation-sémantique}\end{entrée}

\begin{entrée}
\vedette{\hypertarget{Ⓔrloŋrta}{\papi{ rloŋrta}}}\markboth{rloŋrta}{}\classe{n}\acception{1}
\begin{définition}\fra rlung rta\end{définition}
\begin{définition}\cmn 经幡\end{définition}\acception{2}
\begin{définition}\fra chance\end{définition}
\begin{définition}\cmn 运气
\begin{déclaration} \étymologie{\papi{rluŋ.rta}}\end{déclaration}\end{définition}
\begin{exemple}\jya ɯ-rloŋrta ɲɯ-taʁ\cmn 他运气好\end{exemple}\end{entrée}

\begin{entrée}
\vedette{\hypertarget{Ⓔrloʁrloʁ}{\papi{ rloʁrloʁ}}}\markboth{rloʁrloʁ}{}
\classe{idph.2}
\begin{définition}\fra sphérique\end{définition}
\begin{définition}\cmn 球形(比较小)
\begin{déclaration}\use{可以比喻小孩子}\end{déclaration}\end{définition}
\begin{exemple}\jya ɯ-ku rloʁrloʁ ʑo ɲɯ-pa\cmn 他的头是圆形的\end{exemple}\begin{sous-entrée}
\vedette{\hypertarget{}{\papi{ rloʁnɤrloʁ}}}\markboth{rloʁnɤrloʁ}{}\classe{idph.3}
\begin{définition}\fra qui a une tête ronde\end{définition}
\begin{définition}\cmn 头部圆圆的,动作又灵活又可爱\end{définition}
\begin{exemple}\jya tɤ-pɤtso rloʁrloʁ nɤ rloʁrloʁ ɲɯ-ɤnɯɣro\cmn 小孩子在玩\end{exemple}
\begin{exemple}\jya tɤ-pɤtso chɤ-wxti tɕe rloʁnɤrloʁ ʑo tu-ŋke to-cha\cmn 小孩子大了就能走路了\end{exemple}
\begin{relation-sémantique}\confer{
\hyperlink{Ⓔrloŋrloŋ}{\textit{ \papi{rloŋrloŋ}}}
}\end{relation-sémantique}
\begin{relation-sémantique}\confer{
\hyperlink{Ⓔrwoʁrwoʁ}{\textit{ \papi{rwoʁrwoʁ}}}
}\end{relation-sémantique}
\end{sous-entrée}\end{entrée}

\begin{entrée}
\vedette{\hypertarget{Ⓔrlɯm}{\papi{ rlɯm}}}\markboth{rlɯm}{}\classe{idph.1}
\begin{définition}\fra complètement\end{définition}
\begin{définition}\cmn 全部\end{définition}
\begin{exemple}\jya nɤ-tʂha rlɯm kɤ-tshi\cmn 你把你的茶全部喝完\end{exemple}
\begin{exemple}\jya jiʑo rlɯm kɯ kɤ-tshi-j ɕti\cmn 我们都喝了\end{exemple}\end{entrée}

\begin{entrée}
\vedette{\hypertarget{Ⓔrma}{\papi{ rma}}}\markboth{rma}{}
\classe{vi}
\paradigme{\textit{dir :} \jya kɤ-}
\begin{définition}\fra habiter chez quelqu'un\end{définition}
\begin{définition}\cmn 留宿\end{définition}
\begin{exemple}\jya jɯɣmɯr kutɕu kɤ-nɯ-rma\cmn 你今天晚上在这里留宿吧\end{exemple}
\begin{exemple}\jya jɯɣmɯr mbarkhom kɤ-rma\cmn 你今天晚上在马尔康留宿吧\end{exemple}
\begin{exemple}\jya ku-nɯ-rma ɲɯ-sɯsɤm\cmn 他想在这里留宿\end{exemple}
\begin{relation-sémantique}\synonyme{
\hyperlink{Ⓔnɯkho}{\textit{ \papi{nɯkho}}}
}\end{relation-sémantique}
\begin{relation-sémantique}\confer{
\hyperlink{Ⓔsɯrma}{\textit{ \papi{sɯrma}}}
}\end{relation-sémantique}\end{entrée}

\begin{entrée}
\vedette{\hypertarget{Ⓔrmɤβja}{\papi{ rmɤβja}}}\markboth{rmɤβja}{}
\classe{n}
\begin{définition}\fra paon\end{définition}
\begin{définition}\cmn 孔雀
\begin{déclaration} \étymologie{\papi{rma.bʲa}}\end{déclaration}\end{définition}\end{entrée}

\begin{entrée}
\vedette{\hypertarget{Ⓔrmɤβrmɤβ}{\papi{ rmɤβrmɤβ}}}\markboth{rmɤβrmɤβ}{}\classe{idph.2}
\begin{définition}\fra une couche fine\end{définition}
\begin{définition}\cmn 形容薄薄的一层,不完全透明的样子\end{définition}
\begin{exemple}\jya jisŋi zdɯm ci rmɤβrmɤβ ɣɤʑu\cmn 今天云很薄,不完全透明\end{exemple}
\begin{exemple}\jya tɯ-ci ɲɤ-nɤrʑaʁ tɕe ɯ-taʁ ɯ-ɕom kɯ-fse ci rmɤβrmɤβ ko-ta, tɕe kɤ-tshi mɯ-ɲɤ-sna\cmn 水放久了就会在表面上形成薄薄的一层,不能再喝了\end{exemple}
\begin{relation-sémantique}\synonyme{
\hyperlink{Ⓔʂmɤβʂmɤβ}{\textit{ \papi{ʂmɤβʂmɤβ}}}
}\end{relation-sémantique}\end{entrée}

\begin{entrée}
\vedette{\hypertarget{Ⓔrmɤmbe}{\papi{ rmɤmbe}}}\markboth{rmɤmbe}{}\classe{n}
\begin{définition}\fra mue (mammifère)\end{définition}
\begin{définition}\cmn 脱毛;换毛\end{définition}
\begin{relation-sémantique}\confer{
\hyperlink{Ⓔtɤ-rme}{\textit{ \papi{tɤ-rme}}}
}\end{relation-sémantique}
\begin{relation-sémantique}\confer{
\hyperlink{Ⓔmbe}{\textit{ \papi{mbe}}}
}\end{relation-sémantique}
\begin{relation-sémantique}\confer{
\hyperlink{Ⓔnɯrmɤmbe}{\textit{ \papi{nɯrmɤmbe}}}
}\end{relation-sémantique}\end{entrée}

\begin{entrée}
\vedette{\hypertarget{Ⓔrmbatɕɯβ}{\papi{ rmbatɕɯβ}}}\markboth{rmbatɕɯβ}{}
\classe{n}
\begin{définition}\fra espèce de plante\end{définition}
\begin{définition}\cmn 【灰灰菜】\end{définition}
\begin{exemple}\jya rmbatɕɯβ nɯ sɯjno ci ŋu, tɯ-ɟom jamar tu-mbro cha, ɲɯ-ɤɣɯrtɯ-rtaʁ cha, ɯ-rɣi wuma ʑo dɤn, ɯ-jwaʁ ɯ-qhu chu nɯ kɯ-pɣi tu, kɯ-ɤɣɯrnɯɕɯr tu, tɯ-ɣndʑɤr tɤ-kɤ-mar kɯ-fse tu, pha ɯ-phoŋbu nɯ kɯ-pɣi ŋu, kɤ-ndza sna, paʁ wuma ʑo rga\cmn 灰灰菜是一种植物,可以长一米多高,可以长出很多枝桠,结的种子很多,叶子的背面有的是灰色的,也有的是淡红色的,好像上面涂了一层面粉,全身是灰色的,人可以吃,猪特别喜欢。\end{exemple}\end{entrée}

\begin{entrée}
\vedette{\hypertarget{Ⓔrmbɯ}{\papi{ rmbɯ}}}\markboth{rmbɯ}{}
\classe{vt}
\paradigme{\textit{dir :} \jya tɤ-}
\begin{définition}\fra amasser\end{définition}
\begin{définition}\cmn 堆起来(不整齐)\end{définition}
\begin{exemple}\jya tɯjpu ta-rmbɯ\cmn 他把粮食堆起来了\end{exemple}
\begin{exemple}\jya rdɤstaʁ tɤ-rmbɯ-t-a\cmn 我把石头堆起来了\end{exemple}
\begin{exemple}\jya tɯ-ɣli tɤ-rmbɯ-t-a\cmn 我把肥料堆起来了\end{exemple}
\begin{exemple}\jya tɤjpa to-rmbɯ\cmn 她把雪堆起来了\end{exemple}
\begin{relation-sémantique}\confer{
\hyperlink{Ⓔamɯrmbɯ}{\textit{ \papi{amɯrmbɯ}}}
}\end{relation-sémantique}
\begin{relation-sémantique}\confer{
\hyperlink{Ⓔtɯ-rmbɯ}{\textit{ \papi{tɯ-rmbɯ}}}
}\end{relation-sémantique}\end{entrée}

\begin{entrée}
\vedette{\hypertarget{Ⓔrmi}{\papi{ rmi}}}\markboth{rmi}{}
\classe{vi}
\paradigme{\textit{dir :} \jya tɤ-}
\begin{définition}\fra s’appeler\end{définition}
\begin{définition}\cmn 名字叫\end{définition}
\begin{exemple}\jya jiɕqha nɯ nɯ ɲɯ-rmi\cmn 他叫这个\end{exemple}
\begin{exemple}\jya nɤʑo tɕhi tɯ-rmi?\cmn 你叫什么名字\end{exemple}
\begin{exemple}\jya aʑo χpɤltɕin rmi-a, nɤʑo @xiangbolin ɲɯ-tɯ-rmi ɣe\cmn 我叫柏尔青,你叫向柏霖\end{exemple}
\begin{relation-sémantique}\confer{
\hyperlink{Ⓔtɤ-rmi}{\textit{ \papi{tɤ-rmi}}}
}\end{relation-sémantique}\end{entrée}

\begin{entrée}
\vedette{\hypertarget{Ⓔrmɯrmi}{\papi{ rmɯrmi}}}\markboth{rmɯrmi}{}
\classe{adv}
\begin{définition}\fra tous sans exception\end{définition}
\begin{définition}\cmn 每一个,一个都不漏\end{définition}
\begin{exemple}\jya kɯmdza rmɯrmi nɯ jo-ɣi-nɯ\cmn 每一个亲戚都到了\end{exemple}\end{entrée}

\begin{entrée}
\vedette{\hypertarget{Ⓔrnaʁ}{\papi{ rnaʁ}}}\markboth{rnaʁ}{}\classe{vs}
\paradigme{\textit{dir :} \jya pɯ-}
\begin{définition}\fra profond\end{définition}
\begin{définition}\cmn 深\end{définition}
\begin{exemple}\jya jɯ-xtu ɯ-tɯ-rnaʁ nɯ\cmn 我们很饿(肚子很深)\end{exemple}
\end{entrée}

\begin{entrée}
\vedette{\hypertarget{Ⓔrnɤβʑi}{\papi{ rnɤβʑi}}}\markboth{rnɤβʑi}{}\classe{n}\acception{1}
\begin{définition}\fra chapeau à quatre bords\end{définition}
\begin{définition}\cmn 圆盔耳帽\end{définition}\acception{2}
\begin{définition}\fra casserole en fer\end{définition}
\begin{définition}\cmn 生铁锅,有四个把子
\begin{déclaration} \étymologie{\papi{rna.bʑi}}\end{déclaration}\end{définition}\end{entrée}

\begin{entrée}
\vedette{\hypertarget{Ⓔrnɤjɯ}{\papi{ rnɤjɯ}}}\markboth{rnɤjɯ}{}\classe{n}
\begin{définition}\fra boucle d'oreille\end{définition}
\begin{définition}\cmn 耳环
\begin{déclaration} \étymologie{\papi{rna.ju}}\end{déclaration}\end{définition}\end{entrée}

\begin{entrée}
\vedette{\hypertarget{Ⓔrnɤlu}{\papi{ rnɤlu}}}\markboth{rnɤlu}{}
\classe{n}
\begin{définition}\fra sans oreille\end{définition}
\begin{définition}\cmn 缺耳朵的\end{définition}\end{entrée}

\begin{entrée}
\vedette{\hypertarget{Ⓔrnɤrɯ}{\papi{ rnɤrɯ}}}\markboth{rnɤrɯ}{}
\classe{n}
\begin{définition}\fra casserole en fer\end{définition}
\begin{définition}\cmn 生铁锅,有两个把子
\begin{déclaration} \étymologie{\papi{rna.ru?}}\end{déclaration}\end{définition}\end{entrée}

\begin{entrée}
\vedette{\hypertarget{Ⓔrndu}{\papi{ rndu}}}\markboth{rndu}{}
\classe{vt}
\paradigme{\textit{dir :} \jya pɯ-}\acception{1}
\begin{définition}\fra obtenir\end{définition}
\begin{définition}\cmn 拿到\end{définition}\acception{2}
\begin{définition}\fra trouver\end{définition}
\begin{définition}\cmn 找到\end{définition}
\begin{exemple}\jya tɤjmɤɣ pjɤ-rndu\cmn 他找到蘑菇了\end{exemple}
\begin{exemple}\jya laχtɕha kɤ-χtɯ pjɤ-rndu\cmn 买到东西了\end{exemple}
\begin{exemple}\jya pɯ-rndu-tɕi\cmn 我们俩找到了\end{exemple}
\begin{exemple}\jya kɤ-rndu sɤznɤ pɯ-rnde-t-a\cmn 不但没有得到好处,反而遭殃了\end{exemple}\end{entrée}

\begin{entrée}
\vedette{\hypertarget{Ⓔrnde}{\papi{ rnde}}}\markboth{rnde}{}\classe{vt}
\paradigme{\textit{dir :} \jya pɯ-}
\begin{définition}\fra subir un désastre (ne s'emploie pas seul)\end{définition}
\begin{définition}\cmn 吃亏;遭殃
\begin{déclaration}\use{不能单独出现,只能在以上例句中使用}\end{déclaration}\end{définition}
\begin{exemple}\jya kɤ-rndu sɤznɤ pɯ-rnde-t-a\cmn 我不但没有得到好处,反而遭殃了\end{exemple}
\begin{exemple}\jya kɤ-rndu sɤznɤ kɤ-rnde\cmn 不但没有得到好处,反而遭殃了\end{exemple}\end{entrée}

\begin{entrée}
\vedette{\hypertarget{Ⓔrndi}{\papi{ rndi}}}\markboth{rndi}{}
\classe{vi}
\paradigme{\textit{dir :} \jya tɤ-}
\begin{définition}\fra sage, qui ne s'enfuit pas à chaque occasion\end{définition}
\begin{définition}\cmn 不爱到处乱跑,听话\end{définition}
\begin{exemple}\jya ki fsapaʁ ki ɲɯ-rndi\cmn 这个动物不爱到处乱跑\end{exemple}
\begin{exemple}\jya ɕɯŋgɯ staʁ to-rndi\cmn 以前很调皮,爱到处乱跑,现在不调皮了\end{exemple}\end{entrée}

\begin{entrée}
\vedette{\hypertarget{Ⓔrndzɤkɤŋe}{\papi{ rndzɤkɤŋe}}}\markboth{rndzɤkɤŋe}{}\classe{n}
\begin{définition}\fra ombre de la montagne\end{définition}
\begin{définition}\cmn 太阳落山的时候山上的阴影\end{définition}
\begin{exemple}\jya rndzɤkɤŋe tɤ-anɯri\end{exemple}\end{entrée}

\begin{entrée}
\vedette{\hypertarget{Ⓔrɲaŋ}{\papi{ rɲaŋ}}}\markboth{rɲaŋ}{}\classe{vs}
\paradigme{\textit{dir :} \jya nɯ-}
\begin{définition}\fra ancien\end{définition}
\begin{définition}\cmn 陈旧
\begin{déclaration} \étymologie{\papi{rɲiŋ}}\end{déclaration}\end{définition}
\begin{exemple}\jya tamar nɯ ʑaʑa ɲɯ-rɲaŋ ɕti\cmn 酥油很快就会变味\end{exemple}\end{entrée}

\begin{entrée}
\vedette{\hypertarget{Ⓔrɲɟaʁlo}{\papi{ rɲɟaʁlo}}}\markboth{rɲɟaʁlo}{}
\classe{n}
\begin{définition}\fra bâton qui sert à caler la porte\end{définition}
\begin{définition}\cmn 门闩\end{définition}
\begin{exemple}\jya rɲɟaʁlo ɯ-thaʁ pjɯ́-wɣ-lɤt tɕe, kɤ-cɯ mɤ-khɯ\cmn 插上插销,门就不能打开了\end{exemple}
\begin{exemple}\jya rɲɟaʁlo nɯ-lat-a\cmn 我闩了门。\end{exemple}
\begin{exemple}\jya rɲɟaʁlo-ɣɲɟɯ\cmn 门闩的洞\end{exemple}
\begin{relation-sémantique}\confer{
\hyperlink{Ⓔrɟɤthaʁ}{\textit{ \papi{rɟɤthaʁ}}}
}\end{relation-sémantique}\end{entrée}

\begin{entrée}
\vedette{\hypertarget{Ⓔrɲɟi}{\papi{ rɲɟi}}}\markboth{rɲɟi}{}
\classe{vs}
\paradigme{\textit{dir :} \jya thɯ-}
\begin{définition}\fra long\end{définition}
\begin{définition}\cmn 长\end{définition}
\begin{exemple}\jya tɯmbri ɲɯ-rɲɟi\cmn 绳子很长\end{exemple}
\begin{exemple}\jya qha kɯrɯ-ŋga nɯ ɲɯ-rɲɟi\cmn 这件藏装很长\end{exemple}
\begin{relation-sémantique}\confer{
 \papi{ɣɤrɲɟi}
}\end{relation-sémantique}
\begin{relation-sémantique}\antonyme{
\hyperlink{ⒺxtɯtⒽ1}{\textit{ \papi{xtɯt}}}
}\end{relation-sémantique}
\begin{relation-sémantique}\synonyme{
\hyperlink{Ⓔzri}{\textit{ \papi{zri}}}
}\end{relation-sémantique}
\begin{relation-sémantique}\confer{
\hyperlink{Ⓔxtɯrɲɟi}{\textit{ \papi{xtɯrɲɟi}}}
}\end{relation-sémantique}\end{entrée}

\begin{entrée}
\vedette{\hypertarget{Ⓔrɲo}{\papi{ rɲo}}}\markboth{rɲo}{}
\classe{vt}\acception{1}
\paradigme{\textit{dir :} \jya tɤ-}
\begin{définition}\fra essayer, goûter\end{définition}
\begin{définition}\cmn 尝试\end{définition}
\begin{exemple}\jya tɤjko kɤ-ndza tɤ-rɲo-t-a ri, pjɤ-tɕur\cmn 我尝过酸菜,很酸\end{exemple}
\begin{exemple}\jya ɯ-ɲɯ-mɯm kɯ tu-rɲam-a ɲɯ-ra\cmn 我要尝一下好不好吃\end{exemple}\acception{2}
\paradigme{\textit{dir :} \jya pɯ-}
\begin{définition}\fra faire l'expérience de, avoir déjà\end{définition}
\begin{définition}\cmn 体会;曾经……有过\end{définition}
\begin{exemple}\jya pjɯ́-wɣ-rɲo mɤɕtʂa mɤ-kɯ-tso\cmn 自己亲身体会之前不能了解\end{exemple}
\begin{exemple}\jya tɕhi pɯ-nɯ-ŋɯ-ŋu pjɯ́-wɣ-rɲo ra\cmn 无论什么事情都要亲身体会\end{exemple}
\begin{exemple}\jya tɤjko kɤ-ndza pɯ-rɲo-t-a\cmn 我曾经吃过酸菜\end{exemple}
\begin{exemple}\jya kɤ-ɕe pɯ-rɲo-t-a\cmn 我去过\end{exemple}\end{entrée}

\begin{entrée}
\vedette{\hypertarget{Ⓔrɲɯɣrɲɯɣ}{\papi{ rɲɯɣrɲɯɣ}}}\markboth{rɲɯɣrɲɯɣ}{}\classe{idph.2}
\begin{définition}\fra long, fin et flexible\end{définition}
\begin{définition}\cmn 形容又细又长又柔软,很没有精神的样子\end{définition}
\begin{exemple}\jya khɯɣɲɟɯ zɯ, laʁjɯɣ rɲɯɣrɲɯɣ ɲɤ-tɕɤt\cmn 他把从窗户(往外面)探出来根木棒\end{exemple}
\begin{exemple}\jya jiɕqha nɯ rɲɯɣrɲɯɣ ɲɤ-nɯ-ɬoʁ\cmn (细长的东西)出来了\end{exemple}\begin{sous-entrée}
\vedette{\hypertarget{}{\papi{ sɤrɲɯɣrɲɯɣ}}}\markboth{sɤrɲɯɣrɲɯɣ}{}\classe{vt}
\begin{exemple}\jya qapri kɯ ɯ-mdʑu ɲɯ-sɤrɲɯɣrɲɯɣ\cmn 蛇把舌头伸出来\end{exemple}
\end{sous-entrée}\end{entrée}

\begin{entrée}
\vedette{\hypertarget{Ⓔrɲɯl}{\papi{ rɲɯl}}}\markboth{rɲɯl}{}
\classe{vi}
\paradigme{\textit{dir :} \jya pɯ-}\acception{1}
\begin{définition}\fra fâner\end{définition}
\begin{définition}\cmn 凋谢\end{définition}
\begin{exemple}\jya mɯntoʁ pjɤ-rɲɯl\cmn 花凋谢了\end{exemple}\acception{2}
\begin{définition}\fra se délabrer (maison)\end{définition}
\begin{définition}\cmn 塌下来了\end{définition}
\begin{exemple}\jya kha pjɤ-rɲɯl (=pjɤ-mbɯt)\cmn 房子塌下来了\end{exemple}
\begin{relation-sémantique}\synonyme{
\hyperlink{Ⓔmbɯt}{\textit{ \papi{mbɯt}}}
}\end{relation-sémantique}\acception{3}
\begin{définition}\fra avoir complètement pourri\end{définition}
\begin{définition}\cmn 完全腐烂掉了
\begin{déclaration} \étymologie{\papi{rɲid}}\end{déclaration}\end{définition}
\begin{exemple}\jya pjɤ-tsɣi tɕe pjɤ-rɲɯl\cmn 腐烂了\end{exemple}
\begin{exemple}\jya rɯdaʁ pjɤ-si tɕe pjɤ-rɲɯl\cmn 动物死了然后就腐烂了\end{exemple}
\begin{relation-sémantique}\synonyme{
 \papi{zɯɣ2}
}\end{relation-sémantique}
\begin{relation-sémantique}\confer{
\hyperlink{ⒺjaⒽ2}{\textit{ \papi{ja2}}}
}\end{relation-sémantique}\end{entrée}

\begin{entrée}
\vedette{\hypertarget{Ⓔrŋu}{\papi{ rŋu}}}\markboth{rŋu}{}
\classe{vl}
\paradigme{\textit{dir :} \jya tɤ-}
\paradigme{\textit{dir :} \jya thɯ-}
\begin{définition}\fra frire (le blé)\end{définition}
\begin{définition}\cmn 干炒(麦子)
\begin{déclaration} \étymologie{\papi{rŋo}}\end{déclaration}\end{définition}
\begin{exemple}\jya tɤɕi tɤ-rŋu-t-a\cmn 我炒了青稞\end{exemple}
\begin{exemple}\jya tɤɕi chɯ́-wɣ-rŋu tɕe nɯ kóʁmɯz nɤ tɯsqar ɲɯ-βze ɕti\cmn 炒了青稞就可以做糌粑\end{exemple}\end{entrée}

\begin{entrée}
\vedette{\hypertarget{Ⓔrŋama}{\papi{ rŋama}}}\markboth{rŋama}{}\classe{n}
\begin{définition}\fra (porter à) complétion\end{définition}
\begin{définition}\cmn (做)到底,(做)得彻底
\begin{déclaration} \étymologie{\papi{rŋa.ma}}\end{déclaration}\end{définition}
\begin{exemple}\jya ɯʑo kɯ kɤ-nɤma ra rŋama mɤ-kɯ-ɬoʁ ɲɯ-βde ɲɯ-ɕti\cmn 我做事做到一半就放弃\end{exemple}\begin{sous-entrée}
\vedette{\hypertarget{}{\papi{ rŋama,tɕɤt}}}\markboth{rŋama,tɕɤt}{}
\paradigme{\textit{dir :} \jya thɯ-}
\begin{définition}\fra faire complètement, porter à complétion\end{définition}
\begin{définition}\cmn 做到底\end{définition}
\begin{exemple}\jya rŋama mɯ-chɤ-tɯ-tɕɤt\cmn 你没有把事情做得彻底\end{exemple}
\end{sous-entrée}\end{entrée}

\begin{entrée}
\vedette{\hypertarget{Ⓔrŋamoŋ}{\papi{ rŋamoŋ}}}\markboth{rŋamoŋ}{}
\classe{n}
\begin{définition}\fra chameau\end{définition}
\begin{définition}\cmn 骆驼
\begin{déclaration} \étymologie{\papi{rŋa.moŋ}}\end{déclaration}\end{définition}
\begin{exemple}\jya rŋamoŋ raŋzga\cmn 骆驼(固有的)鞍子\end{exemple}\end{entrée}

\begin{entrée}
\vedette{\hypertarget{Ⓔrŋapa}{\papi{ rŋapa}}}\markboth{rŋapa}{}\classe{n}
\begin{définition}\fra cinquième mois\end{définition}
\begin{définition}\cmn 五月
\begin{déclaration} \étymologie{\papi{lŋa.pa}}\end{déclaration}\end{définition}
\end{entrée}

\begin{entrée}
\vedette{\hypertarget{Ⓔrŋawa}{\papi{ rŋawa}}}\markboth{rŋawa}{}\classe{n}
\begin{définition}\ 
\begin{déclaration}\grammar{n.lieu}\end{déclaration}\end{définition}
\begin{définition}\fra Rngaba\end{définition}
\begin{définition}\cmn 阿坝\end{définition}\end{entrée}

\begin{entrée}
\vedette{\hypertarget{Ⓔrŋɤβrŋɤβ}{\papi{ rŋɤβrŋɤβ}}}\markboth{rŋɤβrŋɤβ}{}
\classe{idph.2}
\begin{définition}\fra haut et fin\end{définition}
\begin{définition}\cmn 形容细而高的样子\end{définition}
\begin{exemple}\jya kumpɣa kɯ ɯ-ku rŋɤβrŋɤβ ʑo to-joʁ\cmn 鸡把头伸得很高(东看西看)\end{exemple}\begin{sous-entrée}
\vedette{\hypertarget{}{\papi{ rŋɤβnɤrŋɤβ}}}\markboth{rŋɤβnɤrŋɤβ}{}\classe{idph.3}
\end{sous-entrée}\end{entrée}

\begin{entrée}
\vedette{\hypertarget{Ⓔrŋɤfsoʁ}{\papi{ rŋɤfsoʁ}}}\markboth{rŋɤfsoʁ}{}\classe{n}
\begin{définition}\fra vache dont la tête est blanche\end{définition}
\begin{définition}\cmn 白头牛\end{définition}\end{entrée}

\begin{entrée}
\vedette{\hypertarget{Ⓔrŋɤɣndʑɯr}{\papi{ rŋɤɣndʑɯr}}}\markboth{rŋɤɣndʑɯr}{}\classe{vi}
\paradigme{\textit{dir :} \jya thɯ-}
\begin{définition}\fra faire frire de la tsampa et moudre des grains d'orge\end{définition}
\begin{définition}\cmn 又炒糌粑又磨面\end{définition}
\begin{relation-sémantique}\confer{
\hyperlink{Ⓔrŋu}{\textit{ \papi{rŋu}}}
}\end{relation-sémantique}
\begin{relation-sémantique}\confer{
\hyperlink{Ⓔɣndʑɯr}{\textit{ \papi{ɣndʑɯr}}}
}\end{relation-sémantique}
\begin{relation-sémantique}\confer{
\hyperlink{Ⓔsɤrŋɤɣndʑɯr}{\textit{ \papi{sɤrŋɤɣndʑɯr}}}
}\end{relation-sémantique}\end{entrée}

\begin{entrée}
\vedette{\hypertarget{Ⓔrŋɤmboʁ}{\papi{ rŋɤmboʁ}}}\markboth{rŋɤmboʁ}{}
\classe{n}
\begin{définition}\fra grains d'orge grillés\end{définition}
\begin{définition}\cmn 青稞爆花\end{définition}\end{entrée}

\begin{entrée}
\vedette{\hypertarget{Ⓔrŋɤrŋɤt}{\papi{ rŋɤrŋɤt}}}\markboth{rŋɤrŋɤt}{}\classe{idph.2}
\begin{définition}\fra imposant\end{définition}
\begin{définition}\cmn 形容雄伟;又高又宽的样子\end{définition}
\begin{exemple}\jya praʁ rŋɤrŋɤt ʑo ɲɯ-pa\cmn 悬崖又高又宽\end{exemple}\begin{sous-entrée}
\vedette{\hypertarget{}{\papi{ mɤlɤrŋɤt}}}\markboth{mɤlɤrŋɤt}{}\classe{idph.6}
\begin{exemple}\jya @wenchuan zgo ra mɤlɤrŋɤt ɲɯ-ŋu\cmn 汶川的山又高又宽\end{exemple}
\end{sous-entrée}\end{entrée}

\begin{entrée}
\vedette{\hypertarget{Ⓔrŋɤʁjoʁ}{\papi{ rŋɤʁjoʁ}}}\markboth{rŋɤʁjoʁ}{}
\classe{n}
\begin{définition}\fra bâton courbé avec lequel on frappe le tambour\end{définition}
\begin{définition}\cmn 用来打鼓的棍子
\begin{déclaration} \étymologie{\papi{rŋa.gjog}}\end{déclaration}\end{définition}\end{entrée}

\begin{entrée}
\vedette{\hypertarget{Ⓔruŋgu}{\papi{ ruŋgu}}}\markboth{ruŋgu}{} (\variante{rɯŋgu}) 
\classe{n}
\begin{définition}\fra pâturage\end{définition}
\begin{définition}\cmn 牧场\end{définition}\end{entrée}

\begin{entrée}
\vedette{\hypertarget{Ⓔrŋgɤβ}{\papi{ rŋgɤβ}}}\markboth{rŋgɤβ}{}
\classe{vt}
\paradigme{\textit{dir :} \jya tɤ-}
\begin{définition}\fra attacher\end{définition}
\begin{définition}\cmn 捆绑\end{définition}
\begin{exemple}\jya tɯrme ka-ndo-nɯ tɕe ɯ-jaʁ tu-rŋgɤβ-nɯ ŋu\cmn 他们抓人的时候就把他的手捆起来\end{exemple}
\begin{exemple}\jya aʑo kɯ ɯ-jaʁ tɤ-rŋgaβ-a\cmn 我捆了他的手\end{exemple}
\begin{relation-sémantique}\synonyme{
\hyperlink{ⒺzgroʁⒽ1}{\textit{ \papi{zgroʁ1}}}
}\end{relation-sémantique}
\begin{relation-sémantique}\confer{
\hyperlink{Ⓔtɯ-mthɤrɴɢɤβ}{\textit{ \papi{tɯ-mthɤrɴɢɤβ}}}
}\end{relation-sémantique}
\begin{relation-sémantique}\confer{
 \papi{ɯ-jaqhɤrŋɤβ}
}\end{relation-sémantique}\end{entrée}

\begin{entrée}
\vedette{\hypertarget{Ⓔrŋgɤm}{\papi{ rŋgɤm}}}\markboth{rŋgɤm}{}
\classe{n}
\begin{définition}\fra morceau dur\end{définition}
\begin{définition}\cmn 硬块;固体\end{définition}
\begin{exemple}\jya χɕɤlkara ɯ-rŋgɤm ɲɯ-ŋu ma ɯ-ɣndʑɤr ɲɯ-maʁ\cmn 冰糖是由硬块组成的,不是粉状的。\end{exemple}\end{entrée}

\begin{entrée}
\vedette{\hypertarget{ⒺrŋgɯⒽ2}{\papi{ rŋgɯ}}}\markboth{rŋgɯ}{}\homonyme{2}\classe{n}
\begin{définition}\fra gros rocher\end{définition}
\begin{définition}\cmn 岩石\end{définition}\end{entrée}

\begin{entrée}
\vedette{\hypertarget{ⒺrŋgɯⒽ1}{\papi{ rŋgɯ}}}\markboth{rŋgɯ}{}\homonyme{1}\classe{vi}\acception{1}
\paradigme{\textit{dir :} \jya kɤ-}
\paradigme{\textit{dir :} \jya lɤ-}
\begin{définition}\fra s'allonger\end{définition}
\begin{définition}\cmn 躺\end{définition}
\begin{exemple}\jya nɯŋa ko-rŋgɯ\cmn 牛躺下了\end{exemple}
\begin{exemple}\jya jla ko-rŋgɯ\cmn 犏牛躺下了\end{exemple}
\begin{exemple}\jya tɯrme kɤ-nɯ-rŋgɯ\cmn 人躺下了\end{exemple}
\begin{exemple}\jya kɤ-nɯ-rŋgɯ-j\cmn 我们躺下了\end{exemple}
\begin{exemple}\jya ɯ-thoʁ lɤ-rŋgɯ\cmn 他在地上躺下了\end{exemple}\acception{2}
\paradigme{\textit{dir :} \jya kɤ-}
\begin{définition}\fra dormir\end{définition}
\begin{définition}\cmn 睡觉\end{définition}
\begin{relation-sémantique}\confer{
\hyperlink{Ⓔnɯkhɤrŋgɯ}{\textit{ \papi{nɯkhɤrŋgɯ}}}
}\end{relation-sémantique}\end{entrée}

\begin{entrée}
\vedette{\hypertarget{Ⓔrŋi}{\papi{ rŋi}}}\markboth{rŋi}{}\classe{vs}
\begin{définition}\fra être encore rouges (braises)\end{définition}
\begin{définition}\cmn (火种)还在燃烧,还没有熄灭,还有复燃的可能\end{définition}
\begin{exemple}\jya smi ɲɯ-rŋi\cmn 火种还没有熄灭\end{exemple}\end{entrée}

\begin{entrée}
\vedette{\hypertarget{Ⓔrŋo}{\papi{ rŋo}}}\markboth{rŋo}{}
\classe{vt}
\paradigme{\textit{dir :} \jya nɯ-}
\begin{définition}\fra emprunter (un objet)\end{définition}
\begin{définition}\cmn 向别人借(能归还原物)\end{définition}
\begin{exemple}\jya nɤ-ɕki sɲɯɣjɯ nɯ-rŋo-t-a\cmn 我向你借了一只笔\end{exemple}
\begin{relation-sémantique}\confer{
\hyperlink{Ⓔɕɯrŋo}{\textit{ \papi{ɕɯrŋo}}}
}\end{relation-sémantique}\end{entrée}

\begin{entrée}
\vedette{\hypertarget{Ⓔrŋɯl}{\papi{ rŋɯl}}}\markboth{rŋɯl}{}
\classe{n}
\begin{définition}\fra argent\end{définition}
\begin{définition}\cmn 银子
\begin{déclaration} \étymologie{\papi{dŋul}}\end{déclaration}\end{définition}
\begin{relation-sémantique}\confer{
\hyperlink{Ⓔaɣɯrŋɯl}{\textit{ \papi{aɣɯrŋɯl}}}
}\end{relation-sémantique}\end{entrée}

\begin{entrée}
\vedette{\hypertarget{Ⓔrŋɯlkhoz}{\papi{ rŋɯlkhoz}}}\markboth{rŋɯlkhoz}{}\classe{n}
\begin{définition}\fra sac pour mettre de l'argent\end{définition}
\begin{définition}\cmn 装银子的口袋\end{définition}\end{entrée}

\begin{entrée}
\vedette{\hypertarget{Ⓔrŋɯzrŋɯz}{\papi{ rŋɯzrŋɯz}}}\markboth{rŋɯzrŋɯz}{} (\variante{rŋɯzŋɯz}) \classe{idph.2}
\begin{définition}\fra osseux, maigrichon\end{définition}
\begin{définition}\cmn 过瘦\end{définition}
\begin{exemple}\jya fsapaʁ mɤ-kɯ-mthu ci rŋɯzrŋɯz ɲɯ-ŋu\cmn 那个牲畜又弱又瘦\end{exemple}
\begin{exemple}\jya tɯrme mɤ-kɯ-mthu ci rŋɯzrŋɯz ɲɯ-ŋu\cmn 那个人又弱又瘦\end{exemple}\begin{sous-entrée}
\vedette{\hypertarget{}{\papi{ rŋɯznɤrŋɯz}}}\markboth{rŋɯznɤrŋɯz}{}\classe{idph.2}
\begin{exemple}\jya rŋɯznɤrŋɯz ɲɯ-tɯ-nɤŋkɯŋke\cmn 你那么瘦,在那里走来走去(骂人的话)\end{exemple}
\end{sous-entrée}\end{entrée}

\begin{entrée}
\vedette{\hypertarget{Ⓔro}{\papi{ ro}}}\markboth{ro}{}\classe{vs}\acception{1}
\paradigme{\textit{dir :} \jya nɯ-}
\begin{définition}\fra en trop\end{définition}
\begin{définition}\cmn 多余的\end{définition}
\begin{exemple}\jya ki tɯ-ŋga ki ɲɯ-ro, kɤ-ɕɣɤz ɲɯ-ra\cmn 这件衣服是多余的,要还给人家(多拿了属于别人的衣服)\end{exemple}
\begin{exemple}\jya ki ɯ-phɯ ki kɤ-kho ɲɤ-ro tɕe ɲɯ-ta-fsɯɣ\cmn 你多给了钱,我找一下零钱给你\end{exemple}\acception{2}
\paradigme{\textit{dir :} \jya tɤ-}
\begin{définition}\fra qui ressort\end{définition}
\begin{définition}\cmn 高出;凸出来\end{définition}
\begin{exemple}\jya tu-ro ʑɣɤrʑɣɤr ʑo ɲɯ-ŋu\cmn 有(一两根)凸出来\end{exemple}\end{entrée}

\begin{entrée}
\vedette{\hypertarget{Ⓔroko}{\papi{ roko}}}\markboth{roko}{}
\classe{n}
\begin{définition}\fra type de métal, ressemble au laiton\end{définition}
\begin{définition}\cmn 金属的一种,类似于黄铜\end{définition}
\begin{exemple}\jya roko tɕhoma\cmn 铜的皮带扣子\end{exemple}\end{entrée}

\begin{entrée}
\vedette{\hypertarget{Ⓔrom}{\papi{ rom}}}\markboth{rom}{}\classe{vs}
\paradigme{\textit{dir :} \jya tɤ-}
\begin{définition}\fra séché\end{définition}
\begin{définition}\cmn 晒干的\end{définition}
\begin{exemple}\jya si to-rom\cmn 木料干了\end{exemple}
\begin{relation-sémantique}\confer{
\hyperlink{Ⓔɯ-ɣrom}{\textit{ \papi{ɯ-ɣrom}}}
}\end{relation-sémantique}\begin{sous-entrée}
\vedette{\hypertarget{}{\papi{ sɯɣrom}}}\markboth{sɯɣrom}{}\classe{vt}
\paradigme{\textit{dir :} \jya nɯ-}
\paradigme{\textit{dir :} \jya tɤ-}
\begin{définition}\fra sécher\end{définition}
\begin{définition}\cmn 晒干\end{définition}
\begin{exemple}\jya paʁndza tɤ-sɯɣrom-a\cmn 我把猪食晒干了\end{exemple}
\begin{exemple}\jya sɯjno tɤ-sɯɣrom-a\cmn 我把草晒干了\end{exemple}
\begin{exemple}\jya sɯjno ɲɤ-sɯɣrom\cmn 他把草晒干了\end{exemple}
\begin{exemple}\jya lɤpɯɣ ɲɤ-sɯɣrom\cmn 他把萝卜晒干了\end{exemple}
\end{sous-entrée}\end{entrée}

\begin{entrée}
\vedette{\hypertarget{Ⓔromɲa}{\papi{ romɲa}}}\markboth{romɲa}{}
\classe{n}
\begin{définition}\fra poutre\end{définition}
\begin{définition}\cmn 小梁\end{définition}\end{entrée}

\begin{entrée}
\vedette{\hypertarget{Ⓔroŋba}{\papi{ roŋba}}}\markboth{roŋba}{}
\classe{n}
\begin{définition}\fra locuteurs du rgyalrong oriental\end{définition}
\begin{définition}\cmn 讲四土话的藏族
\begin{déclaration} \étymologie{\papi{roŋ.pa}}\end{déclaration}\end{définition}
\begin{exemple}\jya ɣnɤsqi-xpa ɕɯŋgɯ zɯ roŋba-skɤt pjɤ-βzjoz\cmn 他20年前学了四土话\end{exemple}\end{entrée}

\begin{entrée}
\vedette{\hypertarget{Ⓔroŋwa}{\papi{ roŋwa}}}\markboth{roŋwa}{}
\classe{n}
\begin{définition}\fra agriculteurs\end{définition}
\begin{définition}\cmn 农民
\begin{déclaration} \étymologie{\papi{roŋ.ba}}\end{déclaration}\end{définition}\end{entrée}

\begin{entrée}
\vedette{\hypertarget{Ⓔroŋzga}{\papi{ roŋzga}}}\markboth{roŋzga}{}\classe{n}
\begin{définition}\fra bosse (chameau)\end{définition}
\begin{définition}\cmn 峰(骆驼)\end{définition}
\begin{exemple}\jya rŋamoŋ roŋzga\end{exemple}\end{entrée}

\begin{entrée}
\vedette{\hypertarget{Ⓔrorʁe}{\papi{ rorʁe}}}\markboth{rorʁe}{}\classe{n}
\begin{définition}\cmn 走缘边小柱头之间的穿杆\end{définition}
\begin{exemple}\jya jɤɣɤt laχtsɯ cho mɤro ɣɯ ɯ-kɯ-spoʁ ɯ-ŋgɯ jɯ-kɤ-rʁe laʁjɯɣ nɯ rorʁe rmi\cmn 
走缘柱头和梁架的洞里穿过去的木棒叫\stylefv{rorʁe}
\end{exemple}
\end{entrée}

\begin{entrée}
\vedette{\hypertarget{Ⓔroʁ}{\papi{ roʁ}}}\markboth{roʁ}{}\classe{vt}
\paradigme{\textit{dir :} \jya kɤ-}
\paradigme{\textit{dir :} \jya thɯ-}\acception{1}
\begin{définition}\fra graver\end{définition}
\begin{définition}\cmn 雕刻\end{définition}\acception{2}
\begin{définition}\fra acculer (chasseur)\end{définition}
\begin{définition}\cmn 追到最险要的地方不让动物逃走(猎人)\end{définition}
\begin{exemple}\jya ɕoŋβzu kɯ ji-lɤtɕhom tha-roʁ\cmn 木匠刻了我们的大奶桶\end{exemple}\begin{sous-entrée}
\vedette{\hypertarget{}{\papi{ rɤroʁ}}}\markboth{rɤroʁ}{}\classe{vi}
\paradigme{\textit{dir :} \jya tɤ-}
\begin{définition}\ 
\begin{déclaration}\grammar{apass}\end{déclaration}\end{définition}
\begin{définition}\fra graver\end{définition}
\begin{définition}\cmn 刻\end{définition}
\end{sous-entrée}\end{entrée}

\begin{entrée}
\vedette{\hypertarget{Ⓔroʁrɯz}{\papi{ roʁrɯz}}}\markboth{roʁrɯz}{}
\classe{n}
\begin{définition}\fra changement et balayage\end{définition}
\begin{définition}\cmn 收拾东西和扫地\end{définition}
\begin{exemple}\jya kha tɕe sɤskɯsku ʑo roʁrɯz kɤ-βzu ra\cmn 家里每天早上都要打扫和收拾东西\end{exemple}
\begin{relation-sémantique}\confer{
\hyperlink{Ⓔrɤroʁrɯz}{\textit{ \papi{rɤroʁrɯz}}}
}\end{relation-sémantique}
\begin{relation-sémantique}\confer{
\hyperlink{Ⓔraʁrɯz}{\textit{ \papi{raʁrɯz}}}
}\end{relation-sémantique}\end{entrée}

\begin{entrée}
\vedette{\hypertarget{Ⓔrpu}{\papi{ rpu}}}\markboth{rpu}{}
\classe{vl}
\paradigme{\textit{dir :} \jya kɤ-}
\begin{définition}\fra cogner\end{définition}
\begin{définition}\cmn 撞;碰撞\end{définition}
\begin{exemple}\jya ɯ-taʁ kɤ-rpu-a\cmn 我撞了他\end{exemple}\begin{sous-entrée}
\vedette{\hypertarget{}{\papi{ nɯrpu}}}\markboth{nɯrpu}{}\classe{vt}
\paradigme{\textit{dir :} \jya kɤ-}
\begin{définition}\ 
\begin{déclaration}\grammar{autoben}\end{déclaration}\end{définition}
\begin{définition}\fra se cogner\end{définition}
\begin{définition}\cmn 撞\end{définition}
\begin{exemple}\jya nɤ-ku ko-tɯ-nɯrpu-t tɕe pjɤ-tɯ-phɤβ pjɤ-ra\cmn 你撞了头,你本来应该低头\end{exemple}
\begin{exemple}\jya alo jiʑo ji-kha nɤ-ku ko-tɯ-nɯrpu-t khi loβtɕi\cmn (听说)你在我们干木鸟的家里撞了头,对吧?\end{exemple}
\begin{exemple}\jya a-mi kɤ-nɯrpu-t-a\cmn 我撞到脚了\end{exemple}
\end{sous-entrée}\begin{sous-entrée}
\vedette{\hypertarget{}{\papi{ ɯ-rpu,lɤt}}}\markboth{ɯ-rpu,lɤt}{}
\begin{définition}\fra frapper avec ...\end{définition}
\begin{définition}\cmn 用……打\end{définition}
\begin{exemple}\jya a-taʁ taʁndzɤr-rpu ta-lɤt\cmn 他用喂猪槽打了我\end{exemple}
\begin{exemple}\jya a-taʁ khɯtsa-rpu ta-lɤt\cmn 他用碗打了我\end{exemple}
\begin{relation-sémantique}\ComponentA{\classe{np}
 \papi{ɯ-rpu}
}\end{relation-sémantique}
\begin{relation-sémantique}\ComponentB{\classe{vt}
\hyperlink{ⒺlɤtⒽ1}{\textit{ \papi{lɤt}}}
}\end{relation-sémantique}
\end{sous-entrée}\begin{sous-entrée}
\vedette{\hypertarget{}{\papi{ ʑɣɤrpu}}}\markboth{ʑɣɤrpu}{}\classe{vi}
\paradigme{\textit{dir :} \jya kɤ-}
\begin{définition}\ 
\begin{déclaration}\grammar{refl}\end{déclaration}\end{définition}
\begin{définition}\fra se cogner soi-même\end{définition}
\begin{définition}\cmn 撞到自己\end{définition}
\end{sous-entrée}\end{entrée}

\begin{entrée}
\vedette{\hypertarget{Ⓔrpɤŋgɯ}{\papi{ rpɤŋgɯ}}}\markboth{rpɤŋgɯ}{}\classe{n}
\begin{définition}\ 
\begin{déclaration}\grammar{n.lieu}\end{déclaration}\end{définition}
\begin{définition}\fra l'un des hameaux de Kamnyu\end{définition}
\begin{définition}\cmn 干木鸟的大队之一\end{définition}
\end{entrée}

\begin{entrée}
\vedette{\hypertarget{Ⓔrpɣo}{\papi{ rpɣo}}}\markboth{rpɣo}{}\classe{n}
\begin{définition}\fra en haut de la montagne\end{définition}
\begin{définition}\cmn 高山上\end{définition}\end{entrée}

\begin{entrée}
\vedette{\hypertarget{Ⓔrpɣorɤku}{\papi{ rpɣorɤku}}}\markboth{rpɣorɤku}{}
\classe{n}
\begin{définition}\fra cultures de haute montagne\end{définition}
\begin{définition}\cmn 高山作物(胡豆、豌豆、青稞、小麦、原根、莜麦)\end{définition}
\begin{relation-sémantique}\confer{
\hyperlink{Ⓔrpɣo}{\textit{ \papi{rpɣo}}}
}\end{relation-sémantique}
\begin{relation-sémantique}\confer{
\hyperlink{Ⓔtɤ-rɤku}{\textit{ \papi{tɤ-rɤku}}}
}\end{relation-sémantique}\end{entrée}

\begin{entrée}
\vedette{\hypertarget{Ⓔrpjɯ}{\papi{ rpjɯ}}}\markboth{rpjɯ}{}
\classe{vs}
\paradigme{\textit{dir :} \jya tɤ-}
\begin{définition}\fra abîmé (lait)\end{définition}
\begin{définition}\cmn 变质(牛奶)\end{définition}
\begin{exemple}\jya tɤ-lu to-rpjɯ\cmn 牛奶坏了\end{exemple}\begin{sous-entrée}
\vedette{\hypertarget{}{\papi{ sɯrpjɯ}}}\markboth{sɯrpjɯ}{}\classe{vt}
\paradigme{\textit{dir :} \jya tɤ-}
\begin{définition}\fra laisser s'abîmer (laità\end{définition}
\begin{définition}\cmn 让(牛奶)变质\end{définition}
\begin{exemple}\jya tɤ-lu a-mɤ-tɤ-tɯ-sɯrpji ma nɤja\cmn 你不要让牛奶变质,不然太可惜\end{exemple}
\end{sous-entrée}\end{entrée}

\begin{entrée}
\vedette{\hypertarget{Ⓔrpɯ}{\papi{ rpɯ}}}\markboth{rpɯ}{}
\classe{vs}
\paradigme{\textit{dir :} \jya tɤ-}
\paradigme{\textit{dir :} \jya thɯ-}
\begin{définition}\fra sale (cheveux)\end{définition}
\begin{définition}\cmn 头发又脏又长\end{définition}
\begin{exemple}\jya ɯ-ku chɤ-rpɯ\cmn 他的头发脏了\end{exemple}\end{entrée}

\begin{entrée}
\vedette{\hypertarget{Ⓔrqaco}{\papi{ rqaco}}}\markboth{rqaco}{}\classe{n}
\begin{définition}\ 
\begin{déclaration}\grammar{n.lieu}\end{déclaration}\end{définition}
\begin{définition}\fra Rqakyo (village de Gdongbrgyad)\end{définition}
\begin{définition}\cmn 尕脚村\end{définition}
\end{entrée}

\begin{entrée}
\vedette{\hypertarget{Ⓔrqɤnrqɤn}{\papi{ rqɤnrqɤn}}}\markboth{rqɤnrqɤn}{}\classe{idph.2}
\begin{définition}\fra doré\end{définition}
\begin{définition}\cmn 形容红中带有金黄的晚霞的颜色\end{définition}
\begin{exemple}\jya prɤɲi ɲɯ-ɣɯrni ʑo rqɤnrqɤn\cmn 晚霞是红色的\end{exemple}
\begin{exemple}\jya ɯ-mɲaʁ ɕɤwɤr to-βzu tɕe ɲɯ-ɣɯrni ʑo rqɤnrqɤn\cmn 他眼睛有结膜炎,非常红\end{exemple}\end{entrée}

\begin{entrée}
\vedette{\hypertarget{Ⓔrqhaŋrqhaŋ}{\papi{ rqhaŋrqhaŋ}}}\markboth{rqhaŋrqhaŋ}{}\classe{idph.2}
\begin{définition}\fra grand et mince\end{définition}
\begin{définition}\cmn 形容大而瘦的样子\end{définition}
\begin{exemple}\jya si pjɤ-rom tɕe rqhaŋrqhaŋ ʑo ɲɯ-pa\cmn 树干了,显得又大又瘦\end{exemple}\end{entrée}

\begin{entrée}
\vedette{\hypertarget{Ⓔrqhɤrqhɤt}{\papi{ rqhɤrqhɤt}}}\markboth{rqhɤrqhɤt}{}\classe{idph.2}
\begin{définition}\fra qui vient de sortir (champignon, plante)\end{définition}
\begin{définition}\cmn 形容菌子、草等又新鲜又结实的样子\end{définition}\end{entrée}

\begin{entrée}
\vedette{\hypertarget{Ⓔrqhoʁ}{\papi{ rqhoʁ}}}\markboth{rqhoʁ}{}
\classe{idph.1}
\begin{définition}\fra coup de fusil\end{définition}
\begin{définition}\cmn 打枪的声音\end{définition}
\begin{exemple}\jya ɕɤmɯɣdɯ rqhoʁ ʑo ta-lɤt\cmn 他啪一声地射了枪\end{exemple}
\begin{relation-sémantique}\confer{
\hyperlink{Ⓔɣɤrqhoʁrqhoʁ}{\textit{ \papi{ɣɤrqhoʁrqhoʁ}}}
}\end{relation-sémantique}\end{entrée}

\begin{entrée}
\vedette{\hypertarget{Ⓔrqoʁ}{\papi{ rqoʁ}}}\markboth{rqoʁ}{}
\classe{vt}\acception{1}
\paradigme{\textit{dir :} \jya kɤ-}
\begin{définition}\fra prendre dans les bras\end{définition}
\begin{définition}\cmn 抱;搂\end{définition}
\begin{exemple}\jya kɤ́-wɣ-rqoʁ-a\cmn 他抱了我\end{exemple}
\begin{exemple}\jya ko-rqoʁ\cmn 抱了他\end{exemple}\acception{2}
\paradigme{\textit{dir :} \jya tɤ-}
\begin{définition}\fra prendre dans les bras et relever\end{définition}
\begin{définition}\cmn 抱起来\end{définition}
\begin{exemple}\jya ta-rqoʁ\cmn 他把他抱起来了\end{exemple}
\begin{relation-sémantique}\confer{
\hyperlink{Ⓔtɯ-rqoʁ}{\textit{ \papi{tɯ-rqoʁ}}}
}\end{relation-sémantique}
\begin{relation-sémantique}\confer{
\hyperlink{Ⓔarqɯrqoʁ}{\textit{ \papi{arqɯrqoʁ}}}
}\end{relation-sémantique}\end{entrée}

\begin{entrée}
\vedette{\hypertarget{Ⓔrʁu}{\papi{ rʁu}}}\markboth{rʁu}{}\classe{vs}\acception{1}
\paradigme{\textit{dir :} \jya pɯ-}
\begin{définition}\fra s'évaporer\end{définition}
\begin{définition}\cmn 蒸发掉\end{définition}
\begin{exemple}\jya tɯ-ci tú-wɣ-sɤla qhe pjɯ-rʁu ɕti\cmn 烧开水的时候水会蒸发掉\end{exemple}\acception{2}
\paradigme{\textit{dir :} \jya nɯ-}
\begin{définition}\fra se tarir (lait d'une vache)\end{définition}
\begin{définition}\cmn 干(奶水)\end{définition}
\begin{exemple}\jya nɯŋa nɯ-rɤpɯ ɯ-qhu χsɯ-sla jamar tɕe ɯ-lu ɲɯ-rʁu ɕti\cmn 奶牛生崽三个月后,奶水就干了\end{exemple}\end{entrée}

\begin{entrée}
\vedette{\hypertarget{Ⓔrʁɤβrʁɤβ}{\papi{ rʁɤβrʁɤβ}}}\markboth{rʁɤβrʁɤβ}{}\classe{idph.2}\acception{1}
\begin{définition}\fra rugueux\end{définition}
\begin{définition}\cmn 形容粗糙的样子\end{définition}\acception{2}
\begin{définition}\fra séparées, pas ramassées ensemble (feuilles)\end{définition}
\begin{définition}\cmn 不连贯;分散(叶子)\end{définition}
\begin{exemple}\jya zdɯɬa ɣɯ ɯ-jwaʁ nɯ kú-wɣ-rtoʁ qhe mɤ-andzoʁjoʁ kɯ rʁɤβrʁɤβ ʑo pa, tɕeri ɲɯ́-wɣ-nɤmɯma tɕe mnu ma mɤ-rʁom\cmn 芍药花的叶子看起来不连贯、零碎,但是摸起来很光滑,不粗糙\end{exemple}\end{entrée}

\begin{entrée}
\vedette{\hypertarget{Ⓔrʁɤrʁɤt}{\papi{ rʁɤrʁɤt}}}\markboth{rʁɤrʁɤt}{}\classe{idph.2}
\begin{définition}\fra tenant qqch très fort\end{définition}
\begin{définition}\cmn 形容抓得很紧、抱得很紧的样子\end{définition}
\begin{exemple}\jya qajɯ nɯ sɯjno ɯ-ku rʁɤrʁɤt ʑo ku-ɴqoʁ ŋu\cmn 虫子紧紧地抓住植物\end{exemple}\end{entrée}

\begin{entrée}
\vedette{\hypertarget{Ⓔrʁe}{\papi{ rʁe}}}\markboth{rʁe}{}
\classe{vt}
\paradigme{\textit{dir :} \jya nɯ-}
\paradigme{\textit{dir :} \jya thɯ-}
\begin{définition}\fra enfiler, passer à travers\end{définition}
\begin{définition}\cmn 穿\end{définition}
\begin{exemple}\jya ɕnɤloʁ na-rʁe\cmn 我穿了牛鼻圈\end{exemple}
\begin{exemple}\jya zndɤrchɤβ na-rʁe\cmn 他(把东西)穿进墙缝了\end{exemple}
\begin{exemple}\jya taqaβrna na-rʁe (nɯ-rʁe-t-a)\cmn 他穿了针(我穿了针)\end{exemple}
\begin{exemple}\jya ɯ-jaʁ ɯ-pɤloʁ ɯ-ŋgɯ tha-rʁe\cmn 他把手穿进袖子里了\end{exemple}
\begin{exemple}\jya rkɤsnom ɯ-mi tha-rʁe\cmn 他把脚穿进裤子里了\end{exemple}
\begin{relation-sémantique}\confer{
\hyperlink{Ⓔnɯrʁe}{\textit{ \papi{nɯrʁe}}}
}\end{relation-sémantique}\end{entrée}

\begin{entrée}
\vedette{\hypertarget{Ⓔrʁom}{\papi{ rʁom}}}\markboth{rʁom}{}
\classe{vs}
\paradigme{\textit{dir :} \jya tɤ-}
\paradigme{\textit{dir :} \jya thɯ-}
\begin{définition}\fra rugueux\end{définition}
\begin{définition}\cmn 粗糙\end{définition}
\begin{exemple}\jya tɯ-ŋga ɲɯ-rʁom\cmn 衣服很粗糙\end{exemple}
\begin{exemple}\jya nɯ ɯ-ɣmbɤrme ɲɯ-rʁom\cmn 他的胡子很粗糙\end{exemple}
\begin{relation-sémantique}\antonyme{
\hyperlink{Ⓔmnu}{\textit{ \papi{mnu}}}
}\end{relation-sémantique}\end{entrée}

\begin{entrée}
\vedette{\hypertarget{Ⓔrʁoʁrʁoʁ}{\papi{ rʁoʁrʁoʁ}}}\markboth{rʁoʁrʁoʁ}{}\classe{idph.2}\acception{1}
\begin{définition}\fra frisé (cheveu)\end{définition}
\begin{définition}\cmn 卷(头发)\end{définition}\acception{2}
\begin{définition}\fra peu agile\end{définition}
\begin{définition}\cmn 不灵活\end{définition}
\begin{exemple}\jya ɯ-ku rʁoʁrʁoʁ ʑo ɲɯ-pa\end{exemple}
\begin{exemple}\jya ɯ-ku kɯ-ɤrʁɯrʁu ci rʁoʁrʁoʁ ɲɯ-ŋu\cmn 他的头发是卷卷的\end{exemple}
\begin{exemple}\jya jiɕqha tɯrme nɯ mɤ-kɯ-ɤɕpala ci rʁoʁrʁoʁ ɲɯ-ɕti\cmn 那个人动作不灵活\end{exemple}\end{entrée}

\begin{entrée}
\vedette{\hypertarget{Ⓔrʁɯβrʁɯβ}{\papi{ rʁɯβrʁɯβ}}}\markboth{rʁɯβrʁɯβ}{}\classe{idph.2}\acception{1}
\begin{définition}\fra qui porte beaucoup de fruit\end{définition}
\begin{définition}\cmn 结的果子很多\end{définition}\acception{2}
\begin{définition}\fra devenu rugueux après avoir été séché\end{définition}
\begin{définition}\cmn 由于干燥而变得很粗糙\end{définition}
\begin{exemple}\jya jiɕqha si nɯ rʁɯβrʁɯβ kɯ-ɤɣɯmat ci ɲɯ-ŋu\cmn 那棵树结的果子很多\end{exemple}
\begin{exemple}\jya ɯ-mat rʁɯβrʁɯβ ʑo ɲɯ-pa\cmn 果子很多\end{exemple}
\begin{exemple}\jya si ɯ-rtaʁ rʁɯβrʁɯβ ɲɯ-pa\cmn 树枝很多\end{exemple}
\begin{exemple}\jya sɤtɕha to-khrɯ rʁɯβrʁɯβ ʑo\cmn 地变得很干\end{exemple}\begin{sous-entrée}
\vedette{\hypertarget{}{\papi{ rʁɯβnɤrʁɯβ}}}\markboth{rʁɯβnɤrʁɯβ}{}\classe{idph.3}
\begin{définition}\fra pressé\end{définition}
\begin{définition}\cmn 急躁\end{définition}
\begin{exemple}\jya jiɕqha tɯrme rcánɯ rʁɯβnɤrʁɯβ kɯ-znɤʁamɟa ci ɲɯ-ŋu\cmn 那个人做事很急躁\end{exemple}
\begin{exemple}\jya rʁɯβnɤrʁɯβ ɲɯ-ɤsɯ-ndza\cmn 他在大声地吃(又脆又干的东西)\end{exemple}
\begin{relation-sémantique}\confer{
\hyperlink{Ⓔsɤrʁɯrʁɯβ}{\textit{ \papi{sɤrʁɯrʁɯβ}}}
}\end{relation-sémantique}
\end{sous-entrée}\end{entrée}

\begin{entrée}
\vedette{\hypertarget{Ⓔrʁɯm}{\papi{ rʁɯm}}}\markboth{rʁɯm}{}\classe{idph.1}
\begin{définition}\fra se recroqueviller d'un seul coup\end{définition}
\begin{définition}\cmn 突然间卷起来;突然收拢起来\end{définition}
\begin{exemple}\jya qarma mtshalu ɣɯ ɯ-mat, tɯ-jaʁ a-nɯ-ɤtɯɣ qhe, rʁɯm ʑo ɲɯ-ti tɕe, ɯ-rɣi nɯ pjɯ-nɯɬoʁ, ɯ-rqhu nɯ lu-orʁɯrʁu ŋu\cmn 荨麻的果子碰到手就会突然间收拢起来,种子很快落地\end{exemple}
\begin{exemple}\jya tɤŋkɯ chɯ́-wɣ-nɯ-pu tɕe tɤ-sɤɕke tɕe, rʁɯm ʑo tu-ti tɕe ku-owɯwum ŋu\cmn 烤猪皮的时候,烧烫了就会马上卷起来,收拢起来\end{exemple}\end{entrée}

\begin{entrée}
\vedette{\hypertarget{Ⓔrsoŋrsoŋ}{\papi{ rsoŋrsoŋ}}}\markboth{rsoŋrsoŋ}{}\classe{idph.2}
\begin{définition}\fra très poilu\end{définition}
\begin{définition}\cmn 形容毛茸茸\end{définition}
\begin{exemple}\jya ɯ-mtɕhirme rsoŋrsoŋ ʑo ɲɯ-pa\cmn 他的胡须是毛茸茸的\end{exemple}\end{entrée}

\begin{entrée}
\vedette{\hypertarget{Ⓔrsɯβrsɯβ}{\papi{ rsɯβrsɯβ}}}\markboth{rsɯβrsɯβ}{} (\variante{rsɯprsɯp}) \classe{idph.2}
\begin{définition}\fra poilu\end{définition}
\begin{définition}\cmn 毛多的\end{définition}
\begin{exemple}\jya kɤɣɯrme ci rsɯprsɯp ɲ-ɯŋu\cmn 毛很多\end{exemple}
\begin{exemple}\jya ɯ-mi ɯ-rme rsɯprsɯp ʑo ɲɯ-pa\cmn 他脚上的毛很多\end{exemple}
\begin{exemple}\jya sɤtɕha ra rsɯprsɯp ʑo ɲɯ-pa\cmn 地上草很多\end{exemple}\begin{sous-entrée}
\vedette{\hypertarget{}{\papi{ rsɯβnɤrsɯβ}}}\markboth{rsɯβnɤrsɯβ}{}
\begin{définition}\fra bruit de feuilles mortes\end{définition}
\begin{définition}\cmn 树叶沙沙响的声音\end{définition}
\begin{exemple}\jya sɯŋgɯ rsɯpnɤrsɯp ɲɯ-ŋke\cmn 他在森林里走动,发出沙沙声\end{exemple}
\begin{exemple}\jya soʁma ɯ-ŋgɯ rsɯpnɤrsɯp kɤ-ari\cmn 他在干草里去了,发出沙沙声\end{exemple}
\begin{relation-sémantique}\confer{
\hyperlink{Ⓔɣɤrsɯβrsɯβ}{\textit{ \papi{ɣɤrsɯβrsɯβ}}}
}\end{relation-sémantique}
\begin{relation-sémantique}\confer{
\hyperlink{Ⓔsɯβsɯβ}{\textit{ \papi{sɯβsɯβ}}}
}\end{relation-sémantique}
\end{sous-entrée}\end{entrée}

\begin{entrée}
\vedette{\hypertarget{Ⓔrtaβrɤn}{\papi{ rtaβrɤn}}}\markboth{rtaβrɤn}{}
\classe{n}
\begin{définition}\fra cheval castré\end{définition}
\begin{définition}\cmn 骟马
\begin{déclaration} \étymologie{\papi{rta}}\end{déclaration}\end{définition}\end{entrée}

\begin{entrée}
\vedette{\hypertarget{Ⓔrtakhaŋ}{\papi{ rtakhaŋ}}}\markboth{rtakhaŋ}{}\classe{n}
\begin{définition}\fra écurie\end{définition}
\begin{définition}\cmn 马厩\end{définition}
\begin{relation-sémantique}\synonyme{
\hyperlink{Ⓔmbrosta}{\textit{ \papi{mbrosta}}}
}\end{relation-sémantique}\end{entrée}

\begin{entrée}
\vedette{\hypertarget{Ⓔrtalu}{\papi{ rtalu}}}\markboth{rtalu}{}\classe{n}
\begin{définition}\fra année du cheval\end{définition}
\begin{définition}\cmn 马年
\begin{déclaration} \étymologie{\papi{rta.lo}}\end{déclaration}\end{définition}
\end{entrée}

\begin{entrée}
\vedette{\hypertarget{Ⓔrtamu}{\papi{ rtamu}}}\markboth{rtamu}{}\classe{n}
\begin{définition}\fra jument\end{définition}
\begin{définition}\cmn 母马\end{définition}
\begin{relation-sémantique}\synonyme{
\hyperlink{Ⓔrgonma}{\textit{ \papi{rgonma}}}
}\end{relation-sémantique}\end{entrée}

\begin{entrée}
\vedette{\hypertarget{Ⓔrtamdɯt}{\papi{ rtamdɯt}}}\markboth{rtamdɯt}{}
\classe{n}
\begin{définition}\fra un type de nœud\end{définition}
\begin{définition}\cmn 一种打结的方式
\begin{déclaration} \étymologie{\papi{rta.mdud}}\end{déclaration}\end{définition}\end{entrée}

\begin{entrée}
\vedette{\hypertarget{Ⓔrtaphu}{\papi{ rtaphu}}}\markboth{rtaphu}{}\classe{n}
\begin{définition}\fra étalon\end{définition}
\begin{définition}\cmn 公马\end{définition}
\begin{relation-sémantique}\confer{
\hyperlink{Ⓔχsɤβ}{\textit{ \papi{χsɤβ}}}
}\end{relation-sémantique}\end{entrée}

\begin{entrée}
\vedette{\hypertarget{Ⓔrtaʁ}{\papi{ rtaʁ}}}\markboth{rtaʁ}{}
\classe{vs}
\paradigme{\textit{dir :} \jya tɤ-}
\begin{définition}\fra assez\end{définition}
\begin{définition}\cmn 足够\end{définition}
\begin{exemple}\jya kɤ-ndza ɲɯ-rtaʁ\cmn 够吃\end{exemple}
\begin{exemple}\jya kɤ-ŋga ɲɯ-rtaʁ\cmn 够穿\end{exemple}
\begin{exemple}\jya ɯʑo kɯ tɯ-rtaʁ kɯ-me ʑo ɲɯ́-wɣ-rɯɣne-a ɲɯ-ŋu\cmn 他无端地埋怨我\end{exemple}\begin{sous-entrée}
\vedette{\hypertarget{}{\papi{ artaʁlaʁ}}}\markboth{artaʁlaʁ}{}\classe{vs}
\begin{définition}\fra suffisant\end{définition}
\begin{définition}\cmn 足够\end{définition}
\begin{exemple}\jya kɤ-mbi ɯ-spa ɲɯ-rkɯn tɕe kɯ-ɤrtaʁlaʁ maŋe\cmn 给的东西太少,不够分给大家\end{exemple}
\begin{relation-sémantique}\confer{
\hyperlink{Ⓔɣɤrtaʁ}{\textit{ \papi{ɣɤrtaʁ}}}
}\end{relation-sémantique}
\end{sous-entrée}\end{entrée}

\begin{entrée}
\vedette{\hypertarget{Ⓔrtɤβ}{\papi{ rtɤβ}}}\markboth{rtɤβ}{}
\classe{vt}\acception{1}
\paradigme{\textit{dir :} \jya pɯ-}
\begin{définition}\fra frapper\end{définition}
\begin{définition}\cmn 打(用细长的东西)
\begin{déclaration}\use{该动词的宾语是用来打的工具,而不是挨打的人或者东西}\end{déclaration}\end{définition}
\begin{exemple}\jya laʁjɯɣ pa-rtɤβ\cmn 他用棍子打了\end{exemple}
\begin{exemple}\jya tɤ-pɤtso mɯ́j-khɯ tɕe, tɤtar pɯ-rtaβ-a\cmn 小孩子不听话,我用了一根木条打他\end{exemple}\acception{2}
\paradigme{\textit{dir :} \jya nɯ-}
\paradigme{\textit{dir :} \jya kɤ-}
\begin{définition}\fra attacher\end{définition}
\begin{définition}\cmn 缠线;拴(鞋带);围起来\end{définition}
\begin{exemple}\jya mthɯxtɕɤr na-rtɤβ\cmn 他拴了腰带\end{exemple}
\begin{exemple}\jya xtsɤxtɕɤr na-nɯ-rtɤβ\cmn 他系了鞋带\end{exemple}
\begin{relation-sémantique}\confer{
\hyperlink{Ⓔɯ-rtɯrtɤβ}{\textit{ \papi{ɯ-rtɯrtɤβ}}}
}\end{relation-sémantique}\end{entrée}

\begin{entrée}
\vedette{\hypertarget{Ⓔrtɤdaʁ}{\papi{ rtɤdaʁ}}}\markboth{rtɤdaʁ}{}
\classe{n}
\begin{définition}\fra corde en poil\end{définition}
\begin{définition}\cmn 牛毛搓成的绳子\end{définition}\end{entrée}

\begin{entrée}
\vedette{\hypertarget{Ⓔrtɤltɕaʁ}{\papi{ rtɤltɕaʁ}}}\markboth{rtɤltɕaʁ}{}
\classe{n}
\begin{définition}\fra fouet de cheval\end{définition}
\begin{définition}\cmn 打马的鞭子
\begin{déclaration} \étymologie{\papi{rta.ltɕag}}\end{déclaration}\end{définition}\end{entrée}

\begin{entrée}
\vedette{\hypertarget{Ⓔrtɤmkɯχsɤl}{\papi{ rtɤmkɯχsɤl}}}\markboth{rtɤmkɯχsɤl}{}\classe{adv}
\begin{définition}\fra directement\end{définition}
\begin{définition}\cmn 干脆\end{définition}
\begin{exemple}\jya kɤ-ɕe mɯ́j-tɯ-sɯsɤm ɕti qhe rtɤmkɯχsɤl kɤ-nɯ-rɤʑi\cmn 你既然不想去,就干脆留在这里\end{exemple}
\end{entrée}

\begin{entrée}
\vedette{\hypertarget{Ⓔrtɕhɯɣrtɕhɯɣ}{\papi{ rtɕhɯɣrtɕhɯɣ}}}\markboth{rtɕhɯɣrtɕhɯɣ}{}\classe{idph.2}
\begin{définition}\fra qui a un problème\end{définition}
\begin{définition}\cmn 形容有毛病,让人看不顺眼的样子\end{définition}
\begin{exemple}\jya ki tɯrme ki rtɕɯɣrtɕɯɣ ʑo ɯ-tshɯɣa mɯ́j-βdi\cmn 这个人样子不顺眼\end{exemple}
\begin{exemple}\jya ki laχtɕha ki rtɕɯɣrtɕɯɣ, sna maŋe\cmn 这个东西有毛病,不能要\end{exemple}\end{entrée}

\begin{entrée}
\vedette{\hypertarget{Ⓔrtɕhɯʁjɯ}{\papi{ rtɕhɯʁjɯ}}}\markboth{rtɕhɯʁjɯ}{}
\classe{n}
\begin{définition}\fra chenille\end{définition}
\begin{définition}\cmn 毛虫\end{définition}\end{entrée}

\begin{entrée}
\vedette{\hypertarget{Ⓔrtoʁ}{\papi{ rtoʁ}}}\markboth{rtoʁ}{}
\classe{vt}\acception{1}
\paradigme{\textit{dir :} \jya tɤ-}
\begin{définition}\fra regarder\end{définition}
\begin{définition}\cmn 看
\begin{déclaration} \étymologie{\papi{rtogs}}\end{déclaration}\end{définition}
\begin{exemple}\jya laχtɕha tɤ-rtoʁ-a\cmn 我看了东西\end{exemple}
\begin{exemple}\jya jɯɣi ta-rtoʁ\cmn 他看了书\end{exemple}\acception{2}
\paradigme{\textit{dir :} \jya tɤ-}
\begin{définition}\fra vérifier\end{définition}
\begin{définition}\cmn 查看\end{définition}
\begin{exemple}\jya smɤnba kɯ tɤ́-wɣ-rto-ʁa\cmn 医生给我看病了\end{exemple}\acception{3}
\paradigme{\textit{dir :} \jya kɤ-}
\begin{définition}\fra observer\end{définition}
\begin{définition}\cmn 观察\end{définition}
\begin{exemple}\jya ɯʑo kɤ-rtoʁ-a ri, wuma ʑo ɲɯ-stu\cmn 经过我的观察,我认为他是很努力的人\end{exemple}
\begin{relation-sémantique}\synonyme{
\hyperlink{Ⓔχpjɤt}{\textit{ \papi{χpjɤt}}}
}\end{relation-sémantique}
\begin{relation-sémantique}\confer{
\hyperlink{Ⓔnɯsɯrtoʁ}{\textit{ \papi{nɯsɯrtoʁ}}}
}\end{relation-sémantique}
\begin{sous-entrée}
\vedette{\hypertarget{}{\papi{ artɯrtoʁ}}}\markboth{artɯrtoʁ}{}\classe{vi}
\begin{définition}\ 
\begin{déclaration}\grammar{recip}\end{déclaration}\end{définition}
\begin{définition}\fra se regarder les uns les autres\end{définition}
\begin{définition}\cmn 互相看\end{définition}
\end{sous-entrée}\begin{sous-entrée}
\vedette{\hypertarget{}{\papi{ nɤrtɯrtoʁ}}}\markboth{nɤrtɯrtoʁ}{}\classe{vt}
\begin{définition}\fra regarder dans tous les sens\end{définition}
\begin{définition}\cmn 看来看去\end{définition}
\end{sous-entrée}\begin{sous-entrée}
\vedette{\hypertarget{}{\papi{ sɯrtoʁ}}}\markboth{sɯrtoʁ}{}\classe{vt}
\begin{définition}\ 
\begin{déclaration}\grammar{caus}\end{déclaration}\end{définition}
\begin{exemple}\jya smɤnba kɯ tɤ́-wɣ-sɯrtoʁ-a\cmn 他请了医生给我看病\end{exemple}
\end{sous-entrée}\begin{sous-entrée}
\vedette{\hypertarget{}{\papi{ ʑɣɤsɯrtoʁ}}}\markboth{ʑɣɤsɯrtoʁ}{}\classe{vi}
\begin{définition}\ 
\begin{déclaration}\grammar{refl}\end{déclaration}\end{définition}
\begin{exemple}\jya smɤnba ɯ-phe kɯ-ʑɣɤsɯrtoʁ lɤ-ari\cmn 他去看病了\end{exemple}
\begin{exemple}\jya ɕ-tɤ-ʑɣɤsɯrtoʁ\cmn 你去看病吧\end{exemple}
\begin{exemple}\jya smɤnba a-tɤ-tɯ-ʑɣɤsɯrtoʁ\cmn 你找医生看你的病吧\end{exemple}
\begin{exemple}\jya smɤnba ci kɯ-ʑɣɤsɯrtoʁ jɤ-ari-a\cmn 我去看医生了\end{exemple}
\end{sous-entrée}\end{entrée}

\begin{entrée}
\vedette{\hypertarget{Ⓔrtoʁldɤn}{\papi{ rtoʁldɤn}}}\markboth{rtoʁldɤn}{}\classe{n}
\begin{définition}\fra sage\end{définition}
\begin{définition}\cmn 得道者
\begin{déclaration} \étymologie{\papi{rtogs.ldan}}\end{déclaration}\end{définition}\end{entrée}

\begin{entrée}
\vedette{\hypertarget{Ⓔrtoʁldɤn mɯntoʁ}{\papi{ rtoʁldɤn mɯntoʁ}}}\markboth{rtoʁldɤn mɯntoʁ}{}\classe{n}
\begin{définition}\fra type de fleur\end{définition}
\begin{définition}\cmn 和尚花\end{définition}\end{entrée}

\begin{entrée}
\vedette{\hypertarget{Ⓔrtsa}{\papi{ rtsa}}}\markboth{rtsa}{}
\classe{vt}
\paradigme{\textit{dir :} \jya nɯ-}
\begin{définition}\fra enlever les organes sexuels des animaux femelles\end{définition}
\begin{définition}\cmn 取出雌性动物的胎盘或子宫\end{définition}
\begin{exemple}\jya paʁ na-rtsa\cmn 他阉割了母猪\end{exemple}\end{entrée}

\begin{entrée}
\vedette{\hypertarget{Ⓔrtsaka}{\papi{ rtsaka}}}\markboth{rtsaka}{}\classe{n}
\begin{définition}\fra herbe verte\end{définition}
\begin{définition}\cmn 青草\end{définition}\end{entrée}

\begin{entrée}
\vedette{\hypertarget{Ⓔrtsaʁjɯɣ}{\papi{ rtsaʁjɯɣ}}}\markboth{rtsaʁjɯɣ}{}
\classe{n}
\begin{définition}\fra bâton pour frapper les contrevenants à l'ordre dans le monastère\end{définition}
\begin{définition}\cmn 宗教活动时,维持纪律的和尚用来打人的棍子
\begin{déclaration} \étymologie{\papi{dbʲug}}\end{déclaration}\end{définition}\end{entrée}

\begin{entrée}
\vedette{\hypertarget{Ⓔrtsatɯɣ}{\papi{ rtsatɯɣ}}}\markboth{rtsatɯɣ}{}\classe{n}
\begin{définition}\fra herbe non identifiée qui rend malade le bétail qui l'absorbe\end{définition}
\begin{définition}\cmn 有毒的草,学名不明,牲畜吃了就生病
\begin{déclaration} \étymologie{\papi{rtswa.dug}}\end{déclaration}\end{définition}\end{entrée}

\begin{entrée}
\vedette{\hypertarget{Ⓔrtsawa}{\papi{ rtsawa}}}\markboth{rtsawa}{}
\classe{n}
\begin{définition}\fra importance\end{définition}
\begin{définition}\cmn 重要性
\begin{déclaration} \étymologie{\papi{rtsa.ba}}\end{déclaration}\end{définition}
\begin{exemple}\jya ɯ-rtsawa ɲɯ-wxti\cmn 很重要\end{exemple}\begin{sous-entrée}
\vedette{\hypertarget{}{\papi{ ɯ-rtsawa,ndo}}}\markboth{ɯ-rtsawa,ndo}{}
\begin{définition}\fra contrôler\end{définition}
\begin{définition}\cmn 掌握\end{définition}
\begin{exemple}\jya jiʑo rɟɤlkhɤβ ɣɯ ji-rtsawa ɯ-kɯ-ndo nɯ @gongchandang ŋu\cmn 共产党是掌握我们中国的\end{exemple}
\begin{relation-sémantique}\ComponentA{\classe{np}
 \papi{ɯ-rtsawa}
}\end{relation-sémantique}
\begin{relation-sémantique}\ComponentB{\classe{vt}
\hyperlink{Ⓔndo}{\textit{ \papi{ndo}}}
}\end{relation-sémantique}
\end{sous-entrée}\end{entrée}

\begin{entrée}
\vedette{\hypertarget{Ⓔrtsɤmkɯɣ}{\papi{ rtsɤmkɯɣ}}}\markboth{rtsɤmkɯɣ}{}
\classe{n}
\begin{définition}\fra sac à rtsampa\end{définition}
\begin{définition}\cmn 糌粑口袋
\begin{déclaration} \étymologie{\papi{rtsam.kʰug}}\end{déclaration}\end{définition}\end{entrée}

\begin{entrée}
\vedette{\hypertarget{Ⓔrtsɤmtɕhɯ}{\papi{ rtsɤmtɕhɯ}}}\markboth{rtsɤmtɕhɯ}{}
\classe{n}
\begin{définition}\fra eau que l'on met dans le bol pendant que l'on mange de la tsampa\end{définition}
\begin{définition}\cmn 挼糌粑时倒进碗里的水\end{définition}\end{entrée}

\begin{entrée}
\vedette{\hypertarget{Ⓔrtsɤxtɕɤr}{\papi{ rtsɤxtɕɤr}}}\markboth{rtsɤxtɕɤr}{}\classe{n}
\begin{définition}\fra bande colorée\end{définition}
\begin{définition}\cmn 花带子(小孩子带的)\end{définition}
\begin{exemple}\jya rtsɤxtɕɤr nɯ-nɯrtaβ-a\cmn 我带了花带子\end{exemple}\end{entrée}

\begin{entrée}
\vedette{\hypertarget{Ⓔrtshartsha}{\papi{ rtshartsha}}}\markboth{rtshartsha}{}\classe{idph.2}
\begin{définition}\fra un peu rugueux\end{définition}
\begin{définition}\cmn 形容略粗糙的样子\end{définition}
\begin{exemple}\jya (qaɕparaz) khro mɤ-mpɕu, rʁom tsa rtshartsha\cmn (那种草)不光滑,有点粗糙\end{exemple}\end{entrée}

\begin{entrée}
\vedette{\hypertarget{Ⓔrtshɤrtshɤt}{\papi{ rtshɤrtshɤt}}}\markboth{rtshɤrtshɤt}{}\classe{idph.2}
\begin{définition}\fra fin et résistant (feuille)\end{définition}
\begin{définition}\cmn 形容叶子、纸等薄而结实,不易折破的样子\end{définition}\end{entrée}

\begin{entrée}
\vedette{\hypertarget{Ⓔrtshom}{\papi{ rtshom}}}\markboth{rtshom}{}
\classe{vi}
\begin{définition}\fra avoir une fente (seau en bois)\end{définition}
\begin{définition}\cmn 木板间的隙缝中出现裂口(木桶变干之后)\end{définition}
\begin{exemple}\jya zɯm ɲɤ-rtshom tɕe ɲɤ-ri\cmn 木桶有了裂口就在漏水\end{exemple}
\begin{exemple}\jya zɯm tɯ-ɕoʁ tɯ-ɕoʁ tɤ-kɤ-sprɤt ɲɯ-ŋu tɕe, a-tɤ-zbaʁ tɕe ɲɯ-rtshom ɲɯ-ŋu, tɯ-ci tɤ-me tɕe ɲɯ-rtshom ɲɯ-ŋu\cmn 木桶是由一条一条的木板条组成的,只要变干它就会出现裂口\end{exemple}\end{entrée}

\begin{entrée}
\vedette{\hypertarget{Ⓔrtshɯβrtshɯβ}{\papi{ rtshɯβrtshɯβ}}}\markboth{rtshɯβrtshɯβ}{}\classe{idph.2}
\begin{définition}\fra grossier et piquant (surface)\end{définition}
\begin{définition}\cmn 形容平面表面粗糙的样子\end{définition}\end{entrée}

\begin{entrée}
\vedette{\hypertarget{Ⓔrtshɯrtshi}{\papi{ rtshɯrtshi}}}\markboth{rtshɯrtshi}{}\classe{idph.2}
\begin{définition}\fra râpeux\end{définition}
\begin{définition}\cmn 形容物体摸起来粗糙刮手的样子\end{définition}
\end{entrée}

\begin{entrée}
\vedette{\hypertarget{Ⓔrtsi}{\papi{ rtsi}}}\markboth{rtsi}{}
\classe{vt}
\paradigme{\textit{dir :} \jya tɤ-}\acception{1}
\begin{définition}\fra calculer\end{définition}
\begin{définition}\cmn 算\end{définition}
\begin{exemple}\jya tɤ-rtsi-t-a\cmn 我算了\end{exemple}
\begin{exemple}\jya ji-nɯŋa thɤstɯɣ ɣɤʑu kɯ tɤ-rtsi-t-a\cmn 我数了一下我们家的牛有多少头\end{exemple}\acception{2}
\begin{définition}\fra considérer comme\end{définition}
\begin{définition}\cmn 当成
\begin{déclaration} \étymologie{\papi{rtsi}}\end{déclaration}\end{définition}
\begin{exemple}\jya ki kɤ-rtsi kɯ-tu me nɤ\cmn 这不算什么\end{exemple}
\begin{relation-sémantique}\synonyme{
\hyperlink{ⒺχsɤrⒽ1}{\textit{ \papi{χsɤr1}}}
}\end{relation-sémantique}
\begin{relation-sémantique}\confer{
\hyperlink{Ⓔkɤrtsi}{\textit{ \papi{kɤrtsi}}}
}\end{relation-sémantique}\begin{sous-entrée}
\vedette{\hypertarget{}{\papi{ sɤrtsi}}}\markboth{sɤrtsi}{}\classe{vt}
\begin{définition}\fra considérer comme\end{définition}
\begin{définition}\cmn 视为\end{définition}
\begin{exemple}\jya mkhɤrmaŋ ra ɯ-rɟit ʑo tu-nɯ-sɤrtsi pjɤ-ŋu (=tu-nɯ-ste)\cmn 他把群众当成自己的子女一样\end{exemple}
\begin{exemple}\jya aʑo a-tɕɯ tu-ta-nɯ-sɤrtsi ŋu\cmn 我把你当儿子一样看待\end{exemple}
\end{sous-entrée}\begin{sous-entrée}
\vedette{\hypertarget{}{\papi{ ʑɣɤrtsi}}}\markboth{ʑɣɤrtsi}{}\classe{vs}
\begin{définition}\ 
\begin{déclaration}\grammar{refl}\end{déclaration}\end{définition}\acception{1}
\begin{définition}\fra se considérer comme\end{définition}
\begin{définition}\cmn 自称;把自己视为\end{définition}
\begin{relation-sémantique}\confer{
\hyperlink{ⒺʑɣɤpaⒽ2}{\textit{ \papi{ʑɣɤpa2}}}
}\end{relation-sémantique}\acception{2}
\begin{définition}\fra se compter parmi\end{définition}
\begin{définition}\cmn 把自己算在里面\end{définition}
\begin{exemple}\jya ɯʑo to-nɯ-ʑɣɤrtsi\cmn 他没有吧自己算在里面\end{exemple}
\end{sous-entrée}\end{entrée}

\begin{entrée}
\vedette{\hypertarget{Ⓔrtsiaʁ}{\papi{ rtsiaʁ}}}\markboth{rtsiaʁ}{}
\classe{vs}
\begin{définition}\fra escarpé et sinueux (chemin)\end{définition}
\begin{définition}\cmn 陡峭;难走(路)\end{définition}
\begin{exemple}\jya tʂu ɲɯ-rtsiaʁ\cmn 路很陡峭\end{exemple}\end{entrée}

\begin{entrée}
\vedette{\hypertarget{Ⓔrtsimu}{\papi{ rtsimu}}}\markboth{rtsimu}{}\classe{n}
\begin{définition}\fra façon dont poussent les branches (arbre)\end{définition}
\begin{définition}\cmn (树木的)长势\end{définition}
\begin{exemple}\jya ki si ki ɯ-rtsimu ɲɯ-βdi\cmn 这棵树长势很美\end{exemple}\end{entrée}

\begin{entrée}
\vedette{\hypertarget{Ⓔrtsot}{\papi{ rtsot}}}\markboth{rtsot}{}
\classe{n}
\begin{définition}\fra vengeance\end{définition}
\begin{définition}\cmn 报仇
\begin{déclaration} \étymologie{\papi{rtsod}}\end{déclaration}\end{définition}
\begin{exemple}\jya maka rtsot tu-βze-a ɲɯ-ntshi\cmn 我要报仇\end{exemple}\end{entrée}

\begin{entrée}
\vedette{\hypertarget{Ⓔrtsɯβ}{\papi{ rtsɯβ}}}\markboth{rtsɯβ}{}\classe{vs}
\begin{définition}\fra qui contient beaucoup de gros grains\end{définition}
\begin{définition}\cmn 粗粮多\end{définition}
\begin{exemple}\jya tɤjlu ɲɯ-rtsɯβ\cmn 面粉加的粗粮多\end{exemple}\begin{sous-entrée}
\vedette{\hypertarget{}{\papi{ ɯ-lu,rtsɯβ}}}\markboth{ɯ-lu,rtsɯβ}{}
\begin{définition}\fra avoir un cycle astrologique complet\end{définition}
\begin{définition}\cmn 是……的本命年\end{définition}
\begin{exemple}\jya ɣɯjpa ɯ-lu rtsɯβ\cmn 今年是他的本命年\end{exemple}
\begin{exemple}\jya nɤ-lu sɤ-rtsɯβ\cmn 你的本命年\end{exemple}
\begin{relation-sémantique}\ComponentA{\classe{np}
 \papi{ɯ-lu}
}\end{relation-sémantique}
\begin{relation-sémantique}\ComponentB{\classe{vi}
\hyperlink{Ⓔrtsɯβ}{\textit{ \papi{rtsɯβ}}}
}\end{relation-sémantique}
\end{sous-entrée}\end{entrée}

\begin{entrée}
\vedette{\hypertarget{Ⓔrtsɯɕaŋlaŋmtɕɤt}{\papi{ rtsɯɕaŋlaŋmtɕɤt}}}\markboth{rtsɯɕaŋlaŋmtɕɤt}{}\classe{n}
\begin{définition}\fra toutes les plantes\end{définition}
\begin{définition}\cmn 所有的草木
\begin{déclaration} \étymologie{\papi{rtsi.ɕiŋ.tham.tɕɤt}}\end{déclaration}\end{définition}
\end{entrée}

\begin{entrée}
\vedette{\hypertarget{Ⓔrtsɯɣ}{\papi{ rtsɯɣ}}}\markboth{rtsɯɣ}{}
\classe{vt}
\paradigme{\textit{dir :} \jya \_}
\begin{définition}\fra empiler\end{définition}
\begin{définition}\cmn 堆起来
\begin{déclaration} \étymologie{\papi{rtsigs}}\end{déclaration}\end{définition}
\begin{exemple}\jya si pɯ-rtsɯɣ-a\cmn 我把木头堆起来了\end{exemple}
\begin{exemple}\jya tɤɕi pɯ-rtsɯɣ-a\cmn 我把青稞堆起来了\cmn 
\stylefv{kɤ-rmbɯ}和\stylefv{kɤ-rtsɯɣ}不一样,后者用于木料,而前者用于肥料
\end{exemple}\end{entrée}

\begin{entrée}
\vedette{\hypertarget{Ⓔrtsɯpɣaʁ}{\papi{ rtsɯpɣaʁ}}}\markboth{rtsɯpɣaʁ}{}\classe{n}
\begin{définition}\fra labourage après la récolte\end{définition}
\begin{définition}\cmn 庄稼收割了以后重新翻地\end{définition}
\begin{exemple}\jya rtsɯpɣaʁ lɤ-lɤt-i\cmn 我们翻了地\end{exemple}
\begin{relation-sémantique}\confer{
\hyperlink{Ⓔrɯrtsɯpɣaʁ}{\textit{ \papi{rɯrtsɯpɣaʁ}}}
}\end{relation-sémantique}
\begin{relation-sémantique}\confer{
\hyperlink{Ⓔnɯrtsɯpɣaʁ}{\textit{ \papi{nɯrtsɯpɣaʁ}}}
}\end{relation-sémantique}\end{entrée}

\begin{entrée}
\vedette{\hypertarget{Ⓔrtsɯtpa}{\papi{ rtsɯtpa}}}\markboth{rtsɯtpa}{}
\classe{n}
\begin{définition}\fra poils épais\end{définition}
\begin{définition}\cmn 粗毛\end{définition}\end{entrée}

\begin{entrée}
\vedette{\hypertarget{Ⓔrtsɯtʂɯɣ}{\papi{ rtsɯtʂɯɣ}}}\markboth{rtsɯtʂɯɣ}{}
\classe{n}
\begin{définition}\fra compte\end{définition}
\begin{définition}\cmn 帐
\begin{déclaration} \étymologie{\papi{rtsi.sgrig}}\end{déclaration}\end{définition}
\begin{exemple}\jya nɤ-rtsɯtʂɯɣ te-a ra, ma-ta-ta\cmn 我要跟你算帐,我不会放过你的\end{exemple}\end{entrée}

\begin{entrée}
\vedette{\hypertarget{Ⓔrtsɯz}{\papi{ rtsɯz}}}\markboth{rtsɯz}{}
\classe{vt}
\paradigme{\textit{dir :} \jya nɯ-}
\paradigme{\textit{dir :} \jya tɤ-}
\begin{définition}\fra calculer\end{définition}
\begin{définition}\cmn 算\end{définition}
\begin{exemple}\jya thɤstɯɣ ɲɯ-ɤmɯβɟɤt-i nɯ-rtsɯz-a\cmn 我算了一下我们每个人可以分多少\end{exemple}
\begin{exemple}\jya fsapaʁ tɤ-rtsɯz-a\cmn 我数了一下牲畜\end{exemple}
\begin{relation-sémantique}\synonyme{
\hyperlink{ⒺχsɤrⒽ1}{\textit{ \papi{χsɤr1}}}
}\end{relation-sémantique}
\begin{relation-sémantique}\confer{
\hyperlink{Ⓔrtsi}{\textit{ \papi{rtsi}}}
}\end{relation-sémantique}\end{entrée}

\begin{entrée}
\vedette{\hypertarget{ⒺrɯⒽ2}{\papi{ rɯ}}}\markboth{rɯ}{}\homonyme{2}\classe{n}
\begin{définition}\fra lieu d'habitation temporaire dans la montagne\end{définition}
\begin{définition}\cmn 山上、牧草上暂住的地方(帐篷里)
\begin{déclaration} \étymologie{\papi{ri}}\end{déclaration}\end{définition}
\begin{exemple}\jya rɯ ɲɯ-scat-a\cmn 我在搬帐篷\end{exemple}\end{entrée}

\begin{entrée}
\vedette{\hypertarget{ⒺrɯⒽ1}{\papi{ rɯ}}}\markboth{rɯ}{}\homonyme{1}\classe{vs}
\begin{définition}\fra épais (liquide qui contient beaucoup de matière grasse)\end{définition}
\begin{définition}\cmn 浓(液体里的牛奶或者油)
\end{définition}
\begin{exemple}\jya tɤ-lu kɯ-rɯ to-lɤt\cmn 他倒了很多牛奶(茶里的牛奶很浓)\end{exemple}
\begin{exemple}\jya tɤ-lu mɯ́j-rɯ tɕe qhluqhlu ʑo ɲɯ-pa\cmn 牛奶不浓,(茶里)只有一点点白色\end{exemple}
\end{entrée}

\begin{entrée}
\vedette{\hypertarget{Ⓔrɯβ}{\papi{ rɯβ}}}\markboth{rɯβ}{}
\classe{vs}
\paradigme{\textit{dir :} \jya kɤ-}
\begin{définition}\fra impraticable\end{définition}
\begin{définition}\cmn 不好走(杂草、灌木丛生)\end{définition}
\begin{exemple}\jya sɯŋgɯ ɲɯ-rɯβ\cmn 森林不好走\end{exemple}\end{entrée}

\begin{entrée}
\vedette{\hypertarget{Ⓔrɯβluβra}{\papi{ rɯβluβra}}}\markboth{rɯβluβra}{}\classe{vs}
\begin{définition}\ 
\begin{déclaration}\grammar{denom}\end{déclaration}\end{définition}
\begin{définition}\fra qui donne de bons conseils\end{définition}
\begin{définition}\cmn 善于出主意\end{définition}
\begin{exemple}\jya ɯʑo kɯ-rɯβluβra ci ɲɯ-ŋu\cmn 这个人善于出主意\end{exemple}
\begin{relation-sémantique}\confer{
\hyperlink{Ⓔβluβra}{\textit{ \papi{βluβra}}}
}\end{relation-sémantique}\end{entrée}

\begin{entrée}
\vedette{\hypertarget{Ⓔrɯβnɤrɯβ}{\papi{ rɯβnɤrɯβ}}}\markboth{rɯβnɤrɯβ}{}
\classe{idph.3}
\begin{définition}\fra qui coule sans s'arrêter goute à goute\end{définition}
\begin{définition}\cmn 不停地漏出来;不停地滴出来\end{définition}
\begin{exemple}\jya tɤ-se rɯβrɯβ nɤ rɯβrɯβ ɲɯ-nɯɬoʁ\cmn 血一滴一滴地流出来\end{exemple}
\begin{exemple}\jya tɯ-ci rɯβrɯβ nɤ rɯβrɯβ ɲɯ-nɯftsaʁ\cmn 一滴一滴地漏水\end{exemple}\begin{sous-entrée}
\vedette{\hypertarget{}{\papi{ rɯwɯrawi}}}\markboth{rɯwɯrawi}{}\classe{idph.8}
\begin{définition}\fra confus\end{définition}
\begin{définition}\cmn 心情烦乱\end{définition}
\begin{exemple}\jya ɯ-kɤ-nɯzdɯɣ ɲɯ-dɤn tɕe, ɯ-sɯm rɯwɯrawi ɲɯ-xtsu\cmn 他担心的事情很多,心烦意乱\end{exemple}
\begin{exemple}\jya ɯ-sɯm rɯwɯrawi ʑo ɲɯ-βze\cmn 心情很烦乱\end{exemple}
\begin{relation-sémantique}\confer{
\hyperlink{Ⓔɣɤrɯβrɯβ}{\textit{ \papi{ɣɤrɯβrɯβ}}}
}\end{relation-sémantique}
\end{sous-entrée}\begin{sous-entrée}
\vedette{\hypertarget{}{\papi{ rɯwɯwi}}}\markboth{rɯwɯwi}{}\classe{idph.7}
\begin{exemple}\jya tɯ-ci rɯwɯwi ʑo pɯ-ɣe\cmn 水慢慢地往下流\end{exemple}
\end{sous-entrée}\end{entrée}

\begin{entrée}
\vedette{\hypertarget{Ⓔrɯcɤβŋgɤβ}{\papi{ rɯcɤβŋgɤβ}}}\markboth{rɯcɤβŋgɤβ}{}
\classe{vs}
\paradigme{\textit{dir :} \jya tɤ-}
\begin{définition}\fra être orgueilleux\end{définition}
\begin{définition}\cmn 骄傲
\begin{déclaration} \étymologie{\papi{\stylefn{骄傲}}}\end{déclaration}\end{définition}
\begin{exemple}\jya ma-tɯ-rɯcɤβŋgɤβ\cmn 你不要骄傲\end{exemple}
\begin{exemple}\jya tɤ-rɯcɤβŋgaβ-a\cmn 我在他面前骄傲了一下\end{exemple}
\begin{exemple}\jya ɯʑo kɤ-rɯcɤŋgɤβ mɤ-spe\cmn 他不会骄傲自大\end{exemple}\end{entrée}

\begin{entrée}
\vedette{\hypertarget{Ⓔrɯcɯnmu}{\papi{ rɯcɯnmu}}}\markboth{rɯcɯnmu}{}
\classe{vi}
\paradigme{\textit{dir :} \jya tɤ-}
\begin{définition}\fra répandre des rumeurs\end{définition}
\begin{définition}\cmn 挑拨离间\end{définition}
\begin{exemple}\jya a-mɤ-tɯ-rɯcɯnmu\cmn 你不要挑拨离间\end{exemple}
\begin{exemple}\jya jiɕqha kɯ-rɯcɯnmu ci ɲɯ-ŋu\cmn 这个人爱挑拨离间\end{exemple}
\begin{relation-sémantique}\confer{
\hyperlink{Ⓔcɯnmu}{\textit{ \papi{cɯnmu}}}
}\end{relation-sémantique}\end{entrée}

\begin{entrée}
\vedette{\hypertarget{Ⓔrɯɕaŋchi}{\papi{ rɯɕaŋchi}}}\markboth{rɯɕaŋchi}{}\classe{vi}
\paradigme{\textit{dir :} \jya tɤ-}
\begin{définition}\fra aimer se maquiller et porter des habits luxueux\end{définition}
\begin{définition}\cmn 喜欢打扮,穿豪华的衣服,爱美\end{définition}
\begin{exemple}\jya iɕqha tɕheme nɯ ɲɯ-rɯɕaŋchi\cmn 这个女子喜欢打扮\end{exemple}
\begin{exemple}\jya ɕɯŋgɯ mɯ-pɯ-rɯɕaŋchi, tham to-rɯɕaŋchi\cmn 他以前不打扮,现在就最爱打扮了\end{exemple}\end{entrée}

\begin{entrée}
\vedette{\hypertarget{Ⓔrɯɕɤtsha}{\papi{ rɯɕɤtsha}}}\markboth{rɯɕɤtsha}{}\classe{vi}
\paradigme{\textit{dir :} \jya tɤ-}
\begin{définition}\fra faire attention\end{définition}
\begin{définition}\cmn 小心\end{définition}
\begin{exemple}\jya tɤ-rɯɕɤtsha ma ɲɯ-sɤɣʑɯr\cmn 你小心,很危险\end{exemple}
\begin{relation-sémantique}\synonyme{
\hyperlink{Ⓔrɯndzaŋspa}{\textit{ \papi{rɯndzaŋspa}}}
}\end{relation-sémantique}\end{entrée}

\begin{entrée}
\vedette{\hypertarget{Ⓔrɯɕmi}{\papi{ rɯɕmi}}}\markboth{rɯɕmi}{}
\classe{vi}
\paradigme{\textit{dir :} \jya tɤ-}
\begin{définition}\fra parler\end{définition}
\begin{définition}\cmn 讲\end{définition}
\begin{exemple}\jya jiɕqha nɯ ɲɯ-rɯɕmi\cmn 那个人在说话\end{exemple}
\begin{exemple}\jya qajdo to-rɯɕmi\cmn 乌鸦说话了\end{exemple}
\begin{exemple}\jya li ci tɤti ma jiɕqha tu-rɯɕmi tɕe mɯ-kɤ-tso-a\cmn 你再讲一次,刚才他在讲话,我没有听清楚\end{exemple}
\begin{exemple}\jya tɤ-rɯɕmi jɤɣ\cmn 你可以讲\end{exemple}\begin{sous-entrée}
\vedette{\hypertarget{}{\papi{ anɯrɯɕmɯɕmi}}}\markboth{anɯrɯɕmɯɕmi}{}\classe{vi}
\begin{définition}\ 
\begin{déclaration}\grammar{recip}\end{déclaration}\end{définition}
\begin{définition}\fra se parler l'un à l'autre\end{définition}
\begin{définition}\cmn 互相讲话\end{définition}
\end{sous-entrée}\begin{sous-entrée}
\vedette{\hypertarget{}{\papi{ zrɯɕmi}}}\markboth{zrɯɕmi}{}\classe{vt}
\paradigme{\textit{dir :} \jya tɤ-}
\begin{définition}\ 
\begin{déclaration}\grammar{caus}\end{déclaration}\end{définition}
\begin{définition}\fra faire parler\end{définition}
\begin{définition}\cmn 令……说话\end{définition}
\end{sous-entrée}\end{entrée}

\begin{entrée}
\vedette{\hypertarget{Ⓔrɯɕmɯlaʁ}{\papi{ rɯɕmɯlaʁ}}}\markboth{rɯɕmɯlaʁ}{}\classe{vi}
\begin{définition}\fra parler\end{définition}
\begin{définition}\cmn 讲话\end{définition}
\begin{exemple}\jya aʑo kɤ-rɯɕmɯlaʁ mɤ-cha-a wo\cmn 我不善于给别人打招呼\end{exemple}
\begin{exemple}\jya kɯm ɯ-pɕi nɯtɕu kɯ-rɯɕmɯlaʁ kɯ-fse ci ɣɤʑu\cmn 门后面好像有人在讲话\end{exemple}\end{entrée}

\begin{entrée}
\vedette{\hypertarget{Ⓔrɯɕmɯχtɤm}{\papi{ rɯɕmɯχtɤm}}}\markboth{rɯɕmɯχtɤm}{}\classe{vi}
\paradigme{\textit{dir :} \jya pɯ-}
\begin{définition}\fra dire des balivernes\end{définition}
\begin{définition}\cmn 啰嗦,说废话\end{définition}
\begin{exemple}\jya ma-tɯ-rɯɕmɯχtɤm kɯ nɤ-ma nɯ ʑ-nɯ-nɤme\cmn 不要说废话,去干你的事\end{exemple}
\begin{relation-sémantique}\confer{
\hyperlink{Ⓔrɯɕmi}{\textit{ \papi{rɯɕmi}}}
}\end{relation-sémantique}\end{entrée}

\begin{entrée}
\vedette{\hypertarget{Ⓔrɯɕoŋβzu}{\papi{ rɯɕoŋβzu}}}\markboth{rɯɕoŋβzu}{}\classe{vi}
\paradigme{\textit{dir :} \jya nɯ-}
\begin{définition}\fra faire des travaux du bois\end{définition}
\begin{définition}\cmn 做木工\end{définition}
\begin{exemple}\jya ʑara ɣɯ ku-rɯɕoŋβzu-a\cmn 我在他们家里做木工\end{exemple}
\begin{relation-sémantique}\confer{
\hyperlink{Ⓔɕoŋβzu}{\textit{ \papi{ɕoŋβzu}}}
}\end{relation-sémantique}\end{entrée}

\begin{entrée}
\vedette{\hypertarget{Ⓔrɯdaʁ}{\papi{ rɯdaʁ}}}\markboth{rɯdaʁ}{}
\classe{n}
\begin{définition}\fra bête sauvage\end{définition}
\begin{définition}\cmn 野兽
\begin{déclaration} \étymologie{\papi{ri.dʷags}}\end{déclaration}\end{définition}\end{entrée}

\begin{entrée}
\vedette{\hypertarget{Ⓔrɯfsɤri}{\papi{ rɯfsɤri}}}\markboth{rɯfsɤri}{}\classe{vt}
\paradigme{\textit{dir :} \jya lɤ-}
\begin{définition}\fra filer pour faire une ficelle\end{définition}
\begin{définition}\cmn 搓成线\end{définition}
\begin{exemple}\jya tasa cho mphɯli lɤ-rɯfsɤri-t-a\cmn 我把大麻和亚麻搓成线\end{exemple}
\begin{relation-sémantique}\confer{
\hyperlink{Ⓔtɤ-fsɤri}{\textit{ \papi{tɤ-fsɤri}}}
}\end{relation-sémantique}\end{entrée}

\begin{entrée}
\vedette{\hypertarget{Ⓔrɯftɕaka}{\papi{ rɯftɕaka}}}\markboth{rɯftɕaka}{}
\classe{vi}
\paradigme{\textit{dir :} \jya tɤ-}
\begin{définition}\ 
\begin{déclaration}\grammar{denom}\end{déclaration}\end{définition}
\begin{définition}\fra préparer\end{définition}
\begin{définition}\cmn 准备\end{définition}
\begin{exemple}\jya kɯ-ɕe ɲɯ-rɯftɕaka\cmn 他准备去\end{exemple}
\begin{relation-sémantique}\confer{
\hyperlink{Ⓔftɕaka}{\textit{ \papi{ftɕaka}}}
}\end{relation-sémantique}
\begin{relation-sémantique}\confer{
\hyperlink{Ⓔnɯftɕaka}{\textit{ \papi{nɯftɕaka}}}
}\end{relation-sémantique}
\begin{relation-sémantique}\confer{
\hyperlink{Ⓔsɤftɕaka}{\textit{ \papi{sɤftɕaka}}}
}\end{relation-sémantique}\end{entrée}

\begin{entrée}
\vedette{\hypertarget{Ⓔrɯftɕɤfkɤt}{\papi{ rɯftɕɤfkɤt}}}\markboth{rɯftɕɤfkɤt}{}
\classe{vi}
\paradigme{\textit{dir :} \jya tɤ-}
\begin{définition}\ 
\begin{déclaration}\grammar{denom}\end{déclaration}\end{définition}
\begin{définition}\fra donner son avis, donner des suggestions\end{définition}
\begin{définition}\cmn 给别人出主意(自作多情)\end{définition}
\begin{exemple}\jya jiɕqha nɯ ɲɯ-rɯftɕɤfkɤt\cmn 那个人在给别人出主意\end{exemple}
\begin{relation-sémantique}\confer{
\hyperlink{Ⓔftɕɤfkɤt}{\textit{ \papi{ftɕɤfkɤt}}}
}\end{relation-sémantique}\end{entrée}

\begin{entrée}
\vedette{\hypertarget{Ⓔrɯɣ}{\papi{ rɯɣ}}}\markboth{rɯɣ}{}\classe{vs}
\begin{définition}\fra précieux\end{définition}
\begin{définition}\cmn 贵重\end{définition}\end{entrée}

\begin{entrée}
\vedette{\hypertarget{Ⓔrɯɣnɤn}{\papi{ rɯɣnɤn}}}\markboth{rɯɣnɤn}{}
\classe{vi}
\paradigme{\textit{dir :} \jya nɯ-}
\begin{définition}\fra s'opposer, chercher à causer des ennuis\end{définition}
\begin{définition}\cmn 作对;找茬\end{définition}
\begin{exemple}\jya a-ɕki ɲɯ-rɯɣnɤn\cmn 他跟我作对\end{exemple}
\begin{exemple}\jya ma-nɯ-tɯ-rɯɣnɤn\cmn 你不要(跟他)作对\end{exemple}\begin{sous-entrée}
\vedette{\hypertarget{}{\papi{ zrɯɣnɤn}}}\markboth{zrɯɣnɤn}{}\classe{vt}
\paradigme{\textit{dir :} \jya nɯ-}
\begin{définition}\fra plaisanter\end{définition}
\begin{définition}\cmn 开玩笑,逗着玩\end{définition}
\begin{exemple}\jya ma-nɯ-kɯ-zrɯɣnan-a\cmn 你不要跟我开玩笑\end{exemple}
\begin{exemple}\jya nɯ́-wɣ-zrɯɣnan-a\cmn 他跟我开了玩笑\end{exemple}
\end{sous-entrée}\end{entrée}

\begin{entrée}
\vedette{\hypertarget{Ⓔrɯɣne}{\papi{ rɯɣne}}}\markboth{rɯɣne}{}
\classe{vt}
\paradigme{\textit{dir :} \jya nɯ-}
\begin{définition}\fra critiquer\end{définition}
\begin{définition}\cmn 责怪\end{définition}
\begin{exemple}\jya nɯ-rɯɣne-t-a\cmn 我骂了他\end{exemple}
\begin{exemple}\jya nɯ́-wɣ-rɯɣne-a\cmn 他骂了我\end{exemple}
\begin{exemple}\jya pɯ-az-rɯɣne\cmn (以前)他在骂他\end{exemple}
\begin{exemple}\jya aj pɯ-ɣɤtɕa-a, ma-nɯ-kɯ-rɯɣne-a\cmn 我错了,你不要骂我\end{exemple}\end{entrée}

\begin{entrée}
\vedette{\hypertarget{Ⓔrɯjɤɣɤt}{\papi{ rɯjɤɣɤt}}}\markboth{rɯjɤɣɤt}{}\classe{vi}
\paradigme{\textit{dir :} \jya pɯ-}
\begin{définition}\ 
\begin{déclaration}\grammar{denom}\end{déclaration}\end{définition}
\begin{définition}\fra aller aux toilettes\end{définition}
\begin{définition}\cmn 上厕所\end{définition}
\begin{relation-sémantique}\confer{
\hyperlink{Ⓔjɤɣɤt}{\textit{ \papi{jɤɣɤt}}}
}\end{relation-sémantique}\end{entrée}

\begin{entrée}
\vedette{\hypertarget{Ⓔrɯɟuli}{\papi{ rɯɟuli}}}\markboth{rɯɟuli}{}\classe{vi}
\begin{définition}\ 
\begin{déclaration}\grammar{denom}\end{déclaration}\end{définition}
\begin{définition}\fra jouer de la flûte\end{définition}
\begin{définition}\cmn 吹笛子\end{définition}
\begin{relation-sémantique}\confer{
\hyperlink{Ⓔɟuli}{\textit{ \papi{ɟuli}}}
}\end{relation-sémantique}\end{entrée}

\begin{entrée}
\vedette{\hypertarget{Ⓔrɯkɤtɯm}{\papi{ rɯkɤtɯm}}}\markboth{rɯkɤtɯm}{}
\classe{vi}
\paradigme{\textit{dir :} \jya tɤ-}
\begin{définition}\ 
\begin{déclaration}\grammar{denom}\end{déclaration}\end{définition}
\begin{définition}\fra enrouler de fil l'appareil pour tisser, enrouler un fil en boule\end{définition}
\begin{définition}\cmn 左右缠绕,把线缠成球形\end{définition}
\begin{exemple}\jya tɤ-ri tɤ-rɯkɤtɯm\cmn 你缠线吧\end{exemple}\end{entrée}

\begin{entrée}
\vedette{\hypertarget{Ⓔrɯkhɤcɤl}{\papi{ rɯkhɤcɤl}}}\markboth{rɯkhɤcɤl}{}
\classe{vi}
\paradigme{\textit{dir :} \jya pɯ-}
\begin{définition}\ 
\begin{déclaration}\grammar{denom}\end{déclaration}\end{définition}
\begin{définition}\fra bavarder\end{définition}
\begin{définition}\cmn 聊天\end{définition}
\begin{exemple}\jya jiɕqha nɯ ɲɯ-rɯkhɤcɤl\cmn 那个人在聊天\end{exemple}
\begin{exemple}\jya pɯ-rɯkhɤcɤl-i\cmn 我们聊天了\end{exemple}\end{entrée}

\begin{entrée}
\vedette{\hypertarget{Ⓔrɯkhɤrlɤn}{\papi{ rɯkhɤrlɤn}}}\markboth{rɯkhɤrlɤn}{}\classe{vi}
\paradigme{\textit{dir :} \jya tɤ-}
\begin{définition}\ 
\begin{déclaration}\grammar{denom}\end{déclaration}\end{définition}
\begin{définition}\fra construire une maison\end{définition}
\begin{définition}\cmn 修房子\end{définition}
\begin{exemple}\jya roŋwa thɯ-mɤɕi rɯkhɤrlɤn\cmn 农民富有了就修房子\end{exemple}
\begin{relation-sémantique}\confer{
\hyperlink{Ⓔkhɤrlɤn}{\textit{ \papi{khɤrlɤn}}}
}\end{relation-sémantique}\end{entrée}

\begin{entrée}
\vedette{\hypertarget{Ⓔrɯkhon}{\papi{ rɯkhon}}}\markboth{rɯkhon}{}
\classe{vi}
\paradigme{\textit{dir :} \jya tɤ-}
\begin{définition}\ 
\begin{déclaration}\grammar{denom}\end{déclaration}\end{définition}
\begin{définition}\fra préparer au cas où\end{définition}
\begin{définition}\cmn 预备\end{définition}
\begin{exemple}\jya ɯ-qhu kɤ-nɤma nɯnɯ tham tɕe tu-kɯ-rɯkhon ra\cmn 以后的工作要现在预备好(以防万一)\end{exemple}
\begin{relation-sémantique}\confer{
\hyperlink{Ⓔɯ-khon}{\textit{ \papi{ɯ-khon}}}
}\end{relation-sémantique}\end{entrée}

\begin{entrée}
\vedette{\hypertarget{Ⓔrɯkhramba}{\papi{ rɯkhramba}}}\markboth{rɯkhramba}{}
\classe{vi}
\paradigme{\textit{dir :} \jya tɤ-}
\begin{définition}\fra mentir\end{définition}
\begin{définition}\cmn 撒谎
\begin{déclaration}\grammar{denom}\end{déclaration}\end{définition}
\begin{exemple}\jya jiɕqha nɯ rɯkhramba\cmn 那个人在撒谎\end{exemple}
\begin{exemple}\jya mɤ-kɯ-rɯkhramba ci ŋu\cmn 他是一个不撒谎的人\end{exemple}
\begin{exemple}\jya nɯkhramba\end{exemple}\end{entrée}

\begin{entrée}
\vedette{\hypertarget{Ⓔrɯkɯɕnom}{\papi{ rɯkɯɕnom}}}\markboth{rɯkɯɕnom}{}\classe{vi}
\paradigme{\textit{dir :} \jya tɤ-}
\begin{définition}\ 
\begin{déclaration}\grammar{denom}\end{déclaration}\end{définition}
\begin{définition}\fra monter en épi\end{définition}
\begin{définition}\cmn 抽穗\end{définition}
\begin{relation-sémantique}\confer{
\hyperlink{Ⓔkɯɕnom}{\textit{ \papi{kɯɕnom}}}
}\end{relation-sémantique}\end{entrée}

\begin{entrée}
\vedette{\hypertarget{Ⓔrɯkɯmaʁ}{\papi{ rɯkɯmaʁ}}}\markboth{rɯkɯmaʁ}{}
\classe{vi}
\paradigme{\textit{dir :} \jya thɯ-}
\begin{définition}\ 
\begin{déclaration}\grammar{denom}\end{déclaration}\end{définition}
\begin{définition}\fra maladroit\end{définition}
\begin{définition}\cmn 笨拙,经常损坏东西\end{définition}
\begin{exemple}\jya jiɕqha nɯ ɲɯ-rɯkɯmaʁ\cmn 那个人动作笨拙\end{exemple}
\begin{relation-sémantique}\confer{
\hyperlink{ⒺmaʁⒽ1}{\textit{ \papi{maʁ1}}}
}\end{relation-sémantique}
\begin{relation-sémantique}\confer{
\hyperlink{Ⓔnɯkɯmaʁ}{\textit{ \papi{nɯkɯmaʁ}}}
}\end{relation-sémantique}\end{entrée}

\begin{entrée}
\vedette{\hypertarget{Ⓔrɯkɯŋu}{\papi{ rɯkɯŋu}}}\markboth{rɯkɯŋu}{}
\classe{vs}
\paradigme{\textit{dir :} \jya tɤ-}
\begin{définition}\ 
\begin{déclaration}\grammar{denom}\end{déclaration}\end{définition}
\begin{définition}\fra attentionné envers sa famille\end{définition}
\begin{définition}\cmn 关心家庭\end{définition}
\begin{exemple}\jya tɯrme kɯ-rɯkɯŋu ci ɲɯ-ŋu\cmn 他是个顾家的人\end{exemple}
\begin{exemple}\jya jiɕqha tɯrme ɲɯ-rɯkɯŋu tɕe, laχtɕha wuma ɲɯ-ɤsɯ-χtɯ\cmn 那个人很顾家,买很多东西回家\end{exemple}
\begin{relation-sémantique}\antonyme{
\hyperlink{Ⓔrɯkɯmaʁ}{\textit{ \papi{rɯkɯmaʁ}}}
}\end{relation-sémantique}
\begin{relation-sémantique}\confer{
\hyperlink{Ⓔŋu}{\textit{ \papi{ŋu}}}
}\end{relation-sémantique}\end{entrée}

\begin{entrée}
\vedette{\hypertarget{Ⓔrɯlajɯ}{\papi{ rɯlajɯ}}}\markboth{rɯlajɯ}{}\classe{vi}
\paradigme{\textit{dir :} \jya thɯ-}
\begin{définition}\ 
\begin{déclaration}\grammar{denom}\end{déclaration}\end{définition}
\begin{définition}\fra chanter un chant de montagne\end{définition}
\begin{définition}\cmn 唱山歌\end{définition}
\begin{exemple}\jya chɤ-rɯlajɯ\cmn 他唱了山歌\end{exemple}
\begin{relation-sémantique}\confer{
\hyperlink{Ⓔlajɯ}{\textit{ \papi{lajɯ}}}
}\end{relation-sémantique}\end{entrée}

\begin{entrée}
\vedette{\hypertarget{Ⓔrɯlaʁjɤt}{\papi{ rɯlaʁjɤt}}}\markboth{rɯlaʁjɤt}{}\classe{vi}
\begin{définition}\ 
\begin{déclaration}\grammar{denom}\end{déclaration}\end{définition}
\begin{définition}\fra faire du travail manuel\end{définition}
\begin{définition}\cmn 做手工活\end{définition}
\begin{relation-sémantique}\confer{
\hyperlink{Ⓔlaʁjɤt}{\textit{ \papi{laʁjɤt}}}
}\end{relation-sémantique}\end{entrée}

\begin{entrée}
\vedette{\hypertarget{Ⓔrɯlɯ}{\papi{ rɯlɯ}}}\markboth{rɯlɯ}{}
\classe{n}
\begin{définition}\fra boulette\end{définition}
\begin{définition}\cmn 小丸;小团;小球\end{définition}\end{entrée}

\begin{entrée}
\vedette{\hypertarget{Ⓔrɯmu}{\papi{ rɯmu}}}\markboth{rɯmu}{}\classe{n}
\begin{définition}\fra motif\end{définition}
\begin{définition}\cmn 纹路
\begin{déclaration} \étymologie{\papi{ri.mo}}\end{déclaration}\end{définition}
\begin{exemple}\jya jaχpa rɯmu\cmn 手纹\end{exemple}\end{entrée}

\begin{entrée}
\vedette{\hypertarget{Ⓔrɯm}{\papi{ rɯm}}}\markboth{rɯm}{}
\classe{vt}
\paradigme{\textit{dir :} \jya lɤ-}
\begin{définition}\ 
\begin{déclaration}\grammar{denom}\end{déclaration}\end{définition}
\begin{définition}\fra faire de la ficelle en roulant dans les mains (sens inverse des aiguilles d'une montre)\end{définition}
\begin{définition}\cmn 搓线(逆时针方向)\end{définition}
\begin{exemple}\jya tɤ-ri lɤ-rɯm-a\cmn 我搓了线\end{exemple}
\begin{relation-sémantique}\synonyme{
\hyperlink{Ⓔpɣo}{\textit{ \papi{pɣo}}}
}\end{relation-sémantique}
\begin{relation-sémantique}\synonyme{
\hyperlink{Ⓔrɤjɯɣ}{\textit{ \papi{rɤjɯɣ}}}
}\end{relation-sémantique}\end{entrée}

\begin{entrée}
\vedette{\hypertarget{Ⓔrɯmani}{\papi{ rɯmani}}}\markboth{rɯmani}{}
\classe{vi}
\paradigme{\textit{dir :} \jya nɯ-}
\begin{définition}\ 
\begin{déclaration}\grammar{denom}\end{déclaration}\end{définition}
\begin{définition}\fra réciter les mantras\end{définition}
\begin{définition}\cmn 念玛尼\end{définition}
\begin{exemple}\jya rgɤrgɯn ra ɲɯ-rɯmani-nɯ\cmn 老年人们在念玛尼\end{exemple}\end{entrée}

\begin{entrée}
\vedette{\hypertarget{Ⓔrɯmba}{\papi{ rɯmba}}}\markboth{rɯmba}{}
\classe{n}
\begin{définition}\fra espèce\end{définition}
\begin{définition}\cmn 种类
\begin{déclaration} \étymologie{\papi{rim.pa}}\end{déclaration}\end{définition}\end{entrée}

\begin{entrée}
\vedette{\hypertarget{Ⓔrɯmboʁkhɯr}{\papi{ rɯmboʁkhɯr}}}\markboth{rɯmboʁkhɯr}{}\classe{vt}
\paradigme{\textit{dir :} \jya thɯ-}
\begin{définition}\ 
\begin{déclaration}\grammar{denom}\end{déclaration}
\begin{déclaration}\grammar{denom}\end{déclaration}\end{définition}
\begin{définition}\fra envelopper avec un tissu rectangulaire\end{définition}
\begin{définition}\cmn 用正方形的布包起来\end{définition}
\begin{exemple}\jya cho-rɯmboʁkhɯr\cmn 他把它包起来了\end{exemple}
\begin{relation-sémantique}\confer{
\hyperlink{Ⓔmboʁkhɯr}{\textit{ \papi{mboʁkhɯr}}}
}\end{relation-sémantique}\end{entrée}

\begin{entrée}
\vedette{\hypertarget{Ⓔrɯmpɕɯmɤr}{\papi{ rɯmpɕɯmɤr}}}\markboth{rɯmpɕɯmɤr}{}
\classe{vi}
\paradigme{\textit{dir :} \jya tɤ-}
\begin{définition}\ 
\begin{déclaration}\grammar{denom}\end{déclaration}\end{définition}
\begin{définition}\fra célébrer\end{définition}
\begin{définition}\cmn 庆祝\end{définition}
\begin{exemple}\jya tɤ-rɯmpɕɯmɤr-i\cmn 我们庆祝了\end{exemple}\end{entrée}

\begin{entrée}
\vedette{\hypertarget{Ⓔrɯmphrɯmɯ}{\papi{ rɯmphrɯmɯ}}}\markboth{rɯmphrɯmɯ}{}
\classe{vi}
\paradigme{\textit{dir :} \jya pɯ-}
\begin{définition}\ 
\begin{déclaration}\grammar{denom}\end{déclaration}\end{définition}\acception{1}
\begin{définition}\fra prédire l'avenir\end{définition}
\begin{définition}\cmn 算命\end{définition}\acception{2}
\begin{définition}\fra se faire les griffes (chat)\end{définition}
\begin{définition}\cmn 抓(猫)\end{définition}
\begin{exemple}\jya lɯlu ɲɯ-rɯmphrɯmɯ\cmn 猫喜欢乱抓\end{exemple}\begin{sous-entrée}
\vedette{\hypertarget{}{\papi{ zrɯmphrɯmɯ}}}\markboth{zrɯmphrɯmɯ}{}\classe{vt}
\begin{définition}\fra faire regarder l'avenir\end{définition}
\begin{définition}\cmn 请别人算命\end{définition}
\begin{exemple}\jya nɯ-sqar-a tɕe pɯ-zrɯmphrɯmɯ-t-a\cmn 我请了他算命\end{exemple}
\begin{relation-sémantique}\confer{
\hyperlink{Ⓔmphrɯmɯ}{\textit{ \papi{mphrɯmɯ}}}
}\end{relation-sémantique}
\end{sous-entrée}\end{entrée}

\begin{entrée}
\vedette{\hypertarget{Ⓔrɯmɯntoʁ}{\papi{ rɯmɯntoʁ}}}\markboth{rɯmɯntoʁ}{}
\classe{vi}
\paradigme{\textit{dir :} \jya nɯ-}
\begin{définition}\fra fleurir\end{définition}
\begin{définition}\cmn 开花
\begin{déclaration}\grammar{denom}\end{déclaration}\end{définition}
\begin{exemple}\jya khɯjŋga ɲɤ-rɯmɯntoʁ\cmn 杜鹃花开了\end{exemple}
\begin{exemple}\jya pɤjka ɲɤ-rɯmɯntoʁ\cmn 白瓜开花了\end{exemple}
\begin{relation-sémantique}\confer{
\hyperlink{Ⓔmɯntoʁ}{\textit{ \papi{mɯntoʁ}}}
}\end{relation-sémantique}\end{entrée}

\begin{entrée}
\vedette{\hypertarget{Ⓔrɯndzaŋspa}{\papi{ rɯndzaŋspa}}}\markboth{rɯndzaŋspa}{}\classe{vi}
\paradigme{\textit{dir :} \jya tɤ-}
\begin{définition}\ 
\begin{déclaration}\grammar{denom}\end{déclaration}\end{définition}
\begin{définition}\fra faire attention\end{définition}
\begin{définition}\cmn 小心
\begin{déclaration} \étymologie{\papi{mdzaŋs.pa}}\end{déclaration}\end{définition}
\begin{exemple}\jya tɤ-rɯndzaŋspa ma kɯ-mɯrkɯ tɯ-ɕlɯɣ\cmn 你小心一点,不然会被偷东西的\end{exemple}\end{entrée}

\begin{entrée}
\vedette{\hypertarget{Ⓔrɯndzɤqhɤjɯ}{\papi{ rɯndzɤqhɤjɯ}}}\markboth{rɯndzɤqhɤjɯ}{}
\classe{vi}
\paradigme{\textit{dir :} \jya tɤ-}
\begin{définition}\fra manger derrière le dos des autres\end{définition}
\begin{définition}\cmn 瞒着别人偷吃\end{définition}
\begin{exemple}\jya jiɕqha ɲɯ-rɯndzɤqhɤjɯ\cmn 那个人瞒着别人偷吃东西\end{exemple}
\begin{exemple}\jya tɤ-rɯndzɤqhɤjɯ-tɕi\cmn 我们俩偷吃了东西\end{exemple}
\begin{exemple}\jya ma-tɤ-tɯ-rɯndzɤqhɤjɯ\cmn 你不要瞒着别人偷吃!\end{exemple}\begin{sous-entrée}
\vedette{\hypertarget{}{\papi{ nɯndzɤqhɤjɯ}}}\markboth{nɯndzɤqhɤjɯ}{}\classe{vt}
\paradigme{\textit{dir :} \jya tɤ-}
\begin{définition}\fra manger derrière le dos de ...\end{définition}
\begin{définition}\cmn 瞒着……偷吃\end{définition}
\begin{exemple}\jya tɤ-ta-nɯndzɤqhɤjɯ\cmn 我瞒着你偷吃东西了\end{exemple}
\begin{relation-sémantique}\confer{
\hyperlink{Ⓔndzɤqhɤjɯ}{\textit{ \papi{ndzɤqhɤjɯ}}}
}\end{relation-sémantique}
\end{sous-entrée}\end{entrée}

\begin{entrée}
\vedette{\hypertarget{Ⓔrɯndzɤtshi}{\papi{ rɯndzɤtshi}}}\markboth{rɯndzɤtshi}{}
\classe{vi}
\paradigme{\textit{dir :} \jya tɤ-}
\paradigme{\textit{dir :} \jya thɯ-}
\begin{définition}\ 
\begin{déclaration}\grammar{denom}\end{déclaration}\end{définition}
\begin{définition}\fra prendre un repas\end{définition}
\begin{définition}\cmn 吃一顿\end{définition}
\begin{exemple}\jya thɯ-rɯndzɤtshi\cmn 你吃饭吧!\end{exemple}
\begin{exemple}\jya tu-rɯndzɤtshi ɯ-ɲɯ́-cha?\cmn 他能不能吃饭?(说一个病人)\end{exemple}
\begin{relation-sémantique}\confer{
\hyperlink{ⒺndzɤtshiⒽ2}{\textit{ \papi{ndzɤtshi2}}}
}\end{relation-sémantique}\begin{sous-entrée}
\vedette{\hypertarget{}{\papi{ zrɯndzɤtshi}}}\markboth{zrɯndzɤtshi}{}\classe{vt}
\paradigme{\textit{dir :} \jya tɤ-}
\begin{définition}\fra donner un repas\end{définition}
\begin{définition}\cmn 请别人吃一顿\end{définition}
\end{sous-entrée}\end{entrée}

\begin{entrée}
\vedette{\hypertarget{Ⓔrɯɲɟele}{\papi{ rɯɲɟele}}}\markboth{rɯɲɟele}{}
\classe{vi}
\paradigme{\textit{dir :} \jya thɯ-}
\begin{définition}\fra tendre les jambes\end{définition}
\begin{définition}\cmn 伸脚\end{définition}
\begin{exemple}\jya thɯ-rɯɲɟele-a\cmn 我伸了脚\end{exemple}
\begin{exemple}\jya sɤrɲɟɤle\end{exemple}\end{entrée}

\begin{entrée}
\vedette{\hypertarget{Ⓔrɯŋgɤlwoʁ}{\papi{ rɯŋgɤlwoʁ}}}\markboth{rɯŋgɤlwoʁ}{}
\classe{vi}
\begin{définition}\ 
\begin{déclaration}\grammar{incorp}\end{déclaration}\end{définition}
\begin{définition}\fra gaspiller, éparpiller\end{définition}
\begin{définition}\cmn 乱撒;浪费\end{définition}
\begin{exemple}\jya ma-pɯ-tɯ-rɯŋgɤlwoʁ\cmn 你不要乱撒\end{exemple}
\begin{relation-sémantique}\synonyme{
\hyperlink{Ⓔrɯtɕhɯχtɤr}{\textit{ \papi{rɯtɕhɯχtɤr}}}
}\end{relation-sémantique}
\begin{relation-sémantique}\synonyme{
\hyperlink{Ⓔlwoʁ}{\textit{ \papi{lwoʁ}}}
}\end{relation-sémantique}\end{entrée}

\begin{entrée}
\vedette{\hypertarget{Ⓔrɯŋgoŋpu}{\papi{ rɯŋgoŋpu}}}\markboth{rɯŋgoŋpu}{}
\classe{vi}
\paradigme{\textit{dir :} \jya tɤ-}
\begin{définition}\fra provoquer des désastres\end{définition}
\begin{définition}\cmn 惹祸;破坏东西\end{définition}
\begin{exemple}\jya jisŋi tɤ-rɯŋgoŋpu-a\cmn 我今天破坏了东西\end{exemple}\end{entrée}

\begin{entrée}
\vedette{\hypertarget{Ⓔrɯŋundʑu}{\papi{ rɯŋundʑu}}}\markboth{rɯŋundʑu}{}\classe{vi}
\paradigme{\textit{dir :} \jya tɤ-}
\begin{définition}\fra chercher à s'attirer les faveurs des gens\end{définition}
\begin{définition}\cmn 讨好;跟别人说好话\end{définition}
\begin{exemple}\jya ɯ-phe tɤ-rɯŋundʑu-a\cmn 我跟他说了好话\end{exemple}
\begin{relation-sémantique}\confer{
\hyperlink{Ⓔnɯŋundʑu}{\textit{ \papi{nɯŋundʑu}}}
}\end{relation-sémantique}\end{entrée}

\begin{entrée}
\vedette{\hypertarget{Ⓔrɯŋɯŋɤn}{\papi{ rɯŋɯŋɤn}}}\markboth{rɯŋɯŋɤn}{}
\classe{vi}
\paradigme{\textit{dir :} \jya tɤ-}
\begin{définition}\fra causer des dégâts\end{définition}
\begin{définition}\cmn 搞破坏;捣乱\end{définition}
\begin{exemple}\jya tɤ-rɟit kɯ-rɯŋɯŋɤn ci ɲɯ-ŋu\cmn 他是一个(爱)搞破坏的孩子\end{exemple}
\begin{exemple}\jya tɕhɯthɤn chɤ-ɣi tɕe to-rɯŋɯŋɤn\cmn 洪水来了,搞了破坏\end{exemple}\end{entrée}

\begin{entrée}
\vedette{\hypertarget{Ⓔrɯphɯrɤm}{\papi{ rɯphɯrɤm}}}\markboth{rɯphɯrɤm}{}
\begin{relation-sémantique}\confer{
\hyperlink{Ⓔnɯphɯrɤm}{\textit{ \papi{nɯphɯrɤm}}}
}\end{relation-sémantique}\end{entrée}

\begin{entrée}
\vedette{\hypertarget{Ⓔrɯphɯrlaʁ}{\papi{ rɯphɯrlaʁ}}}\markboth{rɯphɯrlaʁ}{}
\classe{vi}
\paradigme{\textit{dir :} \jya thɯ-}
\begin{définition}\fra ruiner\end{définition}
\begin{définition}\cmn 倾家荡产;破坏
\begin{déclaration} \étymologie{\papi{ⁿpʰro.brlag}}\end{déclaration}\end{définition}
\begin{exemple}\jya nɯ-kha thamtɕɤt cho-phɯt, nɯ-laχtɕha chɤ-sɤrɕo tɕe chɤ-rɯphɯrlaʁ\cmn 他拆了房子,浪费了他们的东西,把财产花光了\end{exemple}\end{entrée}

\begin{entrée}
\vedette{\hypertarget{Ⓔrɯpjɤβlaʁ}{\papi{ rɯpjɤβlaʁ}}}\markboth{rɯpjɤβlaʁ}{}
\classe{vs}
\paradigme{\textit{dir :} \jya tɤ-}
\begin{définition}\fra rusé\end{définition}
\begin{définition}\cmn 狡猾\end{définition}
\begin{exemple}\jya jiɕqha nɯ ɲɯ-rɯpjɤβlaʁ\cmn 那个人很狡猾\end{exemple}\end{entrée}

\begin{entrée}
\vedette{\hypertarget{Ⓔrɯpjɤŋkhɤr}{\papi{ rɯpjɤŋkhɤr}}}\markboth{rɯpjɤŋkhɤr}{}
\classe{vi}
\paradigme{\textit{dir :} \jya tɤ-}
\begin{définition}\fra tourner dans le ciel (oiseau)\end{définition}
\begin{définition}\cmn 盘旋
\begin{déclaration} \étymologie{\papi{ⁿkʰor}}\end{déclaration}\end{définition}
\begin{exemple}\jya qaliaʁ ɲɯ-rɯpjɤŋkhɤr\cmn 老鹰在盘旋\end{exemple}\end{entrée}

\begin{entrée}
\vedette{\hypertarget{Ⓔrɯqajɯ}{\papi{ rɯqajɯ}}}\markboth{rɯqajɯ}{}\classe{vi}
\paradigme{\textit{dir :} \jya nɯ-}
\begin{définition}\ 
\begin{déclaration}\grammar{denom}\end{déclaration}\end{définition}
\begin{définition}\fra avoir des vers\end{définition}
\begin{définition}\cmn 生蛆\end{définition}
\begin{exemple}\jya tɤ-mthɯm ɲɤ-rɯqajɯ\cmn 肉生蛆了\end{exemple}
\begin{relation-sémantique}\confer{
\hyperlink{Ⓔqajɯ}{\textit{ \papi{qajɯ}}}
}\end{relation-sémantique}
\begin{relation-sémantique}\confer{
\hyperlink{Ⓔnɯqajɯ}{\textit{ \papi{nɯqajɯ}}}
}\end{relation-sémantique}\end{entrée}

\begin{entrée}
\vedette{\hypertarget{Ⓔrɯqartsɤβ}{\papi{ rɯqartsɤβ}}}\markboth{rɯqartsɤβ}{}
\classe{vi}
\paradigme{\textit{dir :} \jya kɤ-}
\begin{définition}\ 
\begin{déclaration}\grammar{denom}\end{déclaration}
\begin{déclaration}\grammar{denom}\end{déclaration}\end{définition}
\begin{définition}\fra récolter\end{définition}
\begin{définition}\cmn 收割\end{définition}
\begin{exemple}\jya kɤ-rɯqartsɤβ-i\cmn 我们收割了\end{exemple}
\begin{exemple}\jya ɲɯ-rɯqartsɤβ-nɯ\cmn 他们在收割\end{exemple}
\begin{relation-sémantique}\confer{
\hyperlink{Ⓔqartsɤβ}{\textit{ \papi{qartsɤβ}}}
}\end{relation-sémantique}\end{entrée}

\begin{entrée}
\vedette{\hypertarget{Ⓔrɯqhaχɕu}{\papi{ rɯqhaχɕu}}}\markboth{rɯqhaχɕu}{}\classe{vi}
\paradigme{\textit{dir :} \jya tɤ-}
\begin{définition}\fra se vanter\end{définition}
\begin{définition}\cmn 炫耀\end{définition}
\begin{exemple}\jya jiɕqha nɯ ɲɯ-rɯqhaχɕu\cmn 那个人在夸耀自己\end{exemple}
\begin{exemple}\jya laχtɕha to-nɯχtɯ tɕe ɲɯ-rɯqhaχɕu\cmn 他买了东西就炫耀\end{exemple}
\begin{exemple}\jya ma-tɯ-rɯqhaχɕu ntsɯ\cmn 你不要总是夸耀自己\end{exemple}
\begin{exemple}\jya ``nɤʑo tɯ-rɯqhaχɕu ntsɯ sɤznɤ, nɤ-rɟɯ nɯra mɤ-tɯ-rɯre kɯ" toti\cmn 他对算命先生说:“比起夸耀自己(很会算命),你还不如看好你的财产”(算命先生被贼偷了东西的故事)\end{exemple}\begin{sous-entrée}
\vedette{\hypertarget{}{\papi{ nɯqhaχɕu}}}\markboth{nɯqhaχɕu}{}\classe{vt}
\paradigme{\textit{dir :} \jya tɤ-}
\begin{définition}\fra se vanter de\end{définition}
\begin{définition}\cmn 炫耀(某种东西)\end{définition}
\begin{exemple}\jya ɯ-ŋga ci to-nɯ-χtɯ, nɯ tu-nɯqhaχɕe ɲɯ-ŋu\cmn 他买了一件衣服,现在一直在炫耀\end{exemple}
\begin{relation-sémantique}\confer{
\hyperlink{Ⓔqhaχɕu}{\textit{ \papi{qhaχɕu}}}
}\end{relation-sémantique}
\begin{relation-sémantique}\confer{
\hyperlink{Ⓔχɕu}{\textit{ \papi{χɕu}}}
}\end{relation-sémantique}
\begin{relation-sémantique}\confer{
\hyperlink{Ⓔznaχɕɯχɕu}{\textit{ \papi{znaχɕɯχɕu}}}
}\end{relation-sémantique}
\end{sous-entrée}\end{entrée}

\begin{entrée}
\vedette{\hypertarget{Ⓔrɯru}{\papi{ rɯru}}}\markboth{rɯru}{}
\classe{vt}
\paradigme{\textit{dir :} \jya nɯ-}
\begin{définition}\fra garder, surveiller\end{définition}
\begin{définition}\cmn 守卫;看守\end{définition}
\begin{exemple}\jya fsapaʁ nɯ-rɯre\cmn 我看一下牲畜吧\end{exemple}
\begin{exemple}\jya smi nɯ-rɯre\cmn 你看火吧\end{exemple}
\begin{relation-sémantique}\confer{
\hyperlink{ⒺruⒽ1}{\textit{ \papi{ru1}}}
}\end{relation-sémantique}
\begin{relation-sémantique}\synonyme{
\hyperlink{Ⓔnɤmdzɯ}{\textit{ \papi{nɤmdzɯ}}}
}\end{relation-sémantique}\end{entrée}

\begin{entrée}
\vedette{\hypertarget{Ⓔrɯra}{\papi{ rɯra}}}\markboth{rɯra}{}\classe{vi}
\paradigme{\textit{dir :} \jya \_}
\begin{définition}\fra aller voir\end{définition}
\begin{définition}\cmn 探望
\begin{déclaration}\use{沙尔宗方言}\end{déclaration}\end{définition}
\begin{exemple}\jya ɯʑo ɲɯ-ngo tɕe z-jɤ-rɯra-a (=z-jɤ-rtoʁ-a)\cmn 他生病,我就去看他了\end{exemple}\end{entrée}

\begin{entrée}
\vedette{\hypertarget{Ⓔrɯrawa}{\papi{ rɯrawa}}}\markboth{rɯrawa}{}\classe{vi}
\paradigme{\textit{dir :} \jya nɯ-}
\begin{définition}\fra exiger des autres\end{définition}
\begin{définition}\cmn 要求别人为自己付出
\begin{déclaration} \étymologie{\papi{re.ba}}\end{déclaration}\end{définition}
\begin{exemple}\jya ɯʑo a-ɕki rŋɯl kɤ-mbi ɲɯ-nɯrawa\cmn 他要求我给他钱\end{exemple}
\begin{exemple}\jya ɯʑo a-ɕki tu-kɤ-qur ntsɯ ɲɯ-nɯrawa ŋu\cmn 他总是要求我帮他\end{exemple}\end{entrée}

\begin{entrée}
\vedette{\hypertarget{Ⓔrɯrɤt}{\papi{ rɯrɤt}}}\markboth{rɯrɤt}{}\classe{vt}
\paradigme{\textit{dir :} \jya tɤ-}
\begin{définition}\fra décider des soutras à réciter pour quelqu'un\end{définition}
\begin{définition}\cmn 喇嘛规定(给别人)念经\end{définition}
\begin{exemple}\jya nɯ-rpi kɯ-ɴqɯ-ɴqa ʑo to-rɯrɤt-nɯ\cmn (喇嘛们)规定给他们念隆重的佛经\end{exemple}
\begin{relation-sémantique}\synonyme{
\hyperlink{ⒺkhrɤtⒽ2}{\textit{ \papi{khrɤt2}}}
}\end{relation-sémantique}\end{entrée}

\begin{entrée}
\vedette{\hypertarget{Ⓔrɯrcaŋpɕaʁ}{\papi{ rɯrcaŋpɕaʁ}}}\markboth{rɯrcaŋpɕaʁ}{}\classe{vi}
\begin{définition}\ 
\begin{déclaration}\grammar{denom}\end{déclaration}\end{définition}
\begin{définition}\fra se prosterner jusqu'à un lieu saint tout le long de la route\end{définition}
\begin{définition}\cmn 磕长头(到观音桥)
\begin{déclaration} \étymologie{\papi{brkʲaŋs.pʰʲag}}\end{déclaration}\end{définition}
\begin{relation-sémantique}\confer{
\hyperlink{Ⓔrcaŋpɕaʁ}{\textit{ \papi{rcaŋpɕaʁ}}}
}\end{relation-sémantique}\end{entrée}

\begin{entrée}
\vedette{\hypertarget{Ⓔrɯrdɤβzu}{\papi{ rɯrdɤβzu}}}\markboth{rɯrdɤβzu}{}\classe{vi}
\begin{définition}\fra faire de la maçonnerie\end{définition}
\begin{définition}\cmn 做石工\end{définition}
\begin{exemple}\jya ʑara ɣɯ ku-rɯrdɤβzu-a\cmn 我在给他们做石工\end{exemple}
\begin{relation-sémantique}\confer{
\hyperlink{Ⓔrdɤβzu}{\textit{ \papi{rdɤβzu}}}
}\end{relation-sémantique}\end{entrée}

\begin{entrée}
\vedette{\hypertarget{Ⓔrɯrgɤm}{\papi{ rɯrgɤm}}}\markboth{rɯrgɤm}{}\classe{n}
\begin{définition}\fra cercueil\end{définition}
\begin{définition}\cmn 棺材
\begin{déclaration} \étymologie{\papi{rus.sgam?}}\end{déclaration}\end{définition}\end{entrée}

\begin{entrée}
\vedette{\hypertarget{Ⓔrɯri}{\papi{ rɯri}}}\markboth{rɯri}{}
\begin{relation-sémantique}\confer{
\hyperlink{Ⓔraŋri}{\textit{ \papi{raŋri}}}
}\end{relation-sémantique}\end{entrée}

\begin{entrée}
\vedette{\hypertarget{Ⓔrɯrɟa}{\papi{ rɯrɟa}}}\markboth{rɯrɟa}{}\classe{vt}
\paradigme{\textit{dir :} \jya pɯ-}
\begin{définition}\ 
\begin{déclaration}\grammar{denom}\end{déclaration}\end{définition}
\begin{définition}\fra maudire\end{définition}
\begin{définition}\cmn 咒骂,诅咒\end{définition}
\begin{exemple}\jya nɤʑo kɯ pɯ-kɯ-rɯrɟa-a\cmn 你咒骂我了\end{exemple}
\begin{relation-sémantique}\confer{
\hyperlink{Ⓔɯ-rɟa}{\textit{ \papi{ɯ-rɟa}}}
}\end{relation-sémantique}\end{entrée}

\begin{entrée}
\vedette{\hypertarget{Ⓔrɯrɟaŋrɟɤz}{\papi{ rɯrɟaŋrɟɤz}}}\markboth{rɯrɟaŋrɟɤz}{}\classe{vi}
\begin{définition}\fra perdre du temps\end{définition}
\begin{définition}\cmn 拖延时间\end{définition}
\begin{exemple}\jya ma-tɯ-rɯrɟaŋrɟɤz kɯ tɤ-mbɣom ma kɤ-nɤma kɤ-sthɯt mɤ-tsu\cmn 别拖延时间,抓紧时间,不然的话这件事情做不完\end{exemple}\end{entrée}

\begin{entrée}
\vedette{\hypertarget{Ⓔrɯrɟɯfsoʁ}{\papi{ rɯrɟɯfsoʁ}}}\markboth{rɯrɟɯfsoʁ}{}
\begin{relation-sémantique}\confer{
\hyperlink{Ⓔɣɯrɟɯfsoʁ}{\textit{ \papi{ɣɯrɟɯfsoʁ}}}
}\end{relation-sémantique}\end{entrée}

\begin{entrée}
\vedette{\hypertarget{Ⓔrɯrtsi}{\papi{ rɯrtsi}}}\markboth{rɯrtsi}{}
\classe{n}\acception{1}
\begin{définition}\fra montagne\end{définition}
\begin{définition}\cmn 高山\end{définition}\acception{2}
\begin{définition}\fra dieu de la montagne\end{définition}
\begin{définition}\cmn 山神
\begin{déclaration} \étymologie{\papi{ri.rtse}}\end{déclaration}\end{définition}\end{entrée}

\begin{entrée}
\vedette{\hypertarget{Ⓔrɯrtsɯpɣaʁ}{\papi{ rɯrtsɯpɣaʁ}}}\markboth{rɯrtsɯpɣaʁ}{}\paradigme{\textit{dir :} \jya lɤ-}
\begin{définition}\ 
\begin{déclaration}\grammar{denom}\end{déclaration}\end{définition}
\begin{définition}\fra retourner la terre après la récolte\end{définition}
\begin{définition}\cmn 庄稼收割了以后重新翻地\end{définition}
\begin{relation-sémantique}\confer{
\hyperlink{Ⓔrtsɯpɣaʁ}{\textit{ \papi{rtsɯpɣaʁ}}}
}\end{relation-sémantique}
\begin{relation-sémantique}\confer{
\hyperlink{Ⓔnɯrtsɯpɣaʁ}{\textit{ \papi{nɯrtsɯpɣaʁ}}}
}\end{relation-sémantique}\classe{vi}\end{entrée}

\begin{entrée}
\vedette{\hypertarget{Ⓔrɯrtsɯtʂɯɣ}{\papi{ rɯrtsɯtʂɯɣ}}}\markboth{rɯrtsɯtʂɯɣ}{}\classe{vi}
\paradigme{\textit{dir :} \jya nɯ-}
\paradigme{\textit{dir :} \jya thɯ-}
\begin{définition}\fra faire les comptes\end{définition}
\begin{définition}\cmn 算帐
\begin{déclaration} \étymologie{\papi{rtsi.sgrig}}\end{déclaration}\end{définition}
\begin{exemple}\jya thɯ-rɯrtsɯtʂɯɣ-a / rtsɯtʂɯɣ thɯ-ta-t-a\cmn 我算账了\end{exemple}\end{entrée}

\begin{entrée}
\vedette{\hypertarget{Ⓔrɯʁdɯʁdɯɣ}{\papi{ rɯʁdɯʁdɯɣ}}}\markboth{rɯʁdɯʁdɯɣ}{}
\classe{vs}
\paradigme{\textit{dir :} \jya tɤ-}
\paradigme{\textit{case :} \jya ɣɯ}
\begin{définition}\fra gêner\end{définition}
\begin{définition}\cmn 妨碍别人做事\end{définition}
\begin{exemple}\jya tɯrme ɣɯ ɲɯ-rɯʁdɯʁdɯɣ\cmn 他妨碍别人的工作\end{exemple}
\begin{exemple}\jya aʑɯɣ ɲɯ-rɯʁdɯʁdɯɣ\cmn 他妨碍我的工作\end{exemple}\end{entrée}

\begin{entrée}
\vedette{\hypertarget{Ⓔrɯʁdɯxpa}{\papi{ rɯʁdɯxpa}}}\markboth{rɯʁdɯxpa}{}\classe{vs}
\paradigme{\textit{dir :} \jya tɤ-}
\begin{définition}\ 
\begin{déclaration}\grammar{denom}\end{déclaration}\end{définition}
\begin{définition}\fra empêcher\end{définition}
\begin{définition}\cmn 妨碍\end{définition}
\begin{exemple}\jya ɲɯ-rɯʁdɯxpa\cmn 他在妨碍人\end{exemple}
\begin{exemple}\jya phɤnba kɤ-βzu mɤ-kɯ-cha ci pɯ-ŋu kɯnɤ, ma-tɤ-kɯ-rɯʁdɯxpa ra\cmn 没有能力帮别人的话,也不可以妨碍别人\end{exemple}
\begin{relation-sémantique}\confer{
\hyperlink{Ⓔrɯʁdɯʁdɯɣ}{\textit{ \papi{rɯʁdɯʁdɯɣ}}}
}\end{relation-sémantique}
\begin{relation-sémantique}\confer{
\hyperlink{Ⓔʁdɯxpa}{\textit{ \papi{ʁdɯxpa}}}
}\end{relation-sémantique}\end{entrée}

\begin{entrée}
\vedette{\hypertarget{Ⓔrɯʁgiwa}{\papi{ rɯʁgiwa}}}\markboth{rɯʁgiwa}{}\classe{vi}
\paradigme{\textit{dir :} \jya tɤ-}
\begin{définition}\fra faire lire des soutras pour les morts\end{définition}
\begin{définition}\cmn 请人念经\end{définition}
\begin{relation-sémantique}\confer{
\hyperlink{Ⓔʁgiwa}{\textit{ \papi{ʁgiwa}}}
}\end{relation-sémantique}\end{entrée}

\begin{entrée}
\vedette{\hypertarget{Ⓔrɯʁlɤwɯr}{\papi{ rɯʁlɤwɯr}}}\markboth{rɯʁlɤwɯr}{}
\classe{vi}
\paradigme{\textit{dir :} \jya tɤ-}
\begin{définition}\fra soudain\end{définition}
\begin{définition}\cmn 突然
\begin{déclaration} \étymologie{\papi{glo.bur}}\end{déclaration}\end{définition}
\begin{exemple}\jya @wenchuan waɟɯ to-rɯʁlɤwɯr ɕti\cmn 汶川大地震发生得很突然\end{exemple}\end{entrée}

\begin{entrée}
\vedette{\hypertarget{Ⓔrɯscɯscit}{\papi{ rɯscɯscit}}}\markboth{rɯscɯscit}{}\classe{vs}
\begin{définition}\fra oisif\end{définition}
\begin{définition}\cmn 清闲;安逸\end{définition}
\begin{exemple}\jya ɯ-tɯ-rɯscɯscit nɯ!\cmn 他真清闲\end{exemple}\end{entrée}

\begin{entrée}
\vedette{\hypertarget{Ⓔrɯskɤrwa}{\papi{ rɯskɤrwa}}}\markboth{rɯskɤrwa}{}
\classe{vi}
\paradigme{\textit{dir :} \jya kɤ-}
\begin{définition}\ 
\begin{déclaration}\grammar{denom}\end{déclaration}\end{définition}
\begin{définition}\fra faire tourner les moulins à prière\end{définition}
\begin{définition}\cmn 转经\end{définition}
\begin{exemple}\jya kɯ-rɯskɤrwa jɤ-ari-a\cmn 我去转经了\end{exemple}
\begin{exemple}\jya ɕ-kɤ-rɯskɤrwa-a\cmn 我去转经了\end{exemple}
\begin{exemple}\jya aʑo χsɯ-tɤxɯr kɤ-rɯskɤrwa-a\cmn 我转经转了三周\end{exemple}
\begin{relation-sémantique}\confer{
\hyperlink{Ⓔskɤrwa}{\textit{ \papi{skɤrwa}}}
}\end{relation-sémantique}\end{entrée}

\begin{entrée}
\vedette{\hypertarget{Ⓔrɯsɲaŋne}{\papi{ rɯsɲaŋne}}}\markboth{rɯsɲaŋne}{}\classe{vi}
\paradigme{\textit{dir :} \jya pɯ-}
\begin{définition}\fra jeûner\end{définition}
\begin{définition}\cmn 念哑巴经(禁食斋) 我念了哑巴经\end{définition}
\begin{relation-sémantique}\confer{
\hyperlink{Ⓔsɲaŋne}{\textit{ \papi{sɲaŋne}}}
}\end{relation-sémantique}
\begin{relation-sémantique}\confer{
\hyperlink{Ⓔnɯsɲaŋne}{\textit{ \papi{nɯsɲaŋne}}}
}\end{relation-sémantique}\end{entrée}

\begin{entrée}
\vedette{\hypertarget{Ⓔrɯspa}{\papi{ rɯspa}}}\markboth{rɯspa}{}\classe{n}
\begin{définition}\fra génie\end{définition}
\begin{définition}\cmn 天才
\begin{déclaration} \étymologie{\papi{rigs.pa}}\end{déclaration}\end{définition}
\end{entrée}

\begin{entrée}
\vedette{\hypertarget{Ⓔrɯstɯnmɯ}{\papi{ rɯstɯnmɯ}}}\markboth{rɯstɯnmɯ}{}
\classe{vi}
\paradigme{\textit{dir :} \jya tɤ-}
\begin{définition}\fra se marier\end{définition}
\begin{définition}\cmn 结婚
\begin{déclaration} \étymologie{\papi{ston.mo}}\end{déclaration}\end{définition}
\begin{exemple}\jya ji-me tɤ-rɯstɯnmɯ\cmn 我们的女儿结了婚\end{exemple}
\begin{exemple}\jya ji-tɕɯ tɤ-rɯstɯnmɯ\cmn 我们的儿子结了婚\end{exemple}
\begin{exemple}\jya kɯ-rɯstɯnmɯ ɣɤʑu\cmn 有人在结婚\end{exemple}
\begin{exemple}\jya ɕ-tɤ-rɯstɯnmɯ-a\cmn 我去结婚了\end{exemple}
\begin{exemple}\jya kɯ-rɯstɯnmɯ tɤrca ju-ɕe-a ŋu\cmn 我去参加婚礼\end{exemple}\begin{sous-entrée}
\vedette{\hypertarget{}{\papi{ zrɯstɯnmɯ}}}\markboth{zrɯstɯnmɯ}{}\classe{vt}
\paradigme{\textit{dir :} \jya tɤ-}
\begin{définition}\ 
\begin{déclaration}\grammar{caus}\end{déclaration}\end{définition}
\begin{définition}\fra marier, faire se marier\end{définition}
\begin{définition}\cmn 使……结婚\end{définition}
\end{sous-entrée}\end{entrée}

\begin{entrée}
\vedette{\hypertarget{Ⓔrɯsɯso}{\papi{ rɯsɯso}}}\markboth{rɯsɯso}{}
\classe{vi}
\begin{définition}\ 
\begin{déclaration}\grammar{denom}\end{déclaration}\end{définition}\acception{1}
\paradigme{\textit{dir :} \jya thɯ-}
\begin{définition}\fra réfléchir\end{définition}
\begin{définition}\cmn 想\end{définition}\acception{2}
\begin{définition}\fra se souvenir\end{définition}
\begin{définition}\cmn 回忆\end{définition}
\begin{exemple}\jya ɲɯ-rɯsɯso\cmn 他在想\end{exemple}\acception{3}
\paradigme{\textit{dir :} \jya pɯ-}
\begin{définition}\fra comparer\end{définition}
\begin{définition}\cmn 比较\end{définition}
\begin{exemple}\jya nɤʑo cho pjɯ-kɯ-rɯsɯso tɕe, aʑo kɯ a-laz ɲɯ-sna\cmn 跟你比较的话,我的命要好一些\end{exemple}
\begin{relation-sémantique}\confer{
\hyperlink{Ⓔsɯso}{\textit{ \papi{sɯso}}}
}\end{relation-sémantique}\end{entrée}

\begin{entrée}
\vedette{\hypertarget{Ⓔrɯtɤmtɯ}{\papi{ rɯtɤmtɯ}}}\markboth{rɯtɤmtɯ}{}\classe{vt}
\paradigme{\textit{dir :} \jya thɯ-}
\begin{définition}\ 
\begin{déclaration}\grammar{denom}\end{déclaration}\end{définition}
\begin{définition}\fra faire un nœud\end{définition}
\begin{définition}\cmn 打结\end{définition}
\begin{exemple}\jya tɤ-ri ɯ-ndo thɯ-rɯtɤmtɯ-t-a\cmn 我在线的一头打了个结\end{exemple}
\begin{relation-sémantique}\confer{
\hyperlink{Ⓔtɤ-mtɯ}{\textit{ \papi{tɤ-mtɯ}}}
}\end{relation-sémantique}\end{entrée}

\begin{entrée}
\vedette{\hypertarget{Ⓔrɯtɕɤmɯ}{\papi{ rɯtɕɤmɯ}}}\markboth{rɯtɕɤmɯ}{}\classe{vi}
\paradigme{\textit{dir :} \jya lɤ-}
\begin{définition}\ 
\begin{déclaration}\grammar{denom}\end{déclaration}\end{définition}
\begin{définition}\fra devenir none\end{définition}
\begin{définition}\cmn 当尼姑\end{définition}
\begin{relation-sémantique}\confer{
\hyperlink{Ⓔtɕɤmɯ}{\textit{ \papi{tɕɤmɯ}}}
}\end{relation-sémantique}
\begin{relation-sémantique}\confer{
\hyperlink{Ⓔnɯtɕɤmɯ}{\textit{ \papi{nɯtɕɤmɯ}}}
}\end{relation-sémantique}\end{entrée}

\begin{entrée}
\vedette{\hypertarget{Ⓔrɯtɕhɤβ}{\papi{ rɯtɕhɤβ}}}\markboth{rɯtɕhɤβ}{}
\classe{n}
\begin{définition}\fra endroit où il n'y a que des rochers et pas d'herbe\end{définition}
\begin{définition}\cmn 高山上只有岩石没有草的地方\end{définition}\end{entrée}

\begin{entrée}
\vedette{\hypertarget{Ⓔrɯtɕhɤfɕɤt}{\papi{ rɯtɕhɤfɕɤt}}}\markboth{rɯtɕhɤfɕɤt}{}\classe{vi}
\begin{définition}\fra participer à un débat philosophique\end{définition}
\begin{définition}\cmn 辩经\end{définition}
\begin{exemple}\jya χpɯn ra ɲɯ-rɯtɕhɤfɕɤt-nɯ\cmn 和尚们在辩经\end{exemple}
\begin{relation-sémantique}\confer{
\hyperlink{Ⓔtɕhɤfɕɤt}{\textit{ \papi{tɕhɤfɕɤt}}}
}\end{relation-sémantique}\end{entrée}

\begin{entrée}
\vedette{\hypertarget{Ⓔrɯtɕhɯtɕhi}{\papi{ rɯtɕhɯtɕhi}}}\markboth{rɯtɕhɯtɕhi}{}\classe{vi}
\begin{définition}\fra chicaner\end{définition}
\begin{définition}\cmn 讲究,计较\end{définition}
\begin{exemple}\jya nɤ-kɤ-qha ɯ-tɯ-dɤn nɯ, a-mɤ-tɯ-rɯtɕhɯtɕhi\cmn 你不喜欢的东西太多了,不要这么计较\end{exemple}\end{entrée}

\begin{entrée}
\vedette{\hypertarget{Ⓔrɯtɕhɯχtɤr}{\papi{ rɯtɕhɯχtɤr}}}\markboth{rɯtɕhɯχtɤr}{}
\classe{vi}
\paradigme{\textit{dir :} \jya thɯ-}
\begin{définition}\fra éparpiller, gaspiller\end{définition}
\begin{définition}\cmn 乱撒,浪费\end{définition}
\begin{exemple}\jya ma-pɯ-tɯ-rɯtɕhɯχtɤr\cmn 你不要浪费\end{exemple}
\begin{relation-sémantique}\confer{
\hyperlink{Ⓔrɯŋgɤlwoʁ}{\textit{ \papi{rɯŋgɤlwoʁ}}}
}\end{relation-sémantique}
\begin{relation-sémantique}\confer{
\hyperlink{Ⓔphɤtɕhɯχtɤr}{\textit{ \papi{phɤtɕhɯχtɤr}}}
}\end{relation-sémantique}\end{entrée}

\begin{entrée}
\vedette{\hypertarget{Ⓔrɯtɕi}{\papi{ rɯtɕi}}}\markboth{rɯtɕi}{}\classe{cnj}
\begin{définition}\fra mais\end{définition}
\begin{définition}\cmn 但是\end{définition}\end{entrée}

\begin{entrée}
\vedette{\hypertarget{Ⓔrɯtshoŋpa}{\papi{ rɯtshoŋpa}}}\markboth{rɯtshoŋpa}{}
\classe{vi}
\paradigme{\textit{dir :} \jya pɯ-}
\begin{définition}\fra faire du commerce\end{définition}
\begin{définition}\cmn 做生意
\begin{déclaration} \étymologie{\papi{tsʰoŋ.pa}}\end{déclaration}\end{définition}
\begin{exemple}\jya pjɤ-rɯtshoŋpa\cmn 他以前做生意(现在不做了)\end{exemple}\end{entrée}

\begin{entrée}
\vedette{\hypertarget{Ⓔrɯtʂa}{\papi{ rɯtʂa}}}\markboth{rɯtʂa}{}
\classe{n}
\begin{définition}\fra envie\end{définition}
\begin{définition}\cmn 妒忌\end{définition}
\begin{exemple}\jya rɯtʂa ʁo kɯ-pe ci maʁ\cmn 妒忌是不好的\end{exemple}
\begin{relation-sémantique}\confer{
\hyperlink{Ⓔnɯrɯtʂa}{\textit{ \papi{nɯrɯtʂa}}}
}\end{relation-sémantique}\end{entrée}

\begin{entrée}
\vedette{\hypertarget{Ⓔrɯtɯsqa}{\papi{ rɯtɯsqa}}}\markboth{rɯtɯsqa}{}\classe{vi}
\paradigme{\textit{dir :} \jya lɤ-}
\paradigme{\textit{dir :} \jya kɤ-}
\begin{définition}\ 
\begin{déclaration}\grammar{denom}\end{déclaration}\end{définition}
\begin{définition}\fra manger du gruau de blé\end{définition}
\begin{définition}\cmn 吃麦子粥\end{définition}
\begin{exemple}\jya lɤ-rɯtɯsqa-j\cmn 我们吃了麦子粥\end{exemple}
\begin{exemple}\jya ʑara kɤ-rɯtɯsqa rga-nɯ\cmn 他们喜欢吃麦子粥\end{exemple}
\begin{relation-sémantique}\confer{
\hyperlink{Ⓔtɯsqa}{\textit{ \papi{tɯsqa}}}
}\end{relation-sémantique}\end{entrée}

\begin{entrée}
\vedette{\hypertarget{Ⓔrɯtɯwɯ}{\papi{ rɯtɯwɯ}}}\markboth{rɯtɯwɯ}{}
\classe{vs}
\paradigme{\textit{dir :} \jya tɤ-}
\begin{définition}\fra être sur le point de pousser des épis (de l'orge)\end{définition}
\begin{définition}\cmn 快要抽穗(青稞)
\end{définition}
\begin{exemple}\jya tɤɕi to-rɯtɯwɯ\cmn 青稞快要抽穗\end{exemple}\end{entrée}

\begin{entrée}
\vedette{\hypertarget{Ⓔrɯxpa}{\papi{ rɯxpa}}}\markboth{rɯxpa}{}
\classe{n}
\begin{définition}\fra mémoire\end{définition}
\begin{définition}\cmn 记性
\begin{déclaration} \étymologie{\papi{rig.pa}}\end{déclaration}\end{définition}\end{entrée}

\begin{entrée}
\vedette{\hypertarget{Ⓔrɯxtuxti}{\papi{ rɯxtuxti}}}\markboth{rɯxtuxti}{}\classe{vt}
\paradigme{\textit{dir :} \jya tɤ-}
\begin{définition}\fra respecter\end{définition}
\begin{définition}\cmn 尊重;抬高;奉承\end{définition}
\begin{exemple}\jya tu-ta-rɯxtuxti ŋu nɤ!\cmn 我尊重你\end{exemple}\begin{sous-entrée}
\vedette{\hypertarget{}{\papi{ zrɯxtuxti}}}\markboth{zrɯxtuxti}{}
\begin{exemple}\jya tu-zrɯxtuxti-a zgɤt\cmn 我应该尊重他\end{exemple}
\end{sous-entrée}\begin{sous-entrée}
\vedette{\hypertarget{}{\papi{ ʑɣɤrɯxtuxti}}}\markboth{ʑɣɤrɯxtuxti}{}\classe{vi}
\begin{définition}\ 
\begin{déclaration}\grammar{refl}\end{déclaration}\end{définition}
\begin{définition}\fra être vaniteux\end{définition}
\begin{définition}\cmn 自大\end{définition}
\begin{relation-sémantique}\confer{
\hyperlink{Ⓔwxti}{\textit{ \papi{wxti}}}
}\end{relation-sémantique}
\end{sous-entrée}\end{entrée}

\begin{entrée}
\vedette{\hypertarget{Ⓔrɯχamba}{\papi{ rɯχamba}}}\markboth{rɯχamba}{}\classe{vs}
\begin{définition}\fra être présomptueux\end{définition}
\begin{définition}\cmn 骄傲;自大\end{définition}
\begin{sous-entrée}
\vedette{\hypertarget{}{\papi{ znɯχamba}}}\markboth{znɯχamba}{}\classe{vt}
\begin{définition}\fra agir de façon présomptueuse\end{définition}
\begin{définition}\cmn 骄傲;自大\end{définition}
\begin{exemple}\jya nɤ-kɤ-znɯχamba me\cmn 你没有什么理由骄傲\end{exemple}
\end{sous-entrée}\end{entrée}

\begin{entrée}
\vedette{\hypertarget{Ⓔrɯχɕɯχɕɤβ}{\papi{ rɯχɕɯχɕɤβ}}}\markboth{rɯχɕɯχɕɤβ}{}\classe{vi}
\paradigme{\textit{dir :} \jya tɤ-}
\begin{définition}\fra parler de façon exagérée\end{définition}
\begin{définition}\cmn (讲话)夸张\end{définition}
\begin{exemple}\jya nɤʑo ʁo tɯ-rɯχɕɯχɕɤβ ntsɯ ɕti nɤ\cmn 你讲话总是很夸张\end{exemple}
\begin{relation-sémantique}\confer{
\hyperlink{Ⓔχɕɤβ}{\textit{ \papi{χɕɤβ}}}
}\end{relation-sémantique}\end{entrée}

\begin{entrée}
\vedette{\hypertarget{Ⓔrɯχparɤβ}{\papi{ rɯχparɤβ}}}\markboth{rɯχparɤβ}{}\classe{vt}
\paradigme{\textit{dir :} \jya tɤ-}
\begin{définition}\fra se vanter\end{définition}
\begin{définition}\cmn 卖弄,夸耀自己\end{définition}
\begin{relation-sémantique}\synonyme{
\hyperlink{Ⓔznɤchacha}{\textit{ \papi{znɤchacha}}}
}\end{relation-sémantique}
\end{entrée}

\begin{entrée}
\vedette{\hypertarget{Ⓔrɯχtɕɯrɯ}{\papi{ rɯχtɕɯrɯ}}}\markboth{rɯχtɕɯrɯ}{}\classe{vs}
\begin{définition}\fra tout nu\end{définition}
\begin{définition}\cmn 裸体;光着身子
\begin{déclaration} \étymologie{\papi{gtɕer.bu}}\end{déclaration}\end{définition}
\begin{relation-sémantique}\confer{
\hyperlink{Ⓔχtɕɯrɯpa}{\textit{ \papi{χtɕɯrɯpa}}}
}\end{relation-sémantique}\end{entrée}

\begin{entrée}
\vedette{\hypertarget{Ⓔrɯχtsɯχtso}{\papi{ rɯχtsɯχtso}}}\markboth{rɯχtsɯχtso}{}
\classe{vi}
\begin{définition}\fra aimer la propreté\end{définition}
\begin{définition}\cmn 爱干净(有洁癖)\end{définition}
\begin{exemple}\jya ɲɯ-tɯ-rɯχtsɯχtso\cmn 你爱干净\end{exemple}
\begin{exemple}\jya mɯ́j-tɯ-rɯχtsɯχtso\cmn 你不爱干净\end{exemple}
\begin{exemple}\jya ɲɯ-tɯ-rɯχtsɯχtso tɕe, aʑo a-@beibei mɯ́j-tɯ-ntɕhoz (tɯrme ɯ-@beibei mɯ́j-tɯ-ntɕhoz)\cmn 你爱干净,所以不愿意用我的杯子(你不用别人的杯子)\end{exemple}
\begin{relation-sémantique}\confer{
\hyperlink{Ⓔχtso}{\textit{ \papi{χtso}}}
}\end{relation-sémantique}\end{entrée}

\begin{entrée}
\vedette{\hypertarget{Ⓔrɯz}{\papi{ rɯz}}}\markboth{rɯz}{}\classe{vs}
\paradigme{\textit{dir :} \jya pɯ-}
\begin{définition}\fra vrai\end{définition}
\begin{définition}\cmn 真实
\begin{déclaration} \étymologie{\papi{rigs}}\end{déclaration}\end{définition}
\begin{exemple}\jya tɯ-mɯ ɲɯ-ɤsɯ-lɤt ɲɯ-ti-nɯ, ɕ-thɯ-nɤrɯra-a ri, ɲɯ-rɯz\cmn 我听说外面在下雨,我去看了一下是真的\end{exemple}\end{entrée}

\begin{entrée}
\vedette{\hypertarget{Ⓔrɯzdɯzdɯɣ}{\papi{ rɯzdɯzdɯɣ}}}\markboth{rɯzdɯzdɯɣ}{}\classe{vi}
\paradigme{\textit{dir :} \jya tɤ-}
\begin{définition}\fra raconter ses malheurs\end{définition}
\begin{définition}\cmn 诉苦\end{définition}
\begin{exemple}\jya ɯʑo a-phe ɲɯ-rɯzdɯzdɯɣ\cmn 他向我诉苦\end{exemple}
\begin{exemple}\jya ɯʑo pjɤ-nɤɴqa tɕe pɯ-rɯzdɯzdɯɣ ntsɯ\cmn 因为觉得辛苦,他不停地诉苦\end{exemple}\end{entrée}

\begin{entrée}
\vedette{\hypertarget{Ⓔrwa}{\papi{ rwa}}}\markboth{rwa}{}
\classe{n}
\begin{définition}\fra tente de nomade en poil de yak\end{définition}
\begin{définition}\cmn 牧民的帐篷(毛制成的)
\begin{déclaration} \étymologie{\papi{sbra.ba}}\end{déclaration}\end{définition}\end{entrée}

\begin{entrée}
\vedette{\hypertarget{Ⓔrwɤt}{\papi{ rwɤt}}}\markboth{rwɤt}{}\classe{vt}
\paradigme{\textit{dir :} \jya pɯ-}
\paradigme{\textit{dir :} \jya lɤ-}
\begin{définition}\fra creuser\end{définition}
\begin{définition}\cmn 挖\end{définition}
\begin{exemple}\jya ŋgɤm lɤ-rwat-a\cmn 我挖了土墙\end{exemple}\begin{sous-entrée}
\vedette{\hypertarget{}{\papi{ sɯrwɤt}}}\markboth{sɯrwɤt}{}
\begin{exemple}\jya qaʁ kɯ sɤtɕha pɯ-sɯrwat-a\cmn 我用锄头挖了地\end{exemple}
\begin{relation-sémantique}\synonyme{
\hyperlink{Ⓔlɣa}{\textit{ \papi{lɣa}}}
}\end{relation-sémantique}
\end{sous-entrée}\end{entrée}

\begin{entrée}
\vedette{\hypertarget{Ⓔrwoʁrwoʁ}{\papi{ rwoʁrwoʁ}}}\markboth{rwoʁrwoʁ}{}\classe{idph.2}
\begin{définition}\fra plein de petites boules ou de petits morceaux de même taille\end{définition}
\begin{définition}\cmn 很多小球或大小一致的小块\end{définition}
\begin{exemple}\jya staχpɯ rwoʁrwoʁ ʑo ɲɯ-pa\cmn 豌豆是圆的\end{exemple}
\begin{exemple}\jya tɤ-pɤtso kɯxtɕɯxtɕi rwoʁrwoʁ ɲɯ-ɤʑɯrja-nɯ\cmn 小孩子在排队\end{exemple}
\begin{relation-sémantique}\confer{
\hyperlink{Ⓔrloʁrloʁ}{\textit{ \papi{rloʁrloʁ}}}
}\end{relation-sémantique}
\begin{relation-sémantique}\confer{
\hyperlink{Ⓔrloŋrloŋ}{\textit{ \papi{rloŋrloŋ}}}
}\end{relation-sémantique}
\begin{relation-sémantique}\confer{
\hyperlink{Ⓔrlaŋrlaŋ}{\textit{ \papi{rlaŋrlaŋ}}}
}\end{relation-sémantique}\end{entrée}

\begin{entrée}
\vedette{\hypertarget{Ⓔrzɤβrzɤβ}{\papi{ rzɤβrzɤβ}}}\markboth{rzɤβrzɤβ}{}\classe{idph.2}
\begin{définition}\fra flou\end{définition}
\begin{définition}\cmn 模模糊糊,不清楚\end{définition}
\begin{exemple}\jya χɕɤlmɯɣ a-pɯ-me tɕe, tɤscoz ra rzɤβrzɤβ ʑo ɲɯ-pa tɕe mɯ́j-sɯχsal-a\cmn 没有眼镜的话,文字模模糊糊,我根本看不清楚\end{exemple}
\begin{exemple}\jya zdɯm lɤ-kɯ-ɣe ʑo rzɤβrzɤβ ɲɯ-fse\cmn 好像起了雾一样,模模糊糊地看不清\end{exemple}\begin{sous-entrée}
\vedette{\hypertarget{}{\papi{ ɣɤrzɤβrzɤβ}}}\markboth{ɣɤrzɤβrzɤβ}{}
\begin{définition}\fra être flou\end{définition}
\begin{définition}\cmn 模模糊糊,不清楚\end{définition}
\begin{exemple}\jya a-mɲaʁndo ɲɯ-ɣɤrzɤβrzɤβ tɕe mɯ́j-sɯχsal-a\cmn 我眼边很模糊,看不清楚\end{exemple}
\end{sous-entrée}\end{entrée}

\begin{entrée}
\vedette{\hypertarget{Ⓔrzoŋ}{\papi{ rzoŋ}}}\markboth{rzoŋ}{}
\classe{vt}
\paradigme{\textit{dir :} \jya thɯ-}
\begin{définition}\fra mettre dedans\end{définition}
\begin{définition}\cmn 往里装
\begin{déclaration} \étymologie{\papi{rdzoŋ}}\end{déclaration}\end{définition}
\begin{exemple}\jya thɯ-rzoŋ-a\cmn 我装了\end{exemple}
\begin{exemple}\jya tha-rzoŋ\cmn 他装了\end{exemple}
\begin{exemple}\jya ɕɤmɯɣdɯ thɯ-rzoŋ\cmn 你给枪装子弹吧\end{exemple}\end{entrée}

\begin{entrée}
\vedette{\hypertarget{Ⓔrzoŋlu}{\papi{ rzoŋlu}}}\markboth{rzoŋlu}{}\classe{idph}
\begin{définition}\fra très occupé\end{définition}
\begin{définition}\cmn 忙得不可开交\end{définition}
\begin{exemple}\jya a-ma ɯ-tɯ-dɤn kɯ rzoŋlu ʑo ɲɯ-xtsu\cmn 我事情很多,忙得不可开交\end{exemple}\end{entrée}

\begin{entrée}
\vedette{\hypertarget{Ⓔrzoŋwa}{\papi{ rzoŋwa}}}\markboth{rzoŋwa}{}\classe{n}
\begin{définition}\fra dot\end{définition}
\begin{définition}\cmn 嫁妆\end{définition}\end{entrée}

\begin{entrée}
\vedette{\hypertarget{Ⓔrzoʁ}{\papi{ rzoʁ}}}\markboth{rzoʁ}{}\classe{vi}
\begin{définition}\fra pousser (complète)\end{définition}
\begin{définition}\cmn (完全)长出了
\begin{déclaration} \étymologie{\papi{rdzogs}}\end{déclaration}\end{définition}
\begin{exemple}\jya lɯtoʁ ɲɤ-rzoʁ\cmn (春天的时候),草木复苏\end{exemple}
\begin{exemple}\jya a-ʁɲɤlwa nɯ-rzoʁ\cmn 我受了很多苦\end{exemple}
\begin{exemple}\jya tɤ-rɤku ra ɲɤ-rzoʁ\cmn 庄稼全部生长起来了\end{exemple}
\begin{exemple}\jya ta-ma ɲɯ-rzoʁ ʑo ɕti\cmn 很多事情同一个时间堆在一起,无从下手\end{exemple}\end{entrée}

\begin{entrée}
\vedette{\hypertarget{Ⓔrzɯɴɢaʁ}{\papi{ rzɯɴɢaʁ}}}\markboth{rzɯɴɢaʁ}{}\classe{n}
\begin{définition}\ 
\begin{déclaration}\grammar{n.lieu}\end{déclaration}\end{définition}
\begin{définition}\fra Rdzong.'gag\end{définition}
\begin{définition}\cmn 松岗乡\end{définition}\end{entrée}

\begin{entrée}
\vedette{\hypertarget{Ⓔrzɯrzi}{\papi{ rzɯrzi}}}\markboth{rzɯrzi}{} (\variante{rdzɯrdzi}) \classe{idph.2}\acception{1}
\begin{définition}\fra inquiet\end{définition}
\begin{définition}\cmn 形容担心,放心不下的状态\end{définition}
\begin{exemple}\jya rdzɯrdzi ɲɯ-mu-a\cmn 我害怕\end{exemple}
\begin{exemple}\jya rzɯrzi ɲɯ-mu-a\cmn 我害怕\end{exemple}\acception{2}
\begin{définition}\fra frais (temps)\end{définition}
\begin{définition}\cmn 冷飕飕\end{définition}
\begin{exemple}\jya jisŋi tɯ-mɯ rzɯrzi ci ɲɯ-ŋu, ɲɯ-ɣɤndʐo\cmn 今天冷飕飕的\end{exemple}
\begin{relation-sémantique}\confer{
\hyperlink{Ⓔrdzɯrdzi}{\textit{ \papi{rdzɯrdzi}}}
}\end{relation-sémantique}\end{entrée}

\begin{entrée}
\vedette{\hypertarget{Ⓔrʑaʁ}{\papi{ rʑaʁ}}}\markboth{rʑaʁ}{}
\classe{vi}
\paradigme{\textit{dir :} \jya tɤ-}
\begin{définition}\fra se passer un certain nombre de jours\end{définition}
\begin{définition}\cmn 过几天\end{définition}
\begin{exemple}\jya χsɯm to-rʑaʁ\cmn 过了三天\end{exemple}
\begin{relation-sémantique}\confer{
 \papi{tɤ-rʑaʁ1}
}\end{relation-sémantique}\end{entrée}

\begin{entrée}
\vedette{\hypertarget{Ⓔrʑaʁtɕhɤt}{\papi{ rʑaʁtɕhɤt}}}\markboth{rʑaʁtɕhɤt}{}
\classe{n}
\begin{définition}\fra limite de temps\end{définition}
\begin{définition}\cmn 期限\end{définition}\end{entrée}

\begin{entrée}
\vedette{\hypertarget{Ⓔrʑi}{\papi{ rʑi}}}\markboth{rʑi}{}
\classe{vs}
\paradigme{\textit{dir :} \jya tɤ-}
\begin{définition}\fra lourd\end{définition}
\begin{définition}\cmn 重\end{définition}
\begin{exemple}\jya ɯ-fkur ɲɯ-rʑi\cmn 他背的东西很重\end{exemple}
\begin{exemple}\jya rdɤstaʁ ɲɯ-rʑi\cmn 石头很重\end{exemple}
\begin{relation-sémantique}\antonyme{
\hyperlink{ⒺʑoⒽ1}{\textit{ \papi{ʑo1}}}
}\end{relation-sémantique}\end{entrée}

\begin{entrée}
\vedette{\hypertarget{Ⓔrʑɯɣrʑɯɣ}{\papi{ rʑɯɣrʑɯɣ}}}\markboth{rʑɯɣrʑɯɣ}{}\classe{idph.2}
\begin{définition}\fra gros et lisse\end{définition}
\begin{définition}\cmn 形容横放在那里,又粗又光滑的样子\end{définition}
\begin{exemple}\jya lɤpɯɣ ɯ-tɯ-wxti kɯ rʑɯɣrʑɯɣ ʑo ɲɯ-pa\cmn 萝卜放在那里又粗又光滑\end{exemple}
\begin{sous-entrée}
\vedette{\hypertarget{}{\papi{ ɣɤrʑɯɣlɯɣ}}}\markboth{ɣɤrʑɯɣlɯɣ}{}\classe{vi}
\begin{définition}\fra serpenter, ramper en se tortillant\end{définition}
\begin{définition}\cmn 蜿蜒爬行\end{définition}
\begin{définition}\fra qapri ɲɯ-ɣɤrʑɯɣlɯɣ ʑo thɯ-ɣe\end{définition}
\begin{définition}\cmn 蛇爬下来了\end{définition}
\end{sous-entrée}\begin{sous-entrée}
\vedette{\hypertarget{}{\papi{ rʑɯɣnɤlɯɣ}}}\markboth{rʑɯɣnɤlɯɣ}{}
\begin{définition}\fra qui rampe en se tortillant\end{définition}
\begin{définition}\cmn 形容蜿蜒爬行的样子\end{définition}
\begin{exemple}\jya qapri nɯ rʑɯɣnɤlɯɣ ʑo jɤ-ari\cmn 蛇蜿蜒爬行的样子\end{exemple}
\end{sous-entrée}\end{entrée}

\newpage\caractère{ʁ}

\begin{entrée}
\vedette{\hypertarget{Ⓔʁarphɤβ}{\papi{ ʁarphɤβ}}}\markboth{ʁarphɤβ}{}\classe{n}
\begin{définition}\fra battement d'ailes\end{définition}
\begin{définition}\cmn 拍翅膀\end{définition}
\begin{exemple}\jya qaliaʁ kɯ ʁarphɤβ pjɤ-lɤt\cmn 老鹰拍了翅膀\end{exemple}
\begin{relation-sémantique}\confer{
\hyperlink{Ⓔtɤ-ʁar}{\textit{ \papi{tɤ-ʁar}}}
}\end{relation-sémantique}
\begin{relation-sémantique}\confer{
\hyperlink{Ⓔnɤʁarphɤβ}{\textit{ \papi{nɤʁarphɤβ}}}
}\end{relation-sémantique}
\begin{relation-sémantique}\confer{
\hyperlink{Ⓔɣɤrphɤrphɤβ}{\textit{ \papi{ɣɤrphɤrphɤβ}}}
}\end{relation-sémantique}\end{entrée}

\begin{entrée}
\vedette{\hypertarget{Ⓔʁaʁ}{\papi{ ʁaʁ}}}\markboth{ʁaʁ}{}\classe{vi}
\paradigme{\textit{dir :} \jya nɯ-}\acception{1}
\begin{définition}\fra éclore\end{définition}
\begin{définition}\cmn 孵出来\end{définition}
\begin{exemple}\jya tɤ-ŋgɯm ɲɤ-ʁaʁ\cmn 蛋孵出来了\end{exemple}\acception{2}
\begin{définition}\fra s'épanouir (fleur)\end{définition}
\begin{définition}\cmn 开花\end{définition}
\begin{exemple}\jya mɯntoʁ ɲo-ʁaʁ\cmn 花开了\end{exemple}\begin{sous-entrée}
\vedette{\hypertarget{}{\papi{ sɯɣʁaʁ}}}\markboth{sɯɣʁaʁ}{} (\variante{sɯʁaʁ}) \classe{vt}
\paradigme{\textit{dir :} \jya nɯ-}
\begin{définition}\fra faire éclore\end{définition}
\begin{définition}\cmn 使孵出来;使开花\end{définition}
\begin{exemple}\jya kumpɣa ɯ-ŋgɯm nɯ-sɯɣʁaʁ-a\cmn 我令鸡蛋孵化(用人工的手段)\end{exemple}
\end{sous-entrée}\end{entrée}

\begin{entrée}
\vedette{\hypertarget{Ⓔʁatɯl}{\papi{ ʁatɯl}}}\markboth{ʁatɯl}{}\classe{n}
\begin{définition}\fra habit en peau de renard\end{définition}
\begin{définition}\cmn 狐狸皮皮袄
\begin{déclaration} \étymologie{\papi{wa.dol}}\end{déclaration}\end{définition}\end{entrée}

\begin{entrée}
\vedette{\hypertarget{Ⓔʁaz}{\papi{ ʁaz}}}\markboth{ʁaz}{}
\begin{relation-sémantique}\confer{
\hyperlink{Ⓔʁaznɤ}{\textit{ \papi{ʁaznɤ}}}
}\end{relation-sémantique}\end{entrée}

\begin{entrée}
\vedette{\hypertarget{Ⓔʁaznɤ}{\papi{ ʁaznɤ}}}\markboth{ʁaznɤ}{}
\classe{adv}
\begin{définition}\fra profiter de l'occasion\end{définition}
\begin{définition}\cmn 趁……的机会\end{définition}
\begin{exemple}\jya tɕiʑo ʁna ɣɤʑu-tɕi ʁaznɤ nɯkrɤz-tɕi\cmn 要趁我们俩都在的时候商量\end{exemple}
\begin{exemple}\jya nɤ-tɤ-lu ɲɯ-sɤɕke ʁaznɤ kɤ-tshi, tɕe a-mɤ-nɯmɯɕtaʁ\cmn 牛奶要趁热喝,不要放冷了\end{exemple}
\begin{exemple}\jya sɤɕke ʁaz tɤ-ndze\cmn 趁热吃吧\end{exemple}\end{entrée}

\begin{entrée}
\vedette{\hypertarget{Ⓔʁbɤβʁbɤβ}{\papi{ ʁbɤβʁbɤβ}}}\markboth{ʁbɤβʁbɤβ}{}\classe{idph.2}
\begin{définition}\fra épais et gros\end{définition}
\begin{définition}\cmn 形容厚而大的样子\end{définition}
\begin{exemple}\jya jiɕqha tɯrme nɯ ɯ-rŋa ra ɲɯ-tshu ʑo ʁbɤʁbɤβ\cmn 这个人的脸又粗又胖\end{exemple}\end{entrée}

\begin{entrée}
\vedette{\hypertarget{Ⓔʁdɤnba}{\papi{ ʁdɤnba}}}\markboth{ʁdɤnba}{}\classe{n}
\begin{définition}\fra capital\end{définition}
\begin{définition}\cmn 本钱\end{définition}
\begin{exemple}\jya kɯki nɤ-tɯtsɣe ɯ-ʁdɤnba a-pɯ-ŋu\cmn 把这个用来做你做生意的本钱\end{exemple}\end{entrée}

\begin{entrée}
\vedette{\hypertarget{Ⓔʁdɤʁdɤt}{\papi{ ʁdɤʁdɤt}}}\markboth{ʁdɤʁdɤt}{}\classe{idph.2}
\begin{définition}\fra rectangulaire, solide\end{définition}
\begin{définition}\cmn 正方形、结实的样子\end{définition}
\begin{exemple}\jya rgɤm ʁdɤrdɤt ʑo ɲɯ-pa\cmn 盒子又小又是正方形的,很结实的样子\end{exemple}
\begin{relation-sémantique}\synonyme{
\hyperlink{Ⓔɣdɤɣdɤt}{\textit{ \papi{ɣdɤɣdɤt}}}
}\end{relation-sémantique}\end{entrée}

\begin{entrée}
\vedette{\hypertarget{Ⓔʁdɯβʁdɯβ}{\papi{ ʁdɯβʁdɯβ}}}\markboth{ʁdɯβʁdɯβ}{}
\classe{idph.2}
\begin{définition}\fra rectangulaire\end{définition}
\begin{définition}\cmn 形容四四方方的形状(手上拿得起的东西)\end{définition}
\begin{exemple}\jya @larou ʁdɯβʁdɯβ ci nɯ́-wɣ-mbi-a\cmn 他把腊肉给我了\end{exemple}
\begin{exemple}\jya tɤ-ŋgɤr ci ʁdɯβʁdɯβ pjɤ-phɯt\cmn 他割下一大块(四四方方的)猪膘\end{exemple}
\begin{exemple}\jya ɯ-phoŋbu ʁdɯβʁdɯβ ʑo ɲɯ-pa\cmn 他身子矮胖\end{exemple}\end{entrée}

\begin{entrée}
\vedette{\hypertarget{ⒺʁdɯɣⒽ2}{\papi{ ʁdɯɣ}}}\markboth{ʁdɯɣ}{}\homonyme{2}\classe{n}
\begin{définition}\fra parasol (réservé aux sprul sku)\end{définition}
\begin{définition}\cmn 华盖(活佛专用)
\begin{déclaration} \étymologie{\papi{gdug}}\end{déclaration}\end{définition}\end{entrée}

\begin{entrée}
\vedette{\hypertarget{ⒺʁdɯɣⒽ1}{\papi{ ʁdɯɣ}}}\markboth{ʁdɯɣ}{}\homonyme{1}
\classe{vi}
\paradigme{\textit{dir :} \jya tɤ-}
\begin{définition}\fra grave\end{définition}
\begin{définition}\cmn 有害\end{définition}
\begin{exemple}\jya nɯnɯ kɤ-ʁdɯɣ me\cmn 没有关系\end{exemple}
\begin{exemple}\jya tɕhomba mɯ́j-ʁdɯɣ\cmn 感冒不严重\end{exemple}
\begin{exemple}\jya nɯ mɤ-ʁdɯɣ\cmn 那个没有关系\end{exemple}
\begin{exemple}\jya smɤn a-tɤ-ndze tɕe mɯ́j-ʁdɯɣ\cmn 吃了药就没有关系\end{exemple}\end{entrée}

\begin{entrée}
\vedette{\hypertarget{Ⓔʁdɯn}{\papi{ ʁdɯn}}}\markboth{ʁdɯn}{}\classe{n}
\begin{définition}\fra malheur\end{définition}
\begin{définition}\cmn 邪恶
\begin{déclaration} \étymologie{\papi{gdon}}\end{déclaration}\end{définition}
\end{entrée}

\begin{entrée}
\vedette{\hypertarget{Ⓔʁdɯrɟɤt}{\papi{ ʁdɯrɟɤt}}}\markboth{ʁdɯrɟɤt}{}\classe{n}
\begin{définition}\ 
\begin{déclaration}\grammar{n.lieu}\end{déclaration}\end{définition}
\begin{définition}\fra Gdongbrgyad\end{définition}
\begin{définition}\cmn 龙尔甲乡\end{définition}
\end{entrée}

\begin{entrée}
\vedette{\hypertarget{Ⓔʁdɯrtoʁ}{\papi{ ʁdɯrtoʁ}}}\markboth{ʁdɯrtoʁ}{} (\variante{ʁdɯrto}) \classe{n}
\begin{définition}\fra espace entre les poutres du plafond et les poutres du toit\end{définition}
\begin{définition}\cmn 横梁和房背之间的平板\end{définition}
\begin{exemple}\jya ʁdɯrtoʁ nɯ komɤl ɯ-taʁ stukɤr ɯ-pa tɤrɤm kɯ-jaʁ ku-kɯ-ɕe nɯ ŋu\cmn 
\stylefv{ʁdɯrtoʁ} 是在横梁之上,大梁之下横过去的厚木板
\end{exemple}\end{entrée}

\begin{entrée}
\vedette{\hypertarget{Ⓔʁdɯrtsa}{\papi{ ʁdɯrtsa}}}\markboth{ʁdɯrtsa}{}
\classe{n}
\begin{définition}\fra amadou\end{définition}
\begin{définition}\cmn 火绒(把小麻皮捶绒)\end{définition}\end{entrée}

\begin{entrée}
\vedette{\hypertarget{Ⓔʁdɯskɤr}{\papi{ ʁdɯskɤr}}}\markboth{ʁdɯskɤr}{}\classe{n}
\begin{définition}\fra drapeau à prière\end{définition}
\begin{définition}\cmn 玛尼旗\end{définition}
\begin{relation-sémantique}\synonyme{
 \papi{tartɕin}
}\end{relation-sémantique}\end{entrée}

\begin{entrée}
\vedette{\hypertarget{Ⓔʁdɯxpa}{\papi{ ʁdɯxpa}}}\markboth{ʁdɯxpa}{}\classe{n}
\begin{définition}\fra empêchement, gêne\end{définition}
\begin{définition}\cmn 障碍;坏处\end{définition}
\begin{exemple}\jya kɤ-rɤma tɤ-ŋu tɕe, tɯʑo kɯ-rɯsɯso mɤ-βdi ma, tɯ-zda ra nɯ-ndaŋ kɯnɤ pjɯ́-wɣ-lɤt ra ma, nɯ-ʁdɯxpa ɣɯ-βzu mɤ-βdi\cmn 工作的时候,不要只按照自己的想法来,要考虑到别人,不要妨碍别人\end{exemple}
\begin{exemple}\jya a-ʁdɯxpa ta-βzu\cmn 他妨碍我了\end{exemple}
\begin{relation-sémantique}\confer{
\hyperlink{Ⓔrɯʁdɯxpa}{\textit{ \papi{rɯʁdɯxpa}}}
}\end{relation-sémantique}
\begin{relation-sémantique}\antonyme{
\hyperlink{Ⓔphɤnba}{\textit{ \papi{phɤnba}}}
}\end{relation-sémantique}\end{entrée}

\begin{entrée}
\vedette{\hypertarget{Ⓔʁe}{\papi{ ʁe}}}\markboth{ʁe}{}\classe{n}
\begin{définition}\fra gauche\end{définition}
\begin{définition}\cmn 左边\end{définition}\end{entrée}

\begin{entrée}
\vedette{\hypertarget{Ⓔʁejlu}{\papi{ ʁejlu}}}\markboth{ʁejlu}{}\classe{n}
\begin{définition}\fra gauche\end{définition}
\begin{définition}\cmn 左边\end{définition}
\begin{exemple}\jya ʁejlu tɤ-lat-a, tɤ-ntɕhoz-a\cmn 我用了左手\end{exemple}
\begin{relation-sémantique}\confer{
\hyperlink{Ⓔsɯʁejlu}{\textit{ \papi{sɯʁejlu}}}
}\end{relation-sémantique}\end{entrée}

\begin{entrée}
\vedette{\hypertarget{Ⓔʁejlɤɕkɤr}{\papi{ ʁejlɤɕkɤr}}}\markboth{ʁejlɤɕkɤr}{}
\classe{n}
\begin{définition}\fra gaucher\end{définition}
\begin{définition}\cmn 左撇子(贬义)\end{définition}\end{entrée}

\begin{entrée}
\vedette{\hypertarget{Ⓔʁgɤskɯ}{\papi{ ʁgɤskɯ}}}\markboth{ʁgɤskɯ}{}
\classe{n}
\begin{définition}\fra surveillant dans le monastère\end{définition}
\begin{définition}\cmn 负责惩罚违律行为的和尚
\begin{déclaration} \étymologie{\papi{*dge.sku}}\end{déclaration}\end{définition}\end{entrée}

\begin{entrée}
\vedette{\hypertarget{Ⓔʁgɤsloŋ}{\papi{ ʁgɤsloŋ}}}\markboth{ʁgɤsloŋ}{}\classe{n}
\begin{définition}\fra bhiksu\end{définition}
\begin{définition}\cmn 比丘
\begin{déclaration} \étymologie{\papi{dge.sloŋ}}\end{déclaration}\end{définition}
\end{entrée}

\begin{entrée}
\vedette{\hypertarget{Ⓔʁgiwa}{\papi{ ʁgiwa}}}\markboth{ʁgiwa}{}\classe{n}
\begin{définition}\fra récitation de mantras\end{définition}
\begin{définition}\cmn 念经
\begin{déclaration} \étymologie{\papi{dge.ba}}\end{déclaration}\end{définition}
\begin{exemple}\jya ʁgiwa tɤ-βzu-j\cmn 我们请人念经了\end{exemple}
\begin{relation-sémantique}\confer{
\hyperlink{Ⓔrɯʁgiwa}{\textit{ \papi{rɯʁgiwa}}}
}\end{relation-sémantique}\end{entrée}

\begin{entrée}
\vedette{\hypertarget{Ⓔʁgra}{\papi{ ʁgra}}}\markboth{ʁgra}{}\classe{n}
\begin{définition}\fra ennemi\end{définition}
\begin{définition}\cmn 敌人\end{définition}
\begin{exemple}\jya tɯ-ʁgra nɯ tɯ-mɲaʁrme ma a-pɯ-me ra, tɯ-βzaŋsa nɯ tɯ-kɤrme jamar a-pɯ-dɤn ra\cmn 敌人要少得像眉毛上的毛,朋友要多得像头上的头发\end{exemple}
\begin{relation-sémantique}\antonyme{
\hyperlink{Ⓔtɯ-ɣɲi}{\textit{ \papi{tɯ-ɣɲi}}}
}\end{relation-sémantique}
\begin{relation-sémantique}\confer{
\hyperlink{Ⓔʁgraja}{\textit{ \papi{ʁgraja}}}
}\end{relation-sémantique}\end{entrée}

\begin{entrée}
\vedette{\hypertarget{Ⓔʁgraja}{\papi{ ʁgraja}}}\markboth{ʁgraja}{}
\classe{n}
\begin{définition}\fra ennemi\end{définition}
\begin{définition}\cmn 仇人
\begin{déclaration} \étymologie{\papi{dgra.ja}}\end{déclaration}\end{définition}\end{entrée}

\begin{entrée}
\vedette{\hypertarget{Ⓔʁgusloŋ}{\papi{ ʁgusloŋ}}}\markboth{ʁgusloŋ}{}\classe{n}
\begin{définition}\fra type de moine\end{définition}
\begin{définition}\cmn 和尚的一种\end{définition}\end{entrée}

\begin{entrée}
\vedette{\hypertarget{ⒺʁjuⒽ1}{\papi{ ʁju}}}\markboth{ʁju}{}\homonyme{1}
\classe{n}
\begin{définition}\fra saucisson au bœuf\end{définition}
\begin{définition}\cmn 牛肉香肠\end{définition}\end{entrée}

\begin{entrée}
\vedette{\hypertarget{ⒺʁjuⒽ2}{\papi{ ʁju}}}\markboth{ʁju}{}\homonyme{2}\classe{n}
\begin{définition}\fra turquoise\end{définition}
\begin{définition}\cmn 碧玉;绿松石
\begin{déclaration}\use{古语}\end{déclaration}
\begin{déclaration} \étymologie{\papi{gju}}\end{déclaration}\end{définition}
\begin{relation-sémantique}\synonyme{
\hyperlink{Ⓔmti}{\textit{ \papi{mti}}}
}\end{relation-sémantique}\end{entrée}

\begin{entrée}
\vedette{\hypertarget{Ⓔʁja}{\papi{ ʁja}}}\markboth{ʁja}{}
\classe{n}
\begin{définition}\fra vert-de-gris\end{définition}
\begin{définition}\cmn 铜锈
\begin{déclaration} \étymologie{\papi{gja}}\end{déclaration}\end{définition}\end{entrée}

\begin{entrée}
\vedette{\hypertarget{Ⓔʁjaŋ}{\papi{ ʁjaŋ}}}\markboth{ʁjaŋ}{}
\classe{n}
\begin{définition}\fra bon présage, bonne fortune\end{définition}
\begin{définition}\cmn 吉祥
\begin{déclaration} \étymologie{\papi{gjaŋ}}\end{déclaration}\end{définition}
\begin{exemple}\jya kha ɯ-ŋgɯ ʁjaŋ a-pɯ-tu ra\cmn 一个家庭需要吉祥\end{exemple}\end{entrée}

\begin{entrée}
\vedette{\hypertarget{Ⓔʁjaŋsɯ}{\papi{ ʁjaŋsɯ}}}\markboth{ʁjaŋsɯ}{}\classe{n}
\begin{définition}\fra feutre\end{définition}
\begin{définition}\cmn 毡子\end{définition}
\begin{exemple}\jya ʁjaŋsɯ kɤ-sɯta ra\cmn 要擀毡子\end{exemple}\end{entrée}

\begin{entrée}
\vedette{\hypertarget{Ⓔʁjaŋtʂoŋ}{\papi{ ʁjaŋtʂoŋ}}}\markboth{ʁjaŋtʂoŋ}{}
\classe{n}
\begin{définition}\fra svastika\end{définition}
\begin{définition}\cmn 卍字
\begin{déclaration} \étymologie{\papi{gjung.druŋ}}\end{déclaration}\end{définition}\end{entrée}

\begin{entrée}
\vedette{\hypertarget{Ⓔʁjɤr}{\papi{ ʁjɤr}}}\markboth{ʁjɤr}{}
\classe{vs}
\paradigme{\textit{dir :} \jya tɤ-}
\begin{définition}\fra pousser dru\end{définition}
\begin{définition}\cmn 茂盛\end{définition}
\begin{exemple}\jya ɯ-muj ɲɯ-ʁjɤr\cmn 它的羽毛长得很密\end{exemple}\end{entrée}

\begin{entrée}
\vedette{\hypertarget{Ⓔʁjɤrsa}{\papi{ ʁjɤrsa}}}\markboth{ʁjɤrsa}{}
\classe{n}
\begin{définition}\fra pâturage d'été\end{définition}
\begin{définition}\cmn 夏天的牧场
\begin{déclaration} \étymologie{\papi{dbʲar.sa}}\end{déclaration}\end{définition}\end{entrée}

\begin{entrée}
\vedette{\hypertarget{Ⓔʁjɤʁjɤβ}{\papi{ ʁjɤʁjɤβ}}}\markboth{ʁjɤʁjɤβ}{}\classe{idph.2}
\begin{définition}\fra peu foncée (couleur)\end{définition}
\begin{définition}\cmn 形容颜色不深\end{définition}
\begin{relation-sémantique}\confer{
\hyperlink{Ⓔʁjɯʁji}{\textit{ \papi{ʁjɯʁji}}}
}\end{relation-sémantique}\end{entrée}

\begin{entrée}
\vedette{\hypertarget{Ⓔʁjit}{\papi{ ʁjit}}}\markboth{ʁjit}{}
\classe{vt}
\paradigme{\textit{dir :} \jya tɤ-}
\begin{définition}\fra se souvenir, manquer à\end{définition}
\begin{définition}\cmn 想起;想念\end{définition}
\begin{exemple}\jya jiɕqha nɯ kɯ tɤ́-wɣ-ʁjit-a\cmn 他想起我了\end{exemple}
\begin{exemple}\jya ɯ-rʑaβ ɲɯ-ʁjit\cmn 他想起他的妻子\end{exemple}
\begin{exemple}\jya ɯʑo pɯ-kɤ-βzu ra tɤ-ʁjit-a\cmn 我想起他所做的事情\end{exemple}
\begin{exemple}\jya aʑo tɤ-ta-ʁjit\cmn 我想起你了\end{exemple}\begin{sous-entrée}
\vedette{\hypertarget{}{\papi{ ɯ-sɤʁjɯʁjit}}}\markboth{ɯ-sɤʁjɯʁjit}{}
\begin{définition}\fra rappeller\end{définition}
\begin{définition}\cmn 提醒\end{définition}
\begin{exemple}\jya a-sɤʁjɯʁjit tu-tɯ-βze ɲɯ-ŋu\cmn 你在提醒我\end{exemple}
\end{sous-entrée}\end{entrée}

\begin{entrée}
\vedette{\hypertarget{Ⓔʁjitpa}{\papi{ ʁjitpa}}}\markboth{ʁjitpa}{}\classe{n}
\begin{définition}\fra assurance\end{définition}
\begin{définition}\cmn 心里清楚\end{définition}
\begin{exemple}\jya tɯ-ti mɤ-ra ma a-ʁjitpa tu\cmn 不用你说,我心里很清楚\end{exemple}
\begin{exemple}\jya kɯki sɤtɕha ki nɤ-ʁjitpa ɯ́-tu?\cmn 你熟悉这个地方吗\end{exemple}\end{entrée}

\begin{entrée}
\vedette{\hypertarget{Ⓔʁjoʁ}{\papi{ ʁjoʁ}}}\markboth{ʁjoʁ}{}
\classe{n}
\begin{définition}\fra serviteur\end{définition}
\begin{définition}\cmn 仆人
\begin{déclaration} \étymologie{\papi{gjog}}\end{déclaration}\end{définition}\begin{sous-entrée}
\vedette{\hypertarget{}{\papi{ tɤ-ʁjoʁ}}}\markboth{tɤ-ʁjoʁ}{}\classe{np}
\begin{définition}\fra serviteur de\end{définition}
\begin{définition}\cmn 仆人\end{définition}
\end{sous-entrée}\end{entrée}

\begin{entrée}
\vedette{\hypertarget{Ⓔʁjɯβkɯ}{\papi{ ʁjɯβkɯ}}}\markboth{ʁjɯβkɯ}{}\classe{adv}
\begin{définition}\fra qui ressemble un peu\end{définition}
\begin{définition}\cmn 仿佛像\end{définition}
\begin{exemple}\jya ʁjɯβkɯ ɯ-mu tsa ɲɯ-fse\cmn 他有点像他母亲\end{exemple}\end{entrée}

\begin{entrée}
\vedette{\hypertarget{Ⓔʁjɯβtshɤt}{\papi{ ʁjɯβtshɤt}}}\markboth{ʁjɯβtshɤt}{}
\classe{n}
\begin{définition}\fra estimation\end{définition}
\begin{définition}\cmn 估计\end{définition}
\begin{exemple}\jya fso tɯjpu fsusqi-tɯrpa wɣɯ́-kho ɲɯ-ra ri, ʁjɯβtshɤt ci tɤ-rku-t-a\cmn 明天要交三十斤粮食,我大概估计了一下就装了\end{exemple}
\begin{exemple}\jya ɯ-tɯ-zri @liangmi ɲɯ-ra tɕe ʁjɯβtshɤt ci pɯ-ʁndzar-a\cmn 需要两米长(的木料),我大概估计了一下就锯了\end{exemple}
\begin{relation-sémantique}\confer{
\hyperlink{Ⓔnɯʁjɯβtshɤt}{\textit{ \papi{nɯʁjɯβtshɤt}}}
}\end{relation-sémantique}\end{entrée}

\begin{entrée}
\vedette{\hypertarget{Ⓔʁjɯmbrɯɣ}{\papi{ ʁjɯmbrɯɣ}}}\markboth{ʁjɯmbrɯɣ}{}\classe{n}
\begin{définition}\fra dragon\end{définition}
\begin{définition}\cmn 龙
\begin{déclaration} \étymologie{\papi{ⁿbrug}}\end{déclaration}\end{définition}\end{entrée}

\begin{entrée}
\vedette{\hypertarget{Ⓔʁjɯmbrɯɣma}{\papi{ ʁjɯmbrɯɣma}}}\markboth{ʁjɯmbrɯɣma}{}\classe{n}
\begin{définition}\fra bol\end{définition}
\begin{définition}\cmn 瓷碗,画着龙形状的图案
\begin{déclaration} \étymologie{\papi{ⁿbrug.ma}}\end{déclaration}\end{définition}
\end{entrée}

\begin{entrée}
\vedette{\hypertarget{Ⓔʁjɯʁji}{\papi{ ʁjɯʁji}}}\markboth{ʁjɯʁji}{}
\classe{idph.2}\acception{1}
\begin{définition}\fra peu foncé\end{définition}
\begin{définition}\cmn 形容颜色不深
\end{définition}\acception{2}
\begin{définition}\fra se sentant un peu mal\end{définition}
\begin{définition}\cmn 形容有点不舒服的感觉\end{définition}
\begin{exemple}\jya a-ʑi ci ʁjɯʁji ɲɯ-loʁ\cmn 我有点想吐\end{exemple}\acception{3}
\begin{définition}\fra air de chien battu\end{définition}
\begin{définition}\cmn 形容受委屈的样子\end{définition}
\begin{exemple}\jya khɯna kɯ ɯ-jme pjɤ-ɕɯɴqoʁ tɕe ʁjɯʁji ʑo to-ʑɣɤstu\cmn 狗夹起尾巴做出了一副很委屈的样子\end{exemple}
\begin{exemple}\jya ɯ-wa kɯ tó-wɣ-nɤmqe tɕe ʁjɯʁji ʑo to-ʑɣɤstu\cmn 他被父亲骂了以后做出了一副受委屈的样子\end{exemple}
\begin{relation-sémantique}\confer{
\hyperlink{Ⓔʁjɤʁjɤβ}{\textit{ \papi{ʁjɤʁjɤβ}}}
}\end{relation-sémantique}\end{entrée}

\begin{entrée}
\vedette{\hypertarget{Ⓔʁɟa}{\papi{ ʁɟa}}}\markboth{ʁɟa}{}
\classe{adv}\acception{1}
\begin{définition}\fra seulement\end{définition}
\begin{définition}\cmn 光是\end{définition}
\begin{exemple}\jya jima ʁɟa ʑo lu-ji-nɯ\cmn 他们种的都是玉米\end{exemple}\acception{2}
\begin{définition}\fra tout le temps\end{définition}
\begin{définition}\cmn 一直;总是\end{définition}
\begin{exemple}\jya @kaihui kɯ-fse tɤ-ra tɕe, kɤ-ŋke ʁɟa ʑo ju-kɯ-ɕe pɯ-ra\cmn (以前交通不方便),要开会的时候,一直是走路去的\end{exemple}
\begin{relation-sémantique}\confer{
\hyperlink{Ⓔaʁɟa}{\textit{ \papi{aʁɟa}}}
}\end{relation-sémantique}\end{entrée}

\begin{entrée}
\vedette{\hypertarget{Ⓔʁɟo}{\papi{ ʁɟo}}}\markboth{ʁɟo}{}
\classe{vt}
\paradigme{\textit{dir :} \jya nɯ-}
\begin{définition}\fra rincer\end{définition}
\begin{définition}\cmn 冲水\end{définition}
\begin{exemple}\jya khɯtsa nɯ-ʁɟɤm\cmn 你把碗冲一下\end{exemple}
\begin{exemple}\jya khɯtsa na-ʁɟo\cmn 他把碗冲了一下\end{exemple}
\begin{exemple}\jya tɯ-ŋga na-ʁɟo\cmn 他把衣服冲一下\end{exemple}
\begin{exemple}\jya nɯ-ʁɟo-t-a\cmn 我冲了\end{exemple}
\begin{exemple}\jya tɯthɯ na-ʁɟo\cmn 他把锅子冲了一下\end{exemple}\end{entrée}

\begin{entrée}
\vedette{\hypertarget{Ⓔʁɟoʁɟe}{\papi{ ʁɟoʁɟe}}}\markboth{ʁɟoʁɟe}{}
\classe{n}
\begin{définition}\fra vin ou lait dilué dans l'eau\end{définition}
\begin{définition}\cmn 掺了水的酒或者牛奶\end{définition}\end{entrée}

\begin{entrée}
\vedette{\hypertarget{Ⓔʁɟɯβʁɟɯβ}{\papi{ ʁɟɯβʁɟɯβ}}}\markboth{ʁɟɯβʁɟɯβ}{}
\classe{idph.2}
\begin{définition}\fra gros\end{définition}
\begin{définition}\cmn 形容人全身都胖的样子\end{définition}
\begin{exemple}\jya ɯ-tɯ-tshu kɯ ʁɟɯβʁɟɯβ ʑo ɲɯ-pa\cmn 他全身都很胖\end{exemple}\end{entrée}

\begin{entrée}
\vedette{\hypertarget{Ⓔʁɟɯʁɟri}{\papi{ ʁɟɯʁɟri}}}\markboth{ʁɟɯʁɟri}{}
\classe{idph.2}\acception{1}
\begin{définition}\fra gras et mou\end{définition}
\begin{définition}\cmn 形容胖而软的样子\end{définition}\acception{2}
\begin{définition}\fra humide\end{définition}
\begin{définition}\cmn 形容潮湿的样子\end{définition}
\begin{relation-sémantique}\confer{
\hyperlink{Ⓔχcɯχcri}{\textit{ \papi{χcɯχcri}}}
}\end{relation-sémantique}\end{entrée}

\begin{entrée}
\vedette{\hypertarget{Ⓔʁlaŋlu}{\papi{ ʁlaŋlu}}}\markboth{ʁlaŋlu}{}\classe{n}
\begin{définition}\fra année du bœuf\end{définition}
\begin{définition}\cmn 牛年
\begin{déclaration} \étymologie{\papi{glaŋ.lo}}\end{déclaration}\end{définition}
\end{entrée}

\begin{entrée}
\vedette{\hypertarget{Ⓔʁlɤwɯr}{\papi{ ʁlɤwɯr}}}\markboth{ʁlɤwɯr}{}\classe{adv}
\begin{définition}\fra soudain\end{définition}
\begin{définition}\cmn 突然
\begin{déclaration} \étymologie{\papi{glo.bur}}\end{déclaration}\end{définition}
\begin{exemple}\jya ʁlɤwɯr to-rɤŋgat\cmn 他突然出发了\end{exemple}
\begin{exemple}\jya ʁlɤwɯr to-ngo\cmn 他突然生病了\end{exemple}
\begin{relation-sémantique}\confer{
\hyperlink{Ⓔnɯʁlɤwɯr}{\textit{ \papi{nɯʁlɤwɯr}}}
}\end{relation-sémantique}\end{entrée}

\begin{entrée}
\vedette{\hypertarget{Ⓔʁlɯβʁlɯβ}{\papi{ ʁlɯβʁlɯβ}}}\markboth{ʁlɯβʁlɯβ}{}
\classe{idph.2}
\begin{définition}\fra concave\end{définition}
\begin{définition}\cmn 凹进去\end{définition}
\begin{exemple}\jya khɯsta kɯ-rnaʁ tsa ci ʁlɯβʁlɯβ ɲɯ-ŋu\cmn 碗有点深,是凹进去的\end{exemple}
\begin{exemple}\jya scoʁ ʁlɯβʁlɯβ kɯ-pa\cmn 凹进去(很深)的水瓢\end{exemple}
\begin{exemple}\jya praʁ ɯ-pa rɯdaʁ ci ku-rŋgɯ ɲɯ-ŋu ma ɯ-sta ʁlɯβʁlɯβ ɣɤʑu\cmn 悬崖下有个动物的窝,地是凹进去的\end{exemple}
\begin{relation-sémantique}\confer{
 \papi{aχchowɤlu}
}\end{relation-sémantique}
\begin{relation-sémantique}\confer{
\hyperlink{Ⓔaqhoβlu}{\textit{ \papi{aqhoβlu}}}
}\end{relation-sémantique}\end{entrée}

\begin{entrée}
\vedette{\hypertarget{Ⓔʁlɯm}{\papi{ ʁlɯm}}}\markboth{ʁlɯm}{}
\classe{n}
\begin{définition}\fra orge qui vient juste de fermenter\end{définition}
\begin{définition}\cmn 刚刚发酵的青稞
\begin{déclaration} \étymologie{\papi{glum}}\end{déclaration}\end{définition}\end{entrée}

\begin{entrée}
\vedette{\hypertarget{Ⓔʁlɯmbɯɣ}{\papi{ ʁlɯmbɯɣ}}}\markboth{ʁlɯmbɯɣ}{}\classe{n}
\begin{définition}\fra estimation\end{définition}
\begin{définition}\cmn 估计\end{définition}
\begin{exemple}\jya ʁlɯmbɯɣ ci tɤ-βzu-t-a\cmn 我估计了一下\end{exemple}
\begin{relation-sémantique}\synonyme{
\hyperlink{Ⓔʁjɯβtshɤt}{\textit{ \papi{ʁjɯβtshɤt}}}
}\end{relation-sémantique}
\begin{relation-sémantique}\confer{
\hyperlink{Ⓔnɯʁlɯmbɯɣ}{\textit{ \papi{nɯʁlɯmbɯɣ}}}
}\end{relation-sémantique}\end{entrée}

\begin{entrée}
\vedette{\hypertarget{Ⓔʁlɯmci}{\papi{ ʁlɯmci}}}\markboth{ʁlɯmci}{}\classe{n}
\begin{définition}\fra tchang qui vient juste de fermenter\end{définition}
\begin{définition}\cmn 刚刚发酵的青稞酒
\begin{déclaration} \étymologie{\papi{glum}}\end{déclaration}\end{définition}
\end{entrée}

\begin{entrée}
\vedette{\hypertarget{Ⓔʁlɯmsɯsi}{\papi{ ʁlɯmsɯsi}}}\markboth{ʁlɯmsɯsi}{}\classe{n}
\begin{définition}\fra Polygonatum sibiricum\end{définition}
\begin{définition}\cmn 黄精\end{définition}
\begin{exemple}\jya ʁlɯmsɯsi nɯ sɯjno ci ŋu. tɯ-ji mŋu ndo cho si kɯ-xtɕi ɯ-ŋgɯ ra tu-ɬoʁ ŋu, ɯ-zrɤm nɯ tɕazga fse, ɯ-ru kɯ-ɣɯrni tsa ŋu, ɯ-jwaʁ kɯ-tɕɤr kɯ-ɤmtɕoʁ tsa ŋu, ndɯβ. ɯ-mat nɯ ɯ-ru tu-nɯɴqhe tɕe, tu-oʑɯrja ŋu. ɯ-ru tɯ-ldʑa ma me, ɯ-mat thɯ-tɯt tɕe, chɯ-ɣɯrni ŋu. tú-wɣ-ndza tɕe chi.\cmn 黄精是一种植物。生长在地边和灌木林里,根部像姜一样,茎是红色的,叶子窄而尖,很小。果实是顺着茎往上长。只有一根茎,果实成熟后是红色的,可以吃,很甜。\end{exemple}
\end{entrée}

\begin{entrée}
\vedette{\hypertarget{Ⓔʁlɯn}{\papi{ ʁlɯn}}}\markboth{ʁlɯn}{}\classe{vs}
\paradigme{\textit{dir :} \jya tɤ-}
\begin{définition}\fra insensible aux malheurs\end{définition}
\begin{définition}\cmn 承受得住磨难,不容易动摇
\begin{déclaration} \étymologie{\papi{glen.pa}}\end{déclaration}\end{définition}
\begin{relation-sémantique}\confer{
\hyperlink{Ⓔʁlɯnba}{\textit{ \papi{ʁlɯnba}}}
}\end{relation-sémantique}
\end{entrée}

\begin{entrée}
\vedette{\hypertarget{Ⓔʁlɯnba}{\papi{ ʁlɯnba}}}\markboth{ʁlɯnba}{}
\classe{n}
\begin{définition}\fra insensible aux malheurs\end{définition}
\begin{définition}\cmn 承受得住磨难,不容易动摇
\begin{déclaration} \étymologie{\papi{glen.pa}}\end{déclaration}\end{définition}
\begin{exemple}\jya nɤki tɯrme ki ʁlɯnba ci ɕti tɕe mɤ-naʁdɯɣ\cmn 他是个承受得住磨难的人,他不会动摇的\end{exemple}
\begin{relation-sémantique}\confer{
\hyperlink{Ⓔʁlɯn}{\textit{ \papi{ʁlɯn}}}
}\end{relation-sémantique}\end{entrée}

\begin{entrée}
\vedette{\hypertarget{Ⓔʁma}{\papi{ ʁma}}}\markboth{ʁma}{}\classe{vs}
\paradigme{\textit{dir :} \jya pɯ-}
\begin{définition}\fra trop bas (coup de fusil)\end{définition}
\begin{définition}\cmn 打枪瞄低
\begin{déclaration} \étymologie{\papi{dma}}\end{déclaration}\end{définition}
\begin{exemple}\jya ɕɤmɯɣdɯ ɯ-tɯ-ʁma nɯ\cmn 那杆枪打得很低\end{exemple}
\begin{relation-sémantique}\antonyme{
\hyperlink{Ⓔmthu}{\textit{ \papi{mthu}}}
}\end{relation-sémantique}
\end{entrée}

\begin{entrée}
\vedette{\hypertarget{Ⓔʁmaʁ}{\papi{ ʁmaʁ}}}\markboth{ʁmaʁ}{}\classe{n}
\begin{définition}\fra armée\end{définition}
\begin{définition}\cmn 军队\end{définition}
\begin{exemple}\jya pɣɤtɕɯ ʁmaʁ\cmn 鸟群\end{exemple}\end{entrée}

\begin{entrée}
\vedette{\hypertarget{Ⓔʁmaʁdɤr}{\papi{ ʁmaʁdɤr}}}\markboth{ʁmaʁdɤr}{}\classe{n}
\begin{définition}\fra drapeau\end{définition}
\begin{définition}\cmn 旗
\begin{déclaration} \étymologie{\papi{dmag.dar}}\end{déclaration}\end{définition}
\end{entrée}

\begin{entrée}
\vedette{\hypertarget{Ⓔʁmaʁmi}{\papi{ ʁmaʁmi}}}\markboth{ʁmaʁmi}{}
\classe{n}
\begin{définition}\fra militaire\end{définition}
\begin{définition}\cmn 士兵
\begin{déclaration} \étymologie{\papi{dmag.mi}}\end{déclaration}\end{définition}
\begin{exemple}\jya ʁmaʁmi pɯ-kɯ-ɬoʁ\cmn 退役的士兵\end{exemple}
\begin{relation-sémantique}\confer{
\hyperlink{Ⓔnɯʁmaʁmi}{\textit{ \papi{nɯʁmaʁmi}}}
}\end{relation-sémantique}\end{entrée}

\begin{entrée}
\vedette{\hypertarget{Ⓔʁmazgrɯβ}{\papi{ ʁmazgrɯβ}}}\markboth{ʁmazgrɯβ}{}
\classe{n}
\begin{définition}\fra cicatrice\end{définition}
\begin{définition}\cmn 伤疤\end{définition}\end{entrée}

\begin{entrée}
\vedette{\hypertarget{Ⓔʁmɤrɲɯɣ}{\papi{ ʁmɤrɲɯɣ}}}\markboth{ʁmɤrɲɯɣ}{}\classe{n}
\begin{définition}\fra moustique\end{définition}
\begin{définition}\cmn 蚊子\end{définition}
\begin{exemple}\jya ʁmɤrɲɯɣ nɯ qajɯ ɯ-ʁar kɯ-tu ci ŋu, ɲɯ-nɯqambɯmbjom cha, ɯ-mtɕhi kɯ-ɤmtɕɯ-mtɕoʁ ci ŋu, ɯ-mtɕhi ɯ-taʁ taqaβ kɯ-fse kɯ-xtshɯ-xtshɯm, tɤ-rme jamar ʑo kɯ-xtshɯm tu, ɯ-mɤlɤjaʁ kɯtʂɤ-ldʑa tu, kɯ-rɲɟɯ-rɲɟi kɯ-xtshɯ-xtshɯm ʑo tu, pha ɯ-phoŋbu nɯ kɯ-ɲaʁ ŋu, wuma ʑo sɤmtsɯɣ tɕe aɣɯtɯɣ tɕe, tɯ-ndʐi kɤ-mtsɯɣ tɕe, kɤ-mtsɯɣ ɯ-rkɯ nɯ ɯ-kho kɯ-jom ʑo tu-z-nɯɣmbɤβ cha tɕe wuma ʑo rɤʑa tɕe pɯ́-wɣ-qraʁ kɯnɤ ɣɯ-rɤβraʁ ra.\cmn 蚊子是带有翅膀的昆虫,会飞。嘴很尖,嘴有一种针,细得像毛一样。有六只脚,又细又长,全身都是黑色的。蜇人,有毒。蜇皮肤时,蜇过的地方周围会有较大面积的皮肤肿起来,很痒,而且抠破了还继续发痒。\end{exemple}\end{entrée}

\begin{entrée}
\vedette{\hypertarget{Ⓔʁmɤrsɤr}{\papi{ ʁmɤrsɤr}}}\markboth{ʁmɤrsɤr}{}
\classe{n}
\begin{définition}\fra doré\end{définition}
\begin{définition}\cmn 金黄色\end{définition}\end{entrée}

\begin{entrée}
\vedette{\hypertarget{Ⓔʁmɤrsmɯɣ}{\papi{ ʁmɤrsmɯɣ}}}\markboth{ʁmɤrsmɯɣ}{}
\classe{n}
\begin{définition}\fra bordeau\end{définition}
\begin{définition}\cmn 紫色
\begin{déclaration} \étymologie{\papi{dmar.smug}}\end{déclaration}\end{définition}
\begin{relation-sémantique}\confer{
\hyperlink{Ⓔarɯʁmɤrsmɯɣ}{\textit{ \papi{arɯʁmɤrsmɯɣ}}}
}\end{relation-sémantique}\end{entrée}

\begin{entrée}
\vedette{\hypertarget{Ⓔʁmɤrɯɣ}{\papi{ ʁmɤrɯɣ}}}\markboth{ʁmɤrɯɣ}{}
\classe{n}
\begin{définition}\fra casserole en cuivre\end{définition}
\begin{définition}\cmn 红铜锅,用来熬茶
\begin{déclaration} \étymologie{\papi{dmar.rigs.?}}\end{déclaration}\end{définition}\end{entrée}

\begin{entrée}
\vedette{\hypertarget{Ⓔʁmbɣi}{\papi{ ʁmbɣi}}}\markboth{ʁmbɣi}{}
\classe{n}
\begin{définition}\fra soleil\end{définition}
\begin{définition}\cmn 太阳\end{définition}
\begin{relation-sémantique}\confer{
\hyperlink{Ⓔʁmbɣɯzɯn}{\textit{ \papi{ʁmbɣɯzɯn}}}
}\end{relation-sémantique}\end{entrée}

\begin{entrée}
\vedette{\hypertarget{Ⓔʁmbɣɯzɯn}{\papi{ ʁmbɣɯzɯn}}}\markboth{ʁmbɣɯzɯn}{}\classe{n}
\begin{définition}\fra éclipse de soleil\end{définition}
\begin{définition}\cmn 日蚀\end{définition}
\begin{relation-sémantique}\confer{
\hyperlink{Ⓔʁmbɣi}{\textit{ \papi{ʁmbɣi}}}
}\end{relation-sémantique}\end{entrée}

\begin{entrée}
\vedette{\hypertarget{Ⓔʁmbroŋ}{\papi{ ʁmbroŋ}}}\markboth{ʁmbroŋ}{} (\variante{mbroŋ}) 
\classe{n}
\begin{définition}\fra yak sauvage\end{définition}
\begin{définition}\cmn 野牦牛
\begin{déclaration} \étymologie{\papi{ⁿbroŋ}}\end{déclaration}\end{définition}\end{entrée}

\begin{entrée}
\vedette{\hypertarget{Ⓔʁmɯɣ}{\papi{ ʁmɯɣ}}}\markboth{ʁmɯɣ}{}\classe{vt}
\paradigme{\textit{dir :} \jya tɤ-}
\begin{définition}\fra réfléchir à un plan\end{définition}
\begin{définition}\cmn 计划;计谋;怀疑\end{définition}
\begin{exemple}\jya nɯ ɯ-mɤ-kɯ-ŋu ci ɲɯ-ʁmɯɣ\cmn 他怀疑是不是\end{exemple}
\begin{exemple}\jya kɯki a-pɯ-jɤɣ tɕe, tu-nɯna-a ku-ʁmɯɣ-a\cmn 这件事结束的时候,我打算休息\end{exemple}
\begin{exemple}\jya fsɤqhe tɕe mbarkhom kɤ-ɕe ku-ʁmɯɣ-a\end{exemple}
\begin{exemple}\jya mbarkhom ju-ɕe-a ku-ʁmɯɣ-a\cmn 我打算明年去马尔康\end{exemple}
\begin{exemple}\jya nɤ-qajɣi tu-kɤ-χtɯ tɤ-ʁmɯɣ-a ri, pɤjkhu mɯ-ko-smi\cmn 我本来想你买馍馍(给你吃)但是没有熟\end{exemple}\begin{sous-entrée}
\vedette{\hypertarget{}{\papi{ ʑɣɤʁmɯɣ}}}\markboth{ʑɣɤʁmɯɣ}{}\classe{vt}
\paradigme{\textit{dir :} \jya tɤ-}
\begin{définition}\fra décider par soi-même\end{définition}
\begin{définition}\cmn 自己决定\end{définition}
\begin{exemple}\jya jisŋi kɯ-nɯɕe ku-ʑɣɤʁmɯɣ-a (kɤ-nɯɕe ku-ʑɣɤʁmɯɣ-a)\cmn 我决定今天回去\end{exemple}
\end{sous-entrée}\end{entrée}

\begin{entrée}
\vedette{\hypertarget{Ⓔʁmɯrcɯ}{\papi{ ʁmɯrcɯ}}}\markboth{ʁmɯrcɯ}{}
\classe{n}
\begin{définition}\fra grive (garrulax maximus)\end{définition}
\begin{définition}\cmn 大噪鹛【画眉鸟】\end{définition}
\begin{exemple}\jya ʁmɯrcɯ nɯ khro mɤ-wxti, kha ɯ-rkɯ sɯŋgɯ ra ku-rɤʑi rga, ɯ-mdoʁ kɯ-pɣi ɯ-ŋgɯz kɯnɤ ldʑaŋkɯ kɯ-ɤrɤɕɯɕrɤz tu, ɯ-jme nɯ kɯ-ɲaʁ tɕe ɯ-ku kɯ-wɣrum tu, nɯ-nɯqambɯmbjom tɕe ɯ-ʁar ɯ-rca ɯ-jme kɯnɤ ɲɯ-sqhiar ŋu, tɤ-rɤku kɤ-ndza wuma rga, qajɯ nɯ ra kɤ-ndza rga tɕe qambalɯla ta-ndza tɕe ɯ-phoŋbu tu-ndze tɕe ɯ-ʁar nɯ ra pjɯ-βde ŋu. qambalɯla nɯ tɤtʂu ɯ-ɕki kɤ-ɣi rga tɕe ʁmɯrcɯ kɯ tɤtʂu ɯ-pa nɯ ra qambalɯla ɯ-ʁar pjɯ-prɤm ʑo ŋgrɤl.\cmn 画眉鸟长得不大,喜欢住在房子周围的树林里,身体是灰色的,上面有灰色的纹路,尾巴是黑色,顶端有白点。翅膀展开飞翔时,尾巴也跟着展开。喜欢吃庄稼、虫子。当它吃蝴蝶的时候,只吃身子,不吃翅膀,把翅膀扔在地上。因为蝴蝶喜欢靠近灯光,画眉鸟就会在灯下的地上留下很多蝴蝶翅膀。\end{exemple}\end{entrée}

\begin{entrée}
\vedette{\hypertarget{Ⓔʁmɯrɲɯɣ}{\papi{ ʁmɯrɲɯɣ}}}\markboth{ʁmɯrɲɯɣ}{}\classe{n}
\begin{définition}\fra grive (pomatorhinus erythrocnemis)\end{définition}
\begin{définition}\cmn 斑胸钩嘴鹛\end{définition}
\end{entrée}

\begin{entrée}
\vedette{\hypertarget{Ⓔʁmɯrqaʁ}{\papi{ ʁmɯrqaʁ}}}\markboth{ʁmɯrqaʁ}{}\classe{n}
\begin{définition}\fra grive (garrulax ocellatus)\end{définition}
\begin{définition}\cmn 眼纹噪鹛【呱呱鸡】\end{définition}\end{entrée}

\begin{entrée}
\vedette{\hypertarget{Ⓔʁmɯrtsɯ}{\papi{ ʁmɯrtsɯ}}}\markboth{ʁmɯrtsɯ}{}
\classe{n}
\begin{définition}\fra espèce d'arbrisseau\end{définition}
\begin{définition}\cmn 灌木的一种\end{définition}
\begin{exemple}\jya ʁmɯrtsɯ nɯ si kɯ-mbɤr ci ŋu, ɯʑo sti tu-ɬoʁ tɕe, kɯ-ndɯβ kɯ-dɤn tsa tɯtɯrca tu-ɬoʁ ŋu, ɯ-rkɯ si kɯ-mbro tsa a-pɯ-tu tɕe, ɯ-taʁ ku-rtɤβ nɤ ku-rtɤβ tɕe tu-ɕe nɤ tu-ɕe ɯ-kɤχcɤl mɤɕtʂa tu-ɕe ŋu. ɣɯjpa ɲɯ-rɯmɯntoʁ tɕe fsaqhe tɕe ɯ-mat ku-tshoʁ ŋu, ɯ-mat ku-tɯ-tshoʁ nɯ ɣɯrni thɯ-tɯt tɕe chɯ-ɲaʁ ŋu, kɤ-ndza sna tɕe chi. ɯ-jwaʁ kɯ-ɤrtɯm kɯ-zri tsa ŋu. ɯ-ru ɯ-mdoʁ nɯ kɯ-qarŋe tsa ŋu.\cmn 
\stylefv{ʁmɯrtsɯ} 是比较矮的树,当它单独生长时,长得细而密的,一丛一丛的,当它靠近其他较高的树时,它就会缠着向上爬,一直爬到树梢。今年开花明年结果,果实刚结时是红色的,成熟时是黑色的,可以吃,是甜的,叶子是椭圆形的,树干是黄色的。
\end{exemple}\end{entrée}

\begin{entrée}
\vedette{\hypertarget{Ⓔʁnamʑa}{\papi{ ʁnamʑa}}}\markboth{ʁnamʑa}{}
\classe{n}
\begin{définition}\fra diadème\end{définition}
\begin{définition}\cmn 天冠
\begin{déclaration} \étymologie{\papi{gnam.ʑwa}}\end{déclaration}\end{définition}\end{entrée}

\begin{entrée}
\vedette{\hypertarget{Ⓔʁnaʁna}{\papi{ ʁnaʁna}}}\markboth{ʁnaʁna}{}\classe{n}
\begin{définition}\fra les deux\end{définition}
\begin{définition}\cmn 两个都\end{définition}
\begin{exemple}\jya tɕiʑo ʁnaʁna ʑo tɕi-ʑɯβ ɲɯ-ɣi\cmn 我们俩都有瞌睡\end{exemple}
\end{entrée}

\begin{entrée}
\vedette{\hypertarget{Ⓔʁnɤmchi}{\papi{ ʁnɤmchi}}}\markboth{ʁnɤmchi}{}
\classe{n}
\begin{définition}\fra grande ourse\end{définition}
\begin{définition}\cmn 北斗
\begin{déclaration} \étymologie{\papi{gnam.kʰʲi}}\end{déclaration}\end{définition}\end{entrée}

\begin{entrée}
\vedette{\hypertarget{Ⓔʁnɤmjaŋ}{\papi{ ʁnɤmjaŋ}}}\markboth{ʁnɤmjaŋ}{}
\classe{n}
\begin{définition}\fra plafond\end{définition}
\begin{définition}\cmn 天花板
\begin{déclaration} \étymologie{\papi{gnam.jaŋs}}\end{déclaration}\end{définition}\end{entrée}

\begin{entrée}
\vedette{\hypertarget{Ⓔʁnɤt}{\papi{ ʁnɤt}}}\markboth{ʁnɤt}{}
\classe{vs}
\paradigme{\textit{dir :} \jya tɤ-}
\begin{définition}\fra nuisible\end{définition}
\begin{définition}\cmn 有害\end{définition}
\begin{exemple}\jya nɤʑo nɤ-χti ɯ-taʁ ɲɯ-tɯ-ʁnɤt\cmn 你欺负他\end{exemple}\begin{sous-entrée}
\vedette{\hypertarget{}{\papi{ ɣɤʁnɤt}}}\markboth{ɣɤʁnɤt}{}\classe{vs}
\begin{définition}\fra s'abîmer facilement\end{définition}
\begin{définition}\cmn 容易损坏
\begin{déclaration} \étymologie{\papi{gnod}}\end{déclaration}\end{définition}
\begin{exemple}\jya tɯ-rnom ɯ-ɕɤrɯ nɯ wuma ʑo ɣɤʁnɤt, pjɯ-kɯ-ndʐaβ, kú-wɣ-nɯ-rpu nɯ ra tɕe tu-ʁnɤt mbat\cmn 肋骨容易裂伤\end{exemple}
\begin{relation-sémantique}\confer{
\hyperlink{Ⓔsaʁnɤt}{\textit{ \papi{saʁnɤt}}}
}\end{relation-sémantique}
\end{sous-entrée}\begin{sous-entrée}
\vedette{\hypertarget{}{\papi{ naʁnɤt}}}\markboth{naʁnɤt}{}\classe{vt}
\paradigme{\textit{dir :} \jya nɯ-}
\begin{définition}\fra être gêné de\end{définition}
\begin{définition}\cmn 不好意思,过意不去\end{définition}
\begin{exemple}\jya ɯʑo kɯ kɤ-sɯzdɯɣ ɲɯ-naʁnɤt\cmn 他不好意思麻烦别人\end{exemple}
\end{sous-entrée}\end{entrée}

\begin{entrée}
\vedette{\hypertarget{Ⓔʁndɤr}{\papi{ ʁndɤr}}}\markboth{ʁndɤr}{}
\classe{vi}
\paradigme{\textit{dir :} \jya pɯ-}
\begin{définition}\ 
\begin{déclaration}\grammar{acaus}\end{déclaration}\end{définition}
\begin{définition}\fra se disperser\end{définition}
\begin{définition}\cmn 散开
\begin{déclaration} \étymologie{\papi{gtor}}\end{déclaration}\end{définition}
\begin{exemple}\jya tɯjpu pjɤ-ʁndɤr\cmn 面粉撒得到处都是\end{exemple}
\begin{exemple}\jya laχtɕha pjɤ-ʁndɤr\cmn 东西撒得到处都是\end{exemple}
\begin{exemple}\jya fsapaʁ ra pjɤ-ʁndɤr-nɯ\cmn 牲畜分散了\end{exemple}
\begin{exemple}\jya ɯ-sɯm ra pjɤ-ʁndɤr ʑo (=ɯ-ku ra pjɤ-mɯɕtaʁ ʑo)\cmn 他被吓到了\end{exemple}
\begin{relation-sémantique}\confer{
\hyperlink{Ⓔχtɤr}{\textit{ \papi{χtɤr}}}
}\end{relation-sémantique}\end{entrée}

\begin{entrée}
\vedette{\hypertarget{Ⓔʁndɯ}{\papi{ ʁndɯ}}}\markboth{ʁndɯ}{}
\classe{vt}
\paradigme{\textit{dir :} \jya tɤ-}
\paradigme{\textit{dir :} \jya pɯ-}
\begin{définition}\fra battre, frapper\end{définition}
\begin{définition}\cmn 打\end{définition}
\begin{exemple}\jya nɤ-stu tɤ-fse ma ta-ʁndɯ\cmn 你表现好一点,不然就打你\end{exemple}\begin{sous-entrée}
\vedette{\hypertarget{}{\papi{ saʁndɯ}}}\markboth{saʁndɯ}{}\classe{vi}
\paradigme{\textit{dir :} \jya tɤ-}
\begin{définition}\ 
\begin{déclaration}\grammar{apass}\end{déclaration}\end{définition}
\begin{définition}\fra frapper (des gens)\end{définition}
\begin{définition}\cmn 打人\end{définition}
\begin{relation-sémantique}\synonyme{
\hyperlink{Ⓔrdoŋ}{\textit{ \papi{rdoŋ}}}
}\end{relation-sémantique}
\end{sous-entrée}\end{entrée}

\begin{entrée}
\vedette{\hypertarget{Ⓔʁndzɤr}{\papi{ ʁndzɤr}}}\markboth{ʁndzɤr}{}
\classe{vt}
\paradigme{\textit{dir :} \jya pɯ-}
\paradigme{\textit{dir :} \jya nɯ-}
\paradigme{\textit{dir :} \jya \_}
\begin{définition}\fra couper avec des ciseaux, couper un arbre, scier\end{définition}
\begin{définition}\cmn 剪;砍树;锯断\end{définition}
\begin{exemple}\jya ɕoŋtɕa pa-ʁndzɤr\cmn 他把木料锯断了\end{exemple}
\begin{exemple}\jya si pa-ʁndzɤr\cmn 他把木料锯断了\end{exemple}
\begin{exemple}\jya tɤ-ri pa-ʁndzɤr\cmn 他把线剪断了\end{exemple}\begin{sous-entrée}
\vedette{\hypertarget{}{\papi{ sɯʁndzɤr}}}\markboth{sɯʁndzɤr}{}
\paradigme{\textit{dir :} \jya pɯ-}
\begin{définition}\fra couper avec\end{définition}
\begin{définition}\cmn 用……来剪\end{définition}
\begin{exemple}\jya rɟaŋsoʁ kɯ si pɯ-sɯʁndzar-a\cmn 我用锯子把木料锯断了\end{exemple}
\begin{exemple}\jya tsɯntu kɯ pɯ-sɯʁndzar-a\cmn 我用剪刀剪了\end{exemple}
\end{sous-entrée}\end{entrée}

\begin{entrée}
\vedette{\hypertarget{Ⓔʁnɯ}{\papi{ ʁnɯ}}}\markboth{ʁnɯ}{}\classe{vi}
\paradigme{\textit{dir :} \jya tɤ-}
\begin{définition}\fra supposer\end{définition}
\begin{définition}\cmn 起疑心\end{définition}
\begin{exemple}\jya tɤ-ʁnɯ-a\cmn 我怀疑了\end{exemple}
\begin{exemple}\jya ɯʑo kɯ-ti maŋe nɤri, aʑo tɤ-ʁnɯ-a ɕti ma\cmn 他倒没有说,是我自己怀疑的\end{exemple}
\begin{exemple}\jya ɯ-re wuma ʑo ɲɯ-ɬoʁ tɕe, tɤ-ʁnɯ-a ma ɯ-tshɯɣa mɯ́j-βdi\cmn 他很想笑的样子,所以我就有点怀疑了\end{exemple}\begin{sous-entrée}
\vedette{\hypertarget{}{\papi{ sɯʁnɯ}}}\markboth{sɯʁnɯ}{}\classe{vt}
\paradigme{\textit{dir :} \jya tɤ-}
\begin{définition}\ 
\begin{déclaration}\grammar{caus}\end{déclaration}\end{définition}
\begin{définition}\fra causer la suspicion\end{définition}
\begin{définition}\cmn 令人起疑心\end{définition}
\begin{exemple}\jya kɤ-ti pjɯ-sthɯt maŋe tɕe, tɤ́-wɣ-sɯʁnɯ-a\cmn 他说总是没完没了(一会这样说,一会那样说),令我起了疑心\end{exemple}
\end{sous-entrée}\end{entrée}

\begin{entrée}
\vedette{\hypertarget{Ⓔʁnɯz}{\papi{ ʁnɯz}}}\markboth{ʁnɯz}{}\classe{num}
\begin{définition}\fra deux\end{définition}
\begin{définition}\cmn 二\end{définition}\end{entrée}

\begin{entrée}
\vedette{\hypertarget{Ⓔʁnɯzmɯz}{\papi{ ʁnɯzmɯz}}}\markboth{ʁnɯzmɯz}{}\classe{n}
\begin{définition}\fra deux sortes (expression toute faite, emprunt)\end{définition}
\begin{définition}\cmn 两种\end{définition}\end{entrée}

\begin{entrée}
\vedette{\hypertarget{Ⓔʁɲɤlwa}{\papi{ ʁɲɤlwa}}}\markboth{ʁɲɤlwa}{}\classe{n}
\begin{définition}\fra enfer\end{définition}
\begin{définition}\cmn 阴间
\begin{déclaration} \étymologie{\papi{dmʲal.ba}}\end{déclaration}\end{définition}
\begin{exemple}\jya ʁɲɤlwa sɤtɕha jɤ-ari\cmn 他去了阴间(去世了)\end{exemple}
\begin{exemple}\jya a-ʁɲɤlwa ʑo nɯ-rzoʁ\cmn 我受了很多苦\end{exemple}
\begin{relation-sémantique}\confer{
\hyperlink{Ⓔnɯʁɲɤlwa}{\textit{ \papi{nɯʁɲɤlwa}}}
}\end{relation-sémantique}\end{entrée}

\begin{entrée}
\vedette{\hypertarget{Ⓔʁɲɤrpa}{\papi{ ʁɲɤrpa}}}\markboth{ʁɲɤrpa}{}
\classe{n}
\begin{définition}\fra intendant du monastère\end{définition}
\begin{définition}\cmn 寺庙的管家
\begin{déclaration} \étymologie{\papi{gɲer.pa}}\end{déclaration}\end{définition}\end{entrée}

\begin{entrée}
\vedette{\hypertarget{Ⓔʁɲɟiʁɲɟi}{\papi{ ʁɲɟiʁɲɟi}}}\markboth{ʁɲɟiʁɲɟi}{}
\classe{idph.2}
\begin{définition}\fra énorme\end{définition}
\begin{définition}\cmn 形容粗壮高大的样子\end{définition}
\begin{définition}\cmn 他长的又粗壮又高大\end{définition}
\begin{exemple}\jya ʁɲɟiʁɲɟi ʑo ɲɯ-pa\end{exemple}\begin{sous-entrée}
\vedette{\hypertarget{}{\papi{ ʁɲɟinɤʁɲɟi}}}\markboth{ʁɲɟinɤʁɲɟi}{}\classe{idph.3}
\begin{relation-sémantique}\confer{
\hyperlink{Ⓔʁɲɟliʁɲɟli}{\textit{ \papi{ʁɲɟliʁɲɟli}}}
}\end{relation-sémantique}
\end{sous-entrée}\end{entrée}

\begin{entrée}
\vedette{\hypertarget{Ⓔʁɲɟliʁɲɟli}{\papi{ ʁɲɟliʁɲɟli}}}\markboth{ʁɲɟliʁɲɟli}{}\classe{idph.2}
\begin{définition}\fra énorme\end{définition}
\begin{définition}\cmn 肥壮;高大\end{définition}
\begin{exemple}\jya loŋbutɕhi ʁɲɟliʁɲɟli ʑo ɲɯ-pa\cmn 大象又肥壮又高大\end{exemple}
\begin{exemple}\jya ɯʑo chɤ-tshu tɕe ʁɲɟliʁɲɟli ʑo chɤ-pa\cmn 他变胖了,身体变得很肥壮\end{exemple}
\begin{relation-sémantique}\confer{
\hyperlink{Ⓔʁɲɟiʁɲɟi}{\textit{ \papi{ʁɲɟiʁɲɟi}}}
}\end{relation-sémantique}\end{entrée}

\begin{entrée}
\vedette{\hypertarget{Ⓔʁɲɯɣnaʁ}{\papi{ ʁɲɯɣnaʁ}}}\markboth{ʁɲɯɣnaʁ}{}\classe{n}
\begin{définition}\fra bovidé à tête blanche, mais dont les yeux sont cerclés de noir\end{définition}
\begin{définition}\cmn 头白而眼圈黑的牛
\begin{déclaration} \étymologie{\papi{mig.nag}}\end{déclaration}\end{définition}
\end{entrée}

\begin{entrée}
\vedette{\hypertarget{Ⓔʁɲɯɣra}{\papi{ ʁɲɯɣra}}}\markboth{ʁɲɯɣra}{}
\classe{n}
\begin{définition}\fra masque pour se couvrir les yeux\end{définition}
\begin{définition}\cmn 眼罩
\begin{déclaration} \étymologie{\papi{mig.ra}}\end{déclaration}\end{définition}\end{entrée}

\begin{entrée}
\vedette{\hypertarget{Ⓔʁɲɯm}{\papi{ ʁɲɯm}}}\markboth{ʁɲɯm}{}\classe{vt}
\paradigme{\textit{dir :} \jya tɤ-}
\begin{définition}\ 
\begin{déclaration}\use{古语}\end{déclaration}\end{définition}
\begin{définition}\fra avoir des soupçons à l'égard de\end{définition}
\begin{définition}\cmn 怀疑,起疑心\end{définition}
\begin{exemple}\jya ɯʑo kɯ tɤ́-wɣ-ʁɲɯm-a\cmn 他对我起了疑心\end{exemple}\end{entrée}

\begin{entrée}
\vedette{\hypertarget{Ⓔʁɲɯspa}{\papi{ ʁɲɯspa}}}\markboth{ʁɲɯspa}{}\classe{n}
\begin{définition}\fra deuxième mois\end{définition}
\begin{définition}\cmn 二月
\begin{déclaration} \étymologie{\papi{gɲis.pa}}\end{déclaration}\end{définition}
\end{entrée}

\begin{entrée}
\vedette{\hypertarget{Ⓔʁŋaftɕɤt}{\papi{ ʁŋaftɕɤt}}}\markboth{ʁŋaftɕɤt}{}
\classe{n}
\begin{définition}\fra dernier jour du jeune\end{définition}
\begin{définition}\cmn 禁食斋的最后一天\end{définition}
\begin{exemple}\jya sɲaŋne kɯ-maqhu ɯ-sŋi nɯ ʁŋaftɕɤt tu-kɯ-ti ɲɯ-ŋu\cmn 禁食斋的最后一天叫做\end{exemple}\end{entrée}

\begin{entrée}
\vedette{\hypertarget{Ⓔʁo}{\papi{ ʁo}}}\markboth{ʁo}{}\classe{adv}
\begin{définition}\fra adversatif\end{définition}
\begin{définition}\cmn 倒\end{définition}\end{entrée}

\begin{entrée}
\vedette{\hypertarget{Ⓔʁoŋmdzɤt}{\papi{ ʁoŋmdzɤt}}}\markboth{ʁoŋmdzɤt}{}\classe{n}
\begin{définition}\fra moine en charge de la récitation des soutras\end{définition}
\begin{définition}\cmn 掌管念经活到的和尚\end{définition}\end{entrée}

\begin{entrée}
\vedette{\hypertarget{Ⓔʁoŋtɕhaʁ}{\papi{ ʁoŋtɕhaʁ}}}\markboth{ʁoŋtɕhaʁ}{}\classe{n}
\begin{définition}\fra menace\end{définition}
\begin{définition}\cmn 威胁;镇压\end{définition}
\begin{exemple}\jya nɯ-ʁoŋtɕhaʁ sɤ-lɤt ci tú-wɣ-βzu ɲɯ-ra\cmn 要用威胁的手段镇压他们\end{exemple}\end{entrée}

\begin{entrée}
\vedette{\hypertarget{Ⓔʁrɯlu}{\papi{ ʁrɯlu}}}\markboth{ʁrɯlu}{}
\classe{n}
\begin{définition}\fra sans corne\end{définition}
\begin{définition}\cmn 无角的\end{définition}
\begin{relation-sémantique}\confer{
\hyperlink{Ⓔta-ʁrɯ}{\textit{ \papi{ta-ʁrɯ}}}
}\end{relation-sémantique}\end{entrée}

\begin{entrée}
\vedette{\hypertarget{Ⓔʁrɯrpu}{\papi{ ʁrɯrpu}}}\markboth{ʁrɯrpu}{}\classe{n}
\begin{définition}\fra coup de corne\end{définition}
\begin{définition}\cmn 用角打\end{définition}
\begin{exemple}\jya jla kɯ a-taʁ ʁrɯrpu ta-lɤt\cmn 犏牛用角打了我\end{exemple}
\begin{relation-sémantique}\confer{
\hyperlink{Ⓔta-ʁrɯ}{\textit{ \papi{ta-ʁrɯ}}}
}\end{relation-sémantique}
\begin{relation-sémantique}\confer{
\hyperlink{Ⓔrpu}{\textit{ \papi{rpu}}}
}\end{relation-sémantique}
\begin{relation-sémantique}\confer{
 \papi{nɯʁrɯrpu}
}\end{relation-sémantique}\end{entrée}

\begin{entrée}
\vedette{\hypertarget{Ⓔʁzɤβ}{\papi{ ʁzɤβ}}}\markboth{ʁzɤβ}{}
\classe{vi}
\paradigme{\textit{dir :} \jya tɤ-}
\begin{définition}\fra attentif, soigneux\end{définition}
\begin{définition}\cmn 细心;细致
\begin{déclaration} \étymologie{\papi{gzab}}\end{déclaration}\end{définition}
\begin{exemple}\jya ɯ-ma ɲɯ-ʁzɤβ\cmn 他做事很细致\end{exemple}
\begin{exemple}\jya ɯ-βraʁ kɤ-ʁzaβ-a\cmn 我拴好了\end{exemple}
\begin{exemple}\jya kɯm ɯ-ɣɤβdi kɤ-ʁzaβ-a\cmn 我把门顶住好了\end{exemple}
\begin{exemple}\jya tɤ-rɤku kɤ-saχsi nɯ-ʁzaβ-a\cmn 我把粮食跟石头泥巴分开好了\end{exemple}
\begin{exemple}\jya smɤɣ ɯ-saʁ nɯ-ʁzaβ-a\cmn 我把羊毛撕好了\end{exemple}
\begin{exemple}\jya ɯ-pa kɯ-ʁzɤβ ci ɲɯ-ŋu\cmn 他是个做事做得很细致的人\end{exemple}\begin{sous-entrée}
\vedette{\hypertarget{}{\papi{ sɯʁzɤβ}}}\markboth{sɯʁzɤβ}{}\classe{vt}
\begin{exemple}\jya nɤ-kɯ-mŋɤm koŋla kɤ-rtoʁ a-pɯ-tɯ-sɯ-ʁzɤβ\end{exemple}
\end{sous-entrée}\end{entrée}

\begin{entrée}
\vedette{\hypertarget{Ⓔʁzɤmi}{\papi{ ʁzɤmi}}}\markboth{ʁzɤmi}{}
\classe{n}
\begin{définition}\fra couple\end{définition}
\begin{définition}\cmn 夫妻
\begin{déclaration} \étymologie{\papi{bza.mi}}\end{déclaration}\end{définition}\end{entrée}

\begin{entrée}
\vedette{\hypertarget{Ⓔʁzɤn}{\papi{ ʁzɤn}}}\markboth{ʁzɤn}{}\classe{n}
\begin{définition}\fra kyasha\end{définition}
\begin{définition}\cmn 袈裟
\begin{déclaration} \étymologie{\papi{gzan}}\end{déclaration}\end{définition}
\end{entrée}

\begin{entrée}
\vedette{\hypertarget{Ⓔʁzɤr}{\papi{ ʁzɤr}}}\markboth{ʁzɤr}{}
\classe{n}
\begin{définition}\fra difficulté à respirer\end{définition}
\begin{définition}\cmn 呼吸困难\end{définition}
\begin{exemple}\jya ʁzɤr nɯ tu-kɯ-ɤɕqhe ri mŋɤm, tu-kɯ-ɤtɕhɯz ri mŋɤm tɯ-sŋɯro kɯ-wxti lú-wɣ-tɕɤt ri mŋɤm tɯ-ŋgo kɯ-ŋɤn ci ŋu\cmn 
\stylefv{ʁzɤr}是一种严重的病,(如果得了这个病)你咳嗽时,打喷嚏时,出大气时都感到痛。
\end{exemple}\end{entrée}

\begin{entrée}
\vedette{\hypertarget{Ⓔʁzi}{\papi{ ʁzi}}}\markboth{ʁzi}{}\classe{vs}
\paradigme{\textit{dir :} \jya thɯ-}
\begin{définition}\fra nécessaire\end{définition}
\begin{définition}\cmn 需要\end{définition}
\begin{exemple}\jya ki ɯʑo ɣɯ ɲɯ-ʁzi\cmn 他需要这个\end{exemple}
\begin{exemple}\jya ɯʑo kɯ pa-nɤpe ri ɕɤxɕo tɕe cho-ʁzi tɕe ra ɲɯ-ti\cmn 他原来不喜欢,但是现在就需要了,就说要\end{exemple}
\begin{relation-sémantique}\confer{
\hyperlink{Ⓔnaʁzi}{\textit{ \papi{naʁzi}}}
}\end{relation-sémantique}\end{entrée}

\begin{entrée}
\vedette{\hypertarget{Ⓔʁzraŋʁzraŋ}{\papi{ ʁzraŋʁzraŋ}}}\markboth{ʁzraŋʁzraŋ}{}\classe{idph.2}
\begin{définition}\fra ébouriffé\end{définition}
\begin{définition}\cmn 乱蓬蓬\end{définition}
\begin{exemple}\jya ɯ-kɤrme ʁzraŋʁzraŋ ʑo ɲɯ-pa\cmn 他头发是乱蓬蓬的\end{exemple}\end{entrée}

\begin{entrée}
\vedette{\hypertarget{Ⓔʁzɯɣ}{\papi{ ʁzɯɣ}}}\markboth{ʁzɯɣ}{}\classe{n}
\begin{définition}\fra apparence\end{définition}
\begin{définition}\cmn 相貌
\begin{déclaration} \étymologie{\papi{gzugs}}\end{déclaration}\end{définition}
\end{entrée}

\begin{entrée}
\vedette{\hypertarget{Ⓔʁzɯlu}{\papi{ ʁzɯlu}}}\markboth{ʁzɯlu}{}\classe{n}
\begin{définition}\fra maladie de l'œil\end{définition}
\begin{définition}\cmn 白眼病\end{définition}
\begin{exemple}\jya tɯrme ʁzɯlu nɯ mbro ɯ-phɯ ci ŋu, mbro ʁzɯlu nɯ khɯna ɯ-phɯ ci ŋu, khɯna ʁzɯlu nɯ mbro ɯ-phɯ ci ŋu\cmn 患白眼病的人有马的价格,患白眼病的马只有狗的价格,患白眼病的狗有马的价格\end{exemple}\end{entrée}

\begin{entrée}
\vedette{\hypertarget{Ⓔʁʑɯnɯ}{\papi{ ʁʑɯnɯ}}}\markboth{ʁʑɯnɯ}{}
\classe{n}
\begin{définition}\fra jeune homme\end{définition}
\begin{définition}\cmn 小伙子,青年
\begin{déclaration} \étymologie{\papi{gʑon.nu}}\end{déclaration}\end{définition}
\begin{relation-sémantique}\confer{
\hyperlink{Ⓔnɯʁʑɯnɯ}{\textit{ \papi{nɯʁʑɯnɯ}}}
}\end{relation-sémantique}\end{entrée}

\newpage\caractère{s}

\begin{entrée}
\vedette{\hypertarget{Ⓔsa}{\papi{ sa}}}\markboth{sa}{}
\classe{vs}
\paradigme{\textit{dir :} \jya nɯ-}
\paradigme{\textit{dir :} \jya tɤ-}
\begin{définition}\fra s'émousser\end{définition}
\begin{définition}\cmn 磨损;变钝\end{définition}
\begin{exemple}\jya mbrɯtɕɯ to-sa\cmn 刀变钝了\end{exemple}
\begin{exemple}\jya qraʁ ɲɤ-sa\cmn 铧变钝了\end{exemple}
\begin{relation-sémantique}\synonyme{
\hyperlink{Ⓔzɤt}{\textit{ \papi{zɤt}}}
}\end{relation-sémantique}\begin{sous-entrée}
\vedette{\hypertarget{}{\papi{ ɣɤsa}}}\markboth{ɣɤsa}{}\classe{vs}\acception{1}
\begin{définition}\fra s'émousser facilement\end{définition}
\begin{définition}\cmn 容易变钝\end{définition}\acception{2}
\begin{définition}\fra qui diminue vite\end{définition}
\begin{définition}\cmn 减少得快(食物;衣服)\end{définition}
\begin{exemple}\jya tɯ-mgo ɲɯ-ɣɤsa\cmn 饭很快吃完(不经吃)\end{exemple}
\begin{exemple}\jya nɤ-ɕoʁɕoʁ ɲɯ-ɣɤsa\cmn 你纸巾用得很快\end{exemple}
\begin{relation-sémantique}\synonyme{
\hyperlink{Ⓔɣɤrɕo}{\textit{ \papi{ɣɤrɕo}}}
}\end{relation-sémantique}
\end{sous-entrée}\begin{sous-entrée}
\vedette{\hypertarget{}{\papi{ sɯxsa}}}\markboth{sɯxsa}{}
\paradigme{\textit{dir :} \jya nɯ-}
\paradigme{\textit{dir :} \jya tɤ-}
\begin{définition}\fra émousser\end{définition}
\begin{définition}\cmn 弄钝\end{définition}
\begin{exemple}\jya mbrɯtɕɯ nɯ-sɯxsa-t-a\cmn 我把刀弄钝了\end{exemple}\classe{vt}
\end{sous-entrée}\end{entrée}

\begin{entrée}
\vedette{\hypertarget{Ⓔsaɕɯ}{\papi{ saɕɯ}}}\markboth{saɕɯ}{}
\classe{n}
\begin{définition}\fra mélèze\end{définition}
\begin{définition}\cmn 落叶松\end{définition}
\begin{exemple}\jya saɕɯ nɯ ɯ-tshɯɣa tɯrgi ɲɯ-fse, ɯ-mdoʁ nɯ ra ɕɤɣ ɲɯ-fse, ɯ-ru nɯ tɯrgi ɲɯ-fse, tɕe tɯrgi cho ɕɤɣ ni pjɯ-nɯɕɯrɲɟo-ndʑi mɤ-cha, qartsɯmɤftɕar arŋi-ndʑi. saɕɯ pjɯ-nɯɕɯrɲɟo tɕe, ɯ-jwaʁ pjɤ-ŋgra ɕti, tɕe tɯrme ra kɯ saɕɯ nɯ tɯrgi ɣɯ ɯ-ftsa ŋu ri tɯrgi sɤz kɯ-mbro tu-ɬoʁ ŋu tɕe, nɯ mɤ-kɯ-tʂaŋ ɯ-ndʐa kɯ pjɯ-nɯɕɯrɲɟo ŋu tu-ti-nɯ ŋu.\cmn 落叶松的形状像杉树,但是颜色和柏树相同。树干像杉树的一样。杉树和柏树叶子秋天不会变色,不分冬夏都是绿的,落叶松叶子秋天变色,然后落叶。人们说落叶松虽然是杉树的侄子但生长的地区海拔比杉树生长的地区高,因为这样不合理,所以它叶子秋天变色。\end{exemple}
\begin{exemple}\jya saɕɯ cho tɯrgi nɯ ni kɤndʑɯrpɯftsa ɲɯ-ŋu-ndʑi. ndʑi-sta nɯnɯ, ɯ-rpɯ ɯ-sta saɕɯ lo-ɕe tɕe, qartsɯ tɕe kɯ-nɯɕɯrɲɟo ŋu. tɯrgi cho-maŋthi tɕe núndʐa nɯnɯ tɯrgi mɤ-nɯɕɯrɲɟo tu-kɯ-ti ɲɯ-ŋu.\cmn 落叶松和杉树是舅甥的关系,落叶松去了舅舅的位子所以到了秋天变色,杉树到了下游所以不变色\end{exemple}\end{entrée}

\begin{entrée}
\vedette{\hypertarget{Ⓔsakaβ}{\papi{ sakaβ}}}\markboth{sakaβ}{}
\classe{n}
\begin{définition}\fra puits\end{définition}
\begin{définition}\cmn 井\end{définition}
\begin{relation-sémantique}\confer{
\hyperlink{Ⓔkaβ}{\textit{ \papi{kaβ}}}
}\end{relation-sémantique}\end{entrée}

\begin{entrée}
\vedette{\hypertarget{Ⓔsakhar}{\papi{ sakhar}}}\markboth{sakhar}{}
\classe{n}
\begin{définition}\fra porcherie\end{définition}
\begin{définition}\cmn 猪圈\end{définition}\end{entrée}

\begin{entrée}
\vedette{\hypertarget{Ⓔsalaboŋboŋ}{\papi{ salaboŋboŋ}}}\markboth{salaboŋboŋ}{}
\classe{n}
\begin{définition}\fra vesse-de-loup\end{définition}
\begin{définition}\cmn 马勃\end{définition}
\begin{exemple}\jya salaboŋboŋ nɯ stɤmku ri tu-ɬoʁ ŋu tɕe, kɯ-wɣrum ŋu kɯ-ɤrtɯm rloʁ rloʁ ŋu ɯ-ru me, wuma ʑo mpɯ, nɯ-rom tɕe pjɯ́-wɣ-rɤtɕaʁ tɕe rdɯl ɲɯ-nɯɬoʁ ŋu.\cmn 马勃长在草地上,全白色,呈球形,没有杆,很软,干了以后踩上去就会扬起灰尘(孢子粉)来。\end{exemple}
\begin{relation-sémantique}\synonyme{
\hyperlink{Ⓔɬɤndʐithamaka}{\textit{ \papi{ɬɤndʐithamaka}}}
}\end{relation-sémantique}\end{entrée}

\begin{entrée}
\vedette{\hypertarget{Ⓔsalakɯndzur}{\papi{ salakɯndzur}}}\markboth{salakɯndzur}{}\classe{n}
\begin{définition}\fra galipette\end{définition}
\begin{définition}\cmn (翻)筋斗\end{définition}
\begin{exemple}\jya a-tɕɯ kɯ salakɯndzur ta-βzu\cmn 我儿子翻了筋斗\end{exemple}\end{entrée}

\begin{entrée}
\vedette{\hypertarget{Ⓔsalaʁdɯʁdɯɣ}{\papi{ salaʁdɯʁdɯɣ}}}\markboth{salaʁdɯʁdɯɣ}{}\classe{n}
\begin{définition}\fra (personne, chose) qui gêne\end{définition}
\begin{définition}\cmn 碍事的(人;东西)\end{définition}
\begin{relation-sémantique}\confer{
\hyperlink{Ⓔsala,zrɯ}{\textit{ \papi{sala,zrɯ}}}
}\end{relation-sémantique}\end{entrée}

\begin{entrée}
\vedette{\hypertarget{Ⓔsala,zrɯ}{\papi{ sala,zrɯ}}}\markboth{sala,zrɯ}{}\paradigme{\textit{dir :} \jya kɤ-}
\begin{définition}\fra prendre de la place, gêner\end{définition}
\begin{définition}\cmn 占(不该占的)地方\end{définition}
\begin{exemple}\jya nɯtɕu tɤ-rɤru ma sala ɲɯ-tɯ-zri ɲɯ-ŋu\cmn 起来,你在那里很碍事\end{exemple}
\begin{exemple}\jya sala kɯ-zrɯ ma-tɯ-βze\cmn 别占地方\end{exemple}
\begin{relation-sémantique}\ComponentA{\classe{n}
 \papi{sala}
}\end{relation-sémantique}
\begin{relation-sémantique}\ComponentB{\classe{vt}
\hyperlink{ⒺzrɯⒽ1}{\textit{ \papi{zrɯ}}}
}\end{relation-sémantique}
\begin{relation-sémantique}\confer{
\hyperlink{Ⓔsalaʁdɯʁdɯɣ}{\textit{ \papi{salaʁdɯʁdɯɣ}}}
}\end{relation-sémantique}\end{entrée}

\begin{entrée}
\vedette{\hypertarget{Ⓔsali}{\papi{ sali}}}\markboth{sali}{}
\classe{n}
\begin{définition}\fra arbalète\end{définition}
\begin{définition}\cmn 弩弓\end{définition}\end{entrée}

\begin{entrée}
\vedette{\hypertarget{Ⓔsaloŋ}{\papi{ saloŋ}}}\markboth{saloŋ}{}
\classe{adv}
\begin{définition}\fra partout\end{définition}
\begin{définition}\cmn 到处\end{définition}
\begin{définition}\cmn 他走遍了所有的地方\end{définition}
\begin{exemple}\jya saloŋ ri ʑo ɕ-to-khɤt\end{exemple}
\begin{relation-sémantique}\confer{
\hyperlink{Ⓔlaloŋ}{\textit{ \papi{laloŋ}}}
}\end{relation-sémantique}\end{entrée}

\begin{entrée}
\vedette{\hypertarget{Ⓔsan}{\papi{ san}}}\markboth{san}{}\classe{n}
\begin{définition}\fra parapluie\end{définition}
\begin{définition}\cmn 伞\end{définition}
\end{entrée}

\begin{entrée}
\vedette{\hypertarget{Ⓔsaŋdi}{\papi{ saŋdi}}}\markboth{saŋdi}{}
\classe{n}
\begin{définition}\fra place des serviteurs, à l'ouest\end{définition}
\begin{définition}\cmn 下等人坐的地方(往西方)\end{définition}\end{entrée}

\begin{entrée}
\vedette{\hypertarget{Ⓔsaŋrɟɤz}{\papi{ saŋrɟɤz}}}\markboth{saŋrɟɤz}{}\classe{n}
\begin{définition}\fra bouddha\end{définition}
\begin{définition}\cmn 佛,神仙
\begin{déclaration} \étymologie{\papi{saŋs.rgʲas}}\end{déclaration}\end{définition}
\end{entrée}

\begin{entrée}
\vedette{\hypertarget{Ⓔsaqru}{\papi{ saqru}}}\markboth{saqru}{}
\begin{relation-sémantique}\confer{
\hyperlink{Ⓔqru}{\textit{ \papi{qru}}}
}\end{relation-sémantique}\end{entrée}

\begin{entrée}
\vedette{\hypertarget{Ⓔsar}{\papi{ sar}}}\markboth{sar}{}
\classe{vt}
\paradigme{\textit{dir :} \jya pɯ-}
\begin{définition}\fra filtrer\end{définition}
\begin{définition}\cmn 滤\end{définition}
\begin{exemple}\jya mbrɤz pɯ-sar\cmn 你把米过滤一下\end{exemple}
\begin{exemple}\jya tɤ-lu pɯ-sar\cmn 你把牛奶过滤一下\end{exemple}\end{entrée}

\begin{entrée}
\vedette{\hypertarget{Ⓔsarndzu}{\papi{ sarndzu}}}\markboth{sarndzu}{}\classe{n}
\begin{définition}\ 
\begin{déclaration}\grammar{n.lieu}\end{déclaration}\end{définition}
\begin{définition}\fra Gsar.rdzong\end{définition}
\begin{définition}\cmn 沙尔宗乡\end{définition}\end{entrée}

\begin{entrée}
\vedette{\hypertarget{Ⓔsarsi}{\papi{ sarsi}}}\markboth{sarsi}{}
\classe{n}
\begin{définition}\fra abricot\end{définition}
\begin{définition}\cmn 杏\end{définition}
\begin{exemple}\jya sarsi nɯ zgo kɯ-mbɤr tu-ɬoʁ ŋu, si mbro ɯ-rtaʁ dɤn, ɯ-βri nɯ kɯ-pɣi ŋu, ɯ-jwaʁ nɯ qaɕti ɯ-jwaʁ cho naχtɕɯɣ, ɯ-mɯntoʁ kɯ-ɣɯrni ɯ-ŋgɯz kɯnɤ kɯ-wɣrum tsa ŋu. ɯ-jwaʁ ɲɯ-lɤt ɕɯŋgɯ ɯ-mɯntoʁ ɲɯ-lɤt ŋu. ɯ-mat thɯ-aβzu tɕe, qaɕti tsa fse ri ndɯβ. thɯ-tɯt tɕe, ɯ-phaʁ ntsi ɣɯrni, ɯ-phaʁ ntsi qarŋe, tú-wɣ-ndza qaɕti sɤz mɯm ma chi tɕe, kɤ́rqhɯrqhu kɤ-ndza sna. sarsi nɯ kha ɯ-rkɯ pɯ-kɤ-nɯ-ji tɕi tu, ɕɯŋgɯ zɯ ɯʑo tu-kɯ-nɯ-ɬoʁ tɕi tu.\cmn 杏树生长在下半山上,长得很高,枝桠多,树皮是灰色的,叶子和桃树的叶子一样,花是粉红色的。在长叶子之前就开花。结果像桃子,但小一些。成熟,半边是黄色,半边是红色的。吃起来比桃子好吃,因为很甜,可以连皮一起吃。杏树,有的种在房子旁边,也有自己生长在野外的。\end{exemple}
\begin{relation-sémantique}\confer{
\hyperlink{Ⓔnɯsarsi}{\textit{ \papi{nɯsarsi}}}
}\end{relation-sémantique}\end{entrée}

\begin{entrée}
\vedette{\hypertarget{Ⓔsarwɯ}{\papi{ sarwɯ}}}\markboth{sarwɯ}{}
\classe{n}
\begin{définition}\fra élément du métier à tisser\end{définition}
\begin{définition}\cmn 搓杆(纺锤的木棒)\end{définition}\end{entrée}

\begin{entrée}
\vedette{\hypertarget{Ⓔsaʁ}{\papi{ saʁ}}}\markboth{saʁ}{}
\classe{vt}
\paradigme{\textit{dir :} \jya nɯ-}
\begin{définition}\fra séparer des fils emmêlés\end{définition}
\begin{définition}\cmn 撕开\end{définition}
\begin{exemple}\jya smɤɣ nɯ-saʁ-a\cmn 我撕开了羊毛\end{exemple}
\begin{exemple}\jya tɤ-rme nɯ-saʁ-a\cmn 我撕开了毛\end{exemple}\end{entrée}

\begin{entrée}
\vedette{\hypertarget{Ⓔsaʁdɤt}{\papi{ saʁdɤt}}}\markboth{saʁdɤt}{}\classe{vs}
\paradigme{\textit{dir :} \jya tɤ-}
\begin{définition}\ 
\begin{déclaration}\grammar{deexp}\end{déclaration}\end{définition}
\begin{définition}\fra glissant\end{définition}
\begin{définition}\cmn 滑(路)\end{définition}
\begin{exemple}\jya a-pɯ-zbaʁ tɕe mɤ-saʁdɤt, tɯ-mɯ a-pɯ-lɤt tɕe saʁdɤt\cmn 地干就不滑,下雨的话就滑\end{exemple}
\begin{exemple}\jya tɯ-mɯ pjɤ-lɤt tɕe ɲɯ-saʁdɤt\cmn 下雨了,地很滑\end{exemple}
\begin{exemple}\jya qaɟy ɲɯ-saʁdɤt\cmn 鱼很滑\end{exemple}
\begin{relation-sémantique}\confer{
\hyperlink{Ⓔaʁdɤt}{\textit{ \papi{aʁdɤt}}}
}\end{relation-sémantique}
\begin{relation-sémantique}\synonyme{
\hyperlink{Ⓔsɤŋgio}{\textit{ \papi{sɤŋgio}}}
}\end{relation-sémantique}\end{entrée}

\begin{entrée}
\vedette{\hypertarget{Ⓔsaʁdɯɣ}{\papi{ saʁdɯɣ}}}\markboth{saʁdɯɣ}{}
\classe{vs}
\paradigme{\textit{dir :} \jya tɤ-}
\begin{définition}\ 
\begin{déclaration}\grammar{deexp}\end{déclaration}\end{définition}
\begin{définition}\fra ennuyer, empêcher\end{définition}
\begin{définition}\cmn 干扰;防碍\end{définition}
\begin{exemple}\jya tɤ-rɤru ma ɲɯ-tɯ-saʁdɯɣ\cmn 你起来,你在那里碍事\end{exemple}\end{entrée}

\begin{entrée}
\vedette{\hypertarget{Ⓔsaʁjɤr}{\papi{ saʁjɤr}}}\markboth{saʁjɤr}{}
\begin{relation-sémantique}\confer{
\hyperlink{Ⓔaʁjɤr}{\textit{ \papi{aʁjɤr}}}
}\end{relation-sémantique}\end{entrée}

\begin{entrée}
\vedette{\hypertarget{Ⓔsaʁjɯβ}{\papi{ saʁjɯβ}}}\markboth{saʁjɯβ}{}\classe{vt}
\paradigme{\textit{dir :} \jya tɤ-}
\begin{définition}\fra cacher\end{définition}
\begin{définition}\cmn 遮住,遮掩\end{définition}
\begin{exemple}\jya a-mɤ-pɯ́-wɣ-mto nɯ-sɯso-t-a tɕe tɤ-saʁjɯβ-a\cmn 为了不让别人发现,我把它遮掩了\end{exemple}
\begin{relation-sémantique}\synonyme{
\hyperlink{Ⓔsqaβjɯβ}{\textit{ \papi{sqaβjɯβ}}}
}\end{relation-sémantique}
\begin{relation-sémantique}\confer{
 \papi{naʁjɯɣ}
}\end{relation-sémantique}
\begin{relation-sémantique}\confer{
\hyperlink{Ⓔta-ʁjɯβ}{\textit{ \papi{ta-ʁjɯβ}}}
}\end{relation-sémantique}\end{entrée}

\begin{entrée}
\vedette{\hypertarget{Ⓔsaʁɟa}{\papi{ saʁɟa}}}\markboth{saʁɟa}{}
\begin{relation-sémantique}\confer{
\hyperlink{Ⓔaʁɟa}{\textit{ \papi{aʁɟa}}}
}\end{relation-sémantique}\end{entrée}

\begin{entrée}
\vedette{\hypertarget{Ⓔsaʁlɤt}{\papi{ saʁlɤt}}}\markboth{saʁlɤt}{}
\classe{vt}
\paradigme{\textit{dir :} \jya lɤ-}
\paradigme{\textit{dir :} \jya pɯ-}
\begin{définition}\fra vanner\end{définition}
\begin{définition}\cmn 扬场\end{définition}
\begin{exemple}\jya tɤ-rɤku pɯ-saʁlat-a (=qale pɯ-lat-a)\cmn 我扬了粮食\end{exemple}
\begin{exemple}\jya stoʁ lɤ-saʁlat-a\cmn 我扬了胡豆\end{exemple}\end{entrée}

\begin{entrée}
\vedette{\hypertarget{Ⓔsaʁnɤt}{\papi{ saʁnɤt}}}\markboth{saʁnɤt}{}\classe{vs}
\paradigme{\textit{dir :} \jya thɯ-}\acception{1}
\begin{définition}\fra faire mal\end{définition}
\begin{définition}\cmn 伤到\end{définition}
\begin{exemple}\jya a-ɕa ɯ-taʁ ɲɯ-saʁnɤt\cmn 伤到我了\end{exemple}\acception{2}
\begin{définition}\fra faire du mal\end{définition}
\begin{définition}\cmn 伤害
\begin{déclaration} \étymologie{\papi{gnod}}\end{déclaration}\end{définition}
\begin{exemple}\jya a-taʁ pɯ-saʁnɤt\cmn 伤害我了\end{exemple}
\begin{relation-sémantique}\confer{
\hyperlink{Ⓔʁnɤt}{\textit{ \papi{ʁnɤt}}}
}\end{relation-sémantique}\end{entrée}

\begin{entrée}
\vedette{\hypertarget{Ⓔsaʁre}{\papi{ saʁre}}}\markboth{saʁre}{}\classe{vs}
\paradigme{\textit{dir :} \jya tɤ-}
\begin{définition}\fra austère, impressionnant\end{définition}
\begin{définition}\cmn 严肃,庄重\end{définition}
\begin{relation-sémantique}\confer{
\hyperlink{Ⓔɣɤʁre}{\textit{ \papi{ɣɤʁre}}}
}\end{relation-sémantique}
\begin{relation-sémantique}\confer{
 \papi{ɯ-rʁe}
}\end{relation-sémantique}\end{entrée}

\begin{entrée}
\vedette{\hypertarget{Ⓔsaʁrɯm}{\papi{ saʁrɯm}}}\markboth{saʁrɯm}{}\classe{vt}
\paradigme{\textit{dir :} \jya tɤ-}
\begin{définition}\ 
\begin{déclaration}\grammar{denom}\end{déclaration}\end{définition}
\begin{définition}\fra cacher, couvrir\end{définition}
\begin{définition}\cmn 遮住\end{définition}
\begin{exemple}\jya nɤ-tɤŋe tɤ-saʁrɯm-a\cmn 我帮你遮了太阳\end{exemple}
\begin{exemple}\jya tɤ-rte tɤ-ŋge tɕe tɤŋe tɤ-saʁrɯm\cmn 你戴上帽子,遮住太阳\end{exemple}
\begin{exemple}\jya ɯ-kɯ-saʁrɯm ɲo-me tɕe to-sɤmto\cmn 遮住它的东西没有了,别人就看得见它了\end{exemple}
\begin{relation-sémantique}\confer{
\hyperlink{Ⓔta-ʁrɯm}{\textit{ \papi{ta-ʁrɯm}}}
}\end{relation-sémantique}\end{entrée}

\begin{entrée}
\vedette{\hypertarget{Ⓔsat}{\papi{ sat}}}\markboth{sat}{}\classe{vt}
\paradigme{\textit{dir :} \jya pɯ-}
\begin{définition}\fra tuer\end{définition}
\begin{définition}\cmn 杀\end{définition}\begin{sous-entrée}
\vedette{\hypertarget{}{\papi{ asɯsat}}}\markboth{asɯsat}{}\classe{vi}
\begin{définition}\ 
\begin{déclaration}\grammar{refl}\end{déclaration}\end{définition}
\begin{définition}\fra se tuer les uns les autres\end{définition}
\begin{définition}\cmn 互相残杀\end{définition}
\begin{exemple}\jya pjɤ-k-ɤsɯsat-ndʑi-ci\cmn 他们俩互相残杀了\end{exemple}
\end{sous-entrée}\begin{sous-entrée}
\vedette{\hypertarget{}{\papi{ rɤsat}}}\markboth{rɤsat}{}\classe{vi}
\begin{définition}\ 
\begin{déclaration}\grammar{apass}\end{déclaration}\end{définition}
\begin{définition}\fra tuer des animaux\end{définition}
\begin{définition}\cmn 杀动物\end{définition}
\begin{relation-sémantique}\confer{
 \papi{sɤsat1}
}\end{relation-sémantique}
\end{sous-entrée}\begin{sous-entrée}
\vedette{\hypertarget{}{\papi{ sɯsat}}}\markboth{sɯsat}{}\classe{vt}
\paradigme{\textit{dir :} \jya pɯ-}
\begin{définition}\ 
\begin{déclaration}\grammar{caus}\end{déclaration}\end{définition}\acception{1}
\begin{définition}\fra faire tuer\end{définition}
\begin{définition}\cmn 使人杀(另一个人)\end{définition}\acception{2}
\begin{définition}\fra tuer avec\end{définition}
\begin{définition}\cmn 用……杀\end{définition}
\end{sous-entrée}\begin{sous-entrée}
\vedette{\hypertarget{}{\papi{ ʑɣɤsɯsat}}}\markboth{ʑɣɤsɯsat}{}\classe{vi}
\paradigme{\textit{dir :} \jya pɯ-}
\begin{définition}\ 
\begin{déclaration}\grammar{caus}\end{déclaration}
\begin{déclaration}\grammar{refl}\end{déclaration}\end{définition}
\begin{définition}\fra se faire tuer\end{définition}
\begin{définition}\cmn 遭人杀害\end{définition}
\end{sous-entrée}\end{entrée}

\begin{entrée}
\vedette{\hypertarget{Ⓔsaχaʁ}{\papi{ saχaʁ}}}\markboth{saχaʁ}{}\classe{vs}
\paradigme{\textit{dir :} \jya nɯ-}
\begin{définition}\fra être extrême\end{définition}
\begin{définition}\cmn 极度的\end{définition}
\begin{relation-sémantique}\confer{
\hyperlink{Ⓔnaχaʁ}{\textit{ \papi{naχaʁ}}}
}\end{relation-sémantique}
\end{entrée}

\begin{entrée}
\vedette{\hypertarget{Ⓔsaχɕɤra}{\papi{ saχɕɤra}}}\markboth{saχɕɤra}{}
\classe{vt}
\paradigme{\textit{dir :} \jya nɯ-}
\begin{définition}\fra étendre\end{définition}
\begin{définition}\cmn 铺平;弄得平整;陈展\end{définition}
\begin{exemple}\jya tɯ-sta ɯ-taʁ @pugai nɯ-saχɕɤre\cmn 你把铺盖在床上铺平\end{exemple}\end{entrée}

\begin{entrée}
\vedette{\hypertarget{Ⓔsaχɕɯβ}{\papi{ saχɕɯβ}}}\markboth{saχɕɯβ}{}
\classe{vt}
\paradigme{\textit{dir :} \jya tɤ-}
\begin{définition}\ 
\begin{déclaration}\grammar{caus}\end{déclaration}\end{définition}
\begin{définition}\fra enrouler les brins (d'une corde)\end{définition}
\begin{définition}\cmn 把几股绳子 合并拧在一起、弄成双股\end{définition}
\begin{exemple}\jya tɯmbri tɤ-saχɕɯβ-i\cmn 我们把绳子拧在一起了\end{exemple}
\begin{exemple}\jya tɯmbri χsɯ-ldʑa ʑo to-saχɕɯβ tɕe ɲɯ-ngɯt\cmn 他把三股绳子拧在一起了,这样很结实。\end{exemple}
\begin{exemple}\jya stukɤr thɯ-saχɕɯβ-i\cmn 我们把梁弄成双根\end{exemple}
\begin{relation-sémantique}\confer{
\hyperlink{Ⓔaχɕɯβ}{\textit{ \papi{aχɕɯβ}}}
}\end{relation-sémantique}
\begin{relation-sémantique}\confer{
\hyperlink{Ⓔnɯsaχɕɯβ}{\textit{ \papi{nɯsaχɕɯβ}}}
}\end{relation-sémantique}\end{entrée}

\begin{entrée}
\vedette{\hypertarget{Ⓔsaχɕɯn}{\papi{ saχɕɯn}}}\markboth{saχɕɯn}{}
\classe{vs}
\paradigme{\textit{dir :} \jya nɯ-}
\begin{définition}\fra frais, propre, hygiénique\end{définition}
\begin{définition}\cmn 新鲜;卫生\end{définition}
\begin{exemple}\jya tɤ-mthɯm mɯ́j-saχɕɯn, ɲɯ-saχɕɯn\cmn 肉不新鲜,很新鲜\end{exemple}
\begin{exemple}\jya qaɟy mɯ́j-saχɕɯn\cmn 鱼不新鲜\end{exemple}
\begin{relation-sémantique}\confer{
\hyperlink{Ⓔnaχɕɯn}{\textit{ \papi{naχɕɯn}}}
}\end{relation-sémantique}\end{entrée}

\begin{entrée}
\vedette{\hypertarget{Ⓔsaχpaʁ}{\papi{ saχpaʁ}}}\markboth{saχpaʁ}{}\classe{vt}
\paradigme{\textit{dir :} \jya tɤ-}
\begin{définition}\fra respecter\end{définition}
\begin{définition}\cmn 尊重\end{définition}
\begin{exemple}\jya jiɕqha nɯ ɲɯ-ɣɤʁre tɕe, kɤ-saχpaʁ ɲɯ-sna\cmn 那个人是个受尊重的人,要尊重他\end{exemple}
\begin{exemple}\jya aʑo ɣɯ-saχpaʁ-a\cmn 他尊重我\end{exemple}\begin{sous-entrée}
\vedette{\hypertarget{}{\papi{ asaχpɯχpaʁ}}}\markboth{asaχpɯχpaʁ}{}\classe{vi}
\begin{définition}\ 
\begin{déclaration}\grammar{refl}\end{déclaration}\end{définition}
\begin{définition}\fra se respecter les uns les autres\end{définition}
\begin{définition}\cmn 互相尊重\end{définition}
\begin{exemple}\jya tɕiʑo ni ʑɤŋgɯz asaχpɯχpaʁ-tɕi\cmn 我们俩互相尊重\end{exemple}
\end{sous-entrée}\begin{sous-entrée}
\vedette{\hypertarget{}{\papi{ sɤsaχpaʁ}}}\markboth{sɤsaχpaʁ}{}\classe{vs}
\begin{définition}\ 
\begin{déclaration}\grammar{deexp}\end{déclaration}\end{définition}
\begin{définition}\fra être respecté\end{définition}
\begin{définition}\cmn 受尊重\end{définition}
\begin{exemple}\jya jiɕqha nɯ ɲɯ-sɤsaχpaʁ\cmn 那个人受人尊重\end{exemple}
\end{sous-entrée}\end{entrée}

\begin{entrée}
\vedette{\hypertarget{ⒺsaχsɤlⒽ1Ⓗ1}{\papi{ saχsɤl}}}\markboth{saχsɤl}{}\homonyme{1}\classe{vs}
\paradigme{\textit{dir :} \jya tɤ-}
\begin{définition}\fra clair, évident\end{définition}
\begin{définition}\cmn 明显\end{définition}
\begin{exemple}\jya to-saχsɤl\cmn 变得很明显\end{exemple}\begin{sous-entrée}
\vedette{\hypertarget{}{\papi{ saχsɤl}}}\markboth{saχsɤl}{} (\variante{sɯsaχsɤl}) \classe{vt}
\begin{définition}\fra révéler\end{définition}
\begin{définition}\cmn 透露出,让……显现出来\end{définition}
\begin{exemple}\jya kɯ-ŋu ɯ-kɯ-ŋu nɯ kɤ-sɯsaχsɤl ra\cmn 要把真相透露出去\end{exemple}
\end{sous-entrée}\end{entrée}

\begin{entrée}
\vedette{\hypertarget{Ⓔsaχsi}{\papi{ saχsi}}}\markboth{saχsi}{}
\classe{vt}
\paradigme{\textit{dir :} \jya nɯ-}
\paradigme{\textit{dir :} \jya tɤ-}
\begin{définition}\fra faire complètement\end{définition}
\begin{définition}\cmn 做得彻底\end{définition}
\begin{exemple}\jya tɤ-rɤku nɯ-saχsi-t-a\cmn 我把粮食跟石头泥巴分开了\end{exemple}
\begin{exemple}\jya nɤ-khɯtsa tɤ-saχsi\cmn 你把碗里的东西吃完!\end{exemple}
\begin{exemple}\jya ɕ-kɤ-nɯzʁe tɤ-saχsi\cmn 你把东西搬过去,不要漏掉一些(你给我搬得干净彻底)\end{exemple}
\begin{exemple}\jya mɯ-ɲɤ-saχsi\cmn 没有做得彻底\end{exemple}
\begin{exemple}\jya kɤntɕhaʁ kɯ-rɤma ɲɯ-dɤn, kɤ-saχsi mɯ́jkhɯ\cmn 街上清道夫很多,但是不能清洁干净\end{exemple}
\begin{relation-sémantique}\confer{
\hyperlink{Ⓔaχsi}{\textit{ \papi{aχsi}}}
}\end{relation-sémantique}\end{entrée}

\begin{entrée}
\vedette{\hypertarget{Ⓔsaχsom}{\papi{ saχsom}}}\markboth{saχsom}{}
\begin{relation-sémantique}\confer{
\hyperlink{Ⓔaχsom}{\textit{ \papi{aχsom}}}
}\end{relation-sémantique}\end{entrée}

\begin{entrée}
\vedette{\hypertarget{ⒺsaχsɯⒽ1}{\papi{ saχsɯ}}}\markboth{saχsɯ}{}\homonyme{1}\classe{n}
\begin{définition}\fra repas de midi\end{définition}
\begin{définition}\cmn 中午饭,午饭\end{définition}
\begin{exemple}\jya nɤʑo nɤ-saχsɯ ɯ-pɯ́-tu?\cmn 你吃中午饭了没有\end{exemple}
\begin{exemple}\jya pɤjkhu a-saχsɯ pɯ-me\cmn 还没有吃中午餐\end{exemple}\end{entrée}

\begin{entrée}
\vedette{\hypertarget{ⒺsaχsɯⒽ2}{\papi{ saχsɯ}}}\markboth{saχsɯ}{}\homonyme{2}
\classe{n}
\begin{définition}\fra brosse pour laver la vaisselle\end{définition}
\begin{définition}\cmn 耍把\end{définition}\end{entrée}

\begin{entrée}
\vedette{\hypertarget{Ⓔsaχsɯko}{\papi{ saχsɯko}}}\markboth{saχsɯko}{}
\begin{relation-sémantique}\confer{
\hyperlink{Ⓔaχsɯko}{\textit{ \papi{aχsɯko}}}
}\end{relation-sémantique}\end{entrée}

\begin{entrée}
\vedette{\hypertarget{Ⓔsaχthɯm}{\papi{ saχthɯm}}}\markboth{saχthɯm}{}
\classe{vt}
\paradigme{\textit{dir :} \jya tɤ-}
\begin{définition}\fra puiser avec\end{définition}
\begin{définition}\cmn 用……舀(水、颗粒)\end{définition}
\begin{exemple}\jya khɯtsa ta-saχthɯm\cmn 他用碗舀了\end{exemple}
\begin{exemple}\jya scoʁ ta-saχthɯm\cmn 他用瓢子舀了\end{exemple}\end{entrée}

\begin{entrée}
\vedette{\hypertarget{Ⓔsaχti}{\papi{ saχti}}}\markboth{saχti}{}\classe{vs}
\begin{définition}\fra aimable\end{définition}
\begin{définition}\cmn 很好相处,合得来\end{définition}
\begin{relation-sémantique}\synonyme{
\hyperlink{Ⓔsɤzda}{\textit{ \papi{sɤzda}}}
}\end{relation-sémantique}
\begin{relation-sémantique}\confer{
\hyperlink{Ⓔtɯ-χti}{\textit{ \papi{tɯ-χti}}}
}\end{relation-sémantique}
\begin{relation-sémantique}\confer{
\hyperlink{Ⓔnaχti}{\textit{ \papi{naχti}}}
}\end{relation-sémantique}
\end{entrée}

\begin{entrée}
\vedette{\hypertarget{Ⓔsɤβdaʁ}{\papi{ sɤβdaʁ}}}\markboth{sɤβdaʁ}{}\classe{n}
\begin{définition}\fra divinité locale\end{définition}
\begin{définition}\cmn 地方神
\begin{déclaration} \étymologie{\papi{sa.bdag}}\end{déclaration}\end{définition}\end{entrée}

\begin{entrée}
\vedette{\hypertarget{Ⓔsɤβdɤβde}{\papi{ sɤβdɤβde}}}\markboth{sɤβdɤβde}{}
\begin{relation-sémantique}\confer{
\hyperlink{Ⓔaβdɤβde}{\textit{ \papi{aβdɤβde}}}
}\end{relation-sémantique}\end{entrée}

\begin{entrée}
\vedette{\hypertarget{Ⓔsɤβlo}{\papi{ sɤβlo}}}\markboth{sɤβlo}{}
\classe{vt}
\paradigme{\textit{dir :} \jya nɯ-}
\begin{définition}\fra s'occuper de (à propos des enfants)\end{définition}
\begin{définition}\cmn 带;看(孩子)\end{définition}
\begin{exemple}\jya aʑo a-ɣe ku-sɤβlam-a\cmn 我在带我的孙子\end{exemple}
\begin{relation-sémantique}\confer{
\hyperlink{Ⓔnɤpɤβdaʁ}{\textit{ \papi{nɤpɤβdaʁ}}}
}\end{relation-sémantique}\end{entrée}

\begin{entrée}
\vedette{\hypertarget{Ⓔsɤβlɯβlɯɣ}{\papi{ sɤβlɯβlɯɣ}}}\markboth{sɤβlɯβlɯɣ}{}
\begin{relation-sémantique}\confer{
\hyperlink{Ⓔɣɤβlɯβlɯɣ}{\textit{ \papi{ɣɤβlɯβlɯɣ}}}
}\end{relation-sémantique}\end{entrée}

\begin{entrée}
\vedette{\hypertarget{Ⓔsɤβri}{\papi{ sɤβri}}}\markboth{sɤβri}{}
\begin{relation-sémantique}\confer{
\hyperlink{Ⓔβri}{\textit{ \papi{βri}}}
}\end{relation-sémantique}\end{entrée}

\begin{entrée}
\vedette{\hypertarget{Ⓔsɤβzu}{\papi{ sɤβzu}}}\markboth{sɤβzu}{}\classe{vt}\acception{1}
\paradigme{\textit{dir :} \jya tɤ-}
\paradigme{\textit{dir :} \jya nɯ-}
\begin{définition}\fra préparer, faire une sorte que..., rendre...\end{définition}
\begin{définition}\cmn 准备,把...做成,让……变成\end{définition}
\begin{exemple}\jya kɤ-ŋga to-sɤβzu\cmn 他把衣服准备好了,可以穿了\end{exemple}
\begin{exemple}\jya kɤ-ndza to-sɤβzu\cmn 他把食物准备好了,可以吃了\end{exemple}
\begin{exemple}\jya tɤɕi kɤ-rŋu tɤ-sɤβzu-t-a\cmn 我把青稞准备好了,可以炒了\end{exemple}
\begin{exemple}\jya tɤɕi tɯsqar nɯ-sɤβzu-t-a\end{exemple}
\begin{exemple}\jya tɤ-pɤtso kɤ-ntɕhoz kɯ-sna nɯ-sɤβzu-t-a\cmn 我把孩子培养成有用的人了\end{exemple}
\begin{exemple}\jya tɤ-mu nɯ kɯ paʁ kɤ-ntɕha to-sɤβzu\cmn 那位大娘把猪喂肥了,可以宰了\end{exemple}
\begin{exemple}\jya jiʑora kɯnɤ tu-kɤ-ndza kɯ-me ɲɯ-tɯ-sɤβze ɲɯ-ŋu\cmn 你令我们没有东西吃\end{exemple}
\begin{exemple}\jya jiʑora kɤ-nɯʑɯβ mɤ-kɯ-khɯ tu-tɯ-sɤβze ɲɯ-ŋu\cmn 你令我们无法睡觉\end{exemple}
\begin{relation-sémantique}\confer{
\hyperlink{ⒺβzuⒽ1}{\textit{ \papi{βzu1}}}
}\end{relation-sémantique}
\begin{relation-sémantique}\confer{
\hyperlink{Ⓔaβzu}{\textit{ \papi{aβzu}}}
}\end{relation-sémantique}\end{entrée}

\begin{entrée}
\vedette{\hypertarget{Ⓔsɤβzdoʁβzdɯ}{\papi{ sɤβzdoʁβzdɯ}}}\markboth{sɤβzdoʁβzdɯ}{}\classe{vt}
\paradigme{\textit{dir :} \jya tɤ-}
\begin{définition}\fra rassembler\end{définition}
\begin{définition}\cmn 聚集\end{définition}
\begin{exemple}\jya laχtɕha ra tɤ-sɤβzdoʁβzdɯ-t-a\cmn 我把东西聚在一起了\end{exemple}
\begin{exemple}\jya fsapaʁ ra tɤ-sɤβzdoʁβzdɯ-t-a\cmn 我把牲畜聚在一起了\end{exemple}
\begin{relation-sémantique}\confer{
\hyperlink{Ⓔaβzdoʁβzdɯ}{\textit{ \papi{aβzdoʁβzdɯ}}}
}\end{relation-sémantique}\end{entrée}

\begin{entrée}
\vedette{\hypertarget{Ⓔsɤβzi}{\papi{ sɤβzi}}}\markboth{sɤβzi}{}
\begin{relation-sémantique}\confer{
\hyperlink{Ⓔβzi}{\textit{ \papi{βzi}}}
}\end{relation-sémantique}\end{entrée}

\begin{entrée}
\vedette{\hypertarget{Ⓔsɤβzɯβzu}{\papi{ sɤβzɯβzu}}}\markboth{sɤβzɯβzu}{}\classe{vt}
\paradigme{\textit{dir :} \jya tɤ-}
\begin{définition}\fra se débrouiller, broder un peu une histoire\end{définition}
\begin{définition}\cmn 想办法;把故事的一些情节编一点\end{définition}
\begin{exemple}\jya tɤ-thu-t-a, ɯ-kɯ-spa maŋe tɕe, aʑo tɤ-nɯsɤβzɯβzu-t-a\cmn 我问了,没人知道,我就自己编了一些\end{exemple}
\begin{relation-sémantique}\confer{
\hyperlink{Ⓔsɤβzu}{\textit{ \papi{sɤβzu}}}
}\end{relation-sémantique}\end{entrée}

\begin{entrée}
\vedette{\hypertarget{Ⓔsɤcɤrlu}{\papi{ sɤcɤrlu}}}\markboth{sɤcɤrlu}{}
\begin{relation-sémantique}\confer{
\hyperlink{Ⓔacɤrlu}{\textit{ \papi{acɤrlu}}}
}\end{relation-sémantique}
\end{entrée}

\begin{entrée}
\vedette{\hypertarget{Ⓔsɤcha}{\papi{ sɤcha}}}\markboth{sɤcha}{}\classe{vs}
\paradigme{\textit{dir :} \jya tɤ-}
\begin{définition}\ 
\begin{déclaration}\grammar{deexp}\end{déclaration}\end{définition}
\begin{définition}\fra être possible\end{définition}
\begin{définition}\cmn 可能\end{définition}
\begin{exemple}\jya ku-kɯ-ɕe ɲɯ-sɤcha\cmn (这个地方)可以去\end{exemple}
\begin{exemple}\jya kɤ-fkɯr ɲɯ-sɤcha\cmn (这个东西)背得了\end{exemple}
\begin{exemple}\jya zgo ɯ-tɯ-mbro nɯ, kɤ-ɕe mɤ-sɤcha\cmn 山高得没人能上去\end{exemple}
\begin{exemple}\jya tɤɕi tɯ-lʁa kɤ-fkur sɤcha\cmn 一般来说,一袋青稞,人家背得走\end{exemple}
\begin{exemple}\jya kɯ-ɤntɤm kɤ-nɤrɟɯɣrɟɯɣ sɤcha, tɤton kɤ-rɟɯɣ mɤ-sɤcha\cmn 在平路跑步还是受得了,在山坡跑步就受不了了\end{exemple}
\begin{exemple}\jya tú-wɣ-fkur sɤcha\end{exemple}
\begin{relation-sémantique}\confer{
\hyperlink{ⒺchaⒽ1}{\textit{ \papi{cha1}}}
}\end{relation-sémantique}\end{entrée}

\begin{entrée}
\vedette{\hypertarget{Ⓔsɤchɯchɯβ}{\papi{ sɤchɯchɯβ}}}\markboth{sɤchɯchɯβ}{}
\classe{vt}
\paradigme{\textit{dir :} \jya tɤ-}
\begin{définition}\fra foncer sans se préoccuper de rien (cheval), manger à toute vitesse\end{définition}
\begin{définition}\cmn 不顾一切地闯过去(马)、吃得很急\end{définition}
\begin{exemple}\jya tɤ-sɤchɯchɯβ ʑo tɕe, jɤ-ɕe\cmn 你快点吃就走\end{exemple}
\begin{exemple}\jya tɤ-sɤchɯchɯβ-a lɤ-ari-a\cmn 我不顾一切地冲上去了\end{exemple}\end{entrée}

\begin{entrée}
\vedette{\hypertarget{Ⓔsɤchɯrʁu}{\papi{ sɤchɯrʁu}}}\markboth{sɤchɯrʁu}{}
\begin{relation-sémantique}\confer{
\hyperlink{Ⓔachɯrʁu}{\textit{ \papi{achɯrʁu}}}
}\end{relation-sémantique}\end{entrée}

\begin{entrée}
\vedette{\hypertarget{Ⓔsɤci}{\papi{ sɤci}}}\markboth{sɤci}{}
\begin{relation-sémantique}\confer{
\hyperlink{Ⓔaci}{\textit{ \papi{aci}}}
}\end{relation-sémantique}\end{entrée}

\begin{entrée}
\vedette{\hypertarget{Ⓔsɤcɯ}{\papi{ sɤcɯ}}}\markboth{sɤcɯ}{}
\classe{n}
\begin{définition}\fra clé\end{définition}
\begin{définition}\cmn 钥匙\end{définition}
\begin{relation-sémantique}\confer{
\hyperlink{ⒺcɯⒽ1}{\textit{ \papi{cɯ1}}}
}\end{relation-sémantique}\end{entrée}

\begin{entrée}
\vedette{\hypertarget{Ⓔsɤcɯqhlɯβ}{\papi{ sɤcɯqhlɯβ}}}\markboth{sɤcɯqhlɯβ}{}
\begin{relation-sémantique}\confer{
\hyperlink{Ⓔɣɤcɯqhlɯβ}{\textit{ \papi{ɣɤcɯqhlɯβ}}}
}\end{relation-sémantique}\end{entrée}

\begin{entrée}
\vedette{\hypertarget{Ⓔsɤɕaβ}{\papi{ sɤɕaβ}}}\markboth{sɤɕaβ}{}
\begin{relation-sémantique}\confer{
\hyperlink{ⒺɕaβⒽ1}{\textit{ \papi{ɕaβ1}}}
}\end{relation-sémantique}\end{entrée}

\begin{entrée}
\vedette{\hypertarget{Ⓔsɤɕar}{\papi{ sɤɕar}}}\markboth{sɤɕar}{}
\begin{relation-sémantique}\confer{
\hyperlink{Ⓔɕar}{\textit{ \papi{ɕar}}}
}\end{relation-sémantique}\end{entrée}

\begin{entrée}
\vedette{\hypertarget{Ⓔsɤɕɤt}{\papi{ sɤɕɤt}}}\markboth{sɤɕɤt}{}
\classe{vt}
\paradigme{\textit{dir :} \jya thɯ-}
\paradigme{\textit{dir :} \jya pɯ-}
\begin{définition}\fra peigner\end{définition}
\begin{définition}\cmn 梳
\begin{déclaration} \étymologie{\papi{ɕad}}\end{déclaration}\end{définition}
\begin{exemple}\jya nɤ-ku thɯ-sɤɕɤt\cmn 你梳一下头\end{exemple}
\begin{exemple}\jya ɯ-ku ɲɯ-ɤʁzrɤwolu, pɯ-sɤɕat-a\cmn 他头发很乱,所以我给他梳了头\end{exemple}
\begin{relation-sémantique}\confer{
\hyperlink{Ⓔtɤɕɤt}{\textit{ \papi{tɤɕɤt}}}
}\end{relation-sémantique}\end{entrée}

\begin{entrée}
\vedette{\hypertarget{Ⓔsɤɕkɤɣɕkɤɣ}{\papi{ sɤɕkɤɣɕkɤɣ}}}\markboth{sɤɕkɤɣɕkɤɣ}{}
\begin{relation-sémantique}\confer{
\hyperlink{Ⓔɣɤɕkɤɣɕkɤɣ}{\textit{ \papi{ɣɤɕkɤɣɕkɤɣ}}}
}\end{relation-sémantique}\end{entrée}

\begin{entrée}
\vedette{\hypertarget{ⒺsɤɕkeⒽ1}{\papi{ sɤɕke}}}\markboth{sɤɕke}{}\homonyme{1}\classe{vs}
\paradigme{\textit{dir :} \jya tɤ-}
\begin{définition}\fra brûlant\end{définition}
\begin{définition}\cmn 烫\end{définition}
\begin{exemple}\jya ki kɯ-sɤɕke sthɯci ɲɯ-maʁ\cmn 没有那么烫\end{exemple}
\begin{exemple}\jya kɯ-sɤɕke tɤ-ndze !\cmn 趁热吃!\end{exemple}
\begin{relation-sémantique}\confer{
\hyperlink{Ⓔʑɣɤsɤɕke}{\textit{ \papi{ʑɣɤsɤɕke}}}
}\end{relation-sémantique}\begin{sous-entrée}
\vedette{\hypertarget{}{\papi{ nɤsɤɕke}}}\markboth{nɤsɤɕke}{}\classe{vt}
\begin{définition}\fra trouver brûlant\end{définition}
\begin{définition}\cmn 觉得烫\end{définition}
\begin{exemple}\jya ɲɯ-nɤsɤɕke-a\cmn 我觉得很烫\end{exemple}
\end{sous-entrée}\end{entrée}

\begin{entrée}
\vedette{\hypertarget{ⒺsɤɕkeⒽ2}{\papi{ sɤɕke}}}\markboth{sɤɕke}{}\homonyme{2}
\classe{vt}
\paradigme{\textit{dir :} \jya kɤ-}
\begin{définition}\fra brûler\end{définition}
\begin{définition}\cmn 烤焦\end{définition}
\begin{exemple}\jya qajɣi ko-sɤɕke-t-a\cmn 我不小心把馍馍烤焦了\end{exemple}
\begin{relation-sémantique}\confer{
\hyperlink{Ⓔɕke}{\textit{ \papi{ɕke}}}
}\end{relation-sémantique}\end{entrée}

\begin{entrée}
\vedette{\hypertarget{Ⓔsɤɕoχɕi}{\papi{ sɤɕoχɕi}}}\markboth{sɤɕoχɕi}{}
\begin{relation-sémantique}\confer{
\hyperlink{Ⓔaɕoχɕi}{\textit{ \papi{aɕoχɕi}}}
}\end{relation-sémantique}\end{entrée}

\begin{entrée}
\vedette{\hypertarget{Ⓔsɤɕphɤɣɕphɤɣ}{\papi{ sɤɕphɤɣɕphɤɣ}}}\markboth{sɤɕphɤɣɕphɤɣ}{}
\classe{vt}
\begin{définition}\fra faire claquer\end{définition}
\begin{définition}\cmn 令(衣服)啪啪响\end{définition}
\begin{exemple}\jya qale kɯ tɯ-ŋga ɲɯ-sɤɕphɤɣɕphɤɣ ʑo\cmn 风吹,令衣服啪啪地响\end{exemple}
\begin{relation-sémantique}\confer{
\hyperlink{Ⓔɕphɤɣnɤɕphɤɣ}{\textit{ \papi{ɕphɤɣnɤɕphɤɣ}}}
}\end{relation-sémantique}\end{entrée}

\begin{entrée}
\vedette{\hypertarget{Ⓔsɤɕprɯm}{\papi{ sɤɕprɯm}}}\markboth{sɤɕprɯm}{}
\begin{relation-sémantique}\confer{
\hyperlink{Ⓔaɕprɯm}{\textit{ \papi{aɕprɯm}}}
}\end{relation-sémantique}\end{entrée}

\begin{entrée}
\vedette{\hypertarget{Ⓔsɤɕpɯɕpa}{\papi{ sɤɕpɯɕpa}}}\markboth{sɤɕpɯɕpa}{}
\begin{relation-sémantique}\confer{
\hyperlink{Ⓔaɕpɯɕpa}{\textit{ \papi{aɕpɯɕpa}}}
}\end{relation-sémantique}\end{entrée}

\begin{entrée}
\vedette{\hypertarget{Ⓔsɤɕqa}{\papi{ sɤɕqa}}}\markboth{sɤɕqa}{}
\classe{vs}
\paradigme{\textit{dir :} \jya tɤ-}
\begin{définition}\fra être supportable\end{définition}
\begin{définition}\cmn 受得了\end{définition}
\begin{exemple}\jya a-χpɯm ɲɯ-mŋɤm ri, mɤ-kɯ-sɤɕqa maŋe\cmn 我的膝盖痛但是没有什么受不了的\end{exemple}
\begin{exemple}\jya ɕɤxɕo rcanɯ, tɯ-mɯ mɯ́j-lɤt tɕe, ɯ-tɯ-sɤɕke kɯ mɯ́j-sɤɕqa ʑo\cmn 这几天没有下雨,天气热得使人受不了\end{exemple}
\begin{relation-sémantique}\confer{
\hyperlink{Ⓔnɤɕqa}{\textit{ \papi{nɤɕqa}}}
}\end{relation-sémantique}\end{entrée}

\begin{entrée}
\vedette{\hypertarget{Ⓔsɤɕqali}{\papi{ sɤɕqali}}}\markboth{sɤɕqali}{}
\begin{relation-sémantique}\confer{
\hyperlink{Ⓔɣɤɕqali}{\textit{ \papi{ɣɤɕqali}}}
}\end{relation-sémantique}\end{entrée}

\begin{entrée}
\vedette{\hypertarget{Ⓔsɤɕqhe}{\papi{ sɤɕqhe}}}\markboth{sɤɕqhe}{}
\begin{relation-sémantique}\confer{
\hyperlink{Ⓔaɕqhe}{\textit{ \papi{aɕqhe}}}
}\end{relation-sémantique}\end{entrée}

\begin{entrée}
\vedette{\hypertarget{Ⓔsɤɕtar}{\papi{ sɤɕtar}}}\markboth{sɤɕtar}{}
\begin{relation-sémantique}\confer{
\hyperlink{Ⓔnɯɕtar}{\textit{ \papi{nɯɕtar}}}
}\end{relation-sémantique}
\end{entrée}

\begin{entrée}
\vedette{\hypertarget{Ⓔsɤɕtɕɯɣ}{\papi{ sɤɕtɕɯɣ}}}\markboth{sɤɕtɕɯɣ}{}\classe{n}
\begin{définition}\fra lanière pour porter les enfants sur le dos\end{définition}
\begin{définition}\cmn 背孩子的带子
\end{définition}\end{entrée}

\begin{entrée}
\vedette{\hypertarget{Ⓔsɤɕte}{\papi{ sɤɕte}}}\markboth{sɤɕte}{}
\begin{relation-sémantique}\confer{
\hyperlink{Ⓔɕte}{\textit{ \papi{ɕte}}}
}\end{relation-sémantique}\end{entrée}

\begin{entrée}
\vedette{\hypertarget{Ⓔsɤɕtʂaŋlaŋ}{\papi{ sɤɕtʂaŋlaŋ}}}\markboth{sɤɕtʂaŋlaŋ}{}
\begin{relation-sémantique}\confer{
\hyperlink{Ⓔɕtʂaŋɕtʂaŋ}{\textit{ \papi{ɕtʂaŋɕtʂaŋ}}}
}\end{relation-sémantique}\end{entrée}

\begin{entrée}
\vedette{\hypertarget{Ⓔsɤɕtʂɯlɯɣ}{\papi{ sɤɕtʂɯlɯɣ}}}\markboth{sɤɕtʂɯlɯɣ}{}
\begin{relation-sémantique}\confer{
\hyperlink{Ⓔɕtʂɯɣɕtʂɯɣ}{\textit{ \papi{ɕtʂɯɣɕtʂɯɣ}}}
}\end{relation-sémantique}\end{entrée}

\begin{entrée}
\vedette{\hypertarget{Ⓔsɤɕɯmɕɯm}{\papi{ sɤɕɯmɕɯm}}}\markboth{sɤɕɯmɕɯm}{}
\begin{relation-sémantique}\confer{
\hyperlink{Ⓔɕɯmɕɯm}{\textit{ \papi{ɕɯmɕɯm}}}
}\end{relation-sémantique}\end{entrée}

\begin{entrée}
\vedette{\hypertarget{Ⓔsɤdʐaŋlaŋ}{\papi{ sɤdʐaŋlaŋ}}}\markboth{sɤdʐaŋlaŋ}{}
\classe{vt}
\paradigme{\textit{dir :} \jya tɤ-}
\begin{définition}\fra balancer\end{définition}
\begin{définition}\cmn 甩来甩去\end{définition}
\begin{exemple}\jya tɤ-sɤdʐaŋlaŋ\cmn 他(把这个东西)甩来甩去了\end{exemple}
\begin{relation-sémantique}\synonyme{
\hyperlink{Ⓔsɤɕtʂaŋlaŋ}{\textit{ \papi{sɤɕtʂaŋlaŋ}}}
}\end{relation-sémantique}\end{entrée}

\begin{entrée}
\vedette{\hypertarget{Ⓔsɤdɤmɲi}{\papi{ sɤdɤmɲi}}}\markboth{sɤdɤmɲi}{}
\begin{relation-sémantique}\confer{
\hyperlink{Ⓔmɲi}{\textit{ \papi{mɲi}}}
}\end{relation-sémantique}\end{entrée}

\begin{entrée}
\vedette{\hypertarget{Ⓔsɤdoŋdoŋ}{\papi{ sɤdoŋdoŋ}}}\markboth{sɤdoŋdoŋ}{}
\begin{relation-sémantique}\confer{
\hyperlink{Ⓔɣɤdoŋdoŋ}{\textit{ \papi{ɣɤdoŋdoŋ}}}
}\end{relation-sémantique}\end{entrée}

\begin{entrée}
\vedette{\hypertarget{Ⓔsɤdrɤt}{\papi{ sɤdrɤt}}}\markboth{sɤdrɤt}{}
\begin{relation-sémantique}\confer{
\hyperlink{Ⓔadrɤt}{\textit{ \papi{adrɤt}}}
}\end{relation-sémantique}\end{entrée}

\begin{entrée}
\vedette{\hypertarget{Ⓔsɤdɯxpa}{\papi{ sɤdɯxpa}}}\markboth{sɤdɯxpa}{}
\begin{relation-sémantique}\confer{
\hyperlink{Ⓔsɤzdɯxpa}{\textit{ \papi{sɤzdɯxpa}}}
}\end{relation-sémantique}
\end{entrée}

\begin{entrée}
\vedette{\hypertarget{Ⓔsɤdzɯlɯt}{\papi{ sɤdzɯlɯt}}}\markboth{sɤdzɯlɯt}{}\classe{vt}
\paradigme{\textit{dir :} \jya nɯ-}
\paradigme{\textit{dir :} \jya tɤ-}
\begin{définition}\fra agiter\end{définition}
\begin{définition}\cmn 使扭动\end{définition}
\begin{relation-sémantique}\confer{
\hyperlink{Ⓔɣɤdzɯlɯt}{\textit{ \papi{ɣɤdzɯlɯt}}}
}\end{relation-sémantique}\end{entrée}

\begin{entrée}
\vedette{\hypertarget{Ⓔsɤdʑɯɣdʑɯɣ}{\papi{ sɤdʑɯɣdʑɯɣ}}}\markboth{sɤdʑɯɣdʑɯɣ}{}
\begin{sous-entrée}
\vedette{\hypertarget{}{\papi{ znɤdʑɯɣdʑɯɣ}}}\markboth{znɤdʑɯɣdʑɯɣ}{}\classe{vi}
\paradigme{\textit{dir :} \jya tɤ-}
\begin{définition}\fra donner un petit coup (avec un bâton)\end{définition}
\begin{définition}\cmn (用棍子)戳一下\end{définition}
\begin{exemple}\jya tɤɲi kɯ ɲɤ-znɤdʑɯɣdʑɯɣ\end{exemple}
\begin{exemple}\jya laχtɕha ko-znɤdʑɯɣdʑɯɣ\cmn 他用拐棍戳那个东西\end{exemple}
\end{sous-entrée}\end{entrée}

\begin{entrée}
\vedette{\hypertarget{Ⓔsɤfɕu}{\papi{ sɤfɕu}}}\markboth{sɤfɕu}{}
\begin{relation-sémantique}\confer{
\hyperlink{Ⓔafɕu}{\textit{ \papi{afɕu}}}
}\end{relation-sémantique}\end{entrée}

\begin{entrée}
\vedette{\hypertarget{Ⓔsɤfɕɤra}{\papi{ sɤfɕɤra}}}\markboth{sɤfɕɤra}{}
\classe{vt}\acception{1}
\paradigme{\textit{dir :} \jya tɤ-}
\begin{définition}\fra discuter\end{définition}
\begin{définition}\cmn 讨论;议论
\begin{déclaration}\use{一般用于复数}\end{déclaration}\end{définition}
\begin{exemple}\jya ɲɤ-sɤfɕɤra-nɯ\cmn 他们商量了\end{exemple}
\begin{exemple}\jya tɤ-sɤfɕɤra-j\cmn 我们商量了\end{exemple}
\begin{exemple}\jya kɤ-nɤma tɕhi tu-kɤ-stu nɯra tɕiʑo tɤ-sɤfɕɤra-tɕi\cmn 我们商量了怎么做\end{exemple}
\begin{exemple}\jya jɤ-ɣi-nɯ tɕe sɤfɕɤra-j\cmn 你们来,我们讨论一下\end{exemple}\acception{2}
\paradigme{\textit{dir :} \jya nɯ-}
\begin{définition}\fra dévoiler (une information non publique)\end{définition}
\begin{définition}\cmn 传出去\end{définition}
\begin{exemple}\jya ma-nɯ́-wɣ-sɤfɕɤra\cmn 不要公开\end{exemple}\end{entrée}

\begin{entrée}
\vedette{\hypertarget{Ⓔsɤfɕi}{\papi{ sɤfɕi}}}\markboth{sɤfɕi}{}
\classe{vi}
\paradigme{\textit{dir :} \jya nɯ-}
\begin{définition}\fra être enceinte (après le septième mois de grossesse)\end{définition}
\begin{définition}\cmn 怀孕(快要生;怀孕七月以后)\end{définition}
\begin{exemple}\jya jiɕqha tɕheme ɲɤ-sɤfɕi, ɯ-xtu jɤrjɤr ɲɤ-pa\cmn 那个女子怀孕了,肚子变得很沉\end{exemple}\end{entrée}

\begin{entrée}
\vedette{\hypertarget{Ⓔsɤfka}{\papi{ sɤfka}}}\markboth{sɤfka}{}
\begin{relation-sémantique}\confer{
\hyperlink{ⒺfkaⒽ1}{\textit{ \papi{fka}}}
}\end{relation-sémantique}\end{entrée}

\begin{entrée}
\vedette{\hypertarget{Ⓔsɤfsu}{\papi{ sɤfsu}}}\markboth{sɤfsu}{}
\classe{vt}\acception{1}
\paradigme{\textit{dir :} \jya tɤ-}
\begin{définition}\fra comparer la taille\end{définition}
\begin{définition}\cmn 比较长短\end{définition}
\begin{exemple}\jya a-jaʁndzu tɤ-sɤfsu-t-a ri mɯ-ɲɯ-ɤfsɯfsu\cmn 我比了一下手指的长短,结果长短不一\end{exemple}
\begin{exemple}\jya xtɯrɲɟi ta-sɤfsu\cmn 他比了一下长短\end{exemple}
\begin{exemple}\jya tɕi-mthɯxtɕɤr tɤ-sɤfsu-t-a\cmn 我比了我们俩的腰带有多长\end{exemple}\acception{2}
\paradigme{\textit{dir :} \jya nɯ-}
\begin{définition}\fra uniformiser la taille\end{définition}
\begin{définition}\cmn 弄得一样长\end{définition}
\begin{exemple}\jya χtsiɯ kɯ tɯjpu tɤ́-wɣ-sɯ-ɕtʂo tɕe, ɯ-mŋu nɯ raŋri ʑo ɲɯ́-wɣ-sɤfsu ra\cmn 用瓢子量粮食的时候,要把每一个瓢的口抹平(确定每一瓢相等)\end{exemple}
\begin{relation-sémantique}\confer{
\hyperlink{Ⓔɯ-fsu}{\textit{ \papi{ɯ-fsu}}}
}\end{relation-sémantique}
\begin{relation-sémantique}\confer{
\hyperlink{Ⓔsɤfsuja}{\textit{ \papi{sɤfsuja}}}
}\end{relation-sémantique}
\begin{relation-sémantique}\confer{
\hyperlink{Ⓔafsɯfsu}{\textit{ \papi{afsɯfsu}}}
}\end{relation-sémantique}\end{entrée}

\begin{entrée}
\vedette{\hypertarget{Ⓔsɤfse}{\papi{ sɤfse}}}\markboth{sɤfse}{}\classe{adv}
\begin{définition}\fra comme\end{définition}
\begin{définition}\cmn 很像\end{définition}
\begin{exemple}\jya nɤʑo tɤ-pɤtso sɤfse kɯ kɯ-chi ɲɯ-tɯ-rga\cmn 你很像小孩子,喜欢吃糖\end{exemple}
\begin{exemple}\jya nɤʑo aʑo sɤfse kɯ tɤndʐo mɯ́j-tɯ-schi\cmn 你很像我,很怕冷\end{exemple}
\begin{relation-sémantique}\confer{
\hyperlink{ⒺfseⒽ1}{\textit{ \papi{fse1}}}
}\end{relation-sémantique}\end{entrée}

\begin{entrée}
\vedette{\hypertarget{Ⓔsɤfsuja}{\papi{ sɤfsuja}}}\markboth{sɤfsuja}{}\classe{vt}
\paradigme{\textit{dir :} \jya tɤ-}
\begin{définition}\fra comparer la taille\end{définition}
\begin{définition}\cmn 比较长短\end{définition}
\begin{exemple}\jya tɕi-mthɯxtɕɤr tɤ-sɤfsuja-t-a\cmn 我比了我们俩的腰带有多长\end{exemple}
\begin{exemple}\jya ki tɯmbri ni ŋotɕu nɯ ɲɯ-rɲɟi kɯ tu-sɤfsuje-a\cmn 我把这两根绳子比较一下,看哪一根长\end{exemple}
\begin{relation-sémantique}\synonyme{
\hyperlink{Ⓔsɤfsu}{\textit{ \papi{sɤfsu}}}
}\end{relation-sémantique}\end{entrée}

\begin{entrée}
\vedette{\hypertarget{Ⓔsɤfstɯn}{\papi{ sɤfstɯn}}}\markboth{sɤfstɯn}{}
\begin{relation-sémantique}\confer{
\hyperlink{Ⓔfstɯn}{\textit{ \papi{fstɯn}}}
}\end{relation-sémantique}
\end{entrée}

\begin{entrée}
\vedette{\hypertarget{Ⓔsɤfsɯfse}{\papi{ sɤfsɯfse}}}\markboth{sɤfsɯfse}{}
\begin{relation-sémantique}\confer{
\hyperlink{ⒺfseⒽ2}{\textit{ \papi{fse2}}}
}\end{relation-sémantique}
\end{entrée}

\begin{entrée}
\vedette{\hypertarget{Ⓔsɤftɕaka}{\papi{ sɤftɕaka}}}\markboth{sɤftɕaka}{}
\classe{vt}
\paradigme{\textit{dir :} \jya tɤ-}
\begin{définition}\fra préparer\end{définition}
\begin{définition}\cmn 准备;收拾\end{définition}
\begin{exemple}\jya to-sɤftɕaka\cmn 他准备了东西\end{exemple}
\begin{exemple}\jya laχtɕha tɤ-sɤftɕaka-t-a\cmn 我准备了东西\end{exemple}
\begin{exemple}\jya kɤ-ndza tɤ-sɤftɕaka-t-a\cmn 我准备了食物\end{exemple}
\begin{exemple}\jya tʂha lɤ-sɤftɕaka-t-a\cmn 我准备了茶\end{exemple}
\begin{relation-sémantique}\confer{
\hyperlink{Ⓔftɕaka}{\textit{ \papi{ftɕaka}}}
}\end{relation-sémantique}\end{entrée}

\begin{entrée}
\vedette{\hypertarget{Ⓔsɤftɕaʁ}{\papi{ sɤftɕaʁ}}}\markboth{sɤftɕaʁ}{}
\begin{relation-sémantique}\confer{
\hyperlink{Ⓔftɕaʁ}{\textit{ \papi{ftɕaʁ}}}
}\end{relation-sémantique}
\end{entrée}

\begin{entrée}
\vedette{\hypertarget{Ⓔsɤftɕɤl}{\papi{ sɤftɕɤl}}}\markboth{sɤftɕɤl}{}
\begin{relation-sémantique}\confer{
\hyperlink{Ⓔftɕɤl}{\textit{ \papi{ftɕɤl}}}
}\end{relation-sémantique}\end{entrée}

\begin{entrée}
\vedette{\hypertarget{Ⓔsɤglɤglɤɣ}{\papi{ sɤglɤglɤɣ}}}\markboth{sɤglɤglɤɣ}{}\classe{vi}
\paradigme{\textit{dir :} \jya tɤ-}
\begin{définition}\fra frapper en faisant du bruit\end{définition}
\begin{définition}\cmn 敲得很响;震动\end{définition}
\begin{relation-sémantique}\confer{
\hyperlink{Ⓔɣɤglɤglɤɣ}{\textit{ \papi{ɣɤglɤglɤɣ}}}
}\end{relation-sémantique}
\begin{relation-sémantique}\confer{
\hyperlink{Ⓔglɤɣglɤɣ}{\textit{ \papi{glɤɣglɤɣ}}}
}\end{relation-sémantique}\end{entrée}

\begin{entrée}
\vedette{\hypertarget{Ⓔsɤgrɤl}{\papi{ sɤgrɤl}}}\markboth{sɤgrɤl}{}
\classe{n}
\begin{définition}\fra limite\end{définition}
\begin{définition}\cmn 界限
\begin{déclaration} \étymologie{\papi{gral}}\end{déclaration}\end{définition}\end{entrée}

\begin{entrée}
\vedette{\hypertarget{Ⓔsɤɣa}{\papi{ sɤɣa}}}\markboth{sɤɣa}{}
\classe{vs}
\paradigme{\textit{dir :} \jya tɤ-}
\begin{définition}\fra endroit, chemin non dangereux\end{définition}
\begin{définition}\cmn 安全;平坦的路;地方\end{définition}
\begin{exemple}\jya tʂu ɲɯ-sɤɣa\cmn 路很安全\end{exemple}
\begin{exemple}\jya stɤmku ɲɯ-sɤɣa\cmn 草坪很平坦\end{exemple}
\begin{relation-sémantique}\confer{
\hyperlink{Ⓔnɤɣa}{\textit{ \papi{nɤɣa}}}
}\end{relation-sémantique}\end{entrée}

\begin{entrée}
\vedette{\hypertarget{Ⓔsɤɣɤmɯ}{\papi{ sɤɣɤmɯ}}}\markboth{sɤɣɤmɯ}{}
\begin{relation-sémantique}\confer{
\hyperlink{Ⓔɣɤmɯ}{\textit{ \papi{ɣɤmɯ}}}
}\end{relation-sémantique}\end{entrée}

\begin{entrée}
\vedette{\hypertarget{Ⓔsɤɣdoŋɣdoŋ}{\papi{ sɤɣdoŋɣdoŋ}}}\markboth{sɤɣdoŋɣdoŋ}{}
\begin{relation-sémantique}\confer{
\hyperlink{Ⓔɣdoŋnɤɣdoŋ}{\textit{ \papi{ɣdoŋnɤɣdoŋ}}}
}\end{relation-sémantique}
\end{entrée}

\begin{entrée}
\vedette{\hypertarget{Ⓔsɤɣdɯɣ}{\papi{ sɤɣdɯɣ}}}\markboth{sɤɣdɯɣ}{}\classe{vs}
\paradigme{\textit{dir :} \jya tɤ-}
\begin{définition}\fra être désagréable\end{définition}
\begin{définition}\cmn 讨厌; 不舒服\end{définition}
\begin{exemple}\jya a-mgɯr ɯ-qhu ma-tɯ-ɣɤjɤβjɤβ ɲɯ-tɯ-sɤɣdɯɣ\cmn 你不要乱摸我的背部,你很讨厌\end{exemple}
\begin{exemple}\jya a-phe ma-tɯ-ɣɤsɯɣsɯɣ, ɲɯ-sɤɣdɯɣ\cmn 你不要在我身边乱动,很讨厌\end{exemple}
\begin{exemple}\jya ɲɯ-nɯtɕhomba-a, ɲɯ-sɤɣdɯɣ\cmn 我感冒了,很不舒服\end{exemple}
\begin{exemple}\jya a-mgɯr ɯ-qhu ɲɯ-sɤɣdɯɣ\cmn 我背上不舒服\end{exemple}
\begin{exemple}\jya a-sɯm ɲɯ-sɤɣdɯɣ\cmn 我心里不舒服\end{exemple}
\begin{exemple}\jya a-qhoχpa ɲɯ-sɤɣdɯɣ\cmn 我心里不舒服\end{exemple}
\begin{exemple}\jya a-xtu ɲɯ-sɤɣdɯɣ\cmn 我肚子不舒服\end{exemple}
\begin{relation-sémantique}\confer{
\hyperlink{Ⓔnɤsɤɣdɯɣ}{\textit{ \papi{nɤsɤɣdɯɣ}}}
}\end{relation-sémantique}\end{entrée}

\begin{entrée}
\vedette{\hypertarget{Ⓔsɤɣmu}{\papi{ sɤɣmu}}}\markboth{sɤɣmu}{}
\classe{vs}
\paradigme{\textit{dir :} \jya tɤ-}
\paradigme{\textit{dir :} \jya thɯ-}
\begin{définition}\ 
\begin{déclaration}\grammar{deexp}\end{déclaration}\end{définition}
\begin{définition}\fra terrifiant\end{définition}
\begin{définition}\cmn 恐怖;可怕\end{définition}
\begin{exemple}\jya jiɕqha kɯ-sɤɣmu ci ɲɯ-ŋu\cmn 那个很恐怖\end{exemple}
\begin{exemple}\jya jla ɲɯ-sɤtɕhɯ tɕe ɲɯ-sɤɣmu\cmn 犏牛顶人很可怕\end{exemple}
\begin{exemple}\jya khɯna ɲɯ-sɤmtsɯɣ tɕe ɲɯ-sɤɣmu\cmn 狗咬人很可怕\end{exemple}
\begin{exemple}\jya cho-sɤɣmu\cmn 变得很恐怖\end{exemple}
\begin{relation-sémantique}\confer{
\hyperlink{ⒺmuⒽ1}{\textit{ \papi{mu1}}}
}\end{relation-sémantique}\end{entrée}

\begin{entrée}
\vedette{\hypertarget{Ⓔsɤɣɲat}{\papi{ sɤɣɲat}}}\markboth{sɤɣɲat}{}
\classe{vs}
\begin{définition}\ 
\begin{déclaration}\grammar{deexp}\end{déclaration}\end{définition}
\begin{définition}\fra fatigant\end{définition}
\begin{définition}\cmn 令人很累\end{définition}
\begin{exemple}\jya ki kɤ-nɤma ki wuma ɲɯ-sɤɣɲat\cmn 这种工作很累人\end{exemple}
\begin{relation-sémantique}\confer{
\hyperlink{Ⓔɲat}{\textit{ \papi{ɲat}}}
}\end{relation-sémantique}\end{entrée}

\begin{entrée}
\vedette{\hypertarget{Ⓔsɤɣur}{\papi{ sɤɣur}}}\markboth{sɤɣur}{}
\classe{vt}
\paradigme{\textit{dir :} \jya tɤ-}
\paradigme{\textit{dir :} \jya kɤ-}
\paradigme{\textit{dir :} \jya pɯ-}
\begin{définition}\fra couvrir de tous les côtés (mais pas le dessus), bloquer\end{définition}
\begin{définition}\cmn 遮拦四周(但没有遮住上面)、挡住去路\end{définition}
\begin{exemple}\jya fsapaʁ tɤ-sɤɣur-a (=kɤ-ja-t-a)\cmn 我把牲畜围起来了\end{exemple}
\begin{exemple}\jya sɤxɕe me ma kɤ-sɤɣur-a\cmn 没有去路,因为被我挡住了\end{exemple}
\begin{exemple}\jya kumpɣa ɕɯ-nɤru ɲɯ-ŋu tɕe, (tɯjpu) tɤ-sɤɣur-a\cmn 鸡要去偷吃粮食,我就把(粮食)围起来了\end{exemple}
\begin{exemple}\jya rdɤstaʁ kɯ smi kɤ-sɤɣur-i\cmn 我用石头把火围起来了\end{exemple}
\begin{relation-sémantique}\confer{
\hyperlink{Ⓔtɤ-ɣur}{\textit{ \papi{tɤ-ɣur}}}
}\end{relation-sémantique}\end{entrée}

\begin{entrée}
\vedette{\hypertarget{Ⓔsɤɣɯrɣɯr}{\papi{ sɤɣɯrɣɯr}}}\markboth{sɤɣɯrɣɯr}{}
\begin{relation-sémantique}\confer{
\hyperlink{Ⓔɣɤɣɯrɣɯr}{\textit{ \papi{ɣɤɣɯrɣɯr}}}
}\end{relation-sémantique}\end{entrée}

\begin{entrée}
\vedette{\hypertarget{Ⓔsɤɣʑɯr}{\papi{ sɤɣʑɯr}}}\markboth{sɤɣʑɯr}{}\classe{vs}
\paradigme{\textit{dir :} \jya tɤ-}
\begin{définition}\fra dangereux\end{définition}
\begin{définition}\cmn 危险\end{définition}
\end{entrée}

\begin{entrée}
\vedette{\hypertarget{Ⓔsɤja}{\papi{ sɤja}}}\markboth{sɤja}{}\classe{vt}
\paradigme{\textit{dir :} \jya nɯ-}
\begin{définition}\fra rendre\end{définition}
\begin{définition}\cmn 还东西\end{définition}
\begin{exemple}\jya laʁtɕha ɲɤ-nɯβde-t-a tɕe nɯ́-wɣ-sɤja-a\cmn 我把东西弄丢了,他送还给我了\end{exemple}
\begin{exemple}\jya ɲɤ-nɯβde tɕe nɯ-sɤja-t-a\cmn 他把东西弄丢了,我送还给他了\end{exemple}
\begin{exemple}\jya nɤ-laʁtɕha nɯ-ta-sɤja\cmn 我已经把你的东西还给你了\end{exemple}\end{entrée}

\begin{entrée}
\vedette{\hypertarget{Ⓔsɤjɤr}{\papi{ sɤjɤr}}}\markboth{sɤjɤr}{}
\begin{relation-sémantique}\confer{
\hyperlink{Ⓔajɤr}{\textit{ \papi{ajɤr}}}
}\end{relation-sémantique}\end{entrée}

\begin{entrée}
\vedette{\hypertarget{Ⓔsɤjɤrjɤr}{\papi{ sɤjɤrjɤr}}}\markboth{sɤjɤrjɤr}{}
\begin{relation-sémantique}\confer{
\hyperlink{Ⓔjɤrjɤr}{\textit{ \papi{jɤrjɤr}}}
}\end{relation-sémantique}\end{entrée}

\begin{entrée}
\vedette{\hypertarget{Ⓔsɤjku}{\papi{ sɤjku}}}\markboth{sɤjku}{}
\classe{n}
\begin{définition}\fra bouleau\end{définition}
\begin{définition}\cmn 白桦树\end{définition}
\begin{exemple}\jya sɤjku nɯ ɯ-jwaʁ mbraj cho naχtɕɯɣ, ɯ-rqhu nɯ wuma ʑo jaʁ cho ngɯt tɕe ɲchɣaʁ rmi, kɯ-wɣrum ŋu, si wuma ʑo mbro, ɯ-si nɯ ngɯt\cmn 
白桦树叶子长得和红桦树一样,树皮又厚又结实,叫\stylefv{ɲchɣaʁ},树皮的外层是白色的,是高大的树种,木质很结实。
\end{exemple}\end{entrée}

\begin{entrée}
\vedette{\hypertarget{Ⓔsɤjlɤβ}{\papi{ sɤjlɤβ}}}\markboth{sɤjlɤβ}{}
\classe{n}
\begin{définition}\fra la vapeur du sol\end{définition}
\begin{définition}\cmn 土地上的蒸汽\end{définition}
\begin{relation-sémantique}\confer{
\hyperlink{Ⓔtɤjlɤβ}{\textit{ \papi{tɤjlɤβ}}}
}\end{relation-sémantique}\end{entrée}

\begin{entrée}
\vedette{\hypertarget{Ⓔsɤjloʁ}{\papi{ sɤjloʁ}}}\markboth{sɤjloʁ}{}
\classe{vs}
\paradigme{\textit{dir :} \jya nɯ-}\acception{1}
\begin{définition}\fra laid\end{définition}
\begin{définition}\cmn 丑陋;难看\end{définition}\acception{2}
\begin{définition}\fra dégoutant\end{définition}
\begin{définition}\cmn 难吃\end{définition}\begin{sous-entrée}
\vedette{\hypertarget{}{\papi{ nɤsɤjloʁ}}}\markboth{nɤsɤjloʁ}{}\classe{vt}
\begin{définition}\ 
\begin{déclaration}\grammar{trop}\end{déclaration}\end{définition}
\begin{définition}\fra trouver laid, trouver dégoutant\end{définition}
\begin{définition}\cmn 觉得难看,觉得难吃\end{définition}
\begin{exemple}\jya nɤki ɯ-mdoʁ nɯ ɲɯ-nɤsɤjloʁ-a\cmn 我觉得这个颜色很难看\end{exemple}
\end{sous-entrée}\end{entrée}

\begin{entrée}
\vedette{\hypertarget{Ⓔsɤjndɤt}{\papi{ sɤjndɤt}}}\markboth{sɤjndɤt}{}
\classe{vs}
\paradigme{\textit{dir :} \jya thɯ-}
\begin{définition}\fra mignon, sage (enfant)\end{définition}
\begin{définition}\cmn 可爱;乖\end{définition}
\begin{exemple}\jya laχtɕha ɲɯ-sɤjndɤt\cmn 东西很可爱\end{exemple}
\begin{exemple}\jya tɕheme ɲɯ-sɤjndɤt\cmn 女孩子很可爱\end{exemple}
\begin{exemple}\jya tɯrme ɲɯ-sɤjndɤt\cmn 人很可爱\end{exemple}
\begin{relation-sémantique}\confer{
\hyperlink{Ⓔnɤjndɤt}{\textit{ \papi{nɤjndɤt}}}
}\end{relation-sémantique}\begin{sous-entrée}
\vedette{\hypertarget{}{\papi{ nɤsɤjndɤt}}}\markboth{nɤsɤjndɤt}{}\classe{vt}
\begin{définition}\ 
\begin{déclaration}\grammar{trop}\end{déclaration}\end{définition}
\begin{définition}\fra trouver mignon\end{définition}
\begin{définition}\cmn 觉得可爱\end{définition}
\begin{relation-sémantique}\synonyme{
\hyperlink{Ⓔnɤjndɤt}{\textit{ \papi{nɤjndɤt}}}
}\end{relation-sémantique}
\end{sous-entrée}\end{entrée}

\begin{entrée}
\vedette{\hypertarget{Ⓔsɤjoʁjoʁ}{\papi{ sɤjoʁjoʁ}}}\markboth{sɤjoʁjoʁ}{}
\classe{vt}
\paradigme{\textit{dir :} \jya tɤ-}
\begin{définition}\fra lever un peu, ranger\end{définition}
\begin{définition}\cmn 弄高一点,收拾东西\end{définition}
\begin{exemple}\jya si tɤ-sɤjoʁjoʁ-a\cmn 我把木料弄上去一点了\end{exemple}
\begin{exemple}\jya thɤfka ɯ-ŋgɯ si nɯ ra tɤ-sɤjoʁjoʁ tɕe nɯt\cmn 你把炉子里的柴弄上去一点就会燃烧\end{exemple}
\begin{relation-sémantique}\confer{
\hyperlink{Ⓔjoʁ}{\textit{ \papi{joʁ}}}
}\end{relation-sémantique}\end{entrée}

\begin{entrée}
\vedette{\hypertarget{Ⓔsɤjqu}{\papi{ sɤjqu}}}\markboth{sɤjqu}{}
\begin{relation-sémantique}\confer{
\hyperlink{Ⓔjqu}{\textit{ \papi{jqu}}}
}\end{relation-sémantique}\end{entrée}

\begin{entrée}
\vedette{\hypertarget{Ⓔsɤjʁu}{\papi{ sɤjʁu}}}\markboth{sɤjʁu}{}
\begin{relation-sémantique}\confer{
\hyperlink{Ⓔajʁu}{\textit{ \papi{ajʁu}}}
}\end{relation-sémantique}\end{entrée}

\begin{entrée}
\vedette{\hypertarget{Ⓔsɤjtshi}{\papi{ sɤjtshi}}}\markboth{sɤjtshi}{}
\begin{relation-sémantique}\confer{
\hyperlink{Ⓔjtshi}{\textit{ \papi{jtshi}}}
}\end{relation-sémantique}\end{entrée}

\begin{entrée}
\vedette{\hypertarget{Ⓔsɤjtɯ}{\papi{ sɤjtɯ}}}\markboth{sɤjtɯ}{}
\begin{relation-sémantique}\confer{
\hyperlink{Ⓔajtɯ}{\textit{ \papi{ajtɯ}}}
}\end{relation-sémantique}\end{entrée}

\begin{entrée}
\vedette{\hypertarget{Ⓔsɤjwɤrlɤr}{\papi{ sɤjwɤrlɤr}}}\markboth{sɤjwɤrlɤr}{}
\begin{relation-sémantique}\confer{
\hyperlink{Ⓔɣɤjwɤrlɤr}{\textit{ \papi{ɣɤjwɤrlɤr}}}
}\end{relation-sémantique}\end{entrée}

\begin{entrée}
\vedette{\hypertarget{Ⓔsɤɟɯɣlɯɣ}{\papi{ sɤɟɯɣlɯɣ}}}\markboth{sɤɟɯɣlɯɣ}{}\classe{vt}
\paradigme{\textit{dir :} \jya tɤ-}
\begin{définition}\fra hausser (les épaules)\end{définition}
\begin{définition}\cmn 耸(肩)\end{définition}
\begin{exemple}\jya ɯ-rpaʁ ra to-sɤɟɯɣlɯɣ\cmn 他耸了一下肩\end{exemple}
\end{entrée}

\begin{entrée}
\vedette{\hypertarget{Ⓔsɤɟɯɟrɯɣ}{\papi{ sɤɟɯɟrɯɣ}}}\markboth{sɤɟɯɟrɯɣ}{}
\begin{relation-sémantique}\confer{
\hyperlink{Ⓔɣɤɟɯɟrɯɣ}{\textit{ \papi{ɣɤɟɯɟrɯɣ}}}
}\end{relation-sémantique}\end{entrée}

\begin{entrée}
\vedette{\hypertarget{Ⓔsɤkɤβjɤβ}{\papi{ sɤkɤβjɤβ}}}\markboth{sɤkɤβjɤβ}{}
\begin{relation-sémantique}\confer{
\hyperlink{Ⓔɣɤkɤβjɤβ}{\textit{ \papi{ɣɤkɤβjɤβ}}}
}\end{relation-sémantique}\end{entrée}

\begin{entrée}
\vedette{\hypertarget{Ⓔsɤkɤlɤt}{\papi{ sɤkɤlɤt}}}\markboth{sɤkɤlɤt}{}
\begin{relation-sémantique}\confer{
\hyperlink{Ⓔakɤlɤt}{\textit{ \papi{akɤlɤt}}}
}\end{relation-sémantique}
\end{entrée}

\begin{entrée}
\vedette{\hypertarget{Ⓔsɤkɤsci}{\papi{ sɤkɤsci}}}\markboth{sɤkɤsci}{}
\classe{vt}
\paradigme{\textit{dir :} \jya tɤ-}
\paradigme{\textit{dir :} \jya pɯ-}
\paradigme{\textit{dir :} \jya thɯ-}
\begin{définition}\fra changer\end{définition}
\begin{définition}\cmn 换;调换\end{définition}
\begin{exemple}\jya tɕi-ŋga tɤ-nɯ-sɤkɤsci-tɕi\cmn 我们俩调换了衣服\end{exemple}
\begin{exemple}\jya ɯ-ŋga tshɯrɟɯn chɯ-sɤkɤsci ŋu\cmn 他经常换衣服\end{exemple}
\begin{relation-sémantique}\synonyme{
\hyperlink{Ⓔsɤscɯndu}{\textit{ \papi{sɤscɯndu}}}
}\end{relation-sémantique}
\begin{relation-sémantique}\confer{
\hyperlink{Ⓔnɤsci}{\textit{ \papi{nɤsci}}}
}\end{relation-sémantique}\end{entrée}

\begin{entrée}
\vedette{\hypertarget{Ⓔsɤkɤtɕɤβ}{\papi{ sɤkɤtɕɤβ}}}\markboth{sɤkɤtɕɤβ}{}
\begin{relation-sémantique}\confer{
\hyperlink{Ⓔakɤtɕɤβ}{\textit{ \papi{akɤtɕɤβ}}}
}\end{relation-sémantique}\end{entrée}

\begin{entrée}
\vedette{\hypertarget{Ⓔsɤkhar}{\papi{ sɤkhar}}}\markboth{sɤkhar}{}
\classe{vt}
\paradigme{\textit{dir :} \jya kɤ-}
\paradigme{\textit{dir :} \jya thɯ-}
\begin{définition}\fra enfermer\end{définition}
\begin{définition}\cmn 用……围起来
\begin{déclaration} \étymologie{\papi{ⁿkʰor}}\end{déclaration}\end{définition}
\begin{exemple}\jya si kɤ-sɤkhar-i\cmn 我们用树叉围成圆圈了\end{exemple}
\begin{relation-sémantique}\confer{
\hyperlink{Ⓔakhar}{\textit{ \papi{akhar}}}
}\end{relation-sémantique}
\begin{relation-sémantique}\confer{
\hyperlink{Ⓔnɤkhar}{\textit{ \papi{nɤkhar}}}
}\end{relation-sémantique}\end{entrée}

\begin{entrée}
\vedette{\hypertarget{Ⓔsɤkhra}{\papi{ sɤkhra}}}\markboth{sɤkhra}{}
\classe{vt}
\paradigme{\textit{dir :} \jya pɯ-}
\paradigme{\textit{dir :} \jya kɤ-}
\begin{définition}\fra colorer de toutes sortes de couleurs\end{définition}
\begin{définition}\cmn 配好各种颜色\end{définition}
\begin{exemple}\jya pɯ-sɤkhra-t-a\cmn 我画了很多颜色\end{exemple}
\begin{relation-sémantique}\confer{
\hyperlink{Ⓔakhra}{\textit{ \papi{akhra}}}
}\end{relation-sémantique}
\begin{relation-sémantique}\confer{
\hyperlink{Ⓔkhra}{\textit{ \papi{khra}}}
}\end{relation-sémantique}\end{entrée}

\begin{entrée}
\vedette{\hypertarget{Ⓔsɤkhɯ}{\papi{ sɤkhɯ}}}\markboth{sɤkhɯ}{}
\classe{vt}
\paradigme{\textit{dir :} \jya tɤ-}
\begin{définition}\fra fumer\end{définition}
\begin{définition}\cmn 熏\end{définition}
\begin{exemple}\jya rgɯnba kɤ-sɤkhɯ tɤ-tsɯm\cmn 你把求烟的东送上去\end{exemple}
\begin{exemple}\jya rgɯnba kɤ-sɤkhɯ ɯ-spa tɤ-tsɯm-a\cmn 我把供神求烟的东西送上去里了\end{exemple}
\begin{exemple}\jya tɤ-mthɯm tɤ-sɤkhɯ-t-a\cmn 我熏了肉\end{exemple}
\begin{relation-sémantique}\confer{
\hyperlink{Ⓔtɤ-khɯ}{\textit{ \papi{tɤ-khɯ}}}
}\end{relation-sémantique}
\begin{relation-sémantique}\confer{
\hyperlink{ⒺɣɤkhɯⒽ1}{\textit{ \papi{ɣɤkhɯ1}}}
}\end{relation-sémantique}
\begin{relation-sémantique}\confer{
\hyperlink{Ⓔnɤkhɯ}{\textit{ \papi{nɤkhɯ}}}
}\end{relation-sémantique}\end{entrée}

\begin{entrée}
\vedette{\hypertarget{Ⓔsɤkhɯkhɯɣ}{\papi{ sɤkhɯkhɯɣ}}}\markboth{sɤkhɯkhɯɣ}{}
\classe{vt}
\paradigme{\textit{dir :} \jya kɤ-}
\begin{définition}\fra boire à toute vitesse\end{définition}
\begin{définition}\cmn 急着喝\end{définition}
\begin{exemple}\jya tɯ-ci ka-sɤkhɯkhɯɣ\cmn 他急着喝了水\end{exemple}
\begin{exemple}\jya cha ka-sɤkhɯkhɯɣ\cmn 他急着喝了酒\end{exemple}
\begin{exemple}\jya tɯcɯrqɯ ka-sɤkhɯkhɯɣ\cmn 他急着喝了冷水\end{exemple}
\begin{exemple}\jya ma-kɤ-tɯ-sɤkhɯkhɯɣ\cmn 你不要急着喝\end{exemple}\end{entrée}

\begin{entrée}
\vedette{\hypertarget{Ⓔsɤla}{\papi{ sɤla}}}\markboth{sɤla}{}
\begin{relation-sémantique}\confer{
\hyperlink{Ⓔala}{\textit{ \papi{ala}}}
}\end{relation-sémantique}\end{entrée}

\begin{entrée}
\vedette{\hypertarget{Ⓔsɤlaŋphɤn}{\papi{ sɤlaŋphɤn}}}\markboth{sɤlaŋphɤn}{}
\classe{n}
\begin{définition}\fra bassine\end{définition}
\begin{définition}\cmn 盆子
\begin{déclaration} \étymologie{\papi{\stylefn{洗脸盆}}}\end{déclaration}\end{définition}\end{entrée}

\begin{entrée}
\vedette{\hypertarget{Ⓔsɤlɤɣɯ}{\papi{ sɤlɤɣɯ}}}\markboth{sɤlɤɣɯ}{}
\classe{vt}
\paradigme{\textit{dir :} \jya nɯ-}
\paradigme{\textit{dir :} \jya kɤ-}
\begin{définition}\fra connecter, rattacher\end{définition}
\begin{définition}\cmn 连接(断了的东西)\end{définition}
\begin{exemple}\jya tɤ-fsɤri ka-sɤlɤɣɯ\cmn 他把麻绳接上了\end{exemple}
\begin{exemple}\jya tɯ-mbri ka-sɤlɤɣɯ\cmn 他把绳子接上了\end{exemple}
\begin{relation-sémantique}\synonyme{
\hyperlink{Ⓔsɤmthoʁmthɯt}{\textit{ \papi{sɤmthoʁmthɯt}}}
}\end{relation-sémantique}
\begin{relation-sémantique}\confer{
\hyperlink{Ⓔalɤɣɯ}{\textit{ \papi{alɤɣɯ}}}
}\end{relation-sémantique}\begin{sous-entrée}
\vedette{\hypertarget{}{\papi{ nɯɣɯsɤlɤɣɯ}}}\markboth{nɯɣɯsɤlɤɣɯ}{}\classe{vs}
\begin{définition}\fra difficile à connecter (parole)\end{définition}
\begin{définition}\cmn 不容易连贯\end{définition}
\begin{exemple}\jya ɯ-rju mɯ́j-nɯɣɯsɤlɤɣɯ\cmn 他的话不容易连起来\end{exemple}
\end{sous-entrée}\end{entrée}

\begin{entrée}
\vedette{\hypertarget{Ⓔsɤlɤt}{\papi{ sɤlɤt}}}\markboth{sɤlɤt}{}
\classe{vi}
\paradigme{\textit{dir :} \jya \_}
\begin{définition}\fra raccompagner\end{définition}
\begin{définition}\cmn 送行\end{définition}
\begin{exemple}\jya kɯ-sɤlɤt jɤ-ari\cmn 他去送行了\end{exemple}
\begin{exemple}\jya ɕ-kɤ-sɤlɤt, ʑ-nɯ-sɤlɤt\cmn 他去送行了\end{exemple}
\begin{relation-sémantique}\synonyme{
\hyperlink{Ⓔsɤsco}{\textit{ \papi{sɤsco}}}
}\end{relation-sémantique}\end{entrée}

\begin{entrée}
\vedette{\hypertarget{Ⓔsɤljɤljɤt}{\papi{ sɤljɤljɤt}}}\markboth{sɤljɤljɤt}{}\classe{vt}
\paradigme{\textit{dir :} \jya nɯ-}
\begin{définition}\fra remuer (queue)\end{définition}
\begin{définition}\cmn 摇(尾巴)\end{définition}
\begin{exemple}\jya khɯna kɯ ɯ-jme ɲɯ-ɤsɯ-sɤljɤljɤt\cmn 狗在摇尾巴\end{exemple}\end{entrée}

\begin{entrée}
\vedette{\hypertarget{Ⓔsɤlothi}{\papi{ sɤlothi}}}\markboth{sɤlothi}{}
\begin{relation-sémantique}\confer{
\hyperlink{Ⓔalothi}{\textit{ \papi{alothi}}}
}\end{relation-sémantique}\end{entrée}

\begin{entrée}
\vedette{\hypertarget{Ⓔsɤlpɯm}{\papi{ sɤlpɯm}}}\markboth{sɤlpɯm}{}
\begin{relation-sémantique}\confer{
\hyperlink{Ⓔalpɯm}{\textit{ \papi{alpɯm}}}
}\end{relation-sémantique}\end{entrée}

\begin{entrée}
\vedette{\hypertarget{Ⓔsɤlqɤlqɤt}{\papi{ sɤlqɤlqɤt}}}\markboth{sɤlqɤlqɤt}{}\classe{vt}
\paradigme{\textit{dir :} \jya tɤ-}
\begin{définition}\fra agiter légèrement (des ailes)\end{définition}
\begin{définition}\cmn 轻轻地扇动(翅膀)\end{définition}\end{entrée}

\begin{entrée}
\vedette{\hypertarget{Ⓔsɤltɕhɤltɕhɤt}{\papi{ sɤltɕhɤltɕhɤt}}}\markboth{sɤltɕhɤltɕhɤt}{}
\begin{relation-sémantique}\confer{
\hyperlink{Ⓔltɕhɤltɕhɤt}{\textit{ \papi{ltɕhɤltɕhɤt}}}
}\end{relation-sémantique}\end{entrée}

\begin{entrée}
\vedette{\hypertarget{Ⓔsɤltɕhɯɣlɯɣ}{\papi{ sɤltɕhɯɣlɯɣ}}}\markboth{sɤltɕhɯɣlɯɣ}{}
\begin{relation-sémantique}\confer{
\hyperlink{Ⓔltɕhɯɣltɕhɯɣ}{\textit{ \papi{ltɕhɯɣltɕhɯɣ}}}
}\end{relation-sémantique}\end{entrée}

\begin{entrée}
\vedette{\hypertarget{Ⓔsɤlthɤlthɤβ}{\papi{ sɤlthɤlthɤβ}}}\markboth{sɤlthɤlthɤβ}{}\classe{vt}
\paradigme{\textit{dir :} \jya pɯ-}
\begin{définition}\fra cligner de l'œil\end{définition}
\begin{définition}\cmn 眨眼\end{définition}
\begin{relation-sémantique}\confer{
 \papi{thɤβ1}
}\end{relation-sémantique}\end{entrée}

\begin{entrée}
\vedette{\hypertarget{Ⓔsɤltshɤltshɤt}{\papi{ sɤltshɤltshɤt}}}\markboth{sɤltshɤltshɤt}{}
\classe{vt}
\paradigme{\textit{dir :} \jya nɯ-}
\begin{définition}\fra agiter légèrement\end{définition}
\begin{définition}\cmn 轻轻地摇动(有毛、有絮絮的东西)\end{définition}
\begin{exemple}\jya ɲɯ-sɤltshɤltshɤt\cmn 他在摇动\end{exemple}
\begin{exemple}\jya ɯʑo kɯ na-sɤltshɤltshɤt\cmn 他摇动了\end{exemple}
\begin{exemple}\jya si nɯ ɲɯ-sɤltshɤltshɤt\cmn 他在摇树\end{exemple}
\begin{exemple}\jya rɟɤskhi ɲɯ-sɤltshɤltshɤt\cmn 他在抖动簸箕\end{exemple}
\begin{exemple}\jya ɲɯ-saʁlɤt tɕe ɲɯ-sɤltshɤltshɤt ntsɯ\cmn 他扬粮食,把(簸箕)抖来抖去\end{exemple}\begin{sous-entrée}
\vedette{\hypertarget{}{\papi{ ɣɤltshɤltshɤt}}}\markboth{ɣɤltshɤltshɤt}{}\classe{vi}
\begin{définition}\fra trembler, se secouer\end{définition}
\begin{définition}\cmn 发抖\end{définition}
\begin{exemple}\jya ɯ-tɯ-nɤndʐo kɯ ɲɯ-ɣɤltshɤltshɤt ʑo\cmn 小孩子冷得发抖\end{exemple}
\end{sous-entrée}\end{entrée}

\begin{entrée}
\vedette{\hypertarget{Ⓔsɤltshɯltshɯɣ}{\papi{ sɤltshɯltshɯɣ}}}\markboth{sɤltshɯltshɯɣ}{}
\classe{vt}
\paradigme{\textit{dir :} \jya nɯ-}
\begin{définition}\fra bercer\end{définition}
\begin{définition}\cmn 摇(小孩子)\end{définition}
\begin{exemple}\jya nɯ-sɤltshɯltshɯɣ-a tɕe tɕe ɲɯ-rga tɕe mɯ́j-ɣɤwu\cmn 我把他摇了一下,他高兴就不哭了\end{exemple}\end{entrée}

\begin{entrée}
\vedette{\hypertarget{Ⓔsɤlɯrlɯr}{\papi{ sɤlɯrlɯr}}}\markboth{sɤlɯrlɯr}{}
\begin{relation-sémantique}\confer{
\hyperlink{Ⓔlɯrlɯr}{\textit{ \papi{lɯrlɯr}}}
}\end{relation-sémantique}\end{entrée}

\begin{entrée}
\vedette{\hypertarget{Ⓔsɤlɯzlɯz}{\papi{ sɤlɯzlɯz}}}\markboth{sɤlɯzlɯz}{}
\begin{relation-sémantique}\confer{
\hyperlink{Ⓔɣɤlɯzlɯz}{\textit{ \papi{ɣɤlɯzlɯz}}}
}\end{relation-sémantique}\end{entrée}

\begin{entrée}
\vedette{\hypertarget{Ⓔsɤlwɤlwɤt}{\papi{ sɤlwɤlwɤt}}}\markboth{sɤlwɤlwɤt}{}\classe{vt}
\paradigme{\textit{dir :} \jya tɤ-}
\paradigme{\textit{dir :} \jya nɯ-}
\begin{définition}\fra agiter\end{définition}
\begin{définition}\cmn 挥动\end{définition}
\begin{exemple}\jya ɯ-jaʁ ta-sɤlwɤlwɤt\cmn 他挥了手\end{exemple}
\begin{exemple}\jya ɯ-rna ɲɯ-sɤlwɤlwɤt tɕe βɣɤza ɲɯ-ɤsɯ-no\cmn 它在动耳朵赶蚊子\end{exemple}
\begin{exemple}\jya qale ta-βzu tɕe, rloŋrta ɲɯ-sɤlwɤlwɤt\cmn 风吹,使玛尼旗飘动\end{exemple}
\begin{relation-sémantique}\confer{
\hyperlink{Ⓔɣɤlwɤlwɤt}{\textit{ \papi{ɣɤlwɤlwɤt}}}
}\end{relation-sémantique}\end{entrée}

\begin{entrée}
\vedette{\hypertarget{Ⓔsɤɬɯt}{\papi{ sɤɬɯt}}}\markboth{sɤɬɯt}{}
\begin{relation-sémantique}\confer{
\hyperlink{Ⓔaɬɯt}{\textit{ \papi{aɬɯt}}}
}\end{relation-sémantique}\end{entrée}

\begin{entrée}
\vedette{\hypertarget{Ⓔsɤmbɤldʑɤm}{\papi{ sɤmbɤldʑɤm}}}\markboth{sɤmbɤldʑɤm}{}
\begin{relation-sémantique}\confer{
\hyperlink{Ⓔambɤldʑɤm}{\textit{ \papi{ambɤldʑɤm}}}
}\end{relation-sémantique}\end{entrée}

\begin{entrée}
\vedette{\hypertarget{Ⓔsɤmbi}{\papi{ sɤmbi}}}\markboth{sɤmbi}{}\classe{vl}
\paradigme{\textit{dir :} \jya nɯ-}
\begin{définition}\fra réclamer à qqn\end{définition}
\begin{définition}\cmn 向别人去要\end{définition}
\begin{exemple}\jya laχtɕha nɯ-sɤmbi\cmn 你向他要东西\end{exemple}
\begin{exemple}\jya ɕ-pɯ-sɤmbi-j\cmn 我们去要了\end{exemple}
\begin{exemple}\jya aʑo tɤ-ŋgɯm ci nɯ-sɤmbi-a\cmn 我要了一个鸡蛋\end{exemple}
\begin{exemple}\jya aʑo api ɯ-ɕki kɯmtɕhɯ ci nɯ-sɤmbi-a\cmn 我向哥哥要了玩具\end{exemple}
\begin{exemple}\jya nɯ-sɤmbi-t-a\end{exemple}
\begin{relation-sémantique}\confer{
\hyperlink{Ⓔmbi}{\textit{ \papi{mbi}}}
}\end{relation-sémantique}\end{entrée}

\begin{entrée}
\vedette{\hypertarget{Ⓔsɤmbrɤqɤt}{\papi{ sɤmbrɤqɤt}}}\markboth{sɤmbrɤqɤt}{}
\classe{vt}
\paradigme{\textit{dir :} \jya nɯ-}
\begin{définition}\fra différencier\end{définition}
\begin{définition}\cmn 分辨\end{définition}
\begin{exemple}\jya stoʁ staχpɯ na-sɤmbrɤqɤt\cmn 他分辨了胡豆和豌豆\end{exemple}
\begin{exemple}\jya laχtɕha na-sɤmbrɤqɤt\cmn 他区分了东西\end{exemple}
\begin{exemple}\jya mbraj cho tɯrgi na-sɤmbrɤqɤt\cmn 他分辨了红桦树和杉树\end{exemple}
\begin{exemple}\jya kɯ-ŋu kɯ-maʁ ɲɯ-sɤmbrɤqɤt\cmn 他分辨真的和假的\end{exemple}
\begin{exemple}\jya tɯ-rju tɯ-ŋka ɯ-ŋgɯ nɯ tɕu kɤ-sɤmbrɤqɤt tu\cmn 一句话有几种意思\end{exemple}
\begin{exemple}\jya tɕe qale kɯ tɤɕi ɯ-rdoʁ cho ɯ-βɣi nɯ ra ɲɤ-sɤmbrɤqɤt ɕti\cmn 风把青稞颗粒和糠秕分开\end{exemple}
\begin{relation-sémantique}\confer{
\hyperlink{Ⓔambrɤqɤt}{\textit{ \papi{ambrɤqɤt}}}
}\end{relation-sémantique}\end{entrée}

\begin{entrée}
\vedette{\hypertarget{Ⓔsɤmbrɯ}{\papi{ sɤmbrɯ}}}\markboth{sɤmbrɯ}{}\classe{vi}
\paradigme{\textit{dir :} \jya tɤ-}
\begin{définition}\ 
\begin{déclaration}\grammar{denom}\end{déclaration}\end{définition}
\begin{définition}\fra s’énerver\end{définition}
\begin{définition}\cmn 生气\end{définition}
\begin{exemple}\jya aʑo tɤ-sɤmbrɯ-a\cmn 我生气了\end{exemple}
\begin{exemple}\jya ɲɯ-sɤmbrɯ\cmn 他在生气\end{exemple}
\begin{relation-sémantique}\confer{
\hyperlink{Ⓔsɤzmbrɯ}{\textit{ \papi{sɤzmbrɯ}}}
}\end{relation-sémantique}
\begin{relation-sémantique}\confer{
\hyperlink{Ⓔnɤmbrɯ}{\textit{ \papi{nɤmbrɯ}}}
}\end{relation-sémantique}
\begin{relation-sémantique}\confer{
\hyperlink{Ⓔɣɤsɤmbrɯ}{\textit{ \papi{ɣɤsɤmbrɯ}}}
}\end{relation-sémantique}
\begin{relation-sémantique}\confer{
\hyperlink{Ⓔtɤ-mbrɯ}{\textit{ \papi{tɤ-mbrɯ}}}
}\end{relation-sémantique}\end{entrée}

\begin{entrée}
\vedette{\hypertarget{Ⓔsɤmbrɯŋgɯ}{\papi{ sɤmbrɯŋgɯ}}}\markboth{sɤmbrɯŋgɯ}{}
\classe{vs}
\paradigme{\textit{dir :} \jya tɤ-}
\begin{définition}\ 
\begin{déclaration}\grammar{incorp}\end{déclaration}\end{définition}
\begin{définition}\fra être détestable\end{définition}
\begin{définition}\cmn 讨厌\end{définition}
\begin{exemple}\jya tɯrme kɯ-sɤmbrɯŋgɯ ci ɲɯ-ŋu\cmn 他是一个讨厌的人\end{exemple}
\begin{exemple}\jya kɤ-sɤmbrɯŋgɯ ɲɯ-tɯ-χɕu\cmn 你最会叫人生气\end{exemple}
\begin{relation-sémantique}\confer{
\hyperlink{Ⓔsɤmbrɯ}{\textit{ \papi{sɤmbrɯ}}}
}\end{relation-sémantique}
\begin{relation-sémantique}\confer{
\hyperlink{Ⓔtɤ-mbrɯ,ŋgɯ}{\textit{ \papi{tɤ-mbrɯ,ŋgɯ}}}
}\end{relation-sémantique}\end{entrée}

\begin{entrée}
\vedette{\hypertarget{Ⓔsɤmdzɯ}{\papi{ sɤmdzɯ}}}\markboth{sɤmdzɯ}{}
\begin{relation-sémantique}\confer{
\hyperlink{Ⓔamdzɯ}{\textit{ \papi{amdzɯ}}}
}\end{relation-sémantique}\end{entrée}

\begin{entrée}
\vedette{\hypertarget{Ⓔsɤmgri}{\papi{ sɤmgri}}}\markboth{sɤmgri}{}
\classe{vt}
\paradigme{\textit{dir :} \jya nɯ-}
\begin{définition}\fra rendre claire (l’eau)\end{définition}
\begin{définition}\cmn 澄清
\begin{déclaration}\use{tɯ-ci kɤ-sɤmgri nɯ, tɯ-ci kɯ-qarndɯm nɯ ɲɯ́-wɣ-ɣɤntaβ tɕe, @penzi ɯ-ŋgɯ tú-wɣ-rku tɕe ɲɯ́-wɣ-ɣɤntaβ tɕe ɯʑo ɲɯ-ɤnɯmgri ɲɯ-ŋu. kɯ-ɴqhi nɯra ɯ-qa pjɯ-ɕe ŋu, tɕe kɯ-maŋtaʁ nɯra ɲɯ-ɤmgri ŋu.}\end{déclaration}\end{définition}
\begin{exemple}\jya tɯ-ci ɲɯ́-wɣ-sɤmgri tɕe kɤ-tshi sna\cmn 水澄清了就可以喝\end{exemple}
\begin{exemple}\jya mɯ-nɯ-kɤ-sɤmgri nɯ kɤ-tshi mɤ-sna\cmn 没有澄清的不能喝\end{exemple}\end{entrée}

\begin{entrée}
\vedette{\hypertarget{Ⓔsɤmgro}{\papi{ sɤmgro}}}\markboth{sɤmgro}{}
\classe{vs}
\paradigme{\textit{dir :} \jya tɤ-}
\begin{définition}\fra qui donne envie\end{définition}
\begin{définition}\cmn 令人有……的欲望\end{définition}
\begin{exemple}\jya aʑo kɤ-mbi mɯ́j-sɤmgro-a wo\cmn 别人把东西给我,我就不领情\end{exemple}
\begin{exemple}\jya kɤ-ndza ɲɯ-sɤmgro\cmn 令人有想吃的感觉\end{exemple}
\begin{relation-sémantique}\antonyme{
\hyperlink{Ⓔsɤŋɤβ}{\textit{ \papi{sɤŋɤβ}}}
}\end{relation-sémantique}
\begin{relation-sémantique}\confer{
 \papi{nɤmgro}
}\end{relation-sémantique}\end{entrée}

\begin{entrée}
\vedette{\hypertarget{Ⓔsɤmgrɯn}{\papi{ sɤmgrɯn}}}\markboth{sɤmgrɯn}{}
\begin{relation-sémantique}\confer{
\hyperlink{Ⓔmgrɯn}{\textit{ \papi{mgrɯn}}}
}\end{relation-sémantique}\end{entrée}

\begin{entrée}
\vedette{\hypertarget{Ⓔsɤmɲɤm}{\papi{ sɤmɲɤm}}}\markboth{sɤmɲɤm}{}
\begin{relation-sémantique}\confer{
\hyperlink{Ⓔamɲɤm}{\textit{ \papi{amɲɤm}}}
}\end{relation-sémantique}\end{entrée}

\begin{entrée}
\vedette{\hypertarget{Ⓔsɤmɲo}{\papi{ sɤmɲo}}}\markboth{sɤmɲo}{}\classe{vs}
\begin{définition}\fra magnifique, splendide (spectacle, paysage)\end{définition}
\begin{définition}\cmn 精彩(表演);值得观看\end{définition}
\begin{exemple}\jya jisŋi stɯnmɯ wuma pɯ-sɤmɲo\cmn 今天的婚礼很精彩\end{exemple}
\begin{exemple}\jya tɕetu zgoku tu-kɯ-ɕe ɲɯ-kɯ-nɤrɯra tɕe wuma ʑo sɤmɲo\cmn 山上的风景很值得观看\end{exemple}
\begin{relation-sémantique}\confer{
\hyperlink{Ⓔnɤmɲo}{\textit{ \papi{nɤmɲo}}}
}\end{relation-sémantique}
\end{entrée}

\begin{entrée}
\vedette{\hypertarget{Ⓔsɤmŋaʁ}{\papi{ sɤmŋaʁ}}}\markboth{sɤmŋaʁ}{}
\begin{relation-sémantique}\confer{
\hyperlink{Ⓔamŋaʁ}{\textit{ \papi{amŋaʁ}}}
}\end{relation-sémantique}\end{entrée}

\begin{entrée}
\vedette{\hypertarget{Ⓔsɤmŋo}{\papi{ sɤmŋo}}}\markboth{sɤmŋo}{}\classe{n}
\begin{définition}\ 
\begin{déclaration}\grammar{n.lieu}\end{déclaration}\end{définition}
\begin{définition}\fra Somang\end{définition}
\begin{définition}\cmn 梭磨河\end{définition}\end{entrée}

\begin{entrée}
\vedette{\hypertarget{Ⓔsɤmŋɯr}{\papi{ sɤmŋɯr}}}\markboth{sɤmŋɯr}{}
\classe{vs}
\paradigme{\textit{dir :} \jya nɯ-}
\begin{définition}\fra goût huileux écœurant\end{définition}
\begin{définition}\cmn 腻(食物)\end{définition}
\begin{exemple}\jya kɤndza ɲɯ-sɤmŋɯr\cmn 食物很腻\end{exemple}
\begin{exemple}\jya tɤ-mthɯm ɲɯ-sɤmŋɯr\cmn 肉很腻\end{exemple}\end{entrée}

\begin{entrée}
\vedette{\hypertarget{Ⓔsɤmokhɯtsa}{\papi{ sɤmokhɯtsa}}}\markboth{sɤmokhɯtsa}{}
\classe{n}
\begin{définition}\fra bol en bois\end{définition}
\begin{définition}\cmn 木碗\end{définition}\end{entrée}

\begin{entrée}
\vedette{\hypertarget{Ⓔsɤmtɕhoʁ}{\papi{ sɤmtɕhoʁ}}}\markboth{sɤmtɕhoʁ}{}
\classe{vt}
\paradigme{\textit{dir :} \jya nɯ-}
\paradigme{\textit{dir :} \jya tɤ-}
\begin{définition}\ 
\begin{déclaration}\grammar{caus}\end{déclaration}\end{définition}
\begin{définition}\fra mettre en ordre\end{définition}
\begin{définition}\cmn 摆整齐;使均匀\end{définition}
\begin{exemple}\jya ndʑu nɯ-sɤmtɕhoʁ-a\cmn 我把筷子弄整齐了\end{exemple}
\begin{exemple}\jya nɤki tɯ-ŋga ra nɯ-sɤmtɕhoʁ\cmn 把衣服收拾整齐\end{exemple}
\begin{exemple}\jya jisŋi ji-ma nɯ ra sɤmtɕhoʁ-a ra\cmn 我要整理今天的工作\end{exemple}
\begin{relation-sémantique}\confer{
\hyperlink{Ⓔamtɕhoʁ}{\textit{ \papi{amtɕhoʁ}}}
}\end{relation-sémantique}\end{entrée}

\begin{entrée}
\vedette{\hypertarget{Ⓔsɤmtɕhɯβ}{\papi{ sɤmtɕhɯβ}}}\markboth{sɤmtɕhɯβ}{}
\begin{relation-sémantique}\confer{
\hyperlink{Ⓔmtɕhɯβ}{\textit{ \papi{mtɕhɯβ}}}
}\end{relation-sémantique}\end{entrée}

\begin{entrée}
\vedette{\hypertarget{Ⓔsɤmthoʁmthɯt}{\papi{ sɤmthoʁmthɯt}}}\markboth{sɤmthoʁmthɯt}{}
\classe{vt}
\paradigme{\textit{dir :} \jya lɤ-}
\begin{définition}\fra relier\end{définition}
\begin{définition}\cmn 连接\end{définition}
\begin{exemple}\jya tɤfsɤri lɤ-sɤmthoʁmthɯt-a\cmn 我把麻绳接起来了\end{exemple}
\begin{exemple}\jya tɯmbri lɤ-sɤmthoʁmthɯt-a\cmn 我把绳子接起来了\end{exemple}
\begin{relation-sémantique}\synonyme{
\hyperlink{Ⓔsɤlɤɣɯ}{\textit{ \papi{sɤlɤɣɯ}}}
}\end{relation-sémantique}
\begin{relation-sémantique}\confer{
\hyperlink{Ⓔamthoʁmthɯt}{\textit{ \papi{amthoʁmthɯt}}}
}\end{relation-sémantique}\end{entrée}

\begin{entrée}
\vedette{\hypertarget{Ⓔsɤmto}{\papi{ sɤmto}}}\markboth{sɤmto}{}
\classe{vs}
\paradigme{\textit{dir :} \jya tɤ-}
\begin{définition}\ 
\begin{déclaration}\grammar{deexp}\end{déclaration}\end{définition}
\begin{définition}\fra visible\end{définition}
\begin{définition}\cmn 看得见的\end{définition}
\begin{exemple}\jya ɯ-phoŋbu lonba ɲɯ-sɤmto\cmn 他的身体完全看得到\end{exemple}
\begin{exemple}\jya ɯ-qiɯ ɲɯ-sɤmto\cmn 看得到一半\end{exemple}
\begin{exemple}\jya nɤ-ku ɲɯ-sɤmto\cmn 看得见你的头\end{exemple}
\begin{exemple}\jya kutɕu ku-kɯ-rɤʑi tɕe, kha sɤmto\cmn 在这里看得见房子\end{exemple}\end{entrée}

\begin{entrée}
\vedette{\hypertarget{Ⓔsɤmtshɤm}{\papi{ sɤmtshɤm}}}\markboth{sɤmtshɤm}{}\classe{vs}
\paradigme{\textit{dir :} \jya tɤ-}
\begin{définition}\fra facile à entendre\end{définition}
\begin{définition}\cmn 听得见\end{définition}
\begin{exemple}\jya aʑo tu-ti-a ɯ-ɲɯ́-sɤmtshɤm\cmn 我说的话听得见吗?\end{exemple}
\begin{exemple}\jya nɤʑɯɣ ɯ-ɲɯ́-sɤmtshɤm?\cmn 你听得到吗?\end{exemple}
\begin{relation-sémantique}\confer{
\hyperlink{Ⓔmtshɤm}{\textit{ \papi{mtshɤm}}}
}\end{relation-sémantique}\end{entrée}

\begin{entrée}
\vedette{\hypertarget{Ⓔsɤmtshɤr}{\papi{ sɤmtshɤr}}}\markboth{sɤmtshɤr}{}
\classe{vs}
\paradigme{\textit{dir :} \jya tɤ-}
\begin{définition}\fra étrange\end{définition}
\begin{définition}\cmn 奇怪\end{définition}
\begin{exemple}\jya alo tʂu nɯ kɯ-sɤmtshɤr pjɤ-mbɯt\cmn 上游,路塌方得很吓人\end{exemple}
\begin{relation-sémantique}\confer{
\hyperlink{Ⓔnɤmtshɤr}{\textit{ \papi{nɤmtshɤr}}}
}\end{relation-sémantique}\end{entrée}

\begin{entrée}
\vedette{\hypertarget{Ⓔsɤmtshi}{\papi{ sɤmtshi}}}\markboth{sɤmtshi}{}
\begin{relation-sémantique}\confer{
\hyperlink{Ⓔmtshi}{\textit{ \papi{mtshi}}}
}\end{relation-sémantique}\end{entrée}

\begin{entrée}
\vedette{\hypertarget{Ⓔsɤmtsɯɣ}{\papi{ sɤmtsɯɣ}}}\markboth{sɤmtsɯɣ}{}
\begin{relation-sémantique}\confer{
\hyperlink{Ⓔmtsɯɣ}{\textit{ \papi{mtsɯɣ}}}
}\end{relation-sémantique}\end{entrée}

\begin{entrée}
\vedette{\hypertarget{Ⓔsɤmtsɯr}{\papi{ sɤmtsɯr}}}\markboth{sɤmtsɯr}{}
\classe{vs}
\begin{définition}\ 
\begin{déclaration}\grammar{deexp}\end{déclaration}\end{définition}
\begin{définition}\fra famine (y avoir une)\end{définition}
\begin{définition}\cmn 饥荒\end{définition}
\begin{exemple}\jya ɯ-tɯ-sɤmtsɯr saχaʁ\cmn 有饥荒\end{exemple}
\begin{relation-sémantique}\confer{
\hyperlink{Ⓔmtsɯr}{\textit{ \papi{mtsɯr}}}
}\end{relation-sémantique}\end{entrée}

\begin{entrée}
\vedette{\hypertarget{Ⓔsɤmɯβde}{\papi{ sɤmɯβde}}}\markboth{sɤmɯβde}{}
\begin{relation-sémantique}\confer{
\hyperlink{Ⓔβde}{\textit{ \papi{βde}}}
}\end{relation-sémantique}\end{entrée}

\begin{entrée}
\vedette{\hypertarget{Ⓔsɤmɯmto}{\papi{ sɤmɯmto}}}\markboth{sɤmɯmto}{}
\begin{relation-sémantique}\confer{
\hyperlink{Ⓔamɯmto}{\textit{ \papi{amɯmto}}}
}\end{relation-sémantique}\end{entrée}

\begin{entrée}
\vedette{\hypertarget{Ⓔsɤmɯrpu}{\papi{ sɤmɯrpu}}}\markboth{sɤmɯrpu}{}
\begin{relation-sémantique}\confer{
\hyperlink{Ⓔamɯrpu}{\textit{ \papi{amɯrpu}}}
}\end{relation-sémantique}\end{entrée}

\begin{entrée}
\vedette{\hypertarget{Ⓔsɤmɯrtsɯɣ}{\papi{ sɤmɯrtsɯɣ}}}\markboth{sɤmɯrtsɯɣ}{}
\begin{relation-sémantique}\confer{
\hyperlink{Ⓔmɯrtsɯɣ}{\textit{ \papi{mɯrtsɯɣ}}}
}\end{relation-sémantique}\end{entrée}

\begin{entrée}
\vedette{\hypertarget{Ⓔsɤmɯsthaβ}{\papi{ sɤmɯsthaβ}}}\markboth{sɤmɯsthaβ}{}
\begin{relation-sémantique}\confer{
\hyperlink{Ⓔamɯsthaβ}{\textit{ \papi{amɯsthaβ}}}
}\end{relation-sémantique}\end{entrée}

\begin{entrée}
\vedette{\hypertarget{Ⓔsɤmɯsɯz}{\papi{ sɤmɯsɯz}}}\markboth{sɤmɯsɯz}{}
\begin{relation-sémantique}\confer{
\hyperlink{Ⓔamɯsɯz}{\textit{ \papi{amɯsɯz}}}
}\end{relation-sémantique}\end{entrée}

\begin{entrée}
\vedette{\hypertarget{Ⓔsɤmɯtso}{\papi{ sɤmɯtso}}}\markboth{sɤmɯtso}{}
\begin{relation-sémantique}\confer{
\hyperlink{Ⓔamɯtso}{\textit{ \papi{amɯtso}}}
}\end{relation-sémantique}\end{entrée}

\begin{entrée}
\vedette{\hypertarget{Ⓔsɤmɯtɯɣ}{\papi{ sɤmɯtɯɣ}}}\markboth{sɤmɯtɯɣ}{}
\begin{relation-sémantique}\confer{
\hyperlink{Ⓔamɯtɯɣ}{\textit{ \papi{amɯtɯɣ}}}
}\end{relation-sémantique}\end{entrée}

\begin{entrée}
\vedette{\hypertarget{Ⓔsɤmɯzɣɯt}{\papi{ sɤmɯzɣɯt}}}\markboth{sɤmɯzɣɯt}{}
\begin{relation-sémantique}\confer{
\hyperlink{Ⓔamɯzɣɯt}{\textit{ \papi{amɯzɣɯt}}}
}\end{relation-sémantique}\end{entrée}

\begin{entrée}
\vedette{\hypertarget{Ⓔsɤnaʁdɤz}{\papi{ sɤnaʁdɤz}}}\markboth{sɤnaʁdɤz}{}
\begin{relation-sémantique}\confer{
\hyperlink{Ⓔnaʁdɤz}{\textit{ \papi{naʁdɤz}}}
}\end{relation-sémantique}\end{entrée}

\begin{entrée}
\vedette{\hypertarget{Ⓔsɤnaχsoz}{\papi{ sɤnaχsoz}}}\markboth{sɤnaχsoz}{}
\begin{relation-sémantique}\confer{
\hyperlink{Ⓔnaχsoz}{\textit{ \papi{naχsoz}}}
}\end{relation-sémantique}\end{entrée}

\begin{entrée}
\vedette{\hypertarget{Ⓔsɤnɤjkɯz}{\papi{ sɤnɤjkɯz}}}\markboth{sɤnɤjkɯz}{}
\begin{relation-sémantique}\confer{
\hyperlink{Ⓔnɤjkɯz}{\textit{ \papi{nɤjkɯz}}}
}\end{relation-sémantique}\end{entrée}

\begin{entrée}
\vedette{\hypertarget{Ⓔsɤnɤkhe}{\papi{ sɤnɤkhe}}}\markboth{sɤnɤkhe}{}
\begin{relation-sémantique}\confer{
\hyperlink{Ⓔnɤkhe}{\textit{ \papi{nɤkhe}}}
}\end{relation-sémantique}\end{entrée}

\begin{entrée}
\vedette{\hypertarget{Ⓔsɤnɤmpɕɤr}{\papi{ sɤnɤmpɕɤr}}}\markboth{sɤnɤmpɕɤr}{}
\begin{relation-sémantique}\confer{
\hyperlink{Ⓔnɤmpɕɤr}{\textit{ \papi{nɤmpɕɤr}}}
}\end{relation-sémantique}\end{entrée}

\begin{entrée}
\vedette{\hypertarget{Ⓔsɤnɤmtsioʁ}{\papi{ sɤnɤmtsioʁ}}}\markboth{sɤnɤmtsioʁ}{}
\begin{relation-sémantique}\confer{
\hyperlink{Ⓔnɤmtsioʁ}{\textit{ \papi{nɤmtsioʁ}}}
}\end{relation-sémantique}\end{entrée}

\begin{entrée}
\vedette{\hypertarget{Ⓔsɤnɤntshɣɤz}{\papi{ sɤnɤntshɣɤz}}}\markboth{sɤnɤntshɣɤz}{}
\begin{relation-sémantique}\confer{
\hyperlink{Ⓔnɤntshɣɤz}{\textit{ \papi{nɤntshɣɤz}}}
}\end{relation-sémantique}\end{entrée}

\begin{entrée}
\vedette{\hypertarget{Ⓔsɤnɤre}{\papi{ sɤnɤre}}}\markboth{sɤnɤre}{}
\begin{relation-sémantique}\confer{
 \papi{nɤre}
}\end{relation-sémantique}\end{entrée}

\begin{entrée}
\vedette{\hypertarget{Ⓔsɤnɤsɤɣ}{\papi{ sɤnɤsɤɣ}}}\markboth{sɤnɤsɤɣ}{}
\begin{relation-sémantique}\confer{
\hyperlink{Ⓔnɤsɤɣ}{\textit{ \papi{nɤsɤɣ}}}
}\end{relation-sémantique}\end{entrée}

\begin{entrée}
\vedette{\hypertarget{Ⓔsɤnɤsma}{\papi{ sɤnɤsma}}}\markboth{sɤnɤsma}{}
\begin{relation-sémantique}\confer{
\hyperlink{Ⓔnɤsma}{\textit{ \papi{nɤsma}}}
}\end{relation-sémantique}\end{entrée}

\begin{entrée}
\vedette{\hypertarget{Ⓔsɤnɤstu}{\papi{ sɤnɤstu}}}\markboth{sɤnɤstu}{}
\begin{relation-sémantique}\confer{
\hyperlink{Ⓔnɤstu}{\textit{ \papi{nɤstu}}}
}\end{relation-sémantique}\end{entrée}

\begin{entrée}
\vedette{\hypertarget{Ⓔsɤnɤz}{\papi{ sɤnɤz}}}\markboth{sɤnɤz}{}
\begin{relation-sémantique}\confer{
\hyperlink{Ⓔnɤz}{\textit{ \papi{nɤz}}}
}\end{relation-sémantique}\end{entrée}

\begin{entrée}
\vedette{\hypertarget{Ⓔsɤndu}{\papi{ sɤndu}}}\markboth{sɤndu}{}
\classe{vt}
\paradigme{\textit{dir :} \jya tɤ-}
\begin{définition}\fra échanger\end{définition}
\begin{définition}\cmn 交换\end{définition}
\begin{exemple}\jya tɯ-rɣi tɤ-sɤndu-j\cmn 我们交换了种子\end{exemple}
\begin{relation-sémantique}\synonyme{
\hyperlink{Ⓔnɤsci}{\textit{ \papi{nɤsci}}}
}\end{relation-sémantique}
\begin{relation-sémantique}\synonyme{
\hyperlink{Ⓔsɤscɯndu}{\textit{ \papi{sɤscɯndu}}}
}\end{relation-sémantique}
\begin{relation-sémantique}\confer{
\hyperlink{Ⓔantsɤndu}{\textit{ \papi{antsɤndu}}}
}\end{relation-sémantique}\end{entrée}

\begin{entrée}
\vedette{\hypertarget{Ⓔsɤndʐaβ}{\papi{ sɤndʐaβ}}}\markboth{sɤndʐaβ}{}
\classe{vs}
\paradigme{\textit{dir :} \jya tɤ-}
\begin{définition}\ 
\begin{déclaration}\grammar{deexp}\end{déclaration}
\begin{déclaration}\grammar{acaus}\end{déclaration}\end{définition}
\begin{définition}\fra endroit où il est courant de tomber\end{définition}
\begin{définition}\cmn 令人容易摔倒的地方\end{définition}
\begin{exemple}\jya tʂu ɲɯ-ɣɤrɤβ ɲɯ-sɤndʐaβ\cmn 路很陡峭,容易摔倒\end{exemple}
\begin{exemple}\jya kutɕu sɤndʐaβ ŋgrɤl\cmn 这里容易摔倒\end{exemple}
\begin{relation-sémantique}\confer{
\hyperlink{Ⓔndʐaβ}{\textit{ \papi{ndʐaβ}}}
}\end{relation-sémantique}\end{entrée}

\begin{entrée}
\vedette{\hypertarget{Ⓔsɤndɤɣ}{\papi{ sɤndɤɣ}}}\markboth{sɤndɤɣ}{}
\classe{vs}
\paradigme{\textit{dir :} \jya tɤ-}
\begin{définition}\ 
\begin{déclaration}\grammar{denom}\end{déclaration}\end{définition}
\begin{définition}\fra vénéneux\end{définition}
\begin{définition}\cmn 有毒性;导致中毒(食物)\end{définition}
\begin{exemple}\jya tɤjmɤɣ ɲɯ-sɤndɤɣ\cmn 蘑菇是有毒的\end{exemple}
\begin{relation-sémantique}\confer{
\hyperlink{Ⓔtɤndɤɣ}{\textit{ \papi{tɤndɤɣ}}}
}\end{relation-sémantique}
\begin{relation-sémantique}\confer{
\hyperlink{Ⓔznɤndɤɣ}{\textit{ \papi{znɤndɤɣ}}}
}\end{relation-sémantique}\end{entrée}

\begin{entrée}
\vedette{\hypertarget{ⒺsɤndɤrⒽ2}{\papi{ sɤndɤr}}}\markboth{sɤndɤr}{}\homonyme{2}\classe{n}
\begin{définition}\fra dé à coudre\end{définition}
\begin{définition}\cmn 顶针\end{définition}
\begin{exemple}\jya kɤ-rɤtʂɯβ ɯ-tshɯɣa nɯ kɯrɯ cho kupa ɣɯ ɯ-tɕhɤjlɯz mɤ-naχtɕɯɣ tɕe, taqaβ kɤ-ndo ɯ-tshɯɣa mɤ-naχtɕɯɣ tɕe, sɤndɤr kɤ-ntɕhoz ɯ-tshɯɣa kɯnɤ mɤ-naχtɕɯɣ tɕe, kupa ra kɯ sɤndɤr nɯ-jaʁndzu mɤpaχcɤl ɣɯ ɯ-qa zɯ lu-rʁe-nɯ ŋu ma taqaβ nɯ kɯ chɯ-sɯ-sthoʁ-nɯ ɲɯ-ra, kɯrɯ ra kɯ sɤndɤr nɯ ɯ-jaʁndzu maŋlo nɯ ɣɯ ɯ-kɤχcɤl nɯ tɕu tu-sɯ-ndo-j ŋu ma nɯ kɯ taqaβ sɯ-sthoʁ-i ra. tɕe kupa ra thɯ-rɤtʂɯβ-nɯ tɕe ɯ-ʁɤri lu-ɕe-nɯ ŋu, kɯrɯ ra chɯ-rɤtʂɯβ-i tɕe, ɯ-qhu chu chɯ-cit-i ŋu.\cmn 藏族和汉族缝针的风俗不同,拿针的方式不一样,所以戴顶针的部位也不一样。汉族是把顶针戴在中指的根部,用这个部位来顶针,而我们藏族是把顶针戴在食指顶上,用这个部位来顶针。汉族缝布时是往前逢,我们藏族是向后缝。\end{exemple}\end{entrée}

\begin{entrée}
\vedette{\hypertarget{ⒺsɤndɤrⒽ1}{\papi{ sɤndɤr}}}\markboth{sɤndɤr}{}\homonyme{1}
\classe{vt}
\paradigme{\textit{dir :} \jya nɯ-}
\begin{définition}\fra faire mal en touchant une blessure\end{définition}
\begin{définition}\cmn 碰(伤口)\end{définition}
\begin{exemple}\jya a-jaʁ na-sɤndɤr\cmn 他碰了我的手,把我弄得很痛\end{exemple}
\begin{exemple}\jya tɯ-ɣmaz nɯ-tɯ-sɤndɤr\cmn 你碰了伤口\end{exemple}
\begin{exemple}\jya a-tɯ-ɣmaz ma-nɯ-tɯ-sɤndɤr\cmn 你不要碰我的伤口\end{exemple}
\begin{exemple}\jya nɤ-jaʁ ɲɯ-mŋɤm tɕe, ma-nɯ-tɯ-sɤndɤr\cmn 你的手既然很痛,不要碰它\end{exemple}
\begin{relation-sémantique}\confer{
\hyperlink{Ⓔandɤr}{\textit{ \papi{andɤr}}}
}\end{relation-sémantique}\end{entrée}

\begin{entrée}
\vedette{\hypertarget{Ⓔsɤndɤrndɤr}{\papi{ sɤndɤrndɤr}}}\markboth{sɤndɤrndɤr}{}
\classe{vt}
\paradigme{\textit{dir :} \jya tɤ-}
\paradigme{\textit{dir :} \jya nɯ-}
\begin{définition}\ 
\begin{déclaration}\grammar{deidph}\end{déclaration}\end{définition}
\begin{définition}\fra faire du bruit en vibrant fortement\end{définition}
\begin{définition}\cmn 震动得很响,发出声音\end{définition}
\begin{exemple}\jya tɤrmbɣo ɲɯ-sɤndɤrndɤr\cmn 打鼓打得很响\end{exemple}\end{entrée}

\begin{entrée}
\vedette{\hypertarget{Ⓔsɤndɣɤndɣɤt}{\papi{ sɤndɣɤndɣɤt}}}\markboth{sɤndɣɤndɣɤt}{}
\begin{relation-sémantique}\confer{
\hyperlink{Ⓔɣɤndɣɤndɣɤt}{\textit{ \papi{ɣɤndɣɤndɣɤt}}}
}\end{relation-sémantique}\end{entrée}

\begin{entrée}
\vedette{\hypertarget{Ⓔsɤndɯja}{\papi{ sɤndɯja}}}\markboth{sɤndɯja}{}
\begin{relation-sémantique}\confer{
\hyperlink{Ⓔandɯja}{\textit{ \papi{andɯja}}}
}\end{relation-sémantique}\end{entrée}

\begin{entrée}
\vedette{\hypertarget{Ⓔsɤndɯndo}{\papi{ sɤndɯndo}}}\markboth{sɤndɯndo}{}
\begin{relation-sémantique}\confer{
\hyperlink{Ⓔndo}{\textit{ \papi{ndo}}}
}\end{relation-sémantique}\end{entrée}

\begin{entrée}
\vedette{\hypertarget{Ⓔsɤndza}{\papi{ sɤndza}}}\markboth{sɤndza}{}
\begin{relation-sémantique}\confer{
\hyperlink{Ⓔndza}{\textit{ \papi{ndza}}}
}\end{relation-sémantique}\end{entrée}

\begin{entrée}
\vedette{\hypertarget{Ⓔsɤndzoʁjoʁ}{\papi{ sɤndzoʁjoʁ}}}\markboth{sɤndzoʁjoʁ}{}
\begin{relation-sémantique}\confer{
\hyperlink{Ⓔandzoʁjoʁ}{\textit{ \papi{andzoʁjoʁ}}}
}\end{relation-sémantique}\end{entrée}

\begin{entrée}
\vedette{\hypertarget{Ⓔsɤndzɯrndzɯr}{\papi{ sɤndzɯrndzɯr}}}\markboth{sɤndzɯrndzɯr}{}
\begin{relation-sémantique}\confer{
\hyperlink{Ⓔɣɤndzɯrndzɯr}{\textit{ \papi{ɣɤndzɯrndzɯr}}}
}\end{relation-sémantique}\end{entrée}

\begin{entrée}
\vedette{\hypertarget{Ⓔsɤndʑɤmstu}{\papi{ sɤndʑɤmstu}}}\markboth{sɤndʑɤmstu}{}
\begin{relation-sémantique}\confer{
\hyperlink{Ⓔandʑɤmstu}{\textit{ \papi{andʑɤmstu}}}
}\end{relation-sémantique}\end{entrée}

\begin{entrée}
\vedette{\hypertarget{Ⓔsɤndʑɯ}{\papi{ sɤndʑɯ}}}\markboth{sɤndʑɯ}{}
\begin{relation-sémantique}\confer{
\hyperlink{Ⓔndʑɯ}{\textit{ \papi{ndʑɯ}}}
}\end{relation-sémantique}
\end{entrée}

\begin{entrée}
\vedette{\hypertarget{Ⓔsɤntɕhoʁjɤr}{\papi{ sɤntɕhoʁjɤr}}}\markboth{sɤntɕhoʁjɤr}{}
\begin{relation-sémantique}\confer{
\hyperlink{Ⓔantɕhoʁjɤr}{\textit{ \papi{antɕhoʁjɤr}}}
}\end{relation-sémantique}\end{entrée}

\begin{entrée}
\vedette{\hypertarget{Ⓔsɤntɕhoz}{\papi{ sɤntɕhoz}}}\markboth{sɤntɕhoz}{}
\classe{n}
\begin{définition}\fra utilité\end{définition}
\begin{définition}\cmn 用处\end{définition}
\begin{exemple}\jya ki tɯ-rju tɯ-ŋka ki ɯ-sɤntɕhoz ɲɯ-dɤn\cmn 这一句话很有用\end{exemple}\end{entrée}

\begin{entrée}
\vedette{\hypertarget{Ⓔsɤntɕhɯ}{\papi{ sɤntɕhɯ}}}\markboth{sɤntɕhɯ}{}
\begin{relation-sémantique}\confer{
\hyperlink{Ⓔantɕhɯ}{\textit{ \papi{antɕhɯ}}}
}\end{relation-sémantique}\end{entrée}

\begin{entrée}
\vedette{\hypertarget{Ⓔsɤnthɣar}{\papi{ sɤnthɣar}}}\markboth{sɤnthɣar}{}
\begin{relation-sémantique}\confer{
\hyperlink{Ⓔanthɣar}{\textit{ \papi{anthɣar}}}
}\end{relation-sémantique}\end{entrée}

\begin{entrée}
\vedette{\hypertarget{Ⓔsɤntsɤndu}{\papi{ sɤntsɤndu}}}\markboth{sɤntsɤndu}{}
\classe{vt}
\paradigme{\textit{dir :} \jya tɤ-}
\begin{définition}\ 
\begin{déclaration}\grammar{caus}\end{déclaration}\end{définition}
\begin{définition}\fra échanger\end{définition}
\begin{définition}\cmn 交换\end{définition}
\begin{exemple}\jya tɕi-ŋga to-nɯ-sɤntsɤndu-tɕi\cmn 我们不小心把衣服交换了\end{exemple}
\begin{exemple}\jya khɯtsa to-sɤntsɤndu-tɕi\cmn 我们无意中把碗交换了\end{exemple}
\begin{exemple}\jya ndʑi-rmi ɲɤ-sɤntsɤndu-t-a\cmn 我说错了他们的名字\end{exemple}
\begin{relation-sémantique}\confer{
\hyperlink{Ⓔantsɤndu}{\textit{ \papi{antsɤndu}}}
}\end{relation-sémantique}\end{entrée}

\begin{entrée}
\vedette{\hypertarget{Ⓔsɤntɯ}{\papi{ sɤntɯ}}}\markboth{sɤntɯ}{}
\begin{relation-sémantique}\confer{
\hyperlink{Ⓔantɯ}{\textit{ \papi{antɯ}}}
}\end{relation-sémantique}\end{entrée}

\begin{entrée}
\vedette{\hypertarget{Ⓔsɤnɯβlu}{\papi{ sɤnɯβlu}}}\markboth{sɤnɯβlu}{}
\begin{relation-sémantique}\confer{
\hyperlink{Ⓔnɯβlu}{\textit{ \papi{nɯβlu}}}
}\end{relation-sémantique}\end{entrée}

\begin{entrée}
\vedette{\hypertarget{Ⓔsɤnɯɕtar}{\papi{ sɤnɯɕtar}}}\markboth{sɤnɯɕtar}{}
\begin{relation-sémantique}\confer{
\hyperlink{Ⓔnɯɕtar}{\textit{ \papi{nɯɕtar}}}
}\end{relation-sémantique}
\end{entrée}

\begin{entrée}
\vedette{\hypertarget{Ⓔsɤnɯkhramba}{\papi{ sɤnɯkhramba}}}\markboth{sɤnɯkhramba}{}
\begin{relation-sémantique}\confer{
\hyperlink{Ⓔnɯkhramba}{\textit{ \papi{nɯkhramba}}}
}\end{relation-sémantique}\end{entrée}

\begin{entrée}
\vedette{\hypertarget{Ⓔsɤnɯmpa}{\papi{ sɤnɯmpa}}}\markboth{sɤnɯmpa}{}
\begin{relation-sémantique}\confer{
\hyperlink{Ⓔnɯmpa}{\textit{ \papi{nɯmpa}}}
}\end{relation-sémantique}
\end{entrée}

\begin{entrée}
\vedette{\hypertarget{Ⓔsɤnɯmtɕhu}{\papi{ sɤnɯmtɕhu}}}\markboth{sɤnɯmtɕhu}{}
\begin{relation-sémantique}\confer{
\hyperlink{Ⓔnɯmtɕhu}{\textit{ \papi{nɯmtɕhu}}}
}\end{relation-sémantique}\end{entrée}

\begin{entrée}
\vedette{\hypertarget{Ⓔsɤnɯmthɯ}{\papi{ sɤnɯmthɯ}}}\markboth{sɤnɯmthɯ}{}
\begin{relation-sémantique}\confer{
\hyperlink{ⒺnɯmthɯⒽ1}{\textit{ \papi{nɯmthɯ}}}
}\end{relation-sémantique}\end{entrée}

\begin{entrée}
\vedette{\hypertarget{Ⓔsɤnɯŋumit}{\papi{ sɤnɯŋumit}}}\markboth{sɤnɯŋumit}{}
\begin{relation-sémantique}\confer{
\hyperlink{Ⓔnɯŋumit}{\textit{ \papi{nɯŋumit}}}
}\end{relation-sémantique}\end{entrée}

\begin{entrée}
\vedette{\hypertarget{Ⓔsɤnɯrga}{\papi{ sɤnɯrga}}}\markboth{sɤnɯrga}{}
\begin{relation-sémantique}\confer{
\hyperlink{Ⓔnɯrga}{\textit{ \papi{nɯrga}}}
}\end{relation-sémantique}\end{entrée}

\begin{entrée}
\vedette{\hypertarget{Ⓔsɤnɯrtɕa}{\papi{ sɤnɯrtɕa}}}\markboth{sɤnɯrtɕa}{}
\begin{relation-sémantique}\confer{
\hyperlink{Ⓔnɯrtɕa}{\textit{ \papi{nɯrtɕa}}}
}\end{relation-sémantique}\end{entrée}

\begin{entrée}
\vedette{\hypertarget{Ⓔsɤnɯrɯtʂa}{\papi{ sɤnɯrɯtʂa}}}\markboth{sɤnɯrɯtʂa}{}
\begin{relation-sémantique}\confer{
\hyperlink{Ⓔnɯrɯtʂa}{\textit{ \papi{nɯrɯtʂa}}}
}\end{relation-sémantique}\end{entrée}

\begin{entrée}
\vedette{\hypertarget{Ⓔsɤnɯsɯkho}{\papi{ sɤnɯsɯkho}}}\markboth{sɤnɯsɯkho}{}
\begin{relation-sémantique}\confer{
\hyperlink{Ⓔnɯsɯkho}{\textit{ \papi{nɯsɯkho}}}
}\end{relation-sémantique}\end{entrée}

\begin{entrée}
\vedette{\hypertarget{Ⓔsɤnɯtɕetha}{\papi{ sɤnɯtɕetha}}}\markboth{sɤnɯtɕetha}{}
\begin{relation-sémantique}\confer{
\hyperlink{Ⓔnɯtɕetha}{\textit{ \papi{nɯtɕetha}}}
}\end{relation-sémantique}\end{entrée}

\begin{entrée}
\vedette{\hypertarget{Ⓔsɤnɯtɯtɕhɯ}{\papi{ sɤnɯtɯtɕhɯ}}}\markboth{sɤnɯtɯtɕhɯ}{}
\begin{relation-sémantique}\confer{
\hyperlink{Ⓔnɯtɯtɕhɯ}{\textit{ \papi{nɯtɯtɕhɯ}}}
}\end{relation-sémantique}\end{entrée}

\begin{entrée}
\vedette{\hypertarget{Ⓔsɤnɯzdɯɣ}{\papi{ sɤnɯzdɯɣ}}}\markboth{sɤnɯzdɯɣ}{}
\begin{relation-sémantique}\confer{
\hyperlink{Ⓔnɯzdɯɣ}{\textit{ \papi{nɯzdɯɣ}}}
}\end{relation-sémantique}
\end{entrée}

\begin{entrée}
\vedette{\hypertarget{Ⓔsɤnɯʑɤzdaŋ}{\papi{ sɤnɯʑɤzdaŋ}}}\markboth{sɤnɯʑɤzdaŋ}{}
\begin{relation-sémantique}\confer{
\hyperlink{Ⓔnɯʑɤzdaŋ}{\textit{ \papi{nɯʑɤzdaŋ}}}
}\end{relation-sémantique}
\end{entrée}

\begin{entrée}
\vedette{\hypertarget{Ⓔsɤɲaj}{\papi{ sɤɲaj}}}\markboth{sɤɲaj}{}
\classe{vt}
\paradigme{\textit{dir :} \jya \_}
\begin{définition}\ 
\begin{déclaration}\grammar{caus}\end{déclaration}\end{définition}
\begin{définition}\fra accélérer un travail\end{définition}
\begin{définition}\cmn 赶紧把工作做完\end{définition}
\begin{exemple}\jya tɯ-ntʂu la-sɤɲaj\cmn 他赶紧锄草了\end{exemple}
\begin{exemple}\jya kɤ-ntʂu lɤ-sɤɲaj-a\cmn 我赶紧锄了草\end{exemple}
\begin{exemple}\jya kɤ-nɯzʁe ka-sɤɲaj\cmn 他赶紧搬了东西\end{exemple}
\begin{exemple}\jya a-zda ra pɯ-nɯna-nɯ ɯ-tsi aʑo pɯ-nɯ-sɤɲaj-a (ʑ-lɤ-sɤɲaj-a)\cmn 我的伙伴休息的时候,我趁机继续工作,早点把它做完\end{exemple}
\begin{exemple}\jya kɤ-raχtɕi nɯ-sɤɲaj-a\cmn 我赶紧地洗了\end{exemple}
\begin{exemple}\jya kɤ-rɤrɤt pɯ-sɤɲaj-a\cmn 我赶紧地写了\end{exemple}
\begin{relation-sémantique}\confer{
\hyperlink{Ⓔaɲaj}{\textit{ \papi{aɲaj}}}
}\end{relation-sémantique}\end{entrée}

\begin{entrée}
\vedette{\hypertarget{Ⓔsɤɲcɣɤɲcɣɤt}{\papi{ sɤɲcɣɤɲcɣɤt}}}\markboth{sɤɲcɣɤɲcɣɤt}{}
\begin{relation-sémantique}\confer{
\hyperlink{Ⓔɣɤɲcɣɤɲcɣɤt}{\textit{ \papi{ɣɤɲcɣɤɲcɣɤt}}}
}\end{relation-sémantique}\end{entrée}

\begin{entrée}
\vedette{\hypertarget{Ⓔsɤɲizɲiz}{\papi{ sɤɲizɲiz}}}\markboth{sɤɲizɲiz}{}
\begin{relation-sémantique}\confer{
\hyperlink{Ⓔɣɤɲizɲiz}{\textit{ \papi{ɣɤɲizɲiz}}}
}\end{relation-sémantique}\end{entrée}

\begin{entrée}
\vedette{\hypertarget{Ⓔsɤɲɟu}{\papi{ sɤɲɟu}}}\markboth{sɤɲɟu}{}
\classe{n}
\begin{définition}\fra herbe qu'on donne à manger à la vache pendant la traite\end{définition}
\begin{définition}\cmn 挤奶时,喂奶牛的草\end{définition}\end{entrée}

\begin{entrée}
\vedette{\hypertarget{Ⓔsɤɲɟɯrnor}{\papi{ sɤɲɟɯrnor}}}\markboth{sɤɲɟɯrnor}{}\classe{vt}
\paradigme{\textit{dir :} \jya nɯ-}
\begin{définition}\fra se tromper\end{définition}
\begin{définition}\cmn 弄错\end{définition}
\begin{exemple}\jya tʂu ma-nɯ-tɯ-sɤɲɟɯrnor\cmn 你不要把路弄错了\end{exemple}
\begin{relation-sémantique}\confer{
\hyperlink{Ⓔɲɟɯrnor}{\textit{ \papi{ɲɟɯrnor}}}
}\end{relation-sémantique}\end{entrée}

\begin{entrée}
\vedette{\hypertarget{Ⓔsɤŋu}{\papi{ sɤŋu}}}\markboth{sɤŋu}{}\classe{n}
\begin{définition}\ 
\begin{déclaration}\grammar{n.lieu}\end{déclaration}\end{définition}
\begin{définition}\fra Gdong.brgyad\end{définition}
\begin{définition}\cmn 龙尔甲\end{définition}
\begin{exemple}\jya sɤŋupɯ\cmn 龙尔甲人\end{exemple}\end{entrée}

\begin{entrée}
\vedette{\hypertarget{Ⓔsɤŋɤβ}{\papi{ sɤŋɤβ}}}\markboth{sɤŋɤβ}{}\classe{vs}
\paradigme{\textit{dir :} \jya nɯ-}\acception{1}
\begin{définition}\fra faible, en mauvaise santé\end{définition}
\begin{définition}\cmn 弱;不健康\end{définition}
\begin{exemple}\jya tɤ-pɤtso ɲɯ-sɤŋɤβ, ɲɯ-xtɕi\cmn 小孩子又小又弱\end{exemple}\acception{2}
\begin{définition}\fra qui dérange, qui ne donne pas envie de\end{définition}
\begin{définition}\cmn 令人不愿意;不好意思
\end{définition}
\begin{exemple}\jya kɤ-ɕe ɲɯ-sɤŋɤβ\cmn 不想去\end{exemple}
\begin{exemple}\jya kɯ-rɤma jɤ-ari-a ri, ɲɤ-maqhu-a tɕe ɲɯ-sɤŋɤβ\cmn 我去工作迟到了,有点不好意思\end{exemple}
\begin{exemple}\jya kɤ-nɤma ɯ-tɯ-dɤn tɕe, kɤ-sɤʑa ɲɯ-sɤŋɤβ\cmn 工作很多,不想开工\end{exemple}
\begin{exemple}\jya rasti laji nɯ ra to-ɬoʁ ri ɲɯ-sɤŋɤβ ma ɯʑo ri ɲɯ-xtɕi, qajɯ kɯ ri to-ndza\cmn 他种的大头菜虽然长出来了,但是令人不想要,因为又小又加上被虫吃了\end{exemple}\acception{3}
\begin{définition}\fra à l'apparence peu avenante\end{définition}
\begin{définition}\cmn 丑陋,看起来不顺眼\end{définition}\begin{sous-entrée}
\vedette{\hypertarget{}{\papi{ nɤŋɤβ}}}\markboth{nɤŋɤβ}{}\classe{vt}
\paradigme{\textit{dir :} \jya nɯ-}
\begin{définition}\ 
\begin{déclaration}\grammar{trop}\end{déclaration}\end{définition}
\begin{définition}\fra être gêné par, hésiter à\end{définition}
\begin{définition}\cmn 觉得不好意思\end{définition}
\begin{exemple}\jya nɯ-nɤŋaβ-a\cmn 我觉得不好意思了\end{exemple}
\end{sous-entrée}\begin{sous-entrée}
\vedette{\hypertarget{}{\papi{ znɤŋɤβ}}}\markboth{znɤŋɤβ}{}\classe{vt}
\paradigme{\textit{dir :} \jya nɯ-}
\begin{définition}\fra gêner, embarrasser\end{définition}
\begin{définition}\cmn 令人不好意思\end{définition}
\begin{relation-sémantique}\antonyme{
\hyperlink{Ⓔsɤmgro}{\textit{ \papi{sɤmgro}}}
}\end{relation-sémantique}
\end{sous-entrée}\end{entrée}

\begin{entrée}
\vedette{\hypertarget{Ⓔsɤŋɤβdi}{\papi{ sɤŋɤβdi}}}\markboth{sɤŋɤβdi}{}\classe{n}
\begin{définition}\fra mauvaise odeur\end{définition}
\begin{définition}\cmn 臭味\end{définition}
\begin{exemple}\jya sɤŋɤβdi ʑo ɲɯ-mnɤm\cmn 有臭味\end{exemple}
\begin{relation-sémantique}\confer{
\hyperlink{Ⓔtɤ-di}{\textit{ \papi{tɤ-di}}}
}\end{relation-sémantique}\end{entrée}

\begin{entrée}
\vedette{\hypertarget{Ⓔsɤŋgɤrɤt}{\papi{ sɤŋgɤrɤt}}}\markboth{sɤŋgɤrɤt}{}
\classe{n}
\begin{définition}\fra ustensile de cuisine pour mélanger la tsampa lorsqu'on la fait frire\end{définition}
\begin{définition}\cmn 炒糌粑时,用来搅拌青稞的工具\end{définition}\end{entrée}

\begin{entrée}
\vedette{\hypertarget{Ⓔsɤŋgio}{\papi{ sɤŋgio}}}\markboth{sɤŋgio}{}\classe{vs}
\paradigme{\textit{dir :} \jya tɤ-}
\begin{définition}\ 
\begin{déclaration}\grammar{deexp}\end{déclaration}
\begin{déclaration}\grammar{acaus}\end{déclaration}\end{définition}
\begin{définition}\fra glissant\end{définition}
\begin{définition}\cmn 滑(路)\end{définition}
\begin{relation-sémantique}\confer{
\hyperlink{Ⓔŋgio}{\textit{ \papi{ŋgio}}}
}\end{relation-sémantique}\end{entrée}

\begin{entrée}
\vedette{\hypertarget{ⒺsɤŋoⒽ1Ⓗ1}{\papi{ sɤŋo}}}\markboth{sɤŋo}{}\homonyme{1}
\classe{vi-t}
\paradigme{\textit{dir :} \jya nɯ-}
\begin{définition}\fra écouter, ressentir\end{définition}
\begin{définition}\cmn 听;感受到\end{définition}\begin{sous-entrée}
\vedette{\hypertarget{}{\papi{ sɤŋo}}}\markboth{sɤŋo}{}
\paradigme{\textit{dir :} \jya kɤ-}
\begin{définition}\fra obéir\end{définition}
\begin{définition}\cmn 服从;听别人的劝告\end{définition}
\begin{exemple}\jya nɯ-sɤŋo-a\cmn 我听了\end{exemple}
\begin{exemple}\jya tɤ-ti jɤɣ ma kɯre ku-sɤŋo-a\cmn 你讲吧,我在这里听着\end{exemple}
\begin{exemple}\jya kɤ-sɤŋo-t-a\cmn 我听从他了\end{exemple}
\begin{exemple}\jya wuma ɣɯ-sɤŋo-a\cmn 他很听我的话\end{exemple}
\begin{exemple}\jya @dianhua khro jɤ-lat-a ri, kɯ-sɤŋo maŋe\cmn 我打了很多次电话,没有人接\end{exemple}
\begin{exemple}\jya ŋgumdʑɯɣ ra kɯ ta-tɯt nɯ kɤ-sɤŋo-t-a\cmn 他听从了领导的话\end{exemple}
\begin{exemple}\jya a-wa kɯ ta-tɯt nɯ kɤ-sɤŋo-t-a\cmn 我听从了我父亲的话\end{exemple}
\begin{relation-sémantique}\confer{
\hyperlink{Ⓔmɤrnɤsɤŋo}{\textit{ \papi{mɤrnɤsɤŋo}}}
}\end{relation-sémantique}
\begin{relation-sémantique}\confer{
\hyperlink{Ⓔsɤsɤŋo}{\textit{ \papi{sɤsɤŋo}}}
}\end{relation-sémantique}\classe{vt}
\end{sous-entrée}\end{entrée}

\begin{entrée}
\vedette{\hypertarget{Ⓔsɤŋoʁŋoʁ}{\papi{ sɤŋoʁŋoʁ}}}\markboth{sɤŋoʁŋoʁ}{}
\classe{vt}
\paradigme{\textit{dir :} \jya pɯ-}
\paradigme{\textit{dir :} \jya tɤ-}
\begin{définition}\fra hocher de la tête\end{définition}
\begin{définition}\cmn 点头\end{définition}
\begin{exemple}\jya ɯ-ku ɲɯ-sɤŋoʁŋoʁ\cmn 他在点头\end{exemple}
\begin{relation-sémantique}\confer{
\hyperlink{Ⓔɣɤŋoʁ}{\textit{ \papi{ɣɤŋoʁ}}}
}\end{relation-sémantique}\end{entrée}

\begin{entrée}
\vedette{\hypertarget{Ⓔsɤŋɯr}{\papi{ sɤŋɯr}}}\markboth{sɤŋɯr}{}
\begin{relation-sémantique}\confer{
\hyperlink{Ⓔnɤŋɯr}{\textit{ \papi{nɤŋɯr}}}
}\end{relation-sémantique}\end{entrée}

\begin{entrée}
\vedette{\hypertarget{Ⓔsɤpa}{\papi{ sɤpa}}}\markboth{sɤpa}{}\classe{vt}
\paradigme{\textit{dir :} \jya nɯ-}
\begin{définition}\fra transformer en\end{définition}
\begin{définition}\cmn 使其变成\end{définition}
\begin{exemple}\jya mɤ-kɯ-mpɕɤr nɯ kɯ-mpɕɤr kɤ-sɤpa mɤ-khɯ\cmn 不能把不漂亮的变成漂亮的\end{exemple}\begin{sous-entrée}
\vedette{\hypertarget{}{\papi{ ʑɣɤsɤpa}}}\markboth{ʑɣɤsɤpa}{}\classe{vi}
\paradigme{\textit{dir :} \jya nɯ-}
\begin{définition}\ 
\begin{déclaration}\grammar{refl}\end{déclaration}\end{définition}
\begin{définition}\fra se transformer en\end{définition}
\begin{définition}\cmn 变成\end{définition}
\begin{relation-sémantique}\synonyme{
\hyperlink{Ⓔʑɣɤβzɟɯr}{\textit{ \papi{ʑɣɤβzɟɯr}}}
}\end{relation-sémantique}
\end{sous-entrée}\end{entrée}

\begin{entrée}
\vedette{\hypertarget{Ⓔsɤpɤmbat}{\papi{ sɤpɤmbat}}}\markboth{sɤpɤmbat}{}
\begin{relation-sémantique}\confer{
\hyperlink{Ⓔapɤmbat}{\textit{ \papi{apɤmbat}}}
}\end{relation-sémantique}\end{entrée}

\begin{entrée}
\vedette{\hypertarget{Ⓔsɤpɕoʁ}{\papi{ sɤpɕoʁ}}}\markboth{sɤpɕoʁ}{}
\classe{n}
\begin{définition}\fra endroit\end{définition}
\begin{définition}\cmn 地区
\begin{déclaration} \étymologie{\papi{sa.pʰʲogs}}\end{déclaration}\end{définition}\end{entrée}

\begin{entrée}
\vedette{\hypertarget{Ⓔsɤpɕɯβjɤl}{\papi{ sɤpɕɯβjɤl}}}\markboth{sɤpɕɯβjɤl}{}
\begin{relation-sémantique}\confer{
\hyperlink{Ⓔapɕɯβjɤl}{\textit{ \papi{apɕɯβjɤl}}}
}\end{relation-sémantique}\end{entrée}

\begin{entrée}
\vedette{\hypertarget{Ⓔsɤpe}{\papi{ sɤpe}}}\markboth{sɤpe}{}\classe{vt}
\paradigme{\textit{dir :} \jya tɤ-}
\begin{définition}\fra faire bien\end{définition}
\begin{définition}\cmn 弄好\end{définition}
\begin{exemple}\jya ta-sɤpe\cmn 他弄好了\end{exemple}
\begin{exemple}\jya kɤ-sɤpe ɯ-pɯ-tɯ-cha-nɯ ?\cmn 你们弄好没有?\end{exemple}
\begin{exemple}\jya nɤʑo sɤznɤ aʑo kɤ-nɤma sɤpe-a\cmn 我工作得比你好\end{exemple}
\begin{relation-sémantique}\confer{
\hyperlink{Ⓔpe}{\textit{ \papi{pe}}}
}\end{relation-sémantique}\end{entrée}

\begin{entrée}
\vedette{\hypertarget{Ⓔsɤpɣaʁsci}{\papi{ sɤpɣaʁsci}}}\markboth{sɤpɣaʁsci}{}
\begin{relation-sémantique}\confer{
\hyperlink{Ⓔapɣaʁsci}{\textit{ \papi{apɣaʁsci}}}
}\end{relation-sémantique}\end{entrée}

\begin{entrée}
\vedette{\hypertarget{Ⓔsɤphɤlɤjɤt}{\papi{ sɤphɤlɤjɤt}}}\markboth{sɤphɤlɤjɤt}{}
\begin{relation-sémantique}\confer{
\hyperlink{Ⓔaphɤlɤjɤt}{\textit{ \papi{aphɤlɤjɤt}}}
}\end{relation-sémantique}\end{entrée}

\begin{entrée}
\vedette{\hypertarget{Ⓔsɤphɤr}{\papi{ sɤphɤr}}}\markboth{sɤphɤr}{}\classe{vt}
\paradigme{\textit{dir :} \jya nɯ-}
\begin{définition}\fra secouer, épousseter, faire tomber d'un récipient\end{définition}
\begin{définition}\cmn 拍打,抖掉\end{définition}
\begin{exemple}\jya nɤ-ŋga nɯ-sɤphɤr\cmn 你抖一下衣服\end{exemple}
\begin{exemple}\jya nɯ-sɤphɤr ma thɯci to-ndzoʁ, to-pɣi\cmn 抖一下,(衣服上)粘上了一个什么东西\end{exemple}
\begin{exemple}\jya tɤ-fkɯm, tɯ-ŋga tha-sɤphɤr\cmn 他把口袋、衣服抖了一下\end{exemple}
\begin{exemple}\jya nɯ-sɤphar-a, thɯ-sɤphar-a\cmn 我抖了一下\end{exemple}
\begin{exemple}\jya nɯ-sɤphar-a tɕe a-ŋga ɯ-taʁ rdɯl ra pɯ-mbɤr\cmn 我把衣服抖了一下,上面的尘土掉下来了\end{exemple}
\begin{exemple}\jya tɤjko ɯ-tɯ-tɕur kɯ tɯ-ku ɲɯ-kɯ-sɯ-sɤphɤr ɲɯ-ŋu\cmn 酸菜发酸,酸到昏了头\end{exemple}
\begin{exemple}\jya rtsɤmkɯɣ tha-sɤphɤr\cmn 他把糌粑袋子抖了一下\end{exemple}
\begin{relation-sémantique}\confer{
\hyperlink{Ⓔphɤr}{\textit{ \papi{phɤr}}}
}\end{relation-sémantique}
\begin{relation-sémantique}\confer{
\hyperlink{ⒺmbɤrⒽ2}{\textit{ \papi{mbɤr2}}}
}\end{relation-sémantique}\end{entrée}

\begin{entrée}
\vedette{\hypertarget{Ⓔsɤphɯphrɯɣ}{\papi{ sɤphɯphrɯɣ}}}\markboth{sɤphɯphrɯɣ}{}\classe{vt}
\begin{définition}\fra qui coule bruyament à grosses gouttes\end{définition}
\begin{définition}\cmn 形容答滴答滴地流出来,发出响声的样子\end{définition}
\begin{exemple}\jya ɯ-qom to-sɤphɯphrɯɣ ʑo pjɤ-tɕɤt\cmn 她热泪滚滚\end{exemple}
\begin{exemple}\jya tɯ-mɯ ɲɯ-sɤphɯphrɯɣ ʑo ɲɯ-ɤsɯ-lɤt\cmn 下大雨(雨点比较大)\end{exemple}\end{entrée}

\begin{entrée}
\vedette{\hypertarget{Ⓔsɤpjɤntɤm}{\papi{ sɤpjɤntɤm}}}\markboth{sɤpjɤntɤm}{}
\begin{relation-sémantique}\confer{
\hyperlink{Ⓔapjɤntɤm}{\textit{ \papi{apjɤntɤm}}}
}\end{relation-sémantique}\end{entrée}

\begin{entrée}
\vedette{\hypertarget{Ⓔsɤplaʁplaʁ}{\papi{ sɤplaʁplaʁ}}}\markboth{sɤplaʁplaʁ}{}
\begin{relation-sémantique}\confer{
\hyperlink{Ⓔplaʁplaʁ}{\textit{ \papi{plaʁplaʁ}}}
}\end{relation-sémantique}\end{entrée}

\begin{entrée}
\vedette{\hypertarget{Ⓔsɤprɤtprɤt}{\papi{ sɤprɤtprɤt}}}\markboth{sɤprɤtprɤt}{}\classe{vt}
\paradigme{\textit{dir :} \jya nɯ-}
\paradigme{\textit{dir :} \jya tɤ-}
\begin{définition}\fra frapper le sol du pied\end{définition}
\begin{définition}\cmn 跺脚\end{définition}
\begin{exemple}\jya ɯ-mi ɲɯ-sɤprɤtprɤt\cmn 他在跺脚\end{exemple}
\begin{exemple}\jya ɯ-mi ta-sɤprɤtprɤt\cmn 他跺了脚\end{exemple}\end{entrée}

\begin{entrée}
\vedette{\hypertarget{Ⓔsɤpɯpa}{\papi{ sɤpɯpa}}}\markboth{sɤpɯpa}{}
\classe{vt}
\paradigme{\textit{dir :} \jya tɤ-}
\paradigme{\textit{dir :} \jya nɯ-}\acception{1}
\begin{définition}\fra rassembler et mettre en ordre, préparer\end{définition}
\begin{définition}\cmn 收集;准备好;使其慢慢发展(种植物、养动物)\end{définition}
\begin{exemple}\jya laχtɕha tɤ-sɤpɯpa-t-a\cmn 我把东西准备好了\end{exemple}
\begin{exemple}\jya tɤ-sɤpɯpa-j\cmn 我们准备好了\end{exemple}
\begin{exemple}\jya to-sɤpɯpa\cmn 他准备好了\end{exemple}
\begin{exemple}\jya a-tɕɯ laχtɕha kɯ-tshoz ʑo tɤ-sɤpɯpa-t-a\cmn 我给儿子把东西准备齐全\end{exemple}\acception{2}
\begin{définition}\fra faire se développer\end{définition}
\begin{définition}\cmn 使其慢慢发展(种植物、养动物)\end{définition}
\begin{exemple}\jya ki ji-nɯŋa-mbala kɯ-dɯ-dɤn kɯra ji-nɯŋa mu kɯ-ɤkhra ci pɯ-tu tɕe, nɯ kɯ nɯ-kɤ-sɤpɯpa ŋu\cmn 我们有这么多黄牛和奶牛,完全是从我们过去的一头老花母牛发展起来的\end{exemple}
\begin{exemple}\jya smɤn kɤntɕhɯ-ɣjɤn to-tɯ-χtɯ-t tɕe, khro to-tɯ-sɤpɯpa-t\cmn 你买了很多次药,积累了很多\end{exemple}\end{entrée}

\begin{entrée}
\vedette{\hypertarget{Ⓔsɤpɯpri}{\papi{ sɤpɯpri}}}\markboth{sɤpɯpri}{}
\begin{relation-sémantique}\confer{
\hyperlink{Ⓔapɯpri}{\textit{ \papi{apɯpri}}}
}\end{relation-sémantique}\end{entrée}

\begin{entrée}
\vedette{\hypertarget{Ⓔsɤqɤrle}{\papi{ sɤqɤrle}}}\markboth{sɤqɤrle}{}\classe{vt}
\paradigme{\textit{dir :} \jya nɯ-}
\begin{définition}\ 
\begin{déclaration}\grammar{caus}\end{déclaration}\end{définition}
\begin{définition}\fra sélectionner, diviser, classer\end{définition}
\begin{définition}\cmn 分类;区分;分开\end{définition}
\begin{exemple}\jya fsapaʁ nɯ-sɤqɤrle\cmn 你把牲畜分开一下(例如,把大的跟小的分开)\end{exemple}
\begin{exemple}\jya fsapaʁ nɯ-sɤqɤrle-t-a\cmn 我(根据大小)分开了牲畜\end{exemple}
\begin{exemple}\jya jɯfɕɯr nɯ-sɤqɤrle-t-a, jisŋi kɤ-sɤqɤrle mɤ-ra\cmn 我昨天已经分开了,今天不用分了\end{exemple}
\begin{exemple}\jya mbro jla nɯ-sɤqɤrle-t-a\cmn 我把马和犏牛分开了\end{exemple}
\begin{exemple}\jya tshɤt qaʑo nɯ-sɤqɤrle-j\cmn 我们把山羊和绵羊分开了\end{exemple}
\begin{relation-sémantique}\synonyme{
\hyperlink{Ⓔqɤt}{\textit{ \papi{qɤt}}}
}\end{relation-sémantique}
\begin{relation-sémantique}\confer{
\hyperlink{Ⓔqɤr}{\textit{ \papi{qɤr}}}
}\end{relation-sémantique}
\begin{relation-sémantique}\confer{
\hyperlink{Ⓔaqɤrle}{\textit{ \papi{aqɤrle}}}
}\end{relation-sémantique}\end{entrée}

\begin{entrée}
\vedette{\hypertarget{Ⓔsɤqɤtsa}{\papi{ sɤqɤtsa}}}\markboth{sɤqɤtsa}{}
\begin{relation-sémantique}\confer{
\hyperlink{Ⓔaqɤtsa}{\textit{ \papi{aqɤtsa}}}
}\end{relation-sémantique}\end{entrée}

\begin{entrée}
\vedette{\hypertarget{Ⓔsɤqɤtʂha}{\papi{ sɤqɤtʂha}}}\markboth{sɤqɤtʂha}{}
\begin{relation-sémantique}\confer{
\hyperlink{Ⓔaqɤtʂha}{\textit{ \papi{aqɤtʂha}}}
}\end{relation-sémantique}\end{entrée}

\begin{entrée}
\vedette{\hypertarget{Ⓔsɤqhe}{\papi{ sɤqhe}}}\markboth{sɤqhe}{}
\begin{relation-sémantique}\confer{
\hyperlink{Ⓔaqhe}{\textit{ \papi{aqhe}}}
}\end{relation-sémantique}\end{entrée}

\begin{entrée}
\vedette{\hypertarget{Ⓔsɤqhlɤβlɤβ}{\papi{ sɤqhlɤβlɤβ}}}\markboth{sɤqhlɤβlɤβ}{}\classe{vt}
\paradigme{\textit{dir :} \jya pɯ-}
\begin{définition}\fra émettre un bruit d'éclaboussures\end{définition}
\begin{définition}\cmn (洗衣服的时候)发出溅水声\end{définition}
\begin{exemple}\jya sɤlaŋphɤn ɯ-ŋgɯ tɯ-ŋga pɯ-χtɕi-t-a tɕe pɯ-sɤqhlɤβlaβ-a\cmn 我在盆子里洗衣服的时候发出了很响的溅水声\end{exemple}\end{entrée}

\begin{entrée}
\vedette{\hypertarget{Ⓔsɤqhloŋloŋ}{\papi{ sɤqhloŋloŋ}}}\markboth{sɤqhloŋloŋ}{}
\begin{relation-sémantique}\confer{
\hyperlink{Ⓔqhloŋ}{\textit{ \papi{qhloŋ}}}
}\end{relation-sémantique}\end{entrée}

\begin{entrée}
\vedette{\hypertarget{Ⓔsɤqhlɯβlɯβ}{\papi{ sɤqhlɯβlɯβ}}}\markboth{sɤqhlɯβlɯβ}{}\classe{vt}
\paradigme{\textit{dir :} \jya tɤ-}
\begin{définition}\fra émettre du bruit en agitant l'eau\end{définition}
\begin{définition}\cmn 发出水晃动声\end{définition}
\begin{exemple}\jya tɯ-ŋga nɯ́-wɣ-ʁɟo tɕe tɯ-ci tú-wɣ-sɤqhlɯβlɯβ ʑo ra\cmn 冲衣服的时候就会发出水晃动的声音\end{exemple}\end{entrée}

\begin{entrée}
\vedette{\hypertarget{Ⓔsɤqhɯqha}{\papi{ sɤqhɯqha}}}\markboth{sɤqhɯqha}{}
\classe{vs}
\paradigme{\textit{dir :} \jya tɤ-}
\paradigme{\textit{dir :} \jya thɯ-}
\begin{définition}\fra très actif\end{définition}
\begin{définition}\cmn 调皮;活泼(孩子)\end{définition}
\begin{exemple}\jya ma-tɯ-sɤqhɯqha\cmn 你不要调皮\end{exemple}\end{entrée}

\begin{entrée}
\vedette{\hypertarget{Ⓔsɤqur}{\papi{ sɤqur}}}\markboth{sɤqur}{}
\begin{relation-sémantique}\confer{
\hyperlink{Ⓔqur}{\textit{ \papi{qur}}}
}\end{relation-sémantique}\end{entrée}

\begin{entrée}
\vedette{\hypertarget{Ⓔsɤqrɤcha}{\papi{ sɤqrɤcha}}}\markboth{sɤqrɤcha}{}\classe{n}
\begin{définition}\fra alcool pour recevoir les invités\end{définition}
\begin{définition}\cmn 款待人的酒\end{définition}
\begin{relation-sémantique}\confer{
\hyperlink{Ⓔqru}{\textit{ \papi{qru}}}
}\end{relation-sémantique}
\begin{relation-sémantique}\confer{
\hyperlink{ⒺchaⒽ1}{\textit{ \papi{cha}}}
}\end{relation-sémantique}\end{entrée}

\begin{entrée}
\vedette{\hypertarget{Ⓔsɤr}{\papi{ sɤr}}}\markboth{sɤr}{}
\classe{vs}
\paradigme{\textit{dir :} \jya tɤ-}
\paradigme{\textit{dir :} \jya thɯ-}
\begin{définition}\fra frais\end{définition}
\begin{définition}\cmn 新鲜(食物)\end{définition}
\begin{exemple}\jya ta-mar ɲɯ-sɤr\cmn 酥油很新鲜\end{exemple}
\begin{exemple}\jya tɤ-lu sɤr ʁaznɤ kɤ-tshi\cmn 你要趁牛奶新鲜的时候喝\end{exemple}\end{entrée}

\begin{entrée}
\vedette{\hypertarget{Ⓔsɤraχtɕɤz}{\papi{ sɤraχtɕɤz}}}\markboth{sɤraχtɕɤz}{}
\begin{relation-sémantique}\confer{
\hyperlink{Ⓔraχtɕɤz}{\textit{ \papi{raχtɕɤz}}}
}\end{relation-sémantique}\end{entrée}

\begin{entrée}
\vedette{\hypertarget{Ⓔsɤrɤpa}{\papi{ sɤrɤpa}}}\markboth{sɤrɤpa}{}
\classe{vi}
\paradigme{\textit{dir :} \jya tɤ-}
\begin{définition}\ 
\begin{déclaration}\grammar{comp}\end{déclaration}\end{définition}
\begin{définition}\fra se moquer\end{définition}
\begin{définition}\cmn 取笑\end{définition}
\begin{exemple}\jya ɯʑo ɲɯ-sɤrɤpa\cmn 他取笑(别人)\end{exemple}
\begin{exemple}\jya aj tɤ-sɤrɤpa-a\cmn 我取笑(别人)了\end{exemple}
\begin{exemple}\jya jiɕqha kɯ-sɤrɤpa ci ɲɯ-ŋu\cmn 他是一个爱取笑人的人\end{exemple}
\begin{exemple}\jya a-ɕki ma-tɯ-sɤrɤpa\cmn 你不要跟我开玩笑\end{exemple}
\begin{relation-sémantique}\confer{
\hyperlink{Ⓔsɤre}{\textit{ \papi{sɤre}}}
}\end{relation-sémantique}\end{entrée}

\begin{entrée}
\vedette{\hypertarget{Ⓔsɤrɤt}{\papi{ sɤrɤt}}}\markboth{sɤrɤt}{}
\classe{vt}
\paradigme{\textit{dir :} \jya kɤ-}
\begin{définition}\fra dérouler un fil\end{définition}
\begin{définition}\cmn 卸下来(线)\end{définition}
\begin{exemple}\jya tɤ-ri ka-sɤrɤt\cmn 他把线卸下来了\end{exemple}
\end{entrée}

\begin{entrée}
\vedette{\hypertarget{Ⓔsɤrchɤrchɤt}{\papi{ sɤrchɤrchɤt}}}\markboth{sɤrchɤrchɤt}{}
\begin{relation-sémantique}\confer{
\hyperlink{Ⓔrchɤrchɤt}{\textit{ \papi{rchɤrchɤt}}}
}\end{relation-sémantique}\end{entrée}

\begin{entrée}
\vedette{\hypertarget{Ⓔsɤrchɯɣlɯɣ}{\papi{ sɤrchɯɣlɯɣ}}}\markboth{sɤrchɯɣlɯɣ}{}
\begin{relation-sémantique}\confer{
\hyperlink{Ⓔrchɯɣnɤlɯɣ}{\textit{ \papi{rchɯɣnɤlɯɣ}}}
}\end{relation-sémantique}\end{entrée}

\begin{entrée}
\vedette{\hypertarget{Ⓔsɤrchɯɣrchɯɣ}{\papi{ sɤrchɯɣrchɯɣ}}}\markboth{sɤrchɯɣrchɯɣ}{}
\begin{relation-sémantique}\confer{
\hyperlink{Ⓔrchɯɣnɤlɯɣ}{\textit{ \papi{rchɯɣnɤlɯɣ}}}
}\end{relation-sémantique}\end{entrée}

\begin{entrée}
\vedette{\hypertarget{Ⓔsɤrɕo}{\papi{ sɤrɕo}}}\markboth{sɤrɕo}{}
\begin{relation-sémantique}\confer{
\hyperlink{Ⓔarɕo}{\textit{ \papi{arɕo}}}
}\end{relation-sémantique}\end{entrée}

\begin{entrée}
\vedette{\hypertarget{Ⓔsɤrɕɯβrɕɯβ}{\papi{ sɤrɕɯβrɕɯβ}}}\markboth{sɤrɕɯβrɕɯβ}{}
\begin{relation-sémantique}\confer{
 \papi{ɣɤrɕɯβrɕɯβ}
}\end{relation-sémantique}\end{entrée}

\begin{entrée}
\vedette{\hypertarget{Ⓔsɤre}{\papi{ sɤre}}}\markboth{sɤre}{}\classe{vs}
\paradigme{\textit{dir :} \jya tɤ-}
\paradigme{\textit{dir :} \jya thɯ-}\acception{1}
\begin{définition}\fra amusant\end{définition}
\begin{définition}\cmn 可笑\end{définition}\acception{2}
\begin{définition}\fra excessif\end{définition}
\begin{définition}\cmn 过分\end{définition}
\begin{exemple}\jya nɤ-tɯ-sɤre nɯ\cmn 你好大的胆子\end{exemple}\acception{3}
\begin{définition}\fra extrêmement\end{définition}
\begin{définition}\cmn 极度\end{définition}
\begin{exemple}\jya a-xtu ɯ-tɯ-mŋɤm ɲɯ-sɤre ʑo\cmn 我肚子非常痛\end{exemple}\begin{sous-entrée}
\vedette{\hypertarget{}{\papi{ nɤsɤre}}}\markboth{nɤsɤre}{}\classe{vt}
\begin{définition}\fra trouver amusant\end{définition}
\begin{définition}\cmn 觉得好玩;觉得可笑\end{définition}
\begin{exemple}\jya nɯ kɯ-fse tu-kɤ-ti nɯ aʑo ɲɯ-nɤsɤre-a\cmn 我觉得那种说法很好笑\end{exemple}
\begin{relation-sémantique}\confer{
 \papi{nɤre}
}\end{relation-sémantique}
\begin{relation-sémantique}\confer{
\hyperlink{Ⓔsɤrɤpa}{\textit{ \papi{sɤrɤpa}}}
}\end{relation-sémantique}
\begin{relation-sémantique}\confer{
\hyperlink{Ⓔtɤ-re}{\textit{ \papi{tɤ-re}}}
}\end{relation-sémantique}
\end{sous-entrée}\end{entrée}

\begin{entrée}
\vedette{\hypertarget{Ⓔsɤrga}{\papi{ sɤrga}}}\markboth{sɤrga}{}
\classe{vi}
\paradigme{\textit{dir :} \jya thɯ-}
\begin{définition}\ 
\begin{déclaration}\grammar{deexp}\end{déclaration}\end{définition}
\begin{définition}\fra aimable\end{définition}
\begin{définition}\cmn 可爱\end{définition}
\begin{exemple}\jya jiɕqha nɯ ɲɯ-sɤrga\cmn 那个(孩子)很可爱\end{exemple}
\begin{relation-sémantique}\confer{
 \papi{rga}
}\end{relation-sémantique}
\begin{relation-sémantique}\confer{
\hyperlink{Ⓔnɯrga}{\textit{ \papi{nɯrga}}}
}\end{relation-sémantique}\end{entrée}

\begin{entrée}
\vedette{\hypertarget{Ⓔsɤrɣɤβrɣɤβ}{\papi{ sɤrɣɤβrɣɤβ}}}\markboth{sɤrɣɤβrɣɤβ}{}
\begin{relation-sémantique}\confer{
\hyperlink{Ⓔrɣɤβrɣɤβ}{\textit{ \papi{rɣɤβrɣɤβ}}}
}\end{relation-sémantique}\end{entrée}

\begin{entrée}
\vedette{\hypertarget{Ⓔsɤrɣi}{\papi{ sɤrɣi}}}\markboth{sɤrɣi}{}
\begin{relation-sémantique}\confer{
\hyperlink{Ⓔarɣi}{\textit{ \papi{arɣi}}}
}\end{relation-sémantique}\end{entrée}

\begin{entrée}
\vedette{\hypertarget{Ⓔsɤri}{\papi{ sɤri}}}\markboth{sɤri}{}
\classe{vt}
\paradigme{\textit{dir :} \jya tɤ-}
\begin{définition}\fra ajouter dans\end{définition}
\begin{définition}\cmn 加进;加入
\begin{déclaration}\use{\stylefv{ɣɤjɯ}与\stylefv{sɤri}都可以翻译成“加”,但是用法不同。前者只有“加上去”的意思,而后者带有“加进去混在一起”的意味}\end{déclaration}\end{définition}
\begin{exemple}\jya mtɕhɤnmbrɯ ʑ-lɤ-sɤri-t-a\cmn 我(把粮食)加进供给寺庙的供品里\end{exemple}
\begin{exemple}\jya tɤlɤɕom pɯ-sɤri\cmn 你把奶皮加进去\end{exemple}
\begin{exemple}\jya khɯtsa ɯ-ŋgɯ kɯki ɯ-ro ki pɯ-sɤri-t-a\cmn 我把剩下的加进了碗里\end{exemple}
\begin{relation-sémantique}\confer{
\hyperlink{Ⓔari}{\textit{ \papi{ari}}}
}\end{relation-sémantique}\begin{sous-entrée}
\vedette{\hypertarget{}{\papi{ ʑɣɤsɤri}}}\markboth{ʑɣɤsɤri}{}\classe{vi}
\paradigme{\textit{dir :} \jya tɤ-}
\begin{définition}\ 
\begin{déclaration}\grammar{refl}\end{déclaration}\end{définition}
\begin{définition}\fra devenir membre, être sociable\end{définition}
\begin{définition}\cmn 加入;合群\end{définition}
\begin{exemple}\jya kɯ-ʑɣɤsɤri ci ɲɯ-ŋu\cmn 他是一个合群的人\end{exemple}
\begin{exemple}\jya ndʑi-rca to-zɣɤsɤri\cmn 他加入了他们俩的队伍\end{exemple}
\begin{exemple}\jya aʑo tɤrca tɤ-ʑɣɤsɤri-a\cmn 我加入其中了\end{exemple}
\begin{relation-sémantique}\synonyme{
\hyperlink{Ⓔʑɣɤsɯrku}{\textit{ \papi{ʑɣɤsɯrku}}}
}\end{relation-sémantique}
\end{sous-entrée}\end{entrée}

\begin{entrée}
\vedette{\hypertarget{Ⓔsɤrju}{\papi{ sɤrju}}}\markboth{sɤrju}{}
\begin{relation-sémantique}\confer{
\hyperlink{Ⓔarju}{\textit{ \papi{arju}}}
}\end{relation-sémantique}\end{entrée}

\begin{entrée}
\vedette{\hypertarget{Ⓔsɤrɟɤsno}{\papi{ sɤrɟɤsno}}}\markboth{sɤrɟɤsno}{}
\classe{vt}
\paradigme{\textit{dir :} \jya thɯ-}
\begin{définition}\fra seller\end{définition}
\begin{définition}\cmn 套上马鞍\end{définition}
\begin{exemple}\jya nɤʑo mbro ɕ-thɯ-sɤrɟɤsnɤm\cmn 你给马套上马鞍\end{exemple}
\begin{relation-sémantique}\confer{
\hyperlink{Ⓔtɤ-sno}{\textit{ \papi{tɤ-sno}}}
}\end{relation-sémantique}\end{entrée}

\begin{entrée}
\vedette{\hypertarget{Ⓔsɤrkhɯβrkhɯβ}{\papi{ sɤrkhɯβrkhɯβ}}}\markboth{sɤrkhɯβrkhɯβ}{}
\begin{relation-sémantique}\confer{
\hyperlink{Ⓔnɤrkhɯrkhɯβ}{\textit{ \papi{nɤrkhɯrkhɯβ}}}
}\end{relation-sémantique}\end{entrée}

\begin{entrée}
\vedette{\hypertarget{Ⓔsɤrkɯrku}{\papi{ sɤrkɯrku}}}\markboth{sɤrkɯrku}{}
\begin{relation-sémantique}\confer{
\hyperlink{Ⓔrku}{\textit{ \papi{rku}}}
}\end{relation-sémantique}\end{entrée}

\begin{entrée}
\vedette{\hypertarget{Ⓔsɤrlɤɣrlɤɣ}{\papi{ sɤrlɤɣrlɤɣ}}}\markboth{sɤrlɤɣrlɤɣ}{}
\classe{vt}
\paradigme{\textit{dir :} \jya nɯ-}
\paradigme{\textit{dir :} \jya kɤ-}
\begin{définition}\fra secouer la tête\end{définition}
\begin{définition}\cmn 摇头
\begin{déclaration}\use{表示反对、不同意、没有}\end{déclaration}\end{définition}
\begin{exemple}\jya ɯ-ku ɲɯ-sɤrlɤɣrlɤɣ\cmn 他在摇头\end{exemple}
\begin{exemple}\jya nɯ-sɤrlɤɣrlɤɣ-a\cmn 我摇头了\end{exemple}\end{entrée}

\begin{entrée}
\vedette{\hypertarget{Ⓔsɤrlɤn}{\papi{ sɤrlɤn}}}\markboth{sɤrlɤn}{}
\classe{n}
\begin{définition}\fra trou humide\end{définition}
\begin{définition}\cmn 潮湿地
\begin{déclaration} \étymologie{\papi{sa.rlon}}\end{déclaration}\end{définition}\end{entrée}

\begin{entrée}
\vedette{\hypertarget{Ⓔsɤrlɯrla}{\papi{ sɤrlɯrla}}}\markboth{sɤrlɯrla}{}
\begin{relation-sémantique}\confer{
\hyperlink{Ⓔarlɯrla}{\textit{ \papi{arlɯrla}}}
}\end{relation-sémantique}\end{entrée}

\begin{entrée}
\vedette{\hypertarget{Ⓔsɤrma}{\papi{ sɤrma}}}\markboth{sɤrma}{}\classe{vi}
\begin{définition}\fra au revoir\end{définition}
\begin{définition}\cmn 再见\end{définition}
\begin{exemple}\jya sɤrma-ndʑi je\cmn 你们俩再见\end{exemple}\end{entrée}

\begin{entrée}
\vedette{\hypertarget{Ⓔsɤrmɤβrmɤβ}{\papi{ sɤrmɤβrmɤβ}}}\markboth{sɤrmɤβrmɤβ}{}
\begin{relation-sémantique}\confer{
\hyperlink{Ⓔɣɤrmɤβrmɤβ}{\textit{ \papi{ɣɤrmɤβrmɤβ}}}
}\end{relation-sémantique}\end{entrée}

\begin{entrée}
\vedette{\hypertarget{Ⓔsɤrmbat}{\papi{ sɤrmbat}}}\markboth{sɤrmbat}{}
\begin{relation-sémantique}\confer{
\hyperlink{Ⓔarmbat}{\textit{ \papi{armbat}}}
}\end{relation-sémantique}\end{entrée}

\begin{entrée}
\vedette{\hypertarget{Ⓔsɤrmi}{\papi{ sɤrmi}}}\markboth{sɤrmi}{}\classe{vt}
\paradigme{\textit{dir :} \jya tɤ-}
\begin{définition}\fra nommer\end{définition}
\begin{définition}\cmn 起名\end{définition}
\begin{exemple}\jya kɯnɯβzaŋ to-sɤrmi-nɯ\cmn (他们叫她)她叫做根桑\end{exemple}
\begin{exemple}\jya lɤβzaŋ to-sɤrmi-nɯ\cmn 他们叫他洛桑\end{exemple}\begin{sous-entrée}
\vedette{\hypertarget{}{\papi{ ʑɣɤsɤrmi}}}\markboth{ʑɣɤsɤrmi}{}\classe{vi}
\paradigme{\textit{dir :} \jya tɤ-}
\begin{définition}\fra se choisir comme nom\end{définition}
\begin{définition}\cmn 给自己取名\end{définition}
\begin{relation-sémantique}\confer{
\hyperlink{Ⓔrmi}{\textit{ \papi{rmi}}}
}\end{relation-sémantique}
\begin{relation-sémantique}\confer{
\hyperlink{Ⓔtɤ-rmi}{\textit{ \papi{tɤ-rmi}}}
}\end{relation-sémantique}
\end{sous-entrée}\end{entrée}

\begin{entrée}
\vedette{\hypertarget{Ⓔsɤrndzo}{\papi{ sɤrndzo}}}\markboth{sɤrndzo}{}\classe{n}
\begin{définition}\fra baguette servant à séparer les fils (avant de tisser)\end{définition}
\begin{définition}\cmn 牵杆,用来牵线的工具\end{définition}\end{entrée}

\begin{entrée}
\vedette{\hypertarget{Ⓔsɤrɲɟɤle}{\papi{ sɤrɲɟɤle}}}\markboth{sɤrɲɟɤle}{}
\begin{relation-sémantique}\confer{
\hyperlink{Ⓔarɲɟɤle}{\textit{ \papi{arɲɟɤle}}}
}\end{relation-sémantique}\end{entrée}

\begin{entrée}
\vedette{\hypertarget{Ⓔsɤrɲɯɣrɲɯɣ}{\papi{ sɤrɲɯɣrɲɯɣ}}}\markboth{sɤrɲɯɣrɲɯɣ}{}
\begin{relation-sémantique}\confer{
\hyperlink{Ⓔrɲɯɣrɲɯɣ}{\textit{ \papi{rɲɯɣrɲɯɣ}}}
}\end{relation-sémantique}\end{entrée}

\begin{entrée}
\vedette{\hypertarget{Ⓔsɤrŋɤɣndʑɯr}{\papi{ sɤrŋɤɣndʑɯr}}}\markboth{sɤrŋɤɣndʑɯr}{}
\classe{vt}
\paradigme{\textit{dir :} \jya nɯ-}
\begin{définition}\fra serrer les dents\end{définition}
\begin{définition}\cmn 咬牙,咬牙切齿\end{définition}
\begin{exemple}\jya a-ɕɣa nɯ-sɤrŋɤɣndʑɯr-a\cmn 我咬了牙\end{exemple}
\begin{exemple}\jya nɤ-ɕɣa ɲɯ-tɯ-sɤrŋɤɣndʑɯr ʑo ɲɯ-ŋu\cmn 你在咬牙\end{exemple}
\begin{relation-sémantique}\confer{
\hyperlink{Ⓔrŋɤɣndʑɯr}{\textit{ \papi{rŋɤɣndʑɯr}}}
}\end{relation-sémantique}\end{entrée}

\begin{entrée}
\vedette{\hypertarget{Ⓔsɤrŋgɯŋga}{\papi{ sɤrŋgɯŋga}}}\markboth{sɤrŋgɯŋga}{}\classe{n}
\begin{définition}\fra couverture\end{définition}
\begin{définition}\cmn 被子;铺盖\end{définition}
\begin{relation-sémantique}\confer{
\hyperlink{Ⓔtɯ-ŋga}{\textit{ \papi{tɯ-ŋga}}}
}\end{relation-sémantique}
\begin{relation-sémantique}\confer{
\hyperlink{ⒺrŋgɯⒽ1}{\textit{ \papi{rŋgɯ}}}
}\end{relation-sémantique}\end{entrée}

\begin{entrée}
\vedette{\hypertarget{Ⓔsɤrŋi}{\papi{ sɤrŋi}}}\markboth{sɤrŋi}{}
\begin{relation-sémantique}\confer{
\hyperlink{Ⓔarŋi}{\textit{ \papi{arŋi}}}
}\end{relation-sémantique}\end{entrée}

\begin{entrée}
\vedette{\hypertarget{Ⓔsɤrɴɢlɯm}{\papi{ sɤrɴɢlɯm}}}\markboth{sɤrɴɢlɯm}{}
\begin{relation-sémantique}\confer{
\hyperlink{Ⓔarɴɢlɯm}{\textit{ \papi{arɴɢlɯm}}}
}\end{relation-sémantique}\end{entrée}

\begin{entrée}
\vedette{\hypertarget{Ⓔsɤro}{\papi{ sɤro}}}\markboth{sɤro}{}
\classe{vl}
\paradigme{\textit{dir :} \jya tɤ-}
\begin{définition}\fra poser sur un étalage\end{définition}
\begin{définition}\cmn 摆在架子上\end{définition}
\begin{exemple}\jya ku-kɯ-tɣa tɕe, mɤro ɯ-taʁ tɤɕi qaj, stoʁ staχpɯ tú-wɣ-sɤro ŋu\cmn 收割的时候,要把青稞、小麦、胡豆、豌豆都摆在架子上\end{exemple}\end{entrée}

\begin{entrée}
\vedette{\hypertarget{Ⓔsɤrphɤrphɤβ}{\papi{ sɤrphɤrphɤβ}}}\markboth{sɤrphɤrphɤβ}{}
\begin{relation-sémantique}\confer{
\hyperlink{Ⓔɣɤrphɤrphɤβ}{\textit{ \papi{ɣɤrphɤrphɤβ}}}
}\end{relation-sémantique}\end{entrée}

\begin{entrée}
\vedette{\hypertarget{Ⓔsɤrqhi}{\papi{ sɤrqhi}}}\markboth{sɤrqhi}{}
\begin{relation-sémantique}\confer{
\hyperlink{Ⓔarqhi}{\textit{ \papi{arqhi}}}
}\end{relation-sémantique}\end{entrée}

\begin{entrée}
\vedette{\hypertarget{Ⓔsɤrqhɯrqhɯβ}{\papi{ sɤrqhɯrqhɯβ}}}\markboth{sɤrqhɯrqhɯβ}{}
\classe{vt}
\paradigme{\textit{dir :} \jya tɤ-}
\begin{définition}\fra bruit d'objets durs et secs qui s'entrechoquent\end{définition}
\begin{définition}\cmn 又干又硬的东西相撞发出声音\end{définition}
\begin{exemple}\jya (paʁ kɯ) stoʁ ɲɯ-sɤrqhɯrqhɯβ ɲɯ-ɤsɯ-ndza\cmn (猪)在吃胡豆发出很多声音\end{exemple}
\begin{exemple}\jya rdɤstaʁ ɲɯ-sɤrqhɯrqhɯβ pa-βde\cmn 他把石头扔了,发出很多声音\end{exemple}
\begin{exemple}\jya rdɤstaʁ ɲɯ-sɤrqhɯrqhɯβ ʑo tɯ-ci kɯ tha-tsɯm\cmn 水把石头冲走了,发出很多声音\end{exemple}
\begin{relation-sémantique}\confer{
\hyperlink{Ⓔɣɤrqhɯβrqhɯβ}{\textit{ \papi{ɣɤrqhɯβrqhɯβ}}}
}\end{relation-sémantique}\end{entrée}

\begin{entrée}
\vedette{\hypertarget{Ⓔsɤrʁɯrʁu}{\papi{ sɤrʁɯrʁu}}}\markboth{sɤrʁɯrʁu}{}
\begin{relation-sémantique}\confer{
\hyperlink{Ⓔarʁɯrʁu}{\textit{ \papi{arʁɯrʁu}}}
}\end{relation-sémantique}\end{entrée}

\begin{entrée}
\vedette{\hypertarget{Ⓔsɤrʁɯrʁɯβ}{\papi{ sɤrʁɯrʁɯβ}}}\markboth{sɤrʁɯrʁɯβ}{}\classe{vt}
\paradigme{\textit{dir :} \jya tɤ-}
\begin{définition}\fra faire du bruit en grignotant une nourriture sèche à toute vitesse\end{définition}
\begin{définition}\cmn 吃又干又脆的东西时发出声音\end{définition}
\begin{exemple}\jya tɤ-sɤrʁɯrʁɯβ-a tɤ-ndza-t-a\cmn 我大声地吃了\end{exemple}
\begin{relation-sémantique}\confer{
\hyperlink{Ⓔsɤrqhɯrqhɯβ}{\textit{ \papi{sɤrqhɯrqhɯβ}}}
}\end{relation-sémantique}
\begin{relation-sémantique}\confer{
\hyperlink{Ⓔrʁɯβrʁɯβ}{\textit{ \papi{rʁɯβrʁɯβ}}}
}\end{relation-sémantique}\end{entrée}

\begin{entrée}
\vedette{\hypertarget{Ⓔsɤrtɕhɣaʁ}{\papi{ sɤrtɕhɣaʁ}}}\markboth{sɤrtɕhɣaʁ}{}\classe{vt}
\paradigme{\textit{dir :} \jya nɯ-}
\begin{définition}\fra chicaner (à propos de quelquechose)\end{définition}
\begin{définition}\cmn 计较;找毛病(针对一件事)\end{définition}
\begin{exemple}\jya tɯtɯrca ɕ-tu-rɯndzɤtshi-j pɯ-ŋu ri, ɯʑo kɯ na-sɤrtɕhɣaʁ\cmn 我们本来要一起去吃饭,但是他在那里计较(没有跟我们一起)\end{exemple}
\begin{relation-sémantique}\confer{
\hyperlink{Ⓔɣɤrtɕhɣaʁ}{\textit{ \papi{ɣɤrtɕhɣaʁ}}}
}\end{relation-sémantique}
\begin{relation-sémantique}\synonyme{
\hyperlink{Ⓔsɤtɕɯqaʁ}{\textit{ \papi{sɤtɕɯqaʁ}}}
}\end{relation-sémantique}
\begin{relation-sémantique}\confer{
\hyperlink{Ⓔtɤ-rtɕhɣaʁ,tɕɤt}{\textit{ \papi{tɤ-rtɕhɣaʁ,tɕɤt}}}
}\end{relation-sémantique}
\end{entrée}

\begin{entrée}
\vedette{\hypertarget{Ⓔsɤrtɕhoʁ}{\papi{ sɤrtɕhoʁ}}}\markboth{sɤrtɕhoʁ}{}
\begin{relation-sémantique}\confer{
\hyperlink{Ⓔartɕhoʁ}{\textit{ \papi{artɕhoʁ}}}
}\end{relation-sémantique}\end{entrée}

\begin{entrée}
\vedette{\hypertarget{Ⓔsɤrtɕi}{\papi{ sɤrtɕi}}}\markboth{sɤrtɕi}{}
\begin{relation-sémantique}\confer{
\hyperlink{Ⓔartɕi}{\textit{ \papi{artɕi}}}
}\end{relation-sémantique}\end{entrée}

\begin{entrée}
\vedette{\hypertarget{Ⓔsɤrtsi}{\papi{ sɤrtsi}}}\markboth{sɤrtsi}{}
\begin{relation-sémantique}\confer{
\hyperlink{Ⓔrtsi}{\textit{ \papi{rtsi}}}
}\end{relation-sémantique}
\end{entrée}

\begin{entrée}
\vedette{\hypertarget{Ⓔsɤrtsɯrtso}{\papi{ sɤrtsɯrtso}}}\markboth{sɤrtsɯrtso}{}\classe{vt}
\paradigme{\textit{dir :} \jya tɤ-}
\begin{définition}\fra empiler\end{définition}
\begin{définition}\cmn 堆起来;堆整齐\end{définition}
\begin{exemple}\jya laχtɕha ta-sɤrtsɯrtso\cmn 他把东西堆起来了\end{exemple}
\begin{exemple}\jya jɯɣi ta-sɤrtsɯrtso\cmn 他把书堆起来了\end{exemple}
\begin{exemple}\jya jɯɣi ra tɤ-sɤrtsɯrtso-t-a\cmn 我把书堆起来了\end{exemple}
\begin{relation-sémantique}\synonyme{
\hyperlink{Ⓔrtsɯɣ}{\textit{ \papi{rtsɯɣ}}}
}\end{relation-sémantique}\end{entrée}

\begin{entrée}
\vedette{\hypertarget{Ⓔsɤrtɯmloʁ}{\papi{ sɤrtɯmloʁ}}}\markboth{sɤrtɯmloʁ}{}
\begin{relation-sémantique}\confer{
\hyperlink{Ⓔartɯmloʁ}{\textit{ \papi{artɯmloʁ}}}
}\end{relation-sémantique}\end{entrée}

\begin{entrée}
\vedette{\hypertarget{Ⓔsɤrtɯrtɤβ}{\papi{ sɤrtɯrtɤβ}}}\markboth{sɤrtɯrtɤβ}{}
\begin{relation-sémantique}\confer{
\hyperlink{Ⓔartɯrtɤβ}{\textit{ \papi{artɯrtɤβ}}}
}\end{relation-sémantique}\end{entrée}

\begin{entrée}
\vedette{\hypertarget{Ⓔsɤrɯru}{\papi{ sɤrɯru}}}\markboth{sɤrɯru}{}\classe{vt}
\paradigme{\textit{dir :} \jya tɤ-}
\begin{définition}\fra comparer\end{définition}
\begin{définition}\cmn 比较,核对\end{définition}
\begin{exemple}\jya tɤ-sɤrɯre\cmn 你比较一下\end{exemple}
\begin{exemple}\jya tɤ-sɤrɯru-t-a\cmn 我比较了\end{exemple}
\begin{exemple}\jya ɯʑo cho kɤ-sɤrɯru me\cmn 你不要跟他比\end{exemple}
\begin{exemple}\jya laχtɕha ʁnɯz ɯ-ɲɯ́-naxtɕɯɣ kɯ tú-wɣ-sɤrɯru\cmn 比较一下两个东西是不是一样的\end{exemple}
\begin{exemple}\jya ɕɯ ɣɯ kɯ ɲɯ-dɤn kɯ kɤ-sɤrɯru ɲɯ-ra\cmn 要比较一下谁有最多\end{exemple}
\begin{exemple}\jya tɕiʑo ni tú-wɣ-sɤrɯru tɕe, nɤʑo kɯ ɲɯ-tɯ-tshu\cmn 我们俩做比较的话,你胖一些\end{exemple}\end{entrée}

\begin{entrée}
\vedette{\hypertarget{Ⓔsɤrwa}{\papi{ sɤrwa}}}\markboth{sɤrwa}{}\classe{n}
\begin{définition}\fra grêle\end{définition}
\begin{définition}\cmn 冰雹
\begin{déclaration} \étymologie{\papi{ser.ba}}\end{déclaration}\end{définition}
\begin{exemple}\jya jɯfɕɯndʐi sɤrwa chɤ-lɤt tɕe, tɤ-rɤku ra pjɤ-xtsɯ tɕe mɯ-to-sɤpe\cmn 前天下了冰雹,破坏了庄稼\end{exemple}\end{entrée}

\begin{entrée}
\vedette{\hypertarget{Ⓔsɤrwɤrwɤt}{\papi{ sɤrwɤrwɤt}}}\markboth{sɤrwɤrwɤt}{}
\begin{relation-sémantique}\confer{
\hyperlink{Ⓔɣɤrwɤrwɤt}{\textit{ \papi{ɣɤrwɤrwɤt}}}
}\end{relation-sémantique}\end{entrée}

\begin{entrée}
\vedette{\hypertarget{Ⓔsɤʁe}{\papi{ sɤʁe}}}\markboth{sɤʁe}{}
\begin{relation-sémantique}\confer{
\hyperlink{Ⓔaʁe}{\textit{ \papi{aʁe}}}
}\end{relation-sémantique}\end{entrée}

\begin{entrée}
\vedette{\hypertarget{Ⓔsɤʁombi}{\papi{ sɤʁombi}}}\markboth{sɤʁombi}{}\classe{vs}
\begin{définition}\fra être désespérant\end{définition}
\begin{définition}\cmn 令人没有希望\end{définition}
\begin{exemple}\jya jiɕqha nɯ ɯ-kɯ-mŋɤm ɲɯ-thɯ kɤ-mna ɲɯ-sɤʁombi ɕti\cmn 他的病很严重,没有痊愈的希望\end{exemple}
\begin{relation-sémantique}\confer{
\hyperlink{Ⓔnɤʁombi}{\textit{ \papi{nɤʁombi}}}
}\end{relation-sémantique}\end{entrée}

\begin{entrée}
\vedette{\hypertarget{Ⓔsɤʁʑi}{\papi{ sɤʁʑi}}}\markboth{sɤʁʑi}{}
\classe{n}
\begin{définition}\fra monde\end{définition}
\begin{définition}\cmn 世界
\begin{déclaration} \étymologie{\papi{sa.gʑi}}\end{déclaration}\end{définition}\end{entrée}

\begin{entrée}
\vedette{\hypertarget{Ⓔsɤsaʁjɤr}{\papi{ sɤsaʁjɤr}}}\markboth{sɤsaʁjɤr}{}
\begin{relation-sémantique}\confer{
\hyperlink{Ⓔsaʁjɤr}{\textit{ \papi{saʁjɤr}}}
}\end{relation-sémantique}\end{entrée}

\begin{entrée}
\vedette{\hypertarget{ⒺsɤsatⒽ1Ⓗ1}{\papi{ sɤsat}}}\markboth{sɤsat}{}\homonyme{1}
\classe{vi}
\paradigme{\textit{dir :} \jya pɯ-}
\begin{définition}\ 
\begin{déclaration}\grammar{apass}\end{déclaration}\end{définition}
\begin{définition}\fra tuer des gens\end{définition}
\begin{définition}\cmn 杀人\end{définition}
\begin{exemple}\jya kɤ-sɤsat pjɤ-rɲo\cmn 他曾经杀过人\end{exemple}\begin{sous-entrée}
\vedette{\hypertarget{}{\papi{ sɤsat}}}\markboth{sɤsat}{}\classe{vs}
\begin{définition}\fra mortel (arme)\end{définition}
\begin{définition}\cmn 杀伤力强\end{définition}
\begin{exemple}\jya ki ɕɤmɯɣdɯ ki ɲɯ-sɤsat\cmn 这把枪杀伤力强\end{exemple}
\begin{relation-sémantique}\confer{
\hyperlink{Ⓔsat}{\textit{ \papi{sat}}}
}\end{relation-sémantique}
\end{sous-entrée}\end{entrée}

\begin{entrée}
\vedette{\hypertarget{Ⓔsɤsaχpaʁ}{\papi{ sɤsaχpaʁ}}}\markboth{sɤsaχpaʁ}{}
\begin{relation-sémantique}\confer{
\hyperlink{Ⓔsaχpaʁ}{\textit{ \papi{saχpaʁ}}}
}\end{relation-sémantique}\end{entrée}

\begin{entrée}
\vedette{\hypertarget{Ⓔsɤsɤŋo}{\papi{ sɤsɤŋo}}}\markboth{sɤsɤŋo}{}\paradigme{\textit{dir :} \jya tɤ-}
\begin{définition}\ 
\begin{déclaration}\grammar{apass}\end{déclaration}\end{définition}
\begin{définition}\fra écouter les conseils\end{définition}
\begin{définition}\cmn 听别人的劝告\end{définition}
\begin{exemple}\jya ɲɯ-sɤsɤŋo\cmn 他听别人的意见\end{exemple}
\begin{relation-sémantique}\confer{
 \papi{sɤŋo1}
}\end{relation-sémantique}\classe{vi}\end{entrée}

\begin{entrée}
\vedette{\hypertarget{Ⓔsɤschrɤβlɤβ}{\papi{ sɤschrɤβlɤβ}}}\markboth{sɤschrɤβlɤβ}{}
\begin{relation-sémantique}\confer{
\hyperlink{Ⓔɣɤchrɤβchrɤβ}{\textit{ \papi{ɣɤchrɤβchrɤβ}}}
}\end{relation-sémantique}\end{entrée}

\begin{entrée}
\vedette{\hypertarget{Ⓔsɤscit}{\papi{ sɤscit}}}\markboth{sɤscit}{}\classe{vs}
\paradigme{\textit{dir :} \jya thɯ-}
\begin{définition}\fra heureux, agréable (environnement, époque) amusant (personne)\end{définition}
\begin{définition}\cmn 幸福(环境、时代、生活)、好玩(人)
\begin{déclaration} \étymologie{\papi{skʲid}}\end{déclaration}\end{définition}
\begin{exemple}\jya ji-tɯrma ɲɯ-sɤscit\end{exemple}
\begin{relation-sémantique}\confer{
\hyperlink{Ⓔscit}{\textit{ \papi{scit}}}
}\end{relation-sémantique}
\begin{relation-sémantique}\confer{
\hyperlink{Ⓔnɤsɤscit}{\textit{ \papi{nɤsɤscit}}}
}\end{relation-sémantique}\end{entrée}

\begin{entrée}
\vedette{\hypertarget{Ⓔsɤsco}{\papi{ sɤsco}}}\markboth{sɤsco}{}
\begin{relation-sémantique}\confer{
\hyperlink{Ⓔsco}{\textit{ \papi{sco}}}
}\end{relation-sémantique}\end{entrée}

\begin{entrée}
\vedette{\hypertarget{Ⓔsɤscur}{\papi{ sɤscur}}}\markboth{sɤscur}{}
\classe{n}
\begin{définition}\fra lanière\end{définition}
\begin{définition}\cmn 背带\end{définition}\end{entrée}

\begin{entrée}
\vedette{\hypertarget{Ⓔsɤscɯndu}{\papi{ sɤscɯndu}}}\markboth{sɤscɯndu}{}
\classe{vt}
\paradigme{\textit{dir :} \jya tɤ-}
\paradigme{\textit{dir :} \jya pɯ-}
\begin{définition}\fra échanger\end{définition}
\begin{définition}\cmn 调换\end{définition}
\begin{exemple}\jya tɕi-ŋga tɤ-nɯ-sɤscɯndu-tɕi\cmn 我们调换了衣服\end{exemple}
\begin{exemple}\jya tɯ-rju ɯ-qhu ɯ-ʁɤri ɲɤ-sɤscɯndu-t-a\cmn 我不小心颠倒说话了\end{exemple}
\begin{relation-sémantique}\synonyme{
\hyperlink{Ⓔsɤkɤsci}{\textit{ \papi{sɤkɤsci}}}
}\end{relation-sémantique}
\begin{relation-sémantique}\confer{
\hyperlink{Ⓔsɤndu}{\textit{ \papi{sɤndu}}}
}\end{relation-sémantique}\end{entrée}

\begin{entrée}
\vedette{\hypertarget{Ⓔsɤskɤt}{\papi{ sɤskɤt}}}\markboth{sɤskɤt}{}
\begin{relation-sémantique}\confer{
\hyperlink{Ⓔskɤt}{\textit{ \papi{skɤt}}}
}\end{relation-sémantique}\end{entrée}

\begin{entrée}
\vedette{\hypertarget{Ⓔsɤskɯsku}{\papi{ sɤskɯsku}}}\markboth{sɤskɯsku}{}
\classe{adv}
\begin{définition}\fra tous les matins\end{définition}
\begin{définition}\cmn 每天早上\end{définition}\end{entrée}

\begin{entrée}
\vedette{\hypertarget{Ⓔsɤsma}{\papi{ sɤsma}}}\markboth{sɤsma}{}
\begin{relation-sémantique}\confer{
\hyperlink{Ⓔnɤsma}{\textit{ \papi{nɤsma}}}
}\end{relation-sémantique}\end{entrée}

\begin{entrée}
\vedette{\hypertarget{Ⓔsɤsŋom}{\papi{ sɤsŋom}}}\markboth{sɤsŋom}{}
\begin{relation-sémantique}\confer{
\hyperlink{Ⓔsŋom}{\textit{ \papi{sŋom}}}
}\end{relation-sémantique}\end{entrée}

\begin{entrée}
\vedette{\hypertarget{Ⓔsɤspa}{\papi{ sɤspa}}}\markboth{sɤspa}{}
\begin{relation-sémantique}\confer{
\hyperlink{Ⓔspa}{\textit{ \papi{spa}}}
}\end{relation-sémantique}\end{entrée}

\begin{entrée}
\vedette{\hypertarget{Ⓔsɤsphɯt}{\papi{ sɤsphɯt}}}\markboth{sɤsphɯt}{}
\begin{relation-sémantique}\confer{
\hyperlink{Ⓔsphɯt}{\textit{ \papi{sphɯt}}}
}\end{relation-sémantique}\end{entrée}

\begin{entrée}
\vedette{\hypertarget{Ⓔsɤsqɤr}{\papi{ sɤsqɤr}}}\markboth{sɤsqɤr}{}
\begin{relation-sémantique}\confer{
\hyperlink{Ⓔsqɤr}{\textit{ \papi{sqɤr}}}
}\end{relation-sémantique}\end{entrée}

\begin{entrée}
\vedette{\hypertarget{Ⓔsɤsqra}{\papi{ sɤsqra}}}\markboth{sɤsqra}{}
\classe{n}
\begin{définition}\fra limite\end{définition}
\begin{définition}\cmn 界限\end{définition}\end{entrée}

\begin{entrée}
\vedette{\hypertarget{ⒺsɤstuⒽ1}{\papi{ sɤstu}}}\markboth{sɤstu}{}\homonyme{1}
\classe{vs}
\begin{définition}\fra inspirer confiance\end{définition}
\begin{définition}\cmn 令人相信\end{définition}
\begin{exemple}\jya ɯ-phɯ chondɤre ɯ-@zhiliang nɯra kɯ-sɤstu tsa nɯ tɕu ɕ-tú-wɣ-χtɯ ra\cmn 要在价格和质量都令人信服的地方买\end{exemple}
\begin{relation-sémantique}\confer{
\hyperlink{ⒺstuⒽ1}{\textit{ \papi{stu}}}
}\end{relation-sémantique}\end{entrée}

\begin{entrée}
\vedette{\hypertarget{ⒺsɤstuⒽ2}{\papi{ sɤstu}}}\markboth{sɤstu}{}\homonyme{2}
\classe{vt}\acception{1}
\paradigme{\textit{dir :} \jya tɤ-}
\begin{définition}\fra soutenir, faire correctement un travail\end{définition}
\begin{définition}\cmn 端平;把任务作好\end{définition}
\begin{exemple}\jya khɯsta tɤ-sɤste ma tɤ-lwoʁ\cmn 你要把碗端平不然就会倒出来\end{exemple}\acception{2}
\paradigme{\textit{dir :} \jya \_}
\begin{définition}\fra aller directement\end{définition}
\begin{définition}\cmn 直(走);直接……\end{définition}
\begin{exemple}\jya lú-wɣ-sɤstu ʑo lu-kɯ-ɕe qhe lu-kɯ-zɣɯt ɕti\cmn 直接往前就会到\end{exemple}
\begin{relation-sémantique}\confer{
\hyperlink{Ⓔastu}{\textit{ \papi{astu}}}
}\end{relation-sémantique}
\begin{relation-sémantique}\confer{
\hyperlink{Ⓔsɤstɤko}{\textit{ \papi{sɤstɤko}}}
}\end{relation-sémantique}\end{entrée}

\begin{entrée}
\vedette{\hypertarget{Ⓔsɤstɤko}{\papi{ sɤstɤko}}}\markboth{sɤstɤko}{}
\classe{vt}
\paradigme{\textit{dir :} \jya thɯ-}
\paradigme{\textit{dir :} \jya \_}
\begin{définition}\fra tendre\end{définition}
\begin{définition}\cmn 伸直\end{définition}
\begin{exemple}\jya ɯ-mi tha-sɤstɤko\cmn 他伸了脚\end{exemple}
\begin{exemple}\jya ɕom tha-sɤstɤko\cmn 他把铁打成直条了\end{exemple}
\begin{exemple}\jya nɤ-βri ra nɯ-sɤstɤkɤm\cmn 你舒展一下筋骨吧\end{exemple}
\begin{relation-sémantique}\confer{
\hyperlink{Ⓔastɤko}{\textit{ \papi{astɤko}}}
}\end{relation-sémantique}\begin{sous-entrée}
\vedette{\hypertarget{}{\papi{ ʑɣɤsɤstɤko}}}\markboth{ʑɣɤsɤstɤko}{}\classe{vi}
\begin{définition}\ 
\begin{déclaration}\grammar{refl}\end{déclaration}\end{définition}
\begin{définition}\fra tendre son corps\end{définition}
\begin{définition}\cmn 舒展筋骨\end{définition}
\begin{relation-sémantique}\confer{
\hyperlink{Ⓔastu}{\textit{ \papi{astu}}}
}\end{relation-sémantique}
\begin{relation-sémantique}\confer{
\hyperlink{ⒺsɤstuⒽ2}{\textit{ \papi{sɤstu2}}}
}\end{relation-sémantique}
\end{sous-entrée}\end{entrée}

\begin{entrée}
\vedette{\hypertarget{Ⓔsɤstoŋ}{\papi{ sɤstoŋ}}}\markboth{sɤstoŋ}{}
\classe{n}
\begin{définition}\fra endroit inhabité\end{définition}
\begin{définition}\cmn 荒野
\begin{déclaration} \étymologie{\papi{sa.stoŋ}}\end{déclaration}\end{définition}\end{entrée}

\begin{entrée}
\vedette{\hypertarget{Ⓔsɤsɯβzi}{\papi{ sɤsɯβzi}}}\markboth{sɤsɯβzi}{}
\begin{relation-sémantique}\confer{
\hyperlink{Ⓔβzi}{\textit{ \papi{βzi}}}
}\end{relation-sémantique}\end{entrée}

\begin{entrée}
\vedette{\hypertarget{Ⓔsɤsɯɣ}{\papi{ sɤsɯɣ}}}\markboth{sɤsɯɣ}{}\classe{vt}
\paradigme{\textit{dir :} \jya kɤ-}
\paradigme{\textit{dir :} \jya tɤ-}
\begin{définition}\ 
\begin{déclaration}\grammar{caus}\end{déclaration}\end{définition}
\begin{définition}\fra presser, serrer\end{définition}
\begin{définition}\cmn 挤\end{définition}
\begin{exemple}\jya ɯ-mtɕhi ka-sɤsɯɣ\cmn 他抿了嘴巴\end{exemple}
\begin{exemple}\jya nɤ-jaʁ kɤ-sɤsɯɣ\cmn 你把拳头握紧\end{exemple}
\begin{exemple}\jya tɤ-mtɯ kɤ-sɤsɯɣ\cmn 把结打紧\end{exemple}
\begin{exemple}\jya a-mthɤɣ kɤ-sɤsɯɣ-a\cmn 我把腰带束紧了\end{exemple}
\begin{exemple}\jya a-fkur tɤ-sɤsɯɣ-a\cmn 我把背子捆紧了\end{exemple}
\begin{exemple}\jya tɤ-fkɯm ɯ-mŋu kɤ-sɤsɯɣ-a\cmn 我把袋子的口收紧了\end{exemple}\begin{sous-entrée}
\vedette{\hypertarget{}{\papi{ ʑɣɤsɤsɯɣ}}}\markboth{ʑɣɤsɤsɯɣ}{}\classe{vi}
\paradigme{\textit{dir :} \jya tɤ-}
\begin{définition}\ 
\begin{déclaration}\grammar{refl}\end{déclaration}\end{définition}
\begin{définition}\fra faire le maximum\end{définition}
\begin{définition}\cmn 努力;抓紧时间\end{définition}
\begin{exemple}\jya tɯ-βzjoz kɤ-ʑɣɤsɤsɯɣ ra\cmn 要努力学习\end{exemple}
\begin{relation-sémantique}\confer{
\hyperlink{Ⓔasɯɣ}{\textit{ \papi{asɯɣ}}}
}\end{relation-sémantique}
\end{sous-entrée}\end{entrée}

\begin{entrée}
\vedette{\hypertarget{Ⓔsɤsɯɣli}{\papi{ sɤsɯɣli}}}\markboth{sɤsɯɣli}{}
\begin{relation-sémantique}\confer{
\hyperlink{ⒺliⒽ3}{\textit{ \papi{li3}}}
}\end{relation-sémantique}\end{entrée}

\begin{entrée}
\vedette{\hypertarget{Ⓔsɤsɯɣsɯɣ}{\papi{ sɤsɯɣsɯɣ}}}\markboth{sɤsɯɣsɯɣ}{}\classe{vt}
\paradigme{\textit{dir :} \jya nɯ-}
\begin{définition}\fra toucher avec ...\end{définition}
\begin{définition}\cmn 用……轻轻地擦过、碰到\end{définition}
\begin{exemple}\jya nɤ-ŋga nɯ znde ɯ-taʁ ma-nɯ-tɯ-sɤsɯɣsɯɣ ma sɤɴqhi\cmn 你不要把衣服碰到墙上,很脏\end{exemple}\end{entrée}

\begin{entrée}
\vedette{\hypertarget{Ⓔsɤsɯxɕɤt}{\papi{ sɤsɯxɕɤt}}}\markboth{sɤsɯxɕɤt}{}
\begin{relation-sémantique}\confer{
\hyperlink{Ⓔsɯxɕɤt}{\textit{ \papi{sɯxɕɤt}}}
}\end{relation-sémantique}\end{entrée}

\begin{entrée}
\vedette{\hypertarget{Ⓔsɤsɯz}{\papi{ sɤsɯz}}}\markboth{sɤsɯz}{}\classe{vs}
\begin{définition}\ 
\begin{déclaration}\grammar{deexp}\end{déclaration}\end{définition}
\begin{définition}\fra être connu\end{définition}
\begin{définition}\cmn (人们)知道的\end{définition}
\begin{exemple}\jya ɯ-rmi ɲɯ-sɤsɯz\cmn 人们知道他的名字\end{exemple}
\begin{relation-sémantique}\confer{
\hyperlink{Ⓔsɯz}{\textit{ \papi{sɯz}}}
}\end{relation-sémantique}
\end{entrée}

\begin{entrée}
\vedette{\hypertarget{Ⓔsɤsɯzdɯɣ}{\papi{ sɤsɯzdɯɣ}}}\markboth{sɤsɯzdɯɣ}{}
\begin{relation-sémantique}\confer{
\hyperlink{Ⓔsɯzdɯɣ}{\textit{ \papi{sɯzdɯɣ}}}
}\end{relation-sémantique}\end{entrée}

\begin{entrée}
\vedette{\hypertarget{Ⓔsɤʂɤʂɤt}{\papi{ sɤʂɤʂɤt}}}\markboth{sɤʂɤʂɤt}{}
\classe{vt}
\paradigme{\textit{dir :} \jya tɤ-}
\paradigme{\textit{dir :} \jya kɤ-}
\begin{définition}\fra lire / écrire de manière très fluide\end{définition}
\begin{définition}\cmn 念/写得很流利、吊羊毛吊得很顺手\end{définition}
\begin{exemple}\jya thɯ-rɤrɤt ɲɯ-sɤʂɤʂɤt ʑo\cmn 他写了,写得很流利\end{exemple}
\begin{exemple}\jya kɤ-pɣo ta-sɤʂɤʂɤt ʑo\cmn 他吊了羊毛,吊得很顺手\end{exemple}\end{entrée}

\begin{entrée}
\vedette{\hypertarget{Ⓔsɤʂχɯʂχɯβ}{\papi{ sɤʂχɯʂχɯβ}}}\markboth{sɤʂχɯʂχɯβ}{}
\classe{vt}
\paradigme{\textit{dir :} \jya kɤ-}
\begin{définition}\fra siroter, froisser\end{définition}
\begin{définition}\cmn (喝水时)发出啧啧声音,发出沙沙声\end{définition}
\begin{exemple}\jya tʂha kɤ-sɤʂχɯʂχɯβ\cmn 喝茶时发出啧啧声\end{exemple}
\begin{exemple}\jya ɕoʁɕoʁ ɲɯ-sɤʂχɯʂχɯβ\cmn 他把纸弄皱发出沙沙声\end{exemple}\begin{sous-entrée}
\vedette{\hypertarget{}{\papi{ ɣɤʂχɯʂχɯβ}}}\markboth{ɣɤʂχɯʂχɯβ}{}\classe{vi}
\paradigme{\textit{dir :} \jya kɤ-}
\begin{définition}\fra émettre un bruit de froissement\end{définition}
\begin{définition}\cmn 发出沙沙声\end{définition}
\begin{exemple}\jya tɯ-ndʐi nɯ ɲɯ-ɣɤʂχɯʂχɯβ (kɤ-χtsɤβ mɯ-pjɤ-βdi)\cmn 皮子发出沙沙声(表示没有揉好、很粗糙)\end{exemple}
\begin{relation-sémantique}\confer{
\hyperlink{Ⓔɣɤʂχaβʂχaβ}{\textit{ \papi{ɣɤʂχaβʂχaβ}}}
}\end{relation-sémantique}
\end{sous-entrée}\end{entrée}

\begin{entrée}
\vedette{\hypertarget{Ⓔsɤtaʁki}{\papi{ sɤtaʁki}}}\markboth{sɤtaʁki}{}
\begin{relation-sémantique}\confer{
\hyperlink{Ⓔataʁki}{\textit{ \papi{ataʁki}}}
}\end{relation-sémantique}\end{entrée}

\begin{entrée}
\vedette{\hypertarget{Ⓔsɤtaʁtaʁ}{\papi{ sɤtaʁtaʁ}}}\markboth{sɤtaʁtaʁ}{}
\classe{vt}
\paradigme{\textit{dir :} \jya tɤ-}
\begin{définition}\fra amasser\end{définition}
\begin{définition}\cmn 堆迭\end{définition}
\begin{exemple}\jya laχtɕha tɤ-sɤtaʁtaʁ\cmn 你把东西堆起来\end{exemple}
\begin{exemple}\jya tɯjpu tɤ-sɤtaʁtaʁ\cmn 你把面堆起来\end{exemple}
\begin{exemple}\jya jɯɣi tɤ-sɤtaʁtaʁ\cmn 你把书堆起来\end{exemple}\end{entrée}

\begin{entrée}
\vedette{\hypertarget{Ⓔsɤtɤβ}{\papi{ sɤtɤβ}}}\markboth{sɤtɤβ}{}
\classe{n}
\begin{définition}\fra aire à battre\end{définition}
\begin{définition}\cmn 打场\end{définition}\end{entrée}

\begin{entrée}
\vedette{\hypertarget{Ⓔsɤtɕaʁ}{\papi{ sɤtɕaʁ}}}\markboth{sɤtɕaʁ}{}
\begin{relation-sémantique}\confer{
\hyperlink{Ⓔatɕaʁ}{\textit{ \papi{atɕaʁ}}}
}\end{relation-sémantique}\end{entrée}

\begin{entrée}
\vedette{\hypertarget{Ⓔsɤtɕaʁlaʁ}{\papi{ sɤtɕaʁlaʁ}}}\markboth{sɤtɕaʁlaʁ}{}
\begin{relation-sémantique}\confer{
\hyperlink{Ⓔatɕaʁ}{\textit{ \papi{atɕaʁ}}}
}\end{relation-sémantique}\end{entrée}

\begin{entrée}
\vedette{\hypertarget{Ⓔsɤtɕɤβ}{\papi{ sɤtɕɤβ}}}\markboth{sɤtɕɤβ}{}
\begin{relation-sémantique}\confer{
\hyperlink{Ⓔatɕɤβ}{\textit{ \papi{atɕɤβ}}}
}\end{relation-sémantique}\end{entrée}

\begin{entrée}
\vedette{\hypertarget{Ⓔsɤtɕɤt}{\papi{ sɤtɕɤt}}}\markboth{sɤtɕɤt}{}
\classe{vt}
\paradigme{\textit{dir :} \jya thɯ-}
\begin{définition}\fra attiser le feu\end{définition}
\begin{définition}\cmn 把柴推进火里,把火拨旺\end{définition}
\begin{exemple}\jya smi thɯ-sɤtɕɤt\cmn 你把火拨旺吧\end{exemple}
\begin{exemple}\jya smi kɤ-βlɯ ɲɯ-ra tɕe, nɤʑo smi thɯ-sɤtɕɤt ɲɯ-ntshi\cmn 因为要烧火,请你把柴火推一下\end{exemple}\end{entrée}

\begin{entrée}
\vedette{\hypertarget{Ⓔsɤtɕɣɤrtɕɣɤr}{\papi{ sɤtɕɣɤrtɕɣɤr}}}\markboth{sɤtɕɣɤrtɕɣɤr}{}
\begin{relation-sémantique}\confer{
\hyperlink{Ⓔtɕɣɤrtɕɣɤr}{\textit{ \papi{tɕɣɤrtɕɣɤr}}}
}\end{relation-sémantique}\end{entrée}

\begin{entrée}
\vedette{\hypertarget{Ⓔsɤtɕha}{\papi{ sɤtɕha}}}\markboth{sɤtɕha}{}
\classe{n}
\begin{définition}\fra endroit\end{définition}
\begin{définition}\cmn 地方
\begin{déclaration} \étymologie{\papi{sa.tɕʰa}}\end{déclaration}\end{définition}\end{entrée}

\begin{entrée}
\vedette{\hypertarget{Ⓔsɤtɕhɯ}{\papi{ sɤtɕhɯ}}}\markboth{sɤtɕhɯ}{}\classe{vi}
\paradigme{\textit{dir :} \jya tɤ-}
\begin{définition}\ 
\begin{déclaration}\grammar{apass}\end{déclaration}\end{définition}
\begin{définition}\fra attaquer avec ses cornes\end{définition}
\begin{définition}\cmn 用角打人(牛)\end{définition}
\begin{exemple}\jya mbala kɯ-sɤtɕhɯ\cmn 顶人的公牛\end{exemple}
\begin{relation-sémantique}\confer{
\hyperlink{Ⓔtɕhɯ}{\textit{ \papi{tɕhɯ}}}
}\end{relation-sémantique}\end{entrée}

\begin{entrée}
\vedette{\hypertarget{Ⓔsɤtɕhɯβtɕhɯβ}{\papi{ sɤtɕhɯβtɕhɯβ}}}\markboth{sɤtɕhɯβtɕhɯβ}{}
\classe{vt}
\paradigme{\textit{dir :} \jya pɯ-}
\paradigme{\textit{dir :} \jya tɤ-}
\begin{définition}\fra cligner des yeux\end{définition}
\begin{définition}\cmn 眨眼\end{définition}
\begin{exemple}\jya ɯ-mɲaʁ ɲɯ-sɤtɕhɯβtɕhɯβ\cmn 他在眨眼\end{exemple}
\begin{exemple}\jya a-mɲaʁ tɤ-sɤtɕhɯβtɕhɯβ-a (pɯ-sɤtɕhɯβtɕhɯβ-a)\cmn 我眨了眼\end{exemple}
\begin{relation-sémantique}\confer{
\hyperlink{Ⓔsɤthɤβthɤβ}{\textit{ \papi{sɤthɤβthɤβ}}}
}\end{relation-sémantique}\end{entrée}

\begin{entrée}
\vedette{\hypertarget{Ⓔsɤtɕhɯtɕhɯ}{\papi{ sɤtɕhɯtɕhɯ}}}\markboth{sɤtɕhɯtɕhɯ}{}
\classe{vt}
\paradigme{\textit{dir :} \jya thɯ-}
\begin{définition}\fra se couvrir (de plusieurs couches de vêtements)\end{définition}
\begin{définition}\cmn 套衣服\end{définition}
\begin{exemple}\jya tɯ-ŋga thɯ-sɤtɕhɯtɕhi\cmn 你在上面套上衣服!\end{exemple}
\begin{exemple}\jya tɯ-ŋga kɤntɕhɯ thɯ-sɤtɕhɯtɕhɯ-t-a\cmn 我套了很多件衣服\end{exemple}\end{entrée}

\begin{entrée}
\vedette{\hypertarget{Ⓔsɤtɕhɯz}{\papi{ sɤtɕhɯz}}}\markboth{sɤtɕhɯz}{}
\begin{relation-sémantique}\confer{
\hyperlink{Ⓔatɕhɯz}{\textit{ \papi{atɕhɯz}}}
}\end{relation-sémantique}\end{entrée}

\begin{entrée}
\vedette{\hypertarget{Ⓔsɤtɕɯɣtaʁ}{\papi{ sɤtɕɯɣtaʁ}}}\markboth{sɤtɕɯɣtaʁ}{}
\begin{relation-sémantique}\confer{
\hyperlink{ⒺtaʁⒽ2}{\textit{ \papi{taʁ2}}}
}\end{relation-sémantique}\end{entrée}

\begin{entrée}
\vedette{\hypertarget{Ⓔsɤtɕɯmthɯt}{\papi{ sɤtɕɯmthɯt}}}\markboth{sɤtɕɯmthɯt}{}
\classe{vt}
\paradigme{\textit{dir :} \jya kɤ-}
\begin{définition}\fra mettre ensemble des objets séparés\end{définition}
\begin{définition}\cmn 把零散的东西(线、布片)拼成整体\end{définition}
\begin{exemple}\jya ka-sɤtɕɯmthɯt\cmn 他拼在一起了\end{exemple}
\begin{exemple}\jya ɕomskrɯt kɤ-sɤtɕɯmthɯt-a\cmn 我把铁丝拼在一起了\end{exemple}
\begin{exemple}\jya tɯ-ŋga kɤ-sɤtɕɯmthɯt\cmn 把(零碎的布片)拼在一起,做成一件衣服\end{exemple}
\begin{exemple}\jya tɤ-ri kɤ-sɤtɕɯmthɯt\cmn 把线拼在一起\end{exemple}
\begin{relation-sémantique}\synonyme{
\hyperlink{Ⓔsɤlɤɣɯ}{\textit{ \papi{sɤlɤɣɯ}}}
}\end{relation-sémantique}
\begin{relation-sémantique}\synonyme{
\hyperlink{Ⓔsɤthɤri}{\textit{ \papi{sɤthɤri}}}
}\end{relation-sémantique}\end{entrée}

\begin{entrée}
\vedette{\hypertarget{Ⓔsɤtɕɯqaʁ}{\papi{ sɤtɕɯqaʁ}}}\markboth{sɤtɕɯqaʁ}{}
\classe{vt}
\paradigme{\textit{dir :} \jya nɯ-}
\begin{définition}\fra chicaner, s'opposer à\end{définition}
\begin{définition}\cmn 计较;反驳;找毛病;评理\end{définition}
\begin{exemple}\jya nɯ-sɤtɕɯqaʁ-a\cmn 我跟他计较了\end{exemple}
\begin{exemple}\jya mbro jla ɲɯ-sɤtɕɯqaʁ\cmn 他计较马和犏牛的事情\end{exemple}
\begin{exemple}\jya ɯ-rɟɯ ɲɯ-sɤtɕɯqaʁ\cmn 他计较他的财产\end{exemple}
\begin{exemple}\jya nɤki tɯrme ɯ-kɤ-sɤtɕɯqaʁ dɤn\cmn 那个人很爱计较\end{exemple}
\begin{exemple}\jya ki tɤ-kɤ-tɯt nɯ tɤ-ste, ma-nɯ-tɯ-sɤtɕɯqaʁ\cmn 你要按照他说的去做,不要反驳\end{exemple}
\begin{exemple}\jya nɤ-kɤ-sɤtɕɯqaʁ ʁɟa ʑo ɲɯ-ɕti\cmn 你所说的话全部都是跟人家计较的\end{exemple}
\begin{relation-sémantique}\synonyme{
\hyperlink{Ⓔsɤrtɕhɣaʁ}{\textit{ \papi{sɤrtɕhɣaʁ}}}
}\end{relation-sémantique}\begin{sous-entrée}
\vedette{\hypertarget{}{\papi{ ɣɤtɕɯqaʁ}}}\markboth{ɣɤtɕɯqaʁ}{}\classe{vi}
\paradigme{\textit{dir :} \jya pɯ-}
\begin{définition}\fra chicaner\end{définition}
\begin{définition}\cmn 计较;反驳\end{définition}
\begin{exemple}\jya pɯ-ɣɤtɕɯqaʁ-a\cmn 我计较了\end{exemple}
\begin{relation-sémantique}\confer{
\hyperlink{Ⓔɣɤʁrɯqa}{\textit{ \papi{ɣɤʁrɯqa}}}
}\end{relation-sémantique}
\end{sous-entrée}\end{entrée}

\begin{entrée}
\vedette{\hypertarget{Ⓔsɤtɕɯtɕit}{\papi{ sɤtɕɯtɕit}}}\markboth{sɤtɕɯtɕit}{}
\begin{relation-sémantique}\confer{
\hyperlink{Ⓔatɕɯtɕit}{\textit{ \papi{atɕɯtɕit}}}
}\end{relation-sémantique}\end{entrée}

\begin{entrée}
\vedette{\hypertarget{Ⓔsɤtɕɯtʂi}{\papi{ sɤtɕɯtʂi}}}\markboth{sɤtɕɯtʂi}{}\classe{vt}
\paradigme{\textit{dir :} \jya \_}
\begin{définition}\fra continuer, aller directement sans s'arrêter, en profiter pour faire quelque chose d'autre\end{définition}
\begin{définition}\cmn 继续;直接过去(可以停的地方没有停),顺便做另外一件事\end{définition}
\begin{exemple}\jya tha-sɤtɕɯtʂi, la-sɤtɕɯtʂi\cmn 他顺便带了\end{exemple}
\begin{exemple}\jya tɯpri kɤ-ti tɤ-sɤtɕɯtʂi-t-a\cmn 我顺便转告了口信\end{exemple}
\begin{exemple}\jya ma-tɤ-tɯ-znɯne kɯ pɯ-sɤtɕɯtʂi\cmn 你不要停下來,一定要做下去\end{exemple}
\begin{exemple}\jya tɤ-nɯsɤtɕɯtʂi-nɯ jɤɣ ma aʑɯɣ mɤ-ʁdɯɣ\cmn 你们继续(吃),不用管我\end{exemple}
\begin{exemple}\jya kɤ-nɤma tɤ-sɤtɕɯtʂi jɤɣ\cmn 你继续工作吧\end{exemple}\end{entrée}

\begin{entrée}
\vedette{\hypertarget{Ⓔsɤtɕɯxtʂot}{\papi{ sɤtɕɯxtʂot}}}\markboth{sɤtɕɯxtʂot}{}
\begin{relation-sémantique}\confer{
\hyperlink{Ⓔatɕɯxtʂot}{\textit{ \papi{atɕɯxtʂot}}}
}\end{relation-sémantique}\end{entrée}

\begin{entrée}
\vedette{\hypertarget{Ⓔsɤthɤβthɤβ}{\papi{ sɤthɤβthɤβ}}}\markboth{sɤthɤβthɤβ}{}
\classe{vt}
\begin{définition}\fra cligner des yeux\end{définition}
\begin{définition}\cmn 眨眼(快)\end{définition}
\begin{exemple}\jya ɯ-mɲaʁ ta-sɤthɤβthɤβ\cmn 他眨了眼\end{exemple}\begin{sous-entrée}
\vedette{\hypertarget{}{\papi{ ɣɤthɤβthɤβ}}}\markboth{ɣɤthɤβthɤβ}{}\classe{vi}
\begin{définition}\fra faire l'effronté derrière le dos\end{définition}
\begin{définition}\cmn 悄悄地顶嘴\end{définition}
\begin{exemple}\jya ma-tɯ-ɣɤthɤβthɤβ\cmn 你不要悄悄地顶嘴\end{exemple}
\begin{relation-sémantique}\synonyme{
\hyperlink{Ⓔsɤtɕhɯβtɕhɯβ}{\textit{ \papi{sɤtɕhɯβtɕhɯβ}}}
}\end{relation-sémantique}
\end{sous-entrée}\end{entrée}

\begin{entrée}
\vedette{\hypertarget{Ⓔsɤthɤri}{\papi{ sɤthɤri}}}\markboth{sɤthɤri}{}
\classe{vt}
\paradigme{\textit{dir :} \jya nɯ-}
\begin{définition}\fra connecter, rattacher\end{définition}
\begin{définition}\cmn 连接\end{définition}
\begin{exemple}\jya tɯmbri kɤ-sɤthɤri\cmn 把(几根)绳子连接起来\end{exemple}
\begin{exemple}\jya nɤki tɤ-ri nɯ nɯ-sɤthɤri-t-a\cmn 我把那几根线连接在一起\end{exemple}
\begin{exemple}\jya tɤ-pɤtso kɯ ji-xtsa ɲɯ-sɤthɤri\cmn 小孩子把我们的鞋子系在一起\end{exemple}
\begin{relation-sémantique}\confer{
\hyperlink{Ⓔathɤri}{\textit{ \papi{athɤri}}}
}\end{relation-sémantique}
\begin{relation-sémantique}\synonyme{
\hyperlink{Ⓔsɤtɕɯmthɯt}{\textit{ \papi{sɤtɕɯmthɯt}}}
}\end{relation-sémantique}
\begin{relation-sémantique}\synonyme{
\hyperlink{Ⓔsɤlɤɣɯ}{\textit{ \papi{sɤlɤɣɯ}}}
}\end{relation-sémantique}\end{entrée}

\begin{entrée}
\vedette{\hypertarget{Ⓔsɤthɣɤthɣɤt}{\papi{ sɤthɣɤthɣɤt}}}\markboth{sɤthɣɤthɣɤt}{}
\begin{relation-sémantique}\confer{
\hyperlink{Ⓔɣɤthɣɤthɣɤt}{\textit{ \papi{ɣɤthɣɤthɣɤt}}}
}\end{relation-sémantique}\end{entrée}

\begin{entrée}
\vedette{\hypertarget{Ⓔsɤthoʁmphrɤt}{\papi{ sɤthoʁmphrɤt}}}\markboth{sɤthoʁmphrɤt}{}
\classe{vt}
\paradigme{\textit{dir :} \jya tɤ-}
\begin{définition}\fra installer, mettre ensemble les pièces d'une machine, d'un habit de manière adéquate\end{définition}
\begin{définition}\cmn 把零件组装;令零件相吻合\end{définition}
\begin{exemple}\jya tʂɤm ta-sɤthoʁmphrɤt\cmn 他把板壁组装了\end{exemple}
\begin{exemple}\jya mkhɯrlu ta-sɤthoʁmphrɤt\cmn 他把机器组装了\end{exemple}
\begin{relation-sémantique}\confer{
\hyperlink{Ⓔathoʁmphrɤt}{\textit{ \papi{athoʁmphrɤt}}}
}\end{relation-sémantique}\end{entrée}

\begin{entrée}
\vedette{\hypertarget{Ⓔsɤtsu}{\papi{ sɤtsu}}}\markboth{sɤtsu}{}\classe{vs}
\begin{définition}\fra que l'on a le temps de faire\end{définition}
\begin{définition}\cmn 来得及做的\end{définition}
\begin{exemple}\jya fso tɕe ɕɯ-kɤ-nɤmɲo ɲɯ-sɤtso\cmn 明天有时间去看(节目)\end{exemple}
\begin{relation-sémantique}\confer{
\hyperlink{Ⓔtsu}{\textit{ \papi{tsu}}}
}\end{relation-sémantique}
\end{entrée}

\begin{entrée}
\vedette{\hypertarget{Ⓔsɤtsa}{\papi{ sɤtsa}}}\markboth{sɤtsa}{}\classe{vt}
\paradigme{\textit{dir :} \jya pɯ-}
\begin{définition}\ 
\begin{déclaration}\grammar{caus}\end{déclaration}\end{définition}
\begin{définition}\fra insérer, planter\end{définition}
\begin{définition}\cmn 插;戳;刺痛\end{définition}
\begin{exemple}\jya tɤtshoʁ pɯ-sɤtsa-t-a (=pɯ-no-t-a)\cmn 我钉了钉子\end{exemple}
\begin{exemple}\jya ɕɤmtshoʁ kɤ-sɤtsa-t-a (=kɤ-lat-a, kɤ-no-t-a)\cmn 我钉了铁钉\end{exemple}
\begin{exemple}\jya kɯm pɯ-sɤtsa-ta\cmn 我锁了门\end{exemple}
\begin{exemple}\jya tɤtshoʁ ko-sɤtsa\cmn 他钉了钉子\end{exemple}
\begin{relation-sémantique}\confer{
\hyperlink{Ⓔatsa}{\textit{ \papi{atsa}}}
}\end{relation-sémantique}\end{entrée}

\begin{entrée}
\vedette{\hypertarget{Ⓔsɤtso}{\papi{ sɤtso}}}\markboth{sɤtso}{}
\begin{relation-sémantique}\confer{
\hyperlink{Ⓔtso}{\textit{ \papi{tso}}}
}\end{relation-sémantique}\end{entrée}

\begin{entrée}
\vedette{\hypertarget{Ⓔsɤtʂu}{\papi{ sɤtʂu}}}\markboth{sɤtʂu}{}\classe{vt}
\paradigme{\textit{dir :} \jya tɤ-}
\begin{définition}\fra illuminer avec une lampe\end{définition}
\begin{définition}\cmn 用灯照亮\end{définition}
\begin{exemple}\jya ɣɟɯ kɯngɯ-rtsɤɣ tu ri kɤ-sɤtʂu khɯ, tɯrme kɯngɯ-tɣa ma me ri, kɤ-sɤtʂu mɤ-khɯ\cmn 碉楼虽然有九层高可以用灯照亮,人虽然只有九拃高,但是不能用灯照亮(人心难测)\end{exemple}
\begin{relation-sémantique}\confer{
\hyperlink{Ⓔtɤtʂu}{\textit{ \papi{tɤtʂu}}}
}\end{relation-sémantique}\end{entrée}

\begin{entrée}
\vedette{\hypertarget{Ⓔsɤtʂoʁloʁ}{\papi{ sɤtʂoʁloʁ}}}\markboth{sɤtʂoʁloʁ}{}
\begin{relation-sémantique}\confer{
\hyperlink{Ⓔatʂoʁloʁ}{\textit{ \papi{atʂoʁloʁ}}}
}\end{relation-sémantique}\end{entrée}

\begin{entrée}
\vedette{\hypertarget{Ⓔsɤtɯta}{\papi{ sɤtɯta}}}\markboth{sɤtɯta}{}
\begin{relation-sémantique}\confer{
\hyperlink{Ⓔatɯta}{\textit{ \papi{atɯta}}}
}\end{relation-sémantique}\end{entrée}

\begin{entrée}
\vedette{\hypertarget{Ⓔsɤwi}{\papi{ sɤwi}}}\markboth{sɤwi}{} (\variante{sɤwij}) \classe{vt}
\paradigme{\textit{dir :} \jya pɯ-}
\paradigme{\textit{dir :} \jya kɤ-}
\begin{définition}\fra fermer (yeux)\end{définition}
\begin{définition}\cmn 闭上眼睛\end{définition}
\begin{exemple}\jya ɯ-mɲaʁ ka-sɤwi\cmn 他闭上眼睛了\end{exemple}
\begin{exemple}\jya a-mɲaʁ kɤ-sɤwi-t-a ri a-ʑɯβ kɤ-sɯɣe mɯ́j-khɯ\cmn 我虽然闭上眼睛,还是睡不着\end{exemple}
\begin{relation-sémantique}\confer{
\hyperlink{Ⓔawij}{\textit{ \papi{awij}}}
}\end{relation-sémantique}\end{entrée}

\begin{entrée}
\vedette{\hypertarget{Ⓔsɤwij}{\papi{ sɤwij}}}\markboth{sɤwij}{}
\begin{relation-sémantique}\confer{
\hyperlink{Ⓔawij}{\textit{ \papi{awij}}}
}\end{relation-sémantique}
\end{entrée}

\begin{entrée}
\vedette{\hypertarget{Ⓔsɤwija}{\papi{ sɤwija}}}\markboth{sɤwija}{}
\classe{vt}
\paradigme{\textit{dir :} \jya kɤ-}
\begin{définition}\fra enrouler autour du fuseau\end{définition}
\begin{définition}\cmn 缠在纺锤上\end{définition}
\begin{exemple}\jya tɤ-ri ka-sɤwija\cmn 他把线缠在纺锤上了\end{exemple}\end{entrée}

\begin{entrée}
\vedette{\hypertarget{Ⓔsɤwum}{\papi{ sɤwum}}}\markboth{sɤwum}{}
\classe{vt}
\paradigme{\textit{dir :} \jya tɤ-}
\paradigme{\textit{dir :} \jya pɯ-}
\begin{définition}\fra ranger ensemble, fermer la bouche\end{définition}
\begin{définition}\cmn 收起來;闭嘴\end{définition}
\begin{exemple}\jya nɤmtɕhi pɯ-sɤwum\cmn 闭嘴!\end{exemple}\begin{sous-entrée}
\vedette{\hypertarget{}{\papi{ sɤwɯwum}}}\markboth{sɤwɯwum}{}
\paradigme{\textit{dir :} \jya tɤ-}
\begin{définition}\fra ranger ensemble\end{définition}
\begin{définition}\cmn 收拾在一起\end{définition}
\begin{exemple}\jya fsapaʁ ra tɤ-sɤwɯwum-a\cmn 我把牲畜聚集在一起了\end{exemple}
\begin{exemple}\jya tɤ-pɤtso ɯ-kɯmtɕhɯ ra pjɤ-ʁndɤr tɕe, pɯ-sɤwɯwum-a\cmn 因为孩子的玩具散了,我把这些收集在一起\end{exemple}
\end{sous-entrée}\end{entrée}

\begin{entrée}
\vedette{\hypertarget{Ⓔsɤwɯwum}{\papi{ sɤwɯwum}}}\markboth{sɤwɯwum}{}
\begin{relation-sémantique}\confer{
\hyperlink{Ⓔsɤwum}{\textit{ \papi{sɤwum}}}
}\end{relation-sémantique}\end{entrée}

\begin{entrée}
\vedette{\hypertarget{Ⓔsɤxoŋxoŋ}{\papi{ sɤxoŋxoŋ}}}\markboth{sɤxoŋxoŋ}{}
\begin{relation-sémantique}\confer{
\hyperlink{Ⓔxoŋnɤxoŋ}{\textit{ \papi{xoŋnɤxoŋ}}}
}\end{relation-sémantique}\end{entrée}

\begin{entrée}
\vedette{\hypertarget{Ⓔsɤxphɤn}{\papi{ sɤxphɤn}}}\markboth{sɤxphɤn}{}
\classe{n}
\begin{définition}\fra avantage\end{définition}
\begin{définition}\cmn 好处\end{définition}
\begin{exemple}\jya tɕhi pjɯ́-wɣ-nɤβzjɯβzjoz ʑo ɯ-sɤxphɤn tu ɕti wo\cmn 学什么都有用\end{exemple}\end{entrée}

\begin{entrée}
\vedette{\hypertarget{Ⓔsɤxtɕhɯxtɕhɯβ}{\papi{ sɤxtɕhɯxtɕhɯβ}}}\markboth{sɤxtɕhɯxtɕhɯβ}{}
\classe{vt}
\paradigme{\textit{dir :} \jya tɤ-}
\begin{définition}\fra émettre un bruit de froissement (sac en plastique)\end{définition}
\begin{définition}\cmn 发出沙沙声(塑料袋)\end{définition}
\begin{exemple}\jya a-rna ɯ-ŋgɯ βɣɤza ko-ɕe tɕe, ɲɯ-sɤxtɕhɯxtɕhɯβ ʑo\cmn 苍蝇钻到我耳朵里,发出沙沙声\end{exemple}\begin{sous-entrée}
\vedette{\hypertarget{}{\papi{ ɣɤxtɕhɯxtɕhɯβ}}}\markboth{ɣɤxtɕhɯxtɕhɯβ}{}\classe{vi}
\begin{définition}\fra y avoir un bruit de froissement\end{définition}
\begin{définition}\cmn 沙沙地响\end{définition}
\begin{exemple}\jya a-rna ɯ-ŋgɯ thɯci ko-ɕe, ɲɯ-ɣɤxtɕhɯxtɕhɯβ\cmn 有东西钻到我耳朵里,发出沙沙声\end{exemple}
\end{sous-entrée}\end{entrée}

\begin{entrée}
\vedette{\hypertarget{Ⓔsɤxtɕɯɣ}{\papi{ sɤxtɕɯɣ}}}\markboth{sɤxtɕɯɣ}{}\classe{n}
\begin{définition}\fra lanière pour porter les enfants sur le dos\end{définition}
\begin{définition}\cmn 背小孩子的背带\end{définition}\end{entrée}

\begin{entrée}
\vedette{\hypertarget{Ⓔsɤxtɕɯxtɕi}{\papi{ sɤxtɕɯxtɕi}}}\markboth{sɤxtɕɯxtɕi}{}\classe{adv}
\begin{définition}\fra depuis tout petit\end{définition}
\begin{définition}\cmn 从小\end{définition}
\begin{exemple}\jya sɤxtɕɯxtɕi kɯrɯ skɤt spe\cmn 他从小都会讲藏语\end{exemple}\end{entrée}

\begin{entrée}
\vedette{\hypertarget{Ⓔsɤxɯβxɯβ}{\papi{ sɤxɯβxɯβ}}}\markboth{sɤxɯβxɯβ}{}
\begin{relation-sémantique}\confer{
\hyperlink{Ⓔxɯβxɯβ}{\textit{ \papi{xɯβxɯβ}}}
}\end{relation-sémantique}\end{entrée}

\begin{entrée}
\vedette{\hypertarget{Ⓔsɤxɯxɯɣ}{\papi{ sɤxɯxɯɣ}}}\markboth{sɤxɯxɯɣ}{}\classe{vt}\acception{1}
\paradigme{\textit{dir :} \jya nɯ-}
\begin{définition}\fra agiter\end{définition}
\begin{définition}\cmn 挥动\end{définition}
\begin{exemple}\jya laʁjɯɣ nɯ a-ku ɯ-taʁ kɤ-sɤxɯxɯɣ-a\cmn 我把棍子在我头上挥动了\end{exemple}\acception{2}
\paradigme{\textit{dir :} \jya pɯ-}
\begin{définition}\fra souffler bruyamment\end{définition}
\begin{définition}\cmn 风吹,发出很紧的声音\end{définition}
\begin{exemple}\jya qale ɲɯ-sɤxɯxɯɣ\cmn 风在吹,发出声音\end{exemple}
\begin{relation-sémantique}\confer{
\hyperlink{Ⓔɣɤxɯxɯɣ}{\textit{ \papi{ɣɤxɯxɯɣ}}}
}\end{relation-sémantique}\end{entrée}

\begin{entrée}
\vedette{\hypertarget{Ⓔsɤχa}{\papi{ sɤχa}}}\markboth{sɤχa}{}
\begin{relation-sémantique}\confer{
\hyperlink{Ⓔaχa}{\textit{ \papi{aχa}}}
}\end{relation-sémantique}\end{entrée}

\begin{entrée}
\vedette{\hypertarget{Ⓔsɤχsɯχsjɯβ}{\papi{ sɤχsɯχsjɯβ}}}\markboth{sɤχsɯχsjɯβ}{}
\classe{vt}
\paradigme{\textit{dir :} \jya tɤ-}
\begin{définition}\fra renifler\end{définition}
\begin{définition}\cmn 用鼻吸气,发出嘶嘶声\end{définition}
\begin{exemple}\jya ɯ-ɕna to-sɤχsɯχsjɯβ (=χsjɯβnɤχsjɯβ to-stu)\cmn 他用鼻吸了气\end{exemple}
\begin{relation-sémantique}\confer{
\hyperlink{Ⓔχsjɯβnɤχsjɯβ}{\textit{ \papi{χsjɯβnɤχsjɯβ}}}
}\end{relation-sémantique}\end{entrée}

\begin{entrée}
\vedette{\hypertarget{Ⓔsɤz}{\papi{ sɤz}}}\markboth{sɤz}{}\classe{postp}
\begin{définition}\fra par rapport à\end{définition}
\begin{définition}\cmn 比\end{définition}
\begin{relation-sémantique}\synonyme{
\hyperlink{Ⓔstaʁ}{\textit{ \papi{staʁ}}}
}\end{relation-sémantique}
\begin{relation-sémantique}\synonyme{
\hyperlink{Ⓔsɤznɤ}{\textit{ \papi{sɤznɤ}}}
}\end{relation-sémantique}\end{entrée}

\begin{entrée}
\vedette{\hypertarget{Ⓔsɤzda}{\papi{ sɤzda}}}\markboth{sɤzda}{}\classe{vs}
\begin{définition}\fra aimable\end{définition}
\begin{définition}\cmn 很好相处\end{définition}
\begin{relation-sémantique}\synonyme{
\hyperlink{Ⓔsaχti}{\textit{ \papi{saχti}}}
}\end{relation-sémantique}
\begin{relation-sémantique}\confer{
\hyperlink{Ⓔnɤzda}{\textit{ \papi{nɤzda}}}
}\end{relation-sémantique}
\begin{relation-sémantique}\confer{
\hyperlink{Ⓔɣɤzda}{\textit{ \papi{ɣɤzda}}}
}\end{relation-sémantique}\end{entrée}

\begin{entrée}
\vedette{\hypertarget{Ⓔsɤzdaʁ}{\papi{ sɤzdaʁ}}}\markboth{sɤzdaʁ}{}
\begin{relation-sémantique}\confer{
\hyperlink{Ⓔazdaʁ}{\textit{ \papi{azdaʁ}}}
}\end{relation-sémantique}\end{entrée}

\begin{entrée}
\vedette{\hypertarget{Ⓔsɤzdɯm}{\papi{ sɤzdɯm}}}\markboth{sɤzdɯm}{}\classe{n}
\begin{définition}\fra nuage de pluie\end{définition}
\begin{définition}\cmn 乌云(下雨之前的)\end{définition}
\begin{relation-sémantique}\confer{
\hyperlink{Ⓔzdɯm}{\textit{ \papi{zdɯm}}}
}\end{relation-sémantique}\end{entrée}

\begin{entrée}
\vedette{\hypertarget{Ⓔsɤzdɯxpa}{\papi{ sɤzdɯxpa}}}\markboth{sɤzdɯxpa}{} (\variante{sɤdɯxpa}) \classe{vs}
\paradigme{\textit{dir :} \jya tɤ-}
\paradigme{\textit{dir :} \jya thɯ-}
\begin{définition}\fra pitoyable, pauvre\end{définition}
\begin{définition}\cmn 可怜\end{définition}
\begin{exemple}\jya pɯ-sɤzdɯxpa\cmn 他很可怜\end{exemple}
\begin{relation-sémantique}\confer{
\hyperlink{Ⓔnɯzdɯxpa}{\textit{ \papi{nɯzdɯxpa}}}
}\end{relation-sémantique}\end{entrée}

\begin{entrée}
\vedette{\hypertarget{Ⓔsɤzɣɤkhe}{\papi{ sɤzɣɤkhe}}}\markboth{sɤzɣɤkhe}{}
\begin{relation-sémantique}\confer{
\hyperlink{Ⓔkhe}{\textit{ \papi{khe}}}
}\end{relation-sémantique}\end{entrée}

\begin{entrée}
\vedette{\hypertarget{Ⓔsɤzɣɤmɯ}{\papi{ sɤzɣɤmɯ}}}\markboth{sɤzɣɤmɯ}{}
\begin{relation-sémantique}\confer{
\hyperlink{Ⓔɣɤmɯ}{\textit{ \papi{ɣɤmɯ}}}
}\end{relation-sémantique}\end{entrée}

\begin{entrée}
\vedette{\hypertarget{Ⓔsɤzɣɤxpra}{\papi{ sɤzɣɤxpra}}}\markboth{sɤzɣɤxpra}{}
\begin{relation-sémantique}\confer{
\hyperlink{Ⓔɣɤxpra}{\textit{ \papi{ɣɤxpra}}}
}\end{relation-sémantique}\end{entrée}

\begin{entrée}
\vedette{\hypertarget{Ⓔsɤzɣɯt}{\papi{ sɤzɣɯt}}}\markboth{sɤzɣɯt}{}\classe{vt}
\paradigme{\textit{dir :} \jya \_}
\begin{définition}\fra ramener\end{définition}
\begin{définition}\cmn 带到(目的地)\end{définition}
\begin{exemple}\jya ta-sɤzɣɯt-nɯ\cmn 他们带到了\end{exemple}
\begin{exemple}\jya nɤ-mu ɣɯ ɯ-tɕɣom ɯ-nɯ́-tɯ-sɤzɣɯt\cmn 你把花椒带给你母亲了吗?\end{exemple}\begin{sous-entrée}
\vedette{\hypertarget{}{\papi{ ʑɣɤsɤzɣɯt}}}\markboth{ʑɣɤsɤzɣɯt}{}\classe{vi}
\paradigme{\textit{dir :} \jya \_}
\begin{définition}\ 
\begin{déclaration}\grammar{refl}\end{déclaration}\end{définition}
\begin{définition}\fra se rendre à, auprès de\end{définition}
\begin{définition}\cmn 去到\end{définition}
\begin{relation-sémantique}\confer{
\hyperlink{Ⓔzɣɯt}{\textit{ \papi{zɣɯt}}}
}\end{relation-sémantique}
\begin{relation-sémantique}\confer{
\hyperlink{Ⓔɣɯt}{\textit{ \papi{ɣɯt}}}
}\end{relation-sémantique}
\end{sous-entrée}\end{entrée}

\begin{entrée}
\vedette{\hypertarget{Ⓔsɤzjaŋlaŋ}{\papi{ sɤzjaŋlaŋ}}}\markboth{sɤzjaŋlaŋ}{}
\begin{relation-sémantique}\confer{
\hyperlink{Ⓔɣɤzjaŋlaŋ}{\textit{ \papi{ɣɤzjaŋlaŋ}}}
}\end{relation-sémantique}
\end{entrée}

\begin{entrée}
\vedette{\hypertarget{Ⓔsɤzjaŋzjaŋ}{\papi{ sɤzjaŋzjaŋ}}}\markboth{sɤzjaŋzjaŋ}{}
\begin{relation-sémantique}\confer{
\hyperlink{Ⓔɣɤzjaŋlaŋ}{\textit{ \papi{ɣɤzjaŋlaŋ}}}
}\end{relation-sémantique}\end{entrée}

\begin{entrée}
\vedette{\hypertarget{Ⓔsɤzjɤɣlɤɣ}{\papi{ sɤzjɤɣlɤɣ}}}\markboth{sɤzjɤɣlɤɣ}{}
\begin{relation-sémantique}\confer{
\hyperlink{Ⓔzjɤɣzjɤɣ}{\textit{ \papi{zjɤɣzjɤɣ}}}
}\end{relation-sémantique}\end{entrée}

\begin{entrée}
\vedette{\hypertarget{Ⓔsɤzmbrɯ}{\papi{ sɤzmbrɯ}}}\markboth{sɤzmbrɯ}{}
\classe{vt}
\paradigme{\textit{dir :} \jya tɤ-}
\begin{définition}\ 
\begin{déclaration}\grammar{caus}\end{déclaration}\end{définition}
\begin{définition}\fra énerver\end{définition}
\begin{définition}\cmn 惹人生气\end{définition}
\begin{exemple}\jya tɤ-sɤzmbrɯ-t-a\cmn 我惹他生气了\end{exemple}
\begin{exemple}\jya tɤ-ta-sɤzmbrɯ\cmn 我惹你生气了\end{exemple}
\begin{exemple}\jya tɤ́-wɣ-sɤzmbrɯ-a\cmn 他惹我生气了\end{exemple}
\begin{exemple}\jya ma-tɤ-tɯ-sɤzmbri\cmn 你不要惹他生气\end{exemple}
\begin{exemple}\jya nɤʑo qhlɯ to-tɯ-sɤzmbrɯ-t\cmn 你惹了水神\end{exemple}
\begin{relation-sémantique}\confer{
\hyperlink{Ⓔsɤmbrɯ}{\textit{ \papi{sɤmbrɯ}}}
}\end{relation-sémantique}\end{entrée}

\begin{entrée}
\vedette{\hypertarget{Ⓔsɤznɤ}{\papi{ sɤznɤ}}}\markboth{sɤznɤ}{}
\classe{adv}
\begin{définition}\fra par rapport à\end{définition}
\begin{définition}\cmn 比\end{définition}
\begin{exemple}\jya nɯ ɕɯŋgɯ sɤznɤ tɕi-χpi nɯra ɲɯ-tɯ-tso ɣe?\cmn 你懂我们的故事懂得比以前多,对吧?\end{exemple}
\begin{exemple}\jya nɯ sɤznɤ\cmn 相反的、反而\end{exemple}
\begin{relation-sémantique}\confer{
\hyperlink{Ⓔsɤz}{\textit{ \papi{sɤz}}}
}\end{relation-sémantique}\end{entrée}

\begin{entrée}
\vedette{\hypertarget{Ⓔsɤzoŋzoŋ}{\papi{ sɤzoŋzoŋ}}}\markboth{sɤzoŋzoŋ}{}
\begin{relation-sémantique}\confer{
\hyperlink{Ⓔɣɤzoŋzoŋ}{\textit{ \papi{ɣɤzoŋzoŋ}}}
}\end{relation-sémantique}\end{entrée}

\begin{entrée}
\vedette{\hypertarget{Ⓔsɤzraʁ}{\papi{ sɤzraʁ}}}\markboth{sɤzraʁ}{}
\begin{relation-sémantique}\confer{
\hyperlink{Ⓔnɤzraʁ}{\textit{ \papi{nɤzraʁ}}}
}\end{relation-sémantique}\end{entrée}

\begin{entrée}
\vedette{\hypertarget{Ⓔsɤʑa}{\papi{ sɤʑa}}}\markboth{sɤʑa}{}
\classe{vt}
\paradigme{\textit{dir :} \jya \_}
\begin{définition}\fra commencer\end{définition}
\begin{définition}\cmn 开始
\begin{déclaration}\use{趋向前缀和补语动词一致,例如\stylefv{fɕɤt}“讲述”的固有趋向前缀是\stylefv{pɯ-},所以当\stylefv{fɕɤt}作为\stylefv{sɤʑa}的补语时,\stylefv{sɤʑa}就必须带\stylefv{pɯ-}这个趋向前缀}\end{déclaration}\end{définition}
\begin{exemple}\jya tɕhomba ta-sɤʑa\cmn 他开始感冒\end{exemple}
\begin{exemple}\jya kɤ-rɤma ta-sɤʑa\cmn 他开始工作了\end{exemple}
\begin{exemple}\jya kɤ-nɤma mɯ-tɤ-tɯ-sɤʑa-t\cmn 你没有开始工作\end{exemple}
\begin{exemple}\jya ʑa tɤ-tɯ-sɤʑa-t\cmn 你早就开始了\end{exemple}
\begin{exemple}\jya kɤ-fɕɤt pɯ-sɤʑe\cmn 请开始讲\end{exemple}
\begin{exemple}\jya rɤɣo thɯ-sɤʑe\cmn 请开始唱歌\end{exemple}
\begin{exemple}\jya kutɕu tɕe ʁdɯrɟɤt sɤtɕha lu-sɤʑe ʑo ŋu\cmn 从这里开始就是龙尔甲乡\end{exemple}
\begin{relation-sémantique}\confer{
\hyperlink{ⒺʑaⒽ1}{\textit{ \papi{ʑa1}}}
}\end{relation-sémantique}
\begin{relation-sémantique}\confer{
\hyperlink{Ⓔnɤmphruʑa}{\textit{ \papi{nɤmphruʑa}}}
}\end{relation-sémantique}\end{entrée}

\begin{entrée}
\vedette{\hypertarget{Ⓔsɤʑaŋ}{\papi{ sɤʑaŋ}}}\markboth{sɤʑaŋ}{}
\classe{n}
\begin{définition}\fra champs\end{définition}
\begin{définition}\cmn 田地
\begin{déclaration} \étymologie{\papi{sa.ʑiŋ}}\end{déclaration}\end{définition}\end{entrée}

\begin{entrée}
\vedette{\hypertarget{Ⓔsɤʑdraŋlaŋ}{\papi{ sɤʑdraŋlaŋ}}}\markboth{sɤʑdraŋlaŋ}{}
\begin{relation-sémantique}\confer{
\hyperlink{Ⓔʑdraŋʑdraŋ}{\textit{ \papi{ʑdraŋʑdraŋ}}}
}\end{relation-sémantique}\end{entrée}

\begin{entrée}
\vedette{\hypertarget{Ⓔsɤʑɣɤlɤt}{\papi{ sɤʑɣɤlɤt}}}\markboth{sɤʑɣɤlɤt}{}
\begin{relation-sémantique}\confer{
\hyperlink{Ⓔsɤʑɣɤʑɣɤt}{\textit{ \papi{sɤʑɣɤʑɣɤt}}}
}\end{relation-sémantique}\end{entrée}

\begin{entrée}
\vedette{\hypertarget{Ⓔsɤʑɣɤʑɣɤt}{\papi{ sɤʑɣɤʑɣɤt}}}\markboth{sɤʑɣɤʑɣɤt}{} (\variante{znɯʑɣɤʑɣɤt}) \classe{vt}
\begin{définition}\fra brandir\end{définition}
\begin{définition}\cmn 举起,挥舞\end{définition}
\begin{exemple}\jya mbrɯtɕɯ ɲɯ-sɤʑɣɤʑɣɤt ʑo tha-tsɯm\cmn 他挥着刀跑下去了\end{exemple}\begin{sous-entrée}
\vedette{\hypertarget{}{\papi{ sɤʑɣɤlɤt}}}\markboth{sɤʑɣɤlɤt}{}\classe{vt}
\begin{définition}\fra brandir et agiter dans tous les sens\end{définition}
\begin{définition}\cmn 乱挥乱舞\end{définition}
\end{sous-entrée}\end{entrée}

\begin{entrée}
\vedette{\hypertarget{Ⓔsɤʑɯloʁ}{\papi{ sɤʑɯloʁ}}}\markboth{sɤʑɯloʁ}{}
\classe{vs}
\paradigme{\textit{dir :} \jya nɯ-}
\begin{définition}\ 
\begin{déclaration}\grammar{incorp}\end{déclaration}\end{définition}
\begin{définition}\fra dégouter\end{définition}
\begin{définition}\cmn 使人感到恶心\end{définition}
\begin{exemple}\jya jiɕqha nɯ mɯ́j-saχɕɯn, ɲɯ-sɤʑɯloʁ\cmn 不卫生,令人感到恶心\end{exemple}\begin{sous-entrée}
\vedette{\hypertarget{}{\papi{ nɤsɤʑɯloʁ}}}\markboth{nɤsɤʑɯloʁ}{}\classe{vt}
\begin{définition}\fra trouver dégoutant\end{définition}
\begin{définition}\cmn 觉得恶心\end{définition}
\begin{relation-sémantique}\confer{
\hyperlink{Ⓔnɤʑɯloʁ}{\textit{ \papi{nɤʑɯloʁ}}}
}\end{relation-sémantique}
\begin{relation-sémantique}\confer{
\hyperlink{Ⓔtɯ-ʑi,loʁ}{\textit{ \papi{tɯ-ʑi,loʁ}}}
}\end{relation-sémantique}
\end{sous-entrée}\end{entrée}

\begin{entrée}
\vedette{\hypertarget{Ⓔsɤʑɯrja}{\papi{ sɤʑɯrja}}}\markboth{sɤʑɯrja}{}
\begin{relation-sémantique}\confer{
\hyperlink{Ⓔaʑɯrja}{\textit{ \papi{aʑɯrja}}}
}\end{relation-sémantique}\end{entrée}

\begin{entrée}
\vedette{\hypertarget{Ⓔsɤʑɯχtso}{\papi{ sɤʑɯχtso}}}\markboth{sɤʑɯχtso}{}
\begin{relation-sémantique}\confer{
\hyperlink{Ⓔaʑɯχtso}{\textit{ \papi{aʑɯχtso}}}
}\end{relation-sémantique}\end{entrée}

\begin{entrée}
\vedette{\hypertarget{Ⓔscapa}{\papi{ scapa}}}\markboth{scapa}{}
\classe{n}
\begin{définition}\fra dague\end{définition}
\begin{définition}\cmn 刺刀\end{définition}\end{entrée}

\begin{entrée}
\vedette{\hypertarget{Ⓔscapafkɯm}{\papi{ scapafkɯm}}}\markboth{scapafkɯm}{}
\classe{n}
\begin{définition}\fra fourreau\end{définition}
\begin{définition}\cmn 刀鞘\end{définition}\end{entrée}

\begin{entrée}
\vedette{\hypertarget{Ⓔscaʁa}{\papi{ scaʁa}}}\markboth{scaʁa}{}
\classe{n}
\begin{définition}\fra pie, animal domestique dont le corps est noir et le milieu du corps blanc\end{définition}
\begin{définition}\cmn 喜鹊,全身黑色,腰白色的牲畜
\begin{déclaration} \étymologie{\papi{skʲa.ka}}\end{déclaration}\end{définition}\end{entrée}

\begin{entrée}
\vedette{\hypertarget{Ⓔscɤmdʑɯɣ}{\papi{ scɤmdʑɯɣ}}}\markboth{scɤmdʑɯɣ}{}
\classe{n}
\begin{définition}\fra protection\end{définition}
\begin{définition}\cmn 保佑
\begin{déclaration} \étymologie{\papi{skʲabs.ⁿdʑug}}\end{déclaration}\end{définition}\end{entrée}

\begin{entrée}
\vedette{\hypertarget{Ⓔscɤt}{\papi{ scɤt}}}\markboth{scɤt}{}
\classe{vt}
\paradigme{\textit{dir :} \jya \_}
\begin{définition}\fra déplacer\end{définition}
\begin{définition}\cmn 摆动,搬\end{définition}
\begin{exemple}\jya pa-scɤt\cmn 他搬了\end{exemple}
\begin{exemple}\jya rɯ ɲɯ-scat-a\cmn 我搬帐篷\end{exemple}
\begin{exemple}\jya ɯ-sɤz-rɤʑi ɲo-scɤt\cmn 他搬家了\end{exemple}
\begin{exemple}\jya ɯ-khɯrthaŋ pjɤ-scɤt (=pjɤ-phɤβ)\cmn 他被降职了\end{exemple}
\begin{exemple}\jya ɯ-khɯrthaŋ to-scɤt\cmn 他升职了\end{exemple}\begin{sous-entrée}
\vedette{\hypertarget{}{\papi{ nɤscɯscɤt}}}\markboth{nɤscɯscɤt}{}\classe{vt}
\paradigme{\textit{dir :} \jya \_}
\begin{définition}\ 
\begin{déclaration}\grammar{n.orient}\end{déclaration}\end{définition}
\begin{définition}\fra déplacer partout\end{définition}
\begin{définition}\cmn 搬来搬去\end{définition}
\begin{exemple}\jya pa-nɤscɯscɤt ntsɯ\cmn 他不停地搬来搬去了\end{exemple}
\end{sous-entrée}\begin{sous-entrée}
\vedette{\hypertarget{}{\papi{ rɤscɤt}}}\markboth{rɤscɤt}{}\classe{vi}
\paradigme{\textit{dir :} \jya \_}
\begin{définition}\ 
\begin{déclaration}\grammar{apass}\end{déclaration}\end{définition}
\begin{définition}\fra déménager, déplacer\end{définition}
\begin{définition}\cmn 搬走\end{définition}
\end{sous-entrée}\end{entrée}

\begin{entrée}
\vedette{\hypertarget{Ⓔschɤt}{\papi{ schɤt}}}\markboth{schɤt}{}
\classe{vi}
\paradigme{\textit{dir :} \jya pɯ-}
\begin{définition}\fra se retirer (eau)\end{définition}
\begin{définition}\cmn 下降(水)\end{définition}
\begin{exemple}\jya tɯ-ci pjɤ-schɤt\cmn 水下降了\end{exemple}\begin{sous-entrée}
\vedette{\hypertarget{}{\papi{ sɯschɤt}}}\markboth{sɯschɤt}{}\classe{vt}
\paradigme{\textit{dir :} \jya pɯ-}
\begin{définition}\ 
\begin{déclaration}\grammar{caus}\end{déclaration}\end{définition}
\begin{définition}\fra faire en sorte que l'eau se retire\end{définition}
\begin{définition}\cmn 使(水)下降\end{définition}
\end{sous-entrée}\end{entrée}

\begin{entrée}
\vedette{\hypertarget{Ⓔschi}{\papi{ schi}}}\markboth{schi}{}\classe{vs}
\paradigme{\textit{dir :} \jya tɤ-}
\begin{définition}\fra supporter\end{définition}
\begin{définition}\cmn 经得住\end{définition}
\begin{exemple}\jya aʑo tɤndʐo schi-a ma qaʑo ɕa tɤ-ndza-t-a\cmn 我经得住冷,因为吃了羊肉\end{exemple}
\begin{exemple}\jya ɯʑo tɤ-ŋɤm ɲɯ-schi\cmn 他经得住痛\end{exemple}
\begin{exemple}\jya tɤɕpaʁ schi-a\cmn 我经得住渴\end{exemple}
\begin{exemple}\jya tshɤdɯɣ mɤ-schi-a\cmn 我经不起热的天气\end{exemple}
\begin{exemple}\jya tɤndʐo mɯ́j-tɯ-schi\cmn 你很怕冷\end{exemple}\end{entrée}

\begin{entrée}
\vedette{\hypertarget{Ⓔschɯɣschɯɣ}{\papi{ schɯɣschɯɣ}}}\markboth{schɯɣschɯɣ}{}\classe{idph.2}
\begin{définition}\fra très froid\end{définition}
\begin{définition}\cmn 形容极冷\end{définition}
\begin{exemple}\jya ɯ-jaʁ ɲɯ-mɯɕtaʁ schɯɣschɯɣ ʑo\cmn 他的手冷冰冰的\end{exemple}\end{entrée}

\begin{entrée}
\vedette{\hypertarget{ⒺsciⒽ2}{\papi{ sci}}}\markboth{sci}{}\homonyme{2}
\classe{n}
\begin{définition}\ 
\begin{déclaration}\grammar{n.lieu}\end{déclaration}\end{définition}
\begin{définition}\fra l'un des hameaux de Gyutshapa\end{définition}
\begin{définition}\cmn 二茶村的大队之一\end{définition}\end{entrée}

\begin{entrée}
\vedette{\hypertarget{ⒺsciⒽ1}{\papi{ sci}}}\markboth{sci}{}\homonyme{1}\classe{vi}
\paradigme{\textit{dir :} \jya tɤ-}
\paradigme{\textit{dir :} \jya thɯ-}\acception{1}
\begin{définition}\fra naître\end{définition}
\begin{définition}\cmn 出生\end{définition}
\begin{exemple}\jya ɯ-rɟit to-sci\cmn 她的孩子出生了\end{exemple}\acception{2}
\begin{définition}\fra grandir\end{définition}
\begin{définition}\cmn 生长
\begin{déclaration} \étymologie{\papi{skʲe}}\end{déclaration}\end{définition}
\begin{exemple}\jya tɤ-pɤtso ra kɯ-ŋu kɤ-sɯxɕɤt ɯ-khɯkha a-thɯ-sci-nɯ (a-thɯ-ndzɤt-nɯ) tɕe pe\cmn 在孩子生长的过程中要进行正面教育就会好\end{exemple}
\begin{relation-sémantique}\synonyme{
\hyperlink{Ⓔndzɤt}{\textit{ \papi{ndzɤt}}}
}\end{relation-sémantique}\begin{sous-entrée}
\vedette{\hypertarget{}{\papi{ sɯsci}}}\markboth{sɯsci}{}\classe{vt}
\paradigme{\textit{dir :} \jya tɤ-}
\begin{définition}\ 
\begin{déclaration}\grammar{caus}\end{déclaration}\end{définition}
\begin{définition}\fra donner naissance\end{définition}
\begin{définition}\cmn 生(小孩子)\end{définition}
\begin{relation-sémantique}\confer{
\hyperlink{Ⓔtɯ-sɤsci}{\textit{ \papi{tɯ-sɤsci}}}
}\end{relation-sémantique}
\end{sous-entrée}\end{entrée}

\begin{entrée}
\vedette{\hypertarget{Ⓔscinde}{\papi{ scinde}}}\markboth{scinde}{}\classe{n}
\begin{définition}\ 
\begin{déclaration}\grammar{n.lieu}\end{déclaration}\end{définition}
\begin{définition}\fra l'un des hameaux de Gyutshapa\end{définition}
\begin{définition}\cmn 二茶村的大队之一\end{définition}\end{entrée}

\begin{entrée}
\vedette{\hypertarget{Ⓔscit}{\papi{ scit}}}\markboth{scit}{}\classe{vs}
\paradigme{\textit{dir :} \jya tɤ-}
\paradigme{\textit{dir :} \jya thɯ-}
\begin{définition}\fra heureux\end{définition}
\begin{définition}\cmn 幸福
\begin{déclaration} \étymologie{\papi{skʲid}}\end{déclaration}\end{définition}\begin{sous-entrée}
\vedette{\hypertarget{}{\papi{ sɯscit}}}\markboth{sɯscit}{}\classe{vt}
\begin{définition}\ 
\begin{déclaration}\grammar{caus}\end{déclaration}\end{définition}
\begin{relation-sémantique}\confer{
\hyperlink{Ⓔsɤscit}{\textit{ \papi{sɤscit}}}
}\end{relation-sémantique}
\end{sous-entrée}\end{entrée}

\begin{entrée}
\vedette{\hypertarget{Ⓔsciwa}{\papi{ sciwa}}}\markboth{sciwa}{}\classe{n}
\begin{définition}\fra vie, existence\end{définition}
\begin{définition}\cmn 生命
\begin{déclaration} \étymologie{\papi{skʲe.ba}}\end{déclaration}\end{définition}
\begin{exemple}\jya sciwa ɲo-nɤsci\cmn 他换了一条生命,死了以后活过来了\end{exemple}
\end{entrée}

\begin{entrée}
\vedette{\hypertarget{Ⓔsclaŋsclaŋ}{\papi{ sclaŋsclaŋ}}}\markboth{sclaŋsclaŋ}{}
\classe{idph.2}
\begin{définition}\fra chauve\end{définition}
\begin{définition}\cmn 形容光溜溜的样子\end{définition}
\begin{exemple}\jya ɯ-ku sclaŋsclaŋ ɲɯ-pa\cmn 他的头是光溜溜的\end{exemple}
\begin{exemple}\jya ɯ-ku sclaŋsclaŋ chɤ-nɯ-sɯ-qrɤz\cmn 他请人把头剃得光溜溜\end{exemple}\end{entrée}

\begin{entrée}
\vedette{\hypertarget{Ⓔscun}{\papi{ scun}}}\markboth{scun}{}\classe{n}
\begin{définition}\fra faute\end{définition}
\begin{définition}\cmn 过错
\begin{déclaration} \étymologie{\papi{skʲon}}\end{déclaration}\end{définition}\end{entrée}

\begin{entrée}
\vedette{\hypertarget{Ⓔsco}{\papi{ sco}}}\markboth{sco}{}
\classe{vt}
\paradigme{\textit{dir :} \jya \_}
\begin{définition}\fra raccompagner\end{définition}
\begin{définition}\cmn 送行\end{définition}
\begin{exemple}\jya ɕ-kɤ-sco-t-a\cmn 我去送了他\end{exemple}
\begin{exemple}\jya nɯ́-wɣ-sco-a\cmn 他送了我\end{exemple}\begin{sous-entrée}
\vedette{\hypertarget{}{\papi{ ascɯsco}}}\markboth{ascɯsco}{}\classe{vi}
\begin{définition}\ 
\begin{déclaration}\grammar{refl}\end{déclaration}\end{définition}
\begin{définition}\fra se raccompagner les uns les autres\end{définition}
\begin{définition}\cmn 互相送行\end{définition}
\end{sous-entrée}\begin{sous-entrée}
\vedette{\hypertarget{}{\papi{ sɤsco}}}\markboth{sɤsco}{}\classe{vi}
\begin{définition}\ 
\begin{déclaration}\grammar{apass}\end{déclaration}\end{définition}
\begin{relation-sémantique}\confer{
\hyperlink{Ⓔnɤscɤlɤt}{\textit{ \papi{nɤscɤlɤt}}}
}\end{relation-sémantique}
\end{sous-entrée}\end{entrée}

\begin{entrée}
\vedette{\hypertarget{Ⓔscoʁ}{\papi{ scoʁ}}}\markboth{scoʁ}{}
\classe{n}
\begin{définition}\fra louche\end{définition}
\begin{définition}\cmn 勺子;瓢
\begin{déclaration} \étymologie{\papi{skʲogs}}\end{déclaration}\end{définition}\end{entrée}

\begin{entrée}
\vedette{\hypertarget{Ⓔscoʁɲaʁ}{\papi{ scoʁɲaʁ}}}\markboth{scoʁɲaʁ}{}\classe{n}
\begin{définition}\fra louche en fer\end{définition}
\begin{définition}\cmn 铁铸成的勺子\end{définition}
\end{entrée}

\begin{entrée}
\vedette{\hypertarget{Ⓔscoʁrlɯ}{\papi{ scoʁrlɯ}}}\markboth{scoʁrlɯ}{}\classe{n}
\begin{définition}\fra louche en cuivre\end{définition}
\begin{définition}\cmn 红铜铸成的勺子\end{définition}
\end{entrée}

\begin{entrée}
\vedette{\hypertarget{Ⓔscoʁtshaʁ}{\papi{ scoʁtshaʁ}}}\markboth{scoʁtshaʁ}{}
\classe{n}
\begin{définition}\fra filtre à thé\end{définition}
\begin{définition}\cmn 茶漏勺\end{définition}
\begin{exemple}\jya scoʁtshaʁ nɯ tʂha tɤ́-wɣ-rku tɕe ɯ-sɤ-sɯxtshaʁ ŋu, tʂha ɯ-ŋgɯ tʂhɤzwa tu tɕe ɯ-tshaʁ pɕoʁ nɯ tɕu pjɯ́-wɣ-lɤt tɕe, tʂha ɯ-ci nɯ pjɯ-nɯɬoʁ ŋu ma ɯ-zwa nɯ scoʁ ɯ-ŋgɯ ku-nɯ-rɤʑi ŋu tɕe núndʐa scoʁtshaʁ rmi. ɯ-jɯ nɯ lú-wɣ-ltɤβ kɯ-khɯ ŋu, tɕe ndʐɯnbu kɤ-nɤndɯndo pe.\cmn 茶漏勺是倒茶的时候用来滤茶的器具。因为茶里有茶叶等渣滓,朝着漏勺的方向倒过去,这样茶水会流出来,渣滓就会留在勺子里,所以这种勺子叫漏勺(滤器)。勺子把可以折起来,所以出远门时带着方便。\end{exemple}\end{entrée}

\begin{entrée}
\vedette{\hypertarget{Ⓔscraʁscraʁ}{\papi{ scraʁscraʁ}}}\markboth{scraʁscraʁ}{}
\classe{idph.2}
\begin{définition}\fra tout petit\end{définition}
\begin{définition}\cmn 矮小,挨着地面\end{définition}\begin{sous-entrée}
\vedette{\hypertarget{}{\papi{ ɣɤscraʁlaʁ}}}\markboth{ɣɤscraʁlaʁ}{}\classe{vi}
\begin{exemple}\jya ɯ-ku sɤrku kɯ-me ɲɯ-ɣɤscraʁlaʁ\cmn 没有他的事,还是在多嘴\end{exemple}
\end{sous-entrée}\begin{sous-entrée}
\vedette{\hypertarget{}{\papi{ ɣɤscraʁscraʁ}}}\markboth{ɣɤscraʁscraʁ}{}\classe{vi}
\paradigme{\textit{dir :} \jya nɯ-}
\begin{définition}\fra parler sans arrêt sans se préoccuper des conventions sociales\end{définition}
\begin{définition}\cmn 不知安分守己,多嘴\end{définition}
\begin{exemple}\jya jiɕqha nɯ ɲɯ-ɣɤscraʁscraʁ ntsɯ\cmn 那个人总是多嘴\end{exemple}
\end{sous-entrée}\begin{sous-entrée}
\vedette{\hypertarget{}{\papi{ scraʁnɤscraʁ}}}\markboth{scraʁnɤscraʁ}{}\classe{idph.3}
\begin{exemple}\jya qaɕpa nɯ scraʁnɤscraʁ lu-ɕe pjɤ-ŋu\cmn 青蛙一跳一跳地往上游去\end{exemple}
\end{sous-entrée}\end{entrée}

\begin{entrée}
\vedette{\hypertarget{Ⓔscrɯscri}{\papi{ scrɯscri}}}\markboth{scrɯscri}{}\classe{idph.2}
\begin{définition}\fra liquide, dilué (boue)\end{définition}
\begin{définition}\cmn 形容(泥巴)稀,流体状\end{définition}
\begin{relation-sémantique}\confer{
\hyperlink{Ⓔɲcrɯɣɲcrɯɣ}{\textit{ \papi{ɲcrɯɣɲcrɯɣ}}}
}\end{relation-sémantique}
\begin{relation-sémantique}\confer{
\hyperlink{Ⓔχcrɯχcri}{\textit{ \papi{χcrɯχcri}}}
}\end{relation-sémantique}\end{entrée}

\begin{entrée}
\vedette{\hypertarget{Ⓔscɯʁzɯɣ}{\papi{ scɯʁzɯɣ}}}\markboth{scɯʁzɯɣ}{}
\classe{n}
\begin{définition}\fra apparence\end{définition}
\begin{définition}\cmn 外貌
\begin{déclaration} \étymologie{\papi{skʲe.gzugs}}\end{déclaration}\end{définition}
\begin{relation-sémantique}\confer{
\hyperlink{Ⓔscɯʁzɯɣ}{\textit{ \papi{scɯʁzɯɣ}}}
}\end{relation-sémantique}\end{entrée}

\begin{entrée}
\vedette{\hypertarget{Ⓔscuz}{\papi{ scuz}}}\markboth{scuz}{}
\classe{n}
\begin{définition}\fra faisan (Ithaginis cruentus)\end{définition}
\begin{définition}\cmn 血雉【松鸡子】\end{définition}
\begin{exemple}\jya scuz nɯ pɣa ci ŋu, kɯmpɣa jamar wxti. ɯ-kɤχcɤl kɯ-xtɕɯ-xtɕi ɲaʁ, ɯ-phoŋbu kɯ-ɤrŋi ŋgɯz kɯ-pɣi tsa ŋu. ɯ-mi kɯ-qarŋe ŋu, ɯ-mtsioʁ kɯ-qarŋe ŋu, ɯ-jme ɯ-ku kɯ-rɟum tsa ŋu, ɯ-mɲaʁ ɯ-rkɯ qarma ɣɯ kɯ-fse kɯ-ɣɯrni a-fskɤr, sɯŋgɯ ʁɟa ʑo ku-rɤʑi ma tɯ-ji ɯ-ŋgɯ ɣi mɤ-ŋgrɤl, sɯmat cho qajɯ ʁɟa tu-ndze ŋu. tɤ-mbri tɕe, ``tɕur tɕur tɕur tɕur tɕur" tu-ti ŋu tɕe núndʐa ɯ-ɕa tɕur tu-kɯ-ti ɲɯ-ŋu.\cmn 
松鸡子是一种鸟,有鸡那么大。头顶上有黑点,身子蓝里带灰,脚是黄色的,嘴是黄色的,尾巴的顶端有点宽,眼睛周围有一圈红色的皮像马鸡的一样。它一直生活在森林里,从不下地,专门吃野果和虫子。叫起来时,它发出\stylefv{tɕur tɕur tɕur tɕur tɕur}的声音,所以人家说它的肉很酸。
\end{exemple}\end{entrée}

\begin{entrée}
\vedette{\hypertarget{Ⓔsɣa}{\papi{ sɣa}}}\markboth{sɣa}{}
\classe{n}
\begin{définition}\fra rouille\end{définition}
\begin{définition}\cmn 锈\end{définition}
\begin{relation-sémantique}\confer{
\hyperlink{Ⓔnɯsɣa}{\textit{ \papi{nɯsɣa}}}
}\end{relation-sémantique}\end{entrée}

\begin{entrée}
\vedette{\hypertarget{ⒺsiⒽ1}{\papi{ si}}}\markboth{si}{}\homonyme{1}
\classe{n}
\begin{définition}\fra arbre\end{définition}
\begin{définition}\cmn 树\end{définition}
\end{entrée}

\begin{entrée}
\vedette{\hypertarget{ⒺsiⒽ2}{\papi{ si}}}\markboth{si}{}\homonyme{2}
\classe{vi}
\paradigme{\textit{dir :} \jya pɯ-}
\paradigme{\textit{dir :} \jya nɯ-}
\begin{définition}\fra mourir\end{définition}
\begin{définition}\cmn 死(人)\end{définition}\end{entrée}

\begin{entrée}
\vedette{\hypertarget{Ⓔsijmɤɣ}{\papi{ sijmɤɣ}}}\markboth{sijmɤɣ}{}
\classe{n}
\begin{définition}\fra russule\end{définition}
\begin{définition}\cmn 【辣辣菌】\end{définition}
\begin{exemple}\jya sijmɤɣ nɯ sɤjku ɯ-ŋgɯ tu-ɬoʁ ŋu, ɯ-tshɯɣa cho ɯ-mdoʁ nɯ jmɤɣni fse ri tú-wɣ-ndza tɕe mɤrtsaβ.\cmn 辣辣菌生长在白桦树林里,形状和颜色和杉木菌一样,但吃起来很辣。\end{exemple}\end{entrée}

\begin{entrée}
\vedette{\hypertarget{Ⓔsindzɯ}{\papi{ sindzɯ}}}\markboth{sindzɯ}{}\classe{n}
\begin{définition}\fra testament\end{définition}
\begin{définition}\cmn 遗嘱\end{définition}
\begin{exemple}\jya sindzɯ ci ra chɤ-rɤt.\cmn 他写了遗嘱\end{exemple}
\begin{relation-sémantique}\confer{
\hyperlink{Ⓔndzɯ}{\textit{ \papi{ndzɯ}}}
}\end{relation-sémantique}
\begin{relation-sémantique}\confer{
\hyperlink{ⒺsiⒽ1}{\textit{ \papi{si1}}}
}\end{relation-sémantique}
\end{entrée}

\begin{entrée}
\vedette{\hypertarget{Ⓔsjaŋnɤsjaŋ}{\papi{ sjaŋnɤsjaŋ}}}\markboth{sjaŋnɤsjaŋ}{}\classe{idph.3}
\begin{définition}\fra dodelinant de la tête\end{définition}
\begin{définition}\cmn 形容昂着头,头晃来晃去的样子\end{définition}\end{entrée}

\begin{entrée}
\vedette{\hypertarget{Ⓔsjoŋsjoŋ}{\papi{ sjoŋsjoŋ}}}\markboth{sjoŋsjoŋ}{}\classe{idph.2}
\begin{définition}\fra gris\end{définition}
\begin{définition}\cmn 形容灰中带白的颜色\end{définition}\end{entrée}

\begin{entrée}
\vedette{\hypertarget{Ⓔsjɯŋsjɯŋ}{\papi{ sjɯŋsjɯŋ}}}\markboth{sjɯŋsjɯŋ}{}\classe{idph.2}
\begin{définition}\fra blanc et haut\end{définition}
\begin{définition}\cmn 形容又高又白的样子\end{définition}
\begin{exemple}\jya mtɕhortɯn ci sjɯŋsjɯŋ ʑo ɣɤʑu\cmn 有一座白塔\end{exemple}
\end{entrée}

\begin{entrée}
\vedette{\hypertarget{Ⓔsjɯrnɤsjɯr}{\papi{ sjɯrnɤsjɯr}}}\markboth{sjɯrnɤsjɯr}{}\classe{idph.3}
\begin{définition}\fra qui émet de la fumée par intervalles réguliers\end{définition}
\begin{définition}\cmn 形容一次又一次地冒烟\end{définition}
\begin{exemple}\jya thamaka sjɯrnɤsjɯr ɲɯ-ɤsɯ-sko\cmn 他抽着烟,一次又一次地吐出烟来\end{exemple}\begin{sous-entrée}
\vedette{\hypertarget{}{\papi{ sjɯrɯri}}}\markboth{sjɯrɯri}{}\classe{idph.7}
\begin{définition}\fra qui s'élève lentement (fumée)\end{définition}
\begin{définition}\cmn 形容(烟)慢慢升起来的样子\end{définition}
\begin{exemple}\jya ɯ-thamaka ɯ-khɯ sjɯrɯri ʑo tu-ɬoʁ ɲɯ-ŋu\cmn 他抽的烟慢慢升起\end{exemple}
\end{sous-entrée}\end{entrée}

\begin{entrée}
\vedette{\hypertarget{Ⓔskɤɣ}{\papi{ skɤɣ}}}\markboth{skɤɣ}{}\classe{vt}
\paradigme{\textit{dir :} \jya kɤ-}
\begin{définition}\fra engraisser\end{définition}
\begin{définition}\cmn 催肥\end{définition}
\begin{exemple}\jya paʁ kɤ-skaɣ-a\cmn 我把猪催肥了\end{exemple}\end{entrée}

\begin{entrée}
\vedette{\hypertarget{Ⓔskɤlɤn}{\papi{ skɤlɤn}}}\markboth{skɤlɤn}{}
\classe{n}
\begin{définition}\fra réponse\end{définition}
\begin{définition}\cmn 回句;回音
\begin{déclaration} \étymologie{\papi{skad.len}}\end{déclaration}\end{définition}\end{entrée}

\begin{entrée}
\vedette{\hypertarget{Ⓔskɤlpa}{\papi{ skɤlpa}}}\markboth{skɤlpa}{}\classe{n}
\begin{définition}\fra monde\end{définition}
\begin{définition}\cmn 世界
\begin{déclaration} \étymologie{\papi{bskal.pa}}\end{déclaration}\end{définition}
\end{entrée}

\begin{entrée}
\vedette{\hypertarget{ⒺskɤmⒽ1}{\papi{ skɤm}}}\markboth{skɤm}{}\homonyme{1}\classe{n}
\begin{définition}\fra bœuf à viande\end{définition}
\begin{définition}\cmn 菜牛
\begin{déclaration} \étymologie{\papi{skom.po}}\end{déclaration}\end{définition}
\begin{exemple}\jya skɤmndʐi kɤ-qaʁ\cmn 剥牛皮\end{exemple}\end{entrée}

\begin{entrée}
\vedette{\hypertarget{ⒺskɤmⒽ2}{\papi{ skɤm}}}\markboth{skɤm}{}\homonyme{2}
\classe{vi}
\paradigme{\textit{dir :} \jya pɯ-}
\begin{définition}\fra s'assécher (lac, étang)\end{définition}
\begin{définition}\cmn 干涸(湖、池塘)
\begin{déclaration} \étymologie{\papi{skam}}\end{déclaration}\end{définition}
\begin{exemple}\jya ɕɯβloʁ pjɤ-skɤm\cmn 池塘干涸了\end{exemple}\end{entrée}

\begin{entrée}
\vedette{\hypertarget{Ⓔskɤmtɕhaŋ}{\papi{ skɤmtɕhaŋ}}}\markboth{skɤmtɕhaŋ}{}\classe{n}
\begin{définition}\fra tchang en jarre\end{définition}
\begin{définition}\cmn 青稞酒的一种做法\end{définition}
\begin{relation-sémantique}\synonyme{
\hyperlink{Ⓔchɤmda}{\textit{ \papi{chɤmda}}}
}\end{relation-sémantique}
\begin{relation-sémantique}\synonyme{
\hyperlink{Ⓔchɤci}{\textit{ \papi{chɤci}}}
}\end{relation-sémantique}\end{entrée}

\begin{entrée}
\vedette{\hypertarget{Ⓔskɤnɲɟɯr}{\papi{ skɤnɲɟɯr}}}\markboth{skɤnɲɟɯr}{}
\classe{n}
\begin{définition}\fra mélodie\end{définition}
\begin{définition}\cmn 曲调
\begin{déclaration} \étymologie{\papi{skad.ⁿgʲur}}\end{déclaration}\end{définition}
\begin{exemple}\jya nɯ rɤɣo nɯ ɣɯ ɯ-skɤnɲɟɯr ɲɯ-mpɕɤr\cmn 那首歌的曲子很好听\end{exemple}
\end{entrée}

\begin{entrée}
\vedette{\hypertarget{Ⓔskɤr}{\papi{ skɤr}}}\markboth{skɤr}{}\classe{vt}
\paradigme{\textit{dir :} \jya tɤ-}
\begin{définition}\fra peser\end{définition}
\begin{définition}\cmn 称
\begin{déclaration} \étymologie{\papi{skar}}\end{déclaration}\end{définition}
\begin{exemple}\jya tɤ-skar-a\cmn 我称了\end{exemple}
\begin{exemple}\jya ta-skɤr\cmn 他称了\end{exemple}
\begin{exemple}\jya laχtɕha tú-wɣ-skɤr\cmn 称东西\end{exemple}
\begin{exemple}\jya mbrɤz ta-skɤr\cmn 他称了米\end{exemple}
\begin{exemple}\jya tɤ-mthɯm ta-skɤr\cmn 他称了肉\end{exemple}
\begin{exemple}\jya tɯjpu tɤ-skar-a\cmn 我称了粮食\end{exemple}\begin{sous-entrée}
\vedette{\hypertarget{}{\papi{ rɤskɤr}}}\markboth{rɤskɤr}{}\classe{vi}
\paradigme{\textit{dir :} \jya tɤ-}
\begin{définition}\ 
\begin{déclaration}\grammar{apass}\end{déclaration}\end{définition}
\begin{définition}\fra peser\end{définition}
\begin{définition}\cmn 称东西\end{définition}
\end{sous-entrée}\end{entrée}

\begin{entrée}
\vedette{\hypertarget{Ⓔskɤrlɤm}{\papi{ skɤrlɤm}}}\markboth{skɤrlɤm}{}\classe{n}
\begin{définition}\fra chemin des moulins à prière\end{définition}
\begin{définition}\cmn 转经的路
\begin{déclaration} \étymologie{\papi{bskor.lam}}\end{déclaration}\end{définition}\end{entrée}

\begin{entrée}
\vedette{\hypertarget{Ⓔskɤrma}{\papi{ skɤrma}}}\markboth{skɤrma}{}
\classe{n}\acception{1}
\begin{définition}\fra règle\end{définition}
\begin{définition}\cmn 尺\end{définition}\acception{2}
\begin{définition}\fra jour\end{définition}
\begin{définition}\cmn 日子
\begin{déclaration} \étymologie{\papi{skar.ma}}\end{déclaration}\end{définition}
\begin{exemple}\jya skɤrma kɯ-sna\cmn 好日子\end{exemple}\end{entrée}

\begin{entrée}
\vedette{\hypertarget{Ⓔskɤrtɕɯn}{\papi{ skɤrtɕɯn}}}\markboth{skɤrtɕɯn}{}\classe{n}
\begin{définition}\fra vénus, étoile du soir\end{définition}
\begin{définition}\cmn 金星
\begin{déclaration} \étymologie{\papi{skar.tɕʰen}}\end{déclaration}\end{définition}\end{entrée}

\begin{entrée}
\vedette{\hypertarget{Ⓔskɤrwa}{\papi{ skɤrwa}}}\markboth{skɤrwa}{}\classe{n}
\begin{définition}\fra faire tourner les moulins à prière\end{définition}
\begin{définition}\cmn 转经
\begin{déclaration} \étymologie{\papi{skor.ba}}\end{déclaration}\end{définition}
\begin{exemple}\jya skɤrwa ko-ɕe\cmn 他去转经了\end{exemple}
\begin{exemple}\jya skɤrwa tɯ-tɤxɯr kɤ-ari-a\cmn 我转经转了一周\end{exemple}
\begin{relation-sémantique}\confer{
\hyperlink{Ⓔrɯskɤrwa}{\textit{ \papi{rɯskɤrwa}}}
}\end{relation-sémantique}\end{entrée}

\begin{entrée}
\vedette{\hypertarget{Ⓔskɤt}{\papi{ skɤt}}}\markboth{skɤt}{}\classe{vt}
\paradigme{\textit{dir :} \jya nɯ-}\acception{1}
\begin{définition}\fra rendre un objet\end{définition}
\begin{définition}\cmn 退(东西)\end{définition}
\begin{exemple}\jya laχtɕha na-skɤt\cmn 他把东西退了\end{exemple}
\begin{exemple}\jya jla na-skɤt\cmn 他把犏牛退了\end{exemple}\acception{2}
\begin{définition}\fra refuser\end{définition}
\begin{définition}\cmn 拒绝
\begin{déclaration}\use{不能带补语从句}\end{déclaration}\end{définition}\begin{sous-entrée}
\vedette{\hypertarget{}{\papi{ sɤskɤt}}}\markboth{sɤskɤt}{}\classe{vi}
\paradigme{\textit{dir :} \jya nɯ-}
\begin{définition}\ 
\begin{déclaration}\grammar{apass}\end{déclaration}\end{définition}
\begin{définition}\fra refuser\end{définition}
\begin{définition}\cmn 拒绝别人\end{définition}
\begin{exemple}\jya nɯ-sɤskat-a\cmn 我拒绝了\end{exemple}
\begin{exemple}\jya kɤ-sɤskɤt mɤ-nɤtsa\cmn 不好拒绝\end{exemple}
\end{sous-entrée}\end{entrée}

\begin{entrée}
\vedette{\hypertarget{Ⓔsko}{\papi{ sko}}}\markboth{sko}{}
\classe{vt}
\paradigme{\textit{dir :} \jya pɯ-}
\begin{définition}\fra fumer\end{définition}
\begin{définition}\cmn 抽烟\end{définition}
\begin{exemple}\jya thamakha pa-sko\cmn 他抽了烟\end{exemple}
\begin{exemple}\jya ma-pɯ-tɯ-sko-nɯ\cmn 你们不要抽烟\end{exemple}
\begin{exemple}\jya thamakha ma-pɯ-tɯ-skɤm ɯ́-jɤɣ?\cmn 麻烦你不要抽烟\end{exemple}
\begin{exemple}\jya thamakha kɤ-sko mɯ́j-pe ma ɲɯ́-ɣw-sɤɕqhe-a\cmn 抽烟是不好的,会令我咳嗽\end{exemple}\end{entrée}

\begin{entrée}
\vedette{\hypertarget{Ⓔskraskra}{\papi{ skraskra}}}\markboth{skraskra}{}
\classe{idph.2}\acception{1}
\begin{définition}\fra gris et dépourvu de végétation\end{définition}
\begin{définition}\cmn 形容(地方)灰扑扑、光秃秃\end{définition}
\begin{exemple}\jya sɤtɕha skraskra ʑo ɲɯ-pa, mɯ́j-sɤscit\cmn 那个地方是灰扑扑、光秃秃的,不舒服\end{exemple}\acception{2}
\begin{définition}\fra mal poli\end{définition}
\begin{définition}\cmn 形容调皮的样子\end{définition}
\begin{exemple}\jya tɯrme ɯ-stu mɤ-kɯ-fse ci skraskra ɲɯ-ŋu\cmn 那个人特别调皮\end{exemple}\end{entrée}

\begin{entrée}
\vedette{\hypertarget{Ⓔskrɯt}{\papi{ skrɯt}}}\markboth{skrɯt}{}
\classe{n}
\begin{définition}\fra fil très fin\end{définition}
\begin{définition}\cmn 丝\end{définition}
\begin{relation-sémantique}\confer{
\hyperlink{Ⓔɕomskrɯt}{\textit{ \papi{ɕomskrɯt}}}
}\end{relation-sémantique}\end{entrée}

\begin{entrée}
\vedette{\hypertarget{ⒺskɯⒽ2}{\papi{ skɯ}}}\markboth{skɯ}{}\homonyme{2}
\classe{n}
\begin{définition}\fra statue de bouddha\end{définition}
\begin{définition}\cmn 佛像
\begin{déclaration} \étymologie{\papi{sku}}\end{déclaration}\end{définition}
\end{entrée}

\begin{entrée}
\vedette{\hypertarget{ⒺskɯⒽ1}{\papi{ skɯ}}}\markboth{skɯ}{}\homonyme{1}\classe{vt}
\paradigme{\textit{dir :} \jya pɯ-}
\begin{définition}\fra enterrer\end{définition}
\begin{définition}\cmn 埋\end{définition}
\begin{exemple}\jya pjɤ-skɯ\cmn 把他埋了\end{exemple}
\begin{exemple}\jya fsapaʁ ɯ-ŋgo kɯ-tu pɯ-kɯ-si nɯ pjɯ́-wɣ-skɯ ŋu\cmn 病死了的牲畜要埋(不能吃它的肉)\end{exemple}\end{entrée}

\begin{entrée}
\vedette{\hypertarget{Ⓔskɯβli}{\papi{ skɯβli}}}\markboth{skɯβli}{}\classe{n}
\begin{définition}\fra cérémonie bouddhique\end{définition}
\begin{définition}\cmn 佛教仪式\end{définition}\end{entrée}

\begin{entrée}
\vedette{\hypertarget{Ⓔskɯrloʁ}{\papi{ skɯrloʁ}}}\markboth{skɯrloʁ}{}\classe{n}
\begin{définition}\fra type de pas d'aiguille\end{définition}
\begin{définition}\cmn 缝针的方法\end{définition}\end{entrée}

\begin{entrée}
\vedette{\hypertarget{Ⓔskɯrma}{\papi{ skɯrma}}}\markboth{skɯrma}{}\classe{n}
\begin{définition}\fra cadeau\end{définition}
\begin{définition}\cmn 礼物(请别人带的)
\begin{déclaration} \étymologie{\papi{skur.ma}}\end{déclaration}\end{définition}
\begin{exemple}\jya kɯki a-mu kɯ nɤ-skɯrma jɤ́-wɣ-sɯɣɯt-a ŋu\cmn 这是我母亲要我带给你的礼物\end{exemple}
\begin{relation-sémantique}\synonyme{
\hyperlink{Ⓔtɤ-pɤro}{\textit{ \papi{tɤ-pɤro}}}
}\end{relation-sémantique}
\begin{relation-sémantique}\synonyme{
\hyperlink{Ⓔtɤ-rkuz}{\textit{ \papi{tɤ-rkuz}}}
}\end{relation-sémantique}\end{entrée}

\begin{entrée}
\vedette{\hypertarget{Ⓔskɯχɕaʁ}{\papi{ skɯχɕaʁ}}}\markboth{skɯχɕaʁ}{}\classe{n}
\begin{définition}\fra feu (sprulsku)\end{définition}
\begin{définition}\cmn 圆寂了的活佛
\begin{déclaration} \étymologie{\papi{sku.gɕegs}}\end{déclaration}\end{définition}
\end{entrée}

\begin{entrée}
\vedette{\hypertarget{Ⓔsla}{\papi{ sla}}}\markboth{sla}{}
\classe{n}
\begin{définition}\fra lune\end{définition}
\begin{définition}\cmn 月亮\end{définition}
\begin{exemple}\jya sla qajɣi ɯ-rkɯ kɯ-fse ci ma ɲo-me\cmn 月亮只有馍馍那么大\end{exemple}
\begin{relation-sémantique}\confer{
\hyperlink{Ⓔslɤzɯn}{\textit{ \papi{slɤzɯn}}}
}\end{relation-sémantique}
\begin{relation-sémantique}\confer{
\hyperlink{Ⓔtɯ-sla}{\textit{ \papi{tɯ-sla}}}
}\end{relation-sémantique}
\begin{relation-sémantique}\confer{
\hyperlink{Ⓔslɤŋe}{\textit{ \papi{slɤŋe}}}
}\end{relation-sémantique}\end{entrée}

\begin{entrée}
\vedette{\hypertarget{Ⓔslama}{\papi{ slama}}}\markboth{slama}{}\classe{n}
\begin{définition}\fra étudiant\end{définition}
\begin{définition}\cmn 学生
\begin{déclaration} \étymologie{\papi{slob.ma}}\end{déclaration}\end{définition}
\begin{exemple}\jya nɤ-slama ɯ́-dɤn\cmn 你的学生很多吗?\end{exemple}
\begin{exemple}\jya slama tu-dɤn tɕe kɤ-sɤsɯxɕɤt mɤ-mbat ma\cmn 学生很多的话,不容易教\end{exemple}\end{entrée}

\begin{entrée}
\vedette{\hypertarget{Ⓔslaŋslaŋ}{\papi{ slaŋslaŋ}}}\markboth{slaŋslaŋ}{}
\classe{idph.2}
\begin{définition}\fra blanc et rond\end{définition}
\begin{définition}\cmn 形容又白又圆(美观)的样子\end{définition}
\begin{exemple}\jya sla slaŋslaŋ to-nɯɬoʁ\cmn 月亮出来了,皎洁圆润\end{exemple}
\begin{exemple}\jya ʑara nɯtɕu ʁʑɯnɯ ɲɤ-βzu tɕe slaŋslaŋ ʑo ɲɯ-pa, wuma ʑo ɲɯ-ɣɤχsrɯ\cmn 他们的儿子现在长成了小伙子了,又白又胖,非常英俊\end{exemple}\begin{sous-entrée}
\vedette{\hypertarget{}{\papi{ slaŋnɤslaŋ}}}\markboth{slaŋnɤslaŋ}{}\classe{idph.3}
\begin{exemple}\jya rŋgɯ slaŋnɤslaŋ pɯ-ndʐaβ\cmn 圆形的大石包滚下去了\end{exemple}
\begin{relation-sémantique}\confer{
\hyperlink{Ⓔrlaŋrlaŋ}{\textit{ \papi{rlaŋrlaŋ}}}
}\end{relation-sémantique}
\begin{relation-sémantique}\confer{
\hyperlink{Ⓔɕlaŋɕlaŋ}{\textit{ \papi{ɕlaŋɕlaŋ}}}
}\end{relation-sémantique}
\begin{relation-sémantique}\confer{
\hyperlink{Ⓔclaŋclaŋ}{\textit{ \papi{claŋclaŋ}}}
}\end{relation-sémantique}
\end{sous-entrée}\end{entrée}

\begin{entrée}
\vedette{\hypertarget{Ⓔslaʁ}{\papi{ slaʁ}}}\markboth{slaʁ}{}
\classe{idph.1}
\begin{définition}\fra d'un coup\end{définition}
\begin{définition}\cmn 一下子\end{définition}
\begin{exemple}\jya slaʁ ta-ndza\cmn 他一下子就吃了\end{exemple}
\begin{exemple}\jya slaʁ thɯ-ɕqhlɤt\cmn 他一下子就消失了\end{exemple}\begin{sous-entrée}
\vedette{\hypertarget{}{\papi{ phɯslaʁ}}}\markboth{phɯslaʁ}{}\classe{idph.5}
\begin{exemple}\jya pɣa phɯslaʁ ʑo thɯ-nɯqambɯmbjom\cmn 鸟突然就飞走了\end{exemple}
\begin{exemple}\jya phɯslaʁ ʑo nɯ-ʑɣɤɣɤme\cmn 突然间消失了\end{exemple}
\begin{relation-sémantique}\confer{
\hyperlink{Ⓔɕlaʁ}{\textit{ \papi{ɕlaʁ}}}
}\end{relation-sémantique}
\end{sous-entrée}\begin{sous-entrée}
\vedette{\hypertarget{}{\papi{ slaʁnɤslaʁ}}}\markboth{slaʁnɤslaʁ}{}\classe{idph.3}
\begin{définition}\fra en plusieurs coups\end{définition}
\begin{définition}\cmn 几下\end{définition}
\begin{exemple}\jya ta-ma slaʁnɤslaʁ ɲɯ-ɤz-nɤma\cmn 他几下子就做完了工作\end{exemple}
\end{sous-entrée}\end{entrée}

\begin{entrée}
\vedette{\hypertarget{Ⓔslɤβkhaŋ}{\papi{ slɤβkhaŋ}}}\markboth{slɤβkhaŋ}{}\classe{n}
\begin{définition}\fra école\end{définition}
\begin{définition}\cmn 学校
\begin{déclaration} \étymologie{\papi{slob.kʰaŋ}}\end{déclaration}\end{définition}\end{entrée}

\begin{entrée}
\vedette{\hypertarget{Ⓔslɤŋe}{\papi{ slɤŋe}}}\markboth{slɤŋe}{}
\classe{n}
\begin{définition}\fra clair de lune\end{définition}
\begin{définition}\cmn 月光\end{définition}
\begin{relation-sémantique}\confer{
\hyperlink{Ⓔtɤŋe}{\textit{ \papi{tɤŋe}}}
}\end{relation-sémantique}\end{entrée}

\begin{entrée}
\vedette{\hypertarget{Ⓔslɤrɯri}{\papi{ slɤrɯri}}}\markboth{slɤrɯri}{}
\classe{adv}
\begin{définition}\fra tous les mois\end{définition}
\begin{définition}\cmn 每个月\end{définition}\end{entrée}

\begin{entrée}
\vedette{\hypertarget{Ⓔslɤzɯn}{\papi{ slɤzɯn}}}\markboth{slɤzɯn}{}
\classe{n}
\begin{définition}\fra éclipse de lune\end{définition}
\begin{définition}\cmn 月蚀\end{définition}\end{entrée}

\begin{entrée}
\vedette{\hypertarget{Ⓔsloŋnɤsloŋ}{\papi{ sloŋnɤsloŋ}}}\markboth{sloŋnɤsloŋ}{}
\classe{idph.3}
\begin{définition}\fra vagues déferlantes\end{définition}
\begin{définition}\cmn 形容波浪汹涌的样子\end{définition}
\begin{exemple}\jya tɯ-ci chɤ-wxti tɕe sloŋnɤsloŋ ɲɯ-ʑɣɤstu\cmn 河水暴涨,波浪汹涌\end{exemple}\end{entrée}

\begin{entrée}
\vedette{\hypertarget{Ⓔsloʁ}{\papi{ sloʁ}}}\markboth{sloʁ}{}
\classe{vt}\acception{1}
\paradigme{\textit{dir :} \ }
\begin{définition}\fra fouir (cochon)\end{définition}
\begin{définition}\cmn 用鼻子拱(猪)\end{définition}
\begin{exemple}\jya phaʁrgot kɯ sɤtɕha ɲɤ-sloʁ\cmn 野猪拱了土\end{exemple}\acception{2}
\paradigme{\textit{dir :} \jya tɤ-}
\begin{définition}\fra déterrer\end{définition}
\begin{définition}\cmn 挖出来;掏出来\end{définition}
\begin{exemple}\jya laχtɕha tɤ-sloʁ-a\cmn 我把东西挖出来了\end{exemple}\begin{sous-entrée}
\vedette{\hypertarget{}{\papi{ rɤsloʁ}}}\markboth{rɤsloʁ}{}\classe{vi}
\paradigme{\textit{dir :} \jya nɯ-}
\paradigme{\textit{dir :} \jya tɤ-}
\begin{définition}\ 
\begin{déclaration}\grammar{apass}\end{déclaration}\end{définition}
\begin{exemple}\jya paʁ ɲɤ-rɤsloʁ\cmn 猪拱地了\end{exemple}
\end{sous-entrée}\begin{sous-entrée}
\vedette{\hypertarget{}{\papi{ sɯsloʁ}}}\markboth{sɯsloʁ}{}\classe{vt}
\begin{définition}\fra fouir avec\end{définition}
\begin{définition}\cmn 用……拱(猪)\end{définition}
\begin{exemple}\jya ɯ-ɕna kɯ ɲɯ-sɯsloʁ tɕe pjɯ-lɣe ɲɯ-ɕti\cmn (野猪)用鼻子拱地\end{exemple}
\end{sous-entrée}\end{entrée}

\begin{entrée}
\vedette{\hypertarget{Ⓔsloχpɯn}{\papi{ sloχpɯn}}}\markboth{sloχpɯn}{}
\classe{n}
\begin{définition}\fra professeur\end{définition}
\begin{définition}\cmn 老师
\begin{déclaration} \étymologie{\papi{slob.dpon}}\end{déclaration}\end{définition}\end{entrée}

\begin{entrée}
\vedette{\hypertarget{Ⓔslɯŋslɯŋ}{\papi{ slɯŋslɯŋ}}}\markboth{slɯŋslɯŋ}{}\classe{idph.2}
\begin{définition}\fra lumineux et rond\end{définition}
\begin{définition}\cmn 形容又亮又圆的样子\end{définition}\begin{sous-entrée}
\vedette{\hypertarget{}{\papi{ slɯŋɯŋi}}}\markboth{slɯŋɯŋi}{}\classe{idph.9}
\begin{exemple}\jya tɤŋe slɯŋɯŋi ʑo to-nɯɬoʁ\cmn 太阳慢慢地放着光辉升起来了\end{exemple}
\end{sous-entrée}\end{entrée}

\begin{entrée}
\vedette{\hypertarget{Ⓔsmar}{\papi{ smar}}}\markboth{smar}{}
\classe{n}
\begin{définition}\fra fleuve\end{définition}
\begin{définition}\cmn 河流\end{définition}\end{entrée}

\begin{entrée}
\vedette{\hypertarget{Ⓔsmɤɣ}{\papi{ smɤɣ}}}\markboth{smɤɣ}{}
\classe{n}
\begin{définition}\fra laine\end{définition}
\begin{définition}\cmn 羊毛
\begin{déclaration}\use{羊毛剪下来了以后才是\stylefv{smɤɣ},绵羊身上的叫\stylefv{ɯ-rme}}\end{déclaration}\end{définition}\end{entrée}

\begin{entrée}
\vedette{\hypertarget{Ⓔsmɤɣri}{\papi{ smɤɣri}}}\markboth{smɤɣri}{}\classe{n}
\begin{définition}\fra fil de laine\end{définition}
\begin{définition}\cmn 羊毛线\end{définition}
\begin{relation-sémantique}\confer{
\hyperlink{Ⓔsmɤɣ}{\textit{ \papi{smɤɣ}}}
}\end{relation-sémantique}
\begin{relation-sémantique}\confer{
\hyperlink{Ⓔtɤ-ri}{\textit{ \papi{tɤ-ri}}}
}\end{relation-sémantique}
\end{entrée}

\begin{entrée}
\vedette{\hypertarget{Ⓔsmɤn}{\papi{ smɤn}}}\markboth{smɤn}{}\classe{n}
\begin{définition}\fra médicament\end{définition}
\begin{définition}\cmn 药
\begin{déclaration} \étymologie{\papi{sman}}\end{déclaration}\end{définition}
\begin{exemple}\jya smɤn kɤ-ndza a-mɤ-nɯ-tɯ-jmɯt je!\cmn 你不要忘记吃药\end{exemple}
\begin{exemple}\jya smɤn tɕhi pɯ-nɯ-ŋɯ-ŋu ʑo nɤ ɯ-tshɤt kɤ-ndza ra\cmn 无论什么药,都不能吃太多\end{exemple}\end{entrée}

\begin{entrée}
\vedette{\hypertarget{Ⓔsmɤnba}{\papi{ smɤnba}}}\markboth{smɤnba}{}\classe{n}
\begin{définition}\fra médecin\end{définition}
\begin{définition}\cmn 医生
\begin{déclaration} \étymologie{\papi{sman.pa}}\end{déclaration}\end{définition}
\end{entrée}

\begin{entrée}
\vedette{\hypertarget{Ⓔsmɤnkhaŋ}{\papi{ smɤnkhaŋ}}}\markboth{smɤnkhaŋ}{}
\classe{n}
\begin{définition}\fra hôpital\end{définition}
\begin{définition}\cmn 医院\end{définition}
\begin{exemple}\jya smɤnkhaŋ jɤ-tɯ-ari ?\cmn 你去了医院吗?\end{exemple}\end{entrée}

\begin{entrée}
\vedette{\hypertarget{Ⓔsmɤnrɯɣ}{\papi{ smɤnrɯɣ}}}\markboth{smɤnrɯɣ}{}\classe{n}
\begin{définition}\fra médicament\end{définition}
\begin{définition}\cmn 药材
\begin{déclaration} \étymologie{\papi{sman.rigs}}\end{déclaration}\end{définition}\end{entrée}

\begin{entrée}
\vedette{\hypertarget{Ⓔsmɤʁjoʁ}{\papi{ smɤʁjoʁ}}}\markboth{smɤʁjoʁ}{}\classe{n}
\begin{définition}\fra habit de moine\end{définition}
\begin{définition}\cmn 和尚服装的一种,穿在腰上
\end{définition}
\begin{relation-sémantique}\confer{
\hyperlink{Ⓔtɯ-smɤt}{\textit{ \papi{tɯ-smɤt}}}
}\end{relation-sémantique}\end{entrée}

\begin{entrée}
\vedette{\hypertarget{Ⓔsmɤt}{\papi{ smɤt}}}\markboth{smɤt}{}\classe{vt}
\paradigme{\textit{dir :} \jya pɯ-}
\begin{définition}\fra rabaisser qqn\end{définition}
\begin{définition}\cmn 贬低\end{définition}
\begin{exemple}\jya ɯʑo kɯ aʑo pɯ́-wɣ-smat-a\cmn 他贬低了我\end{exemple}
\begin{relation-sémantique}\confer{
\hyperlink{Ⓔnɯsmɤphɤβ}{\textit{ \papi{nɯsmɤphɤβ}}}
}\end{relation-sémantique}\end{entrée}

\begin{entrée}
\vedette{\hypertarget{ⒺsmiⒽ2}{\papi{ smi}}}\markboth{smi}{}\homonyme{2}
\classe{n}
\begin{définition}\fra feu\end{définition}
\begin{définition}\cmn 火\end{définition}
\begin{exemple}\jya tɕetha mɤʑɯ tɯtshot kɯmŋu-skɤrma tɕe kɤ-lɤt ma, aj smi ci tu-βze-a ntshi ma ɲɯ-mɯɕtaʁ wo\cmn 你过五分钟再打电话,我先烧一点火,因为很冷\end{exemple}\end{entrée}

\begin{entrée}
\vedette{\hypertarget{ⒺsmiⒽ1}{\papi{ smi}}}\markboth{smi}{}\homonyme{1}
\classe{vs}
\paradigme{\textit{dir :} \jya kɤ-}
\begin{définition}\fra cuit (bouilli ou à la vapeur)\end{définition}
\begin{définition}\cmn (煮、蒸)熟的\end{définition}
\begin{exemple}\jya mbrɤz ko-smi\cmn 饭熟了\end{exemple}
\begin{exemple}\jya tɤ-mthɯm ko-smi\cmn 肉熟了\end{exemple}
\begin{relation-sémantique}\confer{
 \papi{ɣɤsmi1}
}\end{relation-sémantique}\end{entrée}

\begin{entrée}
\vedette{\hypertarget{Ⓔsmɯ}{\papi{ smɯ}}}\markboth{smɯ}{}
\classe{n}
\begin{définition}\fra vache\end{définition}
\begin{définition}\cmn 小奶牛\end{définition}\end{entrée}

\begin{entrée}
\vedette{\hypertarget{Ⓔsmɯɣdɯm}{\papi{ smɯɣdɯm}}}\markboth{smɯɣdɯm}{}
\classe{n}
\begin{définition}\fra bois de chauffage\end{définition}
\begin{définition}\cmn 柴\end{définition}\end{entrée}

\begin{entrée}
\vedette{\hypertarget{Ⓔsmɯɣot}{\papi{ smɯɣot}}}\markboth{smɯɣot}{}
\classe{n}
\begin{définition}\fra lumière du feu\end{définition}
\begin{définition}\cmn 火的热光\end{définition}
\begin{exemple}\jya smɯɣot ɲɯ-mpja\cmn 火光发热\end{exemple}
\begin{relation-sémantique}\confer{
\hyperlink{ⒺsmiⒽ2}{\textit{ \papi{smi2}}}
}\end{relation-sémantique}\end{entrée}

\begin{entrée}
\vedette{\hypertarget{Ⓔsmɯɣsmɯɣ}{\papi{ smɯɣsmɯɣ}}}\markboth{smɯɣsmɯɣ}{}
\classe{idph.2}
\begin{définition}\fra vert vif\end{définition}
\begin{définition}\cmn 形容绿油油的色泽\end{définition}
\begin{exemple}\jya tɯji nɯ smɯɣsmɯɣ ɲɯ-pa\cmn 田是绿油油的\end{exemple}\end{entrée}

\begin{entrée}
\vedette{\hypertarget{Ⓔsmɯlɤm}{\papi{ smɯlɤm}}}\markboth{smɯlɤm}{}\classe{n}
\begin{définition}\fra prière, espoir\end{définition}
\begin{définition}\cmn 愿望
\begin{déclaration} \étymologie{\papi{smon.lam}}\end{déclaration}\end{définition}
\begin{exemple}\jya a-pɯ-cha smɯlɤm\cmn 希望能够成功!\end{exemple}\end{entrée}

\begin{entrée}
\vedette{\hypertarget{Ⓔsmɯlju}{\papi{ smɯlju}}}\markboth{smɯlju}{}\classe{n}
\begin{définition}\ 
\begin{déclaration}\grammar{n.lieu}\end{déclaration}\end{définition}
\begin{définition}\fra Smeliou (village de Gdongbrgyad)\end{définition}
\begin{définition}\cmn 石木留村\end{définition}
\end{entrée}

\begin{entrée}
\vedette{\hypertarget{Ⓔsmɯmba}{\papi{ smɯmba}}}\markboth{smɯmba}{}
\classe{n}
\begin{définition}\fra flamme\end{définition}
\begin{définition}\cmn 火焰\end{définition}
\begin{relation-sémantique}\confer{
\hyperlink{ⒺsmiⒽ2}{\textit{ \papi{smi2}}}
}\end{relation-sémantique}\end{entrée}

\begin{entrée}
\vedette{\hypertarget{Ⓔsmɯn}{\papi{ smɯn}}}\markboth{smɯn}{}\classe{vs}
\paradigme{\textit{dir :} \jya pɯ-}
\begin{définition}\fra porter des fruits\end{définition}
\begin{définition}\cmn 成熟
\begin{déclaration} \étymologie{\papi{smin}}\end{déclaration}\end{définition}
\begin{exemple}\jya ɯ-mbrɤzɯ smɯn\cmn 会有好结果\end{exemple}\end{entrée}

\begin{entrée}
\vedette{\hypertarget{Ⓔsmɯntʂɯɣ}{\papi{ smɯntʂɯɣ}}}\markboth{smɯntʂɯɣ}{}\classe{n}
\begin{définition}\fra pléiades\end{définition}
\begin{définition}\cmn 昴宿
\begin{déclaration} \étymologie{\papi{smin.drug}}\end{déclaration}\end{définition}
\end{entrée}

\begin{entrée}
\vedette{\hypertarget{Ⓔsmɯŋgɯ}{\papi{ smɯŋgɯ}}}\markboth{smɯŋgɯ}{}\classe{n}
\begin{définition}\fra milieu du feu\end{définition}
\begin{définition}\cmn 火中间\end{définition}
\begin{exemple}\jya smɯŋgɯ kɤ-ɕe mɤ-βdi ma sɤɕke\cmn 不要走到火中间,会烫到\end{exemple}
\begin{relation-sémantique}\confer{
\hyperlink{ⒺsmiⒽ2}{\textit{ \papi{smi2}}}
}\end{relation-sémantique}
\begin{relation-sémantique}\confer{
\hyperlink{Ⓔɯ-ŋgɯ}{\textit{ \papi{ɯ-ŋgɯ}}}
}\end{relation-sémantique}\end{entrée}

\begin{entrée}
\vedette{\hypertarget{Ⓔsmɯr}{\papi{ smɯr}}}\markboth{smɯr}{}
\classe{n}
\begin{définition}\fra morceau de glace flottant sur le fleuve\end{définition}
\begin{définition}\cmn 河流上的冰块\end{définition}\end{entrée}

\begin{entrée}
\vedette{\hypertarget{Ⓔsmɯrqom}{\papi{ smɯrqom}}}\markboth{smɯrqom}{}\classe{n}
\begin{définition}\fra espèce de plante\end{définition}
\begin{définition}\cmn 植物的一种\end{définition}
\begin{relation-sémantique}\confer{
\hyperlink{Ⓔɬɤndʐitɤtsoʁ}{\textit{ \papi{ɬɤndʐitɤtsoʁ}}}
}\end{relation-sémantique}\end{entrée}

\begin{entrée}
\vedette{\hypertarget{Ⓔsmɯʁjoʁ}{\papi{ smɯʁjoʁ}}}\markboth{smɯʁjoʁ}{}
\classe{n}
\begin{définition}\fra crochet de cheminée\end{définition}
\begin{définition}\cmn 火钩\end{définition}
\begin{relation-sémantique}\synonyme{
\hyperlink{Ⓔɕɤmiŋoʁ}{\textit{ \papi{ɕɤmiŋoʁ}}}
}\end{relation-sémantique}
\begin{relation-sémantique}\confer{
\hyperlink{ⒺsmiⒽ2}{\textit{ \papi{smi2}}}
}\end{relation-sémantique}\end{entrée}

\begin{entrée}
\vedette{\hypertarget{Ⓔsmɯʁrɤt}{\papi{ smɯʁrɤt}}}\markboth{smɯʁrɤt}{}
\classe{n}
\begin{définition}\fra cendre\end{définition}
\begin{définition}\cmn 火炭\end{définition}\end{entrée}

\begin{entrée}
\vedette{\hypertarget{Ⓔsmɯsmi}{\papi{ smɯsmi}}}\markboth{smɯsmi}{}
\classe{n}
\begin{définition}\fra bûche utilisée pour rallumer le feu\end{définition}
\begin{définition}\cmn 火种\end{définition}\end{entrée}

\begin{entrée}
\vedette{\hypertarget{Ⓔsmɯtɕɣom}{\papi{ smɯtɕɣom}}}\markboth{smɯtɕɣom}{}
\classe{n}
\begin{définition}\fra étincelle\end{définition}
\begin{définition}\cmn 火星\end{définition}\end{entrée}

\begin{entrée}
\vedette{\hypertarget{Ⓔsna}{\papi{ sna}}}\markboth{sna}{}
\classe{vs}
\paradigme{\textit{dir :} \jya tɤ-}\acception{1}
\begin{définition}\fra utilisable\end{définition}
\begin{définition}\cmn 质量好\end{définition}
\begin{exemple}\jya laχtɕha ɲɯ-sna\cmn 那个东西的质量很好\end{exemple}
\begin{exemple}\jya ɕoŋtɕa ɲɯ-sna\cmn 木料很好\end{exemple}
\begin{exemple}\jya tɯ-rɟɯ ɲɯ-sna\cmn 财物的质量好\end{exemple}
\begin{exemple}\jya tʂha ɲɯ-sna\cmn 茶很浓\end{exemple}
\begin{exemple}\jya nɤʑo sna mataŋe\cmn 你真没有用\end{exemple}
\begin{exemple}\jya sna ɣɤtɤʑu nɤ, ɲɯ-tɯ-rkaŋ!\cmn 你真能干!\end{exemple}\acception{2}
\begin{définition}\fra généreux\end{définition}
\begin{définition}\cmn 大方;热情;善良\end{définition}\end{entrée}

\begin{entrée}
\vedette{\hypertarget{Ⓔsnalŋaɕthɤβ}{\papi{ snalŋaɕthɤβ}}}\markboth{snalŋaɕthɤβ}{}\classe{n}
\begin{définition}\fra lanière colorée (pantalon, chaussures)\end{définition}
\begin{définition}\cmn 彩色的带子(系衣服、鞋子)\end{définition}\end{entrée}

\begin{entrée}
\vedette{\hypertarget{Ⓔsnama}{\papi{ snama}}}\markboth{snama}{}
\classe{n}
\begin{définition}\fra animal domestique capable de porter des charges\end{définition}
\begin{définition}\cmn 专门用来驮东西的牲畜\end{définition}\end{entrée}

\begin{entrée}
\vedette{\hypertarget{Ⓔsnaŋwa}{\papi{ snaŋwa}}}\markboth{snaŋwa}{}\classe{n}
\begin{définition}\fra état d'esprit\end{définition}
\begin{définition}\cmn 性情
\begin{déclaration} \étymologie{\papi{snaŋ.ba}}\end{déclaration}\end{définition}
\begin{exemple}\jya snaŋwa ɲɯ-mthu\cmn 自信心强\end{exemple}
\end{entrée}

\begin{entrée}
\vedette{\hypertarget{Ⓔsnaʁtsa}{\papi{ snaʁtsa}}}\markboth{snaʁtsa}{}\classe{n}
\begin{définition}\fra encre\end{définition}
\begin{définition}\cmn 墨
\begin{déclaration} \étymologie{\papi{snag.tsʰa}}\end{déclaration}\end{définition}
\end{entrée}

\begin{entrée}
\vedette{\hypertarget{Ⓔsnoŋwa}{\papi{ snoŋwa}}}\markboth{snoŋwa}{}
\classe{n}
\begin{définition}\fra confiance en soi\end{définition}
\begin{définition}\cmn 自信
\begin{déclaration} \étymologie{\papi{snaŋ.ba}}\end{déclaration}\end{définition}
\begin{exemple}\jya ɯ-snoŋwa ɲɯ-mthu\cmn 他很有自信,觉得自高自大\end{exemple}\end{entrée}

\begin{entrée}
\vedette{\hypertarget{Ⓔsnɯm}{\papi{ snɯm}}}\markboth{snɯm}{}
\classe{n}
\begin{définition}\fra huile contenu dans les poils; viande, huile\end{définition}
\begin{définition}\cmn 羊毛和牛毛中含着的油脂;肉;油的统称
\begin{déclaration} \étymologie{\papi{snum}}\end{déclaration}\end{définition}\end{entrée}

\begin{entrée}
\vedette{\hypertarget{Ⓔsnɯɲaʁ}{\papi{ snɯɲaʁ}}}\markboth{snɯɲaʁ}{}
\classe{n}
\begin{définition}\fra fait de causer du tort\end{définition}
\begin{définition}\cmn 害人(的阴谋)\end{définition}
\begin{exemple}\jya snɯɲaʁ ma-tɤ-tɯ-βze\cmn 你不要害人\end{exemple}
\begin{relation-sémantique}\confer{
\hyperlink{Ⓔtɯ-sni}{\textit{ \papi{tɯ-sni}}}
}\end{relation-sémantique}
\begin{relation-sémantique}\confer{
\hyperlink{Ⓔɲaʁ}{\textit{ \papi{ɲaʁ}}}
}\end{relation-sémantique}
\begin{relation-sémantique}\confer{
\hyperlink{Ⓔnɯsnɯɲaʁ}{\textit{ \papi{nɯsnɯɲaʁ}}}
}\end{relation-sémantique}\end{entrée}

\begin{entrée}
\vedette{\hypertarget{Ⓔsɲu}{\papi{ sɲu}}}\markboth{sɲu}{}
\classe{vi}
\paradigme{\textit{dir :} \jya nɯ-}
\begin{définition}\fra être fou\end{définition}
\begin{définition}\cmn 疯
\begin{déclaration} \étymologie{\papi{smʲo}}\end{déclaration}\end{définition}
\begin{exemple}\jya jiɕqha nɯ ɲɤ-sɲu\cmn 那个人疯了\end{exemple}
\begin{exemple}\jya kɯ-sɲu ci ɲɯ-ŋu\cmn 他是疯子\end{exemple}
\begin{relation-sémantique}\confer{
\hyperlink{Ⓔchɯsɲu}{\textit{ \papi{chɯsɲu}}}
}\end{relation-sémantique}\begin{sous-entrée}
\vedette{\hypertarget{}{\papi{ nɤsɲɯsɲu}}}\markboth{nɤsɲɯsɲu}{}\classe{vs}
\begin{définition}\fra un peu fou (parfois normal, parfois fou)\end{définition}
\begin{définition}\cmn 疯疯癫癫
\end{définition}
\begin{exemple}\jya tɯrme kɯ-nɤsɲɯsɲu ci jɤ-ɣe\cmn 来了一个疯疯癫癫的人\end{exemple}
\end{sous-entrée}\begin{sous-entrée}
\vedette{\hypertarget{}{\papi{ sɯsɲu}}}\markboth{sɯsɲu}{}\classe{vt}
\paradigme{\textit{dir :} \jya nɯ-}
\begin{définition}\ 
\begin{déclaration}\grammar{caus}\end{déclaration}\end{définition}
\begin{définition}\fra rendre fou\end{définition}
\begin{définition}\cmn 令……疯\end{définition}
\end{sous-entrée}\end{entrée}

\begin{entrée}
\vedette{\hypertarget{Ⓔsɲaŋne}{\papi{ sɲaŋne}}}\markboth{sɲaŋne}{}
\classe{n}
\begin{définition}\fra upavâsa, fête du jeûne\end{définition}
\begin{définition}\cmn 哑巴经(禁食斋)
\begin{déclaration} \étymologie{\papi{smʲuŋ.gnas}}\end{déclaration}\end{définition}
\begin{exemple}\jya sɲaŋne kɤ-ndo-t-a\cmn 我念了哑巴经\end{exemple}
\begin{relation-sémantique}\confer{
\hyperlink{Ⓔrɯsɲaŋne}{\textit{ \papi{rɯsɲaŋne}}}
}\end{relation-sémantique}\end{entrée}

\begin{entrée}
\vedette{\hypertarget{Ⓔsɲaʁsɲaʁ}{\papi{ sɲaʁsɲaʁ}}}\markboth{sɲaʁsɲaʁ}{}
\classe{idph.2}
\begin{définition}\fra très pointu\end{définition}
\begin{définition}\cmn 形容很尖\end{définition}
\begin{exemple}\jya ɲɯ-ɤmtɕoʁ ʑo sɲaʁsɲaʁ\cmn 非常尖\end{exemple}
\begin{exemple}\jya mbrɯtɕɯ ɯ-ku amtɕoʁ ʑo sɲaʁsɲaʁ\cmn 刀子很尖\end{exemple}\end{entrée}

\begin{entrée}
\vedette{\hypertarget{Ⓔsɲɤβnɤlɤβ}{\papi{ sɲɤβnɤlɤβ}}}\markboth{sɲɤβnɤlɤβ}{}
\classe{idph.4}
\begin{définition}\fra titubant\end{définition}
\begin{définition}\cmn 形容走路摇晃不稳\end{définition}
\begin{exemple}\jya lo-βzi tɕe sɲɤβnɤlɤβ ʑo ɲɯ-ʑɣɤstu\cmn 他喝醉了,走路东摇西摆的\end{exemple}\end{entrée}

\begin{entrée}
\vedette{\hypertarget{Ⓔsɲɤɣsɲɤɣ}{\papi{ sɲɤɣsɲɤɣ}}}\markboth{sɲɤɣsɲɤɣ}{}
\classe{idph.2}\acception{1}
\begin{définition}\fra grand et gros (corps)\end{définition}
\begin{définition}\cmn 形容身体大而胖的样子\end{définition}
\begin{exemple}\jya tɤ-pɤtso cho-wxti sɲɤɣsɲɤɣ ʑo\cmn 小孩子长大了,变得又大又胖\end{exemple}\acception{2}
\begin{définition}\fra bête\end{définition}
\begin{définition}\cmn 形容笨重或者不聪明的样子
\end{définition}
\begin{exemple}\jya sɲɤɣsɲɤɣ ʑo ma-tɤ-tɯ-ʑɣɤstu\cmn 你要做出傻乎乎的样子\end{exemple}\end{entrée}

\begin{entrée}
\vedette{\hypertarget{Ⓔsɲɤt}{\papi{ sɲɤt}}}\markboth{sɲɤt}{}
\classe{n}
\begin{définition}\fra harnais\end{définition}
\begin{définition}\cmn 后輶
\begin{déclaration} \étymologie{\papi{rmed}}\end{déclaration}\end{définition}\end{entrée}

\begin{entrée}
\vedette{\hypertarget{Ⓔsɲikuku}{\papi{ sɲikuku}}}\markboth{sɲikuku}{}
\classe{adv}
\begin{définition}\fra tous les jours\end{définition}
\begin{définition}\cmn 每天\end{définition}\end{entrée}

\begin{entrée}
\vedette{\hypertarget{Ⓔsɲoʁ}{\papi{ sɲoʁ}}}\markboth{sɲoʁ}{}
\classe{vs}
\begin{définition}\fra épaisse (huile)\end{définition}
\begin{définition}\cmn 稠(油)\end{définition}
\begin{exemple}\jya pɤlɤtɕɯ wuma ʑo ɲɯ-sɲoʁ tɕe kɯ-dɤn kɤ-ndza mɯ́j-sɤcha\cmn 酥油馍馍有多油,吃不下很多\end{exemple}\end{entrée}

\begin{entrée}
\vedette{\hypertarget{Ⓔsɲɯɣ}{\papi{ sɲɯɣ}}}\markboth{sɲɯɣ}{}\classe{idph.1}
\begin{définition}\fra douleur lancinante\end{définition}
\begin{définition}\cmn 形容痛得钻心\end{définition}
\begin{exemple}\jya ɯ-tɯ-mŋɤm kɯ sɲɯɣ ʑo ɲɯ-ti\cmn 他一下子就痛得钻心\end{exemple}\begin{sous-entrée}
\vedette{\hypertarget{}{\papi{ sɲɯɣnɤsɲɯɣ}}}\markboth{sɲɯɣnɤsɲɯɣ}{}\classe{idph.3}
\begin{exemple}\jya sɲɯɣnɤsɲɯɣ ʑo ɲɯ-ti ɲɯ-mŋɤm\cmn 他痛得一阵又一阵\end{exemple}
\begin{relation-sémantique}\confer{
\hyperlink{Ⓔɕɲɯɣ}{\textit{ \papi{ɕɲɯɣ}}}
}\end{relation-sémantique}
\end{sous-entrée}\end{entrée}

\begin{entrée}
\vedette{\hypertarget{Ⓔsɲɯɣjɯ}{\papi{ sɲɯɣjɯ}}}\markboth{sɲɯɣjɯ}{}
\classe{n}
\begin{définition}\fra pinceau\end{définition}
\begin{définition}\cmn 笔
\begin{déclaration} \étymologie{\papi{smʲu.gu}}\end{déclaration}\end{définition}\end{entrée}

\begin{entrée}
\vedette{\hypertarget{Ⓔsɲɯŋgɯpala}{\papi{ sɲɯŋgɯpala}}}\markboth{sɲɯŋgɯpala}{}
\classe{adv}
\begin{définition}\fra en plein jour\end{définition}
\begin{définition}\cmn 大白天\end{définition}\end{entrée}

\begin{entrée}
\vedette{\hypertarget{Ⓔsɲɯŋɯŋi}{\papi{ sɲɯŋɯŋi}}}\markboth{sɲɯŋɯŋi}{}\classe{idph.7}
\begin{définition}\fra s'élevant (fumée)\end{définition}
\begin{définition}\cmn 形容青烟慢慢升起的样子\end{définition}
\begin{exemple}\jya tɤ-khɯ sɲɯŋɯŋi ʑo to-ɬoʁ\cmn 青烟袅袅升起\end{exemple}\end{entrée}

\begin{entrée}
\vedette{\hypertarget{Ⓔsŋu}{\papi{ sŋu}}}\markboth{sŋu}{}\classe{vt}
\paradigme{\textit{dir :} \jya tɤ-}
\begin{définition}\fra ordonner\end{définition}
\begin{définition}\cmn 嘱咐\end{définition}
\begin{exemple}\jya sprɯskɯ ra kɯ a-ɕki ``nɯ a-tɤ-tɯ-fse ra" ta-sŋu-nɯ\cmn 活佛们嘱咐我要这样做\end{exemple}\end{entrée}

\begin{entrée}
\vedette{\hypertarget{Ⓔsŋa}{\papi{ sŋa}}}\markboth{sŋa}{}
\classe{vi}
\paradigme{\textit{dir :} \jya tɤ-}
\begin{définition}\fra revivre, revenir à soi\end{définition}
\begin{définition}\cmn 复活;苏醒\end{définition}
\begin{exemple}\jya to-sŋa\cmn 他苏醒了\end{exemple}\begin{sous-entrée}
\vedette{\hypertarget{}{\papi{ sɯsŋa}}}\markboth{sɯsŋa}{}\classe{vt}
\paradigme{\textit{dir :} \jya tɤ-}
\begin{définition}\ 
\begin{déclaration}\grammar{caus}\end{déclaration}\end{définition}
\begin{définition}\fra faire revivre\end{définition}
\begin{définition}\cmn 令人复活、苏醒\end{définition}
\end{sous-entrée}\end{entrée}

\begin{entrée}
\vedette{\hypertarget{Ⓔsŋarɤβ}{\papi{ sŋarɤβ}}}\markboth{sŋarɤβ}{}
\classe{n}
\begin{définition}\fra autrefois\end{définition}
\begin{définition}\cmn 古时候
\begin{déclaration} \étymologie{\papi{sŋa.rabs}}\end{déclaration}\end{définition}\end{entrée}

\begin{entrée}
\vedette{\hypertarget{ⒺsŋaʁⒽ2}{\papi{ sŋaʁ}}}\markboth{sŋaʁ}{}\homonyme{2}\classe{n}
\begin{définition}\fra sorcellerie\end{définition}
\begin{définition}\cmn 法术
\begin{déclaration} \étymologie{\papi{sŋags}}\end{déclaration}\end{définition}
\begin{exemple}\jya ɯʑo sŋaʁ pjɤ-k-ɤʁe-ci\cmn 他被施了法术\end{exemple}
\begin{exemple}\jya sŋaʁ nɯra maka mɯ-pjɤ-mɟa\cmn 法术对她根本无效\end{exemple}\end{entrée}

\begin{entrée}
\vedette{\hypertarget{ⒺsŋaʁⒽ1}{\papi{ sŋaʁ}}}\markboth{sŋaʁ}{}\homonyme{1}\classe{vt}
\paradigme{\textit{dir :} \jya tɤ-}
\begin{définition}\fra maudire, ensorceller\end{définition}
\begin{définition}\cmn 施法术;诅咒
\begin{déclaration} \étymologie{\papi{sŋags}}\end{déclaration}\end{définition}
\begin{exemple}\jya to-sŋaʁ (=tɯ-sŋaʁ to-ɕe)\cmn 他咒了他\end{exemple}
\begin{exemple}\jya tɤ́-wɣ-sŋaʁ-a\cmn 他咒了我\end{exemple}\begin{sous-entrée}
\vedette{\hypertarget{}{\papi{ sɯsŋaʁ}}}\markboth{sɯsŋaʁ}{}\classe{vt}
\paradigme{\textit{dir :} \jya tɤ-}
\begin{définition}\fra faire ensorceller\end{définition}
\begin{définition}\cmn 叫人用法术\end{définition}
\end{sous-entrée}\begin{sous-entrée}
\vedette{\hypertarget{}{\papi{ ʑɣɤsɯsŋaʁ}}}\markboth{ʑɣɤsɯsŋaʁ}{}\classe{vi}
\paradigme{\textit{dir :} \jya tɤ-}
\begin{définition}\fra se faire ensorceller\end{définition}
\begin{définition}\cmn 招人给自己施法术\end{définition}
\end{sous-entrée}\end{entrée}

\begin{entrée}
\vedette{\hypertarget{Ⓔsŋaʁspa}{\papi{ sŋaʁspa}}}\markboth{sŋaʁspa}{}
\classe{n}
\begin{définition}\fra sorcier\end{définition}
\begin{définition}\cmn 法师;巫师
\begin{déclaration} \étymologie{\papi{sŋags.pa}}\end{déclaration}\end{définition}\end{entrée}

\begin{entrée}
\vedette{\hypertarget{Ⓔsŋɤrɯ}{\papi{ sŋɤrɯ}}}\markboth{sŋɤrɯ}{}
\classe{n}
\begin{définition}\fra avant de la selle\end{définition}
\begin{définition}\cmn 前鞍桥
\begin{déclaration} \étymologie{\papi{sŋa.ru}}\end{déclaration}\end{définition}\end{entrée}

\begin{entrée}
\vedette{\hypertarget{Ⓔsŋi}{\papi{ sŋi}}}\markboth{sŋi}{}
\classe{n}
\begin{définition}\fra journée\end{définition}
\begin{définition}\cmn 白天\end{définition}\end{entrée}

\begin{entrée}
\vedette{\hypertarget{Ⓔsŋiɕɤr}{\papi{ sŋiɕɤr}}}\markboth{sŋiɕɤr}{}
\classe{n}
\begin{définition}\fra jour et nuit\end{définition}
\begin{définition}\cmn 白天黑夜\end{définition}\end{entrée}

\begin{entrée}
\vedette{\hypertarget{Ⓔsŋo}{\papi{ sŋo}}}\markboth{sŋo}{}
\classe{n}
\begin{définition}\fra rhododendron (dessous des feuilles orange)\end{définition}
\begin{définition}\cmn 羊角花(叶子背面黄色)\end{définition}\end{entrée}

\begin{entrée}
\vedette{\hypertarget{Ⓔsŋom}{\papi{ sŋom}}}\markboth{sŋom}{}
\classe{vi}
\paradigme{\textit{dir :} \jya nɯ-}
\begin{définition}\fra envier\end{définition}
\begin{définition}\cmn 贪;羡慕
\begin{déclaration} \étymologie{\papi{rŋam}}\end{déclaration}\end{définition}
\begin{exemple}\jya tɯrɟaʁ ɯ-kɯ-nɤmɲo ju-ɕe-nɯ ɲɯ-ŋu tɕe, aʑo ɲɯ-sŋom-a.\cmn 他们要去看舞蹈,我很羡慕(我去不成)\end{exemple}
\begin{relation-sémantique}\confer{
\hyperlink{Ⓔnɯsŋom}{\textit{ \papi{nɯsŋom}}}
}\end{relation-sémantique}
\begin{relation-sémantique}\confer{
\hyperlink{Ⓔrɟɯrŋom}{\textit{ \papi{rɟɯrŋom}}}
}\end{relation-sémantique}
\begin{relation-sémantique}\confer{\classe{vs}
\hyperlink{Ⓔnɯrɟɯrŋom}{\textit{ \papi{nɯrɟɯrŋom}}}
}\end{relation-sémantique}\begin{sous-entrée}
\vedette{\hypertarget{}{\papi{ sɤsŋom}}}\markboth{sɤsŋom}{}
\begin{définition}\ 
\begin{déclaration}\grammar{deexp}\end{déclaration}\end{définition}
\begin{définition}\fra donner envie\end{définition}
\begin{définition}\cmn 令人羡慕\end{définition}
\end{sous-entrée}\end{entrée}

\begin{entrée}
\vedette{\hypertarget{Ⓔsŋorma}{\papi{ sŋorma}}}\markboth{sŋorma}{}
\classe{n}
\begin{définition}\fra plantes qui poussent au printemps\end{définition}
\begin{définition}\cmn 春天发芽了的各种植物
\begin{déclaration} \étymologie{\papi{sŋo}}\end{déclaration}\end{définition}\end{entrée}

\begin{entrée}
\vedette{\hypertarget{Ⓔsŋoʁmɤr}{\papi{ sŋoʁmɤr}}}\markboth{sŋoʁmɤr}{}\classe{n}
\begin{définition}\fra beurre sur lequel un moine a soufflé\end{définition}
\begin{définition}\cmn 和尚念经后,在上面吹过的酥油
\begin{déclaration} \étymologie{\papi{mar}}\end{déclaration}\end{définition}
\end{entrée}

\begin{entrée}
\vedette{\hypertarget{Ⓔsŋur}{\papi{ sŋur}}}\markboth{sŋur}{}\classe{vi}
\paradigme{\textit{dir :} \jya tɤ-}
\begin{définition}\fra ronfler\end{définition}
\begin{définition}\cmn 打鼾
\begin{déclaration} \étymologie{\papi{sŋur}}\end{déclaration}\end{définition}
\begin{exemple}\jya nɤ-tɯ-sŋur nɯ\cmn 你鼾声打得很大声\end{exemple}
\begin{exemple}\jya jɯfɕɯɕɤr nɤʑo pɯ-tɯ-sŋur\cmn 你昨天晚上打了鼾\end{exemple}
\begin{exemple}\jya to-tɯ-sŋur\cmn 你原来不打鼾,现在就打鼾了\end{exemple}
\begin{exemple}\jya aʑo kɤ-nɯʑɯβ-a tɕe, wuma ɲɯ-sŋur-a\cmn 我睡的时候打鼾\end{exemple}\end{entrée}

\begin{entrée}
\vedette{\hypertarget{Ⓔsŋɯqiɯ}{\papi{ sŋɯqiɯ}}}\markboth{sŋɯqiɯ}{}\classe{n}
\begin{définition}\fra une demi journée\end{définition}
\begin{définition}\cmn 半天\end{définition}\end{entrée}

\begin{entrée}
\vedette{\hypertarget{Ⓔsŋɯχcɤl}{\papi{ sŋɯχcɤl}}}\markboth{sŋɯχcɤl}{}
\classe{n}
\begin{définition}\fra midi\end{définition}
\begin{définition}\cmn 中午\end{définition}\end{entrée}

\begin{entrée}
\vedette{\hypertarget{Ⓔsŋɯχcɤlpala}{\papi{ sŋɯχcɤlpala}}}\markboth{sŋɯχcɤlpala}{}\classe{n}
\begin{définition}\fra en plein jour\end{définition}
\begin{définition}\cmn 大白天\end{définition}
\end{entrée}

\begin{entrée}
\vedette{\hypertarget{Ⓔso}{\papi{ so}}}\markboth{so}{}\classe{vs}
\paradigme{\textit{dir :} \jya tɤ-}
\begin{définition}\fra vide\end{définition}
\begin{définition}\cmn 空\end{définition}
\begin{exemple}\jya ɲɯ-so\cmn 是空的\end{exemple}
\begin{exemple}\jya tɯji kɯ-so tu\cmn 有空着的地(没有种庄稼的)\end{exemple}
\begin{exemple}\jya kha kɯ-so ɕti\cmn 房间是空的\end{exemple}
\begin{relation-sémantique}\confer{
\hyperlink{Ⓔɯ-xso}{\textit{ \papi{ɯ-xso}}}
}\end{relation-sémantique}
\begin{relation-sémantique}\confer{
\hyperlink{Ⓔnɯxso}{\textit{ \papi{nɯxso}}}
}\end{relation-sémantique}\begin{sous-entrée}
\vedette{\hypertarget{}{\papi{ sɯxso}}}\markboth{sɯxso}{}\classe{vt}
\paradigme{\textit{dir :} \jya pɯ-}
\paradigme{\textit{dir :} \jya thɯ-}
\begin{définition}\fra vider\end{définition}
\begin{définition}\cmn 掏空\end{définition}
\begin{exemple}\jya tɤ-fkɯm pɯ-sɯxsɤm\cmn 你把口袋腾空\end{exemple}
\end{sous-entrée}\end{entrée}

\begin{entrée}
\vedette{\hypertarget{Ⓔsomo khɯtsa}{\papi{ somo khɯtsa}}}\markboth{somo khɯtsa}{}\classe{n}
\begin{définition}\fra bol en bois\end{définition}
\begin{définition}\cmn 木碗\end{définition}\end{entrée}

\begin{entrée}
\vedette{\hypertarget{Ⓔsoʁdɤr}{\papi{ soʁdɤr}}}\markboth{soʁdɤr}{}
\classe{n}
\begin{définition}\fra rabot\end{définition}
\begin{définition}\cmn 锉刀\end{définition}\end{entrée}

\begin{entrée}
\vedette{\hypertarget{Ⓔsoʁma}{\papi{ soʁma}}}\markboth{soʁma}{}\classe{n}
\begin{définition}\fra paille de blé\end{définition}
\begin{définition}\cmn 麦秸
\begin{déclaration} \étymologie{\papi{sog.ma}}\end{déclaration}\end{définition}
\end{entrée}

\begin{entrée}
\vedette{\hypertarget{Ⓔsoskɯsku}{\papi{ soskɯsku}}}\markboth{soskɯsku}{} (\variante{soskuku}) 
\classe{adv}
\begin{définition}\fra tous les matins\end{définition}
\begin{définition}\cmn 每天早上\end{définition}\end{entrée}

\begin{entrée}
\vedette{\hypertarget{Ⓔsoχpu}{\papi{ soχpu}}}\markboth{soχpu}{}\classe{n}
\begin{définition}\fra mongol\end{définition}
\begin{définition}\cmn 蒙古人
\begin{déclaration} \étymologie{\papi{sog.po}}\end{déclaration}\end{définition}
\end{entrée}

\begin{entrée}
\vedette{\hypertarget{Ⓔsoz}{\papi{ soz}}}\markboth{soz}{}\classe{adv}
\begin{définition}\fra le matin\end{définition}
\begin{définition}\cmn 早上\end{définition}\end{entrée}

\begin{entrée}
\vedette{\hypertarget{Ⓔsozdɯmtɕi}{\papi{ sozdɯmtɕi}}}\markboth{sozdɯmtɕi}{} (\variante{sostɯmtɕi}) 
\classe{n}
\begin{définition}\fra au point du jour\end{définition}
\begin{définition}\cmn 清早\end{définition}\end{entrée}

\begin{entrée}
\vedette{\hypertarget{Ⓔspa}{\papi{ spa}}}\markboth{spa}{}\classe{vt}
\paradigme{\textit{dir :} \jya kɤ-}
\begin{définition}\fra pouvoir\end{définition}
\begin{définition}\cmn 会
\begin{déclaration}\use{“会”的否定形式带有“做对不起人的事情”的意思}\end{déclaration}\end{définition}
\begin{exemple}\jya ɕoŋβzu ko-spa\cmn 他会木工了\end{exemple}
\begin{exemple}\jya kɤ-rɤrɤt ko-spa\cmn 他会写字了\end{exemple}
\begin{exemple}\jya ɯ-kɤ-spa ɲɯ-dɤn\cmn 他会做的事情很多\end{exemple}
\begin{exemple}\jya tɤ-rʑaʁ pɯ-zri ri, pɤjkhu mɯ́j-spe-a\cmn 过了很久,我还是不会\end{exemple}
\begin{exemple}\jya ɯʑo kɯ spe tɕeri, ɯʑo ɯ-ʁa khro maŋe wo\cmn 她会做,但是她没有空\end{exemple}
\begin{exemple}\jya "a-wa" tu-ti ɯ́-spe, "a-mu tu-ti" ra ɯ́-spe?\cmn (你儿子现在)会不会说“爸爸、妈妈”?\end{exemple}
\begin{exemple}\jya mɯ-tɤ-spa-t-a ri me, pɯ-rɯɣnan-a ri me\cmn 我没有做对不起(你)的事情,也没有跟你作对\end{exemple}
\begin{exemple}\jya nɤʑɯɣ mɯ-ta-spa ri me, nɤʑɯɣ pɯ-rɯɣnɤn ri me\cmn 他没有做对不起你的事情,也没有跟你作对\end{exemple}
\begin{exemple}\jya kɤ-rɯkhɤcɤl a-kɤ-spe-a ɲɯ-ra\cmn 我要学会(用藏语)讲话\end{exemple}
\begin{exemple}\jya nɤʑo tɤ-rʑaʁ thɤstɯɣ jamar kɤ-βzjoz kɤ-tɯ-spa-t?\cmn 你学了多久才学会了?\end{exemple}
\begin{exemple}\jya ɯʑo kɤ-rɯcɤβŋgɤβ mɤ-spe\cmn 他不会骄傲自大(没有那个性格)\end{exemple}\begin{sous-entrée}
\vedette{\hypertarget{}{\papi{ nɯspɯspa}}}\markboth{nɯspɯspa}{}\classe{vt}
\paradigme{\textit{dir :} \jya pɯ-}
\begin{exemple}\jya pɯ-nɯspɯspa-t-a ɕti\cmn 我本来就会了\end{exemple}
\begin{exemple}\jya @qiche kɤ-lɤt pa-nɯspɯspa ɕti\cmn 他本来就会开车\end{exemple}
\end{sous-entrée}\begin{sous-entrée}
\vedette{\hypertarget{}{\papi{ sɤspa}}}\markboth{sɤspa}{}\classe{vs}
\begin{définition}\ 
\begin{déclaration}\grammar{deexp}\end{déclaration}\end{définition}
\begin{définition}\fra que l'on connait\end{définition}
\begin{définition}\cmn (人)会做的,知道的\end{définition}
\begin{exemple}\jya mɤʑɯ ɯ-rmi mɤ-kɯ-sɤspa xcat ɕti\cmn 还有很多不知道名字的\end{exemple}
\end{sous-entrée}\begin{sous-entrée}
\vedette{\hypertarget{}{\papi{ sɯspa}}}\markboth{sɯspa}{}\classe{vt}
\paradigme{\textit{dir :} \jya kɤ-}
\begin{définition}\ 
\begin{déclaration}\grammar{caus}\end{déclaration}\end{définition}
\begin{définition}\fra rendre capable\end{définition}
\begin{définition}\cmn 令别人学会\end{définition}
\begin{exemple}\jya kɤ-sɯspa-t-a\cmn 我令他学会了\end{exemple}
\end{sous-entrée}\end{entrée}

\begin{entrée}
\vedette{\hypertarget{Ⓔspa,rka}{\papi{ spa,rka}}}\markboth{spa,rka}{}\paradigme{\textit{dir :} \jya tɤ-}
\begin{définition}\fra être innocent\end{définition}
\begin{définition}\cmn 无辜
\begin{déclaration}\use{\stylefv{spa}和\stylefv{rka}只用于否定形式,带\stylefv{mɤ-}前缀,同时必须与否定存在动词\stylefv{me}“没有”连用}\end{déclaration}\end{définition}
\begin{exemple}\jya mɤ-spe mɤ-rke kɯ-me\cmn 无辜的人\end{exemple}
\begin{exemple}\jya mkhɤrmaŋ mɤ-spa-nɯ mɤ-rka-nɯ kɯ-me\cmn 无辜的老百姓\end{exemple}
\begin{relation-sémantique}\ComponentA{\classe{vt}
\hyperlink{Ⓔspa}{\textit{ \papi{spa}}}
}\end{relation-sémantique}
\begin{relation-sémantique}\ComponentB{\classe{vt}
 \papi{rka}
}\end{relation-sémantique}\begin{forme-mot}1s : \papi{mɤ-spe-a mɤ-rke-a me}\end{forme-mot}\begin{forme-mot}2s : \papi{mɤ-tɯ-spe mɤ-tɯ-rke me}\end{forme-mot}\begin{forme-mot}2s : \papi{mɤ-tɯ-sperke me}\end{forme-mot}\begin{forme-mot}2s : \papi{mɯ-tɤ-tɯ-spa-t-rka-t me}\end{forme-mot}\end{entrée}

\begin{entrée}
\vedette{\hypertarget{Ⓔspɤr}{\papi{ spɤr}}}\markboth{spɤr}{}
\classe{vt}
\paradigme{\textit{dir :} \jya \_}
\begin{définition}\fra changer de lieu de résidence, déménager\end{définition}
\begin{définition}\cmn 搬迁
\begin{déclaration}\use{一般指游牧民族搬帐篷,很少用于“搬家”的意思}\end{déclaration}\end{définition}
\begin{exemple}\jya rɯ na-spɤr-nɯ\cmn 他们搬帐篷了\end{exemple}\begin{sous-entrée}
\vedette{\hypertarget{}{\papi{ rɤspɤr}}}\markboth{rɤspɤr}{}\classe{vi}
\begin{définition}\ 
\begin{déclaration}\grammar{apass}\end{déclaration}\end{définition}
\begin{définition}\fra changer de lieu de résidence\end{définition}
\begin{définition}\cmn 搬迁\end{définition}
\begin{relation-sémantique}\synonyme{
\hyperlink{Ⓔcit}{\textit{ \papi{cit}}}
}\end{relation-sémantique}
\end{sous-entrée}\end{entrée}

\begin{entrée}
\vedette{\hypertarget{Ⓔspɤrmbɯt}{\papi{ spɤrmbɯt}}}\markboth{spɤrmbɯt}{}
\classe{adv}
\begin{définition}\fra partir tout d'un coup\end{définition}
\begin{définition}\cmn 突然就走\end{définition}
\begin{exemple}\jya spɤrmbɯt jo-ɕe\cmn 他突然就走了\end{exemple}
\begin{relation-sémantique}\confer{
\hyperlink{Ⓔspɤr}{\textit{ \papi{spɤr}}}
}\end{relation-sémantique}\end{entrée}

\begin{entrée}
\vedette{\hypertarget{Ⓔspɤt}{\papi{ spɤt}}}\markboth{spɤt}{}\classe{vi}
\paradigme{\textit{dir :} \jya \_}
\begin{définition}\fra se déchirer, s'ouvrir jusqu'au bord (après qu'on ait tiré dessus)\end{définition}
\begin{définition}\cmn (一拉就)扯断、拉破\end{définition}
\begin{exemple}\jya tɯ-ŋga mɯ́j-ngɯt tɕe, jú-wɣ-rɤɕi tɕe ju-spɤt ɲɯ-ɕti\cmn 衣服不结实,一扯就会拉破\end{exemple}
\begin{exemple}\jya ɯ-mtsɯ pjɤ-spɤt\cmn 扣子的眼子拉破了\end{exemple}\begin{sous-entrée}
\vedette{\hypertarget{}{\papi{ sɯspɤt}}}\markboth{sɯspɤt}{}\classe{vt}
\paradigme{\textit{dir :} \jya \_}
\begin{définition}\fra tirer en mordant ou en déchirant\end{définition}
\begin{définition}\cmn 扯断;一边拉一边咬断\end{définition}
\begin{exemple}\jya tɯ-ŋga pɯ-kɯ-ɴɢraʁ nɯ nɯ-sɯspat-a\cmn 我把破烂的衣服扯断了\end{exemple}
\end{sous-entrée}\end{entrée}

\begin{entrée}
\vedette{\hypertarget{Ⓔspɣɤnthar}{\papi{ spɣɤnthar}}}\markboth{spɣɤnthar}{} (\variante{spɣɤnthɣar}) \classe{n}
\begin{définition}\fra chiquenaude\end{définition}
\begin{définition}\cmn 弹手指\end{définition}
\begin{exemple}\jya spɣɤnthɣar tɤ-lat-a\cmn 我弹了手指\end{exemple}\end{entrée}

\begin{entrée}
\vedette{\hypertarget{Ⓔspɣi}{\papi{ spɣi}}}\markboth{spɣi}{}
\classe{n}
\begin{définition}\fra grenier\end{définition}
\begin{définition}\cmn 用木头建造的仓库\end{définition}\end{entrée}

\begin{entrée}
\vedette{\hypertarget{Ⓔspɣɯthoʁ}{\papi{ spɣɯthoʁ}}}\markboth{spɣɯthoʁ}{}
\classe{n}
\begin{définition}\fra sol du grenier\end{définition}
\begin{définition}\cmn 仓库的地面\end{définition}\end{entrée}

\begin{entrée}
\vedette{\hypertarget{Ⓔsphjar}{\papi{ sphjar}}}\markboth{sphjar}{}
\classe{vt}
\paradigme{\textit{dir :} \jya tɤ-}
\begin{définition}\fra étendre pour sécher\end{définition}
\begin{définition}\cmn 展开(晾干)\end{définition}
\begin{exemple}\jya a-ŋga ɲɯ-ɤci tɕe tɤ-sphjar-a\cmn 我衣服是湿的,所以我把它展开了(晾干)\end{exemple}
\begin{exemple}\jya tɯ-ndʐi pɯ-tɯ-qar tɕe tɤ-sphjar\cmn 你剥了皮就把它展开\end{exemple}
\begin{relation-sémantique}\synonyme{
\hyperlink{Ⓔsqhiar}{\textit{ \papi{sqhiar}}}
}\end{relation-sémantique}\end{entrée}

\begin{entrée}
\vedette{\hypertarget{Ⓔsphjaʁ}{\papi{ sphjaʁ}}}\markboth{sphjaʁ}{}\classe{vt}
\paradigme{\textit{dir :} \jya kɤ-}
\begin{définition}\fra mouiller, s’infiltrer complètement\end{définition}
\begin{définition}\cmn 浸透;烫透;冷透\end{définition}
\begin{exemple}\jya tɯ-mɯ kɤ-lɤt tɕe, kɤ́-wɣ-sphjaʁ-a\cmn 下雨了,我就被雨水淋湿了\end{exemple}
\begin{exemple}\jya tɯ-ci kɯ kɤ́-wɣ-sphjaʁ-a\cmn 我被淋湿了\end{exemple}\begin{sous-entrée}
\vedette{\hypertarget{}{\papi{ sɯsphjaʁ}}}\markboth{sɯsphjaʁ}{}\classe{vt}
\begin{définition}\ 
\begin{déclaration}\grammar{habil}\end{déclaration}\end{définition}
\begin{définition}\fra traverser\end{définition}
\begin{définition}\cmn 穿透\end{définition}
\begin{exemple}\jya tɤtʂu ɯ-tɯ-fsoʁ kɯ tɯ-ŋga ra ku-sɯsphjaʁ ɲɯ-ɕti\cmn 灯光穿透衣服\end{exemple}
\begin{exemple}\jya tɯ-rju kɤ-sɯsphjaʁ mɯ-pɯ-cha-a\cmn 我说了一句,但他们没有理睬我\end{exemple}
\end{sous-entrée}\end{entrée}

\begin{entrée}
\vedette{\hypertarget{Ⓔsphɯt}{\papi{ sphɯt}}}\markboth{sphɯt}{}
\begin{relation-sémantique}\confer{
\hyperlink{Ⓔphɯt}{\textit{ \papi{phɯt}}}
}\end{relation-sémantique}\end{entrée}

\begin{entrée}
\vedette{\hypertarget{Ⓔspjaŋkɯ}{\papi{ spjaŋkɯ}}}\markboth{spjaŋkɯ}{}\classe{n}
\begin{définition}\fra loup\end{définition}
\begin{définition}\cmn 狼
\begin{déclaration} \étymologie{\papi{spʲaŋ.ki}}\end{déclaration}\end{définition}
\end{entrée}

\begin{entrée}
\vedette{\hypertarget{Ⓔspjɤt}{\papi{ spjɤt}}}\markboth{spjɤt}{}\classe{vt}
\paradigme{\textit{dir :} \jya tɤ-}
\begin{définition}\fra montrer, faire une démonstration (de ses capacités)\end{définition}
\begin{définition}\cmn 展现出来,显示(自己的本领)
\begin{déclaration} \étymologie{\papi{spʲod}}\end{déclaration}\end{définition}
\begin{exemple}\jya tɤ-spjat-a\cmn 我显示了\end{exemple}
\begin{exemple}\jya ɯ-kɤ-ro kɯ-tu ra ta-nɯ-spjɤt\cmn 他把他拥有的东西展现出来了\end{exemple}
\begin{exemple}\jya nɤ-kɤ-cha tɤ-nɯ-spjɤt\cmn 你显示一下你的能力\end{exemple}\end{entrée}

\begin{entrée}
\vedette{\hypertarget{Ⓔspjɤtɕha}{\papi{ spjɤtɕha}}}\markboth{spjɤtɕha}{}
\classe{n}
\begin{définition}\fra action\end{définition}
\begin{définition}\cmn 行为
\begin{déclaration} \étymologie{\papi{spʲod.tɕʰa?}}\end{déclaration}\end{définition}
\begin{exemple}\jya ki nɤʑo nɤ-spjɤtɕha ŋu\cmn 这是你搞的鬼\end{exemple}
\begin{exemple}\jya nɤ-spjɤtɕha ɯ-tɯ-ŋɤn\cmn 你的行为很恶毒!\end{exemple}
\begin{relation-sémantique}\confer{
\hyperlink{Ⓔnɯspjɤtɕha}{\textit{ \papi{nɯspjɤtɕha}}}
}\end{relation-sémantique}\end{entrée}

\begin{entrée}
\vedette{\hypertarget{Ⓔspoŋ}{\papi{ spoŋ}}}\markboth{spoŋ}{}\classe{vi}
\paradigme{\textit{dir :} \jya \_}
\begin{définition}\fra se réfugier dans un autre pays\end{définition}
\begin{définition}\cmn 逃亡他乡\end{définition}\end{entrée}

\begin{entrée}
\vedette{\hypertarget{Ⓔspoŋspoz}{\papi{ spoŋspoz}}}\markboth{spoŋspoz}{}
\classe{n}
\begin{définition}\fra une plante\end{définition}
\begin{définition}\cmn 一种植物\end{définition}
\begin{exemple}\jya spoŋspoz nɯ sɤtɕha kɯ-mbɤr tsa pɕoʁ tu-ɬoʁ ɲɯ-ŋu, ɯ-zrɤm ɲɯ-dɤn, ɲɯ-zri, ɯ-ku ɯ-jwaʁ nɯ ɲɯ-pɣi tɕe ɯ-rme kɯ-fse ɣɤʑu, ɯ-jwaʁ nɯ ɯ-qa ri ɲɤ-ɬoʁ tɕe ɯ-thoʁ pjɯ-ɤɲɟoʁ kɯ-fse ɲɯ-ŋu. ɯ-ru tu-ɬoʁ tɕe, ɯ-kɤχcɤl tɕe ɲɯ-rɯmɯntoʁ ɲɯ-ŋu. ɯ-mɯntoʁ hanɯni ɲɯ-ɣɯrni, ɯ-mɯntoʁ ɯ-rqhu kɤntɕhɯ-tɤlɤβ ɣɤʑu, ɯ-mɯntoʁ tɯ-rdoʁ ma maŋe, ɯ-rɣi ɣɤʑu ri tu-ɬoʁ mɯ́j-cha, ɯ-zrɤm ɯ-taʁ ɲɯ-mphɤl ɲɯ-ɕti. tɯ-xpa tu-ɬoʁ tɕe pjɯ-khrɯ ɲɯ-ɕti, ki sɯjno ki ɯ-di wuma ɲɯ-mɯm, smɤn ɲɯ-sna khi.\cmn 
\stylefv{spoŋspoz} 生长在海拔低的地方(半山以下),根又多又长,苗、叶子的是灰色的,有毛。叶子长在根部上,好像贴在地面上,茎长出来以后,顶端上开花。花有点红,花萼有几层,只有一朵花,虽然有种子但是不能发芽,是靠它的根繁殖的。长了一年就会干枯。这种草香味很浓,据说可以入药。
\end{exemple}\end{entrée}

\begin{entrée}
\vedette{\hypertarget{Ⓔspoŋsrɤm}{\papi{ spoŋsrɤm}}}\markboth{spoŋsrɤm}{}\classe{n}
\begin{définition}\fra mammifère non identifié\end{définition}
\begin{définition}\cmn 一种动物
\begin{déclaration} \étymologie{\papi{spaŋ.sram}}\end{déclaration}\end{définition}
\begin{exemple}\jya spoŋsrɤm nɯ rɯdaʁ ruŋgu kɯ-mbro ku-kɯ-rɤʑi ci ŋu tɕe ɯ-rme nɯ wuma ʑo mpɕɤr tɕe tɕhɯɕrɤm cho kɯ-naχtɕɯɣ ŋu\cmn 
\stylefv{spoŋsrɤm}是生活在高山草地上的动物,它的毛长得很美,和水獭一样的贵重。
\end{exemple}
\end{entrée}

\begin{entrée}
\vedette{\hypertarget{Ⓔsporɟɤlɯla}{\papi{ sporɟɤlɯla}}}\markboth{sporɟɤlɯla}{}
\classe{n}
\begin{définition}\fra orvet\end{définition}
\begin{définition}\cmn 玻璃蛇\end{définition}
\begin{exemple}\jya sporɟɤlɯla nɯ qapri kɯ-fse ci ŋu ri kɯ-xtɯt kɯ-xtshɯm ci ŋu. ɯ-βri ɯ-mdoʁ nɯ ra qapri fsɯ-fse ʑo fse, rdɤstaʁ ɯ-rchɤβ, praʁ ɯ-rchɤβ ra ku-rɤʑi ŋu. ɯ-βri nɯ mbju ʑo ku-sɤmtsɯɣ kɤ-mtshɤm me ri ɯ-kɯ-nɯɣmu wuma dɤn. kɯ-sɤmtshɤr ci tu tɕe, tɯ-jaχpa ɣɯ ɯ-βzɯr ``hu" tu-kɯ-ti tɕe ɯ-taʁ pjɯ́-wɣ-lɤt tɕe, tɯ-jaʁ pjɯ-tɯɣ ʑo ma mɤ-ra ma tɕe ɲɯ-mbrɤt ɕti. nɯ-mbrɤt tɕe, ɯ-rtshɯm nɯ pɤjkhu tu-nɤrɟɯrɟɯɣ ɕti.\cmn 玻璃蛇像蛇一样,但比较短,也比较细。身子的颜色和蛇完全相同,生活石头缝和岩石缝里。身体有光泽。没有听说过咬人,但很多人怕它。奇怪的是,人们把手掌的边缘吹一下,然后打在它身上,只要轻轻地碰一下它就会断裂。断裂后的那一节还能继续跑。\end{exemple}\end{entrée}

\begin{entrée}
\vedette{\hypertarget{Ⓔspoʁ}{\papi{ spoʁ}}}\markboth{spoʁ}{}\classe{vi}
\paradigme{\textit{dir :} \jya nɯ-}
\begin{définition}\fra être troué\end{définition}
\begin{définition}\cmn 有洞\end{définition}
\begin{exemple}\jya a-khɯtsa ɲɤ-spoʁ\cmn 我的碗有洞\end{exemple}
\begin{relation-sémantique}\confer{
\hyperlink{Ⓔkɯspoʁ}{\textit{ \papi{kɯspoʁ}}}
}\end{relation-sémantique}\end{entrée}

\begin{entrée}
\vedette{\hypertarget{Ⓔspoz}{\papi{ spoz}}}\markboth{spoz}{}
\classe{n}
\begin{définition}\fra encens\end{définition}
\begin{définition}\cmn 香
\begin{déclaration} \étymologie{\papi{spos}}\end{déclaration}\end{définition}\end{entrée}

\begin{entrée}
\vedette{\hypertarget{Ⓔsprɤt}{\papi{ sprɤt}}}\markboth{sprɤt}{}\classe{vt}
\paradigme{\textit{dir :} \jya tɤ-}\acception{1}
\begin{définition}\fra installer\end{définition}
\begin{définition}\cmn 安装
\begin{déclaration}\use{把零件拼起来}\end{déclaration}\end{définition}
\begin{exemple}\jya mkhɯrlu (jiqi) to-sprɤt\cmn 他安装了机器\end{exemple}
\begin{exemple}\jya ɕoŋβzu kɯ to-sprɤt\cmn 木匠安装了\end{exemple}\acception{2}
\begin{définition}\fra remettre entre les mains de\end{définition}
\begin{définition}\cmn 上缴
\begin{déclaration} \étymologie{\papi{sprod}}\end{déclaration}\end{définition}
\begin{exemple}\jya aʑo kɯ-mɯrkɯ pɯ-mto-t-a tɕe, ɯ-taʁ ra nɯ-jaʁ tɤ-sprat-a\cmn 我看到小偷,移交给领导了\end{exemple}\begin{sous-entrée}
\vedette{\hypertarget{}{\papi{ ʑɣɤsprɤt}}}\markboth{ʑɣɤsprɤt}{}\classe{vi}
\paradigme{\textit{dir :} \jya tɤ-}
\begin{définition}\ 
\begin{déclaration}\grammar{refl}\end{déclaration}\end{définition}
\begin{définition}\fra se livrer (à la justice)\end{définition}
\begin{définition}\cmn 把自己交给上级,自首\end{définition}
\begin{exemple}\jya aʑo tɤ-ʑɣɤsprat-a\cmn 我自首了\end{exemple}
\end{sous-entrée}\end{entrée}

\begin{entrée}
\vedette{\hypertarget{Ⓔsprilu}{\papi{ sprilu}}}\markboth{sprilu}{}\classe{n}
\begin{définition}\fra année du singe\end{définition}
\begin{définition}\cmn 猴年
\begin{déclaration} \étymologie{\papi{spreɦu.lo}}\end{déclaration}\end{définition}\end{entrée}

\begin{entrée}
\vedette{\hypertarget{Ⓔsprɯlpa}{\papi{ sprɯlpa}}}\markboth{sprɯlpa}{}
\classe{n}
\begin{définition}\fra magie\end{définition}
\begin{définition}\cmn 法术
\begin{déclaration} \étymologie{\papi{sprul.pa}}\end{déclaration}\end{définition}\end{entrée}

\begin{entrée}
\vedette{\hypertarget{Ⓔsprɯskɯ}{\papi{ sprɯskɯ}}}\markboth{sprɯskɯ}{}\classe{n}
\begin{définition}\fra sprulsku\end{définition}
\begin{définition}\cmn 活佛
\begin{déclaration} \étymologie{\papi{sprul.sku}}\end{déclaration}\end{définition}\end{entrée}

\begin{entrée}
\vedette{\hypertarget{Ⓔspɯ}{\papi{ spɯ}}}\markboth{spɯ}{}\classe{vs}
\paradigme{\textit{dir :} \jya pɯ-}
\begin{définition}\fra sec\end{définition}
\begin{définition}\cmn 水分少;干(面、糌粑、稀泥)\end{définition}
\begin{exemple}\jya rɟɤɣi pjɤ-spɯ\cmn 糌粑水分偏少,吃起来不要干燥\end{exemple}
\begin{relation-sémantique}\antonyme{
\hyperlink{Ⓔŋgri}{\textit{ \papi{ŋgri}}}
}\end{relation-sémantique}\begin{sous-entrée}
\vedette{\hypertarget{}{\papi{ ɣɤspɯ}}}\markboth{ɣɤspɯ}{}\classe{vt}
\paradigme{\textit{dir :} \jya pɯ-}
\begin{définition}\ 
\begin{déclaration}\grammar{caus}\end{déclaration}\end{définition}
\begin{définition}\fra rendre sec\end{définition}
\begin{définition}\cmn 弄干\end{définition}
\begin{exemple}\jya a-rɟɤɣi ɯ-ŋgɯ tɯ-ci pjɤ-ɣɤrkɯn-a tɕe, pjɤ-ɣɤspɯ-t-a\cmn 我减少了糌粑里的水分,令它很干\end{exemple}
\end{sous-entrée}\end{entrée}

\begin{entrée}
\vedette{\hypertarget{Ⓔspɯrtɯm}{\papi{ spɯrtɯm}}}\markboth{spɯrtɯm}{}
\classe{n}
\begin{définition}\fra une espèce de champignon\end{définition}
\begin{définition}\cmn 一种菌\end{définition}
\begin{exemple}\jya spɯrtɯm nɯ tɯrgi ɯ-ŋgɯ tu-ɬoʁ ŋu, ɯ-mdoʁ nɯ kɯ-qandʐi ŋu, ɯ-tshɯɣa nɯ tɯrgi grɯβgrɯβ cho naχtɕɯɣ, kɯ-rko tsa ŋu ɯ-ru nɯ mɤ-ndoʁ, kɤ-ndza sna\cmn 
\stylefv{spɯrtɯm} 长在杉木林里,颜色是浅黑色,形状和杉木菌一样,比较硬,干而不脆,可以吃。
\end{exemple}\end{entrée}

\begin{entrée}
\vedette{\hypertarget{Ⓔsqa}{\papi{ sqa}}}\markboth{sqa}{}
\classe{vt}
\paradigme{\textit{dir :} \jya kɤ-}
\paradigme{\textit{dir :} \jya pɯ-}
\begin{définition}\fra cuire\end{définition}
\begin{définition}\cmn 炖;煮\end{définition}
\begin{exemple}\jya tɤ-mthɯm ka-sqa\cmn 他煮了肉\end{exemple}
\begin{exemple}\jya paʁndza ka-sqa\cmn 他煮了猪食\end{exemple}
\begin{exemple}\jya kɤ-sqa-j\cmn 我们煮了\end{exemple}
\begin{exemple}\jya tɯ-mɯ kɯ kɤ́-wɣ-sqa ʑo\cmn 被雨淋湿了\end{exemple}\end{entrée}

\begin{entrée}
\vedette{\hypertarget{Ⓔsqaβde}{\papi{ sqaβde}}}\markboth{sqaβde}{}\classe{num}
\begin{définition}\fra quatorze\end{définition}
\begin{définition}\cmn 十四\end{définition}\end{entrée}

\begin{entrée}
\vedette{\hypertarget{Ⓔsqaβjɯβ}{\papi{ sqaβjɯβ}}}\markboth{sqaβjɯβ}{}\classe{vt}
\paradigme{\textit{dir :} \jya tɤ-}
\begin{définition}\fra cacher, couvrir\end{définition}
\begin{définition}\cmn 遮掩,遮住\end{définition}
\begin{exemple}\jya a-mɲaʁ tɤ-sqaβjɯβ-a ma ɲɯ-nɤmbju\cmn 因为很亮,我遮住了眼睛\end{exemple}
\begin{relation-sémantique}\synonyme{
\hyperlink{Ⓔsaʁjɯβ}{\textit{ \papi{saʁjɯβ}}}
}\end{relation-sémantique}\end{entrée}

\begin{entrée}
\vedette{\hypertarget{Ⓔsqaɕnɯz}{\papi{ sqaɕnɯz}}}\markboth{sqaɕnɯz}{}\classe{num}
\begin{définition}\fra dix-sept\end{définition}
\begin{définition}\cmn 十七\end{définition}\end{entrée}

\begin{entrée}
\vedette{\hypertarget{Ⓔsqafsum}{\papi{ sqafsum}}}\markboth{sqafsum}{}\classe{num}
\begin{définition}\fra treize\end{définition}
\begin{définition}\cmn 十三\end{définition}\end{entrée}

\begin{entrée}
\vedette{\hypertarget{Ⓔsqamnɯz}{\papi{ sqamnɯz}}}\markboth{sqamnɯz}{}\classe{num}
\begin{définition}\fra douze\end{définition}
\begin{définition}\cmn 十二\end{définition}\end{entrée}

\begin{entrée}
\vedette{\hypertarget{Ⓔsqamŋu}{\papi{ sqamŋu}}}\markboth{sqamŋu}{}\classe{num}
\begin{définition}\fra quinze\end{définition}
\begin{définition}\cmn 十五\end{définition}\end{entrée}

\begin{entrée}
\vedette{\hypertarget{Ⓔsqandʐi}{\papi{ sqandʐi}}}\markboth{sqandʐi}{}
\begin{relation-sémantique}\confer{
\hyperlink{ⒺqandʐiⒽ1}{\textit{ \papi{qandʐi1}}}
}\end{relation-sémantique}\end{entrée}

\begin{entrée}
\vedette{\hypertarget{Ⓔsqane}{\papi{ sqane}}}\markboth{sqane}{}\classe{vt}
\paradigme{\textit{dir :} \jya kɤ-}
\paradigme{\textit{dir :} \jya tɤ-}
\begin{définition}\fra couvrir, plonger dans l'obscurité, éteindre (lumière)\end{définition}
\begin{définition}\cmn 遮光;关灯\end{définition}
\begin{relation-sémantique}\confer{
\hyperlink{Ⓔsqaʁjɯβ}{\textit{ \papi{sqaʁjɯβ}}}
}\end{relation-sémantique}
\begin{relation-sémantique}\confer{
\hyperlink{Ⓔsqanɯ}{\textit{ \papi{sqanɯ}}}
}\end{relation-sémantique}\end{entrée}

\begin{entrée}
\vedette{\hypertarget{Ⓔsqangɯt}{\papi{ sqangɯt}}}\markboth{sqangɯt}{}\classe{num}
\begin{définition}\fra dix-neuf\end{définition}
\begin{définition}\cmn 十九\end{définition}\end{entrée}

\begin{entrée}
\vedette{\hypertarget{Ⓔsqanɯ}{\papi{ sqanɯ}}}\markboth{sqanɯ}{}\classe{vt}
\paradigme{\textit{dir :} \jya kɤ-}
\begin{définition}\ 
\begin{déclaration}\grammar{caus}\end{déclaration}\end{définition}
\begin{définition}\fra plonger dans l'obscurité\end{définition}
\begin{définition}\cmn 遮光\end{définition}
\begin{exemple}\jya qale ta-βzu tɕe, rdɯl kɯ ka-sqanɯ ʑo\cmn 风吹得尘土满天飞扬,遮住了阳光\end{exemple}
\begin{exemple}\jya tɤ-khɯ kɯ ka-sqanɯ ʑo\cmn 浓烟遮住了阳光\end{exemple}
\begin{relation-sémantique}\confer{
\hyperlink{Ⓔqanɯ}{\textit{ \papi{qanɯ}}}
}\end{relation-sémantique}\end{entrée}

\begin{entrée}
\vedette{\hypertarget{Ⓔsqaprɤɣ}{\papi{ sqaprɤɣ}}}\markboth{sqaprɤɣ}{}\classe{num}
\begin{définition}\fra seize\end{définition}
\begin{définition}\cmn 十六\end{définition}\end{entrée}

\begin{entrée}
\vedette{\hypertarget{Ⓔsqaptɯɣ}{\papi{ sqaptɯɣ}}}\markboth{sqaptɯɣ}{}\classe{num}
\begin{définition}\fra onze\end{définition}
\begin{définition}\cmn 十一\end{définition}\end{entrée}

\begin{entrée}
\vedette{\hypertarget{Ⓔsqapɯ}{\papi{ sqapɯ}}}\markboth{sqapɯ}{}
\begin{relation-sémantique}\confer{
\hyperlink{Ⓔqapɯ}{\textit{ \papi{qapɯ}}}
}\end{relation-sémantique}\end{entrée}

\begin{entrée}
\vedette{\hypertarget{Ⓔsqarcat}{\papi{ sqarcat}}}\markboth{sqarcat}{}\classe{num}
\begin{définition}\fra dix-huit\end{définition}
\begin{définition}\cmn 十八\end{définition}\end{entrée}

\begin{entrée}
\vedette{\hypertarget{Ⓔsqarcɯm}{\papi{ sqarcɯm}}}\markboth{sqarcɯm}{}
\classe{vt}
\paradigme{\textit{dir :} \jya pɯ-}
\paradigme{\textit{dir :} \jya thɯ-}
\begin{définition}\fra froncer les sourcils\end{définition}
\begin{définition}\cmn 皱着眉毛,表现出不高兴的表情\end{définition}
\begin{exemple}\jya ɯʑo kɯ ɯ-rŋa pjɤ-sqarcɯm\cmn 他皱了眉头\end{exemple}
\begin{relation-sémantique}\confer{
\hyperlink{Ⓔqarcɯm}{\textit{ \papi{qarcɯm}}}
}\end{relation-sémantique}\end{entrée}

\begin{entrée}
\vedette{\hypertarget{Ⓔsqarndɯm}{\papi{ sqarndɯm}}}\markboth{sqarndɯm}{}
\begin{relation-sémantique}\confer{
\hyperlink{Ⓔqarndɯm}{\textit{ \papi{qarndɯm}}}
}\end{relation-sémantique}\end{entrée}

\begin{entrée}
\vedette{\hypertarget{Ⓔsqaʁjɯβ}{\papi{ sqaʁjɯβ}}}\markboth{sqaʁjɯβ}{}
\classe{vt}
\paradigme{\textit{dir :} \jya tɤ-}
\begin{définition}\fra couvrir (un côté)\end{définition}
\begin{définition}\cmn 遮住(一面);挡住\end{définition}
\begin{exemple}\jya tɤ́-wɣ-sqaʁjɯβ-a\cmn 把我挡住了\end{exemple}
\begin{exemple}\jya si kɯ tu-sqaʁjɯβ ɲɯ-ŋu\cmn 被树遮住\end{exemple}
\begin{exemple}\jya zgo kɯ tu-sqaʁjɯβ ɲɯ-ŋu\cmn 被山遮住\end{exemple}
\begin{exemple}\jya tɤ-kɯ-sqaʁjɯβ-a\cmn 你把我挡住了\end{exemple}
\begin{exemple}\jya tɤ-ta-sqaβjɯɣ\cmn 我把你挡住了\end{exemple}
\begin{exemple}\jya tɤŋe zdɯm kɯ to-sqaβjɯβ ɲɯ-ŋu, tɕe mɯ-ɲɤ-sɤmto\cmn 云遮住了太阳,所以就看不见\end{exemple}
\begin{relation-sémantique}\synonyme{
\hyperlink{Ⓔsqaβjɯβ}{\textit{ \papi{sqaβjɯβ}}}
}\end{relation-sémantique}\end{entrée}

\begin{entrée}
\vedette{\hypertarget{Ⓔsqɤr}{\papi{ sqɤr}}}\markboth{sqɤr}{}
\classe{vt}
\paradigme{\textit{dir :} \jya nɯ-}
\paradigme{\textit{dir :} \jya \_}
\begin{définition}\fra demander à quelqu'un de faire un travail\end{définition}
\begin{définition}\cmn 请人做事;求助\end{définition}
\begin{exemple}\jya nɯ-sqar-a\cmn 我请了他\end{exemple}
\begin{exemple}\jya nɯ́-wɣ-sqar-a\cmn 他请了我\end{exemple}
\begin{exemple}\jya a-kɯ-sqɤr me\cmn 没有人请我\end{exemple}
\begin{exemple}\jya tɯrɣi ɯ-kɯ\_-lɤt nɯ́-wɣ-sqar-a\cmn 他请我撒种子\end{exemple}
\begin{exemple}\jya ɕ-tɤ-sqar-a, ɕ-pɯ-sqar-a\cmn 我去请他了(往上、往下)\end{exemple}\begin{sous-entrée}
\vedette{\hypertarget{}{\papi{ asqɯsqɤr}}}\markboth{asqɯsqɤr}{}\classe{vi}
\begin{définition}\ 
\begin{déclaration}\grammar{recip}\end{déclaration}\end{définition}
\begin{définition}\fra s'aider les uns les autres pour faire des travaux\end{définition}
\begin{définition}\cmn 互相帮忙做事\end{définition}
\begin{exemple}\jya asqɯsqɤr-i\cmn 我们互相帮忙(互相请)\end{exemple}
\end{sous-entrée}\begin{sous-entrée}
\vedette{\hypertarget{}{\papi{ sɤsqɤr}}}\markboth{sɤsqɤr}{}\classe{vi}
\paradigme{\textit{dir :} \jya tɤ-}
\begin{définition}\fra demander à des gens de faire un travail\end{définition}
\begin{définition}\cmn 请人帮忙\end{définition}
\begin{relation-sémantique}\synonyme{
\hyperlink{Ⓔftɕɤl}{\textit{ \papi{ftɕɤl}}}
}\end{relation-sémantique}
\end{sous-entrée}\end{entrée}

\begin{entrée}
\vedette{\hypertarget{Ⓔsqhɤtɤjɯm}{\papi{ sqhɤtɤjɯm}}}\markboth{sqhɤtɤjɯm}{}\classe{n}
\begin{définition}\fra instruments de cuisine\end{définition}
\begin{définition}\cmn 三脚架和锅子;炊具\end{définition}
\begin{relation-sémantique}\confer{
\hyperlink{Ⓔsqhi}{\textit{ \papi{sqhi}}}
}\end{relation-sémantique}\end{entrée}

\begin{entrée}
\vedette{\hypertarget{Ⓔsqhɤthɤlɤɣi}{\papi{ sqhɤthɤlɤɣi}}}\markboth{sqhɤthɤlɤɣi}{}
\classe{n}
\begin{définition}\fra cendre\end{définition}
\begin{définition}\cmn 草木灰\end{définition}
\begin{relation-sémantique}\synonyme{
\hyperlink{Ⓔthɤfkɤlɤɣi}{\textit{ \papi{thɤfkɤlɤɣi}}}
}\end{relation-sémantique}
\begin{relation-sémantique}\confer{
\hyperlink{Ⓔsqhi}{\textit{ \papi{sqhi}}}
}\end{relation-sémantique}\end{entrée}

\begin{entrée}
\vedette{\hypertarget{Ⓔsqhi}{\papi{ sqhi}}}\markboth{sqhi}{}
\classe{n}
\begin{définition}\fra trépied\end{définition}
\begin{définition}\cmn 三脚架\end{définition}
\begin{relation-sémantique}\confer{
\hyperlink{Ⓔsqhɤtɤjɯm}{\textit{ \papi{sqhɤtɤjɯm}}}
}\end{relation-sémantique}\end{entrée}

\begin{entrée}
\vedette{\hypertarget{Ⓔsqhiar}{\papi{ sqhiar}}}\markboth{sqhiar}{}
\classe{vt}
\paradigme{\textit{dir :} \jya \_}
\begin{définition}\fra ouvrir, étendre\end{définition}
\begin{définition}\cmn 展开(布料、鸟的翅膀)\end{définition}
\begin{exemple}\jya pɣa kɯ ɯ-ʁar nɯ na-sqhiar\cmn 鸟展开了翅膀\end{exemple}
\begin{exemple}\jya tɯ-ŋga tɤ-tɯ-ɕkho-t tɕe nɯ-sqhiar ra ma mɤ-zbaʁ\cmn 你晒衣服时候,一定要把它展开不然就不会干\end{exemple}\end{entrée}

\begin{entrée}
\vedette{\hypertarget{Ⓔsqi}{\papi{ sqi}}}\markboth{sqi}{}\classe{num}
\begin{définition}\fra dix\end{définition}
\begin{définition}\cmn 十\end{définition}
\begin{exemple}\jya ɯ-sqɯ-xpa\cmn 好几十年\end{exemple}\end{entrée}

\begin{entrée}
\vedette{\hypertarget{Ⓔsqlɯm}{\papi{ sqlɯm}}}\markboth{sqlɯm}{}
\classe{vi}
\paradigme{\textit{dir :} \jya pɯ-}
\paradigme{\textit{dir :} \jya kɤ-}
\begin{définition}\fra s'affaisser, avoir une crevasse\end{définition}
\begin{définition}\cmn 陷下去;塌下去\end{définition}
\begin{exemple}\jya khɤxtu pɯ-sqlɯm\cmn 房背陷下去了\end{exemple}
\begin{exemple}\jya rnda pjɤ-sqlɯm\cmn 楼层陷下去了\end{exemple}
\begin{exemple}\jya sɤtɕha (tɯ-khɤl) pjɤ-sqlɯm\cmn (有一个)地方陷下去了\end{exemple}
\begin{relation-sémantique}\synonyme{
\hyperlink{Ⓔmbɯt}{\textit{ \papi{mbɯt}}}
}\end{relation-sémantique}
\begin{relation-sémantique}\confer{
\hyperlink{Ⓔarɴɢlɯm}{\textit{ \papi{arɴɢlɯm}}}
}\end{relation-sémantique}\end{entrée}

\begin{entrée}
\vedette{\hypertarget{Ⓔsrɤz}{\papi{ srɤz}}}\markboth{srɤz}{}\classe{n}
\begin{définition}\fra prince\end{définition}
\begin{définition}\cmn 王子
\begin{déclaration} \étymologie{\papi{sras}}\end{déclaration}\end{définition}
\end{entrée}

\begin{entrée}
\vedette{\hypertarget{Ⓔsroχtɕɤn}{\papi{ sroχtɕɤn}}}\markboth{sroχtɕɤn}{}\classe{n}
\begin{définition}\fra tuer des être vivants\end{définition}
\begin{définition}\cmn 杀生\end{définition}
\begin{exemple}\jya sroχtɕɤn ma-pɯ-tɯ-lɤt\cmn 你不要杀生\end{exemple}\end{entrée}

\begin{entrée}
\vedette{\hypertarget{Ⓔsrɯβzɤn}{\papi{ srɯβzɤn}}}\markboth{srɯβzɤn}{}
\classe{n}
\begin{définition}\fra pièce de tissu placée entre deux autres morceaux de tissus au niveau de la couture\end{définition}
\begin{définition}\cmn 在两块布料的缝合处另外夹上一块布料\end{définition}\end{entrée}

\begin{entrée}
\vedette{\hypertarget{ⒺsrɯnⒽ1}{\papi{ srɯn}}}\markboth{srɯn}{}\homonyme{1}
\classe{n}
\begin{définition}\fra inflammation des sinus\end{définition}
\begin{définition}\cmn 鼻窦炎\end{définition}
\begin{relation-sémantique}\confer{
\hyperlink{Ⓔsrɯsmɤn}{\textit{ \papi{srɯsmɤn}}}
}\end{relation-sémantique}\end{entrée}

\begin{entrée}
\vedette{\hypertarget{ⒺsrɯnⒽ2}{\papi{ srɯn}}}\markboth{srɯn}{}\homonyme{2}
\classe{n}
\begin{définition}\fra coton\end{définition}
\begin{définition}\cmn 棉花
\begin{déclaration} \étymologie{\papi{srin}}\end{déclaration}\end{définition}
\end{entrée}

\begin{entrée}
\vedette{\hypertarget{ⒺsrɯnⒽ3}{\papi{ srɯn}}}\markboth{srɯn}{}\homonyme{3}
\classe{n}
\begin{définition}\fra pellicules\end{définition}
\begin{définition}\cmn 头屑\end{définition}
\begin{exemple}\jya ɯ-ku srɯn ɲɯ-dɤn\cmn 他们头上很多头屑\end{exemple}\end{entrée}

\begin{entrée}
\vedette{\hypertarget{Ⓔsrɯnbu}{\papi{ srɯnbu}}}\markboth{srɯnbu}{} (\variante{srɯtphu}) \classe{n}
\begin{définition}\fra râkshasa\end{définition}
\begin{définition}\cmn 妖精\end{définition}
\begin{relation-sémantique}\confer{
\hyperlink{Ⓔsrɯnmɯ}{\textit{ \papi{srɯnmɯ}}}
}\end{relation-sémantique}\end{entrée}

\begin{entrée}
\vedette{\hypertarget{Ⓔsrɯndɤr}{\papi{ srɯndɤr}}}\markboth{srɯndɤr}{}
\classe{n}
\begin{définition}\fra acné\end{définition}
\begin{définition}\cmn 青春痘\end{définition}
\begin{exemple}\jya nɤ-srɯndɤr ɲɤ-ɬoʁ\cmn 你长了青春痘\end{exemple}
\begin{relation-sémantique}\confer{
\hyperlink{Ⓔnɯsrɯɣndɤr}{\textit{ \papi{nɯsrɯɣndɤr}}}
}\end{relation-sémantique}\end{entrée}

\begin{entrée}
\vedette{\hypertarget{Ⓔsrɯnloʁ}{\papi{ srɯnloʁ}}}\markboth{srɯnloʁ}{}
\classe{n}\acception{1}
\begin{définition}\fra diadème en argent\end{définition}
\begin{définition}\cmn 头上的银制装饰品\end{définition}\acception{2}
\begin{définition}\fra anneau\end{définition}
\begin{définition}\cmn 戒指\end{définition}
\begin{exemple}\jya srɯnloʁ lɤ-nɯrʁe-t-a\cmn 我戴了戒指\end{exemple}
\begin{exemple}\jya srɯnloʁ-pɯ\cmn 小戒指\end{exemple}\end{entrée}

\begin{entrée}
\vedette{\hypertarget{Ⓔsrɯnmɯ}{\papi{ srɯnmɯ}}}\markboth{srɯnmɯ}{}
\classe{n}
\begin{définition}\fra râkshasî\end{définition}
\begin{définition}\cmn 妖精
\begin{déclaration} \étymologie{\papi{srin.mo}}\end{déclaration}\end{définition}\end{entrée}

\begin{entrée}
\vedette{\hypertarget{Ⓔsrɯsmɤn}{\papi{ srɯsmɤn}}}\markboth{srɯsmɤn}{}\classe{n}
\begin{définition}\fra médicament contre l'inflammation des sinus\end{définition}
\begin{définition}\cmn 鼻窦炎的药(鼻烟)\end{définition}
\begin{relation-sémantique}\confer{
\hyperlink{Ⓔsmɤn}{\textit{ \papi{smɤn}}}
}\end{relation-sémantique}\end{entrée}

\begin{entrée}
\vedette{\hypertarget{Ⓔsrɯtphɯ}{\papi{ srɯtphɯ}}}\markboth{srɯtphɯ}{}\classe{n}
\begin{définition}\fra râkshasa\end{définition}
\begin{définition}\cmn 男妖\end{définition}
\begin{relation-sémantique}\confer{
\hyperlink{Ⓔsrɯnbu}{\textit{ \papi{srɯnbu}}}
}\end{relation-sémantique}\end{entrée}

\begin{entrée}
\vedette{\hypertarget{ⒺstuⒽ1}{\papi{ stu}}}\markboth{stu}{}\homonyme{1}\classe{vs}
\paradigme{\textit{dir :} \jya tɤ-}
\begin{définition}\fra assidu, travailleur\end{définition}
\begin{définition}\cmn 努力\end{définition}
\begin{exemple}\jya kɤ-rɤma to-stu\cmn 他工作很努力\end{exemple}\acception{2}
\begin{définition}\fra faire attention à\end{définition}
\begin{définition}\cmn 注意安全\end{définition}
\begin{exemple}\jya smi tu-kɯ-stu ɲɯ-ra\cmn 一定要注意安全用火\end{exemple}\begin{sous-entrée}
\vedette{\hypertarget{}{\papi{ nɤstumbat}}}\markboth{nɤstumbat}{}\classe{vt}
\begin{définition}\fra trouver assidu\end{définition}
\begin{définition}\cmn 觉得努力\end{définition}
\begin{exemple}\jya ɯʑo ɲɯ-nɤstumbat-a\cmn 我觉得他很努力\end{exemple}
\end{sous-entrée}\begin{sous-entrée}
\vedette{\hypertarget{}{\papi{ stu,mbat}}}\markboth{stu,mbat}{}
\paradigme{\textit{dir :} \jya tɤ-}
\begin{définition}\fra assidu, travailleur\end{définition}
\begin{définition}\cmn 努力\end{définition}
\begin{exemple}\jya stu-a mbat-a\cmn 我会努力的\end{exemple}
\begin{exemple}\jya tɤ-stu tɤ-mbat je\cmn 你努力吧\end{exemple}
\begin{relation-sémantique}\ComponentA{\classe{vs}
\hyperlink{ⒺstuⒽ1}{\textit{ \papi{stu}}}
}\end{relation-sémantique}
\begin{relation-sémantique}\ComponentB{\classe{vs}
\hyperlink{Ⓔmbat}{\textit{ \papi{mbat}}}
}\end{relation-sémantique}
\end{sous-entrée}\end{entrée}

\begin{entrée}
\vedette{\hypertarget{ⒺstuⒽ2}{\papi{ stu}}}\markboth{stu}{}\homonyme{2}\classe{vi-t}
\paradigme{\textit{dir :} \jya nɯ-}
\paradigme{\textit{dir :} \jya nɯ-}
\paradigme{\textit{dir :} \jya nɯ-}
\begin{définition}\fra croire (une parole)\end{définition}
\begin{définition}\cmn 相信\end{définition}
\begin{exemple}\jya ɯ-tɯfɕɤt pɯ-βzu-t-a ri, ɯʑo ɲɯ-stu\cmn 我告诉他了,他相信\end{exemple}
\begin{exemple}\jya nɯ tɯ-tɕha nɯ jɤ-azɣɯt ri, ɯʑo ɲɯ-stu\cmn 来了这个消息,他相信\end{exemple}
\begin{relation-sémantique}\confer{
\hyperlink{Ⓔɯ-stuⒽ2}{\textit{ \papi{ɯ-stu2}}}
}\end{relation-sémantique}
\begin{relation-sémantique}\confer{
\hyperlink{Ⓔnɤstu}{\textit{ \papi{nɤstu}}}
}\end{relation-sémantique}
\begin{relation-sémantique}\confer{
\hyperlink{ⒺsɤstuⒽ1}{\textit{ \papi{sɤstu1}}}
}\end{relation-sémantique}
\begin{relation-sémantique}\confer{
\hyperlink{Ⓔsɤnɤstu}{\textit{ \papi{sɤnɤstu}}}
}\end{relation-sémantique}\begin{sous-entrée}
\vedette{\hypertarget{}{\papi{ sɯstu}}}\markboth{sɯstu}{}\classe{vt}
\begin{définition}\ 
\begin{déclaration}\grammar{caus}\end{déclaration}\end{définition}
\begin{définition}\fra faire croire\end{définition}
\begin{définition}\cmn 令……相信\end{définition}
\end{sous-entrée}\end{entrée}

\begin{entrée}
\vedette{\hypertarget{ⒺstuⒽ3}{\papi{ stu}}}\markboth{stu}{}\homonyme{3}
\classe{vt}
\paradigme{\textit{dir :} \jya tɤ-}
\begin{définition}\fra faire d'une certaine manière\end{définition}
\begin{définition}\cmn 那样做\end{définition}
\begin{exemple}\jya nɤʑo rɟɤɣi ɯ-ŋgɯ kɯ-chi pjɯ-tɯ-nɯ-lɤt ɲɯ-ŋu tɕe, aʑo kɯnɤ nɯ tɤ-stu-t-a\cmn 你吃糌粑加糖,我也这样吃\end{exemple}
\begin{exemple}\jya kɤ-stu ɲɯ-me\cmn 没有办法\end{exemple}
\begin{exemple}\jya aʑo rŋɯl ste-a me\cmn 钱对我没有用\end{exemple}
\begin{exemple}\jya kɯmaʁ to-stu-t-a tɕe ɲɯ-βzɟɯr-a ɲɯ-ntshi\cmn 我弄错了,要纠正过来\end{exemple}\begin{sous-entrée}
\vedette{\hypertarget{}{\papi{ nɯstu}}}\markboth{nɯstu}{}\classe{vt}
\paradigme{\textit{dir :} \jya tɤ-}
\begin{définition}\fra considérer comme\end{définition}
\begin{définition}\cmn 当成\end{définition}
\begin{exemple}\jya ɯ-rɟit ʑo tú-wɣ-nɯstu-a ŋu\cmn 他把我当成自己的孩子一样\end{exemple}
\begin{relation-sémantique}\confer{
\hyperlink{Ⓔsɤrtsi}{\textit{ \papi{sɤrtsi}}}
}\end{relation-sémantique}
\begin{relation-sémantique}\confer{
\hyperlink{ⒺsɯpaⒽ1}{\textit{ \papi{sɯpa}}}
}\end{relation-sémantique}
\end{sous-entrée}\end{entrée}

\begin{entrée}
\vedette{\hypertarget{Ⓔsta}{\papi{ sta}}}\markboth{sta}{}\classe{vi}
\paradigme{\textit{dir :} \jya thɯ-}
\begin{définition}\fra se réveiller\end{définition}
\begin{définition}\cmn 醒\end{définition}
\begin{exemple}\jya tɤ-pɤtso cho-sta\cmn 小孩子醒了\end{exemple}
\begin{exemple}\jya aj ʑa thɯ-sta-a\cmn 我早就醒了\end{exemple}
\begin{exemple}\jya a-tɯ-sta ɲɯ-maqhu\cmn 我醒得很晚\end{exemple}\begin{sous-entrée}
\vedette{\hypertarget{}{\papi{ sɯsta}}}\markboth{sɯsta}{}\classe{vt}
\paradigme{\textit{dir :} \jya thɯ-}
\begin{définition}\ 
\begin{déclaration}\grammar{caus}\end{déclaration}\end{définition}
\begin{définition}\fra réveiller\end{définition}
\begin{définition}\cmn 弄醒\end{définition}
\begin{exemple}\jya kɯm ɯ-zgra nɯ kɯ chó-wɣ-sɯsta\cmn 门的声音把他弄醒了\end{exemple}
\end{sous-entrée}\end{entrée}

\begin{entrée}
\vedette{\hypertarget{Ⓔstaʁ}{\papi{ staʁ}}}\markboth{staʁ}{}\classe{postp}
\begin{définition}\fra par rapport à\end{définition}
\begin{définition}\cmn 比\end{définition}
\begin{relation-sémantique}\synonyme{
\hyperlink{Ⓔsɤz}{\textit{ \papi{sɤz}}}
}\end{relation-sémantique}
\begin{relation-sémantique}\synonyme{
\hyperlink{Ⓔstaʁnɤ}{\textit{ \papi{staʁnɤ}}}
}\end{relation-sémantique}\end{entrée}

\begin{entrée}
\vedette{\hypertarget{Ⓔstaʁɕɤr}{\papi{ staʁɕɤr}}}\markboth{staʁɕɤr}{}\classe{n}
\begin{définition}\fra petit cochon dont la peau est bariolée de rose, de noir et de blanc\end{définition}
\begin{définition}\cmn 毛色黑白红相间的小猪\end{définition}
\end{entrée}

\begin{entrée}
\vedette{\hypertarget{Ⓔstaʁlu}{\papi{ staʁlu}}}\markboth{staʁlu}{}\classe{n}
\begin{définition}\fra année du tigre\end{définition}
\begin{définition}\cmn 虎年
\begin{déclaration} \étymologie{\papi{stag.lo}}\end{déclaration}\end{définition}
\end{entrée}

\begin{entrée}
\vedette{\hypertarget{Ⓔstaʁnɤ}{\papi{ staʁnɤ}}}\markboth{staʁnɤ}{}\classe{postp}
\begin{définition}\fra par rapport à\end{définition}
\begin{définition}\cmn 比\end{définition}
\begin{exemple}\jya nɯ staʁnɤ\cmn 还不如\end{exemple}
\begin{relation-sémantique}\synonyme{
\hyperlink{Ⓔstaʁ}{\textit{ \papi{staʁ}}}
}\end{relation-sémantique}
\begin{relation-sémantique}\confer{
\hyperlink{Ⓔsɤz}{\textit{ \papi{sɤz}}}
}\end{relation-sémantique}\end{entrée}

\begin{entrée}
\vedette{\hypertarget{Ⓔstaʁrɟɤnma}{\papi{ staʁrɟɤnma}}}\markboth{staʁrɟɤnma}{}\classe{n}
\begin{définition}\fra bol\end{définition}
\begin{définition}\cmn 瓷碗,画着佛像的图案
\begin{déclaration} \étymologie{\papi{stag rgʲan.ma}}\end{déclaration}\end{définition}
\end{entrée}

\begin{entrée}
\vedette{\hypertarget{Ⓔstat}{\papi{ stat}}}\markboth{stat}{}\classe{vi}
\paradigme{\textit{dir :} \jya tɤ-}
\paradigme{\textit{dir :} \jya tɤ-}
\paradigme{\textit{dir :} \jya tɤ-}
\begin{définition}\fra s'arrêter\end{définition}
\begin{définition}\cmn 停止\end{définition}
\begin{exemple}\jya tɯ-mɯ to-stat (= tɯ-mɯ kɤ-lɤt to-znɯna)\cmn 雨停了\end{exemple}
\begin{exemple}\jya ɯ-ɕnɤse to-stat (= ɯ-ɕnɤse kɯ-ɬoʁ to-nɯna)\cmn 他的鼻血止住了\end{exemple}\begin{sous-entrée}
\vedette{\hypertarget{}{\papi{ sɯstat}}}\markboth{sɯstat}{}\classe{vt}
\begin{définition}\fra arrêter\end{définition}
\begin{définition}\cmn 使……停止\end{définition}
\begin{exemple}\jya ɯ-ɕnɤse to-sɯstat pjɤ-cha\cmn 他成功地把鼻血止住了\end{exemple}
\end{sous-entrée}\end{entrée}

\begin{entrée}
\vedette{\hypertarget{Ⓔstaχpɯ}{\papi{ staχpɯ}}}\markboth{staχpɯ}{}\classe{n}
\begin{définition}\fra haricot\end{définition}
\begin{définition}\cmn 豌豆\end{définition}\end{entrée}

\begin{entrée}
\vedette{\hypertarget{Ⓔstaχpɯldzɣɤm}{\papi{ staχpɯldzɣɤm}}}\markboth{staχpɯldzɣɤm}{}
\classe{n}
\begin{définition}\fra paille de haricots\end{définition}
\begin{définition}\cmn 豌豆秸\end{définition}\end{entrée}

\begin{entrée}
\vedette{\hypertarget{Ⓔstaχpɯqajɯ}{\papi{ staχpɯqajɯ}}}\markboth{staχpɯqajɯ}{}\classe{n}
\begin{définition}\fra espèce de chenille\end{définition}
\begin{définition}\cmn 毛虫的一种\end{définition}\end{entrée}

\begin{entrée}
\vedette{\hypertarget{Ⓔstaχpɯrɟɤskhi}{\papi{ staχpɯrɟɤskhi}}}\markboth{staχpɯrɟɤskhi}{}
\classe{n}
\begin{définition}\fra une plante\end{définition}
\begin{définition}\cmn 植物的一种\end{définition}
\begin{exemple}\jya staχpɯrɟɤskhi nɯ sɯjno kɯ-ʁjɤr ɯ-ŋgɯ tu-ɬoʁ ŋu, ɯ-jwaʁ kɯ-ɤrtɯ-rtɯm kɯ-xtɕɯ-xtɕi ŋu, ɯ-rme kɯ-fse tu, jaʁ tsa. ɯ-ru kɯ-xtɯ-xtɯt ma me, ɯ-mɯntoʁ staχpɯ mɯntoʁ fse tɕe kɯ-ɣɯrni ɲɯ-lɤt ŋu. ɯ-mat nɯ ɯ-cɤβ chɯ-βze ŋu. fsapaʁ ra kɯ tu-ndza-nɯ sna, tɯrme kɤ-ndza mɤ-sna.\cmn 
\stylefv{staχpɯ rɟɤskhi} 生长茂盛的草丛里,叶子圆圆的、小小的,上面有毛。叶子有点厚。只有短短的茎,花像豌豆的一样,是红色的,结的是荚果。牲畜可以吃,人不能吃。
\end{exemple}\end{entrée}

\begin{entrée}
\vedette{\hypertarget{Ⓔstɤβtshɤt}{\papi{ stɤβtshɤt}}}\markboth{stɤβtshɤt}{}\classe{n}
\begin{définition}\fra épreuves de force\end{définition}
\begin{définition}\cmn 比力气
\begin{déclaration} \étymologie{\papi{stobs.tsʰad}}\end{déclaration}\end{définition}
\begin{relation-sémantique}\confer{
\hyperlink{Ⓔnɯstɤβtshɤt}{\textit{ \papi{nɯstɤβtshɤt}}}
}\end{relation-sémantique}\end{entrée}

\begin{entrée}
\vedette{\hypertarget{Ⓔstɤɣdo}{\papi{ stɤɣdo}}}\markboth{stɤɣdo}{}\classe{n}
\begin{définition}\fra enfant unique\end{définition}
\begin{définition}\cmn 独生子\end{définition}\end{entrée}

\begin{entrée}
\vedette{\hypertarget{Ⓔstɤjstɤj}{\papi{ stɤjstɤj}}}\markboth{stɤjstɤj}{}
\classe{idph.2}
\begin{définition}\fra petit et trapu\end{définition}
\begin{définition}\cmn 形容又矮又圆的样子\end{définition}\begin{sous-entrée}
\vedette{\hypertarget{}{\papi{ ɣɤstɤjlɤj}}}\markboth{ɣɤstɤjlɤj}{}\classe{vi}
\begin{définition}\fra qui rebondit, qui saute\end{définition}
\begin{définition}\cmn 弹来弹去、跳来跳去(圆的、小的东西)
\begin{déclaration}\use{\stylefv{ɣɤstɯlɯr}、\stylefv{ɣɤstɤlɤj}和\stylefv{ɣɤstalaŋ}都表示圆形的东西在跳来跳去,但是大小不同,\stylefv{stɯrstɯr}用于细小的东西(珠子等),\stylefv{stɤjstɤj}用于小孩子、小动物、皮球,\stylefv{staŋstaŋ}用于大石头、大的动物}\end{déclaration}\end{définition}
\begin{exemple}\jya @piqiu ɲɯ-ɣɤstɤjlɤj ntsɯ\cmn 皮球弹来弹去\end{exemple}
\begin{relation-sémantique}\synonyme{
\hyperlink{Ⓔstɯrstɯr}{\textit{ \papi{stɯrstɯr}}}
}\end{relation-sémantique}
\begin{relation-sémantique}\synonyme{
\hyperlink{Ⓔzdɯzdɯr}{\textit{ \papi{zdɯzdɯr}}}
}\end{relation-sémantique}
\begin{relation-sémantique}\synonyme{
\hyperlink{Ⓔɣɤstaŋlaŋ}{\textit{ \papi{ɣɤstaŋlaŋ}}}
}\end{relation-sémantique}
\end{sous-entrée}\begin{sous-entrée}
\vedette{\hypertarget{}{\papi{ stɤjnɤlɤj}}}\markboth{stɤjnɤlɤj}{}
\begin{exemple}\jya staχpɯ stɤjnɤlɤj ʑo ɲɯ-nɤmdɯmdar\cmn 豌豆在弹来弹去\end{exemple}
\end{sous-entrée}\end{entrée}

\begin{entrée}
\vedette{\hypertarget{Ⓔstɤm}{\papi{ stɤm}}}\markboth{stɤm}{}\classe{vi}
\paradigme{\textit{dir :} \jya kɤ-}
\begin{définition}\fra se solidifier\end{définition}
\begin{définition}\cmn 凝固\end{définition}
\begin{exemple}\jya tɯkri kɤ-stɤm\cmn 油凝固了\end{exemple}
\begin{exemple}\jya sɤtɕha ko-stɤm\cmn 地凝固了\end{exemple}\begin{sous-entrée}
\vedette{\hypertarget{}{\papi{ sɯstɤm}}}\markboth{sɯstɤm}{}\classe{vt}
\paradigme{\textit{dir :} \jya kɤ-}
\begin{définition}\fra laisser se solidifier\end{définition}
\begin{définition}\cmn 使凝固\end{définition}
\begin{exemple}\jya ta-mar thɯ-ftʂi-t-a tɕe kɤ-sɯstam-a\cmn 我把酥油融化成液体,然后又让它凝固了\end{exemple}
\end{sous-entrée}\end{entrée}

\begin{entrée}
\vedette{\hypertarget{Ⓔstɤmku}{\papi{ stɤmku}}}\markboth{stɤmku}{}
\classe{n}
\begin{définition}\fra plaine\end{définition}
\begin{définition}\cmn 草坪\end{définition}\end{entrée}

\begin{entrée}
\vedette{\hypertarget{Ⓔstɤnga}{\papi{ stɤnga}}}\markboth{stɤnga}{}\classe{n}
\begin{définition}\fra manteau\end{définition}
\begin{définition}\cmn 上衣\end{définition}
\begin{relation-sémantique}\confer{
\hyperlink{Ⓔtɯ-stɤt}{\textit{ \papi{tɯ-stɤt}}}
}\end{relation-sémantique}
\begin{relation-sémantique}\confer{
\hyperlink{Ⓔtɯ-ŋga}{\textit{ \papi{tɯ-ŋga}}}
}\end{relation-sémantique}\end{entrée}

\begin{entrée}
\vedette{\hypertarget{Ⓔstɤɴɢaʁ}{\papi{ stɤɴɢaʁ}}}\markboth{stɤɴɢaʁ}{}\classe{n}
\begin{définition}\fra chemise de moine\end{définition}
\begin{définition}\cmn 和尚穿的上内衣
\begin{déclaration} \étymologie{\papi{stod}}\end{déclaration}\end{définition}\end{entrée}

\begin{entrée}
\vedette{\hypertarget{Ⓔstɤrɟɯɣ}{\papi{ stɤrɟɯɣ}}}\markboth{stɤrɟɯɣ}{}\classe{idph}
\paradigme{\textit{emphatic :} \jya stɤrɟɯɣ jɤrɟɯɣ}
\begin{définition}\fra en courant\end{définition}
\begin{définition}\cmn 跑着\end{définition}
\begin{relation-sémantique}\confer{
\hyperlink{ⒺrɟɯɣⒽ1}{\textit{ \papi{rɟɯɣ1}}}
}\end{relation-sémantique}
\begin{relation-sémantique}\confer{
\hyperlink{Ⓔnɯstɤrɟɯɣ}{\textit{ \papi{nɯstɤrɟɯɣ}}}
}\end{relation-sémantique}\end{entrée}

\begin{entrée}
\vedette{\hypertarget{Ⓔstɤsmɤt}{\papi{ stɤsmɤt}}}\markboth{stɤsmɤt}{}\classe{n}
\begin{définition}\fra tête et queue\end{définition}
\begin{définition}\cmn 头尾
\begin{déclaration} \étymologie{\papi{stod.smad}}\end{déclaration}\end{définition}\end{entrée}

\begin{entrée}
\vedette{\hypertarget{Ⓔstɤsqa}{\papi{ stɤsqa}}}\markboth{stɤsqa}{}\classe{n}
\begin{définition}\fra fève cuite\end{définition}
\begin{définition}\cmn 煮熟了的胡豆\end{définition}
\begin{relation-sémantique}\confer{
\hyperlink{Ⓔstoʁ}{\textit{ \papi{stoʁ}}}
}\end{relation-sémantique}
\begin{relation-sémantique}\confer{
\hyperlink{Ⓔsqa}{\textit{ \papi{sqa}}}
}\end{relation-sémantique}\end{entrée}

\begin{entrée}
\vedette{\hypertarget{Ⓔstɤt}{\papi{ stɤt}}}\markboth{stɤt}{}
\classe{vt}
\paradigme{\textit{dir :} \jya pɯ-}
\begin{définition}\fra attacher (bovidés)\end{définition}
\begin{définition}\cmn 拴在草茂盛的地方(犏牛、牛)
\end{définition}
\begin{exemple}\jya fsapaʁ pjɯ́-wɣ-stɤt\cmn 拴牲畜\end{exemple}
\begin{exemple}\jya jla pjɯ́-wɣ-stɤt\cmn 拴牛\end{exemple}\end{entrée}

\begin{entrée}
\vedette{\hypertarget{Ⓔstɤtoŋ}{\papi{ stɤtoŋ}}}\markboth{stɤtoŋ}{}
\classe{n}
\begin{définition}\fra manteau\end{définition}
\begin{définition}\cmn 上衣
\begin{déclaration} \étymologie{\papi{stod.tʰuŋ}}\end{déclaration}\end{définition}\end{entrée}

\begin{entrée}
\vedette{\hypertarget{Ⓔstɤtpa}{\papi{ stɤtpa}}}\markboth{stɤtpa}{}\classe{n}
\begin{définition}\ 
\begin{déclaration}\grammar{n.lieu}\end{déclaration}\end{définition}
\begin{définition}\fra Stodpa\end{définition}
\begin{définition}\cmn 四大坝
\begin{déclaration} \étymologie{\papi{stod.pa}}\end{déclaration}\end{définition}
\begin{exemple}\jya stɤtpapɯ\cmn 四大坝人\end{exemple}\end{entrée}

\begin{entrée}
\vedette{\hypertarget{Ⓔstɣɤrnɤstɣɤr}{\papi{ stɣɤrnɤstɣɤr}}}\markboth{stɣɤrnɤstɣɤr}{}\classe{idph.2}
\begin{définition}\fra en bondissant\end{définition}
\begin{définition}\cmn 一跳一跳\end{définition}\end{entrée}

\begin{entrée}
\vedette{\hypertarget{Ⓔsthaβ}{\papi{ sthaβ}}}\markboth{sthaβ}{}\classe{vt}\acception{1}
\paradigme{\textit{dir :} \jya \_}
\begin{définition}\fra toucher\end{définition}
\begin{définition}\cmn 靠;碰\end{définition}
\begin{exemple}\jya ɯ-taʁ ka-sthaβ\cmn 他碰了一下(上面)\end{exemple}
\begin{exemple}\jya kɤ-sthaβ-a\cmn 我碰了一下\end{exemple}
\begin{exemple}\jya ɯ-rŋa ɯ-taʁ a-jaʁ kɤ-sthaβ-a\cmn 我用手碰了一下他的脸\end{exemple}
\begin{exemple}\jya ɯʑo kɯ ɯ-jaʁ smi ɯ-taʁ ko-sthaβ tɕe pjɤ-sɤɕke\cmn 他触摸到火了,很烫\end{exemple}\acception{2}
\paradigme{\textit{dir :} \jya lɤ-}
\begin{définition}\fra mettre à chauffer sur le four\end{définition}
\begin{définition}\cmn 放在炉子上烤\end{définition}
\begin{exemple}\jya tɯ-ci la-sthaβ\cmn 他把水放在火上做热了\end{exemple}\end{entrée}

\begin{entrée}
\vedette{\hypertarget{Ⓔsthoŋsthoŋ}{\papi{ sthoŋsthoŋ}}}\markboth{sthoŋsthoŋ}{}\classe{idph.2}
\begin{définition}\fra être déformé après avoir été trop rempli\end{définition}
\begin{définition}\cmn 形容口袋里的东西装得很满,变形了(指软的东西,如口袋、枕头棉花等等)\end{définition}
\begin{exemple}\jya tɤ-mkɯm ɯ-rku mɯ-chɤ-βdi, sthoŋsthoŋ ʑo ɲɯ-pa\cmn 枕头没有塞好,显得很满,不平整(显得很鼓鼓的)\end{exemple}\begin{sous-entrée}
\vedette{\hypertarget{}{\papi{ lɤmɤsthoŋ}}}\markboth{lɤmɤsthoŋ}{}
\begin{exemple}\jya lɤmɤtshoŋ ci ʑo tu\cmn 他膀大腰粗的\end{exemple}
\end{sous-entrée}\begin{sous-entrée}
\vedette{\hypertarget{}{\papi{ phɯsthoŋ}}}\markboth{phɯsthoŋ}{}
\begin{exemple}\jya ɯ-re phɯsthoŋ ʑo ɲɤ-ɕlɯɣ\cmn 他失声突然笑出来\end{exemple}
\end{sous-entrée}\begin{sous-entrée}
\vedette{\hypertarget{}{\papi{ sthoŋ}}}\markboth{sthoŋ}{}\classe{idph.1}
\begin{exemple}\jya tɯmbri kɯ pjɤ-k-ɤsɯxtɕɤr-ci ri ɲɤ-mbrɤt sthoŋ ʑo to-ti\cmn 绳子系得太紧就突然断了\end{exemple}
\end{sous-entrée}\begin{sous-entrée}
\vedette{\hypertarget{}{\papi{ sthoŋnɤloŋ}}}\markboth{sthoŋnɤloŋ}{}
\begin{exemple}\jya ɯ-mbrɯ ɯ-tɯ-ŋgɯ kɯ sthoŋnɤloŋ ʑo ɲɯ-rɤma\cmn 他一边发脾气一边做事的样子\end{exemple}
\end{sous-entrée}\begin{sous-entrée}
\vedette{\hypertarget{}{\papi{ sthoŋnɤsthoŋ}}}\markboth{sthoŋnɤsthoŋ}{}\classe{idph.3}
\begin{exemple}\jya ɯ-skhrɯ mɯ́j-βdi tɕe, sthoŋnɤsthoŋ ʑo ɲɯ-ŋke\cmn 她怀孕了,走路显得很臃肿\end{exemple}
\end{sous-entrée}\end{entrée}

\begin{entrée}
\vedette{\hypertarget{Ⓔsthoʁ}{\papi{ sthoʁ}}}\markboth{sthoʁ}{}
\classe{vt}\acception{1}
\paradigme{\textit{dir :} \jya \_}
\begin{définition}\fra appuyer, pousser\end{définition}
\begin{définition}\cmn 按(用手)、推\end{définition}
\begin{exemple}\jya kɯm thɯ-sthoʁ-a\cmn 我把门推了一下(关门)\end{exemple}
\begin{exemple}\jya kɯm kɤ-sthoʁ\cmn 你关门吧\end{exemple}\acception{2}
\paradigme{\textit{dir :} \jya pɯ-}
\begin{définition}\fra oppresser\end{définition}
\begin{définition}\cmn 压迫\end{définition}
\begin{exemple}\jya rɟɤlpu kɯ mkhɤrmaŋ ra pjɯ-sthoʁ pjɤ-ŋu\cmn 土司压迫老百姓\end{exemple}
\begin{relation-sémantique}\confer{
\hyperlink{Ⓔnɯsthoʁ}{\textit{ \papi{nɯsthoʁ}}}
}\end{relation-sémantique}\end{entrée}

\begin{entrée}
\vedette{\hypertarget{Ⓔsthrɯβ}{\papi{ sthrɯβ}}}\markboth{sthrɯβ}{}\classe{idph.1}
\begin{définition}\fra bruit de mucus qui sort du nez\end{définition}
\begin{définition}\cmn 鼻涕突然出来的声音\end{définition}
\begin{exemple}\jya ɯ-ɕnaβ sthrɯβ ʑo thɯ-nɯɬoʁ tɕe ɲɯ-sɤjloʁ\cmn 他的鼻涕噗的一声就出来了,很恶心\end{exemple}\begin{sous-entrée}
\vedette{\hypertarget{}{\papi{ ɣɤsthɯsthrɯβ}}}\markboth{ɣɤsthɯsthrɯβ}{}
\end{sous-entrée}\begin{sous-entrée}
\vedette{\hypertarget{}{\papi{ phɯsthrɯβ}}}\markboth{phɯsthrɯβ}{}\classe{idph.7}
\end{sous-entrée}\begin{sous-entrée}
\vedette{\hypertarget{}{\papi{ sthrɯβnɤsthrɯβ}}}\markboth{sthrɯβnɤsthrɯβ}{}\classe{idph.3}
\end{sous-entrée}\end{entrée}

\begin{entrée}
\vedette{\hypertarget{Ⓔsthɯβsthɯβ}{\papi{ sthɯβsthɯβ}}}\markboth{sthɯβsthɯβ}{}\classe{idph.2}
\begin{définition}\fra sur le point d'apparaître\end{définition}
\begin{définition}\cmn 形容东西将露未露的样子\end{définition}
\begin{exemple}\jya @baobao ɯ-ŋgɯ laχtɕha sthɯβsthɯβ ʑo ɲɯ-nɯxsɯ\cmn 包里的东西露了一点出来\end{exemple}\begin{sous-entrée}
\vedette{\hypertarget{}{\papi{ phɯsthɯβ}}}\markboth{phɯsthɯβ}{}\classe{idph.7}
\begin{définition}\fra en un instant\end{définition}
\begin{définition}\cmn 突然间\end{définition}
\begin{exemple}\jya phɯsthɯβ ʑo ɲɤ-nɤre\cmn 他突然间笑起来了\end{exemple}
\end{sous-entrée}\end{entrée}

\begin{entrée}
\vedette{\hypertarget{Ⓔsthɯci}{\papi{ sthɯci}}}\markboth{sthɯci}{}\classe{adv}
\begin{définition}\fra autant\end{définition}
\begin{définition}\cmn 那么多\end{définition}
\end{entrée}

\begin{entrée}
\vedette{\hypertarget{Ⓔsthɯt}{\papi{ sthɯt}}}\markboth{sthɯt}{}
\classe{vt}\acception{1}
\paradigme{\textit{dir :} \jya \_}
\begin{définition}\fra finir\end{définition}
\begin{définition}\cmn 完\end{définition}
\begin{exemple}\jya kɤ-nɤma ta-sthɯt\cmn 他把工作做完了\cmn kɤ-nɤma nɯ-sthɯt-a\cmn 我把工作做完了\end{exemple}
\begin{exemple}\jya kɤ-βzjoz pa-sthɯt\cmn 他学完了\end{exemple}
\begin{exemple}\jya tɤ-scoz kɤ-rɤt pjɤ-sthɯt\cmn 他把信写完了\end{exemple}
\begin{exemple}\jya kɤ-ndza chɤ-sthɯt\cmn 他吃完了\end{exemple}
\begin{exemple}\jya kɤ-ntʂu la-nɯ-sthɯt\cmn 他锄完草了\end{exemple}\acception{2}
\paradigme{\textit{dir :} \jya pɯ-}
\begin{définition}\fra être fini, être perdu\end{définition}
\begin{définition}\cmn 完蛋了\end{définition}
\begin{exemple}\jya pɯ-tɯ-nɯ-sthɯt\cmn 你完蛋了!\end{exemple}\end{entrée}

\begin{entrée}
\vedette{\hypertarget{ⒺstiⒽ1}{\papi{ sti}}}\markboth{sti}{}\homonyme{1}\classe{vt}
\paradigme{\textit{dir :} \jya nɯ-}
\paradigme{\textit{dir :} \jya \_}
\begin{définition}\fra boucher\end{définition}
\begin{définition}\cmn 堵塞\end{définition}
\begin{exemple}\jya kɯ-spoʁ ɲɯ́-wɣ-sti\cmn 把洞堵住了\end{exemple}
\begin{exemple}\jya nɤj nɤ-βra aj ʑ-nɯ-sti-t-a\cmn 我替你去(做工)\end{exemple}
\begin{exemple}\jya nɤ-mtɕhi lɤ-sti\cmn 你捂住嘴巴吧!\end{exemple}
\begin{exemple}\jya ɯ-mtɕhi kɤ-sti-t-a\cmn 我捂住了他的嘴巴\end{exemple}
\begin{exemple}\jya phoŋ pɯ-sti-t-a\cmn 我把瓶子盖上\end{exemple}
\begin{exemple}\jya tɤχsɤr ɯ-kɯ-sti ɕti-a ma koŋla a-kɤ-spa me\cmn 我只是充数的,我什么也不会\end{exemple}
\begin{relation-sémantique}\confer{
\hyperlink{Ⓔphoŋsti}{\textit{ \papi{phoŋsti}}}
}\end{relation-sémantique}
\begin{relation-sémantique}\confer{
\hyperlink{Ⓔasti}{\textit{ \papi{asti}}}
}\end{relation-sémantique}\end{entrée}

\begin{entrée}
\vedette{\hypertarget{ⒺstiⒽ2}{\papi{ sti}}}\markboth{sti}{}\homonyme{2}
\classe{vt}
\paradigme{\textit{dir :} \jya tɤ-}
\begin{définition}\fra enlever ce qui est en trop\end{définition}
\begin{définition}\cmn 把多余的东西减掉一些\end{définition}
\begin{exemple}\jya ɯ-tɯ-mtshɤt to-tɕhom tɕe tú-wɣ-sti ɲɯ-ra\cmn 太满了,要舀一点(水)出来(不然就会扑出来)\end{exemple}
\begin{exemple}\jya ki tʂha ki ku-sti-a ɲɯ-ntshi-a ma ɯ-tɯ-mtshɤt to-tɕhom\cmn 我喝了一口,不然这个太满了\end{exemple}
\begin{exemple}\jya a-ŋga ɲɯ-sti-a ɲɯ-ntshi ma ɲɯ-sɤɕke\cmn 我要脱一些衣服,太热了\end{exemple}
\begin{relation-sémantique}\synonyme{
\hyperlink{Ⓔnɯβʑit}{\textit{ \papi{nɯβʑit}}}
}\end{relation-sémantique}\end{entrée}

\begin{entrée}
\vedette{\hypertarget{Ⓔstiaŋnɤstiaŋ}{\papi{ stiaŋnɤstiaŋ}}}\markboth{stiaŋnɤstiaŋ}{}\classe{idph.2}
\begin{définition}\fra en bondissant\end{définition}
\begin{définition}\cmn 一跳一跳(蚱蜢)\end{définition}\end{entrée}

\begin{entrée}
\vedette{\hypertarget{Ⓔstukɤr}{\papi{ stukɤr}}}\markboth{stukɤr}{}
\classe{n}
\begin{définition}\fra poutre\end{définition}
\begin{définition}\cmn 梁\end{définition}\end{entrée}

\begin{entrée}
\vedette{\hypertarget{Ⓔstonka}{\papi{ stonka}}}\markboth{stonka}{}
\classe{n}
\begin{définition}\fra automne\end{définition}
\begin{définition}\cmn 秋天
\begin{déclaration} \étymologie{\papi{ston.ka}}\end{déclaration}\end{définition}\end{entrée}

\begin{entrée}
\vedette{\hypertarget{Ⓔstoŋtsu}{\papi{ stoŋtsu}}}\markboth{stoŋtsu}{}
\classe{num}
\begin{définition}\fra mille\end{définition}
\begin{définition}\cmn 一千
\begin{déclaration} \étymologie{\papi{stoŋ}}\end{déclaration}\end{définition}
\begin{exemple}\jya ɯ-stoŋtsu\cmn 几千个\end{exemple}
\begin{exemple}\jya stoŋtu kɤ-χsɤr\cmn 上千(个、次)\end{exemple}\end{entrée}

\begin{entrée}
\vedette{\hypertarget{Ⓔstoʁ}{\papi{ stoʁ}}}\markboth{stoʁ}{}\classe{n}
\begin{définition}\fra pois\end{définition}
\begin{définition}\cmn 胡豆\end{définition}\end{entrée}

\begin{entrée}
\vedette{\hypertarget{Ⓔstoʁldzɣɤm}{\papi{ stoʁldzɣɤm}}}\markboth{stoʁldzɣɤm}{}
\classe{n}
\begin{définition}\fra paille de pois\end{définition}
\begin{définition}\cmn 胡豆秸\end{définition}\end{entrée}

\begin{entrée}
\vedette{\hypertarget{Ⓔstoʁmboʁ}{\papi{ stoʁmboʁ}}}\markboth{stoʁmboʁ}{}\classe{n}
\begin{définition}\fra explosion (fusil)\end{définition}
\begin{définition}\cmn 爆炸(枪)\end{définition}
\begin{exemple}\jya ɕɤmɯɣdɯ stoʁmboʁ ɲɤ-ɕe (nɯ-ari)\cmn 枪不小心爆炸了\end{exemple}
\begin{relation-sémantique}\confer{
\hyperlink{Ⓔamboʁ}{\textit{ \papi{amboʁ}}}
}\end{relation-sémantique}\end{entrée}

\begin{entrée}
\vedette{\hypertarget{Ⓔstoʁrŋu}{\papi{ stoʁrŋu}}}\markboth{stoʁrŋu}{}\classe{n}
\begin{définition}\fra fèves grillées\end{définition}
\begin{définition}\cmn 炒胡豆\end{définition}
\begin{relation-sémantique}\confer{
\hyperlink{Ⓔstoʁ}{\textit{ \papi{stoʁ}}}
}\end{relation-sémantique}
\begin{relation-sémantique}\confer{
\hyperlink{Ⓔrŋu}{\textit{ \papi{rŋu}}}
}\end{relation-sémantique}\end{entrée}

\begin{entrée}
\vedette{\hypertarget{Ⓔstoʁthɤβ}{\papi{ stoʁthɤβ}}}\markboth{stoʁthɤβ}{}\classe{n}
\begin{définition}\fra culture parallèle\end{définition}
\begin{définition}\cmn 兼种(胡豆中种豌豆)\end{définition}\end{entrée}

\begin{entrée}
\vedette{\hypertarget{Ⓔstoʁtsa}{\papi{ stoʁtsa}}}\markboth{stoʁtsa}{}
\classe{n}
\begin{définition}\fra une plante\end{définition}
\begin{définition}\cmn 植物的一种\end{définition}
\begin{exemple}\jya stoʁtsa nɯ li sɯjno ci ŋu, stɤmku cho tɯji ɯ-rkɯ ra tu-ɬoʁ ŋu, mɤ-mbro. ɯ-ru nɯ kɯ-ngɯ-ngɯt ŋu, kɯ-pɣi tsa ŋu, ɯ-jwaʁ ŋɯ arŋi, mba ri nɤrko, ɯ-mɯntoʁ kɯ-dɤn ʑo tɯtɯrca kɯ-ɤʑɯrja ɲɯ-lɤt ŋu. ɯ-mdoʁ aɣɯrnɯɕɯr. ɯ-tshɯɣa stoʁ ɯ-mɯntoʁ fse. ɯ-cɤβ chɯ-βze ŋu, thɯ-tɯt tɕe, ɯ-rdoʁ nɯ rko, tɕe stoʁtsa rmi tɕe, stoʁ ɯ-ftsa kɤ-ti ɲɯ-ŋu. fsapaʁ ra kɤ-ndza rga-nɯ, tɯrme kɤ-ndza mɤ-sna.\cmn 
\stylefv{stoʁtsa}是一种草,生长在草地和田野边,长得不高。茎长得很结实,是灰色的,叶子是绿色的,薄而结实。花在一起排列着,是淡红色的,形状像胡豆的花一样。结的是荚果,成熟后,种子变硬,所以叫作 \stylefv{stoʁtsa},就是“胡豆粒粒儿”的意思。牲畜喜欢吃,人不能吃。
\end{exemple}\end{entrée}

\begin{entrée}
\vedette{\hypertarget{Ⓔstosqa}{\papi{ stosqa}}}\markboth{stosqa}{}\classe{n}
\begin{définition}\fra fève pas encore mûre\end{définition}
\begin{définition}\cmn 未熟的胡豆\end{définition}
\begin{relation-sémantique}\confer{
\hyperlink{Ⓔstoʁ}{\textit{ \papi{stoʁ}}}
}\end{relation-sémantique}\end{entrée}

\begin{entrée}
\vedette{\hypertarget{Ⓔstɯm}{\papi{ stɯm}}}\markboth{stɯm}{}
\classe{vt}
\paradigme{\textit{dir :} \jya lɤ-}
\begin{définition}\fra ramasser (jambes)\end{définition}
\begin{définition}\cmn 收拢(手、脚)\end{définition}
\begin{exemple}\jya a-mi lɤ-stɯm-a\cmn 我把脚收拢了\end{exemple}
\begin{relation-sémantique}\confer{
\hyperlink{Ⓔrɤstɯm}{\textit{ \papi{rɤstɯm}}}
}\end{relation-sémantique}\end{entrée}

\begin{entrée}
\vedette{\hypertarget{Ⓔstɯnmɯ}{\papi{ stɯnmɯ}}}\markboth{stɯnmɯ}{}
\classe{n}
\begin{définition}\fra mariage\end{définition}
\begin{définition}\cmn 婚姻
\begin{déclaration} \étymologie{\papi{ston.mo}}\end{déclaration}\end{définition}
\begin{relation-sémantique}\confer{
\hyperlink{Ⓔrɯstɯnmɯ}{\textit{ \papi{rɯstɯnmɯ}}}
}\end{relation-sémantique}\end{entrée}

\begin{entrée}
\vedette{\hypertarget{Ⓔstɯrstɯr}{\papi{ stɯrstɯr}}}\markboth{stɯrstɯr}{}
\classe{idph.2}
\begin{définition}\fra objet ronds et petits\end{définition}
\begin{définition}\cmn 形容圆形,很细小的东西(如珠子、豌豆等)\end{définition}\begin{sous-entrée}
\vedette{\hypertarget{}{\papi{ ɣɤstɯrlɯr}}}\markboth{ɣɤstɯrlɯr}{}\classe{vi}
\begin{définition}\fra sautiller, rebondir (petits objets ronds)\end{définition}
\begin{définition}\cmn 弹来弹去(豌豆、珠子等)\end{définition}
\begin{relation-sémantique}\synonyme{
\hyperlink{Ⓔɣɤzdɯzdɯr}{\textit{ \papi{ɣɤzdɯzdɯr}}}
}\end{relation-sémantique}
\begin{relation-sémantique}\synonyme{
\hyperlink{Ⓔɣɤstaŋlaŋ}{\textit{ \papi{ɣɤstaŋlaŋ}}}
}\end{relation-sémantique}
\begin{relation-sémantique}\synonyme{
\hyperlink{Ⓔɣɤstɤjlɤj}{\textit{ \papi{ɣɤstɤjlɤj}}}
}\end{relation-sémantique}
\begin{relation-sémantique}\synonyme{
\hyperlink{Ⓔstɤjstɤj}{\textit{ \papi{stɤjstɤj}}}
}\end{relation-sémantique}
\end{sous-entrée}\end{entrée}

\begin{entrée}
\vedette{\hypertarget{Ⓔstuxsi}{\papi{ stuxsi}}}\markboth{stuxsi}{} (\variante{stuksi}) 
\classe{n}
\begin{définition}\fra joug, attelage pour deux animaux\end{définition}
\begin{définition}\cmn 牛轭(双行)\end{définition}\end{entrée}

\begin{entrée}
\vedette{\hypertarget{Ⓔsɯ}{\papi{ sɯ}}}\markboth{sɯ}{}\classe{vs}
\paradigme{\textit{dir :} \jya tɤ-}
\paradigme{\textit{dir :} \jya thɯ-}
\begin{définition}\fra riche, fructueux\end{définition}
\begin{définition}\cmn 茂盛;丰富\end{définition}
\begin{exemple}\jya jiɕqha nɯ ɯ-ɲɤm ɲɯ-sɯ\cmn 那个(动物)很壮\end{exemple}
\begin{exemple}\jya tɤ-pɤtso ɯ-rgu ɲɯ-sɯ\cmn 那个小孩子很有能力(不要小看他)\end{exemple}
\begin{exemple}\jya ɣɯjpa taχpa ɲɯ-sɯ\cmn 今年的庄稼很好\end{exemple}
\begin{relation-sémantique}\confer{
\hyperlink{Ⓔɯ-rgu,sɯ}{\textit{ \papi{ɯ-rgu,sɯ}}}
}\end{relation-sémantique}
\begin{relation-sémantique}\confer{
 \papi{ɯ-ɲɤm,sɯ}
}\end{relation-sémantique}
\begin{relation-sémantique}\confer{
\hyperlink{Ⓔnɯɲɤmsɯ}{\textit{ \papi{nɯɲɤmsɯ}}}
}\end{relation-sémantique}\end{entrée}

\begin{entrée}
\vedette{\hypertarget{Ⓔsɯbɣi}{\papi{ sɯbɣi}}}\markboth{sɯbɣi}{}
\classe{n}
\begin{définition}\fra espèce d'arbrisseau\end{définition}
\begin{définition}\cmn 灌木的一种\end{définition}
\begin{exemple}\jya sɯbɣi χsɯ-tɯphu tu, tɯ-tɯphu nɯ kɯ-xtshɯm kɯ-rɲɟi ŋu, tɯ-phɯ ɯ-ŋgɯ tɕe kɯ-dɯ-dɤn ʑo tu. ɯ-mat nɯ staχpɯ ɯ-mat kɯ-fse ŋu, ɯ-ru nɯ kuxtɕo kɤ-βzu sna, mɤʑɯ tɯ-tɯphu nɯ, sɯbɣi nɯ jpum tsa mbro tsa ɯ-mnɯ nɯ ɯ-spjɯŋ nɯ kɯ-wɣrum tɕe kɯ-mpɯ ŋu, ɯ-mɯntoʁ nɯ kɯ-wɣrum ɲɯ-lɤt tɕe mpɕɤr, ɯ-mat me, mɤʑɯ tɯ-tɯphu nɯ sɯŋgɯ kɯ-wxti ɯ-ŋgɯ tɯ-tɯphu ma me, xtɕi ri ɯ-ru nɯ ngɯt ɯ-mɯntoʁ cho ɯ-mat me. sɯbɣi χsɯ-tɯphu nɯ nɯ-rqhu kɯ-pɣi ŋu, tɕe núndʐa sɯbɣi rmi\cmn 
\stylefv{sɯbɣi} 有三种,一种长得细长,一株长有许多根,果实像豌豆的果实,干了可以编背篼。另一种长得又粗又高,枝条主心是白色的,很软,能开出好看的白花,没有果实。还有一种长在高大的森林里,比较少见,虽然矮小但主干非常结实,既没有花,又没有果实。这三种\stylefv{sɯbɣi}树皮都是灰色的,所以叫作\stylefv{sɯbɣi}。
\end{exemple}\end{entrée}

\begin{entrée}
\vedette{\hypertarget{Ⓔsɯβde}{\papi{ sɯβde}}}\markboth{sɯβde}{}
\begin{relation-sémantique}\confer{
\hyperlink{Ⓔβde}{\textit{ \papi{βde}}}
}\end{relation-sémantique}\end{entrée}

\begin{entrée}
\vedette{\hypertarget{Ⓔsɯβɣi}{\papi{ sɯβɣi}}}\markboth{sɯβɣi}{}\classe{n}
\begin{définition}\fra sciure\end{définition}
\begin{définition}\cmn 锯末\end{définition}
\end{entrée}

\begin{entrée}
\vedette{\hypertarget{Ⓔsɯβɣɯt}{\papi{ sɯβɣɯt}}}\markboth{sɯβɣɯt}{}
\begin{relation-sémantique}\confer{
\hyperlink{Ⓔβɣɯt}{\textit{ \papi{βɣɯt}}}
}\end{relation-sémantique}\end{entrée}

\begin{entrée}
\vedette{\hypertarget{Ⓔsɯβɟɤt}{\papi{ sɯβɟɤt}}}\markboth{sɯβɟɤt}{}
\begin{relation-sémantique}\confer{
\hyperlink{Ⓔβɟɤt}{\textit{ \papi{βɟɤt}}}
}\end{relation-sémantique}\end{entrée}

\begin{entrée}
\vedette{\hypertarget{Ⓔsɯβɟi}{\papi{ sɯβɟi}}}\markboth{sɯβɟi}{}
\begin{relation-sémantique}\confer{
\hyperlink{ⒺβɟiⒽ1}{\textit{ \papi{βɟi1}}}
}\end{relation-sémantique}
\end{entrée}

\begin{entrée}
\vedette{\hypertarget{Ⓔsɯβʁa}{\papi{ sɯβʁa}}}\markboth{sɯβʁa}{}
\begin{relation-sémantique}\confer{
 \papi{βra}
}\end{relation-sémantique}\end{entrée}

\begin{entrée}
\vedette{\hypertarget{Ⓔsɯβsɯβ}{\papi{ sɯβsɯβ}}}\markboth{sɯβsɯβ}{}\classe{idph.2}
\begin{définition}\fra recouvert de poils fins\end{définition}
\begin{définition}\cmn 形容毛茸茸的样子\end{définition}
\begin{exemple}\jya ɯ-jaʁ sɯβsɯβ ʑo ɲɯ-pa\cmn 他手上长满了毛\end{exemple}\begin{sous-entrée}
\vedette{\hypertarget{}{\papi{ ɣɤsɯβsɯβ}}}\markboth{ɣɤsɯβsɯβ}{}\classe{vs}
\begin{définition}\fra avoir une douleur lancinante\end{définition}
\begin{définition}\cmn 一阵一阵地痛
\end{définition}
\begin{relation-sémantique}\confer{
\hyperlink{Ⓔrsɯβrsɯβ}{\textit{ \papi{rsɯβrsɯβ}}}
}\end{relation-sémantique}
\end{sous-entrée}\begin{sous-entrée}
\vedette{\hypertarget{}{\papi{ sɯβnɤsɯβ}}}\markboth{sɯβnɤsɯβ}{}
\begin{définition}\fra douleur lancinante\end{définition}
\begin{définition}\cmn 形容一阵一阵地痛\end{définition}
\begin{exemple}\jya a-tɯ-ɣmaz sɯβnɤsɯβ ɲɯ-mŋɤm\cmn 我的伤口一阵一阵地痛\end{exemple}
\end{sous-entrée}\end{entrée}

\begin{entrée}
\vedette{\hypertarget{Ⓔsɯβzu}{\papi{ sɯβzu}}}\markboth{sɯβzu}{}
\begin{relation-sémantique}\confer{
\hyperlink{ⒺβzuⒽ1}{\textit{ \papi{βzu1}}}
}\end{relation-sémantique}\end{entrée}

\begin{entrée}
\vedette{\hypertarget{Ⓔsɯβzi}{\papi{ sɯβzi}}}\markboth{sɯβzi}{}
\begin{relation-sémantique}\confer{
\hyperlink{Ⓔβzi}{\textit{ \papi{βzi}}}
}\end{relation-sémantique}\end{entrée}

\begin{entrée}
\vedette{\hypertarget{Ⓔsɯɕku}{\papi{ sɯɕku}}}\markboth{sɯɕku}{}\classe{n}
\begin{définition}\fra poireau\end{définition}
\begin{définition}\cmn 韭葱【扁担韭】\end{définition}
\begin{exemple}\jya sɯɕku nɯ zgo khro mɤ-kɯ-mbro ɣɯ sɤjku cho mɲɤm ɯ-ŋgɯ tu-ɬoʁ ɲɯ-ŋu, ɯ-jwaʁ ɲɯ-pɣi tɕe ɲɯ-rʁom, kɯ-tɕɤr tɕe kɯ-rɲɟi ci ɲɯ-ŋu, ɯ-ru maŋe, ɯ-tho maŋe, tú-wɣ-ndza tɕe, ɕkɤphɤr cho ɲɯ-naχtɕɯɣ, ɲɯ-nɤkɤro.\cmn 
\stylefv{sɯɕku}生长在半山的白桦树和野白杨树的树林里,叶子是灰色的,粗糙,又窄又长,没有茎,没有花,吃起来和\stylefv{ɕkɤphɤr}一样,还可以。
\end{exemple}\end{entrée}

\begin{entrée}
\vedette{\hypertarget{Ⓔsɯɕke}{\papi{ sɯɕke}}}\markboth{sɯɕke}{}
\classe{vt}
\paradigme{\textit{dir :} \jya kɤ-}
\paradigme{\textit{dir :} \jya lɤ-}
\begin{définition}\fra faire brûler\end{définition}
\begin{définition}\cmn 烧焦\end{définition}
\begin{exemple}\jya qajɣi lo-tɯ-sɯɕke-t\cmn 你把馍馍烧焦了\end{exemple}
\begin{relation-sémantique}\confer{
\hyperlink{Ⓔɕke}{\textit{ \papi{ɕke}}}
}\end{relation-sémantique}\end{entrée}

\begin{entrée}
\vedette{\hypertarget{Ⓔsɯɕlɯɣ}{\papi{ sɯɕlɯɣ}}}\markboth{sɯɕlɯɣ}{}
\begin{relation-sémantique}\confer{
\hyperlink{Ⓔɕlɯɣ}{\textit{ \papi{ɕlɯɣ}}}
}\end{relation-sémantique}
\end{entrée}

\begin{entrée}
\vedette{\hypertarget{Ⓔsɯɕqhlɤt}{\papi{ sɯɕqhlɤt}}}\markboth{sɯɕqhlɤt}{}
\begin{relation-sémantique}\confer{
\hyperlink{Ⓔɕqhlɤt}{\textit{ \papi{ɕqhlɤt}}}
}\end{relation-sémantique}\end{entrée}

\begin{entrée}
\vedette{\hypertarget{Ⓔsɯɕtʂi}{\papi{ sɯɕtʂi}}}\markboth{sɯɕtʂi}{}\classe{vt}
\paradigme{\textit{dir :} \jya tɤ-}
\begin{définition}\ 
\begin{déclaration}\grammar{denom}\end{déclaration}\end{définition}
\begin{définition}\fra faire suer\end{définition}
\begin{définition}\cmn 令人流汗\end{définition}
\begin{exemple}\jya ki ta-ma ɲɯ-ɴqa tɕe tɤ́-wɣ-sɯɕtʂi-a\cmn 这个工作很辛苦,令我一身都是汗\end{exemple}
\begin{relation-sémantique}\confer{
\hyperlink{Ⓔtɯ-ɕtʂi}{\textit{ \papi{tɯ-ɕtʂi}}}
}\end{relation-sémantique}\end{entrée}

\begin{entrée}
\vedette{\hypertarget{Ⓔsɯɕɯɣra}{\papi{ sɯɕɯɣra}}}\markboth{sɯɕɯɣra}{}
\classe{vt}
\paradigme{\textit{dir :} \jya pɯ-}
\paradigme{\textit{dir :} \jya nɯ-}
\begin{définition}\ 
\begin{déclaration}\grammar{denom}\end{déclaration}\end{définition}
\begin{définition}\fra tamiser\end{définition}
\begin{définition}\cmn 筛\end{définition}
\begin{exemple}\jya ɲɤ-sɯɕɯɣra\cmn 他筛了\end{exemple}
\begin{exemple}\jya tɯjpu pjɤ-sɯɕɯɣ-ra\cmn 他筛了粮食\end{exemple}
\begin{relation-sémantique}\synonyme{
\hyperlink{Ⓔsɯxtshaʁ}{\textit{ \papi{sɯxtshaʁ}}}
}\end{relation-sémantique}
\begin{relation-sémantique}\confer{
\hyperlink{Ⓔɕɯɣra}{\textit{ \papi{ɕɯɣra}}}
}\end{relation-sémantique}\end{entrée}

\begin{entrée}
\vedette{\hypertarget{Ⓔsɯfɕɤl}{\papi{ sɯfɕɤl}}}\markboth{sɯfɕɤl}{}
\begin{relation-sémantique}\confer{
\hyperlink{Ⓔfɕɤl}{\textit{ \papi{fɕɤl}}}
}\end{relation-sémantique}\end{entrée}

\begin{entrée}
\vedette{\hypertarget{Ⓔsɯfkrɯz}{\papi{ sɯfkrɯz}}}\markboth{sɯfkrɯz}{}
\begin{relation-sémantique}\confer{
\hyperlink{Ⓔfkrɯz}{\textit{ \papi{fkrɯz}}}
}\end{relation-sémantique}\end{entrée}

\begin{entrée}
\vedette{\hypertarget{Ⓔsɯfsaŋ}{\papi{ sɯfsaŋ}}}\markboth{sɯfsaŋ}{}
\classe{vt}
\paradigme{\textit{dir :} \jya tɤ-}
\begin{définition}\fra faire des fumigations rituelles\end{définition}
\begin{définition}\cmn 求烟子
\begin{déclaration}\use{用柏树的叶子熏}\end{déclaration}\end{définition}
\begin{exemple}\jya tɤfsaŋ ɯ-kɯ-sɯfsaŋ me, tɯ-ci ɯ-kɯ-χtɕi me\cmn 柏树叶没有人为它求烟,水没有人洗\end{exemple}\begin{sous-entrée}
\vedette{\hypertarget{}{\papi{ ʑɣɤsɯfsaŋ}}}\markboth{ʑɣɤsɯfsaŋ}{}\classe{vi}
\paradigme{\textit{dir :} \jya tɤ-}
\begin{définition}\ 
\begin{déclaration}\grammar{refl}\end{déclaration}\end{définition}
\begin{définition}\fra faire des fumigations rituelles pour soi-même\end{définition}
\begin{définition}\cmn 为自己求烟子\end{définition}
\begin{exemple}\jya tɤ-ʑɣɤsɯfsaŋ-a\cmn 我为自己求了烟子\end{exemple}
\begin{relation-sémantique}\confer{
\hyperlink{Ⓔfsaŋ}{\textit{ \papi{fsaŋ}}}
}\end{relation-sémantique}
\end{sous-entrée}\end{entrée}

\begin{entrée}
\vedette{\hypertarget{Ⓔsɯfsoʁ}{\papi{ sɯfsoʁ}}}\markboth{sɯfsoʁ}{}
\begin{relation-sémantique}\confer{
\hyperlink{ⒺfsoʁⒽ2}{\textit{ \papi{fsoʁ2}}}
}\end{relation-sémantique}\end{entrée}

\begin{entrée}
\vedette{\hypertarget{Ⓔsɯftɕaʁ}{\papi{ sɯftɕaʁ}}}\markboth{sɯftɕaʁ}{}
\classe{vt}
\begin{définition}\fra abîmer, salir\end{définition}
\begin{définition}\cmn (把本来很干净的东西)弄脏\end{définition}
\begin{exemple}\jya ɯ-thoʁ kɯ-ɤχsɯko nɯ, tɤɲɟoʁɲɟi pɯ-tɯ-χtɤr tɕe pɯ-tɯ-sɯftɕaʁ\cmn 你在干净的地面上到处扔了垃圾,(把好端端的地面)弄脏了\end{exemple}
\begin{relation-sémantique}\confer{
\hyperlink{Ⓔftɕaʁ}{\textit{ \papi{ftɕaʁ}}}
}\end{relation-sémantique}\end{entrée}

\begin{entrée}
\vedette{\hypertarget{Ⓔsɯftɕɯm}{\papi{ sɯftɕɯm}}}\markboth{sɯftɕɯm}{}
\begin{relation-sémantique}\confer{
\hyperlink{Ⓔftɕɯm}{\textit{ \papi{ftɕɯm}}}
}\end{relation-sémantique}
\end{entrée}

\begin{entrée}
\vedette{\hypertarget{Ⓔsɯftshi}{\papi{ sɯftshi}}}\markboth{sɯftshi}{}
\begin{relation-sémantique}\confer{
\hyperlink{Ⓔftshi}{\textit{ \papi{ftshi}}}
}\end{relation-sémantique}\end{entrée}

\begin{entrée}
\vedette{\hypertarget{Ⓔsɯɣdɯɣ}{\papi{ sɯɣdɯɣ}}}\markboth{sɯɣdɯɣ}{}
\begin{relation-sémantique}\confer{
\hyperlink{Ⓔdɯɣ}{\textit{ \papi{dɯɣ}}}
}\end{relation-sémantique}\end{entrée}

\begin{entrée}
\vedette{\hypertarget{Ⓔsɯɣe}{\papi{ sɯɣe}}}\markboth{sɯɣe}{}\classe{vt}
\paradigme{\textit{dir :} \jya \_}
\begin{définition}\fra faire venir, inviter à venir\end{définition}
\begin{définition}\cmn 请人来\end{définition}
\begin{exemple}\jya ɯʑo kha jɤ-sɯɣe-t-a\cmn 我把他请到家里来\end{exemple}\end{entrée}

\begin{entrée}
\vedette{\hypertarget{Ⓔsɯɣjɤɣ}{\papi{ sɯɣjɤɣ}}}\markboth{sɯɣjɤɣ}{}\classe{vt}
\paradigme{\textit{dir :} \jya \_}
\begin{définition}\fra finir\end{définition}
\begin{définition}\cmn 结束
\begin{déclaration}\use{干木鸟话用得少}\end{déclaration}\end{définition}
\begin{exemple}\jya kɤ-rɤtʂɯβ kɤ-sɯɣjaɣ-a\cmn 我缝完了\end{exemple}
\begin{relation-sémantique}\synonyme{
\hyperlink{Ⓔsthɯt}{\textit{ \papi{sthɯt}}}
}\end{relation-sémantique}
\begin{relation-sémantique}\confer{
\hyperlink{Ⓔjɤɣ}{\textit{ \papi{jɤɣ}}}
}\end{relation-sémantique}\end{entrée}

\begin{entrée}
\vedette{\hypertarget{Ⓔsɯɣjɯm}{\papi{ sɯɣjɯm}}}\markboth{sɯɣjɯm}{}
\begin{relation-sémantique}\confer{
\hyperlink{ⒺjɯmⒽ1}{\textit{ \papi{jɯm}}}
}\end{relation-sémantique}\end{entrée}

\begin{entrée}
\vedette{\hypertarget{Ⓔsɯɣli}{\papi{ sɯɣli}}}\markboth{sɯɣli}{}
\begin{relation-sémantique}\confer{
\hyperlink{ⒺliⒽ3}{\textit{ \papi{li3}}}
}\end{relation-sémantique}\end{entrée}

\begin{entrée}
\vedette{\hypertarget{Ⓔsɯɣlɯɣ}{\papi{ sɯɣlɯɣ}}}\markboth{sɯɣlɯɣ}{}
\begin{relation-sémantique}\confer{
\hyperlink{Ⓔlɯɣ}{\textit{ \papi{lɯɣ}}}
}\end{relation-sémantique}\end{entrée}

\begin{entrée}
\vedette{\hypertarget{Ⓔsɯɣlɯz}{\papi{ sɯɣlɯz}}}\markboth{sɯɣlɯz}{}\classe{vt}
\paradigme{\textit{dir :} \jya nɯ-}
\begin{définition}\ 
\begin{déclaration}\grammar{caus}\end{déclaration}\end{définition}
\begin{définition}\fra laisser\end{définition}
\begin{définition}\cmn 留下一部分\end{définition}
\begin{exemple}\jya lonba ma-pɯ-tɯ-ci tɕe, tsuku nɯ-sɯɣlɯz\cmn 你不要全部打掉,留一些(茶)\end{exemple}
\begin{relation-sémantique}\confer{
\hyperlink{Ⓔlɯz}{\textit{ \papi{lɯz}}}
}\end{relation-sémantique}\end{entrée}

\begin{entrée}
\vedette{\hypertarget{Ⓔsɯɣmbɤβ}{\papi{ sɯɣmbɤβ}}}\markboth{sɯɣmbɤβ}{}
\begin{relation-sémantique}\confer{
\hyperlink{ⒺmbɤβⒽ1}{\textit{ \papi{mbɤβ1}}}
}\end{relation-sémantique}\end{entrée}

\begin{entrée}
\vedette{\hypertarget{Ⓔsɯɣmbuz}{\papi{ sɯɣmbuz}}}\markboth{sɯɣmbuz}{}
\begin{relation-sémantique}\confer{
\hyperlink{Ⓔmbuz}{\textit{ \papi{mbuz}}}
}\end{relation-sémantique}\end{entrée}

\begin{entrée}
\vedette{\hypertarget{Ⓔsɯɣnɤz}{\papi{ sɯɣnɤz}}}\markboth{sɯɣnɤz}{}
\begin{relation-sémantique}\confer{
\hyperlink{Ⓔnɤz}{\textit{ \papi{nɤz}}}
}\end{relation-sémantique}\end{entrée}

\begin{entrée}
\vedette{\hypertarget{Ⓔsɯɣndɤɣ}{\papi{ sɯɣndɤɣ}}}\markboth{sɯɣndɤɣ}{}
\begin{relation-sémantique}\confer{
\hyperlink{Ⓔndɤɣ}{\textit{ \papi{ndɤɣ}}}
}\end{relation-sémantique}\end{entrée}

\begin{entrée}
\vedette{\hypertarget{Ⓔsɯɣndʐi}{\papi{ sɯɣndʐi}}}\markboth{sɯɣndʐi}{}
\begin{relation-sémantique}\confer{
\hyperlink{Ⓔndʐi}{\textit{ \papi{ndʐi}}}
}\end{relation-sémantique}
\begin{relation-sémantique}\confer{
\hyperlink{Ⓔftʂi}{\textit{ \papi{ftʂi}}}
}\end{relation-sémantique}\end{entrée}

\begin{entrée}
\vedette{\hypertarget{Ⓔsɯɣndʐoʁ}{\papi{ sɯɣndʐoʁ}}}\markboth{sɯɣndʐoʁ}{}
\begin{relation-sémantique}\confer{
\hyperlink{Ⓔndʐoʁ}{\textit{ \papi{ndʐoʁ}}}
}\end{relation-sémantique}\end{entrée}

\begin{entrée}
\vedette{\hypertarget{Ⓔsɯɣndɯl}{\papi{ sɯɣndɯl}}}\markboth{sɯɣndɯl}{}
\begin{relation-sémantique}\confer{
\hyperlink{ⒺndɯlⒽ1}{\textit{ \papi{ndɯl1}}}
}\end{relation-sémantique}\end{entrée}

\begin{entrée}
\vedette{\hypertarget{Ⓔsɯɣndʐɯm}{\papi{ sɯɣndʐɯm}}}\markboth{sɯɣndʐɯm}{}
\begin{relation-sémantique}\confer{
\hyperlink{Ⓔndʐɯm}{\textit{ \papi{ndʐɯm}}}
}\end{relation-sémantique}
\end{entrée}

\begin{entrée}
\vedette{\hypertarget{Ⓔsɯɣndzu}{\papi{ sɯɣndzu}}}\markboth{sɯɣndzu}{}
\begin{relation-sémantique}\confer{
\hyperlink{Ⓔndzu}{\textit{ \papi{ndzu}}}
}\end{relation-sémantique}\end{entrée}

\begin{entrée}
\vedette{\hypertarget{Ⓔsɯɣndzar}{\papi{ sɯɣndzar}}}\markboth{sɯɣndzar}{}
\begin{relation-sémantique}\confer{
\hyperlink{Ⓔndzar}{\textit{ \papi{ndzar}}}
}\end{relation-sémantique}\end{entrée}

\begin{entrée}
\vedette{\hypertarget{Ⓔsɯɣndzi}{\papi{ sɯɣndzi}}}\markboth{sɯɣndzi}{}
\begin{relation-sémantique}\confer{
\hyperlink{Ⓔndzi}{\textit{ \papi{ndzi}}}
}\end{relation-sémantique}
\end{entrée}

\begin{entrée}
\vedette{\hypertarget{Ⓔsɯɣndzur}{\papi{ sɯɣndzur}}}\markboth{sɯɣndzur}{}
\classe{vt}
\paradigme{\textit{dir :} \jya tɤ-}
\begin{définition}\fra dresser, relever\end{définition}
\begin{définition}\cmn 立起来\end{définition}
\begin{exemple}\jya tɤ-jtsi tɤ-sɯɣndzur\cmn 把柱子立起来吧\end{exemple}
\begin{exemple}\jya romɲa tɤ-sɯɣndzur\cmn 把小梁立起来吧\end{exemple}
\begin{exemple}\jya tɤ-pɤtso tɤ-sɯɣndzur\cmn 把孩子扶起来吧\end{exemple}
\begin{relation-sémantique}\confer{
\hyperlink{Ⓔndzur}{\textit{ \papi{ndzur}}}
}\end{relation-sémantique}\end{entrée}

\begin{entrée}
\vedette{\hypertarget{Ⓔsɯɣndzɯr}{\papi{ sɯɣndzɯr}}}\markboth{sɯɣndzɯr}{}
\begin{relation-sémantique}\confer{
\hyperlink{Ⓔndzɯr}{\textit{ \papi{ndzɯr}}}
}\end{relation-sémantique}\end{entrée}

\begin{entrée}
\vedette{\hypertarget{Ⓔsɯɣɲaʁ}{\papi{ sɯɣɲaʁ}}}\markboth{sɯɣɲaʁ}{}\classe{vt}\acception{1}
\paradigme{\textit{dir :} \jya nɯ-}
\begin{définition}\fra rendre noir\end{définition}
\begin{définition}\cmn 使其变黑\end{définition}
\begin{exemple}\jya tɯthɯ ta-sɯɣɲaʁ\cmn 火把锅子烧黑了\end{exemple}
\begin{exemple}\jya ʁmɯrtsɯ ɯ-mat kɯ a-ɕɣa ɲɤ-sɯɣɲaʁ\cmn 果子把我的牙齿弄黑\end{exemple}
\begin{exemple}\jya tɯ-ŋga nɯ-sɯɣɲaʁ-a\cmn 我把衣服染成黑色了\end{exemple}\acception{2}
\paradigme{\textit{dir :} \jya pɯ-}
\begin{définition}\fra calomnier, médire de\end{définition}
\begin{définition}\cmn 诬蔑\end{définition}
\begin{exemple}\jya aʑo tɤ-nɯndo-t-a me tɕe, kɤ-sɯɣɲaʁ mɤ-tɯ-cha\cmn 不是我拿的,你不可以诬蔑我\end{exemple}
\begin{exemple}\jya ma-pɯ-kɯ-sɯɣɲaʁ-a\cmn 你不要诬蔑我\end{exemple}\begin{sous-entrée}
\vedette{\hypertarget{}{\papi{ asɯɣɲɯɣɲaʁ}}}\markboth{asɯɣɲɯɣɲaʁ}{}\classe{vi}
\begin{définition}\ 
\begin{déclaration}\grammar{recip}\end{déclaration}\end{définition}
\begin{définition}\fra se noircir les uns les autres\end{définition}
\begin{définition}\cmn 互相抹黑\end{définition}
\begin{exemple}\jya ɲɤ-k-ɤsɯɴqhɯɴqhi-ndʑi tɕe, ɲɤ-k-ɤsɯɣɲɯɣɲaʁ-ndʑi-ci ʑo\cmn 他们俩互相弄脏了,互相弄黑了\end{exemple}
\end{sous-entrée}\begin{sous-entrée}
\vedette{\hypertarget{}{\papi{ nɯɣɯsɯɣɲaʁ}}}\markboth{nɯɣɯsɯɣɲaʁ}{}\classe{vs}
\begin{définition}\fra être facile à noircir\end{définition}
\begin{définition}\cmn 容易抹黑\end{définition}
\begin{relation-sémantique}\confer{
\hyperlink{Ⓔɲaʁ}{\textit{ \papi{ɲaʁ}}}
}\end{relation-sémantique}
\end{sous-entrée}\begin{sous-entrée}
\vedette{\hypertarget{}{\papi{ ʑɣɤsɯɣɲaʁ}}}\markboth{ʑɣɤsɯɣɲaʁ}{}\classe{vi}
\paradigme{\textit{dir :} \jya nɯ-}
\begin{définition}\ 
\begin{déclaration}\grammar{refl}\end{déclaration}\end{définition}
\begin{définition}\fra se noircir\end{définition}
\begin{définition}\cmn 给自己抹黑\end{définition}
\begin{exemple}\jya tɤ-pɤtso ni tɤɣro kɯ ɲɯ-ʑɣɤsɯɣɲaʁ-ndʑi ʑo\cmn 小孩子玩着把自己弄黑了\end{exemple}
\end{sous-entrée}\end{entrée}

\begin{entrée}
\vedette{\hypertarget{Ⓔsɯɣɲat}{\papi{ sɯɣɲat}}}\markboth{sɯɣɲat}{}
\begin{relation-sémantique}\confer{
\hyperlink{Ⓔɲat}{\textit{ \papi{ɲat}}}
}\end{relation-sémantique}\end{entrée}

\begin{entrée}
\vedette{\hypertarget{Ⓔsɯɣɲɟo}{\papi{ sɯɣɲɟo}}}\markboth{sɯɣɲɟo}{}
\begin{relation-sémantique}\confer{
\hyperlink{Ⓔɲɟo}{\textit{ \papi{ɲɟo}}}
}\end{relation-sémantique}\end{entrée}

\begin{entrée}
\vedette{\hypertarget{Ⓔsɯɣɲo}{\papi{ sɯɣɲo}}}\markboth{sɯɣɲo}{}
\classe{vt}
\paradigme{\textit{dir :} \jya tɤ-}
\begin{définition}\ 
\begin{déclaration}\grammar{caus}\end{déclaration}\end{définition}
\begin{définition}\fra préparer\end{définition}
\begin{définition}\cmn 准备\end{définition}
\begin{exemple}\jya kɤ-ndza tɤ-sɯɣɲo-t-a\cmn 我准备了吃的东西\end{exemple}
\begin{exemple}\jya kɤ-ŋga tɤ-sɯɣɲno-t-a\cmn 我准备了吃的东西\end{exemple}
\begin{exemple}\jya kɤ-tshi tɤ-sɯɣɲo-t-a\cmn 我准备了喝的东西\end{exemple}
\begin{relation-sémantique}\confer{
\hyperlink{Ⓔɲo}{\textit{ \papi{ɲo}}}
}\end{relation-sémantique}
\begin{relation-sémantique}\synonyme{
\hyperlink{ⒺmɲoⒽ1}{\textit{ \papi{mɲo}}}
}\end{relation-sémantique}\end{entrée}

\begin{entrée}
\vedette{\hypertarget{Ⓔsɯɣru}{\papi{ sɯɣru}}}\markboth{sɯɣru}{}
\begin{relation-sémantique}\confer{
\hyperlink{ⒺruⒽ1}{\textit{ \papi{ru1}}}
}\end{relation-sémantique}\end{entrée}

\begin{entrée}
\vedette{\hypertarget{Ⓔsɯɣraʁ}{\papi{ sɯɣraʁ}}}\markboth{sɯɣraʁ}{}
\begin{relation-sémantique}\confer{
\hyperlink{ⒺraʁⒽ1}{\textit{ \papi{raʁ1}}}
}\end{relation-sémantique}\end{entrée}

\begin{entrée}
\vedette{\hypertarget{Ⓔsɯɣri}{\papi{ sɯɣri}}}\markboth{sɯɣri}{}
\classe{vt}
\paradigme{\textit{dir :} \jya pɯ-}
\begin{définition}\fra annuler\end{définition}
\begin{définition}\cmn 取消\end{définition}
\begin{exemple}\jya jɯfɕɯr ju-nɯɕe-a nɯ-sɯso-t-a ri, mɯ́j-ŋgrɯ tɕe pɯ-sɯɣri-t-a\cmn 我昨天准备回家,因为某种原因取消了(这种想法)\end{exemple}\end{entrée}

\begin{entrée}
\vedette{\hypertarget{Ⓔsɯɣrum}{\papi{ sɯɣrum}}}\markboth{sɯɣrum}{}
\begin{relation-sémantique}\confer{
\hyperlink{Ⓔwɣrum}{\textit{ \papi{wɣrum}}}
}\end{relation-sémantique}\end{entrée}

\begin{entrée}
\vedette{\hypertarget{Ⓔsɯɣrom}{\papi{ sɯɣrom}}}\markboth{sɯɣrom}{}
\begin{relation-sémantique}\confer{
\hyperlink{Ⓔrom}{\textit{ \papi{rom}}}
}\end{relation-sémantique}\end{entrée}

\begin{entrée}
\vedette{\hypertarget{Ⓔsɯɣʑaʁ}{\papi{ sɯɣʑaʁ}}}\markboth{sɯɣʑaʁ}{}\classe{vt}
\paradigme{\textit{dir :} \jya nɯ-}
\begin{définition}\fra s'entraîner\end{définition}
\begin{définition}\cmn 练习\end{définition}
\begin{exemple}\jya tɤ-scoz (jɯɣi) ɲɯ-sɯɣʑaʁ ɲɯ-ŋu\cmn 他在念书\end{exemple}
\begin{exemple}\jya jɯɣi na-sɯɣʑaʁ\cmn 他练了字\end{exemple}
\begin{exemple}\jya ɕɤmɯɣdɯ kɤ-lɤt ɲɯ-sɯɣʑaʁ\cmn 他在练习射枪\end{exemple}
\begin{exemple}\jya kɤ-ŋke na-sɯɣʑaʁ, tɯ-ŋke na-sɯɣʑaʁ\cmn 他练习走路了\end{exemple}\end{entrée}

\begin{entrée}
\vedette{\hypertarget{Ⓔsɯɣʑi}{\papi{ sɯɣʑi}}}\markboth{sɯɣʑi}{}
\begin{relation-sémantique}\confer{
\hyperlink{Ⓔʑi}{\textit{ \papi{ʑi}}}
}\end{relation-sémantique}\end{entrée}

\begin{entrée}
\vedette{\hypertarget{Ⓔsɯjaʁndzu}{\papi{ sɯjaʁndzu}}}\markboth{sɯjaʁndzu}{}\classe{vt}
\paradigme{\textit{dir :} \jya kɤ-}
\begin{définition}\fra montrer du doigt\end{définition}
\begin{définition}\cmn 指\end{définition}
\begin{exemple}\jya ma-kɤ-kɯ-sɯjaʁndzu-a\cmn 你不要指我\end{exemple}
\begin{exemple}\jya tɤtʂu ŋotɕu ɲɯ-ŋu, kɤ-sɯjaʁndze\cmn 灯在哪里,你指(给我看)\end{exemple}
\begin{exemple}\jya ɯ-mphrɯmɯ ɲɯ-mto tɕe, kɤ-kɤ-sɯjaʁndzu tu-ste ɲɯ-ŋu\cmn 他的卦应验了,非常准\end{exemple}
\begin{exemple}\jya sla tú-wɣ-sɯjaʁndzu tɕe, tɯ-rna pjɯ-ɴɢraʁ ŋu\cmn 如果用手指指月亮,耳朵就会受伤\end{exemple}\begin{sous-entrée}
\vedette{\hypertarget{}{\papi{ asɯjaʁndzɯʁndzu}}}\markboth{asɯjaʁndzɯʁndzu}{}\classe{vi}
\paradigme{\textit{dir :} \jya kɤ-}
\begin{définition}\fra se montrer du doigt les uns les autres\end{définition}
\begin{définition}\cmn 互相指\end{définition}
\end{sous-entrée}\end{entrée}

\begin{entrée}
\vedette{\hypertarget{Ⓔsɯjɣɤt}{\papi{ sɯjɣɤt}}}\markboth{sɯjɣɤt}{}
\begin{relation-sémantique}\confer{
\hyperlink{Ⓔjɣɤt}{\textit{ \papi{jɣɤt}}}
}\end{relation-sémantique}\end{entrée}

\begin{entrée}
\vedette{\hypertarget{ⒺsɯjnoⒽ1}{\papi{ sɯjno}}}\markboth{sɯjno}{}\homonyme{1}\classe{n}
\begin{définition}\fra herbe\end{définition}
\begin{définition}\cmn 草\end{définition}
\begin{relation-sémantique}\confer{
\hyperlink{Ⓔarɯsɯjno}{\textit{ \papi{arɯsɯjno}}}
}\end{relation-sémantique}\end{entrée}

\begin{entrée}
\vedette{\hypertarget{ⒺsɯjnoⒽ2}{\papi{ sɯjno}}}\markboth{sɯjno}{}\homonyme{2}\classe{vi}
\paradigme{\textit{dir :} \jya pɯ-}
\paradigme{\textit{dir :} \jya lɤ-}
\begin{définition}\fra enlever les mauvaises herbes\end{définition}
\begin{définition}\cmn 拔杂草\end{définition}
\begin{exemple}\jya tɤɕi qaj ɯ-ŋgɯ ɕ-pɯ-sɯjno-a\cmn 我去拔了在青稞和麦子里面的杂草\end{exemple}
\begin{exemple}\jya ʑ-lo-sɯjno\cmn 他去拔了杂草\end{exemple}
\begin{exemple}\jya pɯ-sɯjno-j\cmn 我们拔了杂草\end{exemple}\end{entrée}

\begin{entrée}
\vedette{\hypertarget{Ⓔsɯjnombrombro}{\papi{ sɯjnombrombro}}}\markboth{sɯjnombrombro}{}
\classe{n}
\begin{définition}\fra phasme\end{définition}
\begin{définition}\cmn 树枝虫,竹节虫【秤杆虫】\end{définition}
\begin{exemple}\jya sɯjnombrombro nɯ qajɯ ci ŋu, ɯ-phoŋbu nɯ qaʑmbri kɯ-ɤrtɯ-rtaʁ fse, ɯ-mdoʁ ldʑaŋkɯ ci tu, kɯ-pɣi ci tu, kɯ-pɣi nɯ kɯ-jpum. ftɕar tɕe kɤ-mto tu ma qartsɯ tɕe kɤ-mto me.\cmn 秤杆虫是一种虫子,身子像藤子树的枝桠。有一种是绿色的,还有一种是灰色的,灰色的那种粗壮一些。夏天常见,冬天看不到。\end{exemple}\end{entrée}

\begin{entrée}
\vedette{\hypertarget{Ⓔsɯjnoqa}{\papi{ sɯjnoqa}}}\markboth{sɯjnoqa}{}\classe{n}
\begin{définition}\fra racines\end{définition}
\begin{définition}\cmn 根\end{définition}
\end{entrée}

\begin{entrée}
\vedette{\hypertarget{Ⓔsɯjnormbjɤβ}{\papi{ sɯjnormbjɤβ}}}\markboth{sɯjnormbjɤβ}{}\classe{n}
\begin{définition}\fra herbes coupées en bottes\end{définition}
\begin{définition}\cmn 捆成一把的杂草\end{définition}
\end{entrée}

\begin{entrée}
\vedette{\hypertarget{Ⓔsɯjpɣom}{\papi{ sɯjpɣom}}}\markboth{sɯjpɣom}{}
\begin{relation-sémantique}\confer{
\hyperlink{Ⓔjpɣom}{\textit{ \papi{jpɣom}}}
}\end{relation-sémantique}\end{entrée}

\begin{entrée}
\vedette{\hypertarget{Ⓔsɯjɯ}{\papi{ sɯjɯ}}}\markboth{sɯjɯ}{}
\classe{n}
\begin{définition}\fra louche en bois\end{définition}
\begin{définition}\cmn 木瓢\end{définition}\end{entrée}

\begin{entrée}
\vedette{\hypertarget{Ⓔsɯku}{\papi{ sɯku}}}\markboth{sɯku}{}
\classe{n}
\begin{définition}\fra arbre\end{définition}
\begin{définition}\cmn 树\end{définition}\end{entrée}

\begin{entrée}
\vedette{\hypertarget{Ⓔsɯkɤcɯku}{\papi{ sɯkɤcɯku}}}\markboth{sɯkɤcɯku}{}\classe{n}
\begin{définition}\fra petites branches\end{définition}
\begin{définition}\cmn 各种树的枝桠;枝枝桠桠\end{définition}\end{entrée}

\begin{entrée}
\vedette{\hypertarget{Ⓔsɯkɤku}{\papi{ sɯkɤku}}}\markboth{sɯkɤku}{}\classe{n}
\begin{définition}\fra sommet de l'arbre\end{définition}
\begin{définition}\cmn 树梢\end{définition}
\begin{relation-sémantique}\confer{
\hyperlink{ⒺsiⒽ2}{\textit{ \papi{si2}}}
}\end{relation-sémantique}
\begin{relation-sémantique}\confer{
\hyperlink{Ⓔtɯ-ku}{\textit{ \papi{tɯ-ku}}}
}\end{relation-sémantique}
\end{entrée}

\begin{entrée}
\vedette{\hypertarget{Ⓔsɯkhɤrma}{\papi{ sɯkhɤrma}}}\markboth{sɯkhɤrma}{}\classe{vt}
\begin{définition}\fra maudire\end{définition}
\begin{définition}\cmn 诅咒\end{définition}
\begin{exemple}\jya pɯ́-wɣ-sɯkhɤrma-a\cmn 他咒了我\end{exemple}
\begin{relation-sémantique}\confer{
\hyperlink{Ⓔkhɤrma}{\textit{ \papi{khɤrma}}}
}\end{relation-sémantique}\end{entrée}

\begin{entrée}
\vedette{\hypertarget{Ⓔsɯkho}{\papi{ sɯkho}}}\markboth{sɯkho}{}
\classe{vt}
\paradigme{\textit{dir :} \jya nɯ-}
\paradigme{\textit{dir :} \jya kɤ-}
\paradigme{\textit{dir :} \jya \_}
\begin{définition}\ 
\begin{déclaration}\grammar{caus}\end{déclaration}\end{définition}
\begin{définition}\fra inciter qqn à donner un objet, faire envoyer un objet\end{définition}
\begin{définition}\cmn 使交出来;使递过来\end{définition}
\begin{exemple}\jya laχtɕha ɲɤ-sɯkho\cmn 他让他把东西递过去了\end{exemple}
\begin{exemple}\jya ɕ-kɤ-sɯkho-t-a, ʑ-nɯ-sɯkho-t-a\cmn 我让他递过来了(递过去了)\end{exemple}
\begin{relation-sémantique}\confer{
\hyperlink{ⒺkhoⒽ1}{\textit{ \papi{kho1}}}
}\end{relation-sémantique}\end{entrée}

\begin{entrée}
\vedette{\hypertarget{Ⓔsɯkhrɤt}{\papi{ sɯkhrɤt}}}\markboth{sɯkhrɤt}{}
\begin{relation-sémantique}\confer{
\hyperlink{ⒺkhrɤtⒽ1}{\textit{ \papi{khrɤt1}}}
}\end{relation-sémantique}\end{entrée}

\begin{entrée}
\vedette{\hypertarget{Ⓔsɯko}{\papi{ sɯko}}}\markboth{sɯko}{}
\begin{relation-sémantique}\confer{
\hyperlink{ⒺkoⒽ1}{\textit{ \papi{ko}}}
}\end{relation-sémantique}\end{entrée}

\begin{entrée}
\vedette{\hypertarget{Ⓔsɯkɯm}{\papi{ sɯkɯm}}}\markboth{sɯkɯm}{}\classe{n}
\begin{définition}\fra orifice pour insérer le bois (dans les socs de charrue ou les pioches)\end{définition}
\begin{définition}\cmn 装木头用的洞(铧、锄头里面)\end{définition}\end{entrée}

\begin{entrée}
\vedette{\hypertarget{Ⓔsɯkɯnthoʁ}{\papi{ sɯkɯnthoʁ}}}\markboth{sɯkɯnthoʁ}{}
\classe{n}
\begin{définition}\fra pic-vert\end{définition}
\begin{définition}\cmn 啄木鸟\end{définition}
\begin{exemple}\jya sɯkɯnthoʁ nɯ pɣa ci ŋu, qapɣɤmtɯmtɯ sɤznɤ wxti, ɯ-mtsioʁ kɯ-rɲɟɯ-rɲɟi kɯ-ɤmtɕɯ-mtɕoʁ ŋu tɕe, si ɯ-ru ku-sɯspoʁ tɕe, ɯ-ŋgɯ qajɯ ɲɯ-nɯ-tɕɤt cha, tɕe si ɯ-ŋgɯ qajɯ nɯ ʁɟa ʑo tu-ndze ŋu, sɯkɯnthoʁ ɯ-βri kɯ-ɣɯrni tu, kɯ-ɲaʁ tu, kɯ-qarŋe tu, kɯ-wɣrum tu nɯ ra akhra tɕe wuma ʑo mpɕɤr. ɯ-jme rɲɟi tsa. aʁɤndɯndɤt ʑo ju-ɕe cha.\cmn 啄木鸟是一种鸟,比戴胜大,嘴长而尖,在树干上打洞,自己能够啄出虫子,专门吃树干里的虫子。啄木鸟身子上有红、黑、黄、白等颜色的花纹,非常美丽。尾巴有点长。到处都能去。\end{exemple}\end{entrée}

\begin{entrée}
\vedette{\hypertarget{Ⓔsɯlaʁrdɤβ}{\papi{ sɯlaʁrdɤβ}}}\markboth{sɯlaʁrdɤβ}{}
\classe{vl}
\paradigme{\textit{dir :} \jya tɤ-}
\begin{définition}\ 
\begin{déclaration}\grammar{denom}\end{déclaration}\end{définition}
\begin{définition}\fra donner un coup de pied (animal)\end{définition}
\begin{définition}\cmn 踢(用前肢)\end{définition}
\begin{exemple}\jya mbro ɲɯ-sɯlaʁrdɤβ\cmn 马在踢\end{exemple}
\begin{exemple}\jya jla ɲɯ-sɯlaʁrdɤβ\cmn 犏牛在踢\end{exemple}
\begin{exemple}\jya tɤ́-wɣ-sɯlaʁrdaβ-a\cmn 它踢了我一脚\end{exemple}
\begin{relation-sémantique}\confer{
\hyperlink{Ⓔlaʁrdɤβ}{\textit{ \papi{laʁrdɤβ}}}
}\end{relation-sémantique}\end{entrée}

\begin{entrée}
\vedette{\hypertarget{Ⓔsɯlɤt}{\papi{ sɯlɤt}}}\markboth{sɯlɤt}{}
\begin{relation-sémantique}\confer{
\hyperlink{ⒺlɤtⒽ1}{\textit{ \papi{lɤt}}}
}\end{relation-sémantique}
\end{entrée}

\begin{entrée}
\vedette{\hypertarget{Ⓔsɯldʑoʁ}{\papi{ sɯldʑoʁ}}}\markboth{sɯldʑoʁ}{}
\begin{relation-sémantique}\confer{
\hyperlink{Ⓔldʑoʁ}{\textit{ \papi{ldʑoʁ}}}
}\end{relation-sémantique}\end{entrée}

\begin{entrée}
\vedette{\hypertarget{Ⓔsɯli}{\papi{ sɯli}}}\markboth{sɯli}{}\classe{n}
\begin{définition}\fra bovidé de couleur noire dont la tête, le haut du dos et la queue sont blancs\end{définition}
\begin{définition}\cmn 全身黑色,头、背梁和尾巴白色的牛\end{définition}
\end{entrée}

\begin{entrée}
\vedette{\hypertarget{Ⓔsɯluj}{\papi{ sɯluj}}}\markboth{sɯluj}{}
\classe{vt}
\paradigme{\textit{dir :} \jya \_}
\begin{définition}\fra recouvrir complètement la surface pour cacher\end{définition}
\begin{définition}\cmn 包起来;遮蔽\end{définition}
\begin{exemple}\jya laχtɕha pɯ-sɯluj\cmn 你把东西盖起来吧\end{exemple}
\begin{exemple}\jya tɯjpu pɯ-sɯluj\cmn 你把粮食盖起来吧\end{exemple}
\begin{exemple}\jya tɤrcoʁ kɯ @yangyu to-sɯluj ʑo\cmn 洋芋被泥巴覆盖了\end{exemple}
\begin{exemple}\jya raz kɯ khɯtsa tɤ-sɯluj-a\cmn 我用布把碗盖起来了\end{exemple}
\begin{relation-sémantique}\confer{
\hyperlink{Ⓔɯ-luj}{\textit{ \papi{ɯ-luj}}}
}\end{relation-sémantique}
\begin{relation-sémantique}\synonyme{
\hyperlink{Ⓔɕɯfkaβ}{\textit{ \papi{ɕɯfkaβ}}}
}\end{relation-sémantique}\end{entrée}

\begin{entrée}
\vedette{\hypertarget{Ⓔsɯmat}{\papi{ sɯmat}}}\markboth{sɯmat}{}
\classe{n}
\begin{définition}\fra fruit\end{définition}
\begin{définition}\cmn 水果\end{définition}\end{entrée}

\begin{entrée}
\vedette{\hypertarget{Ⓔsɯmciphɯt}{\papi{ sɯmciphɯt}}}\markboth{sɯmciphɯt}{}
\classe{vt}
\paradigme{\textit{dir :} \jya tɤ-}
\begin{définition}\fra cracher\end{définition}
\begin{définition}\cmn 吐唾液\end{définition}\begin{sous-entrée}
\vedette{\hypertarget{}{\papi{ nɯmciphɯt}}}\markboth{nɯmciphɯt}{}\classe{vt}
\begin{définition}\fra cracher\end{définition}
\begin{définition}\cmn 吐唾液\end{définition}
\begin{exemple}\jya tɤ-nɯmciphɯt-a\cmn 我向他吐了唾液\end{exemple}
\begin{relation-sémantique}\confer{
 \papi{mciphɯt}
}\end{relation-sémantique}
\end{sous-entrée}\end{entrée}

\begin{entrée}
\vedette{\hypertarget{Ⓔsɯmdʑɯtɕoʁ}{\papi{ sɯmdʑɯtɕoʁ}}}\markboth{sɯmdʑɯtɕoʁ}{}\classe{vi}
\paradigme{\textit{dir :} \jya tɤ-}
\begin{définition}\fra s'agenouiller d'une genou\end{définition}
\begin{définition}\cmn 半跪\end{définition}\end{entrée}

\begin{entrée}
\vedette{\hypertarget{Ⓔsɯmdʑɯtɕoʁ}{\papi{ sɯmdʑɯtɕoʁ}}}\markboth{sɯmdʑɯtɕoʁ}{}
\classe{vi}
\paradigme{\textit{dir :} \jya tɤ-}
\begin{définition}\fra s'agenouiller sur une jambe (la manière dont les femmes et les serviteurs doivent s'asseoir)\end{définition}
\begin{définition}\cmn 跪(只跪一只膝盖)
\begin{déclaration}\use{仆人和家庭妇女坐的姿势}\end{déclaration}\end{définition}
\begin{exemple}\jya tɤ-sɯmdʑɯtɕoʁ-a\cmn 我跪下了\end{exemple}
\begin{exemple}\jya to-sɯmdʑɯtɕoʁ\cmn 她跪下了\end{exemple}\end{entrée}

\begin{entrée}
\vedette{\hypertarget{Ⓔsɯmɟa}{\papi{ sɯmɟa}}}\markboth{sɯmɟa}{}\classe{n}
\begin{définition}\fra allumage d'un feu\end{définition}
\begin{définition}\cmn 引燃;点火\end{définition}
\begin{exemple}\jya sɯmɟa tɤ-βzu-t-a\cmn 我点了火\end{exemple}
\begin{relation-sémantique}\confer{
\hyperlink{Ⓔmɟa}{\textit{ \papi{mɟa}}}
}\end{relation-sémantique}\end{entrée}

\begin{entrée}
\vedette{\hypertarget{Ⓔsɯmɟa}{\papi{ sɯmɟa}}}\markboth{sɯmɟa}{}
\begin{relation-sémantique}\confer{
\hyperlink{Ⓔmɟa}{\textit{ \papi{mɟa}}}
}\end{relation-sémantique}\end{entrée}

\begin{entrée}
\vedette{\hypertarget{Ⓔsɯmnɤr}{\papi{ sɯmnɤr}}}\markboth{sɯmnɤr}{}
\classe{n}
\begin{définition}\fra pensée\end{définition}
\begin{définition}\cmn 想法\end{définition}\end{entrée}

\begin{entrée}
\vedette{\hypertarget{Ⓔsɯmɲo}{\papi{ sɯmɲo}}}\markboth{sɯmɲo}{}
\begin{relation-sémantique}\confer{
\hyperlink{ⒺmɲoⒽ1}{\textit{ \papi{mɲo1}}}
}\end{relation-sémantique}
\end{entrée}

\begin{entrée}
\vedette{\hypertarget{Ⓔsɯmɲo}{\papi{ sɯmɲo}}}\markboth{sɯmɲo}{}
\classe{vt}\acception{1}
\paradigme{\textit{dir :} \jya nɯ-}
\begin{définition}\fra comprendre\end{définition}
\begin{définition}\cmn 理解\end{définition}
\begin{exemple}\jya nɯ ɲɯ-sɯmɲam-a ɕti\cmn 我可以理解他\end{exemple}
\begin{exemple}\jya ɲɯ́-wɣ-sɯmɲo-a\cmn 他可以理解我\end{exemple}
\begin{exemple}\jya jiɕqha nɯ ɲɯ-nɯzdɯɣ tɕe, aj ɲɯ-sɯmɲam-a ɕti\cmn 他很担心,我可以理解他\end{exemple}
\begin{exemple}\jya ta-sɯmɲo\cmn 我可以理解你\end{exemple}\acception{2}
\begin{définition}\fra faire préparer\end{définition}
\begin{définition}\cmn 让……准备好\end{définition}
\begin{relation-sémantique}\confer{
\hyperlink{ⒺmɲoⒽ1}{\textit{ \papi{mɲo1}}}
}\end{relation-sémantique}\end{entrée}

\begin{entrée}
\vedette{\hypertarget{Ⓔsɯmŋɤn}{\papi{ sɯmŋɤn}}}\markboth{sɯmŋɤn}{}\classe{n}
\begin{définition}\fra doute\end{définition}
\begin{définition}\cmn 疑心\end{définition}
\begin{exemple}\jya sɯmŋɤn ma-tɤ-tɯ-βze\cmn 你不要起疑心\end{exemple}
\begin{relation-sémantique}\confer{
\hyperlink{Ⓔnɯsɯmŋɤn}{\textit{ \papi{nɯsɯmŋɤn}}}
}\end{relation-sémantique}\end{entrée}

\begin{entrée}
\vedette{\hypertarget{Ⓔsɯmoʁ}{\papi{ sɯmoʁ}}}\markboth{sɯmoʁ}{}
\begin{relation-sémantique}\confer{
\hyperlink{Ⓔmoʁ}{\textit{ \papi{moʁ}}}
}\end{relation-sémantique}\end{entrée}

\begin{entrée}
\vedette{\hypertarget{Ⓔsɯmphru}{\papi{ sɯmphru}}}\markboth{sɯmphru}{}
\classe{n}
\begin{définition}\fra le bois que l'on n'a pas encore fini de couper\end{définition}
\begin{définition}\cmn 尚未砍完的柴\end{définition}
\begin{exemple}\jya a-sɯmphru\cmn 我没有砍完的那一部分\end{exemple}
\begin{relation-sémantique}\confer{
\hyperlink{ⒺsiⒽ1}{\textit{ \papi{si1}}}
}\end{relation-sémantique}
\begin{relation-sémantique}\confer{
\hyperlink{Ⓔɯ-mphru}{\textit{ \papi{ɯ-mphru}}}
}\end{relation-sémantique}\end{entrée}

\begin{entrée}
\vedette{\hypertarget{Ⓔsɯmphrɤt}{\papi{ sɯmphrɤt}}}\markboth{sɯmphrɤt}{}
\begin{relation-sémantique}\confer{
\hyperlink{Ⓔmphrɤt}{\textit{ \papi{mphrɤt}}}
}\end{relation-sémantique}\end{entrée}

\begin{entrée}
\vedette{\hypertarget{Ⓔsɯmphɯ}{\papi{ sɯmphɯ}}}\markboth{sɯmphɯ}{}\classe{n}
\begin{définition}\fra outil pour casser les mottes de terre\end{définition}
\begin{définition}\cmn 土巴捶\end{définition}
\begin{exemple}\jya sɯmphɯ ɯ-ru\cmn 土巴捶的把子\end{exemple}
\begin{exemple}\jya sɯmphɯ ɯ-pɤl\cmn 土巴捶的刀\end{exemple}\end{entrée}

\begin{entrée}
\vedette{\hypertarget{Ⓔsɯmsɯm}{\papi{ sɯmsɯm}}}\markboth{sɯmsɯm}{}
\classe{idph.2}
\begin{définition}\fra formant une couche fine\end{définition}
\begin{définition}\cmn 构成了薄薄的一层\end{définition}
\begin{exemple}\jya tɤjpa sɯmsɯm ko-lɤt\cmn 下了薄薄的一层雪\end{exemple}
\begin{exemple}\jya nɤ-βri ɯ-rme sɯmsɯm ci ɣɤʑu\cmn 你身上有点毛茸茸的\end{exemple}
\begin{exemple}\jya tɤ-tsrɯ sɯmsɯm ʑo to-ɬoʁ\cmn 新发的芽,(很短)仿佛看得见\end{exemple}
\begin{relation-sémantique}\confer{
\hyperlink{Ⓔɕɯmɕɯm}{\textit{ \papi{ɕɯmɕɯm}}}
}\end{relation-sémantique}\end{entrée}

\begin{entrée}
\vedette{\hypertarget{Ⓔsɯmtɕɤn}{\papi{ sɯmtɕɤn}}}\markboth{sɯmtɕɤn}{}
\classe{n}
\begin{définition}\fra animaux\end{définition}
\begin{définition}\cmn 动物
\begin{déclaration} \étymologie{\papi{sems.tɕan}}\end{déclaration}\end{définition}\end{entrée}

\begin{entrée}
\vedette{\hypertarget{Ⓔsɯmtɕɤnrtazoʁ}{\papi{ sɯmtɕɤnrtazoʁ}}}\markboth{sɯmtɕɤnrtazoʁ}{}
\classe{n}
\begin{définition}\fra animaux domestiques\end{définition}
\begin{définition}\cmn 牲畜
\begin{déclaration} \étymologie{\papi{sems.tɕan rta.zog}}\end{déclaration}\end{définition}\end{entrée}

\begin{entrée}
\vedette{\hypertarget{Ⓔsɯmtɕɯr}{\papi{ sɯmtɕɯr}}}\markboth{sɯmtɕɯr}{}
\classe{vt}
\paradigme{\textit{dir :} \jya \_}
\begin{définition}\fra faire tourner\end{définition}
\begin{définition}\cmn 使转动
\begin{déclaration}\grammar{caus}\end{déclaration}\end{définition}
\begin{exemple}\jya laʁŋkhɤr thɯ-sɯmtɕɯr-a\cmn 我把转经筒转动了\end{exemple}
\begin{exemple}\jya mkhɯrlu kɤ-sɯmtɕɯr\cmn 把轮子转动\end{exemple}
\begin{exemple}\jya nɯ-sɯmtɕɯr-a\cmn 我把它转动了\end{exemple}
\begin{exemple}\jya pa-sɯmtɕɯr\cmn 他把它转动了\end{exemple}
\begin{exemple}\jya tɯ-ku tú-wɣ-sɯmtɕɯr tsa ra\cmn 要动脑筋\end{exemple}
\begin{relation-sémantique}\confer{
\hyperlink{Ⓔmtɕɯr}{\textit{ \papi{mtɕɯr}}}
}\end{relation-sémantique}\end{entrée}

\begin{entrée}
\vedette{\hypertarget{Ⓔsɯmto}{\papi{ sɯmto}}}\markboth{sɯmto}{}
\begin{relation-sémantique}\confer{
 \papi{mto}
}\end{relation-sémantique}\end{entrée}

\begin{entrée}
\vedette{\hypertarget{Ⓔsɯmtshɤm}{\papi{ sɯmtshɤm}}}\markboth{sɯmtshɤm}{}\classe{vt}
\paradigme{\textit{dir :} \jya pɯ-}
\begin{définition}\ 
\begin{déclaration}\grammar{caus}\end{déclaration}\end{définition}
\begin{définition}\fra informer\end{définition}
\begin{définition}\cmn 通知\end{définition}
\begin{exemple}\jya ɯʑo kha mɯ́j-rɤʑi tɕe pɯ-sɯmtsham-a\cmn 因为他没在家里,我就把事情转告他了\end{exemple}
\begin{relation-sémantique}\confer{
\hyperlink{Ⓔmtshɤm}{\textit{ \papi{mtshɤm}}}
}\end{relation-sémantique}\end{entrée}

\begin{entrée}
\vedette{\hypertarget{Ⓔsɯmtshɤt}{\papi{ sɯmtshɤt}}}\markboth{sɯmtshɤt}{}
\classe{vt}
\paradigme{\textit{dir :} \jya tɤ-}
\paradigme{\textit{dir :} \jya \_}
\begin{définition}\fra remplir\end{définition}
\begin{définition}\cmn 填满\end{définition}
\begin{exemple}\jya tɤ-fkɯm ta-sɯmtshɤt\cmn 他把袋子填满了\end{exemple}
\begin{exemple}\jya kha nɯ-sɯmtshɤt-i\cmn 我们坐满了房间\end{exemple}
\begin{relation-sémantique}\confer{
\hyperlink{Ⓔmtshɤt}{\textit{ \papi{mtshɤt}}}
}\end{relation-sémantique}\end{entrée}

\begin{entrée}
\vedette{\hypertarget{Ⓔsɯmtshoŋ}{\papi{ sɯmtshoŋ}}}\markboth{sɯmtshoŋ}{}
\begin{relation-sémantique}\confer{
\hyperlink{Ⓔmtshoŋ}{\textit{ \papi{mtshoŋ}}}
}\end{relation-sémantique}\end{entrée}

\begin{entrée}
\vedette{\hypertarget{Ⓔsɯmtshɯβ}{\papi{ sɯmtshɯβ}}}\markboth{sɯmtshɯβ}{}
\begin{relation-sémantique}\confer{
\hyperlink{Ⓔmtshɯβ}{\textit{ \papi{mtshɯβ}}}
}\end{relation-sémantique}\end{entrée}

\begin{entrée}
\vedette{\hypertarget{Ⓔsɯmtso}{\papi{ sɯmtso}}}\markboth{sɯmtso}{}
\classe{n}
\begin{définition}\fra récit de ce qui s'est passé depuis que l'on s'est séparé\end{définition}
\begin{définition}\cmn 分开以来的经过\end{définition}
\begin{exemple}\jya ɯ-sɯmtso na-βzu\cmn 他给他讲了(分开以后发生的事情)\end{exemple}
\begin{exemple}\jya ɯ-sɯmtso nɯ-βzu-t-a\cmn 我给他讲了\end{exemple}
\begin{exemple}\jya a-sɯmtso na-βzu\cmn 他给我讲了\end{exemple}\end{entrée}

\begin{entrée}
\vedette{\hypertarget{Ⓔsɯmtsɯr}{\papi{ sɯmtsɯr}}}\markboth{sɯmtsɯr}{}
\begin{relation-sémantique}\confer{
\hyperlink{Ⓔmtsɯr}{\textit{ \papi{mtsɯr}}}
}\end{relation-sémantique}\end{entrée}

\begin{entrée}
\vedette{\hypertarget{Ⓔsɯmɯzdɯɣ}{\papi{ sɯmɯzdɯɣ}}}\markboth{sɯmɯzdɯɣ}{}
\classe{n}
\begin{définition}\fra inquiétude\end{définition}
\begin{définition}\cmn 操心
\begin{déclaration} \étymologie{\papi{sems.sdug}}\end{déclaration}\end{définition}
\begin{relation-sémantique}\confer{
\hyperlink{Ⓔnɯsɯmɯzdɯɣ}{\textit{ \papi{nɯsɯmɯzdɯɣ}}}
}\end{relation-sémantique}\end{entrée}

\begin{entrée}
\vedette{\hypertarget{Ⓔsɯndɤrmbjom}{\papi{ sɯndɤrmbjom}}}\markboth{sɯndɤrmbjom}{}
\begin{relation-sémantique}\confer{
\hyperlink{Ⓔndɤrmbjom}{\textit{ \papi{ndɤrmbjom}}}
}\end{relation-sémantique}\end{entrée}

\begin{entrée}
\vedette{\hypertarget{Ⓔsɯndo}{\papi{ sɯndo}}}\markboth{sɯndo}{}
\begin{relation-sémantique}\confer{
\hyperlink{Ⓔndo}{\textit{ \papi{ndo}}}
}\end{relation-sémantique}\end{entrée}

\begin{entrée}
\vedette{\hypertarget{Ⓔsɯndza}{\papi{ sɯndza}}}\markboth{sɯndza}{}
\begin{relation-sémantique}\confer{
\hyperlink{Ⓔndza}{\textit{ \papi{ndza}}}
}\end{relation-sémantique}\end{entrée}

\begin{entrée}
\vedette{\hypertarget{Ⓔsɯndzɯ}{\papi{ sɯndzɯ}}}\markboth{sɯndzɯ}{}
\begin{relation-sémantique}\confer{
\hyperlink{Ⓔndzɯ}{\textit{ \papi{ndzɯ}}}
}\end{relation-sémantique}\end{entrée}

\begin{entrée}
\vedette{\hypertarget{Ⓔsɯndzɯpe}{\papi{ sɯndzɯpe}}}\markboth{sɯndzɯpe}{}
\classe{vi}
\paradigme{\textit{dir :} \jya nɯ-}
\begin{définition}\fra s'asseoir par terre avec les deux jambes l'une sur l'autre en travers (la manière dont les femmes doivent s'asseoir lorsqu'elle n'ont pas de travail à faire)\end{définition}
\begin{définition}\cmn 双腿斜着坐(藏族妇女坐的姿势)\end{définition}
\begin{exemple}\jya nɯtɕu nɯ-sɯndzɯpe\cmn 你坐在那里\end{exemple}
\begin{exemple}\jya nɯ-sɯndzɯpe-a\cmn 我坐下了\end{exemple}\end{entrée}

\begin{entrée}
\vedette{\hypertarget{Ⓔsɯndʑaʁskɯsko}{\papi{ sɯndʑaʁskɯsko}}}\markboth{sɯndʑaʁskɯsko}{} (\variante{sɯndʑaʁfskɯfsko}) 
\classe{vi}
\paradigme{\textit{dir :} \jya nɯ-}
\begin{définition}\fra s'étirer\end{définition}
\begin{définition}\cmn 舒展筋骨;伸懒腰\end{définition}
\begin{exemple}\jya tɤ-pɤtso ɲɯ-sɯndʑaʁfskɯfsko ɲɯ-cha\cmn 小孩子会舒展筋骨\end{exemple}
\begin{exemple}\jya nɯ-sɯndʑaʁfskɯfsko-a\cmn 我舒展了筋骨\end{exemple}\end{entrée}

\begin{entrée}
\vedette{\hypertarget{Ⓔsɯndʑutɤndʑu}{\papi{ sɯndʑutɤndʑu}}}\markboth{sɯndʑutɤndʑu}{}
\classe{n}
\begin{définition}\fra petites pousses d'arbre autour des champs cultivés\end{définition}
\begin{définition}\cmn 田周围长的树芽\end{définition}\end{entrée}

\begin{entrée}
\vedette{\hypertarget{Ⓔsɯngrɯβ}{\papi{ sɯngrɯβ}}}\markboth{sɯngrɯβ}{}
\begin{relation-sémantique}\confer{
\hyperlink{Ⓔngrɯβ}{\textit{ \papi{ngrɯβ}}}
}\end{relation-sémantique}\end{entrée}

\begin{entrée}
\vedette{\hypertarget{Ⓔsɯntɕhɣaʁ}{\papi{ sɯntɕhɣaʁ}}}\markboth{sɯntɕhɣaʁ}{}
\begin{relation-sémantique}\confer{
\hyperlink{Ⓔntɕhɣaʁ}{\textit{ \papi{ntɕhɣaʁ}}}
}\end{relation-sémantique}\end{entrée}

\begin{entrée}
\vedette{\hypertarget{Ⓔsɯnthɯ}{\papi{ sɯnthɯ}}}\markboth{sɯnthɯ}{}
\classe{n}
\begin{définition}\fra emplanture\end{définition}
\begin{définition}\cmn 榫头
\begin{déclaration} \étymologie{\papi{\stylefn{榫头}}}\end{déclaration}\end{définition}\end{entrée}

\begin{entrée}
\vedette{\hypertarget{Ⓔsɯntshɤβ}{\papi{ sɯntshɤβ}}}\markboth{sɯntshɤβ}{}
\begin{relation-sémantique}\confer{
\hyperlink{Ⓔntshɤβ}{\textit{ \papi{ntshɤβ}}}
}\end{relation-sémantique}\end{entrée}

\begin{entrée}
\vedette{\hypertarget{Ⓔsɯŋgi}{\papi{ sɯŋgi}}}\markboth{sɯŋgi}{}
\classe{n}
\begin{définition}\fra lion\end{définition}
\begin{définition}\cmn 狮子
\begin{déclaration} \étymologie{\papi{seŋ.ge}}\end{déclaration}\end{définition}\end{entrée}

\begin{entrée}
\vedette{\hypertarget{Ⓔsɯŋgo}{\papi{ sɯŋgo}}}\markboth{sɯŋgo}{}
\begin{relation-sémantique}\confer{
\hyperlink{Ⓔŋgo}{\textit{ \papi{ŋgo}}}
}\end{relation-sémantique}\end{entrée}

\begin{entrée}
\vedette{\hypertarget{Ⓔsɯŋgɯ}{\papi{ sɯŋgɯ}}}\markboth{sɯŋgɯ}{}
\classe{n}
\begin{définition}\fra forêt\end{définition}
\begin{définition}\cmn 森林\end{définition}
\begin{relation-sémantique}\confer{
\hyperlink{ⒺsiⒽ1}{\textit{ \papi{si1}}}
}\end{relation-sémantique}
\begin{relation-sémantique}\confer{
\hyperlink{Ⓔnɯsɯŋgɯ}{\textit{ \papi{nɯsɯŋgɯ}}}
}\end{relation-sémantique}
\begin{relation-sémantique}\confer{
\hyperlink{Ⓔsɯŋgɯɟu}{\textit{ \papi{sɯŋgɯɟu}}}
}\end{relation-sémantique}
\begin{relation-sémantique}\confer{
 \papi{sɯŋgɯnaχtɕin}
}\end{relation-sémantique}\end{entrée}

\begin{entrée}
\vedette{\hypertarget{Ⓔsɯŋgɯɟu}{\papi{ sɯŋgɯɟu}}}\markboth{sɯŋgɯɟu}{}\classe{n}
\begin{définition}\fra une espèce d'arbrisseau\end{définition}
\begin{définition}\cmn 灌木的一种\end{définition}
\begin{exemple}\jya sɯŋgɯ ɟu nɯ ɯ-jwaʁ kɯ-ɤrtɯm tɕe kɯ-ɤmtɕoʁ ŋu. ɯ-ru cho ɯ-jwaʁ qartsɯmɤftɕar ʑo anɯrŋi ɕti, ɯ-mat kɯnɤ kɯ-ɤrŋi ŋu thɯ-tɯt tɕe chɯ-ɲaʁ, ɯ-ru wuma ʑo mpɕu, ɲɯ́-wɣ-phɯt tɕe ndoʁ, tu-wxti wuma mɤ-cha tɕe, ndɯβ tɕe, zɣɤmbu ɲɯ́-wɣ-βzu tɕe pe.\cmn 
\stylefv{sɯŋgɯɟu}有圆而尖的叶子。树干和叶子一年四季都是绿的,果子也是绿色的,成熟后变黑。树干很光滑,折断的时候是脆的,长不大,细小,可以用来做扫把
\end{exemple}\end{entrée}

\begin{entrée}
\vedette{\hypertarget{Ⓔsɯŋgɯnaχtɕɯn}{\papi{ sɯŋgɯnaχtɕɯn}}}\markboth{sɯŋgɯnaχtɕɯn}{}\classe{n}
\begin{définition}\fra forêt primaire de la montagne\end{définition}
\begin{définition}\cmn 深山老林
\begin{déclaration} \étymologie{\papi{nags.tɕʰen}}\end{déclaration}\end{définition}
\end{entrée}

\begin{entrée}
\vedette{\hypertarget{Ⓔsɯŋgɯpɤjka}{\papi{ sɯŋgɯpɤjka}}}\markboth{sɯŋgɯpɤjka}{}\classe{n}
\begin{définition}\fra courge sauvage\end{définition}
\begin{définition}\cmn 野生瓜子的一种\end{définition}
\begin{exemple}\jya sɯŋgɯ pɤjka nɯ sɯjno ci ɯ-ru kɯ-zɯ-zri tɕe kɯ-ɤrqhi tsa jɯ-ɕe tɕe, kɯmaʁ si ɯ-taʁ cho sɯjno ɯ-ru ɯ-taʁ tu-ortɯrtɤβ tɕe, ɲɯ-rɯmɯntoʁ tɕe ɯ-mat ku-tshoʁ ɲɯ-ŋu. ɯ-mɯntoʁ nɯ @laba kɯ-fse ci ŋu, tɕe kɯ-qarŋe ɲɯ-ŋu. ɯ-mat nɯ @huanggua kɯ-fsɯ-fse ci ŋu, kɯ-xtɕi tsa ɲɯ-ŋu. ɯ-jwaʁ nɯ pɤjka ɯ-jwaʁ cho ɲɯ-naχtɕɯɣ, ɯ-jwaʁ ɯ-taʁ ɯ-mdzu ɣɤʑu, ftɕar tɕe tu-ɬoʁ, qartsɯ tɕe pjɯ-rom ɲɯ-ŋu.\cmn 
\stylefv{sɯŋgɯpɤjka}是一种草,茎很长,可以爬到很远,可以缠在其他树干和草的茎上开花结果。花像喇叭一样,是黄色的。果实很像黄瓜,但小一点。叶子和南瓜的叶子一样,叶子上还长有刺,春夏生长,冬天枯萎。
\end{exemple}
\end{entrée}

\begin{entrée}
\vedette{\hypertarget{Ⓔsɯŋgɯrmɤβja}{\papi{ sɯŋgɯrmɤβja}}}\markboth{sɯŋgɯrmɤβja}{}
\classe{n}
\begin{définition}\fra faisan (lophophorus lhuysii)\end{définition}
\begin{définition}\cmn 绿尾虹雉\end{définition}\end{entrée}

\begin{entrée}
\vedette{\hypertarget{Ⓔsɯŋsɯŋ}{\papi{ sɯŋsɯŋ}}}\markboth{sɯŋsɯŋ}{}
\classe{idph.2}
\begin{définition}\fra blanc, pur, propre\end{définition}
\begin{définition}\cmn 形容又白又清洁的样子\end{définition}
\begin{exemple}\jya ɯ-khɯtsa sɯŋsɯŋ to-nɯntsɯɣ\cmn 他把碗舔得很干净了\end{exemple}
\begin{exemple}\jya sɯŋsɯŋ ʑo to-ɕkɯt\cmn 他吃得很干净了\end{exemple}
\begin{exemple}\jya smɤɣ ɲɯ-wɣrum sɯŋsɯŋ ʑo\cmn 羊毛又白又清洁\end{exemple}\end{entrée}

\begin{entrée}
\vedette{\hypertarget{Ⓔsɯɴɢoʁ}{\papi{ sɯɴɢoʁ}}}\markboth{sɯɴɢoʁ}{}\classe{n}
\begin{définition}\fra bois mort\end{définition}
\begin{définition}\cmn 干柴\end{définition}\end{entrée}

\begin{entrée}
\vedette{\hypertarget{Ⓔsɯɴqhi}{\papi{ sɯɴqhi}}}\markboth{sɯɴqhi}{}
\begin{relation-sémantique}\confer{
\hyperlink{Ⓔɴqhi}{\textit{ \papi{ɴqhi}}}
}\end{relation-sémantique}\end{entrée}

\begin{entrée}
\vedette{\hypertarget{ⒺsɯpaⒽ2}{\papi{ sɯpa}}}\markboth{sɯpa}{}\homonyme{2}\classe{n}
\begin{définition}\fra bois de chauffage découpé\end{définition}
\begin{définition}\cmn 劈开了的木柴【柴划子】\end{définition}
\begin{relation-sémantique}\antonyme{
\hyperlink{Ⓔɕɯrdɯm}{\textit{ \papi{ɕɯrdɯm}}}
}\end{relation-sémantique}\end{entrée}

\begin{entrée}
\vedette{\hypertarget{ⒺsɯpaⒽ1}{\papi{ sɯpa}}}\markboth{sɯpa}{}\homonyme{1}\classe{vt}
\paradigme{\textit{dir :} \jya tɤ-}
\begin{définition}\fra considérer\end{définition}
\begin{définition}\cmn 认为,当作\end{définition}
\begin{exemple}\jya ɯʑo kɯ-ɕqraʁ tu-sɯpe-a pɯ-ŋu ri, mɯ́j-ɕqraʁ\cmn 我以为他很聪明,其实他不聪明\end{exemple}
\begin{sous-entrée}
\vedette{\hypertarget{}{\papi{ ci,sɯpa}}}\markboth{ci,sɯpa}{}\acception{1}
\begin{définition}\fra mettre ensemble\end{définition}
\begin{définition}\cmn 加起来\end{définition}
\begin{exemple}\jya jɯfɕɯr aʑo ʁnɯ-ɣjɤn kɤ-rŋgɯ-a tɕe, ci tú-wɣ-sɯpa tɕe tɯtshot kɯngɯt jamar pɯ-nɯʑɯβ-a\cmn 我昨天睡了两次,加起来一共睡了大概九个小时。\end{exemple}\acception{2}
\begin{définition}\fra marier\end{définition}
\begin{définition}\cmn 让……结婚\end{définition}
\begin{relation-sémantique}\ComponentA{\classe{num}
\hyperlink{ⒺciⒽ1}{\textit{ \papi{ci}}}
}\end{relation-sémantique}
\begin{relation-sémantique}\ComponentB{\classe{vt}
\hyperlink{ⒺsɯpaⒽ1}{\textit{ \papi{sɯpa}}}
}\end{relation-sémantique}
\end{sous-entrée}\end{entrée}

\begin{entrée}
\vedette{\hypertarget{Ⓔsɯpɣo}{\papi{ sɯpɣo}}}\markboth{sɯpɣo}{}
\classe{n}
\begin{définition}\fra une pile de bois\end{définition}
\begin{définition}\cmn 柴垛子【柴码子】\end{définition}\end{entrée}

\begin{entrée}
\vedette{\hypertarget{Ⓔsɯphɯ}{\papi{ sɯphɯ}}}\markboth{sɯphɯ}{}
\classe{n}
\begin{définition}\fra arbre\end{définition}
\begin{définition}\cmn 树
\begin{déclaration} \étymologie{\papi{ɕiŋ}}\end{déclaration}\end{définition}\end{entrée}

\begin{entrée}
\vedette{\hypertarget{Ⓔsɯphɯt}{\papi{ sɯphɯt}}}\markboth{sɯphɯt}{}\classe{n}
\begin{définition}\fra fait de couper du bois\end{définition}
\begin{définition}\cmn 砍柴\end{définition}
\begin{exemple}\jya qartsɯ tɕe, sɯphɯt kɤ-βzu ra\cmn 到了冬天要砍柴\end{exemple}
\begin{exemple}\jya a-βɣo sɯphɯt wuma ʑo χɕu\cmn 我的叔叔砍柴很厉害\end{exemple}
\begin{relation-sémantique}\confer{
\hyperlink{ⒺsiⒽ1}{\textit{ \papi{si1}}}
}\end{relation-sémantique}
\begin{relation-sémantique}\confer{
\hyperlink{Ⓔphɯt}{\textit{ \papi{phɯt}}}
}\end{relation-sémantique}
\begin{relation-sémantique}\confer{
\hyperlink{Ⓔɣɯsɯphɯt}{\textit{ \papi{ɣɯsɯphɯt}}}
}\end{relation-sémantique}\end{entrée}

\begin{entrée}
\vedette{\hypertarget{Ⓔsɯprɤt}{\papi{ sɯprɤt}}}\markboth{sɯprɤt}{}
\begin{relation-sémantique}\confer{
\hyperlink{Ⓔprɤt}{\textit{ \papi{prɤt}}}
}\end{relation-sémantique}\end{entrée}

\begin{entrée}
\vedette{\hypertarget{Ⓔsɯqartsɯ}{\papi{ sɯqartsɯ}}}\markboth{sɯqartsɯ}{}
\classe{vl}
\paradigme{\textit{dir :} \jya tɤ-}
\begin{définition}\fra donner un coup de pied\end{définition}
\begin{définition}\cmn 踢(用后肢)\end{définition}
\begin{exemple}\jya jla ɲɯ-sɯqartsɯ (=tɯ-qartsɯ ta-lɤt)\cmn 犏牛在踢\end{exemple}
\begin{exemple}\jya mdzadi tɤ-sɯqartsɯ tɕe tɯ-mɯ mɤ-ɕaβ\cmn 跳蚤再踢也登不上天(你有再大的能力也影响不了人家)\end{exemple}
\begin{relation-sémantique}\synonyme{
\hyperlink{Ⓔsɯlaʁrdɤβ}{\textit{ \papi{sɯlaʁrdɤβ}}}
}\end{relation-sémantique}
\begin{relation-sémantique}\confer{
\hyperlink{Ⓔtɯ-qartsɯ}{\textit{ \papi{tɯ-qartsɯ}}}
}\end{relation-sémantique}\end{entrée}

\begin{entrée}
\vedette{\hypertarget{Ⓔsɯrɤt}{\papi{ sɯrɤt}}}\markboth{sɯrɤt}{}
\begin{relation-sémantique}\confer{
\hyperlink{Ⓔrɤt}{\textit{ \papi{rɤt}}}
}\end{relation-sémantique}
\end{entrée}

\begin{entrée}
\vedette{\hypertarget{Ⓔsɯrdɤl}{\papi{ sɯrdɤl}}}\markboth{sɯrdɤl}{}
\begin{relation-sémantique}\confer{
\hyperlink{Ⓔrdɤl}{\textit{ \papi{rdɤl}}}
}\end{relation-sémantique}\end{entrée}

\begin{entrée}
\vedette{\hypertarget{Ⓔsɯrɟɯɣ}{\papi{ sɯrɟɯɣ}}}\markboth{sɯrɟɯɣ}{}
\begin{relation-sémantique}\confer{
\hyperlink{ⒺrɟɯɣⒽ1}{\textit{ \papi{rɟɯɣ}}}
}\end{relation-sémantique}\end{entrée}

\begin{entrée}
\vedette{\hypertarget{Ⓔsɯrku}{\papi{ sɯrku}}}\markboth{sɯrku}{}
\begin{relation-sémantique}\confer{
\hyperlink{Ⓔrku}{\textit{ \papi{rku}}}
}\end{relation-sémantique}\end{entrée}

\begin{entrée}
\vedette{\hypertarget{Ⓔsɯrkɤz}{\papi{ sɯrkɤz}}}\markboth{sɯrkɤz}{}
\classe{n}
\begin{définition}\fra gravures sur bois\end{définition}
\begin{définition}\cmn 刻的木板\end{définition}
\begin{exemple}\jya sɯrkɤz ta-βzu\cmn 他刻了木板\end{exemple}\end{entrée}

\begin{entrée}
\vedette{\hypertarget{Ⓔsɯrma}{\papi{ sɯrma}}}\markboth{sɯrma}{}
\classe{vt}
\paradigme{\textit{dir :} \jya kɤ-}
\paradigme{\textit{dir :} \jya pɯ-}\acception{1}
\begin{définition}\fra laisser habiter\end{définition}
\begin{définition}\cmn 留宿\end{définition}
\begin{exemple}\jya jɯɣmɯr kutɕu kɤ-sɯrma-t-a\cmn 今天晚上让他在这里睡了\end{exemple}\acception{2}
\begin{définition}\fra enterrer\end{définition}
\begin{définition}\cmn 掩埋\end{définition}
\begin{exemple}\jya smi ka-sɯrma\cmn (用草木灰)把火盖住了\end{exemple}
\begin{relation-sémantique}\synonyme{
\hyperlink{Ⓔsɯrʑaʁ}{\textit{ \papi{sɯrʑaʁ}}}
}\end{relation-sémantique}\end{entrée}

\begin{entrée}
\vedette{\hypertarget{Ⓔsɯrmbɣotɯm}{\papi{ sɯrmbɣotɯm}}}\markboth{sɯrmbɣotɯm}{}\classe{vi}
\begin{définition}\fra s'asseoir en tailleur\end{définition}
\begin{définition}\cmn 缠腿坐
\begin{déclaration}\use{尕脚方言}\end{déclaration}\end{définition}
\begin{relation-sémantique}\synonyme{
\hyperlink{Ⓔsɯχcoŋkroŋ}{\textit{ \papi{sɯχcoŋkroŋ}}}
}\end{relation-sémantique}\end{entrée}

\begin{entrée}
\vedette{\hypertarget{ⒺsɯrnaⒽ1}{\papi{ sɯrna}}}\markboth{sɯrna}{}\homonyme{1}
\classe{n}
\begin{définition}\fra figurine en glaise\end{définition}
\begin{définition}\cmn 泥巴捏成的偶像\end{définition}
\end{entrée}

\begin{entrée}
\vedette{\hypertarget{ⒺsɯrnaⒽ2}{\papi{ sɯrna}}}\markboth{sɯrna}{}\homonyme{2}\classe{n}
\begin{définition}\fra champignon noir\end{définition}
\begin{définition}\cmn 木耳\end{définition}
\begin{relation-sémantique}\confer{
\hyperlink{ⒺsiⒽ1}{\textit{ \papi{si1}}}
}\end{relation-sémantique}
\begin{relation-sémantique}\confer{
\hyperlink{Ⓔtɯ-rna}{\textit{ \papi{tɯ-rna}}}
}\end{relation-sémantique}\end{entrée}

\begin{entrée}
\vedette{\hypertarget{ⒺsɯrnaⒽ3}{\papi{ sɯrna}}}\markboth{sɯrna}{}\homonyme{3}
\classe{n}
\begin{définition}\fra bélier\end{définition}
\begin{définition}\cmn 公绵羊\end{définition}
\end{entrée}

\begin{entrée}
\vedette{\hypertarget{Ⓔsɯrpjɯ}{\papi{ sɯrpjɯ}}}\markboth{sɯrpjɯ}{}
\begin{relation-sémantique}\confer{
\hyperlink{Ⓔrpjɯ}{\textit{ \papi{rpjɯ}}}
}\end{relation-sémantique}\end{entrée}

\begin{entrée}
\vedette{\hypertarget{Ⓔsɯrsɯr}{\papi{ sɯrsɯr}}}\markboth{sɯrsɯr}{}\classe{idph.2}
\begin{définition}\fra rond\end{définition}
\begin{définition}\cmn 形容圆形\end{définition}
\begin{exemple}\jya ɲɯ-ɤrtɯm sɯrsɯr ʑo\cmn 圆溜溜的\end{exemple}
\begin{exemple}\jya lɯlu cho khɯna kɤ-nɯ-rŋgɯ-nɯ tɕe ku-ortɯm-nɯ sɯrsɯr ʑo ŋu\cmn 猫和狗睡觉的时候蜷成一团\end{exemple}\begin{sous-entrée}
\vedette{\hypertarget{}{\papi{ sɯrinɤsɯri}}}\markboth{sɯrinɤsɯri}{}\classe{idph.8}
\begin{définition}\fra qui tourne vite\end{définition}
\begin{définition}\cmn 形容旋转得很快的样子\end{définition}
\begin{exemple}\jya sɯrinɤsɯri ʑo ɲɯ-mtɕɯr\cmn 转得飞快\end{exemple}
\begin{relation-sémantique}\synonyme{
\hyperlink{Ⓔxɯrxɯr}{\textit{ \papi{xɯrxɯr}}}
}\end{relation-sémantique}
\end{sous-entrée}\end{entrée}

\begin{entrée}
\vedette{\hypertarget{Ⓔsɯrtaʁ}{\papi{ sɯrtaʁ}}}\markboth{sɯrtaʁ}{}
\classe{n}
\begin{définition}\fra branche\end{définition}
\begin{définition}\cmn 树枝\end{définition}
\begin{relation-sémantique}\confer{
\hyperlink{Ⓔtɤ-rtaʁ}{\textit{ \papi{tɤ-rtaʁ}}}
}\end{relation-sémantique}\end{entrée}

\begin{entrée}
\vedette{\hypertarget{Ⓔsɯrtoʁ}{\papi{ sɯrtoʁ}}}\markboth{sɯrtoʁ}{}
\begin{relation-sémantique}\confer{
\hyperlink{Ⓔrtoʁ}{\textit{ \papi{rtoʁ}}}
}\end{relation-sémantique}\end{entrée}

\begin{entrée}
\vedette{\hypertarget{Ⓔsɯrtsho}{\papi{ sɯrtsho}}}\markboth{sɯrtsho}{}
\classe{n}
\begin{définition}\fra partie ouverte de la souche\end{définition}
\begin{définition}\cmn 树墩(被锯掉的)表面\end{définition}\end{entrée}

\begin{entrée}
\vedette{\hypertarget{Ⓔsɯrtshɯm}{\papi{ sɯrtshɯm}}}\markboth{sɯrtshɯm}{}
\classe{n}
\begin{définition}\fra souche\end{définition}
\begin{définition}\cmn 树墩;树桩\end{définition}\end{entrée}

\begin{entrée}
\vedette{\hypertarget{Ⓔsɯrtsi}{\papi{ sɯrtsi}}}\markboth{sɯrtsi}{}
\classe{vt}
\paradigme{\textit{dir :} \jya tɤ-}
\paradigme{\textit{dir :} \jya thɯ-}
\begin{définition}\fra appliquer de la laque\end{définition}
\begin{définition}\cmn 上漆\end{définition}
\begin{exemple}\jya tɕoχtsi thɯ-sɯrtsi-t-a\cmn 我给桌子上了漆\end{exemple}
\begin{exemple}\jya to-sɯrtsi (ɯ-rtsi to-lɤt)\cmn 他上了漆\end{exemple}
\begin{relation-sémantique}\confer{
\hyperlink{Ⓔɯ-rtsi}{\textit{ \papi{ɯ-rtsi}}}
}\end{relation-sémantique}\end{entrée}

\begin{entrée}
\vedette{\hypertarget{Ⓔsɯrʑaʁ}{\papi{ sɯrʑaʁ}}}\markboth{sɯrʑaʁ}{}
\classe{vt}
\paradigme{\textit{dir :} \jya kɤ-}
\paradigme{\textit{dir :} \jya pɯ-}
\begin{définition}\fra enterrer\end{définition}
\begin{définition}\cmn 掩埋\end{définition}
\begin{exemple}\jya smi thɤlwa kɯ ka-sɯrʑaʁ\cmn 他用草木灰把火盖住了\end{exemple}
\begin{exemple}\jya lɯlu kɯ ɯ-qe ka-sɯrʑaʁ\cmn 猫把它自己的屎(用土)盖住了\end{exemple}
\begin{relation-sémantique}\synonyme{
\hyperlink{Ⓔsɯrma}{\textit{ \papi{sɯrma}}}
}\end{relation-sémantique}\end{entrée}

\begin{entrée}
\vedette{\hypertarget{Ⓔsɯʁaʁ}{\papi{ sɯʁaʁ}}}\markboth{sɯʁaʁ}{}
\begin{relation-sémantique}\confer{
\hyperlink{Ⓔʁaʁ}{\textit{ \papi{ʁaʁ}}}
}\end{relation-sémantique}\end{entrée}

\begin{entrée}
\vedette{\hypertarget{Ⓔsɯʁejlu}{\papi{ sɯʁejlu}}}\markboth{sɯʁejlu}{}
\classe{vi}
\paradigme{\textit{dir :} \jya tɤ-}
\begin{définition}\fra être gaucher\end{définition}
\begin{définition}\cmn 左撇子\end{définition}
\begin{exemple}\jya jiɕqha nɯ ɲɯ-sɯʁejlu\cmn 那个人是左撇子\end{exemple}
\begin{relation-sémantique}\confer{
\hyperlink{Ⓔʁejlu}{\textit{ \papi{ʁejlu}}}
}\end{relation-sémantique}\end{entrée}

\begin{entrée}
\vedette{\hypertarget{Ⓔsɯʁjit}{\papi{ sɯʁjit}}}\markboth{sɯʁjit}{}\classe{vt}
\paradigme{\textit{dir :} \jya tɤ-}\acception{1}
\begin{définition}\fra se souvenir\end{définition}
\begin{définition}\cmn 记得;想起\end{définition}
\begin{exemple}\jya mɤ-tɯ-sɯʁjit ʑo maŋe\cmn 没有你想不出来的事情\end{exemple}\acception{2}
\begin{définition}\fra manquer à\end{définition}
\begin{définition}\cmn 想念\end{définition}
\begin{exemple}\jya ɯ-kha ra to-sɯʁjit\cmn 他想念他的家属\end{exemple}
\begin{exemple}\jya ɯ-mu to-sɯʁjit\cmn 他想念他母亲\end{exemple}
\begin{exemple}\jya tɤ́-wɣ-sɯʁjit-a\cmn 他想念我了\end{exemple}
\begin{relation-sémantique}\confer{
\hyperlink{Ⓔʁjit}{\textit{ \papi{ʁjit}}}
}\end{relation-sémantique}\end{entrée}

\begin{entrée}
\vedette{\hypertarget{Ⓔsɯʁjoʁ}{\papi{ sɯʁjoʁ}}}\markboth{sɯʁjoʁ}{}
\classe{vt}
\paradigme{\textit{dir :} \jya pɯ-}
\begin{définition}\ 
\begin{déclaration}\grammar{denom}\end{déclaration}\end{définition}
\begin{définition}\fra se servir (d'un tissu) pour confectionner la couche extérieure d'un vêtement\end{définition}
\begin{définition}\cmn 做成衣服的外层\end{définition}
\begin{exemple}\jya raz kɯ-ɣɯrni kɯ χpɯn ɯ-ŋga ɯ-smɤʁjoʁ nɯ pjɯ́-wɣ-sɯʁjoʁ ra\cmn 要用红色的布料制作和尚衣服的外层\end{exemple}
\begin{relation-sémantique}\confer{
\hyperlink{Ⓔɯ-ʁjoʁ}{\textit{ \papi{ɯ-ʁjoʁ}}}
}\end{relation-sémantique}\end{entrée}

\begin{entrée}
\vedette{\hypertarget{Ⓔsɯʁndzɤr}{\papi{ sɯʁndzɤr}}}\markboth{sɯʁndzɤr}{}
\begin{relation-sémantique}\confer{
\hyperlink{Ⓔʁndzɤr}{\textit{ \papi{ʁndzɤr}}}
}\end{relation-sémantique}\end{entrée}

\begin{entrée}
\vedette{\hypertarget{Ⓔsɯʁnɯ}{\papi{ sɯʁnɯ}}}\markboth{sɯʁnɯ}{}
\begin{relation-sémantique}\confer{
\hyperlink{Ⓔʁnɯ}{\textit{ \papi{ʁnɯ}}}
}\end{relation-sémantique}\end{entrée}

\begin{entrée}
\vedette{\hypertarget{Ⓔsɯsu}{\papi{ sɯsu}}}\markboth{sɯsu}{}
\classe{vi}
\paradigme{\textit{dir :} \jya tɤ-}
\begin{définition}\fra vivant\end{définition}
\begin{définition}\cmn 活\end{définition}
\begin{exemple}\jya ɯ-kɯ-mɲɤm wuma ʑo pjɤ-thɯ ri, to-sɯsu\cmn 他差一点死了,又活过来了\end{exemple}\end{entrée}

\begin{entrée}
\vedette{\hypertarget{Ⓔsɯsat}{\papi{ sɯsat}}}\markboth{sɯsat}{}
\begin{relation-sémantique}\confer{
\hyperlink{Ⓔsat}{\textit{ \papi{sat}}}
}\end{relation-sémantique}\end{entrée}

\begin{entrée}
\vedette{\hypertarget{Ⓔsɯsaχsɤl}{\papi{ sɯsaχsɤl}}}\markboth{sɯsaχsɤl}{}
\begin{relation-sémantique}\confer{
 \papi{saχsɤl1}
}\end{relation-sémantique}\end{entrée}

\begin{entrée}
\vedette{\hypertarget{Ⓔsɯsɤɕqali}{\papi{ sɯsɤɕqali}}}\markboth{sɯsɤɕqali}{}
\begin{relation-sémantique}\confer{
\hyperlink{Ⓔɣɤɕqali}{\textit{ \papi{ɣɤɕqali}}}
}\end{relation-sémantique}\end{entrée}

\begin{entrée}
\vedette{\hypertarget{Ⓔsɯschɤt}{\papi{ sɯschɤt}}}\markboth{sɯschɤt}{}
\begin{relation-sémantique}\confer{
\hyperlink{Ⓔschɤt}{\textit{ \papi{schɤt}}}
}\end{relation-sémantique}\end{entrée}

\begin{entrée}
\vedette{\hypertarget{Ⓔsɯsci}{\papi{ sɯsci}}}\markboth{sɯsci}{}
\begin{relation-sémantique}\confer{
\hyperlink{ⒺsciⒽ1}{\textit{ \papi{sci}}}
}\end{relation-sémantique}
\end{entrée}

\begin{entrée}
\vedette{\hypertarget{Ⓔsɯskɯrma}{\papi{ sɯskɯrma}}}\markboth{sɯskɯrma}{}\classe{vt}
\paradigme{\textit{dir :} \jya \_}
\begin{définition}\ 
\begin{déclaration}\grammar{denom}\end{déclaration}\end{définition}
\begin{définition}\fra demander à quelqu'un d'emporter un cadeau à quelqu'un\end{définition}
\begin{définition}\cmn 请人带礼物\end{définition}
\begin{exemple}\jya jɤ-ta-sɯskɯrma nɯ nɤ-jaʁ ɯ-jɤ-ázɣɯt?\cmn 我送给你的礼物,你收到了没有\end{exemple}
\begin{relation-sémantique}\confer{
\hyperlink{Ⓔskɯrma}{\textit{ \papi{skɯrma}}}
}\end{relation-sémantique}\end{entrée}

\begin{entrée}
\vedette{\hypertarget{Ⓔsɯsloʁ}{\papi{ sɯsloʁ}}}\markboth{sɯsloʁ}{}
\begin{relation-sémantique}\confer{
\hyperlink{Ⓔsloʁ}{\textit{ \papi{sloʁ}}}
}\end{relation-sémantique}\end{entrée}

\begin{entrée}
\vedette{\hypertarget{Ⓔsɯsɲu}{\papi{ sɯsɲu}}}\markboth{sɯsɲu}{}
\begin{relation-sémantique}\confer{
\hyperlink{Ⓔsɲu}{\textit{ \papi{sɲu}}}
}\end{relation-sémantique}
\end{entrée}

\begin{entrée}
\vedette{\hypertarget{Ⓔsɯsŋa}{\papi{ sɯsŋa}}}\markboth{sɯsŋa}{}
\begin{relation-sémantique}\confer{
\hyperlink{Ⓔsŋa}{\textit{ \papi{sŋa}}}
}\end{relation-sémantique}\end{entrée}

\begin{entrée}
\vedette{\hypertarget{Ⓔsɯsŋaʁ}{\papi{ sɯsŋaʁ}}}\markboth{sɯsŋaʁ}{}
\begin{relation-sémantique}\confer{
\hyperlink{ⒺsŋaʁⒽ1}{\textit{ \papi{sŋaʁ1}}}
}\end{relation-sémantique}\end{entrée}

\begin{entrée}
\vedette{\hypertarget{Ⓔsɯso}{\papi{ sɯso}}}\markboth{sɯso}{}\classe{vt}
\paradigme{\textit{dir :} \jya nɯ-}\acception{1}
\begin{définition}\fra penser\end{définition}
\begin{définition}\cmn 想\end{définition}
\begin{exemple}\jya @dianhua ɯ-kɯ-lɤt mataŋe tɕe tɕhindʐa kɯ-ŋu kɯ nɯ-sɯso-t-a\cmn 你没有打电话,我想了怎么回事\end{exemple}
\begin{exemple}\jya aʑo a-kɤ-sɯso nɯ ʑo ŋu\cmn 这就是我的意思\end{exemple}
\begin{exemple}\jya ɯ-kɤ-sɯso ɲɯ-dɤn\cmn 他想的事情很多\end{exemple}
\begin{exemple}\jya ``aʑo kɯ-fse kɯ-mpɕɤr me" ɲɯ-nɯ-sɯsɤm pjɤ-ŋu\cmn 他想着,没有比我漂亮的人\end{exemple}\acception{2}
\begin{définition}\fra vouloir\end{définition}
\begin{définition}\cmn 想要\end{définition}
\begin{exemple}\jya laχtɕha kɤ-ntsɣe ɲɯ-sɯsɤm\cmn 他想卖东西\end{exemple}\acception{3}
\begin{définition}\fra manquer\end{définition}
\begin{définition}\cmn 想念\end{définition}\begin{sous-entrée}
\vedette{\hypertarget{}{\papi{ kɤsɯso}}}\markboth{kɤsɯso}{}
\begin{définition}\fra pensant, se disant que\end{définition}
\begin{définition}\cmn 想着;等到……的时候\end{définition}
\begin{exemple}\jya nɯ jamar rtaʁ kɤsɯso tɕe, a-tɤ-tɯ-z-nɯne ra\cmn (你估计)够的时候就要停\end{exemple}
\end{sous-entrée}\begin{sous-entrée}
\vedette{\hypertarget{}{\papi{ ɲɯkɯsɯso}}}\markboth{ɲɯkɯsɯso}{} (\variante{pjɯkɯsɯso}) \classe{adv}
\begin{définition}\fra en comparaison avec...\end{définition}
\begin{définition}\cmn 比起……\end{définition}
\begin{exemple}\jya nɤʑo ɲɯ-kɯ-sɯso tɕe, ɯʑo kɯ-wxtɯ-wxti ɕti\cmn 跟你比起来,他长得很大\end{exemple}
\begin{exemple}\jya aʑo pjɯ-kɯ-sɯso tɕe nɤʑo ʁo ɲɯ-tɯ-mbro\cmn 跟我比起来,你倒是长得很高\end{exemple}
\end{sous-entrée}\end{entrée}

\begin{entrée}
\vedette{\hypertarget{Ⓔsɯspa}{\papi{ sɯspa}}}\markboth{sɯspa}{}
\begin{relation-sémantique}\confer{
\hyperlink{Ⓔspa}{\textit{ \papi{spa}}}
}\end{relation-sémantique}\end{entrée}

\begin{entrée}
\vedette{\hypertarget{Ⓔsɯsphjaʁ}{\papi{ sɯsphjaʁ}}}\markboth{sɯsphjaʁ}{}
\begin{relation-sémantique}\confer{
\hyperlink{Ⓔsphjaʁ}{\textit{ \papi{sphjaʁ}}}
}\end{relation-sémantique}\end{entrée}

\begin{entrée}
\vedette{\hypertarget{Ⓔsɯspoʁ}{\papi{ sɯspoʁ}}}\markboth{sɯspoʁ}{}\classe{vt}
\paradigme{\textit{dir :} \jya \_}
\begin{définition}\fra faire un trou\end{définition}
\begin{définition}\cmn 穿孔\end{définition}
\begin{exemple}\jya tɯ-ŋga ko-sɯspoʁ\cmn 他把衣服戳了个洞\end{exemple}
\begin{exemple}\jya tɤ-fkɯm ko-sɯspoʁ\cmn 他把口袋戳了个洞\end{exemple}
\begin{exemple}\jya khoxtu pjɤ-sɯspoʁ\cmn 他把屋顶戳了个洞\end{exemple}
\begin{exemple}\jya rdɤstaʁ jɤ-lat-a tɕe, ɯ-ku kɤ-sɯspoʁ-a\cmn 我扔了一块石头,把他的头打伤了\end{exemple}
\begin{relation-sémantique}\confer{
\hyperlink{Ⓔspoʁ}{\textit{ \papi{spoʁ}}}
}\end{relation-sémantique}\begin{sous-entrée}
\vedette{\hypertarget{}{\papi{ sɯsɯspoʁ}}}\markboth{sɯsɯspoʁ}{}\classe{vt}
\paradigme{\textit{dir :} \jya \_}
\begin{définition}\fra faire un trou avec....\end{définition}
\begin{définition}\cmn 用……来打洞\end{définition}
\begin{exemple}\jya ndzrɯ kɯ pɯ-sɯ-sɯspoʁ-a\cmn 我用凿子打了洞\end{exemple}
\end{sous-entrée}\end{entrée}

\begin{entrée}
\vedette{\hypertarget{Ⓔsɯstu}{\papi{ sɯstu}}}\markboth{sɯstu}{}
\begin{relation-sémantique}\confer{
\hyperlink{ⒺstuⒽ2}{\textit{ \papi{stu2}}}
}\end{relation-sémantique}\end{entrée}

\begin{entrée}
\vedette{\hypertarget{Ⓔsɯsta}{\papi{ sɯsta}}}\markboth{sɯsta}{}
\begin{relation-sémantique}\confer{
\hyperlink{Ⓔsta}{\textit{ \papi{sta}}}
}\end{relation-sémantique}\end{entrée}

\begin{entrée}
\vedette{\hypertarget{Ⓔsɯstɤm}{\papi{ sɯstɤm}}}\markboth{sɯstɤm}{}
\begin{relation-sémantique}\confer{
\hyperlink{Ⓔstɤm}{\textit{ \papi{stɤm}}}
}\end{relation-sémantique}\end{entrée}

\begin{entrée}
\vedette{\hypertarget{Ⓔsɯsɯspoʁ}{\papi{ sɯsɯspoʁ}}}\markboth{sɯsɯspoʁ}{}
\begin{relation-sémantique}\confer{
\hyperlink{Ⓔsɯspoʁ}{\textit{ \papi{sɯspoʁ}}}
}\end{relation-sémantique}\end{entrée}

\begin{entrée}
\vedette{\hypertarget{Ⓔsɯta}{\papi{ sɯta}}}\markboth{sɯta}{}
\classe{vt}
\paradigme{\textit{dir :} \jya nɯ-}
\begin{définition}\fra détacher\end{définition}
\begin{définition}\cmn 解开\end{définition}
\begin{exemple}\jya nɯki ɲɤ-raʁ tɕe nɯ-sɯte\cmn 线卡住了,你解开吧\end{exemple}
\begin{exemple}\jya tɤ-ri nɯ-kɯ-raʁ nɯ-sɯta-t-a tɕe, kɤ-rɯkɤtɯm jɤɣ\cmn 我把缠了的线解开了,可以牵线了\end{exemple}
\begin{relation-sémantique}\confer{
\hyperlink{Ⓔsɯɕlɯɣ}{\textit{ \papi{sɯɕlɯɣ}}}
}\end{relation-sémantique}\end{entrée}

\begin{entrée}
\vedette{\hypertarget{Ⓔsɯtɤpɯz}{\papi{ sɯtɤpɯz}}}\markboth{sɯtɤpɯz}{}
\classe{n}
\begin{définition}\fra bois pourri\end{définition}
\begin{définition}\cmn 朽木\end{définition}\end{entrée}

\begin{entrée}
\vedette{\hypertarget{Ⓔsɯtɕɤt}{\papi{ sɯtɕɤt}}}\markboth{sɯtɕɤt}{}
\begin{relation-sémantique}\confer{
\hyperlink{Ⓔtɕɤt}{\textit{ \papi{tɕɤt}}}
}\end{relation-sémantique}
\end{entrée}

\begin{entrée}
\vedette{\hypertarget{Ⓔsɯtɕhaʁ}{\papi{ sɯtɕhaʁ}}}\markboth{sɯtɕhaʁ}{}
\classe{n}
\begin{définition}\fra se rétrécir (bois)\end{définition}
\begin{définition}\cmn 收缩(木头)\end{définition}
\begin{exemple}\jya sɯtɕhaʁ ko-ɕe\cmn 木头收缩了\end{exemple}
\begin{relation-sémantique}\confer{
\hyperlink{ⒺsiⒽ1}{\textit{ \papi{si1}}}
}\end{relation-sémantique}
\begin{relation-sémantique}\confer{
\hyperlink{Ⓔtɕhaʁ}{\textit{ \papi{tɕhaʁ}}}
}\end{relation-sémantique}\end{entrée}

\begin{entrée}
\vedette{\hypertarget{Ⓔsɯtɕɯn}{\papi{ sɯtɕɯn}}}\markboth{sɯtɕɯn}{}\classe{n}
\begin{définition}\fra grande forêt\end{définition}
\begin{définition}\cmn 大森林\end{définition}
\begin{relation-sémantique}\synonyme{
 \papi{sɯŋgɯ kɯ-rnaʁ}
}\end{relation-sémantique}\end{entrée}

\begin{entrée}
\vedette{\hypertarget{Ⓔsɯtɕɯnjmɤɣ}{\papi{ sɯtɕɯnjmɤɣ}}}\markboth{sɯtɕɯnjmɤɣ}{}\classe{n}
\begin{définition}\fra russule rouge\end{définition}
\begin{définition}\cmn 红菇【杉木菌】\end{définition}
\begin{relation-sémantique}\synonyme{
\hyperlink{Ⓔjmɤɣni}{\textit{ \papi{jmɤɣni}}}
}\end{relation-sémantique}\end{entrée}

\begin{entrée}
\vedette{\hypertarget{Ⓔsɯti}{\papi{ sɯti}}}\markboth{sɯti}{}
\begin{relation-sémantique}\confer{
\hyperlink{Ⓔti}{\textit{ \papi{ti}}}
}\end{relation-sémantique}\end{entrée}

\begin{entrée}
\vedette{\hypertarget{Ⓔsɯtsu}{\papi{ sɯtsu}}}\markboth{sɯtsu}{}
\begin{relation-sémantique}\confer{
\hyperlink{Ⓔtsu}{\textit{ \papi{tsu}}}
}\end{relation-sémantique}\end{entrée}

\begin{entrée}
\vedette{\hypertarget{Ⓔsɯtsɯm}{\papi{ sɯtsɯm}}}\markboth{sɯtsɯm}{}
\begin{relation-sémantique}\confer{
\hyperlink{Ⓔtsɯm}{\textit{ \papi{tsɯm}}}
}\end{relation-sémantique}\end{entrée}

\begin{entrée}
\vedette{\hypertarget{Ⓔsɯxcha}{\papi{ sɯxcha}}}\markboth{sɯxcha}{}
\classe{vt}
\paradigme{\textit{dir :} \jya tɤ-}
\begin{définition}\ 
\begin{déclaration}\grammar{habil}\end{déclaration}\end{définition}
\begin{définition}\fra être capable\end{définition}
\begin{définition}\cmn 有能力
\begin{déclaration}\use{用于反向形式}\end{déclaration}\end{définition}
\begin{exemple}\jya ɲɯ-rʑi tɕe, mɤ-tɯ́-wɣ-sɯxcha\cmn 很重,你不行(你抬不动)\end{exemple}
\begin{exemple}\jya kɤ-fkur mɯ́j-wɣ-sɯxcha\cmn 他背不起\end{exemple}
\begin{exemple}\jya nɤki nɯ ʁo mɯ́j-ɴqa tɕe tɯ́-wɣ-sɯxcha loβ\cmn 这个倒是没有很难,你能行吧\end{exemple}
\begin{relation-sémantique}\confer{
\hyperlink{ⒺchaⒽ1}{\textit{ \papi{cha1}}}
}\end{relation-sémantique}\end{entrée}

\begin{entrée}
\vedette{\hypertarget{Ⓔsɯxchi}{\papi{ sɯxchi}}}\markboth{sɯxchi}{}
\begin{relation-sémantique}\confer{
\hyperlink{Ⓔchi}{\textit{ \papi{chi}}}
}\end{relation-sémantique}\end{entrée}

\begin{entrée}
\vedette{\hypertarget{Ⓔsɯxcɯ}{\papi{ sɯxcɯ}}}\markboth{sɯxcɯ}{}
\begin{relation-sémantique}\confer{
\hyperlink{Ⓔɯ-lu,cɯ}{\textit{ \papi{ɯ-lu,cɯ}}}
}\end{relation-sémantique}\end{entrée}

\begin{entrée}
\vedette{\hypertarget{Ⓔsɯxɕɤt}{\papi{ sɯxɕɤt}}}\markboth{sɯxɕɤt}{}
\classe{vt}\acception{1}
\paradigme{\textit{dir :} \jya pɯ-}
\paradigme{\textit{dir :} \jya kɤ-}
\begin{définition}\fra enseigner\end{définition}
\begin{définition}\cmn 教\end{définition}
\begin{exemple}\jya kɤ-taʁ pɯ-sɯxɕat-a\cmn 我教他织布了\end{exemple}
\begin{exemple}\jya pɯ-sɯxɕat-a tɕe kɤ-sɯspa-t-a\cmn 我教会了他\end{exemple}\acception{2}
\paradigme{\textit{dir :} \jya nɯ-}
\begin{définition}\fra habituer à\end{définition}
\begin{définition}\cmn 训练,令……养成习惯\end{définition}
\begin{exemple}\jya kɤ-nɤma nɯ-sɯxɕat-a\cmn 我训练了他\end{exemple}
\begin{exemple}\jya kɤ-nɯndzɤmdɯm nɯ-ta-sɯxɕɤt\cmn 我让你养成吃零食的习惯\end{exemple}\acception{3}
\paradigme{\textit{dir :} \jya tɤ-}
\begin{définition}\fra indiquer\end{définition}
\begin{définition}\cmn 指点\end{définition}
\begin{exemple}\jya rgɯnba sɤxɕe ɣɯ ɯ-tʂu nɯ tɤ-sɯxɕat-a\cmn 我给他指点了去寺庙的路\end{exemple}\begin{sous-entrée}
\vedette{\hypertarget{}{\papi{ asɯxɕɯxɕɤt}}}\markboth{asɯxɕɯxɕɤt}{}\classe{vi}
\paradigme{\textit{dir :} \jya pɯ-}
\begin{définition}\ 
\begin{déclaration}\grammar{recip}\end{déclaration}\end{définition}
\begin{définition}\fra s'enseigner les uns aux autres\end{définition}
\begin{définition}\cmn 互相教\end{définition}
\end{sous-entrée}\begin{sous-entrée}
\vedette{\hypertarget{}{\papi{ sɤsɯxɕɤt}}}\markboth{sɤsɯxɕɤt}{}\classe{vi}
\paradigme{\textit{dir :} \jya pɯ-}
\begin{définition}\ 
\begin{déclaration}\grammar{apass}\end{déclaration}\end{définition}
\begin{définition}\fra enseigner\end{définition}
\begin{définition}\cmn 教书\end{définition}
\begin{exemple}\jya tɕhi pjɯ-tɯ-sɤsɯxɕɤt ŋu?\cmn 你教什么?\end{exemple}
\begin{exemple}\jya tɤrtsɯz pjɯ-sɤsɯxɕat-a ŋu\cmn 你教数学\end{exemple}
\end{sous-entrée}\begin{sous-entrée}
\vedette{\hypertarget{}{\papi{ ʑɣɤsɯxɕɤt}}}\markboth{ʑɣɤsɯxɕɤt}{}\classe{vi}
\paradigme{\textit{dir :} \jya nɯ-}
\begin{définition}\ 
\begin{déclaration}\grammar{refl}\end{déclaration}\end{définition}
\begin{définition}\fra s'entraîner\end{définition}
\begin{définition}\cmn 训练自己\end{définition}
\begin{exemple}\jya ɲɯ-ʑɣɤsɯxɕat-a\cmn 我自己训练\end{exemple}
\begin{relation-sémantique}\synonyme{
\hyperlink{Ⓔsɯɣʑaʁ}{\textit{ \papi{sɯɣʑaʁ}}}
}\end{relation-sémantique}
\begin{relation-sémantique}\confer{
\hyperlink{Ⓔɕɤt}{\textit{ \papi{ɕɤt}}}
}\end{relation-sémantique}
\end{sous-entrée}\end{entrée}

\begin{entrée}
\vedette{\hypertarget{Ⓔsɯxɕe}{\papi{ sɯxɕe}}}\markboth{sɯxɕe}{}\classe{vt}
\paradigme{\textit{dir :} \jya \_}
\paradigme{\textit{past stem :} \jya sɤɣri}
\begin{définition}\fra envoyer qqn\end{définition}
\begin{définition}\cmn 派人\end{définition}
\begin{exemple}\jya kɯ-nɤphɯphu jɤ-sɤɣri-t-a (jɤ-no-t-a)\cmn 我把乞丐赶走了\end{exemple}
\begin{exemple}\jya ki kɯ-fse tɤ-rʑaʁ ɲɯ-sɯxɕe-a ɬoʁ\cmn 我只有这样打发时间\end{exemple}
\begin{relation-sémantique}\confer{
\hyperlink{Ⓔɕe}{\textit{ \papi{ɕe}}}
}\end{relation-sémantique}
\begin{relation-sémantique}\confer{
 \papi{ɯ-pa,sɯxɕe}
}\end{relation-sémantique}\end{entrée}

\begin{entrée}
\vedette{\hypertarget{Ⓔsɯxsa}{\papi{ sɯxsa}}}\markboth{sɯxsa}{}
\begin{relation-sémantique}\confer{
\hyperlink{Ⓔsa}{\textit{ \papi{sa}}}
}\end{relation-sémantique}\end{entrée}

\begin{entrée}
\vedette{\hypertarget{Ⓔsɯxso}{\papi{ sɯxso}}}\markboth{sɯxso}{}
\begin{relation-sémantique}\confer{
\hyperlink{Ⓔso}{\textit{ \papi{so}}}
}\end{relation-sémantique}\end{entrée}

\begin{entrée}
\vedette{\hypertarget{Ⓔsɯxtar}{\papi{ sɯxtar}}}\markboth{sɯxtar}{}
\begin{relation-sémantique}\confer{
\hyperlink{Ⓔtar}{\textit{ \papi{tar}}}
}\end{relation-sémantique}\end{entrée}

\begin{entrée}
\vedette{\hypertarget{Ⓔsɯxtɕhaʁ}{\papi{ sɯxtɕhaʁ}}}\markboth{sɯxtɕhaʁ}{}
\classe{vt}
\paradigme{\textit{dir :} \jya pɯ-}
\paradigme{\textit{dir :} \jya nɯ-}
\begin{définition}\fra faire diminuer\end{définition}
\begin{définition}\cmn 减
\begin{déclaration} \étymologie{\papi{tɕʰag}}\end{déclaration}\end{définition}
\begin{exemple}\jya pɯ-sɯxtɕhaʁ-a\cmn 我减了\end{exemple}
\begin{exemple}\jya @laoban kɯ a-ŋgra pjɤ-sɯxtɕhaʁ\cmn 老板把我的工资给减少了\end{exemple}
\begin{exemple}\jya sqi ɯ-ngɯ kɯtʂɤɣ chɯ́-wɣ-sɯxtɕhaʁ tɕe, ɯ-ro kɯβde ɲɯ-ri ŋu\cmn 十减六等于四\end{exemple}
\begin{relation-sémantique}\synonyme{
\hyperlink{Ⓔɣɤtɕhaʁ}{\textit{ \papi{ɣɤtɕhaʁ}}}
}\end{relation-sémantique}
\begin{relation-sémantique}\confer{
\hyperlink{Ⓔtɕhaʁ}{\textit{ \papi{tɕhaʁ}}}
}\end{relation-sémantique}\begin{sous-entrée}
\vedette{\hypertarget{}{\papi{ ʑɣɤsɯxtɕhaʁ}}}\markboth{ʑɣɤsɯxtɕhaʁ}{}\classe{vi}
\begin{définition}\ 
\begin{déclaration}\grammar{refl}\end{déclaration}
\begin{déclaration}\grammar{caus}\end{déclaration}\end{définition}
\begin{définition}\fra diminuer de lui-même\end{définition}
\begin{définition}\cmn 令自己变少\end{définition}
\begin{exemple}\jya pɕawtsɯ ɯʑo ndɤre mɤ-nɯ-ʑɣɤsɯxtɕhaʁ\cmn 钱不会自己变少\end{exemple}
\end{sous-entrée}\end{entrée}

\begin{entrée}
\vedette{\hypertarget{Ⓔsɯxtɕhɤt}{\papi{ sɯxtɕhɤt}}}\markboth{sɯxtɕhɤt}{}
\begin{relation-sémantique}\confer{
\hyperlink{ⒺtɕhɤtⒽ1}{\textit{ \papi{tɕhɤt1}}}
}\end{relation-sémantique}\end{entrée}

\begin{entrée}
\vedette{\hypertarget{Ⓔsɯxtɕhɯt}{\papi{ sɯxtɕhɯt}}}\markboth{sɯxtɕhɯt}{}
\begin{relation-sémantique}\confer{
\hyperlink{Ⓔtɕhɯt}{\textit{ \papi{tɕhɯt}}}
}\end{relation-sémantique}\end{entrée}

\begin{entrée}
\vedette{\hypertarget{ⒺsɯxtsuⒽ1}{\papi{ sɯxtsu}}}\markboth{sɯxtsu}{}\homonyme{1} (\variante{sɯtsu}) 
\classe{vt}
\begin{définition}\ 
\begin{déclaration}\grammar{caus}\end{déclaration}\end{définition}
\begin{définition}\fra laisser du temps\end{définition}
\begin{définition}\cmn 给别人时间\end{définition}
\begin{exemple}\jya ɲɯ́-wɣ-sɯtsu-a-nɯ\cmn 他们给我时间\end{exemple}
\begin{exemple}\jya jisŋi aʑo ju-ɣi-a jɤɣ ma ɲɤ́-wɣ-sɯtsu-a-nɯ\cmn 我今天可以来,因为他们给我时间\end{exemple}
\begin{relation-sémantique}\confer{
\hyperlink{Ⓔtsu}{\textit{ \papi{tsu}}}
}\end{relation-sémantique}
\end{entrée}

\begin{entrée}
\vedette{\hypertarget{ⒺsɯxtsuⒽ2}{\papi{ sɯxtsu}}}\markboth{sɯxtsu}{}\homonyme{2}
\classe{vt}
\paradigme{\textit{dir :} \jya nɯ-}
\begin{définition}\fra faire fermenter\end{définition}
\begin{définition}\cmn 使发酵\end{définition}
\begin{relation-sémantique}\confer{
\hyperlink{Ⓔxtsu}{\textit{ \papi{xtsu}}}
}\end{relation-sémantique}
\end{entrée}

\begin{entrée}
\vedette{\hypertarget{Ⓔsɯxtshu}{\papi{ sɯxtshu}}}\markboth{sɯxtshu}{}
\begin{relation-sémantique}\confer{
\hyperlink{Ⓔtshu}{\textit{ \papi{tshu}}}
}\end{relation-sémantique}\end{entrée}

\begin{entrée}
\vedette{\hypertarget{Ⓔsɯxtshaʁ}{\papi{ sɯxtshaʁ}}}\markboth{sɯxtshaʁ}{}
\classe{vt}
\paradigme{\textit{dir :} \jya pɯ-}
\begin{définition}\ 
\begin{déclaration}\grammar{denom}\end{déclaration}\end{définition}
\begin{définition}\fra tamiser\end{définition}
\begin{définition}\cmn 筛\end{définition}
\begin{exemple}\jya tɯjpu pɯ-sɯxtshaʁ-a\cmn 我筛了粮食\end{exemple}
\begin{exemple}\jya thɤlwa pɯ-sɯxtshaʁ-a\cmn 我筛了灰\end{exemple}
\begin{exemple}\jya tɤɕi nɯ pɯ-sɯxtshaʁ-a\cmn 我筛了青稞\end{exemple}
\begin{relation-sémantique}\synonyme{
\hyperlink{Ⓔsɯɕɯɣra}{\textit{ \papi{sɯɕɯɣra}}}
}\end{relation-sémantique}
\begin{relation-sémantique}\confer{
\hyperlink{Ⓔtshaʁ}{\textit{ \papi{tshaʁ}}}
}\end{relation-sémantique}\end{entrée}

\begin{entrée}
\vedette{\hypertarget{Ⓔsɯxtshoz}{\papi{ sɯxtshoz}}}\markboth{sɯxtshoz}{}\classe{vt}
\paradigme{\textit{dir :} \jya tɤ-}
\begin{définition}\fra avoir au complet\end{définition}
\begin{définition}\cmn 具备齐全;一个也没有漏掉
\begin{déclaration} \étymologie{\papi{tsʰaŋs}}\end{déclaration}\end{définition}
\begin{exemple}\jya laʁdɯn tɤ-sɯxtshoz-a tɕe, kɤ-rɤma khɯ\cmn 农具具备齐全,可以工作了\end{exemple}
\begin{relation-sémantique}\synonyme{
\hyperlink{Ⓔɣɤtshoz}{\textit{ \papi{ɣɤtshoz}}}
}\end{relation-sémantique}
\begin{relation-sémantique}\confer{
\hyperlink{Ⓔtshoz}{\textit{ \papi{tshoz}}}
}\end{relation-sémantique}\end{entrée}

\begin{entrée}
\vedette{\hypertarget{Ⓔsɯxtshwi}{\papi{ sɯxtshwi}}}\markboth{sɯxtshwi}{}
\classe{vt}
\paradigme{\textit{dir :} \jya pɯ-}
\begin{définition}\ 
\begin{déclaration}\grammar{denom}\end{déclaration}\end{définition}
\begin{définition}\fra teindre\end{définition}
\begin{définition}\cmn 染
\begin{déclaration} \étymologie{\papi{tsʰos}}\end{déclaration}\end{définition}
\begin{exemple}\jya tɤ-ri pjɤ-sɯxtshwi\cmn 他染了线\end{exemple}
\begin{exemple}\jya tɯ-ŋga pjɤ-sɯxtshwi\cmn 他染了衣服\end{exemple}
\begin{exemple}\jya raz pɯ-sɯxtshwi-t-a / tshwi pɯ-lat-a\cmn 我染了布料\end{exemple}
\begin{relation-sémantique}\confer{
\hyperlink{Ⓔtshwi}{\textit{ \papi{tshwi}}}
}\end{relation-sémantique}\end{entrée}

\begin{entrée}
\vedette{\hypertarget{Ⓔsɯxtso}{\papi{ sɯxtso}}}\markboth{sɯxtso}{}
\begin{relation-sémantique}\confer{
\hyperlink{Ⓔtso}{\textit{ \papi{tso}}}
}\end{relation-sémantique}\end{entrée}

\begin{entrée}
\vedette{\hypertarget{Ⓔsɯxtsɯɣ}{\papi{ sɯxtsɯɣ}}}\markboth{sɯxtsɯɣ}{}
\begin{relation-sémantique}\confer{
\hyperlink{Ⓔxtsɯɣ}{\textit{ \papi{xtsɯɣ}}}
}\end{relation-sémantique}\end{entrée}

\begin{entrée}
\vedette{\hypertarget{Ⓔsɯxtʂaŋ}{\papi{ sɯxtʂaŋ}}}\markboth{sɯxtʂaŋ}{}
\begin{relation-sémantique}\confer{
\hyperlink{Ⓔtʂaŋ}{\textit{ \papi{tʂaŋ}}}
}\end{relation-sémantique}
\end{entrée}

\begin{entrée}
\vedette{\hypertarget{Ⓔsɯxtʂɯn}{\papi{ sɯxtʂɯn}}}\markboth{sɯxtʂɯn}{}\classe{vt}
\paradigme{\textit{dir :} \jya pɯ-}
\begin{définition}\fra faire bénéficier de ses bienfaits\end{définition}
\begin{définition}\cmn 对别人好\end{définition}
\begin{exemple}\jya ɯʑɤɣ pɯ-sɯxtʂɯn-a\cmn 我对他很好\end{exemple}
\begin{exemple}\jya a-mgɯr nɯ-tɯ-rɤβraʁ, tɤ-tɯ-sɯxtsɯɣ, ɲɯ-tɯ-sɯxtʂɯn\cmn 你帮我抠背部,你抠到正确的地方,做的很好\end{exemple}
\begin{relation-sémantique}\confer{
\hyperlink{Ⓔtɯ-tʂɯn}{\textit{ \papi{tɯ-tʂɯn}}}
}\end{relation-sémantique}\end{entrée}

\begin{entrée}
\vedette{\hypertarget{Ⓔsɯxtɯɣ}{\papi{ sɯxtɯɣ}}}\markboth{sɯxtɯɣ}{}
\begin{relation-sémantique}\confer{
\hyperlink{ⒺtɯɣⒽ1}{\textit{ \papi{tɯɣ1}}}
}\end{relation-sémantique}\end{entrée}

\begin{entrée}
\vedette{\hypertarget{Ⓔsɯχcoŋkroŋ}{\papi{ sɯχcoŋkroŋ}}}\markboth{sɯχcoŋkroŋ}{}
\classe{vi}
\paradigme{\textit{dir :} \jya nɯ-}
\begin{définition}\ 
\begin{déclaration}\grammar{denom}\end{déclaration}\end{définition}
\begin{définition}\fra s'asseoir en tailleur\end{définition}
\begin{définition}\cmn 盘腿坐(男人坐的姿势)\end{définition}
\begin{exemple}\jya nɯ-sɯχcoŋkroŋ-a (=χcoŋkroŋ nɯ-βzu-t-a)\cmn 我盘着腿坐下了\end{exemple}
\begin{relation-sémantique}\confer{
\hyperlink{Ⓔχcoŋkroŋ}{\textit{ \papi{χcoŋkroŋ}}}
}\end{relation-sémantique}
\begin{relation-sémantique}\confer{
\hyperlink{Ⓔtɯ-rpɣo}{\textit{ \papi{tɯ-rpɣo}}}
}\end{relation-sémantique}
\begin{relation-sémantique}\synonyme{
\hyperlink{Ⓔsɯrmbɣotɯm}{\textit{ \papi{sɯrmbɣotɯm}}}
}\end{relation-sémantique}\end{entrée}

\begin{entrée}
\vedette{\hypertarget{Ⓔsɯχpjɤt}{\papi{ sɯχpjɤt}}}\markboth{sɯχpjɤt}{}
\begin{relation-sémantique}\confer{
\hyperlink{Ⓔχpjɤt}{\textit{ \papi{χpjɤt}}}
}\end{relation-sémantique}\end{entrée}

\begin{entrée}
\vedette{\hypertarget{Ⓔsɯχsu}{\papi{ sɯχsu}}}\markboth{sɯχsu}{}
\begin{relation-sémantique}\confer{
\hyperlink{Ⓔχsu}{\textit{ \papi{χsu}}}
}\end{relation-sémantique}
\end{entrée}

\begin{entrée}
\vedette{\hypertarget{Ⓔsɯχsɤl}{\papi{ sɯχsɤl}}}\markboth{sɯχsɤl}{}
\classe{vt}
\paradigme{\textit{dir :} \jya pɯ-}
\begin{définition}\fra reconnaître, s'apercevoir\end{définition}
\begin{définition}\cmn 认得,发现
\begin{déclaration} \étymologie{\papi{gsal}}\end{déclaration}\end{définition}
\begin{exemple}\jya ɲɯ-sɯχsɤl\cmn 他认得他\end{exemple}
\begin{exemple}\jya pɯ-sɯχsal-a\cmn 我发现了他\end{exemple}
\begin{exemple}\jya pɯ́-wɣ-sɯχsal-a\cmn 他发现了我\end{exemple}
\begin{exemple}\jya pɯ-sɯχsɤl-tɕi\cmn 我们俩发现了他\end{exemple}\end{entrée}

\begin{entrée}
\vedette{\hypertarget{Ⓔsɯχta}{\papi{ sɯχta}}}\markboth{sɯχta}{}\classe{postp}
\begin{définition}\fra par rapport à\end{définition}
\begin{définition}\cmn 比\end{définition}
\begin{relation-sémantique}\confer{
\hyperlink{Ⓔstaʁ}{\textit{ \papi{staʁ}}}
}\end{relation-sémantique}
\begin{relation-sémantique}\confer{
\hyperlink{Ⓔsɤz}{\textit{ \papi{sɤz}}}
}\end{relation-sémantique}\end{entrée}

\begin{entrée}
\vedette{\hypertarget{Ⓔsɯχtɯ}{\papi{ sɯχtɯ}}}\markboth{sɯχtɯ}{}
\begin{relation-sémantique}\confer{
\hyperlink{Ⓔχtɯ}{\textit{ \papi{χtɯ}}}
}\end{relation-sémantique}
\end{entrée}

\begin{entrée}
\vedette{\hypertarget{Ⓔsɯz}{\papi{ sɯz}}}\markboth{sɯz}{}\classe{vt}
\paradigme{\textit{dir :} \jya pɯ-}
\paradigme{\textit{generic negative :} \jya mɤ-xsi}
\begin{définition}\fra savoir\end{définition}
\begin{définition}\cmn 知道\end{définition}
\begin{exemple}\jya pɯ-mto-t-a kɯ-fse ri, ŋu maʁ mɤxsi\cmn 我好像看见了,但是不确定\end{exemple}
\begin{relation-sémantique}\confer{
\hyperlink{Ⓔamɯsɯz}{\textit{ \papi{amɯsɯz}}}
}\end{relation-sémantique}\end{entrée}

\begin{entrée}
\vedette{\hypertarget{Ⓔsɯzbaʁ}{\papi{ sɯzbaʁ}}}\markboth{sɯzbaʁ}{}
\begin{relation-sémantique}\confer{
\hyperlink{Ⓔzbaʁ}{\textit{ \papi{zbaʁ}}}
}\end{relation-sémantique}\end{entrée}

\begin{entrée}
\vedette{\hypertarget{Ⓔsɯzbɤβ}{\papi{ sɯzbɤβ}}}\markboth{sɯzbɤβ}{}
\classe{n}
\begin{définition}\fra nœud (sur un arbre)\end{définition}
\begin{définition}\cmn 树瘤\end{définition}\end{entrée}

\begin{entrée}
\vedette{\hypertarget{Ⓔsɯzdɯɣ}{\papi{ sɯzdɯɣ}}}\markboth{sɯzdɯɣ}{}
\classe{vt}
\paradigme{\textit{dir :} \jya pɯ-}
\begin{définition}\ 
\begin{déclaration}\grammar{caus}\end{déclaration}\end{définition}
\begin{définition}\fra causer du souci\end{définition}
\begin{définition}\cmn 令别人受苦\end{définition}
\begin{exemple}\jya jiɕqha nɯ pɯ-sɯzdɯɣ-a\cmn 我令他受苦了\end{exemple}
\begin{exemple}\jya pɯ́-wɣ-sɯzdɯɣ-a\cmn 他令我受苦了\end{exemple}
\begin{exemple}\jya pɯ-ta-sɯzdɯɣ\cmn 我令你担心了\end{exemple}
\begin{exemple}\jya ɲɯ-ta-sɯzdɯɣ nɯ!\cmn 我令你受苦了!\end{exemple}\begin{sous-entrée}
\vedette{\hypertarget{}{\papi{ sɤsɯzdɯɣ}}}\markboth{sɤsɯzdɯɣ}{}\classe{vs}
\begin{définition}\fra gêner les gens\end{définition}
\begin{définition}\cmn 麻烦别人\end{définition}
\begin{exemple}\jya nɤki tɤ-pɤtso nɯ kɤ-ndo ɲɯ-ɴqa ma ɲɯ-sɤsɯzdɯɣ\cmn 带这个小孩子很难,因为他很麻烦\end{exemple}
\begin{relation-sémantique}\confer{
\hyperlink{Ⓔzdɯɣ}{\textit{ \papi{zdɯɣ}}}
}\end{relation-sémantique}
\begin{relation-sémantique}\confer{
\hyperlink{Ⓔnɯzdɯɣ}{\textit{ \papi{nɯzdɯɣ}}}
}\end{relation-sémantique}
\end{sous-entrée}\begin{sous-entrée}
\vedette{\hypertarget{}{\papi{ ʑɣɤsɯzdɯɣ}}}\markboth{ʑɣɤsɯzdɯɣ}{}\classe{vi}
\begin{définition}\ 
\begin{déclaration}\grammar{refl}\end{déclaration}\end{définition}
\begin{définition}\fra se donner du mal\end{définition}
\begin{définition}\cmn 大费周章;不辞辛苦\end{définition}
\end{sous-entrée}\end{entrée}

\begin{entrée}
\vedette{\hypertarget{Ⓔsɯzgrɯtɕhɯ}{\papi{ sɯzgrɯtɕhɯ}}}\markboth{sɯzgrɯtɕhɯ}{}\classe{vt}
\paradigme{\textit{dir :} \jya tɤ-}
\begin{définition}\ 
\begin{déclaration}\grammar{incorp}\end{déclaration}\end{définition}
\begin{définition}\fra donner un coup de coude\end{définition}
\begin{définition}\cmn 打一肘\end{définition}
\begin{exemple}\jya ɯʑo ɲɯ-nɯɕmɯrga tɕe tɤ-sɯzgrɯtɕhɯ-t-a\cmn 他多嘴了,我给他打了一肘\end{exemple}
\begin{relation-sémantique}\confer{
\hyperlink{Ⓔzgrɯtɕhɯ}{\textit{ \papi{zgrɯtɕhɯ}}}
}\end{relation-sémantique}\end{entrée}

\begin{entrée}
\vedette{\hypertarget{Ⓔsɯzʁe}{\papi{ sɯzʁe}}}\markboth{sɯzʁe}{}\classe{n}
\begin{définition}\fra action de transporter du bois\end{définition}
\begin{définition}\cmn 背柴\end{définition}
\begin{relation-sémantique}\confer{
\hyperlink{Ⓔnɯsɯzʁe}{\textit{ \papi{nɯsɯzʁe}}}
}\end{relation-sémantique}\end{entrée}

\begin{entrée}
\vedette{\hypertarget{Ⓔsɯʑŋgrɯt}{\papi{ sɯʑŋgrɯt}}}\markboth{sɯʑŋgrɯt}{}
\classe{vi}
\paradigme{\textit{dir :} \jya tɤ-}
\begin{définition}\fra ravaler ses larmes, gémir\end{définition}
\begin{définition}\cmn 啜泣,呻吟\end{définition}
\begin{exemple}\jya kɤ-rŋgɯ tɕe, ɲɯ-sɯʑŋgrɯt\cmn 他睡了就呻吟\end{exemple}\end{entrée}

\begin{entrée}
\vedette{\hypertarget{Ⓔsuwa}{\papi{ suwa}}}\markboth{suwa}{}
\classe{n}
\begin{définition}\fra viseur\end{définition}
\begin{définition}\cmn 准星\end{définition}
\begin{exemple}\jya suwa nɯ ɕɤmɯɣdɯ ɣɯ-lɤt tɤ-mda tɕe ɯ-sɤ-z-nɯpɯmɲɯɣ ŋu\cmn 准星是打枪时用来瞄准的东西。\end{exemple}
\begin{relation-sémantique}\confer{
\hyperlink{Ⓔnɯsuwa}{\textit{ \papi{nɯsuwa}}}
}\end{relation-sémantique}\end{entrée}

\newpage\caractère{ʂ}

\begin{entrée}
\vedette{\hypertarget{Ⓔʂa}{\papi{ ʂa}}}\markboth{ʂa}{}
\classe{vs}
\paradigme{\textit{dir :} \jya thɯ-}
\begin{définition}\fra capable\end{définition}
\begin{définition}\cmn 能干
\begin{déclaration} \étymologie{\papi{sra}}\end{déclaration}\end{définition}
\begin{exemple}\jya kɤ-ŋke kɯ-ʂa ci ɲɯ-ŋu\cmn 他是一个很能走路的人\end{exemple}\end{entrée}

\begin{entrée}
\vedette{\hypertarget{Ⓔʂaʁ,ta}{\papi{ ʂaʁ,ta}}}\markboth{ʂaʁ,ta}{}
\paradigme{\textit{dir :} \jya kɤ-}
\begin{définition}\fra marquer au fer rouge\end{définition}
\begin{définition}\cmn 烙印\end{définition}
\begin{exemple}\jya a-jaʁ ʂaʁ kɤ-nɯ-sɯta-t-a\cmn 我烙(烫)到了手\end{exemple}
\begin{exemple}\jya ʂaʁ ɯ-rŋa ko-ta\cmn 在他脸上烙了印\end{exemple}
\begin{exemple}\jya ɕom chɤ-z-ɣɯrni tɕe, ɯ-ɕa ɯ-taʁ ʂaʁ ko-ta\cmn 把铁烧红了,就在它的肉上烙了印子\end{exemple}
\begin{relation-sémantique}\ComponentA{\classe{adv}
 \papi{ʂaʁ}
}\end{relation-sémantique}
\begin{relation-sémantique}\ComponentB{\classe{vt}
\hyperlink{Ⓔta}{\textit{ \papi{ta}}}
}\end{relation-sémantique}\end{entrée}

\begin{entrée}
\vedette{\hypertarget{Ⓔʂɣɤlʂɣɤl}{\papi{ ʂɣɤlʂɣɤl}}}\markboth{ʂɣɤlʂɣɤl}{}\classe{idph.2}
\begin{définition}\fra transparent et brillant comme la rosée du matin\end{définition}
\begin{définition}\cmn 形容像草叶上滚动的露珠一样透明发亮的样子\end{définition}
\begin{exemple}\jya tɤ-rɣe ʂɣɤlʂɣɤl ci ɲɯ-ŋu\cmn 珍珠透明发亮\end{exemple}
\begin{exemple}\jya tɤ-tsrɯ ɯ-taʁ tɤʑri ʂɣɤlʂɣɤl ɲɯ-pa\cmn 萌芽上有透明发亮的露珠\end{exemple}
\begin{exemple}\jya kɯ-wxti ra kɯ tu-ti-nɯ, ftɕar tɤ-tsrɯ to-ɬoʁ tɕe, tɤ-tsrɯ ɯ-ku ɯ-taʁ zɯ tɤʑri ʂɣɤlʂɣɤl rɯri ɣɯ tɯka tu tɕe, tɯrme tɯ-fsonam nɯ, nɯ kɯ-fse, rɯri tɯkaka tɯ-fsonam nɯ nɯ fse tu-ti-nɯ tu-raχpi-nɯ ɲɯ-ŋgrɤl\cmn 老年人说,夏天萌芽长出来的时候,每个萌芽上面有透明的露珠,代表着每个人的运气\end{exemple}
\begin{exemple}\jya ɯ-jaʁ ɯ-taʁ cimbɤrom ʂɣɤlʂɣɤl ʑo to-rku\cmn 他手上起了透明的水疱\end{exemple}\end{entrée}

\begin{entrée}
\vedette{\hypertarget{Ⓔʂmɤβʂmɤβ}{\papi{ ʂmɤβʂmɤβ}}}\markboth{ʂmɤβʂmɤβ}{}\classe{idph.2}
\begin{définition}\fra qui forme une fine couche\end{définition}
\begin{définition}\cmn 形容物体薄薄一层,不完全透明的样子\end{définition}
\begin{relation-sémantique}\confer{
\hyperlink{Ⓔrmɤβrmɤβ}{\textit{ \papi{rmɤβrmɤβ}}}
}\end{relation-sémantique}\end{entrée}

\begin{entrée}
\vedette{\hypertarget{Ⓔʂɲoʁʂɲoʁ}{\papi{ ʂɲoʁʂɲoʁ}}}\markboth{ʂɲoʁʂɲoʁ}{}
\classe{idph.2}
\begin{définition}\fra svelte\end{définition}
\begin{définition}\cmn 形容身材苗条,瘦高瘦高的模样\end{définition}
\begin{exemple}\jya tɤ-pɤtso ɯ-phoŋbu ʂɲoʁʂɲoʁ ɲɯ-pa\cmn 小孩子(身材)瘦高瘦高的。\end{exemple}\end{entrée}

\begin{entrée}
\vedette{\hypertarget{Ⓔʂɲɯɣnɤʂɲɯɣ}{\papi{ ʂɲɯɣnɤʂɲɯɣ}}}\markboth{ʂɲɯɣnɤʂɲɯɣ}{}\classe{idph.3}
\begin{définition}\fra intermittent\end{définition}
\begin{définition}\cmn 形容运动、感觉等间发的样子\end{définition}
\begin{exemple}\jya a-mthɤɣ ʂɲɯɣnɤʂɲɯɣ ɲɯ-mŋɤm\cmn 我的腰一阵一阵地痛\end{exemple}
\begin{relation-sémantique}\confer{
\hyperlink{Ⓔɣɤʂɲɯɣlɯɣ}{\textit{ \papi{ɣɤʂɲɯɣlɯɣ}}}
}\end{relation-sémantique}\end{entrée}

\begin{entrée}
\vedette{\hypertarget{Ⓔʂɯɣʂɯɣ}{\papi{ ʂɯɣʂɯɣ}}}\markboth{ʂɯɣʂɯɣ}{}
\classe{idph.2}
\begin{définition}\fra clair\end{définition}
\begin{définition}\cmn 亮\end{définition}
\begin{exemple}\jya ndʑa ʂɯɣʂɯɣ pjɤ-ndɯ\cmn 彩虹出来了\end{exemple}
\begin{relation-sémantique}\confer{
\hyperlink{Ⓔʂɯŋʂɯŋ}{\textit{ \papi{ʂɯŋʂɯŋ}}}
}\end{relation-sémantique}\end{entrée}

\begin{entrée}
\vedette{\hypertarget{Ⓔʂɯŋʂɯŋ}{\papi{ ʂɯŋʂɯŋ}}}\markboth{ʂɯŋʂɯŋ}{}\classe{idph.2}
\begin{définition}\fra clair\end{définition}
\begin{définition}\cmn 天晴\end{définition}
\begin{exemple}\jya tɤŋe ʂɯŋʂɯŋ ɲɤ-ɬoʁ\cmn 太阳出来了\end{exemple}
\begin{exemple}\jya ʂɯŋʂɯŋ ɲɤ-jɯm\cmn 天晴了\end{exemple}
\begin{exemple}\jya ndʑa ʂɯŋʂɯŋ pjɤ-ndɯ\cmn 彩虹出来了\end{exemple}
\begin{exemple}\jya ɯ-mɲaʁ ɲɯ-mto ʂɯŋʂɯŋ ʑo\cmn 他眼睛看得很清楚\end{exemple}
\begin{relation-sémantique}\confer{
\hyperlink{Ⓔxɯŋxɯŋ}{\textit{ \papi{xɯŋxɯŋ}}}
}\end{relation-sémantique}\end{entrée}

\begin{entrée}
\vedette{\hypertarget{Ⓔʂɯt}{\papi{ ʂɯt}}}\markboth{ʂɯt}{}\classe{idph.1}
\begin{définition}\fra bruit que fait un oiseau en s'envolant brusquement\end{définition}
\begin{définition}\cmn 小鸟突然飞出来的声音\end{définition}
\end{entrée}

\begin{entrée}
\vedette{\hypertarget{Ⓔʂχinɤʂχi}{\papi{ ʂχinɤʂχi}}}\markboth{ʂχinɤʂχi}{}
\classe{idph.2}
\begin{définition}\fra haletant\end{définition}
\begin{définition}\cmn 形容气喘吁吁的样子\end{définition}
\begin{exemple}\jya jiɕqha nɯ ɲɯ-nɯtɕhomba tɕe, ɯ-rqo ʂχinɤʂχi ɲɤ-stu\cmn 他感冒了,气喘吁吁的\end{exemple}\end{entrée}

\begin{entrée}
\vedette{\hypertarget{Ⓔʂχɯʂχaj}{\papi{ ʂχɯʂχaj}}}\markboth{ʂχɯʂχaj}{}\classe{idph.2}
\begin{définition}\fra ayant plein de trous\end{définition}
\begin{définition}\cmn 形容满是洞洞眼眼的样子\end{définition}
\begin{exemple}\jya a-ŋga ʂχɯʂχaj ɲɯ-xtsu ɕti ri, nɯ kɯnɤ ɲɯ-mpja\cmn 虽然我衣服上到处都是洞洞眼眼,但还是很暖\end{exemple}
\begin{relation-sémantique}\confer{
\hyperlink{Ⓔʂχɯʂχi}{\textit{ \papi{ʂχɯʂχi}}}
}\end{relation-sémantique}\end{entrée}

\begin{entrée}
\vedette{\hypertarget{Ⓔʂχɯʂχi}{\papi{ ʂχɯʂχi}}}\markboth{ʂχɯʂχi}{}\classe{idph.2}
\begin{définition}\fra ayant de grosses narines\end{définition}
\begin{définition}\cmn 形容鼻孔很大\end{définition}
\begin{exemple}\jya mbro ɯ-ɕna ʂχɯʂχi to-stu, tɕe ndʐoʁ ɯβrɤ-ŋu ma\cmn 马把鼻孔张大了,是不是要调皮了\end{exemple}\begin{sous-entrée}
\vedette{\hypertarget{}{\papi{ ʂχɯwɯʂχawi}}}\markboth{ʂχɯwɯʂχawi}{}\classe{idph.8}
\begin{définition}\fra qui a des trous partout\end{définition}
\begin{définition}\cmn 到处都是洞\end{définition}
\begin{exemple}\jya βʑɯ kɯ lʁa to-ndza tɕe ʂχɯwɯʂχawi ʑo ɲɤ-sɤβzu\cmn 老鼠把口袋啃了,啃得到处都是洞\end{exemple}
\begin{relation-sémantique}\confer{
\hyperlink{Ⓔʂχɯʂχaj}{\textit{ \papi{ʂχɯʂχaj}}}
}\end{relation-sémantique}
\end{sous-entrée}\end{entrée}

\newpage\caractère{t}

\begin{entrée}
\vedette{\hypertarget{Ⓔtu}{\papi{ tu}}}\markboth{tu}{}\classe{vs}
\paradigme{\textit{dir :} \jya tɤ-}
\begin{définition}\fra exister, se trouver\end{définition}
\begin{définition}\cmn 有\end{définition}
\begin{exemple}\jya aj nɯ kóʁmɯz @jieshang ɕ-pɯ-tu-a tɕe jɤ-azɣɯt-a\cmn 我刚才在街上,刚刚到(家)\end{exemple}
\begin{exemple}\jya mɤ-kɯ-pe tɤ-tu tɕe tu-ti-a\cmn 有错误的时候我就说\end{exemple}
\begin{exemple}\jya aʑɯɣ rŋɯl tu\cmn 我有钱\end{exemple}
\begin{exemple}\jya khɯtsa-ŋgɯ tɯ-ci tu\cmn 碗里有水\end{exemple}\end{entrée}

\begin{entrée}
\vedette{\hypertarget{Ⓔta}{\papi{ ta}}}\markboth{ta}{}
\classe{vt}\acception{1}
\paradigme{\textit{dir :} \jya \_}
\begin{définition}\fra poser, mettre\end{définition}
\begin{définition}\cmn 放置\end{définition}
\begin{exemple}\jya ɯ-ŋga pɯ-ta-t-a\cmn 我给他盖了被子\end{exemple}\acception{2}
\paradigme{\textit{dir :} \jya tɤ-}
\paradigme{\textit{dir :} \jya pɯ-}
\begin{définition}\fra porter (lunettes, chapeau)\end{définition}
\begin{définition}\cmn 戴(眼镜。帽子)\end{définition}
\begin{exemple}\jya χɕɤlmɯɣ tu-nɯ-te-a\cmn 我戴眼镜\end{exemple}
\begin{exemple}\jya χɕɤlmɯɣ tɤ-nɯ-te\cmn 你戴上眼镜吧\end{exemple}
\begin{exemple}\jya nɤ-ku ɯ-taʁ tɤ-rte ma-pɯ-tɯ-nɯ-te ma nɤ-tɯ-sɤjloʁ\cmn 你不要戴上帽子,不好看\end{exemple}\acception{3}
\paradigme{\textit{dir :} \jya nɯ-}
\begin{définition}\fra relâcher, laisser, abandonner\end{définition}
\begin{définition}\cmn 放(走);放下\end{définition}
\begin{exemple}\jya mɤ-ta-ta\cmn 我不会放过你的\end{exemple}
\begin{exemple}\jya tɯrɣi nɯ-ta-t-a\cmn 我留了种子\end{exemple}
\begin{exemple}\jya kɯki kɤ-nɤma ki kɤ-ta a-ʁjiz mɯ́j-ɣi\cmn 我不想放弃这个工作\end{exemple}
\begin{exemple}\jya nɤ-ndza-ro ma-nɯ-tɯ-te ma ɯ-kɯ-ndza me (nɤ-khɯɲcɤr ma-nɯ-tɯ-te)\cmn 你不要吃剩一口,不会有人吃你的剩饭\end{exemple}\acception{4}
\paradigme{\textit{dir :} \jya kɤ-}
\begin{définition}\fra cuire\end{définition}
\begin{définition}\cmn 蒸;熬\end{définition}
\begin{exemple}\jya qajɣi khon kɤ-ta-t-a\cmn 我蒸了馍馍\end{exemple}
\begin{exemple}\jya tʂha kɤ-ta-t-a\cmn 我熬了茶\end{exemple}\acception{5}
\paradigme{\textit{dir :} \jya pɯ-}
\begin{définition}\fra se produire (gel)\end{définition}
\begin{définition}\cmn 下(霜)\end{définition}
\begin{exemple}\jya ɕŋɤr pjɤ-t-a\cmn 下了霜\end{exemple}\begin{sous-entrée}
\vedette{\hypertarget{}{\papi{ nɤtɯta}}}\markboth{nɤtɯta}{}\classe{vt}
\begin{définition}\fra mettre n'importe comment\end{définition}
\begin{définition}\cmn 随便放,放了放去\end{définition}
\begin{exemple}\jya nɤ-laχtɕha ra aʁɤndɯndɤt ma-tɯ-nɤtɯte ma ɲɯ-me ŋu\cmn 你不要随便吧东西乱放,不然丢失的\end{exemple}
\end{sous-entrée}\begin{sous-entrée}
\vedette{\hypertarget{}{\papi{ nɯta}}}\markboth{nɯta}{}\classe{vt}
\paradigme{\textit{dir :} \jya kɤ-}
\begin{définition}\fra épouser\end{définition}
\begin{définition}\cmn 娶妻\end{définition}
\begin{exemple}\jya rɟɤlpu kɯ ɯ-rʑaβ ko-nɯ-ta\cmn 土司把她娶为妻子\end{exemple}
\begin{exemple}\jya a-me nɤ-rʑaβ ku-tɯ-nɯte jɤɣ\cmn 你可以娶我的女儿为妻\end{exemple}
\end{sous-entrée}\begin{sous-entrée}
\vedette{\hypertarget{}{\papi{ ɯ-taʁ,ta}}}\markboth{ɯ-taʁ,ta}{}
\paradigme{\textit{dir :} \jya nɯ-}
\begin{définition}\fra reporter la faute sur\end{définition}
\begin{définition}\cmn 归罪于\end{définition}
\begin{exemple}\jya mɤ-kɯ-pe ci tu-tɯ-tu tɕe ɯʑo ɯ-taʁ ɲɯ-ta-nɯ pjɤ-ŋu\cmn 每一次发生不好的事情的时候都归罪于他\end{exemple}
\begin{exemple}\jya aʑo ki tɯ-ŋgo ki kɤ-ɣɤmna a-mɤ-pɯ-cha-a tɕe smɤnba kɤ-βzu mɤ-kɯ-spa ɯ-taʁ ɲɯ-ta-nɯ ɲɯ-ŋu ma tɯ-ŋgo kɤ-ɣɤmna mɤ-kɯ-khɯ nɯmɯ́j-tso-nɯ\cmn 如果我治不好那个病,人们会认为医术不行,他们不懂这种病根本治不了\end{exemple}
\begin{relation-sémantique}\ComponentA{\classe{np}
 \papi{ɯ-taʁ}
}\end{relation-sémantique}
\begin{relation-sémantique}\ComponentB{\classe{vt}
\hyperlink{Ⓔta}{\textit{ \papi{ta}}}
}\end{relation-sémantique}
\end{sous-entrée}\begin{sous-entrée}
\vedette{\hypertarget{}{\papi{ znɯta}}}\markboth{znɯta}{}\classe{vt}\acception{1}
\paradigme{\textit{dir :} \jya kɤ-}
\begin{définition}\fra donner en marriage\end{définition}
\begin{définition}\cmn 许配\end{définition}\acception{2}
\paradigme{\textit{dir :} \jya nɯ-}
\begin{définition}\fra passer à\end{définition}
\begin{définition}\cmn 让给别人\end{définition}
\begin{exemple}\jya nɤʑo tɤ-tɯ-χtɯ-t nɯ, nɤʑɯɣ mɯ-mɤ-ɲɯ-ra nɤ, aʑo ɲɯ-kɯ-znɯta-a ɯ́-jɤɣ?\cmn 你买的东西如果对你没有用的话就让给我吗?\end{exemple}
\begin{relation-sémantique}\confer{
\hyperlink{Ⓔsɯta}{\textit{ \papi{sɯta}}}
}\end{relation-sémantique}
\begin{relation-sémantique}\confer{
\hyperlink{Ⓔatɯta}{\textit{ \papi{atɯta}}}
}\end{relation-sémantique}
\end{sous-entrée}\end{entrée}

\begin{entrée}
\vedette{\hypertarget{Ⓔtacɯn}{\papi{ tacɯn}}}\markboth{tacɯn}{}\classe{n}
\begin{définition}\fra habit que l'on rabat au niveau de l'épaule\end{définition}
\begin{définition}\cmn 大襟
\begin{déclaration} \étymologie{\papi{\stylefn{大襟}}}\end{déclaration}\end{définition}
\end{entrée}

\begin{entrée}
\vedette{\hypertarget{Ⓔtal}{\papi{ tal}}}\markboth{tal}{}\classe{n}
\begin{définition}\fra jianzi\end{définition}
\begin{définition}\cmn 毽子\end{définition}\end{entrée}

\begin{entrée}
\vedette{\hypertarget{Ⓔtalɤn}{\papi{ talɤn}}}\markboth{talɤn}{}\classe{n}
\begin{définition}\fra besace\end{définition}
\begin{définition}\cmn 褡裢
\begin{déclaration} \étymologie{\papi{\stylefn{褡裢}}}\end{déclaration}\end{définition}
\end{entrée}

\begin{entrée}
\vedette{\hypertarget{Ⓔta-ma}{\papi{ ta-ma}}}\markboth{ta-ma}{}\classe{np}
\begin{définition}\fra occupation\end{définition}
\begin{définition}\cmn 事情\end{définition}
\begin{exemple}\jya aʑo a-ma ku-nɤma-nɯ ɕti (=a-tshɤt ku-rɤma-nɯ ɕti)\cmn 他们在替我工作\end{exemple}\end{entrée}

\begin{entrée}
\vedette{\hypertarget{Ⓔta-mar}{\papi{ ta-mar}}}\markboth{ta-mar}{}
\classe{np}
\begin{définition}\fra beurre\end{définition}
\begin{définition}\cmn 酥油\end{définition}
\begin{relation-sémantique}\confer{
\hyperlink{Ⓔʑɯnmar}{\textit{ \papi{ʑɯnmar}}}
}\end{relation-sémantique}\end{entrée}

\begin{entrée}
\vedette{\hypertarget{Ⓔtamɢom}{\papi{ tamɢom}}}\markboth{tamɢom}{}
\classe{n}
\begin{définition}\fra tenaille, pince, pincette\end{définition}
\begin{définition}\cmn 夹子\end{définition}
\begin{relation-sémantique}\confer{
\hyperlink{Ⓔmɢom}{\textit{ \papi{mɢom}}}
}\end{relation-sémantique}
\begin{relation-sémantique}\confer{
\hyperlink{Ⓔɟɯmɢom}{\textit{ \papi{ɟɯmɢom}}}
}\end{relation-sémantique}\end{entrée}

\begin{entrée}
\vedette{\hypertarget{Ⓔtaŋ}{\papi{ taŋ}}}\markboth{taŋ}{}
\classe{vs}
\paradigme{\textit{dir :} \jya tɤ-}
\begin{définition}\fra intelligent (personne), authentique\end{définition}
\begin{définition}\cmn 聪明;真的\end{définition}
\begin{exemple}\jya pjɯrɯ ɲɯ-taŋ\cmn 珊瑚是真货\end{exemple}
\begin{exemple}\jya rŋɯl ɲɯ-taŋ\cmn 银是纯的\end{exemple}
\begin{exemple}\jya smɤɣ kɯ-taŋ\cmn 纯毛\end{exemple}\end{entrée}

\begin{entrée}
\vedette{\hypertarget{Ⓔtaŋbu}{\papi{ taŋbu}}}\markboth{taŋbu}{}\classe{n}
\begin{définition}\fra premier mois\end{définition}
\begin{définition}\cmn 一月
\begin{déclaration} \étymologie{\papi{daŋ.po}}\end{déclaration}\end{définition}
\end{entrée}

\begin{entrée}
\vedette{\hypertarget{Ⓔtaŋi}{\papi{ taŋi}}}\markboth{taŋi}{}\classe{n}
\begin{définition}\fra sécheresse\end{définition}
\begin{définition}\cmn 旱灾\end{définition}
\begin{exemple}\jya taŋi to-ɣi\cmn 有了旱灾\end{exemple}
\begin{exemple}\jya taŋi ɲɯ-thɯ ma tɯ-mɯ mɯ́j-lɤt\cmn 旱灾很严重,不下雨\end{exemple}\end{entrée}

\begin{entrée}
\vedette{\hypertarget{Ⓔtaɴɢoʁ}{\papi{ taɴɢoʁ}}}\markboth{taɴɢoʁ}{}
\classe{n}
\begin{définition}\fra vanneries, cage\end{définition}
\begin{définition}\cmn 篮子;笼子\end{définition}
\begin{exemple}\jya taɴɢoʁ tɤ-βzu-t-a\cmn 我编了篮子\end{exemple}\end{entrée}

\begin{entrée}
\vedette{\hypertarget{Ⓔtaqaβ}{\papi{ taqaβ}}}\markboth{taqaβ}{}
\classe{n}
\begin{définition}\fra aiguille\end{définition}
\begin{définition}\cmn 针\end{définition}
\begin{relation-sémantique}\confer{
\hyperlink{Ⓔrasqaβ}{\textit{ \papi{rasqaβ}}}
}\end{relation-sémantique}\end{entrée}

\begin{entrée}
\vedette{\hypertarget{Ⓔtaqaβrna}{\papi{ taqaβrna}}}\markboth{taqaβrna}{}
\classe{n}
\begin{définition}\fra chas de l'aiguille\end{définition}
\begin{définition}\cmn 针眼\end{définition}\end{entrée}

\begin{entrée}
\vedette{\hypertarget{Ⓔtar}{\papi{ tar}}}\markboth{tar}{}\classe{vi}
\paradigme{\textit{dir :} \jya thɯ-}
\paradigme{\textit{dir :} \jya tɤ-}
\begin{définition}\fra se développer\end{définition}
\begin{définition}\cmn 发展
\begin{déclaration} \étymologie{\papi{dar}}\end{déclaration}\end{définition}
\begin{exemple}\jya rgali chɤ-tar\cmn 小奶牛身体壮起来了\end{exemple}
\begin{exemple}\jya jiɕqha nɯra to-tar-nɯ\cmn 他变得比较富裕了\end{exemple}\begin{sous-entrée}
\vedette{\hypertarget{}{\papi{ sɯxtar}}}\markboth{sɯxtar}{}\classe{vt}
\begin{définition}\ 
\begin{déclaration}\grammar{caus}\end{déclaration}\end{définition}
\end{sous-entrée}\end{entrée}

\begin{entrée}
\vedette{\hypertarget{Ⓔtartɕɯn}{\papi{ tartɕɯn}}}\markboth{tartɕɯn}{}
\classe{n}
\begin{définition}\fra grand drapeau flottant au vent\end{définition}
\begin{définition}\cmn 大经幡
\begin{déclaration} \étymologie{\papi{dar.tɕʰen}}\end{déclaration}\end{définition}\end{entrée}

\begin{entrée}
\vedette{\hypertarget{ⒺtaʁⒽ3}{\papi{ taʁ}}}\markboth{taʁ}{}\homonyme{3}
\classe{adv}
\begin{définition}\fra en haut\end{définition}
\begin{définition}\cmn 上面\end{définition}
\begin{sous-entrée}
\vedette{\hypertarget{}{\papi{ ɯ-taʁ}}}\markboth{ɯ-taʁ}{}\classe{np}
\begin{définition}\fra le haut\end{définition}
\begin{définition}\cmn 上面\end{définition}
\end{sous-entrée}\end{entrée}

\begin{entrée}
\vedette{\hypertarget{ⒺtaʁⒽ1}{\papi{ taʁ}}}\markboth{taʁ}{}\homonyme{1}
\classe{vl}
\paradigme{\textit{dir :} \jya thɯ-}
\begin{définition}\fra tisser\end{définition}
\begin{définition}\cmn 织\end{définition}
\begin{exemple}\jya tɯ-ŋga ɲɯ-ɤsɯ-taʁ\cmn 他在织衣服\end{exemple}
\begin{exemple}\jya tɯŋgar ɲɯ-ɤsɯ-taʁ\cmn 他在织羊毛\end{exemple}
\begin{exemple}\jya a-mu pɯ-taʁ\cmn 我的母亲织布\end{exemple}\end{entrée}

\begin{entrée}
\vedette{\hypertarget{ⒺtaʁⒽ2}{\papi{ taʁ}}}\markboth{taʁ}{}\homonyme{2}\classe{vs}
\paradigme{\textit{dir :} \jya pɯ-}
\begin{définition}\fra clair\end{définition}
\begin{définition}\cmn 清楚;准确\end{définition}
\begin{exemple}\jya kɯ-tɯ-taʁ ɲɯ-ɕti wo\cmn 很清楚嘛\end{exemple}
\begin{exemple}\jya ɯ-tɤ-rʑaʁ wuma ɲɯ-taʁ\cmn 他的时间很准确\end{exemple}
\begin{exemple}\jya ɯ-ʁndo ɲɯ-taʁ\cmn 他的话很清楚\end{exemple}\begin{sous-entrée}
\vedette{\hypertarget{}{\papi{ atɕɯxtaʁ}}}\markboth{atɕɯxtaʁ}{}\classe{vs}
\paradigme{\textit{dir :} \jya tɤ-}
\begin{exemple}\jya kɤ-fɕɤt stɯsti kɯ mɯ-ɲɯ-ɤtɕɯxtaʁ, mɯ-ɲɯ-ɤmɯtso tɕe tɯpɤr pjɯ́-wɣ-lɤt ɲɯ-ra\cmn 光是讲述不够清楚,所以要拍照片\end{exemple}
\end{sous-entrée}\begin{sous-entrée}
\vedette{\hypertarget{}{\papi{ ɣɤtaʁ}}}\markboth{ɣɤtaʁ}{}\classe{vt}
\paradigme{\textit{dir :} \jya tɤ-}
\begin{définition}\fra dire clairement\end{définition}
\begin{définition}\cmn 说得清楚\end{définition}
\begin{exemple}\jya ki tɯ-rju ki kɤ-ɣɤtaʁ ɲɯ-ɴqa\cmn 这个词的发音很难发准\end{exemple}
\begin{exemple}\jya tú-wɣ-ɣɤtaʁ tɕe ``zdɯxpa" tu-kɯ-ti ŋu ri, tú-wɣ-sɤpɤmbat tɕe ``dɯxpa" tu-kɯ-ti ŋu\cmn 
要讲得清楚一点就应该说\stylefv{zdɯxpa},讲得简单一点就说\stylefv{dɯxpa}
\end{exemple}
\end{sous-entrée}\begin{sous-entrée}
\vedette{\hypertarget{}{\papi{ sɤtɕɯxtaʁ}}}\markboth{sɤtɕɯxtaʁ}{}\classe{vt}
\paradigme{\textit{dir :} \jya tɤ-}
\begin{définition}\fra dire clairement\end{définition}
\begin{définition}\cmn 说清楚\end{définition}
\begin{exemple}\jya tɯ-rju ta-sɤtɕɯxtaʁ\end{exemple}
\end{sous-entrée}\end{entrée}

\begin{entrée}
\vedette{\hypertarget{Ⓔta-ʁa}{\papi{ ta-ʁa}}}\markboth{ta-ʁa}{}
\classe{np}
\begin{définition}\fra temps libre\end{définition}
\begin{définition}\cmn 空的时间\end{définition}
\begin{exemple}\jya jisŋi a-ʁa tu\cmn 我今天有空\end{exemple}
\begin{exemple}\jya @xingqier a-ʁa me ma ɯ-ro ra tu\cmn 星期二没有空,剩下的时间有空\end{exemple}
\begin{exemple}\jya a-ʁa qajdo-ʁa\cmn 我像乌鸦一样闲\end{exemple}
\begin{exemple}\jya a-ʁa me\cmn 我没有空\end{exemple}\end{entrée}

\begin{entrée}
\vedette{\hypertarget{Ⓔtaʁɤndu}{\papi{ taʁɤndu}}}\markboth{taʁɤndu}{}
\classe{n}
\begin{définition}\fra échange de travail\end{définition}
\begin{définition}\cmn 还工\end{définition}
\begin{exemple}\jya a-taʁɤndu ɣɯ-tɤ-lɤt, tɕe aʑo nɤ-ʁɤndɤsi ju-ɣi-a\cmn 你先来给我做工,然后我就来给你还工\end{exemple}
\begin{exemple}\jya ɯʑo kɯ a-taʁɤndu ɣɯ-ta-lɤt tɕe, aʑo ɯ-ʁɤndɤsi ɕe-a ra\cmn 他先来给我做工,我要给他还工\end{exemple}
\begin{relation-sémantique}\confer{
\hyperlink{Ⓔta-ʁɤndɤsi}{\textit{ \papi{ta-ʁɤndɤsi}}}
}\end{relation-sémantique}
\begin{relation-sémantique}\confer{
\hyperlink{Ⓔandu}{\textit{ \papi{andu}}}
}\end{relation-sémantique}\end{entrée}

\begin{entrée}
\vedette{\hypertarget{Ⓔta-ʁɤndɤsci}{\papi{ ta-ʁɤndɤsci}}}\markboth{ta-ʁɤndɤsci}{}
\begin{relation-sémantique}\confer{
\hyperlink{Ⓔta-ʁɤndɤsi}{\textit{ \papi{ta-ʁɤndɤsi}}}
}\end{relation-sémantique}\end{entrée}

\begin{entrée}
\vedette{\hypertarget{Ⓔta-ʁɤndɤsi}{\papi{ ta-ʁɤndɤsi}}}\markboth{ta-ʁɤndɤsi}{}\classe{np}
\begin{définition}\fra échange de travail\end{définition}
\begin{définition}\cmn 还工\end{définition}
\begin{exemple}\jya a-taʁɤndu ɣɯ-tɤ-lɤt, tɕe aʑo nɤ-ʁɤndɤsi ju-ɣi-a\cmn 你先来给我做工,然后我就来给你还工\end{exemple}
\begin{exemple}\jya ɯʑo kɯ a-taʁɤndu ɣɯ-ta-lɤt tɕe, aʑo ɯ-ʁɤndɤsi ɕe-a ra\cmn 他先来给我做工,我要给他还工\end{exemple}
\begin{relation-sémantique}\confer{
\hyperlink{Ⓔandu}{\textit{ \papi{andu}}}
}\end{relation-sémantique}
\begin{relation-sémantique}\confer{
\hyperlink{Ⓔtaʁɤndu}{\textit{ \papi{taʁɤndu}}}
}\end{relation-sémantique}\end{entrée}

\begin{entrée}
\vedette{\hypertarget{Ⓔta-ʁi}{\papi{ ta-ʁi}}}\markboth{ta-ʁi}{}
\classe{np}
\begin{définition}\fra petit frère, petite sœur\end{définition}
\begin{définition}\cmn 弟弟;妹妹\end{définition}\end{entrée}

\begin{entrée}
\vedette{\hypertarget{Ⓔta-ʁjɯβ}{\papi{ ta-ʁjɯβ}}}\markboth{ta-ʁjɯβ}{}
\classe{np}
\begin{définition}\fra ombre\end{définition}
\begin{définition}\cmn 影子\end{définition}
\begin{relation-sémantique}\confer{
\hyperlink{Ⓔnaʁjɯβ}{\textit{ \papi{naʁjɯβ}}}
}\end{relation-sémantique}
\begin{relation-sémantique}\confer{
\hyperlink{Ⓔsqaβjɯβ}{\textit{ \papi{sqaβjɯβ}}}
}\end{relation-sémantique}\end{entrée}

\begin{entrée}
\vedette{\hypertarget{Ⓔta-ʁɟaz}{\papi{ ta-ʁɟaz}}}\markboth{ta-ʁɟaz}{}
\classe{np}\acception{1}
\begin{définition}\fra suie\end{définition}
\begin{définition}\cmn 碳黑【锅烟墨】\end{définition}\acception{2}
\begin{définition}\fra défaut\end{définition}
\begin{définition}\cmn 缺点\end{définition}
\begin{sous-entrée}
\vedette{\hypertarget{}{\papi{ ta-ʁɟaz,mar}}}\markboth{ta-ʁɟaz,mar}{}
\paradigme{\textit{dir :} \jya nɯ-}
\begin{définition}\fra décharger sa faute sur\end{définition}
\begin{définition}\cmn 令……背黑锅\end{définition}
\begin{exemple}\jya nɤʑo nɤ-ʁɟaz nɯ aʑo ma-nɯ-kɯ-mar-a\cmn 别让我替你背黑锅\end{exemple}
\begin{relation-sémantique}\ComponentA{\classe{np}
\hyperlink{Ⓔta-ʁɟaz}{\textit{ \papi{ta-ʁɟaz}}}
}\end{relation-sémantique}
\begin{relation-sémantique}\ComponentB{\classe{vt}
\hyperlink{Ⓔmar}{\textit{ \papi{mar}}}
}\end{relation-sémantique}
\end{sous-entrée}\end{entrée}

\begin{entrée}
\vedette{\hypertarget{Ⓔtaʁki}{\papi{ taʁki}}}\markboth{taʁki}{}\classe{n}
\begin{définition}\fra de haut en bas\end{définition}
\begin{définition}\cmn 上下\end{définition}
\begin{relation-sémantique}\confer{
\hyperlink{ⒺtaʁⒽ3}{\textit{ \papi{taʁ3}}}
}\end{relation-sémantique}
\end{entrée}

\begin{entrée}
\vedette{\hypertarget{Ⓔtaʁlɤkɯm}{\papi{ taʁlɤkɯm}}}\markboth{taʁlɤkɯm}{}\classe{n}
\begin{définition}\fra balcon\end{définition}
\begin{définition}\cmn 藏式房屋的阳台\end{définition}
\end{entrée}

\begin{entrée}
\vedette{\hypertarget{Ⓔtaʁmbra,lɤt}{\papi{ taʁmbra,lɤt}}}\markboth{taʁmbra,lɤt}{}\paradigme{\textit{dir :} \jya nɯ-}\acception{1}
\begin{définition}\fra pleurer en s'agitant de toutes ses forces (enfant)\end{définition}
\begin{définition}\cmn 哭得大吵大闹(小孩子)\end{définition}
\begin{exemple}\jya tɤ-pɤtso tɤ-wu ɯ-tɯ-χɕu kɯ taʁmbra ʑo ɲɯ-lɤt ɲɯ-ɕti\cmn 小孩哭乱叫乱吼\end{exemple}\acception{2}
\begin{définition}\fra s'élancer en sautant (cheval)\end{définition}
\begin{définition}\cmn 跳过去(马)\end{définition}
\begin{relation-sémantique}\ComponentA{\classe{n}
 \papi{taʁmbra}
}\end{relation-sémantique}
\begin{relation-sémantique}\ComponentB{\classe{vt}
\hyperlink{ⒺlɤtⒽ1}{\textit{ \papi{lɤt}}}
}\end{relation-sémantique}
\begin{relation-sémantique}\confer{
\hyperlink{ⒺlɤtⒽ1}{\textit{ \papi{lɤt1}}}
}\end{relation-sémantique}\end{entrée}

\begin{entrée}
\vedette{\hypertarget{Ⓔtaʁndo}{\papi{ taʁndo}}}\markboth{taʁndo}{}
\classe{n}
\begin{définition}\fra parole\end{définition}
\begin{définition}\cmn 话(长辈对下辈、老师对学生、领导对属下等)\end{définition}
\begin{relation-sémantique}\confer{
\hyperlink{Ⓔnɯstɤraʁndo}{\textit{ \papi{nɯstɤraʁndo}}}
}\end{relation-sémantique}
\begin{relation-sémantique}\confer{
\hyperlink{Ⓔnɯʁndomnɤt}{\textit{ \papi{nɯʁndomnɤt}}}
}\end{relation-sémantique}\end{entrée}

\begin{entrée}
\vedette{\hypertarget{Ⓔtaʁndzɤr}{\papi{ taʁndzɤr}}}\markboth{taʁndzɤr}{}
\classe{n}
\begin{définition}\fra seau servant à mettre la pâtée des cochons\end{définition}
\begin{définition}\cmn 喂猪用的木桶\end{définition}\end{entrée}

\begin{entrée}
\vedette{\hypertarget{Ⓔta-ʁrɤt}{\papi{ ta-ʁrɤt}}}\markboth{ta-ʁrɤt}{}
\classe{np}
\begin{définition}\fra charbon de bois\end{définition}
\begin{définition}\cmn 碳\end{définition}
\begin{exemple}\jya si pɯ-nɯt ɯ-qhu tɕe ɯ-ʁrɤt ɲɯ-βze ŋu\cmn 把树木烧了之后就会变成碳\end{exemple}
\begin{relation-sémantique}\confer{
\hyperlink{Ⓔraʁrɤt}{\textit{ \papi{raʁrɤt}}}
}\end{relation-sémantique}\end{entrée}

\begin{entrée}
\vedette{\hypertarget{Ⓔtaʁrdo}{\papi{ taʁrdo}}}\markboth{taʁrdo}{}\classe{n}
\begin{définition}\ 
\begin{déclaration}\grammar{n.lieu}\end{déclaration}\end{définition}
\begin{définition}\fra l'un des hameaux de Kamnyu\end{définition}
\begin{définition}\cmn 干木鸟的大队之一\end{définition}
\end{entrée}

\begin{entrée}
\vedette{\hypertarget{Ⓔta-ʁri}{\papi{ ta-ʁri}}}\markboth{ta-ʁri}{}
\classe{np}
\begin{définition}\fra saleté\end{définition}
\begin{définition}\cmn 污垢\end{définition}
\begin{exemple}\jya ɯʑo ɯ-ʁri\cmn 他身上的污垢\end{exemple}\end{entrée}

\begin{entrée}
\vedette{\hypertarget{Ⓔta-ʁrɯ}{\papi{ ta-ʁrɯ}}}\markboth{ta-ʁrɯ}{}
\classe{np}\acception{1}
\begin{définition}\fra corne\end{définition}
\begin{définition}\cmn 角\end{définition}\acception{2}
\begin{définition}\fra cor\end{définition}
\begin{définition}\cmn 趼子\end{définition}
\begin{relation-sémantique}\confer{
\hyperlink{Ⓔɣɤʁrɯ}{\textit{ \papi{ɣɤʁrɯ}}}
}\end{relation-sémantique}
\begin{relation-sémantique}\confer{
\hyperlink{Ⓔʁrɯlu}{\textit{ \papi{ʁrɯlu}}}
}\end{relation-sémantique}\end{entrée}

\begin{entrée}
\vedette{\hypertarget{Ⓔta-ʁrɯm}{\papi{ ta-ʁrɯm}}}\markboth{ta-ʁrɯm}{}
\classe{np}\acception{1}
\begin{définition}\fra lumière\end{définition}
\begin{définition}\cmn 光\end{définition}
\begin{exemple}\jya tɤtʂu ɯ-ʁrɯm\cmn 灯的光\end{exemple}\acception{2}
\begin{définition}\fra reflet\end{définition}
\begin{définition}\cmn 倒影\end{définition}
\begin{exemple}\jya tɯ-ci ɯ-ŋgɯ a-ʁrɯm pjɤ-ntɕhɤr\cmn 水中有我的倒影\end{exemple}\acception{3}
\begin{définition}\fra endroit frais à l'ombre, ombre\end{définition}
\begin{définition}\cmn 阴凉的地方,影子\end{définition}
\begin{exemple}\jya kha ɯ-ʁrɯm\cmn 房子的影子\end{exemple}
\begin{relation-sémantique}\confer{
\hyperlink{Ⓔtɯmɯʁrɯm}{\textit{ \papi{tɯmɯʁrɯm}}}
}\end{relation-sémantique}
\begin{relation-sémantique}\confer{
\hyperlink{Ⓔsaʁrɯm}{\textit{ \papi{saʁrɯm}}}
}\end{relation-sémantique}\end{entrée}

\begin{entrée}
\vedette{\hypertarget{Ⓔtasa}{\papi{ tasa}}}\markboth{tasa}{}
\classe{n}
\begin{définition}\fra chanvre\end{définition}
\begin{définition}\cmn 大麻\end{définition}
\begin{relation-sémantique}\confer{
\hyperlink{Ⓔtɤsɤmu}{\textit{ \papi{tɤsɤmu}}}
}\end{relation-sémantique}
\begin{relation-sémantique}\confer{
\hyperlink{Ⓔtɤsɤɣʑa}{\textit{ \papi{tɤsɤɣʑa}}}
}\end{relation-sémantique}
\begin{relation-sémantique}\confer{
\hyperlink{Ⓔtɤsɤrŋu}{\textit{ \papi{tɤsɤrŋu}}}
}\end{relation-sémantique}
\begin{relation-sémantique}\confer{
\hyperlink{Ⓔtɤsɤsqɤri}{\textit{ \papi{tɤsɤsqɤri}}}
}\end{relation-sémantique}\end{entrée}

\begin{entrée}
\vedette{\hypertarget{Ⓔtatɕhoŋtɕhoŋ}{\papi{ tatɕhoŋtɕhoŋ}}}\markboth{tatɕhoŋtɕhoŋ}{}\classe{n}
\begin{définition}\fra chute d'eau\end{définition}
\begin{définition}\cmn 瀑布\end{définition}\end{entrée}

\begin{entrée}
\vedette{\hypertarget{Ⓔtatpa}{\papi{ tatpa}}}\markboth{tatpa}{}\classe{n}
\begin{définition}\fra foi\end{définition}
\begin{définition}\cmn 信仰
\begin{déclaration} \étymologie{\papi{dad.pa}}\end{déclaration}\end{définition}
\begin{exemple}\jya tatpa tɤ-ta-t-a\cmn 我崇拜(了)他\end{exemple}\end{entrée}

\begin{entrée}
\vedette{\hypertarget{Ⓔtatshi}{\papi{ tatshi}}}\markboth{tatshi}{}\classe{n}
\begin{définition}\ 
\begin{déclaration}\grammar{n.lieu}\end{déclaration}\end{définition}
\begin{définition}\fra Datshang\end{définition}
\begin{définition}\cmn 大藏乡\end{définition}\end{entrée}

\begin{entrée}
\vedette{\hypertarget{Ⓔtaχpa}{\papi{ taχpa}}}\markboth{taχpa}{}\classe{n}
\begin{définition}\fra récolte d'un an\end{définition}
\begin{définition}\cmn 一年的庄稼\end{définition}
\begin{exemple}\jya ɣɯjpa taχpa ɯ-ɲɯ-pe ?\cmn 今年收成好不好?\end{exemple}
\begin{exemple}\jya taχpa ɲɯ-nɤkɤro, ɲɯ-pe ɕti\cmn 庄稼还可以\end{exemple}\end{entrée}

\begin{entrée}
\vedette{\hypertarget{Ⓔtaχphe}{\papi{ taχphe}}}\markboth{taχphe}{}\classe{n}
\begin{définition}\fra paume de la main (utilisée pour baffer)\end{définition}
\begin{définition}\cmn 手掌(掴耳光)\end{définition}
\begin{exemple}\jya tɕoχtsi ɯ-taʁ taχphe pɯ-lat-a\cmn 我拍了桌子\end{exemple}
\begin{relation-sémantique}\confer{
\hyperlink{Ⓔnɤχphe}{\textit{ \papi{nɤχphe}}}
}\end{relation-sémantique}\end{entrée}

\begin{entrée}
\vedette{\hypertarget{Ⓔta-χpi}{\papi{ ta-χpi}}}\markboth{ta-χpi}{}
\classe{np}
\begin{définition}\fra forme, modèle\end{définition}
\begin{définition}\cmn 形状;榜样
\begin{déclaration} \étymologie{\papi{dpe}}\end{déclaration}\end{définition}
\begin{exemple}\jya ɯʑo a-ta-χpi sna\cmn 他值得做我的榜样\end{exemple}
\begin{relation-sémantique}\confer{
\hyperlink{Ⓔznɯχpi}{\textit{ \papi{znɯχpi}}}
}\end{relation-sémantique}\end{entrée}

\begin{entrée}
\vedette{\hypertarget{Ⓔtaχti}{\papi{ taχti}}}\markboth{taχti}{}\classe{n}
\begin{définition}\fra modèle\end{définition}
\begin{définition}\cmn 榜样\end{définition}
\begin{exemple}\jya ɯ-taχti ɲɯ-sna\cmn 可以当作榜样\end{exemple}
\end{entrée}

\begin{entrée}
\vedette{\hypertarget{Ⓔtɤβ}{\papi{ tɤβ}}}\markboth{tɤβ}{}
\classe{vl}
\paradigme{\textit{dir :} \jya pɯ-}
\paradigme{\textit{dir :} \jya kɤ-}
\begin{définition}\fra battre le grain\end{définition}
\begin{définition}\cmn 脱粒\end{définition}
\begin{exemple}\jya ʑara pɯ-tɤβ-nɯ\cmn 他们脱粒了\end{exemple}
\begin{exemple}\jya tɤɕi pɯ-taβ-a\cmn 我把青稞脱了粒\end{exemple}
\begin{exemple}\jya qaj pɯ-taβ-a\cmn 我把小麦脱了粒\end{exemple}\end{entrée}

\begin{entrée}
\vedette{\hypertarget{Ⓔtɤβɣemu}{\papi{ tɤβɣemu}}}\markboth{tɤβɣemu}{}
\classe{n}
\begin{définition}\fra veuve\end{définition}
\begin{définition}\cmn 寡妇\end{définition}\end{entrée}

\begin{entrée}
\vedette{\hypertarget{Ⓔtɤβɣepɯ}{\papi{ tɤβɣepɯ}}}\markboth{tɤβɣepɯ}{}\classe{n}
\begin{définition}\fra orphelin\end{définition}
\begin{définition}\cmn 孤儿\end{définition}
\end{entrée}

\begin{entrée}
\vedette{\hypertarget{Ⓔtɤβɣerɟit}{\papi{ tɤβɣerɟit}}}\markboth{tɤβɣerɟit}{}\classe{n}
\begin{définition}\fra orphelin\end{définition}
\begin{définition}\cmn 孤儿\end{définition}
\end{entrée}

\begin{entrée}
\vedette{\hypertarget{Ⓔtɤβɣewa}{\papi{ tɤβɣewa}}}\markboth{tɤβɣewa}{}\classe{n}
\begin{définition}\fra veuf\end{définition}
\begin{définition}\cmn 鳏夫\end{définition}
\end{entrée}

\begin{entrée}
\vedette{\hypertarget{Ⓔtɤ-βɣo}{\papi{ tɤ-βɣo}}}\markboth{tɤ-βɣo}{}\classe{np}\acception{1}
\begin{définition}\fra oncle (frère du père, mari de la sœur du père ou mari de la sœur de la mère)\end{définition}
\begin{définition}\cmn 伯父;叔叔\end{définition}\acception{2}
\begin{définition}\fra lama\end{définition}
\begin{définition}\cmn 对喇嘛的尊称\end{définition}\end{entrée}

\begin{entrée}
\vedette{\hypertarget{Ⓔtɤ-βɟu}{\papi{ tɤ-βɟu}}}\markboth{tɤ-βɟu}{}
\classe{np}
\begin{définition}\fra matelas\end{définition}
\begin{définition}\cmn 褥子\end{définition}
\begin{exemple}\jya a-βɟu\cmn 我的褥子\end{exemple}\end{entrée}

\begin{entrée}
\vedette{\hypertarget{Ⓔtɤβri}{\papi{ tɤβri}}}\markboth{tɤβri}{}
\classe{n}
\begin{définition}\fra écheveau\end{définition}
\begin{définition}\cmn 一绞\end{définition}
\begin{exemple}\jya tasa lɤ-pɣo-t-a, lɤ-rɯm-a, tɤβri tɤ-βzu-t-a\cmn 我捻了大麻,搓成绞\end{exemple}\end{entrée}

\begin{entrée}
\vedette{\hypertarget{Ⓔtɤβʁa}{\papi{ tɤβʁa}}}\markboth{tɤβʁa}{}\classe{n}
\begin{définition}\fra (avec) agressivité\end{définition}
\begin{définition}\cmn 气势汹汹\end{définition}
\begin{exemple}\jya tɤβʁa kɯ kha jo-zɣɯt\cmn 他气势汹汹地到了他家\end{exemple}
\begin{relation-sémantique}\confer{
\hyperlink{Ⓔβʁa}{\textit{ \papi{βʁa}}}
}\end{relation-sémantique}\end{entrée}

\begin{entrée}
\vedette{\hypertarget{Ⓔtɤ-βzdɤr}{\papi{ tɤ-βzdɤr}}}\markboth{tɤ-βzdɤr}{}
\classe{np}
\begin{définition}\fra beurre (que l'on met dans le thé ou la tsampa)\end{définition}
\begin{définition}\cmn 加(在糌粑里、在茶里)的酥油
\begin{déclaration} \étymologie{\papi{sdor}}\end{déclaration}\end{définition}
\begin{exemple}\jya a-mu kɯ a-βzdɤr pa-lɤt\cmn 母亲给我加了酥油\end{exemple}
\begin{relation-sémantique}\confer{
\hyperlink{Ⓔβzdɤr}{\textit{ \papi{βzdɤr}}}
}\end{relation-sémantique}\end{entrée}

\begin{entrée}
\vedette{\hypertarget{Ⓔtɤ-cɤβ}{\papi{ tɤ-cɤβ}}}\markboth{tɤ-cɤβ}{}\classe{np}
\begin{définition}\fra espèce\end{définition}
\begin{définition}\cmn 种类\end{définition}
\begin{exemple}\jya qarmɯrwa nɯ βʑɯ tɤ-cɤβ ŋu ɕi, pɣa tɤ-cɤβ ŋu mɤ-χsɤl\cmn 不知道蝙蝠是老鼠的一种、还是鸟的一种\end{exemple}\end{entrée}

\begin{entrée}
\vedette{\hypertarget{Ⓔtɤ-chɯ}{\papi{ tɤ-chɯ}}}\markboth{tɤ-chɯ}{}
\classe{np}
\begin{définition}\fra coin\end{définition}
\begin{définition}\cmn 楔子
\begin{déclaration} \étymologie{\papi{kʰʲewu}}\end{déclaration}\end{définition}\end{entrée}

\begin{entrée}
\vedette{\hypertarget{Ⓔtɤcoʁcoʁ}{\papi{ tɤcoʁcoʁ}}}\markboth{tɤcoʁcoʁ}{}\classe{n}
\begin{définition}\fra espèce d'oiseau\end{définition}
\begin{définition}\cmn 一种鸟\end{définition}
\end{entrée}

\begin{entrée}
\vedette{\hypertarget{Ⓔtɤɕu}{\papi{ tɤɕu}}}\markboth{tɤɕu}{}\classe{n}
\begin{définition}\fra fraicheur\end{définition}
\begin{définition}\cmn 阴凉\end{définition}
\begin{exemple}\jya tɤɕu ɣɤʑu (=ɲɯ-ɣɤɕu)\cmn (这里)很凉快\end{exemple}
\begin{relation-sémantique}\confer{
\hyperlink{Ⓔnɤɕu}{\textit{ \papi{nɤɕu}}}
}\end{relation-sémantique}
\begin{relation-sémantique}\confer{
\hyperlink{Ⓔɣɤɕu}{\textit{ \papi{ɣɤɕu}}}
}\end{relation-sémantique}\end{entrée}

\begin{entrée}
\vedette{\hypertarget{Ⓔtɤɕɤɣrum}{\papi{ tɤɕɤɣrum}}}\markboth{tɤɕɤɣrum}{}
\classe{n}
\begin{définition}\fra orge blanc\end{définition}
\begin{définition}\cmn 白青稞\end{définition}\end{entrée}

\begin{entrée}
\vedette{\hypertarget{Ⓔtɤɕɤɲaʁ}{\papi{ tɤɕɤɲaʁ}}}\markboth{tɤɕɤɲaʁ}{}\classe{n}
\begin{définition}\fra orge noir\end{définition}
\begin{définition}\cmn 黑青稞\end{définition}\end{entrée}

\begin{entrée}
\vedette{\hypertarget{Ⓔtɤɕɤrloʁ}{\papi{ tɤɕɤrloʁ}}}\markboth{tɤɕɤrloʁ}{}
\classe{n}
\begin{définition}\fra orge à barbe courte\end{définition}
\begin{définition}\cmn 短芒的青稞\end{définition}\end{entrée}

\begin{entrée}
\vedette{\hypertarget{Ⓔtɤɕɤrmbjɤβ}{\papi{ tɤɕɤrmbjɤβ}}}\markboth{tɤɕɤrmbjɤβ}{}
\classe{n}
\begin{définition}\fra orge en bottes\end{définition}
\begin{définition}\cmn 捆成一把的青稞杆\end{définition}\end{entrée}

\begin{entrée}
\vedette{\hypertarget{Ⓔtɤɕɤt}{\papi{ tɤɕɤt}}}\markboth{tɤɕɤt}{}
\classe{n}
\begin{définition}\fra peigne\end{définition}
\begin{définition}\cmn 梳子
\begin{déclaration} \étymologie{\papi{ɕad}}\end{déclaration}\end{définition}\end{entrée}

\begin{entrée}
\vedette{\hypertarget{Ⓔtɤɕi}{\papi{ tɤɕi}}}\markboth{tɤɕi}{}\classe{n}
\begin{définition}\fra orge\end{définition}
\begin{définition}\cmn 青稞
\end{définition}
\begin{exemple}\jya tɤɕi nɯ jiʑo ra ji-kɤ-ndza pjɯ-me mɤ-kɯ-khɯ ʑo ŋu, tɤɕi nɯ tɤ-ɬoʁ ɕɯmɯma tɕe, ɯ-jwaʁ rɟum tsa tɕe tɤ-rɲɟi tɕe ɯ-ku amtɕoʁ, ɯ-zrɤm dɤn, ɯ-ru lo-rɤrtsɯrtsɤɣ ŋu, ɯ-rtsɤɣ ɯ-pɤrthɤβ nɯ ɯ-ru tɯ-tɣa ro ro ntsɯ tu, ɯ-rtsɤɣ kɯβde jamar ta-lɤt tɕe, ɯ-kɯɕnom tu-lɤt ŋu. ɯ-kɯɕnom ɣɯ ɯ-ru nɯ lɤ-ɬoʁ tɕe ɯ-rtsɤɣ me, tɕe ɯ-kɯɕnom kɯ-wxti nɯ ra tɯ-tɣa kɯ-tu tu, ɯ-kɯɕnom ɯ-rdoʁ raŋri ɣɯ ɯ-ʑmbraʁ kɯ-zɯ-zri ʑo tu, ɯ-ʑmbraʁ wuma rʁom. tɤɕi nɯ wuma ʑo arɤphɤjqa tɕe tɯ-phɯ nɯ tɕu ɕnɤcɤ-ldʑa ŋgɯsqɯ-ldʑa jamar ɲɯ-βze kɯ-cha tu. tɤɕi nɯ tɯ-sqar ɯ-spa ŋu.\cmn 青稞是我们必不可少的食物。青稞刚长出来的时候,叶子比较宽,顶端是尖的,根须多,茎是一节一节的,每一节之间的茎有一拃多长,一般长出四节就抽穗。穗干没有节,穗子有的有一拃长,穗子上每一颗粒上都有一根芒,十分粗糙。一颗青稞可以长出很多根苗,七八根,九十根的都有。青稞是糌粑的原料。\end{exemple}\end{entrée}

\begin{entrée}
\vedette{\hypertarget{Ⓔtɤɕime}{\papi{ tɤɕime}}}\markboth{tɤɕime}{}\classe{n}
\begin{définition}\fra jeune fille\end{définition}
\begin{définition}\cmn 小姐\end{définition}
\end{entrée}

\begin{entrée}
\vedette{\hypertarget{Ⓔtɤɕiʑmbraʁ}{\papi{ tɤɕiʑmbraʁ}}}\markboth{tɤɕiʑmbraʁ}{}
\classe{n}
\begin{définition}\fra barbe d'orge\end{définition}
\begin{définition}\cmn 青稞芒\end{définition}\end{entrée}

\begin{entrée}
\vedette{\hypertarget{Ⓔtɤ-ɕnɤz}{\papi{ tɤ-ɕnɤz}}}\markboth{tɤ-ɕnɤz}{}
\classe{np}
\begin{définition}\fra bout d'un fil\end{définition}
\begin{définition}\cmn 线的一端\end{définition}
\begin{exemple}\jya tɤ-ri ɲɤ-k-ɤɬɯt-ci tɕe ɯ-ɕnɤz mɯ-ɲɤ-sɤmto tɕe kɤ-ɕar ɲɯ-ra\cmn 线乱了看不到头绪,要把它的一端找出来\end{exemple}
\begin{exemple}\jya tɤ-ri ɯ-ɕnɤz ɲɯ́-wɣ-ɕar tɕe tú-wɣ-ɣɤrtɯm\cmn 把线的一端找出来,(然后)把线缠起来\end{exemple}
\begin{relation-sémantique}\confer{
\hyperlink{Ⓔtɯ-pɤɕnɤz}{\textit{ \papi{tɯ-pɤɕnɤz}}}
}\end{relation-sémantique}\end{entrée}

\begin{entrée}
\vedette{\hypertarget{Ⓔtɤɕpaʁ}{\papi{ tɤɕpaʁ}}}\markboth{tɤɕpaʁ}{}
\classe{n}
\begin{définition}\fra soif\end{définition}
\begin{définition}\cmn 口干\end{définition}
\begin{exemple}\jya tɤɕpaʁ ɲɤ-nɤɕqa tɕe mɯ-ko-tshi\cmn 他忍住口干没有喝\end{exemple}
\begin{relation-sémantique}\confer{
\hyperlink{Ⓔɕpaʁ}{\textit{ \papi{ɕpaʁ}}}
}\end{relation-sémantique}\end{entrée}

\begin{entrée}
\vedette{\hypertarget{Ⓔtɤ-ɕphɤt}{\papi{ tɤ-ɕphɤt}}}\markboth{tɤ-ɕphɤt}{}
\classe{np}
\begin{définition}\fra pièce de tissu pour raccommoder les habits\end{définition}
\begin{définition}\cmn 补丁\end{définition}
\begin{exemple}\jya tɯ-ŋga ɯ-ɕphɤt kɤ-ta-t-a\cmn 衣服上打了补丁\end{exemple}
\begin{relation-sémantique}\confer{
\hyperlink{Ⓔɕphɤt}{\textit{ \papi{ɕphɤt}}}
}\end{relation-sémantique}\end{entrée}

\begin{entrée}
\vedette{\hypertarget{Ⓔtɤɕphɤtta}{\papi{ tɤɕphɤtta}}}\markboth{tɤɕphɤtta}{}\classe{n}
\begin{définition}\fra type de pas d'aiguille\end{définition}
\begin{définition}\cmn 小针脚的缝法(沿着补丁的边缘)\end{définition}\end{entrée}

\begin{entrée}
\vedette{\hypertarget{Ⓔtɤɕqali}{\papi{ tɤɕqali}}}\markboth{tɤɕqali}{}\classe{n}
\begin{définition}\fra cri\end{définition}
\begin{définition}\cmn 叫声\end{définition}
\begin{exemple}\jya tɤɕqalɯli ta-βzu (=tɤ-ɣɤɕqali)\cmn 大声地叫喊了一下\end{exemple}
\begin{relation-sémantique}\confer{
\hyperlink{Ⓔɣɤɕqali}{\textit{ \papi{ɣɤɕqali}}}
}\end{relation-sémantique}\end{entrée}

\begin{entrée}
\vedette{\hypertarget{Ⓔtɤ-ɕqhe}{\papi{ tɤ-ɕqhe}}}\markboth{tɤ-ɕqhe}{}\classe{np}
\begin{définition}\fra toux\end{définition}
\begin{définition}\cmn 咳嗽\end{définition}
\begin{exemple}\jya nɤʑo tɤ-ɕqhe ɯ-ɲɯ-mna ?\cmn 你的咳嗽好了没有?\end{exemple}
\begin{exemple}\jya nɤʑo nɤ-ɕqhe ɯβrɤ-ɣɤʑu?\cmn 你有没有咳嗽?\end{exemple}
\begin{exemple}\jya nɤʑo nɤ-ɕqhe ɯ-ɲɯ-mna ?\cmn 你的咳嗽好了没有?\end{exemple}
\begin{exemple}\jya tɤ-ɕqhe ɣɤʑu ɕi ?\cmn 你咳嗽吗?\end{exemple}
\begin{relation-sémantique}\confer{
\hyperlink{Ⓔaɕqhe}{\textit{ \papi{aɕqhe}}}
}\end{relation-sémantique}\end{entrée}

\begin{entrée}
\vedette{\hypertarget{Ⓔtɤɕqraʁ}{\papi{ tɤɕqraʁ}}}\markboth{tɤɕqraʁ}{}\classe{n}
\begin{définition}\fra astuce\end{définition}
\begin{définition}\cmn (耍)聪明,(耍)诡计\end{définition}
\begin{exemple}\jya tɤɕqraʁ kɯ pjɯ-ʑɣɤsɯβʁa ɕti\cmn 他耍了诡计就赢了\end{exemple}
\begin{exemple}\jya ɯʑo kɯ tɤɕqraʁ ntsɯ tu-βze ŋgrɤl\cmn 他总是耍聪明\end{exemple}
\begin{relation-sémantique}\antonyme{
\hyperlink{Ⓔtɤkhe}{\textit{ \papi{tɤkhe}}}
}\end{relation-sémantique}
\begin{relation-sémantique}\confer{
\hyperlink{Ⓔɕqraʁ}{\textit{ \papi{ɕqraʁ}}}
}\end{relation-sémantique}\end{entrée}

\begin{entrée}
\vedette{\hypertarget{Ⓔtɤ-di}{\papi{ tɤ-di}}}\markboth{tɤ-di}{}\classe{np}\acception{1}
\begin{définition}\fra odeur\end{définition}
\begin{définition}\cmn 气味\end{définition}\acception{2}
\begin{définition}\fra puanteur\end{définition}
\begin{définition}\cmn 臭味\end{définition}
\begin{relation-sémantique}\confer{
\hyperlink{Ⓔɯ-dɯɕŋaʁ}{\textit{ \papi{ɯ-dɯɕŋaʁ}}}
}\end{relation-sémantique}
\begin{relation-sémantique}\confer{
\hyperlink{Ⓔɯ-dɯχɯn}{\textit{ \papi{ɯ-dɯχɯn}}}
}\end{relation-sémantique}
\begin{relation-sémantique}\confer{
\hyperlink{Ⓔkɯɲidi}{\textit{ \papi{kɯɲidi}}}
}\end{relation-sémantique}
\end{entrée}

\begin{entrée}
\vedette{\hypertarget{Ⓔtɤ-fkaβ}{\papi{ tɤ-fkaβ}}}\markboth{tɤ-fkaβ}{}
\classe{np}
\begin{définition}\fra couvercle\end{définition}
\begin{définition}\cmn 盖子
\begin{déclaration} \étymologie{\papi{ⁿgebs}}\end{déclaration}\end{définition}
\begin{relation-sémantique}\confer{
\hyperlink{Ⓔfkaβ}{\textit{ \papi{fkaβ}}}
}\end{relation-sémantique}\end{entrée}

\begin{entrée}
\vedette{\hypertarget{Ⓔtɤ-fkɯm}{\papi{ tɤ-fkɯm}}}\markboth{tɤ-fkɯm}{}
\classe{np}
\begin{définition}\fra récipient\end{définition}
\begin{définition}\cmn 可以装东西的物体(如口袋、盆子等)\end{définition}\end{entrée}

\begin{entrée}
\vedette{\hypertarget{Ⓔtɤ-fsa}{\papi{ tɤ-fsa}}}\markboth{tɤ-fsa}{}
\classe{np}
\begin{définition}\fra piège\end{définition}
\begin{définition}\cmn 圈套;陷阱\end{définition}
\begin{exemple}\jya kɯki qala ɯ-fsa tɤ-lat-a\cmn 这是我给兔子布下的陷阱\end{exemple}
\begin{exemple}\jya kɯki aʑo a-tɤ-fsa ŋu\cmn 这是我的圈套\end{exemple}\end{entrée}

\begin{entrée}
\vedette{\hypertarget{Ⓔtɤfsaŋ}{\papi{ tɤfsaŋ}}}\markboth{tɤfsaŋ}{}\classe{n}
\begin{définition}\fra feuilles de genévrier\end{définition}
\begin{définition}\cmn 柏树叶\end{définition}
\begin{relation-sémantique}\confer{
\hyperlink{Ⓔfsaŋ}{\textit{ \papi{fsaŋ}}}
}\end{relation-sémantique}\end{entrée}

\begin{entrée}
\vedette{\hypertarget{Ⓔtɤ-fsɤri}{\papi{ tɤ-fsɤri}}}\markboth{tɤ-fsɤri}{}\classe{np}
\begin{définition}\fra ficelle en lin\end{définition}
\begin{définition}\cmn 麻绳\end{définition}
\begin{relation-sémantique}\confer{
\hyperlink{Ⓔtasa}{\textit{ \papi{tasa}}}
}\end{relation-sémantique}
\begin{relation-sémantique}\confer{
\hyperlink{Ⓔtɤ-ri}{\textit{ \papi{tɤ-ri}}}
}\end{relation-sémantique}
\begin{relation-sémantique}\confer{
\hyperlink{Ⓔrɯfsɤri}{\textit{ \papi{rɯfsɤri}}}
}\end{relation-sémantique}\end{entrée}

\begin{entrée}
\vedette{\hypertarget{Ⓔtɤfsjit}{\papi{ tɤfsjit}}}\markboth{tɤfsjit}{}
\classe{n}
\begin{définition}\fra siffler\end{définition}
\begin{définition}\cmn 口哨\end{définition}
\begin{exemple}\jya tɤfsjit ci thɯ-lat-a\cmn 我吹了口哨\end{exemple}
\begin{relation-sémantique}\confer{
\hyperlink{Ⓔrɤfsjit}{\textit{ \papi{rɤfsjit}}}
}\end{relation-sémantique}\end{entrée}

\begin{entrée}
\vedette{\hypertarget{Ⓔtɤfsɯr}{\papi{ tɤfsɯr}}}\markboth{tɤfsɯr}{}
\classe{n}
\begin{définition}\fra target\end{définition}
\begin{définition}\cmn 靶子\end{définition}\end{entrée}

\begin{entrée}
\vedette{\hypertarget{Ⓔtɤ-ftaʁ}{\papi{ tɤ-ftaʁ}}}\markboth{tɤ-ftaʁ}{}\classe{np}
\begin{définition}\fra signe\end{définition}
\begin{définition}\cmn 记号
\begin{déclaration} \étymologie{\papi{rtags}}\end{déclaration}\end{définition}
\begin{exemple}\jya ɯ-ftaʁ to-ta\cmn 他做了记号\end{exemple}\end{entrée}

\begin{entrée}
\vedette{\hypertarget{Ⓔtɤ-ftsa}{\papi{ tɤ-ftsa}}}\markboth{tɤ-ftsa}{}
\classe{np}
\begin{définition}\fra neveux (enfants de la sœur)\end{définition}
\begin{définition}\cmn 外甥\end{définition}
\begin{exemple}\jya a-ftsa\cmn 我的外甥\end{exemple}\end{entrée}

\begin{entrée}
\vedette{\hypertarget{Ⓔtɤɣ}{\papi{ tɤɣ}}}\markboth{tɤɣ}{}\classe{num}
\begin{définition}\fra un\end{définition}
\begin{définition}\cmn 一\end{définition}
\begin{relation-sémantique}\synonyme{
\hyperlink{ⒺciⒽ1}{\textit{ \papi{ci}}}
}\end{relation-sémantique}
\end{entrée}

\begin{entrée}
\vedette{\hypertarget{Ⓔtɤɣa}{\papi{ tɤɣa}}}\markboth{tɤɣa}{} (\variante{tɤɣal}) 
\classe{adv}
\begin{définition}\fra visible\end{définition}
\begin{définition}\cmn 看得见的\end{définition}
\begin{exemple}\jya tɤɣal ɣɤʑu tɕe, kɤfsɯfse ɲɯ-sɤmto\cmn 完全看得见\end{exemple}
\begin{exemple}\jya tɤɣal mɯ́j-rɤʑi tɕe, mɯ́j-sɤmto\cmn 他不在看得见的地方,看不见他\end{exemple}\end{entrée}

\begin{entrée}
\vedette{\hypertarget{Ⓔtɤɣɤco}{\papi{ tɤɣɤco}}}\markboth{tɤɣɤco}{}\classe{n}
\begin{définition}\ 
\begin{déclaration}\grammar{n.lieu}\end{déclaration}\end{définition}
\begin{définition}\fra l'un des hameaux de Kamnyu\end{définition}
\begin{définition}\cmn 干木鸟的大队之一\end{définition}
\end{entrée}

\begin{entrée}
\vedette{\hypertarget{Ⓔtɤ-ɣe}{\papi{ tɤ-ɣe}}}\markboth{tɤ-ɣe}{}
\classe{np}
\begin{définition}\fra petits enfants\end{définition}
\begin{définition}\cmn 孙子\end{définition}\end{entrée}

\begin{entrée}
\vedette{\hypertarget{Ⓔtɤ-ɣiⒽ1}{\papi{ tɤ-ɣi}}}\markboth{tɤ-ɣi}{}\homonyme{1}\classe{np}
\begin{définition}\fra gens de la famille\end{définition}
\begin{définition}\cmn 家人\end{définition}
\begin{exemple}\jya a-ɣi\cmn 我的家人\end{exemple}\end{entrée}

\begin{entrée}
\vedette{\hypertarget{Ⓔtɤ-ɣiⒽ2}{\papi{ tɤ-ɣi}}}\markboth{tɤ-ɣi}{}\homonyme{2}
\classe{np}
\begin{définition}\fra glaise que l'on applique sur le toit\end{définition}
\begin{définition}\cmn 涂在房顶上的黄泥巴\end{définition}\end{entrée}

\begin{entrée}
\vedette{\hypertarget{ⒺtɤɣɟajⒽ1}{\papi{ tɤɣɟaj}}}\markboth{tɤɣɟaj}{}\homonyme{1}
\classe{n}
\begin{définition}\fra fait de forcer\end{définition}
\begin{définition}\cmn 撬开\end{définition}
\begin{exemple}\jya kɯm kɤ-cɯ mɯ́j-khɯ tɕe tɤɣɟaj tɤ-lɤt-i\cmn 门打不开我们就把它撬开了\end{exemple}
\begin{relation-sémantique}\confer{
\hyperlink{Ⓔnɤɣɟaj}{\textit{ \papi{nɤɣɟaj}}}
}\end{relation-sémantique}\end{entrée}

\begin{entrée}
\vedette{\hypertarget{Ⓔtɤ-ɣɟajⒽ2}{\papi{ tɤ-ɣɟaj}}}\markboth{tɤ-ɣɟaj}{}\homonyme{2}
\classe{np}
\begin{définition}\fra rame\end{définition}
\begin{définition}\cmn 桨\end{définition}
\begin{exemple}\jya ɯʑo kɯ tɤ-βɟaj ta-lɤt\cmn 他划船了\end{exemple}
\begin{relation-sémantique}\confer{
\hyperlink{Ⓔʑmbrɯβɟaj}{\textit{ \papi{ʑmbrɯβɟaj}}}
}\end{relation-sémantique}
\begin{relation-sémantique}\confer{
\hyperlink{Ⓔtɕhaŋβɟaj}{\textit{ \papi{tɕhaŋβɟaj}}}
}\end{relation-sémantique}
\begin{relation-sémantique}\confer{
 \papi{nɤβɟaj}
}\end{relation-sémantique}\end{entrée}

\begin{entrée}
\vedette{\hypertarget{Ⓔtɤɣle}{\papi{ tɤɣle}}}\markboth{tɤɣle}{}\classe{n}
\begin{définition}\fra bâton servant à maintenir à trame\end{définition}
\begin{définition}\cmn 拉住经线的木棒\end{définition}
\end{entrée}

\begin{entrée}
\vedette{\hypertarget{Ⓔtɤ-ɣmbaj}{\papi{ tɤ-ɣmbaj}}}\markboth{tɤ-ɣmbaj}{}\classe{np}
\begin{définition}\fra face d'une montagne\end{définition}
\begin{définition}\cmn 山的一面\end{définition}
\begin{exemple}\jya jɯɣi tɯ-tɤ-ɣmbaj pɯ-sthɯt-a\cmn 我写完了一页书\end{exemple}\end{entrée}

\begin{entrée}
\vedette{\hypertarget{Ⓔtɤ-ɣur}{\papi{ tɤ-ɣur}}}\markboth{tɤ-ɣur}{}
\classe{np}
\begin{définition}\fra haie\end{définition}
\begin{définition}\cmn 篱笆\end{définition}
\begin{exemple}\jya ɯ-ɣur tu-βzu-nɯ tɕe ku-omdzɯ-nɯ ɲɯ-ŋu\cmn 他们围着坐\end{exemple}\end{entrée}

\begin{entrée}
\vedette{\hypertarget{Ⓔtɤɣro}{\papi{ tɤɣro}}}\markboth{tɤɣro}{}\classe{n}
\begin{définition}\fra jeu\end{définition}
\begin{définition}\cmn 游戏\end{définition}
\begin{exemple}\jya tɤ-pɤtso nɯ tɤɣro kɤ-βzu rga\cmn 小孩子喜欢游戏\end{exemple}
\begin{exemple}\jya ɯ-tɤɣro ra to-βzu\cmn 他做了一下逗他玩的动作\end{exemple}\begin{sous-entrée}
\vedette{\hypertarget{}{\papi{ tɤɣro tɤle}}}\markboth{tɤɣro tɤle}{}
\begin{définition}\ 
\begin{déclaration}\grammar{emph}\end{déclaration}\end{définition}
\begin{exemple}\jya tɤɣro tɤle kɯ ku-rɤʑi\cmn 他一直都在玩\end{exemple}
\end{sous-entrée}\end{entrée}

\begin{entrée}
\vedette{\hypertarget{Ⓔtɤɣursi}{\papi{ tɤɣursi}}}\markboth{tɤɣursi}{}
\classe{n}
\begin{définition}\fra branches flexibles sur le balcon pour parer le vent\end{définition}
\begin{définition}\cmn 走缘边用来档风的树苗\end{définition}
\begin{exemple}\jya jɤɣɤt laχtsɯ ɯ-pɤrthɤβ qale sɤ-tshi tɤ-kɤ-βzu si ɯ-mnɯ thɯ-kɤ-mphɯr nɯ, nɯ maʁ nɤ thɯ-kɤ-ndzri nɯ; rorʁe ɯ-taʁ pɯ-kɤ-sɤqatʂha nɯ tɤ-ɣursi rmi\cmn 
走缘柱头之间,用来挡风的树苗裹成或者拧成的,穿插在穿杆上的叫\stylefv{tɤɣursi}
\end{exemple}\end{entrée}

\begin{entrée}
\vedette{\hypertarget{Ⓔtɤjkɤspa}{\papi{ tɤjkɤspa}}}\markboth{tɤjkɤspa}{}\classe{n}
\begin{définition}\fra navet (Brassica sp.)\end{définition}
\begin{définition}\cmn 芜菁【圆根】\end{définition}
\begin{relation-sémantique}\synonyme{
\hyperlink{Ⓔrasti}{\textit{ \papi{rasti}}}
}\end{relation-sémantique}
\begin{relation-sémantique}\confer{
\hyperlink{Ⓔtɤjko}{\textit{ \papi{tɤjko}}}
}\end{relation-sémantique}\end{entrée}

\begin{entrée}
\vedette{\hypertarget{Ⓔtɤjko}{\papi{ tɤjko}}}\markboth{tɤjko}{}
\classe{n}
\begin{définition}\fra feuilles de navet\end{définition}
\begin{définition}\cmn 芜菁叶子【酸菜】\end{définition}\end{entrée}

\begin{entrée}
\vedette{\hypertarget{Ⓔtɤjkopu}{\papi{ tɤjkopu}}}\markboth{tɤjkopu}{}\classe{n}
\begin{définition}\fra saucisse aux légumes\end{définition}
\begin{définition}\cmn 大肠\end{définition}
\begin{relation-sémantique}\confer{
\hyperlink{Ⓔtɯ-pu}{\textit{ \papi{tɯ-pu}}}
}\end{relation-sémantique}\end{entrée}

\begin{entrée}
\vedette{\hypertarget{Ⓔtɤjkɯz}{\papi{ tɤjkɯz}}}\markboth{tɤjkɯz}{}
\classe{n}
\begin{définition}\fra en secret\end{définition}
\begin{définition}\cmn 偷偷地\end{définition}
\begin{exemple}\jya tɤjkɯz tu-βze ɲɯ-ŋu\cmn 他瞒着别人做\end{exemple}
\begin{relation-sémantique}\confer{
\hyperlink{Ⓔnɤjkɯz}{\textit{ \papi{nɤjkɯz}}}
}\end{relation-sémantique}\end{entrée}

\begin{entrée}
\vedette{\hypertarget{Ⓔtɤjlu}{\papi{ tɤjlu}}}\markboth{tɤjlu}{}
\classe{n}
\begin{définition}\fra farine\end{définition}
\begin{définition}\cmn 面粉\end{définition}\end{entrée}

\begin{entrée}
\vedette{\hypertarget{Ⓔtɤjlɤβ}{\papi{ tɤjlɤβ}}}\markboth{tɤjlɤβ}{}
\classe{n}
\begin{définition}\fra vapeur\end{définition}
\begin{définition}\cmn 蒸汽\end{définition}
\begin{relation-sémantique}\confer{
\hyperlink{Ⓔsɤjlɤβ}{\textit{ \papi{sɤjlɤβ}}}
}\end{relation-sémantique}\end{entrée}

\begin{entrée}
\vedette{\hypertarget{Ⓔtɤjlɤpi}{\papi{ tɤjlɤpi}}}\markboth{tɤjlɤpi}{}\classe{n}
\begin{définition}\fra pâte\end{définition}
\begin{définition}\cmn 面团\end{définition}
\begin{relation-sémantique}\confer{
\hyperlink{Ⓔtɤjlu}{\textit{ \papi{tɤjlu}}}
}\end{relation-sémantique}\end{entrée}

\begin{entrée}
\vedette{\hypertarget{Ⓔtɤ-jli}{\papi{ tɤ-jli}}}\markboth{tɤ-jli}{}\classe{np}
\begin{définition}\fra valeur (d'une personne)\end{définition}
\begin{définition}\cmn 身价\end{définition}
\begin{exemple}\jya tɤ-pɤtso ɯ-jli ɲɯ-wxti tɕe kɤ-sɤβlo ɲɯ-ɴqa\cmn 小孩子很宝贝,不好伺候\end{exemple}\end{entrée}

\begin{entrée}
\vedette{\hypertarget{Ⓔtɤjmɤɣ}{\papi{ tɤjmɤɣ}}}\markboth{tɤjmɤɣ}{}
\classe{n}
\begin{définition}\fra champignon\end{définition}
\begin{définition}\cmn 蘑菇\end{définition}\end{entrée}

\begin{entrée}
\vedette{\hypertarget{Ⓔtɤjmɤɣrʑɯɣ}{\papi{ tɤjmɤɣrʑɯɣ}}}\markboth{tɤjmɤɣrʑɯɣ}{}\classe{n}
\begin{définition}\fra lamelles des champignons\end{définition}
\begin{définition}\cmn 菌褶\end{définition}
\end{entrée}

\begin{entrée}
\vedette{\hypertarget{Ⓔtɤ-jme}{\papi{ tɤ-jme}}}\markboth{tɤ-jme}{}
\classe{np}
\begin{définition}\fra queue\end{définition}
\begin{définition}\cmn 尾巴\end{définition}
\begin{exemple}\jya ɯ-jme\cmn 它的尾巴\end{exemple}
\begin{relation-sémantique}\confer{
\hyperlink{Ⓔjmɤrtaʁ}{\textit{ \papi{jmɤrtaʁ}}}
}\end{relation-sémantique}
\begin{relation-sémantique}\confer{
\hyperlink{Ⓔjmɤlu}{\textit{ \papi{jmɤlu}}}
}\end{relation-sémantique}
\begin{relation-sémantique}\confer{
\hyperlink{Ⓔɯ-kɤlɤjme}{\textit{ \papi{ɯ-kɤlɤjme}}}
}\end{relation-sémantique}\end{entrée}

\begin{entrée}
\vedette{\hypertarget{Ⓔtɤjmŋozdɯɣ}{\papi{ tɤjmŋozdɯɣ}}}\markboth{tɤjmŋozdɯɣ}{}\classe{n}
\begin{définition}\fra cauchemar\end{définition}
\begin{définition}\cmn 噩梦\end{définition}
\begin{exemple}\jya tɤjmŋozdɯɣ pɯ-tu\cmn 我做了噩梦\end{exemple}
\begin{relation-sémantique}\confer{
\hyperlink{Ⓔnɤjmŋozdɯɣ}{\textit{ \papi{nɤjmŋozdɯɣ}}}
}\end{relation-sémantique}
\begin{relation-sémantique}\confer{
\hyperlink{Ⓔtɯ-jmŋo}{\textit{ \papi{tɯ-jmŋo}}}
}\end{relation-sémantique}\end{entrée}

\begin{entrée}
\vedette{\hypertarget{Ⓔtɤ-jŋoʁ}{\papi{ tɤ-jŋoʁ}}}\markboth{tɤ-jŋoʁ}{}
\classe{np}
\begin{définition}\fra crochet\end{définition}
\begin{définition}\cmn 钩子\end{définition}\end{entrée}

\begin{entrée}
\vedette{\hypertarget{Ⓔtɤjpa}{\papi{ tɤjpa}}}\markboth{tɤjpa}{}
\classe{n}
\begin{définition}\fra neige\end{définition}
\begin{définition}\cmn 雪\end{définition}
\begin{exemple}\jya tɤjpa ko-lɤt\cmn 下雪了\end{exemple}
\begin{exemple}\jya qartsɯ ja-zɣɯt tɤjpa ku-lɤt ŋgrɤl ɕti wo\cmn 到了冬天就会下雪\end{exemple}
\begin{exemple}\jya kutɕu ko tɤjpa mɯ-ka-lɤt\cmn 这里倒没有下雪\end{exemple}
\begin{exemple}\jya tɤjpa pɤjkhu mbarkhom @jieshang mɯ-pa-sɤzɣɯt\cmn 雪还没有到马尔康街上\end{exemple}
\begin{relation-sémantique}\confer{
\hyperlink{Ⓔarɯtɤjpa}{\textit{ \papi{arɯtɤjpa}}}
}\end{relation-sémantique}\end{entrée}

\begin{entrée}
\vedette{\hypertarget{Ⓔtɤjpɤqe}{\papi{ tɤjpɤqe}}}\markboth{tɤjpɤqe}{}\classe{n}
\begin{définition}\fra espèce de corbeau\end{définition}
\begin{définition}\cmn 寒鸦\end{définition}
\begin{relation-sémantique}\confer{
\hyperlink{Ⓔqajdo}{\textit{ \papi{qajdo}}}
}\end{relation-sémantique}\end{entrée}

\begin{entrée}
\vedette{\hypertarget{Ⓔtɤjpɣom}{\papi{ tɤjpɣom}}}\markboth{tɤjpɣom}{}\classe{n}
\begin{définition}\fra glace\end{définition}
\begin{définition}\cmn 冰\end{définition}
\begin{exemple}\jya tɤjpɣom lɤ-k-ɤβzu-ci\cmn 结了冰\end{exemple}
\begin{exemple}\jya tɤjpɣom ko-ta\cmn 结了冰\end{exemple}
\begin{exemple}\jya tɯ-ci ɯ-taʁ tɤjpɣom kɤ-kɯ-ta\cmn 结了冰的地方\end{exemple}
\begin{relation-sémantique}\confer{
\hyperlink{Ⓔjpɣom}{\textit{ \papi{jpɣom}}}
}\end{relation-sémantique}\end{entrée}

\begin{entrée}
\vedette{\hypertarget{Ⓔtɤ-jroʁ}{\papi{ tɤ-jroʁ}}}\markboth{tɤ-jroʁ}{}
\classe{np}
\begin{définition}\fra trace\end{définition}
\begin{définition}\cmn 痕迹\end{définition}
\begin{exemple}\jya a-jroʁ\cmn 我的痕迹\end{exemple}
\begin{exemple}\jya ɯ-jroʁ jo-thɯ (jo-tɕɤt)\cmn 他留下了痕迹\end{exemple}\end{entrée}

\begin{entrée}
\vedette{\hypertarget{Ⓔtɤjʁa}{\papi{ tɤjʁa}}}\markboth{tɤjʁa}{}
\classe{n}
\begin{définition}\fra col\end{définition}
\begin{définition}\cmn 垭口\end{définition}\end{entrée}

\begin{entrée}
\vedette{\hypertarget{Ⓔtɤjsaʁ}{\papi{ tɤjsaʁ}}}\markboth{tɤjsaʁ}{}
\classe{n}
\begin{définition}\fra débris\end{définition}
\begin{définition}\cmn 赃物;落叶\end{définition}
\begin{exemple}\jya sɤtɕha pjɤ-mbɯt tɕe, ndzom ɯ-pa tɤjsaʁ cho-ɣi\cmn 地塌下来了,漂浮物流到桥下\end{exemple}
\begin{relation-sémantique}\synonyme{
 \papi{tɤɲɟoʁɲɟi}
}\end{relation-sémantique}\end{entrée}

\begin{entrée}
\vedette{\hypertarget{Ⓔtɤ-jtsi}{\papi{ tɤ-jtsi}}}\markboth{tɤ-jtsi}{}
\classe{np}
\begin{définition}\fra pilier\end{définition}
\begin{définition}\cmn 柱子\end{définition}\end{entrée}

\begin{entrée}
\vedette{\hypertarget{Ⓔtɤ-jwaʁ}{\papi{ tɤ-jwaʁ}}}\markboth{tɤ-jwaʁ}{}\classe{np}
\paradigme{\textit{comit :} \jya kɤ́jwɯjwaʁ}
\begin{définition}\fra feuille\end{définition}
\begin{définition}\cmn 叶子
\end{définition}
\begin{exemple}\jya kɤ́jwɯjwaʁ ʑo nɯ-phɯt-a\cmn 我连着叶子一起砍掉了\end{exemple}
\begin{exemple}\jya sɯjwaʁ\cmn 树叶\end{exemple}\end{entrée}

\begin{entrée}
\vedette{\hypertarget{Ⓔtɤkɤɣrum}{\papi{ tɤkɤɣrum}}}\markboth{tɤkɤɣrum}{}
\classe{n}
\begin{définition}\fra à la chevelure blanche\end{définition}
\begin{définition}\cmn 白发\end{définition}
\begin{exemple}\jya rgɤtpu tɤkɤɣrum\cmn 白发老头\end{exemple}
\begin{relation-sémantique}\confer{
\hyperlink{Ⓔtɯ-ku}{\textit{ \papi{tɯ-ku}}}
}\end{relation-sémantique}
\begin{relation-sémantique}\confer{
\hyperlink{Ⓔwɣrum}{\textit{ \papi{wɣrum}}}
}\end{relation-sémantique}\end{entrée}

\begin{entrée}
\vedette{\hypertarget{Ⓔtɤ-kɤrme}{\papi{ tɤ-kɤrme}}}\markboth{tɤ-kɤrme}{}\classe{np}
\begin{définition}\fra cheveux\end{définition}
\begin{définition}\cmn 头发\end{définition}
\begin{relation-sémantique}\confer{
\hyperlink{Ⓔtɯ-ku}{\textit{ \papi{tɯ-ku}}}
}\end{relation-sémantique}
\begin{relation-sémantique}\confer{
\hyperlink{Ⓔtɤ-rme}{\textit{ \papi{tɤ-rme}}}
}\end{relation-sémantique}\end{entrée}

\begin{entrée}
\vedette{\hypertarget{Ⓔtɤ-kɤrtshi}{\papi{ tɤ-kɤrtshi}}}\markboth{tɤ-kɤrtshi}{}
\classe{np}
\begin{définition}\fra épaisseur des cheveux\end{définition}
\begin{définition}\cmn 头发的密度\end{définition}
\begin{exemple}\jya a-kɤrtshi ɲɯ-mba\cmn 我的头发很稀疏\end{exemple}
\begin{exemple}\jya a-kɤrtshi ɲɯ-jaʁ\cmn 我的头发很密\end{exemple}
\begin{relation-sémantique}\confer{
\hyperlink{Ⓔtɯ-ku}{\textit{ \papi{tɯ-ku}}}
}\end{relation-sémantique}\end{entrée}

\begin{entrée}
\vedette{\hypertarget{Ⓔtɤkɤzbɣaʁ}{\papi{ tɤkɤzbɣaʁ}}}\markboth{tɤkɤzbɣaʁ}{}
\classe{n}
\begin{définition}\fra migraine\end{définition}
\begin{définition}\cmn 头风病\end{définition}
\begin{exemple}\jya tɤkɤzbɣaʁ nɯ jɯsŋi soz tɕe a-tɤ-ʑe tɕe, fsosoz tɕe ɯ-tɯ-mŋɤm ɲɯ-ʑi ŋu\cmn 头风病,如果今天早上得了这个病,要到明天早上才能好起来。\end{exemple}\end{entrée}

\begin{entrée}
\vedette{\hypertarget{Ⓔtɤ-kɤʑmbri}{\papi{ tɤ-kɤʑmbri}}}\markboth{tɤ-kɤʑmbri}{}
\classe{np}
\begin{définition}\fra insolation\end{définition}
\begin{définition}\cmn 中暑\end{définition}
\begin{exemple}\jya tɤ-kɤʑmbri nɯ, nɯ-ɣɯtshɤdɯɣ nɯ-tɕhom tɕe tɯ-kɯr ɯ-ŋgɯ tɯ-mdʑu ɯ-taʁ tɤ-ndɤr ɲɯ-ɬoʁ ŋu tɕe wuma ʑo sɤɣdɯɣ. tɤ-pɤtso nɯ-kɤʑmbri kɯ-ɣi ŋgrɤl\cmn 气候太热容易导致中暑。嘴里,舌头上出很多痘痘,很难受。小孩子中暑得比较多。\end{exemple}
\begin{exemple}\jya tɤŋe wuma ʑo ɲɯ-sɤɕke tɕe a-kɤʑmbri pjɯ-sɯɣe ɲɯ-ŋu\cmn 太阳很晒,令我中暑了\end{exemple}\end{entrée}

\begin{entrée}
\vedette{\hypertarget{Ⓔtɤkha}{\papi{ tɤkha}}}\markboth{tɤkha}{}\classe{cnj}
\begin{définition}\fra au moment de\end{définition}
\begin{définition}\cmn 临...之前\end{définition}\end{entrée}

\begin{entrée}
\vedette{\hypertarget{Ⓔtɤkhe}{\papi{ tɤkhe}}}\markboth{tɤkhe}{}
\classe{n}
\begin{définition}\fra imbécile\end{définition}
\begin{définition}\cmn 傻瓜\end{définition}
\begin{relation-sémantique}\antonyme{
\hyperlink{Ⓔtɤɕqraʁ}{\textit{ \papi{tɤɕqraʁ}}}
}\end{relation-sémantique}\end{entrée}

\begin{entrée}
\vedette{\hypertarget{Ⓔtɤkhepɣɤtɕɯ}{\papi{ tɤkhepɣɤtɕɯ}}}\markboth{tɤkhepɣɤtɕɯ}{}\classe{n}
\begin{définition}\fra Emberiza sp.\end{définition}
\begin{définition}\cmn 鹀\end{définition}
\begin{exemple}\jya tɤkhepɣɤtɕɯ nɯ pɣa kɯ-xtɕi ci ŋu, phu nɯ pɣi ri ɯ-mke cho ɯ-rqo pa nɯ ra kɯ-ɣɯrni ŋu, ɯ-mi kɯ-qarŋe ŋu, ɯ-jme ɯ-ku nɯ ra kɯ-wɣrumtɕe mpɕɤr, mu nɯ mɤ-mpɕɤr, pha ɯ-phoŋbu ʑo pɣi, ɯ-taʁ kɯ-ɲaʁ kɯ-khra tu, tɤ-khe pɣɤtɕɯ nɯ kɯ qajɯ cho tɤ-rɤku tu-ndze ŋu, tɕe ɯ-kɯ-qha dɤn, tɕe pɣɤtɕɯ nɯ ɯ-ŋgɯz khe tu-kɯ-ti ɲɯ-ŋu. kha ɯ-rkarkɯ ra kɤ-rɤʑi χɕu. ɯ-loʁ nɯ tʂu ɯ-rkɯ kɯ-ɤrmbat ʑo sɯphɯ kɯ-xtɕi ɯ-qa, xɕaj ɯ-qa ra ku-βze ŋu tɕe, nɯɣɯmto. tɤkhepɣɤtɕɯ ɲɯ-khe tɕe βʑar kɯ ɯʑo ɯ-loʁ ɯ-ŋgɯ ɯ-ŋgɯm nɯ ra tu-ndze tɕe, ɯ-sta nɯ tɕu βʑar kɯ ɯʑo ɯ-ŋgɯm ko-lɤt tɕe, ɯ-ŋgɯm nɯ nɯ-ʁaʁ tɕe chɯ-wxti ɲɯ-ŋu tɕe, tɕe tɤkhepɣɤtɕɯ kɯ sɤ-mɯ-mu ʑo ku-χse ɲɯ-ra.\cmn 
\stylefv{tɤkhe pɣɤtɕɯ}是一种小鸟,公的虽然是灰的,颈和脖子下面带有红色,脚是黄色的,尾巴顶端有白色,很漂亮。母的不漂亮,全身是灰色的,上面还带有黑色的斑点。\stylefv{tɤkhe pɣɤtɕɯ}吃虫子和粮食,很多人讨厌它。据说它在鸟类当中是比较笨的一只。它喜欢在房子周边活动,把窝打在离路边不远的小树和草丛底下,容易发现。因为\stylefv{tɤkhe pɣɤtɕɯ}笨,所以鹞子就会在它的窝里吃掉它的蛋,然后把自己的蛋下在里面。蛋孵出了以后,小鹞子会长大,它也只好带着害怕的心理喂养它们。
\end{exemple}\end{entrée}

\begin{entrée}
\vedette{\hypertarget{Ⓔtɤkhespɤlbu}{\papi{ tɤkhespɤlbu}}}\markboth{tɤkhespɤlbu}{}\classe{n}
\begin{définition}\fra bêta\end{définition}
\begin{définition}\cmn 傻乎乎\end{définition}\end{entrée}

\begin{entrée}
\vedette{\hypertarget{Ⓔtɤ-khrɤl}{\papi{ tɤ-khrɤl}}}\markboth{tɤ-khrɤl}{}
\classe{np}
\begin{définition}\fra prix à payer\end{définition}
\begin{définition}\cmn 自己应该承担的责任;应该付出的代价
\begin{déclaration} \étymologie{\papi{kʰral}}\end{déclaration}\end{définition}
\begin{exemple}\jya ki a-khrɤl ŋu\cmn 这是我的责任\end{exemple}
\begin{exemple}\jya a-khrɤl nɯ-n-nɤma-t-a\cmn 我做了我的那一份(工作)\end{exemple}\end{entrée}

\begin{entrée}
\vedette{\hypertarget{Ⓔtɤ-khɯ}{\papi{ tɤ-khɯ}}}\markboth{tɤ-khɯ}{}
\classe{np}
\begin{définition}\fra fumée\end{définition}
\begin{définition}\cmn 烟\end{définition}
\begin{exemple}\jya tɤ-khɯ ta-tɕɤt\cmn 他求了烟\end{exemple}
\begin{relation-sémantique}\confer{
\hyperlink{Ⓔnɤkhɯ}{\textit{ \papi{nɤkhɯ}}}
}\end{relation-sémantique}
\begin{relation-sémantique}\confer{
\hyperlink{ⒺɣɤkhɯⒽ1}{\textit{ \papi{ɣɤkhɯ1}}}
}\end{relation-sémantique}
\begin{relation-sémantique}\confer{
\hyperlink{Ⓔsɤkhɯ}{\textit{ \papi{sɤkhɯ}}}
}\end{relation-sémantique}
\begin{relation-sémantique}\confer{
\hyperlink{Ⓔkhɯɣɲɟɯ}{\textit{ \papi{khɯɣɲɟɯ}}}
}\end{relation-sémantique}\end{entrée}

\begin{entrée}
\vedette{\hypertarget{Ⓔtɤkhɯɣɲɟɯ}{\papi{ tɤkhɯɣɲɟɯ}}}\markboth{tɤkhɯɣɲɟɯ}{}\classe{n}
\begin{définition}\fra cheminée\end{définition}
\begin{définition}\cmn 烟囱\end{définition}
\begin{relation-sémantique}\confer{
\hyperlink{Ⓔɯ-ɣɲɟɯ}{\textit{ \papi{ɯ-ɣɲɟɯ}}}
}\end{relation-sémantique}\end{entrée}

\begin{entrée}
\vedette{\hypertarget{Ⓔtɤkhɯrɲɯl}{\papi{ tɤkhɯrɲɯl}}}\markboth{tɤkhɯrɲɯl}{}\classe{n}
\begin{définition}\fra fumée qui ne se dissipe pas\end{définition}
\begin{définition}\cmn 驱散不了;在山腰停留的烟子(人为的烟子、炊烟)\end{définition}
\begin{exemple}\jya tɤkhɯrɲɯl ɣɤʑu\cmn 有驱散不了的烟子\end{exemple}\end{entrée}

\begin{entrée}
\vedette{\hypertarget{Ⓔtɤkusci}{\papi{ tɤkusci}}}\markboth{tɤkusci}{}\classe{n}
\begin{définition}\fra il était une fois\end{définition}
\begin{définition}\cmn 故事的开头语\end{définition}
\end{entrée}

\begin{entrée}
\vedette{\hypertarget{Ⓔtɤ-lu}{\papi{ tɤ-lu}}}\markboth{tɤ-lu}{}\classe{np}
\paradigme{\textit{comit :} \jya kɤ́lɯlu}
\begin{définition}\fra lait\end{définition}
\begin{définition}\cmn 奶汁\end{définition}
\begin{exemple}\jya tɤ-lu pɯ-tɕɤt\cmn 你挤奶吧\end{exemple}
\begin{relation-sémantique}\confer{
\hyperlink{Ⓔnɤlu}{\textit{ \papi{nɤlu}}}
}\end{relation-sémantique}\end{entrée}

\begin{entrée}
\vedette{\hypertarget{Ⓔtɤlɤβʑɤzu}{\papi{ tɤlɤβʑɤzu}}}\markboth{tɤlɤβʑɤzu}{}
\classe{n}
\begin{définition}\fra trompettes de la mort\end{définition}
\begin{définition}\cmn 灰喇叭菌\end{définition}
\begin{exemple}\jya tɤlɤβʑɤzu nɯ tɯrgi ɯ-ŋgɯ tu-ɬoʁ ŋu, ɯ-mdoʁ nɯ qromke mdoʁ ŋu, ɯ-tshɯɣa nɯ @laba ɯ-taʁ tɤ-kɤ-ɕthɯz kɯ-fse ŋu, kɤ-ndza sna, ɯ-rʑɯɣ me, ndoʁ\cmn 灰喇叭菌长在杉木林里,紫色,形状像朝天的喇叭,可以吃,没有菌褶,脆。\end{exemple}\end{entrée}

\begin{entrée}
\vedette{\hypertarget{Ⓔtɤlɤɕom}{\papi{ tɤlɤɕom}}}\markboth{tɤlɤɕom}{}
\classe{n}
\begin{définition}\fra peau du lait\end{définition}
\begin{définition}\cmn 奶皮\end{définition}
\begin{relation-sémantique}\confer{
\hyperlink{ⒺɕomⒽ2}{\textit{ \papi{ɕom2}}}
}\end{relation-sémantique}
\end{entrée}

\begin{entrée}
\vedette{\hypertarget{Ⓔtɤlɤndʑu}{\papi{ tɤlɤndʑu}}}\markboth{tɤlɤndʑu}{}
\classe{n}
\begin{définition}\fra bâton à baratter\end{définition}
\begin{définition}\cmn 搅奶的棍子\end{définition}
\begin{relation-sémantique}\confer{
\hyperlink{Ⓔtɤ-lu}{\textit{ \papi{tɤ-lu}}}
}\end{relation-sémantique}
\begin{relation-sémantique}\confer{
\hyperlink{Ⓔndʑu}{\textit{ \papi{ndʑu}}}
}\end{relation-sémantique}\end{entrée}

\begin{entrée}
\vedette{\hypertarget{Ⓔtɤlɤɴqhi}{\papi{ tɤlɤɴqhi}}}\markboth{tɤlɤɴqhi}{}\classe{n}
\begin{définition}\fra lait séché (sur les casseroles)\end{définition}
\begin{définition}\cmn 干了的奶渍(锅子上)\end{définition}
\begin{relation-sémantique}\confer{
\hyperlink{Ⓔɴqhi}{\textit{ \papi{ɴqhi}}}
}\end{relation-sémantique}
\begin{relation-sémantique}\confer{
\hyperlink{Ⓔtɤ-lu}{\textit{ \papi{tɤ-lu}}}
}\end{relation-sémantique}
\end{entrée}

\begin{entrée}
\vedette{\hypertarget{Ⓔtɤlɤrpjɯ}{\papi{ tɤlɤrpjɯ}}}\markboth{tɤlɤrpjɯ}{}
\classe{n}
\begin{définition}\fra lait caillé\end{définition}
\begin{définition}\cmn 变质了的牛奶\end{définition}
\begin{relation-sémantique}\confer{
\hyperlink{Ⓔtɤ-lu}{\textit{ \papi{tɤ-lu}}}
}\end{relation-sémantique}
\begin{relation-sémantique}\confer{
\hyperlink{Ⓔrpjɯ}{\textit{ \papi{rpjɯ}}}
}\end{relation-sémantique}\end{entrée}

\begin{entrée}
\vedette{\hypertarget{Ⓔtɤlɤtshaʁ}{\papi{ tɤlɤtshaʁ}}}\markboth{tɤlɤtshaʁ}{}
\classe{n}
\begin{définition}\fra filtre à lait\end{définition}
\begin{définition}\cmn 用来过滤牛奶的瓢\end{définition}
\begin{relation-sémantique}\confer{
\hyperlink{Ⓔtɤ-lu}{\textit{ \papi{tɤ-lu}}}
}\end{relation-sémantique}
\begin{relation-sémantique}\confer{
\hyperlink{Ⓔtshaʁ}{\textit{ \papi{tshaʁ}}}
}\end{relation-sémantique}\end{entrée}

\begin{entrée}
\vedette{\hypertarget{Ⓔtɤlɤxchi}{\papi{ tɤlɤxchi}}}\markboth{tɤlɤxchi}{}
\classe{n}
\begin{définition}\fra lait frais\end{définition}
\begin{définition}\cmn 新鲜牛奶\end{définition}
\begin{relation-sémantique}\confer{
\hyperlink{Ⓔtɤ-lu}{\textit{ \papi{tɤ-lu}}}
}\end{relation-sémantique}
\begin{relation-sémantique}\confer{
\hyperlink{Ⓔchi}{\textit{ \papi{chi}}}
}\end{relation-sémantique}\end{entrée}

\begin{entrée}
\vedette{\hypertarget{Ⓔtɤlɤxtɕur}{\papi{ tɤlɤxtɕur}}}\markboth{tɤlɤxtɕur}{}
\classe{n}
\begin{définition}\fra lait aigre\end{définition}
\begin{définition}\cmn 酸奶,把酥油打出来以后剩下的奶水\end{définition}
\begin{relation-sémantique}\confer{
\hyperlink{Ⓔtɤ-lu}{\textit{ \papi{tɤ-lu}}}
}\end{relation-sémantique}
\begin{relation-sémantique}\confer{
\hyperlink{ⒺtɕurⒽ1}{\textit{ \papi{tɕur1}}}
}\end{relation-sémantique}\end{entrée}

\begin{entrée}
\vedette{\hypertarget{Ⓔtɤlɟɣo,lɤt}{\papi{ tɤlɟɣo,lɤt}}}\markboth{tɤlɟɣo,lɤt}{}
\paradigme{\textit{dir :} \jya kɤ-}
\begin{définition}\fra attraper au collet (cheval, bovidé)\end{définition}
\begin{définition}\cmn 套(动物)\end{définition}
\begin{relation-sémantique}\ComponentA{\classe{n}
 \papi{tɤlɟɣo}
}\end{relation-sémantique}
\begin{relation-sémantique}\ComponentB{\classe{vt}
\hyperlink{ⒺlɤtⒽ1}{\textit{ \papi{lɤt}}}
}\end{relation-sémantique}
\begin{relation-sémantique}\confer{
\hyperlink{ⒺlɤtⒽ1}{\textit{ \papi{lɤt1}}}
}\end{relation-sémantique}\end{entrée}

\begin{entrée}
\vedette{\hypertarget{Ⓔtɤlmɯz}{\papi{ tɤlmɯz}}}\markboth{tɤlmɯz}{}
\classe{n}
\begin{définition}\fra branches ou paille dont on recouvre les balcons\end{définition}
\begin{définition}\cmn 铺在走缘地上的泥土下面的麦草;豌豆;枝桠,用来防止泥土漏掉\end{définition}\end{entrée}

\begin{entrée}
\vedette{\hypertarget{Ⓔtɤ-loʁⒽ1}{\papi{ tɤ-loʁ}}}\markboth{tɤ-loʁ}{}\homonyme{1}\classe{np}\acception{1}
\begin{définition}\fra terrier, nid\end{définition}
\begin{définition}\cmn 鸟窝,野兽的洞\end{définition}\acception{2}
\begin{définition}\fra berceau\end{définition}
\begin{définition}\cmn 摇篮\end{définition}\end{entrée}

\begin{entrée}
\vedette{\hypertarget{Ⓔtɤ-loʁⒽ2}{\papi{ tɤ-loʁ}}}\markboth{tɤ-loʁ}{}\homonyme{2}
\classe{np}
\begin{définition}\fra anneau\end{définition}
\begin{définition}\cmn 圆圈\end{définition}\end{entrée}

\begin{entrée}
\vedette{\hypertarget{Ⓔtɤlɯlɤt}{\papi{ tɤlɯlɤt}}}\markboth{tɤlɯlɤt}{}\classe{n}
\begin{définition}\fra guerre\end{définition}
\begin{définition}\cmn 战争\end{définition}
\begin{exemple}\jya tɤlɯlɤt to-rɤru tɕe tɯŋgo jo-ɣi\cmn 发生了战乱和瘟疫\end{exemple}
\begin{exemple}\jya tɤlɯlɤt to-βzu-nɯ\cmn 他们打仗了\end{exemple}
\begin{relation-sémantique}\confer{
\hyperlink{Ⓔalɯlɤt}{\textit{ \papi{alɯlɤt}}}
}\end{relation-sémantique}\end{entrée}

\begin{entrée}
\vedette{\hypertarget{Ⓔtɤ-ɬaʁ}{\papi{ tɤ-ɬaʁ}}}\markboth{tɤ-ɬaʁ}{}
\classe{np}
\begin{définition}\fra tante (sœur de la mère, femme du frère du père, femme du frère de la mère, femme du frère)\end{définition}
\begin{définition}\cmn 姨母;伯母;婶母;舅母;嫂子\end{définition}
\begin{exemple}\jya a-ɬaʁ\cmn 我的姨母\end{exemple}\end{entrée}

\begin{entrée}
\vedette{\hypertarget{Ⓔtɤ-mu}{\papi{ tɤ-mu}}}\markboth{tɤ-mu}{}
\classe{np}
\begin{définition}\fra mère\end{définition}
\begin{définition}\cmn 母亲\end{définition}
\begin{exemple}\jya a-mu a-wa\cmn 我的父母\end{exemple}\end{entrée}

\begin{entrée}
\vedette{\hypertarget{Ⓔtɤ-ma}{\papi{ tɤ-ma}}}\markboth{tɤ-ma}{}\classe{np}
\begin{définition}\fra mère (noble)\end{définition}
\begin{définition}\cmn 母亲(贵族用语)\end{définition}
\begin{exemple}\jya a-pa a-ma\cmn 我父母\end{exemple}\end{entrée}

\begin{entrée}
\vedette{\hypertarget{Ⓔtɤ-mɤtɕɤz}{\papi{ tɤ-mɤtɕɤz}}}\markboth{tɤ-mɤtɕɤz}{}
\classe{np}
\begin{définition}\fra trace de pied\end{définition}
\begin{définition}\cmn 脚印\end{définition}
\begin{exemple}\jya a-mɤtɕɤz\cmn 我的脚印\end{exemple}
\begin{relation-sémantique}\confer{
\hyperlink{Ⓔtɤ-tɕɤz}{\textit{ \papi{tɤ-tɕɤz}}}
}\end{relation-sémantique}\end{entrée}

\begin{entrée}
\vedette{\hypertarget{Ⓔtɤ-mɤtsa}{\papi{ tɤ-mɤtsa}}}\markboth{tɤ-mɤtsa}{}
\classe{np}
\begin{définition}\fra cousin\end{définition}
\begin{définition}\cmn 堂兄弟姐妹\end{définition}\end{entrée}

\begin{entrée}
\vedette{\hypertarget{Ⓔtɤ-mbe}{\papi{ tɤ-mbe}}}\markboth{tɤ-mbe}{}
\classe{np}\acception{1}
\begin{définition}\fra vieux et abîmé\end{définition}
\begin{définition}\cmn 破旧的\end{définition}\acception{2}
\begin{définition}\fra habit rapiécé\end{définition}
\begin{définition}\cmn 烂衣服(背缝缝补补很多次的破旧衣服)\end{définition}
\begin{relation-sémantique}\confer{
\hyperlink{Ⓔmbe}{\textit{ \papi{mbe}}}
}\end{relation-sémantique}
\begin{relation-sémantique}\confer{
 \papi{ɯ-mbambe}
}\end{relation-sémantique}\end{entrée}

\begin{entrée}
\vedette{\hypertarget{Ⓔtɤmbextsa}{\papi{ tɤmbextsa}}}\markboth{tɤmbextsa}{}\classe{n}
\begin{définition}\fra botte faite de lin, de laine et de cuir\end{définition}
\begin{définition}\cmn 用麻布、羊毛和皮子作成的靴子\end{définition}
\end{entrée}

\begin{entrée}
\vedette{\hypertarget{Ⓔtɤmbɣo}{\papi{ tɤmbɣo}}}\markboth{tɤmbɣo}{}\classe{n}
\begin{définition}\fra sourd\end{définition}
\begin{définition}\cmn 聋子\end{définition}
\begin{relation-sémantique}\confer{
\hyperlink{Ⓔɣɤmbɣo}{\textit{ \papi{ɣɤmbɣo}}}
}\end{relation-sémantique}
\end{entrée}

\begin{entrée}
\vedette{\hypertarget{Ⓔtɤ-mbrɯ}{\papi{ tɤ-mbrɯ}}}\markboth{tɤ-mbrɯ}{}
\classe{np}
\paradigme{\textit{emphatic :} \jya tɤmbrɯ tɤʁɟa}
\begin{définition}\fra colère\end{définition}
\begin{définition}\cmn 生气(状态)\end{définition}
\begin{exemple}\jya tɤmbrɯ tɤʁɟa kɯ ku-rɤʑi\cmn 他非常生气\end{exemple}
\begin{relation-sémantique}\confer{
\hyperlink{Ⓔtɤ-mbrɯ,ŋgɯ}{\textit{ \papi{tɤ-mbrɯ,ŋgɯ}}}
}\end{relation-sémantique}\end{entrée}

\begin{entrée}
\vedette{\hypertarget{Ⓔtɤmbrɯm}{\papi{ tɤmbrɯm}}}\markboth{tɤmbrɯm}{}
\classe{n}
\begin{définition}\fra rougeole\end{définition}
\begin{définition}\cmn 疹子
\begin{déclaration} \étymologie{\papi{ⁿbrum}}\end{déclaration}\end{définition}\end{entrée}

\begin{entrée}
\vedette{\hypertarget{Ⓔtɤ-mbrɯ,ŋgɯ}{\papi{ tɤ-mbrɯ,ŋgɯ}}}\markboth{tɤ-mbrɯ,ŋgɯ}{}
\paradigme{\textit{dir :} \jya tɤ-}
\begin{définition}\fra s'énerver\end{définition}
\begin{définition}\cmn 生气\end{définition}
\begin{exemple}\jya a-taʁ ɯ-mbrɯ ɲɯ-ŋgɯ\cmn 他生我的气\end{exemple}
\begin{relation-sémantique}\ComponentA{\classe{np}
\hyperlink{Ⓔtɤ-mbrɯ}{\textit{ \papi{tɤ-mbrɯ}}}
}\end{relation-sémantique}
\begin{relation-sémantique}\ComponentB{\classe{vi}
 \papi{\_ŋgɯ}
}\end{relation-sémantique}\begin{sous-entrée}
\vedette{\hypertarget{}{\papi{ tɤ-mbrɯ,ɕɯŋgɯ}}}\markboth{tɤ-mbrɯ,ɕɯŋgɯ}{}\classe{vt}
\paradigme{\textit{dir :} \jya tɤ-}
\begin{définition}\ 
\begin{déclaration}\grammar{caus}\end{déclaration}\end{définition}
\begin{définition}\fra énerver\end{définition}
\begin{définition}\cmn 惹人生气\end{définition}
\begin{exemple}\jya a-mbrɯ ta-ɕɯŋgɯ\cmn 他惹我生气了\end{exemple}
\begin{relation-sémantique}\ComponentA{\classe{np}
\hyperlink{Ⓔtɤ-mbrɯ}{\textit{ \papi{tɤ-mbrɯ}}}
}\end{relation-sémantique}
\begin{relation-sémantique}\ComponentB{\classe{vt}
 \papi{\_ɕɯŋgɯ}
}\end{relation-sémantique}
\begin{relation-sémantique}\confer{
\hyperlink{Ⓔsɤmbrɯ}{\textit{ \papi{sɤmbrɯ}}}
}\end{relation-sémantique}
\begin{relation-sémantique}\confer{
\hyperlink{Ⓔsɤmbrɯŋgɯ}{\textit{ \papi{sɤmbrɯŋgɯ}}}
}\end{relation-sémantique}
\end{sous-entrée}\end{entrée}

\begin{entrée}
\vedette{\hypertarget{Ⓔtɤmcar}{\papi{ tɤmcar}}}\markboth{tɤmcar}{}
\classe{n}
\begin{définition}\fra pinces\end{définition}
\begin{définition}\cmn 火钳\end{définition}
\begin{relation-sémantique}\confer{
\hyperlink{Ⓔɯ-tɤmcar}{\textit{ \papi{ɯ-tɤmcar}}}
}\end{relation-sémantique}\end{entrée}

\begin{entrée}
\vedette{\hypertarget{Ⓔtɤ-mdɯ}{\papi{ tɤ-mdɯ}}}\markboth{tɤ-mdɯ}{}
\classe{np}
\begin{définition}\fra neveux (enfants du frère)\end{définition}
\begin{définition}\cmn 侄子\end{définition}
\begin{exemple}\jya a-mdɯ\cmn 我的侄子\end{exemple}\end{entrée}

\begin{entrée}
\vedette{\hypertarget{Ⓔtɤ-mdzu}{\papi{ tɤ-mdzu}}}\markboth{tɤ-mdzu}{}
\classe{np}
\begin{définition}\fra épine\end{définition}
\begin{définition}\cmn 刺\end{définition}\end{entrée}

\begin{entrée}
\vedette{\hypertarget{Ⓔtɤmdzɤqaqa}{\papi{ tɤmdzɤqaqa}}}\markboth{tɤmdzɤqaqa}{}
\classe{n}
\begin{définition}\fra 
pousses du \stylefv{ɴɢolo}
\end{définition}
\begin{définition}\cmn 
\stylefv{ɴɢolo}的新苗
\end{définition}\end{entrée}

\begin{entrée}
\vedette{\hypertarget{Ⓔtɤmdzɤrgi}{\papi{ tɤmdzɤrgi}}}\markboth{tɤmdzɤrgi}{}
\classe{n}
\begin{définition}\fra chardon\end{définition}
\begin{définition}\cmn 大蓟\end{définition}
\begin{exemple}\jya tɤmdzɤrgi nɯ sɯjno ci ŋu, ɯ-zrɤm wɣrum khro mɤ-wxti, nɯɕɯmɯma ʑo kɤ-phɯt khɯ, ɯ-jwaʁ nɯ ɯ-thoʁ pjɯ-tɯɣ ʑo ŋu, ɯ-jwaʁ ɯ-mdzu wuma ʑo dɤn, ɯ-jwaʁ ɯ-βzɯr ɣɯ ɯ-mdzu nɯ mɤʑɯ ʑo mtɕoʁ cho rɲɟi. ɯ-χcɤl ɯ-ru tu-ɬoʁ tɕe, ɯ-βzɯr tu. ɯ-ru ɯ-taʁ nɯ ra kɯnɤ ɯ-mdzu tu. ɯ-ru tɤ-zri tsa tɕe, li ɯ-jwaʁ ɲɯ-ɬoʁ tɕe, ɯ-mɯntoʁ ɲɯ-lɤt ŋu. ɯ-mɯntoʁ ʁmɤrsmɯɣ tsa ŋu, ɯ-mɯntoʁ pɯ-ŋgra tɕe, ɯ-rɣi ɲɯ-βze tɕe, ɯ-rɣi ɯ-ku zɯ li ɯ-rme kɯ-fse tu, qale kɯ ju-nɯtsɯm cha. tɤmdzɤrgi nɯ ɲɯ́-wɣ-phɯt tɕe, ɯ-di ci tu.\cmn 大蓟是一种植物,根是白色的,长得不大,一下子就可以拔掉,叶子贴在地面上,叶子上长满刺,叶子边的刺(比叶面的刺)长和尖。中间长茎,有棱角。茎上也长有刺。茎长高后,又长叶子,开花。花是紫色的,花凋谢后,就结种子,种子上也长有细毛状的东西,可以被风吹走。扯大蓟时会发出臭味。\end{exemple}\end{entrée}

\begin{entrée}
\vedette{\hypertarget{Ⓔtɤ-muj}{\papi{ tɤ-muj}}}\markboth{tɤ-muj}{}
\classe{np}
\begin{définition}\fra plumes\end{définition}
\begin{définition}\cmn 羽毛\end{définition}
\begin{relation-sémantique}\confer{
\hyperlink{Ⓔpɣɤmuj}{\textit{ \papi{pɣɤmuj}}}
}\end{relation-sémantique}\end{entrée}

\begin{entrée}
\vedette{\hypertarget{Ⓔtɤ-mkɯm}{\papi{ tɤ-mkɯm}}}\markboth{tɤ-mkɯm}{}\classe{np}
\begin{définition}\fra oreiller\end{définition}
\begin{définition}\cmn 枕头\end{définition}
\begin{exemple}\jya aʑo a-mkɯm\cmn 我的枕头\end{exemple}\end{entrée}

\begin{entrée}
\vedette{\hypertarget{Ⓔtɤmphoʁ}{\papi{ tɤmphoʁ}}}\markboth{tɤmphoʁ}{}
\classe{n}
\begin{définition}\fra tsampa\end{définition}
\begin{définition}\cmn 糌粑的一种吃法\end{définition}
\begin{exemple}\jya khɯtsa ɯ-ŋgɯ tʂha ɯ-qiɯ kɯ-xtɕi tú-wɣ-rku tɕe ɯ-taʁ tɯ-sqar pjɯ́-wɣ-lɤt tɕe tɕhɯβroʁ staʁnɤ kɯ-spɯ tsa ɲɯ́-wɣ-ɕmi tɕe tú-wɣ-ndza tɕe nɯnɯ tɤmphoʁ rmi.\cmn 
在碗里倒小半碗的水,再放上少量糌粑,搅均匀后就可以吃。比\stylefv{tɕhɯβroʁ}干一点 。这种吃法叫作\stylefv{tɤmphoʁ}。
\end{exemple}\end{entrée}

\begin{entrée}
\vedette{\hypertarget{Ⓔtɤ-mtɕar}{\papi{ tɤ-mtɕar}}}\markboth{tɤ-mtɕar}{}\classe{np}
\begin{définition}\fra pièce de tissu triangulaire utilisée dans les habits tibétains\end{définition}
\begin{définition}\cmn 藏式服装中的长三角形的布料\end{définition}
\begin{exemple}\jya kɯrɯŋga chɯ́-wɣ-tʂɯβ tɕe, tɯ-ŋga ɣɯ χchoʁe ʑo tɤ-mtɕar tú-wɣ-lɤt ra ma nɯ mɤɕtʂa mɤ-nɯɣɯŋke\cmn 缝藏装时,衣服的左右两边必须要缝上三角形布料,不然不便于走路。\end{exemple}\end{entrée}

\begin{entrée}
\vedette{\hypertarget{Ⓔtɤ-mtɕho}{\papi{ tɤ-mtɕho}}}\markboth{tɤ-mtɕho}{}\classe{np}
\begin{définition}\fra cale, coin\end{définition}
\begin{définition}\cmn 楔子【尖】\end{définition}
\begin{exemple}\jya qaʁ ɯ-mtɕho\cmn 锄头的楔子\end{exemple}
\begin{relation-sémantique}\synonyme{
\hyperlink{Ⓔtɤ-chɯ}{\textit{ \papi{tɤ-chɯ}}}
}\end{relation-sémantique}\end{entrée}

\begin{entrée}
\vedette{\hypertarget{Ⓔtɤmtɕhoʁ}{\papi{ tɤmtɕhoʁ}}}\markboth{tɤmtɕhoʁ}{}\classe{n}
\begin{définition}\fra écharde\end{définition}
\begin{définition}\cmn 木刺(插入皮肉)\end{définition}
\begin{exemple}\jya a-jaʁ tɤmtɕhoʁ to-ɕe\cmn 我的手被木刺刺到\end{exemple}
\begin{exemple}\jya a-jaʁ tɤmtɕhoʁ thɯ-ari tɕe ɲɯ-mŋɤm\cmn 我的手被木刺刺到,很痛\end{exemple}\end{entrée}

\begin{entrée}
\vedette{\hypertarget{Ⓔtɤ-mthɯm}{\papi{ tɤ-mthɯm}}}\markboth{tɤ-mthɯm}{}
\classe{np}
\begin{définition}\fra viande cuite\end{définition}
\begin{définition}\cmn 熟肉\end{définition}\end{entrée}

\begin{entrée}
\vedette{\hypertarget{Ⓔtɤmtshɤr}{\papi{ tɤmtshɤr}}}\markboth{tɤmtshɤr}{}
\classe{n}
\begin{définition}\fra chose étrange\end{définition}
\begin{définition}\cmn 怪事\end{définition}
\begin{exemple}\jya ki ʑo tɤ-fse tɕe, tɤmtshɤr ci ɬoʁ\cmn 这种情况下,会出现怪事\end{exemple}
\begin{relation-sémantique}\confer{
\hyperlink{Ⓔsɤmtshɤr}{\textit{ \papi{sɤmtshɤr}}}
}\end{relation-sémantique}
\begin{relation-sémantique}\confer{
\hyperlink{Ⓔnɤmtshɤr}{\textit{ \papi{nɤmtshɤr}}}
}\end{relation-sémantique}\end{entrée}

\begin{entrée}
\vedette{\hypertarget{Ⓔtɤmtshɤz}{\papi{ tɤmtshɤz}}}\markboth{tɤmtshɤz}{}\paradigme{\textit{dir :} \jya kɤ-}
\begin{définition}\fra hyperostose\end{définition}
\begin{définition}\cmn 骨质增生\end{définition}
\begin{exemple}\jya tɤmtshɤz ko-ndo\cmn 他得了骨质增生\end{exemple}
\begin{exemple}\jya a-tɤmtshɤz ɲɤ-ta\cmn 我的骨质增生又发作了\end{exemple}
\begin{exemple}\jya ɕɤrɯ pɯ-mɲɤt tɕe ɯ-rɯruz kɤ-ɣɤmna mɯ-pɯ-kɯ-khɯ tɕe, nɯ ɯ-stu nɯ ɯ-zbɤβ ku-ndzoʁ ɲɯ-ŋgrɤl tɕe, ɯ-nɯnɯ tɤ-rʑaʁ tɤ-rɲɟi tɕe, kɤ-nɤma nɤɴqa kɯ-fse, tɯ-mɯ tɤ-ɲɟɯr kɯ-fse tɕe wuma ʑo ɲɯ-mŋɤm. nɯnɯ tɤmtshɤz kɤ-kɯ-ndo rmi.\cmn 骨折当时没能治好,在骨折的地方会长出胞来,时间一长,劳累了,天气变化了都会很痛。这种病叫骨质增生病。\end{exemple}\end{entrée}

\begin{entrée}
\vedette{\hypertarget{Ⓔtɤ-mtsɯ}{\papi{ tɤ-mtsɯ}}}\markboth{tɤ-mtsɯ}{}
\classe{np}
\begin{définition}\fra bouton\end{définition}
\begin{définition}\cmn 扣子\end{définition}
\begin{exemple}\jya tɤmtsɯ kɤ-lɤt / nɯ-rle\cmn 你把扣子扣上/解开\end{exemple}\end{entrée}

\begin{entrée}
\vedette{\hypertarget{Ⓔtɤmtsɯr}{\papi{ tɤmtsɯr}}}\markboth{tɤmtsɯr}{}\classe{n}
\begin{définition}\fra faim\end{définition}
\begin{définition}\cmn 饥饿\end{définition}
\begin{exemple}\jya tɤmtsɯr ɲɤ-nɤɕqa tɕe mɯ-to-ndza\cmn 他忍住饥饿没有吃\end{exemple}
\begin{relation-sémantique}\confer{
\hyperlink{Ⓔmtsɯr}{\textit{ \papi{mtsɯr}}}
}\end{relation-sémantique}\end{entrée}

\begin{entrée}
\vedette{\hypertarget{Ⓔtɤ-mtɯ}{\papi{ tɤ-mtɯ}}}\markboth{tɤ-mtɯ}{}
\classe{np}
\begin{définition}\fra nœud\end{définition}
\begin{définition}\cmn 结\end{définition}
\begin{exemple}\jya tɤ-mtɯ tɤ-lat-a\cmn 我打了(个)结\end{exemple}
\begin{exemple}\jya ɯ-ku ɯ-mtɯ (= ɯ-kɤχcɤl)\cmn 他头顶上\end{exemple}
\begin{relation-sémantique}\confer{
\hyperlink{Ⓔrɯtɤmtɯ}{\textit{ \papi{rɯtɤmtɯ}}}
}\end{relation-sémantique}\end{entrée}

\begin{entrée}
\vedette{\hypertarget{Ⓔtɤmtɯkɯnɤ}{\papi{ tɤmtɯkɯnɤ}}}\markboth{tɤmtɯkɯnɤ}{}\classe{adv}
\begin{définition}\fra exprès\end{définition}
\begin{définition}\cmn 故意\end{définition}
\begin{exemple}\jya tɤmtɯkɯnɤ mɯ-ɲɤ-sɤŋo\cmn 他是故意没有听的\end{exemple}
\begin{relation-sémantique}\synonyme{
\hyperlink{Ⓔtɤrkoz}{\textit{ \papi{tɤrkoz}}}
}\end{relation-sémantique}\end{entrée}

\begin{entrée}
\vedette{\hypertarget{Ⓔtɤmtɯɲaʁ}{\papi{ tɤmtɯɲaʁ}}}\markboth{tɤmtɯɲaʁ}{}\classe{n}
\begin{définition}\fra nœud\end{définition}
\begin{définition}\cmn 死结\end{définition}\end{entrée}

\begin{entrée}
\vedette{\hypertarget{Ⓔtɤmɯm}{\papi{ tɤmɯm}}}\markboth{tɤmɯm}{}\classe{n}
\begin{définition}\fra chose que l'on aime manger\end{définition}
\begin{définition}\cmn 自己喜欢吃的东西\end{définition}
\begin{exemple}\jya nɤ-tɤmɯm pɯ-ndzoʁ\cmn 你可以享用好吃的东西了\end{exemple}
\begin{relation-sémantique}\confer{
\hyperlink{Ⓔmɯm}{\textit{ \papi{mɯm}}}
}\end{relation-sémantique}\end{entrée}

\begin{entrée}
\vedette{\hypertarget{Ⓔtɤmɯmɯm}{\papi{ tɤmɯmɯm}}}\markboth{tɤmɯmɯm}{}
\classe{n}
\begin{définition}\fra clochette\end{définition}
\begin{définition}\cmn 铃铛\end{définition}\end{entrée}

\begin{entrée}
\vedette{\hypertarget{Ⓔtɤmɯt}{\papi{ tɤmɯt}}}\markboth{tɤmɯt}{}\classe{n}
\begin{définition}\fra souffle\end{définition}
\begin{définition}\cmn 吹出来的气\end{définition}
\begin{relation-sémantique}\confer{
\hyperlink{Ⓔɣɤmɯt}{\textit{ \papi{ɣɤmɯt}}}
}\end{relation-sémantique}\end{entrée}

\begin{entrée}
\vedette{\hypertarget{Ⓔtɤndɤɣ}{\papi{ tɤndɤɣ}}}\markboth{tɤndɤɣ}{}
\classe{n}
\begin{définition}\fra poison\end{définition}
\begin{définition}\cmn 毒
\begin{déclaration} \étymologie{\papi{dug}}\end{déclaration}\end{définition}\end{entrée}

\begin{entrée}
\vedette{\hypertarget{Ⓔtɤndɤɣri}{\papi{ tɤndɤɣri}}}\markboth{tɤndɤɣri}{}\classe{n}
\begin{définition}\fra enfant illégitime\end{définition}
\begin{définition}\cmn 私生子\end{définition}
\begin{exemple}\jya tɤndɤɣri pjɤ-tɕɤt-ndʑi\cmn 他们俩有了私生子\end{exemple}
\begin{relation-sémantique}\confer{
\hyperlink{Ⓔnɤndɤɣri}{\textit{ \papi{nɤndɤɣri}}}
}\end{relation-sémantique}\end{entrée}

\begin{entrée}
\vedette{\hypertarget{Ⓔtɤndɤku}{\papi{ tɤndɤku}}}\markboth{tɤndɤku}{}
\classe{n}
\begin{définition}\fra espèce de plante\end{définition}
\begin{définition}\cmn 【万年青】\end{définition}
\begin{exemple}\jya tɤndɤku nɯ si kɯ-mbɯ-mbɤr ci ŋu, ɯ-jwaʁ cho ɯ-mɯntoʁ nɯ ra khɯjŋga fse ri xtɕi, ɯ-ru ldʑɯz, sɤndɤɣ\cmn 野生万年青是矮小的树种,叶子和花和洋角花相似,但小一些,树干柔软,有毒性。\end{exemple}\end{entrée}

\begin{entrée}
\vedette{\hypertarget{Ⓔtɤndɤr}{\papi{ tɤndɤr}}}\markboth{tɤndɤr}{}
\classe{n}
\begin{définition}\fra bouton\end{définition}
\begin{définition}\cmn 粉刺\end{définition}\end{entrée}

\begin{entrée}
\vedette{\hypertarget{Ⓔtɤndʐo}{\papi{ tɤndʐo}}}\markboth{tɤndʐo}{}\classe{n}
\begin{définition}\fra froid\end{définition}
\begin{définition}\cmn 寒冷的(天气)\end{définition}
\begin{relation-sémantique}\confer{
\hyperlink{Ⓔɣɤndʐo}{\textit{ \papi{ɣɤndʐo}}}
}\end{relation-sémantique}
\begin{relation-sémantique}\confer{
\hyperlink{Ⓔnɤndʐo}{\textit{ \papi{nɤndʐo}}}
}\end{relation-sémantique}
\begin{sous-entrée}
\vedette{\hypertarget{}{\papi{ tɤndʐo,tɤrqɯ}}}\markboth{tɤndʐo,tɤrqɯ}{}\classe{n}
\begin{exemple}\jya ɣɯjpa rcanɯ tɤndʐo tɤrqɯ pɯ-thɯɣ\cmn 今年非常寒冷\end{exemple}
\begin{relation-sémantique}\confer{
\hyperlink{Ⓔnɤndʐɤrqɯ}{\textit{ \papi{nɤndʐɤrqɯ}}}
}\end{relation-sémantique}
\end{sous-entrée}\end{entrée}

\begin{entrée}
\vedette{\hypertarget{Ⓔtɤndoʁ}{\papi{ tɤndoʁ}}}\markboth{tɤndoʁ}{}
\classe{n}
\begin{définition}\fra copeaux (à la hache)\end{définition}
\begin{définition}\cmn 切屑(斧头)\end{définition}\end{entrée}

\begin{entrée}
\vedette{\hypertarget{Ⓔtɤ-ndɯr}{\papi{ tɤ-ndɯr}}}\markboth{tɤ-ndɯr}{}
\classe{np}
\begin{définition}\fra débris, lie\end{définition}
\begin{définition}\cmn 渣滓\end{définition}
\begin{exemple}\jya tɤ-ndɯr ɲo-ri\cmn 剩下了渣滓\end{exemple}\end{entrée}

\begin{entrée}
\vedette{\hypertarget{Ⓔtɤ-ndzraʁ}{\papi{ tɤ-ndzraʁ}}}\markboth{tɤ-ndzraʁ}{}
\classe{np}
\begin{définition}\fra morceau de tsampa, de glaise roulé en boule\end{définition}
\begin{définition}\cmn 糌粑坨\end{définition}
\begin{exemple}\jya chɤ-rgɤz tɕe, tɤ-ndzraʁ ma ɲo-me\cmn 他老得只剩下一坨\end{exemple}
\begin{exemple}\jya rɟɤɣi-ndzraʁ ɯ-tɯ́-ndze\cmn 你吃不吃糌粑坨坨\end{exemple}\end{entrée}

\begin{entrée}
\vedette{\hypertarget{Ⓔtɤ-ndʑɯɣ}{\papi{ tɤ-ndʑɯɣ}}}\markboth{tɤ-ndʑɯɣ}{}\classe{np}
\begin{définition}\fra résine\end{définition}
\begin{définition}\cmn 松香;树脂\end{définition}
\begin{exemple}\jya tɯrgi ɯ-ndʑɯɣ\cmn 杉树树脂\end{exemple}
\begin{relation-sémantique}\confer{
\hyperlink{Ⓔaɣɯndʑɯɣ}{\textit{ \papi{aɣɯndʑɯɣ}}}
}\end{relation-sémantique}\end{entrée}

\begin{entrée}
\vedette{\hypertarget{Ⓔtɤngɯt}{\papi{ tɤngɯt}}}\markboth{tɤngɯt}{}\classe{n}
\begin{définition}\fra possessions en commun\end{définition}
\begin{définition}\cmn 共同拥有的东西\end{définition}
\begin{exemple}\jya kɯki @luyinji tɕi-tɤngɯt ŋu\cmn 这个录音机是我们俩共同拥有的\end{exemple}
\begin{relation-sémantique}\confer{
\hyperlink{Ⓔnɤngɯt}{\textit{ \papi{nɤngɯt}}}
}\end{relation-sémantique}\end{entrée}

\begin{entrée}
\vedette{\hypertarget{Ⓔtɤ-nmaʁ}{\papi{ tɤ-nmaʁ}}}\markboth{tɤ-nmaʁ}{}\classe{np}
\begin{définition}\fra mari\end{définition}
\begin{définition}\cmn 丈夫\end{définition}\end{entrée}

\begin{entrée}
\vedette{\hypertarget{Ⓔtɤntɤβ}{\papi{ tɤntɤβ}}}\markboth{tɤntɤβ}{}
\classe{n}\acception{1}
\begin{définition}\fra bulle\end{définition}
\begin{définition}\cmn 水泡\end{définition}\acception{2}
\begin{définition}\fra écume\end{définition}
\begin{définition}\cmn 泡沫\end{définition}
\begin{relation-sémantique}\confer{
\hyperlink{Ⓔaɣɯntɤβ}{\textit{ \papi{aɣɯntɤβ}}}
}\end{relation-sémantique}\end{entrée}

\begin{entrée}
\vedette{\hypertarget{Ⓔtɤɲi}{\papi{ tɤɲi}}}\markboth{tɤɲi}{}\classe{n}
\begin{définition}\fra bâton\end{définition}
\begin{définition}\cmn 拐棍\end{définition}
\begin{exemple}\jya tɤɲi kɤ-ndo\cmn 拄着拐棍\end{exemple}
\begin{relation-sémantique}\confer{
\hyperlink{Ⓔnɤɲi}{\textit{ \papi{nɤɲi}}}
}\end{relation-sémantique}\end{entrée}

\begin{entrée}
\vedette{\hypertarget{Ⓔtɤ-ɲi}{\papi{ tɤ-ɲi}}}\markboth{tɤ-ɲi}{}
\classe{np}
\begin{définition}\fra tante (sœur du père)\end{définition}
\begin{définition}\cmn 姑母\end{définition}
\begin{exemple}\jya a-ɲi\cmn 我的姑妈\end{exemple}\end{entrée}

\begin{entrée}
\vedette{\hypertarget{Ⓔtɤ-ɲɟoʁɲɟi}{\papi{ tɤ-ɲɟoʁɲɟi}}}\markboth{tɤ-ɲɟoʁɲɟi}{}\classe{np}
\begin{définition}\fra ordure\end{définition}
\begin{définition}\cmn 垃圾\end{définition}
\begin{exemple}\jya tɕiʑo saχsɯ nɯ-anɯri-tɕi ɯ-qhu tɕi-tɤ-ɲɟoʁɲɟi ra ɣɯ-jo-ru-nɯ\cmn 在我们俩出去吃中午饭之后,他们捡了垃圾\end{exemple}\end{entrée}

\begin{entrée}
\vedette{\hypertarget{Ⓔtɤ-ŋɤm}{\papi{ tɤ-ŋɤm}}}\markboth{tɤ-ŋɤm}{}\classe{np}
\begin{définition}\fra douleur\end{définition}
\begin{définition}\cmn 痛\end{définition}
\begin{exemple}\jya ndʑi-ŋgo ndʑi-ŋɤm a-pɯ-me\cmn 但愿你们俩没有什么病痛\end{exemple}
\begin{relation-sémantique}\confer{
\hyperlink{Ⓔmŋɤm}{\textit{ \papi{mŋɤm}}}
}\end{relation-sémantique}\end{entrée}

\begin{entrée}
\vedette{\hypertarget{Ⓔtɤŋɤmɕɣɤphɯt}{\papi{ tɤŋɤmɕɣɤphɯt}}}\markboth{tɤŋɤmɕɣɤphɯt}{}
\classe{n}
\begin{définition}\fra sensation de soulagement lorsqu'on arrache une dent qui fait souffrir\end{définition}
\begin{définition}\cmn 好得又快又彻底(牙齿痛得厉害的时候,把发痛的牙齿拔掉了就一下子不痛了)\end{définition}
\begin{exemple}\jya tɤŋɤmɕɣɤphɯt to-βzu tɕe nɯɕɯmɯma ʑo to-mna\cmn 牙一拔就不痛了\end{exemple}
\begin{relation-sémantique}\confer{
\hyperlink{Ⓔtɯ-ɕɣa}{\textit{ \papi{tɯ-ɕɣa}}}
}\end{relation-sémantique}
\begin{relation-sémantique}\confer{
\hyperlink{Ⓔphɯt}{\textit{ \papi{phɯt}}}
}\end{relation-sémantique}\end{entrée}

\begin{entrée}
\vedette{\hypertarget{Ⓔtɤŋe}{\papi{ tɤŋe}}}\markboth{tɤŋe}{}
\classe{n}
\begin{définition}\fra soleil\end{définition}
\begin{définition}\cmn 太阳\end{définition}
\begin{exemple}\jya tɤŋe ci ci ɣɤʑu ci ci maŋe\cmn 一会有太阳,一会没有\end{exemple}
\begin{exemple}\jya tɤŋe tɤ-ɬoʁ\cmn 太阳升起了\end{exemple}
\begin{exemple}\jya tɤŋe ɲɤ-k-ɤβzu-ci\cmn (云散了,)太阳就露面了\end{exemple}
\begin{relation-sémantique}\confer{
\hyperlink{Ⓔslɤŋe}{\textit{ \papi{slɤŋe}}}
}\end{relation-sémantique}\end{entrée}

\begin{entrée}
\vedette{\hypertarget{Ⓔtɤŋgɤr}{\papi{ tɤŋgɤr}}}\markboth{tɤŋgɤr}{}
\classe{n}
\begin{définition}\fra lard\end{définition}
\begin{définition}\cmn 膘\end{définition}\end{entrée}

\begin{entrée}
\vedette{\hypertarget{Ⓔtɤŋgɯ}{\papi{ tɤŋgɯ}}}\markboth{tɤŋgɯ}{}
\classe{n}
\begin{définition}\fra prêt\end{définition}
\begin{définition}\cmn 借的东西\end{définition}
\begin{exemple}\jya tɯjpu tɤŋgɯ na-mɟa (=tɯjpu na-nɤŋgɯ)\cmn 他借了粮食\end{exemple}
\begin{relation-sémantique}\confer{
\hyperlink{Ⓔnɤŋgɯ}{\textit{ \papi{nɤŋgɯ}}}
}\end{relation-sémantique}\end{entrée}

\begin{entrée}
\vedette{\hypertarget{Ⓔtɤ-ŋgɯm}{\papi{ tɤ-ŋgɯm}}}\markboth{tɤ-ŋgɯm}{}
\classe{np}
\begin{définition}\fra œuf\end{définition}
\begin{définition}\cmn 蛋\end{définition}\end{entrée}

\begin{entrée}
\vedette{\hypertarget{Ⓔtɤŋkhɯt}{\papi{ tɤŋkhɯt}}}\markboth{tɤŋkhɯt}{}
\classe{n}
\begin{définition}\fra poing\end{définition}
\begin{définition}\cmn 拳\end{définition}
\begin{exemple}\jya a-tɤŋkhɯt\cmn 我的拳头\end{exemple}
\begin{relation-sémantique}\confer{
\hyperlink{Ⓔnɤŋkhɯt}{\textit{ \papi{nɤŋkhɯt}}}
}\end{relation-sémantique}\end{entrée}

\begin{entrée}
\vedette{\hypertarget{Ⓔtɤ-ŋkɯ}{\papi{ tɤ-ŋkɯ}}}\markboth{tɤ-ŋkɯ}{}\classe{np}
\begin{définition}\fra couenne\end{définition}
\begin{définition}\cmn 猪皮\end{définition}
\begin{exemple}\jya paʁ ɯ-ŋkɯ ɲɯ-jaʁ\cmn 猪的皮很厚\end{exemple}\end{entrée}

\begin{entrée}
\vedette{\hypertarget{Ⓔtɤɴqa}{\papi{ tɤɴqa}}}\markboth{tɤɴqa}{}\classe{n}
\begin{définition}\fra difficulté\end{définition}
\begin{définition}\cmn 辛苦\end{définition}
\begin{exemple}\jya kɯmɤlɤxso ji-tɤɴqa pjɤ-ɕti\cmn 我们白辛苦了\end{exemple}
\begin{relation-sémantique}\confer{
\hyperlink{Ⓔɴqa}{\textit{ \papi{ɴqa}}}
}\end{relation-sémantique}\end{entrée}

\begin{entrée}
\vedette{\hypertarget{Ⓔtɤɴqhi}{\papi{ tɤɴqhi}}}\markboth{tɤɴqhi}{}
\classe{n}
\begin{définition}\fra saletés\end{définition}
\begin{définition}\cmn 污垢\end{définition}
\begin{relation-sémantique}\confer{
\hyperlink{Ⓔɴqhi}{\textit{ \papi{ɴqhi}}}
}\end{relation-sémantique}
\begin{relation-sémantique}\confer{
\hyperlink{Ⓔtɤlɤɴqhi}{\textit{ \papi{tɤlɤɴqhi}}}
}\end{relation-sémantique}\end{entrée}

\begin{entrée}
\vedette{\hypertarget{Ⓔtɤ-pa}{\papi{ tɤ-pa}}}\markboth{tɤ-pa}{}\classe{np}
\begin{définition}\fra père (noble)\end{définition}
\begin{définition}\cmn 父亲(贵族用语)\end{définition}
\begin{exemple}\jya a-pa a-ma\cmn 我的父母\end{exemple}\end{entrée}

\begin{entrée}
\vedette{\hypertarget{Ⓔtɤ-pɤloʁ}{\papi{ tɤ-pɤloʁ}}}\markboth{tɤ-pɤloʁ}{}
\classe{np}
\begin{définition}\fra manche\end{définition}
\begin{définition}\cmn 袖子\end{définition}
\begin{exemple}\jya a-pɤloʁ\cmn 我的袖子\end{exemple}\end{entrée}

\begin{entrée}
\vedette{\hypertarget{Ⓔtɤ-pɤndɯr}{\papi{ tɤ-pɤndɯr}}}\markboth{tɤ-pɤndɯr}{}
\classe{np}
\begin{définition}\fra mauvais caractère\end{définition}
\begin{définition}\cmn 做事遇到挫折回家就发脾气\end{définition}
\begin{exemple}\jya nɤ-pɤndɯr a-mɤ-jɤ-tɯ-ɣɯt je\cmn 别回来发脾气\end{exemple}\end{entrée}

\begin{entrée}
\vedette{\hypertarget{Ⓔtɤpɤr}{\papi{ tɤpɤr}}}\markboth{tɤpɤr}{}
\classe{n}
\begin{définition}\fra épi de maïs\end{définition}
\begin{définition}\cmn 玉米包\end{définition}\end{entrée}

\begin{entrée}
\vedette{\hypertarget{Ⓔtɤ-pɤri}{\papi{ tɤ-pɤri}}}\markboth{tɤ-pɤri}{}
\classe{np}
\begin{définition}\fra repas du soir\end{définition}
\begin{définition}\cmn 晚饭\end{définition}
\begin{exemple}\jya a-pɤri to-mda\cmn 我要吃晚餐\end{exemple}\end{entrée}

\begin{entrée}
\vedette{\hypertarget{Ⓔtɤ-pɤro}{\papi{ tɤ-pɤro}}}\markboth{tɤ-pɤro}{}
\classe{np}
\begin{définition}\fra cadeau\end{définition}
\begin{définition}\cmn 礼物(自己亲手拿给别人)\end{définition}
\begin{exemple}\jya a-pɤro\cmn 我给别人的礼物\end{exemple}
\begin{exemple}\jya a-pɤro ɲɯ-ta-mbi ŋu\cmn 我把礼物送给你\end{exemple}
\begin{exemple}\jya aʑɯɣ nɤ-pɤro jo-tɯ-ɣɯt\cmn 你给我带了礼物\end{exemple}
\begin{relation-sémantique}\synonyme{
\hyperlink{Ⓔskɯrma}{\textit{ \papi{skɯrma}}}
}\end{relation-sémantique}
\begin{relation-sémantique}\synonyme{
\hyperlink{Ⓔtɤ-rkuz}{\textit{ \papi{tɤ-rkuz}}}
}\end{relation-sémantique}
\begin{relation-sémantique}\synonyme{
\hyperlink{Ⓔmpɕɯmɤr}{\textit{ \papi{mpɕɯmɤr}}}
}\end{relation-sémantique}\end{entrée}

\begin{entrée}
\vedette{\hypertarget{Ⓔtɤ-pɤtso}{\papi{ tɤ-pɤtso}}}\markboth{tɤ-pɤtso}{}
\classe{np}
\begin{définition}\fra enfant\end{définition}
\begin{définition}\cmn 孩子\end{définition}
\begin{exemple}\jya tɤ-pɤtso ɯ-skɤt\cmn 小孩子的语气\end{exemple}
\begin{exemple}\jya ɯ-pɤtso ɣɤʑu\cmn 她怀上了小孩\end{exemple}
\begin{relation-sémantique}\confer{
\hyperlink{Ⓔnɯtɤpɤtso}{\textit{ \papi{nɯtɤpɤtso}}}
}\end{relation-sémantique}
\begin{relation-sémantique}\confer{
\hyperlink{Ⓔarɯtɤpɤtso}{\textit{ \papi{arɯtɤpɤtso}}}
}\end{relation-sémantique}\end{entrée}

\begin{entrée}
\vedette{\hypertarget{Ⓔtɤpɤtsoβraʁ}{\papi{ tɤpɤtsoβraʁ}}}\markboth{tɤpɤtsoβraʁ}{}\classe{n}
\begin{définition}\fra petit phasme\end{définition}
\begin{définition}\cmn 小的树枝虫\end{définition}
\begin{exemple}\jya tɤpɤtsoβraʁ nɯ sɯjnombrombro cho kɯ-naχtɕɯ-χtɕɯɣ ŋu, li ʁnɯ-tɯphu tu, ldʑaŋkɯ ci kɯ-pɣi ci tu, ndʑi-tɯ-xtshɯm naχtɕɯɣ tɕe nɯ a-pɯ́-wɣ-mto tɕe nɯ maʁ nɤ tɯ-taʁ a-tɤ-ɣi tɕe, nɯ maʁ nɤ tɯʑo tɤ-rɟit tu, nɯ maʁ nɤ tɯ-kɯmdza ra nɯ-rɟit tu tu-kɯ-ti ɲɯ-ŋu tɕe núndʐa tɤ-pɤtso βraʁ ɲɯ-rmi\cmn 
\stylefv{tɤpɤtsoβraʁ}和秤杆虫一模一样,也有两种,绿色的和灰色的,两种一样细。据说如果人看见了它,或者如果它爬到人的身上来了,要么自己会有身孕,要么自己亲戚会有身孕,所以叫\stylefv{tɤpɤtsoβraʁ}(小孩子的象征)
\end{exemple}\end{entrée}

\begin{entrée}
\vedette{\hypertarget{Ⓔtɤpɣi}{\papi{ tɤpɣi}}}\markboth{tɤpɣi}{}
\classe{n}
\begin{définition}\fra maladie de l'œil\end{définition}
\begin{définition}\cmn 眼病\end{définition}
\begin{exemple}\jya a-mɲaʁ tɤpɣi to-ɣi\cmn 我眼睛上长了白点\end{exemple}
\begin{exemple}\jya tɤ-mɲaʁ-rdu kɯ-ɲaʁ ɯ-taʁ kɯ-wɣrum kɯ-xtɕɯ-xtɕi nɯ-kɯ-ɬoʁ nɯ, wuma ʑo mŋɤm tɕe tɤpɣi rmi\cmn 黑眼珠上长了白点,很疼。\end{exemple}\end{entrée}

\begin{entrée}
\vedette{\hypertarget{Ⓔtɤphɯ}{\papi{ tɤphɯ}}}\markboth{tɤphɯ}{}
\classe{n}
\begin{définition}\fra motte de terre\end{définition}
\begin{définition}\cmn 土块\end{définition}\end{entrée}

\begin{entrée}
\vedette{\hypertarget{Ⓔtɤphɯxtsɯ}{\papi{ tɤphɯxtsɯ}}}\markboth{tɤphɯxtsɯ}{}
\classe{n}
\begin{définition}\fra fait d'écraser les mottes de terre\end{définition}
\begin{définition}\cmn 打土巴\end{définition}
\begin{exemple}\jya tɤphɯxtsɯ tɤ-βzu-t-a\cmn 我打了土巴\end{exemple}
\begin{relation-sémantique}\confer{
\hyperlink{Ⓔtɤphɯ}{\textit{ \papi{tɤphɯ}}}
}\end{relation-sémantique}
\begin{relation-sémantique}\confer{
\hyperlink{Ⓔxtsɯ}{\textit{ \papi{xtsɯ}}}
}\end{relation-sémantique}
\begin{relation-sémantique}\confer{
\hyperlink{Ⓔnɤphɯxtsɯ}{\textit{ \papi{nɤphɯxtsɯ}}}
}\end{relation-sémantique}\end{entrée}

\begin{entrée}
\vedette{\hypertarget{Ⓔtɤ-pi}{\papi{ tɤ-pi}}}\markboth{tɤ-pi}{}\classe{np}\acception{1}
\begin{définition}\fra grand frère, grande sœur\end{définition}
\begin{définition}\cmn 哥哥;姐姐\end{définition}
\begin{exemple}\jya a-pi\cmn 我的哥哥(我的姐姐)\end{exemple}\acception{2}
\begin{définition}\fra hôte\end{définition}
\begin{définition}\cmn 客人
\begin{déclaration}\use{沙尔宗方言}\end{déclaration}\end{définition}
\begin{exemple}\jya tɤ-ndza-nɯ je ma tɤ-pi ɯ-zɤz mɤ-sɤfka kɤ-ti tɕe tha mɤ-tɯ-fka-nɯ\cmn 你们吃吧,俗话说:“做客人的食物吃不饱”,你们会吃不饱的\end{exemple}\end{entrée}

\begin{entrée}
\vedette{\hypertarget{Ⓔtɤpjaʁ}{\papi{ tɤpjaʁ}}}\markboth{tɤpjaʁ}{}
\classe{n}
\begin{définition}\fra morceau de bois coupé en parallélépipède\end{définition}
\begin{définition}\cmn 木方条\end{définition}\end{entrée}

\begin{entrée}
\vedette{\hypertarget{Ⓔtɤpjɤz}{\papi{ tɤpjɤz}}}\markboth{tɤpjɤz}{}
\classe{n}
\begin{définition}\fra tresse\end{définition}
\begin{définition}\cmn 辫子\end{définition}
\begin{exemple}\jya a-tɤpjɤz\cmn 我的辫子\end{exemple}
\begin{exemple}\jya tɤpjɤz tha-βzu\cmn 他编了辫子\end{exemple}
\begin{relation-sémantique}\confer{
\hyperlink{Ⓔrɤpjɤz}{\textit{ \papi{rɤpjɤz}}}
}\end{relation-sémantique}\end{entrée}

\begin{entrée}
\vedette{\hypertarget{Ⓔtɤpra}{\papi{ tɤpra}}}\markboth{tɤpra}{}\classe{n}
\begin{définition}\fra messager, envoyé\end{définition}
\begin{définition}\cmn 使者;派出去的人\end{définition}
\begin{exemple}\jya aʑo nɤ-tɤpra tu-βze-a jɤɣ\cmn 我可以当你的使者\end{exemple}
\begin{relation-sémantique}\confer{
\hyperlink{Ⓔɣɤxpra}{\textit{ \papi{ɣɤxpra}}}
}\end{relation-sémantique}\end{entrée}

\begin{entrée}
\vedette{\hypertarget{Ⓔtɤ-prɤm}{\papi{ tɤ-prɤm}}}\markboth{tɤ-prɤm}{}\classe{np}
\begin{définition}\fra nourriture en poudre\end{définition}
\begin{définition}\cmn 粉状粮食\end{définition}
\begin{exemple}\jya tɤ-prɤm pɯ-lɤt\cmn 加一点粉吧\end{exemple}\end{entrée}

\begin{entrée}
\vedette{\hypertarget{Ⓔtɤprɯ}{\papi{ tɤprɯ}}}\markboth{tɤprɯ}{}\classe{n}
\begin{définition}\fra abri de pluie\end{définition}
\begin{définition}\cmn 避雨的地方\end{définition}
\begin{relation-sémantique}\confer{
\hyperlink{Ⓔprɯ}{\textit{ \papi{prɯ}}}
}\end{relation-sémantique}
\begin{relation-sémantique}\confer{
\hyperlink{Ⓔnɤprɯ}{\textit{ \papi{nɤprɯ}}}
}\end{relation-sémantique}\end{entrée}

\begin{entrée}
\vedette{\hypertarget{Ⓔtɤ-pɯ}{\papi{ tɤ-pɯ}}}\markboth{tɤ-pɯ}{}\classe{np}\acception{1}
\begin{définition}\fra petit (animal)\end{définition}
\begin{définition}\cmn 崽子\end{définition}\acception{2}
\begin{définition}\fra intérêt\end{définition}
\begin{définition}\cmn 利息
\begin{déclaration}\use{\stylefv{ɯpɯ}同动词\stylefv{pa}连用时带有“收藏”的意思}\end{déclaration}\end{définition}
\begin{exemple}\jya tɤ-pɯ nɯ ɯʑoz kú-wɣ-ja\cmn 要把小的关要另外的圈里\end{exemple}
\begin{exemple}\jya ɯ-pɯ tɤ-nɯ-pe\cmn 你把它收藏起来\end{exemple}
\begin{exemple}\jya ɯ-pɯ to-nɯ-pa\cmn 他收藏起来了\end{exemple}
\begin{relation-sémantique}\confer{
 \papi{ɯ-pɯ,pa}
}\end{relation-sémantique}\end{entrée}

\begin{entrée}
\vedette{\hypertarget{Ⓔtɤ-qaʁrɯ}{\papi{ tɤ-qaʁrɯ}}}\markboth{tɤ-qaʁrɯ}{} (\variante{ɯ-qataʁrɯ}) 
\classe{np}
\begin{définition}\fra sabot\end{définition}
\begin{définition}\cmn 蹄子\end{définition}
\begin{relation-sémantique}\confer{
\hyperlink{Ⓔtɯ-qa}{\textit{ \papi{tɯ-qa}}}
}\end{relation-sémantique}\end{entrée}

\begin{entrée}
\vedette{\hypertarget{Ⓔtɤ-qɤtɕɤz}{\papi{ tɤ-qɤtɕɤz}}}\markboth{tɤ-qɤtɕɤz}{}\classe{np}
\begin{définition}\fra trace de patte\end{définition}
\begin{définition}\cmn 脚印(动物)\end{définition}
\begin{relation-sémantique}\confer{
\hyperlink{Ⓔtɤ-tɕɤz}{\textit{ \papi{tɤ-tɕɤz}}}
}\end{relation-sémantique}
\begin{relation-sémantique}\confer{
\hyperlink{Ⓔtɯ-qa}{\textit{ \papi{tɯ-qa}}}
}\end{relation-sémantique}\end{entrée}

\begin{entrée}
\vedette{\hypertarget{Ⓔtɤqɤt,lɤt}{\papi{ tɤqɤt,lɤt}}}\markboth{tɤqɤt,lɤt}{}\paradigme{\textit{dir :} \jya lɤ-}
\begin{définition}\fra séparer une chambre en deux\end{définition}
\begin{définition}\cmn 把大房间隔成两个小房间\end{définition}
\begin{exemple}\jya kha tɤqɤt lɤ-lat-a\cmn 我把房子隔开了\end{exemple}
\begin{relation-sémantique}\ComponentA{\classe{n}
 \papi{tɤqɤt}
}\end{relation-sémantique}
\begin{relation-sémantique}\ComponentB{\classe{vt}
\hyperlink{ⒺlɤtⒽ1}{\textit{ \papi{lɤt}}}
}\end{relation-sémantique}
\begin{relation-sémantique}\confer{
\hyperlink{Ⓔqɤt}{\textit{ \papi{qɤt}}}
}\end{relation-sémantique}
\begin{relation-sémantique}\confer{
\hyperlink{ⒺlɤtⒽ1}{\textit{ \papi{lɤt1}}}
}\end{relation-sémantique}\end{entrée}

\begin{entrée}
\vedette{\hypertarget{Ⓔtɤqiaβjmɤɣ}{\papi{ tɤqiaβjmɤɣ}}}\markboth{tɤqiaβjmɤɣ}{}\classe{n}
\begin{définition}\fra lactaire\end{définition}
\begin{définition}\cmn 乳菇【苦苦菌】\end{définition}
\begin{exemple}\jya tɤqiaβjmɤɣ nɯ tɯrgi ɕkrɤz ɯ-ŋgɯ ra tu-ɬoʁ ŋu. ɯ-tɯ-wxti nɯ jmɤɣni jamar fse, ɯ-mdoʁ nɯ pɣi, pjɯ́-wɣ-qru tɕe ɯ-ŋgɯ tɤ-lu kɯ-fse ɲɯ-nɯɬoʁ ŋu, kɤ-ndza mɤ-mɯm, qiaβ ri mɤ-sɤndɤɣ\cmn 苦苦菌长在杉木林和青冈树林里,长得和杉木菌一样大小,颜色是灰色的,把它打烂时里面会流出像牛奶一样的汁,不好吃,因为太苦,但是没有毒。\end{exemple}\end{entrée}

\begin{entrée}
\vedette{\hypertarget{ⒺtɤruⒽ2}{\papi{ tɤru}}}\markboth{tɤru}{}\homonyme{2}\classe{n}
\begin{définition}\fra espèce d'arbre\end{définition}
\begin{définition}\cmn 【火棘】\end{définition}
\begin{exemple}\jya tɤ-ru nɯ si wuma mɤ-kɯ-mbro ci ŋu, ɯ-jwaʁ ɯ-qhu nɯ kɯ-pɣi tsa ŋu, ɯ-ru ɯ-rqhu nɯ li kɯ-pɣi tsa ŋu, kɯ-nɤrko ci ŋu, kɯ-rɤma ra ɣɯ nɯ-laʁdɯn ɯ-jɯ kɤ-nɯ-βzu sna. ɯ-mat nɯ thɯ-tɯt tɕe ɣɯrni, paʁ kɤ-sɯχsu sna. zgoku ɯ-taʁ pa ʑo tu-ɬoʁ cha.\cmn 火棘是一种比较矮的树,叶子背面是灰色的,树皮也是灰色的,比较坚实,农民可以用来制造各种农具的把儿。果实成熟时是红色的,可以喂猪。山上山下都可以生长。\end{exemple}\end{entrée}

\begin{entrée}
\vedette{\hypertarget{Ⓔtɤ-ruⒽ1}{\papi{ tɤ-ru}}}\markboth{tɤ-ru}{}\homonyme{1}
\classe{n}
\begin{définition}\fra chef de village\end{définition}
\begin{définition}\cmn 寨首\end{définition}
\begin{exemple}\jya a-ru\cmn 先生(对别人的尊称)\end{exemple}\end{entrée}

\begin{entrée}
\vedette{\hypertarget{Ⓔtɤrɤɕom}{\papi{ tɤrɤɕom}}}\markboth{tɤrɤɕom}{}
\classe{n}
\begin{définition}\fra lame de binette\end{définition}
\begin{définition}\cmn 锄刃\end{définition}
\begin{exemple}\jya tɤrɤɕom nɯ tɤrɤt ɯ-pa tu-kɤ-tshoʁ ɕom ci ŋu. sɤ-ntʂu ɣɯ ɯ-laʁdɯn nɯ tɤrɤt ŋu tɕe, tɤrɤt ɯ-spa nɯ si ɯ-rtaʁ pjɯ́-wɣ-phɯt tɕe, ɯ-rtaʁ nɯ li ɯ-rtaʁ kɯ-tu pjɯ-ŋu ra tɕe, ɯ-rtaʁ tɯ-rdoʁ nɯ pjɯ́-wɣ-ɣɤ-zri ɲɯ́-wɣ-βzu tɕe tɯ-rdoʁ nɯ pjɯ́-wɣ-ɣɤ-xtɯt, ɯ-rtaʁ kɯ-xtɯt pɯ-kɤ-βzu nɯ chɯ́-wɣ-sɯ-ɤmtɕoʁ tɕe nɯ tɕu tɤ-rɤɕom tú-wɣ-tshoʁ. ɯ-rtaʁ kɯ-zri nɯ chɯ́-wɣ-βʑoʁ chɯ́-wɣ-ɣɤ-mpɕu tɕe, nɯnɯ tɤrɤt ɯ-jɯ ŋu, kɤ-ntʂu tɕe nɯ tú-wɣ-ntɕhoz tɕe, kɤ-ntʂu aɲaj tɕe tɤ-rɤku mɤ-sɯ-mɲɤt.\cmn 锄刃是安装在锄头下面的铁。锄草的专用工具叫锄头。锄头是用砍下的树枝作成的,树枝要有叉,其中的一支砍长,另一支砍短一点,砍得较短的那个叉要削尖一些,在那里安装锄刃。长的那一支要削光滑,成了锄头的把子。锄草的时候用它就速度快,不损坏庄稼。\end{exemple}\end{entrée}

\begin{entrée}
\vedette{\hypertarget{Ⓔtɤ-rɤku}{\papi{ tɤ-rɤku}}}\markboth{tɤ-rɤku}{}\classe{np}
\begin{définition}\fra récolte\end{définition}
\begin{définition}\cmn 庄稼\end{définition}
\begin{exemple}\jya ji-rɤku\cmn 我们的庄稼\end{exemple}\end{entrée}

\begin{entrée}
\vedette{\hypertarget{Ⓔtɤrɤm}{\papi{ tɤrɤm}}}\markboth{tɤrɤm}{}
\classe{n}
\begin{définition}\fra planche de bois\end{définition}
\begin{définition}\cmn 木板\end{définition}\end{entrée}

\begin{entrée}
\vedette{\hypertarget{Ⓔtɤrɤmɕkho}{\papi{ tɤrɤmɕkho}}}\markboth{tɤrɤmɕkho}{}
\classe{n}
\begin{définition}\fra parquet\end{définition}
\begin{définition}\cmn 地板\end{définition}\end{entrée}

\begin{entrée}
\vedette{\hypertarget{Ⓔtɤrɤt}{\papi{ tɤrɤt}}}\markboth{tɤrɤt}{}
\classe{n}
\begin{définition}\fra binette\end{définition}
\begin{définition}\cmn 锄草用的锄头\end{définition}
\begin{relation-sémantique}\confer{
\hyperlink{Ⓔtɤrɤɕom}{\textit{ \papi{tɤrɤɕom}}}
}\end{relation-sémantique}\end{entrée}

\begin{entrée}
\vedette{\hypertarget{Ⓔtɤrɤze}{\papi{ tɤrɤze}}}\markboth{tɤrɤze}{}\classe{n}
\begin{définition}\fra prince, jeune maître de maison\end{définition}
\begin{définition}\cmn 少爷\end{définition}\end{entrée}

\begin{entrée}
\vedette{\hypertarget{Ⓔtɤ-rca}{\papi{ tɤ-rca}}}\markboth{tɤ-rca}{}
\classe{np}
\begin{définition}\fra avec, en suivant\end{définition}
\begin{définition}\cmn 跟……一起\end{définition}
\begin{exemple}\jya a-rca jɤ-ɣi\cmn 跟我来!\end{exemple}
\begin{exemple}\jya ɯʑo kɯnɤ a-rca lu-nɯɣi ŋu\cmn 他也跟我回去\end{exemple}\begin{sous-entrée}
\vedette{\hypertarget{}{\papi{ tɤ-rca,me}}}\markboth{tɤ-rca,me}{}
\begin{définition}\fra irrémédiable\end{définition}
\begin{définition}\cmn 无法挽救;无从下手\end{définition}
\begin{exemple}\jya a-laʁtɕha ra thɯ-arɕo tɕe a-rca nɯ-me\cmn 我东西没有了,再也无法挽救\end{exemple}
\begin{exemple}\jya a-rca ci na-ɣɤme\cmn 他把我的事情弄得很乱\end{exemple}
\begin{relation-sémantique}\ComponentA{\classe{np}
\hyperlink{Ⓔtɤ-rca}{\textit{ \papi{tɤ-rca}}}
}\end{relation-sémantique}
\begin{relation-sémantique}\ComponentB{\classe{vs}
\hyperlink{ⒺmeⒽ1}{\textit{ \papi{me}}}
}\end{relation-sémantique}
\end{sous-entrée}\end{entrée}

\begin{entrée}
\vedette{\hypertarget{Ⓔtɤrcoʁ}{\papi{ tɤrcoʁ}}}\markboth{tɤrcoʁ}{}
\classe{n}
\begin{définition}\fra boue\end{définition}
\begin{définition}\cmn 泥巴\end{définition}
\begin{exemple}\jya tɤrcoʁ ɕ-pɯ-βzu-t-a\cmn 我和了泥\end{exemple}
\begin{relation-sémantique}\confer{
\hyperlink{Ⓔrɤrcoʁ}{\textit{ \papi{rɤrcoʁ}}}
}\end{relation-sémantique}
\begin{relation-sémantique}\confer{
\hyperlink{Ⓔɣɤrcoʁ}{\textit{ \papi{ɣɤrcoʁ}}}
}\end{relation-sémantique}\end{entrée}

\begin{entrée}
\vedette{\hypertarget{Ⓔtɤrɕɤz}{\papi{ tɤrɕɤz}}}\markboth{tɤrɕɤz}{}
\classe{n}
\begin{définition}\fra mur en latte de saule\end{définition}
\begin{définition}\cmn 用杨柳树的细条编成的墙壁【巴巴】\end{définition}
\begin{exemple}\jya tɤrɕɤz nɯ kɯɕɯŋgɯ tɤrɤm kɤ-tɕɤt tʂɤm kɤ-rku mɤ-kɯ-cha ra kɯ tɤqɤt ɯ-sɤ-lɤt nɯ-kɤ-βzu pjɤ-ŋu tɕe ʑmbri ɣɯ ɯ-rtaʁ kɯ-xtshɯm tsa kɤ-mɲɤm ɲɯ-ɕar-nɯ tɕe tɤ-jtsi pɤrthɤβ rorʁe ɲɯ-lɤt-nɯ tɕe nɯ ɯ-taʁ nɯ tɕu ɲɯ-taʁ-nɯ kɯ-fse tɕe ɲɯ-βzu-nɯ pjɤ-ŋu, tɕe nɯnɯ kɤ-βzu tɕe si pjɯ-ɣɯrŋi ra ma nɯ-rom tɕe tu-rko ɕti tɕe kɤ-taʁ mɤ-khɯ. tɕe nɯ tɤrɕɤz nɯ́-wɣ-rku nɯ-rom tɕe, wuma ʑo nɤrko, ɯ-ŋgɯ ku-kɯ-rɤʑi tɕe mpja. tsuku kɯ ɯ-pɕi tɤrcoʁ ʑala tu-lɤt-nɯ pjɤ-ŋu tɕe, nɯ kɯ-fse nɯ mɤʑɯ ʑo mpja.\cmn 柳条墙是过去那些没钱改木板装板壁的人家用来隔房间的。他们找来比较细的、均匀的柳枝条,在柱子之间装上横干,(把枝条)编在上面,就成了柳条墙。要趁柳条没干的时候(编),因为干了就变硬,不能编。柳条墙装了以后,干了,就比较坚固,住在里面暖和。有的人在外面糊上细泥巴,这样更暖和。\end{exemple}\end{entrée}

\begin{entrée}
\vedette{\hypertarget{Ⓔtɤ-re}{\papi{ tɤ-re}}}\markboth{tɤ-re}{}
\classe{np}
\begin{définition}\fra rire\end{définition}
\begin{définition}\cmn 笑\end{définition}
\begin{exemple}\jya a-re ma-tɯ-tɕɤt\cmn 你不要让我耻笑你(你小看我了,我不是那种人)\end{exemple}
\begin{exemple}\jya tɤ-re sɤ-tɕɯ-tɕɤt\cmn 当笑话\end{exemple}
\begin{exemple}\jya tɤ-re sɤ-tɕɯ-tɕɤt ma-tɤ-kɯ-sɯβzu-a\cmn 你不要取笑我\end{exemple}
\begin{relation-sémantique}\confer{
\hyperlink{Ⓔtɤre tɤɟaʁ}{\textit{ \papi{tɤre tɤɟaʁ}}}
}\end{relation-sémantique}
\begin{relation-sémantique}\confer{
 \papi{nɤre}
}\end{relation-sémantique}
\begin{relation-sémantique}\confer{
\hyperlink{Ⓔsɤre}{\textit{ \papi{sɤre}}}
}\end{relation-sémantique}\end{entrée}

\begin{entrée}
\vedette{\hypertarget{Ⓔtɤresɤpɯpa}{\papi{ tɤresɤpɯpa}}}\markboth{tɤresɤpɯpa}{}\classe{n}
\begin{définition}\fra moquerie\end{définition}
\begin{définition}\cmn 取笑人\end{définition}
\begin{exemple}\jya tɤresɤpɯpa ma-tɤ-kɯ-sɯβzu-a\cmn 你不要嘲笑我\end{exemple}
\begin{exemple}\jya tɤresɤpɯpa ta-βzu\cmn 他取笑了他\end{exemple}
\begin{relation-sémantique}\confer{
\hyperlink{Ⓔtɤ-re}{\textit{ \papi{tɤ-re}}}
}\end{relation-sémantique}\end{entrée}

\begin{entrée}
\vedette{\hypertarget{Ⓔtɤre tɤɟaʁ}{\papi{ tɤre tɤɟaʁ}}}\markboth{tɤre tɤɟaʁ}{}
\classe{n}
\begin{définition}\fra plaisanteries\end{définition}
\begin{définition}\cmn 说说笑笑\end{définition}
\begin{relation-sémantique}\confer{
\hyperlink{Ⓔnɤrɤɟaʁ}{\textit{ \papi{nɤrɤɟaʁ}}}
}\end{relation-sémantique}\end{entrée}

\begin{entrée}
\vedette{\hypertarget{Ⓔtɤrga}{\papi{ tɤrga}}}\markboth{tɤrga}{}\classe{n}
\paradigme{\textit{emphatic :} \jya tɤrga tɤχi}
\paradigme{\textit{emphatic :} \jya tɤrga tɤle}
\begin{définition}\fra bonheur\end{définition}
\begin{définition}\cmn 幸福\end{définition}
\begin{exemple}\jya tɤrga tɤχi kɯ ku-rɤʑi-a\cmn 我非常幸福\end{exemple}
\begin{exemple}\jya tɤrga tɤle kɯ jɤ-nɯɣe-a\cmn 我兴高采烈地回家了\end{exemple}\end{entrée}

\begin{entrée}
\vedette{\hypertarget{Ⓔtɤ-rɣa}{\papi{ tɤ-rɣa}}}\markboth{tɤ-rɣa}{}
\classe{np}
\begin{définition}\fra voisin\end{définition}
\begin{définition}\cmn 邻居\end{définition}
\begin{exemple}\jya a-rɣa\cmn 我的邻居\end{exemple}
\begin{relation-sémantique}\synonyme{
\hyperlink{Ⓔjɯlco}{\textit{ \papi{jɯlco}}}
}\end{relation-sémantique}
\begin{relation-sémantique}\confer{
\hyperlink{Ⓔandʑɯrɣa}{\textit{ \papi{andʑɯrɣa}}}
}\end{relation-sémantique}\end{entrée}

\begin{entrée}
\vedette{\hypertarget{Ⓔtɤ-rɣe}{\papi{ tɤ-rɣe}}}\markboth{tɤ-rɣe}{}
\classe{np}
\begin{définition}\fra perle\end{définition}
\begin{définition}\cmn 珍珠\end{définition}
\begin{exemple}\jya a-rɣe\cmn 我的珍珠\end{exemple}\end{entrée}

\begin{entrée}
\vedette{\hypertarget{Ⓔtɤ-ri}{\papi{ tɤ-ri}}}\markboth{tɤ-ri}{}
\classe{np}
\begin{définition}\fra fil\end{définition}
\begin{définition}\cmn 线\end{définition}
\begin{exemple}\jya tɤ-ri nɯ-sɤβzu-t-a\cmn 我把毛搓成了线\end{exemple}
\begin{exemple}\jya nɤ-xtsa ɯ-ri nɯ-βzu-t-a\cmn 我给你做了鞋带\end{exemple}
\begin{relation-sémantique}\confer{
\hyperlink{Ⓔɣɯri}{\textit{ \papi{ɣɯri}}}
}\end{relation-sémantique}
\begin{relation-sémantique}\confer{
\hyperlink{Ⓔsmɤɣri}{\textit{ \papi{smɤɣri}}}
}\end{relation-sémantique}
\begin{relation-sémantique}\confer{
\hyperlink{Ⓔrazri}{\textit{ \papi{razri}}}
}\end{relation-sémantique}\end{entrée}

\begin{entrée}
\vedette{\hypertarget{Ⓔtɤ-rɟit}{\papi{ tɤ-rɟit}}}\markboth{tɤ-rɟit}{}\classe{np}
\begin{définition}\fra enfant\end{définition}
\begin{définition}\cmn 孩子
\begin{déclaration} \étymologie{\papi{rgʲud}}\end{déclaration}\end{définition}
\begin{relation-sémantique}\confer{
\hyperlink{Ⓔrɤrɟit}{\textit{ \papi{rɤrɟit}}}
}\end{relation-sémantique}\end{entrée}

\begin{entrée}
\vedette{\hypertarget{Ⓔtɤrɟɯsti}{\papi{ tɤrɟɯsti}}}\markboth{tɤrɟɯsti}{}\classe{n}
\begin{définition}\fra enfant unique\end{définition}
\begin{définition}\cmn 独生子\end{définition}
\begin{relation-sémantique}\confer{
\hyperlink{Ⓔtɤ-rɟit}{\textit{ \papi{tɤ-rɟit}}}
}\end{relation-sémantique}\end{entrée}

\begin{entrée}
\vedette{\hypertarget{ⒺtɤrkaⒽ1}{\papi{ tɤrka}}}\markboth{tɤrka}{}\homonyme{1}
\classe{n}
\begin{définition}\fra mule\end{définition}
\begin{définition}\cmn 骡子\end{définition}\end{entrée}

\begin{entrée}
\vedette{\hypertarget{ⒺtɤrkaⒽ2}{\papi{ tɤrka}}}\markboth{tɤrka}{}\homonyme{2}\classe{n}
\begin{définition}\fra jumeaux\end{définition}
\begin{définition}\cmn 双胞胎\end{définition}\end{entrée}

\begin{entrée}
\vedette{\hypertarget{Ⓔtɤrkakɕi}{\papi{ tɤrkakɕi}}}\markboth{tɤrkakɕi}{}\classe{n}
\begin{définition}\fra chien de berger\end{définition}
\begin{définition}\cmn 牧羊犬
\begin{déclaration} \étymologie{\papi{kʰʲi}}\end{déclaration}\end{définition}\end{entrée}

\begin{entrée}
\vedette{\hypertarget{Ⓔtɤ-rkhɤrkhɤt}{\papi{ tɤ-rkhɤrkhɤt}}}\markboth{tɤ-rkhɤrkhɤt}{}
\classe{np}
\begin{définition}\fra chemin de montagne en pierre avec des marches\end{définition}
\begin{définition}\cmn 用石板铺成的山路\end{définition}
\begin{exemple}\jya cupa-rkhɤrkhɤt\cmn 石板山路\end{exemple}\end{entrée}

\begin{entrée}
\vedette{\hypertarget{Ⓔtɤrkhɤz}{\papi{ tɤrkhɤz}}}\markboth{tɤrkhɤz}{}
\classe{n}
\begin{définition}\fra crasse qui s'accumule lorsqu'on ne se lave pas pendant longtemps\end{définition}
\begin{définition}\cmn 长期不洗而积累下来的污垢\end{définition}\end{entrée}

\begin{entrée}
\vedette{\hypertarget{Ⓔtɤ-rkhom}{\papi{ tɤ-rkhom}}}\markboth{tɤ-rkhom}{}
\classe{np}
\begin{définition}\fra rachis (plume)\end{définition}
\begin{définition}\cmn 羽干\end{définition}\end{entrée}

\begin{entrée}
\vedette{\hypertarget{Ⓔtɤrkopa}{\papi{ tɤrkopa}}}\markboth{tɤrkopa}{}\classe{n}
\begin{définition}\fra forcer\end{définition}
\begin{définition}\cmn 迫使\end{définition}
\begin{exemple}\jya ɯ-tɕɯ tɤrkopa ʑo jo-sɯxɕe ɕti ma ɯʑo kɯ pjɤ-nɤla pjɤ-maʁ\cmn 他是强迫儿子去的,他儿子不是自愿的\end{exemple}
\begin{exemple}\jya ɯ-tɕɯ tɤrkopa ʑo chɤ-sɯɕkɯt\cmn 他强迫儿子把饭吃完了\end{exemple}
\begin{relation-sémantique}\synonyme{
\hyperlink{Ⓔmɤkɯftshi}{\textit{ \papi{mɤkɯftshi}}}
}\end{relation-sémantique}\end{entrée}

\begin{entrée}
\vedette{\hypertarget{Ⓔtɤrkoz}{\papi{ tɤrkoz}}}\markboth{tɤrkoz}{}\classe{n}
\begin{définition}\fra exprès, de force\end{définition}
\begin{définition}\cmn 故意,强迫\end{définition}
\begin{exemple}\jya kɤ-ndza a-ʁjiz mɯ́j-ɣi ri ɯʑo kɯ tɤrkoz thɯ́-wɣ-sɯ-ndza-a\cmn 虽然我没饿,但他强迫我吃\end{exemple}
\begin{exemple}\jya tɤrkoz tɤ-ndza-t-a pɯ-ra\cmn 我被迫吃了\end{exemple}
\begin{exemple}\jya ɯʑo kɯ tɤrkoz ta-lɤt ɕti\cmn 是他故意打的\end{exemple}
\begin{exemple}\jya ɯ-jaʁ tɤrcoʁ kɯ-tu nɯ, a-ŋga ɯ-taʁ tɤrkoz na-mar/na-sɤtɕaʁ\cmn 他把手上的泥巴故意擦在了我衣服上\end{exemple}
\begin{relation-sémantique}\synonyme{
\hyperlink{Ⓔmɤkɯftshi}{\textit{ \papi{mɤkɯftshi}}}
}\end{relation-sémantique}\end{entrée}

\begin{entrée}
\vedette{\hypertarget{Ⓔtɤrkɯ}{\papi{ tɤrkɯ}}}\markboth{tɤrkɯ}{}\classe{n}
\begin{définition}\fra support pour les seaux d'eau que l'on porte sur le dos\end{définition}
\begin{définition}\cmn 背水的时候,用来垫水桶底子的圆圈\end{définition}
\begin{relation-sémantique}\confer{
\hyperlink{Ⓔaɣɯrkɯrkɯ}{\textit{ \papi{aɣɯrkɯrkɯ}}}
}\end{relation-sémantique}\end{entrée}

\begin{entrée}
\vedette{\hypertarget{Ⓔtɤ-rkuz}{\papi{ tɤ-rkuz}}}\markboth{tɤ-rkuz}{}
\classe{np}
\begin{définition}\fra cadeau\end{définition}
\begin{définition}\cmn 礼物(临走之前给的)\end{définition}
\begin{exemple}\jya aʑo tɤ-rɤŋga-t-a, tɕe a-me kɯ a-rkuz rŋɯl ta-rku (ta-βzu)\cmn 我临走之前,我女儿给了我一点钱\end{exemple}\end{entrée}

\begin{entrée}
\vedette{\hypertarget{Ⓔtɤrmbɣo}{\papi{ tɤrmbɣo}}}\markboth{tɤrmbɣo}{}
\classe{n}
\begin{définition}\fra tambour\end{définition}
\begin{définition}\cmn 鼓\end{définition}\end{entrée}

\begin{entrée}
\vedette{\hypertarget{Ⓔtɤrmbja}{\papi{ tɤrmbja}}}\markboth{tɤrmbja}{}
\classe{n}
\begin{définition}\fra éclair\end{définition}
\begin{définition}\cmn 闪电\end{définition}
\begin{exemple}\jya ftɕar tɕe tɯ-mɯ lɤt tɯ-kha tɕe tɤrmbja tu-βze ŋgrɤl, ɯ-mphru tɕe mbɣɯrloʁ tu-βze ŋu\cmn 夏天下雨的时候经常会出现闪电,然后紧接着就会打雷\end{exemple}
\begin{exemple}\jya tɤrmbja ɲɯ-ɤsɯ-βzu\cmn 在闪电\end{exemple}\end{entrée}

\begin{entrée}
\vedette{\hypertarget{Ⓔtɤrmbjajmɤɣ}{\papi{ tɤrmbjajmɤɣ}}}\markboth{tɤrmbjajmɤɣ}{}
\classe{n}
\begin{définition}\fra une espèce de champignon\end{définition}
\begin{définition}\cmn 一种蘑菇\end{définition}
\begin{exemple}\jya tɤrmbjajmɤɣ to-ɬoʁ\cmn 蓝菌长出来了\end{exemple}
\begin{relation-sémantique}\synonyme{
\hyperlink{Ⓔkɤrŋijmɤɣ}{\textit{ \papi{kɤrŋijmɤɣ}}}
}\end{relation-sémantique}\end{entrée}

\begin{entrée}
\vedette{\hypertarget{Ⓔtɤrmbjɤβ}{\papi{ tɤrmbjɤβ}}}\markboth{tɤrmbjɤβ}{}
\classe{n}
\begin{définition}\fra blé en botte\end{définition}
\begin{définition}\cmn 捆成一把的麦杆\end{définition}\end{entrée}

\begin{entrée}
\vedette{\hypertarget{Ⓔtɤ-rme}{\papi{ tɤ-rme}}}\markboth{tɤ-rme}{}
\classe{np}
\paradigme{\textit{comit :} \jya kɤ́rmɯrme}
\begin{définition}\fra poils\end{définition}
\begin{définition}\cmn 毛\end{définition}
\begin{relation-sémantique}\confer{
\hyperlink{Ⓔaɣɯrme}{\textit{ \papi{aɣɯrme}}}
}\end{relation-sémantique}\end{entrée}

\begin{entrée}
\vedette{\hypertarget{Ⓔtɤ-rmi}{\papi{ tɤ-rmi}}}\markboth{tɤ-rmi}{}\classe{np}
\begin{définition}\fra name\end{définition}
\begin{définition}\cmn 名字\end{définition}
\begin{exemple}\jya ɯ-rmi tɤ-tɕɤt-i (=tɤ-sɤrmi-j)\cmn 我们给他取了名字\end{exemple}
\begin{exemple}\jya a-tɤ-rmi pɯ-rɤt\cmn 你给我写名单\end{exemple}
\begin{exemple}\jya ɯ-rmi ɲɯ-ɬoʁ\cmn 很出名\end{exemple}
\begin{relation-sémantique}\confer{
\hyperlink{Ⓔrmi}{\textit{ \papi{rmi}}}
}\end{relation-sémantique}
\begin{relation-sémantique}\confer{
\hyperlink{Ⓔsɤrmi}{\textit{ \papi{sɤrmi}}}
}\end{relation-sémantique}\end{entrée}

\begin{entrée}
\vedette{\hypertarget{Ⓔtɤrmɯɣlu}{\papi{ tɤrmɯɣlu}}}\markboth{tɤrmɯɣlu}{}\classe{n}
\begin{définition}\fra année du dragon\end{définition}
\begin{définition}\cmn 龙年\end{définition}
\end{entrée}

\begin{entrée}
\vedette{\hypertarget{Ⓔtɤ-rmɯχtɕɤz}{\papi{ tɤ-rmɯχtɕɤz}}}\markboth{tɤ-rmɯχtɕɤz}{}\classe{np}
\begin{définition}\fra surnom\end{définition}
\begin{définition}\cmn 小名\end{définition}
\begin{relation-sémantique}\confer{
\hyperlink{Ⓔtɤ-rmi}{\textit{ \papi{tɤ-rmi}}}
}\end{relation-sémantique}\end{entrée}

\begin{entrée}
\vedette{\hypertarget{Ⓔtɤrɲɟo}{\papi{ tɤrɲɟo}}}\markboth{tɤrɲɟo}{}\classe{n}
\begin{définition}\fra étagère où l'on pose les outils de cuisine\end{définition}
\begin{définition}\cmn 厨架;放厨具的木板(钉在墙上)\end{définition}\end{entrée}

\begin{entrée}
\vedette{\hypertarget{Ⓔtɤ-rɴɢioʁ}{\papi{ tɤ-rɴɢioʁ}}}\markboth{tɤ-rɴɢioʁ}{}\classe{np}
\begin{définition}\fra invagination\end{définition}
\begin{définition}\cmn 槽\end{définition}
\begin{exemple}\jya tɤ-rɴɢioʁ thɯ-tɕat-a / thɯ-βzu-t-a\cmn 我挖了一条槽\end{exemple}
\begin{relation-sémantique}\confer{
\hyperlink{Ⓔtɤ-rqhioʁ}{\textit{ \papi{tɤ-rqhioʁ}}}
}\end{relation-sémantique}
\begin{relation-sémantique}\confer{
\hyperlink{Ⓔarɤrɴɢioʁ}{\textit{ \papi{arɤrɴɢioʁ}}}
}\end{relation-sémantique}\end{entrée}

\begin{entrée}
\vedette{\hypertarget{Ⓔtɤ-ro}{\papi{ tɤ-ro}}}\markboth{tɤ-ro}{}\classe{np}\acception{1}
\begin{définition}\fra en trop\end{définition}
\begin{définition}\cmn 多余的\end{définition}\acception{2}
\begin{définition}\fra reste\end{définition}
\begin{définition}\cmn 剩下的部分\end{définition}
\begin{exemple}\jya kɯki tɤ-ro tɕe, nɯ ma mɯ́j-ra\cmn 太多了,不需要了\end{exemple}
\begin{exemple}\jya tɯ-tɣa ro ro kɯ-rɲɟi\cmn 一拃多一点\end{exemple}
\begin{exemple}\jya ki aʑo a-ro ŋu tɕe, ɯ-tɯ-ndze\cmn 这是我吃剩的,你吃不吃?\end{exemple}
\begin{relation-sémantique}\confer{
\hyperlink{Ⓔɯ-rozre}{\textit{ \papi{ɯ-rozre}}}
}\end{relation-sémantique}\end{entrée}

\begin{entrée}
\vedette{\hypertarget{Ⓔtɤrpat}{\papi{ tɤrpat}}}\markboth{tɤrpat}{}
\classe{n}
\begin{définition}\fra suie sur le plafond\end{définition}
\begin{définition}\cmn 沾在天花板上的烟黑【烟层】\end{définition}
\begin{exemple}\jya tɤrpat ɯ-mdoʁ\cmn 咖啡色\end{exemple}\end{entrée}

\begin{entrée}
\vedette{\hypertarget{Ⓔtɤ-rpi}{\papi{ tɤ-rpi}}}\markboth{tɤ-rpi}{}
\classe{np}
\begin{définition}\fra soutra\end{définition}
\begin{définition}\cmn (诵)经\end{définition}
\begin{exemple}\jya tɤ-rpi wuma ʑo kɯ-wxti ɲɤ-sɯ-βzu-nɯ pjɤ-ra\cmn 只好请(喇嘛)诵经\end{exemple}\end{entrée}

\begin{entrée}
\vedette{\hypertarget{Ⓔtɤ-rpɯ}{\papi{ tɤ-rpɯ}}}\markboth{tɤ-rpɯ}{}
\classe{np}
\begin{définition}\fra oncle (frère de la mère et ses fils)\end{définition}
\begin{définition}\cmn 舅舅;舅舅的儿子\end{définition}
\begin{exemple}\jya a-rpɯ\cmn 我的舅舅\end{exemple}\end{entrée}

\begin{entrée}
\vedette{\hypertarget{Ⓔtɤ-rqhu}{\papi{ tɤ-rqhu}}}\markboth{tɤ-rqhu}{}
\classe{np}\acception{1}
\begin{définition}\fra enveloppe, coquille, carapace\end{définition}
\begin{définition}\cmn 壳\end{définition}\acception{2}
\begin{définition}\fra écorce\end{définition}
\begin{définition}\cmn 树皮\end{définition}\end{entrée}

\begin{entrée}
\vedette{\hypertarget{Ⓔtɤ-rqhioʁ}{\papi{ tɤ-rqhioʁ}}}\markboth{tɤ-rqhioʁ}{}\classe{np}
\begin{définition}\fra invagination, entaille\end{définition}
\begin{définition}\cmn 槽\end{définition}
\begin{relation-sémantique}\confer{
\hyperlink{Ⓔtɤ-rɴɢioʁ}{\textit{ \papi{tɤ-rɴɢioʁ}}}
}\end{relation-sémantique}\end{entrée}

\begin{entrée}
\vedette{\hypertarget{Ⓔtɤrʁaʁ}{\papi{ tɤrʁaʁ}}}\markboth{tɤrʁaʁ}{}\classe{n}
\begin{définition}\fra proie\end{définition}
\begin{définition}\cmn 猎物\end{définition}
\begin{relation-sémantique}\confer{
\hyperlink{Ⓔɣɤrʁaʁ}{\textit{ \papi{ɣɤrʁaʁ}}}
}\end{relation-sémantique}
\begin{relation-sémantique}\confer{
\hyperlink{Ⓔnɤrʁaʁ}{\textit{ \papi{nɤrʁaʁ}}}
}\end{relation-sémantique}
\end{entrée}

\begin{entrée}
\vedette{\hypertarget{Ⓔtɤrʁaʁɕa}{\papi{ tɤrʁaʁɕa}}}\markboth{tɤrʁaʁɕa}{}\classe{n}
\begin{définition}\fra viande issue de la chasse\end{définition}
\begin{définition}\cmn 猎物的肉\end{définition}\end{entrée}

\begin{entrée}
\vedette{\hypertarget{Ⓔtɤrʁaʁkɕi}{\papi{ tɤrʁaʁkɕi}}}\markboth{tɤrʁaʁkɕi}{}
\classe{n}
\begin{définition}\fra chien de chasse\end{définition}
\begin{définition}\cmn 猎狗\end{définition}\end{entrée}

\begin{entrée}
\vedette{\hypertarget{Ⓔtɤ-rtaʁ}{\papi{ tɤ-rtaʁ}}}\markboth{tɤ-rtaʁ}{}
\classe{np}
\begin{définition}\fra branche\end{définition}
\begin{définition}\cmn 树杈\end{définition}
\begin{relation-sémantique}\confer{
\hyperlink{Ⓔartaʁ}{\textit{ \papi{artaʁ}}}
}\end{relation-sémantique}\end{entrée}

\begin{entrée}
\vedette{\hypertarget{Ⓔtɤ-rtɕhɣaʁ,tɕɤt}{\papi{ tɤ-rtɕhɣaʁ,tɕɤt}}}\markboth{tɤ-rtɕhɣaʁ,tɕɤt}{}
\begin{définition}\fra mettre des bâtons dans les roues, entraver\end{définition}
\begin{définition}\cmn 作梗\end{définition}
\begin{exemple}\jya ɯ-rtɕhɣaʁ ɲɤ-tɕɤt\cmn (工作本来很顺利),是他从中作梗\end{exemple}
\begin{relation-sémantique}\ComponentA{\classe{np}
 \papi{tɤ-rtɕhɣaʁ}
}\end{relation-sémantique}
\begin{relation-sémantique}\ComponentB{\classe{vt}
\hyperlink{Ⓔtɕɤt}{\textit{ \papi{tɕɤt}}}
}\end{relation-sémantique}
\end{entrée}

\begin{entrée}
\vedette{\hypertarget{Ⓔtɤ-rtɕi}{\papi{ tɤ-rtɕi}}}\markboth{tɤ-rtɕi}{}
\classe{np}
\begin{définition}\fra complément alimentaire\end{définition}
\begin{définition}\cmn 补品\end{définition}
\begin{exemple}\jya a-rtɕi kɤ-βzu ɲɯ-ra ma a-qhoχpa mɯ́j-sna\cmn 要补养身体,因为内脏不好\end{exemple}\end{entrée}

\begin{entrée}
\vedette{\hypertarget{Ⓔtɤ-rte}{\papi{ tɤ-rte}}}\markboth{tɤ-rte}{}
\classe{np}
\paradigme{\textit{comit :} \jya kɤ́rtɯrte}
\begin{définition}\fra coiffe, chapeau\end{définition}
\begin{définition}\cmn 头帕;帽子\end{définition}
\begin{exemple}\jya a-rte\cmn 我的帽子\end{exemple}
\begin{exemple}\jya nɤ-rte ma-nɯ-tɯ-nɯ-βde\cmn 你不要把帽子弄丢了\end{exemple}
\begin{relation-sémantique}\confer{
\hyperlink{Ⓔnɤrte}{\textit{ \papi{nɤrte}}}
}\end{relation-sémantique}\end{entrée}

\begin{entrée}
\vedette{\hypertarget{Ⓔtɤrtoʁlu}{\papi{ tɤrtoʁlu}}}\markboth{tɤrtoʁlu}{}
\classe{n}
\begin{définition}\fra colostrum\end{définition}
\begin{définition}\cmn 初乳,母牛下了牛犊之后第一次挤的奶\end{définition}
\begin{exemple}\jya nɯŋa ɲo-ɬoʁ tɕe ɯ-tɤrtoʁlu pɯ-arɕo pɯ-tsu\cmn 奶牛生了仔,但是初乳的阶段已经过了\end{exemple}\end{entrée}

\begin{entrée}
\vedette{\hypertarget{Ⓔtɤrtsa}{\papi{ tɤrtsa}}}\markboth{tɤrtsa}{}
\classe{n}
\begin{définition}\fra vague\end{définition}
\begin{définition}\cmn 波浪;波纹\end{définition}
\begin{exemple}\jya tɤrtsa to-βzu\cmn 起了波浪\end{exemple}\end{entrée}

\begin{entrée}
\vedette{\hypertarget{Ⓔtɤ-rtsɤɣ}{\papi{ tɤ-rtsɤɣ}}}\markboth{tɤ-rtsɤɣ}{}\classe{clf}
\begin{définition}\fra un étage\end{définition}
\begin{définition}\cmn 一层楼
\begin{déclaration} \étymologie{\papi{rtseg}}\end{déclaration}\end{définition}
\begin{exemple}\jya χsɤ-rtsɤɣ\cmn 三层楼\end{exemple}
\begin{exemple}\jya ki kha ki ɯ-tɤ-rtsɤɣ ɲɯ-mbro\cmn 这个房子(每一)层楼都很高\end{exemple}\end{entrée}

\begin{entrée}
\vedette{\hypertarget{Ⓔtɤ-rtsho}{\papi{ tɤ-rtsho}}}\markboth{tɤ-rtsho}{}\classe{np}\acception{1}
\begin{définition}\fra surface de la partie coupée\end{définition}
\begin{définition}\cmn 锯过;砍过;剪过的口子\end{définition}\acception{2}
\begin{définition}\fra éteule (de blé)\end{définition}
\begin{définition}\cmn (麦)桩、(麦)茬\end{définition}\end{entrée}

\begin{entrée}
\vedette{\hypertarget{Ⓔtɤ-rtshom}{\papi{ tɤ-rtshom}}}\markboth{tɤ-rtshom}{}\classe{np}
\begin{définition}\fra bruit\end{définition}
\begin{définition}\cmn 噪音;声音\end{définition}
\begin{exemple}\jya tɯrme ɲɯ-nɤŋkɯŋke ma ɯ-rtshom ɣɤʑu\cmn 有人在走来走去,(听得到)声音\end{exemple}\end{entrée}

\begin{entrée}
\vedette{\hypertarget{Ⓔtɤ-rtsi}{\papi{ tɤ-rtsi}}}\markboth{tɤ-rtsi}{}\classe{np}
\begin{définition}\fra huile de porc\end{définition}
\begin{définition}\cmn 猪油\begin{déclaration} \étymologie{\papi{rtsi}}\end{déclaration}\end{définition}
\begin{relation-sémantique}\confer{
\hyperlink{Ⓔaɣɯrtsi}{\textit{ \papi{aɣɯrtsi}}}
}\end{relation-sémantique}\end{entrée}

\begin{entrée}
\vedette{\hypertarget{Ⓔtɤ-rtsɯz}{\papi{ tɤ-rtsɯz}}}\markboth{tɤ-rtsɯz}{}\classe{np}
\begin{définition}\fra nombre, chiffre\end{définition}
\begin{définition}\cmn 数目(计算的结果)
\begin{déclaration} \étymologie{\papi{rtsis}}\end{déclaration}\end{définition}
\begin{exemple}\jya ɯ-rtsɯz ko-ndo (=ɯ-χsɤr ko-ndo)\cmn 他记下了数字\end{exemple}
\begin{relation-sémantique}\confer{
\hyperlink{Ⓔrtsi}{\textit{ \papi{rtsi}}}
}\end{relation-sémantique}\end{entrée}

\begin{entrée}
\vedette{\hypertarget{Ⓔtɤ-rʑaβ}{\papi{ tɤ-rʑaβ}}}\markboth{tɤ-rʑaβ}{}
\classe{np}
\begin{définition}\fra épouse\end{définition}
\begin{définition}\cmn 妻子\end{définition}
\begin{exemple}\jya a-rʑaβ\cmn 我的妻子\end{exemple}\end{entrée}

\begin{entrée}
\vedette{\hypertarget{Ⓔtɤ-rʑaʁⒽ1Ⓗ1}{\papi{ tɤ-rʑaʁ}}}\markboth{tɤ-rʑaʁ}{}\homonyme{1}\classe{clf}
\begin{définition}\fra une nuit\end{définition}
\begin{définition}\cmn 一夜\end{définition}\begin{sous-entrée}
\vedette{\hypertarget{}{\papi{ tɤ-rʑaʁ}}}\markboth{tɤ-rʑaʁ}{}\classe{np}
\begin{définition}\fra temps\end{définition}
\begin{définition}\cmn 时间\end{définition}
\begin{exemple}\jya tɤ-rʑaʁ tɤ-rɲɟi tɕe, mɤ-saχsɤl\cmn 时间长了就会不清楚\end{exemple}
\begin{exemple}\jya ɯ-rʑaʁ ɲɯ-zri / ɯ-rʑaʁ mɯ́j-zri\cmn 他很无聊/他不无聊\end{exemple}
\begin{exemple}\jya a-rʑaʁ mɯ́j-ɕe\cmn 我很无聊\end{exemple}
\begin{exemple}\jya tɤ-rʑaʁ kɯmɤlɤxso a-mɤ-nɯ-ɕe ma nɤja\cmn 不要浪费时间,因为可惜\end{exemple}
\begin{exemple}\jya tɤ-rʑaʁ nɯ pɤrmɤloŋ ɲɯ-ɕe mɤ-pe\cmn 浪费时间是不好的\end{exemple}
\begin{exemple}\jya nɤ-rʑaʁ nɯfse ʑo a-mɤ-nɯ-tɯ-nɯɕe\cmn 你不要白白浪费时间\end{exemple}
\begin{exemple}\jya tɤ-rʑaʁ ɯ-taʁ nɯ tɕu ju-kɯ-zɣɯt ra\cmn 要在规定的时间到达\end{exemple}
\end{sous-entrée}\end{entrée}

\begin{entrée}
\vedette{\hypertarget{Ⓔtɤ-rʑɯɣ}{\papi{ tɤ-rʑɯɣ}}}\markboth{tɤ-rʑɯɣ}{}
\classe{np}
\begin{définition}\fra ride\end{définition}
\begin{définition}\cmn 皱纹\end{définition}\end{entrée}

\begin{entrée}
\vedette{\hypertarget{Ⓔtɤ-ʁamɟa}{\papi{ tɤ-ʁamɟa}}}\markboth{tɤ-ʁamɟa}{}
\classe{np}
\begin{définition}\fra retard\end{définition}
\begin{définition}\cmn 耽误\end{définition}
\begin{exemple}\jya a-ʁamɟa pɯ-tu\cmn 耽误了我的时间\end{exemple}
\begin{exemple}\jya nɤ-ʁamɟa pɯ-sɤβzu-t-a\cmn 我耽误了你的时间\end{exemple}
\begin{exemple}\jya nɤ-ʁamɟa ɣɤʑu\cmn 耽误了你的时间\end{exemple}
\begin{relation-sémantique}\confer{
\hyperlink{Ⓔznɤʁamɟa}{\textit{ \papi{znɤʁamɟa}}}
}\end{relation-sémantique}\end{entrée}

\begin{entrée}
\vedette{\hypertarget{Ⓔtɤ-ʁar}{\papi{ tɤ-ʁar}}}\markboth{tɤ-ʁar}{}
\classe{np}\acception{1}
\begin{définition}\fra ailes\end{définition}
\begin{définition}\cmn 翅膀\end{définition}\acception{2}
\begin{définition}\fra longueur d'un bras\end{définition}
\begin{définition}\cmn 人的一只手那么长\end{définition}\end{entrée}

\begin{entrée}
\vedette{\hypertarget{Ⓔtɤ-ʁarndzom}{\papi{ tɤ-ʁarndzom}}}\markboth{tɤ-ʁarndzom}{}
\classe{np}
\begin{définition}\fra os des ailes\end{définition}
\begin{définition}\cmn 翅膀的骨头
\end{définition}\end{entrée}

\begin{entrée}
\vedette{\hypertarget{Ⓔtɤʁaʁ}{\papi{ tɤʁaʁ}}}\markboth{tɤʁaʁ}{}
\classe{n}
\begin{définition}\fra fête, réunion\end{définition}
\begin{définition}\cmn 聚会\end{définition}
\begin{exemple}\jya tɤʁaʁ ɲɯ-sɤscit\cmn 聚会很开心\end{exemple}
\begin{relation-sémantique}\confer{
\hyperlink{Ⓔnɤʁaʁ}{\textit{ \papi{nɤʁaʁ}}}
}\end{relation-sémantique}
\begin{relation-sémantique}\confer{
 \papi{sɤʁaʁ}
}\end{relation-sémantique}\end{entrée}

\begin{entrée}
\vedette{\hypertarget{Ⓔtɤ-ʁdɤn}{\papi{ tɤ-ʁdɤn}}}\markboth{tɤ-ʁdɤn}{}
\classe{np}
\begin{définition}\fra coussin\end{définition}
\begin{définition}\cmn 垫子
\begin{déclaration} \étymologie{\papi{gdan}}\end{déclaration}\end{définition}
\begin{relation-sémantique}\confer{
\hyperlink{Ⓔnɤʁdɤn}{\textit{ \papi{nɤʁdɤn}}}
}\end{relation-sémantique}\end{entrée}

\begin{entrée}
\vedette{\hypertarget{Ⓔtɤ-ʁjar}{\papi{ tɤ-ʁjar}}}\markboth{tɤ-ʁjar}{}\classe{np}
\begin{définition}\fra fils de chaîne\end{définition}
\begin{définition}\cmn 经线\end{définition}
\begin{exemple}\jya kɤ-taʁ chɯ́-wɣ-βzu tɕe tɤ-ri lo-thi lu-kɯ-ɕe nɯ ɯ-ʁjar ŋu tɤ-ri ku-ndi ku-kɤ-lɤt nɯ ɯ-jlɤβ ŋu\cmn 织布时,上下竖着的线叫经线,左右穿过去的线叫纬线。\end{exemple}
\begin{relation-sémantique}\antonyme{
\hyperlink{Ⓔtɯ-jlɤβ}{\textit{ \papi{tɯ-jlɤβ}}}
}\end{relation-sémantique}\end{entrée}

\begin{entrée}
\vedette{\hypertarget{Ⓔtɤ-ʁjoʁ}{\papi{ tɤ-ʁjoʁ}}}\markboth{tɤ-ʁjoʁ}{}
\begin{relation-sémantique}\confer{
\hyperlink{Ⓔʁjoʁ}{\textit{ \papi{ʁjoʁ}}}
}\end{relation-sémantique}\end{entrée}

\begin{entrée}
\vedette{\hypertarget{Ⓔtɤ-ʁlapaʁtsa}{\papi{ tɤ-ʁlapaʁtsa}}}\markboth{tɤ-ʁlapaʁtsa}{}
\classe{np}
\begin{définition}\fra arrière-bras\end{définition}
\begin{définition}\cmn 胳膊\end{définition}\end{entrée}

\begin{entrée}
\vedette{\hypertarget{Ⓔtɤsapɣɤtɕɯ}{\papi{ tɤsapɣɤtɕɯ}}}\markboth{tɤsapɣɤtɕɯ}{}\classe{n}
\begin{définition}\fra parus sp.\end{définition}
\begin{définition}\cmn 山雀\end{définition}
\end{entrée}

\begin{entrée}
\vedette{\hypertarget{Ⓔtɤsɤɣ}{\papi{ tɤsɤɣ}}}\markboth{tɤsɤɣ}{}
\classe{n}
\begin{définition}\fra amant\end{définition}
\begin{définition}\cmn 情夫\end{définition}
\begin{exemple}\jya aʑo a-tɤsɤɣ me\cmn 我没有情夫\end{exemple}
\begin{relation-sémantique}\confer{
\hyperlink{Ⓔnɤsɤɣ}{\textit{ \papi{nɤsɤɣ}}}
}\end{relation-sémantique}\end{entrée}

\begin{entrée}
\vedette{\hypertarget{Ⓔtɤsɤɣʑa}{\papi{ tɤsɤɣʑa}}}\markboth{tɤsɤɣʑa}{}\classe{n}
\begin{définition}\fra type de chanvre\end{définition}
\begin{définition}\cmn 大麻的一种\end{définition}
\begin{relation-sémantique}\confer{
\hyperlink{Ⓔtasa}{\textit{ \papi{tasa}}}
}\end{relation-sémantique}\end{entrée}

\begin{entrée}
\vedette{\hypertarget{Ⓔtɤsɤmu}{\papi{ tɤsɤmu}}}\markboth{tɤsɤmu}{}\classe{n}
\begin{définition}\fra type de chanvre\end{définition}
\begin{définition}\cmn 大麻的一种(能结种子的)\end{définition}
\begin{relation-sémantique}\confer{
\hyperlink{Ⓔtasa}{\textit{ \papi{tasa}}}
}\end{relation-sémantique}\end{entrée}

\begin{entrée}
\vedette{\hypertarget{Ⓔtɤsɤrŋu}{\papi{ tɤsɤrŋu}}}\markboth{tɤsɤrŋu}{}\classe{n}
\begin{définition}\fra grains de chanvre frits\end{définition}
\begin{définition}\cmn 炒的麻子\end{définition}
\begin{relation-sémantique}\confer{
\hyperlink{Ⓔrŋu}{\textit{ \papi{rŋu}}}
}\end{relation-sémantique}
\begin{relation-sémantique}\confer{
\hyperlink{Ⓔtasa}{\textit{ \papi{tasa}}}
}\end{relation-sémantique}\end{entrée}

\begin{entrée}
\vedette{\hypertarget{Ⓔtɤsɤsqɤri}{\papi{ tɤsɤsqɤri}}}\markboth{tɤsɤsqɤri}{}\classe{n}
\begin{définition}\fra fil de lin\end{définition}
\begin{définition}\cmn 麻线\end{définition}
\begin{relation-sémantique}\confer{
\hyperlink{Ⓔtasa}{\textit{ \papi{tasa}}}
}\end{relation-sémantique}\end{entrée}

\begin{entrée}
\vedette{\hypertarget{Ⓔtɤscɤr}{\papi{ tɤscɤr}}}\markboth{tɤscɤr}{}\classe{n}
\begin{définition}\fra frayeur\end{définition}
\begin{définition}\cmn 恐惧\end{définition}
\begin{exemple}\jya tɤ-scɤr kɯ nɯ-kɤpa ʑo ɲɤ-me\cmn 他们吓得不知所措\end{exemple}
\begin{relation-sémantique}\confer{
\hyperlink{Ⓔnɤscɤr}{\textit{ \papi{nɤscɤr}}}
}\end{relation-sémantique}\end{entrée}

\begin{entrée}
\vedette{\hypertarget{Ⓔtɤ-scoz}{\papi{ tɤ-scoz}}}\markboth{tɤ-scoz}{}\classe{np}
\begin{définition}\fra lettre\end{définition}
\begin{définition}\cmn 字;信\end{définition}
\begin{exemple}\jya @xiangbolin kɯ tɤ-scoz ɲɯ-ɤsɯ-rɤt\cmn 向柏霖在写字\end{exemple}
\begin{exemple}\jya a-jaʁ tɤ-scoz jɤ-ɣe\cmn 我收到了一封信\end{exemple}
\begin{exemple}\jya iʑo ji-rju nɯ ɯ-scoz maŋe tɕe mɯ́j-pe\cmn 我们的语言没有文字是不好的\end{exemple}
\begin{relation-sémantique}\confer{
\hyperlink{Ⓔrɤscoz}{\textit{ \papi{rɤscoz}}}
}\end{relation-sémantique}\end{entrée}

\begin{entrée}
\vedette{\hypertarget{Ⓔtɤ-se}{\papi{ tɤ-se}}}\markboth{tɤ-se}{}
\classe{np}
\begin{définition}\fra sang\end{définition}
\begin{définition}\cmn 血\end{définition}\end{entrée}

\begin{entrée}
\vedette{\hypertarget{Ⓔtɤsepu}{\papi{ tɤsepu}}}\markboth{tɤsepu}{}\classe{n}
\begin{définition}\fra boudin\end{définition}
\begin{définition}\cmn 血肠\end{définition}
\end{entrée}

\begin{entrée}
\vedette{\hypertarget{Ⓔtɤ-skrɤβ}{\papi{ tɤ-skrɤβ}}}\markboth{tɤ-skrɤβ}{}\classe{np}
\begin{définition}\fra fil très fin, cheveu\end{définition}
\begin{définition}\cmn 细线;头发\end{définition}
\begin{exemple}\jya @cai ɯ-ŋgɯ tɤ-skrɤβ ɣɤʑu\cmn 菜里面有头发\end{exemple}\end{entrée}

\begin{entrée}
\vedette{\hypertarget{Ⓔtɤ-sno}{\papi{ tɤ-sno}}}\markboth{tɤ-sno}{}
\classe{np}
\begin{définition}\fra selle\end{définition}
\begin{définition}\cmn 马鞍\end{définition}
\begin{exemple}\jya mbro ɯ-sno thɯ-ta-t-a\cmn 我给马装上了鞍子\end{exemple}\begin{sous-entrée}
\vedette{\hypertarget{}{\papi{ tɤ-sno ɯ-jaʁ}}}\markboth{tɤ-sno ɯ-jaʁ}{}\classe{n}
\begin{définition}\fra partie inférieure de la selle en contact avec le dos du cheval, faite en laine\end{définition}
\begin{définition}\cmn 马鞍垫;鞍鞯\end{définition}
\end{sous-entrée}\end{entrée}

\begin{entrée}
\vedette{\hypertarget{Ⓔtɤ-snom}{\papi{ tɤ-snom}}}\markboth{tɤ-snom}{}
\classe{np}
\begin{définition}\fra sœur (employé par les garçons)\end{définition}
\begin{définition}\cmn 姐姐;妹妹(男性专用)\end{définition}
\begin{exemple}\jya a-snom\cmn 我的姐姐\end{exemple}\end{entrée}

\begin{entrée}
\vedette{\hypertarget{Ⓔtɤ-sɲa}{\papi{ tɤ-sɲa}}}\markboth{tɤ-sɲa}{}
\classe{np}
\begin{définition}\fra tresse\end{définition}
\begin{définition}\cmn 辫子\end{définition}
\begin{exemple}\jya a-sɲa\cmn 我的辫子\end{exemple}
\begin{relation-sémantique}\synonyme{
\hyperlink{Ⓔtɤpjɤz}{\textit{ \papi{tɤpjɤz}}}
}\end{relation-sémantique}\end{entrée}

\begin{entrée}
\vedette{\hypertarget{Ⓔtɤsɲɤmtsɯ}{\papi{ tɤsɲɤmtsɯ}}}\markboth{tɤsɲɤmtsɯ}{}\classe{n}
\begin{définition}\fra broche\end{définition}
\begin{définition}\cmn 夹头发的装饰品\end{définition}
\end{entrée}

\begin{entrée}
\vedette{\hypertarget{Ⓔtɤ-sŋɯt}{\papi{ tɤ-sŋɯt}}}\markboth{tɤ-sŋɯt}{}\classe{np}
\begin{définition}\fra morsure\end{définition}
\begin{définition}\cmn (咬)一口\end{définition}
\begin{exemple}\jya a-jaʁ mɯ́j-so tɕe, a-sŋɯt kɯ kɤ-sɯ-ndo-t-a\cmn 因为我手上拿着东西,所以用牙齿咬住了\end{exemple}
\begin{exemple}\jya aʑo paχɕi ɯ-taʁ tɤ-sŋɯt tu-nɯ-lat-a ɲɯ-jɤɣ ma mbrɯtɕɯ mɯ́j-ra\cmn 我可以就这么咬苹果,不需要刀子\end{exemple}\end{entrée}

\begin{entrée}
\vedette{\hypertarget{Ⓔtɤ-spa}{\papi{ tɤ-spa}}}\markboth{tɤ-spa}{}
\classe{np}\acception{1}
\begin{définition}\fra matériau\end{définition}
\begin{définition}\cmn 材料\end{définition}\acception{2}
\begin{définition}\fra utilité, but\end{définition}
\begin{définition}\cmn 用途;目标\end{définition}\acception{3}
\begin{définition}\fra pour\end{définition}
\begin{définition}\cmn 用来……
\begin{déclaration}\use{用于目标从句时,从句的主动词带不定式\stylefv{kɤ-}前缀}\end{déclaration}\end{définition}
\begin{exemple}\jya khɯna nɯ kha kɯ-rɯru ɯ-spa ŋu\cmn 看家是狗的义务\end{exemple}
\begin{exemple}\jya nɯ tɕhi ɯ-spa ɲɯ-ŋu ?\cmn 那个有什么用呢?\end{exemple}
\begin{exemple}\jya tɤ-pɤtso kɤ-mbi ɯ-spa kɯ-chi to-χtɯ.\cmn 他买糖果给小孩子吃了\end{exemple}
\end{entrée}

\begin{entrée}
\vedette{\hypertarget{Ⓔtɤ-spɯ}{\papi{ tɤ-spɯ}}}\markboth{tɤ-spɯ}{}
\classe{np}
\begin{définition}\fra pus\end{définition}
\begin{définition}\cmn 脓\end{définition}
\begin{relation-sémantique}\confer{
\hyperlink{Ⓔrɤspɯ}{\textit{ \papi{rɤspɯ}}}
}\end{relation-sémantique}\end{entrée}

\begin{entrée}
\vedette{\hypertarget{Ⓔtɤspɯɕku}{\papi{ tɤspɯɕku}}}\markboth{tɤspɯɕku}{}\classe{n}
\begin{définition}\fra poireau sauvage\end{définition}
\begin{définition}\cmn 野韭菜\end{définition}
\begin{exemple}\jya tɤ-spɯ ɕku nɯ kha ɯ-mbe ɣɯ znde ku kɯ-fse nɯ ra tu-ɬoʁ rga, ɯ-mdoʁ pɣi, ɯ-jwaʁ kɤ-lɯ-lju ŋu, ɯ-ru me, tɯ-phɯ ɯ-ŋgɯ kɯ-dɯ-dɤn tu-ɬoʁ ɕti, ɯ-mɯntoʁ ndɯβ ri dɤn. ɯ-di nɯ kɯ-ɣɤjlu kɯ-fse tu. ɕku di mnɤm ri, kɤ-ndza mɤ-mɯm. ɯ-zrɤm dɤn.\cmn 
\stylefv{tɤ-spɯ ɕku} 一般生长在旧房子墙壁顶上,是灰色的,叶子是圆柱形的,没有茎,一秆里有很多根。花小而多。发出腥的味道。有点葱的味道,但不好吃。根很多。
\end{exemple}\end{entrée}

\begin{entrée}
\vedette{\hypertarget{Ⓔtɤ-sqhaj}{\papi{ tɤ-sqhaj}}}\markboth{tɤ-sqhaj}{}
\classe{np}
\begin{définition}\fra sœur (employé par les filles)\end{définition}
\begin{définition}\cmn 姐姐;妹妹(女性专用)\end{définition}
\begin{exemple}\jya a-sqhaj\cmn 我的姐姐\end{exemple}\end{entrée}

\begin{entrée}
\vedette{\hypertarget{Ⓔtɤstu}{\papi{ tɤstu}}}\markboth{tɤstu}{}
\classe{vi}
\begin{définition}\fra au revoir\end{définition}
\begin{définition}\cmn 再见\end{définition}
\begin{exemple}\jya kɯ-sɤfstɯn tɤ-stu je\cmn 再见,照顾人家\end{exemple}\end{entrée}

\begin{entrée}
\vedette{\hypertarget{Ⓔtɤ-sta}{\papi{ tɤ-sta}}}\markboth{tɤ-sta}{}\classe{np}
\begin{définition}\fra endroit où on va enterrer un mort\end{définition}
\begin{définition}\cmn 坟地\end{définition}
\begin{exemple}\jya tɤ-sta pjɯ́-wɣ-lɣa tɕe, tɯ-ɕpɤβ nɯ pjɯ́-wɣ-rku ŋu\cmn 挖了坟地,就把尸体装下去了\end{exemple}
\begin{relation-sémantique}\confer{
\hyperlink{Ⓔtɯ-sta}{\textit{ \papi{tɯ-sta}}}
}\end{relation-sémantique}
\begin{relation-sémantique}\confer{
\hyperlink{Ⓔɯ-sta}{\textit{ \papi{ɯ-sta}}}
}\end{relation-sémantique}\end{entrée}

\begin{entrée}
\vedette{\hypertarget{Ⓔtɤ-ste}{\papi{ tɤ-ste}}}\markboth{tɤ-ste}{}
\classe{np}
\begin{définition}\fra vessie\end{définition}
\begin{définition}\cmn 膀胱\end{définition}\end{entrée}

\begin{entrée}
\vedette{\hypertarget{Ⓔtɤsthoʁsi}{\papi{ tɤsthoʁsi}}}\markboth{tɤsthoʁsi}{}
\classe{n}\acception{1}
\begin{définition}\fra poutre\end{définition}
\begin{définition}\cmn 梁\end{définition}\acception{2}
\begin{définition}\fra poutre du balcon\end{définition}
\begin{définition}\cmn 走缘上的小梁【撑杆】\end{définition}
\begin{exemple}\jya jɤɣɤt ɯ-taʁ khɤxtu ɯ-pa stukɤr ɯ-tshɤt ɕoŋtɕa kɯ-xtshɯm chɯ-kɤ-lɤt nɯ tɤsthoʁsi rmi\cmn 
在走缘和房背之间代替大梁的细梁叫\stylefv{tɤsthoʁsi}
\end{exemple}
\begin{relation-sémantique}\synonyme{
\hyperlink{Ⓔɕɯjaʁ}{\textit{ \papi{ɕɯjaʁ}}}
}\end{relation-sémantique}\end{entrée}

\begin{entrée}
\vedette{\hypertarget{Ⓔtɤsto}{\papi{ tɤsto}}}\markboth{tɤsto}{}\classe{n}
\begin{définition}\fra grande jarre\end{définition}
\begin{définition}\cmn 大坛子\end{définition}\end{entrée}

\begin{entrée}
\vedette{\hypertarget{Ⓔtɤtar}{\papi{ tɤtar}}}\markboth{tɤtar}{}
\classe{n}
\begin{définition}\fra bâton fin\end{définition}
\begin{définition}\cmn 细木棍\end{définition}
\begin{relation-sémantique}\confer{
\hyperlink{Ⓔnɤtar}{\textit{ \papi{nɤtar}}}
}\end{relation-sémantique}
\begin{relation-sémantique}\confer{
\hyperlink{Ⓔjaχpɤtar}{\textit{ \papi{jaχpɤtar}}}
}\end{relation-sémantique}\end{entrée}

\begin{entrée}
\vedette{\hypertarget{Ⓔtɤtɤɣ}{\papi{ tɤtɤɣ}}}\markboth{tɤtɤɣ}{}
\classe{n}
\begin{définition}\fra armoire\end{définition}
\begin{définition}\cmn 柜子(装粮食)
\end{définition}\end{entrée}

\begin{entrée}
\vedette{\hypertarget{Ⓔtɤtɕɤfkɯm}{\papi{ tɤtɕɤfkɯm}}}\markboth{tɤtɕɤfkɯm}{}
\classe{n}
\begin{définition}\fra pommette\end{définition}
\begin{définition}\cmn 酒窝\end{définition}
\begin{relation-sémantique}\synonyme{
\hyperlink{Ⓔkhrambakɯm}{\textit{ \papi{khrambakɯm}}}
}\end{relation-sémantique}\end{entrée}

\begin{entrée}
\vedette{\hypertarget{Ⓔtɤtɕɤri}{\papi{ tɤtɕɤri}}}\markboth{tɤtɕɤri}{}\classe{n}
\begin{définition}\fra type de pas d'aiguille\end{définition}
\begin{définition}\cmn 大针脚的线\end{définition}
\begin{exemple}\jya kɯ-mɤku tɤtɕɤri pɯ-lat-a tɕe nɯ kóʁmɯz tɤ-ɕphɤt ɯ-ta thɯ-βzu-t-a\cmn 我先用针脚大一点的线把补丁固定了,然后就(用小针脚)把补丁做好了\end{exemple}\end{entrée}

\begin{entrée}
\vedette{\hypertarget{Ⓔtɤ-tɕɤz}{\papi{ tɤ-tɕɤz}}}\markboth{tɤ-tɕɤz}{}
\classe{np}
\begin{définition}\fra trace de pied\end{définition}
\begin{définition}\cmn 痕迹;足印\end{définition}
\begin{exemple}\jya a-tɕɤz\cmn 我的脚印\end{exemple}\end{entrée}

\begin{entrée}
\vedette{\hypertarget{Ⓔtɤ-tɕhɤz}{\papi{ tɤ-tɕhɤz}}}\markboth{tɤ-tɕhɤz}{}\classe{np}
\begin{définition}\fra franges colorées\end{définition}
\begin{définition}\cmn 吊边布(衣服边缘的彩色布料)\end{définition}
\begin{exemple}\jya ɯ-ŋga ɯ-tɕhɤz ɯ-tɯ-dɤn kɯ ɲɯ-ɣɤβlɯβlɯɣ ʑo\cmn 他衣服的彩色布料很多,显得很耀眼\end{exemple}\end{entrée}

\begin{entrée}
\vedette{\hypertarget{Ⓔtɤ-tɕɯ}{\papi{ tɤ-tɕɯ}}}\markboth{tɤ-tɕɯ}{}
\classe{np}\acception{1}
\begin{définition}\fra fils\end{définition}
\begin{définition}\cmn 儿子\end{définition}
\begin{exemple}\jya a-tɕɯ\cmn 我的儿子\end{exemple}\acception{2}
\begin{définition}\fra garçon\end{définition}
\begin{définition}\cmn 男孩;男子\end{définition}
\begin{relation-sémantique}\confer{
\hyperlink{Ⓔarɯtɤtɕɯ}{\textit{ \papi{arɯtɤtɕɯ}}}
}\end{relation-sémantique}\end{entrée}

\begin{entrée}
\vedette{\hypertarget{Ⓔtɤtɕɯβraʁ}{\papi{ tɤtɕɯβraʁ}}}\markboth{tɤtɕɯβraʁ}{}\classe{n}
\begin{définition}\fra bardane\end{définition}
\begin{définition}\cmn 牛蒡\end{définition}
\begin{exemple}\jya tɤtɕɯβraʁ nɯ arɤndɯndɤt tu-ɬoʁ ɕti, tɕe ɯ-qa rɲɟi, ngɯt, pakuku tu-ɬoʁ cha, ɯ-ru nɯ aɣɯrnɯɕɯr tsa tɕe tu-rɲɟi tsa cha. tɯrme tɯ-fsu jamar tu-mbro cha, ɯ-ru ɯ-χcɤl tɤ-kɯ-ɣe nɯ tɕu ɯ-jwaʁ ku-ndzoʁ tɕe nɯ ɯ-rca nɯ tɕu ɯ-rtaʁ ɲɯ-ɬoʁ, ɯ-jwaʁ wuma ʑo wxti, ɯ-jwaʁ ɯ-qhu chu nɯ kɯ-wɣrum tsa ŋu, ɯ-ʁɤri nɯ kɯ-ɤrŋi tsa ŋu. ɯ-rtaʁ ɯ-kɤχcɤl raŋri tɕu ɯ-mat kɯ-dɤn ʑo ku-ndzoʁ ŋu. ɯ-mat ɯ-rqhu nɯ ɯ-mdzu tu, kɯ-tɕɯ-tɕɤr kɯ-rɲɟi tsa ŋu, kɯ-dɤn ʑo aɣɯŋgɯŋgɯ tɕe nɯ ɯ-ŋgɯ kóʁmɯz ɯ-rɣi ŋu. ɯ-rɣi ɯ-kɤχcɤl zɯ ɯ-rme tu, ɯ-rɣi wuma ʑo dɤn, ɯ-rɣi ɣɯ ɯ-rme nɯ tɯ-ɕa ɯ-taʁ nɤ tɕaʁ tɕe rɤʑa, ɯ-mat kɤ́rqhɯrqhu nɯ ku-ondzoʁjoʁ cha tɕe tɯrme tɯ-ŋga ɯ-taʁ ku-ndɤm cha tɕe βʑɯ kɯ wuma ʑo nɯɣme ma ɯ-rme ɯ-taʁ ka-ndzoʁ tɕe kɤ-sɯ-ta mɯ́j-khɯ tɕe pjɯ́-wɣ-sat ɲɯ-ŋgrɤl, tɕe núndʐa ʑɯɣsɯr rmi\cmn 
牛蒡到处可以生长,根又长又结实,年年生长,茎是淡红色的,长得比较高,和人一样高,茎长出来再长出叶子,叶子长出来的地方就分叉,叶子很大,背面白色,正面绿色。每一根杈子顶上结很多果实,果实的外壳有刺,又细又长,有很多层,最里面才是种子。种子头上有很多毛,那些毛碰到皮肤上就会发痒。果实连同壳会粘在一起,也会粘在人的衣服上,老鼠最怕它,因为一旦粘在皮毛上,他们无法弄掉,它甚至会使他们丧命,所以它叫“\stylefv{ʑɯxsɯr}”
\end{exemple}
\begin{relation-sémantique}\confer{
\hyperlink{Ⓔʑɯxsɯr}{\textit{ \papi{ʑɯxsɯr}}}
}\end{relation-sémantique}
\end{entrée}

\begin{entrée}
\vedette{\hypertarget{Ⓔtɤ-tɕɯɣ}{\papi{ tɤ-tɕɯɣ}}}\markboth{tɤ-tɕɯɣ}{}\classe{np}
\begin{définition}\fra germe d'arbre\end{définition}
\begin{définition}\cmn 树的萌芽;新发出来的叶子\end{définition}
\begin{exemple}\jya tɤ-tɕɯɣ lo-lɤt\cmn (树)发芽了\end{exemple}
\begin{relation-sémantique}\confer{
\hyperlink{Ⓔɣɤtɕɯɣ}{\textit{ \papi{ɣɤtɕɯɣ}}}
}\end{relation-sémantique}\end{entrée}

\begin{entrée}
\vedette{\hypertarget{Ⓔtɤ-tɕɯpɯ}{\papi{ tɤ-tɕɯpɯ}}}\markboth{tɤ-tɕɯpɯ}{}\classe{np}
\begin{définition}\fra garçon\end{définition}
\begin{définition}\cmn 小男孩\end{définition}
\begin{relation-sémantique}\confer{
\hyperlink{Ⓔtɤ-tɕɯ}{\textit{ \papi{tɤ-tɕɯ}}}
}\end{relation-sémantique}\end{entrée}

\begin{entrée}
\vedette{\hypertarget{Ⓔtɤ-tɕɯrʑaβ}{\papi{ tɤ-tɕɯrʑaβ}}}\markboth{tɤ-tɕɯrʑaβ}{}\classe{np}
\begin{définition}\fra bru\end{définition}
\begin{définition}\cmn 媳妇\end{définition}
\begin{relation-sémantique}\confer{
\hyperlink{Ⓔtɤ-rʑaβ}{\textit{ \papi{tɤ-rʑaβ}}}
}\end{relation-sémantique}
\begin{relation-sémantique}\confer{
\hyperlink{Ⓔtɤ-tɕɯ}{\textit{ \papi{tɤ-tɕɯ}}}
}\end{relation-sémantique}\end{entrée}

\begin{entrée}
\vedette{\hypertarget{Ⓔtɤte}{\papi{ tɤte}}}\markboth{tɤte}{}\classe{adv}
\begin{définition}\fra c'est à dire, en gros\end{définition}
\begin{définition}\cmn 总的来说\end{définition}\end{entrée}

\begin{entrée}
\vedette{\hypertarget{Ⓔtɤthu}{\papi{ tɤthu}}}\markboth{tɤthu}{}\classe{n}
\begin{définition}\fra laine\end{définition}
\begin{définition}\cmn 毛布\end{définition}
\begin{relation-sémantique}\synonyme{
\hyperlink{Ⓔtɯŋgar}{\textit{ \papi{tɯŋgar}}}
}\end{relation-sémantique}\end{entrée}

\begin{entrée}
\vedette{\hypertarget{Ⓔtɤ-thɤβ}{\papi{ tɤ-thɤβ}}}\markboth{tɤ-thɤβ}{}\classe{np}
\begin{définition}\fra mésentente\end{définition}
\begin{définition}\cmn 纠纷\end{définition}
\begin{exemple}\jya ndʑi-thɤβ tɤ-βzu-t-a\cmn 我给他们俩调解了纠纷\end{exemple}
\begin{exemple}\jya tɤ-thɤβ tɤ-fɕɤt-i\cmn 我们调解了纠纷\end{exemple}
\begin{relation-sémantique}\synonyme{
\hyperlink{Ⓔɣɤɕphɤr}{\textit{ \papi{ɣɤɕphɤr}}}
}\end{relation-sémantique}\end{entrée}

\begin{entrée}
\vedette{\hypertarget{Ⓔtɤtho}{\papi{ tɤtho}}}\markboth{tɤtho}{}
\classe{n}
\begin{définition}\fra pin\end{définition}
\begin{définition}\cmn 松树
\begin{déclaration} \étymologie{\papi{tʰaŋ}}\end{déclaration}\end{définition}
\begin{exemple}\jya tɤtho nɯ zgoku kɯ-mbro tsa tu-ɬoʁ ŋu, ɯ-jwaʁ nɯ taqaβ kɯ-fse kɯ-zri ŋu, ɯ-jwaʁ nɯ ɯ-rtaʁ ɣɯ ɯ-βri aʁɤndɯndɤt ku-ndzoʁ ŋu, aɣɯjwaʁ, ɯ-mdoʁ nɯ tɯrgi ɯ-jwaʁ mdoʁ cho naχtɕɯɣ, tɯ-xpa lɤ-skɤr ɯ-mdoʁ ɲɯ-nɤsci mɤ-cha. ɯ-ru jpum tsa aɣɯrtɯrtaʁ, ɯ-tɯ-ɤɣɯrtɯrtaʁ nɯ ɕɤɣ ʑo fse, ɯ-mat tu, tɯrgi laŋlaŋ cho naχtɕɯɣ ri, artɯm. tɤtho ɯ-ru nɯ li ɕɤɣ jamar ma kɤ-rɤɣdɤt mɤ-rtaʁ. ŋgɤjpɤn nɯ li kɯ-ʑru kɤ-sɯpa ŋu, ma qajɯ kɯ mɤ-ndze. tɤtho si nɯ wuma ʑo ɯ-tɤ-ndʑɯɣ dɤn tɕe wuma ʑo kɤ-βlɯ pe.\cmn 松树生长在高山上,叶子长得像针一样,比较长,叶子在枝桠上到处生长,很茂盛,颜色和杉树的颜色一样,一年四季不变色。树干比较粗,长出很多枝桠,枝桠生长的方式很像柏树,有果实,像杉树的果实一样但是是球形的。松树的树干也只能锯成和柏树那么多的几段。用松树的木料制造的木板算是比较优质的,因为蛀虫不爱咬。因为松树的树脂多,所以很好烧。\end{exemple}\end{entrée}

\begin{entrée}
\vedette{\hypertarget{Ⓔtɤthotʂu}{\papi{ tɤthotʂu}}}\markboth{tɤthotʂu}{}\classe{n}
\begin{définition}\fra torche en pin\end{définition}
\begin{définition}\cmn 松明\end{définition}
\end{entrée}

\begin{entrée}
\vedette{\hypertarget{Ⓔtɤton}{\papi{ tɤton}}}\markboth{tɤton}{}
\classe{n}
\begin{définition}\fra vers le haut, vers l'amont\end{définition}
\begin{définition}\cmn 往上;往上游\end{définition}\end{entrée}

\begin{entrée}
\vedette{\hypertarget{Ⓔtɤtshoʁ}{\papi{ tɤtshoʁ}}}\markboth{tɤtshoʁ}{}\classe{n}
\begin{définition}\fra clou\end{définition}
\begin{définition}\cmn 钉子\end{définition}
\end{entrée}

\begin{entrée}
\vedette{\hypertarget{Ⓔtɤ-tshɯɣ}{\papi{ tɤ-tshɯɣ}}}\markboth{tɤ-tshɯɣ}{}
\classe{np}
\begin{définition}\fra œillère\end{définition}
\begin{définition}\cmn 马笼头\end{définition}
\begin{exemple}\jya mbro ɯ-tshɯɣ tɤ-ta-t-a\cmn 我给马戴上了马笼头\end{exemple}\end{entrée}

\begin{entrée}
\vedette{\hypertarget{Ⓔtɤtshɯtsha}{\papi{ tɤtshɯtsha}}}\markboth{tɤtshɯtsha}{}\classe{n}
\begin{définition}\fra salpêtre\end{définition}
\begin{définition}\cmn 硝\end{définition}\end{entrée}

\begin{entrée}
\vedette{\hypertarget{Ⓔtɤtsoʁ}{\papi{ tɤtsoʁ}}}\markboth{tɤtsoʁ}{}
\classe{n}
\begin{définition}\fra Potentilla anserina (gro-ma)\end{définition}
\begin{définition}\cmn 人参果\end{définition}\end{entrée}

\begin{entrée}
\vedette{\hypertarget{Ⓔtɤ-tsrɯ}{\papi{ tɤ-tsrɯ}}}\markboth{tɤ-tsrɯ}{}
\classe{np}
\begin{définition}\fra pousses\end{définition}
\begin{définition}\cmn 萌芽\end{définition}\end{entrée}

\begin{entrée}
\vedette{\hypertarget{Ⓔtɤ-tsɯr}{\papi{ tɤ-tsɯr}}}\markboth{tɤ-tsɯr}{}\classe{np}
\begin{définition}\fra fissure\end{définition}
\begin{définition}\cmn 裂缝\end{définition}
\begin{exemple}\jya ɯ-rnom ɕɤrɯ ɣɯ ɯ-tsɯr lo-ɕe (=lo-ɣɤtsɯr)\cmn 他肋骨骨折了\end{exemple}
\begin{relation-sémantique}\confer{
\hyperlink{Ⓔɣɤtsɯr}{\textit{ \papi{ɣɤtsɯr}}}
}\end{relation-sémantique}\end{entrée}

\begin{entrée}
\vedette{\hypertarget{Ⓔtɤtʂu}{\papi{ tɤtʂu}}}\markboth{tɤtʂu}{}
\classe{n}
\begin{définition}\fra lampe\end{définition}
\begin{définition}\cmn 灯\end{définition}
\begin{exemple}\jya aʑo tɤtʂu ci tu-ci-a ɲɯ-ntshi\cmn 我要开灯\end{exemple}
\begin{relation-sémantique}\confer{
\hyperlink{Ⓔsɤtʂu}{\textit{ \papi{sɤtʂu}}}
}\end{relation-sémantique}\end{entrée}

\begin{entrée}
\vedette{\hypertarget{Ⓔtɤ-tʂɤm}{\papi{ tɤ-tʂɤm}}}\markboth{tɤ-tʂɤm}{}
\classe{np}
\begin{définition}\fra huile animale\end{définition}
\begin{définition}\cmn 动物的油脂\end{définition}\end{entrée}

\begin{entrée}
\vedette{\hypertarget{Ⓔtɤtʂo}{\papi{ tɤtʂo}}}\markboth{tɤtʂo}{}\classe{n}
\begin{définition}\fra lœss\end{définition}
\begin{définition}\cmn 粘土;黄土\end{définition}\end{entrée}

\begin{entrée}
\vedette{\hypertarget{Ⓔtɤtɯr}{\papi{ tɤtɯr}}}\markboth{tɤtɯr}{}
\classe{n}
\begin{définition}\fra outil pour graver les motifs sur l'argent\end{définition}
\begin{définition}\cmn 刻花纹的工具\end{définition}\end{entrée}

\begin{entrée}
\vedette{\hypertarget{Ⓔtɤwu}{\papi{ tɤwu}}}\markboth{tɤwu}{}\classe{n}
\begin{définition}\fra pleurs\end{définition}
\begin{définition}\cmn 哭\end{définition}
\begin{exemple}\jya tɤwu kɯ a-ku ʑo tɤ-mŋɤm\cmn 我哭得令我头疼\end{exemple}
\begin{relation-sémantique}\confer{
\hyperlink{Ⓔɣɤwu}{\textit{ \papi{ɣɤwu}}}
}\end{relation-sémantique}
\begin{relation-sémantique}\confer{
\hyperlink{Ⓔnɤwu}{\textit{ \papi{nɤwu}}}
}\end{relation-sémantique}\end{entrée}

\begin{entrée}
\vedette{\hypertarget{Ⓔtɤ-wa}{\papi{ tɤ-wa}}}\markboth{tɤ-wa}{}
\classe{np}
\begin{définition}\fra père\end{définition}
\begin{définition}\cmn 父亲\end{définition}
\begin{exemple}\jya a-mu a-wa\cmn 我父母\end{exemple}\end{entrée}

\begin{entrée}
\vedette{\hypertarget{Ⓔtɤ-wɤmɯ}{\papi{ tɤ-wɤmɯ}}}\markboth{tɤ-wɤmɯ}{}
\classe{np}
\begin{définition}\fra frère (employé par les filles)\end{définition}
\begin{définition}\cmn 兄弟(女性专用)\end{définition}
\begin{exemple}\jya a-wɤmɯ\cmn 我的哥哥(弟弟)\end{exemple}\end{entrée}

\begin{entrée}
\vedette{\hypertarget{Ⓔtɤ-wi}{\papi{ tɤ-wi}}}\markboth{tɤ-wi}{}\classe{np}
\begin{définition}\fra grand-mère\end{définition}
\begin{définition}\cmn 奶奶;婆婆\end{définition}
\end{entrée}

\begin{entrée}
\vedette{\hypertarget{ⒺtɤwɯⒽ2}{\papi{ tɤwɯ}}}\markboth{tɤwɯ}{}\homonyme{2}\classe{n}
\begin{définition}\fra couverture de feutre\end{définition}
\begin{définition}\cmn 遮雨的毡子\end{définition}\end{entrée}

\begin{entrée}
\vedette{\hypertarget{Ⓔtɤ-wɯⒽ1}{\papi{ tɤ-wɯ}}}\markboth{tɤ-wɯ}{}\homonyme{1}
\classe{np}
\begin{définition}\fra grand-père (c'est le terme par lequel les animaux s'adressent aux être humains dans les histoires)\end{définition}
\begin{définition}\cmn 爷爷;公公 (在故事里面,也是动物对人的称呼)\end{définition}
\end{entrée}

\begin{entrée}
\vedette{\hypertarget{Ⓔtɤwɯrte}{\papi{ tɤwɯrte}}}\markboth{tɤwɯrte}{}
\classe{n}
\begin{définition}\fra chapeau en feutre\end{définition}
\begin{définition}\cmn 毡子制成的帽子\end{définition}
\begin{relation-sémantique}\confer{
\hyperlink{Ⓔtɤ-rte}{\textit{ \papi{tɤ-rte}}}
}\end{relation-sémantique}\end{entrée}

\begin{entrée}
\vedette{\hypertarget{Ⓔtɤ-xtɤɣ}{\papi{ tɤ-xtɤɣ}}}\markboth{tɤ-xtɤɣ}{}
\classe{np}
\begin{définition}\fra frère (employé par les garçons)\end{définition}
\begin{définition}\cmn 兄弟(男性专用)\end{définition}
\begin{exemple}\jya a-xtɤɣ\cmn 我的兄弟\end{exemple}\end{entrée}

\begin{entrée}
\vedette{\hypertarget{Ⓔtɤxtɕɤβ}{\papi{ tɤxtɕɤβ}}}\markboth{tɤxtɕɤβ}{}\classe{n}
\begin{définition}\fra herbe pour les animaux\end{définition}
\begin{définition}\cmn 喂动物的草\end{définition}
\begin{relation-sémantique}\confer{
\hyperlink{Ⓔɣɤxtɕɤβ}{\textit{ \papi{ɣɤxtɕɤβ}}}
}\end{relation-sémantique}\end{entrée}

\begin{entrée}
\vedette{\hypertarget{Ⓔtɤ-xtɕɤr}{\papi{ tɤ-xtɕɤr}}}\markboth{tɤ-xtɕɤr}{}\classe{np}
\begin{définition}\fra corde\end{définition}
\begin{définition}\cmn 用来捆东西的绳索\end{définition}
\begin{relation-sémantique}\confer{
\hyperlink{Ⓔxtɕɤr}{\textit{ \papi{xtɕɤr}}}
}\end{relation-sémantique}\end{entrée}

\begin{entrée}
\vedette{\hypertarget{Ⓔtɤ-xtsɤr}{\papi{ tɤ-xtsɤr}}}\markboth{tɤ-xtsɤr}{}\classe{np}
\begin{définition}\fra enceinte (vache)\end{définition}
\begin{définition}\cmn 怀孕(母牛)\end{définition}
\begin{exemple}\jya nɯŋa ɯ-xtsɤr tu tɕe, kɯ-ɤrqhi ma-jɤ́-wɣ-no ra\cmn 这个母牛怀孕了,不可以赶到远处去\end{exemple}
\begin{exemple}\jya ftsoʁ ɯ-xtsɤr kɯ-tu\cmn 怀孕的母犏牛\end{exemple}
\begin{exemple}\jya qra ɯ-xtsɤr kɯ-mbro\cmn 快要生的母牦牛\end{exemple}\end{entrée}

\begin{entrée}
\vedette{\hypertarget{Ⓔtɤχsɤr}{\papi{ tɤχsɤr}}}\markboth{tɤχsɤr}{}\classe{n}
\begin{définition}\fra nombre\end{définition}
\begin{définition}\cmn 数目\end{définition}
\begin{exemple}\jya tɤχsɤr ɯ-kɯ-sti ɕti-a ma koŋla a-kɤ-spa me\cmn 我只是充数的,我什么也不会\end{exemple}\end{entrée}

\begin{entrée}
\vedette{\hypertarget{Ⓔtɤχtɯχtɤt}{\papi{ tɤχtɯχtɤt}}}\markboth{tɤχtɯχtɤt}{}
\classe{n}
\begin{définition}\fra avis, information\end{définition}
\begin{définition}\cmn 通知\end{définition}
\begin{exemple}\jya tɤχtɯχtɤt ɯ-kɯ-lɤt ɯʑo pɯ-ŋu\cmn 通知大家的是他\end{exemple}
\begin{exemple}\jya a-tɤχtɯχtɤt na-lɤt\cmn 他通知了我\end{exemple}\end{entrée}

\begin{entrée}
\vedette{\hypertarget{Ⓔtɤzdɯɣ}{\papi{ tɤzdɯɣ}}}\markboth{tɤzdɯɣ}{}\classe{n}
\paradigme{\textit{emphatic :} \jya tɤzdɯɣ tɤsŋɤl}
\begin{définition}\fra peine\end{définition}
\begin{définition}\cmn 辛苦;艰苦
\begin{déclaration} \étymologie{\papi{sdug.bsŋal}}\end{déclaration}\end{définition}
\begin{exemple}\jya a-tɤzdɯɣ a-tɤsŋɤl pɯ-rtaʁ\cmn 我受够了苦难\end{exemple}
\begin{relation-sémantique}\confer{
\hyperlink{Ⓔnɯzdɯsŋɤl}{\textit{ \papi{nɯzdɯsŋɤl}}}
}\end{relation-sémantique}\end{entrée}

\begin{entrée}
\vedette{\hypertarget{Ⓔtɤ-zgra}{\papi{ tɤ-zgra}}}\markboth{tɤ-zgra}{}
\classe{np}
\begin{définition}\fra son\end{définition}
\begin{définition}\cmn 声音;噪音\end{définition}
\begin{exemple}\jya ɯ-pɕi tɤ-zgra ɣɤʑu\cmn 外面很吵\end{exemple}
\begin{exemple}\jya tɤ-zgra ɲɯ-thɯ tɕe, koŋla mɯ́j-sɤmtshɤm\cmn 噪音很严重,根本就听不见\end{exemple}
\begin{exemple}\jya tɤ-zgra ɲɤ-ftshi\cmn 没有那么吵了\end{exemple}\end{entrée}

\begin{entrée}
\vedette{\hypertarget{Ⓔtɤzmbɯr}{\papi{ tɤzmbɯr}}}\markboth{tɤzmbɯr}{}\classe{n}
\begin{définition}\fra boue\end{définition}
\begin{définition}\cmn 泥沙\end{définition}
\begin{exemple}\jya zgo pjɤ-mbɯt tɕe, tɤzmbɯr chɤ-ɣi tɕe, ndzom ɯ-pa chɤ-sti\cmn 山塌下来了,泥沙把桥下堵住了,\end{exemple}\end{entrée}

\begin{entrée}
\vedette{\hypertarget{Ⓔtɤzɲɟoʁ}{\papi{ tɤzɲɟoʁ}}}\markboth{tɤzɲɟoʁ}{}\classe{n}
\begin{définition}\fra branche flexible utilisée pour fouetter les animaux\end{définition}
\begin{définition}\cmn 用来打牲畜的木条\end{définition}
\begin{relation-sémantique}\confer{
\hyperlink{Ⓔnɤzɲɟoʁ}{\textit{ \papi{nɤzɲɟoʁ}}}
}\end{relation-sémantique}\end{entrée}

\begin{entrée}
\vedette{\hypertarget{Ⓔtɤzraj}{\papi{ tɤzraj}}}\markboth{tɤzraj}{}\classe{n}
\begin{définition}\fra espèce d'arbre\end{définition}
\begin{définition}\cmn 银木\end{définition}\end{entrée}

\begin{entrée}
\vedette{\hypertarget{Ⓔtɤzraʁ}{\papi{ tɤzraʁ}}}\markboth{tɤzraʁ}{}\classe{n}
\begin{définition}\fra honte\end{définition}
\begin{définition}\cmn 廉耻心\end{définition}
\begin{exemple}\jya aʑo tɤzraʁ kɯ pɯ-thɯɣ ʑo\cmn 我羞愧极了\end{exemple}
\begin{exemple}\jya kɯki tɯrme ki tɤzraʁ mɯ́j-tso, tɤzraʁ mɯ́j-mtshɤm\cmn 这个人不要脸,不懂羞耻\end{exemple}
\begin{relation-sémantique}\confer{
\hyperlink{Ⓔnɤzraʁ}{\textit{ \papi{nɤzraʁ}}}
}\end{relation-sémantique}\end{entrée}

\begin{entrée}
\vedette{\hypertarget{Ⓔtɤ-zrɤm}{\papi{ tɤ-zrɤm}}}\markboth{tɤ-zrɤm}{}
\classe{np}
\begin{définition}\fra racine\end{définition}
\begin{définition}\cmn 根\end{définition}\end{entrée}

\begin{entrée}
\vedette{\hypertarget{Ⓔtɤ-ʑi}{\papi{ tɤ-ʑi}}}\markboth{tɤ-ʑi}{}
\classe{np}
\begin{définition}\fra jeune femme (ayant des enfants)\end{définition}
\begin{définition}\cmn 少奶奶\end{définition}\end{entrée}

\begin{entrée}
\vedette{\hypertarget{Ⓔtɤʑŋgrɯt}{\papi{ tɤʑŋgrɯt}}}\markboth{tɤʑŋgrɯt}{}
\classe{n}
\begin{définition}\fra cicatrice\end{définition}
\begin{définition}\cmn 疤痕\end{définition}
\begin{exemple}\jya a-tɤʑŋgrɯt\cmn 我的疤痕\end{exemple}\end{entrée}

\begin{entrée}
\vedette{\hypertarget{Ⓔtɤʑri}{\papi{ tɤʑri}}}\markboth{tɤʑri}{}
\classe{n}
\begin{définition}\fra rosée\end{définition}
\begin{définition}\cmn 露水\end{définition}
\begin{relation-sémantique}\confer{
\hyperlink{Ⓔnɤʑri}{\textit{ \papi{nɤʑri}}}
}\end{relation-sémantique}\end{entrée}

\begin{entrée}
\vedette{\hypertarget{Ⓔtɤʑɯn}{\papi{ tɤʑɯn}}}\markboth{tɤʑɯn}{}\classe{n}
\begin{définition}\fra vers le bas, vers l'aval\end{définition}
\begin{définition}\cmn 往下;往下游\end{définition}\end{entrée}

\begin{entrée}
\vedette{\hypertarget{Ⓔtɕa}{\papi{ tɕa}}}\markboth{tɕa}{}
\classe{vs}
\paradigme{\textit{dir :} \jya nɯ-}
\begin{définition}\fra maigre\end{définition}
\begin{définition}\cmn 瘦弱\end{définition}
\begin{exemple}\jya ɯ-sɯm ɲɯ-tɕa\cmn 他没有把握,没有信心\end{exemple}
\begin{relation-sémantique}\confer{
\hyperlink{Ⓔtɕale}{\textit{ \papi{tɕale}}}
}\end{relation-sémantique}\end{entrée}

\begin{entrée}
\vedette{\hypertarget{Ⓔtɕaɣi}{\papi{ tɕaɣi}}}\markboth{tɕaɣi}{}\classe{n}
\begin{définition}\fra perroquet\end{définition}
\begin{définition}\cmn 鹦鹉\end{définition}
\begin{exemple}\jya tɕaɣi ɲɯ-tɯ-fse\cmn 你话多,唠叨得不停(你像鹦鹉一样)\end{exemple}\end{entrée}

\begin{entrée}
\vedette{\hypertarget{Ⓔtɕakɯɣ}{\papi{ tɕakɯɣ}}}\markboth{tɕakɯɣ}{}\classe{n}
\begin{définition}\fra sac pour les feuilles de thé\end{définition}
\begin{définition}\cmn 茶叶袋
\begin{déclaration} \étymologie{\papi{dʑa.kʰug}}\end{déclaration}\end{définition}
\end{entrée}

\begin{entrée}
\vedette{\hypertarget{Ⓔtɕale}{\papi{ tɕale}}}\markboth{tɕale}{}\classe{vs}
\begin{définition}\fra maigre\end{définition}
\begin{définition}\cmn 瘦\end{définition}
\begin{relation-sémantique}\confer{
\hyperlink{Ⓔtɕa}{\textit{ \papi{tɕa}}}
}\end{relation-sémantique}
\begin{relation-sémantique}\antonyme{
\hyperlink{Ⓔjpumqa}{\textit{ \papi{jpumqa}}}
}\end{relation-sémantique}\end{entrée}

\begin{entrée}
\vedette{\hypertarget{Ⓔtɕamu}{\papi{ tɕamu}}}\markboth{tɕamu}{}\classe{n}
\begin{définition}\fra espèce de plante\end{définition}
\begin{définition}\cmn 植物的一种
\begin{déclaration} \étymologie{\papi{dʑa.mo}}\end{déclaration}\end{définition}
\begin{exemple}\jya tɕamu nɯ χsɯ-tɯphu tu tɕe, sɯjno ci tu tɕe ɯ-jwaʁ kɯ-rʁom tsa ci ŋu, mɤ-mbro, ɯ-mɯntoʁ ɯ-ru kɯ-xtshɯm kɯ-zri tsa tu-ɬoʁ tɕe ɯ-mɯntoʁ kɯ-wɣrum ɲɯ-lɤt ŋu, ɯ-mɯntoʁ ɯ-qhu pɕoʁ nɯ kɯ-ɣɯrni kɯ-qandʐi tsa ŋu, ɯ-jwaʁ nɯ ɲɯ́-wɣ-sɯɣ-rom tɕe, tʂha kɤ-nɯ-ta sna, tɕe núndʐa tɕamu rmi. li ci tɯ-tɯphu nɯ, ɕkrɤz kɯ-do tsa ɣɯ ɯ-ci kɯ-ɣɯrni tɤ-se ʑo kɯ-fse pjɯ-nɯɬoʁ tɕe, cischiz tu-ojtɯ tɕe tɤ-rʑaʁ tɤ-rɲɟi tɕe ku-jkrɯt tɕe ɯ-rŋgɤm ɲɯ-βze tɕe ɯ-nɯnɯ wuma ʑo tʂha kɯ-ʑru tɯ-mtshi kɯ-mŋɤm kɯ-phɤn ɲɯ-ŋu khi, tɕeri wuma ʑo kɤ-mto rkɯn. ɕkrɤz ɯ-kɤχcɤl ri kɯ-rko ku-kɯ-ndzoʁ ci tu tɕe nɯ li tʂha ku-nɯ-ta-nɯ pɯ-ŋgrɤl. mɤʑɯ tɯ-tɯphu nɯ sɤjku ɯ-taʁ ku-kɯ-ndzoʁ tɕe, kɯ-rko ci ɲɯ-ŋu tɕe ɯ-nɯnɯ pjɯ́-wɣ-qrɯ tɕe ɲɯ-ɣɯrni tsa tɕe nɯnɯ li tʂha ɯ-rca kú-wɣ-nɯ-ta tɕe tʂha ɯ-mdoʁ ɲɯ-ɣɤmpɕɤr cha.\cmn 
\stylefv{tɕamu}有三种,其中一个是一种草,叶子有点粗糙,长得不高,花茎比较细长,开白花,花背面是暗红色的,把叶子晒干了以后,可以熬茶喝,所以叫作\stylefv{tɕamu}。还有一种是比较老的青冈树上流出红色像血一样的液体滴在某个地方时,时间一长就会凝结成的坚硬的固体,那是一种优质的茶,据说可以治胃病,但是很罕见。在青冈树上也会长出一种硬东西,以前曾经有人把它当茶喝。还有一种是长在白桦树上的硬东西,打碎了里面是红色的,把它放在马茶里熬使茶水颜色好看。
\end{exemple}
\end{entrée}

\begin{entrée}
\vedette{\hypertarget{Ⓔtɕamba}{\papi{ tɕamba}}}\markboth{tɕamba}{}\classe{n}
\begin{définition}\fra une plante\end{définition}
\begin{définition}\cmn 【冬寒菜】\end{définition}
\begin{exemple}\jya tɕamba nɯ sɯjno ci ɯ-jwaʁ kɯ-ɤrtɯm tɕe βzɯr kɯ-tu ci ŋu. ɯ-ru kɯ-zri tsa tu-βze cha, ɯ-ru ɯ-pɕi nɯ mɤ-mpɕu, qaɕpa ɯ-mgɯr tsa fse. ɯ-jwaʁ ɯ-sɤɣ-ndzoʁ nɯ tɕu, ɯ-mɯntoʁ ɲɯ-lɤt tɕe, ɯ-mat ɲɯ-βze ŋu. ɯ-mɯntoʁ kɯ-wɣrum ɯ-ŋgɯz kɯ-ɣɯrni ci ŋu, kɤ-ndza sna.\cmn 冬寒菜是叶子圆形,有角的植物,茎长得比较高,茎表面不光滑,像青蛙的背一样,在长叶子的部位,开花结果,花是白里透红的。可以吃。\end{exemple}
\end{entrée}

\begin{entrée}
\vedette{\hypertarget{Ⓔtɕaŋ}{\papi{ tɕaŋ}}}\markboth{tɕaŋ}{}
\classe{n}
\begin{définition}\fra fer\end{définition}
\begin{définition}\cmn 马蹄铁\end{définition}\end{entrée}

\begin{entrée}
\vedette{\hypertarget{Ⓔtɕaŋnɤ}{\papi{ tɕaŋnɤ}}}\markboth{tɕaŋnɤ}{}\classe{cnj}
\begin{définition}\fra alors, sinon\end{définition}
\begin{définition}\cmn 那就,不然就(提出自己的条件才答应别人的请求)\end{définition}\end{entrée}

\begin{entrée}
\vedette{\hypertarget{Ⓔtɕaŋtɣa}{\papi{ tɕaŋtɣa}}}\markboth{tɕaŋtɣa}{}\classe{n}
\begin{définition}\fra couteau\end{définition}
\begin{définition}\cmn 菜刀\end{définition}
\end{entrée}

\begin{entrée}
\vedette{\hypertarget{Ⓔtɕaʁ}{\papi{ tɕaʁ}}}\markboth{tɕaʁ}{}\classe{vi}
\paradigme{\textit{dir :} \jya pɯ-}
\begin{définition}\fra être mûr (pêche)\end{définition}
\begin{définition}\cmn 成熟(桃子)\end{définition}
\begin{exemple}\jya qaɕti pjɤ-tɕaʁ\cmn 桃子成熟了\end{exemple}
\end{entrée}

\begin{entrée}
\vedette{\hypertarget{Ⓔtɕaʁkɤr khɯtsa}{\papi{ tɕaʁkɤr khɯtsa}}}\markboth{tɕaʁkɤr khɯtsa}{}\classe{n}
\begin{définition}\fra bol en fer\end{définition}
\begin{définition}\cmn 铁碗\end{définition}\end{entrée}

\begin{entrée}
\vedette{\hypertarget{Ⓔtɕaʁmɤr}{\papi{ tɕaʁmɤr}}}\markboth{tɕaʁmɤr}{}
\classe{n}
\begin{définition}\fra briquet\end{définition}
\begin{définition}\cmn 火镰
\begin{déclaration} \étymologie{\papi{ltɕags.dmar}}\end{déclaration}\end{définition}\end{entrée}

\begin{entrée}
\vedette{\hypertarget{Ⓔtɕaʁtshɯɣ}{\papi{ tɕaʁtshɯɣ}}}\markboth{tɕaʁtshɯɣ}{}\classe{n}
\begin{définition}\fra brûlure au fer rouge pour soigner les migraines\end{définition}
\begin{définition}\cmn 烙(治头风的方式)
\begin{déclaration}\use{同\stylefv{ta (kɤ-)}连用}\end{déclaration}
\begin{déclaration} \étymologie{\papi{ltɕags.tsʰigs}}\end{déclaration}\end{définition}
\begin{exemple}\jya tɕaʁtshɯɣ ko-ta\cmn 他给他烙了印子。\end{exemple}
\begin{exemple}\jya smɤnba kɯ tu-kɯ-nɯsmɤn, tɯ-ku ɯ-taʁ ku-te nɯ tɕaʁtshɯɣ\cmn 
医生治病的时候,放在头上的那个叫\stylefv{tɕaʁtshɯɣ}
\end{exemple}\end{entrée}

\begin{entrée}
\vedette{\hypertarget{Ⓔtɕaʁzgroʁ}{\papi{ tɕaʁzgroʁ}}}\markboth{tɕaʁzgroʁ}{}\classe{n}
\begin{définition}\fra fers (pieds)\end{définition}
\begin{définition}\cmn 脚镣
\begin{déclaration} \étymologie{\papi{ltɕags.sgrog}}\end{déclaration}\end{définition}\end{entrée}

\begin{entrée}
\vedette{\hypertarget{Ⓔtɕaχkɤr}{\papi{ tɕaχkɤr}}}\markboth{tɕaχkɤr}{}\classe{n}
\begin{définition}\fra fer blanc\end{définition}
\begin{définition}\cmn 白铁皮
\begin{déclaration} \étymologie{\papi{ltɕags.dkar}}\end{déclaration}\end{définition}\end{entrée}

\begin{entrée}
\vedette{\hypertarget{Ⓔtɕaχkhɤβ}{\papi{ tɕaχkhɤβ}}}\markboth{tɕaχkhɤβ}{}\classe{n}
\begin{définition}\fra cheminée\end{définition}
\begin{définition}\cmn 火炉
\begin{déclaration} \étymologie{\papi{ltɕags.khab}}\end{déclaration}\end{définition}
\end{entrée}

\begin{entrée}
\vedette{\hypertarget{Ⓔtɕaχpa}{\papi{ tɕaχpa}}}\markboth{tɕaχpa}{}\classe{n}
\begin{définition}\fra bandit\end{définition}
\begin{définition}\cmn 强盗
\begin{déclaration} \étymologie{\papi{dʑag.pa}}\end{déclaration}\end{définition}
\begin{relation-sémantique}\confer{
\hyperlink{Ⓔnɯtɕaχpa}{\textit{ \papi{nɯtɕaχpa}}}
}\end{relation-sémantique}\end{entrée}

\begin{entrée}
\vedette{\hypertarget{Ⓔtɕazga}{\papi{ tɕazga}}}\markboth{tɕazga}{}\classe{n}
\begin{définition}\fra gingembre\end{définition}
\begin{définition}\cmn 姜
\begin{déclaration} \étymologie{\papi{sga.skʲa}}\end{déclaration}\end{définition}\end{entrée}

\begin{entrée}
\vedette{\hypertarget{Ⓔtɕɤβ}{\papi{ tɕɤβ}}}\markboth{tɕɤβ}{}\classe{vt}
\paradigme{\textit{dir :} \jya pɯ-}
\paradigme{\textit{dir :} \jya lɤ-}
\begin{définition}\fra brûler\end{définition}
\begin{définition}\cmn 烧\end{définition}
\begin{exemple}\jya pa-tɕɤβ\cmn 他烧了\end{exemple}\begin{sous-entrée}
\vedette{\hypertarget{}{\papi{ rɤtɕɤβ}}}\markboth{rɤtɕɤβ}{}\classe{vi}
\paradigme{\textit{dir :} \jya pɯ-}
\begin{définition}\ 
\begin{déclaration}\grammar{apass}\end{déclaration}\end{définition}
\begin{définition}\fra défricher par le feu\end{définition}
\begin{définition}\cmn 烧荒\end{définition}
\begin{exemple}\jya jisŋi tɯji ɯ-ŋgɯ ɕ-pɯ-rɤtɕaβ-a\cmn 我今天到田地里去烧荒了\end{exemple}
\begin{relation-sémantique}\confer{
\hyperlink{Ⓔɲɟɤβ}{\textit{ \papi{ɲɟɤβ}}}
}\end{relation-sémantique}
\end{sous-entrée}\end{entrée}

\begin{entrée}
\vedette{\hypertarget{Ⓔtɕɤkɯ}{\papi{ tɕɤkɯ}}}\markboth{tɕɤkɯ}{}
\classe{adv}
\begin{définition}\fra à l'est\end{définition}
\begin{définition}\cmn 在东边\end{définition}
\begin{exemple}\jya tɕɤkɯ ku-ɕe-a\cmn 我往东边去\end{exemple}
\begin{exemple}\jya tɕɤkɯkɯ ʑo ri ku-rɤʑi-a\cmn 我在东边\end{exemple}
\begin{relation-sémantique}\antonyme{
\hyperlink{Ⓔtɕɤndi}{\textit{ \papi{tɕɤndi}}}
}\end{relation-sémantique}
\begin{relation-sémantique}\confer{
\hyperlink{Ⓔakɯ}{\textit{ \papi{akɯ}}}
}\end{relation-sémantique}\end{entrée}

\begin{entrée}
\vedette{\hypertarget{Ⓔtɕɤlo}{\papi{ tɕɤlo}}}\markboth{tɕɤlo}{}\classe{adv}
\begin{définition}\fra en amont\end{définition}
\begin{définition}\cmn 上游\end{définition}
\begin{relation-sémantique}\confer{
\hyperlink{Ⓔalo}{\textit{ \papi{alo}}}
}\end{relation-sémantique}\end{entrée}

\begin{entrée}
\vedette{\hypertarget{Ⓔtɕɤmɯ}{\papi{ tɕɤmɯ}}}\markboth{tɕɤmɯ}{}\classe{n}
\begin{définition}\fra none\end{définition}
\begin{définition}\cmn 尼姑
\begin{déclaration} \étymologie{\papi{dʑo.mo}}\end{déclaration}\end{définition}
\end{entrée}

\begin{entrée}
\vedette{\hypertarget{Ⓔtɕɤndi}{\papi{ tɕɤndi}}}\markboth{tɕɤndi}{}\classe{adv}
\begin{définition}\fra à l'ouest\end{définition}
\begin{définition}\cmn 在西边\end{définition}
\begin{exemple}\jya tɕɤndɯndi ʑo ku-rɤʑi-a\cmn 我在西边\end{exemple}
\begin{relation-sémantique}\antonyme{
\hyperlink{Ⓔtɕɤkɯ}{\textit{ \papi{tɕɤkɯ}}}
}\end{relation-sémantique}
\begin{relation-sémantique}\confer{
\hyperlink{Ⓔandi}{\textit{ \papi{andi}}}
}\end{relation-sémantique}\end{entrée}

\begin{entrée}
\vedette{\hypertarget{Ⓔtɕɤphɯ}{\papi{ tɕɤphɯ}}}\markboth{tɕɤphɯ}{} (\variante{tɕhɤphɯ}) \classe{n}
\begin{définition}\ 
\begin{déclaration}\grammar{n.lieu}\end{déclaration}\end{définition}
\begin{définition}\fra Japhug\end{définition}
\begin{définition}\cmn 茶堡区\end{définition}
\begin{exemple}\jya tɕɤphɯpɯ\cmn 茶堡人\end{exemple}\end{entrée}

\begin{entrée}
\vedette{\hypertarget{Ⓔtɕɤr}{\papi{ tɕɤr}}}\markboth{tɕɤr}{}
\classe{vs}
\paradigme{\textit{dir :} \jya kɤ-}
\begin{définition}\fra étroit\end{définition}
\begin{définition}\cmn 窄\end{définition}
\begin{exemple}\jya nɯ ɯ-spa ɲɯ-tɕɤr\cmn (布、板子的)材料很窄\end{exemple}
\begin{relation-sémantique}\antonyme{
\hyperlink{Ⓔrɟum}{\textit{ \papi{rɟum}}}
}\end{relation-sémantique}
\begin{relation-sémantique}\confer{
\hyperlink{Ⓔarɟumtɕɤr}{\textit{ \papi{arɟumtɕɤr}}}
}\end{relation-sémantique}\end{entrée}

\begin{entrée}
\vedette{\hypertarget{Ⓔtɕɤt}{\papi{ tɕɤt}}}\markboth{tɕɤt}{}\classe{vt}\acception{1}
\paradigme{\textit{dir :} \jya \_}
\begin{définition}\fra retirer de, extraire de\end{définition}
\begin{définition}\cmn 取出\end{définition}
\begin{exemple}\jya sɤcɯ nɯ-tɕat-a\cmn 我从锁里把钥匙取出来了\end{exemple}
\begin{exemple}\jya qaɟy ɯ-naŋtɕɯ thɯ-tɕat-a\cmn 我把鱼的内脏取出来了\end{exemple}
\begin{exemple}\jya ɯ-kri nɯ-tɕat-a\cmn 我把猪油熬出来了\end{exemple}
\begin{exemple}\jya tɯ-ɣli thɯ-tɕɤt-i\cmn 我们出圈肥了\end{exemple}
\begin{exemple}\jya tɤ-lu pɯ-tɕɤt\cmn 你挤奶吧\end{exemple}
\begin{exemple}\jya tɤ-ro tɤ-tɕat-a\cmn 我挺起胸膛了\end{exemple}
\begin{exemple}\jya ɯ-qom ra pjɤ-tɕɤt\cmn 她掉下了眼泪\end{exemple}\acception{2}
\paradigme{\textit{dir :} \jya nɯ-}
\begin{définition}\fra enlever (habits)\end{définition}
\begin{définition}\cmn 脱(衣服)\end{définition}
\begin{exemple}\jya a-ŋga nɯ-tɕat-a\cmn 我脱了衣服\end{exemple}\acception{3}
\paradigme{\textit{dir :} \jya tɤ-}
\begin{définition}\fra attraper, pêcher\end{définition}
\begin{définition}\cmn 捕到\end{définition}
\begin{exemple}\jya qaɟy tɤ-tɕat-a\cmn 我钓到鱼了\end{exemple}\acception{4}
\paradigme{\textit{dir :} \jya thɯ-}
\begin{définition}\fra élever\end{définition}
\begin{définition}\cmn 抚养(孩子)\end{définition}
\begin{exemple}\jya a-mu a-wa cho a-pi ra kɯ thɯ́-wɣ-tɕat-a-nɯ ŋu\cmn 我父母和姐姐把我抚养成人\end{exemple}\acception{5}
\paradigme{\textit{dir :} \jya thɯ-}
\paradigme{\textit{dir :} \jya pɯ-}
\begin{définition}\fra bannir de la maison, sortir de la maison\end{définition}
\begin{définition}\cmn 赶出家门\end{définition}
\begin{exemple}\jya ɯʑo taʁndo mɯ-pjɤ-tso tɕe ɯ-ɣi ra kɯ pjɤ́-wɣ-tɕɤt\cmn 他不听指挥,家人就把他赶出家门了\end{exemple}
\begin{exemple}\jya tʂapa ɯ-ŋgɯ fsapaʁ ra chɯ́-wɣ-tɕɤt ɲɯ-ra\cmn 要把圈里的牲畜赶出来\end{exemple}
\begin{exemple}\jya kha ɯ-pɕi chɤ-tɕɤt\cmn 把他赶出家门了\end{exemple}\acception{6}
\paradigme{\textit{dir :} \jya nɯ-}
\begin{définition}\fra causatif\end{définition}
\begin{définition}\cmn 使\end{définition}
\begin{exemple}\jya nɤʑo pjɯ-kɯ-nɯβlu-a mɤ-kɯ-ŋgrɯ ɲɯ-tɕat-a ra\cmn 我不会让你骗我的\end{exemple}
\begin{exemple}\jya kɤ-ɤlɯlɤt mɤ-kɯ-ra ɲɤ-tɯ-tɕɤt\cmn 你们令他们不要打架\end{exemple}
\begin{relation-sémantique}\synonyme{
\hyperlink{Ⓔsɤβzu}{\textit{ \papi{sɤβzu}}}
}\end{relation-sémantique}\acception{7}
\paradigme{\textit{dir :} \jya tɤ-}
\paradigme{\textit{dir :} \jya thɯ-}
\begin{définition}\fra publier\end{définition}
\begin{définition}\cmn 出版\end{définition}
\begin{exemple}\jya jɯɣi to-tɕɤt (chɤ-tɕɤt)\cmn 他出版了一本书\end{exemple}\acception{8}
\paradigme{\textit{dir :} \jya tɤ-}
\begin{définition}\fra choisir (nom)\end{définition}
\begin{définition}\cmn 取(名)\end{définition}
\begin{exemple}\jya ɯ-rmi tɤ-tɕat-a\cmn 我给他取了名字\end{exemple}\acception{9}
\paradigme{\textit{dir :} \jya thɯ-}
\begin{définition}\fra finir\end{définition}
\begin{définition}\cmn 结束\end{définition}
\begin{exemple}\jya kɤ-rɤndɯn ɯ-qa chɤ-tɕɤt pjɤ-ra\cmn 他把经念到最后\end{exemple}
\begin{exemple}\jya a-ɣe kɤ-ndo nɯ, ɯ-ndo ʑo thɯ-tɕat-a pɯ-ra\cmn 我把孙子养到他长大成人\end{exemple}
\begin{exemple}\jya kɤ-nɤma ɯ-jme chɯ-tɕat-a ra\cmn 我要把事情做到底\end{exemple}
\begin{relation-sémantique}\confer{
 \papi{rŋama,tɕɤt}
}\end{relation-sémantique}
\begin{relation-sémantique}\synonyme{
\hyperlink{Ⓔsɯɣjɤɣ}{\textit{ \papi{sɯɣjɤɣ}}}
}\end{relation-sémantique}
\begin{sous-entrée}
\vedette{\hypertarget{}{\papi{ sɯtɕɤt}}}\markboth{sɯtɕɤt}{}\classe{vt}
\begin{définition}\ 
\begin{déclaration}\grammar{caus}\end{déclaration}\end{définition}
\end{sous-entrée}\begin{sous-entrée}
\vedette{\hypertarget{}{\papi{ ɯ-βlu,sɯtɕɤt}}}\markboth{ɯ-βlu,sɯtɕɤt}{}
\paradigme{\textit{dir :} \jya tɤ-}
\begin{définition}\ 
\begin{déclaration}\grammar{habil}\end{déclaration}\end{définition}
\begin{définition}\fra parvenir à trouver une solution\end{définition}
\begin{définition}\cmn 想得出办法\end{définition}
\begin{exemple}\jya a-βlu mɯ́j-sɯtɕat-a\cmn 我想不出办法来,我犹豫不决\end{exemple}
\begin{exemple}\jya ki kɤ-nɤma tɕhi tu-ste-a ra ma a-βlu mɯ́j-sɯtɕat-a\cmn 我想不出怎么做这个工作\end{exemple}
\begin{exemple}\jya a-βlu tɤ-sɯtɕa-t-a\cmn 我想出了办法\end{exemple}
\begin{exemple}\jya a-βlu ɲɯ-sɯtɕa-t-a\cmn 我想得出办法\end{exemple}
\begin{exemple}\jya nɤ-βlu ɯ-ɲɯ-tɯ-sɯtɕɤt\cmn 你想得出办法吗?\end{exemple}
\begin{relation-sémantique}\ComponentA{\classe{np}
\hyperlink{Ⓔɯ-βlu}{\textit{ \papi{ɯ-βlu}}}
}\end{relation-sémantique}
\begin{relation-sémantique}\ComponentB{\classe{vt}
\hyperlink{Ⓔsɯtɕɤt}{\textit{ \papi{sɯtɕɤt}}}
}\end{relation-sémantique}\classe{vt}
\end{sous-entrée}\begin{sous-entrée}
\vedette{\hypertarget{}{\papi{ ɯ-βlu,tɕɤt}}}\markboth{ɯ-βlu,tɕɤt}{}
\begin{définition}\fra trouver une solution\end{définition}
\begin{définition}\cmn 想办法,出主意\end{définition}
\begin{exemple}\jya nɤ-βlu ci tɤ-tɕɤt\cmn 你想一下办法\end{exemple}
\begin{relation-sémantique}\ComponentA{\classe{np}
\hyperlink{Ⓔɯ-βlu}{\textit{ \papi{ɯ-βlu}}}
}\end{relation-sémantique}
\begin{relation-sémantique}\ComponentB{\classe{vt}
\hyperlink{Ⓔtɕɤt}{\textit{ \papi{tɕɤt}}}
}\end{relation-sémantique}
\end{sous-entrée}\begin{sous-entrée}
\vedette{\hypertarget{}{\papi{ ʑɣɤsɯtɕɤt}}}\markboth{ʑɣɤsɯtɕɤt}{}\classe{vs}
\begin{définition}\ 
\begin{déclaration}\grammar{caus}\end{déclaration}
\begin{déclaration}\grammar{refl}\end{déclaration}\end{définition}
\end{sous-entrée}\end{entrée}

\begin{entrée}
\vedette{\hypertarget{Ⓔtɕɤthi}{\papi{ tɕɤthi}}}\markboth{tɕɤthi}{}\classe{adv}
\begin{définition}\fra en aval\end{définition}
\begin{définition}\cmn 在下游\end{définition}
\begin{relation-sémantique}\confer{
\hyperlink{Ⓔathi}{\textit{ \papi{athi}}}
}\end{relation-sémantique}\end{entrée}

\begin{entrée}
\vedette{\hypertarget{Ⓔtɕɤtsaʁ}{\papi{ tɕɤtsaʁ}}}\markboth{tɕɤtsaʁ}{}
\classe{n}
\begin{définition}\fra petites tresses (coiffure de femme)\end{définition}
\begin{définition}\cmn 女人在额头上辫的小辫子\end{définition}\end{entrée}

\begin{entrée}
\vedette{\hypertarget{Ⓔtɕendɤre}{\papi{ tɕendɤre}}}\markboth{tɕendɤre}{}\classe{cnj}
\begin{définition}\fra ensuite\end{définition}
\begin{définition}\cmn 然后\end{définition}\end{entrée}

\begin{entrée}
\vedette{\hypertarget{Ⓔtɕeri}{\papi{ tɕeri}}}\markboth{tɕeri}{}\classe{cnj}
\begin{définition}\fra mais\end{définition}
\begin{définition}\cmn 但是\end{définition}
\end{entrée}

\begin{entrée}
\vedette{\hypertarget{Ⓔtɕetha}{\papi{ tɕetha}}}\markboth{tɕetha}{}\classe{n}
\begin{définition}\fra (action de) sonder les gens\end{définition}
\begin{définition}\cmn 试探人\end{définition}
\begin{exemple}\jya tɕetha kɤ-βzu mɤ-mbat\cmn 试探人是不容易的\end{exemple}
\begin{relation-sémantique}\confer{
\hyperlink{Ⓔnɯtɕetha}{\textit{ \papi{nɯtɕetha}}}
}\end{relation-sémantique}\end{entrée}

\begin{entrée}
\vedette{\hypertarget{Ⓔtɕɣaʁ}{\papi{ tɕɣaʁ}}}\markboth{tɕɣaʁ}{}
\classe{vt}
\paradigme{\textit{dir :} \jya tɤ-}
\paradigme{\textit{dir :} \jya nɯ-}
\begin{définition}\fra extraire en pressant sur\end{définition}
\begin{définition}\cmn 挤出来\end{définition}
\begin{exemple}\jya nɤ-jaʁ to-rɤspɯ, tɤ-tɕɣaʁ\cmn 你手上化脓,你挤一下\end{exemple}
\begin{exemple}\jya tɯ-ɣmbɤβ tɤ-tɕɣaʁ-a\cmn 我挤了脓肿\end{exemple}
\begin{exemple}\jya sɯmat nɯ-tɕɣaʁ-a\cmn 我把果子挤出来了\end{exemple}
\begin{relation-sémantique}\confer{
\hyperlink{Ⓔndʑɣaʁ}{\textit{ \papi{ndʑɣaʁ}}}
}\end{relation-sémantique}\end{entrée}

\begin{entrée}
\vedette{\hypertarget{Ⓔtɕɣɤrtɕɣɤr}{\papi{ tɕɣɤrtɕɣɤr}}}\markboth{tɕɣɤrtɕɣɤr}{}\classe{idph.2}
\begin{définition}\fra rouge vif\end{définition}
\begin{définition}\cmn 形容(颜色)红艳艳\end{définition}
\begin{exemple}\jya ɯ-ŋga ɲɯ-ɣɯrni tɕɣɤrtɕɣɤr ʑo\cmn 衣服颜色非常红\end{exemple}\begin{sous-entrée}
\vedette{\hypertarget{}{\papi{ ɣɤtɕɣɤrtɕɣɤr}}}\markboth{ɣɤtɕɣɤrtɕɣɤr}{}\classe{vi}\acception{1}
\begin{définition}\fra hurler\end{définition}
\begin{définition}\cmn 叫唤\end{définition}\acception{2}
\begin{définition}\fra scintiller\end{définition}
\begin{définition}\cmn 一闪一闪地发光\end{définition}
\begin{exemple}\jya paʁ ɲɯ-ɣɤwu ɲɯ-ɣɤtɕɣɤrtɕɣɤr\cmn 猪在叫唤(要宰猪的时候)\end{exemple}
\begin{exemple}\jya tɤrmbja ɲɯ-ɣɤtɕɣɤrtɕɣɤr (ʑo ɲɯ-ɤsɯ-βzu)\cmn 一道道电光闪过。\end{exemple}
\end{sous-entrée}\begin{sous-entrée}
\vedette{\hypertarget{}{\papi{ sɤtɕɣɤrtɕɣɤr}}}\markboth{sɤtɕɣɤrtɕɣɤr}{}\classe{vt}
\end{sous-entrée}\begin{sous-entrée}
\vedette{\hypertarget{}{\papi{ tɕɣɤrnɤtɕɣɤr}}}\markboth{tɕɣɤrnɤtɕɣɤr}{}\classe{idph.3}
\begin{exemple}\jya tɤrmbja tɕɣɤrnɤtɕɣɤr ɲɯ-ɤsɯ-βzu\cmn 形容闪电一道一道地闪过发光\end{exemple}
\end{sous-entrée}\end{entrée}

\begin{entrée}
\vedette{\hypertarget{Ⓔtɕɣɤtɕɣɤt}{\papi{ tɕɣɤtɕɣɤt}}}\markboth{tɕɣɤtɕɣɤt}{}
\classe{idph.2}
\begin{définition}\fra qui retient ses larmes\end{définition}
\begin{définition}\cmn 形容(眼泪)含在眼眶里,快要流出来的样子\end{définition}
\begin{exemple}\jya ɯ-qom tɕɣɤtɕɣɤt ʑo to-stu\cmn 他的眼泪含在眼眶里\end{exemple}\end{entrée}

\begin{entrée}
\vedette{\hypertarget{Ⓔtɕɣom}{\papi{ tɕɣom}}}\markboth{tɕɣom}{}
\classe{n}
\begin{définition}\fra xanthoxyle\end{définition}
\begin{définition}\cmn 花椒\end{définition}
\begin{exemple}\jya tɕɣom nɯ si mɤ-kɯ-mbro ci ŋu, ɯ-mdzu tu, ɯ-ru ɯ-taʁ ɯ-mdzu rkɯn, ɯ-rtaʁ ɯ-taʁ ɯ-mdzu dɤn, tɕɣom nɯ sɯŋgɯ tɕɣom ci tu, kha ɯ-rkɯ pɯ-kɤ-nɯ-ji tɕɣom ci tu, pɯ-kɤ-nɯji tɕɣom nɯ tɕɣomte rmi, sɯŋgɯ tɕɣom nɯ tɕɣomzaʁ rmi, tɕɣomzaʁ ɣɯ ɯ-mat ndɯβ, tɕɣomte ɣɯ ɯ-mat jndʐɤz, thɯ-tɯt tɕe, chɯ-ɣɯrni ŋu, ɯ-mat ɯ-pɕi zɯ ɯ-kri tu, tɕe kú-wɣ-rtoʁ tɕe nɤmbju ʑo. ɲɯ́-wɣ-phɯt tɕe, ɯ-kri nɯ tɯ-mɲaʁ ɯ-ŋgɯ ku-ɕe tɕe toʁde wuma ʑo ɕɯmŋɤm. tú-wɣ-ndza tɕe mɤrtsaβ tɕe ɣɤzɯβzɯβ ʑo, ɯ-dɯχɯn mɯm. ndzɤtshi ɯ-ŋgɯ kú-wɣ-nɯ-lɤt tɕe mɯm.\cmn 
花椒是一种矮小的树木,有刺,树干上刺不多,枝桠上刺多,有野生花椒,也有家边自己种的。自己种的叫\stylefv{tɕɣomte}(真花椒),野生的叫\stylefv{tɕɣomzaʁ}。野花椒的果实小一些,家花椒的果实大一些。成熟了就变红,果实外面有一层油,看起来有光泽。摘花椒的时候,油进了眼睛,会痛一阵子。吃起来是麻的,很香。放在食物里很香。
\end{exemple}
\begin{relation-sémantique}\confer{
\hyperlink{Ⓔtɕɣomte}{\textit{ \papi{tɕɣomte}}}
}\end{relation-sémantique}
\begin{relation-sémantique}\confer{
\hyperlink{Ⓔtɕɣomzaʁ}{\textit{ \papi{tɕɣomzaʁ}}}
}\end{relation-sémantique}
\begin{relation-sémantique}\confer{
\hyperlink{Ⓔnɯtɕɣom}{\textit{ \papi{nɯtɕɣom}}}
}\end{relation-sémantique}\end{entrée}

\begin{entrée}
\vedette{\hypertarget{Ⓔtɕɣomte}{\papi{ tɕɣomte}}}\markboth{tɕɣomte}{}\classe{n}
\begin{définition}\fra xanthoxyle cultivé\end{définition}
\begin{définition}\cmn 家花椒\end{définition}
\end{entrée}

\begin{entrée}
\vedette{\hypertarget{Ⓔtɕɣomzaʁ}{\papi{ tɕɣomzaʁ}}}\markboth{tɕɣomzaʁ}{}\classe{n}
\begin{définition}\fra xanthoxyle sauvage\end{définition}
\begin{définition}\cmn 野花椒\end{définition}
\end{entrée}

\begin{entrée}
\vedette{\hypertarget{Ⓔtɕhaŋβɟaj}{\papi{ tɕhaŋβɟaj}}}\markboth{tɕhaŋβɟaj}{}\classe{n}
\begin{définition}\fra outil pour brasser la bière d'orge\end{définition}
\begin{définition}\cmn 搅拌青稞酒的工具\end{définition}\end{entrée}

\begin{entrée}
\vedette{\hypertarget{Ⓔtɕhaŋkha}{\papi{ tɕhaŋkha}}}\markboth{tɕhaŋkha}{}\classe{n}
\begin{définition}\fra entonnoir\end{définition}
\begin{définition}\cmn 漏斗\end{définition}
\begin{exemple}\jya cha nɯ tɕhaŋkha kɯ tɤ-sɯ-rku-t-a\cmn 我用漏斗倒了酒进去\end{exemple}
\begin{exemple}\jya tɕhaŋkha a-pɯ-tu tɕe, cha kɤ-rku tɤ-mda tɕe mɤ-jit\cmn 有漏斗的话,倒酒就不会洒\end{exemple}\end{entrée}

\begin{entrée}
\vedette{\hypertarget{Ⓔtɕhaʁ}{\papi{ tɕhaʁ}}}\markboth{tɕhaʁ}{}\classe{vi}
\paradigme{\textit{dir :} \jya tɤ-}
\paradigme{\textit{dir :} \jya nɯ-}\acception{1}
\begin{définition}\fra diminuer\end{définition}
\begin{définition}\cmn 减少\end{définition}
\begin{exemple}\jya tɯ-rdoʁ ɲɤ-tɕhaʁ\cmn 少了一个\end{exemple}\acception{2}
\begin{définition}\fra supporter, pouvoir rester en place\end{définition}
\begin{définition}\cmn 稳得住;坐得住‘忍得住
\begin{déclaration} \étymologie{\papi{tɕʰag}}\end{déclaration}\end{définition}
\begin{exemple}\jya mɯ́j-tɕhaʁ-a\cmn 我坐不住\end{exemple}
\begin{exemple}\jya tɤ-pɤtso kɤ-rɤʑi mɯ́j-tɕhaʁ\cmn 小孩子坐不住\end{exemple}
\begin{exemple}\jya ɯ-tɯ-rga kɯ mɯ-pjɤ-tɕhaʁ\cmn 她高兴得忍不住(说漏了)\end{exemple}
\begin{relation-sémantique}\confer{
\hyperlink{Ⓔɯ-tɕhaʁⒽ2}{\textit{ \papi{ɯ-tɕhaʁ2}}}
}\end{relation-sémantique}
\begin{relation-sémantique}\confer{
\hyperlink{Ⓔsɯxtɕhaʁ}{\textit{ \papi{sɯxtɕhaʁ}}}
}\end{relation-sémantique}
\begin{relation-sémantique}\confer{
\hyperlink{Ⓔɣɤtɕhaʁ}{\textit{ \papi{ɣɤtɕhaʁ}}}
}\end{relation-sémantique}
\begin{relation-sémantique}\confer{
\hyperlink{Ⓔzndɤtɕhaʁ}{\textit{ \papi{zndɤtɕhaʁ}}}
}\end{relation-sémantique}\end{entrée}

\begin{entrée}
\vedette{\hypertarget{Ⓔtɕhaʁla}{\papi{ tɕhaʁla}}}\markboth{tɕhaʁla}{}\classe{n}
\begin{définition}\fra cour\end{définition}
\begin{définition}\cmn 院子\end{définition}
\begin{relation-sémantique}\synonyme{
\hyperlink{Ⓔrɟara}{\textit{ \papi{rɟara}}}
}\end{relation-sémantique}\end{entrée}

\begin{entrée}
\vedette{\hypertarget{Ⓔtɕhaʁra}{\papi{ tɕhaʁra}}}\markboth{tɕhaʁra}{}\classe{n}
\begin{définition}\fra toilettes\end{définition}
\begin{définition}\cmn 厕所\end{définition}\end{entrée}

\begin{entrée}
\vedette{\hypertarget{Ⓔtɕhaχɕaŋ}{\papi{ tɕhaχɕaŋ}}}\markboth{tɕhaχɕaŋ}{}\classe{n}
\begin{définition}\fra attelle\end{définition}
\begin{définition}\cmn (骨折)夹板\end{définition}\end{entrée}

\begin{entrée}
\vedette{\hypertarget{Ⓔtɕhaχpu}{\papi{ tɕhaχpu}}}\markboth{tɕhaχpu}{}\classe{n}
\begin{définition}\fra handicapé\end{définition}
\begin{définition}\cmn 残疾人\end{définition}
\begin{relation-sémantique}\confer{
\hyperlink{Ⓔɯ-tɕhaʁⒽ2}{\textit{ \papi{ɯ-tɕhaʁ2}}}
}\end{relation-sémantique}\end{entrée}

\begin{entrée}
\vedette{\hypertarget{Ⓔtɕhaχɯ}{\papi{ tɕhaχɯ}}}\markboth{tɕhaχɯ}{}\classe{n}
\begin{définition}\fra théière\end{définition}
\begin{définition}\cmn 茶壶
\begin{déclaration} \étymologie{\papi{\stylefn{茶壶}}}\end{déclaration}\end{définition}
\end{entrée}

\begin{entrée}
\vedette{\hypertarget{Ⓔtɕhɤfɕɤt}{\papi{ tɕhɤfɕɤt}}}\markboth{tɕhɤfɕɤt}{}\classe{n}
\begin{définition}\fra débats philolophiques sur le bouddhisme\end{définition}
\begin{définition}\cmn 辩经\end{définition}
\begin{exemple}\jya χpɯn ra tɕhɤfɕɤt ɲɯ-ɤsɯ-βzu-nɯ (=ɲɯ-rɯtɕhɤfɕɤt-nɯ)\cmn 和尚们在辩经\end{exemple}
\begin{relation-sémantique}\confer{
\hyperlink{Ⓔrɯtɕhɤfɕɤt}{\textit{ \papi{rɯtɕhɤfɕɤt}}}
}\end{relation-sémantique}
\begin{relation-sémantique}\confer{
\hyperlink{ⒺfɕɤtⒽ1}{\textit{ \papi{fɕɤt1}}}
}\end{relation-sémantique}\end{entrée}

\begin{entrée}
\vedette{\hypertarget{Ⓔtɕhɤɣdɯ}{\papi{ tɕhɤɣdɯ}}}\markboth{tɕhɤɣdɯ}{}\classe{n}
\begin{définition}\fra jarre de vin\end{définition}
\begin{définition}\cmn 酒坛子\end{définition}
\begin{relation-sémantique}\synonyme{
\hyperlink{Ⓔtɕhorzi}{\textit{ \papi{tɕhorzi}}}
}\end{relation-sémantique}\begin{sous-entrée}
\vedette{\hypertarget{}{\papi{ tɯ-tɕhɤɣdɯ}}}\markboth{tɯ-tɕhɤɣdɯ}{}\classe{clf}
\end{sous-entrée}\end{entrée}

\begin{entrée}
\vedette{\hypertarget{Ⓔtɕhɤjlɯz}{\papi{ tɕhɤjlɯz}}}\markboth{tɕhɤjlɯz}{}
\classe{n}
\begin{définition}\fra coutume\end{définition}
\begin{définition}\cmn 风俗
\begin{déclaration} \étymologie{\papi{tɕʰos.lugs}}\end{déclaration}\end{définition}
\begin{exemple}\jya aʑo kɯɕɯŋgɯ ɣɯ ɯ-tɕhɤjlɯz nɯ kɤ-βzu rga-a\cmn 我喜欢根据古代的风俗穿衣服打扮\end{exemple}
\begin{relation-sémantique}\confer{
\hyperlink{Ⓔnɯtɕhɤjlɯz}{\textit{ \papi{nɯtɕhɤjlɯz}}}
}\end{relation-sémantique}\end{entrée}

\begin{entrée}
\vedette{\hypertarget{Ⓔtɕhɤjʁo}{\papi{ tɕhɤjʁo}}}\markboth{tɕhɤjʁo}{}
\classe{n}
\begin{définition}\fra état d'ébriété\end{définition}
\begin{définition}\cmn 发酒疯\end{définition}
\begin{relation-sémantique}\confer{
\hyperlink{Ⓔnɯtɕhɤjʁo}{\textit{ \papi{nɯtɕhɤjʁo}}}
}\end{relation-sémantique}\end{entrée}

\begin{entrée}
\vedette{\hypertarget{Ⓔtɕhɤmkɯm}{\papi{ tɕhɤmkɯm}}}\markboth{tɕhɤmkɯm}{}
\classe{n}
\begin{définition}\fra jarre\end{définition}
\begin{définition}\cmn 坛子\end{définition}\end{entrée}

\begin{entrée}
\vedette{\hypertarget{Ⓔtɕhɤmlaŋ}{\papi{ tɕhɤmlaŋ}}}\markboth{tɕhɤmlaŋ}{}
\classe{n}
\begin{définition}\fra gobelet\end{définition}
\begin{définition}\cmn 缸子\end{définition}\end{entrée}

\begin{entrée}
\vedette{\hypertarget{Ⓔtɕhɤphɯ}{\papi{ tɕhɤphɯ}}}\markboth{tɕhɤphɯ}{}\classe{n}
\begin{définition}\fra prix à payer, avantage\end{définition}
\begin{définition}\cmn 代价;收获\end{définition}
\begin{exemple}\jya ɯ-tɕhɤphɯ me\cmn 不值得\end{exemple}
\begin{relation-sémantique}\confer{
\hyperlink{Ⓔɯ-phɯ}{\textit{ \papi{ɯ-phɯ}}}
}\end{relation-sémantique}\end{entrée}

\begin{entrée}
\vedette{\hypertarget{Ⓔtɕhɤrɕɤt}{\papi{ tɕhɤrɕɤt}}}\markboth{tɕhɤrɕɤt}{}\classe{n}
\begin{définition}\fra pluie torrentielle avec des grêlons\end{définition}
\begin{définition}\cmn 夹着小冰雹的猛雨\end{définition}\end{entrée}

\begin{entrée}
\vedette{\hypertarget{Ⓔtɕhɤrnaʁ}{\papi{ tɕhɤrnaʁ}}}\markboth{tɕhɤrnaʁ}{}\classe{n}
\begin{définition}\fra pluie\end{définition}
\begin{définition}\cmn 雨
\begin{déclaration} \étymologie{\papi{tɕʰar.nag}}\end{déclaration}\end{définition}
\begin{exemple}\jya tɕhɤrnaʁ ɲɯ-ɤsɯ-lɤt\cmn 下雨\end{exemple}\end{entrée}

\begin{entrée}
\vedette{\hypertarget{Ⓔtɕhɤrprɯ}{\papi{ tɕhɤrprɯ}}}\markboth{tɕhɤrprɯ}{}
\classe{n}
\begin{définition}\fra abri de pluie\end{définition}
\begin{définition}\cmn 遮雨的地方\end{définition}\end{entrée}

\begin{entrée}
\vedette{\hypertarget{Ⓔtɕhɤrʁu}{\papi{ tɕhɤrʁu}}}\markboth{tɕhɤrʁu}{}
\classe{n}
\begin{définition}\fra habit de femme en laine\end{définition}
\begin{définition}\cmn 女式藏装,毛织品\end{définition}\end{entrée}

\begin{entrée}
\vedette{\hypertarget{Ⓔtɕhɤrʁɯ}{\papi{ tɕhɤrʁɯ}}}\markboth{tɕhɤrʁɯ}{}\classe{n}
\begin{définition}\fra habit de femme en tissu\end{définition}
\begin{définition}\cmn 氆氇、羊毛、牛毛制成的女装\end{définition}
\end{entrée}

\begin{entrée}
\vedette{\hypertarget{Ⓔtɕhɤrzɤthɯm}{\papi{ tɕhɤrzɤthɯm}}}\markboth{tɕhɤrzɤthɯm}{}\classe{n}
\begin{définition}\fra bouchon au fond des jarres d'alcool\end{définition}
\begin{définition}\cmn 酒缸底部的塞子\end{définition}
\begin{relation-sémantique}\confer{
\hyperlink{Ⓔtɕhorzi}{\textit{ \papi{tɕhorzi}}}
}\end{relation-sémantique}
\begin{relation-sémantique}\confer{
\hyperlink{Ⓔɯ-thɯm}{\textit{ \papi{ɯ-thɯm}}}
}\end{relation-sémantique}\end{entrée}

\begin{entrée}
\vedette{\hypertarget{Ⓔtɕhɤska}{\papi{ tɕhɤska}}}\markboth{tɕhɤska}{}\classe{n}
\begin{définition}\fra clarinette\end{définition}
\begin{définition}\cmn 唢呐\end{définition}
\end{entrée}

\begin{entrée}
\vedette{\hypertarget{ⒺtɕhɤtⒽ2}{\papi{ tɕhɤt}}}\markboth{tɕhɤt}{}\homonyme{2}
\classe{n}
\begin{définition}\fra rendez-vous\end{définition}
\begin{définition}\cmn 约会\end{définition}
\begin{exemple}\jya tɕhɤt nɯ-βzu-tɕi\cmn 我们俩约了时间\end{exemple}
\end{entrée}

\begin{entrée}
\vedette{\hypertarget{ⒺtɕhɤtⒽ1}{\papi{ tɕhɤt}}}\markboth{tɕhɤt}{}\homonyme{1}
\classe{vi}\acception{1}
\paradigme{\textit{dir :} \jya nɯ-}
\paradigme{\textit{dir :} \jya lɤ-}
\begin{définition}\fra s'effondrer de fatigue\end{définition}
\begin{définition}\cmn 累倒
\begin{déclaration}\use{沙尔宗方言}\end{déclaration}
\begin{déclaration} \étymologie{\papi{tɕʰad}}\end{déclaration}\end{définition}
\begin{exemple}\jya ɲɤ-tɕhat-a\cmn 我累倒了\end{exemple}
\begin{exemple}\jya mbro lo-tɕhɤt\cmn 马走不动了\end{exemple}\acception{2}
\begin{définition}\fra disparaître (culture)\end{définition}
\begin{définition}\cmn 消失(庄稼)\end{définition}
\begin{exemple}\jya tɯpɕi jinde mɯ-la-ji-nɯ tɕe, ɲɤ-tɕhɤt\cmn 现在不种亚麻了,连种子都没有\end{exemple}
\begin{relation-sémantique}\synonyme{
\hyperlink{Ⓔɲat}{\textit{ \papi{ɲat}}}
}\end{relation-sémantique}\begin{sous-entrée}
\vedette{\hypertarget{}{\papi{ sɯxtɕhɤt}}}\markboth{sɯxtɕhɤt}{}\classe{vt}
\paradigme{\textit{dir :} \jya nɯ-}
\begin{définition}\fra supprimer\end{définition}
\begin{définition}\cmn 消除\end{définition}
\begin{exemple}\jya ji-@baicai pɯ-dɤn ri, ɲɤ-sɯxtɕhɤt-nɯ\cmn 以前我们的白菜很多,但他们把它们拔除了\end{exemple}
\end{sous-entrée}\end{entrée}

\begin{entrée}
\vedette{\hypertarget{Ⓔtɕhɤtpa}{\papi{ tɕhɤtpa}}}\markboth{tɕhɤtpa}{}
\classe{n}
\begin{définition}\fra punition\end{définition}
\begin{définition}\cmn 惩罚
\begin{déclaration} \étymologie{\papi{tɕʰad.pa}}\end{déclaration}\end{définition}
\begin{exemple}\jya nɤ-tɕhɤtpa βze (=tɯ́-wɣ-znɯtɕhɤl)\cmn 他会惩罚你的\end{exemple}
\begin{relation-sémantique}\confer{
\hyperlink{Ⓔznɯtɕhɤtpa}{\textit{ \papi{znɯtɕhɤtpa}}}
}\end{relation-sémantique}
\begin{relation-sémantique}\synonyme{
\hyperlink{Ⓔɯ-tɕhɤl}{\textit{ \papi{ɯ-tɕhɤl}}}
}\end{relation-sémantique}\end{entrée}

\begin{entrée}
\vedette{\hypertarget{Ⓔtɕhɤz}{\papi{ tɕhɤz}}}\markboth{tɕhɤz}{}\classe{vs}
\begin{définition}\ 
\begin{déclaration}\use{只用于否定式}\end{déclaration}\end{définition}
\begin{définition}\fra contraire au bouddhisme\end{définition}
\begin{définition}\cmn 违背佛教\end{définition}
\begin{exemple}\jya mɤ-kɯ-tɕhɤz me\cmn 不违背佛教\end{exemple}
\begin{exemple}\jya sroχtɕɤn kɤ-lɤt mɤ-tɕhɤz\cmn 杀生是违背佛教的行为\end{exemple}\end{entrée}

\begin{entrée}
\vedette{\hypertarget{Ⓔtɕhemɤli}{\papi{ tɕhemɤli}}}\markboth{tɕhemɤli}{}
\classe{n}
\begin{définition}\fra jeune fille\end{définition}
\begin{définition}\cmn 姑娘\end{définition}\end{entrée}

\begin{entrée}
\vedette{\hypertarget{Ⓔtɕhemɤpɯ}{\papi{ tɕhemɤpɯ}}}\markboth{tɕhemɤpɯ}{}\classe{n}
\begin{définition}\fra jeune fille\end{définition}
\begin{définition}\cmn 姑娘\end{définition}
\begin{relation-sémantique}\confer{
\hyperlink{Ⓔtɕheme}{\textit{ \papi{tɕheme}}}
}\end{relation-sémantique}\end{entrée}

\begin{entrée}
\vedette{\hypertarget{Ⓔtɕheme}{\papi{ tɕheme}}}\markboth{tɕheme}{}\classe{n}
\begin{définition}\fra fille\end{définition}
\begin{définition}\cmn 妇女\end{définition}
\end{entrée}

\begin{entrée}
\vedette{\hypertarget{Ⓔtɕhemekɤtsa}{\papi{ tɕhemekɤtsa}}}\markboth{tɕhemekɤtsa}{}\classe{n}
\begin{définition}\fra une plante\end{définition}
\begin{définition}\cmn 植物的一种\end{définition}
\begin{exemple}\jya tɕhemekɤtsa nɯ sɯjno kɯ-mbɯ-mbɤr ci ŋu. ɯ-ru kɯ-ɤβʑɯrdu fse, ɯ-mdoʁ ɣɯrni. ɯ-βzɯr kɯ-fse kɯ-tu ŋu. ɯ-ru tu-ɬoʁ tɕe, ɯ-jwaʁ nɯ ɯ-ru ɯ-taʁ tɯ-rtsɤɣ tɯ-rtsɤɣ ku-fskɤr tɕe ku-ndzoʁ ŋu. ɯ-jwaʁ ɯ-rchɤβ nɯ tɕu ɯ-mɯntoʁ ku-oʑɯrja ŋu, tɕe ɯ-mɯntoʁ ku-kɤ-fskɤr nɯ ndʐa tɕhemekɤtsa ɲɯ-rmi. ɯ-ru tɯ-ldʑa ɯ-taʁ nɯ ɯ-jwaʁ χsɯ-tɤxɯr kɯβde-tɤxɯr jamar ma me, ɯ-jwaʁ nɯ kɯ-ɤɲaʁndzɯm ŋu.\cmn 
\stylefv{tɕheme kɤtsa}是矮小的植物。茎四方形、红色、有棱角。茎长出来时,叶子就在茎上一节绕一圈地生长。花排列在叶子之间。因为花是绕着茎而长的,所以叫它\stylefv{tɕheme kɤ-tsa}(母女难分的意思)。一根茎上只有三四圈叶子,叶子是深绿色的。
\end{exemple}\end{entrée}

\begin{entrée}
\vedette{\hypertarget{Ⓔtɕhɣaʁtɕhɣaʁ}{\papi{ tɕhɣaʁtɕhɣaʁ}}}\markboth{tɕhɣaʁtɕhɣaʁ}{}
\classe{idph.2}
\begin{définition}\fra propre, sans dommage\end{définition}
\begin{définition}\cmn 形容完整、干净的样子,没有受到损伤\end{définition}
\begin{exemple}\jya tɕhɣaʁtɕhɣaʁ ɲɤ-χtɕi\cmn 他洗得很干净了\end{exemple}
\begin{exemple}\jya si tɕhɣaʁtɕhɣaʁ lo-ftɕhur\cmn (原来倒下来的)柴全部立起来了\end{exemple}\end{entrée}

\begin{entrée}
\vedette{\hypertarget{ⒺtɕhiⒽ1}{\papi{ tɕhi}}}\markboth{tɕhi}{}\homonyme{1}
\classe{pro}
\begin{définition}\fra quoi\end{définition}
\begin{définition}\cmn 什么
\begin{déclaration} \étymologie{\papi{tɕʰi}}\end{déclaration}
\begin{déclaration}\use{同\stylefv{maʁ nɤ}连用时有“至少”的意思}\end{déclaration}\end{définition}
\begin{exemple}\jya kutɕu tɕhi ɯ-kɯ-pa jɤ-tɯ-ɣe?\cmn 你来这里做什么?\end{exemple}
\begin{exemple}\jya tɕhi kɯ-ra ʑo kɤ-nɤma cha\cmn 她什么都能做\end{exemple}
\begin{exemple}\jya tɕhi ɯ-skɤt ŋu, ɯ-ɲɯ́-tɯ-tso?\cmn 你知道什么意思吗?\end{exemple}
\end{entrée}

\begin{entrée}
\vedette{\hypertarget{ⒺtɕhiⒽ2}{\papi{ tɕhi}}}\markboth{tɕhi}{}\homonyme{2}
\classe{n}
\begin{définition}\fra escalier taillé dans un tronc\end{définition}
\begin{définition}\cmn 用树干刻成的楼梯\end{définition}\end{entrée}

\begin{entrée}
\vedette{\hypertarget{Ⓔtɕhi napapa}{\papi{ tɕhi napapa}}}\markboth{tɕhi napapa}{}\classe{adv}
\begin{définition}\fra quoi qu'il arrive\end{définition}
\begin{définition}\cmn 不顾一切,不管发生什么\end{définition}
\begin{exemple}\jya tɕhi napapa ʑo kɤ-nɤɕqa ɬoʁ\cmn 不管发生什么都要坚持\end{exemple}\end{entrée}

\begin{entrée}
\vedette{\hypertarget{Ⓔtɕhindʐa}{\papi{ tɕhindʐa}}}\markboth{tɕhindʐa}{}\classe{pro}
\begin{définition}\fra pourquoi\end{définition}
\begin{définition}\cmn 为什么
\begin{déclaration} \étymologie{\papi{tɕʰi.ⁿdra}}\end{déclaration}\end{définition}\end{entrée}

\begin{entrée}
\vedette{\hypertarget{Ⓔtɕhinthaʁ}{\papi{ tɕhinthaʁ}}}\markboth{tɕhinthaʁ}{}\classe{n}
\begin{définition}\fra corde de tente\end{définition}
\begin{définition}\cmn 用来搭帐篷的拉线\end{définition}
\end{entrée}

\begin{entrée}
\vedette{\hypertarget{Ⓔtɕhitsuku}{\papi{ tɕhitsuku}}}\markboth{tɕhitsuku}{}\classe{pro}
\begin{définition}\fra quoi que ce soit\end{définition}
\begin{définition}\cmn 无论什么,一切\end{définition}\end{entrée}

\begin{entrée}
\vedette{\hypertarget{Ⓔtɕhom}{\papi{ tɕhom}}}\markboth{tɕhom}{}
\classe{vs}
\paradigme{\textit{dir :} \jya tɤ-}
\paradigme{\textit{dir :} \jya nɯ-}
\begin{définition}\fra trop, excessif\end{définition}
\begin{définition}\cmn 过分;太\end{définition}
\begin{exemple}\jya ɯ-tɯ-xtɕi ɲo-tɕhom\cmn 变得太小\end{exemple}
\begin{exemple}\jya a-tɯ-ɤɕqhe ɲɯ-tɕhom\cmn 我咳得太多\end{exemple}\end{entrée}

\begin{entrée}
\vedette{\hypertarget{Ⓔtɕhoma}{\papi{ tɕhoma}}}\markboth{tɕhoma}{}\classe{n}
\begin{définition}\fra ceinture\end{définition}
\begin{définition}\cmn 皮带\end{définition}
\begin{exemple}\jya tɕhoma ɯ-mdʑu\cmn 腰带口子里面的钉子\end{exemple}\end{entrée}

\begin{entrée}
\vedette{\hypertarget{Ⓔtɕhomba}{\papi{ tɕhomba}}}\markboth{tɕhomba}{}
\classe{n}
\begin{définition}\fra rhume\end{définition}
\begin{définition}\cmn 感冒
\begin{déclaration} \étymologie{\papi{tɕʰam.pa}}\end{déclaration}\end{définition}\end{entrée}

\begin{entrée}
\vedette{\hypertarget{Ⓔtɕhorzi}{\papi{ tɕhorzi}}}\markboth{tɕhorzi}{}
\classe{n}
\begin{définition}\fra jarre\end{définition}
\begin{définition}\cmn 坛子\end{définition}\end{entrée}

\begin{entrée}
\vedette{\hypertarget{Ⓔtɕhoʁtɕhoʁ}{\papi{ tɕhoʁtɕhoʁ}}}\markboth{tɕhoʁtɕhoʁ}{} (\variante{ftɕhoʁftɕhoʁ}) 
\classe{idph.2}
\begin{définition}\fra les deux oreilles dressées\end{définition}
\begin{définition}\cmn 形容两只耳朵立起来,看起来很灵活的样子\end{définition}
\begin{exemple}\jya qala ɣɯ ɯ-rna nɯ tu-z-nɯndzi tɕhoʁtɕhoʁ ʑo\cmn 兔子把两只耳朵立起来了\end{exemple}
\begin{exemple}\jya sɯjno to-ɬoʁ tɕe tɕhoʁtɕhoʁ ʑo to-pa\cmn 草刚出土,直直地竖起来\end{exemple}\end{entrée}

\begin{entrée}
\vedette{\hypertarget{Ⓔtɕhoz}{\papi{ tɕhoz}}}\markboth{tɕhoz}{}\classe{n}
\begin{définition}\fra religion\end{définition}
\begin{définition}\cmn 佛学
\begin{déclaration} \étymologie{\papi{tɕʰos}}\end{déclaration}\end{définition}
\end{entrée}

\begin{entrée}
\vedette{\hypertarget{Ⓔtɕhʁɯβnɤtɕhʁɯβ}{\papi{ tɕhʁɯβnɤtɕhʁɯβ}}}\markboth{tɕhʁɯβnɤtɕhʁɯβ}{}\classe{idph.3}
\begin{définition}\fra croquant\end{définition}
\begin{définition}\cmn 形容食物脆
\begin{déclaration}\use{咀嚼的声音比\stylefv{tɕʁɯznɤtɕʁɯz}大一些}\end{déclaration}\end{définition}\begin{sous-entrée}
\vedette{\hypertarget{}{\papi{ nɯtɕhʁɯβ}}}\markboth{nɯtɕhʁɯβ}{}\classe{vt}
\paradigme{\textit{dir :} \jya pɯ-}
\begin{définition}\fra croquer\end{définition}
\begin{définition}\cmn 使劲地咀嚼(脆的食物)\end{définition}
\begin{relation-sémantique}\antonyme{
\hyperlink{Ⓔtɕʁɯβnɤtɕʁɯβ}{\textit{ \papi{tɕʁɯβnɤtɕʁɯβ}}}
}\end{relation-sémantique}
\end{sous-entrée}\end{entrée}

\begin{entrée}
\vedette{\hypertarget{Ⓔtɕhɯ}{\papi{ tɕhɯ}}}\markboth{tɕhɯ}{}\classe{vt}
\paradigme{\textit{dir :} \jya tɤ-}
\begin{définition}\fra encorner\end{définition}
\begin{définition}\cmn 用角顶(牛)\end{définition}
\begin{exemple}\jya mbala kɯ tɤ́-wɣ-tɕhɯ-a\cmn 牛用角顶了我\end{exemple}
\begin{exemple}\jya zdoŋbu kɯ tɯ-mɲaʁ mɤ-tɕhi, xɕɤndʑu kɯ tɯ-mɲaʁ tɕhi\cmn 大事影响不了人,小事会伤害到人\end{exemple}
\begin{relation-sémantique}\confer{
\hyperlink{Ⓔsɤtɕhɯ}{\textit{ \papi{sɤtɕhɯ}}}
}\end{relation-sémantique}
\begin{relation-sémantique}\confer{
\hyperlink{Ⓔnɤkɤtɕhɯ}{\textit{ \papi{nɤkɤtɕhɯ}}}
}\end{relation-sémantique}\begin{sous-entrée}
\vedette{\hypertarget{}{\papi{ atɕhɯtɕhɯ}}}\markboth{atɕhɯtɕhɯ}{}\classe{vi}
\paradigme{\textit{dir :} \jya tɤ-}
\begin{définition}\fra s'encorner les uns les autres\end{définition}
\begin{définition}\cmn 互相顶\end{définition}
\begin{exemple}\jya mbala nɯ ɲɯ-ɤtɕhɯtɕhɯ-ndʑi\cmn 公牛互相顶着\end{exemple}
\end{sous-entrée}\begin{sous-entrée}
\vedette{\hypertarget{}{\papi{ ɯ-tɕhɯ,lɤt}}}\markboth{ɯ-tɕhɯ,lɤt}{}
\begin{définition}\fra donner un coup avec de bout de ...\end{définition}
\begin{définition}\cmn 用……戳\end{définition}
\end{sous-entrée}\end{entrée}

\begin{entrée}
\vedette{\hypertarget{Ⓔtɕhɯβja}{\papi{ tɕhɯβja}}}\markboth{tɕhɯβja}{}
\classe{n}
\begin{définition}\fra espèce d'oiseau\end{définition}
\begin{définition}\cmn 一种鸟
\begin{déclaration} \étymologie{\papi{tɕʰɯ.bʲa}}\end{déclaration}\end{définition}\end{entrée}

\begin{entrée}
\vedette{\hypertarget{Ⓔtɕhɯβroʁ}{\papi{ tɕhɯβroʁ}}}\markboth{tɕhɯβroʁ}{}
\classe{n}
\begin{définition}\fra tsampa\end{définition}
\begin{définition}\cmn 糌粑的一种吃法\end{définition}
\begin{exemple}\jya tɤ-prɤm tɯ-sqar tɯ-mtɕoʁ khɯtsa ɯ-ŋgɯ pjɯ́-wɣ-lɤt tɕe ɯ-taʁ tʂha tú-wɣ-rku tɕe ɲɯ́-wɣ-ɕmi tɕe tɯ-tshi kɯ-fse nɯ kú-wɣ-tshi tɕe nɯ tɕhɯβroʁ rmi\cmn 
在碗里放一撮糌粑,倒上马茶,搅成粥就可以吃,这叫作\stylefv{tɕhɯβroʁ}。
\end{exemple}\end{entrée}

\begin{entrée}
\vedette{\hypertarget{Ⓔtɕhɯβtɕhɯβ}{\papi{ tɕhɯβtɕhɯβ}}}\markboth{tɕhɯβtɕhɯβ}{} (\variante{tɕhɯptɕhɯp}) \classe{idph.2}
\begin{définition}\fra apparition de gouttes d'eau\end{définition}
\begin{définition}\cmn (看得到)水珠\end{définition}
\begin{définition}\cmn 在下雨(有珍珠般的雨点)\end{définition}
\begin{exemple}\jya tɯ-ci ɯ-pɕi tɕhɯβtɕhɯβ ʑo ɲɤ-nɯ-ɬoʁ\cmn (口袋、布)冒了水珠\end{exemple}
\begin{exemple}\jya tɯ-mɯ tɕhɯptɕhɯp ɲɯ-ɤsɯ-lɤt\end{exemple}\begin{sous-entrée}
\vedette{\hypertarget{}{\papi{ ɣɤtɕhɯβtɕhɯβ}}}\markboth{ɣɤtɕhɯβtɕhɯβ}{}\classe{vs}
\begin{définition}\fra qui coule\end{définition}
\begin{définition}\cmn 漏水的\end{définition}
\begin{relation-sémantique}\confer{
\hyperlink{Ⓔtshɯptshɯp}{\textit{ \papi{tshɯptshɯp}}}
}\end{relation-sémantique}
\end{sous-entrée}\end{entrée}

\begin{entrée}
\vedette{\hypertarget{Ⓔtɕhɯɕɤl}{\papi{ tɕhɯɕɤl}}}\markboth{tɕhɯɕɤl}{}
\classe{n}
\begin{définition}\fra cristal\end{définition}
\begin{définition}\cmn 水晶
\begin{déclaration} \étymologie{\papi{tɕʰu.ɕel}}\end{déclaration}\end{définition}
\begin{exemple}\jya tɕhɯɕɤl-mprɯwa\cmn 水晶的念珠\end{exemple}\end{entrée}

\begin{entrée}
\vedette{\hypertarget{Ⓔtɕhɯɕrɤm}{\papi{ tɕhɯɕrɤm}}}\markboth{tɕhɯɕrɤm}{}\classe{n}
\begin{définition}\fra loutre\end{définition}
\begin{définition}\cmn 水獭
\begin{déclaration} \étymologie{\papi{tɕʰu.sram}}\end{déclaration}\end{définition}\end{entrée}

\begin{entrée}
\vedette{\hypertarget{Ⓔtɕhɯɣur}{\papi{ tɕhɯɣur}}}\markboth{tɕhɯɣur}{}\classe{n}
\begin{définition}\fra digue\end{définition}
\begin{définition}\cmn 堤坝\end{définition}
\end{entrée}

\begin{entrée}
\vedette{\hypertarget{Ⓔtɕhɯjɤr}{\papi{ tɕhɯjɤr}}}\markboth{tɕhɯjɤr}{}
\classe{n}
\begin{définition}\fra digue\end{définition}
\begin{définition}\cmn 堤坝
\begin{déclaration} \étymologie{\papi{tɕʰu.jur}}\end{déclaration}\end{définition}\end{entrée}

\begin{entrée}
\vedette{\hypertarget{Ⓔtɕhɯkɤɣar}{\papi{ tɕhɯkɤɣar}}}\markboth{tɕhɯkɤɣar}{}\classe{n}
\begin{définition}\fra bord de l'eau\end{définition}
\begin{définition}\cmn 水边\end{définition}
\begin{exemple}\jya ftɕar tɕe tɕhɯkɤɣar ɲ-ɯkɯ-nɯrɤχtɕi tɕe sɤscit\cmn 夏天的时候在水边洗东西很舒服\end{exemple}
\begin{exemple}\jya tɤ-pɤtso ra tɕhɯkɤɣar k-ɤɕe rga-nɯ\cmn 小孩子喜欢到水边去\end{exemple}
\begin{exemple}\jya tɤ-pɤtso ra tɕhɯkɤɣar kɤ-ɤnɯɣro rga-nɯ\cmn 小孩子喜欢在水边玩\end{exemple}\end{entrée}

\begin{entrée}
\vedette{\hypertarget{Ⓔtɕhɯla}{\papi{ tɕhɯla}}}\markboth{tɕhɯla}{}\classe{n}
\begin{définition}\fra habit d'homme en laine\end{définition}
\begin{définition}\cmn 男式藏装,毛织品\end{définition}\end{entrée}

\begin{entrée}
\vedette{\hypertarget{Ⓔtɕhɯlɤɣrum}{\papi{ tɕhɯlɤɣrum}}}\markboth{tɕhɯlɤɣrum}{}\classe{n}
\begin{définition}\fra habit tibétain blanc\end{définition}
\begin{définition}\cmn 白色的藏式服装\end{définition}
\begin{relation-sémantique}\confer{
\hyperlink{Ⓔwɣrum}{\textit{ \papi{wɣrum}}}
}\end{relation-sémantique}
\begin{relation-sémantique}\confer{
\hyperlink{Ⓔtɕhɯla}{\textit{ \papi{tɕhɯla}}}
}\end{relation-sémantique}\end{entrée}

\begin{entrée}
\vedette{\hypertarget{Ⓔtɕhɯma}{\papi{ tɕhɯma}}}\markboth{tɕhɯma}{}\classe{n}
\begin{définition}\fra navet (Brassica sp.)\end{définition}
\begin{définition}\cmn 芜菁的一种【冬圆根】\end{définition}
\begin{exemple}\jya tɕhɯma nɯ ɯ-tshɯɣa ra rasti cho naχtɕɯɣ, tɕeri ɯ-mdoʁ mɤ-nɤχtɕɯɣ, tɕhɯma nɯ ɯ-jwaʁ nɯ mpɕu, kɯ-xtɕɯ-xtɕi nɤmbju, ɯ-jwaʁ cho ɯ-jndoʁ nɯ ra rasti sɤznɤ mpɯ, rasti nɯ χɕitka tɕe pjɯ́-wɣ-ji, tɕe zgoku nɯ ra stonka mɤɕtʂa mɤ-kɤ-phɯt khɯ. a-pɯ-nɯɣur tɕe, mɤʑɯ arŋi cho mpɯ. tɕhɯma nɯ χɕitka sthɯci mɤ-lɤ́-wɣ-ji kɯ stonka tɕe lú-wɣ-ji. kɯmaʁ tɤ-rɤku kɤ́-wɣ-phɯt ɯ-mphru ɕɯmɯma tɕe lú-wɣ-ji tɕe tɯ-sla jamar tɕe kɤ-ndza tu-βze cha. li tɤjko ɯ-spa kɯ-pe ŋu. ɯ-jndoʁ nɯ paʁndza pe, kɤ́-wɣ-sqa tɕe tɯrme kɤ-ndza sna.\cmn 冬圆根样子和圆根一样,但是颜色不同,冬圆根的叶子是光滑的,有点光泽,叶子和根比圆根嫩。圆根在春天种下,在高山上的那些可以等到秋天才拔,打霜后才收会更绿、更嫩。冬圆根不要在春天种,等其他庄稼收完以后,马上把它种下,不到一个月就可以吃了。也是煮酸菜的好材料,根有是喂猪的好饲料,煮熟后人也可以吃。\end{exemple}\end{entrée}

\begin{entrée}
\vedette{\hypertarget{Ⓔtɕhɯmɲɯɣ}{\papi{ tɕhɯmɲɯɣ}}}\markboth{tɕhɯmɲɯɣ}{}
\classe{n}
\begin{définition}\fra source\end{définition}
\begin{définition}\cmn 泉水\end{définition}\end{entrée}

\begin{entrée}
\vedette{\hypertarget{Ⓔtɕhɯndza}{\papi{ tɕhɯndza}}}\markboth{tɕhɯndza}{}\classe{n}
\begin{définition}\fra flot\end{définition}
\begin{définition}\cmn 水流\end{définition}
\begin{exemple}\jya tɕhɯndza mɤ-kɯ-βdi\cmn 水流很急(水面不平稳)\end{exemple}
\begin{exemple}\jya tɕhɯndza a-mɤ-pɯ-βdi tɕe, tɯ-ci ɯ-zgra wxti\cmn 水流很急的时候很吵\end{exemple}\end{entrée}

\begin{entrée}
\vedette{\hypertarget{Ⓔtɕhɯŋ}{\papi{ tɕhɯŋ}}}\markboth{tɕhɯŋ}{}
\classe{idph.1}
\begin{définition}\fra cliquetis de métaux\end{définition}
\begin{définition}\cmn 形容(金属)撞击声,叮当声\end{définition}\begin{sous-entrée}
\vedette{\hypertarget{}{\papi{ sɤtɕhɯŋtɕhɯŋ}}}\markboth{sɤtɕhɯŋtɕhɯŋ}{}\classe{vt}
\begin{exemple}\jya ɲɯ-sɤtɕhɯŋtɕhɯŋ ɲɯ-ɤsɯ-ʑmbri\cmn 他(敲着铁的东西)发出叮当声\end{exemple}
\end{sous-entrée}\end{entrée}

\begin{entrée}
\vedette{\hypertarget{Ⓔtɕhɯŋkhɤr}{\papi{ tɕhɯŋkhɤr}}}\markboth{tɕhɯŋkhɤr}{}
\classe{n}
\begin{définition}\fra moulin à eau\end{définition}
\begin{définition}\cmn 水车
\begin{déclaration} \étymologie{\papi{tɕʰu.ⁿkʰor}}\end{déclaration}\end{définition}\end{entrée}

\begin{entrée}
\vedette{\hypertarget{Ⓔtɕhɯŋkhɤru}{\papi{ tɕhɯŋkhɤru}}}\markboth{tɕhɯŋkhɤru}{}\classe{n}
\begin{définition}\fra axe du moulin\end{définition}
\begin{définition}\cmn 水磨的主杆\end{définition}
\begin{exemple}\jya tɕhɯŋkhɤru nɯ tɕhɯŋkhɤr ɯ-χcɤl ʑo ɕoŋtɕa tu-kɯ-ɕe tɕe βɣɤsni kɯ-ndo nɯ ŋu, ɯ-taʁ βɣɤrnɤjwaʁ kú-wɣ-tshoʁ ra\cmn 
\stylefv{tɕhɯŋkhɤr-ru}是水磨中间立着的部件,用来支撑磨心,前后左右装有木板。
\end{exemple}
\end{entrée}

\begin{entrée}
\vedette{\hypertarget{Ⓔtɕhɯŋtɕhɯŋ}{\papi{ tɕhɯŋtɕhɯŋ}}}\markboth{tɕhɯŋtɕhɯŋ}{}
\classe{idph.2}
\begin{définition}\fra pure, propre (eau)\end{définition}
\begin{définition}\cmn 形容(水)纯净、干净的样子\end{définition}
\begin{exemple}\jya tɯ-ci tɕhɯŋtɕhɯŋ ɲɯ-ɤmgri\cmn 水澄清得一点渣滓也没有的样子\end{exemple}\end{entrée}

\begin{entrée}
\vedette{\hypertarget{Ⓔtɕhɯpɣa}{\papi{ tɕhɯpɣa}}}\markboth{tɕhɯpɣa}{}
\classe{n}
\begin{définition}\fra canard\end{définition}
\begin{définition}\cmn 鸭子\end{définition}
\begin{relation-sémantique}\confer{
\hyperlink{Ⓔpɣa}{\textit{ \papi{pɣa}}}
}\end{relation-sémantique}\end{entrée}

\begin{entrée}
\vedette{\hypertarget{Ⓔtɕhɯphɯɣ}{\papi{ tɕhɯphɯɣ}}}\markboth{tɕhɯphɯɣ}{}
\classe{n}
\begin{définition}\fra source d'un fleuve\end{définition}
\begin{définition}\cmn 水源
\begin{déclaration} \étymologie{\papi{tɕʰu.pʰugs}}\end{déclaration}\end{définition}
\begin{relation-sémantique}\confer{
\hyperlink{Ⓔɯ-phɯɣ}{\textit{ \papi{ɯ-phɯɣ}}}
}\end{relation-sémantique}\end{entrée}

\begin{entrée}
\vedette{\hypertarget{Ⓔtɕhɯpi}{\papi{ tɕhɯpi}}}\markboth{tɕhɯpi}{}\classe{n}
\begin{définition}\fra être trempé par la pluie\end{définition}
\begin{définition}\cmn 被雨淋得湿透了\end{définition}
\begin{exemple}\jya tɕhɯpi ʑo ɲɤ-tɯ́-wɣ-ta\cmn 你被雨淋到了\end{exemple}\end{entrée}

\begin{entrée}
\vedette{\hypertarget{Ⓔtɕhɯqhu}{\papi{ tɕhɯqhu}}}\markboth{tɕhɯqhu}{}\classe{adv}
\begin{définition}\fra dessous de l'échelle\end{définition}
\begin{définition}\cmn 梯子后面\end{définition}
\begin{relation-sémantique}\confer{
\hyperlink{ⒺtɕhiⒽ2}{\textit{ \papi{tɕhi2}}}
}\end{relation-sémantique}\end{entrée}

\begin{entrée}
\vedette{\hypertarget{Ⓔtɕhɯra}{\papi{ tɕhɯra}}}\markboth{tɕhɯra}{}\classe{n}
\begin{définition}\fra cuve à eau\end{définition}
\begin{définition}\cmn 水缸(积累水的缸子)\end{définition}\end{entrée}

\begin{entrée}
\vedette{\hypertarget{Ⓔtɕhɯrdu}{\papi{ tɕhɯrdu}}}\markboth{tɕhɯrdu}{}
\classe{n}
\begin{définition}\fra galet\end{définition}
\begin{définition}\cmn 卵石
\begin{déclaration} \étymologie{\papi{tɕʰu.rdo}}\end{déclaration}\end{définition}\end{entrée}

\begin{entrée}
\vedette{\hypertarget{Ⓔtɕhɯrkɯ}{\papi{ tɕhɯrkɯ}}}\markboth{tɕhɯrkɯ}{}\classe{n}
\begin{définition}\fra gamelle de chien\end{définition}
\begin{définition}\cmn 狗的饭盆,给狗吃饭的盆\end{définition}\end{entrée}

\begin{entrée}
\vedette{\hypertarget{Ⓔtɕhɯrqhioʁ}{\papi{ tɕhɯrqhioʁ}}}\markboth{tɕhɯrqhioʁ}{}\classe{n}
\begin{définition}\fra canal\end{définition}
\begin{définition}\cmn 渠道\end{définition}\end{entrée}

\begin{entrée}
\vedette{\hypertarget{Ⓔtɕhɯrtsɤm}{\papi{ tɕhɯrtsɤm}}}\markboth{tɕhɯrtsɤm}{}
\classe{n}
\begin{définition}\fra tsampa\end{définition}
\begin{définition}\cmn 糌粑的一种吃法
\begin{déclaration} \étymologie{\papi{tɕʰu.rtsam.pa}}\end{déclaration}\end{définition}\end{entrée}

\begin{entrée}
\vedette{\hypertarget{Ⓔtɕhɯrwa}{\papi{ tɕhɯrwa}}}\markboth{tɕhɯrwa}{}\classe{n}
\begin{définition}\fra quark, tvorog\end{définition}
\begin{définition}\cmn 奶渣
\begin{déclaration} \étymologie{\papi{pʰʲur.ba}}\end{déclaration}\end{définition}\end{entrée}

\begin{entrée}
\vedette{\hypertarget{Ⓔtɕhɯʁja}{\papi{ tɕhɯʁja}}}\markboth{tɕhɯʁja}{}
\classe{n}
\begin{définition}\fra lentille d'eau\end{définition}
\begin{définition}\cmn 浮萍
\begin{déclaration} \étymologie{\papi{*tɕʰu.gja}}\end{déclaration}\end{définition}\end{entrée}

\begin{entrée}
\vedette{\hypertarget{Ⓔtɕhɯʁjɯ}{\papi{ tɕhɯʁjɯ}}}\markboth{tɕhɯʁjɯ}{}\classe{n}
\begin{définition}\fra animaux aquatiques\end{définition}
\begin{définition}\cmn 水虫\end{définition}
\end{entrée}

\begin{entrée}
\vedette{\hypertarget{Ⓔtɕhɯʁnɤz}{\papi{ tɕhɯʁnɤz}}}\markboth{tɕhɯʁnɤz}{} (\variante{tɕhɯʁɲɤz}) 
\classe{n}
\begin{définition}\fra monstre aquatique\end{définition}
\begin{définition}\cmn 水怪\end{définition}\end{entrée}

\begin{entrée}
\vedette{\hypertarget{Ⓔtɕhɯskrɯt}{\papi{ tɕhɯskrɯt}}}\markboth{tɕhɯskrɯt}{}\classe{n}
\begin{définition}\fra fil épais que l'on coud sur le bord des habits tibétains\end{définition}
\begin{définition}\cmn (藏装)缝在衣服边缘的粗线(头星子)\end{définition}\end{entrée}

\begin{entrée}
\vedette{\hypertarget{Ⓔtɕhɯsloŋ}{\papi{ tɕhɯsloŋ}}}\markboth{tɕhɯsloŋ}{}\classe{n}
\begin{définition}\fra inondation\end{définition}
\begin{définition}\cmn 洪水\end{définition}
\end{entrée}

\begin{entrée}
\vedette{\hypertarget{Ⓔtɕhɯsmɤn}{\papi{ tɕhɯsmɤn}}}\markboth{tɕhɯsmɤn}{}\classe{n}
\begin{définition}\fra source chaude\end{définition}
\begin{définition}\cmn 温泉\end{définition}\end{entrée}

\begin{entrée}
\vedette{\hypertarget{Ⓔtɕhɯt}{\papi{ tɕhɯt}}}\markboth{tɕhɯt}{} (\variante{xtɕhɯt}) \classe{vs}
\paradigme{\textit{dir :} \jya tɤ-}
\begin{définition}\fra pouvoir contenir\end{définition}
\begin{définition}\cmn 容得下;装得下\end{définition}
\begin{exemple}\jya tɤ-fkɯm ɲɯ-tɕhɯt\cmn 口袋容得下(那么多)\end{exemple}
\begin{exemple}\jya tɤ-fkɯm ɯ-ŋgɯ ɲɯ-xtɕhɯt (mɯ́j-xtɕhɯt)\cmn 袋子里装得下(装不下)\end{exemple}
\begin{exemple}\jya lʁa ɯ-ŋgɯ thɯ-ɣnde ma a-pɯ-xtɕhɯt ɬoʁ\cmn 你把口袋塞紧一点因为必须装下\end{exemple}\begin{sous-entrée}
\vedette{\hypertarget{}{\papi{ ɣɤtɕhɯt}}}\markboth{ɣɤtɕhɯt}{}\classe{vt}
\paradigme{\textit{dir :} \jya tɤ-}
\begin{définition}\fra faire de la place\end{définition}
\begin{définition}\cmn 空出……的位子\end{définition}
\end{sous-entrée}\begin{sous-entrée}
\vedette{\hypertarget{}{\papi{ sɯxtɕhɯt}}}\markboth{sɯxtɕhɯt}{}\classe{vt}
\paradigme{\textit{dir :} \jya tɤ-}
\begin{définition}\fra faire de la place\end{définition}
\begin{définition}\cmn 空出……的位子\end{définition}
\begin{exemple}\jya kɯmaʁ laχtɕha ra nɯ-nɤscɯscat-a tɕe, rgɤm kɤ-ta tɤ-ɣɤtɕhɯt-a\cmn 我把其他东西搬走了,空出了放箱子的位子\end{exemple}
\begin{exemple}\jya tɤ-sɯxtɕhɯt-a\cmn 我都装进去了\end{exemple}
\begin{exemple}\jya ɯʑo kɤ-ɤmdzɯ tɤ-sɯxtɕhɯt-a\cmn 我给他找了个位子坐(本来没有位子)\end{exemple}
\end{sous-entrée}\end{entrée}

\begin{entrée}
\vedette{\hypertarget{Ⓔtɕhɯtɕhɯtɕhɯt}{\papi{ tɕhɯtɕhɯtɕhɯt}}}\markboth{tɕhɯtɕhɯtɕhɯt}{}
\classe{adv}
\begin{définition}\fra en mettre autant que possible\end{définition}
\begin{définition}\cmn 能装多少就装多少\end{définition}
\begin{exemple}\jya nɤ-khɯtsa tɕhɯtɕhɯtɕhɯt nɯ pɯ-nɯrke tɕe, ɯ-ro nɯ nɯ-βde\cmn 你碗里能装多少就装多少,把剩余的放在那里\end{exemple}\end{entrée}

\begin{entrée}
\vedette{\hypertarget{Ⓔtɕhɯtɕɯn}{\papi{ tɕhɯtɕɯn}}}\markboth{tɕhɯtɕɯn}{}\classe{n}
\begin{définition}\ 
\begin{déclaration}\grammar{n.lieu}\end{déclaration}\end{définition}
\begin{définition}\fra Chuchen\end{définition}
\begin{définition}\cmn 金川\end{définition}\end{entrée}

\begin{entrée}
\vedette{\hypertarget{Ⓔtɕhɯtɕɯnpaχɕi}{\papi{ tɕhɯtɕɯnpaχɕi}}}\markboth{tɕhɯtɕɯnpaχɕi}{}\classe{n}
\begin{définition}\fra poire\end{définition}
\begin{définition}\cmn 梨子\end{définition}
\begin{relation-sémantique}\confer{
\hyperlink{Ⓔnɯtɕhɯtɕɯnpaχɕi}{\textit{ \papi{nɯtɕhɯtɕɯnpaχɕi}}}
}\end{relation-sémantique}\end{entrée}

\begin{entrée}
\vedette{\hypertarget{Ⓔtɕhɯte}{\papi{ tɕhɯte}}}\markboth{tɕhɯte}{}\classe{n}
\begin{définition}\fra grand fleuve\end{définition}
\begin{définition}\cmn 大河\end{définition}\end{entrée}

\begin{entrée}
\vedette{\hypertarget{Ⓔtɕhɯthɤn}{\papi{ tɕhɯthɤn}}}\markboth{tɕhɯthɤn}{}\classe{n}
\begin{définition}\fra coulée de boue\end{définition}
\begin{définition}\cmn 洪水;泥石流
\begin{déclaration} \étymologie{\papi{tɕʰu.tʰan}}\end{déclaration}\end{définition}
\begin{exemple}\jya tɕhɯthɤn chɤ-ɣi\cmn 来了泥石流\end{exemple}\end{entrée}

\begin{entrée}
\vedette{\hypertarget{Ⓔtɕhɯtoʁ}{\papi{ tɕhɯtoʁ}}}\markboth{tɕhɯtoʁ}{}\classe{n}
\begin{définition}\fra marais, endroit humide\end{définition}
\begin{définition}\cmn 湿地\end{définition}
\begin{exemple}\jya tɕhɯtoʁ ɯ-ku\cmn 沼泽\end{exemple}\end{entrée}

\begin{entrée}
\vedette{\hypertarget{Ⓔtɕhɯtsa}{\papi{ tɕhɯtsa}}}\markboth{tɕhɯtsa}{}\classe{n}
\begin{définition}\fra petit ruisseau\end{définition}
\begin{définition}\cmn 小河\end{définition}\end{entrée}

\begin{entrée}
\vedette{\hypertarget{Ⓔtɕhɯtɯɣ}{\papi{ tɕhɯtɯɣ}}}\markboth{tɕhɯtɯɣ}{}\classe{n}
\begin{définition}\fra source empoisonnée\end{définition}
\begin{définition}\cmn 有毒的山泉
\begin{déclaration} \étymologie{\papi{tɕʰu.dug}}\end{déclaration}\end{définition}\end{entrée}

\begin{entrée}
\vedette{\hypertarget{Ⓔtɕhɯwɯr}{\papi{ tɕhɯwɯr}}}\markboth{tɕhɯwɯr}{}\classe{n}
\begin{définition}\fra ampoule\end{définition}
\begin{définition}\cmn 水疱
\begin{déclaration} \étymologie{\papi{tɕʰu.bur}}\end{déclaration}\end{définition}
\begin{exemple}\jya a-jaʁ tɕhɯwɯr rɣurɣu ʑo to-rku\cmn 我手上长了水泡\end{exemple}
\begin{relation-sémantique}\confer{
\hyperlink{Ⓔcɯmbɤrom}{\textit{ \papi{cɯmbɤrom}}}
}\end{relation-sémantique}\end{entrée}

\begin{entrée}
\vedette{\hypertarget{Ⓔtɕhɯχpri}{\papi{ tɕhɯχpri}}}\markboth{tɕhɯχpri}{}
\classe{n}
\begin{définition}\fra salamandre\end{définition}
\begin{définition}\cmn 四脚蛇\end{définition}
\begin{exemple}\jya tɕhɯχpri nɯ tɯ-ci ɯ-ŋgɯ ku-rɤʑi ŋu, qajɯ ci ŋu, kɯ-ɲɯ-ɲaʁ kɯ-nɤmbɯ-mbju ci ŋu, ɯ-mi kɯβdɤ-ldʑa tu, ɯ-mɤndzu kɯβdɤ-ldʑa ɣɤʑu, tɤ-ŋke tɕe ɯ-mɤlɤjaʁ ju-scɤt ɲɯ-ŋu, ɯ-jme ju-nɤkhɯkhrɯt ŋu, ɯ-ku nɯ qapri cho naχtɕɯɣ, ɯ-phoŋbu acilaj ʑo, tɯ-mtshi kɯ-mŋɤm smɤn ɲɯ-ŋu. kɯ-mdoʁmdi mɤ-kɯ-si chɯ́-wɣ-mqlaʁ tɕe stu kɯ-phɤn ɲɯ-ŋu khi.\cmn 四脚蛇栖息在水里,是一种虫,黑色,有光泽。有四只脚,每只脚上有四只脚趾,用四肢爬行,拖着尾巴。头部像蛇的一样,全身湿漉漉的,是治胃病的好药材,据说如果能整个活吞效果最好。\end{exemple}\end{entrée}

\begin{entrée}
\vedette{\hypertarget{Ⓔtɕhɯχtɤrci}{\papi{ tɕhɯχtɤrci}}}\markboth{tɕhɯχtɤrci}{}\classe{n}
\begin{définition}\fra source d'où coule une eau de couleur blanche\end{définition}
\begin{définition}\cmn 流出白色的水的山泉\end{définition}
\end{entrée}

\begin{entrée}
\vedette{\hypertarget{Ⓔtɕhɯχtso}{\papi{ tɕhɯχtso}}}\markboth{tɕhɯχtso}{}\classe{n}
\begin{définition}\fra eau propre, potable\end{définition}
\begin{définition}\cmn 净水
\begin{déclaration} \étymologie{\papi{tɕʰu.gtsaŋ}}\end{déclaration}\end{définition}\end{entrée}

\begin{entrée}
\vedette{\hypertarget{Ⓔtɕhɯzɯ}{\papi{ tɕhɯzɯ}}}\markboth{tɕhɯzɯ}{}
\classe{n}
\begin{définition}\fra élément du métier à tisser\end{définition}
\begin{définition}\cmn 筘\end{définition}
\begin{exemple}\jya tɕhɯzɯ nɯ thɯ-kɯ-taʁ tɕe kɤ-taʁ ɯ-sɤ-ndo ɯ-kɯ-z-rɤsta spa ɣɯ si nɯ-kɤ-βzu ŋu, tsuku tɕe phu mu tu, tsuku tɕe kɯ-ɤntɤm ɕti\cmn 
\stylefv{tɕhɯzɯ}是用来夹住布料控制它的松紧程度的木条,有些是由凸出来和凹进去的两个部分组成的,有的是平的。
\end{exemple}\end{entrée}

\begin{entrée}
\vedette{\hypertarget{Ⓔtɕhɯʑaŋ}{\papi{ tɕhɯʑaŋ}}}\markboth{tɕhɯʑaŋ}{}
\classe{n}
\begin{définition}\fra irrigation\end{définition}
\begin{définition}\cmn 灌溉
\begin{déclaration} \étymologie{\papi{tɕʰu.ʑiŋ}}\end{déclaration}\end{définition}
\begin{exemple}\jya tɕhɯʑaŋ thɯ-lat-a\cmn 我灌溉了(农田)\end{exemple}\end{entrée}

\begin{entrée}
\vedette{\hypertarget{ⒺtɕiⒽ1}{\papi{ tɕi}}}\markboth{tɕi}{}\homonyme{1}
\classe{adv}
\begin{définition}\fra aussi\end{définition}
\begin{définition}\cmn 也\end{définition}
\end{entrée}

\begin{entrée}
\vedette{\hypertarget{ⒺtɕiⒽ2}{\papi{ tɕi}}}\markboth{tɕi}{}\homonyme{2}
\classe{part}
\begin{définition}\fra marque de topique\end{définition}
\begin{définition}\cmn 嘛\end{définition}
\begin{exemple}\jya kɤ-ntɕhoz ɯ-spa tɕi, ɕ-tɤ-χti\cmn 要用的东西嘛,得去买\end{exemple}\end{entrée}

\begin{entrée}
\vedette{\hypertarget{Ⓔtɕiʑo}{\papi{ tɕiʑo}}}\markboth{tɕiʑo}{}\classe{pro}
\begin{définition}\fra nous deux\end{définition}
\begin{définition}\cmn 我们俩\end{définition}
\end{entrée}

\begin{entrée}
\vedette{\hypertarget{Ⓔtɕoχtsi}{\papi{ tɕoχtsi}}}\markboth{tɕoχtsi}{}
\classe{n}
\begin{définition}\fra table\end{définition}
\begin{définition}\cmn 桌子
\begin{déclaration} \étymologie{\papi{ltɕog.rtsi}}\end{déclaration}\end{définition}\end{entrée}

\begin{entrée}
\vedette{\hypertarget{ⒺtɕurⒽ1}{\papi{ tɕur}}}\markboth{tɕur}{}\homonyme{1}\classe{vs}
\paradigme{\textit{dir :} \jya nɯ-}
\begin{définition}\fra acide\end{définition}
\begin{définition}\cmn 酸\end{définition}
\begin{exemple}\jya ɯ-tɯ-tɕur kɯ tɯ-ku ɲɯ-kɯ-sɯ-sɤphɤr ɲɯ-ŋu\cmn 酸到昏了头\end{exemple}\begin{sous-entrée}
\vedette{\hypertarget{}{\papi{ ɣɤtɕur}}}\markboth{ɣɤtɕur}{}\classe{vt}
\paradigme{\textit{dir :} \jya pɯ-}
\begin{définition}\ 
\begin{déclaration}\grammar{caus}\end{déclaration}\end{définition}
\begin{définition}\fra rendre acide\end{définition}
\begin{définition}\cmn 令……变得更酸\end{définition}
\begin{définition}\cmn 因为菜不够酸,我放了一点醋,把菜弄得更酸\end{définition}
\begin{exemple}\jya @cai ɯ-tɯ-tɕur mɯ́j-rtaʁ tɕe, @cu pɯ-lat-a tɕe pɯ-ɣɤtɕur-a\end{exemple}
\end{sous-entrée}\begin{sous-entrée}
\vedette{\hypertarget{}{\papi{ nɤxtɕur}}}\markboth{nɤxtɕur}{}\classe{vt}
\paradigme{\textit{dir :} \jya pɯ-}
\begin{définition}\ 
\begin{déclaration}\grammar{trop}\end{déclaration}\end{définition}
\begin{définition}\fra trouver acide\end{définition}
\begin{définition}\cmn 觉得酸\end{définition}
\begin{exemple}\jya pɯ-nɤxtɕur-a\cmn 我觉得很酸\end{exemple}
\end{sous-entrée}\begin{sous-entrée}
\vedette{\hypertarget{}{\papi{ sɯxtɕur}}}\markboth{sɯxtɕur}{}
\paradigme{\textit{dir :} \jya pɯ-}
\paradigme{\textit{dir :} \jya thɯ-}
\begin{définition}\fra rendre acide\end{définition}
\begin{définition}\cmn 令……变酸\end{définition}
\begin{exemple}\jya tɤjko thɯ-sɯxtɕur-a\cmn 我把圆根(煮熟了以后)弄酸了\end{exemple}
\end{sous-entrée}\end{entrée}

\begin{entrée}
\vedette{\hypertarget{ⒺtɕurⒽ2}{\papi{ tɕur}}}\markboth{tɕur}{}\homonyme{2}
\classe{vt}
\begin{définition}\fra insérer dans\end{définition}
\begin{définition}\cmn 插(筷子、吸管、笔)\end{définition}
\begin{exemple}\jya @cai ɯ-ŋgɯ ndʑu pɯ-tɕur-a\cmn 你用筷子夹菜\end{exemple}
\begin{exemple}\jya chɤmda ɯ-ŋgɯ tɕe chɤmdɤru pɯ-tɕur-a\cmn 你把吸管插进坛坛酒了\end{exemple}
\begin{relation-sémantique}\synonyme{
 \papi{sɤsta}
}\end{relation-sémantique}
\end{entrée}

\begin{entrée}
\vedette{\hypertarget{Ⓔtɕrɯɣnɤtɕrɯɣ}{\papi{ tɕrɯɣnɤtɕrɯɣ}}}\markboth{tɕrɯɣnɤtɕrɯɣ}{}\classe{idph.3}
\begin{définition}\fra grincement de dent\end{définition}
\begin{définition}\cmn 形容牙齿摩擦的响声\end{définition}
\end{entrée}

\begin{entrée}
\vedette{\hypertarget{Ⓔtɕʁɯβnɤtɕʁɯβ}{\papi{ tɕʁɯβnɤtɕʁɯβ}}}\markboth{tɕʁɯβnɤtɕʁɯβ}{}\classe{idph.3}
\begin{définition}\fra croquant\end{définition}
\begin{définition}\cmn 形容食物脆
\begin{déclaration}\use{咀嚼的声音比\stylefv{tɕʁɯznɤtɕʁɯz}大一些}\end{déclaration}\end{définition}\begin{sous-entrée}
\vedette{\hypertarget{}{\papi{ nɯtɕʁɯβ}}}\markboth{nɯtɕʁɯβ}{}\classe{vt}
\paradigme{\textit{dir :} \jya pɯ-}
\begin{définition}\fra croquer\end{définition}
\begin{définition}\cmn 使劲地咀嚼(脆的食物)\end{définition}
\begin{exemple}\jya paχɕi tɤ-nɯtɕʁɯβ-a ʑo tɤ-ndza-t-a\cmn 我吃了苹果,发出咀嚼的声音\end{exemple}
\begin{relation-sémantique}\antonyme{
\hyperlink{Ⓔtɕʁɯznɤtɕʁɯz}{\textit{ \papi{tɕʁɯznɤtɕʁɯz}}}
}\end{relation-sémantique}
\begin{relation-sémantique}\antonyme{
\hyperlink{Ⓔtɕhʁɯβnɤtɕhʁɯβ}{\textit{ \papi{tɕhʁɯβnɤtɕhʁɯβ}}}
}\end{relation-sémantique}
\end{sous-entrée}\end{entrée}

\begin{entrée}
\vedette{\hypertarget{Ⓔtɕʁɯznɤtɕʁɯz}{\papi{ tɕʁɯznɤtɕʁɯz}}}\markboth{tɕʁɯznɤtɕʁɯz}{}
\classe{idph.3}
\begin{définition}\fra croquant\end{définition}
\begin{définition}\cmn 形容食物脆\end{définition}
\begin{exemple}\jya tú-wɣ-ndza tɕe tɕʁɯznɤtɕʁɯz ɲɯ-ti\cmn 吃起来很脆\end{exemple}
\begin{relation-sémantique}\antonyme{
\hyperlink{Ⓔzwaʁnɤzwaʁ}{\textit{ \papi{zwaʁnɤzwaʁ}}}
}\end{relation-sémantique}
\begin{relation-sémantique}\antonyme{
\hyperlink{Ⓔtɕʁɯβnɤtɕʁɯβ}{\textit{ \papi{tɕʁɯβnɤtɕʁɯβ}}}
}\end{relation-sémantique}\end{entrée}

\begin{entrée}
\vedette{\hypertarget{Ⓔtɕɯlɤβ}{\papi{ tɕɯlɤβ}}}\markboth{tɕɯlɤβ}{}\classe{n}
\begin{définition}\fra pipe\end{définition}
\begin{définition}\cmn 烟斗\end{définition}\end{entrée}

\begin{entrée}
\vedette{\hypertarget{Ⓔtɕɯxpa}{\papi{ tɕɯxpa}}}\markboth{tɕɯxpa}{}\classe{n}
\begin{définition}\fra conclue, réglée (affaire)\end{définition}
\begin{définition}\cmn 定好(交易)\end{définition}
\begin{exemple}\jya tɕɯxpa pɯ-nɯ-rku-tɕi\cmn 我们把事情定好了\end{exemple}\end{entrée}

\begin{entrée}
\vedette{\hypertarget{Ⓔtɕɯχtsi}{\papi{ tɕɯχtsi}}}\markboth{tɕɯχtsi}{}\classe{n}
\begin{définition}\fra Tchogtse\end{définition}
\begin{définition}\cmn 卓克基\end{définition}\end{entrée}

\begin{entrée}
\vedette{\hypertarget{Ⓔtɣa}{\papi{ tɣa}}}\markboth{tɣa}{}
\classe{vi}
\paradigme{\textit{dir :} \jya kɤ-}
\begin{définition}\fra récolter\end{définition}
\begin{définition}\cmn 收割\end{définition}
\begin{exemple}\jya kɤ-tɣa tɤ-sɤsqɤr-i\cmn 我们请了别人帮忙收割\end{exemple}
\begin{relation-sémantique}\confer{
\hyperlink{ⒺtɯtɣaⒽ2}{\textit{ \papi{tɯtɣa2}}}
}\end{relation-sémantique}\end{entrée}

\begin{entrée}
\vedette{\hypertarget{ⒺthuⒽ2}{\papi{ thu}}}\markboth{thu}{}\homonyme{2}
\classe{n}
\begin{définition}\fra borne, marque (pile de pierre)\end{définition}
\begin{définition}\cmn 用来当标记的石堆\end{définition}
\end{entrée}

\begin{entrée}
\vedette{\hypertarget{ⒺthuⒽ1}{\papi{ thu}}}\markboth{thu}{}\homonyme{1}\classe{vt}
\paradigme{\textit{dir :} \jya tɤ-}
\begin{définition}\fra demander\end{définition}
\begin{définition}\cmn 问\end{définition}
\begin{exemple}\jya tɤ-thu-t-a\cmn 我问了他\end{exemple}
\begin{exemple}\jya nɤ-kɤ-thu tɤ-the jɤɣ\cmn 你可以问你的问题\end{exemple}
\begin{exemple}\jya a-kɤ-thu nɯ ma thɯ-arɕo\cmn 我没有其他问题了\end{exemple}
\begin{exemple}\jya tɤ́-wɣ-thu-a\cmn 他向别人问了我的情况\end{exemple}
\begin{exemple}\jya a-ɕki ta-thu\cmn 他问了我\end{exemple}
\begin{exemple}\jya kɯmaʁ ci tu-the-a ŋu\cmn 我再问一个问题\cmn tʂu tɤ-thu-t-a\cmn 我问了路\end{exemple}\begin{sous-entrée}
\vedette{\hypertarget{}{\papi{ rɤthu}}}\markboth{rɤthu}{}\classe{vi}
\begin{définition}\ 
\begin{déclaration}\grammar{apass}\end{déclaration}\end{définition}
\begin{définition}\fra demander à quelqu'un\end{définition}
\begin{définition}\cmn 问某人;向某人请教\end{définition}
\begin{exemple}\jya ɯʑo ɯ-ɕki ɕɯ-rɤthu-a\cmn 我去向他请教\end{exemple}
\begin{exemple}\jya a-ɕki tɤ-tɯ-rɤthu\cmn 你问了我\end{exemple}
\begin{relation-sémantique}\confer{
\hyperlink{Ⓔrɤthuthe}{\textit{ \papi{rɤthuthe}}}
}\end{relation-sémantique}
\end{sous-entrée}\end{entrée}

\begin{entrée}
\vedette{\hypertarget{Ⓔtha}{\papi{ tha}}}\markboth{tha}{}\classe{adv}
\begin{définition}\fra dans un moment\end{définition}
\begin{définition}\cmn (不然)过一会儿就\end{définition}
\begin{exemple}\jya tɕe tha tɕe a-pɯ-ŋu\cmn 一会再说吧\end{exemple}
\begin{exemple}\jya nɤ-ŋga tɤ-ŋge ma tha tɯ-nɯtɕhomba\cmn 你把衣服穿上,不然会感冒\end{exemple}\end{entrée}

\begin{entrée}
\vedette{\hypertarget{Ⓔthamaka}{\papi{ thamaka}}}\markboth{thamaka}{}
\classe{n}
\begin{définition}\fra tabac\end{définition}
\begin{définition}\cmn 烟
\begin{déclaration} \étymologie{\papi{tʰa.mag}}\end{déclaration}\end{définition}
\begin{exemple}\jya thamaka a-nɯ-tɯ-ftɕɤt\cmn 你戒烟吧\end{exemple}\end{entrée}

\begin{entrée}
\vedette{\hypertarget{Ⓔthamatham}{\papi{ thamatham}}}\markboth{thamatham}{}
\begin{relation-sémantique}\confer{
\hyperlink{Ⓔthamtham}{\textit{ \papi{thamtham}}}
}\end{relation-sémantique}\end{entrée}

\begin{entrée}
\vedette{\hypertarget{Ⓔthamtɕɤt}{\papi{ thamtɕɤt}}}\markboth{thamtɕɤt}{}\classe{adv}
\begin{définition}\fra complètement, tout\end{définition}
\begin{définition}\cmn 全部
\begin{déclaration} \étymologie{\papi{tʰams.tɕad}}\end{déclaration}\end{définition}\end{entrée}

\begin{entrée}
\vedette{\hypertarget{Ⓔthamtham}{\papi{ thamtham}}}\markboth{thamtham}{}\classe{adv}
\begin{définition}\fra maintenant\end{définition}
\begin{définition}\cmn 现在
\begin{déclaration} \étymologie{\papi{tʰam}}\end{déclaration}\end{définition}
\end{entrée}

\begin{entrée}
\vedette{\hypertarget{ⒺthaŋⒽ1}{\papi{ thaŋ}}}\markboth{thaŋ}{}\homonyme{1}
\classe{n}
\begin{définition}\fra plaine\end{définition}
\begin{définition}\cmn 平坝
\begin{déclaration} \étymologie{\papi{tʰaŋ}}\end{déclaration}\end{définition}\end{entrée}

\begin{entrée}
\vedette{\hypertarget{ⒺthaŋⒽ2}{\papi{ thaŋ}}}\markboth{thaŋ}{}\homonyme{2}
\classe{part}
\begin{définition}\fra marqueur de supposition\end{définition}
\begin{définition}\cmn 表示推测\end{définition}
\begin{exemple}\jya wo nɯnɯ wuma ʑo mɯm thaŋ nɤ!\cmn 这个应该很好吃吧!\end{exemple}\end{entrée}

\begin{entrée}
\vedette{\hypertarget{Ⓔthaʁɕa}{\papi{ thaʁɕa}}}\markboth{thaʁɕa}{}\classe{n}
\begin{définition}\fra élément du métier à tisser\end{définition}
\begin{définition}\cmn 筘【板板】
\begin{déclaration} \étymologie{\papi{ⁿtʰag.ɕa}}\end{déclaration}\end{définition}
\begin{exemple}\jya thaʁɕa nɯ tɤrɤm thɯ-kɤ-βʑoʁ tɕe nɯ-kɤ-ɣɤmpɕu tɕe stɤsmɤt thɯ-kɤ-sɤmtɕoʁ ŋu, thɯ-kɯ-taʁ tɕe kɤ-taʁ ɯ-sqar ɯ-sɤ-lɤt ɯ-spa ŋu\cmn 
\stylefv{thaʁɕa}是把木板削平了以后,两头削尖了,用来让经线和纬线的交叉部分上下移动的工具
\end{exemple}\end{entrée}

\begin{entrée}
\vedette{\hypertarget{Ⓔthaʁmu}{\papi{ thaʁmu}}}\markboth{thaʁmu}{}
\classe{n}
\begin{définition}\fra élément du métier à tisser\end{définition}
\begin{définition}\cmn 榨刀
\begin{déclaration} \étymologie{\papi{ⁿtʰag.ma}}\end{déclaration}\end{définition}
\begin{exemple}\jya thaʁmu nɯ thɯ-kɯ-taʁ tɕe kɤ-taʁ ɯ-sqar ɯ-chɯ-sɤ-ɤsɯɣ ɯ-jlɤβ ɯ-ɲɯ-sɤ-ɤrʁe ɯ-spa ŋu. tɤrɤm ɯ-rkɯ zɯ ɕom kɯ-mba tsa pɯ-kɤ-tshoʁ ci ŋu\cmn 织布时,榨刀是用来把经线和纬线的交叉部分弄紧,然后穿纬线的工具。是边上装着一块薄铁片的木板。\end{exemple}\end{entrée}

\begin{entrée}
\vedette{\hypertarget{Ⓔthaʁŋkhor}{\papi{ thaʁŋkhor}}}\markboth{thaʁŋkhor}{}\classe{n}
\begin{définition}\fra moulin à prière que l'on fait tourner avec les doigts\end{définition}
\begin{définition}\cmn 指捻转经筒
\begin{déclaration} \étymologie{\papi{tʰag.ⁿkʰor}}\end{déclaration}\end{définition}\end{entrée}

\begin{entrée}
\vedette{\hypertarget{Ⓔthaʁtɕɤz}{\papi{ thaʁtɕɤz}}}\markboth{thaʁtɕɤz}{}
\classe{n}
\begin{définition}\fra appareil à tisser\end{définition}
\begin{définition}\cmn 织布机
\begin{déclaration} \étymologie{\papi{tʰag.btɕas}}\end{déclaration}\end{définition}\end{entrée}

\begin{entrée}
\vedette{\hypertarget{Ⓔthaʁ,tɕhot}{\papi{ thaʁ,tɕhot}}}\markboth{thaʁ,tɕhot}{}
\paradigme{\textit{dir :} \jya nɯ-}
\paradigme{\textit{dir :} \jya pɯ-}
\begin{définition}\fra prendre une décision\end{définition}
\begin{définition}\cmn 决定
\begin{déclaration} \étymologie{\papi{tʰag.tɕʰod}}\end{déclaration}\end{définition}
\begin{exemple}\jya thaʁ pɯ-tɕhot-a\cmn 我决定了\end{exemple}
\begin{exemple}\jya tɕi-tɯkrɤz tɤ-ɣe tɕe thaʁ nɯ-tɕhot-tɕi\cmn 我们商议好了就做了决定\end{exemple}
\begin{relation-sémantique}\ComponentA{\classe{n}
 \papi{thaʁ}
}\end{relation-sémantique}
\begin{relation-sémantique}\ComponentB{\classe{vt}
 \papi{tɕhot}
}\end{relation-sémantique}\end{entrée}

\begin{entrée}
\vedette{\hypertarget{Ⓔthaʁtsa}{\papi{ thaʁtsa}}}\markboth{thaʁtsa}{}\classe{n}
\begin{définition}\fra ceinture colorée\end{définition}
\begin{définition}\cmn 花带\end{définition}
\end{entrée}

\begin{entrée}
\vedette{\hypertarget{Ⓔthathor}{\papi{ thathor}}}\markboth{thathor}{}\classe{n}
\begin{définition}\fra partie en métal de la ceinture\end{définition}
\begin{définition}\cmn 腰带的扣子\end{définition}
\begin{relation-sémantique}\confer{
\hyperlink{Ⓔtɕhoma}{\textit{ \papi{tɕhoma}}}
}\end{relation-sémantique}\end{entrée}

\begin{entrée}
\vedette{\hypertarget{Ⓔthawaʁ}{\papi{ thawaʁ}}}\markboth{thawaʁ}{}
\classe{n}
\begin{définition}\fra assiette\end{définition}
\begin{définition}\cmn 木盘子\end{définition}\end{entrée}

\begin{entrée}
\vedette{\hypertarget{Ⓔthaχthi}{\papi{ thaχthi}}}\markboth{thaχthi}{}
\classe{n}
\begin{définition}\fra lanière\end{définition}
\begin{définition}\cmn 背带(用线织成的)\end{définition}\end{entrée}

\begin{entrée}
\vedette{\hypertarget{Ⓔthɤβ}{\papi{ thɤβ}}}\markboth{thɤβ}{} (\variante{lthɤβ}) \classe{n}
\begin{définition}\fra clin d'œil\end{définition}
\begin{définition}\cmn 眨眼\end{définition}
\begin{exemple}\jya a-mɲaʁ thɤβ ʑo tɤ-stu-t-a\cmn 他眨了眼\end{exemple}\end{entrée}

\begin{entrée}
\vedette{\hypertarget{Ⓔthɤfka}{\papi{ thɤfka}}}\markboth{thɤfka}{}\classe{n}
\begin{définition}\fra foyer\end{définition}
\begin{définition}\cmn 灶
\begin{déclaration} \étymologie{\papi{tʰab.ka}}\end{déclaration}\end{définition}\end{entrée}

\begin{entrée}
\vedette{\hypertarget{Ⓔthɤfkɤlɤɣi}{\papi{ thɤfkɤlɤɣi}}}\markboth{thɤfkɤlɤɣi}{}
\classe{n}
\begin{définition}\fra cendre végétale\end{définition}
\begin{définition}\cmn 草木灰\end{définition}
\begin{relation-sémantique}\synonyme{
\hyperlink{Ⓔsqhɤthɤlɤɣi}{\textit{ \papi{sqhɤthɤlɤɣi}}}
}\end{relation-sémantique}\end{entrée}

\begin{entrée}
\vedette{\hypertarget{Ⓔthɤjbra}{\papi{ thɤjbra}}}\markboth{thɤjbra}{}\classe{n}
\begin{définition}\fra type de herse\end{définition}
\begin{définition}\cmn 簧耙\end{définition}\end{entrée}

\begin{entrée}
\vedette{\hypertarget{Ⓔthɤjco}{\papi{ thɤjco}}}\markboth{thɤjco}{}\classe{n}
\begin{définition}\fra palanquin\end{définition}
\begin{définition}\cmn 轿子
\begin{déclaration} \étymologie{\papi{\stylefn{抬轿}}}\end{déclaration}\end{définition}\end{entrée}

\begin{entrée}
\vedette{\hypertarget{Ⓔthɤjtɕu}{\papi{ thɤjtɕu}}}\markboth{thɤjtɕu}{}\classe{pro}
\begin{définition}\fra quand\end{définition}
\begin{définition}\cmn 什么时候\end{définition}
\begin{exemple}\jya thɤjtɕu jamar tɯ-lɤt?\cmn 你大概什么时候打电话?\end{exemple}
\begin{exemple}\jya thɤjtɕu chiz tɯ-nɯɣi kɯ?\cmn 不知道你什么时候回来?\end{exemple}\end{entrée}

\begin{entrée}
\vedette{\hypertarget{Ⓔthɤlwa}{\papi{ thɤlwa}}}\markboth{thɤlwa}{}
\classe{n}
\paradigme{\textit{comit :} \jya kɤ́thɤlwɯlwa}
\begin{définition}\fra terre\end{définition}
\begin{définition}\cmn 土
\begin{déclaration} \étymologie{\papi{tʰal.ba}}\end{déclaration}\end{définition}\end{entrée}

\begin{entrée}
\vedette{\hypertarget{Ⓔthɤlwaɲaʁ}{\papi{ thɤlwaɲaʁ}}}\markboth{thɤlwaɲaʁ}{}\classe{n}
\begin{définition}\fra tchernozyom\end{définition}
\begin{définition}\cmn 黑土地\end{définition}
\begin{relation-sémantique}\confer{
\hyperlink{Ⓔɲaʁ}{\textit{ \papi{ɲaʁ}}}
}\end{relation-sémantique}\end{entrée}

\begin{entrée}
\vedette{\hypertarget{ⒺthɤrⒽ1}{\papi{ thɤr}}}\markboth{thɤr}{}\homonyme{1}
\classe{vi}
\paradigme{\textit{dir :} \jya tɤ-}
\begin{définition}\fra se sauver\end{définition}
\begin{définition}\cmn 保命
\begin{déclaration}\use{古语}\end{déclaration}
\begin{déclaration} \étymologie{\papi{tʰar}}\end{déclaration}\end{définition}
\begin{exemple}\jya to-thɤr\cmn 他得救了\end{exemple}\end{entrée}

\begin{entrée}
\vedette{\hypertarget{ⒺthɤrⒽ2}{\papi{ thɤr}}}\markboth{thɤr}{}\homonyme{2}
\classe{vs}
\paradigme{\textit{dir :} \jya tɤ-}
\begin{définition}\fra complet\end{définition}
\begin{définition}\cmn 满满\end{définition}
\begin{exemple}\jya tɯ-xpa kɯ-thɤr ʑo\cmn 整整一年\end{exemple}
\begin{relation-sémantique}\synonyme{
\hyperlink{Ⓔmtshɤt}{\textit{ \papi{mtshɤt}}}
}\end{relation-sémantique}\end{entrée}

\begin{entrée}
\vedette{\hypertarget{Ⓔthɤstɯɣ}{\papi{ thɤstɯɣ}}}\markboth{thɤstɯɣ}{}\classe{pro}
\begin{définition}\fra combien\end{définition}
\begin{définition}\cmn 多少\end{définition}
\begin{exemple}\jya thɤstɯɣ tɯ-zɣɯt ?\cmn 你有多大?\end{exemple}
\begin{exemple}\jya thɤstɯɣ thɯ-tɯ-ɤzɣɯt ?\cmn 你多大了?\end{exemple}
\begin{exemple}\jya nɤʑo kɯrɯskɤt pɯ-tɯ-βzjoz nɯ thɤstɯɣ to-tsu ?\cmn 你藏语学了多久?\end{exemple}
\begin{exemple}\jya thɤstɯ-tɯrpa\cmn 几斤\end{exemple}\end{entrée}

\begin{entrée}
\vedette{\hypertarget{Ⓔthɤtɕɯ}{\papi{ thɤtɕɯ}}}\markboth{thɤtɕɯ}{}
\classe{n}
\begin{définition}\fra marteau\end{définition}
\begin{définition}\cmn 二锤\end{définition}
\begin{exemple}\jya thɤtɕɯ nɯ kɯ-rɤznde ra ɣɯ nɯ-laʁdɤn ŋu rdɤstaʁ ra ɯ-sɤz-ɣɤβdoʁβdi ŋu, ɯ-ku nɯ ɕom tɕe ɯ-jɯ laʁjɯɣ thɯ-kɤ-tshoʁ ci ŋu\cmn 榔头是石匠的工具,是用来修理石头的,头是铁作成,把子是一节木棒。\end{exemple}\end{entrée}

\begin{entrée}
\vedette{\hypertarget{Ⓔthɤwum}{\papi{ thɤwum}}}\markboth{thɤwum}{}\classe{n}
\begin{définition}\fra une espèce d'arbre\end{définition}
\begin{définition}\cmn 乔木的一种【马鹿柴】\end{définition}
\begin{exemple}\jya thɤwum nɯ si kɯ-mbro kɯ-jpum tsa ci ŋu, aʁɤndɯndɤt ʑo tu-ɬoʁ cha. wuma ʑo ɲɯ-ɤɣɯrtɯrtaʁ cha. ɯ-jwaʁ artɯm tɕe ɯ-ku lu-omtɕoʁ ŋu. si ɯ-ru cho ɯ-rtaʁ nɯ kɯ-ɣɯrni ŋu. ɯ-si mɤ-ngɯt, ndoʁ, sna me, kɤ-nɯβlɯ kɯnɤ khro mɤ-pe, ɯ-mɯntoʁ kɯ-wɣrum ɲɯ-lɤt ŋu. ɯ-mɯntoʁ ɯ-jɯ nɯ zri tsa ɯ-mat thɯ-tɯt tɕe kɯ-ɣɯrni ʂɣɤlʂɣɤl ʑo kɯ-pa ŋu, laŋlaŋ ɯ-mat cho naχtɕɯɣ, tɕeri kɤ-ndza mɤ-sna ma wuma ʑo qiaβ.\cmn 
马鹿柴是长得又高又粗的树种,到处都可以生长,长很多枝条。叶子是圆形的,一头尖。树干和树枝都是紫红色的。木质不结实,很脆,不能做什么材料,连烧火都不好。开白色的花。花梗有点长。果实成熟时是红而透明的,像\stylefv{laŋlaŋ}的果实,但是不能吃,因为很苦。
\end{exemple}\end{entrée}

\begin{entrée}
\vedette{\hypertarget{Ⓔthɣe}{\papi{ thɣe}}}\markboth{thɣe}{}
\classe{n}
\begin{définition}\fra gland\end{définition}
\begin{définition}\cmn 橡子\end{définition}
\begin{relation-sémantique}\confer{
\hyperlink{Ⓔnɯthɣe}{\textit{ \papi{nɯthɣe}}}
}\end{relation-sémantique}\end{entrée}

\begin{entrée}
\vedette{\hypertarget{Ⓔthoɲa}{\papi{ thoɲa}}}\markboth{thoɲa}{}\classe{n}
\begin{définition}\fra ovidé\end{définition}
\begin{définition}\cmn 羊\end{définition}
\end{entrée}

\begin{entrée}
\vedette{\hypertarget{Ⓔthoŋkɤn}{\papi{ thoŋkɤn}}}\markboth{thoŋkɤn}{}\classe{n}
\begin{définition}\fra récipient en cuivre\end{définition}
\begin{définition}\cmn 红铜铸成的罐子,没有盖子\end{définition}
\end{entrée}

\begin{entrée}
\vedette{\hypertarget{Ⓔthoŋlaʁ}{\papi{ thoŋlaʁ}}}\markboth{thoŋlaʁ}{}\classe{n}
\begin{définition}\fra une période, un endroit\end{définition}
\begin{définition}\cmn 一段(时间、地方)\end{définition}
\begin{exemple}\jya nɤʑo jɤxtshi jɤ-tɯ-ɣe ɯ-thoŋlaʁ nɯ mɯ́j-ɣɤndʐo, mɯ́j-ɣɯtshɤdɯɣ tɕe ɲɯ-sɤscit\cmn 你这一次来的那一段时间既不冷也不热,很舒服\end{exemple}\end{entrée}

\begin{entrée}
\vedette{\hypertarget{Ⓔthoŋthɤr}{\papi{ thoŋthɤr}}}\markboth{thoŋthɤr}{}\classe{n}
\begin{définition}\fra baguette de tassage\end{définition}
\begin{définition}\cmn 推弹杆\end{définition}
\begin{exemple}\jya thoŋthɤr nɯ thɯ́-wɣ-rzoŋ tɕe qandʑi sɤ-ɣnda ɯ-spa ŋu\cmn 推弹杆是用来把子弹塞进枪筒里的工具。\end{exemple}\end{entrée}

\begin{entrée}
\vedette{\hypertarget{ⒺthoʁⒽ2}{\papi{ thoʁ}}}\markboth{thoʁ}{}\homonyme{2}
\classe{n}
\begin{définition}\fra foudre\end{définition}
\begin{définition}\cmn 霹雷
\begin{déclaration} \étymologie{\papi{tʰog}}\end{déclaration}\end{définition}
\begin{exemple}\jya thoʁ pjɤ-ɣi\cmn 打雷了\end{exemple}
\begin{exemple}\jya ɣɯjpa kɯ-nɯtʂoŋtshaβ lo-ɕe-nɯ tɕe, tɤ-tɕɯ ci thoʁ kɯ tó-wɣ-tsɯɣ tɕe pjɤ-si\cmn 今年,他们去采虫草时,有一个男人给雷劈死了。\end{exemple}\end{entrée}

\begin{entrée}
\vedette{\hypertarget{ⒺthoʁⒽ1}{\papi{ thoʁ}}}\markboth{thoʁ}{}\homonyme{1}\classe{vt}
\paradigme{\textit{dir :} \jya pɯ-}
\begin{définition}\fra marcher sur\end{définition}
\begin{définition}\cmn 踏\end{définition}
\begin{exemple}\jya a-mi pɯ-thoʁ-a (a-mi pɯ-ta-t-a, a-mi kɯ pɯ-zrɤtɕaʁ-a)\cmn 我踩上了\end{exemple}
\begin{relation-sémantique}\synonyme{
\hyperlink{Ⓔrɤtɕaʁ}{\textit{ \papi{rɤtɕaʁ}}}
}\end{relation-sémantique}\end{entrée}

\begin{entrée}
\vedette{\hypertarget{Ⓔthoʁltɕi}{\papi{ thoʁltɕi}}}\markboth{thoʁltɕi}{}\classe{n}
\begin{définition}\fra fer météoritique?\end{définition}
\begin{définition}\cmn 铁块(陨石)
\end{définition}\end{entrée}

\begin{entrée}
\vedette{\hypertarget{Ⓔtho,thɯɣ}{\papi{ tho,thɯɣ}}}\markboth{tho,thɯɣ}{} (\variante{thoʁ,thɯɣ}) \paradigme{\textit{dir :} \jya tɤ-}
\begin{définition}\fra concordant\end{définition}
\begin{définition}\cmn 相符;一致(消息);很巧\end{définition}
\begin{exemple}\jya tho ɲɯ-thɯɣ\cmn 是相符的\end{exemple}
\begin{exemple}\jya ɯʑo kɯ ta-tɯt cho nɤj tu-tɯ-ti nɯ tho ɲɯ-thɯɣ\cmn 他说的和你说的完全相符\end{exemple}
\begin{relation-sémantique}\ComponentA{\classe{n}
 \papi{tho}
}\end{relation-sémantique}
\begin{relation-sémantique}\ComponentB{\classe{vs}
\hyperlink{ⒺthɯɣⒽ1}{\textit{ \papi{thɯɣ}}}
}\end{relation-sémantique}\end{entrée}

\begin{entrée}
\vedette{\hypertarget{Ⓔthotsi}{\papi{ thotsi}}}\markboth{thotsi}{}\classe{n}
\begin{définition}\fra sceau\end{définition}
\begin{définition}\cmn 印章(在馍馍上)\end{définition}
\begin{exemple}\jya qajɣi ɯ-taʁ thotsi kɤ-ta\cmn 要在馍馍上盖印章\end{exemple}
\begin{relation-sémantique}\synonyme{
\hyperlink{Ⓔmthɯmɤr}{\textit{ \papi{mthɯmɤr}}}
}\end{relation-sémantique}\end{entrée}

\begin{entrée}
\vedette{\hypertarget{Ⓔthoχtɤm}{\papi{ thoχtɤm}}}\markboth{thoχtɤm}{}
\classe{n}
\begin{définition}\fra impôt\end{définition}
\begin{définition}\cmn 税\end{définition}
\begin{exemple}\jya ɯ-thoχtɤm lɤ-kho-j\cmn 我们(给土司)交了粮食\end{exemple}\end{entrée}

\begin{entrée}
\vedette{\hypertarget{Ⓔthrɤβthrɤβ}{\papi{ thrɤβthrɤβ}}}\markboth{thrɤβthrɤβ}{}\classe{idph.2}
\begin{définition}\fra dont l'épaisseur n'est pas uniforme (soupe de riz)\end{définition}
\begin{définition}\cmn 形容稀稠不均匀的样子(稀饭)\end{définition}
\begin{exemple}\jya tɯtshi thrɤβthrɤβ ʑo ɲɯ-pa\cmn 稀饭稀稠不均匀\end{exemple}\end{entrée}

\begin{entrée}
\vedette{\hypertarget{ⒺthɯⒽ1}{\papi{ thɯ}}}\markboth{thɯ}{}\homonyme{1}\classe{vt}\acception{1}
\paradigme{\textit{dir :} \jya kɤ-}
\paradigme{\textit{dir :} \jya pɯ-}
\begin{définition}\fra monter une tente, faire un pont\end{définition}
\begin{définition}\cmn 搭(桥、帐篷)\end{définition}
\begin{exemple}\jya ndzom kɤ-thɯ-t-a (kú-wɣ-ta, kú-wɣ-thɯ)\cmn 我搭了桥\end{exemple}
\begin{exemple}\jya zgɤr pɯ-thɯ-t-a\cmn 我搭了帐篷\end{exemple}\acception{2}
\begin{définition}\fra construire une route\end{définition}
\begin{définition}\cmn 修(路)\end{définition}
\begin{exemple}\jya tʂu lɤ-thɯ-t-a (=lɤ-tɕat-a)\cmn 我修了路\end{exemple}\acception{3}
\begin{définition}\fra séparer les fils\end{définition}
\begin{définition}\cmn 牵(线)\end{définition}
\begin{exemple}\jya kɤ-pɣo lɤ-jɤɣ tɕe, kɤtaʁ kɤ-thɯ-t-a\cmn 搓完线,我就把它牵了\end{exemple}\acception{4}
\begin{définition}\fra laisser (une trace)\end{définition}
\begin{définition}\cmn 留下痕迹\end{définition}
\begin{exemple}\jya ɯ-jroʁ jo-thɯ (jo-tɕɤt)\cmn 它留了痕迹\end{exemple}
\begin{relation-sémantique}\confer{
\hyperlink{ⒺndɯⒽ1}{\textit{ \papi{ndɯ1}}}
}\end{relation-sémantique}
\end{entrée}

\begin{entrée}
\vedette{\hypertarget{ⒺthɯⒽ2}{\papi{ thɯ}}}\markboth{thɯ}{}\homonyme{2}
\classe{vi}
\paradigme{\textit{dir :} \jya nɯ-}
\begin{définition}\fra grave\end{définition}
\begin{définition}\cmn 严重\end{définition}
\begin{exemple}\jya wuma sthɯci mɯ́j-thɯ ɲɯ-ti, ɯ́-ŋu\cmn 他说没有那么严重,是吗?\end{exemple}
\begin{exemple}\jya a-rna ɲɯ-thɯ ɲɯ-ŋu tɕe, koŋla mɯ́j-mtsham-a\cmn 我耳背,听不清楚\end{exemple}
\begin{exemple}\jya ɯ-ku kɯ-mɲɤm ɲɯ-thɯ\cmn 他头疼得很厉害\end{exemple}\begin{sous-entrée}
\vedette{\hypertarget{}{\papi{ ɣɤthɯ}}}\markboth{ɣɤthɯ}{}\classe{vt}
\paradigme{\textit{dir :} \jya nɯ-}
\begin{définition}\ 
\begin{déclaration}\grammar{caus}\end{déclaration}\end{définition}
\begin{définition}\fra aggraver\end{définition}
\begin{définition}\cmn 令……变得更严重\end{définition}
\begin{relation-sémantique}\confer{
\hyperlink{Ⓔnɤxthɯ}{\textit{ \papi{nɤxthɯ}}}
}\end{relation-sémantique}
\end{sous-entrée}\end{entrée}

\begin{entrée}
\vedette{\hypertarget{Ⓔthɯchu}{\papi{ thɯchu}}}\markboth{thɯchu}{}\classe{adv}
\begin{définition}\fra en aval\end{définition}
\begin{définition}\cmn 在下游\end{définition}
\begin{relation-sémantique}\confer{
\hyperlink{Ⓔɯ-thɤcu}{\textit{ \papi{ɯ-thɤcu}}}
}\end{relation-sémantique}\end{entrée}

\begin{entrée}
\vedette{\hypertarget{Ⓔthɯci}{\papi{ thɯci}}}\markboth{thɯci}{}\classe{pro}
\begin{définition}\fra quelque chose, n'importe quoi, n'importe lequel\end{définition}
\begin{définition}\cmn 某个;随便什么;任何一个\end{définition}
\begin{relation-sémantique}\confer{
\hyperlink{Ⓔthɯthɤci}{\textit{ \papi{thɯthɤci}}}
}\end{relation-sémantique}\begin{sous-entrée}
\vedette{\hypertarget{}{\papi{ thɯci fse ci ndʐa ɕti kɯ}}}\markboth{thɯci fse ci ndʐa ɕti kɯ}{}
\begin{définition}\fra il y a sans doute une raison\end{définition}
\begin{définition}\cmn 也许会有办法;也许会过得去;也许是应该的\end{définition}
\begin{exemple}\jya tɯmgo kɤ-sɯ-ndo mɯ́j-khɯ ri, thɯci fse ci ndʐa ɕti kɯ (mɤ-mtsɯr ndʐa ɕti kɯ)\cmn 他不肯带食物走,他可能自己有办法(他可能不饿)\end{exemple}
\end{sous-entrée}\end{entrée}

\begin{entrée}
\vedette{\hypertarget{ⒺthɯɣⒽ1}{\papi{ thɯɣ}}}\markboth{thɯɣ}{}\homonyme{1}
\classe{n}
\begin{définition}\fra taureau, bouc non castré\end{définition}
\begin{définition}\cmn 种羊;种牛
\begin{déclaration} \étymologie{\papi{tʰug}}\end{déclaration}\end{définition}\end{entrée}

\begin{entrée}
\vedette{\hypertarget{ⒺthɯɣⒽ3}{\papi{ thɯɣ}}}\markboth{thɯɣ}{}\homonyme{3}
\classe{n}
\begin{définition}\fra signe\end{définition}
\begin{définition}\cmn 记号\end{définition}
\begin{exemple}\jya thɯɣ tɤ-ta-t-a\cmn 我打了记号\end{exemple}\end{entrée}

\begin{entrée}
\vedette{\hypertarget{ⒺthɯɣⒽ2}{\papi{ thɯɣ}}}\markboth{thɯɣ}{}\homonyme{2}
\classe{vi}
\paradigme{\textit{dir :} \jya kɤ-}
\begin{définition}\fra être détestable\end{définition}
\begin{définition}\cmn 令人讨厌\end{définition}
\begin{exemple}\jya kɤ-thɯɣ!\cmn 糟糕!\end{exemple}
\begin{exemple}\jya nɤʑo ki ndɤre, ko-tɯ-thɯɣ!\cmn 你这个人糟糕到无可救药的地步\end{exemple}
\begin{exemple}\jya nɤʑɯɣ thɯɣ ma!\cmn 到最后受苦就是你自己\end{exemple}
\begin{exemple}\jya ji-zdɯɣ ɲɯ-thɯɣ\cmn 我们在受苦受难\end{exemple}\end{entrée}

\begin{entrée}
\vedette{\hypertarget{Ⓔthɯɣɕɤr}{\papi{ thɯɣɕɤr}}}\markboth{thɯɣɕɤr}{}\classe{n}
\begin{définition}\fra marque des charpentiers sur le bois\end{définition}
\begin{définition}\cmn 木工打记号的线;墨线\end{définition}
\end{entrée}

\begin{entrée}
\vedette{\hypertarget{Ⓔthɯɣskrɯt}{\papi{ thɯɣskrɯt}}}\markboth{thɯɣskrɯt}{}
\classe{n}
\begin{définition}\fra marque des tailleurs sur les tissus\end{définition}
\begin{définition}\cmn 裁缝在布料上用来打记号的线\end{définition}\end{entrée}

\begin{entrée}
\vedette{\hypertarget{Ⓔthɯm}{\papi{ thɯm}}}\markboth{thɯm}{}\classe{n}
\begin{définition}\fra récipient\end{définition}
\begin{définition}\cmn 瓢子\end{définition}
\begin{exemple}\jya ɲchɣaʁthɯm\cmn 桦树皮的瓢子\end{exemple}\end{entrée}

\begin{entrée}
\vedette{\hypertarget{Ⓔthɯraŋ}{\papi{ thɯraŋ}}}\markboth{thɯraŋ}{}
\classe{n}
\begin{définition}\fra nain, une sorte de démon\end{définition}
\begin{définition}\cmn 一种鬼,矮人
\begin{déclaration} \étymologie{\papi{tʰeɦu.raŋ}}\end{déclaration}\end{définition}\end{entrée}

\begin{entrée}
\vedette{\hypertarget{Ⓔthɯrdu}{\papi{ thɯrdu}}}\markboth{thɯrdu}{}\classe{n}
\begin{définition}\fra poids (d'une balance)\end{définition}
\begin{définition}\cmn 砝码,秤砣
\begin{déclaration} \étymologie{\papi{tʰur.rdo}}\end{déclaration}\end{définition}\end{entrée}

\begin{entrée}
\vedette{\hypertarget{Ⓔthɯrnaʁ}{\papi{ thɯrnaʁ}}}\markboth{thɯrnaʁ}{}
\classe{n}
\begin{définition}\fra balance de précision\end{définition}
\begin{définition}\cmn 秤
\begin{déclaration} \étymologie{\papi{tʰur.sraŋ}}\end{déclaration}\end{définition}\end{entrée}

\begin{entrée}
\vedette{\hypertarget{Ⓔthɯrʑi}{\papi{ thɯrʑi}}}\markboth{thɯrʑi}{}\classe{n}
\begin{définition}\fra compassion\end{définition}
\begin{définition}\cmn 同情,怜悯
\begin{déclaration} \étymologie{\papi{tʰugs.rje}}\end{déclaration}\end{définition}
\end{entrée}

\begin{entrée}
\vedette{\hypertarget{Ⓔthɯrʑi,ʑɯ}{\papi{ thɯrʑi,ʑɯ}}}\markboth{thɯrʑi,ʑɯ}{}
\paradigme{\textit{dir :} \jya tɤ-}
\begin{définition}\fra implorer la miséricorde\end{définition}
\begin{définition}\cmn 求饶
\begin{déclaration} \étymologie{\papi{thugs.rdʑe.ʑu}}\end{déclaration}\end{définition}
\begin{exemple}\jya nɤ-ɕki thɯrʑi tu-nɯʑi-a\cmn 我向求饶\end{exemple}
\begin{relation-sémantique}\ComponentA{\classe{n}
\hyperlink{Ⓔthɯrʑi}{\textit{ \papi{thɯrʑi}}}
}\end{relation-sémantique}
\begin{relation-sémantique}\ComponentB{\classe{vt}
\hyperlink{ⒺʑɯⒽ1}{\textit{ \papi{ʑɯ}}}
}\end{relation-sémantique}\end{entrée}

\begin{entrée}
\vedette{\hypertarget{Ⓔthɯthɤci}{\papi{ thɯthɤci}}}\markboth{thɯthɤci}{}\classe{pro}
\begin{définition}\fra quel, lesquels\end{définition}
\begin{définition}\cmn 一些什么\end{définition}
\begin{exemple}\jya phɯrkhɯɣ ɯ-ŋgɯ nɤ-ŋga thɯthɤci arku\cmn 在你挎包装了一些什么衣服\end{exemple}
\begin{relation-sémantique}\confer{
\hyperlink{Ⓔthɯci}{\textit{ \papi{thɯci}}}
}\end{relation-sémantique}\end{entrée}

\begin{entrée}
\vedette{\hypertarget{Ⓔti}{\papi{ ti}}}\markboth{ti}{}
\classe{vt}\acception{1}
\paradigme{\textit{dir :} \jya tɤ-}
\paradigme{\textit{past stem :} \jya tɯt}
\paradigme{\textit{generic :} \jya kɯ-ti}
\paradigme{\textit{inf.1sg :} \jya to-ti-a}
\begin{définition}\fra dire\end{définition}
\begin{définition}\cmn 说
\begin{déclaration}\use{参看\stylefv{ʑɣɤta}或\stylefv{ʑɣɤχtɤt}“靠在”}\end{déclaration}\end{définition}
\begin{exemple}\jya nɯ tu-kɯ-ti mɤ-ŋgrɤl\cmn 不能这样说\end{exemple}
\begin{exemple}\jya li ci tɤ-ti\cmn 你再说一遍\end{exemple}
\begin{exemple}\jya pɤjkhu a-ʁa tu me mɤ-xsi, a-kɤ-ti me\cmn 我现在说不准\end{exemple}
\begin{exemple}\jya tɕhi tu-ti-a a-pɯ-ŋu ɲɯ-ra?\cmn 我本应该说什么呢?\end{exemple}
\begin{exemple}\jya ki kɤ-nɤma ki nɤj nɤ-taʁ ɲɯ-ti-a ŋu nɤ!\cmn 这件事情全靠你了\end{exemple}
\begin{exemple}\jya ki kɤ-nɤma ki ɯʑo ɯ-taʁ ɲɯ-ti-a ntshi\cmn 这件事情只好靠他了\end{exemple}
\begin{exemple}\jya tɕhi nɯ mɤ-tɯ-nɯ-ti\cmn 你什么话都说出来\end{exemple}
\begin{exemple}\jya nɤ-kɤ-nɯ-ti ci ɣɤʑu\cmn 你还好意思说\end{exemple}\acception{2}
\paradigme{\textit{dir :} \jya thɯ-}
\paradigme{\textit{dir :} \jya pɯ-}
\begin{définition}\fra annoncer\end{définition}
\begin{définition}\cmn 宣布\end{définition}
\begin{exemple}\jya tɤ-kɤ-nɯkrɤz nɯ tɯrme ra nɯ-ɕki chɯ-tɯ-ti ra\cmn 你要向人们宣布决议\end{exemple}\begin{sous-entrée}
\vedette{\hypertarget{}{\papi{ kɤti}}}\markboth{kɤti}{}
\begin{définition}\fra on dirait que\end{définition}
\begin{définition}\cmn 看起来,表面上\end{définition}
\begin{exemple}\jya pjɯ-rɤβzjoz kɤti ŋu\cmn 他看起来是在读书\end{exemple}
\end{sous-entrée}\begin{sous-entrée}
\vedette{\hypertarget{}{\papi{ nɯɣɯti}}}\markboth{nɯɣɯti}{}\classe{vs}
\begin{définition}\fra facile à dire\end{définition}
\begin{définition}\cmn 容易说\end{définition}
\end{sous-entrée}\begin{sous-entrée}
\vedette{\hypertarget{}{\papi{ sɯti}}}\markboth{sɯti}{}
\paradigme{\textit{past stem :} \jya sɯtɯt}
\begin{définition}\ 
\begin{déclaration}\grammar{caus}\end{déclaration}\end{définition}
\begin{définition}\fra faire parler\end{définition}
\begin{définition}\cmn 使讲话;播放\end{définition}
\begin{exemple}\jya kɯ-dɤn tsa kɤ-sɯti\cmn 播放多一点\end{exemple}\classe{vt}
\end{sous-entrée}\end{entrée}

\begin{entrée}
\vedette{\hypertarget{Ⓔtoŋku}{\papi{ toŋku}}}\markboth{toŋku}{}\classe{n}
\begin{définition}\fra casserole en cuivre\end{définition}
\begin{définition}\cmn 铜、生铁铸造的锅子【鼎锅】\end{définition}
\end{entrée}

\begin{entrée}
\vedette{\hypertarget{Ⓔtoŋkɤr}{\papi{ toŋkɤr}}}\markboth{toŋkɤr}{}
\classe{n}
\begin{définition}\fra conque\end{définition}
\begin{définition}\cmn 螺
\begin{déclaration} \étymologie{\papi{duŋ.dkar}}\end{déclaration}\end{définition}\end{entrée}

\begin{entrée}
\vedette{\hypertarget{Ⓔtoŋtsi}{\papi{ toŋtsi}}}\markboth{toŋtsi}{}\classe{n}
\begin{définition}\fra centime\end{définition}
\begin{définition}\cmn 一角
\begin{déclaration} \étymologie{\papi{doŋ.rtse}}\end{déclaration}\end{définition}
\end{entrée}

\begin{entrée}
\vedette{\hypertarget{Ⓔtoʁde}{\papi{ toʁde}}}\markboth{toʁde}{}\classe{adv}
\begin{définition}\fra moment\end{définition}
\begin{définition}\cmn 一会儿\end{définition}
\end{entrée}

\begin{entrée}
\vedette{\hypertarget{Ⓔtsu}{\papi{ tsu}}}\markboth{tsu}{}\classe{vi}\acception{1}
\begin{définition}\fra avoir le temps\end{définition}
\begin{définition}\cmn 来得及\end{définition}
\begin{exemple}\jya mɯ́j-tsu-a\cmn 我来不及\end{exemple}
\begin{exemple}\jya pɤjkhu tɤ-rʑaʁ ɣɤʑu, ɲɯ-tsu\cmn 还有时间,还来得及\end{exemple}\acception{2}
\paradigme{\textit{dir :} \jya tɤ-}
\begin{définition}\fra se passer ... (temps)\end{définition}
\begin{définition}\cmn 到(时间)\end{définition}
\begin{exemple}\jya kɯmŋu-xpa pɯ-tsu\cmn 已经过了五年\end{exemple}
\begin{exemple}\jya kɯmŋu-xpa to-tsu\cmn 到了五年\end{exemple}
\begin{exemple}\jya kɯtʂɤ-sŋi ma mɯ-jɤ-tsu-a\cmn 我只来了六天了\end{exemple}\acception{3}
\paradigme{\textit{dir :} \jya pɯ-}
\begin{définition}\fra en arriver au point de ...\end{définition}
\begin{définition}\cmn 到了……的地步\end{définition}
\begin{exemple}\jya ɲɤ-nɯzdɯɣ ʑo pjɤ-tsu\cmn 已经到了非常担心的地步了\end{exemple}
\begin{exemple}\jya nɤʑo ʑaʑa ʑo mɯ-jɤ-tɯ-ɤzɣɯt tɕe, nɯ-ta-nɯzdɯɣ ʑo pɯ-tsu\cmn 你很久都不到,我已经到了很担心你的地步了\end{exemple}
\begin{exemple}\jya jɤ-ɣe-a pɯ-tsu, mɤʑɯ ɲɯ-nɤrɯra-a ra\cmn 我既然来了,我就再看一下\end{exemple}
\begin{relation-sémantique}\confer{
\hyperlink{Ⓔatsɯtsu}{\textit{ \papi{atsɯtsu}}}
}\end{relation-sémantique}
\begin{relation-sémantique}\confer{
\hyperlink{Ⓔsɤtsu}{\textit{ \papi{sɤtsu}}}
}\end{relation-sémantique}
\begin{relation-sémantique}\confer{
\hyperlink{ⒺsɯxtsuⒽ1}{\textit{ \papi{sɯxtsu1}}}
}\end{relation-sémantique}\end{entrée}

\begin{entrée}
\vedette{\hypertarget{Ⓔtsa}{\papi{ tsa}}}\markboth{tsa}{}\classe{adv}
\begin{définition}\fra plutôt, un peu\end{définition}
\begin{définition}\cmn 比较;稍微
\begin{déclaration} \étymologie{\papi{tsa}}\end{déclaration}\end{définition}
\begin{exemple}\jya nɤʑo ndi tsa nɯ-cit\cmn 你稍微转过去一点\end{exemple}
\begin{exemple}\jya khɯtsa ɯ-ŋgɯ tɯ-ci kɯ-dɤn tsa tɤ-rke\cmn 你在碗里倒多一点水\end{exemple}\end{entrée}

\begin{entrée}
\vedette{\hypertarget{Ⓔtsaβ}{\papi{ tsaβ}}}\markboth{tsaβ}{}\classe{vs}
\begin{définition}\fra fort (alcool)\end{définition}
\begin{définition}\cmn 浓度高(酒)\end{définition}
\begin{exemple}\jya cha ɲɯ-tsaβ\cmn 酒的浓度高\end{exemple}
\begin{exemple}\jya ki tɕheme ki ɯ-sɯm sna ri, ɯ-mtɕhi tsaβ wo!\cmn 这个女子心地好,但是嘴上很泼辣\end{exemple}
\begin{relation-sémantique}\antonyme{
\hyperlink{Ⓔmnu}{\textit{ \papi{mnu}}}
}\end{relation-sémantique}\end{entrée}

\begin{entrée}
\vedette{\hypertarget{Ⓔtsanla}{\papi{ tsanla}}}\markboth{tsanla}{}\classe{n}
\begin{définition}\ 
\begin{déclaration}\grammar{n.lieu}\end{déclaration}\end{définition}
\begin{définition}\fra Btsanlha\end{définition}
\begin{définition}\cmn 小金\end{définition}\end{entrée}

\begin{entrée}
\vedette{\hypertarget{Ⓔtsantʂa}{\papi{ tsantʂa}}}\markboth{tsantʂa}{}\classe{n}
\begin{définition}\fra if\end{définition}
\begin{définition}\cmn 红豆杉\end{définition}
\end{entrée}

\begin{entrée}
\vedette{\hypertarget{Ⓔtsaŋga}{\papi{ tsaŋga}}}\markboth{tsaŋga}{}
\classe{n}
\begin{définition}\fra corbeau (corvus dauuricus)\end{définition}
\begin{définition}\cmn 达乌里寒鸦\end{définition}\end{entrée}

\begin{entrée}
\vedette{\hypertarget{Ⓔtsaʁ}{\papi{ tsaʁ}}}\markboth{tsaʁ}{}\classe{adv}
\begin{définition}\fra au moins\end{définition}
\begin{définition}\cmn 至少,起码
\begin{déclaration}\use{一般同\stylefv{tɕhi maʁ nɤ}连用}\end{déclaration}\end{définition}
\begin{exemple}\jya tɤ-rɤru tɕe, tɕhi maʁ nɤ nɤ-rŋa tsaʁ pɯ-χtɕi ma\cmn 你起床,起码把脸洗一下\end{exemple}
\begin{exemple}\jya tɕhi maʁ nɤ tɯ-sŋi smɤn tɯ-ɣjɤn kɤ-ndza ra\cmn 一天至少要吃一次药\end{exemple}\end{entrée}

\begin{entrée}
\vedette{\hypertarget{Ⓔtsɤndɤn}{\papi{ tsɤndɤn}}}\markboth{tsɤndɤn}{}\classe{n}
\begin{définition}\fra santal\end{définition}
\begin{définition}\cmn 檀木
\begin{déclaration} \étymologie{\papi{tsan.dan}}\end{déclaration}\end{définition}
\end{entrée}

\begin{entrée}
\vedette{\hypertarget{Ⓔtsɣaʁtsɣaʁ}{\papi{ tsɣaʁtsɣaʁ}}}\markboth{tsɣaʁtsɣaʁ}{}
\classe{idph.2}
\begin{définition}\fra rouge vif\end{définition}
\begin{définition}\cmn 红艳艳
\begin{déclaration}\use{颜色没有\stylefv{tɕɣɤrtɕɣɤr}那么深}\end{déclaration}\end{définition}
\begin{exemple}\jya mɯntoʁ ɲɯ-ɣɯrni tsɣaʁtsɣaʁ ʑo\cmn 花红艳艳\end{exemple}
\begin{relation-sémantique}\confer{
\hyperlink{Ⓔtɕɣɤrtɕɣɤr}{\textit{ \papi{tɕɣɤrtɕɣɤr}}}
}\end{relation-sémantique}\end{entrée}

\begin{entrée}
\vedette{\hypertarget{Ⓔtsɣi}{\papi{ tsɣi}}}\markboth{tsɣi}{}
\classe{vi}
\paradigme{\textit{dir :} \jya pɯ-}
\paradigme{\textit{dir :} \jya nɯ-}
\begin{définition}\fra pourrir\end{définition}
\begin{définition}\cmn 腐烂\end{définition}
\begin{exemple}\jya @yangyu pjɤ-tsɣi\cmn 洋芋腐烂了\end{exemple}
\begin{exemple}\jya tɯ-ŋga pjɤ-tsɣi\cmn 衣服烂了\end{exemple}
\begin{exemple}\jya ɕoŋtɕa pjɤ-tsɣi\cmn 木料腐烂了\end{exemple}\begin{sous-entrée}
\vedette{\hypertarget{}{\papi{ sɯtsɣi}}}\markboth{sɯtsɣi}{}\classe{vt}
\paradigme{\textit{dir :} \jya pɯ-}
\begin{définition}\ 
\begin{déclaration}\grammar{caus}\end{déclaration}\end{définition}
\begin{définition}\fra laisser pourrir\end{définition}
\begin{définition}\cmn 让腐烂\end{définition}
\begin{exemple}\jya kɯki kɤ-ndza ki ʑa mɯ-tɤ-nɯβdaʁ-a tɕe, pjɤ-sɯtsɣi-t-a\cmn 我很久没有管这个食物,它就腐烂了\end{exemple}
\end{sous-entrée}\end{entrée}

\begin{entrée}
\vedette{\hypertarget{Ⓔtshu}{\papi{ tshu}}}\markboth{tshu}{}\classe{vs}
\paradigme{\textit{dir :} \jya kɤ-}
\paradigme{\textit{dir :} \jya thɯ-}
\begin{définition}\fra gros\end{définition}
\begin{définition}\cmn 胖\end{définition}
\begin{exemple}\jya paʁ ko-tshu\cmn 猪变胖了\end{exemple}
\begin{exemple}\jya skɤm pjɤ-tshu\cmn 菜牛是胖的\end{exemple}
\begin{exemple}\jya nɤ-tɕɯ ɯ-ɲɯ́-tshu?\cmn 你儿子胖不胖?\end{exemple}\begin{sous-entrée}
\vedette{\hypertarget{}{\papi{ sɯxtshu}}}\markboth{sɯxtshu}{}\classe{vt}
\paradigme{\textit{dir :} \jya thɯ-}
\begin{définition}\fra faire grossir\end{définition}
\begin{définition}\cmn 催肥;令……变胖\end{définition}
\begin{exemple}\jya paʁ kɤ-skaɣ-a tɕe thɯ-sɯxtshu-t-a\cmn 我把猪喂得很肥了\end{exemple}
\end{sous-entrée}\begin{sous-entrée}
\vedette{\hypertarget{}{\papi{ zɣɤsɯxtshu}}}\markboth{zɣɤsɯxtshu}{}\classe{vi}
\begin{définition}\fra se faire grossir\end{définition}
\begin{définition}\cmn 令自己变胖\end{définition}
\begin{exemple}\jya thamtham kɯ-xtɕi ra koŋla mɯ-chɯ-rɯndzɤtshi-nɯ tɕe, mɯ-chɯ-ʑɣɤsɯxtshu-nɯ ɲɯ-ŋu\cmn 现在年轻女子吃得少,是为了不要令自己变胖\end{exemple}
\end{sous-entrée}\end{entrée}

\begin{entrée}
\vedette{\hypertarget{Ⓔtsha}{\papi{ tsha}}}\markboth{tsha}{}\classe{n}
\begin{définition}\fra sel\end{définition}
\begin{définition}\cmn 盐
\begin{déclaration} \étymologie{\papi{tsʰwa}}\end{déclaration}\end{définition}
\end{entrée}

\begin{entrée}
\vedette{\hypertarget{Ⓔtshajqajɯ}{\papi{ tshajqajɯ}}}\markboth{tshajqajɯ}{}\classe{n}
\begin{définition}\fra puceron\end{définition}
\begin{définition}\cmn 蚜虫\end{définition}\end{entrée}

\begin{entrée}
\vedette{\hypertarget{Ⓔtshala}{\papi{ tshala}}}\markboth{tshala}{}
\classe{n}
\begin{définition}\fra soudage\end{définition}
\begin{définition}\cmn (焊接用的)铁水
\begin{déclaration} \étymologie{\papi{tsʰa.la}}\end{déclaration}\end{définition}
\begin{exemple}\jya tshala ka-lɤt\cmn 他(把东西)焊了。\end{exemple}\end{entrée}

\begin{entrée}
\vedette{\hypertarget{Ⓔtshaŋ}{\papi{ tshaŋ}}}\markboth{tshaŋ}{}\classe{n}
\begin{définition}\fra armoire\end{définition}
\begin{définition}\cmn 柜子\end{définition}
\begin{relation-sémantique}\synonyme{
\hyperlink{Ⓔwaŋtshaŋ}{\textit{ \papi{waŋtshaŋ}}}
}\end{relation-sémantique}\end{entrée}

\begin{entrée}
\vedette{\hypertarget{Ⓔtshaŋlaŋ}{\papi{ tshaŋlaŋ}}}\markboth{tshaŋlaŋ}{}
\classe{n}
\begin{définition}\fra clochette\end{définition}
\begin{définition}\cmn 铃\end{définition}\end{entrée}

\begin{entrée}
\vedette{\hypertarget{Ⓔtshapa}{\papi{ tshapa}}}\markboth{tshapa}{}\classe{n}
\begin{définition}\ 
\begin{déclaration}\grammar{n.lieu}\end{déclaration}\end{définition}
\begin{définition}\fra l'un des hameaux de Gyutshapa\end{définition}
\begin{définition}\cmn 二茶村的大队之一\end{définition}\end{entrée}

\begin{entrée}
\vedette{\hypertarget{Ⓔtshaʁ}{\papi{ tshaʁ}}}\markboth{tshaʁ}{}
\classe{n}
\begin{définition}\fra crible\end{définition}
\begin{définition}\cmn 筛子
\begin{déclaration} \étymologie{\papi{tsʰags}}\end{déclaration}\end{définition}
\begin{exemple}\jya ki tɤɕi ki tshaʁ pɯ-lat-a\cmn 我筛了青稞\end{exemple}
\begin{exemple}\jya tshaʁ ɯ-mɲaʁ\cmn 筛子的漏孔,眼子\end{exemple}
\begin{relation-sémantique}\confer{
\hyperlink{Ⓔsɯxtshaʁ}{\textit{ \papi{sɯxtshaʁ}}}
}\end{relation-sémantique}\end{entrée}

\begin{entrée}
\vedette{\hypertarget{Ⓔtshatsha}{\papi{ tshatsha}}}\markboth{tshatsha}{}\classe{n}
\begin{définition}\fra impressions sur argile\end{définition}
\begin{définition}\cmn (用模型印出来的)小泥像
\begin{déclaration} \étymologie{\papi{tsha.tsha}}\end{déclaration}\end{définition}
\begin{relation-sémantique}\confer{
\hyperlink{Ⓔtshɤcɯm}{\textit{ \papi{tshɤcɯm}}}
}\end{relation-sémantique}
\begin{relation-sémantique}\confer{
\hyperlink{Ⓔtshɤrkɯ}{\textit{ \papi{tshɤrkɯ}}}
}\end{relation-sémantique}\end{entrée}

\begin{entrée}
\vedette{\hypertarget{Ⓔtshawa}{\papi{ tshawa}}}\markboth{tshawa}{}
\classe{n}
\begin{définition}\fra tuberculose\end{définition}
\begin{définition}\cmn 肺结核
\begin{déclaration} \étymologie{\papi{tsʰa.ba}}\end{déclaration}\end{définition}\end{entrée}

\begin{entrée}
\vedette{\hypertarget{Ⓔtshɤcɯm}{\papi{ tshɤcɯm}}}\markboth{tshɤcɯm}{}\classe{n}
\begin{définition}\fra petite maison où l'on met les tshatsha\end{définition}
\begin{définition}\cmn 装泥像用的小房子\end{définition}
\begin{relation-sémantique}\confer{
\hyperlink{Ⓔtshatsha}{\textit{ \papi{tshatsha}}}
}\end{relation-sémantique}\end{entrée}

\begin{entrée}
\vedette{\hypertarget{Ⓔtshɤdɯɣ}{\papi{ tshɤdɯɣ}}}\markboth{tshɤdɯɣ}{}\classe{n}
\begin{définition}\fra chaleur\end{définition}
\begin{définition}\cmn 天气热\end{définition}
\begin{relation-sémantique}\confer{
\hyperlink{Ⓔɣɯtshɤdɯɣ}{\textit{ \papi{ɣɯtshɤdɯɣ}}}
}\end{relation-sémantique}
\begin{relation-sémantique}\confer{
\hyperlink{Ⓔnɯtshɤdɯɣ}{\textit{ \papi{nɯtshɤdɯɣ}}}
}\end{relation-sémantique}
\begin{relation-sémantique}\antonyme{
\hyperlink{Ⓔtɤndʐo}{\textit{ \papi{tɤndʐo}}}
}\end{relation-sémantique}\end{entrée}

\begin{entrée}
\vedette{\hypertarget{Ⓔtshɤko}{\papi{ tshɤko}}}\markboth{tshɤko}{}
\classe{n}
\begin{définition}\fra tas de pierre, symbole bouddhique\end{définition}
\begin{définition}\cmn 玛尼堆\end{définition}\end{entrée}

\begin{entrée}
\vedette{\hypertarget{Ⓔtshɤmbɤr}{\papi{ tshɤmbɤr}}}\markboth{tshɤmbɤr}{}\classe{n}
\begin{définition}\fra grande lampe à beurre\end{définition}
\begin{définition}\cmn 最大的酥油灯\end{définition}
\end{entrée}

\begin{entrée}
\vedette{\hypertarget{Ⓔtshɤmdzu}{\papi{ tshɤmdzu}}}\markboth{tshɤmdzu}{}
\classe{n}
\begin{définition}\fra espèce d'arbrisseau\end{définition}
\begin{définition}\cmn 灌木的一种\end{définition}
\begin{exemple}\jya tshɤmdzu nɯ si ci ŋu, wuma ʑo mbɤr, ɯ-mdzu rɲɟi cho mtɕoʁ, ɯ-jwaʁ ɯ-βzɯr ra kɯnɤ ɯ-mdzu tu, ɯ-zrɤm nɯ kɯ-qarŋe kɯ-qiaβ ŋu tɕe tɯ-xtu kɯ-mŋɤm ɯ-smɤn ɯ-spa ŋu\cmn 
\stylefv{tshɤmdzu}是一种树,比较矮,刺长而尖,叶子边缘也有刺,根是黄色的,带有苦味,是治拉肚子的药材原料之一。
\end{exemple}\end{entrée}

\begin{entrée}
\vedette{\hypertarget{Ⓔtshɤndʐi}{\papi{ tshɤndʐi}}}\markboth{tshɤndʐi}{}\classe{n}
\begin{définition}\fra peau de chèvre\end{définition}
\begin{définition}\cmn 山羊皮\end{définition}
\begin{relation-sémantique}\confer{
\hyperlink{ⒺtshɤtⒽ2}{\textit{ \papi{tshɤt2}}}
}\end{relation-sémantique}
\begin{relation-sémantique}\confer{
\hyperlink{Ⓔtɯ-ndʐi}{\textit{ \papi{tɯ-ndʐi}}}
}\end{relation-sémantique}
\end{entrée}

\begin{entrée}
\vedette{\hypertarget{Ⓔtshɤnmu}{\papi{ tshɤnmu}}}\markboth{tshɤnmu}{}\classe{n}
\begin{définition}\fra chèvre\end{définition}
\begin{définition}\cmn 母山羊\end{définition}
\begin{relation-sémantique}\confer{
\hyperlink{ⒺtshɤtⒽ2}{\textit{ \papi{tshɤt2}}}
}\end{relation-sémantique}\end{entrée}

\begin{entrée}
\vedette{\hypertarget{Ⓔtshɤɲcɤnɯ}{\papi{ tshɤɲcɤnɯ}}}\markboth{tshɤɲcɤnɯ}{}\classe{n}
\begin{définition}\fra une espèce de champignon\end{définition}
\begin{définition}\cmn 【刷把菌】\end{définition}
\begin{exemple}\jya tshɤɲcɤnɯ nɯ ɕkrɤz ɯ-ŋgɯ tɯrgi ɯ-ŋgɯ tu-ɬoʁ ŋu, ɯ-mdoʁ nɯ kɯ-qarŋe tu, ɕa ɯ-mdoʁ tu, ɯ-tshɯɣa nɯ zɣɤmbu fse, ɯ-ku nɯ tɯ-jaʁndzu kɯ-fse kɯ-xtshɯm-xtshɯm ʑo ŋu, kɤ-ndza sna\cmn 刷把菌长在青冈树林里,有的是黄色的,有的颜色像人的皮肤一样,形状像扫把,尖端像细小的手指,能吃。\end{exemple}\end{entrée}

\begin{entrée}
\vedette{\hypertarget{Ⓔtshɤrkɯ}{\papi{ tshɤrkɯ}}}\markboth{tshɤrkɯ}{}\classe{n}
\begin{définition}\fra moule pour tshatsha\end{définition}
\begin{définition}\cmn 泥像的模型\end{définition}
\begin{relation-sémantique}\confer{
\hyperlink{Ⓔtshatsha}{\textit{ \papi{tshatsha}}}
}\end{relation-sémantique}\end{entrée}

\begin{entrée}
\vedette{\hypertarget{Ⓔtshɤrme}{\papi{ tshɤrme}}}\markboth{tshɤrme}{}\classe{n}
\begin{définition}\fra poil de chèvre\end{définition}
\begin{définition}\cmn 山羊毛\end{définition}
\begin{relation-sémantique}\confer{
\hyperlink{ⒺtshɤtⒽ2}{\textit{ \papi{tshɤt2}}}
}\end{relation-sémantique}
\begin{relation-sémantique}\confer{
\hyperlink{Ⓔtɤ-rme}{\textit{ \papi{tɤ-rme}}}
}\end{relation-sémantique}\end{entrée}

\begin{entrée}
\vedette{\hypertarget{Ⓔtshɤrqhu}{\papi{ tshɤrqhu}}}\markboth{tshɤrqhu}{}
\classe{n}
\begin{définition}\fra natte\end{définition}
\begin{définition}\cmn 席子\end{définition}\end{entrée}

\begin{entrée}
\vedette{\hypertarget{Ⓔtshɤrtɯl}{\papi{ tshɤrtɯl}}}\markboth{tshɤrtɯl}{}\classe{n}
\begin{définition}\fra habit en peau d'agneau\end{définition}
\begin{définition}\cmn 羔羊皮袄\end{définition}\end{entrée}

\begin{entrée}
\vedette{\hypertarget{Ⓔtshɤrɯ}{\papi{ tshɤrɯ}}}\markboth{tshɤrɯ}{}\classe{n}
\begin{définition}\fra veste de peau d'agneau\end{définition}
\begin{définition}\cmn 羔羊皮袄\end{définition}\end{entrée}

\begin{entrée}
\vedette{\hypertarget{Ⓔtshɤʁrɯ}{\papi{ tshɤʁrɯ}}}\markboth{tshɤʁrɯ}{}\classe{n}
\begin{définition}\fra corne de chèvre\end{définition}
\begin{définition}\cmn 山羊角\end{définition}
\begin{relation-sémantique}\confer{
\hyperlink{ⒺtshɤtⒽ2}{\textit{ \papi{tshɤt2}}}
}\end{relation-sémantique}
\begin{relation-sémantique}\confer{
\hyperlink{Ⓔta-ʁrɯ}{\textit{ \papi{ta-ʁrɯ}}}
}\end{relation-sémantique}\end{entrée}

\begin{entrée}
\vedette{\hypertarget{ⒺtshɤtⒽ2}{\papi{ tshɤt}}}\markboth{tshɤt}{}\homonyme{2}\classe{n}
\begin{définition}\fra chèvre\end{définition}
\begin{définition}\cmn 山羊\end{définition}
\begin{relation-sémantique}\confer{
\hyperlink{Ⓔtshɤnmu}{\textit{ \papi{tshɤnmu}}}
}\end{relation-sémantique}\end{entrée}

\begin{entrée}
\vedette{\hypertarget{ⒺtshɤtⒽ1}{\papi{ tshɤt}}}\markboth{tshɤt}{}\homonyme{1}
\classe{vt}
\paradigme{\textit{dir :} \jya tɤ-}
\paradigme{\textit{dir :} \jya kɤ-}\acception{1}
\begin{définition}\fra essayer\end{définition}
\begin{définition}\cmn 试\end{définition}
\begin{exemple}\jya nɤ-tsa ɯ-ɲɯ-βze tɤ-tshɤt\cmn 你试一下适不适合你\end{exemple}
\begin{exemple}\jya @xingqiwu a-kɤ-tɯ-tshɤt\cmn 你星期五试一下吧\end{exemple}\acception{2}
\begin{définition}\fra tester\end{définition}
\begin{définition}\cmn 考验\end{définition}
\begin{relation-sémantique}\confer{
\hyperlink{Ⓔrɤtshɤt}{\textit{ \papi{rɤtshɤt}}}
}\end{relation-sémantique}\end{entrée}

\begin{entrée}
\vedette{\hypertarget{Ⓔtshɤthɤr}{\papi{ tshɤthɤr}}}\markboth{tshɤthɤr}{}\classe{n}
\begin{définition}\fra libérer des animaux vivant\end{définition}
\begin{définition}\cmn 放生
\begin{déclaration} \étymologie{\papi{tsʰe.tʰar}}\end{déclaration}\end{définition}
\begin{exemple}\jya tshɤthɤr lo-lɤt\cmn 他放生了(某种生物)\end{exemple}\end{entrée}

\begin{entrée}
\vedette{\hypertarget{Ⓔtshɤtʂot}{\papi{ tshɤtʂot}}}\markboth{tshɤtʂot}{}\classe{n}
\begin{définition}\fra fièvre\end{définition}
\begin{définition}\cmn 发烧
\begin{déclaration} \étymologie{\papi{tsʰa.drod}}\end{déclaration}\end{définition}
\begin{relation-sémantique}\confer{
\hyperlink{Ⓔnɯtshɤtʂot}{\textit{ \papi{nɯtshɤtʂot}}}
}\end{relation-sémantique}\end{entrée}

\begin{entrée}
\vedette{\hypertarget{Ⓔtshɤwɤre}{\papi{ tshɤwɤre}}}\markboth{tshɤwɤre}{}
\classe{n}
\begin{définition}\fra lézard\end{définition}
\begin{définition}\cmn 壁虎\end{définition}
\begin{exemple}\jya tshɤwɤre nɯ praʁ ɯ-rchɤβ, zndɤrchɤβ ra ku-rɤʑi ɲɯ-ŋu, qajɯ kɯ-wxti tsa ci ɲɯ-ŋu, ɯ-phoŋbu alɯlju, ɯ-jme nɯ ɲɯ-ɤɕpɯɕpa tɕe, ɯ-phaʁ βzɯr nɯ ɯ-thoʁ pjɯ-tɯɣ ŋu, ɯ-mɤlɤjaʁ ɣɤʑu, ɯ-mdoʁ nɯ qapri mdoʁ cho naχtɕɯɣ, ɯʑo stu ʑo kɤ-cha nɯ ju-mtsaʁ tɕe qapri ɲɯ-prɤt, tɕe li tɯ-jwaʁ ci z-ɲɯ-ɕar tɕe, qapri na-prɤt ɯ-stu nɯ tɕu ku-tshoʁ tɕe qapri li ku-sɤndɯ-ndɤm ɲɯ-cha, tɕe qapri tu-nɯsmɤn ɲɯ-cha.\cmn 壁虎生活在岩石缝里,墙壁缝里,是一种大虫。身子是圆柱形的,尾巴扁,尾巴的棱边着地。有四只脚,颜色和蛇一样。(传说)壁虎从蛇背跳过时会使蛇断裂,它最大的本事就是去找一片草叶,夹在蛇断裂的部位,可以把蛇的两段接起来,把蛇治好。\end{exemple}\end{entrée}

\begin{entrée}
\vedette{\hypertarget{Ⓔtshɤχɕaŋ}{\papi{ tshɤχɕaŋ}}}\markboth{tshɤχɕaŋ}{}
\classe{n}
\begin{définition}\fra branche d'arbre que l'on met sur le toit après les cérémonie religieuse\end{définition}
\begin{définition}\cmn 念完经后插在屋顶上的树枝\end{définition}
\begin{exemple}\jya tshɤχɕaŋ nɯ pakuku ʑo nɯ-kɯ-ɣɤrpi tɕe pjɯ-tu ra, sɤjku ɣɯ ɯ-rtaʁ ɯ-rtsimu kɯ-βdi, ɯ-rtaʁ ra ɯ-tɯ-nɯgrɤl kɯ-ɤmɲɤm tsa pjɯ-ŋu ra, ɯ-rtaʁ ɣɯ ɯ-mujmaj raŋri nɯ tɕu qarma muj kɯ-wɣrum nɯ maʁ nɤ smɤɣ kɯ-wɣrum tɯ-sna ntsɯ kú-wɣ-sthoʁ. tɕe tɤ-rpi ɲɯ-jɤɣ tɕe, kha ɣɯ lɤftsɤz ɯ-taʁ pjɯ́-wɣ-sɤtsa ra, kɯ-ɤrqhi ju-kɯ-ru tɕe, mɯntoʁ kɯ-wɣrum nɯ-kɯ-lɤt fse.\cmn 
每年请和尚来家里念经时,必须要有\stylefv{tshɤχɕaŋ},必须是长得美观的白桦树枝,枝桠排列得比较均匀的,在每一根枝桠顶端上都要扎上白马鸡的羽毛或者白羊毛。念完经后就要插在房背上的白石头堆中,从远方看,像盛开的白花。
\end{exemple}\end{entrée}

\begin{entrée}
\vedette{\hypertarget{Ⓔtshɤz}{\papi{ tshɤz}}}\markboth{tshɤz}{}\classe{vs}
\begin{définition}\fra frais et tendre\end{définition}
\begin{définition}\cmn 新鲜;清脆(吃起来很脆)
\end{définition}
\begin{exemple}\jya ki @yangyu ki ɲɯ-tshɤz tɕe ɲɯ-mɯm\cmn 这个洋芋又新鲜又好吃\end{exemple}\end{entrée}

\begin{entrée}
\vedette{\hypertarget{Ⓔtshuβdɯn}{\papi{ tshuβdɯn}}}\markboth{tshuβdɯn}{}\classe{n}
\begin{définition}\ 
\begin{déclaration}\grammar{n.lieu}\end{déclaration}\end{définition}
\begin{définition}\fra Tshobdun\end{définition}
\begin{définition}\cmn 草登乡\end{définition}\end{entrée}

\begin{entrée}
\vedette{\hypertarget{ⒺtshiⒽ4}{\papi{ tshi}}}\markboth{tshi}{}\homonyme{4}
\classe{pro}
\begin{définition}\fra quoi\end{définition}
\begin{définition}\cmn 什么\end{définition}
\begin{exemple}\jya tshi tú-wɣ-pa ɲɯ-ra?\cmn 应该怎么办?\end{exemple}
\begin{relation-sémantique}\synonyme{
\hyperlink{ⒺtɕhiⒽ1}{\textit{ \papi{tɕhi1}}}
}\end{relation-sémantique}
\begin{relation-sémantique}\confer{
\hyperlink{Ⓔtshitsuku}{\textit{ \papi{tshitsuku}}}
}\end{relation-sémantique}\end{entrée}

\begin{entrée}
\vedette{\hypertarget{ⒺtshiⒽ1}{\papi{ tshi}}}\markboth{tshi}{}\homonyme{1}
\classe{vt}
\paradigme{\textit{dir :} \jya kɤ-}
\paradigme{\textit{dir :} \jya pɯ-}
\begin{définition}\fra boire\end{définition}
\begin{définition}\cmn 喝\end{définition}
\begin{exemple}\jya tɯ-ci ka-tshi\cmn 他喝了水\end{exemple}
\begin{exemple}\jya cha pɯ-asɯ-tshi-j\cmn 我们在喝酒\end{exemple}
\begin{exemple}\jya chɤmda pɯ-tshi-t-a\cmn 我喝了坛坛酒\end{exemple}
\begin{exemple}\jya tʂha ku-tshi-tɕi pɯ-ŋu\cmn 我们俩在喝茶(原来)\end{exemple}
\begin{relation-sémantique}\confer{
\hyperlink{Ⓔjtshi}{\textit{ \papi{jtshi}}}
}\end{relation-sémantique}\begin{sous-entrée}
\vedette{\hypertarget{}{\papi{ nɯɣɯtshi}}}\markboth{nɯɣɯtshi}{}\classe{vs}
\begin{définition}\fra agréable à boire\end{définition}
\begin{définition}\cmn 喝着爽口\end{définition}
\begin{exemple}\jya ki tʂha ki wuma ɲɯ-nɯɣɯtshi\cmn 这个茶喝着很爽口\end{exemple}
\end{sous-entrée}\end{entrée}

\begin{entrée}
\vedette{\hypertarget{ⒺtshiⒽ2}{\papi{ tshi}}}\markboth{tshi}{}\homonyme{2}
\classe{vt}
\paradigme{\textit{dir :} \jya nɯ-}
\begin{définition}\fra étrangler\end{définition}
\begin{définition}\cmn 勒住\end{définition}
\begin{exemple}\jya ɯ-mke ɲɤ-tshi\cmn 他把他的脖子勒住了(横着)\end{exemple}
\begin{exemple}\jya ɯ-mke to-tshi\cmn 他把他吊死了\end{exemple}
\begin{relation-sémantique}\confer{
\hyperlink{Ⓔʑɣɤtshi}{\textit{ \papi{ʑɣɤtshi}}}
}\end{relation-sémantique}\end{entrée}

\begin{entrée}
\vedette{\hypertarget{ⒺtshiⒽ3}{\papi{ tshi}}}\markboth{tshi}{}\homonyme{3}
\classe{vt}
\paradigme{\textit{dir :} \jya \_}
\begin{définition}\fra bloquer\end{définition}
\begin{définition}\cmn 挡住,拦住\end{définition}
\begin{exemple}\jya ɕkom nɯ jɤ-tshi\cmn 你把麂子拦住吧!\end{exemple}
\begin{exemple}\jya fsapaʁ nɯ-tshi\cmn 你把牲畜拦住吧!\end{exemple}
\begin{exemple}\jya ʑmbri to-wxti tɕe qale ju-tshi, mɯ-ju-sɯɣe ɲɯ-cha\cmn 杨树长出高了以后可以挡风\end{exemple}
\begin{relation-sémantique}\confer{
\hyperlink{Ⓔnɤtʂɤtshi}{\textit{ \papi{nɤtʂɤtshi}}}
}\end{relation-sémantique}\end{entrée}

\begin{entrée}
\vedette{\hypertarget{Ⓔtshitsuku}{\papi{ tshitsuku}}}\markboth{tshitsuku}{}\classe{pro}
\begin{définition}\fra quoi que ce soit\end{définition}
\begin{définition}\cmn 无论什么,一切\end{définition}
\end{entrée}

\begin{entrée}
\vedette{\hypertarget{Ⓔtshjencɯ}{\papi{ tshjencɯ}}}\markboth{tshjencɯ}{}\classe{n}
\begin{définition}\fra couteau\end{définition}
\begin{définition}\cmn 猎刀\end{définition}
\end{entrée}

\begin{entrée}
\vedette{\hypertarget{Ⓔtshoŋ}{\papi{ tshoŋ}}}\markboth{tshoŋ}{}\classe{n}
\begin{définition}\fra commerce\end{définition}
\begin{définition}\cmn 生意
\begin{déclaration} \étymologie{\papi{tsʰoŋ}}\end{déclaration}\end{définition}
\end{entrée}

\begin{entrée}
\vedette{\hypertarget{Ⓔtshoŋpawa}{\papi{ tshoŋpawa}}}\markboth{tshoŋpawa}{}
\classe{n}
\begin{définition}\fra marchand\end{définition}
\begin{définition}\cmn 老板;商人
\begin{déclaration} \étymologie{\papi{tsʰoŋ.pa.ba}}\end{déclaration}\end{définition}\end{entrée}

\begin{entrée}
\vedette{\hypertarget{Ⓔtshoŋwa}{\papi{ tshoŋwa}}}\markboth{tshoŋwa}{}\classe{n}
\begin{définition}\fra marchand\end{définition}
\begin{définition}\cmn 商人
\begin{déclaration} \étymologie{\papi{tsʰoŋ.ba}}\end{déclaration}\end{définition}\end{entrée}

\begin{entrée}
\vedette{\hypertarget{Ⓔtshoʁ}{\papi{ tshoʁ}}}\markboth{tshoʁ}{}
\classe{vt}\acception{1}
\paradigme{\textit{dir :} \jya kɤ-}
\begin{définition}\fra attacher sur, mettre sur\end{définition}
\begin{définition}\cmn 带上,放在……上\end{définition}
\begin{exemple}\jya khɯna ɯ-mke tɤmɯmɯm ko-tshoʁ\cmn 他在狗的脖子上拴了铃铛\end{exemple}
\begin{exemple}\jya @dianlu kɤ-nɯ-tshoʁ-i\cmn 我们装了电炉\end{exemple}\acception{2}
\begin{définition}\fra faire des fruits\end{définition}
\begin{définition}\cmn 结(果子)\end{définition}
\begin{exemple}\jya paχɕi kɯ ɯ-mat ko-tshoʁ\cmn 苹果树结了果\end{exemple}
\begin{exemple}\jya ʑɴɢɯloʁ kɯ ɯ-mat ko-tshoʁ\cmn 核桃结了果\end{exemple}\begin{sous-entrée}
\vedette{\hypertarget{}{\papi{ atshoʁ}}}\markboth{atshoʁ}{}\classe{vi}
\begin{définition}\fra être attaché\end{définition}
\begin{définition}\cmn 附着\end{définition}
\begin{relation-sémantique}\confer{
\hyperlink{Ⓔndzoʁ}{\textit{ \papi{ndzoʁ}}}
}\end{relation-sémantique}
\end{sous-entrée}\begin{sous-entrée}
\vedette{\hypertarget{}{\papi{ khɯna,tshoʁ}}}\markboth{khɯna,tshoʁ}{}
\paradigme{\textit{dir :} \jya thɯ-}
\begin{définition}\fra lâcher les chiens\end{définition}
\begin{définition}\cmn 放狗\end{définition}
\begin{exemple}\jya khɯna thɯ-tshoʁ-i\cmn 我们放了狗\end{exemple}
\begin{relation-sémantique}\ComponentA{\classe{n}
\hyperlink{Ⓔkhɯna}{\textit{ \papi{khɯna}}}
}\end{relation-sémantique}
\begin{relation-sémantique}\ComponentB{\classe{vt}
\hyperlink{Ⓔtshoʁ}{\textit{ \papi{tshoʁ}}}
}\end{relation-sémantique}
\begin{relation-sémantique}\synonyme{
\hyperlink{Ⓔɣɯkhɯtshoʁ}{\textit{ \papi{ɣɯkhɯtshoʁ}}}
}\end{relation-sémantique}
\end{sous-entrée}\begin{sous-entrée}
\vedette{\hypertarget{}{\papi{ tɯ-χpɯm,sɯtshoʁ}}}\markboth{tɯ-χpɯm,sɯtshoʁ}{}
\paradigme{\textit{dir :} \jya pɯ-}
\begin{définition}\fra faire s'agenouiller\end{définition}
\begin{définition}\cmn 使……跪下\end{définition}
\end{sous-entrée}\begin{sous-entrée}
\vedette{\hypertarget{}{\papi{ tɯ-χpɯm,tshoʁ}}}\markboth{tɯ-χpɯm,tshoʁ}{}
\paradigme{\textit{dir :} \jya pɯ-}
\begin{définition}\fra s'agenouiller\end{définition}
\begin{définition}\cmn 跪下\end{définition}
\begin{exemple}\jya a-χpɯm pɯ-tshoʁ-a\cmn 我跪下了\end{exemple}
\end{sous-entrée}\end{entrée}

\begin{entrée}
\vedette{\hypertarget{Ⓔtshoʁɕaŋ}{\papi{ tshoʁɕaŋ}}}\markboth{tshoʁɕaŋ}{}\classe{n}
\begin{définition}\fra décoration faite avec des branches et des plumes de crossoptilon\end{définition}
\begin{définition}\cmn 用树枝和白马鸡的羽毛做成的装饰\end{définition}\end{entrée}

\begin{entrée}
\vedette{\hypertarget{Ⓔtshoz}{\papi{ tshoz}}}\markboth{tshoz}{}
\classe{vi}
\paradigme{\textit{dir :} \jya tɤ-}
\begin{définition}\fra être au complet\end{définition}
\begin{définition}\cmn 齐全(零散的东西,几个人)
\begin{déclaration} \étymologie{\papi{tsʰaŋs}}\end{déclaration}\end{définition}
\begin{exemple}\jya a-rɣe pɯ-nɯ-prat-a tɕe, pɯ-ʁndɤr tɕeri tɤ-wum-a tɕe nɯ ʁo tɤ-tshoz\cmn 我把珠子弄断,撒在地上了捡了以后倒是齐全的\end{exemple}
\begin{exemple}\jya kɯki jɯɣi ki ɲɯ-tshoz\cmn 这本书是完整的\end{exemple}
\begin{relation-sémantique}\synonyme{
\hyperlink{Ⓔndzɯr}{\textit{ \papi{ndzɯr}}}
}\end{relation-sémantique}\begin{sous-entrée}
\vedette{\hypertarget{}{\papi{ ɣɤtshoz}}}\markboth{ɣɤtshoz}{}\classe{vt}
\begin{définition}\fra rendre complet\end{définition}
\begin{définition}\cmn 使……齐全\end{définition}
\begin{exemple}\jya nɤ-kɯ-ra ra kɤ-ɕar kɤ-ɣɤtshoz pɯ-cha-a\cmn 我找到了你所有需要的东西\end{exemple}
\begin{relation-sémantique}\confer{
\hyperlink{Ⓔsɯxtshoz}{\textit{ \papi{sɯxtshoz}}}
}\end{relation-sémantique}
\end{sous-entrée}\end{entrée}

\begin{entrée}
\vedette{\hypertarget{Ⓔtshupa}{\papi{ tshupa}}}\markboth{tshupa}{}\classe{n}
\begin{définition}\fra village\end{définition}
\begin{définition}\cmn 村子
\begin{déclaration} \étymologie{\papi{tsʰo.pa}}\end{déclaration}\end{définition}
\end{entrée}

\begin{entrée}
\vedette{\hypertarget{Ⓔtshɯ}{\papi{ tshɯ}}}\markboth{tshɯ}{}
\classe{vt}
\paradigme{\textit{dir :} \jya \_}
\begin{définition}\fra s'accommoder de ce qu'il y a\end{définition}
\begin{définition}\cmn 去一个地方使用本地的东西(不带自己的东西)、将就\end{définition}
\begin{exemple}\jya kɤ-tshɯ mɯ́j-khɯ ma kutɕu maŋe tɕe, kɤ-nɯɣɯt pɯ-ra\cmn 没有办法使用本地的东西,只好带过来了\end{exemple}
\begin{exemple}\jya kutɕu ɣɯ kɤndza nɯra kɤ-tshɯ-t-a mɯ-kɤ-nɯɣɯt-a\cmn 我没有把自己地方的食物带来,我吃了这里的食物\end{exemple}\end{entrée}

\begin{entrée}
\vedette{\hypertarget{Ⓔtshɯɣru}{\papi{ tshɯɣru}}}\markboth{tshɯɣru}{}
\classe{n}
\begin{définition}\fra soude; alcali\end{définition}
\begin{définition}\cmn 碱\end{définition}\end{entrée}

\begin{entrée}
\vedette{\hypertarget{Ⓔtshɯmɕtʂat}{\papi{ tshɯmɕtʂat}}}\markboth{tshɯmɕtʂat}{}
\classe{n}
\begin{définition}\fra capacité à économiser\end{définition}
\begin{définition}\cmn 节约开支\end{définition}
\begin{exemple}\jya kɤndzɤtshi tshɯmɕtʂat ɯ-kɯ-βzu nɯ kha tɤ-mu nɯ ŋu\cmn 一家人的开支是家庭主妇计划的\end{exemple}\end{entrée}

\begin{entrée}
\vedette{\hypertarget{Ⓔtshɯntshɯn}{\papi{ tshɯntshɯn}}}\markboth{tshɯntshɯn}{}
\classe{idph.2}
\begin{définition}\fra en bon état\end{définition}
\begin{définition}\cmn 形容塌实、齐全的状态\end{définition}
\begin{exemple}\jya laχtɕha nɯtɕu nɯ-ta-t-a tɕe, tshɯntshɯn ɲɯ-ɤ<nɯ>ta\cmn 我把东西放在那里,没人会碰\end{exemple}
\begin{exemple}\jya nɤ-ŋga tɤ-tɯ-ɕɯɴqoʁ ɯ-sta tshɯntshɯn ɲɯ-ɤ<nɯ>ta\cmn 你把衣服挂在那里,没人会碰\end{exemple}
\begin{exemple}\jya kha ra tɤ-rɤwum-a tshɯntshɯn ʑo\cmn 我把家的东西收拾得很好\end{exemple}\end{entrée}

\begin{entrée}
\vedette{\hypertarget{Ⓔtshɯptshɯp}{\papi{ tshɯptshɯp}}}\markboth{tshɯptshɯp}{}
\classe{idph.2}
\begin{définition}\fra sensation de condensation (dans la brume)\end{définition}
\begin{définition}\cmn 形容空气中弥漫着潮气(水蒸气凝结)的感觉
\begin{déclaration}\use{\stylefv{tshɯptshɯp}和\stylefv{tɕhɯβtɕhɯβ}的区别在于,前者只表示有水分的感觉(看不到水),而后者看得到水珠}\end{déclaration}\end{définition}\begin{sous-entrée}
\vedette{\hypertarget{}{\papi{ ɣɤtshɯptshɯp}}}\markboth{ɣɤtshɯptshɯp}{}\classe{vs}
\begin{exemple}\jya ɲɯ-ɣɤtshɯptshɯp\cmn 空气中弥漫着潮气\end{exemple}
\begin{relation-sémantique}\synonyme{
\hyperlink{Ⓔtɕhɯβtɕhɯβ}{\textit{ \papi{tɕhɯβtɕhɯβ}}}
}\end{relation-sémantique}
\end{sous-entrée}\end{entrée}

\begin{entrée}
\vedette{\hypertarget{Ⓔtshɯrɟɯn}{\papi{ tshɯrɟɯn}}}\markboth{tshɯrɟɯn}{}\classe{adv}\begin{définition}\fra souvent\end{définition}
\begin{définition}\cmn 经常
\begin{déclaration} \étymologie{\papi{tsʰe.rgʲun}}\end{déclaration}\end{définition}
\begin{exemple}\jya tshɯrɟɯn nɯɣi ɕti\cmn 他经常回来\end{exemple}\end{entrée}

\begin{entrée}
\vedette{\hypertarget{Ⓔtshɯtho}{\papi{ tshɯtho}}}\markboth{tshɯtho}{}
\classe{n}
\begin{définition}\fra cabri\end{définition}
\begin{définition}\cmn 山羊羔\end{définition}\end{entrée}

\begin{entrée}
\vedette{\hypertarget{Ⓔtshwi}{\papi{ tshwi}}}\markboth{tshwi}{}
\classe{n}
\begin{définition}\fra teinture\end{définition}
\begin{définition}\cmn 染料
\begin{déclaration} \étymologie{\papi{tsʰos}}\end{déclaration}\end{définition}\end{entrée}

\begin{entrée}
\vedette{\hypertarget{Ⓔtshuxtoʁ}{\papi{ tshuxtoʁ}}}\markboth{tshuxtoʁ}{}\classe{n}
\begin{définition}\fra confiance\end{définition}
\begin{définition}\cmn 信用\end{définition}
\begin{exemple}\jya kɯki tɯrme ki ɯ-tshuxtoʁ kɯ-tu ci ŋu\cmn 他是个讲信用的人\end{exemple}
\begin{exemple}\jya nɤʑo nɤ-tshuxtoʁ ɣɤʑu\cmn 你讲信用\end{exemple}
\begin{exemple}\jya aʑo ki tshuxtoʁ ɯ-ku rɤʑi-a ɕti\cmn 我会守信用的\end{exemple}
\begin{exemple}\jya nɤʑo tshuxtoʁ a-kɤ-tɯ-rɤʑi ra nɤ!\cmn 你要讲信用\end{exemple}\end{entrée}

\begin{entrée}
\vedette{\hypertarget{Ⓔtsjaŋtsjaŋ}{\papi{ tsjaŋtsjaŋ}}}\markboth{tsjaŋtsjaŋ}{}\classe{idph.2}
\begin{définition}\fra haut\end{définition}
\begin{définition}\cmn 身子高(比其他人高)\end{définition}
\begin{exemple}\jya ɲɯ-mbro ʑo tsjaŋtsjaŋ\cmn 他很高\end{exemple}\begin{sous-entrée}
\vedette{\hypertarget{}{\papi{ tsjaŋnɤtsjaŋ}}}\markboth{tsjaŋnɤtsjaŋ}{}\classe{idph.3}
\begin{relation-sémantique}\confer{
\hyperlink{Ⓔzjaŋzjaŋ}{\textit{ \papi{zjaŋzjaŋ}}}
}\end{relation-sémantique}
\begin{relation-sémantique}\confer{
\hyperlink{Ⓔzjɤɣzjɤɣ}{\textit{ \papi{zjɤɣzjɤɣ}}}
}\end{relation-sémantique}
\end{sous-entrée}\end{entrée}

\begin{entrée}
\vedette{\hypertarget{Ⓔtsuku}{\papi{ tsuku}}}\markboth{tsuku}{}\classe{n}
\begin{définition}\fra certains\end{définition}
\begin{définition}\cmn 一些;某些\end{définition}
\end{entrée}

\begin{entrée}
\vedette{\hypertarget{Ⓔtslɯɣtslɯɣ}{\papi{ tslɯɣtslɯɣ}}}\markboth{tslɯɣtslɯɣ}{}\classe{idph.2}\acception{1}
\begin{définition}\fra envelopper complètement\end{définition}
\begin{définition}\cmn 形容包成一团(成了圆形)、包得又紧又大状\end{définition}
\begin{exemple}\jya ɯ-ku pjɤ-ɴɢraʁ tɕe, tslɯɣtslɯɣ to-mphɯr\cmn 他的头破了,被包成一团了\end{exemple}
\begin{exemple}\jya ɯ-jaʁ to-mphɯr tslɯɣtslɯɣ ʑo\cmn 他把手包成一团了\end{exemple}\acception{2}
\begin{définition}\fra rond et dur\end{définition}
\begin{définition}\cmn 形容又圆又硬的样子\end{définition}
\begin{exemple}\jya a-rqo ci thɯci chɤ-ɕe tɕe tslɯɣtslɯɣ ʑo ɲɯ-ɤrku tɕe kɤ-tɕɤt mɯ́j-khɯ\cmn 我喉咙里卡了个东西,取不出来\end{exemple}\begin{sous-entrée}
\vedette{\hypertarget{}{\papi{ ɣɤtslɯɣtslɯɣ}}}\markboth{ɣɤtslɯɣtslɯɣ}{}\classe{vi}
\begin{exemple}\jya tɯ-ci ɲɯ-ɤla ɲɯ-ɣɤtslɯɣtslɯɣ\cmn 水沸腾了\end{exemple}
\end{sous-entrée}\begin{sous-entrée}
\vedette{\hypertarget{}{\papi{ tslɯɣnɤtslɯɣ}}}\markboth{tslɯɣnɤtslɯɣ}{}\classe{idph.3}
\begin{définition}\fra bouillir en faisant de grosses bulles\end{définition}
\begin{définition}\cmn 水沸腾,翻滚的样子\end{définition}
\end{sous-entrée}\end{entrée}

\begin{entrée}
\vedette{\hypertarget{Ⓔtso}{\papi{ tso}}}\markboth{tso}{}\classe{vi-t}
\paradigme{\textit{dir :} \jya kɤ-}
\begin{définition}\fra comprendre, se rendre compte, savoir\end{définition}
\begin{définition}\cmn 懂;发觉;知道\end{définition}
\begin{exemple}\jya hanɯni ci, mɤʑɯ kɯ-taʁ tsa a-kɤ-tso-a ɲɯ-ra\cmn 我还需要弄清楚一点(你再讲一次)\end{exemple}
\begin{exemple}\jya lonba kɤ-tso-a\cmn 我完全懂了\end{exemple}
\begin{exemple}\jya tu-kɯ-ti a-pɯ-ŋu tɕe, tɯrme ra ʑatsa mɤ-tso-nɯ\cmn 如果说这句话,人家听不懂\end{exemple}
\begin{exemple}\jya nɤj tɤ-tɯ-tɯt nɯ aj mɯ́j-tso-a\cmn 我不明白你刚说的话\end{exemple}\begin{sous-entrée}
\vedette{\hypertarget{}{\papi{ sɤtso}}}\markboth{sɤtso}{}\classe{vs}
\begin{définition}\ 
\begin{déclaration}\grammar{deexp}\end{déclaration}\end{définition}
\begin{définition}\fra être compréhensible\end{définition}
\begin{définition}\cmn 听起来清楚\end{définition}
\begin{exemple}\jya tɕhi tu-ti ŋu mɤ-sɤtso wo?\cmn 听不懂他在讲什么吗?\end{exemple}
\begin{exemple}\jya ɲɯ-sɤtso\cmn 很清楚\end{exemple}
\end{sous-entrée}\begin{sous-entrée}
\vedette{\hypertarget{}{\papi{ sɯxtso}}}\markboth{sɯxtso}{}\classe{vt}
\paradigme{\textit{dir :} \jya kɤ-}
\begin{définition}\fra faire comprendre\end{définition}
\begin{définition}\cmn 令人明白\end{définition}
\begin{exemple}\jya tɯ-rju kɤ-sɯxtso-t-a\cmn 我令他明白了这句话的意思\end{exemple}
\end{sous-entrée}\begin{sous-entrée}
\vedette{\hypertarget{}{\papi{ tso me}}}\markboth{tso me}{}\acception{1}
\begin{définition}\fra perdre conscience\end{définition}
\begin{définition}\cmn 失去知觉\end{définition}
\begin{exemple}\jya tso ɲɤ-me-a tɕe pjɤ-nɯʑɯβ-a\cmn 我不只知不觉地睡着了\end{exemple}\acception{2}
\begin{définition}\fra ne pas être en âge de comprendre\end{définition}
\begin{définition}\cmn 不懂事,幼稚\end{définition}
\begin{exemple}\jya ki tɤ-pɤtso ki pɤjkhu tso maŋe\cmn 这个小孩子还不懂事\end{exemple}
\end{sous-entrée}\begin{sous-entrée}
\vedette{\hypertarget{}{\papi{ ʑɣɤsɯxtso}}}\markboth{ʑɣɤsɯxtso}{}\classe{vi}
\begin{définition}\ 
\begin{déclaration}\grammar{refl}\end{déclaration}
\begin{déclaration}\grammar{caus}\end{déclaration}\end{définition}
\begin{définition}\fra faire en sorte de comprendre\end{définition}
\begin{définition}\cmn 让自己明白\end{définition}
\begin{exemple}\jya tɯ-rju kɤ-ʑɣɤsɯxtso ra\cmn 要弄明白这句话的意思\end{exemple}
\begin{exemple}\jya tɤ-kɤ-tɯt nɯra kɤ-ʑɣɤsɯxtso ma tha nɤ-ɕɯ-kɤ-rɤfɕɤt me\cmn 你要弄清他所讲的话,不然的话你转述不出来\end{exemple}
\end{sous-entrée}\end{entrée}

\begin{entrée}
\vedette{\hypertarget{Ⓔtsrɤt}{\papi{ tsrɤt}}}\markboth{tsrɤt}{}
\classe{vt}
\paradigme{\textit{dir :} \jya lɤ-}
\paradigme{\textit{dir :} \jya \_}
\begin{définition}\fra allonger\end{définition}
\begin{définition}\cmn 伸长\end{définition}
\begin{exemple}\jya ɯ-mke lo-tsrɤt\cmn 他伸了脖子\end{exemple}\end{entrée}

\begin{entrée}
\vedette{\hypertarget{Ⓔtsri}{\papi{ tsri}}}\markboth{tsri}{}\classe{vs}
\paradigme{\textit{dir :} \jya nɯ-}
\begin{définition}\fra salé\end{définition}
\begin{définition}\cmn 咸\end{définition}
\begin{exemple}\jya tsha nɯ kɯ-tsri ŋu\cmn 盐是咸的\end{exemple}
\begin{exemple}\jya ɯ-tɯ-tsri ko-tɕhom\cmn 太咸了\end{exemple}
\begin{exemple}\jya ɯ-tɯ-tsri mɯ́j-rtaʁ\cmn 不够咸\end{exemple}\begin{sous-entrée}
\vedette{\hypertarget{}{\papi{ nɤtsri}}}\markboth{nɤtsri}{}\classe{vt}
\paradigme{\textit{dir :} \jya pɯ-}
\begin{définition}\ 
\begin{déclaration}\grammar{trop}\end{déclaration}\end{définition}
\begin{définition}\fra trouver salé\end{définition}
\begin{définition}\cmn 觉得咸\end{définition}
\begin{exemple}\jya pɯ-nɤtsri-t-a\cmn 我觉得很咸\end{exemple}
\begin{relation-sémantique}\confer{
\hyperlink{Ⓔɣɤtsri}{\textit{ \papi{ɣɤtsri}}}
}\end{relation-sémantique}
\end{sous-entrée}\end{entrée}

\begin{entrée}
\vedette{\hypertarget{Ⓔtsɯm}{\papi{ tsɯm}}}\markboth{tsɯm}{}\classe{vt}
\paradigme{\textit{dir :} \jya \_}
\begin{définition}\fra emporter\end{définition}
\begin{définition}\cmn 拿走;带去\end{définition}
\begin{exemple}\jya tɯrme ɣɯ ɯ-laχtɕha nɯ tɯʑo kɯ jú-wɣ-tsɯm ra ma kɤ-sɯtsɯm mɤ-pe ma a-pɯ-tshoz ra\cmn 别人的东西要亲自送过去,不要请人,因为要齐全\end{exemple}\begin{sous-entrée}
\vedette{\hypertarget{}{\papi{ nɤtsɯtsɯm}}}\markboth{nɤtsɯtsɯm}{} (\variante{nɤtsɯmtsɯm}) \classe{vt}
\paradigme{\textit{dir :} \jya \_}
\begin{définition}\fra emporter partout\end{définition}
\begin{définition}\cmn 带来带去\end{définition}
\end{sous-entrée}\begin{sous-entrée}
\vedette{\hypertarget{}{\papi{ nɯtsɯm}}}\markboth{nɯtsɯm}{}\classe{vt}
\paradigme{\textit{dir :} \jya \_}
\begin{définition}\ 
\begin{déclaration}\grammar{vert}\end{déclaration}\end{définition}
\begin{définition}\fra emporter (chez soi)\end{définition}
\begin{définition}\cmn 带回家\end{définition}
\end{sous-entrée}\begin{sous-entrée}
\vedette{\hypertarget{}{\papi{ sɯtsɯm}}}\markboth{sɯtsɯm}{}\classe{vt}
\paradigme{\textit{dir :} \jya \_}
\begin{définition}\fra faire emporter\end{définition}
\begin{définition}\cmn 令人带走\end{définition}
\end{sous-entrée}\end{entrée}

\begin{entrée}
\vedette{\hypertarget{Ⓔtsɯmnɤtsɯm}{\papi{ tsɯmnɤtsɯm}}}\markboth{tsɯmnɤtsɯm}{}\classe{idph.3}
\begin{définition}\fra scintillant\end{définition}
\begin{définition}\cmn 形容星星点点的闪光\end{définition}
\begin{exemple}\jya tɕheme ɯ-rte ɯ-taʁ kɯ-nɤmbju tsɯmnɤtsɯm ʑo ɲɯ-pa\cmn 妇女的帽子一闪一闪地发光\end{exemple}\begin{sous-entrée}
\vedette{\hypertarget{}{\papi{ tsɯmɯtsami}}}\markboth{tsɯmɯtsami}{}\classe{idph.8}
\begin{exemple}\jya ɯ-rŋa ra qame tsɯmɯtsami ɣɤʑu\cmn 他满脸都是黑痣\end{exemple}
\end{sous-entrée}\end{entrée}

\begin{entrée}
\vedette{\hypertarget{Ⓔtsɯntu}{\papi{ tsɯntu}}}\markboth{tsɯntu}{}\classe{n}
\begin{définition}\fra ciseaux\end{définition}
\begin{définition}\cmn 剪刀\end{définition}\end{entrée}

\begin{entrée}
\vedette{\hypertarget{Ⓔtsɯʁot}{\papi{ tsɯʁot}}}\markboth{tsɯʁot}{}
\classe{n}
\begin{définition}\fra faisan (phasianus colchicus)\end{définition}
\begin{définition}\cmn 雉鸡【野鸡】\end{définition}
\begin{exemple}\jya tsɯʁot nɯ pɣa ci ŋu, khro mɤ-wxti, kumpɣa jamar tu, phu nɯ wuma ʑo mpɕɤr, ɯ-jme rɲɟi, ɯ-ku kɯ-ɲaʁ ŋu, ɯ-taʁ ldʑaŋkɯ kɯ-nɤmbju tu, ɯ-phoŋbu kɯ-ɣɯrni, kɯ-qarŋe, ldʑaŋkɯ kɯ-ɲaʁ nɯ ra kɯ-ɤtʂoʁloʁ tu, tɕe wuma ʑo mpɕɤr, ɯ-mi aqarŋɯrŋe tsa ŋu, tɕe mpɕɤr, tsɯʁot mu nɯ kɯ-pɣi tsa ŋu tɕe nɯ mɤ-mpɕɤr. kɯ-mbro ʑo ɲɯ-nɯqambɯmbjom mɤ-cha. sɯku jamar ma tu-ɕe mɤ-cha. sɯmat, tɤ-rɤku, qajɯ nɯ ra tu-ndze ŋu. ɯʑo ɯ-kɯ-ndza dɤn. qandʑɣi, qaliaʁ, xɕiri, khɯna, lɯlu nɯnɯ ra kɯ tú-wɣ-ndza ɕti, ɯʑo tɯ-ji ɯ-rkɯ kɤ-nɤru wuma ʑo χɕu. tsɯʁot nɯ tɯ-ji ɯ-ŋgɯ sɯŋgɯ nɯ ra chɯ-rɤŋgɯm ŋu, ɯ-ŋgɯm nɯ kumpɣa ɯ-ŋgɯm jamar tu. kɤ-ndza sna.\cmn 野鸡是一种鸟,有鸡那么大,公的很漂亮,尾巴长,头是黑的,上面有绿色的光泽,身子有红色、黄色、绿色、黑色交错着,非常漂亮,脚是浅黄色。母野鸡全身是灰色的,不漂亮。飞不高,只能飞到树上去。吃野果、粮食、虫子。很多动物吃野鸡,隼、雕、黄鼠狼、狗、猫都吃。野鸡吃粮食很厉害,在田地里和森林里下蛋。蛋和鸡蛋一样大,可以吃。\end{exemple}\end{entrée}

\begin{entrée}
\vedette{\hypertarget{Ⓔtʂu}{\papi{ tʂu}}}\markboth{tʂu}{}
\classe{n}
\begin{définition}\fra chemin\end{définition}
\begin{définition}\cmn 路\end{définition}
\begin{exemple}\jya ɯ-tɯ-tsɣe wuma ɯ-tʂu ɲɯ-ɕe\cmn 他生意做得很顺利\end{exemple}
\begin{exemple}\jya tʂu lo-βzu\cmn 开路了\end{exemple}
\begin{exemple}\jya nɤ-tʂu!\cmn 请进!(主人对客人的客套话)\end{exemple}
\begin{exemple}\jya ɯ-tʂu tɤ-cɯ-t-a\cmn 我给他让了路\end{exemple}
\begin{exemple}\jya a-tʂu nɯ-βze (=tɤ-ci)\cmn 给我让路\end{exemple}
\begin{relation-sémantique}\confer{
\hyperlink{Ⓔtʂɯtʂu}{\textit{ \papi{tʂɯtʂu}}}
}\end{relation-sémantique}
\begin{relation-sémantique}\confer{
\hyperlink{Ⓔftɕɤru}{\textit{ \papi{ftɕɤru}}}
}\end{relation-sémantique}
\begin{relation-sémantique}\confer{
\hyperlink{ⒺnɯtʂuⒽ1}{\textit{ \papi{nɯtʂu1}}}
}\end{relation-sémantique}
\begin{relation-sémantique}\confer{
\hyperlink{ⒺnɯtʂuⒽ2}{\textit{ \papi{nɯtʂu2}}}
}\end{relation-sémantique}
\begin{relation-sémantique}\confer{
\hyperlink{Ⓔntʂu}{\textit{ \papi{ntʂu}}}
}\end{relation-sémantique}
\begin{relation-sémantique}\confer{
\hyperlink{Ⓔnɤtʂɤtshi}{\textit{ \papi{nɤtʂɤtshi}}}
}\end{relation-sémantique}\end{entrée}

\begin{entrée}
\vedette{\hypertarget{Ⓔtʂaβ}{\papi{ tʂaβ}}}\markboth{tʂaβ}{}\classe{vt}
\paradigme{\textit{dir :} \jya pɯ-}
\paradigme{\textit{dir :} \jya thɯ-}
\paradigme{\textit{dir :} \jya \_}\acception{1}
\begin{définition}\fra faire dégringoler, rouler\end{définition}
\begin{définition}\cmn 令……滚下去\end{définition}\acception{2}
\begin{définition}\fra faire s'effrondrer\end{définition}
\begin{définition}\cmn 令……倒下\end{définition}
\begin{exemple}\jya pɯ-kɯ-tʂaβ-a\cmn 你把我绊倒了\end{exemple}
\begin{exemple}\jya pɯ-tʂaβ-a\cmn 我把他绊倒了\end{exemple}
\begin{exemple}\jya a-ʑɯβ ɯ-tɯ-ɣi kɯ ɣɯ-tʂaβ-a ɲɯ-ɕti nɤma, ɲɯ-βʁa\cmn 我困得快要倒下去了,因为睡意很厉害\end{exemple}
\begin{relation-sémantique}\confer{
\hyperlink{Ⓔndʐaβ}{\textit{ \papi{ndʐaβ}}}
}\end{relation-sémantique}
\begin{relation-sémantique}\confer{
\hyperlink{Ⓔnɤtʂaβlaβ}{\textit{ \papi{nɤtʂaβlaβ}}}
}\end{relation-sémantique}\begin{sous-entrée}
\vedette{\hypertarget{}{\papi{ ʑɣɤtʂaβ}}}\markboth{ʑɣɤtʂaβ}{}\classe{vi}
\paradigme{\textit{dir :} \jya \_}
\begin{définition}\fra se laisser tomber par terre\end{définition}
\begin{définition}\cmn (故意)倒在地上\end{définition}
\end{sous-entrée}\end{entrée}

\begin{entrée}
\vedette{\hypertarget{Ⓔtʂamɯɣ}{\papi{ tʂamɯɣ}}}\markboth{tʂamɯɣ}{}\classe{n}
\begin{définition}\fra placard\end{définition}
\begin{définition}\cmn 藏式的碗柜\end{définition}\end{entrée}

\begin{entrée}
\vedette{\hypertarget{Ⓔtʂaŋ}{\papi{ tʂaŋ}}}\markboth{tʂaŋ}{}\classe{vs}
\paradigme{\textit{dir :} \jya tɤ-}
\begin{définition}\fra être juste\end{définition}
\begin{définition}\cmn 公平
\begin{déclaration} \étymologie{\papi{draŋ}}\end{déclaration}\end{définition}
\begin{exemple}\jya tɯ-rju ɲɯ-tʂaŋ\cmn 那句话是公平的\end{exemple}
\begin{exemple}\jya rɟama ɲɯ-tʂaŋ\cmn 称很准\end{exemple}
\begin{relation-sémantique}\confer{
\hyperlink{Ⓔtɯtʂaŋ}{\textit{ \papi{tɯtʂaŋ}}}
}\end{relation-sémantique}\begin{sous-entrée}
\vedette{\hypertarget{}{\papi{ nɤtʂaŋ}}}\markboth{nɤtʂaŋ}{}\classe{vt}
\begin{définition}\fra trouver juste\end{définition}
\begin{définition}\cmn 觉得公平\end{définition}
\end{sous-entrée}\begin{sous-entrée}
\vedette{\hypertarget{}{\papi{ nɯtɯtʂaŋ}}}\markboth{nɯtɯtʂaŋ}{}\classe{vs}
\begin{définition}\fra être juste\end{définition}
\begin{définition}\cmn 公平\end{définition}
\end{sous-entrée}\begin{sous-entrée}
\vedette{\hypertarget{}{\papi{ sɯxtʂaŋ}}}\markboth{sɯxtʂaŋ}{}\classe{vs}
\begin{définition}\fra régler de façon juste un contentieux\end{définition}
\begin{définition}\cmn 讨个公道;解决纠纷\end{définition}
\begin{exemple}\jya kɯki iʑora ji-kɤ-nɤndɯt ki kɤ-sɯxtʂaŋ mɯ-mɤ-pɯ-khɯ nɤ, mɤ-nɤɕqe-a\cmn 如果不能把我们之间的纠纷解决好的话,我不会放过你的\end{exemple}
\end{sous-entrée}\end{entrée}

\begin{entrée}
\vedette{\hypertarget{Ⓔtʂaŋka}{\papi{ tʂaŋka}}}\markboth{tʂaŋka}{}\classe{n}
\begin{définition}\fra pièce (d'or, d'argent)\end{définition}
\begin{définition}\cmn (金、银)币\end{définition}
\end{entrée}

\begin{entrée}
\vedette{\hypertarget{Ⓔtʂaŋχtɤm}{\papi{ tʂaŋχtɤm}}}\markboth{tʂaŋχtɤm}{}\classe{n}
\begin{définition}\fra vérité\end{définition}
\begin{définition}\cmn 真话
\begin{déclaration} \étymologie{\papi{draŋ.gtam}}\end{déclaration}\end{définition}
\begin{relation-sémantique}\confer{
\hyperlink{Ⓔɯ-stɤrju}{\textit{ \papi{ɯ-stɤrju}}}
}\end{relation-sémantique}\end{entrée}

\begin{entrée}
\vedette{\hypertarget{Ⓔtʂapa}{\papi{ tʂapa}}}\markboth{tʂapa}{}\classe{n}\begin{définition}\fra étable (bovins, ovins)\end{définition}
\begin{définition}\cmn 牛圈;羊圈\end{définition}\end{entrée}

\begin{entrée}
\vedette{\hypertarget{Ⓔtʂaphɤr}{\papi{ tʂaphɤr}}}\markboth{tʂaphɤr}{}
\classe{n}
\begin{définition}\fra bol de moine\end{définition}
\begin{définition}\cmn 僧碗
\begin{déclaration} \étymologie{\papi{grʷa.pʰor}}\end{déclaration}\end{définition}\end{entrée}

\begin{entrée}
\vedette{\hypertarget{Ⓔtʂaqhu}{\papi{ tʂaqhu}}}\markboth{tʂaqhu}{}
\classe{n}
\begin{définition}\fra bord du chemin du côté de la montagne, du côté le plus haut\end{définition}
\begin{définition}\cmn 靠山的路边\end{définition}
\begin{relation-sémantique}\antonyme{
\hyperlink{Ⓔtʂɤndo}{\textit{ \papi{tʂɤndo}}}
}\end{relation-sémantique}
\begin{relation-sémantique}\confer{
\hyperlink{Ⓔtʂu}{\textit{ \papi{tʂu}}}
}\end{relation-sémantique}
\begin{relation-sémantique}\confer{
\hyperlink{Ⓔɯ-qhu}{\textit{ \papi{ɯ-qhu}}}
}\end{relation-sémantique}\end{entrée}

\begin{entrée}
\vedette{\hypertarget{ⒺtʂaʁⒽ1}{\papi{ tʂaʁ}}}\markboth{tʂaʁ}{}\homonyme{1}
\classe{vs}
\begin{définition}\fra avoir de l'effet, aller mieux\end{définition}
\begin{définition}\cmn 有效果,好
\begin{déclaration} \étymologie{\papi{drag}}\end{déclaration}\end{définition}
\begin{exemple}\jya iɕqha a-ʑɯβ wuma ʑo pɯ-ɣe ri, tɤ-rɯɕmi-tɕi, tʂha kɤ-tshi-t-a tɕe ɲɤ-tʂaʁ, tɕe nɯtshɯci a-ʑɯβ mɯ-ɲɤ-ɣi\cmn 刚才很瞌睡,我们聊一下,我又喝了茶,现在就好多了\end{exemple}
\begin{exemple}\jya a-kɯ-mŋɤm ɲɤ-tʂaʁ\cmn 我病好了\end{exemple}
\begin{relation-sémantique}\synonyme{
\hyperlink{Ⓔmna}{\textit{ \papi{mna}}}
}\end{relation-sémantique}
\begin{relation-sémantique}\synonyme{
\hyperlink{Ⓔftshi}{\textit{ \papi{ftshi}}}
}\end{relation-sémantique}\end{entrée}

\begin{entrée}
\vedette{\hypertarget{ⒺtʂaʁⒽ2}{\papi{ tʂaʁ}}}\markboth{tʂaʁ}{}\homonyme{2}\classe{vt}\acception{1}
\paradigme{\textit{dir :} \jya nɯ-}
\begin{définition}\fra mettre en morceaux avec ses doigts\end{définition}
\begin{définition}\cmn 捏烂;捏细\end{définition}
\begin{exemple}\jya paʁtshi ɲɤ-tʂaʁ\cmn 他把猪食捏碎了\end{exemple}
\begin{exemple}\jya (@yangyu) ɲɤ-rɤndzraʁ nɤ ɲɤ-tʂaʁ\cmn 他把洋芋捏了一把,捏烂了\end{exemple}\acception{2}
\paradigme{\textit{dir :} \jya pɯ-}
\begin{définition}\fra réprimer\end{définition}
\begin{définition}\cmn 镇压\end{définition}
\begin{exemple}\jya ŋgundʑɯɣ ɣɯ-ndo tɤ-ra tɕe, tɯrme kɯ-ŋɤn nɯra kɤ-tʂaʁ pjɯ-kɯ-cha ra, tɯrme kɯ-pe nɯra kɤ-ntɕhoz pjɯ-kɯ-cha ra.\cmn 当了领导要会镇住坏人,重用好人\end{exemple}\end{entrée}

\begin{entrée}
\vedette{\hypertarget{Ⓔtʂɤɕphɤt}{\papi{ tʂɤɕphɤt}}}\markboth{tʂɤɕphɤt}{}
\classe{n}
\begin{définition}\fra plantain\end{définition}
\begin{définition}\cmn 车前草\end{définition}
\begin{exemple}\jya tʂɤɕphɤt nɯ tʂɤrkɯ ra wuma ʑo kɤ-ɬoʁ rga, ɯ-jwaʁ sɤtɕha ɯ-taʁ pjɯ-ɤɲɟoʁ ʑo ŋu, ɯ-mɯntoʁ nɯ ɯ-ru kɯ-zri tsa tu-ɬoʁ tɕe nɯ-taʁ ku-ndzoʁ ŋu, ɯ-mdoʁ kɯ-mpɕɤr ra me, ɲɯ́-wɣ-phɯt tɕe ɯ-ŋgru tu. tɯ-ɕɣa kɯ-mŋɤm ɯ-smɤn ŋu. ɯ-mdoʁ kɯ-ɤpɣɯlu tsa ŋu. pakuku tu-ɬoʁ ŋu.\cmn 车前草一般生长在路边上,叶子贴在地面上,茎长到一定程度后在上面开花,花色不美,撕扯叶子时有(较结实的)筋。可以治牙痛。颜色是淡灰色。年年都长。\end{exemple}
\begin{relation-sémantique}\confer{
\hyperlink{Ⓔtʂu}{\textit{ \papi{tʂu}}}
}\end{relation-sémantique}
\begin{relation-sémantique}\confer{
\hyperlink{Ⓔɕphɤt}{\textit{ \papi{ɕphɤt}}}
}\end{relation-sémantique}\end{entrée}

\begin{entrée}
\vedette{\hypertarget{Ⓔtʂɤkɤcu}{\papi{ tʂɤkɤcu}}}\markboth{tʂɤkɤcu}{}
\classe{n}
\begin{définition}\fra bord du chemin du côté est (lorsqu'il n'y a pas de distinction de hauteur entre les deux bords du chemin)\end{définition}
\begin{définition}\cmn 路的两边没有高低之分时,靠近东方的边缘\end{définition}
\begin{relation-sémantique}\confer{
\hyperlink{Ⓔtʂu}{\textit{ \papi{tʂu}}}
}\end{relation-sémantique}\end{entrée}

\begin{entrée}
\vedette{\hypertarget{Ⓔtʂɤm}{\papi{ tʂɤm}}}\markboth{tʂɤm}{}
\classe{n}
\begin{définition}\fra cloison\end{définition}
\begin{définition}\cmn 板壁\end{définition}\end{entrée}

\begin{entrée}
\vedette{\hypertarget{Ⓔtʂɤmar}{\papi{ tʂɤmar}}}\markboth{tʂɤmar}{}
\classe{n}
\begin{définition}\fra beurre que l'on emporte pour le voyage\end{définition}
\begin{définition}\cmn 路上吃的酥油\end{définition}
\begin{relation-sémantique}\confer{
\hyperlink{Ⓔtʂu}{\textit{ \papi{tʂu}}}
}\end{relation-sémantique}
\begin{relation-sémantique}\confer{
\hyperlink{Ⓔta-mar}{\textit{ \papi{ta-mar}}}
}\end{relation-sémantique}\end{entrée}

\begin{entrée}
\vedette{\hypertarget{Ⓔtʂɤmthɯm}{\papi{ tʂɤmthɯm}}}\markboth{tʂɤmthɯm}{}
\classe{n}
\begin{définition}\fra viande que l'on emporte pour le voyage\end{définition}
\begin{définition}\cmn 路上吃的肉\end{définition}
\begin{relation-sémantique}\confer{
\hyperlink{Ⓔtʂu}{\textit{ \papi{tʂu}}}
}\end{relation-sémantique}
\begin{relation-sémantique}\confer{
\hyperlink{Ⓔtɤ-mthɯm}{\textit{ \papi{tɤ-mthɯm}}}
}\end{relation-sémantique}\end{entrée}

\begin{entrée}
\vedette{\hypertarget{Ⓔtʂɤmtshi}{\papi{ tʂɤmtshi}}}\markboth{tʂɤmtshi}{}\classe{n}
\begin{définition}\fra fait de conduire le chemin\end{définition}
\begin{définition}\cmn 引路\end{définition}
\begin{exemple}\jya ɯʑo kɯ a-tʂɤmtshi ta-βzu\cmn 他给我引了路;他指引了我\end{exemple}
\begin{relation-sémantique}\confer{
\hyperlink{Ⓔmtshi}{\textit{ \papi{mtshi}}}
}\end{relation-sémantique}
\begin{relation-sémantique}\confer{
\hyperlink{Ⓔtʂu}{\textit{ \papi{tʂu}}}
}\end{relation-sémantique}
\begin{relation-sémantique}\confer{
\hyperlink{Ⓔɣɯtʂɤmtshi}{\textit{ \papi{ɣɯtʂɤmtshi}}}
}\end{relation-sémantique}\end{entrée}

\begin{entrée}
\vedette{\hypertarget{Ⓔtʂɤmtshi}{\papi{ tʂɤmtshi}}}\markboth{tʂɤmtshi}{}
\classe{n}
\begin{définition}\fra fait de guider le chemin\end{définition}
\begin{définition}\cmn 带路;引路\end{définition}
\begin{exemple}\jya kɯki tʂu ki nɤ-tʂɤmtshi aj tu-βze-a\cmn 我给你带路\end{exemple}
\begin{relation-sémantique}\confer{
\hyperlink{Ⓔɣɯtʂɤmtshi}{\textit{ \papi{ɣɯtʂɤmtshi}}}
}\end{relation-sémantique}\end{entrée}

\begin{entrée}
\vedette{\hypertarget{Ⓔtʂɤndɤcu}{\papi{ tʂɤndɤcu}}}\markboth{tʂɤndɤcu}{}\classe{n}
\begin{définition}\fra bord du chemin du côté ouest (lorsqu'il n'y a pas de distinction de hauteur entre les deux bords du chemin)\end{définition}
\begin{définition}\cmn 路的两边没有高低之分时,靠近西方的边缘\end{définition}
\end{entrée}

\begin{entrée}
\vedette{\hypertarget{Ⓔtʂɤndo}{\papi{ tʂɤndo}}}\markboth{tʂɤndo}{}
\classe{n}
\begin{définition}\fra bord du chemin du côté du fleuve, du côté le plus bas\end{définition}
\begin{définition}\cmn 靠水的路边\end{définition}
\begin{relation-sémantique}\antonyme{
\hyperlink{Ⓔtʂaqhu}{\textit{ \papi{tʂaqhu}}}
}\end{relation-sémantique}
\begin{relation-sémantique}\confer{
\hyperlink{Ⓔtʂu}{\textit{ \papi{tʂu}}}
}\end{relation-sémantique}
\begin{relation-sémantique}\confer{
\hyperlink{Ⓔɯ-ndo}{\textit{ \papi{ɯ-ndo}}}
}\end{relation-sémantique}\end{entrée}

\begin{entrée}
\vedette{\hypertarget{Ⓔtʂɤrkɯ}{\papi{ tʂɤrkɯ}}}\markboth{tʂɤrkɯ}{}
\classe{n}
\begin{définition}\fra bord du chemin\end{définition}
\begin{définition}\cmn 路边\end{définition}
\begin{relation-sémantique}\confer{
\hyperlink{Ⓔtʂu}{\textit{ \papi{tʂu}}}
}\end{relation-sémantique}
\begin{relation-sémantique}\confer{
\hyperlink{Ⓔɯ-rkɯ}{\textit{ \papi{ɯ-rkɯ}}}
}\end{relation-sémantique}\end{entrée}

\begin{entrée}
\vedette{\hypertarget{Ⓔtʂɤrtaʁ}{\papi{ tʂɤrtaʁ}}}\markboth{tʂɤrtaʁ}{}\classe{n}
\begin{définition}\fra carrefour\end{définition}
\begin{définition}\cmn 岔路\end{définition}
\begin{relation-sémantique}\confer{
\hyperlink{Ⓔtʂu}{\textit{ \papi{tʂu}}}
}\end{relation-sémantique}
\begin{relation-sémantique}\confer{
\hyperlink{Ⓔtɤ-rtaʁ}{\textit{ \papi{tɤ-rtaʁ}}}
}\end{relation-sémantique}\end{entrée}

\begin{entrée}
\vedette{\hypertarget{Ⓔtʂɤsɤɴɢɤt}{\papi{ tʂɤsɤɴɢɤt}}}\markboth{tʂɤsɤɴɢɤt}{}
\classe{n}
\begin{définition}\fra croisée de chemin\end{définition}
\begin{définition}\cmn 岔路\end{définition}
\begin{relation-sémantique}\confer{
\hyperlink{Ⓔtʂu}{\textit{ \papi{tʂu}}}
}\end{relation-sémantique}
\begin{relation-sémantique}\confer{
\hyperlink{Ⓔɴɢɤt}{\textit{ \papi{ɴɢɤt}}}
}\end{relation-sémantique}\end{entrée}

\begin{entrée}
\vedette{\hypertarget{Ⓔtʂɤχa}{\papi{ tʂɤχa}}}\markboth{tʂɤχa}{}
\classe{n}
\begin{définition}\fra fondrière, nid-de-poule\end{définition}
\begin{définition}\cmn 路上的坑洼(缺口)\end{définition}
\begin{exemple}\jya tɯ-ɤtɤr pɯ-kɯ-rɲo kɯ tʂɤχa nɯɣme\cmn 曾经摔倒过的人怕路上的缺口(惊弓之鸟)\end{exemple}
\begin{relation-sémantique}\confer{
\hyperlink{Ⓔaχa}{\textit{ \papi{aχa}}}
}\end{relation-sémantique}
\begin{relation-sémantique}\confer{
\hyperlink{Ⓔtʂu}{\textit{ \papi{tʂu}}}
}\end{relation-sémantique}\end{entrée}

\begin{entrée}
\vedette{\hypertarget{Ⓔtʂɤχcɤl}{\papi{ tʂɤχcɤl}}}\markboth{tʂɤχcɤl}{}
\classe{n}
\begin{définition}\fra milieu du chemin\end{définition}
\begin{définition}\cmn 路中间\end{définition}
\begin{relation-sémantique}\confer{
\hyperlink{Ⓔtʂu}{\textit{ \papi{tʂu}}}
}\end{relation-sémantique}
\begin{relation-sémantique}\confer{
\hyperlink{Ⓔɯ-χcɤl}{\textit{ \papi{ɯ-χcɤl}}}
}\end{relation-sémantique}\end{entrée}

\begin{entrée}
\vedette{\hypertarget{Ⓔtʂɤzɤn}{\papi{ tʂɤzɤn}}}\markboth{tʂɤzɤn}{}\classe{n}
\begin{définition}\fra ration pour la route\end{définition}
\begin{définition}\cmn 盘缠\end{définition}
\begin{relation-sémantique}\confer{
\hyperlink{Ⓔtʂu}{\textit{ \papi{tʂu}}}
}\end{relation-sémantique}\end{entrée}

\begin{entrée}
\vedette{\hypertarget{Ⓔtʂha}{\papi{ tʂha}}}\markboth{tʂha}{}
\classe{n}
\begin{définition}\fra thé, petit déjeuner\end{définition}
\begin{définition}\cmn 茶,早饭\end{définition}
\begin{exemple}\jya tʂha kɤ́tɤlɯlu\cmn 奶茶\end{exemple}\end{entrée}

\begin{entrée}
\vedette{\hypertarget{Ⓔtʂhazwa}{\papi{ tʂhazwa}}}\markboth{tʂhazwa}{}\classe{n}
\begin{définition}\fra feuilles de thé qui restent après que le thé ait été bu\end{définition}
\begin{définition}\cmn 茶喝完了以后留下的茶叶\end{définition}
\end{entrée}

\begin{entrée}
\vedette{\hypertarget{Ⓔtʂhɤlu}{\papi{ tʂhɤlu}}}\markboth{tʂhɤlu}{}\classe{n}
\begin{définition}\fra thé au lait\end{définition}
\begin{définition}\cmn 奶茶\end{définition}
\begin{relation-sémantique}\confer{
\hyperlink{Ⓔnɯtʂhɤlu}{\textit{ \papi{nɯtʂhɤlu}}}
}\end{relation-sémantique}\end{entrée}

\begin{entrée}
\vedette{\hypertarget{Ⓔtʂhɤrqhu}{\papi{ tʂhɤrqhu}}}\markboth{tʂhɤrqhu}{}\classe{n}
\begin{définition}\fra natte utilisée pour conserver le thé\end{définition}
\begin{définition}\cmn 装茶的席子\end{définition}\end{entrée}

\begin{entrée}
\vedette{\hypertarget{ⒺtʂhɤtⒽ2}{\papi{ tʂhɤt}}}\markboth{tʂhɤt}{}\homonyme{2}
\classe{idph.1}
\begin{définition}\fra bruit de gouttes qui tombent\end{définition}
\begin{définition}\cmn 滴水的声音\end{définition}\begin{sous-entrée}
\vedette{\hypertarget{}{\papi{ tʂhɤtnɤtʂhɤt}}}\markboth{tʂhɤtnɤtʂhɤt}{}\classe{idph.3}
\begin{exemple}\jya tɯ-ci tʂhɤtnɤtʂhɤt ɲɯ-nɯftsaʁ\end{exemple}
\begin{exemple}\jya tɯftsaʁ tʂhɤtnɤtʂhɤt ʑo ɲɯ-nɯftsaʁ\end{exemple}
\begin{exemple}\jya tɯftsaʁ tʂhɤtnɤtʂhɤt ʑo ɲɯ-ɣi\end{exemple}
\begin{exemple}\jya tɯftsaʁ tʂhɤtnɤtʂhɤt ʑo ɲɯ-ti\cmn 一滴一滴地漏水\end{exemple}
\end{sous-entrée}\end{entrée}

\begin{entrée}
\vedette{\hypertarget{ⒺtʂhɤtⒽ1}{\papi{ tʂhɤt}}}\markboth{tʂhɤt}{}\homonyme{1}
\classe{vs}
\begin{définition}\fra arrogant\end{définition}
\begin{définition}\cmn 傲慢\end{définition}
\begin{exemple}\jya jiɕqha tɯrme nɯ kɯ-tʂhɤt ci ŋu\cmn 那个人\end{exemple}\begin{sous-entrée}
\vedette{\hypertarget{}{\papi{ znɤtʂhɯtʂhɯt}}}\markboth{znɤtʂhɯtʂhɯt}{} (\variante{znɤtʂhɤtʂhɤt}) \classe{vs}
\begin{définition}\fra arrogant\end{définition}
\begin{définition}\cmn 自以为是;霸道\end{définition}
\begin{exemple}\jya kɯ-znɤtʂhɯtʂhɯt ci ɲɯ-ŋu\cmn 他是个霸道的人\end{exemple}
\end{sous-entrée}\end{entrée}

\begin{entrée}
\vedette{\hypertarget{Ⓔtʂhɤzwa}{\papi{ tʂhɤzwa}}}\markboth{tʂhɤzwa}{}
\classe{n}
\begin{définition}\fra lie du thé\end{définition}
\begin{définition}\cmn 茶里的渣滓\end{définition}\end{entrée}

\begin{entrée}
\vedette{\hypertarget{Ⓔtʂhɯβnɤtʂhɯβ}{\papi{ tʂhɯβnɤtʂhɯβ}}}\markboth{tʂhɯβnɤtʂhɯβ}{}\classe{idph.3}
\begin{définition}\fra bruit produit lorsque l'on se mouche le nez\end{définition}
\begin{définition}\cmn 擤鼻涕的声音\end{définition}
\begin{exemple}\jya tɤ-pɤtso kɯ ɯ-ɕna mɯ-chɯ-pɕiz tɕe, tʂhɯβnɤtʂhɯβ tu-sɯ-ti ɲɯ-ŋu\cmn 小孩子没有擤好鼻涕就发出呼哧呼哧的声音\end{exemple}\begin{sous-entrée}
\vedette{\hypertarget{}{\papi{ phɯtʂhɯβ}}}\markboth{phɯtʂhɯβ}{}\classe{idph.7}
\end{sous-entrée}\end{entrée}

\begin{entrée}
\vedette{\hypertarget{Ⓔtʂhɯɣ}{\papi{ tʂhɯɣ}}}\markboth{tʂhɯɣ}{}
\classe{adv}
\begin{définition}\fra peut-être\end{définition}
\begin{définition}\cmn 大概;可能\end{définition}
\begin{exemple}\jya nɤʑo tʂhɯɣ thɯ-tɯ-nɯkɯmaʁ\cmn 你可能错了\end{exemple}
\begin{exemple}\jya nɯ tʂhɯɣ maʁ lo\cmn 大概不是吧\end{exemple}\end{entrée}

\begin{entrée}
\vedette{\hypertarget{Ⓔtʂhɯznɤtʂhɯz}{\papi{ tʂhɯznɤtʂhɯz}}}\markboth{tʂhɯznɤtʂhɯz}{}\classe{idph.3}
\begin{définition}\fra petit bruit d'explosion\end{définition}
\begin{définition}\cmn 形容爆炸的细小声音\end{définition}
\begin{exemple}\jya tɤɕi chɯ́-wɣ-rŋu tɕe, tɤ-smi tɕe, tʂhɯznɤtʂhɯz ʑo ɲɯ-ɤmboʁ ŋu\cmn 炒青稞的时候,炒熟了就会爆炸发出声音\end{exemple}\begin{sous-entrée}
\vedette{\hypertarget{}{\papi{ phɯtʂhɯz}}}\markboth{phɯtʂhɯz}{}\classe{idph.7}
\begin{définition}\fra bruit d'explosion soudaine\end{définition}
\begin{définition}\cmn 形容突然爆炸的声音\end{définition}
\begin{exemple}\jya tɤntɤβ phɯtʂhɯz ʑo ɲɯ-ɤmboʁ\cmn 水泡噗嗤一声就爆裂了\end{exemple}
\end{sous-entrée}\end{entrée}

\begin{entrée}
\vedette{\hypertarget{Ⓔtʂo}{\papi{ tʂo}}}\markboth{tʂo}{}
\classe{vt}
\paradigme{\textit{dir :} \jya nɯ-}
\paradigme{\textit{dir :} \jya pɯ-}
\begin{définition}\fra payer\end{définition}
\begin{définition}\cmn 付\end{définition}
\begin{exemple}\jya ɯ-nŋa ɲɤ-tʂo\cmn 他还了债\end{exemple}
\begin{relation-sémantique}\confer{
\hyperlink{Ⓔnɯnŋɤtʂo}{\textit{ \papi{nɯnŋɤtʂo}}}
}\end{relation-sémantique}\end{entrée}

\begin{entrée}
\vedette{\hypertarget{Ⓔtʂoʁ}{\papi{ tʂoʁ}}}\markboth{tʂoʁ}{}
\classe{vt}
\paradigme{\textit{dir :} \jya kɤ-}
\paradigme{\textit{dir :} \jya tɤ-}
\paradigme{\textit{dir :} \jya pɯ-}
\begin{définition}\fra ajouter de l’eau\end{définition}
\begin{définition}\cmn 掺和\end{définition}
\begin{exemple}\jya tɯ-ci kɤ-tʂoʁ\cmn 掺水吧\end{exemple}
\begin{exemple}\jya cha to-tʂoʁ\cmn 他掺了酒\end{exemple}\end{entrée}

\begin{entrée}
\vedette{\hypertarget{Ⓔtʂot}{\papi{ tʂot}}}\markboth{tʂot}{}
\classe{vs}
\paradigme{\textit{dir :} \jya tɤ-}
\begin{définition}\fra clair\end{définition}
\begin{définition}\cmn 清楚;清晰;明显\end{définition}
\begin{exemple}\jya tʂu ɲɯ-tʂot\cmn 路很清晰(因为走的人多,路没有消失)\end{exemple}\end{entrée}

\begin{entrée}
\vedette{\hypertarget{Ⓔtʂɯβ}{\papi{ tʂɯβ}}}\markboth{tʂɯβ}{}\classe{vt}
\paradigme{\textit{dir :} \jya kɤ-}
\paradigme{\textit{dir :} \jya thɯ-}
\begin{définition}\fra coudre\end{définition}
\begin{définition}\cmn 缝
\begin{déclaration} \étymologie{\papi{ⁿdrub}}\end{déclaration}\end{définition}
\begin{exemple}\jya tɯ-xtsa kɤ-tʂɯβ-a\cmn 我缝了鞋子\end{exemple}\begin{sous-entrée}
\vedette{\hypertarget{}{\papi{ rɤtʂɯβ}}}\markboth{rɤtʂɯβ}{}\classe{vi}
\paradigme{\textit{dir :} \jya thɯ-}
\paradigme{\textit{dir :} \jya pɯ-}
\begin{définition}\ 
\begin{déclaration}\grammar{apass}\end{déclaration}\end{définition}
\begin{définition}\fra coudre\end{définition}
\begin{définition}\cmn 缝;补\end{définition}
\begin{exemple}\jya chɤ-rɤtʂɯβ\cmn 他缝了\end{exemple}
\end{sous-entrée}\end{entrée}

\begin{entrée}
\vedette{\hypertarget{Ⓔtʂɯɣlaʁ}{\papi{ tʂɯɣlaʁ}}}\markboth{tʂɯɣlaʁ}{}\classe{n}
\begin{définition}\fra ramure de cerf à 12 branches\end{définition}
\begin{définition}\cmn 十二叉的鹿角\end{définition}
\end{entrée}

\begin{entrée}
\vedette{\hypertarget{Ⓔtʂɯɣpa}{\papi{ tʂɯɣpa}}}\markboth{tʂɯɣpa}{}\classe{n}
\begin{définition}\fra sixième mois\end{définition}
\begin{définition}\cmn 六月
\begin{déclaration} \étymologie{\papi{drug.pa}}\end{déclaration}\end{définition}
\end{entrée}

\begin{entrée}
\vedette{\hypertarget{Ⓔtʂɯl}{\papi{ tʂɯl}}}\markboth{tʂɯl}{}
\classe{vt}
\paradigme{\textit{dir :} \jya tɤ-}
\paradigme{\textit{dir :} \jya thɯ-}
\begin{définition}\fra enrouler (vêtement)\end{définition}
\begin{définition}\cmn 包、裹(衣服)
\begin{déclaration} \étymologie{\papi{sgril}}\end{déclaration}\end{définition}
\begin{exemple}\jya thɯ-tʂɯl-a\cmn 我裹了(衣服)\end{exemple}
\begin{exemple}\jya tɯ-ŋga thɯ-tʂɯl\cmn 你把衣服裹起来吧\end{exemple}
\begin{exemple}\jya ɯ-ku to-tʂɯl tslɯɣtslɯɣ ʑo (=to-mphɯr)\cmn 他把头包得又厚有紧\end{exemple}
\begin{relation-sémantique}\synonyme{
\hyperlink{Ⓔmphɯr}{\textit{ \papi{mphɯr}}}
}\end{relation-sémantique}\end{entrée}

\begin{entrée}
\vedette{\hypertarget{Ⓔtʂɯmpa}{\papi{ tʂɯmpa}}}\markboth{tʂɯmpa}{}
\classe{n}
\begin{définition}\fra tablier\end{définition}
\begin{définition}\cmn 围腰帕\end{définition}\end{entrée}

\begin{entrée}
\vedette{\hypertarget{Ⓔtʂɯnlɤn}{\papi{ tʂɯnlɤn}}}\markboth{tʂɯnlɤn}{}
\classe{n}
\begin{définition}\fra bienfait\end{définition}
\begin{définition}\cmn 报恩
\begin{déclaration} \étymologie{\papi{drin.len}}\end{déclaration}\end{définition}
\begin{exemple}\jya nɤ-tʂɯnlɤn ɲɯ-ta-fsɯɣ ra\cmn 我要回报你的恩情\end{exemple}
\end{entrée}

\begin{entrée}
\vedette{\hypertarget{Ⓔtʂɯtʂu}{\papi{ tʂɯtʂu}}}\markboth{tʂɯtʂu}{}\classe{adv}
\begin{définition}\fra en chemin\end{définition}
\begin{définition}\cmn 路上\end{définition}
\begin{relation-sémantique}\confer{
\hyperlink{Ⓔtʂu}{\textit{ \papi{tʂu}}}
}\end{relation-sémantique}\end{entrée}

\begin{entrée}
\vedette{\hypertarget{Ⓔtɯ-boʁ}{\papi{ tɯ-boʁ}}}\markboth{tɯ-boʁ}{}\classe{clf}
\begin{définition}\fra troupeau, groupe\end{définition}
\begin{définition}\cmn 一群;一队\end{définition}
\begin{exemple}\jya qaʑo tɯ-boʁ\cmn 一群羊\end{exemple}
\begin{exemple}\jya kɯ-nɤʁaʁ tɯrme tɯboʁ tu\cmn 有一群人在休息\end{exemple}\end{entrée}

\begin{entrée}
\vedette{\hypertarget{Ⓔtɯβɣi}{\papi{ tɯβɣi}}}\markboth{tɯβɣi}{}
\classe{n}
\begin{définition}\fra balle\end{définition}
\begin{définition}\cmn 糠\end{définition}\end{entrée}

\begin{entrée}
\vedette{\hypertarget{Ⓔtɯ-βlɤz}{\papi{ tɯ-βlɤz}}}\markboth{tɯ-βlɤz}{}
\classe{np}
\begin{définition}\fra devoir\end{définition}
\begin{définition}\cmn 义务
\begin{déclaration} \étymologie{\papi{blas}}\end{déclaration}\end{définition}
\begin{exemple}\jya tɕhaʁra chɯ-raʁrɯz-a ma a-βlɤz ɕti\cmn 我扫厕所,因为这是我的义务\end{exemple}
\begin{exemple}\jya aʑo nɤme-a ra ma a-βlɤz ɕti\cmn 我一定要做,因为是我的义务\end{exemple}
\begin{exemple}\jya a-βlɤz kɯ-tu me\cmn 我没有什么要承担的责任\end{exemple}\end{entrée}

\begin{entrée}
\vedette{\hypertarget{Ⓔtɯ-βli}{\papi{ tɯ-βli}}}\markboth{tɯ-βli}{}\classe{np}
\begin{définition}\fra pousses\end{définition}
\begin{définition}\cmn 秧\end{définition}
\begin{exemple}\jya tɯ-βli pɯ-ta-t-a\end{exemple}
\begin{exemple}\jya tɯ-βli pɯ-ji-t-a\cmn 我插了秧了\end{exemple}\end{entrée}

\begin{entrée}
\vedette{\hypertarget{Ⓔtɯ-βlɯz}{\papi{ tɯ-βlɯz}}}\markboth{tɯ-βlɯz}{}\classe{np}
\begin{définition}\fra par cœur\end{définition}
\begin{définition}\cmn 背诵,不需要模型\end{définition}
\begin{exemple}\jya a-βlɯz pɯ-ndɯn-a\cmn 我背诵了\end{exemple}
\begin{exemple}\jya a-βlɯz nɯ-ari\cmn 我不需要模型就做了\end{exemple}
\begin{exemple}\jya a-βlɯz chɯ-taʁ-a khɯ\cmn 我不用模型就会织(花带)\end{exemple}
\begin{relation-sémantique}\confer{
\hyperlink{Ⓔnɯβlɯz}{\textit{ \papi{nɯβlɯz}}}
}\end{relation-sémantique}\end{entrée}

\begin{entrée}
\vedette{\hypertarget{Ⓔtɯ-βra}{\papi{ tɯ-βra}}}\markboth{tɯ-βra}{}\classe{np}\acception{1}
\begin{définition}\fra part\end{définition}
\begin{définition}\cmn 自己的那一份\end{définition}
\begin{exemple}\jya a-βra nɯ-nɯ-ta-t-a\cmn 我给自己留了一份\end{exemple}\acception{2}
\begin{définition}\fra au tour de...\end{définition}
\begin{définition}\cmn 轮到……\end{définition}
\begin{exemple}\jya aʑo tɤ-ndza-t-a tɕe nɤʑo (nɤ-)βra tɤ-ndze\cmn 我已经吃了,轮到你吃了\end{exemple}\end{entrée}

\begin{entrée}
\vedette{\hypertarget{Ⓔtɯ-βri}{\papi{ tɯ-βri}}}\markboth{tɯ-βri}{}
\classe{np}
\begin{définition}\fra corps\end{définition}
\begin{définition}\cmn 身体(上的皮肤)
\end{définition}\begin{sous-entrée}
\vedette{\hypertarget{}{\papi{ tɯ-βri,ɕe}}}\markboth{tɯ-βri,ɕe}{}
\begin{définition}\fra se faire avoir\end{définition}
\begin{définition}\cmn 吃亏\end{définition}
\begin{exemple}\jya nɤ-βri ɕe\cmn 你要吃亏\end{exemple}
\begin{relation-sémantique}\ComponentA{\classe{np}
\hyperlink{Ⓔtɯ-βri}{\textit{ \papi{tɯ-βri}}}
}\end{relation-sémantique}
\begin{relation-sémantique}\ComponentB{\classe{vi}
\hyperlink{Ⓔɕe}{\textit{ \papi{ɕe}}}
}\end{relation-sémantique}
\end{sous-entrée}\end{entrée}

\begin{entrée}
\vedette{\hypertarget{Ⓔtɯ-ci}{\papi{ tɯ-ci}}}\markboth{tɯ-ci}{}\classe{np}
\acception{1}
\begin{définition}\fra eau\end{définition}
\begin{définition}\cmn 水\end{définition}
\begin{exemple}\jya mɯntoʁ ɯ-tɯ-ci kɤ-lɤt ɲɯ-ra\cmn 要给花浇水\end{exemple}\acception{2}
\begin{définition}\fra jus, lait (d'une plante)\end{définition}
\begin{définition}\cmn 汁(植物)\end{définition}
\begin{relation-sémantique}\confer{
\hyperlink{Ⓔaɣɯci}{\textit{ \papi{aɣɯci}}}
}\end{relation-sémantique}\end{entrée}

\begin{entrée}
\vedette{\hypertarget{Ⓔtɯciβɣuβɣu}{\papi{ tɯciβɣuβɣu}}}\markboth{tɯciβɣuβɣu}{}
\classe{n}
\begin{définition}\fra têtard\end{définition}
\begin{définition}\cmn 蝌蚪\end{définition}\end{entrée}

\begin{entrée}
\vedette{\hypertarget{Ⓔtɯciɕku}{\papi{ tɯciɕku}}}\markboth{tɯciɕku}{}\classe{n}
\begin{définition}\fra poireau sauvage\end{définition}
\begin{définition}\cmn 野韭菜\end{définition}
\begin{exemple}\jya tɯciɕku nɯ tɯ-ci ɯ-rkɯ qambɯt ɯ-ŋgɯ rdɤstaʁ ɯ-rchɤβ tu-ɬoʁ ŋu. ɯ-tshɯɣa ra lonba tɤ-spɯ ɕku fse, tɕeri ɯ-mdoʁ arŋi. kɯ-ɣɤjlu di me. tu-tɯ-ɬoʁ tú-wɣ-ndza tɕe mɯm.\cmn 
\stylefv{tɯciɕku}生长在大河边的沙滩里和石头缝里,形状完全和\stylefv{tɤspɯɕku} 一样,但是颜色是绿色的,没有腥味。刚长出来时吃起来很香。
\end{exemple}\end{entrée}

\begin{entrée}
\vedette{\hypertarget{Ⓔtɯcifkɯm}{\papi{ tɯcifkɯm}}}\markboth{tɯcifkɯm}{}\classe{n}
\begin{définition}\fra gourde\end{définition}
\begin{définition}\cmn 水壶,装水的东西\end{définition}
\begin{relation-sémantique}\confer{
\hyperlink{Ⓔtɤ-fkɯm}{\textit{ \papi{tɤ-fkɯm}}}
}\end{relation-sémantique}\end{entrée}

\begin{entrée}
\vedette{\hypertarget{Ⓔtɯci pɣɤtɕɯ}{\papi{ tɯci pɣɤtɕɯ}}}\markboth{tɯci pɣɤtɕɯ}{}\classe{n}
\begin{définition}\fra espèce d'oiseau\end{définition}
\begin{définition}\cmn 一种鸟\end{définition}\end{entrée}

\begin{entrée}
\vedette{\hypertarget{Ⓔtɯciwɯrwɯr}{\papi{ tɯciwɯrwɯr}}}\markboth{tɯciwɯrwɯr}{}
\classe{n}
\begin{définition}\fra tourbillon\end{définition}
\begin{définition}\cmn 漩涡\end{définition}\end{entrée}

\begin{entrée}
\vedette{\hypertarget{Ⓔtɯcizʁe}{\papi{ tɯcizʁe}}}\markboth{tɯcizʁe}{}\classe{n}
\begin{définition}\fra action de transporter de l'eau\end{définition}
\begin{définition}\cmn 背水\end{définition}
\begin{exemple}\jya tɯcizʁe tɤ-βzu-t-a\cmn 我背了很多水\end{exemple}
\begin{relation-sémantique}\confer{
\hyperlink{Ⓔnɯtɯcizʁe}{\textit{ \papi{nɯtɯcizʁe}}}
}\end{relation-sémantique}\end{entrée}

\begin{entrée}
\vedette{\hypertarget{Ⓔtɯcɯla}{\papi{ tɯcɯla}}}\markboth{tɯcɯla}{}
\classe{n}
\begin{définition}\fra eau bouillante\end{définition}
\begin{définition}\cmn 开水\end{définition}
\begin{relation-sémantique}\confer{
\hyperlink{Ⓔala}{\textit{ \papi{ala}}}
}\end{relation-sémantique}\end{entrée}

\begin{entrée}
\vedette{\hypertarget{Ⓔtɯcɯrqɯ}{\papi{ tɯcɯrqɯ}}}\markboth{tɯcɯrqɯ}{}
\classe{n}
\begin{définition}\fra eau froide\end{définition}
\begin{définition}\cmn 冷水\end{définition}\end{entrée}

\begin{entrée}
\vedette{\hypertarget{Ⓔtɯcɯste}{\papi{ tɯcɯste}}}\markboth{tɯcɯste}{}\classe{n}
\begin{définition}\fra sac amniotique\end{définition}
\begin{définition}\cmn 羊膜囊\end{définition}
\begin{exemple}\jya ɯ-tɯcɯste chɤ-ndʑɣaʁ\cmn 她羊水破了\end{exemple}\end{entrée}

\begin{entrée}
\vedette{\hypertarget{Ⓔtɯ-ɕa}{\papi{ tɯ-ɕa}}}\markboth{tɯ-ɕa}{}
\classe{np}
\begin{définition}\fra muscle\end{définition}
\begin{définition}\cmn 肌肉
\begin{déclaration} \étymologie{\papi{ɕa}}\end{déclaration}\end{définition}\end{entrée}

\begin{entrée}
\vedette{\hypertarget{Ⓔtɯ-ɕaqraʁ}{\papi{ tɯ-ɕaqraʁ}}}\markboth{tɯ-ɕaqraʁ}{}
\classe{np}
\begin{définition}\fra omoplate\end{définition}
\begin{définition}\cmn 肩胛骨\end{définition}\end{entrée}

\begin{entrée}
\vedette{\hypertarget{Ⓔtɯɕɤt}{\papi{ tɯɕɤt}}}\markboth{tɯɕɤt}{}\classe{n}
\begin{définition}\fra habitude\end{définition}
\begin{définition}\cmn 习惯\end{définition}
\begin{exemple}\jya ki nɤʑo nɤ-tɯɕɤt ɕti\cmn 这是你的习惯\end{exemple}\end{entrée}

\begin{entrée}
\vedette{\hypertarget{Ⓔtɯ-ɕɣa}{\papi{ tɯ-ɕɣa}}}\markboth{tɯ-ɕɣa}{}\classe{np}
\begin{définition}\fra dent\end{définition}
\begin{définition}\cmn 牙齿\end{définition}
\begin{exemple}\jya a-ɕɣa qajɯ kɯ ɲɯ-ɤsɯ-ndza\cmn 我长了虫牙了。\end{exemple}
\begin{exemple}\jya tɯ-ɕɣa qajɯ kɯ tu-ndze tɕe wuma ʑo mŋɤm\cmn 有蛀牙,很痛\end{exemple}
\begin{exemple}\jya a-ɕɣa ɲɯ-mŋɤm\cmn 牙疼!\end{exemple}
\begin{relation-sémantique}\confer{
\hyperlink{Ⓔaɕɣa}{\textit{ \papi{aɕɣa}}}
}\end{relation-sémantique}\end{entrée}

\begin{entrée}
\vedette{\hypertarget{Ⓔtɯ-ɕɣɤdi}{\papi{ tɯ-ɕɣɤdi}}}\markboth{tɯ-ɕɣɤdi}{}\classe{np}
\begin{définition}\fra haleine fétide\end{définition}
\begin{définition}\cmn 口臭\end{définition}
\begin{exemple}\jya nɤ-ɕɣa nɯ-χtɕi ma nɤ-ɕɣɤdi mnɤm ko\cmn 你要刷牙,你有口臭\end{exemple}
\begin{relation-sémantique}\confer{
\hyperlink{Ⓔtɯ-ɕɣa}{\textit{ \papi{tɯ-ɕɣa}}}
}\end{relation-sémantique}
\begin{relation-sémantique}\confer{
\hyperlink{Ⓔtɤ-di}{\textit{ \papi{tɤ-di}}}
}\end{relation-sémantique}\end{entrée}

\begin{entrée}
\vedette{\hypertarget{Ⓔtɯɕɣɤŋɤm}{\papi{ tɯɕɣɤŋɤm}}}\markboth{tɯɕɣɤŋɤm}{}
\classe{n}
\begin{définition}\fra mal aux dents\end{définition}
\begin{définition}\cmn 牙疼\end{définition}
\begin{relation-sémantique}\confer{
\hyperlink{Ⓔtɯ-ɕɣa}{\textit{ \papi{tɯ-ɕɣa}}}
}\end{relation-sémantique}
\begin{relation-sémantique}\confer{
\hyperlink{Ⓔmŋɤm}{\textit{ \papi{mŋɤm}}}
}\end{relation-sémantique}\end{entrée}

\begin{entrée}
\vedette{\hypertarget{Ⓔtɯ-ɕɣɤrgu}{\papi{ tɯ-ɕɣɤrgu}}}\markboth{tɯ-ɕɣɤrgu}{}\classe{np}
\begin{définition}\fra qualité de la dentition\end{définition}
\begin{définition}\cmn 牙齿的质量\end{définition}
\begin{exemple}\jya ɯ-ɕɣɤrgu ɲɯ-sna\cmn 他牙口很好\end{exemple}
\begin{relation-sémantique}\confer{
\hyperlink{Ⓔtɯ-ɕɣa}{\textit{ \papi{tɯ-ɕɣa}}}
}\end{relation-sémantique}\end{entrée}

\begin{entrée}
\vedette{\hypertarget{Ⓔtɯ-ɕɣɤse}{\papi{ tɯ-ɕɣɤse}}}\markboth{tɯ-ɕɣɤse}{}\classe{np}
\begin{définition}\fra saignement des dents\end{définition}
\begin{définition}\cmn 牙龈流血\end{définition}
\begin{exemple}\jya ɯ-ɕɣɤse pjɤ-ɬoʁ\cmn 他牙龈流血了\end{exemple}\end{entrée}

\begin{entrée}
\vedette{\hypertarget{Ⓔtɯ-ɕɣɤte}{\papi{ tɯ-ɕɣɤte}}}\markboth{tɯ-ɕɣɤte}{}
\classe{np}
\begin{définition}\fra molaires et prémolaires\end{définition}
\begin{définition}\cmn 臼齿\end{définition}
\begin{relation-sémantique}\confer{
\hyperlink{Ⓔtɯ-sa}{\textit{ \papi{tɯ-sa}}}
}\end{relation-sémantique}\end{entrée}

\begin{entrée}
\vedette{\hypertarget{Ⓔtɯ-ɕkat}{\papi{ tɯ-ɕkat}}}\markboth{tɯ-ɕkat}{}\classe{clf}
\begin{définition}\fra charge sur une bête de somme\end{définition}
\begin{définition}\cmn 驮子\end{définition}
\begin{relation-sémantique}\confer{
\hyperlink{Ⓔɣɯɕkat}{\textit{ \papi{ɣɯɕkat}}}
}\end{relation-sémantique}\end{entrée}

\begin{entrée}
\vedette{\hypertarget{Ⓔtɯɕkho}{\papi{ tɯɕkho}}}\markboth{tɯɕkho}{}\classe{n}
\begin{définition}\fra chose en train d'être séchée\end{définition}
\begin{définition}\cmn 正在晒干的东西\end{définition}
\begin{relation-sémantique}\confer{
\hyperlink{Ⓔɕkho}{\textit{ \papi{ɕkho}}}
}\end{relation-sémantique}\end{entrée}

\begin{entrée}
\vedette{\hypertarget{Ⓔtɯ-ɕkrɯt}{\papi{ tɯ-ɕkrɯt}}}\markboth{tɯ-ɕkrɯt}{}
\classe{np}
\begin{définition}\fra bile\end{définition}
\begin{définition}\cmn 胆
\begin{déclaration} \étymologie{\papi{mkʰris}}\end{déclaration}\end{définition}\end{entrée}

\begin{entrée}
\vedette{\hypertarget{Ⓔtɯɕlu}{\papi{ tɯɕlu}}}\markboth{tɯɕlu}{}\classe{n}
\begin{définition}\fra labour\end{définition}
\begin{définition}\cmn 耕地\end{définition}
\begin{exemple}\jya jla kɯ tɯɕlu ɲɯ-rɤɕi pɯ-ra tɕe, jla a-pɯ-me tɕe tɯ-mgo kɤ-ndza mɯ-pɯ-ŋgrɯ\cmn 只有犏牛才能耕地,没有犏牛就没有饭吃\end{exemple}
\begin{relation-sémantique}\confer{
\hyperlink{Ⓔɕlu}{\textit{ \papi{ɕlu}}}
}\end{relation-sémantique}\end{entrée}

\begin{entrée}
\vedette{\hypertarget{Ⓔtɯ-ɕmi}{\papi{ tɯ-ɕmi}}}\markboth{tɯ-ɕmi}{}\classe{np}
\begin{définition}\fra parole\end{définition}
\begin{définition}\cmn 话\end{définition}
\begin{exemple}\jya ɯ-ɕmi ɲɯ-dɤn\cmn 他话很多\end{exemple}
\begin{relation-sémantique}\confer{
\hyperlink{Ⓔrɯɕmi}{\textit{ \papi{rɯɕmi}}}
}\end{relation-sémantique}\end{entrée}

\begin{entrée}
\vedette{\hypertarget{Ⓔtɯ-ɕna}{\papi{ tɯ-ɕna}}}\markboth{tɯ-ɕna}{}\classe{np}
\begin{définition}\fra nez\end{définition}
\begin{définition}\cmn 鼻子\end{définition}
\begin{relation-sémantique}\confer{
\hyperlink{Ⓔɕnɤxsɯr}{\textit{ \papi{ɕnɤxsɯr}}}
}\end{relation-sémantique}
\begin{relation-sémantique}\confer{
\hyperlink{Ⓔtɯ-ɕnɤɣɲɟɯ}{\textit{ \papi{tɯ-ɕnɤɣɲɟɯ}}}
}\end{relation-sémantique}\end{entrée}

\begin{entrée}
\vedette{\hypertarget{Ⓔtɯ-ɕnaβ}{\papi{ tɯ-ɕnaβ}}}\markboth{tɯ-ɕnaβ}{}\classe{np}
\begin{définition}\fra morve sèche\end{définition}
\begin{définition}\cmn 干的鼻涕\end{définition}
\begin{relation-sémantique}\confer{
\hyperlink{Ⓔaɣɯɕnɯɕnaβ}{\textit{ \papi{aɣɯɕnɯɕnaβ}}}
}\end{relation-sémantique}\end{entrée}

\begin{entrée}
\vedette{\hypertarget{Ⓔtɯ-ɕnɤɣɲɟɯ}{\papi{ tɯ-ɕnɤɣɲɟɯ}}}\markboth{tɯ-ɕnɤɣɲɟɯ}{}\classe{np}
\begin{définition}\fra narine\end{définition}
\begin{définition}\cmn 鼻孔\end{définition}
\begin{relation-sémantique}\confer{
\hyperlink{Ⓔtɯ-ɕna}{\textit{ \papi{tɯ-ɕna}}}
}\end{relation-sémantique}
\begin{relation-sémantique}\confer{
\hyperlink{Ⓔɯ-ɣɲɟɯ}{\textit{ \papi{ɯ-ɣɲɟɯ}}}
}\end{relation-sémantique}\end{entrée}

\begin{entrée}
\vedette{\hypertarget{Ⓔtɯ-ɕnɤjtsi}{\papi{ tɯ-ɕnɤjtsi}}}\markboth{tɯ-ɕnɤjtsi}{}\classe{np}
\begin{définition}\fra arête du nez\end{définition}
\begin{définition}\cmn 鼻梁\end{définition}
\begin{relation-sémantique}\confer{
\hyperlink{Ⓔtɤ-jtsi}{\textit{ \papi{tɤ-jtsi}}}
}\end{relation-sémantique}\end{entrée}

\begin{entrée}
\vedette{\hypertarget{Ⓔtɯ-ɕnɤku}{\papi{ tɯ-ɕnɤku}}}\markboth{tɯ-ɕnɤku}{}\classe{np}
\begin{définition}\fra bout du nez\end{définition}
\begin{définition}\cmn 鼻尖\end{définition}\end{entrée}

\begin{entrée}
\vedette{\hypertarget{Ⓔtɯ-ɕnɤkɯm}{\papi{ tɯ-ɕnɤkɯm}}}\markboth{tɯ-ɕnɤkɯm}{}\classe{np}
\begin{définition}\fra partie entre le nez et la lèvre\end{définition}
\begin{définition}\cmn 人中\end{définition}\end{entrée}

\begin{entrée}
\vedette{\hypertarget{Ⓔtɯ-ɕnɤmtsrɯɣ}{\papi{ tɯ-ɕnɤmtsrɯɣ}}}\markboth{tɯ-ɕnɤmtsrɯɣ}{}\classe{np}
\begin{définition}\fra morve liquide\end{définition}
\begin{définition}\cmn 湿鼻涕\end{définition}
\begin{relation-sémantique}\confer{
\hyperlink{Ⓔtɯ-ɕnaβ}{\textit{ \papi{tɯ-ɕnaβ}}}
}\end{relation-sémantique}\end{entrée}

\begin{entrée}
\vedette{\hypertarget{Ⓔtɯ-ɕnɤɴqhi}{\papi{ tɯ-ɕnɤɴqhi}}}\markboth{tɯ-ɕnɤɴqhi}{}\classe{np}
\begin{définition}\fra crottes de nez\end{définition}
\begin{définition}\cmn 鼻子上的干鼻涕\end{définition}
\end{entrée}

\begin{entrée}
\vedette{\hypertarget{Ⓔtɯ-ɕnɤse}{\papi{ tɯ-ɕnɤse}}}\markboth{tɯ-ɕnɤse}{}\classe{np}
\begin{définition}\fra saigner du nez\end{définition}
\begin{définition}\cmn 鼻血\end{définition}
\begin{exemple}\jya a-ɕnɤse ɲɯ-ɬoʁ\cmn 我正流鼻血\end{exemple}
\end{entrée}

\begin{entrée}
\vedette{\hypertarget{ⒺtɯɕoʁⒽ1}{\papi{ tɯɕoʁ}}}\markboth{tɯɕoʁ}{}\homonyme{1}
\classe{n}
\begin{définition}\fra type de servage\end{définition}
\begin{définition}\cmn 负责交粮食以及干徭役的农民\end{définition}\end{entrée}

\begin{entrée}
\vedette{\hypertarget{Ⓔtɯ-ɕoʁⒽ2}{\papi{ tɯ-ɕoʁ}}}\markboth{tɯ-ɕoʁ}{}\homonyme{2}
\classe{clf}
\begin{définition}\fra une douelle\end{définition}
\begin{définition}\cmn 一条桶板\end{définition}
\begin{exemple}\jya zɯm ɯ-ɕoʁ\cmn 木桶的板\end{exemple}\end{entrée}

\begin{entrée}
\vedette{\hypertarget{Ⓔtɯ-ɕpɤβ}{\papi{ tɯ-ɕpɤβ}}}\markboth{tɯ-ɕpɤβ}{}\classe{np}
\begin{définition}\fra cadavre\end{définition}
\begin{définition}\cmn 尸体\end{définition}
\end{entrée}

\begin{entrée}
\vedette{\hypertarget{Ⓔtɯ-ɕtʂi}{\papi{ tɯ-ɕtʂi}}}\markboth{tɯ-ɕtʂi}{}\classe{np}
\begin{définition}\fra sueur\end{définition}
\begin{définition}\cmn 汗\end{définition}
\begin{exemple}\jya a-ɕtʂi pɯ-ɬoʁ\cmn 我流了汗\end{exemple}
\begin{exemple}\jya ki ta-ma ɲɯ-ɴqa tɕe a-ɕtʂi ʑo pa-tɕɤt\cmn 这个工作很辛苦,搞得我一身都是汗\end{exemple}
\begin{relation-sémantique}\confer{
\hyperlink{Ⓔsɯɕtʂi}{\textit{ \papi{sɯɕtʂi}}}
}\end{relation-sémantique}
\end{entrée}

\begin{entrée}
\vedette{\hypertarget{Ⓔtɯ-ɕtɯ}{\papi{ tɯ-ɕtɯ}}}\markboth{tɯ-ɕtɯ}{}\classe{np}
\begin{définition}\fra organe sexuel féminin\end{définition}
\begin{définition}\cmn 女性生殖器
\begin{déclaration} \étymologie{\papi{stu}}\end{déclaration}\end{définition}
\end{entrée}

\begin{entrée}
\vedette{\hypertarget{Ⓔtɯ-ɕɯrɲo}{\papi{ tɯ-ɕɯrɲo}}}\markboth{tɯ-ɕɯrɲo}{}\classe{np}
\begin{définition}\fra expérience\end{définition}
\begin{définition}\cmn 经验\end{définition}
\begin{exemple}\jya nɤ-ɕɯrɲo kɯ-tu ci tɯ-ɕti\cmn 你是个见过世面的人\end{exemple}
\begin{exemple}\jya ki kɯ-fse kɤ-nɤma aʑɯɣ a-ɕɯrɲo me\cmn 这种事情我没尝试过\end{exemple}
\begin{relation-sémantique}\confer{
\hyperlink{Ⓔrɲo}{\textit{ \papi{rɲo}}}
}\end{relation-sémantique}\end{entrée}

\begin{entrée}
\vedette{\hypertarget{Ⓔtɯdi}{\papi{ tɯdi}}}\markboth{tɯdi}{}\classe{n}
\begin{définition}\fra arc\end{définition}
\begin{définition}\cmn 弓\end{définition}
\begin{exemple}\jya kɯɣɤrʁaʁ kɯ tɯdi to-lɤt\cmn 猎人射了箭\end{exemple}\end{entrée}

\begin{entrée}
\vedette{\hypertarget{Ⓔtɯfcaʁ}{\papi{ tɯfcaʁ}}}\markboth{tɯfcaʁ}{}\classe{n}
\begin{définition}\fra tissu ou morceau de cuir utilisé pour protéger les vêtements lorsque l'on porte quelque chose sur le dos\end{définition}
\begin{définition}\cmn 背东西时垫在背上保护衣服的麻布或皮子\end{définition}
\begin{relation-sémantique}\confer{
\hyperlink{Ⓔnɤfcaʁ}{\textit{ \papi{nɤfcaʁ}}}
}\end{relation-sémantique}\end{entrée}

\begin{entrée}
\vedette{\hypertarget{Ⓔtɯfcɤr}{\papi{ tɯfcɤr}}}\markboth{tɯfcɤr}{}\classe{n}
\begin{définition}\fra poterie\end{définition}
\begin{définition}\cmn 泥工\end{définition}
\begin{exemple}\jya tɯfcɤr tɤ-βzu-t-a\cmn 我做了泥工\end{exemple}
\begin{relation-sémantique}\confer{
\hyperlink{Ⓔrɤfcɤr}{\textit{ \papi{rɤfcɤr}}}
}\end{relation-sémantique}\end{entrée}

\begin{entrée}
\vedette{\hypertarget{Ⓔtɯfɕɤl}{\papi{ tɯfɕɤl}}}\markboth{tɯfɕɤl}{}\classe{n}
\begin{définition}\fra diarrhée\end{définition}
\begin{définition}\cmn 拉肚子
\begin{déclaration} \étymologie{\papi{bɕal}}\end{déclaration}\end{définition}\end{entrée}

\begin{entrée}
\vedette{\hypertarget{ⒺtɯfɕɤtⒽ1}{\papi{ tɯfɕɤt}}}\markboth{tɯfɕɤt}{}\homonyme{1}
\classe{n}
\begin{définition}\fra récit (de ce dont on a été témoin)\end{définition}
\begin{définition}\cmn 叙述(自己的所见所闻)\end{définition}
\begin{exemple}\jya a-tɯfɕɤt pɯ-βze\cmn 跟我说一下你所经历的事情\end{exemple}
\begin{relation-sémantique}\confer{
\hyperlink{ⒺfɕɤtⒽ1}{\textit{ \papi{fɕɤt1}}}
}\end{relation-sémantique}\end{entrée}

\begin{entrée}
\vedette{\hypertarget{Ⓔtɯ-fɕɤtⒽ2}{\papi{ tɯ-fɕɤt}}}\markboth{tɯ-fɕɤt}{}\homonyme{2}\classe{clf}
\begin{définition}\fra un brin\end{définition}
\begin{définition}\cmn 一股;一根\end{définition}
\begin{exemple}\jya tɤrɤm χsɯ-fɕɤt ci tɤ-kɤ-sɯpa\cmn 把三张木板拼在一起(平着放)\end{exemple}\end{entrée}

\begin{entrée}
\vedette{\hypertarget{Ⓔtɯ-fkur}{\papi{ tɯ-fkur}}}\markboth{tɯ-fkur}{}
\classe{clf}
\begin{définition}\fra fardeau\end{définition}
\begin{définition}\cmn 一背\end{définition}
\begin{exemple}\jya si tɯ-fkur\cmn 一背柴\end{exemple}
\begin{exemple}\jya si sqɯ-fkur z-jɤ-re ra\cmn 你要背十背柴回来\end{exemple}\end{entrée}

\begin{entrée}
\vedette{\hypertarget{Ⓔtɯfsɤkha}{\papi{ tɯfsɤkha}}}\markboth{tɯfsɤkha}{}\classe{n}
\begin{définition}\fra aube\end{définition}
\begin{définition}\cmn 黎明\end{définition}
\begin{exemple}\jya tɯfsɤkha ɲɤ-k-ɤβzu-ci\cmn 到了黎明时分\end{exemple}
\begin{exemple}\jya tɯfsɤkha ʑŋgri\cmn 金星\end{exemple}
\begin{relation-sémantique}\confer{
\hyperlink{ⒺfsoʁⒽ2}{\textit{ \papi{fsoʁ2}}}
}\end{relation-sémantique}
\begin{relation-sémantique}\confer{
\hyperlink{Ⓔtɤkha}{\textit{ \papi{tɤkha}}}
}\end{relation-sémantique}\end{entrée}

\begin{entrée}
\vedette{\hypertarget{Ⓔtɯ-fsonam}{\papi{ tɯ-fsonam}}}\markboth{tɯ-fsonam}{}\classe{np}
\begin{définition}\fra chance\end{définition}
\begin{définition}\cmn 运气
\begin{déclaration} \étymologie{\papi{bsod.nams}}\end{déclaration}\end{définition}\end{entrée}

\begin{entrée}
\vedette{\hypertarget{Ⓔtɯftsaʁ}{\papi{ tɯftsaʁ}}}\markboth{tɯftsaʁ}{}
\classe{n}
\begin{définition}\fra eau qui coule dans la maison lorsqu'il pleut\end{définition}
\begin{définition}\cmn 下雨的时候,房子里漏水\end{définition}
\begin{exemple}\jya tɯftsaʁ tʂhɤtnɤtʂhɤt ʑo ɲɯ-nɯftsaʁ\end{exemple}
\begin{exemple}\jya tɯftsaʁ tʂhɤtnɤtʂhɤt ʑo ɲɯ-ɣi\end{exemple}
\begin{exemple}\jya tɯftsaʁ tʂhɤtnɤtʂhɤt ʑo ɲɯ-ti\cmn 一滴一滴地漏水\end{exemple}
\begin{relation-sémantique}\confer{
\hyperlink{Ⓔnɯftsaʁ}{\textit{ \papi{nɯftsaʁ}}}
}\end{relation-sémantique}\end{entrée}

\begin{entrée}
\vedette{\hypertarget{ⒺtɯɣⒽ2}{\papi{ tɯɣ}}}\markboth{tɯɣ}{}\homonyme{2}
\classe{n}
\begin{définition}\fra poison\end{définition}
\begin{définition}\cmn 毒
\begin{déclaration} \étymologie{\papi{dug}}\end{déclaration}\end{définition}
\end{entrée}

\begin{entrée}
\vedette{\hypertarget{ⒺtɯɣⒽ1}{\papi{ tɯɣ}}}\markboth{tɯɣ}{}\homonyme{1}\classe{vi}
\paradigme{\textit{dir :} \jya \_}
\begin{définition}\fra entrer en contact avec\end{définition}
\begin{définition}\cmn 接触到,碰到\end{définition}
\begin{exemple}\jya ɯ-thoʁ pɯ-tɯɣ\cmn 着地了\end{exemple}\begin{sous-entrée}
\vedette{\hypertarget{}{\papi{ sɯxtɯɣ}}}\markboth{sɯxtɯɣ}{}\classe{vt}
\begin{définition}\fra mettre en contact avec\end{définition}
\begin{définition}\cmn 使……接触到\end{définition}
\begin{exemple}\jya laχtɕha thɯ-sthoʁ-a tɕe, znde ɯ-taʁ thɯ-sɯxtɯɣ-a\cmn 我把东西推过去,靠到墙上了\end{exemple}
\end{sous-entrée}\end{entrée}

\begin{entrée}
\vedette{\hypertarget{Ⓔtɯ-ɣdɤt}{\papi{ tɯ-ɣdɤt}}}\markboth{tɯ-ɣdɤt}{}\classe{clf}
\begin{définition}\fra section\end{définition}
\begin{définition}\cmn 一段\end{définition}
\begin{exemple}\jya tʂu tɯ-ɣdɤt\cmn 一段路\end{exemple}
\begin{exemple}\jya tɤ-ri χsɯ-ɣdɤt tɤ-sɯxɕe-t-a (tɤ-βzu-t-a; tɤ-lat-a; nɯ-sɤɣri-t-a)\cmn 我把线切成了三段\end{exemple}\end{entrée}

\begin{entrée}
\vedette{\hypertarget{Ⓔtɯ-ɣjɤn}{\papi{ tɯ-ɣjɤn}}}\markboth{tɯ-ɣjɤn}{}\classe{clf}
\begin{définition}\fra fois\end{définition}
\begin{définition}\cmn 一次\end{définition}
\begin{exemple}\jya χsjɤn; χsɯ-ɣjɤn\cmn 三次\end{exemple}
\end{entrée}

\begin{entrée}
\vedette{\hypertarget{Ⓔtɯɣɟaβ}{\papi{ tɯɣɟaβ}}}\markboth{tɯɣɟaβ}{}\classe{n}
\begin{définition}\fra barattage\end{définition}
\begin{définition}\cmn 搅酥油\end{définition}
\begin{exemple}\jya ta-mar tɯ-tɯɣɟaβ\cmn 一桶酥油\end{exemple}
\begin{relation-sémantique}\confer{
\hyperlink{Ⓔɣɟaβ}{\textit{ \papi{ɣɟaβ}}}
}\end{relation-sémantique}\end{entrée}

\begin{entrée}
\vedette{\hypertarget{Ⓔtɯ-ɣli}{\papi{ tɯ-ɣli}}}\markboth{tɯ-ɣli}{}\classe{np}
\begin{définition}\fra engrais, purin\end{définition}
\begin{définition}\cmn 肥料;粪\end{définition}
\begin{relation-sémantique}\confer{
\hyperlink{Ⓔaɣɯɣli}{\textit{ \papi{aɣɯɣli}}}
}\end{relation-sémantique}\end{entrée}

\begin{entrée}
\vedette{\hypertarget{Ⓔtɯ-ɣmaz}{\papi{ tɯ-ɣmaz}}}\markboth{tɯ-ɣmaz}{}\classe{np}
\begin{définition}\fra blessure\end{définition}
\begin{définition}\cmn 伤口\end{définition}
\begin{exemple}\jya ɯ-ɣmaz pɯ-tɕat-a\cmn 我给他留了伤口\end{exemple}
\begin{relation-sémantique}\confer{
\hyperlink{Ⓔnɯɣmaz}{\textit{ \papi{nɯɣmaz}}}
}\end{relation-sémantique}\end{entrée}

\begin{entrée}
\vedette{\hypertarget{Ⓔtɯ-ɣmɤr}{\papi{ tɯ-ɣmɤr}}}\markboth{tɯ-ɣmɤr}{}\classe{np}
\begin{définition}\fra bouche\end{définition}
\begin{définition}\cmn 嘴\end{définition}
\end{entrée}

\begin{entrée}
\vedette{\hypertarget{Ⓔtɯ-ɣmba}{\papi{ tɯ-ɣmba}}}\markboth{tɯ-ɣmba}{}\classe{np}
\begin{définition}\fra joue\end{définition}
\begin{définition}\cmn 腮\end{définition}\end{entrée}

\begin{entrée}
\vedette{\hypertarget{Ⓔtɯ-ɣmbaɕɤrɯ}{\papi{ tɯ-ɣmbaɕɤrɯ}}}\markboth{tɯ-ɣmbaɕɤrɯ}{}\classe{np}
\begin{définition}\fra pommettes\end{définition}
\begin{définition}\cmn 颧骨\end{définition}
\end{entrée}

\begin{entrée}
\vedette{\hypertarget{Ⓔtɯ-ɣmbɤβ}{\papi{ tɯ-ɣmbɤβ}}}\markboth{tɯ-ɣmbɤβ}{}\classe{np}
\begin{définition}\fra ulcère\end{définition}
\begin{définition}\cmn 脓包\end{définition}
\end{entrée}

\begin{entrée}
\vedette{\hypertarget{Ⓔtɯ-ɣmɯr}{\papi{ tɯ-ɣmɯr}}}\markboth{tɯ-ɣmɯr}{}\classe{clf}
\begin{définition}\fra un soir\end{définition}
\begin{définition}\cmn 一天晚上\end{définition}
\begin{exemple}\jya nɯ ɯ-ɣmɯr\cmn 那一天晚上\end{exemple}
\begin{relation-sémantique}\confer{
\hyperlink{Ⓔjɯɣmɯr}{\textit{ \papi{jɯɣmɯr}}}
}\end{relation-sémantique}
\begin{relation-sémantique}\confer{
\hyperlink{Ⓔɕɯŋgɯmɯr}{\textit{ \papi{ɕɯŋgɯmɯr}}}
}\end{relation-sémantique}
\end{entrée}

\begin{entrée}
\vedette{\hypertarget{Ⓔtɯ-ɣna}{\papi{ tɯ-ɣna}}}\markboth{tɯ-ɣna}{}\classe{clf}
\begin{définition}\fra trente boisseaux\end{définition}
\begin{définition}\cmn 三十升\end{définition}\end{entrée}

\begin{entrée}
\vedette{\hypertarget{Ⓔtɯ-ɣndʑɤr}{\papi{ tɯ-ɣndʑɤr}}}\markboth{tɯ-ɣndʑɤr}{}\classe{np}
\begin{définition}\fra farine, tsampa\end{définition}
\begin{définition}\cmn 面粉;糌粑\end{définition}
\begin{relation-sémantique}\confer{
\hyperlink{Ⓔɣndʑɯr}{\textit{ \papi{ɣndʑɯr}}}
}\end{relation-sémantique}
\end{entrée}

\begin{entrée}
\vedette{\hypertarget{Ⓔtɯ-ɣɲi}{\papi{ tɯ-ɣɲi}}}\markboth{tɯ-ɣɲi}{}
\classe{np}
\begin{définition}\fra ami, allié\end{définition}
\begin{définition}\cmn 朋友;友人\end{définition}
\begin{relation-sémantique}\antonyme{
\hyperlink{Ⓔʁgra}{\textit{ \papi{ʁgra}}}
}\end{relation-sémantique}\end{entrée}

\begin{entrée}
\vedette{\hypertarget{Ⓔtɯɣur}{\papi{ tɯɣur}}}\markboth{tɯɣur}{}\classe{n}
\begin{définition}\fra givre\end{définition}
\begin{définition}\cmn 霜\end{définition}
\begin{exemple}\jya tɯɣur pjɤ-ta\cmn 下了霜\end{exemple}
\begin{relation-sémantique}\confer{
\hyperlink{Ⓔɕŋɤr}{\textit{ \papi{ɕŋɤr}}}
}\end{relation-sémantique}\end{entrée}

\begin{entrée}
\vedette{\hypertarget{Ⓔtɯ-ɣrɤz}{\papi{ tɯ-ɣrɤz}}}\markboth{tɯ-ɣrɤz}{}\classe{np}
\begin{définition}\fra ensemble\end{définition}
\begin{définition}\cmn 跟别人一起\end{définition}\end{entrée}

\begin{entrée}
\vedette{\hypertarget{Ⓔtɯɣro}{\papi{ tɯɣro}}}\markboth{tɯɣro}{}\classe{n}
\begin{définition}\fra paille\end{définition}
\begin{définition}\cmn 干草\end{définition}
\begin{exemple}\jya tɯɣro sɤ-ɕkho\cmn 晒草的地方\end{exemple}\end{entrée}

\begin{entrée}
\vedette{\hypertarget{Ⓔtɯ-ɣrɯmke}{\papi{ tɯ-ɣrɯmke}}}\markboth{tɯ-ɣrɯmke}{}\classe{np}
\begin{définition}\fra poignet\end{définition}
\begin{définition}\cmn 手腕\end{définition}
\begin{relation-sémantique}\confer{
\hyperlink{Ⓔtɯ-zgrɯ}{\textit{ \papi{tɯ-zgrɯ}}}
}\end{relation-sémantique}\end{entrée}

\begin{entrée}
\vedette{\hypertarget{Ⓔtɯɣurʑaʁ}{\papi{ tɯɣurʑaʁ}}}\markboth{tɯɣurʑaʁ}{}\classe{n}
\begin{définition}\fra blé d'hiver\end{définition}
\begin{définition}\cmn 冬种\end{définition}\end{entrée}

\begin{entrée}
\vedette{\hypertarget{Ⓔtɯ-ja}{\papi{ tɯ-ja}}}\markboth{tɯ-ja}{}\classe{np}
\begin{définition}\fra grand frère, grande sœur (terme utilisé par les nobles)\end{définition}
\begin{définition}\cmn 哥哥;姐姐(贵族用语)\end{définition}
\end{entrée}

\begin{entrée}
\vedette{\hypertarget{Ⓔtɯ-jaʁ}{\papi{ tɯ-jaʁ}}}\markboth{tɯ-jaʁ}{}\classe{np}
\begin{définition}\fra main; bras\end{définition}
\begin{définition}\cmn 手\end{définition}
\begin{exemple}\jya a-jaʁqhu / a-jaʁ ɯ-qhu\cmn 我的手背\end{exemple}
\begin{exemple}\jya a-jaχpa\cmn 我的手掌\end{exemple}
\begin{exemple}\jya jiɕqha nɯ ɯ-jaʁ kɯ-rɲɟi ɕi ŋu\cmn 那个人喜欢偷东西\end{exemple}
\begin{exemple}\jya ɯ-jaʁ kɯ-βdi\cmn 手艺很好的人\end{exemple}\end{entrée}

\begin{entrée}
\vedette{\hypertarget{Ⓔtɯ-jaʁfkɯm}{\papi{ tɯ-jaʁfkɯm}}}\markboth{tɯ-jaʁfkɯm}{}\classe{np}
\begin{définition}\fra gant\end{définition}
\begin{définition}\cmn 手套\end{définition}\end{entrée}

\begin{entrée}
\vedette{\hypertarget{Ⓔtɯ-jaʁmu}{\papi{ tɯ-jaʁmu}}}\markboth{tɯ-jaʁmu}{}\classe{np}
\begin{définition}\fra pouce\end{définition}
\begin{définition}\cmn 大拇指\end{définition}
\end{entrée}

\begin{entrée}
\vedette{\hypertarget{Ⓔtɯ-jaʁmɤχa}{\papi{ tɯ-jaʁmɤχa}}}\markboth{tɯ-jaʁmɤχa}{} (\variante{tɯ-jaʁmuχa}) \classe{np}
\begin{définition}\fra espace entre le pouce et l'index\end{définition}
\begin{définition}\cmn 虎口\end{définition}\end{entrée}

\begin{entrée}
\vedette{\hypertarget{Ⓔtɯ-jaʁmundzoʁ}{\papi{ tɯ-jaʁmundzoʁ}}}\markboth{tɯ-jaʁmundzoʁ}{}\classe{clf}
\begin{définition}\fra un doigt\end{définition}
\begin{définition}\cmn 一根指头\end{définition}
\begin{relation-sémantique}\confer{
\hyperlink{Ⓔtɯ-ndzoʁ}{\textit{ \papi{tɯ-ndzoʁ}}}
}\end{relation-sémantique}\end{entrée}

\begin{entrée}
\vedette{\hypertarget{Ⓔtɯ-jaʁndzu}{\papi{ tɯ-jaʁndzu}}}\markboth{tɯ-jaʁndzu}{}\classe{np}
\begin{définition}\fra doigt\end{définition}
\begin{définition}\cmn 手指\end{définition}
\end{entrée}

\begin{entrée}
\vedette{\hypertarget{Ⓔtɯ-jaʁndzu aŋɤn}{\papi{ tɯ-jaʁndzu aŋɤn}}}\markboth{tɯ-jaʁndzu aŋɤn}{}\classe{np}
\begin{définition}\fra auriculaire\end{définition}
\begin{définition}\cmn 小指\end{définition}\end{entrée}

\begin{entrée}
\vedette{\hypertarget{Ⓔtɯ-jaʁndzumaŋlo}{\papi{ tɯ-jaʁndzumaŋlo}}}\markboth{tɯ-jaʁndzumaŋlo}{}\classe{np}
\begin{définition}\fra index\end{définition}
\begin{définition}\cmn 食指\end{définition}\end{entrée}

\begin{entrée}
\vedette{\hypertarget{Ⓔtɯ-jaʁndzumɤpaχcɤl}{\papi{ tɯ-jaʁndzumɤpaχcɤl}}}\markboth{tɯ-jaʁndzumɤpaχcɤl}{}\classe{np}
\begin{définition}\fra majeur\end{définition}
\begin{définition}\cmn 中指\end{définition}\end{entrée}

\begin{entrée}
\vedette{\hypertarget{Ⓔtɯ-jaʁqhu}{\papi{ tɯ-jaʁqhu}}}\markboth{tɯ-jaʁqhu}{}\classe{np}
\begin{définition}\fra dessus de la main\end{définition}
\begin{définition}\cmn 手背\end{définition}
\end{entrée}

\begin{entrée}
\vedette{\hypertarget{Ⓔtɯ-jaʁsta}{\papi{ tɯ-jaʁsta}}}\markboth{tɯ-jaʁsta}{}
\classe{n}
\begin{définition}\fra assurance, entraînement\end{définition}
\begin{définition}\cmn 心中有数;熟练\end{définition}
\begin{exemple}\jya a-jaʁsta nɯ-aβzu\cmn 我心中有数要怎么做\end{exemple}
\begin{exemple}\jya tɤ-scoz kɤ-nɤma nɯ aʑo a-jaʁsta ɕti\cmn 我对写字(整理文件)心中有数\end{exemple}
\begin{exemple}\jya tɕhi kɤ-nɤma pɯ-nnɯ-ŋɯ-ŋu nɯnɯ tɯ-jaʁsta ɲɯ-βze kɯ-ra ɲɯ-ɕti ɲɯ-ŋu, nɯ maʁ nɤ kɤ-ɤɣɯmphɯphru mɯ́j-khɯ\cmn 无论什么工作都要练习才行\end{exemple}
\begin{relation-sémantique}\confer{
\hyperlink{Ⓔtɯ-jaʁ}{\textit{ \papi{tɯ-jaʁ}}}
}\end{relation-sémantique}\end{entrée}

\begin{entrée}
\vedette{\hypertarget{Ⓔtɯ-jaχpa}{\papi{ tɯ-jaχpa}}}\markboth{tɯ-jaχpa}{}\classe{np}
\begin{définition}\fra paume\end{définition}
\begin{définition}\cmn 手掌\end{définition}
\begin{exemple}\jya jaχpa rɯmu\cmn 手纹\end{exemple}
\begin{relation-sémantique}\confer{
\hyperlink{Ⓔtɯ-jaʁ}{\textit{ \papi{tɯ-jaʁ}}}
}\end{relation-sémantique}
\begin{relation-sémantique}\confer{
 \papi{ɯ-pa}
}\end{relation-sémantique}\end{entrée}

\begin{entrée}
\vedette{\hypertarget{ⒺtɯjiⒽ2}{\papi{ tɯji}}}\markboth{tɯji}{}\homonyme{2}
\classe{n}
\begin{définition}\fra semailles\end{définition}
\begin{définition}\cmn 播种\end{définition}
\begin{exemple}\jya tɯji to-mda\cmn 播种的时间到了\end{exemple}
\begin{relation-sémantique}\confer{
\hyperlink{Ⓔji}{\textit{ \papi{ji}}}
}\end{relation-sémantique}\end{entrée}

\begin{entrée}
\vedette{\hypertarget{Ⓔtɯ-jiⒽ1}{\papi{ tɯ-ji}}}\markboth{tɯ-ji}{}\homonyme{1}
\classe{np}
\begin{définition}\fra champs\end{définition}
\begin{définition}\cmn 田地\end{définition}
\begin{relation-sémantique}\confer{
\hyperlink{Ⓔji}{\textit{ \papi{ji}}}
}\end{relation-sémantique}
\end{entrée}

\begin{entrée}
\vedette{\hypertarget{Ⓔtɯjimŋu}{\papi{ tɯjimŋu}}}\markboth{tɯjimŋu}{}\classe{n}
\begin{définition}\fra le bord du champs du côté de la montagne\end{définition}
\begin{définition}\cmn 靠山的田边\end{définition}\end{entrée}

\begin{entrée}
\vedette{\hypertarget{Ⓔtɯjindo}{\papi{ tɯjindo}}}\markboth{tɯjindo}{}\classe{n}
\begin{définition}\fra le bord du champs du côte opposé à la montagne\end{définition}
\begin{définition}\cmn 靠水的田边,靠山那一边的对面\end{définition}\end{entrée}

\begin{entrée}
\vedette{\hypertarget{Ⓔtɯ-jlɤβ}{\papi{ tɯ-jlɤβ}}}\markboth{tɯ-jlɤβ}{}\classe{np}
\begin{définition}\fra fils de trame\end{définition}
\begin{définition}\cmn 纬线\end{définition}
\begin{relation-sémantique}\antonyme{
\hyperlink{Ⓔtɤ-ʁjar}{\textit{ \papi{tɤ-ʁjar}}}
}\end{relation-sémantique}
\begin{relation-sémantique}\confer{
\hyperlink{Ⓔjlɤβndʑu}{\textit{ \papi{jlɤβndʑu}}}
}\end{relation-sémantique}\end{entrée}

\begin{entrée}
\vedette{\hypertarget{Ⓔtɯ-jmetɕɯrɯrɯ}{\papi{ tɯ-jmetɕɯrɯrɯ}}}\markboth{tɯ-jmetɕɯrɯrɯ}{}\classe{n}
\begin{définition}\fra excroissance du bassin\end{définition}
\begin{définition}\cmn 尾椎骨\end{définition}\end{entrée}

\begin{entrée}
\vedette{\hypertarget{Ⓔtɯ-jmŋo}{\papi{ tɯ-jmŋo}}}\markboth{tɯ-jmŋo}{}\classe{np}
\begin{définition}\fra rêve\end{définition}
\begin{définition}\cmn 梦\end{définition}
\begin{exemple}\jya ɯ-jmŋo ko-ntɕhɤr\cmn 他做了梦\end{exemple}
\begin{relation-sémantique}\confer{
\hyperlink{Ⓔɣɤjmŋo}{\textit{ \papi{ɣɤjmŋo}}}
}\end{relation-sémantique}\end{entrée}

\begin{entrée}
\vedette{\hypertarget{Ⓔtɯjno}{\papi{ tɯjno}}}\markboth{tɯjno}{}
\classe{n}
\begin{définition}\fra légume\end{définition}
\begin{définition}\cmn 蔬菜\end{définition}
\begin{exemple}\jya tɯjno tɤ-kɤ-xsɯr\cmn 炒的菜\end{exemple}\end{entrée}

\begin{entrée}
\vedette{\hypertarget{Ⓔtɯjnozwa}{\papi{ tɯjnozwa}}}\markboth{tɯjnozwa}{}\classe{n}
\begin{définition}\fra légumes dans la soupe\end{définition}
\begin{définition}\cmn 菜汤里的菜叶\end{définition}
\end{entrée}

\begin{entrée}
\vedette{\hypertarget{Ⓔtɯjpu}{\papi{ tɯjpu}}}\markboth{tɯjpu}{}\classe{n}
\begin{définition}\fra nourriture\end{définition}
\begin{définition}\cmn 粮食\end{définition}\end{entrée}

\begin{entrée}
\vedette{\hypertarget{Ⓔtɯ-jroʁ}{\papi{ tɯ-jroʁ}}}\markboth{tɯ-jroʁ}{}\classe{clf}
\begin{définition}\fra une rangée\end{définition}
\begin{définition}\cmn 一行,一路线\end{définition}
\begin{exemple}\jya qaj tɯ-jroʁ\cmn 一行麦子\end{exemple}
\begin{relation-sémantique}\synonyme{
\hyperlink{Ⓔtɯ-rɣɯt}{\textit{ \papi{tɯ-rɣɯt}}}
}\end{relation-sémantique}
\begin{relation-sémantique}\confer{
\hyperlink{Ⓔtɤ-jroʁ}{\textit{ \papi{tɤ-jroʁ}}}
}\end{relation-sémantique}
\begin{relation-sémantique}\confer{
\hyperlink{Ⓔnɯjroʁ}{\textit{ \papi{nɯjroʁ}}}
}\end{relation-sémantique}
\begin{relation-sémantique}\confer{
\hyperlink{Ⓔrɤjroʁ}{\textit{ \papi{rɤjroʁ}}}
}\end{relation-sémantique}\end{entrée}

\begin{entrée}
\vedette{\hypertarget{Ⓔtɯjʁo}{\papi{ tɯjʁo}}}\markboth{tɯjʁo}{}\classe{n}
\begin{définition}\fra insulte\end{définition}
\begin{définition}\cmn 骂人的话\end{définition}
\begin{exemple}\jya tɯjʁo ta-khɤt\cmn 他骂了很久\end{exemple}
\begin{exemple}\jya tɯjʁo χɕu\cmn 他骂人很厉害\end{exemple}\end{entrée}

\begin{entrée}
\vedette{\hypertarget{Ⓔtɯ-jɯɣ}{\papi{ tɯ-jɯɣ}}}\markboth{tɯ-jɯɣ}{}\classe{clf}
\begin{définition}\fra pièce de tissu\end{définition}
\begin{définition}\cmn 一匹布\end{définition}
\begin{exemple}\jya raz tɯ-jɯɣ\cmn 一匹布\end{exemple}\end{entrée}

\begin{entrée}
\vedette{\hypertarget{Ⓔtɯɟo}{\papi{ tɯɟo}}}\markboth{tɯɟo}{}\classe{n}
\begin{définition}\fra fantôme\end{définition}
\begin{définition}\cmn 鬼(草登话)\end{définition}\end{entrée}

\begin{entrée}
\vedette{\hypertarget{Ⓔtɯ-ɟom}{\papi{ tɯ-ɟom}}}\markboth{tɯ-ɟom}{}\classe{clf}
\begin{définition}\fra longueur des deux bras étendus\end{définition}
\begin{définition}\cmn 一庹【一排】
\begin{déclaration} \étymologie{\papi{ⁿdom.pa}}\end{déclaration}\end{définition}\end{entrée}

\begin{entrée}
\vedette{\hypertarget{Ⓔtɯ-ɟrɯɣ}{\papi{ tɯ-ɟrɯɣ}}}\markboth{tɯ-ɟrɯɣ}{}\classe{clf}
\begin{définition}\fra plein de choses en désordre\end{définition}
\begin{définition}\cmn 一大堆(很乱)\end{définition}
\begin{exemple}\jya laχtɕha tɯ-ɟrɯɣ ʑo jo-ɣɯt\cmn 他带来一大堆很乱的东西\end{exemple}
\begin{relation-sémantique}\confer{
\hyperlink{Ⓔɟrɯɣɟrɯɣ}{\textit{ \papi{ɟrɯɣɟrɯɣ}}}
}\end{relation-sémantique}\end{entrée}

\begin{entrée}
\vedette{\hypertarget{Ⓔtɯ-ɟɯɣ}{\papi{ tɯ-ɟɯɣ}}}\markboth{tɯ-ɟɯɣ}{}\classe{clf}
\begin{définition}\fra un troupeau\end{définition}
\begin{définition}\cmn 一群\end{définition}
\begin{exemple}\jya mbro tɯ-ɟɯɣ\cmn 一群马\end{exemple}
\end{entrée}

\begin{entrée}
\vedette{\hypertarget{Ⓔtɯ-ku}{\papi{ tɯ-ku}}}\markboth{tɯ-ku}{}
\classe{np}\acception{1}
\begin{définition}\fra tête\end{définition}
\begin{définition}\cmn 头\end{définition}
\begin{exemple}\jya a-ku nɯ-βze\cmn 给我编头发吧\end{exemple}
\begin{exemple}\jya ndʑi-ku a-nɯ-tɯ-ɤnɯβzɯβzu-ndʑi je\cmn 你们互相编头发吧\end{exemple}\acception{2}
\begin{définition}\fra haut\end{définition}
\begin{définition}\cmn 上面\end{définition}
\begin{exemple}\jya sɯjno ɣɯ ɯ-ku\cmn 草的叶子和茎\end{exemple}
\begin{relation-sémantique}\confer{
\hyperlink{Ⓔnɤkɤtɕhɯ}{\textit{ \papi{nɤkɤtɕhɯ}}}
}\end{relation-sémantique}
\begin{relation-sémantique}\confer{
\hyperlink{Ⓔkɤlu}{\textit{ \papi{kɤlu}}}
}\end{relation-sémantique}
\begin{relation-sémantique}\confer{
\hyperlink{Ⓔɯ-kɤlɤjme}{\textit{ \papi{ɯ-kɤlɤjme}}}
}\end{relation-sémantique}\begin{sous-entrée}
\vedette{\hypertarget{}{\papi{ tɯ-ku,ta}}}\markboth{tɯ-ku,ta}{}
\begin{définition}\fra s'allonger, poser la tête\end{définition}
\begin{définition}\cmn 躺下\end{définition}
\begin{exemple}\jya a-ku pɯ-nɯ-ta-t-a tɕe pjɤ-nɯʑɯβ-a\cmn 我躺下睡觉了\end{exemple}
\end{sous-entrée}\begin{sous-entrée}
\vedette{\hypertarget{}{\papi{ ɯ-kuɕɯku}}}\markboth{ɯ-kuɕɯku}{}\classe{np}
\begin{définition}\fra sommet\end{définition}
\begin{définition}\cmn 最顶端\end{définition}
\begin{exemple}\jya pɣɤtɕɯ si ɯ-kuɕɯku ʑo zɯ ko-zo\cmn 鸟落在树的顶端\end{exemple}
\end{sous-entrée}\end{entrée}

\begin{entrée}
\vedette{\hypertarget{Ⓔtɯ-kɤcɯɣ}{\papi{ tɯ-kɤcɯɣ}}}\markboth{tɯ-kɤcɯɣ}{}\classe{np}
\begin{définition}\fra sommet de la tête, fontanelle\end{définition}
\begin{définition}\cmn 头顶;囟门\end{définition}\end{entrée}

\begin{entrée}
\vedette{\hypertarget{Ⓔtɯ-kɤftɕaka}{\papi{ tɯ-kɤftɕaka}}}\markboth{tɯ-kɤftɕaka}{}\classe{np}
\begin{définition}\fra coiffure, bijoux portés sur la tête\end{définition}
\begin{définition}\cmn 头饰\end{définition}
\begin{relation-sémantique}\confer{
\hyperlink{Ⓔtɯ-ku}{\textit{ \papi{tɯ-ku}}}
}\end{relation-sémantique}
\begin{relation-sémantique}\confer{
\hyperlink{Ⓔftɕaka}{\textit{ \papi{ftɕaka}}}
}\end{relation-sémantique}\end{entrée}

\begin{entrée}
\vedette{\hypertarget{Ⓔtɯ-kɤlɤmɲaʁ}{\papi{ tɯ-kɤlɤmɲaʁ}}}\markboth{tɯ-kɤlɤmɲaʁ}{}\classe{np}
\begin{définition}\fra traits du visage\end{définition}
\begin{définition}\cmn 面容的五官\end{définition}
\begin{relation-sémantique}\confer{
\hyperlink{Ⓔtɯ-mɤlɤjaʁ}{\textit{ \papi{tɯ-mɤlɤjaʁ}}}
}\end{relation-sémantique}\end{entrée}

\begin{entrée}
\vedette{\hypertarget{Ⓔtɯ-kɤpɤla}{\papi{ tɯ-kɤpɤla}}}\markboth{tɯ-kɤpɤla}{}\classe{np}
\begin{définition}\fra sommet du crâne\end{définition}
\begin{définition}\cmn 头盖骨
\begin{déclaration} \étymologie{\papi{kapāla}}\end{déclaration}\end{définition}
\end{entrée}

\begin{entrée}
\vedette{\hypertarget{Ⓔtɯ-kɤrnoʁ,mtɕɯr}{\papi{ tɯ-kɤrnoʁ,mtɕɯr}}}\markboth{tɯ-kɤrnoʁ,mtɕɯr}{}\classe{vi}
\begin{définition}\fra avoir un vertige\end{définition}
\begin{définition}\cmn 头晕\end{définition}
\begin{exemple}\jya a-kɤrnoʁ ɲɯ-mtɕɯr ntsɯ pɯ-ŋu.\cmn 我当时总是头晕\end{exemple}\end{entrée}

\begin{entrée}
\vedette{\hypertarget{Ⓔtɯ-kha}{\papi{ tɯ-kha}}}\markboth{tɯ-kha}{}\classe{clf}
\begin{définition}\fra pied\end{définition}
\begin{définition}\cmn 尺
\begin{déclaration} \étymologie{\papi{kʰa}}\end{déclaration}\end{définition}
\end{entrée}

\begin{entrée}
\vedette{\hypertarget{Ⓔtɯ-khɤftsɯɣ}{\papi{ tɯ-khɤftsɯɣ}}}\markboth{tɯ-khɤftsɯɣ}{}\classe{clf}
\begin{définition}\fra un tour d'aiguille\end{définition}
\begin{définition}\cmn 一针(缝衣服的时候)\end{définition}\end{entrée}

\begin{entrée}
\vedette{\hypertarget{Ⓔtɯ-khɤl}{\papi{ tɯ-khɤl}}}\markboth{tɯ-khɤl}{}\classe{clf}
\begin{définition}\fra endroit\end{définition}
\begin{définition}\cmn 一个地方
\begin{déclaration} \étymologie{\papi{kʰol?}}\end{déclaration}\end{définition}
\begin{exemple}\jya jiʑo zgo tɤ-ari tɕe, χsɯ-khɤl tɤ-nɯna-j\cmn 我们上山时,在三个不同的地方休息了一下\end{exemple}
\begin{relation-sémantique}\confer{
\hyperlink{Ⓔɯ-khɯkhɤl}{\textit{ \papi{ɯ-khɯkhɤl}}}
}\end{relation-sémantique}
\begin{relation-sémantique}\confer{
\hyperlink{Ⓔarɤkhɯmkhɤl}{\textit{ \papi{arɤkhɯmkhɤl}}}
}\end{relation-sémantique}
\end{entrée}

\begin{entrée}
\vedette{\hypertarget{Ⓔtɯ-khi}{\papi{ tɯ-khi}}}\markboth{tɯ-khi}{}\classe{np}
\begin{définition}\fra chance\end{définition}
\begin{définition}\cmn 运气\end{définition}
\begin{exemple}\jya ji-khi ma jisŋi tɯ-mɯ ɲɯ-jɯm\cmn 我们很幸运今天天气很好\end{exemple}
\begin{exemple}\jya jɯfɕɯr, ji-rɣa ra kɯ @cai nɯ́-wɣ-mbi-j tɕe ji-khi pɯ-ŋu\cmn 昨天邻居给了我们菜,我们很幸运\end{exemple}
\begin{exemple}\jya nɤ-khi ɲɯ-ŋgɯ\cmn 你运气真好\end{exemple}\end{entrée}

\begin{entrée}
\vedette{\hypertarget{Ⓔtɯ-khroŋkhroŋ}{\papi{ tɯ-khroŋkhroŋ}}}\markboth{tɯ-khroŋkhroŋ}{}\classe{np}
\begin{définition}\fra trachée\end{définition}
\begin{définition}\cmn 喉管\end{définition}\end{entrée}

\begin{entrée}
\vedette{\hypertarget{Ⓔtɯ-khɯr}{\papi{ tɯ-khɯr}}}\markboth{tɯ-khɯr}{}\classe{np}
\begin{définition}\fra place dans la hiérarchie\end{définition}
\begin{définition}\cmn 官职
\begin{déclaration} \étymologie{\papi{kʰur}}\end{déclaration}\end{définition}\end{entrée}

\begin{entrée}
\vedette{\hypertarget{Ⓔtɯ-kuŋa}{\papi{ tɯ-kuŋa}}}\markboth{tɯ-kuŋa}{}\classe{n}
\begin{définition}\fra col\end{définition}
\begin{définition}\cmn 衣领\end{définition}
\begin{relation-sémantique}\synonyme{
\hyperlink{Ⓔtɯ-mkɤscur}{\textit{ \papi{tɯ-mkɤscur}}}
}\end{relation-sémantique}\end{entrée}

\begin{entrée}
\vedette{\hypertarget{Ⓔtɯkon}{\papi{ tɯkon}}}\markboth{tɯkon}{}\classe{n}
\begin{définition}\fra pouvoir\end{définition}
\begin{définition}\cmn 权力
\begin{déclaration} \étymologie{\papi{\stylefn{权}}}\end{déclaration}\end{définition}\end{entrée}

\begin{entrée}
\vedette{\hypertarget{Ⓔtɯkrɤz}{\papi{ tɯkrɤz}}}\markboth{tɯkrɤz}{}\classe{n}
\begin{définition}\fra discussion\end{définition}
\begin{définition}\cmn 商量
\begin{déclaration} \étymologie{\papi{gros}}\end{déclaration}\end{définition}
\begin{exemple}\jya nɯ-tɯkrɤz to-ɣi\cmn 他们商议好了\end{exemple}
\begin{relation-sémantique}\confer{
\hyperlink{Ⓔrɤkrɤz}{\textit{ \papi{rɤkrɤz}}}
}\end{relation-sémantique}
\begin{relation-sémantique}\confer{
\hyperlink{Ⓔnɯkrɤz}{\textit{ \papi{nɯkrɤz}}}
}\end{relation-sémantique}\end{entrée}

\begin{entrée}
\vedette{\hypertarget{Ⓔtɯ-kri}{\papi{ tɯ-kri}}}\markboth{tɯ-kri}{}\classe{np}
\begin{définition}\fra huile\end{définition}
\begin{définition}\cmn 油\end{définition}
\begin{exemple}\jya nɤ-tɯ-kri ɲɯ-sɤle-a\cmn 我给你熬一点油\end{exemple}\end{entrée}

\begin{entrée}
\vedette{\hypertarget{Ⓔtɯkrimgo}{\papi{ tɯkrimgo}}}\markboth{tɯkrimgo}{}\classe{n}
\begin{définition}\fra pain frit\end{définition}
\begin{définition}\cmn 油馍馍\end{définition}\end{entrée}

\begin{entrée}
\vedette{\hypertarget{Ⓔtɯ-kɯr}{\papi{ tɯ-kɯr}}}\markboth{tɯ-kɯr}{}\classe{np}
\begin{définition}\fra bouche\end{définition}
\begin{définition}\cmn 嘴\end{définition}
\begin{exemple}\jya nɤʑo nɤ-kɯr ɯ-tɯ-wxti!\cmn 你真自吹自捧!\end{exemple}\end{entrée}

\begin{entrée}
\vedette{\hypertarget{Ⓔtɯlu}{\papi{ tɯlu}}}\markboth{tɯlu}{}\classe{n}
\begin{définition}\ 
\begin{déclaration}\grammar{n.lieu}\end{déclaration}\end{définition}
\begin{définition}\fra l'un des hameaux de Kamnyu\end{définition}
\begin{définition}\cmn 干木鸟的大队之一\end{définition}
\end{entrée}

\begin{entrée}
\vedette{\hypertarget{Ⓔtɯl}{\papi{ tɯl}}}\markboth{tɯl}{}
\classe{vi}
\paradigme{\textit{dir :} \jya nɯ-}
\begin{définition}\fra devenir mauvais à manger (tsampa)\end{définition}
\begin{définition}\cmn 面或者糌粑变味了\end{définition}
\begin{exemple}\jya tɯsqar ɲɤ-tɯl\cmn 糌粑变味了\end{exemple}
\begin{exemple}\jya tɤjlu ɲɤ-tɯl\cmn 面变味了\end{exemple}\end{entrée}

\begin{entrée}
\vedette{\hypertarget{Ⓔtɯ-lasqra}{\papi{ tɯ-lasqra}}}\markboth{tɯ-lasqra}{}\classe{np}
\begin{définition}\fra raie des cheveux\end{définition}
\begin{définition}\cmn 头路\end{définition}
\end{entrée}

\begin{entrée}
\vedette{\hypertarget{Ⓔtɯ-lazⒽ1}{\papi{ tɯ-laz}}}\markboth{tɯ-laz}{}\homonyme{1}
\classe{np}
\begin{définition}\fra front\end{définition}
\begin{définition}\cmn 额头\end{définition}\end{entrée}

\begin{entrée}
\vedette{\hypertarget{Ⓔtɯ-lazⒽ2}{\papi{ tɯ-laz}}}\markboth{tɯ-laz}{}\homonyme{2}\classe{np}
\begin{définition}\fra karma\end{définition}
\begin{définition}\cmn 运气
\begin{déclaration} \étymologie{\papi{las}}\end{déclaration}\end{définition}
\begin{exemple}\jya a-laz ɯ-tɯ-khe\cmn 我的命很苦\end{exemple}
\begin{exemple}\jya a-laz tu\cmn 我运气好\end{exemple}
\begin{exemple}\jya nɤʑo nɤ-laz ɲɯ-sna\cmn 你命好\end{exemple}
\begin{exemple}\jya tɕi-laz mɤ-khɤm\cmn 我们没有缘分\end{exemple}\end{entrée}

\begin{entrée}
\vedette{\hypertarget{Ⓔtɯ-lɤn}{\papi{ tɯ-lɤn}}}\markboth{tɯ-lɤn}{}\classe{np}
\begin{définition}\fra réponse\end{définition}
\begin{définition}\cmn 答案\end{définition}
\begin{exemple}\jya ɯʑo kɯ tɤrkoz ʑo a-lɤn mɯ-tu-βze ɲɯ-ŋu\cmn 他故意不给我回音(不理我)\end{exemple}
\begin{relation-sémantique}\confer{
\hyperlink{Ⓔɣɯlɤn}{\textit{ \papi{ɣɯlɤn}}}
}\end{relation-sémantique}\end{entrée}

\begin{entrée}
\vedette{\hypertarget{Ⓔtɯ-lɤt}{\papi{ tɯ-lɤt}}}\markboth{tɯ-lɤt}{}\classe{np}
\begin{définition}\fra puîné\end{définition}
\begin{définition}\cmn 老二(兄弟姐妹)\end{définition}\end{entrée}

\begin{entrée}
\vedette{\hypertarget{Ⓔtɯ-lchɯɣ}{\papi{ tɯ-lchɯɣ}}}\markboth{tɯ-lchɯɣ}{}\classe{clf}
\begin{définition}\fra section (d'un sac, d'un récipient)\end{définition}
\begin{définition}\cmn 一节(口袋、容器)\end{définition}\end{entrée}

\begin{entrée}
\vedette{\hypertarget{Ⓔtɯ-ldʑa}{\papi{ tɯ-ldʑa}}}\markboth{tɯ-ldʑa}{}\classe{clf}
\begin{définition}\fra un brin\end{définition}
\begin{définition}\cmn 一根;一条\end{définition}
\begin{exemple}\jya tɯ-mbri tɯ-ldʑa\cmn 一根绳子\end{exemple}
\begin{exemple}\jya tɤ-ri tɯ-ldʑa\cmn 一条线\end{exemple}
\begin{exemple}\jya sɯjno tɯ-ldʑa\cmn 一根草\end{exemple}
\begin{exemple}\jya tɯ-ci tɯ-ldʑa\cmn 一条河\end{exemple}
\begin{exemple}\jya qapri tɯ-ldʑa\cmn 一条蛇\end{exemple}
\begin{exemple}\jya qaɟy tɯ-ldʑa\cmn 一条鱼\end{exemple}
\begin{exemple}\jya ɯ-mi tɯ-ldʑa\cmn 一只脚\end{exemple}
\end{entrée}

\begin{entrée}
\vedette{\hypertarget{Ⓔtɯ-lpɤɣ}{\papi{ tɯ-lpɤɣ}}}\markboth{tɯ-lpɤɣ}{}\classe{clf}
\begin{définition}\fra gros morceau\end{définition}
\begin{définition}\cmn 一大块\end{définition}
\begin{exemple}\jya tɤ-rcoʁ tɯ-lpɤɣ\cmn 一滩泥\end{exemple}
\begin{exemple}\jya smɤɣ tɯ-lpɤɣ\cmn 一团一团的羊毛\end{exemple}
\begin{exemple}\jya tɯ-mtɕhi ʁnɯ-lpɤɣ kɤ-sɤmɯrpu ʁo mbat ɕti\cmn 动嘴唇倒很容易\end{exemple}
\begin{exemple}\jya nɤki nɯ ɣɯ ʁnɯ-lpɤɣ nɯ χo, nɯ ma ɯ-kɤ-spa me\cmn 那个人只会动嘴,没有什么本事\end{exemple}\end{entrée}

\begin{entrée}
\vedette{\hypertarget{Ⓔtɯ-ltɕhɯz}{\papi{ tɯ-ltɕhɯz}}}\markboth{tɯ-ltɕhɯz}{}\classe{clf}
\begin{définition}\fra une touffe\end{définition}
\begin{définition}\cmn 一绺(头发)\end{définition}
\begin{exemple}\jya tɯ-kɤrme tɯ-ltɕhɯz\cmn 一绺头发\end{exemple}\end{entrée}

\begin{entrée}
\vedette{\hypertarget{Ⓔtɯ-lɯm}{\papi{ tɯ-lɯm}}}\markboth{tɯ-lɯm}{}
\classe{np}
\begin{définition}\fra dimension\end{définition}
\begin{définition}\cmn 体积\end{définition}
\begin{exemple}\jya tɤ-fkɯm ɣɯ tɯjpu chɯ́-wɣ-rku tɕe, kɯdɤn chɯ́-wɣ-rku tɕe, ɯ-lɯm wxti, kɯrkɯn chɯ́-wɣ-rku tɕe ɯ-lɯm xtɕi\cmn 口袋里的粮食装的多,(体积)就撑得大,装得少,(体积)就撑不大\end{exemple}
\begin{exemple}\jya nɤki tɯrme nɯ ɯ-lɯm wxti mɤ-wxti maŋe ma ɯ-sni kɯ-xtɕɯ-xtɕi ɲɯ-ɕti\cmn 这个人体积大不大都没有用,胆子很小\end{exemple}
\begin{exemple}\jya nɤki kha ɯ-lɯm ndɤre ɲɯ-wxti ri, ɯ-ŋgɯ ra ku-nɯ-pe mɯ-ku-nɯ-pe mɤ-xsi\cmn 这个房子看起来体积很大,不知里面好还是不好\end{exemple}
\begin{relation-sémantique}\confer{
\hyperlink{Ⓔmɤlɯm}{\textit{ \papi{mɤlɯm}}}
}\end{relation-sémantique}\end{entrée}

\begin{entrée}
\vedette{\hypertarget{Ⓔtɯ-lɯz}{\papi{ tɯ-lɯz}}}\markboth{tɯ-lɯz}{}
\classe{np}
\begin{définition}\fra âge\end{définition}
\begin{définition}\cmn 年龄\end{définition}\begin{sous-entrée}
\vedette{\hypertarget{}{\papi{ tɯ-lɯz,ɣi}}}\markboth{tɯ-lɯz,ɣi}{}
\begin{définition}\fra prendre de l'âge\end{définition}
\begin{définition}\cmn 上了年纪\end{définition}
\begin{exemple}\jya a-wi ɯ-lɯz thɯ-ɣe\cmn 我奶奶年龄大了\end{exemple}
\begin{exemple}\jya a-mu ɯ-lɯz thɯ-ɣe\cmn 我母亲年龄大了\end{exemple}
\end{sous-entrée}\end{entrée}

\begin{entrée}
\vedette{\hypertarget{Ⓔtɯ-ɬɯm}{\papi{ tɯ-ɬɯm}}}\markboth{tɯ-ɬɯm}{}\classe{clf}
\begin{définition}\fra une période de sommeil\end{définition}
\begin{définition}\cmn (睡)一觉\end{définition}
\begin{exemple}\jya tɯ-ɬɯm pɯ-nɯʑɯβ-a\cmn 我睡一觉\end{exemple}
\begin{exemple}\jya a-ʑɯβ ci tɯ-ɬɯm pɯ-ɣe\cmn 我睡一觉\end{exemple}\end{entrée}

\begin{entrée}
\vedette{\hypertarget{Ⓔtɯmu}{\papi{ tɯmu}}}\markboth{tɯmu}{}
\classe{n}
\begin{définition}\fra peur\end{définition}
\begin{définition}\cmn 害怕\end{définition}
\begin{exemple}\jya tɯmu kɯ ɯ-lu ʑo pjɤ-cɯ\cmn 他被吓得魂飞魄散\end{exemple}\end{entrée}

\begin{entrée}
\vedette{\hypertarget{Ⓔtɯ-mɤkhɤxtu}{\papi{ tɯ-mɤkhɤxtu}}}\markboth{tɯ-mɤkhɤxtu}{}\classe{np}
\begin{définition}\fra partie supérieure du pied\end{définition}
\begin{définition}\cmn 脚背\end{définition}
\end{entrée}

\begin{entrée}
\vedette{\hypertarget{Ⓔtɯ-mɤlɤjaʁ}{\papi{ tɯ-mɤlɤjaʁ}}}\markboth{tɯ-mɤlɤjaʁ}{}\classe{np}
\begin{définition}\fra quatre membres\end{définition}
\begin{définition}\cmn 四肢,手脚\end{définition}
\begin{relation-sémantique}\confer{
\hyperlink{Ⓔtɯ-kɤlɤmɲaʁ}{\textit{ \papi{tɯ-kɤlɤmɲaʁ}}}
}\end{relation-sémantique}
\end{entrée}

\begin{entrée}
\vedette{\hypertarget{Ⓔtɯ-mɤmu}{\papi{ tɯ-mɤmu}}}\markboth{tɯ-mɤmu}{}\classe{np}
\begin{définition}\fra gros orteil\end{définition}
\begin{définition}\cmn 大脚趾\end{définition}\end{entrée}

\begin{entrée}
\vedette{\hypertarget{Ⓔtɯ-mɤmke}{\papi{ tɯ-mɤmke}}}\markboth{tɯ-mɤmke}{}\classe{np}
\begin{définition}\fra partie de la jambe entre le mollet et la cheville\end{définition}
\begin{définition}\cmn 小腿和踝骨节相连的部分\end{définition}
\end{entrée}

\begin{entrée}
\vedette{\hypertarget{Ⓔtɯ-mɤmɲaʁ}{\papi{ tɯ-mɤmɲaʁ}}}\markboth{tɯ-mɤmɲaʁ}{}\classe{np}
\begin{définition}\fra astragale\end{définition}
\begin{définition}\cmn 距骨\end{définition}
\end{entrée}

\begin{entrée}
\vedette{\hypertarget{Ⓔtɯ-mɤndzu}{\papi{ tɯ-mɤndzu}}}\markboth{tɯ-mɤndzu}{}\classe{np}
\begin{définition}\fra doigts de pied\end{définition}
\begin{définition}\cmn 脚趾\end{définition}
\begin{relation-sémantique}\confer{
\hyperlink{Ⓔtɯ-mi}{\textit{ \papi{tɯ-mi}}}
}\end{relation-sémantique}\end{entrée}

\begin{entrée}
\vedette{\hypertarget{Ⓔtɯ-mɤndzoʁ}{\papi{ tɯ-mɤndzoʁ}}}\markboth{tɯ-mɤndzoʁ}{}\classe{np}
\begin{définition}\fra orteil\end{définition}
\begin{définition}\cmn 脚趾\end{définition}\end{entrée}

\begin{entrée}
\vedette{\hypertarget{Ⓔtɯ-mɤndzrɯ}{\papi{ tɯ-mɤndzrɯ}}}\markboth{tɯ-mɤndzrɯ}{}\classe{np}
\begin{définition}\fra griffes\end{définition}
\begin{définition}\cmn 爪子\end{définition}
\end{entrée}

\begin{entrée}
\vedette{\hypertarget{Ⓔtɯ-mɤŋɤn}{\papi{ tɯ-mɤŋɤn}}}\markboth{tɯ-mɤŋɤn}{}\classe{np}
\begin{définition}\fra petit orteil\end{définition}
\begin{définition}\cmn 小脚趾\end{définition}\end{entrée}

\begin{entrée}
\vedette{\hypertarget{Ⓔtɯ-mɤpu}{\papi{ tɯ-mɤpu}}}\markboth{tɯ-mɤpu}{}\classe{np}
\begin{définition}\fra mollet\end{définition}
\begin{définition}\cmn 小腿\end{définition}
\end{entrée}

\begin{entrée}
\vedette{\hypertarget{Ⓔtɯ-mɤpa}{\papi{ tɯ-mɤpa}}}\markboth{tɯ-mɤpa}{}\classe{np}
\begin{définition}\fra plante du pied\end{définition}
\begin{définition}\cmn 脚底\end{définition}
\begin{relation-sémantique}\confer{
\hyperlink{Ⓔtɯ-mi}{\textit{ \papi{tɯ-mi}}}
}\end{relation-sémantique}\end{entrée}

\begin{entrée}
\vedette{\hypertarget{Ⓔtɯ-mɤpɤl}{\papi{ tɯ-mɤpɤl}}}\markboth{tɯ-mɤpɤl}{}\classe{np}
\begin{définition}\fra plante du pied\end{définition}
\begin{définition}\cmn 脚底
\begin{déclaration} \étymologie{\papi{sbar.mo}}\end{déclaration}\end{définition}
\end{entrée}

\begin{entrée}
\vedette{\hypertarget{Ⓔtɯ-mɤru}{\papi{ tɯ-mɤru}}}\markboth{tɯ-mɤru}{}\classe{np}
\begin{définition}\fra trace de pieds\end{définition}
\begin{définition}\cmn 脚印\end{définition}
\begin{relation-sémantique}\confer{
\hyperlink{Ⓔtɯ-mi}{\textit{ \papi{tɯ-mi}}}
}\end{relation-sémantique}\end{entrée}

\begin{entrée}
\vedette{\hypertarget{Ⓔtɯ-mɤsɯmsɯm}{\papi{ tɯ-mɤsɯmsɯm}}}\markboth{tɯ-mɤsɯmsɯm}{}\classe{np}
\begin{définition}\fra talon\end{définition}
\begin{définition}\cmn 脚跟\end{définition}
\end{entrée}

\begin{entrée}
\vedette{\hypertarget{Ⓔtɯ-mɤtɕɤŋoʁ}{\papi{ tɯ-mɤtɕɤŋoʁ}}}\markboth{tɯ-mɤtɕɤŋoʁ}{}\classe{np}
\begin{définition}\fra creux du genou\end{définition}
\begin{définition}\cmn 膝盖后面;腘\end{définition}\end{entrée}

\begin{entrée}
\vedette{\hypertarget{Ⓔtɯmbar}{\papi{ tɯmbar}}}\markboth{tɯmbar}{}\classe{n}
\begin{définition}\fra ventre de bovidé\end{définition}
\begin{définition}\cmn 牛肚子\end{définition}
\end{entrée}

\begin{entrée}
\vedette{\hypertarget{Ⓔtɯmbaz}{\papi{ tɯmbaz}}}\markboth{tɯmbaz}{}\classe{n}
\begin{définition}\fra coulée de boue\end{définition}
\begin{définition}\cmn 泥石流\end{définition}
\begin{exemple}\jya tɯmbaz chɤ-ɣi\cmn 发生了泥石流\end{exemple}\end{entrée}

\begin{entrée}
\vedette{\hypertarget{Ⓔtɯ-mbɤtɯm}{\papi{ tɯ-mbɤtɯm}}}\markboth{tɯ-mbɤtɯm}{}\classe{np}
\begin{définition}\fra rein\end{définition}
\begin{définition}\cmn 肾\end{définition}
\end{entrée}

\begin{entrée}
\vedette{\hypertarget{Ⓔtɯ-mbur}{\papi{ tɯ-mbur}}}\markboth{tɯ-mbur}{}\classe{np}
\begin{définition}\fra sur les cuisses (lorsqu'on est assis en tailleur)\end{définition}
\begin{définition}\cmn 盘着坐时,腿上的部位\end{définition}
\begin{relation-sémantique}\synonyme{
\hyperlink{Ⓔtɯ-rpɣo}{\textit{ \papi{tɯ-rpɣo}}}
}\end{relation-sémantique}\end{entrée}

\begin{entrée}
\vedette{\hypertarget{Ⓔtɯmbri}{\papi{ tɯmbri}}}\markboth{tɯmbri}{}\classe{n}
\begin{définition}\fra corde\end{définition}
\begin{définition}\cmn 绳子
\begin{déclaration} \étymologie{\papi{ⁿbreŋ}}\end{déclaration}\end{définition}
\begin{relation-sémantique}\confer{
\hyperlink{Ⓔnɯmbrɯmtsaʁ}{\textit{ \papi{nɯmbrɯmtsaʁ}}}
}\end{relation-sémantique}\end{entrée}

\begin{entrée}
\vedette{\hypertarget{Ⓔtɯ-mbɯ}{\papi{ tɯ-mbɯ}}}\markboth{tɯ-mbɯ}{}\classe{np}
\begin{définition}\fra organe sexuel masculin\end{définition}
\begin{définition}\cmn 男生殖器\end{définition}
\end{entrée}

\begin{entrée}
\vedette{\hypertarget{Ⓔtɯ-mchi}{\papi{ tɯ-mchi}}}\markboth{tɯ-mchi}{}\classe{np}
\begin{définition}\fra bile\end{définition}
\begin{définition}\cmn 胆(动物)\end{définition}
\begin{relation-sémantique}\confer{
\hyperlink{Ⓔprɤmchi}{\textit{ \papi{prɤmchi}}}
}\end{relation-sémantique}
\end{entrée}

\begin{entrée}
\vedette{\hypertarget{Ⓔtɯ-mci}{\papi{ tɯ-mci}}}\markboth{tɯ-mci}{}\classe{np}
\begin{définition}\fra salive\end{définition}
\begin{définition}\cmn 口水\end{définition}
\begin{exemple}\jya tɯ-mci thɯ́-wɣ-βde ɯ-qhu kɤ-nɯɕɣɤz mɤ-khɯ\cmn 吐出了口水不能收回去(你送出的礼物不能拿回来)\end{exemple}
\begin{relation-sémantique}\confer{
\hyperlink{Ⓔmcɯrɯβrɯβ}{\textit{ \papi{mcɯrɯβrɯβ}}}
}\end{relation-sémantique}
\begin{relation-sémantique}\confer{
\hyperlink{Ⓔmcɯphɯt}{\textit{ \papi{mcɯphɯt}}}
}\end{relation-sémantique}\end{entrée}

\begin{entrée}
\vedette{\hypertarget{Ⓔtɯmda}{\papi{ tɯmda}}}\markboth{tɯmda}{}\classe{n}
\begin{définition}\fra fusils traditionnels\end{définition}
\begin{définition}\cmn 土枪
\begin{déclaration} \étymologie{\papi{mda}}\end{déclaration}\end{définition}
\end{entrée}

\begin{entrée}
\vedette{\hypertarget{Ⓔtɯmdi}{\papi{ tɯmdi}}}\markboth{tɯmdi}{}\classe{adv}
\begin{définition}\fra tout le monde\end{définition}
\begin{définition}\cmn 大家\end{définition}\end{entrée}

\begin{entrée}
\vedette{\hypertarget{Ⓔtɯ-mdzɤɣ}{\papi{ tɯ-mdzɤɣ}}}\markboth{tɯ-mdzɤɣ}{}\classe{np}
\begin{définition}\fra clitoris\end{définition}
\begin{définition}\cmn 阴蒂\end{définition}\end{entrée}

\begin{entrée}
\vedette{\hypertarget{Ⓔtɯmdzoz}{\papi{ tɯmdzoz}}}\markboth{tɯmdzoz}{}\classe{n}
\begin{définition}\fra interdit, tabou\end{définition}
\begin{définition}\cmn 忌讳\end{définition}
\begin{relation-sémantique}\confer{
\hyperlink{Ⓔmdzoz}{\textit{ \papi{mdzoz}}}
}\end{relation-sémantique}
\end{entrée}

\begin{entrée}
\vedette{\hypertarget{Ⓔtɯ-mdzɯt}{\papi{ tɯ-mdzɯt}}}\markboth{tɯ-mdzɯt}{}\classe{np}
\begin{définition}\fra ordre\end{définition}
\begin{définition}\cmn 命令\end{définition}
\begin{exemple}\jya rɟɤlpu nɯ kɯ ɯ-pa ra ɣɯ nɯ-mdzɯt pjɤ-lɤt\cmn 土司给他的属下下来命令\end{exemple}\end{entrée}

\begin{entrée}
\vedette{\hypertarget{Ⓔtɯ-mdʑu}{\papi{ tɯ-mdʑu}}}\markboth{tɯ-mdʑu}{}\classe{np}
\begin{définition}\fra language\end{définition}
\begin{définition}\cmn 舌头\end{définition}
\begin{exemple}\jya tɕhoma ɯ-mdʑu\cmn 皮带的尖头\end{exemple}\end{entrée}

\begin{entrée}
\vedette{\hypertarget{Ⓔtɯ-mdʑuri}{\papi{ tɯ-mdʑuri}}}\markboth{tɯ-mdʑuri}{}\classe{np}
\begin{définition}\fra tendon de la langue\end{définition}
\begin{définition}\cmn 舌头的筋\end{définition}
\end{entrée}

\begin{entrée}
\vedette{\hypertarget{Ⓔtɯ-me}{\papi{ tɯ-me}}}\markboth{tɯ-me}{}\classe{np}
\begin{définition}\fra fille\end{définition}
\begin{définition}\cmn 女儿\end{définition}
\end{entrée}

\begin{entrée}
\vedette{\hypertarget{Ⓔtɯ-menmaʁ}{\papi{ tɯ-menmaʁ}}}\markboth{tɯ-menmaʁ}{}\classe{np}
\begin{définition}\fra beau-fils\end{définition}
\begin{définition}\cmn 女婿
\begin{déclaration} \étymologie{\papi{mag.pa}}\end{déclaration}\end{définition}
\end{entrée}

\begin{entrée}
\vedette{\hypertarget{Ⓔtɯ-mga}{\papi{ tɯ-mga}}}\markboth{tɯ-mga}{}\classe{np}
\begin{définition}\fra ce que l'on obtient\end{définition}
\begin{définition}\cmn 收获\end{définition}
\begin{exemple}\jya kɯmɤlɤxso pjɯ-tɯ-zdɯɣ ɲɯ-ɕti ma nɤ-mga me\cmn 你白白辛苦了,没有得到任何好处\end{exemple}
\begin{relation-sémantique}\synonyme{
\hyperlink{Ⓔɯ-mbrɤzɯ}{\textit{ \papi{ɯ-mbrɤzɯ}}}
}\end{relation-sémantique}
\begin{relation-sémantique}\confer{
\hyperlink{Ⓔnɯmga}{\textit{ \papi{nɯmga}}}
}\end{relation-sémantique}\end{entrée}

\begin{entrée}
\vedette{\hypertarget{Ⓔtɯ-mgo}{\papi{ tɯ-mgo}}}\markboth{tɯ-mgo}{}\classe{np}
\begin{définition}\fra nourriture\end{définition}
\begin{définition}\cmn 粮食\end{définition}
\end{entrée}

\begin{entrée}
\vedette{\hypertarget{Ⓔtɯmgozmɤrɤβ}{\papi{ tɯmgozmɤrɤβ}}}\markboth{tɯmgozmɤrɤβ}{}\classe{n}
\begin{définition}\fra plat\end{définition}
\begin{définition}\cmn 菜\end{définition}
\end{entrée}

\begin{entrée}
\vedette{\hypertarget{Ⓔtɯmgrɯnphoŋ}{\papi{ tɯmgrɯnphoŋ}}}\markboth{tɯmgrɯnphoŋ}{}\classe{n}
\begin{définition}\fra bouteille d'alcool pour offrir aux hôtes\end{définition}
\begin{définition}\cmn 招待客人的酒瓶\end{définition}\end{entrée}

\begin{entrée}
\vedette{\hypertarget{Ⓔtɯ-mgɯr}{\papi{ tɯ-mgɯr}}}\markboth{tɯ-mgɯr}{}\classe{np}
\begin{définition}\fra dos\end{définition}
\begin{définition}\cmn 背\end{définition}
\end{entrée}

\begin{entrée}
\vedette{\hypertarget{Ⓔtɯ-mi}{\papi{ tɯ-mi}}}\markboth{tɯ-mi}{}\classe{np}
\begin{définition}\fra jambe; pied\end{définition}
\begin{définition}\cmn 脚\end{définition}
\begin{exemple}\jya tɯ-mi ɯ-pɤl\cmn 脚掌\end{exemple}
\begin{relation-sémantique}\confer{
\hyperlink{Ⓔtɯ-mɤndzu}{\textit{ \papi{tɯ-mɤndzu}}}
}\end{relation-sémantique}
\begin{relation-sémantique}\confer{
\hyperlink{Ⓔamɤʁu}{\textit{ \papi{amɤʁu}}}
}\end{relation-sémantique}\end{entrée}

\begin{entrée}
\vedette{\hypertarget{Ⓔtɯ-midi}{\papi{ tɯ-midi}}}\markboth{tɯ-midi}{}\classe{np}
\begin{définition}\fra odeur de pied\end{définition}
\begin{définition}\cmn 脚臭\end{définition}
\begin{exemple}\jya nɤ-midi ɯ-tɯ-mnɤm nɯ\cmn 你的脚很臭啊\end{exemple}\end{entrée}

\begin{entrée}
\vedette{\hypertarget{Ⓔtɯ-mɟaⒽ1}{\papi{ tɯ-mɟa}}}\markboth{tɯ-mɟa}{}\homonyme{1}
\classe{np}
\begin{définition}\fra mâchoire inférieure\end{définition}
\begin{définition}\cmn 下巴\end{définition}
\end{entrée}

\begin{entrée}
\vedette{\hypertarget{Ⓔtɯ-mɟaⒽ2}{\papi{ tɯ-mɟa}}}\markboth{tɯ-mɟa}{}\homonyme{2}
\classe{np}
\begin{définition}\fra avantage, ce que l'on reçoit\end{définition}
\begin{définition}\cmn 收入\end{définition}
\begin{relation-sémantique}\confer{
\hyperlink{Ⓔmɟa}{\textit{ \papi{mɟa}}}
}\end{relation-sémantique}\end{entrée}

\begin{entrée}
\vedette{\hypertarget{Ⓔtɯ-mɟɤrme}{\papi{ tɯ-mɟɤrme}}}\markboth{tɯ-mɟɤrme}{}\classe{np}
\begin{définition}\fra barbe\end{définition}
\begin{définition}\cmn 胡子\end{définition}
\begin{relation-sémantique}\confer{
\hyperlink{Ⓔtɤ-rme}{\textit{ \papi{tɤ-rme}}}
}\end{relation-sémantique}
\end{entrée}

\begin{entrée}
\vedette{\hypertarget{Ⓔtɯ-mkɤqhu}{\papi{ tɯ-mkɤqhu}}}\markboth{tɯ-mkɤqhu}{}\classe{np}
\begin{définition}\fra nuque\end{définition}
\begin{définition}\cmn 项,颈背【脑后勺】\end{définition}
\begin{relation-sémantique}\confer{
\hyperlink{Ⓔɯ-qhu}{\textit{ \papi{ɯ-qhu}}}
}\end{relation-sémantique}
\end{entrée}

\begin{entrée}
\vedette{\hypertarget{Ⓔtɯ-mkɤscur}{\papi{ tɯ-mkɤscur}}}\markboth{tɯ-mkɤscur}{}
\classe{np}
\begin{définition}\fra col\end{définition}
\begin{définition}\cmn 衣领\end{définition}
\begin{relation-sémantique}\confer{
\hyperlink{Ⓔtɯ-mke}{\textit{ \papi{tɯ-mke}}}
}\end{relation-sémantique}
\begin{relation-sémantique}\synonyme{
\hyperlink{Ⓔtɯ-kuŋa}{\textit{ \papi{tɯ-kuŋa}}}
}\end{relation-sémantique}\end{entrée}

\begin{entrée}
\vedette{\hypertarget{Ⓔtɯ-mke}{\papi{ tɯ-mke}}}\markboth{tɯ-mke}{}\classe{np}
\begin{définition}\fra cou\end{définition}
\begin{définition}\cmn 脖子
\begin{déclaration} \étymologie{\papi{ske}}\end{déclaration}\end{définition}
\end{entrée}

\begin{entrée}
\vedette{\hypertarget{Ⓔtɯ-mɢla}{\papi{ tɯ-mɢla}}}\markboth{tɯ-mɢla}{}\classe{clf}
\begin{définition}\fra pas\end{définition}
\begin{définition}\cmn 一步\end{définition}
\begin{exemple}\jya tɯ-mɢla kɤ-scɤt ɯ-tɤ́-cha\cmn (你儿子)会走路了吗?\end{exemple}
\begin{exemple}\jya tɯ-mɢla cinɤ ma-jɤ-tɯ-te!\cmn 一步都不要迈进!\end{exemple}
\begin{exemple}\jya tɯ-mɢla ju-cit ŋu\cmn (小孩子会)走路\end{exemple}\end{entrée}

\begin{entrée}
\vedette{\hypertarget{Ⓔtɯmnɯ}{\papi{ tɯmnɯ}}}\markboth{tɯmnɯ}{}\classe{n}
\begin{définition}\fra alène\end{définition}
\begin{définition}\cmn 锥\end{définition}
\end{entrée}

\begin{entrée}
\vedette{\hypertarget{Ⓔtɯmɲa}{\papi{ tɯmɲa}}}\markboth{tɯmɲa}{}\classe{n}
\begin{définition}\fra flèche\end{définition}
\begin{définition}\cmn 箭\end{définition}
\end{entrée}

\begin{entrée}
\vedette{\hypertarget{Ⓔtɯ-mɲaʁ}{\papi{ tɯ-mɲaʁ}}}\markboth{tɯ-mɲaʁ}{}\classe{np}
\begin{définition}\fra œil\end{définition}
\begin{définition}\cmn 眼睛\end{définition}
\end{entrée}

\begin{entrée}
\vedette{\hypertarget{Ⓔtɯ-mɲaʁfkaβ}{\papi{ tɯ-mɲaʁfkaβ}}}\markboth{tɯ-mɲaʁfkaβ}{}\classe{np}
\begin{définition}\fra paupière supérieure\end{définition}
\begin{définition}\cmn 上眼皮\end{définition}
\begin{relation-sémantique}\confer{
\hyperlink{Ⓔfkaβ}{\textit{ \papi{fkaβ}}}
}\end{relation-sémantique}
\end{entrée}

\begin{entrée}
\vedette{\hypertarget{Ⓔtɯ-mɲaʁndo}{\papi{ tɯ-mɲaʁndo}}}\markboth{tɯ-mɲaʁndo}{}\classe{np}
\begin{définition}\fra coin du l'œil\end{définition}
\begin{définition}\cmn 眼边\end{définition}
\begin{relation-sémantique}\confer{
\hyperlink{Ⓔɯ-ndo}{\textit{ \papi{ɯ-ndo}}}
}\end{relation-sémantique}\end{entrée}

\begin{entrée}
\vedette{\hypertarget{Ⓔtɯ-mɲaʁrdu}{\papi{ tɯ-mɲaʁrdu}}}\markboth{tɯ-mɲaʁrdu}{}\classe{np}
\begin{définition}\fra prunelle\end{définition}
\begin{définition}\cmn 眼珠\end{définition}
\end{entrée}

\begin{entrée}
\vedette{\hypertarget{Ⓔtɯ-mɲaʁrme}{\papi{ tɯ-mɲaʁrme}}}\markboth{tɯ-mɲaʁrme}{}\classe{np}
\begin{définition}\fra sourcil\end{définition}
\begin{définition}\cmn 眉毛\end{définition}
\end{entrée}

\begin{entrée}
\vedette{\hypertarget{Ⓔtɯ-mɲaʁspɯ}{\papi{ tɯ-mɲaʁspɯ}}}\markboth{tɯ-mɲaʁspɯ}{}\classe{np}
\begin{définition}\fra chassie\end{définition}
\begin{définition}\cmn 眼屎\end{définition}
\begin{relation-sémantique}\confer{
\hyperlink{Ⓔtɤ-spɯ}{\textit{ \papi{tɤ-spɯ}}}
}\end{relation-sémantique}
\end{entrée}

\begin{entrée}
\vedette{\hypertarget{Ⓔtɯmɲɯɣ}{\papi{ tɯmɲɯɣ}}}\markboth{tɯmɲɯɣ}{}\classe{n}
\begin{définition}\fra cancer de l'estomac\end{définition}
\begin{définition}\cmn 胃癌\end{définition}
\begin{exemple}\jya lɯlu ɯ-rme chɯ́-wɣ-ndza tɕe, tɯmɲɯɣ βze\cmn 吞了猫的毛就会得胃癌。\end{exemple}
\begin{relation-sémantique}\confer{
\hyperlink{Ⓔnɯmɲɯɣ}{\textit{ \papi{nɯmɲɯɣ}}}
}\end{relation-sémantique}\end{entrée}

\begin{entrée}
\vedette{\hypertarget{Ⓔtɯ-mɲɯɣ}{\papi{ tɯ-mɲɯɣ}}}\markboth{tɯ-mɲɯɣ}{}\classe{np}
\begin{définition}\fra œsophage\end{définition}
\begin{définition}\cmn 食道
\begin{déclaration} \étymologie{\papi{mid}}\end{déclaration}\end{définition}
\begin{relation-sémantique}\confer{
\hyperlink{Ⓔnɯmɲɯɣ}{\textit{ \papi{nɯmɲɯɣ}}}
}\end{relation-sémantique}\end{entrée}

\begin{entrée}
\vedette{\hypertarget{Ⓔtɯ-mɲɯtsi}{\papi{ tɯ-mɲɯtsi}}}\markboth{tɯ-mɲɯtsi}{}\classe{clf}
\begin{définition}\fra vie\end{définition}
\begin{définition}\cmn 一辈子
\begin{déclaration} \étymologie{\papi{mi.tsʰe}}\end{déclaration}\end{définition}
\begin{exemple}\jya kɯki kha ki χsɯ-tɯmɲɯtsi to-tsu (=chɤ-mda)\cmn 这座房子已经住过三代人\end{exemple}\end{entrée}

\begin{entrée}
\vedette{\hypertarget{Ⓔtɯmŋu}{\papi{ tɯmŋu}}}\markboth{tɯmŋu}{}\classe{n}
\begin{définition}\ 
\begin{déclaration}\grammar{n.lieu}\end{déclaration}
\begin{déclaration}\grammar{n.lieu}\end{déclaration}\end{définition}
\begin{définition}\fra nom commun à plusieurs champs à Kamnyu\end{définition}
\begin{définition}\cmn 干木鸟村几块田地的统称\end{définition}\end{entrée}

\begin{entrée}
\vedette{\hypertarget{Ⓔtɯ-mpɕar}{\papi{ tɯ-mpɕar}}}\markboth{tɯ-mpɕar}{}\classe{clf}
\begin{définition}\fra feuille, un yuan\end{définition}
\begin{définition}\cmn 一张;一元\end{définition}
\begin{exemple}\jya sɯjwaʁ tɯ-mpɕar\cmn 一片叶子\end{exemple}
\begin{exemple}\jya jɯɣi tɯ-mpɕar\cmn 一张纸\end{exemple}\end{entrée}

\begin{entrée}
\vedette{\hypertarget{Ⓔtɯ-mphɯr}{\papi{ tɯ-mphɯr}}}\markboth{tɯ-mphɯr}{}\classe{clf}
\begin{définition}\fra rouleau, paquet\end{définition}
\begin{définition}\cmn 一包;一卷\end{définition}
\begin{exemple}\jya raz tɯ-mphɯr\cmn 一卷布\end{exemple}
\end{entrée}

\begin{entrée}
\vedette{\hypertarget{Ⓔtɯ-mphɯz}{\papi{ tɯ-mphɯz}}}\markboth{tɯ-mphɯz}{}\classe{np}
\begin{définition}\fra fesse\end{définition}
\begin{définition}\cmn 屁股\end{définition}
\end{entrée}

\begin{entrée}
\vedette{\hypertarget{Ⓔtɯ-mqaj}{\papi{ tɯ-mqaj}}}\markboth{tɯ-mqaj}{}\classe{np}
\begin{définition}\fra critique\end{définition}
\begin{définition}\cmn 批评\end{définition}
\begin{exemple}\jya a-mqaj pɯ-tu\cmn 我被批评了\end{exemple}\end{entrée}

\begin{entrée}
\vedette{\hypertarget{Ⓔtɯ-mtɕhi}{\papi{ tɯ-mtɕhi}}}\markboth{tɯ-mtɕhi}{}
\classe{np}\acception{1}
\begin{définition}\fra lèvres\end{définition}
\begin{définition}\cmn 嘴唇\end{définition}\acception{2}
\begin{définition}\fra bouche\end{définition}
\begin{définition}\cmn 嘴
\begin{déclaration} \étymologie{\papi{mtɕʰu}}\end{déclaration}\end{définition}
\begin{exemple}\jya nɤ-mtɕhi kɤ-ndɤm\cmn 你闭嘴\end{exemple}
\begin{exemple}\jya cha ɯ-kɯ-tshi ɯ-mtɕhi kɯ-nɯʑɯβ ɯ-qe\cmn 喝了酒的人的嘴,瞌睡的人的屁(酒话是心理话)\end{exemple}
\begin{exemple}\jya nɤ-mtɕhi mɤ-mpɕɤr\cmn 你讲话很不客气(没有甜言蜜语)\end{exemple}
\begin{relation-sémantique}\confer{
\hyperlink{Ⓔtɯ-mtɕhi,χo}{\textit{ \papi{tɯ-mtɕhi,χo}}}
}\end{relation-sémantique}\end{entrée}

\begin{entrée}
\vedette{\hypertarget{Ⓔtɯ-mtɕhirme}{\papi{ tɯ-mtɕhirme}}}\markboth{tɯ-mtɕhirme}{}\classe{np}
\begin{définition}\fra moustaches\end{définition}
\begin{définition}\cmn 胡子,胡须\end{définition}
\end{entrée}

\begin{entrée}
\vedette{\hypertarget{Ⓔtɯ-mtɕhi,χo}{\papi{ tɯ-mtɕhi,χo}}}\markboth{tɯ-mtɕhi,χo}{}\paradigme{\textit{dir :} \jya tɤ-}
\begin{définition}\fra fanfaronner\end{définition}
\begin{définition}\cmn 说大话\end{définition}
\begin{exemple}\jya ɯ-mtɕhi ɲɯ-χo\cmn 他爱说大话\end{exemple}
\begin{relation-sémantique}\ComponentA{\classe{np}
\hyperlink{Ⓔtɯ-mtɕhi}{\textit{ \papi{tɯ-mtɕhi}}}
}\end{relation-sémantique}
\begin{relation-sémantique}\ComponentB{\classe{vs}
 \papi{χo}
}\end{relation-sémantique}
\begin{sous-entrée}
\vedette{\hypertarget{}{\papi{ tɯ-mtɕhi,ɣɤχo}}}\markboth{tɯ-mtɕhi,ɣɤχo}{}
\begin{définition}\fra fanfaronner\end{définition}
\begin{définition}\cmn 说大话\end{définition}
\begin{exemple}\jya ɯ-mtɕhi ɲɯ-ɣɤχɤm\cmn 他在说大话\end{exemple}
\begin{exemple}\jya a-mtɕhi ku-ɣɤχam-a\cmn 我正在说大话\end{exemple}
\begin{relation-sémantique}\ComponentA{\classe{np}
\hyperlink{Ⓔtɯ-mtɕhi}{\textit{ \papi{tɯ-mtɕhi}}}
}\end{relation-sémantique}
\begin{relation-sémantique}\ComponentB{\classe{vt}
 \papi{ɣɤχo}
}\end{relation-sémantique}
\begin{relation-sémantique}\confer{
\hyperlink{Ⓔtɯ-mtɕhi}{\textit{ \papi{tɯ-mtɕhi}}}
}\end{relation-sémantique}
\end{sous-entrée}\end{entrée}

\begin{entrée}
\vedette{\hypertarget{Ⓔtɯmtɕhɯ}{\papi{ tɯmtɕhɯ}}}\markboth{tɯmtɕhɯ}{}\classe{n}
\begin{définition}\fra crachat rituel\end{définition}
\begin{définition}\cmn 念经时,一边吹一边吐口水,治病的方式
\begin{déclaration} \étymologie{\papi{mtɕʰu}}\end{déclaration}\end{définition}
\end{entrée}

\begin{entrée}
\vedette{\hypertarget{Ⓔtɯmtɕi}{\papi{ tɯmtɕi}}}\markboth{tɯmtɕi}{}\classe{n}
\begin{définition}\fra matin\end{définition}
\begin{définition}\cmn 早晨\end{définition}
\end{entrée}

\begin{entrée}
\vedette{\hypertarget{Ⓔtɯmtɕiβzɤrnaʁ}{\papi{ tɯmtɕiβzɤrnaʁ}}}\markboth{tɯmtɕiβzɤrnaʁ}{}\classe{n}
\begin{définition}\fra très tôt le matin\end{définition}
\begin{définition}\cmn 大清早\end{définition}
\begin{exemple}\jya tɯmtɕiβzɤrnaʁ ʑo tɤ-rɤru-a\cmn 我起得很早\end{exemple}\end{entrée}

\begin{entrée}
\vedette{\hypertarget{Ⓔtɯ-mtɕoʁ}{\papi{ tɯ-mtɕoʁ}}}\markboth{tɯ-mtɕoʁ}{}\classe{clf}
\begin{définition}\fra une pincée; une touffe\end{définition}
\begin{définition}\cmn 一撮\end{définition}\end{entrée}

\begin{entrée}
\vedette{\hypertarget{Ⓔtɯ-mthɤɣ}{\papi{ tɯ-mthɤɣ}}}\markboth{tɯ-mthɤɣ}{}\classe{np}
\begin{définition}\fra taille\end{définition}
\begin{définition}\cmn 腰\end{définition}
\begin{relation-sémantique}\confer{
\hyperlink{Ⓔtɯ-mthɤrɴɢɤβ}{\textit{ \papi{tɯ-mthɤrɴɢɤβ}}}
}\end{relation-sémantique}
\begin{relation-sémantique}\confer{
\hyperlink{Ⓔmthɯxtɕɤr}{\textit{ \papi{mthɯxtɕɤr}}}
}\end{relation-sémantique}\end{entrée}

\begin{entrée}
\vedette{\hypertarget{Ⓔtɯ-mthɤrɴɢɤβ}{\papi{ tɯ-mthɤrɴɢɤβ}}}\markboth{tɯ-mthɤrɴɢɤβ}{}\classe{n}
\begin{définition}\fra encolure du pantalon\end{définition}
\begin{définition}\cmn 裤兜\end{définition}
\begin{exemple}\jya ɯ-mthɤrɴɢɤβ chɤmdɤru chɯ-nɯrʁe\cmn 他把竹竿插在裤兜里\end{exemple}
\begin{exemple}\jya tɯɲcɣa ɯ-mthɤrɴɢɤβ to-rʁe\cmn 他把镰刀插在裤兜里\end{exemple}
\begin{relation-sémantique}\confer{
\hyperlink{Ⓔtɯ-mthɤɣ}{\textit{ \papi{tɯ-mthɤɣ}}}
}\end{relation-sémantique}
\begin{relation-sémantique}\confer{
\hyperlink{Ⓔrŋgɤβ}{\textit{ \papi{rŋgɤβ}}}
}\end{relation-sémantique}\end{entrée}

\begin{entrée}
\vedette{\hypertarget{Ⓔtɯ-mthɯ}{\papi{ tɯ-mthɯ}}}\markboth{tɯ-mthɯ}{}\classe{np}
\begin{définition}\fra bénéfice\end{définition}
\begin{définition}\cmn 赚到的钱(收入减去开支)\end{définition}
\begin{exemple}\jya tɯ-sla nɤ-mthɯ tɕhi jamar ɣɤʑu\cmn 你一个月能赚多少钱?\end{exemple}
\begin{exemple}\jya ɯ-mthɯ to-ndza\cmn 他赚了他的钱\end{exemple}
\begin{relation-sémantique}\confer{
\hyperlink{ⒺnɯmthɯⒽ1}{\textit{ \papi{nɯmthɯ}}}
}\end{relation-sémantique}\end{entrée}

\begin{entrée}
\vedette{\hypertarget{Ⓔtɯ-mtshi}{\papi{ tɯ-mtshi}}}\markboth{tɯ-mtshi}{}\classe{np}
\begin{définition}\fra foie\end{définition}
\begin{définition}\cmn 肝
\begin{déclaration} \étymologie{\papi{mtɕʰin.pa}}\end{déclaration}\end{définition}
\end{entrée}

\begin{entrée}
\vedette{\hypertarget{Ⓔtɯ-mtso}{\papi{ tɯ-mtso}}}\markboth{tɯ-mtso}{}
\classe{np}
\begin{définition}\fra repas du midi, goûter\end{définition}
\begin{définition}\cmn 带去山上吃的中午饭【路餐】、五点钟吃的那顿饭\end{définition}
\begin{exemple}\jya tɯ-mtsomthɯm\cmn 路餐里面的肉\end{exemple}\end{entrée}

\begin{entrée}
\vedette{\hypertarget{Ⓔtɯ-mtɯ}{\papi{ tɯ-mtɯ}}}\markboth{tɯ-mtɯ}{}\classe{np}
\begin{définition}\fra coiffure traditionnelle des hommes tibétains\end{définition}
\begin{définition}\cmn 藏族男人的传统发型\end{définition}
\begin{exemple}\jya ɯ-mtɯ ɲɤ-nɯ-sɯ-ta\cmn 他剃头的时候在头顶上留了一撮头发\end{exemple}\end{entrée}

\begin{entrée}
\vedette{\hypertarget{Ⓔtɯ-mɯ}{\papi{ tɯ-mɯ}}}\markboth{tɯ-mɯ}{}\classe{np}
\begin{définition}\fra temps, pluie\end{définition}
\begin{définition}\cmn 天气;雨\end{définition}
\begin{exemple}\jya tɯ-mɯ kɯ-ɤrŋi\cmn 青天(天宫)\end{exemple}
\begin{exemple}\jya tɯ-mɯ ɲɯ-ɤsɯ-lɤt\cmn 在下雨\end{exemple}
\begin{exemple}\jya tɯ-mɯ ko-lɤt (=jo-ɣɯt)\cmn 下雨了\end{exemple}
\begin{exemple}\jya tɯ-mɯ ci ci ku-lɤt, ci ci mɯ́j-lɤt\cmn 有时候下雨,有时候不下雨\end{exemple}
\begin{exemple}\jya tɯ-mɯ lɤt ɲɯ-ŋu rca\cmn 快要下雨了\end{exemple}
\begin{exemple}\jya tɯ-mɯ chɯ tu-ru\cmn 他仰着(睡)\end{exemple}
\begin{exemple}\jya jɯfɕɯr ji-mɯ pɯ-pe\cmn 我们昨天(遇到了)很好的天气\end{exemple}
\begin{exemple}\jya ɯ-ndzɯ mɤ-kɯ-sɤŋo ɯ-mɯ mbɯt\cmn 不听劝告的人没有好下场(他的天要垮下来)\end{exemple}
\begin{exemple}\jya tɯ-mɯ pɯ-pa-fkaβ ʑo ɕ-tɤ-khat-a\cmn 我走遍了天下\end{exemple}\end{entrée}

\begin{entrée}
\vedette{\hypertarget{Ⓔtɯmɯɕoʁ}{\papi{ tɯmɯɕoʁ}}}\markboth{tɯmɯɕoʁ}{}\classe{n}
\begin{définition}\fra sarrasin\end{définition}
\begin{définition}\cmn 荞麦\end{définition}
\end{entrée}

\begin{entrée}
\vedette{\hypertarget{Ⓔtɯmɯkɤmpɕi}{\papi{ tɯmɯkɤmpɕi}}}\markboth{tɯmɯkɤmpɕi}{} (\variante{tɯmɯ kɯmɤɕi}) \classe{n}
\begin{définition}\fra paradis\end{définition}
\begin{définition}\cmn 天堂\end{définition}
\end{entrée}

\begin{entrée}
\vedette{\hypertarget{Ⓔtɯ-mɯm}{\papi{ tɯ-mɯm}}}\markboth{tɯ-mɯm}{}\classe{clf}
\begin{définition}\fra une gorgée\end{définition}
\begin{définition}\cmn 一口(水)\end{définition}
\end{entrée}

\begin{entrée}
\vedette{\hypertarget{Ⓔtɯmɯpaʁ}{\papi{ tɯmɯpaʁ}}}\markboth{tɯmɯpaʁ}{}\classe{n}
\begin{définition}\fra limace\end{définition}
\begin{définition}\cmn 蛞蝓\end{définition}
\end{entrée}

\begin{entrée}
\vedette{\hypertarget{Ⓔtɯ-mɯrʁɯz}{\papi{ tɯ-mɯrʁɯz}}}\markboth{tɯ-mɯrʁɯz}{}\classe{clf}
\begin{définition}\fra coup de griffe\end{définition}
\begin{définition}\cmn (抓)一下\end{définition}
\begin{exemple}\jya lɯlu kɯ tɯ-mɯrʁɯz ci to-lɤt\cmn 猫抓了一下\end{exemple}
\begin{relation-sémantique}\confer{
\hyperlink{Ⓔmɯrʁɯz}{\textit{ \papi{mɯrʁɯz}}}
}\end{relation-sémantique}\end{entrée}

\begin{entrée}
\vedette{\hypertarget{Ⓔtɯmɯrtsɯɣ}{\papi{ tɯmɯrtsɯɣ}}}\markboth{tɯmɯrtsɯɣ}{}\classe{n}
\begin{définition}\fra pincer\end{définition}
\begin{définition}\cmn 捏,掐,拧
\begin{déclaration}\use{一般和\stylefv{lɤt}连用}\end{déclaration}\end{définition}
\begin{exemple}\jya a-rʑaβ kɯ tɯmɯrtsɯɣ ta-lɤt\cmn 我的妻子掐了我一下\end{exemple}
\begin{exemple}\jya nɤ-qe thɯ-tɯ-rɤrɕɯβ tɕe, kutɕu tɯrme kɯmŋu ɣɤʑu-j tɕe, tɯmɯrtsɯɣ kɯmŋu tu-lat-a ra\cmn 你放了屁,我们这里有五个人,所以我就要掐你五次\end{exemple}
\begin{relation-sémantique}\confer{
\hyperlink{Ⓔmɯrtsɯɣ}{\textit{ \papi{mɯrtsɯɣ}}}
}\end{relation-sémantique}\end{entrée}

\begin{entrée}
\vedette{\hypertarget{Ⓔtɯmɯʁrɯm}{\papi{ tɯmɯʁrɯm}}}\markboth{tɯmɯʁrɯm}{}\classe{n}
\begin{définition}\fra si haut qu'il cache le ciel\end{définition}
\begin{définition}\cmn 高得遮住蓝天\end{définition}
\begin{exemple}\jya kha nɯ tɯmɯʁrɯm ʑo to-zɣɯt\cmn 这个房子高得遮住蓝天\end{exemple}
\begin{relation-sémantique}\confer{
\hyperlink{Ⓔta-ʁrɯm}{\textit{ \papi{ta-ʁrɯm}}}
}\end{relation-sémantique}
\begin{relation-sémantique}\confer{
\hyperlink{Ⓔtɯ-mɯ}{\textit{ \papi{tɯ-mɯ}}}
}\end{relation-sémantique}\end{entrée}

\begin{entrée}
\vedette{\hypertarget{Ⓔtɯmɢɯt}{\papi{ tɯmɢɯt}}}\markboth{tɯmɢɯt}{}\classe{n}
\begin{définition}\fra poutre qui soutient le balcon\end{définition}
\begin{définition}\cmn 支撑走檐的梁\end{définition}
\begin{exemple}\jya jɤɣɤt ɯ-pa stukɤr ɯ-tshɤt ɕoŋtɕa nɯ tɯmɢɯt rmi\cmn 
走缘下面起大梁作用的木料叫\stylefv{tɯmɢɯt}
\end{exemple}\end{entrée}

\begin{entrée}
\vedette{\hypertarget{Ⓔtɯ-ndʐi}{\papi{ tɯ-ndʐi}}}\markboth{tɯ-ndʐi}{}\classe{np}
\begin{définition}\fra peau\end{définition}
\begin{définition}\cmn 皮肤\end{définition}
\begin{relation-sémantique}\confer{
\hyperlink{Ⓔcɤndʐi}{\textit{ \papi{cɤndʐi}}}
}\end{relation-sémantique}
\begin{relation-sémantique}\confer{
\hyperlink{Ⓔtshɤndʐi}{\textit{ \papi{tshɤndʐi}}}
}\end{relation-sémantique}
\begin{relation-sémantique}\confer{
\hyperlink{Ⓔqartshɤndʐi}{\textit{ \papi{qartshɤndʐi}}}
}\end{relation-sémantique}
\end{entrée}

\begin{entrée}
\vedette{\hypertarget{Ⓔtɯndʐiŋga}{\papi{ tɯndʐiŋga}}}\markboth{tɯndʐiŋga}{}\classe{n}
\begin{définition}\fra habit en peau\end{définition}
\begin{définition}\cmn 皮衣\end{définition}\end{entrée}

\begin{entrée}
\vedette{\hypertarget{Ⓔtɯ-ndzɤfkɯm}{\papi{ tɯ-ndzɤfkɯm}}}\markboth{tɯ-ndzɤfkɯm}{} (\variante{tɯ-ndzɤkɯm}) \classe{np}
\begin{définition}\fra estomac\end{définition}
\begin{définition}\cmn 胃\end{définition}\end{entrée}

\begin{entrée}
\vedette{\hypertarget{Ⓔtɯ-ndzɤŋgrɯm}{\papi{ tɯ-ndzɤŋgrɯm}}}\markboth{tɯ-ndzɤŋgrɯm}{}\classe{np}
\begin{définition}\fra tempes\end{définition}
\begin{définition}\cmn 太阳穴\end{définition}
\end{entrée}

\begin{entrée}
\vedette{\hypertarget{Ⓔtɯ-ndzɣi}{\papi{ tɯ-ndzɣi}}}\markboth{tɯ-ndzɣi}{}\classe{np}
\begin{définition}\fra canines\end{définition}
\begin{définition}\cmn 獠牙\end{définition}
\end{entrée}

\begin{entrée}
\vedette{\hypertarget{Ⓔtɯ-ndzoʁ}{\papi{ tɯ-ndzoʁ}}}\markboth{tɯ-ndzoʁ}{}\classe{clf}
\begin{définition}\fra gousse\end{définition}
\begin{définition}\cmn 蒜瓣\end{définition}
\begin{exemple}\jya kɯmɕku tɯ-ndzoʁ\cmn 一瓣蒜\end{exemple}\end{entrée}

\begin{entrée}
\vedette{\hypertarget{Ⓔtɯ-ndzrɯ}{\papi{ tɯ-ndzrɯ}}}\markboth{tɯ-ndzrɯ}{}\classe{np}
\begin{définition}\fra ongle\end{définition}
\begin{définition}\cmn 指甲\end{définition}
\begin{exemple}\jya a-ndzrɯ nɯ-nɯ-phɯt-a\cmn 我剪了指甲\end{exemple}\end{entrée}

\begin{entrée}
\vedette{\hypertarget{Ⓔtɯ-ndzʁi}{\papi{ tɯ-ndzʁi}}}\markboth{tɯ-ndzʁi}{}\classe{np}
\begin{définition}\fra clavicule\end{définition}
\begin{définition}\cmn 锁骨\end{définition}
\end{entrée}

\begin{entrée}
\vedette{\hypertarget{Ⓔtɯ-ndzɯ}{\papi{ tɯ-ndzɯ}}}\markboth{tɯ-ndzɯ}{}\classe{np}
\begin{définition}\fra conseil\end{définition}
\begin{définition}\cmn 教育、劝告的话\end{définition}
\begin{exemple}\jya ɯ-ndzɯ mɤ-kɯ-sɤŋo ɯ-mɯ mbɯt\cmn 不听劝告的人没有好下场(他的天要垮下来)\end{exemple}
\begin{sous-entrée}
\vedette{\hypertarget{}{\papi{ tɯ-ndzɯ,βzu}}}\markboth{tɯ-ndzɯ,βzu}{}
\paradigme{\textit{dir :} \jya pɯ-}
\begin{définition}\fra ordonner\end{définition}
\begin{définition}\cmn 命令\end{définition}
\begin{exemple}\jya nɤ-ndzɯ ɲɯ-βze-a\cmn 我命令你\end{exemple}
\begin{exemple}\jya @dangzhongyang kɯ ji-ndzɯ pa-βzu\cmn 党中央命令我们\end{exemple}
\begin{relation-sémantique}\ComponentA{\classe{np}
\hyperlink{Ⓔtɯ-ndzɯ}{\textit{ \papi{tɯ-ndzɯ}}}
}\end{relation-sémantique}
\begin{relation-sémantique}\ComponentB{\classe{vt}
\hyperlink{ⒺβzuⒽ1}{\textit{ \papi{βzu}}}
}\end{relation-sémantique}
\begin{relation-sémantique}\confer{
\hyperlink{ⒺβzuⒽ1}{\textit{ \papi{βzu1}}}
}\end{relation-sémantique}
\end{sous-entrée}\end{entrée}

\begin{entrée}
\vedette{\hypertarget{Ⓔtɯ-nŋa}{\papi{ tɯ-nŋa}}}\markboth{tɯ-nŋa}{}
\classe{np}
\begin{définition}\fra dette\end{définition}
\begin{définition}\cmn 债\end{définition}
\begin{exemple}\jya tɯ-nŋa sqɯ-mpɕar nɯ-tɕat-a\cmn 我欠了十块钱\end{exemple}
\begin{exemple}\jya a-nŋa nɯ mɤʑɯ tʂam-a ra\cmn 我还得还债\end{exemple}
\begin{relation-sémantique}\confer{
\hyperlink{Ⓔŋa}{\textit{ \papi{ŋa}}}
}\end{relation-sémantique}\end{entrée}

\begin{entrée}
\vedette{\hypertarget{Ⓔtɯnoʁ}{\papi{ tɯnoʁ}}}\markboth{tɯnoʁ}{}\classe{n}
\begin{définition}\fra sauce\end{définition}
\begin{définition}\cmn 沾水;酱\end{définition}\end{entrée}

\begin{entrée}
\vedette{\hypertarget{Ⓔtɯ-ntɕhaʁ}{\papi{ tɯ-ntɕhaʁ}}}\markboth{tɯ-ntɕhaʁ}{}\classe{clf}
\begin{définition}\fra goutte\end{définition}
\begin{définition}\cmn 一滴\end{définition}
\begin{exemple}\jya tɯ-mɯ ɯ-ntɕhɯ-ntɕhaʁ ɲɯ-ɤsɯ-lɤt\cmn 点点滴滴地下雨(大雨点子)\end{exemple}\end{entrée}

\begin{entrée}
\vedette{\hypertarget{Ⓔtɯ-ntɕhɯr}{\papi{ tɯ-ntɕhɯr}}}\markboth{tɯ-ntɕhɯr}{}\classe{clf}
\begin{définition}\fra morceau, débris\end{définition}
\begin{définition}\cmn 碎片\end{définition}
\begin{exemple}\jya rdɤstaʁ tɯ-ntɕhɯr\cmn 一个石头碎片\end{exemple}
\begin{exemple}\jya tɯ-ji tɯ-ntɕhɯr\cmn 一块地\end{exemple}
\begin{relation-sémantique}\confer{
\hyperlink{Ⓔɯ-ntɕhantɕhɯr}{\textit{ \papi{ɯ-ntɕhantɕhɯr}}}
}\end{relation-sémantique}\end{entrée}

\begin{entrée}
\vedette{\hypertarget{Ⓔtɯ-nthoʁ}{\papi{ tɯ-nthoʁ}}}\markboth{tɯ-nthoʁ}{}\classe{clf}
\begin{définition}\fra petit rond\end{définition}
\begin{définition}\cmn 小圆点\end{définition}
\begin{exemple}\jya tɤ-se tɯ-nthoʁ pjɤ-ɕe\cmn 地上滴了一滴血\end{exemple}\end{entrée}

\begin{entrée}
\vedette{\hypertarget{Ⓔtɯ-ntsi}{\papi{ tɯ-ntsi}}}\markboth{tɯ-ntsi}{}\classe{clf}
\begin{définition}\fra un membre d'une paire\end{définition}
\begin{définition}\cmn 一只\end{définition}
\begin{exemple}\jya tɯ-xtsa tɯ-ntsi\cmn 一只鞋子\end{exemple}
\begin{sous-entrée}
\vedette{\hypertarget{}{\papi{ ɯ-ntsi,βzu}}}\markboth{ɯ-ntsi,βzu}{}
\begin{définition}\fra répondre\end{définition}
\begin{définition}\cmn 答复\end{définition}
\begin{relation-sémantique}\synonyme{
 \papi{ɯ-lɤn,βzu}
}\end{relation-sémantique}
\begin{relation-sémantique}\synonyme{
 \papi{ɯ-sci,βzu}
}\end{relation-sémantique}
\end{sous-entrée}\end{entrée}

\begin{entrée}
\vedette{\hypertarget{Ⓔtɯ-nɯ}{\papi{ tɯ-nɯ}}}\markboth{tɯ-nɯ}{}\classe{np}
\begin{définition}\fra sein\end{définition}
\begin{définition}\cmn 乳房\end{définition}
\end{entrée}

\begin{entrée}
\vedette{\hypertarget{Ⓔtɯ-ɲɤm}{\papi{ tɯ-ɲɤm}}}\markboth{tɯ-ɲɤm}{}\classe{np}
\begin{définition}\fra chair, gras\end{définition}
\begin{définition}\cmn 身上的肉(人、动物)
\begin{déclaration} \étymologie{\papi{ɲam}}\end{déclaration}\end{définition}\begin{sous-entrée}
\vedette{\hypertarget{}{\papi{ tɯ-ɲɤm,khe}}}\markboth{tɯ-ɲɤm,khe}{}
\begin{définition}\fra maigre\end{définition}
\begin{définition}\cmn 瘦\end{définition}
\begin{relation-sémantique}\ComponentA{\classe{np}
\hyperlink{Ⓔtɯ-ɲɤm}{\textit{ \papi{tɯ-ɲɤm}}}
}\end{relation-sémantique}
\begin{relation-sémantique}\ComponentB{\classe{vs}
\hyperlink{Ⓔkhe}{\textit{ \papi{khe}}}
}\end{relation-sémantique}
\begin{relation-sémantique}\synonyme{
\hyperlink{Ⓔnɯɲɤmkhe}{\textit{ \papi{nɯɲɤmkhe}}}
}\end{relation-sémantique}
\end{sous-entrée}\begin{sous-entrée}
\vedette{\hypertarget{}{\papi{ tɯ-ɲɤm,phɤn}}}\markboth{tɯ-ɲɤm,phɤn}{}
\begin{définition}\fra très utile\end{définition}
\begin{définition}\cmn 有用\end{définition}
\begin{exemple}\jya kɯki sɲɯɣjɯ ki a-ɲɤm wuma pɯ-phɤn ma khro tɤ-ntɕhoz-a\cmn 这支笔对我很有用,我用了很久\end{exemple}
\begin{relation-sémantique}\ComponentA{\classe{np}
\hyperlink{Ⓔtɯ-ɲɤm}{\textit{ \papi{tɯ-ɲɤm}}}
}\end{relation-sémantique}
\begin{relation-sémantique}\ComponentB{\classe{vs}
\hyperlink{Ⓔphɤn}{\textit{ \papi{phɤn}}}
}\end{relation-sémantique}
\end{sous-entrée}\begin{sous-entrée}
\vedette{\hypertarget{}{\papi{ tɯ-ɲɤm,sɯ}}}\markboth{tɯ-ɲɤm,sɯ}{}
\begin{définition}\fra gros, gras\end{définition}
\begin{définition}\cmn 肥;胖\end{définition}
\begin{relation-sémantique}\ComponentA{\classe{np}
\hyperlink{Ⓔtɯ-ɲɤm}{\textit{ \papi{tɯ-ɲɤm}}}
}\end{relation-sémantique}
\begin{relation-sémantique}\ComponentB{\classe{vs}
\hyperlink{Ⓔsɯ}{\textit{ \papi{sɯ}}}
}\end{relation-sémantique}
\begin{relation-sémantique}\confer{
\hyperlink{Ⓔnɯɲɤmsɯ}{\textit{ \papi{nɯɲɤmsɯ}}}
}\end{relation-sémantique}
\end{sous-entrée}\end{entrée}

\begin{entrée}
\vedette{\hypertarget{Ⓔtɯɲɤt}{\papi{ tɯɲɤt}}}\markboth{tɯɲɤt}{}\classe{n}
\begin{définition}\fra éboulement\end{définition}
\begin{définition}\cmn 山崩;滑坡\end{définition}
\begin{exemple}\jya tɯɲɤt pjɤ-ɣi\cmn 出现了山崩\end{exemple}\end{entrée}

\begin{entrée}
\vedette{\hypertarget{Ⓔtɯɲcɣa}{\papi{ tɯɲcɣa}}}\markboth{tɯɲcɣa}{}\classe{n}
\begin{définition}\fra faucille\end{définition}
\begin{définition}\cmn 镰刀\end{définition}
\end{entrée}

\begin{entrée}
\vedette{\hypertarget{Ⓔtɯɲɟoʁ}{\papi{ tɯɲɟoʁ}}}\markboth{tɯɲɟoʁ}{}\classe{n}
\begin{définition}\fra homme de main\end{définition}
\begin{définition}\cmn 助手\end{définition}\end{entrée}

\begin{entrée}
\vedette{\hypertarget{Ⓔtɯɲoʁ}{\papi{ tɯɲoʁ}}}\markboth{tɯɲoʁ}{}\classe{n}
\begin{définition}\fra grains et balle\end{définition}
\begin{définition}\cmn 颗粒和糠秕混合\end{définition}
\end{entrée}

\begin{entrée}
\vedette{\hypertarget{Ⓔtɯŋgu}{\papi{ tɯŋgu}}}\markboth{tɯŋgu}{}
\classe{n}
\begin{définition}\fra casserole pour faire frire la tsampa\end{définition}
\begin{définition}\cmn 炒青稞的锅\end{définition}
\begin{exemple}\jya tɯŋgu nɯ tɯsqar ɯ-sɤ-rŋu ɯ-rkoz ŋu, tɯthɯ nɯ rnaʁ, tɯŋgu nɯ mɤ-rnaʁ, antɤm\cmn 炒锅是专门用来炒青稞的\end{exemple}\end{entrée}

\begin{entrée}
\vedette{\hypertarget{Ⓔtɯ-ŋga}{\papi{ tɯ-ŋga}}}\markboth{tɯ-ŋga}{}\classe{np}
\begin{définition}\fra habit\end{définition}
\begin{définition}\cmn 衣服\end{définition}
\begin{relation-sémantique}\confer{
\hyperlink{Ⓔŋga}{\textit{ \papi{ŋga}}}
}\end{relation-sémantique}
\begin{relation-sémantique}\confer{
\hyperlink{Ⓔstɤnga}{\textit{ \papi{stɤnga}}}
}\end{relation-sémantique}
\begin{relation-sémantique}\confer{
\hyperlink{Ⓔkɯrɯŋga}{\textit{ \papi{kɯrɯŋga}}}
}\end{relation-sémantique}
\begin{relation-sémantique}\confer{
\hyperlink{Ⓔkupaŋga}{\textit{ \papi{kupaŋga}}}
}\end{relation-sémantique}\end{entrée}

\begin{entrée}
\vedette{\hypertarget{Ⓔtɯŋgar}{\papi{ tɯŋgar}}}\markboth{tɯŋgar}{}\classe{n}
\begin{définition}\fra tissu de laine\end{définition}
\begin{définition}\cmn 羊毛布(尚未缝成衣服)\end{définition}
\end{entrée}

\begin{entrée}
\vedette{\hypertarget{Ⓔtɯŋgɤmbe}{\papi{ tɯŋgɤmbe}}}\markboth{tɯŋgɤmbe}{}
\classe{n}
\begin{définition}\fra vêtements abîmés\end{définition}
\begin{définition}\cmn 破旧的衣服\end{définition}
\begin{relation-sémantique}\confer{
\hyperlink{Ⓔtɯ-ŋga}{\textit{ \papi{tɯ-ŋga}}}
}\end{relation-sémantique}
\begin{relation-sémantique}\confer{
\hyperlink{Ⓔtɤ-mbe}{\textit{ \papi{tɤ-mbe}}}
}\end{relation-sémantique}\end{entrée}

\begin{entrée}
\vedette{\hypertarget{Ⓔtɯ-ŋgɤndo}{\papi{ tɯ-ŋgɤndo}}}\markboth{tɯ-ŋgɤndo}{}\classe{np}
\begin{définition}\fra bord des vêtements\end{définition}
\begin{définition}\cmn 衣角\end{définition}
\end{entrée}

\begin{entrée}
\vedette{\hypertarget{Ⓔtɯ-ŋgo}{\papi{ tɯ-ŋgo}}}\markboth{tɯ-ŋgo}{}\classe{np}
\begin{définition}\fra maladie\end{définition}
\begin{définition}\cmn 病\end{définition}
\begin{relation-sémantique}\confer{
\hyperlink{Ⓔngo}{\textit{ \papi{ngo}}}
}\end{relation-sémantique}\end{entrée}

\begin{entrée}
\vedette{\hypertarget{Ⓔtɯ-ŋgru}{\papi{ tɯ-ŋgru}}}\markboth{tɯ-ŋgru}{}\classe{np}
\begin{définition}\fra tendon\end{définition}
\begin{définition}\cmn 筋\end{définition}
\end{entrée}

\begin{entrée}
\vedette{\hypertarget{Ⓔtɯ-ŋgra}{\papi{ tɯ-ŋgra}}}\markboth{tɯ-ŋgra}{}\classe{np}
\begin{définition}\fra salaire\end{définition}
\begin{définition}\cmn 工资\end{définition}
\begin{relation-sémantique}\confer{
\hyperlink{Ⓔnɯŋgra}{\textit{ \papi{nɯŋgra}}}
}\end{relation-sémantique}\end{entrée}

\begin{entrée}
\vedette{\hypertarget{Ⓔtɯ-ŋgɯl}{\papi{ tɯ-ŋgɯl}}}\markboth{tɯ-ŋgɯl}{}\classe{clf}
\begin{définition}\fra une boucle, un tour (à propos d'intestins enroulés comme des cordes)\end{définition}
\begin{définition}\cmn 一圈(肠子)\end{définition}
\begin{exemple}\jya tɯ-pu tɯ-ŋgɯl\cmn 一圈肠子\end{exemple}
\begin{relation-sémantique}\confer{
\hyperlink{Ⓔtɯ-tɤjŋgɤɣ}{\textit{ \papi{tɯ-tɤjŋgɤɣ}}}
}\end{relation-sémantique}\end{entrée}

\begin{entrée}
\vedette{\hypertarget{Ⓔtɯ-ŋka}{\papi{ tɯ-ŋka}}}\markboth{tɯ-ŋka}{}\classe{clf}\acception{1}
\begin{définition}\fra une parole, un bruit\end{définition}
\begin{définition}\cmn 一声;一句\end{définition}
\begin{exemple}\jya tɯ-ŋka tɯ-ŋka tu-ti-a ŋu nɤ\cmn 我一句一句地说\end{exemple}\acception{2}
\begin{définition}\fra une bouchée\end{définition}
\begin{définition}\cmn 一口,嚼过的食物\end{définition}
\begin{exemple}\jya tɤ-rɟit nɯ tɯ-ŋka kɤ-mbi kɯ chɯ́-wɣ-ɣɤwxti ɕti\cmn 小孩子是用大人嚼过的食物喂大的\end{exemple}
\begin{relation-sémantique}\confer{
\hyperlink{Ⓔnɤŋka}{\textit{ \papi{nɤŋka}}}
}\end{relation-sémantique}\end{entrée}

\begin{entrée}
\vedette{\hypertarget{Ⓔtɯ-ɴɢar}{\papi{ tɯ-ɴɢar}}}\markboth{tɯ-ɴɢar}{}\classe{np}
\begin{définition}\fra crachat\end{définition}
\begin{définition}\cmn 痰\end{définition}
\begin{exemple}\jya ɯ-thoʁ nɤ-ɴɢar ma-thɯ-βde ma ɲɯ-sɤʑɯloʁ\cmn 你不要在地上吐痰,很恶心。\end{exemple}
\end{entrée}

\begin{entrée}
\vedette{\hypertarget{Ⓔtɯpu}{\papi{ tɯpu}}}\markboth{tɯpu}{}\classe{n}
\begin{définition}\fra moxibustion\end{définition}
\begin{définition}\cmn 艾灸\end{définition}
\begin{exemple}\jya a-tɯpu ka-ta\cmn 他给我烧艾灸\end{exemple}
\end{entrée}

\begin{entrée}
\vedette{\hypertarget{Ⓔtɯ-pu}{\papi{ tɯ-pu}}}\markboth{tɯ-pu}{}\classe{np}
\begin{définition}\fra intestin\end{définition}
\begin{définition}\cmn 肠\end{définition}
\end{entrée}

\begin{entrée}
\vedette{\hypertarget{Ⓔtɯ-pɤchaʁ}{\papi{ tɯ-pɤchaʁ}}}\markboth{tɯ-pɤchaʁ}{}\classe{np}
\begin{définition}\fra nombril\end{définition}
\begin{définition}\cmn 肚脐\end{définition}
\end{entrée}

\begin{entrée}
\vedette{\hypertarget{Ⓔtɯ-pɤɕnɤz}{\papi{ tɯ-pɤɕnɤz}}}\markboth{tɯ-pɤɕnɤz}{}\classe{np}
\begin{définition}\fra anus\end{définition}
\begin{définition}\cmn 肛门
\begin{déclaration}\use{古语}\end{déclaration}\end{définition}
\begin{relation-sémantique}\confer{
\hyperlink{Ⓔtɯ-pu}{\textit{ \papi{tɯ-pu}}}
}\end{relation-sémantique}
\begin{relation-sémantique}\confer{
\hyperlink{Ⓔtɤ-ɕnɤz}{\textit{ \papi{tɤ-ɕnɤz}}}
}\end{relation-sémantique}\end{entrée}

\begin{entrée}
\vedette{\hypertarget{Ⓔtɯ-pɤɣrum}{\papi{ tɯ-pɤɣrum}}}\markboth{tɯ-pɤɣrum}{}\classe{np}
\begin{définition}\fra gros intestin\end{définition}
\begin{définition}\cmn 大肠\end{définition}
\begin{relation-sémantique}\confer{
\hyperlink{Ⓔtɯ-pu}{\textit{ \papi{tɯ-pu}}}
}\end{relation-sémantique}\end{entrée}

\begin{entrée}
\vedette{\hypertarget{Ⓔtɯ-pɤɲɟi}{\papi{ tɯ-pɤɲɟi}}}\markboth{tɯ-pɤɲɟi}{}\classe{np}
\begin{définition}\fra bas-ventre\end{définition}
\begin{définition}\cmn 小肚子\end{définition}
\end{entrée}

\begin{entrée}
\vedette{\hypertarget{Ⓔtɯ-pɤŋi}{\papi{ tɯ-pɤŋi}}}\markboth{tɯ-pɤŋi}{}\classe{np}
\begin{définition}\fra intestin grêle\end{définition}
\begin{définition}\cmn 小肠\end{définition}
\begin{relation-sémantique}\confer{
\hyperlink{Ⓔtɯ-pu}{\textit{ \papi{tɯ-pu}}}
}\end{relation-sémantique}\end{entrée}

\begin{entrée}
\vedette{\hypertarget{Ⓔtɯpɤr}{\papi{ tɯpɤr}}}\markboth{tɯpɤr}{}
\classe{n}
\begin{définition}\fra dessin\end{définition}
\begin{définition}\cmn 画,照片
\begin{déclaration} \étymologie{\papi{par}}\end{déclaration}\end{définition}
\begin{exemple}\jya nɤ-tɯpɤr pjɯ-lat-a tɕe, tɯrme ɯ-ɕki ɲɯ-kham-a\cmn 我给你拍照片,给别人看\end{exemple}\end{entrée}

\begin{entrée}
\vedette{\hypertarget{Ⓔtɯ-pɤrme}{\papi{ tɯ-pɤrme}}}\markboth{tɯ-pɤrme}{}
\classe{clf}
\begin{définition}\fra année\end{définition}
\begin{définition}\cmn 一岁(年龄)\end{définition}
\begin{exemple}\jya nɤʑo thɤstɯ-pɤrme thɯ-tɯ-azɣɯt?\cmn 你多大了?\end{exemple}
\begin{exemple}\jya aʑo fsusqafsum-pɤrme thɯ-azɣɯt-a\cmn 我三十三岁\end{exemple}\end{entrée}

\begin{entrée}
\vedette{\hypertarget{Ⓔtɯpɕi}{\papi{ tɯpɕi}}}\markboth{tɯpɕi}{}\classe{n}
\begin{définition}\fra lin\end{définition}
\begin{définition}\cmn 亚麻\end{définition}
\end{entrée}

\begin{entrée}
\vedette{\hypertarget{Ⓔtɯ-pɕoʁ}{\papi{ tɯ-pɕoʁ}}}\markboth{tɯ-pɕoʁ}{}\classe{n}
\begin{définition}\fra côté, direction\end{définition}
\begin{définition}\cmn 方向
\begin{déclaration} \étymologie{\papi{pʰʲogs}}\end{déclaration}\end{définition}
\begin{exemple}\jya tɯ-pɕoʁ ci pjɯ-phɤn, tɯ-pɕoʁ ci pjɯ-ʁdɯɣ ɲɯ-ɕti\cmn (这种药)一方面有好处,一方面又有副作用\end{exemple}
\begin{exemple}\jya ɯ-pɕoʁ a-mɤ-pɯ-naχtɕɯɣ qhe li ɯ-ti mɤ-naχtɕɯɣ\cmn 只要方向不一样说法就不一样(解释动词的趋向前缀的时候)\end{exemple}\end{entrée}

\begin{entrée}
\vedette{\hypertarget{Ⓔtɯ-pɕɯrtɕhaʁ}{\papi{ tɯ-pɕɯrtɕhaʁ}}}\markboth{tɯ-pɕɯrtɕhaʁ}{}\classe{clf}
\begin{définition}\fra une fois\end{définition}
\begin{définition}\cmn 一倍\end{définition}
\begin{relation-sémantique}\synonyme{
\hyperlink{Ⓔtɯ-tɤlɤβ}{\textit{ \papi{tɯ-tɤlɤβ}}}
}\end{relation-sémantique}
\begin{relation-sémantique}\confer{
\hyperlink{Ⓔtɯ-tɯpɕɯrtɕhaʁ}{\textit{ \papi{tɯ-tɯpɕɯrtɕhaʁ}}}
}\end{relation-sémantique}\end{entrée}

\begin{entrée}
\vedette{\hypertarget{Ⓔtɯpɣaʁ}{\papi{ tɯpɣaʁ}}}\markboth{tɯpɣaʁ}{}\classe{n}
\begin{définition}\fra défrichage\end{définition}
\begin{définition}\cmn 开荒\end{définition}
\begin{exemple}\jya tɯpɣaʁ lo-tɕɤt-ndʑi\cmn 他们俩开荒了\end{exemple}
\begin{relation-sémantique}\confer{
\hyperlink{Ⓔpɣaʁ}{\textit{ \papi{pɣaʁ}}}
}\end{relation-sémantique}\end{entrée}

\begin{entrée}
\vedette{\hypertarget{Ⓔtɯ-phaʁ}{\papi{ tɯ-phaʁ}}}\markboth{tɯ-phaʁ}{}\classe{np}
\begin{définition}\fra un côté\end{définition}
\begin{définition}\cmn 一边;半边\end{définition}
\begin{exemple}\jya ɯ-phaʁ ntsi (kɯ) ko-sɯ-rtoʁ\cmn 他斜着眼睛看了\end{exemple}\end{entrée}

\begin{entrée}
\vedette{\hypertarget{Ⓔtɯ-phaʁja}{\papi{ tɯ-phaʁja}}}\markboth{tɯ-phaʁja}{}\classe{np}
\begin{définition}\fra époux\end{définition}
\begin{définition}\cmn 丈夫;伴侣\end{définition}
\end{entrée}

\begin{entrée}
\vedette{\hypertarget{Ⓔtɯ-phoŋ}{\papi{ tɯ-phoŋ}}}\markboth{tɯ-phoŋ}{}\classe{clf}
\begin{définition}\fra bouteille\end{définition}
\begin{définition}\cmn 一瓶
\begin{déclaration} \étymologie{\papi{bum-pa}}\end{déclaration}\end{définition}
\begin{exemple}\jya cha tɯ-phoŋ\cmn 一瓶酒\end{exemple}
\end{entrée}

\begin{entrée}
\vedette{\hypertarget{Ⓔtɯ-phoŋbu}{\papi{ tɯ-phoŋbu}}}\markboth{tɯ-phoŋbu}{}
\classe{np}
\begin{définition}\fra corps\end{définition}
\begin{définition}\cmn 身体
\begin{déclaration} \étymologie{\papi{pʰuŋ.po}}\end{déclaration}\end{définition}
\begin{exemple}\jya nɤ-phoŋbu ɣɯ ɯ-βri ma-pɯ-tɯ-sɯxɕe ma!\cmn 你不要伤着身体!\end{exemple}
\begin{relation-sémantique}\confer{
 \papi{tɯ-phoŋbu,ndo}
}\end{relation-sémantique}\end{entrée}

\begin{entrée}
\vedette{\hypertarget{Ⓔtɯ-phoʁ}{\papi{ tɯ-phoʁ}}}\markboth{tɯ-phoʁ}{}\classe{np}
\begin{définition}\fra salaire\end{définition}
\begin{définition}\cmn 工资\end{définition}
\begin{relation-sémantique}\synonyme{
\hyperlink{Ⓔtɯ-ŋgra}{\textit{ \papi{tɯ-ŋgra}}}
}\end{relation-sémantique}\end{entrée}

\begin{entrée}
\vedette{\hypertarget{Ⓔtɯ-phɯ}{\papi{ tɯ-phɯ}}}\markboth{tɯ-phɯ}{}\classe{clf}
\begin{définition}\fra tronc\end{définition}
\begin{définition}\cmn 一棵\end{définition}
\begin{exemple}\jya si tɯ-phɯ\cmn 一棵树\end{exemple}\end{entrée}

\begin{entrée}
\vedette{\hypertarget{Ⓔtɯ-phɯɣ}{\papi{ tɯ-phɯɣ}}}\markboth{tɯ-phɯɣ}{}\classe{np}
\begin{définition}\fra fortune\end{définition}
\begin{définition}\cmn 财产,财力\end{définition}
\begin{exemple}\jya jiɕqha tɯrme nɯ ɯ-phɯɣ tu\cmn 那个人很有财力\end{exemple}
\end{entrée}

\begin{entrée}
\vedette{\hypertarget{Ⓔtɯ-phɯm}{\papi{ tɯ-phɯm}}}\markboth{tɯ-phɯm}{}\classe{np}
\begin{définition}\fra pan du vêtement\end{définition}
\begin{définition}\cmn 衣兜\end{définition}
\begin{exemple}\jya ɯ-phɯm nɯ tɕu tasa-rŋu ci tɯ-lʁɤtɕɯ ɲɤ-rku\cmn 他把一袋炒麻籽装在衣兜里了\end{exemple}
\end{entrée}

\begin{entrée}
\vedette{\hypertarget{Ⓔtɯ-phɯxpa}{\papi{ tɯ-phɯxpa}}}\markboth{tɯ-phɯxpa}{}\classe{np}
\begin{définition}\fra cuisse\end{définition}
\begin{définition}\cmn 大腿\end{définition}
\end{entrée}

\begin{entrée}
\vedette{\hypertarget{Ⓔtɯ-pju}{\papi{ tɯ-pju}}}\markboth{tɯ-pju}{}\classe{np}
\begin{définition}\fra moelle\end{définition}
\begin{définition}\cmn 骨髓\end{définition}\end{entrée}

\begin{entrée}
\vedette{\hypertarget{Ⓔtɯ-pjaχpa}{\papi{ tɯ-pjaχpa}}}\markboth{tɯ-pjaχpa}{}\classe{np}
\begin{définition}\fra aisselle\end{définition}
\begin{définition}\cmn 膈肢窝\end{définition}
\begin{exemple}\jya jɯɣi ɯ-pjaχpa to-rku\cmn 他把书夹在腋下\end{exemple}
\begin{relation-sémantique}\confer{
\hyperlink{Ⓔnɯpjaχpa}{\textit{ \papi{nɯpjaχpa}}}
}\end{relation-sémantique}\end{entrée}

\begin{entrée}
\vedette{\hypertarget{Ⓔtɯ-po}{\papi{ tɯ-po}}}\markboth{tɯ-po}{}\classe{clf}
\begin{définition}\fra unité de mesure\end{définition}
\begin{définition}\cmn 斗
\begin{déclaration} \étymologie{\papi{ⁿbo}}\end{déclaration}\end{définition}\end{entrée}

\begin{entrée}
\vedette{\hypertarget{Ⓔtɯpoli}{\papi{ tɯpoli}}}\markboth{tɯpoli}{}\classe{n}
\begin{définition}\fra une gerbe d'herbe\end{définition}
\begin{définition}\cmn 一捆青草\end{définition}
\end{entrée}

\begin{entrée}
\vedette{\hypertarget{Ⓔtɯpri}{\papi{ tɯpri}}}\markboth{tɯpri}{}\classe{n}
\begin{définition}\fra message\end{définition}
\begin{définition}\cmn 口信\end{définition}
\begin{exemple}\jya a-tɯpri jo-lɤt\cmn 他给我带了口信\end{exemple}
\begin{relation-sémantique}\confer{
\hyperlink{Ⓔznɯxpri}{\textit{ \papi{znɯxpri}}}
}\end{relation-sémantique}
\end{entrée}

\begin{entrée}
\vedette{\hypertarget{Ⓔtɯ-pɯsqhɯt}{\papi{ tɯ-pɯsqhɯt}}}\markboth{tɯ-pɯsqhɯt}{}\classe{np}
\begin{définition}\fra fin de l'œsophage\end{définition}
\begin{définition}\cmn 食管的末端\end{définition}\end{entrée}

\begin{entrée}
\vedette{\hypertarget{Ⓔtɯ-qa}{\papi{ tɯ-qa}}}\markboth{tɯ-qa}{}\classe{np}\acception{1}
\begin{définition}\fra racine\end{définition}
\begin{définition}\cmn 根\end{définition}\acception{2}
\begin{définition}\fra patte\end{définition}
\begin{définition}\cmn (动物)的脚\end{définition}\acception{3}
\begin{définition}\fra fond\end{définition}
\begin{définition}\cmn 底部\end{définition}
\begin{exemple}\jya tɤ-fkɯm ɣɯ ɯ-qa\cmn 袋子的底部\end{exemple}
\begin{exemple}\jya mtshu ɯ-qa zɯ\cmn 在湖底\end{exemple}
\begin{exemple}\jya ɯ-qa ʑo tu-nɯɬoʁ naʁzi-a\cmn 我想寻根问底\end{exemple}
\begin{exemple}\jya ɯ-kɤ-thu nɯ ɯ-qa ʑo tu-nɯɬoʁ naʁzi\cmn 他想寻根问底\end{exemple}
\begin{exemple}\jya nɯ-tɯ-khe kɯ ɯ-qa ʑo ɲɤ-me\cmn 他们笨到极点\end{exemple}\begin{sous-entrée}
\vedette{\hypertarget{}{\papi{ ɯ-qaɕɯqa}}}\markboth{ɯ-qaɕɯqa}{}
\begin{définition}\fra le plus profond\end{définition}
\begin{définition}\cmn 最底层\end{définition}
\begin{exemple}\jya rɟɤmtshu ɯ-qaɕɯqa\cmn 海洋的最底部\end{exemple}
\begin{relation-sémantique}\confer{
\hyperlink{Ⓔtɤ-qaʁrɯ}{\textit{ \papi{tɤ-qaʁrɯ}}}
}\end{relation-sémantique}
\end{sous-entrée}\end{entrée}

\begin{entrée}
\vedette{\hypertarget{Ⓔtɯ-qartsɯ}{\papi{ tɯ-qartsɯ}}}\markboth{tɯ-qartsɯ}{}
\classe{clf}
\begin{définition}\fra coup de pied\end{définition}
\begin{définition}\cmn 踢一脚\end{définition}
\begin{exemple}\jya tɯ-qartsɯ ta-lɤt\cmn 它踢了一脚\end{exemple}
\begin{relation-sémantique}\confer{
\hyperlink{Ⓔsɯqartsɯ}{\textit{ \papi{sɯqartsɯ}}}
}\end{relation-sémantique}\end{entrée}

\begin{entrée}
\vedette{\hypertarget{Ⓔtɯ-qazgra}{\papi{ tɯ-qazgra}}}\markboth{tɯ-qazgra}{}\classe{np}
\begin{définition}\fra bruit de pas\end{définition}
\begin{définition}\cmn 脚步声\end{définition}
\begin{relation-sémantique}\confer{
\hyperlink{Ⓔtɤ-zgra}{\textit{ \papi{tɤ-zgra}}}
}\end{relation-sémantique}\end{entrée}

\begin{entrée}
\vedette{\hypertarget{Ⓔtɯ-qɤsɤlɤt}{\papi{ tɯ-qɤsɤlɤt}}}\markboth{tɯ-qɤsɤlɤt}{}\classe{np}
\begin{définition}\fra anus\end{définition}
\begin{définition}\cmn 肛门\end{définition}
\end{entrée}

\begin{entrée}
\vedette{\hypertarget{Ⓔtɯ-qe}{\papi{ tɯ-qe}}}\markboth{tɯ-qe}{}\classe{np}
\begin{définition}\fra excrément, pet\end{définition}
\begin{définition}\cmn 屎;屁\end{définition}
\begin{exemple}\jya a-qe nɯ-lat-a\cmn 我拉了屎\end{exemple}
\begin{exemple}\jya a-qe thɯ-lat-a\cmn 我放了屁\end{exemple}
\begin{relation-sémantique}\confer{
\hyperlink{Ⓔkhrambaqe}{\textit{ \papi{khrambaqe}}}
}\end{relation-sémantique}
\begin{relation-sémantique}\confer{
\hyperlink{Ⓔzdɯmqe}{\textit{ \papi{zdɯmqe}}}
}\end{relation-sémantique}
\end{entrée}

\begin{entrée}
\vedette{\hypertarget{Ⓔtɯ-qejdi}{\papi{ tɯ-qejdi}}}\markboth{tɯ-qejdi}{}\classe{np}
\begin{définition}\fra odeur de bouse\end{définition}
\begin{définition}\cmn 屎的臭味\end{définition}
\begin{exemple}\jya nɤ-qejdi ɯ-tɯ-sɤjloʁ nɯ\cmn 你的屎的臭味很难闻\end{exemple}
\begin{relation-sémantique}\confer{
\hyperlink{Ⓔtɯ-qe}{\textit{ \papi{tɯ-qe}}}
}\end{relation-sémantique}
\begin{relation-sémantique}\confer{
\hyperlink{Ⓔtɤ-di}{\textit{ \papi{tɤ-di}}}
}\end{relation-sémantique}\end{entrée}

\begin{entrée}
\vedette{\hypertarget{Ⓔtɯqejmɤɣ}{\papi{ tɯqejmɤɣ}}}\markboth{tɯqejmɤɣ}{}\classe{n}
\begin{définition}\fra une espèce de champignon\end{définition}
\begin{définition}\cmn 【牛屎菌】\end{définition}
\begin{exemple}\jya tɯ-qe jmɤɣ nɯ tɯ-qe ɯ-taʁ tu-ɬoʁ ŋu, ɯ-mdoʁ kɯ-wɣrum tu, kɯ-qandʐi tu, kɯ-ɣɯrni tu, kɯ-sɤndɤɣ me ri ɯ-kɯ-ndza me\cmn 牛屎菌长在粪上,有的是白色的,有的是乌色的,有的是红的,没有毒性,但也没人吃。\end{exemple}
\end{entrée}

\begin{entrée}
\vedette{\hypertarget{Ⓔtɯ-qe,rɤrɕɯβ}{\papi{ tɯ-qe,rɤrɕɯβ}}}\markboth{tɯ-qe,rɤrɕɯβ}{}\paradigme{\textit{dir :} \jya thɯ-}
\begin{définition}\fra péter sans faire de bruit\end{définition}
\begin{définition}\cmn 悄悄地放屁\end{définition}
\begin{exemple}\jya ɯ-qe tha-rɤrɕɯβ\cmn 他放了屁\end{exemple}
\begin{relation-sémantique}\ComponentA{\classe{np}
\hyperlink{Ⓔtɯ-qe}{\textit{ \papi{tɯ-qe}}}
}\end{relation-sémantique}
\begin{relation-sémantique}\ComponentB{\classe{vt}
 \papi{rɤrɕɯβ}
}\end{relation-sémantique}\end{entrée}

\begin{entrée}
\vedette{\hypertarget{Ⓔtɯ-qhoχpa}{\papi{ tɯ-qhoχpa}}}\markboth{tɯ-qhoχpa}{}\classe{np}\acception{1}
\begin{définition}\fra organes\end{définition}
\begin{définition}\cmn 内脏\end{définition}\acception{2}
\begin{définition}\fra état d'esprit\end{définition}
\begin{définition}\cmn 性情
\begin{déclaration} \étymologie{\papi{kʰog.pa}}\end{déclaration}\end{définition}\end{entrée}

\begin{entrée}
\vedette{\hypertarget{Ⓔtɯ-qhrɯmbɤβ}{\papi{ tɯ-qhrɯmbɤβ}}}\markboth{tɯ-qhrɯmbɤβ}{}
\classe{np}
\begin{définition}\fra rot\end{définition}
\begin{définition}\cmn 饱嗝\end{définition}
\begin{exemple}\jya a-qhrɯmbɤβ ɲɯ-sɯɣe\cmn 我打了个嗝儿\end{exemple}
\begin{exemple}\jya tɕɣom tɤ-ndza-t-a, a-qhrɯmbɤβ la-sɯɣe\cmn 我吃了花椒,就打嗝了\end{exemple}
\begin{exemple}\jya @pijiu kɤ-tshi-t-a, a-qhrɯmbɤβ ja-sɯɣe\cmn 我一喝啤酒就要打嗝\end{exemple}
\begin{relation-sémantique}\confer{
\hyperlink{Ⓔnɤqhrɯmbɤβ}{\textit{ \papi{nɤqhrɯmbɤβ}}}
}\end{relation-sémantique}\end{entrée}

\begin{entrée}
\vedette{\hypertarget{Ⓔtɯqioʁ}{\papi{ tɯqioʁ}}}\markboth{tɯqioʁ}{}\classe{n}
\begin{définition}\fra vomi\end{définition}
\begin{définition}\cmn 呕吐物\end{définition}
\begin{exemple}\jya tɯqioʁ tɤ-khat-a\cmn 我吐了很久(很多)\end{exemple}
\begin{relation-sémantique}\confer{
\hyperlink{Ⓔqioʁ}{\textit{ \papi{qioʁ}}}
}\end{relation-sémantique}\end{entrée}

\begin{entrée}
\vedette{\hypertarget{Ⓔtɯ-qom}{\papi{ tɯ-qom}}}\markboth{tɯ-qom}{}\classe{np}
\begin{définition}\fra larme\end{définition}
\begin{définition}\cmn 眼泪\end{définition}
\end{entrée}

\begin{entrée}
\vedette{\hypertarget{Ⓔtɯr}{\papi{ tɯr}}}\markboth{tɯr}{}
\classe{vt}
\paradigme{\textit{dir :} \jya \_}
\begin{définition}\fra dépasser\end{définition}
\begin{définition}\cmn 冲过去\end{définition}
\begin{exemple}\jya fsapaʁ kɯ kɤ́-wɣ-tɯr-a\cmn 牲畜在我面前冲过去了\end{exemple}\end{entrée}

\begin{entrée}
\vedette{\hypertarget{Ⓔtɯ-rɤʁaŋ}{\papi{ tɯ-rɤʁaŋ}}}\markboth{tɯ-rɤʁaŋ}{}\classe{np}
\begin{définition}\fra capacité de décision\end{définition}
\begin{définition}\cmn 自我主张的权利
\begin{déclaration} \étymologie{\papi{raŋ.dbaŋ}}\end{déclaration}\end{définition}
\begin{exemple}\jya a-rɤʁaŋ me\cmn 我身不由己\end{exemple}
\begin{relation-sémantique}\confer{
\hyperlink{Ⓔznɯrɤʁaŋ}{\textit{ \papi{znɯrɤʁaŋ}}}
}\end{relation-sémantique}\end{entrée}

\begin{entrée}
\vedette{\hypertarget{Ⓔtɯrɤt}{\papi{ tɯrɤt}}}\markboth{tɯrɤt}{}\classe{n}
\begin{définition}\fra style d'écriture, façon d'écrire\end{définition}
\begin{définition}\cmn 字体\end{définition}
\begin{relation-sémantique}\confer{
\hyperlink{Ⓔrɤt}{\textit{ \papi{rɤt}}}
}\end{relation-sémantique}\end{entrée}

\begin{entrée}
\vedette{\hypertarget{Ⓔtɯ-rcu}{\papi{ tɯ-rcu}}}\markboth{tɯ-rcu}{}
\classe{np}
\begin{définition}\fra veste\end{définition}
\begin{définition}\cmn 皮袄\end{définition}
\begin{exemple}\jya a-rcu\cmn 我的皮袄\end{exemple}\end{entrée}

\begin{entrée}
\vedette{\hypertarget{Ⓔtɯ-rdoʁ}{\papi{ tɯ-rdoʁ}}}\markboth{tɯ-rdoʁ}{}
\classe{clf}\acception{1}
\begin{définition}\fra un morceau\end{définition}
\begin{définition}\cmn 一块;一个
\begin{déclaration}\use{茶堡话的默认量词}\end{déclaration}\end{définition}
\begin{exemple}\jya khɯtsa tɯ-rdoʁ\cmn 一个碗\end{exemple}
\begin{exemple}\jya tɤ-ŋgɯm tɯ-rdoʁ\cmn 一个鸡蛋\end{exemple}
\begin{exemple}\jya mbrɤz tɯ-rdoʁ\cmn 一粒米\end{exemple}
\begin{exemple}\jya zɣɤmbu tɯ-rdoʁ\cmn 一把扫把\end{exemple}
\begin{exemple}\jya mbrɯtɕɯ tɯ-rdoʁ\cmn 一把刀\end{exemple}
\begin{exemple}\jya ndzom tɯ-rdoʁ\cmn 一座桥\end{exemple}
\begin{exemple}\jya rɤɣo tɯ-rdoʁ\cmn 一首歌\end{exemple}
\begin{exemple}\jya tɯ-ŋga tɯ-rdoʁ\cmn 一件衣服\end{exemple}
\begin{exemple}\jya ʁmaʁdɤr tɯ-rdoʁ\cmn 一面旗\end{exemple}
\begin{exemple}\jya mɯntoʁ tɯ-rdoʁ\cmn 一朵花\end{exemple}
\begin{exemple}\jya ɯ-rdɯ-rdoʁ ʑo ma me\cmn 只剩下几个\end{exemple}
\begin{exemple}\jya ɯ-zda ra ʁnɯz ɣɯ nɯ-kɤ-ndza nɯ ɯʑo tɯ-rdoʁ kɯ tu-ndze mɤɕtʂa mɯ́j-rtaʁ\cmn 他的吃量是其他(小孩子)的两倍\end{exemple}\acception{2}
\begin{définition}\fra grains\end{définition}
\begin{définition}\cmn 粮食
\begin{déclaration} \étymologie{\papi{rdog.po}}\end{déclaration}\end{définition}
\begin{relation-sémantique}\synonyme{
\hyperlink{Ⓔtɯjpu}{\textit{ \papi{tɯjpu}}}
}\end{relation-sémantique}\end{entrée}

\begin{entrée}
\vedette{\hypertarget{Ⓔtɯrgi}{\papi{ tɯrgi}}}\markboth{tɯrgi}{}\classe{n}
\begin{définition}\fra sapin\end{définition}
\begin{définition}\cmn 杉树\end{définition}
\begin{exemple}\jya tɯrgi nɯ zgo kɯ-mbro ɴqiaβ tsa tu-ɬoʁ ŋu, si wuma ʑo kɯ-mbro, ɯ-ru nɯ kɯ-jpɯ-jpum kɯ-mbɯ-mbro ŋu, ɯ-rtaʁ nɯ khro mɤ-jpum tɕe, ɯ-βri nɯ tɕu mɤ-kɯ-ɤmtɕhoʁ tu-oʑɯrja ŋu. ɯ-rtaʁ maŋpa nɯ ra zri, taʁ tɤ-ari ɯ-jija ɯ-rtaʁ nɯ tu-xtɯt ŋu, ɯ-rtaʁ nɯ ɯ-taʁ nɯ tɕu, li ɯ-rtaʁ ɲɯ-nɯ-ɴɢɤt ŋu, ɯ-jwaʁ nɯ taqaβ fse ri, xtɯt aɕpɯɕpa, tɕe ɯ-rtaʁ ɯ-taʁ ɯ-jwaʁ ɯ-tɯ-ndzoʁ nɯ tɤ-muj kɯ-fse ɯ-tshɯɣa ŋu. tɯrgi ɯ-jwaʁ nɯ arŋi tɕe pɣi tsa, ɯ-mat nɯ qaɟy ɯ-rqhu tsa fse, alɯlju tɕe rɲɟi tsa, tɕe nɯ ɯ-ŋgɯ nɯ tɕu, ɯ-rɣi arku, thɯ-tɯt tɕe, qaɟy rqhu kɯ-fse nɯ raŋri ʑo ɲɤ-ɴɢaʁ, tɕe ɯ-rɣi pjɯ-nɯɬoʁ ŋu ma ɯ-zrɤm taʁ tu-mphɯl mɤ-cha, tɕe ɯ-mat nɯ tɯrgi laŋlaŋ rmi. tɯrgi ɯ-ru nɯ wuma ʑo tɤrɤm ɯ-spa kɯ-ʑru, tɯrgi tɯ-phɯ nɯ kɯβdɤsqi-ɟom, kɯmŋɤsqi-ɟom kɯ-mbro ɲɯ-ɕti. tɕe nɯ nɯ́-wɣ-phɯt tɕe ɕoŋtɕa ɕnɤcɤ-rzɯɣ kɤ-βzu rtaʁ. tɕe ɯ-rtaʁ nɯ ra kɤ-nɯβlɯ ma mɯ́j-sna, ɯ-ru nɯ ŋgɤjpɤn chɯ́-wɣ-lɤt tɕe, laχtɕha tɕhi kɯ-ra kɤ-βzu sna.\cmn 
杉树生长在比较高的山阴(背阴的山坡)上,是一种高大的树。树干长得又粗又高,枝桠都不粗,在树上长得不整齐。树枝下面的长,越是长在上面就越短。在枝桠上有分杈,叶子长得像针,短而扁,叶子在枝桠上的长法像羽毛的形状。杉树的叶子青而灰,果实像鱼鳞,是圆柱形的,种子装在里面。成熟后,像鳞片的那些东西个个都展开了,种子就会出来,因为杉树不能用根繁殖。这种果实叫\stylefv{tɯrgilaŋlaŋ}。杉树的树干是做木板最好的原料,一棵杉树有14-15米高,砍下来足够锯成七八节木料。枝桠只能烧火用,树干锯成木板,可以作成各种家具。
\end{exemple}
\end{entrée}

\begin{entrée}
\vedette{\hypertarget{Ⓔtɯrgigrɯβgrɯβ}{\papi{ tɯrgigrɯβgrɯβ}}}\markboth{tɯrgigrɯβgrɯβ}{}\classe{n}
\begin{définition}\fra une espèce de champignon\end{définition}
\begin{définition}\cmn 【杉木蘑菇】\end{définition}
\begin{exemple}\jya tɯrgi grɯβgrɯβ nɯ tɯrgi kɯ-wxti kɯ-ʁjɤr ɯ-ŋgɯ tu-ɬoʁ ŋu, stonka mɤɕtʂa mɤ-ɬoʁ, grɯβgrɯβ cho ɯ-mdoʁ naχtɕɯɣ ɯ-ru jpum cho wxti ɯ-di mɤ-naχtɕɯɣ\cmn 杉木蘑菇是长在茂密高大的杉木林里,到秋天才能生长,颜色和松茸一样,但主干比较粗大,味道不一样。\end{exemple}
\end{entrée}

\begin{entrée}
\vedette{\hypertarget{Ⓔtɯrgilaŋlaŋ}{\papi{ tɯrgilaŋlaŋ}}}\markboth{tɯrgilaŋlaŋ}{}\classe{n}
\begin{définition}\fra pomme du sapin\end{définition}
\begin{définition}\cmn 杉树果\end{définition}
\begin{exemple}\jya nɯ-tɯrgilaŋlaŋ\end{exemple}\end{entrée}

\begin{entrée}
\vedette{\hypertarget{Ⓔtɯrgipaʁtsa}{\papi{ tɯrgipaʁtsa}}}\markboth{tɯrgipaʁtsa}{}\classe{n}
\begin{définition}\fra écureuil\end{définition}
\begin{définition}\cmn 松鼠\end{définition}
\end{entrée}

\begin{entrée}
\vedette{\hypertarget{Ⓔtɯrgismɤɣ}{\papi{ tɯrgismɤɣ}}}\markboth{tɯrgismɤɣ}{}\classe{n}
\begin{définition}\fra Usnea sp.\end{définition}
\begin{définition}\cmn 松萝\end{définition}
\end{entrée}

\begin{entrée}
\vedette{\hypertarget{Ⓔtɯ-rɣi}{\papi{ tɯ-rɣi}}}\markboth{tɯ-rɣi}{}\classe{n}
\begin{définition}\fra graine\end{définition}
\begin{définition}\cmn 种子\end{définition}\end{entrée}

\begin{entrée}
\vedette{\hypertarget{Ⓔtɯ-rɣɯt}{\papi{ tɯ-rɣɯt}}}\markboth{tɯ-rɣɯt}{}\classe{clf}
\begin{définition}\fra brin de fil\end{définition}
\begin{définition}\cmn 一股线\end{définition}\end{entrée}

\begin{entrée}
\vedette{\hypertarget{Ⓔtɯ-ri}{\papi{ tɯ-ri}}}\markboth{tɯ-ri}{}\classe{clf}
\begin{définition}\fra cent\end{définition}
\begin{définition}\cmn 一百\end{définition}
\begin{relation-sémantique}\synonyme{
\hyperlink{Ⓔɣurʑa}{\textit{ \papi{ɣurʑa}}}
}\end{relation-sémantique}\end{entrée}

\begin{entrée}
\vedette{\hypertarget{Ⓔtɯ-rju}{\papi{ tɯ-rju}}}\markboth{tɯ-rju}{}\classe{np}
\begin{définition}\fra parole\end{définition}
\begin{définition}\cmn 话\end{définition}
\begin{exemple}\jya tɯ-rju to-nɤtsɯmɣɯt\cmn 他传播了谣言\end{exemple}
\begin{exemple}\jya tɯ-rju kɯ-ɕɤɣ tɯ-ŋka kɤ-spa-t-a\cmn 我学了一个新词\end{exemple}\end{entrée}

\begin{entrée}
\vedette{\hypertarget{Ⓔtɯrɟaʁ}{\papi{ tɯrɟaʁ}}}\markboth{tɯrɟaʁ}{}
\classe{n}
\begin{définition}\fra danse\end{définition}
\begin{définition}\cmn 舞蹈\end{définition}
\begin{exemple}\jya a-tɯrɟaʁ ci kɤ-fɕɤt\cmn 你给我跳一支舞\end{exemple}
\begin{exemple}\jya tɯrɟaʁ kɯ-rɲɟɯ-rɲɟi ʑo ko-rɤɕi-nɯ (ko-mtshi-nɯ)\cmn 他们跳舞的队伍拉得很长\end{exemple}
\begin{relation-sémantique}\confer{
\hyperlink{Ⓔrɟaʁ}{\textit{ \papi{rɟaʁ}}}
}\end{relation-sémantique}
\begin{relation-sémantique}\confer{
\hyperlink{ⒺfɕɤtⒽ1}{\textit{ \papi{fɕɤt1}}}
}\end{relation-sémantique}\end{entrée}

\begin{entrée}
\vedette{\hypertarget{Ⓔtɯ-rɟɯ}{\papi{ tɯ-rɟɯ}}}\markboth{tɯ-rɟɯ}{}\classe{np}
\begin{définition}\fra fortune\end{définition}
\begin{définition}\cmn 财富
\begin{déclaration} \étymologie{\papi{rgʲu}}\end{déclaration}\end{définition}
\begin{exemple}\jya tɯ-tsɣe kɤ-βzu ɣɯ ɯ-rɟɯ ɲɯ-rtaʁ\cmn 他有做生意的本钱\end{exemple}\end{entrée}

\begin{entrée}
\vedette{\hypertarget{Ⓔtɯ-rkɤn}{\papi{ tɯ-rkɤn}}}\markboth{tɯ-rkɤn}{}\classe{np}
\begin{définition}\fra palais\end{définition}
\begin{définition}\cmn 上腭
\begin{déclaration} \étymologie{\papi{rkan}}\end{déclaration}\end{définition}
\end{entrée}

\begin{entrée}
\vedette{\hypertarget{Ⓔtɯrkɤz}{\papi{ tɯrkɤz}}}\markboth{tɯrkɤz}{}\classe{n}
\begin{définition}\fra sculpture\end{définition}
\begin{définition}\cmn 雕塑
\begin{déclaration} \étymologie{\papi{rkos}}\end{déclaration}\end{définition}
\end{entrée}

\begin{entrée}
\vedette{\hypertarget{Ⓔtɯ-rkoŋɕɤl}{\papi{ tɯ-rkoŋɕɤl}}}\markboth{tɯ-rkoŋɕɤl}{}\classe{clf}
\begin{définition}\fra une are\end{définition}
\begin{définition}\cmn 一亩
\begin{déclaration} \étymologie{\papi{rkaŋ}}\end{déclaration}\end{définition}
\end{entrée}

\begin{entrée}
\vedette{\hypertarget{Ⓔtɯ-rla}{\papi{ tɯ-rla}}}\markboth{tɯ-rla}{}\classe{np}
\begin{définition}\fra âme, principe vital\end{définition}
\begin{définition}\cmn 灵魂,生命的根源\end{définition}
\begin{exemple}\jya tɯ-mu kɯ ɯ-rla ɲɤ-me\cmn 他吓坏了\end{exemple}\end{entrée}

\begin{entrée}
\vedette{\hypertarget{Ⓔtɯrma}{\papi{ tɯrma}}}\markboth{tɯrma}{}\classe{n}
\begin{définition}\fra vie quotidienne; tâches quotidiennes\end{définition}
\begin{définition}\cmn 过日子;日常生活\end{définition}
\begin{exemple}\jya tɯrma ko-ndo (tɯrma ɯʑoz ko-ndo)\cmn 他安家过日子(他离家在另一个地方过日子)\end{exemple}
\begin{exemple}\jya ndʑi-tɯrma nɯ wuma pjɤ-khɯ\cmn 他们俩日子过得很好\end{exemple}\end{entrée}

\begin{entrée}
\vedette{\hypertarget{Ⓔtɯrmbɣi}{\papi{ tɯrmbɣi}}}\markboth{tɯrmbɣi}{}\classe{n}
\begin{définition}\ 
\begin{déclaration}\grammar{n.lieu}\end{déclaration}\end{définition}
\begin{définition}\fra l'un des hameaux de Gyutshapa\end{définition}
\begin{définition}\cmn 二茶村的大队之一\end{définition}\end{entrée}

\begin{entrée}
\vedette{\hypertarget{Ⓔtɯ-rmbi}{\papi{ tɯ-rmbi}}}\markboth{tɯ-rmbi}{}\classe{np}
\begin{définition}\fra urine\end{définition}
\begin{définition}\cmn 尿\end{définition}
\end{entrée}

\begin{entrée}
\vedette{\hypertarget{Ⓔtɯ-rmbɯ}{\papi{ tɯ-rmbɯ}}}\markboth{tɯ-rmbɯ}{}\classe{clf}
\begin{définition}\fra tas\end{définition}
\begin{définition}\cmn 一堆\end{définition}
\begin{exemple}\jya tɯ-ɣli tɯ-tɯ-rmbɯ\cmn 一堆粪\end{exemple}
\begin{relation-sémantique}\confer{
\hyperlink{Ⓔrmbɯ}{\textit{ \papi{rmbɯ}}}
}\end{relation-sémantique}\end{entrée}

\begin{entrée}
\vedette{\hypertarget{Ⓔtɯrme}{\papi{ tɯrme}}}\markboth{tɯrme}{}\classe{n}\acception{1}
\begin{définition}\fra homme\end{définition}
\begin{définition}\cmn 人\end{définition}\acception{2}
\begin{définition}\fra quelqu'un d'autre\end{définition}
\begin{définition}\cmn 别人\end{définition}
\begin{exemple}\jya kɯki tɯrme ɣɯ ɯ-@shouji ŋu\cmn 这是别人的手机\end{exemple}\end{entrée}

\begin{entrée}
\vedette{\hypertarget{Ⓔtɯrmɯ}{\papi{ tɯrmɯ}}}\markboth{tɯrmɯ}{}
\classe{n}
\begin{définition}\fra après midi\end{définition}
\begin{définition}\cmn 下午\end{définition}
\begin{exemple}\jya nɤ-tɯrmɯ ko-ɣi\cmn 已经很晚,你来不及了\end{exemple}
\begin{exemple}\jya soz tɕi tu ŋgrɤl tɯrmɯ tɕi tu ŋgrɤl\cmn 早上也有,下午也有\end{exemple}
\begin{relation-sémantique}\confer{
\hyperlink{Ⓔnɯrmɯ}{\textit{ \papi{nɯrmɯ}}}
}\end{relation-sémantique}\end{entrée}

\begin{entrée}
\vedette{\hypertarget{Ⓔtɯrmɯkha}{\papi{ tɯrmɯkha}}}\markboth{tɯrmɯkha}{}\classe{n}
\begin{définition}\fra crépuscule\end{définition}
\begin{définition}\cmn 黄昏\end{définition}
\end{entrée}

\begin{entrée}
\vedette{\hypertarget{Ⓔtɯ-rna}{\papi{ tɯ-rna}}}\markboth{tɯ-rna}{}\classe{np}
\paradigme{\textit{comit :} \jya kɤ́rnɯrna}
\begin{définition}\fra oreille\end{définition}
\begin{définition}\cmn 耳朵
\begin{déclaration} \étymologie{\papi{rna}}\end{déclaration}\end{définition}
\begin{relation-sémantique}\confer{
\hyperlink{Ⓔmɤrnɤsɤŋo}{\textit{ \papi{mɤrnɤsɤŋo}}}
}\end{relation-sémantique}\end{entrée}

\begin{entrée}
\vedette{\hypertarget{Ⓔtɯ-rnamɕɤz}{\papi{ tɯ-rnamɕɤz}}}\markboth{tɯ-rnamɕɤz}{}\classe{np}
\begin{définition}\fra âme\end{définition}
\begin{définition}\cmn 灵魂
\begin{déclaration} \étymologie{\papi{rnam.ɕes}}\end{déclaration}\end{définition}
\end{entrée}

\begin{entrée}
\vedette{\hypertarget{Ⓔtɯ-rnɤfsɯr}{\papi{ tɯ-rnɤfsɯr}}}\markboth{tɯ-rnɤfsɯr}{}\classe{np}
\begin{définition}\fra partie poilue devant les oreilles\end{définition}
\begin{définition}\cmn 耳朵前面长小头发的部位\end{définition}\end{entrée}

\begin{entrée}
\vedette{\hypertarget{Ⓔtɯ-rnɤɣɲɟɯ}{\papi{ tɯ-rnɤɣɲɟɯ}}}\markboth{tɯ-rnɤɣɲɟɯ}{}\classe{np}
\begin{définition}\fra conduit auditif\end{définition}
\begin{définition}\cmn 耳孔\end{définition}
\begin{relation-sémantique}\confer{
\hyperlink{Ⓔtɯ-rna}{\textit{ \papi{tɯ-rna}}}
}\end{relation-sémantique}
\begin{relation-sémantique}\confer{
\hyperlink{Ⓔɯ-ɣɲɟɯ}{\textit{ \papi{ɯ-ɣɲɟɯ}}}
}\end{relation-sémantique}\end{entrée}

\begin{entrée}
\vedette{\hypertarget{Ⓔtɯ-rnɤpɤl}{\papi{ tɯ-rnɤpɤl}}}\markboth{tɯ-rnɤpɤl}{}\classe{np}
\begin{définition}\fra lobe de l'oreille\end{définition}
\begin{définition}\cmn 耳垂\end{définition}\end{entrée}

\begin{entrée}
\vedette{\hypertarget{Ⓔtɯrnda}{\papi{ tɯrnda}}}\markboth{tɯrnda}{}\classe{n}
\begin{définition}\fra partie en bois des maisons tibétains\end{définition}
\begin{définition}\cmn 藏式房屋的用木料做成的部分\end{définition}
\begin{exemple}\jya kɯrɯ kha ɯ-taʁ ɕoŋtɕa kɤ-ntɕhoz kɯ-kɯ-ra nɯ jɤɣɤt cho khɤxtu nɯ ra tɯrnda rmi\cmn 
藏式房屋上所有用木料做成的部分,如走缘、房背等都叫\stylefv{tɯrnda}
\end{exemple}
\end{entrée}

\begin{entrée}
\vedette{\hypertarget{Ⓔtɯrndaku}{\papi{ tɯrndaku}}}\markboth{tɯrndaku}{}\classe{n}
\begin{définition}\fra étage au dessus\end{définition}
\begin{définition}\cmn 楼上\end{définition}
\end{entrée}

\begin{entrée}
\vedette{\hypertarget{Ⓔtɯ-rni}{\papi{ tɯ-rni}}}\markboth{tɯ-rni}{}\classe{np}
\begin{définition}\fra gencive\end{définition}
\begin{définition}\cmn 牙龈\end{définition}
\end{entrée}

\begin{entrée}
\vedette{\hypertarget{Ⓔtɯ-rnom}{\papi{ tɯ-rnom}}}\markboth{tɯ-rnom}{}\classe{np}
\begin{définition}\fra côte\end{définition}
\begin{définition}\cmn 肋骨\end{définition}
\end{entrée}

\begin{entrée}
\vedette{\hypertarget{Ⓔtɯ-rnoʁ}{\papi{ tɯ-rnoʁ}}}\markboth{tɯ-rnoʁ}{}\classe{np}
\begin{définition}\fra cerveau\end{définition}
\begin{définition}\cmn 脑子\end{définition}
\begin{exemple}\jya a-rnoʁ ma-tɯ-ci\cmn 你不要(在我耳边)这么吵(令我受不了)\end{exemple}
\begin{relation-sémantique}\confer{
\hyperlink{Ⓔnɤrnoʁ}{\textit{ \papi{nɤrnoʁ}}}
}\end{relation-sémantique}\end{entrée}

\begin{entrée}
\vedette{\hypertarget{Ⓔtɯrŋu}{\papi{ tɯrŋu}}}\markboth{tɯrŋu}{}
\classe{n}
\begin{définition}\fra orge frit\end{définition}
\begin{définition}\cmn 炒青稞\end{définition}
\begin{relation-sémantique}\confer{
\hyperlink{Ⓔrŋu}{\textit{ \papi{rŋu}}}
}\end{relation-sémantique}\end{entrée}

\begin{entrée}
\vedette{\hypertarget{Ⓔtɯ-rŋa}{\papi{ tɯ-rŋa}}}\markboth{tɯ-rŋa}{}\classe{np}
\begin{définition}\fra visage\end{définition}
\begin{définition}\cmn 脸\end{définition}
\begin{exemple}\jya a-rŋa tɤ-chaβ-a ma ɲɯ-qiaβ\cmn 我做鬼脸因为很苦\end{exemple}
\begin{exemple}\jya tɯʑo tɯ-rŋa qambrɯ tɤ-ari kɯnɤ mɤ́-wɣ-nɯ-mto, tɯrme ra nɯ-rŋa zrɯɣ tɤ-ari kɯnɤ ɣɯ́-mto\cmn 容易看到别人的缺点,不容易看到自己的缺点\end{exemple}
\begin{relation-sémantique}\confer{
\hyperlink{Ⓔaɣɯrŋa}{\textit{ \papi{aɣɯrŋa}}}
}\end{relation-sémantique}
\begin{relation-sémantique}\confer{
\hyperlink{Ⓔɣɤrŋa}{\textit{ \papi{ɣɤrŋa}}}
}\end{relation-sémantique}
\begin{relation-sémantique}\confer{
\hyperlink{Ⓔanɯrŋɤrɯru}{\textit{ \papi{anɯrŋɤrɯru}}}
}\end{relation-sémantique}\end{entrée}

\begin{entrée}
\vedette{\hypertarget{Ⓔtɯrŋɤt}{\papi{ tɯrŋɤt}}}\markboth{tɯrŋɤt}{}
\classe{n}
\begin{définition}\fra piège\end{définition}
\begin{définition}\cmn 陷阱\end{définition}
\begin{exemple}\jya tɯrŋɤt pɯ-βzu-t-a\cmn 我设了陷阱\end{exemple}\end{entrée}

\begin{entrée}
\vedette{\hypertarget{Ⓔtɯ-ro}{\papi{ tɯ-ro}}}\markboth{tɯ-ro}{}\classe{np}\acception{1}
\begin{définition}\fra poitrine\end{définition}
\begin{définition}\cmn 胸膛
\begin{déclaration} \étymologie{\papi{braŋ}}\end{déclaration}\end{définition}
\begin{exemple}\jya ɯ-ro ɲɯ-ŋgɤr\cmn 他心胸狭窄(小气)\end{exemple}
\begin{exemple}\jya ɯ-ro ɲɯ-jom\cmn 他心胸广阔(不计较)\end{exemple}\acception{2}
\begin{définition}\fra colère\end{définition}
\begin{définition}\cmn (生的)气\end{définition}
\begin{exemple}\jya ɯ-ro lo-ɣi\cmn 他生气了\end{exemple}
\begin{exemple}\jya ɯ-ro jo-zɣɯt\cmn 他的坏脾气又发了\end{exemple}
\begin{exemple}\jya ɯ-ro ɲɤ-ʑi\cmn 他气消了\end{exemple}
\begin{exemple}\jya nɯ kɯnɤ ʑo ɯ-ro mɯ-pjɤ-k-ɤfɕu-ci\cmn 这样他都不解恨\end{exemple}\end{entrée}

\begin{entrée}
\vedette{\hypertarget{Ⓔtɯrpa}{\papi{ tɯrpa}}}\markboth{tɯrpa}{}\classe{n}
\begin{définition}\fra hache\end{définition}
\begin{définition}\cmn 斧头\end{définition}
\begin{relation-sémantique}\confer{
\hyperlink{Ⓔtɯ-tɯrpa}{\textit{ \papi{tɯ-tɯrpa}}}
}\end{relation-sémantique}
\end{entrée}

\begin{entrée}
\vedette{\hypertarget{Ⓔtɯ-rpaʁ}{\papi{ tɯ-rpaʁ}}}\markboth{tɯ-rpaʁ}{}\classe{np}
\begin{définition}\fra épaule\end{définition}
\begin{définition}\cmn 肩膀
\begin{déclaration} \étymologie{\papi{pʰrag}}\end{déclaration}\end{définition}
\end{entrée}

\begin{entrée}
\vedette{\hypertarget{Ⓔtɯ-rpɣo}{\papi{ tɯ-rpɣo}}}\markboth{tɯ-rpɣo}{}
\classe{np}
\begin{définition}\fra sur les cuisses (lorsqu'on est assis en tailleur)\end{définition}
\begin{définition}\cmn 盘着坐时,腿上的部位
\end{définition}
\begin{exemple}\jya a-rpɣo\cmn 我的腿上\end{exemple}
\begin{relation-sémantique}\synonyme{
\hyperlink{Ⓔtɯ-mbur}{\textit{ \papi{tɯ-mbur}}}
}\end{relation-sémantique}\end{entrée}

\begin{entrée}
\vedette{\hypertarget{Ⓔtɯ-rqɤpa}{\papi{ tɯ-rqɤpa}}}\markboth{tɯ-rqɤpa}{}\classe{np}
\begin{définition}\fra poitrine\end{définition}
\begin{définition}\cmn 胸膛(的上半部分)\end{définition}
\begin{exemple}\jya tɤ-pɤtso ɯ-rqɤpa ɲɯ-ɤci tɕe ɯ-mtʂɤkhoz ɲɯ-ra\cmn 小孩子的胸膛很湿,要给他带口水巾\end{exemple}
\begin{relation-sémantique}\confer{
\hyperlink{Ⓔtɯ-rqo}{\textit{ \papi{tɯ-rqo}}}
}\end{relation-sémantique}\end{entrée}

\begin{entrée}
\vedette{\hypertarget{Ⓔtɯ-rqo}{\papi{ tɯ-rqo}}}\markboth{tɯ-rqo}{}
\classe{np}
\begin{définition}\fra gorge\end{définition}
\begin{définition}\cmn 喉咙\end{définition}
\begin{exemple}\jya a-pɯ-sɤtso ra ma a-rqo ɲɯ-qhrɯt ʑo tɕe, mɤ-tɯ-tso thaŋ ɲɯ-sɯsam-a ŋu\cmn 希望听得清楚,因为我的喉咙倒了(嗓子哑了),我怕你听不清楚\end{exemple}\end{entrée}

\begin{entrée}
\vedette{\hypertarget{Ⓔtɯ-rqoloʁloʁ}{\papi{ tɯ-rqoloʁloʁ}}}\markboth{tɯ-rqoloʁloʁ}{}\classe{np}
\begin{définition}\fra pomme d'Adam\end{définition}
\begin{définition}\cmn 喉结\end{définition}
\end{entrée}

\begin{entrée}
\vedette{\hypertarget{Ⓔtɯ-rqopa}{\papi{ tɯ-rqopa}}}\markboth{tɯ-rqopa}{}\classe{np}
\begin{définition}\fra bas de la gorge\end{définition}
\begin{définition}\cmn 喉咙的下半部分\end{définition}
\begin{relation-sémantique}\confer{
\hyperlink{Ⓔtɯ-rqo}{\textit{ \papi{tɯ-rqo}}}
}\end{relation-sémantique}\end{entrée}

\begin{entrée}
\vedette{\hypertarget{Ⓔtɯ-rqorɣe}{\papi{ tɯ-rqorɣe}}}\markboth{tɯ-rqorɣe}{}\classe{np}
\begin{définition}\fra collier\end{définition}
\begin{définition}\cmn 项圈\end{définition}
\end{entrée}

\begin{entrée}
\vedette{\hypertarget{Ⓔtɯ-rqoʁ}{\papi{ tɯ-rqoʁ}}}\markboth{tɯ-rqoʁ}{}\classe{clf}
\begin{définition}\fra quantité qui peut tenir entre l'avant-bras et le bras fléchi\end{définition}
\begin{définition}\cmn 一抱\end{définition}
\begin{exemple}\jya si tɯ-rqoʁ tɤ-ɣɯt-a\cmn 我抱了一抱的柴\end{exemple}
\begin{relation-sémantique}\confer{
\hyperlink{Ⓔrqoʁ}{\textit{ \papi{rqoʁ}}}
}\end{relation-sémantique}\end{entrée}

\begin{entrée}
\vedette{\hypertarget{Ⓔtɯ-rqozbɤβ}{\papi{ tɯ-rqozbɤβ}}}\markboth{tɯ-rqozbɤβ}{}\classe{np}
\begin{définition}\fra goitre\end{définition}
\begin{définition}\cmn 甲状腺肿瘤\end{définition}
\end{entrée}

\begin{entrée}
\vedette{\hypertarget{Ⓔtɯrsa}{\papi{ tɯrsa}}}\markboth{tɯrsa}{}\classe{n}
\begin{définition}\fra tombe\end{définition}
\begin{définition}\cmn 坟墓
\begin{déclaration} \étymologie{\papi{dur.sa}}\end{déclaration}\end{définition}\end{entrée}

\begin{entrée}
\vedette{\hypertarget{Ⓔtɯrtɕhi}{\papi{ tɯrtɕhi}}}\markboth{tɯrtɕhi}{}\classe{n}
\begin{définition}\fra espèce de plante\end{définition}
\begin{définition}\cmn 【酸酸草】\end{définition}
\begin{exemple}\jya tɯrtɕhi nɯ sɯjno kɯ-mbɤr tsa ci ŋu, ɯ-zrɤm wuma ʑo rɲɟi dɤn, ɯ-jwaʁ nɯ kɯ-ɤrtɯm xɯrxɯr tɕe ɯ-rkɯ nɯ ra kɯ-rʁom tsa tu, ɯ-ʁɤri nɯ kɯ-ɤrŋi ŋu, ɯ-qhu nɯ kɯ-ɤrŋi tu kɯ-ɣɯrni tu, ɯ-ru tú-wɣ-ndza tɕe wuma zo tɕur, paʁ kɤ-mbi sna.\cmn 酸酸草是一种矮小的草。根又多又长。叶子是圆形的,边缘有些粗糙的小齿,正面是绿色的,背面有的是绿色的,有的是红色的。茎吃起来很酸,可以喂猪。\end{exemple}
\end{entrée}

\begin{entrée}
\vedette{\hypertarget{Ⓔtɯ-rti}{\papi{ tɯ-rti}}}\markboth{tɯ-rti}{}\classe{np}
\begin{définition}\fra jupe\end{définition}
\begin{définition}\cmn 裙子\end{définition}
\begin{relation-sémantique}\confer{
\hyperlink{Ⓔɯ-rti}{\textit{ \papi{ɯ-rti}}}
}\end{relation-sémantique}
\end{entrée}

\begin{entrée}
\vedette{\hypertarget{Ⓔtɯ-rtsa}{\papi{ tɯ-rtsa}}}\markboth{tɯ-rtsa}{}\classe{np}
\begin{définition}\fra pouls\end{définition}
\begin{définition}\cmn 脉搏
\begin{déclaration} \étymologie{\papi{rtsa}}\end{déclaration}\end{définition}
\end{entrée}

\begin{entrée}
\vedette{\hypertarget{Ⓔtɯ-rtsaku}{\papi{ tɯ-rtsaku}}}\markboth{tɯ-rtsaku}{}\classe{np}
\begin{définition}\fra point d'acupuncture\end{définition}
\begin{définition}\cmn 穴位
\begin{déclaration} \étymologie{\papi{rtsa}}\end{déclaration}\end{définition}
\end{entrée}

\begin{entrée}
\vedette{\hypertarget{Ⓔtɯ-rtsɤɣ}{\papi{ tɯ-rtsɤɣ}}}\markboth{tɯ-rtsɤɣ}{}
\classe{clf}
\begin{définition}\fra une section, un étage, une toise\end{définition}
\begin{définition}\cmn 一节;一层楼;一丈\end{définition}\end{entrée}

\begin{entrée}
\vedette{\hypertarget{Ⓔtɯ-rtshɤz}{\papi{ tɯ-rtshɤz}}}\markboth{tɯ-rtshɤz}{}\classe{np}
\begin{définition}\fra poumon\end{définition}
\begin{définition}\cmn 肺\end{définition}
\end{entrée}

\begin{entrée}
\vedette{\hypertarget{Ⓔtɯ-rtsi}{\papi{ tɯ-rtsi}}}\markboth{tɯ-rtsi}{}\classe{np}
\begin{définition}\fra nourriture pour bovidés\end{définition}
\begin{définition}\cmn 牛的食物\end{définition}\end{entrée}

\begin{entrée}
\vedette{\hypertarget{Ⓔtɯrtɯthɯ}{\papi{ tɯrtɯthɯ}}}\markboth{tɯrtɯthɯ}{}\classe{n}
\begin{définition}\fra tissu de lin\end{définition}
\begin{définition}\cmn 麻布\end{définition}\end{entrée}

\begin{entrée}
\vedette{\hypertarget{Ⓔtɯrɯm}{\papi{ tɯrɯm}}}\markboth{tɯrɯm}{}\classe{n}
\begin{définition}\fra sorte\end{définition}
\begin{définition}\cmn 种类\end{définition}
\begin{exemple}\jya jɯfɕɯr tɤ-rɯndzɤtshi tɕe jɯ-mgo zmɤrɤβ kɤntɕhɯ tɯrɯm pɯ-tu\cmn 昨天我们吃饭的时候,有好几种菜\end{exemple}
\end{entrée}

\begin{entrée}
\vedette{\hypertarget{Ⓔtɯ-rɯrtsɤɣ}{\papi{ tɯ-rɯrtsɤɣ}}}\markboth{tɯ-rɯrtsɤɣ}{}\classe{np}
\begin{définition}\fra articulation\end{définition}
\begin{définition}\cmn 关节
\begin{déclaration} \étymologie{\papi{tsʰigs}}\end{déclaration}\end{définition}
\end{entrée}

\begin{entrée}
\vedette{\hypertarget{Ⓔtɯ-rɯxpa}{\papi{ tɯ-rɯxpa}}}\markboth{tɯ-rɯxpa}{}\classe{np}
\begin{définition}\fra mémoire\end{définition}
\begin{définition}\cmn 记性
\begin{déclaration} \étymologie{\papi{rig.pa}}\end{déclaration}\end{définition}
\begin{exemple}\jya a-rɯxpa mɯ-ɲo-sna\cmn 我记性不好\end{exemple}
\end{entrée}

\begin{entrée}
\vedette{\hypertarget{Ⓔtɯ-rzɤz}{\papi{ tɯ-rzɤz}}}\markboth{tɯ-rzɤz}{}\classe{np}
\begin{définition}\ 
\begin{déclaration}\use{古语}\end{déclaration}\end{définition}\acception{1}
\begin{définition}\fra bagages (cadeau que l'on offre avant le départ)\end{définition}
\begin{définition}\cmn 行李\end{définition}
\begin{relation-sémantique}\synonyme{
\hyperlink{Ⓔtɤ-rkuz}{\textit{ \papi{tɤ-rkuz}}}
}\end{relation-sémantique}\acception{2}
\begin{définition}\fra dot\end{définition}
\begin{définition}\cmn 嫁妆
\begin{déclaration} \étymologie{\papi{rdzas}}\end{déclaration}\end{définition}\end{entrée}

\begin{entrée}
\vedette{\hypertarget{Ⓔtɯ-rzɯɣ}{\papi{ tɯ-rzɯɣ}}}\markboth{tɯ-rzɯɣ}{}
\classe{clf}\acception{1}
\begin{définition}\fra section\end{définition}
\begin{définition}\cmn 一段\end{définition}\acception{2}
\begin{définition}\fra un moment\end{définition}
\begin{définition}\cmn 一会儿\end{définition}
\begin{exemple}\jya nɤʑo pɤjkhu tɯ-rzɯɣ tɤ-nɯna tɕe tɕetha rɤma-tɕi\cmn 你暂时休息一下,等一会我们再工作\end{exemple}
\begin{exemple}\jya tʂu tɯ-rzɯɣ\cmn 一段路\end{exemple}
\begin{exemple}\jya ɟu tɯ-rzɯɣ\cmn 一节竹子\end{exemple}
\begin{relation-sémantique}\confer{
 \papi{rɤrzɯrzɯɣ}
}\end{relation-sémantique}\end{entrée}

\begin{entrée}
\vedette{\hypertarget{Ⓔtɯ-rʑaβspa}{\papi{ tɯ-rʑaβspa}}}\markboth{tɯ-rʑaβspa}{}\classe{np}
\begin{définition}\fra fiancée\end{définition}
\begin{définition}\cmn 未婚妻\end{définition}
\end{entrée}

\begin{entrée}
\vedette{\hypertarget{Ⓔtɯ-ʁɤjtshɯz}{\papi{ tɯ-ʁɤjtshɯz}}}\markboth{tɯ-ʁɤjtshɯz}{}\classe{np}
\begin{définition}\fra avoir une utilité\end{définition}
\begin{définition}\cmn 发挥作用\end{définition}
\begin{exemple}\jya tɤtʂu nɯ ɕɤr tɕe tɯ-ʁɤjtshɯz ɲɯ-ɕe\cmn 晚上的时候,灯发挥很大的作用\end{exemple}
\begin{exemple}\jya nɤ-rɟit nɯ ju-nɯ-ɕe ɲɯ-ɕti tɕe, nɤ-ʁɤjtshɯz mɯ́j-ɕe\cmn 你的孩子离开家了,对你没有用了(不会再照顾你了)\end{exemple}\end{entrée}

\begin{entrée}
\vedette{\hypertarget{Ⓔtɯ-ʁɤriɕɣa}{\papi{ tɯ-ʁɤriɕɣa}}}\markboth{tɯ-ʁɤriɕɣa}{}\classe{np}
\begin{définition}\fra incisives\end{définition}
\begin{définition}\cmn 门牙\end{définition}
\end{entrée}

\begin{entrée}
\vedette{\hypertarget{Ⓔtɯ-ʁɤt}{\papi{ tɯ-ʁɤt}}}\markboth{tɯ-ʁɤt}{}\classe{np}
\begin{définition}\fra capacité de travail\end{définition}
\begin{définition}\cmn (工作或办事的)能力\end{définition}
\begin{exemple}\jya a-ʁɤt maŋe\cmn 我没有能力做那么多事情\end{exemple}
\end{entrée}

\begin{entrée}
\vedette{\hypertarget{Ⓔtɯ-ʁjiz,ɣi}{\papi{ tɯ-ʁjiz,ɣi}}}\markboth{tɯ-ʁjiz,ɣi}{}\classe{np}
\begin{définition}\fra avoir envie\end{définition}
\begin{définition}\cmn 想吃,想要\end{définition}
\begin{exemple}\jya a-ʁjiz mɯ́j-ɣi\cmn 我不想\end{exemple}
\begin{relation-sémantique}\ComponentA{\classe{np}
 \papi{tɯ-ʁjiz}
}\end{relation-sémantique}
\begin{relation-sémantique}\ComponentB{\classe{vi}
\hyperlink{Ⓔɣi}{\textit{ \papi{ɣi}}}
}\end{relation-sémantique}\end{entrée}

\begin{entrée}
\vedette{\hypertarget{Ⓔtɯ-ʁla}{\papi{ tɯ-ʁla}}}\markboth{tɯ-ʁla}{}\classe{np}
\begin{définition}\fra avant-bras\end{définition}
\begin{définition}\cmn 胳膊\end{définition}
\end{entrée}

\begin{entrée}
\vedette{\hypertarget{Ⓔtɯ-ʁnoŋ}{\papi{ tɯ-ʁnoŋ}}}\markboth{tɯ-ʁnoŋ}{}
\classe{np}
\begin{définition}\fra remords\end{définition}
\begin{définition}\cmn 内疚
\begin{déclaration} \étymologie{\papi{gnoŋ}}\end{déclaration}\end{définition}
\begin{exemple}\jya a-ʁnoŋ me\cmn 我问心无愧\end{exemple}
\begin{exemple}\jya nɤ-ʁnoŋ maŋe\cmn 你问心无愧\end{exemple}
\begin{relation-sémantique}\confer{
\hyperlink{Ⓔnɯʁnoŋ}{\textit{ \papi{nɯʁnoŋ}}}
}\end{relation-sémantique}\end{entrée}

\begin{entrée}
\vedette{\hypertarget{Ⓔtɯ-ʁɲɯrgɯm}{\papi{ tɯ-ʁɲɯrgɯm}}}\markboth{tɯ-ʁɲɯrgɯm}{}\classe{np}
\begin{définition}\fra pourtour des yeux\end{définition}
\begin{définition}\cmn 眼圈骨\end{définition}
\begin{exemple}\jya nɤ-ʁɲɯrgɯm chɯ-tɯ-phɤβ\cmn 你的脸色变了,你的表情变了。\end{exemple}
\end{entrée}

\begin{entrée}
\vedette{\hypertarget{Ⓔtɯ-ʁo,mbi}{\papi{ tɯ-ʁo,mbi}}}\markboth{tɯ-ʁo,mbi}{}
\paradigme{\textit{dir :} \jya nɯ-}
\begin{définition}\ 
\begin{déclaration}\grammar{acaus}\end{déclaration}\end{définition}
\begin{définition}\fra regretter, ne pas être satisfait, être déçu\end{définition}
\begin{définition}\cmn 介意;灰心;失望\end{définition}
\begin{exemple}\jya a-ʁo mɤ-mbi\cmn 我不介意\end{exemple}
\begin{exemple}\jya a-ʁo ɲɯ-mbi ɕti\cmn 我很失望\end{exemple}
\begin{relation-sémantique}\confer{
\hyperlink{Ⓔnɤʁombi}{\textit{ \papi{nɤʁombi}}}
}\end{relation-sémantique}
\begin{relation-sémantique}\confer{
\hyperlink{Ⓔtɯ-ʁo,phi}{\textit{ \papi{tɯ-ʁo,phi}}}
}\end{relation-sémantique}
\begin{relation-sémantique}\ComponentA{\classe{np}
 \papi{tɯ-ʁo}
}\end{relation-sémantique}
\begin{relation-sémantique}\ComponentB{\classe{vi}
\hyperlink{Ⓔmbi}{\textit{ \papi{mbi}}}
}\end{relation-sémantique}\end{entrée}

\begin{entrée}
\vedette{\hypertarget{Ⓔtɯ-ʁo,phi}{\papi{ tɯ-ʁo,phi}}}\markboth{tɯ-ʁo,phi}{}\begin{définition}\fra décevoir\end{définition}
\begin{définition}\cmn 令……失望\end{définition}
\begin{exemple}\jya ɯʑo kɯ a-ʁo ɲɯ-phi\cmn 我对他很失望\end{exemple}
\begin{exemple}\jya a-ʁo ɲɯ-tɯ-phi\cmn 我对你很失望\end{exemple}
\begin{exemple}\jya a-ʁo ɲɯ-nɯ-phi-a\cmn 我对我自己很失望\end{exemple}
\begin{relation-sémantique}\confer{
\hyperlink{Ⓔtɯ-ʁo,mbi}{\textit{ \papi{tɯ-ʁo,mbi}}}
}\end{relation-sémantique}
\begin{relation-sémantique}\ComponentA{\classe{np}
 \papi{tɯ-ʁo}
}\end{relation-sémantique}
\begin{relation-sémantique}\ComponentB{\classe{vt}
 \papi{phi}
}\end{relation-sémantique}\end{entrée}

\begin{entrée}
\vedette{\hypertarget{Ⓔtɯ-ʁwaŋ}{\papi{ tɯ-ʁwaŋ}}}\markboth{tɯ-ʁwaŋ}{}\classe{np}
\begin{définition}\fra pouvoir\end{définition}
\begin{définition}\cmn 权力
\begin{déclaration} \étymologie{\papi{dbaŋ}}\end{déclaration}\end{définition}
\end{entrée}

\begin{entrée}
\vedette{\hypertarget{Ⓔtɯ-sa}{\papi{ tɯ-sa}}}\markboth{tɯ-sa}{}\classe{np}
\begin{définition}\fra dents de sagesse\end{définition}
\begin{définition}\cmn 智齿
\end{définition}
\end{entrée}

\begin{entrée}
\vedette{\hypertarget{Ⓔtɯ-sɤsci}{\papi{ tɯ-sɤsci}}}\markboth{tɯ-sɤsci}{}
\classe{np}
\begin{définition}\fra anniversaire, lieu de naissance\end{définition}
\begin{définition}\cmn 生日;出生的地方
\begin{déclaration} \étymologie{\papi{skʲe}}\end{déclaration}\end{définition}
\begin{exemple}\jya nɤ-sɤsci ŋotɕu pɯ-ŋu?\cmn 你生在哪里?\end{exemple}\end{entrée}

\begin{entrée}
\vedette{\hypertarget{Ⓔtɯ-sɤxɕe}{\papi{ tɯ-sɤxɕe}}}\markboth{tɯ-sɤxɕe}{}\classe{np}
\begin{définition}\fra direction, but vers lequel on va\end{définition}
\begin{définition}\cmn 去向;去的目标\end{définition}
\begin{relation-sémantique}\confer{
\hyperlink{Ⓔɕe}{\textit{ \papi{ɕe}}}
}\end{relation-sémantique}
\end{entrée}

\begin{entrée}
\vedette{\hypertarget{Ⓔtɯ-scur}{\papi{ tɯ-scur}}}\markboth{tɯ-scur}{}\classe{np}
\begin{définition}\fra creux de la main\end{définition}
\begin{définition}\cmn 手掌心\end{définition}
\end{entrée}

\begin{entrée}
\vedette{\hypertarget{Ⓔtɯsi}{\papi{ tɯsi}}}\markboth{tɯsi}{}\classe{n}
\begin{définition}\fra mort\end{définition}
\begin{définition}\cmn 死亡\end{définition}
\begin{exemple}\jya tɯsi kɤ-nɤjo ɕti\cmn 只有等死\end{exemple}
\begin{exemple}\jya ɯ-tɯsi mda\cmn 他的死期到了\end{exemple}
\begin{relation-sémantique}\confer{
\hyperlink{ⒺsiⒽ1}{\textit{ \papi{si1}}}
}\end{relation-sémantique}\end{entrée}

\begin{entrée}
\vedette{\hypertarget{Ⓔtɯ-skɤlwa}{\papi{ tɯ-skɤlwa}}}\markboth{tɯ-skɤlwa}{}\classe{np}
\begin{définition}\fra partie\end{définition}
\begin{définition}\cmn 份
\begin{déclaration} \étymologie{\papi{skal.ba}}\end{déclaration}\end{définition}
\end{entrée}

\begin{entrée}
\vedette{\hypertarget{Ⓔtɯ-skɤt}{\papi{ tɯ-skɤt}}}\markboth{tɯ-skɤt}{}\classe{np}
\begin{définition}\fra voix\end{définition}
\begin{définition}\cmn 声音\end{définition}
\begin{définition}\fra parole\end{définition}
\begin{définition}\cmn 话\end{définition}
\begin{définition}\fra language\end{définition}
\begin{définition}\cmn 语言
\begin{déclaration} \étymologie{\papi{skad}}\end{déclaration}\end{définition}
\begin{exemple}\jya kɤ-nɯrɤɣo ɯ-skɤt ɲɯ-sna\cmn 他唱歌的声音\end{exemple}
\begin{exemple}\jya tɕhi ɯ-skɤt ɲɯ-ŋu, ɯ-ɲɯ-tɯ-tso\cmn 你懂不懂什么意思?\end{exemple}
\begin{exemple}\jya mbro tɯ-skɤt kɯ-tso ci, a-mbro tɤ-rku-nɯ ra\cmn 给我准备一匹能听懂人话的马\end{exemple}
\begin{exemple}\jya aʑo kɯ-ŋɤn ɯ-skɤt ntsɯ tu-βze ɲɯ-ŋu\cmn 他总是说我的坏话\end{exemple}\begin{sous-entrée}
\vedette{\hypertarget{}{\papi{ tɯ-skɤt,βzu}}}\markboth{tɯ-skɤt,βzu}{}
\begin{définition}\fra imiter\end{définition}
\begin{définition}\cmn 模仿;装作\end{définition}
\begin{exemple}\jya tɕaɣi nɯ pjɯ́-wɣ-sɯxɕɤt tɕe, tɯrme ɯ-skɤt tu-βze spe tu-ti-nɯ ŋgrɤl\cmn 如果教它的话,鹦鹉会模仿人话\end{exemple}
\begin{relation-sémantique}\ComponentA{\classe{np}
\hyperlink{Ⓔtɯ-skɤt}{\textit{ \papi{tɯ-skɤt}}}
}\end{relation-sémantique}
\begin{relation-sémantique}\ComponentB{\classe{vt}
\hyperlink{ⒺβzuⒽ1}{\textit{ \papi{βzu}}}
}\end{relation-sémantique}
\begin{relation-sémantique}\confer{
\hyperlink{ⒺβzuⒽ1}{\textit{ \papi{βzu1}}}
}\end{relation-sémantique}
\end{sous-entrée}\end{entrée}

\begin{entrée}
\vedette{\hypertarget{Ⓔtɯ-skhrɯ}{\papi{ tɯ-skhrɯ}}}\markboth{tɯ-skhrɯ}{}\classe{np}
\begin{définition}\fra corps\end{définition}
\begin{définition}\cmn 身体\end{définition}
\begin{exemple}\jya ɯ-skhrɯ mɯ-pjɤ-βdi\cmn 她怀孕了\end{exemple}\end{entrée}

\begin{entrée}
\vedette{\hypertarget{Ⓔtɯ-skhɯrdoʁ}{\papi{ tɯ-skhɯrdoʁ}}}\markboth{tɯ-skhɯrdoʁ}{}\classe{np}
\begin{définition}\fra testicules\end{définition}
\begin{définition}\cmn 睾丸\end{définition}\end{entrée}

\begin{entrée}
\vedette{\hypertarget{Ⓔtɯ-sla}{\papi{ tɯ-sla}}}\markboth{tɯ-sla}{}\classe{clf}
\begin{définition}\fra un mois\end{définition}
\begin{définition}\cmn 一个月\end{définition}
\begin{relation-sémantique}\confer{
\hyperlink{Ⓔsla}{\textit{ \papi{sla}}}
}\end{relation-sémantique}
\begin{relation-sémantique}\confer{
\hyperlink{Ⓔkɤrɤsla}{\textit{ \papi{kɤrɤsla}}}
}\end{relation-sémantique}
\begin{relation-sémantique}\confer{
\hyperlink{Ⓔslɤŋe}{\textit{ \papi{slɤŋe}}}
}\end{relation-sémantique}
\end{entrée}

\begin{entrée}
\vedette{\hypertarget{Ⓔtɯ-smɤt}{\papi{ tɯ-smɤt}}}\markboth{tɯ-smɤt}{}\classe{np}
\begin{définition}\fra le bas du corps\end{définition}
\begin{définition}\cmn 下半身
\begin{déclaration} \étymologie{\papi{smad}}\end{déclaration}\end{définition}
\begin{relation-sémantique}\confer{
\hyperlink{Ⓔsmɤʁjoʁ}{\textit{ \papi{smɤʁjoʁ}}}
}\end{relation-sémantique}\end{entrée}

\begin{entrée}
\vedette{\hypertarget{Ⓔtɯ-snaʁ}{\papi{ tɯ-snaʁ}}}\markboth{tɯ-snaʁ}{}\classe{clf}
\begin{définition}\fra morceau\end{définition}
\begin{définition}\cmn 一小块\end{définition}
\begin{exemple}\jya tɯ-ji tɯ-snaʁ\cmn 一块地\end{exemple}
\end{entrée}

\begin{entrée}
\vedette{\hypertarget{Ⓔtɯ-sni}{\papi{ tɯ-sni}}}\markboth{tɯ-sni}{}\classe{np}
\begin{définition}\fra cœur\end{définition}
\begin{définition}\cmn 心脏\end{définition}
\begin{exemple}\jya ɯ-sni ɲɯ-xtɕi\cmn 他很胆小\end{exemple}
\begin{exemple}\jya ɯ-sni ɲɤ-zdɯɣ\cmn 他很伤心\end{exemple}
\begin{relation-sémantique}\confer{
\hyperlink{Ⓔnɤsnɯndo}{\textit{ \papi{nɤsnɯndo}}}
}\end{relation-sémantique}\end{entrée}

\begin{entrée}
\vedette{\hypertarget{Ⓔtɯ-sŋi}{\papi{ tɯ-sŋi}}}\markboth{tɯ-sŋi}{}
\classe{clf}
\begin{définition}\fra un jour\end{définition}
\begin{définition}\cmn 一天\end{définition}
\begin{exemple}\jya wo, nɯ ɯ-sŋi ʑo tɕe a-pɯ-ŋu\cmn 就在那一天吧!\end{exemple}\end{entrée}

\begin{entrée}
\vedette{\hypertarget{Ⓔtɯsŋi ɲɯntaʁ}{\papi{ tɯsŋi ɲɯntaʁ}}}\markboth{tɯsŋi ɲɯntaʁ}{}\classe{adv}
\begin{définition}\fra un jour entier\end{définition}
\begin{définition}\cmn 一整天\end{définition}\end{entrée}

\begin{entrée}
\vedette{\hypertarget{Ⓔtɯ-sŋɯro}{\papi{ tɯ-sŋɯro}}}\markboth{tɯ-sŋɯro}{}
\classe{np}
\begin{définition}\fra souffle\end{définition}
\begin{définition}\cmn 气(一口)\end{définition}
\begin{exemple}\jya a-sŋɯro ci lɤ-tɕat-a\cmn 我呼了气\end{exemple}
\begin{exemple}\jya a-sŋɯro ci lɤ-rɤɕi-t-a\cmn 我吸了一口气\end{exemple}
\begin{exemple}\jya ɯ-sŋɯro kɤ-lɤt ɲɤ-nɤɕqa\cmn 他屏住了呼吸\end{exemple}
\begin{exemple}\jya ɯ-sŋɯro lo-k-ɤrɕo-ci (=ɯ-sroʁ lo-mbrɤt)\cmn 他断气了\end{exemple}\end{entrée}

\begin{entrée}
\vedette{\hypertarget{Ⓔtɯ-spra}{\papi{ tɯ-spra}}}\markboth{tɯ-spra}{}\classe{clf}
\begin{définition}\fra poignée\end{définition}
\begin{définition}\cmn 一捧\end{définition}
\begin{exemple}\jya mbrɤz tɯ-spra\cmn 一捧米\end{exemple}
\begin{exemple}\jya tɯ-jaʁ ɯ-spra\cmn 双手捧东西的姿势\end{exemple}\end{entrée}

\begin{entrée}
\vedette{\hypertarget{Ⓔtɯsqa}{\papi{ tɯsqa}}}\markboth{tɯsqa}{}
\classe{n}
\begin{définition}\fra soupe d'avoine, de blé et de haricots que l'on donne aux enfants et à ceux qui récupèrent les excréments de chevaux pour faire de l'engrais\end{définition}
\begin{définition}\cmn 粥
\begin{déclaration}\use{把莜麦、小麦和豌豆煮在一起,再加一些香料,给小孩子和出圈肥的人吃}\end{déclaration}\end{définition}
\begin{exemple}\jya tɯsqa kɤ-βzu-t-a\cmn 我熬了粥\end{exemple}
\begin{exemple}\jya qaj tɤɕi kú-wɣ-sqa tɕe, nɯnɯ tɯsqa tu-kɯ-ti ɲɯ-ŋu\cmn 煮了小麦和青稞,那叫做粥\end{exemple}
\begin{relation-sémantique}\confer{
\hyperlink{Ⓔsqa}{\textit{ \papi{sqa}}}
}\end{relation-sémantique}
\begin{relation-sémantique}\confer{
\hyperlink{Ⓔrɯtɯsqa}{\textit{ \papi{rɯtɯsqa}}}
}\end{relation-sémantique}\end{entrée}

\begin{entrée}
\vedette{\hypertarget{Ⓔtɯsqar}{\papi{ tɯsqar}}}\markboth{tɯsqar}{}\classe{n}
\begin{définition}\fra tsampa\end{définition}
\begin{définition}\cmn 糌粑\end{définition}\end{entrée}

\begin{entrée}
\vedette{\hypertarget{Ⓔtɯ-sraŋ}{\papi{ tɯ-sraŋ}}}\markboth{tɯ-sraŋ}{}\classe{clf}
\begin{définition}\fra once\end{définition}
\begin{définition}\cmn 一两
\begin{déclaration} \étymologie{\papi{sraŋ}}\end{déclaration}\end{définition}
\end{entrée}

\begin{entrée}
\vedette{\hypertarget{Ⓔtɯ-srɤm}{\papi{ tɯ-srɤm}}}\markboth{tɯ-srɤm}{}\classe{np}
\begin{définition}\fra lignée\end{définition}
\begin{définition}\cmn 人的根源(祖宗,财富等)\end{définition}
\begin{relation-sémantique}\confer{
\hyperlink{Ⓔtɤ-zrɤm}{\textit{ \papi{tɤ-zrɤm}}}
}\end{relation-sémantique}
\end{entrée}

\begin{entrée}
\vedette{\hypertarget{Ⓔtɯ-sroʁ}{\papi{ tɯ-sroʁ}}}\markboth{tɯ-sroʁ}{}\classe{np}
\begin{définition}\fra vie\end{définition}
\begin{définition}\cmn 生命
\begin{déclaration} \étymologie{\papi{srog}}\end{déclaration}\end{définition}
\begin{exemple}\jya a-sroʁ ngɯt\cmn 我命大(不容易死)\end{exemple}
\begin{exemple}\jya jla nɯ ŋkhorwapa ra nɯ-sroʁ ɯ-kɯ-ndo pɯ-ŋu\cmn 犏牛是农民们的命根子\end{exemple}
\begin{exemple}\jya ɯ-sroʁ to-tɕɤt\cmn 让他丧命了\end{exemple}\begin{sous-entrée}
\vedette{\hypertarget{}{\papi{ tɯ-sroʁ,ɕe}}}\markboth{tɯ-sroʁ,ɕe}{}
\paradigme{\textit{dir :} \jya pɯ-}
\begin{définition}\fra perdre la vie\end{définition}
\begin{définition}\cmn 丧命\end{définition}
\begin{exemple}\jya nɯŋa dɯxpa ma ɯ-sroʁ pjɤ-ɕe, tɯ-ci ri mɯ-pjɤ-k-ɤʁe-ci\cmn 奶牛很可怜,(去喝水的时候)丧了命,一口水也没有喝到\end{exemple}
\begin{relation-sémantique}\ComponentA{\classe{np}
\hyperlink{Ⓔtɯ-sroʁ}{\textit{ \papi{tɯ-sroʁ}}}
}\end{relation-sémantique}
\begin{relation-sémantique}\ComponentB{\classe{vi}
\hyperlink{Ⓔɕe}{\textit{ \papi{ɕe}}}
}\end{relation-sémantique}
\end{sous-entrée}\begin{sous-entrée}
\vedette{\hypertarget{}{\papi{ tɯ-sroʁ,mbrɤt}}}\markboth{tɯ-sroʁ,mbrɤt}{}
\paradigme{\textit{dir :} \jya lɤ-}
\begin{définition}\fra expirer, rendre le dernier soupir\end{définition}
\begin{définition}\cmn 断气\end{définition}
\begin{relation-sémantique}\confer{
\hyperlink{Ⓔnɯsroʁmbrɤt}{\textit{ \papi{nɯsroʁmbrɤt}}}
}\end{relation-sémantique}
\end{sous-entrée}\end{entrée}

\begin{entrée}
\vedette{\hypertarget{Ⓔtɯ-sroʁ,nɯwɤtku}{\papi{ tɯ-sroʁ,nɯwɤtku}}}\markboth{tɯ-sroʁ,nɯwɤtku}{}
\paradigme{\textit{dir :} \jya tɤ-}
\begin{définition}\fra risquer sa vie\end{définition}
\begin{définition}\cmn 冒生命危险\end{définition}
\begin{exemple}\jya a-sroʁ ku-nɯwɤtke-a ɕti\cmn 我正冒着生命危险\end{exemple}
\begin{exemple}\jya a-sroʁ kɤ-nɯwɤtku ʑo kutɕu jɤ-ɣe-a ɕti\cmn 我冒着生命危险来到这里\end{exemple}
\begin{relation-sémantique}\ComponentA{\classe{np}
\hyperlink{Ⓔtɯ-sroʁ}{\textit{ \papi{tɯ-sroʁ}}}
}\end{relation-sémantique}
\begin{relation-sémantique}\ComponentB{\classe{vt}
 \papi{nɯwɤtku}
}\end{relation-sémantique}
\begin{relation-sémantique}\confer{
\hyperlink{Ⓔtɯ-wɤtku}{\textit{ \papi{tɯ-wɤtku}}}
}\end{relation-sémantique}\end{entrée}

\begin{entrée}
\vedette{\hypertarget{Ⓔtɯ-sroʁ,ri}{\papi{ tɯ-sroʁ,ri}}}\markboth{tɯ-sroʁ,ri}{}\classe{vt}
\paradigme{\textit{dir :} \jya kɤ-}
\begin{définition}\fra sauver\end{définition}
\begin{définition}\cmn 救\end{définition}
\begin{exemple}\jya nɤ-sroʁ kɤ-ri-t-a\cmn 我救了你的命\end{exemple}
\begin{relation-sémantique}\ComponentA{\classe{np}
\hyperlink{Ⓔtɯ-sroʁ}{\textit{ \papi{tɯ-sroʁ}}}
}\end{relation-sémantique}
\begin{relation-sémantique}\ComponentB{\classe{vt}
\hyperlink{ⒺriⒽ1}{\textit{ \papi{ri}}}
}\end{relation-sémantique}
\begin{relation-sémantique}\synonyme{
\hyperlink{Ⓔfsraŋ}{\textit{ \papi{fsraŋ}}}
}\end{relation-sémantique}
\begin{relation-sémantique}\confer{
\hyperlink{ⒺriⒽ3}{\textit{ \papi{ri3}}}
}\end{relation-sémantique}\end{entrée}

\begin{entrée}
\vedette{\hypertarget{Ⓔtɯ-srɯt}{\papi{ tɯ-srɯt}}}\markboth{tɯ-srɯt}{}\classe{clf}
\begin{définition}\fra un filet de lumière\end{définition}
\begin{définition}\cmn 细小缝隙里透过的光\end{définition}
\begin{exemple}\jya tɤŋe tɯ-srɯt\cmn 一丝阳光\end{exemple}\end{entrée}

\begin{entrée}
\vedette{\hypertarget{Ⓔtɯ-sta}{\papi{ tɯ-sta}}}\markboth{tɯ-sta}{}\classe{np}\acception{1}
\begin{définition}\fra lit\end{définition}
\begin{définition}\cmn 床位(睡过的地方)\end{définition}
\begin{exemple}\jya a-sta ku-βze-a\cmn 我在铺床\end{exemple}
\begin{exemple}\jya a-sta na-βzu\cmn 他给我留了床位(座位)\end{exemple}\acception{2}
\begin{définition}\fra place\end{définition}
\begin{définition}\cmn 位子;原位(坐过的地方)\end{définition}
\begin{exemple}\jya khɯtsa a-tʂha pɯ-kɤ-rku ɯ-sta nɯ tɕu a-tɯjno tɤ-rke\cmn 在我原来喝过茶的那个碗里给我装菜\end{exemple}
\begin{exemple}\jya jɤ-nɯ-ɣi jɤɣ ma nɤ-sta nɯ-βzu-t-a\cmn 你来吧,我给你让位了\end{exemple}
\begin{exemple}\jya nɤ-sta tɤ-nɯ-fse\cmn 你规矩一下!;你恢复原来的面目吧!\end{exemple}
\begin{relation-sémantique}\confer{
\hyperlink{Ⓔtɤ-sta}{\textit{ \papi{tɤ-sta}}}
}\end{relation-sémantique}
\begin{relation-sémantique}\confer{
\hyperlink{Ⓔɯ-sta}{\textit{ \papi{ɯ-sta}}}
}\end{relation-sémantique}\end{entrée}

\begin{entrée}
\vedette{\hypertarget{Ⓔtɯ-staʁ}{\papi{ tɯ-staʁ}}}\markboth{tɯ-staʁ}{}
\classe{clf}
\begin{définition}\fra une poêlée\end{définition}
\begin{définition}\cmn 一锅,炒出一锅糌粑的时间\end{définition}
\begin{exemple}\jya tɯ-staʁ thɯ-rŋu-t-a\cmn 我炒了一锅糌粑\end{exemple}
\begin{exemple}\jya tɯ-staʁ-rŋu jamar pɯ-atsɯtsu\cmn 过了炒一锅糌粑的时间\end{exemple}\end{entrée}

\begin{entrée}
\vedette{\hypertarget{Ⓔtɯ-stɤt}{\papi{ tɯ-stɤt}}}\markboth{tɯ-stɤt}{}\classe{np}
\begin{définition}\fra le haut du corps\end{définition}
\begin{définition}\cmn 上半身
\begin{déclaration} \étymologie{\papi{stot}}\end{déclaration}\end{définition}
\begin{exemple}\jya tɯ-stɤt kɤ-βde\cmn 脱下藏装右手的袖子(便于做事)\end{exemple}
\begin{relation-sémantique}\confer{
\hyperlink{Ⓔstɤnga}{\textit{ \papi{stɤnga}}}
}\end{relation-sémantique}\end{entrée}

\begin{entrée}
\vedette{\hypertarget{Ⓔtɯ-sɯm}{\papi{ tɯ-sɯm}}}\markboth{tɯ-sɯm}{}
\classe{np}
\begin{définition}\fra état d'esprit\end{définition}
\begin{définition}\cmn 性情
\begin{déclaration} \étymologie{\papi{sems}}\end{déclaration}\end{définition}
\begin{exemple}\jya ɯ-sɯm wuma ʑo ŋɤn\cmn 他疑心很重\end{exemple}
\begin{exemple}\jya a-sɯm mba, kɤ-rɤntɕha mɤ-cha-a\cmn 我心很软,不能宰猪\end{exemple}
\begin{exemple}\jya ɯ-sɯm mɯ́j-βdi\cmn 他不放心\end{exemple}
\begin{exemple}\jya ɯ-sɯm ɲɯ-ɤstu\cmn 他很正直,很忠诚\end{exemple}
\begin{exemple}\jya ɯ-sɯm kɯ-sna tu-βze ɲɯ-ŋu\cmn 他安的是好心\end{exemple}
\begin{exemple}\jya aʑo a-sɯm tɕe ...\cmn 我认为,……\end{exemple}
\begin{exemple}\jya ɯ-sɯm kɤ-χtɤt ʑo pjɯ-rɤβzjoz ŋu\cmn 他很专心,一心一意地读书\end{exemple}
\begin{exemple}\jya ɯ-sɯm lu-mtshɤt kɯ-me ci pjɤ-ŋu\cmn 他是个贪得无厌的人\end{exemple}
\begin{exemple}\jya aʑo nɯtɕu tɕe nɤ-ɕki ɣi-a tɕe, nɤ-sɯm pjɯ-tu ra\cmn 你要有心理准备,我那个时候去你家\end{exemple}
\begin{exemple}\jya tɯrme ɯ-sɯm, mbro ɯ-xtu\cmn 人的心,马的肚子(人心无法满足)\end{exemple}\begin{sous-entrée}
\vedette{\hypertarget{}{\papi{ tɯ-sɯm,ɕe}}}\markboth{tɯ-sɯm,ɕe}{}
\begin{définition}\fra vouloir\end{définition}
\begin{définition}\cmn 想要\end{définition}
\begin{exemple}\jya a-sɯm mɯ-pɯ-ari ri, tɤ-fse-a pɯ-ra\cmn 虽然我不愿意,但是只好照着办\end{exemple}
\begin{exemple}\jya ɕɯ-kɤ-rŋgɯ a-sɯm mɯ́j-ɕe\cmn 我不想去睡觉\end{exemple}
\begin{relation-sémantique}\ComponentA{\classe{np}
\hyperlink{Ⓔtɯ-sɯm}{\textit{ \papi{tɯ-sɯm}}}
}\end{relation-sémantique}
\begin{relation-sémantique}\ComponentB{\classe{vi}
\hyperlink{Ⓔɕe}{\textit{ \papi{ɕe}}}
}\end{relation-sémantique}
\end{sous-entrée}\end{entrée}

\begin{entrée}
\vedette{\hypertarget{Ⓔtɯ-sɯmpa}{\papi{ tɯ-sɯmpa}}}\markboth{tɯ-sɯmpa}{}\classe{n}
\begin{définition}\fra pensée\end{définition}
\begin{définition}\cmn 想法
\begin{déclaration} \étymologie{\papi{sems.pa}}\end{déclaration}\end{définition}
\begin{exemple}\jya tɯ-sɯmpa kɯ-sɤɣdɯɣ\cmn 心里很烦\end{exemple}
\begin{exemple}\jya nɤ-sɯmpa kɤ-rku tɯrme\cmn 你的意中人\end{exemple}\end{entrée}

\begin{entrée}
\vedette{\hypertarget{Ⓔtɯ-sɯso}{\papi{ tɯ-sɯso}}}\markboth{tɯ-sɯso}{}
\classe{np}
\begin{définition}\fra souvenir, pensée\end{définition}
\begin{définition}\cmn 记忆;思维\end{définition}
\begin{exemple}\jya a-sɯso jɤ-ɣe\cmn 我想起来了\end{exemple}
\begin{relation-sémantique}\confer{
 \papi{tɯ-ʑosɯso}
}\end{relation-sémantique}
\begin{relation-sémantique}\confer{
\hyperlink{Ⓔrɯsɯso}{\textit{ \papi{rɯsɯso}}}
}\end{relation-sémantique}
\begin{relation-sémantique}\confer{
\hyperlink{Ⓔsɯso}{\textit{ \papi{sɯso}}}
}\end{relation-sémantique}\begin{sous-entrée}
\vedette{\hypertarget{}{\papi{ tɯ-sɯso,lɤt}}}\markboth{tɯ-sɯso,lɤt}{}
\paradigme{\textit{dir :} \jya thɯ-}
\begin{définition}\fra avoir l'intention de\end{définition}
\begin{définition}\cmn 做……的打算;有……的想法\end{définition}
\begin{exemple}\jya kɤ-rɤʑi ɣɯ ɯ-sɯso ma-thɯ-tɯ-lɤt\cmn 你不要做留下来的打算\end{exemple}
\begin{relation-sémantique}\ComponentA{\classe{np}
\hyperlink{Ⓔtɯ-sɯso}{\textit{ \papi{tɯ-sɯso}}}
}\end{relation-sémantique}
\begin{relation-sémantique}\ComponentB{\classe{vt}
\hyperlink{ⒺlɤtⒽ1}{\textit{ \papi{lɤt}}}
}\end{relation-sémantique}
\end{sous-entrée}\end{entrée}

\begin{entrée}
\vedette{\hypertarget{Ⓔtɯ-ʂɯl}{\papi{ tɯ-ʂɯl}}}\markboth{tɯ-ʂɯl}{}\classe{clf}
\begin{définition}\fra une bande de couleur\end{définition}
\begin{définition}\cmn 一条很细的条纹
\begin{déclaration} \étymologie{\papi{srol}}\end{déclaration}\end{définition}
\end{entrée}

\begin{entrée}
\vedette{\hypertarget{Ⓔtɯt}{\papi{ tɯt}}}\markboth{tɯt}{}
\classe{vi}
\paradigme{\textit{dir :} \jya thɯ-}
\paradigme{\textit{dir :} \jya pɯ-}
\begin{définition}\fra mûrir\end{définition}
\begin{définition}\cmn 成熟\end{définition}
\begin{exemple}\jya qaj chɤ-tɯt\cmn 小麦熟了\end{exemple}
\begin{exemple}\jya sɯmat pjɤ-tɯt\cmn 水果熟了\end{exemple}\begin{sous-entrée}
\vedette{\hypertarget{}{\papi{ ɣɤtɯt}}}\markboth{ɣɤtɯt}{}\classe{vs}
\begin{définition}\fra qui mûrit vite\end{définition}
\begin{définition}\cmn 早熟\end{définition}
\end{sous-entrée}\end{entrée}

\begin{entrée}
\vedette{\hypertarget{Ⓔtɯtaʁ}{\papi{ tɯtaʁ}}}\markboth{tɯtaʁ}{}\classe{n}
\begin{définition}\fra textile\end{définition}
\begin{définition}\cmn 纺织品\end{définition}
\begin{relation-sémantique}\confer{
\hyperlink{ⒺtaʁⒽ1}{\textit{ \papi{taʁ1}}}
}\end{relation-sémantique}\end{entrée}

\begin{entrée}
\vedette{\hypertarget{Ⓔtɯtaχte}{\papi{ tɯtaχte}}}\markboth{tɯtaχte}{}
\classe{n}
\begin{définition}\fra méthode de tissage\end{définition}
\begin{définition}\cmn 织布的方法,四根线交错着【斜纹子】\end{définition}
\begin{exemple}\jya raz nɯ tɯtaχte pɯ-nɯ-ŋu nɤ, tɯ-mpɕar, ŋgɤrom pɯ-nɯ-ŋu nɤ, tɯ-mpɕar\cmn 斜纹子只穿一件,单巴子也只穿一件(因为衣服少)\end{exemple}\end{entrée}

\begin{entrée}
\vedette{\hypertarget{Ⓔtɯ-tɤɕrɤz}{\papi{ tɯ-tɤɕrɤz}}}\markboth{tɯ-tɤɕrɤz}{}\classe{n}
\begin{définition}\fra une bande colorée\end{définition}
\begin{définition}\cmn 一溜(花纹)\end{définition}
\begin{relation-sémantique}\confer{
\hyperlink{Ⓔarɤɕɯɕrɤz}{\textit{ \papi{arɤɕɯɕrɤz}}}
}\end{relation-sémantique}\end{entrée}

\begin{entrée}
\vedette{\hypertarget{Ⓔtɯ-tɤfskɤr}{\papi{ tɯ-tɤfskɤr}}}\markboth{tɯ-tɤfskɤr}{}\classe{clf}
\begin{définition}\fra un tour\end{définition}
\begin{définition}\cmn 一圈\end{définition}
\begin{exemple}\jya tɯ-tɤfskɤr to-lɤt-nɯ\cmn 他们绕了一圈\end{exemple}
\begin{relation-sémantique}\confer{
\hyperlink{Ⓔfskɤr}{\textit{ \papi{fskɤr}}}
}\end{relation-sémantique}\end{entrée}

\begin{entrée}
\vedette{\hypertarget{Ⓔtɯ-tɤjŋgɤɣ}{\papi{ tɯ-tɤjŋgɤɣ}}}\markboth{tɯ-tɤjŋgɤɣ}{}\classe{clf}
\begin{définition}\fra une boucle, un tour (à propos d'intestins enroulés comme des cordes)\end{définition}
\begin{définition}\cmn 一圈(肠子)\end{définition}
\begin{exemple}\jya tɯ-pu tɯ-tɤjŋgɤɣ\cmn 一圈肠子\end{exemple}
\begin{relation-sémantique}\confer{
\hyperlink{Ⓔtɯ-ŋgɯl}{\textit{ \papi{tɯ-ŋgɯl}}}
}\end{relation-sémantique}\end{entrée}

\begin{entrée}
\vedette{\hypertarget{Ⓔtɯ-tɤkhar}{\papi{ tɯ-tɤkhar}}}\markboth{tɯ-tɤkhar}{}\classe{clf}\acception{1}
\begin{définition}\fra une muraille\end{définition}
\begin{définition}\cmn 一堵围墙\end{définition}\acception{2}
\begin{définition}\fra une troupe de gens qui entourent un endroit\end{définition}
\begin{définition}\cmn 包围某个地方的一群人\end{définition}
\end{entrée}

\begin{entrée}
\vedette{\hypertarget{Ⓔtɯ-tɤkhrɤz}{\papi{ tɯ-tɤkhrɤz}}}\markboth{tɯ-tɤkhrɤz}{}\classe{clf}
\begin{définition}\fra une rayure, une ligne\end{définition}
\begin{définition}\cmn 一排;一路;一层\end{définition}
\begin{exemple}\jya ki tɯ-ŋga ki kɯ-wɣrum tɯ-tɤkhrɤz kɯ-ɲaʁ tɯ-tɤkhrɤz ku-ɕe ɲɯ-ŋu\cmn 这件衣服的花纹是一路白色、一路黑色的\end{exemple}
\begin{relation-sémantique}\synonyme{
\hyperlink{Ⓔtɯ-tɤmphrɯm}{\textit{ \papi{tɯ-tɤmphrɯm}}}
}\end{relation-sémantique}
\begin{relation-sémantique}\synonyme{
\hyperlink{Ⓔtɯ-tɤɕrɤz}{\textit{ \papi{tɯ-tɤɕrɤz}}}
}\end{relation-sémantique}
\begin{relation-sémantique}\confer{
\hyperlink{ⒺkhrɤtⒽ1}{\textit{ \papi{khrɤt1}}}
}\end{relation-sémantique}\end{entrée}

\begin{entrée}
\vedette{\hypertarget{Ⓔtɯ-tɤlɤβ}{\papi{ tɯ-tɤlɤβ}}}\markboth{tɯ-tɤlɤβ}{}
\classe{clf}
\begin{définition}\fra couche\end{définition}
\begin{définition}\cmn 层\end{définition}
\begin{exemple}\jya tɤ-βɟu kɤntɕhɯ-tɤlɤβ lɤ-ta-j\cmn 我们铺了几层褥子\end{exemple}\end{entrée}

\begin{entrée}
\vedette{\hypertarget{Ⓔtɯ-tɤlia}{\papi{ tɯ-tɤlia}}}\markboth{tɯ-tɤlia}{}\classe{clf}
\begin{définition}\fra un petit écheveau\end{définition}
\begin{définition}\cmn 一小绞\end{définition}
\begin{relation-sémantique}\confer{
\hyperlink{Ⓔtɤβri}{\textit{ \papi{tɤβri}}}
}\end{relation-sémantique}\end{entrée}

\begin{entrée}
\vedette{\hypertarget{Ⓔtɯ-tɤlkɯɣ}{\papi{ tɯ-tɤlkɯɣ}}}\markboth{tɯ-tɤlkɯɣ}{}\classe{clf}
\begin{définition}\fra un tour de corde (enroulée en cercle)\end{définition}
\begin{définition}\cmn 一圈绳子\end{définition}
\end{entrée}

\begin{entrée}
\vedette{\hypertarget{Ⓔtɯtɤmda}{\papi{ tɯtɤmda}}}\markboth{tɯtɤmda}{}\classe{adv}
\begin{définition}\fra chaque fois\end{définition}
\begin{définition}\cmn 每次\end{définition}
\begin{relation-sémantique}\confer{
\hyperlink{Ⓔmda}{\textit{ \papi{mda}}}
}\end{relation-sémantique}\end{entrée}

\begin{entrée}
\vedette{\hypertarget{Ⓔtɯ-tɤmphrɯm}{\papi{ tɯ-tɤmphrɯm}}}\markboth{tɯ-tɤmphrɯm}{}
\classe{clf}
\begin{définition}\fra une rangée\end{définition}
\begin{définition}\cmn 一路;一排\end{définition}
\begin{exemple}\jya jiʑora kutɕu ɲɯ-ɤʑɯrja-j tɕe, tɤ-tɕɯ tɯ-tɤmphrɯm, tɕheme tɯ-tɤmphrɯm nɯ kɯ-fse tu-kɯ-ndzur ɲɯ-ra\cmn 我们在这里排队,男的一排、女的一排\end{exemple}
\begin{relation-sémantique}\confer{
\hyperlink{Ⓔrɤmphrɯm}{\textit{ \papi{rɤmphrɯm}}}
}\end{relation-sémantique}\end{entrée}

\begin{entrée}
\vedette{\hypertarget{Ⓔtɯ-tɤndzri}{\papi{ tɯ-tɤndzri}}}\markboth{tɯ-tɤndzri}{}\classe{clf}
\begin{définition}\fra une poignée\end{définition}
\begin{définition}\cmn 一绞\end{définition}
\begin{exemple}\jya tɯɣro tɯ-tɤndzri\cmn 一绞青草\end{exemple}\end{entrée}

\begin{entrée}
\vedette{\hypertarget{Ⓔtɯ-tɤri}{\papi{ tɯ-tɤri}}}\markboth{tɯ-tɤri}{}\classe{clf}
\begin{définition}\fra ligature\end{définition}
\begin{définition}\cmn 一串\end{définition}
\begin{relation-sémantique}\confer{
\hyperlink{Ⓔtɤ-ri}{\textit{ \papi{tɤ-ri}}}
}\end{relation-sémantique}
\end{entrée}

\begin{entrée}
\vedette{\hypertarget{Ⓔtɯ-tɤrtsɯɣ}{\papi{ tɯ-tɤrtsɯɣ}}}\markboth{tɯ-tɤrtsɯɣ}{}\classe{clf}
\begin{définition}\fra pile\end{définition}
\begin{définition}\cmn 一堆\end{définition}
\end{entrée}

\begin{entrée}
\vedette{\hypertarget{Ⓔtɯ-tɤrzɯɣ}{\papi{ tɯ-tɤrzɯɣ}}}\markboth{tɯ-tɤrzɯɣ}{}\classe{clf}
\begin{définition}\fra une section\end{définition}
\begin{définition}\cmn 一段\end{définition}
\begin{relation-sémantique}\confer{
\hyperlink{Ⓔtɯ-rzɯɣ}{\textit{ \papi{tɯ-rzɯɣ}}}
}\end{relation-sémantique}\end{entrée}

\begin{entrée}
\vedette{\hypertarget{Ⓔtɯ-tɤʁol}{\papi{ tɯ-tɤʁol}}}\markboth{tɯ-tɤʁol}{}\classe{np}
\begin{définition}\fra se préoccuper de choses qui ne le regarde pas\end{définition}
\begin{définition}\cmn 多管闲事\end{définition}
\end{entrée}

\begin{entrée}
\vedette{\hypertarget{Ⓔtɯ-tɤsɯm}{\papi{ tɯ-tɤsɯm}}}\markboth{tɯ-tɤsɯm}{}\classe{clf}
\begin{définition}\fra une partie\end{définition}
\begin{définition}\cmn 一份\end{définition}
\begin{exemple}\jya χsɯ-tɤsɯm ɯ-ŋgɯ zɯ tɯ-tɯsɯm\cmn 三分之一\end{exemple}
\begin{relation-sémantique}\synonyme{
\hyperlink{Ⓔtɯ-tɯcɯr}{\textit{ \papi{tɯ-tɯcɯr}}}
}\end{relation-sémantique}\end{entrée}

\begin{entrée}
\vedette{\hypertarget{Ⓔtɯ-tɤtɕhɯ}{\papi{ tɯ-tɤtɕhɯ}}}\markboth{tɯ-tɤtɕhɯ}{}\classe{clf}
\begin{définition}\fra un coup de bêche\end{définition}
\begin{définition}\cmn 挖一锄头\end{définition}
\end{entrée}

\begin{entrée}
\vedette{\hypertarget{Ⓔtɯ-tɤxɯr}{\papi{ tɯ-tɤxɯr}}}\markboth{tɯ-tɤxɯr}{}\classe{clf}
\begin{définition}\fra un tour\end{définition}
\begin{définition}\cmn 一圈\end{définition}
\begin{exemple}\jya jiʑo tɯ-tɤxɯr tɤ-lɤt-i\cmn 我们绕了一圈\end{exemple}
\begin{relation-sémantique}\confer{
\hyperlink{Ⓔxɯrxɯr}{\textit{ \papi{xɯrxɯr}}}
}\end{relation-sémantique}
\end{entrée}

\begin{entrée}
\vedette{\hypertarget{Ⓔtɯ-tɕa}{\papi{ tɯ-tɕa}}}\markboth{tɯ-tɕa}{}\classe{np}
\begin{définition}\fra faute\end{définition}
\begin{définition}\cmn 错\end{définition}
\begin{exemple}\jya a-tɕa tu\cmn 我有错\end{exemple}
\begin{exemple}\jya ki aʑo a-tɕa ŋu\cmn 这是我的错\end{exemple}
\end{entrée}

\begin{entrée}
\vedette{\hypertarget{Ⓔtɯ-tɕa,nɯjɤt}{\papi{ tɯ-tɕa,nɯjɤt}}}\markboth{tɯ-tɕa,nɯjɤt}{}\classe{vt}
\paradigme{\textit{dir :} \jya nɯ-}
\begin{définition}\fra présenter ses excuses\end{définition}
\begin{définition}\cmn 赔罪\end{définition}
\begin{exemple}\jya nɤ-tɕa ɣɯ-nɯ-nɯjɤt\cmn 你来赔罪\end{exemple}
\begin{relation-sémantique}\ComponentA{\classe{np}
\hyperlink{Ⓔtɯ-tɕa}{\textit{ \papi{tɯ-tɕa}}}
}\end{relation-sémantique}
\begin{relation-sémantique}\ComponentB{\classe{vt}
 \papi{nɯjɤt}
}\end{relation-sémantique}\end{entrée}

\begin{entrée}
\vedette{\hypertarget{Ⓔtɯ-tɕhaⒽ1}{\papi{ tɯ-tɕha}}}\markboth{tɯ-tɕha}{}\homonyme{1}\classe{np}
\begin{définition}\fra information\end{définition}
\begin{définition}\cmn 消息
\begin{déclaration}\use{\stylefv{a-tɕha}可以表示“关于我的消息”,也可以表示“我需要的消息”}\end{déclaration}\end{définition}
\begin{exemple}\jya a-tɕha nɯ-khɤm\cmn 你要给我做解答\end{exemple}
\begin{exemple}\jya tɯ-tɕha na-tsɯm\cmn 他带了消息\end{exemple}
\begin{exemple}\jya tɯ-tɕha ja-ɣɯt\cmn 他把消息带给我了\end{exemple}
\begin{exemple}\jya tɕe tɤ-tɯ-ʑɣɤsɯrtoʁ tɕe, ɯ-ɲɯ́-tɯ-mna nɤ, a-tɕha a-kɤ-tɯ-khɤm\cmn 你看了病以后,如果病好的话,你要告诉我\end{exemple}
\begin{exemple}\jya nɤ-kɯ-mŋɤm ɯ-ɲɯ́-mna nɯra a-tɕha a-kɤ-tɯ-khɤm\cmn 你病好的消息要告诉我\end{exemple}
\begin{exemple}\jya tɯ-tɕha ja-sɤzɣɯt\cmn 他把消息带到了\end{exemple}
\begin{relation-sémantique}\confer{
\hyperlink{Ⓔɣɯtɕha}{\textit{ \papi{ɣɯtɕha}}}
}\end{relation-sémantique}\end{entrée}

\begin{entrée}
\vedette{\hypertarget{Ⓔtɯ-tɕhaⒽ2}{\papi{ tɯ-tɕha}}}\markboth{tɯ-tɕha}{}\homonyme{2}
\classe{clf}
\begin{définition}\fra paire\end{définition}
\begin{définition}\cmn 一对\end{définition}
\begin{exemple}\jya tɯ-xtsa tɯ-tɕha\cmn 一双鞋子\end{exemple}
\begin{exemple}\jya qala tɯ-tɕha\cmn 一对兔子\end{exemple}
\begin{exemple}\jya ɯ-me tɯ-tɕha to-sci\cmn 她生了一对女儿\end{exemple}\end{entrée}

\begin{entrée}
\vedette{\hypertarget{Ⓔtɯ-tɕhaʁ}{\papi{ tɯ-tɕhaʁ}}}\markboth{tɯ-tɕhaʁ}{}\classe{clf}
\begin{définition}\fra bouquet, boisseau\end{définition}
\begin{définition}\cmn 一捆\end{définition}
\end{entrée}

\begin{entrée}
\vedette{\hypertarget{Ⓔtɯ-tɕhaʁa}{\papi{ tɯ-tɕhaʁa}}}\markboth{tɯ-tɕhaʁa}{}\classe{np}
\begin{définition}\fra à double paupière\end{définition}
\begin{définition}\cmn 双眼皮\end{définition}
\begin{exemple}\jya aʑo a-tɕhaʁa me\cmn 我没有双眼皮\end{exemple}\end{entrée}

\begin{entrée}
\vedette{\hypertarget{Ⓔtɯtɕhɯ}{\papi{ tɯtɕhɯ}}}\markboth{tɯtɕhɯ}{}\classe{n}
\begin{définition}\fra assassinat\end{définition}
\begin{définition}\cmn 刺杀\end{définition}
\begin{exemple}\jya tɯtɕhɯ to-lɤt\cmn 他刺杀了人\end{exemple}
\begin{relation-sémantique}\confer{
\hyperlink{Ⓔnɯtɯtɕhɯ}{\textit{ \papi{nɯtɯtɕhɯ}}}
}\end{relation-sémantique}\end{entrée}

\begin{entrée}
\vedette{\hypertarget{Ⓔtɯ-tɕɯlɤβ}{\papi{ tɯ-tɕɯlɤβ}}}\markboth{tɯ-tɕɯlɤβ}{}\classe{clf}
\begin{définition}\fra temps de fumer une pipe\end{définition}
\begin{définition}\cmn 一个烟斗的功夫\end{définition}
\begin{exemple}\jya thamakha tɯ-tɕɯlɤβ kɤ-sko ɯ-raŋ jamar\cmn 抽一个烟斗的时间(一个烟斗的功夫)\end{exemple}
\begin{relation-sémantique}\confer{
\hyperlink{Ⓔtɕɯlɤβ}{\textit{ \papi{tɕɯlɤβ}}}
}\end{relation-sémantique}\end{entrée}

\begin{entrée}
\vedette{\hypertarget{ⒺtɯtɣaⒽ2}{\papi{ tɯtɣa}}}\markboth{tɯtɣa}{}\homonyme{2}
\classe{n}
\begin{définition}\fra récolte\end{définition}
\begin{définition}\cmn 庄稼\end{définition}
\begin{relation-sémantique}\confer{
\hyperlink{Ⓔtɣa}{\textit{ \papi{tɣa}}}
}\end{relation-sémantique}
\end{entrée}

\begin{entrée}
\vedette{\hypertarget{Ⓔtɯ-tɣaⒽ1}{\papi{ tɯ-tɣa}}}\markboth{tɯ-tɣa}{}\homonyme{1}
\classe{clf}
\begin{définition}\fra empan\end{définition}
\begin{définition}\cmn 一拃,张开大拇指和中指来测距的长度单位。\end{définition}
\begin{exemple}\jya χsɯ-tɣa jamar ma tu-mbro mɤ-cha\cmn 只能长到三拃高\end{exemple}\begin{sous-entrée}
\vedette{\hypertarget{}{\papi{ tɯ-tɣa}}}\markboth{tɯ-tɣa}{}\classe{np}
\begin{exemple}\jya χsɤ-tɣa, χsɯ-tɣa\cmn 三拃\end{exemple}
\begin{exemple}\jya aʑo a-tɣa xtɯt ɕti\cmn 我的拃(手指的长度)很短\end{exemple}
\end{sous-entrée}\end{entrée}

\begin{entrée}
\vedette{\hypertarget{Ⓔtɯ-thɯ}{\papi{ tɯ-thɯ}}}\markboth{tɯ-thɯ}{}\classe{np}
\begin{définition}\fra casserole\end{définition}
\begin{définition}\cmn 锅子\end{définition}
\begin{exemple}\jya a-thɯ\cmn 我的锅子\end{exemple}
\begin{relation-sémantique}\confer{
\hyperlink{Ⓔnɯthɯ}{\textit{ \papi{nɯthɯ}}}
}\end{relation-sémantique}\end{entrée}

\begin{entrée}
\vedette{\hypertarget{Ⓔtɯ-tun}{\papi{ tɯ-tun}}}\markboth{tɯ-tun}{}
\classe{np}
\begin{définition}\fra but, sens\end{définition}
\begin{définition}\cmn 目的;意思
\begin{déclaration} \étymologie{\papi{don}}\end{déclaration}\end{définition}
\begin{exemple}\jya ɯ-sɤ-ntɕhoz mɯ́j-naχtɕɯɣ tɕe, tɕe ɯ-tun nɯ mɯ́j-naχtɕɯɣ ɲɯ-ŋu\cmn 用在不同的语境里(这个词)的意思也不一样\end{exemple}
\begin{exemple}\jya ʁnaʁna ɯ-tun naχtɕɯɣ ɕti\cmn 两种(说法)的意思是一样的\end{exemple}
\begin{exemple}\jya ɯ-ti mɤ-naχtɕɯɣ ma ɯ-tun naχtɕɯɣ ɕti\cmn 说法不一样,意思一样\end{exemple}\end{entrée}

\begin{entrée}
\vedette{\hypertarget{Ⓔtɯtsa}{\papi{ tɯtsa}}}\markboth{tɯtsa}{}\classe{n}
\begin{définition}\fra enclume\end{définition}
\begin{définition}\cmn 铁镦\end{définition}
\end{entrée}

\begin{entrée}
\vedette{\hypertarget{Ⓔtɯtsɣe}{\papi{ tɯtsɣe}}}\markboth{tɯtsɣe}{}\classe{n}
\begin{définition}\fra commerce\end{définition}
\begin{définition}\cmn 生意\end{définition}
\begin{exemple}\jya tɯtsɣe chɤ-ta\cmn 他摆了摊子\end{exemple}
\begin{relation-sémantique}\confer{
\hyperlink{Ⓔntsɣe}{\textit{ \papi{ntsɣe}}}
}\end{relation-sémantique}
\end{entrée}

\begin{entrée}
\vedette{\hypertarget{Ⓔtɯtshi}{\papi{ tɯtshi}}}\markboth{tɯtshi}{}
\classe{n}
\begin{définition}\fra gruau de riz\end{définition}
\begin{définition}\cmn 粥;稀饭\end{définition}
\begin{définition}\cmn 我喝了粥\end{définition}
\begin{définition}\cmn 我喝了我的粥\end{définition}
\begin{exemple}\jya tɯtshi kɤ-tshi-t-a\end{exemple}
\begin{exemple}\jya a-tɯtshi kɤ-nɯ-tshi-t-a\end{exemple}
\begin{relation-sémantique}\confer{
\hyperlink{ⒺtshiⒽ1}{\textit{ \papi{tshi1}}}
}\end{relation-sémantique}\end{entrée}

\begin{entrée}
\vedette{\hypertarget{Ⓔtɯtshot}{\papi{ tɯtshot}}}\markboth{tɯtshot}{}\classe{n}\begin{définition}\fra heure\end{définition}
\begin{définition}\cmn 钟
\begin{déclaration} \étymologie{\papi{dus.tsʰod}}\end{déclaration}\end{définition}
\begin{exemple}\jya mɤʑɯ tɯtshot ʁnɯz cho tɯ-phaʁ jamar tɕe tu\cmn 还有两个半小时\end{exemple}
\begin{exemple}\jya mɤʑɯ sqamŋu skɤrma tɕe tɯtshot kɯtʂɤɣ zɣɯt ɲɯ-ŋu\cmn 还有十五分钟就到了六点种了\end{exemple}
\begin{exemple}\jya kɯkutɕu tɯtshot kɯβde kɤ-azɣɯt\cmn 这里已经是四点钟了\end{exemple}
\begin{exemple}\jya tɯtshot χsɯm cho ɣnɤsqi-skɤrma ko-zɣɯt\cmn 已经三点二十分了\end{exemple}
\begin{exemple}\jya tɯtshot mɯ-tɤ-rtoʁ-a\cmn 我没有看时间\end{exemple}
\begin{exemple}\jya tɯtshot ʑatsa zɣɯt ɲɯ-ŋu\cmn 时间差不多到了\end{exemple}\end{entrée}

\begin{entrée}
\vedette{\hypertarget{Ⓔtɯ-tshoz}{\papi{ tɯ-tshoz}}}\markboth{tɯ-tshoz}{}\classe{clf}
\begin{définition}\fra un ensemble\end{définition}
\begin{définition}\cmn 一套\end{définition}\end{entrée}

\begin{entrée}
\vedette{\hypertarget{Ⓔtɯ-tsi}{\papi{ tɯ-tsi}}}\markboth{tɯ-tsi}{}\classe{np}
\begin{définition}\fra longévité\end{définition}
\begin{définition}\cmn 寿命\end{définition}
\begin{définition}\cmn 她很长寿\end{définition}
\begin{exemple}\jya ɯ-tsi pjɤ-zri\end{exemple}\end{entrée}

\begin{entrée}
\vedette{\hypertarget{Ⓔtɯtʂaŋ}{\papi{ tɯtʂaŋ}}}\markboth{tɯtʂaŋ}{}\classe{n}
\begin{définition}\fra impartialité\end{définition}
\begin{définition}\cmn 公道\end{définition}
\begin{exemple}\jya tɕi-tɯtʂaŋ ci nɯ-lɤt\cmn 给我们讨个公道\end{exemple}
\begin{relation-sémantique}\confer{
\hyperlink{Ⓔtʂaŋ}{\textit{ \papi{tʂaŋ}}}
}\end{relation-sémantique}\end{entrée}

\begin{entrée}
\vedette{\hypertarget{Ⓔtɯ-tʂɯn}{\papi{ tɯ-tʂɯn}}}\markboth{tɯ-tʂɯn}{}\classe{np}
\begin{définition}\fra faveur, bonté\end{définition}
\begin{définition}\cmn 恩\end{définition}
\begin{exemple}\jya a-taʁ nɤ-tʂɯn wxti\cmn 你对我恩重如山\end{exemple}\end{entrée}

\begin{entrée}
\vedette{\hypertarget{Ⓔtɯ-tɯcɤβ}{\papi{ tɯ-tɯcɤβ}}}\markboth{tɯ-tɯcɤβ}{}\classe{clf}
\begin{définition}\fra une espèce\end{définition}
\begin{définition}\cmn 一类;一种\end{définition}
\begin{exemple}\jya nɯ tɯ-tɯcɤβ ɲɯ-maʁ-nɯ\cmn 不是同一类的\end{exemple}
\begin{relation-sémantique}\confer{
\hyperlink{Ⓔtɤ-cɤβ}{\textit{ \papi{tɤ-cɤβ}}}
}\end{relation-sémantique}\end{entrée}

\begin{entrée}
\vedette{\hypertarget{Ⓔtɯ-tɯcɯr}{\papi{ tɯ-tɯcɯr}}}\markboth{tɯ-tɯcɯr}{}\classe{clf}
\begin{définition}\fra une partie\end{définition}
\begin{définition}\cmn 一份\end{définition}
\begin{exemple}\jya χsɯ-tɯcɯr tú-wɣ-lɤt tɕe tɯ-tɯcɯr\cmn 三分之一\end{exemple}
\begin{exemple}\jya kɯmŋu-tɯcɯr ɯ-ŋgɯ χsɯ-tɯcɯr\cmn 五分之三\end{exemple}
\begin{relation-sémantique}\synonyme{
\hyperlink{Ⓔtɯ-tɤsɯm}{\textit{ \papi{tɯ-tɤsɯm}}}
}\end{relation-sémantique}\end{entrée}

\begin{entrée}
\vedette{\hypertarget{Ⓔtɯ-tɯkro}{\papi{ tɯ-tɯkro}}}\markboth{tɯ-tɯkro}{}\classe{clf}
\begin{définition}\fra une part\end{définition}
\begin{définition}\cmn 一份\end{définition}
\end{entrée}

\begin{entrée}
\vedette{\hypertarget{Ⓔtɯ-tɯm}{\papi{ tɯ-tɯm}}}\markboth{tɯ-tɯm}{}\classe{clf}
\begin{définition}\fra gousse d'ail\end{définition}
\begin{définition}\cmn 一头蒜\end{définition}\end{entrée}

\begin{entrée}
\vedette{\hypertarget{Ⓔtɯ-tɯmbrɯ}{\papi{ tɯ-tɯmbrɯ}}}\markboth{tɯ-tɯmbrɯ}{}
\classe{clf}
\begin{définition}\fra section (du balcon, pour mettre la nourriture)\end{définition}
\begin{définition}\cmn 一格;一空(走檐上,两个柱头之间可以装的粮食)\end{définition}
\begin{exemple}\jya tɤ-rɤku jɤɣɤt zɯ kɯtʂɤɣ-tɯmbrɯ tú-wɣ-ta tɕe mɤro ɯ-taʁ tɕe tɯ-tɯmbrɯ ma me\cmn 走檐的六格等于粮架的一格\end{exemple}\end{entrée}

\begin{entrée}
\vedette{\hypertarget{Ⓔtɯ-tɯndzɯm}{\papi{ tɯ-tɯndzɯm}}}\markboth{tɯ-tɯndzɯm}{}\classe{clf}
\begin{définition}\fra bœuf (pour tirer la charrue)\end{définition}
\begin{définition}\cmn (套上犁的)一架牛\end{définition}
\end{entrée}

\begin{entrée}
\vedette{\hypertarget{Ⓔtɯ-tɯnɯna}{\papi{ tɯ-tɯnɯna}}}\markboth{tɯ-tɯnɯna}{}\classe{clf}
\begin{définition}\fra borne (entre deux auberges de repos)\end{définition}
\begin{définition}\cmn 一里(两个休息的地点之间的距离)\end{définition}
\end{entrée}

\begin{entrée}
\vedette{\hypertarget{Ⓔtɯ-tɯŋgɯ}{\papi{ tɯ-tɯŋgɯ}}}\markboth{tɯ-tɯŋgɯ}{}\classe{clf}
\begin{définition}\fra des épis d'orge sur tout le toit\end{définition}
\begin{définition}\cmn 满满一房背的青稞穗\end{définition}
\begin{exemple}\jya saχsɯ ɕɯŋgɯ tɯ-tɯŋgɯ pɯ-tɤβ-i\cmn 我们在午饭前打完了满满一房背的青稞穗\end{exemple}\end{entrée}

\begin{entrée}
\vedette{\hypertarget{Ⓔtɯ-tɯpɕɯrtɕhaʁ}{\papi{ tɯ-tɯpɕɯrtɕhaʁ}}}\markboth{tɯ-tɯpɕɯrtɕhaʁ}{}\classe{clf}\acception{1}
\begin{définition}\fra une génération, une classe\end{définition}
\begin{définition}\cmn 一辈;一代、年级\end{définition}
\begin{exemple}\jya jiʑo tɯ-tɯpɕɯrtɕhaʁ pɯ-ŋu-j\cmn 我们是同一个年级的\end{exemple}\acception{2}
\begin{définition}\fra un groupe (de gens)\end{définition}
\begin{définition}\cmn 一批(人)\end{définition}
\begin{relation-sémantique}\confer{
\hyperlink{Ⓔtɯ-pɕɯrtɕhaʁ}{\textit{ \papi{tɯ-pɕɯrtɕhaʁ}}}
}\end{relation-sémantique}\end{entrée}

\begin{entrée}
\vedette{\hypertarget{Ⓔtɯ-tɯphu}{\papi{ tɯ-tɯphu}}}\markboth{tɯ-tɯphu}{}
\classe{clf}\acception{1}
\begin{définition}\fra espèce\end{définition}
\begin{définition}\cmn 种类,物种\end{définition}
\begin{exemple}\jya tɯ-tɯphu ɕti, ɯ-rmi mɤ-naχtɕɯɣ\cmn 是同一种,但是名字不一样\end{exemple}
\begin{exemple}\jya pɣa ɯ-tɯphu dɤn\cmn 鸟的种类很多\end{exemple}\acception{2}
\begin{définition}\fra une ruche (abeilles)\end{définition}
\begin{définition}\cmn 一窝(蜜蜂)\end{définition}
\end{entrée}

\begin{entrée}
\vedette{\hypertarget{Ⓔtɯ-tɯpɯ}{\papi{ tɯ-tɯpɯ}}}\markboth{tɯ-tɯpɯ}{}\classe{clf}
\begin{définition}\fra famille\end{définition}
\begin{définition}\cmn 一户人\end{définition}
\end{entrée}

\begin{entrée}
\vedette{\hypertarget{Ⓔtɯ-tɯrpa}{\papi{ tɯ-tɯrpa}}}\markboth{tɯ-tɯrpa}{}\classe{clf}
\begin{définition}\fra livre\end{définition}
\begin{définition}\cmn 一斤\end{définition}
\begin{relation-sémantique}\confer{
\hyperlink{Ⓔtɯrpa}{\textit{ \papi{tɯrpa}}}
}\end{relation-sémantique}
\end{entrée}

\begin{entrée}
\vedette{\hypertarget{Ⓔtɯ-wɤt}{\papi{ tɯ-wɤt}}}\markboth{tɯ-wɤt}{}\classe{np}
\begin{définition}\fra manche\end{définition}
\begin{définition}\cmn 袖子\end{définition}
\begin{exemple}\jya a-wɤt lɤ-pɣaʁ-a\cmn 我挽起袖子了\end{exemple}
\begin{relation-sémantique}\synonyme{
\hyperlink{Ⓔtɤ-pɤloʁ}{\textit{ \papi{tɤ-pɤloʁ}}}
}\end{relation-sémantique}\end{entrée}

\begin{entrée}
\vedette{\hypertarget{Ⓔtɯ-wɤtku}{\papi{ tɯ-wɤtku}}}\markboth{tɯ-wɤtku}{}\classe{np}
\begin{définition}\fra bout des manches\end{définition}
\begin{définition}\cmn 袖口\end{définition}
\begin{relation-sémantique}\confer{
\hyperlink{Ⓔtɯ-sroʁ,nɯwɤtku}{\textit{ \papi{tɯ-sroʁ,nɯwɤtku}}}
}\end{relation-sémantique}
\begin{relation-sémantique}\confer{
\hyperlink{Ⓔtɯ-wɤt}{\textit{ \papi{tɯ-wɤt}}}
}\end{relation-sémantique}\end{entrée}

\begin{entrée}
\vedette{\hypertarget{Ⓔtɯwɯ}{\papi{ tɯwɯ}}}\markboth{tɯwɯ}{}\classe{n}
\begin{définition}\fra bâton du fuseau\end{définition}
\begin{définition}\cmn 吊干(纺锤的木棒)\end{définition}
\end{entrée}

\begin{entrée}
\vedette{\hypertarget{Ⓔtɯwɯr}{\papi{ tɯwɯr}}}\markboth{tɯwɯr}{}\classe{n}
\begin{définition}\fra habit de pluie\end{définition}
\begin{définition}\cmn 蓑衣\end{définition}\end{entrée}

\begin{entrée}
\vedette{\hypertarget{Ⓔtɯ-xɕɤt}{\papi{ tɯ-xɕɤt}}}\markboth{tɯ-xɕɤt}{}\classe{np}
\begin{définition}\fra force\end{définition}
\begin{définition}\cmn 力气
\begin{déclaration} \étymologie{\papi{ɕed}}\end{déclaration}\end{définition}
\begin{exemple}\jya ɯ-xɕɤt tɯ-tu ʑo tha-ɣɤmɯt\cmn 他用尽全力吹\end{exemple}\begin{sous-entrée}
\vedette{\hypertarget{}{\papi{ ɯ-xɕɤt kɯ}}}\markboth{ɯ-xɕɤt kɯ}{}\classe{lnk}
\begin{exemple}\jya a-mu kɯ a-nɯzdɯɣ ɯ-xɕɤt kɯ to-ngo\cmn 我母亲因为担心我生病了\end{exemple}
\end{sous-entrée}\end{entrée}

\begin{entrée}
\vedette{\hypertarget{Ⓔtɯ-xɕiridi}{\papi{ tɯ-xɕiridi}}}\markboth{tɯ-xɕiridi}{}\classe{np}
\begin{définition}\fra odeur de dessous de bras\end{définition}
\begin{définition}\cmn 狐臭\end{définition}
\begin{exemple}\jya nɤ-xɕiridi ɯ-tɯ-sɤjloʁ nɯ\cmn 你的狐臭很难闻\end{exemple}
\begin{relation-sémantique}\confer{
\hyperlink{Ⓔtɤ-di}{\textit{ \papi{tɤ-di}}}
}\end{relation-sémantique}\end{entrée}

\begin{entrée}
\vedette{\hypertarget{Ⓔtɯ-xpa}{\papi{ tɯ-xpa}}}\markboth{tɯ-xpa}{}\classe{clf}
\begin{définition}\fra une année\end{définition}
\begin{définition}\cmn 一年\end{définition}
\begin{exemple}\jya kɯmŋu-xpa to-tsu (=kɯmŋu-xpa ɲɤ-ɕe)\cmn 过了五年\end{exemple}
\begin{relation-sémantique}\confer{
\hyperlink{Ⓔkɤrɤxpa}{\textit{ \papi{kɤrɤxpa}}}
}\end{relation-sémantique}
\begin{relation-sémantique}\confer{
\hyperlink{Ⓔjapa}{\textit{ \papi{japa}}}
}\end{relation-sémantique}
\begin{relation-sémantique}\confer{
\hyperlink{Ⓔɣɯjpa}{\textit{ \papi{ɣɯjpa}}}
}\end{relation-sémantique}
\end{entrée}

\begin{entrée}
\vedette{\hypertarget{Ⓔtɯxpalɤskɤr}{\papi{ tɯxpalɤskɤr}}}\markboth{tɯxpalɤskɤr}{}\classe{adv}
\begin{définition}\fra toute l'année\end{définition}
\begin{définition}\cmn 一年到头
\begin{déclaration} \étymologie{\papi{lo.skor}}\end{déclaration}\end{définition}
\end{entrée}

\begin{entrée}
\vedette{\hypertarget{Ⓔtɯ-xsoz}{\papi{ tɯ-xsoz}}}\markboth{tɯ-xsoz}{}\classe{clf}
\begin{définition}\fra un matin\end{définition}
\begin{définition}\cmn 一个早晨\end{définition}
\end{entrée}

\begin{entrée}
\vedette{\hypertarget{Ⓔtɯ-xtu}{\papi{ tɯ-xtu}}}\markboth{tɯ-xtu}{}\classe{np}
\begin{définition}\fra ventre\end{définition}
\begin{définition}\cmn 肚子\end{définition}
\begin{exemple}\jya ɯ-xtu ɯ́-khɯ?\cmn 他胃口好不好?\end{exemple}
\begin{relation-sémantique}\confer{
\hyperlink{Ⓔɯ-xtɤfka}{\textit{ \papi{ɯ-xtɤfka}}}
}\end{relation-sémantique}
\begin{relation-sémantique}\confer{
\hyperlink{Ⓔxtɤtshɤt}{\textit{ \papi{xtɤtshɤt}}}
}\end{relation-sémantique}
\end{entrée}

\begin{entrée}
\vedette{\hypertarget{Ⓔtɯ-xtɤci}{\papi{ tɯ-xtɤci}}}\markboth{tɯ-xtɤci}{}\classe{np}
\begin{définition}\fra remontées gastriques\end{définition}
\begin{définition}\cmn 吐酸水\end{définition}
\begin{exemple}\jya a-xtɤci lo-ɣi\cmn 我吐酸水。\end{exemple}
\begin{relation-sémantique}\confer{
\hyperlink{Ⓔtɯ-xtu}{\textit{ \papi{tɯ-xtu}}}
}\end{relation-sémantique}
\begin{relation-sémantique}\confer{
\hyperlink{Ⓔtɯ-ci}{\textit{ \papi{tɯ-ci}}}
}\end{relation-sémantique}
\end{entrée}

\begin{entrée}
\vedette{\hypertarget{Ⓔtɯxtɤŋɤm}{\papi{ tɯxtɤŋɤm}}}\markboth{tɯxtɤŋɤm}{}\classe{n}
\begin{définition}\fra dysenterie\end{définition}
\begin{définition}\cmn 痢疾\end{définition}
\begin{relation-sémantique}\confer{
\hyperlink{Ⓔtɯ-xtu}{\textit{ \papi{tɯ-xtu}}}
}\end{relation-sémantique}
\begin{relation-sémantique}\confer{
\hyperlink{Ⓔmŋɤm}{\textit{ \papi{mŋɤm}}}
}\end{relation-sémantique}\end{entrée}

\begin{entrée}
\vedette{\hypertarget{Ⓔtɯ-xtɤpa}{\papi{ tɯ-xtɤpa}}}\markboth{tɯ-xtɤpa}{}\classe{np}
\begin{définition}\fra ventre (animal)\end{définition}
\begin{définition}\cmn 动物的身下 (因为动物通过四肢站立,因此指的是与肚子相连的部分)\end{définition}
\begin{relation-sémantique}\confer{
\hyperlink{Ⓔtɯ-xtu}{\textit{ \papi{tɯ-xtu}}}
}\end{relation-sémantique}\end{entrée}

\begin{entrée}
\vedette{\hypertarget{Ⓔtɯ-xtsa}{\papi{ tɯ-xtsa}}}\markboth{tɯ-xtsa}{}\classe{np}
\begin{définition}\fra chaussure\end{définition}
\begin{définition}\cmn 鞋子\end{définition}
\begin{exemple}\jya nɤ-xtsa pɯ-tɕɤt\cmn 你脱鞋子吧\end{exemple}\end{entrée}

\begin{entrée}
\vedette{\hypertarget{Ⓔtɯxtsakɤɣɯɕqri}{\papi{ tɯxtsakɤɣɯɕqri}}}\markboth{tɯxtsakɤɣɯɕqri}{}\classe{n}
\begin{définition}\fra botte dont le haut est en peau de chevrotain, et le milieu en cuir teint en rouge\end{définition}
\begin{définition}\cmn 靴筒上部是獐皮子,下部是染成红色的牛皮的一种靴子\end{définition}
\begin{exemple}\jya tɯ-xtsa kɤɣɯɕqri nɯ ɯ-ɕna ɯ-rkɯ cho ɯ-komɤr nɯ ra li mbanaʁtsa cho naχtɕɯɣ ri komɤr ɯ-sɤ-tʂɯβ tɤ-ri nɯ ɯ-mdoʁ kɯ-mpɕɤr ku-lɤt-nɯ ŋgrɤl. ɯ-rkɯ ɯ-tɯ-tʂɯβ nɯ ra kɤntɕhɯ ku-lɤt-nɯ ŋgrɤl. ɯ-rkɯ ɯ-taʁ tɕe komɤr pjɯ-tshoʁ-nɯ ɯ-taʁ kóʁmɯz cɤndʐi, nɯ maʁ nɤ mphrɯɣ nɯ maʁ nɤ raz kɯ-mpɕɤr kɯ-fse pjɯ-tshoʁ-nɯ. ɯ-xtsɤkɤŋgɯ nɯ li smɤɣ thɯ-kɤ-βzu pjɯ-tshoʁ-nɯ ŋu. tɕe nɯ tɯ-xtsa nɯ kɤ-tʂɯβ ɴqa, tɕe kɯ-mpɕɤr kɤ-nɯ-sɯpa ŋu\cmn 
\stylefv{tɯxtsa kɤɣɯɕqri}(鞋的一种)的鞋尖、鞋边和鞋背上用红皮子,(做法)和黑皮鞋一样。缝红皮子要用彩色的线,缝鞋边的时候要缝几道。鞋边上面先缝上红皮子(作鞋面),然后上面再缝上麝香鹿皮、氆氇或者美观的棉布(作鞋筒)。鞋筒的内层又是用羊毛制成的。缝这种鞋子很困难,但是可以当作是(藏族服装)的装饰品之一。
\end{exemple}\end{entrée}

\begin{entrée}
\vedette{\hypertarget{Ⓔtɯ-xtsɤkɤŋgɯ}{\papi{ tɯ-xtsɤkɤŋgɯ}}}\markboth{tɯ-xtsɤkɤŋgɯ}{}\classe{np}
\begin{définition}\fra intérieur de la botte\end{définition}
\begin{définition}\cmn 靴筒内衬\end{définition}
\end{entrée}

\begin{entrée}
\vedette{\hypertarget{Ⓔtɯ-xtsɤmbe}{\papi{ tɯ-xtsɤmbe}}}\markboth{tɯ-xtsɤmbe}{}\classe{np}
\begin{définition}\fra chaussures abîmées\end{définition}
\begin{définition}\cmn 破旧的鞋子\end{définition}
\begin{relation-sémantique}\confer{
\hyperlink{Ⓔtɯ-xtsa}{\textit{ \papi{tɯ-xtsa}}}
}\end{relation-sémantique}
\begin{relation-sémantique}\confer{
\hyperlink{Ⓔtɤ-mbe}{\textit{ \papi{tɤ-mbe}}}
}\end{relation-sémantique}\end{entrée}

\begin{entrée}
\vedette{\hypertarget{Ⓔtɯ-xtshi}{\papi{ tɯ-xtshi}}}\markboth{tɯ-xtshi}{}\classe{clf}
\begin{définition}\fra une fois\end{définition}
\begin{définition}\cmn 一次\end{définition}
\end{entrée}

\begin{entrée}
\vedette{\hypertarget{Ⓔtɯ-χpɣi}{\papi{ tɯ-χpɣi}}}\markboth{tɯ-χpɣi}{}\classe{np}
\begin{définition}\fra cuisse\end{définition}
\begin{définition}\cmn 大腿
\begin{déclaration} \étymologie{\papi{bʲin.pa}}\end{déclaration}\end{définition}
\end{entrée}

\begin{entrée}
\vedette{\hypertarget{Ⓔtɯ-χpɯm}{\papi{ tɯ-χpɯm}}}\markboth{tɯ-χpɯm}{}
\classe{np}
\begin{définition}\fra genou\end{définition}
\begin{définition}\cmn 膝盖\end{définition}
\begin{exemple}\jya a-χpɯm pɯ-tshoʁ-a (lɤ-tshoʁ-a)\cmn 我跪下了\end{exemple}
\begin{relation-sémantique}\confer{
 \papi{tɯ-χpɯm,tshoʁ}
}\end{relation-sémantique}\end{entrée}

\begin{entrée}
\vedette{\hypertarget{Ⓔtɯ-χpɯmɯrna}{\papi{ tɯ-χpɯmɯrna}}}\markboth{tɯ-χpɯmɯrna}{}\classe{np}
\begin{définition}\fra os du genou\end{définition}
\begin{définition}\cmn 膝盖骨\end{définition}
\end{entrée}

\begin{entrée}
\vedette{\hypertarget{Ⓔtɯ-χsɯmχsoz}{\papi{ tɯ-χsɯmχsoz}}}\markboth{tɯ-χsɯmχsoz}{} (\variante{tɯ-χsoŋχsɤz}) \classe{np}
\begin{définition}\fra vigueur\end{définition}
\begin{définition}\cmn 精神\end{définition}
\begin{exemple}\jya nɤ-χsɯmχsoz ɯ-tɯ-me nɯ!\cmn 你真没有精神\end{exemple}\begin{sous-entrée}
\vedette{\hypertarget{}{\papi{ ɯ-χsoŋχsɤz,ɣi}}}\markboth{ɯ-χsoŋχsɤz,ɣi}{}
\begin{définition}\fra retrouver son énergie\end{définition}
\begin{définition}\cmn 有精神\end{définition}
\begin{exemple}\jya ɯ-χsoŋχsɤz to-ɣi\cmn 精神提起来了\end{exemple}
\begin{exemple}\jya a-χsoŋχsɤz mɯ́j-ɣi\cmn 我没有精神\end{exemple}
\begin{relation-sémantique}\ComponentA{\classe{np}
 \papi{ɯ-χsoŋχsɤz}
}\end{relation-sémantique}
\begin{relation-sémantique}\ComponentB{\classe{vi}
\hyperlink{Ⓔɣi}{\textit{ \papi{ɣi}}}
}\end{relation-sémantique}
\end{sous-entrée}\begin{sous-entrée}
\vedette{\hypertarget{}{\papi{ ɯ-χsoŋχsɤz,sɯɣe}}}\markboth{ɯ-χsoŋχsɤz,sɯɣe}{}
\begin{définition}\fra donner de l'énergie\end{définition}
\begin{définition}\cmn 提精神\end{définition}
\begin{exemple}\jya tʂha kú-wɣ-tshi tɕe tɯ-χsoŋχsɤz tu-sɯɣe ɲɯ-ŋu\cmn 喝茶就可以提精神\end{exemple}
\end{sous-entrée}\end{entrée}

\begin{entrée}
\vedette{\hypertarget{Ⓔtɯ-χti}{\papi{ tɯ-χti}}}\markboth{tɯ-χti}{}\classe{np}
\begin{définition}\fra compagnon\end{définition}
\begin{définition}\cmn 伙伴\end{définition}
\end{entrée}

\begin{entrée}
\vedette{\hypertarget{Ⓔtɯ-χtispa}{\papi{ tɯ-χtispa}}}\markboth{tɯ-χtispa}{}\classe{np}
\begin{définition}\fra fiancé\end{définition}
\begin{définition}\cmn 未婚夫\end{définition}
\end{entrée}

\begin{entrée}
\vedette{\hypertarget{Ⓔtɯz}{\papi{ tɯz}}}\markboth{tɯz}{}\classe{n}
\begin{définition}\fra époque\end{définition}
\begin{définition}\cmn 时代
\begin{déclaration} \étymologie{\papi{dus}}\end{déclaration}\end{définition}
\begin{exemple}\jya ndʑi-tɯz a-pɯ-βdi\cmn 祝你们俩一生平安\end{exemple}\end{entrée}

\begin{entrée}
\vedette{\hypertarget{Ⓔtɯ-zboʁ}{\papi{ tɯ-zboʁ}}}\markboth{tɯ-zboʁ}{}\classe{clf}
\begin{définition}\fra une poignée (herbes, oignon)\end{définition}
\begin{définition}\cmn 一把(草,葱)\end{définition}
\begin{exemple}\jya xɕaj tɯ-zboʁ\cmn 一把草\end{exemple}
\begin{exemple}\jya ɕku tɯ-zboʁ\cmn 一把葱子\end{exemple}
\begin{exemple}\jya tasa tɯ-zboʁ\cmn 一把麻皮\end{exemple}\end{entrée}

\begin{entrée}
\vedette{\hypertarget{Ⓔtɯ-zda}{\papi{ tɯ-zda}}}\markboth{tɯ-zda}{}\classe{np}
\begin{définition}\fra compagnon, autre\end{définition}
\begin{définition}\cmn 伙伴;别人\end{définition}
\begin{exemple}\jya kha ɯ-zda kɤ-rɤʑit-a\cmn 我待在家里陪他了\end{exemple}
\begin{relation-sémantique}\confer{
\hyperlink{Ⓔɣɤzda}{\textit{ \papi{ɣɤzda}}}
}\end{relation-sémantique}
\begin{relation-sémantique}\confer{
\hyperlink{Ⓔrɤzda}{\textit{ \papi{rɤzda}}}
}\end{relation-sémantique}
\begin{relation-sémantique}\confer{
\hyperlink{Ⓔsɤzda}{\textit{ \papi{sɤzda}}}
}\end{relation-sémantique}\end{entrée}

\begin{entrée}
\vedette{\hypertarget{Ⓔtɯ-zgoɕɤrɯ}{\papi{ tɯ-zgoɕɤrɯ}}}\markboth{tɯ-zgoɕɤrɯ}{}\classe{np}
\begin{définition}\fra colonne vertébrale\end{définition}
\begin{définition}\cmn 脊椎骨\end{définition}
\end{entrée}

\begin{entrée}
\vedette{\hypertarget{Ⓔtɯ-zgrɯ}{\papi{ tɯ-zgrɯ}}}\markboth{tɯ-zgrɯ}{}\classe{np}
\begin{définition}\fra coude\end{définition}
\begin{définition}\cmn 肘\end{définition}
\begin{relation-sémantique}\confer{
\hyperlink{Ⓔtɯ-ɣrɯmke}{\textit{ \papi{tɯ-ɣrɯmke}}}
}\end{relation-sémantique}
\begin{relation-sémantique}\confer{
\hyperlink{Ⓔzgrɯtɕhɯ}{\textit{ \papi{zgrɯtɕhɯ}}}
}\end{relation-sémantique}
\end{entrée}

\begin{entrée}
\vedette{\hypertarget{Ⓔtɯ-zloʁ}{\papi{ tɯ-zloʁ}}}\markboth{tɯ-zloʁ}{}\classe{clf}
\begin{définition}\fra fois\end{définition}
\begin{définition}\cmn 一倍\end{définition}
\begin{exemple}\jya ʁɲɯ-zloʁ\cmn 两倍\end{exemple}
\begin{exemple}\jya kɯβde nɯ ʁnɯz ɣɯ ʁɲɯ-zloʁ ŋu\cmn 四是二的两倍\end{exemple}\end{entrée}

\begin{entrée}
\vedette{\hypertarget{Ⓔtɯ-ʑi,loʁ}{\papi{ tɯ-ʑi,loʁ}}}\markboth{tɯ-ʑi,loʁ}{}\begin{définition}\fra avoir la nausée\end{définition}
\begin{définition}\cmn 感到恶心\end{définition}
\begin{exemple}\jya ɯʑo ɯ-ʑi ɲɯ-loʁ\cmn 他感到恶心\end{exemple}
\begin{exemple}\jya a-ʑi ɲɯ-loʁ\cmn 我感到恶心\end{exemple}
\begin{relation-sémantique}\confer{
\hyperlink{Ⓔnɤʑɯloʁ}{\textit{ \papi{nɤʑɯloʁ}}}
}\end{relation-sémantique}
\begin{relation-sémantique}\ComponentA{\classe{np}
 \papi{tɯ-ʑi}
}\end{relation-sémantique}
\begin{relation-sémantique}\ComponentB{\classe{vs}
\hyperlink{Ⓔloʁ}{\textit{ \papi{loʁ}}}
}\end{relation-sémantique}\end{entrée}

\begin{entrée}
\vedette{\hypertarget{Ⓔtɯʑŋgrɯm}{\papi{ tɯʑŋgrɯm}}}\markboth{tɯʑŋgrɯm}{}\classe{n}
\begin{définition}\fra cartilage\end{définition}
\begin{définition}\cmn 软骨\end{définition}
\begin{exemple}\jya a-ɕna ɣɯ ɯ-tɯʑŋgrɯm\cmn 我鼻子的软骨\end{exemple}\end{entrée}

\begin{entrée}
\vedette{\hypertarget{Ⓔtɯʑo}{\papi{ tɯʑo}}}\markboth{tɯʑo}{}\classe{pro}
\begin{définition}\fra soi-même (générique)\end{définition}
\begin{définition}\cmn 自己(泛指的人称代词)\end{définition}
\end{entrée}

\begin{entrée}
\vedette{\hypertarget{Ⓔtɯʑo-sɯso}{\papi{ tɯʑo-sɯso}}}\markboth{tɯʑo-sɯso}{}
\classe{np}
\begin{définition}\fra n'en faire qu'à sa tête\end{définition}
\begin{définition}\cmn 随心所欲\end{définition}
\begin{exemple}\jya aʑo-sɯso kɤ-nɯpa mɯ́j-khɯ\cmn 我不能随心所欲\end{exemple}
\begin{relation-sémantique}\confer{
\hyperlink{Ⓔtɯ-sɯso}{\textit{ \papi{tɯ-sɯso}}}
}\end{relation-sémantique}
\begin{relation-sémantique}\confer{
\hyperlink{Ⓔsɯso}{\textit{ \papi{sɯso}}}
}\end{relation-sémantique}\end{entrée}

\begin{entrée}
\vedette{\hypertarget{Ⓔtɯ-ʑrɤz}{\papi{ tɯ-ʑrɤz}}}\markboth{tɯ-ʑrɤz}{}\classe{clf}
\begin{définition}\fra une bande (couleur)\end{définition}
\begin{définition}\cmn 一路(颜色)\end{définition}
\end{entrée}

\begin{entrée}
\vedette{\hypertarget{Ⓔtɯ-ʑɯβ}{\papi{ tɯ-ʑɯβ}}}\markboth{tɯ-ʑɯβ}{}
\classe{np}
\begin{définition}\fra somnolence\end{définition}
\begin{définition}\cmn 瞌睡\end{définition}
\begin{exemple}\jya a-ʑɯβ mɯ́j-ɣi\cmn 我睡不着\end{exemple}
\begin{exemple}\jya a-ʑɯβ mɯ́j-sɯɣe-nɯ\cmn 他们不让我睡觉\end{exemple}
\begin{exemple}\jya kɤntɕhaʁ ɲɯ-ɣɤɣɯrɣɯrnɯ tɕe, a-ʑɯβ mɯ́j-sɯɣe-nɯ\cmn 街上很吵,弄得我睡不着\end{exemple}
\begin{exemple}\jya a-rɣa ɲɯ-ɣɤɕqali-nɯ tɕe, a-ʑɯβ mɯ́j-sɯɣe-nɯ\end{exemple}
\begin{exemple}\jya a-rɣa ɲɯ-ɣɤɲcɣɤlɤt-nɯ tɕe, a-ʑɯβ mɯ́j-sɯɣe-nɯ\cmn 我的邻居很吵,弄得我睡不着\end{exemple}
\begin{relation-sémantique}\confer{
\hyperlink{Ⓔnɯʑɯβ}{\textit{ \papi{nɯʑɯβ}}}
}\end{relation-sémantique}\end{entrée}

\newpage\caractère{ɯ}

\begin{entrée}
\vedette{\hypertarget{Ⓔɯ-βdoʁ}{\papi{ ɯ-βdoʁ}}}\markboth{ɯ-βdoʁ}{}\classe{np}
\begin{définition}\fra pénis, zizi (enfant)\end{définition}
\begin{définition}\cmn 阴茎\end{définition}\end{entrée}

\begin{entrée}
\vedette{\hypertarget{Ⓔɯ-βlu}{\papi{ ɯ-βlu}}}\markboth{ɯ-βlu}{}\classe{np}
\begin{définition}\fra méthode, astuce, stratagème\end{définition}
\begin{définition}\cmn 办法,计谋
\begin{déclaration} \étymologie{\papi{blo}}\end{déclaration}\end{définition}
\begin{exemple}\jya nɤ-βlu ci tu-tɕat-a\cmn 我给你出个主意\end{exemple}
\begin{relation-sémantique}\confer{
\hyperlink{Ⓔɯ-βlaβlu}{\textit{ \papi{ɯ-βlaβlu}}}
}\end{relation-sémantique}
\begin{relation-sémantique}\confer{
\hyperlink{Ⓔnɯβlu}{\textit{ \papi{nɯβlu}}}
}\end{relation-sémantique}
\begin{relation-sémantique}\confer{
\hyperlink{Ⓔaɣɯβlu}{\textit{ \papi{aɣɯβlu}}}
}\end{relation-sémantique}\end{entrée}

\begin{entrée}
\vedette{\hypertarget{Ⓔɯ-βlaβlu}{\papi{ ɯ-βlaβlu}}}\markboth{ɯ-βlaβlu}{}\classe{n}
\begin{définition}\fra astuce, méthode\end{définition}
\begin{définition}\cmn 办法\end{définition}
\begin{exemple}\jya nɤ-βlaβlu ɲɯ-dɤn\cmn 你有很多办法\end{exemple}
\begin{relation-sémantique}\confer{
\hyperlink{Ⓔɯ-βlu}{\textit{ \papi{ɯ-βlu}}}
}\end{relation-sémantique}\end{entrée}

\begin{entrée}
\vedette{\hypertarget{Ⓔɯ-βraʁ}{\papi{ ɯ-βraʁ}}}\markboth{ɯ-βraʁ}{}
\classe{np}
\begin{définition}\fra symbole, signe\end{définition}
\begin{définition}\cmn 象征;预兆\end{définition}
\begin{exemple}\jya ɣɯjpa qartsɯ ʁmɯrcɯ ɲɯ-dɤn tɕe, nɯ tɤɕi kɯ-pe ɯ-βraʁ ŋu\cmn 今年夏天画眉很多,表示今年青稞的产量很高\end{exemple}
\end{entrée}

\begin{entrée}
\vedette{\hypertarget{Ⓔɯ-βzɯr}{\papi{ ɯ-βzɯr}}}\markboth{ɯ-βzɯr}{}\classe{np}
\begin{définition}\fra angle\end{définition}
\begin{définition}\cmn 角落
\begin{déclaration} \étymologie{\papi{bzur}}\end{déclaration}\end{définition}
\begin{exemple}\jya tɕoχtsi ɣɯ ɯ-βzɯr ri a-jaʁ kɤ-nɯ-rpu-t-a tɕe ɲɯ-mŋɤm\cmn 我把手撞到桌子的一角,很痛\end{exemple}\end{entrée}

\begin{entrée}
\vedette{\hypertarget{Ⓔɯ-cu}{\papi{ ɯ-cu}}}\markboth{ɯ-cu}{}
\classe{np}
\begin{définition}\fra ingrédients\end{définition}
\begin{définition}\cmn 配料\end{définition}
\begin{exemple}\jya hajtsu cho ɕku nɯ tɤjko ɯ-cu pjɯ́-wɣ-lɤt tɕe mɯm\cmn 在酸菜里添加黑椒和蒜就好吃\end{exemple}
\begin{exemple}\jya a-pi kɯ tatshi ɯ-skɤt ɣɯ ɯ-cu kɤmɲɯ tu-βze ɲɯ-ŋu tɕe ɲɯ-sɤtso, tatshi skɤt ʁɟa a-tɤ-ti tɕe mɯ́j-sɤtso\cmn 他把干木鸟话掺到大藏话里面说就好懂一些\end{exemple}
\begin{relation-sémantique}\confer{
\hyperlink{Ⓔacu}{\textit{ \papi{acu}}}
}\end{relation-sémantique}\end{entrée}

\begin{entrée}
\vedette{\hypertarget{Ⓔɯ-cɤβ}{\papi{ ɯ-cɤβ}}}\markboth{ɯ-cɤβ}{}\classe{np}
\begin{définition}\fra gousse, cosse\end{définition}
\begin{définition}\cmn 荚果\end{définition}
\begin{exemple}\jya staχpɯ ɯ-cɤβ\cmn 豌豆的荚果\end{exemple}
\begin{relation-sémantique}\confer{
\hyperlink{Ⓔrɤcɤβ}{\textit{ \papi{rɤcɤβ}}}
}\end{relation-sémantique}\end{entrée}

\begin{entrée}
\vedette{\hypertarget{Ⓔɯ-ciqa}{\papi{ ɯ-ciqa}}}\markboth{ɯ-ciqa}{}\classe{adv}
\begin{définition}\fra origine, depuis toujours, en fait\end{définition}
\begin{définition}\cmn 根源;从来;说到底\end{définition}
\begin{exemple}\jya ɯʑo ɯ-ciqa ʑo pɯ-nɯ-pɯ-pe ɕti\cmn 他从来都是个好人\end{exemple}\end{entrée}

\begin{entrée}
\vedette{\hypertarget{Ⓔɯ-cɯma}{\papi{ ɯ-cɯma}}}\markboth{ɯ-cɯma}{}\classe{np}
\begin{définition}\fra malheur, calamité\end{définition}
\begin{définition}\cmn 闯的祸\end{définition}
\begin{exemple}\jya aʑo a-cɯma ŋu\cmn 是我闯的祸\end{exemple}\end{entrée}

\begin{entrée}
\vedette{\hypertarget{Ⓔɯ-ɕi}{\papi{ ɯ-ɕi}}}\markboth{ɯ-ɕi}{}\classe{np}
\begin{définition}\fra plante fanée\end{définition}
\begin{définition}\cmn 枯萎了的草(一年生的植物)\end{définition}
\begin{exemple}\jya dɤrʁɯ ɯ-ɕi\cmn 枯萎了的蕨苔\end{exemple}\end{entrée}

\begin{entrée}
\vedette{\hypertarget{Ⓔɯ-ɕki}{\papi{ ɯ-ɕki}}}\markboth{ɯ-ɕki}{}\classe{postp}\acception{1}
\begin{définition}\fra à côté\end{définition}
\begin{définition}\cmn 旁边\end{définition}
\begin{relation-sémantique}\synonyme{
\hyperlink{Ⓔɯ-rkɯ}{\textit{ \papi{ɯ-rkɯ}}}
}\end{relation-sémantique}\acception{2}
\begin{définition}\fra datif\end{définition}
\begin{définition}\cmn 与格\end{définition}
\begin{relation-sémantique}\synonyme{
\hyperlink{Ⓔɯ-phe}{\textit{ \papi{ɯ-phe}}}
}\end{relation-sémantique}\end{entrée}

\begin{entrée}
\vedette{\hypertarget{Ⓔɯ-ɕpɯz}{\papi{ ɯ-ɕpɯz}}}\markboth{ɯ-ɕpɯz}{}
\classe{np}
\begin{définition}\fra imitation, dessin\end{définition}
\begin{définition}\cmn 模样;假冒\end{définition}
\begin{exemple}\jya jɯɣi ɯ-taʁ tɯrme ɯ-ɕpɯz ci ɣɤʑu\cmn 书上有人的模样\end{exemple}
\begin{exemple}\jya tɯrme ɯ-ɕpɯz to-βzu\cmn 他做了个人的模样\end{exemple}
\begin{exemple}\jya nɤki nɤ-khɯtsa ɯ-ɕpɯz ɲɯ-ɕti ma koŋla ɲɯ-maʁ\cmn 你那个碗是伪造品,不是真的\end{exemple}
\begin{relation-sémantique}\confer{
\hyperlink{Ⓔnɯɕpɯz}{\textit{ \papi{nɯɕpɯz}}}
}\end{relation-sémantique}\end{entrée}

\begin{entrée}
\vedette{\hypertarget{Ⓔɯ-ɕtɯrme}{\papi{ ɯ-ɕtɯrme}}}\markboth{ɯ-ɕtɯrme}{}\classe{np}
\begin{définition}\fra poils pubien (femme)\end{définition}
\begin{définition}\cmn (女子的)阴毛\end{définition}\end{entrée}

\begin{entrée}
\vedette{\hypertarget{Ⓔɯ-ɕɯɕaŋ}{\papi{ ɯ-ɕɯɕaŋ}}}\markboth{ɯ-ɕɯɕaŋ}{}\classe{np}
\begin{définition}\fra limite\end{définition}
\begin{définition}\cmn 界限\end{définition}
\begin{exemple}\jya ki ɯ-ɕɯɕaŋ ki kɤmɲɯ sɤtɕha ŋu\cmn 以这个为界就是干木鸟村的地盘\end{exemple}
\end{entrée}

\begin{entrée}
\vedette{\hypertarget{Ⓔɯ-du}{\papi{ ɯ-du}}}\markboth{ɯ-du}{}\classe{np}
\begin{définition}\fra (être tiré) au sort\end{définition}
\begin{définition}\cmn 抽到了(抽签的时候)\end{définition}
\begin{exemple}\jya ɯ-du to-ɣi\cmn 他抽到了签\end{exemple}
\begin{exemple}\jya a-du tɤ-ɣe\cmn 我抽到了签\end{exemple}\end{entrée}

\begin{entrée}
\vedette{\hypertarget{Ⓔɯ-do}{\papi{ ɯ-do}}}\markboth{ɯ-do}{}\classe{np}
\begin{définition}\fra vieux\end{définition}
\begin{définition}\cmn 老的\end{définition}\end{entrée}

\begin{entrée}
\vedette{\hypertarget{Ⓔɯ-dɯɕŋaʁ}{\papi{ ɯ-dɯɕŋaʁ}}}\markboth{ɯ-dɯɕŋaʁ}{}\classe{np}
\begin{définition}\fra puanteur très forte\end{définition}
\begin{définition}\cmn 特别浓的臭味\end{définition}
\begin{exemple}\jya ndʑiŋgri ɯ-dɯɕŋaʁ ɲɯ-sɤjloʁ\cmn 臭草非常臭\end{exemple}\end{entrée}

\begin{entrée}
\vedette{\hypertarget{Ⓔɯ-dɯχɯn}{\papi{ ɯ-dɯχɯn}}}\markboth{ɯ-dɯχɯn}{}
\classe{np}
\begin{définition}\fra bonne odeur\end{définition}
\begin{définition}\cmn 香味\end{définition}
\begin{exemple}\jya ɯ-dɯχɯn pɯ-mtsham-a\cmn 我闻到了香味\end{exemple}
\begin{exemple}\jya ɯ-dɯχɯn tɤ-nɤmnam-a\cmn 我闻了一下香味\end{exemple}
\begin{relation-sémantique}\synonyme{
\hyperlink{Ⓔɯ-ɟɤm}{\textit{ \papi{ɯ-ɟɤm}}}
}\end{relation-sémantique}
\begin{relation-sémantique}\confer{
\hyperlink{Ⓔaɣɯdɯχɯn}{\textit{ \papi{aɣɯdɯχɯn}}}
}\end{relation-sémantique}
\begin{relation-sémantique}\confer{
\hyperlink{Ⓔtɤ-di}{\textit{ \papi{tɤ-di}}}
}\end{relation-sémantique}\end{entrée}

\begin{entrée}
\vedette{\hypertarget{Ⓔɯ-fsu}{\papi{ ɯ-fsu}}}\markboth{ɯ-fsu}{}\classe{np}
\begin{définition}\fra égal, autant\end{définition}
\begin{définition}\cmn 一样\end{définition}
\begin{exemple}\jya tɤ-pɤtso cho-wxti tɕe, ɯ-mu ɯ-fsu ʑo cho-ɬoʁ\cmn 孩子长大了,长得跟他母亲一样高\end{exemple}
\begin{relation-sémantique}\confer{
\hyperlink{Ⓔsɤfsu}{\textit{ \papi{sɤfsu}}}
}\end{relation-sémantique}
\begin{relation-sémantique}\confer{
\hyperlink{Ⓔɣɯfsu}{\textit{ \papi{ɣɯfsu}}}
}\end{relation-sémantique}
\begin{relation-sémantique}\confer{
\hyperlink{Ⓔafsɯfsu}{\textit{ \papi{afsɯfsu}}}
}\end{relation-sémantique}\end{entrée}

\begin{entrée}
\vedette{\hypertarget{Ⓔɯ-grɤl}{\papi{ ɯ-grɤl}}}\markboth{ɯ-grɤl}{}\classe{np}
\begin{définition}\fra raison\end{définition}
\begin{définition}\cmn 理由;规律;条理
\begin{déclaration} \étymologie{\papi{gral}}\end{déclaration}\end{définition}
\begin{exemple}\jya laχtɕha ɯ-grɤl kɯ-me ʑo ɲɤ-ta\cmn 东西放得没有条理\end{exemple}
\begin{exemple}\jya nɤ-grɤl ɯ-tɯ-me\cmn 你无法无天,没有分寸,没有礼貌\end{exemple}
\begin{relation-sémantique}\confer{
\hyperlink{Ⓔnɯgrɤl}{\textit{ \papi{nɯgrɤl}}}
}\end{relation-sémantique}\end{entrée}

\begin{entrée}
\vedette{\hypertarget{Ⓔɯ-ɣɲaʁ}{\papi{ ɯ-ɣɲaʁ}}}\markboth{ɯ-ɣɲaʁ}{}\classe{np}
\begin{définition}\fra résultat désastreux (que l'on se cause à soi-même)\end{définition}
\begin{définition}\cmn 恶果(自己给自己带来的)\end{définition}
\begin{exemple}\jya a-ɣɲaʁ to-nɯ-βzu-t-a\cmn 我自己给自己找了麻烦\end{exemple}
\begin{exemple}\jya ɯ-ɣɲaʁ ɲɤ-ɕe\cmn 对他不利了\end{exemple}
\begin{exemple}\jya mtshalu ɲɤ-phɯt ri kó-wɣ-mtsɯɣ tɕe mɤ́ɣrɤz nɤ ɯ-ɣɲaʁ ɲɤ-ɕe\cmn 他摘荨麻的时候被蛰到了,倒是自找麻烦\end{exemple}\end{entrée}

\begin{entrée}
\vedette{\hypertarget{Ⓔɯ-ɣɲɟɯ}{\papi{ ɯ-ɣɲɟɯ}}}\markboth{ɯ-ɣɲɟɯ}{}\classe{np}
\begin{définition}\fra orifice\end{définition}
\begin{définition}\cmn 洞\end{définition}
\begin{relation-sémantique}\confer{
\hyperlink{Ⓔɲɟɯ}{\textit{ \papi{ɲɟɯ}}}
}\end{relation-sémantique}
\begin{relation-sémantique}\confer{
\hyperlink{Ⓔkhɯɣɲɟɯ}{\textit{ \papi{khɯɣɲɟɯ}}}
}\end{relation-sémantique}
\begin{relation-sémantique}\confer{
\hyperlink{Ⓔqaɲɯɣɲɟɯ}{\textit{ \papi{qaɲɯɣɲɟɯ}}}
}\end{relation-sémantique}
\begin{relation-sémantique}\confer{
\hyperlink{Ⓔtɯ-ɕnɤɣɲɟɯ}{\textit{ \papi{tɯ-ɕnɤɣɲɟɯ}}}
}\end{relation-sémantique}
\begin{relation-sémantique}\confer{
\hyperlink{Ⓔtɤkhɯɣɲɟɯ}{\textit{ \papi{tɤkhɯɣɲɟɯ}}}
}\end{relation-sémantique}
\begin{relation-sémantique}\confer{
\hyperlink{Ⓔtɯ-rnɤɣɲɟɯ}{\textit{ \papi{tɯ-rnɤɣɲɟɯ}}}
}\end{relation-sémantique}
\end{entrée}

\begin{entrée}
\vedette{\hypertarget{Ⓔɯ-ɣrom}{\papi{ ɯ-ɣrom}}}\markboth{ɯ-ɣrom}{}\classe{np}
\begin{définition}\fra (chose) séchée\end{définition}
\begin{définition}\cmn 干的\end{définition}
\begin{définition}\jya \end{définition}
\begin{exemple}\jya tɤjmɤɣ ɯ-ɣrom\cmn 干菌子\end{exemple}
\begin{exemple}\jya tɤ-mthɯm ɯ-ɣrom\cmn 干的肉\end{exemple}
\begin{relation-sémantique}\confer{
\hyperlink{Ⓔrom}{\textit{ \papi{rom}}}
}\end{relation-sémantique}\end{entrée}

\begin{entrée}
\vedette{\hypertarget{Ⓔɯ-jlu}{\papi{ ɯ-jlu}}}\markboth{ɯ-jlu}{}\classe{np}\acception{1}
\begin{définition}\fra cru, pas cuit\end{définition}
\begin{définition}\cmn 生的\end{définition}
\begin{exemple}\jya stoʁ ɯ-jlu kɤ-ndza mɤ-mɯm\cmn 生胡豆不好吃\end{exemple}\acception{2}
\begin{définition}\fra uniquement\end{définition}
\begin{définition}\cmn 光是\end{définition}
\begin{exemple}\jya @cai kɯnɤ tɤ-ndze, tɯmgo ɯ-jlu ʑo ma-tɤ-tɯ-ndze\cmn 你也吃菜,不要光吃饭\end{exemple}\end{entrée}

\begin{entrée}
\vedette{\hypertarget{Ⓔɯ-jndoʁ}{\papi{ ɯ-jndoʁ}}}\markboth{ɯ-jndoʁ}{}\classe{np}
\begin{définition}\fra racine de radis ou de navet\end{définition}
\begin{définition}\cmn 芜菁;萝卜的根\end{définition}
\end{entrée}

\begin{entrée}
\vedette{\hypertarget{Ⓔɯ-jndɯz}{\papi{ ɯ-jndɯz}}}\markboth{ɯ-jndɯz}{}\classe{np}
\begin{définition}\fra fils qui débordent, verticilles\end{définition}
\begin{définition}\cmn 絮线\end{définition}
\begin{exemple}\jya thaχtsa ɯ-sɤ-sɤʑa cho ɯ-sɤɣ-jɤɣ ɯ-tɤ-ri nɯ ɯ-jndɯz rmi\cmn 花带的开头和结尾部分的线叫絮线\end{exemple}\end{entrée}

\begin{entrée}
\vedette{\hypertarget{Ⓔɯ-jŋgɯ}{\papi{ ɯ-jŋgɯ}}}\markboth{ɯ-jŋgɯ}{}
\classe{np}\acception{1}
\begin{définition}\fra mangeoir\end{définition}
\begin{définition}\cmn 槽\end{définition}\acception{2}
\begin{définition}\fra gamelle (du chien)\end{définition}
\begin{définition}\cmn 狗槽;狗碗;槽\end{définition}
\begin{exemple}\jya khɯna kɯ ɯ-jŋgɯ to-nɯntsɯɣ\cmn 狗舔了它的碗\end{exemple}
\begin{relation-sémantique}\confer{
\hyperlink{Ⓔpɤjŋgɯ}{\textit{ \papi{pɤjŋgɯ}}}
}\end{relation-sémantique}
\begin{relation-sémantique}\confer{
\hyperlink{Ⓔqhajŋgɯ}{\textit{ \papi{qhajŋgɯ}}}
}\end{relation-sémantique}
\begin{relation-sémantique}\confer{
\hyperlink{Ⓔkhɯjŋgɯ}{\textit{ \papi{khɯjŋgɯ}}}
}\end{relation-sémantique}\end{entrée}

\begin{entrée}
\vedette{\hypertarget{Ⓔɯjo}{\papi{ ɯjo}}}\markboth{ɯjo}{}
\classe{n}
\begin{définition}\fra cochon de deux ans\end{définition}
\begin{définition}\cmn 两岁的猪
\end{définition}
\begin{exemple}\jya paʁ ɯ-jo\cmn 两岁的猪\end{exemple}
\begin{exemple}\jya phaʁrgot ɯ-jo\cmn 两岁的野猪\end{exemple}\end{entrée}

\begin{entrée}
\vedette{\hypertarget{Ⓔɯ-jɯ}{\papi{ ɯ-jɯ}}}\markboth{ɯ-jɯ}{}\classe{np}
\begin{définition}\fra poignée\end{définition}
\begin{définition}\cmn 把子
\begin{déclaration} \étymologie{\papi{ju.ba}}\end{déclaration}\end{définition}\end{entrée}

\begin{entrée}
\vedette{\hypertarget{Ⓔɯ-jɯja}{\papi{ ɯ-jɯja}}}\markboth{ɯ-jɯja}{}\classe{np}
\begin{définition}\fra au fur et à mesure que (+verbe)\end{définition}
\begin{définition}\cmn 随着(加动词)\end{définition}
\begin{exemple}\jya tɯ-mɯ kɤ-lɤt ɯ-jɯja tɕe, ɯ-thoʁ tu-ɣɤrcoʁ ɲɯ-ŋu\cmn 雨水越大,地面的稀泥越多\end{exemple}
\begin{exemple}\jya tɤ-pɤtso thɯ-wxti ɯ-jɯja nɤ, ɯ-ndzɯmbra kɤ-sɤsɯɣ ra\cmn 随着孩子长大,要加强对他的教育\end{exemple}\end{entrée}

\begin{entrée}
\vedette{\hypertarget{Ⓔɯ-ɟɤm}{\papi{ ɯ-ɟɤm}}}\markboth{ɯ-ɟɤm}{}\classe{np}
\begin{définition}\fra goût\end{définition}
\begin{définition}\cmn 香味
\begin{déclaration}\use{古语}\end{déclaration}\end{définition}
\begin{exemple}\jya ki cha ki ɯ-ɟɤm kɯ-mɯm ci ɲɯ-ŋu\cmn 这种茶味道很香\end{exemple}
\begin{relation-sémantique}\synonyme{
\hyperlink{Ⓔɯ-dɯχɯn}{\textit{ \papi{ɯ-dɯχɯn}}}
}\end{relation-sémantique}\end{entrée}

\begin{entrée}
\vedette{\hypertarget{Ⓔɯ-kɤcu}{\papi{ ɯ-kɤcu}}}\markboth{ɯ-kɤcu}{}\classe{np}
\begin{définition}\fra à l'est\end{définition}
\begin{définition}\cmn 在东方\end{définition}\end{entrée}

\begin{entrée}
\vedette{\hypertarget{Ⓔɯ-kɤlɤjme}{\papi{ ɯ-kɤlɤjme}}}\markboth{ɯ-kɤlɤjme}{}\classe{np}
\begin{définition}\fra tête à queue\end{définition}
\begin{définition}\cmn 头尾颠倒\end{définition}
\begin{exemple}\jya a-kɤlɤjme pɯ-ru tɕe, tɤ-rɤru-a\cmn 我跌倒了再爬起来\end{exemple}
\end{entrée}

\begin{entrée}
\vedette{\hypertarget{Ⓔɯ-kɤndɯt}{\papi{ ɯ-kɤndɯt}}}\markboth{ɯ-kɤndɯt}{}\classe{np}
\begin{définition}\fra amant, relation extra-conjugale\end{définition}
\begin{définition}\cmn 情人
\begin{déclaration} \étymologie{\papi{ɴdod}}\end{déclaration}\end{définition}\end{entrée}

\begin{entrée}
\vedette{\hypertarget{Ⓔɯ-kɤɲɟoʁ}{\papi{ ɯ-kɤɲɟoʁ}}}\markboth{ɯ-kɤɲɟoʁ}{}\classe{np}
\begin{définition}\fra tissu cousu sur l'ouverture au bas des robes\end{définition}
\begin{définition}\cmn 缝在衣服叉开的部分上面的布块\end{définition}\end{entrée}

\begin{entrée}
\vedette{\hypertarget{Ⓔɯ-kɤχcɤl}{\papi{ ɯ-kɤχcɤl}}}\markboth{ɯ-kɤχcɤl}{}\classe{np}
\begin{définition}\fra sur le dessus\end{définition}
\begin{définition}\cmn 在顶上,顶端\end{définition}
\begin{exemple}\jya nɤ-kɤχcɤl\cmn 在你的头顶部\end{exemple}
\begin{exemple}\jya si ɯ-kɤχcɤl\cmn 树的顶端\end{exemple}\end{entrée}

\begin{entrée}
\vedette{\hypertarget{Ⓔɯ-khon}{\papi{ ɯ-khon}}}\markboth{ɯ-khon}{}
\classe{np}
\begin{définition}\fra au cas où\end{définition}
\begin{définition}\cmn 以防万一\end{définition}
\begin{exemple}\jya aʑo tɯ-ci khro tɤ-rku-t-a tɕe, fso ɣɯ ɯ-khon ŋu\cmn 万一明天停水,我预备了一些水\end{exemple}
\begin{relation-sémantique}\confer{
\hyperlink{Ⓔrɯkhon}{\textit{ \papi{rɯkhon}}}
}\end{relation-sémantique}\end{entrée}

\begin{entrée}
\vedette{\hypertarget{Ⓔɯ-khrakhrɯ}{\papi{ ɯ-khrakhrɯ}}}\markboth{ɯ-khrakhrɯ}{}\classe{np}
\begin{définition}\fra (pains) rassis\end{définition}
\begin{définition}\cmn 干了的(馍馍)\end{définition}
\begin{exemple}\jya qajɣi ɯ-khrakhrɯ nɯ ɲɤ-mbi\cmn (大女儿)把干了的馍馍给他吃\end{exemple}
\begin{relation-sémantique}\confer{
\hyperlink{Ⓔkhrɯ}{\textit{ \papi{khrɯ}}}
}\end{relation-sémantique}\end{entrée}

\begin{entrée}
\vedette{\hypertarget{Ⓔɯ-khrɤt}{\papi{ ɯ-khrɤt}}}\markboth{ɯ-khrɤt}{}\classe{np}
\begin{définition}\fra déterminé\end{définition}
\begin{définition}\cmn 规定的\end{définition}
\begin{exemple}\jya smɤn kɤ-ndza nɯ ɯ-khrɤt kɯ-tu ɕti\cmn 吃是有规定的(时间和数量)\end{exemple}
\begin{exemple}\jya mkhɯrlu ɯ-ŋgɯ laχtɕha ɯ-khrɤt ma tú-wɣ-rku mɤ-khɯ\cmn 在车里装东西不能超过规定的重量\end{exemple}
\begin{exemple}\jya @xingqier tɯtshot χsɯm nɯ tɕiʑo @dianhua kɤ-lɤt ɯ-khrɤt nɯ-βzu-tɕi ŋu.\cmn 我们星期二三点钟约了时间打电话\end{exemple}
\begin{sous-entrée}
\vedette{\hypertarget{}{\papi{ ɯ-khrɤt,ndo}}}\markboth{ɯ-khrɤt,ndo}{}
\paradigme{\textit{dir :} \jya kɤ-}
\begin{définition}\fra contrôler\end{définition}
\begin{définition}\cmn 掌握\end{définition}
\begin{exemple}\jya soz kɤ-rɤru tɕhi jamar mda ɯ-khrɤt kú-wɣ-nɯ-ndo ra\cmn 要掌握好自己早上几点钟起床\end{exemple}
\begin{relation-sémantique}\ComponentA{\classe{np}
\hyperlink{Ⓔɯ-khrɤt}{\textit{ \papi{ɯ-khrɤt}}}
}\end{relation-sémantique}
\begin{relation-sémantique}\ComponentB{\classe{vt}
\hyperlink{Ⓔndo}{\textit{ \papi{ndo}}}
}\end{relation-sémantique}
\end{sous-entrée}\end{entrée}

\begin{entrée}
\vedette{\hypertarget{Ⓔɯ-khɯkha}{\papi{ ɯ-khɯkha}}}\markboth{ɯ-khɯkha}{}\classe{np}\acception{1}
\begin{définition}\fra l'un après l'autre\end{définition}
\begin{définition}\cmn 一个接着一个\end{définition}\acception{2}
\begin{définition}\fra au moment où\end{définition}
\begin{définition}\cmn 的时候\end{définition}
\begin{exemple}\jya ɲɯ-nɤqambɯmbjom ɯ-khɯkha ʑo ju-βji\cmn 小鸟飞的时候,隼来追它\end{exemple}\end{entrée}

\begin{entrée}
\vedette{\hypertarget{Ⓔɯ-khɯkhɤl}{\papi{ ɯ-khɯkhɤl}}}\markboth{ɯ-khɯkhɤl}{}\classe{np}
\begin{définition}\fra à certains endroits\end{définition}
\begin{définition}\cmn 某些地方或部位,或指东一块儿西一块儿(不是完整的一块)\end{définition}
\begin{exemple}\jya tɯ-ŋga ɯ-khɯkhɤl ko-spoʁ\cmn 衣服有些地方破了\end{exemple}
\begin{exemple}\jya tɯ-mɯ ɯ-khɯkhɤl ɲɤ-jɯm\cmn 有些地方没有云\end{exemple}
\begin{relation-sémantique}\confer{
\hyperlink{Ⓔtɯ-khɤl}{\textit{ \papi{tɯ-khɤl}}}
}\end{relation-sémantique}\end{entrée}

\begin{entrée}
\vedette{\hypertarget{Ⓔɯkɯki}{\papi{ ɯkɯki}}}\markboth{ɯkɯki}{}\classe{pro}
\begin{définition}\fra celui-ci\end{définition}
\begin{définition}\cmn 这个\end{définition}
\begin{relation-sémantique}\confer{
\hyperlink{Ⓔkɯki}{\textit{ \papi{kɯki}}}
}\end{relation-sémantique}
\end{entrée}

\begin{entrée}
\vedette{\hypertarget{Ⓔɯkɯkɯra}{\papi{ ɯkɯkɯra}}}\markboth{ɯkɯkɯra}{}\classe{pro}
\begin{définition}\fra ceux-ci\end{définition}
\begin{définition}\cmn 这些\end{définition}
\begin{relation-sémantique}\confer{
\hyperlink{Ⓔkɯra}{\textit{ \papi{kɯra}}}
}\end{relation-sémantique}
\end{entrée}

\begin{entrée}
\vedette{\hypertarget{Ⓔɯ-kɯmpɕɤr}{\papi{ ɯ-kɯmpɕɤr}}}\markboth{ɯ-kɯmpɕɤr}{}\classe{np}
\begin{définition}\fra décoration\end{définition}
\begin{définition}\cmn 装饰\end{définition}
\begin{exemple}\jya ɯ-kɯmpɕɤr ci nɤme-a\cmn 我要给他好看的!\end{exemple}
\begin{relation-sémantique}\confer{
\hyperlink{Ⓔmpɕɤr}{\textit{ \papi{mpɕɤr}}}
}\end{relation-sémantique}\end{entrée}

\begin{entrée}
\vedette{\hypertarget{Ⓔɯ-lu,cɯ}{\papi{ ɯ-lu,cɯ}}}\markboth{ɯ-lu,cɯ}{}
\paradigme{\textit{dir :} \jya pɯ-}
\begin{définition}\fra perdre conscience\end{définition}
\begin{définition}\cmn 昏倒;昏迷\end{définition}
\begin{exemple}\jya aʑo pɯ-xtɕi-a, khɤxtɤndo pɯ-atar-a tɕe, a-lu pjɤ-cɯ khi\cmn 我小的时候,从楼梯上摔下来,就昏倒了\end{exemple}
\begin{relation-sémantique}\ComponentA{\classe{np}
 \papi{ɯ-lu}
}\end{relation-sémantique}
\begin{relation-sémantique}\ComponentB{\classe{vt}
\hyperlink{ⒺcɯⒽ1}{\textit{ \papi{cɯ}}}
}\end{relation-sémantique}\begin{sous-entrée}
\vedette{\hypertarget{}{\papi{ ɯ-lu,sɯxcɯ}}}\markboth{ɯ-lu,sɯxcɯ}{}\classe{vt}
\paradigme{\textit{dir :} \jya pɯ-}
\begin{définition}\fra faire perdre conscience\end{définition}
\begin{définition}\cmn 令人失去知觉\end{définition}
\begin{exemple}\jya to-ʁndɯ tɕe ɯ-lu ʑo pjɤ-sɯxcɯ\cmn 他打了他,令他失去了知觉\end{exemple}
\begin{relation-sémantique}\ComponentA{\classe{np}
 \papi{ɯ-lu}
}\end{relation-sémantique}
\begin{relation-sémantique}\ComponentB{\classe{vt}
\hyperlink{Ⓔsɯxcɯ}{\textit{ \papi{sɯxcɯ}}}
}\end{relation-sémantique}
\end{sous-entrée}\end{entrée}

\begin{entrée}
\vedette{\hypertarget{Ⓔɯ-luj}{\papi{ ɯ-luj}}}\markboth{ɯ-luj}{}\classe{np}
\begin{définition}\fra enveloppe extérieure\end{définition}
\begin{définition}\cmn 外层\end{définition}\end{entrée}

\begin{entrée}
\vedette{\hypertarget{Ⓔɯ-locu}{\papi{ ɯ-locu}}}\markboth{ɯ-locu}{}\classe{np}
\begin{définition}\fra en amont\end{définition}
\begin{définition}\cmn 在上游\end{définition}\end{entrée}

\begin{entrée}
\vedette{\hypertarget{Ⓔɯ-ltɕi}{\papi{ ɯ-ltɕi}}}\markboth{ɯ-ltɕi}{}\classe{np}
\begin{définition}\fra franges (sac)\end{définition}
\begin{définition}\cmn 【须须】(挎包的)
\end{définition}\end{entrée}

\begin{entrée}
\vedette{\hypertarget{Ⓔɯ-lɯtshɤt}{\papi{ ɯ-lɯtshɤt}}}\markboth{ɯ-lɯtshɤt}{}\classe{np}
\begin{définition}\fra du même âge\end{définition}
\begin{définition}\cmn 同岁,跟自己年龄差不多\end{définition}
\begin{exemple}\jya ɯ-lɯtshɤt ra nɯ-rca ɲɯ-rga\cmn 他喜欢跟自己同龄的小孩子在一起\end{exemple}
\begin{relation-sémantique}\confer{
\hyperlink{Ⓔaɣɯlɯtshɤt}{\textit{ \papi{aɣɯlɯtshɤt}}}
}\end{relation-sémantique}\end{entrée}

\begin{entrée}
\vedette{\hypertarget{Ⓔɯ-lwa}{\papi{ ɯ-lwa}}}\markboth{ɯ-lwa}{}
\classe{np}
\begin{définition}\fra crinière\end{définition}
\begin{définition}\cmn 马鬃\end{définition}\end{entrée}

\begin{entrée}
\vedette{\hypertarget{Ⓔɯ-maŋ}{\papi{ ɯ-maŋ}}}\markboth{ɯ-maŋ}{}\classe{np}
\begin{définition}\fra en groupes\end{définition}
\begin{définition}\cmn 很多群\end{définition}
\begin{exemple}\jya tɯrme ɯ-maŋ ʑo to-k-ɤwɯwum-nɯ-ci\cmn 很多人集中在那里了\end{exemple}\end{entrée}

\begin{entrée}
\vedette{\hypertarget{Ⓔɯ-mat}{\papi{ ɯ-mat}}}\markboth{ɯ-mat}{}\classe{np}
\begin{définition}\fra fruit\end{définition}
\begin{définition}\cmn 果子\end{définition}\end{entrée}

\begin{entrée}
\vedette{\hypertarget{Ⓔɯ-mbrɤzɯ}{\papi{ ɯ-mbrɤzɯ}}}\markboth{ɯ-mbrɤzɯ}{}\classe{np}
\begin{définition}\fra résultat\end{définition}
\begin{définition}\cmn 结果
\begin{déclaration} \étymologie{\papi{ⁿbras}}\end{déclaration}\end{définition}
\begin{exemple}\jya a-mbrɤzɯ pɯ-smɯn\cmn 我得到了(好)结果\end{exemple}
\begin{exemple}\jya tɯʑo kɯ ɯ-mbrɤzɯ ɣɯ-nɯ-ndza kɯ-ra ŋu\cmn 要自食其果的\end{exemple}\end{entrée}

\begin{entrée}
\vedette{\hypertarget{Ⓔɯ-mbɯrme}{\papi{ ɯ-mbɯrme}}}\markboth{ɯ-mbɯrme}{}\classe{np}
\begin{définition}\fra poils pubiens (homme)\end{définition}
\begin{définition}\cmn 阴毛\end{définition}
\begin{relation-sémantique}\confer{
\hyperlink{Ⓔtɯ-mbɯ}{\textit{ \papi{tɯ-mbɯ}}}
}\end{relation-sémantique}\end{entrée}

\begin{entrée}
\vedette{\hypertarget{Ⓔɯ-mbɯrqhu}{\papi{ ɯ-mbɯrqhu}}}\markboth{ɯ-mbɯrqhu}{}\classe{np}
\begin{définition}\fra prépuce\end{définition}
\begin{définition}\cmn 包皮\end{définition}
\begin{relation-sémantique}\confer{
\hyperlink{Ⓔtɯ-mbɯ}{\textit{ \papi{tɯ-mbɯ}}}
}\end{relation-sémantique}\end{entrée}

\begin{entrée}
\vedette{\hypertarget{Ⓔɯ-mdoʁ}{\papi{ ɯ-mdoʁ}}}\markboth{ɯ-mdoʁ}{}
\classe{np}
\begin{définition}\fra couleur\end{définition}
\begin{définition}\cmn 颜色
\begin{déclaration} \étymologie{\papi{mdog}}\end{déclaration}\end{définition}
\begin{exemple}\jya qromke ɯ-mdoʁ asɯ-ndo\cmn 带有紫色\end{exemple}\end{entrée}

\begin{entrée}
\vedette{\hypertarget{Ⓔɯ-mdʑɯɣ}{\papi{ ɯ-mdʑɯɣ}}}\markboth{ɯ-mdʑɯɣ}{}\classe{np}
\begin{définition}\fra à la fin\end{définition}
\begin{définition}\cmn 最后\end{définition}
\begin{exemple}\jya ɯ-mdʑɯɣ tɕe saχsɤl ɕti nɤ\cmn 最后就会清楚的\end{exemple}\end{entrée}

\begin{entrée}
\vedette{\hypertarget{Ⓔɯ-mujmaj}{\papi{ ɯ-mujmaj}}}\markboth{ɯ-mujmaj}{}\classe{np}
\begin{définition}\fra branche d'arbre\end{définition}
\begin{définition}\cmn 树的枝桠\end{définition}
\begin{relation-sémantique}\confer{
\hyperlink{Ⓔnɤmujmaj}{\textit{ \papi{nɤmujmaj}}}
}\end{relation-sémantique}\end{entrée}

\begin{entrée}
\vedette{\hypertarget{Ⓔɯ-mnɯ}{\papi{ ɯ-mnɯ}}}\markboth{ɯ-mnɯ}{}\classe{np}
\begin{définition}\fra pousse, petite branche, pousse de bambou\end{définition}
\begin{définition}\cmn 树木新发出来的枝条;笋\end{définition}\end{entrée}

\begin{entrée}
\vedette{\hypertarget{Ⓔɯ-mɲaʁsta}{\papi{ ɯ-mɲaʁsta}}}\markboth{ɯ-mɲaʁsta}{}\classe{np}
\begin{définition}\fra magnanimité\end{définition}
\begin{définition}\cmn 心胸\end{définition}
\begin{exemple}\jya ɯ-mɲaʁsta kɯ-ŋgɤr ci ŋu, kɯ-xtɕɯ-xtɕi kɯnɤ ʑo naʁdɯɣ ɕti\cmn 他是个心胸狭窄的人,小事情也计较\end{exemple}
\begin{exemple}\jya ɯ-mɲaʁsta jom\cmn 他心胸宽阔\end{exemple}\end{entrée}

\begin{entrée}
\vedette{\hypertarget{Ⓔɯ-mɲoz}{\papi{ ɯ-mɲoz}}}\markboth{ɯ-mɲoz}{}
\begin{relation-sémantique}\confer{
\hyperlink{ⒺmɲoⒽ1}{\textit{ \papi{mɲo1}}}
}\end{relation-sémantique}\end{entrée}

\begin{entrée}
\vedette{\hypertarget{Ⓔɯ-mŋu}{\papi{ ɯ-mŋu}}}\markboth{ɯ-mŋu}{}
\classe{np}\acception{1}
\begin{définition}\fra ouverture (sac)\end{définition}
\begin{définition}\cmn (背篼、口袋)的口\end{définition}
\begin{exemple}\jya tɤ-fkɯm ɯ-mŋu\cmn 口袋的口\end{exemple}
\begin{relation-sémantique}\confer{
\hyperlink{Ⓔnɯmŋu}{\textit{ \papi{nɯmŋu}}}
}\end{relation-sémantique}\acception{2}
\begin{définition}\fra partie du toit sous laquelle il n'y a pas de balcon\end{définition}
\begin{définition}\cmn 房顶靠墙的一边(没有走檐)\end{définition}
\begin{exemple}\jya khɤxtu ɯ-mŋu\end{exemple}\acception{3}
\begin{définition}\fra côté du champs près de la montagne\end{définition}
\begin{définition}\cmn (田地)靠山的那一边\end{définition}
\begin{exemple}\jya tɯji ɯ-mŋu\cmn 田地靠山的那一角\end{exemple}
\begin{relation-sémantique}\confer{
\hyperlink{Ⓔkhɯmŋu}{\textit{ \papi{khɯmŋu}}}
}\end{relation-sémantique}\begin{sous-entrée}
\vedette{\hypertarget{}{\papi{ ɯ-mŋuɕɯmŋu}}}\markboth{ɯ-mŋuɕɯmŋu}{}\classe{np}
\begin{définition}\fra le plus au bord\end{définition}
\begin{définition}\cmn 最边缘\end{définition}
\begin{exemple}\jya tʂha tɤ́-wɣ-rku tɕe khɯtsa ɯ-mŋuɕɯmŋu sthɯci tu-zɣɯt mɤ-ra ma kɤ-ndo tɕe sɤɕke\cmn 倒茶的时候,不要倒得太满,不然端的时候会烫手\end{exemple}
\end{sous-entrée}\end{entrée}

\begin{entrée}
\vedette{\hypertarget{Ⓔɯ-mphru}{\papi{ ɯ-mphru}}}\markboth{ɯ-mphru}{}\classe{np}
\begin{définition}\fra à la suite de\end{définition}
\begin{définition}\cmn ……以后
\begin{déclaration} \étymologie{\papi{ⁿpʰro (deɦi.ⁿpʰror)}}\end{déclaration}\end{définition}
\begin{exemple}\jya ɯ-mphru pjɯ-ʑe-a tɕe pjɯ-ndɯn-a ŋu nɤ\cmn 我从头开始读\end{exemple}
\begin{exemple}\jya saχsɯ ɯ-qhu tɕe ɯ-mphru nɯ-mɟa-tɕi\cmn 我们午后再继续\end{exemple}\end{entrée}

\begin{entrée}
\vedette{\hypertarget{Ⓔɯ-mphɯsku}{\papi{ ɯ-mphɯsku}}}\markboth{ɯ-mphɯsku}{}\classe{np}
\begin{définition}\fra croupe (bovins, ovins)\end{définition}
\begin{définition}\cmn 臀部(牛、羊)\end{définition}
\begin{relation-sémantique}\confer{
\hyperlink{Ⓔtɯ-mphɯz}{\textit{ \papi{tɯ-mphɯz}}}
}\end{relation-sémantique}
\begin{relation-sémantique}\confer{
\hyperlink{Ⓔtɯ-ku}{\textit{ \papi{tɯ-ku}}}
}\end{relation-sémantique}\end{entrée}

\begin{entrée}
\vedette{\hypertarget{Ⓔɯ-mthoŋ,nɯɬoʁ}{\papi{ ɯ-mthoŋ,nɯɬoʁ}}}\markboth{ɯ-mthoŋ,nɯɬoʁ}{}\classe{np}
\paradigme{\textit{dir :} \jya nɯ-}
\begin{définition}\fra se trahir, dévoiler ses défauts\end{définition}
\begin{définition}\cmn 露馅\end{définition}
\begin{exemple}\jya a-mthoŋ nɯ-ɬoʁ ɲɯ-ŋu\cmn 我要露馅了\end{exemple}
\begin{exemple}\jya ɯ-mthoŋ pjɤ-nɯ-ɬoʁ\cmn 他露馅了\end{exemple}
\begin{relation-sémantique}\ComponentA{\classe{np}
 \papi{ɯ-mthoŋ}
}\end{relation-sémantique}
\begin{relation-sémantique}\ComponentB{\classe{vi}
\hyperlink{Ⓔnɯɬoʁ}{\textit{ \papi{nɯɬoʁ}}}
}\end{relation-sémantique}\end{entrée}

\begin{entrée}
\vedette{\hypertarget{Ⓔɯ-mtsioʁ}{\papi{ ɯ-mtsioʁ}}}\markboth{ɯ-mtsioʁ}{}\classe{np}
\begin{définition}\fra bec\end{définition}
\begin{définition}\cmn 鸟嘴\end{définition}\end{entrée}

\begin{entrée}
\vedette{\hypertarget{Ⓔɯ-ndaŋ,lɤt}{\papi{ ɯ-ndaŋ,lɤt}}}\markboth{ɯ-ndaŋ,lɤt}{}
\classe{np}
\paradigme{\textit{dir :} \jya pɯ-}
\begin{définition}\fra penser à\end{définition}
\begin{définition}\cmn (为某人)着想
\begin{déclaration} \étymologie{\papi{ⁿdaŋ}}\end{déclaration}\end{définition}
\begin{exemple}\jya nɯ ɯ-ndaŋ pjɯ-lat-a ŋu\cmn 我为他着想\end{exemple}
\begin{exemple}\jya tɯ-zda ra nɯ-ndaŋ pjɯ́-wɣ-lɤt ra ma nɯ-ʁdɯxpa ɣɯ-βzu mɤ-βdi\cmn 他考虑到别人,不妨碍别人\end{exemple}
\begin{relation-sémantique}\ComponentA{\classe{np}
 \papi{ɯ-ndaŋ}
}\end{relation-sémantique}
\begin{relation-sémantique}\ComponentB{\classe{vt}
\hyperlink{ⒺlɤtⒽ1}{\textit{ \papi{lɤt}}}
}\end{relation-sémantique}
\begin{relation-sémantique}\confer{
\hyperlink{Ⓔrɤndaŋ}{\textit{ \papi{rɤndaŋ}}}
}\end{relation-sémantique}
\begin{relation-sémantique}\confer{
\hyperlink{ⒺlɤtⒽ1}{\textit{ \papi{lɤt1}}}
}\end{relation-sémantique}\end{entrée}

\begin{entrée}
\vedette{\hypertarget{Ⓔɯ-ndɤcu}{\papi{ ɯ-ndɤcu}}}\markboth{ɯ-ndɤcu}{}\classe{np}
\begin{définition}\fra à l'ouest\end{définition}
\begin{définition}\cmn 在西方\end{définition}\end{entrée}

\begin{entrée}
\vedette{\hypertarget{Ⓔɯ-ndo}{\papi{ ɯ-ndo}}}\markboth{ɯ-ndo}{}
\classe{np}\acception{1}
\begin{définition}\fra bord, côté du champs en direction du fleuve\end{définition}
\begin{définition}\cmn 边缘,田地靠河流的那一边\end{définition}
\begin{exemple}\jya tɯ-ŋga ɯ-ndo\cmn 衣服的下边\end{exemple}\acception{2}
\begin{définition}\fra fin\end{définition}
\begin{définition}\cmn 结尾\end{définition}
\begin{exemple}\jya tu-kɯ-stu ra ma ɯ-ndo tɕe mɤ-pe\cmn 要注意一点,不然到最后就没有好结果\end{exemple}
\begin{exemple}\jya ɯ-ndo tɕe mɤʑɯ kɤ-sɤpe ra\cmn 事情到了最后要做得更好\end{exemple}\begin{sous-entrée}
\vedette{\hypertarget{}{\papi{ ɯ-ndoɕɯndo}}}\markboth{ɯ-ndoɕɯndo}{}
\begin{définition}\fra le côté le plus lointain\end{définition}
\begin{définition}\cmn 最边缘\end{définition}
\begin{exemple}\jya nɤmkha ɯ-ndoɕɯndo ʑo zɯ ʑŋgri kɯ-tʂɯ-tʂot ʑo tɯ-rdoʁ ɣɤʑu\cmn 在天的最边缘有一颗很明亮的星星\end{exemple}
\end{sous-entrée}\end{entrée}

\begin{entrée}
\vedette{\hypertarget{Ⓔɯ-ndzɯɣlɯz}{\papi{ ɯ-ndzɯɣlɯz}}}\markboth{ɯ-ndzɯɣlɯz}{}\classe{np}
\begin{définition}\fra comportement\end{définition}
\begin{définition}\cmn 举止\end{définition}
\begin{exemple}\jya ɯʑo ɯ-ndzɯɣlɯz mɤ-kɯ-βdi ɕti ma tɯrme pe ɯ-sɯm sna\cmn 这个人虽然举止不好,但是心底很善良\end{exemple}\end{entrée}

\begin{entrée}
\vedette{\hypertarget{Ⓔɯ-ndzɯɣ,maʁ}{\papi{ ɯ-ndzɯɣ,maʁ}}}\markboth{ɯ-ndzɯɣ,maʁ}{}
\paradigme{\textit{dir :} \jya pɯ-}
\begin{définition}\fra être terrible\end{définition}
\begin{définition}\cmn 厉害\end{définition}
\begin{exemple}\jya a-mu kɯ ``a-ku ɯ-tɯ-mŋɤm kɯ ɯ-ndzɯɣ ʑo ɲɯ-maʁ"\cmn 妈妈说“我头疼得很厉害”\end{exemple}
\begin{exemple}\jya jisŋi kɤntɕhaʁ tɯrme ɯ-tɯ-dɤn kɯ ɯ-ndzɯɣ ɲɯ-maʁ\cmn 今天街上人特别多\end{exemple}
\begin{exemple}\jya khɤcɤl ɯ-ndzɯɣ to-ɣɤmaʁ-ndʑi ɕti\cmn 他们谈了很久\end{exemple}
\begin{relation-sémantique}\ComponentA{\classe{np}
 \papi{ɯ-ndzɯɣ}
}\end{relation-sémantique}
\begin{relation-sémantique}\ComponentB{\classe{vs}
\hyperlink{ⒺmaʁⒽ1}{\textit{ \papi{maʁ}}}
}\end{relation-sémantique}\end{entrée}

\begin{entrée}
\vedette{\hypertarget{Ⓔɯ-ndzɯndzoʁ}{\papi{ ɯ-ndzɯndzoʁ}}}\markboth{ɯ-ndzɯndzoʁ}{}\classe{np}
\begin{définition}\fra suivant sans relâche\end{définition}
\begin{définition}\cmn 紧紧地跟着\end{définition}
\begin{relation-sémantique}\confer{
\hyperlink{Ⓔndzoʁ}{\textit{ \papi{ndzoʁ}}}
}\end{relation-sémantique}
\end{entrée}

\begin{entrée}
\vedette{\hypertarget{Ⓔɯntɕe}{\papi{ ɯntɕe}}}\markboth{ɯntɕe}{}\classe{cnj}
\begin{définition}\fra ensuite\end{définition}
\begin{définition}\cmn 以后\end{définition}
\end{entrée}

\begin{entrée}
\vedette{\hypertarget{Ⓔɯ-ntɕhantɕhɯr}{\papi{ ɯ-ntɕhantɕhɯr}}}\markboth{ɯ-ntɕhantɕhɯr}{}\classe{np}
\begin{définition}\ 
\begin{déclaration}\grammar{n.rdpl}\end{déclaration}\end{définition}
\begin{définition}\fra morceaux, débris\end{définition}
\begin{définition}\cmn 碎片\end{définition}
\begin{relation-sémantique}\confer{
\hyperlink{Ⓔtɯ-ntɕhɯr}{\textit{ \papi{tɯ-ntɕhɯr}}}
}\end{relation-sémantique}
\end{entrée}

\begin{entrée}
\vedette{\hypertarget{Ⓔɯnɯnɯ}{\papi{ ɯnɯnɯ}}}\markboth{ɯnɯnɯ}{}\classe{pro}
\begin{définition}\fra celui-là\end{définition}
\begin{définition}\cmn 那个\end{définition}
\begin{relation-sémantique}\confer{
\hyperlink{ⒺnɯnɯⒽ2}{\textit{ \papi{nɯnɯ2}}}
}\end{relation-sémantique}
\end{entrée}

\begin{entrée}
\vedette{\hypertarget{Ⓔɯnɯnɯra}{\papi{ ɯnɯnɯra}}}\markboth{ɯnɯnɯra}{}\classe{pro}
\begin{définition}\fra ceux-là\end{définition}
\begin{définition}\cmn 那些\end{définition}
\begin{relation-sémantique}\confer{
\hyperlink{Ⓔnɯra}{\textit{ \papi{nɯra}}}
}\end{relation-sémantique}
\end{entrée}

\begin{entrée}
\vedette{\hypertarget{Ⓔɯŋ}{\papi{ ɯŋ}}}\markboth{ɯŋ}{}\classe{intj}
\begin{définition}\fra exprime l'hésitation\end{définition}
\begin{définition}\cmn 表示犹豫,不高兴\end{définition}
\end{entrée}

\begin{entrée}
\vedette{\hypertarget{Ⓔɯŋaj,βzu}{\papi{ ɯŋaj,βzu}}}\markboth{ɯŋaj,βzu}{}
\paradigme{\textit{dir :} \jya nɯ-}
\begin{définition}\fra être autosatisfait\end{définition}
\begin{définition}\cmn 得意,原谅自己,自满,无视自己的过错\end{définition}
\begin{exemple}\jya ɯŋaj ma-nɯ-tɯ-nɯ-βze kɯ tɤ-stu tɤ-mbat\cmn 你不要这么得意,努力一点\end{exemple}
\begin{relation-sémantique}\ComponentA{\classe{np}
 \papi{ɯ-ŋaj}
}\end{relation-sémantique}
\begin{relation-sémantique}\ComponentB{\classe{vt}
\hyperlink{ⒺβzuⒽ1}{\textit{ \papi{βzu}}}
}\end{relation-sémantique}
\begin{relation-sémantique}\confer{
\hyperlink{ⒺβzuⒽ1}{\textit{ \papi{βzu1}}}
}\end{relation-sémantique}\end{entrée}

\begin{entrée}
\vedette{\hypertarget{Ⓔɯŋgu}{\papi{ ɯŋgu}}}\markboth{ɯŋgu}{}
\classe{adv}
\begin{définition}\fra autrefois, d'abord\end{définition}
\begin{définition}\cmn 本来,首先
\begin{déclaration} \étymologie{\papi{ⁿgo}}\end{déclaration}\end{définition}
\begin{exemple}\jya ɯʑo ɯŋgu jɤznɤ taʁndo kɯ-tso ci pjɤ-ŋu ri, ɯ-ndo tɕe taʁndo mɯ-ɲɤ-tso\cmn 他本来很听话的,后来就不听话了\end{exemple}
\begin{relation-sémantique}\antonyme{
\hyperlink{Ⓔɯ-ndo}{\textit{ \papi{ɯ-ndo}}}
}\end{relation-sémantique}
\begin{relation-sémantique}\confer{
\hyperlink{Ⓔnɯŋgu}{\textit{ \papi{nɯŋgu}}}
}\end{relation-sémantique}\end{entrée}

\begin{entrée}
\vedette{\hypertarget{Ⓔɯ-ŋgɤrmoz}{\papi{ ɯ-ŋgɤrmoz}}}\markboth{ɯ-ŋgɤrmoz}{}
\classe{np}
\begin{définition}\fra dérangement\end{définition}
\begin{définition}\cmn 打搅,麻烦人\end{définition}
\begin{exemple}\jya aʑo nɤʑo mɤ-ɣi-a ma nɤ-ŋgɤrmoz sɤβze-a\cmn 我不来你家,我会打搅你的\end{exemple}\end{entrée}

\begin{entrée}
\vedette{\hypertarget{Ⓔɯ-ŋgumdʑɯɣ}{\papi{ ɯ-ŋgumdʑɯɣ}}}\markboth{ɯ-ŋgumdʑɯɣ}{}
\classe{np}
\begin{définition}\fra chef\end{définition}
\begin{définition}\cmn 领导
\begin{déclaration} \étymologie{\papi{ⁿgo.mdʑug}}\end{déclaration}\end{définition}
\begin{exemple}\jya aʑo nɯ-ŋgumdʑɯɣ ŋu-a\cmn 我是你们的领导\end{exemple}
\begin{relation-sémantique}\confer{
 \papi{nɯŋgɯdʑɯɣ}
}\end{relation-sémantique}\begin{sous-entrée}
\vedette{\hypertarget{}{\papi{ ŋgumdʑɯɣpa}}}\markboth{ŋgumdʑɯɣpa}{}
\begin{définition}\ 
\begin{déclaration}\grammar{n}\end{déclaration}\end{définition}
\begin{définition}\fra chef\end{définition}
\begin{définition}\cmn 领导人\end{définition}
\begin{exemple}\jya ŋgumdʑɯɣpa to-ndo\cmn 他当了领导\end{exemple}
\end{sous-entrée}\end{entrée}

\begin{entrée}
\vedette{\hypertarget{Ⓔɯ-ŋgu,thon}{\papi{ ɯ-ŋgu,thon}}}\markboth{ɯ-ŋgu,thon}{}\paradigme{\textit{dir :} \jya thɯ-}
\begin{définition}\fra avoir une bonne situation, être aisé\end{définition}
\begin{définition}\cmn 家境好;富裕
\begin{déclaration} \étymologie{\papi{mgo.tʰon}}\end{déclaration}
\begin{déclaration}\use{\stylefv{ɯ-ŋgu} \stylefv{thon}不能附加人称标记,人称标记用\stylefv{ɯ-ŋgu}的领属前缀来反映。}\end{déclaration}\end{définition}
\begin{exemple}\jya ɯ-ŋgu mɯ́j-thon\cmn 他什么都没有\end{exemple}
\begin{exemple}\jya jiɕqha nɯ ɯ-ŋgu thon, mɤɕi\cmn 他很有钱,很富有\end{exemple}
\begin{exemple}\jya jiʑo ɕaŋtaʁ tɯ-ŋgu mɤ-kɯ-thon me\cmn 没有比我们穷的人了\end{exemple}
\begin{relation-sémantique}\ComponentA{\classe{np}
 \papi{ɯ-ŋgu}
}\end{relation-sémantique}
\begin{relation-sémantique}\ComponentB{\classe{vs}
 \papi{thon}
}\end{relation-sémantique}\end{entrée}

\begin{entrée}
\vedette{\hypertarget{Ⓔɯ-ŋgɯ}{\papi{ ɯ-ŋgɯ}}}\markboth{ɯ-ŋgɯ}{}\classe{n}
\begin{définition}\fra dedans\end{définition}
\begin{définition}\cmn 里面\end{définition}
\begin{exemple}\jya a-ŋga ɯ-ŋgɯɕɯŋgɯ nɯ ɯ-poloʁ me\cmn 我最里层的衣服没有袖子\end{exemple}\end{entrée}

\begin{entrée}
\vedette{\hypertarget{Ⓔɯ-ŋgɯmɤpɕi}{\papi{ ɯ-ŋgɯmɤpɕi}}}\markboth{ɯ-ŋgɯmɤpɕi}{}\classe{adv}
\begin{définition}\fra à l'intérieur et à l'extérieur\end{définition}
\begin{définition}\cmn 里里外外\end{définition}
\begin{relation-sémantique}\confer{
\hyperlink{Ⓔɯ-pɕi}{\textit{ \papi{ɯ-pɕi}}}
}\end{relation-sémantique}\end{entrée}

\begin{entrée}
\vedette{\hypertarget{Ⓔɯ-ŋgɯsni}{\papi{ ɯ-ŋgɯsni}}}\markboth{ɯ-ŋgɯsni}{}\classe{np}
\begin{définition}\fra au cœur même de\end{définition}
\begin{définition}\cmn 最里面
\end{définition}
\begin{relation-sémantique}\confer{
\hyperlink{Ⓔtɯ-sni}{\textit{ \papi{tɯ-sni}}}
}\end{relation-sémantique}\end{entrée}

\begin{entrée}
\vedette{\hypertarget{Ⓔɯ-ɴqra}{\papi{ ɯ-ɴqra}}}\markboth{ɯ-ɴqra}{}\classe{np}
\begin{définition}\fra délabré\end{définition}
\begin{définition}\cmn 破烂\end{définition}
\begin{relation-sémantique}\confer{
\hyperlink{Ⓔkhɤɴqra}{\textit{ \papi{khɤɴqra}}}
}\end{relation-sémantique}
\begin{relation-sémantique}\confer{
\hyperlink{Ⓔrɤɴqra}{\textit{ \papi{rɤɴqra}}}
}\end{relation-sémantique}\end{entrée}

\begin{entrée}
\vedette{\hypertarget{Ⓔɯ-pa,ɕe}{\papi{ ɯ-pa,ɕe}}}\markboth{ɯ-pa,ɕe}{}
\paradigme{\textit{dir :} \jya nɯ-}
\begin{définition}\fra être accaparé par\end{définition}
\begin{définition}\cmn 被……拿去自己用\end{définition}
\begin{exemple}\jya iʑora ji-kɤndzɤtshi ɯ-ro pɯ-dɤn ri, ɯʑo ɯ-pa ɲɤ-nɯ-ɕe\cmn 我们在一起玩的时候没有吃完的食物都被他占有了\end{exemple}
\begin{relation-sémantique}\ComponentA{\classe{np}
 \papi{ɯ-pa}
}\end{relation-sémantique}
\begin{relation-sémantique}\ComponentB{\classe{vi}
\hyperlink{Ⓔɕe}{\textit{ \papi{ɕe}}}
}\end{relation-sémantique}
\begin{sous-entrée}
\vedette{\hypertarget{}{\papi{ ɯ-pa,sɯxɕe}}}\markboth{ɯ-pa,sɯxɕe}{}
\paradigme{\textit{dir :} \jya nɯ-}
\begin{définition}\fra s'accaparer des objets qui appartiennent à d'autres\end{définition}
\begin{définition}\cmn 归为私用,拿去自己用;占有\end{définition}
\begin{exemple}\jya nɤj nɤ-pa ma-nɯ-tɯ-nɯ-sɯxɕe\cmn 你不要拿去自己用\end{exemple}
\begin{relation-sémantique}\ComponentA{\classe{np}
 \papi{ɯ-pa}
}\end{relation-sémantique}
\begin{relation-sémantique}\ComponentB{\classe{vt}
\hyperlink{Ⓔsɯxɕe}{\textit{ \papi{sɯxɕe}}}
}\end{relation-sémantique}
\end{sous-entrée}\end{entrée}

\begin{entrée}
\vedette{\hypertarget{Ⓔɯ-palɤjlɯz}{\papi{ ɯ-palɤjlɯz}}}\markboth{ɯ-palɤjlɯz}{}\classe{n}
\begin{définition}\fra méthode, façon\end{définition}
\begin{définition}\cmn 办法;措施\end{définition}
\end{entrée}

\begin{entrée}
\vedette{\hypertarget{Ⓔɯ-pɤl}{\papi{ ɯ-pɤl}}}\markboth{ɯ-pɤl}{}\classe{np}\acception{1}
\begin{définition}\fra paume\end{définition}
\begin{définition}\cmn (手、脚)掌\end{définition}
\begin{exemple}\jya tɯ-jaʁ ɯ-pɤl\cmn 手掌\end{exemple}
\begin{exemple}\jya tɯ-mi ɯ-pɤl\cmn 脚掌\end{exemple}\acception{2}
\begin{définition}\fra partie de la louche qui sert à contenir le liquide\end{définition}
\begin{définition}\cmn 勺子容水的部分\end{définition}
\end{entrée}

\begin{entrée}
\vedette{\hypertarget{Ⓔɯ-pɤrthɤβ}{\papi{ ɯ-pɤrthɤβ}}}\markboth{ɯ-pɤrthɤβ}{}\classe{np}
\begin{définition}\fra entre\end{définition}
\begin{définition}\cmn 两个之间
\begin{déclaration} \étymologie{\papi{bar}}\end{déclaration}\end{définition}
\begin{exemple}\jya ndʑi-pɤrthɤβ\cmn 在他们俩之间\end{exemple}
\begin{relation-sémantique}\confer{
\hyperlink{Ⓔɯ-thɤβ}{\textit{ \papi{ɯ-thɤβ}}}
}\end{relation-sémantique}\end{entrée}

\begin{entrée}
\vedette{\hypertarget{Ⓔɯ-pɕi}{\papi{ ɯ-pɕi}}}\markboth{ɯ-pɕi}{}
\classe{n}
\begin{définition}\fra dehors\end{définition}
\begin{définition}\cmn 外面
\begin{déclaration} \étymologie{\papi{pʰʲi}}\end{déclaration}\end{définition}
\begin{exemple}\jya ɯ-pɕi qale ɣɤʑu wo\cmn 外面有风\end{exemple}
\begin{relation-sémantique}\confer{
\hyperlink{Ⓔmɤpɕi}{\textit{ \papi{mɤpɕi}}}
}\end{relation-sémantique}\end{entrée}

\begin{entrée}
\vedette{\hypertarget{Ⓔɯ-phe}{\papi{ ɯ-phe}}}\markboth{ɯ-phe}{}\classe{postp}
\begin{définition}\fra datif\end{définition}
\begin{définition}\cmn 与格\end{définition}
\begin{relation-sémantique}\synonyme{
\hyperlink{Ⓔɯ-ɕki}{\textit{ \papi{ɯ-ɕki}}}
}\end{relation-sémantique}\end{entrée}

\begin{entrée}
\vedette{\hypertarget{Ⓔɯ-phɯ}{\papi{ ɯ-phɯ}}}\markboth{ɯ-phɯ}{}\classe{np}
\begin{définition}\fra prix\end{définition}
\begin{définition}\cmn 价钱\end{définition}
\begin{relation-sémantique}\confer{
\hyperlink{Ⓔrɤphɯ}{\textit{ \papi{rɤphɯ}}}
}\end{relation-sémantique}
\begin{relation-sémantique}\confer{
\hyperlink{Ⓔnɯphɯ}{\textit{ \papi{nɯphɯ}}}
}\end{relation-sémantique}
\end{entrée}

\begin{entrée}
\vedette{\hypertarget{Ⓔɯ-phɯɣ}{\papi{ ɯ-phɯɣ}}}\markboth{ɯ-phɯɣ}{}\classe{np}
\begin{définition}\fra source (fleuve)\end{définition}
\begin{définition}\cmn 水源
\begin{déclaration} \étymologie{\papi{pʰugs}}\end{déclaration}\end{définition}
\begin{relation-sémantique}\confer{
\hyperlink{Ⓔtɕhɯphɯɣ}{\textit{ \papi{tɕhɯphɯɣ}}}
}\end{relation-sémantique}
\end{entrée}

\begin{entrée}
\vedette{\hypertarget{Ⓔɯ-phɯl}{\papi{ ɯ-phɯl}}}\markboth{ɯ-phɯl}{}\classe{np}
\begin{définition}\fra serti, incrusté de\end{définition}
\begin{définition}\cmn 镶着\end{définition}
\begin{exemple}\jya mbrɯtɕɯ ɯ-phɯl rɯnbotɕhi kɤ-rku\cmn 镶着宝石的刀\end{exemple}\end{entrée}

\begin{entrée}
\vedette{\hypertarget{Ⓔɯ-phɯŋgɯ}{\papi{ ɯ-phɯŋgɯ}}}\markboth{ɯ-phɯŋgɯ}{}\classe{np}
\begin{définition}\fra giron\end{définition}
\begin{définition}\cmn 怀里\end{définition}\end{entrée}

\begin{entrée}
\vedette{\hypertarget{Ⓔɯ-phɯphi}{\papi{ ɯ-phɯphi}}}\markboth{ɯ-phɯphi}{}\classe{np}
\begin{définition}\fra vagin (enfant)\end{définition}
\begin{définition}\cmn (小孩子的)阴道\end{définition}\end{entrée}

\begin{entrée}
\vedette{\hypertarget{Ⓔɯ-phɯphɯ}{\papi{ ɯ-phɯphɯ}}}\markboth{ɯ-phɯphɯ}{}\classe{np}
\begin{définition}\fra ce que l'on mendie\end{définition}
\begin{définition}\cmn 乞讨的(东西、钱)\end{définition}
\begin{exemple}\jya kɤ-nɤjɤm tɕe nɤ-phɯphɯ ju-ɣɯt-a\cmn 你在这里等着,给你送东西来(对乞丐说的话)\end{exemple}
\begin{relation-sémantique}\confer{
\hyperlink{Ⓔnɤphɯphɯ}{\textit{ \papi{nɤphɯphɯ}}}
}\end{relation-sémantique}
\end{entrée}

\begin{entrée}
\vedette{\hypertarget{Ⓔɯ-punaŋtɕa}{\papi{ ɯ-punaŋtɕa}}}\markboth{ɯ-punaŋtɕa}{}\classe{np}
\begin{définition}\fra organes internes\end{définition}
\begin{définition}\cmn 内脏\end{définition}\end{entrée}

\begin{entrée}
\vedette{\hypertarget{Ⓔɯ-qaɕɯqa}{\papi{ ɯ-qaɕɯqa}}}\markboth{ɯ-qaɕɯqa}{}\classe{np}
\begin{définition}\fra le fond\end{définition}
\begin{définition}\cmn 最深处\end{définition}
\begin{exemple}\jya rɟɤmtshu ɯ-qaɕɯqa nɯtɕu zɯ\cmn 在海的最深处\end{exemple}\end{entrée}

\begin{entrée}
\vedette{\hypertarget{Ⓔɯ-qhu}{\papi{ ɯ-qhu}}}\markboth{ɯ-qhu}{}
\classe{np}
\begin{définition}\fra arrière\end{définition}
\begin{définition}\cmn 后面\end{définition}
\begin{exemple}\jya nɤʑo nɯ-tɯ-ɤnɯri ɯ-qhu aʑo ki pɯ-rat-a\cmn 你回去了以后我就写了这些\end{exemple}
\begin{exemple}\jya nɯ ɯ-qhu nɯ tɕe kɯmaʁ kɯrɯχpi ci rat-a ŋu\cmn 下一次,我再写一个藏语故事\end{exemple}
\begin{relation-sémantique}\confer{
\hyperlink{Ⓔmaqhu}{\textit{ \papi{maqhu}}}
}\end{relation-sémantique}
\begin{relation-sémantique}\confer{
\hyperlink{Ⓔɯ-qhɤchu}{\textit{ \papi{ɯ-qhɤchu}}}
}\end{relation-sémantique}
\begin{relation-sémantique}\confer{
\hyperlink{Ⓔqhaqhu}{\textit{ \papi{qhaqhu}}}
}\end{relation-sémantique}
\begin{relation-sémantique}\confer{
\hyperlink{Ⓔtʂaqhu}{\textit{ \papi{tʂaqhu}}}
}\end{relation-sémantique}
\begin{relation-sémantique}\confer{
\hyperlink{Ⓔqharu}{\textit{ \papi{qharu}}}
}\end{relation-sémantique}
\begin{relation-sémantique}\confer{
\hyperlink{Ⓔtɕhɯqhu}{\textit{ \papi{tɕhɯqhu}}}
}\end{relation-sémantique}\begin{sous-entrée}
\vedette{\hypertarget{}{\papi{ ɯ-qhu,βzu}}}\markboth{ɯ-qhu,βzu}{}
\paradigme{\textit{dir :} \jya tɤ-}
\paradigme{\textit{dir :} \jya thɯ-}
\begin{définition}\fra défendre, soutenir, prendre le parti de\end{définition}
\begin{définition}\cmn 维护;为……做主\end{définition}
\begin{exemple}\jya tɯtʂaŋ kɤ-βzu ra ma, tɯ-pɕoʁ ɯ-qhu kɤ-βzu mɤ-khɯ\cmn 要公平,不要维护一方\end{exemple}
\begin{exemple}\jya jiʑora ji-qhu thɯ-βze ra ma mɤ-jɤɣ nɤ!\cmn 一定要为我们做主!\end{exemple}
\begin{exemple}\jya aʑo a-qhu tɤ-βze ra (thɯ-βze ra) = aʑo pjɯ-kɯ-zɣɤŋgi-a ra\cmn 你要为我做主!\end{exemple}
\begin{relation-sémantique}\synonyme{
\hyperlink{Ⓔzɣɤŋgi}{\textit{ \papi{zɣɤŋgi}}}
}\end{relation-sémantique}
\begin{relation-sémantique}\ComponentA{\classe{np}
\hyperlink{Ⓔɯ-qhu}{\textit{ \papi{ɯ-qhu}}}
}\end{relation-sémantique}
\begin{relation-sémantique}\ComponentB{\classe{vt}
\hyperlink{ⒺβzuⒽ1}{\textit{ \papi{βzu}}}
}\end{relation-sémantique}
\end{sous-entrée}\end{entrée}

\begin{entrée}
\vedette{\hypertarget{Ⓔɯ-qhɤchu}{\papi{ ɯ-qhɤchu}}}\markboth{ɯ-qhɤchu}{}
\classe{adv}
\begin{définition}\fra arrière\end{définition}
\begin{définition}\cmn 背面,后面\end{définition}
\begin{exemple}\jya nɤki tɯrme ɣɯ ɯ-qhɤchu nɯtɕu laχtɕha ata\cmn 那个人的后面有个东西\end{exemple}
\begin{relation-sémantique}\confer{
\hyperlink{Ⓔɯ-qhu}{\textit{ \papi{ɯ-qhu}}}
}\end{relation-sémantique}\end{entrée}

\begin{entrée}
\vedette{\hypertarget{Ⓔɯ-qhoʁ}{\papi{ ɯ-qhoʁ}}}\markboth{ɯ-qhoʁ}{}\classe{np}
\begin{définition}\fra largeur des habits (tronc)\end{définition}
\begin{définition}\cmn 衣服的宽度(胸膛和肚子)\end{définition}
\begin{exemple}\jya a-ŋga ɯ-qhoʁ ɲɯ-ŋgɤr tɕe, kɤ-ŋga mɯ́j-khɯ\cmn 因为我那件衣服太小,穿不下\end{exemple}\end{entrée}

\begin{entrée}
\vedette{\hypertarget{Ⓔɯ-qiɯ}{\papi{ ɯ-qiɯ}}}\markboth{ɯ-qiɯ}{}\classe{np}
\begin{définition}\fra moitié\end{définition}
\begin{définition}\cmn 一半\end{définition}
\begin{exemple}\jya ɯ-qiɯ pɯ-mtsham-a\cmn 我听了一半\end{exemple}\end{entrée}

\begin{entrée}
\vedette{\hypertarget{Ⓔɯ-qoʁ}{\papi{ ɯ-qoʁ}}}\markboth{ɯ-qoʁ}{}
\classe{np}
\begin{définition}\fra un an (enfant)\end{définition}
\begin{définition}\cmn 周岁(孩子)\end{définition}
\begin{exemple}\jya ɯ-qoʁ jɤ-azɣɯt ɯ́-ŋu?\cmn (你儿子)满周岁了吗?\end{exemple}\end{entrée}

\begin{entrée}
\vedette{\hypertarget{Ⓔɯ-ru}{\papi{ ɯ-ru}}}\markboth{ɯ-ru}{}\classe{np}
\paradigme{\textit{comit :} \jya kɤ́rɯru}
\paradigme{\textit{comit :} \jya kɤɣɯrɯru}
\begin{définition}\fra tige\end{définition}
\begin{définition}\cmn 杆子\end{définition}
\begin{relation-sémantique}\confer{
\hyperlink{ⒺaɣɯrɯruⒽ2}{\textit{ \papi{aɣɯrɯru2}}}
}\end{relation-sémantique}\end{entrée}

\begin{entrée}
\vedette{\hypertarget{Ⓔɯ-raŋ}{\papi{ ɯ-raŋ}}}\markboth{ɯ-raŋ}{}\classe{np}
\begin{définition}\fra génération, au moment de\end{définition}
\begin{définition}\cmn 年代,正当那个时候
\begin{déclaration} \étymologie{\papi{riŋ}}\end{déclaration}\end{définition}
\end{entrée}

\begin{entrée}
\vedette{\hypertarget{Ⓔɯ-rɤɣ}{\papi{ ɯ-rɤɣ}}}\markboth{ɯ-rɤɣ}{}
\classe{np}
\begin{définition}\fra au moment prévu, au même moment\end{définition}
\begin{définition}\cmn 在预定的时间\end{définition}
\begin{exemple}\jya stonka ɯ-rɤɣ ja-zɣɯt tɕe tɤ-rɤku chɯ-mda ɕti\cmn 秋天到了,庄稼就会成熟\end{exemple}
\begin{relation-sémantique}\confer{
\hyperlink{Ⓔarɤrɤɣ}{\textit{ \papi{arɤrɤɣ}}}
}\end{relation-sémantique}\end{entrée}

\begin{entrée}
\vedette{\hypertarget{Ⓔɯ-rcharchɤβ}{\papi{ ɯ-rcharchɤβ}}}\markboth{ɯ-rcharchɤβ}{}\classe{np}
\begin{définition}\fra interstice\end{définition}
\begin{définition}\cmn 缝隙;之间\end{définition}
\begin{relation-sémantique}\confer{
\hyperlink{Ⓔɯ-rchɤβ}{\textit{ \papi{ɯ-rchɤβ}}}
}\end{relation-sémantique}\end{entrée}

\begin{entrée}
\vedette{\hypertarget{Ⓔɯ-rchɤβ}{\papi{ ɯ-rchɤβ}}}\markboth{ɯ-rchɤβ}{}\classe{np}\acception{1}
\begin{définition}\fra interstice\end{définition}
\begin{définition}\cmn 缝隙\end{définition}\acception{2}
\begin{définition}\fra milieu\end{définition}
\begin{définition}\cmn 之间\end{définition}
\begin{relation-sémantique}\confer{
\hyperlink{Ⓔɯ-rcharchɤβ}{\textit{ \papi{ɯ-rcharchɤβ}}}
}\end{relation-sémantique}\end{entrée}

\begin{entrée}
\vedette{\hypertarget{Ⓔɯ-rɕa,mŋɤm}{\papi{ ɯ-rɕa,mŋɤm}}}\markboth{ɯ-rɕa,mŋɤm}{}
\begin{définition}\fra chérir\end{définition}
\begin{définition}\cmn 疼爱\end{définition}
\begin{exemple}\jya a-ɣe a-rɕa mŋɤm\cmn 我疼爱我的孙子\cmn 无辜的老百姓\end{exemple}
\begin{relation-sémantique}\confer{
\hyperlink{Ⓔnɤrɕɤmŋɤm}{\textit{ \papi{nɤrɕɤmŋɤm}}}
}\end{relation-sémantique}
\begin{relation-sémantique}\ComponentA{\classe{np}
 \papi{ɯ-rɕa}
}\end{relation-sémantique}
\begin{relation-sémantique}\ComponentB{\classe{vs}
\hyperlink{Ⓔmŋɤm}{\textit{ \papi{mŋɤm}}}
}\end{relation-sémantique}\end{entrée}

\begin{entrée}
\vedette{\hypertarget{Ⓔɯ-rɕa,tsha}{\papi{ ɯ-rɕa,tsha}}}\markboth{ɯ-rɕa,tsha}{}\classe{vi}
\paradigme{\textit{dir :} \jya kɤ-}
\begin{définition}\fra être délicat et prévenant\end{définition}
\begin{définition}\cmn 体贴\end{définition}
\begin{exemple}\jya nɤʑo nɤ-rɕa wuma ɲɯ-tsha\cmn 你很体贴人的\end{exemple}
\begin{relation-sémantique}\ComponentA{\classe{np}
 \papi{ɯ-rɕa}
}\end{relation-sémantique}
\begin{relation-sémantique}\ComponentB{\classe{vs}
\hyperlink{Ⓔtsha}{\textit{ \papi{tsha}}}
}\end{relation-sémantique}\begin{sous-entrée}
\vedette{\hypertarget{}{\papi{ ɯ-rɕa,ɣɤtsha}}}\markboth{ɯ-rɕa,ɣɤtsha}{}
\begin{exemple}\jya nɤ-taʁ a-rɕa tu-ɣɤtshe-a ra\cmn 我要对你体贴一点\end{exemple}
\begin{exemple}\jya nɤ-mu nɤ-wa ndʑɪ-ɕki nɤ-rɕa tɤ-ɣɤtshe ra\cmn 你要孝顺你父母\end{exemple}
\begin{relation-sémantique}\ComponentA{\classe{np}
 \papi{ɯ-rɕa}
}\end{relation-sémantique}
\begin{relation-sémantique}\ComponentB{\classe{vt}
\hyperlink{Ⓔɣɤtsha}{\textit{ \papi{ɣɤtsha}}}
}\end{relation-sémantique}
\end{sous-entrée}\end{entrée}

\begin{entrée}
\vedette{\hypertarget{Ⓔɯ-rɕa,χtɤt}{\papi{ ɯ-rɕa,χtɤt}}}\markboth{ɯ-rɕa,χtɤt}{}\classe{vt}
\paradigme{\textit{dir :} \jya kɤ-}
\begin{définition}\fra se concentrer\end{définition}
\begin{définition}\cmn 专心,集中\end{définition}
\begin{exemple}\jya nɤ-rɕa kɤ-χtɤt ɲɯ-ra\cmn 你要专心一点\end{exemple}
\begin{exemple}\jya nɯ-rɕa kɤ-χtɤt mɤ-cha-nɯ tɕe kɯmɤlɤxso ɕti\cmn 他们不能专心学习,就白费了\end{exemple}
\begin{relation-sémantique}\ComponentA{\classe{np}
 \papi{ɯ-rɕa}
}\end{relation-sémantique}
\begin{relation-sémantique}\ComponentB{\classe{vt}
\hyperlink{Ⓔχtɤt}{\textit{ \papi{χtɤt}}}
}\end{relation-sémantique}\end{entrée}

\begin{entrée}
\vedette{\hypertarget{Ⓔɯ-rdoʁ}{\papi{ ɯ-rdoʁ}}}\markboth{ɯ-rdoʁ}{}\classe{np}
\begin{définition}\fra nourriture pour animaux\end{définition}
\begin{définition}\cmn 整体的牲畜的粮食\end{définition}
\begin{relation-sémantique}\confer{
\hyperlink{Ⓔtɯ-rdoʁ}{\textit{ \papi{tɯ-rdoʁ}}}
}\end{relation-sémantique}
\end{entrée}

\begin{entrée}
\vedette{\hypertarget{Ⓔɯ-rgu,sɯ}{\papi{ ɯ-rgu,sɯ}}}\markboth{ɯ-rgu,sɯ}{}\classe{np}
\begin{définition}\fra robuste, fort (malgré les apparences)\end{définition}
\begin{définition}\cmn 有(出乎意料的)能力,体力\end{définition}
\begin{exemple}\jya nɤ-rgu (ɯ-tɯ-sɯ) nɯ!\cmn (没有想到)你办得到\end{exemple}
\begin{exemple}\jya tɤ-pɤtso ɯ-rgu ɲɯ-sɯ tɕe, ɲɯ-rkaŋ\cmn 那个小孩子很有能力,很壮(不要小看他)\end{exemple}
\begin{relation-sémantique}\ComponentA{\classe{np}
 \papi{ɯ-rgu}
}\end{relation-sémantique}
\begin{relation-sémantique}\ComponentB{\classe{vs}
\hyperlink{Ⓔsɯ}{\textit{ \papi{sɯ}}}
}\end{relation-sémantique}\end{entrée}

\begin{entrée}
\vedette{\hypertarget{Ⓔɯ-rɟa}{\papi{ ɯ-rɟa}}}\markboth{ɯ-rɟa}{}\classe{n}
\begin{définition}\fra injure\end{définition}
\begin{définition}\cmn 咒人的话\end{définition}
\begin{relation-sémantique}\synonyme{
\hyperlink{Ⓔkhɤrma}{\textit{ \papi{khɤrma}}}
}\end{relation-sémantique}
\begin{relation-sémantique}\confer{
\hyperlink{Ⓔrɯrɟa}{\textit{ \papi{rɯrɟa}}}
}\end{relation-sémantique}\end{entrée}

\begin{entrée}
\vedette{\hypertarget{Ⓔɯ-rɟɤŋgo}{\papi{ ɯ-rɟɤŋgo}}}\markboth{ɯ-rɟɤŋgo}{}\classe{np}
\begin{définition}\fra complication (maladie)\end{définition}
\begin{définition}\cmn 并发症\end{définition}
\end{entrée}

\begin{entrée}
\vedette{\hypertarget{Ⓔɯ-rka,ŋɤn}{\papi{ ɯ-rka,ŋɤn}}}\markboth{ɯ-rka,ŋɤn}{}
\begin{définition}\fra être ingrat, mauvais\end{définition}
\begin{définition}\cmn 心底不好,不怀好意\end{définition}
\begin{exemple}\jya nɤ-rka ɯ-tɯ-ŋɤn\cmn 你心底不好\end{exemple}
\begin{exemple}\jya ɯ-rka ɲɯ-ŋɤn ma ɯ-taʁ wuma ʑo pɯ-pe-a ɕti ri, tham tɕe ɯʑo kɯ mɯ́j-wɣ-nɯkon-a\cmn 他心底不好,我原来对他很好,现在他却不理我\end{exemple}
\begin{relation-sémantique}\ComponentA{\classe{np}
 \papi{ɯ-rka}
}\end{relation-sémantique}
\begin{relation-sémantique}\ComponentB{\classe{vs}
\hyperlink{Ⓔŋɤn}{\textit{ \papi{ŋɤn}}}
}\end{relation-sémantique}
\begin{sous-entrée}
\vedette{\hypertarget{}{\papi{ ɯ-rka,ɣɤŋɤn}}}\markboth{ɯ-rka,ɣɤŋɤn}{}
\begin{exemple}\jya nɤ-rka ma-tɯ-ɣɤŋɤn\cmn 不要起坏心\end{exemple}
\begin{relation-sémantique}\ComponentA{\classe{np}
 \papi{ɯ-rka}
}\end{relation-sémantique}
\begin{relation-sémantique}\ComponentB{\classe{vt}
\hyperlink{Ⓔɣɤŋɤn}{\textit{ \papi{ɣɤŋɤn}}}
}\end{relation-sémantique}
\end{sous-entrée}\end{entrée}

\begin{entrée}
\vedette{\hypertarget{Ⓔɯ-rkarkɯ}{\papi{ ɯ-rkarkɯ}}}\markboth{ɯ-rkarkɯ}{}\classe{np}
\begin{définition}\ 
\begin{déclaration}\grammar{n.rdpl}\end{déclaration}\end{définition}
\begin{définition}\fra bord\end{définition}
\begin{définition}\cmn 边缘\end{définition}
\begin{relation-sémantique}\synonyme{
\hyperlink{Ⓔɯ-zarzɯr}{\textit{ \papi{ɯ-zarzɯr}}}
}\end{relation-sémantique}
\begin{relation-sémantique}\confer{
\hyperlink{Ⓔɯ-rkɯ}{\textit{ \papi{ɯ-rkɯ}}}
}\end{relation-sémantique}\end{entrée}

\begin{entrée}
\vedette{\hypertarget{Ⓔɯrkoz}{\papi{ ɯrkoz}}}\markboth{ɯrkoz}{}\classe{adv}
\begin{définition}\fra spécialement\end{définition}
\begin{définition}\cmn 专门\end{définition}\end{entrée}

\begin{entrée}
\vedette{\hypertarget{Ⓔɯ-rkɯ}{\papi{ ɯ-rkɯ}}}\markboth{ɯ-rkɯ}{}\classe{np}
\begin{définition}\fra côté\end{définition}
\begin{définition}\cmn 旁边;角落\end{définition}
\begin{relation-sémantique}\confer{
\hyperlink{Ⓔɯ-rkarkɯ}{\textit{ \papi{ɯ-rkarkɯ}}}
}\end{relation-sémantique}\begin{sous-entrée}
\vedette{\hypertarget{}{\papi{ ɯ-rkɯɕɯrkɯ}}}\markboth{ɯ-rkɯɕɯrkɯ}{}
\begin{définition}\fra le côté le plus au bord\end{définition}
\begin{définition}\cmn 最边缘\end{définition}
\begin{exemple}\jya zɣɤmbu nɯ kha ɯ-rkɯɕɯrkɯ nɯtɕu ɲɯ́-wɣ-ta ra\cmn 扫把要放在房子的最边缘\end{exemple}
\end{sous-entrée}\end{entrée}

\begin{entrée}
\vedette{\hypertarget{Ⓔɯ-rkɯm}{\papi{ ɯ-rkɯm}}}\markboth{ɯ-rkɯm}{}\classe{np}
\begin{définition}\fra cotylédon\end{définition}
\begin{définition}\cmn 子叶\end{définition}
\begin{exemple}\jya lɤpɯɣ ɯ-rkɯm\cmn 萝卜的子叶\end{exemple}
\begin{relation-sémantique}\confer{
\hyperlink{Ⓔkarkɯm}{\textit{ \papi{karkɯm}}}
}\end{relation-sémantique}\end{entrée}

\begin{entrée}
\vedette{\hypertarget{Ⓔɯ-rma}{\papi{ ɯ-rma}}}\markboth{ɯ-rma}{}\classe{np}
\begin{définition}\fra ferment\end{définition}
\begin{définition}\cmn 酵母;曲子\end{définition}\end{entrée}

\begin{entrée}
\vedette{\hypertarget{Ⓔɯ-rmɯrɟa}{\papi{ ɯ-rmɯrɟa}}}\markboth{ɯ-rmɯrɟa}{}\classe{np}
\begin{définition}\fra sobriquet, surnom\end{définition}
\begin{définition}\cmn 外号(贬义)\end{définition}
\begin{relation-sémantique}\confer{
\hyperlink{Ⓔɯ-rɟa}{\textit{ \papi{ɯ-rɟa}}}
}\end{relation-sémantique}
\begin{relation-sémantique}\confer{
\hyperlink{Ⓔtɤ-rmi}{\textit{ \papi{tɤ-rmi}}}
}\end{relation-sémantique}\end{entrée}

\begin{entrée}
\vedette{\hypertarget{Ⓔɯ-rnɤɣmbaj}{\papi{ ɯ-rnɤɣmbaj}}}\markboth{ɯ-rnɤɣmbaj}{}\classe{np}
\begin{définition}\fra côté de l'oreille\end{définition}
\begin{définition}\cmn 耳边\end{définition}
\begin{relation-sémantique}\confer{
\hyperlink{Ⓔtɯ-rna}{\textit{ \papi{tɯ-rna}}}
}\end{relation-sémantique}\end{entrée}

\begin{entrée}
\vedette{\hypertarget{Ⓔɯ-rnɤqhu}{\papi{ ɯ-rnɤqhu}}}\markboth{ɯ-rnɤqhu}{}
\classe{np}
\begin{définition}\fra derrière les oreilles\end{définition}
\begin{définition}\cmn 耳朵后面\end{définition}\end{entrée}

\begin{entrée}
\vedette{\hypertarget{Ⓔɯ-rozre}{\papi{ ɯ-rozre}}}\markboth{ɯ-rozre}{}
\classe{np}
\begin{définition}\fra reste\end{définition}
\begin{définition}\cmn 剩余的;余留的残渣\end{définition}
\begin{exemple}\jya tɯ-ŋga ɯ-rozre nɯra jɤ-tsɯm\cmn 你把剩下的衣服带走\end{exemple}
\begin{relation-sémantique}\confer{
\hyperlink{Ⓔtɤ-ro}{\textit{ \papi{tɤ-ro}}}
}\end{relation-sémantique}\end{entrée}

\begin{entrée}
\vedette{\hypertarget{Ⓔɯ-rqɯ}{\papi{ ɯ-rqɯ}}}\markboth{ɯ-rqɯ}{}\classe{np}
\begin{définition}\fra objet froid\end{définition}
\begin{définition}\cmn 冷的东西\end{définition}
\begin{relation-sémantique}\confer{
\hyperlink{Ⓔtɯcɯrqɯ}{\textit{ \papi{tɯcɯrqɯ}}}
}\end{relation-sémantique}\end{entrée}

\begin{entrée}
\vedette{\hypertarget{Ⓔɯ-rtɤβ}{\papi{ ɯ-rtɤβ}}}\markboth{ɯ-rtɤβ}{}\classe{np}
\begin{définition}\fra lanière ornée\end{définition}
\begin{définition}\cmn 带有装饰的带子
\end{définition}\end{entrée}

\begin{entrée}
\vedette{\hypertarget{Ⓔɯ-rti}{\papi{ ɯ-rti}}}\markboth{ɯ-rti}{}\classe{np}
\begin{définition}\fra embryon de poulain\end{définition}
\begin{définition}\cmn 马的胚胎
\begin{déclaration} \étymologie{\papi{rteɦu}}\end{déclaration}\end{définition}
\begin{exemple}\jya rgonma ɯ-rti kɯ-mbro\cmn 快要生的母马\end{exemple}
\begin{exemple}\jya mbro ɯ-rti kɯ-tu\cmn 怀孕的母马\end{exemple}
\begin{relation-sémantique}\confer{
\hyperlink{Ⓔrɤrti}{\textit{ \papi{rɤrti}}}
}\end{relation-sémantique}
\begin{relation-sémantique}\confer{
\hyperlink{Ⓔtɯ-rti}{\textit{ \papi{tɯ-rti}}}
}\end{relation-sémantique}\end{entrée}

\begin{entrée}
\vedette{\hypertarget{Ⓔɯ-rtsa,tɕɤt}{\papi{ ɯ-rtsa,tɕɤt}}}\markboth{ɯ-rtsa,tɕɤt}{}
\paradigme{\textit{dir :} \jya nɯ-}
\begin{définition}\fra rechercher la cause de\end{définition}
\begin{définition}\cmn 追究\end{définition}
\begin{exemple}\jya ɯ-rtsa ɲɤ-tɕɤt\cmn 他追究了\end{exemple}
\begin{relation-sémantique}\confer{
\hyperlink{Ⓔnɯrtsa}{\textit{ \papi{nɯrtsa}}}
}\end{relation-sémantique}
\begin{relation-sémantique}\ComponentA{\classe{np}
 \papi{ɯ-rtsa}
}\end{relation-sémantique}
\begin{relation-sémantique}\ComponentB{\classe{vt}
\hyperlink{Ⓔtɕɤt}{\textit{ \papi{tɕɤt}}}
}\end{relation-sémantique}\end{entrée}

\begin{entrée}
\vedette{\hypertarget{Ⓔɯ-rtshɯ}{\papi{ ɯ-rtshɯ}}}\markboth{ɯ-rtshɯ}{}\classe{np}
\begin{définition}\fra écorce de légumineuse (pour nourrir les bovidés)\end{définition}
\begin{définition}\cmn 豆类的粗糠秕,喂牛\end{définition}
\begin{exemple}\jya stoʁ rtshɯ\cmn 胡豆的粗糠秕\end{exemple}\end{entrée}

\begin{entrée}
\vedette{\hypertarget{Ⓔɯ-rtshɯm}{\papi{ ɯ-rtshɯm}}}\markboth{ɯ-rtshɯm}{}\classe{np}
\begin{définition}\fra section\end{définition}
\begin{définition}\cmn 一段(不完整)\end{définition}
\end{entrée}

\begin{entrée}
\vedette{\hypertarget{Ⓔɯ-rtsi}{\papi{ ɯ-rtsi}}}\markboth{ɯ-rtsi}{}
\classe{np}
\begin{définition}\fra laque\end{définition}
\begin{définition}\cmn 漆;油
\begin{déclaration} \étymologie{\papi{rtsi}}\end{déclaration}\end{définition}
\begin{exemple}\jya ɯ-rtsi chɤ-lɤt\cmn 他上了漆\end{exemple}
\begin{exemple}\jya rgɯnba ɯ-rtsi to-lɤt\cmn 在庙里上了漆\end{exemple}
\begin{exemple}\jya laχtɕha ɯ-rtsi to-lɤt\cmn 他给家具上了漆\end{exemple}
\begin{relation-sémantique}\confer{
\hyperlink{Ⓔsɯrtsi}{\textit{ \papi{sɯrtsi}}}
}\end{relation-sémantique}\end{entrée}

\begin{entrée}
\vedette{\hypertarget{Ⓔɯ-rtɯrtɤβ}{\papi{ ɯ-rtɯrtɤβ}}}\markboth{ɯ-rtɯrtɤβ}{}\classe{np}
\begin{définition}\fra personne collante\end{définition}
\begin{définition}\cmn 缠着别人不放\end{définition}
\begin{relation-sémantique}\confer{
\hyperlink{Ⓔrtɤβ}{\textit{ \papi{rtɤβ}}}
}\end{relation-sémantique}\end{entrée}

\begin{entrée}
\vedette{\hypertarget{Ⓔɯ-rɯɣ}{\papi{ ɯ-rɯɣ}}}\markboth{ɯ-rɯɣ}{}\classe{np}
\begin{définition}\fra nationalité, race\end{définition}
\begin{définition}\cmn 族
\begin{déclaration} \étymologie{\papi{rigs}}\end{déclaration}\end{définition}
\end{entrée}

\begin{entrée}
\vedette{\hypertarget{Ⓔɯrɯruz}{\papi{ ɯrɯruz}}}\markboth{ɯrɯruz}{}\classe{adv}
\begin{définition}\fra à ce moment\end{définition}
\begin{définition}\cmn 当时;眼前\end{définition}
\begin{exemple}\jya ɯʑo tɤ-ngo tɕe ɯrɯruz nɯ wuma pɯ-sɤɣʑɯr, kɯ-maqhu ʁo tɕe to-mna.\cmn 他生病的时候,当时有生命危险,最后还是痊愈了\end{exemple}
\begin{exemple}\jya ɯrɯruz ɣɯ ɯ-ndaŋ ma ɯ-qhu ɣɯ ɯ-ndaŋ kɤ-lɤt mɯ́j-spe\cmn 他只会考虑眼前的事,不会考虑后果\end{exemple}\end{entrée}

\begin{entrée}
\vedette{\hypertarget{Ⓔɯ-rɯz}{\papi{ ɯ-rɯz}}}\markboth{ɯ-rɯz}{}\classe{np}\acception{1}
\begin{définition}\fra sorte, espèce\end{définition}
\begin{définition}\cmn 类别;品种\end{définition}
\begin{exemple}\jya nɤki rɤjndoʁ ɯ-rɯz ɲɯ-ŋu\cmn 这是大头菜的一种\end{exemple}\acception{2}
\begin{définition}\fra tour (travail)\end{définition}
\begin{définition}\cmn 轮到自己(办事)\end{définition}
\begin{exemple}\jya jisŋi kɤ-rɤma a-rɯz ŋu\cmn 今天轮到我上班\end{exemple}
\begin{exemple}\jya a-rɯz jɤ-azɣɯt\cmn 轮到我了\end{exemple}
\begin{relation-sémantique}\synonyme{
 \papi{ɯ-βra}
}\end{relation-sémantique}\acception{3}
\begin{définition}\fra clairvoyance\end{définition}
\begin{définition}\cmn 预见
\begin{déclaration} \étymologie{\papi{rigs}}\end{déclaration}\end{définition}\end{entrée}

\begin{entrée}
\vedette{\hypertarget{Ⓔɯ-rwarwa}{\papi{ ɯ-rwarwa}}}\markboth{ɯ-rwarwa}{}
\classe{np}
\begin{définition}\fra crête\end{définition}
\begin{définition}\cmn 鸡冠\end{définition}\end{entrée}

\begin{entrée}
\vedette{\hypertarget{Ⓔɯ-ʁɤri}{\papi{ ɯ-ʁɤri}}}\markboth{ɯ-ʁɤri}{}\classe{np}
\begin{définition}\fra avant\end{définition}
\begin{définition}\cmn 前面\end{définition}
\begin{relation-sémantique}\confer{
 \papi{ɯ-ʁɤri,astu}
}\end{relation-sémantique}
\end{entrée}

\begin{entrée}
\vedette{\hypertarget{Ⓔɯ-ʁdɤz}{\papi{ ɯ-ʁdɤz}}}\markboth{ɯ-ʁdɤz}{}\classe{np}
\begin{définition}\fra charge, souci\end{définition}
\begin{définition}\cmn 负担\end{définition}
\begin{exemple}\jya ndʑi-ʁdɤz tɕɤt-tɕi\cmn 我们加重你们俩的负担\end{exemple}
\begin{relation-sémantique}\confer{
\hyperlink{Ⓔnaʁdɤz}{\textit{ \papi{naʁdɤz}}}
}\end{relation-sémantique}\end{entrée}

\begin{entrée}
\vedette{\hypertarget{Ⓔɯ-ʁjoʁ}{\papi{ ɯ-ʁjoʁ}}}\markboth{ɯ-ʁjoʁ}{}
\classe{np}
\begin{définition}\fra partie extérieure des vêtements\end{définition}
\begin{définition}\cmn 衣服的外层\end{définition}
\begin{exemple}\jya tɯ-ŋga ɯ-ʁjoʁ\cmn 衣服的外层\end{exemple}
\begin{relation-sémantique}\antonyme{
\hyperlink{Ⓔnaŋɕa}{\textit{ \papi{naŋɕa}}}
}\end{relation-sémantique}
\begin{relation-sémantique}\confer{
\hyperlink{Ⓔsɯʁjoʁ}{\textit{ \papi{sɯʁjoʁ}}}
}\end{relation-sémantique}\end{entrée}

\begin{entrée}
\vedette{\hypertarget{Ⓔɯ-ʁlu}{\papi{ ɯ-ʁlu}}}\markboth{ɯ-ʁlu}{}\classe{np}
\begin{définition}\fra endroit concave\end{définition}
\begin{définition}\cmn 凹下去的地形\end{définition}
\begin{relation-sémantique}\confer{
\hyperlink{Ⓔaʁloʁlu}{\textit{ \papi{aʁloʁlu}}}
}\end{relation-sémantique}\end{entrée}

\begin{entrée}
\vedette{\hypertarget{Ⓔɯ-ʁlɤt}{\papi{ ɯ-ʁlɤt}}}\markboth{ɯ-ʁlɤt}{}\classe{np}
\begin{définition}\fra canon\end{définition}
\begin{définition}\cmn 枪杆【枪肚子】\end{définition}
\begin{exemple}\jya ɯ-ʁlɤt nɯ ɕɤmɯɣdɯ mɯzi sɤ-rkɯ ŋu, ɕɤmɯɣdɯ sna mɤ-sna nɯ ʁlɤt rɤmdzɯt\cmn 枪肚子是用来装火药的洞,枪肚子决定枪的好与坏。\end{exemple}
\end{entrée}

\begin{entrée}
\vedette{\hypertarget{Ⓔɯ-ʁle}{\papi{ ɯ-ʁle}}}\markboth{ɯ-ʁle}{}\classe{np}
\begin{définition}\fra réputation\end{définition}
\begin{définition}\cmn 名声,面子\end{définition}
\begin{exemple}\jya ɯ-ʁle ɣɤʑu\cmn 他有好名声\end{exemple}
\begin{exemple}\jya ɯ-ʁle ko-ru tɕe to-nɯŋgumtha\cmn 他为了得到好名声就照顾他了\end{exemple}
\begin{exemple}\jya nɤʑo ɯ-ʁle ma-kɤ-tɯ-ru kɯ koŋla tú-wɣ-sɤpe ra\cmn 不要只顾名声,要干实际的事情\end{exemple}
\begin{relation-sémantique}\confer{
\hyperlink{Ⓔraʁle}{\textit{ \papi{raʁle}}}
}\end{relation-sémantique}
\begin{relation-sémantique}\confer{
\hyperlink{Ⓔqale}{\textit{ \papi{qale}}}
}\end{relation-sémantique}\end{entrée}

\begin{entrée}
\vedette{\hypertarget{Ⓔɯ-ʁnawa}{\papi{ ɯ-ʁnawa}}}\markboth{ɯ-ʁnawa}{}\classe{np}
\begin{définition}\fra vacances\end{définition}
\begin{définition}\cmn (请)假\end{définition}
\begin{exemple}\jya ji-ŋgundʑɯɣ kɯ a-ʁnawa mɯ́j-nɤle\cmn 我们领导不给我请假\end{exemple}
\begin{exemple}\jya a-ʁnawa tɤ-thu-t-a\cmn 我请了假\end{exemple}\end{entrée}

\begin{entrée}
\vedette{\hypertarget{Ⓔɯ-ʁɲɤrŋa}{\papi{ ɯ-ʁɲɤrŋa}}}\markboth{ɯ-ʁɲɤrŋa}{}\classe{np}
\begin{définition}\fra crosse\end{définition}
\begin{définition}\cmn 枪把\end{définition}
\begin{exemple}\jya ɯ-ʁɲɤrŋa nɯ tɯ-rpaʁ ɯ-sɤ-χtɤt ŋu\cmn 枪把是用来抵住肩膀的部件。\end{exemple}\end{entrée}

\begin{entrée}
\vedette{\hypertarget{Ⓔɯ-ʁre}{\papi{ ɯ-ʁre}}}\markboth{ɯ-ʁre}{}
\classe{np}
\begin{définition}\fra respect, prestige, authorité\end{définition}
\begin{définition}\cmn 威望\end{définition}
\begin{exemple}\jya ɯ-ʁre ɣɤʑu (=ɲɯ-ɣɤʁre), ɯʑo ɯ-ʁre kɯ-tu ci ɲɯ-ŋu\cmn 他是有威望的人\end{exemple}
\begin{relation-sémantique}\confer{
\hyperlink{Ⓔsaʁre}{\textit{ \papi{saʁre}}}
}\end{relation-sémantique}
\begin{relation-sémantique}\confer{
\hyperlink{Ⓔnaʁre}{\textit{ \papi{naʁre}}}
}\end{relation-sémantique}
\begin{relation-sémantique}\confer{
\hyperlink{Ⓔɣɤʁre}{\textit{ \papi{ɣɤʁre}}}
}\end{relation-sémantique}\end{entrée}

\begin{entrée}
\vedette{\hypertarget{Ⓔɯ-sɤɣjɤɣ}{\papi{ ɯ-sɤɣjɤɣ}}}\markboth{ɯ-sɤɣjɤɣ}{}\classe{np}
\begin{définition}\fra fin\end{définition}
\begin{définition}\cmn 结尾\end{définition}
\begin{exemple}\jya χpi ɯ-sɤɣjɤɣ\cmn 故事的结尾\end{exemple}
\begin{relation-sémantique}\confer{
\hyperlink{Ⓔjɤɣ}{\textit{ \papi{jɤɣ}}}
}\end{relation-sémantique}
\end{entrée}

\begin{entrée}
\vedette{\hypertarget{Ⓔɯ-sɤɣɬoʁ}{\papi{ ɯ-sɤɣɬoʁ}}}\markboth{ɯ-sɤɣɬoʁ}{}\classe{np}
\begin{définition}\fra endroit où une plante pousse\end{définition}
\begin{définition}\cmn 生长的地方(植物、蘑菇)\end{définition}
\begin{relation-sémantique}\confer{
\hyperlink{ⒺɬoʁⒽ2}{\textit{ \papi{ɬoʁ2}}}
}\end{relation-sémantique}
\end{entrée}

\begin{entrée}
\vedette{\hypertarget{Ⓔɯ-sɤpe}{\papi{ ɯ-sɤpe}}}\markboth{ɯ-sɤpe}{}\classe{np}
\begin{définition}\fra avantage\end{définition}
\begin{définition}\cmn 好处\end{définition}
\begin{relation-sémantique}\confer{
\hyperlink{Ⓔpe}{\textit{ \papi{pe}}}
}\end{relation-sémantique}
\end{entrée}

\begin{entrée}
\vedette{\hypertarget{Ⓔɯ-sɤʁjɯʁjit}{\papi{ ɯ-sɤʁjɯʁjit}}}\markboth{ɯ-sɤʁjɯʁjit}{}
\begin{relation-sémantique}\confer{
\hyperlink{Ⓔʁjit}{\textit{ \papi{ʁjit}}}
}\end{relation-sémantique}\end{entrée}

\begin{entrée}
\vedette{\hypertarget{Ⓔɯ-sɤsɤʑa}{\papi{ ɯ-sɤsɤʑa}}}\markboth{ɯ-sɤsɤʑa}{}\classe{np}
\begin{définition}\fra début\end{définition}
\begin{définition}\cmn 开头\end{définition}
\begin{exemple}\jya nɤ-kɤ-ti ɯ-sɤsɤʑa kɯ-tu maŋe\cmn 你的说法无从说起(没有根据)\end{exemple}
\begin{relation-sémantique}\confer{
\hyperlink{ⒺʑaⒽ1}{\textit{ \papi{ʑa}}}
}\end{relation-sémantique}
\begin{relation-sémantique}\confer{
\hyperlink{Ⓔsɤʑa}{\textit{ \papi{sɤʑa}}}
}\end{relation-sémantique}\end{entrée}

\begin{entrée}
\vedette{\hypertarget{Ⓔɯ-sɤti}{\papi{ ɯ-sɤti}}}\markboth{ɯ-sɤti}{}\classe{np}
\begin{définition}\fra prétexte\end{définition}
\begin{définition}\cmn 借口\end{définition}
\begin{relation-sémantique}\confer{
\hyperlink{Ⓔti}{\textit{ \papi{ti}}}
}\end{relation-sémantique}
\end{entrée}

\begin{entrée}
\vedette{\hypertarget{Ⓔɯ-scawa}{\papi{ ɯ-scawa}}}\markboth{ɯ-scawa}{}\classe{np}
\begin{définition}\fra pitoyable\end{définition}
\begin{définition}\cmn 可怜,无奈
\begin{déclaration} \étymologie{\papi{skʲo.ba}}\end{déclaration}\end{définition}
\begin{exemple}\jya nɤ-scawa ɲɯ-saχaʁ\cmn 你很可怜\end{exemple}
\end{entrée}

\begin{entrée}
\vedette{\hypertarget{Ⓔɯ-sci}{\papi{ ɯ-sci}}}\markboth{ɯ-sci}{}
\classe{np}
\begin{définition}\fra à la place de\end{définition}
\begin{définition}\cmn 替
\begin{déclaration} \étymologie{\papi{skʲi}}\end{déclaration}\end{définition}
\begin{exemple}\jya nɤ-sci aʑo ju-ɕe-a\cmn 我替你去\end{exemple}
\begin{exemple}\jya aʑo a-ʁa maŋe tɕe, nɤʑo a-sci tu-tɯ-βze ɯ-tɯ-cha?\cmn 我没有空,你可以代替我做吗?\end{exemple}\begin{sous-entrée}
\vedette{\hypertarget{}{\papi{ ɯ-sci,βzu}}}\markboth{ɯ-sci,βzu}{}
\begin{relation-sémantique}\ComponentA{\classe{np}
\hyperlink{Ⓔɯ-sci}{\textit{ \papi{ɯ-sci}}}
}\end{relation-sémantique}
\begin{relation-sémantique}\ComponentB{\classe{vt}
\hyperlink{ⒺβzuⒽ1}{\textit{ \papi{βzu}}}
}\end{relation-sémantique}\acception{1}
\begin{définition}\fra se venger de\end{définition}
\begin{définition}\cmn 报仇\end{définition}
\begin{relation-sémantique}\synonyme{
 \papi{ɯ-rtsot,βzu}
}\end{relation-sémantique}\acception{2}
\begin{définition}\fra remplacer\end{définition}
\begin{définition}\cmn 代替\end{définition}
\begin{relation-sémantique}\synonyme{
 \papi{ɯ-tshɤt,βzu}
}\end{relation-sémantique}\acception{3}
\begin{définition}\fra répondre\end{définition}
\begin{définition}\cmn 答复\end{définition}
\begin{exemple}\jya ɯ-sci to-βzu (=ɯ-tshɤt to-βzu; ɯ-rtsot to-βzu; ɯ-lɤn to-βzu)\cmn 他报复了他;他代替了他;他答复了他\end{exemple}
\begin{relation-sémantique}\synonyme{
 \papi{ɯ-lɤn,βzu}
}\end{relation-sémantique}
\begin{relation-sémantique}\synonyme{
 \papi{ɯ-ntsi,βzu}
}\end{relation-sémantique}
\end{sous-entrée}\end{entrée}

\begin{entrée}
\vedette{\hypertarget{Ⓔɯ-sku}{\papi{ ɯ-sku}}}\markboth{ɯ-sku}{}\classe{np}
\begin{définition}\fra tiges et feuilles du navet\end{définition}
\begin{définition}\cmn 圆根的茎和叶子\end{définition}\end{entrée}

\begin{entrée}
\vedette{\hypertarget{Ⓔɯ-smɤnjɯn}{\papi{ ɯ-smɤnjɯn}}}\markboth{ɯ-smɤnjɯn}{}\classe{np}
\begin{définition}\fra prix du traitement\end{définition}
\begin{définition}\cmn 药费\end{définition}
\begin{relation-sémantique}\confer{
\hyperlink{Ⓔsmɤn}{\textit{ \papi{smɤn}}}
}\end{relation-sémantique}\end{entrée}

\begin{entrée}
\vedette{\hypertarget{Ⓔɯ-spjɯŋ}{\papi{ ɯ-spjɯŋ}}}\markboth{ɯ-spjɯŋ}{}\classe{np}
\begin{définition}\fra tige centrale\end{définition}
\begin{définition}\cmn 主心干\end{définition}
\end{entrée}

\begin{entrée}
\vedette{\hypertarget{Ⓔɯ-spɯɣ}{\papi{ ɯ-spɯɣ}}}\markboth{ɯ-spɯɣ}{}\classe{np}
\begin{définition}\fra partie proche du corps (membre)\end{définition}
\begin{définition}\cmn 靠近身体的部分;根部(肢体)\end{définition}
\begin{exemple}\jya tɯ-jaʁ ɯ-spɯɣ\cmn 肩膀\end{exemple}
\begin{exemple}\jya tɯ-jaʁndzu ɯ-spɯɣ\cmn 手指的根部\end{exemple}\end{entrée}

\begin{entrée}
\vedette{\hypertarget{Ⓔɯ-sqar}{\papi{ ɯ-sqar}}}\markboth{ɯ-sqar}{}\classe{np}
\begin{définition}\fra endroit où les fils se croisent (pendant le tissage)\end{définition}
\begin{définition}\cmn 线上下交叉的地方(织布的时候)\end{définition}\end{entrée}

\begin{entrée}
\vedette{\hypertarget{Ⓔɯ-srɯβ}{\papi{ ɯ-srɯβ}}}\markboth{ɯ-srɯβ}{}\classe{np}
\begin{définition}\fra fissure, interstice, couture\end{définition}
\begin{définition}\cmn 裂缝;针脚
\begin{déclaration} \étymologie{\papi{srubs}}\end{déclaration}\end{définition}
\end{entrée}

\begin{entrée}
\vedette{\hypertarget{Ⓔɯ-stuⒽ1}{\papi{ ɯ-stu}}}\markboth{ɯ-stu}{}\homonyme{1}
\classe{np}
\begin{définition}\fra vers l'avant, directement\end{définition}
\begin{définition}\cmn 对面的地方或方向,直接\end{définition}
\begin{exemple}\jya ɯ-stu ʑo kɤ-ɕe tɕe tɤ-atɯɣ\cmn 你直走就会遇到\end{exemple}
\begin{relation-sémantique}\confer{
\hyperlink{Ⓔastu}{\textit{ \papi{astu}}}
}\end{relation-sémantique}\end{entrée}

\begin{entrée}
\vedette{\hypertarget{Ⓔɯ-stuⒽ2}{\papi{ ɯ-stu}}}\markboth{ɯ-stu}{}\homonyme{2}\classe{np}
\begin{définition}\fra vrai\end{définition}
\begin{définition}\cmn 真心,准确,实话\end{définition}
\begin{exemple}\jya nɤ-stu tɤ-fse\cmn 你要注意一下,你要守规矩一点\end{exemple}
\begin{exemple}\jya a-stu tu-ti-a ŋu ma\cmn 我是说真心话\end{exemple}
\begin{exemple}\jya ɯʑo kɯ ɯ-stu tu-ti ɲɯ-ŋu\cmn 他是说真话\end{exemple}
\begin{exemple}\jya kɤ-nɤma ɯ-stu tú-wɣ-nɤma ra ma mɤ-pe\cmn 事要用心做,不然会不好的\end{exemple}
\begin{exemple}\jya a-stu tu-ti-a\cmn 我说实话\end{exemple}
\begin{exemple}\jya nɤ-stu tɤ-ti\cmn 你说实话吧\end{exemple}
\begin{exemple}\jya nɤ-stu tu-tɯ-ti ɯ́-ŋu ?\cmn 你说实话吗?\end{exemple}
\begin{exemple}\jya nɤ-stu ɯ́-ŋu?\cmn 你说的是不是实话?\end{exemple}
\begin{relation-sémantique}\confer{
\hyperlink{ⒺstuⒽ2}{\textit{ \papi{stu2}}}
}\end{relation-sémantique}\end{entrée}

\begin{entrée}
\vedette{\hypertarget{Ⓔɯ-sta}{\papi{ ɯ-sta}}}\markboth{ɯ-sta}{}\classe{np}
\begin{définition}\fra habitude\end{définition}
\begin{définition}\cmn 习惯\end{définition}
\begin{exemple}\jya aʑɯɣ ɯ-sta a-nɯ-βze ra (=a-nɯ-ɕat-a ra)\cmn 我要养成习惯\end{exemple}
\begin{exemple}\jya aʑo kɤ-nɯmtɕi ɯ-sta na-βzu\cmn 我早起惯了\end{exemple}
\begin{exemple}\jya nɤʑo kɤ-nɯmtɕi ɯ-sta ɲɤ-k-ɤβzu-ci (=nɤʑo kɤ-nɯmtɕi ɲɤ-tɯ-ɕɤt)\cmn 你早起惯了\end{exemple}
\begin{exemple}\jya nɤki nɯ ɯ-kɯ-mŋɤm to-mna tɕe ɯ-sta ʑo to-fse\cmn 那个人病好了,恢复了原状\end{exemple}
\begin{relation-sémantique}\confer{
\hyperlink{Ⓔtɤ-sta}{\textit{ \papi{tɤ-sta}}}
}\end{relation-sémantique}
\begin{relation-sémantique}\confer{
\hyperlink{Ⓔtɯ-sta}{\textit{ \papi{tɯ-sta}}}
}\end{relation-sémantique}
\begin{relation-sémantique}\confer{
\hyperlink{Ⓔta}{\textit{ \papi{ta}}}
}\end{relation-sémantique}\end{entrée}

\begin{entrée}
\vedette{\hypertarget{Ⓔɯ-stɤrju}{\papi{ ɯ-stɤrju}}}\markboth{ɯ-stɤrju}{}\classe{np}
\begin{définition}\fra vérité\end{définition}
\begin{définition}\cmn 真话\end{définition}
\begin{relation-sémantique}\synonyme{
\hyperlink{Ⓔtʂaŋχtɤm}{\textit{ \papi{tʂaŋχtɤm}}}
}\end{relation-sémantique}
\begin{relation-sémantique}\confer{
\hyperlink{Ⓔtɯ-rju}{\textit{ \papi{tɯ-rju}}}
}\end{relation-sémantique}
\begin{relation-sémantique}\confer{
\hyperlink{Ⓔɯ-stuⒽ2}{\textit{ \papi{ɯ-stu2}}}
}\end{relation-sémantique}\end{entrée}

\begin{entrée}
\vedette{\hypertarget{Ⓔɯ-sti}{\papi{ ɯ-sti}}}\markboth{ɯ-sti}{}
\classe{np}
\begin{définition}\fra seul\end{définition}
\begin{définition}\cmn 孤单\end{définition}
\begin{exemple}\jya aʑo-sti ma me-a\cmn 只有我一个人\end{exemple}
\begin{relation-sémantique}\confer{
\hyperlink{Ⓔmɯsti}{\textit{ \papi{mɯsti}}}
}\end{relation-sémantique}
\begin{relation-sémantique}\confer{
\hyperlink{Ⓔnɯstɤraʁndo}{\textit{ \papi{nɯstɤraʁndo}}}
}\end{relation-sémantique}
\end{entrée}

\begin{entrée}
\vedette{\hypertarget{Ⓔɯ-tɤjɯ}{\papi{ ɯ-tɤjɯ}}}\markboth{ɯ-tɤjɯ}{}\classe{np}\acception{1}
\begin{définition}\fra ajouté\end{définition}
\begin{définition}\cmn 填充的\end{définition}
\begin{exemple}\jya ki nɤ-ŋga ɯ-tɤjɯ a-pɯ-ŋu ma tɯ-nɤndʐo\cmn 再给你这件衣服,不然你会冷的\end{exemple}
\begin{exemple}\jya kɯki kɯ mɤ-tɯ́-wɣ-ɕɯfka tɕe, nɤ-tɤjɯ a-pɯ-tu ra\cmn 这一点东西吃不饱,就拿这个填肚子吧\end{exemple}
\begin{relation-sémantique}\confer{
\hyperlink{Ⓔɣɤjɯ}{\textit{ \papi{ɣɤjɯ}}}
}\end{relation-sémantique}\acception{2}
\begin{définition}\fra non seulement ... mais\end{définition}
\begin{définition}\cmn 不但……而且\end{définition}
\begin{exemple}\jya ɲɯ-ɕɯmŋɤm ɯ-tɤjɯ tɕe ɲɯ-sɤzoŋzoŋ ʑo ŋu\cmn (荨麻)不但把人弄痛,而且让人发麻\end{exemple}
\begin{relation-sémantique}\synonyme{
 \papi{ʁo alala ri}
}\end{relation-sémantique}
\begin{relation-sémantique}\synonyme{
 \papi{mɤra ma}
}\end{relation-sémantique}
\begin{relation-sémantique}\synonyme{
 \papi{maʁ kɯ}
}\end{relation-sémantique}\end{entrée}

\begin{entrée}
\vedette{\hypertarget{Ⓔɯ-tɤmcar}{\papi{ ɯ-tɤmcar}}}\markboth{ɯ-tɤmcar}{}\classe{np}
\begin{définition}\fra percuteur\end{définition}
\begin{définition}\cmn 撞针\end{définition}
\begin{exemple}\jya ɯ-tɤmcar nɯ pɯlthi kɯ-ndo smi sɤ-sthɤβ ŋu\cmn 撞针是夹住火绳点火用的部件\end{exemple}
\begin{relation-sémantique}\confer{
\hyperlink{Ⓔtɤmcar}{\textit{ \papi{tɤmcar}}}
}\end{relation-sémantique}\end{entrée}

\begin{entrée}
\vedette{\hypertarget{Ⓔɯ-tɕhaʁⒽ1}{\papi{ ɯ-tɕhaʁ}}}\markboth{ɯ-tɕhaʁ}{}\homonyme{1}\classe{np}
\begin{définition}\fra fourrage (pour cheval)\end{définition}
\begin{définition}\cmn 马料(没有磨成粉)
\begin{déclaration} \étymologie{\papi{tɕʰag}}\end{déclaration}\end{définition}
\begin{exemple}\jya mbro ɯ-tɕhaʁ tɤ-ta-t-a\cmn 我喂了马\end{exemple}
\begin{relation-sémantique}\confer{
\hyperlink{Ⓔnɯtɕhaʁ}{\textit{ \papi{nɯtɕhaʁ}}}
}\end{relation-sémantique}\end{entrée}

\begin{entrée}
\vedette{\hypertarget{Ⓔɯ-tɕhaʁⒽ2}{\papi{ ɯ-tɕhaʁ}}}\markboth{ɯ-tɕhaʁ}{}\homonyme{2}
\classe{np}
\begin{définition}\fra handicap\end{définition}
\begin{définition}\cmn 残疾\end{définition}
\begin{exemple}\jya a-ku tɤ-mna ri, a-mi ɯ-tɕhaʁ pɯ-ɬoʁ\cmn 我的头愈好,但是脚成了残疾的\end{exemple}
\begin{exemple}\jya ɯ-tɕhaʁ pa-tɕɤt\cmn 他变成成了残疾人\end{exemple}
\begin{relation-sémantique}\confer{
\hyperlink{Ⓔtɕhaʁ}{\textit{ \papi{tɕhaʁ}}}
}\end{relation-sémantique}\end{entrée}

\begin{entrée}
\vedette{\hypertarget{Ⓔɯ-tɕhɤl}{\papi{ ɯ-tɕhɤl}}}\markboth{ɯ-tɕhɤl}{}
\classe{np}
\begin{définition}\fra amende, punition\end{définition}
\begin{définition}\cmn 罚款
\begin{déclaration} \étymologie{\papi{tɕʰad}}\end{déclaration}\end{définition}
\begin{exemple}\jya a-tɕhɤl nɯ-kho-t-a\cmn 我交了罚款\end{exemple}
\begin{relation-sémantique}\confer{
\hyperlink{Ⓔnɯtɕhɤl}{\textit{ \papi{nɯtɕhɤl}}}
}\end{relation-sémantique}
\begin{relation-sémantique}\confer{
\hyperlink{Ⓔtɕhɤtpa}{\textit{ \papi{tɕhɤtpa}}}
}\end{relation-sémantique}\end{entrée}

\begin{entrée}
\vedette{\hypertarget{Ⓔɯ-tɕhɯβ}{\papi{ ɯ-tɕhɯβ}}}\markboth{ɯ-tɕhɯβ}{}\classe{np}\acception{1}
\begin{définition}\fra prendre en compte\end{définition}
\begin{définition}\cmn 考虑到……\end{définition}
\begin{exemple}\jya a-tɕɯ ɯ-tɕhɯβ βze-a ɲɯ-ra\cmn 我要考虑到我儿子的情况\end{exemple}\acception{2}
\begin{définition}\fra afin de\end{définition}
\begin{définition}\cmn 便于\end{définition}
\begin{relation-sémantique}\confer{
\hyperlink{Ⓔnɤxtɕhɯβ}{\textit{ \papi{nɤxtɕhɯβ}}}
}\end{relation-sémantique}\end{entrée}

\begin{entrée}
\vedette{\hypertarget{Ⓔɯ-tɕhɯz}{\papi{ ɯ-tɕhɯz}}}\markboth{ɯ-tɕhɯz}{}\classe{np}
\begin{définition}\fra éternuement\end{définition}
\begin{définition}\cmn 喷嚏\end{définition}
\begin{exemple}\jya ɯ-tɕhɯz to-ɣi\cmn 他打了喷嚏\end{exemple}
\begin{relation-sémantique}\confer{
\hyperlink{Ⓔatɕhɯz}{\textit{ \papi{atɕhɯz}}}
}\end{relation-sémantique}\end{entrée}

\begin{entrée}
\vedette{\hypertarget{Ⓔɯtɕɯn}{\papi{ ɯtɕɯn}}}\markboth{ɯtɕɯn}{}\classe{n}
\begin{définition}\fra énorme\end{définition}
\begin{définition}\cmn 巨大
\begin{déclaration} \étymologie{\papi{tɕʰen}}\end{déclaration}\end{définition}
\end{entrée}

\begin{entrée}
\vedette{\hypertarget{Ⓔɯ-tɕɯtɕu}{\papi{ ɯ-tɕɯtɕu}}}\markboth{ɯ-tɕɯtɕu}{}\classe{np}
\begin{définition}\fra pénis, zizi (enfant)\end{définition}
\begin{définition}\cmn 阴茎,(小孩的)小鸡鸡\end{définition}\end{entrée}

\begin{entrée}
\vedette{\hypertarget{Ⓔɯte}{\papi{ ɯte}}}\markboth{ɯte}{}\classe{adv}
\begin{définition}\fra au bout du compte\end{définition}
\begin{définition}\cmn 本来;归根到底\end{définition}\end{entrée}

\begin{entrée}
\vedette{\hypertarget{Ⓔɯ-thaʁ}{\papi{ ɯ-thaʁ}}}\markboth{ɯ-thaʁ}{}\classe{np}
\begin{définition}\fra verrou en bois\end{définition}
\begin{définition}\cmn 插销\end{définition}
\begin{exemple}\jya ɯ-thaʁ pjɯ́-wɣ-rku ra / pjɯ́-wɣ-lɤt ra\cmn 要插上插销\end{exemple}
\begin{relation-sémantique}\confer{
\hyperlink{Ⓔrɟɤthaʁ}{\textit{ \papi{rɟɤthaʁ}}}
}\end{relation-sémantique}\end{entrée}

\begin{entrée}
\vedette{\hypertarget{Ⓔɯ-tha,ɯ-scoz}{\papi{ ɯ-tha,ɯ-scoz}}}\markboth{ɯ-tha,ɯ-scoz}{}\classe{np}
\begin{définition}\fra culture\end{définition}
\begin{définition}\cmn 文化\end{définition}
\begin{exemple}\jya nɤʑo nɤ-tha nɤ-scoz ɲɯ-rnaʁ\cmn 你文化水平高\end{exemple}\end{entrée}

\begin{entrée}
\vedette{\hypertarget{Ⓔɯ-thɤβ}{\papi{ ɯ-thɤβ}}}\markboth{ɯ-thɤβ}{}\classe{np}
\begin{définition}\fra au milieu\end{définition}
\begin{définition}\cmn 中间\end{définition}
\begin{exemple}\jya aʑo ndʑi-thɤβ tu-βze-a\cmn 我来调解你们之间的纠纷\end{exemple}
\begin{exemple}\jya tɕi-thɤβ kɯ-dɤn me\cmn 我们俩年龄相差不多,我们俩之间没有很远\end{exemple}
\begin{relation-sémantique}\confer{
\hyperlink{Ⓔɯ-pɤrthɤβ}{\textit{ \papi{ɯ-pɤrthɤβ}}}
}\end{relation-sémantique}
\end{entrée}

\begin{entrée}
\vedette{\hypertarget{Ⓔɯ-thɤcu}{\papi{ ɯ-thɤcu}}}\markboth{ɯ-thɤcu}{}\classe{np}
\begin{définition}\fra en aval\end{définition}
\begin{définition}\cmn 在下游\end{définition}\end{entrée}

\begin{entrée}
\vedette{\hypertarget{Ⓔɯ-tho}{\papi{ ɯ-tho}}}\markboth{ɯ-tho}{}\classe{np}
\begin{définition}\fra pédoncule\end{définition}
\begin{définition}\cmn 花梗\end{définition}\end{entrée}

\begin{entrée}
\vedette{\hypertarget{Ⓔɯ-thoβ}{\papi{ ɯ-thoβ}}}\markboth{ɯ-thoβ}{}\classe{np}
\begin{définition}\fra puissance\end{définition}
\begin{définition}\cmn 权力
\begin{déclaration} \étymologie{\papi{tʰob}}\end{déclaration}\end{définition}
\begin{exemple}\jya rɟɤlpu ɣɯ ɯ-thoβ nɯ tɤru ɣɯ sɤznɤ kɯ-wxti ŋu\cmn 国王的权力比头人的大\end{exemple}\end{entrée}

\begin{entrée}
\vedette{\hypertarget{Ⓔɯ-thoʁ}{\papi{ ɯ-thoʁ}}}\markboth{ɯ-thoʁ}{}\classe{np}
\begin{définition}\fra sol\end{définition}
\begin{définition}\cmn 地上
\begin{déclaration} \étymologie{\papi{tʰog (sa.tʰog)}}\end{déclaration}\end{définition}
\end{entrée}

\begin{entrée}
\vedette{\hypertarget{Ⓔɯ-thɯɣli}{\papi{ ɯ-thɯɣli}}}\markboth{ɯ-thɯɣli}{}\classe{np}
\begin{définition}\fra taches (sur le pelage)\end{définition}
\begin{définition}\cmn 花纹(带斑点)\end{définition}
\begin{exemple}\jya mphrɯɣ ɣɯ ɯ-thɯɣli ɲɯ-tʂot\cmn 氆氇的花纹很清晰\end{exemple}
\begin{exemple}\jya kɯrtsɤɣ ɣɯ ɯ-thɯɣli ɲɯ-fkra\cmn 豹子的斑点很清晰\end{exemple}\end{entrée}

\begin{entrée}
\vedette{\hypertarget{Ⓔɯ-thɯm}{\papi{ ɯ-thɯm}}}\markboth{ɯ-thɯm}{}\classe{np}
\begin{définition}\fra bouchon au fond des jarres d'alcool\end{définition}
\begin{définition}\cmn 酒缸底部的塞子\end{définition}
\begin{relation-sémantique}\confer{
\hyperlink{Ⓔtɕhɤrzɤthɯm}{\textit{ \papi{tɕhɤrzɤthɯm}}}
}\end{relation-sémantique}\end{entrée}

\begin{entrée}
\vedette{\hypertarget{Ⓔɯ-tsa}{\papi{ ɯ-tsa}}}\markboth{ɯ-tsa}{}
\classe{np}
\begin{définition}\fra qui convient\end{définition}
\begin{définition}\cmn 适合别人的东西\end{définition}
\begin{exemple}\jya a-mi ɯ-tsa ɲɯ-βze, mɯ́j-wxti, mɯ́j-xtɕi\cmn (那双鞋子)很适合我的脚,不大也不小\end{exemple}
\begin{exemple}\jya ɯ-ŋga ɯ-tsa ʑo ɲɯ-βze\cmn 这件衣服穿着很合身\end{exemple}\end{entrée}

\begin{entrée}
\vedette{\hypertarget{Ⓔɯ-tshɤt}{\papi{ ɯ-tshɤt}}}\markboth{ɯ-tshɤt}{}
\classe{np}\acception{1}
\begin{définition}\fra à la place de\end{définition}
\begin{définition}\cmn 代替\end{définition}
\begin{exemple}\jya ki wɯɟa ki ndʑu ɯ-tshɤt ŋu\cmn 调羹是可以代替筷子的\end{exemple}
\begin{exemple}\jya a-tshɤt ɣɯ-tu-βze ɲɯ-ŋu\cmn 他来替代我\end{exemple}\acception{2}
\begin{définition}\fra tout juste\end{définition}
\begin{définition}\cmn 刚好\end{définition}
\end{entrée}

\begin{entrée}
\vedette{\hypertarget{Ⓔɯ-tshot}{\papi{ ɯ-tshot}}}\markboth{ɯ-tshot}{}
\classe{np}
\begin{définition}\fra qui convient tout juste\end{définition}
\begin{définition}\cmn 刚好合适\end{définition}
\begin{exemple}\jya a-xtsa ɯ-tshot ɲɯ-βze\cmn 我的鞋子刚刚合适\end{exemple}
\begin{exemple}\jya aʑo ɯ-rkoz ɯ-tshot kɤ-ndo-t-a me ri, ɯʑo nɯ ma mɯ-chɯ-ɤnɯrɕo ɕti\cmn 我不是故意带的刚刚好(巧克力粉的数量),自然就用了那么多\end{exemple}
\begin{relation-sémantique}\synonyme{
\hyperlink{Ⓔɯ-tsa}{\textit{ \papi{ɯ-tsa}}}
}\end{relation-sémantique}\end{entrée}

\begin{entrée}
\vedette{\hypertarget{Ⓔɯ-tshɯɣa}{\papi{ ɯ-tshɯɣa}}}\markboth{ɯ-tshɯɣa}{}
\classe{np}\acception{1}
\begin{définition}\fra forme\end{définition}
\begin{définition}\cmn 形状\end{définition}\acception{2}
\begin{définition}\fra méthode\end{définition}
\begin{définition}\cmn 方法\end{définition}
\begin{exemple}\jya kɤ-rɯɕmi kɯnɤ ɯ-tshɯɣa tɕe tu\cmn 说话也是有方法的\end{exemple}
\begin{exemple}\jya kɤ-ti ɯ-tshɯɣa mɤ-naχtɕɯɣ ri, ɯ-tun naχtɕɯɣ\cmn 说法不一样,意思一样\end{exemple}
\begin{exemple}\jya ndʑi-tɯ-rɯndzɤtshi ɯ-tshɯɣa ɲɯ-naχtɕɯɣ (kɯ-rɯndzɤtshi ndʑi-tshɯɣa)\cmn 他们吃饭的样子是一样的\end{exemple}\end{entrée}

\begin{entrée}
\vedette{\hypertarget{Ⓔɯ-tshɯɣrtsa}{\papi{ ɯ-tshɯɣrtsa}}}\markboth{ɯ-tshɯɣrtsa}{}
\classe{np}
\begin{définition}\fra sens, contenu (d'un texte)\end{définition}
\begin{définition}\cmn 内容;意义\end{définition}
\begin{exemple}\jya ki tɯ-rju ɯ-tshɯɣrtsa ɲɯ-rnaʁ\cmn 这句话的含义很深奥\end{exemple}\end{entrée}

\begin{entrée}
\vedette{\hypertarget{Ⓔɯ-tsi}{\papi{ ɯ-tsi}}}\markboth{ɯ-tsi}{}\classe{np}
\begin{définition}\fra moment\end{définition}
\begin{définition}\cmn 时间\end{définition}
\begin{exemple}\jya toʁde ɯ-tsi ʑo qhe kɤ-mto ɲɤ-me\cmn 一瞬间就看不见了\end{exemple}\end{entrée}

\begin{entrée}
\vedette{\hypertarget{Ⓔɯ-tsololot}{\papi{ ɯ-tsololot}}}\markboth{ɯ-tsololot}{}\classe{np}
\begin{définition}\fra pénis, zizi (enfant)\end{définition}
\begin{définition}\cmn 阴茎,(小孩的)小鸡鸡\end{définition}
\begin{relation-sémantique}\synonyme{
\hyperlink{Ⓔɯ-tɕɯtɕu}{\textit{ \papi{ɯ-tɕɯtɕu}}}
}\end{relation-sémantique}\end{entrée}

\begin{entrée}
\vedette{\hypertarget{Ⓔɯ-tsɯ,rnaʁ}{\papi{ ɯ-tsɯ,rnaʁ}}}\markboth{ɯ-tsɯ,rnaʁ}{}
\begin{définition}\fra garder un secret\end{définition}
\begin{définition}\cmn 保守秘密\end{définition}
\begin{exemple}\jya nɤ-tsɯ ɲɯ-rnaʁ\cmn 你把秘密保守好\end{exemple}
\begin{exemple}\jya nɤ-tsɯ a-kɤ-tɯ-ɣɤrnaʁ ra\cmn 你要保守秘密!\end{exemple}
\begin{relation-sémantique}\ComponentA{\classe{np}
 \papi{ɯ-tsɯ}
}\end{relation-sémantique}
\begin{relation-sémantique}\ComponentB{\classe{vs}
\hyperlink{Ⓔrnaʁ}{\textit{ \papi{rnaʁ}}}
}\end{relation-sémantique}
\begin{relation-sémantique}\confer{
\hyperlink{Ⓔnɤtsɯ}{\textit{ \papi{nɤtsɯ}}}
}\end{relation-sémantique}\end{entrée}

\begin{entrée}
\vedette{\hypertarget{Ⓔɯ-tʂɯmpɤri}{\papi{ ɯ-tʂɯmpɤri}}}\markboth{ɯ-tʂɯmpɤri}{}
\classe{n}
\begin{définition}\fra lanière du tablier\end{définition}
\begin{définition}\cmn 围裙的带子\end{définition}
\begin{exemple}\jya ɯ-tʂɯmpɤri ɲɤ-nɯrtɤβ\cmn 他拴了带子\end{exemple}
\begin{exemple}\jya ɯ-tʂɯmpɤri ra ltɕhɤltɕhɤt ʑo pjɤ-nɯ-ɕɯɴqoʁ\cmn 围裙的带子吊着,小巧玲珑的。\end{exemple}
\begin{relation-sémantique}\confer{
\hyperlink{Ⓔtʂɯmpa}{\textit{ \papi{tʂɯmpa}}}
}\end{relation-sémantique}
\begin{relation-sémantique}\confer{
\hyperlink{Ⓔtɤ-ri}{\textit{ \papi{tɤ-ri}}}
}\end{relation-sémantique}\end{entrée}

\begin{entrée}
\vedette{\hypertarget{Ⓔɯ-xso}{\papi{ ɯ-xso}}}\markboth{ɯ-xso}{}
\classe{np}\acception{1}
\begin{définition}\fra vide\end{définition}
\begin{définition}\cmn 空\end{définition}
\begin{exemple}\jya tɤ-fkɯm ɯ-xso\cmn 空口袋\end{exemple}
\begin{exemple}\jya khɯtsa ɯ-xso\cmn 空的碗\end{exemple}
\begin{exemple}\jya nɯ-xso chɤ-nɯ-ɬoʁ-nɯ\cmn 他们空手走出了\end{exemple}\acception{2}
\begin{définition}\fra normal\end{définition}
\begin{définition}\cmn 随便,普通\end{définition}
\begin{exemple}\jya ɯ-xso jɤ-ari-a ɕti\cmn 我是随便去的\end{exemple}
\begin{exemple}\jya a-rʑaβ maʁ, ɯ-xso a-βzaŋsa ɕti\cmn 不是我的妻子,是个普通朋友\end{exemple}\end{entrée}

\begin{entrée}
\vedette{\hypertarget{Ⓔɯ-xtɤfka}{\papi{ ɯ-xtɤfka}}}\markboth{ɯ-xtɤfka}{}\classe{np}
\begin{définition}\fra le ventre rempli\end{définition}
\begin{définition}\cmn 肚子饱\end{définition}
\begin{exemple}\jya a-xtɤfka ʑo tɤ-nɯ-ndza-t-a\cmn 我吃了个饱\end{exemple}
\begin{relation-sémantique}\confer{
\hyperlink{Ⓔtɯ-xtu}{\textit{ \papi{tɯ-xtu}}}
}\end{relation-sémantique}
\begin{relation-sémantique}\confer{
\hyperlink{ⒺfkaⒽ1}{\textit{ \papi{fka}}}
}\end{relation-sémantique}\end{entrée}

\begin{entrée}
\vedette{\hypertarget{Ⓔɯ-χaʁ}{\papi{ ɯ-χaʁ}}}\markboth{ɯ-χaʁ}{}\classe{np}
\begin{définition}\fra malheur\end{définition}
\begin{définition}\cmn 遭殃\end{définition}
\begin{exemple}\jya a-χaʁ ʑo pɯ-ari\cmn 我遭殃了\end{exemple}
\begin{exemple}\jya nɤ-χaʁ ʑo pa-lɤt\cmn 你遭殃了\end{exemple}
\begin{exemple}\jya pɯwɯ nɯ ɯ-χaʁ ʑo pɯ-tɯ-sɤɣri-t (pɯ-tɯ-ta-t)\cmn 你让驴子遭殃了\end{exemple}
\end{entrée}

\begin{entrée}
\vedette{\hypertarget{Ⓔɯ-χcɤl}{\papi{ ɯ-χcɤl}}}\markboth{ɯ-χcɤl}{}
\classe{np}
\begin{définition}\fra milieu\end{définition}
\begin{définition}\cmn 中间
\begin{déclaration} \étymologie{\papi{dkʲil}}\end{déclaration}\end{définition}
\begin{exemple}\jya kɯ-sɤmtshi nɯ kɯ kɯ-rɟaʁ ra tɕhaʁla ɯ-χcɤl ʑo ka-tsɯm\cmn 领舞者把舞蹈队伍带到了坝子中间\end{exemple}\end{entrée}

\begin{entrée}
\vedette{\hypertarget{Ⓔɯ-χpoʁ}{\papi{ ɯ-χpoʁ}}}\markboth{ɯ-χpoʁ}{}
\classe{np}
\begin{définition}\fra chapeau (champignon)\end{définition}
\begin{définition}\cmn 菌盖\end{définition}
\begin{exemple}\jya zdɯmqe nɯnɯra, ɯ-χpoʁ cho ɯ-jɯ nɯra ɯ-grɤl me\cmn 黑银耳那些,看不清楚哪里是盖盖,那里是茎\end{exemple}
\begin{relation-sémantique}\synonyme{
\hyperlink{Ⓔtɤ-fkaβ}{\textit{ \papi{tɤ-fkaβ}}}
}\end{relation-sémantique}\end{entrée}

\begin{entrée}
\vedette{\hypertarget{Ⓔɯ-χsɤr}{\papi{ ɯ-χsɤr}}}\markboth{ɯ-χsɤr}{}
\classe{np}
\begin{définition}\fra calcul\end{définition}
\begin{définition}\cmn 数数\end{définition}
\begin{exemple}\jya tɯrme thɤstɯɣ kɯ-tu nɯ ɯ-χsɤr ko-ndo\cmn 他记下了有几个人\end{exemple}
\begin{relation-sémantique}\synonyme{
\hyperlink{Ⓔtɤ-rtsɯz}{\textit{ \papi{tɤ-rtsɯz}}}
}\end{relation-sémantique}
\begin{relation-sémantique}\confer{
\hyperlink{ⒺχsɤrⒽ1}{\textit{ \papi{χsɤr1}}}
}\end{relation-sémantique}\end{entrée}

\begin{entrée}
\vedette{\hypertarget{Ⓔɯ-χsɤrtoʁ}{\papi{ ɯ-χsɤrtoʁ}}}\markboth{ɯ-χsɤrtoʁ}{}\classe{np}
\begin{définition}\fra sommet pointu, excroissance pointue\end{définition}
\begin{définition}\cmn 尖顶
\begin{déclaration} \étymologie{\papi{gser.tʰog}}\end{déclaration}\end{définition}\end{entrée}

\begin{entrée}
\vedette{\hypertarget{Ⓔɯ-χto}{\papi{ ɯ-χto}}}\markboth{ɯ-χto}{}\classe{np}
\begin{définition}\fra encoche\end{définition}
\begin{définition}\cmn 插口\end{définition}
\begin{exemple}\jya tɤ-jtsi stɤsmɤt komɤl ɯ-kɯ-ndo ɯ-spa pɯ-kɤ-saχaʁ ɯ-χto rmi.\cmn 
柱头两头用来支撑横梁的插口叫\stylefv{ɯ-χto}。
\end{exemple}
\begin{exemple}\jya tɯwɯ cho sarwɯ li ndʑi-χto tu, tɕeri kɯ-ɤβʑɯrdu maʁ, kɤ-kɤ-rkhe ŋu, ndʑu ɯ-phoŋbu kɤ-kɤ-znɯjɯn ŋu\cmn 
纺锤和搓杆也有\stylefv{ɯ-χto},但不是方形,是顺着木条的圆形刻着的
\end{exemple}\end{entrée}

\begin{entrée}
\vedette{\hypertarget{Ⓔɯ-χtɯ}{\papi{ ɯ-χtɯ}}}\markboth{ɯ-χtɯ}{}\classe{np}
\begin{définition}\fra endroit difficile à voir\end{définition}
\begin{définition}\cmn 不容易被发现的地方\end{définition}
\begin{exemple}\jya ki sɤtɕha ɯ-χtɯ ʑo ri ɕti tɕe, tɯrme kɯ-ɕe rkɯn\cmn 这个地方不容易被别人发现,去的人很少\end{exemple}
\begin{relation-sémantique}\confer{
\hyperlink{Ⓔaraχtɯ}{\textit{ \papi{araχtɯ}}}
}\end{relation-sémantique}\end{entrée}

\begin{entrée}
\vedette{\hypertarget{Ⓔɯ-χtɯkrɤl}{\papi{ ɯ-χtɯkrɤl}}}\markboth{ɯ-χtɯkrɤl}{}\classe{np}
\begin{définition}\fra comme les autres\end{définition}
\begin{définition}\cmn 跟其他人一样\end{définition}
\begin{exemple}\jya nɤ-χtɯkrɤl nɤ-mɤ-tɯ-fse nɯ!\cmn 你真的不正常!\end{exemple}
\begin{relation-sémantique}\synonyme{
\hyperlink{Ⓔɯ-χtɯrca}{\textit{ \papi{ɯ-χtɯrca}}}
}\end{relation-sémantique}\end{entrée}

\begin{entrée}
\vedette{\hypertarget{Ⓔɯ-χtɯrca}{\papi{ ɯ-χtɯrca}}}\markboth{ɯ-χtɯrca}{}\classe{adv}
\begin{définition}\fra comme les autres, avec les autres\end{définition}
\begin{définition}\cmn 和别人一样,和别人一起\end{définition}
\begin{exemple}\jya nɤ-χtɯrca tɤ-fse\cmn 你要跟其他人一样\end{exemple}
\begin{exemple}\jya ɯ-χtɯrca mɤ-fse\cmn 他跟正常人不一样\end{exemple}
\begin{exemple}\jya ɯ-χtɯrca mɤ-kɯ-ɤri\cmn 不合群的人\end{exemple}
\begin{relation-sémantique}\synonyme{
\hyperlink{Ⓔɯ-χtɯkrɤl}{\textit{ \papi{ɯ-χtɯkrɤl}}}
}\end{relation-sémantique}\end{entrée}

\begin{entrée}
\vedette{\hypertarget{Ⓔɯ-zarzɯr}{\papi{ ɯ-zarzɯr}}}\markboth{ɯ-zarzɯr}{}
\classe{np}
\begin{définition}\ 
\begin{déclaration}\grammar{n.rdpl}\end{déclaration}\end{définition}
\begin{définition}\fra environs\end{définition}
\begin{définition}\cmn 边缘\end{définition}
\begin{exemple}\jya tɯji ɯ-zarzɯr ra sɯjno dɤn\cmn 地边杂草多\end{exemple}
\begin{relation-sémantique}\synonyme{
\hyperlink{Ⓔɯ-rkarkɯ}{\textit{ \papi{ɯ-rkarkɯ}}}
}\end{relation-sémantique}\end{entrée}

\begin{entrée}
\vedette{\hypertarget{Ⓔɯ-zbroŋ}{\papi{ ɯ-zbroŋ}}}\markboth{ɯ-zbroŋ}{}\classe{np}
\begin{définition}\fra dessin sur le bord des pains\end{définition}
\begin{définition}\cmn 馍馍边缘的花纹\end{définition}
\begin{exemple}\jya qajɣi ɯ-zbroŋ kɯ-tu nɯ mpɕɤr\cmn 有花纹的馍馍好看\end{exemple}\end{entrée}

\begin{entrée}
\vedette{\hypertarget{Ⓔɯ-zdɤrca}{\papi{ ɯ-zdɤrca}}}\markboth{ɯ-zdɤrca}{}\classe{adv}
\begin{définition}\fra avec les autres\end{définition}
\begin{définition}\cmn 和别人\end{définition}
\begin{exemple}\jya ɯ-zdɤrca mɤ-kɯ-ɤri\cmn 不合群的人\end{exemple}
\begin{exemple}\jya nɤ-zdɤrca jɤ-ɣi!\cmn 你跟同伴一起来吧\end{exemple}
\begin{relation-sémantique}\synonyme{
\hyperlink{Ⓔɯ-χtɯrca}{\textit{ \papi{ɯ-χtɯrca}}}
}\end{relation-sémantique}
\begin{relation-sémantique}\synonyme{
\hyperlink{Ⓔɯ-χtɯkrɤl}{\textit{ \papi{ɯ-χtɯkrɤl}}}
}\end{relation-sémantique}
\begin{relation-sémantique}\confer{
 \papi{ɯ-zda}
}\end{relation-sémantique}
\begin{relation-sémantique}\confer{
\hyperlink{Ⓔtɤ-rca}{\textit{ \papi{tɤ-rca}}}
}\end{relation-sémantique}\end{entrée}

\begin{entrée}
\vedette{\hypertarget{Ⓔɯ-zgɯr}{\papi{ ɯ-zgɯr}}}\markboth{ɯ-zgɯr}{}
\classe{np}
\begin{définition}\fra partie recourbée\end{définition}
\begin{définition}\cmn 卷起来的部分\end{définition}
\begin{exemple}\jya tsuku tɯrme ra ɣɯ nɯ-mgɯr mɤ-kɯ-ɤstu tɕe, kɯ-ɤzgrɯ kɯ-fse, nɯ ɯ-zgɯr ɣɤʑu tu-kɯ-ti ŋgrɤl\cmn 有些人背部不直,就可以说他有驼背\end{exemple}
\begin{relation-sémantique}\confer{
\hyperlink{Ⓔazgɯr}{\textit{ \papi{azgɯr}}}
}\end{relation-sémantique}\end{entrée}

\begin{entrée}
\vedette{\hypertarget{Ⓔɯ-zɯr}{\papi{ ɯ-zɯr}}}\markboth{ɯ-zɯr}{}\classe{np}
\begin{définition}\fra côté\end{définition}
\begin{définition}\cmn 旁边\end{définition}
\begin{exemple}\jya ndʑu nɯ khɯtsa ɯ-zɯr nɯ-te\cmn 把筷子放在碗的旁边\end{exemple}
\begin{relation-sémantique}\synonyme{
\hyperlink{Ⓔɯ-βzɯr}{\textit{ \papi{ɯ-βzɯr}}}
}\end{relation-sémantique}\begin{sous-entrée}
\vedette{\hypertarget{}{\papi{ ɯ-zɯrɕɯzɯr}}}\markboth{ɯ-zɯrɕɯzɯr}{}\classe{np}
\begin{définition}\fra le plus au bord\end{définition}
\begin{définition}\cmn 最边缘\end{définition}
\end{sous-entrée}\end{entrée}

\begin{entrée}
\vedette{\hypertarget{Ⓔɯ-ʑat}{\papi{ ɯ-ʑat}}}\markboth{ɯ-ʑat}{}\classe{np}
\begin{définition}\fra caractère propre\end{définition}
\begin{définition}\cmn 本性;自己的性格\end{définition}
\begin{exemple}\jya ɯʑo tɤtɕɯpɯ ɲɯ-ɕti tɕe, ɯ-ʑat ci ɣɤʑu\cmn 他既然是男孩子,调皮一点是自然的\end{exemple}
\begin{exemple}\jya tɤjpa ko-lɤt tɕe ɯ-ʑat ci ɣɤʑu, tɕe tɯ-ŋga pjɯ-jaʁ ɲɯ-ra\cmn 下了雪肯定会冷一点,要多穿一点衣服\end{exemple}
\begin{exemple}\jya ftɕar jɤ-kɯ-zɣɯt nɯ ɯ-ʑat ci ɣɤʑu nɤ ma ɲɯ-sɤɕke\cmn 天气热一点,那是因为春天到了的缘故\end{exemple}
\begin{relation-sémantique}\confer{
\hyperlink{Ⓔaɣɯʑɯʑat}{\textit{ \papi{aɣɯʑɯʑat}}}
}\end{relation-sémantique}\end{entrée}

\begin{entrée}
\vedette{\hypertarget{Ⓔɯ-ʑɤŋɤn}{\papi{ ɯ-ʑɤŋɤn}}}\markboth{ɯ-ʑɤŋɤn}{}\classe{np}
\begin{définition}\fra pour se venger de ...\end{définition}
\begin{définition}\cmn 为了报复……
\begin{déclaration} \étymologie{\papi{ʑe.ŋan}}\end{déclaration}\end{définition}
\begin{exemple}\jya nɤʑo taʁndo mɯ́j-tɯ-tso, nɤ-ʑɤŋɤn kɯ-ɴqa ɲɯ-ta-znɤma ŋu\cmn 因为你不听话,为了报复你,就让你干重货\end{exemple}\end{entrée}

\begin{entrée}
\vedette{\hypertarget{Ⓔɯ-ʑɤrʑɯr}{\papi{ ɯ-ʑɤrʑɯr}}}\markboth{ɯ-ʑɤrʑɯr}{}\classe{np}
\begin{définition}\fra faire en même temps\end{définition}
\begin{définition}\cmn 一边……一边\end{définition}
\begin{exemple}\jya aʑo pjɯ-ta-sɯxɕɤt ɯ-ʑɤrʑɯr lu-taʁ-a ŋu\cmn 我一边教你,一边织衣服。\end{exemple}\end{entrée}

\begin{entrée}
\vedette{\hypertarget{Ⓔɯʑo}{\papi{ ɯʑo}}}\markboth{ɯʑo}{}\classe{pro}
\begin{définition}\fra lui\end{définition}
\begin{définition}\cmn 他\end{définition}
\begin{exemple}\jya ɯʑo ʑo\cmn 他自己\end{exemple}
\end{entrée}

\begin{entrée}
\vedette{\hypertarget{Ⓔɯʑoz}{\papi{ ɯʑoz}}}\markboth{ɯʑoz}{}\classe{adv}
\begin{définition}\fra à part\end{définition}
\begin{définition}\cmn 另外\end{définition}\end{entrée}

\begin{entrée}
\vedette{\hypertarget{Ⓔɯʑɯʑur}{\papi{ ɯʑɯʑur}}}\markboth{ɯʑɯʑur}{}\classe{adv}\acception{1}
\begin{définition}\fra au moment de\end{définition}
\begin{définition}\cmn ……的时候\end{définition}\acception{2}
\begin{définition}\fra en même temps que\end{définition}
\begin{définition}\cmn 一边……一边\end{définition}
\end{entrée}

\newpage\caractère{w}

\begin{entrée}
\vedette{\hypertarget{Ⓔwaji}{\papi{ waji}}}\markboth{waji}{}\classe{n}
\begin{définition}\fra petit de yak\end{définition}
\begin{définition}\cmn 牦牛犊\end{définition}
\end{entrée}

\begin{entrée}
\vedette{\hypertarget{Ⓔwaɟɯ}{\papi{ waɟɯ}}}\markboth{waɟɯ}{}
\classe{n}
\begin{définition}\fra tremblement de terre\end{définition}
\begin{définition}\cmn 地震\end{définition}
\begin{exemple}\jya waɟɯ to-βzu\cmn 发生了地震\end{exemple}
\begin{exemple}\jya waɟɯ ɲɤ-nmu\cmn 发生了地震\end{exemple}\end{entrée}

\begin{entrée}
\vedette{\hypertarget{Ⓔwaŋtshaŋ}{\papi{ waŋtshaŋ}}}\markboth{waŋtshaŋ}{}\classe{n}
\begin{définition}\fra armoire\end{définition}
\begin{définition}\cmn 柜子\end{définition}
\end{entrée}

\begin{entrée}
\vedette{\hypertarget{Ⓔwɤrwɤr}{\papi{ wɤrwɤr}}}\markboth{wɤrwɤr}{}
\begin{relation-sémantique}\confer{
\hyperlink{Ⓔhwɤrhwɤr}{\textit{ \papi{hwɤrhwɤr}}}
}\end{relation-sémantique}
\end{entrée}

\begin{entrée}
\vedette{\hypertarget{Ⓔwudɯŋ}{\papi{ wudɯŋ}}}\markboth{wudɯŋ}{}\classe{n}
\begin{définition}\fra petite jarre\end{définition}
\begin{définition}\cmn 小坛子(大小像瓶子一样)\end{définition}\end{entrée}

\begin{entrée}
\vedette{\hypertarget{Ⓔwudzɯdzi}{\papi{ wudzɯdzi}}}\markboth{wudzɯdzi}{}\classe{intj}
\begin{définition}\fra exprime la peur\end{définition}
\begin{définition}\cmn 表示害怕(看到怪物的时候)\end{définition}\end{entrée}

\begin{entrée}
\vedette{\hypertarget{Ⓔwɣrum}{\papi{ wɣrum}}}\markboth{wɣrum}{}
\classe{vs}
\paradigme{\textit{dir :} \jya tɤ-}
\paradigme{\textit{dir :} \jya thɯ-}
\begin{définition}\fra blanc\end{définition}
\begin{définition}\cmn 白\end{définition}\begin{sous-entrée}
\vedette{\hypertarget{}{\papi{ sɯɣrum}}}\markboth{sɯɣrum}{}\classe{vt}
\begin{définition}\ 
\begin{déclaration}\grammar{caus}\end{déclaration}\end{définition}
\begin{définition}\fra blanchir, rendre blanc\end{définition}
\begin{définition}\cmn 使其变白\end{définition}
\begin{relation-sémantique}\confer{
\hyperlink{Ⓔaɣrɤɣrum}{\textit{ \papi{aɣrɤɣrum}}}
}\end{relation-sémantique}
\end{sous-entrée}\end{entrée}

\begin{entrée}
\vedette{\hypertarget{Ⓔwulaʁ}{\papi{ wulaʁ}}}\markboth{wulaʁ}{}\classe{n}
\begin{définition}\fra corvée\end{définition}
\begin{définition}\cmn 乌拉,徭役
\begin{déclaration} \étymologie{\papi{ɦu.lag}}\end{déclaration}\end{définition}
\end{entrée}

\begin{entrée}
\vedette{\hypertarget{Ⓔwum}{\papi{ wum}}}\markboth{wum}{}\classe{vt}
\paradigme{\textit{dir :} \jya \_}
\begin{définition}\fra fermer (sac, parapluie)\end{définition}
\begin{définition}\cmn 收;收紧;合拢\end{définition}
\begin{exemple}\jya san pɯ-wum-a\cmn 我收了伞\end{exemple}
\begin{exemple}\jya lʁa ɯ-mŋu kɤ-wum-a\cmn 我收紧了口子\end{exemple}
\begin{exemple}\jya lʁa ɯ-ŋgɯ laχtɕha thɯ-wum-a\cmn 我把东西装在口袋里了\end{exemple}
\begin{relation-sémantique}\confer{
\hyperlink{Ⓔawɯwum}{\textit{ \papi{awɯwum}}}
}\end{relation-sémantique}\begin{sous-entrée}
\vedette{\hypertarget{}{\papi{ ʑɣɤwum}}}\markboth{ʑɣɤwum}{}\classe{ps}
\begin{définition}\ 
\begin{déclaration}\grammar{refl}\end{déclaration}\end{définition}
\begin{définition}\fra se refermer\end{définition}
\begin{définition}\cmn 自己合拢;自然合拢\end{définition}
\end{sous-entrée}\end{entrée}

\begin{entrée}
\vedette{\hypertarget{Ⓔwuma}{\papi{ wuma}}}\markboth{wuma}{}
\classe{adv}\acception{1}
\begin{définition}\fra très\end{définition}
\begin{définition}\cmn 很;非常\end{définition}
\begin{exemple}\jya wuma ʑo rɤɣo ra pjɤ-mpɕɤr ɲɯ-ŋu\cmn 歌非常好听\end{exemple}\acception{2}
\begin{définition}\fra vrai\end{définition}
\begin{définition}\cmn 真正的
\begin{déclaration} \étymologie{\papi{ŋo.ma}}\end{déclaration}\end{définition}
\begin{exemple}\jya ɬɤndʐi wuma nɯ nɤʑo ɲɯ-tɯ-ŋu\cmn 你才是真正的魔鬼\end{exemple}
\begin{sous-entrée}
\vedette{\hypertarget{}{\papi{ wuma tɤŋu tɕe}}}\markboth{wuma tɤŋu tɕe}{}
\begin{définition}\fra en fait, à vrai dire\end{définition}
\begin{définition}\cmn 说实话;实际上\end{définition}
\begin{exemple}\jya wuma tɤŋu tɕe, aʑo kɤ-ndza a-ʁjiz mɯ́j-ɣi.\cmn 说实话,我不像吃东西\end{exemple}
\end{sous-entrée}\end{entrée}

\begin{entrée}
\vedette{\hypertarget{Ⓔwuŋgru}{\papi{ wuŋgru}}}\markboth{wuŋgru}{}\classe{n}
\begin{définition}\fra vesce\end{définition}
\begin{définition}\cmn 野豌豆\end{définition}
\begin{exemple}\jya wuŋgru nɯ sɯjno kɯ-mbɤr ŋu, ɯ-tshɯɣa nɯ staχpɯ fse, tɯ-ji ɯ-ŋgɯ tɯ-ji ɯ-rkɯ arɤndɯndɤt ʑo tu-ɬoʁ ɕti, ɯ-mɯntoʁ kɯ-ɤɣɯrnɯɕɯr ŋu, ɯ-ru nɯ kɯ-ɤβʑɯrdu ŋu, wuma ʑo nɤrko, fsapaʁndza sna, tɯrme kɤ-ndza mɤ-sna.\cmn 野豌豆是矮小的植物,样子像豌豆,地里和地边到处都可以生长,花淡红色,茎四方形、非常坚实。可以喂牲畜,人不能吃。\end{exemple}
\end{entrée}

\begin{entrée}
\vedette{\hypertarget{Ⓔwo}{\papi{ wo}}}\markboth{wo}{}\classe{part}
\begin{définition}\fra impératif intensifié\end{définition}
\begin{définition}\cmn 加强命令式的语气\end{définition}
\begin{exemple}\jya kɯm nɯ-thɯ-pe wo\cmn 你把门关上嘛\end{exemple}
\end{entrée}

\begin{entrée}
\vedette{\hypertarget{Ⓔwoɬaʁ}{\papi{ woɬaʁ}}}\markboth{woɬaʁ}{}\classe{n}
\begin{définition}\fra marâtre\end{définition}
\begin{définition}\cmn 继母\end{définition}
\end{entrée}

\begin{entrée}
\vedette{\hypertarget{Ⓔworɟɤmchɤn}{\papi{ worɟɤmchɤn}}}\markboth{worɟɤmchɤn}{}\classe{intj}
\begin{définition}\fra exprime l'étonnement\end{définition}
\begin{définition}\cmn 表示自己感触很深,惊叹
\begin{déclaration} \étymologie{\papi{ʔo.rgʲan.mkʰʲen}}\end{déclaration}\end{définition}
\begin{exemple}\jya worɟɤmchɤn, jisŋi tɯ-mɯ ɯ-tɯ-dɤn\cmn 天啊,今天雨怎么这么多!\end{exemple}
\begin{exemple}\jya worɟɤmchɤn ɯ-tɯ-mbro nɯ!\cmn 天啊,你这么高!\end{exemple}
\begin{exemple}\jya worɟɤmchɤn ɯ-tɯ-nɯɲɤmkhe nɯ!\cmn 天啊,你这么瘦!\end{exemple}\end{entrée}

\begin{entrée}
\vedette{\hypertarget{Ⓔwortɕhi}{\papi{ wortɕhi}}}\markboth{wortɕhi}{}\classe{intj}
\begin{définition}\fra je vous en prie\end{définition}
\begin{définition}\cmn 求您了
\begin{déclaration} \étymologie{\papi{ɦor.tɕʰe}}\end{déclaration}\end{définition}
\begin{exemple}\jya wortɕhi tu-kɯ-qur-a-nɯ ɲɯ-ntshi\cmn 求你们帮一下我\end{exemple}
\end{entrée}

\begin{entrée}
\vedette{\hypertarget{Ⓔwortɕhi wojɤr}{\papi{ wortɕhi wojɤr}}}\markboth{wortɕhi wojɤr}{}\classe{intj}
\begin{définition}\fra je vous en prie\end{définition}
\begin{définition}\cmn 求您了
\begin{déclaration} \étymologie{\papi{ɦor.tɕʰe}}\end{déclaration}\end{définition}
\end{entrée}

\begin{entrée}
\vedette{\hypertarget{Ⓔwutɕhɯtɕhɯ}{\papi{ wutɕhɯtɕhɯ}}}\markboth{wutɕhɯtɕhɯ}{}\classe{intj}
\begin{définition}\fra exprime que le locuteur a froid\end{définition}
\begin{définition}\cmn 表示冷\end{définition}\end{entrée}

\begin{entrée}
\vedette{\hypertarget{Ⓔwɯdɯŋ}{\papi{ wɯdɯŋ}}}\markboth{wɯdɯŋ}{}\classe{n}
\begin{définition}\fra petite jarre d'alcool\end{définition}
\begin{définition}\cmn 小酒坛\end{définition}\end{entrée}

\begin{entrée}
\vedette{\hypertarget{Ⓔwɯɟa}{\papi{ wɯɟa}}}\markboth{wɯɟa}{}\classe{n}
\begin{définition}\fra cuillère\end{définition}
\begin{définition}\cmn 调羹\end{définition}
\end{entrée}

\begin{entrée}
\vedette{\hypertarget{Ⓔwɯrna}{\papi{ wɯrna}}}\markboth{wɯrna}{}\classe{n}
\begin{définition}\fra poids du fuseau\end{définition}
\begin{définition}\cmn 纺锤的秤砣\end{définition}
\end{entrée}

\begin{entrée}
\vedette{\hypertarget{Ⓔwɯwɯ}{\papi{ wɯwɯ}}}\markboth{wɯwɯ}{}\classe{n}
\begin{définition}\fra bolet\end{définition}
\begin{définition}\cmn 牛肝菌\end{définition}
\begin{exemple}\jya wɯwɯ jmɤɣ nɯ ɕkrɤz, tɯrgi, sɤjku ɯ-ŋgɯ ra tu-ɬoʁ ŋu, ɯ-mgɯrqhu nɯ kɯ-qandʐɯlu tɕe kɯ-nɤmbju ŋu, ɯ-rʑɯɣ me kú-wɣ-nɤmɯma tɕe kɯ-mpɯ-mpɯ ʑo ŋu, ɯ-mdoʁ nɯ ɯ-pa ɯ-pɕoʁ kɯ-wɣrum tu, kɯ-qarŋe tu, ɯ-pa kɯ-wɣrum nɯ kɤ-ndza sna, ɯ-pa kɯ-qarŋe nɯ kɤ-ndza mɤ-sna, ɯ-ru nɯ pjɯ́-wɣ-qlɯt tɕe mɤ-ndoʁ kɤ-saʁ khɯ\cmn 牛肝菌长在青冈树林、杉木林和白桦树林里,背面是乌色的,光滑,没有菌褶,摸起来很软。它下部有的是白色,有的是黄色的,白色的那些可以吃,黄色不可以吃。菌干不脆,不能折,只能撕。\end{exemple}
\end{entrée}

\begin{entrée}
\vedette{\hypertarget{Ⓔwxti}{\papi{ wxti}}}\markboth{wxti}{}\classe{vs}
\paradigme{\textit{dir :} \jya tɤ-}
\paradigme{\textit{dir :} \jya thɯ-}
\begin{définition}\fra grand\end{définition}
\begin{définition}\cmn 大\end{définition}
\begin{exemple}\jya tɤ-pɤtso chɤ-wxti\cmn 小孩子大了\end{exemple}
\begin{exemple}\jya tɯ-ci chɤ-wxti\cmn 水涨了\end{exemple}
\begin{exemple}\jya @dianhua ɯ-skɤt ci ci tu-wxti, ci ci ɲɯ-xtɕi ɲɯ-ŋu tɕe, koŋla mɯ́j-sɤmtshɤm\cmn 电话的声音一会大,一会小,听得不完整\end{exemple}
\begin{relation-sémantique}\confer{
\hyperlink{Ⓔrɯxtuxti}{\textit{ \papi{rɯxtuxti}}}
}\end{relation-sémantique}\begin{sous-entrée}
\vedette{\hypertarget{}{\papi{ ɣɤwxti}}}\markboth{ɣɤwxti}{}\classe{vt}
\paradigme{\textit{dir :} \jya tɤ-}
\begin{définition}\fra rendre grand\end{définition}
\begin{définition}\cmn 弄大\end{définition}
\begin{exemple}\jya tɤ-ɣɤwxti-t-a\cmn 我弄大了\end{exemple}
\begin{relation-sémantique}\antonyme{
\hyperlink{Ⓔxtɕi}{\textit{ \papi{xtɕi}}}
}\end{relation-sémantique}
\end{sous-entrée}\end{entrée}

\newpage\caractère{x}

\begin{entrée}
\vedette{\hypertarget{Ⓔxɤlxɤl}{\papi{ xɤlxɤl}}}\markboth{xɤlxɤl}{}\classe{idph.2}
\begin{définition}\fra qui porte des habits pas trop serrés\end{définition}
\begin{définition}\cmn 形容衣服穿得宽松的样子\end{définition}\begin{sous-entrée}
\vedette{\hypertarget{}{\papi{ xɤlnɤxɤl}}}\markboth{xɤlnɤxɤl}{}
\begin{exemple}\jya tɯ-ŋga kɯ-ɤŋgɤjom tsa ɲɯ́-wɣ-nɯ-βzu tɕe, tɤ-kɯ-ŋke tɕe xɤlnɤxɤl pa tɕe sɤscit\cmn 衣服宽一点的话,穿起来走路方便,舒服\end{exemple}
\end{sous-entrée}\end{entrée}

\begin{entrée}
\vedette{\hypertarget{Ⓔxɤtxɤt}{\papi{ xɤtxɤt}}}\markboth{xɤtxɤt}{}\classe{idph.2}
\begin{définition}\fra énorme et somptueux\end{définition}
\begin{définition}\cmn 形容又宽大又豪华的样子(房子、建筑物)\end{définition}
\begin{exemple}\jya kha xɤtxɤt ʑo ɲɯ-pa\cmn 房子又宽大又豪华\end{exemple}\end{entrée}

\begin{entrée}
\vedette{\hypertarget{Ⓔxcaŋxcaŋ}{\papi{ xcaŋxcaŋ}}}\markboth{xcaŋxcaŋ}{}
\classe{idph.2}
\begin{définition}\fra gros et plat\end{définition}
\begin{définition}\cmn 又大又扁状\end{définition}
\begin{exemple}\jya ɯ-xtsa ɲɯ-nɤwxti-a tɕe xcaŋxcaŋ mɯ́j-nɯɣɯŋga\cmn 他的鞋子我穿太大了,不合适\end{exemple}\begin{sous-entrée}
\vedette{\hypertarget{}{\papi{ xcaŋnɤxcaŋ}}}\markboth{xcaŋnɤxcaŋ}{}\classe{idph.3}
\begin{exemple}\jya xcaŋnɤxcaŋ ʑo ɲɯ-nɤŋkɯŋke\cmn 他穿着太大的鞋子走路\end{exemple}
\end{sous-entrée}\end{entrée}

\begin{entrée}
\vedette{\hypertarget{Ⓔxcat}{\papi{ xcat}}}\markboth{xcat}{}
\classe{vs}
\paradigme{\textit{dir :} \jya nɯ-}
\paradigme{\textit{dir :} \jya tɤ-}
\begin{définition}\fra nombreux\end{définition}
\begin{définition}\cmn 有很多\end{définition}
\begin{exemple}\jya tɯrme xcat\cmn 人很多\end{exemple}
\begin{exemple}\jya a-ŋga xcat ʑo ɕti\cmn 我有很多衣服\end{exemple}
\begin{relation-sémantique}\synonyme{
\hyperlink{Ⓔdɤn}{\textit{ \papi{dɤn}}}
}\end{relation-sémantique}\begin{sous-entrée}
\vedette{\hypertarget{}{\papi{ sɯxcat}}}\markboth{sɯxcat}{}\classe{vt}
\begin{définition}\ 
\begin{déclaration}\grammar{caus}\end{déclaration}\end{définition}
\begin{exemple}\jya tɯ-mɯ qale kɯ ɲɯ-ɤsɯ-sɯxcat ʑo\cmn 下着狂风暴雨\end{exemple}
\end{sous-entrée}\end{entrée}

\begin{entrée}
\vedette{\hypertarget{Ⓔxcɤxcɤt}{\papi{ xcɤxcɤt}}}\markboth{xcɤxcɤt}{}\classe{idph.2}
\begin{définition}\fra petit et vif, mignon\end{définition}
\begin{définition}\cmn 形容小巧玲珑的样子\end{définition}
\begin{exemple}\jya tɤ-pɤtso ɯ-rŋa ra xcɤxcɤt ʑo pa tɕe ɲɯ-sɤjndɤt\cmn 小孩子的脸又小又可爱\end{exemple}\end{entrée}

\begin{entrée}
\vedette{\hypertarget{Ⓔxchɯxcho}{\papi{ xchɯxcho}}}\markboth{xchɯxcho}{}\classe{idph.2}
\begin{définition}\fra creux\end{définition}
\begin{définition}\cmn 形容空心的样子\end{définition}\end{entrée}

\begin{entrée}
\vedette{\hypertarget{Ⓔxchuxchu}{\papi{ xchuxchu}}}\markboth{xchuxchu}{} (\variante{xcuxcu}) \classe{idph.2}
\begin{définition}\fra épais et résistant (pétales d'une fleur)\end{définition}
\begin{définition}\cmn 形容花等厚而结实的样子\end{définition}
\begin{exemple}\jya ɯ-mɯntoʁ nɯ xchuxchu ʑo ɲɯ-pa\cmn 它的花显得又厚又结实\end{exemple}
\begin{exemple}\jya tɤ-pɤtso kɯ ɯ-mtɕhi xchuxchu ʑo to-stu\cmn 小孩子嘟着嘴巴\end{exemple}\end{entrée}

\begin{entrée}
\vedette{\hypertarget{ⒺxɕajⒽ1}{\papi{ xɕaj}}}\markboth{xɕaj}{}\homonyme{1}
\classe{n}
\begin{définition}\fra herbe\end{définition}
\begin{définition}\cmn 一种草\end{définition}
\begin{exemple}\jya xɕaj ʁnɯ-tɯphu tu tɕe, si ci tu, sɯjno ci tu. sɯjno nɯ ɯ-jwaʁ nɯ kɯ-rɲɟi kɯ-tɕɤr lu-kɯ-ɤmtɕoʁ ŋu. ɯ-ru nɯ ɯ-rtsɤɣ lu-oʑɯrja ŋu, ɯ-rtsɤɣ nɯtɕu ɯ-jwaʁ ntsɯ ku-ndzoʁ ŋu. ɯ-jwaʁ ɯ-qa nɯtɕu, ɯ-ru nɯ ku-mphɯr kɯ-fse ŋu, tɕe ɯ-ru lɤ-zri tɕe ɯ-lɤcu tɕe, li ɯ-rtsɤɣ ɲɯ-βze tɕe, li ɯ-jwaʁ lu-ɬoʁ ŋu. xɕaj nɯ tɯ-phɯ ɯ-ŋgɯ, lɤŋɤtʂɤ-ldʑa ɲɯ-βze cha, ɯ-kɤχcɤl nɯ tɕu ɯ-mat kɯ-ɕnom kɯ-fse lu-βze ŋu. sɯjno xɕaj nɯ kɤntɕhɯ-tɯphɯ tu, pɣɤtɕɯxɕaj kɤ-ti tu, pɤŋɤxɕaj kɤ-ti tu, cɤmi xɕaj kɤ-ti tu, xsɤrɯ kɤ-ti tu, pɣɤjmɤt kɤ-ti tu, tɯ-ci xɕaj kɤ-ti tu, tɕeri nɯ-tshɯɣa ra kɯ-naχtɕɯɣ tsa ɕti, nɯ ɯ-ŋgɯ zɯ cɤmi xɕaj stu ʑo wxti, cɤmi tsa tu-ɬoʁ ŋu, tɯ-ci xɕaj kɯ-mbɯ-mbɤr ma me, tɕhɯtoʁ ku tu-ɬoʁ ŋu, ɯ-ro nɯ ra zgoku tsa tu-ɬoʁ ŋu. nɯŋa cho jla kɤ-mbi wuma ʑo pe.\cmn 
\stylefv{xɕaj}有两种,一种是树,另一种是草。草的那种叶子细长,上面尖。茎由许多节组成,每一节上有长有叶子。叶子的根部好像是裹着茎长出来的,叶子裹着的茎长高了又长出节来,又长叶子。每一秆的可以分五、六根,在顶上抽穗结果。这种草有很多种,有\stylefv{pɣɤtɕɯxɕaj}、\stylefv{pɤŋɤxɕaj}、\stylefv{cɤmi xɕaj}、\stylefv{xsɤrɯ}、\stylefv{pɣɤjmɤt}、\stylefv{tɯ-ci xɕaj}六种,但它们的形状都差不多相同。其中\stylefv{cɤmi xɕaj}最大,生长在河坝,\stylefv{tɯ-ci xɕaj}很矮,长在水草地上,其它的都生长在高山上。是喂奶牛和犏牛的好饲料。
\end{exemple}
\end{entrée}

\begin{entrée}
\vedette{\hypertarget{ⒺxɕajⒽ2}{\papi{ xɕaj}}}\markboth{xɕaj}{}\homonyme{2}
\classe{n}
\begin{définition}\fra une espèce d'arbre\end{définition}
\begin{définition}\cmn 乔木的一种\end{définition}
\begin{exemple}\jya xɕaj ʁnɯ-tɯphu tu tɕe, si ci tu, sɯjno ci tu. si nɯ kɯ-mbɯ-mbro ci ŋu, ɯ-jpum tsa ɲɯ-βze cha, tɕe zgo kɯ-mbɤr tsa zɯ tu-ɬoʁ ŋu, ɯ-ru ɯ-pɕi nɯ kɯ-pɣi ŋu, ɯ-rtaʁ dɤn, ɯ-jwaʁ ʑmbri ɯ-jwaʁ tsa fse, ɯ-mɯntoʁ kɯ-ndɯ-ndɯβ ɕti khro mɤ-χsɤl, ɯ-mat nɯ ɯ-cɤβ chɯ-βze ŋu, tɕeri ɯ-rɣi nɯ pɯ-ŋgra tɕe tu-ɬoʁ mɤ-cha. ɯʑo ɯ-kɯ-sɯ-mphɯl nɯ, ɯ-zrɤm ɕti. ɯ-si ngɯt tɕe, laʁdɯn ɯ-jɯ kɤ-βzu pe. kɤ-nɯ-βlɯ kɯnɤ wuma ʑo pe ma thɯ́-wɣ-nɯ-βlɯ tɕe ɯ-khɯ mɤ-wxti. ɯ-smi sɤɕke. si ɯ-xɕaj nɯ li ʁnɯ-tɯphu tu. tɕe xɕaj ɲaʁ kɤ-ti ci tu, xɕaj wɣrum kɤ-ti ci tu. xɕaj ɯ-rqhu nɯ fsapaʁ ɯ-ɕɤrɯ tɤ-mtshɤz kɤ-kɯ-ndo ɣɯ ɯ-smɤn ɲɯ-ŋu khi.\cmn 
{xɕaj}有两种,一种是树,另一种是草。树的那种长得很高、比较粗,生长在下半山。树皮呈灰色,枝桠多。叶子类似柳树的叶子。花很小,看不清楚。结荚果,但种子掉了以后不能发芽,繁殖靠根。因为木质结实,可以作各种农具的把子。也是烧火的好柴,因为烧起来发出很少烟,火气又高。这种树也分成两种,一种叫\stylefv{xɕaj ɲaʁ},另一种叫\stylefv{xɕajɣrum} 。\stylefv{xɕaj}的树皮是牲畜骨质增生病的良药。
\end{exemple}\end{entrée}

\begin{entrée}
\vedette{\hypertarget{Ⓔxɕajɲaʁ}{\papi{ xɕajɲaʁ}}}\markboth{xɕajɲaʁ}{}\classe{n}
\begin{définition}\fra espèce d'arbre\end{définition}
\begin{définition}\cmn 乔木的一种\end{définition}\end{entrée}

\begin{entrée}
\vedette{\hypertarget{Ⓔxɕɤfsa}{\papi{ xɕɤfsa}}}\markboth{xɕɤfsa}{}\classe{n}
\begin{définition}\fra ficelle (faite de paille)\end{définition}
\begin{définition}\cmn 草绳\end{définition}
\begin{relation-sémantique}\confer{
\hyperlink{ⒺxɕajⒽ1}{\textit{ \papi{xɕaj}}}
}\end{relation-sémantique}\end{entrée}

\begin{entrée}
\vedette{\hypertarget{Ⓔxɕɤɣrum}{\papi{ xɕɤɣrum}}}\markboth{xɕɤɣrum}{}\classe{n}
\begin{définition}\fra espèce d'arbre\end{définition}
\begin{définition}\cmn 乔木的一种\end{définition}
\begin{relation-sémantique}\confer{
\hyperlink{ⒺxɕajⒽ1}{\textit{ \papi{xɕaj}}}
}\end{relation-sémantique}\end{entrée}

\begin{entrée}
\vedette{\hypertarget{Ⓔxɕɤndʑu}{\papi{ xɕɤndʑu}}}\markboth{xɕɤndʑu}{}\classe{n}
\begin{définition}\fra bâton très fin\end{définition}
\begin{définition}\cmn 又细又短的小木棒\end{définition}
\end{entrée}

\begin{entrée}
\vedette{\hypertarget{Ⓔxɕelwi}{\papi{ xɕelwi}}}\markboth{xɕelwi}{}\classe{n}
\begin{définition}\fra tique\end{définition}
\begin{définition}\cmn 蜱【草虱】\end{définition}
\begin{exemple}\jya xɕelwi nɯ qajɯ ci ŋu. ɯ-mdoʁ kɯ-ɣɯrni ŋu. ɯ-mɤlɤjaʁ kɯβdɤ-ldʑa tu, ɯ-ku kɯ-xtɕɯ-xtɕi ŋu, ɯ-mtɕhi ɲɯ-ɤmtɕoʁ, sɯŋgɯ kɯ-mbɤr cho stɤmku ra ɣɤʑu tɕe nɯŋa cho jla mbala nɯ ra nɯ-taʁ kɤ-ndzoʁ rga, tɕe nɯ-se ku-tshi tɕe tɤ-fka tɕe ɯʑo kɯ-wxtɯ-wxti ʑo ɲɯ-βze ŋu, fsapaʁ ɯ-se ku-tshi ɕɯŋgɯ staʁnɤ sqi jamar ɲɯ-wxti ɲɯ-ŋu. tɕe xɕelwi nɯ tɤ-fka tɕe pjɤ-tɤr ɲɯ-ɕti ma ɯ-jɤɣɤt sɤ-lɤt maŋe, tɕe tu-ndze nɤ tu-ndze ma nɯ ma kɤ-lɤt mɯ́j-khɯ. tɕe wuma ʑo dɤn, fsapaʁ tɯ-rdoʁ ɯ-taʁ kɤ-rtsi mɤ-kɯ-sɤcha ʑo ku-ndzoʁ ɲɯ-ɕti, tɕe fsapaʁ ɯ-taʁ wuma ʑo ʁnɤt, tɕe tham tɕe ɯ-smɤn ɣɤʑu tɕe, tú-wɣ-lɤt tɕe ɲɯ-ɣɤme ɲɯ-ŋu ri ɯ-qhu tɕe li ku-ndzoʁ ɲɯ-ɕti. tɯrme ɯ-taʁ kɯnɤ ku-ndzoʁ ŋgrɤl tɕe ɣɯ-phɯt tɤ-kha tɕe ɯ-rmi tu-βzu mɤ-βdi ma tɕe ɯ-phoŋbu nɯ kɤ-phɯt khɯ ma ɯ-ku nɯ ku-raʁ ŋu tɕe kɤ-phɯt mɤ-khɯ. mɤ-kɤ-nɤrmi ɲɯ́-wɣ-phɯt tɕe kɤ́kɯku kɤ-phɯt khɯ tu-kɯ-ti ɲɯ-ŋu.\cmn 蜱是一种虫,是红色的。有四只脚,头部很小,嘴很尖。生活在灌木丛和草地上,喜欢爬在奶牛、犏牛和黄牛身上吸血。吸饱以后,它的身体就变得很大,是它吸血前的十倍以上,蜱吸饱了就会掉下来。因为身上没有肛门,只能吃了又吃,不能解便。蜱非常多,一头牛身上的蜱不计其数,对牲畜非常有害。现在有一种农药可以消灭它,但过了一段时间它会重新出现。蜱也会爬在人身上。据说拔掉的时候不能叫它的名字,不然只能拔掉它的身子,它的头部会卡在里面。不叫它的名字的话就可以连头一起拔掉。\end{exemple}
\end{entrée}

\begin{entrée}
\vedette{\hypertarget{Ⓔxɕiri}{\papi{ xɕiri}}}\markboth{xɕiri}{}\classe{n}
\begin{définition}\fra belette\end{définition}
\begin{définition}\cmn 黄鼠狼\end{définition}
\begin{exemple}\jya xɕiri nɯ ɯ-ku lɯlu fse, ɯ-rna ra kɯnɤ lɯlu fse, ɯ-mi nɯ ra ɯ-ndzrɯ tu, ɯ-phoŋbu ra kɯ-ɤɣɯrnɯɕɯr ŋu, ɯ-jme jpum cho rɲɟi, ɯ-phoŋbu cho ɯ-jme ra ɯ-tɯ-jpum ɯ-tɯ-zri afsu. tɤ-rɤku mɤ-ndze, tɕeri ɯʑo staʁ kɯ-xtɕi ɣɯ pɣɤtɕɯ, βʑɯ nɯ ra tu-ndze ŋu. ci ci tɕe lɯlu kɯnɤ ku-mtsɯɣ ŋgrɤl, li ci ci tɕe tɤ-pɤtso nɤrŋi tsa ku-mtsɯɣ cho pjɯ-mɯrʁɯz ra ŋgrɤl, tɕe rɯdaʁ kɯ-ŋɤn tsa ci ŋu.\cmn 黄鼠狼的头像猫,耳朵也像猫,脚上有爪子,身体是淡红色的,尾巴粗而长,身子的粗细长短与尾巴相同。不吃粮食,但是吃比自己小的鸟、老鼠等。有时候还会咬猫,咬伤和抓伤婴儿,是比较坏的动物。\end{exemple}
\end{entrée}

\begin{entrée}
\vedette{\hypertarget{Ⓔxɕɯβ}{\papi{ xɕɯβ}}}\markboth{xɕɯβ}{}
\classe{vs}
\paradigme{\textit{dir :} \jya nɯ-}
\paradigme{\textit{dir :} \jya pɯ-}
\paradigme{\textit{dir :} \jya lɤ-}
\begin{définition}\fra se dégonfler\end{définition}
\begin{définition}\cmn 瘪下去;缩下去\end{définition}
\begin{exemple}\jya ɯ-ŋgɯ tɯ-ci ɲɤ-me tɕe ɲɤ-xɕɯβ\cmn 里面的水没有了,缩下去了\end{exemple}
\begin{relation-sémantique}\synonyme{
\hyperlink{Ⓔɲchoʁ}{\textit{ \papi{ɲchoʁ}}}
}\end{relation-sémantique}\end{entrée}

\begin{entrée}
\vedette{\hypertarget{Ⓔxoŋnɤxoŋ}{\papi{ xoŋnɤxoŋ}}}\markboth{xoŋnɤxoŋ}{}\classe{idph.2}
\begin{définition}\fra sensation d'engourdissement dans la bouche (après avoir mangé du xanthoxyle)\end{définition}
\begin{définition}\cmn 形容吃了花椒以后麻的感觉\end{définition}\begin{sous-entrée}
\vedette{\hypertarget{}{\papi{ sɤxoŋxoŋ}}}\markboth{sɤxoŋxoŋ}{}\classe{vt}
\begin{définition}\fra engourdir la bouche (xanthoxyle)\end{définition}
\begin{définition}\cmn 发麻\end{définition}
\begin{exemple}\jya tɕɣom a-kɯr ɯ-ŋgɯ na-sɤxoŋxoŋ ʑo\cmn 花椒吃在嘴里麻得很\end{exemple}
\begin{relation-sémantique}\confer{
\hyperlink{Ⓔɣɤzɯβzɯβ}{\textit{ \papi{ɣɤzɯβzɯβ}}}
}\end{relation-sémantique}
\end{sous-entrée}\end{entrée}

\begin{entrée}
\vedette{\hypertarget{Ⓔxoŋxoŋ}{\papi{ xoŋxoŋ}}}\markboth{xoŋxoŋ}{}\classe{idph.2}
\begin{définition}\fra blanchâtre\end{définition}
\begin{définition}\cmn 形容灰白\end{définition}
\begin{exemple}\jya tɯɣur xoŋxoŋ ʑo pjɤ-ta\cmn 打了灰白的霜\end{exemple}\end{entrée}

\begin{entrée}
\vedette{\hypertarget{Ⓔxsar}{\papi{ xsar}}}\markboth{xsar}{}\classe{n}
\begin{définition}\fra naemorhedus goral\end{définition}
\begin{définition}\cmn 岩羊【青羊】\end{définition}
\begin{exemple}\jya xsar nɯ cɤmi praʁ ɯ-ŋgɯ ku-rɤʑi ɲɯ-ŋu, praʁ ɯ-taʁ kɤ-ŋke wuma kɯ-cha ŋu, tɯ-mɯ kɯ pɯ-pa-χtɕi tu-ŋke-a cha tu-ti ŋu tu-kɯ-ti ɲɯ-ŋu, ɯ-mdoʁ nɯ kɯ-pɣi ɲɯ-ŋu, tu-nɯsɯku ŋgrɤl, sɯjwaʁ tu-ndze, ɯ-ʁrɯ nɯ tɯ-tɕha kɯ-mtɕɯ-mtɕoʁ tu, khɯna kɤ-sat wuma kɯ-cha ŋu. khɯna kɯ praʁ ɯ-taʁ ku-roʁ ŋgrɤl tɕe, nɯ tɕu tɕe tu-tɕhi tɕe pjɤ-sat ŋgrɤl. qartsɯ tɤjpɣom kɤ-ta ɯ-raŋ tɕe, tshu, tɕe nɯ chɯ-nɯrmɤmbe ŋu. ɯ-qa taʁrɯ nɯ tshɤt cho naχtɕɯɣ\cmn 青羊一般生活在河坝的岩石里,善于在岩石上行走。据说它自吹:“凡雨水能淋到的地方,我都能走”。青羊是灰色的,能爬树,吃树叶,有一对很尖的角,能把狗杀死。狗把它追到悬崖时,青羊就会用角把狗顶死。冬天结冰时,青羊会变得很肥,会脱毛。蹄子像山羊的蹄子一样。\end{exemple}
\end{entrée}

\begin{entrée}
\vedette{\hypertarget{Ⓔxsɤndʐi}{\papi{ xsɤndʐi}}}\markboth{xsɤndʐi}{}\classe{n}
\begin{définition}\fra peau de goral\end{définition}
\begin{définition}\cmn 岩羊【青羊】皮子\end{définition}
\begin{relation-sémantique}\confer{
\hyperlink{Ⓔxsar}{\textit{ \papi{xsar}}}
}\end{relation-sémantique}
\begin{relation-sémantique}\confer{
\hyperlink{Ⓔtɯ-ndʐi}{\textit{ \papi{tɯ-ndʐi}}}
}\end{relation-sémantique}\end{entrée}

\begin{entrée}
\vedette{\hypertarget{Ⓔxsɤrɯ}{\papi{ xsɤrɯ}}}\markboth{xsɤrɯ}{}\classe{n}
\begin{définition}\fra une plante\end{définition}
\begin{définition}\cmn 植物的一种\end{définition}
\begin{exemple}\jya xsɤrɯ nɯ tɯ-ji ɯ-rkɯ tu-kɯ-ɬoʁ ci sɯjno ŋu. ɯ-jwaʁ kɯ-mba kɯ-zri tsa ŋu, ɯ-zrɤm nɯ kɯ-wɣrum kɯ-ɤrɤrtsɯ-rtsɤɣ ci ŋu, rko. jla nɯŋa ra wuma rga-nɯ, tɤrɤku ɯ-ŋgɯ tɤ-ɬoʁ tɕe, mɤ-sɤpe. ɯ-lu tu.\cmn 
\stylefv{xsɤrɯ}是生长在田边的植物,叶子细长,根是白色的,有节且硬。犏牛、奶牛都喜欢吃。长在庄稼地里时,对庄稼不好。有乳汁。
\end{exemple}\end{entrée}

\begin{entrée}
\vedette{\hypertarget{Ⓔxsɯr}{\papi{ xsɯr}}}\markboth{xsɯr}{}\classe{vt}
\paradigme{\textit{dir :} \jya tɤ-}
\begin{définition}\fra poêler\end{définition}
\begin{définition}\cmn 炒\end{définition}
\begin{exemple}\jya aʑo tɤjmɤɣ tɤ-xsɯr-a\cmn 我把蘑菇炒了\end{exemple}
\begin{relation-sémantique}\confer{
\hyperlink{Ⓔɕnɤxsɯr}{\textit{ \papi{ɕnɤxsɯr}}}
}\end{relation-sémantique}\begin{sous-entrée}
\vedette{\hypertarget{}{\papi{ rɤxsɯr}}}\markboth{rɤxsɯr}{}\classe{vi}
\paradigme{\textit{dir :} \jya tɤ-}
\begin{définition}\ 
\begin{déclaration}\grammar{apass}\end{déclaration}\end{définition}
\begin{exemple}\jya z-rɤxsɯr\cmn 炒锅(用来炒菜的工具)\end{exemple}
\end{sous-entrée}\end{entrée}

\begin{entrée}
\vedette{\hypertarget{Ⓔxsɯxsɯ}{\papi{ xsɯxsɯ}}}\markboth{xsɯxsɯ}{}\classe{n}
\begin{définition}\fra type de sac\end{définition}
\begin{définition}\cmn 口袋的一种\end{définition}
\begin{exemple}\jya xsɯxsɯ (khorca) nɯ laχtɕha fkɯm tu-kɤ-fkur ci ŋu, rgali pɯ ɯ-ndʐi nɯ kɯ-mdoʁmdi pjɯ́-wɣ-qaʁ tɕe tɤ-rom tɕe ta-mar ɲɯ́-wɣ-mar pjɯ́-wɣ-χtsɤβ tɕe, nɯ mpɯ ʑo tɕe, ɯ-rme nɯ ɯ-pɕi ɲɯ́-wɣ-ɕthɯz tɕe, ɯ-mɤlɤjaʁ kɯβde ɣɯ ɯ-ndʐi nɯ tɕu ɯ-sɤ-fkur ɯ-ri kú-wɣ-tshoʁ, thaχthi maʁ nɤ tɯ-ndʐi qase thɯ-kɤ-tɕɤt kú-wɣ-tshoʁ tɕe, nɯ ɯ-sɤ-fkur ŋu. ɯ-mke stu nɯ pjɯ́-wɣ-pri tɕe, nɯ tɤ-fkɯm ɣɯ ɯ-mŋu ɲɯ́-wɣ-βzu ŋu, tɕe nɯtɕu qase ci kú-wɣ-βraʁ tɕe tɤ-fkɯm ɯ-ŋgɯ laχtɕha thɯ́-wɣ-rku tɕe, qase nɯ kɯ ɯ-mŋu nɯ kú-wɣ-sɯ-xtɕɤr ŋu, ɯ-mŋu nɯ ɯ-taʁ pɕoʁ tú-wɣ-ɕthɯz tɕe tú-wɣ-fkur ŋu.\cmn 
\stylefv{xsɯxsɯ}是装东西的口袋,可以背。把小牛的皮整块剥下来,干了以后,就擦上酥油,搓揉,等到揉得软了,毛向外翻,在四条腿的皮子上系上背带,或者用线制作的背带,或者用剖成的皮绳作背带。在脖颈部位破一条口子,作为口袋的开口,在那里扎一根皮绳,把东西装好后,就用皮绳捆住口子。背的时候,口袋的口子向上。
\end{exemple}
\end{entrée}

\begin{entrée}
\vedette{\hypertarget{Ⓔxʂɤxʂɤt}{\papi{ xʂɤxʂɤt}}}\markboth{xʂɤxʂɤt}{}
\classe{idph.2}
\begin{définition}\fra long et fin, flexible\end{définition}
\begin{définition}\cmn 形容苗条、纤细、柔软的样子\end{définition}
\begin{exemple}\jya ɯ-xtu ra xʂɤxʂɤt to-stu\cmn 她肚子很瘦。\end{exemple}
\begin{exemple}\jya jiɕqha tɯrme ɯ-ŋga ɯ-tɯ-xtɕi kɯ xʂɤxʂɤt ɲɯ-pa\cmn 那个人衣服穿得很紧\end{exemple}
\begin{exemple}\jya kɯ-xtshɯm jnom ci xʂɤxʂɤt ɲɯ-ŋu\cmn 又细又软\end{exemple}
\begin{exemple}\jya ɯ-phoŋbu ra xʂɤxʂɤt ʑo kɯ-pa ci ɲɯ-ŋu\cmn 他身材苗条\end{exemple}
\begin{relation-sémantique}\confer{
\hyperlink{Ⓔχʂɤχʂɤt}{\textit{ \papi{χʂɤχʂɤt}}}
}\end{relation-sémantique}\end{entrée}

\begin{entrée}
\vedette{\hypertarget{Ⓔxtaŋxtaŋ}{\papi{ xtaŋxtaŋ}}}\markboth{xtaŋxtaŋ}{}\classe{ideo.2}
\begin{définition}\fra gonflé\end{définition}
\begin{définition}\cmn 形容胀得很鼓的样子\end{définition}
\begin{exemple}\jya ɯ-xtu xtaŋxtaŋ ʑo ɲɯ-nɤmbɤβ\cmn 他肚子胀得很鼓\end{exemple}\end{entrée}

\begin{entrée}
\vedette{\hypertarget{Ⓔxtɤβxtɤβ}{\papi{ xtɤβxtɤβ}}}\markboth{xtɤβxtɤβ}{}\classe{idph.2}
\begin{définition}\fra épais et court\end{définition}
\begin{définition}\cmn 形容粗而短的样子
\end{définition}
\begin{exemple}\jya tɤ-pɤtso ɯ-mi ɲɯ-tshu xtɤβxtɤβ ʑo\cmn 小伙子的腿又粗又短\end{exemple}\end{entrée}

\begin{entrée}
\vedette{\hypertarget{Ⓔxtɤqa}{\papi{ xtɤqa}}}\markboth{xtɤqa}{}\classe{n}
\begin{définition}\fra bas-ventre\end{définition}
\begin{définition}\cmn 小肚子\end{définition}
\begin{relation-sémantique}\confer{
\hyperlink{Ⓔtɯ-xtu}{\textit{ \papi{tɯ-xtu}}}
}\end{relation-sémantique}
\begin{relation-sémantique}\confer{
\hyperlink{Ⓔtɯ-qa}{\textit{ \papi{tɯ-qa}}}
}\end{relation-sémantique}\end{entrée}

\begin{entrée}
\vedette{\hypertarget{Ⓔxtɤtshɤt}{\papi{ xtɤtshɤt}}}\markboth{xtɤtshɤt}{}\classe{n}
\begin{définition}\fra contrôle de son appétit\end{définition}
\begin{définition}\cmn 饮食节制有度\end{définition}
\begin{exemple}\jya tsuku lu-βzi-nɯ tɕe, cha kɤ-tshi lu-ɣɤtɕhom-nɯ tɕe, nɯ-ɣi ra kɯ tú-wɣ-nɤmqe-nɯ tɕe, ``nɤ-xtɤtshɤt ɯ-tɯ-me nɯ" tu-ti-nɯ ɲɯ-ŋu\cmn 有些人喝酒喝太多,醉了,他们的家人骂他们说:“你完全是饮食无度啊”\end{exemple}
\begin{relation-sémantique}\confer{
\hyperlink{Ⓔtɯ-xtu}{\textit{ \papi{tɯ-xtu}}}
}\end{relation-sémantique}\end{entrée}

\begin{entrée}
\vedette{\hypertarget{Ⓔxtɕɤr}{\papi{ xtɕɤr}}}\markboth{xtɕɤr}{}
\classe{vt}
\paradigme{\textit{dir :} \jya kɤ-}
\paradigme{\textit{dir :} \jya tɤ-}
\begin{définition}\fra attacher\end{définition}
\begin{définition}\cmn 系\end{définition}
\begin{exemple}\jya a-xtsa tɤ-xtɕar-a (ɯ-ri kɤ-lat-a)\cmn 我系了鞋带\end{exemple}
\begin{exemple}\jya tɤ-fkɯm kɤ-xtɕɤr\cmn 你把口袋系一下\end{exemple}
\begin{relation-sémantique}\confer{
\hyperlink{Ⓔxtsɤxtɕɤr}{\textit{ \papi{xtsɤxtɕɤr}}}
}\end{relation-sémantique}
\begin{relation-sémantique}\confer{
\hyperlink{Ⓔmthɯxtɕɤr}{\textit{ \papi{mthɯxtɕɤr}}}
}\end{relation-sémantique}\end{entrée}

\begin{entrée}
\vedette{\hypertarget{Ⓔxtɕi}{\papi{ xtɕi}}}\markboth{xtɕi}{}
\classe{vs}
\paradigme{\textit{dir :} \jya nɯ-}
\paradigme{\textit{dir :} \jya pɯ-}
\begin{définition}\fra petit\end{définition}
\begin{définition}\cmn 小\end{définition}
\begin{exemple}\jya wuma ɲɯ-xtɕi\cmn 很小\end{exemple}
\begin{exemple}\jya ɯ-phoŋbu ɲɯ-xtɕi\cmn 他身体很小\end{exemple}
\begin{exemple}\jya ɯ-lɯz kɯ-xtɕi\cmn 他很年轻\end{exemple}
\begin{exemple}\jya @shouji ɯ-skɤt ɲɯ-xtɕi tɕe, khro mɯ́j-mtsham-a\cmn 手机的声音很低,我听不见\end{exemple}
\begin{exemple}\jya nɤ-skɤt pɯ-xtɕi\cmn 你的声音变小了\end{exemple}
\begin{exemple}\jya nɤʑo nɤ-mu pɯ-xtɕi tɕe ɯ-pɯ́-mpɕɤr?\cmn 你母亲小的时候漂亮吗?\end{exemple}\begin{sous-entrée}
\vedette{\hypertarget{}{\papi{ ɣɤxtɕi}}}\markboth{ɣɤxtɕi}{}\classe{vt}
\paradigme{\textit{dir :} \jya nɯ-}
\begin{définition}\fra rendre petit\end{définition}
\begin{définition}\cmn 弄小\end{définition}
\begin{exemple}\jya ɲɯ-dɤn tsa ɕti ri, pɯ-ɣɤxtɕi-t-a\cmn 有点多,我弄小了\end{exemple}
\begin{relation-sémantique}\antonyme{
\hyperlink{Ⓔwxti}{\textit{ \papi{wxti}}}
}\end{relation-sémantique}
\begin{relation-sémantique}\confer{
\hyperlink{Ⓔaxtɕɯxte}{\textit{ \papi{axtɕɯxte}}}
}\end{relation-sémantique}
\end{sous-entrée}\begin{sous-entrée}
\vedette{\hypertarget{}{\papi{ kɯxtɕɯxtɕi}}}\markboth{kɯxtɕɯxtɕi}{}
\begin{définition}\fra un peu\end{définition}
\begin{définition}\cmn 一点\end{définition}
\begin{exemple}\jya ɯʑo kɯ kɯ-xtɕɯ-xtɕi ma na-nɤma me\cmn 他只做了一点点\end{exemple}
\end{sous-entrée}\end{entrée}

\begin{entrée}
\vedette{\hypertarget{Ⓔxtɕɯxte}{\papi{ xtɕɯxte}}}\markboth{xtɕɯxte}{}\classe{n}
\begin{définition}\fra taille\end{définition}
\begin{définition}\cmn 大小\end{définition}
\begin{relation-sémantique}\confer{
\hyperlink{Ⓔxtɕi}{\textit{ \papi{xtɕi}}}
}\end{relation-sémantique}
\begin{relation-sémantique}\confer{
\hyperlink{Ⓔmɯxte}{\textit{ \papi{mɯxte}}}
}\end{relation-sémantique}
\begin{relation-sémantique}\confer{
\hyperlink{Ⓔaxtɕɯxte}{\textit{ \papi{axtɕɯxte}}}
}\end{relation-sémantique}\end{entrée}

\begin{entrée}
\vedette{\hypertarget{Ⓔxthom}{\papi{ xthom}}}\markboth{xthom}{}\classe{vt}
\paradigme{\textit{dir :} \jya nɯ-}
\begin{définition}\fra poser horizontalement\end{définition}
\begin{définition}\cmn 横着放;放平\end{définition}
\begin{exemple}\jya ɕoŋtɕa na-xthom\cmn 他把木料横着放了\end{exemple}
\begin{exemple}\jya ɯ-ndɤcu @luyinji na-xthom\cmn 他把录音机放平了\end{exemple}
\begin{exemple}\jya laʁjɯɣ pɯ-xthom-a\cmn 我把棍子放平了\end{exemple}
\begin{relation-sémantique}\confer{
\hyperlink{Ⓔndom}{\textit{ \papi{ndom}}}
}\end{relation-sémantique}\begin{sous-entrée}
\vedette{\hypertarget{}{\papi{ axthom}}}\markboth{axthom}{}\classe{vi}
\begin{définition}\ 
\begin{déclaration}\grammar{pass}\end{déclaration}\end{définition}
\begin{définition}\fra être posé\end{définition}
\begin{définition}\cmn 横放着\end{définition}
\begin{exemple}\jya nɯre ri axthom tɕe ata ɕti\cmn 在那里放着\end{exemple}
\end{sous-entrée}\begin{sous-entrée}
\vedette{\hypertarget{}{\papi{ ʑɣɤxthom}}}\markboth{ʑɣɤxthom}{}\classe{vi}
\paradigme{\textit{dir :} \jya lɤ-}
\begin{définition}\ 
\begin{déclaration}\grammar{refl}\end{déclaration}\end{définition}
\begin{définition}\fra s'allonger horizontalement\end{définition}
\begin{définition}\cmn 躺下\end{définition}
\begin{exemple}\jya nɯ ɕɯmɯma ʑo ɯ-thoʁ nɯ tɕu lo-ʑɣɤxthom\cmn 他马上就躺在地上\end{exemple}
\end{sous-entrée}\end{entrée}

\begin{entrée}
\vedette{\hypertarget{Ⓔxtsu}{\papi{ xtsu}}}\markboth{xtsu}{}\classe{vi}
\paradigme{\textit{dir :} \jya nɯ-}
\begin{définition}\fra fermenter\end{définition}
\begin{définition}\cmn 发酵\end{définition}
\begin{exemple}\jya tɯ-ɣli ɲo-xtsu\cmn 肥料发酵了\end{exemple}
\begin{exemple}\jya cha ɲo-xtsu\cmn 酒发酵了\end{exemple}
\begin{exemple}\jya ɯ-sɯm rɯwɯrawi ɲɯ-xtsu\cmn 心情很烦乱\end{exemple}
\begin{exemple}\jya rzoŋlu ʑo ɲɯ-xtsu\cmn 忙得不可开交\end{exemple}
\begin{relation-sémantique}\confer{
\hyperlink{ⒺsɯxtsuⒽ2}{\textit{ \papi{sɯxtsu2}}}
}\end{relation-sémantique}
\end{entrée}

\begin{entrée}
\vedette{\hypertarget{Ⓔxtsɤɕna}{\papi{ xtsɤɕna}}}\markboth{xtsɤɕna}{}\classe{n}
\begin{définition}\fra pointe recourbée des bottes\end{définition}
\begin{définition}\cmn 鞋子钩起的顶端\end{définition}
\begin{exemple}\jya xtsɤɕna nɯ tɯ-xtsa ɣɯ ɯ-ʁɤri tɯ-ɕna kɯ-fse tɯ-kɯ-ŋgɤɣ nɯ ŋu tɕe xtsɤrkɯ ɯ-ʁɤri ku-kɯ-ɤndɯndo ɯ-stu nɯ ŋu, ɯ-qhuchu lu-kɯ-ɣe komɤr jaʁndzu χsɯm jamar kɯ-rɟum, tɯ-tɣa ro ro kɯ-rɲɟi nɯ li xtsɤɕna rmi.\cmn 
\stylefv{xtsɤɕna}(鞋鼻子)是鞋子前面钩着的部分,是\stylefv{xtsɤrkɯ}(鞋边)的接头部分,后面有一块三指宽、一拃多长的红皮子,这块皮子叫\stylefv{xtsɤɕna}(鞋鼻子)。
\end{exemple}\end{entrée}

\begin{entrée}
\vedette{\hypertarget{Ⓔxtsɤku}{\papi{ xtsɤku}}}\markboth{xtsɤku}{}\classe{n}
\begin{définition}\fra partie de la botte qui recouvre les mollets\end{définition}
\begin{définition}\cmn 靴筒(靴子盖小腿的部分)\end{définition}
\end{entrée}

\begin{entrée}
\vedette{\hypertarget{Ⓔxtsɤpɤl}{\papi{ xtsɤpɤl}}}\markboth{xtsɤpɤl}{}\classe{n}
\begin{définition}\fra chaussure (ne dépasse pas la cheville)\end{définition}
\begin{définition}\cmn 鞋子(没有筒)\end{définition}
\end{entrée}

\begin{entrée}
\vedette{\hypertarget{Ⓔxtsɤqar}{\papi{ xtsɤqar}}}\markboth{xtsɤqar}{}\classe{n}
\begin{définition}\fra endroit où la semelle et la chaussure sont cousues ensemble\end{définition}
\begin{définition}\cmn 鞋底和鞋子的接头部分\end{définition}
\end{entrée}

\begin{entrée}
\vedette{\hypertarget{Ⓔxtsɤqarmbe}{\papi{ xtsɤqarmbe}}}\markboth{xtsɤqarmbe}{}\classe{n}
\begin{définition}\fra vieille semelle de chaussure en cuir\end{définition}
\begin{définition}\cmn 被扔掉的皮鞋底\end{définition}
\end{entrée}

\begin{entrée}
\vedette{\hypertarget{Ⓔxtsɤqɤr}{\papi{ xtsɤqɤr}}}\markboth{xtsɤqɤr}{}\classe{n}
\begin{définition}\fra bordure de la semelle\end{définition}
\begin{définition}\cmn 鞋底子的周边\end{définition}
\begin{relation-sémantique}\confer{
\hyperlink{Ⓔtɯ-xtsa}{\textit{ \papi{tɯ-xtsa}}}
}\end{relation-sémantique}\end{entrée}

\begin{entrée}
\vedette{\hypertarget{Ⓔxtsɤqɤrqaβ}{\papi{ xtsɤqɤrqaβ}}}\markboth{xtsɤqɤrqaβ}{}\classe{n}
\begin{définition}\fra aiguille pour réparer les chaussures\end{définition}
\begin{définition}\cmn 补鞋子的针\end{définition}
\begin{relation-sémantique}\confer{
 \papi{qaβ}
}\end{relation-sémantique}\end{entrée}

\begin{entrée}
\vedette{\hypertarget{Ⓔxtsɤrkɯ}{\papi{ xtsɤrkɯ}}}\markboth{xtsɤrkɯ}{}\classe{n}
\begin{définition}\fra partie de la chaussure recouvrant les pieds\end{définition}
\begin{définition}\cmn 鞋边(鞋子、靴子盖脚的部分)\end{définition}
\begin{exemple}\jya xtsɤrkɯ nɯ tɯ-xtsa χchoʁe ɯ-pa mŋulɤn ɯ-sɤ-tshoʁ nɯ ŋu, xtsɤrkɯ ɯ-taʁ nɯ tɕu xtsɤku pjɯ́-wɣ-tshoʁ, ɯ-ŋgɯ ɯ-pɕoʁ nɯ tɕu xtsɤkɤŋgɯ pjɯ-tu ra, xtsɤkɤŋgɯ nɯ tɯ-ŋgar pjɯ-ŋu ra, xtsɤku nɯ cɤndʐi nɯ maʁ nɤ ɕkom ndʐi, nɯ maʁ nɤ qartshɤndʐi kɯ-mba pjɯ-ŋu ɲɯ-ra. tɯ-xtsa nɯ ɯ-qhu tɕe ɯ-srɯβzɤn pjɯ-tu ɲɯ-ra, tɯ-xtsa nɯ tú-wɣ-ŋga tɕe wuma ʑo mpja. tɯ-xtsa ʁnɯ-tɯphu tu, tɯ-tɯphu nɯ konaʁxtsa rmi, ɯ-xtsɤrkɯ nɯ komɤr thɯ-kɤ-sɯɣ-ɲaʁ ŋu, mɤʑɯ tɯ-tɯphu nɯ koscaxtsa rmi, tɕe ɯ-xtsɤrkɯ nɯ mɯ-pɯ-kɤ-sɯxtshwi ŋu.\cmn 
鞋帮的左右两边的下侧装鞋底,上面装靴筒,在内层要有内衬,是用羊毛织出来的布料。靴筒是用麝香鹿皮、麂子皮或是比较薄的鹿子皮做成的。在鞋后面要有\stylefv{srɯβzɤn}(在缝合处夹的一块布料)。这种鞋子穿起来很暖和。鞋子分成两种,一种叫\stylefv{konaʁ xtsa},鞋帮是染成黑色的红皮子,另一种叫\stylefv{kosca xtsa},鞋帮是根本没有染色的皮子。
\end{exemple}\end{entrée}

\begin{entrée}
\vedette{\hypertarget{Ⓔxtsɤrtɯm}{\papi{ xtsɤrtɯm}}}\markboth{xtsɤrtɯm}{}\classe{n}
\begin{définition}\fra botte en cuir\end{définition}
\begin{définition}\cmn 皮靴子\end{définition}
\begin{exemple}\jya xtsɤrtɯm nɯ ɯ-xtsɤku nɯ tɯ-ndʐi ŋu tɕe, ɯ-ɕnɤku ɯ-stu nɯ lú-wɣ-phaʁ tɕe, nɯ ɯ-stu nɯ tɯ-ndʐi kɤ-βzɯχsɯm tú-wɣ-rku tɕe, ɯ-xtsɤrkɯ ɯ-ʑoz kɯ-me. xtsɤrtɯm nɯ li ʁnɯ-tɯphu tu tɕe, tɯ-tɯphu nɯ ɯ-pɕi tɯ-ndʐi ŋu, ɯ-ŋgɯ tɯ-ŋgar ŋu, li ci tɯ-tɯphu nɯ tɤ-mbextsa rmi, ɯ-pɕi nɯ tɯ-rtɯthɯ ŋu, ɯ-ŋgɯ nɯ li tɯ-ŋgar ŋu. ki ʁnɯ-tɯphu ki tɯ-xtsa ni ndʑi-tʂɯβ naχtɕɯɣ, ndʑi-mŋulɤn ra.\cmn 
皮靴子的靴筒是一块皮子,把那块皮子的前端(对着脚背的部分)破一个口,那里装上三角形的皮子,没有另外的鞋帮。这种靴子有两种,一种外层是皮子,内层是羊毛布,另一种叫 \stylefv{tɤmbextsa},外层是麻布,里面还是羊毛布。这两种的缝法一样,都需要鞋底。
\end{exemple}
\end{entrée}

\begin{entrée}
\vedette{\hypertarget{Ⓔxtsɤsoʁ}{\papi{ xtsɤsoʁ}}}\markboth{xtsɤsoʁ}{}\classe{n}
\begin{définition}\fra semelle en paille\end{définition}
\begin{définition}\cmn 鞋垫\end{définition}
\end{entrée}

\begin{entrée}
\vedette{\hypertarget{Ⓔxtsɤxtɕɤr}{\papi{ xtsɤxtɕɤr}}}\markboth{xtsɤxtɕɤr}{}
\classe{n}
\begin{définition}\fra lacet\end{définition}
\begin{définition}\cmn 鞋带\end{définition}
\begin{exemple}\jya xtsɤxtɕɤr na-nɯ-rtɤβ\cmn 他系了鞋带\end{exemple}
\begin{exemple}\jya nɤ-xtsɤxtɕɤr ɲɤ-nɯ-ɬoʁ nɤ!\cmn 你的鞋带解开了\end{exemple}
\begin{exemple}\jya nɤ-xtsɤxtɕɤr kɤ-lɤt ma tɯ́-wɣ-tʂaβ\cmn 你把鞋带系上,不然会摔跤的\end{exemple}\end{entrée}

\begin{entrée}
\vedette{\hypertarget{Ⓔxtshɯm}{\papi{ xtshɯm}}}\markboth{xtshɯm}{}
\classe{vs}
\paradigme{\textit{dir :} \jya nɯ-}
\begin{définition}\fra fin\end{définition}
\begin{définition}\cmn 细(直径)\end{définition}
\begin{exemple}\jya tɤ-ri kɯ-xtshɯm\cmn 细线\end{exemple}
\begin{exemple}\jya si kɯ-xtshɯm\cmn 很细的树\end{exemple}\begin{sous-entrée}
\vedette{\hypertarget{}{\papi{ ɣɤxtshɯm}}}\markboth{ɣɤxtshɯm}{}\classe{vt}
\paradigme{\textit{dir :} \jya thɯ-}
\begin{définition}\fra rendre fin\end{définition}
\begin{définition}\cmn 使变细\end{définition}
\begin{relation-sémantique}\antonyme{
\hyperlink{Ⓔjpum}{\textit{ \papi{jpum}}}
}\end{relation-sémantique}
\end{sous-entrée}\end{entrée}

\begin{entrée}
\vedette{\hypertarget{Ⓔxtsɯ}{\papi{ xtsɯ}}}\markboth{xtsɯ}{}
\classe{vt}
\paradigme{\textit{dir :} \jya pɯ-}
\begin{définition}\fra piler\end{définition}
\begin{définition}\cmn 捣碎,砸碎,碾磨\end{définition}
\begin{exemple}\jya tʂha pa-xtsɯ\cmn 他捣碎了(马茶)\end{exemple}
\begin{exemple}\jya ɕom pa-xtsɯ\cmn 他打了铁\end{exemple}
\begin{exemple}\jya hajtsu pa-xtsɯ\cmn 他捣碎了辣椒\end{exemple}
\begin{exemple}\jya mɯzi pa-xtsɯ\cmn 他捣碎了黑火药\end{exemple}\end{entrée}

\begin{entrée}
\vedette{\hypertarget{Ⓔxtsɯɣ}{\papi{ xtsɯɣ}}}\markboth{xtsɯɣ}{}
\classe{vt}
\paradigme{\textit{dir :} \jya tɤ-}
\begin{définition}\fra toucher, atteindre, frapper\end{définition}
\begin{définition}\cmn 打中\end{définition}
\begin{exemple}\jya ta-xtsɯɣ\cmn 他打中了\end{exemple}
\begin{exemple}\jya tɤ-xtsɯɣa\cmn 我打中了\end{exemple}
\begin{exemple}\jya (pri) to-xtsɯɣ ri jo-nɯɕe\cmn 虽然我打中了熊,它回去了(逃走了 )\end{exemple}\begin{sous-entrée}
\vedette{\hypertarget{}{\papi{ sɯxtsɯɣ}}}\markboth{sɯxtsɯɣ}{}\classe{vt}
\paradigme{\textit{dir :} \jya tɤ-}
\begin{définition}\ 
\begin{déclaration}\grammar{caus}\end{déclaration}\end{définition}
\begin{définition}\fra atteindre avec\end{définition}
\begin{définition}\cmn 用……射中、打中\end{définition}
\begin{exemple}\jya rdɤstaʁ kɯ tɤ́-wɣ-sɯxtsɯɣ-a\cmn 他扔石头打中了我\end{exemple}
\begin{exemple}\jya ɯ-zgrɯ kɯ tɤ́-wɣ-sɯxtsɯɣ-a\cmn 他用肘打中了我\end{exemple}
\begin{exemple}\jya tɯdi kɯ tɤ́-wɣsɯxtsɯɣ-a\cmn 他射箭射中了我\end{exemple}
\begin{exemple}\jya kɯki mbrɯtɕɯ ɲɯ-mtɕoʁ tɕe, nɤ-jaʁ tɯ-sɯxtsɯɣ ma\cmn 刀很锋利,小心不要割到手\end{exemple}
\end{sous-entrée}\end{entrée}

\begin{entrée}
\vedette{\hypertarget{Ⓔxtsɯsna}{\papi{ xtsɯsna}}}\markboth{xtsɯsna}{}\classe{n}
\begin{définition}\fra toute sorte de\end{définition}
\begin{définition}\cmn 各种各样\end{définition}
\end{entrée}

\begin{entrée}
\vedette{\hypertarget{Ⓔxtʂoŋxtʂoŋ}{\papi{ xtʂoŋxtʂoŋ}}}\markboth{xtʂoŋxtʂoŋ}{}\classe{idph.2}
\begin{définition}\fra mou et gonflé\end{définition}
\begin{définition}\cmn 形容饱满而软的样子\end{définition}
\begin{exemple}\jya @qiqiu nɯ qale xtʂoŋxtʂoŋ ʑo chɤ-mtshɤt\cmn 气球吹得胀鼓鼓的\end{exemple}\end{entrée}

\begin{entrée}
\vedette{\hypertarget{Ⓔxtɯrkɯ}{\papi{ xtɯrkɯ}}}\markboth{xtɯrkɯ}{}\classe{n}
\begin{définition}\fra cordes pour attacher la charrue au joug\end{définition}
\begin{définition}\cmn 牛皮绳【纤绳】\end{définition}
\begin{exemple}\jya xtɯrkɯ nɯ mbɣɤru cho stuxsi ndʑi-kɯ-sɤthɤri tɯ-ndʐi kɯ-rɟum kɯ-zri tsa nɯ ŋu\cmn 
\stylefv{xtɯrkɯ}是连接犁杆和牛轭的又宽又长的牛皮绳
\end{exemple}\end{entrée}

\begin{entrée}
\vedette{\hypertarget{Ⓔxtɯrɲɟi}{\papi{ xtɯrɲɟi}}}\markboth{xtɯrɲɟi}{}
\classe{n}
\begin{définition}\fra longueur\end{définition}
\begin{définition}\cmn 长度\end{définition}
\begin{relation-sémantique}\confer{
\hyperlink{ⒺxtɯtⒽ1}{\textit{ \papi{xtɯt1}}}
}\end{relation-sémantique}
\begin{relation-sémantique}\confer{
\hyperlink{Ⓔrɲɟi}{\textit{ \papi{rɲɟi}}}
}\end{relation-sémantique}\end{entrée}

\begin{entrée}
\vedette{\hypertarget{ⒺxtɯtⒽ2}{\papi{ xtɯt}}}\markboth{xtɯt}{}\homonyme{2}
\classe{n}
\begin{définition}\fra chat sauvage\end{définition}
\begin{définition}\cmn 野猫\end{définition}
\begin{exemple}\jya xtɯt nɯ lɯlu cho kɯ-naχtɕɯ-χtɕɯɣ ʑo ŋu, tɕeri xtɯt nɯ sɯŋgɯ, praʁ ɯ-rchɤβ ku-rɤʑi ŋu, ɯʑo sɤznɤ rɯdaʁ kɯ-xtɕi ra tu-ndze ɲɯ-ŋu, lɯlu sɤznɤ kɯ-xtɕɯ-xtɕi ɲɯ-wxti cho ɲɯ-rkaŋ. ɯ-mdoʁ nɯ kɯ-pɣi tɕe ɯ-taʁ kɯ-ɲaʁ kɯ-ɤkhra ɲɯ-ŋu. ci ci tɕe lɯlu cho ɲawa tu-βzu-ndʑi ɲɯ-ŋgrɤl.\cmn 野猫和家猫一模一样,但是野猫生活在森林里和岩洞里,吃比自己小的动物,比家猫大一些,强一些。颜色是灰色,上面有黑色的斑纹。有时候会和家猫交配。\end{exemple}
\end{entrée}

\begin{entrée}
\vedette{\hypertarget{ⒺxtɯtⒽ1}{\papi{ xtɯt}}}\markboth{xtɯt}{}\homonyme{1}\classe{vs}
\paradigme{\textit{dir :} \jya nɯ-}
\begin{définition}\fra court\end{définition}
\begin{définition}\cmn 短\end{définition}
\begin{exemple}\jya ɯ-phoŋbu ɲɯ-xtɯt\cmn 他的身体很小\end{exemple}
\begin{exemple}\jya ɕoŋtɕa ɲɯ-xtɯt\cmn 木料很短\end{exemple}
\begin{relation-sémantique}\confer{
\hyperlink{Ⓔnɤxtɯt}{\textit{ \papi{nɤxtɯt}}}
}\end{relation-sémantique}
\begin{relation-sémantique}\antonyme{
\hyperlink{Ⓔzri}{\textit{ \papi{zri}}}
}\end{relation-sémantique}
\begin{relation-sémantique}\antonyme{
\hyperlink{Ⓔrɲɟi}{\textit{ \papi{rɲɟi}}}
}\end{relation-sémantique}
\begin{relation-sémantique}\confer{
\hyperlink{Ⓔxtɯrɲɟi}{\textit{ \papi{xtɯrɲɟi}}}
}\end{relation-sémantique}\begin{sous-entrée}
\vedette{\hypertarget{}{\papi{ ɣɤxtɯt}}}\markboth{ɣɤxtɯt}{}\classe{vt}
\paradigme{\textit{dir :} \jya nɯ-}
\paradigme{\textit{dir :} \jya \_}
\begin{définition}\ 
\begin{déclaration}\grammar{caus}\end{déclaration}\end{définition}
\begin{définition}\fra raccourcir\end{définition}
\begin{définition}\cmn 弄短
\begin{déclaration}\use{用\stylefv{tɤ-}表示“穿短”(把袖子卷起来),\stylefv{nɯ-}表示“剪短”}\end{déclaration}\end{définition}
\begin{exemple}\jya ɯ-ŋga to-ɣɤxtɯt\cmn 他衣服穿得很短\end{exemple}
\begin{exemple}\jya nɤʑo tɕheme tɯ-ɕti tɕe, nɤ-ŋga kɤ-ɣɤxtɯt mɤ-ra\cmn 你是女孩子,衣服不能穿得太短\end{exemple}
\begin{exemple}\jya aʑo mɯ-to-rɯndzaŋspa-a tɕe, tɯ-ŋga kɤ-qrɯ ɲɤ-ɣɤxtɯt-a\cmn 我不小心把衣服裁得很短\end{exemple}
\end{sous-entrée}\begin{sous-entrée}
\vedette{\hypertarget{}{\papi{ sɯxtɯt}}}\markboth{sɯxtɯt}{}\classe{vt}
\begin{définition}\ 
\begin{déclaration}\grammar{caus}\end{déclaration}\end{définition}
\begin{définition}\fra raccourcir\end{définition}
\begin{définition}\cmn 弄短\end{définition}
\begin{exemple}\jya tɤ-rʑaʁ kɤ-sɯxtɯt khɯ\cmn 可以把时间缩短\end{exemple}
\end{sous-entrée}\begin{sous-entrée}
\vedette{\hypertarget{}{\papi{ zɣɤxtɯt}}}\markboth{zɣɤxtɯt}{}\classe{vt}
\paradigme{\textit{dir :} \jya nɯ-}
\begin{définition}\fra raccourcir avec, faire raccourcir\end{définition}
\begin{définition}\cmn 使弄短;用……弄短\end{définition}
\begin{exemple}\jya a-ŋga ɲɯ-zri tɕe, nɯ-zɣɤxtɯt-a\cmn 我的衣服太长,就请人弄短了\end{exemple}
\end{sous-entrée}\end{entrée}

\begin{entrée}
\vedette{\hypertarget{Ⓔxɯβxɯβ}{\papi{ xɯβxɯβ}}}\markboth{xɯβxɯβ}{}
\classe{idph.2}\acception{1}
\begin{définition}\fra rose\end{définition}
\begin{définition}\cmn 粉红状\end{définition}\acception{2}
\begin{définition}\fra chaud\end{définition}
\begin{définition}\cmn 形容(天气) 暖暖的\end{définition}
\begin{exemple}\jya kɯ-ɣɯrni xɯβxɯβ ci ɲɯ-ŋu\cmn 是粉红色的\end{exemple}
\begin{exemple}\jya tɯrme ɯ-rŋa kɯ-ɣɯrni xɯβxɯβ ci ɲɯ-ŋu\cmn 那个人的脸是粉红的\end{exemple}
\begin{exemple}\jya kha jɤ-azɣɯt-a tɕe, xɯβxɯβ ɲɯ-mpja\cmn 我到家里了,很暖\end{exemple}\begin{sous-entrée}
\vedette{\hypertarget{}{\papi{ ɣɤxɯβxɯβ}}}\markboth{ɣɤxɯβxɯβ}{}\classe{vi}
\begin{exemple}\jya ɲɯ-mɤrtsaβ ɲɯ-ɣɤxɯβxɯβ\cmn 很辣\end{exemple}
\begin{exemple}\jya ɯ-mbrɯ ɲɯ-ɣɤxɯβxɯβ ʑo ɲɯ-ŋgɯ\cmn 他很生气\end{exemple}
\end{sous-entrée}\begin{sous-entrée}
\vedette{\hypertarget{}{\papi{ sɤxɯβxɯβ}}}\markboth{sɤxɯβxɯβ}{}\classe{vt}
\begin{exemple}\jya qale ɲɯ-sɤxɯβxɯβ\cmn 风在吹到脸上(很冷的感觉)\end{exemple}
\end{sous-entrée}\begin{sous-entrée}
\vedette{\hypertarget{}{\papi{ xɯβ}}}\markboth{xɯβ}{}\classe{idph.1}
\begin{exemple}\jya ɯ-mbrɯ xɯβ ʑo tɤ-ŋgɯ\cmn 他一下子就生气了\end{exemple}
\end{sous-entrée}\begin{sous-entrée}
\vedette{\hypertarget{}{\papi{ xɯβnɤlɯβ}}}\markboth{xɯβnɤlɯβ}{}\classe{idph.4}
\end{sous-entrée}\begin{sous-entrée}
\vedette{\hypertarget{}{\papi{ xɯβnɤxɯβ}}}\markboth{xɯβnɤxɯβ}{}\classe{idph.3}
\begin{définition}\fra douleur lancinante\end{définition}
\begin{définition}\cmn 一阵一阵地痛(没有出血)\end{définition}
\begin{exemple}\jya a-βri xɯβnɤxɯβ ɲɯ-ti\cmn 我身上一阵一阵地痛\end{exemple}
\begin{exemple}\jya mtshalu kɯ kɤ́-wɣ-mtsɯɣ-a tɕe, a-βri xɯβnɤxɯβ ɲɯ-ti\cmn 我被荨麻刺到了,一阵一阵地痛\end{exemple}
\end{sous-entrée}\begin{sous-entrée}
\vedette{\hypertarget{}{\papi{ xɯwɯwi}}}\markboth{xɯwɯwi}{}\classe{idph.6}
\begin{exemple}\jya xɯwɯwi ʑo ɲɯ-mpja\cmn 慢慢地暖起来\end{exemple}
\end{sous-entrée}\end{entrée}

\begin{entrée}
\vedette{\hypertarget{Ⓔxɯchɯcho}{\papi{ xɯchɯcho}}}\markboth{xɯchɯcho}{}
\classe{intj}
\begin{définition}\fra soupir de fatigue\end{définition}
\begin{définition}\cmn 表示自己很累的感叹声\end{définition}
\begin{exemple}\jya nɯ-kɯ-ɲat tɕe xɯchɯcho tu-kɯ-ti ŋu\cmn 
累了就说“\stylefv{xɯchɯcho}”
\end{exemple}\end{entrée}

\begin{entrée}
\vedette{\hypertarget{Ⓔxɯŋxɯŋ}{\papi{ xɯŋxɯŋ}}}\markboth{xɯŋxɯŋ}{}
\classe{idph.2}
\begin{définition}\fra claire (pièce)\end{définition}
\begin{définition}\cmn 形容(房间)明亮\end{définition}
\begin{exemple}\jya kha ɲɯ-fsoʁ kɯ xɯŋxɯŋ ʑo\cmn 房子很明亮\end{exemple}
\begin{exemple}\jya ɲɯ-qarŋe xɯŋxɯŋ ʑo\cmn 很黄\end{exemple}
\begin{exemple}\jya tɤŋe ko-ntɕhɤr, xɯŋxɯŋ ʑo kha ɲɯ-fsoʁ\cmn 太阳发光,(照得)房间很明亮\end{exemple}\begin{sous-entrée}
\vedette{\hypertarget{}{\papi{ xɯŋɯŋi}}}\markboth{xɯŋɯŋi}{}\classe{idph.7}
\begin{exemple}\jya tɤŋe xɯŋɯŋi ʑo pɯ-ɣe\cmn 太阳慢慢地下山了,很明亮\end{exemple}
\begin{exemple}\jya tɤŋe xɯŋɯŋi ʑo to-nɯ-ɬoʁ\cmn 太阳慢慢地出来了,很明亮\end{exemple}
\begin{relation-sémantique}\confer{
\hyperlink{Ⓔʂɯŋʂɯŋ}{\textit{ \papi{ʂɯŋʂɯŋ}}}
}\end{relation-sémantique}
\end{sous-entrée}\end{entrée}

\begin{entrée}
\vedette{\hypertarget{Ⓔxɯrxɯr}{\papi{ xɯrxɯr}}}\markboth{xɯrxɯr}{}
\classe{idph.2}
\begin{définition}\fra rond\end{définition}
\begin{définition}\cmn 圆形\end{définition}
\begin{exemple}\jya tɤŋe xɯrxɯr ʑo ɲɯ-ɤrtɯm\cmn 太阳是圆的\end{exemple}
\begin{exemple}\jya mbrɤsɤm xɯrxɯr ɲɯ-ɤrtɯm\cmn 晒粮食的簸箕是圆形的\end{exemple}\begin{sous-entrée}
\vedette{\hypertarget{}{\papi{ ɣɤxɯrxɯr}}}\markboth{ɣɤxɯrxɯr}{}\classe{vi}
\begin{définition}\fra qui tourne\end{définition}
\begin{définition}\cmn 在转动\end{définition}
\begin{exemple}\jya mkhɯrlu ɲɯ-ɣɤxɯrxɯr ɲɯ-mtɕɯr\cmn 轮子在转动\end{exemple}
\begin{relation-sémantique}\confer{
\hyperlink{Ⓔtɯ-tɤxɯr}{\textit{ \papi{tɯ-tɤxɯr}}}
}\end{relation-sémantique}
\begin{relation-sémantique}\synonyme{
\hyperlink{Ⓔsɯrsɯr}{\textit{ \papi{sɯrsɯr}}}
}\end{relation-sémantique}
\end{sous-entrée}\begin{sous-entrée}
\vedette{\hypertarget{}{\papi{ xɯrinɤxɯri}}}\markboth{xɯrinɤxɯri}{}\classe{idph.8}
\begin{définition}\fra qui tourne vite\end{définition}
\begin{définition}\cmn 旋转得很快的样子\end{définition}
\begin{exemple}\jya xɯrinɤxɯri ʑo ko-mtɕɯr\cmn 转得很快了\end{exemple}
\end{sous-entrée}\begin{sous-entrée}
\vedette{\hypertarget{}{\papi{ xɯrnɤxɯr}}}\markboth{xɯrnɤxɯr}{}\classe{idph.3}
\begin{définition}\fra qui tourne\end{définition}
\begin{définition}\cmn 在转动\end{définition}
\end{sous-entrée}\end{entrée}

\begin{entrée}
\vedette{\hypertarget{Ⓔxwɤrnɤxwɤr}{\papi{ xwɤrnɤxwɤr}}}\markboth{xwɤrnɤxwɤr}{} (\variante{xwaranɤxwara}) 
\classe{idph.3}
\begin{définition}\fra qui tourne vite\end{définition}
\begin{définition}\cmn 形容转得很快的样子\end{définition}
\begin{exemple}\jya xwɤrxwɤr nɤ xwɤrxwɤr ʑo ɲɯ-mtɕɯr\cmn 哗哗哗地飞快旋转\end{exemple}\end{entrée}

\newpage\caractère{χ}

\begin{entrée}
\vedette{\hypertarget{Ⓔχajaŋ}{\papi{ χajaŋ}}}\markboth{χajaŋ}{}\classe{n}
\begin{définition}\fra aluminium\end{définition}
\begin{définition}\cmn 铝
\begin{déclaration} \étymologie{\papi{ha.jaŋ}}\end{déclaration}\end{définition}\end{entrée}

\begin{entrée}
\vedette{\hypertarget{Ⓔχajχaj}{\papi{ χajχaj}}}\markboth{χajχaj}{}\classe{idpd.2}
\begin{définition}\fra attendre (sans bouger)\end{définition}
\begin{définition}\cmn 呆呆地瞪着\end{définition}
\begin{exemple}\jya dɯxpa ma χajχaj ʑo ɲɯ́-wɣ-nɤjo-a\cmn 我呆呆地盼着他\end{exemple}\end{entrée}

\begin{entrée}
\vedette{\hypertarget{Ⓔχaŋχaŋ}{\papi{ χaŋχaŋ}}}\markboth{χaŋχaŋ}{}
\classe{idph.2}
\begin{définition}\fra un peu orange\end{définition}
\begin{définition}\cmn 形容淡橘黄色\end{définition}
\begin{exemple}\jya ɲɯ-qarŋe χaŋχaŋ ʑo\cmn 是淡黄的\end{exemple}
\begin{exemple}\jya tɯrmɯkha tɕe, prɤɲi χaŋχaŋ ɲɤ-k-ɤβzu-ci\cmn 傍晚的时候,晚霞带有橘黄色\end{exemple}\end{entrée}

\begin{entrée}
\vedette{\hypertarget{Ⓔχawo}{\papi{ χawo}}}\markboth{χawo}{}\classe{intj}
\begin{définition}\fra expression du regret, de l'espoir\end{définition}
\begin{définition}\cmn 表示惋惜、希望\end{définition}
\begin{exemple}\jya χawo ʑo nɤ-ɕqhe a-nɯ-me kɯ!\cmn 唉,真希望你的咳嗽会治好\end{exemple}\end{entrée}

\begin{entrée}
\vedette{\hypertarget{Ⓔχɤβ}{\papi{ χɤβ}}}\markboth{χɤβ}{}
\classe{vt}
\paradigme{\textit{dir :} \jya kɤ-}
\paradigme{\textit{dir :} \jya lɤ-}
\begin{définition}\fra boire complètement\end{définition}
\begin{définition}\cmn 喝干;喝尽最后一滴\end{définition}
\begin{exemple}\jya kɤ-χɤβ kɯ-fse ci ɲɯ-ɕti\cmn 只剩下一点点\end{exemple}
\begin{exemple}\jya a-sŋɯro lɤ-χaβ-a\cmn 我吸了气\end{exemple}
\begin{exemple}\jya cha kɤ-χaβ-a\cmn 我把酒喝光了\end{exemple}\end{entrée}

\begin{entrée}
\vedette{\hypertarget{Ⓔχɤjnɤχɤj}{\papi{ χɤjnɤχɤj}}}\markboth{χɤjnɤχɤj}{}\classe{idph.3}
\begin{définition}\fra essouflé\end{définition}
\begin{définition}\cmn 喘气\end{définition}
\begin{exemple}\jya tɤ-rɟɯɣ-a tɕe χɤjnɤχɤj ʑo tɤ-tɯt-a pɯ-ra\cmn 我跑得气喘吁吁的\end{exemple}
\begin{relation-sémantique}\synonyme{
\hyperlink{Ⓔχinɤχi}{\textit{ \papi{χinɤχi}}}
}\end{relation-sémantique}\end{entrée}

\begin{entrée}
\vedette{\hypertarget{Ⓔχɤlnɤχɤl}{\papi{ χɤlnɤχɤl}}}\markboth{χɤlnɤχɤl}{}\classe{idph.3}
\begin{définition}\fra marcher d'un pas assuré\end{définition}
\begin{définition}\cmn 形容走路步伐稳健的样子\end{définition}\end{entrée}

\begin{entrée}
\vedette{\hypertarget{Ⓔχɤlχɤl}{\papi{ χɤlχɤl}}}\markboth{χɤlχɤl}{}\classe{idph.2}\acception{1}
\begin{définition}\fra relâché, guéri, rassuré\end{définition}
\begin{définition}\cmn 形容捆绑以后解开的宽松感、痊愈、放心的感觉\end{définition}
\begin{exemple}\jya khapa fsapaʁ pɯ-rɤʑi-nɯ ɕti tɕe, ɕɯ-nam-a pɯ-ŋu ri, khapa pɯ-azɣɯt-a tɕe, χɤlχɤl ʑo jo-ɕe-nɯ\cmn 牲畜原来在楼下,我正要去赶它们的时候,它们就消失得无影无踪了\end{exemple}
\begin{exemple}\jya jɯfɕɯr tɤ-pɤtso wuma ʑo pɯ-nɯzdɯɣ-a ri, jɤ-azɣɯt tɕe, a-sɯm χɤlχɤl ʑo ɲo-pa\cmn 昨天我很担心小孩子,他到了之后我就放心了\end{exemple}
\begin{exemple}\jya χɤlχɤl ʑo to-mna\cmn 完全痊愈\end{exemple}\acception{2}
\begin{définition}\fra disparaître\end{définition}
\begin{définition}\cmn 消失得无影无踪(比较嘈杂的人或者动物)\end{définition}
\begin{exemple}\jya χɤlχɤl chɤ-k-ɤrɕo-ci\cmn 完全用完了\end{exemple}
\begin{exemple}\jya χɤlχɤl to-ɕkɯt\cmn 他吃完了\end{exemple}
\begin{exemple}\jya slama ra jɤ-anɯri-nɯ, sloχpɯn ɯ-rkɯ χɤlχɤl ʑo ɲɯ-pa\cmn 当学生都走了之后,老师觉得心里空落落的\end{exemple}
\begin{exemple}\jya a-kɯ-mŋɤm χɤlχɤl ʑo tɤ-pa\cmn 我的病完全痊愈了\end{exemple}
\begin{relation-sémantique}\confer{
\hyperlink{Ⓔɣɤχɤlχɤl}{\textit{ \papi{ɣɤχɤlχɤl}}}
}\end{relation-sémantique}\end{entrée}

\begin{entrée}
\vedette{\hypertarget{Ⓔχɤnku}{\papi{ χɤnku}}}\markboth{χɤnku}{}\classe{n}
\begin{définition}\fra casserole en fer pour cuire la pâté des cochons\end{définition}
\begin{définition}\cmn 煮猪草的生铁锅\end{définition}
\begin{exemple}\jya χɤnku nɯ tɯ-thɯ kɯ-wxti ci ŋu, ɯ-spa khru ɲɯ-ŋu, mba tɕe wxti ri mɤ-rʑi, paʁndza sɤ-sqa wuma ʑo pe. χɤnku nɯ ɯ-spa khro ɲɯ-ŋu tɕe, khru mɤ-ngɯt ma ɲɯ-ndoʁ tɕe kɤ-ta kɤ-mɟa ra ɣɯ-mdzoz tsa ra.\cmn 
\stylefv{χɤnku}是一种大锅,用生铁铸成。因为很薄,所以大而轻,最适合于煮猪草。因为用生铁做成,所以不结实,很脆,所以放下拿起的时候要特别小心。
\end{exemple}
\end{entrée}

\begin{entrée}
\vedette{\hypertarget{Ⓔχɤnχɤn}{\papi{ χɤnχɤn}}}\markboth{χɤnχɤn}{}\classe{idph.2}
\begin{définition}\fra large et vide\end{définition}
\begin{définition}\cmn 形容又大又空的样子\end{définition}
\begin{exemple}\jya χɤnχɤn ʑo ɲɯ-ɤz-nɤjo\cmn 他呆呆地站在那里等着(望着一个方向不动)\end{exemple}
\begin{exemple}\jya kha ɯ-tɯ-wxti kɯ χɤnχɤn ʑo ɲɯ-pa\cmn 房子又大又空\end{exemple}
\begin{exemple}\jya nɤ-khɯtsa ɯ-tɯ-wxti nɯ χɤnχɤn kɯ\cmn 你的碗很大!\end{exemple}
\begin{exemple}\jya χɤnχɤn ʑo ma-tɤ-tɯ-ʑɣɤstu\cmn 不要傻乎乎的那样\end{exemple}
\begin{relation-sémantique}\synonyme{
\hyperlink{Ⓔχajχaj}{\textit{ \papi{χajχaj}}}
}\end{relation-sémantique}\end{entrée}

\begin{entrée}
\vedette{\hypertarget{Ⓔχɤpɤχɤle}{\papi{ χɤpɤχɤle}}}\markboth{χɤpɤχɤle}{}\classe{n}
\begin{définition}\fra extraverti\end{définition}
\begin{définition}\cmn 外向\end{définition}
\begin{exemple}\jya nɯki tɯrme χɤpɤχɤle ci ɯ-qhoχpa kɤ-rku kɯ-me ci ɕti\cmn 这个人比较外向,有什么事都不会放在心里的。\end{exemple}
\end{entrée}

\begin{entrée}
\vedette{\hypertarget{Ⓔχcha}{\papi{ χcha}}}\markboth{χcha}{}\classe{n}
\begin{définition}\fra droite\end{définition}
\begin{définition}\cmn 右边\end{définition}\end{entrée}

\begin{entrée}
\vedette{\hypertarget{Ⓔχchoʁe}{\papi{ χchoʁe}}}\markboth{χchoʁe}{}\classe{adv}
\paradigme{\textit{emphatic :} \jya χchɯχchoʁɯʁe}
\begin{définition}\fra à droite et à gauche\end{définition}
\begin{définition}\cmn 左右\end{définition}
\begin{exemple}\jya jɯfɕɯr a-jaʁ χchoʁe ʑo laχtɕha tɤ-ndo-t-a\cmn 昨天我两只手都拿了东西。\end{exemple}\end{entrée}

\begin{entrée}
\vedette{\hypertarget{Ⓔχcoŋkroŋ}{\papi{ χcoŋkroŋ}}}\markboth{χcoŋkroŋ}{}
\classe{n}
\begin{définition}\fra en tailleur\end{définition}
\begin{définition}\cmn 盘腿\end{définition}
\begin{exemple}\jya χcoŋkroŋ ɲɤ-βzu\cmn 他盘腿坐了\end{exemple}
\begin{exemple}\jya tɤ-tɕɯ kɯ χcoŋkroŋ ɲɯ-βze ŋgrɤl, tɕheme kɯ ndzɯpe ɲɯ-βze ŋgrɤl\cmn 男子盘腿坐,女子跪着坐\end{exemple}
\begin{exemple}\jya thamtham tɕe, tɕheme ra kɯ χcoŋkroŋ ɲɯ-kɯ-βzu tu, ɕɯŋgɯ tɕe, tɕheme kɯ χcoŋkroŋ ɲɯ-βze mɯ-pjɤ-jɤɣ\cmn 现在,有些女子会盘腿坐,以前是不允许的\end{exemple}\end{entrée}

\begin{entrée}
\vedette{\hypertarget{Ⓔχcrɯχcri}{\papi{ χcrɯχcri}}}\markboth{χcrɯχcri}{} (\variante{χcɯχcri}) 
\classe{idph.2}
\begin{définition}\fra dilué, peu épais\end{définition}
\begin{définition}\cmn 形容流体稀\end{définition}
\begin{exemple}\jya nɯŋa ɯ-qe χcɯχcri ʑo ɲɯ-pa\cmn 牛屎很稀\end{exemple}
\begin{relation-sémantique}\confer{
\hyperlink{Ⓔscrɯscri}{\textit{ \papi{scrɯscri}}}
}\end{relation-sémantique}
\begin{relation-sémantique}\confer{
\hyperlink{Ⓔɲcriɲcri}{\textit{ \papi{ɲcriɲcri}}}
}\end{relation-sémantique}\end{entrée}

\begin{entrée}
\vedette{\hypertarget{Ⓔχcɯχcri}{\papi{ χcɯχcri}}}\markboth{χcɯχcri}{}\classe{idph.2}
\begin{définition}\fra gras et mou\end{définition}
\begin{définition}\cmn 形容胖而软的样子\end{définition}
\begin{relation-sémantique}\confer{
\hyperlink{Ⓔʁɟɯʁɟri}{\textit{ \papi{ʁɟɯʁɟri}}}
}\end{relation-sémantique}\end{entrée}

\begin{entrée}
\vedette{\hypertarget{Ⓔχɕu}{\papi{ χɕu}}}\markboth{χɕu}{}
\classe{vs}
\paradigme{\textit{dir :} \jya tɤ-}
\begin{définition}\fra fort, résistant\end{définition}
\begin{définition}\cmn 健壮;力量大;有耐性\end{définition}
\begin{exemple}\jya jla ɲɯ-χɕu\cmn 犏牛很强壮\end{exemple}
\begin{exemple}\jya mkhɯrlu ɲɯ-χɕu\cmn 汽车马力大\end{exemple}
\begin{exemple}\jya pɯ-tɯ-χɕu\cmn 谢谢\end{exemple}
\begin{relation-sémantique}\confer{
\hyperlink{Ⓔχɕuχɕe}{\textit{ \papi{χɕuχɕe}}}
}\end{relation-sémantique}\end{entrée}

\begin{entrée}
\vedette{\hypertarget{Ⓔχɕaʁ}{\papi{ χɕaʁ}}}\markboth{χɕaʁ}{}\classe{n}
\begin{définition}\fra morceau de bois coupé à la hache\end{définition}
\begin{définition}\cmn 劈好了的木料(房背、走缘当石板用)
\begin{déclaration} \étymologie{\papi{gɕag}}\end{déclaration}\end{définition}\end{entrée}

\begin{entrée}
\vedette{\hypertarget{Ⓔχɕaʁ}{\papi{ χɕaʁ}}}\markboth{χɕaʁ}{}\classe{vi}
\paradigme{\textit{dir :} \jya nɯ-}
\begin{définition}\fra décéder (honorifique, réservé aux lamas et aux sprulsku)\end{définition}
\begin{définition}\cmn 圆寂(敬语)
\begin{déclaration} \étymologie{\papi{gɕegs}}\end{déclaration}\end{définition}\end{entrée}

\begin{entrée}
\vedette{\hypertarget{Ⓔχɕɤβ}{\papi{ χɕɤβ}}}\markboth{χɕɤβ}{}
\classe{vs}
\paradigme{\textit{dir :} \jya tɤ-}\acception{1}
\begin{définition}\fra excessif (parole)\end{définition}
\begin{définition}\cmn 夸张(话)\end{définition}
\begin{exemple}\jya ɯ-rju ɲɯ-χɕɤβ\cmn 他说的话很夸张\end{exemple}\acception{2}
\begin{définition}\fra éclatante, vive (couleur)\end{définition}
\begin{définition}\cmn 鲜艳(颜色)\end{définition}
\begin{exemple}\jya ɯ-mdoʁ ɲɯ-χɕɤβ\cmn 颜色很鲜艳(红色、黄色)\end{exemple}\acception{3}
\begin{définition}\fra fort (bruit)\end{définition}
\begin{définition}\cmn 响;吵\end{définition}
\begin{exemple}\jya ɯ-zgra ɲɯ-χɕɤβ\cmn 声音很响(很吵)\end{exemple}\acception{4}
\begin{définition}\fra forte (odeur)\end{définition}
\begin{définition}\cmn 浓(气味)\end{définition}
\begin{exemple}\jya ɯ-di ɲɯ-χɕɤβ\cmn 气味很浓\end{exemple}
\begin{relation-sémantique}\confer{
\hyperlink{Ⓔrɯχɕɯχɕɤβ}{\textit{ \papi{rɯχɕɯχɕɤβ}}}
}\end{relation-sémantique}\end{entrée}

\begin{entrée}
\vedette{\hypertarget{Ⓔχɕɤl}{\papi{ χɕɤl}}}\markboth{χɕɤl}{}\classe{n}
\begin{définition}\fra verre\end{définition}
\begin{définition}\cmn 玻璃
\begin{déclaration} \étymologie{\papi{ɕel}}\end{déclaration}\end{définition}\end{entrée}

\begin{entrée}
\vedette{\hypertarget{Ⓔχɕɤlkara}{\papi{ χɕɤlkara}}}\markboth{χɕɤlkara}{}\classe{n}
\begin{définition}\fra sucre en morceau\end{définition}
\begin{définition}\cmn 冰糖
\begin{déclaration} \étymologie{\papi{ɕel.dkar}}\end{déclaration}\end{définition}\end{entrée}

\begin{entrée}
\vedette{\hypertarget{Ⓔχɕɤlmdoŋ}{\papi{ χɕɤlmdoŋ}}}\markboth{χɕɤlmdoŋ}{}\classe{n}
\begin{définition}\fra lunette, télescope\end{définition}
\begin{définition}\cmn 望远镜\end{définition}\end{entrée}

\begin{entrée}
\vedette{\hypertarget{Ⓔχɕɤlmɯɣ}{\papi{ χɕɤlmɯɣ}}}\markboth{χɕɤlmɯɣ}{}\classe{n}
\paradigme{\textit{comit :} \jya kɤ́χɕɤlmɯlmɯɣ}
\begin{définition}\fra lunettes\end{définition}
\begin{définition}\cmn 眼镜
\begin{déclaration} \étymologie{\papi{ɕel.mig}}\end{déclaration}\end{définition}
\begin{exemple}\jya χɕɤlmɯɣ tɤ-nɯ-ta-t-a (tɤ-nɯ-tshoʁ-a)\cmn 我戴上了眼镜\end{exemple}\end{entrée}

\begin{entrée}
\vedette{\hypertarget{Ⓔχɕɤlzgoŋ}{\papi{ χɕɤlzgoŋ}}}\markboth{χɕɤlzgoŋ}{}\classe{n}
\begin{définition}\fra miroir\end{définition}
\begin{définition}\cmn 镜子
\begin{déclaration} \étymologie{\papi{ɕel.sgo}}\end{déclaration}\end{définition}
\begin{relation-sémantique}\synonyme{
\hyperlink{Ⓔkɯsɤɣru}{\textit{ \papi{kɯsɤɣru}}}
}\end{relation-sémantique}\end{entrée}

\begin{entrée}
\vedette{\hypertarget{Ⓔχɕitka}{\papi{ χɕitka}}}\markboth{χɕitka}{}\classe{n}
\begin{définition}\fra printemps\end{définition}
\begin{définition}\cmn 春天
\begin{déclaration} \étymologie{\papi{dpʲid.ka}}\end{déclaration}\end{définition}\end{entrée}

\begin{entrée}
\vedette{\hypertarget{Ⓔχɕoʁ}{\papi{ χɕoʁ}}}\markboth{χɕoʁ}{}
\classe{vt}
\paradigme{\textit{dir :} \jya \_}
\begin{définition}\fra tirer\end{définition}
\begin{définition}\cmn 抽出\end{définition}
\begin{exemple}\jya ndʑu tɤ-χɕoʁ-a\cmn 我把筷子抽出来了\end{exemple}
\begin{exemple}\jya si thɯ-χɕoʁ\cmn (从柴堆里)取一根\end{exemple}
\begin{exemple}\jya scapa la-χɕoʁ\cmn 他把剑抽出来了\end{exemple}\end{entrée}

\begin{entrée}
\vedette{\hypertarget{Ⓔχɕu,rnaʁ}{\papi{ χɕu,rnaʁ}}}\markboth{χɕu,rnaʁ}{}
\begin{définition}\fra merci beaucoup\end{définition}
\begin{définition}\cmn 万分感谢\end{définition}
\begin{exemple}\jya pɯ-tɯ-χɕu pɯ-tɯ-rnaʁ\cmn 感谢你了\end{exemple}
\begin{exemple}\jya pɯ-χɕu pɯ-rnaʁ\cmn 万分感谢他\end{exemple}
\begin{relation-sémantique}\ComponentA{\classe{vs}
\hyperlink{Ⓔχɕu}{\textit{ \papi{χɕu}}}
}\end{relation-sémantique}
\begin{relation-sémantique}\ComponentB{\classe{vs}
\hyperlink{Ⓔrnaʁ}{\textit{ \papi{rnaʁ}}}
}\end{relation-sémantique}\end{entrée}

\begin{entrée}
\vedette{\hypertarget{Ⓔχɕɯldɤn}{\papi{ χɕɯldɤn}}}\markboth{χɕɯldɤn}{}\classe{n}
\begin{définition}\fra en sécurité\end{définition}
\begin{définition}\cmn 安全;安康\end{définition}
\begin{exemple}\jya ɯʑo ɯ-kha ra mɤ-kɯ-pe ku-me tɕe, χɕɯldɤn ɕti\cmn 他的家里人都好,平安无事\end{exemple}
\begin{relation-sémantique}\confer{
\hyperlink{Ⓔaχɕɯldɤn}{\textit{ \papi{aχɕɯldɤn}}}
}\end{relation-sémantique}\end{entrée}

\begin{entrée}
\vedette{\hypertarget{Ⓔχɕɯn}{\papi{ χɕɯn}}}\markboth{χɕɯn}{}
\classe{vi}\acception{1}
\paradigme{\textit{dir :} \jya pɯ-}
\begin{définition}\fra sain et sauf\end{définition}
\begin{définition}\cmn 安全
\begin{déclaration} \étymologie{\papi{gɕin}}\end{déclaration}\end{définition}
\begin{exemple}\jya kɤ-χɕɯn lo-zɣɯt-ndʑi\cmn 他们俩安全到达了\end{exemple}
\begin{exemple}\jya kɤ-χɕɯn kɤ-nɯʑɯβ\cmn 安心睡觉吧!\end{exemple}\acception{2}
\begin{définition}\fra être fini (travail)\end{définition}
\begin{définition}\cmn 完;结束(工作)
\begin{déclaration}\use{沙尔宗方言}\end{déclaration}\end{définition}
\begin{exemple}\jya ta-ma pjɤ-χɕɯn\cmn 工作完了\end{exemple}\end{entrée}

\begin{entrée}
\vedette{\hypertarget{Ⓔχɕɯnrʑi}{\papi{ χɕɯnrʑi}}}\markboth{χɕɯnrʑi}{}\classe{n}
\begin{définition}\fra Yama\end{définition}
\begin{définition}\cmn 阎王
\begin{déclaration} \étymologie{\papi{gɕen.rdʑe}}\end{déclaration}\end{définition}\end{entrée}

\begin{entrée}
\vedette{\hypertarget{Ⓔχɕɯχɕi}{\papi{ χɕɯχɕi}}}\markboth{χɕɯχɕi}{}\classe{idph.2}
\begin{définition}\fra qui écoute en silence\end{définition}
\begin{définition}\cmn 形容悄悄地听的样子\end{définition}
\begin{exemple}\jya nɯtɕu nɯ-rkɯ χɕɯχɕi ʑo pjɤ-tɯ-sɤŋo\cmn 你在那里悄悄地听他们交谈\end{exemple}\end{entrée}

\begin{entrée}
\vedette{\hypertarget{Ⓔχɕuχɕe}{\papi{ χɕuχɕe}}}\markboth{χɕuχɕe}{}\classe{vs}
\begin{définition}\fra fort, robuste\end{définition}
\begin{définition}\cmn 身强力壮\end{définition}
\begin{exemple}\jya thamtham ʁʑɯnɯ ɲɯ-ɕti tɕe, wuma ʑo ɲɯ-χɕuχɕe\cmn 他是青年,非常强壮\end{exemple}
\begin{relation-sémantique}\confer{
\hyperlink{Ⓔχɕu}{\textit{ \papi{χɕu}}}
}\end{relation-sémantique}\end{entrée}

\begin{entrée}
\vedette{\hypertarget{Ⓔχinɤχi}{\papi{ χinɤχi}}}\markboth{χinɤχi}{}
\classe{idph.3}
\begin{définition}\fra éssouflé\end{définition}
\begin{définition}\cmn 形容气喘吁吁的样子
\end{définition}
\begin{exemple}\jya ɯʑo jo-rɟɯɣ pjɤ-ra tɕe, χinɤχi ʑo ɲɯ-ʑɣɤstu jɤ-azɣɯt\cmn 因为他要跑过来,气喘吁吁地到了\end{exemple}
\begin{exemple}\jya ɲɤ-nɤɴqa, ɯ-fkur pjɤ-rʑi tɕe, χinɤχi ʑo jɤ-azɣɯt\cmn 因为他很辛苦,负担很重,气喘吁吁地到了\end{exemple}
\begin{relation-sémantique}\synonyme{
\hyperlink{Ⓔqhinɤqhi}{\textit{ \papi{qhinɤqhi}}}
}\end{relation-sémantique}\end{entrée}

\begin{entrée}
\vedette{\hypertarget{Ⓔχɲɤβχɲɤβ}{\papi{ χɲɤβχɲɤβ}}}\markboth{χɲɤβχɲɤβ}{}\classe{idph.2}
\begin{définition}\fra mou et humide\end{définition}
\begin{définition}\cmn 形容软而潮湿的样子\end{définition}
\begin{relation-sémantique}\synonyme{
\hyperlink{Ⓔχɲɤχɲɤr}{\textit{ \papi{χɲɤχɲɤr}}}
}\end{relation-sémantique}\end{entrée}

\begin{entrée}
\vedette{\hypertarget{Ⓔχɲɤχɲɤr}{\papi{ χɲɤχɲɤr}}}\markboth{χɲɤχɲɤr}{}\classe{idph.2}
\begin{définition}\fra mou et humide\end{définition}
\begin{définition}\cmn 形容软而潮湿的样子\end{définition}
\begin{exemple}\jya a-ʑɯβ ɲɯ-ɣi tɕe a-phoŋbu ra χɲɤχɲɤr ʑo ɲɯ-pa\cmn 我睡意来了,觉得浑身软趴趴的\end{exemple}
\begin{relation-sémantique}\synonyme{
\hyperlink{Ⓔχɲɤβχɲɤβ}{\textit{ \papi{χɲɤβχɲɤβ}}}
}\end{relation-sémantique}\end{entrée}

\begin{entrée}
\vedette{\hypertarget{Ⓔχɲɯχɲi}{\papi{ χɲɯχɲi}}}\markboth{χɲɯχɲi}{}\classe{idph.2}\acception{1}
\begin{définition}\fra mou et en bouillie\end{définition}
\begin{définition}\cmn 形容又软又稀;晒嫣了的食物\end{définition}
\begin{définition}\jya \end{définition}
\begin{exemple}\jya kɯ-mpɯ kɯ-fse χɲɯχɲi kɯ-pa nɯra mɤ-rga-a\cmn 我不喜欢又的那些(食物)\end{exemple}\acception{2}
\begin{définition}\fra sans force\end{définition}
\begin{définition}\cmn 形容没有精神,没有力气的样子\end{définition}
\begin{exemple}\jya iɕqha tɯrme nɯ χɲɯχɲi kɯ-pa ci ɲɯ-ŋu\cmn 这个人没有精神\end{exemple}\end{entrée}

\begin{entrée}
\vedette{\hypertarget{Ⓔχoŋ}{\papi{ χoŋ}}}\markboth{χoŋ}{}\classe{idph.1}
\begin{définition}\fra qui tombe tout d'un coup dans un trou\end{définition}
\begin{définition}\cmn 突然踏空,掉进洞里\end{définition}
\begin{exemple}\jya ɯ-mi χoŋ ʑo pjɤ-ɕqhlɤt\cmn 他突然踏空了\end{exemple}\begin{sous-entrée}
\vedette{\hypertarget{}{\papi{ χoŋχoŋ}}}\markboth{χoŋχoŋ}{}\classe{idph.2}
\begin{définition}\fra ayant un grand trou\end{définition}
\begin{définition}\cmn 有洞(洞口很大)\end{définition}
\begin{exemple}\jya ɯ-ŋga χoŋχoŋ ʑo kɯ-pa ko-spoʁ\cmn 他衣服上有个大窟窿\end{exemple}\acception{2}
\begin{définition}\fra aube\end{définition}
\begin{définition}\cmn 天亮\end{définition}
\begin{exemple}\jya lo-fsoʁ χoŋχoŋ ri mɯ́j-tɯ-rɤru\cmn 天亮了,你还不起床\end{exemple}
\end{sous-entrée}\end{entrée}

\begin{entrée}
\vedette{\hypertarget{Ⓔχpa}{\papi{ χpa}}}\markboth{χpa}{}
\classe{vs}
\paradigme{\textit{dir :} \jya tɤ-}
\paradigme{\textit{dir :} \jya thɯ-}
\begin{définition}\fra fier, plein de confiance en soi\end{définition}
\begin{définition}\cmn 自豪,很有自信
\begin{déclaration} \étymologie{\papi{dpa}}\end{déclaration}\end{définition}
\begin{exemple}\jya tɯrme kɯ-χpa ci ɲɯ-ŋu\cmn 他是一个自豪的人\end{exemple}
\begin{exemple}\jya ɯ-sɯm lo-χpa\cmn 他变得骄傲了\end{exemple}
\begin{relation-sémantique}\confer{
\hyperlink{Ⓔznɤχpɯχpa}{\textit{ \papi{znɤχpɯχpa}}}
}\end{relation-sémantique}\end{entrée}

\begin{entrée}
\vedette{\hypertarget{Ⓔχpaχtshɤt}{\papi{ χpaχtshɤt}}}\markboth{χpaχtshɤt}{}\classe{n}
\begin{définition}\fra yojana\end{définition}
\begin{définition}\cmn 由旬,长度单位(非常远)
\begin{déclaration} \étymologie{\papi{dpag.tsʰad}}\end{déclaration}\end{définition}\end{entrée}

\begin{entrée}
\vedette{\hypertarget{Ⓔχpɤlwi}{\papi{ χpɤlwi}}}\markboth{χpɤlwi}{}\classe{n}
\begin{définition}\fra un motif bouddhique\end{définition}
\begin{définition}\cmn 一种佛教图纹
\begin{déclaration} \étymologie{\papi{dpal.beɦu}}\end{déclaration}\end{définition}\end{entrée}

\begin{entrée}
\vedette{\hypertarget{Ⓔχphjɤrχphjɤr}{\papi{ χphjɤrχphjɤr}}}\markboth{χphjɤrχphjɤr}{}\classe{idph.2}
\begin{définition}\fra fade\end{définition}
\begin{définition}\cmn 形容没有香味的\end{définition}
\begin{exemple}\jya tɤ-mthɯm kɤ́-wɣ-sqa a-kɤ-smi ɲɯ-ra ma nɯ maʁ nɤ tɤ́-wɣ-ndza tɕe χphjɤrχphjɤr ʑo ɲɯ-pa tɕe mɯ́j-mɯm\cmn 煮肉的的时候要煮熟,不然吃起来一点香味也没有,不好吃\end{exemple}\end{entrée}

\begin{entrée}
\vedette{\hypertarget{Ⓔχpi}{\papi{ χpi}}}\markboth{χpi}{}\classe{n}\acception{1}
\begin{définition}\fra histoire\end{définition}
\begin{définition}\cmn 故事
\begin{déclaration} \étymologie{\papi{dpe}}\end{déclaration}\end{définition}
\begin{exemple}\jya ɯʑo kɯ χpi wuma kɯ-mpɕɤr ci ɲɤ-sɤβzu\cmn 他编了一个很精彩的故事\end{exemple}\acception{2}
\begin{définition}\fra example\end{définition}
\begin{définition}\cmn 例子\end{définition}
\begin{exemple}\jya ɯ-χpi zɯ (= a-pɯ-ŋu nɤ)\cmn 例如\end{exemple}
\begin{exemple}\jya nɤ-χpi ci ku-te-a\cmn 我给你举个例子\end{exemple}
\begin{relation-sémantique}\confer{
\hyperlink{Ⓔta-χpi}{\textit{ \papi{ta-χpi}}}
}\end{relation-sémantique}
\begin{relation-sémantique}\confer{
\hyperlink{Ⓔraχpi}{\textit{ \papi{raχpi}}}
}\end{relation-sémantique}\end{entrée}

\begin{entrée}
\vedette{\hypertarget{Ⓔχpjɤt}{\papi{ χpjɤt}}}\markboth{χpjɤt}{}
\classe{vt}
\paradigme{\textit{dir :} \jya kɤ-}
\paradigme{\textit{dir :} \jya tɤ-}
\begin{définition}\fra observer\end{définition}
\begin{définition}\cmn 观察\end{définition}
\begin{exemple}\jya nɤ-zda kɤ-χpjɤt\cmn 你观察一下你的同伴\end{exemple}
\begin{exemple}\jya aʑo ku-kɯ-χpjat-a\cmn 你在观察我\end{exemple}
\begin{exemple}\jya nɯ kɤ-χpjɤt me nɤ\cmn 那个是说不准的\end{exemple}\begin{sous-entrée}
\vedette{\hypertarget{}{\papi{ aχpɯχpjɤt}}}\markboth{aχpɯχpjɤt}{}\classe{vi}
\begin{définition}\ 
\begin{déclaration}\grammar{recip}\end{déclaration}\end{définition}
\begin{définition}\fra se regarder les uns les autres\end{définition}
\begin{définition}\cmn 互相观察(行为)\end{définition}
\begin{exemple}\jya ma-tɯ-ɤχpɯχpjɤt-nɯ\cmn 你们不要互相学坏\end{exemple}
\end{sous-entrée}\begin{sous-entrée}
\vedette{\hypertarget{}{\papi{ naχpjɤt}}}\markboth{naχpjɤt}{}
\paradigme{\textit{dir :} \jya tɤ-}
\begin{définition}\ 
\begin{déclaration}\use{只出现在未然式和不定式}\end{déclaration}\end{définition}
\begin{exemple}\jya nɯ-βde a-tɤ-naχpjɤt\cmn 你放弃吧,没有必要坚持\end{exemple}
\begin{exemple}\jya mɯ-ɲɯ-ɣi nɤ a-tɤ-naχpjɤt\cmn 如果他不来就算了吧\end{exemple}
\begin{exemple}\jya kɤ-naχpjɤt mɤ-βze\cmn 不能这样就过去了(不能放弃)\end{exemple}
\end{sous-entrée}\begin{sous-entrée}
\vedette{\hypertarget{}{\papi{ nɯχpjɤt}}}\markboth{nɯχpjɤt}{}\classe{vt}
\paradigme{\textit{dir :} \jya tɤ-}
\begin{définition}\fra cela dépend de\end{définition}
\begin{définition}\cmn 随便……,由……决定,自己看着办\end{définition}
\begin{exemple}\jya nɤʑo tɯ-ɕe mɤ-tɯ-ɕe, nɤʑo tɤ-nɯχpjɤt\cmn 去不去由你\end{exemple}
\begin{exemple}\jya nɤʑo ma-tɯ-ɤrju a-tɤ-nɯχpjɤt\cmn 你不要说,由他自己决定\end{exemple}
\end{sous-entrée}\begin{sous-entrée}
\vedette{\hypertarget{}{\papi{ sɯχpjɤt}}}\markboth{sɯχpjɤt}{}\classe{vt}
\paradigme{\textit{dir :} \jya tɤ-}
\begin{définition}\fra demander l'avis de\end{définition}
\begin{définition}\cmn 征求意见\end{définition}
\begin{exemple}\jya laχtɕha nɯ tú-wɣ-χtɯ ɕi kɯ tu-ta-sɯχpjɤt\cmn 我征求一下你的意见,买不买这个东西\end{exemple}
\end{sous-entrée}\begin{sous-entrée}
\vedette{\hypertarget{}{\papi{ ʑɣɤχpjɤt}}}\markboth{ʑɣɤχpjɤt}{}\classe{vi}
\paradigme{\textit{dir :} \jya kɤ-}
\begin{définition}\ 
\begin{déclaration}\grammar{refl}\end{déclaration}\end{définition}
\begin{définition}\fra s'observer\end{définition}
\begin{définition}\cmn 观察自己\end{définition}
\begin{exemple}\jya nɤʑo kɤ-nɯ-ʑɣɤχpjɤt tɕe, tɤ-wa ɯ-ɲɯ-tɯ-fse kɯ?\cmn 你观察一下自己,是不是像一个父亲的样子\end{exemple}
\end{sous-entrée}\end{entrée}

\begin{entrée}
\vedette{\hypertarget{Ⓔχploʁploʁ}{\papi{ χploʁploʁ}}}\markboth{χploʁploʁ}{}\classe{idph.2}
\begin{définition}\fra en boule\end{définition}
\begin{définition}\cmn 形容球形\end{définition}
\begin{exemple}\jya ɯ-rte nɯ χploʁploʁ kɯ-pa ci ɲɯ-ŋu\cmn 他的帽子是球形的\end{exemple}
\begin{exemple}\jya tɤjmɤɣ thamtɕɤt nɯ mɯ-tɤ-kɯ-qawɤr nɯ χploʁploʁ kɯ-pa tu-kɯ-ti ɕti\cmn 菌子在开放之前都是球形的\end{exemple}
\begin{relation-sémantique}\confer{
\hyperlink{Ⓔploʁploʁ}{\textit{ \papi{ploʁploʁ}}}
}\end{relation-sémantique}
\begin{relation-sémantique}\confer{
\hyperlink{Ⓔɕploʁɕploʁ}{\textit{ \papi{ɕploʁɕploʁ}}}
}\end{relation-sémantique}\end{entrée}

\begin{entrée}
\vedette{\hypertarget{Ⓔχpɯn}{\papi{ χpɯn}}}\markboth{χpɯn}{}\classe{n}
\begin{définition}\fra moine\end{définition}
\begin{définition}\cmn 和尚
\begin{déclaration} \étymologie{\papi{dpon}}\end{déclaration}\end{définition}
\begin{exemple}\jya χpɯn to-ndo\cmn 他当了和尚\end{exemple}
\begin{exemple}\jya ɯ-tɕɯ χpɯn lu-te-a ɲɯ-sɯsam-a\cmn 我想让我儿子当和尚\end{exemple}
\begin{relation-sémantique}\confer{
\hyperlink{Ⓔnɯχpɯn}{\textit{ \papi{nɯχpɯn}}}
}\end{relation-sémantique}
\begin{relation-sémantique}\confer{
\hyperlink{Ⓔrɤχpɯn}{\textit{ \papi{rɤχpɯn}}}
}\end{relation-sémantique}\end{entrée}

\begin{entrée}
\vedette{\hypertarget{Ⓔχpɯnbu}{\papi{ χpɯnbu}}}\markboth{χpɯnbu}{}\classe{n}
\begin{définition}\fra maître\end{définition}
\begin{définition}\cmn 主人\end{définition}
\begin{exemple}\jya ɯʑo kɯ χpɯnbu la-ndo\cmn 他掌权了\end{exemple}
\begin{relation-sémantique}\confer{
\hyperlink{Ⓔnɯχpɯnbu}{\textit{ \papi{nɯχpɯnbu}}}
}\end{relation-sémantique}\end{entrée}

\begin{entrée}
\vedette{\hypertarget{Ⓔχsu}{\papi{ χsu}}}\markboth{χsu}{}
\classe{vt}
\paradigme{\textit{dir :} \jya nɯ-}
\paradigme{\textit{dir :} \jya pɯ-}
\paradigme{\textit{dir :} \jya thɯ-}
\begin{définition}\fra élever\end{définition}
\begin{définition}\cmn 养;供他吃
\begin{déclaration} \étymologie{\papi{gso}}\end{déclaration}\end{définition}
\begin{exemple}\jya pɣɤtɕɯ kɯ ɯ-pɯ ɲɯ-ɤsɯ-χsu\cmn 鸟在喂它的小鸟\end{exemple}
\begin{exemple}\jya nɯ-χsu-t-a\cmn 我养了他\end{exemple}
\begin{exemple}\jya nɯ́-wɣ-χsu-a\cmn 他养了我\end{exemple}
\begin{exemple}\jya paʁ pɯ-χsu-t-a\cmn 我喂了猪\end{exemple}
\begin{exemple}\jya rgali thɯ-χsu-t-a\cmn 我喂了小奶牛\end{exemple}
\begin{relation-sémantique}\synonyme{
\hyperlink{Ⓔngu}{\textit{ \papi{ngu}}}
}\end{relation-sémantique}\begin{sous-entrée}
\vedette{\hypertarget{}{\papi{ sɯχsu}}}\markboth{sɯχsu}{}\classe{vt}
\begin{définition}\fra nourrir avec\end{définition}
\begin{définition}\cmn 用……喂\end{définition}
\end{sous-entrée}\begin{sous-entrée}
\vedette{\hypertarget{}{\papi{ ʑɣɤsɯχsu}}}\markboth{ʑɣɤsɯχsu}{}\classe{vi}
\begin{définition}\ 
\begin{déclaration}\grammar{refl}\end{déclaration}\end{définition}
\begin{définition}\fra se nourrir\end{définition}
\begin{définition}\cmn 自己喂自己\end{définition}
\begin{relation-sémantique}\synonyme{
\hyperlink{Ⓔngu}{\textit{ \papi{ngu}}}
}\end{relation-sémantique}
\end{sous-entrée}\end{entrée}

\begin{entrée}
\vedette{\hypertarget{Ⓔχsɤβ}{\papi{ χsɤβ}}}\markboth{χsɤβ}{}\classe{n}
\begin{définition}\fra étalon\end{définition}
\begin{définition}\cmn 公马
\begin{déclaration} \étymologie{\papi{gseb}}\end{déclaration}\end{définition}
\end{entrée}

\begin{entrée}
\vedette{\hypertarget{ⒺχsɤlⒽ1}{\papi{ χsɤl}}}\markboth{χsɤl}{}\homonyme{1}\classe{vt}
\paradigme{\textit{dir :} \jya tɤ-}
\paradigme{\textit{dir :} \jya kɤ-}
\begin{définition}\fra manger, boire (honorifique)\end{définition}
\begin{définition}\cmn 用膳(敬语)
\begin{déclaration} \étymologie{\papi{gsol}}\end{déclaration}\end{définition}
\begin{exemple}\jya βlama kɯ to-χsɤl\cmn 喇嘛吃了\end{exemple}\end{entrée}

\begin{entrée}
\vedette{\hypertarget{ⒺχsɤlⒽ2}{\papi{ χsɤl}}}\markboth{χsɤl}{}\homonyme{2}
\classe{vi}
\paradigme{\textit{dir :} \jya tɤ-}
\begin{définition}\fra clair, évident\end{définition}
\begin{définition}\cmn 明显;清晰
\begin{déclaration} \étymologie{\papi{gsal}}\end{déclaration}\end{définition}
\begin{exemple}\jya fso tɕe a-kɤ-tɯ-lɤt tɕe a-tɤ-χsɤl je\cmn 你明天打电话就会知道\end{exemple}
\begin{exemple}\jya kɯki tɤ-scoz ki pjɯ́-wɣ-ndɯn ɲɯ-jɤɣ ma ɲɯ-χsɤl\cmn 可以把这封信读出来,因为写得很清楚\end{exemple}
\begin{relation-sémantique}\confer{
 \papi{saχsɤl}
}\end{relation-sémantique}
\begin{relation-sémantique}\confer{
\hyperlink{Ⓔɣɯχsɤl}{\textit{ \papi{ɣɯχsɤl}}}
}\end{relation-sémantique}\end{entrée}

\begin{entrée}
\vedette{\hypertarget{Ⓔχsɤlkawa}{\papi{ χsɤlkawa}}}\markboth{χsɤlkawa}{}\classe{n}
\begin{définition}\fra moine qui garde la chapelle\end{définition}
\begin{définition}\cmn 守佛堂的和尚
\begin{déclaration} \étymologie{\papi{gsol.ka.ba}}\end{déclaration}\end{définition}
\end{entrée}

\begin{entrée}
\vedette{\hypertarget{Ⓔχsɤltɕhɯ}{\papi{ χsɤltɕhɯ}}}\markboth{χsɤltɕhɯ}{}\classe{n}
\begin{définition}\fra eau (honorifique)\end{définition}
\begin{définition}\cmn 水(敬语)
\begin{déclaration} \étymologie{\papi{gsol.tɕʰu}}\end{déclaration}\end{définition}
\end{entrée}

\begin{entrée}
\vedette{\hypertarget{Ⓔχsɤltɕhɯ}{\papi{ χsɤltɕhɯ}}}\markboth{χsɤltɕhɯ}{} (\variante{ʑɤβtɕhɯ}, \variante{zlɤβtɕhɯ}) \classe{n}
\begin{définition}\fra eau (honorifique)\end{définition}
\begin{définition}\cmn 水(敬语)
\begin{déclaration} \étymologie{\papi{gsol.tɕʰu}}\end{déclaration}\end{définition}\end{entrée}

\begin{entrée}
\vedette{\hypertarget{ⒺχsɤrⒽ2}{\papi{ χsɤr}}}\markboth{χsɤr}{}\homonyme{2}
\classe{n}
\begin{définition}\fra or\end{définition}
\begin{définition}\cmn 金子
\begin{déclaration} \étymologie{\papi{gser}}\end{déclaration}\end{définition}\end{entrée}

\begin{entrée}
\vedette{\hypertarget{ⒺχsɤrⒽ1}{\papi{ χsɤr}}}\markboth{χsɤr}{}\homonyme{1}\classe{vt}
\paradigme{\textit{dir :} \jya tɤ-}
\begin{définition}\fra compter\end{définition}
\begin{définition}\cmn 数\end{définition}
\begin{exemple}\jya ji-nɯŋa thɤstɯɣ ɣɤʑu tɤ-χsar-a\cmn 我数了一下我们家的牛有多少头\end{exemple}
\begin{exemple}\jya tɤ-χsar-a ri kɯmŋu ɣɤʑu\cmn 我数了一下,有五个\end{exemple}
\begin{relation-sémantique}\synonyme{
\hyperlink{Ⓔrtsi}{\textit{ \papi{rtsi}}}
}\end{relation-sémantique}
\begin{relation-sémantique}\confer{
\hyperlink{Ⓔɯ-χsɤr}{\textit{ \papi{ɯ-χsɤr}}}
}\end{relation-sémantique}\end{entrée}

\begin{entrée}
\vedette{\hypertarget{Ⓔχsɤrɣɤt}{\papi{ χsɤrɣɤt}}}\markboth{χsɤrɣɤt}{}\classe{n}
\begin{définition}\fra sûtra pour les animaux\end{définition}
\begin{définition}\cmn 为牛羊念的经
\begin{déclaration} \étymologie{\papi{gser.ɦod}}\end{déclaration}\end{définition}\end{entrée}

\begin{entrée}
\vedette{\hypertarget{Ⓔχsɤrmdoʁ}{\papi{ χsɤrmdoʁ}}}\markboth{χsɤrmdoʁ}{}\classe{n}
\begin{définition}\fra doré\end{définition}
\begin{définition}\cmn 金色
\begin{déclaration} \étymologie{\papi{gser.mdog}}\end{déclaration}\end{définition}\end{entrée}

\begin{entrée}
\vedette{\hypertarget{Ⓔχsɤrnaʁ}{\papi{ χsɤrnaʁ}}}\markboth{χsɤrnaʁ}{}\classe{n}
\begin{définition}\fra or noir\end{définition}
\begin{définition}\cmn 黑色的金
\begin{déclaration} \étymologie{\papi{gser.nag}}\end{déclaration}\end{définition}\end{entrée}

\begin{entrée}
\vedette{\hypertarget{Ⓔχsɤrɲa}{\papi{ χsɤrɲa}}}\markboth{χsɤrɲa}{}\classe{n}
\begin{définition}\fra cyprin\end{définition}
\begin{définition}\cmn 金鱼
\begin{déclaration} \étymologie{\papi{gser.ɲa}}\end{déclaration}\end{définition}
\end{entrée}

\begin{entrée}
\vedette{\hypertarget{Ⓔχsɤrɲɟɤt}{\papi{ χsɤrɲɟɤt}}}\markboth{χsɤrɲɟɤt}{}\classe{n}
\begin{définition}\fra vermeil\end{définition}
\begin{définition}\cmn 镀金的白银\end{définition}\end{entrée}

\begin{entrée}
\vedette{\hypertarget{Ⓔχsɤrthu}{\papi{ χsɤrthu}}}\markboth{χsɤrthu}{}\classe{n}
\begin{définition}\fra la fête de l'été\end{définition}
\begin{définition}\cmn 看花节\end{définition}\end{entrée}

\begin{entrée}
\vedette{\hypertarget{Ⓔχsɤrʑa}{\papi{ χsɤrʑa}}}\markboth{χsɤrʑa}{}\classe{n}
\begin{définition}\fra coiffe en or\end{définition}
\begin{définition}\cmn 金冠
\begin{déclaration} \étymologie{\papi{gser.ʑwa}}\end{déclaration}\end{définition}
\end{entrée}

\begin{entrée}
\vedette{\hypertarget{Ⓔχsjɯβ}{\papi{ χsjɯβ}}}\markboth{χsjɯβ}{}
\classe{n}
\begin{définition}\fra peau de serpent\end{définition}
\begin{définition}\cmn 蛇蜕的皮\end{définition}
\begin{exemple}\jya qapri ɯ-χsjɯβ chɤ-βde\cmn 蛇脱皮了\end{exemple}\end{entrée}

\begin{entrée}
\vedette{\hypertarget{Ⓔχsjɯβnɤχsjɯβ}{\papi{ χsjɯβnɤχsjɯβ}}}\markboth{χsjɯβnɤχsjɯβ}{}\classe{idph.3}
\begin{définition}\fra reniflement\end{définition}
\begin{définition}\cmn 形容用鼻吸气的声音\end{définition}
\begin{exemple}\jya χsjɯβnɤχsjɯβ to-stu (=ɯ-ɕna to-sɤχsɯχsjɯβ)\cmn 他用鼻吸了气\end{exemple}
\begin{relation-sémantique}\confer{
\hyperlink{Ⓔsɤχsɯχsjɯβ}{\textit{ \papi{sɤχsɯχsjɯβ}}}
}\end{relation-sémantique}\end{entrée}

\begin{entrée}
\vedette{\hypertarget{Ⓔχsoz}{\papi{ χsoz}}}\markboth{χsoz}{}
\classe{vs}\acception{1}
\begin{définition}\fra fine (oreille)\end{définition}
\begin{définition}\cmn 敏锐(耳朵)\end{définition}
\begin{exemple}\jya tɕhi rɯdaʁ kɯ-fse ɯ-rna kɯ-χsoz me\cmn 什么动物的听觉是最敏锐的?\end{exemple}
\begin{exemple}\jya tɕhi rɯdaʁ ɣɯ ɯ-rna stu ʑo χsoz\cmn 什么动物的听觉是最敏锐的?\end{exemple}\acception{2}
\begin{définition}\fra fin (odorat)\end{définition}
\begin{définition}\cmn 敏锐(嗅觉)\end{définition}
\begin{exemple}\jya a-ɕna χsoz\cmn 我的嗅觉很敏锐\end{exemple}\acception{3}
\begin{définition}\fra pas bouché (paille)\end{définition}
\begin{définition}\cmn 畅通(管子)\end{définition}
\begin{exemple}\jya chɤmdɤru ɲɯ-χsoz tɕe, chɤmda kɤ-sɯtshi ɲɯ-khu\cmn 吸管是畅通的,可以喝坛坛酒\end{exemple}
\begin{relation-sémantique}\confer{
\hyperlink{Ⓔnaχsoz}{\textit{ \papi{naχsoz}}}
}\end{relation-sémantique}\end{entrée}

\begin{entrée}
\vedette{\hypertarget{Ⓔχsɯm}{\papi{ χsɯm}}}\markboth{χsɯm}{}\classe{num}
\begin{définition}\fra trois\end{définition}
\begin{définition}\cmn 三\end{définition}\end{entrée}

\begin{entrée}
\vedette{\hypertarget{Ⓔχsɯmba}{\papi{ χsɯmba}}}\markboth{χsɯmba}{}\classe{n}
\begin{définition}\fra troisième mois\end{définition}
\begin{définition}\cmn 三月
\begin{déclaration} \étymologie{\papi{gsum.pa}}\end{déclaration}\end{définition}
\end{entrée}

\begin{entrée}
\vedette{\hypertarget{Ⓔχsɯmdu}{\papi{ χsɯmdu}}}\markboth{χsɯmdu}{}\classe{n}
\begin{définition}\fra carrefour en Y\end{définition}
\begin{définition}\cmn 三岔口\end{définition}
\begin{relation-sémantique}\synonyme{
\hyperlink{Ⓔtʂɤsɤɴɢɤt}{\textit{ \papi{tʂɤsɤɴɢɤt}}}
}\end{relation-sémantique}\end{entrée}

\begin{entrée}
\vedette{\hypertarget{Ⓔχsɯmkha}{\papi{ χsɯmkha}}}\markboth{χsɯmkha}{}\classe{n}
\begin{définition}\ 
\begin{déclaration}\use{古语}\end{déclaration}\end{définition}
\begin{définition}\fra la troisième fois\end{définition}
\begin{définition}\cmn 第三次\end{définition}
\begin{exemple}\jya χsɯmkha to-mbri\cmn 已经是第三次了\end{exemple}\end{entrée}

\begin{entrée}
\vedette{\hypertarget{Ⓔχsɯmsna}{\papi{ χsɯmsna}}}\markboth{χsɯmsna}{}\classe{n}
\begin{définition}\fra lorsque l'on tue un animal, part donnée aux amis\end{définition}
\begin{définition}\cmn 宰动物的时候,分给好友的部分
\begin{déclaration} \étymologie{\papi{gsum.sna}}\end{déclaration}\end{définition}\end{entrée}

\begin{entrée}
\vedette{\hypertarget{Ⓔχsɯskɤl}{\papi{ χsɯskɤl}}}\markboth{χsɯskɤl}{}
\classe{n}
\begin{définition}\fra trois repas\end{définition}
\begin{définition}\cmn 三顿\end{définition}
\begin{exemple}\jya zama χsɯskɤl nɯ koŋla tu-kɯ-rɯndzɤtshi ra\cmn 一天三餐要吃好一点\end{exemple}\end{entrée}

\begin{entrée}
\vedette{\hypertarget{Ⓔχʂɤχʂɤt}{\papi{ χʂɤχʂɤt}}}\markboth{χʂɤχʂɤt}{}
\classe{idph.2}\acception{1}
\begin{définition}\fra léger (habit)\end{définition}
\begin{définition}\cmn 形容(衣服)单薄,透风\end{définition}
\begin{exemple}\jya nɤ-ŋga kɯ-mba ci χʂɤχʂɤt tɯ-asɯ-ŋga\cmn 你穿的衣服很单薄\end{exemple}
\begin{exemple}\jya ɯ-ŋga ɲɯ-mba ʑo χʂɤχʂɤt\cmn 他的衣服很薄\end{exemple}\acception{2}
\begin{définition}\fra qui a les traits fins, au regard intelligent\end{définition}
\begin{définition}\cmn 眉目清秀,聪明的样子\end{définition}
\begin{exemple}\jya kɯ-mpɕɤr ci χʂɤχʂɤt ɲɯ-ŋu\cmn 他眉目清秀(有双眼皮,眼睛又圆又亮)\end{exemple}
\begin{relation-sémantique}\confer{
\hyperlink{Ⓔxʂɤxʂɤt}{\textit{ \papi{xʂɤxʂɤt}}}
}\end{relation-sémantique}\end{entrée}

\begin{entrée}
\vedette{\hypertarget{Ⓔχtɤr}{\papi{ χtɤr}}}\markboth{χtɤr}{}\classe{vt}
\paradigme{\textit{dir :} \jya thɯ-}
\paradigme{\textit{dir :} \jya pɯ-}
\begin{définition}\fra disperser\end{définition}
\begin{définition}\cmn 打散;撒
\begin{déclaration} \étymologie{\papi{gtor}}\end{déclaration}\end{définition}
\begin{exemple}\jya tɯjpu pjɤ-χtɤr\cmn 他撒了粮食\end{exemple}
\begin{exemple}\jya tɯ-rɣi tha-χtɤr\cmn 他撒了种子\end{exemple}
\begin{exemple}\jya tɯ-ɣndʑɤr pɯ-nɯ-χtar-a\cmn 我把糌粑打散了\end{exemple}
\begin{exemple}\jya khu ɯ-ɯ-sɤɣmu kɯ rɯdaʁ ra pjɤ́-wɣ-χtɤr-nɯ ʑo\cmn 老虎可怕得让百兽四处逃窜了\end{exemple}
\begin{relation-sémantique}\confer{
\hyperlink{Ⓔʁndɤr}{\textit{ \papi{ʁndɤr}}}
}\end{relation-sémantique}
\begin{relation-sémantique}\confer{
\hyperlink{Ⓔrɯtɕhɯχtɤr}{\textit{ \papi{rɯtɕhɯχtɤr}}}
}\end{relation-sémantique}\end{entrée}

\begin{entrée}
\vedette{\hypertarget{Ⓔχtɤt}{\papi{ χtɤt}}}\markboth{χtɤt}{}
\paradigme{\textit{dir :} \jya kɤ-}
\paradigme{\textit{dir :} \jya pɯ-}
\paradigme{\textit{dir :} \jya nɯ-}
\begin{définition}\fra appuyer contre\end{définition}
\begin{définition}\cmn 靠\end{définition}
\begin{exemple}\jya kɤ-ɤmdzɯ-a tɕe, a-mgɯr kɤ-nɯ-χtat-a\cmn 我坐下的时候就靠了背\end{exemple}
\begin{exemple}\jya znde ɯ-taʁ a-mgɯr kɤ-nɯ-χtat-a\cmn 我把背靠在墙上了\end{exemple}\begin{sous-entrée}
\vedette{\hypertarget{}{\papi{ tɯ-sɯm,χtɤt}}}\markboth{tɯ-sɯm,χtɤt}{}\acception{1}
\begin{définition}\fra être loyal envers\end{définition}
\begin{définition}\cmn 对……忠心\end{définition}
\begin{exemple}\jya nɤ-ɕki a-sɯm ku-χtat-a ɕti\cmn 我对你忠心耿耿\end{exemple}\acception{2}
\begin{définition}\fra se concentrer\end{définition}
\begin{définition}\cmn 专心\end{définition}
\begin{exemple}\jya a-sɯm ku-χtat-a ʑo ku-rɤma-a\cmn 我很专心地做事\end{exemple}
\begin{exemple}\jya nɤ-sɯm kɤ-χtɤt ʑo pɯ-rɤβzjoz ra nɤ!\cmn 你要专心学习\end{exemple}
\end{sous-entrée}\begin{sous-entrée}
\vedette{\hypertarget{}{\papi{ ɯ-rɕa,χtɤt}}}\markboth{ɯ-rɕa,χtɤt}{}
\begin{définition}\fra se concentrer\end{définition}
\begin{définition}\cmn 专心\end{définition}
\end{sous-entrée}\begin{sous-entrée}
\vedette{\hypertarget{}{\papi{ ʑɣɤχtɤt}}}\markboth{ʑɣɤχtɤt}{}\classe{vi}
\begin{définition}\ 
\begin{déclaration}\grammar{refl}\end{déclaration}\end{définition}
\begin{définition}\fra s'appuyer sur\end{définition}
\begin{définition}\cmn 靠
\begin{déclaration}\use{“靠”的引申意义(如“这件事靠你了”)不能用\stylefv{ʑɣɤχtɤt}或\stylefv{ʑɣɤta}来表示,参看\stylefv{ti}“说”}\end{déclaration}
\begin{déclaration} \étymologie{\papi{gtad}}\end{déclaration}\end{définition}
\begin{exemple}\jya si ɯ-taʁ ko-ʑɣɤχtɤt\cmn 他靠在树上\end{exemple}
\begin{exemple}\jya znde ɯ-taʁ ko-ʑɣɤχtɤt\cmn 我靠在墙上\end{exemple}
\begin{exemple}\jya ɯ-zda ɯ-taʁ ko-ʑɣɤχtɤt\cmn 他依靠了他的伙伴\end{exemple}
\begin{exemple}\jya aʑo ɲɯ-ɤɲat-a tɕe, ɯ-taʁ kɤ-ʑɣɤχtat-a tɕe tɤ-nɯna-a\cmn 我累了所以靠在他身上休息了\end{exemple}
\begin{relation-sémantique}\synonyme{
\hyperlink{Ⓔʑɣɤta}{\textit{ \papi{ʑɣɤta}}}
}\end{relation-sémantique}
\end{sous-entrée}\end{entrée}

\begin{entrée}
\vedette{\hypertarget{Ⓔχtɕɤnzɤn}{\papi{ χtɕɤnzɤn}}}\markboth{χtɕɤnzɤn}{}\classe{n}
\begin{définition}\fra bête sauvage\end{définition}
\begin{définition}\cmn 野兽
\begin{déclaration} \étymologie{\papi{gtɕan.gzan}}\end{déclaration}\end{définition}\end{entrée}

\begin{entrée}
\vedette{\hypertarget{Ⓔχtɕɤt}{\papi{ χtɕɤt}}}\markboth{χtɕɤt}{}
\classe{n}
\begin{définition}\fra exorcisme\end{définition}
\begin{définition}\cmn 驱妖的仪式;咒经
\begin{déclaration} \étymologie{\papi{gtɕod}}\end{déclaration}\end{définition}
\begin{exemple}\jya sprɯskɯ kɯ χtɕɤt pa-lɤt\cmn 活佛念了咒经\end{exemple}\end{entrée}

\begin{entrée}
\vedette{\hypertarget{Ⓔχtɕɤz}{\papi{ χtɕɤz}}}\markboth{χtɕɤz}{}\classe{vs}
\paradigme{\textit{dir :} \jya pɯ-}
\begin{définition}\fra être chéri\end{définition}
\begin{définition}\cmn 受宠
\begin{déclaration} \étymologie{\papi{gtɕes}}\end{déclaration}\end{définition}
\begin{exemple}\jya ɯ-mɤ-tɯ-χtɕɤz nɯ dɯχpa!\cmn 他很可怜,根本不受人宠爱\end{exemple}\begin{sous-entrée}
\vedette{\hypertarget{}{\papi{ sɯχtɕɤz}}}\markboth{sɯχtɕɤz}{}\classe{vt}
\begin{définition}\fra adorer\end{définition}
\begin{définition}\cmn 宠爱\end{définition}
\begin{exemple}\jya ɯ-mu ɯ-wa ni kɯ ɲɯ-sɯχtɕɤz-ndʑi\cmn 他父母很宠爱他\end{exemple}
\end{sous-entrée}\end{entrée}

\begin{entrée}
\vedette{\hypertarget{Ⓔχtɕɤzɤz}{\papi{ χtɕɤzɤz}}}\markboth{χtɕɤzɤz}{}\classe{n}
\begin{définition}\fra mets pour bien recevoir les invités\end{définition}
\begin{définition}\cmn 款待客人的食品
\begin{déclaration} \étymologie{\papi{gtɕes.zas}}\end{déclaration}\end{définition}\end{entrée}

\begin{entrée}
\vedette{\hypertarget{Ⓔχtɕi}{\papi{ χtɕi}}}\markboth{χtɕi}{}
\classe{vt}\acception{1}
\paradigme{\textit{dir :} \jya pɯ-}
\paradigme{\textit{dir :} \jya nɯ-}
\begin{définition}\fra laver\end{définition}
\begin{définition}\cmn 洗\end{définition}
\begin{exemple}\jya nɤ-rŋa pɯ-χtɕi\cmn 你洗脸吧\end{exemple}
\begin{exemple}\jya nɤ-ŋga nɯ-χtɕi\cmn 你洗衣服吧\end{exemple}
\begin{exemple}\jya tɯ-ŋga nɯ-χtɕi-t-a\cmn 我洗了衣服\end{exemple}
\begin{exemple}\jya a-βri pɯ-χtɕi-t-a\cmn 我洗了身体\end{exemple}
\begin{exemple}\jya a-ɕɣa nɯ-χtɕi-t-a\cmn 我刷了牙\end{exemple}
\begin{exemple}\jya aʑo jɤ-azɣɯt-a nɯ tɕu, ɯʑo kɯ ɯ-jaʁ pjɤ-k-ɤ<nɯ>sɯ-χtɕi-ci\cmn 我到那里的时候,他正在洗手\end{exemple}
\begin{exemple}\jya nɤʑo nɤ-ku ci pɯ-nɯ-χtɕi\cmn 你洗一下头吧\end{exemple}
\begin{exemple}\jya nɤ-jaʁ pɯ-nɯ-χtɕi\cmn 你洗一下手吧\end{exemple}
\begin{exemple}\jya kha thɯ-χtɕi\cmn 洗一下房子吧\end{exemple}\acception{2}
\paradigme{\textit{dir :} \jya pɯ-}
\begin{définition}\fra tremper (pluie)\end{définition}
\begin{définition}\cmn 淋湿(雨)\end{définition}
\begin{exemple}\jya tɯ-mɯ kɯ pɯ́-wɣ-χtɕi-a\cmn 我被雨淋湿了\end{exemple}
\begin{relation-sémantique}\confer{
\hyperlink{Ⓔraχtɕɯʁɟo}{\textit{ \papi{raχtɕɯʁɟo}}}
}\end{relation-sémantique}\begin{sous-entrée}
\vedette{\hypertarget{}{\papi{ raχtɕi}}}\markboth{raχtɕi}{}\classe{vi}\acception{1}
\paradigme{\textit{dir :} \jya pɯ-}
\begin{définition}\fra se laver le visage, se laver\end{définition}
\begin{définition}\cmn 洗脸;洗澡\end{définition}
\begin{exemple}\jya ɯ-pɯ-tɯ-raχtɕi\cmn 你洗了脸没有?\end{exemple}
\begin{exemple}\jya pɯ-raχtɕi-a\cmn 我洗了脸\end{exemple}\acception{2}
\paradigme{\textit{dir :} \jya nɯ-}
\begin{définition}\fra laver des choses\end{définition}
\begin{définition}\cmn 洗东西\end{définition}
\begin{exemple}\jya jisŋi nɯ-raχtɕi-a\cmn 今天洗了衣服\end{exemple}
\end{sous-entrée}\begin{sous-entrée}
\vedette{\hypertarget{}{\papi{ ʑɣɤχtɕi}}}\markboth{ʑɣɤχtɕi}{}\classe{vi}
\begin{définition}\ 
\begin{déclaration}\grammar{refl}\end{déclaration}\end{définition}
\begin{définition}\fra se laver\end{définition}
\begin{définition}\cmn 洗自己\end{définition}
\end{sous-entrée}\end{entrée}

\begin{entrée}
\vedette{\hypertarget{Ⓔχtɕoŋ}{\papi{ χtɕoŋ}}}\markboth{χtɕoŋ}{}\classe{n}
\begin{définition}\fra rhumatisme\end{définition}
\begin{définition}\cmn 风湿,关节疼
\begin{déclaration} \étymologie{\papi{gtɕoŋ}}\end{déclaration}\end{définition}\end{entrée}

\begin{entrée}
\vedette{\hypertarget{Ⓔχtɕɯrɯ}{\papi{ χtɕɯrɯ}}}\markboth{χtɕɯrɯ}{}\classe{n}
\begin{définition}\fra nu\end{définition}
\begin{définition}\cmn 裸体\end{définition}
\begin{relation-sémantique}\confer{
\hyperlink{Ⓔχtɕɯrɯpa}{\textit{ \papi{χtɕɯrɯpa}}}
}\end{relation-sémantique}
\begin{relation-sémantique}\confer{
\hyperlink{Ⓔrɯχtɕɯrɯ}{\textit{ \papi{rɯχtɕɯrɯ}}}
}\end{relation-sémantique}
\begin{relation-sémantique}\confer{
\hyperlink{Ⓔnɯχtɕɯrɯ}{\textit{ \papi{nɯχtɕɯrɯ}}}
}\end{relation-sémantique}\end{entrée}

\begin{entrée}
\vedette{\hypertarget{Ⓔχtɕɯrɯpa}{\papi{ χtɕɯrɯpa}}}\markboth{χtɕɯrɯpa}{}\classe{n}
\begin{définition}\fra tout nu\end{définition}
\begin{définition}\cmn 裸体,不穿衣服
\begin{déclaration} \étymologie{\papi{gtɕer.bu.pa}}\end{déclaration}\end{définition}
\begin{relation-sémantique}\confer{
\hyperlink{Ⓔrɯχtɕɯrɯ}{\textit{ \papi{rɯχtɕɯrɯ}}}
}\end{relation-sémantique}\end{entrée}

\begin{entrée}
\vedette{\hypertarget{Ⓔχtorma}{\papi{ χtorma}}}\markboth{χtorma}{}\classe{n}
\begin{définition}\fra offrande au dieux\end{définition}
\begin{définition}\cmn 供奉鬼神的物品
\begin{déclaration} \étymologie{\papi{gtor.ma}}\end{déclaration}\end{définition}
\end{entrée}

\begin{entrée}
\vedette{\hypertarget{Ⓔχtsɤβ}{\papi{ χtsɤβ}}}\markboth{χtsɤβ}{}
\classe{vt}
\paradigme{\textit{dir :} \jya nɯ-}
\begin{définition}\fra pétrir, tanner (peau)\end{définition}
\begin{définition}\cmn 揉
\end{définition}
\begin{exemple}\jya tɯ-ndʐi pa-χtsɤβ\cmn 他揉了皮子\end{exemple}
\begin{exemple}\jya pɯ́-wɣ-χtsaβ-a\cmn 他揉了我\end{exemple}
\begin{exemple}\jya tɤ-pɤtso, ma-pɯ-kɯ-χtsaβ-a\cmn 小孩,你不要揉我(你不要一整天麻烦我)\end{exemple}
\begin{exemple}\jya ɯ-χtsɤβ pɯ-ɣe\cmn (皮子)已经揉好了\end{exemple}\end{entrée}

\begin{entrée}
\vedette{\hypertarget{Ⓔχtshɤχtshɤt}{\papi{ χtshɤχtshɤt}}}\markboth{χtshɤχtshɤt}{}
\classe{idph.2}
\begin{définition}\fra sage et très actif (enfant, petit animal)\end{définition}
\begin{définition}\cmn 形容小孩子或者小动物看起来很乖很灵活的样子,小巧玲珑。\end{définition}
\begin{exemple}\jya jiɕqha tɤ-pɤtso χtshɤχtshɤt ci ɲɯ-ŋu\cmn 那个小孩子很灵活\end{exemple}
\begin{exemple}\jya tshɯtho χtshɤχtshɤt ʑo ɲɯ-pa\cmn 那头小羊羔很灵活\end{exemple}\begin{sous-entrée}
\vedette{\hypertarget{}{\papi{ χtshɤnɤχtshɤt}}}\markboth{χtshɤnɤχtshɤt}{}\classe{idph.3}
\begin{exemple}\jya tɤ-pɤtso nɯ χtshɤnɤχtshɤt ʑo ɯ-mu ɯ-phe ko-ɕe\cmn 那个小孩子很敏捷地一下子就到了他母亲身边\end{exemple}
\end{sous-entrée}\end{entrée}

\begin{entrée}
\vedette{\hypertarget{Ⓔχtsiɯ}{\papi{ χtsiɯ}}}\markboth{χtsiɯ}{}\classe{n}
\begin{définition}\fra unité de mesure\end{définition}
\begin{définition}\cmn 升\end{définition}
\end{entrée}

\begin{entrée}
\vedette{\hypertarget{Ⓔχtso}{\papi{ χtso}}}\markboth{χtso}{}\classe{vs}
\paradigme{\textit{dir :} \jya tɤ-}
\begin{définition}\fra propre\end{définition}
\begin{définition}\cmn 干净(本质)
\begin{déclaration} \étymologie{\papi{gtsaŋ}}\end{déclaration}\end{définition}
\begin{exemple}\jya tɯ-ci ɲɯ-χtso\cmn 水很干净\end{exemple}
\begin{exemple}\jya ndzɤtshi ɲɯ-χtso\cmn 食物很干净\end{exemple}\begin{sous-entrée}
\vedette{\hypertarget{}{\papi{ ɣɤχtso}}}\markboth{ɣɤχtso}{}\classe{vt}
\paradigme{\textit{dir :} \jya tɤ-}
\begin{définition}\ 
\begin{déclaration}\grammar{caus}\end{déclaration}\end{définition}
\begin{définition}\fra rendre propre\end{définition}
\begin{définition}\cmn 弄干净\end{définition}
\end{sous-entrée}\begin{sous-entrée}
\vedette{\hypertarget{}{\papi{ naχtso}}}\markboth{naχtso}{} (\variante{aʑɯχtso}) \classe{vt}
\begin{définition}\fra trouver propre\end{définition}
\begin{définition}\cmn 觉得干净\end{définition}
\begin{exemple}\jya mɤ-kɤ-naχtso nɯra s-chɯ́-wɣ-βde\cmn 觉得不干净的东西就要扔掉\end{exemple}
\begin{relation-sémantique}\synonyme{
\hyperlink{Ⓔnaχɕɯn}{\textit{ \papi{naχɕɯn}}}
}\end{relation-sémantique}
\begin{relation-sémantique}\antonyme{
\hyperlink{Ⓔɴqhi}{\textit{ \papi{ɴqhi}}}
}\end{relation-sémantique}
\end{sous-entrée}\end{entrée}

\begin{entrée}
\vedette{\hypertarget{Ⓔχtsur}{\papi{ χtsur}}}\markboth{χtsur}{}\classe{vs}
\begin{définition}\fra important\end{définition}
\begin{définition}\cmn 重要\end{définition}\begin{sous-entrée}
\vedette{\hypertarget{}{\papi{ raχtsur}}}\markboth{raχtsur}{}\classe{vt}
\paradigme{\textit{dir :} \jya tɤ-}
\begin{définition}\fra considérer comme important\end{définition}
\begin{définition}\cmn 觉得重要\end{définition}
\begin{exemple}\jya aʑo kɤ-nɤma ra tu-stu-a pɯ-ŋu tɕe, ŋgumdʑɯɣ ra kɯ tú-wɣ-raχtsur-a-nɯ pɯ-ŋu\cmn 因为我工作得很努力,领导们很器重我\end{exemple}
\begin{exemple}\jya kɤ-nɤma nɯ ra ri, kɤ-rɯndzɤtshi kɯnɤ kɤ-raχtsur ra\cmn 工作是需要的,但是吃饭也重要\end{exemple}
\end{sous-entrée}\end{entrée}

\begin{entrée}
\vedette{\hypertarget{Ⓔχtsɯm}{\papi{ χtsɯm}}}\markboth{χtsɯm}{}\classe{n}
\begin{définition}\fra paille d'orge\end{définition}
\begin{définition}\cmn 青稞杆\end{définition}
\end{entrée}

\begin{entrée}
\vedette{\hypertarget{Ⓔχtsɯmpapɯ}{\papi{ χtsɯmpapɯ}}}\markboth{χtsɯmpapɯ}{}\classe{n}
\begin{définition}\fra paille d'orge en botte\end{définition}
\begin{définition}\cmn 一捆一捆的青稞杆
\end{définition}
\begin{relation-sémantique}\confer{
\hyperlink{Ⓔχtsɯm}{\textit{ \papi{χtsɯm}}}
}\end{relation-sémantique}\end{entrée}

\begin{entrée}
\vedette{\hypertarget{Ⓔχtsɯχtsri}{\papi{ χtsɯχtsri}}}\markboth{χtsɯχtsri}{}
\classe{idph.2}
\begin{définition}\fra les plumes / poils hérissés\end{définition}
\begin{définition}\cmn 形容(鸟;猫)把(羽)毛毛竖起来的样子\end{définition}
\begin{exemple}\jya pɣɤtɕɯ nɯ to-sɤmbrɯ tɕe, ɯ-muj ra χtsɯχtsri ʑo to-sɯɣndzur.\cmn 小鸟生气了就把羽毛竖起来了\end{exemple}\end{entrée}

\begin{entrée}
\vedette{\hypertarget{Ⓔχtʂɯɣdʑa}{\papi{ χtʂɯɣdʑa}}}\markboth{χtʂɯɣdʑa}{}\classe{n}
\begin{définition}\fra thé au beurre\end{définition}
\begin{définition}\cmn 酥油茶
\begin{déclaration} \étymologie{\papi{dkrug.dʑa}}\end{déclaration}\end{définition}\end{entrée}

\begin{entrée}
\vedette{\hypertarget{Ⓔχtɯ}{\papi{ χtɯ}}}\markboth{χtɯ}{}
\classe{vt}
\paradigme{\textit{dir :} \jya tɤ-}
\begin{définition}\fra acheter\end{définition}
\begin{définition}\cmn 买\end{définition}
\begin{exemple}\jya mbarkhom lɤ-tɯ-ɣe ri @beimu tɤ-tɯ-χtɯ-t loβ, nɤ-ɕqhe ɯ-smɤn\cmn 你上次来马尔康的时候,你买了贝母对吧,咳嗽的药\end{exemple}
\begin{exemple}\jya jɯfɕɯr @cai mɯ-ɕ-tɤ-χtɯ-t-a\cmn 我昨天没有去买菜\end{exemple}
\begin{exemple}\jya aʑo kɤntɕhaʁ tɯ-ŋga ɯ-kɯ-χtɯ jɤ-ari-a\cmn 我到街上去买衣服了(还没有买回来)\end{exemple}
\begin{exemple}\jya tɤ-mthɯm ɕ-tɤ-χtɯ-t-a.\cmn 我去买肉了(已经买回来了)\end{exemple}\begin{sous-entrée}
\vedette{\hypertarget{}{\papi{ raχtɯ}}}\markboth{raχtɯ}{}\classe{vi}
\paradigme{\textit{dir :} \jya tɤ-}
\begin{définition}\ 
\begin{déclaration}\grammar{apass}\end{déclaration}\end{définition}
\begin{définition}\fra acheter des choses\end{définition}
\begin{définition}\cmn 买东西\end{définition}
\begin{exemple}\jya @dianhua jɤ-tɯ-lɤt ri, aʑo pɯ-raχtɯ-a\cmn 你打电话的时候我正在买东西\end{exemple}
\begin{exemple}\jya nɤʑo jɤ-tɯ-ɤzɣɯt ri, aʑo tɤ-raχtɯ-a\cmn 你到来的时候,我已经买好东西了\end{exemple}
\end{sous-entrée}\begin{sous-entrée}
\vedette{\hypertarget{}{\papi{ sɯχtɯ}}}\markboth{sɯχtɯ}{}\classe{vt}
\paradigme{\textit{dir :} \jya tɤ-}
\begin{définition}\fra vendre à\end{définition}
\begin{définition}\cmn 使……买、卖给\end{définition}
\begin{exemple}\jya ki tɯ-ŋga ki ɯʑo kɯ tɤ́-wɣ-sɯ-χtɯ-a ŋu (=a-ɕki na-ntsɣe)\cmn 这件衣服是他卖给我的\end{exemple}
\begin{relation-sémantique}\confer{
\hyperlink{Ⓔraχtɯtsɣe}{\textit{ \papi{raχtɯtsɣe}}}
}\end{relation-sémantique}
\end{sous-entrée}\end{entrée}

\begin{entrée}
\vedette{\hypertarget{Ⓔχtɯɣ}{\papi{ χtɯɣ}}}\markboth{χtɯɣ}{}\classe{vt}
\paradigme{\textit{dir :} \jya tɤ-}
\begin{définition}\fra demander une arbitration (auprès de son supérieur)\end{définition}
\begin{définition}\cmn (向上级)请求裁决;请上级评理
\begin{déclaration} \étymologie{\papi{gtug}}\end{déclaration}\end{définition}
\begin{exemple}\jya tɕiʑo kɤ-nɯkrɤz mɯ́j-cha-tɕi tɕe (tɕi-tɯkrɤz mɯ́j-ɣi tɕe), taʁ ɕ-tu-kɤ-χtɯɣ ɲɯ-ɬoʁ\cmn 我们俩既然说不通,只好请上级评理\end{exemple}
\begin{exemple}\jya nɤ-ɕki ɣɯ-tu-χtɯɣ-a ɲɯ-ɬoʁ\cmn 我只好来求您了\end{exemple}
\end{entrée}

\begin{entrée}
\vedette{\hypertarget{Ⓔχtɯmbrɤl}{\papi{ χtɯmbrɤl}}}\markboth{χtɯmbrɤl}{}\classe{n}
\begin{définition}\fra célébration\end{définition}
\begin{définition}\cmn 庆祝
\begin{déclaration} \étymologie{\papi{rten.ⁿbrel}}\end{déclaration}\end{définition}
\end{entrée}

\begin{entrée}
\vedette{\hypertarget{Ⓔχtɯn}{\papi{ χtɯn}}}\markboth{χtɯn}{}\classe{n}
\begin{définition}\fra mortier\end{définition}
\begin{définition}\cmn 臼【坨坨】
\begin{déclaration} \étymologie{\papi{gtun}}\end{déclaration}\end{définition}
\end{entrée}

\begin{entrée}
\vedette{\hypertarget{Ⓔχɯχɯ}{\papi{ χɯχɯ}}}\markboth{χɯχɯ}{}\classe{idph.2}
\begin{définition}\fra qui a des grandes narines\end{définition}
\begin{définition}\cmn 形容鼻孔很大的样子\end{définition}
\begin{exemple}\jya ca kɯ ɯ-ɕna χɯχɯ ʑo to-stu\cmn 麝香鹿把鼻孔弄得很大\end{exemple}\end{entrée}

\begin{entrée}
\vedette{\hypertarget{Ⓔχwara}{\papi{ χwara}}}\markboth{χwara}{}\classe{n}
\begin{définition}\fra un type de tente\end{définition}
\begin{définition}\cmn 一种帐篷
\begin{déclaration} \étymologie{\papi{sbra}}\end{déclaration}\end{définition}
\end{entrée}

\begin{entrée}
\vedette{\hypertarget{Ⓔχwɤr}{\papi{ χwɤr}}}\markboth{χwɤr}{}\classe{n}
\begin{définition}\fra Hor\end{définition}
\begin{définition}\cmn 霍尔
\begin{déclaration} \étymologie{\papi{hor}}\end{déclaration}\end{définition}
\end{entrée}

\begin{entrée}
\vedette{\hypertarget{Ⓔχwɤrχwɤr}{\papi{ χwɤrχwɤr}}}\markboth{χwɤrχwɤr}{}\classe{idph.2}\acception{1}
\begin{définition}\fra abîmé et déchiré\end{définition}
\begin{définition}\cmn 形容破烂的样子
\end{définition}
\begin{exemple}\jya ɯ-ŋga mɯ́j-pe tɕe χwɤrχwɤr ʑo ɲɯ-pa\cmn 我的衣服不好,破破烂烂的\end{exemple}\acception{2}
\begin{définition}\fra ouvert (sac)\end{définition}
\begin{définition}\cmn 形容向外展开的样子\end{définition}\end{entrée}

\begin{entrée}
\vedette{\hypertarget{Ⓔχuχu}{\papi{ χuχu}}}\markboth{χuχu}{}\classe{idph.2}
\begin{définition}\fra qui a une petite ouverture\end{définition}
\begin{définition}\cmn 形容洞口很小\end{définition}
\begin{exemple}\jya tɤ-pɤtso ɯ-mtɕhi ɲɯ-xtɕi χuχu ʑo ɲɯ-pa\cmn 小孩子的嘴很小\end{exemple}\end{entrée}

\newpage\caractère{z}

\begin{entrée}
\vedette{\hypertarget{Ⓔzaŋ}{\papi{ zaŋ}}}\markboth{zaŋ}{}\classe{n}
\begin{définition}\fra cuivre\end{définition}
\begin{définition}\cmn 红铜
\begin{déclaration} \étymologie{\papi{zaŋs}}\end{déclaration}\end{définition}
\end{entrée}

\begin{entrée}
\vedette{\hypertarget{Ⓔzaŋpoŋ}{\papi{ zaŋpoŋ}}}\markboth{zaŋpoŋ}{}\classe{n}
\begin{définition}\fra ventouse\end{définition}
\begin{définition}\cmn 火罐(原来是用红铜铸的)
\begin{déclaration} \étymologie{\papi{zaŋs.bum}}\end{déclaration}\end{définition}
\end{entrée}

\begin{entrée}
\vedette{\hypertarget{Ⓔzaŋzaŋ}{\papi{ zaŋzaŋ}}}\markboth{zaŋzaŋ}{}
\classe{idph.2}
\begin{définition}\fra ébouriffés et longs (cheveux)\end{définition}
\begin{définition}\cmn 形容头发等乱蓬蓬样子\end{définition}
\begin{exemple}\jya ɯ-ku tɤrpɯ zaŋzaŋ ʑo\cmn 他的头发又长又脏,乱蓬蓬的\end{exemple}
\begin{relation-sémantique}\confer{
\hyperlink{Ⓔzoŋzoŋ}{\textit{ \papi{zoŋzoŋ}}}
}\end{relation-sémantique}
\begin{relation-sémantique}\confer{
\hyperlink{Ⓔdzaŋdzaŋ}{\textit{ \papi{dzaŋdzaŋ}}}
}\end{relation-sémantique}\end{entrée}

\begin{entrée}
\vedette{\hypertarget{Ⓔzaχtɤt}{\papi{ zaχtɤt}}}\markboth{zaχtɤt}{}\classe{n}
\begin{définition}\fra nourriture pour les morts\end{définition}
\begin{définition}\cmn 供奉死人的食物\end{définition}
\begin{relation-sémantique}\synonyme{
\hyperlink{Ⓔzɤmpo}{\textit{ \papi{zɤmpo}}}
}\end{relation-sémantique}\end{entrée}

\begin{entrée}
\vedette{\hypertarget{Ⓔzɤjzɤj}{\papi{ zɤjzɤj}}}\markboth{zɤjzɤj}{}\classe{idph.2}\begin{définition}\fra si petit qu'on a peine à le voir\end{définition}
\begin{définition}\cmn 形容小而难以看见的样子
\end{définition}
\begin{exemple}\jya tɕetu zgoku ri tɯrme ci zɤjzɤj ʑo ɲɯ-ndzur\cmn 这座山上有人站着,模模糊糊地看不清楚\end{exemple}
\begin{relation-sémantique}\synonyme{
\hyperlink{Ⓔdzɤjdzɤj}{\textit{ \papi{dzɤjdzɤj}}}
}\end{relation-sémantique}\end{entrée}

\begin{entrée}
\vedette{\hypertarget{Ⓔzɤmpo}{\papi{ zɤmpo}}}\markboth{zɤmpo}{}\classe{n}
\begin{définition}\fra nourriture pour les morts\end{définition}
\begin{définition}\cmn 供奉死人的食物\end{définition}
\begin{relation-sémantique}\confer{
\hyperlink{Ⓔnɯzɤmpo}{\textit{ \papi{nɯzɤmpo}}}
}\end{relation-sémantique}
\begin{relation-sémantique}\synonyme{
\hyperlink{Ⓔzaχtɤt}{\textit{ \papi{zaχtɤt}}}
}\end{relation-sémantique}\end{entrée}

\begin{entrée}
\vedette{\hypertarget{Ⓔzɤntshaʁ}{\papi{ zɤntshaʁ}}}\markboth{zɤntshaʁ}{}\classe{n}
\begin{définition}\fra plat\end{définition}
\begin{définition}\cmn 食品\end{définition}\end{entrée}

\begin{entrée}
\vedette{\hypertarget{Ⓔzɤsna}{\papi{ zɤsna}}}\markboth{zɤsna}{}
\classe{n}
\begin{définition}\fra nourriture pour les morts\end{définition}
\begin{définition}\cmn 祭祀死人用的食物
\begin{déclaration} \étymologie{\papi{zas.sna}}\end{déclaration}\end{définition}
\begin{exemple}\jya tɯrme pɯ-si tɕe zɤsna chɯ́-wɣ-βde ŋu\cmn 人去世了就要给他供奉一些食物\end{exemple}
\begin{relation-sémantique}\confer{
\hyperlink{Ⓔzaχtɤt}{\textit{ \papi{zaχtɤt}}}
}\end{relation-sémantique}
\begin{relation-sémantique}\confer{
\hyperlink{Ⓔnɯzɤsna}{\textit{ \papi{nɯzɤsna}}}
}\end{relation-sémantique}\end{entrée}

\begin{entrée}
\vedette{\hypertarget{Ⓔzɤt}{\papi{ zɤt}}}\markboth{zɤt}{}\classe{vs}
\paradigme{\textit{dir :} \jya nɯ-}\acception{1}
\begin{définition}\fra s'émousser\end{définition}
\begin{définition}\cmn 磨损\end{définition}
\begin{exemple}\jya qraʁ ɲɤ-zɤt\cmn 铧磨损了\end{exemple}
\begin{relation-sémantique}\synonyme{
\hyperlink{Ⓔsa}{\textit{ \papi{sa}}}
}\end{relation-sémantique}\acception{2}
\begin{définition}\fra disparaître\end{définition}
\begin{définition}\cmn 消失
\begin{déclaration} \étymologie{\papi{zad}}\end{déclaration}\end{définition}
\begin{exemple}\jya jiʑo kɤmɲɯ-skɤt a-mɤ-nɯ-zɤt (a-mɤ-nɯ-mbrɤt, a-thɯ-ɤrɕo)\cmn 但愿我们干木鸟话不会消失\end{exemple}\begin{sous-entrée}
\vedette{\hypertarget{}{\papi{ sɯɣzɤt}}}\markboth{sɯɣzɤt}{}\classe{vt}
\begin{définition}\fra laisser disparaître\end{définition}
\begin{définition}\cmn 让……消失\end{définition}
\begin{exemple}\jya kɤ-sɯɣzɤt nɤja tɕe a-mɤ-nɯ-zɤt\cmn 令它消失太可惜,希望不会消失\end{exemple}
\end{sous-entrée}\end{entrée}

\begin{entrée}
\vedette{\hypertarget{Ⓔzbaʁ}{\papi{ zbaʁ}}}\markboth{zbaʁ}{}
\classe{vs}
\paradigme{\textit{dir :} \jya tɤ-}
\begin{définition}\fra sec\end{définition}
\begin{définition}\cmn 干燥\end{définition}
\begin{exemple}\jya tɯ-nga to-zbaʁ\cmn 衣服干了\end{exemple}
\begin{exemple}\jya tɤ-ɕkho-t-a tɕe, to-zbaʁ\cmn 我晾了就干了\end{exemple}\begin{sous-entrée}
\vedette{\hypertarget{}{\papi{ sɯzbaʁ}}}\markboth{sɯzbaʁ}{}\classe{vt}
\paradigme{\textit{dir :} \jya tɤ-}
\begin{définition}\ 
\begin{déclaration}\grammar{caus}\end{déclaration}\end{définition}
\begin{définition}\fra sécher\end{définition}
\begin{définition}\cmn 弄干\end{définition}
\begin{exemple}\jya qale kɯ a-ŋga to-sɯzbaʁ\cmn 风把衣服(吹干)了\end{exemple}
\begin{relation-sémantique}\synonyme{
\hyperlink{Ⓔrom}{\textit{ \papi{rom}}}
}\end{relation-sémantique}
\begin{relation-sémantique}\synonyme{
\hyperlink{Ⓔkhrɯ}{\textit{ \papi{khrɯ}}}
}\end{relation-sémantique}
\begin{relation-sémantique}\antonyme{
\hyperlink{Ⓔaci}{\textit{ \papi{aci}}}
}\end{relation-sémantique}
\begin{relation-sémantique}\confer{
 \papi{ɣɤzbaʁ}
}\end{relation-sémantique}
\end{sous-entrée}\end{entrée}

\begin{entrée}
\vedette{\hypertarget{Ⓔzbaʁzbɯ}{\papi{ zbaʁzbɯ}}}\markboth{zbaʁzbɯ}{}\classe{vs}
\begin{définition}\fra sec\end{définition}
\begin{définition}\cmn 干\end{définition}
\begin{relation-sémantique}\confer{
\hyperlink{Ⓔzbaʁ}{\textit{ \papi{zbaʁ}}}
}\end{relation-sémantique}\end{entrée}

\begin{entrée}
\vedette{\hypertarget{Ⓔzbraʁ}{\papi{ zbraʁ}}}\markboth{zbraʁ}{}\classe{vt}
\paradigme{\textit{dir :} \jya nɯ-}
\paradigme{\textit{dir :} \jya kɤ-}
\begin{définition}\fra attacher sur quelque chose\end{définition}
\begin{définition}\cmn 绑在另外一个物体\end{définition}
\begin{exemple}\jya tɤ-jtsi ɯ-taʁ ɲɤ-zbraʁ\cmn 把他绑在柱子上了\end{exemple}
\begin{relation-sémantique}\confer{
\hyperlink{Ⓔβraʁ}{\textit{ \papi{βraʁ}}}
}\end{relation-sémantique}\begin{sous-entrée}
\vedette{\hypertarget{}{\papi{ azbraʁ}}}\markboth{azbraʁ}{}\classe{vi}
\begin{définition}\ 
\begin{déclaration}\grammar{pass}\end{déclaration}\end{définition}
\begin{définition}\fra être attaché sur quelque chose\end{définition}
\begin{définition}\cmn 被绑在另外一个物体上\end{définition}
\end{sous-entrée}\end{entrée}

\begin{entrée}
\vedette{\hypertarget{Ⓔzbrilu}{\papi{ zbrilu}}}\markboth{zbrilu}{}\classe{n}
\begin{définition}\fra année du serpent\end{définition}
\begin{définition}\cmn 蛇年
\begin{déclaration} \étymologie{\papi{sbrul.lo}}\end{déclaration}\end{définition}
\end{entrée}

\begin{entrée}
\vedette{\hypertarget{Ⓔzbɯ}{\papi{ zbɯ}}}\markboth{zbɯ}{}\classe{n}
\begin{définition}\ 
\begin{déclaration}\grammar{n.lieu}\end{déclaration}\end{définition}
\begin{définition}\fra Zbu\end{définition}
\begin{définition}\cmn 日部乡\end{définition}\end{entrée}

\begin{entrée}
\vedette{\hypertarget{Ⓔzbɯɣ}{\papi{ zbɯɣ}}}\markboth{zbɯɣ}{}\classe{idph.1}
\begin{définition}\fra bruit d'une falaise qui s'écroule\end{définition}
\begin{définition}\cmn 形容房间里突然密闭昏闷的感觉,或岩石突然倒塌的声音\end{définition}
\begin{exemple}\jya kɯm zbɯɣ ʑo ta-stu\cmn 他关了门,房间里一下子就闷了起来\end{exemple}
\end{entrée}

\begin{entrée}
\vedette{\hypertarget{Ⓔzbɯwa}{\papi{ zbɯwa}}}\markboth{zbɯwa}{}\classe{n}
\begin{définition}\fra lanière pour porter les enfants sur le dos\end{définition}
\begin{définition}\cmn 用来背小孩的带子\end{définition}
\begin{relation-sémantique}\confer{
\hyperlink{Ⓔbɯwa}{\textit{ \papi{bɯwa}}}
}\end{relation-sémantique}
\end{entrée}

\begin{entrée}
\vedette{\hypertarget{Ⓔzdɤβ}{\papi{ zdɤβ}}}\markboth{zdɤβ}{}
\classe{vt}
\paradigme{\textit{dir :} \jya tɤ-}
\begin{définition}\fra plier\end{définition}
\begin{définition}\cmn 折叠;裹起来
\begin{déclaration} \étymologie{\papi{sdeb}}\end{déclaration}\end{définition}
\begin{exemple}\jya tɯmbri ta-zdɤβ\cmn 他把绳子折起来了(弄成双股)\end{exemple}
\begin{relation-sémantique}\synonyme{
\hyperlink{Ⓔltɤβ}{\textit{ \papi{ltɤβ}}}
}\end{relation-sémantique}\end{entrée}

\begin{entrée}
\vedette{\hypertarget{Ⓔzdoŋbu}{\papi{ zdoŋbu}}}\markboth{zdoŋbu}{}\classe{n}
\begin{définition}\fra tronc coupé\end{définition}
\begin{définition}\cmn 大木头
\begin{déclaration} \étymologie{\papi{sdoŋ.po}}\end{déclaration}\end{définition}\end{entrée}

\begin{entrée}
\vedette{\hypertarget{Ⓔzdoʁzdoʁ}{\papi{ zdoʁzdoʁ}}}\markboth{zdoʁzdoʁ}{}\classe{idph.2}
\begin{définition}\fra petit et vif\end{définition}
\begin{définition}\cmn 形容小而鲜活的样子\end{définition}
\begin{exemple}\jya kumpɣɤtɕɯ nɯ zdoʁzdoʁ ʑo ɲɯ-pa\cmn 麻雀小巧玲珑\end{exemple}
\begin{relation-sémantique}\confer{
\hyperlink{Ⓔɣɤzdoʁloʁ}{\textit{ \papi{ɣɤzdoʁloʁ}}}
}\end{relation-sémantique}\end{entrée}

\begin{entrée}
\vedette{\hypertarget{Ⓔzdɯɣ}{\papi{ zdɯɣ}}}\markboth{zdɯɣ}{}\classe{vi}
\paradigme{\textit{dir :} \jya pɯ-}
\begin{définition}\fra pénible\end{définition}
\begin{définition}\cmn 辛苦
\begin{déclaration} \étymologie{\papi{sdug}}\end{déclaration}\end{définition}
\begin{exemple}\jya ɯʑo ɲɯ-zdɯɣ\cmn 他很辛苦\end{exemple}\end{entrée}

\begin{entrée}
\vedette{\hypertarget{Ⓔzdɯɬa}{\papi{ zdɯɬa}}}\markboth{zdɯɬa}{}\classe{n}
\begin{définition}\fra Paeonia sp.\end{définition}
\begin{définition}\cmn 芍药\end{définition}
\begin{exemple}\jya zdɯɬa nɯ zgoku kɯ-mbro kɯ-mbɤr aʁɤndɯndɤt sɯŋgɯ tu-ɬoʁ cha. sɯjno mɤ-mbro tɯ-phɯ ɯ-ŋgɯ ɯ-ru lɤŋɤtʂɤ-ldʑa tu-ɬoʁ cha. tɯ-ldʑa ma kɯ-me ci tu. ɯ-mɯntoʁ kɯ-ɣɯrni ɲɯ-kɯ-lɤt ci tu, nɯ ɣɯ ɯ-qa ɯ-zrɤm nɯ ɣɯrni, mɤʑɯ tɯ-tɯphu tɕe, ɯ-mɯntoʁ kɯ-wɣrum ci tu tɕe, nɯ ɣɯ ɯ-qa ɯ-zrɤm nɯ wɣrum. ɯ-mɯntoʁ wxti, tɯ-rdoʁ tɯ-rdoʁ ɲɯ-lɤt ŋu. ɯ-jwaʁ nɯ thɯ-kɤ-rɤfɕɯfɕɤt fse, ʁnɯ-tɯphu ni ndʑi-jwaʁ ra naχtɕɯɣ. smɤn ɲɯ-sna. ɯ-di ɕŋaʁ ci tu.\cmn 芍药生长在山上山下的森林里。这种草长得不高,有的一棵可以长出五、六根茎,有的只长一根茎。有的开红花,根也是红色的,有的开白花,根也是白色的。花很大,一朵一朵地开,叶子好像是被撕下来的一样。两种芍药的叶子相同。可以入药。有一股臭味。\end{exemple}
\end{entrée}

\begin{entrée}
\vedette{\hypertarget{Ⓔzdɯm}{\papi{ zdɯm}}}\markboth{zdɯm}{}\classe{n}
\begin{définition}\fra nuage, brume\end{définition}
\begin{définition}\cmn 云;雾\end{définition}
\end{entrée}

\begin{entrée}
\vedette{\hypertarget{Ⓔzdɯmkhɤtɕhɯ}{\papi{ zdɯmkhɤtɕhɯ}}}\markboth{zdɯmkhɤtɕhɯ}{}\classe{n}
\begin{définition}\fra humidité de la brume\end{définition}
\begin{définition}\cmn 早上起雾时的水分\end{définition}
\begin{exemple}\jya zdɯm lɤ-ɣe tɕe, zdɯmkhɤtɕhɯ ɣɤʑu\cmn 起雾的时候空气中就有水分\end{exemple}
\begin{exemple}\jya zdɯmkhɤtɕhɯ kɯ tɯ-ŋga ɲɤ-znɯrlɤn\cmn 雾气把衣服弄湿了\end{exemple}
\begin{exemple}\jya tɯ-mɯ zdɯmkhɤtɕhɯ jamar ɲɯ-ɤsɯ-lɤt\cmn 下雨了,雨点又细又密\end{exemple}\end{entrée}

\begin{entrée}
\vedette{\hypertarget{Ⓔzdɯmlaʁrɯʁrɯ}{\papi{ zdɯmlaʁrɯʁrɯ}}}\markboth{zdɯmlaʁrɯʁrɯ}{}\classe{n}
\begin{définition}\fra escargot\end{définition}
\begin{définition}\cmn 蜗牛\end{définition}
\begin{exemple}\jya zdɯmlaʁrɯʁrɯ nɯ qajɯ ci ŋu, ɯ-rqhu tu, ɯ-rqhu nɯ kɯ-ɤrtɯm tɕe ɯ-ŋgɯ kɯ-spoʁ ɲɯ-ŋu, ɯʑo nɯ ɯ-rqhu ɯ-ŋgɯ nɯ tɕu ku-rɤʑi ŋu, ftɕar tɕe ɯ-ŋgɯ chɯ-nɯɬoʁ tɕe, ɯ-rqhu nɯ ɯ-mgɯr ɯ-taʁ ɲɯ-ndzoʁ tɕe tu-nɯfkɯfkur ɲɯ-ŋu, ɯʑo ɯ-phoŋbu nɯ ra kɯ-mpɯ-mpɯ ʑo ɲɯ-ŋu, ɯ-ɕɤrɯ maŋe, ɯ-ʁrɯ kɯnɤ kɯ-mpɯ-mpɯ ɲɯ-ŋu, tɕe ci ci lu-tɕɤt, ci ci chɯ-sɯ-ɕqhlɤt ɲɯ-ŋu, qartsɯ tɕe ɯ-ŋgɯ chɯ-ɕqhlɤt tɕe, kɤ-mto maŋe, ɯ-rqhu nɯ cɯrmbɯ ɯ-ŋgɯ aʁɤndɯndɤt ɣɤʑu.\cmn 蜗牛是一种虫子,有壳。壳是圆形的,有洞。它自己住在洞里面,夏天从壳里出来,把壳背来背去。它身子全是软软的,没有骨头,触角也是软软的,有时候伸出来,有时候收回去。冬天它缩进壳里,看不见。壳在石头堆里到处可以看到。\end{exemple}
\end{entrée}

\begin{entrée}
\vedette{\hypertarget{Ⓔzdɯmqe}{\papi{ zdɯmqe}}}\markboth{zdɯmqe}{}\classe{n}
\begin{définition}\fra une espèce de champignon\end{définition}
\begin{définition}\cmn 一种菌子\end{définition}
\begin{exemple}\jya zdɯmqe nɯ ɕɤr tɯ-mɯ kɤ-lɤt tɕe, soz tɕe stɤmku xɕaj ɣɯ ɯ-rchɤβ kɯ-ɤʁɟa nɯ ra ku-ndzoʁ ŋu. kɯ-ɤlɤɣɯ ŋu ma tɯ-rdoʁ tɯ-rdoʁ maʁ, tɤjmɤɣ fse ri tɯ-mɯ tɤ-jɯm tɕe ɲɯ-me ɕti\cmn 
\stylefv{zdɯmqe}生长在草地、草丛之间的空地上。一般晚上下雨,早上有雾和雾刚散的时候才会长出。它不是单独生长的,也没有规则形状,像菌子,但是天一晴就会消失。
\end{exemple}\end{entrée}

\begin{entrée}
\vedette{\hypertarget{Ⓔzdɯxthɯɣ}{\papi{ zdɯxthɯɣ}}}\markboth{zdɯxthɯɣ}{}\classe{n}
\begin{définition}\fra à la limite de l'acceptable\end{définition}
\begin{définition}\cmn 勉强可以,不理想
\begin{déclaration} \étymologie{\papi{sdug.tʰug}}\end{déclaration}
\begin{déclaration} \étymologie{\papi{sdug.tʰug}}\end{déclaration}\end{définition}
\begin{exemple}\jya a-kɤ-nɤma zdɯxthɯɣ ʑo nɯ-sthɯt-a\cmn 我工作完成得差强人意\end{exemple}\end{entrée}

\begin{entrée}
\vedette{\hypertarget{Ⓔzdɯzdu}{\papi{ zdɯzdu}}}\markboth{zdɯzdu}{}
\classe{idph.2}
\begin{définition}\fra petit, rond et dur\end{définition}
\begin{définition}\cmn 形容圆、硬而小的感觉
\end{définition}
\begin{exemple}\jya tɤ-pɤtso ɲɯ-sɤjndɤt ɯ-jaʁ ra zdɯzdu ʑo ɲɯ-pa\cmn 小孩子很可爱,手又小又圆\end{exemple}\end{entrée}

\begin{entrée}
\vedette{\hypertarget{Ⓔzdɯzdɯr}{\papi{ zdɯzdɯr}}}\markboth{zdɯzdɯr}{} (\variante{zdɯrzdɯr}) \classe{idph.2}
\begin{définition}\fra objet ronds et petits\end{définition}
\begin{définition}\cmn 圆形,很细小的东西(如珠子、豌豆等)\end{définition}
\begin{relation-sémantique}\confer{
\hyperlink{Ⓔɣɤzdɯzdɯr}{\textit{ \papi{ɣɤzdɯzdɯr}}}
}\end{relation-sémantique}
\begin{relation-sémantique}\confer{
\hyperlink{Ⓔstɯrstɯr}{\textit{ \papi{stɯrstɯr}}}
}\end{relation-sémantique}\end{entrée}

\begin{entrée}
\vedette{\hypertarget{ⒺzgaⒽ2}{\papi{ zga}}}\markboth{zga}{}\homonyme{2}
\classe{n}
\begin{définition}\fra sauce\end{définition}
\begin{définition}\cmn 酱\end{définition}
\end{entrée}

\begin{entrée}
\vedette{\hypertarget{ⒺzgaⒽ1}{\papi{ zga}}}\markboth{zga}{}\homonyme{1}
\classe{vs}
\paradigme{\textit{dir :} \jya tɤ-}
\begin{définition}\fra être mûr (abcès)\end{définition}
\begin{définition}\cmn 成熟(脓包、粉刺)\end{définition}
\begin{exemple}\jya tɯ-ɣmbɤβ to-zga tɕe kɤ-tɕɣaʁ to-mda\cmn 脓包成熟了,可以挤了\end{exemple}\end{entrée}

\begin{entrée}
\vedette{\hypertarget{Ⓔzgɤr}{\papi{ zgɤr}}}\markboth{zgɤr}{}
\classe{n}
\begin{définition}\fra tente\end{définition}
\begin{définition}\cmn 帐篷(棉布制成)
\begin{déclaration} \étymologie{\papi{sgar}}\end{déclaration}\end{définition}
\begin{exemple}\jya mbroχpa kɯ zgɤr cho-thɯ\cmn 牧民搭了帐篷\end{exemple}
\begin{exemple}\jya zgɤr thɯ-tʂɯβ-i\cmn 我们缝了帐篷\end{exemple}\end{entrée}

\begin{entrée}
\vedette{\hypertarget{Ⓔzgɤrɕaŋ}{\papi{ zgɤrɕaŋ}}}\markboth{zgɤrɕaŋ}{}\classe{n}
\begin{définition}\fra mât de tente\end{définition}
\begin{définition}\cmn 帐篷杆子
\begin{déclaration} \étymologie{\papi{sgar.ɕiŋ}}\end{déclaration}\end{définition}
\end{entrée}

\begin{entrée}
\vedette{\hypertarget{Ⓔzgɤrtshoʁ}{\papi{ zgɤrtshoʁ}}}\markboth{zgɤrtshoʁ}{}\classe{n}
\begin{définition}\fra piquet de tente\end{définition}
\begin{définition}\cmn 帐篷桩\end{définition}
\end{entrée}

\begin{entrée}
\vedette{\hypertarget{Ⓔzgɤt}{\papi{ zgɤt}}}\markboth{zgɤt}{}\classe{vi.nh}
\begin{définition}\fra devoir\end{définition}
\begin{définition}\cmn 应该\end{définition}
\begin{exemple}\jya ɲɯ-khɤm zgɤt\cmn 他应该给\end{exemple}
\begin{exemple}\jya ɲɯ-kham-a zgɤt\cmn 我应该给\end{exemple}
\begin{exemple}\jya tu-zrɯwxtɯwxti-a zgɤt\cmn 我应该尊重他\end{exemple}
\begin{exemple}\jya ɯ-koŋ nɯ jamar ɲɯ-zgɤt\cmn 这个东西值这个价\end{exemple}
\begin{relation-sémantique}\synonyme{
\hyperlink{Ⓔtʂaŋ}{\textit{ \papi{tʂaŋ}}}
}\end{relation-sémantique}\end{entrée}

\begin{entrée}
\vedette{\hypertarget{Ⓔzgo}{\papi{ zgo}}}\markboth{zgo}{}\classe{n}
\begin{définition}\fra montagne\end{définition}
\begin{définition}\cmn 山\end{définition}
\end{entrée}

\begin{entrée}
\vedette{\hypertarget{Ⓔzgoco}{\papi{ zgoco}}}\markboth{zgoco}{}\classe{n}
\begin{définition}\fra vallée\end{définition}
\begin{définition}\cmn 山沟\end{définition}
\end{entrée}

\begin{entrée}
\vedette{\hypertarget{Ⓔzgoku}{\papi{ zgoku}}}\markboth{zgoku}{}\classe{n}
\begin{définition}\fra pente\end{définition}
\begin{définition}\cmn 山坡\end{définition}
\end{entrée}

\begin{entrée}
\vedette{\hypertarget{Ⓔzgomdʑo}{\papi{ zgomdʑo}}}\markboth{zgomdʑo}{}\classe{n}
\begin{définition}\fra nom d'une fête\end{définition}
\begin{définition}\cmn 看花节
\begin{déclaration}\use{古语}\end{déclaration}\end{définition}
\end{entrée}

\begin{entrée}
\vedette{\hypertarget{Ⓔzgoŋzgoŋ}{\papi{ zgoŋzgoŋ}}}\markboth{zgoŋzgoŋ}{}\classe{idph.2}
\begin{définition}\fra courbé\end{définition}
\begin{définition}\cmn 形容弓起来的样子\end{définition}
\begin{exemple}\jya rgɤtpu nɯ ɯ-phoŋbu zgoŋzgoŋ ʑo ɲɯ-pa\cmn 老年人背是弓着的\end{exemple}\end{entrée}

\begin{entrée}
\vedette{\hypertarget{Ⓔzgoʁ}{\papi{ zgoʁ}}}\markboth{zgoʁ}{}\classe{idph.1}
\begin{définition}\fra tout d'un coup (s'agenouiller)\end{définition}
\begin{définition}\cmn 一下子(跪下)\end{définition}
\begin{exemple}\jya ɯ-χpɯm zgoʁ ʑo pjɤ-tshoʁ\cmn 他一下子跪下了(很恭敬的样子)\end{exemple}
\begin{relation-sémantique}\synonyme{
\hyperlink{Ⓔdzoʁ}{\textit{ \papi{dzoʁ}}}
}\end{relation-sémantique}
\begin{relation-sémantique}\synonyme{
\hyperlink{Ⓔgoʁ}{\textit{ \papi{goʁ}}}
}\end{relation-sémantique}\end{entrée}

\begin{entrée}
\vedette{\hypertarget{Ⓔzgotɕɯ}{\papi{ zgotɕɯ}}}\markboth{zgotɕɯ}{}\classe{n}
\begin{définition}\fra pente\end{définition}
\begin{définition}\cmn 小山坡\end{définition}
\end{entrée}

\begin{entrée}
\vedette{\hypertarget{Ⓔzgrawa}{\papi{ zgrawa}}}\markboth{zgrawa}{}\classe{n}
\begin{définition}\fra sac en cuir\end{définition}
\begin{définition}\cmn 用牛皮缝成的口袋
\begin{déclaration} \étymologie{\papi{sgra.ba}}\end{déclaration}\end{définition}
\begin{exemple}\jya qartshaz ɯ-ndʐi zgrawa\cmn 鹿皮制成的口袋\end{exemple}\end{entrée}

\begin{entrée}
\vedette{\hypertarget{Ⓔzgri}{\papi{ zgri}}}\markboth{zgri}{}\classe{n}
\begin{définition}\fra espèce d'herbe\end{définition}
\begin{définition}\cmn 草的一种\end{définition}
\begin{relation-sémantique}\synonyme{
\hyperlink{Ⓔmɯrkuj}{\textit{ \papi{mɯrkuj}}}
}\end{relation-sémantique}\end{entrée}

\begin{entrée}
\vedette{\hypertarget{ⒺzgroʁⒽ2}{\papi{ zgroʁ}}}\markboth{zgroʁ}{}\homonyme{2}
\classe{n}
\begin{définition}\fra bracelet\end{définition}
\begin{définition}\cmn 手镯
\begin{déclaration} \étymologie{\papi{sgrog}}\end{déclaration}\end{définition}
\end{entrée}

\begin{entrée}
\vedette{\hypertarget{ⒺzgroʁⒽ1}{\papi{ zgroʁ}}}\markboth{zgroʁ}{}\homonyme{1}\classe{vt}
\paradigme{\textit{dir :} \jya tɤ-}
\begin{définition}\fra attacher\end{définition}
\begin{définition}\cmn 绑
\begin{déclaration} \étymologie{\papi{sgrog}}\end{déclaration}\end{définition}
\begin{exemple}\jya laχtɕha to-zgroʁ\cmn 他把东西捆起来了\end{exemple}
\begin{exemple}\jya ɯ-fkur to-zgroʁ\cmn 他把背包捆起来了\end{exemple}
\begin{exemple}\jya tó-wɣ-zgroʁ\cmn 他被绑起来了\end{exemple}
\begin{exemple}\jya sɤrŋgɯŋga tɤ-zgroʁ-a\cmn 我把床单捆起来了(要出发的时候)\end{exemple}\begin{sous-entrée}
\vedette{\hypertarget{}{\papi{ azgroʁ}}}\markboth{azgroʁ}{}\classe{vi}
\begin{définition}\ 
\begin{déclaration}\grammar{pass}\end{déclaration}\end{définition}
\begin{définition}\fra être attaché\end{définition}
\begin{définition}\cmn 被绑\end{définition}
\end{sous-entrée}\end{entrée}

\begin{entrée}
\vedette{\hypertarget{Ⓔzgrɯβ}{\papi{ zgrɯβ}}}\markboth{zgrɯβ}{}
\classe{vi}
\paradigme{\textit{dir :} \jya nɯ-}
\begin{définition}\fra faire avec toute son énergie\end{définition}
\begin{définition}\cmn 一心一意地做一件事情
\begin{déclaration} \étymologie{\papi{sgrub}}\end{déclaration}\end{définition}
\begin{exemple}\jya tɕhɤz kɤ-zgrɯβ\cmn 修佛法\end{exemple}\end{entrée}

\begin{entrée}
\vedette{\hypertarget{Ⓔzgrɯl}{\papi{ zgrɯl}}}\markboth{zgrɯl}{}
\classe{vt}
\paradigme{\textit{dir :} \jya nɯ-}
\begin{définition}\fra rouler entre les mains (sens inverse des aiguilles d'une montre)\end{définition}
\begin{définition}\cmn 搓线(逆时针)
\begin{déclaration} \étymologie{\papi{sgril}}\end{déclaration}\end{définition}
\begin{exemple}\jya tɤ-ri na-zgrɯl\cmn 他搓了线\end{exemple}
\begin{exemple}\jya tɯmbri na-zgrɯl\cmn 他搓了绳子\end{exemple}
\begin{exemple}\jya nɯ-zgrɯl-a\cmn 我搓了\end{exemple}
\begin{relation-sémantique}\confer{
\hyperlink{Ⓔrɤjɯɣ}{\textit{ \papi{rɤjɯɣ}}}
}\end{relation-sémantique}\end{entrée}

\begin{entrée}
\vedette{\hypertarget{Ⓔzgrɯtɕhɯ}{\papi{ zgrɯtɕhɯ}}}\markboth{zgrɯtɕhɯ}{}
\classe{n}
\begin{définition}\fra coup de coude\end{définition}
\begin{définition}\cmn 一肘(打)\end{définition}
\begin{exemple}\jya zgrɯtɕhɯ tɤ-lat-a\cmn 我打了一肘\end{exemple}
\begin{exemple}\jya zgrɯtɕhɯ ma-tɤ-tɯ-lɤt\cmn 你不要用肘打人\end{exemple}
\begin{relation-sémantique}\confer{
\hyperlink{Ⓔnɯzgrɯtɕhɯ}{\textit{ \papi{nɯzgrɯtɕhɯ}}}
}\end{relation-sémantique}
\begin{relation-sémantique}\confer{
\hyperlink{Ⓔtɯ-zgrɯ}{\textit{ \papi{tɯ-zgrɯ}}}
}\end{relation-sémantique}
\begin{relation-sémantique}\confer{
\hyperlink{Ⓔtɕhɯ}{\textit{ \papi{tɕhɯ}}}
}\end{relation-sémantique}\end{entrée}

\begin{entrée}
\vedette{\hypertarget{Ⓔzgɯrmɯɣ}{\papi{ zgɯrmɯɣ}}}\markboth{zgɯrmɯɣ}{}\classe{n}
\begin{définition}\fra mousse\end{définition}
\begin{définition}\cmn 青苔【木路苏】\end{définition}
\end{entrée}

\begin{entrée}
\vedette{\hypertarget{Ⓔzgɯrwɯ}{\papi{ zgɯrwɯ}}}\markboth{zgɯrwɯ}{}\classe{n}
\begin{définition}\fra bosse\end{définition}
\begin{définition}\cmn 驼背
\begin{déclaration} \étymologie{\papi{sgur.ba}}\end{déclaration}\end{définition}
\end{entrée}

\begin{entrée}
\vedette{\hypertarget{Ⓔzgɯt}{\papi{ zgɯt}}}\markboth{zgɯt}{}
\classe{vi}
\paradigme{\textit{dir :} \jya kɤ-}
\begin{définition}\fra rétrécir (habits)\end{définition}
\begin{définition}\cmn 缩水(衣服)
\begin{déclaration}\use{木头缩小不能说\stylefv{zgɯt},要说\stylefv{sɯtɕhaʁ}}\end{déclaration}\end{définition}
\begin{exemple}\jya tɯ-ŋga nɯ-χtɕi-t-a ri ko-zgɯt\cmn 我洗了衣服就缩水了\end{exemple}\end{entrée}

\begin{entrée}
\vedette{\hypertarget{Ⓔzɣa}{\papi{ zɣa}}}\markboth{zɣa}{}
\classe{vs}
\paradigme{\textit{dir :} \jya pɯ-}
\begin{définition}\fra normalement il devrait\end{définition}
\begin{définition}\cmn 按理来说应该……\end{définition}
\begin{exemple}\jya jɯfɕɯr tɯ-mɯ kɯ-lɤt pɯ-zɣa\cmn 按道理,昨天晚上应该下雨(结果没有下雨)\end{exemple}
\begin{exemple}\jya qale kɯ-βzu pɯ-zɣa\cmn 按道理,应该有风\end{exemple}
\begin{exemple}\jya tɯ-mɯ kɯ-jɯm pɯ-zɣa\cmn 按道理,应该天晴\end{exemple}
\begin{exemple}\jya pjɯ-ɕaβ ɲɯ-zɣa ri mɯ́j-ɕaβ\cmn 按道理应该够长,但是不够长\end{exemple}
\begin{exemple}\jya tʂu mɤ-kɯ-zɣa ʑo pjɤ-mbɯt\cmn 路莫名其妙地塌下来了\end{exemple}\end{entrée}

\begin{entrée}
\vedette{\hypertarget{Ⓔzɣɤβlo}{\papi{ zɣɤβlo}}}\markboth{zɣɤβlo}{}
\begin{relation-sémantique}\confer{
\hyperlink{Ⓔɣɤβlo}{\textit{ \papi{ɣɤβlo}}}
}\end{relation-sémantique}\end{entrée}

\begin{entrée}
\vedette{\hypertarget{Ⓔzɣɤɕɯɴqoʁ}{\papi{ zɣɤɕɯɴqoʁ}}}\markboth{zɣɤɕɯɴqoʁ}{}
\begin{relation-sémantique}\confer{
\hyperlink{Ⓔɕɯɴqoʁ}{\textit{ \papi{ɕɯɴqoʁ}}}
}\end{relation-sémantique}\end{entrée}

\begin{entrée}
\vedette{\hypertarget{Ⓔzɣɤdi}{\papi{ zɣɤdi}}}\markboth{zɣɤdi}{}
\begin{relation-sémantique}\confer{
\hyperlink{Ⓔɣɤdi}{\textit{ \papi{ɣɤdi}}}
}\end{relation-sémantique}\end{entrée}

\begin{entrée}
\vedette{\hypertarget{Ⓔzɣɤji}{\papi{ zɣɤji}}}\markboth{zɣɤji}{}
\begin{relation-sémantique}\confer{
\hyperlink{Ⓔɣɤji}{\textit{ \papi{ɣɤji}}}
}\end{relation-sémantique}\end{entrée}

\begin{entrée}
\vedette{\hypertarget{Ⓔzɣɤmbu}{\papi{ zɣɤmbu}}}\markboth{zɣɤmbu}{}\classe{n}
\begin{définition}\fra balai\end{définition}
\begin{définition}\cmn 扫帚\end{définition}
\end{entrée}

\begin{entrée}
\vedette{\hypertarget{Ⓔzɣɤngɯt}{\papi{ zɣɤngɯt}}}\markboth{zɣɤngɯt}{}
\begin{relation-sémantique}\confer{
\hyperlink{Ⓔɣɤngɯt}{\textit{ \papi{ɣɤngɯt}}}
}\end{relation-sémantique}\end{entrée}

\begin{entrée}
\vedette{\hypertarget{Ⓔzɣɤŋgi}{\papi{ zɣɤŋgi}}}\markboth{zɣɤŋgi}{}
\begin{relation-sémantique}\confer{
\hyperlink{Ⓔɣɤŋgi}{\textit{ \papi{ɣɤŋgi}}}
}\end{relation-sémantique}\end{entrée}

\begin{entrée}
\vedette{\hypertarget{Ⓔzɣɤrzɣɤr}{\papi{ zɣɤrzɣɤr}}}\markboth{zɣɤrzɣɤr}{}\classe{idph.2}
\begin{définition}\fra qui prend de la place mais qui n'est pas lourd\end{définition}
\begin{définition}\cmn 形容松软的东西(棉絮、泡沫塑料等)虽然不重,但体积很大,很占地方的样子\end{définition}
\end{entrée}

\begin{entrée}
\vedette{\hypertarget{Ⓔzɣɤʁre}{\papi{ zɣɤʁre}}}\markboth{zɣɤʁre}{}
\begin{relation-sémantique}\confer{
\hyperlink{Ⓔɣɤʁre}{\textit{ \papi{ɣɤʁre}}}
}\end{relation-sémantique}\end{entrée}

\begin{entrée}
\vedette{\hypertarget{Ⓔzɣɤtɕa}{\papi{ zɣɤtɕa}}}\markboth{zɣɤtɕa}{}
\begin{relation-sémantique}\confer{
\hyperlink{Ⓔɣɤtɕa}{\textit{ \papi{ɣɤtɕa}}}
}\end{relation-sémantique}\end{entrée}

\begin{entrée}
\vedette{\hypertarget{Ⓔzɣɤwu}{\papi{ zɣɤwu}}}\markboth{zɣɤwu}{}
\begin{relation-sémantique}\confer{
\hyperlink{Ⓔɣɤwu}{\textit{ \papi{ɣɤwu}}}
}\end{relation-sémantique}\end{entrée}

\begin{entrée}
\vedette{\hypertarget{Ⓔzɣoma}{\papi{ zɣoma}}}\markboth{zɣoma}{}\classe{n}
\begin{définition}\fra lie\end{définition}
\begin{définition}\cmn 酒糟\end{définition}\end{entrée}

\begin{entrée}
\vedette{\hypertarget{Ⓔzɣɯmphrɯmphru}{\papi{ zɣɯmphrɯmphru}}}\markboth{zɣɯmphrɯmphru}{} (\variante{\_zɣɯmphɯmphru}) \classe{vt}
\paradigme{\textit{dir :} \jya pɯ-}
\begin{définition}\fra faire en continu, à de nombreuses reprises\end{définition}
\begin{définition}\cmn 接二连三地做\end{définition}
\begin{exemple}\jya tɯ-fkur tɕɤkɯ a-pɯ-ɤta, nɯ z-ɲɯ́-wɣ-zɣɯmphrɯmphru jɤɣ\cmn 那一背(东西)放在那边,可以接着把它背过来\end{exemple}
\begin{exemple}\jya sɯfkur ɕ-pɯ-zɣɯmprɯmphru-t-a\cmn 我接二连三地去背了柴\end{exemple}
\begin{exemple}\jya pɤjkhu kutɕu ɯ-skɤt kɤ-zɣɯmphɯmphru ʑo kɤ-ti mɯ́j-cha\cmn 这里的话,我还不能讲得很流利\end{exemple}\end{entrée}

\begin{entrée}
\vedette{\hypertarget{Ⓔzɣɯmphɯmphru}{\papi{ zɣɯmphɯmphru}}}\markboth{zɣɯmphɯmphru}{}
\begin{relation-sémantique}\confer{
\hyperlink{Ⓔaɣɯmphɯmphru}{\textit{ \papi{aɣɯmphɯmphru}}}
}\end{relation-sémantique}\end{entrée}

\begin{entrée}
\vedette{\hypertarget{Ⓔzɣɯŋgɯŋgɯ}{\papi{ zɣɯŋgɯŋgɯ}}}\markboth{zɣɯŋgɯŋgɯ}{}
\begin{relation-sémantique}\confer{
\hyperlink{Ⓔaɣɯŋgɯŋgɯ}{\textit{ \papi{aɣɯŋgɯŋgɯ}}}
}\end{relation-sémantique}\end{entrée}

\begin{entrée}
\vedette{\hypertarget{Ⓔzɣɯqhu}{\papi{ zɣɯqhu}}}\markboth{zɣɯqhu}{}\classe{n}
\begin{définition}\fra partie du fardeau opposée au dos du porteur\end{définition}
\begin{définition}\cmn 柴捆子的后边部分,不接触人的背部(比较粗的木柴)\end{définition}
\begin{relation-sémantique}\antonyme{
\hyperlink{Ⓔzrɯβɟu}{\textit{ \papi{zrɯβɟu}}}
}\end{relation-sémantique}\end{entrée}

\begin{entrée}
\vedette{\hypertarget{Ⓔzɣɯrkɯrkɯ}{\papi{ zɣɯrkɯrkɯ}}}\markboth{zɣɯrkɯrkɯ}{}
\begin{relation-sémantique}\confer{
\hyperlink{Ⓔaɣɯrkɯrkɯ}{\textit{ \papi{aɣɯrkɯrkɯ}}}
}\end{relation-sémantique}\end{entrée}

\begin{entrée}
\vedette{\hypertarget{Ⓔzɣɯrndi}{\papi{ zɣɯrndi}}}\markboth{zɣɯrndi}{}\classe{n}
\begin{définition}\fra offrandes rituelles\end{définition}
\begin{définition}\cmn (经堂上的)贡品\end{définition}\end{entrée}

\begin{entrée}
\vedette{\hypertarget{Ⓔzɣɯrni}{\papi{ zɣɯrni}}}\markboth{zɣɯrni}{}
\begin{relation-sémantique}\confer{
\hyperlink{Ⓔɣɯrni}{\textit{ \papi{ɣɯrni}}}
}\end{relation-sémantique}\end{entrée}

\begin{entrée}
\vedette{\hypertarget{Ⓔzɣɯt}{\papi{ zɣɯt}}}\markboth{zɣɯt}{}
\classe{vi}
\paradigme{\textit{dir :} \jya \_}
\paradigme{\textit{perfective stem (1st and 3th persons) :} \jya azɣɯt}
\begin{définition}\fra arriver\end{définition}
\begin{définition}\cmn 到达\end{définition}
\begin{exemple}\jya ʑa jo-tɯ-ʑɣɯt\cmn 你早就到了\end{exemple}
\begin{exemple}\jya ʑa jɤ-azɣɯt-a\cmn 我早就到了\end{exemple}
\begin{exemple}\jya nɯ kóʁmɯz lɤ-azɣɯt loβ\cmn 他刚刚才到呢\end{exemple}
\begin{exemple}\jya a-jaʁ jɤ-azɣɯt\cmn 我收到了\end{exemple}
\begin{exemple}\jya mɯ-ɕɯ-tɯ-zɣɯt nɯ-sɯso-t-a\cmn 我怕你到达不了\end{exemple}
\begin{exemple}\jya a-slama nɯ laʁnɤ-rʑaʁ tɕe nɤ-ɕki zɣɯt\cmn 我的学生过几天就会到你那里\end{exemple}\begin{sous-entrée}
\vedette{\hypertarget{}{\papi{ nɯzɣɯt}}}\markboth{nɯzɣɯt}{}\classe{vi}
\begin{définition}\ 
\begin{déclaration}\grammar{vert}\end{déclaration}\end{définition}
\begin{définition}\fra rentrer chez soi\end{définition}
\begin{définition}\cmn 安全回家\end{définition}
\end{sous-entrée}\end{entrée}

\begin{entrée}
\vedette{\hypertarget{Ⓔzɣɯtɕɯtɕɤβ}{\papi{ zɣɯtɕɯtɕɤβ}}}\markboth{zɣɯtɕɯtɕɤβ}{}
\classe{vt}
\paradigme{\textit{dir :} \jya tɤ-}
\begin{définition}\fra entrecroiser, assortir l'un après l'autre\end{définition}
\begin{définition}\cmn 连续地搭配(不同的颜色)\end{définition}
\begin{exemple}\jya ʑaka ɯ-mdoʁ khatoʁ tú-wɣ-zɣɯtɕɯtɕɤβ tɕe mpɕɤr\cmn 把每个颜色搭配成一路一路就美观\end{exemple}
\begin{exemple}\jya tɯ-kɯ-mŋɤm tɤ-tu tɕe, kɤ-nɯna cho kɤ-rɤma tú-wɣ-zɣɯtɕɯtɕɤβ tɕe pe\cmn 生病的时候,最好把工作和休息调理好\end{exemple}\end{entrée}

\begin{entrée}
\vedette{\hypertarget{Ⓔzjaŋzjaŋ}{\papi{ zjaŋzjaŋ}}}\markboth{zjaŋzjaŋ}{}\classe{idph.2}
\begin{définition}\fra haut\end{définition}
\begin{définition}\cmn 身子高(比其他人高)\end{définition}
\begin{exemple}\jya mbro ɯ-taʁ zjaŋzjaŋ to-ɕe\cmn 他骑上了马,显得很高\end{exemple}\begin{sous-entrée}
\vedette{\hypertarget{}{\papi{ mɤlɤzjaŋ}}}\markboth{mɤlɤzjaŋ}{}\classe{idph.6}
\begin{exemple}\jya a-ɣe mɤlɤzjaŋ ʑo thɯ-aβzu (=zjaŋzjaŋ ʑo kɯ-pa thɯ-aβzu)\cmn 我的孙子战法得很高了\end{exemple}
\end{sous-entrée}\begin{sous-entrée}
\vedette{\hypertarget{}{\papi{ phɯzjaŋ}}}\markboth{phɯzjaŋ}{}\classe{idph.5}
\begin{exemple}\jya phɯzjaŋ ʑo tɤ-ndzur\cmn 他突然间站起来了,显得比别人高\end{exemple}
\end{sous-entrée}\begin{sous-entrée}
\vedette{\hypertarget{}{\papi{ zjaŋnɤzjaŋ}}}\markboth{zjaŋnɤzjaŋ}{}\classe{idph.3}
\begin{définition}\ 
\begin{déclaration}\use{双数形式表示马和骑马的人在一起}\end{déclaration}\end{définition}
\begin{exemple}\jya mbro ɯ-taʁ to-ɕe tɕe, zjaŋnɤzjaŋ jɤ-ari-ndʑi\cmn 他骑上了马,就往上游去了,显得很高\end{exemple}
\end{sous-entrée}\begin{sous-entrée}
\vedette{\hypertarget{}{\papi{ zjaŋɯŋi}}}\markboth{zjaŋɯŋi}{}\classe{idph.7}
\begin{exemple}\jya zjaŋɯŋi ʑo jɤ-ari\cmn 他慢慢地走了(身子很高的人)\end{exemple}
\end{sous-entrée}\begin{sous-entrée}
\vedette{\hypertarget{}{\papi{ zjɯŋɯzjaŋi}}}\markboth{zjɯŋɯzjaŋi}{}\classe{idph.8}
\begin{exemple}\jya zjɯŋɯzjaŋi ɲɯ-xcat\cmn 很多人在一起,高矮不一\end{exemple}
\begin{exemple}\jya zgo zjɯŋɯzjaŋi ɲɯ-xcat\cmn 山很多,高低不一\end{exemple}
\begin{relation-sémantique}\confer{
\hyperlink{Ⓔɣɤzjaŋlaŋ}{\textit{ \papi{ɣɤzjaŋlaŋ}}}
}\end{relation-sémantique}
\begin{relation-sémantique}\confer{
\hyperlink{Ⓔsɤzjaŋlaŋ}{\textit{ \papi{sɤzjaŋlaŋ}}}
}\end{relation-sémantique}
\begin{relation-sémantique}\confer{
\hyperlink{Ⓔɣɤzjaŋzjaŋ}{\textit{ \papi{ɣɤzjaŋzjaŋ}}}
}\end{relation-sémantique}
\begin{relation-sémantique}\confer{
\hyperlink{Ⓔsɤzjaŋzjaŋ}{\textit{ \papi{sɤzjaŋzjaŋ}}}
}\end{relation-sémantique}
\begin{relation-sémantique}\confer{
\hyperlink{Ⓔnɯzjaŋ}{\textit{ \papi{nɯzjaŋ}}}
}\end{relation-sémantique}
\begin{relation-sémantique}\confer{
\hyperlink{Ⓔzjɤɣzjɤɣ}{\textit{ \papi{zjɤɣzjɤɣ}}}
}\end{relation-sémantique}
\begin{relation-sémantique}\confer{
\hyperlink{Ⓔtsjaŋtsjaŋ}{\textit{ \papi{tsjaŋtsjaŋ}}}
}\end{relation-sémantique}
\end{sous-entrée}\end{entrée}

\begin{entrée}
\vedette{\hypertarget{Ⓔzjɤɣzjɤɣ}{\papi{ zjɤɣzjɤɣ}}}\markboth{zjɤɣzjɤɣ}{}\classe{idph.2}
\begin{définition}\fra grand, élevé\end{définition}
\begin{définition}\cmn 形容个子高,物体因为数量多而堆得很高的样子;或形容人沉默不语的模样\end{définition}
\begin{exemple}\jya tɯrme zjɤɣzjɤɣ ʑo ɲɯ-ɤmdzɯ ɲɯ-rɤʑi\cmn 那个人沉默不语地在那里坐着\end{exemple}
\begin{exemple}\jya tɯ-ɣli zjɤɣzjɤɣ ʑo to-rmbɯ-nɯ\cmn 他们那肥料堆得很高(肥料多)\end{exemple}
\begin{exemple}\jya tɤ-fkɯm ɯ-ŋgɯ zjɤɣzjɤɣ ʑo cho-rku\cmn 口袋里装得很满\end{exemple}\begin{sous-entrée}
\vedette{\hypertarget{}{\papi{ ɣɤzjɤɣlɤɣ}}}\markboth{ɣɤzjɤɣlɤɣ}{}\classe{vi}
\paradigme{\textit{dir :} \jya tɤ-}
\begin{définition}\fra se balancer, se dandiner, se pas tenir en place sur sa chaise\end{définition}
\begin{définition}\cmn 不停地摇动,坐不住\end{définition}
\begin{exemple}\jya tɕɤndi ɲɯ-nɤŋkɯŋke tɕe ɲɯ-ɣɤzjɤɣlɤɣ\cmn 他在边走动\end{exemple}
\begin{exemple}\jya ma-tɯ-ɣɤzjɤɣlɤɣ ntsɯ, phoʁphoʁ kɤ-ɤmdzɯ\cmn 别动,坐好一点\end{exemple}
\begin{exemple}\jya pri ɲɯ-ɣɤzjɤɣlɤɣ ntsɯ\cmn 老熊在动来动去\end{exemple}
\begin{exemple}\jya χpɯn kɤ-ndɯn ɯ-raŋ tɕe tu-ɣɤzjɤɣlɤɣ ntsɯ ŋu\cmn 和尚念经的时候摇头\end{exemple}
\end{sous-entrée}\begin{sous-entrée}
\vedette{\hypertarget{}{\papi{ mɤlɤzjɤɣ}}}\markboth{mɤlɤzjɤɣ}{}\classe{idph.5}
\begin{exemple}\jya @yangyu mɤlɤzjɤɣ to-rku\cmn 他把洋芋装得满满的\end{exemple}
\end{sous-entrée}\begin{sous-entrée}
\vedette{\hypertarget{}{\papi{ phɯzjɤɣ}}}\markboth{phɯzjɤɣ}{}\classe{idph.6}
\begin{exemple}\jya tɯrme phɯzjɤɣ tɤ-nɯɬoʁ\cmn 突然冒出个人来\end{exemple}
\end{sous-entrée}\begin{sous-entrée}
\vedette{\hypertarget{}{\papi{ sɤzjɤɣlɤɣ}}}\markboth{sɤzjɤɣlɤɣ}{}\classe{vt}
\paradigme{\textit{dir :} \jya tɤ-}
\begin{exemple}\jya si nɯ-ndʐaβ tɕe tɤ́-wɣ-sɤzjɤɣlɤɣ-a\cmn 树倒过来了,差一点把我弄倒了\end{exemple}
\begin{relation-sémantique}\confer{
\hyperlink{Ⓔzjɤɣzjɤɣ}{\textit{ \papi{zjɤɣzjɤɣ}}}
}\end{relation-sémantique}
\end{sous-entrée}\begin{sous-entrée}
\vedette{\hypertarget{}{\papi{ zjɤɣnɤlɤɣ}}}\markboth{zjɤɣnɤlɤɣ}{}\classe{idph.4}
\begin{exemple}\jya zjɤɣnɤlɤɣ ɲɯ-ʑɣɤstu\cmn 他在扭动,到处东瞻西望\end{exemple}
\end{sous-entrée}\begin{sous-entrée}
\vedette{\hypertarget{}{\papi{ zjɤɣnɤzjɤɣ}}}\markboth{zjɤɣnɤzjɤɣ}{}\classe{idph.3}
\begin{exemple}\jya laχtɕha zjɤɣnɤzjɤɣ ɲɯ-ɤsɯ-fkur kɤ-ari\cmn 他背着一大堆东西,走了\end{exemple}
\begin{exemple}\jya zjɤɣnɤzjɤɣ ɲɯ-ŋke\cmn 他走路一脚高一脚低的。\end{exemple}
\end{sous-entrée}\begin{sous-entrée}
\vedette{\hypertarget{}{\papi{ zjɤɣɯɣi}}}\markboth{zjɤɣɯɣi}{}\classe{idph.7}
\begin{exemple}\jya tɯrme zjɤɣɯɣi lɤ-ari\cmn 那个人慢慢地离开了\end{exemple}
\end{sous-entrée}\end{entrée}

\begin{entrée}
\vedette{\hypertarget{Ⓔzɟada}{\papi{ zɟada}}}\markboth{zɟada}{}\classe{n}
\begin{définition}\fra nain\end{définition}
\begin{définition}\cmn 矮人;侏儒\end{définition}\end{entrée}

\begin{entrée}
\vedette{\hypertarget{Ⓔzɟaŋzɟaŋ}{\papi{ zɟaŋzɟaŋ}}}\markboth{zɟaŋzɟaŋ}{}\classe{idph.2}
\begin{définition}\fra mou et enflé\end{définition}
\begin{définition}\cmn 形容软而鼓起的样子\end{définition}
\begin{exemple}\jya lʁa ɯ-ŋgɯ laχtɕha khro to-rku-nɯ zɟaŋzɟaŋ ʑo ɲɯ-pa\cmn 他们在口袋里装了很多东西,显得鼓鼓囊囊的\end{exemple}
\begin{relation-sémantique}\synonyme{
\hyperlink{Ⓔzɟraŋzɟraŋ}{\textit{ \papi{zɟraŋzɟraŋ}}}
}\end{relation-sémantique}\end{entrée}

\begin{entrée}
\vedette{\hypertarget{Ⓔzɟɤɣzɟɤɣ}{\papi{ zɟɤɣzɟɤɣ}}}\markboth{zɟɤɣzɟɤɣ}{}\classe{idph.2}
\begin{définition}\fra court et épais\end{définition}
\begin{définition}\cmn 形容粗而短的样子\end{définition}\end{entrée}

\begin{entrée}
\vedette{\hypertarget{Ⓔzɟi}{\papi{ zɟi}}}\markboth{zɟi}{}\classe{n}
\begin{définition}\fra sac en poils de yak\end{définition}
\begin{définition}\cmn 毛织布袋
\begin{déclaration} \étymologie{\papi{sgʲe}}\end{déclaration}\end{définition}
\end{entrée}

\begin{entrée}
\vedette{\hypertarget{Ⓔzɟoʁzɟoʁ}{\papi{ zɟoʁzɟoʁ}}}\markboth{zɟoʁzɟoʁ}{}\classe{idph.2}
\begin{définition}\fra petit\end{définition}
\begin{définition}\cmn 形容身材矮小\end{définition}
\begin{exemple}\jya tɤ-pɤtso nɯ zɟoʁzɟoʁ ʑo ɲɯ-pa\cmn 那个孩子个子矮矮的\end{exemple}
\begin{relation-sémantique}\antonyme{
\hyperlink{Ⓔzjɤɣzjɤɣ}{\textit{ \papi{zjɤɣzjɤɣ}}}
}\end{relation-sémantique}
\begin{relation-sémantique}\antonyme{
\hyperlink{Ⓔzjaŋzjaŋ}{\textit{ \papi{zjaŋzjaŋ}}}
}\end{relation-sémantique}\end{entrée}

\begin{entrée}
\vedette{\hypertarget{Ⓔzɟraŋzɟraŋ}{\papi{ zɟraŋzɟraŋ}}}\markboth{zɟraŋzɟraŋ}{}\classe{idph.2}
\begin{définition}\fra mou et enflé\end{définition}
\begin{définition}\cmn 形容软而鼓起的样子\end{définition}
\begin{exemple}\jya ɯ-xtu zɟraŋzɟraŋ ʑo pa\cmn 他肚子鼓起来,软乎乎的样子\end{exemple}
\begin{relation-sémantique}\synonyme{
\hyperlink{Ⓔzɟaŋzɟaŋ}{\textit{ \papi{zɟaŋzɟaŋ}}}
}\end{relation-sémantique}\end{entrée}

\begin{entrée}
\vedette{\hypertarget{Ⓔzɟɯɣ}{\papi{ zɟɯɣ}}}\markboth{zɟɯɣ}{}
\classe{idph.1}
\begin{définition}\fra bruit d'un objet lourd qui tombe de très haut\end{définition}
\begin{définition}\cmn 形容重的物体从高处落地(震动地面)发出的声音\end{définition}
\begin{exemple}\jya zɟɯɣ ɲɯ-ti pa-ɣɤrɤt\cmn 他扔下去了,发出震动的声音\end{exemple}\begin{sous-entrée}
\vedette{\hypertarget{}{\papi{ zɟɯɣnɤzɟɯɣ}}}\markboth{zɟɯɣnɤzɟɯɣ}{}\classe{idph.3}
\begin{exemple}\jya ɕɤmɯɣdɯ zɟɯɣnɤzɟɯɣ ɲɯ-ɤsɯ-lɤt\cmn 他在射枪,发出震动的声音\end{exemple}
\end{sous-entrée}\end{entrée}

\begin{entrée}
\vedette{\hypertarget{Ⓔzmɤŋgɯ}{\papi{ zmɤŋgɯ}}}\markboth{zmɤŋgɯ}{}
\begin{relation-sémantique}\confer{
\hyperlink{Ⓔmɤŋgɯ}{\textit{ \papi{mɤŋgɯ}}}
}\end{relation-sémantique}\end{entrée}

\begin{entrée}
\vedette{\hypertarget{Ⓔzmɤpɕi}{\papi{ zmɤpɕi}}}\markboth{zmɤpɕi}{}
\begin{relation-sémantique}\confer{
\hyperlink{Ⓔmɤpɕi}{\textit{ \papi{mɤpɕi}}}
}\end{relation-sémantique}\end{entrée}

\begin{entrée}
\vedette{\hypertarget{Ⓔzmɤrɤβ}{\papi{ zmɤrɤβ}}}\markboth{zmɤrɤβ}{}
\classe{vt}
\paradigme{\textit{dir :} \jya tɤ-}
\begin{définition}\fra manger en mélangeant avec\end{définition}
\begin{définition}\cmn 合着吃\end{définition}
\begin{exemple}\jya qajɣi cho tɤjko tɤ-zmɤraβ-a\cmn 我把馍馍和菜和着吃了\end{exemple}\end{entrée}

\begin{entrée}
\vedette{\hypertarget{Ⓔzmɤrtsaβ}{\papi{ zmɤrtsaβ}}}\markboth{zmɤrtsaβ}{}
\begin{relation-sémantique}\confer{
\hyperlink{Ⓔmɤrtsaβ}{\textit{ \papi{mɤrtsaβ}}}
}\end{relation-sémantique}\end{entrée}

\begin{entrée}
\vedette{\hypertarget{Ⓔzmɯjqha}{\papi{ zmɯjqha}}}\markboth{zmɯjqha}{}\classe{vt}
\paradigme{\textit{dir :} \jya tɤ-}
\begin{définition}\fra offenser\end{définition}
\begin{définition}\cmn 得罪\end{définition}
\begin{exemple}\jya tɤ-zmɯjqha-t-a\cmn 我得罪了他\end{exemple}
\begin{relation-sémantique}\confer{
\hyperlink{Ⓔqha}{\textit{ \papi{qha}}}
}\end{relation-sémantique}\end{entrée}

\begin{entrée}
\vedette{\hypertarget{Ⓔzmɯjrɯ}{\papi{ zmɯjrɯ}}}\markboth{zmɯjrɯ}{}
\begin{relation-sémantique}\confer{
\hyperlink{Ⓔmɯjrɯ}{\textit{ \papi{mɯjrɯ}}}
}\end{relation-sémantique}\end{entrée}

\begin{entrée}
\vedette{\hypertarget{Ⓔzmɯnmu}{\papi{ zmɯnmu}}}\markboth{zmɯnmu}{}
\begin{relation-sémantique}\confer{
\hyperlink{Ⓔmɯnmu}{\textit{ \papi{mɯnmu}}}
}\end{relation-sémantique}\end{entrée}

\begin{entrée}
\vedette{\hypertarget{Ⓔzmɯrmbɯ}{\papi{ zmɯrmbɯ}}}\markboth{zmɯrmbɯ}{}
\begin{relation-sémantique}\confer{
\hyperlink{Ⓔamɯrmbɯ}{\textit{ \papi{amɯrmbɯ}}}
}\end{relation-sémantique}\end{entrée}

\begin{entrée}
\vedette{\hypertarget{Ⓔznaʁjɯβ}{\papi{ znaʁjɯβ}}}\markboth{znaʁjɯβ}{}
\begin{relation-sémantique}\confer{
\hyperlink{Ⓔnaʁjɯβ}{\textit{ \papi{naʁjɯβ}}}
}\end{relation-sémantique}\end{entrée}

\begin{entrée}
\vedette{\hypertarget{Ⓔznaχɕɯχɕu}{\papi{ znaχɕɯχɕu}}}\markboth{znaχɕɯχɕu}{}\classe{vi}
\begin{définition}\ 
\begin{déclaration}\grammar{refl}\end{déclaration}
\begin{déclaration}\grammar{trop}\end{déclaration}\end{définition}
\begin{définition}\fra se croire plus fort (que les autres)\end{définition}
\begin{définition}\cmn 自以为强\end{définition}
\begin{exemple}\jya kɤ-znaχɕɯχɕu mɤ-tɯ-nɯ-cha ma nɤʑo sɤz kɯ-χɕu tu\cmn 你用不着逞能,比你有能力的大有人在\end{exemple}
\begin{relation-sémantique}\confer{
\hyperlink{Ⓔχɕu}{\textit{ \papi{χɕu}}}
}\end{relation-sémantique}
\begin{relation-sémantique}\confer{
\hyperlink{Ⓔrɯqhaχɕu}{\textit{ \papi{rɯqhaχɕu}}}
}\end{relation-sémantique}\end{entrée}

\begin{entrée}
\vedette{\hypertarget{Ⓔznaχsoz}{\papi{ znaχsoz}}}\markboth{znaχsoz}{}
\begin{relation-sémantique}\confer{
\hyperlink{Ⓔnaχsoz}{\textit{ \papi{naχsoz}}}
}\end{relation-sémantique}\end{entrée}

\begin{entrée}
\vedette{\hypertarget{Ⓔznaχtɕɯɣ}{\papi{ znaχtɕɯɣ}}}\markboth{znaχtɕɯɣ}{}
\begin{relation-sémantique}\confer{
\hyperlink{Ⓔnaχtɕɯɣ}{\textit{ \papi{naχtɕɯɣ}}}
}\end{relation-sémantique}\end{entrée}

\begin{entrée}
\vedette{\hypertarget{Ⓔznɤβʁaβʁa}{\papi{ znɤβʁaβʁa}}}\markboth{znɤβʁaβʁa}{}
\begin{relation-sémantique}\confer{
\hyperlink{Ⓔβʁa}{\textit{ \papi{βʁa}}}
}\end{relation-sémantique}\end{entrée}

\begin{entrée}
\vedette{\hypertarget{Ⓔznɤchacha}{\papi{ znɤchacha}}}\markboth{znɤchacha}{}
\classe{vi}
\paradigme{\textit{dir :} \jya tɤ-}
\begin{définition}\ 
\begin{déclaration}\grammar{refl}\end{déclaration}
\begin{déclaration}\grammar{trop}\end{déclaration}\end{définition}
\begin{définition}\fra fanfaronner\end{définition}
\begin{définition}\cmn 逞能\end{définition}
\begin{exemple}\jya jiɕqha kɯ-znɤchacha ci ɲɯ-ŋu\cmn 他是个逞能的人\end{exemple}
\begin{relation-sémantique}\confer{
\hyperlink{Ⓔznɤtʂhɯtʂhɯt}{\textit{ \papi{znɤtʂhɯtʂhɯt}}}
}\end{relation-sémantique}\end{entrée}

\begin{entrée}
\vedette{\hypertarget{Ⓔznɤɕqa}{\papi{ znɤɕqa}}}\markboth{znɤɕqa}{}
\begin{relation-sémantique}\confer{
\hyperlink{Ⓔnɤɕqa}{\textit{ \papi{nɤɕqa}}}
}\end{relation-sémantique}\end{entrée}

\begin{entrée}
\vedette{\hypertarget{Ⓔznɤɕqɯɕqraʁ}{\papi{ znɤɕqɯɕqraʁ}}}\markboth{znɤɕqɯɕqraʁ}{} (\variante{znɤɕqraʁɕqraʁ}) \classe{vi}
\begin{définition}\ 
\begin{déclaration}\grammar{refl}\end{déclaration}
\begin{déclaration}\grammar{trop}\end{déclaration}\end{définition}
\begin{définition}\fra se croire intelligent\end{définition}
\begin{définition}\cmn 自以为聪明\end{définition}
\begin{exemple}\jya kɯ-znɤɕqraʁɕqraʁ ci ɲɯ-ŋu\cmn 他是个自以为聪明的人\end{exemple}
\begin{exemple}\jya nɤʑo ɲɯ-tɯ-znɤɕqraʁɕqraʁ\cmn 你自以为很聪明\end{exemple}
\begin{relation-sémantique}\confer{
\hyperlink{Ⓔɕqraʁ}{\textit{ \papi{ɕqraʁ}}}
}\end{relation-sémantique}
\begin{relation-sémantique}\synonyme{
\hyperlink{Ⓔznɤchacha}{\textit{ \papi{znɤchacha}}}
}\end{relation-sémantique}
\begin{relation-sémantique}\synonyme{
\hyperlink{Ⓔrɯχparɤβ}{\textit{ \papi{rɯχparɤβ}}}
}\end{relation-sémantique}\end{entrée}

\begin{entrée}
\vedette{\hypertarget{Ⓔznɤftɕɤftɕɤl}{\papi{ znɤftɕɤftɕɤl}}}\markboth{znɤftɕɤftɕɤl}{}
\classe{vi}
\paradigme{\textit{dir :} \jya tɤ-}
\begin{définition}\ 
\begin{déclaration}\grammar{refl}\end{déclaration}
\begin{déclaration}\grammar{trop}\end{déclaration}\end{définition}
\begin{définition}\fra se créer des ennuis à soi-même\end{définition}
\begin{définition}\cmn 自作多情;自己给自己找麻烦\end{définition}
\begin{exemple}\jya to-znɤftɕɤftɕal-a\cmn 我自作多情\end{exemple}
\begin{exemple}\jya ɯʑo ɲɯ-znɤftɕɤftɕɤl\cmn 他自作多情\end{exemple}
\begin{exemple}\jya ɯʑo kɯ-znɤftɕɤftɕɤl ci ŋu\cmn 他是一个自作多情的人\end{exemple}\end{entrée}

\begin{entrée}
\vedette{\hypertarget{Ⓔznɤja}{\papi{ znɤja}}}\markboth{znɤja}{}\classe{vt}
\paradigme{\textit{dir :} \jya nɯ-}
\begin{définition}\fra chérir\end{définition}
\begin{définition}\cmn 珍惜;不舍得\end{définition}
\begin{exemple}\jya tɯ-ŋga kɤ-rɤmbi ɲɯ-znɤje-a\cmn 我舍不得把衣服给别人\end{exemple}
\begin{relation-sémantique}\confer{
\hyperlink{Ⓔnɤja}{\textit{ \papi{nɤja}}}
}\end{relation-sémantique}\end{entrée}

\begin{entrée}
\vedette{\hypertarget{Ⓔznɤjo}{\papi{ znɤjo}}}\markboth{znɤjo}{}
\begin{relation-sémantique}\confer{
\hyperlink{Ⓔnɤjo}{\textit{ \papi{nɤjo}}}
}\end{relation-sémantique}\end{entrée}

\begin{entrée}
\vedette{\hypertarget{Ⓔznɤjpɯjpe}{\papi{ znɤjpɯjpe}}}\markboth{znɤjpɯjpe}{} (\variante{znɤjpejpe}) 
\classe{vs}
\paradigme{\textit{dir :} \jya thɯ-}
\begin{définition}\ 
\begin{déclaration}\grammar{refl}\end{déclaration}
\begin{déclaration}\grammar{trop}\end{déclaration}\end{définition}
\begin{définition}\fra se considérer comme quelqu'un de bien, se trouver belle\end{définition}
\begin{définition}\cmn 自以为好;自以为漂亮\end{définition}
\begin{exemple}\jya ɯʑo kɯ-znɤjpɯjpe ci ŋu\cmn 他是一个自以为好的人\end{exemple}
\begin{exemple}\jya ɲɯ-znɤjpɯjpe\cmn 他自以为好\end{exemple}
\begin{exemple}\jya cho-znɤjpɯjpe\cmn 他现在觉得自己很好(以前没有这个习惯)\end{exemple}\end{entrée}

\begin{entrée}
\vedette{\hypertarget{Ⓔznɤkɤro}{\papi{ znɤkɤro}}}\markboth{znɤkɤro}{}
\begin{relation-sémantique}\confer{
\hyperlink{Ⓔnɤkɤro}{\textit{ \papi{nɤkɤro}}}
}\end{relation-sémantique}\end{entrée}

\begin{entrée}
\vedette{\hypertarget{Ⓔznɤkhɯ}{\papi{ znɤkhɯ}}}\markboth{znɤkhɯ}{}
\begin{relation-sémantique}\confer{
\hyperlink{Ⓔnɤkhɯ}{\textit{ \papi{nɤkhɯ}}}
}\end{relation-sémantique}\end{entrée}

\begin{entrée}
\vedette{\hypertarget{Ⓔznɤkɯt}{\papi{ znɤkɯt}}}\markboth{znɤkɯt}{}
\begin{relation-sémantique}\confer{
\hyperlink{Ⓔnɤkɯt}{\textit{ \papi{nɤkɯt}}}
}\end{relation-sémantique}\end{entrée}

\begin{entrée}
\vedette{\hypertarget{Ⓔznɤltɕɤm}{\papi{ znɤltɕɤm}}}\markboth{znɤltɕɤm}{}\classe{vt}
\paradigme{\textit{dir :} \jya pɯ-}
\begin{définition}\fra couvrir\end{définition}
\begin{définition}\cmn 用自己衣服的一角盖在别人身上,跟别人分享\end{définition}
\begin{exemple}\jya pɯ-znɤltɕam-a\cmn 我帮他盖了(衣服)\end{exemple}
\begin{exemple}\jya pɯ́-wɣ-znɤltɕam-a\cmn 他帮我盖了(衣服)\end{exemple}
\begin{exemple}\jya tɯ-ŋga ɯ-βzɯr nɯ ɯ-zda ɯ-taʁ zɯ pa-znɤltɕɤm\cmn 他顺便把铺盖的一角帮朋友盖在身上\end{exemple}\begin{sous-entrée}
\vedette{\hypertarget{}{\papi{ aznɤltɕɯltɕɤm}}}\markboth{aznɤltɕɯltɕɤm}{}\classe{vi}
\paradigme{\textit{dir :} \jya pɯ-}
\begin{définition}\ 
\begin{déclaration}\grammar{recip}\end{déclaration}\end{définition}
\begin{définition}\fra se couvrir les uns les autres\end{définition}
\begin{définition}\cmn 用衣服互相盖\end{définition}
\begin{exemple}\jya pɯ-aznɤltɕɯltɕɤm-tɕi\cmn 我们互相盖了\end{exemple}
\end{sous-entrée}\end{entrée}

\begin{entrée}
\vedette{\hypertarget{Ⓔznɤlɯli}{\papi{ znɤlɯli}}}\markboth{znɤlɯli}{}\classe{vi}
\paradigme{\textit{dir :} \jya tɤ-}
\begin{définition}\fra faire des caprices\end{définition}
\begin{définition}\cmn 撒娇\end{définition}
\begin{exemple}\jya ɯ-mu ɯ-phe ɲɯ-znɤlɯli\cmn 他向他母亲撒娇\end{exemple}
\begin{exemple}\jya a-mu ɯ-ɕki tɤ-znɤlɯli-a\cmn 我在母亲面前撒娇了\end{exemple}\end{entrée}

\begin{entrée}
\vedette{\hypertarget{Ⓔznɤmaʁmaʁ}{\papi{ znɤmaʁmaʁ}}}\markboth{znɤmaʁmaʁ}{}
\classe{vt}
\paradigme{\textit{dir :} \jya tɤ-}
\begin{définition}\fra cacher la vérité\end{définition}
\begin{définition}\cmn 掩盖真相;掩盖自己的行为\end{définition}
\begin{exemple}\jya to-mɯrkɯ ri to-znɤmaʁmaʁ\cmn 他偷了东西但是不承认\end{exemple}
\begin{exemple}\jya pɯ-kɯ-fse nɯ tɤ-ti wo ma ma-tɤ-tɯ-znɤmaʁmaʁ\cmn 你要把发生的事情说清楚,不要掩盖真相\end{exemple}
\begin{relation-sémantique}\confer{
\hyperlink{ⒺmaʁⒽ1}{\textit{ \papi{maʁ1}}}
}\end{relation-sémantique}\end{entrée}

\begin{entrée}
\vedette{\hypertarget{Ⓔznɤmbju}{\papi{ znɤmbju}}}\markboth{znɤmbju}{}
\begin{relation-sémantique}\confer{
\hyperlink{Ⓔnɤmbju}{\textit{ \papi{nɤmbju}}}
}\end{relation-sémantique}\end{entrée}

\begin{entrée}
\vedette{\hypertarget{Ⓔznɤmɲole}{\papi{ znɤmɲole}}}\markboth{znɤmɲole}{}
\begin{relation-sémantique}\confer{
\hyperlink{Ⓔnɤmɲole}{\textit{ \papi{nɤmɲole}}}
}\end{relation-sémantique}\end{entrée}

\begin{entrée}
\vedette{\hypertarget{Ⓔznɤmpɕɤmpɕɤr}{\papi{ znɤmpɕɤmpɕɤr}}}\markboth{znɤmpɕɤmpɕɤr}{}
\classe{vi}
\paradigme{\textit{dir :} \jya thɯ-}
\begin{définition}\ 
\begin{déclaration}\grammar{refl}\end{déclaration}
\begin{déclaration}\grammar{trop}\end{déclaration}\end{définition}
\begin{définition}\fra se croire belle\end{définition}
\begin{définition}\cmn 以为自己很漂亮\end{définition}
\begin{exemple}\jya chɤ-tɯ-znɤmpɕɤmpɕɤr\cmn 你以为自己很漂亮了\end{exemple}
\begin{relation-sémantique}\confer{
\hyperlink{Ⓔmpɕɤr}{\textit{ \papi{mpɕɤr}}}
}\end{relation-sémantique}
\begin{relation-sémantique}\confer{
\hyperlink{Ⓔrɤmpɕɤr}{\textit{ \papi{rɤmpɕɤr}}}
}\end{relation-sémantique}\end{entrée}

\begin{entrée}
\vedette{\hypertarget{Ⓔznɤmqrɯz}{\papi{ znɤmqrɯz}}}\markboth{znɤmqrɯz}{}
\begin{relation-sémantique}\confer{
\hyperlink{Ⓔɣɤmqrɯz}{\textit{ \papi{ɣɤmqrɯz}}}
}\end{relation-sémantique}\end{entrée}

\begin{entrée}
\vedette{\hypertarget{Ⓔznɤmɯma}{\papi{ znɤmɯma}}}\markboth{znɤmɯma}{}
\classe{vt}
\paradigme{\textit{dir :} \jya tɤ-}
\begin{définition}\fra faire toutes sortes de choses\end{définition}
\begin{définition}\cmn 做各种事情
\begin{déclaration}\use{只用于无人称完成体形式}\end{déclaration}\end{définition}
\begin{exemple}\jya tɕhi tɤ́-wɣ-znɤmɯma ʑo tu-kɯ-rɯndzaŋspa ra\cmn 不管做什么,一定要小心\end{exemple}
\begin{exemple}\jya tɕhi tɤ́-wɣ-znɤmɯma ʑo mɯ́j-cha\cmn 不管让他做什么都不行\end{exemple}\end{entrée}

\begin{entrée}
\vedette{\hypertarget{Ⓔznɤndɤɣ}{\papi{ znɤndɤɣ}}}\markboth{znɤndɤɣ}{}
\begin{relation-sémantique}\confer{
\hyperlink{Ⓔnɤndɤɣ}{\textit{ \papi{nɤndɤɣ}}}
}\end{relation-sémantique}\end{entrée}

\begin{entrée}
\vedette{\hypertarget{Ⓔznɤndɤɣri}{\papi{ znɤndɤɣri}}}\markboth{znɤndɤɣri}{}
\begin{relation-sémantique}\confer{
\hyperlink{Ⓔnɤndɤɣri}{\textit{ \papi{nɤndɤɣri}}}
}\end{relation-sémantique}\end{entrée}

\begin{entrée}
\vedette{\hypertarget{Ⓔznɤndɤr}{\papi{ znɤndɤr}}}\markboth{znɤndɤr}{}
\begin{relation-sémantique}\confer{
\hyperlink{Ⓔnɤndɤr}{\textit{ \papi{nɤndɤr}}}
}\end{relation-sémantique}\end{entrée}

\begin{entrée}
\vedette{\hypertarget{Ⓔznɤndɯndɤt}{\papi{ znɤndɯndɤt}}}\markboth{znɤndɯndɤt}{}
\begin{relation-sémantique}\confer{
\hyperlink{Ⓔnɤndɯndɤt}{\textit{ \papi{nɤndɯndɤt}}}
}\end{relation-sémantique}
\end{entrée}

\begin{entrée}
\vedette{\hypertarget{Ⓔznɤngɯt}{\papi{ znɤngɯt}}}\markboth{znɤngɯt}{}
\begin{relation-sémantique}\confer{
\hyperlink{Ⓔnɤngɯt}{\textit{ \papi{nɤngɯt}}}
}\end{relation-sémantique}\end{entrée}

\begin{entrée}
\vedette{\hypertarget{Ⓔznɤŋɤβ}{\papi{ znɤŋɤβ}}}\markboth{znɤŋɤβ}{}
\begin{relation-sémantique}\confer{
\hyperlink{Ⓔsɤŋɤβ}{\textit{ \papi{sɤŋɤβ}}}
}\end{relation-sémantique}\end{entrée}

\begin{entrée}
\vedette{\hypertarget{Ⓔznɤŋgɤr}{\papi{ znɤŋgɤr}}}\markboth{znɤŋgɤr}{}
\paradigme{\textit{dir :} \jya \_}
\begin{définition}\fra pousser vers un côté (par la foule)\end{définition}
\begin{définition}\cmn 挤过去(因为人多,很拥挤)\end{définition}
\begin{exemple}\jya ɲɯ-kɯ-znɤŋgar-a\cmn 你把我挤到那边去\end{exemple}
\begin{relation-sémantique}\confer{
\hyperlink{Ⓔŋgɤr}{\textit{ \papi{ŋgɤr}}}
}\end{relation-sémantique}
\end{entrée}

\begin{entrée}
\vedette{\hypertarget{Ⓔznɤŋgɯ}{\papi{ znɤŋgɯ}}}\markboth{znɤŋgɯ}{}
\begin{relation-sémantique}\confer{
\hyperlink{Ⓔnɤŋgɯ}{\textit{ \papi{nɤŋgɯ}}}
}\end{relation-sémantique}\end{entrée}

\begin{entrée}
\vedette{\hypertarget{Ⓔznɤŋɯŋu}{\papi{ znɤŋɯŋu}}}\markboth{znɤŋɯŋu}{}\classe{vs}
\begin{définition}\ 
\begin{déclaration}\grammar{refl}\end{déclaration}
\begin{déclaration}\grammar{trop}\end{déclaration}\end{définition}
\begin{définition}\fra être présomptueux\end{définition}
\begin{définition}\cmn 自以为是\end{définition}
\begin{relation-sémantique}\confer{
\hyperlink{Ⓔŋu}{\textit{ \papi{ŋu}}}
}\end{relation-sémantique}\end{entrée}

\begin{entrée}
\vedette{\hypertarget{Ⓔznɤpɤri}{\papi{ znɤpɤri}}}\markboth{znɤpɤri}{}
\begin{relation-sémantique}\confer{
\hyperlink{Ⓔnɤpɤri}{\textit{ \papi{nɤpɤri}}}
}\end{relation-sémantique}\end{entrée}

\begin{entrée}
\vedette{\hypertarget{Ⓔznɤphɤtphɤt}{\papi{ znɤphɤtphɤt}}}\markboth{znɤphɤtphɤt}{}
\begin{relation-sémantique}\confer{
\hyperlink{Ⓔnɤphɤtphɤt}{\textit{ \papi{nɤphɤtphɤt}}}
}\end{relation-sémantique}\end{entrée}

\begin{entrée}
\vedette{\hypertarget{Ⓔznɤrɕu}{\papi{ znɤrɕu}}}\markboth{znɤrɕu}{}
\begin{relation-sémantique}\confer{
\hyperlink{Ⓔnɤrɕu}{\textit{ \papi{nɤrɕu}}}
}\end{relation-sémantique}\end{entrée}

\begin{entrée}
\vedette{\hypertarget{Ⓔznɤre}{\papi{ znɤre}}}\markboth{znɤre}{}
\begin{relation-sémantique}\confer{
 \papi{nɤre}
}\end{relation-sémantique}\end{entrée}

\begin{entrée}
\vedette{\hypertarget{Ⓔznɤrko}{\papi{ znɤrko}}}\markboth{znɤrko}{}
\begin{relation-sémantique}\confer{
\hyperlink{ⒺnɤrkoⒽ1}{\textit{ \papi{nɤrko}}}
}\end{relation-sémantique}
\end{entrée}

\begin{entrée}
\vedette{\hypertarget{Ⓔznɤʁamɟa}{\papi{ znɤʁamɟa}}}\markboth{znɤʁamɟa}{}
\classe{vs}
\paradigme{\textit{dir :} \jya tɤ-}
\begin{définition}\fra zêlé\end{définition}
\begin{définition}\cmn 勤快,抓紧时间(舍不得耽误时间)\end{définition}
\begin{exemple}\jya ta-ma ɲɯ-znɤʁamɟa\cmn 他工作很勤快\end{exemple}
\begin{exemple}\jya pɯ-znɤʁamɟa-tɕi\cmn 我们俩很勤快\end{exemple}
\begin{exemple}\jya jiɕqha nɯ kɯ-znɤʁamɟa ci ɲɯ-ŋu\cmn 他是个勤快的人\end{exemple}
\begin{exemple}\jya nɤʑo kɤ-rɤβzjoz ɲɯ-tɯ-znɤʁamɟa\cmn 你学习很勤快\end{exemple}
\begin{relation-sémantique}\confer{
\hyperlink{Ⓔtɤ-ʁamɟa}{\textit{ \papi{tɤ-ʁamɟa}}}
}\end{relation-sémantique}\end{entrée}

\begin{entrée}
\vedette{\hypertarget{Ⓔznɤʁdɤn}{\papi{ znɤʁdɤn}}}\markboth{znɤʁdɤn}{} (\variante{znɯʁdɤn}) 
\begin{relation-sémantique}\confer{
\hyperlink{Ⓔnɤʁdɤn}{\textit{ \papi{nɤʁdɤn}}}
}\end{relation-sémantique}\end{entrée}

\begin{entrée}
\vedette{\hypertarget{Ⓔznɤscɤr}{\papi{ znɤscɤr}}}\markboth{znɤscɤr}{}
\begin{relation-sémantique}\confer{
\hyperlink{Ⓔnɤscɤr}{\textit{ \papi{nɤscɤr}}}
}\end{relation-sémantique}\end{entrée}

\begin{entrée}
\vedette{\hypertarget{Ⓔznɤtʂa}{\papi{ znɤtʂa}}}\markboth{znɤtʂa}{}
\begin{relation-sémantique}\confer{
\hyperlink{Ⓔnɤtʂa}{\textit{ \papi{nɤtʂa}}}
}\end{relation-sémantique}\end{entrée}

\begin{entrée}
\vedette{\hypertarget{Ⓔznɤtʂhɯtʂhɯt}{\papi{ znɤtʂhɯtʂhɯt}}}\markboth{znɤtʂhɯtʂhɯt}{}
\begin{relation-sémantique}\confer{
\hyperlink{ⒺtʂhɤtⒽ1}{\textit{ \papi{tʂhɤt1}}}
}\end{relation-sémantique}\end{entrée}

\begin{entrée}
\vedette{\hypertarget{Ⓔznɤtʂɯntʂɯn}{\papi{ znɤtʂɯntʂɯn}}}\markboth{znɤtʂɯntʂɯn}{}
\classe{vs}
\begin{définition}\fra qui aime se vanter de ses bonnes actions\end{définition}
\begin{définition}\cmn 炫耀自己的功劳
\begin{déclaration} \étymologie{\papi{drin}}\end{déclaration}\end{définition}
\begin{exemple}\jya ɲɯ-sɯxtʂɯn ri, ɲɯ-znɤtʂɯntʂɯn\cmn 他虽然对别人好,但是会炫耀自己\end{exemple}
\begin{relation-sémantique}\confer{
\hyperlink{Ⓔtɯ-tʂɯn}{\textit{ \papi{tɯ-tʂɯn}}}
}\end{relation-sémantique}\end{entrée}

\begin{entrée}
\vedette{\hypertarget{Ⓔznɤtɯɣ}{\papi{ znɤtɯɣ}}}\markboth{znɤtɯɣ}{}
\begin{relation-sémantique}\confer{
\hyperlink{Ⓔatɯɣ}{\textit{ \papi{atɯɣ}}}
}\end{relation-sémantique}\end{entrée}

\begin{entrée}
\vedette{\hypertarget{Ⓔznɤχɤmthi}{\papi{ znɤχɤmthi}}}\markboth{znɤχɤmthi}{}
\begin{relation-sémantique}\confer{
\hyperlink{Ⓔnɤχɤmthi}{\textit{ \papi{nɤχɤmthi}}}
}\end{relation-sémantique}
\end{entrée}

\begin{entrée}
\vedette{\hypertarget{Ⓔznɤχpɯχpa}{\papi{ znɤχpɯχpa}}}\markboth{znɤχpɯχpa}{}
\classe{vs}
\paradigme{\textit{dir :} \jya thɯ-}
\begin{définition}\fra arrogant\end{définition}
\begin{définition}\cmn 傲慢\end{définition}
\begin{exemple}\jya jiɕqha nɯ ɲɯ-znɤχpɯχpa\cmn 那个人很傲慢\end{exemple}
\begin{relation-sémantique}\confer{
\hyperlink{Ⓔχpa}{\textit{ \papi{χpa}}}
}\end{relation-sémantique}\end{entrée}

\begin{entrée}
\vedette{\hypertarget{Ⓔznɤzraʁ}{\papi{ znɤzraʁ}}}\markboth{znɤzraʁ}{}
\begin{relation-sémantique}\confer{
\hyperlink{Ⓔnɤzraʁ}{\textit{ \papi{nɤzraʁ}}}
}\end{relation-sémantique}\end{entrée}

\begin{entrée}
\vedette{\hypertarget{Ⓔzndɤkɤlwa}{\papi{ zndɤkɤlwa}}}\markboth{zndɤkɤlwa}{}\classe{n}
\begin{définition}\fra pierre plate\end{définition}
\begin{définition}\cmn 防雨水的石板\end{définition}
\begin{relation-sémantique}\confer{
\hyperlink{ⒺzndeⒽ1}{\textit{ \papi{znde1}}}
}\end{relation-sémantique}\end{entrée}

\begin{entrée}
\vedette{\hypertarget{Ⓔzndɤqa}{\papi{ zndɤqa}}}\markboth{zndɤqa}{}\classe{n}
\begin{définition}\fra bas du mur\end{définition}
\begin{définition}\cmn 墙脚\end{définition}
\end{entrée}

\begin{entrée}
\vedette{\hypertarget{Ⓔzndɤrchɤβ}{\papi{ zndɤrchɤβ}}}\markboth{zndɤrchɤβ}{}\classe{n}
\begin{définition}\fra fissure sur le mur\end{définition}
\begin{définition}\cmn 墙上的缝隙\end{définition}\end{entrée}

\begin{entrée}
\vedette{\hypertarget{Ⓔzndɤtɕhaʁ}{\papi{ zndɤtɕhaʁ}}}\markboth{zndɤtɕhaʁ}{}
\classe{n}
\begin{définition}\fra se rétrécir (mur)\end{définition}
\begin{définition}\cmn 收缩,变形(墙因受潮等原因)\end{définition}
\begin{exemple}\jya zndɤtɕhaʁ pjɤ-ɕe\cmn 墙(因为受了潮)收缩变形了。\end{exemple}
\begin{relation-sémantique}\confer{
\hyperlink{ⒺzndeⒽ1}{\textit{ \papi{znde1}}}
}\end{relation-sémantique}
\begin{relation-sémantique}\confer{
\hyperlink{Ⓔtɕhaʁ}{\textit{ \papi{tɕhaʁ}}}
}\end{relation-sémantique}\end{entrée}

\begin{entrée}
\vedette{\hypertarget{ⒺzndeⒽ1}{\papi{ znde}}}\markboth{znde}{}\homonyme{1}\classe{n}
\begin{définition}\fra mur en pierre\end{définition}
\begin{définition}\cmn 石墙\end{définition}
\begin{exemple}\jya znde tɤ-βzu-t-a\cmn 我修了墙\end{exemple}
\begin{exemple}\jya qajɯ znde ɯ-taʁ tu-xcat-nɯ ʑo ɲɯ-ŋu\cmn 墙上有很多虫子\end{exemple}
\begin{relation-sémantique}\confer{
\hyperlink{ⒺzndeⒽ2}{\textit{ \papi{znde2}}}
}\end{relation-sémantique}
\begin{relation-sémantique}\confer{
\hyperlink{Ⓔrɤznde}{\textit{ \papi{rɤznde}}}
}\end{relation-sémantique}
\begin{relation-sémantique}\confer{
\hyperlink{Ⓔzndɤtɕhaʁ}{\textit{ \papi{zndɤtɕhaʁ}}}
}\end{relation-sémantique}
\begin{relation-sémantique}\confer{
\hyperlink{Ⓔzndɤrchɤβ}{\textit{ \papi{zndɤrchɤβ}}}
}\end{relation-sémantique}\end{entrée}

\begin{entrée}
\vedette{\hypertarget{ⒺzndeⒽ2}{\papi{ znde}}}\markboth{znde}{}\homonyme{2}\classe{vt}
\paradigme{\textit{dir :} \jya tɤ-}
\begin{définition}\fra réparer un mur à un endroit, empiler des briques là où le mur s'est abîmé\end{définition}
\begin{définition}\cmn 堵住石墙的缺口\end{définition}
\begin{exemple}\jya tɤ-znde-t-a\cmn 我垒起来了\end{exemple}
\begin{exemple}\jya ki ɯ-stu ki ɲɯ-ɤχa tɕe tɤ-znde-t-a\cmn 这个地方有个缺口,我就堵上了\end{exemple}
\begin{relation-sémantique}\confer{
\hyperlink{ⒺzndeⒽ1}{\textit{ \papi{znde1}}}
}\end{relation-sémantique}
\begin{relation-sémantique}\confer{
\hyperlink{Ⓔrɤznde}{\textit{ \papi{rɤznde}}}
}\end{relation-sémantique}\end{entrée}

\begin{entrée}
\vedette{\hypertarget{Ⓔznɯβdaʁ}{\papi{ znɯβdaʁ}}}\markboth{znɯβdaʁ}{}
\begin{relation-sémantique}\confer{
\hyperlink{Ⓔnɯβdaʁ}{\textit{ \papi{nɯβdaʁ}}}
}\end{relation-sémantique}\end{entrée}

\begin{entrée}
\vedette{\hypertarget{Ⓔznɯcaχto}{\papi{ znɯcaχto}}}\markboth{znɯcaχto}{}
\begin{relation-sémantique}\confer{
\hyperlink{Ⓔnɯcaχto}{\textit{ \papi{nɯcaχto}}}
}\end{relation-sémantique}\end{entrée}

\begin{entrée}
\vedette{\hypertarget{Ⓔznɯɕkhɤɣ}{\papi{ znɯɕkhɤɣ}}}\markboth{znɯɕkhɤɣ}{}\classe{vt}
\paradigme{\textit{dir :} \jya nɯ-}
\begin{définition}\fra ne pas se préoccuper de\end{définition}
\begin{définition}\cmn 不在乎,不理\end{définition}
\begin{exemple}\jya nɯ kɯ a-phe nɯ ɲɯ-ti ri, ɲɯ-znɯɕkhaɣ-a ɕti\cmn 他对我说这些话,但是我不在乎\end{exemple}
\begin{exemple}\jya tɯ-mɯ ɲɯ-ɤsɯ-lɤt ri aʑo ɲɯ-znɯɕqhaɣ-a ɕti\cmn 虽然下雨,但是我不在乎\end{exemple}\end{entrée}

\begin{entrée}
\vedette{\hypertarget{Ⓔznɯɕqhɯɕqhu}{\papi{ znɯɕqhɯɕqhu}}}\markboth{znɯɕqhɯɕqhu}{}
\classe{vt}
\paradigme{\textit{dir :} \jya pɯ-}\acception{1}
\begin{définition}\fra s'opposer\end{définition}
\begin{définition}\cmn 违反;反对\end{définition}
\begin{exemple}\jya pɯ-znɯɕqhɯɕqhu-t-a\cmn 我反对了\end{exemple}
\begin{exemple}\jya ɯ-stu mɯ́j-nɤme tɕe pjɯ-znɯɕqhɯɕqhe ntsɯ ɲɯ-ɕti\cmn 他不是诚心想做,总是搞破坏\end{exemple}
\begin{exemple}\jya tɤ-tɯt-a nɯ ma-pɯ-tɯ-znɯɕqhɯɕqhe\cmn 你不要反对我所说的\end{exemple}\acception{2}
\begin{définition}\fra revenir sur (sa parole)\end{définition}
\begin{définition}\cmn 违背;反悔\end{définition}
\begin{exemple}\jya tɤ-tɯ-nɯ-tɯt nɯ ma-pɯ-tɯ-znɯɕqhɯɕqhe\cmn 你不要反悔\end{exemple}
\begin{relation-sémantique}\confer{
\hyperlink{Ⓔnɯɕqhu}{\textit{ \papi{nɯɕqhu}}}
}\end{relation-sémantique}\end{entrée}

\begin{entrée}
\vedette{\hypertarget{Ⓔznɯɕtar}{\papi{ znɯɕtar}}}\markboth{znɯɕtar}{}
\begin{relation-sémantique}\confer{
\hyperlink{Ⓔnɯɕtar}{\textit{ \papi{nɯɕtar}}}
}\end{relation-sémantique}\end{entrée}

\begin{entrée}
\vedette{\hypertarget{Ⓔznɯfsoʁspɤt}{\papi{ znɯfsoʁspɤt}}}\markboth{znɯfsoʁspɤt}{}\classe{vt}
\paradigme{\textit{dir :} \jya lɤ-}
\begin{définition}\fra (faire) toute la nuit jusqu'au lever du jour\end{définition}
\begin{définition}\cmn 从晚上开始一直……到天亮\end{définition}
\begin{exemple}\jya kɤ-rɤrɤt lɤ-znɯfsoʁspat-a ʑo pɯ-ra\cmn 我只好一直写到天亮\end{exemple}
\begin{exemple}\jya kɤ-rɤma lɤ-znɯfsoʁspat-i ʑo\cmn 我们晚上开始工作,一直工作到天亮\end{exemple}
\begin{relation-sémantique}\confer{
\hyperlink{ⒺfsoʁⒽ1}{\textit{ \papi{fsoʁ}}}
}\end{relation-sémantique}
\end{entrée}

\begin{entrée}
\vedette{\hypertarget{Ⓔznɯɣɟɯ}{\papi{ znɯɣɟɯ}}}\markboth{znɯɣɟɯ}{}
\begin{relation-sémantique}\confer{
\hyperlink{Ⓔnɯɣɟɯ}{\textit{ \papi{nɯɣɟɯ}}}
}\end{relation-sémantique}\end{entrée}

\begin{entrée}
\vedette{\hypertarget{Ⓔznɯɣmaz}{\papi{ znɯɣmaz}}}\markboth{znɯɣmaz}{}
\begin{relation-sémantique}\confer{
\hyperlink{Ⓔnɯɣmaz}{\textit{ \papi{nɯɣmaz}}}
}\end{relation-sémantique}\end{entrée}

\begin{entrée}
\vedette{\hypertarget{Ⓔznɯjɯn}{\papi{ znɯjɯn}}}\markboth{znɯjɯn}{}
\classe{vt}
\paradigme{\textit{dir :} \jya tɤ-}
\begin{définition}\fra être conforme à, suivre, aller le long\end{définition}
\begin{définition}\cmn 顺着;依着;沿着\end{définition}
\begin{exemple}\jya ta-znɯjɯn\cmn 他跟着他去了\end{exemple}
\begin{exemple}\jya tɤ́-wɣ-znɯjɯn-a\cmn 他跟着我去了\end{exemple}
\begin{exemple}\jya tʂu tɤ-znɯjɯn-a\cmn 我沿着路去了\end{exemple}
\begin{relation-sémantique}\synonyme{
\hyperlink{Ⓔnɯɴqhu}{\textit{ \papi{nɯɴqhu}}}
}\end{relation-sémantique}\end{entrée}

\begin{entrée}
\vedette{\hypertarget{Ⓔznɯkhrɯm}{\papi{ znɯkhrɯm}}}\markboth{znɯkhrɯm}{}
\begin{relation-sémantique}\confer{
\hyperlink{Ⓔnɯkhrɯm}{\textit{ \papi{nɯkhrɯm}}}
}\end{relation-sémantique}
\end{entrée}

\begin{entrée}
\vedette{\hypertarget{Ⓔznɯkro}{\papi{ znɯkro}}}\markboth{znɯkro}{}\classe{vt}
\paradigme{\textit{dir :} \jya tɤ-}
\begin{définition}\fra donner une part à\end{définition}
\begin{définition}\cmn 分东西给别人\end{définition}
\begin{exemple}\jya aʑo paχɕi a-zda pɯ-znɯkro-t-a\cmn 我把苹果分给我的朋友了\end{exemple}
\begin{exemple}\jya aʑo paχɕi pjɯ-ta-znɯkro\cmn 我把苹果分给你\end{exemple}
\begin{exemple}\jya nɤ-paχɕi nɤ-zda pɯ-znɯkrɤm\cmn 你把苹果分给你的朋友\end{exemple}
\begin{relation-sémantique}\confer{
\hyperlink{Ⓔkro}{\textit{ \papi{kro}}}
}\end{relation-sémantique}\end{entrée}

\begin{entrée}
\vedette{\hypertarget{Ⓔznɯkrɯβ}{\papi{ znɯkrɯβ}}}\markboth{znɯkrɯβ}{}
\begin{relation-sémantique}\confer{
\hyperlink{Ⓔnɯkrɯβ}{\textit{ \papi{nɯkrɯβ}}}
}\end{relation-sémantique}\end{entrée}

\begin{entrée}
\vedette{\hypertarget{Ⓔznɯkɯlu}{\papi{ znɯkɯlu}}}\markboth{znɯkɯlu}{}
\begin{relation-sémantique}\confer{
\hyperlink{Ⓔnɯkɯlu}{\textit{ \papi{nɯkɯlu}}}
}\end{relation-sémantique}
\end{entrée}

\begin{entrée}
\vedette{\hypertarget{Ⓔznɯmgɯrjɯm}{\papi{ znɯmgɯrjɯm}}}\markboth{znɯmgɯrjɯm}{} (\variante{znɯmgɯrjɯβ}) 
\classe{vt}
\paradigme{\textit{dir :} \jya kɤ-}
\paradigme{\textit{dir :} \jya thɯ-}
\begin{définition}\fra chauffer au feu\end{définition}
\begin{définition}\cmn 烘干;烤(在火塘边)\end{définition}
\begin{exemple}\jya smi ɯ-phe kɤ-znɯmgɯrjɯm-a\cmn 我烤火了\end{exemple}
\begin{exemple}\jya @yangyu kɤ-kɤ-sqa kɤ-znɯmgɯrjɯm-a\cmn 把煮过的洋芋烤了一下\end{exemple}
\begin{exemple}\jya qajɣi nɯ chɯ́-wɣ-znɯmgɯrjɯβ tɕe, nɯ kɯ-fse thɯ-kɯ-smi nɯ tú-wɣ-ndza tɕe mɯm\cmn 把馍馍在火塘边烤熟很好吃\end{exemple}\end{entrée}

\begin{entrée}
\vedette{\hypertarget{Ⓔznɯmkɤqloʁ}{\papi{ znɯmkɤqloʁ}}}\markboth{znɯmkɤqloʁ}{}
\begin{relation-sémantique}\confer{
\hyperlink{Ⓔnɯmkɤqloʁ}{\textit{ \papi{nɯmkɤqloʁ}}}
}\end{relation-sémantique}\end{entrée}

\begin{entrée}
\vedette{\hypertarget{Ⓔznɯmnɤl}{\papi{ znɯmnɤl}}}\markboth{znɯmnɤl}{}
\begin{relation-sémantique}\confer{
\hyperlink{Ⓔnɯmnɤl}{\textit{ \papi{nɯmnɤl}}}
}\end{relation-sémantique}\end{entrée}

\begin{entrée}
\vedette{\hypertarget{Ⓔznɯna}{\papi{ znɯna}}}\markboth{znɯna}{}\classe{vt}
\paradigme{\textit{dir :} \jya \_}
\begin{définition}\fra arrêter\end{définition}
\begin{définition}\cmn 停止\end{définition}
\begin{exemple}\jya aʑo thamaka kɤ-sko tɤ-znɯna-t-a ma mɯ́j-pe\cmn 我停止抽烟了\end{exemple}
\begin{exemple}\jya ta-ma tɤ-znɯna-t-a\cmn 我停止工作了\end{exemple}
\begin{relation-sémantique}\confer{
\hyperlink{Ⓔnɯna}{\textit{ \papi{nɯna}}}
}\end{relation-sémantique}\end{entrée}

\begin{entrée}
\vedette{\hypertarget{Ⓔznɯndʐɯnbu}{\papi{ znɯndʐɯnbu}}}\markboth{znɯndʐɯnbu}{}
\begin{relation-sémantique}\confer{
\hyperlink{Ⓔnɯndʐɯnbu}{\textit{ \papi{nɯndʐɯnbu}}}
}\end{relation-sémantique}
\end{entrée}

\begin{entrée}
\vedette{\hypertarget{Ⓔznɯndzɯ}{\papi{ znɯndzɯ}}}\markboth{znɯndzɯ}{}
\begin{relation-sémantique}\confer{
\hyperlink{Ⓔnɯndzɯ}{\textit{ \papi{nɯndzɯ}}}
}\end{relation-sémantique}\end{entrée}

\begin{entrée}
\vedette{\hypertarget{Ⓔznɯnoʁ}{\papi{ znɯnoʁ}}}\markboth{znɯnoʁ}{}\classe{vt}
\paradigme{\textit{dir :} \jya pɯ-}
\begin{définition}\fra mettre de la nourriture dans la sauce\end{définition}
\begin{définition}\cmn 蘸\end{définition}
\begin{exemple}\jya ɕkɤfkri ɯ-ŋgɯ pɯ-znɯnoʁ-a\cmn 我蘸了大蒜沾水\end{exemple}
\begin{exemple}\jya tsha pɯ-znɯnoʁ-a\cmn 我蘸了盐巴\end{exemple}
\begin{exemple}\jya tɯnoʁ tɤ-βzu-t-a tɕe, tɤ-mthɯm pɯ-znɯnoʁ\cmn 我做了汁,你就蘸肉吧\end{exemple}
\begin{relation-sémantique}\confer{
\hyperlink{Ⓔtɯnoʁ}{\textit{ \papi{tɯnoʁ}}}
}\end{relation-sémantique}\end{entrée}

\begin{entrée}
\vedette{\hypertarget{Ⓔznɯntsho}{\papi{ znɯntsho}}}\markboth{znɯntsho}{}
\begin{relation-sémantique}\confer{
\hyperlink{Ⓔnɯntsho}{\textit{ \papi{nɯntsho}}}
}\end{relation-sémantique}\end{entrée}

\begin{entrée}
\vedette{\hypertarget{Ⓔznɯntsɯɣ}{\papi{ znɯntsɯɣ}}}\markboth{znɯntsɯɣ}{}
\begin{relation-sémantique}\confer{
\hyperlink{Ⓔnɯntsɯɣ}{\textit{ \papi{nɯntsɯɣ}}}
}\end{relation-sémantique}
\end{entrée}

\begin{entrée}
\vedette{\hypertarget{Ⓔznɯɲco}{\papi{ znɯɲco}}}\markboth{znɯɲco}{}
\begin{relation-sémantique}\confer{
\hyperlink{Ⓔnɯco}{\textit{ \papi{nɯco}}}
}\end{relation-sémantique}\end{entrée}

\begin{entrée}
\vedette{\hypertarget{Ⓔznɯŋgu}{\papi{ znɯŋgu}}}\markboth{znɯŋgu}{}
\begin{relation-sémantique}\confer{
\hyperlink{Ⓔnɯŋgu}{\textit{ \papi{nɯŋgu}}}
}\end{relation-sémantique}\end{entrée}

\begin{entrée}
\vedette{\hypertarget{Ⓔznɯŋgɤrkɯ}{\papi{ znɯŋgɤrkɯ}}}\markboth{znɯŋgɤrkɯ}{}\classe{vt}
\paradigme{\textit{dir :} \jya kɤ-}
\begin{définition}\fra envelopper les bébés dans des habits et les placer verticalement\end{définition}
\begin{définition}\cmn 用衣服把婴儿竖着包起来\end{définition}\end{entrée}

\begin{entrée}
\vedette{\hypertarget{Ⓔznɯŋgra}{\papi{ znɯŋgra}}}\markboth{znɯŋgra}{}
\begin{relation-sémantique}\confer{
\hyperlink{Ⓔnɯŋgra}{\textit{ \papi{nɯŋgra}}}
}\end{relation-sémantique}
\end{entrée}

\begin{entrée}
\vedette{\hypertarget{Ⓔznɯɴɢɤt}{\papi{ znɯɴɢɤt}}}\markboth{znɯɴɢɤt}{}\classe{vt}
\paradigme{\textit{dir :} \jya nɯ-}
\begin{définition}\fra séparer\end{définition}
\begin{définition}\cmn 分开\end{définition}
\begin{exemple}\jya tɤ-pɤtso ni nɯ-znɯɴɢɤt-i ma ɲɯ-rɯŋɯŋɤn-ndʑi\cmn 我们让那两个孩子分开,因为他们俩准备干坏事\end{exemple}
\begin{relation-sémantique}\confer{
\hyperlink{Ⓔɴɢɤt}{\textit{ \papi{ɴɢɤt}}}
}\end{relation-sémantique}\end{entrée}

\begin{entrée}
\vedette{\hypertarget{Ⓔznɯɴqhu}{\papi{ znɯɴqhu}}}\markboth{znɯɴqhu}{}
\begin{relation-sémantique}\confer{
\hyperlink{Ⓔnɯɴqhu}{\textit{ \papi{nɯɴqhu}}}
}\end{relation-sémantique}\end{entrée}

\begin{entrée}
\vedette{\hypertarget{Ⓔznɯpoʁ}{\papi{ znɯpoʁ}}}\markboth{znɯpoʁ}{}
\begin{relation-sémantique}\confer{
\hyperlink{Ⓔnɯpoʁ}{\textit{ \papi{nɯpoʁ}}}
}\end{relation-sémantique}
\end{entrée}

\begin{entrée}
\vedette{\hypertarget{Ⓔznɯqatɯkɯr}{\papi{ znɯqatɯkɯr}}}\markboth{znɯqatɯkɯr}{}
\classe{vt}
\paradigme{\textit{dir :} \jya nɯ-}
\begin{définition}\fra donner de mauvais conseils\end{définition}
\begin{définition}\cmn 进行反面教育
\begin{déclaration}\use{相应的不及物动词*\stylefv{nɯqatɯkɯr}不存在}\end{déclaration}\end{définition}
\begin{exemple}\jya na-znɯqatɯkɯr\cmn 他对他进行了反面教育\end{exemple}
\begin{exemple}\jya nɯ́-wɣ-znɯqatɯkɯr-a\cmn 他对我进行了反面教育\end{exemple}
\begin{exemple}\jya ɲɤ-znɯqatɯkɯr\cmn 他对他进行了反面教育\end{exemple}\end{entrée}

\begin{entrée}
\vedette{\hypertarget{Ⓔznɯqhɤstɯstu}{\papi{ znɯqhɤstɯstu}}}\markboth{znɯqhɤstɯstu}{}
\begin{relation-sémantique}\confer{
\hyperlink{Ⓔnɯqhɤstɯstu}{\textit{ \papi{nɯqhɤstɯstu}}}
}\end{relation-sémantique}
\end{entrée}

\begin{entrée}
\vedette{\hypertarget{Ⓔznɯrɤʁaŋ}{\papi{ znɯrɤʁaŋ}}}\markboth{znɯrɤʁaŋ}{}\classe{vt}
\begin{définition}\fra qui a le droit de\end{définition}
\begin{définition}\cmn 有权利……;有资格……\end{définition}
\begin{exemple}\jya nɤʑo aʑo kɤ-sɯxɕɤt ra mɤ-tɯ-znɯrɤʁaŋ (=nɤ-rɤʁaŋ me)\cmn 你没有资格教我\end{exemple}
\begin{relation-sémantique}\confer{
\hyperlink{Ⓔtɯ-rɤʁaŋ}{\textit{ \papi{tɯ-rɤʁaŋ}}}
}\end{relation-sémantique}\end{entrée}

\begin{entrée}
\vedette{\hypertarget{Ⓔznɯrdɯl}{\papi{ znɯrdɯl}}}\markboth{znɯrdɯl}{}
\begin{relation-sémantique}\confer{
\hyperlink{Ⓔnɯrdɯl}{\textit{ \papi{nɯrdɯl}}}
}\end{relation-sémantique}\end{entrée}

\begin{entrée}
\vedette{\hypertarget{Ⓔznɯrɯrɯz}{\papi{ znɯrɯrɯz}}}\markboth{znɯrɯrɯz}{}
\begin{relation-sémantique}\confer{
\hyperlink{Ⓔnɯrɯz}{\textit{ \papi{nɯrɯz}}}
}\end{relation-sémantique}\end{entrée}

\begin{entrée}
\vedette{\hypertarget{Ⓔznɯsɤlɤɣ}{\papi{ znɯsɤlɤɣ}}}\markboth{znɯsɤlɤɣ}{}
\begin{relation-sémantique}\confer{
\hyperlink{Ⓔnɯsɤlɤɣ}{\textit{ \papi{nɯsɤlɤɣ}}}
}\end{relation-sémantique}\end{entrée}

\begin{entrée}
\vedette{\hypertarget{Ⓔznɯslɯɣ}{\papi{ znɯslɯɣ}}}\markboth{znɯslɯɣ}{}
\begin{relation-sémantique}\confer{
\hyperlink{Ⓔnɯslɯɣ}{\textit{ \papi{nɯslɯɣ}}}
}\end{relation-sémantique}\end{entrée}

\begin{entrée}
\vedette{\hypertarget{Ⓔznɯsmɤn}{\papi{ znɯsmɤn}}}\markboth{znɯsmɤn}{}
\begin{relation-sémantique}\confer{
\hyperlink{Ⓔnɯsmɤn}{\textit{ \papi{nɯsmɤn}}}
}\end{relation-sémantique}\end{entrée}

\begin{entrée}
\vedette{\hypertarget{Ⓔznɯstu}{\papi{ znɯstu}}}\markboth{znɯstu}{}
\begin{relation-sémantique}\confer{
\hyperlink{Ⓔnɯstu}{\textit{ \papi{nɯstu}}}
}\end{relation-sémantique}\end{entrée}

\begin{entrée}
\vedette{\hypertarget{Ⓔznɯta}{\papi{ znɯta}}}\markboth{znɯta}{}
\begin{relation-sémantique}\confer{
\hyperlink{Ⓔta}{\textit{ \papi{ta}}}
}\end{relation-sémantique}\end{entrée}

\begin{entrée}
\vedette{\hypertarget{Ⓔznɯtɕarloŋ}{\papi{ znɯtɕarloŋ}}}\markboth{znɯtɕarloŋ}{}
\begin{relation-sémantique}\confer{
\hyperlink{Ⓔnɯtɕarloŋ}{\textit{ \papi{nɯtɕarloŋ}}}
}\end{relation-sémantique}\end{entrée}

\begin{entrée}
\vedette{\hypertarget{Ⓔznɯtɕhɤl}{\papi{ znɯtɕhɤl}}}\markboth{znɯtɕhɤl}{}
\begin{relation-sémantique}\confer{
\hyperlink{Ⓔnɯtɕhɤl}{\textit{ \papi{nɯtɕhɤl}}}
}\end{relation-sémantique}\end{entrée}

\begin{entrée}
\vedette{\hypertarget{Ⓔznɯtɕhɤtpa}{\papi{ znɯtɕhɤtpa}}}\markboth{znɯtɕhɤtpa}{}\classe{vt}
\begin{définition}\fra punir\end{définition}
\begin{définition}\cmn 惩罚\end{définition}
\begin{relation-sémantique}\confer{
\hyperlink{Ⓔtɕhɤtpa}{\textit{ \papi{tɕhɤtpa}}}
}\end{relation-sémantique}
\begin{relation-sémantique}\synonyme{
\hyperlink{Ⓔznɯtɕhɤl}{\textit{ \papi{znɯtɕhɤl}}}
}\end{relation-sémantique}
\end{entrée}

\begin{entrée}
\vedette{\hypertarget{Ⓔznɯtɯfɕɤl}{\papi{ znɯtɯfɕɤl}}}\markboth{znɯtɯfɕɤl}{}
\begin{relation-sémantique}\confer{
\hyperlink{Ⓔnɯtɯfɕɤl}{\textit{ \papi{nɯtɯfɕɤl}}}
}\end{relation-sémantique}\end{entrée}

\begin{entrée}
\vedette{\hypertarget{Ⓔznɯxpri}{\papi{ znɯxpri}}}\markboth{znɯxpri}{}\classe{vt}
\paradigme{\textit{dir :} \jya nɯ-}
\begin{définition}\fra prendre ... comme prétexte\end{définition}
\begin{définition}\cmn 以……为借口\end{définition}
\begin{exemple}\jya kɯ-raχtɯ nɯ-znɯxpri-t-a tɕe jɤ-ari-a\cmn 我以买东西为借口去了那边\end{exemple}
\begin{exemple}\jya ɯ-kɤ-znɯxpri ci pjɤ-tu\cmn 他有个借口\end{exemple}
\begin{exemple}\jya ɯ-znɯxpri ɲɤ-ɕar\cmn 他找了个借口\end{exemple}
\begin{relation-sémantique}\confer{
\hyperlink{Ⓔtɯpri}{\textit{ \papi{tɯpri}}}
}\end{relation-sémantique}\end{entrée}

\begin{entrée}
\vedette{\hypertarget{Ⓔznɯχamba}{\papi{ znɯχamba}}}\markboth{znɯχamba}{}
\begin{relation-sémantique}\confer{
\hyperlink{Ⓔrɯχamba}{\textit{ \papi{rɯχamba}}}
}\end{relation-sémantique}\end{entrée}

\begin{entrée}
\vedette{\hypertarget{Ⓔznɯχcɤl}{\papi{ znɯχcɤl}}}\markboth{znɯχcɤl}{}
\classe{vt}
\paradigme{\textit{dir :} \jya tɤ-}
\begin{définition}\fra atteindre la cible\end{définition}
\begin{définition}\cmn 打中\end{définition}
\begin{exemple}\jya tɤ-fsɯr ɯ-taʁ tɤ-znɯχcal-a ʑo tɤ-lat-a\cmn 我对着靶子中心射了枪\end{exemple}
\begin{exemple}\jya ɕɤmɯɣdɯ tɤ-znɯχcal-a pɯ-cha-a\cmn 我射枪成功地射中了\end{exemple}
\begin{relation-sémantique}\confer{
\hyperlink{Ⓔɯ-χcɤl}{\textit{ \papi{ɯ-χcɤl}}}
}\end{relation-sémantique}\end{entrée}

\begin{entrée}
\vedette{\hypertarget{Ⓔznɯχpi}{\papi{ znɯχpi}}}\markboth{znɯχpi}{}\classe{vt}
\paradigme{\textit{dir :} \jya pɯ-}
\begin{définition}\fra imiter\end{définition}
\begin{définition}\cmn 模仿
\begin{déclaration} \étymologie{\papi{dpe}}\end{déclaration}\end{définition}
\begin{exemple}\jya a-tɕhemɤχti ɣɯ ɯ-tɯ-rɤt nɯ pɯ-znɯχpi-t-a\cmn 我模仿了我女朋友的字\end{exemple}\end{entrée}

\begin{entrée}
\vedette{\hypertarget{Ⓔznɯχtɕɯrɯ}{\papi{ znɯχtɕɯrɯ}}}\markboth{znɯχtɕɯrɯ}{}
\begin{relation-sémantique}\confer{
\hyperlink{Ⓔnɯχtɕɯrɯ}{\textit{ \papi{nɯχtɕɯrɯ}}}
}\end{relation-sémantique}\end{entrée}

\begin{entrée}
\vedette{\hypertarget{Ⓔznɯzɤz}{\papi{ znɯzɤz}}}\markboth{znɯzɤz}{}\classe{vt}
\paradigme{\textit{dir :} \jya tɤ-}
\begin{définition}\fra appâter\end{définition}
\begin{définition}\cmn 用……引诱\end{définition}
\begin{exemple}\jya nɤki ŋgumdʑɯɣ kɤ-znɯzɤz mɤ-khɯ\cmn 那个领导不会被引诱\end{exemple}\end{entrée}

\begin{entrée}
\vedette{\hypertarget{Ⓔznɯzdɯɣ}{\papi{ znɯzdɯɣ}}}\markboth{znɯzdɯɣ}{}
\begin{relation-sémantique}\confer{
\hyperlink{Ⓔnɯzdɯɣ}{\textit{ \papi{nɯzdɯɣ}}}
}\end{relation-sémantique}\end{entrée}

\begin{entrée}
\vedette{\hypertarget{Ⓔznɯzɟɯ}{\papi{ znɯzɟɯ}}}\markboth{znɯzɟɯ}{}
\begin{relation-sémantique}\confer{
\hyperlink{Ⓔnɯzɟɯ}{\textit{ \papi{nɯzɟɯ}}}
}\end{relation-sémantique}\end{entrée}

\begin{entrée}
\vedette{\hypertarget{Ⓔznɯʑɣɤʑɣɤt}{\papi{ znɯʑɣɤʑɣɤt}}}\markboth{znɯʑɣɤʑɣɤt}{}
\begin{relation-sémantique}\confer{
\hyperlink{Ⓔsɤʑɣɤʑɣɤt}{\textit{ \papi{sɤʑɣɤʑɣɤt}}}
}\end{relation-sémantique}\end{entrée}

\begin{entrée}
\vedette{\hypertarget{ⒺzɲɟaⒽ2}{\papi{ zɲɟa}}}\markboth{zɲɟa}{}\homonyme{2}
\classe{n}
\begin{définition}\fra une espèce d'arbrisseau\end{définition}
\begin{définition}\cmn 【黄刺泡儿】\end{définition}
\begin{exemple}\jya zɲɟa nɯ tɯ-ji mŋu ndo ra kɤ-ɬoʁ rga, ɯ-jwaʁ kɯ-ɤɲaʁndzɯm ŋu, ɯʑo mɤ-mbro, ɯ-jwaʁ ɯ-taʁ ɯ-ru ɯ-taʁ ɯ-mdzu dɤn, wuma ʑo mtɕoʁ, ɯ-mɯntoʁ kɯ-wɣrum ŋu, ɯ-mat thɯ-tɯt tɕe ʁmɤrsɤr ŋu, wuma ʑo chi. ɯ-mat kɯ-ndɯ-ndɯβ ʑo kɯ-ɤrtɯ-rtɯm ʑo boʁ boʁ ŋu\cmn 黄刺泡儿一般生长在田边地角,叶子是暗绿色的,长得不高,叶子和茎上长满尖锐的刺,花是白色的,果实成熟后呈金黄色,很甜。果实是由聚集在一起的小球组成的。\end{exemple}
\begin{relation-sémantique}\confer{
\hyperlink{Ⓔzɲɟɤsɯsi}{\textit{ \papi{zɲɟɤsɯsi}}}
}\end{relation-sémantique}\end{entrée}

\begin{entrée}
\vedette{\hypertarget{ⒺzɲɟaⒽ1}{\papi{ zɲɟa}}}\markboth{zɲɟa}{}\homonyme{1}\classe{vs}
\paradigme{\textit{dir :} \jya tɤ-}
\begin{définition}\fra stablement maintenu\end{définition}
\begin{définition}\cmn 夹得稳(夹子)\end{définition}
\begin{exemple}\jya tamɢom ɲɯ-zɲɟa\cmn 夹子夹得稳\end{exemple}
\begin{exemple}\jya tamɢom nɯ kɯ a-mtɕhɯrme thɯ-sɯ-phɯt-a ri ɲɯ-zɲɟa tɕe ɲɯ-pe\cmn 我用夹子拔了胡子,它夹得非常稳\end{exemple}\end{entrée}

\begin{entrée}
\vedette{\hypertarget{Ⓔzɲɟɤsɯsi}{\papi{ zɲɟɤsɯsi}}}\markboth{zɲɟɤsɯsi}{}\classe{n}
\begin{définition}\fra espèce de baie\end{définition}
\begin{définition}\cmn 黄刺泡儿的果子\end{définition}
\begin{relation-sémantique}\confer{
\hyperlink{ⒺzɲɟaⒽ2}{\textit{ \papi{zɲɟa2}}}
}\end{relation-sémantique}\end{entrée}

\begin{entrée}
\vedette{\hypertarget{Ⓔzɲɟɤʑru}{\papi{ zɲɟɤʑru}}}\markboth{zɲɟɤʑru}{}\classe{n}
\begin{définition}\fra une plante\end{définition}
\begin{définition}\cmn 植物的一种\end{définition}
\begin{exemple}\jya zɲɟɤʑru nɯ si kɯ-mbɤr ci ŋu, ʁnɯ-tɯphu tu, tɯ-tɯphu nɯ ɯ-ru kɯ-ɣɯrni ŋu, ɯ-jwaʁ ɯ-ru ɯ-taʁ chɯ-ɤʑɯrja ŋu, ɯ-mat nɯ ɯ-jwaʁ rca chɯ-ɤʑɯrja ŋu, tɕe thɯ-tɯt tɕe, ci kɯ-qarŋe ci tu, kɯ-ɲaʁ ci tu, tɕe kɯ-qarŋe nɯ jndʐɤz, kɯ-ɲaʁ nɯ ndɯβ, tú-wɣ-ndza tɕe chi, zɲɟɤʑru ɯ-ru cho ɯ-jwaʁ ɯ-taʁ ra ɯ-mdzu kɯ-ndɯβ tsa tu. mɤʑɯ tɯ-tɯphu nɯ ɯ-ru cho ɯ-jwaʁ tɯ-ɣndʑɤr thɯ-kɤ-mar ʑo kɯ-fse kɯ-wɣrum tu. ɯ-mat pjɯ-ɴqoʁ tsa ŋu. thɯ-tɯt tɕe qarŋe. kɤ-ndza mɯm. li ɯ-ru cho ɯ-jwaʁ ra ɯ-mdzu tu. nɯ kɯ tu-mbro tsa cha.\cmn 
\stylefv{zɲɟɤʑru}是一种矮小的树,分为两种。一种有红色的树干,叶子排列在树干上,果实也和叶子长在一起。成熟后,有的是黄色,有的是黑色的,黄色的较大,黑色的较小,吃起来很甜。\stylefv{zɲɟɤʑru}的树干和叶子上有小刺。还有一种,树干和叶子上好像涂了白粉一样。果实垂吊着,成熟后变黄。可以吃。树干和叶子上也长有刺。这种\stylefv{zɲɟɤʑru}长得比较高一些。
\end{exemple}\end{entrée}

\begin{entrée}
\vedette{\hypertarget{Ⓔzo}{\papi{ zo}}}\markboth{zo}{}
\classe{vi}
\paradigme{\textit{dir :} \jya kɤ-}
\begin{définition}\fra se poser (oiseau)\end{définition}
\begin{définition}\cmn 停落(鸟)\end{définition}
\begin{exemple}\jya pɣa ko-zo\cmn 鸟停落了\end{exemple}
\begin{exemple}\jya ɣʑo ko-zo\cmn 蜜蜂停落了\end{exemple}\end{entrée}

\begin{entrée}
\vedette{\hypertarget{Ⓔzoŋzoŋ}{\papi{ zoŋzoŋ}}}\markboth{zoŋzoŋ}{}\classe{idph.2}
\begin{définition}\fra ébouriffé, en désordre\end{définition}
\begin{définition}\cmn 形容人蓬头垢面,头发乱蓬蓬的样子,或形容动物的尾巴粗而毛发多\end{définition}
\begin{exemple}\jya nɤ-ku pɯ-sɤɕɤt ma zoŋzoŋ ʑo ɲɯ-pa\cmn 你梳一下头,你的头发乱蓬蓬的\end{exemple}
\begin{relation-sémantique}\confer{
\hyperlink{Ⓔzaŋzaŋ}{\textit{ \papi{zaŋzaŋ}}}
}\end{relation-sémantique}\end{entrée}

\begin{entrée}
\vedette{\hypertarget{Ⓔzraʁrɯz}{\papi{ zraʁrɯz}}}\markboth{zraʁrɯz}{}
\begin{relation-sémantique}\confer{
\hyperlink{Ⓔraʁrɯz}{\textit{ \papi{raʁrɯz}}}
}\end{relation-sémantique}\end{entrée}

\begin{entrée}
\vedette{\hypertarget{Ⓔzraχtɕi}{\papi{ zraχtɕi}}}\markboth{zraχtɕi}{}\classe{n}
\begin{définition}\fra savon\end{définition}
\begin{définition}\cmn 肥皂\end{définition}
\end{entrée}

\begin{entrée}
\vedette{\hypertarget{Ⓔzrɤβ}{\papi{ zrɤβ}}}\markboth{zrɤβ}{}\classe{n}
\begin{définition}\fra bouc\end{définition}
\begin{définition}\cmn 公山羊\end{définition}\end{entrée}

\begin{entrée}
\vedette{\hypertarget{Ⓔzrɤβraʁ}{\papi{ zrɤβraʁ}}}\markboth{zrɤβraʁ}{}
\begin{relation-sémantique}\confer{
\hyperlink{Ⓔrɤβraʁ}{\textit{ \papi{rɤβraʁ}}}
}\end{relation-sémantique}\end{entrée}

\begin{entrée}
\vedette{\hypertarget{Ⓔzrɤɣrɯ}{\papi{ zrɤɣrɯ}}}\markboth{zrɤɣrɯ}{}
\begin{relation-sémantique}\confer{
\hyperlink{Ⓔrɤɣrɯ}{\textit{ \papi{rɤɣrɯ}}}
}\end{relation-sémantique}\end{entrée}

\begin{entrée}
\vedette{\hypertarget{Ⓔzrɤjroʁ}{\papi{ zrɤjroʁ}}}\markboth{zrɤjroʁ}{}
\begin{relation-sémantique}\confer{
\hyperlink{Ⓔrɤjroʁ}{\textit{ \papi{rɤjroʁ}}}
}\end{relation-sémantique}\end{entrée}

\begin{entrée}
\vedette{\hypertarget{Ⓔzrɤkrɯ}{\papi{ zrɤkrɯ}}}\markboth{zrɤkrɯ}{}
\begin{relation-sémantique}\confer{
\hyperlink{Ⓔrɤkrɯ}{\textit{ \papi{rɤkrɯ}}}
}\end{relation-sémantique}\end{entrée}

\begin{entrée}
\vedette{\hypertarget{Ⓔzrɤma}{\papi{ zrɤma}}}\markboth{zrɤma}{}
\begin{relation-sémantique}\confer{
\hyperlink{Ⓔrɤma}{\textit{ \papi{rɤma}}}
}\end{relation-sémantique}\end{entrée}

\begin{entrée}
\vedette{\hypertarget{Ⓔzrɤmgo}{\papi{ zrɤmgo}}}\markboth{zrɤmgo}{}
\classe{vt}
\paradigme{\textit{dir :} \jya tɤ-}
\begin{définition}\fra mélanger une poudre avec un liquide et en faire des boules\end{définition}
\begin{définition}\cmn 把粉状的物体跟液体混在一起,揉成一坨一坨\end{définition}
\begin{exemple}\jya tɯ-ci kɯ tɤjlu tɤ-zrɤmgo-t-a\cmn 我在面粉里放了一点水,揉成一坨一坨\end{exemple}
\begin{relation-sémantique}\confer{
\hyperlink{Ⓔtɯ-mgo}{\textit{ \papi{tɯ-mgo}}}
}\end{relation-sémantique}\end{entrée}

\begin{entrée}
\vedette{\hypertarget{Ⓔzrɤmpɕɤr}{\papi{ zrɤmpɕɤr}}}\markboth{zrɤmpɕɤr}{}
\begin{relation-sémantique}\confer{
\hyperlink{Ⓔrɤmpɕɤr}{\textit{ \papi{rɤmpɕɤr}}}
}\end{relation-sémantique}\end{entrée}

\begin{entrée}
\vedette{\hypertarget{Ⓔzrɤntɕɯ}{\papi{ zrɤntɕɯ}}}\markboth{zrɤntɕɯ}{}\classe{n}
\begin{définition}\fra haricot\end{définition}
\begin{définition}\cmn 绿豆
\begin{déclaration} \étymologie{\papi{sran}}\end{déclaration}\end{définition}
\begin{exemple}\jya zrɤntɕɯ nɯ li tɤ-rɤku ci ŋu, tɯ-ji nɯ mɤ-kɯ-sna tsa lu-ji khɯ, ɯʑo kɯ-xtɕi ci ŋu, tu-wxti mɤ-cha, ɯ-tshɯɣa nɯ staχpɯ cho naχtɕɯɣ, ɯ-jwaʁ, ɯ-ru, ɯ-mɯntoʁ, ɯ-cɤβ nɯ ra lonba staχpɯ fsɯ-fse ʑo fse, staχpɯ wuma ʑɤ tu-rɲɟi cha, ɯ-rdoʁ ɯ-jndʐɤz artɯm rloʁrloʁ. zrɤntɕɯ ɯ-rdoʁ xtɕi cho aɕpɯɕpa tsa.\cmn 绿豆是一种庄稼,可以种在贫瘠的地里。它较小,长不大,形状和豌豆一样,叶子、茎、花、荚果和豌豆一模一样。豌豆长得长,颗粒是球形的。绿豆的颗粒小而扁。\end{exemple}\end{entrée}

\begin{entrée}
\vedette{\hypertarget{Ⓔzrɤpɯ}{\papi{ zrɤpɯ}}}\markboth{zrɤpɯ}{}
\begin{relation-sémantique}\confer{
\hyperlink{Ⓔrɤpɯ}{\textit{ \papi{rɤpɯ}}}
}\end{relation-sémantique}
\end{entrée}

\begin{entrée}
\vedette{\hypertarget{Ⓔzrɤru}{\papi{ zrɤru}}}\markboth{zrɤru}{}
\begin{relation-sémantique}\confer{
\hyperlink{Ⓔrɤru}{\textit{ \papi{rɤru}}}
}\end{relation-sémantique}\end{entrée}

\begin{entrée}
\vedette{\hypertarget{Ⓔzrɤrɤt}{\papi{ zrɤrɤt}}}\markboth{zrɤrɤt}{}
\begin{relation-sémantique}\confer{
\hyperlink{Ⓔrɤt}{\textit{ \papi{rɤt}}}
}\end{relation-sémantique}
\end{entrée}

\begin{entrée}
\vedette{\hypertarget{Ⓔzrɤrɟit}{\papi{ zrɤrɟit}}}\markboth{zrɤrɟit}{}
\begin{relation-sémantique}\confer{
\hyperlink{Ⓔrɤrɟit}{\textit{ \papi{rɤrɟit}}}
}\end{relation-sémantique}
\end{entrée}

\begin{entrée}
\vedette{\hypertarget{Ⓔzrɤrmbɣo}{\papi{ zrɤrmbɣo}}}\markboth{zrɤrmbɣo}{}
\begin{relation-sémantique}\confer{
\hyperlink{Ⓔrɤrmbɣo}{\textit{ \papi{rɤrmbɣo}}}
}\end{relation-sémantique}\end{entrée}

\begin{entrée}
\vedette{\hypertarget{Ⓔzrɤsta}{\papi{ zrɤsta}}}\markboth{zrɤsta}{}
\begin{relation-sémantique}\confer{
\hyperlink{Ⓔrɤsta}{\textit{ \papi{rɤsta}}}
}\end{relation-sémantique}\end{entrée}

\begin{entrée}
\vedette{\hypertarget{Ⓔzrɤtɕha}{\papi{ zrɤtɕha}}}\markboth{zrɤtɕha}{}
\classe{vt}
\begin{définition}\fra déterminé à partir de\end{définition}
\begin{définition}\cmn 以……为标准\end{définition}
\begin{exemple}\jya nɤ-sɯm ɕe mɤ-ɕe tɤ-zrɤtɕhe\cmn 以你想不想(做)为标准\end{exemple}
\begin{exemple}\jya ɯ-spa rtaʁ mɤ-rtaʁ tɤ-zrɤtɕhe\cmn (你衣服裁得多不多)看材料够不够\end{exemple}\begin{sous-entrée}
\vedette{\hypertarget{}{\papi{ arɤtɕha}}}\markboth{arɤtɕha}{}\classe{vi}
\begin{exemple}\jya ɯ-ngra pe mɤ-pe nɯ, kɤ-nɤma pe mɤ-pe arɤtɕha\cmn 他的工钱高不高,看他工作做得好不好\end{exemple}
\begin{exemple}\jya kɤ-zɣɯt tɯ-cha mɤ-tɯ-cha nɯ kɤ-rɟɯɣ tɯ-cha mɤ-tɯ-cha arɤtɕha\cmn 你能不能早到,看你能不能跑步\end{exemple}
\end{sous-entrée}\end{entrée}

\begin{entrée}
\vedette{\hypertarget{Ⓔzrɤtshi}{\papi{ zrɤtshi}}}\markboth{zrɤtshi}{}
\begin{relation-sémantique}\confer{
\hyperlink{Ⓔarɤtshi}{\textit{ \papi{arɤtshi}}}
}\end{relation-sémantique}\end{entrée}

\begin{entrée}
\vedette{\hypertarget{Ⓔzrɤʑi}{\papi{ zrɤʑi}}}\markboth{zrɤʑi}{}\classe{vt}
\paradigme{\textit{dir :} \jya kɤ-}\acception{1}
\begin{définition}\fra faire habiter\end{définition}
\begin{définition}\cmn 使……住在\end{définition}\acception{2}
\begin{définition}\fra laisser\end{définition}
\begin{définition}\cmn 留下\end{définition}
\begin{exemple}\jya ɯʑo kɯre kɤ-zrɤʑi-t-a\cmn 我让他待在这里了\end{exemple}
\begin{exemple}\jya izora tɤ-pɤtso kha ɯ-ngɯ ɯʑosti kɤ-zrɤʑi-j\cmn 我们把孩子一个人留在家里\end{exemple}\end{entrée}

\begin{entrée}
\vedette{\hypertarget{Ⓔzri}{\papi{ zri}}}\markboth{zri}{}\classe{vs}
\paradigme{\textit{dir :} \jya thɯ-}
\begin{définition}\fra long\end{définition}
\begin{définition}\cmn 长\end{définition}
\begin{relation-sémantique}\synonyme{
\hyperlink{Ⓔrɲɟi}{\textit{ \papi{rɲɟi}}}
}\end{relation-sémantique}
\begin{relation-sémantique}\antonyme{
\hyperlink{ⒺxtɯtⒽ1}{\textit{ \papi{xtɯt}}}
}\end{relation-sémantique}\end{entrée}

\begin{entrée}
\vedette{\hypertarget{ⒺzrɯⒽ1}{\papi{ zrɯ}}}\markboth{zrɯ}{}\homonyme{1}
\classe{n}
\begin{définition}\fra parasite des bovins\end{définition}
\begin{définition}\cmn 牛的寄生虫\end{définition}\end{entrée}

\begin{entrée}
\vedette{\hypertarget{ⒺzrɯⒽ2}{\papi{ zrɯ}}}\markboth{zrɯ}{}\homonyme{2}
\classe{n}
\begin{définition}\fra adret\end{définition}
\begin{définition}\cmn 山阳,向阳的山坡\end{définition}
\end{entrée}

\begin{entrée}
\vedette{\hypertarget{ⒺzrɯⒽ3}{\papi{ zrɯ}}}\markboth{zrɯ}{}\homonyme{3}
\classe{vt}
\paradigme{\textit{dir :} \jya nɯ-}
\begin{définition}\fra s'accaparer\end{définition}
\begin{définition}\cmn 霸占,占用\end{définition}
\begin{exemple}\jya sɤtɕha nɯ-zrɯ-t-a\cmn 我占用了这个地方\end{exemple}\end{entrée}

\begin{entrée}
\vedette{\hypertarget{Ⓔzrɯβɟu}{\papi{ zrɯβɟu}}}\markboth{zrɯβɟu}{}\classe{n}
\begin{définition}\fra dans un fardeau de bois, la partie qui est en contact avec le dos du porteur\end{définition}
\begin{définition}\cmn 柴捆子里比较细的枝条,接触人的背部\end{définition}
\begin{relation-sémantique}\antonyme{
\hyperlink{Ⓔzɣɯqhu}{\textit{ \papi{zɣɯqhu}}}
}\end{relation-sémantique}\end{entrée}

\begin{entrée}
\vedette{\hypertarget{Ⓔzrɯɕmi}{\papi{ zrɯɕmi}}}\markboth{zrɯɕmi}{}
\begin{relation-sémantique}\confer{
\hyperlink{Ⓔrɯɕmi}{\textit{ \papi{rɯɕmi}}}
}\end{relation-sémantique}
\end{entrée}

\begin{entrée}
\vedette{\hypertarget{Ⓔzrɯɣ}{\papi{ zrɯɣ}}}\markboth{zrɯɣ}{}\classe{n}
\begin{définition}\fra pou\end{définition}
\begin{définition}\cmn 虱子\end{définition}
\begin{exemple}\jya zrɯɣ nɯ-nɤmbɣaʁlaʁ rdɯl mɤ-tɕɤt\cmn 虱子打滚也不会起灰尘(没有什么可怕的)\end{exemple}
\begin{relation-sémantique}\confer{
\hyperlink{Ⓔaɣɯzrɯɣ}{\textit{ \papi{aɣɯzrɯɣ}}}
}\end{relation-sémantique}\end{entrée}

\begin{entrée}
\vedette{\hypertarget{Ⓔzrɯɣnɤn}{\papi{ zrɯɣnɤn}}}\markboth{zrɯɣnɤn}{}
\begin{relation-sémantique}\confer{
\hyperlink{Ⓔrɯɣnɤn}{\textit{ \papi{rɯɣnɤn}}}
}\end{relation-sémantique}\end{entrée}

\begin{entrée}
\vedette{\hypertarget{Ⓔzrɯɣndza}{\papi{ zrɯɣndza}}}\markboth{zrɯɣndza}{}\classe{n}
\begin{définition}\fra mante religieuse\end{définition}
\begin{définition}\cmn 螳螂\end{définition}
\begin{exemple}\jya zrɯɣndza nɯ qajɯ ci ŋu, ɯ-mi kɯtʂɤ-ldʑa tu, ɯ-ku kɯ-xtɕi tɕe kɯ-ɤmtɕoʁ ci ŋu, ɯ-phoŋbu kɯ-wxti tsa ci ŋu, ɯ-smɤt tɕe chɯ-ɤmtɕoʁ tsa ŋu, tɕe ɯ-mgɯr ɯ-qhu ra rko, ɯ-xtɤpa ra mpɯ tɕe zrɯɣ kɤ-ndza wuma rga, kɯ-mɤku tɕe, zrɯɣ pha ɯ-phoŋbu tɯtɯrca chɯ-mqlaʁ ŋu, khro tsa ta-ndza tɕe, tu-fka ɲɯ-ŋu tɕe zrɯɣ nɯ-atɯɣ tɕe ɯ-se ku-tshi tɕe ɲɯ-βde ɲɯ-ŋu. pha ɯ-phoŋbu kɯ-ɤrŋi, sɯjno cho aɣɯmdoʁ.\cmn 螳螂是一种虫,有六只脚,头小而尖,身子较大,尾部是尖的。背部硬,肚皮软。它爱吃虱子,开始是整个吃掉。吃了几个以后,饱了,再遇到虱子的时候,喝了血就扔了。螳螂全身是绿色的,和草的颜色一样。\end{exemple}\end{entrée}

\begin{entrée}
\vedette{\hypertarget{Ⓔzrɯɣru}{\papi{ zrɯɣru}}}\markboth{zrɯɣru}{}\classe{n}
\begin{définition}\fra épouillage\end{définition}
\begin{définition}\cmn 捉虱子\end{définition}
\begin{relation-sémantique}\confer{
\hyperlink{Ⓔzrɯɣ}{\textit{ \papi{zrɯɣ}}}
}\end{relation-sémantique}
\begin{relation-sémantique}\confer{
\hyperlink{ⒺruⒽ2}{\textit{ \papi{ru2}}}
}\end{relation-sémantique}
\begin{relation-sémantique}\confer{
\hyperlink{Ⓔnɯzrɯɣru}{\textit{ \papi{nɯzrɯɣru}}}
}\end{relation-sémantique}\end{entrée}

\begin{entrée}
\vedette{\hypertarget{Ⓔzrɯndzɤtshi}{\papi{ zrɯndzɤtshi}}}\markboth{zrɯndzɤtshi}{}
\begin{relation-sémantique}\confer{
\hyperlink{Ⓔrɯndzɤtshi}{\textit{ \papi{rɯndzɤtshi}}}
}\end{relation-sémantique}\end{entrée}

\begin{entrée}
\vedette{\hypertarget{Ⓔzrɯŋgrɤl}{\papi{ zrɯŋgrɤl}}}\markboth{zrɯŋgrɤl}{}
\begin{relation-sémantique}\confer{
\hyperlink{Ⓔŋgrɤl}{\textit{ \papi{ŋgrɤl}}}
}\end{relation-sémantique}\end{entrée}

\begin{entrée}
\vedette{\hypertarget{Ⓔzrɯstɯnmɯ}{\papi{ zrɯstɯnmɯ}}}\markboth{zrɯstɯnmɯ}{}
\begin{relation-sémantique}\confer{
\hyperlink{Ⓔrɯstɯnmɯ}{\textit{ \papi{rɯstɯnmɯ}}}
}\end{relation-sémantique}
\end{entrée}

\begin{entrée}
\vedette{\hypertarget{Ⓔzrɯxtar}{\papi{ zrɯxtar}}}\markboth{zrɯxtar}{}\classe{vt}
\paradigme{\textit{dir :} \jya tɤ-}
\begin{définition}\fra développer\end{définition}
\begin{définition}\cmn 使兴旺起来\end{définition}
\begin{exemple}\jya jiɕqha ta-zrɯxtar-nɯ\cmn 他们使它兴旺起来\end{exemple}
\begin{exemple}\jya kha kɤ-zrɯxtar pjɤ-cha-nɯ\cmn 他们成功地令自己的家庭兴旺起来了\end{exemple}
\begin{relation-sémantique}\confer{
\hyperlink{Ⓔsɯxtar}{\textit{ \papi{sɯxtar}}}
}\end{relation-sémantique}\end{entrée}

\begin{entrée}
\vedette{\hypertarget{Ⓔzʁaʁzʁaʁ}{\papi{ zʁaʁzʁaʁ}}}\markboth{zʁaʁzʁaʁ}{}\classe{idph.2}
\begin{définition}\fra correctement habillé\end{définition}
\begin{définition}\cmn 形容穿得很精干的样子\end{définition}
\begin{exemple}\jya nɯ-ŋga ra tu-kɯ-ɣɤxtɯt nɯ zʁaʁzʁaʁ ʑo ɲɯ-pa\cmn 穿高一点的衣服看起来很威武精干\end{exemple}
\begin{relation-sémantique}\antonyme{
\hyperlink{Ⓔlɲɯɣlɲɯɣ}{\textit{ \papi{lɲɯɣlɲɯɣ}}}
}\end{relation-sémantique}\end{entrée}

\begin{entrée}
\vedette{\hypertarget{Ⓔzʁɤɲcɯ}{\papi{ zʁɤɲcɯ}}}\markboth{zʁɤɲcɯ}{}\classe{n}
\begin{définition}\fra fronde\end{définition}
\begin{définition}\cmn 投石带【石子带】\end{définition}
\begin{exemple}\jya zʁɤɲcɯ ci to-lɤt tɕe pɣa to-sɯxtsɯɣ\cmn 他用投石带射中了鸟\end{exemple}
\begin{exemple}\jya zʁɤɲcɯ nɯ tɤ-fsɤri maʁ nɤ qase nɯ-kɤ-βzu ŋu tɕe ɯ-χcɤl nɯ tɕu rdɤstaʁ ɯ-sɤɣ-raʁ ci tú-wɣ-βzu tɕe, tú-wɣ-zdɤβ tɕe ɯ-ɕnɤz tɯka nɯ kú-wɣ-ndo, ɯ-χcɤl nɯ tɕu rdɤstaʁ kɯ-xtɕi tsa chɯ́-wɣ-sɯɣraʁ tɕe tɯ-ku ɯ-taʁ tɯ-jaʁ ntsi kɯ χsɯ-tɤxɯr jamar kú-wɣ-sɯ-sɯ-mtɕɯr tɕe ɯ-ɕnɤz tɯ-rdoʁ nɯ ɲɯ́-wɣ-ta tɕe ɲɯ́-wɣ-lɤt, tɕe rdɤstaʁ kɯ-ɤrqhi ʑo ju-ɕe cha. tɕe nɯ rdɤstaʁ sɤ-lɤt ɯ-spa ŋu.\cmn 投石带用麻绳或者皮绳做成。中间做一个能卡住石子的(结),(从中间)叠一下。绳的两头拿在手上,在中间卡住小石子,然后在头上挥转三圈,之后再把绳子的一头放掉,打出去,这样石子就投得远一些。投石带是投掷石头的专用工具。\end{exemple}\end{entrée}

\begin{entrée}
\vedette{\hypertarget{Ⓔzʁɤzʁɤt}{\papi{ zʁɤzʁɤt}}}\markboth{zʁɤzʁɤt}{}\classe{idph.3}
\begin{définition}\fra (enfant) habillé de façon correcte\end{définition}
\begin{définition}\cmn 形容(小孩子)穿得很整齐的样子\end{définition}\end{entrée}

\begin{entrée}
\vedette{\hypertarget{Ⓔzɯ}{\papi{ zɯ}}}\markboth{zɯ}{}\classe{postp}
\begin{définition}\fra locatif\end{définition}
\begin{définition}\cmn 在\end{définition}\end{entrée}

\begin{entrée}
\vedette{\hypertarget{Ⓔzɯɣzɯɣ}{\papi{ zɯɣzɯɣ}}}\markboth{zɯɣzɯɣ}{}
\classe{idph.2}
\begin{définition}\fra stable, immobile\end{définition}
\begin{définition}\cmn 稳定的状态,一动也不动\end{définition}
\begin{exemple}\jya zɯɣzɯɣ ʑo ɲɯ-ɤsɯ-ndo\cmn 他拿着不放,动也不动\end{exemple}
\begin{exemple}\jya zɯɣzɯɣ ʑo ɲɯ-rɤʑi\cmn 他坐在那里,动也不动\end{exemple}
\begin{relation-sémantique}\synonyme{
\hyperlink{Ⓔgrɯɣgrɯɣ}{\textit{ \papi{grɯɣgrɯɣ}}}
}\end{relation-sémantique}\end{entrée}

\begin{entrée}
\vedette{\hypertarget{Ⓔzɯm}{\papi{ zɯm}}}\markboth{zɯm}{}\classe{n}
\begin{définition}\fra seau\end{définition}
\begin{définition}\cmn 桶
\begin{déclaration} \étymologie{\papi{zom}}\end{déclaration}\end{définition}
\end{entrée}

\begin{entrée}
\vedette{\hypertarget{Ⓔzɯmbɯr}{\papi{ zɯmbɯr}}}\markboth{zɯmbɯr}{}\classe{n}
\begin{définition}\fra bouton d'argent\end{définition}
\begin{définition}\cmn 银盆\end{définition}\end{entrée}

\begin{entrée}
\vedette{\hypertarget{Ⓔzɯmi}{\papi{ zɯmi}}}\markboth{zɯmi}{}\classe{adv}
\begin{définition}\fra presque\end{définition}
\begin{définition}\cmn 差一点;几乎\end{définition}
\begin{exemple}\jya rdɤstaʁ ta-lɤt, zɯmi ʑo jɯ-tɤ́-wɣ-tsɯɣ-a\cmn 差一点打到了我\end{exemple}
\begin{exemple}\jya zɯmi ɲɯ-naχtɕɯɣ\cmn 几乎一样\end{exemple}\end{entrée}

\begin{entrée}
\vedette{\hypertarget{Ⓔzɯmjɯ}{\papi{ zɯmjɯ}}}\markboth{zɯmjɯ}{}\classe{n}
\begin{définition}\fra lanière servant à porter les seaux d'eau\end{définition}
\begin{définition}\cmn 背水桶的带子\end{définition}\end{entrée}

\begin{entrée}
\vedette{\hypertarget{Ⓔzɯmzɯm}{\papi{ zɯmzɯm}}}\markboth{zɯmzɯm}{}\classe{idph.2}
\begin{définition}\fra coupé très fin\end{définition}
\begin{définition}\cmn 形容切得很细\end{définition}
\begin{exemple}\jya pjɯ́-wɣ-rɤkrɯ zɯmzɯm ʑo ɲɯ-ra\cmn 要切得很细\end{exemple}\end{entrée}

\begin{entrée}
\vedette{\hypertarget{Ⓔzɯn}{\papi{ zɯn}}}\markboth{zɯn}{}\classe{n}
\begin{définition}\fra argent de l'époque impériale\end{définition}
\begin{définition}\cmn 民国之前通行的货币\end{définition}
\end{entrée}

\begin{entrée}
\vedette{\hypertarget{Ⓔzɯŋzɯŋ}{\papi{ zɯŋzɯŋ}}}\markboth{zɯŋzɯŋ}{}
\classe{idph.2}
\begin{définition}\fra complètement blanc\end{définition}
\begin{définition}\cmn 全白\end{définition}
\begin{exemple}\jya rgɤtpu ɯ-ku cho-wɣrum zɯŋzɯŋ ʑo\cmn 老头子的头发全变白了\end{exemple}\end{entrée}

\begin{entrée}
\vedette{\hypertarget{Ⓔzɯxtɕhɤl}{\papi{ zɯxtɕhɤl}}}\markboth{zɯxtɕhɤl}{}\classe{n}
\begin{définition}\fra cymbales\end{définition}
\begin{définition}\cmn 钹
\begin{déclaration} \étymologie{\papi{sbug.tɕʰal}}\end{déclaration}\end{définition}\end{entrée}

\begin{entrée}
\vedette{\hypertarget{Ⓔzwu}{\papi{ zwu}}}\markboth{zwu}{}\classe{n}
\begin{définition}\fra maladie de l'œil\end{définition}
\begin{définition}\cmn 眼病\end{définition}
\begin{exemple}\jya ɯ-mɲaʁ zwu to-ɣi\cmn 他眼上长了痘痘\end{exemple}\end{entrée}

\begin{entrée}
\vedette{\hypertarget{Ⓔzwaʁnɤzwaʁ}{\papi{ zwaʁnɤzwaʁ}}}\markboth{zwaʁnɤzwaʁ}{}
\classe{idph.3}
\begin{définition}\fra mou, pas ferme\end{définition}
\begin{définition}\cmn 形容物体绵软\end{définition}
\begin{exemple}\jya tú-wɣ-ndza tɕe zwaʁnɤzwaʁ ɲɯ-ti\cmn 吃起来软绵绵的\end{exemple}
\begin{relation-sémantique}\antonyme{
\hyperlink{Ⓔtɕʁɯznɤtɕʁɯz}{\textit{ \papi{tɕʁɯznɤtɕʁɯz}}}
}\end{relation-sémantique}\end{entrée}

\begin{entrée}
\vedette{\hypertarget{Ⓔzwɤɣrum}{\papi{ zwɤɣrum}}}\markboth{zwɤɣrum}{}\classe{n}
\begin{définition}\fra armoise blanche\end{définition}
\begin{définition}\cmn 白艾蒿\end{définition}
\begin{relation-sémantique}\confer{
\hyperlink{ⒺzwɤrⒽ2}{\textit{ \papi{zwɤr2}}}
}\end{relation-sémantique}
\begin{relation-sémantique}\confer{
\hyperlink{Ⓔwɣrum}{\textit{ \papi{wɣrum}}}
}\end{relation-sémantique}\end{entrée}

\begin{entrée}
\vedette{\hypertarget{ⒺzwɤrⒽ2}{\papi{ zwɤr}}}\markboth{zwɤr}{}\homonyme{2}
\classe{n}
\begin{définition}\fra armoise\end{définition}
\begin{définition}\cmn 蒿\end{définition}
\begin{exemple}\jya zwɤr nɯ ɯ-ru cho ɯ-jwaʁ kɤsɯfse kɯ-pɣi ŋu. ɯ-jwaʁ nɯ kɯ-ɤɣɯrʑɯɣʑɯɣ kɯ-fse ŋu. zwɤr ɯ-di χɕɤβ, kɤntɕhɯ-tɯphu tu, zwɤrqha kɯ-rmi ci tu, si ŋu. zwɤɣrum kɤ-ti ci tu tɕe nɯ aɣrɤɣrum. zwɤrɲaʁ kɤ-ti ci tu tɕe nɯ aɲaʁndzɯm. ɴqiazwɤr kɤ-ti ci tu tɕe nɯ ɯ-jwaʁ ra mba tsa ɯ-mdoʁ nɯ arŋi. kɤ-ndza sna. zwɤrqha, zwɤɣrum, zwɤrɲaʁ nɯ ra nɯ-jwaʁ ɯ-taʁ ɯ-rme sɯβsɯβ tu, ɴqiazwɤr ɯ-jwaʁ ɯ-taʁ ɯ-rme me.\cmn 
蒿的茎和叶子全部都是灰色的,叶子上有褶,香味浓。蒿分很多种。叫做\stylefv{zwɤrqha}的是一种树。叫做\stylefv{zwɤrɣrum}的淡白色。叫做\stylefv{zwɤrɲaʁ}的颜色比较深。叫做\stylefv{qiaβzwɤr}(香蒿)的叶子薄,绿色,可以吃。前三种蒿叶子上有细毛,香蒿叶子没有毛。
\end{exemple}
\begin{relation-sémantique}\confer{
\hyperlink{Ⓔzwɤɣrum}{\textit{ \papi{zwɤɣrum}}}
}\end{relation-sémantique}
\begin{relation-sémantique}\confer{
\hyperlink{Ⓔzwɤrqha}{\textit{ \papi{zwɤrqha}}}
}\end{relation-sémantique}
\begin{relation-sémantique}\confer{
\hyperlink{Ⓔɴqiazwɤr}{\textit{ \papi{ɴqiazwɤr}}}
}\end{relation-sémantique}\end{entrée}

\begin{entrée}
\vedette{\hypertarget{ⒺzwɤrⒽ1}{\papi{ zwɤr}}}\markboth{zwɤr}{}\homonyme{1}\classe{vt}
\paradigme{\textit{dir :} \jya kɤ-}
\paradigme{\textit{dir :} \jya tɤ-}
\begin{définition}\fra allumer\end{définition}
\begin{définition}\cmn 点火;点灯
\begin{déclaration} \étymologie{\papi{sbor}}\end{déclaration}\end{définition}
\begin{exemple}\jya smi ka-zwɤr, smi ta-zwɤr\cmn 他点了火\end{exemple}
\begin{exemple}\jya tɤtʂu kɤ-zwar-a\cmn 我点了灯\end{exemple}
\begin{relation-sémantique}\confer{
\hyperlink{Ⓔamɯzwɤr}{\textit{ \papi{amɯzwɤr}}}
}\end{relation-sémantique}\end{entrée}

\begin{entrée}
\vedette{\hypertarget{Ⓔzwɤrqha}{\papi{ zwɤrqha}}}\markboth{zwɤrqha}{}\classe{n}
\begin{définition}\fra espèce d'armoise\end{définition}
\begin{définition}\cmn 艾蒿的一种\end{définition}
\begin{relation-sémantique}\confer{
\hyperlink{ⒺzwɤrⒽ2}{\textit{ \papi{zwɤr2}}}
}\end{relation-sémantique}\end{entrée}

\begin{entrée}
\vedette{\hypertarget{Ⓔzwɤrqhɤjmɤɣ}{\papi{ zwɤrqhɤjmɤɣ}}}\markboth{zwɤrqhɤjmɤɣ}{}\classe{n}
\begin{définition}\fra une espèce de champignon\end{définition}
\begin{définition}\cmn 一种菌子\end{définition}
\begin{exemple}\jya zwɤrqhɤjmɤɣ nɯ zwɤrqha ɯ-ŋgɯ tu-ɬoʁ ŋu, ɯ-taʁ ɯ-pa kɯ-fsɯ-fse ʑo kɯ-wɣrɯ-wɣrum ʑo ŋu, kɤ-ndza sna\cmn 
\stylefv{zwɤrqhɤjmɤɣ}长在\stylefv{zwɤrqha} 树林里,上部和下部一样都是白色的,能吃。
\end{exemple}\end{entrée}

\newpage\caractère{ʑ}

\begin{entrée}
\vedette{\hypertarget{Ⓔʑu}{\papi{ ʑu}}}\markboth{ʑu}{}\classe{n}
\begin{définition}\fra yaourt\end{définition}
\begin{définition}\cmn 酸奶
\begin{déclaration} \étymologie{\papi{ʑo}}\end{déclaration}\end{définition}
\end{entrée}

\begin{entrée}
\vedette{\hypertarget{ⒺʑaⒽ1}{\papi{ ʑa}}}\markboth{ʑa}{}\homonyme{1}
\classe{vt}
\paradigme{\textit{dir :} \jya \_}
\begin{définition}\fra commencer\end{définition}
\begin{définition}\cmn 开始\end{définition}
\begin{exemple}\jya ɯ-mphru pjɯ-ʑe-a tɕe pjɯ-ndɯn-a ŋu ŋɤ\cmn 我紧接着(从头)开始读\end{exemple}
\begin{exemple}\jya tɯ-nɯrɤɣo chɤ-ʑa\cmn 他开始唱起歌来\end{exemple}
\begin{relation-sémantique}\confer{
\hyperlink{Ⓔsɤʑa}{\textit{ \papi{sɤʑa}}}
}\end{relation-sémantique}\end{entrée}

\begin{entrée}
\vedette{\hypertarget{ⒺʑaⒽ2}{\papi{ ʑa}}}\markboth{ʑa}{}\homonyme{2}\classe{vs}
\paradigme{\textit{dir :} \jya nɯ-}
\begin{définition}\fra avoir une atrophie musculaire\end{définition}
\begin{définition}\cmn 肌肉萎缩\end{définition}
\begin{exemple}\jya ɯ-mi ɲɤ-ʑa\cmn 他脚的肌肉萎缩了\end{exemple}
\begin{exemple}\jya ɯ-jaʁ ɲɤ-ʑa\cmn 他手的肌肉萎缩了\end{exemple}\end{entrée}

\begin{entrée}
\vedette{\hypertarget{Ⓔʑaka}{\papi{ ʑaka}}}\markboth{ʑaka}{}
\classe{n}
\begin{définition}\fra chacun\end{définition}
\begin{définition}\cmn 各自\end{définition}
\begin{exemple}\jya ʑaka tu-nɯsaχsɯ-tɕi ɕti\end{exemple}
\begin{exemple}\jya ʑaka kɤ-nɯ-βzu mɯ-rtaʁ-tɕi\end{exemple}
\end{entrée}

\begin{entrée}
\vedette{\hypertarget{Ⓔʑakastaka}{\papi{ ʑakastaka}}}\markboth{ʑakastaka}{}\classe{n}
\begin{définition}\fra par soi-même, chacun dans son coin\end{définition}
\begin{définition}\cmn 各自各地\end{définition}
\begin{exemple}\jya mɯntoʁ ʑakastaka ɯ-mdoʁ xcat ʑo ɕti\cmn 花有各种各样的颜色\end{exemple}
\begin{exemple}\jya jiʑora ʑakastaka ji-ma nɯ-nɤma-j ra\cmn 我们要各自办各自的事情\end{exemple}
\end{entrée}

\begin{entrée}
\vedette{\hypertarget{Ⓔʑala}{\papi{ ʑala}}}\markboth{ʑala}{}\classe{n}
\begin{définition}\fra application de glaise sur les murs pour les rendre plus lisse\end{définition}
\begin{définition}\cmn 在墙壁上涂上水泥使其光滑【敷墙壁】\end{définition}\end{entrée}

\begin{entrée}
\vedette{\hypertarget{Ⓔʑaŋɬa}{\papi{ ʑaŋɬa}}}\markboth{ʑaŋɬa}{}\classe{n}
\begin{définition}\fra silex que l'on place au milieu du champs\end{définition}
\begin{définition}\cmn 放在田地中间的燧石
\begin{déclaration} \étymologie{\papi{ʑiŋ.ɬa}}\end{déclaration}\end{définition}
\begin{relation-sémantique}\confer{
\hyperlink{ⒺqapiⒽ1}{\textit{ \papi{qapi1}}}
}\end{relation-sémantique}
\end{entrée}

\begin{entrée}
\vedette{\hypertarget{Ⓔʑaŋmu}{\papi{ ʑaŋmu}}}\markboth{ʑaŋmu}{}\classe{n}
\begin{définition}\fra le plus grand champs\end{définition}
\begin{définition}\cmn 最大的田地
\begin{déclaration} \étymologie{\papi{ʑiŋ.mo}}\end{déclaration}\end{définition}
\end{entrée}

\begin{entrée}
\vedette{\hypertarget{Ⓔʑaŋndza}{\papi{ ʑaŋndza}}}\markboth{ʑaŋndza}{}\classe{n}
\begin{définition}\fra festin\end{définition}
\begin{définition}\cmn 宴会\end{définition}
\begin{exemple}\jya ʑaŋndza chɤ-lɤt-nɯ\cmn 他们办了宴会\end{exemple}
\begin{exemple}\jya jiʑora ʑaŋndza thɯ-lɤt-i\cmn 我们办了宴会\end{exemple}\end{entrée}

\begin{entrée}
\vedette{\hypertarget{Ⓔʑaŋpjaʁ}{\papi{ ʑaŋpjaʁ}}}\markboth{ʑaŋpjaʁ}{}\classe{n}
\begin{définition}\fra outil pour faire cuire les momo\end{définition}
\begin{définition}\cmn 【烙片】用来炕馍馍的用具\end{définition}\end{entrée}

\begin{entrée}
\vedette{\hypertarget{Ⓔʑara}{\papi{ ʑara}}}\markboth{ʑara}{}\classe{pro}
\begin{définition}\fra eux\end{définition}
\begin{définition}\cmn 他们\end{définition}
\end{entrée}

\begin{entrée}
\vedette{\hypertarget{ⒺʑaʁⒽ2}{\papi{ ʑaʁ}}}\markboth{ʑaʁ}{}\homonyme{2}
\classe{n}
\begin{définition}\fra pellicule de graisse\end{définition}
\begin{définition}\cmn 浮油
\begin{déclaration} \étymologie{\papi{ʑag}}\end{déclaration}\end{définition}\end{entrée}

\begin{entrée}
\vedette{\hypertarget{ⒺʑaʁⒽ3}{\papi{ ʑaʁ}}}\markboth{ʑaʁ}{}\homonyme{3}
\classe{n}
\begin{définition}\fra jour\end{définition}
\begin{définition}\cmn 天
\begin{déclaration} \étymologie{\papi{ʑag}}\end{déclaration}\end{définition}
\begin{exemple}\jya ʑaʁ χsɤ-rʑaʁ\cmn 三天\end{exemple}
\begin{exemple}\jya ʑaʁ χsɯ-sŋi\cmn 三个白天\end{exemple}
\end{entrée}

\begin{entrée}
\vedette{\hypertarget{Ⓔʑatsa}{\papi{ ʑatsa}}}\markboth{ʑatsa}{}
\classe{n}
\begin{définition}\fra bientôt\end{définition}
\begin{définition}\cmn 快要\end{définition}
\begin{exemple}\jya dian ʑatsa arɕo ɲɯ-ŋu\cmn 快没有电了\end{exemple}\end{entrée}

\begin{entrée}
\vedette{\hypertarget{Ⓔʑaʑa}{\papi{ ʑaʑa}}}\markboth{ʑaʑa}{}\classe{n}\acception{1}
\begin{définition}\fra longtemps avant\end{définition}
\begin{définition}\cmn 早就\end{définition}\acception{2}
\begin{définition}\fra pendant longtemps\end{définition}
\begin{définition}\cmn 很久\end{définition}
\begin{exemple}\jya ʑaʑa ʑo jɤ-azɣɯt-a\cmn 我早就到了\end{exemple}
\begin{exemple}\jya ʑaʑa ʑo mɯ-jɤ-azɣɯt\cmn 他很久都没有来\end{exemple}\end{entrée}

\begin{entrée}
\vedette{\hypertarget{Ⓔʑɤn}{\papi{ ʑɤn}}}\markboth{ʑɤn}{}\classe{vs}
\begin{définition}\fra moins bon\end{définition}
\begin{définition}\cmn (比自己)差\end{définition}
\begin{exemple}\jya ɯ-kɤ-spa rkɯn tɕe, aʑo sɤznɤ ʑɤn\cmn 他会做的事情很少,他比我差\end{exemple}
\begin{relation-sémantique}\antonyme{
\hyperlink{Ⓔmna}{\textit{ \papi{mna}}}
}\end{relation-sémantique}\end{entrée}

\begin{entrée}
\vedette{\hypertarget{Ⓔʑɤni}{\papi{ ʑɤni}}}\markboth{ʑɤni}{}\classe{pro}
\begin{définition}\fra eux deux\end{définition}
\begin{définition}\cmn 他们俩\end{définition}
\end{entrée}

\begin{entrée}
\vedette{\hypertarget{Ⓔʑɤŋgɯz}{\papi{ ʑɤŋgɯz}}}\markboth{ʑɤŋgɯz}{}
\classe{adv}
\begin{définition}\fra l'un à l'autre\end{définition}
\begin{définition}\cmn 互相\end{définition}
\begin{exemple}\jya tɕiʑo ʑɤŋgɯz azɣɤʁrɯʁre-tɕi\cmn 我们俩互相尊重\end{exemple}
\begin{relation-sémantique}\confer{
\hyperlink{Ⓔɯ-ŋgɯ}{\textit{ \papi{ɯ-ŋgɯ}}}
}\end{relation-sémantique}\end{entrée}

\begin{entrée}
\vedette{\hypertarget{Ⓔʑɤwu}{\papi{ ʑɤwu}}}\markboth{ʑɤwu}{}\classe{n}
\begin{définition}\fra boiteux\end{définition}
\begin{définition}\cmn 跛子
\begin{déclaration} \étymologie{\papi{ʑa.bo}}\end{déclaration}\end{définition}
\begin{relation-sémantique}\synonyme{
\hyperlink{Ⓔɕkala}{\textit{ \papi{ɕkala}}}
}\end{relation-sémantique}
\begin{relation-sémantique}\confer{
\hyperlink{Ⓔaʑɤwu}{\textit{ \papi{aʑɤwu}}}
}\end{relation-sémantique}\end{entrée}

\begin{entrée}
\vedette{\hypertarget{Ⓔʑɤzdaŋ}{\papi{ ʑɤzdaŋ}}}\markboth{ʑɤzdaŋ}{}\classe{n}
\begin{définition}\fra envie, volonté de dépasser les autres\end{définition}
\begin{définition}\cmn 妒忌;有赶上别人的心
\begin{déclaration} \étymologie{\papi{ʑe.sdaŋ}}\end{déclaration}\end{définition}
\end{entrée}

\begin{entrée}
\vedette{\hypertarget{Ⓔʑdraŋʑdraŋ}{\papi{ ʑdraŋʑdraŋ}}}\markboth{ʑdraŋʑdraŋ}{}\classe{idph.2}
\begin{définition}\fra objet long et flexible\end{définition}
\begin{définition}\cmn 形容物体(衣服,软树枝等)又长又软的样子\end{définition}
\begin{exemple}\jya tɯmbri ʑdraŋʑdraŋ ʑo ɲɯ-ɤta\cmn 绳子放在那里,很凌乱的样子\end{exemple}\begin{sous-entrée}
\vedette{\hypertarget{}{\papi{ sɤʑdraŋlaŋ}}}\markboth{sɤʑdraŋlaŋ}{}\classe{vt}
\begin{définition}\fra secouer légèrement (objet long et flexible)\end{définition}
\begin{définition}\cmn 抖动(又长又软的东西)\end{définition}
\begin{relation-sémantique}\synonyme{
\hyperlink{Ⓔsɤɕtʂɯlɯɣ}{\textit{ \papi{sɤɕtʂɯlɯɣ}}}
}\end{relation-sémantique}
\begin{relation-sémantique}\synonyme{
\hyperlink{Ⓔsɤɕtʂaŋlaŋ}{\textit{ \papi{sɤɕtʂaŋlaŋ}}}
}\end{relation-sémantique}
\end{sous-entrée}\begin{sous-entrée}
\vedette{\hypertarget{}{\papi{ ʑdraŋnɤʑdraŋ}}}\markboth{ʑdraŋnɤʑdraŋ}{}\classe{idph.3}
\begin{définition}\cmn 地上有旧的衣服,他带走了\end{définition}
\begin{exemple}\jya tɯ-ŋgɤmbe ɯ-thoʁ ɲɯ-ɤta tɕe, ʑdraŋnɤʑdraŋ ka-tsɯm\end{exemple}
\end{sous-entrée}\end{entrée}

\begin{entrée}
\vedette{\hypertarget{Ⓔʑdrɤβʑdrɤβ}{\papi{ ʑdrɤβʑdrɤβ}}}\markboth{ʑdrɤβʑdrɤβ}{}\classe{idph.2}
\begin{définition}\fra objet long et mou\end{définition}
\begin{définition}\cmn 形容物体长而软的样子\end{définition}
\begin{exemple}\jya razmbe ʑdrɤβʑdrɤβ ʑo ɲɯ-ɴqoʁ\cmn 烂布条又脏又长地在那里挂着\end{exemple}
\begin{exemple}\jya razmbe ʑdrɤβʑdrɤβ ʑo ɲɯ-ɤta\cmn 烂布条又脏又长地放在那里\end{exemple}
\begin{exemple}\jya nɤ-χsɯmχsoz jɤ-sɯɣe ʑdrɤβʑdrɤβ ʑo ma-tɯ-ʑɣɤstu\cmn 你要打起精神来,不要无精打采的样子\end{exemple}
\begin{relation-sémantique}\confer{
\hyperlink{Ⓔɕthrɤβɕthrɤβ}{\textit{ \papi{ɕthrɤβɕthrɤβ}}}
}\end{relation-sémantique}\end{entrée}

\begin{entrée}
\vedette{\hypertarget{Ⓔʑdɯɣʑdɯɣ}{\papi{ ʑdɯɣʑdɯɣ}}}\markboth{ʑdɯɣʑdɯɣ}{}
\classe{idph.2}
\begin{définition}\fra compact, solide\end{définition}
\begin{définition}\cmn 形容物体紧凑或牢固\end{définition}
\begin{exemple}\jya nɤki tʂɤm ɲɯ-ɤrku tɕe, ʑdɯɣʑdɯɣ ʑo ɲɯ-ɤstu\cmn 板壁装得很紧凑\end{exemple}
\begin{exemple}\jya ɯ-ɕɣa ɯ-tɯ-pe kɯ ʑdɯɣʑdɯɣ ʑo ɲɯ-pa\cmn 他的牙齿很牢固\end{exemple}
\begin{exemple}\jya kɯm ʑdɯɣʑdɯɣ ʑo pjɤ-nɯ-sɤtsa\cmn 他把门锁得死死的\end{exemple}\end{entrée}

\begin{entrée}
\vedette{\hypertarget{Ⓔʑgaʁ}{\papi{ ʑgaʁ}}}\markboth{ʑgaʁ}{}\classe{adv}
\begin{définition}\fra tout juste\end{définition}
\begin{définition}\cmn 刚好\end{définition}
\begin{exemple}\jya jisŋi tɕe, tɯ-sla ʑgaʁ ʑo tu-tsu ŋu\cmn 到今天就刚好满一个月了\end{exemple}
\begin{exemple}\jya kɯtʂɤɣ ɯ-taʁ kɯβde pjɯ́-wɣ-ta tɕe, sqi ʑgaʁ ŋu\cmn 六加四正好等于十\end{exemple}
\end{entrée}

\begin{entrée}
\vedette{\hypertarget{Ⓔʑgɤβʑgɤβ}{\papi{ ʑgɤβʑgɤβ}}}\markboth{ʑgɤβʑgɤβ}{}\classe{idph.2}
\begin{définition}\fra grand, maigre et bossu\end{définition}
\begin{définition}\cmn 形容高、瘦而驼背的样子\end{définition}
\begin{relation-sémantique}\synonyme{
\hyperlink{Ⓔɕkɤɣɕkɤɣ}{\textit{ \papi{ɕkɤɣɕkɤɣ}}}
}\end{relation-sémantique}\end{entrée}

\begin{entrée}
\vedette{\hypertarget{Ⓔʑgrɤɣʑgrɤɣ}{\papi{ ʑgrɤɣʑgrɤɣ}}}\markboth{ʑgrɤɣʑgrɤɣ}{}
\classe{idph.2}
\begin{définition}\fra dur et froid (sensation lorsqu'on s'allonge sur le sol)\end{définition}
\begin{définition}\cmn 形容躺在地上又没有衣服盖的感觉,又冷又硬的感觉。\end{définition}
\begin{exemple}\jya ʑgrɤɣʑgrɤɣ ʑo ɲɯ-rŋgɯ\cmn 他躺在地上,感觉地面又硬又冷\end{exemple}
\begin{relation-sémantique}\confer{
\hyperlink{Ⓔnɯʑgrɤɣ}{\textit{ \papi{nɯʑgrɤɣ}}}
}\end{relation-sémantique}\end{entrée}

\begin{entrée}
\vedette{\hypertarget{Ⓔʑgrɯɣ}{\papi{ ʑgrɯɣ}}}\markboth{ʑgrɯɣ}{} (\variante{ʑgrɯ}) \classe{n}
\begin{définition}\fra certainement\end{définition}
\begin{définition}\cmn 一定\end{définition}\end{entrée}

\begin{entrée}
\vedette{\hypertarget{Ⓔʑɣɤβde}{\papi{ ʑɣɤβde}}}\markboth{ʑɣɤβde}{}\classe{vi}
\paradigme{\textit{dir :} \jya thɯ-}
\begin{définition}\ 
\begin{déclaration}\grammar{refl}\end{déclaration}\end{définition}
\begin{définition}\fra se suicider en se jetant à l'eau\end{définition}
\begin{définition}\cmn 投河自尽\end{définition}
\begin{exemple}\jya ma-ɕ-thɯ-tɯ-ʑɣɤβde ma nɤ-mu ɲɯ-ɤkhu\cmn 你不要去投河自尽,你母亲在叫你\end{exemple}
\begin{relation-sémantique}\confer{
\hyperlink{Ⓔβde}{\textit{ \papi{βde}}}
}\end{relation-sémantique}\end{entrée}

\begin{entrée}
\vedette{\hypertarget{Ⓔʑɣɤβʁum}{\papi{ ʑɣɤβʁum}}}\markboth{ʑɣɤβʁum}{}
\begin{relation-sémantique}\confer{
\hyperlink{Ⓔβʁum}{\textit{ \papi{βʁum}}}
}\end{relation-sémantique}\end{entrée}

\begin{entrée}
\vedette{\hypertarget{Ⓔʑɣɤβzɟɯr}{\papi{ ʑɣɤβzɟɯr}}}\markboth{ʑɣɤβzɟɯr}{}
\begin{relation-sémantique}\confer{
\hyperlink{Ⓔβzɟɯr}{\textit{ \papi{βzɟɯr}}}
}\end{relation-sémantique}
\end{entrée}

\begin{entrée}
\vedette{\hypertarget{Ⓔʑɣɤɕphɣo}{\papi{ ʑɣɤɕphɣo}}}\markboth{ʑɣɤɕphɣo}{}
\begin{relation-sémantique}\confer{
\hyperlink{Ⓔɕphɣo}{\textit{ \papi{ɕphɣo}}}
}\end{relation-sémantique}\end{entrée}

\begin{entrée}
\vedette{\hypertarget{Ⓔʑɣɤɕthɯz}{\papi{ ʑɣɤɕthɯz}}}\markboth{ʑɣɤɕthɯz}{}\classe{vi}
\paradigme{\textit{dir :} \jya kɤ-}
\begin{définition}\ 
\begin{déclaration}\grammar{refl}\end{déclaration}\end{définition}
\begin{définition}\fra dévoiler sa réelle identité\end{définition}
\begin{définition}\cmn 亮相\end{définition}
\begin{exemple}\jya fso tɕe ju-ɣi-a tɕe ɣɯ-ku-ʑɣɤɕthɯz-a\cmn 我明天来亲自亮相(给别人看我的真面目)\end{exemple}
\begin{relation-sémantique}\confer{
\hyperlink{Ⓔɕthɯz}{\textit{ \papi{ɕthɯz}}}
}\end{relation-sémantique}\end{entrée}

\begin{entrée}
\vedette{\hypertarget{Ⓔʑɣɤɕtʂat}{\papi{ ʑɣɤɕtʂat}}}\markboth{ʑɣɤɕtʂat}{}
\begin{relation-sémantique}\confer{
\hyperlink{Ⓔɕtʂat}{\textit{ \papi{ɕtʂat}}}
}\end{relation-sémantique}\end{entrée}

\begin{entrée}
\vedette{\hypertarget{Ⓔʑɣɤɕɯɣmu}{\papi{ ʑɣɤɕɯɣmu}}}\markboth{ʑɣɤɕɯɣmu}{}
\begin{relation-sémantique}\confer{
\hyperlink{Ⓔɕɯɣmu}{\textit{ \papi{ɕɯɣmu}}}
}\end{relation-sémantique}\end{entrée}

\begin{entrée}
\vedette{\hypertarget{Ⓔʑɣɤɕɯmbɣom}{\papi{ ʑɣɤɕɯmbɣom}}}\markboth{ʑɣɤɕɯmbɣom}{}
\begin{relation-sémantique}\confer{
\hyperlink{Ⓔɕɯmbɣom}{\textit{ \papi{ɕɯmbɣom}}}
}\end{relation-sémantique}\end{entrée}

\begin{entrée}
\vedette{\hypertarget{Ⓔʑɣɤɕɯnŋo}{\papi{ ʑɣɤɕɯnŋo}}}\markboth{ʑɣɤɕɯnŋo}{}
\begin{relation-sémantique}\confer{
\hyperlink{Ⓔɕɯnŋo}{\textit{ \papi{ɕɯnŋo}}}
}\end{relation-sémantique}\end{entrée}

\begin{entrée}
\vedette{\hypertarget{Ⓔʑɣɤɕɯrga}{\papi{ ʑɣɤɕɯrga}}}\markboth{ʑɣɤɕɯrga}{}
\begin{relation-sémantique}\confer{
\hyperlink{Ⓔɕɯrga}{\textit{ \papi{ɕɯrga}}}
}\end{relation-sémantique}\end{entrée}

\begin{entrée}
\vedette{\hypertarget{Ⓔʑɣɤfɕɤt}{\papi{ ʑɣɤfɕɤt}}}\markboth{ʑɣɤfɕɤt}{}
\begin{relation-sémantique}\confer{
\hyperlink{ⒺfɕɤtⒽ1}{\textit{ \papi{fɕɤt1}}}
}\end{relation-sémantique}\end{entrée}

\begin{entrée}
\vedette{\hypertarget{Ⓔʑɣɤfsraŋ}{\papi{ ʑɣɤfsraŋ}}}\markboth{ʑɣɤfsraŋ}{}
\begin{relation-sémantique}\confer{
\hyperlink{Ⓔfsraŋ}{\textit{ \papi{fsraŋ}}}
}\end{relation-sémantique}
\end{entrée}

\begin{entrée}
\vedette{\hypertarget{Ⓔʑɣɤfstɯn}{\papi{ ʑɣɤfstɯn}}}\markboth{ʑɣɤfstɯn}{}
\begin{relation-sémantique}\confer{
\hyperlink{Ⓔfstɯn}{\textit{ \papi{fstɯn}}}
}\end{relation-sémantique}\end{entrée}

\begin{entrée}
\vedette{\hypertarget{Ⓔʑɣɤɣɤβdi}{\papi{ ʑɣɤɣɤβdi}}}\markboth{ʑɣɤɣɤβdi}{}
\begin{relation-sémantique}\confer{
\hyperlink{Ⓔɣɤβdi}{\textit{ \papi{ɣɤβdi}}}
}\end{relation-sémantique}\end{entrée}

\begin{entrée}
\vedette{\hypertarget{Ⓔʑɣɤɣɤkhe}{\papi{ ʑɣɤɣɤkhe}}}\markboth{ʑɣɤɣɤkhe}{}
\begin{relation-sémantique}\confer{
\hyperlink{Ⓔkhe}{\textit{ \papi{khe}}}
}\end{relation-sémantique}\end{entrée}

\begin{entrée}
\vedette{\hypertarget{Ⓔʑɣɤɣɤla}{\papi{ ʑɣɤɣɤla}}}\markboth{ʑɣɤɣɤla}{}
\begin{relation-sémantique}\confer{
\hyperlink{Ⓔɣɤla}{\textit{ \papi{ɣɤla}}}
}\end{relation-sémantique}\end{entrée}

\begin{entrée}
\vedette{\hypertarget{Ⓔʑɣɤɣɤme}{\papi{ ʑɣɤɣɤme}}}\markboth{ʑɣɤɣɤme}{}
\begin{relation-sémantique}\confer{
\hyperlink{Ⓔɣɤme}{\textit{ \papi{ɣɤme}}}
}\end{relation-sémantique}
\end{entrée}

\begin{entrée}
\vedette{\hypertarget{Ⓔʑɣɤɣɤntaβ}{\papi{ ʑɣɤɣɤntaβ}}}\markboth{ʑɣɤɣɤntaβ}{}
\begin{relation-sémantique}\confer{
\hyperlink{Ⓔɣɤntaβ}{\textit{ \papi{ɣɤntaβ}}}
}\end{relation-sémantique}
\end{entrée}

\begin{entrée}
\vedette{\hypertarget{Ⓔʑɣɤɣɤŋgi}{\papi{ ʑɣɤɣɤŋgi}}}\markboth{ʑɣɤɣɤŋgi}{}
\begin{relation-sémantique}\confer{
\hyperlink{Ⓔɣɤŋgi}{\textit{ \papi{ɣɤŋgi}}}
}\end{relation-sémantique}\end{entrée}

\begin{entrée}
\vedette{\hypertarget{Ⓔʑɣɤɣɤrndi}{\papi{ ʑɣɤɣɤrndi}}}\markboth{ʑɣɤɣɤrndi}{}
\begin{relation-sémantique}\confer{
\hyperlink{Ⓔɣɤrndi}{\textit{ \papi{ɣɤrndi}}}
}\end{relation-sémantique}\end{entrée}

\begin{entrée}
\vedette{\hypertarget{Ⓔʑɣɤɣɤtɕa}{\papi{ ʑɣɤɣɤtɕa}}}\markboth{ʑɣɤɣɤtɕa}{}
\begin{relation-sémantique}\confer{
\hyperlink{Ⓔɣɤtɕa}{\textit{ \papi{ɣɤtɕa}}}
}\end{relation-sémantique}\end{entrée}

\begin{entrée}
\vedette{\hypertarget{Ⓔʑɣɤɣɤtɕɤt}{\papi{ ʑɣɤɣɤtɕɤt}}}\markboth{ʑɣɤɣɤtɕɤt}{}
\begin{relation-sémantique}\confer{
\hyperlink{Ⓔɣɤtɕɤt}{\textit{ \papi{ɣɤtɕɤt}}}
}\end{relation-sémantique}
\end{entrée}

\begin{entrée}
\vedette{\hypertarget{Ⓔʑɣɤɣɤʑo}{\papi{ ʑɣɤɣɤʑo}}}\markboth{ʑɣɤɣɤʑo}{}
\begin{relation-sémantique}\confer{
\hyperlink{ⒺʑoⒽ1}{\textit{ \papi{ʑo1}}}
}\end{relation-sémantique}\end{entrée}

\begin{entrée}
\vedette{\hypertarget{Ⓔʑɣɤkho}{\papi{ ʑɣɤkho}}}\markboth{ʑɣɤkho}{}
\begin{relation-sémantique}\confer{
\hyperlink{ⒺkhoⒽ1}{\textit{ \papi{kho1}}}
}\end{relation-sémantique}\end{entrée}

\begin{entrée}
\vedette{\hypertarget{Ⓔʑɣɤkro}{\papi{ ʑɣɤkro}}}\markboth{ʑɣɤkro}{}
\begin{relation-sémantique}\confer{
\hyperlink{Ⓔkro}{\textit{ \papi{kro}}}
}\end{relation-sémantique}\end{entrée}

\begin{entrée}
\vedette{\hypertarget{Ⓔʑɣɤlɤrko}{\papi{ ʑɣɤlɤrko}}}\markboth{ʑɣɤlɤrko}{} (\variante{zɣɤnɤrko}) 
\classe{vi}
\paradigme{\textit{dir :} \jya tɤ-}
\begin{définition}\ 
\begin{déclaration}\grammar{refl}\end{déclaration}\end{définition}
\begin{définition}\fra s'encourager soi-même, garder confiance\end{définition}
\begin{définition}\cmn 自己鼓励自己\end{définition}
\begin{exemple}\jya nɤʑo tɤ-ʑɣɤlɤrko, tɕe ʑa a-tɤ-tɯ-mna\cmn 你要坚持,就会早点痊愈\end{exemple}
\begin{relation-sémantique}\confer{
\hyperlink{ⒺnɤrkoⒽ1}{\textit{ \papi{nɤrko1}}}
}\end{relation-sémantique}\end{entrée}

\begin{entrée}
\vedette{\hypertarget{Ⓔʑɣɤmaŋlo}{\papi{ ʑɣɤmaŋlo}}}\markboth{ʑɣɤmaŋlo}{}
\begin{relation-sémantique}\confer{
\hyperlink{Ⓔmaŋlo}{\textit{ \papi{maŋlo}}}
}\end{relation-sémantique}\end{entrée}

\begin{entrée}
\vedette{\hypertarget{Ⓔʑɣɤmar}{\papi{ ʑɣɤmar}}}\markboth{ʑɣɤmar}{}
\begin{relation-sémantique}\confer{
\hyperlink{Ⓔmar}{\textit{ \papi{mar}}}
}\end{relation-sémantique}\end{entrée}

\begin{entrée}
\vedette{\hypertarget{Ⓔʑɣɤmdzoz}{\papi{ ʑɣɤmdzoz}}}\markboth{ʑɣɤmdzoz}{}\classe{vi}
\begin{définition}\fra avoir de l'amour propre\end{définition}
\begin{définition}\cmn 自重,注意自己的言行\end{définition}
\begin{exemple}\jya tɯrme sɤtɕha jɤ-kɯ-ɤri tɕe, tɯ-kɯ-ʑɣɤmdzoz ra\cmn 出远门的时候要会自重\end{exemple}
\begin{relation-sémantique}\confer{
\hyperlink{Ⓔmdzoz}{\textit{ \papi{mdzoz}}}
}\end{relation-sémantique}\end{entrée}

\begin{entrée}
\vedette{\hypertarget{Ⓔʑɣɤmgɯ}{\papi{ ʑɣɤmgɯ}}}\markboth{ʑɣɤmgɯ}{}
\begin{relation-sémantique}\confer{
\hyperlink{Ⓔmgɯ}{\textit{ \papi{mgɯ}}}
}\end{relation-sémantique}\end{entrée}

\begin{entrée}
\vedette{\hypertarget{Ⓔʑɣɤmɲo}{\papi{ ʑɣɤmɲo}}}\markboth{ʑɣɤmɲo}{}
\begin{relation-sémantique}\confer{
\hyperlink{ⒺmɲoⒽ1}{\textit{ \papi{mɲo1}}}
}\end{relation-sémantique}\end{entrée}

\begin{entrée}
\vedette{\hypertarget{Ⓔʑɣɤmphɯr}{\papi{ ʑɣɤmphɯr}}}\markboth{ʑɣɤmphɯr}{}
\begin{relation-sémantique}\confer{
\hyperlink{Ⓔmphɯr}{\textit{ \papi{mphɯr}}}
}\end{relation-sémantique}\end{entrée}

\begin{entrée}
\vedette{\hypertarget{Ⓔʑɣɤmto}{\papi{ ʑɣɤmto}}}\markboth{ʑɣɤmto}{}
\begin{relation-sémantique}\confer{
 \papi{mto}
}\end{relation-sémantique}\end{entrée}

\begin{entrée}
\vedette{\hypertarget{Ⓔʑɣɤnaχsoz}{\papi{ ʑɣɤnaχsoz}}}\markboth{ʑɣɤnaχsoz}{}
\begin{relation-sémantique}\confer{
\hyperlink{Ⓔnaχsoz}{\textit{ \papi{naχsoz}}}
}\end{relation-sémantique}\end{entrée}

\begin{entrée}
\vedette{\hypertarget{Ⓔʑɣɤnɤmpɕɤr}{\papi{ ʑɣɤnɤmpɕɤr}}}\markboth{ʑɣɤnɤmpɕɤr}{}
\begin{relation-sémantique}\confer{
\hyperlink{Ⓔnɤmpɕɤr}{\textit{ \papi{nɤmpɕɤr}}}
}\end{relation-sémantique}\end{entrée}

\begin{entrée}
\vedette{\hypertarget{Ⓔʑɣɤnɤmqe}{\papi{ ʑɣɤnɤmqe}}}\markboth{ʑɣɤnɤmqe}{}
\begin{relation-sémantique}\confer{
\hyperlink{Ⓔnɤmqe}{\textit{ \papi{nɤmqe}}}
}\end{relation-sémantique}
\end{entrée}

\begin{entrée}
\vedette{\hypertarget{Ⓔʑɣɤnɤmtshɤr}{\papi{ ʑɣɤnɤmtshɤr}}}\markboth{ʑɣɤnɤmtshɤr}{}
\begin{relation-sémantique}\confer{
\hyperlink{Ⓔnɤmtshɤr}{\textit{ \papi{nɤmtshɤr}}}
}\end{relation-sémantique}\end{entrée}

\begin{entrée}
\vedette{\hypertarget{Ⓔʑɣɤnɤndʐo}{\papi{ ʑɣɤnɤndʐo}}}\markboth{ʑɣɤnɤndʐo}{}
\begin{relation-sémantique}\confer{
\hyperlink{Ⓔnɤndʐo}{\textit{ \papi{nɤndʐo}}}
}\end{relation-sémantique}\end{entrée}

\begin{entrée}
\vedette{\hypertarget{Ⓔʑɣɤnɤrko}{\papi{ ʑɣɤnɤrko}}}\markboth{ʑɣɤnɤrko}{}
\begin{relation-sémantique}\confer{
\hyperlink{ⒺnɤrkoⒽ1}{\textit{ \papi{nɤrko1}}}
}\end{relation-sémantique}\end{entrée}

\begin{entrée}
\vedette{\hypertarget{Ⓔʑɣɤnɤstu}{\papi{ ʑɣɤnɤstu}}}\markboth{ʑɣɤnɤstu}{}
\begin{relation-sémantique}\confer{
\hyperlink{Ⓔnɤstu}{\textit{ \papi{nɤstu}}}
}\end{relation-sémantique}\end{entrée}

\begin{entrée}
\vedette{\hypertarget{Ⓔʑɣɤnɤtsɯ}{\papi{ ʑɣɤnɤtsɯ}}}\markboth{ʑɣɤnɤtsɯ}{}
\begin{relation-sémantique}\confer{
\hyperlink{Ⓔnɤtsɯ}{\textit{ \papi{nɤtsɯ}}}
}\end{relation-sémantique}\end{entrée}

\begin{entrée}
\vedette{\hypertarget{Ⓔʑɣɤnbaʁ}{\papi{ ʑɣɤnbaʁ}}}\markboth{ʑɣɤnbaʁ}{}
\begin{relation-sémantique}\confer{
 \papi{nbaʁ}
}\end{relation-sémantique}\end{entrée}

\begin{entrée}
\vedette{\hypertarget{Ⓔʑɣɤndzɯ}{\papi{ ʑɣɤndzɯ}}}\markboth{ʑɣɤndzɯ}{}
\begin{relation-sémantique}\confer{
\hyperlink{Ⓔndzɯ}{\textit{ \papi{ndzɯ}}}
}\end{relation-sémantique}\end{entrée}

\begin{entrée}
\vedette{\hypertarget{Ⓔʑɣɤntsɣe}{\papi{ ʑɣɤntsɣe}}}\markboth{ʑɣɤntsɣe}{}
\begin{relation-sémantique}\confer{
\hyperlink{Ⓔntsɣe}{\textit{ \papi{ntsɣe}}}
}\end{relation-sémantique}\end{entrée}

\begin{entrée}
\vedette{\hypertarget{Ⓔʑɣɤnɯβlu}{\papi{ ʑɣɤnɯβlu}}}\markboth{ʑɣɤnɯβlu}{}
\begin{relation-sémantique}\confer{
\hyperlink{Ⓔnɯβlu}{\textit{ \papi{nɯβlu}}}
}\end{relation-sémantique}\end{entrée}

\begin{entrée}
\vedette{\hypertarget{Ⓔʑɣɤnɯkhramba}{\papi{ ʑɣɤnɯkhramba}}}\markboth{ʑɣɤnɯkhramba}{}
\begin{relation-sémantique}\confer{
\hyperlink{Ⓔnɯkhramba}{\textit{ \papi{nɯkhramba}}}
}\end{relation-sémantique}\end{entrée}

\begin{entrée}
\vedette{\hypertarget{Ⓔʑɣɤnɯkon}{\papi{ ʑɣɤnɯkon}}}\markboth{ʑɣɤnɯkon}{}
\begin{relation-sémantique}\confer{
\hyperlink{Ⓔnɯkon}{\textit{ \papi{nɯkon}}}
}\end{relation-sémantique}\end{entrée}

\begin{entrée}
\vedette{\hypertarget{Ⓔʑɣɤnɯmbrɤpɯ}{\papi{ ʑɣɤnɯmbrɤpɯ}}}\markboth{ʑɣɤnɯmbrɤpɯ}{}
\begin{relation-sémantique}\confer{
\hyperlink{Ⓔnɯmbrɤpɯ}{\textit{ \papi{nɯmbrɤpɯ}}}
}\end{relation-sémantique}\end{entrée}

\begin{entrée}
\vedette{\hypertarget{Ⓔʑɣɤnɯmpa}{\papi{ ʑɣɤnɯmpa}}}\markboth{ʑɣɤnɯmpa}{}
\begin{relation-sémantique}\confer{
\hyperlink{Ⓔnɯmpa}{\textit{ \papi{nɯmpa}}}
}\end{relation-sémantique}
\end{entrée}

\begin{entrée}
\vedette{\hypertarget{Ⓔʑɣɤnɯsmɤn}{\papi{ ʑɣɤnɯsmɤn}}}\markboth{ʑɣɤnɯsmɤn}{}
\begin{relation-sémantique}\confer{
\hyperlink{Ⓔnɯsmɤn}{\textit{ \papi{nɯsmɤn}}}
}\end{relation-sémantique}\end{entrée}

\begin{entrée}
\vedette{\hypertarget{Ⓔʑɣɤnɯtʂawku}{\papi{ ʑɣɤnɯtʂawku}}}\markboth{ʑɣɤnɯtʂawku}{}
\begin{relation-sémantique}\confer{
\hyperlink{Ⓔnɯtʂawku}{\textit{ \papi{nɯtʂawku}}}
}\end{relation-sémantique}\end{entrée}

\begin{entrée}
\vedette{\hypertarget{ⒺʑɣɤpaⒽ1}{\papi{ ʑɣɤpa}}}\markboth{ʑɣɤpa}{}\homonyme{1}\classe{vi}
\paradigme{\textit{dir :} \jya tɤ-}
\begin{définition}\ 
\begin{déclaration}\grammar{refl}\end{déclaration}\end{définition}\acception{1}
\paradigme{\textit{construction :} \jya participe sujet}
\begin{définition}\fra faire semblant\end{définition}
\begin{définition}\cmn 装做\end{définition}
\begin{exemple}\jya pɯ-kɯ-maʁ ɲɯ-ʑɣɤpa\cmn 他装作不是他\end{exemple}
\begin{relation-sémantique}\synonyme{
\hyperlink{Ⓔnɯɕpɯz}{\textit{ \papi{nɯɕpɯz}}}
}\end{relation-sémantique}\acception{2}
\begin{définition}\fra être orgueilleux\end{définition}
\begin{définition}\cmn 傲慢\end{définition}
\begin{exemple}\jya jiɕqha nɯ kɯ-ʑɣɤpa ci ɲɯ-ŋu\cmn 他是个傲慢的人\end{exemple}
\begin{relation-sémantique}\synonyme{
\hyperlink{Ⓔznɤjpɯjpe}{\textit{ \papi{znɤjpɯjpe}}}
}\end{relation-sémantique}\acception{3}
\begin{définition}\fra s'appeler soi-même\end{définition}
\begin{définition}\cmn 自称\end{définition}
\begin{exemple}\jya ʁnɯ-xpa pjɤ-wxti qhe tɤ-pi to-ʑɣɤpa\cmn 因为他大两岁,所以自称哥哥\end{exemple}
\begin{relation-sémantique}\synonyme{
\hyperlink{Ⓔʑɣɤrtsi}{\textit{ \papi{ʑɣɤrtsi}}}
}\end{relation-sémantique}\end{entrée}

\begin{entrée}
\vedette{\hypertarget{ⒺʑɣɤpaⒽ2}{\papi{ ʑɣɤpa}}}\markboth{ʑɣɤpa}{}\homonyme{2}\begin{relation-sémantique}\confer{
\hyperlink{ⒺpaⒽ1}{\textit{ \papi{pa1}}}
}\end{relation-sémantique}\end{entrée}

\begin{entrée}
\vedette{\hypertarget{Ⓔʑɣɤpɣaʁ}{\papi{ ʑɣɤpɣaʁ}}}\markboth{ʑɣɤpɣaʁ}{}
\begin{relation-sémantique}\confer{
\hyperlink{Ⓔpɣaʁ}{\textit{ \papi{pɣaʁ}}}
}\end{relation-sémantique}\end{entrée}

\begin{entrée}
\vedette{\hypertarget{Ⓔʑɣɤqɤr}{\papi{ ʑɣɤqɤr}}}\markboth{ʑɣɤqɤr}{}\classe{vi}
\paradigme{\textit{dir :} \jya nɯ-}
\begin{définition}\fra solitaire\end{définition}
\begin{définition}\cmn 孤僻\end{définition}
\begin{relation-sémantique}\confer{
\hyperlink{Ⓔqɤr}{\textit{ \papi{qɤr}}}
}\end{relation-sémantique}\end{entrée}

\begin{entrée}
\vedette{\hypertarget{Ⓔʑɣɤraχtɕɤz}{\papi{ ʑɣɤraχtɕɤz}}}\markboth{ʑɣɤraχtɕɤz}{}
\begin{relation-sémantique}\confer{
\hyperlink{Ⓔraχtɕɤz}{\textit{ \papi{raχtɕɤz}}}
}\end{relation-sémantique}\end{entrée}

\begin{entrée}
\vedette{\hypertarget{Ⓔʑɣɤrɤβraʁ}{\papi{ ʑɣɤrɤβraʁ}}}\markboth{ʑɣɤrɤβraʁ}{}
\begin{relation-sémantique}\confer{
\hyperlink{Ⓔrɤβraʁ}{\textit{ \papi{rɤβraʁ}}}
}\end{relation-sémantique}\end{entrée}

\begin{entrée}
\vedette{\hypertarget{Ⓔʑɣɤrɤɕi}{\papi{ ʑɣɤrɤɕi}}}\markboth{ʑɣɤrɤɕi}{}
\begin{relation-sémantique}\confer{
\hyperlink{Ⓔrɤɕi}{\textit{ \papi{rɤɕi}}}
}\end{relation-sémantique}\end{entrée}

\begin{entrée}
\vedette{\hypertarget{Ⓔʑɣɤrɤphɯ}{\papi{ ʑɣɤrɤphɯ}}}\markboth{ʑɣɤrɤphɯ}{}\classe{vi}
\paradigme{\textit{dir :} \jya pɯ-}
\paradigme{\textit{dir :} \jya kɯ-}
\begin{définition}\fra mesurer ses forces\end{définition}
\begin{définition}\cmn 量力而行\end{définition}
\begin{exemple}\jya tɯʑo pjɯ-kɯ-ʑɣɤrɤphɯ ra\cmn 要量力而行\end{exemple}
\begin{exemple}\jya nɤ-mtɕhi ma-tɯ-ɣɤχɤm kɯ koŋla kɤ-ʑɣɤrɤphɯ\cmn 不要说大话,要量力而行\end{exemple}\end{entrée}

\begin{entrée}
\vedette{\hypertarget{Ⓔʑɣɤrku}{\papi{ ʑɣɤrku}}}\markboth{ʑɣɤrku}{}
\begin{relation-sémantique}\confer{
\hyperlink{Ⓔrku}{\textit{ \papi{rku}}}
}\end{relation-sémantique}\end{entrée}

\begin{entrée}
\vedette{\hypertarget{Ⓔʑɣɤrpu}{\papi{ ʑɣɤrpu}}}\markboth{ʑɣɤrpu}{}
\begin{relation-sémantique}\confer{
\hyperlink{Ⓔrpu}{\textit{ \papi{rpu}}}
}\end{relation-sémantique}\end{entrée}

\begin{entrée}
\vedette{\hypertarget{Ⓔʑɣɤrtsi}{\papi{ ʑɣɤrtsi}}}\markboth{ʑɣɤrtsi}{}
\begin{relation-sémantique}\confer{
\hyperlink{Ⓔrtsi}{\textit{ \papi{rtsi}}}
}\end{relation-sémantique}\end{entrée}

\begin{entrée}
\vedette{\hypertarget{Ⓔʑɣɤrɯxtuxti}{\papi{ ʑɣɤrɯxtuxti}}}\markboth{ʑɣɤrɯxtuxti}{}
\begin{relation-sémantique}\confer{
\hyperlink{Ⓔrɯxtuxti}{\textit{ \papi{rɯxtuxti}}}
}\end{relation-sémantique}\end{entrée}

\begin{entrée}
\vedette{\hypertarget{Ⓔʑɣɤrʑɣɤr}{\papi{ ʑɣɤrʑɣɤr}}}\markboth{ʑɣɤrʑɣɤr}{}\classe{idph.2}
\begin{définition}\fra avoir des brins qui dépassent (dans une touffe)\end{définition}
\begin{définition}\cmn 形容一束东西当中,有一两根凸出来,显得不整齐的样子\end{définition}
\begin{exemple}\jya tu-ro ʑɣɤrʑɣɤr ʑo ɲɯ-ŋu\cmn 有(一两根)凸出来\end{exemple}
\begin{relation-sémantique}\confer{
\hyperlink{Ⓔʑɣɤʑɣɤt}{\textit{ \papi{ʑɣɤʑɣɤt}}}
}\end{relation-sémantique}\end{entrée}

\begin{entrée}
\vedette{\hypertarget{Ⓔʑɣɤʁmɯɣ}{\papi{ ʑɣɤʁmɯɣ}}}\markboth{ʑɣɤʁmɯɣ}{}
\begin{relation-sémantique}\confer{
\hyperlink{Ⓔʁmɯɣ}{\textit{ \papi{ʁmɯɣ}}}
}\end{relation-sémantique}\end{entrée}

\begin{entrée}
\vedette{\hypertarget{Ⓔʑɣɤsaʁjɤr}{\papi{ ʑɣɤsaʁjɤr}}}\markboth{ʑɣɤsaʁjɤr}{}
\begin{relation-sémantique}\confer{
\hyperlink{Ⓔsaʁjɤr}{\textit{ \papi{saʁjɤr}}}
}\end{relation-sémantique}
\end{entrée}

\begin{entrée}
\vedette{\hypertarget{Ⓔʑɣɤsɤɕke}{\papi{ ʑɣɤsɤɕke}}}\markboth{ʑɣɤsɤɕke}{}\classe{vi}
\begin{définition}\ 
\begin{déclaration}\grammar{refl}\end{déclaration}\end{définition}
\begin{définition}\fra se brûler\end{définition}
\begin{définition}\cmn 烫到自己\end{définition}
\begin{exemple}\jya ma-pɯ-tɯ-ʑɣɤsɤɕke\cmn 你不要烫到自己\end{exemple}
\begin{relation-sémantique}\confer{
\hyperlink{Ⓔɕke}{\textit{ \papi{ɕke}}}
}\end{relation-sémantique}
\begin{relation-sémantique}\confer{
\hyperlink{ⒺsɤɕkeⒽ1}{\textit{ \papi{sɤɕke}}}
}\end{relation-sémantique}\end{entrée}

\begin{entrée}
\vedette{\hypertarget{Ⓔʑɣɤsɤfɕu}{\papi{ ʑɣɤsɤfɕu}}}\markboth{ʑɣɤsɤfɕu}{}
\begin{relation-sémantique}\confer{
\hyperlink{Ⓔafɕu}{\textit{ \papi{afɕu}}}
}\end{relation-sémantique}\end{entrée}

\begin{entrée}
\vedette{\hypertarget{Ⓔʑɣɤsɤjɤr}{\papi{ ʑɣɤsɤjɤr}}}\markboth{ʑɣɤsɤjɤr}{}
\begin{relation-sémantique}\confer{
\hyperlink{Ⓔajɤr}{\textit{ \papi{ajɤr}}}
}\end{relation-sémantique}\end{entrée}

\begin{entrée}
\vedette{\hypertarget{Ⓔʑɣɤsɤnbaʁ}{\papi{ ʑɣɤsɤnbaʁ}}}\markboth{ʑɣɤsɤnbaʁ}{}
\begin{relation-sémantique}\confer{
\hyperlink{Ⓔanbaʁ}{\textit{ \papi{anbaʁ}}}
}\end{relation-sémantique}\end{entrée}

\begin{entrée}
\vedette{\hypertarget{Ⓔʑɣɤsɤɲɟoʁ}{\papi{ ʑɣɤsɤɲɟoʁ}}}\markboth{ʑɣɤsɤɲɟoʁ}{}
\begin{relation-sémantique}\confer{
\hyperlink{Ⓔɲɟoʁ}{\textit{ \papi{ɲɟoʁ}}}
}\end{relation-sémantique}\end{entrée}

\begin{entrée}
\vedette{\hypertarget{Ⓔʑɣɤsɤpɣaʁsci}{\papi{ ʑɣɤsɤpɣaʁsci}}}\markboth{ʑɣɤsɤpɣaʁsci}{}
\begin{relation-sémantique}\confer{
\hyperlink{Ⓔapɣaʁsci}{\textit{ \papi{apɣaʁsci}}}
}\end{relation-sémantique}\end{entrée}

\begin{entrée}
\vedette{\hypertarget{Ⓔʑɣɤsɤri}{\papi{ ʑɣɤsɤri}}}\markboth{ʑɣɤsɤri}{}
\begin{relation-sémantique}\confer{
\hyperlink{Ⓔsɤri}{\textit{ \papi{sɤri}}}
}\end{relation-sémantique}\end{entrée}

\begin{entrée}
\vedette{\hypertarget{Ⓔʑɣɤsɤrmbat}{\papi{ ʑɣɤsɤrmbat}}}\markboth{ʑɣɤsɤrmbat}{}
\begin{relation-sémantique}\confer{
\hyperlink{Ⓔarmbat}{\textit{ \papi{armbat}}}
}\end{relation-sémantique}\end{entrée}

\begin{entrée}
\vedette{\hypertarget{Ⓔʑɣɤsɤrmi}{\papi{ ʑɣɤsɤrmi}}}\markboth{ʑɣɤsɤrmi}{}
\begin{relation-sémantique}\confer{
\hyperlink{Ⓔsɤrmi}{\textit{ \papi{sɤrmi}}}
}\end{relation-sémantique}\end{entrée}

\begin{entrée}
\vedette{\hypertarget{Ⓔʑɣɤsɤrqhi}{\papi{ ʑɣɤsɤrqhi}}}\markboth{ʑɣɤsɤrqhi}{}
\begin{relation-sémantique}\confer{
\hyperlink{Ⓔarqhi}{\textit{ \papi{arqhi}}}
}\end{relation-sémantique}\end{entrée}

\begin{entrée}
\vedette{\hypertarget{Ⓔʑɣɤsɤrʁɯrʁu}{\papi{ ʑɣɤsɤrʁɯrʁu}}}\markboth{ʑɣɤsɤrʁɯrʁu}{}
\begin{relation-sémantique}\confer{
\hyperlink{Ⓔarʁɯrʁu}{\textit{ \papi{arʁɯrʁu}}}
}\end{relation-sémantique}
\end{entrée}

\begin{entrée}
\vedette{\hypertarget{Ⓔʑɣɤsɤstɤko}{\papi{ ʑɣɤsɤstɤko}}}\markboth{ʑɣɤsɤstɤko}{}
\begin{relation-sémantique}\confer{
\hyperlink{Ⓔsɤstɤko}{\textit{ \papi{sɤstɤko}}}
}\end{relation-sémantique}
\end{entrée}

\begin{entrée}
\vedette{\hypertarget{Ⓔʑɣɤsɤtsa}{\papi{ ʑɣɤsɤtsa}}}\markboth{ʑɣɤsɤtsa}{}
\begin{relation-sémantique}\confer{
\hyperlink{Ⓔatsa}{\textit{ \papi{atsa}}}
}\end{relation-sémantique}\end{entrée}

\begin{entrée}
\vedette{\hypertarget{Ⓔʑɣɤsɤtɯɣ}{\papi{ ʑɣɤsɤtɯɣ}}}\markboth{ʑɣɤsɤtɯɣ}{}
\begin{relation-sémantique}\confer{
\hyperlink{Ⓔatɯɣ}{\textit{ \papi{atɯɣ}}}
}\end{relation-sémantique}\end{entrée}

\begin{entrée}
\vedette{\hypertarget{Ⓔʑɣɤsɤzɣɯt}{\papi{ ʑɣɤsɤzɣɯt}}}\markboth{ʑɣɤsɤzɣɯt}{}
\begin{relation-sémantique}\confer{
\hyperlink{Ⓔsɤzɣɯt}{\textit{ \papi{sɤzɣɯt}}}
}\end{relation-sémantique}\end{entrée}

\begin{entrée}
\vedette{\hypertarget{Ⓔʑɣɤsprɯl}{\papi{ ʑɣɤsprɯl}}}\markboth{ʑɣɤsprɯl}{}\classe{vi}
\paradigme{\textit{dir :} \jya nɯ-}
\begin{définition}\ 
\begin{déclaration}\grammar{refl}\end{déclaration}\end{définition}
\begin{définition}\fra se déguiser, se transformer\end{définition}
\begin{définition}\cmn 装扮成,转变成
\begin{déclaration} \étymologie{\papi{sprul}}\end{déclaration}\end{définition}
\begin{exemple}\jya ɬɤndʐi nɯ tɯrme ɲɯ-ʑɣɤsprɯl ŋgrɤl\cmn 鬼会把自己变成人\end{exemple}\end{entrée}

\begin{entrée}
\vedette{\hypertarget{Ⓔʑɣɤstu}{\papi{ ʑɣɤstu}}}\markboth{ʑɣɤstu}{}
\classe{vi}
\paradigme{\textit{dir :} \jya tɤ-}
\begin{définition}\ 
\begin{déclaration}\grammar{refl}\end{déclaration}\end{définition}
\begin{définition}\fra faire en sorte de devenir ainsi\end{définition}
\begin{définition}\cmn 使自己变成那样
\begin{déclaration}\use{同状貌词连用}\end{déclaration}\end{définition}
\begin{exemple}\jya tɕhɣaʁtɕhɣaʁ ʑo tɤ-ʑɣɤstu\end{exemple}
\begin{exemple}\jya phoʁphoʁ to-ʑɣɤstu\cmn 他使自己变得很干净\end{exemple}
\begin{relation-sémantique}\confer{
\hyperlink{ⒺstuⒽ1}{\textit{ \papi{stu}}}
}\end{relation-sémantique}\end{entrée}

\begin{entrée}
\vedette{\hypertarget{Ⓔʑɣɤsɯβde}{\papi{ ʑɣɤsɯβde}}}\markboth{ʑɣɤsɯβde}{}
\classe{vi}
\paradigme{\textit{dir :} \jya pɯ-}
\begin{définition}\ 
\begin{déclaration}\grammar{refl}\end{déclaration}
\begin{déclaration}\grammar{caus}\end{déclaration}\end{définition}
\begin{définition}\fra se faire jeter\end{définition}
\begin{définition}\cmn 被摔下来\end{définition}
\begin{exemple}\jya mbro to-nɯmbrɤpɯ tɕe pjɤ-ʑɣɤsɯβde\cmn 他骑了马,被摔下来了\end{exemple}
\begin{relation-sémantique}\confer{
\hyperlink{Ⓔβde}{\textit{ \papi{βde}}}
}\end{relation-sémantique}\end{entrée}

\begin{entrée}
\vedette{\hypertarget{Ⓔʑɣɤsɯβʁa}{\papi{ ʑɣɤsɯβʁa}}}\markboth{ʑɣɤsɯβʁa}{}
\begin{relation-sémantique}\confer{
\hyperlink{Ⓔβʁa}{\textit{ \papi{βʁa}}}
}\end{relation-sémantique}\end{entrée}

\begin{entrée}
\vedette{\hypertarget{Ⓔʑɣɤsɯβzi}{\papi{ ʑɣɤsɯβzi}}}\markboth{ʑɣɤsɯβzi}{}
\begin{relation-sémantique}\confer{
\hyperlink{Ⓔβzi}{\textit{ \papi{βzi}}}
}\end{relation-sémantique}\end{entrée}

\begin{entrée}
\vedette{\hypertarget{Ⓔʑɣɤsɯɕqhlɤt}{\papi{ ʑɣɤsɯɕqhlɤt}}}\markboth{ʑɣɤsɯɕqhlɤt}{}
\begin{relation-sémantique}\confer{
\hyperlink{Ⓔɕqhlɤt}{\textit{ \papi{ɕqhlɤt}}}
}\end{relation-sémantique}\end{entrée}

\begin{entrée}
\vedette{\hypertarget{Ⓔʑɣɤsɯɕqraʁ}{\papi{ ʑɣɤsɯɕqraʁ}}}\markboth{ʑɣɤsɯɕqraʁ}{}
\begin{relation-sémantique}\confer{
\hyperlink{Ⓔɕqraʁ}{\textit{ \papi{ɕqraʁ}}}
}\end{relation-sémantique}\end{entrée}

\begin{entrée}
\vedette{\hypertarget{Ⓔʑɣɤsɯfsaŋ}{\papi{ ʑɣɤsɯfsaŋ}}}\markboth{ʑɣɤsɯfsaŋ}{}
\begin{relation-sémantique}\confer{
\hyperlink{Ⓔsɯfsaŋ}{\textit{ \papi{sɯfsaŋ}}}
}\end{relation-sémantique}\end{entrée}

\begin{entrée}
\vedette{\hypertarget{Ⓔʑɣɤsɯɣlɯɣ}{\papi{ ʑɣɤsɯɣlɯɣ}}}\markboth{ʑɣɤsɯɣlɯɣ}{}
\begin{relation-sémantique}\confer{
\hyperlink{Ⓔlɯɣ}{\textit{ \papi{lɯɣ}}}
}\end{relation-sémantique}\end{entrée}

\begin{entrée}
\vedette{\hypertarget{Ⓔʑɣɤsɯɣɲaʁ}{\papi{ ʑɣɤsɯɣɲaʁ}}}\markboth{ʑɣɤsɯɣɲaʁ}{}
\begin{relation-sémantique}\confer{
\hyperlink{Ⓔsɯɣɲaʁ}{\textit{ \papi{sɯɣɲaʁ}}}
}\end{relation-sémantique}\end{entrée}

\begin{entrée}
\vedette{\hypertarget{Ⓔʑɣɤsɯɣʑi}{\papi{ ʑɣɤsɯɣʑi}}}\markboth{ʑɣɤsɯɣʑi}{}
\begin{relation-sémantique}\confer{
\hyperlink{Ⓔʑi}{\textit{ \papi{ʑi}}}
}\end{relation-sémantique}\end{entrée}

\begin{entrée}
\vedette{\hypertarget{Ⓔʑɣɤsɯmphɯr}{\papi{ ʑɣɤsɯmphɯr}}}\markboth{ʑɣɤsɯmphɯr}{}
\begin{relation-sémantique}\confer{
\hyperlink{Ⓔmphɯr}{\textit{ \papi{mphɯr}}}
}\end{relation-sémantique}\end{entrée}

\begin{entrée}
\vedette{\hypertarget{Ⓔʑɣɤsɯmto}{\papi{ ʑɣɤsɯmto}}}\markboth{ʑɣɤsɯmto}{}
\begin{relation-sémantique}\confer{
 \papi{mto}
}\end{relation-sémantique}
\end{entrée}

\begin{entrée}
\vedette{\hypertarget{Ⓔʑɣɤsɯndo}{\papi{ ʑɣɤsɯndo}}}\markboth{ʑɣɤsɯndo}{}
\begin{relation-sémantique}\confer{
\hyperlink{Ⓔndo}{\textit{ \papi{ndo}}}
}\end{relation-sémantique}
\end{entrée}

\begin{entrée}
\vedette{\hypertarget{Ⓔʑɣɤsɯndo}{\papi{ ʑɣɤsɯndo}}}\markboth{ʑɣɤsɯndo}{}
\begin{relation-sémantique}\confer{
\hyperlink{Ⓔndo}{\textit{ \papi{ndo}}}
}\end{relation-sémantique}\end{entrée}

\begin{entrée}
\vedette{\hypertarget{Ⓔʑɣɤsɯntshɤβ}{\papi{ ʑɣɤsɯntshɤβ}}}\markboth{ʑɣɤsɯntshɤβ}{}
\begin{relation-sémantique}\confer{
\hyperlink{Ⓔntshɤβ}{\textit{ \papi{ntshɤβ}}}
}\end{relation-sémantique}\end{entrée}

\begin{entrée}
\vedette{\hypertarget{Ⓔʑɣɤsɯrku}{\papi{ ʑɣɤsɯrku}}}\markboth{ʑɣɤsɯrku}{}
\begin{relation-sémantique}\confer{
\hyperlink{Ⓔrku}{\textit{ \papi{rku}}}
}\end{relation-sémantique}\end{entrée}

\begin{entrée}
\vedette{\hypertarget{Ⓔʑɣɤsɯrtoʁ}{\papi{ ʑɣɤsɯrtoʁ}}}\markboth{ʑɣɤsɯrtoʁ}{}
\begin{relation-sémantique}\confer{
\hyperlink{Ⓔrtoʁ}{\textit{ \papi{rtoʁ}}}
}\end{relation-sémantique}\end{entrée}

\begin{entrée}
\vedette{\hypertarget{Ⓔʑɣɤsɯsat}{\papi{ ʑɣɤsɯsat}}}\markboth{ʑɣɤsɯsat}{}
\begin{relation-sémantique}\confer{
\hyperlink{Ⓔsat}{\textit{ \papi{sat}}}
}\end{relation-sémantique}\end{entrée}

\begin{entrée}
\vedette{\hypertarget{Ⓔʑɣɤsɯsŋaʁ}{\papi{ ʑɣɤsɯsŋaʁ}}}\markboth{ʑɣɤsɯsŋaʁ}{}
\begin{relation-sémantique}\confer{
\hyperlink{ⒺsŋaʁⒽ1}{\textit{ \papi{sŋaʁ1}}}
}\end{relation-sémantique}\end{entrée}

\begin{entrée}
\vedette{\hypertarget{Ⓔʑɣɤsɯxɕɤt}{\papi{ ʑɣɤsɯxɕɤt}}}\markboth{ʑɣɤsɯxɕɤt}{}
\begin{relation-sémantique}\confer{
\hyperlink{Ⓔsɯxɕɤt}{\textit{ \papi{sɯxɕɤt}}}
}\end{relation-sémantique}\end{entrée}

\begin{entrée}
\vedette{\hypertarget{Ⓔʑɣɤsɯxtshu}{\papi{ ʑɣɤsɯxtshu}}}\markboth{ʑɣɤsɯxtshu}{}
\begin{relation-sémantique}\confer{
\hyperlink{Ⓔtshu}{\textit{ \papi{tshu}}}
}\end{relation-sémantique}\end{entrée}

\begin{entrée}
\vedette{\hypertarget{Ⓔʑɣɤsɯxtso}{\papi{ ʑɣɤsɯxtso}}}\markboth{ʑɣɤsɯxtso}{}
\begin{relation-sémantique}\confer{
\hyperlink{Ⓔtso}{\textit{ \papi{tso}}}
}\end{relation-sémantique}\end{entrée}

\begin{entrée}
\vedette{\hypertarget{Ⓔʑɣɤsɯxtɯɣ}{\papi{ ʑɣɤsɯxtɯɣ}}}\markboth{ʑɣɤsɯxtɯɣ}{}
\classe{vi}
\paradigme{\textit{dir :} \jya \_}
\begin{définition}\ 
\begin{déclaration}\grammar{refl}\end{déclaration}\end{définition}
\begin{définition}\fra se rapprocher et entrer en contact avec\end{définition}
\begin{définition}\cmn 靠拢\end{définition}
\begin{exemple}\jya ma-thɯ-tɯ-ʑɣɤsɯxtɯɣ ma nɤ-ŋga sɯ-pɣi\cmn 你不要靠拢,会把你衣服弄脏\end{exemple}
\begin{relation-sémantique}\synonyme{
\hyperlink{Ⓔʑɣɤχtɤt}{\textit{ \papi{ʑɣɤχtɤt}}}
}\end{relation-sémantique}\end{entrée}

\begin{entrée}
\vedette{\hypertarget{Ⓔʑɣɤsɯχsu}{\papi{ ʑɣɤsɯχsu}}}\markboth{ʑɣɤsɯχsu}{}
\begin{relation-sémantique}\confer{
\hyperlink{Ⓔχsu}{\textit{ \papi{χsu}}}
}\end{relation-sémantique}
\end{entrée}

\begin{entrée}
\vedette{\hypertarget{Ⓔʑɣɤsɯzdɯɣ}{\papi{ ʑɣɤsɯzdɯɣ}}}\markboth{ʑɣɤsɯzdɯɣ}{}
\begin{relation-sémantique}\confer{
\hyperlink{Ⓔsɯzdɯɣ}{\textit{ \papi{sɯzdɯɣ}}}
}\end{relation-sémantique}\end{entrée}

\begin{entrée}
\vedette{\hypertarget{Ⓔʑɣɤta}{\papi{ ʑɣɤta}}}\markboth{ʑɣɤta}{}\classe{vi}\acception{1}
\paradigme{\textit{dir :} \jya \_}
\begin{définition}\ 
\begin{déclaration}\grammar{refl}\end{déclaration}\end{définition}
\begin{définition}\fra s’adosser à, s'appuyer\end{définition}
\begin{définition}\cmn 靠\end{définition}
\begin{exemple}\jya ɯ-zda ɯ-taʁ pjɤ-ʑɣɤta\cmn 他躺在别人的身上了\end{exemple}
\begin{exemple}\jya aʑo ɲɤ-ɲat-a tɕe, ɯ-taʁ kɤ-ʑɣɤta-a\cmn 我很累,所以靠在他身上\end{exemple}\acception{2}
\paradigme{\textit{dir :} \jya nɯ-}
\begin{définition}\fra rester et ne pas vouloir partir\end{définition}
\begin{définition}\cmn 留在别人家里不肯走\end{définition}
\begin{exemple}\jya tɯrme ɯ-kha ɲɤ-ʑɣɤta\cmn 他留在人家的屋子里不肯走\end{exemple}
\begin{exemple}\jya ma-nɯ-tɯ-ʑɣɤta kɯ nɯɕe-tɕi\cmn 你不要待在这里,我们走吧\end{exemple}
\begin{relation-sémantique}\synonyme{
 \papi{zɣɤχtɤt}
}\end{relation-sémantique}
\begin{relation-sémantique}\confer{
\hyperlink{Ⓔta}{\textit{ \papi{ta}}}
}\end{relation-sémantique}\end{entrée}

\begin{entrée}
\vedette{\hypertarget{Ⓔʑɣɤtshi}{\papi{ ʑɣɤtshi}}}\markboth{ʑɣɤtshi}{}\classe{vi}
\paradigme{\textit{dir :} \jya tɤ-}
\begin{définition}\ 
\begin{déclaration}\grammar{refl}\end{déclaration}\end{définition}
\begin{définition}\fra se pendre\end{définition}
\begin{définition}\cmn 上吊\end{définition}
\begin{exemple}\jya to-ʑɣɤtshi\cmn 他上吊自尽了\end{exemple}
\begin{relation-sémantique}\confer{
\hyperlink{ⒺtshiⒽ2}{\textit{ \papi{tshi2}}}
}\end{relation-sémantique}\end{entrée}

\begin{entrée}
\vedette{\hypertarget{Ⓔʑɣɤtʂaβ}{\papi{ ʑɣɤtʂaβ}}}\markboth{ʑɣɤtʂaβ}{}
\begin{relation-sémantique}\confer{
\hyperlink{Ⓔtʂaβ}{\textit{ \papi{tʂaβ}}}
}\end{relation-sémantique}\end{entrée}

\begin{entrée}
\vedette{\hypertarget{Ⓔʑɣɤwum}{\papi{ ʑɣɤwum}}}\markboth{ʑɣɤwum}{}
\begin{relation-sémantique}\confer{
\hyperlink{Ⓔwum}{\textit{ \papi{wum}}}
}\end{relation-sémantique}\end{entrée}

\begin{entrée}
\vedette{\hypertarget{Ⓔʑɣɤxthom}{\papi{ ʑɣɤxthom}}}\markboth{ʑɣɤxthom}{}
\begin{relation-sémantique}\confer{
\hyperlink{Ⓔxthom}{\textit{ \papi{xthom}}}
}\end{relation-sémantique}\end{entrée}

\begin{entrée}
\vedette{\hypertarget{Ⓔʑɣɤχpjɤt}{\papi{ ʑɣɤχpjɤt}}}\markboth{ʑɣɤχpjɤt}{}
\begin{relation-sémantique}\confer{
\hyperlink{Ⓔχpjɤt}{\textit{ \papi{χpjɤt}}}
}\end{relation-sémantique}
\end{entrée}

\begin{entrée}
\vedette{\hypertarget{Ⓔʑɣɤχtɤt}{\papi{ ʑɣɤχtɤt}}}\markboth{ʑɣɤχtɤt}{}
\begin{relation-sémantique}\confer{
\hyperlink{Ⓔχtɤt}{\textit{ \papi{χtɤt}}}
}\end{relation-sémantique}
\end{entrée}

\begin{entrée}
\vedette{\hypertarget{Ⓔʑɣɤχtɕi}{\papi{ ʑɣɤχtɕi}}}\markboth{ʑɣɤχtɕi}{}
\begin{relation-sémantique}\confer{
\hyperlink{Ⓔχtɕi}{\textit{ \papi{χtɕi}}}
}\end{relation-sémantique}\end{entrée}

\begin{entrée}
\vedette{\hypertarget{Ⓔʑɣɤznɯɲɤmkhe}{\papi{ ʑɣɤznɯɲɤmkhe}}}\markboth{ʑɣɤznɯɲɤmkhe}{}
\begin{relation-sémantique}\confer{
\hyperlink{Ⓔnɯɲɤmkhe}{\textit{ \papi{nɯɲɤmkhe}}}
}\end{relation-sémantique}\end{entrée}

\begin{entrée}
\vedette{\hypertarget{Ⓔʑɣɤʑɣɤt}{\papi{ ʑɣɤʑɣɤt}}}\markboth{ʑɣɤʑɣɤt}{}\classe{idph.2}
\begin{définition}\fra avoir des brins qui dépassent (dans une touffe)\end{définition}
\begin{définition}\cmn 形容一束东西当中,有一两根凸出来,显得不整齐的样子\end{définition}
\begin{exemple}\jya kuxtɕo ɯ-taʁ si nɯ ɲɯ-ro ʑɣɤʑɣɤt ʑo\cmn 背篼里装的柴有(一根两根)凸出来\end{exemple}
\begin{relation-sémantique}\confer{
\hyperlink{Ⓔʑɣɤrʑɣɤr}{\textit{ \papi{ʑɣɤrʑɣɤr}}}
}\end{relation-sémantique}\end{entrée}

\begin{entrée}
\vedette{\hypertarget{Ⓔʑi}{\papi{ ʑi}}}\markboth{ʑi}{}
\classe{vi}
\paradigme{\textit{dir :} \jya nɯ-}
\paradigme{\textit{dir :} \jya pɯ-}
\paradigme{\textit{dir :} \jya thɯ-}
\begin{définition}\fra se résorber, se calmer\end{définition}
\begin{définition}\cmn 平静下来;减轻;消肿
\begin{déclaration} \étymologie{\papi{ʑi}}\end{déclaration}\end{définition}
\begin{exemple}\jya ɯ-kɯ-mŋɤm ɲɤ-ʑi\cmn 他的痛减轻了\end{exemple}
\begin{exemple}\jya ɯ-mbrɯ ɲɤ-ʑi\cmn 他的气消了\end{exemple}
\begin{exemple}\jya a-mtshi kɯ-mŋɤm daldaltsɯtsa ʑo nɯ-ʑi\cmn 我的胃疼(肝)慢慢地减轻了\end{exemple}
\begin{exemple}\jya tʂha ku-tshi-a qhe ɲɯ-ʑi ɕti\cmn 喝了茶就会好一点(解渴,或者不再打瞌睡了)\end{exemple}
\begin{exemple}\jya tɯ-mɯ ɲɤ-ʑi\cmn 雨停了\end{exemple}
\begin{exemple}\jya qale ɲɤ-ʑi\cmn 风停了\end{exemple}
\begin{exemple}\jya ɯ-tɯ-ɣmbɤβ pjɤ-ʑi\cmn 他的脓肿消肿了\end{exemple}\begin{sous-entrée}
\vedette{\hypertarget{}{\papi{ sɯɣʑi}}}\markboth{sɯɣʑi}{}
\paradigme{\textit{dir :} \jya pɯ-}
\begin{définition}\fra calmer\end{définition}
\begin{définition}\cmn 令…消肿、令…平静\end{définition}
\begin{exemple}\jya ɯ-mbrɯ ra pjɤ-sɯɣʑi\cmn 她平息了怒气\end{exemple}\classe{vt}
\end{sous-entrée}\begin{sous-entrée}
\vedette{\hypertarget{}{\papi{ ʑɣɤsɯɣʑi}}}\markboth{ʑɣɤsɯɣʑi}{}\classe{vi}
\begin{définition}\ 
\begin{déclaration}\grammar{refl}\end{déclaration}
\begin{déclaration}\grammar{caus}\end{déclaration}\end{définition}
\begin{définition}\fra se calmer\end{définition}
\begin{définition}\cmn 平静下来\end{définition}
\end{sous-entrée}\end{entrée}

\begin{entrée}
\vedette{\hypertarget{Ⓔʑiwarɯmtɕhɤt}{\papi{ ʑiwarɯmtɕhɤt}}}\markboth{ʑiwarɯmtɕhɤt}{}\classe{n}
\begin{définition}\fra une célébration bouddhique\end{définition}
\begin{définition}\cmn 一种法事
\begin{déclaration} \étymologie{\papi{ʑi.ba ri.mtɕʰod}}\end{déclaration}\end{définition}\end{entrée}

\begin{entrée}
\vedette{\hypertarget{Ⓔʑmbɤr}{\papi{ ʑmbɤr}}}\markboth{ʑmbɤr}{}\classe{n}
\begin{définition}\fra ulcère\end{définition}
\begin{définition}\cmn 疮\end{définition}
\begin{exemple}\jya a-rŋa ʑmbɤr ɲɤ-ɬoʁ\cmn 我脸上生了疮\end{exemple}\end{entrée}

\begin{entrée}
\vedette{\hypertarget{Ⓔʑmbraʁlaʁli}{\papi{ ʑmbraʁlaʁli}}}\markboth{ʑmbraʁlaʁli}{}\classe{n}
\begin{définition}\fra une plante\end{définition}
\begin{définition}\cmn 植物的一种\end{définition}
\begin{exemple}\jya ʑmbraʁlaʁli nɯ sɯjno ci ŋu, ɯ-ru kɯ-mpɯ tsa ŋu, kɯ-ɤlɯlju ŋu, ɯ-jwaʁ ʁnɯz ma ku-tshoʁ mɤ-cha. ɯ-jwaʁ ni ndʑi-pɤrthɤβ ri ɯ-mɯntoʁ ɲɯ-βze tɕe ɯ-jwaʁ nɯ nɯ-ɴɢɤt tɕe, ɯ-mɯntoʁ pjɯ-ŋgra tɕe ɯ-mat nɯ ɯ-jwaʁ ɯ-pa pjɤ-ɴqoʁ ŋu. ɯ-mat thɯ-tɯt tɕe ɣɯrni. ɯ-ŋgɯ ɯ-ci cho ɯ-rdoʁ ra kɯnɤ ɣɯrni. mɤ-sɤndɤɣ. tɤ-pɤtso ra kɯ tu-ndza-nɯ ŋgrɤl. nɯ tú-wɣ-ndza tɕe ``tɯ-rqo ʑmbraʁ mɤ-ɕe" tu-ti-nɯ ŋgrɤl.\cmn 
\stylefv{ʑmbraʁlaʁli} 是一种植物,茎有点柔软,呈圆柱形。只长两片叶子。花夹在两片叶子中间,叶子展开了以后,花凋落结成果实吊在叶子下面。果实成熟后变红。没有毒性。小孩子们经常吃这个果实。据说吃了它“青稞的芒不会进入喉咙”。
\end{exemple}\end{entrée}

\begin{entrée}
\vedette{\hypertarget{ⒺʑmbriⒽ2}{\papi{ ʑmbri}}}\markboth{ʑmbri}{}\homonyme{2}
\classe{n}
\begin{définition}\fra saule\end{définition}
\begin{définition}\cmn 柳树\end{définition}
\end{entrée}

\begin{entrée}
\vedette{\hypertarget{ⒺʑmbriⒽ1}{\papi{ ʑmbri}}}\markboth{ʑmbri}{}\homonyme{1}
\paradigme{\textit{dir :} \jya thɯ-}
\begin{définition}\fra faire du bruit, jouer d'un instrument de musique\end{définition}
\begin{définition}\cmn 发出声音;演奏音乐\end{définition}
\begin{exemple}\jya ɟuli thɯ-ʑmbri-t-a\cmn 我吹了竹笛\end{exemple}
\begin{exemple}\jya zɯxtɕhɤl ta-ʑmbri\cmn 他打钹了\end{exemple}
\begin{exemple}\jya mkhɤrŋa ta-ʑmbri\cmn 他敲锣了\end{exemple}
\begin{relation-sémantique}\confer{
\hyperlink{ⒺmbriⒽ1}{\textit{ \papi{mbri1}}}
}\end{relation-sémantique}\classe{vt}\end{entrée}

\begin{entrée}
\vedette{\hypertarget{Ⓔʑmbrijmɤɣ}{\papi{ ʑmbrijmɤɣ}}}\markboth{ʑmbrijmɤɣ}{}\classe{n}
\begin{définition}\fra Hericium erinaceus\end{définition}
\begin{définition}\cmn 猴头菌【杨柳菌】\end{définition}
\begin{exemple}\jya ʑmbrijmɤɣ nɯ ʑmbri kɯ-wxti ɯ-taʁ tu-ɬoʁ ɲɯ-ŋu, kɯmaʁ tɤjmɤɣ ra cho nɯ-tshɯɣa mɯ́j-naχtɕɯɣ, ɯʑo kɯ-ɤrtɯm rloŋrloŋ ɲɯ-ŋu, ɯ-βri nɯ tɤ-rme kɯ-fse ʁɟa ɲɯ-ŋu, kɤ-ndza ɲɯ-sna, ɯ-mdoʁ kɯ-ɤqarŋɯrŋe tsa ɲɯ-ŋu.\cmn 杨柳菌长在较高大的柳树上,样子和其他蘑菇不同。呈圆球形,全身长满毛,可以吃。颜色是淡黄色。\end{exemple}\end{entrée}

\begin{entrée}
\vedette{\hypertarget{Ⓔʑmbroko}{\papi{ ʑmbroko}}}\markboth{ʑmbroko}{}\classe{n}
\begin{définition}\fra Sonchus oleraceus\end{définition}
\begin{définition}\cmn 苦苣菜【空洞菜】\end{définition}
\begin{exemple}\jya ʑmbroko nɯ tɯ-xpa tu-kɯ-ɬoʁ sɯjno ŋu, ɯ-ru ɯ-ŋgɯ nɯ kɯ-so ŋu, pjɯ́-wɣ-qlɯt tɕe mpɯ, tɕe ɯ-lu tu, fsapaʁ ndza wuma pe ɯ-mɯntoʁ kɯ-qarŋe tɕe kɤ-rɯlaba ŋu\cmn 苦苣菜是一年生的植物,茎是空心的,撇断的时候是嫩的,有乳汁。是很好的饲草。花黄色,喇叭形。\end{exemple}
\end{entrée}

\begin{entrée}
\vedette{\hypertarget{Ⓔʑmbrɯ}{\papi{ ʑmbrɯ}}}\markboth{ʑmbrɯ}{}
\classe{n}
\begin{définition}\fra bateau\end{définition}
\begin{définition}\cmn 船
\begin{déclaration} \étymologie{\papi{gru}}\end{déclaration}\end{définition}
\begin{exemple}\jya ʑmbrɯ kɤ-lat-a\cmn 我划了船\end{exemple}\end{entrée}

\begin{entrée}
\vedette{\hypertarget{Ⓔʑmbrɯβɟaj}{\papi{ ʑmbrɯβɟaj}}}\markboth{ʑmbrɯβɟaj}{}\classe{n}
\begin{définition}\fra rame\end{définition}
\begin{définition}\cmn 桨\end{définition}\end{entrée}

\begin{entrée}
\vedette{\hypertarget{Ⓔʑmbrɯɟoʁ}{\papi{ ʑmbrɯɟoʁ}}}\markboth{ʑmbrɯɟoʁ}{}\classe{n}
\begin{définition}\fra clayonnage\end{définition}
\begin{définition}\cmn 杨柳枝条\end{définition}\end{entrée}

\begin{entrée}
\vedette{\hypertarget{Ⓔʑmbrɯkɤlu}{\papi{ ʑmbrɯkɤlu}}}\markboth{ʑmbrɯkɤlu}{}\classe{n}
\begin{définition}\fra type de saule\end{définition}
\begin{définition}\cmn 柳树的一种(看起来被据掉一样)\end{définition}\end{entrée}

\begin{entrée}
\vedette{\hypertarget{Ⓔʑmbrɯpɣa}{\papi{ ʑmbrɯpɣa}}}\markboth{ʑmbrɯpɣa}{}\classe{n}
\begin{définition}\fra espèce d'oiseau\end{définition}
\begin{définition}\cmn 一种鸟\end{définition}
\begin{exemple}\jya ʑmbrɯpɣa nɯ pɣa tɤŋkhɯt staʁnɤ wxti, ɯ-mi kɯ-qarŋe ŋu, ɯ-mtsioʁ nɯ ɲaʁ rɲɟi tsa ɯ-jme nɯ ɯ-phoŋbu sɤz rɲɟi, ɯ-muj ɯ-mdoʁ wuma mpɕɤr, kɯ-ɲaʁ ra kɯnɤ nɤmbju, ɯ-xtɤpa ra ʁmɤrsɤr ɯ-mdoʁ tu, ɯ-rqo pa cho ɯ-ʁar χchoʁe ra kɯ-wɣrum tɯ-snaʁ ka tu. tɯ-ji ɯ-ŋgɯ ju-ɣi mɤ-ŋgrɤl, stɤmku cho sɯŋgɯ ra ku-rɤʑi ɕti.\cmn 
\stylefv{ʑmbrɯpɣa}是一种鸟,略大于拳头。脚黄色,嘴黑色,尾巴比身子长。羽毛颜色很美,黑色有光泽。腹部是橙色的,脖子和翅膀上各有一块白点。这种鸟不会来到庄稼地里,只是在草地和森林里生活。
\end{exemple}\end{entrée}

\begin{entrée}
\vedette{\hypertarget{Ⓔʑmbɯlɯm}{\papi{ ʑmbɯlɯm}}}\markboth{ʑmbɯlɯm}{}\classe{n}
\begin{définition}\fra une espèce de champignon\end{définition}
\begin{définition}\cmn 【油辣枯】\end{définition}
\begin{exemple}\jya ʑmbɯlɯm nɯ sɤjku cho mbraj ɯ-ŋgɯ tu-ɬoʁ ŋu, ɯ-qhu nɯ kɯ-ɤqarŋɯ-rŋe tɕe kɯ-nɤmbju tsa ŋu, ɯ-rʑɯɣ cho ɯ-ru nɯ kɯ-wɣrum ŋu, tú-wɣ-ndza tɕe kɯ-xtɕɯ-xtɕi qiaβ cho mɤrtsaβ\cmn 油辣枯长在白桦树和红桦树林里,背面带有黄色,有点光泽,下面菌褶和干是白色的。吃的时候,有点苦,有点辣。\end{exemple}
\end{entrée}

\begin{entrée}
\vedette{\hypertarget{Ⓔʑŋgu}{\papi{ ʑŋgu}}}\markboth{ʑŋgu}{}
\classe{vi}
\paradigme{\textit{dir :} \jya kɤ-}
\begin{définition}\fra passer une rivière en bateau\end{définition}
\begin{définition}\cmn 坐船渡河\end{définition}
\begin{exemple}\jya tɯ-ci ɯ-taʁ ko-ʑŋgu (=tɯ-ʑŋgu ko-lɤt)\cmn 他坐船渡了河\end{exemple}
\begin{exemple}\jya kɤ-ʑŋgu-a\cmn 我渡了河\end{exemple}
\begin{exemple}\jya kɯ-ʑŋgu\cmn 船夫\end{exemple}\end{entrée}

\begin{entrée}
\vedette{\hypertarget{Ⓔʑŋga}{\papi{ ʑŋga}}}\markboth{ʑŋga}{}
\classe{vt}\acception{1}
\paradigme{\textit{dir :} \jya tɤ-}
\begin{définition}\fra aider qqn à s'habiller\end{définition}
\begin{définition}\cmn 帮别人穿衣服\end{définition}
\begin{exemple}\jya ɯ-ŋga tɤ-ʑŋga-t-a\cmn 我帮他穿衣服了\end{exemple}\acception{2}
\paradigme{\textit{dir :} \jya pɯ-}
\begin{définition}\fra border le lit à qqn\end{définition}
\begin{définition}\cmn 帮别人盖被子\end{définition}
\begin{exemple}\jya pɯ-ʑŋga-t-a\cmn 我帮他盖了被子\end{exemple}
\begin{relation-sémantique}\confer{
\hyperlink{Ⓔŋga}{\textit{ \papi{ŋga}}}
}\end{relation-sémantique}\end{entrée}

\begin{entrée}
\vedette{\hypertarget{Ⓔʑŋgi}{\papi{ ʑŋgi}}}\markboth{ʑŋgi}{}
\classe{vi}
\paradigme{\textit{dir :} \jya \_}
\begin{définition}\fra porter du bois\end{définition}
\begin{définition}\cmn 背柴\end{définition}
\begin{exemple}\jya ɕ-pɯ-ʑŋgi-a\cmn 我去背柴了\end{exemple}\end{entrée}

\begin{entrée}
\vedette{\hypertarget{Ⓔʑŋgri}{\papi{ ʑŋgri}}}\markboth{ʑŋgri}{}\classe{n}
\begin{définition}\fra étoile\end{définition}
\begin{définition}\cmn 星星\end{définition}
\begin{exemple}\jya ʑŋgri kɯ-nɯqambɯmbjom\cmn 流星\end{exemple}\end{entrée}

\begin{entrée}
\vedette{\hypertarget{Ⓔʑɴɢu}{\papi{ ʑɴɢu}}}\markboth{ʑɴɢu}{}
\classe{vt}
\paradigme{\textit{dir :} \jya pɯ-}
\paradigme{\textit{dir :} \jya thɯ-}
\begin{définition}\fra éplucher, décortiquer\end{définition}
\begin{définition}\cmn 削;掰开\end{définition}
\begin{exemple}\jya jima pɯ-ʑɴɢu-t-a\cmn 我剥了玉米\end{exemple}
\begin{exemple}\jya stoʁ pɯ-ʑɴɢu-t-a\cmn 我掰了胡豆\end{exemple}
\begin{exemple}\jya ʑɴɢɯloʁ pɯ-ʑɴɢu-t-a\cmn 我掰了核桃\end{exemple}
\begin{exemple}\jya rɤjndoʁ pɯ-ʑɴɢu-t-a\cmn 我剥了圆根\end{exemple}\begin{sous-entrée}
\vedette{\hypertarget{}{\papi{ nɯɣɯʑɴɢu}}}\markboth{nɯɣɯʑɴɢu}{}\classe{vs}
\begin{définition}\ 
\begin{déclaration}\grammar{facil}\end{déclaration}\end{définition}
\begin{définition}\fra facile à éplucher\end{définition}
\begin{définition}\cmn 容易削\end{définition}
\end{sous-entrée}\end{entrée}

\begin{entrée}
\vedette{\hypertarget{Ⓔʑɴɢoʁ}{\papi{ ʑɴɢoʁ}}}\markboth{ʑɴɢoʁ}{}\classe{vt}
\paradigme{\textit{dir :} \jya kɤ-}
\begin{définition}\fra accrocher\end{définition}
\begin{définition}\cmn 钩住\end{définition}
\begin{exemple}\jya a-tɤ-ri kɤ-ʑɴɢoʁ\cmn 你帮我牵线吧(用两只手的手指把线分开)\end{exemple}
\begin{relation-sémantique}\confer{
\hyperlink{Ⓔɴqoʁ}{\textit{ \papi{ɴqoʁ}}}
}\end{relation-sémantique}\begin{sous-entrée}
\vedette{\hypertarget{}{\papi{ aʑɴɢɯʑɴɢoʁ}}}\markboth{aʑɴɢɯʑɴɢoʁ}{}\classe{vi}
\begin{définition}\fra accroché les uns avec les autres\end{définition}
\begin{définition}\cmn 钩在一起\end{définition}
\begin{exemple}\jya βʑɯxsɯr nɯ aʑɴɢɯʑɴɢoʁ\cmn 牛蒡子的果子钩在一起\end{exemple}
\end{sous-entrée}\begin{sous-entrée}
\vedette{\hypertarget{}{\papi{ sɯʑɴɢoʁ}}}\markboth{sɯʑɴɢoʁ}{}\classe{vt}
\paradigme{\textit{dir :} \jya kɤ-}
\begin{définition}\ 
\begin{déclaration}\grammar{caus}\end{déclaration}\end{définition}
\begin{définition}\fra accrocher avec\end{définition}
\begin{définition}\cmn 用……钩住\end{définition}
\begin{exemple}\jya tɤ-jŋoʁ kɯ tɯ-ŋga ko-sɯʑɴɢoʁ\cmn 他用钩子把衣服钩起来了\end{exemple}
\begin{exemple}\jya tɯ-rju nɯ kɯ-kɯ-mɯ-maʁ ku-tɯ-sɯʑɴɢoʁ\cmn 你把话题东拉西扯\end{exemple}
\end{sous-entrée}\end{entrée}

\begin{entrée}
\vedette{\hypertarget{Ⓔʑɴɢro}{\papi{ ʑɴɢro}}}\markboth{ʑɴɢro}{}\classe{n}
\begin{définition}\fra guimbarde\end{définition}
\begin{définition}\cmn 口簧琴\end{définition}
\begin{exemple}\jya ʑɴɢro nɯ-ʑmbri-t-a\cmn 我吹了口簧琴\end{exemple}
\begin{exemple}\jya ʑɴɢro ɲɯ-ɤsɯ-lɤt\cmn 他在吹口簧琴\end{exemple}\end{entrée}

\begin{entrée}
\vedette{\hypertarget{Ⓔʑɴɢɯloʁ}{\papi{ ʑɴɢɯloʁ}}}\markboth{ʑɴɢɯloʁ}{}\classe{n}\acception{1}
\begin{définition}\fra noix\end{définition}
\begin{définition}\cmn 核桃\end{définition}
\begin{exemple}\jya mbrɯtɕɯ kɯ ʑɴɢɯloʁ nɯ-sɯphaʁ-a\cmn 我用刀子把核桃撬开了\end{exemple}
\begin{exemple}\jya ʑɴɢɯloʁ nɯ si kɯ-wxtɯ-wxti ci ŋu, ɯ-mat nɯ ɯ-rqhu kɯ-rkɯ-rko ŋu, ɯ-ŋgɯ nɯ tɯ-rnoʁ ɯ-tshɯɣa fse, tú-wɣ-ndza tɕe wuma ʑo mɯm, ɯ-kri tu, ɯ-mat nɯ tɤ-rtɕi sna.\cmn 核桃是一种高大的树,果子壳很硬,形状像大脑,吃起来很香,含有油脂,有滋补的作用。\end{exemple}\acception{2}
\begin{définition}\fra Golok\end{définition}
\begin{définition}\cmn 果洛\end{définition}\end{entrée}

\begin{entrée}
\vedette{\hypertarget{ⒺʑoⒽ2}{\papi{ ʑo}}}\markboth{ʑo}{}\homonyme{2}
\classe{part}
\begin{définition}\fra particule emphatique\end{définition}
\begin{définition}\cmn 强调助词\end{définition}
\end{entrée}

\begin{entrée}
\vedette{\hypertarget{ⒺʑoⒽ1}{\papi{ ʑo}}}\markboth{ʑo}{}\homonyme{1}\classe{vs}
\paradigme{\textit{dir :} \jya tɤ-}
\begin{définition}\fra léger\end{définition}
\begin{définition}\cmn 轻\end{définition}
\begin{exemple}\jya ɯ-fkur ɲɯ-ʑo\cmn 他的负担很轻\end{exemple}
\begin{relation-sémantique}\antonyme{
\hyperlink{Ⓔrʑi}{\textit{ \papi{rʑi}}}
}\end{relation-sémantique}\begin{sous-entrée}
\vedette{\hypertarget{}{\papi{ ʑɣɤɣɤʑo}}}\markboth{ʑɣɤɣɤʑo}{}\classe{vi}
\paradigme{\textit{dir :} \jya tɤ-}
\begin{définition}\ 
\begin{déclaration}\grammar{refl}\end{déclaration}
\begin{déclaration}\grammar{caus}\end{déclaration}\end{définition}
\begin{définition}\fra se rendre léger\end{définition}
\begin{définition}\cmn 令自己变轻\end{définition}
\begin{exemple}\jya tɤ-muj jamar ʑo to-ʑɣɤɣɤʑo\cmn 他令自己变得像羽毛一样轻\end{exemple}
\end{sous-entrée}\end{entrée}

\begin{entrée}
\vedette{\hypertarget{Ⓔʑru}{\papi{ ʑru}}}\markboth{ʑru}{}
\classe{vs}\acception{1}
\paradigme{\textit{dir :} \jya tɤ-}
\paradigme{\textit{dir :} \jya thɯ-}
\begin{définition}\fra grand et fort\end{définition}
\begin{définition}\cmn 强壮\end{définition}\acception{2}
\begin{définition}\fra de bonne qualité, précieux\end{définition}
\begin{définition}\cmn 质量好;优质;贵重\end{définition}
\begin{exemple}\jya ɕoŋtɕa ɲɯ-ʑru\cmn 木料质量很好\end{exemple}\begin{sous-entrée}
\vedette{\hypertarget{}{\papi{ nɤʑru}}}\markboth{nɤʑru}{}\classe{vt}
\begin{définition}\fra trouver précieux\end{définition}
\begin{définition}\cmn 觉得贵重\end{définition}
\end{sous-entrée}\end{entrée}

\begin{entrée}
\vedette{\hypertarget{Ⓔʑur}{\papi{ ʑur}}}\markboth{ʑur}{}
\classe{vs}
\paradigme{\textit{dir :} \jya tɤ-}
\begin{définition}\fra en quantité suffisante\end{définition}
\begin{définition}\cmn 足够\end{définition}
\begin{exemple}\jya kɤ-znɯkro mɯ́j-ʑur\cmn 不够分给人家\end{exemple}
\begin{exemple}\jya kɤ-rɤmbi mɯ́j-ʑur\cmn 不够送\end{exemple}
\begin{relation-sémantique}\synonyme{
\hyperlink{Ⓔrtaʁ}{\textit{ \papi{rtaʁ}}}
}\end{relation-sémantique}\end{entrée}

\begin{entrée}
\vedette{\hypertarget{Ⓔʑro}{\papi{ ʑro}}}\markboth{ʑro}{}\classe{n}
\begin{définition}\fra type d'arbrisseau\end{définition}
\begin{définition}\cmn 灌木的一种\end{définition}\end{entrée}

\begin{entrée}
\vedette{\hypertarget{Ⓔʑʁɯnʑʁɯn}{\papi{ ʑʁɯnʑʁɯn}}}\markboth{ʑʁɯnʑʁɯn}{}\classe{idph.2}
\begin{définition}\fra haut sur ses pieds\end{définition}
\begin{définition}\cmn 形容脚高而细长的样子\end{définition}\end{entrée}

\begin{entrée}
\vedette{\hypertarget{ⒺʑɯⒽ1}{\papi{ ʑɯ}}}\markboth{ʑɯ}{}\homonyme{1}
\classe{vs}
\paradigme{\textit{dir :} \jya tɤ-}
\begin{définition}\fra pas seulement\end{définition}
\begin{définition}\cmn 不只
\begin{déclaration}\use{只用于否定式}\end{déclaration}\end{définition}
\begin{exemple}\jya a-ʁi tɤtɕɯ kɯmŋu nɯ mɤ-ʑɯ ma mɤʑɯ tɕheme ci tu\cmn 不只五个弟弟,还有一个妹妹\end{exemple}
\begin{exemple}\jya tɯrme tɯ-rdoʁ mɯ́j-ʑɯ ma laχsɯm ɣɤʑu-nɯ\cmn 不只一个人,有两三个人\end{exemple}
\begin{exemple}\jya paχɕi ɯ-tɯ-wxti kɯ tɤŋkhɯt mɯ́j-ʑɯ (mɤ-kɯ-ʑɯ) jamar ɲɯ-wxti\cmn 苹果比拳头大一点\end{exemple}
\begin{relation-sémantique}\confer{
\hyperlink{Ⓔmɤʑɯ}{\textit{ \papi{mɤʑɯ}}}
}\end{relation-sémantique}\end{entrée}

\begin{entrée}
\vedette{\hypertarget{Ⓔʑɯβdaʁ}{\papi{ ʑɯβdaʁ}}}\markboth{ʑɯβdaʁ}{}\classe{n}
\begin{définition}\fra dieu de la montagne\end{définition}
\begin{définition}\cmn 山神
\begin{déclaration} \étymologie{\papi{gʑi.bdag}}\end{déclaration}\end{définition}\end{entrée}

\begin{entrée}
\vedette{\hypertarget{Ⓔʑɯβʑɯβ}{\papi{ ʑɯβʑɯβ}}}\markboth{ʑɯβʑɯβ}{} (\variante{ʑɯpʑɯp}) \classe{idph.2}
\begin{définition}\fra beaucoup de chose dressées, beaucoup de gens debout\end{définition}
\begin{définition}\cmn 形容很多人、东西、动物等密密麻麻地站着的样子\end{définition}
\begin{exemple}\jya tɤ-rtsho ʑɯβʑɯβ ʑo ɲɯ-pa\cmn 麦桩密密麻麻地立在那里\end{exemple}
\begin{exemple}\jya tɯrme ʑɯβʑɯβ ʑo ɲɯ-ndzur-nɯ\cmn 很多人在那里站着\end{exemple}
\begin{exemple}\jya fsapaʁ ra zgoku zɯ ʑɯβʑɯβ ʑo ɲɯ-rɤʑi-nɯ\cmn 山上牲畜很多,密密麻麻\end{exemple}\begin{sous-entrée}
\vedette{\hypertarget{}{\papi{ ʑɯβnɤlɯβ}}}\markboth{ʑɯβnɤlɯβ}{}\classe{idph.4}
\begin{définition}\fra beaucoup de gens allant dans tous les sens s'occupant chacun de leur tâche\end{définition}
\begin{définition}\cmn 形容很多人来回走动,各自做自己的事情的场面\end{définition}
\begin{exemple}\jya tɯrme ra ʑɯβnɤlɯβ ʑo ɲɯ-rɤma-nɯ\cmn 很多人来回走动,各自做自己的事情\end{exemple}
\begin{relation-sémantique}\confer{
\hyperlink{Ⓔjɯβjɯβ}{\textit{ \papi{jɯβjɯβ}}}
}\end{relation-sémantique}
\begin{relation-sémantique}\confer{
\hyperlink{Ⓔɟɯɣɟɯɣ}{\textit{ \papi{ɟɯɣɟɯɣ}}}
}\end{relation-sémantique}
\end{sous-entrée}\end{entrée}

\begin{entrée}
\vedette{\hypertarget{ⒺʑɯɣⒽ1}{\papi{ ʑɯɣ}}}\markboth{ʑɯɣ}{}\homonyme{1}\classe{vs}
\paradigme{\textit{dir :} \jya pɯ-}
\begin{définition}\fra avoir complètement pourri\end{définition}
\begin{définition}\cmn 完全腐烂掉\end{définition}
\begin{exemple}\jya nɯŋa pjɤ-ndʐaβ tɕe, pjɤ-ʑɯɣ\cmn 奶牛摔倒了,死了然后尸体完全腐烂掉了\end{exemple}
\begin{relation-sémantique}\synonyme{
\hyperlink{Ⓔrɲɯl}{\textit{ \papi{rɲɯl}}}
}\end{relation-sémantique}\end{entrée}

\begin{entrée}
\vedette{\hypertarget{ⒺʑɯɣⒽ2}{\papi{ ʑɯɣ}}}\markboth{ʑɯɣ}{}\homonyme{2}
\classe{vt}
\begin{définition}\fra dire bonsoir\end{définition}
\begin{définition}\cmn 请晚安
\begin{déclaration}\use{古语}\end{déclaration}
\begin{déclaration} \étymologie{\papi{bʑugs}}\end{déclaration}\end{définition}
\begin{exemple}\jya sɤrma a-tɤ-ʑɯɣ-nɯ\cmn 他们要向他请安\end{exemple}\end{entrée}

\begin{entrée}
\vedette{\hypertarget{Ⓔʑɯm}{\papi{ ʑɯm}}}\markboth{ʑɯm}{}\classe{vs}
\paradigme{\textit{dir :} \jya nɯ-}\acception{1}
\begin{définition}\fra bon à manger\end{définition}
\begin{définition}\cmn 好吃\end{définition}
\begin{exemple}\jya jɯfɕɯr, @cai tɤ-ndza-t-a pɯ-ʑɯm\cmn 昨天吃的菜,很好吃\end{exemple}
\begin{exemple}\jya ki kɤ-ndza ki ɲɯ-mɯm tɕe, tú-wɣ-nɯ-ndza tɕe ɲɯ-ʑɯm\cmn 这种食物很好吃,吃了感觉很好\end{exemple}\acception{2}
\begin{définition}\fra désaltérant\end{définition}
\begin{définition}\cmn 解渴的\end{définition}
\begin{exemple}\jya aʑo tɤ-lu kɤ-tshi-t-a, ɲɯ-ʑɯm\cmn 我喝了牛奶,很解渴\end{exemple}\acception{3}
\begin{définition}\fra agréable\end{définition}
\begin{définition}\cmn 舒服\end{définition}
\begin{exemple}\jya ki tɯ-ŋga tɤ-ŋga-t-a tɕe, ɲɯ-mpja tɕe pɯ-ʑɯm\cmn 我穿了这件衣服,很暖很舒服\end{exemple}\end{entrée}

\begin{entrée}
\vedette{\hypertarget{Ⓔʑɯmkhɤm}{\papi{ ʑɯmkhɤm}}}\markboth{ʑɯmkhɤm}{}\classe{n}\acception{1}
\begin{définition}\fra domaine\end{définition}
\begin{définition}\cmn 管辖区\end{définition}\acception{2}
\begin{définition}\fra beaucoup, un long moment\end{définition}
\begin{définition}\cmn 很多;很长时间\end{définition}
\begin{exemple}\jya ʑɯmkhɤm ɯ-xpa\cmn 很多年\end{exemple}
\begin{exemple}\jya tɯrme ʑɯmkhɤm ʑo jo-ɣi-nɯ\cmn 来了很多人\end{exemple}\acception{3}
\begin{définition}\fra un long moment\end{définition}
\begin{définition}\cmn 很长时间
\begin{déclaration} \étymologie{\papi{ʑiŋ.kʰams}}\end{déclaration}\end{définition}
\begin{relation-sémantique}\confer{
\hyperlink{Ⓔʑɯŋkhɤm}{\textit{ \papi{ʑɯŋkhɤm}}}
}\end{relation-sémantique}\end{entrée}

\begin{entrée}
\vedette{\hypertarget{Ⓔʑɯnmar}{\papi{ ʑɯnmar}}}\markboth{ʑɯnmar}{}
\classe{n}
\begin{définition}\fra beurre clarifié\end{définition}
\begin{définition}\cmn 融酥(净化黄油)
\begin{déclaration} \étymologie{\papi{ʑun.mar}}\end{déclaration}
\end{définition}
\begin{relation-sémantique}\confer{
\hyperlink{Ⓔta-mar}{\textit{ \papi{ta-mar}}}
}\end{relation-sémantique}\end{entrée}

\begin{entrée}
\vedette{\hypertarget{Ⓔʑɯŋkhɤm}{\papi{ ʑɯŋkhɤm}}}\markboth{ʑɯŋkhɤm}{}\classe{n}
\begin{définition}\fra la terre entière\end{définition}
\begin{définition}\cmn 全世界
\begin{déclaration} \étymologie{\papi{ʑiŋ.kʰams}}\end{déclaration}\end{définition}
\begin{exemple}\jya ʑɯŋkhɤm ɯ-ku zɯ\cmn 世界上\end{exemple}
\begin{relation-sémantique}\confer{
\hyperlink{Ⓔʑɯmkhɤm}{\textit{ \papi{ʑɯmkhɤm}}}
}\end{relation-sémantique}
\end{entrée}

\begin{entrée}
\vedette{\hypertarget{Ⓔʑɯrɯʑɤri}{\papi{ ʑɯrɯʑɤri}}}\markboth{ʑɯrɯʑɤri}{}\classe{adv}
\begin{définition}\fra progressivement\end{définition}
\begin{définition}\cmn 渐渐\end{définition}
\begin{exemple}\jya ʑɯrɯʑɤri tɕe chɯ-chɯ-mɤɕi-ndʑi nɤ chɯ-chɯ-mɤɕi-ndʑi\cmn 他们俩逐渐变得越来越富裕\end{exemple}
\begin{relation-sémantique}\confer{
\hyperlink{Ⓔɯ-ʑɤrʑɯr}{\textit{ \papi{ɯ-ʑɤrʑɯr}}}
}\end{relation-sémantique}\end{entrée}

\begin{entrée}
\vedette{\hypertarget{Ⓔʑɯxsa}{\papi{ ʑɯxsa}}}\markboth{ʑɯxsa}{}\classe{n}
\begin{définition}\fra siège, place où l'on s'assoit (honorifique)\end{définition}
\begin{définition}\cmn 座位(敬语)
\begin{déclaration} \étymologie{\papi{bʑugs.sa}}\end{déclaration}\end{définition}
\end{entrée}

\begin{entrée}
\vedette{\hypertarget{Ⓔʑɯxsɯr}{\papi{ ʑɯxsɯr}}}\markboth{ʑɯxsɯr}{}\classe{n}
\begin{définition}\fra fruit de la bardane\end{définition}
\begin{définition}\cmn 牛蒡子\end{définition}
\begin{relation-sémantique}\confer{
\hyperlink{Ⓔtɤtɕɯβraʁ}{\textit{ \papi{tɤtɕɯβraʁ}}}
}\end{relation-sémantique}
\end{entrée}

\begin{entrée}
\vedette{\hypertarget{Ⓔʑwɤʑwɤr}{\papi{ ʑwɤʑwɤr}}}\markboth{ʑwɤʑwɤr}{}\classe{idph.2}
\begin{définition}\fra de travers (chapeau)\end{définition}
\begin{définition}\cmn 形容不周正的样子(帽子)\end{définition}
\begin{relation-sémantique}\synonyme{
\hyperlink{Ⓔjwɤjwɤr}{\textit{ \papi{jwɤjwɤr}}}
}\end{relation-sémantique}
\begin{sous-entrée}
\vedette{\hypertarget{}{\papi{ ʑwɤrnɤlɤr}}}\markboth{ʑwɤrnɤlɤr}{}\classe{idph.4}
\begin{définition}\fra gesticulant\end{définition}
\begin{définition}\cmn 形容指手画脚的样子\end{définition}
\begin{exemple}\jya ʑwɤrnɤlɤr ʑo ma-tɯ-ʑɣɤstu kɯ phoʁphoʁ kɤ-rɤʑi\cmn 你不要指手画脚,规矩一点\end{exemple}
\end{sous-entrée}\end{entrée}

\end{multicols}
\end{document}